\begin{document}

\title{Exodus}

Exo 1:1  Numele fiilor lui Israel, care au intrat în Egipt împreuna cu Iacov, tatal lor, aducând fiecare toata casa sa, sunt acestea:
Exo 1:2  Ruben, Simeon, Levi ?i Iuda;
Exo 1:3  Isahar, Zabulon ?i Veniamin;
Exo 1:4  Dan, Neftali, Gad ?i A?er.
Exo 1:5  Sufletele însa ie?ite din Iacov erau de toate ?aptezeci ?i cinci, iar Iosif era de mai înainte în Egipt.
Exo 1:6  Dar au murit ?i Iosif ?i to?i fra?ii lui ?i to?i cei de pe vremea lor.
Exo 1:7  Iar fiii lui Israel s-au nascut în numar mare ?i s-au înmul?it, au crescut ?i s-au întarit foarte, foarte tare, ?i s-a umplut ?ara de ei.
Exo 1:8  Dar s-a ridicat alt rege peste Egipt, care nu cunoscuse pe Iosif.
Exo 1:9  Acesta a zis catre poporul sau: "Iata, neamul fiilor lui Israel e mul?ime mare ?i e mai tare decât noi.
Exo 1:10  Veni?i dar sa-i împilam, ca sa nu se mai înmul?easca ?i ca nu cumva la vreme de razboi sa se uneasca cu vrajma?ii no?tri ?i, batându-ne, sa iasa din ?ara noastra!"
Exo 1:11  De aceea au pus peste ei supraveghetori de lucrari, ca sa-i împileze cu munci grele. Atunci a zidit Israel ceta?i tari lui Faraon: Pitom ?i Ramses, care serveau lui Faraon ca hambare, ?i cetatea On sau Iliopolis.
Exo 1:12  Însa cu cât îi împilau mai mult, cu atât mai mult se înmul?eau ?i se întareau foarte, foarte tare, a?a ca Egiptenii se îngrozeau de fiii lui Israel.
Exo 1:13  De aceea Egiptenii sileau înca ?i mai stra?nic la munca pe fiii lui Israel
Exo 1:14  ?i le faceau via?a amara prin munci grele, la lut, la caramida ?i la tot felul de lucru de câmp ?i prin alte felurite munci, la care-i sileau cu stra?nicie.
Exo 1:15  Ba, regele Egiptului a poruncit moa?elor evreie?ti, care se numeau: una ?ifra ?i alta Pua,
Exo 1:16  ?i le-a zis: "Când mo?i?i la evreice, sa lua?i seama când nasc: de va fi baiat, sa-l omorâ?i, iar de va fi fata, sa o cru?a?i!"
Exo 1:17  Moa?ele însa s-au temut de Dumnezeu ?i n-au facut cum le poruncise regele Egiptului, ci au lasat ?i pe baie?i sa traiasca.
Exo 1:18  Atunci a chemat regele Egiptului pe moa?e ?i le-a zis: "pentru ce a?i facut a?a ?i a?i lasat sa traiasca ?i copiii de parte barbateasca?"
Exo 1:19  Iar moa?ele au raspuns lui Faraon: "Femeile evreice nu sunt ca egiptencele, ci ele sunt voinice ?i nasc pâna nu vin moa?ele la ele".
Exo 1:20  De aceea Dumnezeu a facut bine moa?elor, iar poporul lui Israel se înmul?ea ?i se întarea mereu.
Exo 1:21  ?i fiindca moa?ele se temeau de Dumnezeu, de aceea El le-a întarit neamul.
Exo 1:22  Atunci Faraon a poruncit la tot poporul sau ?i a zis: "Tot copilul de parte barbateasca, ce se va na?te Evreilor, sa-l arunca?i în Nil, iar fetele sa le lasa?i sa traiasca toate!"
Exo 2:1  Un om oarecare, din semin?ia lui Levi, ?i-a luat femeie din fetele lui Levi.
Exo 2:2  Femeia aceea a luat în pântece ?i a nascut un baiat ?i, vazând ca e frumos, l-a ascuns vreme de trei luni.
Exo 2:3  Dar, fiindca nu putea sa-l mai doseasca, a luat mama lui un co? de papura ?i l-a uns cu catran ?i cu smoala ?i punând copilul în el l-a a?ezat în papuri?, la marginea râului.
Exo 2:4  Iar sora copilului pândea de departe ca sa vada ce are sa i se întâmple.
Exo 2:5  Atunci s-a pogorât fata lui Faraon la râu sa se scalde, ?i roabele ei o înso?ira pe malul râului. ?i vazând co?ul în papuri?, ea a trimis pe una din roabele sale sa-l aduca.
Exo 2:6  ?i, deschizându-l, a vazut copilul: era un baiat care plângea. Atunci i s-a facut mila de el fetei lui Faraon ?i a zis: "Acesta este dintre copiii Evreilor".
Exo 2:7  Iar sora copilului a zis catre fata lui Faraon: "Voie?ti sa ma duc sa-?i chem o doica dintre evreice, ca sa alapteze copilul?"
Exo 2:8  Fata lui Faraon i-a zis: "Du-te!" ?i s-a dus copila ?i a chemat pe mama pruncului.
Exo 2:9  Atunci fata lui Faraon i-a zis: "Ia-mi copilul acesta ?i mi-l alapteaza, ca eu am sa-?i platesc! " ?i a luat femeia copilul ?i l-a alaptat.
Exo 2:10  Dupa ce a crescut copilul, doica l-a dus la fata lui Faraon ?i i-a fost ca fiu ?i i-a pus numele Moise, pentru ca î?i zicea: "Din apa l-am scos!"
Exo 2:11  Iar dupa multa vreme, când se facuse mare, Moise a ie?it la fiii lui Israel, fra?ii sai, ?i a vazut muncile lor cele grele. Cu prilejul acesta a vazut el pe un egiptean ca batea pe un evreu dintre fiii lui Israel, fra?ii sai;
Exo 2:12  ?i cautând încoace ?i încolo ?i nevazând pe nimeni, el a ucis pe egiptean ?i l-a ascuns în nisip.
Exo 2:13  Apoi ie?ind iara?i a doua zi, a vazut doi evrei certându-se ?i a zis asupritorului: "pentru ce ba?i pe aproapele tau?"
Exo 2:14  Acela însa i-a raspuns: "Cine te-a pus capetenie ?i judecator peste noi? Nu cumva vrei sa ma omori ?i pe mine, cum ai omorât ieri pe egipteanul acela?" ?i s-a spaimântat Moise ?i a zis: "Cu adevarat s-a vadit fapta aceasta!"
Exo 2:15  Iar daca a aflat Faraon de fapta aceasta, el a voit sa ucida pe Moise. Moise însa a fugit de la fa?a lui Faraon ?i s-a dus în ?ara Madian; ?i sosind în ?ara Madian, s-a oprit la o fântâna.
Exo 2:16  Preotul din Madian însa avea ?apte fete, care pa?teau oile tatalui lor. ?i venind acestea au scos apa ?i au umplut adapatorile, ca sa adape oile tatalui lor.
Exo 2:17  Dar pastorii venind, le-au alungat. Atunci s-a sculat Moise ?i le-a aparat, le-a scos apa ?i le-a adapat oile.
Exo 2:18  Mergând ele la tatal lor Raguel, acesta le-a zis: "Cum de a?i venit astazi a?a de curând?"
Exo 2:19  Iar ele au zis: "Un egiptean oarecare ne-a aparat de pastori, ne-a scos apa ?i ne-a adapat oile noastre!"
Exo 2:20  Zis-a acela catre fiicele sale: "Dar unde este acela? De ce l-a?i lasat? Chema?i-l ?i da?i-i sa manânce pâine!"
Exo 2:21  ?i a ramas Moise la omul acela ?i i-a dat pe fiica sa Sefora de so?ie.
Exo 2:22  Aceasta, luând în pântece, a nascut un fiu ?i i-a pus Moise numele Gher?on, zicând: "Am ajuns pribeag în ?ara straina". ?i luând iara?i în pântece, femeia a nascut alt fiu ?i i-a pus numele Eliezer, pentru ca ?i-a zis: "Dumnezeul tatalui meu mi-a fost ajutor ?i m-a scapat din mâna lui Faraon".
Exo 2:23  Apoi, dupa trecere de vreme multa, a murit regele Egiptului, de care fugise Moise. Fiii lui Israel însa gemeau sub povara muncilor ?i strigau ?i strigarea lor din munca s-a suit pâna la Dumnezeu.
Exo 2:24  Auzind suspinele lor, Dumnezeu ?i-a adus aminte de legamântul Sau pe care îl facuse cu Avraam, cu Isaac ?i cu Iacov.
Exo 2:25  De aceea a cautat Dumnezeu spre fiii lui Israel ?i S-a gândit la ei.
Exo 3:1  În vremea aceea, Moise pa?tea oile lui Ietro, preotul din Madian, socrul sau. ?i departându-se odata cu turma în pustie, a ajuns pâna la muntele lui Dumnezeu, la Horeb;
Exo 3:2  Iar acolo i S-a aratat îngerul Domnului într-o para de foc, ce ie?ea dintr-un rug; ?i a vazut ca rugul ardea, dar nu se mistuia.
Exo 3:3  Atunci Moise ?i-a zis: "Ma duc sa vad aceasta aratare minunata: ca rugul nu se mistuie?te".
Exo 3:4  Iar daca a vazut Domnul ca se apropie sa priveasca, a strigat la el Domnul din rug ?i a zis: "Moise! Moise!". ?i el a raspuns: "Iata-ma, Doamne!"
Exo 3:5  ?i Domnul a zis: "Nu te apropia aici! Ci scoate-?i încal?amintea din picioarele tale, ca locul pe care calci este pamânt sfânt!"
Exo 3:6  Apoi i-a zis iara?i: "Eu sunt Dumnezeul tatalui tau, Dumnezeul lui Avraam ?i Dumnezeul lui Isaac ?i Dumnezeul lui Iacov!" ?i ?i-a acoperit Moise fa?a sa, ca se temea sa priveasca pe Dumnezeu.
Exo 3:7  Zis-a Domnul catre Moise: "Am vazut necazul poporului Meu în Egipt ?i strigarea lui de sub apasatori am auzit ?i durerea lui o ?tiu.
Exo 3:8  M-am pogorât dar sa-l izbavesc din mâna Egiptenilor, sa-l scot din ?ara aceasta ?i sa-l duc într-un pamânt roditor ?i larg, în ?ara unde curge miere ?i lapte, în ?inutul Canaaneilor, al Heteilor, al Amoreilor, al Ferezeilor, al Ghergheseilor, al Heveilor ?i al Iebuseilor.
Exo 3:9  Iata dar ca strigarea fiilor lui Israel a ajuns acum pâna la Mine ?i am vazut chinurile lor, cu care-i pedepsesc Egiptenii.
Exo 3:10  Vino dar sa te trimit la Faraon, regele Egiptului, ca sa sco?i pe fiii lui Israel, poporul Meu, din ?ara Egiptului!"
Exo 3:11  Atunci a zis Moise catre Dumnezeu: "Cine sunt eu, ca sa ma duc la Faraon, regele Egiptului, ?i sa scot pe fiii lui Israel din ?ara Egiptului?"
Exo 3:12  Iar Dumnezeu i-a zis: "Eu voi fi cu tine ?i acesta î?i va fi semnul ca te trimit Eu: când vei scoate pe poporul Meu din ?ara Egiptului, va ve?i închina lui Dumnezeu în muntele acesta!"
Exo 3:13  Zis-a iara?i Moise catre Dumnezeu: "Iata, eu ma voi duce la fiii lui Israel ?i le voi zice: Dumnezeul parin?ilor vo?tri m-a trimis la voi... Dar de-mi vor zice: Cum Îl cheama, ce sa le spun?"
Exo 3:14  Atunci Dumnezeu a raspuns lui Moise: "Eu sunt Cel ce sunt". Apoi i-a zis: "A?a sa spui fiilor lui Israel: Cel ce este m-a trimis la voi!"
Exo 3:15  Apoi a zis Dumnezeu iara?i catre Moise: "A?a sa spui fiilor lui Israel: "Domnul Dumnezeul parin?ilor no?tri, Dumnezeul lui Avraam, Dumnezeul lui Isaac ?i Dumnezeul lui Iacov m-a trimis la voi. Acesta este numele Meu pe veci; aceasta este pomenirea Mea din neam în neam!"
Exo 3:16  Mergând dar, aduna pe batrânii fiilor lui Israel ?i le spune: "Domnul Dumnezeul parin?ilor no?tri, Dumnezeul lui Avraam, Dumnezeul lui Isaac ?i Dumnezeul lui Iacov mi S-a aratat ?i a zis: V-am cercetat de aproape ?i am vazut câte vi se întâmpla în Egipt!"
Exo 3:17  ?i mi-a mai zis: "Va voi scoate din împilarea Egiptului ?i va voi duce în pamântul Canaaneilor, al Heteilor, al Amoreilor, al Ferezeilor, al Ghergheseilor, al Heveilor ?i al Iebuseilor, în pamântul unde curge miere ?i lapte".
Exo 3:18  Iar ei vor asculta glasul tau. Atunci vei intra tu ?i batrânii lui Israel la Faraon, regele Egiptului, ?i-i ve?i zice: "Domnul Dumnezeul Evreilor ne-a chemat. Lasa-ne dar sa mergem în pustie, cale de trei zile, ca sa aducem jertfa Dumnezeului nostru".
Exo 3:19  Eu însa ?tiu ca Faraon, regele Egiptului, nu are sa va lase sa pleca?i, pâna nu îl voi sili Eu cu mâna tare.
Exo 3:20  Voi întinde deci mâna Mea ?i voi lovi Egiptul cu toate minunile, pe care le voi face în mijlocul lui, ?i dupa aceea va va lasa.
Exo 3:21  Voi da poporului acestuia trecere înaintea Egiptenilor ?i când ve?i ie?i, nu ve?i ie?i cu mâinile goale,
Exo 3:22  Ci fiecare femeie va cere la vecina sa ?i de la cea care sta cu ea în casa vase de argint, lucruri de aur ?i haine ?i ve?i împodobi cu ele pe fiii vo?tri ?i pe fetele voastre ?i ve?i prada pe Egipteni".
Exo 4:1  ?i raspunzând, Moise a zis: "Dar de nu ma vor crede ?i nu vor asculta de glasul meu, ci vor zice: "Nu ?i S-a aratat Domnul!", ce sa le spun?"
Exo 4:2  Zis-a Domnul catre el: "Ce ai în mâna?" ?i el a raspuns: "Un toiag!"
Exo 4:3  "Arunca-l jos!" îi zise Domnul. ?i a aruncat Moise toiagul jos ?i s-a facut toiagul ?arpe ?i a fugit Moise de el.
Exo 4:4  ?i a zis Domnul catre Moise: "Întinde mâna ?i-l apuca de coada!" ?i ?i-a întins Moise mâna ?i l-a apucat de coada ?i s-a facut toiag în mâna lui.
Exo 4:5  Apoi a zis Domnul: "A?a sa faci înaintea lor, ca sa te creada ca ?i S-a aratat Dumnezeul parin?ilor lor, Dumnezeul lui Avraam, Dumnezeul lui Isaac ?i Dumnezeul lui Iacov!"
Exo 4:6  Zis-a Domnul iara?i: "Baga-?i mâna în sân!" ?i când a scos-o din sân, iata mâna lui era alba ca zapada de lepra.
Exo 4:7  ?i i-a zis din nou Domnul: "Baga-?i iara?i mina în sân!" ?i ?i-a bagat Moise mâna în sân; ?i când a scos-o din sân, iata, era iar curata, ca tot trupul sau.
Exo 4:8  "Daca nu te vor crede ?i nu vor asculta glasul semnului întâi, te vor crede la savâr?irea semnului al doilea.
Exo 4:9  Iar de nu te vor crede nici dupa amândoua semnele ?i nu vor asculta glasul tau, atunci sa iei apa din fluviu ?i s-o ver?i pe uscat; apa luata din râu se va face pe usca: sânge".
Exo 4:10  Atunci Moise a zis catre Domnul: "O, Doamne, eu nu sunt om îndemânatic la vorba, ci graiesc cu anevoie ?i sunt gângav; ?i aceasta nu de ieri de alaltaieri, nici de când ai început Tu a grai cu robul Tau; gura mea ?i limba mea sunt anevoioase".
Exo 4:11  Dumnezeu însa a zis catre Moise: "Cine a dat omului gura ?i cine face pe om mut, sau surd, sau cu vedere, sau orb? Oare nu Eu, Domnul Dumnezeu?
Exo 4:12  Mergi dar: Eu voi deschide gura ta ?i te voi înva?a ce sa graie?ti".
Exo 4:13  Zis-a Moise: "Rogu-ma, Doamne, trimite pe altul, pe care vei vrea sa-l trimi?i!"
Exo 4:14  Atunci, aprinzându-se mânia Domnului asupra lui Moise, a zis: "Nu ai tu, oare, pe fratele tau Aaron levitul? ?tiu ca el poate sa vorbeasca în locul tau. Iata el te va întâmpina ?i, când te va vedea, se va bucura în inima sa.
Exo 4:15  Tu-i vei grai lui ?i îi vei pune în gura cuvintele Mele, iar Eu voi deschide gura ta ?i gura lui ?i va voi înva?a ce sa face?i.
Exo 4:16  Va grai el poporului, în locul tau, vorbind pentru tine, iar tu îi vei fi graitor din partea lui Dumnezeu.
Exo 4:17  Toiagul acesta, care a fost prefacut în ?arpe, ia-l în mâna ta, caci cu el ai sa faci minuni".
Exo 4:18  Deci, a plecat Moise de acolo ?i s-a întors la Ietro, socrul sau, ?i a zis catre el: "Ma duc înapoi la fra?ii mei, care sunt în Egipt, ca sa vad de mai traiesc". Iar Ietro i-a zis: "Mergi în pace!"
Exo 4:19  Dupa atât de multe zile a murit regele Egiptului care prigonise pe Moise, ?i Domnul a grait aceasta lui Moise în pamântul Madian: "Scoala ?i întoarce-te în Egipt, ca au murit to?i cei ce cautau sufletul tau!"
Exo 4:20  Luând atunci femeia ?i copiii, Moise i-a pus pe asini ?i s-a întors în Egipt. ?i a luat Moise în mâna sa ?i toiagul cel de la Dumnezeu.
Exo 4:21  ?i a zis Domnul catre Moise: "Când vei merge ?i vei ajunge în ?ara Egiptului, cauta sa faci înaintea lui Faraon toate minunile ce ?i-am poruncit. Eu însa voi învârto?a inima lui ?i nu va da drumul poporului.
Exo 4:22  Dar tu sa zici lui Faraon: A?a zice Domnul Dumnezeul Evreilor: Israel este fiul Meu, întâi-nascutul Meu.
Exo 4:23  Î?i zic dar: Lasa pe fiul Meu sa Mi se închine; iar de nu-l vei lasa, iata, î?i voi ucide pe fiul tau cel întâi-nascut".
Exo 4:24  Însa, la un popas de noapte, pe cale, l-a întâmpinat îngerul Domnului ?i a încercat sa-l omoare.
Exo 4:25  Dar Sefora, luând un cu?it de piatra, a taiat împrejur pe fiul sau ?i atingând picioarele lui Moise a zis; "Tu-mi e?ti un so? crud!"
Exo 4:26  ?i S-a dus Domnul de la el; iar ea, din pricina acestei taieri împrejur, i-a zis lui Moise: "So? crud!"
Exo 4:27  Atunci a zis Domnul catre Aaron: "Mergi în întâmpinarea lui Moise în pustie!" ?i s-a dus acesta ?i s-a întâlnit cu el în muntele lui Dumnezeu ?i s-au sarutat amândoi.
Exo 4:28  Atunci a spus Moise lui Aaron toate cuvintele Domnului, pe care i le poruncise, ?i toate semnele ce-i încredin?ase sa faca.
Exo 4:29  Dupa aceea s-au dus Moise ?i Aaron ?i au adunat pe to?i batrânii fiilor lui Israel
Exo 4:30  ?i le-a spus Aaron toate cuvintele pe care le graise Domnul lui Moise, ?i a facut Moise semne înaintea poporului;
Exo 4:31  ?i poporul a crezut ?i s-a bucurat ca a cercetat Domnul pe fiii lui Israel ?i a vazut necazurile lor ?i, plecându-se, s-au închinat.
Exo 5:1  Dupa aceea Moise ?i Aaron au intrat la Faraon ?i au zis catre dânsul: "A?a graie?te Domnul Dumnezeul lui Israel: Lasa pe poporul Meu, ca sa-Mi faca sarbatoare în pustie!"
Exo 5:2  Faraon însa a zis: "Cine este acela Domnul, ca sa-I ascult glasul ?i sa dau drumul fiilor lui Israel? Nu-L cunosc pe Domnul ?i nu voi da drumul lui Israel!"
Exo 5:3  Zis-au ei catre dânsul: "Dumnezeul Evreilor ne-a chemat; lasa-ne sa mergem în pustie cale de trei zile, ca sa aducem jertfa Domnului Dumnezeului nostru, ca sa nu pierim de ciuma sau de sabie!"
Exo 5:4  Iar regele Egiptului le-a zis: "Moise ?i Aaron, pentru ce-mi stingheri?i poporul de la lucru? Duce?i-va fiecare din voi la treburile voastre!"
Exo 5:5  Apoi Faraon a zis iar: "Iata acum s-a înmul?it poporul acesta în ?ara ?i voi îl întrerupe?i de la lucru".
Exo 5:6  ?i chiar în ziua aceea a poruncit Faraon capeteniilor ?i slujba?ilor poporului ?i le-a zis:
Exo 5:7  "De acum înainte sa nu mai da?i poporului paie pentru facerea caramizii, ca ieri ?i ca alaltaieri, ci sa se duca ei sa-?i adune paie.
Exo 5:8  Dar caramizi sa faca tot atâtea câte faceau în fiecare zi; sa-i sili?i ?i sa nu le împu?ina?i munca; fiindca sunt fara treaba ?i de aceea striga ?i zic: Haidem sa aducem jertfa Dumnezeului nostru!
Exo 5:9  Sa fie dar împovara?i de lucru oamenii ace?tia ?i sa se îndeletniceasca cu acestea, iar nu sa se îndeletniceasca cu vorbe mincinoase!"
Exo 5:10  ?i au ie?it capeteniile lor ?i slujba?ii poporului ?i au zis catre popor: "A?a zice Faraon: Nu va mai dau paie.
Exo 5:11  Merge?i voi în?iva ?i va aduna?i paie de unde ve?i gasi, dar din lucrul vostru nu vi se va scadea nimic!"
Exo 5:12  Atunci s-a risipit poporul în tot Egiptul, ca sa strânga trestie în loc de paie.
Exo 5:13  Iar slujba?ii îi sileau, zicând: "Împlini?i-va lucrul dat pentru fiecare zi, ca atunci când vi se dadea paie".
Exo 5:14  Iar pe slujba?ii pu?i peste ei de capeteniile lui Faraon îi bateau, zicând: "Pentru ce n-a?i facut ?i astazi numarul vostru de caramizi, ca ieri ?i ca alaltaieri?"
Exo 5:15  Atunci s-au dus slujba?ii fiilor lui Israel ?i au strigat catre Faraon, zicând: "Pentru ce faci a?a cu robii tai?
Exo 5:16  Paie nu se dau robilor tai, dar ne zic: Face?i caramida. ?i robii tai sunt batu?i ?i poporul tau e mereu vinovat".
Exo 5:17  Iar el le-a zis: "Sunte?i lene?i ?i de aceea zice?i: Haidem sa aducem jertfa Dumnezeului nostru.
Exo 5:18  Acum duce?i-va dar ?i munci?i! Paie nu vi se vor da, dar numarul de caramizi rânduit sa-l face?i!"
Exo 5:19  Deci au vazut slujba?ii fiilor lui Israel pacostea cazuta peste ei, când le spunea: "Numarul de caramizi lucrate pentru fiecare zi nu vi se va împu?ina".
Exo 5:20  ?i, ie?ind de la Faraon, s-au întâlnit ei cu Moise ?i cu Aaron, care veneau înaintea lor
Exo 5:21  ?i au zis catre ei: "Sa va vada ?i sa va judece Dumnezeu, ca ne-a?i facut urâ?i înaintea lui Faraon ?i a slujitorilor lui ?i le-a?i dat sabie la mâna, ca sa ne omoare".
Exo 5:22  Atunci Moise s-a întors la Domnul ?i a zis: "Doamne, pentru ce ai adus necazul acesta asupra acestui popor ?i la ce m-ai trimis pe mine?
Exo 5:23  Caci de când am mers eu la Faraon ?i i-am grait în numele Tau, el a început sa necajeasca ?i mai rau pe poporul acesta, iar de izbavit Tu nu l-ai izbavit pe poporul Tau".
Exo 6:1  Atunci a zis Domnul catre Moise: "Acum ai sa vezi ce am sa fac lui Faraon, ca sub lucrarea mâinii Mele celei tari el îi va lasa ?i sub lucrarea bra?ului Meu celui puternic el îi va alunga din pamântul sau".
Exo 6:2  Apoi a grait Domnul cu Moise ?i a zis catre el: "Eu sunt Domnul
Exo 6:3  ?i M-am aratat lui Avraam, lui Isaac ?i lui Iacov ca Dumnezeu Atotputernic, iar cu numele Meu de Domnul nu M-am facut cunoscut lor.
Exo 6:4  Mai mult: am facut legamânt cu ei, ca sa le dau pamântul Canaan, pamântul pribegiei lor, în care rataceau ei.
Exo 6:5  ?i, în sfâr?it, am auzit suspinul fiilor lui Israel, pe care îi ?in Egiptenii în robie, ?i Mi-am adus aminte de legamântul Meu cu voi.
Exo 6:6  Mergi dar de vorbe?te fiilor lui Israel ?i le spune: Eu sunt Domnul ?i am sa va scot de la munca cea grea a Egiptenilor ?i am sa va izbavesc din robia lor; am sa va izbavesc cu bra? înalt ?i cu pedepse mari;
Exo 6:7  Am sa va primesc sa-Mi fi?i popor, iar Eu sa va fiu Dumnezeu ?i voi ve?i cunoa?te ca Eu sunt Domnul Dumnezeul vostru, Care v-a scos din pamântul Egiptului ?i de sub munca apasatoare a Egiptenilor.
Exo 6:8  Apoi am sa va duc în pamântul acela pentru care Mi-am ridicat mâna sa-l dau lui Avraam, lui Isaac ?i lui Iacov, ?i pe care am sa-l dau voua în stapânire, caci Eu sunt Domnul!"
Exo 6:9  ?i a grait Moise a?a fiilor lui Israel; dar ei n-au ascultat pe Moise din pricina deznadejdii ?i a greuta?ii muncilor lor.
Exo 6:10  ?i iara?i a grait Domnul cu Moise ?i i-a zis:
Exo 6:11  "Intra ?i spune lui Faraon, regele Egiptului, ca sa lase pe fiii lui Israel sa iasa din ?ara lui!"
Exo 6:12  Dar Moise a grait înaintea Domnului ?i a zis: "Iata, fiii lui Israel nu ma asculta. Cum, dar, ma va asculta Faraon? ?i apoi eu sunt ?i gângav".
Exo 6:13  Domnul însa a grait lui Moise ?i Aaron ?i le-a poruncit sa spuna lui Faraon, regele Egiptului, sa dea drumul fiilor lui Israel din ?ara Egiptului.
Exo 6:14  Iata acum începatorii familiilor stramo?e?ti: Fiii lui Ruben, întâi-nascutul lui Israel: Enoh ?i Falu, He?ron ?i Carmi. Acestea sunt familiile lui Ruben.
Exo 6:15  Fiii lui Simeon: Iemuel ?i Iamin, Ohad ?i Iachin, ?ohar ?i Saul, fiii canaaneencii. Acestea sunt familiile lui Simeon.
Exo 6:16  Iar numele fiilor lui Levi, în?ira?i cum s-au nascut, sunt acestea: Gher?on, Cahat ?i Merari. Iar anii vie?ii lui Levi au fost o suta treizeci ?i ?apte.
Exo 6:17  Fiii lui Gher?on: Libni ?i ?imei, cu familiile lor.
Exo 6:18  Fiii lui Cahat: Amram, I?har, Hebron ?i Uziel. Iar anii vie?ii lui Cahat au fost o suta treizeci ?i trei de ani.
Exo 6:19  Fiii lui Merari: Mahli ?i Mu?i. Acesta este neamul lui Levi, dupa familiile lui.
Exo 6:20  Amram însa a luat de femeie pe Iochebed, fata unchiului sau, ?i aceasta i-a nascut pe Aaron ?i pe Moise, precum ?i pe Mariam, sora lor. Iar anii vie?ii lui Amram au fost o suta treizeci ?i ?apte de ani.
Exo 6:21  Fiii lui I?har: Core, Nefeg ?i Zicri.
Exo 6:22  Fiii lui Uziel: Misael, El?afan ?i Sitri.
Exo 6:23  Iar Aaron ?i-a luat de so?ie pe Elisaveta, fata lui Aminadab ?i sora lui Naason; aceasta i-a nascut pe Nadab ?i pe Abiud, pe Eleazar ?i pe Itamar.
Exo 6:24  Fiii lui Core: Asir, Elcana ?i Abiasaf. Acestea sunt familiile lui Core.
Exo 6:25  Eleazar, fiul lui Aaron, ?i-a luat de femeie pe una din fiicele lui Putiel ?i aceasta i-a nascut pe Finees. Ace?tia sunt începatorii familiilor stramo?e?ti ale levi?ilor.
Exo 6:26  Acesta este acel Aaron ?i acel Moise, carora Dumnezeu le-a zis: "Scoate?i pe fiii lui Israel din pamântul Egiptului cu o?tirea lor!"
Exo 6:27  ?i ace?tia au spus lui Faraon, regele Egiptului, sa dea drumul fiilor lui Israel din pamântul Egiptului. Acesta este acel Aaron ?i acesta este acel Moise
Exo 6:28  Din timpul când a grait Domnul cu Moise în ?ara Egiptului
Exo 6:29  ?i i-a zis lui: "Eu sunt Domnul!" Spune lui Faraon, regele Egiptului, câte-?i vorbesc Eu!"
Exo 6:30  Iar Moise a raspuns Domnului: "Iata eu sunt greoi la vorba. Cum dar ma va asculta Faraon?"
Exo 7:1  Raspuns-a Domnul lui Moise ?i i-a zis: "Iata, Eu fac din tine un dumnezeu pentru Faraon, iar Aaron, fratele tau, î?i va fi prooroc.
Exo 7:2  Tu dar vei grai lui Aaron toate câte î?i voi porunci, iar Aaron, fratele tau, va spune lui Faraon, ca sa lase pe fiii lui Israel sa iasa din pamântul lui.
Exo 7:3  Eu însa voi învârto?a inima lui Faraon ?i voi arata mul?imea semnelor Mele ?i a minunilor Mele în pamântul Egiptului.
Exo 7:4  Faraon nu va va asculta, dar Eu îmi voi pune mâna asupra Egiptului ?i voi scoate o?tirile Mele, pe poporul Meu, pe fiii lui Israel din pamântul Egiptului, cu mare izbânda.
Exo 7:5  Când voi întinde mâna Mea asupra Egiptului ?i voi scoate pe fiii lui Israel din mijlocul lui, atunci vor cunoa?te to?i Egiptenii ca Eu sunt Domnul".
Exo 7:6  Moise ?i Aaron s-au supus; cum le-a poruncit Domnul, a?a au facut.
Exo 7:7  Când au început a grai lui Faraon, Moise era de optzeci de ani, iar Aaron, fratele lui, de optzeci ?i trei de ani.
Exo 7:8  ?i a grait Domnul cu Moise ?i cu Aaron ?i a zis:
Exo 7:9  "Daca Faraon va va zice: Da-ne vreun semn sau vreo minune, atunci tu sa zici fratelui tau Aaron: Ia toiagul ?i-l arunca jos înaintea lui Faraon ?i înaintea slujitorilor lui, ?i se va face ?arpe".
Exo 7:10  S-au dus deci Moise ?i Aaron la Faraon ?i la slujitorii lui ?i au facut cum le poruncise Domnul: a aruncat Aaron toiagul sau înaintea lui Faraon ?i înaintea slujitorilor lui ?i s-a facut ?arpe.
Exo 7:11  Atunci a chemat ?i Faraon pe în?elep?ii Egiptului ?i pe vrajitori ?i au facut ?i vrajitorii Egiptenilor asemenea lucru cu vrajile lor:
Exo 7:12  Fiecare din ei ?i-a aruncat toiagul ?i s-a facut ?arpe. Dar toiagul lui Aaron a înghi?it toiegele lor.
Exo 7:13  De aceea s-a învârto?at inima lui Faraon ?i nu i-a ascultat, dupa cum spusese Domnul.
Exo 7:14  Zis-a Domnul catre Moise: "Inima lui Faraon se îndaratnice?te ?i nu lasa poporul.
Exo 7:15  Du-te dar la Faraon diminea?a; iata el are sa iasa la apa, iar tu sa stai în calea lui, pe malul râului, ?i toiagul acesta, care s-a prefacut în ?arpe, sa-l iei în mâna ta;
Exo 7:16  ?i sa zici lui Faraon: Domnul, Dumnezeul Evreilor m-a trimis la tine sa-?i spun: Lasa pe poporul Meu sa-Mi faca slujba în pustie; ?i iata pâna acum nu M-ai ascultat.
Exo 7:17  A?a zice Domnul: Din aceasta vei cunoa?te ca Eu sunt Domnul: iata, cu acest toiag, care e în mâna mea, voi lovi apa din râu ?i se va preface în sânge;
Exo 7:18  Pe?tele din râu va muri, râul se va împu?i ?i Egiptenii nu vor putea bea apa din râu".
Exo 7:19  ?i a mai zis Domnul catre Moise: "Sa zici lui Aaron, fratele tau: Ia toiagul în mâna ?i întinde-?i mâna asupra apelor Egiptului: asupra râurilor lui, asupra lacurilor lui ?i asupra oricarei adunari de apa; ?i se vor preface în sânge ?i va fi sânge în toata ?ara Egiptului, în vasele de lemn ?i în cele de piatra!"
Exo 7:20  ?i au facut Moise ?i Aaron cum le-a poruncit Domnul: a ridicat Aaron toiagul sau ?i a lovit apa râului, înaintea ochilor lui Faraon ?i înaintea ochilor slujitorilor lui, ?i toata apa din râu s-a prefacut în sânge.
Exo 7:21  Atunci pe?tele din râu a murit, râul s-a împu?it ?i Egiptenii nu puteau sa bea apa din râu; ?i era sânge în toata ?ara Egiptului.
Exo 7:22  ?i au facut a?a ?i magii Egipteni cu vrajile lor. De aceea s-a învârto?at inima lui Faraon ?i nu i-a ascultat, precum le spusese Domnul.
Exo 7:23  Întorcându-se, Faraon a intrat în casa sa ?i nu a pus la inima aceasta.
Exo 7:24  Atunci au sapat to?i Egiptenii în preajma râului, ca sa gaseasca sa bea apa, caci din râu nu puteau sa bea apa.
Exo 7:25  Se împlinisera ?apte zile de când lovise Domnul apa.
Exo 8:1  Atunci a zis Domnul catre Moise: "Intra la Faraon ?i-i zi: A?a graie?te Domnul: Lasa pe poporul Meu, ca sa-Mi slujeasca.
Exo 8:2  Iar de nu vei vrea sa-l la?i, iata Eu voi lovi toate ?inuturile tale cu broa?te.
Exo 8:3  Râul va mi?una de broa?te ?i, ie?ind, acestea se vor sui în casele tale, în dormitoarele tale, pe paturile tale, în casele slujitorilor tai ?i ale poporului tau, în cuptoarele tale ?i în aluaturile tale;
Exo 8:4  Pe tine, pe poporul tau ?i pe toate slugile tale se vor sui broa?te".
Exo 8:5  ?i a mai zis Domnul catre Moise: "Spune lui Aaron, fratele tau: întinde toiagul cu mâna ta spre râuri, spre lacuri ?i spre bal?i ?i fa sa iasa broa?te în pamântul Egiptului!"
Exo 8:6  ?i ?i-a întins Aaron mâna sa asupra apelor Egiptului ?i ele au scos broa?te; ?i au ie?it broa?te ?i au acoperit pamântul Egiptului.
Exo 8:7  Dar au facut asemenea ?i vrajitorii Egiptenilor cu vrajile lor ?i au scos broa?te în pamântul Egiptului.
Exo 8:8  Atunci a chemat Faraon pe Moise ?i pe Aaron ?i a zis: "Ruga?i-va pentru mine Domnului sa departeze broa?tele de la mine ?i de la poporul meu ?i voi lasa poporul lui Israel sa jertfeasca Domnului!"
Exo 8:9  Moise însa a zis catre Faraon: "Soroce?te-mi însu?i când sa ma rog pentru tine, pentru slugile tale ?i pentru poporul tau, ca sa piara broa?tele de la tine, de la poporul tau ?i din casele voastre ?i sa ramâna numai în râu",
Exo 8:10  Iar el a zis: "Mâine". Zis-a Moise: "Va fi cum ai zis, ca sa ?tii ca nu este altul ca Domnul Dumnezeul nostru.
Exo 8:11  Se vor departa broa?tele de la tine, din casele tale, din ?arine, de la slugile tale ?i de la poporul tau ?i numai în râu vor ramâne".
Exo 8:12  Ie?ind deci Moise ?i Aaron de la Faraon, a strigat Moise catre Domnul ca sa piara broa?tele pe care le trimisese împotriva lui Faraon.
Exo 8:13  ?i a facut Domnul dupa cuvântul lui Moise ?i au murit broa?tele de prin case, de prin cur?i ?i ds prin ?arini;
Exo 8:14  ?i le-au adunat gramezi, gramezi, ?i s-a împu?it pamântul.
Exo 8:15  Vazând însa ca s-a facut u?urare, Faraon ?i-a învârto?at inima ?i nu i-a ascultat, dupa cum spusese Domnul.
Exo 8:16  Atunci a zis Domnul catre Moise: "Spune lui Aaron: Întinde-?i toiagul tau cu mâna ?i love?te ?arâna pamântului ?i vor fi ?ân?ari pe oameni, pe vite, pe Faraon ?i în casa lui ?i pe slugile lui; toata ?arâna pamântului se va face ?ân?ari în tot pamântul Egiptului".
Exo 8:17  ?i au facut ei a?a: Aaron ?i-a întins toiagul cu mâna sa ?i a lovit ?arâna pamântului ?i s-au ivit ?ân?ari pe oameni ?i pe vite. Toata ?arâna pamântului s-a facut ?ân?ari în tot pamântul Egiptului.
Exo 8:18  Au încercat atunci ?i magii cu vrajile lor sa faca ?ân?ari, dar n-au putut. ?i au ramas ?ân?ari pe oameni ?i pe vite.
Exo 8:19  ?i au zis magii catre Faraon: "Acesta e degetul lui Dumnezeu!" Dar inima lui Faraon s-a învârto?at ?i nu i-a ascultat, dupa cum spusese Domnul.
Exo 8:20  Zis-a Domnul catre Moise: "Scoala mâine de diminea?a ?i ie?i înaintea lui Faraon în vremea când el are sa iasa la apa, iar tu sa-i zici: A?a graie?te Domnul: Lasa pe poporul Meu ca sa-Mi slujeasca în pustie!
Exo 8:21  Daca însa nu vei lasa pe poporul Meu, iata Eu voi trimite asupra ta, asupra slujitorilor tai, asupra poporului tau ?i asupra caselor voastre tauni ?i se vor umple casele Egiptenilor de tauni ?i pamântul pe care traiesc ei.
Exo 8:22  ?i voi osebi în ziua aceea pamântul Go?en în care locuie?te poporul Meu, ca acolo nu vor fi tauni ca sa ?tii ca Eu sunt Domnul, în mijlocul acestei ?ari.
Exo 8:23  Voi face deosebire între poporul Meu ?i poporul tau ?i chiar mâine va fi semnul acesta pe pamânt".
Exo 8:24  ?i a facut Domnul a?a ?i a venit mul?ime de tauni în casa lui Faraon, în casele slujitorilor lui ?i în tot pamântul Egiptului, încât s-a pustiit ?ara de tauni.
Exo 8:25  Atunci a chemat Faraon pe Moise ?i pe Aaron ?i a zis: "Merge?i ?i aduce?i jertfa Domnului Dumnezeului vostru în ?ara!"
Exo 8:26  Moise însa a zis: "Nu este cu putin?a sa se faca a?a, pentru ca cele ce aducem noi jertfa Domnului Dumnezeului nostru sunt urâciune înaintea Egiptenilor. ?i de vom jertfi noi înaintea Egiptenilor cele ce sunt urâciune pentru ei, nu ne vor ucide, oare, cu pietre?
Exo 8:27  De aceea ne vom duce în pustie cale de trei zile ?i vom aduce acolo jertfa Domnului Dumnezeului nostru, dupa cum ne va zice Domnul".
Exo 8:28  Zis-a Faraon: "Eu va voi lasa sa aduce?i jertfa Domnului Dumnezeului vostru, în pustie, dar sa nu va duce?i departe. Ruga?i-va dar Domnului pentru mine!"
Exo 8:29  Iar Moise a zis: "Iata, cum voi ie?i de la tine, ma voi ruga Domnului Dumnezeu ?i mâine se vor îndeparta taunii de la Faraon, de la slujitorii lui ?i de la poporul lui, dar Faraon sa înceteze a mai în?ela, nedând drumul poporului sa aduca jertfa Domnului!"
Exo 8:30  ?i ie?ind Moise de la Faraon, s-a rugat lui Dumnezeu.
Exo 8:31  ?i a facut Domnul dupa cum zisese Moise: a îndepartat taunii de la Faraon, de la slujitorii lui ?i de la poporul lui ?i n-a mai ramas nici unul.
Exo 8:32  Dar Faraon ?i-a învârto?at inima ?i de data aceasta ?i n-a lasat poporul sa se duca.
Exo 9:1  Atunci a zis Domnul catre Moise: "Intra la Faraon ?i-i spune: Acestea zice Domnul Dumnezeul Evreilor: Lasa pe poporul Meu sa-Mi slujeasca!
Exo 9:2  Iar de nu vei vrea sa la?i pe poporul Meu, ci-l vei mai ?ine,
Exo 9:3  Iata, mâna Domnului va fi peste vitele tale cele de la câmp: peste cai, peste asini, peste camile, peste boi ?i oi ?i va fi moarte foarte mare.
Exo 9:4  Dar va face Domnul osebire în ziua aceea între vitele Israeli?ilor ?i vitele Egiptenilor: din toate vitele fiilor lui Israel nu va muri nici una".
Exo 9:5  ?i a pus Domnul soroc ?i a zis: "Mâine va face Domnul aceasta în ?ara aceasta!"
Exo 9:6  ?i a doua zi a facut Domnul aceasta ?i au murit toate vitele Egiptenilor, iar din vitele fiilor lui Israel n-a murit nici una.
Exo 9:7  Atunci a trimis Faraon sa afle ?i iata din toate vitele fiilor lui Israel nu murise nici una. Dar inima lui Faraon s-a învârto?at ?i nu a lasat poporul sa se duca.
Exo 9:8  Iara?i a grait Domnul cu Moise ?i cu Aaron ?i a zis: "Lua?i-va câte o mâna plina de cenu?a din cuptor ?i s-o arunce Moise spre cer înaintea lui Faraon ?i a slujitorilor lui.
Exo 9:9  ?i se va stârni pulbere în tot pamântul Egiptului ?i vor fi pe oameni ?i pe vite rani ?i ba?ici usturatoare în toata ?ara Egiptului".
Exo 9:10  Deci, au luat ei cenu?a din cuptor, au mers înaintea lui Faraon, a aruncat-o Moise spre cer ?i s-au facut bube cu puroi pe oameni ?i pe vite;
Exo 9:11  ?i magii n-au putut sta împotriva lui Moise din pricina ranilor, pentru ca erau bube pe ei ?i în tot Egiptul.
Exo 9:12  Dar Domnul a învârto?at inima lui Faraon ?i nu i-a ascultat, cum zisese Domnul lui Moise.
Exo 9:13  Zis-a Domnul catre Moise: "Sa te scoli mâine de diminea?a, sa te înfa?i?ezi lui Faraon ?i sa-i zici: A?a graie?te Domnul Dumnezeul Evreilor: Lasa pe poporul Meu ca sa-Mi slujeasca,
Exo 9:14  Fiindca de data aceasta voi trimite toate pedepsele Mele împotriva ta, a slugilor tale ?i a poporului tau, ca sa vezi ca nu este altul asemenea Mie în tot pamântul.
Exo 9:15  De Mi-a? fi întins mâna ?i te-a? fi lovit pe tine ?i pe poporul tau cu ciuma, tu ai fi fost ?ters de pe fa?a pamântului;
Exo 9:16  Dar te-am cru?at, ca sa-Mi arat puterea Mea ?i ca sa se vesteasca numele Meu în tot pamântul,
Exo 9:17  ?i tu tot te mai împotrive?ti poporului Meu ?i nu-l la?i.
Exo 9:18  Iata, Eu voi ploua mâine, pe vremea asta, grindina foarte multa, cum n-a mai fost în Egipt de la întemeierea lui ?i pâna în ziua aceasta.
Exo 9:19  Trimite dar acum sa adune turmele tale ?i toate câte ai la câmp, ca asupra tuturor oamenilor ?i vitelor, care vor fi în ?arina ?i nu vor intra în casa, va cadea grindina ?i vor muri".
Exo 9:20  Acei dintre robii lui Faraon, care s-au temut de Domnul, au adunat în graba pe oamenii ?i turmele lor acasa,
Exo 9:21  Iar cei ce n-au luat aminte la cuvântul Domnului, aceia ?i-au lasat slugile ?i vitele lor în câmp.
Exo 9:22  ?i a zis Domnul catre Moise: "Întinde mâna ta spre cer ?i va cadea grindina peste tot pamântul Egiptului: peste oameni, peste turme ?i peste toata iarba câmpului din pamântul Egiptului!"
Exo 9:23  Atunci ?i-a întins Moise mâna spre cer ?i a slobozit Domnul tunete, grindina ?i foc pe pamânt; ?i a plouat Domnul grindina în pamântul Egiptului.
Exo 9:24  Aceasta a fost o grindina foarte mare ?i printre grindina ardea foc, cum nu mai fusese în tot pamântul Egiptului, de când se a?ezasera oamenii pe el.
Exo 9:25  Grindina aceasta a batut în tot pamântul Egiptului, tot ce era pe câmp, oameni ?i dobitoace; toata iarba câmpului a batut-o grindina ?i to?i pomii de pe câmp i-a rupt grindina.
Exo 9:26  Numai în ?inutul Go?en, unde traiau fiii lui Israel, n-a fost grindina.
Exo 9:27  Atunci trimi?ând, Faraon a chemat pe Moise ?i pe Aaron ?i a zis catre ei: "Acum vad ca am pacatuit! Domnul este drept, iar eu ?i poporul meu suntem vinova?i.
Exo 9:28  Ruga?i-va Domnului pentru mine, sa înceteze tunetele, grindina ?i focul pe pamânt ?i va voi lasa ?i, mai mult, nu va voi împiedica!"
Exo 9:29  Iar Moise i-a zis: "Îndata ce voi ie?i din ora?, voi întinde mâna mea spre cer, catre Domnul, ?i vor înceta tunetele; nu va mai fi nici grindina, nici ploaie, ca sa cuno?ti ca al Domnului este pamântul.
Exo 9:30  Dar ?tiu ca tu ?i slujitorii tai nu va teme?i înca de Domnul Dumnezeu".
Exo 9:31  Atunci inul ?i orzul s-au stricat, pentru ca orzul era înspicat ?i inul în floare.
Exo 9:32  Iar grâul ?i ovazul nu s-au stricat, pentru ca acestea erau mai târzii.
Exo 9:33  Ie?ind deci Moise de la Faraon ?i din cetate ?i întinzându-?i mâinile catre Domnul, au încetat tunetele ?i grindina ?i s-a oprit ploaia.
Exo 9:34  Vazând însa ca au încetat ploaia, grindina ?i tunetele, pacatuit-a Faraon înainte ?i ?i-a învârto?at inima ?i el ?i slugile sale.
Exo 9:35  ?i învârto?ata fiind inima lui Faraon ?i a slugilor lui, el n-a lasat pe fiii lui Israel sa plece cum poruncise Dumnezeu prin mâna lui Moise.
Exo 10:1  Atunci a grait iara?i Domnul cu Moise ?i a zis: "Intra la Faraon, ca i-am învârto?at inima lui ?i a slugilor lui, ca sa arat între ei pe rând aceste semne ale Mele;
Exo 10:2  Ca sa istorisi?i în auzul fiilor vo?tri ?i al fiilor fiilor vo?tri câte am facut în Egipt ?i semnele Mele, pe care le-am aratat într-însul, ?i ca sa cunoa?te?i ca Eu sunt Domnul!"
Exo 10:3  ?i a intrat Moise ?i Aaron la Faraon ?i i-au zis: "A?a graie?te Domnul Dumnezeul Evreilor: Pâna când nu vei vrea sa te smere?ti înaintea Mea? Lasa pe poporul Meu, ca sa-Mi slujeasca!
Exo 10:4  Iar de nu vei lasa pe poporul Meu, iata mâine, pe vremea asta, voi aduce lacuste multe în toate hotarele tale;
Exo 10:5  ?i vor acoperi ele fa?a pamântului, încât pamântul nu se va putea vedea, ?i vor mânca tot ce a mai ramas la voi, pe pamânt, nestricat de grindina; to?i pomii ce cresc prin câmpiile voastre;
Exo 10:6  Vor umplea casele tale, casele tuturor slugilor tale ?i toate casele în tot pamântul Egiptenilor, cum n-au vazut parin?ii tai, nici parin?ii parin?ilor tai de când traiesc ei pe pamânt ?i pâna în ziua de astazi". Apoi s-a întors Moise ?i a ie?it de la Faraon.
Exo 10:7  Atunci dregatorii lui Faraon au zis catre acesta: "Oare mult are sa ne chinuiasca omul acesta? Da drumul oamenilor acestora, ca sa faca slujba Dumnezeului lor! Sau vrei sa vezi Egiptul pierind?"
Exo 10:8  ?i ei au întors pe Moise ?i pe Aaron la Faraon; iar Faraon a zis catre ei: "Duce?i-va ?i face?i slujba Domnului Dumnezeului vostru! Dar cine sunt cei care trebuie sa mearga?"
Exo 10:9  Raspuns-a Moise: "Vom merge cu cei tineri ?i cu cei batrâni ai no?tri, cu fiii no?tri, cu fiicele noastre, cu oile noastre ?i cu boii no?tri, caci e sarbatoarea Domnului Dumnezeului nostru".
Exo 10:10  Faraon însa le-a zis: "Fie a?a! Dumnezeu cu voi! Eu sunt gata sa va dau drumul. Dar la ce sa va duce?i cu copiii? Se vede ca ave?i gând rau.
Exo 10:11  Nu! Duce?i-va numai barba?ii ?i face?i slujba Domnului, cum a?i cerut!" ?i au fost da?i afara de la Faraon.
Exo 10:12  Atunci a zis Domnul catre Moise: "Întinde-?i mâna ta asupra pamântului Egiptului ?i vor navali lacustele asupra pamântului Egiptului ?i vor mânca toata iarba pamântului, toate roadele pomilor ?i tot ce a ramas nestricat de grindina".
Exo 10:13  Deci ?i-a ridicat Moise toiagul sau asupra pamântului Egiptului ?i a adus Domnul asupra pamântului acestuia vânt de la rasarit toata ziua aceea ?i toata noaptea ?i, când s-a facut ziua, vântul de la rasarit a adus lacuste.
Exo 10:14  ?i au navalit ele în tot pamântul Egiptului, s-au a?ezat în toate ?inuturile Egiptului mul?ime multa; asemenea lacuste n-au mai fost ?i nu vor mai fi.
Exo 10:15  ?i au acoperit ele toata ?ara, cât nu se mai vedea pamântul; ?i au mâncat toata iarba pamântului ?i toate roadele pomilor, care nu fusesera stricate de grindina; ?i n-a ramas nici un fir de verdea?a, nici în arbori, nici în iarba câmpului în tot pamântul Egiptului.
Exo 10:16  Atunci Faraon a chemat în graba pe Moise ?i pe Aaron ?i le-a zis: "Gre?it-am înaintea Domnului Dumnezeului vostru ?i înaintea voastra!
Exo 10:17  Ierta?i-mi acum înca o data gre?eala mea ?i va ruga?i Domnului Dumnezeului vostru sa abata în orice chip de la mine prapadul acesta!"
Exo 10:18  ?i ie?ind de la Faraon, Moise s-a rugat lui Dumnezeu,
Exo 10:19  ?i Domnul a stârnit vânt puternic de la apus ?i acesta a dus lacustele ?i le-a aruncat în Marea Ro?ie ?i n-a ramas nici o lacusta în tot pamântul Egiptului.
Exo 10:20  Dar Domnul a învârto?at inima lui Faraon ?i acesta n-a dat drumul fiilor lui Israel.
Exo 10:21  Atunci a zis Domnul catre Moise: "Întinde mâna ta spre cer ?i se va face întuneric în pamântul Egiptului, încât sa-l pipai cu mâna".
Exo 10:22  ?i ?i-a întins Moise mâna sa spre cer ?i s-a facut întuneric bezna trei zile în tot pamântul Egiptului,
Exo 10:23  De nu se vedea om cu om, ?i nimeni nu s-a urnit de la locul sau trei zile. Iar la fiii lui Israel a fost lumina peste tot în locuin?ele lor.
Exo 10:24  Atunci a chemat Faraon pe Moise ?i pe Aaron ?i le-a zis: "Duce?i-va ?i face?i slujba Domnului Dumnezeului vostru, dar sa ramâna aici vitele voastre marunte ?i mari, iar copiii sa mearga cu voi".
Exo 10:25  Moise însa a zis: "Ba nu, ci da-ne vite pentru jertfele ?i arderile de tot ce avem sa aducem Domnului Dumnezeului nostru.
Exo 10:26  Deci, sa mearga cu noi ?i turmele noastre ?i sa nu ramâna nici un picior, caci din ele avem sa luam ca sa aducem jertfa Domnului Dumnezeului nostru; dar, pâna nu vom ajunge acolo, nu ?tim ce avem sa aducem jertfa Domnului Dumnezeului nostru".
Exo 10:27  Domnul a învârto?at inima lui Faraon ?i el n-a vrut sa le dea drumul,
Exo 10:28  Ci a zis Faraon catre Moise: "Du-te de aici! Dar baga de seama sa nu te mai ara?i în fa?a mea, caci în ziua când vei vedea fa?a mea, vei muri".
Exo 10:29  Raspuns-a Moise: "Cum ai zis, a?a va fi. Mai mult nu voi mai vedea fa?a ta!"
Exo 11:1  Dupa aceea a zis Domnul catre Moise: "Înca o plaga voi mai aduce asupra lui Faraon ?i asupra Egiptului ?i dupa aceea va vor da drumul de aici. Dar când va vor da drumul, cu grabire va vor alunga de aici.
Exo 11:2  Spune dar poporului în taina, ca fiecare barbat de la vecinul sau ?i fiecare femeie de la vecina ei sa ceara împrumut vase de argint ?i vase de aur ?i haine".
Exo 11:3  ?i a dat Domnul poporului Sau trecere înaintea Egiptenilor ?i ace?tia le-au împrumutat cele cerute. Dar ?i Moise ajunsese mare foarte în pamântul Egiptului, înaintea lui Faraon ?i a slujitorilor lui Faraon ?i a tot poporul.
Exo 11:4  ?i a zis Moise: "A?a graie?te Domnul: La miezul nop?ii voi trece prin Egipt
Exo 11:5  ?i va muri tot întâiul nascut în pamântul Egiptului, de la întâiul nascut al lui Faraon, care urmeaza sa ?ada pe tronul sau, pâna la întâiul nascut al roabei de la râ?ni?a ?i pâna la întâiul nascut al dobitoacelor.
Exo 11:6  ?i va fi plângere mare în tot pamântul Egiptului, cum n-a mai fost ?i cum nu va mai fi.
Exo 11:7  Iar la to?i fiii lui Israel nici câine nu va latra, nici la om, nici la dobitoc, ca sa cunoa?te?i ce deosebire face Domnul între Egipteni ?i Israeli?i.
Exo 11:8  ?i se vor pogorî to?i ace?ti slujitori ai tai la mine ?i, închinându-se mie, vor zice: Ie?i împreuna cu tot poporul tau, pe care-l pova?uie?ti tu. ?i dupa aceea voi ?i ie?i". ?i a ie?it Moise de la Faraon înfierbântat de mânie.
Exo 11:9  Apoi a zis Domnul catre Moise: "Nu va va asculta nici acum Faraon, ca sa se înmul?easca semnele Mele ?i minunile Mele în pamântul Egiptului!"
Exo 11:10  A facut deci Moise ?i Aaron toate semnele ?i minunile acestea înaintea lui Faraon. Dar Domnul a învârto?at inima lui Faraon ?i el n-a ascultat sa lase pe Israel sa iasa din pamântul sau.
Exo 12:1  Apoi a grait Domnul cu Moise ?i Aaron în pamântul Egiptului ?i le-a zis:
Exo 12:2  "Luna aceasta sa va fie începutul lunilor, sa va fie întâia între lunile anului.
Exo 12:3  Vorbe?te deci la toata ob?tea fiilor lui Israel ?i le spune: În ziua a zecea a lunii acesteia sa-?i ia fiecare din capii de familie un miel; câte un miel de familie sa lua?i fiecare.
Exo 12:4  Iar daca vor fi pu?ini în familie, încât sa nu fie deajuns ca sa poata mânca mielul, sa ia cu sine de la vecinul cel mai aproape de dânsul un numar de suflete: numara?i-va la un miel atâ?ia cât pot sa-l manânce.
Exo 12:5  Mielul sa va fie de un an, parte barbateasca ?i fara meteahna, ?i sa lua?i sau un miel, sau un ied,
Exo 12:6  Sa-l ?ine?i pâna în ziua a paisprezecea a lunii acesteia ?i atunci toata adunarea ob?tii fiilor lui Israel sa-l junghie catre seara.
Exo 12:7  Sa ia din sângele lui ?i sa unga amândoi u?orii ?i pragul cel de sus al u?ii casei unde au sa-l manânce.
Exo 12:8  ?i sa manânce în noaptea aceea carnea lui fripta la foc; dar s-o manânce cu azima ?i cu ierburi amare.
Exo 12:9  Dar sa nu-l mânca?i nefript deajuns sau fiert în apa, ci sa mânca?i totul fript bine pe foc, ?i capul cu picioarele ?i maruntaiele.
Exo 12:10  Sa nu lasa?i din el pe a doua zi ?i oasele lui sa nu le zdrobi?i. Ceea ce va ramâne pe a doua zi sa arde?i în foc.
Exo 12:11  Sa-l mânca?i însa a?a: sa ave?i coapsele încinse, încal?amintea în picioare ?i toiegele în mâinile voastre; ?i sa-l mânca?i cu graba, caci este Pa?tile Domnului.
Exo 12:12  În noaptea aceea voi trece peste pamântul Egiptului ?i voi lovi pe tot întâiul nascut în pamântul Egiptului, al oamenilor ?i al dobitoacelor, ?i voi face judecata asupra tuturor dumnezeilor în pamântul Egiptului, caci Eu sunt Domnul.
Exo 12:13  Iar la voi sângele va fi semn pe casele în care va ve?i afla: voi vedea sângele ?i va voi ocoli ?i nu va fi între voi rana omorâtoare, când voi lovi pamântul Egiptului.
Exo 12:14  Ziua aceea sa fie spre pomenire ?i sa praznui?i într-însa sarbatoarea Domnului, din neam în neam; ca a?ezare ve?nica s-o praznui?i.
Exo 12:15  ?apte zile sa mânca?i azime; din ziua întâi sa departa?i din casele voastre dospitura, caci cine va mânca dospit din ziua întâi pâna în ziua a ?aptea, sufletul aceluia se va stârpi din Israel.
Exo 12:16  În ziua întâi sa ave?i adunare sfânta, în ziua a ?aptea iar adunare sfânta; ?i în acele zile sa nu face?i nici un fel de lucru decât numai cele ce trebuie fiecaruia de mâncat, numai acelea sa vi le face?i.
Exo 12:17  Pazi?i sarbatoarea azimilor, ca în ziua aceea am scos taberele voastre din pamântul Egiptului; pazi?i ziua aceasta în neamul vostru ca a?ezamânt ve?nic.
Exo 12:18  Începând din seara zilei a paisprezecea a lunii întâi ?i pâna în seara zilei a douazeci ?i una a aceleia?i luni, sa mânca?i pâine nedospita.
Exo 12:19  ?apte zile sa nu se afle dospitura în casele voastre; tot cel care va mânca dospit, sufletul acela se va stârpi din ob?tea lui Israel, fie strain sau ba?tina? al pamântului aceluia.
Exo 12:20  Tot ce e dospit sa nu mânca?i, ci în toate a?ezarile voastre sa mânca?i azima".
Exo 12:21  Apoi a chemat Moise pe to?i batrânii fiilor lui Israel ?i le-a zis: "Mergeri ?i va lua?i miei dupa familiile voastre ?i junghia?i Pa?tile.
Exo 12:22  Dupa aceea sa lua?i un manunchi de isop ?i, muindu-l în sângele strâns de la miel într-un vas, sa unge?i pragul de sus ?i amândoi u?orii  u?ii cu sângele cel din vas, iar voi sa nu ie?i?i nici unul din casa pâna diminea?a;
Exo 12:23  Caci are sa treaca Domnul sa loveasca Egiptul; ?i vazând sângele de pe pragul de sus ?i de pe cei doi u?ori, Domnul va trece pe lânga u?a ?i nu va îngadui pierzatorului sa intre în casele voastre, ca sa va loveasca.
Exo 12:24  Pazi?i acestea ca un a?ezamânt ve?nic pentru voi ?i pentru copiii vo?tri.
Exo 12:25  Iar dupa ce ve?i intra în pamântul pe care Domnul îl va da voua, cum a zis, sa pazi?i rânduiala aceasta.
Exo 12:26  ?i când va vor zice copiii vo?tri: Ce înseamna rânduiala aceasta?
Exo 12:27  Sa le spune?i: Aceasta este jertfa ce o aducem de Pa?ti Domnului, Care în Egipt a trecut pe lânga casele fiilor lui Israel, când a lovit Egiptul, iar casele noastre le-a izbavit". ?i s-a plecat poporul ?i s-a închinat.
Exo 12:28  Au mers deci fiii lui Israel ?i au facut toate cum poruncise Domnul lui Moise ?i Aaron; a?a au facut.
Exo 12:29  Iar la miezul nop?ii a lovit Domnul pe to?i întâi-nascu?ii în pamântul Egiptului, de la întâi-nascutul lui Faraon, care ?edea pe tron, pâna la întâi-nascutul robului, care sta în închisoare, ?i pe to?i întâi-nascu?ii dobitoacelor.
Exo 12:30  ?i s-a sculat noaptea Faraon însu?i, toate slugile lui ?i to?i Egiptenii, ?i s-a facut bocet mare în toata ?ara Egiptului, caci nu era casa unde sa nu fie mort.
Exo 12:31  În aceea?i noapte a chemat Faraon pe Moise ?i pe Aaron ?i le-a zis: "Scula?i-va ?i ie?i?i din pamântul poporului meu! ?i voi ?i fiii lui Israel! ?i duce?i-va de face?i slujba Domnului Dumnezeului vostru, precum a?i zis.
Exo 12:32  Lua?i cu voi ?i oile ?i boii vo?tri, cum a?i cerut, ?i va duce?i ?i ma binecuvânta?i ?i pe mine!"
Exo 12:33  ?i sileau Egiptenii pe poporul evreu sa iasa degraba din ?ara aceea, caci ziceau: "Pierim cu to?ii!"
Exo 12:34  Atunci poporul a luat pe umeri aluatul sau pâna a nu se dospi, cu cove?ile învelite în hainele lor.
Exo 12:35  ?i facând fiii lui Israel cum le poruncise Moise, ei au cerut de la Egipteni vase de argint ?i de aur ?i haine;
Exo 12:36  Iar Domnul a dat poporului Sau trecere înaintea Egiptenilor, ca sa-i dea tot ce a cerut. ?i astfel au fost prada?i Egiptenii.
Exo 12:37  Fiii lui Israel au plecat din Ramses spre Sucot, ca fa ?ase sute de mii de barba?i pede?tri, afara de copii.
Exo 12:38  ?i a mai ie?it împreuna cu ei mul?ime de oameni de felurite neamuri, ?i oi, ?i boi, ?i turme foarte mari.
Exo 12:39  Iar din aluatul ce l-au scos din Egipt au copt azime, ca nu se dospise înca, pentru ca i-au scos Egiptenii ?i nu putusera zabovi nici macar sa-?i faca de mâncare pentru drum.
Exo 12:40  Timpul însa, cât fiii lui Israel ?i parin?ii lor au trait în Egipt ?i în ?ara Canaan, a fost de patru sute treizeci de ani.
Exo 12:41  Iar dupa trecerea celor patru sute treizeci de ani a ie?it toata o?tirea Domnului din pamântul Egiptului, noaptea.
Exo 12:42  Aceasta a fost noaptea de priveghere a Domnului pentru scoaterea lor din ?ara Egiptului ?i pe aceasta noapte de priveghere pentru Domnul o vor pazi to?i fiii lui Israel din neam în neam.
Exo 12:43  Dupa aceea a zis Domnul catre Moise ?i Aaron: "Rânduiala Pa?telui este aceasta: Nimeni din cei de alt neam sa nu manânce din el.
Exo 12:44  Dar tot robul cumparat cu bani ?i taiat împrejur sa manânce din el.
Exo 12:45  Strainul ?i simbria?ul a?ijderea sa nu manânce din el.
Exo 12:46  Sa se manânce în aceea?i casa; sa nu lasa?i pe a doua zi; carnea sa nu o scoate?i afara din casa ?i oasele sa nu le zdrobi?i.
Exo 12:47  Sa-l praznuiasca toata ob?tea fiilor lui Israel.
Exo 12:48  Iar de va veni la voi vreun strain sa faca Pa?tile Domnului, sa tai împrejur pe to?i cei de parte barbateasca ai lui ?i numai atunci sa-l savâr?easca ?i va fi ca ?i locuitorul de ba?tina al ?arii; dar tot cel netaiat împrejur sa nu manânce din el.
Exo 12:49  O lege sa fie ?i pentru ba?tina? ?i pentru strainul ce se va a?eza la voi!"
Exo 12:50  ?i au facut fiii lui Israel cum poruncise Domnul lui Moise ?i Aaron; a?a au facut.
Exo 12:51  Deci, în ziua aceea a scos Domnul pe fiii lui Israel din ?ara Egiptului, cu o?tirea lor.
Exo 13:1  În vremea aceea a vorbit Domnul cu Moise ?i i-a zis:
Exo 13:2  "Sa-Mi sfin?e?ti pe tot întâiul nascut, pe tot cel ce se na?te întâi la fiii lui Israel, de la om pâna la dobitoc, ca este al Meu!"
Exo 13:3  Iar Moise a zis catre popor: "Sa va aduce?i aminte de ziua aceasta, în care a?i ie?it din pamântul Egiptului, din casa robiei, caci cu mâna tare v-a scos Domnul de acolo ?i sa nu mânca?i dospit;
Exo 13:4  Ca astazi ie?i?i voi, în luna Aviv.
Exo 13:5  Iar când te va duce Domnul Dumnezeul tau în ?ara Canaaneilor, a Heteilor, a Amoreilor, a Heveilor, a Iebuseilor, a Ghergheseilor ?i a Ferezeilor, pentru care S-a jurat El parin?ilor tai sa-?i dea ?ara unde curge miere ?i lapte, sa faci slujba aceasta în aceasta luna.
Exo 13:6  ?apte zile sa manânci azime, iar în ziua a ?aptea este sarbatoarea Domnului:
Exo 13:7  Azime sa mânca?i ?apte zile ?i sa nu se gaseasca la tine pâine dospita, nici aluat dospit în toate hotarele tale.
Exo 13:8  În ziua aceea sa spui fiului tau ?i sa zici: Acestea sunt pentru cele ce a facut Domnul cu mine, când am ie?it din Egipt.
Exo 13:9  Sa fie acestea ca un semn pe mâna ta ?i aducere aminte înaintea ochilor tai, pentru ca legea Domnului sa fie în gura ta, caci cu mâna tare te-a scos Domnul Dumnezeu din Egipt.
Exo 13:10  Sa pazi?i dar legea aceasta din an în an, la vremea hotarâta.
Exo 13:11  ?i când te va duce Domnul Dumnezeul tau în ?ara Canaanului, cum S-a jurat ?ie ?i parin?ilor tai, ?i ti-o va da ?ie,
Exo 13:12  Atunci sa osebe?ti Domnului pe tot cel de parte barbateasca de la oameni, care se na?te întâi; ?i pe tot cel de parte barbateasca, care se va na?te întâi din turmele sau de la vitele ce vei avea, sa-l închini Domnului.
Exo 13:13  Pe tot întâi-nascutul de la asina sa-l rascumperi cu un miel; iar de nu-l vei rascumpara, îi vei frânge gâtul; sa rascumperi ?i pe tot întâi-nascutul din oameni în neamul tau.
Exo 13:14  Când însa te va întreba dupa aceea fiul tau ?i va zice: Ce înseamna aceasta?, sa-i spui: Cu mâna puternica ne-a scos Domnul din pamântul Egiptului, din casa robiei.
Exo 13:15  Ca atunci când se îndaratnicea Faraon sa ne dea drumul, Domnul a omorât pe to?i întâi-nascu?ii în pamântul Egiptului, de la întâi-nascutul oamenilor pâna la întâi-nascutul dobitoacelor. De aceea jertfesc eu Domnului pe tot întâi-nascutul de parte barbateasca ?i pe tot întâi-nascutul din fiii mei îl rascumpar.
Exo 13:16  Sa fie dar aceasta ca un semn la mâna ta ?i ca o tabli?a deasupra ochilor tai, caci cu mâna tare ne-a scos Domnul din Egipt!"
Exo 13:17  Iar dupa ce Faraon a dat drumul poporului, Dumnezeu nu l-a dus pe calea cea catre pamântul Filistenilor, care era mai scurta; caci a zis Dumnezeu: "Nu cumva poporul, vazând razboi, sa-i para rau ?i sa se întoarca în Egipt".
Exo 13:18  Ci a dus Dumnezeu poporul împrejur, pe calea pustiului, catre Marea Ro?ie. ?i fiii lui Israel au ie?it în buna rânduiala din pamântul Egiptului.
Exo 13:19  Atunci a luat Moise cu sine oasele lui Iosif; caci Iosif legase pe fiii lui Israel cu juramânt, zicând: "Are sa va cerceteze Dumnezeu ?i atunci sa lua?i cu voi ?i oasele mele de aici!"
Exo 13:20  Fiii lui Israel au pornit apoi din Sucot ?i ?i-au a?ezat tabara la Etam, la capatul pustiului.
Exo 13:21  Iar Domnul mergea înaintea lor: ziua în stâlp de nor, aratându-le calea, iar noaptea în stâlp de foc,  luminându-le, ca sa poata merge ?i ziua ?i noaptea.
Exo 13:22  ?i n-a lipsit stâlpul de nor ziua, nici stâlpul de foc noaptea dinaintea poporului.
Exo 14:1  Atunci a grait Domnul cu Moise ?i a zis:
Exo 14:2  "Spune fiilor lui Israel sa se întoarca ?i sa-?i a?eze tabara în fa?a Pi-Hahirotului, între Migdal ?i mare, în preajma lui Baal-?efon. Acolo, în preajma lui, lânga mare, sa tabarâ?i.
Exo 14:3  Ca Faraon va zice catre poporul sau: Fiii ace?tia ai lui Israel s-au ratacit în pamântul acesta ?i i-a închis pustiul.
Exo 14:4  Iar Eu voi învârto?a inima lui Faraon ?i va alerga dupa ei. ?i-Mi voi arata slava Mea asupra lui Faraon ?i asupra a toata o?tirea lui; ?i vor cunoa?te to?i Egiptenii ca Eu sunt Domnul!" ?i au facut a?a.
Exo 14:5  Atunci s-a dat de ?tire regelui Egiptului ca poporul evreu a fugit. ?i s-a întors inima lui Faraon ?i a slujitorilor lui asupra poporului acestuia ?i ei au zis: "Ce am facut noi? Cum de am lasat pe fiii lui Israel sa se duca ?i sa nu ne mai robeasca noua?"
Exo 14:6  A înhamat deci Faraon carele sale de razboi ?i a luat poporul sau cu sine:
Exo 14:7  A luat cu sine ?ase sute de caru?e alese ?i toata calarimea Egiptului ?i capeteniile lor.
Exo 14:8  Iar Domnul a învârto?at inima lui Faraon, regele Egiptului, ?i a slujitorilor lui, ?i a alergat acesta dupa fiii lui Israel; dar fiii lui Israel ie?isera sub mâna înalta.
Exo 14:9  ?i au alergat dupa ei Egiptenii cu to?i caii ?i carele lui Faraon, cu calare?ii ?i cu toata o?tirea lui ?i i-au ajuns când poposisera ei la mare, lânga Pi-Hahirot, în fa?a lui Baal-?efon.
Exo 14:10  Dar când s-a apropiat Faraon ?i când s-au uitat fiii lui Israel înapoi ?i au vazut ca Egiptenii vin dupa ei, s-au spaimântat foarte tare fiii lui Israel ?i au strigat catre Domnul;
Exo 14:11  ?i au zis catre Moise: "Oare nu erau morminte în ?ara Egiptului, de ce ne-ai adus sa murim în pustie? Ce ai facut tu cu noi, sco?ându-ne din Egipt?
Exo 14:12  Nu ?i-am spus noi, oare, de aceasta în Egipt, când ?i-am zis: Lasa-ne sa robim Egiptenilor, ca e mai bine sa fim robi Egiptenilor decât sa murim în pustia aceasta?"
Exo 14:13  Moise însa a zis catre popor: "Nu va teme?i! Sta?i ?i ve?i vedea minunea cea de la Domnul, pe care va va face-o El astazi, caci pe Egiptenii pe care îi vede?i astazi nu-i ve?i mai vedea niciodata.
Exo 14:14  Domnul are sa Se lupte pentru voi, iar voi fi?i lini?ti?i!"
Exo 14:15  Atunci a zis Domnul catre Moise: "Ce strigi catre Mine? Spune fiilor lui Israel sa porneasca,
Exo 14:16  Iar tu ridica-n toiagul ?i-?i întinde mâna asupra marii ?i o desparte ?i vor trece fiii lui Israel prin mijlocul marii, ca pe uscat.
Exo 14:17  Iata, Eu voi învârto?a inima lui Faraon ?i a tuturor Egiptenilor, ca sa mearga pe urmele lor. ?i-Mi voi arata slava Mea asupra lui Faraon ?i asupra a toata o?tirea lui, asupra carelor lui ?i asupra calare?ilor lui.
Exo 14:18  ?i vor cunoa?te to?i Egiptenii ca Eu sunt Domnul, când Îmi voi arata slava Mea asupra lui Faraon, asupra carelor lui ?i asupra calare?ilor lui".
Exo 14:19  Atunci s-a ridicat îngerul Domnului, care mergea înaintea taberei fiilor lui Israel, ?i s-a mutat în urma lor; ?i s-a ridicat stâlpul cel de nor dinaintea lor ?i a stat în urma lor.
Exo 14:20  Astfel a trecut el ?i a stat între tabara Egiptenilor ?i tabara fiilor lui Israel; ?i era negura ?i întuneric pentru unii, iar pentru ceilal?i lumina, noaptea, ?i toata noaptea nu s-au apropiat unii de al?ii.
Exo 14:21  Iar Moise ?i-a întins mâna sa asupra marii ?i a alungat Domnul marea toata noaptea cu vânt puternic de la rasarit ?i s-a facut marea uscat, ca s-au despar?it apele.
Exo 14:22  ?i au intrat fiii lui Israel prin mijlocul marii, mergând ca pe uscat, iar apele le erau perete, la dreapta ?i la stânga lor.
Exo 14:23  Iar Egiptenii urmarindu-i, au intrat dupa ei în mijlocul marii to?i caii lui Faraon, carele ?i calare?ii lui.
Exo 14:24  Dar în straja dimine?ii a cautat Domnul din stâlpul cel de foc ?i din nor spre tabara Egiptenilor ?i a umplut tabara Egiptenilor de spaima.
Exo 14:25  ?i a facut sa sara ro?ile de la carele lor, încât cu anevoie mergeau carele. Atunci au zis Egiptenii: "Sa fugim de la fa?a lui Israel, ca Domnul se lupta pentru ei cu Egiptenii!"
Exo 14:26  Iar Domnul a zis catre Moise: "Întinde?i mâna asupra marii, ca sa se întoarca apele asupra Egiptenilor, asupra carelor lor ?i asupra calare?ilor lor".
Exo 14:27  ?i ?i-a întins Moise mâna asupra marii ?i spre ziua s-a întors apa la locul ei, iar Egiptenii fugeau împotriva apei. ?i a?a a înecat Dumnezeu pe Egipteni în mijlocul marii.
Exo 14:28  Iar apele s-au tras la loc ?i au acoperit carele ?i calare?ii întregii o?tiri a lui Faraon, care intrase dupa Israeli?i în mare, ?i nu a ramas nici unul dintre ei.
Exo 14:29  Fiii lui Israel însa au trecut prin mare ca pe uscat ?i apa le-a fost perete la dreapta ?i stânga lor.
Exo 14:30  A?a a izbavit Domnul în ziua aceea pe Israeli?i din mâinile Egiptenilor; ?i au vazut fiii lui Israel pe Egipteni mor?i pe malurile marii.
Exo 14:31  Vazut-a Israel mâna cea tare pe care a întins-o Domnul asupra Egiptenilor, ?i s-a temut poporul de Domnul ?i a crezut în Domnul ?i în Moise, sluga Lui.
Exo 15:1  Atunci Moise ?i fiii lui Israel au cântat Domnului cântarea aceasta ?i au zis: "Sa cântam Domnului, caci cu slava S-a preaslavit! Pe cal ?i pe calare? în mare i-a aruncat!
Exo 15:2  Taria mea ?i marirea mea este Domnul, caci El m-a izbavit. Acesta este Dumnezeul meu ?i-L voi preaslavi, Dumnezeul parintelui meu ?i-L voi preaînal?a!
Exo 15:3  Domnul este viteaz în lupta; Domnul este numele Lui.
Exo 15:4  Carele lui Faraon ?i o?tirea lui în mare le-a aruncat; Pe capeteniile cele de seama ale lui, Marea Ro?ie le-a înghi?it,
Exo 15:5  Adâncul le-a acoperit, În fundul marii ca o piatra s-au pogorât.
Exo 15:6  Dreapta Ta, Doamne, ?i-a aratat taria. Mâna Ta cea dreapta, Doamne, pe vrajma?i i-a sfarâmat.
Exo 15:7  Cu mul?imea slavei Tale ai surpat pe cei potrivnici. Trimis-ai mânia Ta ?i i-a mistuit ca pe ni?te paie.
Exo 15:8  La suflarea narilor Tale s-a despar?it apa, Strânsu-s-au la un loc apele ca un perete ?i s-au închegat valurile în inima marii.
Exo 15:9  Vrajma?ul zicea: "Alerga-voi dupa ei ?i-i voi ajunge; Prada voi împar?i ?i-mi voi satura sufletul de razbunare; Voi scoate sabia ?i mâna mea îi va stârpi".
Exo 15:10  Dar ai trimis Tu duhul Tau ?i marea i-a înghi?it; Afundatu-s-au ca plumbul În apele cele mari.
Exo 15:11  Doamne, cine este asemenea ?ie între dumnezei? Cine este asemenea ?ie preaslavit în sfin?enie, Minunat întru slava ?i facator de minuni?
Exo 15:12  Întins-ai dreapta Ta ?i i-a înghi?it pamântul!
Exo 15:13  Calauzit-ai cu mila Ta acest popor ?i l-ai izbavit; Tu îl pova?uie?ti cu puterea Ta, Spre loca?ul sfin?eniei Tale.
Exo 15:14  Auzit-au neamurile ?i s-au cutremurat, Frica a cuprins pe cei din Filisteia.
Exo 15:15  Atunci s-au spaimântat capeteniile Edomului, Pe conducatorii Moabului cutremur i-a cuprins; ?i to?i câ?i traiesc în Canaan ?i-au pierdut cumpatul.
Exo 15:16  Frica ?i groaza va cadea peste ei. ?i de mare?ia bra?ului Tau, Ca pietrele vor încremeni, Pâna va trece poporul Tau, Doamne, Pâna va trece poporul Tau acesta, pe care l-ai câ?tigat Tu.
Exo 15:17  Tu îl vei duce ?i-l vei sadi în muntele mo?tenirii Tale, În locul ce ?i l-ai facut sala?luire, Doamne, În loca?ul sfânt cel zidit de mâinile Tale, Doamne!
Exo 15:18  Împara?i-va Domnul în veac ?i în veacul veacului.
Exo 15:19  Caci caii lui Faraon cu carele ?i calare?ii lui au intrat în mare. Întors-a Domnul asupra lor apele marii, Iar fiii lui Israel au trecut prin mare, ca pe uscat!"
Exo 15:20  Atunci a luat Mariam prooroci?a, sora lui Aaron, timpanul în mâna sa, ?i au ie?it dupa dânsa toate femeile cu timpane ?i dan?uind.
Exo 15:21  ?i raspundea Mariam înaintea lor: "Sa cântam Domnului, caci cu slava S-a preaslavit! Pe cal ?i pe calare? în mare i-a aruncat!"
Exo 15:22  Apoi a ridicat Moise pe fiii lui Israel de la Marea Ro?ie ?i i-a dus în pustia ?ur ?i au mers trei zile prin pustie ?i n-au gasit apa.
Exo 15:23  Au ajuns apoi la Mara, dar n-au putut sa bea apa nici din Mara, ca era amara, pentru care s-a ?i numit locul acela Mara.
Exo 15:24  De aceea cârtea poporul împotriva lui Moise ?i zicea: "Ce sa bem?"
Exo 15:25  Atunci Moise a strigat catre Domnul ?i Domnul i-a aratat un lemn; ?i l-a aruncat în apa ?i s-a îndulcit apa. Acolo a pus Domnul poporului Sau rânduieli ?i porunci ?i acolo l-a încercat ?i i-a zis:
Exo 15:26  "De vei asculta cu luare-aminte glasul Domnului Dumnezeului tau ?i vei face lucruri drepte înaintea Lui ?i de vei lua aminte la poruncile Lui ?i vei pazi legile Lui, nu voi aduce asupra ta nici una din bolile pe care le-am adus asupra Egiptenilor, ca Eu sunt Domnul Dumnezeul tau Care te vindeca".
Exo 15:27  Apoi au venit în Elim. ?i erau acolo douasprezece izvoare de apa ?i ?aptezeci de pomi de finic. ?i au tabarât acolo lânga apa.
Exo 16:1  Plecând apoi din Elim, a venit toata ob?tea fiilor lui Israel în pustia Sin, câre este între Elim ?i între Sinai, în ziua a cincisprezecea a lunii a doua, dupa ie?irea din Egipt.
Exo 16:2  În pustia aceasta toata ob?tea fiilor lui Israel a cârtit împotriva lui Moise ?i Aaron.
Exo 16:3  ?i au zis catre ei fiii lui Israel: "Mai bine muream batu?i de Domnul în pamântul Egiptului, când ?edeam împrejurul caldarilor cu carne ?i mâncam pâine de ne saturam! Dar voi ne-a?i adus în pustia aceasta, ca toata ob?tea aceasta sa moara de foame".
Exo 16:4  Domnul însa a zis catre Moise: "Iata Eu le voi ploua pâine din cer. Sa iasa dar poporul ?i sa adune în fiecare zi cât trebuie pentru o zi, ca sa-l încerc daca va umbla sau nu dupa legea Mea.
Exo 16:5  Iar în ziua a ?asea sa adune de doua ori mai mult decât adunau în celelalte zile, pentru o zi".
Exo 16:6  Atunci au zis Moise ?i Aaron catre toata adunarea fiilor lui Israel: "Diseara ve?i cunoa?te ca Domnul v-a scos din pamântul Egiptului.
Exo 16:7  ?i diminea?a ve?i vedea slava Domnului, ca El a auzit cârtirea voastra împotriva lui Dumnezeu; iar noi ce suntem de cârti?i împotriva noastra?"
Exo 16:8  ?i a mai zis Moise: "Când Domnul va va da diseara carne sa mânca?i ?i diminea?a pâine sa va satura?i, din aceea veri afla ca a auzit Domnul cârtirea ce a?i ridicat asupra Lui. Caci noi ce suntem? Cârtirea voastra nu este împotriva noastra, ci împotriva lui Dumnezeu".
Exo 16:9  Apoi a zis Moise catre Aaron: "Spune la toata adunarea fiilor lui Israel: Apropia?i-va înaintea lui Dumnezeu, ca a auzit cârtirea voastra!"
Exo 16:10  Iar când vorbea Aaron catre toata adunarea fiilor lui Israel, au cautat ei spre pustie ?i iata slava Domnului s-a aratat în nor.
Exo 16:11  ?i a grait Domnul cu Moise ?i a zis:
Exo 16:12  "Am auzit cârtirea fiilor lui Israel. Spune-le dar: Diseara carne ve?i mânca, iar diminea?a va ve?i satura de pâine ?i ve?i cunoa?te ca Eu, Domnul, sunt Dumnezeul vostru".
Exo 16:13  Iar, daca s-a facut seara, au venit prepeli?e ?i au acoperit tabara, iar diminea?a, dupa ce s-a luat roua dimprejurul taberei,
Exo 16:14  Iata, se afla pe fa?a pustiei ceva marunt, ca ni?te graun?e, ?i albicios, ca grindina pe pamânt.
Exo 16:15  ?i vazând fiii lui Israel, au zis unii catre al?ii: "Ce e asta?" Ca nu ?tiau ce e. Iar Moise le-a zis: "Aceasta e pâinea pe care v-o da Dumnezeu sa o mânca?i.
Exo 16:16  Iata ce a poruncit Domnul: Aduna?i fiecare cât sa va ajunga de mâncat; câte un omer de om, dupa numarul sufletelor voastre; fiecare câ?i are în cort, atâtea omere sa adune!"
Exo 16:17  ?i au facut a?a fiii lui Israel; au adunat unii mai mult, al?ii mai pu?in;
Exo 16:18  Dar masurând cu omerul, nici celui ce adunase mult n-a prisosit, nici celui ce adunase pu?in n-a lipsit, ci fiecare, cât era deajuns la cei ce erau cu sine, atât a adunat.
Exo 16:19  Zis-a iara?i Moise catre ei: "Nimeni sa nu lase din aceasta pe a doua zi".
Exo 16:20  Dar ei n-au ascultat pe Moise, ci unii au lasat din aceasta pe a doua zi; dar a facut viermi ?i s-a stricat. ?i s-a mâniat pe ei Moise.
Exo 16:21  Fiecare aduna mana diminea?a cât îi trebuia pentru mâncat în ziua aceea, caci, daca se înfierbânta soarele, ceea ce ramânea se topea.
Exo 16:22  Iar în ziua a ?asea adunara de doua ori mai multa: câte doua omere de fiecare. ?i au venit toate capeteniile adunarii sa-l în?tiin?eze pe Moise.
Exo 16:23  Iar Moise le-a zis: "Iata ce a zis Domnul: Mâine e odihna, odihna cea sfânta în cinstea Domnului; ce trebuie copt coace?i, ce trebuie fiert, fierbe?i astazi, ?i ce va ramâne, pastra?i pe a doua zi!"
Exo 16:24  ?i au lasat din acestea pâna diminea?a, dupa cum le poruncise Moise, ?i nu s-au stricat nici n-au facut viermi.
Exo 16:25  Apoi a zis Moise: "Mânca?i aceasta astazi, ca astazi este odihna în cinstea Domnului ?i nu ve?i gasi de aceasta astazi pe câmp.
Exo 16:26  ?ase zile sa aduna?i, iar ziua a ?aptea este zi de odihna ?i nu ve?i afla din ea în aceasta zi".
Exo 16:27  Dar unii din popor au ie?it sa adune ?i în ziua a ?aptea ?i n-au gasit.
Exo 16:28  Atunci Domnul a zis catre Moise: "Pâna când nu ve?i voi sa asculta?i de poruncile Mele ?i de înva?aturile Mele?
Exo 16:29  Vede?i ca Domnul v-a dat ziua aceasta de odihna ?i de aceea va da El în ziua a ?asea ?i pâine pentru doua zile; ramâne?i fiecare în casele voastre ?i nimeni sa nu iasa de la locul sau în ziua a ?aptea".
Exo 16:30  ?i s-a odihnit poporul în ziua a ?aptea.
Exo 16:31  Casa lui Israel i-a pus numele mana ?i aceasta era alba, ca samân?a de coriandru, iar la gust ca turta cu miere.
Exo 16:32  Dupa aceea Moise a zis: "Iata ce porunce?te Domnul: Umple?i cu mana un omer, ca sa se pastreze în viitor urma?ilor vo?tri, ca sa vada pâinea cu care v-am hranit Eu în pustie, dupa ce v-am scos din ?ara Egiptului".
Exo 16:33  Iar catre Aaron a zis Moise: "Ia un vas de aur ?i toarna în el un omer plin cu mana ?i pune-l înaintea Domnului, ca sa se pastreze în viitor pentru urma?ii vo?tri!"
Exo 16:34  ?i l-a pus Aaron înaintea chivotului marturiei, ca sa se pastreze, cum poruncise Domnul lui Moise.
Exo 16:35  Iar fiii lui Israel au mâncat mana patruzeci de ani, pâna ce au ajuns în ?ara locuita; pâna ce au ajuns în hotarele pamântului Canaan au mâncat mana.
Exo 16:36  Iar omerul este a zecea parte dintr-o efa.
Exo 17:1  Dupa aceea a plecat la drum toata ob?tea fiilor lui Israel din pustia Sin, dupa porunca Domnului, ?i a tabarât la Rafidim, unde poporul nu avea apa de baut.
Exo 17:2  ?i poporul cauta cearta lui Moise, zicând: "Da-ne apa sa bem!" Iar Moise le-a zis: "De ce ma banui?i ?i de ce ispiti?i pe Domnul?"
Exo 17:3  Atunci poporul, apasat de sete, cârtea împotriva lui Moise ?i zicea: "Ce este aceasta? Ne-ai scos din Egipt ca sa ne omori cu sete pe noi, pe copiii no?tri ?i turmele noastre?"
Exo 17:4  Iar Moise a strigat catre Domnul ?i a zis: "Ce sa fac cu poporul acesta? Caci pu?in lipse?te ca sa ma ucida cu pietre".
Exo 17:5  Zis-a Domnul catre Moise: "Treci pe dinaintea poporului acestuia, dar ia cu tine câ?iva din batrânii lui Israel; ia în mâna ?i toiagul cu care ai lovit Nilul ?i du-te.
Exo 17:6  Iata Eu voi sta înaintea ta acolo la stânca din Horeb, iar tu vei lovi în stânca ?i va curge din ea apa ?i va bea poporul". ?i a facut Moise a?a înaintea batrânilor lui Israel.
Exo 17:7  De aceea s-a pus locului aceluia numele: Masa ?i Meriba, pentru ca acolo cârtisera fiii lui Israel ?i pentru ca ispitisera pe Domnul, zicând: "Este, oare, Domnul în mijlocul nostru sau nu?"
Exo 17:8  Atunci au venit Amaleci?ii sa se bata cu Israeli?ii la Rafidim.
Exo 17:9  Iar Moise a zis catre Iosua: "Alege-fi barba?i voinici ?i du-te de te lupta cu Amaleci?ii! Iar eu ma voi sui mâine în vârful muntelui ?i toiagul lui Dumnezeu va fi în mâna mea".
Exo 17:10  A facut deci Iosua cum îi zisese Moise ?i s-a dus sa bata pe Amaleci?i; iar Moise cu Aaron ?i Or s-au suit în vârful muntelui.
Exo 17:11  Când î?i ridica Moise mâinile, biruia Israel; iar când î?i lasa el mâinile, biruiau Amaleci?ii.
Exo 17:12  Dar obosind mâinile lui Moise, au luat o piatra ?i au pus-o lânga el ?i a ?ezut Moise pe piatra; iar Aaron ?i Or îi sprijineau mâinile, unul de o parte ?i altul de alta parte. ?i au stat mâinile lui ridicate pâna la asfin?itul soarelui.
Exo 17:13  ?i a zdrobit Iosua pe Amalec ?i tot poporul lui cu ascu?i?ul sabiei.
Exo 17:14  Atunci a zis Domnul catre Moise: "Scrie acestea în carte spre pomenire ?i spune lui Iosua ca voi ?terge cu totul pomenirea lui Amalec de sub cer!"
Exo 17:15  Atunci a facut Moise un jertfelnic Domnului ?i i-a pus numele: "Domnul este scaparea mea!"
Exo 17:16  Caci zicea: "Pentru ca mi-au fost mâinile ridicate spre scaunul Domnului, de aceea va bate Domnul pe Amalec din neam în neam!"
Exo 18:1  Auzind însa Ietro, preotul din Madian, socrul lui Moise, de toate câte facuse Dumnezeu pentru Moise ?i pentru Israel, poporul Sau, când a scos Domnul pe Israel din Egipt,
Exo 18:2  A luat Ietro, socrul lui Moise, pe Sefora, femeia lui Moise, care fusese trimisa înainte de acesta acasa,
Exo 18:3  ?i pe cei doi fii ai ei, din care unul se chema Gher?om, pentru ca Moise î?i zisese: "Ratacit sunt eu în pamânt strain",
Exo 18:4  Iar pe altul îl chema Eliezer, pentru ca-?i zisese el: "Dumnezeul parin?ilor mei mi-a fost ajutor ?i m-a scapat de sabia lui Faraon!"
Exo 18:5  ?i a venit Ietro, socrul lui Moise, cu fiii acestuia ?i cu femeia lui, la Moise în pustie, unde-?i a?ezase el tabara, la muntele lui Dumnezeu.
Exo 18:6  Atunci el a trimis vorba lui Moise, zicând: "Iata, eu, Ietro, socrul tau, ?i femeia ta ?i cei doi fii ai ei împreuna cu ea venim la tine".
Exo 18:7  Deci a ie?it Moise în întâmpinarea socrului sau, s-a plecat înaintea lui ?i l-a sarutat. Iar dupa ce s-au binecuvântat unul pe altul, au intrat în cort.
Exo 18:8  Apoi a povestit Moise socrului sau toate câte a facut Domnul cu Faraon ?i cu to?i Egiptenii pentru Israel, toate suferin?ele ce le-au întâlnit ei în cale ?i cum i-a izbavit Domnul din mâinile lui Faraon ?i din mâinile Egiptenilor.
Exo 18:9  Iar Ietro s-a bucurat de toate binefacerile ce a aratat Domnul lui Israel, când l-a izbavit din mâna Egiptenilor ?i din mâna lui Faraon.
Exo 18:10  ?i a zis Ietro: "Binecuvântat este Domnul, Care v-a izbavit din mâinile Egiptenilor ?i din mâna tui Faraon, Cel ce a izbavit pe poporul acesta din stapânirea Egiptenilor.
Exo 18:11  Acum am cunoscut ?i eu ca Domnul este mare peste to?i dumnezeii, pentru ca a smerit pe ace?tia".
Exo 18:12  Apoi Ietro, socrul lui Moise, a adus lui Dumnezeu ardere de tot ?i jertfa. ?i au venit Aaron ?i to?i batrânii lui Israel sa manânce pâine cu socrul lui Moise înaintea lui Dumnezeu.
Exo 18:13  Iar a doua zi a ?ezut Moise sa judece poporul ?i a stat poporul înaintea lui Moise de diminea?a pâna seara.
Exo 18:14  Vazând Ietro, socrul lui Moise, tot ceea ce facea el cu poporul, i-a zis: "Ce faci tu cu poporul? De ce stai tu singur ?i tot poporul tau sta înaintea ta de diminea?a pâna seara?"
Exo 18:15  Iar Moise a zis catre socrul sau: "Poporul vine la mine sa ceara judecata de la Dumnezeu.
Exo 18:16  Când se ivesc între ei neîn?elegeri, vin la mine ?i judec pe fiecare ?i-i înva? poruncile lui Dumnezeu ?i legile Lui".
Exo 18:17  Iar socrul lui Moise a zis catre acesta: "Ceea ce faci, nu faci bine.
Exo 18:18  Caci te vei prapadi ?i tu, ?i poporul acesta, care este cu tine. E grea pentru tine sarcina aceasta ?i nu o vei putea împlini singur.
Exo 18:19  Acum dar asculta-ma pe mine: Am sa-?i dau un sfat ?i Dumnezeu sa fie cu tine! Fii tu pentru popor mijlocitor înaintea lui Dumnezeu ?i înfa?i?eaza la Dumnezeu nevoile lui.
Exo 18:20  Înva?a-i poruncile ?i legile Lui; arata-le calea Lui, pe care trebuie sa mearga, ?i faptele ce trebuie sa faca.
Exo 18:21  Iar mai departe alege-?i din tot poporul oameni drep?i ?i cu frica lui Dumnezeu; oameni drep?i, care urasc lacomia, ?i-i pune capetenii peste mii, capetenii peste sute, capetenii peste cincizeci, capetenii peste zeci.
Exo 18:22  Ace?tia sa judece poporul în toata vremea: pricinile grele sa le aduca la tine, iar pe cele mici sa le judece ei toate. U?ureaza-?i povara ?i ei sa o poarte împreuna cu tine!
Exo 18:23  De vei face lucrul acesta ?i te va întari ?i Dumnezeu cu porunci, vei putea sa faci fa?a, ?i tot poporul acesta va ajunge cu pace la locul sau".
Exo 18:24  ?i a ascultat Moise glasul socrului sau ?i a facut toate câte i-a zis.
Exo 18:25  A ales deci Moise din tot Israelul oameni destoinici ?i i-a pus capetenii în popor: peste mii, peste sute, peste cincizeci, peste zeci.
Exo 18:26  ?i judecau ace?tia poporul în toata vremea; toate pricinile grele le aduceau la Moise, iar pe cele mai u?oare le judecau ei toate.
Exo 18:27  Dupa aceea a petrecut Moise pe socrul sau ?i acesta s-a dus în ?ara lui.
Exo 19:1  Iar în luna a treia de la ie?irea fiilor lui Israel din pamântul Egiptului, chiar în ziua de luna plina, au ajuns în pustia Sinai.
Exo 19:2  Plecase deci Israel de la Rafidim ?i ajungând în pustia Sinai, au tabarât acolo în pustie, în fa?a muntelui.
Exo 19:3  Apoi s-a suit Moise în munte, la Dumnezeu; ?i l-a strigat Domnul din vârful muntelui ?i i-a zis: "Graie?te casei lui Iacov ?i veste?te fiilor lui Israel a?a:
Exo 19:4  "A?i vazut ce am facut Egiptenilor ?i cum v-am luat pe aripi de vultur ?i v-am adus la Mine.
Exo 19:5  Deci, de ve?i asculta glasul Meu ?i de ve?i pazi legamântul Meu, dintre toate neamurile Îmi ve?i fi popor ales ca al Meu este tot pamântul;
Exo 19:6  Îmi ve?i fi împara?ie preo?easca ?i neam sfânt!" Acestea sunt cuvintele pe care le vei spune fiilor lui Israel".
Exo 19:7  ?i venind, Moise a chemat pe batrânii poporului ?i le-a spus toate cuvintele acestea pe care le poruncise Domnul.
Exo 19:8  Atunci tot poporul, raspunzând într-un glas, a zis: "Toate câte a zis Domnul vom face ?i vom fi ascultatori!" ?i a dus Moise cuvintele poporului la Domnul.
Exo 19:9  Iar Domnul a zis catre Moise: "Iata voi veni la tine în stâlp de nor des, ca sa auda poporul ca Eu graiesc cu tine, ?i sa te creada pururea". Iar Moise a spus Domnului cuvintele poporului.
Exo 19:10  Zis-a Domnul catre Moise: "Pogoara-te de graie?te poporului sa se ?ina curat astazi ?i mâine, ?i sa-?i spele hainele,
Exo 19:11  Ca sa fie gata pentru poimâine, caci poimâine Se va pogorî Domnul înaintea ochilor a tot poporul pe Muntele Sinai.
Exo 19:12  Sa-i tragi poporului hotar împrejurul muntelui ?i sa-i spui: Pazi?i-va de a va sui în munte ?i de a va atinge de ceva din el, ca tot cel ce se va atinge de munte va muri.
Exo 19:13  Nici cu mâna sa nu se atinga de el, ca va fi ucis cu pietre sau se va sageta cu sageata; nu va ramâne în via?a, fie om, fie dobitoc. Iar daca se vor îndeparta tunetele ?i trâmbi?ele ?i norul de pe munte, se vor putea sui în munte".
Exo 19:14  Pogorându-se deci Moise din munte la popor, el a sfin?it poporul ?i, spalându-?i ei hainele,
Exo 19:15  Le-a zis Moise: "Sa fi?i gata pentru poimâine ?i de femei sa nu va atinge?i!"
Exo 19:16  Iar a treia zi, când s-a facut ziua, erau tunete ?i fulgere ?i nor des pe Muntele Sinai ?i sunet de trâmbi?e foarte puternic. ?i s-a cutremurat tot poporul în tabara.
Exo 19:17  Atunci a scos Moise poporul din tabara în întâmpinarea lui Dumnezeu ?i au stat la poalele muntelui.
Exo 19:18  Iar Muntele Sinai fumega tot, ca Se pogorâse Dumnezeu pe el în foc; ?i se ridica de pe el fum, ca fumul dintr-un cuptor, ?i tot muntele se cutremura puternic.
Exo 19:19  De asemenea ?i sunetul trâmbi?ei se auzea din ce în ce mai tare; ?i Moise graia, iar Dumnezeu îi raspundea cu glas.
Exo 19:20  Deci, fiind pogorât Domnul pe Muntele Sinai, pe vârful muntelui, a chemat Domnul pe Moise în vârful muntelui ?i s-a suit Moise acolo.
Exo 19:21  Atunci a zis Domnul catre Moise: "Pogoara-te ?i opre?te poporul, ca sa nu navaleasca spre Domnul, sa vada slava Lui, ca vor cadea mul?i dintre ei.
Exo 19:22  Iar preo?ii, care se apropie de Domnul Dumnezeu, sa se sfin?easca, ca nu cumva sa-i loveasca Domnul".
Exo 19:23  Zis-a Moise catre Domnul: "Nu se poate ca poporul sa se suie pe Muntele Sinai, pentru ca Tu ne-ai oprit din vreme, ?i ai zis: Trage hotar împrejurul muntelui ?i-l sfin?e?te!"
Exo 19:24  Iar Domnul i-a raspuns: "Du-te ?i te pogoara ?i apoi te vei sui împreuna cu Aaron; iar preo?ii ?i poporul sa nu îndrazneasca a se sui la Domnul, ca sa nu-i loveasca Domnul".
Exo 19:25  ?i s-a pogorât Moise la popor ?i i-a spus toate.
Exo 20:1  Atunci a rostit Domnul înaintea lui Moise toate cuvintele acestea ?i a zis:
Exo 20:2  "Eu sunt Domnul Dumnezeul tau, Care te-a scos din pamântul Egiptului ?i din casa robiei.
Exo 20:3  Sa nu ai al?i dumnezei afara de Mine!
Exo 20:4  Sa nu-?i faci chip cioplit ?i nici un fel de asemanare a nici unui lucru din câte sunt în cer, sus, ?i din câte sunt pe pamânt, jos, ?i din câte sunt în apele de sub pamânt!
Exo 20:5  Sa nu te închini lor, nici sa le sluje?ti, ca Eu, Domnul Dumnezeul tau, sunt un Dumnezeu zelos, care pedepsesc pe copii pentru vina parin?ilor ce Ma urasc pe Mine, pâna la al treilea ?i al patrulea neam,
Exo 20:6  ?i Ma milostivesc pâna la al miilea neam catre cei ce Ma iubesc ?i pazesc poruncile Mele.
Exo 20:7  Sa nu iei numele Domnului Dumnezeului tau în de?ert, ca nu va lasa Domnul nepedepsit pe cel ce ia în de?ert numele Lui.
Exo 20:8  Adu-?i aminte de ziua odihnei, ca sa o sfin?e?ti.
Exo 20:9  Lucreaza ?ase zile ?i-?i fa în acelea toate treburile tale,
Exo 20:10  Iar ziua a ?aptea este odihna Domnului Dumnezeului tau: sa nu faci în acea zi nici un lucru: nici tu, nici fiul tau, nici fiica ta, nici sluga ta, nici slujnica ta, nici boul tau, nici asinul tau, nici orice dobitoc al tau, nici strainul care ramâne la tine,
Exo 20:11  Ca în ?ase zile a facut Domnul cerul ?i pamântul, marea ?i toate cele ce sunt într-însele, iar în ziua a ?aptea S-a odihnit. De aceea a binecuvântat Domnul ziua a ?aptea ?i a sfin?it-o.
Exo 20:12  Cinste?te pe tatal tau ?i pe mama ta, ca sa-?i fie bine ?i sa traie?ti ani mul?i pe pamântul pe care Domnul Dumnezeul tau ?i-l va da ?ie.
Exo 20:13  Sa nu ucizi!
Exo 20:14  Sa nu fii desfrânat!
Exo 20:15  Sa nu furi!
Exo 20:16  Sa nu marturise?ti strâmb împotriva aproapelui tau!
Exo 20:17  Sa nu dore?ti casa aproapelui tau; sa nu dore?ti femeia aproapelui tau, nici ogorul lui, nici sluga lui, nici slujnica lui, nici boul lui, nici asinul lui ?i nici unul din dobitoacele lui ?i nimic din câte are aproapele tau!"
Exo 20:18  ?i tot poporul a auzit fulgerele ?i tunetele ?i sunetul trâmbi?elor, ?i a vazut muntele fumegând; ?i vazând, tot poporul s-a dat înapoi ?i a stat departe, temându-se.
Exo 20:19  Apoi a zis catre Moise: "Vorbe?te tu cu noi ?i vom asculta, dar Dumnezeu sa nu graiasca cu noi, ca sa nu murim".
Exo 20:20  Zis-a Moise catre popor: "Cuteza?i, ca Dumnezeu a venit la voi, sa va puna la încercare pentru ca frica Lui sa fie în voi, ca sa nu gre?i?i".
Exo 20:21  ?i a stat tot poporul departe, iar Moise s-a apropiat de întunericul unde era Dumnezeu.
Exo 20:22  Atunci Domnul a zis catre Moise: "A?a sa vorbe?ti casei lui Iacov ?i a?a sa veste?ti fiilor lui Israel: A?i vazut ca am grait cu voi din cer!
Exo 20:23  Sa nu va face?i dumnezei de argint ?i nici dumnezei de aur sa nu va face?i.
Exo 20:24  Sa-Mi faci jertfelnic de pamânt ?i sa aduci pe el arderile de tot ale tale, jertfele de izbavire, oile ?i boii tai. În tot locul unde voi pune pomenirea numelui Meu, acolo voi veni la tine, ca sa te binecuvântez.
Exo 20:25  Iar de-Mi vei face jertfelnic de piatra, sa nu-l faci de piatra cioplita; ca de vei pune dalta ta pe ea, o vei spurca.
Exo 20:26  ?i sa nu te sui pe trepte la jertfelnicul Meu, ca sa nu se descopere acolo goliciunea ta!"
Exo 21:1  "Iata acum legiuirile pe care tu le vei pune în vedere lor:
Exo 21:2  De vei cumpara rob evreu, el sa-?i lucreze ?ase ani, iar în anul al ?aptelea sa iasa slobod, în dar.
Exo 21:3  Daca acela a venit în casa ta singur, singur sa iasa; iar de a venit cu femeie, sa iasa cu el ?i femeia lui.
Exo 21:4  Daca însa îi va fi dat stapânul femeie ?i aceasta va fi nascut fii sau fiice, atunci femeia ?i copiii ei vor fi ai stapânului lui, iar el va ie?i singur.
Exo 21:5  Iar daca robul va zice: Îmi iubesc stapânul, femeia ?i copiii ?i nu voi sa ma liberez,
Exo 21:6  Atunci sa-l aduca stapânul lui la judecatori ?i, dupa ce l-a apropiat de u?a sau la u?ori, sa-i gaureasca stapânul urechea cu o sula, ?i-l va robi în veci.
Exo 21:7  Daca cineva î?i va vinde fiica roaba, ea nu va ie?i cum ies roabele.
Exo 21:8  Daca ea nu va placea stapânului sau, care ?i-a ales-o, sa-i îngaduie a se rascumpara, dar el nu va avea voie s-o vânda la familie straina, dupa ce i-a fost necredincios.
Exo 21:9  Daca a logodit-o cu fiul sau, atunci sa se poarte cu ea dupa dreptul fiicelor.
Exo 21:10  Iar daca va mai lua ?i pe alta, atunci ea sa nu fie lipsita de hrana, de îmbracaminte ?i de traiul cu barbatul sau.
Exo 21:11  Iar daca el nu-i va face aceste trei lucruri, sa iasa de la dânsul în dar, fara rascumparare.
Exo 21:12  De va lovi cineva pe un om ?i acela va muri, sa fie dat mor?ii.
Exo 21:13  Iar de nu-l va fi lovit cu voin?a ?i i-a cazut sub mâna din îngaduirea lui Dumnezeu, î?i voi hotarî un loc, unde sa fuga uciga?ul.
Exo 21:14  Daca însa va ucide cineva pe aproapele sau cu buna ?tiin?a ?i cu vicle?ug ?i va fugi la altarul Meu, ?i de la altarul Meu sa-l iei ?i sa-l omori.
Exo 21:15  Cel ce va bate pe tata sau pe mama sa fie omorât.
Exo 21:16  Cel ce va fura un om din fiii lui Israel ?i, facându-l rob, îl va vinde, sau se va gasi în mâinile lui, acela sa fie omorât.
Exo 21:17  Cel ce va grai de rau pe tatal sau sau pe mama sa, acela sa fie omorât.
Exo 21:18  De se vor sfadi doi oameni ?i unul va lovi pe celalalt cu o piatra, sau cu pumnul, ?i acela nu va muri, ci va cadea la pat,
Exo 21:19  De se va scula ?i va ie?i din casa cu ajutorul cârjei, cel ce l-a lovit nu va fi vinovat de moarte, ci va plati numai împiedicarea aceluia de la munca ?i vindecarea lui.
Exo 21:20  Iar de va lovi cineva pe robul sau sau pe slujnica sa cu toiagul, ?i ei vor muri sub mâna lui, aceia trebuie sa fie razbuna?i;
Exo 21:21  Iar de vor mai trai o zi sau doua, ei nu trebuie razbuna?i, ca sunt plati?i cu argintul stapânului lor.
Exo 21:22  De se vor bate doi oameni ?i vor lovi o femeie însarcinata ?i aceasta va lepada copilul sau fara alta vatamare, sa se supuna cel vinovat la despagubirea ce o va cere barbatul acelei femei ?i el va trebui sa plateasca potrivit cu hotarârea judecatorilor.
Exo 21:23  Iar de va fi ?i alta vatamare, atunci sa plateasca suflet pentru suflet,
Exo 21:24  Ochi pentru ochi, dinte pentru dinte, mâna pentru mâna, picior pentru picior,
Exo 21:25  Arsura pentru arsura, rana pentru rana, vânataie pentru vânataie.
Exo 21:26  Iar de va lovi cineva pe robul sau în ochi iar pe slujnica sa o va lovi în ochi ?i ea îl va pierde, sa-l lase liber ca despagubire pentru ochi.
Exo 21:27  ?i de va pricinui caderea unui dinte al robului sau sau al roabei sale, sa le dea drumul pentru acel dinte.
Exo 21:28  Daca un bou va împunge de moarte barbat sau femeie, boul sa fie ucis cu pietre ?i carnea lui sa nu se manânce, iar stapânul boului sa fie nevinovat.
Exo 21:29  Iar daca boul a fost împungator cu o zi sau cu doua sau cu trei înainte, ?i stapânul lui, fiind vestit despre aceasta, nu l-a închis ?i boul a ucis barbat sau femeie, boul sa fie ucis cu pietre ?i stapânul lui sa fie dat mor?ii.
Exo 21:30  Daca însa i se va pune stapânului pre? de rascumparare, pentru sufletul sau, ce va fi pus asupra lui aceea va ?i plati.
Exo 21:31  Tot dupa aceasta lege sa se urmeze, de va împunge boul baiat sau fata.
Exo 21:32  Iar de va împunge boul rob sau roaba, sa se plateasca stapânului acestora treizeci de sicli de argint, iar boul sa fie ucis cu pietre.
Exo 21:33  De va sapa cineva o fântâna sau va descoperi o fântâna ?i nu o va acoperi ?i va cadea în ea un bou sau un asin,
Exo 21:34  Stapânul fântânii trebuie sa plateasca argint stapânului lor, iar boul sau asinul sa fie al lui.
Exo 21:35  Iar daca boul cuiva va împunge boul altuia ?i va muri, sa se vânda boul cel viu ?i pre?ul sa-l împarta pe din doua; de asemenea ?i pe cel ucis sa-l împarta pe din doua.
Exo 21:36  Iar de s-a ?tiut ca boul a fost împungator de multa vreme, dar stapânul lui, fiind în?tiin?at despre aceasta, nu l-a pazit, atunci acesta trebuie sa plateasca bou pentru bou, iar cel ucis sa fie al lui".
Exo 22:1  "De va fura cineva un bou sau o oaie ?i le va junghia, sau le va vinde, sa plateasca cinci boi pentru un bou ?i patru oi pentru oaie!
Exo 22:2  Daca furul va fi prins spargând ?i va fi lovit încât sa moara, cel ce l-a lovit nu va fi vinovat de moartea lui.
Exo 22:3  Iar de se va face aceasta dupa ce a rasarit soarele, va fi vinovat ?i pentru ucidere va fi ucis. Cel ce a furat va trebui sa plateasca tot ?i de nu are cu ce, sa fie vândut el pentru plata celor furate.
Exo 22:4  Iar de se va prinde furul ?i cele furate se vor gasi la el vii, fie bou, oaie sau asin, sa plateasca îndoit.
Exo 22:5  De va pricinui cineva paguba într-o ?arina sau vie, lasând vitele sa pasca, stricând ?arina altuia, sa plateasca din ?arina sa potrivit cu stricaciunea; iar de a pascut toata ?arina, sa plateasca despagubire cu ce are mai bun în ?arina sa ?i cu ce are mai bun în via sa.
Exo 22:6  De va izbucni foc ?i va cuprinde spini ?i, întinzându-se, va arde clai, sau snopi, sau holda, sa plateasca despagubire îndoit cel ce a aprins focul.
Exo 22:7  De va da cineva vecinului sau argint sau lucruri sa le pastreze ?i acelea vor fi furate din casa acestui om, de se va gasi furul, sa le plateasca îndoit;
Exo 22:8  Iar de nu se va gasi furul, sa vina stapânul casei înaintea judecatorilor ?i sa jure ca nu ?i-a întins mâna asupra lucrului aproapelui sau.
Exo 22:9  Pentru tot lucrul care s-ar putea fura: bou sau asin, oaie sau haina, sau orice lucru pierdut, despre care va zice cineva: "Acesta este al meu!" pricina amândurora trebuie sa fie adusa înaintea judecatorilor, ?i cel ce va fi osândit de judecatori sa plateasca aproapelui sau îndoit.
Exo 22:10  De va da cineva spre paza aproapelui sau asin sau bou, sau oaie, sau alt dobitoc ?i va muri, sau va fi vatamat, sau luat fara sa ?tie cineva,
Exo 22:11  Sa faca amândoi juramânt înaintea Domnului, ca cel ce a luat pe seama sa nu ?i-a întins mâna asupra lucrului aproapelui sau, ?i a?a stapânul trebuie sa primeasca juramântul, iar celalalt nu va avea sa-l despagubeasca;
Exo 22:12  Iar de se va fura de la el, sa plateasca stapânului despagubire.
Exo 22:13  Daca însa va fi sfâ?iat de fiara, sa-i aduca ceea ce a ramas ca marturie ?i nu va plati despagubire pentru vita sfâ?iata.
Exo 22:14  De va împrumuta cineva de la aproapele sau vita ?i aceea se va vatama sau va pieri, ?i stapânul ei nu va fi cu ea, sa o plateasca;
Exo 22:15  Iar daca stapânul ei a fost cu ea, sa nu o plateasca. Iar daca a fost închiriata cu bani, se va socoti pentru chiria aceea.
Exo 22:16  De va amagi cineva o fata nelogodita ?i se va culca cu ea, sa o înzestreze ?i sa o ia de so?ie;
Exo 22:17  Iar daca tatal ei se va feri ?i nu va voi sa o dea lui de femeie, atunci el sa plateasca tatalui fetei bani câ?i se cer pentru înzestrarea fetelor.
Exo 22:18  Pe vrajitori sa nu-i lasa?i sa traiasca!
Exo 22:19  Tot cel ce se împreuna cu dobitoc sa fie omorât.
Exo 22:20  Cel ce jertfe?te la al?i dumnezei, afara de Domnul, sa se piarda.
Exo 22:21  Pe strain sa nu-l strâmtorezi, nici sa-l ape?i, caci ?i voi a?i fost straini în pamântul Egiptului.
Exo 22:22  La nici o vaduva ?i la nici un orfan sa nu le face?i rau!
Exo 22:23  Iar de le ve?i face rau ?i vor striga catre Mine, voi auzi plângerea lor,
Exo 22:24  ?i se va aprinde mânia Mea ?i va voi ucide cu sabia ?i vor fi femeile voastre vaduve ?i copiii vo?tri orfani.
Exo 22:25  De vei împrumuta bani fratelui sarac din poporul Meu, sa nu-l strâmtorezi ?i sa nu-i pui camata.
Exo 22:26  De vei lua zalog haina aproapelui tau, sa i-l întorci pâna la asfin?itul soarelui,
Exo 22:27  Caci aceasta este învelitoarea lui, aceasta este singura îmbracaminte pentru goliciunea sa. fn ce va dormi el? Deci de va striga catre Mine, îl voi auzi, pentru ca sunt milostiv.
Exo 22:28  Pe judecatori sa nu-i graie?ti de rau ?i pe capetenia poporului tau sa nu o hule?ti!
Exo 22:29  Nu întârzia a-Mi aduce pârga ariei tale ?i a teascului tau; pe cel întâi-nascut din fiii tai sa Mi-l dai Mic!
Exo 22:30  Asemenea sa faci cu boul tau, cu oaia ta ?i cu asinul tau: ?apte zile sa fie ei la mama lor, iar în ziua a opta sa Mi le dai Mie!
Exo 22:31  Sa-Mi fi?i popor sfânt; sa nu mânca?i carnea dobitocului sfâ?iat de fiara în câmp, ci s-o arunca?i la câini!"
Exo 23:1  "Sa nu iei aminte la zvon de?ert; sa nu te une?ti cu cel nedrept, ca sa fii martor mincinos!
Exo 23:2  Sa nu te iei dupa cei mai mul?i, ca sa faci rau; ?i la judecata sa nu urmezi celor mai mul?i, ca sa te aba?i de la dreptate;
Exo 23:3  Nici saracului sa nu-i fii partinitor la judecata!
Exo 23:4  De vei întâlni boul du?manului tau sau asinul lui ratacit, sa-l întorci ?i sa I-l duci!
Exo 23:5  De vei vedea asinul vrajma?ului tau cazut sub povara, sa nu-l treci cu vederea, ci sa-l ridici împreuna cu el.
Exo 23:6  Sa nu judeci strâmb pricina saracului tau!
Exo 23:7  De orice cuvânt mincinos sa te fere?ti; sa nu ucizi pe cel nevinovat ?i drept, caci Eu nu voi ierta pe nelegiuit.
Exo 23:8  Daruri sa nu prime?ti, caci darurile orbesc ochii celor ce vad ?i strâmba pricinile cele drepte.
Exo 23:9  Pe strain sa nu-l obijduie?ti, nici sa nu-l strâmtorezi, caci voi ?ti?i cum e sufletul pribeagului, ca ?i voi a?i fost pribegi în ?ara Egiptului.
Exo 23:10  ?ase ani sa semeni ?arina ta ?i sa aduni roadele ei,
Exo 23:11  Iar în al ?aptelea, las-o sa se odihneasca; ?i se vor hrani saracii poporului tau, iar rama?i?ele le vor mânca fiarele câmpului. A?a sa faci ?i cu via ta ?i cu maslinii tai.
Exo 23:12  În ?ase zile sa-?i faci treburile tale, iar în ziua a ?aptea sa te odihne?ti, ca sa se odihneasca ?i boul tau ?i asinul tau ?i ca sa rasufle fiul roabei tale ?i strainul care e cu tine.
Exo 23:13  Pazi?i toate câte v-am spus ?i numele altor dumnezei sa nu le pomeni?i, nici sa se auda ele din gura voastra.
Exo 23:14  De trei ori în an sa-Mi praznuie?ti:
Exo 23:15  Sa ?ii sarbatoarea azimelor. ?apte zile sa manânci azime în timpul lunii lui Aviv cum ?i-am poruncit, "ci în acea luna ai ie?it din Egipt; sa nu te înfa?i?ezi înaintea Mea cu mâna goala.
Exo 23:16  Sa ?ii apoi sarbatoarea seceri?ului ?i a strângerii celor dintâi roade ale tale, pe care le-ai semanat în ?arina ta, ?i sarbatoarea strângerii roadelor toamna, când aduni de pe câmp munca ta.
Exo 23:17  De trei ori pe an sa se înfa?i?eze înaintea Domnului Dumnezeului tau to?i cei de parte barbateasca ai tai.
Exo 23:18  Când voi alunga neamurile de la fa?a ta ?i voi largi hotarele tale, sa nu torni sângele jertfei tale pe dospit, nici grasimea de la jertfa Mea cea de la sarbatori sa nu ramâna pe a doua zi.
Exo 23:19  Pârga. din roadele ?arinii tale sa o aduci în casa Domnului Dumnezeului tau! Sa nu fierbi iedul în laptele mamei lui!
Exo 23:20  Iata Eu trimit înaintea ta pe îngerul Meu, ca sa te pazeasca în cale ?i sa te duca la pamântul acela pe care l-am pregatit pentru tine.
Exo 23:21  Ia aminte la tine însu?i; sa-l ascul?i ?i sa nu-i fi necredincios, ca nu te va ierta, pentru ca numele Meu este în el.
Exo 23:22  De vei asculta cu luare aminte glasul sau ?i vei face toate câte î?i poruncesc ?i de vei pazi legamântul Meu, Îmi ve?i fi popor ales dintre toate neamurile, ca al Meu este tot pamântul, iar voi Îmi ve?i fi preo?ie împarateasca ?i neam sfânt. Spune cuvintele acestea fiilor lui Israel: De ve?i asculta cu luare aminte glasul îngerului Meu ?i ve?i împlini toate câte va voi spune, voi fi vrajma? vrajma?ilor tai ?i potrivnicilor tai le voi fi potrivnic.
Exo 23:23  Când va merge înaintea ta îngerul Meu, pova?uitorul tau, ?i te va duce la Amorei, la Hetei, la Ferezei, ia Canaanei, la Gherghesei, la Hevei ?i la Iebusei, ?i-i voi stârpi pe ace?tia de la fa?a voastra,
Exo 23:24  Atunci sa nu te închini la dumnezeii lor, nici sa le sluje?ti, nici sa faci dupa faptele acelora, ci sa-i zdrobe?ti de tot ?i sa strici stâlpii lor.
Exo 23:25  Sa sluje?ti numai Domnului Dumnezeului tau ?i El va binecuvânta pâinea ta, vinul tau, apa ta ?i voi abate bolile de la voi.
Exo 23:26  În fa?a ta nu va fi femeie care sa nasca înainte de vreme sau stearpa; ?i voi umple numarul zilelor tale.
Exo 23:27  Groaza voi trimite înaintea ta ?i voi îngrozi de tot pe poporul asupra caruia ve?i merge ?i voi pune pe fuga pe to?i vrajma?ii tai.
Exo 23:28  Trimite-voi înaintea ta viespi ?i vor alunga de la fa?a voastra pe Amorei, pe Hevei, pe Iebusei, pe Canaanei ?i pe Hetei.
Exo 23:29  Dar nu-i voi alunga de la fa?a voastra într-un an, ca sa nu se pustiiasca pamântul ?i ca sa nu se înmul?easca asupra ta fiarele salbatice;
Exo 23:30  Ci-i voi alunga încetul cu încetul, pâna ce va ve?i înmul?i ?i ve?i lua în stapânire pamântul acela.
Exo 23:31  Întinde-voi hotarele tale de la Marea Ro?ie pâna la Marea Filistenilor ?i de la pustie pâna la râul cel mare al Eufratului, caci voi da în mâinile voastre pe locuitorii pamântului acestuia ?i-i voi alunga de la fa?a ta.
Exo 23:32  Sa nu va amesteca?i ?i sa nu face?i legamânt cu ei, nici cu dumnezeii lor.
Exo 23:33  Sa nu locuiasca ei în ?ara voastra, ca sa nu va faca sa pacatui?i împotriva Mea; ca de ve?i sluji dumnezeilor lor, ace?tia vor fi cursa pentru voi".
Exo 24:1  Apoi a zis Dumnezeu iara?i catre Moise: "Suie-te la Domnul, tu ?i Aaron, Nadab, Abiud ?i ?aptezeci dintre batrânii lui Israel ?i va închina?i Domnului de departe.
Exo 24:2  Numai Moise singur sa se apropie de Domnul, iar ceilal?i sa nu se apropie; poporul de asemenea sa nu se suie cu el!"
Exo 24:3  A venit deci Moise ?i a spus poporului toate cuvintele Domnului ?i legile. Atunci a raspuns tot poporul într-un glas ?i a zis: "Toate cuvintele pe care le-a grait Domnul le vom face ?i le vom asculta!"
Exo 24:4  Iar Moise a scris toate cuvintele Domnului. ?i el, sculându-se dis-de-diminea?a, a zidit jertfelnic sub munte ?i a pus doisprezece stâlpi, dupa cele douasprezece semin?ii ale lui Israel.
Exo 24:5  A trimis apoi tineri dintre fiii lui Israel, de au adus ace?tia arderi de tot ?i au jertfit vitei, ca jertfa de izbavire Domnului Dumnezeu.
Exo 24:6  Atunci Moise, luând jumatate din sânge, l-a turnat într-un vas, iar cu cealalta jumatate de sânge a stropit jertfelnicul.
Exo 24:7  Dupa aceea, luând cartea legamântului, a citit în auzul poporului; iar ei au zis: "Toate câte a grait Domnul le vom face ?i le vom asculta!"
Exo 24:8  Dupa aceea, luând Moise sângele, a stropit poporul, zicând: "Acesta este sângele legamântului, pe care l-a încheiat Domnul cu voi, dupa toate cuvintele acestea".
Exo 24:9  Apoi s-a suit Moise ?i Aaron, Nadab, Abiud ?i ?aptezeci dintre batrânii lui Israel
Exo 24:10  ?i au vazut locul unde statea Dumnezeul lui Israel; sub picioarele Lui era ceva, ce semana cu un lucru de safir, curat ?i limpede ca seninul cerului.
Exo 24:11  Dar El n-a întins mâna Sa împotriva ale?ilor lui Israel, iar ei au vazut pe Dumnezeu, apoi au mâncat ?i au baut.
Exo 24:12  ?i a zis Domnul catre Moise: "Suie-te la Mine în munte ?i fii acolo, ca am sa-?i dau table de piatra, legea ?i poruncile, pe care le-am scris Eu pentru înva?atura lor!"
Exo 24:13  Atunci, sculându-se Moise împreuna cu Iosua, slujitorul sau, s-a suit în muntele Domnului;
Exo 24:14  Iar batrânilor le-a zis: "Ramâne?i aici pâna ne vom întoarce la voi. Iata Aaron ?i Or sunt cu voi; de va avea cineva pricina, sa vina la ei".
Exo 24:15  S-a suit deci Moise ?i Iosua în munte ?i un nor a acoperit muntele.
Exo 24:16  Slava Domnului s-a pogorât pe Muntele Sinai ?i l-a acoperit norul ?ase zile, iar în ziua a ?aptea a strigat Domnul pe Moise din mijlocul norului.
Exo 24:17  Chipul slavei Domnului de pe vârful muntelui era în ochii fiilor lui Israel, ca un foc mistuitor.
Exo 24:18  ?i s-a suit Moise pe munte ?i a intrat în mijlocul norului; ?i a stat Moise pe munte patruzeci de zile ?i patruzeci de nop?i.
Exo 25:1  Atunci a grait Dumnezeu cu Moise ?i a zis:
Exo 25:2  "Spune fiilor lui Israel sa-Mi aduca prinoase: de la tot omul, pe care-l lasa inima sa dea, sa prime?ti prinoase pentru Mine.
Exo 25:3  Iar prinoasele ce vei primi de la ei sunt acestea: aur, argint ?i arama;
Exo 25:4  Matase violeta, purpurie ?i stacojie, în ?i par de capra.
Exo 25:5  Piei de berbec vopsite ro?u, piei de vi?el de mare ?i lemn de salcâm;
Exo 25:6  Untdelemn pentru candele, aromate pentru mirul de uns ?i pentru miresmele de tamâiere;
Exo 25:7  Piatra de sardiu ?i pietre de pus la efod ?i la ho?en.
Exo 25:8  Din acestea sa-Mi faci loca? sfânt ?i voi locui în mijlocul lor.
Exo 25:9  Cortul ?i toate vasele ?i obiectele lui sa le faci dupa modelul ce-?i voi arata Eu; a?a sa le faci!
Exo 25:10  Chivotul legii sa-l faci din lemn de salcâm: lung de doi co?i ?i jumatate, larg de un cot ?i jumatate ?i înalt de un cot ?i jumatate.
Exo 25:11  Sa-l fereci cu aur curat, ?i pe dinauntru ?i pe din afara. Sus, împrejurul lui, sa-i faci cununa împletita de aur.
Exo 25:12  Apoi sa torni pentru el patru inele de aur ?i sa le prinzi în cele patru col?uri de jos ale lui: doua inele pe o latura ?i doua inele pe cealalta latura.
Exo 25:13  Sa faci pârghii din lemn de salcâm ?i sa le îmbraci cu aur.
Exo 25:14  ?i sa vâri pârghiile prin inelele de pe laturile chivotului, încât cu ajutorul lor sa se poarte chivotul.
Exo 25:15  Pârghiile sa fie necontenit în inelele chivotului.
Exo 25:16  Iar în chivot sa pui legea, pe care ji-o voi da.
Exo 25:17  Sa faci ?i capac la chivot, de aur curat, lung de doi co?i ?i jumatate, ?i lat de un cot ?i jumatate.
Exo 25:18  Apoi sa faci doi heruvimi de aur; ?i sa-i faci ca dintr-o bucata, ca ?i cum ar rasari din cele doua capete ale capacului;
Exo 25:19  Sa pui un heruvim la un capat ?i un heruvim la celalalt capat al capacului.
Exo 25:20  ?i heruvimii sa-i faci ca ?i cum ar ie?i din capac. Heruvimii ace?tia sa fie cu aripile întinse pe deasupra capacului, acoperind cu aripile lor capacul, iar fe?ele sa ?i le aiba unul spre altul; spre capac sa fie fe?ele heruvimilor.
Exo 25:21  Apoi sa pui acest capac deasupra la chivot, iar în chivot sa pui legea ce î?i voi da.
Exo 25:22  Acolo, între cei doi heruvimi de deasupra chivotului legii, Ma voi descoperi ?ie ?i î?i voi grai de toate, câte am a porunci prin tine fiilor lui Israel.
Exo 25:23  Sa faci apoi masa din lemn de salcâm: lunga de doi co?i, lata de un cot, înalta de un cot ?i jumatate.
Exo 25:24  S-o îmbraci cu aur curat ?i sa-i faci împrejur cununa de aur, împletita.
Exo 25:25  Sa mai faci împrejurul ei pervaz înalt de o palma ?i împrejurul pervazului sa faci cununa de aur.
Exo 25:26  Sa mai faci patru inele de aur ?i sa prinzi cele patru inele sub cununa, în cele patru col?uri de la picioarele mesei.
Exo 25:27  Inelele sa fie în pervaz ca ni?te tor?i pentru pârghii, ca sa se poarte cu ele masa.
Exo 25:28  Iar pârghiile sa le faci din lemn de salcâm, sa le fereci cu aur curat ?i cu ele se va purta masa.
Exo 25:29  Apoi sa faci pentru ea talere, cadelni?e, pahare ?i cupe, ca sa torni cu ele; acestea sa le faci din aur curat.
Exo 25:30  Iar pe masa sa pui pâinile punerii înainte, care se vor afla pururea înaintea Mea.
Exo 25:31  Sa faci sfe?nic din aur curat. Sfe?nicul sa-l faci batut din ciocan: fusul, bra?ele, cupele, nodurile ?i florile lui sa fie dintr-o bucata.
Exo 25:32  ?ase bra?e sa iasa pe cele doua laturi ale lui: trei bra?e ale sfe?nicului sa fie pe o latura a lui ?i trei bra?e ale sfe?nicului sa fie pe cealalta latura.
Exo 25:33  El va avea la un bra? trei cupe în forma florii de migdal, cu nodurile ?i florile lor, ?i la alt bra? va avea trei cupe în forma florii de migdal, cu nodurile ?i florile lor. A?a vor avea toate cele ?ase bra?e, ce ies din fusul sfe?nicului.
Exo 25:34  Iar pe fusul sfe?nicului sa fie patru cupe în forma florii de migdal, cu nodurile ?i florile lor:
Exo 25:35  Un nod sub doua bra?e, un alt nod sub alte doua bra?e ?i un al treilea nod sub cele din urma doua bra?e, iar în vârful fusului sfe?nicului sa fie înca o cupa în forma florii de migdal cu nodul ?i floarea ei.
Exo 25:36  Nodurile ?i ramurile acestea sa fie dintr-o bucata cu sfe?nicul. El trebuie sa fie lucrat tot cu ciocanul, dintr-o singura bucata de aur curat.
Exo 25:37  Sa-i faci ?apte candele ?i sa pui în el candelele acestea, ca sa lumineze latura din fa?a lui.
Exo 25:38  Sa-i faci mucari ?i tavi?e de aur curat.
Exo 25:39  Dintr-un talant de aur curat sa se faca toate obiectele acestea.
Exo 25:40  Vezi sa faci acestea toate dupa modelul ce ?i s-a aratat în munte".
Exo 26:1  "Cortul însa sa-l faci din zece covoare de in rasucit ?i de matase violeta, stacojie ?i vi?inie; în ?esatura lor sa faci chipuri de heruvimi alese cu iscusin?a.
Exo 26:2  Lungimea fiecarui covor sa fie de douazeci ?i opt de co?i, ?i la?imea fiecarui covor sa fie de patru co?i: toate covoarele sa aiba aceea?i masura.
Exo 26:3  Cinci covoare se vor uni la un loc ?i celelalte cinci iar se vor uni la un loc.
Exo 26:4  Apoi sa faci cheotori de matase violeta, pe marginea covorului din capatul jumata?ii întâi de acoperi?, ?i tot asemenea cheotori sa faci pe marginea covorului din urma de la cealalta jumatate de acoperi?.
Exo 26:5  Cincizeci de cheotori sa faci la un covor ?i pe marginea covorului ce are a se uni cu el sa faci tot cincizeci de cheotori. Cheotorile acestea sa raspunda unele cu altele.
Exo 26:6  Sa faci cincizeci de copci din aur ?i cu copcile acestea sa une?ti cele doua jumata?i de acoperi?, ?i a?a va fi acoperi?ul cortului o singura bucata.
Exo 26:7  Apoi sa faci covoare din par de capra, ca sa acoperi cortul. Unsprezece covoare de acestea sa faci.
Exo 26:8  Lungimea unui covor sa fie de treizeci de co?i, ?i la?imea unui covor sa fie de patru co?i; cele unsprezece covoare sa aiba toate aceea?i masura.
Exo 26:9  Sa une?ti între ele cinci covoare ?i celelalte ?ase covoare iar sa le une?ti între ele. Jumatatea din al ?aselea covor sa o îndoi în partea de dinainte a cortului.
Exo 26:10  Sa faci cincizeci de cheotori pe marginea covorului din capatul unei jumata?i, ca sa se poata uni cu cealalta jumatate; ?i alte cincizeci de cheotori sa faci la marginea celeilalte jumata?i de acoperi?, care trebuie unita cu cea dintâi.
Exo 26:11  Sa faci apoi cincizeci de copci din arama ?i sa vâri copcile acestea în cheotori ?i sa une?ti cele doua jumata?i de acoperi?, ca sa fie unul singur.
Exo 26:12  Iar prisosul de acoperi?, o jumatate de covor, care prisose?te de la acoperi?ul cortului, sa atârne în partea dindarat a cortului.
Exo 26:13  Partea însa din lungimea acoperi?ului, care prisose?te de o parte ?i de alta a cortului, sa atârne peste pere?ii cortului, de o parte un cot ?i de alta un cot, ca sa-i apere.
Exo 26:14  Dupa aceea sa faci cortului un acoperi? de piei ro?ii de berbec ?i înca un acoperi? de piei de vi?el de mare pe deasupra.
Exo 26:15  Sa faci apoi pentru cort scânduri din lemn de salcâm, ca sa stea în picioare.
Exo 26:16  Fiecare scândura sa o faci lunga de zece co?i ?i lata de un cot ?i jumatate sa fie fiecare scândura.
Exo 26:17  O scândura sa aiba doua cepuri la capat, unul în dreptul altuia. A?a sa faci la toate scândurile cortului.
Exo 26:18  ?i scânduri de acestea pentru cort sa faci douazeci, pentru latura dinspre miazazi.
Exo 26:19  Sub aceste douazeci de scânduri sa faci patruzeci de postamente de argint: câte doua postamente sub o scândura, pentru cele doua cepuri ale ei ?i doua postamente pentru alta scândura, pentru cele doua cepuri ale ei.
Exo 26:20  Douazeci de scânduri sa faci pentru cealalta latura, dinspre miazanoapte.
Exo 26:21  ?i pentru acestea sa faci patruzeci de postamente de argint, câte doua postamente sub o scândura ?i doua postamente pentru alta scândura;
Exo 26:22  Iar pentru partea dindarat a cortului, care vine spre asfin?it, sa faci ?ase scânduri.
Exo 26:23  ?i doua scânduri sa faci pentru unghiurile cortului din partea dindarat a lui.
Exo 26:24  Acestea sa fie de doua ori mai groase ?i sus unite prin câte un inel. A?a trebuie sa fie amândoua la fel pentru amândoua unghiurile.
Exo 26:25  ?i a?a vor fi opt scânduri în partea dindarat a cortului ?i pentru cele ?aisprezece postamente de argint, câte doua postamente sub fiecare scândura, pentru cele doua cepuri ale ei.
Exo 26:26  Sa faci apoi pârghii din lemn de salcâm: cinci pârghii pentru scândurile de pe o latura a cortului,
Exo 26:27  Cinci pârghii pentru scândurile de pe cealalta latura a cortului ?i cinci pârghii pentru scândurile din partea de la fundul cortului, care vine spre asfin?it.
Exo 26:28  Iar pârghia din mijloc va trece prin scânduri de la un capat la celalalt al cortului.
Exo 26:29  Scândurile sa le îmbraci cu aur; inelele, prin care se vâra pârghiile, sa le faci de aur ?i sa îmbraci cu aur ?i pârghiile.
Exo 26:30  ?i a?a sa înjghebezi cortul dupa modelul care ti s-a aratat în acest munte.
Exo 26:31  Sa faci o perdea de in rasucit ?i de matase violeta, stacojie ?i vi?inie, rasucita, iar în ?esatura ei sa aiba chipuri de heruvimi alese cu iscusin?a;
Exo 26:32  ?i s-o atârni cu verigi de aur pe patru stâlpi din lemn de salcâm, îmbraca?i cu aur ?i a?eza?i pe patru postamente de argint.
Exo 26:33  Dupa ce vei prinde perdeaua în copci, sa aduci acolo dupa perdea chivotul legii ?i perdeaua va va despar?i astfel sfânta de sfânta sfintelor.
Exo 26:34  La chivotul legii din sfânta sfintelor sa pui capacul.
Exo 26:35  Iar dincoace, în afara de perdea, sa pui masa ?i în fa?a mesei sa pui sfe?nicul; în partea de miazazi a cortului sa-l pui. Masa însa s-o pui în partea de miazanoapte a cortului.
Exo 26:36  Apoi sa faci o perdea la u?a cortului, de matase violeta, stacojie ?i vi?inie, rasucita ?i de in rasucit, cu flori alese în ?esatura ei.
Exo 26:37  Pentru perdeaua aceasta sa faci cinci stâlpi din lemn de salcâm ?i sa-i îmbraci cu aur. La ei sa faci verigi de aur ?i sa torni pentru ei cinci postamente de arama".
Exo 27:1  "Sa faci un jertfelnic din lemn de salcâm, lung de cinci co?i, lat de cinci co?i. Jertfelnicul sa fie în patru col?uri ?i înal?imea lui de trei co?i.
Exo 27:2  În cele patru col?uri ale lui sa faci coarne. Coarnele sa fie ca rasarite din el ?i sa-l îmbraci cu arama.
Exo 27:3  Apoi sa-i faci caldari pentru pus cenu?a, lopa?ele, lighene, furculi?e ?i cle?te. Toate uneltele acestea sa le faci de arama.
Exo 27:4  Sa faci jertfelnicului o împletitura, un fel de cama?a din sârma de arama, ?i la împletitura aceasta sa faci, în cele patru col?uri ale ei, patru inele de arama.
Exo 27:5  Sa îmbraci cu cama?a aceasta partea de jos a jertfelnicului, ca sa fie împletitura pâna la jumatatea jertfelnicului.
Exo 27:6  Sa faci pentru jertfelnic pârghii, din lemn de salcâm, ?i sa le îmbraci cu arama.
Exo 27:7  Drugii ace?tia sa-i vâri prin inele, pe o parte ?i pe alta a jertfelnicului, ca sa poata fi purtat.
Exo 27:8  Iar jertfelnicul sa-l faci de scânduri, gol înauntru. Dupa cum ?i s-a aratat în munte, a?a sa-l faci.
Exo 27:9  Cortului sa-i faci curte. Pe partea dinspre miazazi, perdelele sa fie de in rasucit, lungi de o suta de coti numai pe partea aceasta.
Exo 27:10  Pentru ele sa faci douazeci de stâlpi ?i pentru ei douazeci de postamente de arama; cârligele la stâlpi ?i verigile lor sa fie de argint.
Exo 27:11  Tot a?a ?i pe latura dinspre miazanoapte sa fie perdele de o suta de co?i în lungime ?i la ele douazeci de stâlpi, iar sub ei douazeci de postamente de arama; cârligele ?i verigile stâlpilor sa fie de argint.
Exo 27:12  În latul cur?ii, pe partea dinspre asfin?it, sa fie perdelele lungi de cincizeci de co?i ?i la ele zece stâlpi ?i la stâlpi zece postamente.
Exo 27:13  Tot de cincizeci de co?i sa fie perdelele din latul cur?ii în partea dinainte, cea dinspre rasarit; ?i la ele zece stâlpi ?i sub ei zece postamente.
Exo 27:14  Din ace?tia, cincisprezece co?i la un capat al laturii sa fie perdelele, cu trei stâlpi ai lor ?i cu trei postamente,
Exo 27:15  ?i la celalalt capat cincisprezece co?i sa fie perdele la fel, cu trei stâlpi ai lor ?i cu trei postamente.
Exo 27:16  Iar la mijloc, poarta cur?ii, larga da douazeci de co?i, sa aiba perdele de lâna violeta, stacojie ?i vi?inie, rasucita ?i de in rasucit, cu patru stâlpi ?i patru postamente.
Exo 27:17  To?i stâlpii cur?ii împrejur sa fie fereca?i cu argint ?i uni?i cu legaturi de argint, iar postamentele lor sa fie de arama.
Exo 27:18  A?adar lungimea cur?ii sa fie de o suta de co?i, la?imea peste tot de cincizeci de co?i, înal?imea de cinci co?i, perdelele sa fie de in rasucit, iar postamentele stâlpilor de arama.
Exo 27:19  Toate lucrurile, toate uneltele ?i to?i ?aru?ii cur?ii sa fie de arama.
Exo 27:20  Sa porunce?ti fiilor lui Israel sa-?i aduca untdelemn curat pentru luminat, stors din masline, ca sa arda sfe?nicul în toata vremea, în cortul adunarii, în fa?a perdelei celei de dinaintea chivotului legii.
Exo 27:21  Sfe?nicul îl va aprinde Aaron ?i fiii lui, de seara pâna diminea?a, înaintea Domnului. Aceasta e lege ve?nica pentru fiii lui Israel din neam în neam".
Exo 28:1  "Sa iei la tine pe Aaron, fratele tau, ?i pe fiii lui, ca dintre fiii lui Israel sa-Mi fie preo?i Aaron ?i fiii lui Aaron: Nadab, Abiud, Eleazar ?i Itamar.
Exo 28:2  Sa faci lui Aaron, fratele tau, ve?minte sfin?ite, spre cinste ?i podoaba.
Exo 28:3  Sa spui dar, la to?i cei iscusi?i, pe care i-am umplut de duhul în?elepciunii ?i al priceperii, sa faca lui Aaron ve?minte sfin?ite pentru ziua sfin?irii lui, cu care sa-Mi slujeasca.
Exo 28:4  Iata dar ve?mintele ce trebuie sa faca: ho?en, efod, meil, hiton, chidar ?i cingatoare. Acestea sunt ve?mintele sfin?ite, ce trebuie sa faca ei lui Aaron, fratele tau, ?i fiilor lui, ca sa-Mi slujeasca ei ca preo?i.
Exo 28:5  Pentru acestea vor lua aur curat ?i matasuri violete, purpurii ?i stacojii ?i în sub?ire
Exo 28:6  ?i vor face efod lucrat cu iscusin?a din fire de aur, de matase violeta, stacojie ?i vi?inie, rasucita ?i de in rasucit.
Exo 28:7  Acesta va fi din doua buca?i: una pe piept ?i alta pe spate, unite pe umeri cu doua încheietori.
Exo 28:8  Cingatoarea efodului, care vine peste el, sa fie lucrata la fel cu el, din fire de aur curat, de matase violeta, stacojie ?i vi?inie, rasucita ?i de in rasucit
Exo 28:9  Apoi sa iei doua pietre, amândoua pietrele sa fie de smarald, ?i sa sapi pe ele numele fiilor lui Israel:
Exo 28:10  ?ase nume pe o piatra ?i celelalte ?ase nume pe cealalta piatra, dupa rânduiala în care s-au nascut ei.
Exo 28:11  Cum fac sapatorii în piatra, care sapa pece?i, a?a sa fie sapatura pe cele doua pietre cu numele fiilor lui Israel ?i sa a?ezi pietrele în cuibule?e de aur curat.
Exo 28:12  Aceste doua pietre sa le pui încheietori la efod. Pietrele acestea vor fi spre pomenirea fiilor lui Israel ?i Aaron va purta numele fiilor lui Israel, spre pomenire înaintea Domnului, pe amândoi umerii sai.
Exo 28:13  Sa faci cuibule?e de aur curat.
Exo 28:14  Apoi sa faci doua lan?i?oare tot de aur curat; acestea sa le faci, rasucite ca sfoara; ?i sa prinzi lan?i?oarele cele rasucite de cuibule?ele de la încheietorile efodului, în partea de dinainte.
Exo 28:15  Sa faci ho?enul judeca?ii, lucrat cu iscusin?a, la fel cu efodul: din fire de aur, de matase violeta, stacojie, vi?inie ?i de in rasucit.
Exo 28:16  Acesta sa fie îndoit, în patru col?uri, lung de o palma ?i lat de o palma.
Exo 28:17  Pe el sa a?ezi o înfloritura de pietre scumpe, în?irate în patru rânduri. Un rând de pietre sa fie: un sardeon, un topaz ?i un smarald; acesta e rândul întâi.
Exo 28:18  În rândul al doilea: un rubin, un safir ?i un diamant;
Exo 28:19  În rândul al treilea: un opal, o agata ?i un ametist;
Exo 28:20  ?i în rândul al patrulea: un hrisolit, un onix ?i un iaspis. Acestea trebuie sa fie a?ezate dupa rânduiala lor în cuibule?e de aur.
Exo 28:21  Pietrele acestea trebuie sa fie în numar de douasprezece, dupa numarul numelor celor doisprezece fii ai lui Israel, în?irate pe cele doua pietre de pe umeri, dupa numele lor ?i dupa rânduiala în care s au nascut ei. Pe fiecare trebuie sa sapi, ca pe pecete, câte un nume din numarul celor douasprezece semin?ii.
Exo 28:22  Apoi sa faci pentru ho?en lan?i?oare de aur curat, lucrat rasucit, ca sfoara.
Exo 28:23  Sa mai faci pentru ho?en doua verigi de aur ?i aceste doua verigi de aur sa le prinzi de cele doua col?uri de sus ale ho?enului;
Exo 28:24  Sa introduci cele doua lan?i?oare împletite de aur în cele doua verigi din cele doua col?uri ale ho?enului
Exo 28:25  ?i sa prinzi celelalte doua capete ale lan?i?oarelor de cuibule?ele efodului de pe umeri, în partea de dinainte.
Exo 28:26  ?i sa mai faci doua verigi de aur ?i sa le prinzi de col?urile de jos ale ho?enului, care cad pe cingatoarea efodului.
Exo 28:27  Apoi sa mai faci înca doua verigi de aur ?i sa le prinzi de cele doua margini de jos ale efodului, pe partea de dinainte, deasupra cingatorii efodului
Exo 28:28  ?i sa prinzi verigile ho?enului de verigile efodului cu un ?nur de matase albastra, ca sa stea peste cingatoarea efodului ?i ca ho?enul sa nu se mi?te de pe efod.
Exo 28:29  ?i va purta Aaron, când va intra în cortul adunarii, numele fiilor lui Israel pe ho?enul judeca?ii, la inima sa, spre ve?nica pomenire înaintea Domnului.
Exo 28:30  În ho?enul judeca?ii sa pui Urim ?i Tumim; ?i vor fi acestea la inima lui Aaron, când va intra el în cortul adunarii sa se înfa?i?eze înaintea Domnului. Astfel va purta Aaron pururea la inima sa judecata fiilor lui Israel, înaintea Domnului.
Exo 28:31  Sa faci apoi meilul de sub efod tot de matase vi?inie.
Exo 28:32  Acesta va avea la mijloc, sus, o deschizatura pentru cap ?i deschizatura sa aiba împrejur un guler ?esut ca plato?a, ca sa nu se rupa.
Exo 28:33  Iar pe la poale îi vei face de jur împrejur ciucuri tot de matase violeta, stacojie, vi?inie ?i de in rasucit;
Exo 28:34  ?i printre ciucuri vei pune clopo?ei de aur de jur împrejur a?a: un ciucure ?i un clopo?el de aur, un ciucure ?i un clopo?el de aur.
Exo 28:35  ?i acesta va fi pe Aaron în timpul slujbei, când va intra în cortul sfânt, înaintea Domnului, ?i când va ie?i, ca sa se auda sunetul clopo?eilor ?i sa nu moara.
Exo 28:36  Sa faci dupa aceea o tabli?a ?lefuita, de aur curat, ?i sa sapi pe ea, cum se sapa pe pecete, cuvintele: "Sfin?enia Domnului",
Exo 28:37  ?i s-o prinzi cu ?nur de matase violeta de chidar, a?a ca sa vina în partea de dinainte a chidarului.
Exo 28:38  Aceasta va fi pe fruntea lui Aaron ?i Aaron va purta pe fruntea sa neajunsurile prinoaselor afierosite de fiii lui Israel ?i ale tuturor darurilor aduse de ei; ea va fi pururea pe fruntea lui, pentru a atrage bunavoin?a Domnului spre ei.
Exo 28:39  Hitonul sa-l faci de in ?i tot de in sa faci ?i mitra, iar cingatoarea sa o faci brodata cu matase de felurite culori.
Exo 28:40  Sa faci de asemenea ?i fiilor lui Aaron hitoane ?i cingatori; ?i sa le faci ?i turbane pentru cinste ?i podoaba.
Exo 28:41  Sa îmbraci cu acestea pe fratele tau Aaron ?i împreuna cu el ?i pe fiii lui, sa-i ungi, sa-i întare?ti în slujbele lor ?i sa-i sfin?e?ti, ca sa-Mi fie preo?i.
Exo 28:42  Sa le faci pantaloni de in, de la brâu pâna sub genunchi, ca sa-?i acopere goliciunea trupului lor;
Exo 28:43  Aaron ?i fiii lui sa se îmbrace când vor intra în cortul adunarii sau când se vor apropia de jertfelnic, în sfânta, ca sa slujeasca, pentru a nu-?i atrage pacat asupra lor ?i sa moara. Aceasta sa fie lege ve?nica pentru el ?i pentru urma?ii lui e
Exo 29:1  "Iata ce trebuie sa savâr?e?ti asupra lor, când îi vei sfin?i sa-Mi fie preo?i: sa iei un vi?el din cireada, doi berbeci fara meteahna,
Exo 29:2  Pâini nedospite, azime framântate cu untdelemn ?i turte nedospite, unse cu untdelemn. Acestea sa le faci din faina de grâu aleasa.
Exo 29:3  Sa le pui într-un paner ?i sa le aduci în paner la cortul adunarii o data cu vi?elul ?i cu berbecii.
Exo 29:4  Apoi sa aduci pe Aaron ?i pe fiii lui la intrarea cortului adunarii ?i sa-i speli cu apa.
Exo 29:5  ?i, luând ve?mintele sfinte, sa îmbraci pe Aaron, fratele tau, cu hitonul ?i cu meilul, cu efodul ?i cu ho?enul, ?i sa-l încingi peste efod;
Exo 29:6  Sa-i pui pe cap mitra, iar la mitra sa prinzi diadema sfin?eniei.
Exo 29:7  Apoi sa iei untdelemn de ungere ?i sa-i torni pe cap ?i sa-l ungi.
Exo 29:8  Dupa aceea sa aduci ?i pe fiii lui ?i sa-i îmbraci cu hitoane;
Exo 29:9  Sa-i încingi cu brâie ?i sa le pui turbanele; ?i-Mi vor fi preo?i în veac. A?a vei sfin?i tu pe Aaron ?i pe fiii lui.
Exo 29:10  Sa aduci apoi vi?elul înaintea cortului adunarii ?i sa-?i puna Aaron ?i fiii lui mâinile pe capul vi?elului, înaintea Domnului, la u?a cortului adunarii.
Exo 29:11  Sa junghii vi?elul înaintea Domnului, la u?a cortului adunarii.
Exo 29:12  Sa iei din sângele vi?elului ?i sa pui cu degetul tau pe coarnele jertfelnicului, iar celalalt sânge sa-l torni tot la temelia jertfelnicului.
Exo 29:13  Apoi sa iei toata grasimea cea de pe maruntaie, seul de pe ficat, amândoi rarunchii ?i grasimea de pe ei ?i sa le arzi pe jertfelnic.
Exo 29:14  Iar carnea vi?elului, pielea lui ?i necura?eniile lui sa le arzi cu foc afara din tabara. Caci e jertfa pentru pacat.
Exo 29:15  Dupa aceea sa iei un berbec ?i sa-?i puna Aaron ?i fiii lui mâinile pe capul berbecului;
Exo 29:16  Sa junghii berbecul ?i, luând sângele lui, sa strope?ti jertfelnicul de jur împrejur.
Exo 29:17  Sa tai apoi berbecul în buca?i, sa speli cu apa maruntaiele ?i picioarele ?i sa le pui pe buca?i lânga capa?âna lui.
Exo 29:18  ?i sa arzi berbecul tot pe jertfelnic. Aceasta este ardere de tot Domnului, jertfa Domnului, mireasma placuta înaintea Lui.
Exo 29:19  Sa iei ?i celalalt berbec, sa-?i puna Aaron ?i fiii lui mâinile pe capul berbecului ?i sa-l junghii;
Exo 29:20  Sa iei din sângele lui ?i sa pui pe vârful urechii drepte a lui Aaron, pe vârful degetului mare al mâinii drepte, pe vârful degetului mare al piciorului drept ?i pe vârful urechilor drepte ale fiilor lui ?i pe vârful degetelor mari ale mâinilor drepte ale lor ?i pe vârful degetelor mari ale picioarelor drepte ale lor. ?i sa strope?ti cu sânge jertfelnicul pe toate par?ile.
Exo 29:21  Sa iei din sângele de pe jertfelnic ?i din untdelemnul de ungere ?i sa strope?ti asupra lui Aaron ?i asupra ve?mintelor lui, asupra fiilor lui ?i asupra ve?mintelor fiilor lui, ?i se va sfin?i el ?i ve?mintele lui, fiii lui ?i ve?mintele fiilor lui. Iar celalalt sânge al berbecului sa-l torni lânga altar împrejur.
Exo 29:22  Apoi sa iei din berbec grasimea ?i coada lui, grasimea ce acopera maruntaiele, seul de pe ficat, amândoi rarunchii ?i grasimea de pe ei ?i ?oldul drept; pentru ca acesta este berbecul pentru sfin?irea în preot;
Exo 29:23  Sa mai iei o pâine, din cele cu untdelemn, o turta cu untdelemn ?i o azima din panerul ce este pus înaintea Domnului;
Exo 29:24  Sa le pui toate pe bra?ele lui Aaron ?i pe bra?ele fiilor lui, ca sa le aduca, leganându-le, înaintea Domnului.
Exo 29:25  Apoi sa iei acestea din mâinile lor ?i sa le arzi pe jertfelnic, ardere de tot, spre buna mireasma înaintea Domnului; aceasta este jertfa Domnului.
Exo 29:26  Sa iei pieptul berbecului, care este pentru sfin?irea lui Aaron, ?i sa-l duci înaintea Domnului leganându-l; aceasta va fi partea ta.
Exo 29:27  Sa sfin?e?ti pieptul leganat ?i spata leganata, care au fost ridicate înaintea Domnului, din berbecul cel pentru sfin?irea lui Aaron ?i a fiilor lui.
Exo 29:28  ?i acestea sa fie prin lege ve?nica pentru Aaron ?i fiii lui din cele ce aduc fiii lui Israel, caci acesta e dar ridicat din cele ce vor aduce fiii lui Israel ca jertfa de pace, darul ridicat Domnului.
Exo 29:29  Ve?mintele sfinte cele pentru Aaron sa fie, dupa el, ale fiilor sai, ?i sa fie un?i ?i sfin?i?i, îmbraca?i cu ele.
Exo 29:30  Marele preot dintre fiii lui, care-i va urma ?i care va intra în cortul adunarii, ca sa slujeasca în locul cel sfânt, se va îmbraca cu ele ?apte zile.
Exo 29:31  Sa iei apoi berbecul cel pentru sfin?ire ?i sa fierbi carnea lui în locul cel sfânt;
Exo 29:32  ?i sa manânce Aaron ?i fiii lui carnea berbecului acestuia ?i pâinile cele din paner, la u?a cortului adunarii,
Exo 29:33  Ca prin aceasta s-a facut cura?irea lor pentru a fi sfin?i?i ?i pentru a li se încredin?a preo?ia; nimeni altul sa nu manânce, ca sunt sfin?ite.
Exo 29:34  Iar de va ramâne din aceasta carne de sfin?ire ?i din pâini pe a doua zi, sa arzi rama?i?ele cu foc ?i sa nu se manânce, ca este lucru sfin?it.
Exo 29:35  Deci a?a sa faci cu Aaron ?i cu fiii lui, dupa cum ti-am poruncit: ?apte zile sa tina sfin?irea lor.
Exo 29:36  Vi?elul cel de jertfa pentru pacat sa-l aduci în fiecare zi pentru cura?ire; jertfa pentru pacat s-o savâr?e?ti pe jertfelnic pentru cura?irea lui ?i sa-l ungi pentru sfin?irea lui.
Exo 29:37  ?apte zile sa cure?i astfel jertfelnicul ?i sa-l sfin?e?ti ?i va fi jertfelnicul sfin?enie mare; tot ce se va atinge de el se va sfin?i.
Exo 29:38  Iata ce vei aduce tu pe jertfelnic: doi miei de un an fara meteahna vei aduce în fiecare zi, jertfa necontenita:
Exo 29:39  Un miel sa-l aduci diminea?a ?i celalalt miel sa-l aduci seara.
Exo 29:40  A zecea parte dintr-o efa de faina de grâu, framântata cu a patra parte dintr-un hin de untdelemn curat, iar pentru turnare, a patra parte de hin de vin, pentru un miel.
Exo 29:41  Al doilea miel sa-l aduci seara cu dar de faina, ca ?i pe cel de diminea?a, ?i cu aceea?i turnare de vin; sa-l aduci jertfa Domnului întru miros cu buna mireasma.
Exo 29:42  Aceasta va fi jertfa necontenita în neamul vostru, la u?ile cortului marturiei, unde Ma voi arata pe Mine Însumi voua ca sa graiesc cu tine.
Exo 29:43  Acolo Ma voi pogorî Eu Însumi la fiii lui Israel ?i se va sfin?i locul acesta de slava Mea.
Exo 29:44  Voi sfin?i cortul adunarii ?i jertfelnicul; pe Aaron ?i pe fiii lui de asemenea îi voi sfin?i, ca sa-Mi fie preo?i.
Exo 29:45  ?i voi locui în mijlocul fiilor lui Israel ?i le voi fi Dumnezeu;
Exo 29:46  ?i vor cunoa?te ca Eu, Domnul, sunt Dumnezeul lor, Cel ce i-am scos din pamântul Egiptului, ca sa locuiesc în mijlocul lor ?i sa le fiu Dumnezeu!"
Exo 30:1  "Sa faci de asemenea un jertfelnic de tamâiere, din lemn de salcâm.
Exo 30:2  Dar sa-l faci patrat, lung de un cot ?i lat de un cot ?i înalt de doi coli; coarnele lui sa fie din el.
Exo 30:3  Sa îmbraci cu aur curat partea lui de sus, pere?ii împrejur ?i coarnele lui; ?i sa-i faci împrejur o cununa de aur împletita.
Exo 30:4  Sub cununa lui împletita sa-i faci doua inele de aur curat ?i sa le pui la doua col?uri pe doua laturi ale lui. Acestea sa fie de bagat pârghiile pentru a-l purta cu ele.
Exo 30:5  Pârghiile sa i le faci din lemn de salcâm ?i sa le îmbraci cu aur.
Exo 30:6  Jertfelnicul sa-l a?ezi în fa?a perdelei, care este dinaintea chivotului legii, unde am sa Ma arat Eu ?ie.
Exo 30:7  Pe el Aaron va arde tamâie mirositoare în fiecare diminea?a, când pregate?te candelele.
Exo 30:8  Când va aprinde Aaron seara candelele, iar va arde miresme. Aceasta tamâiere neîntrerupta se va face pururea înaintea Domnului din neam în neam.
Exo 30:9  Sa nu aduce?i pe el nici o ardere de tamâie straina, nici ardere de tot, nici dar de pâine, nici turnare sa nu turna?i pe el.
Exo 30:10  Aaron va savâr?i jertfa de cura?ire peste coarnele lui o data pe an; cu sânge din jertfa de cura?ire cea pentru pacat îl va cura?i el o data pe an; în neamul vostru aceasta este mare sfin?enie înaintea Domnului".
Exo 30:11  Apoi a grait Domnul cu Moise ?i a zis:
Exo 30:12  "Când vei face numaratoarea fiilor lui Israel, la cercetarea lor, sa dea fiecare Domnului rascumparare ca sa nu vina nici o nenorocire asupra lor în timpul numaratului.
Exo 30:13  Cel ce intra la numaratoare sa dea jumatate de siclu, dupa siclul sfânt, care are douazeci de ghere; deci darul Domnului va fi o jumatate de siclu.
Exo 30:14  Tot cel ce intra la numaratoare, de la douazeci de ani în sus, sa aduca aceasta dare Domnului.
Exo 30:15  Sa nu dea bogatul mai mult, nici saracul mai pu?in de jumatate de siclu dar Domnului, pentru rascumpararea sufletului.
Exo 30:16  Sa iei argintul de rascumparare de la fiii lui Israel ?i sa-l dai la trebuin?ele cortului adunarii ?i va fi pentru fiii lui Israel spre pomenire înaintea Domnului, ca sa cru?e sufletele voastre".
Exo 30:17  ?i iara?i a grait Domnul cu Moise ?i a zis:
Exo 30:18  "Sa faci o baie de arama, cu postament de arama, pentru spalat; s-o pui între cortul adunarii ?i jertfelnic ?i sa torni într-însa apa.
Exo 30:19  Aaron ?i fiii lui î?i vor spala în ea mâinile ?i picioarele lor cu apa.
Exo 30:20  Când trebuie sa intre ei în cortul marturiei, sa se spele cu apa de aceasta, ca sa nu moara; ?i când trebuie sa se apropie de jertfelnic, ca sa slujeasca ?i ca sa aduca ardere de tot Domnului, sa-?i spele mâinile lor ?i picioarele lor cu apa, ca sa nu moara.
Exo 30:21  ?i va fi aceasta rânduiala ve?nica pentru el ?i pentru urma?ii lui, din neam în neam".
Exo 30:22  Apoi a grait Domnul cu Moise ?i a zis:
Exo 30:23  "Sa iei din cele mai bune mirodenii: cinci sute sicli de smirna aleasa; jumatate din aceasta, adica doua sute cincizeci sicli de scor?i?oara mirositoare; doua sute cincizeci sicli trestie mirositoare;
Exo 30:24  Cinci sute sicli casie, dupa siclul sfânt, ?i untdelemn de masline un hin,
Exo 30:25  ?i sa faci din acestea mir pentru ungerea sfânta, mir alcatuit dupa me?te?ugul facatorilor de aromate; acesta va fi mirul pentru sfânta ungere.
Exo 30:26  Sa ungi cu el cortul adunarii, chivotul legii ?i toate lucrurile din cort,
Exo 30:27  Masa ?i toate cele de pe ea, sfe?nicul ?i toate lucrurile lui, jertfelnicul tamâierii,
Exo 30:28  Jertfelnicul arderii de tot ?i toate lucrurile lui ?i baia ?i postamentul ei.
Exo 30:29  ?i sa le sfin?e?ti pe acestea ?i va fi sfin?enie mare; tot ce se va atinge de ele se va sfin?i.
Exo 30:30  Sa ungi de asemenea ?i pe Aaron ?i pe fiii lui ?i sa-i sfin?e?ti, ca sa-Mi fie preo?i.
Exo 30:31  Iar fiilor lui Israel sa le spui: Acesta va fi pentru voi mirul sfintei ungeri, în numele Meu, în neamul vostru.
Exo 30:32  Trupurile celorlal?i oameni sa nu le ungi cu el ?i dupa chipul alcatuirii lui sa nu va face?i pentru voi mir la fel. Acesta este lucru sfânt ?i sfânt trebuie sa fie ?i pentru voi.
Exo 30:33  Cine î?i va face ceva asemanator lui, sau cine se va unge cu el din cei ce nu trebuie sa se unga, acela se va stârpi din poporul sau".
Exo 30:34  Apoi a zis Domnul catre Moise: "Ia-?i mirodenii: stacte, oniha, halvan mirositor ?i tamâie curata, din toate aceea?i masura,
Exo 30:35  ?i fa din ele, cu ajutorul me?te?ugului facatorilor de aromate, un amestec de tamâiat, cu adaos de sare, curat ?i sfânt;
Exo 30:36  Piseaza-l marunt ?i-l pune înaintea chivotului legii, în cortul adunarii, unde am sa Ma arat ?ie. Aceasta va fi pentru voi sfin?enie mare.
Exo 30:37  Tamâie, alcatuita în felul acesta, sa nu va face?i pentru voi: sfin?enie sa va fie ea pentru Domnul.
Exo 30:38  Cine î?i va face asemenea amestec, ca sa afume cu el, sufletul acela se va stârpi din poporul sau".
Exo 31:1  Dupa aceea a grait Domnul cu Moise ?i a zis:
Exo 31:2  "Iata, Eu am rânduit anume pe Be?aleel, fiul lui Uri, fiul lui Or, din semin?ia lui Iuda,
Exo 31:3  ?i l-am umplut de duh dumnezeiesc, de în?elepciune, de pricepere, de ?tiin?a ?i de iscusin?a la tot lucrul,
Exo 31:4  Ca sa faca lucruri de aur, de argint ?i de arama, de matase violeta, stacojie ?i vi?inie, ?i de in rasucit,
Exo 31:5  Sa ?lefuiasca pietre scumpe pentru podoabe ?i sa sape în lemn tot felul de lucruri.
Exo 31:6  ?i iata, i-am dat ca ajutor pe Oholiab, fiul lui Ahisamac, din semin?ia lui Dan, ?i am pus în?elepciune în mintea oricarui om iscusit, ca sa faca toate câte ?i-am poruncit:
Exo 31:7  Cortul adunarii, chivotul legii, capacul cel de deasupra lui ?i toate lucrurile cortului;
Exo 31:8  Masa ?i toate vasele ei; sfe?nicul cel de aur curat cu toate obiectele lui ?i jertfelnicul tamâierii;
Exo 31:9  Jertfelnicul pentru arderile de tot cu toate obiectele lui; baia ?i postamentul ei;
Exo 31:10  ?esaturile pentru înveli?, ve?mintele sfin?ite pentru Aaron preotul ?i ve?mintele de slujba pentru fiii lui;
Exo 31:11  Mirul pentru ungere ?i aromatele mirositoare pentru loca?ul cel sfânt; toate le vor face ei a?a, cum ti-am poruncit Eu ?ie".
Exo 31:12  ?i a mai vorbit Domnul cu Moise ?i a zis:
Exo 31:13  "Spune fiilor lui Israel a?a: Baga?i de seama sa pazi?i zilele Mele de odihna, caci acestea sunt semn între Mine ?i voi din neam în neam, ca sa ?ti?i ca Eu sunt Domnul, Cel ce va sfin?e?te.
Exo 31:14  Pazi?i deci ziua de odihna, caci ea este sfânta pentru voi. Cel ce o va întina, acela va fi omorât; tot cel ce va face într-însa vreo lucrare, sufletul acela va fi stârpit din poporul Meu;
Exo 31:15  ?ase zile sa lucreze, iar ziua a ?aptea este zi de odihna, închinata Domnului; tot cel ce va munci în ziua odihnei va fi omorât.
Exo 31:16  Sa pazeasca deci fiii lui Israel ziua odihnei, praznuind ziua odihnei din neam în neam, ca un legamânt ve?nic.
Exo 31:17  Acesta este semn ve?nic între Mine ?i fiii lui Israel, pentru ca în ?ase zile a facut Domnul cerul ?i pamântul, iar în ziua a ?aptea a încetat ?i S-a odihnit".
Exo 31:18  Dupa ce a încetat Dumnezeu de a grai cu Moise, pe Muntele Sinai, i-a dat cele doua table ale legii, table de piatra, scrise cu degetul lui Dumnezeu.
Exo 32:1  Vazând însa poporul ca Moise întârzie a se pogorî din munte, s-a adunat la Aaron ?i i-a zis: "Scoala ?i ne fa dumnezei, care sa mearga înaintea noastra, caci cu omul acesta, cu Moise, care ne-a scos din ?ara Egiptului, nu ?tim ce s-a întâmplat".
Exo 32:2  Iar Aaron le-a zis: "Scoate?i cerceii de aur din urechile femeilor voastre, ale feciorilor vo?tri ?i ale fetelor voastre ?i-i aduce?i la mine".
Exo 32:3  Atunci tot poporul a scos cerceii cei de aur din urechile alor sai ?i i-a adus la Aaron.
Exo 32:4  Luându-i din mâinile lor, i-a turnat în tipar ?i a facut din ei un vi?el turnat ?i l-a cioplit cu dalta. Iar ei au zis: "Iata, Israele, dumnezeul tau, care te-a scos din ?ara Egiptului!
Exo 32:5  Vazând aceasta, Aaron a zidit înaintea lui un jertfelnic; ?i a strigat Aaron ?i a zis: "Mâine este sarbatoarea Domnului!"
Exo 32:6  A doua zi s-au sculat ei de diminea?a ?i au adus arderi de tot ?i jertfe de pace; apoi a ?ezut poporul de a mâncat ?i a baut ?i pe urma s-a sculat ?i a jucat.
Exo 32:7  Atunci a zis Domnul catre Moise: "Grabe?te de te pogoara de aici, caci poporul tau, pe care l-ai scos din ?ara Egiptului, s-a razvratit.
Exo 32:8  Curând s-au abatut de la calea pe care le-am poruncit-o, ?i-au facut un vi?el turnat ?i s-au închinat la el, aducându-i jertfe ?i zicând: "Iata, Israele, dumnezeul tau, care te-a scos din ?ara Egiptului!"
Exo 32:9  ?i a mai zis Domnul catre Moise: "Eu Ma uit la poporul acesta ?i vad ca este popor tare de cerbice;
Exo 32:10  Lasa-Ma dar acum sa se aprinda mânia Mea asupra lor, sa-i pierd ?i sa fac din tine un popor mare!"
Exo 32:11  Moise însa a rugat pe Domnul Dumnezeul sau ?i a zis: "Sa nu se aprinda, Doamne, mânia Ta asupra poporului Tau, pe care l-ai scos din ?ara Egiptului cu putere mare ?i cu bra?ul Tau cel înalt,
Exo 32:12  Ca nu cumva sa zica Egiptenii: I-a dus la pieire, ca sa-i ucida în mun?i ?i sa-i ?tearga de pe fala pamântului. Întoarce-?i iu?imea mâniei Tale, milostive?te-Te ?i nu cauta la rautatea poporului Tau.
Exo 32:13  Adu-?i aminte de Avraam, de Isaac ?i de Iacov, robii Tai, carora Te-ai jurat Tu pe Tine Însu?i, zicând: Voi înmul?i foarte tare neamul vostru, ca stelele cerului; ?i tot pamântul acesta, de care v-am vorbit, îl voi da urma?ilor vo?tri ?i-l vor stapâni în veci!"
Exo 32:14  Atunci a abatut Domnul pieirea ce zisese s-o aduca asupra poporului Sau.
Exo 32:15  Dupa aceea Moise, întorcându-se; s-a pogorât din munte, cu cele doua table ale legii în mâna, scrise pe amândoua par?ile lor - pe o parte ?i pe alta erau scrise.
Exo 32:16  Tablele acestea erau lucrul lui Dumnezeu ?i scrierea era scrierea lui Dumnezeu, sapata pe table.
Exo 32:17  Atunci, auzind Iosua glasul poporului rasunând, a zis catre Moise: "În tabara se aud strigate de razboi".
Exo 32:18  Iar Moise a zis: "Acesta nu este glas de biruitori, nici glas de birui?i; ci eu aud glas de oameni beli".
Exo 32:19  Iar dupa ce s-a apropiat de tabara, el a vazut vi?elul ?i jocurile ?i, aprinzându-se de mânie, a aruncat din mâinile sale cele doua table ?i le-a sfarâmat sub munte.
Exo 32:20  Apoi luând vi?elul, pe care-l facusera ei, l-a ars în foc, l-a facut pulbere ?i, presarându-l în apa, a dat-o sa o bea fiii lui Israel.
Exo 32:21  Dupa aceea a zis catre Aaron: "Ce ti-a facut poporul acesta, de l-ai vârât într-un pacat a?a de mare?"
Exo 32:22  Iar Aaron a raspuns lui Moise: "Sa nu se aprinda mânia domnului meu! Tu ?tii pe poporul acesta ca e razvratitor.
Exo 32:23  Caci ei mi-au zis: Fa-ne dumnezei, care sa mearga înaintea noastra, caci cu omul acesta, cu Moise, care ne-a scos din ?ara Egiptului, nu ?tim ce s-a întâmplat.
Exo 32:24  Atunci eu le-am zis: Cine are aur sa-l scoata. ?i ei l-au scos ?i mi l-au dat mie ?i eu l-am aruncat în foc ?i a ie?it acest vi?el".
Exo 32:25  Moise, vazând ca poporul acesta e neînfrânat, caci Aaron îngaduise sa ajunga neînfrânat ?i de râs înaintea du?manilor lui,
Exo 32:26  A stat la intrarea taberei ?i a zis: "Cine este pentru Domnul sa vina la mine!" ?i s-au adunat la el to?i fiii lui Levi.
Exo 32:27  Iar Moise le-a zis: "A?a zice Domnul Dumnezeul lui Israel: Sa-?i încinga fiecare din voi sabia sa la ?old ?i strabatând tabara de la o intrare pâna la cealalta, înainte ?i înapoi, sa ucida fiecare pe fratele sau, pe prietenul sau ?i pe aproapele sau".
Exo 32:28  ?i au facut fiii lui Levi dupa cuvântul lui Moise. În ziua aceea au cazut din popor ca la trei mii de oameni.
Exo 32:29  Caci Moise le zisese fiilor lui Levi: "Afierosi?i-va astazi mâinile voastre Domnului, fiecare prin fiul sau sau prin fratele sau, ca sa va trimita El astazi binecuvântare!"
Exo 32:30  Iar a doua zi a zis Moise catre popor: "A?i facut pacat mare; ma voi sui acum la Domnul sa vad nu cumva voi ?terge pacatul vostru".
Exo 32:31  ?i s-a întors Moise la Domnul ?i a zis: "O, Doamne, poporul acesta a savâr?it pacat mare, facându-?i dumnezeu de aur.
Exo 32:32  Rogu-ma acum, de vrei sa le ier?i pacatul acesta, iarta-i; iar de nu, ?terge-ma ?i pe mine din cartea Ta, în care m-ai scris!"
Exo 32:33  Zis-a Domnul catre Moise: "Pe acela care a gre?it înaintea Mea îl voi ?terge din cartea Mea.
Exo 32:34  Iar acum mergi ?i du poporul acesta la locul unde ?i-am zis. Iata îngerul Meu va merge înaintea ta ?i în ziua cercetarii Mele voi pedepsi pacatul lor".
Exo 32:35  Astfel a lovit Domnul poporul, pentru vi?elul ce î?i facuse, pe care-l turnase Aaron.
Exo 33:1  Apoi a zis Domnul catre Moise: "Du-te de aici tu ?i poporul tau, pe care l-ai scos din pamântul Egiptului, ?i sui?i-va în pamântul, pentru care M-am jurat lui Avraam, lui Isaac ?i lui Iacov, zicând: Urma?ilor vo?tri îl voi da.
Exo 33:2  Eu voi trimite înaintea ta pe îngerul Meu ?i va izgoni pe Canaanei, pe Amorei, pe Hetei, pe Ferezei, pe Gherghesei, pe Hevei ?i pe Iebusei,
Exo 33:3  ?i te voi duce în ?ara unde curge lapte ?i miere. Dar Eu nu voi merge în mijlocul vostru, ca sa nu va pierd pe cale, pentru ca sunte?i popor îndaratnic!
Exo 33:4  Auzind însa acest cuvânt grozav, poporul a plâns cu jale ?i nimeni n-a mai pus pe sine podoabele sale.
Exo 33:5  Caci Domnul zisese lui Moise: "Spune fiilor lui Israel: Voi sunte?i popor îndaratnic. De voi merge Eu în mijlocul vostru, într-o clipeala va voi pierde. Dezbraca?i acum de pe voi hainele voastre cele frumoase ?i podoabele voastre ?i voi vedea ce voi face cu voi".
Exo 33:6  Atunci fiii lui Israel au dezbracat de pe ei podoabele lor ?i hainele cele frumoase când au plecat de la Muntele Horeb.
Exo 33:7  Iar Moise, luându-?i cortul, l-a întins afara din tabara, departe de ea, ?i-l numi cortul adunarii; ?i tot cel ce cauta pe Domnul venea la cortul adunarii, care se afla afara din tabara.
Exo 33:8  ?i când se îndrepta Moise spre cort, tot poporul se scula ?i sta fiecare la u?a cortului sau ?i se uita dupa Moise, pâna ce intra el în cort.
Exo 33:9  Iar dupa ce intra Moise în cort, se pogora un stâlp de nor ?i se oprea la intrarea cortului ?i Domnul graia cu Moise.
Exo 33:10  ?i vedea tot poporul stâlpul cel de nor, care statea la u?a cortului, ?i se scula tot poporul ?i se închina fiecare din u?a cortului sau.
Exo 33:11  Domnul însa graia cu Moise fa?a catre fa?a, cum ar grai cineva cu prietenul sau. Dupa aceea Moise se întorcea în tabara; iar tânarul sau slujitor Iosua, fiul lui Navi, nu parasea cortul.
Exo 33:12  Atunci a zis Moise catre Domnul: "Iata, Tu îmi spui: Du pe poporul acesta, dar nu mi-ai descoperit pe cine ai sa trimi?i cu mine, de?i mi-ai spus: Te cunosc pe nume ?i ai aflat bunavoin?a înaintea ochilor Mei.
Exo 33:13  Deci, de am aflat bunavoin?a în ochii Tai, arata-Te sa Te vad, ca sa cunosc ?i sa aflu bunavoin?a în ochii Tai ?i ca acest neam e poporul Tau".
Exo 33:14  ?i a zis Domnul catre el: "Eu Însumi voi merge înaintea Ta ?i Te voi duce la odihna!"
Exo 33:15  Zis-a Moise catre Domnul: "Daca nu mergi Tu Însu?i cu noi, atunci sa nu ne sco?i de aici;
Exo 33:16  Caci prin ce se va cunoa?te cu adevarat ca eu ?i poporul Tau am aflat bunavoin?a înaintea Ta? Au nu prin aceea ca Tu sa fii înso?itorul nostru? Atunci eu ?i poporul Tau vom fi cei mai slavi?i dintre toate popoarele de pe pamânt".
Exo 33:17  ?i a zis Domnul catre Moise: "Voi face ?i ceea ce zici tu, pentru ca tu ai aflat bunavoin?a înaintea Mea ?i te cunosc pe tine mai mult decât pe to?i".
Exo 33:18  ?i Moise a zis: "Arata-mi slava Ta! "
Exo 33:19  Zis-a Domnul catre Moise: "Eu voi trece pe dinaintea ta toata slava Mea, voi rosti numele lui Iahve înaintea ta ?i pe cel ce va fi de miluit îl voi milui ?i cine va fi vrednic de îndurare, de acela Ma voi îndura".
Exo 33:20  Apoi a adaugat: "Fa?a Mea însa nu vei putea s-o vezi, ca nu poate vedea omul fa?a Mea ?i sa traiasca".
Exo 33:21  ?i iara?i a zis Domnul: "Iata aici la Mine un loc: ?ezi pe stânca aceasta;
Exo 33:22  Când va trece slava Mea, te voi ascunde în scobitura stâncii ?i voi pune mâna Mea peste tine pâna voi trece;
Exo 33:23  Iar când voi ridica mâna Mea, tu vei vedea spatele Meu, iar fa?a Mea nu o vei vedea!"
Exo 34:1  Zis-a Domnul catre Moise: "Ciople?te doua table de piatra, ca cele dintâi, ?i suie-te la Mine în munte ?i voi scrie pe aceste table cuvintele care au fost scrise pe tablele cele dintâi, pe care le-ai sfarâmat.
Exo 34:2  Sa fii gata dis-de-diminea?a ?i diminea?a sa te sui în Muntele Sinai ?i sa stai înaintea Mea acolo pe vârful muntelui.
Exo 34:3  Dar nimeni sa nu se suie cu tine, nici sa se arate în tot muntele: nici oi, nici vite mari sa nu pasca împrejurul acestui munte".
Exo 34:4  Deci a cioplit Moise doua table de piatra, asemenea cu cele dintâi, ?i, sculându-se dis-de-diminea?a, a luat Moise în mâini cele doua table de piatra ?i s-a suit în Muntele Sinai, cum îi poruncise Domnul.
Exo 34:5  Atunci S-a pogorât Domnul în nor, a stat acolo ?i a rostit numele lui Iahve.
Exo 34:6  ?i Domnul, trecând pe dinaintea lui, a zis: "Iahve, Iahve, Dumnezeu, iubitor da oameni, milostiv, îndelung-rabdator, plin de îndurare ?i de dreptate,
Exo 34:7  Care paze?te adevarul ?i arata mila la mii de neamuri; Care iarta vina ?i razvratirea ?i pacatul, dar nu lasa nepedepsit pe cel ce pacatuie?te; Care pentru pacatele parin?ilor pedepse?te pe copii ?i pe copiii copiilor pâna la al treilea ?i al patrulea neam!"
Exo 34:8  Atunci a cazut Moise îndata la pamânt ?i s-a închinat lui Dumnezeu,
Exo 34:9  Zicând: "De am aflat bunavoin?a în ochii Tai, Stapâne, sa mearga Stapânul în mijlocul nostru, caci poporul acesta e îndaratnic; dar, iarta nelegiuirile noastre ?i pacatele ?i ne fa mo?tenirea Ta!"
Exo 34:10  Domnul însa a zis catre Moise: "Iata, Eu închei legamânt înaintea a tot poporul tau: Voi face lucruri slavite, cum n-au fost în tot pamântul ?i la toate popoarele; ?i tot poporul în mijlocul caruia te vei afla tu, va vedea lucrurile Domnului, caci înfrico?ator va fi ceea ce voi face pentru tine.
Exo 34:11  Pastreaza ceea ce î?i poruncesc Eu acum: Iata Eu voi izgoni de la fa?a ta pe Amorei, pe Canaanei, pe Hetei, pe Ferezei, pe Hevei, pe Gherghesei ?i pe Iebusei.
Exo 34:12  Fere?te-te sa intri în legatura cu locuitorii ?arii aceleia, în care ai sa intri, ca sa nu fie ei o cursa între voi.
Exo 34:13  Jertfelnicele lor sa le strica?i, stâlpii lor sa-i sfarâma?i; sa taia?i dumbravile lor cele sfin?ite ?i dumnezeii lor cei ciopli?i sa-i arde?i în foc,
Exo 34:14  Caci tu nu trebuie sa te închini la alt dumnezeu, fara numai Domnului Dumnezeu, pentru ca numele Lui este "Zelosul"; Dumnezeu este zelos.
Exo 34:15  Nu cumva sa intri în legatura cu locuitorii ?arii aceleia, pentru ca ei, urmând dupa dumnezeii lor ?i aducând jertfe dumnezeilor lor, te vor pofti ?i pe tine sa gu?ti din jertfa lor.
Exo 34:16  ?i vei lua fetele lor so?ii pentru fiii tai ?i fetele tale le vei marita dupa feciorii lor; ?i vor merge fetele tale dupa dumnezeii lor ?i fiii tai vor merge dupa dumnezeii lor.
Exo 34:17  Sa nu-?i faci dumnezei turna?i.
Exo 34:18  Sarbatoarea azimelor sa o paze?ti: ?apte zile, cum ?i-am poruncit Eu, sa manânci azime, la vremea rânduita în luna Aviv, caci în luna Aviv ai ie?it tu din Egipt.
Exo 34:19  Tot întâiul nascut de parte barbateasca este al Meu; asemenea ?i tot întâiul nascut al vacii ?i tot întâiul nascut al oii.
Exo 34:20  Iar întâiul nascut al asinei sa-l rascumperi cu un miel, iar de nu-l vei rascumpara, sa-i frângi gâtul. To?i întâii nascu?i din fiii tai sa-i rascumperi ?i nimeni sa nu se înfa?i?eze înaintea Mea cu mâna goala.
Exo 34:21  ?ase zile lucreaza, iar în ziua a ?aptea sa te odihne?ti; chiar în vremea semanatului ?i a seceri?ului sa te odihne?ti.
Exo 34:22  Sa ?ii ?i sarbatoarea saptamânilor, sarbatoarea pârgei, la seceri?ul grâului, ?i sarbatoarea strângerii roadelor, la sfâr?itul toamnei.
Exo 34:23  De trei ori pe an sa se înfa?i?eze înaintea Domnului Dumnezeului lui Israel to?i cei de parte barbateasca ai tai,
Exo 34:24  Caci când voi goni popoarele de la fa?a ta ?i voi largi hotarele tale, nimeni nu va pofti ogorul tau, de te vei sui sa te înfa?i?ezi înaintea Domnului Dumnezeului tau de trei ori pe an.
Exo 34:25  Sa nu torni sângele jertfei Mele pe pâine dospita ?i jertfa de la sarbatoarea Pa?tilor sa nu ramâna pâna a doua zi.
Exo 34:26  Cele dintâi roade ale ?arinii tale sa le aduci în casa Domnului Dumnezeului tau. Sa nu fierbi iedul în laptele mamei sale".
Exo 34:27  ?i a mai zis Domnul catre Moise: "Scrie-?i cuvintele acestea, caci pe cuvintele acestea închei Eu legamânt cu tine ?i cu Israel!"
Exo 34:28  Moise a stat acolo la Domnul patruzeci de zile ?i patruzeci de nop?i; ?i nici pâine n-a mâncat, nici apa n-a baut. ?i a scris Moise pe table cuvintele legamântului: cele zece porunci.
Exo 34:29  Iar când se pogora Moise din Muntele Sinai, având în mâini cele doua table ale legii, el nu ?tia ca fa?a sa stralucea, pentru ca graise Dumnezeu cu el.
Exo 34:30  Deci Aaron ?i to?i fiii lui Israel, vazând pe Moise ca are fa?a stralucitoare, s-au temut sa se apropie de el.
Exo 34:31  Atunci i-a chemat Moise ?i au venit la el Aaron ?i toate capeteniile ob?tei ?i Moise a grait cu ei.
Exo 34:32  Dupa aceasta s-au apropiat de el to?i fiii lui Israel ?i el le-a poruncit tot ce-i graise Domnul în Muntele Sinai.
Exo 34:33  Iar dupa ce a încetat de a grai cu ei, Moise ?i-a acoperit fa?a cu un val.
Exo 34:34  Când însa intra el înaintea Domnului, ca sa vorbeasca cu El, atunci î?i ridica valul pâna când ie?ea; iar la ie?ire spunea fiilor lui Israel cele ce i se poruncisera de catre Domnul.
Exo 34:35  ?i vedeau fiii lui Israel ca fa?a lui Moise stralucea ?i Moise î?i punea iar valul peste fa?a sa, pâna când intra din nou sa vorbeasca cu Domnul.
Exo 35:1  Atunci a adunat Moise toata ob?tea fiilor lui Israel ?i le-a zis: "Iata ce a poruncit Domnul sa face?i:
Exo 35:2  ?ase zile sa lucra?i, iar ziua a ?aptea sa fie sfânta pentru voi, zi de odihna, odihna Domnului; tot cel ce va lucra în ziua aceea va fi omorât.
Exo 35:3  În ziua odihnei sa nu face?i foc în toate loca?urile voastre. Eu sunt Domnul!"
Exo 35:4  Apoi a grait Moise la toata ob?tea fiilor lui Israel ?i a zis: "Iata ce a mai poruncit Domnul sa va spun:
Exo 35:5  Aduce?i din ale voastre, daruri Domnului; fiecare sa aduca Domnului daruri cât îl lasa inima: aur, argint ?i arama;
Exo 35:6  Matase violeta, stacojie ?i vi?inie, vison rasucit ?i par de capra;
Exo 35:7  Piei de berbec vopsite ro?u, piei de vi?el de mare ?i lemn de salcâm;
Exo 35:8  Untdelemn pentru sfe?nic ?i aromate pentru mirul de uns ?i pentru facut miresme de tamâie,
Exo 35:9  Piatra de sardiu, pietre pentru prins la efod ?i la ho?en.
Exo 35:10  Tot cel cu minte în?eleapta dintre voi sa vina ?i sa faca toate câte a poruncit Domnul:
Exo 35:11  Cortul ?i acoperamintele lui, acoperi?ul lui cel de deasupra, verigile ?i scândurile lui, pârghiile, stâlpii ?i postamentele lor;
Exo 35:12  Chivotul legii ?i pârghiile lui, capacul lui ?i perdeaua despar?itoare, pietre de smarald, tamâie ?i mir de ungere;
Exo 35:13  Masa cu pârghiile ?i toate uneltele ei ?i pâinile pentru punerea înainte,
Exo 35:14  Sfe?nicul pentru luminat cu toate obiectele lui, candelele lui ?i untdelemnul de ars;
Exo 35:15  Jertfelnicul tamâierii ?i pârghiile lui ?i miresme pentru tamâiere,
Exo 35:16  Jertfelnicul pentru arderile de tot, împletitura de sârma pentru el, pârghiile lui ?i toate cele de trebuin?a pentru el; baia ?i postamentul ei,
Exo 35:17  Perdelele cur?ii, stâlpul ei cu postamentele lor ?i perdeaua de la intrarea în curte,
Exo 35:18  ?aru?ii cortului; ?aru?ii cur?ii ?i frânghiile lor,
Exo 35:19  Ve?mintele sfinte pentru facut slujba în loca?ul sfânt; ve?minte sfinte pentru Aaron preotul ?i ve?minte pentru fiii lui, pentru slujba preo?iei".
Exo 35:20  Dupa aceea, plecând toata ob?tea fiilor lui Israel de la Moise,
Exo 35:21  A adus fiecare cât l-a lasat inima sa ?i cât l-a îndemnat cugetul sa aduca dar Domnului, pentru facerea cortului adunarii ?i a tuturor lucrurilor lui ?i pentru toate ve?mintele sfinte.
Exo 35:22  ?i veneau barba?ii cu femeile ?i fiecare, dupa cum îl lasa inima, aducea verigi, cercei, inele, bra?ari ?i tot felul de lucruri de aur; cum voia fiecare sa aduca Domnului daruri de aur.
Exo 35:23  Fiecare din cei ce aveau matase violeta, stacojie ?i vi?inie, în ?i par de capra, piei de berbec vopsite ro?u ?i vânat, le aducea.
Exo 35:24  Fiecare din cei ce puteau sa aduca în dar argint sau arama, aducea din acestea Domnului; ?i fiecare din cei ce aveau lemn de salcâm, aducea pentru toate cele de trebuin?a;
Exo 35:25  Toate femeile cu minte iscusita torceau cu mâinile lor ?i aduceau tort, matase violeta, stacojie ?i vi?inie ?i în.
Exo 35:26  Toate femeile, pe care le tragea inima ?i ?tiau sa toarca, torceau par de capra;
Exo 35:27  Iar capeteniile aduceau pietre de smarald ?i pietre scumpe de pus la efod ?i la ho?en,
Exo 35:28  Precum ?i miresme, untdelemn pentru candelabru, mir de ungere ?i miresme de tamâiere.
Exo 35:29  Deci tot barbatul ?i femeia din fiii lui Israel, pe care i-a tras inima sa aduca pentru toate lucrurile ce poruncise Domnul prin Moise sa se faca, au adus dar de buna voie Domnului.
Exo 35:30  Apoi a zis Moise catre fiii lui Israel: "Iata Domnul a chemat anume pe Be?aleel, fiul lui Uri al lui Or, din semin?ia lui Iuda,
Exo 35:31  ?i l-a umplut de duhul dumnezeiesc al în?elepciunii, al priceperii, al ?tiin?ei ?i a toata iscusin?a,
Exo 35:32  Ca sa lucreze ?esaturi iscusite, sa faca lucruri de aur, de argint ?i de arama;
Exo 35:33  Sa ciopleasca pietrele scumpe pentru încrustat, sa sape în lemn ?i sa faca tot felul de lucruri iscusite.
Exo 35:34  ?i priceperea de a înva?a pe al?ii a pus-o în inima lui, în a lui ?i a lui Oholiab, fiul lui Ahisamac, din semin?ia lui Dan.
Exo 35:35  A umplut inima acestora de în?elepciune, ca sa faca pentru loca?ul sfânt orice lucru de sapator ?i de ?esator iscusit, de cusator pe pânza de matase violeta, stacojie ?i vi?inie ?i de in, ?i de ?esator în stare de a face orice lucru ?i a nascoci ?esaturi iscusite".
Exo 36:1  ?i Be?aleel, Oholiab ?i to?i cei cu minte iscusita, carora Domnul le daduse în?elepciune ?i pricepere, ca sa ?tie sa faca tot felul de lucruri trebuitoare la loca?ul cel sfânt, vor trebui sa faca dupa cum poruncise Domnul.
Exo 36:2  Iar Moise a chemat pe Be?aleel, pe Oholiab ?i pe to?i cei cu minte iscusita, carora le daduse Domnul iscusin?a ?i pe to?i cei ce-i tragea inima sa vina la lucru de buna voie, ca sa ajute la acestea.
Exo 36:3  ?i au luat ei de la Moise toate prinoasele, pe care le adusesera fiii lui Israel pentru toate cele trebuincioase loca?ului sfânt, ca sa le lucreze. Atunci tot se mai aduceau înca la el daruri de buna voie în fiecare diminea?a.
Exo 36:4  Deci to?i cei cu minte iscusita, care împlineau tot felul de lucrari la loca?ul sfânt, au venit fiecare, de la lucru cu care se îndeletnicea,
Exo 36:5  ?i ei au spus lui Moise, zicând: "Poporul aduce mult mai mult decât trebuie pentru lucrurile ce a poruncit Domnul sa se faca".
Exo 36:6  Atunci a poruncit Moise ?i s-a strigat în tabara, ca nici barbat, nici femeie sa nu mai faca nimic pentru daruit la loca?ul sfânt. ?i a încetat poporul de a mai aduce.
Exo 36:7  Caci material adunat era destul pentru toate lucrurile ce trebuiau facute, ba mai ?i prisosea.
Exo 36:8  Atunci to?i cei cu minte iscusita, care se îndeletniceau cu facerea loca?ului sfânt, au facut pentru cort zece covoare de in rasucit ?i de matase violeta, stacojie ?i vi?inie; ?i în ?esatura lor au facut chipuri de heruvimi, alese cu iscusin?a.
Exo 36:9  Lungimea unui covor era de douazeci ?i opt de co?i ?i la?imea unui covor era de patru co?i. Toate covoarele aveau aceea?i masura.
Exo 36:10  Cinci covoare au fost prinse unul de altul ?i celelalte cinci iar au fost prinse unul de altul; ?i a?a s-au facut doua jumata?i de acoperi?.
Exo 36:11  Apoi au facut cheotori de matase violeta pe marginea covorului din marginea jumata?ii întâi de acoperi?, unde trebuia unita cu jumatatea a doua; de asemenea au facut ?i pe marginea jumata?ii a doua, unde aceasta trebuia unita cu cea dintâi;
Exo 36:12  Cincizeci de cheotori au facut la o jumatate de acoperamânt ?i cincizeci de cheotori au facut la cealalta ?i cheotorile acestea erau unele în dreptul altora.
Exo 36:13  Dupa aceea au facut cincizeci de copci de aur ?i cu copcile acestea au unit cele doua jumata?i de acoperi? una cu alta ?i s-a facut un acoperi? întreg al cortului.
Exo 36:14  Apoi au facut covoare de par de capra pentru acoperit cortul peste cele de mai sus. Unsprezece covoare de acestea au facut.
Exo 36:15  Lungimea unui covor era de treizeci de coti, iar la?imea de patru coti; ?i cele unsprezece covoare aveau toate aceea?i masura.
Exo 36:16  ?i au unit cinci covoare la un loc ?i pe celelalte ?ase covoare iar le-au unit la un loc.
Exo 36:17  Apoi au facut cincizeci de cheotori pe marginea covorului celui din marginea unei jumata?i, unde aceasta trebuia sa se uneasca cu cealalta jumatate, iar cincizeci de cheotori le-au facut pe marginea covorului din marginea celeilalte jumata?i, care trebuia sa se uneasca cu cea dintâi.
Exo 36:18  ?i au facut cincizeci de copci de arama ca sa uneasca covoarele spre a se face un singur acoperi? de cort.
Exo 36:19  Apoi au mai facut pentru cort un acoperi? de piei de berbec vopsite în ro?u ?i un acoperi?, pe deasupra, de piei vinete.
Exo 36:20  Dupa aceea au facut pentru cort scânduri din lemn de salcâm de pus în picioare.
Exo 36:21  Fiecare scândura era lunga de zece co?i ?i lata de un cot ?i jumatate.
Exo 36:22  Fiecare scândura avea doua cepuri, a?ezate unul în dreptul celuilalt.
Exo 36:23  A?a au facut toate scândurile cortului ?i anume: douazeci de scânduri pentru latura de miazazi;
Exo 36:24  ?i sub aceste douazeci de scânduri au facut patruzeci de postamente de argint, câte doua postamente la fiecare scândura, pentru cele doua cepuri ale ei.
Exo 36:25  Pentru latura a doua, dinspre miazanoapte, au facut alte douazeci de scânduri
Exo 36:26  ?i patruzeci de postamente de argint, câte doua postamente de fiecare scândura, pentru cele doua cepuri ale ei;
Exo 36:27  Iar pentru partea dindarat a cortului, dinspre asfin?it, au facut ?ase scânduri.
Exo 36:28  Au mai facut doua scânduri pentru unghiurile de la fundul cortului.
Exo 36:29  Acestea erau prinse jos ?i sus prin câte un inel.
Exo 36:30  ?i a?a, cu cele doua scânduri de la cele doua col?uri erau la partea dindarat a cortului opt scânduri, iar postamente de argint ?aisprezece, câte doua postamente sub fiecare scândura.
Exo 36:31  Apoi au facut cinci pârghii din lemn de salcâm pentru scândurile de pe o latura a cortului,
Exo 36:32  ?i cinci pârghii pentru scândurile de pe cealalta latura;
Exo 36:33  Iar pârghia din mijloc au facut-o a?a, ca sa treaca prin scânduri, de la un capat la celalalt al peretelui.
Exo 36:34  Scândurile le-au îmbracat cu aur; inelele, prin care se vârau pârghiile, le-au facut de aur ?i tot cu aur au îmbracat ?i pârghiile.
Exo 36:35  Dupa aceea au facut o perdea de matase violeta, stacojie ?i vi?inie ?i de in rasucit ?i pe ea au facut chipuri de heruvimi cu iscusin?a alese, ca sa o puna între sfânta ?i sfânta sfintelor.
Exo 36:36  Pentru ea au facut patru stâlpi, din lemn de salcâm, i-au îmbracat cu aur, le-au facut cârlige de aur ?i sub ei au turnat patru postamente de argint.
Exo 36:37  Apoi la u?a cortului au facut o perdea de matase violeta, stacojie ?i vi?inie ?i de in rasucit, cu alesaturi.
Exo 36:38  Pentru ea au facut cinci stâlpi cu cârligele lor ?i i-au îmbracat cu aur, turnând pentru ei cinci postamente de arama.
Exo 37:1  Dupa aceea Be?aleel a facut chivotul din lemn de salcâm, lung de doi co?i ?i jumatate, larg de un cot ?i jumatate ?i înalt tot de un cot ?i jumatate;
Exo 37:2  L-a îmbracat cu aur curat pe dinauntru ?i pe din afara, iar împrejur i-a facut o cununa de aur.
Exo 37:3  A turnat pentru el patru inele de aur, pentru cele patru col?uri de jos ale lui: doua inele pe o latura ?i doua inele pe cealalta latura.
Exo 37:4  A facut doua pârghii de lemn de salcâm, le-a îmbracat cu aur,
Exo 37:5  ?i le-a vârât prin inelele de pe laturile chivotului, ca sa poarte chivotul cu ele.
Exo 37:6  A facut apoi capacul chivotului de aur curat: lung de doi co?i ?i jumatate ?i lat de un cot ?i jumatate.
Exo 37:7  A facut de asemenea doi heruvimi de aur, lucra?i din ciocan, pentru cele doua capete ale capacului,
Exo 37:8  ?i i-a a?ezat unul la un capat ?i altul la celalalt capat al capacului. Heruvimii ace?tia erau facu?i ca ie?ind din capac la cele doua capete ale lui.
Exo 37:9  Cei doi heruvimi î?i întindeau aripile unul spre altul, umbrind capacul, iar fe?ele lor erau îndreptate una catre alta, privind spre capac.
Exo 37:10  A facut apoi masa din lemn de salcâm, lunga de doi co?i, lata de un cot ?i înalta de un cot ?i jumatate.
Exo 37:11  A îmbracat-o cu aur curat ?i împrejur i-a facut o cununa de aur.
Exo 37:12  I-a facut de asemenea un pervaz împrejur înalt de o palma, iar împrejurul pervazului a facut cununa împletita de aur.
Exo 37:13  A turnat pentru ea patru inele de aur ?i a prins aceste inele în cele patru col?uri de la cele patru picioare ale ei.
Exo 37:14  Inelele erau prinse de pervaz ?i prin ele se petreceau doua pârghii pentru purtat masa.
Exo 37:15  Pârghiile de purtat masa le-a facut din lemn de salcâm ?i le-a îmbracat cu aur.
Exo 37:16  A facut vase trebuitoare pentru masa: talere, cadelni?e, linguri ?i cupe pentru turnat, toate de aur curat.
Exo 37:17  Dupa aceea a facut un sfe?nic de aur curat ?i sfe?nicul acesta l-a lucrat din ciocan. Fusul lui, bra?ele lui, cupele lui, nodurile lui ?i florile lui erau toate dintr-o bucata.
Exo 37:18  Din laturile lui ie?eau ?ase bra?e: trei bra?e ale sfe?nicului ie?eau dintr-o latura a lui ?i trei bra?e ale sfe?nicului ie?eau din cealalta latura a lui;
Exo 37:19  Un bra? avea trei cupe în forma florii de migdal, cu nodurile ?i florile lor; alt bra? avea trei cupe tot în forma florii de migdal, cu nodurile ?i florile lor; a?a aveau toate cele ?ase bra?e, care ie?eau din laturile sfe?nicului.
Exo 37:20  Iar pe fusul sfe?nicului erau patru cupe în forma florii de migdal, cu nodurile ?i florile lor.
Exo 37:21  Cele ?ase bra?e, care ie?eau din el, aveau: un nod sub primele doua bra?e, un nod sub alte doua bra?e ?i un nod sub ultimele doua bra?e.
Exo 37:22  Nodurile ?i ramurile de pe ele erau una cu fusul. Sfe?nicul întreg era lucrat din ciocan, dintr-o singura bucata de aur curat.
Exo 37:23  Apoi i-a facut ?apte candele, mucari ?i tavi?e de aur curat.
Exo 37:24  Dintr-un talant de aur curat au facut sfe?nicul cu toate cele necesare lui.
Exo 37:25  A facut apoi jertfelnicul tamâierii, din lemn de salcâm, lung de un cot, lat de un cot, adica patrat ?i înalt de doi co?i; coarnele lui erau din el;
Exo 37:26  ?i l-a îmbracat cu aur curat pe deasupra, pe laturile lui de jur împrejur ?i pe coarnele lui, iar împrejur i-a facut cununa de aur.
Exo 37:27  Sub pervazul lui, la doua din col?urile lui, a prins doua inele de aur, ?i le-a prins de o parte ?i de alta a lui, ca sa se petreaca prin ele pârghiile de purtat.
Exo 37:28  Pârghiile le-au facut din lemn de salcâm ?i le-au îmbracat cu aur.
Exo 37:29  A facut de asemenea mir pentru sfânta ungere ?i tamâie mirositoare, curate ?i cu iscusin?a alcatuite de pregatitorii de aromate.
Exo 38:1  Dupa aceea a facut jertfelnicul pentru arderile de tot, din lemn de salcâm, lung de cinci co?i, lat de cinci co?i, adica cu fa?a patrata ?i înalt de trei co?i.
Exo 38:2  I-a facut patru coarne, ce ie?eau din el, în cele patru col?uri ale lui, ?i l-a îmbracat cu arama.
Exo 38:3  A facut apoi toate lucrurile trebuitoare jertfelnicului: oale, lopa?ele, cupe, furculi?e ?i vase pentru carbuni; toate obiectele acestea le-a facut din arama.
Exo 38:4  A mai facut pentru jertfelnic o cama?a, un fel de împletitura de arama, care îmbraca partea lui de jos pâna la jumatatea lui.
Exo 38:5  A turnat apoi patru verigi de arama pentru cele patru col?uri ale împletiturii celei de arama, pentru petrecut prin ele pârghiile de purtat;
Exo 38:6  Iar pârghiile le-au facut din lemn de salcâm ?i le-a îmbracat cu arama.
Exo 38:7  Pârghiile se petreceau prin inelele din laturile jertfelnicului, ca sa poata fi purtat cu ajutorul lor. Jertfelnicul l-a facut din scânduri, gol înauntru.
Exo 38:8  A facut apoi baia de arama ?i postamentul ei tot de arama, cu chipuri iscusit lucrate care împodobeau intrarea cortului adunarii.
Exo 38:9  Dupa aceea a facut curtea. Spre miazazi curtea avea perdele de in rasucit, lungi de o suta de co?i,
Exo 38:10  ?i la ele douazeci de stâlpi cu douazeci de postamente de arama sub ei ?i cu cârligele ?i legatorile lor de argint;
Exo 38:11  Pe latura de miazanoapte a facut perdele lungi de o suta de co?i ?i la ele douazeci de stâlpi, cu douazeci de postamente de arama sub ei, cu cârligele ?i legatorile lor de argint.
Exo 38:12  În partea dinspre asfin?it a facut perdele lungi de cincizeci de co?i ?i pentru ele zece stâlpi, cu zece postamente de arama ?i cu cârligele ?i legaturile lor de argint;
Exo 38:13  Iar în partea de dinainte, dinspre rasarit, a facut perdele lungi de cincizeci de co?i,
Exo 38:14  ?i anume: de o parte a por?ii cur?ii cincisprezece co?i de perdele ?i la ele trei stâlpi cu trei postamente de arama;
Exo 38:15  De cealalta parte a por?ii cur?ii cincisprezece co?i de perdele ?i la ele trei stâlpi cu trei postamente de arama.
Exo 38:16  Toate perdelele pe toate laturile cur?ii erau de in rasucit.
Exo 38:17  Iar stâlpii aveau postamentele de arama, cârligele ?i legatorile lor de argint, vârfurile îmbracate în argint ?i to?i erau uni?i între ei prin legatori de argint.
Exo 38:18  Iar perdeaua pentru poarta cur?ii a facut-o din lâna violeta, stacojie ?i vi?inie ?i din in rasucit, cu alesaturi, lunga de douazeci de co?i ?i înalta de cinci co?i, pe toata întinderea, ca ?i perdelele cur?ii.
Exo 38:19  Pentru ea a facut patru stâlpi cu patru postamente de arama sub ei, cu cârligele ?i legatorile de argint ? cu vârfurile îmbracate în argint.
Exo 38:20  To?i ?aru?ii împrejurul cortului ?i cur?ii erau de arama.
Exo 38:21  Iata acum ?i socoteala lucrurilor ce s-au întrebuin?at la cortul adunarii, care s-a facut dupa porunca lui Moise, prin levi?i, sub supravegherea lui Itamar, fiul preotului Aaron.
Exo 38:22  Toate însa, câte a poruncit Domnul lui Moise, s-au lucrat de Be?aleel, fiul lui Uri al lui Or, din semin?ia lui Iuda,
Exo 38:23  Ajutat de Oholiab, fiul lui Ahisamac, din semin?ia lui Dan, sapator în piatra ?i ?esator iscusit ?i cusator pe pânze de in ?i de matase violeta, stacojie ?i vi?inie.
Exo 38:24  Tot aurul întrebuin?at la cort ?i la toate lucrurile lui a fost douazeci ?i noua de talan?i ?i ?apte sute treizeci sicli de aur, ce s-au adus în dar, socotit dupa siclul sfânt.
Exo 38:25  Iar argintul, ce s-a adus dar de la cei numara?i ai ob?tei, a fost o suta de talan?i ?i o mie ?apte sute ?aptezeci ?i cinci de sicli, socotit dupa siclul sfânt.
Exo 38:26  Argintul acesta s-a luat de la ?ase sute trei mii cinci sute cincizeci de oameni, în vârsta de la douazeci de ani în sus, trecu?i prin numaratoare, câte o jumatate de siclu de cap, socotit dupa siclul sfânt.
Exo 38:27  O suta de talan?i de argint s-au întrebuin?at la turnarea postamentelor scândurilor cortului ?i a postamentelor stâlpilor perdelelor lui: o suta de postamente din o suta de talan?i, câte un talant la postament;
Exo 38:28  Iar din o mie ?apte sute ?aptezeci ?i cinci de sicli au facut cârligele de la stâlpii cur?ii, au îmbracat vârfurile lor ?i au facut pentru ei legatori.
Exo 38:29  Arama, adusa în dar, a fost: trei sute ?aptezeci de talan?i ?i doua mii patru sute de sicli.
Exo 38:30  Din ea au facut postamente pentru stâlpii de la intrarea cortului adunarii, jertfelnicul cel de arama, cama?a de arama a lui ?i toate uneltele jertfelnicului;
Exo 38:31  Postamentele pentru to?i stâlpii cur?ii, postamentele pentru stâlpii de la intrarea cur?ii, to?i ?aru?ii cortului ?i to?i ?aru?ii dimprejurul cur?ii.
Exo 39:1  Iar din matase violeta, stacojie ?i vi?inie au facut ve?minte de slujba, pentru slujit în loca?ul sfânt, ?i au mai facut ve?minte sfinte pentru Aaron, cum poruncise Domnul  lui Moise.
Exo 39:2  Au facut efodul din fire de aur, din matase violeta, stacojie ?i vi?inie ?i din in rasucit.
Exo 39:3  ?i anume: au desfacut aurul în foi ?i au taiat fire, pe care le-au ?esut cu iscusin?a  printre firele de matase violeta, stacojie ?i vi?inie ?i de in rasucit, lucru  iscusit.
Exo 39:4  I-au facut încheietori de încheiat pe umeri ?i au unit amândoua par?ile lui.
Exo 39:5  Brâul efodului, care vine peste el, la fel cu el, l-au facut din fire de aur, din matase violeta, stacojie ?i vi?inie ?i din in rasucit, cum poruncise Domnul lui Moise.
Exo 39:6  Au lucrat apoi doua   pietre de smarald, a?ezându-le în cuibule?e de aur ?i sapând pe ele numele fiilor lui Israel, cum se sapa pe pecete,
Exo 39:7  ?i le-au pus la încheieturile efodului, pe umeri, întru pomenirea fiilor lui Israel, cum poruncise Domnul lui Moise.
Exo 39:8  Au facut apoi ho?enul, lucrare iscusita, la fel cu efodul, din fire de aur ?i din matase violeta, stacojie ?i vi?inie ?i din in rasucit.
Exo 39:9  Ho?enul l-au facut dublu, în patru col?uri, lung de o palma ?i lat de o palma.
Exo 39:10  ?i au pus pe el pietre scumpe, a?ezate în patru rânduri: într-un rând un sardeon, un topaz ?i un smarald - rândul întâi;
Exo 39:11  În rândul al doilea: un rubin, un safir ?i un diamant;
Exo 39:12  În rândul al treilea: un opal, o agata ?i un ametist;
Exo 39:13  ?i în rândul al patrulea: un hrisolit, un onix ?i un iaspis. Ele erau a?ezate în cuibule?e de aur.
Exo 39:14  Pietrele acestea erau în numar de douasprezece, dupa numarul fiilor lui Israel, ?i pe fiecare din ele era sapat, ca pe pecete, câte un nume, din cele ale celor douasprezece semin?ii.
Exo 39:15  La ho?en au facut apoi lan?i?oare groase de aur curat ?i lucrate rasucit, ca sfoara;
Exo 39:16  Au mai facut doua rozete ?i doua verigi de aur ?i au prins cele doua verigi de cele doua col?uri de sus ale ho?enului;
Exo 39:17  ?i au aga?at doua capete ale lan?i?oarelor de cele doua verigi din colturile ho?enului,
Exo 39:18  Iar celelalte doua capete ale celor doua lan?i?oare le-au aga?at de cele doua rozete ?i le-au prins pe acestea de încheieturile efodului, pe fa?a acestuia.
Exo 39:19  Dupa aceea au mai facut înca doua verigi de aur ?i le-au prins de celelalte doua col?uri ale ho?enului pe cealalta parte dinspre efod;
Exo 39:20  ?i au mai facut ?i alte doua verigi de aur ?i le-au prins de cele doua încheieturi ale efodului, dedesubt, pe fa?a lui, unde se unesc, mai sus de încingatoarea efodului.
Exo 39:21  ?i au legat ho?enul cu verigile lui de verigile efodului cu un ?nur de matase violeta, ca sa stea deasupra încingatorii efodului ?i ca sa nu cada ho?enul de pe efod, cum poruncise Domnul lui Moise.
Exo 39:22  Iar meilul care vine sub efod, l-au facut din purpura ?esuta violet.
Exo 39:23  Acesta avea în partea de sus o deschizatura ?i împrejurul acestei deschizaturi avea un guler, ?esut ca o plato?a, ca sa nu se rupa.
Exo 39:24  Meilului i-au facut pe la poale ciucuri de matase violeta, stacojie ?i vi?inie ?i de in rasucit;
Exo 39:25  I-au mai facut ?i clopo?ei de aur curat ?i au pus clopo?ei printre ciucurii de la poalele meilului de jur împrejur;
Exo 39:26  ?i i-au a?ezat pe la poalele meilului de slujba a?a: un clopo?el ?i un ciucure, un clopo?el ?i un ciucure, cum poruncise Domnul lui Moise.
Exo 39:27  Au facut apoi pentru Aaron ?i pentru fiii lui hitoane ?esute din in,
Exo 39:28  Chidare de in, turbane tot de in ?i pantaloni de in rasucit;
Exo 39:29  ?i cingatoare din in rasucit ?i de matase violeta, stacojie ?i vi?inie, ?esuta cu alesaturi, cum poruncise Domnul lui Moise.
Exo 39:30  Dupa aceea au facut o tabli?a de aur curat, diadema sfin?eniei, ?i au sapat pe ea, ca pe pecete, cuvintele: "Sfin?enia Domnului".
Exo 39:31  ?i au prins de ea un ?nur de matase violeta, ca s-o lege peste chidar, cum poruncise Domnul lui Moise.
Exo 39:32  A?a s-au sfâr?it toate lucrarile de la cortul adunarii. ?i au facut fiii lui Israel toate; cum poruncise Domnul lui Moise a?a au facut.
Exo 39:33  Apoi au adus la Moise: cortul, acoperamintele ?i toate cele de trebuin?a ale lui, cârligele lui, scândurile lui, pârghiile lui, stâlpii lui ?i postamentele lui;
Exo 39:34  Acoperi?urile cele cu piei ro?ii de berbec ?i acoperi?urile cele de piei vinete ?i perdeaua din mijloc;
Exo 39:35  Chivotul legii, capacul lui ?i pârghiile;
Exo 39:36  Masa cu toate cele de trebuin?a pentru ea ?i pâinile de pus înainte;
Exo 39:37  Sfe?nicul cel de aur curat, candelele lui, candele puse în el la locul lor, ?i toate cele trebuincioase pentru el ?i untdelemn de ars;
Exo 39:38  Jertfelnicul cel de aur, mirul pentru ungere, miresme pentru tamâiere ?i perdeaua de la intrarea cortului;
Exo 39:39  Jertfelnicul cel de arama, cama?a lui cea de arama, pârghiile lui ?i toate cele trebuitoare pentru el, baia ?i postamentul ei;
Exo 39:40  Perdelele cur?ii, stâlpii ei ?i postamentele lor, perdelele de la intrarea cur?ii, frânghiile, ?aru?ii ?i toate lucrurile trebuitoare la slujba în cortul adunarii;
Exo 39:41  Ve?mintele de slujit în cort, ve?mintele sfinte ale preotului Aaron ?i ve?mintele de slujba pentru fiii lui.
Exo 39:42  Toate aceste lucruri le facusera fiii lui Israel a?a cum poruncise Domnul lui Moise.
Exo 39:43  ?i privi Moise toata lucrarea ?i iata ei o facusera a?a cum poruncise Domnul ?i Moise i-a binecuvântat.
Exo 40:1  Apoi iara?i a grait Domnul cu Moise ?i a zis:
Exo 40:2  "În ziua întâi a lunii întâi, sa a?ezi cortul adunarii.
Exo 40:3  ?i sa pui într-însul chivotul legii ?i dinaintea chivotului sa atârni perdeaua;
Exo 40:4  Apoi sa aduci înauntru masa ?i sa a?ezi pe ea toate lucrurile ei; ?i sfe?nicul sa-l duci înauntru ?i sa aprinzi într-însul candelele lui.
Exo 40:5  Sa a?ezi jertfelnicul cel de aur pentru tamâiere înaintea chivotului legii ?i sa atârni perdeaua la intrare în cortul adunarii.
Exo 40:6  Apoi sa a?ezi jertfelnicul arderilor de tot înaintea intrarii în cortul adunarii.
Exo 40:7  Iar între cortul adunarii ?i jertfelnic sa a?ezi baia ?i sa torni în ea apa;
Exo 40:8  ?i împrejurul cortului sa a?ezi împrejmuirea cur?ii ?i la intrarea cur?ii sa atârni perdeaua.
Exo 40:9  Dupa aceea sa iei mir de ungere ?i sa ungi cortul ?i toate cele din el ?i sa-l sfin?e?ti pe el ?i toate lucrurile lui ?i va fi sfânt;
Exo 40:10  Sa ungi jertfelnicul arderilor de tot ?i toate lucrurile lui ?i sa sfin?e?ti jertfelnicul ?i va fi sfin?enie mare;
Exo 40:11  Sa ungi apoi baia ?i postamentul ei ?i sa o sfin?e?ti.
Exo 40:12  Apoi sa aduci pe Aaron ?i pe fiii lui la u?a cortului adunarii ?i sa-i speli cu apa;
Exo 40:13  Sa îmbraci pe Aaron în sfintele ve?minte ?i sa-l ungi ?i sa-l sfin?e?ti, ca sa-mi fie preot.
Exo 40:14  Sa aduci ?i pe fiii lui, sa-i îmbraci cu hitoane,
Exo 40:15  ?i sa-i ungi, cum ai uns pe tatal lor, ca sa-Mi fie preo?i, ?i aceasta ungere îi va sfin?i preo?i pentru totdeauna în neamul lor!"
Exo 40:16  ?i a facut Moise tot; cum i-a poruncit Domnul a?a a facut:
Exo 40:17  În luna întâi a anului al doilea de la ie?irea lor din Egipt, în ziua întâi a lunii a fost a?ezat cortul.
Exo 40:18  ?i a a?ezat Moise cortul, a pus postamentele lui, scândurile lui, pârghiile lui ?i stâlpii lui;
Exo 40:19  A întins deasupra cortului acoperamintele ?i peste aceste acoperaminte a pus acoperi?ul, cum poruncise Domnul lui Moise.
Exo 40:20  Apoi a luat ?i a pus legea în chivot, a petrecut pârghiile prin inelele chivotului ?i a pus deasupra, la chivot, capacul;
Exo 40:21  A dus apoi chivotul în cort, a atârnat perdeaua ?i a închis chivotul legii, precum poruncise Domnul lui Moise.
Exo 40:22  Dupa aceea a pus masa în cortul adunarii, în partea de miazanoapte a cortului, în afara de perdea,
Exo 40:23  ?i a a?ezat pe ea pâinile punerii înaintea Domnului, cum poruncise Dumnezeu lui Moise.
Exo 40:24  Sfe?nicul l-a a?ezat în cortul adunarii, în fa?a mesei, în partea de miazazi a cortului,
Exo 40:25  ?i a aprins candelele lui înaintea Domnului, precum poruncise Domnul lui Moise.
Exo 40:26  A a?ezat jertfelnicul cel de aur în cortul adunarii, înaintea perdelei,
Exo 40:27  ?i a aprins pe el tamâie mirositoare, cum poruncise Domnul lui Moise;
Exo 40:28  A atârnat perdeaua la u?a cortului;
Exo 40:29  Iar jertfelnicul arderilor de tot 1-a a?ezat la intrarea în cortul adunarii ?i a pus pe el arderi de tot ?i prinoase de pâine, cum poruncise Domnul lui Moise.
Exo 40:30  Apoi a a?ezat baia între cortul adunarii ?i jertfelnic ?i a turnat în ea apa pentru spalat;
Exo 40:31  Moise, Aaron ?i fiii lui trebuia sa-?i spele din ea mâinile ?i picioarele;
Exo 40:32  Când intrau ei în cortul adunarii sau când se apropiau de jertfelnic ca sa slujeasca, se spalau din ea cum poruncise Domnul lui Moise.
Exo 40:33  Dupa aceea au pus împrejmuirea cur?ii împrejurul cortului ?i a jertfelnicului ?i a atârnat perdeaua la intrarea cur?ii. ?i a?a a ispravit Moise lucrarile.
Exo 40:34  Atunci un nor a acoperit cortul adunarii ?i loca?ul s-a umplut de slava Domnului;
Exo 40:35  ?i Moise n-a putut sa intre în cortul adunarii, pentru ca-l cuprinsese pe acesta norul ?i slava Domnului umpluse loca?ul.
Exo 40:36  În tot timpul calatoriei fiilor lui Israel, când se ridica norul de pe cort, atunci plecau la drum,
Exo 40:37  Iar de nu se ridica norul, nici ei nu plecau la drum pâna nu se ridica;
Exo 40:38  Pentru ca în tot timpul calatoriei, ziua statea peste cort norul Domnului, iar noaptea se afla peste el foc, înaintea ochilor întregii case a lui Israel.


\end{document}