\begin{document}

\title{Leviticus}

Lev 1:1  În vremea aceea, chemând pe Moise, Domnul i-a grait din cortul adunarii ?i i-a zis:
Lev 1:2  "Graie?te fiilor lui Israel ?i le spune: De va aduce cineva dintre voi jertfa Domnului din dobitoace, sa aduca jertfa din cireada de vite ?i din turma de oi.
Lev 1:3  De va fi jertfa lui ardere de tot din vite mari, sa fie parte barbateasca, fara meteahna, ?i s-o aduca la u?a cortului adunarii, ca sa fie bine-placuta înaintea Domnului.
Lev 1:4  Apoi sa-?i puna mâna pe capul jertfei cea pentru arderea de tot ?i î?i va afla bunavoin?a spre iertarea pacatelor lui.
Lev 1:5  Apoi sa junghie vi?elul, înaintea Domnului, iar preo?ii, fiii lui Aaron, sa aduca sângele ?i sa stropeasca cu sânge împrejur jertfelnicul de la u?a cortului adunarii.
Lev 1:6  Sa despoaie pielea de pe jertfa arderii de tot ?i sa taie jertfa în buca?i.
Lev 1:7  Dupa aceea preo?ii, fiii lui Aaron, sa puna pe jertfelnic foc ?i pe foc sa puna lemne;
Lev 1:8  ?i pe lemnele de pe focul care e pe jertfelnic, sa puna preo?ii, fiii lui Aaron, buca?ile, capul ?i grasimea, precum ?i maruntaiele ?i picioarele, dupa ce le vor spala cu apa.
Lev 1:9  ?i sa arda preo?ii toate acestea pe jertfelnic, ca ardere de tot, jertfa, mireasma placuta Domnului.
Lev 1:10  Iar daca jertfa adusa de el Domnului ca ardere de tot este din vite mici, sa fie din miei sau din iezi, parte barbateasca, fara meteahna.
Lev 1:11  Sa-?i puna mâna pe capul jertfei ?i s-o înjunghie înaintea Domnului, în partea de miazanoapte a jertfelnicului; iar preo?ii, fiii lui Aaron, sa stropeasca cu sângele ei jertfelnicul împrejur.
Lev 1:12  Sa o taie apoi în buca?i, despar?ind capul ?i grasimea ei; sa a?eze preotul buca?ile pe lemnele care sunt pe focul de pe jertfelnic;
Lev 1:13  Iar maruntaiele ?i picioarele sa le spele cu apa ?i sa aduca preotul toate ?i sa le arda pe jertfelnic, ca ardere de tot, jertfa, mireasma placuta Domnului.
Lev 1:14  Daca aduce el din pasari ardere de tot Domnului, sa aduca jertfa sa din turturele sau din pui de porumbel.
Lev 1:15  Preotul sa o aduca la jertfelnic, sa-i frânga gâtul ?i sa o puna pe jertfelnic, iar sângele sa-l scurga pe peretele jertfelnicului.
Lev 1:16  Gu?a ?i penele sa le scoata ?i sa le arunce lânga jertfelnic, în partea dinspre rasarit, la locul cenu?ei.
Lev 1:17  Apoi sa-i frânga aripile, fara a le desprinde de trup, ?i sa o arda preotul pe jertfelnic, pe lemnele ce sunt pe foc, ca ardere de tot, jertfa, mireasma placuta Domnului".
Lev 2:1  "Daca cineva voie?te sa jertfeasca Domnului prinos de pâine, sa aduca faina buna de grâu, sa toarne peste ea untdelemn ?i sa puna pe ea tamâie;
Lev 2:2  Apoi s-o aduca preo?ilor, fiilor lui Aaron, iar unul din ei sa ia o mâna plina de faina, cu untdelemnul ?i cu toata tamâia ?i s-o arda pe jertfelnic spre pomenire, ca jertfa, mireasma placuta Domnului.
Lev 2:3  Iar rama?i?a din prinosul de pâine va fi a lui Aaron ?i a fiilor lui; aceasta e sfin?enie mare din jertfele Domnului.
Lev 2:4  Daca însa vrei sa aduci jertfa prinosului de pâine din aluaturi coapte în cuptor, sa aduci pâini de faina buna de grâu, nedospite, framântate cu untdelemn, ?i turte nedospite, unse cu untdelemn.
Lev 2:5  Daca jertfa ta este prinos de pâine copt în tigaie, sa fie de faina buna de grâu, framântata cu untdelemn ?i nedospita.
Lev 2:6  Sa-l rupi buca?i ?i sa torni peste el untdelemn; acesta este prinos de pâine, adus Domnului.
Lev 2:7  Daca jertfa ta este prinos de pâine gatit în oala, sa se faca de faina buna de grâu cu untdelemn.
Lev 2:8  Prinosul Domnului, gatit a?a, sa se aduca ?i sa se încredin?eze preotului, iar acesta sa-l duca la jertfelnic.
Lev 2:9  Apoi sa ia preotul din jertfa o parte spre pomenire ?i s-o arda pe jertfelnic, ca ardere, ca mireasma placuta Domnului;
Lev 2:10  Iar rama?itele prinosului de pâine vor fi pentru Aaron ?i fiii lui; acestea-s sfin?enie mare din jertfele Domnului.
Lev 2:11  Orice prinos de pâine, ce aduce?i Domnului, sa nu-l face?i dospit, caci nici dospitura, nici miere nu ve?i arde, ca jertfa înaintea Domnului.
Lev 2:12  Ca prinos de pârga, sa aduce?i ?i de acestea Domnului, dar pe jertfelnic sa nu le înal?a?i întru mireasma bine-placuta Domnului.
Lev 2:13  Toate prinoasele tale de pâine sara-le cu sare; sa nu la?i jertfele tale fara sare, semnul legamântului Dumnezeului tau; cu toate prinoasele tale adu Domnului Dumnezeului tau ?i sare.
Lev 2:14  De aduci Domnului prinos de pâine din cele dintâi roade, adu ca dar din cele dintâi roade ale tale graun?e din spice, prajite pe foc ?i pisate;
Lev 2:15  Toarna peste ele untdelemn ?i pune pe ele tamâie; acesta este prinos de pâine.
Lev 2:16  Preotul sa arda, spre pomenire, o parte din graun?e ?i din untdelemn cu toata tamâia; aceasta este jertfa Domnului".
Lev 3:1  "Daca însa jertfa lui va fi jertfa de împacare ?i daca se va aduce din vite mari, parte barbateasca sau parte femeiasca, sa se aduca înaintea Domnului din cele fara meteahna
Lev 3:2  Sa-?i puna cel ce o aduce mâna sa pe capul jertfei ?i s-o junghie înaintea Domnului, la u?a cortului adunarii; iar preo?ii, fiii lui Aaron, sa stropeasca cu sânge din ea jertfelnicul împrejur.
Lev 3:3  Din jertfa de mântuire sa aduca jertfa Domnului: grasimea care acopera maruntaiele, toata grasimea ce acopera intestinele;
Lev 3:4  Amândoi rarunchii, grasimea de pe ei ?i cea de pe ?olduri, seul de pe ficat ?i cel de pe rarunchi;
Lev 3:5  Iar fiii lui Aaron sa arda acestea pe jertfelnic împreuna cu arderea de tot, care este pe lemnele de pe focul de pe jertfelnic; aceasta este jertfa, mireasma placuta Domnului.
Lev 3:6  Iar daca cineva aduce Domnului jertfa de împacare din vite mici, parte barbateasca sau femeiasca, s-o aduca din cele fara meteahna.
Lev 3:7  Daca aduce jertfa o oaie, sa o înfa?i?eze înaintea Domnului.
Lev 3:8  Sa-?i puna mâna sa pe capul jertfei sale ?i s-o junghie înaintea cortului adunarii; iar preo?ii, fiii lui Aaron, sa stropeasca cu sângele ei jertfelnicul pe toate par?ile.
Lev 3:9  ?i din aceasta jertfa de împacare sa aduca ardere Domnului grasimea ei, toata coada, retezând-o chiar din capatul spinarii, grasimea de pe maruntaie, toata grasimea de pe partea dinauntru;
Lev 3:10  Amândoi rarunchii, grasimea de pe ei ?i cea de pe ?olduri, seul de pe ficat ?i praporul, pe care-l va desprinde cu rarunchii;
Lev 3:11  Iar preotul sa arda acestea pe jertfelnic; aceasta mistuire prin foc este jertfa Domnului.
Lev 3:12  Daca însa jertfa lui este din capre, s-o înfa?i?eze înaintea Domnului,
Lev 3:13  Sa-?i puna mâna sa pe capul caprei ?i s-o junghie la u?a cortului adunarii; iar preo?ii, fiii lui Aaron, sa stropeasca cu sângele ei jertfelnicul împrejur.
Lev 3:14  Din acestea sa aduca prinos ?i jertfa Domnului: grasimea de pe maruntaie, toata grasimea care acopera intestinele,
Lev 3:15  Amândoi rarunchii, grasimea de pe ei ?i cea de pe ?olduri, seul de pe ficat pe care-l va desprinde cu cel de pe rarunchi;
Lev 3:16  ?i sa le arda preotul pe jertfelnic; aceasta ardere pe foc este mireasma placuta Domnului. Toata grasimea este a Domnului.
Lev 3:17  Este lege ve?nica ?i pentru to?i urma?ii vo?tri din toate a?ezarile voastre, ca toata grasimea ?i tot sângele sa nu-l mânca?i".
Lev 4:1  ?i a grait Domnul cu Moise ?i a zis:
Lev 4:2  "Graie?te fiilor lui Israel ?i le spune: Daca vreun om va pacatui din ne?tiin?a împotriva poruncilor Domnului ?i va face ce nu se cuvine, calcând vreuna din ele;
Lev 4:3  De a pacatuit arhiereu miruit ?i a tras pe popor la pacat, pentru pacatul sau, pe care l-a savâr?it, sa aduca un vi?el fara meteahna, ca jertfa Domnului pentru pacat;
Lev 4:4  Sa înfa?i?eze vi?elul înaintea Domnului, la u?a cortului adunarii, sa-?i puna mâna sa pe capul vi?elului ?i sa junghie vi?elul înaintea Domnului.
Lev 4:5  Apoi sa ia preotul cel miruit, ale carui mâini sunt sfin?ite, din sângele vi?elului ?i sa-l duca în cortul adunarii.
Lev 4:6  Acolo sa-?i moaie preotul degetul sau în sânge, sa stropeasca cu sânge de ?apte ori înaintea Domnului, asupra perdelei loca?ului sfânt.
Lev 4:7  Dupa aceea sa puna preotul din sângele vi?elului înaintea Domnului, pe coarnele jertfelnicului tamâierii, care se afla în cortul adunarii, iar toata rama?i?a din sângele vi?elului s-o toarne la temelia jertfelnicului arderii de tot, care se afla înaintea cortului adunarii.
Lev 4:8  Apoi sa scoata din vi?elul adus pentru pacat toata grasimea lui, grasimea cea de pe maruntaie, toata grasimea ce acopera launtrul,
Lev 4:9  Amândoi rarunchii cu grasimea de pe ei ?i cea de pe ?olduri, seul de pe ficat; acestea sa le scoata împreuna cu rarunchii,
Lev 4:10  Precum se ia din vi?elul jertfei de izbavire, ?i sa le arda preotul pe jertfelnicul arderii de tot.
Lev 4:11  Iar pielea vi?elului ?i tot trupul lui cu capul ?i cu picioarele lui, cu maruntaiele lui ?i cu necura?enia lui,
Lev 4:12  Adica tot vi?elul sa-l scoata afara din tabara, la loc curat, unde se arunca cenu?a, ?i sa-l arda pe foc de lemne; unde se arunca cenu?a, acolo sa-l arda.
Lev 4:13  Daca însa toata ob?tea lui Israel va pacatui, din ne?tiin?a, ?i va face împotriva poruncilor Domnului ceva ce nu trebuia facut ?i vrednic de osânda, iar fapta `aceasta va ramâne necunoscuta adunarii,
Lev 4:14  Când se va afla pacatul, pe care l-au savâr?it ei, sa se aduca din partea întregii ob?ti un vi?el fara meteahna, jertfa pentru pacat, sa-l înfa?i?eze înaintea cortului adunarii,
Lev 4:15  Iar batrânii ob?tii sa-?i puna mâinile lor pe capul vi?elului, înaintea Domnului ?i sa junghie vi?elul înaintea Domnului.
Lev 4:16  Apoi preotul miruit sa duca din sângele vi?elului în cortul adunarii.
Lev 4:17  Sa-?i moaie preotul degetul sau în sângele vi?elului ?i sa stropeasca de ?apte ori înaintea Domnului asupra perdelei sfintei sfintelor.
Lev 4:18  Apoi preotul sa puna din sânge pe coarnele jertfelnicului tamâierii, care este înaintea fe?ei Domnului în cortul adunarii, iar celalalt sânge sa-l toarne la temelia jertfelnicului arderii de tot, care este la u?a cortului adunarii.
Lev 4:19  Toata grasimea lui s-o scoata din el ?i s-o arda pe jertfelnic;
Lev 4:20  ?i sa faca cu vi?elul acesta ceea ce s-a facut cu vi?elul adus pentru pacat; a?a sa faca cu el ?i a?a sa-i cure?e preotul ?i li se va ierta pacatul.
Lev 4:21  Dupa aceea sa scoata vi?elul întreg afara din tabara ?i sa-l arda a?a cum a ars ?i vi?elul de care s-a vorbit mai sus. Aceasta e jertfa pentru pacatul ob?tii.
Lev 4:22  Iar daca va gre?i o capetenie ?i din ne?tiin?a va face împotriva uneia din toate poruncile Domnului Dumnezeului sau ceva ce nu trebuia sa faca ?i vrednic de osânda,
Lev 4:23  Când va afla el pacatul sau, pe care l-a savâr?it, sa aduca jertfa pentru pacat un ?ap fara meteahna,
Lev 4:24  Sa-?i puna mâna sa pe capul ?apului ?i sa-l junghie, unde se junghie arderile de tot, înaintea Domnului; aceasta este jertfa pentru pacat.
Lev 4:25  Iar preotul sa ia cu degetul sau sânge de la jertfa pentru pacat ?i sa-l puna pe coarnele jertfelnicului arderii de tot, iar celalalt sânge sa-l toarne la temelia jertfelnicului arderii de tot.
Lev 4:26  Toata grasimea ei s-o arda pe jertfelnic, ca grasimea jertfei de izbavire, ?i a?a îl va cura?i preotul de pacatul lui ?i i se va ierta.
Lev 4:27  Daca însa un om din poporul de rând va gre?i din ne?tiin?a împotriva uneia din toate poruncile Domnului ?i va face ceva ce nu trebuia sa faca ?i vrednic de osânda,
Lev 4:28  Când va afla el pacatul ce l-a savâr?it, sa aduca din caprele sale jertfa o capra fara meteahna, pentru pacatul ce l-a savâr?it,
Lev 4:29  Sa-?i puna mâna sa pe capul jertfei pentru pacat ?i sa junghie capra adusa, jertfa pentru pacat, unde se junghie jertfele arderii de tot.
Lev 4:30  Apoi sa ia preotul din sângele ei cu degetul sau ?i sa puna pe coarnele jertfelnicului arderii de tot, iar celalalt sânge sa-l toarne la temelia jertfelnicului.
Lev 4:31  Toata grasimea ei s-o aleaga, cum se alege grasimea la jertfele de mântuire, ?i s-o arda preotul pe jertfelnic, spre miros bine-placut Domnului; astfel îl va cura?i preotul ?i i se va ierta pacatul.
Lev 4:32  Iar daca cineva vrea sa aduca jertfa pentru pacat din turma de oi, sa aduca parte femeiasca, fara meteahna,
Lev 4:33  Sa-?i puna mâna sa pe capul jertfei pentru pacat ?i s-o junghie, ca jertfa pentru pacat, la locul unde se junghie jertfa arderii de tot.
Lev 4:34  Apoi sa ia preotul cu degetul sau din sângele acestei jertfe pentru pacat ?i sa puna pe coarnele jertfelnicului arderii de tot, iar celalalt sânge sa-l toarne jos lânga jertfelnic.
Lev 4:35  Toata grasimea ei s-o aleaga, cum se alege grasimea din oaia pentru jertfa de izbavire, ?i s-o arda preotul pe jertfelnic, ca jertfa Domnului; ?i a?a îl va cura?i preotul de pacatul ce l-a savâr?it ?i i se va ierta".
Lev 5:1  Daca vreun suflet va pacatui prin aceea ca, fiind pus sa jure ca martor, nu va spune ceea ce a auzit sau ce ?tie, acela va lua asupra sa pacat.
Lev 5:2  Sau de se va atinge cineva de orice lucru necurat, sau de trup necurat de fiara, sau de stârv de dobitoc necurat, sau de stârv de târâtoare necurata, fara sa ?tie, se face necurat ?i vinovat;
Lev 5:3  Sau de se va atinge cineva de necura?enie omeneasca, sau de orice fel de necura?enie care spurca, ?i nu va ?ti, dar apoi va afla, acela e vinovat.
Lev 5:4  Sau de se va jura cineva cu buzele sale nebune?te sa faca ceva rau sau bine, orice fel ar fi fapta pentru care se jura oamenii fara socoteala, de nu va ?ti ca aceasta este rau, ci va afla în urma, e vinovat.
Lev 5:5  Deci, de se va face cineva vinovat de ceva din acestea ?i î?i va marturisi pacatul,
Lev 5:6  Atunci, pentru pacatul sau, pe care l-a savâr?it, sa aduca Domnului jertfa din turma, o oaie sau o capra din caprele sale, pentru vina pacatului, ?i-l va cura?i preotul prin aceasta de pacatul sau ?i i se va ierta pacatul.
Lev 5:7  Iar de nu va fi în stare sa aduca jertfa o oaie, pentru vina pacatului sau, sa aduca Domnului doua turturele sau doi pui de porumbel: unul jertfa pentru pacat, iar altul ardere de tot.
Lev 5:8  Aceste pasari sa le aduca la preot ?i preotul sa jertfeasca mai întâi pe cea pentru pacat, sa-i frânga gâtul, fara sa desparta capul de trup,
Lev 5:9  ?i sa stropeasca cu sângele acestei jertfe pentru pacat peretele jertfelnicului, iar celalalt sânge sa-l scurga jos lânga jertfelnic; aceasta e jertfa pentru pacat.
Lev 5:10  Iar pe cealalta pasare s-o aduca ardere de tot, dupa rânduiala. ?i a?a îl va cura?i preotul de pacatul lui ?i i se va ierta.
Lev 5:11  Daca însa nu-i va da mâna sa aduca nici o pereche de turturele sau doi pui de porumbel, atunci sa aduca pentru gre?eala sa a zecea parte dintr-o efa de faina buna de grâu, ca jertfa pentru pacat, dar sa nu toarne pe ea untdelemn, nici tamâie sa nu puna pe ea, ca aceasta este jertfa pentru pacat.
Lev 5:12  S-o aduca la preot ?i preotul sa ia din ea un pumn plin, spre pomenire, ?i s-o arda pe jertfelnic, ca jertfa Domnului; aceasta este jertfa pentru pacat.
Lev 5:13  Prin aceasta îl va cura?i preotul de pacatul lui, pe care l-a savâr?it în una din întâmplarile acelea, ?i i se va ierta pacatul; rama?i?a de faina va fi a preotului, ca la prinosul de faina".
Lev 5:14  Apoi a grait Domnul cu Moise ?i a zis:
Lev 5:15  De va face cineva gre?eala ?i din ne?tiin?a va pacatui împotriva celor afierosite Domnului, acela, pentru vina sa, sa ia din turma de oi ?i sa aduca Domnului jertfa pentru vina, un berbec fara meteahna, pre?uit la doi sicli de argint, dupa pre?ul siclului sfânt.
Lev 5:16  ?i ce a gre?it împotriva lucrului sfânt, va plati ?i va mai adauga peste pre?ul lui a cincea parte din pre? ?i va da aceasta preotului ?i preotul îl va cura?i prin berbecul jertfei pentru vina ?i i se va ierta.
Lev 5:17  De va gre?i cineva împotriva uneia din toate poruncile Domnului ?i va face ce nu se cuvine sa faca ?i din ne?tiin?a s-a facut vinovat ?i va fi sub pacat,
Lev 5:18  Acela sa aduca la preot din turma de oi, jertfa pentru vina, un berbec fara meteahna, dupa pre?uirea ta, ?i-i va cura?i preotul gre?eala, în care a cazut el din ne?tiin?a ?i i se va ierta.
Lev 5:19  Aceasta este jertfa pentru gre?eala cu care s-a facut vinovat el înaintea Domnului".
Lev 6:1  Grait-a iara?i Domnul cu Moise ?i a zis:
Lev 6:2  "Daca cineva va gre?i ?i cu buna ?tiin?a va nesocoti poruncile Domnului, tagaduind înaintea aproapelui sau ceea ce acesta i-a încredin?at, sau i-a lasat în pastrare, sau ceea ce i-a furat, sau va în?ela pe aproapele sau,
Lev 6:3  Sau gasind un lucru pierdut ?i tagaduind înaintea lui, sau jurându-se strâmb pentru ceva, ce atrage pedeapsa asupra oamenilor,
Lev 6:4  Daca se va dovedi ca a gre?it ?i s-a facut vinovat, sa întoarca ce a furat, sau ce a rapit, sau ce i-a fost încredin?at, sau ce a fost pierdut ?i gasit de el.
Lev 6:5  Tot lucrul, pentru care s-a jurat strâmb, sa-l plateasca deplin ?i sa mai adauge pe deasupra a cincea parte din pre?ul lui ?i sa dea aceluia, al caruia este lucrul, în ziua când î?i va cunoa?te vina sa.
Lev 6:6  Iar pentru vina sa sa ia din turma de oi un berbec fara meteahna, dupa pre?uirea ta, ?i sa-l aduca Domnului prin preot, jertfa pentru vina.
Lev 6:7  Preotul îl va cura?i înaintea Domnului ?i i se va ierta orice ar fi faptuit ?i oricum s-ar fi facut vinovat".
Lev 6:8  ?i a grait Domnul cu Moise ?i a zis:
Lev 6:9  "Porunce?te lui Aaron ?i fiilor lui ?i le zi: Rânduiala arderii de tot este aceasta: arderea de tot sa ramâna pe vatra jertfelnicului toata noaptea pâna diminea?a ?i focul jertfelnicului sa arda pe el ?i sa nu se stinga.
Lev 6:10  Iar diminea?a preotul sa se îmbrace cu haina sa cea de in, dupa ce ?i-a luat pantalonii sai cei de in pe trupul sau, sa ridice cenu?a arderii de tot, pe care a ars-o focul pe jertfelnic, ?i s-o puna lânga jertfelnic.
Lev 6:11  Apoi sa-?i dezbrace hainele sale ?i sa se îmbrace cu alte haine ?i sa scoata cenu?a afara din tabara; la loc curat.
Lev 6:12  Dar focul pe jertfelnic sa arda ?i sa nu se stinga; preotul sa puna pe el lemne în fiecare diminea?a, sa a?eze pe el ardere de tot ?i sa arda pe el grasimea jertfei de mântuire.
Lev 6:13  Iar focul sa arda necontenit pe jertfelnic ?i sa nu se stinga.
Lev 6:14  Rânduiala prinosului de pâine, pe care preo?ii, fiii lui Aaron, trebuie sa-l aduca înaintea Domnului la jertfelnic, este aceasta:
Lev 6:15  Sa ia preotul din prinosul acesta de pâine un pumn de faina de grâu, cu untdelemnul ei ?i cu toata tamâia, care e pe prinos, ?i sa le arda pe jertfelnic mireasma placuta de pomenire înaintea Domnului.
Lev 6:16  Iar rama?i?a din ea s-o manânce Aaron ?i fiii lui ?i s-o manânce nedospita, în locul cel sfânt; în curtea cortului adunarii s-o manânce, dar sa nu o coaca dospita.
Lev 6:17  Aceasta le-o dau parte din jertfele Mele. Aceasta este sfin?enie mare, ca ?i jertfa pentru pacat ?i ca ?i jertfa pentru vina.
Lev 6:18  Tot barbatul din neamul preo?esc poate sa manânce din ea. Aceasta e lege ve?nica în neamul vostru din jertfele Domnului. Tot ce se va atinge de ea, se va sfin?i".
Lev 6:19  A grait Domnul cu Moise ?i a zis:
Lev 6:20  "Prinosul lui Aaron ?i al fiilor lui, pe care-l vor aduce ei Domnului, în ziua ungerii lor, este acesta: faina buna de grâu, a zecea parte din efa, vor aduce jertfa necontenita; jumatate din ea diminea?a ?i jumatate seara.
Lev 6:21  S-o gateasca în tigaie, cu untdelemn; s-o rupi buca?i, cum se rupe prinosul de pâine framântat cu untdelemn; sa o aduci întru mireasma placuta Domnului.
Lev 6:22  Aceasta s-o savâr?easca preotul, care se va mirui în locul lui Aaron, din fiii lui; acesta este a?ezamânt ve?nic. Prinosul acesta sa-l arda tot.
Lev 6:23  Orice prinos de pâine din partea preotului sa se arda tot ?i sa nu se manânce nimic din el".
Lev 6:24  ?i a grait Domnul cu Moise ?i  a zis:
Lev 6:25  "Spune lui Aaron ?i fiilor lui ?i le zi: Rânduiala jertfei pentru pacat este aceasta: jertfa pentru pacat sa se junghie înaintea Domnului, în locul unde se junghie ?i cea pentru arderea de tot. Aceasta este sfin?enie mare.
Lev 6:26  Preotul cel ce savâr?e?te jertfa cea pentru pacat s-o manânce, dar s-o manânce în locul cel sfânt, în curtea cortului adunarii.
Lev 6:27  Tot ce se va atinge de carnea ei se va sfin?i; ?i de se va stropi cu sângele ei haina, haina stropita sa se spele în locul cel sfânt.
Lev 6:28  Oala de lut, în care s-a fiert ea, sa se sparga; iar daca ea s-a fiert în vas de arama, acesta sa se cure?e ?i sa se spele cu apa.
Lev 6:29  To?i cei de parte barbateasca din neamul preo?esc pot sa manânce din ea. Aceasta este mare sfin?enie înaintea Domnului.
Lev 6:30  Dar orice jertfa pentru pacat din al carei sânge s-a dus în cortul adunarii pentru facerea cura?irii în locul cel sfânt, sa nu se manânce, ci sa se arda în foc".
Lev 7:1  "Iata ?i rânduiala jertfei pentru vina: Aceasta este sfin?enie mare.
Lev 7:2  Jertfa pentru vina sa se junghie în locul unde se junghie jertfa arderii de tot ?i cu sângele ei sa se stropeasca jertfelnicul de jur împrejur.
Lev 7:3  Cel ce o aduce sa osebeasca din ea toata grasimea, coada ?i grasimea de pe maruntaie,
Lev 7:4  Amândoi rarunchii, grasimea cea de pe ei ?i seul de pe ficat: toate acestea sa le osebeasca împreuna cu cei doi rarunchi.
Lev 7:5  Acestea sa le arda preotul pe jertfelnic, ca jertfa Domnului. Aceasta este jertfa pentru vina.
Lev 7:6  To?i cei de parte barbateasca din neamul preo?esc sa manânce din ea, dar s-o manânce în locul cel sfânt, ca aceasta este sfin?enie mare.
Lev 7:7  La jertfa pentru vina, ca ?i la jertfa pentru pacat, este aceea?i rânduiala; ele sunt partea preotului, care savâr?e?te cura?irea cu ajutorul lor.
Lev 7:8  Când preotul va aduce jertfa arderii de tot a cuiva, pielea jertfei aduse va fi a preotului.
Lev 7:9  Tot prinosul de pâine copt în cuptor ?i tot prinosul de pâine gatit în oala sau în tigaie va fi al preotului, care-l savâr?e?te.
Lev 7:10  Orice dar de pâine, framântat cu untdelemn sau uscat, va fi al tuturor fiilor lui Aaron deopotriva.
Lev 7:11  Iar rânduiala jertfei de împacare, care se aduce Domnului, este aceasta:
Lev 7:12  Daca se va aduce ca jertfa de mul?umire, atunci sa se aduca pâini framântate cu untdelemn, turte nedospite, unse cu untdelemn, faina de grâu, framântata cu untdelemn;
Lev 7:13  Pe lânga pâinile nedospite sa se mai aduca dar la jertfa de mul?umire ?i pâine dospita.
Lev 7:14  Unul din toate aceste daruri ale sale sa-l aduca Domnului dar ridicat; acesta va fi al preotului, care strope?te cu sângele jertfei de mântuire.
Lev 7:15  ?i carnea jertfei de mântuire, ca dar de mul?umire, va fi tot a lui, însa sa se manânce în ziua aducerii ei ?i sa nu ramâna din ea nimic pe a doua zi.
Lev 7:16  Daca însa jertfa ce se aduce este din fagaduin?a sau de bunavoie, jertfa lui sa se manânce în ziua aducerii ?i ceea ce va ramâne se poate mânca a doua zi.
Lev 7:17  Iar ceea ce va mai ramâne din carnea jertfei pe a treia zi sa se arda cu foc.
Lev 7:18  Daca însa carnea jertfei acesteia o va mânca cineva a treia zi, jertfa aceasta nu va fi primita ?i nu i se va ?ine în seama, ca este întinare ?i cel ce o va mânca va avea asupra sa pacat.
Lev 7:19  Carnea care a fost atinsa de ceva necurat sa nu se manânce, ci sa se arda cu foc; iar carnea curata sa se manânce tot de cel curat.
Lev 7:20  Daca însa vreun om, în stare de necura?ie, va mânca din carnea jertfei de mântuire, adusa Domnului, acel suflet se va stârpi din poporul sau.
Lev 7:21  Daca vreun om, care s-a atins de ceva necurat, de necura?enie omeneasca, sau de dobitoc necurat, sau de vreo târâtoare necurata, va mânca din carnea jertfei de izbavire, adusa Domnului, omul acela se va stârpi din poporul sau".
Lev 7:22  ?i a grait Domnul cu Moise ?i a zis:
Lev 7:23  "Graie?te fiilor lui Israel ?i le zi: Nici un fel de grasime, nici de bou, nici de oaie, nici de ?ap sa nu mânca?i.
Lev 7:24  Grasimea de mortaciune ?i grasimea dobitocului sfâ?iat de fiara sa se întrebuin?eze la orice lucru, iar de mâncat sa nu se manânce.
Lev 7:25  Tot cel ce va mânca grasimea dobitocului, care se aduce jertfa mistuita cu foc Domnului, acela sa se stârpeasca din poporul sau.
Lev 7:26  Nici un fel de sânge sa nu mânca?i în toate ceta?ile voastre, nici de pasari, nici de dobitoace.
Lev 7:27  Tot cel ce va mânca sânge, acela se va stârpi din poporul sau".
Lev 7:28  ?i a grait Domnul cu Moise ?i a zis:
Lev 7:29  "Graie?te fiilor lui Israel ?i le zi: Cel ce î?i înfa?i?eaza Domnului jertfa sa de mântuire, acela din jertfa sa de mântuire sa aduca o parte prinos Domnului,
Lev 7:30  ?i anume: Sa aduca Domnului jertfa cu mâinile sale: grasimea de pe pieptul jertfei ?i seul de pe ficat; sa aduca leganând pieptul jertfei înaintea Domnului.
Lev 7:31  Grasimea s-o arda preotul pe jertfelnic, iar pieptul va fi al lui Aaron ?i al fiilor lui.
Lev 7:32  ?i spata dreapta din jertfele de izbavire ce aduce?i sa o da?i preotului.
Lev 7:33  Spata dreapta va fi partea aceluia din fiii lui Aaron, care va aduce pe jertfelnic sângele ?i grasimea jertfei de izbavire;
Lev 7:34  Caci Eu voi lua de la fiii lui Israel, din jertfele lor de izbavire, pieptul leganat ?i spata dreapta ?i le voi da lui Aaron preotul ?i fiilor lui ca venit ve?nic de la fiii lui Israel.
Lev 7:35  Acestea sunt partea lui Aaron ?i partea fiilor lui din jertfele Domnului, pe care o vor primi din ziua când se vor înfa?i?a ei înaintea Domnului, ca sa slujeasca,
Lev 7:36  ?i pe care a poruncit Domnul sa li se dea de catre fiii lui Israel din ziua ungerii lor. Aceasta este hotarâre ve?nica în neamul lor".
Lev 7:37  Aceasta este rânduiala arderii de tot, a darului de pâine, a jertfei pentru pacat, a jertfei pentru vina, a jertfei afierosirii ?i a jertfei de mântuire,
Lev 7:38  Cum a dat-o Domnul lui Moise pe Muntele Sinai, când a poruncit fiilor lui Israel, în pustiul Sinai, sa-?i aduca prinoasele lor Domnului.
Lev 8:1  ?i a grait Domnul cu Moise ?i a zis:
Lev 8:2  "Ia pe Aaron ?i împreuna cu el ?i pe fiii lui, ve?mintele, mirul de miruit, vi?elul de jertfa cea pentru pacat, panerul cu azimile ?i cei doi berbeci,
Lev 8:3  ?i aduna toata ob?tea la u?a cortului adunarii".
Lev 8:4  ?i a facut Moise a?a cum îi poruncise Domnul ?i a adunat ob?tea la u?a cortului adunarii.
Lev 8:5  Dupa aceea a zis Moise catre ob?te: "Iata ce porunce?te Domnul sa se faca!"
Lev 8:6  Deci a adus Moise pe Aaron ?i pe fiii lui ?i i-a spalat cu apa:
Lev 8:7  Apoi a îmbracat pe Aaron cu hitonul, l-a încins cu brâul, l-a îmbracat cu meilul, i-a pus efodul, l-a încins cu cingatoarea efodului ?i i-a strâns cu ea efodul;
Lev 8:8  Dupa aceea i-a pus ho?enul ?i în ho?en i-a pus Urim ?i Tumim,
Lev 8:9  Iar pe cap i-a pus chidarul ?i la chidar, în partea lui de dinainte, i-a prins tabli?a cea de aur, diadema sfin?eniei, cum poruncise Domnul lui Moise.
Lev 8:10  Apoi a luat Moise mirul de miruit ?i a miruit cortul ?i toate cele din el ?i le-a sfin?it.
Lev 8:11  A stropit cu el de ?apte ori asupra jertfelnicului ?i a miruit jertfelnicul ?i toate obiectele lui, baia ?i capatâiul ei ?i le-a sfin?it.
Lev 8:12  Dupa aceea a turnat Moise mir pe capul lui Aaron ?i l-a uns ?i l-a sfin?it.
Lev 8:13  ?i a adus Moise pe fiii lui Aaron, i-a îmbracat cu hitoane, i-a încins cu brâie ?i le-a pus turbane, cum poruncise Domnul lui Moise.
Lev 8:14  Apoi a adus Moise vi?elul cel de jertfa pentru pacat, iar Aaron ?i fiii lui ?i-au pus mâinile pe capul vi?elului de jertfa pentru pacat;
Lev 8:15  ?i l-a junghiat Moise ?i a luat din sânge ?i cu degetul sau a pus pe coarnele jertfelnicului de toate par?ile ?i a cura?it jertfelnicul, iar celalalt sânge l-a turnat jos lânga jertfelnic ?i a sfin?it jertfelnicul ca sa fie curat.
Lev 8:16  A luat apoi Moise toata grasimea de pe maruntaie, seul de pe ficat, amândoi rarunchii ?i grasimea lor, ?i le-a ars pe jertfelnic.
Lev 8:17  Iar vi?elul, pielea lui, carnea lui ?i necura?enia lui, le-a ars cu foc afara din tabara, cum poruncise Domnul lui Moise.
Lev 8:18  Apoi Moise a adus berbecul cel pentru ardere de tot, iar Aaron ?i fiii lui ?i-au pus mâinile pe capul berbecului.
Lev 8:19  ?i apoi a junghiat Moise berbecul ?i a stropit cu sânge jertfelnicul de jur împrejur.
Lev 8:20  A taiat apoi berbecul în buca?i ?i a adus Moise buca?ile, capa?âna ?i grasimea, iar maruntaiele ?i picioarele le-a spalat cu apa.
Lev 8:21  ?i a ars Moise tot berbecul pe jertfelnic; aceasta era ardere de tot spre mireasma placuta, aceasta era jertfa Domnului, cum poruncise Domnul lui Moise.
Lev 8:22  Dupa aceea a adus Moise al doilea berbec, berbecul cel pentru sfin?ire, ?i ?i-au pus Aaron ?i fiii lui mâinile pe capul berbecului.
Lev 8:23  ?i junghiindu-l, Moise a luat din sângele lui ?i a pus pe vârful urechii drepte a lui Aaron, pe degetul cel mare de la mâna dreapta a lui ?i pe degetul cel mare de la piciorul drept al lui.
Lev 8:24  Apoi a adus Moise pe fiii lui Aaron ?i a pus sânge pe vârful urechilor drepte ale lor, pe degetul cel mare de la mâinile drepte ale lor ?i pe degetul cel mare de la picioarele drepte ale lor; ?i a stropit Moise jertfelnicul cu sânge de jur împrejur.
Lev 8:25  Dupa aceea a luat Moise grasimea ?i coada, toata grasimea de pe maruntaie, seul de pe ficat, amândoi rarunchii, grasimea lor ?i ?oldul drept;
Lev 8:26  Iar din panerul cu pâinile punerii înaintea Domnului a luat o azima, o pâine cu untdelemn ?i turta ?i le-a a?ezat peste grasime ?i peste ?oldul drept;
Lev 8:27  ?i toate acestea le-a pus pe mâinile lui Aaron ?i pe mâinile fiilor sai ?i le-au dus leganându-le înaintea fe?ei Domnului.
Lev 8:28  Apoi a luat Moise acestea din mâinile lor ?i le-a ars pe jertfelnic ardere de tot; aceasta a fost jertfa de sfin?ire, mireasma placuta, jertfa Domnului.
Lev 8:29  Luând apoi pieptul, Moise l-a adus, leganându-l înaintea fe?ei Domnului; aceasta era partea lui Moise din berbecul sfin?irii, cum poruncise Domnul lui Moise.
Lev 8:30  Apoi a luat Moise mir de miruit ?i sânge de lânga jertfelnic ?i a stropit pe Aaron, ve?mintele lui, pe fiii lui ?i ve?mintele fiilor lui împreuna cu el; ?i a?a a sfin?it pe Aaron ?i ve?mintele lui ?i, împreuna cu el, ?i pe fiii lui ?i ve?mintele fiilor lui.
Lev 8:31  Apoi a zis Moise catre Aaron ?i catre fiii lui: "Fierbe?i carnea la intrarea cortului adunarii ?i acolo s-o mânca?i cu pâinea cea din panerul sfin?irii, dupa cum mi s-a poruncit mie ?i mi s-a zis: "Aaron ?i fiii lui s-o manânce!"
Lev 8:32  Iar rama?i?ele de carne ?i de pâine sa le arde?i cu foc.
Lev 8:33  ?apte zile sa nu va departa?i de la u?a cortului adunarii, pâna se vor împlini zilele sfin?irii voastre, ca sfin?irea voastra trebuie sa se savâr?easca în ?apte zile.
Lev 8:34  Cum s-a facut astazi, a?a a poruncit Domnul sa se faca pentru cura?irea voastra ?i în celelalte zile.
Lev 8:35  La u?a cortului adunarii ve?i ?edea ziua ?i noaptea timp de ?apte zile ?i ve?i fi de straja la Domnul, ca sa nu muri?i, ca a?a mi s-a poruncit mie de la Domnul Dumnezeu".
Lev 8:36  ?i au împlinit Aaron ?i fiii lui toate rânduielile, câte le poruncise Domnul prin Moise.
Lev 9:1  Iar în ziua a opta a chemat Moise pe Aaron, pe fiii lui ?i pe batrânii lui Israel,
Lev 9:2  ?i a zis catre Aaron: "Ia-?i din turma un vi?el de jertfa pentru pacat ?i un berbec pentru arderea de tot, amândoi fara meteahna, ?i-i adu înaintea fe?ei Domnului.
Lev 9:3  Iar batrânilor lui Israel sa le graie?ti ?i sa le spui: Lua?i din turma de capre un ?ap, ca jertfa pentru pacat, un berbec, un vi?el ?i un miel, to?i de un an ?i fara meteahna, ca sa fie adu?i ardere de tot,
Lev 9:4  Precum ?i un bou ?i un berbec pentru jertfa de mântuire, ca sa se savâr?easca jertfa înaintea fe?ei Domnului, ?i prinos de pâine, framântat cu untdelemn, ca astazi are sa vi se arate Domnul".
Lev 9:5  Deci au luat ei ?i au adus înaintea cortului adunarii cele ce poruncise Moise ?i a venit toata ob?tea ?i a stat înaintea fe?ei Domnului.
Lev 9:6  Atunci a zis Moise catre ob?te: "Iata ce a poruncit Domnul sa face?i, ca sa vi se arate slava Domnului!"
Lev 9:7  Iar catre Aaron Moise a zis: "Apropie-te de jertfelnic ?i savâr?e?te jertfa ta cea pentru pacat ?i arderea de tot a ta ?i cura?e?te-te pe tine ?i casa ta; apoi adu darurile poporului ?i-l cura?e?te, cum a poruncit Domnul!"
Lev 9:8  Deci, s-a apropiat Aaron de jertfelnic ?i a junghiat vi?elul cel de jertfa pentru pacatul sau.
Lev 9:9  Iar fiii lui Aaron, aducând sângele la el, ?i-a muiat degetul în sânge ?i a pus pe coarnele jertfelnicului, iar celalalt sânge l-a turnat jos lânga jertfelnic.
Lev 9:10  Grasimea, rarunchii ?i seul de pe ficat, de la jertfa cea pentru pacat, le-a ars pe jertfelnic, cum poruncise Domnul lui Moise;
Lev 9:11  Iar carnea ?i pielea le-a ars pe foc afara din tabara.
Lev 9:12  Apoi a junghiat jertfa cea pentru arderea de tot ?i, aducându-i fiii lui Aaron sângele, a stropit jertfelnicul din toate par?ile.
Lev 9:13  Dupa aceea i-au adus arderea de tot în buca?i ?i el le-a pus împreuna cu capa?âna pe jertfelnic,
Lev 9:14  Iar maruntaiele ?i picioarele le-a spalat cu apa ?i le-a pus peste arderea de tot ?i le-a ars pe jertfelnic.
Lev 9:15  Apoi a adus prinosul poporului: luând ?apul cel pentru pacatul poporului l-a junghiat ?i l-a adus jertfa pentru pacat, ca ?i berbecul.
Lev 9:16  Dupa aceea a adus arderea de tot, savâr?ind-o dupa rânduiala.
Lev 9:17  A adus de asemenea prinos de pâine ?i, luând din el o mâna plina, a ars pe jertfelnic, pe lânga arderea de tot cea de diminea?a.
Lev 9:18  Apoi a junghiat boul ?i berbecul pentru jertfa de mântuire a poporului ?i fiii lui Aaron i-au adus sângele ?i el a stropit cu el jertfelnicul de jur împrejur.
Lev 9:19  I-au adus apoi grasimea boului ?i a berbecului, coada ?i grasimea cea de pe maruntaie, rarunchii cu grasimea lor ?i grasimea de pe ficat;
Lev 9:20  ?i a pus grasimea pe pieptul jertfei de izbavire, apoi a ars grasimea pe jertfelnic;
Lev 9:21  Iar pieptul ?i spata dreapta le-a adus Aaron leganându-le înaintea fe?ei Domnului, cum poruncise Moise.
Lev 9:22  ?i ?i-a ridicat Aaron mâinile sale asupra poporului ?i l-a binecuvântat, iar dupa ce a savâr?it jertfa pentru pacat, arderea de tot ?i jertfa de mântuire, s-a coborât.
Lev 9:23  Apoi au intrat Moise ?i Aaron în cortul adunarii ?i când au ie?it au binecuvântat tot poporul; atunci s-a aratat slava Domnului la tot poporul.
Lev 9:24  ?i ie?ind foc de la Domnul, a mistuit pe jertfelnic arderea de tot ?i grasimea. ?i vazând tot poporul a scos strigate de bucurie ?i a cazut cu fa?a la pamânt.
Lev 10:1  În vremea aceea cei doi fii ai lui Aaron, Nadab ?i Abiud, luându-?i fiecare cadelni?a sa, au pus în ea foc, au turnat deasupra tamâie ?i au adus înaintea Domnului foc strain, ce nu le poruncise Domnul.
Lev 10:2  Atunci a ie?it foc de la Domnul ?i i-a mistuit ?i au murit amândoi înaintea Domnului.
Lev 10:3  Iar Moise a zis catre Aaron: "Iata ce a voit sa spuna Domnul când a zis: Voi sa fiu sfin?it prin cei ce se vor apropia de Mine ?i înaintea adunarii a tot poporul preaslavit". Iar Aaron tacea.
Lev 10:4  Atunci a chemat Moise pe Misail ?i El?afan, fiii lui Uziel, unchiul lui Aaron, ?i le-a zis: "Duce?i-va de scoate?i pe fra?ii vo?tri din loca?ul cel sfânt ?i-i duce?i afara din tabara!"
Lev 10:5  ?i ace?tia s-au dus ?i i-au scos în hitoanele lor afara din tabara, cum zisese Moise.
Lev 10:6  Iar lui Aaron ?i fiilor lui, Eleazar ?i Itamar, le-a zis Moise; "Capetele voastre sa nu vi le descoperi?i ?i ve?mintele voastre sa nu vi le sfâ?ia?i, ca sa nu muri?i ?i ca sa nu atrage?i mânia asupra ob?tii întregi. Dar fra?ii vo?tri, toata casa lui Israel, pot sa plânga pe cei ar?i, pe care i-a ars Domnul.
Lev 10:7  Din u?a cortului adunarii sa nu ie?i?i, ca sa nu muri?i, caci ave?i pe voi mirul de ungere al Domnului!" ?i s-a facut drept cuvântul lui Moise.
Lev 10:8  Apoi graind Domnul cu Aaron, a zis:
Lev 10:9  "Vin ?i sichera sa nu be?i, nici tu, nici fiii tai, când intra?i în cortul adunarii sau va apropia?i de jertfelnic, ca sa nu muri?i. Acesta este a?ezamânt ve?nic în neamul vostru.
Lev 10:10  Ca sa pute?i deosebi cele sfinte de cele nesfinte ?i cele necurate de cele curate,
Lev 10:11  ?i ca sa înva?a?i pe fiii lui Israel toate legile, pe care le-a poruncit lor Domnul prin Moise".
Lev 10:12  Iar Moise a zis catre Aaron ?i catre fiii sai, Eleazar ?i Itamar, care-i mai ramasesera: "Lua?i prinosul de pâine, ce a ramas din jertfele Domnului, ?i-l mânca?i nedospit, lânga jertfelnic, ca acesta este sfin?enie mare.
Lev 10:13  Sa-l mânca?i însa în locul cel sfânt, ca aceasta este partea ta ?i partea fiilor tai din jertfele Domnului; a?a mi s-a poruncit mie de la Domnul.
Lev 10:14  Iar pieptul leganat ?i spata ridicata sa le mânca?i la loc curat, tu ?i fiii tai ?i casa ta împreuna cu tine, ca acestea sunt date sa fie partea ta ?i partea fiilor tai din jertfele de izbavire ale fiilor lui Israel.
Lev 10:15  Spata ridicata ?i pieptul leganat sa le aduca ei cu grasime pentru ardere, leganându-le înaintea fe?ei Domnului, ?i sa fie acestea parte ve?nica pentru tine ?i împreuna cu tine ?i pentru fiii tai ?i pentru fiicele tale, cum a poruncit Domnul lui Moise".
Lev 10:16  ?i a cautat Moise ?apul de jertfa pentru pacat ?i iata era ars. ?i s-a mâniat Moise pe Eleazar ?i pe Itamar, fiii lui Aaron, care mai ramasesera,
Lev 10:17  ?i a zis: "Pentru ce n-an mâncat jertfa pentru pacat în locul cel sfânt? Ca aceasta este sfin?enie mare ?i va e data voua, ca sa ridica?i pacatele ob?tii ?i s-o cura?i?i înaintea Domnului.
Lev 10:18  Iata sângele ei nu s-a dus înauntrul loca?ului sfânt ?i voi trebuia s-o mânca?i în locul cel sfânt, cum mi s-a poruncit mie de la Domnul".
Lev 10:19  Aaron însa a zis catre Moise: "Iata, astazi ?i-au adus ei jertfa lor pentru pacat ?i arderea lor de tot înaintea Domnului ?i, dupa cele ce mi s-au întâmplat, de a? fi mâncat astazi jertfa pentru pacat, oare ar fi fost aceasta placut Domnului?"
Lev 10:20  ?i a auzit acestea Moise ?i le-a socotit raspunsul îndrepta?it.
Lev 11:1  În vremea aceea a grait Domnul cu Moise ?i cu Aaron ?i a zis:
Lev 11:2  "Grai?i fiilor lui Israel ?i le zice?i: Iata animalele pe care le pute?i mânca din toate dobitoacele de pe pamânt:
Lev 11:3  Orice animal cu copita despicata, care are copita despar?ita în doua ?i î?i rumega mâncarea, îl pute?i mânca.
Lev 11:4  Dar ?i din cele ce-?i rumega mâncarea, sau î?i au copita despicata sau împar?ita în doua, nu ve?i mânca pe acestea: camila, pentru ca aceasta-?i rumega mâncarea, dar copita n-o are despicata; aceasta e necurata pentru voi.
Lev 11:5  Iepurele de casa î?i rumega mâncarea, dar laba n-o are despicata; acesta este necurat pentru voi.
Lev 11:6  Iepurele de câmp î?i rumega mâncarea, dar laba n-o are despicata; acesta este necurat pentru voi.
Lev 11:7  Porcul are copita despicata ?i despar?ita în doua, dar nu rumega; acesta este necurat pentru voi.
Lev 11:8  Din carnea acestora sa nu mânca?i ?i de stârvurile lor sa nu va atinge?i, ca acestea sunt necurate pentru voi.
Lev 11:9  Din toate vie?uitoarele, care sunt în apa, ve?i mânca pe acestea: toate câte sunt în ape; în mari, în râuri ?i în bal?i, ?i au aripi ?i solzi, pe acelea sa le mânca?i.
Lev 11:10  Iar toate câte sunt în ape, în mari, în râuri, ?i în bal?i, toate cele ce mi?una în ape, dar n-au aripi ?i solzi, spurcaciune sunt pentru voi.
Lev 11:11  De acestea sa va îngre?o?a?i, carnea lor sa n-o mânca?i ?i de stârvurile lor sa va îngre?o?a?i.
Lev 11:12  Toate vieta?ile din ape, care n-au aripi ?i solzi, sunt spurcate pentru voi.
Lev 11:13  Din pasari sa nu mânca?i ?i sa va îngre?o?a?i de acestea, ca sunt spurcate: vulturul, zgrip?orul ?i vulturul de mare;
Lev 11:14  Corbul ?i ?oimul cu soiurile lor;
Lev 11:15  Toata cioara cu soiurile ei;
Lev 11:16  Stru?ul, cucuveaua, rândunica ?i uliul cu soiurile lui;
Lev 11:17  Huhurezul, pescarul ?i ibisul;
Lev 11:18  Lebada, pelicanul ?i cocorul;
Lev 11:19  Cocostârcul, bâtlanul cu soiurile lui; pupaza ?i liliacul.
Lev 11:20  Toate insectele înaripate, care umbla pe patru picioare, sunt spurcate pentru voi.
Lev 11:21  Dar din toate insectele înaripate, care umbla în patru picioare, sa mânca?i numai pe acelea care au fluierele picioarelor de dinapoi mai lungi, ca sa poata sari pe pamânt.
Lev 11:22  Din acestea sa mânca?i urmatoarele: lacusta ?i soiurile ei, solamul ?i soiurile lui, hargolul ?i soiurile lui, ?i hagabul cu soiurile lui.
Lev 11:23  Orice alta insecta înaripata care are patru picioare e spurcata pentru voi ?i va spurca?i de ele.
Lev 11:24  Tot cel ce se va atinge de trupul lor necurat va fi pâna seara;
Lev 11:25  ?i tot cel ce va lua în mâini trupul lor sa-?i spele haina ?i necurat va fi pâna seara.
Lev 11:26  Tot dobitocul cu copita despicata, care n-are copita despar?ita adânc sau nu-?i rumega mâncarea, este necurat pentru voi; tot cel ce se va atinge de el necurat va fi pâna seara.
Lev 11:27  Din toate fiarele cu patru picioare, cele care calca pe labe sunt necurate pentru voi ?i tot cel ce se va atinge de stârvul lor necurat va fi pâna seara.
Lev 11:28  Cel ce va umbla cu stârvul lor sa-?i spele haina ?i necurat va fi pâna seara, caci ele sunt necurate pentru voi.
Lev 11:29  Din dobitoacele ce mi?una pe pamânt, iata care sunt necurate pentru voi: cârti?a, ?oarecele ?i ?opârla, cu soiurile lor;
Lev 11:30  Ariciul, crocodilul, salamandra, melcul ?i cameleonul.
Lev 11:31  Acestea dintre toate cele ce mi?una pe pamânt sunt necurate pentru voi. Tot cel ce se atinge de stârvurile lor necurat va fi pâna seara.
Lev 11:32  Tot lucrul, pe care va cadea vreuna din acestea, moarta, fie vas de lemn, sau haina, sau piele, sau orice fel de lucru ce se întrebuin?eaza la ceva, lucrul acela necurat va fi; sa-l pune?i în apa ?i va fi necurat pâna seara, iar apoi va fi curat.
Lev 11:33  Tot vasul de lut, în care va cadea vreuna din ele, sa-l sparge?i, iar cele din el sunt necurate.
Lev 11:34  Orice lucru de mâncare, peste care "a cadea apa din acel vas, necurat va fi pentru voi ?i toata bautura de baut, din asemenea vas, necurata va fi.
Lev 11:35  Tot lucrul, peste care va cadea ceva din trupul mort al acestora, se va spurca; soba ?i caminul sa le strica?i, ca necurate sunt ?i necurate vor fi pentru voi.
Lev 11:36  Numai izvorul, fântâna ?i adunarile de apa vor ramâne curate, iar cel ce se va atinge de mortaciunile din ele, acela necurat va fi.
Lev 11:37  De va cadea ceva din trupul acestora pe samân?a de semanat, aceasta curata va fi.
Lev 11:38  Daca însa va cadea ceva din trupul lor peste samân?a, dupa ce aceasta s-a muiat cu apa, atunci samân?a necurata sa fie pentru voi.
Lev 11:39  Iar de va muri vreun dobitoc din cele ce se manânca ?i se va atinge cineva de stârvul lui, acela necurat va fi pâna seara;
Lev 11:40  Iar cel ce va mânca mortaciunea lui, sa-?i spele hainele sale ?i necurat va fi pâna seara; cel ce va duce stârvul lui sa-?i spele hainele sale ?i necurat va fi pâna seara.
Lev 11:41  Toata vietatea ce se târa?te pe pamânt este spurcata pentru voi; sa n-o mânca?i.
Lev 11:42  Tot ce se târa?te pe pântece ?i tot ce umbla în patru picioare ?i cele cu picioare multe dintre vieta?ile ce se târasc pe pamânt, sa nu le mânca?i, ca sunt spurcate pentru voi.
Lev 11:43  Sa nu va spurca?i sufletele voastre cu vreo vietate târâtoare ?i sa nu va pângari?i cu ea, ca sa fi?i din pricina ei necura?i,
Lev 11:44  Ca Eu sunt Domnul Dumnezeul vostru. Sfin?i?i-va ?i ve?i fi sfin?i, ca Eu, Domnul Dumnezeul vostru, sfânt sunt; sa nu va pângari?i sufletele voastre cu vreo vietate din cele ce se târasc pe pamânt,
Lev 11:45  Ca Eu sunt Domnul, Cel ce v-am scos din pamântul Egiptului, ca sa va fiu Dumnezeu. Deci fi?i sfin?i, ca Eu, Domnul, sunt sfânt".
Lev 11:46  Aceasta este legea cea pentru dobitoace, pentru pasari, pentru toate vieta?ile ce mi?una în apa ?i pentru toate vieta?ile ce traiesc pe pamânt,
Lev 11:47  Dupa care se pot deosebi cele necurate de cele curate ?i vieta?ile ce se manânca de vietatile ce nu se manânca.
Lev 12:1  ?i a grait Domnul lui Moise ?i a zis:
Lev 12:2  "Graie?te fiilor lui Israel ?i le zi: Daca femeia va zamisli ?i va na?te prunc de parte barbateasca, necurata va fi ?apte zile, cum e necurata ?i în zilele regulei ei.
Lev 12:3  Iar în ziua a opta se va taia pruncul împrejur.
Lev 12:4  Femeia sa mai ?ada treizeci ?i trei de zile ?i sa se cura?e de sângele sau; de nimic sfânt sa nu se atinga, ?i la loca?ul sfânt sa nu mearga, pâna se vor împlini zilele cura?irii ei.
Lev 12:5  Iar de va na?te fata, necurata va fi doua saptamâni, ca ?i în timpul regulei ei; apoi sa mai stea ?aizeci ?i ?ase de zile pentru a se cura?i de sângele sau.
Lev 12:6  Dupa ce se vor împlini zilele cura?irii ei pentru fiu sau pentru fiica, sa aduca preotului la u?a cortului un miel de un an ardere de tot ?i un pui de porumbel sau o turturica, jertfa pentru pacat;
Lev 12:7  Preotul va înfa?i?a acestea înaintea Domnului ?i o va cura?i ?i curata va fi de curgerea sângelui ei. Aceasta e rânduiala pentru ceea ce a nascut prunc de parte barbateasca sau de parte femeiasca.
Lev 12:8  Iar de nu-i va da mâna sa aduca un miel, sa ia doua turturele sau doi pui de porumbel, unul pentru ardere de tot ?i altul jertfa pentru pacat, ?i o va cura?i preotul ?i curata va fi".
Lev 13:1  Grait-a Domnul cu Moise ?i cu Aaron ?i le-a zis:
Lev 13:2  "De se va ivi la vreun om pe pielea trupului lui vreo umflatura, sau buba, sau ba?ica, sau de se va face pe pielea trupului o rana ca de lepra, sa fie adus la Aaron preotul sau la un preot din fiii lui.
Lev 13:3  Preotul va cerceta rana de pe pielea trupului lui ?i de va vedea ca perii de pe rana s-au facut albi ?i ca rana s-a adâncit în pielea trupului, aceea este rana de lepra, iar preotul dupa ce l-a cercetat, îl va declara necurat.
Lev 13:4  Iar daca pata de pe piele, de?i este alba, dar nu este ?i adâncita în pielea lui, ?i perii de pe ea nu s-au facut albi, ci sunt negri, sa închida preotul pe cel cu rana ?apte zile.
Lev 13:5  În ziua a ?aptea sa vada preotul rana: daca rana a ramas ca înainte ?i nu s-a întins rana pe piele, preotul sa-l închida alte ?apte zile.
Lev 13:6  În ziua a ?aptea îl va cerceta preotul din nou ?i daca rana va fi slabita ?i nu se va fi întins rana pe piele, preotul sa-l declare curat. Aceasta este o buba ?i cel ce o are sa-?i spele hainele sale ?i va fi curat.
Lev 13:7  Iar daca, dupa ce omul s-a aratat preotului, din nou buba a început a se întinde pe piele, sa se arate iar preotului;
Lev 13:8  Preotul, vazând ca buba se întinde pe piele, îl va declara necurat, ca aceasta este lepra.
Lev 13:9  De se va ivi pe un om boala leprei, acela sa fie adus la preot.
Lev 13:10  Preotul va cerceta ?i, daca umflatura de pe piele va fi alba ?i parul va fi schimbat în alb ?i daca umflatura va fi carne vie,
Lev 13:11  Aceea e lepra învechita pe pielea trupului; preotul il va declara necurat ?i nu-l va închide, ca este necurat.
Lev 13:12  Daca însa lepra va înflori pe piele ?i daca va acoperi lepra toata pielea bolnavului de la cap pâna la picioare, cât poate sa vada preotul cu ochii,
Lev 13:13  ?i daca va vedea preotul ca lepra a acoperit toata pielea trupului, atunci va declara pe bolnav curat, pentru ca tot s-a schimbat în alb ?i deci este curat.
Lev 13:14  Iar în ziua când se va ivi pe el carne vie, va fi necurat,
Lev 13:15  ?i preotul, vazând carnea vie, îl va declara necurat, caci carnea cea vie este necurata, este lepra.
Lev 13:16  Iar daca se va schimba carnea cea vie ?i se va face alba, sa vina bolnavul la preot,
Lev 13:17  ?i preotul sa-l cerceteze ?i daca rana s-a schimbat în alb, atunci preotul sa-l declare curat, ca e curat.
Lev 13:18  Daca cineva a avut pe pielea trupului o buba ?i s-a vindecat,
Lev 13:19  ?i pe locul bubei s-a ivit o umflatura alba sau o pata alba-ro?iatica, sa se arate preotului.
Lev 13:20  ?i preotul sa-l cerceteze ?i de se va vedea ca umflatura s-a adâncit în piele ?i parul de pe ea s-a schimbat în alb, preotul îl va declara necurat, ca aceasta e lepra ?i s-a ivit în locul bubei.
Lev 13:21  Daca însa preotul va vedea ca parul de pe umflatura nu este alb ?i ea nu este adâncita în pielea trupului ?i e negricioasa, atunci preotul va închide pe bolnav pentru ?apte zile.
Lev 13:22  Daca rana va începe a se la?i tare pe piele, preotul îl va declara necurat, ca este rana de lepra.
Lev 13:23  Iar daca pata va ramâne pe loc ?i nu se va la?i, atunci e o oprire în loc a bubei ?i preotul va declara pe bolnav curat.
Lev 13:24  Daca cineva va avea pe pielea trupului o arsura ?i pe locul tamaduit de arsura se va ivi o pata ro?iatica-albicioasa,
Lev 13:25  ?i daca preotul va vedea ca parul de pe acea pata s-a schimbat în alb ?i ca pata e adâncita sub piele, aceea este lepra ?i s-a ivit pe arsura; preotul va declara pe bolnav necurat, caci e boala leprei.
Lev 13:26  Daca însa preotul va vedea ca parul de pe pata nu este alb ?i ca ea nu este adâncita sub piele ?i ca este negricioasa, preotul va închide pe acela pentru ?apte zile;
Lev 13:27  ?i în ziua a ?aptea preotul îl va cerceta iar ?i, daca pata s-a la?it tare pe piele, preotul îl va declara necurat, ca aceea este rana de lepra.
Lev 13:28  Iar daca pata sta pe loc ?i este negricioasa, aceea este obrinteala arsurii ?i preotul va declara pe om curat, ca este obrinteala a arsurii.
Lev 13:29  Daca un barbat sau o femeie va avea o pata pe cap sau pe barbie,
Lev 13:30  ?i, cercetând-o preotul, se va vedea ca ea este adâncita sub piele ?i ca parul de pe ea este galbui ?i sub?ire, preotul va declara pe unul ca acela necurat, ca aceea este chelbe, lepra în cap, sau lepra în barba.
Lev 13:31  Daca însa preotul, la cercetarea petei de chelbe, va vedea ca ea nu este adâncita sub piele ?i ca parul de pe ea nu este galbui, preotul va închide pe cel cu pata de chelbe ?apte zile;
Lev 13:32  în ziua a ?aptea preotul va cerceta pata iar ?i, daca chelbea nu s-a întins ?i n-are parul de pe ea galbui ?i nici nu s-a adâncit chelbea sub piele,
Lev 13:33  Atunci sa rada pielea, dar locul cu chelbe sa nu-l rada, ?i preotul sa închida pe cel cu pata a doua oara pentru ?apte zile.
Lev 13:34  În ziua a ?aptea preotul va cerceta din nou chelbea ?i, daca chelbea nu se va fi întins pe piele ?i nu se va fi adâncit în piele, preotul va declara pe acela curat ?i acela sa-?i spele hainele sale ?i va fi curat.
Lev 13:35  Iar daca, dupa aceasta cura?ire a lui, chelbea va începe a se la?i foarte tare pe piele,
Lev 13:36  ?i daca preotul va vedea ca chelbea se la?e?te pe piele, atunci preotul sa nu mai caute de e parul galbui, ca acela este necurat.
Lev 13:37  Daca însa chelbea sta pe loc ?i se ive?te pe ea par negru, atunci chelbea a trecut, omul e curat ?i preotul îl va declara curat.
Lev 13:38  Daca un barbat sau o femeie va avea pe pielea trupului pete, pete albe,
Lev 13:39  ?i daca preotul va vedea ca pe pielea trupului aceluia petele sunt albe-vinete, aceea e pecingine care a înflorit pe piele ?i omul ce o are este curat.
Lev 13:40  Daca cuiva i-a cazut parul de pe cap, aceea e ple?uvie ?i omul este curat.
Lev 13:41  Daca cuiva i-a cazut parul din partea de dinainte a capului, aceea este jumatate de ple?uvie ?i omul e curat.
Lev 13:42  Iar daca pe ple?uvia din partea de dinainte sau de dinapoi va fi pata alba sau ro?iatica, atunci pe ple?uvia lui a înflorit lepra.
Lev 13:43  Preotul îl va cerceta ?i de va vedea ca fa?a umflaturii de pe ple?uvia lui este alba sau ro?iatica, semanând cu lepra, ce de obicei se ive?te pe pielea trupului,
Lev 13:44  Acela este om lepros ?i este necurat; preotul sa-l declare necurat, ca pe capul lui este boala leprei.
Lev 13:45  Leprosul, cel ce are aceasta boala, sa fie cu hainele sfâ?iate, cu capul descoperit, învelit pâna la buze, ?i sa strige mereu: necurat! necurat!
Lev 13:46  Tot timpul cât va avea pe el boala, sa fie spurcat, ca necurat este; ?i sa traiasca singuratic ?i afara din tabara sa fie locuin?a lui.
Lev 13:47  Daca boala leprei va fi pe haina, fie pe haina de lâna, sau pe haina de in,
Lev 13:48  Sau pe urzeala, sau pe batatura de in sau de lâna, sau pe piele sau pe vreun lucru de piele,
Lev 13:49  ?i daca va fi pata verzuie sau ro?iatica pe haina sau pe piele, pe batatura sau pe urzeala, sau pe vreun lucru de piele, aceea este boala leprei, ?i el se va arata preotului.
Lev 13:50  Preotul va cerceta boala ?i va închide lucrul atins de boala pentru ?apte zile;
Lev 13:51  În ziua a ?aptea va cerceta preotul lucrul atins de boala ?i daca boala se va fi întins pe haina, sau pe urzeala, sau pe batatura, sau pe piele, sau pe vreun lucru de piele, aceasta este lepra rozatoare, ?i e necurat;
Lev 13:52  ?i el sa arda haina aceea, sau urzeala, sau batatura cea de lâna sau de in, sau orice fel de lucru din piele, pe care va fi boala, ca aceea este lepra rozatoare ?i sa se arda cu foc.
Lev 13:53  Iar daca preotul va vedea ca boala nu s-a întins pe haina, sau pe urzeala, sau pe batatura, sau pe orice fel de lucru din piele,
Lev 13:54  Atunci preotul va porunci sa se spele lucrul pe care s-a ivit boala ?i-l va închide a doua oara pentru ?apte zile.
Lev 13:55  Daca, dupa spalarea lucrului atins, preotul va vedea ca boala nu ?i-a schimbat starea sa, dar s-a întins, atunci este necurat ?i lucrul sa-l arzi în foc, caci lepra a ros fa?a sau dosul.
Lev 13:56  Daca însa preotul va vedea ca pata, dupa spalarea ei, s-a mic?orat, atunci preotul s-o rupa de la haina, sau din piele, sau din urzeala, sau din batatura.
Lev 13:57  Iar daca se va ivi iar pe haina, sau pe batatura, sau pe urzeala, sau pe vreun lucru de piele, aceea este lepra înflorita ?i sa se arda cu foc lucrul pe care s-a ivit boala.
Lev 13:58  Daca însa haina, sau urzeala, sau batatura, sau lucrul de piele îl vei spala ?i se va duce pata de pe el, trebuie sa se spele a doua oara ?i va fi curat.
Lev 13:59  Aceasta este rânduiala pentru boala leprei, ce se va ivi pe haina de lâna sau de in, sau pe urzeala, sau pe batatura, sau pe vreun lucru de piele, ?i cum trebuie hotarât ca acestea sunt curate sau necurate".
Lev 14:1  ?i graind cu Moise, Domnul a zis:
Lev 14:2  "Iata rânduiala pentru cel lepros: Când el se va cura?i, se va duce la preot;
Lev 14:3  Iar preotul va ie?i afara din tabara ?i de va vedea preotul ca leprosul s-a vindecat de boala leprei,
Lev 14:4  Va porunci preotul sa se ia pentru cel cura?it doua pasari vii, curate, lemn de cedru, a?a ro?ie rasucita ?i isop.
Lev 14:5  Dupa aceea preotul va porunci sa se junghie una din pasari deasupra unui vas de lut, la apa curgatoare;
Lev 14:6  Va lua apoi pasarea cea vie, lemnul de cedru, a?a cea ro?ie ?i isopul ?i le va muie pe acestea ?i pasarea cea vie în sângele pasarii junghiate la apa curgatoare;
Lev 14:7  Va stropi de ?apte ori pe cel ce se cura?a de lepra ?i va fi curat; apoi va da drumul pasarii celei vii în câmp.
Lev 14:8  Iar cel cura?it sa-?i spele hainele sale, sa-?i tunda tot parul sau, sa se spele cu apa ?i va fi curat. Apoi sa intre în tabara ?i sa stea ?apte zile afara din cortul sau.
Lev 14:9  În ziua a ?aptea sa-?i rada tot parul sau, capul ?i barba sa, sprâncenele sale, tot parul sau sa ?i-l rada, ?i sa-?i spele iara?i hainele sale ?i trupul sau sa ?i-l spele cu apa ?i va fi curat.
Lev 14:10  În ziua a opta sa ia doi berbeci de câte un an, fara meteahna ?i o oaie de un an, fara meteahna, ?i dintr-o efa de faina de grâu, împar?ita în zece, sa ia trei par?i pentru darul de pâine ?i s-o amestece cu untdelemn ?i un log (pahar) de untdelemn;
Lev 14:11  Iar preotul cel ce cura?e?te va duce pe omul ce se cura?e?te împreuna cu acestea înaintea Domnului, la u?a cortului adunarii;
Lev 14:12  Acolo va lua preotul un berbec, ce voie?te a aduce jertfa pentru vina, ?i logul de untdelemn ?i le va aduce pe acestea leganându-le înaintea Domnului.
Lev 14:13  Berbecul îl va junghia în locul acela, unde se junghie jertfele pentru pacat ?i pentru arderea de tot, la loc sfânt, ca aceasta este jertfa pentru vina ?i, ca ?i jertfa pentru pacat, este a preotului ?i este sfin?enie mare.
Lev 14:14  Apoi va lua preotul din sângele jertfei pentru vina ?i va pune preotul pe vârful urechii drepte a celui ce se cura?e?te, pe degetul cel mare de la mâna dreapta a lui ?i pe degetul cel mare de la piciorul cel drept al lui.
Lev 14:15  De asemenea va lua preotul din logul de untdelemn ?i va turna în palma sa cea stânga;
Lev 14:16  Î?i va muia preotul degetul mâinii sale drepte în untdelemnul cel din palma stânga a sa ?i va stropi de ?apte ori cu degetul sau înaintea fe?ei Domnului;
Lev 14:17  Apoi din untdelemnul ramas în palma lui va pune preotul pe vârful urechii drepte a celui ce se cura?e?te, pe degetul cel mare al mâinii lui drepte ?i pe degetul cel mare de la piciorul cel drept al lui, pe locurile unde a pus ?i sângele jertfei pentru vina;
Lev 14:18  Iar celalalt untdelemn din palma preotului îl va turna pe capul celui ce se cura?e?te ?i-l va cura?i preotul pe acesta înaintea Domnului.
Lev 14:19  Astfel va savâr?i preotul jertfa pentru pacat ?i va cura?i pe cel ce a venit sa se cure?e de necura?enia lui; dupa aceea va junghia jertfa arderii de tot;
Lev 14:20  ?i va pune preotul arderea de tot ?i darul de pâine pe jertfelnic. Astfel îl va cura?i pe el preotul ?i el va fi curat.
Lev 14:21  Daca însa acela va fi sarac ?i nu-i va da mâna, sa ia numai un berbec pentru jertfa de vina leganata pentru cura?irea sa, a zecea parte dintr-o efa de faina de grâu, amestecata cu untdelemn pentru darul de pâine, un log de untdelemn
Lev 14:22  ?i doua turturele sau doi pui de porumbel, cum îi va da mâna: unul jertfa pentru pacat ?i altul ardere de tot.
Lev 14:23  Îi va aduce în ziua a opta cea pentru cura?irea sa la preot, înaintea Domnului, la u?a cortului adunarii.
Lev 14:24  Iar preotul, luând berbecul de jertfa pentru vina ?i logul de untdelemn, le va aduce pe acestea leganându-le înaintea Domnului.
Lev 14:25  Apoi va junghia berbecul de jertfa pentru vina ?i va lua preotul din sângele jertfei pentru vina ?i va pune pe vârful urechii drepte a celui ce se cura?a, pe degetul cel mare de la mâna lui cea dreapta ?i pe degetul cel mare de la piciorul lui cel drept.
Lev 14:26  ?i va turna preotul untdelemn în palma sa cea stânga;
Lev 14:27  ?i cu untdelemn din palma sa cea stânga va stropi preotul de ?apte ori cu degetul mâinii sale celei drepte înaintea fe?ei Domnului;
Lev 14:28  Apoi va pune preotul untdelemn din palma sa cea stânga pe marginea urechii drepte a celui ce se cura?e?te ?i pe degetul cel mare de la mâna lui cea dreapta ?i pe degetul cel marc de la piciorul lui cel drept, pe locurile unde este pus ?i sângele jertfei pentru vina;
Lev 14:29  Iar celalalt untdelemn din palma sa cea stânga îl va turna pe capul celui ce se cura?e?te, ca sa-l cure?e înaintea Domnului.
Lev 14:30  ?i turturelele sau puii de porumbel, cum îi va fi dat mâna celui ce se cura?e?te, dupa starea lui, le va aduce:
Lev 14:31  O pasare jertfa pentru pacat ?i alta pentru ardere de tot, împreuna cu darul de pâine. ?i a?a va cura?i preotul pe cel ce se cura?e?te înaintea Domnului.
Lev 14:32  Aceasta este rânduiala pentru cel bolnav de lepra, caruia nu-i da mâna sa duca tot ce se cere pentru cura?irea sa".
Lev 14:33  ?i a grait Domnul cu Moise ?i Aaron ?i a zis:
Lev 14:34  "Când ve?i intra în pamântul Canaanului, pe care-l voi da voua de mo?tenire, ?i voi aduce boala leprei asupra caselor din pamântul mo?tenirii voastre,
Lev 14:35  Atunci cel cu casa trebuie sa se duca ?i sa spuna preotului, zicând: Pe casa mea s-a ivit, pare-mi-se, boala.
Lev 14:36  Atunci preotul va porunci sa se goleasca casa înainte de a intra preotul sa cerceteze boala, ca sa nu se faca necurate toate cele din casa; dupa aceea va veni preotul sa cerceteze casa.
Lev 14:37  ?i cercetând el boala, daca va vedea ca boala de pe pere?ii casei e în chip de gropi verzui sau ro?ietice, adâncite în perete,
Lev 14:38  Va ie?i din casa, la u?a casei, ?i va închide casa pentru ?apte zile.
Lev 14:39  în ziua a ?aptea va veni preotul iar sa cerceteze casa ?i de va vedea ca boala s-a întins pe pere?ii casei,
Lev 14:40  Preotul va porunci sa se scoata pietrele pe care este boala, sa se arunce afara din ora?, la loc necurat,
Lev 14:41  Casa sa se razuiasca toata pe dinauntru, iar razatura, ce se va razui, sa se arunce afara din ora?, la loc necurat.
Lev 14:42  Sa aduca apoi alte pietre ?i sa le puna în locul pietrelor acelora; sa ia alta tencuiala ?i casa sa se tencuiasca.
Lev 14:43  Daca boala se va ivi iar ?i va înflori pe pere?ii casei, dupa ce s-au scos pietrele ?i s-a razuit ?i s-a tencuit casa,
Lev 14:44  Atunci preotul va veni iar ?i va cerceta ?i de s-a raspândit boala pe pere?ii casei, aceea este lepra rozatoare ?i casa este necurata.
Lev 14:45  Casa aceea sa se darâme, iar pietrele ei, lemnul ei ?i toata tencuiala sa se scoata afara din ora?, la loc necurat.
Lev 14:46  Cel ce va intra în casa aceea, cât va fi ea închisa, acela necurat va fi pâna seara.
Lev 14:47  Cel ce va dormi în casa aceea sa-?i spele hainele sale ?i necurat va fi pâna seara; ?i cel ce va mânca în casa aceea sa-?i spele hainele ?i necurat va fi pâna seara.
Lev 14:48  Daca însa preotul, venind ?i intrând, va vedea ca boala de pe pere?ii casei nu s-a mai întins dupa ce aceasta a fost tencuita din nou, preotul o va declara curata, ca boala a trecut.
Lev 14:49  Ca sa cure?e casa, va lua deci doua pasari vii, curate, lemn de cedru, a?a ro?ie rasucita ?i isop;
Lev 14:50  Va junghia o pasare deasupra unui vas de lut, la apa curgatoare.
Lev 14:51  Va lua lemnul cel de cedru, a?a, isopul ?i pasarea vie ?i le va muia în sângele pasarii junghiate ?i în apa de izvor ?i va stropi casa de ?apte ori.
Lev 14:52  ?i va cura?i astfel casa cu sângele pasarii, cu apa de izvor, cu pasarea cea vie, cu lemnul cel de cedru, cu a?a ro?ie rasucita ?i cu isop.
Lev 14:53  Iar pasarii celei vii îi va da drumul din cetate în câmp ?i se va cura?i casa ?i curata va fi.
Lev 14:54  Aceasta este rânduiala pentru oricare fel de boala a leprei ?i a chelbei.
Lev 14:55  Pentru lepra de pe haine ?i de pe case,
Lev 14:56  ?i pentru umflaturi, pecingine ?i pete,
Lev 14:57  Ca sa se poata afla când acestea sunt necurate ?i când sunt curate: aceasta este rânduiala pentru lepra".
Lev 15:1  ?i a grait Domnul cu Moise ?i cu Aaron, zicând:
Lev 15:2  "Grai?i fiilor lui Israel ?i le spune?i: Daca un barbat va avea curgere din trupul sau, pentru curgerea lui este necurat,
Lev 15:3  ?i legea necura?iei lui este aceasta: Ori de se face curgere din trupul lui, ori de este împiedicata curgerea în trupul lui, el este necurat.
Lev 15:4  Tot patul, pe care doarme cel ce are curgere, este necurat; tot lucrul, pe care va ?edea cel ce are curgere, este necurat.
Lev 15:5  Omul, care se va atinge de patul lui, sa-?i spele hainele sale, sa se spele cu apa ?i va fi necurat pâna seara.
Lev 15:6  Cel ce va ?edea pe vreun lucru, pe care a ?ezut cel ce are curgere, sa-?i spele hainele sale, sa se spele cu apa ?i va fi necurat pâna seara.
Lev 15:7  Cel ce se va atinge de trupul celui ce are curgere sa-?i spele hainele sale, sa se spele cu apa ?i necurat va fi pâna seara.
Lev 15:8  Daca cel ce are curgere va scuipa pe unul curat, acesta sa-?i spele hainele, sa se spele cu apa ?i necurat va fi pâna seara.
Lev 15:9  Toata ?aua, pe care va calari cel ce are curgere, necurata va fi pâna seara.
Lev 15:10  Tot cel ce se atinge de câte au fost sub acela va fi necurat pâna seara, iar cel ce va ridica acestea sa-?i spele hainele sale, sa se spele cu apa ?i necurat va fi pâna seara.
Lev 15:11  Acela, de care se va atinge cel ce are curgere, fara sa-?i fi spalat mâinile cu apa, sa-?i spele hainele sale, sa se spele cu apa ?i va fi necurat pâna seara.
Lev 15:12  Vasul de lut, de care s-a atins cel ce are curgere, sa se sparga ?i tot vasul de lemn sa se spele cu apa ?i va fi curat.
Lev 15:13  Iar când cel ce are curgere se va cura?i de curgerea sa sa numere ?apte zile pentru cura?irea sa, sa-?i spele hainele sale, sa-?i spele trupul cu apa de izvor ?i va fi curat.
Lev 15:14  Apoi în ziua a opta sa-?i ia doua turturele sau doi pui de porumbel, sa vina înaintea fe?ei Domnului, la u?a cortului adunarii ?i sa le dea preotului;
Lev 15:15  Iar preotul sa aduca din ele: una jertfa pentru pacat ?i una ardere de tot; ?i sa-l cure?e preotul înaintea Domnului de curgerea lui.
Lev 15:16  Daca un om va avea din întâmplare curgerea semin?ei, acela sa-?i spele cu apa tot trupul sau ?i va fi necurat pâna seara.
Lev 15:17  Orice haina ?i orice piele, pe care va cadea samân?a, sa se spele cu apa ?i necurata va fi pâna seara.
Lev 15:18  Daca barbatul se va culca cu femeia ?i va avea el curgerea semin?ei, sa se spele amândoi cu apa ?i necura?i sa fie pâna seara.
Lev 15:19  De va avea femeia curgere de sânge, care curge din trupul sau, trebuie sa stea ?apte zile pentru cura?irea sa. Tot cel ce se va atinge de ea, necurat va fi pâna seara.
Lev 15:20  Tot lucrul pe care se va culca ea în timpul necura?iei va fi necurat ?i tot lucrul pe care va ?edea va fi necurat.
Lev 15:21  Tot cel ce se va atinge de patul ei sa-?i spele hainele sale, sa se spele cu apa ?i necurat va fi pâna seara.
Lev 15:22  Tot cel ce se va atinge de vreun lucru, pe care a ?ezut ea, sa-?i spele hainele sale, sa se spele cu apa ?i necurat va fi pâna seara.
Lev 15:23  Iar de se va atinge cineva de ceva din patul ei sau de lucrul pe care a ?ezut ea, acela necurat va fi pâna seara.
Lev 15:24  De va dormi ea cu barbatul, necura?ia ei va fi ?i pe el ?i necurat va fi el ?apte zile, iar tot patul, în care va dormi, necurat va fi.
Lev 15:25  Daca femeii îi va curge sânge mai multe zile ?i nu în timpul regulii ei, sau daca ea are curgere mai mult decât timpul regulii ei obi?nuite, atunci în tot timpul curgerii necura?iei ei va fi necurata, ca ?i în timpul regulii ei.
Lev 15:26  Tot patul, în care va dormi în timpul curgerii ei, va fi necurat, cum e patul ?i în timpul regulii ei, ?i tot lucrul pe care va ?edea ea va fi necurat, cum e necurat în timpul regulii ei.
Lev 15:27  Tot cel ce se va atinge de acel lucru va fi necurat: sa-?i spele hainele sale, sa-?i spele trupul cu apa ?i va fi necurat pâna seara.
Lev 15:28  Iar când se va izbavi ea de curgerea sa, sa se cure?e ?apte zile ?i dupa aceea va fi curata.
Lev 15:29  În ziua a opta sa-?i ia doua turturele sau doi pui de porumbei ?i sa-i aduca preotului, la u?a cortului adunarii
Lev 15:30  Iar preotul va aduce una din pasari jertfa pentru pacat ?i pe cealalta ardere de tot; ?i s-o cure?e preotul înaintea Domnului de curgerea ei cea necurata.
Lev 15:31  A?a sa feri?i pe fiii lui Israel de necura?enia lor, ca sa nu moara ei în necura?enia lor, spurcându-Mi loca?ul Meu cel din mijlocul vostru.
Lev 15:32  Aceasta este rânduiala pentru cel ce are curgere ?i pentru cel ce i se va întâmpla pierderea semin?ei, care-l face necurat,
Lev 15:33  ?i pentru ceea ce sufera de regula sa ?i pentru cei ce au curgere, barbat sau femeie, ?i pentru barbatul ce doarme cu femeie necurata".
Lev 16:1  Dupa moartea celor doi fii ai lui Aaron, care au murit când au adus foc strain înaintea fe?ei Domnului, a grait Domnul cu Moise;
Lev 16:2  ?i a zis Domnul catre Moise: "Spune lui Aaron, fratele tau, sa nu intre oricând în loca?ul sfânt de dupa perdea, înaintea cura?itorului celui de pe chivotul legii, ca sa nu moara, ca deasupra capacului Ma voi arata în nor.
Lev 16:3  Iata rânduiala dupa care trebuie sa intre Aaron în loca?ul sfânt: cu un vi?el, jertfa pentru pacat, ?i cu un berbec pentru ardere de tot.
Lev 16:4  Sa se îmbrace cu hitonul de in sfin?it, sa aiba pe trupul lui pantaloni de in, sa fie încins cu brâu de in ?i sa-?i ia ?i chidar de in: acestea sunt ve?mintele sfin?ite; dar sa-?i spele tot trupul sau cu apa ?i numai a?a sa se îmbrace cu ele.
Lev 16:5  Iar de la ob?tea fiilor lui Israel sa ia din turma lor doi ?api de jertfa pentru pacat ?i un berbec pentru arderea de tot.
Lev 16:6  Sa aduca Aaron. pentru sine vi?elul de jertfa pentru pacat, ca sa se cure?e pe sine ?i casa sa.
Lev 16:7  Apoi sa ia cei doi ?api ?i sa-i puna înaintea fe?ei Domnului la u?a cortului adunarii.
Lev 16:8  ?i sa arunce Aaron sor?i asupra celor doi ?api: un sor? pentru al Domnului ?i un sor? pentru al lui Azazel.
Lev 16:9  Dupa aceea sa ia Aaron ?apul, asupra caruia a cazut sor?ul Domnului, ?i sa-l aduca jertfa pentru pacat,
Lev 16:10  Iar ?apul asupra caruia a cazut sor?ul pentru Azazel sa-l puna viu înaintea Domnului, ca sa savâr?easca asupra lui cura?irea ?i sa-i dea drumul în pustie pentru ispa?ire, ca sa duca acela cu sine nelegiuirile lor în pamânt neumblat.
Lev 16:11  Apoi sa aduca Aaron vi?elul de jertfa pentru pacatele sale, ca sa se cure?e pe sine ?i casa sa ?i sa junghie vi?elul jertfa pentru pacatele sale;
Lev 16:12  Sa ia carbuni aprin?i de pe jertfelnicul cel dinaintea Domnului, o cadelni?a plina, ?i aromate pisate marunt pentru tamâiere doua mâini pline, ?i sa le duca înauntru, dupa perdea;
Lev 16:13  Sa puna aromatele pe focul din cadelni?a înaintea felei Domnului, astfel ca norul de fum sa acopere capacul cel de pe chivotul legii, ca sa nu moara Aaron.
Lev 16:14  Sa ia ?i din sângele vi?elului ?i sa stropeasca cu degetul sau spre rasarit peste capac; ?i înaintea capacului sa stropeasca de ?apte ori sânge cu degetul sau.
Lev 16:15  Dupa aceea sa junghie înaintea Domnului ?apul de jertfa pentru pacatele poporului, sa duca sângele lui înauntru, dupa perdea, ?i sa faca cu sângele acela ce a facut ?i cu sângele vi?elului, stropind cu el pe capac ?i înaintea capacului.
Lev 16:16  A?a va cura?i loca?ul sfânt de necura?ia fiilor lui Israel, de nelegiuirile lor ?i de toate pacatele lor. A?a sa faca el cu cortul adunarii, care se afla la ei, în mijlocul necura?eniilor lor.
Lev 16:17  Nici un om sa nu fie în cortul adunarii, când va intra el sa cure?e loca?ul sfânt ?i pâna va ie?i. A?a se va cura?i el pe sine ?i casa sa ?i toata ob?tea fiilor lui Israel.
Lev 16:18  Apoi va ie?i la jertfelnicul cel dinaintea Domnului ?i-l va cura?i, luând din sângele vi?elului ?i din sângele ?apului ?i punând pe coarnele jertfelnicului de jur împrejur,
Lev 16:19  ?i, stropindu-l cu sânge, cu degetul sau de ?apte ori, îl va cura?i de necura?eniile fiilor lui Israel ?i-l va sfin?i.
Lev 16:20  Iar dupa ce va sfâr?i de cura?at loca?ul sfânt, cortul adunarii ?i jertfelnicul, ?i cura?ind ?i pe preo?i, va aduce ?apul cel viu,
Lev 16:21  Î?i va pune Aaron mâinile sale pe capul ?apului celui viu ?i va marturisi asupra lui toate nelegiuirile fiilor lui Israel, toate nedrepta?ile lor ?i toate pacatele lor; ?i, punându-le pe acestea pe capul ?apului, îl va trimite cu un om anumit în pustie.
Lev 16:22  ?i va duce ?apul cu sine toate nelegiuirile lor în pamânt neumblat ?i omul va da drumul ?apului în pustie.
Lev 16:23  Dupa aceea va intra Aaron în cortul adunarii ?i se va dezbraca de hainele cele de in, cu care se îmbracase la intrarea în locul cel sfânt, ?i le va lasa acolo;
Lev 16:24  Î?i va spala trupul sau cu apa în locul cel sfânt, se va îmbraca cu hainele sale ?i, ie?ind, va savâr?i arderea de tot pentru sine ?i arderea de tot pentru popor, ?i se va cura?i astfel pe sine, casa sa, poporul ?i pe preo?i.
Lev 16:25  Iar grasimea jertfei pentru pacat o va arde pe jertfelnic.
Lev 16:26  Cel ce a dat drumul în pustie ?apului de ispa?ire sa-?i spele hainele sale, sa-?i spele trupul cu apa ?i atunci sa intre în tabara.
Lev 16:27  Iar vi?elul de jertfa pentru pacat ?i ?apul de jertfa pentru pacat al caror sânge a fost adus înauntru pentru cura?irea loca?ului sfânt, sa se scoata afara din tabara ?i sa se arda în foc pielea lor, carnea lor ?i necura?enia lor.
Lev 16:28  Cel ce le va arde sa-?i spele hainele, sa-?i spele trupul sau cu apa ?i numai dupa aceea sa intre în tabara.
Lev 16:29  Aceasta sa fie pentru voi lege ve?nica: în luna a ?aptea, în ziua a zecea a lunii, sa posti?i ?i nici o munca sa nu face?i, nici ba?tina?ul, nici strainul care s-a a?ezat la voi,
Lev 16:30  Caci în ziua aceasta vi se face cura?ire, ca sa fi?i cura?i de toate pacatele voastre, înaintea Domnului, ?i cura?i ve?i fi.
Lev 16:31  Aceasta e cea mai mare zi de odihna pentru voi ?i sa smeri?i sufletele voastre prin post. Aceasta este lege ve?nica.
Lev 16:32  De cura?it însa sa va cure?e preotul care este uns ca sa slujeasca în locul tatalui sau.
Lev 16:33  Sa se îmbrace el cu ve?mintele cele de in ?i cu ve?mintele sfinte; ?i va cura?i sfânta sfintelor, cortul adunarii, va cura?i jertfelnicul ?i pe preo?i ?i va cura?i ?i toata ob?tea poporului.
Lev 16:34  Aceasta sa fie pentru voi lege ve?nica: o data în an sa cura?i?i pe fiii lui Israel de pacatele lor". ?i Aaron a facut a?a cum poruncise Domnul lui Moise.
Lev 17:1  Grait-a Domnul cu Moise ?i a zis:
Lev 17:2  "Vorbe?te lui Aaron, fiilor lui ?i tuturor fiilor lui Israel ?i zi catre ei: Iata ce porunce?te Domnul:
Lev 17:3  Orice om dintre fiii lui Israel sau dintre strainii ce s-au lipit de voi, care va junghia bou, sau oaie, sau capra, în tabara, sau care va junghia afara din tabara,
Lev 17:4  ?i nu le va înfa?i?a la u?a cortului adunarii, ca sa le aduca ardere de tot sau jertfa de izbavire, placuta Domnului, cu miros de buna mireasma; sau care le va junghia afara din tabara ?i la u?a cortului adunarii nu le va aduce, ca sa le faca jertfa Domnului, înaintea loca?ului Domnului, omului aceluia i se va cere sângele, ca a varsat sânge ?i se va stârpi sufletul acela din poporul sau;
Lev 17:5  Pentru ca fiii lui Israel sa-?i aduca jertfele lor, câte le junghie ei în câmp, ?i sa le înfa?i?eze Domnului la u?a cortului adunarii, la preot, ?i sa le faca Domnului jertfa de mântuire.
Lev 17:6  ?i va stropi preotul cu sânge jertfelnicul împrejur, înaintea Domnului, la u?a cortului adunarii, iar grasimea o va arde spre miros de buna mireasma Domnului,
Lev 17:7  Ca sa nu-?i mai aduca ei jertfele lor la idolii dupa care umbla desfrânând. Aceasta sa fie pentru ei a?ezamânt ve?nic în neamul lor.
Lev 17:8  Sa le spui de asemenea: Daca un om dintre fiii lui Israel sau dintre fiii strainilor care locuiesc între ei va face ardere de tot sau jertfa
Lev 17:9  ?i nu o va aduce la u?a cortului adunarii, ca sa o aduca jertfa înaintea Domnului, omul acela se va stârpi din poporul sau.
Lev 17:10  Daca un om dintre fiii lui Israel ?i dintre strainii care traiesc între voi va mânca orice fel de sânge, Îmi voi întoarce fa?a Mea împotriva sufletului celui ce va mânca sânge ?i-l voi stârpi din poporul sau,
Lev 17:11  Pentru ca via?a a tot trupul este în sânge ?i pe acesta vi l-am dat pentru jertfelnic, ca sa va cura?iri sufletele voastre, ca sângele acesta cura?e?te sufletul.
Lev 17:12  De aceea am ?i zis fiilor lui Israel: Nimeni din voi sa nu manânce sânge ?i nici strainul, care locuie?te la voi, sa nu manânce sânge.
Lev 17:13  Oricine dintre fiii lui Israel ?i dintre strainii ce locuiesc la voi va vâna fiara sau pasare, care se manânca, acela sa scurga sângele ei ?i sa-l acopere cu pamânt, caci via?a oricarui trup este în sângele lui.
Lev 17:14  De aceea am zis fiilor lui Israel: Sa nu mânca?i sângele nici unui trup, pentru ca via?a oricarui trup este în sângele lui: tot cel ce-l va mânca se va stârpi,
Lev 17:15  ?i tot cel ce va mânca mortaciune sau sfâ?iat de fiara, acela, fie ba?tina? sau strain, sa-?i spele hainele, sa se spele cu apa ?i necurat va fi pâna seara, iar apoi va fi curat;
Lev 17:16  Iar de nu-?i va spala hainele sale ?i nu-?i va spala trupul sau, va purta asupra sa vina sa".
Lev 18:1  În vremea aceea a grait Domnul cu Moise, zicând:
Lev 18:2  "Vorbe?te fiilor lui Israel ?i zi catre ei: Eu sunt Domnul Dumnezeul vostru.
Lev 18:3  De datinile pamântului Egiptului, în care a?i trait, sa nu va ?ine?i, nici de datinile pamântului Canaanului, în care am sa va duc, sa nu va ?ine?i ?i nici sa umbla?i dupa obiceiurile lor.
Lev 18:4  Ci legile Mele sa le plini?i ?i a?ezamintele Mele sa le pazi?i, umblând dupa cum poruncesc ele, ca Eu sunt Domnul Dumnezeul vostru.
Lev 18:5  Pazi?i toate poruncile Mele ?i toate hotarârile sa le ?ine?i, caci omul care le pline?te va trai prin ele: Eu sunt Domnul Dumnezeul vostru.
Lev 18:6  Nimeni sa nu se apropie de nici o ruda dupa trup, cu gândul ca sa-i descopere goliciunea. Eu sunt Domnul!
Lev 18:7  Goliciunea tatalui tau ?i goliciunea mamei tale sa n-o descoperi! Ca este mama ta, sa nu-i descoperi goliciunea ei.
Lev 18:8  Goliciunea femeii tatalui tau sa n-o descoperi, ca este goliciunea tatalui tau!
Lev 18:9  Goliciunea surorii tale, goliciunea fiicei tatalui tau sau a fiicei mamei tale, care s-a nascut în casa sau afara din casa, sa n-o descoperi!
Lev 18:10  Goliciunea fiicei fiului tau sau a fiicei fiicei tale sa n-o descoperi, caci goliciunea ta este!
Lev 18:11  Goliciunea fiicei femeii tatalui tau; care s-a nascut din tatal tau, sa n-o descoperi, ca sora î?i este dupa tata!
Lev 18:12  Goliciunea surorii tatalui tau sa n-o descoperi, ca este de un sânge cu tatal tau!
Lev 18:13  Goliciunea surorii mamei tale sa n-o descoperi, ca este de un sânge cu mama ta!
Lev 18:14  Goliciunea fratelui tatalui tau sa n-o descoperi ?i de femeia lui sa nu te apropii, ca sunt unchiul ?i matu?a ta!
Lev 18:15  Goliciunea nurorii tale sa n-o descoperi, ca ea este femeia fiului tau; sa nu-i descoperi goliciunea!
Lev 18:16  Goliciunea femeii fratelui tau sa n-o descoperi, ca este goliciunea fratelui tau.
Lev 18:17  Goliciunea unei femei ?i a fiicei ei sa nu descoperi; pe fiica fiului ei ?i pe fiica fiicei ei sa nu le iei, ca sa le descoperi goliciunea; aceasta este nelegiuire, ca sunt rude de sânge cu ea!
Lev 18:18  Sa nu iei concubina pe sora femeii tale, ca sa descoperi ru?inea ei în vremea ei, vie fiind ea.
Lev 18:19  Sa nu te apropii de femeie în timpul regulii ei, ca sa-i descoperi goliciunea.
Lev 18:20  ?i cu femeia aproapelui tau sa nu te culci, ca sa-?i ver?i samân?a ?i ca sa te spurci cu ea.
Lev 18:21  Din copiii tai sa nu dai în slujba lui Moloh, ca sa nu pângare?ti numele Dumnezeului tau. Eu sunt Domnul.
Lev 18:22  Sa nu te culci cu barbat, ca ?i cu femeie; aceasta este spurcaciune.
Lev 18:23  Cu nici un dobitoc sa nu te culci, ca sa-?i ver?i samân?a ?i sa te spurci cu el; nici femeia sa nu stea la dobitoc, ca sa se spurce cu el; aceasta e urâciune.
Lev 18:24  Sa nu va întina?i cu nimic din acestea, ca cu toate acestea s-au întinat pagânii, pe care ?u îi izgonesc dinaintea fe?ei voastre.
Lev 18:25  Ca s-a întinat pamântul ?i am privit la nelegiuirile lor ?i a lepadat pamântul pe cei ce traiau pe el.
Lev 18:26  Iar voi sa pazi?i toate poruncile Mele ?i toate legile Mele ?i sa nu face?i toate ticalo?iile acestea, nici ba?tina?ul, nici strainul care traie?te între ?oi.
Lev 18:27  Ca toate urâciunile acestea le-au facut oamenii pamântului acestuia care e înaintea voastra ?i s-a întinat pamântul;
Lev 18:28  Ca nu cumva sa va lepede ?i pe voi pamântul, când îl ve?i întina, cum a aruncat el de la sine pe popoarele care au fost înainte de voi.
Lev 18:29  Ca tot cel ce va face ticalo?iile acestea, sufletul care va face acestea se va stârpi din poporul sau.
Lev 18:30  Deci pazi?i poruncile Mele ?i sa nu umbla?i dupa obiceiurile urâte, dupa care au umblat cei dinaintea voastra, nici sa va întina?i cu ele. Eu sunt Domnul Dumnezeul vostru".
Lev 19:1  Grait-a Domnul cu Moise ?i a zis:
Lev 19:2  "Vorbe?te la toata ob?tea fiilor lui Israel ?i le zi; Fi?i sfin?i, ca Eu, Domnul Dumnezeul vostru, sunt sfânt.
Lev 19:3  Sa cinsteasca fiecare pe tatal sau ?i pe mama sa ?i zilele Mele de odihna sa le pazi?i, ca Eu sunt Domnul Dumnezeul vostru.
Lev 19:4  Sa nu alerga?i la idoli ?i dumnezei turna?i sa nu va face?i, ca Eu sunt Domnul Dumnezeul vostru.
Lev 19:5  De ve?i aduce Domnului jertfa de izbavire, sa o aduce?i de bunavoie.
Lev 19:6  ?i sa o mânca?i în ziua aducerii ?i a doua zi, iar ce va ramâne pentru a treia zi sa arde?i cu foc.
Lev 19:7  Iar de va mânca cineva a treia zi, va face urâciune ?i jertfa nu va fi primita;
Lev 19:8  Cel ce va mânca va agonisi pacat, ca acela a spurcat lucrul sfânt al Domnului ?i sufletul acela se va stârpi din poporul sau.
Lev 19:9  Când ve?i secera holdele voastre în pamântul vostru, sa nu seceri tot pâna la fir în ogorul tau ?i ceea ce ramâne dupa seceri?ul tau sa nu aduni;
Lev 19:10  ?i în via ta sa nu culegi strugurii rama?i, nici boabele ce cad în via ta sa nu le aduni; lasa-le pe acestea saracului ?i strainului, ca Eu sunt Domnul Dumnezeul tau.
Lev 19:11  Sa nu fura?i, sa nu spune?i minciuna ?i sa nu în?ele nimeni pe aproapele sau.
Lev 19:12  Sa nu va jura?i strâmb pe numele Meu ?i sa nu pângari?i numele cel sfânt al Dumnezeului vostru, ca Eu sunt Domnul Dumnezeul vostru.
Lev 19:13  Sa nu nedrepta?e?ti pe aproapele ?i sa nu-l jefuie?ti. Plata simbria?ului sa nu ramâna la tine pâna a doua zi.
Lev 19:14  Sa nu graie?ti de rau pe surd ?i înaintea orbului sa nu pui piedica. Sa te temi de Domnul Dumnezeul tau. Eu sunt Domnul Dumnezeul tau.
Lev 19:15  Sa nu face?i nedreptate la judecata; sa nu cauta?i la fa?a celui sarac ?i de fa?a celui puternic sa nu te sfie?ti, ci cu dreptate sa judeci pe aproapele tau.
Lev 19:16  Sa nu umbli cu clevetiri în poporul tau ?i asupra vie?ii aproapelui tau sa nu te ridici. Eu sunt Domnul Dumnezeul vostru.
Lev 19:17  Sa nu du?mane?ti pe fratele tau în inima ta, dar sa mustri pe aproapele tau, ca sa nu por?i pacatul lui.
Lev 19:18  Sa nu te razbuni cu mina ta ?i sa nu ai ura asupra fiilor poporului tau, ci sa iube?ti pe aproapele tau ca pe tine însu?i. Eu sunt Domnul Dumnezeul vostru.
Lev 19:19  Legea Mea sa o pazi?i; vitele tale sa nu le faci sa se împreune cu alt soi; ogorul tau sa nu-l semeni deodata cu doua feluri de semin?e; cu haina ?esuta din felurite torturi, de lâna ?i de in, sa nu te îmbraci.
Lev 19:20  De va dormi cineva cu femeie, împreunându-se, ?i aceea va fi roaba, logodita cu un barbat, dar nerascumparata înca sau daca nu i s-a dat înca slobozenia, sa-i pedepsi?i pe amândoi, dar nu cu moarte, pentru ca ea nu este sloboda,
Lev 19:21  Ci sa aduca el Domnului, la u?a cortului adunarii, jertfa de vina; un berbec sa aduca jertfa pentru vina sa;
Lev 19:22  ?i preotul îl va cura?i de pacatul lui înaintea Domnului cu berbecul cel pentru vina ?i i se va ierta lui pacatul pe care l-a facut.
Lev 19:23  Când ve?i intra în pamântul, pe care Domnul Dumnezeul vostru vi-l  va da, ?i ve?i sadi orice pom roditor, sa cura?i?i necura?enia lui: trei ani sa socoti?i roadele lui ca necurate ?i sa nu le mânca?i;
Lev 19:24  Iar în anul al patrulea toate roadele lui sa fie afierosite Domnului, întru lauda Lui.
Lev 19:25  ?i în anul al cincilea sa mânca?i din roadele lui ?i sa va aduna?i toate roadele. Eu sunt Domnul Dumnezeul vostru.
Lev 19:26  Sa nu mânca?i cu sânge; sa nu vraji?i, nici sa ghici?i.
Lev 19:27  Sa nu va tunde?i rotund parul capului vostru, nici sa va strica?i fa?a barbii voastre.
Lev 19:28  Pentru mor?i sa nu va face?i taieturi pe trupurile voastre, nici semne cu împunsaturi sa nu face?i pe voi. Eu sunt Domnul Dumnezeul vostru.
Lev 19:29  Sa nu necinste?ti pe fiica ta, îngaduindu-i sa faca desfrânare, ca sa nu se desfrâneze pamântul ?i ca sa nu se umple pamântul de stricaciune.
Lev 19:30  Zilele Mele de odihna sa le paze?ti ?i loca?ul Meu sa-l cinste?ti. Eu sunt Domnul.
Lev 19:31  Sa nu alerga?i la cei ce cheama mor?ii, pe la vrajitori sa nu umbla?i ?i sa nu va întina?i cu ei. Eu sunt Domnul Dumnezeul vostru.
Lev 19:32  Înaintea celui carunt sa te scoli, sa cinste?ti fa?a batrânului ?i sa te temi de Domnul Dumnezeul tau. Eu sunt Domnul Dumnezeul vostru.
Lev 19:33  De se va a?eza vreun strain în pamântul vostru, sa nu-l strâmtora?i.
Lev 19:34  Strainul, care s-a a?ezat la voi, sa fie pentru voi ca ?i ba?tina?ul vostru; sa-l iubi?i ca pe voi în?iva, ca ?i voi a?i fost straini în pamântul Egiptului. Eu sunt Domnul Dumnezeul vostru.
Lev 19:35  Sa nu face?i nedreptate la judecata, la masura, la cântarit ?i la masuratoare.
Lev 19:36  Cântarul vostru sa fie drept, greuta?ile drepte, efa dreapta ?i hinul drept. Eu sunt Domnul Dumnezeul vostru, Care v-am scos din pamântul Egiptului.
Lev 19:37  Sa pazi?i toate legile Mele ?i toate orânduielile Mele ?i sa le plini?i. Eu sunt Domnul Dumnezeul vostru".
Lev 20:1  Grait-a Domnul cu Moise ?i a zis:
Lev 20:2  "Spune fiilor lui Israel acestea: Daca cineva dintre fiii lui Israel ?i dintre strainii care traiesc printre Israeli?i va da din copiii sai lui Moloh, acela sa fie dat mor?ii: poporul ba?tina? sa-l ucida cu pietre.
Lev 20:3  ?i Eu Îmi voi întoarce fa?a Mea împotriva omului aceluia ?i-l voi stârpi din poporul sau, pentru ca a dat din copiii sai lui Moloh, ca sa întineze loca?ul Meu cel sfânt ?i sa necinsteasca numele Meu cel sfânt.
Lev 20:4  Iar daca poporul ba?tina? î?i va închide ochii sai asupra omului aceluia, când va da din copiii sai lui Moloh,
Lev 20:5  Îmi voi întoarce fa?a Mea împotriva omului aceluia ?i împotriva neamului lui ?i-l voi stârpi din poporul sau pe el ?i pe to?i cei ce fac desfrânari asemenea lui, desfrânând dupa Moloh.
Lev 20:6  Daca vreun suflet va alerga la cei ce cheama mor?ii ?i la vrajitorii, ca sa desfrâneze în urma lor, Eu voi întoarce fa?a Mea împotriva sufletului aceluia ?i-l voi pierde din poporul lui.
Lev 20:7  Sfin?i?i-va pe voi în?iva ?i ve?i fi sfin?i, ca Eu, Domnul Dumnezeul vostru, sunt sfânt.
Lev 20:8  Pazi?i legile Mele ?i le plini?i, ca Eu sunt Domnul, Cel ce va sfin?e?te.
Lev 20:9  Cel ce va grai de rau pe tatal sau sau pe mama sa sa fie dat mor?ii, ca a grait de rau pe tatal sau ?i pe mama sa ?i sângele sau este asupra sa.
Lev 20:10  De se va desfrâna cineva cu femeie maritata, adica de se va desfrâna cu femeia aproapelui sau, sa se omoare desfrânatul ?i desfrânata.
Lev 20:11  Cel ce se va culca cu femeia tatalui sau, acela goliciunea tatalui sau a descoperit; sa se omoare amândoi, caci vinova?i sunt.
Lev 20:12  De se va culca cineva cu nora sa, amândoi sa se omoare, ca au facut urâciune ?i sângele lor este asupra lor.
Lev 20:13  De se va culca cineva cu barbat ca ?i cu femeie, amândoi au facut nelegiuire ?i sa se omoare, ca sângele lor asupra lor este.
Lev 20:14  Daca î?i va lua cineva femeie ?i se va desfrâna cu mama ei, nelegiuire face; pe foc sa se arda ?i el ?i ea, ca sa nu fie nelegiuiri între voi.
Lev 20:15  Cel ce se va amesteca. cu dobitoc sa se omoare ?i sa ucide?i dobitocul.
Lev 20:16  Daca femeia se va duce la vreun dobitoc, ca sa se uneasca cu el, sa ucizi femeia ?i dobitocul sa se omoare, ca sângele lor este asupra lor.
Lev 20:17  De va lua cineva pe sora sa, dupa tata sau dupa mama, ?i-i va vedea goliciunea ?i ea va vedea goliciunea lui: aceasta este ru?ine ?i sa fie stârpi?i înaintea ochilor fiilor poporului lor. El a descoperit goliciunea surorii sale; sa-?i poarte pacatul lor.
Lev 20:18  Barbatul care se va culca cu femeie în timpul curgerii ei ?i-i va descoperi goliciunea, acela a descoperit curgerea sângelui ei ?i ea ?i-a descoperit curgerea sângelui sau: amândoi sa fie stârpi?i din poporul lor.
Lev 20:19  Goliciunea surorii mamei tale ?i a surorii tatalui tau sa n-o descoperi, ca unul ca acela î?i dezgole?te trupul rudei sale ?i-?i vor purta pacatul amândoi.
Lev 20:20  Cel ce se va culca cu matu?a sa descopera goliciunea unchiului sau; sa-?i poarte amândoi pacatul ?i fara copii sa moara.
Lev 20:21  De va lua cineva pe femeia fratelui sau, urâciune este, ca a descoperit goliciunea fratelui sau: fara copii sa moara.
Lev 20:22  Pazi?i toate a?ezamintele Mele ?i toate hotarârile Mele ?i le plini?i ?i nu va va arunca de pe sine pamântul în care va voi duce sa trai?i.
Lev 20:23  Sa nu umbla?i dupa obiceiurile popoarelor pe care le voi alunga de la voi, ca ele au facut acestea toate ?i M-am scârbit de ele.
Lev 20:24  Eu doara v-am spus: Voi ve?i mo?teni pamântul lor ?i Eu va voi da sa mo?teni?i pamântul în care curge lapte ?i miere. Eu sunt Domnul Dumnezeul vostru, Care v-am despar?it de toate popoarele.
Lev 20:25  Sa deosebi?i dobitocul curat de cel necurat ?i pasarea curata de cea necurata; sa nu va întinaâi sufletele voastre cu dobitoc sau cu pasare, nici cu toate cele ce se târasc pe pamânt, pe care Eu le-am deosebit ca necurate.
Lev 20:26  Sa-Mi fi?i sfin?i, ca Eu, Domnul Dumnezeul vostru, sunt sfânt ?i v-am deosebit de toate popoarele, ca sa fi?i ai Mei.
Lev 20:27  Barbatul sau femeia, de vor chema mor?i sau de vor vraji, sa moara neaparat: cu pietre sa fie uci?i, ca sângele lor este asupra lor".
Lev 21:1  Zis-a Domnul catre Moise: "Graie?te preo?ilor, fiilor lui Aaron ?i le spune:
Lev 21:2  Sa nu se spurce prin atingere de mort din poporul lor. Sa se atinga numai de rudenia de aproape a lor, de mama lor ?i de tatal lor, de fiul lor ?i de fiica lor, de fratele lor;
Lev 21:3  De sora lor fecioara, care traie?te la ei ?i e nemaritata, poate sa se atinga, fara sa se spurce.
Lev 21:4  De nimeni altul din poporul sau sa nu se atinga, ca sa nu se spurce.
Lev 21:5  Sa nu-?i rada capul, sa nu-?i tunda marginea barbii ?i sa nu-?i faca taieturi pe trupurile lor pentru mor?i.
Lev 21:6  Sa fie sfin?i ai Dumnezeului lor ?i sa nu pângareasca numele Dumnezeului lor, ca ei aduc jertfa Domnului ?i pâine Dumnezeului lor ?i de aceea sa fie sfin?i.
Lev 21:7  Sa nu-?i ia de so?ie femeie desfrânata sau necinstita; nici femeie lepadata de barbatul ei, caci sunt sfin?i ai Domnului Dumnezeului lor.
Lev 21:8  Cinste?te-i ca pe sfin?i, caci ei aduc pâine Dumnezeului tau; sfin?i sa va fie, caci Eu, Domnul, Cel ce va sfin?esc, sunt sfânt.
Lev 21:9  Daca fiica preotului se va spurca prin desfrânare, ea necinste?te pe tatal sau: sa fie arsa cu foc.
Lev 21:10  Marele preot din fra?ii tai, pe capul caruia s-a turnat mirul de ungere ?i care este sfin?it, ca sa se îmbrace cu ve?mintele sfinte, sa nu-?i descopere capul sau, nici sa-?i sfâ?ie hainele;
Lev 21:11  ?i nici de un mort sa nu se apropie, nici chiar de tatal sau sau de mama sa sa nu se atinga.
Lev 21:12  De loca?ul sfânt sa nu se departeze, ca sa nu necinsteasca loca?ul Dumnezeului sau, caci mirul sfânt de ungere al Dumnezeului lui este asupra lui. Eu sunt Domnul.
Lev 21:13  Acesta î?i va lua de femeie fecioara din poporul sau.
Lev 21:14  Vaduva, sau lepadata, sau necinstita, sau desfrânata sa nu ia, ci fecioara din poporul sau sa-?i ia de femeie.
Lev 21:15  Sa nu-?i spurce samân?a sa în poporul sau, ca Eu sunt Domnul Dumnezeu, Cel ce îl sfin?esc"!
Lev 21:16  Grait-a Domnul cu Moise ?i a zis:
Lev 21:17  "Spune lui Aaron: Nimeni din neamul tau în viitor ?i din rudele tale sa nu se apropie, ca sa aduca daruri Dumnezeului sau, de va avea vreo meteahna pe trupul sau.
Lev 21:18  Tot omul cu meteahna pe trup sa nu se apropie: nici orb, nici ?chiop, nici ciung,
Lev 21:19  Nici cel cu piciorul rupt sau cu mâna rupta, nici ghebos, nici cu vreun madular uscat,
Lev 21:20  Nici cel cu albea?a pe ochi, nici chelul, nici pipernicitul, nici cel cu par?ile barbate?ti vatamate.
Lev 21:21  Nici un om din samân?a preotului Aaron, care va avea pe trupul sau vreo meteahna, sa nu se apropie ca sa aduca jertfa Domnului; ca are meteahna ?i de aceea sa nu se apropie ca sa aduca daruri Dumnezeului sau.
Lev 21:22  Darurile Dumnezeului sau sunt sfin?enii mari, din sfin?enii poate sa manânce,
Lev 21:23  Dar de perdea sa nu treaca ?i la jertfelnic sa nu se apropie; sa nu necinsteasca loca?ul Meu cel sfânt, caci Eu  sunt Domnul, Cel ce îi sfin?esc".
Lev 21:24  ?i a spus Moise acestea lui Aaron, fiilor lui ?i tuturor fiilor lui Israel.
Lev 22:1  Dupa aceea a grait Domnul cu Moise ?i a zis:
Lev 22:2  "Spune lui Aaron ?i fiilor lui sa umble cu bagare de seama cu cele sfinte ale fiilor lui Israel ?i sa nu pângareasca numele cel sfânt al Meu prin prinoasele pe care ei în?i?i Mi le aduc. Eu sunt Domnul.
Lev 22:3  Spune-le: Tot omul din semin?ia voastra ?i din neamul vostru, care va avea pe sine vreo necura?enie ?i se va apropia de cele sfinte, care se afierosesc de fiii lui Israel Domnului, sufletul aceluia se va stârpi de la fa?a Mea. Eu sunt Domnul Dumnezeul vostru.
Lev 22:4  Omul din semin?ia preotului Aaron care va fi lepros sau va avea curgere sa nu manânce din cele sfinte, pâna nu se va cura?i; ?i cine se va atinge de ceva necurat de la mort, sau cine va suferi de curgerea semin?ei,
Lev 22:5  Sau cine se va atinge de vreo târâtoare, de care s-ar spurca, sau de vreun om, care l-ar face necurat prin orice fel de necura?ie a lui,
Lev 22:6  Cel ce s-a atins de acestea necurat va fi pâna seara ?i sa nu manânce din cele sfinte înainte de a-?i spala trupul sau cu apa.
Lev 22:7  Iar dupa ce va apune soarele ?i dupa ce se va cura?i, atunci sa manânce din cele sfinte, ca aceea este hrana lui.
Lev 22:8  Mortaciune ?i sfâ?iat de fiara sa nu manânce, ca sa nu se spurce cu acestea. Eu sunt Domnul.
Lev 22:9  Sa pazeasca poruncile Mele, ca sa nu aiba asupra-le pacat ?i sa nu moara, când vor calca acestea. Eu sunt Domnul Dumnezeu, Cel ce îi sfin?esc pe ei.
Lev 22:10  Nici un strain sa nu manânce din cele sfinte. Nici cel ce locuie?te la un preot ?i nici simbria?ul preotului sa nu manânce din cele sfinte.
Lev 22:11  Iar daca preotul î?i va cumpara un rob cu argint, acela sa manânce din ele; asemenea ?i robul nascut în casa sa sa manânce din pâinea lui.
Lev 22:12  Daca fiica preotului se va marita dupa strain de neamul preo?esc, nici ea sa nu manânce din prinoasele sfinte, cuvenite lui.
Lev 22:13  Când însa fiica preotului va fi vaduva sau despar?ita ?i copii nu va avea ?i se va întoarce în casa tatalui sau, cum era ?i în tinere?ea sa, atunci ea sa manânce pâinea tatalui sau, iar dintre straini nimeni sa nu manânce.
Lev 22:14  Daca cineva manânca din gre?eala din cele sfinte, sa întoarca preotului pre?ul lucrului sfânt ?i sa mai adauge înca a cincea parte din pre?ul lui:
Lev 22:15  Preo?ii sa nu spurce cele sfinte ale fiilor lui Israel, pe care ei le aduc dar Domnului,
Lev 22:16  ?i sa nu atraga asupra-le vinova?ia faradelegii, când vor mânca cele sfinte ale lor, ca Eu sunt Domnul, Cel ce îi sfin?esc".
Lev 22:17  Grait-a Domnul cu Moise ?i a zis:
Lev 22:18  "Vorbe?te lui Aaron, fiilor lui ?i la toata adunarea fiilor lui Israel ?i le zi: Daca cineva dintre fiii lui Israel, sau dintre strainii care s-au a?ezat la ei, în Israel, î?i vor aduce jertfa lor, pe care o aduc Domnului ardere de tot, dupa fagaduin?a sau de evlavie,
Lev 22:19  Ca sa afle prin aceasta bunavoin?a la Dumnezeu, jertfa trebuie sa fie fara meteahna, de parte barbateasca, din vitele mari, sau din oi, sau din capre.
Lev 22:20  Toate câte au meteahna în sine sa nu le aduce?i Domnului, ca nu vor fi primite.
Lev 22:21  De va aduce cineva Domnului jertfa de mântuire, plinind o fagaduin?a, sau aducând jertfa de buna voie, sau la praznicele voastre, din boi, sau din oi, sa fie fara meteahna; ca sa fie primita, sa nu aiba nici o meteahna.
Lev 22:22  Dobitoc orb, vatamat, sau slut, sau bubos, sau rapciugos, sau râios, sa nu aduce?i Domnului ?i nici sa da?i la jertfelnicul Domnului pentru jertfa.
Lev 22:23  Bou sau oaie cu picioarele lungi sau scurte peste masura po?i sa aduci ca jertfa de evlavie, iar pentru jertfa fagaduita acestea nu, sunt primite.
Lev 22:24  Dobitocul care are par?ile barbate?ti strivite, sfarâmate, smulse sau taiate, sa nu-l aduce?i Domnului ?i în ?ara voastra sa nu face?i asemenea lucru.
Lev 22:25  Nici din mâinile celor de alt neam sa nu aduce?i nici unul din asemenea dobitoace în dar Dumnezeului rostru, pentru ca acestea sunt vatamate ?i cu meteahna ?i nu va vor fi primite".
Lev 22:26  ?i a grait Domnul cu Moise ?i a zis:
Lev 22:27  "De se va na?te vi?el, sau miel, sau ied, ?apte zile sa stea la mama lui, iar din ziua a opta înainte va fi bun de adus jertfa Domnului.
Lev 22:28  Dar nici vaca, nici oaie sa nu junghia?i în aceea?i zi cu puiul ei.
Lev 22:29  De aduce?i Domnului jertfa de mul?umire, s-o aduce?i ca sa va fie primita.
Lev 22:30  În aceea?i zi s-o mânca?i ?i sa nu lasa?i din carnea ei pe a doua zi. Eu sunt Domnul.
Lev 22:31  Sa pazi?i poruncile Mele ?i sa le plini?i: Eu sunt Domnul.
Lev 22:32  Sa nu spurca?i numele cel sfânt al Meu, ca sa fiu Eu sfânt între fiii lui Israel.
Lev 22:33  Eu sunt Domnul, Cel ce va sfin?esc pe voi, Care v-am scos din pamântul Egiptului, ca sa fiu Dumnezeul vostru. Eu sunt Domnul".
Lev 23:1  Grait-a Domnul cu Moise ?i a zis:
Lev 23:2  "Vorbe?te fiilor lui Israel ?i le spune care sunt sarbatorile Domnului, în care se vor face adunarile sfinte. Sarbatorile Mele sunt acestea:
Lev 23:3  ?ase zile sa lucra?i, iar ziua a ?aptea este ziua odihnei, adunare sfânta a Domnului: nici o munca sa nu face?i; aceasta este odihna Domnului în toate locuin?ele voastre.
Lev 23:4  Iata ?i celelalte sarbatori ale Domnului, adunarile sfinte, pe care trebuie sa le vesti?i la vremea lor:
Lev 23:5  În luna întâi, în ziua a paisprezecea a lunii, catre seara, sunt Pa?tile Domnului.
Lev 23:6  Iar în ziua a cincisprezecea a aceleia?i luni este sarbatoarea azimei Domnului: ?apte zile sa mânca?i azime.
Lev 23:7  În ziua întâi a sarbatorilor sa ave?i adunare sfânta ?i nici o munca sa nu face?i.
Lev 23:8  Timp de ?apte zile sa aduce?i jertfa Domnului, ?i în ziua a ?aptea iar e adunare sfânta; nici o munca sa nu face?i".
Lev 23:9  ?i a grait Domnul cu Moise ?i a zis:
Lev 23:10  "Vorbe?te fiilor lui Israel ?i le spune: Când ve?i intra în pamântul pe care Eu vi-l dau voua ?i ve?i face seceri? în el, cel dintâi snop al seceri?ului vostru sa-l aduce?i la preot;
Lev 23:11  El va ridica acest snop înaintea Domnului, ca sa afla?i bunavoin?a la El; a doua zi dupa cea dintâi a sarbatorii îl va ridica preotul.
Lev 23:12  În ziua ridicarii snopului ve?i aduce Domnului ardere de tot un miel de un an, fara meteahna.
Lev 23:13  Împreuna cu el ve?i aduce prinos de pâine doua din zece par?i de efa de faina de grâu, amestecata cu untdelemn, ca sa fie jertfa Domnului, mireasma placuta, ?i ve?i face ?i turnare un sfert de hin de vin.
Lev 23:14  Nici un fel de pâine noua, nici graun?e uscate, nici graun?e crude sa nu mânca?i pâna la ziua aceea în care ve?i aduce prinos Dumnezeului vostru. Acesta este a?ezamânt ve?nic în neamul vostru, oriunde ve?i locui.
Lev 23:15  Din ziua a doua dupa întâi a sarbatorii, din ziua în care ve?i aduce snopul leganat, sa numara?i ?apte saptamâni întregi,
Lev 23:16  Pâna la ziua întâi de dupa cea din urma zi a saptamânii a ?aptea, sa numara?i cincizeci de zile ?i atunci sa aduce?i un nou dar de pâine Domnului.
Lev 23:17  Sa aduce?i din locuin?ele voastre dar ridicat: doua pâini facute din doua zecimi de efa de faina de grâu, coapte cu dospitura, ca pârga Domnului.
Lev 23:18  Împreuna cu pâinile sa mai aduce?i ?apte miei fara meteahna, de câte un an, un junc din cireada ?i doi berbeci fara meteahna; ca sa fie acestea Domnului ardere de tot, dar de pâine, turnare ?i jertfa cu mireasma placuta Domnului.
Lev 23:19  Sa jertfi?i de asemenea din turma de capre un ?ap, jertfa pentru pacat, ?i doi miei de câte un an, jertfa de mântuire, împreuna cu pâinile din pârga.
Lev 23:20  Pe acestea sa le aduca preotul leganându-le înaintea Domnului, împreuna cu pâinile din pârga de grâu leganate ?i cu cei doi miei; acestea vor fi sfin?enie Domnului ?i vor fi ale preotului, care le înfa?i?eaza.
Lev 23:21  Sa da?i de ?tire praznuirea în ziua aceasta ?i sa ave?i adunare sfânta, nici o munca sa nu face?i. Acesta este a?ezamânt ve?nic în neamul vostru, în toate a?ezarile voastre.
Lev 23:22  Când ve?i secera holda în pamântul vostru, sa nu aduna?i ce ramâne dupa seceratul ogorului vostru ?i spicele ce cad de sub secere sa nu le aduna?i, ci sa lasa?i pe acestea saracului ?i strainului. Eu sunt Domnul, Dumnezeul vostru".
Lev 23:23  ?i a grait Domnul cu Moise ?i a zis:
Lev 23:24  "Spune fiilor lui Israel: În luna a ?aptea, ziua întâi a lunii sa va fie zi de odihna, sarbatoarea trâmbi?elor ?i adunare sfânta sa ave?i;
Lev 23:25  Nici o munca sa nu face?i, ci sa aduce?i ardere de tot Domnului",
Lev 23:26  Apoi a grait Domnul cu Moise ?i a zis:
Lev 23:27  "?i în ziua a zecea a lunii aceleia a ?aptea, care este ziua cura?irii, sa ave?i adunare sfânta; sa posti?i ?i sa aduce?i ardere de tot Domnului;
Lev 23:28  Nici o munca sa nu face?i în ziua aceea, ca aceasta este ziua cura?irii" ca sa va cura?i?i înaintea fe?ei Domnului Dumnezeului vostru.
Lev 23:29  Tot sufletul care nu va posti în ziua aceea se va stârpi din poporul sau;
Lev 23:30  ?i tot sufletul care va lucra în ziua aceea, îl voi stârpi din mijlocul poporului sau.
Lev 23:31  Nici o munca sa nu face?i: acesta este a?ezamânt ve?nic în neamul vostru în toate ceta?ile voastre.
Lev 23:32  Aceasta este pentru voi zi de odihna; sa posti?i din seara zilei a noua a lunii; din acea seara pâna în seara zilei a zecea a lunii sa praznui?i odihna voastra".
Lev 23:33  Grait-a Domnul cu Moise ?i a zis:
Lev 23:34  "Spune fiilor lui Israel: Din ziua a cincisprezecea a lunii a ?aptea începe sarbatoarea corturilor; ?apte zile sa sarbatore?ti în cinstea Domnului.
Lev 23:35  În ziua întâi va fi adunare sfânta; nici o munca sa nu face?i.
Lev 23:36  ?apte zile sa aduce?i jertfa Domnului ?i în ziua a opta va fi adunare sfânta; sa aduce?i arderi de tot Domnului: aceasta este încheierea sarbatorii; nici o munca sa nu face?i.
Lev 23:37  Acestea sunt sarbatorile Domnului, în care trebuie sa se ?ina adunarile sfinte, ca sa aduca jertfe Domnului, ardere de tot, prinos de pâine, jertfe junghiate ?i turnari, fiecare din ele la ziua hotarâta;
Lev 23:38  Afara de zilele de odihna ale Domnului, afara de darurile voastre, afara de toate afierosirile voastre ?i afara de tot ce aduce?i din evlavie ?i da?i Domnului,
Lev 23:39  În ziua a cincisprezecea a lunii a ?aptea, când va strânge?i roadele pamântului, sa sarbatori?i sarbatoarea Domnului ?apte zile: în ziua întâi este odihna ?i în ziua a opta iara este odihna.
Lev 23:40  În ziua întâi sa lua?i ramuri de copaci frumo?i, ramuri de finici, ramuri de copaci cu frunzele late ?i salcii de râu ?i sa va veseli?i înaintea Domnului Dumnezeului vostru, ?apte zile.
Lev 23:41  Sa praznui?i sarbatoarea aceasta a Domnului ?apte zile în an: acesta este a?ezamânt ve?nic în neamul vostru. În luna a ?aptea sa o praznui?i.
Lev 23:42  Sa locui?i ?apte zile în corturi; tot israelitul ba?tina? sa locuiasca în corturi,
Lev 23:43  Ca sa ?tie urma?ii vo?tri ca în corturi am a?ezat Eu pe fiii lui Israel, când i-am scos din pamântul Egiptului. Eu sunt Domnul Dumnezeul vostru".
Lev 23:44  Astfel a grait Moise fiilor lui Israel despre sarbatorile Domnului.
Lev 24:1  ?i a grait Domnul cu Moise ?i a zis:
Lev 24:2  "Porunce?te fiilor lui Israel sa-?i aduca untdelemn de masline, curat ?i limpede, pentru candele, ca sa arda sfe?nicul necontenit,
Lev 24:3  Înaintea perdelei din cortul adunarii, ?i-l va aprinde Aaron ?i fiii lui înaintea Domnului, ca sa arda totdeauna, de seara pâna diminea?a. Acesta este a?ezamânt ve?nic în neamul vostru.
Lev 24:4  Candelele sa le puna în sfe?nicul cel de aur curat de dinaintea Domnului, ca sa arda sfe?nicul de seara pâna diminea?a.
Lev 24:5  Sa lua?i faina de grâu buna ?i sa face?i din ea douasprezece pâini; fiecare pâine sa fie de doua zecimi de efa.
Lev 24:6  ?i sa le pune?i pe doua rânduri, câte ?ase pâini în rând, pe masa cea de aur curat de dinaintea Domnului;
Lev 24:7  Pe fiecare rând sa pui tamâie curata ?i sare, ?i vor fi acestea, pe lânga pâini, jertfa de pomenire înaintea Domnului.
Lev 24:8  În ziua odihnei sa se puna acestea necontenit înaintea Domnului din partea fiilor lui Israel; acesta este legamânt ve?nic.
Lev 24:9  Ele vor fi ale lui Aaron ?i ale fiilor sai, care le vor mânca în locul cel sfânt, ca acestea sunt sfin?enie mare pentru ei din jertfele Domnului: acesta este a?ezamânt ve?nic".
Lev 24:10  În vremea aceea, fiul unei israelite, nascut între israeli?i dintr-un egiptean, a ie?it la fiii lui Israel ?i s-a sfadit în tabara cu un israelit.
Lev 24:11  ?i fiul israelitei, hulind numele Domnului ?i graindu-l de rau, a fost adus la Moise, iar numele mamei lui era ?elomit, fata lui Dibri, din neamul lui Dan.
Lev 24:12  Acela a fost pus sub straja, ca sa-l judece dupa porunca Domnului.
Lev 24:13  Atunci a grait Domnul cu Moise ?i a zis:
Lev 24:14  "Scoate pe hulitor afara din tabara ?i to?i cei ce au auzit sa-?i puna mâinile lor pe capul lui, iar toata ob?tea sa-l ucida cu pietre.
Lev 24:15  Apoi fiitor lui Israel sa le spui: Omul care va huli pe Dumnezeu î?i va agonisi pacat.
Lev 24:16  Hulitorul numelui Domnului sa fie omorât neaparat; toata ob?tea sa-l ucida cu pietre. Sau strainul, sau ba?tina?ul, de va huli numele Domnului, sa fie omorât.
Lev 24:17  De va lovi cineva un om ?i va muri, acela sa fie omorât.
Lev 24:18  De va lovi cineva dobitoc ?i va muri, acela sa dea dobitoc pentru dobitoc.
Lev 24:19  De va pricinui cineva vatamare aproapelui sau, aceluia sa i se faca ceea ce a facut el altuia:
Lev 24:20  Frântura pentru frântura, ochi pentru ochi, dinte pentru dinte; cum a facut el vatamare altui om, a?a sa i se faca ?i lui.
Lev 24:21  Cel ce va ucide dobitoc sa dea altul; iar cel ce va ucide om sa fie omorât.
Lev 24:22  Aceea?i judecata sa ave?i ?i pentru strain ?i pentru ba?tina?, ca Eu sunt Domnul Dumnezeul vostru".
Lev 24:23  ?i dupa ce Moise a spus acestea fiilor lui Israel ?i au facut fiii lui Israel cum poruncise Domnul lui Moise, au scos pe hulitor afara din tabara ?i l-au ucis cu pietre.
Lev 25:1  Grait-a Domnul cu Moise pe Muntele Sinai ?i a zis:
Lev 25:2  "Vorbe?te fiilor lui Israel ?i le spune: Dupa ce veri intra în pamântul pe care îl voi da voua, sa se odihneasca pamântul; sa fie o odihna în cinstea Domnului.
Lev 25:3  ?ase ani sa semeni ogorul tau, ?ase ani sa lucrezi via ta ?i sa aduni roadele lor;
Lev 25:4  Iar anul al ?aptelea sa fie an de odihna a pamântului, odihna Domnului; ogorul tau sa nu-l semeni ?i via ta sa n-o tai în anul acela.
Lev 25:5  Ceea ce va cre?te de la sine pe ogorul tau sa nu seceri ?i strugurii de pe vi?ele tale netaiate sa nu-i culegi, ca sa fie acest an de odihna pentru pamânt.
Lev 25:6  ?i aceste roade vor fi în timpul odihnei pamântului hrana pentru tine, pentru robul tau ?i pentru roaba ta, pentru simbria?ul tau ?i pentru strainul tau care s-a a?ezat la tine;
Lev 25:7  Pentru dobitocul tau ?i pentru fiarele cele de pe pamântul tau, sa fie de hrana toate roadele lui.
Lev 25:8  Sa numeri apoi ?apte ani de odihna, adica de ?apte ori câte ?apte ani, ca sa ai în cei de ?apte ori câte ?apte ani, patruzeci ?i noua de ani.
Lev 25:9  ?i sa trâmbi?ezi cu trâmbi?a în luna a ?aptea, în ziua a zecea a lunii; în ziua cura?irii sa trâmbi?ezi cu trâmbi?a în toata ?ara voastra.
Lev 25:10  Sa sfin?i?i anul al cincizecilea ?i sa se vesteasca slobozenie pe pamântul vostru pentru to?i locuitorii lui. Sa va fie acesta an de slobozenie, ca sa se întoarca fiecare la mo?ia sa; fiecare sa se întoarca la ogorul sau.
Lev 25:11  Anul al cincizecilea sa va fie an de slobozenie: sa nu semana?i, nici sa secera?i ceea ce va cre?te de la sine pe pamânt, ?i sa nu culege?i poama de pe vi?ele netaiate,
Lev 25:12  Ca acesta e jubileu; sfânt sa fie pentru voi; roadele de pe ogor sa le mânca?i.
Lev 25:13  În anul jubileu sa se întoarca fiecare la mo?ia sa.
Lev 25:14  De vei vinde ceva aproapelui tau sau de vei cumpara ceva de la aproapele tau, sa nu în?ele nimeni pe aproapele sau.
Lev 25:15  Dupa numarul anilor trecu?i de la cel din urma jubileu sa cumperi de la aproapele tau, ?i dupa numarul anilor de cules sa-?i vânda el.
Lev 25:16  Daca au ramas ani mai mul?i pâna la jubileu, spore?te pre?ul, iar daca au ramas pu?ini ani, mic?oreaza pre?ul, caci un anumit numar de seceri?uri î?i vinde el.
Lev 25:17  Sa nu în?ele nimeni pe aproapele sau; teme-te de Domnul Dumnezeul tau; Eu sunt Domnul Dumnezeul vostru.
Lev 25:18  Face?i poruncile Mele ?i toate hotarârile Mele; face?i ?i pazi?i toate acestea ?i ve?i locui lini?ti?i pe pamânt.
Lev 25:19  Pamântul î?i va da rodul sau, ve?i mânca pâna la sa? ?i ve?i trai lini?ti?i pe el.
Lev 25:20  Iar de ve?i zice: Dar ce sa mâncam în anul al ?aptelea, când nici nu vom semana, nici nu vom aduna roadele noastre?
Lev 25:21  Va voi trimite binecuvântarea Mea în anul al ?aselea ?i va aduce roadele sale pentru trei ani.
Lev 25:22  ?i ve?i semana în anul al optulea, dar de mâncat ve?i mânca roadele cele vechi pâna la al noualea an: pâna se vor coace roadele anului al optulea ve?i mânca din cele vechi din anii trecu?i.
Lev 25:23  Pamântul sa nu-l vinde?i de veci, ca pamântul este al Meu; iar voi sunte?i straini ?i venetici înaintea Mea.
Lev 25:24  În toate par?ile stapânirii voastre sa îngadui?i rascumpararea pamântului.
Lev 25:25  Daca fratele tau, care e cu tine, va saraci ?i va vinde din mo?tenirea sa, sa vina ruda sa de aproape ?i sa cumpere ceea ce vinde fratele sau.
Lev 25:26  Daca însa nu va avea cineva rudenie, ci îi va da lui mâna ?i va gasi cât îi trebuie pentru rascumparare,
Lev 25:27  Atunci sa numere anii vânzarii sale, ?i ce trece sa întoarca aceluia, caruia i-a vândut, ?i se va întoarce la mo?ia sa.
Lev 25:28  Iar daca nu va gasi mâna lui cit îi trebuie sa întoarca aceluia, atunci pamântul vândut de el va ramâne în mâinile cumparatorului pâna la anul jubileu, ?i în anul jubileu cumparatorul va ie?i ?i vânzatorul va intra în stapânirea sa.
Lev 25:29  De va vinde cineva casa de locuit în ora? îngradit cu zid, poate s-o rascumpere pâna într-un an de la vânzarea ei: timp de un an poate s-o rascumpere.
Lev 25:30  Iar de nu se va rascumpara pâna la împlinirea unui an întreg, casa cea din ora? îngradit cu zid va ramâne pentru totdeauna aceluia care a cumparat-o ?i urma?ilor lui, ?i în anul jubileu nu va trece de la el.
Lev 25:31  Iar casele din sate, care n-au împrejur zid, sa se socoteasca deopotriva cu pamântul: ele se pot rascumpara oricând ?i în anul jubileu trec la fostul lor stapân.
Lev 25:32  Cât pentru ora?ele levi?ilor ?i casele din ora?ele stapânirii lor, levi?ii le vor putea rascumpara de-a pururi.
Lev 25:33  Iar daca cineva din levi?i nu va face rascumpararea, atunci casa vânduta din ora?ele stapânirii lor se întoarce în anul jubileu, caci casele din ora?ele levi?ilor sunt stapânirea lor între fiii lui Israel.
Lev 25:34  Nici ogoarele dimprejurul ora?elor lor nu se pot vinde, pentru ca acestea sunt mo?tenirea lor ve?nica.
Lev 25:35  Daca fratele tau va saraci ?i va ajunge la strâmtorare înaintea ta, ajuta-l, fie strain, fie ba?tina?, ca sa traiasca cu tine.
Lev 25:36  Sa nu iei de la el dobânda ?i spor, ci sa te temi de Dumnezeul tau, ca sa traiasca fratele tau cu tine. Eu sunt Domnul.
Lev 25:37  Argintul tau sa nu ?i-l dai lui cu camata ?i pâinea ta sa nu i-o dai ca s-o iei cu spor.
Lev 25:38  Eu sunt Domnul Dumnezeul vostru, Cel ce v-am scos din pamântul Egiptului, ca sa va dau pamântul Canaanului ?i ca sa fiu Dumnezeul vostru.
Lev 25:39  Când îi va saraci fratele ?i-i se va vinde ?ie, sa nu-l pui la munca de rob,
Lev 25:40  Ci sa fie el la tine ca simbria? sau strain, ?i sa-?i lucreze pâna la anul jubileu;
Lev 25:41  Iar atunci sa se duca de la tine, ?i el ?i copiii lui împreuna cu el, sa se întoarca în neamul sau ?i sa intre iara?i în stapânirea parin?ilor sai.
Lev 25:42  Pentru ca ei sunt robii Mei, pe care Eu i-am scos din pamântul Egiptului;
Lev 25:43  Sa nu-i vinzi, cum se vând robii, sa nu-i stapâne?ti cu cruzime ?i sa te temi de Dumnezeul tau.
Lev 25:44  Iar ca sa-?i ai robul tau ?i roaba ta, sa-?i cumperi rob ?i roaba de la neamurile dimprejurul vostru.
Lev 25:45  Pute?i sa va cumpara?i ?i din copiii strainilor, care s-au a?ezat la voi, ?i din neamul lor, care este la voi ?i care s-a nascut în pamântul vostru; pot sa fie averea voastra.
Lev 25:46  Pute?i sa-i da?i mo?tenire fiilor vo?tri dupa voi, ca orice avere, ve?nic sa-i stapâni?i, ca pe robi. Iar asupra fra?ilor vo?tri din fiii lui Israel ?i unul asupra altuia, sa nu domni?i cu cruzime.
Lev 25:47  Daca strainul sau veneticul tau s-a îmboga?it pe lânga tine, iar fratele tau a saracit lânga tine ?i s-a vândut veneticului, care s-a a?ezat la tine, sau unui urma? din neamul veneticului,
Lev 25:48  Atunci, dupa vânzare, se va putea rascumpara; careva din fra?ii lui va putea sa-l rascumpere:
Lev 25:49  Sau unchiul lui, sau fiul unchiului va putea sa-l rascumpere, sau careva din neamul lui, din semin?ia lui sa-l rascumpere; sau de va avea îndestulare, sa se rascumpere singur.
Lev 25:50  ?i acela sa se rafuiasca cu cel ce l-a cumparat, de la anul când s-a vândut el pâna la anul jubileu, ?i argintul pentru care s-a vândut sa i-l întoarca dupa numarul anilor; socotindu-i ca sluji?i de un simbria? vremelnic la el.
Lev 25:51  ?i daca ramân înca mul?i ani, el va plati rascumpararea potrivit anilor acestora, dupa pre?ul cu care a fost cumparat.
Lev 25:52  Iar daca ramân pu?ini ani pâna la anul jubileu, sa-i numere ?i sa plateasca, pentru rascumpararea sa, dupa numarul anilor.
Lev 25:53  El sa fie la dânsul cu anul, ca simbria?ul, ?i acela sa nu-l stapâneasca cu asprime înaintea ochilor tai.
Lev 25:54  Iar daca el nu se va rascumpara în chipul acesta, atunci în anul jubileu va ie?i el însu?i, ?i împreuna cu el, ?i copiii lui,
Lev 25:55  Pentru ca fiii lui Israel sunt robii Mei; robii Mei sunt ei, ca Eu i-am scos din pamântul Egiptului. Eu sunt Domnul Dumnezeul vostru.
Lev 26:1  Sa nu va face?i idoli, nici chipuri cioplite; nici stâlpi sa nu va ridica?i; nici pietre cu chipuri cioplite cu dalta sa nu va a?eza?i în pamântul vostru, ca sa va închina?i la ele, ca Eu sunt Domnul Dumnezeul vostru.
Lev 26:2  Zilele de odihna ale Mele sa le pazi?i ?i loca?ul Meu cel sfânt sa-l cinsti?i, ca Eu sunt Domnul.
Lev 26:3  De ve?i umbla dupa legile Mele ?i de ve?i pazi ?i plini poruncile Mele,
Lev 26:4  Va voi da ploaie la timp, pamântul ?i pomii î?i vor da roadele lor.
Lev 26:5  Treieratul vostru va ajunge pâna la culesul viilor, culesul viilor va ajunge pâna la semanat; ve?i mânca pâinea voastra cu mul?umire ?i ve?i trai în pamântul vostru fara primejdie.
Lev 26:6  Voi trimite pace pe pamântul vostru ?i nimeni nu va va tulbura; voi goni fiarele salbatice ?i sabia nu va trece prin pamântul vostru.
Lev 26:7  Ve?i alunga pe vrajma?ii vo?tri ?i vor cadea uci?i înaintea voastra.
Lev 26:8  Cinci din voi vor birui o suta ?i o suta din voi vor goni zece mii ?i vor cadea vrajma?ii vo?tri de sabie înaintea voastra.
Lev 26:9  Cauta-voi spre voi ?i va voi binecuvânta; ve?i avea copii, va voi înmul?i ?i voi fi statornic în legamântul Meu cu voi.
Lev 26:10  Ve?i mânca roadele vechi din anii trecu?i ?i ve?i da la o parte pe cele vechi pentru a face loc celor noi.
Lev 26:11  Voi a?eza loca?ul Meu în mijlocul vostru ?i sufletul Meu nu se va scârbi de voi.
Lev 26:12  Voi umbla printre voi, voi fi Dumnezeul vostru ?i voi poporul Meu.
Lev 26:13  Eu sunt Domnul Dumnezeul vostru, Cel ce v-am scos din pamântul Egiptului, ca sa nu mai fi?i robi acolo; am sfarâmat jugul vostru ?i v-am pova?uit cu fruntea ridicata.
Lev 26:14  Iar de nu Ma ve?i asculta ?i de nu veri pazi aceste porunci ale Mele,
Lev 26:15  De ve?i dispre?ui a?ezamintele Mele ?i de se va scârbi sufletul vostru de legile Mele, neîmplinind poruncile Mele, ?i calcând legamântul Meu,
Lev 26:16  Atunci ?i Eu am sa Ma port cu voi a?a: Voi trimite asupra voastra groaza, lingoarea ?i frigurile, de care vi se vor secatui ochii ?i vi se va istovi sufletul; ve?i semana semin?ele în zadar ?i vrajma?ii vo?tri le vor mânca.
Lev 26:17  Îmi voi întoarce fa?a împotriva voastra ?i ve?i cadea înaintea vrajma?ilor vo?tri; vor domni peste voi du?manii vo?tri ?i ve?i fugi când nimeni nu va va alunga.
Lev 26:18  Daca nici dupa toate acestea nu Ma ve?i asculta, atunci în?eptit voi mari pedeapsa pentru pacatele voastre.
Lev 26:19  Voi frânge îndaratnicia voastra cea mândra ?i cerul vostru îl voi face ca fierul, iar pamântul vostru ca arama.
Lev 26:20  În zadar va ve?i cheltui puterile voastre, ca pamântul vostru nu-?i va da roadele sale, nici pomii din ?ara voastra nu-?i vor da poamele lor.
Lev 26:21  Daca ?i dupa acestea ve?i umbla împotriva Mea ?i nu ve?i vrea sa Ma asculta?i, atunci va voi adauga lovituri în?eptit pentru pacatele voastre;
Lev 26:22  Voi trimite asupra voastra fiarele câmpului, care va vor lipsi de copii; vor prapadi vitele voastre ?i pe voi va voi împu?ina a?a, încât se vor pustii drumurile voastre.
Lev 26:23  Daca nici dupa aceasta nu va ve?i îndrepta, împotrivindu-va Mie,
Lev 26:24  Atunci ?i Eu voi veni cu mânie asupra voastra ?i va voi lovi în?eptit pentru pacatele voastre.
Lev 26:25  Voi aduce asupra voastra sabie razbunatoare, ca sa razbune legamântul Meu. Iar daca va ve?i ascunde în ora?ele voastre, voi trimite asupra voastra molima ?i ve?i fi da?i în mâinile vrajma?ului.
Lev 26:26  Pâinea, care va hrane?te, o voi lua de la voi; zece femei vor coace pâine pentru voi într-un cuptor ?i vor da pâinea voastra cu cântarul ?i ve?i mânca ?i nu va ve?i satura.
Lev 26:27  Daca nici dupa aceasta nu Ma ve?i asculta ?i ve?i pa?i împotriva Mea,
Lev 26:28  Atunci ?i Eu cu mânie voi veni asupra voastra ?i va voi pedepsi în?eptit pentru pacatele voastre;
Lev 26:29  Ve?i mânca din carnea fiilor vo?tri ?i din carnea fiicelor voastre.
Lev 26:30  Darâma-voi înal?imile voastre ?i voi strica stâlpii vo?tri; trupurile voastre le voi prabu?i sub darâmaturile idolilor vo?tri ?i se va scârbi sufletul Meu de voi.
Lev 26:31  Ora?ele voastre le voi preface în ruine, voi pustii loca?urile voastre cele sfinte ?i nu voi mirosi miresmele placute ale jertfelor voastre.
Lev 26:32  Pustii-voi pamântul vostru a?a încât sa se mire de el vrajma?ii vo?tri care se vor a?eza pe el;
Lev 26:33  Iar pe voi va voi risipi printre popoare; în urma voastra Îmi voi ridica sabia ?i va fi pamântul vostru pustiu ?i ora?ele voastre darâmate.
Lev 26:34  Atunci pamântul se va bucura de odihnele sale în zilele pustiirii lui; când voi ve?i fi în pamântul vrajma?ilor vo?tri, atunci pamântul vostru se va odihni ?i se va bucura de odihna lui.
Lev 26:35  În toate zilele pustiirii lui el se va odihni cât nu s-a odihnit în zilele de odihna ale voastre, când locuia?i voi în el.
Lev 26:36  Celor ce vor ramâne dintre voi le voi trimite în inimi frica în pamântul vrajma?ilor lor; pâna ?i freamatul frunzei ce se clatina îi va pune pe fuga ?i vor fugi ca de sabie ?i vor cadea când nimeni nu-i va alunga.
Lev 26:37  Se vor calca unul pe altul, ca cei ce fug de sabie, când nimeni nu-i va urmari ?i nu ve?i avea putere sa va împotrivi?i vrajma?ilor vo?tri.
Lev 26:38  Ve?i pieri printre popoare ?i va va înghi?i pamântul vrajma?ilor vo?tri.
Lev 26:39  Iar cei ce vor ramâne din voi se vor usca pentru pacatele lor în pamânturile vrajma?ilor. vo?tri, se vor usca ?i pentru pacatele parin?ilor lor.
Lev 26:40  Atunci î?i vor marturisi faradelegile lor ?i faradelegile parin?ilor lor, cum au savâr?it ei nelegiuiri împotriva Mea ?i au pa?it împotriva Mea.
Lev 26:41  Pentru care ?i Eu am venit cu mânie asupra lor ?i i-am adus în pamântul vrajma?ilor lor; atunci se va supune inima lor cea netaiata împrejur ?i vor suferi ei pentru nelegiuirile lor.
Lev 26:42  ?i Eu Îmi voi aduce aminte de legamântul Meu cu Iacov, de legamântul Meu cu Isaac, ?i de legamântul Meu cu Avraam Îmi voi aduce aminte ?i de pamânt îmi voi aduce aminte.
Lev 26:43  Când pamântul va fi parasit de ei ?i el se va bucura de odihna lui, golit fiind de ei, ?i ei vor suferi pentru faradelegi ?i pentru ca au nesocotit legile Mele ?i sufletul lor s-a scârbit de a?ezamântul Meu;
Lev 26:44  Când vor fi ei în pamântul vrajma?ilor, Eu nu-i voi dispre?ui ?i nu Ma voi scârbi de ei, a?a încât sa-i pierd ?i sa stric legamântul Meu cu ei, ca Eu sunt Domnul Dumnezeul lor.
Lev 26:45  Îmi voi aminti de ei pentru legamântul încheiat cu stramo?ii lor, pe care i-am scos din pamântul Egiptului, înaintea ochilor popoarelor, ca sa fiu Dumnezeul lor. Eu sunt Domnul".
Lev 26:46  Acestea sunt a?ezamintele, hotarârile ?i legile pe care le-a a?ezat Domnul între Sine ?i fiii lui Israel prin Moise, pe Muntele Sinai.
Lev 27:1  A grait Domnul cu Moise ?i a zis:
Lev 27:2  "Vorbe?te fiilor lui Israel ?i le spune: De va fagadui cineva sa-?i afieroseasca sufletul sau Domnului, pre?uirea ta sa fie a?a:
Lev 27:3  Pre?ul pentru un barbat, de la douazeci pâna la ?aizeci de ani, sa fie cincizeci de sicli de argint, dupa siclul sfânt.
Lev 27:4  Iar daca este femeie, pre?ul sa fie treizeci de sicli.
Lev 27:5  De la cinci pâna la douazeci de ani, pre?ul sa fie pentru barbat douazeci de sicli, iar pentru femeie zece sicli.
Lev 27:6  Iar de la o luna pâna la cinci ani, pre?ul sa fie pentru barbat cinci sicli de argint, ?i pentru femeie trei sicli de argint.
Lev 27:7  De la ?aizeci de ani în sus pre?ul sa fie pentru barbat cincisprezece sicli de argint, ?i pentru femeie zece sicli.
Lev 27:8  Iar daca este sarac ?i nu e în stare sa plateasca pre?ul, atunci sa fie adus la preot ?i sa-l pre?uiasca preotul; potrivit cu starea celui ce ?i-a dat fagaduin?a sa-l pre?uiasca preotul.
Lev 27:9  Daca însa va fi un dobitoc, ce se aduce jertfa Domnului, tot ce se aduce Domnului trebuie sa fie sfânt.
Lev 27:10  Sa nu schimbe nici bun cu rau, nici rau cu bun; iar de schimba cineva dobitoc cu dobitoc, atunci ?i cel schimbat ?i cel dat schimb va fi sfânt.
Lev 27:11  Daca ar fi cumva un dobitoc necurat, care nu se aduce jertfa Domnului, ?i el va fi adus la preot,
Lev 27:12  Preotul îl va pre?ui ori de este bun, ori de este rau; ?i cum îl va pre?ui preotul, a?a sa fie.
Lev 27:13  De va vrea cineva sa-l rascumpere, atunci sa adauge a cincea parte la pre?.
Lev 27:14  De va afierosi cineva casa sa Domnului, s-o pre?uiasca preotul, de este buna sau rea, ?i cum o va pre?ui preotul, a?a sa ramâna.
Lev 27:15  Daca afierositorul va vrea sa rascumpere casa sa, sa adauge a cincea parte de argint la pre?ul  ei ?i va fi a lui.
Lev 27:16  De va afierosi cineva Domnului ?arina din mo?ia sa, pre?uirea sa se faca dupa venitul ei, cincizeci de sicli de argint pentru fiecare gomer de orz semanat.
Lev 27:17  De î?i va afierosi ?arina sa chiar din anul jubileu, sa fie dupa pre?ul hotarât.
Lev 27:18  Iar de î?i afierose?te cineva ?arina sa dupa anul jubileu, atunci preotul sa socoteasca argintul dupa numarul anilor ce mai ramân pâna la anul jubileu ?i sa scada din pre?ul ei.
Lev 27:19  Daca însa va vrea sa-?i rascumpere ?arina cel ce a afierosit-o, atunci el sa adauge a cincea parte de argint la pre?ul ei ?i sa ramâna a lui.
Lev 27:20  Iar daca acela nu-?i va rascumpara ?arina ?i va fi vânduta altui om, atunci nu se mai. poate rascumpara;
Lev 27:21  ?arina aceea, când se va întoarce în anul jubileu, va fi afierosire Domnului, ca ?arina jertfa, ?i va trece în stapânirea preotului.
Lev 27:22  Iar daca cineva va afierosi Domnului o ?arina cumparata, care nu este din ?arinile mo?iei lui,
Lev 27:23  Preotul sa-i socoteasca partea de pre? pâna la anul jubileu, ?i acela sa-i dea pre?ul în aceea?i zi, ca afierosire Domnului,
Lev 27:24  ?i ?arina în anul jubileu va trece iar la acela, de la care a fost cumparata ?i din mo?ia caruia a fost pamântul acela.
Lev 27:25  Toate pre?urile sa fie facute dupa siclul sfânt; siclul sa aiba douazeci de ghere.
Lev 27:26  Numai întâii nascu?i ai dobitoacelor, care dupa întâietatea na?terii sunt ai Domnului, sa nu-i afieroseasca nimeni: fie bou, fie oaie, ca sunt ai Domnului.
Lev 27:27  Iar daca este dobitoc necurat, sa fie rascumparat dupa pre?uirea ta, la care sa se mai adauge a cincea parte, ?i de nu se va rascumpara sa se vânda dupa pre?uirea ta.
Lev 27:28  Toate cele afierosite, pe care omul cu juramânt le da Domnului din ale sale, - fie om, fie dobitoc, fie ?arina din mo?ia sa, - nici nu se rascumpara, nici nu se vând. Tot ce este afierosit cu juramânt este sfin?enie mare a Domnului.
Lev 27:29  Orice om afierosit cu juramânt nu se rascumpara, ci trebuie sa se dea mor?ii.
Lev 27:30  Toata dijma de la pamânt, din roadele pamântului ?i din roadele pomilor este a Domnului, sfin?enia Domnului.
Lev 27:31  ?i de va voi cineva sa-?i rascumpere dijma, sa adauge la pre?ul ei a cincea parte.
Lev 27:32  Toata dijma de la boi ?i de la oi ?i tot al zecelea din câte trec pe sub toiag este afierosit Domnului.
Lev 27:33  Nu trebuie cautat de este bun sau rau ?i nu trebuie schimbat; dar de-l va schimba cineva, atunci ?i cel schimbat ?i schimbul vor fi sfinte ?i nu se vor putea rascumpara".
Lev 27:34  Acestea sunt poruncile pe care le-a poruncit Domnul lui Moise pe Muntele Sinai pentru fiii lui Israel.


\end{document}