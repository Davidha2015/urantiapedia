\begin{document}

\title{Levitic}


\chapter{1}

\par 1 În vremea aceea, chemând pe Moise, Domnul i-a grăit din cortul adunării și i-a zis:
\par 2 "Grăiește fiilor lui Israel și le spune: De va aduce cineva dintre voi jertfă Domnului din dobitoace, să aducă jertfă din cireada de vite și din turma de oi.
\par 3 De va fi jertfa lui ardere de tot din vite mari, să fie parte bărbătească, fără meteahnă, și s-o aducă la ușa cortului adunării, ca să fie bine-plăcută înaintea Domnului.
\par 4 Apoi să-și pună mâna pe capul jertfei cea pentru arderea de tot și își va afla bunăvoință spre iertarea păcatelor lui.
\par 5 Apoi să junghie vițelul, înaintea Domnului, iar preoții, fiii lui Aaron, să aducă sângele și să stropească cu sânge împrejur jertfelnicul de la ușa cortului adunării.
\par 6 Să despoaie pielea de pe jertfa arderii de tot și să taie jertfa în bucăți.
\par 7 După aceea preoții, fiii lui Aaron, să pună pe jertfelnic foc și pe foc să pună lemne;
\par 8 Și pe lemnele de pe focul care e pe jertfelnic, să pună preoții, fiii lui Aaron, bucățile, capul și grăsimea, precum și măruntaiele și picioarele, după ce le vor spăla cu apă.
\par 9 Și să ardă preoții toate acestea pe jertfelnic, ca ardere de tot, jertfă, mireasmă plăcută Domnului.
\par 10 Iar dacă jertfa adusă de el Domnului ca ardere de tot este din vite mici, să fie din miei sau din iezi, parte bărbătească, fără meteahnă.
\par 11 Să-și pună mâna pe capul jertfei și s-o înjunghie înaintea Domnului, în partea de miazănoapte a jertfelnicului; iar preoții, fiii lui Aaron, să stropească cu sângele ei jertfelnicul împrejur.
\par 12 Să o taie apoi în bucăți, despărțind capul și grăsimea ei; să așeze preotul bucățile pe lemnele care sunt pe focul de pe jertfelnic;
\par 13 Iar măruntaiele și picioarele să le spele cu apă și să aducă preotul toate și să le ardă pe jertfelnic, ca ardere de tot, jertfă, mireasmă plăcută Domnului.
\par 14 Dacă aduce el din păsări ardere de tot Domnului, să aducă jertfa sa din turturele sau din pui de porumbel.
\par 15 Preotul să o aducă la jertfelnic, să-i frângă gâtul și să o pună pe jertfelnic, iar sângele să-l scurgă pe peretele jertfelnicului.
\par 16 Gușa și penele să le scoată și să le arunce lângă jertfelnic, în partea dinspre răsărit, la locul cenușei.
\par 17 Apoi să-i frângă aripile, fără a le desprinde de trup, Și să o ardă preotul pe jertfelnic, pe lemnele ce sunt pe foc, ca ardere de tot, jertfă, mireasmă plăcută Domnului".

\chapter{2}

\par 1 "Dacă cineva voiește să jertfească Domnului prinos de pâine, să aducă făină bună de grâu, să toarne peste ea untdelemn și să pună pe ea tămâie;
\par 2 Apoi s-o aducă preoților, fiilor lui Aaron, iar unul din ei să ia o mână plină de făină, cu untdelemnul și cu toată tămâia și s-o ardă pe jertfelnic spre pomenire, ca jertfă, mireasmă plăcută Domnului.
\par 3 Iar rămășița din prinosul de pâine va fi a lui Aaron și a fiilor lui; aceasta e sfințenie mare din jertfele Domnului.
\par 4 Dacă însă vrei să aduci jertfa prinosului de pâine din aluaturi coapte în cuptor, să aduci pâini de făină bună de grâu, nedospite, frământate cu untdelemn, și turte nedospite, unse cu untdelemn.
\par 5 Dacă jertfa ta este prinos de pâine copt în tigaie, să fie de făină bună de grâu, frământată cu untdelemn și nedospită.
\par 6 Să-l rupi bucăți și să torni peste el untdelemn; acesta este prinos de pâine, adus Domnului.
\par 7 Dacă jertfa ta este prinos de pâine gătit în oală, să se facă de făină bună de grâu cu untdelemn.
\par 8 Prinosul Domnului, gătit așa, să se aducă și să se încredințeze preotului, iar acesta să-l ducă la jertfelnic.
\par 9 Apoi să ia preotul din jertfă o parte spre pomenire și s-o ardă pe jertfelnic, ca ardere, ca mireasmă plăcută Domnului;
\par 10 Iar rămășitele prinosului de pâine vor fi pentru Aaron și fiii lui; acestea-s sfințenie mare din jertfele Domnului.
\par 11 Orice prinos de pâine, ce aduceți Domnului, să nu-l faceți dospit, căci nici dospitură, nici miere nu veți arde, ca jertfă înaintea Domnului.
\par 12 Ca prinos de pârgă, să aduceți și de acestea Domnului, dar pe jertfelnic să nu le înălțați întru mireasmă bine-plăcută Domnului.
\par 13 Toate prinoasele tale de pâine sară-le cu sare; să nu lași jertfele tale fără sare, semnul legământului Dumnezeului tău; cu toate prinoasele tale adu Domnului Dumnezeului tău și sare.
\par 14 De aduci Domnului prinos de pâine din cele dintâi roade, adu ca dar din cele dintâi roade ale tale grăunțe din spice, prăjite pe foc și pisate;
\par 15 Toarnă peste ele untdelemn și pune pe ele tămâie; acesta este prinos de pâine.
\par 16 Preotul să ardă, spre pomenire, o parte din grăunțe și din untdelemn cu toată tămâia; aceasta este jertfă Domnului".

\chapter{3}

\par 1 "Dacă însă jertfa lui va fi jertfă de împăcare și dacă se va aduce din vite mari, parte bărbătească sau parte femeiască, să se aducă înaintea Domnului din cele fără meteahnă
\par 2 Să-și pună cel ce o aduce mâna sa pe capul jertfei și s-o junghie înaintea Domnului, la ușa cortului adunării; iar preoții, fiii lui Aaron, să stropească cu sânge din ea jertfelnicul împrejur.
\par 3 Din jertfa de mântuire să aducă jertfă Domnului: grăsimea care acoperă măruntaiele, toată grăsimea ce acoperă intestinele;
\par 4 Amândoi rărunchii, grăsimea de pe ei și cea de pe șolduri, seul de pe ficat și cel de pe rărunchi;
\par 5 Iar fiii lui Aaron să ardă acestea pe jertfelnic împreună cu arderea de tot, care este pe lemnele de pe focul de pe jertfelnic; aceasta este jertfă, mireasmă plăcută Domnului.
\par 6 Iar dacă cineva aduce Domnului jertfă de împăcare din vite mici, parte bărbătească sau femeiască, s-o aducă din cele fără meteahnă.
\par 7 Dacă aduce jertfă o oaie, să o înfățișeze înaintea Domnului.
\par 8 Să-și pună mâna sa pe capul jertfei sale și s-o junghie înaintea cortului adunării; iar preoții, fiii lui Aaron, să stropească cu sângele ei jertfelnicul pe toate părțile.
\par 9 Și din această jertfă de împăcare să aducă ardere Domnului grăsimea ei, toată coada, retezând-o chiar din capătul spinării, grăsimea de pe măruntaie, toată grăsimea de pe partea dinăuntru;
\par 10 Amândoi rărunchii, grăsimea de pe ei și cea de pe șolduri, seul de pe ficat și praporul, pe care-l va desprinde cu rărunchii;
\par 11 Iar preotul să ardă acestea pe jertfelnic; această mistuire prin foc este jertfă Domnului.
\par 12 Dacă însă jertfa lui este din capre, s-o înfățișeze înaintea Domnului,
\par 13 Să-și pună mâna sa pe capul caprei și s-o junghie la ușa cortului adunării; iar preoții, fiii lui Aaron, să stropească cu sângele ei jertfelnicul împrejur.
\par 14 Din acestea să aducă prinos și jertfă Domnului: grăsimea de pe măruntaie, toată grăsimea care acoperă intestinele,
\par 15 Amândoi rărunchii, grăsimea de pe ei și cea de pe șolduri, seul de pe ficat pe care-l va desprinde cu cel de pe rărunchi;
\par 16 Și să le ardă preotul pe jertfelnic; această ardere pe foc este mireasmă plăcută Domnului. Toată grăsimea este a Domnului.
\par 17 Este lege veșnică și pentru toți urmașii voștri din toate așezările voastre, ca toată grăsimea și tot sângele să nu-l mâncați".

\chapter{4}

\par 1 Și a grăit Domnul cu Moise și a zis:
\par 2 "Grăiește fiilor lui Israel și le spune: Dacă vreun om va păcătui din neștiință împotriva poruncilor Domnului și va face ce nu se cuvine, călcând vreuna din ele;
\par 3 De a păcătuit arhiereu miruit și a tras pe popor la păcat, pentru păcatul său, pe care l-a săvârșit, să aducă un vițel fără meteahnă, ca jertfă Domnului pentru păcat;
\par 4 Să înfățișeze vițelul înaintea Domnului, la ușa cortului adunării, să-și pună mâna sa pe capul vițelului și să junghie vițelul înaintea Domnului.
\par 5 Apoi să ia preotul cel miruit, ale cărui mâini sunt sfințite, din sângele vițelului și să-l ducă în cortul adunării.
\par 6 Acolo să-și moaie preotul degetul său în sânge, să stropească cu sânge de șapte ori înaintea Domnului, asupra perdelei locașului sfânt.
\par 7 După aceea să pună preotul din sângele vițelului înaintea Domnului, pe coarnele jertfelnicului tămâierii, care se află în cortul adunării, iar toată rămășița din sângele vițelului s-o toarne la temelia jertfelnicului arderii de tot, care se află înaintea cortului adunării.
\par 8 Apoi să scoată din vițelul adus pentru păcat toată grăsimea lui, grăsimea cea de pe măruntaie, toată grăsimea ce acoperă lăuntrul,
\par 9 Amândoi rărunchii cu grăsimea de pe ei și cea de pe șolduri, seul de pe ficat; acestea să le scoată împreună cu rărunchii,
\par 10 Precum se ia din vițelul jertfei de izbăvire, și să le ardă preotul pe jertfelnicul arderii de tot.
\par 11 Iar pielea vițelului și tot trupul lui cu capul și cu picioarele lui, cu măruntaiele lui și cu necurățenia lui,
\par 12 Adică tot vițelul să-l scoată afară din tabără, la loc curat, unde se aruncă cenușa, și să-l ardă pe foc de lemne; unde se aruncă cenușa, acolo să-l ardă.
\par 13 Dacă însă toată obștea lui Israel va păcătui, din neștiință, și va face împotriva poruncilor Domnului ceva ce nu trebuia făcut și vrednic de osândă, iar fapta `aceasta va rămâne necunoscută adunării,
\par 14 Când se va afla păcatul, pe care l-au săvârșit ei, să se aducă din partea întregii obști un vițel fără meteahnă, jertfă pentru păcat, să-l înfățișeze înaintea cortului adunării,
\par 15 Iar bătrânii obștii să-și pună mâinile lor pe capul vițelului, înaintea Domnului și să junghie vițelul înaintea Domnului.
\par 16 Apoi preotul miruit să ducă din sângele vițelului în cortul adunării.
\par 17 Să-și moaie preotul degetul său în sângele vițelului și să stropească de șapte ori înaintea Domnului asupra perdelei sfintei sfintelor.
\par 18 Apoi preotul să pună din sânge pe coarnele jertfelnicului tămâierii, care este înaintea feței Domnului în cortul adunării, iar celălalt sânge să-l toarne la temelia jertfelnicului arderii de tot, care este la ușa cortului adunării.
\par 19 Toată grăsimea lui s-o scoată din el și s-o ardă pe jertfelnic;
\par 20 Și să facă cu vițelul acesta ceea ce s-a făcut cu vițelul adus pentru păcat; așa să facă cu el și așa să-i curețe preotul și li se va ierta păcatul.
\par 21 După aceea să scoată vițelul întreg afară din tabără și să-l ardă așa cum a ars și vițelul de care s-a vorbit mai sus. Aceasta e jertfă pentru păcatul obștii.
\par 22 Iar dacă va greși o căpetenie și din neștiință va face împotriva uneia din toate poruncile Domnului Dumnezeului său ceva ce nu trebuia să facă și vrednic de osândă,
\par 23 Când va afla el păcatul său, pe care l-a săvârșit, să aducă jertfă pentru păcat un țap fără meteahnă,
\par 24 Să-și pună mâna sa pe capul țapului și să-l junghie, unde se junghie arderile de tot, înaintea Domnului; aceasta este jertfă pentru păcat.
\par 25 Iar preotul să ia cu degetul său sânge de la jertfa pentru păcat și să-l pună pe coarnele jertfelnicului arderii de tot, iar celălalt sânge să-l toarne la temelia jertfelnicului arderii de tot.
\par 26 Toată grăsimea ei s-o ardă pe jertfelnic, ca grăsimea jertfei de izbăvire, și așa îl va curăți preotul de păcatul lui și i se va ierta.
\par 27 Dacă însă un om din poporul de rând va greși din neștiință împotriva uneia din toate poruncile Domnului și va face ceva ce nu trebuia să facă și vrednic de osândă,
\par 28 Când va afla el păcatul ce l-a săvârșit, să aducă din caprele sale jertfă o capră fără meteahnă, pentru păcatul ce l-a săvârșit,
\par 29 Să-și pună mâna sa pe capul jertfei pentru păcat și să junghie capra adusă, jertfă pentru păcat, unde se junghie jertfele arderii de tot.
\par 30 Apoi să ia preotul din sângele ei cu degetul său și să pună pe coarnele jertfelnicului arderii de tot, iar celălalt sânge să-l toarne la temelia jertfelnicului.
\par 31 Toată grăsimea ei s-o aleagă, cum se alege grăsimea la jertfele de mântuire, și s-o ardă preotul pe jertfelnic, spre miros bine-plăcut Domnului; astfel îl va curăți preotul și i se va ierta păcatul.
\par 32 Iar dacă cineva vrea să aducă jertfă pentru păcat din turma de oi, să aducă parte femeiască, fără meteahnă,
\par 33 Să-și pună mâna sa pe capul jertfei pentru păcat și s-o junghie, ca jertfă pentru păcat, la locul unde se junghie jertfa arderii de tot.
\par 34 Apoi să ia preotul cu degetul său din sângele acestei jertfe pentru păcat și să pună pe coarnele jertfelnicului arderii de tot, iar celălalt sânge să-l toarne jos lângă jertfelnic.
\par 35 Toată grăsimea ei s-o aleagă, cum se alege grăsimea din oaia pentru jertfă de izbăvire, și s-o ardă preotul pe jertfelnic, ca jertfă Domnului; și așa îl va curăți preotul de păcatul ce l-a săvârșit și i se va ierta".

\chapter{5}

\par 1 Dacă vreun suflet va păcătui prin aceea că, fiind pus să jure ca martor, nu va spune ceea ce a auzit sau ce știe, acela va lua asupra sa păcat.
\par 2 Sau de se va atinge cineva de orice lucru necurat, sau de trup necurat de fiară, sau de stârv de dobitoc necurat, sau de stârv de târâtoare necurată, fără să știe, se face necurat și vinovat;
\par 3 Sau de se va atinge cineva de necurățenie omenească, sau de orice fel de necurățenie care spurcă, și nu va ști, dar apoi va afla, acela e vinovat.
\par 4 Sau de se va jura cineva cu buzele sale nebunește să facă ceva rău sau bine, orice fel ar fi fapta pentru care se jură oamenii fără socoteală, de nu va ști că aceasta este rău, ci va afla în urmă, e vinovat.
\par 5 Deci, de se va face cineva vinovat de ceva din acestea și își va mărturisi păcatul,
\par 6 Atunci, pentru păcatul său, pe care l-a săvârșit, să aducă Domnului jertfă din turmă, o oaie sau o capră din caprele sale, pentru vina păcatului, și-l va curăți preotul prin aceasta de păcatul său și i se va ierta păcatul.
\par 7 Iar de nu va fi în stare să aducă jertfă o oaie, pentru vina păcatului său, să aducă Domnului două turturele sau doi pui de porumbel: unul jertfă pentru păcat, iar altul ardere de tot.
\par 8 Aceste păsări să le aducă la preot și preotul să jertfească mai întâi pe cea pentru păcat, să-i frângă gâtul, fără să despartă capul de trup,
\par 9 Și să stropească cu sângele acestei jertfe pentru păcat peretele jertfelnicului, iar celălalt sânge să-l scurgă jos lângă jertfelnic; aceasta e jertfă pentru păcat.
\par 10 Iar pe cealaltă pasăre s-o aducă ardere de tot, după rânduială. Și așa îl va curăți preotul de păcatul lui și i se va ierta.
\par 11 Dacă însă nu-i va da mâna să aducă nici o pereche de turturele sau doi pui de porumbel, atunci să aducă pentru greșeala sa a zecea parte dintr-o efă de făină bună de grâu, ca jertfă pentru păcat, dar să nu toarne pe ea untdelemn, nici tămâie să nu pună pe ea, că aceasta este jertfă pentru păcat.
\par 12 S-o aducă la preot și preotul să ia din ea un pumn plin, spre pomenire, și s-o ardă pe jertfelnic, ca jertfă Domnului; aceasta este jertfă pentru păcat.
\par 13 Prin aceasta îl va curăți preotul de păcatul lui, pe care l-a săvârșit în una din întâmplările acelea, și i se va ierta păcatul; rămășița de făină va fi a preotului, ca la prinosul de făină".
\par 14 Apoi a grăit Domnul cu Moise și a zis:
\par 15 De va face cineva greșeală și din neștiință va păcătui împotriva celor afierosite Domnului, acela, pentru vina sa, să ia din turma de oi și să aducă Domnului jertfă pentru vină, un berbec fără meteahnă, prețuit la doi sicli de argint, după prețul siclului sfânt.
\par 16 Și ce a greșit împotriva lucrului sfânt, va plăti și va mai adăuga peste prețul lui a cincea parte din preț și va da aceasta preotului și preotul îl va curăți prin berbecul jertfei pentru vină și i se va ierta.
\par 17 De va greși cineva împotriva uneia din toate poruncile Domnului și va face ce nu se cuvine să facă și din neștiință s-a făcut vinovat și va fi sub păcat,
\par 18 Acela să aducă la preot din turma de oi, jertfă pentru vină, un berbec fără meteahnă, după prețuirea ta, și-i va curăți preotul greșeala, în care a căzut el din neștiință și i se va ierta.
\par 19 Aceasta este jertfă pentru greșeala cu care s-a făcut vinovat el înaintea Domnului".

\chapter{6}

\par 1 Grăit-a iarăși Domnul cu Moise și a zis:
\par 2 "Dacă cineva va greși și cu bună știință va nesocoti poruncile Domnului, tăgăduind înaintea aproapelui său ceea ce acesta i-a încredințat, sau i-a lăsat în păstrare, sau ceea ce i-a furat, sau va înșela pe aproapele său,
\par 3 Sau găsind un lucru pierdut și tăgăduind înaintea lui, sau jurându-se strâmb pentru ceva, ce atrage pedeapsă asupra oamenilor,
\par 4 Dacă se va dovedi că a greșit și s-a făcut vinovat, să întoarcă ce a furat, sau ce a răpit, sau ce i-a fost încredințat, sau ce a fost pierdut și găsit de el.
\par 5 Tot lucrul, pentru care s-a jurat strâmb, să-l plătească deplin și să mai adauge pe deasupra a cincea parte din prețul lui și să dea aceluia, al căruia este lucrul, în ziua când își va cunoaște vina sa.
\par 6 Iar pentru vina sa să ia din turma de oi un berbec fără meteahnă, după prețuirea ta, și să-l aducă Domnului prin preot, jertfă pentru vină.
\par 7 Preotul îl va curăți înaintea Domnului și i se va ierta orice ar fi făptuit și oricum s-ar fi făcut vinovat".
\par 8 Și a grăit Domnul cu Moise și a zis:
\par 9 "Poruncește lui Aaron și fiilor lui și le zi: Rânduiala arderii de tot este aceasta: arderea de tot să rămână pe vatra jertfelnicului toată noaptea până dimineața și focul jertfelnicului să ardă pe el și să nu se stingă.
\par 10 Iar dimineața preotul să se îmbrace cu haina sa cea de in, după ce și-a luat pantalonii săi cei de in pe trupul său, să ridice cenușa arderii de tot, pe care a ars-o focul pe jertfelnic, și s-o pună lângă jertfelnic.
\par 11 Apoi să-și dezbrace hainele sale și să se îmbrace cu alte haine și să scoată cenușa afară din tabără; la loc curat.
\par 12 Dar focul pe jertfelnic să ardă și să nu se stingă; preotul să pună pe el lemne în fiecare dimineață, să așeze pe el ardere de tot și să ardă pe el grăsimea jertfei de mântuire.
\par 13 Iar focul să ardă necontenit pe jertfelnic și să nu se stingă.
\par 14 Rânduiala prinosului de pâine, pe care preoții, fiii lui Aaron, trebuie să-l aducă înaintea Domnului la jertfelnic, este aceasta:
\par 15 Să ia preotul din prinosul acesta de pâine un pumn de făină de grâu, cu untdelemnul ei și cu toată tămâia, care e pe prinos, și să le ardă pe jertfelnic mireasmă plăcută de pomenire înaintea Domnului.
\par 16 Iar rămășița din ea s-o mănânce Aaron și fiii lui și s-o mănânce nedospită, în locul cel sfânt; în curtea cortului adunării s-o mănânce, dar să nu o coacă dospită.
\par 17 Aceasta le-o dau parte din jertfele Mele. Aceasta este sfințenie mare, ca și jertfa pentru păcat și ca și jertfa pentru vină.
\par 18 Tot bărbatul din neamul preoțesc poate să mănânce din ea. Aceasta e lege veșnică în neamul vostru din jertfele Domnului. Tot ce se va atinge de ea, se va sfinți".
\par 19 A grăit Domnul cu Moise și a zis:
\par 20 "Prinosul lui Aaron și al fiilor lui, pe care-l vor aduce ei Domnului, în ziua ungerii lor, este acesta: făină bună de grâu, a zecea parte din efă, vor aduce jertfă necontenită; jumătate din ea dimineața și jumătate seara.
\par 21 S-o gătească în tigaie, cu untdelemn; s-o rupi bucăți, cum se rupe prinosul de pâine frământat cu untdelemn; să o aduci întru mireasmă plăcută Domnului.
\par 22 Aceasta s-o săvârșească preotul, care se va mirui în locul lui Aaron, din fiii lui; acesta este așezământ veșnic. Prinosul acesta să-l ardă tot.
\par 23 Orice prinos de pâine din partea preotului să se ardă tot și să nu se mănânce nimic din el".
\par 24 Și a grăit Domnul cu Moise și  a zis:
\par 25 "Spune lui Aaron și fiilor lui și le zi: Rânduiala jertfei pentru păcat este aceasta: jertfa pentru păcat să se junghie înaintea Domnului, în locul unde se junghie și cea pentru arderea de tot. Aceasta este sfințenie mare.
\par 26 Preotul cel ce săvârșește jertfa cea pentru păcat s-o mănânce, dar s-o mănânce în locul cel sfânt, în curtea cortului adunării.
\par 27 Tot ce se va atinge de carnea ei se va sfinți; și de se va stropi cu sângele ei haina, haina stropită să se spele în locul cel sfânt.
\par 28 Oala de lut, în care s-a fiert ea, să se spargă; iar dacă ea s-a fiert în vas de aramă, acesta să se curețe și să se spele cu apă.
\par 29 Toți cei de parte bărbătească din neamul preoțesc pot să mănânce din ea. Aceasta este mare sfințenie înaintea Domnului.
\par 30 Dar orice jertfă pentru păcat din al cărei sânge s-a dus în cortul adunării pentru facerea curățirii în locul cel sfânt, să nu se mănânce, ci să se ardă în foc".

\chapter{7}

\par 1 "Iată și rânduiala jertfei pentru vină: Aceasta este sfințenie mare.
\par 2 Jertfa pentru vină să se junghie în locul unde se junghie jertfa arderii de tot și cu sângele ei să se stropească jertfelnicul de jur împrejur.
\par 3 Cel ce o aduce să osebească din ea toată grăsimea, coada și grăsimea de pe măruntaie,
\par 4 Amândoi rărunchii, grăsimea cea de pe ei și seul de pe ficat: toate acestea să le osebească împreună cu cei doi rărunchi.
\par 5 Acestea să le ardă preotul pe jertfelnic, ca jertfă Domnului. Aceasta este jertfă pentru vină.
\par 6 Toți cei de parte bărbătească din neamul preoțesc să mănânce din ea, dar s-o mănânce în locul cel sfânt, că aceasta este sfințenie mare.
\par 7 La jertfa pentru vină, ca și la jertfa pentru păcat, este aceeași rânduială; ele sunt partea preotului, care săvârșește curățirea cu ajutorul lor.
\par 8 Când preotul va aduce jertfa arderii de tot a cuiva, pielea jertfei aduse va fi a preotului.
\par 9 Tot prinosul de pâine copt în cuptor și tot prinosul de pâine gătit în oală sau în tigaie va fi al preotului, care-l săvârșește.
\par 10 Orice dar de pâine, frământat cu untdelemn sau uscat, va fi al tuturor fiilor lui Aaron deopotrivă.
\par 11 Iar rânduiala jertfei de împăcare, care se aduce Domnului, este aceasta:
\par 12 Dacă se va aduce ca jertfă de mulțumire, atunci să se aducă pâini frământate cu untdelemn, turte nedospite, unse cu untdelemn, făină de grâu, frământată cu untdelemn;
\par 13 Pe lângă pâinile nedospite să se mai aducă dar la jertfa de mulțumire și pâine dospită.
\par 14 Unul din toate aceste daruri ale sale să-l aducă Domnului dar ridicat; acesta va fi al preotului, care stropește cu sângele jertfei de mântuire.
\par 15 Și carnea jertfei de mântuire, ca dar de mulțumire, va fi tot a lui, însă să se mănânce în ziua aducerii ei și să nu rămână din ea nimic pe a doua zi.
\par 16 Dacă însă jertfa ce se aduce este din făgăduință sau de bunăvoie, jertfa lui să se mănânce în ziua aducerii și ceea ce va rămâne se poate mânca a doua zi.
\par 17 Iar ceea ce va mai rămâne din carnea jertfei pe a treia zi să se ardă cu foc.
\par 18 Dacă însă carnea jertfei acesteia o va mânca cineva a treia zi, jertfa aceasta nu va fi primită și nu i se va ține în seamă, că este întinare și cel ce o va mânca va avea asupra sa păcat.
\par 19 Carnea care a fost atinsă de ceva necurat să nu se mănânce, ci să se ardă cu foc; iar carnea curată să se mănânce tot de cel curat.
\par 20 Dacă însă vreun om, în stare de necurăție, va mânca din carnea jertfei de mântuire, adusă Domnului, acel suflet se va stârpi din poporul său.
\par 21 Dacă vreun om, care s-a atins de ceva necurat, de necurățenie omenească, sau de dobitoc necurat, sau de vreo târâtoare necurată, va mânca din carnea jertfei de izbăvire, adusă Domnului, omul acela se va stârpi din poporul său".
\par 22 Și a grăit Domnul cu Moise și a zis:
\par 23 "Grăiește fiilor lui Israel și le zi: Nici un fel de grăsime, nici de bou, nici de oaie, nici de țap să nu mâncați.
\par 24 Grăsimea de mortăciune și grăsimea dobitocului sfâșiat de fiară să se întrebuințeze la orice lucru, iar de mâncat să nu se mănânce.
\par 25 Tot cel ce va mânca grăsimea dobitocului, care se aduce jertfă mistuită cu foc Domnului, acela să se stârpească din poporul său.
\par 26 Nici un fel de sânge să nu mâncați în toate cetățile voastre, nici de păsări, nici de dobitoace.
\par 27 Tot cel ce va mânca sânge, acela se va stârpi din poporul său".
\par 28 Și a grăit Domnul cu Moise și a zis:
\par 29 "Grăiește fiilor lui Israel și le zi: Cel ce își înfățișează Domnului jertfa sa de mântuire, acela din jertfa sa de mântuire să aducă o parte prinos Domnului,
\par 30 Și anume: Să aducă Domnului jertfă cu mâinile sale: grăsimea de pe pieptul jertfei și seul de pe ficat; să aducă legănând pieptul jertfei înaintea Domnului.
\par 31 Grăsimea s-o ardă preotul pe jertfelnic, iar pieptul va fi al lui Aaron și al fiilor lui.
\par 32 Și spata dreaptă din jertfele de izbăvire ce aduceți să o dați preotului.
\par 33 Spata dreaptă va fi partea aceluia din fiii lui Aaron, care va aduce pe jertfelnic sângele și grăsimea jertfei de izbăvire;
\par 34 Căci Eu voi lua de la fiii lui Israel, din jertfele lor de izbăvire, pieptul legănat și spata dreaptă și le voi da lui Aaron preotul și fiilor lui ca venit veșnic de la fiii lui Israel.
\par 35 Acestea sunt partea lui Aaron și partea fiilor lui din jertfele Domnului, pe care o vor primi din ziua când se vor înfățișa ei înaintea Domnului, ca să slujească,
\par 36 Și pe care a poruncit Domnul să li se dea de către fiii lui Israel din ziua ungerii lor. Aceasta este hotărâre veșnică în neamul lor".
\par 37 Aceasta este rânduiala arderii de tot, a darului de pâine, a jertfei pentru păcat, a jertfei pentru vină, a jertfei afierosirii și a jertfei de mântuire,
\par 38 Cum a dat-o Domnul lui Moise pe Muntele Sinai, când a poruncit fiilor lui Israel, în pustiul Sinai, să-și aducă prinoasele lor Domnului.

\chapter{8}

\par 1 Și a grăit Domnul cu Moise și a zis:
\par 2 "Ia pe Aaron și împreună cu el și pe fiii lui, veșmintele, mirul de miruit, vițelul de jertfa cea pentru păcat, panerul cu azimile și cei doi berbeci,
\par 3 Și adună toată obștea la ușa cortului adunării".
\par 4 Și a făcut Moise așa cum îi poruncise Domnul și a adunat obștea la ușa cortului adunării.
\par 5 După aceea a zis Moise către obște: "Iată ce poruncește Domnul să se facă!"
\par 6 Deci a adus Moise pe Aaron și pe fiii lui și i-a spălat cu apă:
\par 7 Apoi a îmbrăcat pe Aaron cu hitonul, l-a încins cu brâul, l-a îmbrăcat cu meilul, i-a pus efodul, l-a încins cu cingătoarea efodului și i-a strâns cu ea efodul;
\par 8 După aceea i-a pus hoșenul și în hoșen i-a pus Urim și Tumim,
\par 9 Iar pe cap i-a pus chidarul și la chidar, în partea lui de dinainte, i-a prins tăblița cea de aur, diadema sfințeniei, cum poruncise Domnul lui Moise.
\par 10 Apoi a luat Moise mirul de miruit și a miruit cortul și toate cele din el și le-a sfințit.
\par 11 A stropit cu el de șapte ori asupra jertfelnicului și a miruit jertfelnicul și toate obiectele lui, baia și căpătâiul ei și le-a sfințit.
\par 12 După aceea a turnat Moise mir pe capul lui Aaron și l-a uns și l-a sfințit.
\par 13 Și a adus Moise pe fiii lui Aaron, i-a îmbrăcat cu hitoane, i-a încins cu brâie și le-a pus turbane, cum poruncise Domnul lui Moise.
\par 14 Apoi a adus Moise vițelul cel de jertfă pentru păcat, iar Aaron și fiii lui și-au pus mâinile pe capul vițelului de jertfă pentru păcat;
\par 15 Și l-a junghiat Moise și a luat din sânge și cu degetul său a pus pe coarnele jertfelnicului de toate părțile și a curățit jertfelnicul, iar celălalt sânge l-a turnat jos lângă jertfelnic și a sfințit jertfelnicul ca să fie curat.
\par 16 A luat apoi Moise toată grăsimea de pe măruntaie, seul de pe ficat, amândoi rărunchii și grăsimea lor, și le-a ars pe jertfelnic.
\par 17 Iar vițelul, pielea lui, carnea lui și necurățenia lui, le-a ars cu foc afară din tabără, cum poruncise Domnul lui Moise.
\par 18 Apoi Moise a adus berbecul cel pentru ardere de tot, iar Aaron și fiii lui și-au pus mâinile pe capul berbecului.
\par 19 Și apoi a junghiat Moise berbecul și a stropit cu sânge jertfelnicul de jur împrejur.
\par 20 A tăiat apoi berbecul în bucăți și a adus Moise bucățile, căpățâna și grăsimea, iar măruntaiele și picioarele le-a spălat cu apă.
\par 21 Și a ars Moise tot berbecul pe jertfelnic; aceasta era ardere de tot spre mireasmă plăcută, aceasta era jertfă Domnului, cum poruncise Domnul lui Moise.
\par 22 După aceea a adus Moise al doilea berbec, berbecul cel pentru sfințire, și și-au pus Aaron și fiii lui mâinile pe capul berbecului.
\par 23 Și junghiindu-l, Moise a luat din sângele lui și a pus pe vârful urechii drepte a lui Aaron, pe degetul cel mare de la mâna dreaptă a lui și pe degetul cel mare de la piciorul drept al lui.
\par 24 Apoi a adus Moise pe fiii lui Aaron și a pus sânge pe vârful urechilor drepte ale lor, pe degetul cel mare de la mâinile drepte ale lor și pe degetul cel mare de la picioarele drepte ale lor; și a stropit Moise jertfelnicul cu sânge de jur împrejur.
\par 25 După aceea a luat Moise grăsimea și coada, toată grăsimea de pe măruntaie, seul de pe ficat, amândoi rărunchii, grăsimea lor și șoldul drept;
\par 26 Iar din panerul cu pâinile punerii înaintea Domnului a luat o azimă, o pâine cu untdelemn și turtă și le-a așezat peste grăsime și peste șoldul drept;
\par 27 Și toate acestea le-a pus pe mâinile lui Aaron și pe mâinile fiilor săi și le-au dus legănându-le înaintea feței Domnului.
\par 28 Apoi a luat Moise acestea din mâinile lor și le-a ars pe jertfelnic ardere de tot; aceasta a fost jertfa de sfințire, mireasmă plăcută, jertfă Domnului.
\par 29 Luând apoi pieptul, Moise l-a adus, legănându-l înaintea feței Domnului; aceasta era partea lui Moise din berbecul sfințirii, cum poruncise Domnul lui Moise.
\par 30 Apoi a luat Moise mir de miruit și sânge de lângă jertfelnic și a stropit pe Aaron, veșmintele lui, pe fiii lui și veșmintele fiilor lui împreună cu el; și așa a sfințit pe Aaron și veșmintele lui și, împreună cu el, și pe fiii lui și veșmintele fiilor lui.
\par 31 Apoi a zis Moise către Aaron și către fiii lui: "Fierbeți carnea la intrarea cortului adunării și acolo s-o mâncați cu pâinea cea din panerul sfințirii, după cum mi s-a poruncit mie și mi s-a zis: "Aaron și fiii lui s-o mănânce!"
\par 32 Iar rămășițele de carne și de pâine să le ardeți cu foc.
\par 33 Șapte zile să nu vă depărtați de la ușa cortului adunării, până se vor împlini zilele sfințirii voastre, că sfințirea voastră trebuie să se săvârșească în șapte zile.
\par 34 Cum s-a făcut astăzi, așa a poruncit Domnul să se facă pentru curățirea voastră și în celelalte zile.
\par 35 La ușa cortului adunării veți ședea ziua și noaptea timp de șapte zile și veți fi de strajă la Domnul, ca să nu muriți, că așa mi s-a poruncit mie de la Domnul Dumnezeu".
\par 36 Și au împlinit Aaron și fiii lui toate rânduielile, câte le poruncise Domnul prin Moise.

\chapter{9}

\par 1 Iar în ziua a opta a chemat Moise pe Aaron, pe fiii lui și pe bătrânii lui Israel,
\par 2 Și a zis către Aaron: "Ia-ți din turmă un vițel de jertfă pentru păcat și un berbec pentru arderea de tot, amândoi fără meteahnă, și-i adu înaintea feței Domnului.
\par 3 Iar bătrânilor lui Israel să le grăiești și să le spui: Luați din turma de capre un țap, ca jertfă pentru păcat, un berbec, un vițel și un miel, toți de un an și fără meteahnă, ca să fie aduși ardere de tot,
\par 4 Precum și un bou și un berbec pentru jertfă de mântuire, ca să se săvârșească jertfă înaintea feței Domnului, și prinos de pâine, frământat cu untdelemn, că astăzi are să vi se arate Domnul".
\par 5 Deci au luat ei și au adus înaintea cortului adunării cele ce poruncise Moise și a venit toată obștea și a stat înaintea feței Domnului.
\par 6 Atunci a zis Moise către obște: "Iată ce a poruncit Domnul să faceți, ca să vi se arate slava Domnului!"
\par 7 Iar către Aaron Moise a zis: "Apropie-te de jertfelnic și săvârșește jertfa ta cea pentru păcat și arderea de tot a ta și curățește-te pe tine și casa ta; apoi adu darurile poporului și-l curățește, cum a poruncit Domnul!"
\par 8 Deci, s-a apropiat Aaron de jertfelnic și a junghiat vițelul cel de jertfă pentru păcatul său.
\par 9 Iar fiii lui Aaron, aducând sângele la el, și-a muiat degetul în sânge și a pus pe coarnele jertfelnicului, iar celălalt sânge l-a turnat jos lângă jertfelnic.
\par 10 Grăsimea, rărunchii și seul de pe ficat, de la jertfa cea pentru păcat, le-a ars pe jertfelnic, cum poruncise Domnul lui Moise;
\par 11 Iar carnea și pielea le-a ars pe foc afară din tabără.
\par 12 Apoi a junghiat jertfa cea pentru arderea de tot și, aducându-i fiii lui Aaron sângele, a stropit jertfelnicul din toate părțile.
\par 13 După aceea i-au adus arderea de tot în bucăți și el le-a pus împreună cu căpățâna pe jertfelnic,
\par 14 Iar măruntaiele și picioarele le-a spălat cu apă și le-a pus peste arderea de tot și le-a ars pe jertfelnic.
\par 15 Apoi a adus prinosul poporului: luând țapul cel pentru păcatul poporului l-a junghiat și l-a adus jertfă pentru păcat, ca și berbecul.
\par 16 După aceea a adus arderea de tot, săvârșind-o după rânduială.
\par 17 A adus de asemenea prinos de pâine și, luând din el o mână plină, a ars pe jertfelnic, pe lângă arderea de tot cea de dimineață.
\par 18 Apoi a junghiat boul și berbecul pentru jertfa de mântuire a poporului și fiii lui Aaron i-au adus sângele și el a stropit cu el jertfelnicul de jur împrejur.
\par 19 I-au adus apoi grăsimea boului și a berbecului, coada și grăsimea cea de pe măruntaie, rărunchii cu grăsimea lor și grăsimea de pe ficat;
\par 20 și a pus grăsimea pe pieptul jertfei de izbăvire, apoi a ars grăsimea pe jertfelnic;
\par 21 Iar pieptul și spata dreaptă le-a adus Aaron legănându-le înaintea feței Domnului, cum poruncise Moise.
\par 22 Și și-a ridicat Aaron mâinile sale asupra poporului și l-a binecuvântat, iar după ce a săvârșit jertfa pentru păcat, arderea de tot și jertfa de mântuire, s-a coborât.
\par 23 Apoi au intrat Moise și Aaron în cortul adunării și când au ieșit au binecuvântat tot poporul; atunci s-a arătat slava Domnului la tot poporul.
\par 24 Și ieșind foc de la Domnul, a mistuit pe jertfelnic arderea de tot și grăsimea. Și văzând tot poporul a scos strigăte de bucurie și a căzut cu fața la pământ.

\chapter{10}

\par 1 În vremea aceea cei doi fii ai lui Aaron, Nadab și Abiud, luându-și fiecare cădelnița sa, au pus în ea foc, au turnat deasupra tămâie și au adus înaintea Domnului foc străin, ce nu le poruncise Domnul.
\par 2 Atunci a ieșit foc de la Domnul și i-a mistuit și au murit amândoi înaintea Domnului.
\par 3 Iar Moise a zis către Aaron: "Iată ce a voit să spună Domnul când a zis: Voi să fiu sfințit prin cei ce se vor apropia de Mine și înaintea adunării a tot poporul preaslăvit". Iar Aaron tăcea.
\par 4 Atunci a chemat Moise pe Misail și Elțafan, fiii lui Uziel, unchiul lui Aaron, și le-a zis: "Duceți-vă de scoateți pe frații voștri din locașul cel sfânt și-i duceți afară din tabără!"
\par 5 Și aceștia s-au dus și i-au scos în hitoanele lor afară din tabără, cum zisese Moise.
\par 6 Iar lui Aaron și fiilor lui, Eleazar și Itamar, le-a zis Moise; "Capetele voastre să nu vi le descoperiți și veșmintele voastre să nu vi le sfâșiați, ca să nu muriți și ca să nu atrageți mânia asupra obștii întregi. Dar frații voștri, toată casa lui Israel, pot să plângă pe cei arși, pe care i-a ars Domnul.
\par 7 Din ușa cortului adunării să nu ieșiți, ca să nu muriți, căci aveți pe voi mirul de ungere al Domnului!" Și s-a făcut drept cuvântul lui Moise.
\par 8 Apoi grăind Domnul cu Aaron, a zis:
\par 9 "Vin și sicheră să nu beți, nici tu, nici fiii tăi, când intrați în cortul adunării sau vă apropiați de jertfelnic, ca să nu muriți. Acesta este așezământ veșnic în neamul vostru.
\par 10 Ca să puteți deosebi cele sfinte de cele nesfinte și cele necurate de cele curate,
\par 11 Și ca să învățați pe fiii lui Israel toate legile, pe care le-a poruncit lor Domnul prin Moise".
\par 12 Iar Moise a zis către Aaron și către fiii săi, Eleazar și Itamar, care-i mai rămăseseră: "Luați prinosul de pâine, ce a rămas din jertfele Domnului, și-l mâncați nedospit, lângă jertfelnic, că acesta este sfințenie mare.
\par 13 Să-l mâncați însă în locul cel sfânt, că aceasta este partea ta și partea fiilor tăi din jertfele Domnului; așa mi s-a poruncit mie de la Domnul.
\par 14 Iar pieptul legănat și spata ridicată să le mâncați la loc curat, tu și fiii tăi și casa ta împreună cu tine, că acestea sunt date să fie partea ta și partea fiilor tăi din jertfele de izbăvire ale fiilor lui Israel.
\par 15 Spata ridicată și pieptul legănat să le aducă ei cu grăsime pentru ardere, legănându-le înaintea feței Domnului, și să fie acestea parte veșnică pentru tine și împreună cu tine și pentru fiii tăi și pentru fiicele tale, cum a poruncit Domnul lui Moise".
\par 16 Și a căutat Moise țapul de jertfă pentru păcat și iată era ars. Și s-a mâniat Moise pe Eleazar și pe Itamar, fiii lui Aaron, care mai rămăseseră,
\par 17 și a zis: "Pentru ce n-an mâncat jertfa pentru păcat în locul cel sfânt? Că aceasta este sfințenie mare și vă e dată vouă, ca să ridicați păcatele obștii și s-o curățiți înaintea Domnului.
\par 18 Iată sângele ei nu s-a dus înăuntrul locașului sfânt și voi trebuia s-o mâncați în locul cel sfânt, cum mi s-a poruncit mie de la Domnul".
\par 19 Aaron însă a zis către Moise: "Iată, astăzi și-au adus ei jertfa lor pentru păcat și arderea lor de tot înaintea Domnului și, după cele ce mi s-au întâmplat, de aș fi mâncat astăzi jertfa pentru păcat, oare ar fi fost aceasta plăcut Domnului?"
\par 20 Și a auzit acestea Moise și le-a socotit răspunsul îndreptățit.

\chapter{11}

\par 1 În vremea aceea a grăit Domnul cu Moise și cu Aaron și a zis:
\par 2 "Grăiți fiilor lui Israel și le ziceți: Iată animalele pe care le puteți mânca din toate dobitoacele de pe pământ:
\par 3 Orice animal cu copita despicată, care are copita despărțită în două și își rumegă mâncarea, îl puteți mânca.
\par 4 Dar și din cele ce-și rumegă mâncarea, sau își au copita despicată sau împărțită în două, nu veți mânca pe acestea: cămila, pentru că aceasta-și rumegă mâncarea, dar copita n-o are despicată; aceasta e necurată pentru voi.
\par 5 Iepurele de casă își rumegă mâncarea, dar laba n-o are despicată; acesta este necurat pentru voi.
\par 6 Iepurele de câmp își rumegă mâncarea, dar laba n-o are despicată; acesta este necurat pentru voi.
\par 7 Porcul are copita despicată și despărțită în două, dar nu rumegă; acesta este necurat pentru voi.
\par 8 Din carnea acestora să nu mâncați și de stârvurile lor să nu vă atingeți, că acestea sunt necurate pentru voi.
\par 9 Din toate viețuitoarele, care sunt în apă, veți mânca pe acestea: toate câte sunt în ape; în mări, în râuri și în bălți, și au aripi și solzi, pe acelea să le mâncați.
\par 10 Iar toate câte sunt în ape, în mări, în râuri, și în bălți, toate cele ce mișună în ape, dar n-au aripi și solzi, spurcăciune sunt pentru voi.
\par 11 De acestea să vă îngrețoșați, carnea lor să n-o mâncați și de stârvurile lor să vă îngrețoșați.
\par 12 Toate vietățile din ape, care n-au aripi și solzi, sunt spurcate pentru voi.
\par 13 Din păsări să nu mâncați și să vă îngrețoșați de acestea, că sunt spurcate: vulturul, zgripțorul și vulturul de mare;
\par 14 Corbul și șoimul cu soiurile lor;
\par 15 Toată cioara cu soiurile ei;
\par 16 Struțul, cucuveaua, rândunica și uliul cu soiurile lui;
\par 17 Huhurezul, pescarul și ibisul;
\par 18 Lebăda, pelicanul și cocorul;
\par 19 Cocostârcul, bâtlanul cu soiurile lui; pupăza și liliacul.
\par 20 Toate insectele înaripate, care umblă pe patru picioare, sunt spurcate pentru voi.
\par 21 Dar din toate insectele înaripate, care umblă în patru picioare, să mâncați numai pe acelea care au fluierele picioarelor de dinapoi mai lungi, ca să poată sări pe pământ.
\par 22 Din acestea să mâncați următoarele: lăcusta și soiurile ei, solamul și soiurile lui, hargolul și soiurile lui, și hagabul cu soiurile lui.
\par 23 Orice altă insectă înaripată care are patru picioare e spurcată pentru voi și vă spurcați de ele.
\par 24 Tot cel ce se va atinge de trupul lor necurat va fi până seara;
\par 25 Și tot cel ce va lua în mâini trupul lor să-și spele haina și necurat va fi până seara.
\par 26 Tot dobitocul cu copita despicată, care n-are copita despărțită adânc sau nu-și rumegă mâncarea, este necurat pentru voi; tot cel ce se va atinge de el necurat va fi până seara.
\par 27 Din toate fiarele cu patru picioare, cele care calcă pe labe sunt necurate pentru voi și tot cel ce se va atinge de stârvul lor necurat va fi până seara.
\par 28 Cel ce va umbla cu stârvul lor să-și spele haina și necurat va fi până seara, căci ele sunt necurate pentru voi.
\par 29 Din dobitoacele ce mișună pe pământ, iată care sunt necurate pentru voi: cârtița, șoarecele și șopârla, cu soiurile lor;
\par 30 Ariciul, crocodilul, salamandra, melcul și cameleonul.
\par 31 Acestea dintre toate cele ce mișună pe pământ sunt necurate pentru voi. Tot cel ce se atinge de stârvurile lor necurat va fi până seara.
\par 32 Tot lucrul, pe care va cădea vreuna din acestea, moartă, fie vas de lemn, sau haină, sau piele, sau orice fel de lucru ce se întrebuințează la ceva, lucrul acela necurat va fi; să-l puneți în apă și va fi necurat până seara, iar apoi va fi curat.
\par 33 Tot vasul de lut, în care va cădea vreuna din ele, să-l spargeți, iar cele din el sunt necurate.
\par 34 Orice lucru de mâncare, peste care "a cădea apă din acel vas, necurat va fi pentru voi și toată băutura de băut, din asemenea vas, necurată va fi.
\par 35 Tot lucrul, peste care va cădea ceva din trupul mort al acestora, se va spurca; soba și căminul să le stricați, că necurate sunt și necurate vor fi pentru voi.
\par 36 Numai izvorul, fântâna și adunările de apă vor rămâne curate, iar cel ce se va atinge de mortăciunile din ele, acela necurat va fi.
\par 37 De va cădea ceva din trupul acestora pe sămânța de semănat, aceasta curată va fi.
\par 38 Dacă însă va cădea ceva din trupul lor peste sămânță, după ce aceasta s-a muiat cu apă, atunci sămânța necurată să fie pentru voi.
\par 39 Iar de va muri vreun dobitoc din cele ce se mănâncă și se va atinge cineva de stârvul lui, acela necurat va fi până seara;
\par 40 Iar cel ce va mânca mortăciunea lui, să-și spele hainele sale și necurat va fi până seara; cel ce va duce stârvul lui să-și spele hainele sale și necurat va fi până seara.
\par 41 Toată vietatea ce se târăște pe pământ este spurcată pentru voi; să n-o mâncați.
\par 42 Tot ce se târăște pe pântece și tot ce umblă în patru picioare și cele cu picioare multe dintre vietățile ce se târăsc pe pământ, să nu le mâncați, că sunt spurcate pentru voi.
\par 43 Să nu vă spurcați sufletele voastre cu vreo vietate târâtoare și să nu vă pângăriți cu ea, ca să fiți din pricina ei necurați,
\par 44 Că Eu sunt Domnul Dumnezeul vostru. Sfințiți-vă și veți fi sfinți, că Eu, Domnul Dumnezeul vostru, sfânt sunt; să nu vă pângăriți sufletele voastre cu vreo vietate din cele ce se târăsc pe pământ,
\par 45 Că Eu sunt Domnul, Cel ce v-am scos din pământul Egiptului, ca să vă fiu Dumnezeu. Deci fiți sfinți, că Eu, Domnul, sunt sfânt".
\par 46 Aceasta este legea cea pentru dobitoace, pentru păsări, pentru toate vietățile ce mișună în apă și pentru toate vietățile ce trăiesc pe pământ,
\par 47 După care se pot deosebi cele necurate de cele curate și vietățile ce se mănâncă de vietătile ce nu se mănâncă.

\chapter{12}

\par 1 Și a grăit Domnul lui Moise și a zis:
\par 2 "Grăiește fiilor lui Israel și le zi: Dacă femeia va zămisli și va naște prunc de parte bărbătească, necurată va fi șapte zile, cum e necurată și în zilele regulei ei.
\par 3 Iar în ziua a opta se va tăia pruncul împrejur.
\par 4 Femeia să mai șadă treizeci și trei de zile și să se curățe de sângele său; de nimic sfânt să nu se atingă, și la locașul sfânt să nu meargă, până se vor împlini zilele curățirii ei.
\par 5 Iar de va naște fată, necurată va fi două săptămâni, ca și în timpul regulei ei; apoi să mai stea șaizeci și șase de zile pentru a se curăți de sângele său.
\par 6 După ce se vor împlini zilele curățirii ei pentru fiu sau pentru fiică, să aducă preotului la ușa cortului un miel de un an ardere de tot și un pui de porumbel sau o turturică, jertfă pentru păcat;
\par 7 Preotul va înfățișa acestea înaintea Domnului și o va curăți și curată va fi de curgerea sângelui ei. Aceasta e rânduiala pentru ceea ce a născut prunc de parte bărbătească sau de parte femeiască.
\par 8 Iar de nu-i va da mâna să aducă un miel, să ia două turturele sau doi pui de porumbel, unul pentru ardere de tot și altul jertfă pentru păcat, și o va curăți preotul și curată va fi".

\chapter{13}

\par 1 Grăit-a Domnul cu Moise și cu Aaron și le-a zis:
\par 2 "De se va ivi la vreun om pe pielea trupului lui vreo umflătură, sau bubă, sau bășică, sau de se va face pe pielea trupului o rană ca de lepră, să fie adus la Aaron preotul sau la un preot din fiii lui.
\par 3 Preotul va cerceta rana de pe pielea trupului lui și de va vedea că perii de pe rană s-au făcut albi și că rana s-a adâncit în pielea trupului, aceea este rană de lepră, iar preotul după ce l-a cercetat, îl va declara necurat.
\par 4 Iar dacă pata de pe piele, deși este albă, dar nu este și adâncită în pielea lui, și perii de pe ea nu s-au făcut albi, ci sunt negri, să închidă preotul pe cel cu rana șapte zile.
\par 5 În ziua a șaptea să vadă preotul rana: dacă rana a rămas ca înainte și nu s-a întins rana pe piele, preotul să-l închidă alte șapte zile.
\par 6 În ziua a șaptea îl va cerceta preotul din nou și dacă rana va fi slăbită și nu se va fi întins rana pe piele, preotul să-l declare curat. Aceasta este o bubă și cel ce o are să-și spele hainele sale și va fi curat.
\par 7 Iar dacă, după ce omul s-a arătat preotului, din nou buba a început a se întinde pe piele, să se arate iar preotului;
\par 8 Preotul, văzând că buba se întinde pe piele, îl va declara necurat, că aceasta este lepră.
\par 9 De se va ivi pe un om boala leprei, acela să fie adus la preot.
\par 10 Preotul va cerceta și, dacă umflătura de pe piele va fi albă și părul va fi schimbat în alb și dacă umflătura va fi carne vie,
\par 11 Aceea e lepră învechită pe pielea trupului; preotul il va declara necurat și nu-l va închide, că este necurat.
\par 12 Dacă însă lepra va înflori pe piele și dacă va acoperi lepra toată pielea bolnavului de la cap până la picioare, cât poate să vadă preotul cu ochii,
\par 13 Și dacă va vedea preotul că lepra a acoperit toată pielea trupului, atunci va declara pe bolnav curat, pentru că tot s-a schimbat în alb și deci este curat.
\par 14 Iar în ziua când se va ivi pe el carne vie, va fi necurat,
\par 15 Și preotul, văzând carnea vie, îl va declara necurat, căci carnea cea vie este necurată, este lepră.
\par 16 Iar dacă se va schimba carnea cea vie și se va face albă, să vină bolnavul la preot,
\par 17 Și preotul să-l cerceteze și dacă rana s-a schimbat în alb, atunci preotul să-l declare curat, că e curat.
\par 18 Dacă cineva a avut pe pielea trupului o bubă și s-a vindecat,
\par 19 Și pe locul bubei s-a ivit o umflătură albă sau o pată albă-roșiatică, să se arate preotului.
\par 20 Și preotul să-l cerceteze și de se va vedea că umflătura s-a adâncit în piele și părul de pe ea s-a schimbat în alb, preotul îl va declara necurat, că aceasta e lepră și s-a ivit în locul bubei.
\par 21 Dacă însă preotul va vedea că părul de pe umflătură nu este alb și ea nu este adâncită în pielea trupului și e negricioasă, atunci preotul va închide pe bolnav pentru șapte zile.
\par 22 Dacă rana va începe a se lăți tare pe piele, preotul îl va declara necurat, că este rană de lepră.
\par 23 Iar dacă pata va rămâne pe loc și nu se va lăți, atunci e o oprire în loc a bubei și preotul va declara pe bolnav curat.
\par 24 Dacă cineva va avea pe pielea trupului o arsură și pe locul tămăduit de arsură se va ivi o pată roșiatică-albicioasă,
\par 25 Și dacă preotul va vedea că părul de pe acea pată s-a schimbat în alb și că pata e adâncită sub piele, aceea este lepră și s-a ivit pe arsură; preotul va declara pe bolnav necurat, căci e boala leprei.
\par 26 Dacă însă preotul va vedea că părul de pe pată nu este alb și că ea nu este adâncită sub piele și că este negricioasă, preotul va închide pe acela pentru șapte zile;
\par 27 Și în ziua a șaptea preotul îl va cerceta iar și, dacă pata s-a lățit tare pe piele, preotul îl va declara necurat, că aceea este rană de lepră.
\par 28 Iar dacă pata stă pe loc și este negricioasă, aceea este obrinteala arsurii și preotul va declara pe om curat, că este obrinteală a arsurii.
\par 29 Dacă un bărbat sau o femeie va avea o pată pe cap sau pe bărbie,
\par 30 Și, cercetând-o preotul, se va vedea că ea este adâncită sub piele și că părul de pe ea este gălbui și subțire, preotul va declara pe unul ca acela necurat, că aceea este chelbe, lepră în cap, sau lepră în barbă.
\par 31 Dacă însă preotul, la cercetarea petei de chelbe, va vedea că ea nu este adâncită sub piele și că părul de pe ea nu este gălbui, preotul va închide pe cel cu pata de chelbe șapte zile;
\par 32 în ziua a șaptea preotul va cerceta pata iar și, dacă chelbea nu s-a întins și n-are părul de pe ea gălbui și nici nu s-a adâncit chelbea sub piele,
\par 33 Atunci să radă pielea, dar locul cu chelbe să nu-l radă, și preotul să închidă pe cel cu pata a doua oară pentru șapte zile.
\par 34 În ziua a șaptea preotul va cerceta din nou chelbea și, dacă chelbea nu se va fi întins pe piele și nu se va fi adâncit în piele, preotul va declara pe acela curat și acela să-și spele hainele sale și va fi curat.
\par 35 Iar dacă, după această curățire a lui, chelbea va începe a se lăți foarte tare pe piele,
\par 36 Și dacă preotul va vedea că chelbea se lățește pe piele, atunci preotul să nu mai caute de e părul gălbui, că acela este necurat.
\par 37 Dacă însă chelbea stă pe loc și se ivește pe ea păr negru, atunci chelbea a trecut, omul e curat și preotul îl va declara curat.
\par 38 Dacă un bărbat sau o femeie va avea pe pielea trupului pete, pete albe,
\par 39 Și dacă preotul va vedea că pe pielea trupului aceluia petele sunt albe-vinete, aceea e pecingine care a înflorit pe piele și omul ce o are este curat.
\par 40 Dacă cuiva i-a căzut părul de pe cap, aceea e pleșuvie și omul este curat.
\par 41 Dacă cuiva i-a căzut părul din partea de dinainte a capului, aceea este jumătate de pleșuvie și omul e curat.
\par 42 Iar dacă pe pleșuvia din partea de dinainte sau de dinapoi va fi pată albă sau roșiatică, atunci pe pleșuvia lui a înflorit lepra.
\par 43 Preotul îl va cerceta și de va vedea că fața umflăturii de pe pleșuvia lui este albă sau roșiatică, semănând cu lepra, ce de obicei se ivește pe pielea trupului,
\par 44 Acela este om lepros și este necurat; preotul să-l declare necurat, că pe capul lui este boala leprei.
\par 45 Leprosul, cel ce are această boală, să fie cu hainele sfâșiate, cu capul descoperit, învelit până la buze, și să strige mereu: necurat! necurat!
\par 46 Tot timpul cât va avea pe el boala, să fie spurcat, că necurat este; și să trăiască singuratic și afară din tabără să fie locuința lui.
\par 47 Dacă boala leprei va fi pe haină, fie pe haină de lână, sau pe haină de in,
\par 48 Sau pe urzeală, sau pe bătătură de in sau de lână, sau pe piele sau pe vreun lucru de piele,
\par 49 Și dacă va fi pată verzuie sau roșiatică pe haină sau pe piele, pe bătătură sau pe urzeală, sau pe vreun lucru de piele, aceea este boala leprei, și el se va arăta preotului.
\par 50 Preotul va cerceta boala și va închide lucrul atins de boală pentru șapte zile;
\par 51 În ziua a șaptea va cerceta preotul lucrul atins de boală și dacă boala se va fi întins pe haină, sau pe urzeală, sau pe bătătură, sau pe piele, sau pe vreun lucru de piele, aceasta este lepră rozătoare, și e necurat;
\par 52 Și el să ardă haina aceea, sau urzeala, sau bătătura cea de lână sau de in, sau orice fel de lucru din piele, pe care va fi boala, că aceea este lepră rozătoare și să se ardă cu foc.
\par 53 Iar dacă preotul va vedea că boala nu s-a întins pe haină, sau pe urzeală, sau pe bătătură, sau pe orice fel de lucru din piele,
\par 54 Atunci preotul va porunci să se spele lucrul pe care s-a ivit boala și-l va închide a doua oară pentru șapte zile.
\par 55 Dacă, după spălarea lucrului atins, preotul va vedea că boala nu și-a schimbat starea sa, dar s-a întins, atunci este necurat și lucrul să-l arzi în foc, căci lepra a ros fața sau dosul.
\par 56 Dacă însă preotul va vedea că pata, după spălarea ei, s-a micșorat, atunci preotul s-o rupă de la haină, sau din piele, sau din urzeală, sau din bătătură.
\par 57 Iar dacă se va ivi iar pe haină, sau pe bătătură, sau pe urzeală, sau pe vreun lucru de piele, aceea este lepră înflorită și să se ardă cu foc lucrul pe care s-a ivit boala.
\par 58 Dacă însă haina, sau urzeala, sau bătătura, sau lucrul de piele îl vei spăla și se va duce pata de pe el, trebuie să se spele a doua oară și va fi curat.
\par 59 Aceasta este rânduiala pentru boala leprei, ce se va ivi pe haină de lână sau de in, sau pe urzeală, sau pe bătătură, sau pe vreun lucru de piele, și cum trebuie hotărât că acestea sunt curate sau necurate".

\chapter{14}

\par 1 Și grăind cu Moise, Domnul a zis:
\par 2 "Iată rânduiala pentru cel lepros: Când el se va curăți, se va duce la preot;
\par 3 Iar preotul va ieși afară din tabără și de va vedea preotul că leprosul s-a vindecat de boala leprei,
\par 4 Va porunci preotul să se ia pentru cel curățit două păsări vii, curate, lemn de cedru, ață roșie răsucită și isop.
\par 5 După aceea preotul va porunci să se junghie una din păsări deasupra unui vas de lut, la apă curgătoare;
\par 6 Va lua apoi pasărea cea vie, lemnul de cedru, ața cea roșie și isopul și le va muie pe acestea și pasărea cea vie în sângele păsării junghiate la apă curgătoare;
\par 7 Va stropi de șapte ori pe cel ce se curăță de lepră și va fi curat; apoi va da drumul păsării celei vii în câmp.
\par 8 Iar cel curățit să-și spele hainele sale, să-și tundă tot părul său, să se spele cu apă și va fi curat. Apoi să intre în tabără și să stea șapte zile afară din cortul său.
\par 9 În ziua a șaptea să-și radă tot părul său, capul și barba sa, sprâncenele sale, tot părul său să și-l radă, și să-și spele iarăși hainele sale și trupul său să și-l spele cu apă și va fi curat.
\par 10 În ziua a opta să ia doi berbeci de câte un an, fără meteahnă și o oaie de un an, fără meteahnă, și dintr-o efă de făină de grâu, împărțită în zece, să ia trei părți pentru darul de pâine și s-o amestece cu untdelemn și un log (pahar) de untdelemn;
\par 11 Iar preotul cel ce curățește va duce pe omul ce se curățește împreună cu acestea înaintea Domnului, la ușa cortului adunării;
\par 12 Acolo va lua preotul un berbec, ce voiește a aduce jertfă pentru vină, și logul de untdelemn și le va aduce pe acestea legănându-le înaintea Domnului.
\par 13 Berbecul îl va junghia în locul acela, unde se junghie jertfele pentru păcat și pentru arderea de tot, la loc sfânt, că aceasta este jertfă pentru vină și, ca și jertfa pentru păcat, este a preotului și este sfințenie mare.
\par 14 Apoi va lua preotul din sângele jertfei pentru vină și va pune preotul pe vârful urechii drepte a celui ce se curățește, pe degetul cel mare de la mâna dreaptă a lui și pe degetul cel mare de la piciorul cel drept al lui.
\par 15 De asemenea va lua preotul din logul de untdelemn și va turna în palma sa cea stângă;
\par 16 Își va muia preotul degetul mâinii sale drepte în untdelemnul cel din palma stângă a sa și va stropi de șapte ori cu degetul său înaintea feței Domnului;
\par 17 Apoi din untdelemnul rămas în palma lui va pune preotul pe vârful urechii drepte a celui ce se curățește, pe degetul cel mare al mâinii lui drepte și pe degetul cel mare de la piciorul cel drept al lui, pe locurile unde a pus și sângele jertfei pentru vină;
\par 18 Iar celălalt untdelemn din palma preotului îl va turna pe capul celui ce se curățește și-l va curăți preotul pe acesta înaintea Domnului.
\par 19 Astfel va săvârși preotul jertfa pentru păcat și va curăți pe cel ce a venit să se curețe de necurățenia lui; după aceea va junghia jertfa arderii de tot;
\par 20 Și va pune preotul arderea de tot și darul de pâine pe jertfelnic. Astfel îl va curăți pe el preotul și el va fi curat.
\par 21 Dacă însă acela va fi sărac și nu-i va da mâna, să ia numai un berbec pentru jertfa de vină legănată pentru curățirea sa, a zecea parte dintr-o efă de făină de grâu, amestecată cu untdelemn pentru darul de pâine, un log de untdelemn
\par 22 Și două turturele sau doi pui de porumbel, cum îi va da mâna: unul jertfă pentru păcat și altul ardere de tot.
\par 23 Îi va aduce în ziua a opta cea pentru curățirea sa la preot, înaintea Domnului, la ușa cortului adunării.
\par 24 Iar preotul, luând berbecul de jertfă pentru vină și logul de untdelemn, le va aduce pe acestea legănându-le înaintea Domnului.
\par 25 Apoi va junghia berbecul de jertfă pentru vină și va lua preotul din sângele jertfei pentru vină și va pune pe vârful urechii drepte a celui ce se curăță, pe degetul cel mare de la mâna lui cea dreaptă și pe degetul cel mare de la piciorul lui cel drept.
\par 26 Și va turna preotul untdelemn în palma sa cea stângă;
\par 27 Și cu untdelemn din palma sa cea stângă va stropi preotul de șapte ori cu degetul mâinii sale celei drepte înaintea feței Domnului;
\par 28 Apoi va pune preotul untdelemn din palma sa cea stângă pe marginea urechii drepte a celui ce se curățește și pe degetul cel mare de la mâna lui cea dreaptă și pe degetul cel marc de la piciorul lui cel drept, pe locurile unde este pus și sângele jertfei pentru vină;
\par 29 Iar celălalt untdelemn din palma sa cea stângă îl va turna pe capul celui ce se curățește, ca să-l curețe înaintea Domnului.
\par 30 Și turturelele sau puii de porumbel, cum îi va fi dat mâna celui ce se curățește, după starea lui, le va aduce:
\par 31 O pasăre jertfă pentru păcat și alta pentru ardere de tot, împreună cu darul de pâine. Și așa va curăți preotul pe cel ce se curățește înaintea Domnului.
\par 32 Aceasta este rânduiala pentru cel bolnav de lepră, căruia nu-i dă mâna să ducă tot ce se cere pentru curățirea sa".
\par 33 Și a grăit Domnul cu Moise și Aaron și a zis:
\par 34 "Când veți intra în pământul Canaanului, pe care-l voi da vouă de moștenire, și voi aduce boala leprei asupra caselor din pământul moștenirii voastre,
\par 35 Atunci cel cu casa trebuie să se ducă și să spună preotului, zicând: Pe casa mea s-a ivit, pare-mi-se, boala.
\par 36 Atunci preotul va porunci să se golească casa înainte de a intra preotul să cerceteze boala, ca să nu se facă necurate toate cele din casă; după aceea va veni preotul să cerceteze casa.
\par 37 Și cercetând el boala, dacă va vedea că boala de pe pereții casei e în chip de gropi verzui sau roșietice, adâncite în perete,
\par 38 Va ieși din casă, la ușa casei, și va închide casa pentru șapte zile.
\par 39 în ziua a șaptea va veni preotul iar să cerceteze casa și de va vedea că boala s-a întins pe pereții casei,
\par 40 Preotul va porunci să se scoată pietrele pe care este boala, să se arunce afară din oraș, la loc necurat,
\par 41 Casa să se răzuiască toată pe dinăuntru, iar răzătura, ce se va răzui, să se arunce afară din oraș, la loc necurat.
\par 42 Să aducă apoi alte pietre și să le pună în locul pietrelor acelora; să ia altă tencuială și casa să se tencuiască.
\par 43 Dacă boala se va ivi iar și va înflori pe pereții casei, după ce s-au scos pietrele și s-a răzuit și s-a tencuit casa,
\par 44 Atunci preotul va veni iar și va cerceta și de s-a răspândit boala pe pereții casei, aceea este lepră rozătoare și casa este necurată.
\par 45 Casa aceea să se dărâme, iar pietrele ei, lemnul ei și toată tencuiala să se scoată afară din oraș, la loc necurat.
\par 46 Cel ce va intra în casa aceea, cât va fi ea închisă, acela necurat va fi până seara.
\par 47 Cel ce va dormi în casa aceea să-și spele hainele sale și necurat va fi până seara; și cel ce va mânca în casa aceea să-și spele hainele și necurat va fi până seara.
\par 48 Dacă însă preotul, venind și intrând, va vedea că boala de pe pereții casei nu s-a mai întins după ce aceasta a fost tencuită din nou, preotul o va declara curată, că boala a trecut.
\par 49 Ca să curețe casa, va lua deci două păsări vii, curate, lemn de cedru, ață roșie răsucită și isop;
\par 50 Va junghia o pasăre deasupra unui vas de lut, la apă curgătoare.
\par 51 Va lua lemnul cel de cedru, ața, isopul și pasărea vie și le va muia în sângele păsării junghiate și în apa de izvor și va stropi casa de șapte ori.
\par 52 Și va curăți astfel casa cu sângele păsării, cu apă de izvor, cu pasărea cea vie, cu lemnul cel de cedru, cu ața roșie răsucită și cu isop.
\par 53 Iar păsării celei vii îi va da drumul din cetate în câmp și se va curăți casa și curată va fi.
\par 54 Aceasta este rânduiala pentru oricare fel de boală a leprei și a chelbei.
\par 55 Pentru lepra de pe haine și de pe case,
\par 56 Și pentru umflături, pecingine și pete,
\par 57 Ca să se poată afla când acestea sunt necurate și când sunt curate: aceasta este rânduiala pentru lepră".

\chapter{15}

\par 1 Și a grăit Domnul cu Moise și cu Aaron, zicând:
\par 2 "Grăiți fiilor lui Israel și le spuneți: Dacă un bărbat va avea curgere din trupul său, pentru curgerea lui este necurat,
\par 3 Și legea necurăției lui este aceasta: Ori de se face curgere din trupul lui, ori de este împiedicată curgerea în trupul lui, el este necurat.
\par 4 Tot patul, pe care doarme cel ce are curgere, este necurat; tot lucrul, pe care va ședea cel ce are curgere, este necurat.
\par 5 Omul, care se va atinge de patul lui, să-și spele hainele sale, să se spele cu apă și va fi necurat până seara.
\par 6 Cel ce va ședea pe vreun lucru, pe care a șezut cel ce are curgere, să-și spele hainele sale, să se spele cu apă și va fi necurat până seara.
\par 7 Cel ce se va atinge de trupul celui ce are curgere să-și spele hainele sale, să se spele cu apă și necurat va fi până seara.
\par 8 Dacă cel ce are curgere va scuipa pe unul curat, acesta să-și spele hainele, să se spele cu apă și necurat va fi până seara.
\par 9 Toată șaua, pe care va călări cel ce are curgere, necurată va fi până seara.
\par 10 Tot cel ce se atinge de câte au fost sub acela va fi necurat până seara, iar cel ce va ridica acestea să-și spele hainele sale, să se spele cu apă și necurat va fi până seara.
\par 11 Acela, de care se va atinge cel ce are curgere, fără să-și fi spălat mâinile cu apă, să-și spele hainele sale, să se spele cu apă și va fi necurat până seara.
\par 12 Vasul de lut, de care s-a atins cel ce are curgere, să se spargă și tot vasul de lemn să se spele cu apă și va fi curat.
\par 13 Iar când cel ce are curgere se va curăți de curgerea sa să numere șapte zile pentru curățirea sa, să-și spele hainele sale, să-și spele trupul cu apă de izvor și va fi curat.
\par 14 Apoi în ziua a opta să-și ia două turturele sau doi pui de porumbel, să vină înaintea feței Domnului, la ușa cortului adunării și să le dea preotului;
\par 15 Iar preotul să aducă din ele: una jertfă pentru păcat și una ardere de tot; și să-l curețe preotul înaintea Domnului de curgerea lui.
\par 16 Dacă un om va avea din întâmplare curgerea seminței, acela să-și spele cu apă tot trupul său și va fi necurat până seara.
\par 17 Orice haină și orice piele, pe care va cădea sămânța, să se spele cu apă și necurată va fi până seara.
\par 18 Dacă bărbatul se va culca cu femeia și va avea el curgerea seminței, să se spele amândoi cu apă și necurați să fie până seara.
\par 19 De va avea femeia curgere de sânge, care curge din trupul său, trebuie să stea șapte zile pentru curățirea sa. Tot cel ce se va atinge de ea, necurat va fi până seara.
\par 20 Tot lucrul pe care se va culca ea în timpul necurăției va fi necurat și tot lucrul pe care va ședea va fi necurat.
\par 21 Tot cel ce se va atinge de patul ei să-și spele hainele sale, să se spele cu apă și necurat va fi până seara.
\par 22 Tot cel ce se va atinge de vreun lucru, pe care a șezut ea, să-și spele hainele sale, să se spele cu apă și necurat va fi până seara.
\par 23 Iar de se va atinge cineva de ceva din patul ei sau de lucrul pe care a șezut ea, acela necurat va fi până seara.
\par 24 De va dormi ea cu bărbatul, necurăția ei va fi și pe el și necurat va fi el șapte zile, iar tot patul, în care va dormi, necurat va fi.
\par 25 Dacă femeii îi va curge sânge mai multe zile și nu în timpul regulii ei, sau dacă ea are curgere mai mult decât timpul regulii ei obișnuite, atunci în tot timpul curgerii necurăției ei va fi necurată, ca și în timpul regulii ei.
\par 26 Tot patul, în care va dormi în timpul curgerii ei, va fi necurat, cum e patul și în timpul regulii ei, și tot lucrul pe care va ședea ea va fi necurat, cum e necurat în timpul regulii ei.
\par 27 Tot cel ce se va atinge de acel lucru va fi necurat: să-și spele hainele sale, să-și spele trupul cu apă și va fi necurat până seara.
\par 28 Iar când se va izbăvi ea de curgerea sa, să se curețe șapte zile și după aceea va fi curată.
\par 29 În ziua a opta să-și ia două turturele sau doi pui de porumbei și să-i aducă preotului, la ușa cortului adunării
\par 30 Iar preotul va aduce una din păsări jertfă pentru păcat și pe cealaltă ardere de tot; și s-o curețe preotul înaintea Domnului de curgerea ei cea necurată.
\par 31 Așa să feriți pe fiii lui Israel de necurățenia lor, ca să nu moară ei în necurățenia lor, spurcându-Mi locașul Meu cel din mijlocul vostru.
\par 32 Aceasta este rânduiala pentru cel ce are curgere și pentru cel ce i se va întâmpla pierderea seminței, care-l face necurat,
\par 33 Și pentru ceea ce suferă de regula sa și pentru cei ce au curgere, bărbat sau femeie, și pentru bărbatul ce doarme cu femeie necurată".

\chapter{16}

\par 1 După moartea celor doi fii ai lui Aaron, care au murit când au adus foc străin înaintea feței Domnului, a grăit Domnul cu Moise;
\par 2 Și a zis Domnul către Moise: "Spune lui Aaron, fratele tău, să nu intre oricând în locașul sfânt de după perdea, înaintea curățitorului celui de pe chivotul legii, ca să nu moară, că deasupra capacului Mă voi arăta în nor.
\par 3 Iată rânduiala după care trebuie să intre Aaron în locașul sfânt: cu un vițel, jertfă pentru păcat, și cu un berbec pentru ardere de tot.
\par 4 Să se îmbrace cu hitonul de in sfințit, să aibă pe trupul lui pantaloni de in, să fie încins cu brâu de in și să-și ia și chidar de in: acestea sunt veșmintele sfințite; dar să-și spele tot trupul său cu apă și numai așa să se îmbrace cu ele.
\par 5 Iar de la obștea fiilor lui Israel să ia din turma lor doi țapi de jertfă pentru păcat și un berbec pentru arderea de tot.
\par 6 Să aducă Aaron. pentru sine vițelul de jertfă pentru păcat, ca să se curețe pe sine și casa sa.
\par 7 Apoi să ia cei doi țapi și să-i pună înaintea feței Domnului la ușa cortului adunării.
\par 8 Și să arunce Aaron sorți asupra celor doi țapi: un sorț pentru al Domnului și un sorț pentru al lui Azazel.
\par 9 După aceea să ia Aaron țapul, asupra căruia a căzut sorțul Domnului, și să-l aducă jertfă pentru păcat,
\par 10 Iar țapul asupra căruia a căzut sorțul pentru Azazel să-l pună viu înaintea Domnului, ca să săvârșească asupra lui curățirea și să-i dea drumul în pustie pentru ispășire, ca să ducă acela cu sine nelegiuirile lor în pământ neumblat.
\par 11 Apoi să aducă Aaron vițelul de jertfă pentru păcatele sale, ca să se curețe pe sine și casa sa și să junghie vițelul jertfă pentru păcatele sale;
\par 12 Să ia cărbuni aprinși de pe jertfelnicul cel dinaintea Domnului, o cădelniță plină, și aromate pisate mărunt pentru tămâiere două mâini pline, și să le ducă înăuntru, după perdea;
\par 13 Să pună aromatele pe focul din cădelniță înaintea felei Domnului, astfel ca norul de fum să acopere capacul cel de pe chivotul legii, ca să nu moară Aaron.
\par 14 Să ia și din sângele vițelului și să stropească cu degetul său spre răsărit peste capac; și înaintea capacului să stropească de șapte ori sânge cu degetul său.
\par 15 După aceea să junghie înaintea Domnului țapul de jertfă pentru păcatele poporului, să ducă sângele lui înăuntru, după perdea, și să facă cu sângele acela ce a făcut și cu sângele vițelului, stropind cu el pe capac și înaintea capacului.
\par 16 Așa va curăți locașul sfânt de necurăția fiilor lui Israel, de nelegiuirile lor și de toate păcatele lor. Așa să facă el cu cortul adunării, care se află la ei, în mijlocul necurățeniilor lor.
\par 17 Nici un om să nu fie în cortul adunării, când va intra el să curețe locașul sfânt și până va ieși. Așa se va curăți el pe sine și casa sa și toată obștea fiilor lui Israel.
\par 18 Apoi va ieși la jertfelnicul cel dinaintea Domnului și-l va curăți, luând din sângele vițelului și din sângele țapului și punând pe coarnele jertfelnicului de jur împrejur,
\par 19 Și, stropindu-l cu sânge, cu degetul său de șapte ori, îl va curăți de necurățeniile fiilor lui Israel și-l va sfinți.
\par 20 Iar după ce va sfârși de curățat locașul sfânt, cortul adunării și jertfelnicul, și curățind și pe preoți, va aduce țapul cel viu,
\par 21 Își va pune Aaron mâinile sale pe capul țapului celui viu și va mărturisi asupra lui toate nelegiuirile fiilor lui Israel, toate nedreptățile lor și toate păcatele lor; și, punându-le pe acestea pe capul țapului, îl va trimite cu un om anumit în pustie.
\par 22 Și va duce țapul cu sine toate nelegiuirile lor în pământ neumblat și omul va da drumul țapului în pustie.
\par 23 După aceea va intra Aaron în cortul adunării și se va dezbrăca de hainele cele de in, cu care se îmbrăcase la intrarea în locul cel sfânt, și le va lăsa acolo;
\par 24 Își va spăla trupul său cu apă în locul cel sfânt, se va îmbrăca cu hainele sale și, ieșind, va săvârși arderea de tot pentru sine și arderea de tot pentru popor, și se va curăți astfel pe sine, casa sa, poporul și pe preoți.
\par 25 Iar grăsimea jertfei pentru păcat o va arde pe jertfelnic.
\par 26 Cel ce a dat drumul în pustie țapului de ispășire să-și spele hainele sale, să-și spele trupul cu apă și atunci să intre în tabără.
\par 27 Iar vițelul de jertfă pentru păcat Și țapul de jertfă pentru păcat al căror sânge a fost adus înăuntru pentru curățirea locașului sfânt, să se scoată afară din tabără și să se ardă în foc pielea lor, carnea lor și necurățenia lor.
\par 28 Cel ce le va arde să-și spele hainele, să-și spele trupul său cu apă și numai după aceea să intre în tabără.
\par 29 Aceasta să fie pentru voi lege veșnică: în luna a șaptea, în ziua a zecea a lunii, să postiți și nici o muncă să nu faceți, nici băștinașul, nici străinul care s-a așezat la voi,
\par 30 Căci în ziua aceasta vi se face curățire, ca să fiți curați de toate păcatele voastre, înaintea Domnului, și curați veți fi.
\par 31 Aceasta e cea mai mare zi de odihnă pentru voi și să smeriți sufletele voastre prin post. Aceasta este lege veșnică.
\par 32 De curățit însă să vă curețe preotul care este uns ca să slujească în locul tatălui său.
\par 33 Să se îmbrace el cu veșmintele cele de in și cu veșmintele sfinte; și va curăți sfânta sfintelor, cortul adunării, va curăți jertfelnicul și pe preoți și va curăți și toată obștea poporului.
\par 34 Aceasta să fie pentru voi lege veșnică: o dată în an să curățiți pe fiii lui Israel de păcatele lor". Și Aaron a făcut așa cum poruncise Domnul lui Moise.

\chapter{17}

\par 1 Grăit-a Domnul cu Moise și a zis:
\par 2 "Vorbește lui Aaron, fiilor lui și tuturor fiilor lui Israel și zi către ei: Iată ce poruncește Domnul:
\par 3 Orice om dintre fiii lui Israel sau dintre străinii ce s-au lipit de voi, care va junghia bou, sau oaie, sau capră, în tabără, sau care va junghia afară din tabără,
\par 4 Și nu le va înfățișa la ușa cortului adunării, ca să le aducă ardere de tot sau jertfă de izbăvire, plăcută Domnului, cu miros de bună mireasmă; sau care le va junghia afară din tabără și la ușa cortului adunării nu le va aduce, ca să le facă jertfă Domnului, înaintea locașului Domnului, omului aceluia i se va cere sângele, că a vărsat sânge și se va stârpi sufletul acela din poporul său;
\par 5 Pentru ca fiii lui Israel să-și aducă jertfele lor, câte le junghie ei în câmp, și să le înfățișeze Domnului la ușa cortului adunării, la preot, și să le facă Domnului jertfă de mântuire.
\par 6 Și va stropi preotul cu sânge jertfelnicul împrejur, înaintea Domnului, la ușa cortului adunării, iar grăsimea o va arde spre miros de bună mireasmă Domnului,
\par 7 Ca să nu-și mai aducă ei jertfele lor la idolii după care umblă desfrânând. Aceasta să fie pentru ei așezământ veșnic în neamul lor.
\par 8 Să le spui de asemenea: Dacă un om dintre fiii lui Israel sau dintre fiii străinilor care locuiesc între ei va face ardere de tot sau jertfă
\par 9 Și nu o va aduce la ușa cortului adunării, ca să o aducă jertfă înaintea Domnului, omul acela se va stârpi din poporul său.
\par 10 Dacă un om dintre fiii lui Israel și dintre străinii care trăiesc între voi va mânca orice fel de sânge, Îmi voi întoarce fața Mea împotriva sufletului celui ce va mânca sânge și-l voi stârpi din poporul său,
\par 11 Pentru că viața a tot trupul este în sânge și pe acesta vi l-am dat pentru jertfelnic, ca să vă curățiri sufletele voastre, că sângele acesta curățește sufletul.
\par 12 De aceea am și zis fiilor lui Israel: Nimeni din voi să nu mănânce sânge și nici străinul, care locuiește la voi, să nu mănânce sânge.
\par 13 Oricine dintre fiii lui Israel și dintre străinii ce locuiesc la voi va vâna fiară sau pasăre, care se mănâncă, acela să scurgă sângele ei și să-l acopere cu pământ, căci viața oricărui trup este în sângele lui.
\par 14 De aceea am zis fiilor lui Israel: Să nu mâncați sângele nici unui trup, pentru că viața oricărui trup este în sângele lui: tot cel ce-l va mânca se va stârpi,
\par 15 Și tot cel ce va mânca mortăciune sau sfâșiat de fiară, acela, fie băștinaș sau străin, să-și spele hainele, să se spele cu apă și necurat va fi până seara, iar apoi va fi curat;
\par 16 Iar de nu-și va spăla hainele sale și nu-și va spăla trupul său, va purta asupra sa vina sa".

\chapter{18}

\par 1 În vremea aceea a grăit Domnul cu Moise, zicând:
\par 2 "Vorbește fiilor lui Israel și zi către ei: Eu sunt Domnul Dumnezeul vostru.
\par 3 De datinile pământului Egiptului, în care ați trăit, să nu vă țineți, nici de datinile pământului Canaanului, în care am să vă duc, să nu vă țineți și nici să umblați după obiceiurile lor.
\par 4 Ci legile Mele să le pliniți și așezămintele Mele să le păziți, umblând după cum poruncesc ele, că Eu sunt Domnul Dumnezeul vostru.
\par 5 Păziți toate poruncile Mele și toate hotărârile să le țineți, căci omul care le plinește va trăi prin ele: Eu sunt Domnul Dumnezeul vostru.
\par 6 Nimeni să nu se apropie de nici o rudă după trup, cu gândul ca să-i descopere goliciunea. Eu sunt Domnul!
\par 7 Goliciunea tatălui tău și goliciunea mamei tale să n-o descoperi! Că este mama ta, să nu-i descoperi goliciunea ei.
\par 8 Goliciunea femeii tatălui tău să n-o descoperi, că este goliciunea tatălui tău!
\par 9 Goliciunea surorii tale, goliciunea fiicei tatălui tău sau a fiicei mamei tale, care s-a născut în casă sau afară din casă, să n-o descoperi!
\par 10 Goliciunea fiicei fiului tău sau a fiicei fiicei tale să n-o descoperi, căci goliciunea ta este!
\par 11 Goliciunea fiicei femeii tatălui tău; care s-a născut din tatăl tău, să n-o descoperi, că soră îți este după tată!
\par 12 Goliciunea surorii tatălui tău să n-o descoperi, că este de un sânge cu tatăl tău!
\par 13 Goliciunea surorii mamei tale să n-o descoperi, că este de un sânge cu mama ta!
\par 14 Goliciunea fratelui tatălui tău să n-o descoperi și de femeia lui să nu te apropii, că sunt unchiul și mătușa ta!
\par 15 Goliciunea nurorii tale să n-o descoperi, că ea este femeia fiului tău; să nu-i descoperi goliciunea!
\par 16 Goliciunea femeii fratelui tău să n-o descoperi, că este goliciunea fratelui tău.
\par 17 Goliciunea unei femei și a fiicei ei să nu descoperi; pe fiica fiului ei și pe fiica fiicei ei să nu le iei, ca să le descoperi goliciunea; aceasta este nelegiuire, că sunt rude de sânge cu ea!
\par 18 Să nu iei concubină pe sora femeii tale, ca să descoperi rușinea ei în vremea ei, vie fiind ea.
\par 19 Să nu te apropii de femeie în timpul regulii ei, ca să-i descoperi goliciunea.
\par 20 Și cu femeia aproapelui tău să nu te culci, ca să-ți verși sămânța și ca să te spurci cu ea.
\par 21 Din copiii tăi să nu dai în slujba lui Moloh, ca să nu pângărești numele Dumnezeului tău. Eu sunt Domnul.
\par 22 Să nu te culci cu bărbat, ca și cu femeie; aceasta este spurcăciune.
\par 23 Cu nici un dobitoc să nu te culci, ca să-ți verși sămânța și să te spurci cu el; nici femeia să nu stea la dobitoc, ca să se spurce cu el; aceasta e urâciune.
\par 24 Să nu vă întinați cu nimic din acestea, că cu toate acestea s-au întinat păgânii, pe care țu îi izgonesc dinaintea feței voastre.
\par 25 Că s-a întinat pământul și am privit la nelegiuirile lor și a lepădat pământul pe cei ce trăiau pe el.
\par 26 Iar voi să păziți toate poruncile Mele și toate legile Mele și să nu faceți toate ticăloșiile acestea, nici băștinașul, nici străinul care trăiește între țoi.
\par 27 Că toate urâciunile acestea le-au făcut oamenii pământului acestuia care e înaintea voastră și s-a întinat pământul;
\par 28 Ca nu cumva să vă lepede și pe voi pământul, când îl veți întina, cum a aruncat el de la sine pe popoarele care au fost înainte de voi.
\par 29 Că tot cel ce va face ticăloșiile acestea, sufletul care va face acestea se va stârpi din poporul său.
\par 30 Deci păziți poruncile Mele și să nu umblați după obiceiurile urâte, după care au umblat cei dinaintea voastră, nici să vă întinați cu ele. Eu sunt Domnul Dumnezeul vostru".

\chapter{19}

\par 1 Grăit-a Domnul cu Moise și a zis:
\par 2 "Vorbește la toată obștea fiilor lui Israel și le zi; Fiți sfinți, că Eu, Domnul Dumnezeul vostru, sunt sfânt.
\par 3 Să cinstească fiecare pe tatăl său și pe mama sa și zilele Mele de odihnă să le păziți, că Eu sunt Domnul Dumnezeul vostru.
\par 4 Să nu alergați la idoli și dumnezei turnați să nu vă faceți, că Eu sunt Domnul Dumnezeul vostru.
\par 5 De veți aduce Domnului jertfă de izbăvire, să o aduceți de bunăvoie.
\par 6 Și să o mâncați în ziua aducerii și a doua zi, iar ce va rămâne pentru a treia zi să ardeți cu foc.
\par 7 Iar de va mânca cineva a treia zi, va face urâciune și jertfa nu va fi primită;
\par 8 Cel ce va mânca va agonisi păcat, că acela a spurcat lucrul sfânt al Domnului și sufletul acela se va stârpi din poporul său.
\par 9 Când veți secera holdele voastre în pământul vostru, să nu seceri tot până la fir în ogorul tău și ceea ce rămâne după secerișul tău să nu aduni;
\par 10 Și în via ta să nu culegi strugurii rămași, nici boabele ce cad în via ta să nu le aduni; lasă-le pe acestea săracului și străinului, că Eu sunt Domnul Dumnezeul tău.
\par 11 Să nu furați, să nu spuneți minciună și să nu înșele nimeni pe aproapele său.
\par 12 Să nu vă jurați strâmb pe numele Meu și să nu pângăriți numele cel sfânt al Dumnezeului vostru, că Eu sunt Domnul Dumnezeul vostru.
\par 13 Să nu nedreptățești pe aproapele și să nu-l jefuiești. Plata simbriașului să nu rămână la tine până a doua zi.
\par 14 Să nu grăiești de rău pe surd și înaintea orbului să nu pui piedică. Să te temi de Domnul Dumnezeul tău. Eu sunt Domnul Dumnezeul tău.
\par 15 Să nu faceți nedreptate la judecată; să nu căutați la fața celui sărac și de fața celui puternic să nu te sfiești, ci cu dreptate să judeci pe aproapele tău.
\par 16 Să nu umbli cu clevetiri în poporul tău și asupra vieții aproapelui tău să nu te ridici. Eu sunt Domnul Dumnezeul vostru.
\par 17 Să nu dușmănești pe fratele tău în inima ta, dar să mustri pe aproapele tău, ca să nu porți păcatul lui.
\par 18 Să nu te răzbuni cu mina ta și să nu ai ură asupra fiilor poporului tău, ci să iubești pe aproapele tău ca pe tine însuți. Eu sunt Domnul Dumnezeul vostru.
\par 19 Legea Mea să o păziți; vitele tale să nu le faci să se împreune cu alt soi; ogorul tău să nu-l semeni deodată cu două feluri de semințe; cu haină țesută din felurite torturi, de lână și de in, să nu te îmbraci.
\par 20 De va dormi cineva cu femeie, împreunându-se, și aceea va fi roabă, logodită cu un bărbat, dar nerăscumpărată încă sau dacă nu i s-a dat încă slobozenia, să-i pedepsiți pe amândoi, dar nu cu moarte, pentru că ea nu este slobodă,
\par 21 Ci să aducă el Domnului, la ușa cortului adunării, jertfă de vină; un berbec să aducă jertfă pentru vina sa;
\par 22 Și preotul îl va curăți de păcatul lui înaintea Domnului cu berbecul cel pentru vină și i se va ierta lui păcatul pe care l-a făcut.
\par 23 Când veți intra în pământul, pe care Domnul Dumnezeul vostru vi-l  va da, și veți sădi orice pom roditor, să curățiți necurățenia lui: trei ani să socotiți roadele lui ca necurate și să nu le mâncați;
\par 24 Iar în anul al patrulea toate roadele lui să fie afierosite Domnului, întru lauda Lui.
\par 25 Și în anul al cincilea să mâncați din roadele lui și să vă adunați toate roadele. Eu sunt Domnul Dumnezeul vostru.
\par 26 Să nu mâncați cu sânge; să nu vrăjiți, nici să ghiciți.
\par 27 Să nu vă tundeți rotund părul capului vostru, nici să vă stricați fața bărbii voastre.
\par 28 Pentru morți să nu vă faceți tăieturi pe trupurile voastre, nici semne cu împunsături să nu faceți pe voi. Eu sunt Domnul Dumnezeul vostru.
\par 29 Să nu necinstești pe fiica ta, îngăduindu-i să facă desfrânare, ca să nu se desfrâneze pământul și ca să nu se umple pământul de stricăciune.
\par 30 Zilele Mele de odihnă să le păzești și locașul Meu să-l cinstești. Eu sunt Domnul.
\par 31 Să nu alergați la cei ce cheamă morții, pe la vrăjitori să nu umblați și să nu vă întinați cu ei. Eu sunt Domnul Dumnezeul vostru.
\par 32 Înaintea celui cărunt să te scoli, să cinstești fața bătrânului și să te temi de Domnul Dumnezeul tău. Eu sunt Domnul Dumnezeul vostru.
\par 33 De se va așeza vreun străin în pământul vostru, să nu-l strâmtorați.
\par 34 Străinul, care s-a așezat la voi, să fie pentru voi ca și băștinașul vostru; să-l iubiți ca pe voi înșivă, că și voi ați fost străini în pământul Egiptului. Eu sunt Domnul Dumnezeul vostru.
\par 35 Să nu faceți nedreptate la judecată, la măsură, la cântărit și la măsurătoare.
\par 36 Cântarul vostru să fie drept, greutățile drepte, efa dreaptă și hinul drept. Eu sunt Domnul Dumnezeul vostru, Care v-am scos din pământul Egiptului.
\par 37 Să păziți toate legile Mele și toate orânduielile Mele și să le pliniți. Eu sunt Domnul Dumnezeul vostru".

\chapter{20}

\par 1 Grăit-a Domnul cu Moise și a zis:
\par 2 "Spune fiilor lui Israel acestea: Dacă cineva dintre fiii lui Israel și dintre străinii care trăiesc printre Israeliți va da din copiii săi lui Moloh, acela să fie dat morții: poporul băștinaș să-l ucidă cu pietre.
\par 3 Și Eu Îmi voi întoarce fața Mea împotriva omului aceluia și-l voi stârpi din poporul său, pentru că a dat din copiii săi lui Moloh, ca să întineze locașul Meu cel sfânt și să necinstească numele Meu cel sfânt.
\par 4 Iar dacă poporul băștinaș își va închide ochii săi asupra omului aceluia, când va da din copiii săi lui Moloh,
\par 5 Îmi voi întoarce fața Mea împotriva omului aceluia și împotriva neamului lui și-l voi stârpi din poporul său pe el și pe toți cei ce fac desfrânări asemenea lui, desfrânând după Moloh.
\par 6 Dacă vreun suflet va alerga la cei ce cheamă morții și la vrăjitorii, ca să desfrâneze în urma lor, Eu voi întoarce fața Mea împotriva sufletului aceluia și-l voi pierde din poporul lui.
\par 7 Sfințiți-vă pe voi înșivă și veți fi sfinți, că Eu, Domnul Dumnezeul vostru, sunt sfânt.
\par 8 Păziți legile Mele și le pliniți, că Eu sunt Domnul, Cel ce vă sfințește.
\par 9 Cel ce va grăi de rău pe tatăl său sau pe mama sa să fie dat morții, că a grăit de rău pe tatăl său și pe mama sa și sângele său este asupra sa.
\par 10 De se va desfrâna cineva cu femeie măritată, adică de se va desfrâna cu femeia aproapelui său, să se omoare desfrânatul și desfrânata.
\par 11 Cel ce se va culca cu femeia tatălui său, acela goliciunea tatălui său a descoperit; să se omoare amândoi, căci vinovați sunt.
\par 12 De se va culca cineva cu nora sa, amândoi să se omoare, că au făcut urâciune și sângele lor este asupra lor.
\par 13 De se va culca cineva cu bărbat ca și cu femeie, amândoi au făcut nelegiuire și să se omoare, că sângele lor asupra lor este.
\par 14 Dacă își va lua cineva femeie și se va desfrâna cu mama ei, nelegiuire face; pe foc să se ardă și el și ea, ca să nu fie nelegiuiri între voi.
\par 15 Cel ce se va amesteca. cu dobitoc să se omoare și să ucideți dobitocul.
\par 16 Dacă femeia se va duce la vreun dobitoc, ca să se unească cu el, să ucizi femeia și dobitocul să se omoare, că sângele lor este asupra lor.
\par 17 De va lua cineva pe sora sa, după tată sau după mamă, și-i va vedea goliciunea și ea va vedea goliciunea lui: aceasta este rușine și să fie stârpiți înaintea ochilor fiilor poporului lor. El a descoperit goliciunea surorii sale; să-și poarte păcatul lor.
\par 18 Bărbatul care se va culca cu femeie în timpul curgerii ei și-i va descoperi goliciunea, acela a descoperit curgerea sângelui ei și ea și-a descoperit curgerea sângelui său: amândoi să fie stârpiți din poporul lor.
\par 19 Goliciunea surorii mamei tale și a surorii tatălui tău să n-o descoperi, că unul ca acela își dezgolește trupul rudei sale și-și vor purta păcatul amândoi.
\par 20 Cel ce se va culca cu mătușa sa descoperă goliciunea unchiului său; să-și poarte amândoi păcatul și fără copii să moară.
\par 21 De va lua cineva pe femeia fratelui său, urâciune este, că a descoperit goliciunea fratelui său: fără copii să moară.
\par 22 Păziți toate așezămintele Mele și toate hotărârile Mele și le pliniți și nu vă va arunca de pe sine pământul în care vă voi duce să trăiți.
\par 23 Să nu umblați după obiceiurile popoarelor pe care le voi alunga de la voi, că ele au făcut acestea toate și M-am scârbit de ele.
\par 24 Eu doară v-am spus: Voi veți moșteni pământul lor și Eu vă voi da să moșteniți pământul în care curge lapte și miere. Eu sunt Domnul Dumnezeul vostru, Care v-am despărțit de toate popoarele.
\par 25 Să deosebiți dobitocul curat de cel necurat și pasărea curată de cea necurată; să nu vă întinaâi sufletele voastre cu dobitoc sau cu pasăre, nici cu toate cele ce se târăsc pe pământ, pe care Eu le-am deosebit ca necurate.
\par 26 Să-Mi fiți sfinți, că Eu, Domnul Dumnezeul vostru, sunt sfânt și v-am deosebit de toate popoarele, ca să fiți ai Mei.
\par 27 Bărbatul sau femeia, de vor chema morți sau de vor vrăji, să moară neapărat: cu pietre să fie uciși, că sângele lor este asupra lor".

\chapter{21}

\par 1 Zis-a Domnul către Moise: "Grăiește preoților, fiilor lui Aaron și le spune:
\par 2 Să nu se spurce prin atingere de mort din poporul lor. Să se atingă numai de rudenia de aproape a lor, de mama lor și de tatăl lor, de fiul lor și de fiica lor, de fratele lor;
\par 3 De sora lor fecioară, care trăiește la ei și e nemăritată, poate să se atingă, fără să se spurce.
\par 4 De nimeni altul din poporul său să nu se atingă, ca să nu se spurce.
\par 5 Să nu-și radă capul, să nu-și tundă marginea bărbii și să nu-și facă tăieturi pe trupurile lor pentru morți.
\par 6 Să fie sfinți ai Dumnezeului lor și să nu pângărească numele Dumnezeului lor, că ei aduc jertfă Domnului și pâine Dumnezeului lor și de aceea să fie sfinți.
\par 7 Să nu-și ia de soție femeie desfrânată sau necinstită; nici femeie lepădată de bărbatul ei, căci sunt sfinți ai Domnului Dumnezeului lor.
\par 8 Cinstește-i ca pe sfinți, căci ei aduc pâine Dumnezeului tău; sfinți să vă fie, căci Eu, Domnul, Cel ce vă sfințesc, sunt sfânt.
\par 9 Dacă fiica preotului se va spurca prin desfrânare, ea necinstește pe tatăl său: să fie arsă cu foc.
\par 10 Marele preot din frații tăi, pe capul căruia s-a turnat mirul de ungere și care este sfințit, ca să se îmbrace cu veșmintele sfinte, să nu-și descopere capul său, nici să-și sfâșie hainele;
\par 11 Și nici de un mort să nu se apropie, nici chiar de tatăl său sau de mama sa să nu se atingă.
\par 12 De locașul sfânt să nu se depărteze, ca să nu necinstească locașul Dumnezeului său, căci mirul sfânt de ungere al Dumnezeului lui este asupra lui. Eu sunt Domnul.
\par 13 Acesta își va lua de femeie fecioară din poporul său.
\par 14 Văduvă, sau lepădată, sau necinstită, sau desfrânată să nu ia, ci fecioară din poporul său să-și ia de femeie.
\par 15 Să nu-și spurce sămânța sa în poporul său, că Eu sunt Domnul Dumnezeu, Cel ce îl sfințesc"!
\par 16 Grăit-a Domnul cu Moise și a zis:
\par 17 "Spune lui Aaron: Nimeni din neamul tău în viitor și din rudele tale să nu se apropie, ca să aducă daruri Dumnezeului său, de va avea vreo meteahnă pe trupul său.
\par 18 Tot omul cu meteahnă pe trup să nu se apropie: nici orb, nici șchiop, nici ciung,
\par 19 Nici cel cu piciorul rupt sau cu mâna ruptă, nici ghebos, nici cu vreun mădular uscat,
\par 20 Nici cel cu albeață pe ochi, nici chelul, nici pipernicitul, nici cel cu părțile bărbătești vătămate.
\par 21 Nici un om din sămânța preotului Aaron, care va avea pe trupul său vreo meteahnă, să nu se apropie ca să aducă jertfă Domnului; că are meteahnă și de aceea să nu se apropie ca să aducă daruri Dumnezeului său.
\par 22 Darurile Dumnezeului său sunt sfințenii mari, din sfințenii poate să mănânce,
\par 23 Dar de perdea să nu treacă și la jertfelnic să nu se apropie; să nu necinstească locașul Meu cel sfânt, căci Eu  sunt Domnul, Cel ce îi sfințesc".
\par 24 Și a spus Moise acestea lui Aaron, fiilor lui și tuturor fiilor lui Israel.

\chapter{22}

\par 1 După aceea a grăit Domnul cu Moise și a zis:
\par 2 "Spune lui Aaron și fiilor lui să umble cu băgare de seamă cu cele sfinte ale fiilor lui Israel și să nu pângărească numele cel sfânt al Meu prin prinoasele pe care ei înșiși Mi le aduc. Eu sunt Domnul.
\par 3 Spune-le: Tot omul din seminția voastră și din neamul vostru, care va avea pe sine vreo necurățenie și se va apropia de cele sfinte, care se afierosesc de fiii lui Israel Domnului, sufletul aceluia se va stârpi de la fața Mea. Eu sunt Domnul Dumnezeul vostru.
\par 4 Omul din seminția preotului Aaron care va fi lepros sau va avea curgere să nu mănânce din cele sfinte, până nu se va curăți; și cine se va atinge de ceva necurat de la mort, sau cine va suferi de curgerea seminței,
\par 5 Sau cine se va atinge de vreo târâtoare, de care s-ar spurca, sau de vreun om, care l-ar face necurat prin orice fel de necurăție a lui,
\par 6 Cel ce s-a atins de acestea necurat va fi până seara și să nu mănânce din cele sfinte înainte de a-și spăla trupul său cu apă.
\par 7 Iar după ce va apune soarele și după ce se va curăți, atunci să mănânce din cele sfinte, că aceea este hrana lui.
\par 8 Mortăciune și sfâșiat de fiară să nu mănânce, ca să nu se spurce cu acestea. Eu sunt Domnul.
\par 9 Să păzească poruncile Mele, ca să nu aibă asupră-le păcat și să nu moară, când vor călca acestea. Eu sunt Domnul Dumnezeu, Cel ce îi sfințesc pe ei.
\par 10 Nici un străin să nu mănânce din cele sfinte. Nici cel ce locuiește la un preot și nici simbriașul preotului să nu mănânce din cele sfinte.
\par 11 Iar dacă preotul își va cumpăra un rob cu argint, acela să mănânce din ele; asemenea și robul născut în casa sa să mănânce din pâinea lui.
\par 12 Dacă fiica preotului se va mărita după străin de neamul preoțesc, nici ea să nu mănânce din prinoasele sfinte, cuvenite lui.
\par 13 Când însă fiica preotului va fi văduvă sau despărțită și copii nu va avea și se va întoarce în casa tatălui său, cum era și în tinerețea sa, atunci ea să mănânce pâinea tatălui său, iar dintre străini nimeni să nu mănânce.
\par 14 Dacă cineva mănâncă din greșeală din cele sfinte, să întoarcă preotului prețul lucrului sfânt și să mai adauge încă a cincea parte din prețul lui:
\par 15 Preoții să nu spurce cele sfinte ale fiilor lui Israel, pe care ei le aduc dar Domnului,
\par 16 Și să nu atragă asupră-le vinovăția fărădelegii, când vor mânca cele sfinte ale lor, că Eu sunt Domnul, Cel ce îi sfințesc".
\par 17 Grăit-a Domnul cu Moise și a zis:
\par 18 "Vorbește lui Aaron, fiilor lui și la toată adunarea fiilor lui Israel și le zi: Dacă cineva dintre fiii lui Israel, sau dintre străinii care s-au așezat la ei, în Israel, își vor aduce jertfa lor, pe care o aduc Domnului ardere de tot, după făgăduință sau de evlavie,
\par 19 Ca să afle prin aceasta bunăvoință la Dumnezeu, jertfa trebuie să fie fără meteahnă, de parte bărbătească, din vitele mari, sau din oi, sau din capre.
\par 20 Toate câte au meteahnă în sine să nu le aduceți Domnului, că nu vor fi primite.
\par 21 De va aduce cineva Domnului jertfă de mântuire, plinind o făgăduință, sau aducând jertfă de bună voie, sau la praznicele voastre, din boi, sau din oi, să fie fără meteahnă; ca să fie primită, să nu aibă nici o meteahnă.
\par 22 Dobitoc orb, vătămat, sau slut, sau bubos, sau răpciugos, sau râios, să nu aduceți Domnului și nici să dați la jertfelnicul Domnului pentru jertfă.
\par 23 Bou sau oaie cu picioarele lungi sau scurte peste măsură poți să aduci ca jertfă de evlavie, iar pentru jertfa făgăduită acestea nu, sunt primite.
\par 24 Dobitocul care are părțile bărbătești strivite, sfărâmate, smulse sau tăiate, să nu-l aduceți Domnului și în țara voastră să nu faceți asemenea lucru.
\par 25 Nici din mâinile celor de alt neam să nu aduceți nici unul din asemenea dobitoace în dar Dumnezeului rostru, pentru că acestea sunt vătămate și cu meteahnă și nu vă vor fi primite".
\par 26 Și a grăit Domnul cu Moise și a zis:
\par 27 "De se va naște vițel, sau miel, sau ied, șapte zile să stea la mama lui, iar din ziua a opta înainte va fi bun de adus jertfă Domnului.
\par 28 Dar nici vacă, nici oaie să nu junghiați în aceeași zi cu puiul ei.
\par 29 De aduceți Domnului jertfă de mulțumire, s-o aduceți ca să vă fie primită.
\par 30 În aceeași zi s-o mâncați și să nu lăsați din carnea ei pe a doua zi. Eu sunt Domnul.
\par 31 Să păziți poruncile Mele și să le pliniți: Eu sunt Domnul.
\par 32 Să nu spurcați numele cel sfânt al Meu, ca să fiu Eu sfânt între fiii lui Israel.
\par 33 Eu sunt Domnul, Cel ce vă sfințesc pe voi, Care v-am scos din pământul Egiptului, ca să fiu Dumnezeul vostru. Eu sunt Domnul".

\chapter{23}

\par 1 Grăit-a Domnul cu Moise și a zis:
\par 2 "Vorbește fiilor lui Israel și le spune care sunt sărbătorile Domnului, în care se vor face adunările sfinte. Sărbătorile Mele sunt acestea:
\par 3 Șase zile să lucrați, iar ziua a șaptea este ziua odihnei, adunare sfântă a Domnului: nici o muncă să nu faceți; aceasta este odihna Domnului în toate locuințele voastre.
\par 4 Iată și celelalte sărbători ale Domnului, adunările sfinte, pe care trebuie să le vestiți la vremea lor:
\par 5 În luna întâi, în ziua a paisprezecea a lunii, către seară, sunt Paștile Domnului.
\par 6 Iar în ziua a cincisprezecea a aceleiași luni este sărbătoarea azimei Domnului: șapte zile să mâncați azime.
\par 7 În ziua întâi a sărbătorilor să aveți adunare sfântă și nici o muncă să nu faceți.
\par 8 Timp de șapte zile să aduceți jertfă Domnului, și în ziua a șaptea iar e adunare sfântă; nici o muncă să nu faceți".
\par 9 Și a grăit Domnul cu Moise și a zis:
\par 10 "Vorbește fiilor lui Israel și le spune: Când veți intra în pământul pe care Eu vi-l dau vouă și veți face seceriș în el, cel dintâi snop al secerișului vostru să-l aduceți la preot;
\par 11 El va ridica acest snop înaintea Domnului, ca să aflați bunăvoință la El; a doua zi după cea dintâi a sărbătorii îl va ridica preotul.
\par 12 În ziua ridicării snopului veți aduce Domnului ardere de tot un miel de un an, fără meteahnă.
\par 13 Împreună cu el veți aduce prinos de pâine două din zece părți de efă de făină de grâu, amestecată cu untdelemn, ca să fie jertfă Domnului, mireasmă plăcută, și veți face și turnare un sfert de hin de vin.
\par 14 Nici un fel de pâine nouă, nici grăunțe uscate, nici grăunțe crude să nu mâncați până la ziua aceea în care veți aduce prinos Dumnezeului vostru. Acesta este așezământ veșnic în neamul vostru, oriunde veți locui.
\par 15 Din ziua a doua după întâi a sărbătorii, din ziua în care veți aduce snopul legănat, să numărați șapte săptămâni întregi,
\par 16 Până la ziua întâi de după cea din urmă zi a săptămânii a șaptea, să numărați cincizeci de zile și atunci să aduceți un nou dar de pâine Domnului.
\par 17 Să aduceți din locuințele voastre dar ridicat: două pâini făcute din două zecimi de efă de făină de grâu, coapte cu dospitură, ca pârgă Domnului.
\par 18 Împreună cu pâinile să mai aduceți șapte miei fără meteahnă, de câte un an, un junc din cireadă și doi berbeci fără meteahnă; ca să fie acestea Domnului ardere de tot, dar de pâine, turnare și jertfă cu mireasmă plăcută Domnului.
\par 19 Să jertfiți de asemenea din turma de capre un țap, jertfă pentru păcat, și doi miei de câte un an, jertfă de mântuire, împreună cu pâinile din pârgă.
\par 20 Pe acestea să le aducă preotul legănându-le înaintea Domnului, împreună cu pâinile din pârgă de grâu legănate și cu cei doi miei; acestea vor fi sfințenie Domnului și vor fi ale preotului, care le înfățișează.
\par 21 Să dați de știre prăznuirea în ziua aceasta și să aveți adunare sfântă, nici o muncă să nu faceți. Acesta este așezământ veșnic în neamul vostru, în toate așezările voastre.
\par 22 Când veți secera holda în pământul vostru, să nu adunați ce rămâne după seceratul ogorului vostru și spicele ce cad de sub secere să nu le adunați, ci să lăsați pe acestea săracului și străinului. Eu sunt Domnul, Dumnezeul vostru".
\par 23 Și a grăit Domnul cu Moise și a zis:
\par 24 "Spune fiilor lui Israel: În luna a șaptea, ziua întâi a lunii să vă fie zi de odihnă, sărbătoarea trâmbițelor și adunare sfântă să aveți;
\par 25 Nici o muncă să nu faceți, ci să aduceți ardere de tot Domnului",
\par 26 Apoi a grăit Domnul cu Moise și a zis:
\par 27 "Și în ziua a zecea a lunii aceleia a șaptea, care este ziua curățirii, să aveți adunare sfântă; să postiți și să aduceți ardere de tot Domnului;
\par 28 Nici o muncă să nu faceți în ziua aceea, că aceasta este ziua curățirii" ca să vă curățiți înaintea feței Domnului Dumnezeului vostru.
\par 29 Tot sufletul care nu va posti în ziua aceea se va stârpi din poporul său;
\par 30 Și tot sufletul care va lucra în ziua aceea, îl voi stârpi din mijlocul poporului său.
\par 31 Nici o muncă să nu faceți: acesta este așezământ veșnic în neamul vostru în toate cetățile voastre.
\par 32 Aceasta este pentru voi zi de odihnă; să postiți din seara zilei a noua a lunii; din acea seară până în seara zilei a zecea a lunii să prăznuiți odihna voastră".
\par 33 Grăit-a Domnul cu Moise și a zis:
\par 34 "Spune fiilor lui Israel: Din ziua a cincisprezecea a lunii a șaptea începe sărbătoarea corturilor; șapte zile să sărbătorești în cinstea Domnului.
\par 35 În ziua întâi va fi adunare sfântă; nici o muncă să nu faceți.
\par 36 Șapte zile să aduceți jertfă Domnului și în ziua a opta va fi adunare sfântă; să aduceți arderi de tot Domnului: aceasta este încheierea sărbătorii; nici o muncă să nu faceți.
\par 37 Acestea sunt sărbătorile Domnului, în care trebuie să se țină adunările sfinte, ca să aducă jertfe Domnului, ardere de tot, prinos de pâine, jertfe junghiate și turnări, fiecare din ele la ziua hotărâtă;
\par 38 Afară de zilele de odihnă ale Domnului, afară de darurile voastre, afară de toate afierosirile voastre și afară de tot ce aduceți din evlavie și dați Domnului,
\par 39 În ziua a cincisprezecea a lunii a șaptea, când vă strângeți roadele pământului, să sărbătoriți sărbătoarea Domnului șapte zile: în ziua întâi este odihnă și în ziua a opta iară este odihnă.
\par 40 În ziua întâi să luați ramuri de copaci frumoși, ramuri de finici, ramuri de copaci cu frunzele late și sălcii de râu și să vă veseliți înaintea Domnului Dumnezeului vostru, șapte zile.
\par 41 Să prăznuiți sărbătoarea aceasta a Domnului șapte zile în an: acesta este așezământ veșnic în neamul vostru. În luna a șaptea să o prăznuiți.
\par 42 Să locuiți șapte zile în corturi; tot israelitul băștinaș să locuiască în corturi,
\par 43 Ca să știe urmașii voștri că în corturi am așezat Eu pe fiii lui Israel, când i-am scos din pământul Egiptului. Eu sunt Domnul Dumnezeul vostru".
\par 44 Astfel a grăit Moise fiilor lui Israel despre sărbătorile Domnului.

\chapter{24}

\par 1 Și a grăit Domnul cu Moise și a zis:
\par 2 "Poruncește fiilor lui Israel să-ți aducă untdelemn de măsline, curat și limpede, pentru candele, ca să ardă sfeșnicul necontenit,
\par 3 Înaintea perdelei din cortul adunării, și-l va aprinde Aaron și fiii lui înaintea Domnului, ca să ardă totdeauna, de seara până dimineața. Acesta este așezământ veșnic în neamul vostru.
\par 4 Candelele să le pună în sfeșnicul cel de aur curat de dinaintea Domnului, ca să ardă sfeșnicul de seara până dimineața.
\par 5 Să luați făină de grâu bună și să faceți din ea douăsprezece pâini; fiecare pâine să fie de două zecimi de efă.
\par 6 Și să le puneți pe două rânduri, câte șase pâini în rând, pe masa cea de aur curat de dinaintea Domnului;
\par 7 Pe fiecare rând să pui tămâie curată și sare, și vor fi acestea, pe lângă pâini, jertfă de pomenire înaintea Domnului.
\par 8 În ziua odihnei să se pună acestea necontenit înaintea Domnului din partea fiilor lui Israel; acesta este legământ veșnic.
\par 9 Ele vor fi ale lui Aaron și ale fiilor săi, care le vor mânca în locul cel sfânt, că acestea sunt sfințenie mare pentru ei din jertfele Domnului: acesta este așezământ veșnic".
\par 10 În vremea aceea, fiul unei israelite, născut între israeliți dintr-un egiptean, a ieșit la fiii lui Israel și s-a sfădit în tabără cu un israelit.
\par 11 Și fiul israelitei, hulind numele Domnului și grăindu-l de rău, a fost adus la Moise, iar numele mamei lui era Șelomit, fata lui Dibri, din neamul lui Dan.
\par 12 Acela a fost pus sub strajă, ca să-l judece după porunca Domnului.
\par 13 Atunci a grăit Domnul cu Moise și a zis:
\par 14 "Scoate pe hulitor afară din tabără și toți cei ce au auzit să-și pună mâinile lor pe capul lui, iar toată obștea să-l ucidă cu pietre.
\par 15 Apoi fiitor lui Israel să le spui: Omul care va huli pe Dumnezeu își va agonisi păcat.
\par 16 Hulitorul numelui Domnului să fie omorât neapărat; toată obștea să-l ucidă cu pietre. Sau străinul, sau băștinașul, de va huli numele Domnului, să fie omorât.
\par 17 De va lovi cineva un om și va muri, acela să fie omorât.
\par 18 De va lovi cineva dobitoc și va muri, acela să dea dobitoc pentru dobitoc.
\par 19 De va pricinui cineva vătămare aproapelui său, aceluia să i se facă ceea ce a făcut el altuia:
\par 20 Frântură pentru frântură, ochi pentru ochi, dinte pentru dinte; cum a făcut el vătămare altui om, așa să i se facă și lui.
\par 21 Cel ce va ucide dobitoc să dea altul; iar cel ce va ucide om să fie omorât.
\par 22 Aceeași judecată să aveți și pentru străin și pentru băștinaș, că Eu sunt Domnul Dumnezeul vostru".
\par 23 Și după ce Moise a spus acestea fiilor lui Israel și au făcut fiii lui Israel cum poruncise Domnul lui Moise, au scos pe hulitor afară din tabără și l-au ucis cu pietre.

\chapter{25}

\par 1 Grăit-a Domnul cu Moise pe Muntele Sinai și a zis:
\par 2 "Vorbește fiilor lui Israel și le spune: După ce veri intra în pământul pe care îl voi da vouă, să se odihnească pământul; să fie o odihnă în cinstea Domnului.
\par 3 șase ani să semeni ogorul tău, șase ani să lucrezi via ta și să aduni roadele lor;
\par 4 Iar anul al șaptelea să fie an de odihnă a pământului, odihna Domnului; ogorul tău să nu-l semeni și via ta să n-o tai în anul acela.
\par 5 Ceea ce va crește de la sine pe ogorul tău să nu seceri și strugurii de pe vițele tale netăiate să nu-i culegi, ca să fie acest an de odihnă pentru pământ.
\par 6 Și aceste roade vor fi în timpul odihnei pământului hrană pentru tine, pentru robul tău și pentru roaba ta, pentru simbriașul tău și pentru străinul tău care s-a așezat la tine;
\par 7 Pentru dobitocul tău și pentru fiarele cele de pe pământul tău, să fie de hrană toate roadele lui.
\par 8 Să numeri apoi șapte ani de odihnă, adică de șapte ori câte șapte ani, ca să ai în cei de șapte ori câte șapte ani, patruzeci și nouă de ani.
\par 9 Și să trâmbițezi cu trâmbița în luna a șaptea, în ziua a zecea a lunii; în ziua curățirii să trâmbițezi cu trâmbița în toată țara voastră.
\par 10 Să sfințiți anul al cincizecilea și să se vestească slobozenie pe pământul vostru pentru toți locuitorii lui. Să vă fie acesta an de slobozenie, ca să se întoarcă fiecare la moșia sa; fiecare să se întoarcă la ogorul său.
\par 11 Anul al cincizecilea să vă fie an de slobozenie: să nu semănați, nici să secerați ceea ce va crește de la sine pe pământ, și să nu culegeți poama de pe vițele netăiate,
\par 12 Că acesta e jubileu; sfânt să fie pentru voi; roadele de pe ogor să le mâncați.
\par 13 În anul jubileu să se întoarcă fiecare la moșia sa.
\par 14 De vei vinde ceva aproapelui tău sau de vei cumpăra ceva de la aproapele tău, să nu înșele nimeni pe aproapele său.
\par 15 După numărul anilor trecuți de la cel din urmă jubileu să cumperi de la aproapele tău, și după numărul anilor de cules să-ți vândă el.
\par 16 Dacă au rămas ani mai mulți până la jubileu, sporește prețul, iar dacă au rămas puțini ani, micșorează prețul, căci un anumit număr de secerișuri îți vinde el.
\par 17 Să nu înșele nimeni pe aproapele său; teme-te de Domnul Dumnezeul tău; Eu sunt Domnul Dumnezeul vostru.
\par 18 Faceți poruncile Mele și toate hotărârile Mele; faceți și păziți toate acestea și veți locui liniștiți pe pământ.
\par 19 Pământul își va da rodul său, veți mânca până la saț și veți trăi liniștiți pe el.
\par 20 Iar de veți zice: Dar ce să mâncăm în anul al șaptelea, când nici nu vom semăna, nici nu vom aduna roadele noastre?
\par 21 Vă voi trimite binecuvântarea Mea în anul al șaselea și va aduce roadele sale pentru trei ani.
\par 22 Și veți semăna în anul al optulea, dar de mâncat veți mânca roadele cele vechi până la al nouălea an: până se vor coace roadele anului al optulea veți mânca din cele vechi din anii trecuți.
\par 23 Pământul să nu-l vindeți de veci, că pământul este al Meu; iar voi sunteți străini și venetici înaintea Mea.
\par 24 În toate părțile stăpânirii voastre să îngăduiți răscumpărarea pământului.
\par 25 Dacă fratele tău, care e cu tine, va sărăci și va vinde din moștenirea sa, să vină ruda sa de aproape și să cumpere ceea ce vinde fratele său.
\par 26 Dacă însă nu va avea cineva rudenie, ci îi va da lui mâna și va găsi cât îi trebuie pentru răscumpărare,
\par 27 Atunci să numere anii vânzării sale, și ce trece să întoarcă aceluia, căruia i-a vândut, și se va întoarce la moșia sa.
\par 28 Iar dacă nu va găsi mâna lui cit îi trebuie să întoarcă aceluia, atunci pământul vândut de el va rămâne în mâinile cumpărătorului până la anul jubileu, și în anul jubileu cumpărătorul va ieși și vânzătorul va intra în stăpânirea sa.
\par 29 De va vinde cineva casă de locuit în oraș îngrădit cu zid, poate s-o răscumpere până într-un an de la vânzarea ei: timp de un an poate s-o răscumpere.
\par 30 Iar de nu se va răscumpăra până la împlinirea unui an întreg, casa cea din oraș îngrădit cu zid va rămâne pentru totdeauna aceluia care a cumpărat-o și urmașilor lui, și în anul jubileu nu va trece de la el.
\par 31 Iar casele din sate, care n-au împrejur zid, să se socotească deopotrivă cu pământul: ele se pot răscumpăra oricând și în anul jubileu trec la fostul lor stăpân.
\par 32 Cât pentru orașele leviților și casele din orașele stăpânirii lor, leviții le vor putea răscumpăra de-a pururi.
\par 33 Iar dacă cineva din leviți nu va face răscumpărarea, atunci casa vândută din orașele stăpânirii lor se întoarce în anul jubileu, căci casele din orașele leviților sunt stăpânirea lor între fiii lui Israel.
\par 34 Nici ogoarele dimprejurul orașelor lor nu se pot vinde, pentru că acestea sunt moștenirea lor veșnică.
\par 35 Dacă fratele tău va sărăci și va ajunge la strâmtorare înaintea ta, ajută-l, fie străin, fie băștinaș, ca să trăiască cu tine.
\par 36 Să nu iei de la el dobândă și spor, ci să te temi de Dumnezeul tău, ca să trăiască fratele tău cu tine. Eu sunt Domnul.
\par 37 Argintul tău să nu ți-l dai lui cu camătă și pâinea ta să nu i-o dai ca s-o iei cu spor.
\par 38 Eu sunt Domnul Dumnezeul vostru, Cel ce v-am scos din pământul Egiptului, ca să vă dau pământul Canaanului și ca să fiu Dumnezeul vostru.
\par 39 Când îi va sărăci fratele și-i se va vinde ție, să nu-l pui la muncă de rob,
\par 40 Ci să fie el la tine ca simbriaș sau străin, și să-ți lucreze până la anul jubileu;
\par 41 Iar atunci să se ducă de la tine, și el și copiii lui împreună cu el, să se întoarcă în neamul său și să intre iarăși în stăpânirea părinților săi.
\par 42 Pentru că ei sunt robii Mei, pe care Eu i-am scos din pământul Egiptului;
\par 43 Să nu-i vinzi, cum se vând robii, să nu-i stăpânești cu cruzime și să te temi de Dumnezeul tău.
\par 44 Iar ca să-ți ai robul tău și roaba ta, să-ți cumperi rob și roabă de la neamurile dimprejurul vostru.
\par 45 Puteți să vă cumpărați și din copiii străinilor, care s-au așezat la voi, și din neamul lor, care este la voi și care s-a născut în pământul vostru; pot să fie averea voastră.
\par 46 Puteți să-i dați moștenire fiilor voștri după voi, ca orice avere, veșnic să-i stăpâniți, ca pe robi. Iar asupra fraților voștri din fiii lui Israel și unul asupra altuia, să nu domniți cu cruzime.
\par 47 Dacă străinul sau veneticul tău s-a îmbogățit pe lângă tine, iar fratele tău a sărăcit lângă tine și s-a vândut veneticului, care s-a așezat la tine, sau unui urmaș din neamul veneticului,
\par 48 Atunci, după vânzare, se va putea răscumpăra; careva din frații lui va putea să-l răscumpere:
\par 49 Sau unchiul lui, sau fiul unchiului va putea să-l răscumpere, sau careva din neamul lui, din seminția lui să-l răscumpere; sau de va avea îndestulare, să se răscumpere singur.
\par 50 Și acela să se răfuiască cu cel ce l-a cumpărat, de la anul când s-a vândut el până la anul jubileu, și argintul pentru care s-a vândut să i-l întoarcă după numărul anilor; socotindu-i ca slujiți de un simbriaș vremelnic la el.
\par 51 Și dacă rămân încă mulți ani, el va plăti răscumpărarea potrivit anilor acestora, după prețul cu care a fost cumpărat.
\par 52 Iar dacă rămân puțini ani până la anul jubileu, să-i numere și să plătească, pentru răscumpărarea sa, după numărul anilor.
\par 53 El să fie la dânsul cu anul, ca simbriașul, și acela să nu-l stăpânească cu asprime înaintea ochilor tăi.
\par 54 Iar dacă el nu se va răscumpăra în chipul acesta, atunci în anul jubileu va ieși el însuși, și împreună cu el, și copiii lui,
\par 55 Pentru că fiii lui Israel sunt robii Mei; robii Mei sunt ei, că Eu i-am scos din pământul Egiptului. Eu sunt Domnul Dumnezeul vostru.

\chapter{26}

\par 1 Să nu vă faceți idoli, nici chipuri cioplite; nici stâlpi să nu vă ridicați; nici pietre cu chipuri cioplite cu dalta să nu vă așezați în pământul vostru, ca să vă închinați la ele, că Eu sunt Domnul Dumnezeul vostru.
\par 2 Zilele de odihnă ale Mele să le păziți și locașul Meu cel sfânt să-l cinstiți, că Eu sunt Domnul.
\par 3 De veți umbla după legile Mele și de veți păzi și plini poruncile Mele,
\par 4 Vă voi da ploaie la timp, pământul și pomii își vor da roadele lor.
\par 5 Treieratul vostru va ajunge până la culesul viilor, culesul viilor va ajunge până la semănat; veți mânca pâinea voastră cu mulțumire și veți trăi în pământul vostru fără primejdie.
\par 6 Voi trimite pace pe pământul vostru și nimeni nu vă va tulbura; voi goni fiarele sălbatice și sabia nu va trece prin pământul vostru.
\par 7 Veți alunga pe vrăjmașii voștri și vor cădea uciși înaintea voastră.
\par 8 Cinci din voi vor birui o sută și o sută din voi vor goni zece mii și vor cădea vrăjmașii voștri de sabie înaintea voastră.
\par 9 Căuta-voi spre voi și vă voi binecuvânta; veți avea copii, vă voi înmulți și voi fi statornic în legământul Meu cu voi.
\par 10 Veți mânca roadele vechi din anii trecuți și veți da la o parte pe cele vechi pentru a face loc celor noi.
\par 11 Voi așeza locașul Meu în mijlocul vostru și sufletul Meu nu se va scârbi de voi.
\par 12 Voi umbla printre voi, voi fi Dumnezeul vostru și voi poporul Meu.
\par 13 Eu sunt Domnul Dumnezeul vostru, Cel ce v-am scos din pământul Egiptului, ca să nu mai fiți robi acolo; am sfărâmat jugul vostru și v-am povățuit cu fruntea ridicată.
\par 14 Iar de nu Mă veți asculta și de nu veri păzi aceste porunci ale Mele,
\par 15 De veți disprețui așezămintele Mele și de se va scârbi sufletul vostru de legile Mele, neîmplinind poruncile Mele, și călcând legământul Meu,
\par 16 Atunci și Eu am să Mă port cu voi așa: Voi trimite asupra voastră groaza, lingoarea și frigurile, de care vi se vor secătui ochii și vi se va istovi sufletul; veți semăna semințele în zadar și vrăjmașii voștri le vor mânca.
\par 17 Îmi voi întoarce fața împotriva voastră și veți cădea înaintea vrăjmașilor voștri; vor domni peste voi dușmanii voștri și veți fugi când nimeni nu vă va alunga.
\par 18 Dacă nici după toate acestea nu Mă veți asculta, atunci înșeptit voi mări pedeapsa pentru păcatele voastre.
\par 19 Voi frânge îndărătnicia voastră cea mândră și cerul vostru îl voi face ca fierul, iar pământul vostru ca arama.
\par 20 În zadar vă veți cheltui puterile voastre, că pământul vostru nu-și va da roadele sale, nici pomii din țara voastră nu-și vor da poamele lor.
\par 21 Dacă și după acestea veți umbla împotriva Mea și nu veți vrea să Mă ascultați, atunci vă voi adăuga lovituri înșeptit pentru păcatele voastre;
\par 22 Voi trimite asupra voastră fiarele câmpului, care vă vor lipsi de copii; vor prăpădi vitele voastre și pe voi vă voi împuțina așa, încât se vor pustii drumurile voastre.
\par 23 Dacă nici după aceasta nu vă veți îndrepta, împotrivindu-vă Mie,
\par 24 Atunci și Eu voi veni cu mânie asupra voastră și vă voi lovi înșeptit pentru păcatele voastre.
\par 25 Voi aduce asupra voastră sabie răzbunătoare, ca să răzbune legământul Meu. Iar dacă vă veți ascunde în orașele voastre, voi trimite asupra voastră molimă și veți fi dați în mâinile vrăjmașului.
\par 26 Pâinea, care vă hrănește, o voi lua de la voi; zece femei vor coace pâine pentru voi într-un cuptor și vor da pâinea voastră cu cântarul și veți mânca și nu vă veți sătura.
\par 27 Dacă nici după aceasta nu Mă veți asculta și veți păși împotriva Mea,
\par 28 Atunci și Eu cu mânie voi veni asupra voastră și vă voi pedepsi înșeptit pentru păcatele voastre;
\par 29 Veți mânca din carnea fiilor voștri și din carnea fiicelor voastre.
\par 30 Dărâma-voi înălțimile voastre și voi strica stâlpii voștri; trupurile voastre le voi prăbuși sub dărâmăturile idolilor voștri și se va scârbi sufletul Meu de voi.
\par 31 Orașele voastre le voi preface în ruine, voi pustii locașurile voastre cele sfinte și nu voi mirosi miresmele plăcute ale jertfelor voastre.
\par 32 Pustii-voi pământul vostru așa încât să se mire de el vrăjmașii voștri care se vor așeza pe el;
\par 33 Iar pe voi vă voi risipi printre popoare; în urma voastră Îmi voi ridica sabia și va fi pământul vostru pustiu și orașele voastre dărâmate.
\par 34 Atunci pământul se va bucura de odihnele sale în zilele pustiirii lui; când voi veți fi în pământul vrăjmașilor voștri, atunci pământul vostru se va odihni și se va bucura de odihna lui.
\par 35 În toate zilele pustiirii lui el se va odihni cât nu s-a odihnit în zilele de odihnă ale voastre, când locuiați voi în el.
\par 36 Celor ce vor rămâne dintre voi le voi trimite în inimi frica în pământul vrăjmașilor lor; până și freamătul frunzei ce se clatină îi va pune pe fugă și vor fugi ca de sabie și vor cădea când nimeni nu-i va alunga.
\par 37 Se vor călca unul pe altul, ca cei ce fug de sabie, când nimeni nu-i va urmări și nu veți avea putere să vă împotriviți vrăjmașilor voștri.
\par 38 Veți pieri printre popoare și vă va înghiți pământul vrăjmașilor voștri.
\par 39 Iar cei ce vor rămâne din voi se vor usca pentru păcatele lor în pământurile vrăjmașilor. voștri, se vor usca și pentru păcatele părinților lor.
\par 40 Atunci își vor mărturisi fărădelegile lor și fărădelegile părinților lor, cum au săvârșit ei nelegiuiri împotriva Mea și au pășit împotriva Mea.
\par 41 Pentru care și Eu am venit cu mânie asupra lor și i-am adus în pământul vrăjmașilor lor; atunci se va supune inima lor cea netăiată împrejur și vor suferi ei pentru nelegiuirile lor.
\par 42 Și Eu Îmi voi aduce aminte de legământul Meu cu Iacov, de legământul Meu cu Isaac, și de legământul Meu cu Avraam Îmi voi aduce aminte și de pământ îmi voi aduce aminte.
\par 43 Când pământul va fi părăsit de ei și el se va bucura de odihna lui, golit fiind de ei, și ei vor suferi pentru fărădelegi și pentru că au nesocotit legile Mele și sufletul lor s-a scârbit de așezământul Meu;
\par 44 Când vor fi ei în pământul vrăjmașilor, Eu nu-i voi disprețui și nu Mă voi scârbi de ei, așa încât să-i pierd și să stric legământul Meu cu ei, că Eu sunt Domnul Dumnezeul lor.
\par 45 Îmi voi aminti de ei pentru legământul încheiat cu strămoșii lor, pe care i-am scos din pământul Egiptului, înaintea ochilor popoarelor, ca să fiu Dumnezeul lor. Eu sunt Domnul".
\par 46 Acestea sunt așezămintele, hotărârile și legile pe care le-a așezat Domnul între Sine și fiii lui Israel prin Moise, pe Muntele Sinai.

\chapter{27}

\par 1 A grăit Domnul cu Moise și a zis:
\par 2 "Vorbește fiilor lui Israel și le spune: De va făgădui cineva să-și afierosească sufletul său Domnului, prețuirea ta să fie așa:
\par 3 Prețul pentru un bărbat, de la douăzeci până la șaizeci de ani, să fie cincizeci de sicli de argint, după siclul sfânt.
\par 4 Iar dacă este femeie, prețul să fie treizeci de sicli.
\par 5 De la cinci până la douăzeci de ani, prețul să fie pentru bărbat douăzeci de sicli, iar pentru femeie zece sicli.
\par 6 Iar de la o lună până la cinci ani, prețul să fie pentru bărbat cinci sicli de argint, și pentru femeie trei sicli de argint.
\par 7 De la șaizeci de ani în sus prețul să fie pentru bărbat cincisprezece sicli de argint, și pentru femeie zece sicli.
\par 8 Iar dacă este sărac și nu e în stare să plătească prețul, atunci să fie adus la preot și să-l prețuiască preotul; potrivit cu starea celui ce și-a dat făgăduința să-l prețuiască preotul.
\par 9 Dacă însă va fi un dobitoc, ce se aduce jertfă Domnului, tot ce se aduce Domnului trebuie să fie sfânt.
\par 10 Să nu schimbe nici bun cu rău, nici rău cu bun; iar de schimbă cineva dobitoc cu dobitoc, atunci și cel schimbat și cel dat schimb va fi sfânt.
\par 11 Dacă ar fi cumva un dobitoc necurat, care nu se aduce jertfă Domnului, și el va fi adus la preot,
\par 12 Preotul îl va prețui ori de este bun, ori de este rău; și cum îl va prețui preotul, așa să fie.
\par 13 De va vrea cineva să-l răscumpere, atunci să adauge a cincea parte la preț.
\par 14 De va afierosi cineva casa sa Domnului, s-o prețuiască preotul, de este bună sau rea, și cum o va prețui preotul, așa să rămână.
\par 15 Dacă afierositorul va vrea să răscumpere casa sa, să adauge a cincea parte de argint la prețul  ei și va fi a lui.
\par 16 De va afierosi cineva Domnului țarină din moșia sa, prețuirea să se facă după venitul ei, cincizeci de sicli de argint pentru fiecare gomer de orz semănat.
\par 17 De își va afierosi țarina sa chiar din anul jubileu, să fie după prețul hotărât.
\par 18 Iar de își afierosește cineva țarina sa după anul jubileu, atunci preotul să socotească argintul după numărul anilor ce mai rămân până la anul jubileu și să scadă din prețul ei.
\par 19 Dacă însă va vrea să-și răscumpere țarina cel ce a afierosit-o, atunci el să adauge a cincea parte de argint la prețul ei și să rămână a lui.
\par 20 Iar dacă acela nu-și va răscumpăra țarina și va fi vândută altui om, atunci nu se mai. poate răscumpăra;
\par 21 Țarina aceea, când se va întoarce în anul jubileu, va fi afierosire Domnului, ca țarină jertfă, și va trece în stăpânirea preotului.
\par 22 Iar dacă cineva va afierosi Domnului o țarină cumpărată, care nu este din țarinile moșiei lui,
\par 23 Preotul să-i socotească partea de preț până la anul jubileu, și acela să-i dea prețul în aceeași zi, ca afierosire Domnului,
\par 24 Și țarina în anul jubileu va trece iar la acela, de la care a fost cumpărată și din moșia căruia a fost pământul acela.
\par 25 Toate prețurile să fie făcute după siclul sfânt; siclul să aibă douăzeci de ghere.
\par 26 Numai întâii născuți ai dobitoacelor, care după întâietatea nașterii sunt ai Domnului, să nu-i afierosească nimeni: fie bou, fie oaie, că sunt ai Domnului.
\par 27 Iar dacă este dobitoc necurat, să fie răscumpărat după prețuirea ta, la care să se mai adauge a cincea parte, și de nu se va răscumpăra să se vândă după prețuirea ta.
\par 28 Toate cele afierosite, pe care omul cu jurământ le dă Domnului din ale sale, - fie om, fie dobitoc, fie țarină din moșia sa, - nici nu se răscumpără, nici nu se vând. Tot ce este afierosit cu jurământ este sfințenie mare a Domnului.
\par 29 Orice om afierosit cu jurământ nu se răscumpără, ci trebuie să se dea morții.
\par 30 Toată dijma de la pământ, din roadele pământului și din roadele pomilor este a Domnului, sfințenia Domnului.
\par 31 Și de va voi cineva să-și răscumpere dijma, să adauge la prețul ei a cincea parte.
\par 32 Toată dijma de la boi și de la oi și tot al zecelea din câte trec pe sub toiag este afierosit Domnului.
\par 33 Nu trebuie căutat de este bun sau rău și nu trebuie schimbat; dar de-l va schimba cineva, atunci și cel schimbat și schimbul vor fi sfinte și nu se vor putea răscumpăra".
\par 34 Acestea sunt poruncile pe care le-a poruncit Domnul lui Moise pe Muntele Sinai pentru fiii lui Israel.


\end{document}