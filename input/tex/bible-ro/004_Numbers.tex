\begin{document}

\title{Numbers}

Num 1:1  În ziua întâi a lunii a doua din anul al doilea dupa ie?irea Israeli?ilor din pamântul Egiptului, a grait Domnul cu Moise în cortul adunarii, în pustiul Sinai, ?i a zis:
Num 1:2  "Numara?i toata ob?tea fiilor lui Israel dupa semin?iile lor, dupa familiile lor ?i dupa numele lor, om cu om.
Num 1:3  Tot barbatul de la douazeci de ani în sus, tot cel ce poate ie?i la oaste în Israel, sa se numere de tine ?i de Aaron ?i sa se rânduiasca în tabara lui.
Num 1:4  Dar cu voi sa mai fie din fiecare semin?ie câte un om, care e cel mai de seama în neamul sau.
Num 1:5  Iata numele barba?ilor care vor fi cu voi: din Ruben: Eli?ur, fiul lui ?edeur;
Num 1:6  Din Simeon: ?elumiel, fiul lui ?uri?adai;
Num 1:7  Din Iuda: Naason, fiul lui Aminadab;
Num 1:8  Din Isahar: Natanael, fiul lui ?uar;
Num 1:9  Din Zabulon: Eliab, fiul lui Helon;
Num 1:10  Din fiii lui Iosif: Eli?ama, fiul lui Amihud, din Efraim; ?i Gamaliel, fiul lui Peda?ur, din Manase;
Num 1:11  Din Veniamin: Abidan, fiul lui Ghedeon;
Num 1:12  Din Dan: Ahiezer, fiul lui Ami?adai;
Num 1:13  Din A?er: Paghiel, fiul lui Ocran;
Num 1:14  Din Gad: Eliasaf, fiul lui Raguel;
Num 1:15  Din Neftali: Ahira, fiul lui Enan.
Num 1:16  Ace?tia sunt barba?ii ale?i ai ob?tii, capii semin?iilor parin?ilor lor, capeteniile peste mii în Israel".
Num 1:17  Luând deci Moise ?i Aaron pe barba?ii ace?tia, care au fost numi?i pe numele lor,
Num 1:18  Au adunat toata ob?tea în ziua întâi a lunii a doua din anul al doilea ?i au înscris, dupa spi?ele neamului lor, pe to?i barba?ii de la douazeci de ani în sus pe semin?ii, pe familii ?i pe numele lor, om cu om.
Num 1:19  Numaratoarea aceasta a facut-o Moise în pustiul Sinai, cum îi poruncise Domnul.
Num 1:20  Fiii lui Ruben, întâiul nascut al lui Israel, dupa semin?ia lor, dupa neamurile lor, dupa familiile lor, dupa numele lor, to?i barba?ii om cu om, de la douazeci de ani în sus, to?i cei buni de oaste,
Num 1:21  S-au numarat în semin?ia lui Ruben patruzeci ?i ?ase de mii cinci sute.
Num 1:22  Fiii lui Simeon, dupa semin?ia lor, dupa neamurile lor, dupa familiile lor, dupa numele lor, to?i barba?ii om cu om, de la douazeci de ani în sus, to?i cei buni de oaste,
Num 1:23  S-au numarat în semin?ia lui Simeon cincizeci ?i noua de mii trei sute.
Num 1:24  Fiii lui Gad, dupa semin?ia lor, dupa neamurile lor, dupa familiile lor, dupa numele lor, to?i barba?ii om cu om, de la douazeci de ani în sus, to?i cei buni de oaste,
Num 1:25  S-au numarat în semin?ia lui Gad patruzeci ?i cinci de mii ?ase sute cincizeci.
Num 1:26  Fiii lui Iuda, dupa semin?ia lor, dupa neamurile lor, dupa familiile lor, dupa numele lor, to?i barba?ii om cu om, de la douazeci de ani în sus, to?i cei buni de oaste,
Num 1:27  S-au numarat în semin?ia lui Iuda ?aptezeci ?i patru de mii ?ase sute.
Num 1:28  Fiii lui Isahar, dupa semin?ia lor, dupa neamurile lor, dupa familiile lor, dupa numele lor, to?i barba?ii om cu om, de ia douazeci de ani în sus, to?i cei buni de oaste,
Num 1:29  S-au numarat în semin?ia lui Isahar cincizeci ?i patru de mii patru sute.
Num 1:30  Fiii lui Zabulon, dupa semin?iile lor, dupa neamurile lor, dupa familiile lor, dupa numele lor, to?i barba?ii om cu om, de la douazeci de ani în sus, to?i cei buni de oaste,
Num 1:31  S-au numarat în semin?ia lui Zabulon cincizeci ?i ?apte de mii patru sute.
Num 1:32  Fiii lui Iosif: fiii lui Efraim, dupa semin?ia lor, dupa neamurile lor, dupa familiile lor, dupa numele lor, to?i barba?ii om cu om, de la douazeci de ani în sus, to?i cei buni de oaste,
Num 1:33  S-au numarat în semin?ia lui Efraim patruzeci de mii cinci sute.
Num 1:34  Fiii lui Manase, dupa semin?ia lor, dupa neamurile lor, dupa familiile lor, dupa numele lor, to?i barba?ii om cu om, de la douazeci de ani în sus, to?i cei buni de oaste,
Num 1:35  S-au numarat în semin?ia lui Manase treizeci ?i doua de mii doua sute.
Num 1:36  Fiii lui Veniamin, dupa semin?ia lor, dupa neamurile lor, dupa familiile lor, dupa numele lor, to?i barba?ii om cu om, de la douazeci de ani în sus, to?i cei buni de oaste,
Num 1:37  S-au numarat în semin?ia lui Veniamin treizeci ?i cinci de mii patru sute.
Num 1:38  Fiii lui Dan, dupa semin?ia lor, dupa neamurile lor, dupa familiile lor, dupa numele lor, to?i barba?ii om cu om, de la douazeci de ani în sus, to?i cei buni de oaste,
Num 1:39  S-au numarat în semin?ia lui Dan, ?aizeci ?i doua de mii ?apte sute.
Num 1:40  Fiii lui A?er, dupa semin?ia lor, dupa neamurile lor, dupa familiile lor, dupa numele lor, to?i barba?ii om cu om, de la douazeci de ani în sus, to?i cei buni de oaste,
Num 1:41  S-au numarat în semin?ia lui A?er patruzeci ?i una de mii cinci sute.
Num 1:42  Fiii lui Neftali, dupa semin?ia lor, dupa neamurile lor, dupa familiile lor, dupa numele lor, to?i barba?ii om cu om, de la douazeci de ani în sus, to?i cei buni de oaste,
Num 1:43  S-au numarat în semin?ia lui Neftali cincizeci ?i trei de mii patru sute.
Num 1:44  Ace?tia sunt cei care au intrat la numaratoarea facuta de Moise ?i Aaron ?i de cei doisprezece barba?i, capeteniile lui Israel, câte un barbat de fiecare semin?ie, dupa neamul stramo?esc.
Num 1:45  Deci to?i fiii lui Israel de la douazeci de ani în sus, buni de oaste, care au intrat la numaratoare, dupa familiile lor,
Num 1:46  Au fost ?ase sute trei mii cinci sute cincizeci.
Num 1:47  Iar levi?ii, dupa semin?ia parin?ilor lor, n-au fost numara?i între ei.
Num 1:48  ?i a grait Domnul cu Moise ?i a zis:
Num 1:49  "Vezi ca semin?ia lui Levi sa n-o bagi la numaratoare ?i sa nu-i numeri pe fiii lui Levi cu fiii lui Israel;
Num 1:50  Ci rânduie?te pe levi?i la cortul adunarii ?i le încredin?eaza toate lucrurile lui ?i toate câte sunt în el. Ei sa poarte cortul ?i toate lucrurile lui, sa slujeasca în el ?i sa î?i a?eze tabara împrejurul lui.
Num 1:51  Când va fi sa plece cortul, levi?ii sa-l strânga, ?i când va fi sa se opreasca, levi?ii sa-l a?eze; iar de se va apropia unul strain, sa fie omorât.
Num 1:52  Fiii lui Israel sa poposeasca fiecare în tabara sa ?i fiecare sub steagul sau ?i în cetele lor.
Num 1:53  Iar levi?ii sa-?i a?eze tabara aproape, împrejurul cortului adunarii, ca sa nu vina mânia asupra ob?tii fiilor lui Israel; ?i sa strajuiasca levi?ii la cortul adunarii".
Num 1:54  ?i au facut fiii lui Israel toate câte poruncise Domnul lui Moise ?i Aaron; a?a au facut.
Num 2:1  Atunci a grait Domnul cu Moise ?i cu Aaron ?i a zis:
Num 2:2  "Fiii lui Israel sa poposeasca fiecare lânga steagul sau, în preajma semnelor familiei sale, ?i sa-?i a?eze taberele înaintea cortului marturiei ?i împrejurul lui.
Num 2:3  Întâi, spre rasarit, sa poposeasca steagul taberei lui Iuda, cu cetele sale, cu Naason, fiul lui Aminadab, capetenia fiilor lui Iuda,
Num 2:4  ?i cu o?tenii sai în numar de ?aptezeci ?i patru de mii ?ase sute.
Num 2:5  Alaturi sa poposeasca semin?ia lui Isahar, cu Natanael, fiul lui ?uar, capetenia fiilor lui Isahar,
Num 2:6  ?i cu o?tenii sai în numar de cincizeci ?i patru de mii patru sute.
Num 2:7  Mai departe va poposi semin?ia lui Zabulon, cu Eliab, fiul lui Helon, capetenia fiilor lui Zabulon,
Num 2:8  Cu o?tenii sai în numar de cincizeci ?i ?apte de mii patru sute.
Num 2:9  To?i ace?tia în numar de o suta optzeci ?i ?ase de mii patru sute, care ?in de tabara lui Iuda, sa plece întâi.
Num 2:10  Spre miazazi sa se a?eze tabara lui Ruben, cu cetele sale ?i Eli?ur, fiul lui ?edeur, capetenia fiilor lui Ruben,
Num 2:11  ?i cu o?tenii sai în numar de patruzeci ?i ?ase de mii cinci sute.
Num 2:12  Lânga el va poposi semin?ia lui Simeon, cu ?elumiel, fiul lui ?uri?adai, capetenia fiilor lui Simeon,
Num 2:13  ?i cu o?tenii sai în numar de cincizeci ?i noua de mii trei sute.
Num 2:14  Dupa acesta va poposi semin?ia lui Gad, cu Eliasaf, fiul lui Raguel, capetenia fiilor lui Gad,
Num 2:15  ?i cu o?tenii sai în numar de patruzeci ?i cinci de mii ?ase sute cincizeci.
Num 2:16  To?i ace?tia cu luptatorii lor în numar de o suta cincizeci ?i una de mii patru sute cincizeci, care ?in de tabara lui Ruben, rândui?i în tabere, vor pleca în rândul al doilea.
Num 2:17  Dupa aceea, când va pleca cortul adunarii, tabara levi?ilor va fi în mijlocul taberelor ?i precum au poposit a?a sa ?i plece, fiecare la rândul sau ?i sub steagul sau.
Num 2:18  Spre apus va poposi tabara lui Efraim cu cetele sale ?i cu Eli?ama, fiul lui Amihud, capetenia fiilor lui Efraim,
Num 2:19  ?i cu o?tenii sai în numar de patruzeci de mii cinci sute.
Num 2:20  Lânga ea se va a?eza semin?ia lui Manase cu Gamaliel, fiul lui Peda?ur, capetenia fiilor lui Manase,
Num 2:21  ?i cu o?tenii sai în numar de treizeci ?i doua de mii doua sute.
Num 2:22  Dupa acesta semin?ia lui Veniamin cu Abidan, fiul lui Ghedeon, capetenia fiilor lui Veniamin,
Num 2:23  ?i cu o?tenii lui în numar de treizeci ?i cinci de mii patru sute.
Num 2:24  To?i ace?tia cu luptatorii lor în numar de o suta opt mii o suta, care ?in de tabara lui Efraim, vor pleca în al treilea rând, a?eza?i în cete.
Num 2:25  La miazanoapte se va a?eza tabara lui Dan cu cetele sale ?i cu Ahiezer, fiul lui Ami?adai, capetenia fiilor lui Dan,
Num 2:26  ?i cu o?tenii sai în numar de ?aizeci ?i doua de mii ?apte sute.
Num 2:27  Lânga el î?i va a?eza tabara semin?ia lui A?er, cu Paghiel, fiul lui Ocran, capetenia fiilor lui A?er,
Num 2:28  ?i cu o?tenii sai în numar de patruzeci ?i una de mii cinci sute.
Num 2:29  Mai departe î?i va a?eza tabara semin?ia lui Neftali cu Ahira, fiul lui Enan, capetenia fiilor lui Neftali,
Num 2:30  ?i cu o?tenii sai în numar de cincizeci ?i trei de mii patru sute.
Num 2:31  To?i ace?tia cu luptatorii lor în numar de o suta cincizeci ?i ?apte de mii ?ase sute, care ?in de tabara lui Dan, sa plece la urma sub steagurile lor ?i rândui?i în cete".
Num 2:32  Ace?tia sunt fiii lui Israel care au intrat la numaratoare dupa familiile lor. To?i, câ?i au intrat la numaratoare pe tabere ?i pe cete, erau ?ase sute trei mii cinci sute cincizeci.
Num 2:33  Iar levi?ii nu s-au numarat cu ei, dupa cum poruncise Domnul lui Moise.
Num 2:34  ?i au facut fiii lui Israel toate câte poruncise Domnul lui Moise: a?a se a?ezau în tabere sub steagurile lor ?i a?a purcedeau fiecare cu semin?ia sa ?i cu familia sa.
Num 3:1  Iata acum spi?a neamului lui Aaron ?i a lui Moise, din timpul când a grait Domnul cu Moise pe Muntele Sinai, ?i iata numele fiilor lui Aaron:
Num 3:2  Nadab, întâiul nascut, Abiud, Eleazar ?i Itamar.
Num 3:3  Acestea sunt numele fiilor lui Aaron, preo?i mirui?i, care au fost sfin?i?i, ca sa slujeasca cele ale preo?iei.
Num 3:4  Insa Nadab ?: Abiud au murit înaintea fe?ei Domnului, când au adus foc strain înaintea fe?ei Domnului în pustiul Sinai, neavând copii, ?i au ramas preo?i numai Eleazar ?i Itamar cu tatal lor Aaron.
Num 3:5  Atunci a grait Domnul cu Moise ?i a zis:
Num 3:6  "Ia semin?ia lui Levi ?i o pune la îndemâna lui Aaron preotul, ca sa-l ajute în slujba lui.
Num 3:7  Sa fie de paza în locul lui ?i în locul fiilor lui Israel la cortul adunarii; sa faca slujbele la cort;
Num 3:8  Sa pastreze toate lucrurile cortului adunarii, sa strajuiasca în locul fiilor lui Israel ?i sa faca slujbele la cort.
Num 3:9  Da pe levi?i la îndemâna lui Aaron, fratele tau, ?i fiilor lui, preo?ilor; sa-Mi fie darui?i Mie dintre fiii lui Israel.
Num 3:10  Iar lui Aaron ?i fiilor lui încredin?eaza-le cortul adunarii, ca sa-?i pazeasca datoria lor preo?easca ?i toate cele de la jertfelnic ?i de dupa perdea; iar de se va apropia cineva strain, sa fie omorât".
Num 3:11  ?i a grait Domnul cu Moise ?i a zis:
Num 3:12  "Iata, Eu am luat din fiii lui Israel pe levi?i în locul tuturor întâilor nascu?i, în locul tuturor celor ce se nasc întâi în Israel ?i aceia vor fi în locul acestora.
Num 3:13  Levi?ii sa fie ai Mei, caci to?i întâi-nascu?ii sunt ai Mei. În ziua când am lovit pe to?i întâi-nascu?ii în pamântul Egiptului, atunci Mi-am sfin?it pe to?i întâi-nascu?ii lui Israel de la om pâna la dobitoc ?i ace?tia sa fie ai Mei. Eu  sunt Domnul".
Num 3:14  Iara?i a grait Domnul cu Moise, în pustiul Sinai, ?i a zis:
Num 3:15  "Numara pe fiii lui Levi, dupa familiile lor, dupa neamurile lor; pe to?i cei de parte barbateasca, de la o luna în sus sa-i numeri".
Num 3:16  ?i i-au numarat Moise ?i Aaron, dupa cuvântul Domnului, cum le poruncise Domnul.
Num 3:17  Iata dar care sunt fiii lui Levi, dupa numele lor: Gher?on, Cahat ?i Merari.
Num 3:18  Iar numele fiilor lui Gher?on, dupa neamurile lor, sunt: Libni ?i ?imei.
Num 3:19  Fiii lui Cahat, dupa neamurile lor, sunt: Amram, I?har, Hebron ?i Uziel.
Num 3:20  Fiii lui Merari, dupa neamurile lor, sunt: Mahli ?i Mu?i. Acestea sunt neamurile lui Levi, dupa familiile lor.
Num 3:21  Din Gher?on au ie?it neamul lui Libni ?i neamul lui ?imei: aceste neamuri sunt din Gher?on.
Num 3:22  Socotindu-se la numar tot cel de parte barbateasca, de la o luna în sus, s-au numarat în neamul lui Gher?on ?apte mii cinci sute.
Num 3:23  Fiii lui Gher?on trebuia sa se a?eze cu tabara în urma cortului, spre asfin?it.
Num 3:24  Eliasaf, fiul lui Lael, era capetenia familiei fiilor lui Gher?on.
Num 3:25  Fiii lui Gher?on la cortul adunarii aveau sa pazeasca cortul ?i acoperi?ul lui, perdeaua de la u?a cortului adunarii,
Num 3:26  Perdelele cur?ii, perdeaua de la intrarea cur?ii celei dimprejurul cortului ?i jertfelnicului, frânghiile ?i toate uneltele lor.
Num 3:27  Din Cahat a ie?it familia lui Amram, familia lui I?har, familia lui Hebron ?i familia lui Uziel: aceste familii sunt din Cahat.
Num 3:28  Socotindu-se la numar tot cel de parte barbateasca, de la o luna în sus, s-au numarat în neamul acesta opt mii trei sute. Ei pazeau loca?ul sfânt.
Num 3:29  Familiile fiilor lui Cahat trebuia sa-?i a?eze tabara lânga cort, în partea de miazazi;
Num 3:30  Iar capetenie în familiile neamului lui Cahat era El?afan, fiul lui Uziel.
Num 3:31  În paza lor se afla chivotul, masa, sfe?nicul, jertfelnicul, vasele sfinte, care se întrebuin?eaza la slujbe, ?i perdeaua cu toate ale ei.
Num 3:32  Capetenie peste capeteniile levi?ilor era Eleazar, fiul preotului Aaron , care era rânduit sa privegheze pe cei ce aveau în pastrare loca?ul sfânt.
Num 3:33  Din Merari au ie?it familia lui Mahli ?i familia lui Mu?i; aceste familii sunt din Merari.
Num 3:34  Socotindu-se la numar tot cel de parte barbateasca de la o luna în sus, s-au numarat în neamul acesta ?ase mii doua sute;
Num 3:35  Iar capetenie peste familiile din neamul lui Merari era ?uriel, fiul lui Abihael. Ace?tia trebuia sa-?i a?eze tabara lânga cort, în partea de miazanoapte.
Num 3:36  În paza fiilor lui Merari s-au rânduit scândurile dimprejurul cortului, pârghiile lui, stâlpii lui, postamentele acestora, ?i toate lucrurile ?i uneltele lor,
Num 3:37  Stâlpii cur?ii din toate par?ile ei, postamentele lor, ?aru?ii cur?ii ?i frânghiile ei.
Num 3:38  Iar în partea de dinainte a cortului adunarii, spre rasarit, trebuia sa-?i a?eze tabara Moise ?i Aaron ?i fiii acestuia, carora li se încredin?ase paza loca?ului sfânt în locul fiilor lui Israel. Iar de se va apropia vreun strain, sa fie omorât.
Num 3:39  Deci to?i levi?ii numara?i, pe care i-au numarat Moise ?i Aaron, cum poruncise Domnul, dupa neamurile lor, parte barbateasca, de la o luna în sus, au fost douazeci ?i doua de mii.
Num 3:40  Apoi a zis Domnul catre Moise: "Socote?te pe tot barbatul întâi-nascut dintre fiii lui Israel, de la o luna în sus, ?i ia numarul numelor lor.
Num 3:41  ?i în locul tuturor întâi-nascu?ilor ai fiilor lui Israel, sa iei pentru Mine pe levi?i. Eu sunt Domnul. ?i vitele levi?ilor sa le iei în locul a tot întâi-nascutului din vitele fiilor lui Israel".
Num 3:42  ?i a numarat Moise, dupa cum îi poruncise Domnul, pe to?i întâi-nascu?ii dintre fiii lui Israel;
Num 3:43  ?i întâi-nascu?ii de parte barbateasca de la o luna în sus, dupa numarul numelor, au fost to?i douazeci ?i doua de mii doua sute ?aptezeci ?i trei.
Num 3:44  ?i a grait Domnul cu Moise ?i a zis:
Num 3:45  "Ia pe levi?i în locul tuturor întâi-nascu?ilor fiilor lui Israel ?i vitele levi?ilor în locul vitelor lor, ?i sa fie levi?ii ai Mei. Eu sunt Domnul!
Num 3:46  Iar ca rascumparare pentru cei doua sute ?aptezeci ?i trei de întâi-nascu?i ai fiilor lui Israel, care trec peste numarul levi?ilor,
Num 3:47  Sa iei câte cinci sicli de cap, socotind câte douazeci de ghere într-un siclu, dupa siclul sfânt,
Num 3:48  ?i argintul acesta sa-l dai lui Aaron ?i fiilor lui, ca rascumparare pentru cei ce prisosesc peste numarul lor".
Num 3:49  ?i adunând Moise argintul de rascumparare pentru întâi-nascu?ii lui Israel, care treceau peste numarul levi?ilor,
Num 3:50  S-au gasit o mie trei sute ?aizeci ?i cinci de sicli, dupa siclul sfânt.
Num 3:51  ?i a dat Moise argintul de rascumparare, pentru cei ce prisoseau, lui Aaron ?i fiilor lui, dupa cuvântul Domnului, precum poruncise Domnul lui Moise.
Num 4:1  ?i a grait Domnul cu Moise ?i cu Aaron ?i a zis:
Num 4:2  "Numara din fiii lui Levi pe fiii lui Cahat, dupa neamurile ?i dupa familiile lor, pe to?i cei buni de slujba,
Num 4:3  De la treizeci de ani în sus pâna la cincizeci de ani, ca sa lucreze la cortul adunarii.
Num 4:4  Slujba fiilor lui Cahat la cortul adunarii va fi sa duca sfânta sfintelor.
Num 4:5  Când va pleca tabara, sa intre Aaron ?i fiii lui, sa ia perdeaua despar?itoare între sfânta ?i sfânta sfintelor, sa înveleasca cu ea chivotul legii;
Num 4:6  Sa puna apoi un acoperamânt de piei vinete, iar pe deasupra aceluia sa arunce un înveli? de lâna albastra ?i sa puna pârghiile la chivot.
Num 4:7  Apoi sa a?tearna pe masa pâinilor punerii înainte o fa?a de masa violeta ?i sa puna pe ea blidele, talerele, oalele ?i cupele cele pentru turnat, ?i pâinile ei pururea sa fie pe ea.
Num 4:8  Peste acestea sa puna o poala purpurie, iar pe deasupra ei sa puna un acoperamânt de piele vânata ?i sa-i puna pârghiile.
Num 4:9  Sa ia apoi o îmbracaminte violeta ?i sa acopere sfe?nicul ?i candelele lui, cle?tele lui, plutele lui, ?i toate vasele cele pentru untdelemn, care se întrebuin?eaza la el.
Num 4:10  Sa-l acopere pe el ?i toate uneltele lui cu un acoperamânt de piei vinete ?i sa-l puna pe nasalie.
Num 4:11  Peste jertfelnicul cel de aur sa puna o îmbracaminte violeta, sa-l acopere cu un acoperamânt de piei vinete ?i apoi sa-i a?eze pârghiile în verigi.
Num 4:12  Sa ia toate lucrurile cele pentru slujba, care se întrebuin?eaza la slujba în loca?ul sfânt, ?i sa le puna în înveli?uri de lâna violeta, sa le acopere cu acoperaminte de piei vinete ?i sa le puna pe nasalie.
Num 4:13  Dupa aceea sa cure?e jertfelnicul de cenu?a, sa-l acopere cu o îmbracaminte violeta,
Num 4:14  Sa puna pe el toate vasele lui, care se întrebuin?eaza la el în timpul slujbei: cle?tele, furculi?ele, lope?ile, oalele ?i toate vasele jertfelnicului, sa-l acopere cu un acoperamânt de piei vinete ?i sa-i puna pârghiile. Sa ia apoi o îmbracaminte violeta ?i sa acopere baia ?i postamentul ei; sa puna pe deasupra lor un acoperamânt vânat de piele ?i sa le puna pe nasalie.
Num 4:15  Dupa aceea Aaron ?i fiii lui, înainte de plecarea taberei la drum, vor strânge tot cortul ?i vor înveli toate lucrurile loca?ului sfânt, iar fiii lui Cahat vor veni sa le ia; dar nu trebuie sa se atinga ei de sfânta sfintelor, ca sa nu moara. Aceste lucruri ale cortului sa le duca fiii lui Cahat.
Num 4:16  Eleazar, fiul preotului Aaron, va fi supraveghetor peste untdelemnul pentru sfe?nic, aromatele de tamâiat, darul zilnic de pâine ?i mirul; va avea ?i supraveghere peste tot cortul ?i peste câte sunt în el ?i în loca?ul sfânt ?i peste toate lucrurile".
Num 4:17  ?i a grait Domnul cu Moise ?i cu Aaron ?i a zis:
Num 4:18  "Sa nu lasa?i sa se stinga samân?a neamului lui Cahat dintre levi?i.
Num 4:19  Iata ce trebuie sa le face?i, ca sa traiasca ?i sa nu moara, când se vor apropia de sfânta sfintelor: sa vina Aaron ?i fiii lui ?i sa puna pe fiecare la slujba lui ?i la sarcina lui;
Num 4:20  Dar ei sa nu vina sa priveasca la cele sfinte, când le învelesc, ca sa nu moara".
Num 4:21  ?i a grait Domnul cu Moise ?i a zis:
Num 4:22  "Numara ?i pe fiii lui Gher?on, pe familii ?i pe neamuri, de la treizeci de ani pâna la cincizeci de ani;
Num 4:23  ?i sa numeri pe to?i cei buni de slujba, ca sa lucreze la cortul adunarii.
Num 4:24  Iata slujba familiilor lui Gher?on, adica ce au de facut ?i de dus:
Num 4:25  Sa duca acoperi?urile cortului, cortul adunarii, acoperi?ul lui, acoperi?ul cel de piei vinete, care e pe deasupra lor, perdeaua, care se atârna la u?a cortului adunarii,
Num 4:26  Perdeaua de la poarta cur?ii, pânzele cur?ii celei dimprejurul cortului ?i a jertfelnicului, frânghiile lor ?i toate lucrurile lor de slujba ?i tot ce este de facut la ele sa faca ei.
Num 4:27  Toate slujbele fiilor lui Gher?on la ducerea poverilor ?i la toate lucrarile lor trebuie sa se faca dupa porunca lui Aaron ?i a fiilor lui ?i lor sa le încredin?a?i spre pastrare tot ceea ce au ei de dus.
Num 4:28  Acestea sunt slujbele fiilor lui Gher?on la cortul adunarii ?i acestea li se vor încredin?a spre pastrare sub. supravegherea lui Itamar, fiul preotului Aaron.
Num 4:29  Pe fiii lui Merari, iar sa-i numeri, dupa neamurile ?i dupa familiile lor,
Num 4:30  De la treizeci de ani în sus pâna la cincizeci de ani. Sa numeri pe to?i cei buni de slujba, ca sa lucreze la cortul adunarii.
Num 4:31  Iata ce sa duca ei, dupa slujba lor la cortul adunarii: scândurile cortului cu pârghiile lor, stâlpii lui cu postamentele lor, funiile cortului cu ?aru?ii lor;
Num 4:32  Stâlpii cur?ii de pe toate laturile ei cu postamentele lor, ?aru?ii cur?ii cu frânghiile lor, toate uneltele lor ?i tot ce ?ine de ele. Sa numara?i pe nume toate lucrurile ce sunt datori sa duca.
Num 4:33  Aceasta-i slujba neamului fiilor lui Merari ?i tot ce au sa faca la cortul adunarii, sub supravegherea lui Itamar, fiul preotului Aaron".
Num 4:34  Atunci au numarat Moise ?i Aaron cu capeteniile ob?tii pe fiii lui Cahat dupa neamurile ?i dupa familiile lor,
Num 4:35  De la treizeci de ani în sus pâna la cincizeci de ani, pe to?i cei buni de slujba, ea sa lucreze la cortul adunarii.
Num 4:36  ?i s-au gasit la numaratoare, dupa familiile lor, doua mii ?apte sute cincizeci.
Num 4:37  Acesta este numarul fiilor lui Cahat, to?i cei buni de slujba la cortul adunarii, pe care i-au numarat Moise ?i Aaron, dupa porunca Domnului, data prin Moise.
Num 4:38  S-au numarat apoi fiii lui Gher?on, dupa neamurile ?i dupa familiile lor,
Num 4:39  De la treizeci de ani în sus, pâna la cincizeci de ani, to?i cei buni de slujba, ca sa lucreze la cortul adunarii.
Num 4:40  ?i s-au gasit la numaratoare, dupa neamurile ?i dupa familiile lor, doua mii ?ase sute treizeci.
Num 4:41  Acesta este numarul fiilor lui Gher?on, to?i cei buni de slujba la cortul adunarii, pe care i-au numarat Moise ?i Aaron, dupa porunca Domnului.
Num 4:42  S-a numarat dupa aceea ?i neamul fiilor lui Merari, dupa rudeniile ?i dupa familiile lor,
Num 4:43  De la treizeci de ani în sus, pâna la cincizeci de ani, to?i cei buni de slujba, ca sa lucreze la cortul adunarii.
Num 4:44  ?i s-au gasit la numaratoare, dupa neamul lor ?i dupa familii, trei mii doua sute.
Num 4:45  Acesta este numarul fiilor lui Merari, pe care i-au numarat Moise ?i Aaron, dupa porunca Domnului, data prin Moise.
Num 4:46  To?i levi?ii, numara?i de Moise ?i de Aaron ?i de capeteniile lui Israel, dupa neamurile ?i dupa familiile lor,
Num 4:47  De la treizeci de ani în sus, pâna la cincizeci de ani, to?i cei buni de slujba, ca sa lucreze la cortul adunarii ?i sa-l duca,
Num 4:48  S-au gasit la numaratoare opt mii cinci sute optzeci.
Num 4:49  ?i dupa porunca Domnului, data prin Moise, s-au rânduit fiecare la lucrul sau ?i la slujba sa, ?i au fost numara?i, cum poruncise Domnul lui Moise.
Num 5:1  ?i a grait Domnul cu Moise ?i a zis:
Num 5:2  "Porunce?te fiilor lui Israel sa scoata din tabara pe to?i lepro?ii, pe to?i cei ce au scurgere ?i pe to?i cei întina?i prin atingere de mort.
Num 5:3  De la barbat pâna la femeie sa-i scoate?i ?i sa-i trimite?i afara din tabara, ca sa nu pângareasca taberele lor, în mijlocul carora locuiesc Eu".
Num 5:4  ?i au facut a?a fiii lui Israel: i-au scos afara din tabara. Cum poruncise Domnul lui Moise, a?a au facut fiii lui Israel.
Num 5:5  ?i a grait Domnul lui Moise ?i a zis:
Num 5:6  "Spune fiilor lui Israel: Daca un barbat sau o femeie va face vreun pacat fa?a de un om, ?i prin aceasta va pacatui împotriva Domnului ?i va fi vinovat sufletul acela,
Num 5:7  Sa-?i marturiseasca pacatul ce a facut ?i sa întoarca deplin aceea prin ce a pacatuit ?i sa mai adauge la aceea a cincea parte ?i sa dea aceluia fa?a de care a pacatuit.
Num 5:8  Daca însa omul acela nu va avea mo?tenitor, caruia sa se dea cele pentru gre?eala, atunci sa le dea Domnului ?i vor fi ale preotului, pe lânga berbecul de cura?ire, cu care acesta îl va cura?i.
Num 5:9  Toata pârga din toate darurile fiilor lui Israel, pe care le aduc ei la preot, sa fie ale lui.
Num 5:10  Orice lucru afierosit sa fie al lui; ?i orice va da cineva preotului este al lui".
Num 5:11  ?i a grait Domnul lui Moise ?i a zis:
Num 5:12  "Graie?te fiilor lui Israel ?i zi catre ei: De va gre?i femeia unui barbat ?i-l va în?ela,
Num 5:13  ?i va dormi cineva cu ea în pat, ?i lucrul va fi ascuns de barbatul ei, ?i ea se va spurca pe ascuns, ?i nu vor fi martori împotriva ei, nici nu va fi prinsa asupra faptului;
Num 5:14  De va cadea asupra barbatului duhul îndoielii banuind pe femeia sa, vinovata fiind aceasta, sau de va cadea asupra lui duhul îndoielii ?i va banui femeia sa, nevinovata fiind:
Num 5:15  Sa-?i aduca barbatul femeia sa la preot ?i sa aduca jertfa pentru ea a zecea parte de efa de faina de orz, dar sa nu toarne deasupra untdelemn, nici sa puna tamâie, pentru ca acesta este dar de banuiala, dar de amintire, care aminte?te vinova?ia;
Num 5:16  Iar preotul sa o aduca ?i sa o puna înaintea Domnului.
Num 5:17  Apoi sa ia preotul apa curata de izvor într-un vas de lut, sa ia ?arâna din pamânt de dinaintea cortului adunarii ?i sa o puna în apa.
Num 5:18  Dupa aceea sa puna preotul femeia înaintea Domnului, sa descopere capul femeii ?i sa-i dea în mâini darul de pomenire, darul de banuiala, iar preotul sa aiba în mâini apa cea amara, care aduce blestemul.
Num 5:19  Apoi sa jure preotul femeia ?i sa-i zica: Daca n-a dormit nimeni cu tine ?i tu nu te-ai spurcat ?i n-ai calcat credincio?ia catre barbatul tau, nevatamata sa fii de aceasta apa amara care aduce blestem;
Num 5:20  Iar de te-ai abatut, fiind maritata, ?i te-ai spurcat, de a dormit cineva cu tine, afara de barbatul tau,
Num 5:21  Atunci sa dea Domnul sa fii de blestem ?i de ocara în poporul tau; sa faca Domnul ca sânul tau sa cada ?i sa se umfle pântecele tau.
Num 5:22  ?i apa aceasta, care aduce blestem, sa intre înauntrul tau, ca sa ti se umfle pântecele ?i sa-?i cada sânul tau. Iar femeia sa zica: Amin, amin!
Num 5:23  Apoi sa scrie preotul juramintele acestea pe hârtie, sa le moaie în apa cea amara,
Num 5:24  ?i sa dea femeii sa bea apa amara aducatoare de blestem, ?i va înghi?i ea apa aducatoare de blestem spre vatamarea ei.
Num 5:25  Dupa aceea sa ia preotul din mâinile femeii darul de pâine cel pentru banuiala ?i sa ridice acest dar înaintea Domnului ?i sa-l duca la jertfelnic.
Num 5:26  Sa ia apoi preotul cu pumnul o parte din darul de amintire, s-o arda pe jertfelnic ?i dupa aceasta sa dea femeii sa bea apa.
Num 5:27  Dupa ce va bea apa cea amara a blestemului, daca ea va fi necurata ?i daca va fi în?elat pe barbatul sau, se va umfla pântecele ei ?i sânul ei va cadea ?i va fi femeia aceea blestemata în poporul sau.
Num 5:28  Iar daca femeia nu s-a spurcat, ci va fi curata, nevatamata va ramâne ?i va na?te copii.
Num 5:29  Aceasta este rânduiala pentru femeia banuita, care, fiind maritata, s-ar abate ?i s-ar spurca,
Num 5:30  Sau pentru omul, asupra caruia ar cadea duhul geloziei ?i ar banui pe femeia sa. Atunci sa puna el pe femeie înaintea fe?ei Domnului ?i sa faca preotul cu ea dupa legea aceasta.
Num 5:31  ?i va fi barbatul curat de pacat, iar femeia aceea î?i va purta pacatul ei".
Num 6:1  ?i a grait Domnul cu Moise ?i a zis:
Num 6:2  "Vorbe?te fiilor lui Israel ?i zi catre ei: Daca barbat sau femeie va hotarî sa dea fagaduin?a de nazireu, ca sa se afieroseasca nazireu Domnului,
Num 6:3  Sa se fereasca de vin ?i de sichera; o?et de vin ?i o?et de sichera sa nu bea ?i nimic din cele facute din struguri sa nu bea; nici struguri proaspe?i sau usca?i sa nu manânce.
Num 6:4  În toate zilele, cât va fi nazireu, sa nu manânce, nici sa bea vreo bautura facuta din struguri, de la sâmbure pâna la pielita.
Num 6:5  În toate zilele fagaduin?ei sale de nazireu sa nu treaca brici pe capul sau; pâna la împlinirea zilelor, câte a afierosit Domnului, este sfânt ?i trebuie sa creasca parul pe capul lui.
Num 6:6  În toate zilele, pentru care s-a afierosit pe sine sa fie nazireul Domnului, sa nu se apropie de trup mort:
Num 6:7  Când va muri tatal sau, sau mama sa, sau fratele sau, sau sora sa, sa nu se spurce prin atingerea de ei, pentru ca afierosirea lui Dumnezeu este pe capul lui.
Num 6:8  În toate zilele cât va fi nazireu, este sfântul Domnului.
Num 6:9  De va muri însa cineva lânga el fara de veste ?i de naprasna, ?i prin aceasta î?i va întina capul sau de nazireu, sa-?i tunda capul sau în ziua cura?irii sale;
Num 6:10  În ziua a ?aptea sa se tunda, iar în ziua a opta sa aduca preotului doua turturele sau doi pui de porumbel, la u?a cortului adunarii,
Num 6:11  ?i preotul sa aduca o pasare jertfa pentru pacat, iar pe cealalta ardere de tot, ?i sa-l cure?e de spurcarea cea prin atingerea de trupul mort ?i sa-i sfin?easca în ziua aceea capul lui.
Num 6:12  Apoi sa-?i înceapa din nou zilele sale de nazireu, afierosite Domnului, ?i sa aduca un berbec de un an jertfa de iertare, iar zilele dinainte sunt pierdute, pentru ca nazireatul a fost întinat.
Num 6:13  Iata legea cea pentru nazireu: când se vor împlini zilele lui de nazireu, sa se aduca la u?a cortului adunarii;
Num 6:14  Sa aduca darul sau Domnului: un miel de un an, fara meteahna, ardere de tot; o mioara de un an, fara meteahna, jertfa pentru pacat, ?i un berbec de un an, fara meteahna, jertfa de împacare,
Num 6:15  ?i un paner cu azime de faina de grâu, framântate cu untdelemn, ?i cu turte nedospite, unse cu untdelemn, cu darul lor de pâine ?i cu turnarea lor.
Num 6:16  Pe acestea le va înfa?i?a preotul înaintea Domnului, va savâr?i jertfa lui pentru pacat ?i arderea de lot a lui.
Num 6:17  Berbecul îl va aduce Domnului jertfa de împacare cu panerul cel de azime; ?i va aduce preotul prinosul lui de pâine ?i turnarea lui.
Num 6:18  ?i î?i va tunde nazireul la intrarea cortului adunarii capul sau de nazireu ?i va lua parul capului sau de nazireu ?i-l va pune pe focul cel de sub jertfa de împacare.
Num 6:19  Apoi va lua preotul ?oldul cel fiert al berbecului, o pâine nedospita ?i o turta nedospita din paner ?i le va pune nazireului pe mâini, dupa ce acesta ?i-a tuns capul de nazireu,
Num 6:20  ?i sa înal?e preotul acestea, leganându-le înaintea Domnului. Aceasta sfin?enie sa fie a preotului pe lânga pieptul leganat ?i pe lânga ?oldul înal?at. Dupa aceasta nazireul poate sa bea vin.
Num 6:21  Iata rânduiala cea pentru nazireul care a dat fagaduin?a ?i jertfa ce trebuie sa aduca el Domnului pentru nazireatul sau, pe lânga ceea ce-i îngaduiesc mijloacele lui. Dupa fagaduin?a sa, pe care o va da, a?a sa faca, dupa cele legiuite pentru nazireatul sau".
Num 6:22  ?i a grait Domnul cu Moise ?i a zis:
Num 6:23  "Spune lui Aaron ?i fiilor lui ?i le zi: A?a sa binecuvânta?i pe fiii lui Israel ?i sa zice?i catre ei:
Num 6:24  Sa te binecuvânteze Domnul ?i sa te pazeasca!
Num 6:25  Sa caute Domnul asupra ta cu fa?a vesela ?i sa te miluiasca!
Num 6:26  Sa-?i întoarca Domnul fala Sa catre tine ?i sa-li daruiasca pace!
Num 6:27  A?a sa cheme numele Meu asupra fiilor lui Israel ?i Eu, Domnul, îi voi binecuvânta".
Num 7:1  Când a a?ezat Moise cortul ?i l-a miruit ?i l-a sfin?it pe el ?i toate lucrurile lui, jertfelnicul ?i toate obiectele lui, ?i le-a miruit ?i le-a sfin?it,
Num 7:2  Atunci au venit cele douasprezece capetenii ale lui Israel, capii familiilor lor, mai-marii semin?iilor, care supravegheasera numaratoarea,
Num 7:3  ?i au adus Domnului darurile lor, ?ase care acoperite ?i doisprezece boi, câte un car de fiecare doua capetenii ?i câte un bou de fiecare capetenie ?i le-au adus înaintea cortului.
Num 7:4  A grait Domnul lui Moise zicând:
Num 7:5  "Prime?te-le de la ei, ca sa fie pentru facerea lucrarilor trebuitoare la cortul adunarii ?i le da levi?ilor, potrivit cu felul slujbei fiecaruia".
Num 7:6  ?i Moise, luând carele ?i boii, le-a dat levi?ilor:
Num 7:7  Doua care ?i patru boi a dat fiilor lui Gher?on, dupa slujba lor;
Num 7:8  Patru care ?i opt boi a dat fiilor lui Merari, dupa slujba lor, sub pova?a lui Itamar, fiul lui Aaron, preotul.
Num 7:9  Iar fiilor lui Cahat nu le-a dat, pentru ca slujba lor era de a duce lucrurile sfinte, pe care trebuia sa le poarte pe umeri.
Num 7:10  Au mai adus capeteniile jertfe pentru sfin?irea jertfelnicului, în ziua miruirii lui, ?i au înfa?i?at capeteniile prinoasele lor înaintea jertfelnicului.
Num 7:11  Atunci a zis Domnul catre Moise: "Câte o capetenie pe fiecare zi sa aduca prinosul sau pentru sfin?irea jertfelnicului".
Num 7:12  În ziua întâi a adus darul sau Naason, fiul lui Aminadab, capetenia semin?iei lui Iuda.
Num 7:13  ?i darul lui a fost: un blid de argint în greutate de o suta treizeci de sicli ?i o cupa de argint de ?aptezeci de sicli, dupa siclul sfânt, amândoua pline cu faina de grâu, amestecata cu untdelemn, pentru jertfa;
Num 7:14  O cadelni?a de aur de zece sicli, plina cu miresme;
Num 7:15  Un vi?el, un berbec ?i un miel de un an pentru ardere de tot;
Num 7:16  Un ?ap, jertfa pentru pacat;
Num 7:17  Iar ca jertfa de împacare: doi boi, cinci berbeci, cinci ?api, cinci miei de un an. Acestea au fost darurile lui Naason, fiul lui Aminadab.
Num 7:18  În ziua a doua a adus Natanael, fiul lui ?uar, capetenia semin?iei lui Isahar.
Num 7:19  Acesta a adus dar din partea sa: un blid de argint în greutate de o suta treizeci de sicli ?i o cupa de argint de ?aptezeci de sicli, dupa siclul sfânt, amândoua pline cu faina de grâu, amestecata cu untdelemn, pentru jertfa;
Num 7:20  O cadelni?a de aur de zece sicli, plina cu miresme;
Num 7:21  Un vi?el, un berbec ?i un miel de un an pentru ardere de tot;
Num 7:22  Un ?ap, jertfa pentru pacat;
Num 7:23  Iar ca jertfa de împacare: doi boi, cinci berbeci, cinci ?api ?i cinci miei de un an. Acestea au fost darurile lui Natanael, fiul lui ?uar.
Num 7:24  În ziua a treia a adus capetenia fiilor lui Zabulon, Eliab, fiul lui Helon.
Num 7:25  Darurile lui au fost: un blid de argint în greutate de o suta treizeci  de sicli ?i o cupa de argint de ?aptezeci de sicli, dupa siclul sfânt, amândoua pline cu faina de grâu, amestecata cu untdelemn, pentru jertfa;
Num 7:26  O cadelni?a de aur de zece sicli, plina cu miresme;
Num 7:27  Un vi?el, un berbec ?i un miel de un an ardere de tot;
Num 7:28  Un ?ap, jertfa pentru pacat;
Num 7:29  Iar pentru jertfa de împacare: doi boi, cinci berbeci, cinci ?api ?i cinci miei de un an. Acestea sunt darurile lui Eliab, fiul lui Helon.
Num 7:30  În ziua a patra a adus capetenia fiilor lui Ruben, Eli?ur, fiul lui ?edeur.
Num 7:31  Darurile lui au fost: un blid de argint în greutate de o suta treizeci de sicli ?i o cupa de argint de ?aptezeci de sicli, dupa siclul sfânt, amândoua pline cu faina de grâu, amestecata cu untdelemn, pentru jertfa;
Num 7:32  O cadelni?a de aur de zece sicli, plina cu miresme;
Num 7:33  Un vi?el, un berbec ?i un miel de un an, pentru ardere de tot;
Num 7:34  Un ?ap, jertfa pentru pacat;
Num 7:35  Iar pentru jertfa de împacare: doi boi, cinci berbeci, cinci ?api ?i cinci miei de un an. Acestea sunt darurile lui Eli?ur, fiul lui ?edeur.
Num 7:36  În ziua a cincea a adus capetenia fiilor lui Simeon, ?elumiel, fiul lui ?uri?adai.
Num 7:37  Darurile lui au fost: un blid de argint în greutate de o suta treizeci de sicli ?i o cupa de argint de ?aptezeci de sicli, dupa siclul sfânt, amândoua pline cu faina de grâu, amestecata cu untdelemn, pentru jertfa;
Num 7:38  O cadelni?a de aur de zece sicli, plina cu miresme;
Num 7:39  Un vi?el, un berbec ?i un miel de un an, pentru ardere de tot;
Num 7:40  Un ?ap, jertfa pentru pacat;
Num 7:41  Iar pentru jertfa de împacare: doi boi, cinci berbeci, cinci ?api ?i cinci miei de un an. Acestea sunt darurile lui ?elumiel, fiul lui ?uri?adai.
Num 7:42  În ziua a ?asea a adus capetenia fiilor lui Gad, Eliasaf, fiul lui Raguel.
Num 7:43  Darurile lui au fost: un blid de argint în greutate de o suta treizeci de sicli ?i o cupa de argint de ?aptezeci sicli, dupa siclul sfânt, amândoua pline cu faina de grâu, amestecata cu untdelemn, pentru jertfa;
Num 7:44  O cadelni?a de aur de zece sicli, plina cu miresme;
Num 7:45  Un vi?el, un berbec ?i un miel de un an pentru ardere de tot;
Num 7:46  Un ?ap, jertfa pentru pacat;
Num 7:47  Iar pentru jertfa de împacare: doi boi, cinci berbeci, cinci ?api ?i cinci miei de un an. Acestea sunt darurile lui Eliasaf, fiul lui Raguel.
Num 7:48  În ziua a ?aptea a adus capetenia fiilor lui Efraim, Eli?ama, fiul lui Amihud.
Num 7:49  Darurile lui au fost: un blid de argint în greutate de o suta treizeci de sicli ?i o cupa de argint de ?aptezeci de sicli, dupa siclul sfânt, amândoua pline cu faina de grâu, amestecata cu untdelemn, pentru jertfa;
Num 7:50  O cadelni?a de aur de zece sicli, plina cu miresme;
Num 7:51  Un vi?el, un berbec ?i un miel de un an pentru ardere de tot;
Num 7:52  Un ?ap, jertfa pentru pacat;
Num 7:53  Iar pentru jertfa de împacare: doi boi, cinci berbeci, cinci ?api ?i cinci miei de un an. Acestea sunt darurile lui Eli?ama, fiul lui Amihud.
Num 7:54  În ziua a opta a adus capetenia fiilor lui Manase, Gamaliel, fiul lui Peda?ur.
Num 7:55  Darurile lui au fost: un blid de argint în greutate de o suta treizeci de sicli ?i o cupa de argint de ?aptezeci sicli, dupa siclul sfânt, amândoua pline cu faina de grâu, amestecata cu untdelemn, pentru jertfa;
Num 7:56  O cadelni?a de aur de zece sicli, plina cu miresme;
Num 7:57  Un vi?el, un berbec ?i un miel de un an pentru ardere de tot;
Num 7:58  Un ?ap, jertfa pentru pacat;
Num 7:59  Iar pentru jertfa de împacare: doi boi, cinci berbeci, cinci ?api ?i cinci miei de un an. Acestea sunt darurile lui Gamaliel, fiul lui Peda?ur.
Num 7:60  În ziua a noua a adus capetenia fiilor lui Veniamin, Abidan, fiul lui Ghedeon.
Num 7:61  Darurile lui au fost: un blid de argint în greutate de o suta treizeci sicli ?i o cupa de argint de ?aptezeci de sicli, dupa siclul sfânt, amândoua pline cu faina, amestecata cu untdelemn, pentru jertfa;
Num 7:62  O cadelni?a de aur de zece sicli, plina cu miresme;
Num 7:63  Un vi?el, un berbec ?i un miel de un an, pentru ardere de tot;
Num 7:64  Un jap, jertfa pentru pacat;
Num 7:65  Iar pentru jertfa de împacare: doi boi, cinci berbeci, cinci ?api ?i cinci miei de un an. Acestea sunt darurile lui Abidan, fiul lui Ghedeon.
Num 7:66  În ziua a zecea a adus capetenia fiilor lui Dan, Ahiezer, fiul lui Ami?adai.
Num 7:67  Darurile lui au fost: un blid de argint în greutate de o suta treizeci sicli ?i o cupa de argint de ?aptezeci sicli, dupa siclul sfânt, amândoua pline cu faina de grâu, amestecata cu untdelemn, pentru jertfa;
Num 7:68  O cadelni?a de aur de zece sicli, plina cu miresme;
Num 7:69  Un vi?el, un berbec ?i un miel de un an, pentru ardere de tot;
Num 7:70  Un jap, jertfa pentru pacat;
Num 7:71  Iar pentru jertfa de împacare: doi boi, cinci berbeci, cinci ?api ?i cinci miei de un an. Acestea sunt darurile lui Ahiezer, fiul lui Ami?adai.
Num 7:72  În ziua a unsprezecea a adus capetenia fiilor lui A?er, Paghiel, fiul lui Ocran.
Num 7:73  Darurile lui au fost: un blid de argint în greutate de o suta treizeci de sicli ?i o cupa de argint de ?aptezeci de sicli, dupa siclul sfânt, amândoua pline cu faina de grâu, amestecata cu untdelemn, pentru jertfa;
Num 7:74  O cadelni?a de aur de zece sicli, plina cu miresme;
Num 7:75  Un vi?el, un berbec ?i un miel de un an pentru arderea de tot;
Num 7:76  Un ?ap, jertfa pentru pacat;
Num 7:77  Iar pentru jertfa de împacare: doi boi, cinci berbeci, cinci ?api ?i cinci miei de un an. Acestea sunt darurile lui Paghiel, fiul lui Ocran.
Num 7:78  În ziua a douasprezecea a adus capetenia fiilor lui Neftali, Ahira, fiul lui Enan.
Num 7:79  Darurile lui au fost: un blid de argint în greutate de o suta treizeci de sicli ?i o cupa de argint de ?aptezeci de sicli, dupa siclul sfânt, amândoua pline cu faina de grâu, amestecata cu untdelemn, pentru jertfa;
Num 7:80  O cadelni?a de aur de zece sicli, plina cu miresme;
Num 7:81  Un vi?el, un berbec ?i un miel de un an pentru ardere de tot;
Num 7:82  Un ?ap, jertfa pentru pacat;
Num 7:83  Iar pentru jertfa de împacare: doi boi, cinci berbeci, cinci ?api ?i cinci miei de un an. Acestea sunt darurile lui Ahira, fiul lui Enan.
Num 7:84  Acestea au fost darurile din partea capeteniilor lui Israel, aduse la sfin?irea jertfelnicului, în ziua miruirii lui: douasprezece blide de argint, douasprezece cupe de argint, douasprezece cadelni?e de aur,
Num 7:85  Având fiecare blid o suta treizeci sicli de argint ?i fiecare cupa câte ?aptezeci de sicli; deci argintul tot în aceste vase a fost doua mii patru sute sicli, dupa siclul sfânt;
Num 7:86  Douasprezece cadelni?e de aur, pline cu miresme, de câte zece sicli fiecare, dupa siclul sfânt; deci tot aurul cadelni?elor a fost o suta douazeci de sicli;
Num 7:87  Pentru arderi de tot au fost: doisprezece vi?ei din vitele mari, doisprezece berbeci ?i doisprezece miei de un an ?i împreuna cu ei prinosul de pâine ?i turnarea lor; doisprezece ?api, jertfa pentru pacat;
Num 7:88  Iar pentru jertfa de împacare au fost: douazeci ?i patru de boi, ?aizeci de berbeci, ?aizeci de ?api ?i ?aizeci de miei de un an, fara meteahna. Acestea au fost darurile la sfin?irea jertfelnicului, dupa miruirea lui.
Num 7:89  Când a intrat Moise în cortul adunarii, ca sa graiasca cu Domnul, a auzit un glas, care-i graia de sus de pe chivotul legii, dintre cei doi heruvimi. Glasul acela graia cu el.
Num 8:1  Atunci a grait Domnul cu Moise ?i a zis:
Num 8:2  "Vorbe?te cu Aaron ?i-i spune: Când vei pune candelele în sfe?nic, ca sa lumineze partea cea dinaintea lui, sa aprinzi în el ?apte candele".
Num 8:3  ?i a facut Aaron a?a: a aprins în sfe?nic, ca sa lumineze partea cea din fa?a lui, ?apte candele, cum poruncise Domnul lui Moise.
Num 8:4  Iata cum era facut sfe?nicul: fusul lui de aur era lucrat din ciocan; florile lui toate erau tot din ciocan. Dupa modelul pe care îl aratase Domnul lui Moise, a?a s-a facut sfe?nicul.
Num 8:5  ?i a grait cu Moise Domnul ?i i-a zis:
Num 8:6  "Ia pe levi?i din mijlocul fiilor lui Israel ?i-i cura?a;
Num 8:7  ?i ca sa-i cure?i, sa faci cu ei a?a: sa-i strope?ti cu apa cura?irii, sa-?i rada cu briciul tot trupul lor, sa-?i spele hainele ?i vor fi cura?i.
Num 8:8  Apoi ei sa ia un vi?el ?i prinosul de pâine, faina de grâu, amestecata cu untdelemn; iar tu sa mai iei un vi?el, jertfa pentru pacat.
Num 8:9  Adu dupa aceea pe levi?i înaintea cortului adunarii, unde vei aduna toata ob?tea fiilor lui Israel.
Num 8:10  Sa se apropie levi?ii înaintea Domnului ?i fiii lui Israel sa-?i puna mâinile pe levi?i;
Num 8:11  Iar Aaron sa afieroseasca pe levi?i înaintea Domnului, din partea fiilor lui Israel, ca sa faca ei slujba Domnului.
Num 8:12  Apoi levi?ii sa-?i puna mâinile pe capetele vi?eilor ?i tu sa aduci unul jertfa pentru pacat, iar pe celalalt ardere de tot Domnului pentru cura?irea levi?ilor.
Num 8:13  Pune apoi pe levi?i înaintea Domnului ?i înaintea lui Aaron ?i înaintea fiilor lui ?i-i adu dar Domnului.
Num 8:14  A?a vei osebi pe levi?i de fiii lui Israel, ca vor fi ai Mei.
Num 8:15  Dupa aceea vor merge levi?ii sa slujeasca la cortul adunarii, dupa ce îi vei cura?i ?i îi vei afierosi Domnului;
Num 8:16  Caci Îmi sunt da?i Mie dintre fiii lui Israel în locul tuturor celor întâi-nascu?i, care deschide orice pântece;
Num 8:17  Caci al Meu  este tot întâi-nascutul lui Israel, de la om pâna la dobitoc, pentru ca Mi i-am sfin?it Mie în ziua când am lovit în pamântul Egiptului pe to?i întâi-nascu?ii;
Num 8:18  ?i în locul tuturor întâi-nascu?ilor fiilor lui Israel am luat pe levi?i;
Num 8:19  ?i i-am dat pe levi?i dar lui Aaron ?i fiilor lui dintre fiii lui Israel, ca sa slujeasca pentru fiii lui Israel, la cortul adunarii ?i sa se roage pentru fiii lui Israel, ca sa nu-i ajunga pe fiii lui Israel vreo urgie, când s-ar apropia de loca?ul sfânt".
Num 8:20  Moise ?i Aaron ?i toata ob?tea fiilor lui Israel au facut cu levi?ii cum poruncise Domnul lui Moise pentru levi?i; a?a au facut cu ei fiii lui Israel.
Num 8:21  S-au cura?it deci levi?ii ?i ?i-au spalat hainele, iar Aaron a savâr?it sfin?irea lor înaintea Domnului ?i s-a rugat pentru ei, ca sa fie cura?i.
Num 8:22  Dupa aceea au intrat levi?ii sa-?i faca slujbele lor la cortul adunarii, înaintea lui Aaron ?i înaintea fiilor lui. Cum poruncise Domnul lui Moise pentru levi?i, a?a au facut cu ei.
Num 8:23  ?i a grait Domnul cu Moise ?i a zis:
Num 8:24  "Aceasta este legea levi?ilor: de la douazeci ?i cinci de ani în sus sa intre sa lucreze la cortul adunarii;
Num 8:25  Iar la cincizeci de ani sa înceteze ?i sa nu mai lucreze.
Num 8:26  De acolo înainte sa ajute fra?ilor lor a strajui la cortul adunarii, dar de lucrat sa nu mai lucreze. A?a sa faci cu levi?ii, ca fiecare sa fie la slujba lui de paznic".
Num 9:1  În vremea aceea a grait Domnul cu Moise în pustiul Sinai, în anul al doilea dupa ie?irea din Egipt, în luna întâi, ?i a zis:
Num 9:2  "Spune fiilor lui Israel sa faca Pa?tile la vremea rânduita pentru ele:
Num 9:3  În ziua de paisprezece a lunii întâi, spre seara, sa le faca la vremea lor, dupa legea lor ?i dupa regulile lor sa le savâr?i?i".
Num 9:4  ?i a spus Moise fiilor lui Israel sa faca Pa?tile:
Num 9:5  ?i au facut ei Pa?tile în luna întâi, în ziua a paisprezecea, spre seara, în pustiul Sinai; cum poruncise Domnul lui Moise a?a au facut fiii lui Israel.
Num 9:6  Dar erau ?i oameni necura?i, care se atinsesera de trup de om mort, ?i nu puteau sa savâr?easca Pa?tile în ziua aceea. Ace?tia au venit în ziua aceea la Moise ?i Aaron,
Num 9:7  ?i le-au spus oamenii aceia: "Noi suntem necura?i, pentru ca ne-am atins de trup de om mort; do ce sa nu fim lasa?i sa aducem Domnului dar la vremea cea rânduita pentru fiii lui Israel?"
Num 9:8  Iar Moise a zis catre ei: "Sta?i aici, ca am sa ascult ce porunce?te Domnul pentru voi!"
Num 9:9  A grait Domnul lui Moise ?i a zis:
Num 9:10  "Spune fiilor lui Israel: Daca cineva din voi sau din urma?ii vo?tri va fi necurat prin atingere de trup de om mort, sau va fi departe în calatorie, sau între neamuri straine, ?i acela sa faca Pa?tile Domnului.
Num 9:11  Dar sa le faca în ziua a paisprezecea a lunii a doua, seara, ?i sa le manânce cu azime ?i cu ierburi amare;
Num 9:12  Sa nu lase din ele pe a doua zi, nici oasele sa nu le zdrobeasca; ?i sa le savâr?easca dupa toata rânduiala Pa?tilor.
Num 9:13  Iar omul curat, care nu se afla departe în calatorie ?i nu va face Pa?tile, sufletul acela sa se stârpeasca din poporul sau, ca n-a adus dar Domnului la vreme. Omul acela î?i va purta pacatul sau.
Num 9:14  De va trai la voi vreun strain sa faca ?i el Pa?tile Domnului: dupa legea Pa?tilor ?l dupa rânduiala lor sa le faca. O singura lege sa fie ?i pentru voi ?i pentru strain".
Num 9:15  În ziua când a fost a?ezat cortul, nor a acoperit cortul adunarii, ?i de seara pâna diminea?a a fost deasupra cortului, ca o vedere de foc.
Num 9:16  A?a era totdeauna: ziua îl acoperea un nor ?i noaptea o vedere de foc.
Num 9:17  Când se ridica norul de deasupra cortului, atunci fiii lui Israel plecau ?i în locul unde se oprea norul, acolo poposeau cu tabara fiii lui Israel.
Num 9:18  Dupa porunca Domnului se opreau fiii lui Israel cu tabara lor ?i dupa porunca Domnului plecau; tot timpul cât norul statea deasupra cortului, stateau ?i ei cu tabara.
Num 9:19  Când însa norul statea multa vreme deasupra cortului, urmau acestui semn al Domnului ?i fiii lui Israel ?i nu plecau.
Num 9:20  Câteodata se întâmpla ca norul sa stea numai pu?ina vreme deasupra cortului: dupa glasul Domnului se opreau ?i dupa porunca Lui plecau la drum.
Num 9:21  Câteodata norul statea numai de seara pâna diminea?a, iar diminea?a se ridica norul; atunci plecau ?i ei; sau statea norul o zi ?i o noapte, ?i când se ridica, plecau ?i ei;
Num 9:22  Sau de umbrea norul deasupra cortului doua zile, sau o luna, sau un an, fiii lui Israel stateau ?i nu plecau la drum; iar când se ridica el, atunci plecau,
Num 9:23  Ca din porunca Domnului se opreau ?i din porunca Domnului plecau la drum: urmau semnul Domnului, dupa porunca data de Domnul prin Moise.
Num 10:1  ?i a grait Domnul cu Moise ?i a zis:
Num 10:2  "Fa-?i doua trâmbi?e de argint; din ciocan sa le faci, ca sa fie pentru chemarea ob?tii ?i pentru plecarea taberei.
Num 10:3  De se va trâmbi?a din ele, se va aduna toata ob?tea la u?a cortului adunarii.
Num 10:4  De se va trâmbi?a numai din una, se vor aduna la tine toate capeteniile cele mai mari ale lui Israel.
Num 10:5  Când ve?i înso?i sunetele cu strigate, se vor ridica taberele cele dinspre rasarit.
Num 10:6  Când ve?i înso?i a doua oara sunetele cu strigate, se vor ridica taberele cele dinspre miazazi. Când ve?i înso?i a treia oara sunetele cu strigate, se vor ridica taberele cele dinspre mare. Când ve?i înso?i a patra oara sunetele cu strigate, se vor ridica taberele cele dinspre miazanoapte. Sa înso?i?i sunetele cu strigate numai pentru plecare.
Num 10:7  Iar când chema?i adunarea, sa suna?i, dar sa nu înso?i?i sunetele cu strigate.
Num 10:8  Din trâmbi?e vor suna preo?ii, fiii lui Aaron: aceasta-i pentru voi lege ve?nica din neam în neam.
Num 10:9  Când ve?i merge la razboi, în pamântul vostru, împotriva vrajma?ilor care navalesc asupra voastra, înso?i?i sunetele de trâmbi?a cu strigate ?i ve?i fi pomeni?i înaintea Domnului Dumnezeului vostru ?i ve?i fi izbavi?i de vrajma?ii vo?tri.
Num 10:10  În ziua voastra de bucurie, la sarbatorile voastre ?i la lunile noi ale voastre, sa trâmbi?a?i din trâmbi?e la arderile de tot ale voastre ?i la jertfele voastre de împacare ?i prin aceasta ve?i fi pomeni?i înaintea Dumnezeului vostru. Eu sunt Domnul Dumnezeul vostru".
Num 10:11  În anul al doilea, în luna a doua, în douazeci ale lunii, s-a ridicat norul de deasupra cortului adunarii;
Num 10:12  ?i au plecat fiii lui Israel din pustiul Sinai dupa taberele lor ?i s-a oprit norul în pustiul Paran.
Num 10:13  Aceasta a fost întâia plecare, dupa porunca lui Dumnezeu, data prin Moise.
Num 10:14  Întâi s-a ridicat steagul taberei fiilor lui Iuda cu cetele lor ?i peste cetele lor era Naason, fiul lui Aminadab.
Num 10:15  Peste cetele semin?iei lui Isahar era Natanael, fiul lui ?uar;
Num 10:16  Iar peste cetele semin?iei fiilor lui Zabulon era Eliab, fiul lui Helon.
Num 10:17  Apoi s-a ridicat cortul ?i au plecat fiii lui Gher?on ?i fiii lui Merari, care duceau cortul.
Num 10:18  Dupa aceea s-a ridicat steagul taberei lui Ruben cu cetele sale; peste cetele lui era Eli?ur, fiul lui ?edeur;
Num 10:19  Peste cetele semin?iei lui Simeon era ?elumiel, fiul lui ?uri?adai;
Num 10:20  Iar peste cetele semin?iei fiilor lui Gad era Eliasaf, fiul lui Raguel.
Num 10:21  Dupa aceea au plecat fiii lui Cahat, care duceau lucrurile sfinte, caci cortul trebuia sa fie a?ezat înainte de sosirea lor.
Num 10:22  Apoi s-a ridicat steagul taberei fiilor lui Efraim cu cetele lor; peste cetele lui era Eli?ama, fiul lui Amihud.
Num 10:23  Peste cetele fiilor semin?iei lui Manase era Gamaliel, fiul lui Peda?ur;
Num 10:24  Iar peste cetele fiilor semin?iei lui Veniamin era Abidan, fiul lui Ghedeon.
Num 10:25  Dupa toate taberele, cel din urma a fost ridicat steagul taberei fiilor lui Dan cu cetele sale; peste cetele lui era Ahiezer, fiul lui Ami?adai;
Num 10:26  Peste cetele semin?iei fiilor lui A?er era Paghiel, fiul lui Ocran;
Num 10:27  Iar peste cetele semin?iei fiilor lui Neftali era Ahira, fiul lui Enan.
Num 10:28  Aceasta era rânduiala în care mergeau fiii lui Israel cu taberele lor. ?i a?a au plecat.
Num 10:29  Atunci a zis Moise catre Hobab fiul lui Raguel, madianitul, socrul lui Moise: "Noi plecam la locul acela, de care a zis Domnul: Voua vi-l voi da. Hai cu noi ?i-?i vom face bine, caci Domnul a grait bine de Israel".
Num 10:30  Acela însa a zis catre el: "Nu merg, ci ma duc în ?ara mea ?i la neamul meu".
Num 10:31  Dar Moise a zis: "Nu ne parasi, pentru ca tu ?tii cum ne a?ezam noi taberele în pustie ?i vei fi ochiul nostru.
Num 10:32  Daca mergi cu noi, binele ce ni-l va face Domnul, îl vom face ?i noi ?ie".
Num 10:33  Plecând ei de la muntele Domnului, au mers trei zile; iar chivotul legii Domnului a mers înaintea lor cale de trei zile, ca sa aleaga pentru ei loc de odihna.
Num 10:34  Norul Domnului îi umbrea ziua, când plecau de la popas.
Num 10:35  Când se ridica chivotul, ca sa plece la drum, Moise zicea: "Scoala, Doamne, ?i sa se risipeasca vrajma?ii Tai ?i sa fuga de la fa?a Ta cei ce Te urasc pe Tine!"
Num 10:36  Iar când se oprea chivotul, el zicea: "Întoarce-Te, Doamne, la miile ?i zecile de mii ale lui Israel!"
Num 11:1  Poporul însa începu sa cârteasca în auzul Domnului, iar Domnul auzind, se aprinse mânia Lui, izbucni între ei foc de la Domnul ?i începu a mistui marginile taberei.
Num 11:2  Atunci a strigat poporul catre Moise, iar Moise s-a rugat Domnului ?i a încetat focul.
Num 11:3  De aceea s-a numit locul acela: Tabeera, adica ardere, caci acolo a fost aprins între ei focul de la Domnul.
Num 11:4  Strainii, dintre ei, începura sa-?i arate poftele ?i ?edeau cu ei ?i fiii lui Israel ?i plângeau, zicând: "Cine ne va hrani cu carne?
Num 11:5  Caci ne aducem aminte de pe?tele, pe care-l mâncam în Egipt în dar, de castrave?i ?i de pepeni, de ceapa, de praz ?i de usturoi;
Num 11:6  Acum însa sufletul nostru tânje?te; nimic nu mai este înaintea ochilor no?tri decât numai mana".
Num 11:7  Iar mana era ca samân?a de coriandru ?i înfa?i?area ei era ca înfa?i?area cristalului.
Num 11:8  Poporul se ducea ?i o aduna, o râ?neau în râ?ni?e sau o pisau în piua, o fierbeau în caldari ?i faceau din ea turte; iar gustul ei era ca gustul turtelor cu untdelemn.
Num 11:9  Când cadea noaptea roua pe tabara, atunci cadea peste ea ?i mana.
Num 11:10  Moise însa auzea cum plângea fiecare prin familiile sale ?i la u?a cortului sau, ?i s-a aprins tare mânia Domnului ?i s-a mâhnit Moise.
Num 11:11  Atunci a zis Moise catre Domnul: "De ce întristezi pe robul Tau ?i de ce oare n-am aflat mila înaintea ochilor Tai, caci ai pus asupra mea sarcina a tot poporul acesta?
Num 11:12  Oare eu am zamislit tot poporul acesta ?i oare eu l-am nascut, de-mi zici: Ia-l în sânul tau, cum ia doica pe copil, ?i-l du în pamântul pe care cu juramânt l-am fagaduit parin?ilor lui?
Num 11:13  De unde sa iau ?i sa dau eu carne la tot poporul acesta? Caci plâng înaintea mea ?i. zic: Da-ne carne sa mâncam!
Num 11:14  Eu singur nu voi putea sa duc tot poporul acesta, ca acest lucru este greu pentru mine.
Num 11:15  Daca faci a?a cu mine, atunci mai bine omoara-ma, de am aflat mila înaintea ochilor Tai, ca sa nu mai vad necazul acesta".
Num 11:16  Atunci Domnul a zis catre Moise: "Aduna-Mi ?aptezeci de barba?i, dintre batrânii lui Israel, pe care-i ?tii tu ca sunt capetenii poporului ?i supraveghetorii lui, ?i du-i la cortul adunarii, ca sa stea cu tine acolo.
Num 11:17  Ca Ma voi pogorî acolo ?i voi vorbi cu tine ?i voi lua din duhul care este peste tine ?i voi pune peste ei ca sa duca ei cu tine sarcina poporului ?i sa nu o duci numai tu singur.
Num 11:18  Iar poporului spune-i: Sa va cura?i?i pentru ziua de mâine ?i ve?i avea carne, deoarece a?i plâns în auzul Domnului ?i a?i zis: Cine ne va hrani cu carne? Ca ne era bine în Egipt; ca are sa va dea Domnul carne sa mânca?i ?i ve?i mânca;
Num 11:19  ?i ve?i mânca nu numai o zi, nici numai doua sau cinci zile, nici numai zece sau douazeci de zile,
Num 11:20  Ci o luna întreaga ve?i mânca pâna va va da pe nas ?i va ve?i scârbi de ea, pentru ca a?i dispre?uit pe Domnul, Care este între voi, ?i v-a?i plâns înaintea Lui ?i a?i zis: La ce trebuia sa ie?im noi din Egipt?"
Num 11:21  Apoi a zis Moise: "În poporul acesta, în care ma aflu, sunt ?ase sute de mii de pede?tri ?i Tu zici: Am sa le dau carne sa manânce ?i vor mânca o luna de zile!
Num 11:22  Li se vor taia, oare, toate oile ?i to?i boii, ca sa le ajunga? Sau tot pe?tele marii li se va aduna, ca sa-i îndestuleze?"
Num 11:23  Zis-a Domnul catre Moise: "Dar, oare, mâna Domnului e scurta? Acum vei vedea de se va împlini sau nu cuvântul Meu".
Num 11:24  Atunci a ie?it Moise ?i a spus poporului cuvintele Domnului, a adunat ?aptezeci de barba?i dintre batrânii poporului ?i i-a pus împrejurul cortului.
Num 11:25  ?i S-a pogorât Domnul în nor ?i a vorbit cu el; ?i a luat din duhul care era peste el ?i a pus peste cei ?aptezeci de barba?i capetenii. Îndata însa cum a odihnit duhul peste ei, au început a prooroci, dar apoi au încetat.
Num 11:26  Doi dintre barba?i însa au ramas în tabara: pe unul îl chema Eldad ?i pe celalalt îl chema Medad. ?i a odihnit ?i peste ei duhul, caci erau din cei înscri?i, dar nu venisera la cort, ?i au proorocit ?i ei acolo în tabara.
Num 11:27  Atunci a alergat un tânar ?i a spus lui Moise, zicând: "Eldad ?i Medad proorocesc în tabara".
Num 11:28  ?i raspunzând, Iosua, fiul lui Navi, slujitorul lui Moise, unul din ale?ii lui, a zis: "Domnul meu Moise, opre?te-i!"
Num 11:29  Moise însa i-a zis: "Nu cumva e?ti gelos pe mine? O, de ar fi to?i prooroci în poporul Domnului ?i de ar trimite Domnul duhul Sau peste ei!"
Num 11:30  Apoi s-a întors Moise în tabara împreuna cu batrânii lui Israel.
Num 11:31  Atunci s-a stârnit vânt de la Domnul, a adus prepeli?e dinspre mare ?i le-a presarat împrejurul taberei cale de o zi într-o parte ?i cale de o zi în cealalta parte împrejurul taberei, strat gros aproape de doi co?i de la pamânt.
Num 11:32  Atunci s-a sculat poporul ?i toata ziua ?i toata noaptea aceea ?i toata ziua urmatoare au adunat prepeli?e; ?i cine a adunat pu?in, tot a adunat zece co?uri ?i le-au întins împrejurul taberei.
Num 11:33  Dar carnea era înca în gura lor ?i nu ispravisera înca de mâncat, când se aprinse mânia Domnului asupra poporului, ?i a lovit Domnul poporul cu bataie foarte mare.
Num 11:34  ?i s-a pus numele locului aceluia: Chibrot-Hataava, adica mormintele poftei, caci acolo au îngropat pe poporul cel aprins de pofta.
Num 11:35  Apoi a plecat poporul din Chibrot-Hataava la Ha?erot ?i s-a oprit în Ha?erot.
Num 12:1  Mariam ?i Aaron vorbeau însa împotriva lui Moise pentru femeia etiopianca, pe care o luase Moise, caci Moise era casatorit cu o etiopianca.
Num 12:2  Ei ziceau: "Oare numai cu Moise a grait Domnul? N-a grait El, oare, ?i cu noi?" ?i a auzit acestea Domnul.
Num 12:3  Moise însa era omul cel mai blând dintre to?i oamenii de pe pamânt.
Num 12:4  Atunci a zis Domnul fara de veste catre Moise ?i catre Aaron ?i catre Mariam: "Ie?i?i câte?i trei la cortul adunarii". ?i au ie?it tustrei.
Num 12:5  Atunci S-a pogorât Domnul în stâlpul cel de nor, a stat la u?a cortului ?i a chemat pe Aaron ?i pe Mariam ?i au ie?it amândoi.
Num 12:6  Apoi a zis: "Asculta?i cuvintele Mele: De este între voi vreun prooroc al Domnului, Ma arat lui în vedenie ?i în somn vorbesc cu el.
Num 12:7  Nu tot a?a am grait ?i cu robul Meu Moise, - el este credincios în toata casa Mea:
Num 12:8  Cu el graiesc gura catre gura, la aratare ?i aievea, iar nu în ghicituri, ?i el vede fa?a Domnului. Cum de nu v-a?i temut sa cârti?i împotriva robului Meu Moise?"
Num 12:9  ?i s-a aprins mânia Domnului asupra lor ?i, departându-Se Domnul,
Num 12:10  S-a departat ?i norul de la cort ?i iata Mariam s-a facut alba de lepra, ca zapada. ?i când s-a uitat Aaron la Mariam, iata era leproasa.
Num 12:11  Atunci a zis Aaron catre Moise: "Rogu-ma, domnul meu, sa nu ne socote?ti gre?eala ca ne-am purtat rau ?i am pacatuit!
Num 12:12  Nu îngadui dar sa fie Mariam ca cel nascut mort, al carui trup, la ie?irea din pântecele mamei sale, este pe jumatate putred".
Num 12:13  Atunci a strigat Moise catre Domnul ?i a zis: "Dumnezeule, vindec-o!"
Num 12:14  Domnul însa a zis catre Moise: "Daca tatal ei ar fi scuipat-o în obraz, oare n-ar fi trebuit sa se ru?ineze ?apte zile? A?a dar sa fie închisa ?apte zile afara din tabara, dupa aceea sa intre".
Num 12:15  ?i a ?ezut Mariam închisa afara din tabara ?apte zile ?i poporul n-a plecat la drum pâna s-a cura?it Mariam.
Num 12:16  Dupa aceasta a pornit poporul de la Ha?erot ?i a poposit în pustiul Paran.
Num 13:1  Acolo a grait Domnul cu Moise ?i i-a zis:
Num 13:2  "Trimite din partea ta oameni ca sa iscodeasca pamântul Canaanului pe care am sa-l dau Eu fiilor lui Israel spre mo?tenire; câte un om de fiecare semin?ie sa trimi?i; însa ace?tia sa fie to?i capetenii între ei".
Num 13:3  ?i i-a trimis pe ace?tia Moise din pustiul Paran, dupa porunca Domnului, ?i erau to?i capetenii.
Num 13:4  Iata acum ?i numele lor: ?ammua, fiul lui Zahur, din semin?ia lui Ruben.
Num 13:5  Safat, fiul lui Hori, din semin?ia lui Simeon;
Num 13:6  Caleb, fiul lui Iefone, din semin?ia lui Iuda;
Num 13:7  Igal, fiul lui Iosif, din semin?ia lui Isahar;
Num 13:8  Osia, fiul lui Navi, din semin?ia lui Efraim;
Num 13:9  Falti, fiul lui Rafu, din semin?ia lui Veniamin;
Num 13:10  Gadiel, fiul lui Sodi, din semin?ia lui Zabulon;
Num 13:11  Gadi, fiul lui Susi, din Manase, semin?ia lui Iosif;
Num 13:12  Amiel, fiul lui Ghemali, din semin?ia lui Dan;
Num 13:13  Setur, fiul lui Mihael, din semin?ia lui A?er;
Num 13:14  Nahbi, fiul lui Vofsi, din semin?ia lui Neftali;
Num 13:15  Gheuel, fiul lui Machi, din semin?ia lui Gad.
Num 13:16  Acestea sunt numele barba?ilor pe care i-a trimis Moise sa iscodeasca ?ara. Insa pe Osia, fiul lui Navi, Moise l-a numit Iosua.
Num 13:17  Trimi?ându-i pe ace?tia din pustiul Paran ca sa iscodeasca pamântul Canaanului, Moise le-a zis:
Num 13:18  "Sui?i-va din pustiul acesta ?i va urca?i pe munte ?i cerceta?i ce pamânt este ?i ce popor locuie?te în el; de este tare sau slab, mult la numar sau pu?in;
Num 13:19  Cum este ?ara pe care o locuie?te: buna sau rea, cum sunt ora?ele în care traie?te el: cu ziduri sau fara ziduri;
Num 13:20  Cum este pamântul: gras sau slab, de sunt pe el copaci sau nu. Fi?i curajo?i ?i lua?i din roadele pamântului aceluia". Aceasta se petrecea pe vremea coacerii strugurilor.
Num 13:21  ?i s-au dus ei ?i au cercetat pamântul de la pustiul ?in pâna la Rehob, care vine lânga Hamat.
Num 13:22  De acolo au trecut în partea de miazazi a Canaanului ?i au mers pâna la Hebron, unde traiau Ahiman, ?e?ai, ?i Talmai, copiii lui Enac. Hebronul fusese zidit cu ?apte ani înaintea ora?ului egiptean ?oan.
Num 13:23  Apoi au venit pâna în valea E?col, au cercetat-o ?i au taiat de acolo o vi?a de vie cu un strugure de poama ?i au dus-o doi pe pârghie. Au mai luat de asemenea rodii ?i smochine.
Num 13:24  Locul acesta l-au numit ei valea E?col, adica valea strugurelui, de la strugurele de poama pe care l-au taiat de acolo fiii lui Israel.
Num 13:25  ?i dupa ce au cercetat ei pamântul, s-au întors dupa patruzeci de zile
Num 13:26  ?i, mergând, au venit la Moise ?i la Aaron ?i la toata ob?tea fiilor lui Israel, la Cade?, în pustiul Paran, ?i le-au adus lor ?i întregii ob?ti vi?a ?i le-au aratat roadele pamântului aceluia.
Num 13:27  Apoi le-au povestit ?i au zis: "Am fost în pamântul în care ne-ai trimis, pamântul în care curge miere ?i lapte ?i iata roadele lui.
Num 13:28  Dar poporul care locuie?te în el, este îndrazne? ?i ora?ele sunt întarite ?i foarte mari, ba ?i pe fiii lui Enac i-am vazut acolo.
Num 13:29  Amalec locuie?te în partea de miazazi a ?arii; Heteii, Heveii, Iebuseii ?i Amoreii locuiesc în mun?i, iar Canaaneii locuiesc pe lânga mare ?i pe lânga râul Iordanului".
Num 13:30  Caleb însa a lini?tit poporul înaintea lui Moise, zicând: "Nu, ci sa mergem ?i sa-l cuprindem, pentru ca îl vom putea birui!"
Num 13:31  Iar oamenii cei ce fusesera cu el au zis: "Nu putem sa mergem împotriva poporului aceluia, pentru ca e mult mai puternic decât noi".
Num 13:32  ?i au împra?tiat printre fiii lui Israel zvonuri rele despre pamântul pe care-l cercetasera, zicând: "Pamântul pe care l-am strabatut noi, ca sa-l vedem, este un pamânt care manânca pe cei ce locuiesc în el ?i tot poporul, pe care l-am vazut acolo, sunt oameni foarte mari.
Num 13:33  Acolo am vazut noi ?i uria?i, pe fiii lui Enac, din neamul uria?ilor; ?i noua ni se parea ca suntem fa?a de ei ca ni?te lacuste ?i tot a?a le paream ?i noi lor".
Num 14:1  Atunci toata ob?tea a ridicat strigat ?i a plâns poporul toata noaptea aceea;
Num 14:2  Cârtind împotriva lui Moise ?i a lui Aaron, to?i fiii lui Israel ?i toata ob?tea au zis catre ei: "Mai bine era sa fi murit în pamântul Egiptului sau sa murim în pustiul acesta!
Num 14:3  La ce ne duce Domnul în pamântul acela, ca sa cadem în razboi? Femeile noastre ?i copiii no?tri vor fi prada. Nu ar fi, oare, mai bine sa ne întoarcem în Egipt?"
Num 14:4  Apoi au zis unii catre al?ii: "Sa ne alegem capetenie ?i sa ne întoarcem în Egipt".
Num 14:5  Atunci au cazut Moise ?i Aaron cu felele la pamânt înaintea întregii adunari a ob?tii fiilor lui Israel.
Num 14:6  Iar Iosua, fiul lui Navi ?i Caleb al lui Iefone, care erau din cei ce cercetasera ?ara, ?i-au rupt hainele lor
Num 14:7  ?i au zis catre ob?tea fiilor lui Israel: "Pamântul, pe care l-am strabatut noi, este foarte, foarte bun;
Num 14:8  De va fi Domnul bun cu noi, ne va duce în pamântul acela ?i ni-l va da noua; în pamântul acela izvora?te lapte ?i miere.
Num 14:9  Deci nu va ridica?i împotriva Domnului ?i nu va teme?i de poporul pamântului aceluia, caci va ajunge mâncarea noastra: ei n-au aparare, iar cu noi este Domnul. Nu va teme?i de ei!"
Num 14:10  Atunci toata ob?tea a zis: "Sa-i ucidem cu pietre!" Dar iata slava Domnului s-a aratat în nor  tuturor fiilor lui Israel la cortul adunarii.
Num 14:11  ?i a zis Domnul catre Moise: "Pâna când Ma va supara poporul acesta ?i pâna când nu va crede el în Mine, cu toate minunile ce am facut în mijlocul lui?
Num 14:12  Îl voi lovi cu ciuma ?i-l voi pierde ?i te voi face pe tine ?i casa tatalui tau popor numeros ?i mai puternic decât acesta!"
Num 14:13  Moise însa a zis catre Domnul: "Vor auzi de aceasta Egiptenii, din mijlocul carora ai scos Tu, cu puterea Ta, pe poporul acesta
Num 14:14  ?i vor spune locuitorilor pamântului acestuia, care au auzit, ca Tu, Doamne, Te afli în mijlocul poporului acestuia ?i Tu, Doamne, le dai sa Te vada fa?a catre fa?a, ?i ca Tu mergi înaintea lor, ziua în stâlp de nor ?i noaptea în stâlp de foc.
Num 14:15  Iar daca Tu vei pierde pe poporul acesta, ca pe un om, atunci popoarele care au auzit de numele Tau vor zice:
Num 14:16  Domnul n-a putut duce pe poporul acesta în pamântul pe care cu juramânt l-a fagaduit sa-l dea lor ?i de aceea l-a pierdut în pustie.
Num 14:17  Deci, înal?a-se acum puterea Ta, Doamne, cum ai spus Tu, zicând:
Num 14:18  Domnul este îndelung-rabdator, mult-îndurat ?i adevarat, iertând faradelegile, gre?elile ?i pacatele ?i nelasând nepedepsit, ci pedepse?te nelegiuirile parin?ilor în copii pâna la al treilea ?i al patrulea neam.
Num 14:19  Iarta pacatul poporului acestuia, dupa mare mila Ta, precum ai iertat Tu poporul acesta din Egipt ?i pâna acum".
Num 14:20  Zis-a Domnul catre Moise: "Voi ierta, dupa cuvântul tau,
Num 14:21  Dar viu sunt Eu ?i viu e numele Meu ?i de slava Domnului e plin tot pamântul:
Num 14:22  To?i barba?ii câ?i au vazut slava Mea ?i minunile pe care le-am facut în pamântul Egiptului ?i în pustie, ?i M-au ispitit pâna acum de zeci de ori ?i n-au ascultat glasul Meu,
Num 14:23  Nu vor vedea pamântul pe care Eu cu juramânt l-am fagaduit parin?ilor lor; ci numai copiilor lor, care sunt aici cu Mine, care nu ?tiu ce este binele ?i ce este raul ?i tuturor nevârstnicilor, care nu judeca, acelora le voi da pamântul, iar to?i cei ce M-au amarât nu-l vor vedea;
Num 14:24  Iar pe robul Meu Caleb, îl voi duce în pamântul în care a umblat ?i semin?ia lui îl va mo?teni, pentru ca în el a fost alt duh ?i pentru ca el s-a supus Mie.
Num 14:25  Amaleci?ii ?i Canaaneii locuiesc pe vale; mâine sa va întoarce?i ?i sa va duce?i în pustie, spre Marea Ro?ie".
Num 14:26  ?i a mai grait Domnul cu Moise ?i cu Aaron ?i a zis:
Num 14:27  "Pâna când aceasta ob?te rea va cârti împotriva Mea? Cârtirea cu care fiii lui Israel cârtesc împotriva Mea, o aud.
Num 14:28  Deci, spune-le: Viu sunt Eu, zice Domnul! Dupa cum a?i zis în auzul Meu, a?a voi face cu voi:
Num 14:29  În pustia aceasta vor cadea oasele voastre ?i voi to?i cei numara?i, de la douazeci de ani în sus, care a?i cârtit împotriva Mea, oricâ?i a?i fi la numar,
Num 14:30  Nu ve?i intra în pamântul pentru care, ridicându-Mi mâna, M-am jurat sa va a?ez, ci numai Caleb, fiul lui Iefone, ?i Iosua, fiul lui Navi.
Num 14:31  Pe copiii vo?tri, despre care voi zicea?i ca vor ajunge prada vrajma?ilor, îi voi duce acolo ?i ei vor cunoa?te pamântul pe care voi l-a?i nesocotit;
Num 14:32  Iar oasele voastre vor cadea în pustia aceasta.
Num 14:33  Copiii vo?tri vor rataci prin pustie patruzeci de ani ?i vor suferi pedeapsa pentru desfrânarea voastra, pâna vor cadea toate oasele voastre în pustie.
Num 14:34  Dupa numarul celor patruzeci de zile, în care a?i iscodit pamântul Canaan, ve?i purta pedeapsa pentru pacatele voastre patruzeci de ani, câte un an pentru fiecare zi, ca sa cunoa?te?i ce înseamna sa fi?i parasi?i de Mine.
Num 14:35  Eu, Domnul, am grait! ?i a?a voi face cu toata ob?tea aceasta rea, care s-a ridicat împotriva Mea: în pustia aceasta vor pieri ?i vor muri to?i!"
Num 14:36  ?i oamenii pe care-i trimisese Moise sa iscodeasca pamântul ?i care la întoarcere au întarâtat împotriva lui toata ob?tea aceasta, raspândind zvonuri rele despre ?ara aceea,
Num 14:37  Au murit lovi?i înaintea Domnului, pentru ca au raspândit zvonurile despre ?ara aceea.
Num 14:38  Numai Iosua, fiul lui Navi, ?i Caleb, fiul lui Iefone, au ramas vii dintre barba?ii aceia care fusesera sa iscodeasca ?ara Canaan.
Num 14:39  Cuvintele acestea le-a spus Moise înaintea tuturor fiilor lui Israel ?i poporul s-a întristat foarte tare.
Num 14:40  Sculându-se ei deci dis-de-diminea?a, s-au dus pe vârful muntelui, zicând: "Iata, ne ducem la locul acela de care ne-a grait Domnul, caci am gre?it!"
Num 14:41  Moise însa le-a zis: "Pentru ce calca?i porunca Domnului? Nu ve?i izbuti.
Num 14:42  Nu va duce?i, caci Domnul nu este între voi ?i ve?i cadea înaintea vrajma?ilor vo?tri;
Num 14:43  Caci Amaleci?ii ?i Canaaneii sunt acolo înaintea voastra ?i ve?i cadea de sabie, pentru ca v-a?i abatut de la Domnul ?i Domnul nu va fi cu voi".
Num 14:44  Dar ei au îndraznit sa se urce pe vârful muntelui; iar chivotul legii Domnului ?i Moise n-au parasit tabara.
Num 14:45  Atunci s-au suit Amaleci?ii ?i Canaaneii care traiau în muntele acela ?i i-au înfrânt ?i i-au gonit pâna la Horma ?i s-au întors în tabara.
Num 15:1  În vremea aceea a grait Domnul cu Moise ?i a zis:
Num 15:2  "Vorbe?te fiilor lui Israel ?i le spune: Când ve?i intra în pamântul vostru de locuit, pe care Eu îl voi da voua,
Num 15:3  ?i când ve?i face jertfe Domnului din oi sau din boi, ardere de tot, sau jertfa de fagaduin?a sau de buna voie, sau când ve?i face la sarbatorile voastre mireasma placuta Domnului,
Num 15:4  Atunci cel ce aduce darul sau Domnului sa aduca jertfa de pâine a zecea parte de efa de faina de grâu curata, amestecata cu un sfert de hin de untdelemn,
Num 15:5  ?i vin pentru turnare, a patra parte de hin la ardere de tot sau la jertfa de fagaduin?a, la fiecare miel va face la fel întru miros bine-placut Domnului.
Num 15:6  Iar când ve?i aduce berbec, adu jertfa de pâine doua zecimi de efa de faina de grâu curata, amestecata cu a treia parte de hin de untdelemn;
Num 15:7  ?i vin de turnare sa aduci a treia parte de hin, întru miros de buna mireasma Domnului.
Num 15:8  Daca aduce?i junc, ardere de tot, sau jertfa de fagaduin?a, sau jertfa de împacare,
Num 15:9  Atunci cu juncul sa aduci prinos de pâine trei zecimi de efa de faina de grâu, amestecata cu jumatate de hin de untdelemn.
Num 15:10  ?i vin pentru turnare, jumatate de hin la jertfa, întru miros de buna mireasma Domnului.
Num 15:11  A?a sa faci totdeauna, când aduci junc sau berbec, miel sau capra,
Num 15:12  Dupa numarul jertfelor pe care le face?i; a?a sa aduce?i la fiecare, dupa numarul lor.
Num 15:13  Tot ba?tina?ul sa faca a?a când aduce jertfe de acestea întru mireasma placuta Domnului.
Num 15:14  De va trai însa printre voi în pamântul vostru un strain ?i ar fi între voi din neam în neam, ?i va voi sa aduca jertfa pentru miros placut Domnului, sa faca a?a cum face?i voi.
Num 15:15  Pentru voi ob?tea Domnului ?i pentru strainul care locuie?te între voi, o singura lege sa fie, lege ve?nica din neam în neam. Cum sunte?i voi a?a sa fie ?i strainul înaintea Domnului.
Num 15:16  O singura lege ?i acelea?i drepturi sa fie pentru voi ?i pentru strainul care locuie?te la voi".
Num 15:17  ?i a grait Domnul cu Moise ?i a zis:
Num 15:18  "Vorbe?te fiilor lui Israel ?i le spune: Când ve?i intra în pamântul în care va duc,
Num 15:19  ?i ve?i mânca pâinea ?arii aceleia, sa înal?a?i prinos Domnului.
Num 15:20  Pârga din aluatul vostru sa înal?a?i dar Domnului o azima; dar s-o înal?a?i a?a ca prinosul din arie;
Num 15:21  Pârga din aluatul vostru sa înal?a?i dar Domnului din neam în neam.
Num 15:22  Daca însa ve?i gre?i din ne?tiin?a ?i nu ve?i împlini toate poruncile acestea, pe care le-a rostit Domnul lui Moise,
Num 15:23  ?i tot ce v-a poruncit Domnul prin Moise din ziua în care a început Domnul a va porunci,
Num 15:24  Daca gre?eala e din nebagarea de seama a ob?tii, atunci toata ob?tea sa aduca din cireada un junc fara meteahna, ardere de tot, întru miros bineplacut Domnului, cu dar de pâine, cu turnarea lui dupa rânduiala, ?i din turma de capre, un ?ap ca jertfa pentru pacat
Num 15:25  ?i se va ruga preotul pentru toata ob?tea fiilor lui Israel ?i li se va ierta, caci aceasta a fost gre?eala ?i ei au adus darul lor Domnului ?i jertfa pentru pacatul lor înaintea Domnului, pentru gre?eala lor.
Num 15:26  Atunci se va ierta întregii ob?ti a fiilor lui Israel ?i strainului care traie?te între ei, pentru ca tot poporul a facut aceasta din ne?tiin?a.
Num 15:27  Daca vreun suflet a gre?it din ne?tiin?a, sa aduca o capra de un an jertfa pentru pacat
Num 15:28  ?i se va ruga preotul pentru sufletul care a facut pacat din ne?tiin?a înaintea Domnului ?i va afla mila ?i i se va ierta.
Num 15:29  ?i pentru ba?tina?ul din Israel ?i pentru strainul care traie?te între voi, o singura lege sa fie când cineva va pacatui din ne?tiin?a.
Num 15:30  Iar daca cineva dintre ba?tina?i sau dintre straini va face ceva din îndrazneala, acela hule?te pe Domnul ?i sufletul lui se va stârpi din poporul sau,
Num 15:31  Caci a dispre?uit cuvântul Domnului ?i a calcat poruncile Lui; sa se stârpeasca sufletul acela ?i pacatul lui va fi asupra lui".
Num 15:32  Când se aflau fiii lui Israel în pustiu, au gasit un om adunând lemne în ziua odihnei;
Num 15:33  ?i cei ce l-au gasit adunând lemne în ziua odihnei l-au adus la Moise ?i Aaron ?i la toata ob?tea fiilor lui Israel;
Num 15:34  ?i l-au pus sub paza, pentru ca nu era înca hotarât ce sa faca cu el.
Num 15:35  Atunci a zis Domnul catre Moise: "Omul acesta sa moara; sa fie ucis cu pietre de catre toata ob?tea fiilor lui Israel, afara din tabara!"
Num 15:36  L-au scos deci toata ob?tea fiilor lui Israel afara din tabara ?i l-au ucis cu pietre toata ob?tea, afara din tabara, cum poruncise Domnul lui Moise.
Num 15:37  ?i a grait Domnul cu Moise ?i a zis:
Num 15:38  "Vorbe?te fiilor lui Israel ?i le spune sa-?i faca ciucuri la poalele hainelor lor, din neam în neam, ?i pe deasupra ciucurilor de la poalele hainelor lor sa puna un ?iret de matase violeta.
Num 15:39  Ciucurii ace?tia sa fie ca, uitându-va la ei, sa va aduce?i aminte de toate poruncile Domnului ?i sa le împlini?i ?i sa nu umbla?i dupa inima voastra ?i dupa ochii vo?tri care va îndeamna la desfrânare;
Num 15:40  Ca sa va aduce?i aminte ?i sa plini?i toate poruncile Mele ?i sa fi?i sfin?i înaintea Dumnezeului vostru.
Num 15:41  Eu sunt Domnul Dumnezeul vostru, Care v-am scos din pamântul Egiptului, ca sa fiu Dumnezeul vostru. Eu sunt Domnul Dumnezeul vostru".
Num 16:1  Atunci Core, fiul lui I?har, fiul lui Cahat, fiul lui Levi, cu Datan ?i cu Abiron, fiii lui Eliab, cu On, fiul lui Felet, din semin?ia lui Ruben, s-au sculat împotriva lui Moise,
Num 16:2  Împreuna cu doua sute cincizeci de barba?i, capetenii ale ob?tii fiilor lui Israel, oameni însemna?i, care erau chema?i la adunare.
Num 16:3  Adunându-se ace?tia împotriva lui Moise ?i Aaron, le-au zis: "Destul! Toata ob?tea ?i to?i cei ce o alcatuiesc sunt sfin?i ?i Domnul este între ei. Pentru ce va socoti?i voi mai presus de adunarea Domnului!"
Num 16:4  Auzind acestea, Moise a cazut cu fa?a la pamânt
Num 16:5  ?i a grait lui Core ?i tuturor parta?ilor lui ?i le-a zis: "Mâine va arata Domnul cine este al Lui ?i cine este sfânt, ca sa ?i-L apropie; ?i pe cine va alege El, pe acela îl va ?i apropia la Sine.
Num 16:6  Iata ce sa face?i: Core ?i to?i parta?ii tai sa va lua?i cadelni?e
Num 16:7  ?i mâine sa pune?i în acestea foc ?i sa turna?i în ele tamâie înaintea Domnului, ?i pe cine va alege Domnul, acela va fi sfânt. Destul, fiii lui Levi!
Num 16:8  Apoi a zis iara?i Moise catre Core: "Asculta?i, fii ai lui Levi:
Num 16:9  Oare e pu?in lucru pentru voi ca Dumnezeul lui Israel v-a osebit de ob?tea lui Israel ?i v-a apropiat la Sine ca sa face?i slujbe la cortul Domnului ?i sa sta?i înaintea ob?tii Domnului, slujind pentru ea?
Num 16:10  El te-a apropiat pe tine ?i cu tine pe to?i fra?ii tai, fiii lui Levi. Alerga?i acum ?i dupa preo?ie?
Num 16:11  A?adar tu ?i toata ob?tea ta, v-a?i adunat împotriva Domnului. Ce este Aaron, de cârti?i împotriva lui?"
Num 16:12  Atunci a trimis Moise sa cheme pe Datan ?i pe Abiron, fiii lui Eliab. Ei însa au zis: "Nu mergem!
Num 16:13  Oare pu?in lucru e ca ne-ai scos din ?ara unde curge miere ?i lapte ?i ne-ai adus sa ne pierzi în pustie? Vrei sa ?i domne?ti peste noi?
Num 16:14  Dusu-ne-ai tu oare în ?ara unde curge lapte ?i miere ?i datu-ne-ai tu oare în stapânire ?arinile ?i viile ei? Vrei sa sco?i ochii oamenilor acestora? Nu mergem!"
Num 16:15  ?i s-a mâhnit Moise foarte tare ?i a zis catre Domnul: "Sa nu-?i întorci ochii Tai la prinosul lor. Eu nici unuia dintre ei nu i-am luat asinul ?i rau n-am facut nici unuia dintre ei".
Num 16:16  Apoi a zis Moise catre Core: "Sfin?e?te-?i ceata ta ?i mâine sa fi?i gata înaintea Domnului: tu, ei ?i Aaron.
Num 16:17  Lua?i-va fiecare cadelni?e, pune?i în ele tamâie ?i va apropia?i fiecare cu cadelni?a înaintea Domnului, cu doua sute cincizeci de cadelni?e: ?i tu ?i Aaron sa aduce?i fiecare cadelni?a voastra".
Num 16:18  ?i ?i-a luat fiecare cadelni?a sa, au pus în ele foc, au turnat tamâie în ele; ?i au stat înaintea intrarii cortului adunarii Moise ?i Aaron.
Num 16:19  Core însa a adunat împotriva lor toata ob?tea înaintea u?ii cortului adunarii. ?i s-a aratat slava Domnului la toata ob?tea.
Num 16:20  ?i a grait Domnul cu Moise ?i Aaron ?i a zis:
Num 16:21  "Osebi?i-va de ob?tea aceasta ?i-i voi pierde într-o clipa".
Num 16:22  Iar ei au cazut cu fe?ele la pamânt ?i au zis: "Doamne, Dumnezeul duhurilor ?i a tot trupul, un om a gre?it ?i Tu Te mânii pe toata ob?tea? "
Num 16:23  Domnul însa i-a zis lui Moise:
Num 16:24  "Spune ob?tii: Feri?i-va în toate par?ile de locuin?a lui Core, a lui Datan ?i a lui Abiron".
Num 16:25  Atunci, sculându-se, Moise s-a dus la Datan ?i Abiron ?i s-au dus dupa el ?i to?i batrânii lui Israel.
Num 16:26  ?i a zis ob?tii: "Feri?i-va de corturile acestor oameni netrebnici ?i sa nu va atinge?i de tot ce e al lor, ca sa nu pieri?i cu toate pacatele lor".
Num 16:27  ?i ei au ocolit sala?urile lui Core, Datan ?i Abiron; iar Datan ?i Abiron ie?isera ?i stateau la u?ile corturilor lor, cu femeile lor ?i cu fiii lor ?i cu pruncii lor.
Num 16:28  Zis-a Moise: "Ca Domnul m-a trimis sa fac toate lucrurile acestea ?i ca nu le fac eu de la mine, ve?i cunoa?te din aceea:
Num 16:29  De vor muri ace?tia, cum moc to?i oamenii, ?i de-i va ajunge aceea?i pedeapsa, care ajunge pe to?i oamenii, - atunci nu m-a trimis Domnul.
Num 16:30  Iar daca Domnul va face lucru neobi?nuit, de-?i va deschide pamântul gura sa ?i-i va înghi?i pe ei ?i casele lor ?i corturile lor ?i tot ce au ei, ?i daca ei vor fi du?i de vii în locuin?a mor?ilor, atunci sa ?ti?i ca oamenii ace?tia au dispre?uit pe Domnul".
Num 16:31  Cum a încetat el sa spuna toate cuvintele acestea, s-a desfacut pamântul sub aceia
Num 16:32  ?i ?i-a deschis pamântul gura sa ?i i-a înghi?it pe ei ?i casele lor, pe to?i oamenii lui Core ?i toata averea;
Num 16:33  ?i s-au pogorât ei cu toate câte aveau de vii în locuin?a mor?ilor, i-a acoperit pamântul ?i au pierit din mijlocul ob?tii.
Num 16:34  ?i tot Israelul, care era împrejurul lor, a fugit la strigatele lor, ca ziceau: "Sa nu ne înghita ?i pe noi pamântul!"
Num 16:35  A ie?it apoi foc de la Domnul ?i a mistuit pe cei doua sute cincizeci de barba?i care au adus tamâie.
Num 16:36  Dupa aceea a grait Domnul cu Moise ?i a zis:
Num 16:37  "Spune lui Eleazar, fiul preotului Aaron, sa adune cadelni?ele cele de arama ale celor ar?i ?i focul strain sa-l arunce afara, caci s-au sfin?it cadelni?ele pacato?ilor acestora prin moartea lor.
Num 16:38  Sa le sfarâme deci ?i sa le faca foi pentru acoperit jertfelnicul. Pentru ca le-au adus aceia înaintea Domnului, s-au sfin?it ?i vor fi semn pentru fiii lui Israel".
Num 16:39  A luat deci Eleazar, fiul  preotului Aaron, cadelni?ele cele de arama, pe care le adusesera cei ar?i, ?i le-a prefacut în foi pentru acoperit jertfelnicul,
Num 16:40  Ca sa-?i aduca aminte fiii lui Israel ca nimeni din alt neam, care nu e din semin?ia lui Aaron, sa nu se apropie sa aduca tamâiere înaintea Domnului, ?i sa nu fie ca ?i Core ?i parta?ii lui, precum îi graise Domnul prin Moise.
Num 16:41  A doua zi însa toata ob?tea fiilor lui Israel a cârtit împotriva lui Moise ?i a lui Aaron ?i a zis: Voi a?i omorât poporul Domnului.
Num 16:42  ?i când s-a adunat ob?tea împotriva lui Moise ?i Aaron, ace?tia s-au întors catre cortul adunarii, ?i iata norul l-a acoperit ?i s-a aratat slava Domnului.
Num 16:43  ?i a venit Moise ?i Aaron la cortul adunarii.
Num 16:44  Atunci a grait Domnul cu Moise ?i Aaron ?i a zis:
Num 16:45  "Departa?i-va de ob?tea aceasta, ca într-o clipa o voi pierde". Iar ei au cazut cu fa?a la pamânt.
Num 16:46  ?i a zis Moise catre Aaron: "Ia-?i cadelni?a, pune în ea foc de pe jertfelnic, arunca în ea tamâie ?i du-o repede în tabara ?i te roaga pentru ei, ca a ie?it mânie de la fa?a Domnului ?i a început pedepsirea poporului".
Num 16:47  Atunci Aaron a luat, cum îi zisese Moise, a alergat în mijlocul ob?tii ?i iata se începuse moartea în popor; ?i a pus tamâia ?i s-a rugat pentru popor;
Num 16:48  ?i stând el între mor?i ?i vii, a încetat bataia.
Num 16:49  Au murit atunci din pedepsirea aceea paisprezece mii ?apte sute de oameni, afara de cei ce murisera pentru razvratirea lui Core.
Num 16:50  Iar dupa ce a încetat pedepsirea, s-a întors Aaron la Moise, la u?a cortului adunarii.
Num 17:1  Dupa aceea a grait Domnul lui Moise ?i a zis:
Num 17:2  "Spune fiilor lui Israel ?i ia de la ei, de la toate capeteniile lor, dupa semin?ii, douasprezece toiege, câte un toiag de fiecare semin?ie, ?i numele fiecarei capetenii scrie-l pe toiagul sau;
Num 17:3  Iar numele lui Aaron sa-l scrii pe toiagul lui Levi, caci un toiag vor da de fiecare capetenie de semin?ie.
Num 17:4  Toiegele acelea sa le pui în cortul adunarii înaintea chivotului legii, unde Ma arat ?ie.
Num 17:5  ?i va fi ca toiagul omului pe care-l voi alege va odrasli; ?i a?a voi potoli cârtirea fiilor lui Israel, cu care cârtesc ei împotriva voastra".
Num 17:6  ?i Moise a spus acestea fiilor lui Israel ?i toate capeteniile lor i-au dat toiegele, câte un toiag de fiecare capetenie, adica douasprezece toiege, dupa cele douasprezece semin?ii ale lor; ?i toiagul lui Aaron era între toiegele lor.
Num 17:7  Apoi Moise a pus toiegele înaintea Domnului, în cortul adunarii.
Num 17:8  Iar a doua zi a intrat Moise ?i Aaron în cortul adunarii ?i iata toiagul lui Aaron, din casa lui Levi, odraslise, înmugurise, înflorise ?i facuse migdale.
Num 17:9  ?i atunci a scos Moise toate toiegele de la fa?a Domnului la to?i fiii lui Israel; ?i au vazut ?i ?i-au luat fiecare toiagul sau.
Num 17:10  Apoi a zis Domnul catre Moise: Pune iar toiagul lui Aaron înaintea chivotului legii spre pastrare, ca semn pentru fiii neascultatori, ca sa înceteze de a mai cârti împotriva Mea, ca sa nu moara!"
Num 17:11  ?i a facut Moise a?a; cum îi poruncise Domnul a?a a facut.
Num 17:12  ?i au zis fiii lui Israel catre Moise: "Iata murim, pierim, pierim cu to?ii!
Num 17:13  Tot cel ce se apropie de cortul Domnului moare; nu cumva o sa murim cu to?ii?"
Num 18:1  Zis-a Domnul catre Aaron: "Tu, fiii tai ?i casa tatalui tau cu tine ve?i purta pacatul pentru nepasarea de loca?ul sfânt; tu ?i fiii tai împreuna cu tine ve?i purta pacatul pentru nepasarea de preo?ia voastra.
Num 18:2  Apropie-?i pe fra?ii tai, semin?ia lui Levi, neamul tatalui tau, ca sa fie pe lânga tine ?l sa-?i slujeasca; iar tu ?i fiii tai împreuna cu tine ve?i fi la cortul adunarii.
Num 18:3  Levi?ii sa pazeasca cele rânduite de tine ?i sa faca slujba la cort, dar sa nu se apropie de lucrurile loca?ului sfânt ?i de jertfelnic, ca sa nu moara ?i ei ?i voi.
Num 18:4  Sa fie deci pe lânga tine ?i sa faca slujba la cortul adunarii ?i toate lucrarile la cort; iar altul sa nu se apropie de tine.
Num 18:5  A?a sa face?i slujba în loca?ul sfânt ?i la jertfelnic, ?i nu va mai veni mânia asupra fiilor lui Israel;
Num 18:6  Ca am ales din fiii lui Israel pe fra?ii vo?tri, pe levi?i, ?i vi i-am dat ca dar închinat Domnului, sa faca slujba la cortul adunarii;
Num 18:7  Iar tu ?i fiii tai sa va îndeplini?i preo?ia voastra în toate cele ce ?in de jertfelnic ?i ce se afla înauntru dupa perdea, ?i sa savâr?i?i slujbele darului vostru preo?esc, iar altul strain, de se va apropia, sa fie omorât".
Num 18:8  Zis-a Domnul catre Aaron: "Iata Eu am dat în seama voastra pârga Mea din toate cele închinate Mie de fiii lui Israel: ?ie ?i le-am dat acestea ?i dupa tine fiilor tai, pentru cinul vostru, pentru preo?ia voastra, prin lege ve?nica.
Num 18:9  Iata ce este al tau din cele preasfinte, în afara de cele ce se dau focului: orice dar de pâine al lor, orice jertfa pentru pacat a lor ?i orice jertfa pentru vina, ce-Mi aduc ei, aceste lucruri preasfinte sa fie ale tale ?i ale fiilor tai.
Num 18:10  Acestea sa le mânca?i în locul cel sfânt. Tu ?i fiii tai, to?i cei de parte barbateasca ai vo?tri pot sa manânce din ele. Cele sfinte sa fie ale tale.
Num 18:11  ?i iata ce sa mai fie al vostru din darurile lor ridicate: toate darurile ridicate ale fiilor lui Israel ?i toate darurile lor leganate ?i le-am dat ?ie ?i fiilor tai ?i fiicelor tale, care sunt cu tine, prin lege ve?nica. Tot cel curat din casa ta poate sa manânce din acestea.
Num 18:12  Toata pârga de untdelemn ?i toata pârga de struguri ?i pârga grâului lor, toate câte aduc ei Domnului, ?i le-am dat ?ie.
Num 18:13  Cele dintâi roade ale pamântului lor, pe care le aduc ei Domnului, sa fie ale tale, ?i tot cel curat din casa ta poate sa manânce din acestea.
Num 18:14  Tot ce este afierosit în Israel sa fie al tau.
Num 18:15  Tot ce se na?te întâi din tot trupul, din oameni ?i din dobitoace, ?i se aduce Domnului, sa fie al tau; dar întâiul nascut dintre oameni sa se rascumpere ?i întâiul nascut dintre dobitoacele necurate sa se rascumpere;
Num 18:16  Iar pre?ul rascumpararii lui, la o luna dupa na?tere, este cinci sicli de argint, dupa siclul sfânt, care are douazeci de ghere.
Num 18:17  Însa întâiul nascut al vacilor, întâiul nascut al oilor ?i întâiul nascut al caprelor, nu se rascumpara: ace?tia sunt sfin?i?i; cu sângele lor sa strope?ti jertfelnicul. Grasimea lor s-o arzi ca jertfa, întru miros de buna mireasma Domnului;
Num 18:18  Iar carnea lor este a ta ?i tot ale tale sunt pieptul înal?at ?i ?oldul drept.
Num 18:19  Toate darurile sfinte, înal?ate, care se aduc Domnului de fiii lui Israel, ?i le dau ?ie, fiilor tai ?i fiicelor tale care sunt cu tine, prin lege ve?nica. Acest legamânt de necalcat este ve?nic înaintea Domnului pentru tine ?i pentru urma?ii tai".
Num 18:20  Zis-a Domnul catre Aaron: "În pamântul lor nu vei avea nici mo?tenire, nici parte nu vei avea între ei. Eu sunt partea ta ?i mo?tenirea ta între fiii lui Israel,
Num 18:21  Iar fiilor lui Levi, iata, Eu le-am dat mo?tenire toata zeciuiala din toate câte are Israel, pentru slujba lor pe care o fac la cortul adunarii.
Num 18:22  De acum fiii lui Israel sa nu mai vina la cortul adunarii, ca sa nu faca pacat aducator de moarte.
Num 18:23  Ci la cortul adunarii sa faca slujba levi?ii ?i sa ia asupra-?i pacatul lor. Aceasta este lege ve?nica în neamul vostru.
Num 18:24  Dar printre fiii lui Israel ei nu vor avea mo?tenire, caci zeciuiala fiilor lui Israel, pe care ace?tia o aduc dar Domnului, am dat-o levi?ilor mo?tenire ?i de aceea le-am ?i zis Eu ca nu vor avea mo?tenire între fiii lui Israel".
Num 18:25  Apoi a grait Domnul lui Moise ?i a zis:
Num 18:26  "Vorbe?te levi?ilor ?i le zi: Când ve?i lua de la fiii lui Israel zeciuiala, pe care v-am dat-o ca mo?tenire, sa înal?a?i din ea dar Domnului a zecea parte, ca zeciuiala,
Num 18:27  ?i vi se va socoti acest dar al vostru ca grâul din arie ?i ca mustul de la teasc.
Num 18:28  Astfel ve?i aduce ?i voi dar Domnului din toate zeciuielile voastre, câte ve?i lua de la fiii lui Israel, ?i ve?i da din ele, dar Domnului, preotului Aaron.
Num 18:29  Din toate cele daruite voua, cele mai bune din toate cele sfin?ite sa le aduce?i dar Domnului,
Num 18:30  ?i sa le spui: De ve?i aduce din acestea partea cea mai buna, se va socoti levi?ilor ca cele primite de la arie ?i ca cele primite de la teasc.
Num 18:31  Aceasta sa o mânca?i oriunde voi, ?i fiii vo?tri, ?i familiile voastre, caci aceasta va este plata pentru munca voastra la cortul adunarii.
Num 18:32  Pentru aceasta nu ve?i avea pacat, de ve?i aduce cele mai bune din toate; ?i sfintele prinoase ale fiilor lui Israel nu le ve?i întina ?i nu ve?i muri".
Num 19:1  Grait-a Domnul cu Moise ?i Aaron ?i a zis:
Num 19:2  "Iata porunca legii, pe care a dat-o Domnul, când a zis: Spune fiilor lui Israel sa-?i aduca o juninca ro?ie, fara meteahna, care sa nu aiba cusur ?i sa nu fi purtat jug;
Num 19:3  Sa o dai preotului Eleazar, sa o scoata afara din tabara, la loc curat, ?i sa o junghie înaintea lui.
Num 19:4  Apoi sa ia preotul Eleazar din sângele ei ?i sa stropeasca cu sânge spre partea de dinainte a cortului adunarii de ?apte ori.
Num 19:5  Dupa aceea sa o arda de tot înaintea lui; sa arda adica ?i carnea ?i pielea ?i sângele ?i necura?enia ei.
Num 19:6  Apoi sa ia preotul lemn de cedru, isop ?i a?a de lâna ro?ie ?i sa le arunce pe juninca ce se arde.
Num 19:7  Sa-?i spele preotul hainele sale, sa-?i spele trupul cu apa, apoi sa intre în tabara ?i necurat va fi pâna seara.
Num 19:8  Cel ce a ars-o de asemenea sa-?i spele hainele sale, sa-?i spele trupul cu apa ?i necurat va fi pâna seara.
Num 19:9  Un om curat sa strânga cenu?a junincii, s-o puna afara din tabara la loc curat ?i sa se pastreze pentru ob?tea fiilor lui Israel, ca sa se faca cu ea apa de stropire, adica apa de cura?ire.
Num 19:10  Cel ce a adunat cenu?a junincii sa-?i spele hainele sale ?i sa fie necurat pâna seara. Aceasta sa fie a?ezamânt ve?nic pentru fiii lui Israel ?i pentru strainii ce traiesc la dân?ii.
Num 19:11  Cel ce se va atinge de trupul mort al unui om sa fie necurat ?apte zile.
Num 19:12  Acesta sa se cure?e cu aceasta apa în ziua a treia ?i în ziua a ?aptea ?i va fi curat; iar de nu se va cura?i în ziua a treia ?i în ziua a ?aptea, nu va fi curat.
Num 19:13  Tot cel ce sg va atinge de trupul mort al unui om ?i nu se va cura?i, acela va întina loca?ul Domnului; omul acela se va stârpi din Israel, caci n-a fost stropit cu apa cura?itoare ?i este necurat ?i necura?ia lui e înca asupra lui.
Num 19:14  Iata legea: De va muri un om într-o casa, tot cel ce va intra în casa aceea ?i câte sunt în casa vor fi necurate ?apte zile.
Num 19:15  Tot vasul descoperit, care nu este legat la gura ?i n-are capac pe el, este necurat.
Num 19:16  Tot cel ce se va atinge în câmp de cel ucis cu sabia sau de mort sau de os de om sau de mormânt va fi necurat ?apte zile.
Num 19:17  Pentru cel necurat sa se ia din cenu?a jertfei arse pentru cura?ire ?i sa se toarne peste ea într-un vas apa de izvor;
Num 19:18  Apoi un om curat sa ia isop, sa-l moaie în apa aceea ?i sa stropeasca din ea casa, lucrurile ?i oamenii câ?i sunt acolo ?i pe cel ce s-a atins de os de om sau de ucis sau de mort sau de mormânt.
Num 19:19  Cel curat sa stropeasca pe cel necurat în ziua a treia ?i în ziua a ?aptea ?i sa-l cure?e în ziua a ?aptea. Apoi sa-?i spele hainele sale ?i trupul sau cu apa ?i va fi necurat pâna seara.
Num 19:20  Iar daca vreun om va fi necurat ?i nu se va cura?i, omul acela se va stârpi din ob?te, caci a spurcat loca?ul Domnului; caci nu s-a stropit cu apa cura?itoare ?i este necurat.
Num 19:21  Acesta sa fie a?ezamânt ve?nic pentru dân?ii. Cel ce a stropit cu apa cura?itoare sa-?i spele hainele sale; cel ce s-a atins de apa cura?itoare va fi necurat pâna seara.
Num 19:22  Tot lucrul de care se va atinge cel necurat va fi necurat; ?i tot ce se va atinge de acel lucru va fi necurat pâna seara".
Num 20:1  În luna întâi a ajuns toata ob?tea fiilor lui Israel în pustiul Sin ?i s-a oprit poporul în Cade?. ?i a murit Mariam ?i a fost îngropata acolo.
Num 20:2  Acolo însa nu era apa pentru ob?te ?i s-au adunat ei împotriva lui Moise ?i a lui Aaron,
Num 20:3  ?i blestema poporul pe Moise ?i zicea: "O, de am fi murit ?i noi când au murit fra?ii no?tri înaintea Domnului!
Num 20:4  La ce a?i adus voi ob?tea Domnului în pustiul acesta, ca sa ne omorâ?i ?i pe noi ?i dobitoacele noastre?
Num 20:5  ?i la ce ne-a?i scos din Egipt, ca sa ne aduce?i în acest loc rau, unde nu se poate semana ?i nu sunt nici smochini, nici vite, nici rodii ?i nici macar apa de baut?"
Num 20:6  Atunci s-au dus Moise ?i Aaron din fa?a poporului la u?a cortului adunarii ?i au cazut cu fe?ele la pamânt ?i s-a aratat slava Domnului peste ei.
Num 20:7  ?i a grait Domnul cu Moise ?i a zis:
Num 20:8  "Ia toiagul ?i aduna ob?tea, tu ?i Aaron, fratele tau, ?i grai?i stâncii înaintea lor ?i ea va va da apa; ?i le ve?i scoate apa din stânca ?i ve?i adapa ob?tea ?i dobitoacele ei".
Num 20:9  A luat deci Moise toiagul din fa?a Domnului, cum poruncise Domnul.
Num 20:10  ?i au adunat Moise ?i Aaron ob?tea la stânca ?i a zis catre ob?te: "Asculta?i, îndaratnicilor, au doara din stânca aceasta va vom scoate apa?"
Num 20:11  Apoi ?i-a ridicat Moise mâna ?i a lovit în stânca cu toiagul sau de doua ori ?i î ie?it apa multa ?i baut ob?tea ?i dobitoacele ei.
Num 20:12  Atunci a zis Domnul catre Moise ?i Aaron: "Pentru ca nu M-a?i crezut, ca sa arata?i sfin?enia Mea înaintea ochilor fiilor lui Israel, de aceea nu ve?i duce voi adunarea aceasta în pamântul pe care am sa i-l dau".
Num 20:13  Aceasta este apa Meriba, caci aici fiii lui Israel s-au certat înaintea Domnului, iar El S-a sfin?it între ei.
Num 20:14  Din Cade? a trimis Moise soli la regele Edomului, ca sa-i spuna: "A?a zice fratele tau Israel: Tu ?tii toate greuta?ile ce am îndurat.
Num 20:15  Parin?ii no?tri s-au pogorât în Egipt ?i noi am pribegit în Egipt vreme multa; dar Egiptenii ne-au facut rau noua ?i parin?ilor no?tri.
Num 20:16  De aceea am strigat catre Domnul ?i a auzit Domnul glasul nostru ?i a trimis îngerul Sau de ne-a scos din Egipt; ?i acum suntem în Cade?, ora?ul cel mai apropiat de hotarul tau.
Num 20:17  Îngaduie?te-ne sa trecem prin ?ara ta, ca nu ne vom abate pe la ogoare ?i pe la vii, nici apa nu vom bea din fântânile tale; ci vom trece pe drumul împaratesc, neabatându-ne nici la dreapta, nici la stânga, pâna vom ie?i din hotarele tale".
Num 20:18  Edom însa i-a raspuns: "Sa nu treci pe la mine, iar de nu vei asculta voi ie?i cu razboi înaintea ta".
Num 20:19  Zisu-i-au fiii lui Israel: "Vom merge pe drumul cel mare ?i de vom bea din apa ta, noi sau dobitoacele noastre, î?i vom plati; vom trece numai cu piciorul, ceea ce e un lucru de nimic".
Num 20:20  Iar acela i-a raspuns: "Sa nu treci pe la mine!" ?i a ie?it Edom înaintea lui cu popor mult ?i cu mâna puternica.
Num 20:21  Deci nu s-a învoit Edom sa dea voie lui Israel sa treaca prin hotarele lui ?i Israel s-a departat de la el.
Num 20:22  Dupa aceea au pornit fiii lui Israel din Cade? ?i au venit toata ob?tea la muntele Hor.
Num 20:23  Iar la muntele Hor, care e lânga hotarele ?arii lui Edom, a grait Domnul cu Moise ?i cu Aaron ?i a zis:
Num 20:24  "Aaron va fi adaugat la poporul sau, ca el nu va intra în pamântul pe care îl voi da fiilor lui Israel, pentru ca nu v-a?i supus poruncii Mele la apa Meriba.
Num 20:25  Sa iei dar pe fratele tau Aaron ?i pe Eleazar, fiul lui, ?i sa-i sui pe muntele Hor înaintea întregii ob?ti;
Num 20:26  Sa dezbraci acolo de pe Aaron hainele lui ?i sa îmbraci cu ele pe Eleazar, fiul lui, ?i Aaron sa se duca ?i sa moara acolo".
Num 20:27  ?i a facut Moise a?a cum îi poruncise Domnul: i-a suit pe muntele Hor înaintea ochilor întregii ob?ti.
Num 20:28  Acolo a dezbracat Moise de pe Aaron hainele lui ?i a îmbracat cu ele pe Eleazar, fiul lui. ?i a murit Aaron pe vârful muntelui, iar Moise ?i Eleazar s-au pogorât din munte.
Num 20:29  Vazând toata ob?tea ca a murit Aaron, l-a plâns toata casa lui Israel treizeci de zile.
Num 21:1  Auzind însa regele canaanean din Arad, care locuia la miazazi, ca Israel vine pe drumul dinspre Atarim, a intrat în lupta cu Israeli?ii ?i a luat pe unii din ei în robie.
Num 21:2  Atunci a facut Israel fagaduin?a Domnului ?i a zis: "De vei da pe poporul acesta în mâinile mele, îl voi nimici pe el ?i ceta?ile lui".
Num 21:3  ?i a ascultat Domnul glasul lui Israel ?i a dat pe Canaanei în mâinile lui ?i el i-a nimicit pe ei ?i ora?ele lor ?i a pus locului aceluia numele: Horma, adica nimicire.
Num 21:4  De la muntele Hor au apucat pe calea Marii Ro?ii, ca sa ocoleasca pamântul lui Edom, dar pe drum poporul a început sa-?i piarda rabdarea.
Num 21:5  ?i graia poporul împotriva lui Dumnezeu ?i împotriva lui Moise, zicând: "La ce ne-ai scos din pamântul Egiptului, ca sa ne omori în pustiu, ca aici nu este nici pâine, nici apa ?i sufletul nostru s-a scârbit de aceasta hrana saracacioasa".
Num 21:6  Atunci a trimis Domnul asupra poporului ?erpi venino?i, care mu?cau poporul, ?i a murit mul?ime de popor din fiii lui Israel.
Num 21:7  A venit deci poporul la Moise ?i a zis: "Am gre?it, graind împotriva Domnului ?i împotriva ta; roaga-te Domnului, ca sa departeze ?erpii de la noi". ?i s-a rugat Moise Domnului pentru popor.
Num 21:8  Iar Domnul a zis catre Moise: "Fa-?i un ?arpe de arama ?i-l pune pe un stâlp; ?i de va mu?ca ?arpele pe vreun om, tot cel mu?cat care se va uita la el va trai.
Num 21:9  ?i a facut Moise un ?arpe de arama ?i l-a pus pe un stâlp; ?i când un ?arpe mu?ca vreun om, acesta privea la ?arpele cel de arama ?i traia.
Num 21:10  Sculându-se de acolo, fiii lui Israel au tabarât la Obot.
Num 21:11  Iar dupa ce s-au ridicat ?i din Obot, au tabarât la Iie-Abarim, dincolo de pustiu, în fa?a Moabului, catre rasaritul soarelui.
Num 21:12  De acolo s-au ridicat ?i au tabarât în valea Zared.
Num 21:13  Ridicându-se apoi ?i de acolo, au tabarât dincolo de Arnon, în pustia care e afara din hotarele Amoreilor. Caci Arnonul este hotar între Moabi?i ?i Amorei.
Num 21:14  De aceea se ?i zice în "Cartea razboaielor Domnului":
Num 21:15  Domnul a cuprins Vahebul cu curgerile sale navalnice ?i ?uvoaiele Arnonului ?i povârni?ul curgerilor de apa care se întinde pâna la localitatea Ar ?i se opre?te în hotarul Moabului.
Num 21:16  De acolo s-au îndreptat spre Beer, fântâna despre care a zis Domnul lui Moise: "Aduna poporul ?i le voi da apa sa bea".
Num 21:17  Atunci a cântat Israel la fântâna cântarea aceasta: "Lauda?i fântâna aceasta, cânta?i imne în cinstea ei!
Num 21:18  Fântâna pe care principii au sapat-o, pe care mai-marii poporului au deschis-o cu sceptrul, cu toiegele lor!"
Num 21:19  Din Beer au mers la Matana, de la Matana la Nahaliel, de la Nahaliel la Bamot;
Num 21:20  Iar de la Bamot la valea din câmpia Moabului, pe vârful muntelui Fazga, în fa?a pustiului.
Num 21:21  De acolo a trimis Moise soli la Sihon, regele Amoreilor, cu ve?ti de pace, ca sa i se spuna:
Num 21:22  "Da-mi voie sa trec prin ?ara ta. Nu ne vom abate nici la ogorul tau, nici la via ta, nici apa din fântâna ta nu vom bea, ci vom trece de hotarele tale!"
Num 21:23  Dar Sihon n-a îngaduit lui Israel sa treaca prin ?ara lui, ci ?i-a adunat tot poporul sau ?i a pa?it împotriva lui Israel în pustie, înaintând pâna la Iaha?, unde s-a luptat cu Israel.
Num 21:24  Însa Israel l-a batut, macelarindu-l cu sabia, ?i i-a cuprins ?ara de la Arnon pâna la Iaboc, pâna la fiii lui Amon, caci hotarele Amoni?ilor erau întarite.
Num 21:25  Luând toate ceta?ile acestea, Israel s-a a?ezat în ceta?ile Amoreilor: în He?bon ?i în toate satele care ?ineau de el.
Num 21:26  Caci He?bonul era cetatea lui Sihon, regele Amoreilor. Acesta se luptase cu fostul rege al Amoreilor ?i-i luase din mâini toata ?ara Amoreilor de la Aroer pâna la Arnon.
Num 21:27  De aceea ?i zic rapsozii în bataie de joc: "Veni?i la He?bon, ca sa se zideasca ?i sa se întareasca cetatea lui Sihon.
Num 21:28  Ca a ie?it foc din He?bon ?i para de foc din cetatea lui Sihon ?i a mistuit Ar-Moabul ?i pe stapânii mun?ilor Arnonului.
Num 21:29  Vai de tine, Moab! E?ti pierdut, poporul lui Camos! Feciorii lui s-au risipit ?i fetele lui au ajuns roabe la Sihon, regele Amoreilor.
Num 21:30  Tras-am asupra lor cu sage?i. De la He?bon pâna la Dibon tot este darâmat, am pustiit tot pâna la Nofa, care e aproape de Medeba".
Num 21:31  ?i a?a s-a a?ezat Israel în toate ceta?ile Amoreilor.
Num 21:32  De acolo a trimis Moise sa iscodeasca Iazerul, pe care l-a luat împreuna cu satele lui ?i a alungat pe Amoreii care locuiau acolo.
Num 21:33  ?i întorcându-se, a luat calea spre Vasan; iar Og, regele Vasanului, a ie?it înaintea lor cu tot poporul sau, ca sa se razboiasca la Edrei.
Num 21:34  Atunci a zis Domnul catre Moise: "Sa nu te temi de el, ca-l voi da în mâinile tale pe el ?i tot poporul lui ?i toata ?ara lui ?i vei face cu el cum ai facut cu Sihon, regele Amoreilor, care locuia în He?bon".
Num 21:35  ?i l-a batut pe el, pe fiii lui, ?i pe tot poporul lui, de n-a lasat viu nici pe unul din ai lui, ?i a cuprins ?ara lui.
Num 22:1  Purcezând apoi de acolo, fiii lui Israel au tabarât în ?esurile Moabului, lânga Iordan, în fa?a Ierihonului.
Num 22:2  Iar Balac, fiul lui Sefor, vazând toate câte facuse Israel Amoreilor,
Num 22:3  S-a înfrico?at foarte tare de poporul acesta, pentru ca era mult la numar, ?i s-au înspaimântat Moabi?ii de fiii lui Israel
Num 22:4  ?i au zis catre capeteniile Madiani?ilor: "Poporul acesta manânca acum totul împrejurul nostru, cum manânca boul iarba câmpului". Balac însa, feciorul lui Sefor, era atunci regele Moabi?ilor.
Num 22:5  Deci a trimis acesta soli la Valaam, fiul lui Beor, în Petor, care e a?ezat lânga râul Eufrat, în pamântul fiilor poporului sau, ca sa-l cheme ?i sa-i spuna: "Iata a ie?it un popor din Egipt ?i a acoperit fa?a pamântului ?i traie?te lânga mine.
Num 22:6  Vino deci ?i-mi blesteama poporul acesta, ca este mai tare decât mine, ?i atunci poate voi fi în stare sa-l bat ?i sa-l alung din ?ara. Eu ?tiu ca pe cine binecuvântezi tu acela este binecuvântat, ?i pe cine blestemi este blestemat".
Num 22:7  S-au dus deci batrânii Moabi?ilor ?i batrânii Madiani?ilor cu mâinile pline de daruri pentru vraji; ?i ajungând la Valaam, i-au spus vorbele lui Balac.
Num 22:8  Iar el le-a zis: "Ramâne?i aici peste noapte ?i va voi da raspuns cum îmi va spune Domnul". ?i au ramas capeteniile lui Moab la Valaam.
Num 22:9  Atunci a venit Dumnezeu la Valaam ?i a zis: "Cine sunt oamenii aceia de la tine?"
Num 22:10  Iar Valaam a zis catre Dumnezeu: "Balac, fiul lui Sefor, regele Moabului, i-a trimis la mine sa-mi spuna:
Num 22:11  Iata a ie?it din Egipt un popor ?i a acoperit fa?a pamântului ?i locuie?te lânga mine; vino dar de mi-l blesteama, doar l-a? putea birui ?i alunga din ?ara".
Num 22:12  Dumnezeu însa a zis catre Valaam: "Sa nu te duci cu ei ?i sa nu blestemi pe poporul acela, ca este binecuvântat".
Num 22:13  Diminea?a s-a sculat Valaam ?i a zis catre batrânii lui Balac: "Duce?i-va la stapânul vostru, ca nu ma lasa Dumnezeu sa merg cu voi".
Num 22:14  Sculându-se deci, capeteniile Moabului au venit la Balac ?i i-au spus: "Valaam n-a vrut sa vina cu noi".
Num 22:15  Atunci Balac a trimis al?i soli mai mul?i ?i mai însemna?i decât aceia.
Num 22:16  ?i venind ace?tia la Valaam, i-au zis: "A?a graie?te Balac al lui Sefor: Nu te lepada a veni pâna la mine;
Num 22:17  Ca î?i voi da cinste mare ?i-?i voi face toate câte-mi vei zice; vino însa ?i-mi blesteama poporul acesta".
Num 22:18  Iar Valaam a raspuns ?i a zis catre capeteniile lui Balac: "Chiar de mi-ar da Balac casa sa plina de argint ?i de aur, nu pot sa calc porunca Domnului Dumnezeului meu ?i sa fac ceva mic sau mare dupa placul meu.
Num 22:19  Ramâne?i însa acum ?i voi aici peste noapte ?i voi vedea ce-mi va mai spune Domnul".
Num 22:20  Atunci a venit Dumnezeu la Valaam noaptea ?i i-a zis: "Daca oamenii ace?tia au venit sa te cheme, scoala ?i te du cu ei; dar sa faci ceea ce-?i voi zice Eu!"
Num 22:21  A doua zi s-a sculat Valaam, ?i-a pus samarul pe asina sa ?i s-a dus cu capeteniile Moabului.
Num 22:22  Dar se aprinsese mânia lui Dumnezeu pentru ca s-a dus, iar îngerul Domnului s-a sculat, ca sa-l mustre pe cale.
Num 22:23  Cum ?edea el pe asina sa, înso?it de doua slugi ale sale, a vazut asina pe îngerul Domnului, care statea în drum cu sabia ridicata în mâna, ?i s-a abatut din drum pe câmp; iar Valaam a batut asina cu toiagul sau, ca sa o întoarca la drum.
Num 22:24  Dar îngerul Domnului a stat în drumul îngust între vii, unde de o parte ?i de alta era zid;
Num 22:25  ?i asina, vazând îngerul Domnului, s-a tras catre zid ?i a strâns piciorul lui Valaam în zid, ?i acesta iar a început s-o bata.
Num 22:26  Îngerul Domnului însa a trecut iar ?i a stat la loc strâmt, unde nu era chip sa te aba?i nici la dreapta, nici la stânga.
Num 22:27  Iar asina, vazând pe îngerul Domnului, s-a culcat sub Valaam. Atunci s-a mâniat Valaam ?i a început sa bata asina cu toiagul.
Num 22:28  Dar Domnul a deschis gura asinei ?i aceasta a zis catre Valaam: "Ce ?i-am facut eu, de ma ba?i acum pentru a treia oara?"
Num 22:29  ?i Valaam a zis catre asina: "Pentru ca ?i-ai râs de mine; de a? fi avut în mâna o sabie, te-a? fi ucis aici pe loc".
Num 22:30  Raspuns-a asina lui Valaam: "Au nu sunt eu asina ta, pe care ai umblat din tinere?ile tale ?i pâna în ziua aceasta? Avut-am oare deprinderea de a ma purta a?a cu tine?" ?i el a zis: "Nu!"
Num 22:31  Atunci a deschis Domnul ochii lui Valaam ?i acesta a vazut pe îngerul Domnului, care statea în mijlocul drumului cu sabia ridicata în mâna, ?i s-a închinat ?i a cazut cu fa?a la pamânt.
Num 22:32  Iar îngerul Domnului i-a zis: "De ce ai batut asina ta de trei ori? Eu am ie?it sa te împiedic, deoarece calea ta nu este dreapta înaintea mea;
Num 22:33  ?i asina, vazându-ma pe mine, s-a întors de la mine de trei ori pâna acum; daca ea nu s-ar fi întors de Ia mine, eu te-a? fi  ucis pe tine, iar pe ea a? fi lasat-o vie".
Num 22:34  Zis-a Valaam catre îngerul Domnului: "Am pacatuit, pentru ca n-am ?tiut ca stai tu în drum înaintea mea. Deci, daca aceasta nu este placut în ochii tai, atunci ma voi întoarce".
Num 22:35  Iar îngerul Domnului a zis catre Valaam: "Du-te cu  oamenii ace?tia, dar sa graie?ti ceea ce-?i voi spune eu!" ?i s-a dus Valaam cu capeteniile lui Balac.
Num 22:36  Când a auzit Balac ca vine Valaam, a ie?it în întâmpinarea lui în ora?ul moabit, care este lânga hotarul de la Arnon, chiar la hotar.
Num 22:37  ?i a zis Valaam catre Balac: "Iata acum am venit la tine. Dar pot eu, oare, sa-?i spun ceva?
Num 22:38  Ce-mi va pune Dumnezeu în gura, aceea î?i voi grai!"
Num 22:39  Apoi s-a dus Valaam cu Balac ?i au mers la Kiriat-Hu?ot.
Num 22:40  Atunci a junghiat Balac oi ?i boi ?i a trimis lui Valaam ?i capeteniilor ce erau cu el.
Num 22:41  Iar a doua zi de diminea?a, a luat Balac pe Valaam ?i l-a suit pe înal?imile lui Baal, ca sa-i arate de acolo o parte din popor.
Num 23:1  Atunci a zis Valaam catre Balac: "Zide?te-mi aici ?apte jertfelnice ?i pregate?te-mi ?apte vi?ei ?i ?apte berbeci".
Num 23:2  ?i a facut Balac dupa cum zisese Valaam: au ridicat Balac ?i Valaam câte un vi?el ?i câte un berbec pe fiecare jertfelnic.
Num 23:3  Apoi a zis Valaam catre Balac: "Stai lânga jertfa ta, iar eu ma duc, ca poate îmi va ie?i Domnul înainte ?i ce-mi va descoperi El aceea î?i voi spune". ?i a ramas Balac lânga jertfa sa, iar Valaam s-a dus într-un loc înalt sa întrebe pe Dumnezeu.
Num 23:4  ?i S-a aratat Dumnezeu lui Valaam ?i a zis Valaam catre El: "Am zidit ?apte jertfelnice ?i am suit câte un vi?el ?i câte un berbec pe fiecare jertfelnic".
Num 23:5  Iar Domnul a pus cuvânt în gura lui Valaam ?i a zis: "Întoarce-te la Balac ?i sa-i zici a?a!"
Num 23:6  ?i s-a întors la acesta ?i iata el statea la arderile de tot ale lui ?i toate capeteniile Moabului erau cu el. ?i a fost peste el Duhul Domnului ?i ?i-a rostit cuvântul sau, zicând:
Num 23:7  "Din Mesopotamia m-a adus Balac, regele Moabului, din mun?ii Rasaritului ?i mi-a zis: Vino ?i-mi blesteama pe Iacov, vino ?i osânde?te pe Israel!
Num 23:8  Cum sa blestem pe cel ce nu-l blesteama Dumnezeu? Sau cum sa osândesc pe cel ce nu-l osânde?te Dumnezeu?
Num 23:9  De pe vârful muntelui ma uit la el ?i de pe dealuri îl privesc. Iata un popor care traie?te deosebi ?i nu se numara cu alte popoare.
Num 23:10  Cine va numara pe urma?ii lui Iacov ?i gloatele din Israel cine le va socoti? Sa moara sufletul meu moartea drep?ilor acestora ?i sa fie sfâr?itul meu ca sfâr?itul lor!"
Num 23:11  Atunci a zis Balac catre Valaam: "Ce mi-ai facut? Eu te-am adus sa-mi blestemi pe vrajma?ii mei ?i iata tu îi binecuvântezi!"
Num 23:12  Valaam însa a zis catre Balac: "Oare sa nu spun eu lui Balac ceea ce-mi pune Domnul în gura?"
Num 23:13  Iar Balac a zis catre el: "Vino cu mine în alt loc, de unde nu-l vei vedea tot, ci numai o parte din el vei vedea, iar tot nu-l vei vedea: sa mi-l blestemi de acolo".
Num 23:14  ?i l-a dus pe el la locul de straja, pe vârful muntelui Fazga, ?i a zidit acolo ?apte jertfelnice ?i a pus câte un vi?el ?i câte un berbec pe fiecare jertfelnic.
Num 23:15  ?i a zis Valaam catre Balac: "Stai aici lânga jertfa ta, iar eu ma duc sa întreb pe Dumnezeu!"
Num 23:16  Atunci a întâmpinat Dumnezeu pe Valaam ?i a pus cuvânt în gura lui ?i a zis: "Întoarce-te la Balac ?i sa-i graie?ti acestea".
Num 23:17  ?i s-a întors la el ?i statea la jertfa sa cu toate capeteniile Moabului. ?i l-a întrebat Balac: "Ce ?i-a spus Domnul?"
Num 23:18  Iar el ?i-a rostit cuvântul sau ?i a zis: "Scoala ?i asculta Balac! Ia aminte la mine, fiul lui Sefor!
Num 23:19  Dumnezeu nu este ca omul, ca sa-L min?i, nici ca fiul omului, ca sa-I para rau. Au zice-va El ?i nu va face? Sau va grai ?i nu va împlini?
Num 23:20  Iata am primit porunca sa binecuvântez; El a binecuvântat ?i eu nu pot întoarce binecuvântarea.
Num 23:21  El nu vede nedreptate în Iacov ?i nu zare?te silnicie în Israel; Domnul Dumnezeul sau este cu el ?i în mijlocul lui se aude strigat de veselie ca pentru un împarat.
Num 23:22  Dumnezeu l-a scos din Egipt, puterea lui este ca a unui taur.
Num 23:23  Pentru ca nu este vrajitorie în Iacov, nici farmece în Israel, la vreme se va spune lui Iacov ?i lui Israel; cele ce vrea sa plineasca Dumnezeu!
Num 23:24  Iata un popor care se ridica asemenea unei leoaice ?i ca un leu se scoala, care nu se culca pâna n-a sfâ?iat prada ?i pâna n-a baut sângele uci?ilor!"
Num 23:25  Zis-a Balac catre Valaam: "Nici de blestemat sa nu-l blestemi, nici de binecuvântat sa nu-l binecuvântezi".
Num 23:26  Iar Valaam a raspuns ?i a zis catre Balac: "Nu ?i-am grait eu, oare, ca voi face ce-mi va spune Domnul?"
Num 23:27  Atunci a zis Balac catre Valaam: "Hai sa te duc în alt loc: poate-I va placea lui Dumnezeu ?i mi-I vei blestema de acolo".
Num 23:28  ?i a luat Balac pe Valaam pe vârful lui Peor, care prive?te spre pustie.
Num 23:29  Aici Valaam a zis catre Balac: "Zide?te-mi ?apte jertfelnice ?i pregate?te-mi ?apte vi?ei ?i ?apte berbeci",
Num 23:30  ?i a facut Balac, cum a zis Valaam, ?i a pus câte un vi?el ?i câte un berbec pe fiecare jertfelnic.
Num 24:1  Vazând Valaam ca Domnul binevoie?te sa se binecuvânteze Israel, n-a mai alergat dupa obicei la vrajitorii, ci s-a întors cu fa?a spre pustie;
Num 24:2  ?i ridicându-?i Valaam ochii sai, a vazut pe Israel a?ezat dupa semin?iile sale ?i a venit peste dânsul duhul lui Dumnezeu,
Num 24:3  ?i ?i-a rostit el cuvântul sau, zicând: "A?a zice Valaam, fiul lui Beor; a?a graie?te barbatul cel ce vede cu adevarat;
Num 24:4  A?a glasuie?te cel ce asculta cuvântul lui Dumnezeu, cel ce cunoa?te gândurile Celui Atotputernic, cel ce vede descoperirile lui Dumnezeu, ca în vis, dar ochii ?i-i are deschi?i:
Num 24:5  Cât sunt de frumoase sala?urile tale, Iacove, corturile tale, Israele!
Num 24:6  Se desfa?oara ca ni?te vai, ca ni?te gradini pe lânga râuri, ca ni?te cedri pe lânga ape, ca ni?te corturi pe care le-a înfipt Domnul!
Num 24:7  Ie?i-va din samân?a lui un Om, care va stapâni neamuri multe ?i stapânirea Lui va întrece pe a lui Agag ?i împara?ia Lui se va înal?a.
Num 24:8  Dumnezeu l-a scos din Egipt ?i puterea lui va fi ca a taurului; mânca-va popoarele du?mane lui, va sfarâma oasele lor ?i cu sage?ile sale va sageta pe vrajma?i.
Num 24:9  Plecatu-s-a ?i s-a culcat ca un leu ?i ca o leoaica; cine-l va scula? Cel ce te binecuvânteaza, binecuvântat sa fie, ?i cel ce te blesteama sa fie blestemat!"
Num 24:10  Atunci s-a mâniat Balac pe Valaam ?i, frângându-?i mâinile, a zis Balac catre Valaam: "Eu te-am chemat sa-mi blestemi pe vrajma?ii mei, iar tu, iata, i-ai binecuvântat de trei ori pâna acum.
Num 24:11  Fugi dar în ?ara ta! Am zis ca te voi cinsti; dar iata ca Domnul te-a lipsit de cinste".
Num 24:12  Valaam însa a zis catre Balac: "N-am spus eu oare solilor tai pe care i-ai trimis la mine:
Num 24:13  Chiar de mi-ar da Balac casa sa plina de argint ?i de aur, nu voi putea sa calc porunca Domnului, ca sa fac ceva bun sau rau dupa placul meu; câte-mi va spune Domnul, acelea le voi grai?
Num 24:14  Deci, iata, ma duc repede în ?ara mea; dar vino sa-?i spun ce are sa faca poporul acesta cu poporul tau în vremurile viitoare".
Num 24:15  ?i ?i-a urmat Valaam cuvântul sau ?i a zis: "A?a graie?te Valaam, fiul lui Beor; a?a graie?te barbatul cel ce vede cu adevarat,
Num 24:16  Cel ce asculta cuvintele lui Dumnezeu, cel ce are ?tiin?a de la Cel Preaînalt ?i vede descoperirile lui Dumnezeu, ca în vis, dar ochii îi sunt deschi?i:
Num 24:17  Îl vad, dar acum înca nu este; îl privesc, dar nu de aproape; o stea rasare din Iacov; un toiag se ridica din Israel ?i va lovi pe capeteniile Moabului ?i pe to?i fiii lui Set îi va zdrobi.
Num 24:18  Lua-va de mo?tenire pe Edom ?i va stapâni Seirul vrajma?ilor sai ?i Israel î?i va arata puterea.
Num 24:19  Din Iacov se va scula Cel ce va stapâni cu putere ?i va pierde pe cei ce vor ramâne în cetate".
Num 24:20  Apoi vazând pe Amalec, ?i-a urmat cuvântul ?i a zis: "Cel întâi dintre popoare e Amalec, dar ?i neamul lui va pieri".
Num 24:21  Vazând dupa aceea pe Chenei, ?i-a urmat cuvântul ?i a zis: "Locuin?a ta e tare ?i cuibul tau e a?ezat pe stânca;
Num 24:22  Dar Cain va fi darâmat ?i nu este mult pâna ce Asur te va duce în robie".
Num 24:23  Iar când a vazut pe Og, ?i-a urmat cuvântul ?i a zis: "Vai, vai, cine va mai trai când Dumnezeu va aduce acestea!
Num 24:24  Veni-vor corabii de la Chitim ?i vor smeri pe Asur, vor smeri pe Heber, dar ?i acelea vor pieri".
Num 24:25  Sculându-se apoi Valaam s-a întors înapoi, în ?ara sa; ?i s-a dus ?i Balac întru ale sale.
Num 25:1  Atunci s-a a?ezat Israel în Sitim, dar poporul a început sa se spurce, pacatuind cu fetele din Moab.
Num 25:2  Ca acestea îi chemau la jertfele idolilor lor ?i mânca poporul din acele jertfe ?i se închina la dumnezeii lor.
Num 25:3  A?a s-a lipit Israel de Baal-Peor, pentru care s-a aprins mânia lui Dumnezeu asupra lui Israel.
Num 25:4  ?i a zis Domnul catre Moise: "Ia pe toate capeteniile poporului ?i le spânzura de copaci pentru Domnul înainte de asfin?itul soarelui, ca sa se abata de la Israel iu?imea mâniei Domnului".
Num 25:5  Atunci a zis Moise catre judecatorii lui Israel: "Ucide?i fiecare pe oamenii vo?tri care s-au lipit de Baal-Peor".
Num 25:6  Dar iata oarecare din fiii lui Israel a venit ?i a adus între fra?ii sai o madianita, în ochii lui Moise ?i în ochii întregii ob?ti a fiilor lui Israel, când plângeau ei la u?a cortului adunarii.
Num 25:7  Atunci Finees, fiul lui Eleazar, fiul preotului Aaron, vazând aceasta, s-a sculat din mijlocul ob?tii ?i, luând în mâna lancea sa,
Num 25:8  A intrat dupa israelit în sala? ?i i-a strapuns pe amândoi, pe israelit ?i pe femeie, în pântece; ?i a încetat pedepsirea fiilor lui Israel.
Num 25:9  Cei ce au murit de pedeapsa aceasta au fost douazeci ?i patru de mii.
Num 25:10  ?i a grait Domnul cu Moise ?i a zis:
Num 25:11  "Finees, feciorul lui Eleazar, fiul preotului Aaron, a abatut mânia Mea de la fiii lui Israel, râvnind între ei pentru Mine, ?i n-am mai pierdut pe fiii lui Israel în mânia Mea;
Num 25:12  De aceea spune-i ca voi încheia cu el legamântul Meu de pace,
Num 25:13  ?i va fi pentru el ?i pentru urma?ii lui de dupa el legamânt de preo?ie ve?nica, caci a aratat râvna pentru Dumnezeul sau ?i a ispa?it pacatul fiilor lui Israel".
Num 25:14  Numele israelitului ucis, care a fost omorât cu madianita, era Zimri, fiul lui Salu, capetenia semin?iei lui Simeon;
Num 25:15  Iar numele madianitei ucise era Cozbi, fiica lui ?ur, capetenia unei semin?ii ie?ita dintr-o casa patriarhala din Madian.
Num 25:16  ?i a grait Domnul cu Moise ?i a zis: "Vorbe?te fiilor lui Israel ?i le zi:
Num 25:17  Socoti?i pe Madiani?i du?manii vo?tri ?i omorâ?i-i, bate?i-i, ca s-au purtat cu voi du?manos întru vicle?ugul lor,
Num 25:18  Ademenindu-va cu Peor ?i cu Cozbi, sora lor, fiica unei capetenii a Madiani?ilor, care a fost ucisa în ziua urgiei celei pentru Peor".
Num 26:1  Dupa aceasta pedeapsa, a grait Domnul catre Moise ?i catre Eleazar, fiul preotului Aaron, ?i a zis:
Num 26:2  "Numara?i toata ob?tea fiilor lui Israel de la douazeci de ani în sus, pe to?i cei buni de razboi în Israel, dupa familiile lor!"
Num 26:3  Atunci Moise ?i preotul Eleazar le-au grait în ?esurile Moabului, la Iordan, în dreptul Ierihonului, ?i le-au zis:
Num 26:4  "Numara?i pe to?i de la douazeci de ani în sus", cum a grait Domnul lui Moise ?i fiilor lui Israel, care au ie?it din pamântul Egiptului.
Num 26:5  Ruben, este întâiul nascut al lui Israel. Fiii lui Ruben: din Enoh, neamul lui Enoh; din Falu, neamul lui Falu;
Num 26:6  Din He?ron, neamul lui He?ron; din Carmi, neamul lui Carmi;
Num 26:7  Acestea sunt neamurile lui Ruben; ?i s-au numarat patruzeci ?i trei de mii ?apte sute treizeci.
Num 26:8  Fiul lui Falu: Eliab.
Num 26:9  Fiii lui Eliab: Nemuel, Datan ?i Abiron. Datan ?i Abiron sunt aceia care, chema?i fiind în adunare, au stârnit razvratire împotriva lui Moise ?i a lui Aaron împreuna cu parta?ii lui Core, când ace?tia au stârnit razvratire împotriva Domnului
Num 26:10  ?i ?i-a deschis pamântul gura sa ?i i-a înghi?it pe ei ?i pe Core; ?i împreuna cu ei au murit ?i parta?ii lor, când focul a mistuit doua sute cincizeci de oameni ?i au ramas ei ca semn.
Num 26:11  Însa fiii lui Core n-au murit.
Num 26:12  Fiii lui Simeon, dupa familiile lor, sunt: din Nemuel, neamul lui Nemuel; din Iamin, neamul lui Iamin; din Iachin; neamul lui Iachin;
Num 26:13  Din Zerah, neamul lui Zerah; din Saul, neamul lui Saul.
Num 26:14  Acestea sunt neamurile cele din Simeon, care s-au gasit la numaratoarea lor: douazeci ?i doua de mii doua sute.
Num 26:15  Fiii lui Gad, dupa neamurile lor: din ?efon neamul ?efonienilor; din Haghi, neamul Haghi?ilor; din ?unie, neamul ?unienilor;
Num 26:16  Din Ozni, neamul Oznienilor;
Num 26:17  Din Eri, neamul Erienilor; din Arod, neamul Arodeilor; din Areli, neamul Arelienilor.
Num 26:18  Acestea sunt neamurile fiilor lui Gad, care la numaratoarea lor s-au gasit patruzeci de mii cinci sute.
Num 26:19  Fiii lui Iuda sunt: Er ?i Onan, ?ela, Fares ?i Zara; însa Er ?i Onan au murit în pamântul Canaanului.
Num 26:20  ?i fiii lui Iuda, dupa neamurile lor, sunt: din ?ela, neamul ?elaenilor; din Fares, neamul Fareseilor; din Zara, neamul Zaraenilor.
Num 26:21  Iar fiii lui Fares sunt: din Esron neamul Esroneilor; din Hamul, neamul Hamulienilor.
Num 26:22  Acestea sunt neamurile din Iuda ?i la numaratoarea lor s-au gasit ?aptezeci ?i ?ase de mii cinci sute.
Num 26:23  Fiii lui Isahar, dupa neamurile lor, sunt: din Tola, neamul Tolaenilor; din Fuva, neamul Fuvaenilor;
Num 26:24  Din Ia?ub, neamul Ia?ubienilor; din ?imron, neamul ?imronienilor.
Num 26:25  Acestea sunt neamurile din Isahar ?i la numaratoarea lor s-au gasit ?aizeci ?i patru de mii trei sute.
Num 26:26  Fiii lui Zabulon, dupa neamurile lor, sunt: din Sered, neamul Seredienilor; din Elon, neamul Elonienilor; din Iahleil, neamul Iahleililor.
Num 26:27  Acestea sunt neamurile din Zabulon ?i la numaratoarea lor s-au gasit ?aizeci de mii cinci sute.
Num 26:28  Fiii lui Iosif sunt: Manase ?i Efraim.
Num 26:29  Fiii lui Manase, dupa neamurile lor, sunt: din Machir, neamul Machirienilor; din Machir s-a nascut Galaad ?i din Galaad este neamul Galaadenilor.
Num 26:30  Fiii lui Galaad sunt: din Iezer, neamul Iezerienilor; din Helec, neamul Helecienilor;
Num 26:31  Din Asriel, neamul Asrielienilor; din ?echem, neamul ?echemienilor;
Num 26:32  Din ?emida, neamul ?emidienilor; din Hefer, neamul Heferienilor.
Num 26:33  Salfaad, fiul lui Hefer, n-a avut fii, ci numai fiice ?i numele fiicelor lui Salfaad sunt: Mahla, Noa, Hogla, Milca ?i Tir?a.
Num 26:34  Acestea sunt neamurile lui Manase ?i la numaratoarea lor s-au gasit cincizeci ?i doua de mii ?apte sute.
Num 26:35  Fiii lui Efraim, dupa neamurile lor, sunt: din ?utelah, neamul ?utelahienilor; din Becher, neamul Becherienilor; din Tahan, neamul Tahanienilor.
Num 26:36  Iar fiii lui ?utelah sunt: din Eran, neamul Eranienilor.
Num 26:37  Acestea sunt neamurile fiilor lui Efraim ?i la numaratoarea lor s-au gasit treizeci ?i doua de mii cinci sute. Ace?tia sunt fiii lui Iosif dupa neamurile lor.
Num 26:38  Fiii lui Veniamin, dupa neamurile lor, sunt; din Bela, neamul Belaienilor; din A?bel, neamul A?belienilor; din Ahiram, neamul Ahiramienilor;
Num 26:39  Din ?efufam, neamul ?efufamienilor; din Hufam, neamul Hufamienilor.
Num 26:40  Iar fiii lui Bela sunt: Ard ?i Naaman: din Ard, neamul Ardienilor ?i din Naaman, neamul Naamanienilor.
Num 26:41  Ace?tia sunt fiii lui Veniamin dupa neamurile lor ?i la numaratoare s-au gasit patruzeci ?i cinci de mii ?ase sute.
Num 26:42  Fiii lui Dan, dupa neamurile lor, sunt: din ?uham, neamul ?uhamienilor. Acestea sunt familiile lui Dan, dupa neamurile lor.
Num 26:43  ?i neamurile lui ?uham, la numaratoarea lor, au fost de toate ?aizeci ?i patru de mii patru sute.
Num 26:44  Fiii lui A?er, dupa neamurile lor, sunt: din Imna, neamul Imnaenilor; din I?ba, neamul I?baenilor; din Verie, neamul Verienilor.
Num 26:45  Din fiii lui Verie: din Heber, neamul Heberienilor; din Malchiel, neamul Malchielilor.
Num 26:46  ?i numele fiicei lui A?er a fost Serah.
Num 26:47  Acestea sunt neamurile fiilor lui A?er ?i la numaratoare s-au gasit cincizeci ?i trei de mii patru sute.
Num 26:48  Fiii lui Neftali, dupa neamurile lor, sunt: din Iah?eel, neamul Iah?eelienilor; din Guni, neamul Gunienilor;
Num 26:49  Din Ie?er, neamul Ie?erienilor; din ?ilem, neamul ?ilemienilor.
Num 26:50  Acestea sunt neamurile lui Neftali, dupa familiile lor ?i la numaratoare s-au gasit patruzeci ?i cinci de mii patru sute.
Num 26:51  Iata numarul fiilor lui Israel, celor ce au intrat la numaratoare: ?ase sute una mii ?apte sute treizeci.
Num 26:52  Apoi a grait Domnul cu Moise ?i a zis:
Num 26:53  "Acestora sa li se împarta spre mo?tenire pamântul, dupa numarul numelor;
Num 26:54  Celor mai mul?i sa le dai mo?ie mai mare, iar celor mai pu?ini sa le dai mo?ie mai mica; fiecaruia  sa se dea mo?ie potrivit cu numarul celor ce au intrat la numaratoare.
Num 26:55  Pamântul sa-l împar?i?i prin sor?i; dupa numele semin?iilor parin?ilor lor sa-?i primeasca ?i par?ile:
Num 26:56  Prin sor?i sa le împar?i mo?ia, atât celor mul?i la numar, cât ?i celor pu?ini la numar".
Num 26:57  Levi?ii, care au intrat la numaratoare, dupa neamurile lor, sunt ace?tia: din Gher?on, neamul Gher?onienilor; din Cahat, neamul Cahatienilor, din Merari, neamul Merarienilor.
Num 26:58  Iata neamurile lui Levi: neamul lui Libni, neamul lui Hebron, neamul lui Mahli, neamul lui Mu?i ?i neamul lui Core. Din Cahat s-a nascut Amram.
Num 26:59  Numele femeii lui Amram a fost Iohabed, fiica lui Levi, pe care a nascut-o femeia lui Levi în Egipt, iar ea a nascut lui Amram pe Aaron, pe Moise ?i pe Mariam, sora lor.
Num 26:60  Lui Aaron i s-au nascut: Nadab ?i Abiud, Eleazar ?i Itamar.
Num 26:61  Dar Nadab ?i Abiud au murit când au adus foc strain înaintea Domnului, în pustiul Sinai.
Num 26:62  ?i s-au numarat to?i cei de parte barbateasca de la o luna în sus ?i s-au gasit douazeci ?i trei de mii; caci ace?tia nu fusesera numara?i împreuna cu fiii lui Israel, pentru ca nu li s-a dat mo?tenire printre fiii lui Israel.
Num 26:63  Ace?tia sunt cei numara?i de Moise ?i Eleazar preotul, care au numarat pe fiii lui Israel în ?esurile Moabului, lânga Iordan, în dreptul Ierihonului.
Num 26:64  În numarul lor nu se afla niciunul din fiii lui Israel numara?i de Moise ?i de preotul Aaron, în pustiul Sinai,
Num 26:65  Caci Domnul le zisese acestora ca vor muri to?i în pustie, - ?i n-au ramas din ei niciunul, afara de Caleb, fiul lui Iefone ?i de Iosua, fiul lui Navi.
Num 27:1  Atunci au venit fetele lui Salfaad, fiul lui Hefer, fiul lui Galaad, fiul lui Machir, din neamul lui Manase, fiul lui Iosif, ale caror nume sunt acestea: Mahla, Noa, Hogla, Milca ?i Tir?a,
Num 27:2  ?i au stat înaintea lui Moise, a lui Eleazar preotul, înaintea capeteniilor ?i înaintea întregii ob?ti, la u?a cortului adunarii, ?i au zis:
Num 27:3  "Tatal nostru a murit în pustie; el n-a fost din numarul celor care s-au ridicat împotriva Domnului cu adunarea lui Core, ci a murit pentru pacatul sau ?i feciori n-a avut.
Num 27:4  De ce sa piara numele tatalui nostru din neamul lui, pentru ca n-are fii? Da-ne ?i noua mo?ie între fra?ii tatalui nostru!"
Num 27:5  Moise însa a adus cererea lor înaintea Domnului;
Num 27:6  Iar Domnul a zis catre Moise:
Num 27:7  "Drept au grait fetele lui Salfaad; daruie?te-le ?i lor mo?tenire între fra?ii tatalui lor ?i trece-le lor mo?ia tatalui lor.
Num 27:8  Iar fiilor lui Israel sa le graie?ti ?i sa le spui: De va muri cineva, neavând fiu, sa da?i partea lui fiicei lui.
Num 27:9  Iar de nu are nici fiica, sa da?i partea lui fra?ilor lui.
Num 27:10  De nu are însa nici fra?i, sa da?i partea lui fra?ilor tatalui lui.
Num 27:11  Iar de nu are tatal sau fra?i, sa da?i partea lui rudeniei celei mai de aproape din neamul lui, ca sa mo?teneasca ale lui. Aceasta sa fie pentru fiii lui Israel ca o hotarâre din lege, cum a poruncit Domnul lui Moise".
Num 27:12  Apoi a zis Domnul catre Moise: "Suie-te pe acest munte, care este dincoace de Iordan, adica pe muntele Nebo, ?i prive?te pamântul Canaanului, pe care am sa-l dau fiilor lui Israel de mo?tenire.
Num 27:13  Iar dupa ce-l vei vedea, te vei adauga ?i tu la poporul tau, cum s-a adaugat Aaron, fratele tau, pe muntele Hor.
Num 27:14  Pentru ca v-a?i împotrivit poruncii Mele în pustiul Sinai, în vremea tulburarii ob?tii, ca sa arata?i înaintea ochilor lor sfin?enia Mea la ape, adica la apele Meriba de la Cade?, în pustiul Sinai".
Num 27:15  Moise însa a grait Domnului ?i a zis:
Num 27:16  "Domnul Dumnezeul duhurilor ?i a tot trupul sa rânduiasca peste ob?tea aceasta un om,
Num 27:17  Care sa iasa înaintea ei ?i care sa intre înaintea ei, care sa-i duca ?i sa-i aduca, ca sa nu ramâna ob?tea Domnului ca oile ce n-au pastor".
Num 27:18  Iar Domnul a zis catre Moise: "Ia-?i pe Iosua, fiul lui Navi, om cu duh într-însul, pune-?i peste el mâna ta;
Num 27:19  Apoi du-l înaintea preotului Eleazar, înaintea a toata ob?tea ?i da-i pove?e înaintea ochilor lor;
Num 27:20  Da-i din slava ta, ca sa-l asculte toata ob?tea fiilor lui Israel.
Num 27:21  Dupa aceea sa stea înaintea preotului Eleazar ?i acesta va întreba de hotarârile Domnului prin ajutorul Urimului: dupa cuvântul acestuia sa iasa ?i dupa cuvântul acestuia sa intre el ?i to?i fiii lui Israel cei împreuna cu dânsul ?i toata ob?tea".
Num 27:22  ?i a facut Moise cum i-a poruncit Domnul Dumnezeu: a luat pe Iosua ?i l-a pus înaintea preotului Eleazar ?i a toata ob?tea.
Num 27:23  Apoi ?i-a pus peste el mâinile sale ?i i-a dat pove?e, cum zisese Domnul prin Moise.
Num 28:1  Apoi iara?i a grait Domnul cu Moise ?i a zis:
Num 28:2  "Porunce?te fiilor lui Israel ?i le spune: Darurile Mele, darile Mele, jertfele Mele cele întru miros cu buna mireasma, îngriji?i sa Mi se aduca la sarbatorile Mele.
Num 28:3  Spune-le: Iata jertfele care trebuie sa le aduce?i Domnului: doi miei de câte un an fara meteahna, ardere de tot necontenita, pe fiecare zi;
Num 28:4  Un miel sa-l aduci diminea?a ?i pe celalalt miel sa-l aduci seara.
Num 28:5  Jertfa de pâine sa aduci a zecea parte de efa de faina de grâu, amestecata cu un sfert de hin de untdelemn;
Num 28:6  Aceasta este ardere de tot necontenita ?i care a fost savâr?ita la Muntele Sinai, spre miros cu buna mireasma ?i ca jertfa Domnului.
Num 28:7  La ea sa aduci turnare un sfert de hin de vin la un miel; ?i turnarea de vin a Domnului s-o torni la loc sfânt.
Num 28:8  Celalalt miel sa-l aduci spre seara, cu darul lui de pâine ?i cu turnarea lui sa-l aduca jertfa, mireasma placuta Domnului.
Num 28:9  Iar în ziua odihnei sa aduce?i doi miei de câte un an, fara meteahna, ?i ca jertfa doua zecimi de efa de faina de grâu, framântata cu untdelemn ?i cu turnarea ei.
Num 28:10  Aceasta este ardere de tot pentru ziua odihnei afara de arderea de tot cea necontenita cu turnarea ei.
Num 28:11  La începutul lunilor voastre sa aduce?i Domnului ardere de tot: din cireada, doi vi?ei, iar din turma, un berbec ?i ?apte miei de câte un an fara meteahna.
Num 28:12  Iar ca dar de pâine, câte trei zecimi de efa faina de grâu, framântata cu untdelemn, la fiecare vi?el, ?i doua zecimi de efa faina de grâu, framântata cu untdelemn, ca dar de pâine la berbec,
Num 28:13  ?i câte o zecime de efa faina de grâu, framântata cu untdelemn, ca dar de pâine la fiecare miel. Aceasta este ardere de tot, mireasma placuta, jertfa Domnului.
Num 28:14  Turnare la ele sa fie jumatate hin de vin de fiecare vi?el, a treia parte hin de berbec ?i un sfert de hin la fiecare miel. Aceasta este ardere de tot pentru fiecare început de luna, la toate lunile anului.
Num 28:15  Sa mai aduce?i Domnului ?i un ?ap, jertfa pentru pacat, afara de arderea de tot cea necontenita, ?i sa-l aduca cu turnarea lui.
Num 28:16  În ziua a paisprezecea a lunii întâi sunt Pa?tile Domnului.
Num 28:17  În ziua a cincisprezecea este sarbatoare. ?apte zile sa mânca?i azime.
Num 28:18  În ziua întâi sa ave?i adunare sfânta ?i nici un fel de lucru sa nu face?i;
Num 28:19  ?i sa aduce?i Domnului jertfa, ardere de tot: din cireada, doi vi?ei, iar din turma, un berbec ?i ?apte miei de câte un an; ace?tia sa fie fara meteahna.
Num 28:20  Cu ei sa aduce?i dar de pâine, faina de grâu framântata cu untdelemn, trei zecimi de efa de fiecare vi?el, doua zecimi de efa la berbec,
Num 28:21  ?i câte o zecime de efa sa aduci cu fiecare din cei ?apte miei;
Num 28:22  Sa aduce?i un ?ap jertfa pentru pacat, pentru cura?irea voastra.
Num 28:23  Acestea sa le aduce?i, pe lânga arderea de tot de diminea?a, care este ardere de tot necontenita.
Num 28:24  Tot a?a sa aduce?i ?i în fiecare din cele ?apte zile: pâine, jertfa, mireasma placuta Domnului, pe lânga arderea de tot cea necontenita ?i cu turnarea ei.
Num 28:25  În ziua a ?aptea sa ave?i adunare sfânta ?i nici un lucru sa nu lucra?i.
Num 28:26  În ziua celor dintâi roade, când aduce?i Domnului prinosul nou de pâine, la încheierea saptamânilor, sa ave?i adunare sfânta ?i nici un lucru sa nu lucra?i.
Num 28:27  Sa aduce?i ardere de tot spre miros de buna mireasma Domnului: din cireada, doi vitei, iar din turma, un berbec ?i ?apte miei de câte un an fara meteahna.
Num 28:28  Cu ei sa aduce?i dar de pâine, faina de grâu framântata cu untdelemn: trei zecimi de efa la fiecare vi?el, doua zecimi de efa la berbec
Num 28:29  ?i o zecime de efa la fiecare din cei ?apte miei.
Num 28:30  Sa aduce?i un ?ap jertfa pentru pacat, spre cura?irea voastra.
Num 28:31  Acestea sa mi le aduce?i cu turnarile lor, afara de arderile de tot neîncetate cu darul lor de pâine, care se aduc de obicei; acestea trebuie sa fie curate".
Num 29:1  "În ziua întâi a lunii a ?aptea de asemenea sa ave?i adunare sfânta, ?i nici un lucru sa nu lucra?i; pe aceasta sa o socoti?i o zi a suflatului în trâmbi?e.
Num 29:2  Sa aduce?i ardere de tot, spre miros placut Domnului: un vi?el, un berbec, ?apte miei de câte un an fara meteahna;
Num 29:3  La ei, ca dar de pâine, faina de grâu, framântata cu untdelemn: trei zecimi de efa la vi?el, doua zecimi de efa la berbec
Num 29:4  ?i câte o zecime de efa la fiecare din cei ?apte miei.
Num 29:5  Din turma de capre sa aduce?i un ?ap, jertfa pentru pacat, spre cura?irea voastra.
Num 29:6  Acestea sa le aduce?i jertfe pe lânga arderea de tot, cu darul de pâine ?i turnarea lui de la luna noua ?i pe lânga arderea de tot necontenita cu darul ei de pâine ?i turnarea lui, dupa rânduiala, întru miros de buna mireasma Domnului.
Num 29:7  În ziua a zecea a acestei luni sa ave?i adunare sfânta, sa posti?i ?i nici un lucru sa nu face?i.
Num 29:8  Sa aduce?i ardere de tot Domnului spre miros de buna mireasma: un vi?el" un berbec ?i ?apte miei de câte un an.
Num 29:9  La ei sa aduce?i dar de pâine, faina de grâu framântata cu untdelemn, trei zecimi de efa la vi?el, doua zecimi de efa la berbec
Num 29:10  ?i câte o zecime de efa la fiecare din cei ?apte miei.
Num 29:11  Iar din turma de capre sa aduce?i un ?ap jertfa pentru pacat, spre cura?irea voastra; acestea pe linga jertfa pentru pacat din ziua cura?irii ?i pe lânga arderea de tot cea necontenita cu darul ei de pâine ?i turnarea ei, care se aduce dupa rânduiala jertfa Domnului, spre miros cu buna mireasma.
Num 29:12  În ziua a cincisprezecea a lunii a ?aptea sa ave?i iar adunare sfânta; nici un lucru sa nu lucra?i ?i sa sarbatori?i sarbatoarea Domnului ?apte zile.
Num 29:13  În ziua întâi sa aduce?i ardere de tot, jertfa, mireasma placuta Domnului: din cireada, treisprezece vi?ei, iar din turma, doi berbeci, paisprezece miei de câte un an; dar sa fie fara meteahna.
Num 29:14  Cu ei, ca dar de pâine, sa se aduca faina de grâu framântata cu untdelemn: trei zecimi de efa cu fiecare din cei treisprezece vilei, doua zecimi de efa cu fiecare din cei doi berbeci,
Num 29:15  ?i câte o zecime de efa de fiecare din cei paisprezece miei;
Num 29:16  Iar din turma de capre, un ?ap, jertfa pentru pacat, peste arderea de tot necontenita ?i darul ei de pâine cu turnarea ei.
Num 29:17  A doua zi sa se aduca doisprezece vitei, doi berbeci, paisprezece miei de câte un an, fara meteahna;
Num 29:18  Cu ei sa se aduca dar de pâine ?i turnare: la vilei, la berbeci ?i la miei, dupa numarul lor, cum e rânduit;
Num 29:19  Iar din turma de capre, un ?ap, jertfa pentru pacat; acestea sa le aduce?i în afara de arderea de tot necontenita ?i de darul de pâine cu turnarea ei.
Num 29:20  A treia zi sa aduce?i unsprezece vilei, doi berbeci ?i paisprezece miei de câte un an, fara meteahna;
Num 29:21  ?i cu ei dar de pâine ?i turnare pentru vitei, pentru berbeci ?i pentru miei, dupa numarul lor, dupa rânduiala;
Num 29:22  Iar din turma de capre sa aduce?i un ?ap, jertfa pentru pacat, peste arderea de tot necontenita cu darul de pâine ?i turnarea ei.
Num 29:23  A patra zi sa aduce?i zece vilei, doi berbeci ?i paisprezece miei de câte un an, fara meteahna;
Num 29:24  Cu ei sa aduce?i dar de pâine ?i turnare pentru vitei, pentru berbeci ?i pentru miei, dupa numarul lor, cum e rânduiala;
Num 29:25  Iar din turma de capre sa aduce?i un ?ap, jertfa pentru pacat, pe lânga arderea de tot necontenita cu darul de pâine ?i turnarea ei.
Num 29:26  În ziua a cincea sa aduce?i noua vitei, doi berbeci ?i paisprezece miei de câte un an, fara meteahna;
Num 29:27  ?i cu ei dar de pâine ?i turnare pentru vilei, pentru berbeci ?i pentru miei, dupa numarul lor, cum e rânduiala;
Num 29:28  Iar din turma de capre sa aduce?i un ?ap, jertfa pentru pacat, peste arderea necontenita cu darul de pâine ?i cu turnarea ei.
Num 29:29  În ziua a ?asea sa aduce?i opt vitei, doi berbeci ?i paisprezece miei, fara meteahna;
Num 29:30  ?i eu ei dar de pâine ?i turnare pentru vilei, pentru berbeci ?i pentru miei, dupa numarul lor, cum e rânduiala;
Num 29:31  Iar din turma de capre, un ?ap, jertfa pentru pacat, peste arderea de tot necontenita cu darul de pâine ?i cu turnarea ei.
Num 29:32  În ziua a ?aptea sa aduce?i ?apte vitei, doi berbeci ?i paisprezece miei fara meteahna;
Num 29:33  ?i cu ei dar de pâine ?i turnare pentru vilei, pentru berbeci ?i pentru miei, dupa numarul lor, cum e rânduiala;
Num 29:34  Iar din turma de capre sa aduce?i un ?ap, jertfa pentru pacat, peste arderea de tot necontenita cu darul de pâine ?i cu turnarea ei.
Num 29:35  În ziua a opta sa ave?i încheierea sarbatorii; nici un lucru sa nu lucra?i,
Num 29:36  ?i sa aduce?i ardere de tot, jertfa, mireasma placuta Domnului: un vi?el, un berbec ?i ?apte miei fara meteahna.
Num 29:37  ?i cu ei dar de pâine ?i turnare pentru vi?el, pentru berbec ?i pentru miei, dupa numarul lor, cum e rânduiala.
Num 29:38  Iar din turma de capre sa aduce?i un ?ap, jertfa pentru pacat, pe lânga arderea de tot necontenita cu darul de pâine ?i cu turnarea ei.
Num 29:39  Acestea sa le aduce?i Domnului la sarbatorile voastre, pe lânga arderile de tot ale voastre cu darurile de pâine ale voastre ?i cu turnarile voastre ?i jertfele voastre de buna voie, pe care le aduce?i dupa fagaduin?a sau din evlavie".
Num 29:40  Moise a spus fiilor lui Israel toate cele ce-i poruncise Domnul.
Num 30:1  A?adar, a grait Moise catre capeteniile semin?iilor fiilor lui Israel ?i le-a zis: "Iata ce porunce?te Domnul:
Num 30:2  Omul care va face fagaduin?a Domnului sau se va jura cu juramânt, punând legatura asupra sufletului sau, sa nu-?i calce cuvântul, ci sa împlineasca toate câte au ie?it din gura lui.
Num 30:3  Daca vreo femeie va da fagaduin?a Domnului ?i va pune asupra sa legamântul, în casa parintelui sau, în tinere?ea sa,
Num 30:4  ?i va auzi tatal fagaduin?a ei ?i legamântul ce ea ?i-a pus asupra sufletului sau, ?i va tacea tatal ei asupra acestora, atunci toate fagaduin?ele ei se vor ?ine ?i orice legamânt ?i-ar fi pus ea asupra sufletului sau se va ?ine.
Num 30:5  Iar daca tatal ei, auzind, o va opri, atunci toate fagaduin?ele ei ?i legamintele ce ea ?i-ar fi pus asupra sufletului sau nu se vor ?ine ?i Domnul o va ierta, pentru ca a oprit-o tatal ei.
Num 30:6  Daca însa ea se va marita ?i va fi asupra ei fagaduin?a sau cuvântul gurii sale, cu care s-a legat pe sine,
Num 30:7  ?i va auzi barbatul ei ?i, auzind-o, va tacea, atunci fagaduin?ele ei se vor ?ine ?i legamintele ce ea ?i-a pus asupra sufletului sau se vor ?ine.
Num 30:8  Iar daca barbatul ei, auzind, o va opri ?i va lepada fagaduin?a ei, care este asupra ei, ?i cuvântul gurii ei cu care ea s-a legat pe sine, atunci acestea nu se vor ?ine, pentru ca i le-a oprit barbatul ei, ?i Domnul o va ierta.
Num 30:9  Iar fagaduin?a vaduvei ?i a celei despar?ite ?i orice legamânt ?i-ar pune aceasta asupra sufletului ei se va ?ine.
Num 30:10  Daca însa în casa barbatului sau a dat fagaduin?a sau ?i-a pus legamânt asupra sufletului sau cu juramânt,
Num 30:11  ?i barbatul ei a auzit ?i a tacut asupra acesteia ?i n-a oprit-o, atunci toate fagaduin?ele ei se vor ?ine ?i orice legamânt ?i-ar fi pus asupra sufletului sau se va ?ine.
Num 30:12  Daca însa barbatul ei, auzind, a lepadat fagaduin?ele, atunci toate fagaduin?ele ie?ite din gura ei ?i legamintele sufletului sau nu se vor ?ine, pentru ca barbatul ei le-a desfiin?at ?i Domnul o va ierta.
Num 30:13  Orice fagaduin?a ?i orice legamânt cu juramânt pentru smerirea sufletului ei, barbatul ei îl poate întari ?i tot barbatul ei îl poate ?i desfiin?a.
Num 30:14  Daca însa barbatul ei a tacut despre aceasta, zi dupa zi, prin aceasta el a întarit toate fagaduin?ele ei ?i toate legamintele ce sunt asupra ei le-a întarit, pentru ca el a auzit ?i a tacut.
Num 30:15  Iar daca barbatul le-a lepadat dupa ce le-a auzit, atunci a luat el asupra sa pacatul ei".
Num 30:16  Acestea sunt legile, pe care Domnul le-a poruncit lui Moise asupra legamintelor dintre barbat ?i femeia lui, dintre tata ?i fiica lui, cât aceasta este tânara ?i se afla în casa tatalui ei.
Num 31:1  Apoi a grait Domnul cu Moise ?i a zis:
Num 31:2  "Razbuna pe fiii lui Israel împotriva Madiani?ilor, ?i apoi te vei adauga la poporul tau".
Num 31:3  Iar Moise a grait poporului ?i a zis: "Înarma?i dintre voi oameni pentru razboi, ca sa mearga împotriva Madiani?ilor ?i sa savâr?easca razbunarea Domnului asupra Madiani?ilor.
Num 31:4  Din toate semin?iile fiilor lui Israel sa trimite?i la razboi, câte o mie din fiecare semin?ie".
Num 31:5  ?i ?i-au ales din miile lui Israel, câte o mie din fiecare semin?ie, adica douasprezece mii de oameni, înarma?i pentru razboi.
Num 31:6  Pe ace?tia i-a trimis Moise la razboi, câte o mie din fiecare semin?ie; ?i cu ei a trimis la razboi pe Finees, fiul preotului Eleazar, fiul lui Aaron; ?i acesta avea în mâinile sale vasele sfinte ?i trâmbi?ele de strigare.
Num 31:7  ?i au lovit ei pe Madian, cum poruncise Domnul lui Moise, ?i au ucis pe to?i cei de parte barbateasca.
Num 31:8  Împreuna cu uci?ii lor au cazut ?i regii madiani?i: Evi, Rechem, ?ur, Hur ?i Reba - cinci regi madiani?i - ?i Valaam, fiul lui Beor, a cazut de sabie, împreuna cu uci?ii acelora.
Num 31:9  Iar pe femeile Madiani?ilor ?i pe copiii lor le-au luat fiii lui Israel în robie; ?i toate vitele lor, toate turmele lor ?i toate avu?iile lor le-au luat prada.
Num 31:10  Toate ceta?ile lor din ?inuturile lor cu toate satele lor le-au ars cu foc.
Num 31:11  Toata prada ?i tot ce-au apucat de la om pâna la dobitoc au luat cu ei.
Num 31:12  Robii, prada ?i cele apucate le-au dus la Moise, la preotul Eleazar ?i la ob?tea fiilor lui Israel, în tabara din ?esul Moabului, care este lânga Iordan, în fa?a Ierihonului.
Num 31:13  În întâmpinarea lor au ie?it din tabara Moise, Eleazar preotul ?i toate capeteniile ob?tii.
Num 31:14  Atunci s-a mâniat Moise pe capeteniile o?tirii, pe capeteniile miilor ?i pe suta?ii care se întorsesera de la razboi, ?i le-a zis Moise:
Num 31:15  "Pentru ce a?i lasat vii toate femeile?
Num 31:16  Caci ele, dupa sfatul lui Valaam, au facut pe fiii lui Israel sa se abata de la cuvântul Domnului, pentru Peor, pentru care a venit pedeapsa asupra ob?tii Domnului.
Num 31:17  Ucide?i dar to?i copiii de parte barbateasca ?i toate femeile ce-au cunoscut barbat, ucide?i-le.
Num 31:18  Iar pe fetele care n-au cunoscut barbat, lasa?i-le pe toate vii pentru voi.
Num 31:19  ?i sa ?ede?i afara din tabara ?apte zile; to?i cei ce a?i ucis om ?i v-a?i atins de om ucis sa va cura?i?i în ziua a treia ?i în ziua a ?aptea, ?i voi ?i robii vo?tri.
Num 31:20  Toate hainele, toate lucrurile de piele, tot ce este facut din par de capra ?i toate vasele de lemn sa le cura?i?i".
Num 31:21  Apoi a zis preotul Eleazar o?tenilor care fusesera la razboi: "Hotarârea legii, pe care a dat-o Domnul lui Moise, este aceasta:
Num 31:22  Aurul, argintul, arama, fierul, plumbul, cositorul
Num 31:23  ?i tot ce trece prin foc, sa le trece?i prin foc, ca sa se cure?e; afara de aceasta ?i cu apa de cura?ire sa le cura?i?i; iar toate cele ce nu se pot trece prin foc, sa le trece?i prin apa.
Num 31:24  Hainele voastre sa le spala?i în ziua a ?aptea ?i sa va cura?i?i, iar dupa aceea ve?i intra în tabara".
Num 31:25  Iara?i a grait Domnul cu Moise ?i a zis:
Num 31:26  "Socote?te prada de razboi, de la om pâna la dobitoc, împreuna cu Eleazar preotul ?i cu capeteniile semin?iilor ob?tii;
Num 31:27  Apoi împarte prada în doua, între o?tenii care au fost la batalie ?i între toata ob?tea.
Num 31:28  De la o?tenii care au fost la razboi, ia dare pentru Domnul, câte un suflet la cinci sute, din oameni, din vite, din asini ?i din oi.
Num 31:29  Acestea sa le iei din partea lor ?i sa le dai preotului Eleazar ca dar înal?at Domnului.
Num 31:30  Iar din jumatatea cuvenita fiilor lui Israel sa iei unul la cincizeci din oameni, din vite, din asini ?i din oi; ?i pe acestea sa le dai levi?ilor, care slujesc la cortul Domnului".
Num 31:31  ?i a facut Moise ?i Eleazar preotul cum poruncise Domnul lui Moise.
Num 31:32  Atunci s-a gasit prada ramasa din cele luate ?i aduse de cei ce fusesera la razboi: ?ase sute ?aptezeci ?i cinci de mii de oi;
Num 31:33  ?aptezeci ?i doua de mii de boi;
Num 31:34  Asini, ?aizeci ?i una de mii;
Num 31:35  Femei, care n-au cunoscut barbat, de toate, treizeci ?i doua de mii de suflete.
Num 31:36  Jumatate, partea celor ce fusesera la razboi, dupa numaratoare, a fost: oi trei sute treizeci ?i ?apte de mii cinci sute.
Num 31:37  ?i darea Domnului din oi a fost: ?ase sute ?aptezeci ?i cinci;
Num 31:38  Boi treizeci ?i ?ase de mii, ?i din ace?tia darea Domnului a fost ?aptezeci ?i doi;
Num 31:39  Asini, treizeci de mii cinci sute, ?i din ei darea Domnului a fost ?aizeci ?i unul;
Num 31:40  Oameni, ?aisprezece mii, ?i din ei darea Domnului a fost treizeci ?i doua suflete.
Num 31:41  ?i a dat Moise darea Domnului lui Eleazar preotul, cum poruncise Domnul lui Moise.
Num 31:42  Iar partea fiilor lui Israel, pe care a luat-o Moise de la cei ce fusesera la razboi, a fost:
Num 31:43  Oi, trei sute treizeci ?i ?apte de mii cinci sute;
Num 31:44  Boi, treizeci ?i ?ase de mii;
Num 31:45  Asini, treizeci de mii cinci sute;
Num 31:46  Oameni, ?aisprezece mii.
Num 31:47  Din aceasta parte a fiilor lui Israel, a luat Moise unul la cincizeci din oameni ?i din vite, ?i le-a dat levi?ilor, care faceau slujba în cortul Domnului, dupa cum poruncise Domnul lui Moise.
Num 31:48  Atunci au venit la Moise capeteniile o?tirii, capeteniile peste mii ?i suta?ii, ?i au zis catre Moise:
Num 31:49  "Robii tai au numarat pe o?tenii care ne-au fost încredin?a?i ?i n-a lipsit niciunul din ei.
Num 31:50  ?i iata noi am adus prinos Domnului, fiecare ce am putut dobândi din lucrurile de aur: lan?uri, bra?ari, inele, cercei ?i salbe, pentru cura?irea sufletelor noastre înaintea Domnului".
Num 31:51  ?i a luat Moise ?i Eleazar de la ei toate aceste lucruri de aur.
Num 31:52  Aurul acesta, care s-a adus prinos Domnului de catre capeteniile peste mii ?i peste sute, a fost tot ?aisprezece mii ?apte sute cincizeci de sicli.
Num 31:53  O?tenii au pradat fiecare pentru ei.
Num 31:54  ?i a luat Moise ?i preotul Eleazar aurul de la capeteniile peste mii ?i peste sute ?i l-au dus în cortul adunarii, pentru pomenirea fiilor lui Israel înaintea Domnului.
Num 32:1  Iar fiii lui Ruben ?i fiii lui Gad aveau foarte multe turme; dar vazând ca pamântul Iazer ?i pamântul Galaad sunt locuri bune pentru turme,
Num 32:2  Au venit fiii lui Gad ?i fiii lui Ruben ?i au grait cu Moise, cu Eleazar preotul ?i cu capeteniile ob?tii ?i au zis:
Num 32:3  "Atarotul, Dibonul, Iazerul, Nimra, He?bonul, Eleale, Sevam, Nebo ?i Beon,
Num 32:4  ?inuturile, pe care Domnul le-a lovit înaintea ob?tii lui Israel, sunt pamânturi bune pentru turme, ?i robii tai au turme".
Num 32:5  ?i au mai zis: "De-am aflat trecere în ochii tai, da pamânturile acestea robilor tai în stapânire ?i nu ne trece peste Iordan".
Num 32:6  Moise însa a zis catre fiii lui Gad ?i catre fiii lui Ruben: "Fra?ii vo?tri se duc la razboi, iar voi sa ramâne?i aici?
Num 32:7  Pentru ce întoarce?i inima fiilor lui Israel sa nu treaca în pamântul pe care Domnul li-l da?
Num 32:8  A?a au facut ?i parin?ii vo?tri când i-am trimis din Cade?-Barne ca sa cerceteze ?ara:
Num 32:9  Au mers pâna în valea E?col, au vazut pamântul ?i au abatut inima fiilor lui Israel, ca sa nu mearga ace?tia în pamântul pe care Domnul li-l da.
Num 32:10  Dar s-a aprins în ziua aceea mânia Domnului ?i S-a jurat ?i a zis:
Num 32:11  "Oamenii ace?tia, care au ie?it din Egipt ?i care sunt de douazeci de ani ?i mai mari ?i cunosc binele ?i raul, nu vor vedea pamântul, pentru care Eu M-am jurat lui Avraam ?i lui Isaac ?i lui Iacov.
Num 32:12  Pentru ca nu Mi s-au supus Mie, afara de Caleb, fiul lui Iefone Chenezul, ?i de Iosua, fiul lui Navi, pentru ca ace?tia s-au supus Domnului".
Num 32:13  S-a aprins atunci mânia Domnului asupra lui Israel ?i i-a purtat prin pustie patruzeci de ani, pâna când s-a sfâr?it tot neamul care facuse rau înaintea Domnului.
Num 32:14  ?i iata acum, în locul parin?ilor vo?tri v-a?i ridicat voi, samân?a pacato?ilor, ca sa spori?i înca ?i mai mult iu?imea mâniei Domnului asupra lui Israel.
Num 32:15  Daca va ve?i abate de la El, iara?i va lasa pe Israel în pustie ?i ve?i pierde tot poporul acesta".
Num 32:16  Iar ei, apropiindu-se de el, au zis: "Noi ne vom face aici staule pentru turmele noastre ?i ceta?i pentru copiii no?tri;
Num 32:17  Iar noi în?ine cei dintâi ne vom înarma ?i vom merge înaintea fiilor lui Israel, pâna ce îi vom duce la locurile lor; iar copiii no?tri vor ramâne în ceta?ile întarite, pentru ca sa nu fie în primejdie din partea oamenilor locului.
Num 32:18  Nu ne vom întoarce la casele noastre, pâna când fiii lui Israel nu vor intra fiecare în mo?tenirea sa;
Num 32:19  Caci nu vom lua împreuna cu ei mo?tenire dincolo de Iordan ?i nici mai departe, daca ni se va da parte dincoace de Iordan, spre rasarit".
Num 32:20  Atunci a zis Moise catre ei: "De ve?i face aceasta, de ve?i merge înarma?i la razboi înaintea Domnului,
Num 32:21  De va trece fiecare din voi înarmat peste Iordan înaintea Domnului, pâna când va pierde El pe vrajma?ii Sai înaintea Sa ?i pâna când va fi cuprins pamântul înaintea Lui,
Num 32:22  Atunci, dupa ce va ve?i întoarce, ve?i fi fara vina înaintea Domnului ?i înaintea lui Israel ?i ve?i avea pamântul acesta mo?tenire înaintea Domnului.
Num 32:23  Iar de nu ve?i face a?a, ve?i gre?i înaintea Domnului ?i ve?i suferi pedeapsa care va va ajunge pentru pacatul vostru.
Num 32:24  Zidi?i-va ceta?i pentru copiii vo?tri ?i staule pentru oile voastre ?i face?i cele ce a?i rostit cu buzele voastre".
Num 32:25  Zis-au fiii lui Gad ?i fiii lui Ruben catre Moise: "Robii tai vor face cum porunce?te domnul nostru.
Num 32:26  Copiii no?tri, femeile noastre, turmele noastre ?i toate vitele noastre vor ramâne aici în ceta?ile Galaadului;
Num 32:27  Iar robii tai, înarma?i cu to?ii ca o?teni, vor merge înaintea Domnului la razboi, cum zice domnul nostru".
Num 32:28  Atunci a dat Moise porunca pentru ei lui Eleazar preotul, lui Iosua, fiul lui Navi, ?i capeteniilor semin?iilor fiilor lui Israel,
Num 32:29  ?i le-a zis Moise: "Daca fiii lui Gad ?i fiii lui Ruben vor trece cu voi peste Iordan, întrarmându-se cu to?ii pentru razboi înaintea Domnului, dupa ce ?ara va fi supusa înaintea voastra, sa le da?i pamântul Galaad în stapânire.
Num 32:30  Iar daca ei nu vor merge cu voi înarma?i pentru razboi înaintea Domnului, sa trimite?i înaintea voastra averea lor, femeile lor ?i vitele lor în pamântul Canaan ?i ei sa primeasca mo?ie împreuna cu voi în pamântul Canaanului".
Num 32:31  Iar fiii lui Gad ?i fiii lui Ruben au raspuns ?i au zis: "Cum a zis domnul robilor tai a?a vom ?i face.
Num 32:32  Vom merge înarma?i înaintea Domnului în pamântul Canaan, iar partea noastra de mo?ie sa fie de asta parte de Iordan".
Num 32:33  Atunci Moise le-a dat fiilor lui Gad, fiilor lui Ruben ?i la jumatate din semin?ia lui Manase, fiul lui Iosif, ?ara lui Sihon, regele Amoreilor, ?i ?ara lui Og, regele Vasanului, pamântul cu ora?ele lui ?i împrejurimile ?i ceta?ile din toate par?ile jarii.
Num 32:34  ?i au zidit fiii lui Gad: Dibonul, Atarotul, Aroerul,
Num 32:35  Atarot-?ofanul, Iazerul, Iogbeha,
Num 32:36  Bet-Nimra ?i Bet-Haran, ceta?i întarite ?i staule pentru oi.
Num 32:37  Fiii lui Ruben au zidit He?bonul, Eleale, Chiriataimul,
Num 32:38  Nebo, Baal-Meonul ?i Sibma, ale caror nume au fost schimbate ?i au dat alte nume ora?elor pe care le-au zidit ei.
Num 32:39  Iar fiii lui Machir, fiul lui Manase, s-au dus în Galaad ?i l-au luat ?i au alungat pe Amoreii care erau acolo.
Num 32:40  Iar Moise a dat Galaadul lui Machir, fiul lui Manase, ?i s-a a?ezat acela acolo.
Num 32:41  Iair, fiul lui Manase, s-a dus ?i a luat sala?urile lor ?i le-a numit sala?urile lui Iair.
Num 32:42  Iar Nobah s-a dus ?i a luat Chenatul ?i ceta?ile care ?ineau de el ?i l-a numit dupa numele sau: Nobah.
Num 33:1  Iata acum popasurile fiilor lui Israel, dupa ce au ie?it ei din pamântul Egiptului cu o?tirile lor, sub mâna lui Moise ?i Aaron.
Num 33:2  Moise, din porunca Domnului, a scris calatoria lor cu popasurile lor; iar popasurile lor sunt acestea:
Num 33:3  În luna întâi, în ziua a cincisprezecea a lunii întâi, a doua zi de Pa?ti, fiii lui Israel au purces din Ramses (Go?en) ?i au ie?it, sub mâna înalta, înaintea ochilor a tot Egiptul.
Num 33:4  În vremea aceea Egiptenii îngropau pe to?i cei ce murisera dintre ei, pe to?i întâi-nascu?ii, pe care-i lovise Domnul, în ?ara Egiptului, când a facut Domnul judecata asupra dumnezeilor lor.
Num 33:5  Dupa ce au pornit fiii lui Israel din Ramses (Go?en), au poposit în Sucot.
Num 33:6  Pornind apoi din Sucot, au tabarât la Etam, care este la marginea pustiului.
Num 33:7  Din Etam au pornit ?i s-au îndreptat spre Pi-Hahirot, care este în ?ara Baal-?efonului, ?i ?i-au a?ezat tabara înaintea Migdolului.
Num 33:8  Pornind apoi din Pi-Hahirot, au trecut prin mare în pustie ?i, mergând cale de trei zile prin pustiul Etam, ?i-au a?ezat tabara la Mara.
Num 33:9  Plecând de la Mara, au venit la Elim. În Elim insa erau douasprezece izvoare de apa ?i ?aptezeci de finici ?i au tabarât acolo lânga apa.
Num 33:10  Pornind apoi din Elim, au tabarât la Marea Ro?ie.
Num 33:11  Au pornit apoi de la Marea Ro?ie ?i au tabarât în pustiul Sin.
Num 33:12  Pornind din pustiul Sin, au poposit la Dofca.
Num 33:13  Pornind din Dofca, au tabarât la Alu?.
Num 33:14  Pornind din Alu?, ?i-au a?ezat tabara la Rafidim. Acolo nu era apa ca sa bea poporul.
Num 33:15  Pornind din Rafidim, au tabarât în pustiul Sinai.
Num 33:16  Iar dupa ce au pornit din pustiul Sinai, au poposit la Chibrot-Hataava.
Num 33:17  Pornind din Chibrot-Hataava, au tabarât în Ha?erot.
Num 33:18  Pornind din Ha?erot, au poposit la Ritma.
Num 33:19  Pornind din Ritma, ?i-au a?ezat tabara la Rimon-Pere?.
Num 33:20  Pornind din Rimon-Pere?, au tabarât în Libna.
Num 33:21  Pornind din Libna, au tabarât la Risa.
Num 33:22  Pornind din Risa, ?i-au a?ezat tabara la Chehelata.
Num 33:23  Pornind din Chehelata, au tabarât pe Muntele ?afer.
Num 33:24  Pornind de pe Muntele ?afer, au poposit în Harada.
Num 33:25  Pornind din Harada, au tabarât la Machelot.
Num 33:26  Pornind din Machelot, au poposit în Tahat.
Num 33:27  Pornind din Tahat, s-au a?ezat cu tabara în Tarah.
Num 33:28  Pornind din Tarah, au tabarât în Mitca.
Num 33:29  Pornind din Mitca, au tabarât în Ha?mona.
Num 33:30  Pornind din Ha?mona, au poposit la Moserot.
Num 33:31  Pornind din Moserot, ?i-au a?ezat tabara la Bene-Iaakan.
Num 33:32  Pornind din Bene-Iaakan, au tabarât la Hor-Haghidgad.
Num 33:33  Pornind din Hor-Haghidgad, au poposit în Iotbata.
Num 33:34  Pornind din Iotbata, au tabarât la Abrona.
Num 33:35  Pornind din Abrona, ?i-au a?ezat tabara la E?ion-Gheber.
Num 33:36  Pornind din E?ion-Gheber, au poposit în pustiul Sin. Plecând din pustiul Sin, au tabarât în Cade?.
Num 33:37  Iar din Cade? au purces ?i au poposit la muntele Hor, lânga hotarul ?arii Edomului.
Num 33:38  Aici s-a suit Aaron preotul pe muntele Hor, dupa porunca Domnului, ?i a murit acolo, în anul al patruzecilea de la ie?irea fiilor lui Israel din pamântul Egiptului, în luna a cincea, în ziua întâi a lunii.
Num 33:39  Aaron era de o suta douazeci ?i trei de ani, când a murit pe muntele Hor.
Num 33:40  Atunci regele canaanean din Arad, care traia în partea de miazazi a pamântului Canaan, a auzit ca vin fiii lui Israel.
Num 33:41  Ace?tia însa, plecând de la muntele Hor, au tabarât la ?almona.
Num 33:42  Pornind din ?almona, au poposit la Punon.
Num 33:43  Pornind din Punon, au tabarât la Obot.
Num 33:44  Pornind din Obot, au poposit la Iie-Abarim, lânga hotarele lui Moab.
Num 33:45  Pornind din Iie-Abarim, au tabarât la Dibon-Gad.
Num 33:46  Pornind din Dibon-Gad, au poposit ia Almon-Diblataim.
Num 33:47  Pornind din Almon-Diblataim, au tabarât în mun?ii Abarim, în fa?a lui Nebo.
Num 33:48  Pornind de la mun?ii Abarim, au poposit în ?esurile Moabului, la Iordan, în fa?a Ierihonului,
Num 33:49  ?i ?i-au a?ezat ei tabara la Iordan, de la Bet-Ie?imot pâna la Abel-?itim, în ?esurile Moabului.
Num 33:50  Grait-a Domnul cu Moise în ?esurile Moabului, la Iordan, în fa?a Ierihonului, ?i a zis:
Num 33:51  "Vorbe?te fiilor lui Israel ?i le spune: Când ve?i trece peste Iordan, în pamântul Canaanului,
Num 33:52  Sa alunga?i de la voi pe to?i locuitorii ?arii ?i sa strica?i toate chipurile cele cioplite ale lor, to?i idolii lor cei turna?i din argint ?i toate înal?imile lor sa le pustii?i.
Num 33:53  Sa lua?i în stapânire pamântul ?i sa va a?eza?i acolo, caci va dau în stapânire pamântul acesta.
Num 33:54  Sa împar?i?i pamântul prin sor?i la semin?iile voastre: celor mai mul?i la numar sa le da?i parte mai mare, iar celor mai pu?ini la numar sa le da?i parte mai mica; fiecaruia unde-i va cadea sor?ul, acolo sa-i fie partea, dupa semin?iile parin?ilor vo?tri.
Num 33:55  Iar daca nu ve?i alunga de la voi pe locuitorii pamântului, atunci cei rama?i din ei vor fi spini pentru ochii vo?tri ?i bolduri pentru coastele voastre ?i va vor strâmtora în ?ara în care ve?i trai.
Num 33:56  ?i atunci va voi face voua ceea ce aveam de gând sa le fac lor".
Num 34:1  A grait Domnul cu Moise ?i a zis:
Num 34:2  "Porunce?te fiilor lui Israel ?i le zi: Iata, ve?i intra în pamântul Canaan. Acesta va fi mo?tenirea voastra; iar hotarele Canaanului sunt acestea:
Num 34:3  Partea de miazazi va începe de la pustiul Sin de lânga Edom ?i va avea la rasarit, ca hotar, Marea Sarata.
Num 34:4  Acest hotar se va îndrepta spre miazazi, catre înal?imea Acravimului; va trece prin Sin ?i se va întinde pâna la miazazi de Cade?-Barne; apoi va merge catre Ha?ar-Adar trecând la A?mon.
Num 34:5  De la A?mon, hotarul se va îndrepta spre Râul Egiptului ?i se va pogorî pâna la mare.
Num 34:6  Iar hotar dinspre apus va va fi Marea cea Mare. Acesta va fi hotarul vostru dinspre asfin?it.
Num 34:7  Iar spre miazanoapte, hotarul vostru sa-l trage?i de la Marea cea Mare pâna la muntele Hor;
Num 34:8  De la muntele Hor, sa-l trage?i spre Hamat, ?i hotarul va atinge ?edadul.
Num 34:9  De acolo va merge hotarul catre ?ifron ?i va atinge Ha?ar-Enan. Acesta sa va fie hotarul de miazanoapte.
Num 34:10  Iar hotarul dinspre rasarit sa vi-l trage?i de la Ha?ar-Enan catre ?efam;
Num 34:11  De la ?efam hotarul se va pogorî spre Ribla, pe la rasarit de Ain, mergând de-a lungul malului Marii Chineret (Ghenizaret) pe partea de rasarit.
Num 34:12  De aici hotarul se va pogorî pe Iordan ?i se va sfâr?i la Marea Sarata. Acesta va fi pamântul vostru, dupa hotarele lui din toate par?ile".
Num 34:13  Atunci a dat Moise porunca fiilor lui Israel ?i a zis: "Iata pamântul pe care voi îl ve?i împar?i în buca?i, prin sor?i, ?i care a poruncit Domnul sa se dea la noua semin?ii ?i la jumatate din semin?ia lui Manase.
Num 34:14  Caci semin?iilor fiilor lui Ruben cu familiile lor, a fiilor lui Gad cu familiile lor, ?i jumatate din semin?ia lui Manase ?i-au primit partea lor.
Num 34:15  Doua semin?ii întregi ?i o jumatate de semin?ie ?i-au primit partea peste Iordan, pe partea rasariteana, în fa?a Ierihonului".
Num 34:16  A grait Domnul cu Moise ?i a zis:
Num 34:17  "Iata numele barba?ilor care au sa va împarta pamântul: Eleazar preotul ?i Iosua, fiul lui Navi;
Num 34:18  Ve?i mai lua înca ?i câte o capetenie de fiecare semin?ie pentru împar?irea pamântului.
Num 34:19  Numele acestor barba?i sunt: Caleb, fiul lui Iefoni, pentru semin?ia Iudei;
Num 34:20  Samuel, fiul lui Amihud, pentru semin?ia fiilor lui Simeon;
Num 34:21  Elidad, fiul lui Chislon, pentru semin?ia lui Veniamin;
Num 34:22  Capetenia Buchi, fiul lui Iogli, pentru semin?ia fiilor lui Dan;
Num 34:23  Capetenia Haniel, fiul lui Efod, pentru semin?ia fiilor lui Manase;
Num 34:24  Capetenia Chemuel, fiul lui ?iftan, pentru semin?ia fiilor lui Efraim;
Num 34:25  Capetenia Eli?afan, fiul lui Parnac, pentru semin?ia fiilor lui Zabulon,
Num 34:26  Capetenia Paltiel, fiul lui Azan, pentru semin?ia fiilor lui Isahar;
Num 34:27  Capetenia Ahihud, fiul lui ?elomi, pentru semin?ia fiilor lui A?er;
Num 34:28  Capetenia Pedael, fiul lui Amihud, pentru semin?ia fiilor lui Neftali".
Num 34:29  Ace?tia sunt aceia carora a poruncit Domnul sa împarta pamântul Canaan la fiii lui Israel.
Num 35:1  În vremea aceea a grait Domnul cu Moise în ?esurile Moabului, la Iordan, în fa?a Ierihonului, ?i a zis:
Num 35:2  "Porunce?te fiilor lui Israel, ca ei, din par?ile mo?tenirii lor, sa dea levi?ilor ceta?i de locuit; ?i împrejurul ceta?ilor sa le dea levi?ilor locuri.
Num 35:3  Ceta?ile vor fi de locuit; iar locurile vor fi pentru vitele lor, iar averea pentru toate nevoile vie?ii lor.
Num 35:4  Locurile de pe lânga ceta?ile pe care trebuie sa le da?i levi?ilor sa se întinda în toate par?ile, de la zidurile ceta?ii pâna la doua mii de co?i;
Num 35:5  Sa masura?i de la cetate, spre rasarit doua mii de co?i, spre miazazi doua mii de co?i, spre apus doua mii de co?i ?i spre miazanoapte doua mii de co?i, iar în mijloc sa fie cetatea: acestea vor fi pamânturile lor de pe lânga ceta?i.
Num 35:6  Dintre ceta?ile pe care le ve?i da levi?ilor ?ase ceta?i sa fie de scapare, în care ve?i îngadui sa fuga uciga?ii. ?i pe lânga acestea sa le mai da?i patruzeci ?i doua de ceta?i.
Num 35:7  Ceta?ile pe care trebuie sa le da?i levi?ilor sa fie de toate patruzeci ?i opt de ceta?i cu locurile dimprejur.
Num 35:8  ?i când ve?i da ceta?ile acestea din mo?iile fiilor lui Israel, atunci din mo?iile cele mai mari sa da?i mai mult ?i din cele mai mici mai pu?in; fiecare semin?ie sa dea levi?ilor din ceta?ile ei potrivit cu partea primita.
Num 35:9  A grait Domnul cu Moise ?i a zis:
Num 35:10  "Spune fiilor lui Israel ?i le zi:
Num 35:11  Când ve?i trece peste Iordan, în pamântul Canaan, sa va alege?i ceta?ile care au sa va fie ceta?i de scapare, unde sa poata fugi uciga?ul care a ucis om fara sa vrea.
Num 35:12  ?i vor fi ceta?ile acestea loc de scapare de cel ce razbuna sângele varsat, ca sa nu fie omorât cel ce a ucis, înainte de a se înfa?i?a el în fa?a ob?tii la judecata.
Num 35:13  Ceta?ile pe care trebuie sa le da?i ca ceta?i de scapare, sa fie ?ase.
Num 35:14  Trei ceta?i sa da?i de asta parte de Iordan, ?i trei ceta?i sa da?i în pamântul Canaan; acestea trebuie sa fie ceta?ile de scapare.
Num 35:15  Aceste ceta?i sa fie, ?i pentru fiii lui Israel ?i pentru straini ?i pentru cei stramuta?i la voi, loc de scapare; acolo sa fuga uciga?ul fara voie.
Num 35:16  Daca cineva a lovit pe altul cu o unealta de fier ?i acela a murit, acesta este uciga? ?i uciga?ul trebuie omorât.
Num 35:17  Daca cineva a lovit cu piatra pe altul ?i acela a murit, acesta este uciga? ?i uciga?ul trebuie omorât.
Num 35:18  Sau daca cu o unealta de lemn, cu care se poate pricinui moartea, l-a lovit a?a încât acela a murit, acesta este uciga? ?i uciga?ul trebuie dat mor?ii.
Num 35:19  Razbunatorul sângelui varsat poate sa ucida pe fapta? îndata ce-l întâlne?te.
Num 35:20  Daca cineva izbe?te pe altul din ura, sau cu gând rau arunca asupra lui ceva, a?a încât acela moare, sau din du?manie îl love?te cu mâna, a?a încât acela moare,
Num 35:21  Cel ce a lovit trebuie dat mor?ii, ca este uciga?, ?i razbunatorul sângelui varsat poate ucide pe uciga? îndata ce-l va întâlni.
Num 35:22  Daca însa cineva izbe?te pe altul din nebagare de seama, fara du?manie,
Num 35:23  Sau arunca ceva asupra lui fara gând rau, sau vreo piatra a rostogolit asupra lui fara sa-l vada ?i acela moare, iar el nu i-a fost du?man ?i nu i-a dorit raul,
Num 35:24  Atunci ob?tea trebuie sa judece între uciga? ?i razbunatorul sângelui varsat dupa aceste rânduieli;
Num 35:25  ?i ob?tea trebuie sa izbaveasca pe uciga? din mâinile razbunatorului sângelui varsat, ?i sa-l întoarca ob?tea în cetatea lui de scapare, unde a fugit el, ca sa traiasca acolo pâna la moartea marelui preot, care este miruit cu mir sfin?it.
Num 35:26  Daca uciga?ul va ie?i peste hotarele ora?ului de scapare, în care a fugit,
Num 35:27  ?i-l va gasi razbunatorul sângelui varsat, în afara de hotarele ceta?ii lui de scapare, ?i va ucide pe uciga?ul acesta razbunatorul de sânge, acesta nu va fi vinovat de varsare de sânge,
Num 35:28  Pentru ca acela trebuie sa ?ada în ora?ul sau de scapare pâna la moartea marelui preot; iar dupa moartea marelui preot trebuie sa se întoarca uciga?ul în pamântul sau de mo?tenire.
Num 35:29  Aceasta sa va fie rânduiala legiuita în neamul ?i în toate loca?urile voastre.
Num 35:30  Daca cineva va ucide om, uciga?ul trebuie ucis dupa cuvintele martorilor, dar pentru a osândi la moarte, nu este de ajuns un singur martor.
Num 35:31  Sa nu lua?i rascumparare pentru sufletul uciga?ului care este vinovat mor?ii, ci sa-l omorâ?i.
Num 35:32  Sa nu lua?i rascumparare pentru cel ce a fugit în ora?ul de scapare, ca sa-i îngadui?i sa locuiasca în pamântul sau, înainte de moartea marelui preot.
Num 35:33  Sa nu spurca?i pamântul pe care ave?i sa trai?i; ca sângele spurca pamântul ?i pamântul nu se poate cura?i în alt fel de sângele varsat pe el, decât cu sângele celui ce l-a varsat.
Num 35:34  Sa nu spurca?i pamântul pe care trai?i ?i în mijlocul caruia locuiesc Eu; caci Eu, Domnul, locuiesc între fiii lui Israel".
Num 36:1  Atunci au venit capeteniile familiilor din semin?ia fiilor lui Galaad, fiul lui Machir, fiul lui Manase, din semin?ia fiilor lui Iosif, ?i au grait înaintea lui Moise ?i înaintea lui Eleazar preotul ?i înaintea capeteniilor urma?ilor fiilor lui Israel ?i au zis:
Num 36:2  "Domnul a poruncit stapânului nostru sa dea pamânt de mo?tenire fiilor lui Israel prin sor?i, ?i stapânului nostru i s-a poruncit de la Domnul sa dea partea lui Salfaad, fratele nostru, fiicelor lui.
Num 36:3  Daca însa acestea vor fi so?ii ale fiilor unei alte semin?ii a fiilor lui Israel, atunci partea lor se va lua din mo?ia parin?ilor no?tri ?i se va adauga la mo?ia acelei semin?ii, în care ele vor fi so?ii, ?i a?a se va lua din mo?ia noastra ce ni s-a cuvenit prin sor?i.
Num 36:4  ?i chiar când va fi jubileu la fiii lui Israel, atunci partea lor se va adauga la mo?ia acelei semin?ii, în care ele vor fi so?ii, ?i partea lor se va ?terge din mo?ia semin?iei parin?ilor no?tri".
Num 36:5  Deci a dat Moise porunca fiilor lui Israel, dupa cuvântul Domnului, zicând:
Num 36:6  "Adevarat graie?te semin?ia fiilor lui Iosif. Iata ce porunce?te Domnul pentru fiicele lui Salfaad: Ele pot sa fie so?ii ale acelora care vor placea ochilor lor, numai sa fie so?ii în neamul semin?iei tatalui lor,
Num 36:7  Pentru ca partea fiilor lui Israel sa nu treaca de la o semin?ie la alta; ca fiecare din fiii lui Israel trebuie sa fie legat de mo?ia semin?iei parin?ilor sai.
Num 36:8  ?i orice fata care stapâne?te o parte de mo?tenite în una din semin?iile fiilor lui Israel sa fie so?ia cuiva din neamul semin?iei tatalui sau, ca fiii lui Israel sa mo?teneasca fiecare partea parin?ilor sai,
Num 36:9  ?i sa nu treaca partea de la o semin?ie la alta semin?ie, ca fiecare din semin?iile fiilor lui Israel trebuie sa fie legata de mo?ia sa".
Num 36:10  Cum a poruncit Domnul lui Moise, a?a au facut fiicele lui Salfaad.
Num 36:11  ?i fiicele lui Salfaad: Mahla, Tir?a, Hogla, Milca ?i Noa s-au maritat dupa fiii unchiului lor.
Num 36:12  În semin?ia fiilor lui Manase, fiul lui Iosif, au fost ele so?ii, ?i a ramas mo?ia lor neamului tatalui lor.
Num 36:13  Acestea sunt poruncile ?i a?ezamintele pe care le-a dat Domnul fiilor lui Israel, prin Moise, în ?esurile Moabului, la Iordan, în fa?a Ierihonului.


\end{document}