\begin{document}

\title{Numeri}


\chapter{1}

\par 1 În ziua întâi a lunii a doua din anul al doilea după ieșirea Israeliților din pământul Egiptului, a grăit Domnul cu Moise în cortul adunării, în pustiul Sinai, și a zis:
\par 2 "Numărați toată obștea fiilor lui Israel după semințiile lor, după familiile lor și după numele lor, om cu om.
\par 3 Tot bărbatul de la douăzeci de ani în sus, tot cel ce poate ieși la oaste în Israel, să se numere de tine și de Aaron și să se rânduiască în tabăra lui.
\par 4 Dar cu voi să mai fie din fiecare seminție câte un om, care e cel mai de seamă în neamul său.
\par 5 Iată numele bărbaților care vor fi cu voi: din Ruben: Elițur, fiul lui Ședeur;
\par 6 Din Simeon: Șelumiel, fiul lui Țurișadai;
\par 7 Din Iuda: Naason, fiul lui Aminadab;
\par 8 Din Isahar: Natanael, fiul lui Țuar;
\par 9 Din Zabulon: Eliab, fiul lui Helon;
\par 10 Din fiii lui Iosif: Elișama, fiul lui Amihud, din Efraim; și Gamaliel, fiul lui Pedațur, din Manase;
\par 11 Din Veniamin: Abidan, fiul lui Ghedeon;
\par 12 Din Dan: Ahiezer, fiul lui Amișadai;
\par 13 Din Așer: Paghiel, fiul lui Ocran;
\par 14 Din Gad: Eliasaf, fiul lui Raguel;
\par 15 Din Neftali: Ahira, fiul lui Enan.
\par 16 Aceștia sunt bărbații aleși ai obștii, capii semințiilor părinților lor, căpeteniile peste mii în Israel".
\par 17 Luând deci Moise și Aaron pe bărbații aceștia, care au fost numiți pe numele lor,
\par 18 Au adunat toată obștea în ziua întâi a lunii a doua din anul al doilea și au înscris, după spițele neamului lor, pe toți bărbații de la douăzeci de ani în sus pe seminții, pe familii și pe numele lor, om cu om.
\par 19 Numărătoarea aceasta a făcut-o Moise în pustiul Sinai, cum îi poruncise Domnul.
\par 20 Fiii lui Ruben, întâiul născut al lui Israel, după seminția lor, după neamurile lor, după familiile lor, după numele lor, toți bărbații om cu om, de la douăzeci de ani în sus, toți cei buni de oaste,
\par 21 S-au numărat în seminția lui Ruben patruzeci și șase de mii cinci sute.
\par 22 Fiii lui Simeon, după seminția lor, după neamurile lor, după familiile lor, după numele lor, toți bărbații om cu om, de la douăzeci de ani în sus, toți cei buni de oaste,
\par 23 S-au numărat în seminția lui Simeon cincizeci și nouă de mii trei sute.
\par 24 Fiii lui Gad, după seminția lor, după neamurile lor, după familiile lor, după numele lor, toți bărbații om cu om, de la douăzeci de ani în sus, toți cei buni de oaste,
\par 25 S-au numărat în seminția lui Gad patruzeci și cinci de mii șase sute cincizeci.
\par 26 Fiii lui Iuda, după seminția lor, după neamurile lor, după familiile lor, după numele lor, toți bărbații om cu om, de la douăzeci de ani în sus, toți cei buni de oaste,
\par 27 S-au numărat în seminția lui Iuda șaptezeci și patru de mii șase sute.
\par 28 Fiii lui Isahar, după seminția lor, după neamurile lor, după familiile lor, după numele lor, toți bărbații om cu om, de ia douăzeci de ani în sus, toți cei buni de oaste,
\par 29 S-au numărat în seminția lui Isahar cincizeci și patru de mii patru sute.
\par 30 Fiii lui Zabulon, după semințiile lor, după neamurile lor, după familiile lor, după numele lor, toți bărbații om cu om, de la douăzeci de ani în sus, toți cei buni de oaste,
\par 31 S-au numărat în seminția lui Zabulon cincizeci și șapte de mii patru sute.
\par 32 Fiii lui Iosif: fiii lui Efraim, după seminția lor, după neamurile lor, după familiile lor, după numele lor, toți bărbații om cu om, de la douăzeci de ani în sus, toți cei buni de oaste,
\par 33 S-au numărat în seminția lui Efraim patruzeci de mii cinci sute.
\par 34 Fiii lui Manase, după seminția lor, după neamurile lor, după familiile lor, după numele lor, toți bărbații om cu om, de la douăzeci de ani în sus, toți cei buni de oaste,
\par 35 S-au numărat în seminția lui Manase treizeci și două de mii două sute.
\par 36 Fiii lui Veniamin, după seminția lor, după neamurile lor, după familiile lor, după numele lor, toți bărbații om cu om, de la douăzeci de ani în sus, toți cei buni de oaste,
\par 37 S-au numărat în seminția lui Veniamin treizeci și cinci de mii patru sute.
\par 38 Fiii lui Dan, după seminția lor, după neamurile lor, după familiile lor, după numele lor, toți bărbații om cu om, de la douăzeci de ani în sus, toți cei buni de oaste,
\par 39 S-au numărat în seminția lui Dan, șaizeci și două de mii șapte sute.
\par 40 Fiii lui Așer, după seminția lor, după neamurile lor, după familiile lor, după numele lor, toți bărbații om cu om, de la douăzeci de ani în sus, toți cei buni de oaste,
\par 41 S-au numărat în seminția lui Așer patruzeci și una de mii cinci sute.
\par 42 Fiii lui Neftali, după seminția lor, după neamurile lor, după familiile lor, după numele lor, toți bărbații om cu om, de la douăzeci de ani în sus, toți cei buni de oaste,
\par 43 S-au numărat în seminția lui Neftali cincizeci și trei de mii patru sute.
\par 44 Aceștia sunt cei care au intrat la numărătoarea făcută de Moise și Aaron și de cei doisprezece bărbați, căpeteniile lui Israel, câte un bărbat de fiecare seminție, după neamul strămoșesc.
\par 45 Deci toți fiii lui Israel de la douăzeci de ani în sus, buni de oaste, care au intrat la numărătoare, după familiile lor,
\par 46 Au fost șase sute trei mii cinci sute cincizeci.
\par 47 Iar leviții, după seminția părinților lor, n-au fost numărați între ei.
\par 48 Și a grăit Domnul cu Moise și a zis:
\par 49 "Vezi ca seminția lui Levi să n-o bagi la numărătoare și să nu-i numeri pe fiii lui Levi cu fiii lui Israel;
\par 50 Ci rânduiește pe leviți la cortul adunării și le încredințează toate lucrurile lui și toate câte sunt în el. Ei să poarte cortul și toate lucrurile lui, să slujească în el și să își așeze tabăra împrejurul lui.
\par 51 Când va fi să plece cortul, leviții să-l strângă, și când va fi să se oprească, leviții să-l așeze; iar de se va apropia unul străin, să fie omorât.
\par 52 Fiii lui Israel să poposească fiecare în tabăra sa și fiecare sub steagul său și în cetele lor.
\par 53 Iar leviții să-și așeze tabăra aproape, împrejurul cortului adunării, ca să nu vină mânia asupra obștii fiilor lui Israel; și să străjuiască leviții la cortul adunării".
\par 54 Și au făcut fiii lui Israel toate câte poruncise Domnul lui Moise și Aaron; așa au făcut.

\chapter{2}

\par 1 Atunci a grăit Domnul cu Moise și cu Aaron și a zis:
\par 2 "Fiii lui Israel să poposească fiecare lângă steagul său, în preajma semnelor familiei sale, și să-și așeze taberele înaintea cortului mărturiei și împrejurul lui.
\par 3 Întâi, spre răsărit, să poposească steagul taberei lui Iuda, cu cetele sale, cu Naason, fiul lui Aminadab, căpetenia fiilor lui Iuda,
\par 4 Și cu oștenii săi în număr de șaptezeci și patru de mii șase sute.
\par 5 Alături să poposească seminția lui Isahar, cu Natanael, fiul lui Țuar, căpetenia fiilor lui Isahar,
\par 6 Și cu oștenii săi în număr de cincizeci și patru de mii patru sute.
\par 7 Mai departe va poposi seminția lui Zabulon, cu Eliab, fiul lui Helon, căpetenia fiilor lui Zabulon,
\par 8 Cu oștenii săi în număr de cincizeci și șapte de mii patru sute.
\par 9 Toți aceștia în număr de o sută optzeci și șase de mii patru sute, care țin de tabăra lui Iuda, să plece întâi.
\par 10 Spre miazăzi să se așeze tabăra lui Ruben, cu cetele sale și Elițur, fiul lui Ședeur, căpetenia fiilor lui Ruben,
\par 11 Și cu oștenii săi în număr de patruzeci și șase de mii cinci sute.
\par 12 Lângă el va poposi seminția lui Simeon, cu Șelumiel, fiul lui Țurișadai, căpetenia fiilor lui Simeon,
\par 13 Și cu oștenii săi în număr de cincizeci și nouă de mii trei sute.
\par 14 După acesta va poposi seminția lui Gad, cu Eliasaf, fiul lui Raguel, căpetenia fiilor lui Gad,
\par 15 Și cu oștenii săi în număr de patruzeci și cinci de mii șase sute cincizeci.
\par 16 Toți aceștia cu luptătorii lor în număr de o sută cincizeci și una de mii patru sute cincizeci, care țin de tabăra lui Ruben, rânduiți în tabere, vor pleca în rândul al doilea.
\par 17 După aceea, când va pleca cortul adunării, tabăra leviților va fi în mijlocul taberelor și precum au poposit așa să și plece, fiecare la rândul său și sub steagul său.
\par 18 Spre apus va poposi tabăra lui Efraim cu cetele sale și cu Elișama, fiul lui Amihud, căpetenia fiilor lui Efraim,
\par 19 Și cu oștenii săi în număr de patruzeci de mii cinci sute.
\par 20 Lângă ea se va așeza seminția lui Manase cu Gamaliel, fiul lui Pedațur, căpetenia fiilor lui Manase,
\par 21 Și cu oștenii săi în număr de treizeci și două de mii două sute.
\par 22 După acesta seminția lui Veniamin cu Abidan, fiul lui Ghedeon, căpetenia fiilor lui Veniamin,
\par 23 Și cu oștenii lui în număr de treizeci și cinci de mii patru sute.
\par 24 Toți aceștia cu luptătorii lor în număr de o sută opt mii o sută, care țin de tabăra lui Efraim, vor pleca în al treilea rând, așezați în cete.
\par 25 La miazănoapte se va așeza tabăra lui Dan cu cetele sale și cu Ahiezer, fiul lui Amișadai, căpetenia fiilor lui Dan,
\par 26 Și cu oștenii săi în număr de șaizeci și două de mii șapte sute.
\par 27 Lângă el își va așeza tabăra seminția lui Așer, cu Paghiel, fiul lui Ocran, căpetenia fiilor lui Așer,
\par 28 Și cu oștenii săi în număr de patruzeci și una de mii cinci sute.
\par 29 Mai departe își va așeza tabăra seminția lui Neftali cu Ahira, fiul lui Enan, căpetenia fiilor lui Neftali,
\par 30 Și cu oștenii săi în număr de cincizeci și trei de mii patru sute.
\par 31 Toți aceștia cu luptătorii lor în număr de o sută cincizeci și șapte de mii șase sute, care țin de tabăra lui Dan, să plece la urmă sub steagurile lor și rânduiți în cete".
\par 32 Aceștia sunt fiii lui Israel care au intrat la numărătoare după familiile lor. Toți, câți au intrat la numărătoare pe tabere și pe cete, erau șase sute trei mii cinci sute cincizeci.
\par 33 Iar leviții nu s-au numărat cu ei, după cum poruncise Domnul lui Moise.
\par 34 Și au făcut fiii lui Israel toate câte poruncise Domnul lui Moise: așa se așezau în tabere sub steagurile lor și așa purcedeau fiecare cu seminția sa și cu familia sa.

\chapter{3}

\par 1 Iată acum spița neamului lui Aaron și a lui Moise, din timpul când a grăit Domnul cu Moise pe Muntele Sinai, și iată numele fiilor lui Aaron:
\par 2 Nadab, întâiul născut, Abiud, Eleazar și Itamar.
\par 3 Acestea sunt numele fiilor lui Aaron, preoți miruiți, care au fost sfințiți, ca să slujească cele ale preoției.
\par 4 Insă Nadab ș: Abiud au murit înaintea feței Domnului, când au adus foc străin înaintea feței Domnului în pustiul Sinai, neavând copii, și au rămas preoți numai Eleazar și Itamar cu tatăl lor Aaron.
\par 5 Atunci a grăit Domnul cu Moise și a zis:
\par 6 "Ia seminția lui Levi și o pune la îndemâna lui Aaron preotul, ca să-l ajute în slujba lui.
\par 7 Să fie de pază în locul lui și în locul fiilor lui Israel la cortul adunării; să facă slujbele la cort;
\par 8 Să păstreze toate lucrurile cortului adunării, să străjuiască în locul fiilor lui Israel și să facă slujbele la cort.
\par 9 Dă pe leviți la îndemâna lui Aaron, fratele tău, și fiilor lui, preoților; să-Mi fie dăruiți Mie dintre fiii lui Israel.
\par 10 Iar lui Aaron și fiilor lui încredințează-le cortul adunării, ca să-și păzească datoria lor preoțească și toate cele de la jertfelnic și de după perdea; iar de se va apropia cineva străin, să fie omorât".
\par 11 Și a grăit Domnul cu Moise și a zis:
\par 12 "Iată, Eu am luat din fiii lui Israel pe leviți în locul tuturor întâilor născuți, în locul tuturor celor ce se nasc întâi în Israel și aceia vor fi în locul acestora.
\par 13 Leviții să fie ai Mei, căci toți întâi-născuții sunt ai Mei. În ziua când am lovit pe toți întâi-născuții în pământul Egiptului, atunci Mi-am sfințit pe toți întâi-născuții lui Israel de la om până la dobitoc și aceștia să fie ai Mei. Eu  sunt Domnul".
\par 14 Iarăși a grăit Domnul cu Moise, în pustiul Sinai, și a zis:
\par 15 "Numără pe fiii lui Levi, după familiile lor, după neamurile lor; pe toți cei de parte bărbătească, de la o lună în sus să-i numeri".
\par 16 Și i-au numărat Moise și Aaron, după cuvântul Domnului, cum le poruncise Domnul.
\par 17 Iată dar care sunt fiii lui Levi, după numele lor: Gherșon, Cahat și Merari.
\par 18 Iar numele fiilor lui Gherșon, după neamurile lor, sunt: Libni și Șimei.
\par 19 Fiii lui Cahat, după neamurile lor, sunt: Amram, Ițhar, Hebron și Uziel.
\par 20 Fiii lui Merari, după neamurile lor, sunt: Mahli și Muși. Acestea sunt neamurile lui Levi, după familiile lor.
\par 21 Din Gherșon au ieșit neamul lui Libni și neamul lui Șimei: aceste neamuri sunt din Gherșon.
\par 22 Socotindu-se la număr tot cel de parte bărbătească, de la o lună în sus, s-au numărat în neamul lui Gherșon șapte mii cinci sute.
\par 23 Fiii lui Gherșon trebuia să se așeze cu tabăra în urma cortului, spre asfințit.
\par 24 Eliasaf, fiul lui Lael, era căpetenia familiei fiilor lui Gherșon.
\par 25 Fiii lui Gherșon la cortul adunării aveau să păzească cortul și acoperișul lui, perdeaua de la ușa cortului adunării,
\par 26 Perdelele curții, perdeaua de la intrarea curții celei dimprejurul cortului și jertfelnicului, frânghiile și toate uneltele lor.
\par 27 Din Cahat a ieșit familia lui Amram, familia lui Ițhar, familia lui Hebron și familia lui Uziel: aceste familii sunt din Cahat.
\par 28 Socotindu-se la număr tot cel de parte bărbătească, de la o lună în sus, s-au numărat în neamul acesta opt mii trei sute. Ei păzeau locașul sfânt.
\par 29 Familiile fiilor lui Cahat trebuia să-și așeze tabăra lângă cort, în partea de miazăzi;
\par 30 Iar căpetenie în familiile neamului lui Cahat era Elțafan, fiul lui Uziel.
\par 31 În paza lor se afla chivotul, masa, sfeșnicul, jertfelnicul, vasele sfinte, care se întrebuințează la slujbe, și perdeaua cu toate ale ei.
\par 32 Căpetenie peste căpeteniile leviților era Eleazar, fiul preotului Aaron , care era rânduit să privegheze pe cei ce aveau în păstrare locașul sfânt.
\par 33 Din Merari au ieșit familia lui Mahli și familia lui Muși; aceste familii sunt din Merari.
\par 34 Socotindu-se la număr tot cel de parte bărbătească de la o lună în sus, s-au numărat în neamul acesta șase mii două sute;
\par 35 Iar căpetenie peste familiile din neamul lui Merari era Țuriel, fiul lui Abihael. Aceștia trebuia să-și așeze tabăra lângă cort, în partea de miazănoapte.
\par 36 În paza fiilor lui Merari s-au rânduit scândurile dimprejurul cortului, pârghiile lui, stâlpii lui, postamentele acestora, și toate lucrurile și uneltele lor,
\par 37 Stâlpii curții din toate părțile ei, postamentele lor, țărușii curții și frânghiile ei.
\par 38 Iar în partea de dinainte a cortului adunării, spre răsărit, trebuia să-și așeze tabăra Moise și Aaron și fiii acestuia, cărora li se încredințase paza locașului sfânt în locul fiilor lui Israel. Iar de se va apropia vreun străin, să fie omorât.
\par 39 Deci toți leviții numărați, pe care i-au numărat Moise și Aaron, cum poruncise Domnul, după neamurile lor, parte bărbătească, de la o lună în sus, au fost douăzeci și două de mii.
\par 40 Apoi a zis Domnul către Moise: "Socotește pe tot bărbatul întâi-născut dintre fiii lui Israel, de la o lună în sus, și ia numărul numelor lor.
\par 41 Și în locul tuturor întâi-născuților ai fiilor lui Israel, să iei pentru Mine pe leviți. Eu sunt Domnul. Și vitele leviților să le iei în locul a tot întâi-născutului din vitele fiilor lui Israel".
\par 42 Și a numărat Moise, după cum îi poruncise Domnul, pe toți întâi-născuții dintre fiii lui Israel;
\par 43 Și întâi-născuții de parte bărbătească de la o lună în sus, după numărul numelor, au fost toți douăzeci și două de mii două sute șaptezeci și trei.
\par 44 Și a grăit Domnul cu Moise și a zis:
\par 45 "Ia pe leviți în locul tuturor întâi-născuților fiilor lui Israel și vitele leviților în locul vitelor lor, și să fie leviții ai Mei. Eu sunt Domnul!
\par 46 Iar ca răscumpărare pentru cei două sute șaptezeci și trei de întâi-născuți ai fiilor lui Israel, care trec peste numărul leviților,
\par 47 Să iei câte cinci sicli de cap, socotind câte douăzeci de ghere într-un siclu, după siclul sfânt,
\par 48 Și argintul acesta să-l dai lui Aaron și fiilor lui, ca răscumpărare pentru cei ce prisosesc peste numărul lor".
\par 49 Și adunând Moise argintul de răscumpărare pentru întâi-născuții lui Israel, care treceau peste numărul leviților,
\par 50 S-au găsit o mie trei sute șaizeci și cinci de sicli, după siclul sfânt.
\par 51 Și a dat Moise argintul de răscumpărare, pentru cei ce prisoseau, lui Aaron și fiilor lui, după cuvântul Domnului, precum poruncise Domnul lui Moise.

\chapter{4}

\par 1 Și a grăit Domnul cu Moise și cu Aaron și a zis:
\par 2 "Numără din fiii lui Levi pe fiii lui Cahat, după neamurile și după familiile lor, pe toți cei buni de slujbă,
\par 3 De la treizeci de ani în sus până la cincizeci de ani, ca să lucreze la cortul adunării.
\par 4 Slujba fiilor lui Cahat la cortul adunării va fi să ducă sfânta sfintelor.
\par 5 Când va pleca tabăra, să intre Aaron și fiii lui, să ia perdeaua despărțitoare între sfânta și sfânta sfintelor, să învelească cu ea chivotul legii;
\par 6 Să pună apoi un acoperământ de piei vinete, iar pe deasupra aceluia să arunce un înveliș de lână albastră și să pună pârghiile la chivot.
\par 7 Apoi să aștearnă pe masa pâinilor punerii înainte o față de masă violetă și să pună pe ea blidele, talerele, oalele și cupele cele pentru turnat, și pâinile ei pururea să fie pe ea.
\par 8 Peste acestea să pună o poală purpurie, iar pe deasupra ei să pună un acoperământ de piele vânătă și să-i pună pârghiile.
\par 9 Să ia apoi o îmbrăcăminte violetă și să acopere sfeșnicul și candelele lui, cleștele lui, plutele lui, și toate vasele cele pentru untdelemn, care se întrebuințează la el.
\par 10 Să-l acopere pe el și toate uneltele lui cu un acoperământ de piei vinete și să-l pună pe năsălie.
\par 11 Peste jertfelnicul cel de aur să pună o îmbrăcăminte violetă, să-l acopere cu un acoperământ de piei vinete și apoi să-i așeze pârghiile în verigi.
\par 12 Să ia toate lucrurile cele pentru slujbă, care se întrebuințează la slujbă în locașul sfânt, și să le pună în învelișuri de lână violetă, să le acopere cu acoperăminte de piei vinete și să le pună pe năsălie.
\par 13 După aceea să curețe jertfelnicul de cenușă, să-l acopere cu o îmbrăcăminte violetă,
\par 14 Să pună pe el toate vasele lui, care se întrebuințează la el în timpul slujbei: cleștele, furculițele, lopețile, oalele și toate vasele jertfelnicului, să-l acopere cu un acoperământ de piei vinete și să-i pună pârghiile. Să ia apoi o îmbrăcăminte violetă și să acopere baia și postamentul ei; să pună pe deasupra lor un acoperământ vânăt de piele și să le pună pe năsălie.
\par 15 După aceea Aaron și fiii lui, înainte de plecarea taberei la drum, vor strânge tot cortul și vor înveli toate lucrurile locașului sfânt, iar fiii lui Cahat vor veni să le ia; dar nu trebuie să se atingă ei de sfânta sfintelor, ca să nu moară. Aceste lucruri ale cortului să le ducă fiii lui Cahat.
\par 16 Eleazar, fiul preotului Aaron, va fi supraveghetor peste untdelemnul pentru sfeșnic, aromatele de tămâiat, darul zilnic de pâine și mirul; va avea și supraveghere peste tot cortul și peste câte sunt în el și în locașul sfânt și peste toate lucrurile".
\par 17 Și a grăit Domnul cu Moise și cu Aaron și a zis:
\par 18 "Să nu lăsați să se stingă sămânța neamului lui Cahat dintre leviți.
\par 19 Iată ce trebuie să le faceți, ca să trăiască și să nu moară, când se vor apropia de sfânta sfintelor: să vină Aaron și fiii lui și să pună pe fiecare la slujba lui și la sarcina lui;
\par 20 Dar ei să nu vină să privească la cele sfinte, când le învelesc, ca să nu moară".
\par 21 Și a grăit Domnul cu Moise și a zis:
\par 22 "Numără și pe fiii lui Gherșon, pe familii și pe neamuri, de la treizeci de ani până la cincizeci de ani;
\par 23 Și să numeri pe toți cei buni de slujbă, ca să lucreze la cortul adunării.
\par 24 Iată slujba familiilor lui Gherșon, adică ce au de făcut și de dus:
\par 25 Să ducă acoperișurile cortului, cortul adunării, acoperișul lui, acoperișul cel de piei vinete, care e pe deasupra lor, perdeaua, care se atârnă la ușa cortului adunării,
\par 26 Perdeaua de la poarta curții, pânzele curții celei dimprejurul cortului și a jertfelnicului, frânghiile lor și toate lucrurile lor de slujbă și tot ce este de făcut la ele să facă ei.
\par 27 Toate slujbele fiilor lui Gherșon la ducerea poverilor și la toate lucrările lor trebuie să se facă după porunca lui Aaron și a fiilor lui și lor să le încredințați spre păstrare tot ceea ce au ei de dus.
\par 28 Acestea sunt slujbele fiilor lui Gherșon la cortul adunării și acestea li se vor încredința spre păstrare sub. supravegherea lui Itamar, fiul preotului Aaron.
\par 29 Pe fiii lui Merari, iar să-i numeri, după neamurile și după familiile lor,
\par 30 De la treizeci de ani în sus până la cincizeci de ani. Să numeri pe toți cei buni de slujbă, ca să lucreze la cortul adunării.
\par 31 Iată ce să ducă ei, după slujba lor la cortul adunării: scândurile cortului cu pârghiile lor, stâlpii lui cu postamentele lor, funiile cortului cu țărușii lor;
\par 32 Stâlpii curții de pe toate laturile ei cu postamentele lor, țărușii curții cu frânghiile lor, toate uneltele lor și tot ce ține de ele. Să numărați pe nume toate lucrurile ce sunt datori să ducă.
\par 33 Aceasta-i slujba neamului fiilor lui Merari și tot ce au să facă la cortul adunării, sub supravegherea lui Itamar, fiul preotului Aaron".
\par 34 Atunci au numărat Moise și Aaron cu căpeteniile obștii pe fiii lui Cahat după neamurile și după familiile lor,
\par 35 De la treizeci de ani în sus până la cincizeci de ani, pe toți cei buni de slujbă, ea să lucreze la cortul adunării.
\par 36 Și s-au găsit la numărătoare, după familiile lor, două mii șapte sute cincizeci.
\par 37 Acesta este numărul fiilor lui Cahat, toți cei buni de slujbă la cortul adunării, pe care i-au numărat Moise și Aaron, după porunca Domnului, dată prin Moise.
\par 38 S-au numărat apoi fiii lui Gherșon, după neamurile și după familiile lor,
\par 39 De la treizeci de ani în sus, până la cincizeci de ani, toți cei buni de slujbă, ca să lucreze la cortul adunării.
\par 40 Și s-au găsit la numărătoare, după neamurile și după familiile lor, două mii șase sute treizeci.
\par 41 Acesta este numărul fiilor lui Gherșon, toți cei buni de slujbă la cortul adunării, pe care i-au numărat Moise și Aaron, după porunca Domnului.
\par 42 S-a numărat după aceea și neamul fiilor lui Merari, după rudeniile și după familiile lor,
\par 43 De la treizeci de ani în sus, până la cincizeci de ani, toți cei buni de slujbă, ca să lucreze la cortul adunării.
\par 44 Și s-au găsit la numărătoare, după neamul lor și după familii, trei mii două sute.
\par 45 Acesta este numărul fiilor lui Merari, pe care i-au numărat Moise și Aaron, după porunca Domnului, dată prin Moise.
\par 46 Toți leviții, numărați de Moise și de Aaron și de căpeteniile lui Israel, după neamurile și după familiile lor,
\par 47 De la treizeci de ani în sus, până la cincizeci de ani, toți cei buni de slujbă, ca să lucreze la cortul adunării și să-l ducă,
\par 48 S-au găsit la numărătoare opt mii cinci sute optzeci.
\par 49 Și după porunca Domnului, dată prin Moise, s-au rânduit fiecare la lucrul său și la slujba sa, și au fost numărați, cum poruncise Domnul lui Moise.

\chapter{5}

\par 1 Și a grăit Domnul cu Moise și a zis:
\par 2 "Poruncește fiilor lui Israel să scoată din tabără pe toți leproșii, pe toți cei ce au scurgere și pe toți cei întinați prin atingere de mort.
\par 3 De la bărbat până la femeie să-i scoateți și să-i trimiteți afară din tabără, ca să nu pângărească taberele lor, în mijlocul cărora locuiesc Eu".
\par 4 Și au făcut așa fiii lui Israel: i-au scos afară din tabără. Cum poruncise Domnul lui Moise, așa au făcut fiii lui Israel.
\par 5 Și a grăit Domnul lui Moise și a zis:
\par 6 "Spune fiilor lui Israel: Dacă un bărbat sau o femeie va face vreun păcat față de un om, și prin aceasta vă păcătui împotriva Domnului și va fi vinovat sufletul acela,
\par 7 Să-și mărturisească păcatul ce a făcut și să întoarcă deplin aceea prin ce a păcătuit și să mai adauge la aceea a cincea parte și să dea aceluia față de care a păcătuit.
\par 8 Dacă însă omul acela nu va avea moștenitor, căruia să se dea cele pentru greșeală, atunci să le dea Domnului și vor fi ale preotului, pe lângă berbecul de curățire, cu care acesta îl va curăți.
\par 9 Toată pârga din toate darurile fiilor lui Israel, pe care le aduc ei la preot, să fie ale lui.
\par 10 Orice lucru afierosit să fie al lui; și orice va da cineva preotului este al lui".
\par 11 Și a grăit Domnul lui Moise și a zis:
\par 12 "Grăiește fiilor lui Israel și zi către ei: De va greși femeia unui bărbat și-l va înșela,
\par 13 Și va dormi cineva cu ea în pat, și lucrul va fi ascuns de bărbatul ei, și ea se va spurca pe ascuns, și nu vor fi martori împotriva ei, nici nu va fi prinsă asupra faptului;
\par 14 De va cădea asupra bărbatului duhul îndoielii bănuind pe femeia sa, vinovată fiind aceasta, sau de va cădea asupra lui duhul îndoielii și va bănui femeia sa, nevinovată fiind:
\par 15 Să-și aducă bărbatul femeia sa la preot și să aducă jertfă pentru ea a zecea parte de efă de făină de orz, dar să nu toarne deasupra untdelemn, nici să pună tămâie, pentru că acesta este dar de bănuială, dar de amintire, care amintește vinovăția;
\par 16 Iar preotul să o aducă și să o pună înaintea Domnului.
\par 17 Apoi să ia preotul apă curată de izvor într-un vas de lut, să ia țărână din pământ de dinaintea cortului adunării și să o pună în apă.
\par 18 După aceea să pună preotul femeia înaintea Domnului, să descopere capul femeii și să-i dea în mâini darul de pomenire, darul de bănuială, iar preotul să aibă în mâini apa cea amară, care aduce blestemul.
\par 19 Apoi să jure preotul femeia și să-i zică: Dacă n-a dormit nimeni cu tine și tu nu te-ai spurcat și n-ai călcat credincioșia către bărbatul tău, nevătămată să fii de această apă amară care aduce blestem;
\par 20 Iar de te-ai abătut, fiind măritată, și te-ai spurcat, de a dormit cineva cu tine, afară de bărbatul tău,
\par 21 Atunci să dea Domnul să fii de blestem și de ocară în poporul tău; să facă Domnul ca sânul tău să cadă și să se umfle pântecele tău.
\par 22 Și apa aceasta, care aduce blestem, să intre înăuntrul tău, ca să ti se umfle pântecele și să-ți cadă sânul tău. Iar femeia să zică: Amin, amin!
\par 23 Apoi să scrie preotul jurămintele acestea pe hârtie, să le moaie în apa cea amară,
\par 24 Și să dea femeii să bea apa amară aducătoare de blestem, și va înghiți ea apa aducătoare de blestem spre vătămarea ei.
\par 25 După aceea să ia preotul din mâinile femeii darul de pâine cel pentru bănuială și să ridice acest dar înaintea Domnului și să-l ducă la jertfelnic.
\par 26 Să ia apoi preotul cu pumnul o parte din darul de amintire, s-o ardă pe jertfelnic și după aceasta să dea femeii să bea apa.
\par 27 După ce va bea apa cea amară a blestemului, dacă ea va fi necurată și dacă va fi înșelat pe bărbatul său, se va umfla pântecele ei și sânul ei va cădea și va fi femeia aceea blestemată în poporul său.
\par 28 Iar dacă femeia nu s-a spurcat, ci va fi curată, nevătămată va rămâne și va naște copii.
\par 29 Aceasta este rânduiala pentru femeia bănuită, care, fiind măritată, s-ar abate și s-ar spurca,
\par 30 Sau pentru omul, asupra căruia ar cădea duhul geloziei și ar bănui pe femeia sa. Atunci să pună el pe femeie înaintea feței Domnului și să facă preotul cu ea după legea aceasta.
\par 31 Și va fi bărbatul curat de păcat, iar femeia aceea își va purta păcatul ei".

\chapter{6}

\par 1 Și a grăit Domnul cu Moise și a zis:
\par 2 "Vorbește fiilor lui Israel și zi către ei: Dacă bărbat sau femeie va hotărî să dea făgăduință de nazireu, ca să se afierosească nazireu Domnului,
\par 3 Să se ferească de vin și de sicheră; oțet de vin și oțet de sicheră să nu bea și nimic din cele făcute din struguri să nu bea; nici struguri proaspeți sau uscați să nu mănânce.
\par 4 În toate zilele, cât va fi nazireu, să nu mănânce, nici să bea vreo băutură făcută din struguri, de la sâmbure până la pielită.
\par 5 În toate zilele făgăduinței sale de nazireu să nu treacă brici pe capul său; până la împlinirea zilelor, câte a afierosit Domnului, este sfânt și trebuie să crească părul pe capul lui.
\par 6 În toate zilele, pentru care s-a afierosit pe sine să fie nazireul Domnului, să nu se apropie de trup mort:
\par 7 Când va muri tatăl său, sau mama sa, sau fratele său, sau sora sa, să nu se spurce prin atingerea de ei, pentru că afierosirea lui Dumnezeu este pe capul lui.
\par 8 În toate zilele cât va fi nazireu, este sfântul Domnului.
\par 9 De va muri însă cineva lângă el fără de veste și de năprasnă, și prin aceasta își va întina capul său de nazireu, să-și tundă capul său în ziua curățirii sale;
\par 10 În ziua a șaptea să se tundă, iar în ziua a opta să aducă preotului două turturele sau doi pui de porumbel, la ușa cortului adunării,
\par 11 Și preotul să aducă o pasăre jertfă pentru păcat, iar pe cealaltă ardere de tot, și să-l curețe de spurcarea cea prin atingerea de trupul mort și să-i sfințească în ziua aceea capul lui.
\par 12 Apoi să-și înceapă din nou zilele sale de nazireu, afierosite Domnului, și să aducă un berbec de un an jertfă de iertare, iar zilele dinainte sunt pierdute, pentru că nazireatul a fost întinat.
\par 13 Iată legea cea pentru nazireu: când se vor împlini zilele lui de nazireu, să se aducă la ușa cortului adunării;
\par 14 Să aducă darul său Domnului: un miel de un an, fără meteahnă, ardere de tot; o mioară de un an, fără meteahnă, jertfă pentru păcat, și un berbec de un an, fără meteahnă, jertfă de împăcare,
\par 15 Și un paner cu azime de făină de grâu, frământate cu untdelemn, și cu turte nedospite, unse cu untdelemn, cu darul lor de pâine și cu turnarea lor.
\par 16 Pe acestea le va înfățișa preotul înaintea Domnului, va săvârși jertfa lui pentru păcat și arderea de lot a lui.
\par 17 Berbecul îl va aduce Domnului jertfă de împăcare cu panerul cel de azime; și va aduce preotul prinosul lui de pâine și turnarea lui.
\par 18 Și își va tunde nazireul la intrarea cortului adunării capul său de nazireu și va lua părul capului său de nazireu și-l va pune pe focul cel de sub jertfa de împăcare.
\par 19 Apoi va lua preotul șoldul cel fiert al berbecului, o pâine nedospită și o turtă nedospită din paner și le va pune nazireului pe mâini, după ce acesta și-a tuns capul de nazireu,
\par 20 Și să înalțe preotul acestea, legănându-le înaintea Domnului. Această sfințenie să fie a preotului pe lângă pieptul legănat și pe lângă șoldul înălțat. După aceasta nazireul poate să bea vin.
\par 21 Iată rânduiala cea pentru nazireul care a dat făgăduință și jertfa ce trebuie să aducă el Domnului pentru nazireatul său, pe lângă ceea ce-i îngăduiesc mijloacele lui. După făgăduința sa, pe care o va da, așa să facă, după cele legiuite pentru nazireatul său".
\par 22 Și a grăit Domnul cu Moise și a zis:
\par 23 "Spune lui Aaron și fiilor lui și le zi: Așa să binecuvântați pe fiii lui Israel și să ziceți către ei:
\par 24 Să te binecuvânteze Domnul și să te păzească!
\par 25 Să caute Domnul asupra ta cu față veselă și să te miluiască!
\par 26 Să-Și întoarcă Domnul fala Sa către tine și să-li dăruiască pace!
\par 27 Așa să cheme numele Meu asupra fiilor lui Israel și Eu, Domnul, îi voi binecuvânta".

\chapter{7}

\par 1 Când a așezat Moise cortul și l-a miruit și l-a sfințit pe el și toate lucrurile lui, jertfelnicul și toate obiectele lui, și le-a miruit și le-a sfințit,
\par 2 Atunci au venit cele douăsprezece căpetenii ale lui Israel, capii familiilor lor, mai-marii semințiilor, care supravegheaseră numărătoarea,
\par 3 Și au adus Domnului darurile lor, șase care acoperite și doisprezece boi, câte un car de fiecare două căpetenii și câte un bou de fiecare căpetenie și le-au adus înaintea cortului.
\par 4 A grăit Domnul lui Moise zicând:
\par 5 "Primește-le de la ei, ca să fie pentru facerea lucrărilor trebuitoare la cortul adunării și le dă leviților, potrivit cu felul slujbei fiecăruia".
\par 6 Și Moise, luând carele și boii, le-a dat leviților:
\par 7 Două care și patru boi a dat fiilor lui Gherșon, după slujba lor;
\par 8 Patru care și opt boi a dat fiilor lui Merari, după slujba lor, sub povața lui Itamar, fiul lui Aaron, preotul.
\par 9 Iar fiilor lui Cahat nu le-a dat, pentru că slujba lor era de a duce lucrurile sfinte, pe care trebuia să le poarte pe umeri.
\par 10 Au mai adus căpeteniile jertfe pentru sfințirea jertfelnicului, în ziua miruirii lui, și au înfățișat căpeteniile prinoasele lor înaintea jertfelnicului.
\par 11 Atunci a zis Domnul către Moise: "Câte o căpetenie pe fiecare zi să aducă prinosul său pentru sfințirea jertfelnicului".
\par 12 În ziua întâi a adus darul său Naason, fiul lui Aminadab, căpetenia seminției lui Iuda.
\par 13 și darul lui a fost: un blid de argint în greutate de o sută treizeci de sicli și o cupă de argint de șaptezeci de sicli, după siclul sfânt, amândouă pline cu făină de grâu, amestecată cu untdelemn, pentru jertfă;
\par 14 O cădelniță de aur de zece sicli, plină cu miresme;
\par 15 Un vițel, un berbec și un miel de un an pentru ardere de tot;
\par 16 Un țap, jertfă pentru păcat;
\par 17 Iar ca jertfă de împăcare: doi boi, cinci berbeci, cinci țapi, cinci miei de un an. Acestea au fost darurile lui Naason, fiul lui Aminadab.
\par 18 În ziua a doua a adus Natanael, fiul lui Țuar, căpetenia seminției lui Isahar.
\par 19 Acesta a adus dar din partea sa: un blid de argint în greutate de o sută treizeci de sicli și o cupă de argint de șaptezeci de sicli, după siclul sfânt, amândouă pline cu făină de grâu, amestecată cu untdelemn, pentru jertfă;
\par 20 O cădelniță de aur de zece sicli, plină cu miresme;
\par 21 Un vițel, un berbec și un miel de un an pentru ardere de tot;
\par 22 Un țap, jertfă pentru păcat;
\par 23 Iar ca jertfă de împăcare: doi boi, cinci berbeci, cinci țapi și cinci miei de un an. Acestea au fost darurile lui Natanael, fiul lui Țuar.
\par 24 În ziua a treia a adus căpetenia fiilor lui Zabulon, Eliab, fiul lui Helon.
\par 25 Darurile lui au fost: un blid de argint în greutate de o sută treizeci  de sicli și o cupă de argint de șaptezeci de sicli, după siclul sfânt, amândouă pline cu făină de grâu, amestecată cu untdelemn, pentru jertfă;
\par 26 O cădelniță de aur de zece sicli, plină cu miresme;
\par 27 Un vițel, un berbec și un miel de un an ardere de tot;
\par 28 Un țap, jertfă pentru păcat;
\par 29 Iar pentru jertfa de împăcare: doi boi, cinci berbeci, cinci țapi și cinci miei de un an. Acestea sunt darurile lui Eliab, fiul lui Helon.
\par 30 În ziua a patra a adus căpetenia fiilor lui Ruben, Elițur, fiul lui Ședeur.
\par 31 Darurile lui au fost: un blid de argint în greutate de o sută treizeci de sicli și o cupă de argint de șaptezeci de sicli, după siclul sfânt, amândouă pline cu făină de grâu, amestecată cu untdelemn, pentru jertfă;
\par 32 O cădelniță de aur de zece sicli, plină cu miresme;
\par 33 Un vițel, un berbec și un miel de un an, pentru ardere de tot;
\par 34 Un țap, jertfă pentru păcat;
\par 35 Iar pentru jertfa de împăcare: doi boi, cinci berbeci, cinci țapi și cinci miei de un an. Acestea sunt darurile lui Elițur, fiul lui Ședeur.
\par 36 În ziua a cincea a adus căpetenia fiilor lui Simeon, Șelumiel, fiul lui Țurișadai.
\par 37 Darurile lui au fost: un blid de argint în greutate de o sută treizeci de sicli și o cupă de argint de șaptezeci de sicli, după siclul sfânt, amândouă pline cu făină de grâu, amestecată cu untdelemn, pentru jertfă;
\par 38 O cădelniță de aur de zece sicli, plină cu miresme;
\par 39 Un vițel, un berbec și un miel de un an, pentru ardere de tot;
\par 40 Un țap, jertfă pentru păcat;
\par 41 Iar pentru jertfa de împăcare: doi boi, cinci berbeci, cinci țapi și cinci miei de un an. Acestea sunt darurile lui Șelumiel, fiul lui Țurișadai.
\par 42 În ziua a șasea a adus căpetenia fiilor lui Gad, Eliasaf, fiul lui Raguel.
\par 43 Darurile lui au fost: un blid de argint în greutate de o sută treizeci de sicli și o cupă de argint de șaptezeci sicli, după siclul sfânt, amândouă pline cu făină de grâu, amestecată cu untdelemn, pentru jertfă;
\par 44 O cădelniță de aur de zece sicli, plină cu miresme;
\par 45 Un vițel, un berbec și un miel de un an pentru ardere de tot;
\par 46 Un țap, jertfă pentru păcat;
\par 47 Iar pentru jertfa de împăcare: doi boi, cinci berbeci, cinci țapi și cinci miei de un an. Acestea sunt darurile lui Eliasaf, fiul lui Raguel.
\par 48 În ziua a șaptea a adus căpetenia fiilor lui Efraim, Elișama, fiul lui Amihud.
\par 49 Darurile lui au fost: un blid de argint în greutate de o sută treizeci de sicli și o cupă de argint de șaptezeci de sicli, după siclul sfânt, amândouă pline cu făină de grâu, amestecată cu untdelemn, pentru jertfă;
\par 50 O cădelniță de aur de zece sicli, plină cu miresme;
\par 51 Un vițel, un berbec și un miel de un an pentru ardere de tot;
\par 52 Un țap, jertfă pentru păcat;
\par 53 Iar pentru jertfa de împăcare: doi boi, cinci berbeci, cinci țapi și cinci miei de un an. Acestea sunt darurile lui Elișama, fiul lui Amihud.
\par 54 În ziua a opta a adus căpetenia fiilor lui Manase, Gamaliel, fiul lui Pedațur.
\par 55 Darurile lui au fost: un blid de argint în greutate de o sută treizeci de sicli și o cupă de argint de șaptezeci sicli, după siclul sfânt, amândouă pline cu făină de grâu, amestecată cu untdelemn, pentru jertfă;
\par 56 O cădelniță de aur de zece sicli, plină cu miresme;
\par 57 Un vițel, un berbec și un miel de un an pentru ardere de tot;
\par 58 Un țap, jertfă pentru păcat;
\par 59 Iar pentru jertfa de împăcare: doi boi, cinci berbeci, cinci țapi și cinci miei de un an. Acestea sunt darurile lui Gamaliel, fiul lui Pedațur.
\par 60 În ziua a noua a adus căpetenia fiilor lui Veniamin, Abidan, fiul lui Ghedeon.
\par 61 Darurile lui au fost: un blid de argint în greutate de o sută treizeci sicli și o cupă de argint de șaptezeci de sicli, după siclul sfânt, amândouă pline cu făină, amestecată cu untdelemn, pentru jertfă;
\par 62 O cădelniță de aur de zece sicli, plină cu miresme;
\par 63 Un vițel, un berbec și un miel de un an, pentru ardere de tot;
\par 64 Un jap, jertfă pentru păcat;
\par 65 Iar pentru jertfa de împăcare: doi boi, cinci berbeci, cinci țapi și cinci miei de un an. Acestea sunt darurile lui Abidan, fiul lui Ghedeon.
\par 66 În ziua a zecea a adus căpetenia fiilor lui Dan, Ahiezer, fiul lui Amișadai.
\par 67 Darurile lui au fost: un blid de argint în greutate de o sută treizeci sicli și o cupă de argint de șaptezeci sicli, după siclul sfânt, amândouă pline cu făină de grâu, amestecată cu untdelemn, pentru jertfă;
\par 68 O cădelniță de aur de zece sicli, plină cu miresme;
\par 69 Un vițel, un berbec și un miel de un an, pentru ardere de tot;
\par 70 Un jap, jertfă pentru păcat;
\par 71 Iar pentru jertfa de împăcare: doi boi, cinci berbeci, cinci țapi și cinci miei de un an. Acestea sunt darurile lui Ahiezer, fiul lui Amișadai.
\par 72 În ziua a unsprezecea a adus căpetenia fiilor lui Așer, Paghiel, fiul lui Ocran.
\par 73 Darurile lui au fost: un blid de argint în greutate de o sută treizeci de sicli și o cupă de argint de șaptezeci de sicli, după siclul sfânt, amândouă pline cu făină de grâu, amestecată cu untdelemn, pentru jertfă;
\par 74 O cădelniță de aur de zece sicli, plină cu miresme;
\par 75 Un vițel, un berbec și un miel de un an pentru arderea de tot;
\par 76 Un țap, jertfă pentru păcat;
\par 77 Iar pentru jertfa de împăcare: doi boi, cinci berbeci, cinci țapi și cinci miei de un an. Acestea sunt darurile lui Paghiel, fiul lui Ocran.
\par 78 În ziua a douăsprezecea a adus căpetenia fiilor lui Neftali, Ahira, fiul lui Enan.
\par 79 Darurile lui au fost: un blid de argint în greutate de o sută treizeci de sicli și o cupă de argint de șaptezeci de sicli, după siclul sfânt, amândouă pline cu făină de grâu, amestecată cu untdelemn, pentru jertfă;
\par 80 O cădelniță de aur de zece sicli, plină cu miresme;
\par 81 Un vițel, un berbec și un miel de un an pentru ardere de tot;
\par 82 Un țap, jertfă pentru păcat;
\par 83 Iar pentru jertfa de împăcare: doi boi, cinci berbeci, cinci țapi și cinci miei de un an. Acestea sunt darurile lui Ahira, fiul lui Enan.
\par 84 Acestea au fost darurile din partea căpeteniilor lui Israel, aduse la sfințirea jertfelnicului, în ziua miruirii lui: douăsprezece blide de argint, douăsprezece cupe de argint, douăsprezece cădelnițe de aur,
\par 85 Având fiecare blid o sută treizeci sicli de argint și fiecare cupă câte șaptezeci de sicli; deci argintul tot în aceste vase a fost două mii patru sute sicli, după siclul sfânt;
\par 86 Douăsprezece cădelnițe de aur, pline cu miresme, de câte zece sicli fiecare, după siclul sfânt; deci tot aurul cădelnițelor a fost o sută douăzeci de sicli;
\par 87 Pentru arderi de tot au fost: doisprezece viței din vitele mari, doisprezece berbeci și doisprezece miei de un an și împreună cu ei prinosul de pâine și turnarea lor; doisprezece țapi, jertfă pentru păcat;
\par 88 Iar pentru jertfa de împăcare au fost: douăzeci și patru de boi, șaizeci de berbeci, șaizeci de țapi și șaizeci de miei de un an, fără meteahnă. Acestea au fost darurile la sfințirea jertfelnicului, după miruirea lui.
\par 89 Când a intrat Moise în cortul adunării, ca să grăiască cu Domnul, a auzit un glas, care-i grăia de sus de pe chivotul legii, dintre cei doi heruvimi. Glasul acela grăia cu el.

\chapter{8}

\par 1 Atunci a grăit Domnul cu Moise și a zis:
\par 2 "Vorbește cu Aaron și-i spune: Când vei pune candelele în sfeșnic, ca să lumineze partea cea dinaintea lui, să aprinzi în el șapte candele".
\par 3 Și a făcut Aaron așa: a aprins în sfeșnic, ca să lumineze partea cea din fața lui, șapte candele, cum poruncise Domnul lui Moise.
\par 4 Iată cum era făcut sfeșnicul: fusul lui de aur era lucrat din ciocan; florile lui toate erau tot din ciocan. După modelul pe care îl arătase Domnul lui Moise, așa s-a făcut sfeșnicul.
\par 5 Și a grăit cu Moise Domnul și i-a zis:
\par 6 "Ia pe leviți din mijlocul fiilor lui Israel și-i curăță;
\par 7 Și ca să-i cureți, să faci cu ei așa: să-i stropești cu apa curățirii, să-și radă cu briciul tot trupul lor, să-și spele hainele și vor fi curați.
\par 8 Apoi ei să ia un vițel și prinosul de pâine, făină de grâu, amestecată cu untdelemn; iar tu să mai iei un vițel, jertfă pentru păcat.
\par 9 Adu după aceea pe leviți înaintea cortului adunării, unde vei aduna toată obștea fiilor lui Israel.
\par 10 Să se apropie leviții înaintea Domnului și fiii lui Israel să-și pună mâinile pe leviți;
\par 11 Iar Aaron să afierosească pe leviți înaintea Domnului, din partea fiilor lui Israel, ca să facă ei slujbă Domnului.
\par 12 Apoi leviții să-și pună mâinile pe capetele vițeilor și tu să aduci unul jertfă pentru păcat, iar pe celălalt ardere de tot Domnului pentru curățirea leviților.
\par 13 Pune apoi pe leviți înaintea Domnului și înaintea lui Aaron și înaintea fiilor lui și-i adu dar Domnului.
\par 14 Așa vei osebi pe leviți de fiii lui Israel, că vor fi ai Mei.
\par 15 După aceea vor merge leviții să slujească la cortul adunării, după ce îi vei curăți și îi vei afierosi Domnului;
\par 16 Căci Îmi sunt dați Mie dintre fiii lui Israel în locul tuturor celor întâi-născuți, care deschide orice pântece;
\par 17 Căci al Meu  este tot întâi-născutul lui Israel, de la om până la dobitoc, pentru că Mi i-am sfințit Mie în ziua când am lovit în pământul Egiptului pe toți întâi-născuții;
\par 18 Și în locul tuturor întâi-născuților fiilor lui Israel am luat pe leviți;
\par 19 Și i-am dat pe leviți dar lui Aaron și fiilor lui dintre fiii lui Israel, ca să slujească pentru fiii lui Israel, la cortul adunării și să se roage pentru fiii lui Israel, ca să nu-i ajungă pe fiii lui Israel vreo urgie, când s-ar apropia de locașul sfânt".
\par 20 Moise și Aaron și toată obștea fiilor lui Israel au făcut cu leviții cum poruncise Domnul lui Moise pentru leviți; așa au făcut cu ei fiii lui Israel.
\par 21 S-au curățit deci leviții și și-au spălat hainele, iar Aaron a săvârșit sfințirea lor înaintea Domnului și s-a rugat pentru ei, ca să fie curați.
\par 22 După aceea au intrat leviții să-și facă slujbele lor la cortul adunării, înaintea lui Aaron și înaintea fiilor lui. Cum poruncise Domnul lui Moise pentru leviți, așa au făcut cu ei.
\par 23 Și a grăit Domnul cu Moise și a zis:
\par 24 "Aceasta este legea leviților: de la douăzeci și cinci de ani în sus să intre să lucreze la cortul adunării;
\par 25 Iar la cincizeci de ani să înceteze și să nu mai lucreze.
\par 26 De acolo înainte să ajute fraților lor a străjui la cortul adunării, dar de lucrat să nu mai lucreze. Așa să faci cu leviții, ca fiecare să fie la slujba lui de paznic".

\chapter{9}

\par 1 În vremea aceea a grăit Domnul cu Moise în pustiul Sinai, în anul al doilea după ieșirea din Egipt, în luna întâi, și a zis:
\par 2 "Spune fiilor lui Israel să facă Paștile la vremea rânduită pentru ele:
\par 3 În ziua de paisprezece a lunii întâi, spre seară, să le facă la vremea lor, după legea lor și după regulile lor să le săvârșiți".
\par 4 Și a spus Moise fiilor lui Israel să facă Paștile:
\par 5 Și au făcut ei Paștile în luna întâi, în ziua a paisprezecea, spre seară, în pustiul Sinai; cum poruncise Domnul lui Moise așa au făcut fiii lui Israel.
\par 6 Dar erau și oameni necurați, care se atinseseră de trup de om mort, și nu puteau să săvârșească Paștile în ziua aceea. Aceștia au venit în ziua aceea la Moise și Aaron,
\par 7 Și le-au spus oamenii aceia: "Noi suntem necurați, pentru că ne-am atins de trup de om mort; do ce să nu fim lăsați să aducem Domnului dar la vremea cea rânduită pentru fiii lui Israel?"
\par 8 Iar Moise a zis către ei: "Stați aici, că am să ascult ce poruncește Domnul pentru voi!"
\par 9 A grăit Domnul lui Moise și a zis:
\par 10 "Spune fiilor lui Israel: Dacă cineva din voi sau din urmașii voștri va fi necurat prin atingere de trup de om mort, sau va fi departe în călătorie, sau între neamuri străine, și acela să facă Paștile Domnului.
\par 11 Dar să le facă în ziua a paisprezecea a lunii a doua, seara, și să le mănânce cu azime și cu ierburi amare;
\par 12 Să nu lase din ele pe a doua zi, nici oasele să nu le zdrobească; și să le săvârșească după toată rânduiala Paștilor.
\par 13 Iar omul curat, care nu se află departe în călătorie și nu va face Paștile, sufletul acela să se stârpească din poporul său, că n-a adus dar Domnului la vreme. Omul acela își va purta păcatul său.
\par 14 De va trăi la voi vreun străin să facă și el Paștile Domnului: după legea Paștilor șl după rânduiala lor să le facă. O singură lege să fie și pentru voi și pentru străin".
\par 15 În ziua când a fost așezat cortul, nor a acoperit cortul adunării, și de seara până dimineața a fost deasupra cortului, ca o vedere de foc.
\par 16 Așa era totdeauna: ziua îl acoperea un nor și noaptea o vedere de foc.
\par 17 Când se ridica norul de deasupra cortului, atunci fiii lui Israel plecau și în locul unde se oprea norul, acolo poposeau cu tabăra fiii lui Israel.
\par 18 După porunca Domnului se opreau fiii lui Israel cu tabăra lor și după porunca Domnului plecau; tot timpul cât norul stătea deasupra cortului, stăteau și ei cu tabăra.
\par 19 Când însă norul stătea multă vreme deasupra cortului, urmau acestui semn al Domnului și fiii lui Israel și nu plecau.
\par 20 Câteodată se întâmpla ca norul să stea numai puțină vreme deasupra cortului: după glasul Domnului se opreau și după porunca Lui plecau la drum.
\par 21 Câteodată norul stătea numai de seara până dimineața, iar dimineața se ridica norul; atunci plecau și ei; sau stătea norul o zi și o noapte, și când se ridica, plecau și ei;
\par 22 Sau de umbrea norul deasupra cortului două zile, sau o lună, sau un an, fiii lui Israel stăteau și nu plecau la drum; iar când se ridica el, atunci plecau,
\par 23 Că din porunca Domnului se opreau și din porunca Domnului plecau la drum: urmau semnul Domnului, după porunca dată de Domnul prin Moise.

\chapter{10}

\par 1 Și a grăit Domnul cu Moise și a zis:
\par 2 "Fă-ți două trâmbițe de argint; din ciocan să le faci, ca să fie pentru chemarea obștii și pentru plecarea taberei.
\par 3 De se va trâmbița din ele, se va aduna toată obștea la ușa cortului adunării.
\par 4 De se va trâmbița numai din una, se vor aduna la tine toate căpeteniile cele mai mari ale lui Israel.
\par 5 Când veți însoți sunetele cu strigăte, se vor ridica taberele cele dinspre răsărit.
\par 6 Când veți însoți a doua oară sunetele cu strigăte, se vor ridica taberele cele dinspre miazăzi. Când veți însoți a treia oară sunetele cu strigăte, se vor ridica taberele cele dinspre mare. Când veți însoți a patra oară sunetele cu strigăte, se vor ridica taberele cele dinspre miazănoapte. Să însoțiți sunetele cu strigăte numai pentru plecare.
\par 7 Iar când chemați adunarea, să sunați, dar să nu însoțiți sunetele cu strigăte.
\par 8 Din trâmbițe vor suna preoții, fiii lui Aaron: aceasta-i pentru voi lege veșnică din neam în neam.
\par 9 Când veți merge la război, în pământul vostru, împotriva vrăjmașilor care năvălesc asupra voastră, însoțiți sunetele de trâmbiță cu strigăte și veți fi pomeniți înaintea Domnului Dumnezeului vostru și veți fi izbăviți de vrăjmașii voștri.
\par 10 În ziua voastră de bucurie, la sărbătorile voastre și la lunile noi ale voastre, să trâmbițați din trâmbițe la arderile de tot ale voastre și la jertfele voastre de împăcare și prin aceasta veți fi pomeniți înaintea Dumnezeului vostru. Eu sunt Domnul Dumnezeul vostru".
\par 11 În anul al doilea, în luna a doua, în douăzeci ale lunii, s-a ridicat norul de deasupra cortului adunării;
\par 12 Și au plecat fiii lui Israel din pustiul Sinai după taberele lor și s-a oprit norul în pustiul Paran.
\par 13 Aceasta a fost întâia plecare, după porunca lui Dumnezeu, dată prin Moise.
\par 14 Întâi s-a ridicat steagul taberei fiilor lui Iuda cu cetele lor și peste cetele lor era Naason, fiul lui Aminadab.
\par 15 Peste cetele seminției lui Isahar era Natanael, fiul lui Țuar;
\par 16 Iar peste cetele seminției fiilor lui Zabulon era Eliab, fiul lui Helon.
\par 17 Apoi s-a ridicat cortul și au plecat fiii lui Gherșon și fiii lui Merari, care duceau cortul.
\par 18 După aceea s-a ridicat steagul taberei lui Ruben cu cetele sale; peste cetele lui era Elițur, fiul lui Ședeur;
\par 19 Peste cetele seminției lui Simeon era Șelumiel, fiul lui Țurișadai;
\par 20 Iar peste cetele seminției fiilor lui Gad era Eliasaf, fiul lui Raguel.
\par 21 După aceea au plecat fiii lui Cahat, care duceau lucrurile sfinte, căci cortul trebuia să fie așezat înainte de sosirea lor.
\par 22 Apoi s-a ridicat steagul taberei fiilor lui Efraim cu cetele lor; peste cetele lui era Elișama, fiul lui Amihud.
\par 23 Peste cetele fiilor seminției lui Manase era Gamaliel, fiul lui Pedațur;
\par 24 Iar peste cetele fiilor seminției lui Veniamin era Abidan, fiul lui Ghedeon.
\par 25 După toate taberele, cel din urmă a fost ridicat steagul taberei fiilor lui Dan cu cetele sale; peste cetele lui era Ahiezer, fiul lui Amișadai;
\par 26 Peste cetele seminției fiilor lui Așer era Paghiel, fiul lui Ocran;
\par 27 Iar peste cetele seminției fiilor lui Neftali era Ahira, fiul lui Enan.
\par 28 Aceasta era rânduiala în care mergeau fiii lui Israel cu taberele lor. Și așa au plecat.
\par 29 Atunci a zis Moise către Hobab fiul lui Raguel, madianitul, socrul lui Moise: "Noi plecăm la locul acela, de care a zis Domnul: Vouă vi-l voi da. Hai cu noi și-ți vom face bine, căci Domnul a grăit bine de Israel".
\par 30 Acela însă a zis către el: "Nu merg, ci mă duc în țara mea și la neamul meu".
\par 31 Dar Moise a zis: "Nu ne părăsi, pentru că tu știi cum ne așezăm noi taberele în pustie și vei fi ochiul nostru.
\par 32 Dacă mergi cu noi, binele ce ni-l va face Domnul, îl vom face și noi ție".
\par 33 Plecând ei de la muntele Domnului, au mers trei zile; iar chivotul legii Domnului a mers înaintea lor cale de trei zile, ca să aleagă pentru ei loc de odihnă.
\par 34 Norul Domnului îi umbrea ziua, când plecau de la popas.
\par 35 Când se ridica chivotul, ca să plece la drum, Moise zicea: "Scoală, Doamne, și să se risipească vrăjmașii Tăi și să fugă de la fața Ta cei ce Te urăsc pe Tine!"
\par 36 Iar când se oprea chivotul, el zicea: "Întoarce-Te, Doamne, la miile și zecile de mii ale lui Israel!"

\chapter{11}

\par 1 Poporul însă începu să cârtească în auzul Domnului, iar Domnul auzind, se aprinse mânia Lui, izbucni între ei foc de la Domnul și începu a mistui marginile taberei.
\par 2 Atunci a strigat poporul către Moise, iar Moise s-a rugat Domnului și a încetat focul.
\par 3 De aceea s-a numit locul acela: Tabeera, adică ardere, căci acolo a fost aprins între ei focul de la Domnul.
\par 4 Străinii, dintre ei, începură să-și arate poftele și ședeau cu ei și fiii lui Israel și plângeau, zicând: "Cine ne va hrăni cu carne?
\par 5 Căci ne aducem aminte de peștele, pe care-l mâncam în Egipt în dar, de castraveți și de pepeni, de ceapă, de praz și de usturoi;
\par 6 Acum însă sufletul nostru tânjește; nimic nu mai este înaintea ochilor noștri decât numai mana".
\par 7 Iar mana era ca sămânța de coriandru și înfățișarea ei era ca înfățișarea cristalului.
\par 8 Poporul se ducea și o aduna, o râșneau în râșnițe sau o pisau în piuă, o fierbeau în căldări și făceau din ea turte; iar gustul ei era ca gustul turtelor cu untdelemn.
\par 9 Când cădea noaptea roua pe tabără, atunci cădea peste ea și mana.
\par 10 Moise însă auzea cum plângea fiecare prin familiile sale și la ușa cortului său, și s-a aprins tare mânia Domnului și s-a mâhnit Moise.
\par 11 Atunci a zis Moise către Domnul: "De ce întristezi pe robul Tău și de ce oare n-am aflat milă înaintea ochilor Tăi, căci ai pus asupra mea sarcina a tot poporul acesta?
\par 12 Oare eu am zămislit tot poporul acesta și oare eu l-am născut, de-mi zici: Ia-l în sânul tău, cum ia doica pe copil, și-l du în pământul pe care cu jurământ l-am făgăduit părinților lui?
\par 13 De unde să iau și să dau eu carne la tot poporul acesta? Căci plâng înaintea mea și. zic: Dă-ne carne să mâncăm!
\par 14 Eu singur nu voi putea să duc tot poporul acesta, că acest lucru este greu pentru mine.
\par 15 Dacă faci așa cu mine, atunci mai bine omoară-mă, de am aflat milă înaintea ochilor Tăi, ca să nu mai văd necazul acesta".
\par 16 Atunci Domnul a zis către Moise: "Adună-Mi șaptezeci de bărbați, dintre bătrânii lui Israel, pe care-i știi tu că sunt căpetenii poporului și supraveghetorii lui, și du-i la cortul adunării, ca să stea cu tine acolo.
\par 17 Că Mă voi pogorî acolo și voi vorbi cu tine și voi lua din duhul care este peste tine și voi pune peste ei ca să ducă ei cu tine sarcina poporului și să nu o duci numai tu singur.
\par 18 Iar poporului spune-i: Să vă curățiți pentru ziua de mâine și veți avea carne, deoarece ați plâns în auzul Domnului și ați zis: Cine ne va hrăni cu carne? Că ne era bine în Egipt; că are să vă dea Domnul carne să mâncați și veți mânca;
\par 19 Și veți mânca nu numai o zi, nici numai două sau cinci zile, nici numai zece sau douăzeci de zile,
\par 20 Ci o lună întreagă veți mânca până vă va da pe nas și vă veți scârbi de ea, pentru că ați disprețuit pe Domnul, Care este între voi, și v-ați plâns înaintea Lui și ați zis: La ce trebuia să ieșim noi din Egipt?"
\par 21 Apoi a zis Moise: "În poporul acesta, în care mă aflu, sunt șase sute de mii de pedeștri și Tu zici: Am să le dau carne să mănânce și vor mânca o lună de zile!
\par 22 Li se vor tăia, oare, toate oile și toți boii, ca să le ajungă? Sau tot peștele mării li se va aduna, ca să-i îndestuleze?"
\par 23 Zis-a Domnul către Moise: "Dar, oare, mâna Domnului e scurtă? Acum vei vedea de se va împlini sau nu cuvântul Meu".
\par 24 Atunci a ieșit Moise și a spus poporului cuvintele Domnului, a adunat șaptezeci de bărbați dintre bătrânii poporului și i-a pus împrejurul cortului.
\par 25 Și S-a pogorât Domnul în nor și a vorbit cu el; și a luat din duhul care era peste el și a pus peste cei șaptezeci de bărbați căpetenii. Îndată însă cum a odihnit duhul peste ei, au început a prooroci, dar apoi au încetat.
\par 26 Doi dintre bărbați însă au rămas în tabără: pe unul îl chema Eldad și pe celălalt îl chema Medad. Și a odihnit și peste ei duhul, căci erau din cei înscriși, dar nu veniseră la cort, și au proorocit și ei acolo în tabără.
\par 27 Atunci a alergat un tânăr și a spus lui Moise, zicând: "Eldad și Medad proorocesc în tabără".
\par 28 Și răspunzând, Iosua, fiul lui Navi, slujitorul lui Moise, unul din aleșii lui, a zis: "Domnul meu Moise, oprește-i!"
\par 29 Moise însă i-a zis: "Nu cumva ești gelos pe mine? O, de ar fi toți prooroci în poporul Domnului și de ar trimite Domnul duhul Său peste ei!"
\par 30 Apoi s-a întors Moise în tabără împreună cu bătrânii lui Israel.
\par 31 Atunci s-a stârnit vânt de la Domnul, a adus prepelițe dinspre mare și le-a presărat împrejurul taberei cale de o zi într-o parte și cale de o zi în cealaltă parte împrejurul taberei, strat gros aproape de doi coți de la pământ.
\par 32 Atunci s-a sculat poporul și toată ziua și toată noaptea aceea și toată ziua următoare au adunat prepelițe; și cine a adunat puțin, tot a adunat zece coșuri și le-au întins împrejurul taberei.
\par 33 Dar carnea era încă în gura lor și nu isprăviseră încă de mâncat, când se aprinse mânia Domnului asupra poporului, și a lovit Domnul poporul cu bătaie foarte mare.
\par 34 Și s-a pus numele locului aceluia: Chibrot-Hataava, adică mormintele poftei, căci acolo au îngropat pe poporul cel aprins de poftă.
\par 35 Apoi a plecat poporul din Chibrot-Hataava la Hașerot și s-a oprit în Hașerot.

\chapter{12}

\par 1 Mariam și Aaron vorbeau însă împotriva lui Moise pentru femeia etiopiancă, pe care o luase Moise, căci Moise era căsătorit cu o etiopiancă.
\par 2 Ei ziceau: "Oare numai cu Moise a grăit Domnul? N-a grăit El, oare, și cu noi?" Și a auzit acestea Domnul.
\par 3 Moise însă era omul cel mai blând dintre toți oamenii de pe pământ.
\par 4 Atunci a zis Domnul fără de veste către Moise și către Aaron și către Mariam: "Ieșiți câteși trei la cortul adunării". Și au ieșit tustrei.
\par 5 Atunci S-a pogorât Domnul în stâlpul cel de nor, a stat la ușa cortului și a chemat pe Aaron și pe Mariam și au ieșit amândoi.
\par 6 Apoi a zis: "Ascultați cuvintele Mele: De este între voi vreun prooroc al Domnului, Mă arăt lui în vedenie și în somn vorbesc cu el.
\par 7 Nu tot așa am grăit și cu robul Meu Moise, - el este credincios în toată casa Mea:
\par 8 Cu el grăiesc gură către gură, la arătare și aievea, iar nu în ghicituri, și el vede fața Domnului. Cum de nu v-ați temut să cârtiți împotriva robului Meu Moise?"
\par 9 Și s-a aprins mânia Domnului asupra lor și, depărtându-Se Domnul,
\par 10 S-a depărtat și norul de la cort și iată Mariam s-a făcut albă de lepră, ca zăpada. Și când s-a uitat Aaron la Mariam, iată era leproasă.
\par 11 Atunci a zis Aaron către Moise: "Rogu-mă, domnul meu, să nu ne socotești greșeala că ne-am purtat rău și am păcătuit!
\par 12 Nu îngădui dar să fie Mariam ca cel născut mort, al cărui trup, la ieșirea din pântecele mamei sale, este pe jumătate putred".
\par 13 Atunci a strigat Moise către Domnul și a zis: "Dumnezeule, vindec-o!"
\par 14 Domnul însă a zis către Moise: "Dacă tatăl ei ar fi scuipat-o în obraz, oare n-ar fi trebuit să se rușineze șapte zile? Așa dar să fie închisă șapte zile afară din tabără, după aceea să intre".
\par 15 Și a șezut Mariam închisă afară din tabără șapte zile și poporul n-a plecat la drum până s-a curățit Mariam.
\par 16 După aceasta a pornit poporul de la Hașerot și a poposit în pustiul Paran.

\chapter{13}

\par 1 Acolo a grăit Domnul cu Moise și i-a zis:
\par 2 "Trimite din partea ta oameni ca să iscodească pământul Canaanului pe care am să-l dau Eu fiilor lui Israel spre moștenire; câte un om de fiecare seminție să trimiți; însă aceștia să fie toți căpetenii între ei".
\par 3 Și i-a trimis pe aceștia Moise din pustiul Paran, după porunca Domnului, și erau toți căpetenii.
\par 4 Iată acum și numele lor: Șammua, fiul lui Zahur, din seminția lui Ruben.
\par 5 Safat, fiul lui Hori, din seminția lui Simeon;
\par 6 Caleb, fiul lui Iefone, din seminția lui Iuda;
\par 7 Igal, fiul lui Iosif, din seminția lui Isahar;
\par 8 Osia, fiul lui Navi, din seminția lui Efraim;
\par 9 Falti, fiul lui Rafu, din seminția lui Veniamin;
\par 10 Gadiel, fiul lui Sodi, din seminția lui Zabulon;
\par 11 Gadi, fiul lui Susi, din Manase, seminția lui Iosif;
\par 12 Amiel, fiul lui Ghemali, din seminția lui Dan;
\par 13 Setur, fiul lui Mihael, din seminția lui Așer;
\par 14 Nahbi, fiul lui Vofsi, din seminția lui Neftali;
\par 15 Gheuel, fiul lui Machi, din seminția lui Gad.
\par 16 Acestea sunt numele bărbaților pe care i-a trimis Moise să iscodească țara. Insă pe Osia, fiul lui Navi, Moise l-a numit Iosua.
\par 17 Trimițându-i pe aceștia din pustiul Paran ca să iscodească pământul Canaanului, Moise le-a zis:
\par 18 "Suiți-vă din pustiul acesta și vă urcați pe munte și cercetați ce pământ este și ce popor locuiește în el; de este tare sau slab, mult la număr sau puțin;
\par 19 Cum este țara pe care o locuiește: bună sau rea, cum sunt orașele în care trăiește el: cu ziduri sau fără ziduri;
\par 20 Cum este pământul: gras sau slab, de sunt pe el copaci sau nu. Fiți curajoși și luați din roadele pământului aceluia". Aceasta se petrecea pe vremea coacerii strugurilor.
\par 21 Și s-au dus ei și au cercetat pământul de la pustiul Țin până la Rehob, care vine lângă Hamat.
\par 22 De acolo au trecut în partea de miazăzi a Canaanului și au mers până la Hebron, unde trăiau Ahiman, Șeșai, și Talmai, copiii lui Enac. Hebronul fusese zidit cu șapte ani înaintea orașului egiptean Țoan.
\par 23 Apoi au venit până în valea Eșcol, au cercetat-o și au tăiat de acolo o viță de vie cu un strugure de poamă și au dus-o doi pe pârghie. Au mai luat de asemenea rodii și smochine.
\par 24 Locul acesta l-au numit ei valea Eșcol, adică valea strugurelui, de la strugurele de poamă pe care l-au tăiat de acolo fiii lui Israel.
\par 25 Și după ce au cercetat ei pământul, s-au întors după patruzeci de zile
\par 26 Și, mergând, au venit la Moise și la Aaron și la toată obștea fiilor lui Israel, la Cadeș, în pustiul Paran, și le-au adus lor și întregii obști vița și le-au arătat roadele pământului aceluia.
\par 27 Apoi le-au povestit și au zis: "Am fost în pământul în care ne-ai trimis, pământul în care curge miere și lapte și iată roadele lui.
\par 28 Dar poporul care locuiește în el, este îndrăzneț și orașele sunt întărite și foarte mari, ba și pe fiii lui Enac i-am văzut acolo.
\par 29 Amalec locuiește în partea de miazăzi a țării; Heteii, Heveii, Iebuseii și Amoreii locuiesc în munți, iar Canaaneii locuiesc pe lângă mare și pe lângă râul Iordanului".
\par 30 Caleb însă a liniștit poporul înaintea lui Moise, zicând: "Nu, ci să mergem și să-l cuprindem, pentru că îl vom putea birui!"
\par 31 Iar oamenii cei ce fuseseră cu el au zis: "Nu putem să mergem împotriva poporului aceluia, pentru că e mult mai puternic decât noi".
\par 32 Și au împrăștiat printre fiii lui Israel zvonuri rele despre pământul pe care-l cercetaseră, zicând: "Pământul pe care l-am străbătut noi, ca să-l vedem, este un pământ care mănâncă pe cei ce locuiesc în el și tot poporul, pe care l-am văzut acolo, sunt oameni foarte mari.
\par 33 Acolo am văzut noi și uriași, pe fiii lui Enac, din neamul uriașilor; și nouă ni se părea că suntem față de ei ca niște lăcuste și tot așa le păream și noi lor".

\chapter{14}

\par 1 Atunci toată obștea a ridicat strigăt și a plâns poporul toată noaptea aceea;
\par 2 Cârtind împotriva lui Moise și a lui Aaron, toți fiii lui Israel și toată obștea au zis către ei: "Mai bine era să fi murit în pământul Egiptului sau să murim în pustiul acesta!
\par 3 La ce ne duce Domnul în pământul acela, ca să cădem în război? Femeile noastre și copiii noștri vor fi pradă. Nu ar fi, oare, mai bine să ne întoarcem în Egipt?"
\par 4 Apoi au zis unii către alții: "Să ne alegem căpetenie și să ne întoarcem în Egipt".
\par 5 Atunci au căzut Moise și Aaron cu felele la pământ înaintea întregii adunări a obștii fiilor lui Israel.
\par 6 Iar Iosua, fiul lui Navi și Caleb al lui Iefone, care erau din cei ce cercetaseră țara, și-au rupt hainele lor
\par 7 Și au zis către obștea fiilor lui Israel: "Pământul, pe care l-am străbătut noi, este foarte, foarte bun;
\par 8 De va fi Domnul bun cu noi, ne va duce în pământul acela și ni-l va da nouă; în pământul acela izvorăște lapte și miere.
\par 9 Deci nu vă ridicați împotriva Domnului și nu vă temeți de poporul pământului aceluia, căci va ajunge mâncarea noastră: ei n-au apărare, iar cu noi este Domnul. Nu vă temeți de ei!"
\par 10 Atunci toată obștea a zis: "Să-i ucidem cu pietre!" Dar iată slava Domnului s-a arătat în nor  tuturor fiilor lui Israel la cortul adunării.
\par 11 Și a zis Domnul către Moise: "Până când Mă va supăra poporul acesta și până când nu va crede el în Mine, cu toate minunile ce am făcut în mijlocul lui?
\par 12 Îl voi lovi cu ciumă și-l voi pierde și te voi face pe tine și casa tatălui tău popor numeros și mai puternic decât acesta!"
\par 13 Moise însă a zis către Domnul: "Vor auzi de aceasta Egiptenii, din mijlocul cărora ai scos Tu, cu puterea Ta, pe poporul acesta
\par 14 Și vor spune locuitorilor pământului acestuia, care au auzit, că Tu, Doamne, Te afli în mijlocul poporului acestuia și Tu, Doamne, le dai să Te vadă față către față, și că Tu mergi înaintea lor, ziua în stâlp de nor și noaptea în stâlp de foc.
\par 15 Iar dacă Tu vei pierde pe poporul acesta, ca pe un om, atunci popoarele care au auzit de numele Tău vor zice:
\par 16 Domnul n-a putut duce pe poporul acesta în pământul pe care cu jurământ l-a făgăduit să-l dea lor și de aceea l-a pierdut în pustie.
\par 17 Deci, înalță-se acum puterea Ta, Doamne, cum ai spus Tu, zicând:
\par 18 Domnul este îndelung-răbdător, mult-îndurat și adevărat, iertând fărădelegile, greșelile și păcatele și nelăsând nepedepsit, ci pedepsește nelegiuirile părinților în copii până la al treilea și al patrulea neam.
\par 19 Iartă păcatul poporului acestuia, după mare mila Ta, precum ai iertat Tu poporul acesta din Egipt și până acum".
\par 20 Zis-a Domnul către Moise: "Voi ierta, după cuvântul tău,
\par 21 Dar viu sunt Eu și viu e numele Meu și de slava Domnului e plin tot pământul:
\par 22 Toți bărbații câți au văzut slava Mea și minunile pe care le-am făcut în pământul Egiptului și în pustie, și M-au ispitit până acum de zeci de ori și n-au ascultat glasul Meu,
\par 23 Nu vor vedea pământul pe care Eu cu jurământ l-am făgăduit părinților lor; ci numai copiilor lor, care sunt aici cu Mine, care nu știu ce este binele și ce este răul și tuturor nevârstnicilor, care nu judecă, acelora le voi da pământul, iar toți cei ce M-au amărât nu-l vor vedea;
\par 24 Iar pe robul Meu Caleb, îl voi duce în pământul în care a umblat și seminția lui îl va moșteni, pentru că în el a fost alt duh și pentru că el s-a supus Mie.
\par 25 Amaleciții și Canaaneii locuiesc pe vale; mâine să vă întoarceți și să vă duceți în pustie, spre Marea Roșie".
\par 26 Și a mai grăit Domnul cu Moise și cu Aaron și a zis:
\par 27 "Până când această obște rea va cârti împotriva Mea? Cârtirea cu care fiii lui Israel cârtesc împotriva Mea, o aud.
\par 28 Deci, spune-le: Viu sunt Eu, zice Domnul! După cum ați zis în auzul Meu, așa voi face cu voi:
\par 29 În pustia aceasta vor cădea oasele voastre și voi toți cei numărați, de la douăzeci de ani în sus, care ați cârtit împotriva Mea, oricâți ați fi la număr,
\par 30 Nu veți intra în pământul pentru care, ridicându-Mi mâna, M-am jurat să vă așez, ci numai Caleb, fiul lui Iefone, și Iosua, fiul lui Navi.
\par 31 Pe copiii voștri, despre care voi ziceați că vor ajunge pradă vrăjmașilor, îi voi duce acolo și ei vor cunoaște pământul pe care voi l-ați nesocotit;
\par 32 Iar oasele voastre vor cădea în pustia aceasta.
\par 33 Copiii voștri vor rătăci prin pustie patruzeci de ani și vor suferi pedeapsă pentru desfrânarea voastră, până vor cădea toate oasele voastre în pustie.
\par 34 După numărul celor patruzeci de zile, în care ați iscodit pământul Canaan, veți purta pedeapsa pentru păcatele voastre patruzeci de ani, câte un an pentru fiecare zi, ca să cunoașteți ce înseamnă să fiți părăsiți de Mine.
\par 35 Eu, Domnul, am grăit! Și așa voi face cu toată obștea aceasta rea, care s-a ridicat împotriva Mea: în pustia aceasta vor pieri și vor muri toți!"
\par 36 Și oamenii pe care-i trimisese Moise să iscodească pământul și care la întoarcere au întărâtat împotriva lui toată obștea aceasta, răspândind zvonuri rele despre țara aceea,
\par 37 Au murit loviți înaintea Domnului, pentru că au răspândit zvonurile despre țara aceea.
\par 38 Numai Iosua, fiul lui Navi, și Caleb, fiul lui Iefone, au rămas vii dintre bărbații aceia care fuseseră să iscodească țara Canaan.
\par 39 Cuvintele acestea le-a spus Moise înaintea tuturor fiilor lui Israel și poporul s-a întristat foarte tare.
\par 40 Sculându-se ei deci dis-de-dimineață, s-au dus pe vârful muntelui, zicând: "Iată, ne ducem la locul acela de care ne-a grăit Domnul, căci am greșit!"
\par 41 Moise însă le-a zis: "Pentru ce călcați porunca Domnului? Nu veți izbuti.
\par 42 Nu vă duceți, căci Domnul nu este între voi și veți cădea înaintea vrăjmașilor voștri;
\par 43 Căci Amaleciții și Canaaneii sunt acolo înaintea voastră și veți cădea de sabie, pentru că v-ați abătut de la Domnul și Domnul nu va fi cu voi".
\par 44 Dar ei au îndrăznit să se urce pe vârful muntelui; iar chivotul legii Domnului și Moise n-au părăsit tabăra.
\par 45 Atunci s-au suit Amaleciții și Canaaneii care trăiau în muntele acela și i-au înfrânt și i-au gonit până la Horma și s-au întors în tabără.

\chapter{15}

\par 1 În vremea aceea a grăit Domnul cu Moise și a zis:
\par 2 "Vorbește fiilor lui Israel și le spune: Când veți intra în pământul vostru de locuit, pe care Eu îl voi da vouă,
\par 3 Și când veți face jertfe Domnului din oi sau din boi, ardere de tot, sau jertfă de făgăduință sau de bună voie, sau când veți face la sărbătorile voastre mireasmă plăcută Domnului,
\par 4 Atunci cel ce aduce darul său Domnului să aducă jertfă de pâine a zecea parte de efă de făină de grâu curată, amestecată cu un sfert de hin de untdelemn,
\par 5 Și vin pentru turnare, a patra parte de hin la ardere de tot sau la jertfa de făgăduință, la fiecare miel va face la fel întru miros bine-plăcut Domnului.
\par 6 Iar când veți aduce berbec, adu jertfă de pâine două zecimi de efă de făină de grâu curată, amestecată cu a treia parte de hin de untdelemn;
\par 7 Și vin de turnare să aduci a treia parte de hin, întru miros de bună mireasmă Domnului.
\par 8 Dacă aduceți junc, ardere de tot, sau jertfă de făgăduință, sau jertfă de împăcare,
\par 9 Atunci cu juncul să aduci prinos de pâine trei zecimi de efă de făină de grâu, amestecată cu jumătate de hin de untdelemn.
\par 10 Și vin pentru turnare, jumătate de hin la jertfă, întru miros de bună mireasmă Domnului.
\par 11 Așa să faci totdeauna, când aduci junc sau berbec, miel sau capră,
\par 12 După numărul jertfelor pe care le faceți; așa să aduceți la fiecare, după numărul lor.
\par 13 Tot băștinașul să facă așa când aduce jertfe de acestea întru mireasmă plăcută Domnului.
\par 14 De va trăi însă printre voi în pământul vostru un străin și ar fi între voi din neam în neam, și va voi să aducă jertfă pentru miros plăcut Domnului, să facă așa cum faceți voi.
\par 15 Pentru voi obștea Domnului și pentru străinul care locuiește între voi, o singură lege să fie, lege veșnică din neam în neam. Cum sunteți voi așa să fie și străinul înaintea Domnului.
\par 16 O singură lege și aceleași drepturi să fie pentru voi și pentru străinul care locuiește la voi".
\par 17 Și a grăit Domnul cu Moise și a zis:
\par 18 "Vorbește fiilor lui Israel și le spune: Când veți intra în pământul în care vă duc,
\par 19 Și veți mânca pâinea țării aceleia, să înălțați prinos Domnului.
\par 20 Pârgă din aluatul vostru să înălțați dar Domnului o azimă; dar s-o înălțați așa ca prinosul din arie;
\par 21 Pârgă din aluatul vostru să înălțați dar Domnului din neam în neam.
\par 22 Dacă însă veți greși din neștiință și nu veți împlini toate poruncile acestea, pe care le-a rostit Domnul lui Moise,
\par 23 Și tot ce v-a poruncit Domnul prin Moise din ziua în care a început Domnul a vă porunci,
\par 24 Dacă greșeala e din nebăgarea de seamă a obștii, atunci toată obștea să aducă din cireadă un junc fără meteahnă, ardere de tot, întru miros bineplăcut Domnului, cu dar de pâine, cu turnarea lui după rânduială, și din turma de capre, un țap ca jertfă pentru păcat
\par 25 Și se va ruga preotul pentru toată obștea fiilor lui Israel și li se va ierta, căci aceasta a fost greșeală și ei au adus darul lor Domnului și jertfă pentru păcatul lor înaintea Domnului, pentru greșeala lor.
\par 26 Atunci se va ierta întregii obști a fiilor lui Israel și străinului care trăiește între ei, pentru că tot poporul a făcut aceasta din neștiință.
\par 27 Dacă vreun suflet a greșit din neștiință, să aducă o capră de un an jertfă pentru păcat
\par 28 Și se va ruga preotul pentru sufletul care a făcut păcat din neștiință înaintea Domnului și va afla milă și i se va ierta.
\par 29 Și pentru băștinașul din Israel și pentru străinul care trăiește între voi, o singură lege să fie când cineva va păcătui din neștiință.
\par 30 Iar dacă cineva dintre băștinași sau dintre străini va face ceva din îndrăzneală, acela hulește pe Domnul și sufletul lui se va stârpi din poporul său,
\par 31 Căci a disprețuit cuvântul Domnului și a călcat poruncile Lui; să se stârpească sufletul acela și păcatul lui va fi asupra lui".
\par 32 Când se aflau fiii lui Israel în pustiu, au găsit un om adunând lemne în ziua odihnei;
\par 33 Și cei ce l-au găsit adunând lemne în ziua odihnei l-au adus la Moise și Aaron și la toată obștea fiilor lui Israel;
\par 34 Și l-au pus sub pază, pentru că nu era încă hotărât ce să facă cu el.
\par 35 Atunci a zis Domnul către Moise: "Omul acesta să moară; să fie ucis cu pietre de către toată obștea fiilor lui Israel, afară din tabără!"
\par 36 L-au scos deci toată obștea fiilor lui Israel afară din tabără și l-au ucis cu pietre toată obștea, afară din tabără, cum poruncise Domnul lui Moise.
\par 37 Și a grăit Domnul cu Moise și a zis:
\par 38 "Vorbește fiilor lui Israel și le spune să-și facă ciucuri la poalele hainelor lor, din neam în neam, și pe deasupra ciucurilor de la poalele hainelor lor să pună un șiret de mătase violetă.
\par 39 Ciucurii aceștia să fie ca, uitându-vă la ei, să vă aduceți aminte de toate poruncile Domnului și să le împliniți și să nu umblați după inima voastră și după ochii voștri care vă îndeamnă la desfrânare;
\par 40 Ca să vă aduceți aminte și să pliniți toate poruncile Mele și să fiți sfinți înaintea Dumnezeului vostru.
\par 41 Eu sunt Domnul Dumnezeul vostru, Care v-am scos din pământul Egiptului, ca să fiu Dumnezeul vostru. Eu sunt Domnul Dumnezeul vostru".

\chapter{16}

\par 1 Atunci Core, fiul lui Ițhar, fiul lui Cahat, fiul lui Levi, cu Datan și cu Abiron, fiii lui Eliab, cu On, fiul lui Felet, din seminția lui Ruben, s-au sculat împotriva lui Moise,
\par 2 Împreună cu două sute cincizeci de bărbați, căpetenii ale obștii fiilor lui Israel, oameni însemnați, care erau chemați la adunare.
\par 3 Adunându-se aceștia împotriva lui Moise și Aaron, le-au zis: "Destul! Toată obștea și toți cei ce o alcătuiesc sunt sfinți și Domnul este între ei. Pentru ce vă socotiți voi mai presus de adunarea Domnului!"
\par 4 Auzind acestea, Moise a căzut cu fața la pământ
\par 5 Și a grăit lui Core și tuturor părtașilor lui și le-a zis: "Mâine va arăta Domnul cine este al Lui și cine este sfânt, ca să și-L apropie; și pe cine va alege El, pe acela îl va și apropia la Sine.
\par 6 Iată ce să faceți: Core și toți părtașii tăi să vă luați cădelnițe
\par 7 Și mâine să puneți în acestea foc și să turnați în ele tămâie înaintea Domnului, și pe cine va alege Domnul, acela va fi sfânt. Destul, fiii lui Levi!
\par 8 Apoi a zis iarăși Moise către Core: "Ascultați, fii ai lui Levi:
\par 9 Oare e puțin lucru pentru voi că Dumnezeul lui Israel v-a osebit de obștea lui Israel și v-a apropiat la Sine ca să faceți slujbe la cortul Domnului și să stați înaintea obștii Domnului, slujind pentru ea?
\par 10 El te-a apropiat pe tine și cu tine pe toți frații tăi, fiii lui Levi. Alergați acum și după preoție?
\par 11 Așadar tu și toată obștea ta, v-ați adunat împotriva Domnului. Ce este Aaron, de cârtiți împotriva lui?"
\par 12 Atunci a trimis Moise să cheme pe Datan și pe Abiron, fiii lui Eliab. Ei însă au zis: "Nu mergem!
\par 13 Oare puțin lucru e că ne-ai scos din țara unde curge miere și lapte și ne-ai adus să ne pierzi în pustie? Vrei să și domnești peste noi?
\par 14 Dusu-ne-ai tu oare în țara unde curge lapte și miere și datu-ne-ai tu oare în stăpânire țarinile și viile ei? Vrei să scoți ochii oamenilor acestora? Nu mergem!"
\par 15 Și s-a mâhnit Moise foarte tare și a zis către Domnul: "Să nu-ți întorci ochii Tăi la prinosul lor. Eu nici unuia dintre ei nu i-am luat asinul și rău n-am făcut nici unuia dintre ei".
\par 16 Apoi a zis Moise către Core: "Sfințește-ți ceata ta și mâine să fiți gata înaintea Domnului: tu, ei și Aaron.
\par 17 Luați-vă fiecare cădelnițe, puneți în ele tămâie și vă apropiați fiecare cu cădelnița înaintea Domnului, cu două sute cincizeci de cădelnițe: și tu și Aaron să aduceți fiecare cădelnița voastră".
\par 18 Și și-a luat fiecare cădelnița sa, au pus în ele foc, au turnat tămâie în ele; și au stat înaintea intrării cortului adunării Moise și Aaron.
\par 19 Core însă a adunat împotriva lor toată obștea înaintea ușii cortului adunării. Și s-a arătat slava Domnului la toată obștea.
\par 20 Și a grăit Domnul cu Moise și Aaron și a zis:
\par 21 "Osebiți-vă de obștea aceasta și-i voi pierde într-o clipă".
\par 22 Iar ei au căzut cu fețele la pământ și au zis: "Doamne, Dumnezeul duhurilor și a tot trupul, un om a greșit și Tu Te mânii pe toată obștea? "
\par 23 Domnul însă i-a zis lui Moise:
\par 24 "Spune obștii: Feriți-vă în toate părțile de locuința lui Core, a lui Datan și a lui Abiron".
\par 25 Atunci, sculându-se, Moise s-a dus la Datan și Abiron și s-au dus după el și toți bătrânii lui Israel.
\par 26 Și a zis obștii: "Feriți-vă de corturile acestor oameni netrebnici și să nu vă atingeți de tot ce e al lor, ca să nu pieriți cu toate păcatele lor".
\par 27 Și ei au ocolit sălașurile lui Core, Datan și Abiron; iar Datan și Abiron ieșiseră și stăteau la ușile corturilor lor, cu femeile lor și cu fiii lor și cu pruncii lor.
\par 28 Zis-a Moise: "Că Domnul m-a trimis să fac toate lucrurile acestea și că nu le fac eu de la mine, veți cunoaște din aceea:
\par 29 De vor muri aceștia, cum moc toți oamenii, și de-i va ajunge aceeași pedeapsă, care ajunge pe toți oamenii, - atunci nu m-a trimis Domnul.
\par 30 Iar dacă Domnul va face lucru neobișnuit, de-și va deschide pământul gura sa și-i va înghiți pe ei și casele lor și corturile lor și tot ce au ei, și dacă ei vor fi duși de vii în locuința morților, atunci să știți că oamenii aceștia au disprețuit pe Domnul".
\par 31 Cum a încetat el să spună toate cuvintele acestea, s-a desfăcut pământul sub aceia
\par 32 Și și-a deschis pământul gura sa și i-a înghițit pe ei și casele lor, pe toți oamenii lui Core și toată averea;
\par 33 Și s-au pogorât ei cu toate câte aveau de vii în locuința morților, i-a acoperit pământul și au pierit din mijlocul obștii.
\par 34 Și tot Israelul, care era împrejurul lor, a fugit la strigătele lor, că ziceau: "Să nu ne înghită și pe noi pământul!"
\par 35 A ieșit apoi foc de la Domnul și a mistuit pe cei două sute cincizeci de bărbați care au adus tămâie.
\par 36 După aceea a grăit Domnul cu Moise și a zis:
\par 37 "Spune lui Eleazar, fiul preotului Aaron, să adune cădelnițele cele de aramă ale celor arși și focul străin să-l arunce afară, căci s-au sfințit cădelnițele păcătoșilor acestora prin moartea lor.
\par 38 Să le sfărâme deci și să le facă foi pentru acoperit jertfelnicul. Pentru că le-au adus aceia înaintea Domnului, s-au sfințit și vor fi semn pentru fiii lui Israel".
\par 39 A luat deci Eleazar, fiul  preotului Aaron, cădelnițele cele de aramă, pe care le aduseseră cei arși, și le-a prefăcut în foi pentru acoperit jertfelnicul,
\par 40 Ca să-și aducă aminte fiii lui Israel că nimeni din alt neam, care nu e din seminția lui Aaron, să nu se apropie să aducă tămâiere înaintea Domnului, și să nu fie ca și Core și părtașii lui, precum îi grăise Domnul prin Moise.
\par 41 A doua zi însă toată obștea fiilor lui Israel a cârtit împotriva lui Moise și a lui Aaron și a zis: Voi ați omorât poporul Domnului.
\par 42 Și când s-a adunat obștea împotriva lui Moise și Aaron, aceștia s-au întors către cortul adunării, și iată norul l-a acoperit și s-a arătat slava Domnului.
\par 43 Și a venit Moise și Aaron la cortul adunării.
\par 44 Atunci a grăit Domnul cu Moise și Aaron și a zis:
\par 45 "Depărtați-vă de obștea aceasta, că într-o clipă o voi pierde". Iar ei au căzut cu fața la pământ.
\par 46 Și a zis Moise către Aaron: "Ia-ți cădelnița, pune în ea foc de pe jertfelnic, aruncă în ea tămâie și du-o repede în tabără și te roagă pentru ei, că a ieșit mânie de la fața Domnului și a început pedepsirea poporului".
\par 47 Atunci Aaron a luat, cum îi zisese Moise, a alergat în mijlocul obștii și iată se începuse moartea în popor; și a pus tămâia și s-a rugat pentru popor;
\par 48 Și stând el între morți și vii, a încetat bătaia.
\par 49 Au murit atunci din pedepsirea aceea paisprezece mii șapte sute de oameni, afară de cei ce muriseră pentru răzvrătirea lui Core.
\par 50 Iar după ce a încetat pedepsirea, s-a întors Aaron la Moise, la ușa cortului adunării.

\chapter{17}

\par 1 După aceea a grăit Domnul lui Moise și a zis:
\par 2 "Spune fiilor lui Israel și ia de la ei, de la toate căpeteniile lor, după seminții, douăsprezece toiege, câte un toiag de fiecare seminție, și numele fiecărei căpetenii scrie-l pe toiagul său;
\par 3 Iar numele lui Aaron să-l scrii pe toiagul lui Levi, căci un toiag vor da de fiecare căpetenie de seminție.
\par 4 Toiegele acelea să le pui în cortul adunării înaintea chivotului legii, unde Mă arăt ție.
\par 5 Și va fi că toiagul omului pe care-l voi alege va odrăsli; și așa voi potoli cârtirea fiilor lui Israel, cu care cârtesc ei împotriva voastră".
\par 6 Și Moise a spus acestea fiilor lui Israel și toate căpeteniile lor i-au dat toiegele, câte un toiag de fiecare căpetenie, adică douăsprezece toiege, după cele douăsprezece seminții ale lor; și toiagul lui Aaron era între toiegele lor.
\par 7 Apoi Moise a pus toiegele înaintea Domnului, în cortul adunării.
\par 8 Iar a doua zi a intrat Moise și Aaron în cortul adunării și iată toiagul lui Aaron, din casa lui Levi, odrăslise, înmugurise, înflorise și făcuse migdale.
\par 9 Și atunci a scos Moise toate toiegele de la fața Domnului la toți fiii lui Israel; și au văzut și și-au luat fiecare toiagul său.
\par 10 Apoi a zis Domnul către Moise: Pune iar toiagul lui Aaron înaintea chivotului legii spre păstrare, ca semn pentru fiii neascultători, ca să înceteze de a mai cârti împotriva Mea, ca să nu moară!"
\par 11 Și a făcut Moise așa; cum îi poruncise Domnul așa a făcut.
\par 12 Și au zis fiii lui Israel către Moise: "Iată murim, pierim, pierim cu toții!
\par 13 Tot cel ce se apropie de cortul Domnului moare; nu cumva o să murim cu toții?"

\chapter{18}

\par 1 Zis-a Domnul către Aaron: "Tu, fiii tăi și casa tatălui tău cu tine veți purta păcatul pentru nepăsarea de locașul sfânt; tu și fiii tăi împreună cu tine veți purta păcatul pentru nepăsarea de preoția voastră.
\par 2 Apropie-ți pe frații tăi, seminția lui Levi, neamul tatălui tău, ca să fie pe lângă tine șl să-ți slujească; iar tu și fiii tăi împreună cu tine veți fi la cortul adunării.
\par 3 Leviții să păzească cele rânduite de tine și să facă slujbă la cort, dar să nu se apropie de lucrurile locașului sfânt și de jertfelnic, ca să nu moară și ei și voi.
\par 4 Să fie deci pe lângă tine și să facă slujbă la cortul adunării și toate lucrările la cort; iar altul să nu se apropie de tine.
\par 5 Așa să faceți slujba în locașul sfânt și la jertfelnic, și nu va mai veni mânia asupra fiilor lui Israel;
\par 6 Că am ales din fiii lui Israel pe frații voștri, pe leviți, și vi i-am dat ca dar închinat Domnului, să facă slujbă la cortul adunării;
\par 7 Iar tu și fiii tăi să vă îndepliniți preoția voastră în toate cele ce țin de jertfelnic și ce se află înăuntru după perdea, și să săvârșiți slujbele darului vostru preoțesc, iar altul străin, de se va apropia, să fie omorât".
\par 8 Zis-a Domnul către Aaron: "Iată Eu am dat în seama voastră pârga Mea din toate cele închinate Mie de fiii lui Israel: ție ți le-am dat acestea și după tine fiilor tăi, pentru cinul vostru, pentru preoția voastră, prin lege veșnică.
\par 9 Iată ce este al tău din cele preasfinte, în afară de cele ce se dau focului: orice dar de pâine al lor, orice jertfă pentru păcat a lor și orice jertfă pentru vină, ce-Mi aduc ei, aceste lucruri preasfinte să fie ale tale și ale fiilor tăi.
\par 10 Acestea să le mâncați în locul cel sfânt. Tu și fiii tăi, toți cei de parte bărbătească ai voștri pot să mănânce din ele. Cele sfinte să fie ale tale.
\par 11 Și iată ce să mai fie al vostru din darurile lor ridicate: toate darurile ridicate ale fiilor lui Israel și toate darurile lor legănate ți le-am dat ție și fiilor tăi și fiicelor tale, care sunt cu tine, prin lege veșnică. Tot cel curat din casa ta poate să mănânce din acestea.
\par 12 Toată pârga de untdelemn și toată pârga de struguri și pârga grâului lor, toate câte aduc ei Domnului, ți le-am dat ție.
\par 13 Cele dintâi roade ale pământului lor, pe care le aduc ei Domnului, să fie ale tale, și tot cel curat din casa ta poate să mănânce din acestea.
\par 14 Tot ce este afierosit în Israel să fie al tău.
\par 15 Tot ce se naște întâi din tot trupul, din oameni și din dobitoace, și se aduce Domnului, să fie al tău; dar întâiul născut dintre oameni să se răscumpere și întâiul născut dintre dobitoacele necurate să se răscumpere;
\par 16 Iar prețul răscumpărării lui, la o lună după naștere, este cinci sicli de argint, după siclul sfânt, care are douăzeci de ghere.
\par 17 Însă întâiul născut al vacilor, întâiul născut al oilor și întâiul născut al caprelor, nu se răscumpără: aceștia sunt sfințiți; cu sângele lor să stropești jertfelnicul. Grăsimea lor s-o arzi ca jertfă, întru miros de bună mireasmă Domnului;
\par 18 Iar carnea lor este a ta și tot ale tale sunt pieptul înălțat și șoldul drept.
\par 19 Toate darurile sfinte, înălțate, care se aduc Domnului de fiii lui Israel, ți le dau ție, fiilor tăi și fiicelor tale care sunt cu tine, prin lege veșnică. Acest legământ de necălcat este veșnic înaintea Domnului pentru tine și pentru urmașii tăi".
\par 20 Zis-a Domnul către Aaron: "În pământul lor nu vei avea nici moștenire, nici parte nu vei avea între ei. Eu sunt partea ta și moștenirea ta între fiii lui Israel,
\par 21 Iar fiilor lui Levi, iată, Eu le-am dat moștenire toată zeciuiala din toate câte are Israel, pentru slujba lor pe care o fac la cortul adunării.
\par 22 De acum fiii lui Israel să nu mai vină la cortul adunării, ca să nu facă păcat aducător de moarte.
\par 23 Ci la cortul adunării să facă slujba leviții și să ia asupră-și păcatul lor. Aceasta este lege veșnică în neamul vostru.
\par 24 Dar printre fiii lui Israel ei nu vor avea moștenire, căci zeciuiala fiilor lui Israel, pe care aceștia o aduc dar Domnului, am dat-o leviților moștenire și de aceea le-am și zis Eu că nu vor avea moștenire între fiii lui Israel".
\par 25 Apoi a grăit Domnul lui Moise și a zis:
\par 26 "Vorbește leviților și le zi: Când veți lua de la fiii lui Israel zeciuiala, pe care v-am dat-o ca moștenire, să înălțați din ea dar Domnului a zecea parte, ca zeciuială,
\par 27 Și vi se va socoti acest dar al vostru ca grâul din arie și ca mustul de la teasc.
\par 28 Astfel veți aduce și voi dar Domnului din toate zeciuielile voastre, câte veți lua de la fiii lui Israel, și veți da din ele, dar Domnului, preotului Aaron.
\par 29 Din toate cele dăruite vouă, cele mai bune din toate cele sfințite să le aduceți dar Domnului,
\par 30 Și să le spui: De veți aduce din acestea partea cea mai bună, se va socoti leviților ca cele primite de la arie și ca cele primite de la teasc.
\par 31 Aceasta să o mâncați oriunde voi, și fiii voștri, și familiile voastre, căci aceasta vă este plata pentru munca voastră la cortul adunării.
\par 32 Pentru aceasta nu veți avea păcat, de veți aduce cele mai bune din toate; și sfintele prinoase ale fiilor lui Israel nu le veți întina și nu veți muri".

\chapter{19}

\par 1 Grăit-a Domnul cu Moise și Aaron și a zis:
\par 2 "Iată porunca legii, pe care a dat-o Domnul, când a zis: Spune fiilor lui Israel să-ți aducă o junincă roșie, fără meteahnă, care să nu aibă cusur și să nu fi purtat jug;
\par 3 Să o dai preotului Eleazar, să o scoată afară din tabără, la loc curat, și să o junghie înaintea lui.
\par 4 Apoi să ia preotul Eleazar din sângele ei și să stropească cu sânge spre partea de dinainte a cortului adunării de șapte ori.
\par 5 După aceea să o ardă de tot înaintea lui; să ardă adică și carnea și pielea și sângele și necurățenia ei.
\par 6 Apoi să ia preotul lemn de cedru, isop și ață de lână roșie și să le arunce pe juninca ce se arde.
\par 7 Să-și spele preotul hainele sale, să-și spele trupul cu apă, apoi să intre în tabără și necurat va fi până seara.
\par 8 Cel ce a ars-o de asemenea să-și spele hainele sale, să-și spele trupul cu apă și necurat va fi până seara.
\par 9 Un om curat să strângă cenușa junincii, s-o pună afară din tabără la loc curat și să se păstreze pentru obștea fiilor lui Israel, ca să se facă cu ea apă de stropire, adică apă de curățire.
\par 10 Cel ce a adunat cenușa junincii să-și spele hainele sale și să fie necurat până seara. Aceasta să fie așezământ veșnic pentru fiii lui Israel și pentru străinii ce trăiesc la dânșii.
\par 11 Cel ce se va atinge de trupul mort al unui om să fie necurat șapte zile.
\par 12 Acesta să se curețe cu această apă în ziua a treia și în ziua a șaptea și va fi curat; iar de nu se va curăți în ziua a treia și în ziua a șaptea, nu va fi curat.
\par 13 Tot cel ce sg va atinge de trupul mort al unui om și nu se va curăți, acela va întina locașul Domnului; omul acela se va stârpi din Israel, căci n-a fost stropit cu apă curățitoare și este necurat și necurăția lui e încă asupra lui.
\par 14 Iată legea: De va muri un om într-o casă, tot cel ce va intra în casa aceea și câte sunt în casă vor fi necurate șapte zile.
\par 15 Tot vasul descoperit, care nu este legat la gură și n-are capac pe el, este necurat.
\par 16 Tot cel ce se va atinge în câmp de cel ucis cu sabia sau de mort sau de os de om sau de mormânt va fi necurat șapte zile.
\par 17 Pentru cel necurat să se ia din cenușa jertfei arse pentru curățire și să se toarne peste ea într-un vas apă de izvor;
\par 18 Apoi un om curat să ia isop, să-l moaie în apa aceea și să stropească din ea casa, lucrurile și oamenii câți sunt acolo și pe cel ce s-a atins de os de om sau de ucis sau de mort sau de mormânt.
\par 19 Cel curat să stropească pe cel necurat în ziua a treia și în ziua a șaptea și să-l curețe în ziua a șaptea. Apoi să-și spele hainele sale și trupul său cu apă și va fi necurat până seara.
\par 20 Iar dacă vreun om va fi necurat și nu se va curăți, omul acela se va stârpi din obște, căci a spurcat locașul Domnului; căci nu s-a stropit cu apă curățitoare și este necurat.
\par 21 Acesta să fie așezământ veșnic pentru dânșii. Cel ce a stropit cu apă curățitoare să-și spele hainele sale; cel ce s-a atins de apa curățitoare va fi necurat până seara.
\par 22 Tot lucrul de care se va atinge cel necurat va fi necurat; și tot ce se va atinge de acel lucru va fi necurat până seara".

\chapter{20}

\par 1 În luna întâi a ajuns toată obștea fiilor lui Israel în pustiul Sin și s-a oprit poporul în Cadeș. Și a murit Mariam și a fost îngropată acolo.
\par 2 Acolo însă nu era apă pentru obște și s-au adunat ei împotriva lui Moise și a lui Aaron,
\par 3 Și blestema poporul pe Moise și zicea: "O, de am fi murit și noi când au murit frații noștri înaintea Domnului!
\par 4 La ce ați adus voi obștea Domnului în pustiul acesta, ca să ne omorâți și pe noi și dobitoacele noastre?
\par 5 Și la ce ne-ați scos din Egipt, ca să ne aduceți în acest loc rău, unde nu se poate semăna și nu sunt nici smochini, nici vite, nici rodii și nici măcar apă de băut?"
\par 6 Atunci s-au dus Moise și Aaron din fața poporului la ușa cortului adunării și au căzut cu fețele la pământ și s-a arătat slava Domnului peste ei.
\par 7 Și a grăit Domnul cu Moise și a zis:
\par 8 "Ia toiagul și adună obștea, tu și Aaron, fratele tău, și grăiți stâncii înaintea lor și ea vă va da apă; și le veți scoate apă din stâncă și veți adăpa obștea și dobitoacele ei".
\par 9 A luat deci Moise toiagul din fața Domnului, cum poruncise Domnul.
\par 10 Și au adunat Moise și Aaron obștea la stâncă și a zis către obște: "Ascultați, îndărătnicilor, au doară din stânca aceasta vă vom scoate apă?"
\par 11 Apoi și-a ridicat Moise mâna și a lovit în stâncă cu toiagul său de două ori și î ieșit apă multă și băut obștea și dobitoacele ei.
\par 12 Atunci a zis Domnul către Moise și Aaron: "Pentru că nu M-ați crezut, ca să arătați sfințenia Mea înaintea ochilor fiilor lui Israel, de aceea nu veți duce voi adunarea aceasta în pământul pe care am să i-l dau".
\par 13 Aceasta este apa Meriba, căci aici fiii lui Israel s-au certat înaintea Domnului, iar El S-a sfințit între ei.
\par 14 Din Cadeș a trimis Moise soli la regele Edomului, ca să-i spună: "Așa zice fratele tău Israel: Tu știi toate greutățile ce am îndurat.
\par 15 Părinții noștri s-au pogorât în Egipt și noi am pribegit în Egipt vreme multă; dar Egiptenii ne-au făcut rău nouă și părinților noștri.
\par 16 De aceea am strigat către Domnul și a auzit Domnul glasul nostru și a trimis îngerul Său de ne-a scos din Egipt; și acum suntem în Cadeș, orașul cel mai apropiat de hotarul tău.
\par 17 Îngăduiește-ne să trecem prin țara ta, că nu ne vom abate pe la ogoare și pe la vii, nici apă nu vom bea din fântânile tale; ci vom trece pe drumul împărătesc, neabătându-ne nici la dreapta, nici la stânga, până vom ieși din hotarele tale".
\par 18 Edom însă i-a răspuns: "Să nu treci pe la mine, iar de nu vei asculta voi ieși cu război înaintea ta".
\par 19 Zisu-i-au fiii lui Israel: "Vom merge pe drumul cel mare și de vom bea din apa ta, noi sau dobitoacele noastre, îți vom plăti; vom trece numai cu piciorul, ceea ce e un lucru de nimic".
\par 20 Iar acela i-a răspuns: "Să nu treci pe la mine!" Și a ieșit Edom înaintea lui cu popor mult și cu mână puternică.
\par 21 Deci nu s-a învoit Edom să dea voie lui Israel să treacă prin hotarele lui și Israel s-a depărtat de la el.
\par 22 După aceea au pornit fiii lui Israel din Cadeș și au venit toată obștea la muntele Hor.
\par 23 Iar la muntele Hor, care e lângă hotarele țării lui Edom, a grăit Domnul cu Moise și cu Aaron și a zis:
\par 24 "Aaron va fi adăugat la poporul său, că el nu va intra în pământul pe care îl voi da fiilor lui Israel, pentru că nu v-ați supus poruncii Mele la apa Meriba.
\par 25 Să iei dar pe fratele tău Aaron și pe Eleazar, fiul lui, și să-i sui pe muntele Hor înaintea întregii obști;
\par 26 Să dezbraci acolo de pe Aaron hainele lui și să îmbraci cu ele pe Eleazar, fiul lui, și Aaron să se ducă și să moară acolo".
\par 27 Și a făcut Moise așa cum îi poruncise Domnul: i-a suit pe muntele Hor înaintea ochilor întregii obști.
\par 28 Acolo a dezbrăcat Moise de pe Aaron hainele lui și a îmbrăcat cu ele pe Eleazar, fiul lui. Și a murit Aaron pe vârful muntelui, iar Moise și Eleazar s-au pogorât din munte.
\par 29 Văzând toată obștea că a murit Aaron, l-a plâns toată casa lui Israel treizeci de zile.

\chapter{21}

\par 1 Auzind însă regele canaanean din Arad, care locuia la miazăzi, că Israel vine pe drumul dinspre Atarim, a intrat în luptă cu Israeliții și a luat pe unii din ei în robie.
\par 2 Atunci a făcut Israel făgăduință Domnului și a zis: "De vei da pe poporul acesta în mâinile mele, îl voi nimici pe el și cetățile lui".
\par 3 Și a ascultat Domnul glasul lui Israel și a dat pe Canaanei în mâinile lui și el i-a nimicit pe ei și orașele lor și a pus locului aceluia numele: Horma, adică nimicire.
\par 4 De la muntele Hor au apucat pe calea Mării Roșii, ca să ocolească pământul lui Edom, dar pe drum poporul a început să-și piardă răbdarea.
\par 5 Și grăia poporul împotriva lui Dumnezeu și împotriva lui Moise, zicând: "La ce ne-ai scos din pământul Egiptului, ca să ne omori în pustiu, că aici nu este nici pâine, nici apă și sufletul nostru s-a scârbit de această hrană sărăcăcioasă".
\par 6 Atunci a trimis Domnul asupra poporului șerpi veninoși, care mușcau poporul, și a murit mulțime de popor din fiii lui Israel.
\par 7 A venit deci poporul la Moise și a zis: "Am greșit, grăind împotriva Domnului și împotriva ta; roagă-te Domnului, ca să depărteze șerpii de la noi". Și s-a rugat Moise Domnului pentru popor.
\par 8 Iar Domnul a zis către Moise: "Fă-ți un șarpe de aramă și-l pune pe un stâlp; și de va mușca șarpele pe vreun om, tot cel mușcat care se va uita la el va trăi.
\par 9 Și a făcut Moise un șarpe de aramă și l-a pus pe un stâlp; și când un șarpe mușca vreun om, acesta privea la șarpele cel de aramă și trăia.
\par 10 Sculându-se de acolo, fiii lui Israel au tăbărât la Obot.
\par 11 Iar după ce s-au ridicat și din Obot, au tăbărât la Iie-Abarim, dincolo de pustiu, în fața Moabului, către răsăritul soarelui.
\par 12 De acolo s-au ridicat și au tăbărât în valea Zared.
\par 13 Ridicându-se apoi și de acolo, au tăbărât dincolo de Arnon, în pustia care e afară din hotarele Amoreilor. Căci Arnonul este hotar între Moabiți și Amorei.
\par 14 De aceea se și zice în "Cartea războaielor Domnului":
\par 15 Domnul a cuprins Vahebul cu curgerile sale năvalnice și șuvoaiele Arnonului și povârnișul curgerilor de apă care se întinde până la localitatea Ar și se oprește în hotarul Moabului.
\par 16 De acolo s-au îndreptat spre Beer, fântâna despre care a zis Domnul lui Moise: "Adună poporul și le voi da apă să bea".
\par 17 Atunci a cântat Israel la fântână cântarea aceasta: "Lăudați fântâna aceasta, cântați imne în cinstea ei!
\par 18 Fântâna pe care principii au săpat-o, pe care mai-marii poporului au deschis-o cu sceptrul, cu toiegele lor!"
\par 19 Din Beer au mers la Matana, de la Matana la Nahaliel, de la Nahaliel la Bamot;
\par 20 Iar de la Bamot la valea din câmpia Moabului, pe vârful muntelui Fazga, în fața pustiului.
\par 21 De acolo a trimis Moise soli la Sihon, regele Amoreilor, cu vești de pace, ca să i se spună:
\par 22 "Dă-mi voie să trec prin țara ta. Nu ne vom abate nici la ogorul tău, nici la via ta, nici apă din fântâna ta nu vom bea, ci vom trece de hotarele tale!"
\par 23 Dar Sihon n-a îngăduit lui Israel să treacă prin țara lui, ci și-a adunat tot poporul său și a pășit împotriva lui Israel în pustie, înaintând până la Iahaț, unde s-a luptat cu Israel.
\par 24 Însă Israel l-a bătut, măcelărindu-l cu sabia, și i-a cuprins țara de la Arnon până la Iaboc, până la fiii lui Amon, căci hotarele Amoniților erau întărite.
\par 25 Luând toate cetățile acestea, Israel s-a așezat în cetățile Amoreilor: în Heșbon și în toate satele care țineau de el.
\par 26 Căci Heșbonul era cetatea lui Sihon, regele Amoreilor. Acesta se luptase cu fostul rege al Amoreilor și-i luase din mâini toată țara Amoreilor de la Aroer până la Arnon.
\par 27 De aceea și zic rapsozii în bătaie de joc: "Veniți la Heșbon, ca să se zidească și să se întărească cetatea lui Sihon.
\par 28 Că a ieșit foc din Heșbon și pară de foc din cetatea lui Sihon și a mistuit Ar-Moabul și pe stăpânii munților Arnonului.
\par 29 Vai de tine, Moab! Ești pierdut, poporul lui Camos! Feciorii lui s-au risipit și fetele lui au ajuns roabe la Sihon, regele Amoreilor.
\par 30 Tras-am asupra lor cu săgeți. De la Heșbon până la Dibon tot este dărâmat, am pustiit tot până la Nofa, care e aproape de Medeba".
\par 31 Și așa s-a așezat Israel în toate cetățile Amoreilor.
\par 32 De acolo a trimis Moise să iscodească Iazerul, pe care l-a luat împreună cu satele lui și a alungat pe Amoreii care locuiau acolo.
\par 33 Și întorcându-se, a luat calea spre Vasan; iar Og, regele Vasanului, a ieșit înaintea lor cu tot poporul său, ca să se războiască la Edrei.
\par 34 Atunci a zis Domnul către Moise: "Să nu te temi de el, că-l voi da în mâinile tale pe el și tot poporul lui și toată țara lui și vei face cu el cum ai făcut cu Sihon, regele Amoreilor, care locuia în Heșbon".
\par 35 Și l-a bătut pe el, pe fiii lui, și pe tot poporul lui, de n-a lăsat viu nici pe unul din ai lui, și a cuprins țara lui.

\chapter{22}

\par 1 Purcezând apoi de acolo, fiii lui Israel au tăbărât în șesurile Moabului, lângă Iordan, în fața Ierihonului.
\par 2 Iar Balac, fiul lui Sefor, văzând toate câte făcuse Israel Amoreilor,
\par 3 S-a înfricoșat foarte tare de poporul acesta, pentru că era mult la număr, și s-au înspăimântat Moabiții de fiii lui Israel
\par 4 Și au zis către căpeteniile Madianiților: "Poporul acesta mănâncă acum totul împrejurul nostru, cum mănâncă boul iarba câmpului". Balac însă, feciorul lui Sefor, era atunci regele Moabiților.
\par 5 Deci a trimis acesta soli la Valaam, fiul lui Beor, în Petor, care e așezat lângă râul Eufrat, în pământul fiilor poporului său, ca să-l cheme și să-i spună: "Iată a ieșit un popor din Egipt și a acoperit fața pământului și trăiește lângă mine.
\par 6 Vino deci și-mi blesteamă poporul acesta, că este mai tare decât mine, și atunci poate voi fi în stare să-l bat și să-l alung din țară. Eu știu că pe cine binecuvântezi tu acela este binecuvântat, și pe cine blestemi este blestemat".
\par 7 S-au dus deci bătrânii Moabiților și bătrânii Madianiților cu mâinile pline de daruri pentru vrăji; și ajungând la Valaam, i-au spus vorbele lui Balac.
\par 8 Iar el le-a zis: "Rămâneți aici peste noapte și vă voi da răspuns cum îmi va spune Domnul". Și au rămas căpeteniile lui Moab la Valaam.
\par 9 Atunci a venit Dumnezeu la Valaam și a zis: "Cine sunt oamenii aceia de la tine?"
\par 10 Iar Valaam a zis către Dumnezeu: "Balac, fiul lui Sefor, regele Moabului, i-a trimis la mine să-mi spună:
\par 11 Iată a ieșit din Egipt un popor și a acoperit fața pământului și locuiește lângă mine; vino dar de mi-l blesteamă, doar l-aș putea birui și alunga din țară".
\par 12 Dumnezeu însă a zis către Valaam: "Să nu te duci cu ei și să nu blestemi pe poporul acela, că este binecuvântat".
\par 13 Dimineața s-a sculat Valaam și a zis către bătrânii lui Balac: "Duceți-vă la stăpânul vostru, că nu mă lasă Dumnezeu să merg cu voi".
\par 14 Sculându-se deci, căpeteniile Moabului au venit la Balac și i-au spus: "Valaam n-a vrut să vină cu noi".
\par 15 Atunci Balac a trimis alți soli mai mulți și mai însemnați decât aceia.
\par 16 Și venind aceștia la Valaam, i-au zis: "Așa grăiește Balac al lui Sefor: Nu te lepăda a veni până la mine;
\par 17 Că îți voi da cinste mare și-ți voi face toate câte-mi vei zice; vino însă și-mi blesteamă poporul acesta".
\par 18 Iar Valaam a răspuns și a zis către căpeteniile lui Balac: "Chiar de mi-ar da Balac casa sa plină de argint și de aur, nu pot să calc porunca Domnului Dumnezeului meu și să fac ceva mic sau mare după placul meu.
\par 19 Rămâneți însă acum și voi aici peste noapte și voi vedea ce-mi va mai spune Domnul".
\par 20 Atunci a venit Dumnezeu la Valaam noaptea și i-a zis: "Dacă oamenii aceștia au venit să te cheme, scoală și te du cu ei; dar să faci ceea ce-ți voi zice Eu!"
\par 21 A doua zi s-a sculat Valaam, și-a pus samarul pe asina sa și s-a dus cu căpeteniile Moabului.
\par 22 Dar se aprinsese mânia lui Dumnezeu pentru că s-a dus, iar îngerul Domnului s-a sculat, ca să-l mustre pe cale.
\par 23 Cum ședea el pe asina sa, însoțit de două slugi ale sale, a văzut asina pe îngerul Domnului, care stătea în drum cu sabia ridicată în mână, și s-a abătut din drum pe câmp; iar Valaam a bătut asina cu toiagul său, ca să o întoarcă la drum.
\par 24 Dar îngerul Domnului a stat în drumul îngust între vii, unde de o parte și de alta era zid;
\par 25 Și asina, văzând îngerul Domnului, s-a tras către zid și a strâns piciorul lui Valaam în zid, și acesta iar a început s-o bată.
\par 26 Îngerul Domnului însă a trecut iar și a stat la loc strâmt, unde nu era chip să te abați nici la dreapta, nici la stânga.
\par 27 Iar asina, văzând pe îngerul Domnului, s-a culcat sub Valaam. Atunci s-a mâniat Valaam și a început să bată asina cu toiagul.
\par 28 Dar Domnul a deschis gura asinei și aceasta a zis către Valaam: "Ce ți-am făcut eu, de mă bați acum pentru a treia oară?"
\par 29 Și Valaam a zis către asină: "Pentru că ți-ai râs de mine; de aș fi avut în mână o sabie, te-aș fi ucis aici pe loc".
\par 30 Răspuns-a asina lui Valaam: "Au nu sunt eu asina ta, pe care ai umblat din tinerețile tale și până în ziua aceasta? Avut-am oare deprinderea de a mă purta așa cu tine?" Și el a zis: "Nu!"
\par 31 Atunci a deschis Domnul ochii lui Valaam și acesta a văzut pe îngerul Domnului, care stătea în mijlocul drumului cu sabia ridicată în mână, și s-a închinat și a căzut cu fața la pământ.
\par 32 Iar îngerul Domnului i-a zis: "De ce ai bătut asina ta de trei ori? Eu am ieșit să te împiedic, deoarece calea ta nu este dreaptă înaintea mea;
\par 33 Și asina, văzându-mă pe mine, s-a întors de la mine de trei ori până acum; dacă ea nu s-ar fi întors de Ia mine, eu te-aș fi  ucis pe tine, iar pe ea aș fi lăsat-o vie".
\par 34 Zis-a Valaam către îngerul Domnului: "Am păcătuit, pentru că n-am știut că stai tu în drum înaintea mea. Deci, dacă aceasta nu este plăcut în ochii tăi, atunci mă voi întoarce".
\par 35 Iar îngerul Domnului a zis către Valaam: "Du-te cu  oamenii aceștia, dar să grăiești ceea ce-ți voi spune eu!" și s-a dus Valaam cu căpeteniile lui Balac.
\par 36 Când a auzit Balac că vine Valaam, a ieșit în întâmpinarea lui în orașul moabit, care este lângă hotarul de la Arnon, chiar la hotar.
\par 37 Și a zis Valaam către Balac: "Iată acum am venit la tine. Dar pot eu, oare, să-ți spun ceva?
\par 38 Ce-mi va pune Dumnezeu în gură, aceea îți voi grăi!"
\par 39 Apoi s-a dus Valaam cu Balac și au mers la Kiriat-Huțot.
\par 40 Atunci a junghiat Balac oi și boi și a trimis lui Valaam și căpeteniilor ce erau cu el.
\par 41 Iar a doua zi de dimineață, a luat Balac pe Valaam și l-a suit pe înălțimile lui Baal, ca să-i arate de acolo o parte din popor.

\chapter{23}

\par 1 Atunci a zis Valaam către Balac: "Zidește-mi aici șapte jertfelnice și pregătește-mi șapte viței și șapte berbeci".
\par 2 Și a făcut Balac după cum zisese Valaam: au ridicat Balac și Valaam câte un vițel și câte un berbec pe fiecare jertfelnic.
\par 3 Apoi a zis Valaam către Balac: "Stai lângă jertfa ta, iar eu mă duc, că poate îmi va ieși Domnul înainte și ce-mi va descoperi El aceea îți voi spune". Și a rămas Balac lângă jertfa sa, iar Valaam s-a dus într-un loc înalt să întrebe pe Dumnezeu.
\par 4 Și S-a arătat Dumnezeu lui Valaam și a zis Valaam către El: "Am zidit șapte jertfelnice și am suit câte un vițel și câte un berbec pe fiecare jertfelnic".
\par 5 Iar Domnul a pus cuvânt în gura lui Valaam și a zis: "Întoarce-te la Balac și să-i zici așa!"
\par 6 Și s-a întors la acesta și iată el stătea la arderile de tot ale lui și toate căpeteniile Moabului erau cu el. Și a fost peste el Duhul Domnului și și-a rostit cuvântul său, zicând:
\par 7 "Din Mesopotamia m-a adus Balac, regele Moabului, din munții Răsăritului și mi-a zis: Vino și-mi blesteamă pe Iacov, vino și osândește pe Israel!
\par 8 Cum să blestem pe cel ce nu-l blesteamă Dumnezeu? Sau cum să osândesc pe cel ce nu-l osândește Dumnezeu?
\par 9 De pe vârful muntelui mă uit la el și de pe dealuri îl privesc. Iată un popor care trăiește deosebi și nu se numără cu alte popoare.
\par 10 Cine va număra pe urmașii lui Iacov și gloatele din Israel cine le va socoti? Să moară sufletul meu moartea drepților acestora și să fie sfârșitul meu ca sfârșitul lor!"
\par 11 Atunci a zis Balac către Valaam: "Ce mi-ai făcut? Eu te-am adus să-mi blestemi pe vrăjmașii mei și iată tu îi binecuvântezi!"
\par 12 Valaam însă a zis către Balac: "Oare să nu spun eu lui Balac ceea ce-mi pune Domnul în gură?"
\par 13 Iar Balac a zis către el: "Vino cu mine în alt loc, de unde nu-l vei vedea tot, ci numai o parte din el vei vedea, iar tot nu-l vei vedea: să mi-l blestemi de acolo".
\par 14 Și l-a dus pe el la locul de strajă, pe vârful muntelui Fazga, și a zidit acolo șapte jertfelnice și a pus câte un vițel și câte un berbec pe fiecare jertfelnic.
\par 15 Și a zis Valaam către Balac: "Stai aici lângă jertfa ta, iar eu mă duc să întreb pe Dumnezeu!"
\par 16 Atunci a întâmpinat Dumnezeu pe Valaam și a pus cuvânt în gura lui și a zis: "Întoarce-te la Balac și să-i grăiești acestea".
\par 17 Și s-a întors la el și stătea la jertfa sa cu toate căpeteniile Moabului. Și l-a întrebat Balac: "Ce ți-a spus Domnul?"
\par 18 Iar el și-a rostit cuvântul său și a zis: "Scoală și ascultă Balac! Ia aminte la mine, fiul lui Sefor!
\par 19 Dumnezeu nu este ca omul, ca să-L minți, nici ca fiul omului, ca să-I pară rău. Au zice-va El și nu va face? Sau va grăi și nu va împlini?
\par 20 Iată am primit poruncă să binecuvântez; El a binecuvântat și eu nu pot întoarce binecuvântarea.
\par 21 El nu vede nedreptate în Iacov și nu zărește silnicie în Israel; Domnul Dumnezeul său este cu el și în mijlocul lui se aude strigăt de veselie ca pentru un împărat.
\par 22 Dumnezeu l-a scos din Egipt, puterea lui este ca a unui taur.
\par 23 Pentru că nu este vrăjitorie în Iacov, nici farmece în Israel, la vreme se va spune lui Iacov și lui Israel; cele ce vrea să plinească Dumnezeu!
\par 24 Iată un popor care se ridică asemenea unei leoaice și ca un leu se scoală, care nu se culcă până n-a sfâșiat prada și până n-a băut sângele ucișilor!"
\par 25 Zis-a Balac către Valaam: "Nici de blestemat să nu-l blestemi, nici de binecuvântat să nu-l binecuvântezi".
\par 26 Iar Valaam a răspuns și a zis către Balac: "Nu ți-am grăit eu, oare, că voi face ce-mi va spune Domnul?"
\par 27 Atunci a zis Balac către Valaam: "Hai să te duc în alt loc: poate-I va plăcea lui Dumnezeu și mi-I vei blestema de acolo".
\par 28 Și a luat Balac pe Valaam pe vârful lui Peor, care privește spre pustie.
\par 29 Aici Valaam a zis către Balac: "Zidește-mi șapte jertfelnice și pregătește-mi șapte viței și șapte berbeci",
\par 30 Și a făcut Balac, cum a zis Valaam, și a pus câte un vițel și câte un berbec pe fiecare jertfelnic.

\chapter{24}

\par 1 Văzând Valaam că Domnul binevoiește să se binecuvânteze Israel, n-a mai alergat după obicei la vrăjitorii, ci s-a întors cu fața spre pustie;
\par 2 Și ridicându-și Valaam ochii săi, a văzut pe Israel așezat după semințiile sale și a venit peste dânsul duhul lui Dumnezeu,
\par 3 Și și-a rostit el cuvântul său, zicând: "Așa zice Valaam, fiul lui Beor; așa grăiește bărbatul cel ce vede cu adevărat;
\par 4 Așa glăsuiește cel ce ascultă cuvântul lui Dumnezeu, cel ce cunoaște gândurile Celui Atotputernic, cel ce vede descoperirile lui Dumnezeu, ca în vis, dar ochii și-i are deschiși:
\par 5 Cât sunt de frumoase sălașurile tale, Iacove, corturile tale, Israele!
\par 6 Se desfășoară ca niște văi, ca niște grădini pe lângă râuri, ca niște cedri pe lângă ape, ca niște corturi pe care le-a înfipt Domnul!
\par 7 Ieși-va din sămânța lui un Om, care va stăpâni neamuri multe și stăpânirea Lui va întrece pe a lui Agag și împărăția Lui se va înălța.
\par 8 Dumnezeu l-a scos din Egipt și puterea lui va fi ca a taurului; mânca-va popoarele dușmane lui, va sfărâma oasele lor și cu săgețile sale va săgeta pe vrăjmași.
\par 9 Plecatu-s-a și s-a culcat ca un leu și ca o leoaică; cine-l va scula? Cel ce te binecuvântează, binecuvântat să fie, și cel ce te blesteamă să fie blestemat!"
\par 10 Atunci s-a mâniat Balac pe Valaam și, frângându-și mâinile, a zis Balac către Valaam: "Eu te-am chemat să-mi blestemi pe vrăjmașii mei, iar tu, iată, i-ai binecuvântat de trei ori până acum.
\par 11 Fugi dar în țara ta! Am zis că te voi cinsti; dar iată că Domnul te-a lipsit de cinste".
\par 12 Valaam însă a zis către Balac: "N-am spus eu oare solilor tăi pe care i-ai trimis la mine:
\par 13 Chiar de mi-ar da Balac casa sa plină de argint și de aur, nu voi putea să calc porunca Domnului, ca să fac ceva bun sau rău după placul meu; câte-mi va spune Domnul, acelea le voi grăi?
\par 14 Deci, iată, mă duc repede în țara mea; dar vino să-ți spun ce are să facă poporul acesta cu poporul tău în vremurile viitoare".
\par 15 Și și-a urmat Valaam cuvântul său și a zis: "Așa grăiește Valaam, fiul lui Beor; așa grăiește bărbatul cel ce vede cu adevărat,
\par 16 Cel ce ascultă cuvintele lui Dumnezeu, cel ce are știință de la Cel Preaînalt și vede descoperirile lui Dumnezeu, ca în vis, dar ochii îi sunt deschiși:
\par 17 Îl văd, dar acum încă nu este; îl privesc, dar nu de aproape; o stea răsare din Iacov; un toiag se ridică din Israel și va lovi pe căpeteniile Moabului și pe toți fiii lui Set îi va zdrobi.
\par 18 Lua-va de moștenire pe Edom și va stăpâni Seirul vrăjmașilor săi și Israel își va arăta puterea.
\par 19 Din Iacov se va scula Cel ce va stăpâni cu putere și va pierde pe cei ce vor rămâne în cetate".
\par 20 Apoi văzând pe Amalec, și-a urmat cuvântul și a zis: "Cel întâi dintre popoare e Amalec, dar și neamul lui va pieri".
\par 21 Văzând după aceea pe Chenei, și-a urmat cuvântul și a zis: "Locuința ta e tare și cuibul tău e așezat pe stâncă;
\par 22 Dar Cain va fi dărâmat și nu este mult până ce Asur te va duce în robie".
\par 23 Iar când a văzut pe Og, și-a urmat cuvântul și a zis: "Vai, vai, cine va mai trăi când Dumnezeu va aduce acestea!
\par 24 Veni-vor corăbii de la Chitim și vor smeri pe Asur, vor smeri pe Heber, dar și acelea vor pieri".
\par 25 Sculându-se apoi Valaam s-a întors înapoi, în țara sa; și s-a dus și Balac întru ale sale.

\chapter{25}

\par 1 Atunci s-a așezat Israel în Sitim, dar poporul a început să se spurce, păcătuind cu fetele din Moab.
\par 2 Că acestea îi chemau la jertfele idolilor lor și mânca poporul din acele jertfe și se închina la dumnezeii lor.
\par 3 Așa s-a lipit Israel de Baal-Peor, pentru care s-a aprins mânia lui Dumnezeu asupra lui Israel.
\par 4 Și a zis Domnul către Moise: "Ia pe toate căpeteniile poporului și le spânzură de copaci pentru Domnul înainte de asfințitul soarelui, ca să se abată de la Israel iuțimea mâniei Domnului".
\par 5 Atunci a zis Moise către judecătorii lui Israel: "Ucideți fiecare pe oamenii voștri care s-au lipit de Baal-Peor".
\par 6 Dar iată oarecare din fiii lui Israel a venit și a adus între frații săi o madianită, în ochii lui Moise și în ochii întregii obști a fiilor lui Israel, când plângeau ei la ușa cortului adunării.
\par 7 Atunci Finees, fiul lui Eleazar, fiul preotului Aaron, văzând aceasta, s-a sculat din mijlocul obștii și, luând în mână lancea sa,
\par 8 A intrat după israelit în sălaș și i-a străpuns pe amândoi, pe israelit și pe femeie, în pântece; și a încetat pedepsirea fiilor lui Israel.
\par 9 Cei ce au murit de pedeapsa aceasta au fost douăzeci și patru de mii.
\par 10 Și a grăit Domnul cu Moise și a zis:
\par 11 "Finees, feciorul lui Eleazar, fiul preotului Aaron, a abătut mânia Mea de la fiii lui Israel, râvnind între ei pentru Mine, și n-am mai pierdut pe fiii lui Israel în mânia Mea;
\par 12 De aceea spune-i că voi încheia cu el legământul Meu de pace,
\par 13 Și va fi pentru el și pentru urmașii lui de după el legământ de preoție veșnică, căci a arătat râvnă pentru Dumnezeul său și a ispășit păcatul fiilor lui Israel".
\par 14 Numele israelitului ucis, care a fost omorât cu madianita, era Zimri, fiul lui Salu, căpetenia seminției lui Simeon;
\par 15 Iar numele madianitei ucise era Cozbi, fiica lui Țur, căpetenia unei seminții ieșită dintr-o casă patriarhală din Madian.
\par 16 Și a grăit Domnul cu Moise și a zis: "Vorbește fiilor lui Israel și le zi:
\par 17 Socotiți pe Madianiți dușmanii voștri și omorâți-i, bateți-i, că s-au purtat cu voi dușmănos întru vicleșugul lor,
\par 18 Ademenindu-vă cu Peor și cu Cozbi, sora lor, fiica unei căpetenii a Madianiților, care a fost ucisă în ziua urgiei celei pentru Peor".

\chapter{26}

\par 1 După această pedeapsă, a grăit Domnul către Moise și către Eleazar, fiul preotului Aaron, și a zis:
\par 2 "Numărați toată obștea fiilor lui Israel de la douăzeci de ani în sus, pe toți cei buni de război în Israel, după familiile lor!"
\par 3 Atunci Moise și preotul Eleazar le-au grăit în șesurile Moabului, la Iordan, în dreptul Ierihonului, și le-au zis:
\par 4 "Numărați pe toți de la douăzeci de ani în sus", cum a grăit Domnul lui Moise și fiilor lui Israel, care au ieșit din pământul Egiptului.
\par 5 Ruben, este întâiul născut al lui Israel. Fiii lui Ruben: din Enoh, neamul lui Enoh; din Falu, neamul lui Falu;
\par 6 Din Hețron, neamul lui Hețron; din Carmi, neamul lui Carmi;
\par 7 Acestea sunt neamurile lui Ruben; și s-au numărat patruzeci și trei de mii șapte sute treizeci.
\par 8 Fiul lui Falu: Eliab.
\par 9 Fiii lui Eliab: Nemuel, Datan și Abiron. Datan și Abiron sunt aceia care, chemați fiind în adunare, au stârnit răzvrătire împotriva lui Moise și a lui Aaron împreună cu părtașii lui Core, când aceștia au stârnit răzvrătire împotriva Domnului
\par 10 Și și-a deschis pământul gura sa și i-a înghițit pe ei și pe Core; și împreună cu ei au murit și părtașii lor, când focul a mistuit două sute cincizeci de oameni și au rămas ei ca semn.
\par 11 Însă fiii lui Core n-au murit.
\par 12 Fiii lui Simeon, după familiile lor, sunt: din Nemuel, neamul lui Nemuel; din Iamin, neamul lui Iamin; din Iachin; neamul lui Iachin;
\par 13 Din Zerah, neamul lui Zerah; din Saul, neamul lui Saul.
\par 14 Acestea sunt neamurile cele din Simeon, care s-au găsit la numărătoarea lor: douăzeci și două de mii două sute.
\par 15 Fiii lui Gad, după neamurile lor: din Țefon neamul Țefonienilor; din Haghi, neamul Haghiților; din Șunie, neamul Șunienilor;
\par 16 Din Ozni, neamul Oznienilor;
\par 17 Din Eri, neamul Erienilor; din Arod, neamul Arodeilor; din Areli, neamul Arelienilor.
\par 18 Acestea sunt neamurile fiilor lui Gad, care la numărătoarea lor s-au găsit patruzeci de mii cinci sute.
\par 19 Fiii lui Iuda sunt: Er și Onan, Șela, Fares și Zara; însă Er și Onan au murit în pământul Canaanului.
\par 20 Și fiii lui Iuda, după neamurile lor, sunt: din Șela, neamul Șelaenilor; din Fares, neamul Fareseilor; din Zara, neamul Zaraenilor.
\par 21 Iar fiii lui Fares sunt: din Esron neamul Esroneilor; din Hamul, neamul Hamulienilor.
\par 22 Acestea sunt neamurile din Iuda și la numărătoarea lor s-au găsit șaptezeci și șase de mii cinci sute.
\par 23 Fiii lui Isahar, după neamurile lor, sunt: din Tola, neamul Tolaenilor; din Fuva, neamul Fuvaenilor;
\par 24 Din Iașub, neamul Iașubienilor; din Șimron, neamul Șimronienilor.
\par 25 Acestea sunt neamurile din Isahar și la numărătoarea lor s-au găsit șaizeci și patru de mii trei sute.
\par 26 Fiii lui Zabulon, după neamurile lor, sunt: din Sered, neamul Seredienilor; din Elon, neamul Elonienilor; din Iahleil, neamul Iahleililor.
\par 27 Acestea sunt neamurile din Zabulon și la numărătoarea lor s-au găsit șaizeci de mii cinci sute.
\par 28 Fiii lui Iosif sunt: Manase și Efraim.
\par 29 Fiii lui Manase, după neamurile lor, sunt: din Machir, neamul Machirienilor; din Machir s-a născut Galaad și din Galaad este neamul Galaadenilor.
\par 30 Fiii lui Galaad sunt: din Iezer, neamul Iezerienilor; din Helec, neamul Helecienilor;
\par 31 Din Asriel, neamul Asrielienilor; din Șechem, neamul Șechemienilor;
\par 32 Din Șemida, neamul Șemidienilor; din Hefer, neamul Heferienilor.
\par 33 Salfaad, fiul lui Hefer, n-a avut fii, ci numai fiice și numele fiicelor lui Salfaad sunt: Mahla, Noa, Hogla, Milca și Tirța.
\par 34 Acestea sunt neamurile lui Manase și la numărătoarea lor s-au găsit cincizeci și două de mii șapte sute.
\par 35 Fiii lui Efraim, după neamurile lor, sunt: din Șutelah, neamul Șutelahienilor; din Becher, neamul Becherienilor; din Tahan, neamul Tahanienilor.
\par 36 Iar fiii lui Șutelah sunt: din Eran, neamul Eranienilor.
\par 37 Acestea sunt neamurile fiilor lui Efraim și la numărătoarea lor s-au găsit treizeci și două de mii cinci sute. Aceștia sunt fiii lui Iosif după neamurile lor.
\par 38 Fiii lui Veniamin, după neamurile lor, sunt; din Bela, neamul Belaienilor; din Așbel, neamul Așbelienilor; din Ahiram, neamul Ahiramienilor;
\par 39 Din Șefufam, neamul Șefufamienilor; din Hufam, neamul Hufamienilor.
\par 40 Iar fiii lui Bela sunt: Ard și Naaman: din Ard, neamul Ardienilor și din Naaman, neamul Naamanienilor.
\par 41 Aceștia sunt fiii lui Veniamin după neamurile lor și la numărătoare s-au găsit patruzeci și cinci de mii șase sute.
\par 42 Fiii lui Dan, după neamurile lor, sunt: din Șuham, neamul Șuhamienilor. Acestea sunt familiile lui Dan, după neamurile lor.
\par 43 Și neamurile lui Șuham, la numărătoarea lor, au fost de toate șaizeci și patru de mii patru sute.
\par 44 Fiii lui Așer, după neamurile lor, sunt: din Imna, neamul Imnaenilor; din Ișba, neamul Ișbaenilor; din Verie, neamul Verienilor.
\par 45 Din fiii lui Verie: din Heber, neamul Heberienilor; din Malchiel, neamul Malchielilor.
\par 46 Și numele fiicei lui Așer a fost Serah.
\par 47 Acestea sunt neamurile fiilor lui Așer și la numărătoare s-au găsit cincizeci și trei de mii patru sute.
\par 48 Fiii lui Neftali, după neamurile lor, sunt: din Iahțeel, neamul Iahțeelienilor; din Guni, neamul Gunienilor;
\par 49 Din Iețer, neamul Iețerienilor; din Șilem, neamul Șilemienilor.
\par 50 Acestea sunt neamurile lui Neftali, după familiile lor și la numărătoare s-au găsit patruzeci și cinci de mii patru sute.
\par 51 Iată numărul fiilor lui Israel, celor ce au intrat la numărătoare: șase sute una mii șapte sute treizeci.
\par 52 Apoi a grăit Domnul cu Moise și a zis:
\par 53 "Acestora să li se împartă spre moștenire pământul, după numărul numelor;
\par 54 Celor mai mulți să le dai moșie mai mare, iar celor mai puțini să le dai moșie mai mică; fiecăruia  să se dea moșie potrivit cu numărul celor ce au intrat la numărătoare.
\par 55 Pământul să-l împărțiți prin sorți; după numele semințiilor părinților lor să-și primească și părțile:
\par 56 Prin sorți să le împarți moșia, atât celor mulți la număr, cât și celor puțini la număr".
\par 57 Leviții, care au intrat la numărătoare, după neamurile lor, sunt aceștia: din Gherșon, neamul Gherșonienilor; din Cahat, neamul Cahatienilor, din Merari, neamul Merarienilor.
\par 58 Iată neamurile lui Levi: neamul lui Libni, neamul lui Hebron, neamul lui Mahli, neamul lui Muși și neamul lui Core. Din Cahat s-a născut Amram.
\par 59 Numele femeii lui Amram a fost Iohabed, fiica lui Levi, pe care a născut-o femeia lui Levi în Egipt, iar ea a născut lui Amram pe Aaron, pe Moise și pe Mariam, sora lor.
\par 60 Lui Aaron i s-au născut: Nadab și Abiud, Eleazar și Itamar.
\par 61 Dar Nadab și Abiud au murit când au adus foc străin înaintea Domnului, în pustiul Sinai.
\par 62 Și s-au numărat toți cei de parte bărbătească de la o lună în sus și s-au găsit douăzeci și trei de mii; căci aceștia nu fuseseră numărați împreună cu fiii lui Israel, pentru că nu li s-a dat moștenire printre fiii lui Israel.
\par 63 Aceștia sunt cei numărați de Moise și Eleazar preotul, care au numărat pe fiii lui Israel în șesurile Moabului, lângă Iordan, în dreptul Ierihonului.
\par 64 În numărul lor nu se afla niciunul din fiii lui Israel numărați de Moise și de preotul Aaron, în pustiul Sinai,
\par 65 Căci Domnul le zisese acestora că vor muri toți în pustie, - și n-au rămas din ei niciunul, afară de Caleb, fiul lui Iefone și de Iosua, fiul lui Navi.

\chapter{27}

\par 1 Atunci au venit fetele lui Salfaad, fiul lui Hefer, fiul lui Galaad, fiul lui Machir, din neamul lui Manase, fiul lui Iosif, ale căror nume sunt acestea: Mahla, Noa, Hogla, Milca și Tirța,
\par 2 Și au stat înaintea lui Moise, a lui Eleazar preotul, înaintea căpeteniilor și înaintea întregii obști, la ușa cortului adunării, și au zis:
\par 3 "Tatăl nostru a murit în pustie; el n-a fost din numărul celor care s-au ridicat împotriva Domnului cu adunarea lui Core, ci a murit pentru păcatul său și feciori n-a avut.
\par 4 De ce să piară numele tatălui nostru din neamul lui, pentru că n-are fii? Dă-ne și nouă moșie între frații tatălui nostru!"
\par 5 Moise însă a adus cererea lor înaintea Domnului;
\par 6 Iar Domnul a zis către Moise:
\par 7 "Drept au grăit fetele lui Salfaad; dăruiește-le și lor moștenire între frații tatălui lor și trece-le lor moșia tatălui lor.
\par 8 Iar fiilor lui Israel să le grăiești și să le spui: De va muri cineva, neavând fiu, să dați partea lui fiicei lui.
\par 9 Iar de nu are nici fiică, să dați partea lui fraților lui.
\par 10 De nu are însă nici frați, să dați partea lui fraților tatălui lui.
\par 11 Iar de nu are tatăl său frați, să dați partea lui rudeniei celei mai de aproape din neamul lui, ca să moștenească ale lui. Aceasta să fie pentru fiii lui Israel ca o hotărâre din lege, cum a poruncit Domnul lui Moise".
\par 12 Apoi a zis Domnul către Moise: "Suie-te pe acest munte, care este dincoace de Iordan, adică pe muntele Nebo, și privește pământul Canaanului, pe care am să-l dau fiilor lui Israel de moștenire.
\par 13 Iar după ce-l vei vedea, te vei adăuga și tu la poporul tău, cum s-a adăugat Aaron, fratele tău, pe muntele Hor.
\par 14 Pentru că v-ați împotrivit poruncii Mele în pustiul Sinai, în vremea tulburării obștii, ca să arătați înaintea ochilor lor sfințenia Mea la ape, adică la apele Meriba de la Cadeș, în pustiul Sinai".
\par 15 Moise însă a grăit Domnului și a zis:
\par 16 "Domnul Dumnezeul duhurilor și a tot trupul să rânduiască peste obștea aceasta un om,
\par 17 Care să iasă înaintea ei și care să intre înaintea ei, care să-i ducă și să-i aducă, ca să nu rămână obștea Domnului ca oile ce n-au păstor".
\par 18 Iar Domnul a zis către Moise: "Ia-ți pe Iosua, fiul lui Navi, om cu duh într-însul, pune-ți peste el mâna ta;
\par 19 Apoi du-l înaintea preotului Eleazar, înaintea a toată obștea și dă-i povețe înaintea ochilor lor;
\par 20 Dă-i din slava ta, ca să-l asculte toată obștea fiilor lui Israel.
\par 21 După aceea să stea înaintea preotului Eleazar și acesta va întreba de hotărârile Domnului prin ajutorul Urimului: după cuvântul acestuia să iasă și după cuvântul acestuia să intre el și toți fiii lui Israel cei împreună cu dânsul și toată obștea".
\par 22 Și a făcut Moise cum i-a poruncit Domnul Dumnezeu: a luat pe Iosua și l-a pus înaintea preotului Eleazar și a toată obștea.
\par 23 Apoi și-a pus peste el mâinile sale și i-a dat povețe, cum zisese Domnul prin Moise.

\chapter{28}

\par 1 Apoi iarăși a grăit Domnul cu Moise și a zis:
\par 2 "Poruncește fiilor lui Israel și le spune: Darurile Mele, dările Mele, jertfele Mele cele întru miros cu bună mireasmă, îngrijiți să Mi se aducă la sărbătorile Mele.
\par 3 Spune-le: Iată jertfele care trebuie să le aduceți Domnului: doi miei de câte un an fără meteahnă, ardere de tot necontenită, pe fiecare zi;
\par 4 Un miel să-l aduci dimineața și pe celălalt miel să-l aduci seara.
\par 5 Jertfă de pâine să aduci a zecea parte de efă de făină de grâu, amestecată cu un sfert de hin de untdelemn;
\par 6 Aceasta este ardere de tot necontenită și care a fost săvârșită la Muntele Sinai, spre miros cu bună mireasmă și ca jertfă Domnului.
\par 7 La ea să aduci turnare un sfert de hin de vin la un miel; și turnarea de vin a Domnului s-o torni la loc sfânt.
\par 8 Celălalt miel să-l aduci spre seară, cu darul lui de pâine și cu turnarea lui să-l aducă jertfă, mireasmă plăcută Domnului.
\par 9 Iar în ziua odihnei să aduceți doi miei de câte un an, fără meteahnă, și ca jertfă două zecimi de efă de făină de grâu, frământată cu untdelemn și cu turnarea ei.
\par 10 Aceasta este ardere de tot pentru ziua odihnei afară de arderea de tot cea necontenită cu turnarea ei.
\par 11 La începutul lunilor voastre să aduceți Domnului ardere de tot: din cireadă, doi viței, iar din turmă, un berbec și șapte miei de câte un an fără meteahnă.
\par 12 Iar ca dar de pâine, câte trei zecimi de efă făină de grâu, frământată cu untdelemn, la fiecare vițel, și două zecimi de efă făină de grâu, frământată cu untdelemn, ca dar de pâine la berbec,
\par 13 Și câte o zecime de efă făină de grâu, frământată cu untdelemn, ca dar de pâine la fiecare miel. Aceasta este ardere de tot, mireasmă plăcută, jertfă Domnului.
\par 14 Turnare la ele să fie jumătate hin de vin de fiecare vițel, a treia parte hin de berbec și un sfert de hin la fiecare miel. Aceasta este ardere de tot pentru fiecare început de lună, la toate lunile anului.
\par 15 Să mai aduceți Domnului și un țap, jertfă pentru păcat, afară de arderea de tot cea necontenită, și să-l aducă cu turnarea lui.
\par 16 În ziua a paisprezecea a lunii întâi sunt Paștile Domnului.
\par 17 În ziua a cincisprezecea este sărbătoare. Șapte zile să mâncați azime.
\par 18 În ziua întâi să aveți adunare sfântă și nici un fel de lucru să nu faceți;
\par 19 Și să aduceți Domnului jertfă, ardere de tot: din cireadă, doi viței, iar din turmă, un berbec și șapte miei de câte un an; aceștia să fie fără meteahnă.
\par 20 Cu ei să aduceți dar de pâine, făină de grâu frământată cu untdelemn, trei zecimi de efă de fiecare vițel, două zecimi de efă la berbec,
\par 21 Și câte o zecime de efă să aduci cu fiecare din cei șapte miei;
\par 22 Să aduceți un țap jertfă pentru păcat, pentru curățirea voastră.
\par 23 Acestea să le aduceți, pe lângă arderea de tot de dimineață, care este ardere de tot necontenită.
\par 24 Tot așa să aduceți și în fiecare din cele șapte zile: pâine, jertfă, mireasmă plăcută Domnului, pe lângă arderea de tot cea necontenită și cu turnarea ei.
\par 25 În ziua a șaptea să aveți adunare sfântă și nici un lucru să nu lucrați.
\par 26 În ziua celor dintâi roade, când aduceți Domnului prinosul nou de pâine, la încheierea săptămânilor, să aveți adunare sfântă și nici un lucru să nu lucrați.
\par 27 Să aduceți ardere de tot spre miros de bună mireasmă Domnului: din cireadă, doi vitei, iar din turmă, un berbec și șapte miei de câte un an fără meteahnă.
\par 28 Cu ei să aduceți dar de pâine, făină de grâu frământată cu untdelemn: trei zecimi de efă la fiecare vițel, două zecimi de efă la berbec
\par 29 Și o zecime de efă la fiecare din cei șapte miei.
\par 30 Să aduceți un țap jertfă pentru păcat, spre curățirea voastră.
\par 31 Acestea să mi le aduceți cu turnările lor, afară de arderile de tot neîncetate cu darul lor de pâine, care se aduc de obicei; acestea trebuie să fie curate".

\chapter{29}

\par 1 "În ziua întâi a lunii a șaptea de asemenea să aveți adunare sfântă, și nici un lucru să nu lucrați; pe aceasta să o socotiți o zi a suflatului în trâmbițe.
\par 2 Să aduceți ardere de tot, spre miros plăcut Domnului: un vițel, un berbec, șapte miei de câte un an fără meteahnă;
\par 3 La ei, ca dar de pâine, făină de grâu, frământată cu untdelemn: trei zecimi de efă la vițel, două zecimi de efă la berbec
\par 4 Și câte o zecime de efă la fiecare din cei șapte miei.
\par 5 Din turma de capre să aduceți un țap, jertfă pentru păcat, spre curățirea voastră.
\par 6 Acestea să le aduceți jertfe pe lângă arderea de tot, cu darul de pâine și turnarea lui de la lună nouă și pe lângă arderea de tot necontenită cu darul ei de pâine și turnarea lui, după rânduială, întru miros de bună mireasmă Domnului.
\par 7 În ziua a zecea a acestei luni să aveți adunare sfântă, să postiți și nici un lucru să nu faceți.
\par 8 Să aduceți ardere de tot Domnului spre miros de bună mireasmă: un vițel" un berbec și șapte miei de câte un an.
\par 9 La ei să aduceți dar de pâine, făină de grâu frământată cu untdelemn, trei zecimi de efă la vițel, două zecimi de efă la berbec
\par 10 Și câte o zecime de efă la fiecare din cei șapte miei.
\par 11 Iar din turma de capre să aduceți un țap jertfă pentru păcat, spre curățirea voastră; acestea pe lingă jertfa pentru păcat din ziua curățirii și pe lângă arderea de tot cea necontenită cu darul ei de pâine și turnarea ei, care se aduce după rânduială jertfă Domnului, spre miros cu bună mireasmă.
\par 12 În ziua a cincisprezecea a lunii a șaptea să aveți iar adunare sfântă; nici un lucru să nu lucrați și să sărbătoriți sărbătoarea Domnului șapte zile.
\par 13 În ziua întâi să aduceți ardere de tot, jertfă, mireasmă plăcută Domnului: din cireadă, treisprezece viței, iar din turmă, doi berbeci, paisprezece miei de câte un an; dar să fie fără meteahnă.
\par 14 Cu ei, ca dar de pâine, să se aducă făină de grâu frământată cu untdelemn: trei zecimi de efă cu fiecare din cei treisprezece vilei, două zecimi de efă cu fiecare din cei doi berbeci,
\par 15 Și câte o zecime de efă de fiecare din cei paisprezece miei;
\par 16 Iar din turma de capre, un țap, jertfă pentru păcat, peste arderea de tot necontenită și darul ei de pâine cu turnarea ei.
\par 17 A doua zi să se aducă doisprezece vitei, doi berbeci, paisprezece miei de câte un an, fără meteahnă;
\par 18 Cu ei să se aducă dar de pâine și turnare: la vilei, la berbeci și la miei, după numărul lor, cum e rânduit;
\par 19 Iar din turma de capre, un țap, jertfă pentru păcat; acestea să le aduceți în afară de arderea de tot necontenită și de darul de pâine cu turnarea ei.
\par 20 A treia zi să aduceți unsprezece vilei, doi berbeci și paisprezece miei de câte un an, fără meteahnă;
\par 21 Și cu ei dar de pâine și turnare pentru vitei, pentru berbeci și pentru miei, după numărul lor, după rânduială;
\par 22 Iar din turma de capre să aduceți un țap, jertfă pentru păcat, peste arderea de tot necontenită cu darul de pâine și turnarea ei.
\par 23 A patra zi să aduceți zece vilei, doi berbeci și paisprezece miei de câte un an, fără meteahnă;
\par 24 Cu ei să aduceți dar de pâine și turnare pentru vitei, pentru berbeci și pentru miei, după numărul lor, cum e rânduiala;
\par 25 Iar din turma de capre să aduceți un țap, jertfă pentru păcat, pe lângă arderea de tot necontenită cu darul de pâine și turnarea ei.
\par 26 În ziua a cincea să aduceți nouă vitei, doi berbeci și paisprezece miei de câte un an, fără meteahnă;
\par 27 Și cu ei dar de pâine și turnare pentru vilei, pentru berbeci și pentru miei, după numărul lor, cum e rânduiala;
\par 28 Iar din turma de capre să aduceți un țap, jertfă pentru păcat, peste arderea necontenită cu darul de pâine și cu turnarea ei.
\par 29 În ziua a șasea să aduceți opt vitei, doi berbeci și paisprezece miei, fără meteahnă;
\par 30 Și eu ei dar de pâine și turnare pentru vilei, pentru berbeci și pentru miei, după numărul lor, cum e rânduiala;
\par 31 Iar din turma de capre, un țap, jertfă pentru păcat, peste arderea de tot necontenită cu darul de pâine și cu turnarea ei.
\par 32 În ziua a șaptea să aduceți șapte vitei, doi berbeci și paisprezece miei fără meteahnă;
\par 33 Și cu ei dar de pâine și turnare pentru vilei, pentru berbeci și pentru miei, după numărul lor, cum e rânduiala;
\par 34 Iar din turma de capre să aduceți un țap, jertfă pentru păcat, peste arderea de tot necontenită cu darul de pâine și cu turnarea ei.
\par 35 În ziua a opta să aveți încheierea sărbătorii; nici un lucru să nu lucrați,
\par 36 Și să aduceți ardere de tot, jertfă, mireasmă plăcută Domnului: un vițel, un berbec și șapte miei fără meteahnă.
\par 37 Și cu ei dar de pâine și turnare pentru vițel, pentru berbec și pentru miei, după numărul lor, cum e rânduiala.
\par 38 Iar din turma de capre să aduceți un țap, jertfă pentru păcat, pe lângă arderea de tot necontenită cu darul de pâine și cu turnarea ei.
\par 39 Acestea să le aduceți Domnului la sărbătorile voastre, pe lângă arderile de tot ale voastre cu darurile de pâine ale voastre și cu turnările voastre și jertfele voastre de bună voie, pe care le aduceți după făgăduință sau din evlavie".
\par 40 Moise a spus fiilor lui Israel toate cele ce-i poruncise Domnul.

\chapter{30}

\par 1 Așadar, a grăit Moise către căpeteniile semințiilor fiilor lui Israel și le-a zis: "Iată ce poruncește Domnul:
\par 2 Omul care va face făgăduință Domnului sau se va jura cu jurământ, punând legătură asupra sufletului său, să nu-și calce cuvântul, ci să împlinească toate câte au ieșit din gura lui.
\par 3 Dacă vreo femeie va da făgăduință Domnului și va pune asupra sa legământul, în casa părintelui său, în tinerețea sa,
\par 4 Și va auzi tatăl făgăduința ei și legământul ce ea și-a pus asupra sufletului său, și va tăcea tatăl ei asupra acestora, atunci toate făgăduințele ei se vor ține și orice legământ și-ar fi pus ea asupra sufletului său se va ține.
\par 5 Iar dacă tatăl ei, auzind, o va opri, atunci toate făgăduințele ei și legămintele ce ea și-ar fi pus asupra sufletului său nu se vor ține și Domnul o va ierta, pentru că a oprit-o tatăl ei.
\par 6 Dacă însă ea se va mărita și va fi asupra ei făgăduința sau cuvântul gurii sale, cu care s-a legat pe sine,
\par 7 Și va auzi bărbatul ei și, auzind-o, va tăcea, atunci făgăduințele ei se vor ține și legămintele ce ea și-a pus asupra sufletului său se vor ține.
\par 8 Iar dacă bărbatul ei, auzind, o va opri și va lepăda făgăduința ei, care este asupra ei, și cuvântul gurii ei cu care ea s-a legat pe sine, atunci acestea nu se vor ține, pentru că i le-a oprit bărbatul ei, și Domnul o va ierta.
\par 9 Iar făgăduința văduvei și a celei despărțite și orice legământ și-ar pune aceasta asupra sufletului ei se va ține.
\par 10 Dacă însă în casa bărbatului său a dat făgăduință sau și-a pus legământ asupra sufletului său cu jurământ,
\par 11 Și bărbatul ei a auzit și a tăcut asupra acesteia și n-a oprit-o, atunci toate făgăduințele ei se vor ține și orice legământ și-ar fi pus asupra sufletului său se va ține.
\par 12 Dacă însă bărbatul ei, auzind, a lepădat făgăduințele, atunci toate făgăduințele ieșite din gura ei și legămintele sufletului său nu se vor ține, pentru că bărbatul ei le-a desființat și Domnul o va ierta.
\par 13 Orice făgăduință și orice legământ cu jurământ pentru smerirea sufletului ei, bărbatul ei îl poate întări și tot bărbatul ei îl poate și desființa.
\par 14 Dacă însă bărbatul ei a tăcut despre aceasta, zi după zi, prin aceasta el a întărit toate făgăduințele ei și toate legămintele ce sunt asupra ei le-a întărit, pentru că el a auzit și a tăcut.
\par 15 Iar dacă bărbatul le-a lepădat după ce le-a auzit, atunci a luat el asupra sa păcatul ei".
\par 16 Acestea sunt legile, pe care Domnul le-a poruncit lui Moise asupra legămintelor dintre bărbat și femeia lui, dintre tată și fiica lui, cât aceasta este tânără și se află în casa tatălui ei.

\chapter{31}

\par 1 Apoi a grăit Domnul cu Moise și a zis:
\par 2 "Răzbună pe fiii lui Israel împotriva Madianiților, și apoi te vei adăuga la poporul tău".
\par 3 Iar Moise a grăit poporului și a zis: "Înarmați dintre voi oameni pentru război, ca să meargă împotriva Madianiților și să săvârșească răzbunarea Domnului asupra Madianiților.
\par 4 Din toate semințiile fiilor lui Israel să trimiteți la război, câte o mie din fiecare seminție".
\par 5 Și și-au ales din miile lui Israel, câte o mie din fiecare seminție, adică douăsprezece mii de oameni, înarmați pentru război.
\par 6 Pe aceștia i-a trimis Moise la război, câte o mie din fiecare seminție; și cu ei a trimis la război pe Finees, fiul preotului Eleazar, fiul lui Aaron; și acesta avea în mâinile sale vasele sfinte și trâmbițele de strigare.
\par 7 Și au lovit ei pe Madian, cum poruncise Domnul lui Moise, și au ucis pe toți cei de parte bărbătească.
\par 8 Împreună cu ucișii lor au căzut și regii madianiți: Evi, Rechem, Țur, Hur și Reba - cinci regi madianiți - și Valaam, fiul lui Beor, a căzut de sabie, împreună cu ucișii acelora.
\par 9 Iar pe femeile Madianiților și pe copiii lor le-au luat fiii lui Israel în robie; și toate vitele lor, toate turmele lor și toate avuțiile lor le-au luat pradă.
\par 10 Toate cetățile lor din ținuturile lor cu toate satele lor le-au ars cu foc.
\par 11 Toată prada și tot ce-au apucat de la om până la dobitoc au luat cu ei.
\par 12 Robii, prada și cele apucate le-au dus la Moise, la preotul Eleazar și la obștea fiilor lui Israel, în tabăra din șesul Moabului, care este lângă Iordan, în fața Ierihonului.
\par 13 În întâmpinarea lor au ieșit din tabără Moise, Eleazar preotul și toate căpeteniile obștii.
\par 14 Atunci s-a mâniat Moise pe căpeteniile oștirii, pe căpeteniile miilor și pe sutașii care se întorseseră de la război, și le-a zis Moise:
\par 15 "Pentru ce ați lăsat vii toate femeile?
\par 16 Căci ele, după sfatul lui Valaam, au făcut pe fiii lui Israel să se abată de la cuvântul Domnului, pentru Peor, pentru care a venit pedeapsă asupra obștii Domnului.
\par 17 Ucideți dar toți copiii de parte bărbătească și toate femeile ce-au cunoscut bărbat, ucideți-le.
\par 18 Iar pe fetele care n-au cunoscut bărbat, lăsați-le pe toate vii pentru voi.
\par 19 Și să ședeți afară din tabără șapte zile; toți cei ce ați ucis om și v-ați atins de om ucis să vă curățiți în ziua a treia și în ziua a șaptea, și voi și robii voștri.
\par 20 Toate hainele, toate lucrurile de piele, tot ce este făcut din păr de capră și toate vasele de lemn să le curățiți".
\par 21 Apoi a zis preotul Eleazar oștenilor care fuseseră la război: "Hotărârea legii, pe care a dat-o Domnul lui Moise, este aceasta:
\par 22 Aurul, argintul, arama, fierul, plumbul, cositorul
\par 23 Și tot ce trece prin foc, să le treceți prin foc, ca să se curețe; afară de aceasta și cu apă de curățire să le curățiți; iar toate cele ce nu se pot trece prin foc, să le treceți prin apă.
\par 24 Hainele voastre să le spălați în ziua a șaptea și să vă curățiți, iar după aceea veți intra în tabără".
\par 25 Iarăși a grăit Domnul cu Moise și a zis:
\par 26 "Socotește prada de război, de la om până la dobitoc, împreună cu Eleazar preotul și cu căpeteniile semințiilor obștii;
\par 27 Apoi împarte prada în două, între oștenii care au fost la bătălie și între toată obștea.
\par 28 De la oștenii care au fost la război, ia dare pentru Domnul, câte un suflet la cinci sute, din oameni, din vite, din asini și din oi.
\par 29 Acestea să le iei din partea lor și să le dai preotului Eleazar ca dar înălțat Domnului.
\par 30 Iar din jumătatea cuvenită fiilor lui Israel să iei unul la cincizeci din oameni, din vite, din asini și din oi; și pe acestea să le dai leviților, care slujesc la cortul Domnului".
\par 31 Și a făcut Moise și Eleazar preotul cum poruncise Domnul lui Moise.
\par 32 Atunci s-a găsit pradă rămasă din cele luate și aduse de cei ce fuseseră la război: șase sute șaptezeci și cinci de mii de oi;
\par 33 Șaptezeci și două de mii de boi;
\par 34 Asini, șaizeci și una de mii;
\par 35 Femei, care n-au cunoscut bărbat, de toate, treizeci și două de mii de suflete.
\par 36 Jumătate, partea celor ce fuseseră la război, după numărătoare, a fost: oi trei sute treizeci și șapte de mii cinci sute.
\par 37 Și darea Domnului din oi a fost: șase sute șaptezeci și cinci;
\par 38 Boi treizeci și șase de mii, și din aceștia darea Domnului a fost șaptezeci și doi;
\par 39 Asini, treizeci de mii cinci sute, și din ei darea Domnului a fost șaizeci și unul;
\par 40 Oameni, șaisprezece mii, și din ei darea Domnului a fost treizeci și două suflete.
\par 41 Și a dat Moise darea Domnului lui Eleazar preotul, cum poruncise Domnul lui Moise.
\par 42 Iar partea fiilor lui Israel, pe care a luat-o Moise de la cei ce fuseseră la război, a fost:
\par 43 Oi, trei sute treizeci și șapte de mii cinci sute;
\par 44 Boi, treizeci și șase de mii;
\par 45 Asini, treizeci de mii cinci sute;
\par 46 Oameni, șaisprezece mii.
\par 47 Din această parte a fiilor lui Israel, a luat Moise unul la cincizeci din oameni și din vite, și le-a dat leviților, care făceau slujbă în cortul Domnului, după cum poruncise Domnul lui Moise.
\par 48 Atunci au venit la Moise căpeteniile oștirii, căpeteniile peste mii și sutașii, și au zis către Moise:
\par 49 "Robii tăi au numărat pe oștenii care ne-au fost încredințați și n-a lipsit niciunul din ei.
\par 50 Și iată noi am adus prinos Domnului, fiecare ce am putut dobândi din lucrurile de aur: lanțuri, brățări, inele, cercei și salbe, pentru curățirea sufletelor noastre înaintea Domnului".
\par 51 Și a luat Moise și Eleazar de la ei toate aceste lucruri de aur.
\par 52 Aurul acesta, care s-a adus prinos Domnului de către căpeteniile peste mii și peste sute, a fost tot șaisprezece mii șapte sute cincizeci de sicli.
\par 53 Oștenii au prădat fiecare pentru ei.
\par 54 Și a luat Moise și preotul Eleazar aurul de la căpeteniile peste mii și peste sute și l-au dus în cortul adunării, pentru pomenirea fiilor lui Israel înaintea Domnului.

\chapter{32}

\par 1 Iar fiii lui Ruben și fiii lui Gad aveau foarte multe turme; dar văzând că pământul Iazer și pământul Galaad sunt locuri bune pentru turme,
\par 2 Au venit fiii lui Gad și fiii lui Ruben și au grăit cu Moise, cu Eleazar preotul și cu căpeteniile obștii și au zis:
\par 3 "Atarotul, Dibonul, Iazerul, Nimra, Heșbonul, Eleale, Sevam, Nebo și Beon,
\par 4 Ținuturile, pe care Domnul le-a lovit înaintea obștii lui Israel, sunt pământuri bune pentru turme, și robii tăi au turme".
\par 5 Și au mai zis: "De-am aflat trecere în ochii tăi, dă pământurile acestea robilor tăi în stăpânire și nu ne trece peste Iordan".
\par 6 Moise însă a zis către fiii lui Gad și către fiii lui Ruben: "Frații voștri se duc la război, iar voi să rămâneți aici?
\par 7 Pentru ce întoarceți inima fiilor lui Israel să nu treacă în pământul pe care Domnul li-l dă?
\par 8 Așa au făcut și părinții voștri când i-am trimis din Cadeș-Barne ca să cerceteze țara:
\par 9 Au mers până în valea Eșcol, au văzut pământul și au abătut inima fiilor lui Israel, ca să nu meargă aceștia în pământul pe care Domnul li-l dă.
\par 10 Dar s-a aprins în ziua aceea mânia Domnului și S-a jurat și a zis:
\par 11 "Oamenii aceștia, care au ieșit din Egipt și care sunt de douăzeci de ani și mai mari și cunosc binele și răul, nu vor vedea pământul, pentru care Eu M-am jurat lui Avraam și lui Isaac și lui Iacov.
\par 12 Pentru că nu Mi s-au supus Mie, afară de Caleb, fiul lui Iefone Chenezul, și de Iosua, fiul lui Navi, pentru că aceștia s-au supus Domnului".
\par 13 S-a aprins atunci mânia Domnului asupra lui Israel și i-a purtat prin pustie patruzeci de ani, până când s-a sfârșit tot neamul care făcuse rău înaintea Domnului.
\par 14 Și iată acum, în locul părinților voștri v-ați ridicat voi, sămânța păcătoșilor, ca să sporiți încă și mai mult iuțimea mâniei Domnului asupra lui Israel.
\par 15 Dacă vă veți abate de la El, iarăși va lăsa pe Israel în pustie și veți pierde tot poporul acesta".
\par 16 Iar ei, apropiindu-se de el, au zis: "Noi ne vom face aici staule pentru turmele noastre și cetăți pentru copiii noștri;
\par 17 Iar noi înșine cei dintâi ne vom înarma și vom merge înaintea fiilor lui Israel, până ce îi vom duce la locurile lor; iar copiii noștri vor rămâne în cetățile întărite, pentru ca să nu fie în primejdie din partea oamenilor locului.
\par 18 Nu ne vom întoarce la casele noastre, până când fiii lui Israel nu vor intra fiecare în moștenirea sa;
\par 19 Căci nu vom lua împreună cu ei moștenire dincolo de Iordan și nici mai departe, dacă ni se va da parte dincoace de Iordan, spre răsărit".
\par 20 Atunci a zis Moise către ei: "De veți face aceasta, de veți merge înarmați la război înaintea Domnului,
\par 21 De va trece fiecare din voi înarmat peste Iordan înaintea Domnului, până când va pierde El pe vrăjmașii Săi înaintea Sa și până când va fi cuprins pământul înaintea Lui,
\par 22 Atunci, după ce vă veți întoarce, veți fi fără vină înaintea Domnului și înaintea lui Israel și veți avea pământul acesta moștenire înaintea Domnului.
\par 23 Iar de nu veți face așa, veți greși înaintea Domnului și veți suferi pedeapsa care vă va ajunge pentru păcatul vostru.
\par 24 Zidiți-vă cetăți pentru copiii voștri și staule pentru oile voastre și faceți cele ce ați rostit cu buzele voastre".
\par 25 Zis-au fiii lui Gad și fiii lui Ruben către Moise: "Robii tăi vor face cum poruncește domnul nostru.
\par 26 Copiii noștri, femeile noastre, turmele noastre și toate vitele noastre vor rămâne aici în cetățile Galaadului;
\par 27 Iar robii tăi, înarmați cu toții ca oșteni, vor merge înaintea Domnului la război, cum zice domnul nostru".
\par 28 Atunci a dat Moise poruncă pentru ei lui Eleazar preotul, lui Iosua, fiul lui Navi, și căpeteniilor semințiilor fiilor lui Israel,
\par 29 Și le-a zis Moise: "Dacă fiii lui Gad și fiii lui Ruben vor trece cu voi peste Iordan, întrarmându-se cu toții pentru război înaintea Domnului, după ce țara va fi supusă înaintea voastră, să le dați pământul Galaad în stăpânire.
\par 30 Iar dacă ei nu vor merge cu voi înarmați pentru război înaintea Domnului, să trimiteți înaintea voastră averea lor, femeile lor și vitele lor în pământul Canaan și ei să primească moșie împreună cu voi în pământul Canaanului".
\par 31 Iar fiii lui Gad și fiii lui Ruben au răspuns și au zis: "Cum a zis domnul robilor tăi așa vom și face.
\par 32 Vom merge înarmați înaintea Domnului în pământul Canaan, iar partea noastră de moșie să fie de astă parte de Iordan".
\par 33 Atunci Moise le-a dat fiilor lui Gad, fiilor lui Ruben și la jumătate din seminția lui Manase, fiul lui Iosif, țara lui Sihon, regele Amoreilor, și țara lui Og, regele Vasanului, pământul cu orașele lui și împrejurimile și cetățile din toate părțile jării.
\par 34 Și au zidit fiii lui Gad: Dibonul, Atarotul, Aroerul,
\par 35 Atarot-Șofanul, Iazerul, Iogbeha,
\par 36 Bet-Nimra și Bet-Haran, cetăți întărite și staule pentru oi.
\par 37 Fiii lui Ruben au zidit Heșbonul, Eleale, Chiriataimul,
\par 38 Nebo, Baal-Meonul și Sibma, ale căror nume au fost schimbate și au dat alte nume orașelor pe care le-au zidit ei.
\par 39 Iar fiii lui Machir, fiul lui Manase, s-au dus în Galaad și l-au luat și au alungat pe Amoreii care erau acolo.
\par 40 Iar Moise a dat Galaadul lui Machir, fiul lui Manase, și s-a așezat acela acolo.
\par 41 Iair, fiul lui Manase, s-a dus și a luat sălașurile lor și le-a numit sălașurile lui Iair.
\par 42 Iar Nobah s-a dus și a luat Chenatul și cetățile care țineau de el și l-a numit după numele său: Nobah.

\chapter{33}

\par 1 Iată acum popasurile fiilor lui Israel, după ce au ieșit ei din pământul Egiptului cu oștirile lor, sub mâna lui Moise și Aaron.
\par 2 Moise, din porunca Domnului, a scris călătoria lor cu popasurile lor; iar popasurile lor sunt acestea:
\par 3 În luna întâi, în ziua a cincisprezecea a lunii întâi, a doua zi de Paști, fiii lui Israel au purces din Ramses (Goșen) și au ieșit, sub mână înaltă, înaintea ochilor a tot Egiptul.
\par 4 În vremea aceea Egiptenii îngropau pe toți cei ce muriseră dintre ei, pe toți întâi-născuții, pe care-i lovise Domnul, în țara Egiptului, când a făcut Domnul judecată asupra dumnezeilor lor.
\par 5 După ce au pornit fiii lui Israel din Ramses (Goșen), au poposit în Sucot.
\par 6 Pornind apoi din Sucot, au tăbărât la Etam, care este la marginea pustiului.
\par 7 Din Etam au pornit și s-au îndreptat spre Pi-Hahirot, care este în țara Baal-Țefonului, și și-au așezat tabăra înaintea Migdolului.
\par 8 Pornind apoi din Pi-Hahirot, au trecut prin mare în pustie și, mergând cale de trei zile prin pustiul Etam, și-au așezat tabăra la Mara.
\par 9 Plecând de la Mara, au venit la Elim. În Elim insă erau douăsprezece izvoare de apă și șaptezeci de finici și au tăbărât acolo lângă apă.
\par 10 Pornind apoi din Elim, au tăbărât la Marea Roșie.
\par 11 Au pornit apoi de la Marea Roșie și au tăbărât în pustiul Sin.
\par 12 Pornind din pustiul Sin, au poposit la Dofca.
\par 13 Pornind din Dofca, au tăbărât la Aluș.
\par 14 Pornind din Aluș, și-au așezat tabăra la Rafidim. Acolo nu era apă ca să bea poporul.
\par 15 Pornind din Rafidim, au tăbărât în pustiul Sinai.
\par 16 Iar după ce au pornit din pustiul Sinai, au poposit la Chibrot-Hataava.
\par 17 Pornind din Chibrot-Hataava, au tăbărât în Hașerot.
\par 18 Pornind din Hașerot, au poposit la Ritma.
\par 19 Pornind din Ritma, și-au așezat tabăra la Rimon-Pereț.
\par 20 Pornind din Rimon-Pereț, au tăbărât în Libna.
\par 21 Pornind din Libna, au tăbărât la Risa.
\par 22 Pornind din Risa, și-au așezat tabăra la Chehelata.
\par 23 Pornind din Chehelata, au tăbărât pe Muntele Șafer.
\par 24 Pornind de pe Muntele Șafer, au poposit în Harada.
\par 25 Pornind din Harada, au tăbărât la Machelot.
\par 26 Pornind din Machelot, au poposit în Tahat.
\par 27 Pornind din Tahat, s-au așezat cu tabăra în Tarah.
\par 28 Pornind din Tarah, au tăbărât în Mitca.
\par 29 Pornind din Mitca, au tăbărât în Hașmona.
\par 30 Pornind din Hașmona, au poposit la Moserot.
\par 31 Pornind din Moserot, și-au așezat tabăra la Bene-Iaakan.
\par 32 Pornind din Bene-Iaakan, au tăbărât la Hor-Haghidgad.
\par 33 Pornind din Hor-Haghidgad, au poposit în Iotbata.
\par 34 Pornind din Iotbata, au tăbărât la Abrona.
\par 35 Pornind din Abrona, și-au așezat tabăra la Ețion-Gheber.
\par 36 Pornind din Ețion-Gheber, au poposit în pustiul Sin. Plecând din pustiul Sin, au tăbărât în Cadeș.
\par 37 Iar din Cadeș au purces și au poposit la muntele Hor, lângă hotarul țării Edomului.
\par 38 Aici s-a suit Aaron preotul pe muntele Hor, după porunca Domnului, și a murit acolo, în anul al patruzecilea de la ieșirea fiilor lui Israel din pământul Egiptului, în luna a cincea, în ziua întâi a lunii.
\par 39 Aaron era de o sută douăzeci și trei de ani, când a murit pe muntele Hor.
\par 40 Atunci regele canaanean din Arad, care trăia în partea de miazăzi a pământului Canaan, a auzit că vin fiii lui Israel.
\par 41 Aceștia însă, plecând de la muntele Hor, au tăbărât la Țalmona.
\par 42 Pornind din Țalmona, au poposit la Punon.
\par 43 Pornind din Punon, au tăbărât la Obot.
\par 44 Pornind din Obot, au poposit la Iie-Abarim, lângă hotarele lui Moab.
\par 45 Pornind din Iie-Abarim, au tăbărât la Dibon-Gad.
\par 46 Pornind din Dibon-Gad, au poposit ia Almon-Diblataim.
\par 47 Pornind din Almon-Diblataim, au tăbărât în munții Abarim, în fața lui Nebo.
\par 48 Pornind de la munții Abarim, au poposit în șesurile Moabului, la Iordan, în fața Ierihonului,
\par 49 Și și-au așezat ei tabăra la Iordan, de la Bet-Ieșimot până la Abel-Șitim, în șesurile Moabului.
\par 50 Grăit-a Domnul cu Moise în șesurile Moabului, la Iordan, în fața Ierihonului, și a zis:
\par 51 "Vorbește fiilor lui Israel și le spune: Când veți trece peste Iordan, în pământul Canaanului,
\par 52 Să alungați de la voi pe toți locuitorii țării și să stricați toate chipurile cele cioplite ale lor, toți idolii lor cei turnați din argint și toate înălțimile lor să le pustiiți.
\par 53 Să luați în stăpânire pământul și să vă așezați acolo, căci vă dau în stăpânire pământul acesta.
\par 54 Să împărțiți pământul prin sorți la semințiile voastre: celor mai mulți la număr să le dați parte mai mare, iar celor mai puțini la număr să le dați parte mai mică; fiecăruia unde-i va cădea sorțul, acolo să-i fie partea, după semințiile părinților voștri.
\par 55 Iar dacă nu veți alunga de la voi pe locuitorii pământului, atunci cei rămași din ei vor fi spini pentru ochii voștri și bolduri pentru coastele voastre și vă vor strâmtora în țara în care veți trăi.
\par 56 și atunci vă voi face vouă ceea ce aveam de gând să le fac lor".

\chapter{34}

\par 1 A grăit Domnul cu Moise și a zis:
\par 2 "Poruncește fiilor lui Israel și le zi: Iată, veți intra în pământul Canaan. Acesta va fi moștenirea voastră; iar hotarele Canaanului sunt acestea:
\par 3 Partea de miazăzi va începe de la pustiul Sin de lângă Edom și va avea la răsărit, ca hotar, Marea Sărată.
\par 4 Acest hotar se va îndrepta spre miazăzi, către înălțimea Acravimului; va trece prin Sin și se va întinde până la miazăzi de Cadeș-Barne; apoi va merge către Hațar-Adar trecând la Ațmon.
\par 5 De la Ațmon, hotarul se va îndrepta spre Râul Egiptului și se va pogorî până la mare.
\par 6 Iar hotar dinspre apus vă va fi Marea cea Mare. Acesta va fi hotarul vostru dinspre asfințit.
\par 7 Iar spre miazănoapte, hotarul vostru să-l trageți de la Marea cea Mare până la muntele Hor;
\par 8 De la muntele Hor, să-l trageți spre Hamat, și hotarul va atinge Țedadul.
\par 9 De acolo va merge hotarul către Țifron și va atinge Hațar-Enan. Acesta să vă fie hotarul de miazănoapte.
\par 10 Iar hotarul dinspre răsărit să vi-l trageți de la Hațar-Enan către Șefam;
\par 11 De la Șefam hotarul se va pogorî spre Ribla, pe la răsărit de Ain, mergând de-a lungul malului Mării Chineret (Ghenizaret) pe partea de răsărit.
\par 12 De aici hotarul se va pogorî pe Iordan și se va sfârși la Marea Sărată. Acesta va fi pământul vostru, după hotarele lui din toate părțile".
\par 13 Atunci a dat Moise poruncă fiilor lui Israel și a zis: "Iată pământul pe care voi îl veți împărți în bucăți, prin sorți, și care a poruncit Domnul să se dea la nouă seminții și la jumătate din seminția lui Manase.
\par 14 Căci semințiilor fiilor lui Ruben cu familiile lor, a fiilor lui Gad cu familiile lor, și jumătate din seminția lui Manase și-au primit partea lor.
\par 15 Două seminții întregi și o jumătate de seminție și-au primit partea peste Iordan, pe partea răsăriteană, în fața Ierihonului".
\par 16 A grăit Domnul cu Moise și a zis:
\par 17 "Iată numele bărbaților care au să vă împartă pământul: Eleazar preotul și Iosua, fiul lui Navi;
\par 18 Veți mai lua încă și câte o căpetenie de fiecare seminție pentru împărțirea pământului.
\par 19 Numele acestor bărbați sunt: Caleb, fiul lui Iefoni, pentru seminția Iudei;
\par 20 Samuel, fiul lui Amihud, pentru seminția fiilor lui Simeon;
\par 21 Elidad, fiul lui Chislon, pentru seminția lui Veniamin;
\par 22 Căpetenia Buchi, fiul lui Iogli, pentru seminția fiilor lui Dan;
\par 23 Căpetenia Haniel, fiul lui Efod, pentru seminția fiilor lui Manase;
\par 24 Căpetenia Chemuel, fiul lui Șiftan, pentru seminția fiilor lui Efraim;
\par 25 Căpetenia Elițafan, fiul lui Parnac, pentru seminția fiilor lui Zabulon,
\par 26 Căpetenia Paltiel, fiul lui Azan, pentru seminția fiilor lui Isahar;
\par 27 Căpetenia Ahihud, fiul lui Șelomi, pentru seminția fiilor lui Așer;
\par 28 Căpetenia Pedael, fiul lui Amihud, pentru seminția fiilor lui Neftali".
\par 29 Aceștia sunt aceia cărora a poruncit Domnul să împartă pământul Canaan la fiii lui Israel.

\chapter{35}

\par 1 În vremea aceea a grăit Domnul cu Moise în șesurile Moabului, la Iordan, în fața Ierihonului, și a zis:
\par 2 "Poruncește fiilor lui Israel, ca ei, din părțile moștenirii lor, să dea leviților cetăți de locuit; și împrejurul cetăților să le dea leviților locuri.
\par 3 Cetățile vor fi de locuit; iar locurile vor fi pentru vitele lor, iar averea pentru toate nevoile vieții lor.
\par 4 Locurile de pe lângă cetățile pe care trebuie să le dați leviților să se întindă în toate părțile, de la zidurile cetății până la două mii de coți;
\par 5 Să măsurați de la cetate, spre răsărit două mii de coți, spre miazăzi două mii de coți, spre apus două mii de coți și spre miazănoapte două mii de coți, iar în mijloc să fie cetatea: acestea vor fi pământurile lor de pe lângă cetăți.
\par 6 Dintre cetățile pe care le veți da leviților șase cetăți să fie de scăpare, în care veți îngădui să fugă ucigașii. Și pe lângă acestea să le mai dați patruzeci și două de cetăți.
\par 7 Cetățile pe care trebuie să le dați leviților să fie de toate patruzeci și opt de cetăți cu locurile dimprejur.
\par 8 Și când veți da cetățile acestea din moșiile fiilor lui Israel, atunci din moșiile cele mai mari să dați mai mult și din cele mai mici mai puțin; fiecare seminție să dea leviților din cetățile ei potrivit cu partea primită.
\par 9 A grăit Domnul cu Moise și a zis:
\par 10 "Spune fiilor lui Israel și le zi:
\par 11 Când veți trece peste Iordan, în pământul Canaan, să vă alegeți cetățile care au să vă fie cetăți de scăpare, unde să poată fugi ucigașul care a ucis om fără să vrea.
\par 12 Și vor fi cetățile acestea loc de scăpare de cel ce răzbună sângele vărsat, ca să nu fie omorât cel ce a ucis, înainte de a se înfățișa el în fața obștii la judecată.
\par 13 Cetățile pe care trebuie să le dați ca cetăți de scăpare, să fie șase.
\par 14 Trei cetăți să dați de astă parte de Iordan, și trei cetăți să dați în pământul Canaan; acestea trebuie să fie cetățile de scăpare.
\par 15 Aceste cetăți să fie, și pentru fiii lui Israel și pentru străini și pentru cei strămutați la voi, loc de scăpare; acolo să fugă ucigașul fără voie.
\par 16 Dacă cineva a lovit pe altul cu o unealtă de fier și acela a murit, acesta este ucigaș și ucigașul trebuie omorât.
\par 17 Dacă cineva a lovit cu piatră pe altul și acela a murit, acesta este ucigaș și ucigașul trebuie omorât.
\par 18 Sau dacă cu o unealtă de lemn, cu care se poate pricinui moartea, l-a lovit așa încât acela a murit, acesta este ucigaș și ucigașul trebuie dat morții.
\par 19 Răzbunătorul sângelui vărsat poate să ucidă pe făptaș îndată ce-l întâlnește.
\par 20 Dacă cineva izbește pe altul din ură, sau cu gând rău aruncă asupra lui ceva, așa încât acela moare, sau din dușmănie îl lovește cu mâna, așa încât acela moare,
\par 21 Cel ce a lovit trebuie dat morții, că este ucigaș, și răzbunătorul sângelui vărsat poate ucide pe ucigaș îndată ce-l va întâlni.
\par 22 Dacă însă cineva izbește pe altul din nebăgare de seamă, fără dușmănie,
\par 23 Sau aruncă ceva asupra lui fără gând rău, sau vreo piatră a rostogolit asupra lui fără să-l vadă și acela moare, iar el nu i-a fost dușman și nu i-a dorit răul,
\par 24 Atunci obștea trebuie să judece între ucigaș și răzbunătorul sângelui vărsat după aceste rânduieli;
\par 25 Și obștea trebuie să izbăvească pe ucigaș din mâinile răzbunătorului sângelui vărsat, și să-l întoarcă obștea în cetatea lui de scăpare, unde a fugit el, ca să trăiască acolo până la moartea marelui preot, care este miruit cu mir sfințit.
\par 26 Dacă ucigașul va ieși peste hotarele orașului de scăpare, în care a fugit,
\par 27 Și-l va găsi răzbunătorul sângelui vărsat, în afară de hotarele cetății lui de scăpare, și va ucide pe ucigașul acesta răzbunătorul de sânge, acesta nu va fi vinovat de vărsare de sânge,
\par 28 Pentru că acela trebuie să șadă în orașul său de scăpare până la moartea marelui preot; iar după moartea marelui preot trebuie să se întoarcă ucigașul în pământul său de moștenire.
\par 29 Aceasta să vă fie rânduiala legiuită în neamul și în toate locașurile voastre.
\par 30 Dacă cineva va ucide om, ucigașul trebuie ucis după cuvintele martorilor, dar pentru a osândi la moarte, nu este de ajuns un singur martor.
\par 31 Să nu luați răscumpărare pentru sufletul ucigașului care este vinovat morții, ci să-l omorâți.
\par 32 Să nu luați răscumpărare pentru cel ce a fugit în orașul de scăpare, ca să-i îngăduiți să locuiască în pământul său, înainte de moartea marelui preot.
\par 33 Să nu spurcați pământul pe care aveți să trăiți; că sângele spurcă pământul și pământul nu se poate curăți în alt fel de sângele vărsat pe el, decât cu sângele celui ce l-a vărsat.
\par 34 Să nu spurcați pământul pe care trăiți și în mijlocul căruia locuiesc Eu; căci Eu, Domnul, locuiesc între fiii lui Israel".

\chapter{36}

\par 1 Atunci au venit căpeteniile familiilor din seminția fiilor lui Galaad, fiul lui Machir, fiul lui Manase, din seminția fiilor lui Iosif, și au grăit înaintea lui Moise și înaintea lui Eleazar preotul și înaintea căpeteniilor urmașilor fiilor lui Israel și au zis:
\par 2 "Domnul a poruncit stăpânului nostru să dea pământ de moștenire fiilor lui Israel prin sorți, și stăpânului nostru i s-a poruncit de la Domnul să dea partea lui Salfaad, fratele nostru, fiicelor lui.
\par 3 Dacă însă acestea vor fi soții ale fiilor unei alte seminții a fiilor lui Israel, atunci partea lor se va lua din moșia părinților noștri și se va adăuga la moșia acelei seminții, în care ele vor fi soții, și așa se va lua din moșia noastră ce ni s-a cuvenit prin sorți.
\par 4 Și chiar când va fi jubileu la fiii lui Israel, atunci partea lor se va adăuga la moșia acelei seminții, în care ele vor fi soții, și partea lor se va șterge din moșia seminției părinților noștri".
\par 5 Deci a dat Moise poruncă fiilor lui Israel, după cuvântul Domnului, zicând:
\par 6 "Adevărat grăiește seminția fiilor lui Iosif. Iată ce poruncește Domnul pentru fiicele lui Salfaad: Ele pot să fie soții ale acelora care vor plăcea ochilor lor, numai să fie soții în neamul seminției tatălui lor,
\par 7 Pentru ca partea fiilor lui Israel să nu treacă de la o seminție la alta; că fiecare din fiii lui Israel trebuie să fie legat de moșia seminției părinților săi.
\par 8 Și orice fată care stăpânește o parte de moștenite în una din semințiile fiilor lui Israel să fie soția cuiva din neamul seminției tatălui său, ca fiii lui Israel să moștenească fiecare partea părinților săi,
\par 9 Și să nu treacă partea de la o seminție la altă seminție, că fiecare din semințiile fiilor lui Israel trebuie să fie legată de moșia sa".
\par 10 Cum a poruncit Domnul lui Moise, așa au făcut fiicele lui Salfaad.
\par 11 Și fiicele lui Salfaad: Mahla, Tirța, Hogla, Milca și Noa s-au măritat după fiii unchiului lor.
\par 12 În seminția fiilor lui Manase, fiul lui Iosif, au fost ele soții, și a rămas moșia lor neamului tatălui lor.
\par 13 Acestea sunt poruncile și așezămintele pe care le-a dat Domnul fiilor lui Israel, prin Moise, în șesurile Moabului, la Iordan, în fața Ierihonului.


\end{document}