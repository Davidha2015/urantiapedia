\begin{document}

\title{Deuteronomy}

Deu 1:1  Acestea sunt cuvintele pe care le-a grait Moise la tot Israelul peste Iordan, în pustiul Arabah, din fa?a Sufei, între Paran, Tofel, Laban, Ha?erot ?i Di-Zahab,
Deu 1:2  Cale de unsprezece zile de la Horeb, în drumul de la Muntele Seir, catre Cade?-Barnea.
Deu 1:3  În anul al patruzecilea, în luna a unsprezecea, în ziua întâi a lunii acesteia, a grait Moise tuturor fiilor lui Israel toate cît îi poruncise Domnul pentru ei.
Deu 1:4  Dupa ce a batut pe Sihon, regele Amoreilor, care locuia în He?bon, ?i pe Og, regele Vasanului, care locuia în A?tarot ?i în Edrei, dincolo de Iordan, în pamântul Moabului,
Deu 1:5  A început Moise a lamuri legea aceasta ?i a zis:
Deu 1:6  "Domnul Dumnezeul vostru ne-a grait noua pe Horeb ?i a zis: Va ajunge de când locui?i pe muntele acesta!
Deu 1:7  Întoarce?i-va ?i, pornind la drum, duce?i-va la muntele Amoreilor ?i la to?i vecinii lor din Arabah, din munte, din ?efela ?i din Negeb, la malurile marii, în pamântul Canaanului, la Liban, ?i chiar pâna la râul cel mare, la fluviul Eufratului.
Deu 1:8  ?i iata, Eu va dau pamântul acesta; merge?i ?i va lua?i de mo?tenire pamântul pe care Domnul a fagaduit cu juramânt sa-l dea parin?ilor vo?tri, lui Avraam ?i lui Isaac ?i lui Iacov, lor ?i urma?ilor lor.
Deu 1:9  În vremea aceea v-am zis: Nu va mai pot pova?ui singur;
Deu 1:10  Domnul Dumnezeul vostru v-a înmul?it ?i iata acum sunte?i mul?i la numar, ca stelele cerului.
Deu 1:11  Domnul Dumnezeul parin?ilor vo?tri sa va înmul?easca de o mie de ori mai mult decât sunte?i acum ?i sa va binecuvânteze, cum v-a fagaduit El!
Deu 1:12  Cum dar voi purta singur greuta?ile voastre ?i sarcinile voastre ?i neîn?elegerile dintre voi?
Deu 1:13  Alege?i-va din semin?iile voastre barba?i în?elep?i, pricepu?i ?i încerca?i, ?i-i voi pune capetenii peste voi.
Deu 1:14  Atunci mi-a?i raspuns ?i a?i zis: Bun lucru ne porunce?ti sa facem!
Deu 1:15  ?i am luat dintre voi barba?i în?elep?i, pricepu?i ?i încerca?i, ?i i-am pus pova?uitori peste voi: capetenii peste mii, peste sute, peste cincizeci, peste zeci ?i judecatori peste semin?iile voastre.
Deu 1:16  În vremea aceea, am dat porunca judecatorilor vo?tri ?i am zis: Sa asculta?i pe fra?ii vo?tri ?i sa judeca?i drept pricina ce ar avea un om atât cu fratele lui, cît ?i cu cel strain.
Deu 1:17  Sa nu partini?i la judecata, ci sa asculta?i ?i pe cel mare ?i pe cel mic. Sa nu va sfii?i de la fa?a omului, ca judecata este a lui Dumnezeu. Iar pricina care va fi grea pentru voi sa o aduce?i la mine ?i o voi asculta eu.
Deu 1:18  V-am mai dat în vremea aceea porunci pentru toate cele ce trebuie sa face?i.
Deu 1:19  Am plecat apoi de la Horeb, cum ne poruncise Domnul Dumnezeul nostru, ?i am strabatut tot pustiul acesta mare ?i înfrico?ator, pe care l-a?i vazut în drumul spre muntele Amoreilor, ?i am ajuns la Cade?-Barnea.
Deu 1:20  Atunci v-am zis: Iata, a?i ajuns la muntele Amoreilor, pe care Domnul Dumnezeul vostru, îl va da noua.
Deu 1:21  Iata, Israel, Domnul Dumnezeul tau î?i da pamântul acesta: mergi ?i ia-l în stapânire, cum ?i-a zis Domnul Dumnezeul parin?ilor tai; nu te teme, nici nu te înspaimânta!
Deu 1:22  Iar voi a?i venit cu to?ii la mine ?i a?i zis: Sa trimitem înaintea noastra oameni ca sa cerceteze pamântul ?i sa ne aduca ?tire despre drumul pe care sa mergem ?i despre ceta?ile la care sa ne ducem.
Deu 1:23  Cuvântul acesta mi-a placut ?i am luat dintre voi doisprezece oameni, câte unul din fiecare semin?ie.
Deu 1:24  Ace?tia s-au dus ?i s-au suit pe munte, au mers pâna la valea E?col, ?i au cercetat-o.
Deu 1:25  Au luat din roadele pamântului ?i ne-au adus noua; ?i ne-au adus ?i ?tire, spunându-ne: Pamântul pe care Domnul Dumnezeul nostru ni-l da este bun.
Deu 1:26  Voi însa n-a?i vrut sa va duce?i ?i v-a?i împotrivit poruncii Domnului Dumnezeului vostru, a?i cârtit în corturile voastre ?i a?i zis:
Deu 1:27  Domnul din ura catre noi ne-a scos din pamântul Egiptului, ca sa ne dea în mâinile Amoreilor ?i sa ne piarda.
Deu 1:28  Încotro sa ne ducem? Fra?ii no?tri ne-au înfrico?at, spunându-ne: "Poporul acela e mai mare, ma? mult ?i mai puternic decât noi; ceta?ile de acolo sunt mari ?i cu întarituri pâna la cer; ?i am mai vazut acolo ?i pe fiii lui Enac".
Deu 1:29  Atunci v-am zis: Nu va înspaimânta?i ?i nu va teme?i de ei.
Deu 1:30  Domnul Dumnezeul vostru merge înaintea voastra ?i se va lupta El pentru voi, cum a facut cu voi ?i în Egipt, înaintea ochilor vo?tri.
Deu 1:31  ?i cum a facut în pustiul acesta, unde, cum ai vazut tu, Israel, Domnul Dumnezeul tau te-a purtat tot drumul ce l-a?i strabatut pâna ce a?i sosit la locul acesta, cum poarta un om pe fiul sau.
Deu 1:32  Dar voi nici a?a n-a?i crezut pe Domnul Dumnezeul vostru,
Deu 1:33  Care a mers înaintea voastra în calatorie, ca sa va caute loc unde sa poposi?i; ?i mergea noaptea în foc, ca sa va arate calea pe care sa merge?i, iar ziua în nor.
Deu 1:34  ?i auzind Domnul Dumnezeu cuvintele voastre, S-a mâniat ?i S-a jurat, zicând: "Nimeni din oamenii ace?tia, din acest neam rau,
Deu 1:35  Nu va vedea pamântul cel bun, pe care Eu am jurat sa-l dau parin?ilor vo?tri.
Deu 1:36  Numai Caleb, fiul lui Iefone, îl va vedea. Aceluia ?i fiilor lui voi da pamântul pe care 1-a strabatut el, pentru ca acela s-a supus Domnului.
Deu 1:37  Pentru voi s-a mâniat Domnul ?i pe mine ?i a zis: Nici tu nu vei intra acolo!
Deu 1:38  Iosua, fiul lui Navi, care este cu tine, acela va intra acolo; întare?te-l pe el, ca el va duce pe Israel în mo?tenirea sa.
Deu 1:39  Copiii vo?tri, de care voi zicea?i ca vor cadea prada vrajma?ilor, ?i fiii vo?tri, care acum nu cunosc nici binele, nici raul, aceia vor intra acolo; lor îl voi da ?i ei îl vor mo?teni.
Deu 1:40  Iar voi întoarce?i-va ?i va îndrepta?i spre pustie, pe calea cea catre Marea Ro?ie.
Deu 1:41  Voi însa mi-a?i raspuns atunci ?i mi-a?i zis: Am pacatuit înaintea Domnului Dumnezeului nostru! Ne ducem sa ne luptam cum ne-a poruncit Domnul Dumnezeul nostru! ?i v-a?i încins fiecare cu arma sa de lupta ?i v-a?i hotarât nebune?te sa va sui?i pe munte.
Deu 1:42  Iar Domnul mi-a zis: Spune-le: Nu va sui?i ?i nu va lupta?i, ca sa nu va biruiasca vrajma?ii vo?tri, ca Eu nu sunt în mijlocul vostru.
Deu 1:43  ?i eu v-am spus, dar voi n-a?i ascultat, ci v-a?i împotrivit poruncii Domnului ?i, în îndaratnicia voastra, v-a?i suit pe munte.
Deu 1:44  Dar v-a ie?it înainte poporul amoreu care locuia pe muntele acela ?i a tabarât asupra voastra ca albinele ?i v-a zdrobit de la Seir pâna la Horma.
Deu 1:45  Atunci v-a?i întors ?i a?i plâns înaintea Domnului, dar Domnul n-a ascultat plângerea voastra ?i nu v-a luat în seama.
Deu 1:46  ?i a?i locuit în Cade? vreme multa, ca multe au fost zilele cît a?i stat acolo".
Deu 2:1  Apoi, întorcându-ne noi, am pornit prin pustie, spre Marea Ro?ie, cum îmi graise Domnul, ?i am umblat zile multe împrejurul Muntelui Seir.
Deu 2:2  Iar Domnul a zis catre mine:
Deu 2:3  "Ajunge de când umbla?i împrejurul acestui munte! Întoarce?i-va dar spre miazanoapte!
Deu 2:4  Da porunca poporului ?i zi: Voi acum ve?i trece prin hotarele fiilor lui Isav, fra?ii vo?tri, care locuiesc în Seir, ?i ace?tia se vor teme de voi foarte tare.
Deu 2:5  Dar sa nu începe?i razboiul cu ei, caci nu va voi da din pamântul lor nici o palma de loc, pentru ca Muntele Seir l-am dat în stapânirea lui Isav.
Deu 2:6  Mâncare sa va cumpara?i de la ei cu bani ?i sa mânca?i; ?i apa de baut sa va cumpara?i de la ei tot cu bani;
Deu 2:7  Ca Domnul Dumnezeul tau, Israele, te-a binecuvântat în tot lucrul mâinilor tale ?i te-a ocrotit în timpul calatoriei tale prin pustiul acesta mare ?i înfrico?ator. Iata, de patruzeci de ani Domnul Dumnezeul tau este cu tine ?i n-ai dus lipsa de nimic".
Deu 2:8  ?i am trecut pe lânga fiii lui Isav, fra?ii no?tri, care locuiau în Seir, pe calea câmpului, de la Elat ?i E?ion-Gaber, ?i ne-am abatut ?i am mers spre pustiul Moabului.
Deu 2:9  Dar Domnul mi-a zis: "Nu intra în du?manie cu Moab ?i nu începe razboi cu el, ca nu-?i voi da în stapânire nimic din pamântul lui, pentru ca Arul l-am dat în stapânire fiilor lui Lot".
Deu 2:10  Înainte au locuit acolo Emimii, popor mare, mult la numar ?i înalt la statura, ca fiii lui Enac;
Deu 2:11  ?i ace?tia se socoteau printre Refaimi, ca fiii lui Enac; iar Moabi?ii îi numesc Emimi.
Deu 2:12  Pe Seir însa au trait înainte Horeii; dar fiii lui Isav i-au alungat ?i i-au pierdut de la fa?a lor ?i s-au a?ezat în locul lor, cum trebuie sa faca ?i Israel în pamântul sau de mo?tenire, care i-l va da Domnul.
Deu 2:13  Scula?i-va dar ?i trece?i râul Zared. ?i am trecut noi râul Zared.
Deu 2:14  De atunci, de când ne-am dus la Cade?-Barnea ?i pâna ce am trecut râul Zared, au trecut treizeci ?i opt de ani ?i au pierit din tabara noastra to?i cei ce erau atunci buni de razboi, dupa cum li se jurase Domnul;
Deu 2:15  Ca mâna Domnului, pâna au pierit ei, a fost asupra lor, ca sa-i piarda din tabara.
Deu 2:16  Iar daca au pierit to?i cei ce erau atunci buni de razboi din popor,
Deu 2:17  Mi-a grait Domnul ?i a zis:
Deu 2:18  "Acum tu sa treci pe lânga hotarele lui Moab spre Ar,
Deu 2:19  ?i sa te apropii repede de Amoni?i, dar sa nu intri cu ace?tia în du?manie ?i sa nu începi razboi cu ei, caci nu-?i voi da în stapânire nimic din pamântul fiilor lui Amon, pentru ca l-am dat în stapânire fiilor lui Lot".
Deu 2:20  Acesta se socotea a fi pamântul Refaimilor, caci Refaimii locuisera înainte într-însul. Amoni?ii însa îi numeau Zomzomimi.
Deu 2:21  Poporul acesta fusese mare, mult la numar ?i înalt la statura, ca fiii lui Enac; dar Domnul îi pierduse de la fa?a Amoni?ilor ?i-i alungasera ace?tia ?i se a?ezasera în locul lor.
Deu 2:22  Astfel a facut Domnul pentru fiii lui Isav care locuiau în Seir, când a prapadit de la fa?a lor pe Horei. ?i dupa ce ei au fost izgoni?i, s-au a?ezat în locul lor, unde traiesc ?i astazi.
Deu 2:23  ?i pe Hevei, care locuiau prin sate chiar pâna la Gaza, i-au pierdut Caftorimii, care se trageau din Caftorim, ?i s-au a?ezat în locul lor.
Deu 2:24  Scula?i-va ?i va porni?i ?i trece?i râul Arnon, ca iata Eu voi da în mâna ta pe Amoreul Sihon, regele He?bonului ?i rara lui; începe a-l cuprinde ?i du razboi cu el.
Deu 2:25  Din ziua aceasta voi începe Eu a împra?tia înaintea ta frica ?i groaza peste popoare, sub tot cerul; cei ce vor auzi de tine se vor cutremura ?i se vor îngrozi de tine.
Deu 2:26  Din pustiul Chedemot am trimis soli la Sihon, regele He?bonului, cu cuvinte de pace, ca sa spuna:
Deu 2:27  Îngaduie-mi sa trec prin ?ara ta, ca voi merge pe drum ?i nu ma voi abate nici la dreapta, nici la stânga;
Deu 2:28  Hrana sa-mi vinzi pe bani ?i voi mânca, ?i apa de baut sa-mi dai pe bani ?i voi bea,
Deu 2:29  Cum mi-au facut fiii lui Isav, care locuiesc în Seir, ?i Moabi?ii, care locuiesc Arul; numai cu piciorul meu voi merge pâna voi trece Iordanul în pamântul pe care Domnul Dumnezeul nostru ni-l da noua.
Deu 2:30  Dar Sihon, regele He?bonului, n-a voit a ne îngadui sa trecem prin pamântul lui, pentru ca Domnul Dumnezeul tau a îndaratnicit duhul lui ?i inima lui a împietrit-o, ca sa-l dea în mâinile tale, cum se vede acum.
Deu 2:31  Atunci mi-a zis Domnul: "Iata, încep sa-?i dau pe Sihon Amoreul, regele He?bonului, ?i pamântul lui; începe a stapâni pamântul lui".
Deu 2:32  Iar Sihon, regele He?bonului, cu tot poporul sau, ne-a ie?it înainte sa se lupte la Iaha?.
Deu 2:33  Dar Domnul Dumnezeul nostru 1-a dat în mâinile noastre ?i l-am batut pe el ?i pe fiii lui ?i tot poporul lui.
Deu 2:34  În vremea aceea am luat toate ceta?ile lui ?i am nimicit toate ceta?ile lui, barba?i, femei ?i copii, ?i n-am lasat pe nimeni viu.
Deu 2:35  Numai vitele lor ?i cele jefuite din ceta?ile cuprinse de noi ni le-am luat.
Deu 2:36  De la Aroer, care se afla pe malul râului Arnon, ?i de la cetatea cea din vale pâna la muntele Galaad, n-a mai fost cetate în care noi sa nu fi putut patrunde: Domnul Dumnezeu a dat tot în mâinile noastre.
Deu 2:37  Numai de pamântul Amoni?ilor nu te-ai apropiat, nici de locurile ce se întind în apropierea râului Iaboc, nici de ceta?ile ce sunt pe munte, nici de nimic ce nu ne-a poruncit Domnul Dumnezeul nostru".
Deu 3:1  "Ne-am întors apoi de acolo ?i am mers catre Vasan, însa ne-a ie?it înainte cu razboi Og, regele Vasanului, la Edrei, cu tot poporul sau.
Deu 3:2  Dar Domnul mi-a zis: Nu te teme de el, caci îl voi da în mâinile tale pe el ?i tot poporul lui ?i tot pamântul lui, ?i vei face cu el ce-ai facut cu Sihon, regele Amoreilor, care a trait în He?bon.
Deu 3:3  Domnul Dumnezeul nostru a dat în mâinile noastre ?i pe Og, regele Vasanului, cu tot poporul lui, ?i noi l-am batut, încât nimeni de la ei n-a ramas viu.
Deu 3:4  În vremea aceea am luat toate ceta?ile lui, ca n-a fost cetate pe care sa n-o luam de la ei. Am luat ?aizeci de ceta?i, toata latura Argob, ?ara lui Og al Vasanului.
Deu 3:5  Toate ceta?ile acestea erau întarite cu ziduri înalte, cu por?i ?i cu încuietori, afara de ceta?ile neîntarite care erau foarte multe.
Deu 3:6  ?i le-am nimicit, cum facusem ?i cu Sihon, regele He?bonului, pierzând fiecare cetate cu barba?i, femei ?i copii.
Deu 3:7  Iar toate vitele ?i cele jefuite prin ceta?i ni le-am luat ca prada.
Deu 3:8  Am luat în vremea aceea din mâinile celor doi regi amorei pamântul acesta care este dincoace de Iordan, de la râul Arnon pâna la muntele Hermon.
Deu 3:9  (Sidonienii numesc Hermonul, Sirion, iar Amoreii îl numesc Senir).
Deu 3:10  Am luat adica toate ceta?ile din ?es, tot Galaadul ?i tot Vasanul, pâna la Salca ?i Edrei, ceta?ile din ?ara lui Og al Vasanului.
Deu 3:11  Caci numai Og, regele Vasanului, mai ramasese din Refaimi. Iata patul lui, pat de fier, ?i astazi este în Rabat-Amon: lung de noua co?i ?i lat de patru co?i, co?i barbate?ti.
Deu 3:12  Pamântul acesta l-am luat atunci începând de la Aroer, care este lânga râul Arnon; jumatate din muntele Galaadului cu ceta?ile lui l-am dat semin?iilor lui Ruben ?i Gad;
Deu 3:13  Iar rama?i?a cealalta din Galaad ?i tot Vasanul, ?ara lui Og, le-am dat la jumatate din semin?ia lui Manase; tot ?inutul Argob, eu tot Vasanul se nume?te ?ara Refaimilor.
Deu 3:14  Iair, fiul lui Manase, a luat tot ?inutul Argob, pâna la hotarele Ghe?uri?ilor ?i Maacati?ilor, ?i a numit Vasanul, dupa numele sau, sala?urile lui Iair, cum se cheama ?i astazi.
Deu 3:15  Lui Machir i-am dat Galaadul;
Deu 3:16  Iar semin?iilor lui Ruben ?i Gad le-am dat ?ara de la Galaad pâna la râul Arnon, pamântul dintre râu ?i hotar, pâna la râul Iaboc, pâna la hotarul fiilor lui Amon.
Deu 3:17  Precum ?i Arabah ?i Iordanul, care este hotar de la Chineret pâna la marea Arabah, Marea Sarata, la poalele Muntelui Fazga, spre rasarit.
Deu 3:18  V-am mai dat în vremea aceea porunca ?i am zis: Domnul Dumnezeul vostru v-a dat pamântul acesta în stapânire; to?i cei buni de lupta, înarmându-va, merge?i înaintea fiilor lui Israel, fra?ii vo?tri;
Deu 3:19  Numai femeile voastre, copiii vo?tri ?i vitele voastre, ca ?tiu ca ave?i vite multe, sa ramâna în ceta?ile voastre pe care vi le-am dat eu,
Deu 3:20  Pâna când Domnul Dumnezeu va da lini?te fra?ilor vo?tri, ca ?i voua, ?i pâna când î?i vor primi ?i ei în stapânire pamântul pe care Domnul Dumnezeul vostru li-l va da peste Iordan; atunci va ve?i întoarce fiecare la mo?ia sa pe care v-am dat-o eu.
Deu 3:21  Iar lui Iosua i-am poruncit ?i am zis: Ochii tai au vazut tot ceea ce a facut Domnul Dumnezeul vostru cu ace?ti doi regi; tot a?a va face Domnul ?i cu toate ?arile ce le vei strabate tu.
Deu 3:22  Nu te teme de ele, ca Domnul Dumnezeul vostru Însu?i se va lupta pentru voi.
Deu 3:23  În vremea aceea m-am rugat Domnului ?i am zis:
Deu 3:24  Stapâne Doamne, ai început sa ara)i robului Tau slava Ta, puterea Ta, mâna Ta cea tare ?i bra?ul cel înalt, ca cine este Dumnezeu în cer sau pe pamânt, Care sa faca astfel de lucruri cum sunt ale Tale ?i cu o asemenea putere ca a Ta.
Deu 3:25  Îngaduie-mi sa trec ?i sa vad pamântul cel bun care este peste Iordan ?i acel munte frumos, Libanul.
Deu 3:26  Dar Domnul s-a rnâniat pe mine pentru voi ?i nu m-a ascultat, ei mi-a zis Domnul: Ajunge! De acum sa nu-Mi mai graie?ti de aceasta!
Deu 3:27  Suie-te în vârful lui Fazga ?i prive?te cu ochii tai spre apus ?i spre miazanoapte ?i spre miazazi ?i spre rasarit ?i vezi cu ochii tai, caci nu vei trece peste acest Iordan.
Deu 3:28  Da pova?a lui Iosua; întare?te-l ?i-l îmbarbateaza, caci el va merge înaintea acestui popor ?i el va împar?i în buca?i tot pamântul, pe care-l vezi acum.
Deu 3:29  Atunci ne-am oprit noi în vale, în fa?a Bet-Peorului".
Deu 4:1  "Asculta dar, Israele: hotarârile ?i legile care va înva? eu astazi sa le pazi?i, ca sa fi?i vii, sa va înmul?i?i ?i ca sa va duce?i sa mo?teni?i acel pamânt pe care Domnul Dumnezeul parin?ilor vo?tri vi-l da în stapânire
Deu 4:2  Sa nu adauga?i nimic la cele ce va poruncesc eu, nici sa lasa?i ceva din ele; pazi?i poruncile Domnului Dumnezeului vostru, pe care vi le spun eu astazi.
Deu 4:3  Ochii vo?tri au vazut toate câte a facut Domnul Dumnezeul vostru cu Baal-Peor; pe tot omul care a urmat lui Baal-Peor 1-a pierdut Domnul Dumnezeul tau din mijlocul tau;
Deu 4:4  Iar voi, cei ce v-a?i lipit de Domnul Dumnezeul vostru sunte?i vii pâna în ziua de astazi.
Deu 4:5  Iata, v-am înva?at porunci ?i legi, cum îmi poruncise Domnul Dumnezeul meu, ca sa face?i a?a în ?ara aceea în care intra?i ca s-o stapâni?i.
Deu 4:6  Sa le pazi?i a?adar ?i sa le împlini?i, caci în aceasta sta în?elepciunea voastra ?i cumin?enia voastra înaintea ochilor popoarelor care, auzind de toate legiuirile acestea, vor zice: Numai acest popor mare este popor în?elept ?i priceput.
Deu 4:7  Caci este oare vreun popor mare, de care dumnezeii lui sa fie a?a de aproape, cît de aproape este de noi Domnul Dumnezeul nostru, oricând Îl chemam?
Deu 4:8  Sau este vreun popor mare, care sa aiba astfel de a?ezaminte ?i legi drepte, cum este toata legea aceasta pe care v-o înfa?i?ez eu astazi?
Deu 4:9  Decât numai paze?te-te ?i î?i fere?te cu îngrijire sufletul tau, ca sa nu ui?i acele lucruri pe care le-au vazut ochii tai ?i sa nu-?i iasa ele de la inima în toate zilele vie?ii tale; sa le spui fiilor ?i fiilor feciorilor tai.
Deu 4:10  Sa le spui de ziua aceea în care ai stat tu înaintea Domnului Dumnezeului tau, la Horeb, de ziua adunarii când a zis Domnul catre mine: Aduna la Mine poporul ?i Eu îi voi vesti cuvintele Mele, din care se vor înva?a ei a se teme de Mine în toate zilele vie?ii lor de pe pamânt ?i vor înva?a pe fiii lor.
Deu 4:11  Atunci v-a?i apropiat ?i a?i stat sub munte, muntele ardea cu foc pâna la cer ?i era negura, nor ?i întuneric.
Deu 4:12  Iar Domnul v-a grait de pe munte din mijlocul focului; ?i glasul cuvintelor Lui l-a?i auzit, iar fa?a Lui n-a?i vazut-o, ci numai glasul I l-a?i auzit.
Deu 4:13  Atunci v-a descoperit El legamântul Sau, cele zece porunci, pe care v-a poruncit sa le împlini?i, ?i le-a scris pe doua lespezi de piatra.
Deu 4:14  În vremea aceea mi-a poruncit Domnul sa va înva? poruncile ?i legile Lui, ca sa le împlini?i în ?ara aceea în care intra?i ca s-o stapâni?i.
Deu 4:15  ?ine?i dar bine minte ca în ziua aceea, când Domnul v-a grait din mijlocul focului, de pe muntele Horeb, n-a?i vazut nici un chip.
Deu 4:16  Sa nu gre?i?i dar ?i sa nu va face?i chipuri cioplite, sau închipuiri ale vreunui idol, care sa înfa?i?eze barbat sau femeie,
Deu 4:17  Sau închipuirea vreunui dobitoc de pe pamânt, sau închipuirea vreunei pasari ce zboara sub cer,
Deu 4:18  Sau închipuirea vreunei jivine, ce se târa?te pe pamânt, sau închipuirea vreunui pe?te din apa, de sub pamânt;
Deu 4:19  Sau, privind la cer ?i vazând soarele, luna, stelele ?i toata o?tirea cerului, sa nu te la?i amagit ca sa te închini lor, nici sa le sluje?ti, pentru ca Domnul Dumnezeul tau le-a lasat pentru toate popoarele de sub cer.
Deu 4:20  Iar pe voi v-a luat Domnul Dumnezeu ?i v-a scos din cuptorul cel de fier, din ?ara Egiptului, ca sa-I fi?i Lui popor de mo?tenire, cum sunte?i acum.
Deu 4:21  Domnul Dumnezeu S-a mâniat însa pe mine pentru voi ?i S-a jurat ca eu nu voi trece Iordanul acesta ?i nu voi intra în acel pamânt bun pe care Domnul Dumnezeul tau ?i-l da ?ie de mo?tenire.
Deu 4:22  Eu voi muri în pamântul acesta, fara sa trec Iordanul, iar voi ve?i trece ?i ve?i lua în stapânire acel pamânt bun.
Deu 4:23  Lua?i seama sa nu uita?i legamântul Domnului Dumnezeului vostru pe care 1-a încheiat cu voi ?i sa nu va face?i idoli care ar închipui ceva, precum ?i-a poruncit Domnul Dumnezeul tau.
Deu 4:24  Caci Domnul Dumnezeul tau este foc mistuitor, Dumnezeu zelos.
Deu 4:25  Iar de ?i se vor na?te fii ?i fiilor tai fii ?i, traind mult pe pamânt, ve?i cadea în pacat ?i va ve?i face chipuri cioplite, care sa închipuiasca ceva, de ve?i face ceva ce este rau înaintea ochilor Domnului Dumnezeului vostru ca sa-L mânia?i,
Deu 4:26  Va marturisesc astazi pe cer ?i pe pamânt ca ve?i pierde curând pamântul, pentru a carui mo?tenire trece?i acum Iordanul; nu ve?i trai multa vreme pe el, ci ve?i pieri.
Deu 4:27  Domnul va va împra?tia prin toate popoarele ?i ve?i ramâne pu?ini la numar printre toate popoarele la care va va duce Domnul.
Deu 4:28  Ve?i sluji acolo altor dumnezei, facu?i de mâini omene?ti din lemn ?i piatra, care nu vad ?i nu aud, care nu manânca ?i nu au miros.
Deu 4:29  Dar când vei cauta acolo pe Domnul Dumnezeul tau, Îl vei gasi, de-L vei cauta cu toata inima ta ?i cu tot sufletul tau.
Deu 4:30  Când vei fi la necaz ?i când te vor ajunge toate acestea în curgerea vremii, te vei întoarce la Domnul Dumnezeul tau ?i vei asculta glasul Lui.
Deu 4:31  Domnul Dumnezeul tau este Dumnezeu bun ?i îndurat; nu te va lasa, nu te va pierde ?i nici nu va uita legamântul încheiat cu parin?ii tai, pe care l-a întarit cu juramânt înaintea lor.
Deu 4:32  Ca cerceteaza timpurile trecute, care au fost înainte de tine, din ziua când a facut Dumnezeu pe om pe pamânt; cerceteaza de la o margine a cerului pâna la cealalta margine a lui ?i vezi daca s-a mai savâr?it vreo fapta mare ca aceasta ?i daca s-a mai auzit ceva la fel!
Deu 4:33  A mai auzit, oare, vreun popor glasul Dumnezeului celui viu, graind din mijlocul focului, cum ai auzit tu ?i sa ramâna viu?
Deu 4:34  Sau a mai încercat, oare, vreun dumnezeu sa se duca sa-?i ia popor din mijlocul altui popor prin plagi, prin semne, prin vedenii ?i prin razboi, cu mâna tare ?i cu bra? înalt ?i prin minuni mari, cum a facut pentru voi Domnul Dumnezeul vostru în Egipt, înaintea ochilor vo?tri?
Deu 4:35  ?ie, Israele, ?i s-a dat sa vezi aceasta, ca sa ?tii ca numai Domnul Dumnezeul tau este Dumnezeu ?i nu mai este altul afara de El.
Deu 4:36  Din cer te-a învrednicit sa auzi glasul Lui, ca sa te înve?e, ?i pe pamânt ?i-a aratat focul Lui cel mare ?i ai auzit cuvintele Lui din mijlocul focului.
Deu 4:37  Pentru ca El a iubit pe parin?ii tai ?i v-a ales pe voi, urma?ii lor, de aceea te-a ?i scos cu puterea Lui cea mare din Egipt.
Deu 4:38  Ca sa alunge de la fa?a ta popoarele cele mai mari ?i mai puternice decât tine ?i ca sa te duca sa-?i dea pamântul lor mo?tenire, a?a cum ai astazi.
Deu 4:39  Cunoa?te dar astazi ?i ?ine minte ca Domnul Dumnezeul tau este Dumnezeu sus în cer ?i jos pe pamânt ?i nu mai este altul afara de El.
Deu 4:40  Sa paze?ti legile Lui ?i poruncile Lui, pe care ?i le spun eu astazi, ca sa-?i fie bine ?ie ?i fiilor tai de dupa tine ?i ca sa traie?ti multa vreme în pamântul acela pe care Domnul Dumnezeul tau ?i-l da pentru totdeauna".
Deu 4:41  Atunci a ales Moise trei ceta?i dincoace de Iordan, spre rasaritul Soarelui, ca sa fuga acolo uciga?ul
Deu 4:42  Care va ucide pe aproapele sau fara de voie ?i fara sa-i fi fost vrajma? nici cu o zi, nici cu doua înainte, ?i ca, scapând în una din aceste ceta?i, sa ramâna cu via?a.
Deu 4:43  Aceste ceta?i sunt: Be?er în pustie, în câmpia din semin?ia lui Ruben, Ramot în Galaad, în semin?ia lui Gad, ?i Golan în Vasan, în semin?ia lui Manase.
Deu 4:44  Iata legea pe care a înfa?i?at-o Moise fiilor lui Israel.
Deu 4:45  ?i iata poruncile, legile ?i îndreptarile pe care le-a rostit Moise fiilor lui Israel în pustie, dupa ie?irea lor din Egipt,
Deu 4:46  Dincolo de Iordan, în valea din fa?a Bet-Peorului, în pamântul lui Sihon, regele Amoreilor, care a trait în He?bon, pe care l-a batut Moise cu fiii lui Israel, dupa ie?irea lor din Egipt,
Deu 4:47  ?i au mo?tenit pamântul lui ?i pamântul lui Og, regele Vasanului, ?ara celor doi regi amorei de peste Iordan, spre rasaritul soarelui,
Deu 4:48  Care se întinde pe malul râului Arnon de la Aroer pâna la muntele Sihon sau Hermon,
Deu 4:49  ?i tot ?esul Arabah de dincoace de Iordan, catre rasaritul soarelui, pâna la mare, ?esul Arabah de la poalele muntelui Fazga.
Deu 5:1  În vremea aceea a chemat Moise tot Israelul ?i le-a zis: "Asculta, Israele, poruncile ?i legile pe care le voi rosti eu astazi în auzul urechilor voastre: înva?a?i-le ?i sili?i-va sa le plini?i.
Deu 5:2  Domnul Dumnezeul vostru a încheiat cu voi legamânt în Horeb.
Deu 5:3  Legamântul acesta nu 1-a încheiat Domnul cu parin?ii no?tri, ci cu noi, cei ce suntem astazi cu to?ii vii aici.
Deu 5:4  Fa?a catre fa?a a grait Domnul cu voi din mijlocul focului de pe munte;
Deu 5:5  Iar eu am stat în vremea aceea între Domnul ?i între voi, ca sa va spun Cuvântul Domnului, caci voi v-a?i temut de foc ?i nu v-a?i suit în munte, -- ?i a zis Domnul:
Deu 5:6  "Eu sunt Domnul Dumnezeul tau, Care te-am scos din pamântul Egiptului, din casa robiei.
Deu 5:7  Sa nu ai al?i dumnezei afara de Mine.
Deu 5:8  Sa nu-?i faci chip cioplit, nici vreo înfa?i?are a celor ce sunt sus în cer, sau jos pe pamânt, sau în apa ?i sub pamânt.
Deu 5:9  Sa, nu te închini lor, nici sa le sluje?ti, caci Eu Domnul Dumnezeul tau sunt Dumnezeu zelos, Care pedepse?te vina parin?ilor în copii pâna la al treilea ?i al patrulea neam pentru cei ce Ma urasc.
Deu 5:10  ?i Ma milostivesc pâna la a! miilea neam catre cei ce Ma iubesc ?i pazesc poruncile Mele.
Deu 5:11  Sa nu iei numele Domnului Dumnezeului tau în de?ert, ca nu va lasa Domnul Dumnezeul tau nepedepsit pe cel ce ia numele Lui în de?ert.
Deu 5:12  Paze?te ziua odihnei, ca sa o ?ii cu sfin?enie, cum ?i-a poruncit Domnul Dumnezeul tau.
Deu 5:13  ?ase zile lucreaza ?i-?i fa toate treburile tale;
Deu 5:14  Ziua a ?aptea este ziua de odihna a Domnului Dumnezeului tau. Sa nu faci în ziua aceea nici un lucru: nici tu, nici fiul tau, nici fiica ta, nici robul tau, nici roaba ta, nici boul tau, nici asinul tau, sau alt dobitoc al tau, nici strainul tau care se afla la tine, ca sa se odihneasca robul tau ?i roaba ta cum te odihne?ti ?i tu.
Deu 5:15  Adu-?i aminte ca ai fost rob în pamântul Egiptului ?i Domnul Dumnezeul tau te-a scos de acolo cu mina tare ?i cu bra? înalt ?i de aceea îi-a poruncit Domnul Dumnezeul tau sa paze?ti ziua odihnei ?i sa o ?ii cu sfin?enie.
Deu 5:16  Cinste?te pe tatal tau ?i pe mama ?a, cum ?i-a poruncit Domnul Dumnezeul tau, ca sa traie?ti ani mul?i ?i sa-?i fie bine în pamântul acela, pe care Domnul Dumnezeul tau ?i-l da ?ie.
Deu 5:17  Sa nu ucizi!
Deu 5:18  Sa nu fii desfrânat!
Deu 5:19  Sa nu furi!
Deu 5:20  Sa nu dai marturii mincinoase asupra aproapelui tau!
Deu 5:21  Sa nu pofte?ti femeia aproapelui tau ?i sa nu dore?ti casa aproapelui tau, nici ?arina lui, nici robul lui, nici roaba lui, nici boul lui, nici asinul lui, nici orice dobitoc al lui, nici nimic din cele ce sunt ale aproapelui tau!
Deu 5:22  Cuvintele acestea le-a grait Domnul catre toata adunarea voastra, pe munte, din mijlocul focului, al norului, al întunericului ?i al furtunii, cu glas de tunet ?i altceva n-a mai grait ?i le-a scris pe doua lespezi de piatra ?i mi le-a dat mie.
Deu 5:23  ?i când a?i auzit glasul din mijlocul întunericului ?i muntele ardea, v-a?i apropiat de mine toate capeteniile semin?iilor voastre cu batrânii vo?tri
Deu 5:24  ?i a?i zis: "Iata, Domnul Dumnezeul nostru ne-a aratat slava Sa ?i mare?ia Sa ?i glasul Lui l-am auzit din mijlocul focului. Astazi. am vazut ca Dumnezeu graie?te cu omul ?i acesta ramâne viu.
Deu 5:25  ?i de ce sa murim noi? Ca focul acesta ne va mistui ?i de vom mai auzi glasul Domnului Dumnezeului nostru vom muri.
Deu 5:26  Caci este oare vreun om care sa auda glasul Dumnezeului celui viu graind din mijlocul focului, cum am auzit noi, ?i sa ramâna viu?
Deu 5:27  Apropie-te dar tu ?i asculta toate câte-?i va spune Domnul Dumnezeul nostru ?i apoi ne vei spune tu noua toate câte-?i va grai Domnul Dumnezeul nostru, ?i noi vom asculta ?i vom face.
Deu 5:28  Iar Domnul a auzit cuvintele voastre, cum graia?i cu mine, ?i mi-a zis Domnul: Am auzit cuvintele poporului acestuia, pe care le-au grait catre tine, ?i tot ce-au grait este bine.
Deu 5:29  O, de ar fi inima lor a?a, ca sa se teama de Mine ?i sa pazeasca toate poruncile Mele în toata vremea, ca sa le fie bine ?i lor ?i fiilor lor în veci!
Deu 5:30  Du-te ?i le spune: întoarce?i-va în corturile voastre!
Deu 5:31  Iar tu ramâi aici cu Mine; ?i-?i voi spune toate poruncile, hotarârile ?i legile ce trebuie sa-i înve?i ca sa le pazeasca în pamântul acela, pe care li-l dau Eu în stapânire.
Deu 5:32  ?i sa le zici: Vede?i, sa va purta?i a?a cum v-a poruncit Domnul Dumnezeul vostru ?i sa nu va abate?i nici la dreapta, nici la stânga!
Deu 5:33  Sa umbla?i pe calea aceea pe care v-a poruncit Domnul Dumnezeul vostru, ca sa fi?i vii ?i sa va fie bine ?i sa trai?i vreme multa în pamântul acela pe care îl ve?i lua în stapânire".
Deu 6:1  "Iata poruncile, hotarârile ?i legile pe care mi-a poruncit Domnul Dumnezeul vostru sa va înva?, ca sa le pazi?i în pamântul acela în care merge?i, ca sa-l lua?i în stapânire:
Deu 6:2  Sa te temi de Domnul Dumnezeul tau ?i toate hotarârile Lui ?i poruncile Lui, pe care ti le spun eu astazi, sa le paze?ti tu ?i fiii tai ?i fiii fiilor tai, în toate zilele vie?ii tale, ca sa se înmul?easca zilele tale.
Deu 6:3  Asculta dar, Israele, ?i sile?te-te sa împline?ti acestea, ca sa-?i fie bine ?i sa va înmul?i?i foarte, precum ?i-a grait Domnul Dumnezeul parin?ilor tai ca-li va da pamântul unde curge lapte ?i miere. Acestea sunt hotarârile ?i legile pe care le-a dat Domnul Dumnezeu fiilor lui Israel în pustie, dupa ie?irea lor din pamântul Egiptului.
Deu 6:4  Asculta, Israele, Domnul Dumnezeul nostru este singurul Domn.
Deu 6:5  Sa iube?ti pe Domnul Dumnezeul tau, din toata inima ta, din tot sufletul tau ?i din toata puterea ta.
Deu 6:6  Cuvintele acestea, pe care ?i le spun eu astazi, sa le ai în inima ta ?i în sufletul tau;
Deu 6:7  Sa le sade?ti în fiii tai ?i sa vorbe?ti de ele când ?ezi în casa ta, când mergi pe cale, când te culci ?i când te scoli.
Deu 6:8  Sa le legi ca semn la mâna ?i sa le ai ca pe o tabli?a pe fruntea ta.
Deu 6:9  Sa le scrii pe u?orii casei tale ?i pe por?ile tale.
Deu 6:10  Iar când te va duce Domnul Dumnezeul tau în pamântul acela pentru care s-a jurat parin?ilor tai: lui Avraam, lui Isaac ?i lui Iacov, ca sa ?i-l dea cu ceta?i mari ?i frumoase, pe care nu le-ai zidit tu,
Deu 6:11  Cu case pline de toate bunata?ile, pe care nu le-ai umplut tu, cu fântâni sapate în piatra, pe care nu le-ai sapat tu, cu vii ?i cu maslini, pe care nu le-ai sadit tu, ?i vei mânca ?i te vei satura,
Deu 6:12  Atunci, paze?te-te, sa nu se ademeneasca inima ta, ca sa ui?i pe Domnul Care te-a scos din pamântul Egiptului ?i din casa robiei.
Deu 6:13  Sa te temi de Domnul Dumnezeul tau ?i numai Lui sa-I sluje?ti, de El sa te lipe?ti ?i pe numele Lui sa te juri.
Deu 6:14  Sa nu merge?i dupa al?i dumnezei, dupa dumnezeii popoarelor, care se vor afla împrejurul vostru;
Deu 6:15  Ca sa nu se aprinda mânia Domnului Dumnezeului tau asupra ta ?i sa nu te piarda de pe fa?a pamântului, ca Domnul Dumnezeul tau, Care se afla în mijlocul tau, este Dumnezeu zelos.
Deu 6:16  Sa nu ispiti?i pe Domnul Dumnezeul vostru, cum L-a?i ispitit la Masa.
Deu 6:17  Sa ?ine?i bine poruncile Domnului Dumnezeului vostru, legile Lui ?i hotarârile Lui, pe care vi le-a dat El.
Deu 6:18  Sa faci ceea ce este drept ?i bine înaintea ochilor Domnului Dumnezeului tau, ca sa-?i fie bine ?i ca sa intri ?i sa iei în stapânire pamântul pe care Domnul cu juramânt 1-a fagaduit parin?ilor tai
Deu 6:19  ?i ca sa alunge El pe to?i vrajma?ii tai de la fa?a ta, cum a zis Domnul.
Deu 6:20  De te va întreba în viitor fiul tau ?i va zice: Ce înseamna aceste porunci, hotarâri ?i legi pe care vi le-a dat Domnul Dumnezeul vostru? Sa-i spui fiului tau:
Deu 6:21  Am fost robi la Faraon în Egipt ?i Domnul Dumnezeu ne-a scos din Egipt cu mâna tare ?i cu bra? înalt.
Deu 6:22  ?i a aratat Domnul Dumnezeu semne ?i minuni mari ?i pedepse a adus înaintea ochilor no?tri, asupra Egiptului, asupra lui Faraon, asupra a toata casa lui ?i asupra o?tirii lui;
Deu 6:23  Iar pe noi ne-a scos de acolo Domnul Dumnezeul nostru ca sa ne duca ?i sa ne dea pamântul pentru care s-a jurat parin?ilor no?tri ca ni-l va da.
Deu 6:24  Atunci ne-a poruncit Domnul sa împlinim toate hotarârile acestea ?i sa ne temem de Domnul Dumnezeul nostru, ca sa ne fie bine în toate zilele, ca ?i acum.
Deu 6:25  ?i va face mila cu noi de ne vom sili sa împlinim toate aceste porunci ale legii înaintea fe?ei Domnului Dumnezeului nostru, cum ne-a poruncit".
Deu 7:1  "Când Domnul Dumnezeul tau te va duce în pamântul la care mergi ca sa-l mo?tene?ti ?i va izgoni de la fa?a ta neamurile cele mari ?i multe ?i anume: pe Hetei, pe Gherghesei, pe Amorei, pe Canaanei, pe Ferezei, pe Hevei ?i pe Iebusei - ?apte neamuri, care sunt mai mari ?i mai puternice decât tine -
Deu 7:2  ?i le va da Domnul Dumnezeul tau în mâinile tale ?i le vei bate, atunci sa le nimice?ti, sa nu faci cu ele legamânt ?i sa nu le cru?i.
Deu 7:3  Sa nu te încuscre?ti cu ele: pe fiica ta sa nu o dai dupa fiul lui ?i pe fiica lui sa nu o iei pentru fiul tau,
Deu 7:4  Ca vor abate pe fiii tai de la Mine ca sa slujeasca altor dumnezei, ?i se va aprinde asupra voastra mânia Domnului ?i curând te va pierde.
Deu 7:5  Ci sa face?i cu ele a?a: jertfelnicile lor sa le strica?i, stâlpii lor sa-i darâma?i, dumbravile lor sa le taia?i ?i idolii dumnezeilor lor sa-i arde?i cu foc.
Deu 7:6  Ca e?ti poporul sfânt al Domnului Dumnezeului tau ?i te-a ales Domnul Dumnezeul tau ca sa-I fii poporul Lui ales din toate popoarele de pe pamânt.
Deu 7:7  ?i Domnul v-a primit, nu pentru ca sunte?i mai mul?i la numar decât toate popoarele - caci sunte?i mai pu?ini la numar decât toate popoarele, -
Deu 7:8  Ci pentru ca va iube?te Domnul; ?i ca sa Î?i ?ina juramântul pe care 1-a facut parin?ilor vo?tri, v-a scos Domnul cu mâna tare ?i cu bra? înalt ?i v-a scapat din casa robiei ?i din mâna lui Faraon, regele Egiptului.
Deu 7:9  Sa ?tii dar ca Domnul Dumnezeul tau este adevaratul Dumnezeu, Dumnezeu credincios, Care paze?te legamântul Sau ?i mila Sa, pâna la al miilea neam, catre cei ce-L iubesc ?i pazesc poruncile Lui;
Deu 7:10  ?i rasplate?te la fel celor ce-L urasc, pierzându-i, ?i nu întârzie sa rasplateasca, eu aceea?i masura, celor ce-L urasc.
Deu 7:11  Paze?te dar poruncile, hotarârile ?i legile pe care-?i poruncesc astazi sa le împline?ti.
Deu 7:12  De vei asculta legile acestea, de le vei pazi ?i le vei împlini, atunci ?i Domnul Dumnezeul tau va ?ine legamântul ?i mila Sa fa?a de tine, cum S-a jurat El parin?ilor tai;
Deu 7:13  Te va iubi, te va binecuvânta, te va înmul?i ?i va binecuvânta rodul pântecelui tau, rodul pamântului tau, pâinea ta, vinul tau, untdelemnul tau, pe cele nascute ale vitelor tale mari ?i ale oilor turmei tale în pamântul acela, pentru care S-a jurat El parin?ilor tai sa ?i-l dea ?ie.
Deu 7:14  ?i vei fi binecuvântat mai mult decât toate popoarele ?i nu se va afla sterp sau stearpa nici între ai tai, nici între dobitoacele tale.
Deu 7:15  Va departa de la tine Domnul Dumnezeul tau toata neputin?a ?i nici una din bolile cele rele ale Egiptenilor, pe care le-ai vazut ?i le ?tii, nu va aduce asupra ta, ci le va trimite asupra celor ce te urasc pe tine
Deu 7:16  Mânca-vei toata agonisita popoarelor pe care Domnul Dumnezeul tau ?i le va da ?ie; sa nu le cru?e ochiul tau ?i sa nu sluje?ti dumnezeilor lor, ca aceasta este cursa pentru tine.
Deu 7:17  Nu cumva sa zici în inima ta: Popoarele acestea sunt mai mari la numar decât mine, cum le voi putea izgoni?
Deu 7:18  Sa nu te temi de ele, ci adu-?i aminte ce a facut Domnul Dumnezeul tau cu Faraon ?i cu tot Egiptul,
Deu 7:19  ?i de acele încercari mari, pe care le-au vazut ochii tai, de acele semne ?i minuni mari, de mâna cea tare ?i de bra?ul cel înalt cu care te-a scos Domnul Dumnezeul tau. Tot a?a va face Domnul Dumnezeul tau cu toate popoarele de care te temi.
Deu 7:20  Înca ?i viespi va trimite Domnul Dumnezeul tau asupra lor pâna ce vor pieri cei ce au ramas ?i s-au ascuns de la fa?a ta.
Deu 7:21  Nu te înspaimânta de ei, ca Domnul Dumnezeul tau, Cel din mijlocul tau, este Dumnezeu mare ?i minunat.
Deu 7:22  Domnul Dumnezeul tau va izgoni dinaintea ta popoarele acestea încetul cu încetul; nu po?i sa le pierzi repede, ca sa nu se pustiiasca pamântul ?i sa nu se înmul?easca împotriva ta fiarele câmpului;
Deu 7:23  Ci ?i le va da Domnul Dumnezeul tau ?i le va pune în mare tulburare, încât vor pieri.
Deu 7:24  Va da pe regii lor în mâinile tale ?i tu vei pierde numele lor de sub cer: nimeni nu-?i va putea sta înainte pâna îi vei stârpi.
Deu 7:25  Idolii dumnezeilor lor sa-i arde?i cu foc; sa nu dore?ti a lua argintul sau aurul de pe ei, ca sa nu-?i fie aceasta cursa, ca urâciune sunt ace?tia înaintea Domnului Dumnezeului tau,
Deu 7:26  ?i urâciunea idoleasca sa n-o duci în casa ta, ca sa nu cazi sub blestem, ca ea. Fere?te-te de aceasta ?i sa-?i fie scârba de ea, ca este blestemata".
Deu 8:1  "Sili?i-va sa împlini?i toate poruncile acestea pe care vi le dau eu astazi, ca sa fi?i vii ?i sa va înmul?i?i ?i sa va duce?i sa lua?i în stapânire pamântul cel bun pe care cu juramânt 1-a fagaduit Domnul Dumnezeul parin?ilor vo?tri.
Deu 8:2  Sa-?i aduci aminte de toata calea pe care te-a pova?uit Domnul Dumnezeul tau prin pustie de acum patruzeci de ani, ca sa te smereasca, ca sa te încerce ?i ca sa afle ce este în inima ta ?i de ai sa paze?ti sau nu poruncile Lui.
Deu 8:3  Te-a smerit, te-a pedepsit cu foamea ?i te-a hranit cu mana pe care nu o cuno?teai ?i pe care nu o cuno?teau nici parin?ii tai, ca sa-?i arate ca nu numai cu pâine traie?te omul, ci ca omul traie?te ?i cu tot Cuvântul ce iese din gura Domnului.
Deu 8:4  Haina ta nu s-a învechit pe tine ?i piciorul tau n-a capatat bataturi în ace?ti patruzeci de ani.
Deu 8:5  Dar sa ?tii în inima ta ca Domnul Dumnezeul tau te înva?a, cum înva?a omul pe fiul sau.
Deu 8:6  Paze?te dar poruncile Domnului Dumnezeului tau, umblând în caile Lui ?i temându-te de El.
Deu 8:7  Ca Domnul Dumnezeul tau te va duce într-o ?ara buna, ?ara de curgeri de apa, de izvoare ?i de ape adânci, care ?â?nesc în vai ?i în mun?i;
Deu 8:8  ?ara în care se afla: grâu, orz, vi?a de vie, smochine ?i rodii;
Deu 8:9  Într-o ?ara unde sunt maslini, untdelemn ?i miere, în care fara lipsa vei mânca pâinea ta ?i nu vei duce lipsa de nimic; în care pietrele au fier ?i din mun?ii careia vei scoate arama.
Deu 8:10  Când însa vei mânca ?i te vei satura, sa binecuvântezi pe Domnul Dumnezeul tau pentru ?ara cea buna pe care ?i-a dat-o.
Deu 8:11  Fere?te-te de a uita pe Domnul Dumnezeul tau ?i de a nu pazi poruncile Lui, legile Lui ?i hotarârile Lui pe care ti le spun eu astazi.
Deu 8:12  Când vei mânca ?i te vei satura ?i î?i vei face case frumoase ?i vei trai în ele;
Deu 8:13  Când vei avea multe vite mari ?i marunte ?i mult argint ?i aur ?i vei avea de toate la tine,
Deu 8:14  Vezi sa nu se mândreasca inima ta ?i sa nu ui?i pe Domnul Dumnezeul tau Care te-a scos din Egipt, din casa robiei;
Deu 8:15  Care te-a pova?uit prin pustiul cel mare ?i groaznic, unde sunt ?erpi venino?i, scorpioni ?i locuri arse de soare ?i fara de apa;
Deu 8:16  Care a scos pentru tine izvor din stânca de cremene, te-a hranit în pustie cu mana pe care tu n-o cuno?teai ?i n-o cuno?teau nici parin?ii tai, ca sa te smereasca ?i sa te încerce,
Deu 8:17  Ca sa-?i faca bine în urma ?i ca sa nu zici în inima ta: Puterea mea ?i taria mâinii mele mi-au adus boga?ia aceasta;
Deu 8:18  Ci ca sa-?i aduci aminte de Domnul Dumnezeul tau, ca El î?i da putere sa faci boga?ie, ca sa-?i ?ina, ca acum, legamântul Lui, pe care cu juramânt l-a întarit cu parin?ii tai.
Deu 8:19  Iar daca vei uita pe Domnul Dumnezeul tau ?i vei merge dupa dumnezeii altora, vei sluji acelora ?i te vei închina lor, va marturisesc astazi pe cer ?i pe pamânt ca ve?i pieri.
Deu 8:20  Cum pier popoarele, pe care Domnul Dumnezeu le pierde dinaintea voastra, a?a ve?i pieri ?i voi, de nu ve?i asculta glasul Domnului Dumnezeului vostru!"
Deu 9:1  "Asculta, Israele, de acum tu vei trece Iordanul, ca sa intri ?i sa cuprinzi popoare mai mari ?i mai puternice decât tine ?i ceta?i mari cu ziduri pâna la cer,
Deu 9:2  Precum ?i pe poporul cei mare, mult la numar ?i înalt la statura, pe fiii lui Enac, de care tu ai auzit spunându-se: Cine se va împotrivi fiilor lui Enac?"
Deu 9:3  Afla dar astazi ca Domnul Dumnezeul tau merge înaintea ta. Acesta este foc mistuitor: pierde-i-va ?i-i va doborî înaintea ta, ?i tu îi vei izgoni ?i-i vei omorî curând, cum ji-a grait Domnul.
Deu 9:4  Când îi va izgoni Domnul Dumnezeul tau de la fa?a ta, sa nu zici în inima ta: "Pentru dreptatea mea m-a adus Domnul sa stapânesc pe acest pamânt bun", caci pentru necredincio?ia popoarelor acestora le izgone?te Domnul de la fa?a ta.
Deu 9:5  Nu pentru dreptatea ta ?i nici pentru dreptatea inimii tale mergi sa mo?tene?ti pamântul lor, ci pentru necredin?a ?i faradelegile popoarelor acestora le izgone?te Domnul Dumnezeul tau de la fa?a ta ?i ca sa împlineasca fagaduin?a cu care S-a jurat Domnul parin?ilor tai: lui Avraam, lui Isaac ?i lui Iacov.
Deu 9:6  De aceea sa ?tii astazi ca nu pentru dreptatea ta î?i da Domnul Dumnezeul tau sa mo?tene?ti acest pamânt bun, ca tu e?ti un popor tare la cerbice.
Deu 9:7  ?ine minte ?i nu uita de câte ori ai mâniat pe Domnul Dumnezeul tau în pustie; din ziua când a?i ie?it din pamântul Egiptului ?i pâna ce au sosit la locul acesta, necontenit v-a?i împotrivit Domnului.
Deu 9:8  La Horeb a?i mâniat pe Domnul ?i a?i pornit pe Domnul asupra voastra, a?a încât a vrut sa va piarda.
Deu 9:9  Când m-am suit eu pe munte, ca sa primesc lespezile de piatra, tablele legamântului, pe care 1-a încheiat Domnul cu voi, am stat în munte patruzeci de zile ?i patruzeci de nop?i,
Deu 9:10  ?i nici pâine n-am mâncat, nici apa n-am baut. Atunci mi-a dat Domnul doua table de piatra, scrise cu degetul lui Dumnezeu; pe acelea erau scrise toate cuvintele pe care vi le-a grait Domnul pe munte, din mijlocul focului, în ziua adunarii.
Deu 9:11  Dar dupa trecerea celor patruzeci de zile ?i patruzeci de nop?i, când mi-a dat Domnul cele doua table de piatra, tablele legamântului, mi-a zis Domnul:
Deu 9:12  "Scoala ?i te pogoara repede de aici, ca s-a razvratit poporul tau pe care l-ai scos din Egipt; curând s-a abatut el de la calea pe care i-am poruncit sa mearga ?i ?i-a facut chip turnat".
Deu 9:13  Tot atunci mi-a mai zis Domnul: "De mai multe ori ti-am grait ?i ?i-am zis: Ma uit la poporul acesta ?i vad ca este popor tare la cerbice.
Deu 9:14  Lasa-Ma dar acum sa-l pierd ?i sa ?terg numele lui de sub cer ?i voi ridica din tine popor mai mare, mai puternic ?i mai mult la numar decât ei".
Deu 9:15  Atunci eu m-am întors ?i m-am pogorât din munte, iar muntele ardea în foc. Cele doua table ale legamântului erau în amândoua mâinile mele.
Deu 9:16  Am vazut însa ca voi pacatuisera?i înaintea Domnului Dumnezeului vostru, va facusera?i un vi?el turnat ?i va abatusera?i curând de la calea pe care va poruncise Domnul sa o urma?i.
Deu 9:17  Am luat atunci cele doua table ?i, aruncându-le cu amândoua mâinile mele, le-am sfarâmat înaintea ochilor vo?tri.
Deu 9:18  Apoi am îngenunchiat a doua oara înaintea Domnului, ca ?i întâia oara, patruzeci de zile ?i patruzeci de nop?i, fara sa manânc pâine ?i fara sa beau apa; m-am rugat pentru pacatele voastre cu care a?i gre?it voi, facând rau înaintea ochilor Domnului Dumnezeului vostru ?i mîniindu-L.
Deu 9:19  Ca eu am fost îngrozit de mânia ?i de iu?imea cu care Se mâniase Domnul pe voi, voind sa va piarda. ?i m-a auzit Domnul ?i de data aceasta.
Deu 9:20  Atunci Se mâniase Domnul foarte tare ?i pe Aaron, vrând sa-l piarda ?i pe el; dar m-am rugat eu în vremea aceea ?i pentru Aaron.
Deu 9:21  Iar pacatul vostru pe care l-a?i facut, adica vi?elul, l-am luat, l-am ars în foc, l-am sfarâmat ?i l-am pisat bine, pâna când s-a facut marunt ca praful, ?i. praful acesta l-am aruncat în pârâul ce curgea din munte.
Deu 9:22  La Tabeerah, la Masa ?i la Chibrot-Hataava voi iara?i a?i mâniat pe Domnul Dumnezeul vostru.
Deu 9:23  Când v-a trimis Domnul din Cade?-Barnea, zicând: Merge?i de lua?i pamântul pe care vi-l dau Eu, v-a?i împotrivit poruncii Domnului Dumnezeului vostru ?i nu l-a?i crezut, nici n-a?i ascultat glasul Lui.
Deu 9:24  Nesupu?i a?i fost Domnului chiar din ziua când am început a va cunoa?te.
Deu 9:25  Îngenunchind, a?adar, înaintea Domnului, m-am rugat eu patruzeci de zile ?i patruzeci de nop?i, caci Domnul zisese sa va piarda. Eu însa m-am rugat Domnului ?i am zis:
Deu 9:26  Stapâne Doamne, Împarate, Dumnezeule, nu pierde pe poporul Tau ?i mo?tenirea Ta, pe care l-ai izbavit cu marirea puterii Tale ?i pe care l-ai scos din Egipt cu mâna tare ?i cu bra?ul Tau cel înalt.
Deu 9:27  Adu-?i aminte de robii Tai: Avraam, Isaac ?i Iacov, carora Te-ai jurat pe Tine Însu?i; nu Te uita la cerbicia poporului acestuia, nici la necredin?a lui, nici la pacatele lui,
Deu 9:28  Ca cei ce traiesc în pamântul de unde ne-ai scos Tu sa nu zica: Domnul nu i-a putut duce în pamântul ce le-a fagaduit ?i, urându-i, i-a scos ca sa-i omoare în pustiu.
Deu 9:29  Ei sunt poporul Tau ?i mo?tenirea Ta, pe care l-ai scos din pamântul Egiptului cu puterea Ta cea mare ?i cu bra?ul Tau cel înalt".
Deu 10:1  "Atunci mi-a zis Domnul: Ciople?te-?i doua table de piatra, ca ?i cele dintâi, ?i suie-te la Mine în munte ?i-?i fa un chivot de lemn;
Deu 10:2  Ca Eu am sa scriu pe tablele acelea cuvintele ce au fost pe tablele cele dintâi, pe care le-ai sfarâmat, iar tu sa le pui în chivot.
Deu 10:3  Am facut atunci un chivot din lemn de salcâm, am cioplit doua table de piatra, ca ?i cele dintâi, ?i m-am suit în munte cu cele doua table în mâinile mele.
Deu 10:4  Iar El a scris pe table, cum fusese scris ?i pe cele dintâi, cele zece porunci pe care vi le spusese Domnul pe munte din mijlocul focului, în ziua adunarii, ?i mi le-a dat Domnul mie;
Deu 10:5  Iar eu m-am întors ?i m-am pogorât din munte ?i am pus tablele în chivotul pe care-l facusem, ca sa stea acolo, cum îmi poruncise Domnul.
Deu 10:6  Apoi au plecat fiii lui Israel din Beerot-Bene-Iaacan la Mosera. Acolo a murit Aaron ?i a fost îngropat acolo ?i în locul lui s-a facut preot Eleazar, fiul lui.
Deu 10:7  De acolo am plecat la Gudgod ?i din Gudgod la Iotbata, în pamântul unde sunt cursuri de apa.
Deu 10:8  În vremea aceea a ales Domnul semin?ia lui Levi, ca sa poarte chivotul legamântului Domnului, sa stea înaintea Domnului, sa-I slujeasca, sa se roage ?i sa-I binecuvânteze numele Lui, cum face pâna în ziua de astazi.
Deu 10:9  De aceea n-are levitul parte ?i mo?tenire cu fra?ii sai, caci mo?tenirea lui este Domnul, cum i-a spus Domnul Dumnezeul tau.
Deu 10:10  Deci am stat eu pe munte, ca ?i întâia oara, patruzeci de zile ?i patruzeci de nop?i; ?i m-a ascultat Domnul ?i de asta data ?i n-a mai vrut Domnul sa te piarda,
Deu 10:11  Ci mi-a zis Domnul: Scoala ?i mergi înaintea poporului acestuia, ca sa intre ?i sa mo?teneasca pamântul pentru care M-am jurat parin?ilor lor sa li-l dau.
Deu 10:12  A?adar, Israele, ce cere de la tine Domnul Dumnezeul tau? - Numai aceasta: sa te temi de Domnul Dumnezeul tau, sa umbli în toate caile Lui, sa-L iube?ti ?i sa sluje?ti Domnului Dumnezeului tau, din toata inima ta ?i din tot sufletul tau;
Deu 10:13  Sa paze?ti poruncile Domnului Dumnezeului tau ?i hotarârile Lui pe care ?i le spun eu astazi, ca sa-?i fie bine.
Deu 10:14  Iata, al Domnului Dumnezeului tau este cerul ?i cerurile cerurilor, pamântul ?i toate cele de pe el.
Deu 10:15  Dar numai pe parin?ii tai i-a primit Domnul ?i i-a iubit ?i v-a ales pe voi, samân?a lor de dupa ei, din toate popoarele, cum vede?i astazi.
Deu 10:16  Deci sa taia?i împrejur inima voastra ?i de acum înainte sa nu mai fi?i tari la cerbice;
Deu 10:17  Ca Domnul Dumnezeul vostru este Dumnezeul dumnezeilor ?i Stapânul stapânilor, Dumnezeu mare ?i puternic ?i minunat, Care nu cauta la fa?a, nici nu ia mita;
Deu 10:18  Care face dreptate orfanului ?i vaduvei ?i iube?te pe pribeag ?i-i da pâine ?i hrana.
Deu 10:19  Sa iubi?i ?i voi pe pribeag, ca ?i voi a?i fost pribegi în pamântul Egiptului.
Deu 10:20  De Domnul Dumnezeul tau sa te temi, numai Lui sa-I sluje?ti, de El sa te lipe?ti ?i cu numele Lui sa te juri.
Deu 10:21  El este lauda ta ?i El este Dumnezeul tau, Cel ce a facut cu tine acele lucruri mari ?i înfrico?atoare pe care le-au vazut ochii tai.
Deu 10:22  ?aptezeci ?i cinci de suflete erau parin?ii tai când au venit în Egipt, iar acum Domnul Dumnezeul tau ?i-a sporit numarul ca stelele cerului".
Deu 11:1  "Sa iube?ti dar pe Domnul Dumnezeul tau ?i sa paze?ti în toate zilele cele ce ?i-a poruncit El sa paze?ti: hotarârile Lui, legile Lui ?i poruncile Lui.
Deu 11:2  Baga?i de seama dar ca eu nu graiesc cu copiii vo?tri, care nu ?tiu ?i n-au vazut pedeapsa Domnului Dumnezeului vostru, nici slava Lui, nici mâna Lui cea tare, nici bra?ul Lui cel înalt,
Deu 11:3  Nici semnele Lui, nici lucrurile Lui, pe care le-a facut în mijlocul Egiptului cu Faraon, regele egiptean, ?i cu tot pamântul lui,
Deu 11:4  Nici ce a facut El cu o?tirea egipteana, cu caii lui ?i cu carele lui, pe care le-â înecat în apele Marii Ro?ii, când alergau dupa voi; ?i i-a pierdut Domnul Dumnezeu pâna în ziua de astazi;
Deu 11:5  Nici ce a facut El pentru voi în pustie, pâna când a?i ajuns în locul acesta;
Deu 11:6  Nici ce a facut El cu Datan ?i Abiron, fiii lui Eliab, fiul lui Ruben, când ?i-a deschis pamântul gura sa ?i în mijlocul a tot Israelul i-a înghi?it pe ei, pe familiile lor, corturile lor ?i toata averea ce o aveau.
Deu 11:7  Caci ochii vo?tri au vazut toate lucrurile cele mari ale Domnului, pe care le-a facut El.
Deu 11:8  De aceea pazi?i toate poruncile Lui, pe care vi le spun eu astazi, ca sa fi?i vii ?i sa va întari?i, sa va duce?i sa mo?teni?i pamântul în care ve?i trece peste Iordan ca sa-l stapâni?i;
Deu 11:9  ?i ca sa trai?i multa vreme în pamântul acela pentru care Domnul S-a jurat parin?ilor vo?tri sa li-l dea lor ?i semin?iei lor, în pamântul unde curge miere ?i lapte.
Deu 11:10  Caci pamântul la care mergi tu ca sa-l stapâne?ti nu este ca pamântul Egiptului din care ai ie?it, unde, semanând samân?a, o udai cu ajutorul picioarelor tale, ca pe o gradina de legume.
Deu 11:11  Ci pamântul în care trece?i ca sa-l stapâni?i este o ?ara cu mun?i ?i cu vai ?i se adapa cu apa din ploaia cerului.
Deu 11:12  Este ?ara de care poarta grija Domnul Dumnezeul tau; ochii Domnului Dumnezeului tau sunt necontenit asupra ei, de la începutul anului pâna la sfâr?itul lui.
Deu 11:13  De ve?i asculta poruncile Mele pe care vi le dau astazi, zice Domnul, ?i ve?i iubi pe Domnul Dumnezeul vostru ?i-I ve?i sluji din toata inima ?i din tot sufletul vostru,
Deu 11:14  Voi da pamântului vostru ploaie la vreme, timpurie ?i târzie, ?i-?i vei strânge pâinea ta, vinul tau ?i untdelemnul tau;
Deu 11:15  Voi da iarba pe câmpia ta pentru dobitoacele tale ?i vei mânca ?i te vei satura.
Deu 11:16  Pazi?i-va sa nu se mândreasca inima voastra ?i sa nu va abate?i, nici sa va apuca?i sa sluji?i altor dumnezei ?i sa va închina?i lor.
Deu 11:17  Ca atunci se va aprinde mânia Domnului asupra voastra, va închide cerul ?i nu va fi ploaie ?i pamântul nu-?i va da roadele sale; iar voi ve?i pieri curând de pe pamântul cel bun pe care Domnul vi-l da.
Deu 11:18  Pune?i dar aceste cuvinte ale mele în inima voastra ?i în sufletul vostru; lega?i-le ca semn la mâna voastra ?i sa le ave?i ca pe o tabli?a pe fruntea voastra.
Deu 11:19  Sa înva?a?i acestea ?i pe fiii vo?tri, graind de ele când ?ede?i  acasa ?i când merge?i pe cale, când va culca?i ?i când va scula?i.
Deu 11:20  Sa le scrie?i  pe u?orii caselor voastre ?i pe por?ile voastre,
Deu 11:21  Ca zilele voastre ?i zilele copiilor vo?tri în acel pamânt bun, pentru care Domnul S-a jurat parin?ilor vo?tri, sa fie atât de multe, câte vor fi zilele cerului deasupra pamântului.
Deu 11:22  Ca de ve?i pazi voi toate poruncile acestea, pe care va poruncesc sa le pazi?i, ?i de ve?i iubi pe Domnul Dumnezeul vostru, umblând în toate caile Lui ?i lipindu-va de El,
Deu 11:23  Va alunga Domnul toate popoarele acestea de la fa?a, voastra ?i ve?i stapâni popoare mai mari ?i mai puternice decât voi.
Deu 11:24  Tot locul pe care va calca piciorul vostru va fi al vostru; de la pustiu pâna la Liban, de la râul Eufratului ?i pâna la marea cea de la asfin?it, se vor întinde hotarele voastre.
Deu 11:25  Nimeni nu va putea sta înaintea voastra; Domnul Dumnezeul vostru va aduce frica ?i cutremur peste tot pamântul dinaintea voastra, în care ve?i calca voi, dupa cum v-a grait.
Deu 11:26  Iata, eu va pun astazi înainte binecuvântare ?i blestem:
Deu 11:27  Binecuvântare ve?i avea daca ve?i asculta poruncile Domnului Dumnezeului vostru, pe care vi le spun eu astazi; iar blestem, daca nu ve?i asculta poruncile Domnului Dumnezeului vostru,
Deu 11:28  Ci va ve?i abate de la calea pe care v-o poruncesc astazi ?i ve?i merge dupa dumnezei pe care nu-i ?ti?i.
Deu 11:29  Când te va duce Domnul Dumnezeul tau în pamântul acela în care mergi ca sa-l mo?tene?ti, atunci sa roste?ti binecuvântarea pe muntele Garizim ?i blestemul pe muntele Ebal.
Deu 11:30  Iata, ace?tia sunt peste Iordan, în drumul spre asfin?itul soarelui, în pamântul Canaaneilor, care locuiesc pe ?esul Arabah din fa?a Ghilgalei, aproape de stejarul More.
Deu 11:31  Caci voi ve?i trece Iordanul, ca sa mergeri sa lua?i pamântul pe care Domnul Dumnezeul vostru vi-l da mo?tenire ve?nica ?i-l ve?i lua în stapânire ?i ve?i trai în el.
Deu 11:32  A?adar sili?i-va sa pazi?i toate hotarârile ?i poruncile Lui pe care vi le pun eu astazi înainte".
Deu 12:1  "Iata hotarârile ?i legile pe care trebuie sa va sili?i a le împlini în pamântul ce Domnul Dumnezeul parin?ilor vo?tri vi-l da în stapânire, în toate zilele cît ve?i trai în pamântul acela.
Deu 12:2  Sa pustii?i toate locurile în care popoarele ce le ve?i supune au slujit dumnezeilor lor, cele din mun?ii înal?i, cele de pe dealuri ?i cele de sub orice copac umbros.
Deu 12:3  Sa darâma?i jertfelnicele lor, sa strica?i stâlpii lor, sa arderi cu foc copacii lor, sa sfarâma?i idolii dumnezeilor lor ?i sa ?terge?i numele lor din locurile acelea.
Deu 12:4  Iar Domnului Dumnezeului vostru sa nu-I face?i a?a;
Deu 12:5  Ci la locul pe care-l va alege Domnul Dumnezeul vostru din toate semin?iile voastre, ca sa-?i puna numele Sau asupra lui, sa veni?i sa-l cerceta?i.
Deu 12:6  Acolo sa aduce?i arderile de tot ale voastre ?i jertfele voastre, zeciuielile voastre ?i ridicarea mâinilor voastre, fagaduin?ele voastre, prinoasele voastre cele de buna voie ?i jertfele voastre de pace, pe întâii nascu?i ai vitelor voastre mari ?i ai vitelor voastre mici;
Deu 12:7  Sa mânca?i acolo înaintea Domnului Dumnezeului vostru ?i sa va veseli?i cu familiile voastre pentru toate câte au facut mâinile voastre ?i cu câte v-a binecuvântat Domnul Dumnezeul vostru.
Deu 12:8  Sa nu face?i a?a cum facem noi acum aici, adica ceea ce i se pare fiecaruia ca este bine;
Deu 12:9  Caci noi acum n-am intrat în locul odihnei ?i în mo?tenirea pe care ji-o da Domnul Dumnezeul tau;
Deu 12:10  Ci când ve?i trece Iordanul ?i va ve?i a?eza în pamântul ce vi-l da Domnul Dumnezeul vostru mo?tenire, când El va va lini?ti de to?i vrajma?ii vo?tri, care va înconjura, ?i ve?i trai la adapost de primejdii,
Deu 12:11  Atunci la locul pe care-l va alege Domnul Dumnezeul vostru, ca sa-?i puna numele asupra Lui, acolo sa aduce?i tot ce v-am poruncit Eu astazi: arderile  de tot ale voastre, jertfele voastre, zeciuielile voastre, ridicarea mâinilor voastre ?i toate cele alese dupa fagaduin?ele voastre, ce a?i fagaduit Domnului Dumnezeului vostru.
Deu 12:12  Sa va veseli?i înaintea Domnului Dumnezeului vostru, voi, fiii vo?tri ?i fiicele voastre, robii vo?tri, roabele voastre ?i levitul cel din mijlocul sala?urilor voastre, ca acela n-are parte ?i mo?tenire cu voi.
Deu 12:13  Fere?te-te de a-?i aduce arderile de tot ale tale în orice loc s-ar întâmpla,
Deu 12:14  Ci numai în locul acela pe care-l va alege Domnul Dumnezeul tau în una din semin?iile tale, ?i fa tot ce ?i-am poruncit eu astazi.
Deu 12:15  Totu?i, când î?i va pofti sufletul, vei putea sa junghii ?i sa manânci carne oriunde vei trai, dupa binecuvântarea pe care ?i-a dat-o Domnul Dumnezeul tau; cel ce va fi necurat ?i cel ce va fi curat vor putea sa manânce carne, cum se manânca cea de caprioara ?i cea de cerb.
Deu 12:16  Numai sânge sa nu mânca?i, ci sa-l varsa?i jos ca apa.
Deu 12:17  Tu nu vei putea sa manânci în sala?urile tale zeciuiala de la pâinea ta, de la vinul tau ?i de la untdelemnul tau, întâii nascu?i ai vitelor tale mari ?i ai vitelor tale marunte, nici darurile tale de buna voie, pe care le-ai fagaduit, nici prinoasele tale cele de buna voie, nici cele ridicate ale mâinilor tale,
Deu 12:18  Ci acestea sa le manânci înaintea Domnului Dumnezeului tau, în locul acela pe care-l va alege Domnul Dumnezeul tau, tu ?i fiul tau, fiica ta, robul tau ?i roaba ta, levitul ?i strainul care este în loca?urile tale ?i sa te vesele?ti înaintea Domnului Dumnezeului tau de toate câte au facut mâinile tale.
Deu 12:19  Baga de seama sa nu parase?ti pe levit în toate zilele cît vei trai în pamântul tau.
Deu 12:20  Când va largi Domnul Dumnezeul tau hotarele tale, precum ?i-a grait El, ?i vei zice: Voi mânca ?i carne, pentru ca sufletul tau dore?te sa manânce carne, manânca ?i carne, dupa dorin?a sufletului tau.
Deu 12:21  De va fi departe de tine locul pe care l-a ales Domnul Dumnezeul tau, ca sa-?i puna numele asupra lui, atunci sa junghii din vitele tale mari ?i marunte pe care ?i le-a dat Domnul Dumnezeul tau, dupa cum ?i-am poruncit eu, ?i sa manânci în locuin?ele tale, dupa dorin?a sufletului tau.
Deu 12:22  Dar sa le manânci cum se manânca cerbul ?i caprioara; aceasta poate sa manânce ?i cel curat ?i cel necurat al tau.
Deu 12:23  Dar ia bine seama sa nu manânci sânge, pentru ca sângele are în el via?a ?i sa nu manânci via?a laolalta cu carnea.
Deu 12:24  Sa nu manânci sângele, ci sa-l ver?i jos ca apa.
Deu 12:25  Sa nu-l manânci, ca sa-?i fie bine în veci, ?ie ?i copiilor tai de dupa tine, ?i bine-?i va fi de vei face cele bune ?i cele placute înaintea ochilor Domnului Dumnezeului tau.
Deu 12:26  Dar cele închinate ale tale, câte le vei avea, ?i cele fagaduite ale tale, ia-le ?i vino la locul acela pe care 1-a ales Domnul Dumnezeul tau, ca sa se cheme numele Lui acolo.
Deu 12:27  Sa savâr?e?ti arderile de tot ale tale, carnea ?i sângele, pe jertfelnicul Domnului Dumnezeului tau; sângele celorlalte jertfe ale tale sa fie varsat lânga jertfelnicul Domnului Dumnezeului tau, iar carnea lor s-o manânci.
Deu 12:28  Asculta ?i împline?te toate poruncile acestea pe care ?i le dau eu astazi, ca sa-?i fie bine în veci, ?ie ?i copiilor tai ?i bine-?i va fi de vei face cele bune ?i placute înaintea ochilor Domnului Dumnezeului tau.
Deu 12:29  Când Domnul Dumnezeul tau va pierde de la fa?a ta popoarele la care mergi, ca sa le cuprinzi, ?i dupa ce le vei cuprinde ?i te vei a?eza în pamântul lor,
Deu 12:30  Atunci sa te paze?ti ca sa nu cazi în cursa ?i sa le urmezi lor, dupa ce le vei pierde de pe fa?a pamântului, ?i sa nu cau?i pe dumnezeii lor, zicând: Cum au slujit popoarele acestea dumnezeilor lor, a?a voi face ?i eu.
Deu 12:31  Sa nu faci a?a Domnului Dumnezeului tau, caci aceia fac dumnezeilor lor toate de care se îndeparteaza Domnul ?i pe care le ura?te El; aceia ?i pe fiii ?i pe fiicele lor le ard pe foc înaintea dumnezeilor lor.
Deu 12:32  Toate câte va poruncesc sili?i-va sa le împlini?i ?i nici sa adaugi ?i nici Sa la?i ceva din ele".
Deu 13:1  "De se va ridica în mijlocul tau prooroc sau vazator de vise ?i va face înaintea ta semn ?i minune,
Deu 13:2  ?i se va împlini semnul sau minunea aceea, de care ?i-a grait el, ?i-?i va zice atunci: Sa mergem dupa al?i dumnezei, pe care tu nu-i ?tii ?i sa le slujim acelora,
Deu 13:3  Sa nu ascul?i cuvintele proorocului aceluia sau ale acelui vazator de vise, ca prin aceasta va ispite?te Domnul Dumnezeul vostru, ca sa afle de iubi?i pe Domnul Dumnezeul vostru din toata inima voastra ?i din tot sufletul vostru.
Deu 13:4  Domnului Dumnezeului vostru sa-I urma?i ?i de El sa va teme?i; sa pazi?i poruncile Lui ?i glasul Lui sa-l asculta?i; Lui sa-I sluji?i ?i de El sa va lipi?i.
Deu 13:5  Iar pe proorocul acela sau pe vazatorul acela de vise sa-l da?i mor?ii, pentru ca v-a sfatuit sa va abate?i de la Domnul Dumnezeul vostru, Cel ce v-a scos din pamântul Egiptului ?i v-a izbavit din casa robiei, dorind sa te abata de la calea pe care ?i-a poruncit Domnul Dumnezeul tau sa mergi; pierde dar raul din mijlocul tau.
Deu 13:6  De te va îndemna în taina fratele tau, fiul tatalui tau, sau fiul mamei tale, sau fiul tau, sau fiica ta, sau femeia de la sânul tau, sau prietenul tau care este pentru tine ca sufletul tau, zicând: Haidem sa slujim altor dumnezei, pe care tu ?i parin?ii tai nu i-a?i ?tiut,
Deu 13:7  Dumnezeilor acelor popoare, care locuiesc împrejurul tau, aproape sau departe de tine, de la un capat pâna la celalalt al pamântului,
Deu 13:8  Sa nu te învoie?ti cu ei, nici sa-i ascul?i; sa nu-i cru?e ochii tai, sa nu-?i fie mila de ei, nici sa-i ascunzi;
Deu 13:9  Ci ucide-i; mâna ta sa fie înaintea tuturor asupra lor, ca sa-i ucida, ?i apoi sa urmeze mâinile a tot poporul.
Deu 13:10  Sa-i ucizi cu pietre pâna la moarte, ca au încercat sa te abata de la Domnul Dumnezeul tau, Care te-a scos din pamântul Egiptului ?i din casa robiei.
Deu 13:11  Tot Israelul va auzi aceasta ?i se va teme ?i nu se vor mai apuca pe viitor sa mai faca în mijlocul tau asemenea rau.
Deu 13:12  De vei auzi de vreuna din ceta?ile tale, pe care Domnul Dumnezeul tau ?i le da ca sa locuie?ti,
Deu 13:13  Ca s-au ivit în ea oameni necredincio?i dintre ai tai ?i au smintit pe locuitorii ceta?ii lor, zicând: Haidem sa slujim altor dumnezei, pe care voi nu i-a?i ?tiut,
Deu 13:14  Cauta, cerceteaza ?i întreaba bine, ?i de va fi adevarat ca s-a întâmplat urâciunea aceasta în mijlocul tau,
Deu 13:15  Sa love?ti pe locuitorii acelei ceta?i cu ascu?i?ul sabiei, s-o dai, blestemului pe ea ?i tot ce este în ea ?i dobitoacele ei sa le treci prin ascu?i?ul sabiei.
Deu 13:16  Iar prazile ei sa le aduni toate în mijlocul pie?ii ei ?i sa arzi cu foc cetatea ?i toata prada ei, ca ardere de tot Domnului Dumnezeului tau; sa fie ea pe vecie darâmata ?i niciodata sa nu se mai zideasca.
Deu 13:17  Nimic din cele blestemate sa nu se lipeasca de mâna ta, ca sa-?i potoleasca Domnul iu?imea mâniei Sale ?i sa-?i dea mila ?i îndurare, ?i sa te înmul?easca, cum ?i-a grait ?i ?ie ?i cum S-a jurat parin?ilor tai,
Deu 13:18  De vei asculta glasul Domnului Dumnezeului tau, pazind toate poruncile Lui pe care ?i le dau acum ?i facând cele bune ?i placute înaintea ochilor Domnului Dumnezeului tau".
Deu 14:1  "Voi sunte?i fiii Domnului Dumnezeului vostru; sa nu face?i crestaturi pe trupul vostru ?i sa nu va tunde?i parul de deasupra ochilor vo?tri, pentru mor?i;
Deu 14:2  Caci voi sunte?i poporul sfânt al Domnului Dumnezeului vostru ?i pe voi v-a ales Domnul ca sa-I fi?i poporul Lui de mo?tenire dintre toate popoarele câte sunt pe pamânt.
Deu 14:3  Sa nu mânca?i nici un lucru necurat.
Deu 14:4  Iata dobitoacele pe care le pute?i mânca:
Deu 14:5  Boul, oaia, capra, cerbul, gazela, antilopa, ?apul, cerboaica, boul salbatic ?i capra salbatica.
Deu 14:6  Orice dobitoc care are copita despicata, cu spintecatura adânca între amândoua par?ile copitei ?i care dobitoc rumega mâncarea, se manânca.
Deu 14:7  Dintre cele ce î?i rumega mâncarea sau î?i au copita despicata printr-o spintecatura adânca, numai acestea nu se manânca: camila, iepurele ?i iepurele de casa, pentru ca, de?i acestea î?i rumega mâncarea, dar nu-?i au copita despicata, acestea sunt necurate pentru voi.
Deu 14:8  Nu se manânca porcul, pentru ca, de?i are copita despicata, nu î?i rumega mâncarea; acesta este necurat pentru voi. Carnea acestora sa n-o mânca?i ?i de stârvurile lor sa nu va atinge?i.
Deu 14:9  Din toate vieta?ile care sunt în apa, sa mânca?i pe acelea care au aripi ?i solzi;
Deu 14:10  Iar pe toate celelalte, care n-au aripi ?i solzi, sa nu le mânca?i; necurate sunt pentru voi.
Deu 14:11  Orice pasare curata s-o mânca?i.
Deu 14:12  Dar din ele sa nu mânca?i pe acestea: vulturul, vulturul rapitor ?i vulturul de mare,
Deu 14:13  Corbul, ?oimul, gaia cu soiurile ei,
Deu 14:14  Tot soiul de ciori,
Deu 14:15  Stru?ul, cucuveaua, pescaru?ul ?i uliul cu soiurile lui,
Deu 14:16  Huhurezul, ibisul ?i lebada,
Deu 14:17  Pelicanul, porfirionul ?i corbul de mare,
Deu 14:18  Cocostârcul, pupaza cu soiurile ei ?i liliacul.
Deu 14:19  Toate înaripatele târâtoare sunt necurate pentru voi; sa nu le mânca?i.
Deu 14:20  Orice pasare curata s-o mânca?i.
Deu 14:21  Sa nu mânca?i nici o mortaciune, ci s-o dai strainului de alt neam ce se va întâmpla sa locuiasca în casa ta; acela s-o manânce sau sa i-o vinzi, caci tu e?ti poporul sfânt al Domnului Dumnezeului tau. Sa nu fierbi iedul în laptele mamei sale.
Deu 14:22  Sa osebe?ti zeciuiala din toate veniturile semanaturilor tale, care-?i vin din ?arina ta în. fiecare an,
Deu 14:23  ?i sa manânci înaintea Domnului Dumnezeului tau, la locul ce-l va alege El, ca sa-I fie numele acolo; adu zeciuiala din pâinea ta, din vinul tau, din untdelemnul tau ?i pe întâii nascu?i ai vitelor tale mari ?i ai vitelor tale marunte, ca sa te înve?i a te teme de Domnul Dumnezeul tau în toate zilele.
Deu 14:24  Iar de va fi pentru tine drumul lung, încât sa nu po?i aduce acestea, pentru ca este departe de tine locul pe care l-a ales Domnul Dumnezeul tau, ca sa-?i puna acolo numele Sau, ?i Domnul Dumnezeul tau te-a binecuvântat,
Deu 14:25  Atunci schimba acestea pe argint ?i ia argintul în mâna ta ?i vino la locul pe care 1-a ales Domnul Dumnezeul tau;
Deu 14:26  Apoi cumpara pe argintul acesta tot ce dore?te sufletul tau: boi, oi, vin, sichera ?i orice î?i pofte?te sufletul tau, ?i manânca acolo înaintea Domnului Dumnezeului tau ?i te vesele?te, tu ?i familia ta.
Deu 14:27  Dar pe levitul care este în loca?urile tale sa nu-l parase?ti, caci el nu are parte ?i mo?tenire cu tine.
Deu 14:28  Iar dupa trecerea a trei ani, ia toate zeciuielile veniturilor tale din anul acela ?i le pune în loca?urile tale;
Deu 14:29  ?i sa vina levitul, caci el nu are parte ?i mo?tenire cu tine, ?i strainul ?i orfanul ?i vaduva care se afla în sala?urile tale ?i sa manânce ?i sa se sature, ca sa te binecuvânteze Domnul Dumnezeul tau în toate lucrurile mâinilor tale, pe care le vei face tu".
Deu 15:1  "În anul al ?aptelea vei face iertare.
Deu 15:2  Iertarea însa va fi aceasta: tot împrumutatorul, care da împrumut aproapelui sau, sa ierte datoria ?i sa n-o mai ceara de la aproapele sau sau de la fratele sau, ca s-a vestit iertarea în cinstea Domnului Dumnezeului tau.
Deu 15:3  De la cel de alt neam sa ceri datoria; iar ce vei avea la fratele tau, sa-i ier?i.
Deu 15:4  Numai a?a nu va fi sarac printre voi; ca te va binecuvânta Domnul în pamântul acela pe care Domnul Dumnezeul tau ?i-l da în stapânire, ca sa-l ai mo?tenire,
Deu 15:5  Daca vei asculta glasul Domnului Dumnezeului tau ?i te vei sili sa pline?ti toate poruncile acestea, care ?i le spun eu astazi.
Deu 15:6  Caci Domnul Dumnezeul tau te va binecuvânta, dupa cum i-a grait ?i vei da împrumut altor poare, iar tu nu vei lua împrumut; ?i ?i domni peste multe popoare, iar acelea nu vor domni peste tine.
Deu 15:7  Iar de va fi la tine sarac vreunul din fra?ii tai, în vreuna din ceta?ile tale de pe pamântul tau pe care ?i-l da Domnul Dumnezeul tau, sa nu-?i învârto?ezi inima, nici sa-?i închizi mâna ta înaintea fratelui tau celui sarac;
Deu 15:8  Ci sa-i deschizi mâna ta ?i sa-i dai împrumuturi potrivite cu nevoia lui ?i cu lipsa ce sufera.
Deu 15:9  Paze?te-te sa nu intre în inima ta gândul nelegiuit ?i sa zici: Se apropie anul al ?aptelea, anul iertarii; ?i sa nu se faca din pricina aceasta ochiul tau nemilostiv catre fratele tau cel sarac ?i sa-l treci cu vederea; ca acela va striga împotriva ta catre Domnul ?i va fi asupra ta pacat mare.
Deu 15:10  Da-i, da-i ?i împrumuturi câte-?i va cere ?i cît îi va trebui, ?i când îi vei da, sa nu se întristeze inima ta, caci pentru aceasta te va binecuvânta Domnul Dumnezeul tau în toate lucrurile tale ?i în toate câte se vor lucra de mâinile tale.
Deu 15:11  Caci nu va lipsi sarac din pamântul tau; de aceea î?i ?i poruncesc eu: Deschide mâna ta fratelui tau, saracului tau ?i celui lipsit din pamântul tau.
Deu 15:12  De ?i se va vinde ?ie fratele tau, evreu sau evreica, ?ase ani sa fie rob la tine, iar în anul al ?aptelea sa-i dai drumul de la tine, slobod.
Deu 15:13  Iar când îi vei da drumul ca sa fie slobod, sa nu-i dai drumul cu mâinile goale;
Deu 15:14  Ci înzestreaza-l din turmele tale, din aria ta, de la teascul tau; da-i ?i lui din cele cu care te-a binecuvântat Domnul Dumnezeul tau.
Deu 15:15  Adu-?i aminte ca ?i tu ai fost rob în pamântul Egiptului ?i te-a izbavit Domnul Dumnezeul tau. Iata pentru ce î?i poruncesc acestea astazi.
Deu 15:16  Iar daca acela î?i va zice: Nu ma duc de la tine, pentru ca te iubesc pe tine ?i casa ta, ?i deci îi este bine la tine,
Deu 15:17  Sa iei sula ?i sa-i gaure?ti urechea lui de u?or, ?i î?i va fi rob pe vecie. Tot a?a sa faci ?i cu roaba ta.
Deu 15:18  Sa nu socote?ti o greutate pentru tine când va trebui sa-i dai drumul de la tine ca sa fie slobod, caci în ?ase ani ?i-a muncit de doua ori cât plata unui strain ?i te va binecuvânta Domnul Dumnezeul tau în toate câte vei face.
Deu 15:19  Tot întâiul nascut de parte barbateasca, ce se va na?te din vitele tale cele mari ?i din vitele marunte ale tale, sa-l închini Domnului Dumnezeului tau. Sa nu lucrezi cu boul tau întâi-nascut ?i sa nu tunzi pe întâiul nascut din vitele tale marunte.
Deu 15:20  Înaintea Domnului Dumnezeului tau sa manânci acestea în fiecare an, tu ?i familia ta, la locul pe care-l va alege Domnul Dumnezeul tau.
Deu 15:21  Dar daca va avea vreo meteahna, ?chiopatare, sau orbire, sau alta meteahna oarecare, sa nu-l aduci jertfa Domnului Dumnezeului tau,
Deu 15:22  Ci sa-l manânci în ceta?ile tale; atât cel necurat cît ?i cel curat pot sa manânce din el, cum manânca o caprioara sau un cerb.
Deu 15:23  Numai sângele lui sa nu-l manânci, ci sa-l ver?i jos, ca apa".
Deu 16:1  "Sa paze?ti luna Aviv ?i sa praznuie?ti Pa?tile Domnului Dumnezeului tau, pentru ca în luna Aviv te-a scos Domnul Dumnezeul tau din Egipt, noaptea.
Deu 16:2  Sa junghii Pa?tile Domnului Dumnezeului tau din vite mari ?i din vite marunte, la locul pe care-l va alege Domnul, ca sa fie numele Lui acolo.
Deu 16:3  Sa nu manânci în timpul Pa?tilor pâine dospita; ?apte zile sa manânci azime, pâinea durerii, ca sa-?i aduci aminte de ie?irea ta din pamântul Egiptului în toate zilele vie?ii tale, caci cu grabire ai ie?it tu din pamântul Egiptului.
Deu 16:4  Sa nu se afle la tine aluat dospit în tot ?inutul tau, timp de ?apte zile, ?i din carnea care ai adus-o jertfa seara, în ziua întâi, sa nu ramâna nimic pe diminea?a.
Deu 16:5  Tu nu pori sa junghii Pa?tile în vreuna din ceta?ile tale, pe care Domnul Dumnezeul tau ?i le va da.
Deu 16:6  Ci numai în locul acela pe care-l va alege Domnul Dumnezeul tau, ca sa ramâna acolo numele Lui; sa junghii Pa?tile seara, la asfin?itul soarelui, pe vremea când ai ie?it tu din Egipt.
Deu 16:7  Sa frigi ?i sa manânci în locul acela pe care-l va alege Domnul Dumnezeul tau, iar a doua zi po?i sa te întorci ?i sa intri în sala?urile tale.
Deu 16:8  ?ase zile sa manânci pâine nedospita, iar în ziua a ?aptea este încheierea sarbatorii Domnului Dumnezeului tau; sa nu lucrezi în acele zile nimic, fara numai cele pentru suflet.
Deu 16:9  Sa numeri apoi ?apte saptamâni; dar sa începi a numara cele ?apte saptamâni de când se va începe seceri?ul.
Deu 16:10  ?i atunci sa savâr?e?ti sarbatoarea saptamânilor Domnului Dumnezeului tau cu dar de bunavoie, cum î?i va da mâna ?i dupa cum vei putea, din cele cu care te-a binecuvântat Domnul Dumnezeul tau.
Deu 16:11  Sa te vesele?ti înaintea Domnului Dumnezeului tau, tu, fiul tau ?i fiica ta, robul tau ?i roaba ta, levitul din ceta?ile tale ?i strainul, orfanul ?i vaduva, care vor fi în mijlocul tau, în locul pe care l-a ales Domnul Dumnezeul tau, ca sa fie numele Lui acolo.
Deu 16:12  Adu-?i aminte ca ai fost rob în Egipt; ?ine dar ?i paze?te poruncile acestea.
Deu 16:13  Sarbatoarea corturilor s-o savâr?e?ti în ?apte zile, dupa ce vei aduna din aria ta ?i din teascul tau.
Deu 16:14  ?i sa te vesele?ti în sarbatoarea ta: tu, fiul tau ?i fiica ta, robul tau ?i roaba ta, levitul ?i strainul, orfanul ?i vaduva, care sunt în ceta?ile tale.
Deu 16:15  ?apte zile sa sarbatore?ti înaintea Domnului Dumnezeului tau, la locul pe care-l va alege Domnul Dumnezeul tau, ca sa fie numele Lui acolo; ca te va binecuvânta Domnul Dumnezeul tau în toate roadele ?i în tot lucrul mâinilor tale, ?i tu de aceea sa fii vesel.
Deu 16:16  De trei ori pe an sa se înfa?i?eze to?i cei de parte barbateasca înaintea Domnului Dumnezeului tau la locul pe care-l va alege El: la sarbatoarea azimelor, la sarbatoarea saptamânilor ?i la sarbatoarea corturilor, dar nimeni sa nu se înfa?i?eze înaintea fe?ei Domnului cu mâinile goale.
Deu 16:17  Ci fiecare sa vina cu dar în mâna sa, dupa cum 1-a binecuvântat Domnul Dumnezeul tau.
Deu 16:18  în toate ceta?ile tale, pe care ?i le va da Domnul Dumnezeul tau, sa-îi pui judecatori ?i capetenii dupa semin?iile tale, ca sa judece poporul cu judecata dreapta.
Deu 16:19  Sa nu strici legea, sa nu cau?i la fa?a ?i sa nu iei mita, ca mita orbe?te ochii în?elep?ilor ?i strâmba pricinile drepte.
Deu 16:20  Cauta dreptate ?i iar dreptate, ca sa traie?ti ?i sa stapâne?ti pamântul pe care Domnul Dumnezeul tau ?i-l da.
Deu 16:21  Sa nu-?i sade?ti dumbrava de orice fel de copaci împrejurul jertfelnicului pe care-l vei zidi Domnului Dumnezeului tau.
Deu 16:22  ?i sa nu-?i ridici stâlpi idole?ti, care sunt urâ?i de Domnul Dumnezeul tau".
Deu 17:1  Sa nu aduci jertfa Domnului Dumnezeului tau bou sau oaie cu meteahna, sau cu bete?ug, caci aceasta este urâciune înaintea Domnului Dumnezeului tau.
Deu 17:2  De se va afla la tine, în vreuna din ceta?ile tale, pe care ?i le va da Domnul Dumnezeul tau, barbat sau femeie, care sa fi facut rau înaintea ochilor Domnului Dumnezeului tau, calcând legamântul Lui,
Deu 17:3  ?i se va duce ?i se va apuca sa slujeasca altor dumnezei ?i se va închina acelora, sau soarelui, sau lunii, sau la toata o?tirea cereasca, ceea ce eu n-am poruncit;
Deu 17:4  ?i ?i se va vesti ?i vei auzi aceasta, sa cercetezi bine ?i de se va adeveri aceasta ?i se va fi facut urâciunea aceasta în Israel,
Deu 17:5  Sa sco?i pe barbatul acela sau pe femeia aceea care au facut raul acesta la por?ile tale ?i sa-i ucizi cu pietre.
Deu 17:6  Cel osândit la moarte sa moara dupa spusele a doi sau trei martori; iar pe spusa unui singur martor sa nu fie osândit nimeni la moarte.
Deu 17:7  Mâna martorilor sa se ridice asupra lui, ca sa-l ucida înaintea tuturor, ?i apoi sa se ridice mâna a tot poporul. Pierde deci raul din mijlocul tau.
Deu 17:8  Daca în vreo pricina oarecare î?i va fi greu de ales între sânge ?i sânge, între judecata ?i judecata, între batai ?i batai ?i în ceta?ile tale parerile vor fi împar?ite, atunci scoala ?i du-te la locul pe care-l va alege Domnul Dumnezeul tau, ca sa-I fie numele acolo,
Deu 17:9  ?i vino la preo?i, la levi?i ?i la judecatorul care va fi în zilele acelea ?i întreaba-i, iar ei î?i vor spune cum sa judeci.
Deu 17:10  Fa dupa Cuvântul ce-?i vor spune ei în locul pe care-l va alege Domnul Dumnezeul tau, ca sa fie chemat numele Lui acolo ?i sile?te-te sa împline?ti tot ceea ce te vor înva?a ei,
Deu 17:11  Dupa legea pe care te vor înva?a ei ?i dupa hotarârea ce-?i vor spune-o sa faci ?i sa nu te aba?i nici la dreapta, nici la stânga de la cele ce-li vor spune ei.
Deu 17:12  Iar cine se va purta a?a de îndaratnic, încât sa nu asculte pe preotul care sta acolo la slujba înaintea Domnului Dumnezeului tau, sau pe judecatorul care va fi în zilele acelea, unul ca acela sa moara.
Deu 17:13  Pierde deci raul din Israel ?i va auzi tot poporul ?i se va teme ?i nu se va mai purta în viitor cu îndaratnicie.
Deu 17:14  Când vei ajunge tu în pamântul ce ?i-l da Domnul Dumnezeul tau ?i-l vei lua în stapânire ?i te vei a?eza în el ?i vei zice: Îmi voi pune rege peste mine, ca celelalte popoare, care sunt împrejurul meu,
Deu 17:15  Atunci sa-?i pui rege peste tine pe acela pe care-l va alege Domnul Dumnezeul tau: dintre fra?ii tai sa-?i pui rege peste tine; nu vei putea sa pui rege peste tine un strain, care nu este din fra?ii tai.
Deu 17:16  Dar sa nu-?i înmul?easca acela caii ?i sa nu întoarca pe popor în Egipt, pentru ca sa-?i înmul?easca el caii, caci Domnul v-a zis: Sa nu va mai întoarceri pe calea aceasta.
Deu 17:17  Sa nu-?i înmul?easca femeile, ca sa nu se razvrateasca inima lui, ?i nici argintul ?i aurul lui sa nu ?i-l înmul?easca peste masura.
Deu 17:18  Caci, când se va sui pe scaunul regatului sau, trebuie sa-?i scrie pentru sine cartea legii acesteia din cartea care se afla la preo?ii levi?ilor,
Deu 17:19  ?i sa fie aceasta la el ?i el sa o citeasca în toate zilele vie?ii sale, ca sa înve?e a se teme de Domnul Dumnezeul sau ?i sa se sileasca a împlini toate cuvintele legii acesteia ?i toate hotarârile acestea,
Deu 17:20  Ca sa nu se îngâmfe inima lui fa?a de fra?ii lui ?i ca sa nu se abata el de la lege nici la dreapta, nici la stânga, ci ca sa fie el ?i fiii lui zile multe la domnie în Israel".
Deu 18:1  "Preo?ii, levi?ii ?i toata semin?ia lui Levi nu va avea parte ?i mo?tenire cu Israel; ace?tia sa se hraneasca cu jertfele Domnului ?i cu partea Lui;
Deu 18:2  Iar mo?tenire nu va avea el între fra?ii sai, caci Domnul însu?i este mo?tenirea lui, precum i-a grait El.
Deu 18:3  Iata ce sa se dea preo?ilor de la popor: cei ce aduc ca jertfa boi sau oi sa dea preotului spata, falcile ?i stomacul.
Deu 18:4  De asemenea pârga de la grâul tau, de la vinul tau ?i de la untdelemnul tau, pârga de lâna de la oile tale sa i-o dai lui,
Deu 18:5  Ca pe el 1-a ales Domnul Dumnezeul tau din toate semin?iile tale, ca sa stea înaintea Domnului Dumnezeului tau ?i sa slujeasca întru numele Domnului, el ?i fiii lui în toate zilele.
Deu 18:6  De va pleca levitul din una din ceta?ile tale, din tot pamântul fiilor lui Israel, unde locuie?te, ?i va veni, dupa dorin?a sufletului sau, la locul ce 1-a ales Domnul,
Deu 18:7  ?i va sluji în numele Domnului Dumnezeului tau, ca to?i fra?ii sai levi?i care stau înaintea Domnului,
Deu 18:8  Sa se foloseasca de aceea?i parte ca ?i ceilal?i, pe lânga cele primite din vânzarea mo?tenirii parinte?ti.
Deu 18:9  Când vei intra tu în pamântul ce ?i-l da Domnul Dumnezeul tau, sa nu te deprinzi a face urâciunile pe care le fac popoarele acestea.
Deu 18:10  Sa nu se gaseasca la tine de aceia care trec pe fiul sau fiica lor prin foc, nici prezicator, sau ghicitor, sau vrajitor, sau fermecator,
Deu 18:11  Nici descântator, nici chemator de duhuri, nici mag, nici de cei ce graiesc cu mor?ii.
Deu 18:12  Caci urâciune este înaintea Domnului tot cel ce face acestea, ?i pentru aceasta urâciune îi izgone?te Domnul Dumnezeul tau de la fa?a ta.
Deu 18:13  Iar tu fii fara prihana înaintea Domnului Dumnezeului tau;
Deu 18:14  Caci popoarele acestea, pe care le izgone?ti tu, asculta de ghicitori ?i de prevestitori, iar ?ie nu-?i îngaduie aceasta Domnul Dumnezeul tau.
Deu 18:15  Prooroc din mijlocul tau ?i din fra?ii tai, ca ?i mine, î?i va ridica Domnul Dumnezeul tau: pe Acela sa-L asculta?i.
Deu 18:16  Ca tu la Horeb, în ziua adunarii, ai cerut de la Domnul Dumnezeul tau ?i ai zis: Sa nu mai aud glasul Domnului Dumnezeului meu ?i focul acesta mare sa nu-l mai vad, ca sa nu mor.
Deu 18:17  Atunci mi-a zis Domnul: Bine este ceea ce ?i-au spus ei.
Deu 18:18  Eu le voi ridica Prooroc din mijlocul fra?ilor lor, cum e?ti tu, ?i voi pune cuvintele Mele în gura Lui ?i El le va grai tot ce-I voi porunci Eu.
Deu 18:19  Iar cine nu va asculta cuvintele Mele, pe care Proorocul Acela le va grai în numele Meu, aceluia îi voi cere socoteala.
Deu 18:20  Iar proorocul care va îndrazni sa graiasca în numele Meu ceea ce nu i-am poruncit Eu sa graiasca, ?i care va grai în numele altor dumnezei, pe un astfel de prooroc sa-l da?i mor?ii.
Deu 18:21  De vei zice în inima ta: Cum vom cunoa?te Cuvântul pe care nu-l graie?te Domnul?
Deu 18:22  Daca proorocul vorbe?te în numele Domnului, dar Cuvântul acela nu se va împlini ?i nu se va adeveri, atunci nu graie?te Domnul Cuvântul acela, ci-l graie?te proorocul din îndrazneala lui; nu te teme de el".
Deu 19:1  "Când Domnul Dumnezeul tau va pierde pe popoarele al caror pamânt ?i-l da ?ie Domnul Dumnezeul tau, ?i tu vei intra în mo?tenirea lor ?i te vei a?eza în ceta?ile lor ?i în casele lor,
Deu 19:2  Atunci sa-?i alegi trei ceta?i în ?ara ta pe care Domnul Dumnezeul tau ?i-o da în stapânire.
Deu 19:3  Sa-?i faci drum ?i sa împar?i în trei par?i tot pamântul tau pe care ?i-l da Domnul Dumnezeul tau de mo?tenire. Acelea vor sluji ca loc de scapare oricarui uciga?.
Deu 19:4  ?i iata care uciga? va fugi acolo ?i va trai: cel ce va ucide pe aproapele sau fara voie ?i fara sa-i fi fost vrajma? nici cu o zi, nici cu doua înainte;
Deu 19:5  Cel ce se va duce cu aproapele sau în padure sa taie lemne ?i, învârtind mâna sa cu toporul, ca sa taie un copac, va sari toporul din coada ?i va lovi pe aproapele ?i acela va muri, acesta sa fuga în una din aceste ceta?i ale tale, spre a scapa cu via?a,
Deu 19:6  Ca razbunatorul sângelui, în aprinderea inimii lui, sa nu se mânie pe uciga? ?i sa nu-l ajunga pe acesta, daca va fi lung drumul, ?i sa nu-l ucida, întrucât nu este vinovat de moarte, pentru ca nu i-a fost vrajma? nici cu o zi, nici cu doua înainte.
Deu 19:7  De aceea ?i-am dat eu porunca ?i ?i-am zis: Alege-?i trei ceta?i.
Deu 19:8  Iar când Domnul Dumnezeul tau va largi hotarele tale, dupa cum S-a jurat parin?ilor tai, ?i î?i va da tot pamântul pe care a fagaduit sa-l dea parin?ilor tai,
Deu 19:9  Daca te vei sili sa împline?ti toate poruncile acestea, pe care ?i le spun eu astazi, ?i vei iubi pe Domnul Dumnezeul tau ?i vei umbla în caile Lui în toate zilele, atunci la aceste trei ceta?i sa mai adaugi înca trei ceta?i,
Deu 19:10  Ca sa nu se verse sângele nevinovatului în pamântul tau pe care Domnul Dumnezeul tau ?i-l da de mo?tenire ?i sa nu ai asupra ta vina de sânge.
Deu 19:11  Iar daca cineva din ai tai va fi du?man aproapelui tau ?i-l va pândi ?i va sari la acela ?i-l va ucide ?i apoi va fugi în una din ceta?ile acestea,
Deu 19:12  Batrânii ceta?ii lui sa trimita ca sa-l ia de acolo ?i sa-l dea în mâinile razbunatorului sângelui, ca sa moara.
Deu 19:13  Sa nu-l cru?e pe unul ca acela ochiul tau. Spala pe Israel de sângele nevinovat ?i va fi bine.
Deu 19:14  Sa nu mu?i hotarul aproapelui tau, pe care l-au a?ezat stramo?ii mo?iei tale, care ?i s-a cuvenit în pamântul pe care Domnul Dumnezeul tau ?i-l da în stapânire.
Deu 19:15  Nu ajunge numai un martor pentru a vadi pe cineva de vreo vina sau de vreo nelegiuire sau de vreun pacat de care s-ar fi facut vinovat, ci orice pricina sa se dovedeasca prin spusa a doi sau trei martori.
Deu 19:16  De se va ridica asupra cuiva martor nedrept, învinuindu-l de nelegiuire,
Deu 19:17  Amândoi oamenii ace?tia între care este pricina sa se înfa?i?eze înaintea Domnului, la preot sau la judecatorii care vor fi în zilele acelea.
Deu 19:18  ?i judecatorii sa cerceteze bine ?i, daca martorul acela va fi martor mincinos ?i va fi marturisit strâmb asupra fratelui sau,
Deu 19:19  Sa-i face?i ceea ce voise sa faca el fratelui sau. ?i a?a sa stârpe?ti raul din mijlocul tau;
Deu 19:20  ?i vor auzi ?i ceilal?i ?i se vor teme ?i nu se vor apuca sa mai faca în mijlocul tau acest rau.
Deu 19:21  Sa nu-l cru?e ochiul tau, ci sa ceri suflet pentru suflet, ochi pentru ochi, dinte pentru dinte, mâna pentru mâna, picior pentru picior. Cu raul pe care îl va face cineva-aproapelui sau, cu acela trebuie sa i se plateasca".
Deu 20:1  "Când vei ie?i la razboi împotriva du?manului tau ?i vei vedea cai, caru?e ?i oameni mai mul?i decât ai tu, sa nu te temi de ei, caci cu tine este Domnul Dumnezeul tau, Care te-a scos din pamântul Egiptului.
Deu 20:2  Iar când ve?i fi aproape de lupta, sa vina preotul ?i sa vorbeasca poporului ?i sa-i spuna:
Deu 20:3  Asculta, Israele, voi astazi intra?i în lupta cu du?manii vo?tri; sa nu slabeasca inima voastra, nu va teme?i, nu va tulbura?i, nici nu va înspaimânta?i de ei.
Deu 20:4  Ca Domnul Dumnezeul vostru merge cu voi, ca sa se lupte pentru voi cu du?manii vo?tri ?i sa va izbaveasca.
Deu 20:5  Capeteniile o?tirii înca sa graiasca poporului ?i sa zica: Cel ce ?i-a zidit casa noua ?i n-a sfin?it-o, acela sa iasa ?i sa se întoarca la casa sa, ca sa nu moara în batalie ?i sa nu i-o sfin?easca altul.
Deu 20:6  Cel ce ?i-a sadit vie ?i n-a mâncat din ea, acela sa iasa ?i sa se întoarca la casa sa, ca sa nu moara în batalie ?i ca sa nu se foloseasca altul de ea.
Deu 20:7  Cel ce s-a logodit cu femeie ?i n-a luat-o, acela sa iasa ?i sa se întoarca la casa sa, ca sa nu moara în batalie ?i ca sa nu o ia altul.
Deu 20:8  Ba capeteniile o?tirii sa mai spuna poporului ?i sa zica: Cine este fricos ?i pu?in la suflet, acela sa iasa ?i sa se întoarca acasa, ca sa nu faca fricoase ?i inimile fra?ilor lui, cum este inima lui.
Deu 20:9  ?i dupa ce capeteniile o?tirii vor ispravi de spus poporului toate acestea, atunci sa se puna capeteniile de razboi ca pova?uitori ai poporului.
Deu 20:10  Când te vei apropia de cetate ca s-o cuprinzi, fa-i îndemnare de pace.
Deu 20:11  De se va învoi sa primeasca pacea cu tine ?i-?i va deschide por?ile, atunci tot poporul ce se va gasi în ea î?i va plati bir ?i-?i va sluji.
Deu 20:12  Iar de nu se va învoi cu tine la pace ?i va duce razboi cu tine, atunci s-o înconjuri.
Deu 20:13  ?i când Domnul Dumnezeul tau o va da în mâinile tale, sa love?ti cu ascu?i?ul sabiei pe to?i cei de parte barbateasca din ea.
Deu 20:14  Numai femeile ?i copiii, vitele ?i tot ce este în cetate, toata prada ei sa o iei pentru tine ?i sa te folose?ti de prada vrajma?ilor tai, pe care ?i i-a dat Domnul Dumnezeul tau în mâna.
Deu 20:15  A?a sa faci cu toate ceta?ile care sunt foarte departe de tine ?i care nu sunt din ceta?ile popoarelor acestora.
Deu 20:16  Iar în ceta?ile popoarelor acestora pe care Domnul Dumnezeul tau ?i le da în stapânire, sa nu la?i în via?a nici un suflet;
Deu 20:17  Ci sa-i dai blestemului: pe Hetei ?i pe Amorei, pe Canaanei ?i Ferezei, pe Hevei, pe Iebusei ?i pe Gherghesei, precum ?i-a poruncit Domnul Dumnezeul tau,
Deu 20:18  Ca sa nu va înve?e aceia sa face?i acelea?i urâciuni pe care le-au facut ei pentru dumnezeii lor ?i ca sa nu gre?i?i înaintea Domnului Dumnezeului vostru.
Deu 20:19  De ve?i ?ine multa vreme înconjurata vreo cetate, ca s-o cuprinzi ?i s-o iei, sa nu strici pomii ei cu securea, ci sa te hrane?ti din ei ?i sa nu-i dobori la pamânt. Copacul de pe câmp este el oare om ca sa se ascunda de tine dupa întaritura?
Deu 20:20  Iar copacii pe care-i ?tii ca nu-?i aduc nimic de hrana po?i sa-i strici ?i sa-i tai, ca sa-?i faci întarituri împotriva ceta?ii care poarta cu tine razboi, pâna o vei supune".
Deu 21:1  "Daca în pamântul pe care ?i-l da în stapânire Domnul Dumnezeul tau se va gasi om ucis, zacând în câmp, ?i nu se va ?ti cine l-a ucis,
Deu 21:2  Sa iasa batrânii tai ?i judecatorii tai ?i sa masoare ce departare este de la cel ucis pâna la ora?ele dimprejur.
Deu 21:3  ?i batrânii ceta?ii aceleia, care va fi mai aproape de cel ucis, sa ia o juninca ce n-a fost pusa la munca ?i n-a purtat jug,
Deu 21:4  ?i batrânii ceta?ii aceleia sa duca aceasta juninca la apa curgatoare, într-un loc care n-a fost arat, nici semanat, ?i sa junghie juninca acolo în apa cea curgatoare.
Deu 21:5  Apoi sa vina preo?ii, fiii levi?ilor, ca pe ei i-a ales Domnul Dumnezeul tau sa-I slujeasca ?i sa binecuvânteze în numele Lui ?i dupa Cuvântul lor se hotara?te orice lucru îndoielnic ?i toata vatamarea pricinuita.
Deu 21:6  ?i to?i batrânii ceta?ii aceleia, câre sunt mai aproape de cel ucis, sa-?i spele mâinile deasupra capului junincii celei junghiate în râu
Deu 21:7  ?i sa graiasca ?i sa spuna: "Mâinile noastre n-au varsat sângele acesta ?i ochii no?tri n-au vazut;
Deu 21:8  Iarta pe poporul Tau Israel, pe care Tu, Doamne, l-ai rascumparat din pamântul Egiptului ?i nu lasa poporului Tau Israel acest sânge nevinovat!" ?i se vor cura?i de sânge.
Deu 21:9  A?a sa speli tu sângele nevinovat de la tine, daca vrei sa faci cele bune ?i drepte înaintea ochilor Domnului Dumnezeului tau.
Deu 21:10  Când vei ie?i la razboi împotriva vrajma?ilor tai ?i Domnul Dumnezeul tau ?i-i va da în mâinile tale ?i-i vei lua în robie,
Deu 21:11  ?i vei vedea printre robi femeie frumoasa la chip ?i o vei iubi ?i vei vrea s-o iei de so?ie,
Deu 21:12  S-o aduci în casa ta, sa-?i tunda capul sau, sa-?i taie unghiile,
Deu 21:13  Sa-?i dezbrace de pe ea haina sa de robie, sa locuiasca în casa ta ?i sa-?i plânga pe tatal sau ?i pe mama sa timp de o luna; iar dupa aceea vei intra la ea, ca sa fii barbatul ei ?i ea sa-?i fie femeie.
Deu 21:14  Iar daca ea în urma nu-?i va mai placea, sa-i dai drumul sa se duca unde va vrea, dar sa n-o vinzi pe argint ?i sa n-o prefaci în roaba, pentru ca ai umilit-o.
Deu 21:15  De va avea cineva doua femei, una iubita ?i una neiubita ?i atât cea iubita cît ?i cea neiubita îi vor na?te copii ?i întâiul nascut va fi al celei neiubite,
Deu 21:16  Acela, la împar?irea averii sale între fiii sai, nu poate sa dea fiului femeii iubite întâietate înaintea fiului întâi-nascut din cea neiubita,
Deu 21:17  Ci sa cunoasca de întâi-nascut pe fiul celei neiubite ?i sa-i dea acestuia parte îndoita din toate câte va avea, ca acesta este pârga puterii lui ?i al lui este dreptul de întâi-nascut.
Deu 21:18  De va avea cineva fecior rau ?i nesupus, care nu asculta de vorba tatalui sau ?i de vorba mamei sale ?i ace?tia l-au pedepsit, dar el tot nu-i asculta,
Deu 21:19  Sa-l ia tatal lui ?i mama lui ?i sa-l duca la batrânii ceta?ii lor ?i la poarta acelei ceta?i ?i catre preo?ii ceta?ii lor sa zica:
Deu 21:20  Acest fiu al nostru este rau ?i neascultator, nu asculta de Cuvântul nostru ?i este lacom ?i be?iv".
Deu 21:21  Atunci to?i oamenii ceta?ii lui sa-l ucida cu pietre ?i sa-l omoare. ?i a?a sa stârpe?ti raul din mijlocul tau ?i to?i Israeli?ii vor auzi ?i se vor teme.
Deu 21:22  De se va gasi la cineva vinova?ie vrednica de moarte ?i va fi omorât, spânzurat de copac,
Deu 21:23  Trupul lui sa nu ramâna peste noapte spânzurat de copac, ci sa-l îngropi tot în ziua aceea, caci blestemat este înaintea Domnului tot cel spânzurat pe lemn ?i sa nu spurci pamântul tau pe care Domnul Dumnezeul tau ?i-l da mo?tenire".
Deu 22:1  "Când vei vedea boul fratelui tau sau oaia lui ratacite pe câmp; sa nu treci pe lânga ele, ci sa le întorci fratelui tau.
Deu 22:2  Dar daca fratele tau nu va fi aproape de tine sau nu-l cuno?ti, sa le duci la casa ta ?i sa ?ada la tine pâna le va cauta fratele tau ?i atunci sa i le dai.
Deu 22:3  A?a sa faci ?i cu asinul lui, a?a sa faci ?i cu haina lui, a?a sa faci ?i cu orice lucru pierdut al fratelui tau pe care el îl va pierde ?i tu îl vei gasi; de la aceasta nu te po?i da la o parte.
Deu 22:4  Când vei vedea asinul fratelui tau sau boul lui cazu?i în drum, sa nu-i la?i, ci sa-i ridici împreuna cu el.
Deu 22:5  Femeia sa nu poarte ve?minte barbate?ti, nici barbatul sa nu îmbrace haine femeie?ti, ca tot cel ce face aceasta, urâciune este înaintea Domnului Dumnezeului tau.
Deu 22:6  Daca în cale, în vreun copac sau pe pamânt, vei gasi cuib de pasare cu pui sau cu oua ?i mama lor va fi ?ezând pe pui sau pe oua, sa nu iei mama împreuna cu puii;
Deu 22:7  Mamei da-i drumul, iar puii ia-i pentru tine ca sa-?i fie bine ?i sa se înmul?easca zilele tale.
Deu 22:8  De vei zidi casa noua, sa faci aparatoare pe marginea acoperi?ului tau, ca sa nu aduci sânge asupra casei tale când va cadea cineva de pe ea.
Deu 22:9  Sa nu semeni via ta cu doua feluri de semin?e, ca sa nu-?i faci blestemata strângerea semin?elor, pe care tu le semeni împreuna cu roadele viei tale.
Deu 22:10  Sa nu ari cu un bou ?i cu un asin.
Deu 22:11  Sa nu te îmbraci cu haina facuta din doua feluri de fire: de lâna ?i de in.
Deu 22:12  Fa-?i ciucuri în cele patru col?uri ale mantiei tale cu care te acoperi.
Deu 22:13  De î?i va lua cineva femeie ?i va intra la dânsa,
Deu 22:14  Iar apoi o va urî ?i va ridica asupra ei învinuiri de lucruri urâte, va împra?tia zvon rau despre ea ?i va zice: Am luat femeia aceasta ?i am intrat la ea ?i n-am gasit la ea feciorie,
Deu 22:15  Atunci tatal fetei ?i mama ei sa ia ?i sa duca semnele fecioriei fetei la batrânii ceta?ii, în poarta;
Deu 22:16  ?i tatal fetei sa zica batrânilor: Am dat pe fiica mea de femeie acestui om ?i acum el a urât-o,
Deu 22:17  ?i iata ridica asupra ei învinuiri de lucruri urâte, zicând: N-am gasit feciorie la fiica ta; dar iata semnele fecioriei fiicei mele. ?i sa întinda haina înaintea batrânilor ceta?ii.
Deu 22:18  Atunci batrânii acelei ceta?i sa ia pe barbat ?i sa-l pedepseasca;
Deu 22:19  Sa puna asupra lui gloaba de o suta de sicli de argint ?i sa-i dea tatalui fetei, pentru ca a stârnit zvonuri rele despre o fata israelita; ea însa sa-i ramâna femeie ?i el sa nu se poata despar?i de ea toata via?a lui.
Deu 22:20  Iar daca cele spuse vor fi adevarate ?i nu se va gasi feciorie la fata,
Deu 22:21  Atunci fata sa fie adusa la u?a casei tatalui ei ?i locuitorii ceta?ii ei sa o ucida cu pietre ?i sa o omoare, pentru ca a facut lucru de ru?ine în Israel, desfrânându-se în casa tatalui sau. ?i a?a sa stârpe?ti raul din mijlocul tau.
Deu 22:22  De se va gasi cineva dormind cu femeie maritata, pe amândoi sa-i da?i mor?ii: ?i barbatul, care a dormit cu femeia ?i femeia. ?i a?a sa stârpe?ti raul din Israel.
Deu 22:23  De va fi vreo fata tânara, logodita cu barbat ?i cineva o va întâlni în cetate ?i se va culca cu dânsa,
Deu 22:24  Sa-i aduce?i pe amândoi la poarta ceta?ii aceleia ?i sa-i ucide?i cu pietre: pe fata pentru ca n-a ?ipat în cetate, iar pe barbat pentru ca a necinstit pe femeia aproapelui sau. ?i a?a sa stârpe?ti raul din mijlocul tau.
Deu 22:25  Daca vreun barbat va întâlni la câmp o fata logodita ?i, prinzând-o, se va culca cu ea, sa-l ucide?i numai pe barbatul care s-a culcat cu ea;
Deu 22:26  Iar fetei sa nu-i faci nimic. Asupra fetei nu este vina de moarte, caci aceasta este tot una ca ?i cum cineva s-ar ridica asupra aproapelui sau ?i l-ar omorî;
Deu 22:27  Pentru ca el a întâlnit-o în câmp ?i, de?i fata logodita va fi strigat, n-a avut cine s-o scape.
Deu 22:28  De se va întâlni cineva cu o fata nelogodita ?i o va prinde ?i se va culca cu ea ?i vor fi prin?i,
Deu 22:29  Atunci cel ce s-a culcat cu ea sa dea tatalui fetei cincizeci de sicli de argint, iar ea sa-i fie nevasta, pentru ca a necinstit-o; toata via?a lui sa nu se poata despar?i de ea.
Deu 22:30  Nimeni sa nu ia de so?ie pe femeia tatalui sau ?i sa ridice marginea hainei tatalui sau".
Deu 23:1  "Scopitul ?i famenul sa nu intre în ob?tea Domnului.
Deu 23:2  Fiul femeii desfrânate sa nu intre în ob?tea Domnului.
Deu 23:3  Amonitul ?i Moabitul sa nu intre în ob?tea Domnului; nici al zecelea neam al lor în veci sa nu intre în ob?tea Domnului,
Deu 23:4  Pentru ca nu v-au întâmpinat cu pâine ?i cu apa în cale, când venea?i din Egipt, ?i pentru ca ei au platit împotriva ta pe Valaam, fiul lui Beor, din Petorul Mesopotamiei, ca sa va blesteme.
Deu 23:5  Dar Domnul Dumnezeul tau n-a voit sa asculte pe Valaam ?i a prefacut Domnul Dumnezeul tau blestemul lui în binecuvântare pentru tine, pentru ca Domnul Dumnezeul tau te iube?te.
Deu 23:6  Sa nu le dore?ti pace ?i fericire în toate zilele tale, în veci.
Deu 23:7  Sa nu-?i fie scârba de edomit caci acesta î?i este frate. Sa nu-?i fie scârba de egiptean, ca ai fost strain în pamântul lui.
Deu 23:8  Copiii ce se vor na?te acestora în al treilea neam pot intra în ob?tea Domnului.
Deu 23:9  Când vei ie?i cu razboi asupra du?manilor tai, fere?te-te de tot ce este rau.
Deu 23:10  De va fi careva din ai tai necurat de ceea ce i s-a întâmplat noaptea, acela sa iasa afara din tabara ?i sa nu intre în tabara,
Deu 23:11  Ci, dupa ce se va face seara, sa-?i spele trupul cu apa ?i dupa asfin?itul soarelui sa intre în tabara.
Deu 23:12  Sa ai afara din tabara loc ?i acolo sa ie?i afara.
Deu 23:13  Afara de uneltele tale, sa mai ai ?i o lopata ?i, când vei vrea sa ie?i cu scaunul afara din tabara, sa sapi o groapa ?i apoi sa îngropi cu ea necura?eniile tale;
Deu 23:14  Ca Domnul Dumnezeul tau umbla prin tabara ta, ca sa te izbaveasca ?i sa-?i dea pe vrajma?ii tai în mâinile tale; de aceea tabara ta sa fie sfânta, ca sa nu vada El la tine ceva de ru?ine ?i sa Se departeze de la tine.
Deu 23:15  Sa nu dai pe rob în mâinile stapânului sau, când acela va fugi de la Stapânul sau la tine;
Deu 23:16  Lasa-l sa traiasca la tine, lasa-l sa locuiasca în mijlocul vostru, în locul ce-?i va alege, în una din ceta?ile tale unde-i va place; dar sa nu-l strâmtorezi.
Deu 23:17  Sa nu fie desfrânata din fiicele lui Israel, nici desfrânat din fiii lui Israel.
Deu 23:18  Câ?tigul de la desfrânata ?i pre?ul de pe câine sa nu-l duci în casa Domnului Dumnezeului tau pentru împlinirea oricarei fagaduin?e, caci ?i unul ?i altul sunt urâciune înaintea Domnului Dumnezeului tau.
Deu 23:19  Sa nu dai cu camata fratelui tau nici argint, nici pâine, nici nimic din câte se pot da cu camata.
Deu 23:20  Celui de alt neam sa-i dai cu camata; iar fratelui tau sa nu-i dai cu camata, ca Domnul Dumnezeul tau sa te binecuvânteze întru toate câte se fac de mâinile tale în pamântul în care mergi ca sa-l iei în stapânire.
Deu 23:21  De vei da fagaduin?a Domnului Dumnezeului tau, sa nu întârzii a o împlini, caci Domnul Dumnezeul tau o va cere de la tine ?i pacat vei avea asupra ta.
Deu 23:22  Iar de n-ai dat fagaduin?a, nu va fi pacat asupra ta.
Deu 23:23  Ceea ce a ie?it din gura ta sa paze?ti ?i sa împline?ti fagaduin?a pe care tu ai facut-o de buna voie Domnului Dumnezeului tau, ?i de care ai grait cu gura ta.
Deu 23:24  De vei intra în via aproapelui tau, po?i sa manânci poama pâna ce te vei satura, dar în panerul tau sa nu pui.
Deu 23:25  ?i când vei intra în grâul aproapelui tau, rupe spice cu mâinile tale, iar secera sa n-o pui în ogorul aproapelui tau".
Deu 24:1  "De va lua cineva femeie ?i se va face barbat ei, dar ea nu va afla bunavoin?a în ochii lui, pentru ca va gasi el ceva neplacut la ea, ?i-i va scrie carte de despar?ire, i-o va da la mâna ?i o va slobozi din casa sa,
Deu 24:2  Iar ea va ie?i ?i, ducându-se, se va marita cu alt barbat.
Deu 24:3  Dar daca ?i acest din urma barbat o va urî ?i-i va scrie carte de despar?ire ?i i-o va da la mâna ?i o va slobozi din casa sa, sau va muri acest din urma barbat al ei, care a luat-o de so?ie,
Deu 24:4  Barbatul ei cel dintâi care a lasat-o nu o poate lua iar de so?ie, dupa ce a fost întinata, ca aceasta este urâciune înaintea Domnului Dumnezeului tau; sa nu întinezi pamântul, pe care Domnul Dumnezeul tau ti-l da de mo?tenire.
Deu 24:5  Daca cineva ?i-a luat femeie de curând, sa nu se duca la razboi ?i sa nu i se puna nici o sarcina; lasa-l sa ramâna slobod la casa sa timp de un an, sa veseleasca pe femeia sa pe care a luat-o.
Deu 24:6  Nimeni sa nu ia zalog piatra de deasupra sau cea de dedesubt a râ?ni?ei, ca ar însemna ca iei zalog însa?i via?a cuiva.
Deu 24:7  De se va afla ca cineva a furat pe vreunul din fra?ii sai, din fiii lui Israel ?i, facându-l rob, l-a vândut, sa fie omorât talharul acela ?i sa stârpe?ti raul din mijlocul tau.
Deu 24:8  Fii atent cu tine la boala leprei, paze?te foarte bine toata legea care te înva?a preo?ii cei din levi?i, ?i împlini?i cu sfin?enie ceea ce le-am poruncit eu.
Deu 24:9  Adu-?i aminte ce a facut Domnul Dumnezeul tau Mariamei pe drum, când venea?i voi din Egipt.
Deu 24:10  De vei da aproapelui tau ceva împrumut, sa nu te duci în casa lui ca sa iei zalog de la el;
Deu 24:11  Stai în uli?a, ?i acela caruia i-ai dat împrumut sa-?i scoata zalog în uli?a.
Deu 24:12  Iar daca acela va fi om sarac, sa nu te culci, având la tine zalogului;
Deu 24:13  Ci sa-i întorci zalogul la asfin?itul soarelui, ca sa se culce în haina sa ?i sa te binecuvânteze; aceasta ?i se va socoti ca o fapta buna înaintea Domnului Dumnezeului tau.
Deu 24:14  Sa nu nedrepta?e?ti pe cel ce munce?te cu plata, pe sarac ?i pe cel lipsit dintre fra?ii tai sau dintre strainii care sunt în pamântul tau ?i în ceta?ile tale.
Deu 24:15  Ci sa dai plata în aceea?i zi ?i sa nu apuna soarele înainte de aceasta, ca el este sarac ?i sufletul lui a?teapta aceasta plata; ca sa nu strige el asupra ta catre Domnul ?i sa nu ai pacat.
Deu 24:16  Parin?ii sa nu fie pedepsi?i cu moartea pentru vina copiilor ?i nici copiii sa nu fie pedepsi?i cu moartea pentru vina parin?ilor; ci fiecare sa fie pedepsit cu moartea pentru pacatul sau.
Deu 24:17  Sa nu judeci strâmb pe strain, pe orfan ?i pe vaduva, ?i vaduvei sa nu-i iei haina zalog.
Deu 24:18  Adu-?i aminte ca ?i tu ai fost rob în Egipt ?i Domnul Dumnezeul tau te-a izbavit de acolo, de aceea î?i ?i poruncesc eu sa faci aceasta.
Deu 24:19  Când vei secera holda în ?arina ta ?i vei uita vreun snop în ?arina, sa nu te întorci sa-l iei, ci lasa-l sa ramâna al strainului, saracului, orfanului ?i vaduvei, ca Domnul Dumnezeul tau sa te binecuvânteze întru toate lucrurile mâinilor tale.
Deu 24:20  Când vei scutura maslinul tau, sa nu te întorci sa culegi rama?i?ele, ci lasa-le strainului, orfanului ?i vaduvei".
Deu 24:21  Când vei strânge roadele viei tale, sa nu aduni rama?i?ele, ci lasa-le strainului, orfanului ?i vaduvei.
Deu 24:22  Adu-?i aminte ca ?i tu ai fost rob în pamântul Egiptului ?i de aceea î?i poruncesc eu sa faci aceasta".
Deu 25:1  "De va fi neîn?elegere între oameni, sa fie adu?i la judecata ?i sa fie judeca?i; celui drept sa i se dea dreptate, iar cel vinovat sa se osândeasca.
Deu 25:2  Daca celui vinovat i se va cuveni bataie, sa porunceasca judecatorii sa fie pus jos ?i sa fie batut înaintea lor, dupa masura vinova?iei lui.
Deu 25:3  I se pot da pâna la patruzeci de lovituri, iar nu mai mult, ca nu cumva fratele tau, din pricina multelor lovituri, sa fie schilodit înaintea ochilor tai.
Deu 25:4  Sa nu legi gura boului care treiera.
Deu 25:5  De vor trai fra?ii împreuna ?i unul din ei va muri, fara sa aiba fiu, femeia celui mort sa nu se marite în alta parte dupa strain, ci cumnatul ei sa intre la ea, sa ?i-o ia so?ie ?i sa traiasca cu ea.
Deu 25:6  Întâiul nascut pe care-l va na?te ea sa poarte numele fratelui lui cel mort, pentru ca numele acestuia sa nu se ?tearga din Israel.
Deu 25:7  Iar daca el nu va voi sa ia pe cumnata sa, aceasta sa se duca la poarta ceta?ii, înaintea batrânilor ?i sa zica: Cumnatul meu nu vrea sa pastreze numele fratelui sau în Israel, nevrând sa se casatoreasca cu mine.
Deu 25:8  Iar batrânii ceta?ii lui sa-l cheme ?i sa-l sfatuiasca ?i, daca el se va ridica ?i va zice: Nu vreau s-o iau,
Deu 25:9  Atunci cumnata lui sa se duca la el acolo, în fa?a batrânilor, sa-i dezlege sandaua din piciorul lui, sa-l scuipe în obraz ?i sa zica: A?a se cuvine omului care nu vrea sa zideasca fratelui sau casa în Israel.
Deu 25:10  ?i casa acestuia se va numi în Israel casa descul?ului.
Deu 25:11  De se vor bate între dân?ii ni?te barba?i, ?i femeia unuia din ei se va duce ca sa scoata pe barbatul sau din mâna celui ce-l bate ?i, întinzându-?i ea mâna, va apuca pe acesta de par?ile lui ru?inoase,
Deu 25:12  Sa i se taie mâna ei ?i sa nu o cru?e ochiul tau.
Deu 25:13  Sa nu ai în sacule?ul tau doua feluri de greuta?i pentru cumpana: unele mai mari ?i altele mai mici.
Deu 25:14  În casa ta sa nu ai doua feluri de efa: una mai mare ?i alta mai mica.
Deu 25:15  Greuta?ile pentru cumpana ta sa fie adevarate ?i drepte ?i efa ta sa fie adevarata ?i dreapta, ca sa se înmul?easca zilele tale pe pamântul pe care ?i-l da Domnul Dumnezeul tau de mo?tenire;
Deu 25:16  Ca urâciune este înaintea Domnului Dumnezeului tau tot cel ce face strâmbatate.
Deu 25:17  Adu-?i aminte cum s-a purtat cu tine Amalec pe drum, când venea?i voi din Egipt,
Deu 25:18  ?i cum te-a întâmpinat el în cale ?i a ucis în urma ta pe to?i cei slabi?i, când erai ostenit ?i obosit, netemându-se de Dumnezeu.
Deu 25:19  A?adar, când Domnul Dumnezeul tau te va lini?ti de to?i vrajma?ii tai, din toate par?ile, în pamântul pe care Domnul Dumnezeul tau ?i-l da mo?tenire ca sa-l stapâne?ti, sa ?tergi pomenirea lui Amalec de sub cer. Nu uita aceasta".
Deu 26:1  Când vei intra în pamântul pe care Domnul Dumnezeul tau ?i-l da mo?tenire, ?i-l vei lua în stapânire ?i te vei a?eza pe el,
Deu 26:2  Sa iei pârga tuturor roadelor pamântului ce vei lua tu din pamântul tau pe care Domnul Dumnezeul tau ?i-l da, sa le pui în paner ?i sa te duci la locul acela pe care Domnul Dumnezeul tau îl va alege ca sa-i fie acolo numele Lui;
Deu 26:3  Sa mergi la preotul care va fi în zilele acelea ?i sa-i zici: Astazi marturisesc înaintea Domnului Dumnezeului tau ca am intrat în acel pamânt pentru care Domnul S-a jurat parin?ilor no?tri sa ni-l dea noua.
Deu 26:4  Iar când preotul va lua panerul din mâna ta ?i-l va pune înaintea jertfelnicului Domnului Dumnezeului tau,
Deu 26:5  Tu sa raspunzi ?i sa zici înaintea Domnului Dumnezeului tau: Tatal meu a fost un arameian pribeag, s-a dus în Egipt, s-a a?ezat acolo cu pu?inii oameni ai sai ?i acolo s-a ridicat din el popor mare, puternic ?i mult la numar.
Deu 26:6  Dar Egiptenii s-au purtat rau cu noi ?i ne-au strâmtorat ?i ne-au silit la munci grele.
Deu 26:7  De aceea am strigat noi catre Domnul Dumnezeul parin?ilor no?tri; iar Domnul a auzit strigarea noastra, a vazut nenorocirea noastra, muncile noastre ?i împilarea noastra.
Deu 26:8  ?i ne-a scos Domnul din Egipt, singur cu puterea Lui cea mare, cu mâna tare ?i cu bra? înalt, cu înfrico?are mare, cu semne ?i minuni;
Deu 26:9  ?i ne-a adus în locul acesta ?i ne-a dat pamântul acesta, ?ara în care curge lapte ?i miere.
Deu 26:10  Acum iata am adus pârga roadelor din pamântul pe care Tu, Doamne, mi l-ai dat din pamântul unde curge lapte ?i miere. Sa pui  roadele acelea înaintea Domnului Dumnezeului tau, sa te închini înaintea Domnului Dumnezeului tau
Deu 26:11  ?i sa te vesele?ti de toate bunata?ile ce Domnul Dumnezeul tau ?i-a dat ?ie ?i casei tale; dar sa se veseleasca ?i levitul ?i strainul care va fi la tine.
Deu 26:12  Iar când vei osebi toate zeciuielile din roadele pamântului tau în anul al treilea, care este anul zeciuielii, ?i le vei da levitului, strainului, orfanului ?i vaduvei, ca sa manânce ace?tia în loca?urile tale ?i sa se sature,
Deu 26:13  Atunci sa zici înaintea Domnului Dumnezeului tau: Am osebit din casa mea cele sfinte ?i le-am dat levitului, strainului, orfanului ?i vaduvei, dupa toate poruncile Tale pe care mi le-ai dat Tu mie; n-am calcat poruncile Tale, nici nu le-am uitat;
Deu 26:14  N-am mâncat din ele în întristarea mea, nici nu le-am osebit în necura?enie, nici n-am dat din ele cu prilejul vreunui mort, ci m-am supus glasului Domnului Dumnezeului meu ?i am împlinit tot ce mi-ai poruncit Tu.
Deu 26:15  Cauta deci din loca?ul Tau cel sfânt, din ceruri, ?i binecuvinteaza pe poporul Tau, Israel, ?i pamântul pe care ni l-ai dat noua, dupa cum Te-ai jurat parin?ilor no?tri, ca sa ne dai pamântul în care curge lapte ?i miere.
Deu 26:16  În ziua aceasta î?i porunce?te Domnul Dumnezeul tau sa împline?ti toate hotarârile ?i rânduielile acestea; sa le paze?ti ?i sa le împline?ti din toata inima ta ?i din tot sufletul tau.
Deu 26:17  Astazi ai marturisit tu Domnului ca El va fi Dumnezeul tau ?i ca tu vei umbla în caile Lui ?i vei pazi hotarârile Lui, poruncile Lui ?i legile Lui ?i vei asculta glasul Lui.
Deu 26:18  ?i Domnul ?i-a fagaduit astazi ca tu vei fi poporul Lui adevarat, precum ?i-a grait El, de vei pazi toate poruncile Lui;
Deu 26:19  ?i te va pune cu cinstea ?i cu marirea ?i cu faima mai presus de toate popoarele pe care le-a facut El ?i vei fi poporul sfânt al Domnului Dumnezeului tau, precum ?i-a grait El".
Deu 27:1  Moise, împreuna cu batrânii fiilor lui Israel, a poruncit poporului ?i a zis: "Toate poruncile pe care vi le poruncesc eu astazi sa le împlini?i.
Deu 27:2  Când vei trece peste Iordan în pamântul pe care Domnul Dumnezeul tau ?i-l da, sa-?i a?ezi pietre mari ?i sa le varuie?ti cu var;
Deu 27:3  ?i pe pietrele acelea sa scrii toate cuvintele acestei legi când vei trece Iordanul, ca sa intri în pamântul pe care Domnul Dumnezeul tau ?i-l da, în pamântul unde curge lapte ?i miere, dupa cum ?i-a grait Domnul Dumnezeul parin?ilor tai.
Deu 27:4  Dupa ce ve?i fi trecut Iordanul, sa pune?i pietrele acelea, precum va poruncesc eu astazi, pe muntele Ebal ?i sa le varui?i cu var.
Deu 27:5  Sa zide?ti acolo jertfelnic Domnului Dumnezeului tau, jertfelnic facut din pietre, fara sa pui asupra lor fierul.
Deu 27:6  Jertfelnicul Domnului Dumnezeului tau însa sa-l faci din pietre întregi ?i sa aduci pe el Domnului Dumnezeului tau ardere de tot.
Deu 27:7  Sa mai aduci jertfe de pace; sa manânci ?i sa te saturi acolo ?i sa te vesele?ti înaintea Domnului Dumnezeului tau.
Deu 27:8  Dar sa scrii pe pietrele acelea cuvintele legii acesteia foarte lamurit".
Deu 27:9  Moise cu preo?ii cei din levi?i a grait la tot Israelul ?i a zis: "Ia aminte ?i asculta, Israele: Astazi te-ai facut poporul Domnului Dumnezeului tau.
Deu 27:10  Asculta dar glasul Domnului Dumnezeului tau ?i pline?te toate poruncile Lui ?i hotarârile Lui pe care ?i le spun eu astazi".
Deu 27:11  În ziua aceea a mai poruncit Moise poporului ?i a zis:
Deu 27:12  "Dupa ce ve?i trece Iordanul, semin?iile: Simeon, Levi, Iuda, Isahar, Iosif ?i Veniamin sa stea pe muntele Garizim ?i sa binecuvânteze poporul;
Deu 27:13  Iar semin?iile: Ruben, Gad, A?er, Zabulon, Dan ?i Neftali sa stea pe muntele Ebal, ca sa rosteasca blestemul.
Deu 27:14  Atunci levi?ii sa strige ?i sa zica cu glas tare tuturor Israeli?ilor:
Deu 27:15  Blestemat sa fie cel ce va face idol cioplit sau turnat, lucru de mâna de me?ter ?i urâciune înaintea Domnului ?i-l va pune la loc tainic! La aceasta tot poporul sa raspunda ?i sa zica: Amin!
Deu 27:16  Blestemat sa fie cel ce va grai de rau pe tatal sau sau pe mama sa! ?i tot poporul sa zica: Amin!
Deu 27:17  Blestemat sa fie cel ce va muta hotarul aproapelui sau! ?i tot poporul sa zica: Amin!
Deu 27:18  Blestemat sa fie cel ce va abate pe orb din drum! ?i tot poporul sa zica: Amin!
Deu 27:19  Blestemat sa fie cel ce va judeca strâmb pe strain, pe orfan ?i pe vaduva! ?i tot poporul sa zica: Amin!
Deu 27:20  Blestemat sa fie cel ce se va culca cu femeia tatalui sau, ca a ridicat poala hainei tatalui sau! ?i tot poporul sa zica: Amin!
Deu 27:21  Blestemat sa fie cel ce se va culca cu vreun dobitoc! ?i tot poporul sa zica: Amin!
Deu 27:22  Blestemat sa fie cel ce se va culca cu sora sa, fiica tatalui sau, sau fiica mamei sale! ?i tot poporul sa zica: Amin!
Deu 27:23  Blestemat sa fie cel ce se va culca cu soacra sa! ?i tot poporul sa zica: Amin! Blestemat sa fie cel ce se va culca cu sora femeii sale! ?i tot poporul sa zica: Amin!
Deu 27:24  Blestemat sa fie cel ce va ucide în ascuns pe aproapele sau! ?i tot poporul sa zica: Amin!
Deu 27:25  Blestemat sa fie cel ce va lua mita, ca sa ucida suflet ?i sa verse sânge nevinovat! ?i tot poporul sa zica: Amin!
Deu 27:26  Blestemat sa fie tot omul care nu va plini toate cuvintele legii acesteia ?i nu va urma dupa ea! ?i tot poporul sa zica: Amin!"
Deu 28:1  "Daca tu, dupa ce vei trece peste Iordan, în pamântul pe care Domnul Dumnezeul tau îi-l va da, vei asculta glasul Domnului Dumnezeului tau ?i vei împlini cu bagare de seama toate poruncile Lui pe care ?i le dau astazi, atunci Domnul Dumnezeul tau te va pune mai presus de toate popoarele pamântului.
Deu 28:2  De vei asculta glasul Domnului Dumnezeului tau, vor veni asupra ta toate binecuvântarile acestea ?i se vor împlini asupra ta:
Deu 28:3  Binecuvântat sa fii în cetate ?i binecuvântat sa fii în ?arina;
Deu 28:4  Binecuvântat sa fie rodul pântecelui tau, rodul pamântului tau, rodul dobitoacelor tale;
Deu 28:5  Binecuvântate sa fie hambarele tale ?i camarile tale;
Deu 28:6  Binecuvântat sa fii la intrarea ta în casa ?i binecuvântat sa fii la ie?irea ta din casa;
Deu 28:7  Sa bata Domnul înaintea ta pe vrajma?ii tai cei ce se vor ridica asupra ta; pe o cale sa vina asupra ta ?i pe ?apte cai sa fuga de tine;
Deu 28:8  Sa-îi trimita Domnul binecuvântare peste grânarele tale ?i peste tot lucrul mâinilor tale ?i sa te binecuvânteze în pamântul pe care Domnul Dumnezeul tau ?i-l da;
Deu 28:9  Sa faca Domnul Dumnezeu din tine popor sfânt al Sau, precum ?i S-a jurat El ?ie ?i parin?ilor tai, daca vei asculta poruncile Domnului Dumnezeului tau ?i vei umbla în caile Lui;
Deu 28:10  Vor vedea toate popoarele pamântului ca por?i numele Domnului Dumnezeului tau ?i se vor teme de tine;
Deu 28:11  Domnul Dumnezeul tau î?i va da bel?ug în toate bunata?ile, în rodul pântecelui tau, în rodul dobitoacelor tale ?i în rodul ogoarelor tale din pamântul pe care Domnul S-a jurat parin?ilor tai sa ?i-l dea;
Deu 28:12  Domnul î?i va deschide comoara Sa cea buna, cerul, ca sa dea ploaie pamântului tau la vreme ?i ca sa binecuvânteze toate lucrurile mâinilor tale; ?i vei da împrumut multor popoare, iar tu nu vei lua împrumut; vei domni peste multe popoare, iar acelea nu vor domni peste tine;
Deu 28:13  Domnul Dumnezeul tau te va pune cap iar nu coada ?i vei fi numai sus, iar jos nu vei fi, daca te vei supune poruncilor Domnului Dumnezeului tau, care ?i le spun eu astazi sa le ?ii ?i sa le împline?ti
Deu 28:14  ?i daca nu te vei abate de la toate poruncile care ?i le poruncesc eu astazi nici la dreapta nici la stânga, ca sa merge?i dupa al?i dumnezei sa le sluji?i.
Deu 28:15  Iar daca nu vei asculta glasul Domnului Dumnezeului tau ?i nu te vei sili sa împline?ti toate poruncile ?i hotarârile Lui pe care îi le poruncesc eu astazi, sa vina asupra ta toate blestemele acestea ?i sa te ajunga:
Deu 28:16  Blestemat sa fii tu în cetate ?i blestemat sa fii tu în ?arina;
Deu 28:17  Blestemate sa fie grânarele tale ?i camarile tale;
Deu 28:18  Blestemat sa fie rodul pântecelui tau ?i rodul pamântului tau, rodul vacilor tale ?i rodul oilor tale;
Deu 28:19  Blestemat sa fii tu la intrarea ta în casa ?i blestemat la ie?irea ta din casa;
Deu 28:20  Sa trimita Domnul asupra ta blestem, tulburare ?i necaz în tot lucrul mâinilor tale pe care te vei apuca sa-l faci, pâna vei fi stârpit ?i pâna vei pieri curând, pentru faptele tale rele ?i pentru ca M-ai parasit;
Deu 28:21  Ba sa mai trimita Domnul asupra ta ciuma, pâna te va stârpi de pe pamântul în care mergi ca sa-l stapâne?ti;
Deu 28:22  Sa te bata Domnul cu oftica, cu lingoare, cu friguri, cu aprindere, cu seceta, cu vânt rau ?i cu rugina, ?i te vor urmari acestea pâna vei pieri;
Deu 28:23  Cerurile tale, care sunt deasupra capului tau, sa se faca arama ?i pamântul de sub tine fier;
Deu 28:24  În loc de ploaie, Domnul sa dea pamântului tau praf ?i pulbere, care sa cada din cer asupra ta pâna te va pierde ?i pâna vei fi prapadit;
Deu 28:25  Domnul te va da sa fii batut de vrajma?ii tai; pe un drum sa mergi asupra lor ?i pe ?apte drumuri sa fugi de ei ?i sa fii împra?tiat prin toate ?arile pamântului;
Deu 28:26  Trupurile tale sa fie hrana tuturor pasarilor cerului ?i fiarelor ?i nu va fi cine sa le alunge;
Deu 28:27  Te va lovi Domnul cu lepra Egiptului, cu trânji, cu râie ?i cu pecingine, de care sa nu te po?i vindeca;
Deu 28:28  Sa te bata Domnul cu nebunie, cu orbire ?i cu amor?irea inimii;
Deu 28:29  Pe dibuite sa mergi ziua în amiaza mare, cum umbla orbul pipaind pe întuneric, ?i sa te strâmtoreze ?i sa te ocarasca în toate zilele, ?i nimeni sa nu te apere;
Deu 28:30  Cu femeie sa te logode?ti ?i altul sa se culce cu ea; casa sa zide?ti, ?i sa nu traie?ti în ea; vie sa sade?ti, dar de ea sa nu te folose?ti;
Deu 28:31  Boul tau sa fie junghiat sub ochii tai ?i sa nu-l manânci tu; asinul sa ?i-l ia ?i sa nu ?i-l mai aduca; oile tale sa fie date vrajma?ilor tai ?i nimeni sa nu te izbaveasca;
Deu 28:32  Fiii tai ?i fiicele tale sa fie date la popor strain; ochii tai sa-i vada ?i sa se topeasca în toate zilele de mila lor, dar sa nu ai nici o putere în mâinile tale;
Deu 28:33  Roadele pamântului tau ?i toate ostenelile tale sa le manânce un popor pe care tu nu l-ai cunoscut, iar tu sa fii numai strâmtorat ?i chinuit în toate zilele;
Deu 28:34  Din pricina celor ce-?i vor vedea ochii tai, î?i vei ie?i din min?i;
Deu 28:35  Domnul te va lovi cu lepra rea peste genunchi ?i peste fluiere ?i din talpile picioarelor tale pâna în cre?tetul capului tau, de care nu te vei mai putea vindeca;
Deu 28:36  Te va duce Domnul pe tine ?i pe regele tau, pe care-l vei pune peste tine, la poporul pe care nu l-ai cunoscut nici tu, nici parin?ii tai, ?i acolo vei sluji altor dumnezei de lemn ?i de piatra;
Deu 28:37  ?i vei fi de spaima, de pomina ?i de râs la toate popoarele la care te va duce Domnul Dumnezeul tau;
Deu 28:38  Vei semana multa samân?a în ?arina, dar pu?ina vei culege, pentru ca o vor mânca lacustele;
Deu 28:39  Vii vei sadi ?i le vei lucra, dar nu le vei culege, nici nu vei bea vin, pentru ca le vor mânca viermii;
Deu 28:40  Maslini înca vei avea în toate ?inuturile tale, dar cu untdelemn nu te vei unge, pentru ca maslinele tale vor cadea;
Deu 28:41  Fii ?i fiice vei na?te, dar nu-i vei avea, pentru ca vor fi lua?i în robie;
Deu 28:42  To?i pomii tai ?i roadele pamântului le va strica rugina;
Deu 28:43  Strainul cel din mijlocul tau se va înal?a peste tine din ce în ce mai sus, iar tu te vei pogorî din ce în ce mai jos;
Deu 28:44  Acela î?i va da împrumut, iar tu nu-i vei da lui împrumut; acela va fi cap, iar tu vei fi coada.
Deu 28:45  Vor veni asupra ta toate blestemele acestea, te vor urmari ?i te vor ajunge. pâna vei fi stârpit, pentru ca n-ai ascultat glasul Domnului Dumnezeului tau ?i n-ai pazit poruncile Lui, nici hotarârile Lui pe care ?i le-a dat El.
Deu 28:46  ?i vor fi ele semn ?i pecete asupra ta ?i asupra semin?iei tale în veci.
Deu 28:47  Pentru ca tu n-ai slujit Domnului Dumnezeului tau cu veselia ?i cu bucuria inimii, când erai îmbel?ugat de toate.
Deu 28:48  Vei sluji vrajma?ului tau, pe care-l va trimite asupra ta Domnul Dumnezeul tau, când vei fi în foamete ?i în sete ?i în golatate ?i în tot felul de lipsa; acela va pune pe grumazul tau jug de fier ?i te va istovi.
Deu 28:49  Trimite-va Domnul asupra ta popor din departare, de la marginea pamântului; ca un vultur va veni poporul acela a carui limba tu nu o vei în?elege;
Deu 28:50  Popor crunt, care nu va da cinste batrânului ?i nu va cru?a pe cel tânar.
Deu 28:51  Va mânca acela rodul dobitoacelor tale ?i rodul pamântului tau, pâna te va nimici, caci nu-?i va lasa nici pâine, nici vin, nici untdelemn, nici rodul vitelor tale, nici rodul oilor tale, pâna te va pierde.
Deu 28:52  Te va strâmtora în toate ceta?ile tale, pâna ce va darâma, în tot pamântul tau, zidurile tale cele înalte ?i tari în care nadajduie?ti tu ?i te va împila în toate loca?urile tale, în tot pamântul tau pe care Domnul Dumnezeul tau ?i-l da.
Deu 28:53  ?i vei mânca tu rodul pântecelui tau, carnea fiilor ?i fiicelor tale, pe care ?i-i va fi dat Domnul Dumnezeul tau în timpul împresurarii ?i al strâmtorarii cu care te va strâmtora vrajma?ul tau.
Deu 28:54  Barbatul tau, cel rasfa?at ?i trait în alintare, va privi cu ochi nemilostivi la fratele sau; la so?ia de la sânul sau ?i la ceilal?i copii ai sai, care î?i vor ramâne.
Deu 28:55  ?i nu va da nici unuia din ei carne din copiii sai, pe care îi va mânca el, caci nu-i va mai ramâne nimic în vremea împresurarii cu care te va strâmtora vrajma?ul tau în toate ceta?ile tale.
Deu 28:56  Femeia ta, traita în bel?ug ?i rasfa?, care nu ?i-a pus piciorul sau în pamânt din pricina traiului alintat ?i îndestulat dinainte, va privi cu ochi nemilostivi pe barbatul de la sânul ei ?i la fiul sau ?i la fiica sa,
Deu 28:57  ?i nu le va da fatul, care a ie?it din coapsele sale ?i copiii, pe care i-a nascut, pentru ca ea, din pricina lipsei de toate, îi va mânca pe ascuns în timpul împresurarii ?i al strâmtorarii cu care te va strâmtora vrajma?ul tau în ceta?ile tale.
Deu 28:58  De nu te vei sili sa împline?ti toate cuvintele legii acesteia, care sunt scrise în cartea aceasta ?i nu te vei teme de acest nume slavit ?i înfrico?ator al Domnului Dumnezeului tau,
Deu 28:59  Atunci Domnul te va bate pe tine ?i pe urma?ii tai cu plagi nemaiauzite, cu plagi mari ?i nesfâr?ite ?i cu boli rele ?i necurmate;
Deu 28:60  Va aduce asupra ta toate plagile cele rele ale Egiptului, de care te-ai temut ?i se vor lipi acelea de tine.
Deu 28:61  Toata boala, toata plaga scrisa ?i toata cea nescrisa în cartea legii acesteia, o va aduce Domnul asupra ta, pâna vei fi stârpit.
Deu 28:62  Pu?ini din voi vor ramâne, de?i ve?i fi fost ca stelele cerului, pentru ca n-a?i ascultat glasul Domnului Dumnezeului vostru.
Deu 28:63  Cum s-a bucurat Domnul, când v-a facut bine ?i v-a înmul?it, tot a?a se va bucura Domnul când va va pierde ?i va va stârpi ?i ve?i fi arunca?i din pamântul în care intra?i ca sa-l stapâni?i.
Deu 28:64  Atunci te va împra?tia Domnul Dumnezeul tau prin toate popoarele ?i acolo vei sluji altor dumnezei, pe care nu i-ai cunoscut nici tu, nici parin?ii tai; vei sluji la lemne ?i la pietre.
Deu 28:65  Dar ?i între aceste popoare nu te vei lini?ti ?i nu vei avea loc de odihna pentru piciorul tau, ca Domnul î?i va da acolo inima tremuratoare, topirea ochilor ?i durere sufletului.
Deu 28:66  Via?a ta va fi mereu în primejdie înaintea ochilor tai; vei tremura ziua ?i noaptea ?i nu vei fi sigur de via?a ta.
Deu 28:67  De tremurul inimii tale, de care vei fi cuprins, ?i de cele ce vei vedea cu ochii tai, diminea?a vei zice: O, de ar veni seara! Iar seara vei zice: O, de ar veni ziua!
Deu 28:68  ?i te va întoarce Domnul în Egipt, în corabii pe calea aceea de care ?i-a zis: "Sa nu o mai vezi!"; ?i acolo va ve?i da spre vânzare vrajma?ilor vo?tri robi ?i roabe ?i nu va fi cine sa va cumpere".
Deu 29:1  Iata cuvintele legamântului ce a poruncit Domnul lui Moise sa încheie cu fiii lui Israel în pamântul Moabului, afara de legamântul pe care l-a încheiat Domnul cu ei în Horeb.
Deu 29:2  Atunci a chemat Moise pe to?i fiii lui Israel ?i le-a zis: "A?i vazut toate câte a facut Domnul înaintea ochilor vo?tri, în pamântul Egiptului, cu Faraon ?i cu toate slugile lui ?i cu tot pamântul lui,
Deu 29:3  Acele pedepse mari pe care le-au vazut ochii tai ?i acele semne ?i minuni, mâna cea tare ?i bra?ul cel înalt;
Deu 29:4  Dar pâna în ziua de astazi nu v-a dat Domnul Dumnezeu minte ca sa pricepe?i, ochi ca sa vede?i ?i urechi ca sa auzi?i.
Deu 29:5  Patruzeci de ani v-a purtat prin pustie ?i hainele de pe voi nu s-au învechit, nici încal?amintele voastre nu s-au stricat în picioarele voastre.
Deu 29:6  Pâine n-a?i mâncat, vin ?i sichera n-a?i baut, ca sa ?ti?i ca Eu sunt Domnul Dumnezeul vostru.
Deu 29:7  Când însa a?i ajuns la locul acesta, s-a ridicat împotriva voastra Sihon, regele He?bonului, ?i Og, regele Vasanului, ca sa se lupte cu noi.
Deu 29:8  Dar noi i-am batut ?i am cuprins ?ara lor ?i am dat-o mo?tenire semin?iilor lui Ruben ?i Gad ?i la jumatate din semin?ia lui Manase.
Deu 29:9  Pazi?i dar toate cuvintele a?ezamântului acestuia ?i le împlini?i, ca sa ave?i spor la toate câte ve?i face.
Deu 29:10  Voi cu to?ii va înfa?i?a?i astazi înaintea fe?ei Domnului Dumnezeului vostru: capeteniile semin?iilor voastre, batrânii vo?tri, judecatorii vo?tri, mai marii o?tirii voastre ?i to?i Israeli?ii,
Deu 29:11  Copiii vo?tri, femeile voastre ?i strainii tai care se afla în taberele tale, de la taietorul de lemne pâna la caratorul de apa.
Deu 29:12  Ca sa închei legamânt cu Domnul Dumnezeul tau ?i sa ai parte de legamântul facut prin juramânt pe care Domnul Dumnezeul tau l-a încheiat astazi cu tine,
Deu 29:13  Ca sa te faci astazi poporul Lui ?i El sa-?i fie Dumnezeu, precum ?i-a grait El ?i cum S-a jurat parin?ilor tai: lui Avraam, lui Isaac ?i lui Iacov.
Deu 29:14  ?i nu numai cu voi singuri închei Eu acest legamânt ?i fac acest juramânt,
Deu 29:15  Ci atât cu cei ce stau astazi aici cu noi înaintea fe?ei Domnului Dumnezeului vostru, cît ?i cu acei care nu sunt astazi aici cu noi.
Deu 29:16  Ca ?ti?i cum am trait noi în pamântul Egiptului ?i cum am trecut prin mijlocul popoarelor pe la care a?i venit,
Deu 29:17  ?i a?i vazut urâciunile lor ?i idolii lor de lemn ?i de piatra, de argint ?i de aur, pe care îi au ele.
Deu 29:18  Sa nu fie printre voi barbat sau femeie, sau neam, sau semin?ie, a caror inima sa se abata acum de la Domnul Dumnezeul vostru, ca sa mearga sa slujeasca dumnezeilor acelor popoare; sa nu fie printre voi radacina din care sa rasara otrava ?i pelin,
Deu 29:19  Nici astfel de om care, auzind cuvintele blestemului acestuia, s-ac lauda în inima sa, zicând: "Eu voi fi fericit, cu toate ca voi umbla dupa voin?a inimii mele", ?i sa piara astfel cel satul cu cel flamând.
Deu 29:20  Pe unul ca acesta nu-l va ierta Domnul, ci îndata se va aprinde mânia Domnului ?i iu?imea Lui asupra unui astfel de om ?i va cadea asupra lui tot blestemul legamântului acestuia, care este scris în cartea aceasta a legamântului ?i va ?terge Domnul numele lui de sub cer,
Deu 29:21  ?i-l va despar?i Domnul spre pieire din toate semin?iile lui Israel, dupa toate blestemele legamântului, care sunt scrise în cartea aceasta a legii.
Deu 29:22  Rândul de oameni care va urma, copiii vo?tri care vor fi dupa voi, strainul care va veni din ?ara departata ?i toate popoarele, vazând pedepsirea pamântului acestuia ?i bolile cu care îl pustie?te Domnul,
Deu 29:23  Vazând pucioasa ?i sarea ?i ca tot pamântul este zgura, încât nici nu se seamana, nici nu rode?te ?i nu rasare pe el nici un fir de iarba, ca de pe urma Sodomei, Gomorei, Admei ?i ?eboimului, pe care le-a stricat Domnul în mânia Sa ?i în iu?imea Sa,
Deu 29:24  Vor zice: Pentru ce a facut Domnul a?a cu ?ara aceasta? Cât de mare este aprinderea mâniei Lui!
Deu 29:25  Atunci vor raspunde: Pentru ca au parasit legamântul Domnului Dumnezeului parin?ilor lor, pe care Acesta l-a încheiat cu ei, când i-a scos din pamântul Egiptului,
Deu 29:26  ?i s-au dus ?i s-au apucat sa slujeasca altor dumnezei ?i s-au închinat acelor dumnezei pe care ei nu i-au cunoscut ?i pe care El nu i-a hotarât.
Deu 29:27  De aceea s-a aprins mânia Domnului asupra Zarii acesteia ?i a adus El asupra ei toate blestemele legamântului, scrise în aceasta carte a legii,
Deu 29:28  ?i i-a lepadat Domnul din pamântul lor cu mânie, cu iu?ime ?i cu aprindere mare ?i i-a aruncat în alt pamânt, cum vedem  acum.
Deu 29:29  Cele ascunse sunt ale Domnului Dumnezeului nostru, iar cele descoperite sunt ale noastre ?i ale fiilor no?tri pe veci, ca sa plinim toate cuvintele legii acesteia".
Deu 30:1  "Când vor veni asupra ta toate cuvintele acestea, binecuvântarea ?i blestemul, pe care ti le-am spus eu ?i le vei primi în inima ta în toate popoarele printre care te va împra?tia Domnul Dumnezeul tau,
Deu 30:2  ?i te vei întoarce la Domnul Dumnezeul tau ?i, cum ti-am poruncit eu astazi, vei asculta glasul Domnului Dumnezeului tau, tu ?i fiii tai, din toata inima ta ?i din tot sufletul tau,
Deu 30:3  Atunci Domnul Dumnezeul tau va întoarce pe robii tai ?i se va milostivi asupra ta ?i iar te va aduna din toate popoarele printre care te-a împra?tiat Domnul Dumnezeul tau.
Deu 30:4  Chiar de ai fi risipit de la o margine a cerului pâna la cealalta margine a cerului, ?i de acolo te va aduna Domnul Dumnezeul tau ?i te va lua ?i de acolo,
Deu 30:5  ?i te va aduce Domnul Dumnezeul tau în pamântul pe care l-au stapânit parin?ii tai ?i-l vei lua în stapânire ?i te va face fericit ?i te va înmul?i mai mult decât pe parin?ii tai.
Deu 30:6  Va taia Domnul împrejur inima ta ?i inima urma?ilor tai, ca sa iube?ti pe Domnul Dumnezeul tau din toata inima ta ?i din tot sufletul tau, ca sa traie?ti.
Deu 30:7  Atunci Domnul Dumnezeul tau va întoarce toate blestemele acestea asupra vrajma?ilor tai ?i a celor ce te-au urât ?i te-au prigonit;
Deu 30:8  Iar tu te vei întoarce ?i vei asculta glasul Domnului Dumnezeului tau ?i vei împlini toate poruncile Lui pe care ?i le spun astazi.
Deu 30:9  Domnul Dumnezeul tau î?i va da cu prisosin?a spor la tot lucrul mâinilor tale, la rodul pântecelui tau, la rodul dobitoacelor tale, la rodul pamântului tau, ca se va bucura Domnul Dumnezeul tau din nou de tine, cum S-a bucurat de parin?i tai, ?i-?i va face bine,
Deu 30:10  De vei asculta glasul Domnului Dumnezeului tau, pazind ?i împlinind toate poruncile Lui, hotarârile Lui ?i legile Lui, ?i de te vei întoarce la Domnul Dumnezeul tau din toata inima ta ?i din tot sufletul tau.
Deu 30:11  Caci porunca aceasta care ?i-o poruncesc eu astazi nu este neîn?eleasa de tine ?i nu este departe.
Deu 30:12  Ea nu este în cer, ca sa zici: Cine se va sui pentru noi în cer, ca sa ne-o aduca ?i sa ne-o dea s-o auzim ?i s-o facem?
Deu 30:13  ?i nu este ea nici peste mare, ca sa zici: Cine se va duce pentru noi peste mare, ca sa ne-o aduca, sa ne faca s-o auzim ?i s-o împlinim?
Deu 30:14  Ci Cuvântul acesta este foarte aproape de tine; el este în gura ta ?i în inima ta, ca sa-l faci.
Deu 30:15  Iata eu astazi ?i-am pus înainte via?a ?i moartea, binele ?i raul,
Deu 30:16  Poruncindu-?i astazi sa iube?ti pe Domnul Dumnezeul tau, sa umbli în toate caile Lui ?i sa împline?ti poruncile Lui, hotarârile Lui ?i legile Lui, ca sa traie?ti ?i sa te înmul?e?ti ?i sa te binecuvânteze Domnul Dumnezeul tau pe pamântul pe care îl vei stapâni.
Deu 30:17  Iar de se va întoarce inima ta ?i nu vei asculta, ci te vei lasa ademenit ?i te vei închina la al?i dumnezei ?i le vei sluji lor,
Deu 30:18  Va dau de ?tire astazi ca ve?i pieri ?i nu ve?i trai mult în pamântul pe care Domnul Dumnezeu ?i-l da ?i pentru a carui stapânire treci tu acum Iordanul.
Deu 30:19  Ca martori înaintea voastra iau astazi cerul ?i pamântul: via?a ?i moarte Zi-am pus eu astazi înainte, ?i binecuvântare ?i blestem. Alege via?a ca sa traie?ti tu ?i urma?ii tai.
Deu 30:20  Sa iube?ti pe Domnul Dumnezeul tau, sa ascul?i glasul Lui ?i sa te lipe?ti de El; caci în aceasta este via?a ta ?i lungimea zilelor tale, ca sa locuie?ti pe pamântul pe care Domnul Dumnezeul tau cu juramânt l-a fagaduit parin?ilor tai, lui Avraam, lui Isaac ?i lui Iacov ca îl va da lor".
Deu 31:1  Atunci s-a dus Moise ?i a grait cuvintele acestea tuturor fiilor lui Israel
Deu 31:2  ?i le-a zis: "Eu acum sunt de o suta douazeci de ani ?i nu mai pot intra ?i ie?i ?i Domnul mi-a zis: Tu nu vei trece Iordanul acesta,
Deu 31:3  Ci va merge înaintea ta Însu?i Domnul Dumnezeul tau ?i va stârpi El pe popoarele acestea de la fa?a ta ?i tu le vei stapâni; ?i Iosua va merge înaintea ta, cum a zis Domnul.
Deu 31:4  ?i va face Domnul cu ei cum a facut ?i cu Sihon ?i cu Og, regii Amoreilor, care erau dincoace de Iordan, ?i cum a facut cu pamântul acelora pe care i-a pierdut;
Deu 31:5  Îi va da Domnul pe ei voua ?i ve?i face cu ei dupa poruncile pe care vi le-am spus eu.
Deu 31:6  Fi?i tari ?i curajo?i, nu va teme?i, nu va îngrozi?i, nici nu va spaimânta?i de ei, ca Domnul Dumnezeul tau va merge El Însu?i cu tine ?i nu se va departa de tine, nici nu te va parasi".
Deu 31:7  Apoi a chemat Moise pe Iosua ?i înaintea ochilor tuturor Israeli?ilor i-a zis: "Fii tare ?i curajos, ca tu vei intra cu poporul acesta în pamântul pe care Domnul S-a jurat parin?ilor lui sa i-l dea ?i tu i-l vei împar?i în par?i de mo?tenire.
Deu 31:8  Domnul Însu?i va merge înaintea ta; El Însu?i va fi cu tine ?i nu se va departa de tine, nici te va parasi; nu te teme, nici nu te spaimânta".
Deu 31:9  Apoi a scris Moise legea aceasta ?i a dat-o preo?ilor, fiilor levi?ilor, care purtau chivotul legii Domnului, ?i tuturor batrânilor fiilor lui Israel.
Deu 31:10  ?i le-a poruncit Moise acestora ?i le-a zis: "Dupa trecerea a ?apte ani, în anul iertarii, la sarbatoarea corturilor,
Deu 31:11  Când tot Israelul va veni sa se înfa?i?eze înaintea fe?ei Domnului Dumnezeului tau, în locul pe care-l va alege Domnul, sa cite?ti legea aceasta înaintea a tot Israelul ?i în auzul lui;
Deu 31:12  Sa aduni poporul, barba?ii, femeile, copiii ?i pe strainii tai care se vor afla în ceta?ile tale, ca sa auda ?i sa înve?e ?i ca sa se teama de Domnul Dumnezeul vostru ?i ca sa se sileasca sa împlineasca toate cuvintele legii acesteia.
Deu 31:13  Fiii lor care nu ?tiu vor auzi ?i vor înva?a a se teme de Domnul Dumnezeul vostru în toate zilele, cît veri trai pe pamântul în care trece?i voi peste Iordan ca sa-l stapâni?i".
Deu 31:14  Dupa aceea a zis Domnul catre Moise: "Iata s-a apropiat clipa în care sa mori; cheama pe Iosua ?i sta?i la u?a cortului adunarii, ca Eu îi voi da pove?e!" ?i a venit Moise cu Iosua ?i au stat la u?a cortului adunarii.
Deu 31:15  Atunci S-a aratat Domnul în cort, în stâlp de nor, ?i stâlpul de nor a stat la u?a cortului adunarii.
Deu 31:16  ?i a zis Domnul catre Moise: "Iata, tu te vei odihni cu parin?ii tai, iar poporul acesta se va scula ?i se va desfrâna dupa dumnezeii straini ai pamântului aceluia în care va intra, iar pe Mine Ma va parasi ?i va calca legamântul Meu, pe care l-am încheiat cu el;
Deu 31:17  Pentru aceasta se va aprinde mânia Mea asupra lui în ziua aceea ?i-i voi parasi, Îmi voi ascunde fa?a de la ei ?i vor fi omorâ?i ?i-i vor ajunge mul?ime de necazuri ?i greuta?i ?i atunci Israel va zice: Aceste necazuri nu m-au ajuns ele oare pentru ca nu este Domnul Dumnezeul meu în mijlocul meu?
Deu 31:18  Dar Eu Îmi voi ascunde fa?a Mea de la el în ziua aceea, pentru toate faradelegile lui pe care le-a facut el, întorcându-se la al?i dumnezei.
Deu 31:19  A?adar, scrie-?i cuvintele cântarii acesteia ?i înva?a pe fiii lui Israel ?i le-o pune în gura lor, pentru ca aceasta cântare sa-Mi fie marturie printre fiii lui Israel.
Deu 31:20  Caci Eu îi voi duce în pamântul cel bun, unde curge lapte ?i miere, dupa cum M-am jurat parin?ilor lor, ?i vor mânca, se vor satura, se vor îngra?a, se vor îndrepta spre al?i dumnezei ?i vor sluji acelora, iar pe Mine Ma vor lepada ?i vor calca legamântul Meu, pe care l-am dat lor.
Deu 31:21  Dar când îi vor ajunge mul?ime de nenorociri ?i de necazuri, atunci cântarea aceasta va fi marturie împotriva lor, caci ea nu va pieri din gura lor ?i din gura urma?ilor lor. Cunosc Eu cugetele lor pe care le au ei acum, înainte de a-i duce în pamântul cel bun pentru care M-am jurat parin?ilor lor".
Deu 31:22  ?i a scris Moise cântarea aceasta în ziua aceea ?i a spus-o fiilor lui Israel.
Deu 31:23  Iar Domnul a poruncit lui Iosua Navi ?i i-a zis: "Fii tare ?i curajos, caci tu vei duce pe fiii lui Israel în pamântul pentru care M-am jurat lor, ?i Eu voi fi cu tine".
Deu 31:24  Când a scris Moise în carte toate cuvintele legii acesteia pâna la sfâr?it,
Deu 31:25  Atunci Moise a poruncit levi?ilor care purtau chivotul legii Domnului
Deu 31:26  ?i a zis: "Lua?i aceasta carte a legii ?i o pune?i de-a dreapta chivotului legii Domnului Dumnezeului vostru ?i va fi ea acolo marturie împotriva ta.
Deu 31:27  Ca eu cunosc îndaratnicia ta ?i cerbicia ta; daca ?i acum, când traiesc eu cu voi, sunte?i îndaratnici înaintea Domnului, dar cu cît mai mult ve?i fi dupa moartea mea?
Deu 31:28  Chema?i la mine pe to?i batrânii semin?iilor voastre ?i pe judecatorii vo?tri ?i pe capeteniile voastre ?i eu voi spune în auzul lor cuvintele acestea ?i voi chema martori împotriva lor cerul ?i pamântul;
Deu 31:29  Caci ?tiu eu ca dupa moartea mea va ve?i razvrati ?i va ve?i abate de la calea pe care v-am aratat-o eu; în zilele cele de apoi va vor ajunge necazuri, pentru ca ve?i face rau înaintea ochilor Domnului Dumnezeu, mîniindu-L cu lucrurile mâinilor voastre".
Deu 31:30  ?i a rostit Moise în auzul întregii ob?ti a Israeli?ilor cuvintele cântarii acesteia pâna la sfâr?it:
Deu 32:1  "Ia aminte, cerule, ?i voi grai! Asculta, pamântule, cuvintele gurii mele!
Deu 32:2  Ca ploaia sa curga înva?atura mea ?i graiurile mele sa se coboare ca roua, ca bura pe verdea?a ?i ca ploaia repede pe iarba.
Deu 32:3  Caci numele Domnului voi preamari. Da?i slava Dumnezeului nostru!
Deu 32:4  El este taria; desavâr?ite sunt lucrurile Lui, caci toate caile Lui sunt drepte. Credincios este Dumnezeu ?i nu este întru El nedreptate; drept ?i adevarat este El,
Deu 32:5  Iar ei s-au razvratit împotriva Lui; ei, dupa netrebniciile lor, nu sunt fiii Lui, ci neam îndaratnic ?i ticalos. Cu acestea rasplati?i voi Domnului?
Deu 32:6  Popor nechibzuit ?i fara de minte, au nu este El tatal tau, Cel ce te-a zidit, te-a facut ?i te-a întemeiat?
Deu 32:7  Adu-?i aminte de zilele cele de demult, cugeta la anii neamurilor trecute! Întreaba pe tatal tau ?i-?i va da de ?tire, întreaba pe batrâni, ?i-?i vor spune:
Deu 32:8  Când Cel Preaînalt a împar?it mo?tenire popoarelor, când a împar?it pe fiii lui Adam, atunci a statornicit hotarele neamurilor dupa numarul îngerilor lui Dumnezeu;
Deu 32:9  Iar partea Domnului este poporul lui Iacov, Israel e partea lui de mo?tenire.
Deu 32:10  Gasitu-l-a în pamânt pustiu, în pustiu trist ?i cu urlete salbatice, ?i t-a aparat, l-a îngrijit ?i l-a pazit, ca lumina ochiului Sau.
Deu 32:11  Întocmai ca vulturul care îndeamna la zbor puii sai ?i se rote?te pe deasupra lor, întinzându-?i aripile, a luat pe Israel ?i l-a dus pe penele sale.
Deu 32:12  Domnul l-a pova?uit ?i n-a fost cu el dumnezeu strain.
Deu 32:13  ?i l-a a?ezat pe înal?imile pamântului ?i l-a hranit cu roada ?arinilor. I-a dat sa scoata miere din piatra ?i cu untdelemn din stânca vârtoasa l-a hranit;
Deu 32:14  L-a hranit cu unt de vaca ?i cu lapte de oi, cu grasimea mieilor, a berbecilor de Vasan, a ?apilor ?i cu grâu gras; a baut vin, sângele bobi?elor de strugure.
Deu 32:15  A mâncat Iacov, s-a îngra?at Israel ?i s-a facut îndaratnic; îngra?atu-s-a, îngro?atu-s-a ?i s-a umplut de grasime; a parasit pe Dumnezeu, Cel ce l-a facut ?i a dispre?uit cetatea mântuirii sale.
Deu 32:16  Întarâtat-au râvna Lui cu dumnezei straini ?i cu urâciunile lor L-au mâniat;
Deu 32:17  Adus-au jertfe demonilor, ?i nu lui Dumnezeu, unor dumnezei noi, pe care nu i-au ?tiut, care au venit de la vecinii lor ?i pe care parin?ii lor nu i-au cunoscut.
Deu 32:18  Iar pe Aparatorul, Cel ce te-a nascut, L-ai uitat ?i nu ?i-ai adus aminte de Dumnezeu, Cel ce te-a zidit.
Deu 32:19  Vazut-a Domnul ?i S-a mâniat ?i în mânia Sa a trecut cu vederea pe fiii Sai ?i pe fiicele Sale,
Deu 32:20  ?i a zis: îmi voi ascunde fa?a Mea de la ei ?i voi vedea cum va fi sfâr?itul lor; caci neam ticalos sunt ei ?i copii în care nu este credincio?ie.
Deu 32:21  Ei M-au întarâtat la gelozie prin cei ce nu sunt Dumnezeu ?i au aprins mânia Mea prin idolii lor; îi voi întarâta ?i Eu pe ei printr-un popor care nu e popor, le voi aprinde mânia printr-un neam fara pricepere.
Deu 32:22  Ca foc s-a aprins din pricina mâniei Mele: va arde pâna în fundul locuin?ei mor?ilor, va mânca pamântul ?i roadele lui ?i va pârjoli temeliile mun?ilor.
Deu 32:23  Voi strânge împotriva lor necazuri ?i voi cheltui asupra lor toate sage?ile Mele;
Deu 32:24  Istovi?i vor fi de foame ?i prapadi?i de lingoare ?i molima rea; voi trimite asupra lor din?ii fiarelor, veninul târâtoarelor din pulbere voi trimite.
Deu 32:25  De din afara îi va pierde sabia, iar prin case groaza, pierzând pe tânar ?i pe tânara, pe copilul de ?â?a ?i pe batrânul acoperit de carunte?e.
Deu 32:26  Am zis: ?i voi împra?tia ?i voi ?terge pomenirea lor dintre oameni.
Deu 32:27  Dar am amânat aceasta, pentru rautatea vrajma?ilor, ca vrajma?ii lor sa nu se mândreasca ?i sa zica: Mâna noastra este puternica ?i toate acestea nu le-a facut Domnul.
Deu 32:28  Ca ace?tia sunt oameni care ?i-au pierdut mintea ?i n-au nici o pricepere.
Deu 32:29  O, de ar judeca ei ?i de s-ar gândi la aceasta! De ar pricepe ce are sa fie cu ei mai pe urma:
Deu 32:30  Cum ar putea unul sa puna pe fuga o mie, ?i doi, zece mii, daca aparatorul lor nu i-ar vinde ?i Domnul nu i-ar parasi!
Deu 32:31  Caci aparatorul lor nu este ca Aparatorul nostru ?i la aceasta chiar vrajma?ii no?tri sunt martori.
Deu 32:32  Ca via lor este din vi?a de vie a Sodomei ?i din ?esurile Gomorei; strugurii lor sunt struguri otravi?i ?i bobi?ele lor amare;
Deu 32:33  Vinul lor este venin de scorpion ?i otrava pierzatoare de aspida.
Deu 32:34  Au nu sunt acestea ascunse la Mine? ?i nu sunt ele pecetluite în camarile Mele?
Deu 32:35  A Mea este razbunarea ?i rasplatirea când se va poticni piciorul lor; ca aproape este ziua pieirii lor ?i curând vor veni cele gatite pentru ei.
Deu 32:36  Iar Domnul va judeca pe poporul Sau ?i Se va milostivi asupra robilor Sai, când va vedea ca a slabit taria lor ?i ca nu se mai afla nici robi, nici slobozi.
Deu 32:37  Atunci Domnul va zice: Unde sunt dumnezeii lor ?i taria în care nadajduiau ei?
Deu 32:38  Unde sunt cei ce au mâncat grasimea jertfelor lor ?i au baut vinul turnarilor lor? Sa se scoale, sa va ajute ?i sa va fie ocrotire.
Deu 32:39  Vede?i, vede?i, dar, ca Eu sunt ?i nu este alt Dumnezeu afara de Mine: Eu omor ?i înviez, Eu ranesc ?i tamaduiesc ?i nimeni nu poate scapa din mâna Mea!
Deu 32:40  Eu ridic la cer mâna Mea ?i Ma jur pe dreapta Mea ?i zic: Viu sunt Eu în veac!
Deu 32:41  Când voi ascu?i sabia Mea cea lucitoare ?i va începe mâna Mea a judeca, Ma voi razbuna pe vrajma?ii Mei ?i celor ce Ma urasc le voi rasplati.
Deu 32:42  Adapa-voi sage?ile Mele cu sânge ?i sabia Mea se va satura de carnea ?i de sângele celor uci?i ?i robi?i ?i de capetele capeteniilor vrajma?ului.
Deu 32:43  Veseli?i-va, ceruri, împreuna cu El ?i va închina?i Lui to?i îngerii lui Dumnezeu! Veseli?i-va, neamuri, împreuna cu poporul Lui ?i sa se întareasca to?i fiii lui Dumnezeu! Caci El va razbuna sângele robilor Sai ?i va rasplati cu razbunare vrajma?ilor Sai ?i celor ce-L urasc le va rasplati ?i va cura?i Domnul pamântul poporului Sau!"
Deu 32:44  În ziua aceea a scris Moise cântarea aceasta ?i a spus-o fiilor lui Israel. Atunci a venit Moise la popor ?i a rostit toate cuvintele cântarii acesteia în auzul poporului, el ?i Iosua, fiul lui Navi.
Deu 32:45  Iar dupa ce a rostit Moise toate cuvintele acestea înaintea a tot Israelul, le-a zis:
Deu 32:46  "Pune?i la inima voastra toate cuvintele pe care vi le-am spus eu astazi ?i sa le lasa?i mo?tenire copiilor vo?tri, ca sa se sileasca ?i ei a împlini toate poruncile legii acesteia;
Deu 32:47  Caci acestea nu sunt în de?ert date voua, ci acestea sunt via?a voastra ?i prin acestea ve?i trai multa vreme în pamântul acela în care trece?i acum peste Iordan, ca sa-l stapâni?i".
Deu 32:48  Tot în ziua aceea a grait Domnul cu Moise ?i a zis:
Deu 32:49  "Suie-te în muntele acesta al Abarimului, în muntele Nebo, care este în pamântul Moabului, în fa?a Ierihonului, ?i prive?te asupra Canaanului, pe care-l dau în stapânirea fiilor lui Israel,
Deu 32:50  ?i mori pe munte ?i te adauga la poporul tau, cum a murit ?i Aaron, fratele tau, pe muntele Hor ?i s-a adaugat la poporul sau,
Deu 32:51  Pentru ca a?i gre?it înaintea Mea în mijlocul fiilor lui Israel, la apele Meribei, la Cade?, în pustiul Sin, ?i pentru ca n-a?i aratat sfin?enia Mea între fiii lui Israel.
Deu 32:52  Numai de departe vei vedea pamântul pe care Eu îl dau fiilor lui Israel, dar de intrat nu vei intra în pamântul acela".
Deu 33:1  Iata acum ?i binecuvântarea cu care Moise, omul lui Dumnezeu, a binecuvântat pe fiii lui Israel înainte de moartea sa.
Deu 33:2  ?i a zis el: "Venit-a Domnul din Sinai ?i ni S-a descoperit întru slava Sa în Seir; stralucit-a din Mun?ii Paranului ?i a ie?it cu mul?ime mare de sfin?i, având la dreapta focul legii.
Deu 33:3  Cu adevarat El a iubit pe poporul Sau. To?i sfin?ii Lui sunt sub mâna Lui ?i au cazut la picioarele Lui ca sa asculte cuvintele Lui.
Deu 33:4  Moise ne-a dat o lege, mo?tenire a ob?tii lui Iacov;
Deu 33:5  El a fost regele lui Israel, când se adunau capeteniile popoarelor împreuna cu semin?iile lui Israel.
Deu 33:6  Sa traiasca Ruben ?i sa nu moara, ?i Simeon sa nu fie pu?in la numar!
Deu 33:7  Iar pentru Iuda a zis acestea: "Asculta, Doamne, glasul lui Iuda ?i adu-l la poporul sau; cu mâinile sale sa se apere ?i Tu sa-i fii ajutor împotriva vrajma?ilor lui".
Deu 33:8  Pentru Levi a zis: "Urimul Tau, Doamne, ?i Tumimul Tau sa fie pentru barbatul Tau cel sfânt, pe care Tu l-ai încercat la Massa ?i cu care Tu Te-ai certat la apele Meribei;
Deu 33:9  Care a zis de tatal sau ?i de mama sa: "Nu i-am vazut", ?i pe fra?ii sai nu i-a cunoscut ?i de fiii sai nu ?tie nimic; caci ei au ?inut cuvintele Tale ?i legamântul Tau l-au pazit;
Deu 33:10  Înva?a pe Iacov legile Tale ?i pe Israel poruncile Tale; pune tamâie înaintea fe?ei Tale ?i arderi de tot pe jertfelnicul Tau.
Deu 33:11  Binecuvinteaza, Doamne, puterea lui ?i lucrul mâinilor lui fie-?i placut; love?te coapsele celar ce se ridica împotriva lui ?i celor ce-l urasc, ca sa nu se poata împotrivi".
Deu 33:12  Pentru Veniamin a zis: "Iubitul Domnului va sta lânga El fara primejdie ?i Dumnezeu îl va ocroti în toata vremea ?i el va odihni pe umerii Lui".
Deu 33:13  Pentru Iosif a zis: "Sa binecuvânteze Domnul pamântul lui cu darurile cele alese ale cerului, cu roua ?i cu darurile adâncului celui dedesubt;
Deu 33:14  Cu roade alese din cele pe care le face sa creasca soarele ?i cu cele mai bune roade care odraslesc în fiecare luna;
Deu 33:15  Cu cele mai alese din câte dau mun?ii cei vechi ?i cu darurile alese ale dealurilor celor ve?nice;
Deu 33:16  ?i cu darurile cele alese ale pamântului ?i ale celor ce-l umplu. Binecuvântarea Celui ce S-a aratat în rug sa vina pe capul lui Iosif, pe cre?tetul celui dintâi dintre fra?ii lui!
Deu 33:17  Frumuse?ea lui sa fie ca a taurului întâi nascut ?i coarnele lui sa fie ca ?i coarnele bivolului; cu acelea va împunge popoarele toate pâna la marginile pamântului; acestea sunt mul?imile mari ale lui Efraim, acestea sunt miile lui Manase".
Deu 33:18  Pentru Zabulon a zis: "Vesele?te-te, Zabulon, în caile tale ?i tu, Isahare, în corturile tale!
Deu 33:19  Chema-vor ace?tia poporul pe munte ?i acolo vor junghia jertfele cele legiuite, caci se hranesc cu boga?ia marii ?i cu comorile cele ascunse în nisip".
Deu 33:20  Pentru Gad a zis: "Binecuvântat. sa fie cel ce a sporit pe Gad, care odihne?te ca un pui de leu ?i sfarâma ?i bra? ?i cap.
Deu 33:21  Alesu-?i-a el pârga cea buna a ?arii; acolo fost-a pastrata o mo?ie de capetenie; venit-a în fruntea poporului ?i a împlinit dreptatea Domnului ?i judeca?ile lui Israel".
Deu 33:22  Pentru Dan a zis: "Dan este pui de leu, care se arunca din Vasan".
Deu 33:23  Pentru Neftali a zis: "Neftali, tu cel satul de bunavoin?a ?i plin de binecuvântarea Domnului, ia marea ?i partea cea de miazazi în stapânire".
Deu 33:24  Pentru A?er a zis: "Binecuvântat sa fie A?er între fii, iubit sa fie de fra?ii lui ?i sa-?i afunde în untdelemn piciorul lui.
Deu 33:25  De fier ?i de arama sa fie zavoarele ?i lini?tea ta sa ?ina cît zilele vie?ii tale.
Deu 33:26  Nimeni, o Israele, nu este ca Dumnezeu, Care sa mearga pe ceruri întru ajutorul tau ?i pe nori întru slava Sa.
Deu 33:27  Dumnezeu este liman din vremi stravechi; caci cu bra?ul Lui cel ve?nic El te sprijina ?i din fa?a ta gonind vrajma?ii, zice: "Stârpe?te-i!"
Deu 33:28  Israel locuie?te neprimejduit. Ochiul lui Iacov prive?te îmbel?ugat de pâine ?i de vin, ?i cerurile lui picura roua.
Deu 33:29  Ferice de tine, Israele! Cine este asemenea ?ie, popor izbavit de Domnul, scutul ?i ajutorul tau, sabia ?i slava ta. Vrajma?ii tai se vor da înapoi înaintea ta ?i tu vei calca peste grumajii lor!"
Deu 34:1  Atunci s-a suit Moise din ?esurile Moabului în Muntele Nebo, pe vârful Fazga, care este în fa?a Ierihonului, ?i i-a aratat Domnul tot pamântul Galaad pâna la Dan,
Deu 34:2  Tot pamântul lui Neftali, tot pamântul lui Efraim ?i Manase ?i tot pamântul lui Iuda pâna la marea cea de la asfin?it,
Deu 34:3  Partea de la miazazi a ?arii, ?esul Ierihonului, cetatea Palmierilor, pâna la ?oar.
Deu 34:4  ?i i-a zis Domnul: "Iata pamântul pentru care M-am jurat lui Avraam, lui Isaac ?i lui Iacov, zicând: Semin?iei tale îl voi da. Te-am învrednicit sa-l vezi cu ochii tai; dar în el nu vei intra!
Deu 34:5  ?i a murit Moise, robul lui Dumnezeu, acolo, în pamântul Moabului, dupa Cuvântul Domnului;
Deu 34:6  ?i a fost îngropat în vale, în pamântul Moabului, în fa?a Bet-Peorului, dar nimeni nu ?tie mormântul lui nici pâna în ziua de astazi.
Deu 34:7  ?i era Moise de o suta douazeci de ani, când a murit; dar vederea lui nu slabise ?i taria lui nu se împu?inase.
Deu 34:8  ?i au plâns fiii lui Israel pe Moise, în ?esurile Moabului, la Iordan, aproape de Ierihon, treizeci de zile, pâna s-au împlinit zilele de jelit ?i de plâns dupa Moise.
Deu 34:9  Iar Iosua, fiul lui Navi, s-a umplut de duhul în?elepciunii, pentru ca î?i pusese Moise mâinile asupra lui, ?i i s-au supus fiii lui Israel ?i au facut a?a dupa cum le poruncise Domnul prin Moise.
Deu 34:10  De atunci nu s-a mai ridicat în Israel prooroc asemenea lui Moise pe care Dumnezeu sa-l fi cunoscut fa?a catre fa?a,
Deu 34:11  Nici sa savâr?easca toate semnele ?i minunile cu care Domnul l-a trimis în pamântul Egiptului asupra lui Faraon ?i asupra tuturor dregatorilor lui ?i asupra a tot pamântul lui;
Deu 34:12  Nici sa faca cu mâna tare ?i cu mari înfrico?ari ceea ce a facut Moise înaintea ochilor a tot Israelul.


\end{document}