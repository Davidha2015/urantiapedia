\begin{document}

\title{Deuteronom}


\chapter{1}

\par 1 Acestea sunt cuvintele pe care le-a grăit Moise la tot Israelul peste Iordan, în pustiul Arabah, din fața Sufei, între Paran, Tofel, Laban, Hașerot și Di-Zahab,
\par 2 Cale de unsprezece zile de la Horeb, în drumul de la Muntele Seir, către Cadeș-Barnea.
\par 3 În anul al patruzecilea, în luna a unsprezecea, în ziua întâi a lunii acesteia, a grăit Moise tuturor fiilor lui Israel toate cît îi poruncise Domnul pentru ei.
\par 4 După ce a bătut pe Sihon, regele Amoreilor, care locuia în Heșbon, și pe Og, regele Vasanului, care locuia în Aștarot și în Edrei, dincolo de Iordan, în pământul Moabului,
\par 5 A început Moise a lămuri legea aceasta și a zis:
\par 6 "Domnul Dumnezeul vostru ne-a grăit nouă pe Horeb și a zis: Vă ajunge de când locuiți pe muntele acesta!
\par 7 Întoarceți-vă și, pornind la drum, duceți-vă la muntele Amoreilor și la toți vecinii lor din Arabah, din munte, din Șefela și din Negeb, la malurile mării, în pământul Canaanului, la Liban, și chiar până la râul cel mare, la fluviul Eufratului.
\par 8 Și iată, Eu vă dau pământul acesta; mergeți și vă luați de moștenire pământul pe care Domnul a făgăduit cu jurământ să-l dea părinților voștri, lui Avraam și lui Isaac și lui Iacov, lor și urmașilor lor.
\par 9 În vremea aceea v-am zis: Nu vă mai pot povățui singur;
\par 10 Domnul Dumnezeul vostru v-a înmulțit și iată acum sunteți mulți la număr, ca stelele cerului.
\par 11 Domnul Dumnezeul părinților voștri să vă înmulțească de o mie de ori mai mult decât sunteți acum și să vă binecuvânteze, cum v-a făgăduit El!
\par 12 Cum dar voi purta singur greutățile voastre și sarcinile voastre și neînțelegerile dintre voi?
\par 13 Alegeți-vă din semințiile voastre bărbați înțelepți, pricepuți și încercați, și-i voi pune căpetenii peste voi.
\par 14 Atunci mi-ați răspuns și ați zis: Bun lucru ne poruncești să facem!
\par 15 Și am luat dintre voi bărbați înțelepți, pricepuți și încercați, și i-am pus povățuitori peste voi: căpetenii peste mii, peste sute, peste cincizeci, peste zeci și judecători peste semințiile voastre.
\par 16 În vremea aceea, am dat poruncă judecătorilor voștri și am zis: Să ascultați pe frații voștri și să judecați drept pricina ce ar avea un om atât cu fratele lui, cît și cu cel străin.
\par 17 Să nu părtiniți la judecată, ci să ascultați și pe cel mare și pe cel mic. Să nu vă sfiiți de la fața omului, că judecata este a lui Dumnezeu. Iar pricina care va fi grea pentru voi să o aduceți la mine și o voi asculta eu.
\par 18 V-am mai dat în vremea aceea porunci pentru toate cele ce trebuie să faceți.
\par 19 Am plecat apoi de la Horeb, cum ne poruncise Domnul Dumnezeul nostru, și am străbătut tot pustiul acesta mare și înfricoșător, pe care l-ați văzut în drumul spre muntele Amoreilor, și am ajuns la Cadeș-Barnea.
\par 20 Atunci v-am zis: Iată, ați ajuns la muntele Amoreilor, pe care Domnul Dumnezeul vostru, îl va da nouă.
\par 21 Iată, Israel, Domnul Dumnezeul tău îți dă pământul acesta: mergi și ia-l în stăpânire, cum ți-a zis Domnul Dumnezeul părinților tăi; nu te teme, nici nu te înspăimânta!
\par 22 Iar voi ați venit cu toții la mine și ați zis: Să trimitem înaintea noastră oameni ca să cerceteze pământul și să ne aducă știre despre drumul pe care să mergem și despre cetățile la care să ne ducem.
\par 23 Cuvântul acesta mi-a plăcut și am luat dintre voi doisprezece oameni, câte unul din fiecare seminție.
\par 24 Aceștia s-au dus și s-au suit pe munte, au mers până la valea Eșcol, și au cercetat-o.
\par 25 Au luat din roadele pământului și ne-au adus nouă; și ne-au adus și știre, spunându-ne: Pământul pe care Domnul Dumnezeul nostru ni-l dă este bun.
\par 26 Voi însă n-ați vrut să vă duceți și v-ați împotrivit poruncii Domnului Dumnezeului vostru, ați cârtit în corturile voastre și ați zis:
\par 27 Domnul din ură către noi ne-a scos din pământul Egiptului, ca să ne dea în mâinile Amoreilor și să ne piardă.
\par 28 Încotro să ne ducem? Frații noștri ne-au înfricoșat, spunându-ne: "Poporul acela e mai mare, maț mult și mai puternic decât noi; cetățile de acolo sunt mari și cu întărituri până la cer; și am mai văzut acolo și pe fiii lui Enac".
\par 29 Atunci v-am zis: Nu vă înspăimântați și nu vă temeți de ei.
\par 30 Domnul Dumnezeul vostru merge înaintea voastră și se va lupta El pentru voi, cum a făcut cu voi și în Egipt, înaintea ochilor voștri.
\par 31 Și cum a făcut în pustiul acesta, unde, cum ai văzut tu, Israel, Domnul Dumnezeul tău te-a purtat tot drumul ce l-ați străbătut până ce ați sosit la locul acesta, cum poartă un om pe fiul său.
\par 32 Dar voi nici așa n-ați crezut pe Domnul Dumnezeul vostru,
\par 33 Care a mers înaintea voastră în călătorie, ca să vă caute loc unde să poposiți; și mergea noaptea în foc, ca să vă arate calea pe care să mergeți, iar ziua în nor.
\par 34 Și auzind Domnul Dumnezeu cuvintele voastre, S-a mâniat și S-a jurat, zicând: "Nimeni din oamenii aceștia, din acest neam rău,
\par 35 Nu va vedea pământul cel bun, pe care Eu am jurat să-l dau părinților voștri.
\par 36 Numai Caleb, fiul lui Iefone, îl va vedea. Aceluia și fiilor lui voi da pământul pe care 1-a străbătut el, pentru că acela s-a supus Domnului.
\par 37 Pentru voi s-a mâniat Domnul și pe mine și a zis: Nici tu nu vei intra acolo!
\par 38 Iosua, fiul lui Navi, care este cu tine, acela va intra acolo; întărește-l pe el, că el va duce pe Israel în moștenirea sa.
\par 39 Copiii voștri, de care voi ziceați că vor cădea pradă vrăjmașilor, și fiii voștri, care acum nu cunosc nici binele, nici răul, aceia vor intra acolo; lor îl voi da și ei îl vor moșteni.
\par 40 Iar voi întoarceți-vă și vă îndreptați spre pustie, pe calea cea către Marea Roșie.
\par 41 Voi însă mi-ați răspuns atunci și mi-ați zis: Am păcătuit înaintea Domnului Dumnezeului nostru! Ne ducem să ne luptăm cum ne-a poruncit Domnul Dumnezeul nostru! Și v-ați încins fiecare cu arma sa de luptă și v-ați hotărât nebunește să vă suiți pe munte.
\par 42 Iar Domnul mi-a zis: Spune-le: Nu vă suiți și nu vă luptați, ca să nu vă biruiască vrăjmașii voștri, că Eu nu sunt în mijlocul vostru.
\par 43 Și eu v-am spus, dar voi n-ați ascultat, ci v-ați împotrivit poruncii Domnului și, în îndărătnicia voastră, v-ați suit pe munte.
\par 44 Dar v-a ieșit înainte poporul amoreu care locuia pe muntele acela și a tăbărât asupra voastră ca albinele și v-a zdrobit de la Seir până la Horma.
\par 45 Atunci v-ați întors și ați plâns înaintea Domnului, dar Domnul n-a ascultat plângerea voastră și nu v-a luat în seamă.
\par 46 Și ați locuit în Cadeș vreme multă, că multe au fost zilele cît ați stat acolo".

\chapter{2}

\par 1 Apoi, întorcându-ne noi, am pornit prin pustie, spre Marea Roșie, cum îmi grăise Domnul, și am umblat zile multe împrejurul Muntelui Seir.
\par 2 Iar Domnul a zis către mine:
\par 3 "Ajunge de când umblați împrejurul acestui munte! Întoarceți-vă dar spre miazănoapte!
\par 4 Dă poruncă poporului și zi: Voi acum veți trece prin hotarele fiilor lui Isav, frații voștri, care locuiesc în Seir, și aceștia se vor teme de voi foarte tare.
\par 5 Dar să nu începeți războiul cu ei, căci nu vă voi da din pământul lor nici o palmă de loc, pentru că Muntele Seir l-am dat în stăpânirea lui Isav.
\par 6 Mâncare să vă cumpărați de la ei cu bani și să mâncați; și apă de băut să vă cumpărați de la ei tot cu bani;
\par 7 Că Domnul Dumnezeul tău, Israele, te-a binecuvântat în tot lucrul mâinilor tale și te-a ocrotit în timpul călătoriei tale prin pustiul acesta mare și înfricoșător. Iată, de patruzeci de ani Domnul Dumnezeul tău este cu tine și n-ai dus lipsă de nimic".
\par 8 Și am trecut pe lângă fiii lui Isav, frații noștri, care locuiau în Seir, pe calea câmpului, de la Elat și Ețion-Gaber, și ne-am abătut și am mers spre pustiul Moabului.
\par 9 Dar Domnul mi-a zis: "Nu intra în dușmănie cu Moab și nu începe război cu el, că nu-ți voi da în stăpânire nimic din pământul lui, pentru că Arul l-am dat în stăpânire fiilor lui Lot".
\par 10 Înainte au locuit acolo Emimii, popor mare, mult la număr și înalt la statură, ca fiii lui Enac;
\par 11 Și aceștia se socoteau printre Refaimi, ca fiii lui Enac; iar Moabiții îi numesc Emimi.
\par 12 Pe Seir însă au trăit înainte Horeii; dar fiii lui Isav i-au alungat și i-au pierdut de la fața lor și s-au așezat în locul lor, cum trebuie să facă și Israel în pământul său de moștenire, care i-l va da Domnul.
\par 13 Sculați-vă dar și treceți râul Zared. Și am trecut noi râul Zared.
\par 14 De atunci, de când ne-am dus la Cadeș-Barnea și până ce am trecut râul Zared, au trecut treizeci și opt de ani și au pierit din tabăra noastră toți cei ce erau atunci buni de război, după cum li se jurase Domnul;
\par 15 Că mâna Domnului, până au pierit ei, a fost asupra lor, ca să-i piardă din tabără.
\par 16 Iar dacă au pierit toți cei ce erau atunci buni de război din popor,
\par 17 Mi-a grăit Domnul și a zis:
\par 18 "Acum tu să treci pe lângă hotarele lui Moab spre Ar,
\par 19 Și să te apropii repede de Amoniți, dar să nu intri cu aceștia în dușmănie și să nu începi război cu ei, căci nu-ți voi da în stăpânire nimic din pământul fiilor lui Amon, pentru că l-am dat în stăpânire fiilor lui Lot".
\par 20 Acesta se socotea a fi pământul Refaimilor, căci Refaimii locuiseră înainte într-însul. Amoniții însă îi numeau Zomzomimi.
\par 21 Poporul acesta fusese mare, mult la număr și înalt la statură, ca fiii lui Enac; dar Domnul îi pierduse de la fața Amoniților și-i alungaseră aceștia și se așezaseră în locul lor.
\par 22 Astfel a făcut Domnul pentru fiii lui Isav care locuiau în Seir, când a prăpădit de la fața lor pe Horei. Și după ce ei au fost izgoniți, s-au așezat în locul lor, unde trăiesc și astăzi.
\par 23 Și pe Hevei, care locuiau prin sate chiar până la Gaza, i-au pierdut Caftorimii, care se trăgeau din Caftorim, și s-au așezat în locul lor.
\par 24 Sculați-vă și vă porniți și treceți râul Arnon, că iată Eu voi da în mâna ta pe Amoreul Sihon, regele Heșbonului și rara lui; începe a-l cuprinde și du război cu el.
\par 25 Din ziua aceasta voi începe Eu a împrăștia înaintea ta frică și groază peste popoare, sub tot cerul; cei ce vor auzi de tine se vor cutremura și se vor îngrozi de tine.
\par 26 Din pustiul Chedemot am trimis soli la Sihon, regele Heșbonului, cu cuvinte de pace, ca să spună:
\par 27 Îngăduie-mi să trec prin țara ta, că voi merge pe drum și nu mă voi abate nici la dreapta, nici la stânga;
\par 28 Hrană să-mi vinzi pe bani și voi mânca, și apă de băut să-mi dai pe bani și voi bea,
\par 29 Cum mi-au făcut fiii lui Isav, care locuiesc în Seir, și Moabiții, care locuiesc Arul; numai cu piciorul meu voi merge până voi trece Iordanul în pământul pe care Domnul Dumnezeul nostru ni-l dă nouă.
\par 30 Dar Sihon, regele Heșbonului, n-a voit a ne îngădui să trecem prin pământul lui, pentru că Domnul Dumnezeul tău a îndărătnicit duhul lui și inima lui a împietrit-o, ca să-l dea în mâinile tale, cum se vede acum.
\par 31 Atunci mi-a zis Domnul: "Iată, încep să-ți dau pe Sihon Amoreul, regele Heșbonului, și pământul lui; începe a stăpâni pământul lui".
\par 32 Iar Sihon, regele Heșbonului, cu tot poporul său, ne-a ieșit înainte să se lupte la Iahaț.
\par 33 Dar Domnul Dumnezeul nostru 1-a dat în mâinile noastre și l-am bătut pe el și pe fiii lui și tot poporul lui.
\par 34 În vremea aceea am luat toate cetățile lui și am nimicit toate cetățile lui, bărbați, femei și copii, și n-am lăsat pe nimeni viu.
\par 35 Numai vitele lor și cele jefuite din cetățile cuprinse de noi ni le-am luat.
\par 36 De la Aroer, care se află pe malul râului Arnon, și de la cetatea cea din vale până la muntele Galaad, n-a mai fost cetate în care noi să nu fi putut pătrunde: Domnul Dumnezeu a dat tot în mâinile noastre.
\par 37 Numai de pământul Amoniților nu te-ai apropiat, nici de locurile ce se întind în apropierea râului Iaboc, nici de cetățile ce sunt pe munte, nici de nimic ce nu ne-a poruncit Domnul Dumnezeul nostru".

\chapter{3}

\par 1 "Ne-am întors apoi de acolo și am mers către Vasan, însă ne-a ieșit înainte cu război Og, regele Vasanului, la Edrei, cu tot poporul său.
\par 2 Dar Domnul mi-a zis: Nu te teme de el, căci îl voi da în mâinile tale pe el și tot poporul lui și tot pământul lui, și vei face cu el ce-ai făcut cu Sihon, regele Amoreilor, care a trăit în Heșbon.
\par 3 Domnul Dumnezeul nostru a dat în mâinile noastre și pe Og, regele Vasanului, cu tot poporul lui, și noi l-am bătut, încât nimeni de la ei n-a rămas viu.
\par 4 În vremea aceea am luat toate cetățile lui, că n-a fost cetate pe care să n-o luăm de la ei. Am luat șaizeci de cetăți, toată latura Argob, țara lui Og al Vasanului.
\par 5 Toate cetățile acestea erau întărite cu ziduri înalte, cu porți și cu încuietori, afară de cetățile neîntărite care erau foarte multe.
\par 6 Și le-am nimicit, cum făcusem și cu Sihon, regele Heșbonului, pierzând fiecare cetate cu bărbați, femei și copii.
\par 7 Iar toate vitele și cele jefuite prin cetăți ni le-am luat ca pradă.
\par 8 Am luat în vremea aceea din mâinile celor doi regi amorei pământul acesta care este dincoace de Iordan, de la râul Arnon până la muntele Hermon.
\par 9 (Sidonienii numesc Hermonul, Sirion, iar Amoreii îl numesc Senir).
\par 10 Am luat adică toate cetățile din șes, tot Galaadul și tot Vasanul, până la Salca și Edrei, cetățile din țara lui Og al Vasanului.
\par 11 Căci numai Og, regele Vasanului, mai rămăsese din Refaimi. Iată patul lui, pat de fier, și astăzi este în Rabat-Amon: lung de nouă coți și lat de patru coți, coți bărbătești.
\par 12 Pământul acesta l-am luat atunci începând de la Aroer, care este lângă râul Arnon; jumătate din muntele Galaadului cu cetățile lui l-am dat semințiilor lui Ruben și Gad;
\par 13 Iar rămășița cealaltă din Galaad și tot Vasanul, țara lui Og, le-am dat la jumătate din seminția lui Manase; tot ținutul Argob, eu tot Vasanul se numește țara Refaimilor.
\par 14 Iair, fiul lui Manase, a luat tot ținutul Argob, până la hotarele Gheșuriților și Maacatiților, și a numit Vasanul, după numele său, sălașurile lui Iair, cum se cheamă și astăzi.
\par 15 Lui Machir i-am dat Galaadul;
\par 16 Iar semințiilor lui Ruben și Gad le-am dat țara de la Galaad până la râul Arnon, pământul dintre râu și hotar, până la râul Iaboc, până la hotarul fiilor lui Amon.
\par 17 Precum și Arabah și Iordanul, care este hotar de la Chineret până la marea Arabah, Marea Sărată, la poalele Muntelui Fazga, spre răsărit.
\par 18 V-am mai dat în vremea aceea poruncă și am zis: Domnul Dumnezeul vostru v-a dat pământul acesta în stăpânire; toți cei buni de luptă, înarmându-vă, mergeți înaintea fiilor lui Israel, frații voștri;
\par 19 Numai femeile voastre, copiii voștri și vitele voastre, că știu că aveți vite multe, să rămână în cetățile voastre pe care vi le-am dat eu,
\par 20 Până când Domnul Dumnezeu va da liniște fraților voștri, ca și vouă, și până când își vor primi și ei în stăpânire pământul pe care Domnul Dumnezeul vostru li-l va da peste Iordan; atunci vă veți întoarce fiecare la moșia sa pe care v-am dat-o eu.
\par 21 Iar lui Iosua i-am poruncit și am zis: Ochii tăi au văzut tot ceea ce a făcut Domnul Dumnezeul vostru cu acești doi regi; tot așa va face Domnul și cu toate țările ce le vei străbate tu.
\par 22 Nu te teme de ele, că Domnul Dumnezeul vostru Însuși se va lupta pentru voi.
\par 23 În vremea aceea m-am rugat Domnului și am zis:
\par 24 Stăpâne Doamne, ai început să ară)i robului Tău slava Ta, puterea Ta, mâna Ta cea tare și brațul cel înalt, că cine este Dumnezeu în cer sau pe pământ, Care să facă astfel de lucruri cum sunt ale Tale și cu o asemenea putere ca a Ta.
\par 25 Îngăduie-mi să trec și să văd pământul cel bun care este peste Iordan și acel munte frumos, Libanul.
\par 26 Dar Domnul s-a rnâniat pe mine pentru voi și nu m-a ascultat, ei mi-a zis Domnul: Ajunge! De acum să nu-Mi mai grăiești de aceasta!
\par 27 Suie-te în vârful lui Fazga și privește cu ochii tăi spre apus și spre miazănoapte și spre miazăzi și spre răsărit și vezi cu ochii tăi, căci nu vei trece peste acest Iordan.
\par 28 Dă povață lui Iosua; întărește-l și-l îmbărbătează, căci el va merge înaintea acestui popor și el va împărți în bucăți tot pământul, pe care-l vezi acum.
\par 29 Atunci ne-am oprit noi în vale, în fața Bet-Peorului".

\chapter{4}

\par 1 "Ascultă dar, Israele: hotărârile și legile care vă învăț eu astăzi să le păziți, ca să fiți vii, să vă înmulțiți și ca să vă duceți să moșteniți acel pământ pe care Domnul Dumnezeul părinților voștri vi-l dă în stăpânire
\par 2 Să nu adăugați nimic la cele ce vă poruncesc eu, nici să lăsați ceva din ele; păziți poruncile Domnului Dumnezeului vostru, pe care vi le spun eu astăzi.
\par 3 Ochii voștri au văzut toate câte a făcut Domnul Dumnezeul vostru cu Baal-Peor; pe tot omul care a urmat lui Baal-Peor 1-a pierdut Domnul Dumnezeul tău din mijlocul tău;
\par 4 Iar voi, cei ce v-ați lipit de Domnul Dumnezeul vostru sunteți vii până în ziua de astăzi.
\par 5 Iată, v-am învățat porunci și legi, cum îmi poruncise Domnul Dumnezeul meu, ca să faceți așa în țara aceea în care intrați ca s-o stăpâniți.
\par 6 Să le păziți așadar și să le împliniți, căci în aceasta stă înțelepciunea voastră și cumințenia voastră înaintea ochilor popoarelor care, auzind de toate legiuirile acestea, vor zice: Numai acest popor mare este popor înțelept și priceput.
\par 7 Căci este oare vreun popor mare, de care dumnezeii lui să fie așa de aproape, cît de aproape este de noi Domnul Dumnezeul nostru, oricând Îl chemăm?
\par 8 Sau este vreun popor mare, care să aibă astfel de așezăminte și legi drepte, cum este toată legea aceasta pe care v-o înfățișez eu astăzi?
\par 9 Decât numai păzește-te și îți ferește cu îngrijire sufletul tău, ca să nu uiți acele lucruri pe care le-au văzut ochii tăi și să nu-ți iasă ele de la inimă în toate zilele vieții tale; să le spui fiilor și fiilor feciorilor tăi.
\par 10 Să le spui de ziua aceea în care ai stat tu înaintea Domnului Dumnezeului tău, la Horeb, de ziua adunării când a zis Domnul către mine: Adună la Mine poporul și Eu îi voi vesti cuvintele Mele, din care se vor învăța ei a se teme de Mine în toate zilele vieții lor de pe pământ și vor învăța pe fiii lor.
\par 11 Atunci v-ați apropiat și ați stat sub munte, muntele ardea cu foc până la cer și era negură, nor și întuneric.
\par 12 Iar Domnul v-a grăit de pe munte din mijlocul focului; și glasul cuvintelor Lui l-ați auzit, iar fața Lui n-ați văzut-o, ci numai glasul I l-ați auzit.
\par 13 Atunci v-a descoperit El legământul Său, cele zece porunci, pe care v-a poruncit să le împliniți, și le-a scris pe două lespezi de piatră.
\par 14 În vremea aceea mi-a poruncit Domnul să vă învăț poruncile și legile Lui, ca să le împliniți în țara aceea în care intrați ca s-o stăpâniți.
\par 15 Țineți dar bine minte că în ziua aceea, când Domnul v-a grăit din mijlocul focului, de pe muntele Horeb, n-ați văzut nici un chip.
\par 16 Să nu greșiți dar și să nu vă faceți chipuri cioplite, sau închipuiri ale vreunui idol, care să înfățișeze bărbat sau femeie,
\par 17 Sau închipuirea vreunui dobitoc de pe pământ, sau închipuirea vreunei păsări ce zboară sub cer,
\par 18 Sau închipuirea vreunei jivine, ce se târăște pe pământ, sau închipuirea vreunui pește din apă, de sub pământ;
\par 19 Sau, privind la cer și văzând soarele, luna, stelele și toată oștirea cerului, să nu te lași amăgit ca să te închini lor, nici să le slujești, pentru că Domnul Dumnezeul tău le-a lăsat pentru toate popoarele de sub cer.
\par 20 Iar pe voi v-a luat Domnul Dumnezeu și v-a scos din cuptorul cel de fier, din țara Egiptului, ca să-I fiți Lui popor de moștenire, cum sunteți acum.
\par 21 Domnul Dumnezeu S-a mâniat însă pe mine pentru voi și S-a jurat că eu nu voi trece Iordanul acesta și nu voi intra în acel pământ bun pe care Domnul Dumnezeul tău ți-l dă ție de moștenire.
\par 22 Eu voi muri în pământul acesta, fără să trec Iordanul, iar voi veți trece și veți lua în stăpânire acel pământ bun.
\par 23 Luați seama să nu uitați legământul Domnului Dumnezeului vostru pe care 1-a încheiat cu voi și să nu vă faceți idoli care ar închipui ceva, precum ți-a poruncit Domnul Dumnezeul tău.
\par 24 Căci Domnul Dumnezeul tău este foc mistuitor, Dumnezeu zelos.
\par 25 Iar de ți se vor naște fii și fiilor tăi fii și, trăind mult pe pământ, veți cădea în păcat și vă veți face chipuri cioplite, care să închipuiască ceva, de veți face ceva ce este rău înaintea ochilor Domnului Dumnezeului vostru ca să-L mâniați,
\par 26 Vă mărturisesc astăzi pe cer și pe pământ că veți pierde curând pământul, pentru a cărui moștenire treceți acum Iordanul; nu veți trăi multă vreme pe el, ci veți pieri.
\par 27 Domnul vă va împrăștia prin toate popoarele și veți rămâne puțini la număr printre toate popoarele la care vă va duce Domnul.
\par 28 Veți sluji acolo altor dumnezei, făcuți de mâini omenești din lemn și piatră, care nu văd și nu aud, care nu mănâncă și nu au miros.
\par 29 Dar când vei căuta acolo pe Domnul Dumnezeul tău, Îl vei găsi, de-L vei căuta cu toată inima ta și cu tot sufletul tău.
\par 30 Când vei fi la necaz și când te vor ajunge toate acestea în curgerea vremii, te vei întoarce la Domnul Dumnezeul tău și vei asculta glasul Lui.
\par 31 Domnul Dumnezeul tău este Dumnezeu bun și îndurat; nu te va lăsa, nu te va pierde și nici nu va uita legământul încheiat cu părinții tăi, pe care l-a întărit cu jurământ înaintea lor.
\par 32 Că cercetează timpurile trecute, care au fost înainte de tine, din ziua când a făcut Dumnezeu pe om pe pământ; cercetează de la o margine a cerului până la cealaltă margine a lui și vezi dacă s-a mai săvârșit vreo faptă mare ca aceasta și dacă s-a mai auzit ceva la fel!
\par 33 A mai auzit, oare, vreun popor glasul Dumnezeului celui viu, grăind din mijlocul focului, cum ai auzit tu și să rămână viu?
\par 34 Sau a mai încercat, oare, vreun dumnezeu să se ducă să-și ia popor din mijlocul altui popor prin plăgi, prin semne, prin vedenii și prin război, cu mână tare și cu braț înalt și prin minuni mari, cum a făcut pentru voi Domnul Dumnezeul vostru în Egipt, înaintea ochilor voștri?
\par 35 Ție, Israele, ți s-a dat să vezi aceasta, ca să știi că numai Domnul Dumnezeul tău este Dumnezeu și nu mai este altul afară de El.
\par 36 Din cer te-a învrednicit să auzi glasul Lui, ca să te învețe, și pe pământ ți-a arătat focul Lui cel mare și ai auzit cuvintele Lui din mijlocul focului.
\par 37 Pentru că El a iubit pe părinții tăi și v-a ales pe voi, urmașii lor, de aceea te-a Și scos cu puterea Lui cea mare din Egipt.
\par 38 Ca să alunge de la fața ta popoarele cele mai mari și mai puternice decât tine și ca să te ducă să-ți dea pământul lor moștenire, așa cum ai astăzi.
\par 39 Cunoaște dar astăzi și ține minte că Domnul Dumnezeul tău este Dumnezeu sus în cer și jos pe pământ și nu mai este altul afară de El.
\par 40 Să păzești legile Lui și poruncile Lui, pe care ți le spun eu astăzi, ca să-ți fie bine ție și fiilor tăi de după tine și ca să trăiești multă vreme în pământul acela pe care Domnul Dumnezeul tău ți-l dă pentru totdeauna".
\par 41 Atunci a ales Moise trei cetăți dincoace de Iordan, spre răsăritul Soarelui, ca să fugă acolo ucigașul
\par 42 Care va ucide pe aproapele său fără de voie și fără să-i fi fost vrăjmaș nici cu o zi, nici cu două înainte, și ca, scăpând în una din aceste cetăți, să rămână cu viață.
\par 43 Aceste cetăți sunt: Bețer în pustie, în câmpia din seminția lui Ruben, Ramot în Galaad, în seminția lui Gad, și Golan în Vasan, în seminția lui Manase.
\par 44 Iată legea pe care a înfățișat-o Moise fiilor lui Israel.
\par 45 Și iată poruncile, legile și îndreptările pe care le-a rostit Moise fiilor lui Israel în pustie, după ieșirea lor din Egipt,
\par 46 Dincolo de Iordan, în valea din fața Bet-Peorului, în pământul lui Sihon, regele Amoreilor, care a trăit în Heșbon, pe care l-a bătut Moise cu fiii lui Israel, după ieșirea lor din Egipt,
\par 47 Și au moștenit pământul lui și pământul lui Og, regele Vasanului, țara celor doi regi amorei de peste Iordan, spre răsăritul soarelui,
\par 48 Care se întinde pe malul râului Arnon de la Aroer până la muntele Sihon sau Hermon,
\par 49 Și tot șesul Arabah de dincoace de Iordan, către răsăritul soarelui, până la mare, șesul Arabah de la poalele muntelui Fazga.

\chapter{5}

\par 1 În vremea aceea a chemat Moise tot Israelul și le-a zis: "Ascultă, Israele, poruncile și legile pe care le voi rosti eu astăzi în auzul urechilor voastre: învățați-le și siliți-vă să le pliniți.
\par 2 Domnul Dumnezeul vostru a încheiat cu voi legământ în Horeb.
\par 3 Legământul acesta nu 1-a încheiat Domnul cu părinții noștri, ci cu noi, cei ce suntem astăzi cu toții vii aici.
\par 4 Față către față a grăit Domnul cu voi din mijlocul focului de pe munte;
\par 5 Iar eu am stat în vremea aceea între Domnul și între voi, ca să vă spun Cuvântul Domnului, căci voi v-ați temut de foc și nu v-ați suit în munte, -- și a zis Domnul:
\par 6 "Eu sunt Domnul Dumnezeul tău, Care te-am scos din pământul Egiptului, din casa robiei.
\par 7 Să nu ai alți dumnezei afară de Mine.
\par 8 Să nu-ți faci chip cioplit, nici vreo înfățișare a celor ce sunt sus în cer, sau jos pe pământ, sau în apă și sub pământ.
\par 9 Să, nu te închini lor, nici să le slujești, căci Eu Domnul Dumnezeul tău sunt Dumnezeu zelos, Care pedepsește vina părinților în copii până la al treilea și al patrulea neam pentru cei ce Mă urăsc.
\par 10 Și Mă milostivesc până la a! miilea neam către cei ce Mă iubesc și păzesc poruncile Mele.
\par 11 Să nu iei numele Domnului Dumnezeului tău în deșert, că nu va lăsa Domnul Dumnezeul tău nepedepsit pe cel ce ia numele Lui în deșert.
\par 12 Păzește ziua odihnei, ca să o ții cu sfințenie, cum ți-a poruncit Domnul Dumnezeul tău.
\par 13 Șase zile lucrează și-ți fă toate treburile tale;
\par 14 Ziua a șaptea este ziua de odihnă a Domnului Dumnezeului tău. Să nu faci în ziua aceea nici un lucru: nici tu, nici fiul tău, nici fiica ta, nici robul tău, nici roaba ta, nici boul tău, nici asinul tău, sau alt dobitoc al tău, nici străinul tău care se află la tine, ca să se odihnească robul tău și roaba ta cum te odihnești și tu.
\par 15 Adu-ți aminte că ai fost rob în pământul Egiptului și Domnul Dumnezeul tău te-a scos de acolo cu mină tare și cu braț înalt și de aceea îi-a poruncit Domnul Dumnezeul tău să păzești ziua odihnei și să o ții cu sfințenie.
\par 16 Cinstește pe tatăl tău și pe mama ța, cum ți-a poruncit Domnul Dumnezeul tău, ca să trăiești ani mulți și să-ți fie bine în pământul acela, pe care Domnul Dumnezeul tău ți-l dă ție.
\par 17 Să nu ucizi!
\par 18 Să nu fii desfrânat!
\par 19 Să nu furi!
\par 20 Să nu dai mărturii mincinoase asupra aproapelui tău!
\par 21 Să nu poftești femeia aproapelui tău și să nu dorești casa aproapelui tău, nici țarina lui, nici robul lui, nici roaba lui, nici boul lui, nici asinul lui, nici orice dobitoc al lui, nici nimic din cele ce sunt ale aproapelui tău!
\par 22 Cuvintele acestea le-a grăit Domnul către toată adunarea voastră, pe munte, din mijlocul focului, al norului, al întunericului și al furtunii, cu glas de tunet și altceva n-a mai grăit și le-a scris pe două lespezi de piatră și mi le-a dat mie.
\par 23 Și când ați auzit glasul din mijlocul întunericului și muntele ardea, v-ați apropiat de mine toate căpeteniile semințiilor voastre cu bătrânii voștri
\par 24 Și ați zis: "Iată, Domnul Dumnezeul nostru ne-a arătat slava Sa și măreția Sa și glasul Lui l-am auzit din mijlocul focului. Astăzi. am văzut că Dumnezeu grăiește cu omul și acesta rămâne viu.
\par 25 Și de ce să murim noi? Că focul acesta ne va mistui și de vom mai auzi glasul Domnului Dumnezeului nostru vom muri.
\par 26 Căci este oare vreun om care să audă glasul Dumnezeului celui viu grăind din mijlocul focului, cum am auzit noi, și să rămână viu?
\par 27 Apropie-te dar tu și ascultă toate câte-ți va spune Domnul Dumnezeul nostru și apoi ne vei spune tu nouă toate câte-ți va grăi Domnul Dumnezeul nostru, și noi vom asculta și vom face.
\par 28 Iar Domnul a auzit cuvintele voastre, cum grăiați cu mine, și mi-a zis Domnul: Am auzit cuvintele poporului acestuia, pe care le-au grăit către tine, și tot ce-au grăit este bine.
\par 29 O, de ar fi inima lor așa, ca să se teamă de Mine și să păzească toate poruncile Mele în toată vremea, ca să le fie bine și lor și fiilor lor în veci!
\par 30 Du-te și le spune: întoarceți-vă în corturile voastre!
\par 31 Iar tu rămâi aici cu Mine; și-ți voi spune toate poruncile, hotărârile și legile ce trebuie să-i înveți ca să le păzească în pământul acela, pe care li-l dau Eu în stăpânire.
\par 32 Și să le zici: Vedeți, să vă purtați așa cum v-a poruncit Domnul Dumnezeul vostru și să nu vă abateți nici la dreapta, nici la stânga!
\par 33 Să umblați pe calea aceea pe care v-a poruncit Domnul Dumnezeul vostru, ca să fiți vii și să vă fie bine și să trăiți vreme multă în pământul acela pe care îl veți lua în stăpânire".

\chapter{6}

\par 1 "Iată poruncile, hotărârile și legile pe care mi-a poruncit Domnul Dumnezeul vostru să vă învăț, ca să le păziți în pământul acela în care mergeți, ca să-l luați în stăpânire:
\par 2 Să te temi de Domnul Dumnezeul tău și toate hotărârile Lui și poruncile Lui, pe care ti le spun eu astăzi, să le păzești tu și fiii tăi și fiii fiilor tăi, în toate zilele vieții tale, ca să se înmulțească zilele tale.
\par 3 Ascultă dar, Israele, și silește-te să împlinești acestea, ca să-ți fie bine Și să vă înmulțiți foarte, precum ți-a grăit Domnul Dumnezeul părinților tăi că-li va da pământul unde curge lapte și miere. Acestea sunt hotărârile și legile pe care le-a dat Domnul Dumnezeu fiilor lui Israel în pustie, după ieșirea lor din pământul Egiptului.
\par 4 Ascultă, Israele, Domnul Dumnezeul nostru este singurul Domn.
\par 5 Să iubești pe Domnul Dumnezeul tău, din toată inima ta, din tot sufletul tău și din toată puterea ta.
\par 6 Cuvintele acestea, pe care ți le spun eu astăzi, să le ai în inima ta și în sufletul tău;
\par 7 Să le sădești în fiii tăi și să vorbești de ele când șezi în casa ta, când mergi pe cale, când te culci și când te scoli.
\par 8 Să le legi ca semn la mână și să le ai ca pe o tăbliță pe fruntea ta.
\par 9 Să le scrii pe ușorii casei tale și pe porțile tale.
\par 10 Iar când te va duce Domnul Dumnezeul tău în pământul acela pentru care s-a jurat părinților tăi: lui Avraam, lui Isaac și lui Iacov, ca să ți-l dea cu cetăți mari și frumoase, pe care nu le-ai zidit tu,
\par 11 Cu case pline de toate bunătățile, pe care nu le-ai umplut tu, cu fântâni săpate în piatră, pe care nu le-ai săpat tu, cu vii și cu măslini, pe care nu le-ai sădit tu, și vei mânca și te vei sătura,
\par 12 Atunci, păzește-te, să nu se ademenească inima ta, ca să uiți pe Domnul Care te-a scos din pământul Egiptului și din casa robiei.
\par 13 Să te temi de Domnul Dumnezeul tău și numai Lui să-I slujești, de El să te lipești și pe numele Lui să te juri.
\par 14 Să nu mergeți după alți dumnezei, după dumnezeii popoarelor, care se vor afla împrejurul vostru;
\par 15 Ca să nu se aprindă mânia Domnului Dumnezeului tău asupra ta și să nu te piardă de pe fața pământului, că Domnul Dumnezeul tău, Care se află în mijlocul tău, este Dumnezeu zelos.
\par 16 Să nu ispitiți pe Domnul Dumnezeul vostru, cum L-ați ispitit la Masa.
\par 17 Să țineți bine poruncile Domnului Dumnezeului vostru, legile Lui și hotărârile Lui, pe care vi le-a dat El.
\par 18 Să faci ceea ce este drept și bine înaintea ochilor Domnului Dumnezeului tău, ca să-ți fie bine și ca să intri și să iei în stăpânire pământul pe care Domnul cu jurământ 1-a făgăduit părinților tăi
\par 19 Și ca să alunge El pe toți vrăjmașii tăi de la fața ta, cum a zis Domnul.
\par 20 De te va întreba în viitor fiul tău și va zice: Ce înseamnă aceste porunci, hotărâri și legi pe care vi le-a dat Domnul Dumnezeul vostru? Să-i spui fiului tău:
\par 21 Am fost robi la Faraon în Egipt și Domnul Dumnezeu ne-a scos din Egipt cu mână tare și cu braț înalt.
\par 22 Și a arătat Domnul Dumnezeu semne și minuni mari și pedepse a adus înaintea ochilor noștri, asupra Egiptului, asupra lui Faraon, asupra a toată casa lui și asupra oștirii lui;
\par 23 Iar pe noi ne-a scos de acolo Domnul Dumnezeul nostru ca să ne ducă și să ne dea pământul pentru care s-a jurat părinților noștri că ni-l va da.
\par 24 Atunci ne-a poruncit Domnul să împlinim toate hotărârile acestea și să ne temem de Domnul Dumnezeul nostru, ca să ne fie bine în toate zilele, ca și acum.
\par 25 Și va face milă cu noi de ne vom sili să împlinim toate aceste porunci ale legii înaintea feței Domnului Dumnezeului nostru, cum ne-a poruncit".

\chapter{7}

\par 1 "Când Domnul Dumnezeul tău te va duce în pământul la care mergi ca să-l moștenești și va izgoni de la fața ta neamurile cele mari și multe și anume: pe Hetei, pe Gherghesei, pe Amorei, pe Canaanei, pe Ferezei, pe Hevei și pe Iebusei - șapte neamuri, care sunt mai mari și mai puternice decât tine -
\par 2 Și le va da Domnul Dumnezeul tău în mâinile tale și le vei bate, atunci să le nimicești, să nu faci cu ele legământ și să nu le cruți.
\par 3 Să nu te încuscrești cu ele: pe fiica ta să nu o dai după fiul lui și pe fiica lui să nu o iei pentru fiul tău,
\par 4 Că vor abate pe fiii tăi de la Mine ca să slujească altor dumnezei, și se va aprinde asupra voastră mânia Domnului și curând te va pierde.
\par 5 Ci să faceți cu ele așa: jertfelnicile lor să le stricați, stâlpii lor să-i dărâmați, dumbrăvile lor să le tăiați și idolii dumnezeilor lor să-i ardeți cu foc.
\par 6 Că ești poporul sfânt al Domnului Dumnezeului tău și te-a ales Domnul Dumnezeul tău ca să-I fii poporul Lui ales din toate popoarele de pe pământ.
\par 7 Și Domnul v-a primit, nu pentru că sunteți mai mulți la număr decât toate popoarele - căci sunteți mai puțini la număr decât toate popoarele, -
\par 8 Ci pentru că vă iubește Domnul; și ca să Își țină jurământul pe care 1-a făcut părinților voștri, v-a scos Domnul cu mână tare și cu braț înalt și v-a scăpat din casa robiei și din mâna lui Faraon, regele Egiptului.
\par 9 Să știi dar că Domnul Dumnezeul tău este adevăratul Dumnezeu, Dumnezeu credincios, Care păzește legământul Său și mila Sa, până la al miilea neam, către cei ce-L iubesc și păzesc poruncile Lui;
\par 10 Și răsplătește la fel celor ce-L urăsc, pierzându-i, și nu întârzie să răsplătească, eu aceeași măsură, celor ce-L urăsc.
\par 11 Păzește dar poruncile, hotărârile și legile pe care-ți poruncesc astăzi să le împlinești.
\par 12 De vei asculta legile acestea, de le vei păzi și le vei împlini, atunci și Domnul Dumnezeul tău va ține legământul și mila Sa față de tine, cum S-a jurat El părinților tăi;
\par 13 Te va iubi, te va binecuvânta, te va înmulți și va binecuvânta rodul pântecelui tău, rodul pământului tău, pâinea ta, vinul tău, untdelemnul tău, pe cele născute ale vitelor tale mari și ale oilor turmei tale în pământul acela, pentru care S-a jurat El părinților tăi să ți-l dea ție.
\par 14 Și vei fi binecuvântat mai mult decât toate popoarele și nu se va afla sterp sau stearpă nici între ai tăi, nici între dobitoacele tale.
\par 15 Va depărta de la tine Domnul Dumnezeul tău toată neputința și nici una din bolile cele rele ale Egiptenilor, pe care le-ai văzut și le știi, nu va aduce asupra ta, ci le va trimite asupra celor ce te urăsc pe tine
\par 16 Mânca-vei toată agonisita popoarelor pe care Domnul Dumnezeul tău ți le va da ție; să nu le cruțe ochiul tău și să nu slujești dumnezeilor lor, că aceasta este cursă pentru tine.
\par 17 Nu cumva să zici în inima ta: Popoarele acestea sunt mai mari la număr decât mine, cum le voi putea izgoni?
\par 18 Să nu te temi de ele, ci adu-ți aminte ce a făcut Domnul Dumnezeul tău cu Faraon și cu tot Egiptul,
\par 19 Și de acele încercări mari, pe care le-au văzut ochii tăi, de acele semne și minuni mari, de mâna cea tare și de brațul cel înalt cu care te-a scos Domnul Dumnezeul tău. Tot așa va face Domnul Dumnezeul tău cu toate popoarele de care te temi.
\par 20 Încă și viespi va trimite Domnul Dumnezeul tău asupra lor până ce vor pieri cei ce au rămas și s-au ascuns de la fața ta.
\par 21 Nu te înspăimânta de ei, că Domnul Dumnezeul tău, Cel din mijlocul tău, este Dumnezeu mare și minunat.
\par 22 Domnul Dumnezeul tău va izgoni dinaintea ta popoarele acestea încetul cu încetul; nu poți să le pierzi repede, ca să nu se pustiiască pământul și să nu se înmulțească împotriva ta fiarele câmpului;
\par 23 Ci ți le va da Domnul Dumnezeul tău și le va pune în mare tulburare, încât vor pieri.
\par 24 Va da pe regii lor în mâinile tale și tu vei pierde numele lor de sub cer: nimeni nu-ți va putea sta înainte până îi vei stârpi.
\par 25 Idolii dumnezeilor lor să-i ardeți cu foc; să nu dorești a lua argintul sau aurul de pe ei, ca să nu-ți fie aceasta cursă, că urâciune sunt aceștia înaintea Domnului Dumnezeului tău,
\par 26 Și urâciunea idolească să n-o duci în casa ta, ca să nu cazi sub blestem, ca ea. Ferește-te de aceasta și să-ți fie scârbă de ea, că este blestemată".

\chapter{8}

\par 1 "Siliți-vă să împliniți toate poruncile acestea pe care vi le dau eu astăzi, ca să fiți vii și să vă înmulțiți și să vă duceți să luați în stăpânire pământul cel bun pe care cu jurământ 1-a făgăduit Domnul Dumnezeul părinților voștri.
\par 2 Să-ți aduci aminte de toată calea pe care te-a povățuit Domnul Dumnezeul tău prin pustie de acum patruzeci de ani, ca să te smerească, ca să te încerce și ca să afle ce este în inima ta și de ai să păzești sau nu poruncile Lui.
\par 3 Te-a smerit, te-a pedepsit cu foamea și te-a hrănit cu mana pe care nu o cunoșteai și pe care nu o cunoșteau nici părinții tăi, ca să-ți arate că nu numai cu pâine trăiește omul, ci că omul trăiește și cu tot Cuvântul ce iese din gura Domnului.
\par 4 Haina ta nu s-a învechit pe tine și piciorul tău n-a căpătat bătături în acești patruzeci de ani.
\par 5 Dar să știi în inima ta că Domnul Dumnezeul tău te învață, cum învață omul pe fiul său.
\par 6 Păzește dar poruncile Domnului Dumnezeului tău, umblând în căile Lui și temându-te de El.
\par 7 Că Domnul Dumnezeul tău te va duce într-o țară bună, țară de curgeri de apă, de izvoare și de ape adânci, care țâșnesc în văi și în munți;
\par 8 Țară în care se află: grâu, orz, viță de vie, smochine și rodii;
\par 9 Într-o țară unde sunt măslini, untdelemn și miere, în care fără lipsă vei mânca pâinea ta și nu vei duce lipsă de nimic; în care pietrele au fier și din munții căreia vei scoate aramă.
\par 10 Când însă vei mânca și te vei sătura, să binecuvântezi pe Domnul Dumnezeul tău pentru țara cea bună pe care ți-a dat-o.
\par 11 Ferește-te de a uita pe Domnul Dumnezeul tău și de a nu păzi poruncile Lui, legile Lui și hotărârile Lui pe care ti le spun eu astăzi.
\par 12 Când vei mânca și te vei sătura și îți vei face case frumoase și vei trăi în ele;
\par 13 Când vei avea multe vite mari și mărunte și mult argint și aur și vei avea de toate la tine,
\par 14 Vezi să nu se mândrească inima ta și să nu uiți pe Domnul Dumnezeul tău Care te-a scos din Egipt, din casa robiei;
\par 15 Care te-a povățuit prin pustiul cel mare și groaznic, unde sunt șerpi veninoși, scorpioni și locuri arse de soare și fără de apă;
\par 16 Care a scos pentru tine izvor din stânca de cremene, te-a hrănit în pustie cu mana pe care tu n-o cunoșteai și n-o cunoșteau nici părinții tăi, ca să te smerească și să te încerce,
\par 17 Ca să-ți facă bine în urmă și ca să nu zici în inima ta: Puterea mea și tăria mâinii mele mi-au adus bogăția aceasta;
\par 18 Ci ca să-ți aduci aminte de Domnul Dumnezeul tău, că El îți dă putere să faci bogăție, ca să-Și țină, ca acum, legământul Lui, pe care cu jurământ l-a întărit cu părinții tăi.
\par 19 Iar dacă vei uita pe Domnul Dumnezeul tău și vei merge după dumnezeii altora, vei sluji acelora și te vei închina lor, vă mărturisesc astăzi pe cer și pe pământ că veți pieri.
\par 20 Cum pier popoarele, pe care Domnul Dumnezeu le pierde dinaintea voastră, așa veți pieri și voi, de nu veți asculta glasul Domnului Dumnezeului vostru!"

\chapter{9}

\par 1 "Ascultă, Israele, de acum tu vei trece Iordanul, ca să intri și să cuprinzi popoare mai mari și mai puternice decât tine și cetăți mari cu ziduri până la cer,
\par 2 Precum și pe poporul cei mare, mult la număr și înalt la statură, pe fiii lui Enac, de care tu ai auzit spunându-se: Cine se va împotrivi fiilor lui Enac?"
\par 3 Află dar astăzi că Domnul Dumnezeul tău merge înaintea ta. Acesta este foc mistuitor: pierde-i-va și-i va doborî înaintea ta, și tu îi vei izgoni și-i vei omorî curând, cum ji-a grăit Domnul.
\par 4 Când îi va izgoni Domnul Dumnezeul tău de la fața ta, să nu zici în inima ta: "Pentru dreptatea mea m-a adus Domnul să stăpânesc pe acest pământ bun", căci pentru necredincioșia popoarelor acestora le izgonește Domnul de la fața ta.
\par 5 Nu pentru dreptatea ta și nici pentru dreptatea inimii tale mergi să moștenești pământul lor, ci pentru necredința și fărădelegile popoarelor acestora le izgonește Domnul Dumnezeul tău de la fața ta și ca să împlinească făgăduința cu care S-a jurat Domnul părinților tăi: lui Avraam, lui Isaac și lui Iacov.
\par 6 De aceea să știi astăzi că nu pentru dreptatea ta îți dă Domnul Dumnezeul tău să moștenești acest pământ bun, că tu ești un popor tare la cerbice.
\par 7 Ține minte și nu uita de câte ori ai mâniat pe Domnul Dumnezeul tău în pustie; din ziua când ați ieșit din pământul Egiptului și până ce au sosit la locul acesta, necontenit v-ați împotrivit Domnului.
\par 8 La Horeb ați mâniat pe Domnul și ați pornit pe Domnul asupra voastră, așa încât a vrut să vă piardă.
\par 9 Când m-am suit eu pe munte, ca să primesc lespezile de piatră, tablele legământului, pe care 1-a încheiat Domnul cu voi, am stat în munte patruzeci de zile și patruzeci de nopți,
\par 10 Și nici pâine n-am mâncat, nici apă n-am băut. Atunci mi-a dat Domnul două table de piatră, scrise cu degetul lui Dumnezeu; pe acelea erau scrise toate cuvintele pe care vi le-a grăit Domnul pe munte, din mijlocul focului, în ziua adunării.
\par 11 Dar după trecerea celor patruzeci de zile și patruzeci de nopți, când mi-a dat Domnul cele două table de piatră, tablele legământului, mi-a zis Domnul:
\par 12 "Scoală și te pogoară repede de aici, că s-a răzvrătit poporul tău pe care l-ai scos din Egipt; curând s-a abătut el de la calea pe care i-am poruncit să meargă și și-a făcut chip turnat".
\par 13 Tot atunci mi-a mai zis Domnul: "De mai multe ori ti-am grăit și ți-am zis: Mă uit la poporul acesta și văd că este popor tare la cerbice.
\par 14 Lasă-Mă dar acum să-l pierd și să șterg numele lui de sub cer și voi ridica din tine popor mai mare, mai puternic și mai mult la număr decât ei".
\par 15 Atunci eu m-am întors și m-am pogorât din munte, iar muntele ardea în foc. Cele două table ale legământului erau în amândouă mâinile mele.
\par 16 Am văzut însă că voi păcătuiserăți înaintea Domnului Dumnezeului vostru, vă făcuserăți un vițel turnat și vă abătuserăți curând de la calea pe care vă poruncise Domnul să o urmați.
\par 17 Am luat atunci cele două table și, aruncându-le cu amândouă mâinile mele, le-am sfărâmat înaintea ochilor voștri.
\par 18 Apoi am îngenunchiat a doua oară înaintea Domnului, ca și întâia oară, patruzeci de zile și patruzeci de nopți, fără să mănânc pâine și fără să beau apă; m-am rugat pentru păcatele voastre cu care ați greșit voi, făcând rău înaintea ochilor Domnului Dumnezeului vostru și mîniindu-L.
\par 19 Că eu am fost îngrozit de mânia și de iuțimea cu care Se mâniase Domnul pe voi, voind să vă piardă. Și m-a auzit Domnul și de data aceasta.
\par 20 Atunci Se mâniase Domnul foarte tare și pe Aaron, vrând să-l piardă și pe el; dar m-am rugat eu în vremea aceea și pentru Aaron.
\par 21 Iar păcatul vostru pe care l-ați făcut, adică vițelul, l-am luat, l-am ars în foc, l-am sfărâmat și l-am pisat bine, până când s-a făcut mărunt ca praful, și. praful acesta l-am aruncat în pârâul ce curgea din munte.
\par 22 La Tabeerah, la Masa și la Chibrot-Hataava voi iarăși ați mâniat pe Domnul Dumnezeul vostru.
\par 23 Când v-a trimis Domnul din Cadeș-Barnea, zicând: Mergeți de luați pământul pe care vi-l dau Eu, v-ați împotrivit poruncii Domnului Dumnezeului vostru și nu l-ați crezut, nici n-ați ascultat glasul Lui.
\par 24 Nesupuși ați fost Domnului chiar din ziua când am început a vă cunoaște.
\par 25 Îngenunchind, așadar, înaintea Domnului, m-am rugat eu patruzeci de zile și patruzeci de nopți, căci Domnul zisese să vă piardă. Eu însă m-am rugat Domnului și am zis:
\par 26 Stăpâne Doamne, Împărate, Dumnezeule, nu pierde pe poporul Tău și moștenirea Ta, pe care l-ai izbăvit cu mărirea puterii Tale și pe care l-ai scos din Egipt cu mână tare și cu brațul Tău cel înalt.
\par 27 Adu-Ți aminte de robii Tăi: Avraam, Isaac și Iacov, cărora Te-ai jurat pe Tine Însuți; nu Te uita la cerbicia poporului acestuia, nici la necredința lui, nici la păcatele lui,
\par 28 Ca cei ce trăiesc în pământul de unde ne-ai scos Tu să nu zică: Domnul nu i-a putut duce în pământul ce le-a făgăduit și, urându-i, i-a scos ca să-i omoare în pustiu.
\par 29 Ei sunt poporul Tău și moștenirea Ta, pe care l-ai scos din pământul Egiptului cu puterea Ta cea mare și cu brațul Tău cel înalt".

\chapter{10}

\par 1 "Atunci mi-a zis Domnul: Cioplește-ți două table de piatră, ca și cele dintâi, și suie-te la Mine în munte și-ți fă un chivot de lemn;
\par 2 Că Eu am să scriu pe tablele acelea cuvintele ce au fost pe tablele cele dintâi, pe care le-ai sfărâmat, iar tu să le pui în chivot.
\par 3 Am făcut atunci un chivot din lemn de salcâm, am cioplit două table de piatră, ca și cele dintâi, și m-am suit în munte cu cele două table în mâinile mele.
\par 4 Iar El a scris pe table, cum fusese scris și pe cele dintâi, cele zece porunci pe care vi le spusese Domnul pe munte din mijlocul focului, în ziua adunării, și mi le-a dat Domnul mie;
\par 5 Iar eu m-am întors și m-am pogorât din munte și am pus tablele în chivotul pe care-l făcusem, ca să stea acolo, cum îmi poruncise Domnul.
\par 6 Apoi au plecat fiii lui Israel din Beerot-Bene-Iaacan la Mosera. Acolo a murit Aaron și a fost îngropat acolo și în locul lui s-a făcut preot Eleazar, fiul lui.
\par 7 De acolo am plecat la Gudgod și din Gudgod la Iotbata, în pământul unde sunt cursuri de apă.
\par 8 În vremea aceea a ales Domnul seminția lui Levi, ca să poarte chivotul legământului Domnului, să stea înaintea Domnului, să-I slujească, să se roage și să-I binecuvânteze numele Lui, cum face până în ziua de astăzi.
\par 9 De aceea n-are levitul parte și moștenire cu frații săi, căci moștenirea lui este Domnul, cum i-a spus Domnul Dumnezeul tău.
\par 10 Deci am stat eu pe munte, ca și întâia oară, patruzeci de zile și patruzeci de nopți; și m-a ascultat Domnul și de astă dată și n-a mai vrut Domnul să te piardă,
\par 11 Ci mi-a zis Domnul: Scoală și mergi înaintea poporului acestuia, ca să intre și să moștenească pământul pentru care M-am jurat părinților lor să li-l dau.
\par 12 Așadar, Israele, ce cere de la tine Domnul Dumnezeul tău? - Numai aceasta: să te temi de Domnul Dumnezeul tău, să umbli în toate căile Lui, să-L iubești și să slujești Domnului Dumnezeului tău, din toată inima ta și din tot sufletul tău;
\par 13 Să păzești poruncile Domnului Dumnezeului tău și hotărârile Lui pe care ți le spun eu astăzi, ca să-ți fie bine.
\par 14 Iată, al Domnului Dumnezeului tău este cerul și cerurile cerurilor, pământul și toate cele de pe el.
\par 15 Dar numai pe părinții tăi i-a primit Domnul și i-a iubit și v-a ales pe voi, sămânța lor de după ei, din toate popoarele, cum vedeți astăzi.
\par 16 Deci să tăiați împrejur inima voastră și de acum înainte să nu mai fiți tari la cerbice;
\par 17 Că Domnul Dumnezeul vostru este Dumnezeul dumnezeilor și Stăpânul stăpânilor, Dumnezeu mare și puternic și minunat, Care nu caută la față, nici nu ia mită;
\par 18 Care face dreptate orfanului și văduvei și iubește pe pribeag și-i dă pâine și hrană.
\par 19 Să iubiți și voi pe pribeag, că și voi ați fost pribegi în pământul Egiptului.
\par 20 De Domnul Dumnezeul tău să te temi, numai Lui să-I slujești, de El să te lipești și cu numele Lui să te juri.
\par 21 El este lauda ta și El este Dumnezeul tău, Cel ce a făcut cu tine acele lucruri mari și înfricoșătoare pe care le-au văzut ochii tăi.
\par 22 Șaptezeci și cinci de suflete erau părinții tăi când au venit în Egipt, iar acum Domnul Dumnezeul tău ți-a sporit numărul ca stelele cerului".

\chapter{11}

\par 1 "Să iubești dar pe Domnul Dumnezeul tău și să păzești în toate zilele cele ce ți-a poruncit El să păzești: hotărârile Lui, legile Lui și poruncile Lui.
\par 2 Băgați de seamă dar că eu nu grăiesc cu copiii voștri, care nu știu și n-au văzut pedeapsa Domnului Dumnezeului vostru, nici slava Lui, nici mâna Lui cea tare, nici brațul Lui cel înalt,
\par 3 Nici semnele Lui, nici lucrurile Lui, pe care le-a făcut în mijlocul Egiptului cu Faraon, regele egiptean, și cu tot pământul lui,
\par 4 Nici ce a făcut El cu oștirea egipteană, cu caii lui și cu carele lui, pe care le-â înecat în apele Mării Roșii, când alergau după voi; și i-a pierdut Domnul Dumnezeu până în ziua de astăzi;
\par 5 Nici ce a făcut El pentru voi în pustie, până când ați ajuns în locul acesta;
\par 6 Nici ce a făcut El cu Datan și Abiron, fiii lui Eliab, fiul lui Ruben, când și-a deschis pământul gura sa și în mijlocul a tot Israelul i-a înghițit pe ei, pe familiile lor, corturile lor și toată averea ce o aveau.
\par 7 Căci ochii voștri au văzut toate lucrurile cele mari ale Domnului, pe care le-a făcut El.
\par 8 De aceea păziți toate poruncile Lui, pe care vi le spun eu astăzi, ca să fiți vii și să vă întăriți, să vă duceți să moșteniți pământul în care veți trece peste Iordan ca să-l stăpâniți;
\par 9 Și ca să trăiți multă vreme în pământul acela pentru care Domnul S-a jurat părinților voștri să li-l dea lor și seminției lor, în pământul unde curge miere și lapte.
\par 10 Căci pământul la care mergi tu ca să-l stăpânești nu este ca pământul Egiptului din care ai ieșit, unde, semănând sămânța, o udai cu ajutorul picioarelor tale, ca pe o grădină de legume.
\par 11 Ci pământul în care treceți ca să-l stăpâniți este o țară cu munți și cu văi și se adapă cu apă din ploaia cerului.
\par 12 Este țara de care poartă grijă Domnul Dumnezeul tău; ochii Domnului Dumnezeului tău sunt necontenit asupra ei, de la începutul anului până la sfârșitul lui.
\par 13 De veți asculta poruncile Mele pe care vi le dau astăzi, zice Domnul, și veți iubi pe Domnul Dumnezeul vostru și-I veți sluji din toată inima și din tot sufletul vostru,
\par 14 Voi da pământului vostru ploaie la vreme, timpurie și târzie, și-ți vei strânge pâinea ta, vinul tău și untdelemnul tău;
\par 15 Voi da iarbă pe câmpia ta pentru dobitoacele tale și vei mânca și te vei sătura.
\par 16 Păziți-vă să nu se mândrească inima voastră și să nu vă abateți, nici să vă apucați să slujiți altor dumnezei și să vă închinați lor.
\par 17 Că atunci se va aprinde mânia Domnului asupra voastră, va închide cerul și nu va fi ploaie și pământul nu-și va da roadele sale; iar voi veți pieri curând de pe pământul cel bun pe care Domnul vi-l dă.
\par 18 Puneți dar aceste cuvinte ale mele în inima voastră și în sufletul vostru; legați-le ca semn la mâna voastră și să le aveți ca pe o tăbliță pe fruntea voastră.
\par 19 Să învățați acestea și pe fiii voștri, grăind de ele când ședeți  acasă și când mergeți pe cale, când vă culcați și când vă sculați.
\par 20 Să le scrieți  pe ușorii caselor voastre și pe porțile voastre,
\par 21 Ca zilele voastre și zilele copiilor voștri în acel pământ bun, pentru care Domnul S-a jurat părinților voștri, să fie atât de multe, câte vor fi zilele cerului deasupra pământului.
\par 22 Că de veți păzi voi toate poruncile acestea, pe care vă poruncesc să le păziți, și de veți iubi pe Domnul Dumnezeul vostru, umblând în toate căile Lui și lipindu-vă de El,
\par 23 Va alunga Domnul toate popoarele acestea de la fața, voastră și veți stăpâni popoare mai mari și mai puternice decât voi.
\par 24 Tot locul pe care va călca piciorul vostru va fi al vostru; de la pustiu până la Liban, de la râul Eufratului și până la marea cea de la asfințit, se vor întinde hotarele voastre.
\par 25 Nimeni nu va putea sta înaintea voastră; Domnul Dumnezeul vostru va aduce frică și cutremur peste tot pământul dinaintea voastră, în care veți călca voi, după cum v-a grăit.
\par 26 Iată, eu vă pun astăzi înainte binecuvântare și blestem:
\par 27 Binecuvântare veți avea dacă veți asculta poruncile Domnului Dumnezeului vostru, pe care vi le spun eu astăzi; iar blestem, dacă nu veți asculta poruncile Domnului Dumnezeului vostru,
\par 28 Ci vă veți abate de la calea pe care v-o poruncesc astăzi și veți merge după dumnezei pe care nu-i știți.
\par 29 Când te va duce Domnul Dumnezeul tău în pământul acela în care mergi ca să-l moștenești, atunci să rostești binecuvântarea pe muntele Garizim și blestemul pe muntele Ebal.
\par 30 Iată, aceștia sunt peste Iordan, în drumul spre asfințitul soarelui, în pământul Canaaneilor, care locuiesc pe șesul Arabah din fața Ghilgalei, aproape de stejarul More.
\par 31 Căci voi veți trece Iordanul, ca să mergeri să luați pământul pe care Domnul Dumnezeul vostru vi-l dă moștenire veșnică și-l veți lua în stăpânire și veți trăi în el.
\par 32 Așadar siliți-vă să păziți toate hotărârile și poruncile Lui pe care vi le pun eu astăzi înainte".

\chapter{12}

\par 1 "Iată hotărârile și legile pe care trebuie să vă siliți a le împlini în pământul ce Domnul Dumnezeul părinților voștri vi-l dă în stăpânire, în toate zilele cît veți trăi în pământul acela.
\par 2 Să pustiiți toate locurile în care popoarele ce le veți supune au slujit dumnezeilor lor, cele din munții înalți, cele de pe dealuri și cele de sub orice copac umbros.
\par 3 Să dărâmați jertfelnicele lor, să stricați stâlpii lor, să arderi cu foc copacii lor, să sfărâmați idolii dumnezeilor lor și să ștergeți numele lor din locurile acelea.
\par 4 Iar Domnului Dumnezeului vostru să nu-I faceți așa;
\par 5 Ci la locul pe care-l va alege Domnul Dumnezeul vostru din toate semințiile voastre, ca să-Și pună numele Său asupra lui, să veniți să-l cercetați.
\par 6 Acolo să aduceți arderile de tot ale voastre și jertfele voastre, zeciuielile voastre și ridicarea mâinilor voastre, făgăduințele voastre, prinoasele voastre cele de bună voie și jertfele voastre de pace, pe întâii născuți ai vitelor voastre mari și ai vitelor voastre mici;
\par 7 Să mâncați acolo înaintea Domnului Dumnezeului vostru și să vă veseliți cu familiile voastre pentru toate câte au făcut mâinile voastre și cu câte v-a binecuvântat Domnul Dumnezeul vostru.
\par 8 Să nu faceți așa cum facem noi acum aici, adică ceea ce i se pare fiecăruia că este bine;
\par 9 Căci noi acum n-am intrat în locul odihnei și în moștenirea pe care ji-o dă Domnul Dumnezeul tău;
\par 10 Ci când veți trece Iordanul și vă veți așeza în pământul ce vi-l dă Domnul Dumnezeul vostru moștenire, când El vă va liniști de toți vrăjmașii voștri, care vă înconjură, și veți trăi la adăpost de primejdii,
\par 11 Atunci la locul pe care-l va alege Domnul Dumnezeul vostru, ca să-Și pună numele asupra Lui, acolo să aduceți tot ce v-am poruncit Eu astăzi: arderile  de tot ale voastre, jertfele voastre, zeciuielile voastre, ridicarea mâinilor voastre și toate cele alese după făgăduințele voastre, ce ați făgăduit Domnului Dumnezeului vostru.
\par 12 Să vă veseliți înaintea Domnului Dumnezeului vostru, voi, fiii voștri și fiicele voastre, robii voștri, roabele voastre și levitul cel din mijlocul sălașurilor voastre, că acela n-are parte și moștenire cu voi.
\par 13 Ferește-te de a-ți aduce arderile de tot ale tale în orice loc s-ar întâmpla,
\par 14 Ci numai în locul acela pe care-l va alege Domnul Dumnezeul tău în una din semințiile tale, și fă tot ce ți-am poruncit eu astăzi.
\par 15 Totuși, când îți va pofti sufletul, vei putea să junghii și să mănânci carne oriunde vei trăi, după binecuvântarea pe care ți-a dat-o Domnul Dumnezeul tău; cel ce va fi necurat și cel ce va fi curat vor putea să mănânce carne, cum se mănâncă cea de căprioară și cea de cerb.
\par 16 Numai sânge să nu mâncați, ci să-l vărsați jos ca apa.
\par 17 Tu nu vei putea să mănânci în sălașurile tale zeciuiala de la pâinea ta, de la vinul tău și de la untdelemnul tău, întâii născuți ai vitelor tale mari și ai vitelor tale mărunte, nici darurile tale de bună voie, pe care le-ai făgăduit, nici prinoasele tale cele de bună voie, nici cele ridicate ale mâinilor tale,
\par 18 Ci acestea să le mănânci înaintea Domnului Dumnezeului tău, în locul acela pe care-l va alege Domnul Dumnezeul tău, tu și fiul tău, fiica ta, robul tău și roaba ta, levitul și străinul care este în locașurile tale și să te veselești înaintea Domnului Dumnezeului tău de toate câte au făcut mâinile tale.
\par 19 Bagă de seamă să nu părăsești pe levit în toate zilele cît vei trăi în pământul tău.
\par 20 Când va lărgi Domnul Dumnezeul tău hotarele tale, precum ți-a grăit El, și vei zice: Voi mânca și carne, pentru că sufletul tău dorește să mănânce carne, mănâncă și carne, după dorința sufletului tău.
\par 21 De va fi departe de tine locul pe care l-a ales Domnul Dumnezeul tău, ca să-Și pună numele asupra lui, atunci să junghii din vitele tale mari și mărunte pe care ți le-a dat Domnul Dumnezeul tău, după cum ți-am poruncit eu, și să mănânci în locuințele tale, după dorința sufletului tău.
\par 22 Dar să le mănânci cum se mănâncă cerbul și căprioara; aceasta poate să mănânce și cel curat și cel necurat al tău.
\par 23 Dar ia bine seama să nu mănânci sânge, pentru că sângele are în el viață și să nu mănânci viața laolaltă cu carnea.
\par 24 Să nu mănânci sângele, ci să-l verși jos ca apa.
\par 25 Să nu-l mănânci, ca să-ți fie bine în veci, ție și copiilor tăi de după tine, și bine-ți va fi de vei face cele bune și cele plăcute înaintea ochilor Domnului Dumnezeului tău.
\par 26 Dar cele închinate ale tale, câte le vei avea, și cele făgăduite ale tale, ia-le și vino la locul acela pe care 1-a ales Domnul Dumnezeul tău, ca să se cheme numele Lui acolo.
\par 27 Să săvârșești arderile de tot ale tale, carnea și sângele, pe jertfelnicul Domnului Dumnezeului tău; sângele celorlalte jertfe ale tale să fie vărsat lângă jertfelnicul Domnului Dumnezeului tău, iar carnea lor s-o mănânci.
\par 28 Ascultă și împlinește toate poruncile acestea pe care ți le dau eu astăzi, ca să-ți fie bine în veci, ție și copiilor tăi și bine-ți va fi de vei face cele bune și plăcute înaintea ochilor Domnului Dumnezeului tău.
\par 29 Când Domnul Dumnezeul tău va pierde de la fața ta popoarele la care mergi, ca să le cuprinzi, și după ce le vei cuprinde și te vei așeza în pământul lor,
\par 30 Atunci să te păzești ca să nu cazi în cursă și să le urmezi lor, după ce le vei pierde de pe fața pământului, și să nu cauți pe dumnezeii lor, zicând: Cum au slujit popoarele acestea dumnezeilor lor, așa voi face și eu.
\par 31 Să nu faci așa Domnului Dumnezeului tău, căci aceia fac dumnezeilor lor toate de care se îndepărtează Domnul și pe care le urăște El; aceia și pe fiii și pe fiicele lor le ard pe foc înaintea dumnezeilor lor.
\par 32 Toate câte vă poruncesc siliți-vă să le împliniți și nici să adaugi și nici Să lași ceva din ele".

\chapter{13}

\par 1 "De se va ridica în mijlocul tău prooroc sau văzător de vise și va face înaintea ta semn și minune,
\par 2 Și se va împlini semnul sau minunea aceea, de care ți-a grăit el, și-ți va zice atunci: Să mergem după alți dumnezei, pe care tu nu-i știi și să le slujim acelora,
\par 3 Să nu asculți cuvintele proorocului aceluia sau ale acelui văzător de vise, că prin aceasta vă ispitește Domnul Dumnezeul vostru, ca să afle de iubiți pe Domnul Dumnezeul vostru din toată inima voastră și din tot sufletul vostru.
\par 4 Domnului Dumnezeului vostru să-I urmați și de El să vă temeți; să păziți poruncile Lui și glasul Lui să-l ascultați; Lui să-I slujiți și de El să vă lipiți.
\par 5 Iar pe proorocul acela sau pe văzătorul acela de vise să-l dați morții, pentru că v-a sfătuit să vă abateți de la Domnul Dumnezeul vostru, Cel ce v-a scos din pământul Egiptului și v-a izbăvit din casa robiei, dorind să te abată de la calea pe care ți-a poruncit Domnul Dumnezeul tău să mergi; pierde dar răul din mijlocul tău.
\par 6 De te va îndemna în taină fratele tău, fiul tatălui tău, sau fiul mamei tale, sau fiul tău, sau fiica ta, sau femeia de la sânul tău, sau prietenul tău care este pentru tine ca sufletul tău, zicând: Haidem să slujim altor dumnezei, pe care tu și părinții tăi nu i-ați știut,
\par 7 Dumnezeilor acelor popoare, care locuiesc împrejurul tău, aproape sau departe de tine, de la un capăt până la celălalt al pământului,
\par 8 Să nu te învoiești cu ei, nici să-i asculți; să nu-i cruțe ochii tăi, să nu-ți fie milă de ei, nici să-i ascunzi;
\par 9 Ci ucide-i; mâna ta să fie înaintea tuturor asupra lor, ca să-i ucidă, și apoi să urmeze mâinile a tot poporul.
\par 10 Să-i ucizi cu pietre până la moarte, că au încercat să te abată de la Domnul Dumnezeul tău, Care te-a scos din pământul Egiptului și din casa robiei.
\par 11 Tot Israelul va auzi aceasta și se va teme și nu se vor mai apuca pe viitor să mai facă în mijlocul tău asemenea rău.
\par 12 De vei auzi de vreuna din cetățile tale, pe care Domnul Dumnezeul tău ți le dă ca să locuiești,
\par 13 Că s-au ivit în ea oameni necredincioși dintre ai tăi și au smintit pe locuitorii cetății lor, zicând: Haidem să slujim altor dumnezei, pe care voi nu i-ați știut,
\par 14 Caută, cercetează și întreabă bine, și de va fi adevărat că s-a întâmplat urâciunea aceasta în mijlocul tău,
\par 15 Să lovești pe locuitorii acelei cetăți cu ascuțișul sabiei, s-o dai, blestemului pe ea și tot ce este în ea și dobitoacele ei să le treci prin ascuțișul sabiei.
\par 16 Iar prăzile ei să le aduni toate în mijlocul pieții ei și să arzi cu foc cetatea și toată prada ei, ca ardere de tot Domnului Dumnezeului tău; să fie ea pe vecie dărâmată și niciodată să nu se mai zidească.
\par 17 Nimic din cele blestemate să nu se lipească de mâna ta, ca să-Și potolească Domnul iuțimea mâniei Sale și să-ți dea milă și îndurare, și să te înmulțească, cum ți-a grăit și ție și cum S-a jurat părinților tăi,
\par 18 De vei asculta glasul Domnului Dumnezeului tău, păzind toate poruncile Lui pe care ți le dau acum și făcând cele bune și plăcute înaintea ochilor Domnului Dumnezeului tău".

\chapter{14}

\par 1 "Voi sunteți fiii Domnului Dumnezeului vostru; să nu faceți crestături pe trupul vostru și să nu vă tundeți părul de deasupra ochilor voștri, pentru morți;
\par 2 Căci voi sunteți poporul sfânt al Domnului Dumnezeului vostru și pe voi v-a ales Domnul ca să-I fiți poporul Lui de moștenire dintre toate popoarele câte sunt pe pământ.
\par 3 Să nu mâncați nici un lucru necurat.
\par 4 Iată dobitoacele pe care le puteți mânca:
\par 5 Boul, oaia, capra, cerbul, gazela, antilopa, țapul, cerboaica, boul sălbatic și capra sălbatică.
\par 6 Orice dobitoc care are copita despicată, cu spintecătură adâncă între amândouă părțile copitei și care dobitoc rumegă mâncarea, se mănâncă.
\par 7 Dintre cele ce își rumegă mâncarea sau își au copita despicată printr-o spintecătură adâncă, numai acestea nu se mănâncă: cămila, iepurele și iepurele de casă, pentru că, deși acestea își rumegă mâncarea, dar nu-și au copita despicată, acestea sunt necurate pentru voi.
\par 8 Nu se mănâncă porcul, pentru că, deși are copita despicată, nu își rumegă mâncarea; acesta este necurat pentru voi. Carnea acestora să n-o mâncați și de stârvurile lor să nu vă atingeți.
\par 9 Din toate vietățile care sunt în apă, să mâncați pe acelea care au aripi și solzi;
\par 10 Iar pe toate celelalte, care n-au aripi și solzi, să nu le mâncați; necurate sunt pentru voi.
\par 11 Orice pasăre curată s-o mâncați.
\par 12 Dar din ele să nu mâncați pe acestea: vulturul, vulturul răpitor și vulturul de mare,
\par 13 Corbul, șoimul, gaia cu soiurile ei,
\par 14 Tot soiul de ciori,
\par 15 Struțul, cucuveaua, pescărușul și uliul cu soiurile lui,
\par 16 Huhurezul, ibisul și lebăda,
\par 17 Pelicanul, porfirionul și corbul de mare,
\par 18 Cocostârcul, pupăza cu soiurile ei și liliacul.
\par 19 Toate înaripatele târâtoare sunt necurate pentru voi; să nu le mâncați.
\par 20 Orice pasăre curată s-o mâncați.
\par 21 Să nu mâncați nici o mortăciune, ci s-o dai străinului de alt neam ce se va întâmpla să locuiască în casa ta; acela s-o mănânce sau să i-o vinzi, căci tu ești poporul sfânt al Domnului Dumnezeului tău. Să nu fierbi iedul în laptele mamei sale.
\par 22 Să osebești zeciuială din toate veniturile semănăturilor tale, care-ți vin din țarina ta în. fiecare an,
\par 23 Și să mănânci înaintea Domnului Dumnezeului tău, la locul ce-l va alege El, ca să-I fie numele acolo; adu zeciuială din pâinea ta, din vinul tău, din untdelemnul tău și pe întâii născuți ai vitelor tale mari și ai vitelor tale mărunte, ca să te înveți a te teme de Domnul Dumnezeul tău în toate zilele.
\par 24 Iar de va fi pentru tine drumul lung, încât să nu poți aduce acestea, pentru că este departe de tine locul pe care l-a ales Domnul Dumnezeul tău, ca să-și pună acolo numele Său, și Domnul Dumnezeul tău te-a binecuvântat,
\par 25 Atunci schimbă acestea pe argint și ia argintul în mâna ta și vino la locul pe care 1-a ales Domnul Dumnezeul tău;
\par 26 Apoi cumpără pe argintul acesta tot ce dorește sufletul tău: boi, oi, vin, sicheră și orice îți poftește sufletul tău, și mănâncă acolo înaintea Domnului Dumnezeului tău și te veselește, tu și familia ta.
\par 27 Dar pe levitul care este în locașurile tale să nu-l părăsești, căci el nu are parte și moștenire cu tine.
\par 28 Iar după trecerea a trei ani, ia toate zeciuielile veniturilor tale din anul acela și le pune în locașurile tale;
\par 29 Și să vină levitul, căci el nu are parte și moștenire cu tine, și străinul și orfanul și văduva care se află în sălașurile tale și să mănânce și să se sature, ca să te binecuvânteze Domnul Dumnezeul tău în toate lucrurile mâinilor tale, pe care le vei face tu".

\chapter{15}

\par 1 "În anul al șaptelea vei face iertare.
\par 2 Iertarea însă va fi aceasta: tot împrumutătorul, care dă împrumut aproapelui său, să ierte datoria și să n-o mai ceară de la aproapele său sau de la fratele său, că s-a vestit iertarea în cinstea Domnului Dumnezeului tău.
\par 3 De la cel de alt neam să ceri datoria; iar ce vei avea la fratele tău, să-i ierți.
\par 4 Numai așa nu va fi sărac printre voi; că te va binecuvânta Domnul în pământul acela pe care Domnul Dumnezeul tău ți-l dă în stăpânire, ca să-l ai moștenire,
\par 5 Dacă vei asculta glasul Domnului Dumnezeului tău și te vei sili să plinești toate poruncile acestea, care ți le spun eu astăzi.
\par 6 Căci Domnul Dumnezeul tău te va binecuvânta, după cum i-a grăit și vei da împrumut altor poare, iar tu nu vei lua împrumut; și ți domni peste multe popoare, iar acelea nu vor domni peste tine.
\par 7 Iar de va fi la tine sărac vreunul din frații tăi, în vreuna din cetățile tale de pe pământul tău pe care ți-l dă Domnul Dumnezeul tău, să nu-ți învârtoșezi inima, nici să-ți închizi mâna ta înaintea fratelui tău celui sărac;
\par 8 Ci să-i deschizi mâna ta și să-i dai împrumuturi potrivite cu nevoia lui și cu lipsa ce suferă.
\par 9 Păzește-te să nu intre în inima ta gândul nelegiuit și să zici: Se apropie anul al șaptelea, anul iertării; și să nu se facă din pricina aceasta ochiul tău nemilostiv către fratele tău cel sărac și să-l treci cu vederea; că acela va striga împotriva ta către Domnul și va fi asupra ta păcat mare.
\par 10 Dă-i, dă-i și împrumuturi câte-ți va cere și cît îi va trebui, și când îi vei da, să nu se întristeze inima ta, căci pentru aceasta te va binecuvânta Domnul Dumnezeul tău în toate lucrurile tale și în toate câte se vor lucra de mâinile tale.
\par 11 Căci nu va lipsi sărac din pământul tău; de aceea îți și poruncesc eu: Deschide mâna ta fratelui tău, săracului tău și celui lipsit din pământul tău.
\par 12 De ți se va vinde ție fratele tău, evreu sau evreică, șase ani să fie rob la tine, iar în anul al șaptelea să-i dai drumul de la tine, slobod.
\par 13 Iar când îi vei da drumul ca să fie slobod, să nu-i dai drumul cu mâinile goale;
\par 14 Ci înzestrează-l din turmele tale, din aria ta, de la teascul tău; dă-i și lui din cele cu care te-a binecuvântat Domnul Dumnezeul tău.
\par 15 Adu-ți aminte că și tu ai fost rob în pământul Egiptului și te-a izbăvit Domnul Dumnezeul tău. Iată pentru ce îți poruncesc acestea astăzi.
\par 16 Iar dacă acela îți va zice: Nu mă duc de la tine, pentru că te iubesc pe tine și casa ta, și deci îi este bine la tine,
\par 17 Să iei sula și să-i găurești urechea lui de ușor, și îți va fi rob pe vecie. Tot așa să faci și cu roaba ta.
\par 18 Să nu socotești o greutate pentru tine când va trebui să-i dai drumul de la tine ca să fie slobod, căci în șase ani ți-a muncit de două ori cât plata unui străin și te va binecuvânta Domnul Dumnezeul tău în toate câte vei face.
\par 19 Tot întâiul născut de parte bărbătească, ce se va naște din vitele tale cele mari și din vitele mărunte ale tale, să-l închini Domnului Dumnezeului tău. Să nu lucrezi cu boul tău întâi-născut și să nu tunzi pe întâiul născut din vitele tale mărunte.
\par 20 Înaintea Domnului Dumnezeului tău să mănânci acestea în fiecare an, tu și familia ta, la locul pe care-l va alege Domnul Dumnezeul tău.
\par 21 Dar dacă va avea vreo meteahnă, șchiopătare, sau orbire, sau altă meteahnă oarecare, să nu-l aduci jertfă Domnului Dumnezeului tău,
\par 22 Ci să-l mănânci în cetățile tale; atât cel necurat cît și cel curat pot să mănânce din el, cum mănâncă o căprioară sau un cerb.
\par 23 Numai sângele lui să nu-l mănânci, ci să-l verși jos, ca apa".

\chapter{16}

\par 1 "Să păzești luna Aviv și să prăznuiești Paștile Domnului Dumnezeului tău, pentru că în luna Aviv te-a scos Domnul Dumnezeul tău din Egipt, noaptea.
\par 2 Să junghii Paștile Domnului Dumnezeului tău din vite mari și din vite mărunte, la locul pe care-l va alege Domnul, ca să fie numele Lui acolo.
\par 3 Să nu mănânci în timpul Paștilor pâine dospită; șapte zile să mănânci azime, pâinea durerii, ca să-ți aduci aminte de ieșirea ta din pământul Egiptului în toate zilele vieții tale, căci cu grăbire ai ieșit tu din pământul Egiptului.
\par 4 Să nu se afle la tine aluat dospit în tot ținutul tău, timp de șapte zile, și din carnea care ai adus-o jertfă seara, în ziua întâi, să nu rămână nimic pe dimineață.
\par 5 Tu nu pori să junghii Paștile în vreuna din cetățile tale, pe care Domnul Dumnezeul tău ți le va da.
\par 6 Ci numai în locul acela pe care-l va alege Domnul Dumnezeul tău, ca să rămână acolo numele Lui; să junghii Paștile seara, la asfințitul soarelui, pe vremea când ai ieșit tu din Egipt.
\par 7 Să frigi și să mănânci în locul acela pe care-l va alege Domnul Dumnezeul tău, iar a doua zi poți să te întorci și să intri în sălașurile tale.
\par 8 Șase zile să mănânci pâine nedospită, iar în ziua a șaptea este încheierea sărbătorii Domnului Dumnezeului tău; să nu lucrezi în acele zile nimic, fără numai cele pentru suflet.
\par 9 Să numeri apoi șapte săptămâni; dar să începi a număra cele șapte săptămâni de când se va începe secerișul.
\par 10 Și atunci să săvârșești sărbătoarea săptămânilor Domnului Dumnezeului tău cu dar de bunăvoie, cum îți va da mâna și după cum vei putea, din cele cu care te-a binecuvântat Domnul Dumnezeul tău.
\par 11 Să te veselești înaintea Domnului Dumnezeului tău, tu, fiul tău și fiica ta, robul tău și roaba ta, levitul din cetățile tale și străinul, orfanul și văduva, care vor fi în mijlocul tău, în locul pe care l-a ales Domnul Dumnezeul tău, ca să fie numele Lui acolo.
\par 12 Adu-ți aminte că ai fost rob în Egipt; ține dar și păzește poruncile acestea.
\par 13 Sărbătoarea corturilor s-o săvârșești în șapte zile, după ce vei aduna din aria ta și din teascul tău.
\par 14 Și să te veselești în sărbătoarea ta: tu, fiul tău și fiica ta, robul tău și roaba ta, levitul și străinul, orfanul și văduva, care sunt în cetățile tale.
\par 15 Șapte zile să sărbătorești înaintea Domnului Dumnezeului tău, la locul pe care-l va alege Domnul Dumnezeul tău, ca să fie numele Lui acolo; că te va binecuvânta Domnul Dumnezeul tău în toate roadele și în tot lucrul mâinilor tale, și tu de aceea să fii vesel.
\par 16 De trei ori pe an să se înfățișeze toți cei de parte bărbătească înaintea Domnului Dumnezeului tău la locul pe care-l va alege El: la sărbătoarea azimelor, la sărbătoarea săptămânilor și la sărbătoarea corturilor, dar nimeni să nu se înfățișeze înaintea feței Domnului cu mâinile goale.
\par 17 Ci fiecare să vină cu dar în mâna sa, după cum 1-a binecuvântat Domnul Dumnezeul tău.
\par 18 în toate cetățile tale, pe care ți le va da Domnul Dumnezeul tău, să-îi pui judecători și căpetenii după semințiile tale, ca să judece poporul cu judecată dreaptă.
\par 19 Să nu strici legea, să nu cauți la față și să nu iei mită, că mita orbește ochii înțelepților și strâmbă pricinile drepte.
\par 20 Caută dreptate și iar dreptate, ca să trăiești și să stăpânești pământul pe care Domnul Dumnezeul tău ți-l dă.
\par 21 Să nu-ți sădești dumbravă de orice fel de copaci împrejurul jertfelnicului pe care-l vei zidi Domnului Dumnezeului tău.
\par 22 Și să nu-ți ridici stâlpi idolești, care sunt urâți de Domnul Dumnezeul tău".

\chapter{17}

\par 1 Să nu aduci jertfă Domnului Dumnezeului tău bou sau oaie cu meteahnă, sau cu beteșug, căci aceasta este urâciune înaintea Domnului Dumnezeului tău.
\par 2 De se va afla la tine, în vreuna din cetățile tale, pe care ți le va da Domnul Dumnezeul tău, bărbat sau femeie, care să fi făcut rău înaintea ochilor Domnului Dumnezeului tău, călcând legământul Lui,
\par 3 Și se va duce și se va apuca să slujească altor dumnezei și se va închina acelora, sau soarelui, sau lunii, sau la toată oștirea cerească, ceea ce eu n-am poruncit;
\par 4 Și ți se va vesti și vei auzi aceasta, să cercetezi bine și de se va adeveri aceasta și se va fi făcut urâciunea aceasta în Israel,
\par 5 Să scoți pe bărbatul acela sau pe femeia aceea care au făcut răul acesta la porțile tale și să-i ucizi cu pietre.
\par 6 Cel osândit la moarte să moară după spusele a doi sau trei martori; iar pe spusa unui singur martor să nu fie osândit nimeni la moarte.
\par 7 Mâna martorilor să se ridice asupra lui, ca să-l ucidă înaintea tuturor, și apoi să se ridice mâna a tot poporul. Pierde deci răul din mijlocul tău.
\par 8 Dacă în vreo pricină oarecare îți va fi greu de ales între sânge și sânge, între judecată și judecată, între bătăi și bătăi și în cetățile tale părerile vor fi împărțite, atunci scoală și du-te la locul pe care-l va alege Domnul Dumnezeul tău, ca să-I fie numele acolo,
\par 9 Și vino la preoți, la leviți și la judecătorul care va fi în zilele acelea și întreabă-i, iar ei îți vor spune cum să judeci.
\par 10 Fă după Cuvântul ce-ți vor spune ei în locul pe care-l va alege Domnul Dumnezeul tău, ca să fie chemat numele Lui acolo și silește-te să împlinești tot ceea ce te vor învăța ei,
\par 11 După legea pe care te vor învăța ei și după hotărârea ce-ți vor spune-o să faci și să nu te abați nici la dreapta, nici la stânga de la cele ce-li vor spune ei.
\par 12 Iar cine se va purta așa de îndărătnic, încât să nu asculte pe preotul care stă acolo la slujbă înaintea Domnului Dumnezeului tău, sau pe judecătorul care va fi în zilele acelea, unul ca acela să moară.
\par 13 Pierde deci răul din Israel și va auzi tot poporul și se va teme și nu se va mai purta în viitor cu îndărătnicie.
\par 14 Când vei ajunge tu în pământul ce ți-l dă Domnul Dumnezeul tău și-l vei lua în stăpânire și te vei așeza în el și vei zice: Îmi voi pune rege peste mine, ca celelalte popoare, care sunt împrejurul meu,
\par 15 Atunci să-ți pui rege peste tine pe acela pe care-l va alege Domnul Dumnezeul tău: dintre frații tăi să-ți pui rege peste tine; nu vei putea să pui rege peste tine un străin, care nu este din frații tăi.
\par 16 Dar să nu-și înmulțească acela caii și să nu întoarcă pe popor în Egipt, pentru ca să-și înmulțească el caii, căci Domnul v-a zis: Să nu vă mai întoarceri pe calea aceasta.
\par 17 Să nu-și înmulțească femeile, ca să nu se răzvrătească inima lui, și nici argintul și aurul lui să nu și-l înmulțească peste măsură.
\par 18 Căci, când se va sui pe scaunul regatului său, trebuie să-și scrie pentru sine cartea legii acesteia din cartea care se află la preoții leviților,
\par 19 Și să fie aceasta la el și el să o citească în toate zilele vieții sale, ca să învețe a se teme de Domnul Dumnezeul său și să se silească a împlini toate cuvintele legii acesteia și toate hotărârile acestea,
\par 20 Ca să nu se îngâmfe inima lui față de frații lui și ca să nu se abată el de la lege nici la dreapta, nici la stânga, ci ca să fie el și fiii lui zile multe la domnie în Israel".

\chapter{18}

\par 1 "Preoții, leviții și toată seminția lui Levi nu va avea parte și moștenire cu Israel; aceștia să se hrănească cu jertfele Domnului și cu partea Lui;
\par 2 Iar moștenire nu va avea el între frații săi, căci Domnul însuși este moștenirea lui, precum i-a grăit El.
\par 3 Iată ce să se dea preoților de la popor: cei ce aduc ca jertfă boi sau oi să dea preotului spata, fălcile și stomacul.
\par 4 De asemenea pârga de la grâul tău, de la vinul tău și de la untdelemnul tău, pârga de lână de la oile tale să i-o dai lui,
\par 5 Că pe el 1-a ales Domnul Dumnezeul tău din toate semințiile tale, ca să stea înaintea Domnului Dumnezeului tău și să slujească întru numele Domnului, el și fiii lui în toate zilele.
\par 6 De va pleca levitul din una din cetățile tale, din tot pământul fiilor lui Israel, unde locuiește, și va veni, după dorința sufletului său, la locul ce 1-a ales Domnul,
\par 7 Și va sluji în numele Domnului Dumnezeului tău, ca toți frații săi leviți care stau înaintea Domnului,
\par 8 Să se folosească de aceeași parte ca și ceilalți, pe lângă cele primite din vânzarea moștenirii părintești.
\par 9 Când vei intra tu în pământul ce ți-l dă Domnul Dumnezeul tău, să nu te deprinzi a face urâciunile pe care le fac popoarele acestea.
\par 10 Să nu se găsească la tine de aceia care trec pe fiul sau fiica lor prin foc, nici prezicător, sau ghicitor, sau vrăjitor, sau fermecător,
\par 11 Nici descântător, nici chemător de duhuri, nici mag, nici de cei ce grăiesc cu morții.
\par 12 Căci urâciune este înaintea Domnului tot cel ce face acestea, și pentru această urâciune îi izgonește Domnul Dumnezeul tău de la fața ta.
\par 13 Iar tu fii fără prihană înaintea Domnului Dumnezeului tău;
\par 14 Căci popoarele acestea, pe care le izgonești tu, ascultă de ghicitori și de prevestitori, iar ție nu-ți îngăduie aceasta Domnul Dumnezeul tău.
\par 15 Prooroc din mijlocul tău și din frații tăi, ca și mine, îți va ridica Domnul Dumnezeul tău: pe Acela să-L ascultați.
\par 16 Că tu la Horeb, în ziua adunării, ai cerut de la Domnul Dumnezeul tău și ai zis: Să nu mai aud glasul Domnului Dumnezeului meu și focul acesta mare să nu-l mai văd, ca să nu mor.
\par 17 Atunci mi-a zis Domnul: Bine este ceea ce ți-au spus ei.
\par 18 Eu le voi ridica Prooroc din mijlocul fraților lor, cum ești tu, și voi pune cuvintele Mele în gura Lui și El le va grăi tot ce-I voi porunci Eu.
\par 19 Iar cine nu va asculta cuvintele Mele, pe care Proorocul Acela le va grăi în numele Meu, aceluia îi voi cere socoteală.
\par 20 Iar proorocul care va îndrăzni să grăiască în numele Meu ceea ce nu i-am poruncit Eu să grăiască, și care va grăi în numele altor dumnezei, pe un astfel de prooroc să-l dați morții.
\par 21 De vei zice în inima ta: Cum vom cunoaște Cuvântul pe care nu-l grăiește Domnul?
\par 22 Dacă proorocul vorbește în numele Domnului, dar Cuvântul acela nu se va împlini și nu se va adeveri, atunci nu grăiește Domnul Cuvântul acela, ci-l grăiește proorocul din îndrăzneala lui; nu te teme de el".

\chapter{19}

\par 1 "Când Domnul Dumnezeul tău va pierde pe popoarele al căror pământ ți-l dă ție Domnul Dumnezeul tău, și tu vei intra în moștenirea lor și te vei așeza în cetățile lor și în casele lor,
\par 2 Atunci să-ți alegi trei cetăți în țara ta pe care Domnul Dumnezeul tău ți-o dă în stăpânire.
\par 3 Să-ți faci drum și să împarți în trei părți tot pământul tău pe care ți-l dă Domnul Dumnezeul tău de moștenire. Acelea vor sluji ca loc de scăpare oricărui ucigaș.
\par 4 Și iată care ucigaș va fugi acolo și va trăi: cel ce va ucide pe aproapele său fără voie și fără să-i fi fost vrăjmaș nici cu o zi, nici cu două înainte;
\par 5 Cel ce se va duce cu aproapele său în pădure să taie lemne și, învârtind mâna sa cu toporul, ca să taie un copac, va sări toporul din coadă și va lovi pe aproapele și acela va muri, acesta să fugă în una din aceste cetăți ale tale, spre a scăpa cu viață,
\par 6 Ca răzbunătorul sângelui, în aprinderea inimii lui, să nu se mânie pe ucigaș și să nu-l ajungă pe acesta, dacă va fi lung drumul, și să nu-l ucidă, întrucât nu este vinovat de moarte, pentru că nu i-a fost vrăjmaș nici cu o zi, nici cu două înainte.
\par 7 De aceea ți-am dat eu poruncă și ți-am zis: Alege-ți trei cetăți.
\par 8 Iar când Domnul Dumnezeul tău va lărgi hotarele tale, după cum S-a jurat părinților tăi, și îți va da tot pământul pe care a făgăduit să-l dea părinților tăi,
\par 9 Dacă te vei sili să împlinești toate poruncile acestea, pe care ți le spun eu astăzi, și vei iubi pe Domnul Dumnezeul tău și vei umbla în căile Lui în toate zilele, atunci la aceste trei cetăți să mai adaugi încă trei cetăți,
\par 10 Ca să nu se verse sângele nevinovatului în pământul tău pe care Domnul Dumnezeul tău ți-l dă de moștenire și să nu ai asupra ta vină de sânge.
\par 11 Iar dacă cineva din ai tăi va fi dușman aproapelui tău și-l va pândi și va sări la acela și-l va ucide și apoi va fugi în una din cetățile acestea,
\par 12 Bătrânii cetății lui să trimită ca să-l ia de acolo și să-l dea în mâinile răzbunătorului sângelui, ca să moară.
\par 13 Să nu-l cruțe pe unul ca acela ochiul tău. Spală pe Israel de sângele nevinovat și va fi bine.
\par 14 Să nu muți hotarul aproapelui tău, pe care l-au așezat strămoșii moșiei tale, care ți s-a cuvenit în pământul pe care Domnul Dumnezeul tău ți-l dă în stăpânire.
\par 15 Nu ajunge numai un martor pentru a vădi pe cineva de vreo vină sau de vreo nelegiuire sau de vreun păcat de care s-ar fi făcut vinovat, ci orice pricină să se dovedească prin spusa a doi sau trei martori.
\par 16 De se va ridica asupra cuiva martor nedrept, învinuindu-l de nelegiuire,
\par 17 Amândoi oamenii aceștia între care este pricina să se înfățișeze înaintea Domnului, la preot sau la judecătorii care vor fi în zilele acelea.
\par 18 Și judecătorii să cerceteze bine și, dacă martorul acela va fi martor mincinos și va fi mărturisit strâmb asupra fratelui său,
\par 19 Să-i faceți ceea ce voise să facă el fratelui său. Și așa să stârpești răul din mijlocul tău;
\par 20 Și vor auzi și ceilalți și se vor teme și nu se vor apuca să mai facă în mijlocul tău acest rău.
\par 21 Să nu-l cruțe ochiul tău, ci să ceri suflet pentru suflet, ochi pentru ochi, dinte pentru dinte, mână pentru mână, picior pentru picior. Cu răul pe care îl va face cineva-aproapelui său, cu acela trebuie să i se plătească".

\chapter{20}

\par 1 "Când vei ieși la război împotriva dușmanului tău și vei vedea cai, căruțe și oameni mai mulți decât ai tu, să nu te temi de ei, căci cu tine este Domnul Dumnezeul tău, Care te-a scos din pământul Egiptului.
\par 2 Iar când veți fi aproape de luptă, să vină preotul și să vorbească poporului și să-i spună:
\par 3 Ascultă, Israele, voi astăzi intrați în luptă cu dușmanii voștri; să nu slăbească inima voastră, nu vă temeți, nu vă tulburați, nici nu vă înspăimântați de ei.
\par 4 Că Domnul Dumnezeul vostru merge cu voi, ca să se lupte pentru voi cu dușmanii voștri și să vă izbăvească.
\par 5 Căpeteniile oștirii încă să grăiască poporului și să zică: Cel ce și-a zidit casă nouă și n-a sfințit-o, acela să iasă și să se întoarcă la casa sa, ca să nu moară în bătălie și să nu i-o sfințească altul.
\par 6 Cel ce și-a sădit vie și n-a mâncat din ea, acela să iasă și să se întoarcă la casa sa, ca să nu moară în bătălie și ca să nu se folosească altul de ea.
\par 7 Cel ce s-a logodit cu femeie și n-a luat-o, acela să iasă și să se întoarcă la casa sa, ca să nu moară în bătălie și ca să nu o ia altul.
\par 8 Ba căpeteniile oștirii să mai spună poporului și să zică: Cine este fricos și puțin la suflet, acela să iasă și să se întoarcă acasă, ca să nu facă fricoase și inimile fraților lui, cum este inima lui.
\par 9 Și după ce căpeteniile oștirii vor isprăvi de spus poporului toate acestea, atunci să se pună căpeteniile de război ca povățuitori ai poporului.
\par 10 Când te vei apropia de cetate ca s-o cuprinzi, fă-i îndemnare de pace.
\par 11 De se va învoi să primească pacea cu tine și-ți va deschide porțile, atunci tot poporul ce se va găsi în ea îți va plăti bir și-ți va sluji.
\par 12 Iar de nu se va învoi cu tine la pace și va duce război cu tine, atunci s-o înconjuri.
\par 13 Și când Domnul Dumnezeul tău o va da în mâinile tale, să lovești cu ascuțișul sabiei pe toți cei de parte bărbătească din ea.
\par 14 Numai femeile și copiii, vitele și tot ce este în cetate, toată prada ei să o iei pentru tine și să te folosești de prada vrăjmașilor tăi, pe care ți i-a dat Domnul Dumnezeul tău în mână.
\par 15 Așa să faci cu toate cetățile care sunt foarte departe de tine și care nu sunt din cetățile popoarelor acestora.
\par 16 Iar în cetățile popoarelor acestora pe care Domnul Dumnezeul tău ți le dă în stăpânire, să nu lași în viață nici un suflet;
\par 17 Ci să-i dai blestemului: pe Hetei și pe Amorei, pe Canaanei și Ferezei, pe Hevei, pe Iebusei și pe Gherghesei, precum ți-a poruncit Domnul Dumnezeul tău,
\par 18 Ca să nu vă învețe aceia să faceți aceleași urâciuni pe care le-au făcut ei pentru dumnezeii lor și ca să nu greșiți înaintea Domnului Dumnezeului vostru.
\par 19 De veți ține multă vreme înconjurată vreo cetate, ca s-o cuprinzi și s-o iei, să nu strici pomii ei cu securea, ci să te hrănești din ei și să nu-i dobori la pământ. Copacul de pe câmp este el oare om ca să se ascundă de tine după întăritură?
\par 20 Iar copacii pe care-i știi că nu-ți aduc nimic de hrană poți să-i strici și să-i tai, ca să-ți faci întărituri împotriva cetății care poartă cu tine război, până o vei supune".

\chapter{21}

\par 1 "Dacă în pământul pe care ți-l dă în stăpânire Domnul Dumnezeul tău se va găsi om ucis, zăcând în câmp, și nu se va ști cine l-a ucis,
\par 2 Să iasă bătrânii tăi și judecătorii tăi și să măsoare ce depărtare este de la cel ucis până la orașele dimprejur.
\par 3 Și bătrânii cetății aceleia, care va fi mai aproape de cel ucis, să ia o junincă ce n-a fost pusă la muncă și n-a purtat jug,
\par 4 Și bătrânii cetății aceleia să ducă această junincă la apă curgătoare, într-un loc care n-a fost arat, nici semănat, și să junghie juninca acolo în apa cea curgătoare.
\par 5 Apoi să vină preoții, fiii leviților, că pe ei i-a ales Domnul Dumnezeul tău să-I slujească și să binecuvânteze în numele Lui și după Cuvântul lor se hotărăște orice lucru îndoielnic și toată vătămarea pricinuită.
\par 6 Și toți bătrânii cetății aceleia, câre sunt mai aproape de cel ucis, să-și spele mâinile deasupra capului junincii celei junghiate în râu
\par 7 Și să grăiască și să spună: "Mâinile noastre n-au vărsat sângele acesta și ochii noștri n-au văzut;
\par 8 Iartă pe poporul Tău Israel, pe care Tu, Doamne, l-ai răscumpărat din pământul Egiptului și nu lăsa poporului Tău Israel acest sânge nevinovat!" Și se vor curăți de sânge.
\par 9 Așa să speli tu sângele nevinovat de la tine, dacă vrei să faci cele bune și drepte înaintea ochilor Domnului Dumnezeului tău.
\par 10 Când vei ieși la război împotriva vrăjmașilor tăi și Domnul Dumnezeul tău ți-i va da în mâinile tale și-i vei lua în robie,
\par 11 Și vei vedea printre robi femeie frumoasă la chip și o vei iubi și vei vrea s-o iei de soție,
\par 12 S-o aduci în casa ta, să-și tundă capul său, să-și taie unghiile,
\par 13 Să-și dezbrace de pe ea haina sa de robie, să locuiască în casa ta și să-și plângă pe tatăl său și pe mama sa timp de o lună; iar după aceea vei intra la ea, ca să fii bărbatul ei și ea să-ți fie femeie.
\par 14 Iar dacă ea în urmă nu-ți va mai plăcea, să-i dai drumul să se ducă unde va vrea, dar să n-o vinzi pe argint și să n-o prefaci în roabă, pentru că ai umilit-o.
\par 15 De va avea cineva două femei, una iubită și una neiubită și atât cea iubită cît și cea neiubită îi vor naște copii și întâiul născut va fi al celei neiubite,
\par 16 Acela, la împărțirea averii sale între fiii săi, nu poate să dea fiului femeii iubite întâietate înaintea fiului întâi-născut din cea neiubită,
\par 17 Ci să cunoască de întâi-născut pe fiul celei neiubite și să-i dea acestuia parte îndoită din toate câte va avea, că acesta este pârga puterii lui și al lui este dreptul de întâi-născut.
\par 18 De va avea cineva fecior rău și nesupus, care nu ascultă de vorba tatălui său și de vorba mamei sale și aceștia l-au pedepsit, dar el tot nu-i ascultă,
\par 19 Să-l ia tatăl lui și mama lui și să-l ducă la bătrânii cetății lor și la poarta acelei cetăți și către preoții cetății lor să zică:
\par 20 Acest fiu al nostru este rău și neascultător, nu ascultă de Cuvântul nostru și este lacom și bețiv".
\par 21 Atunci toți oamenii cetății lui să-l ucidă cu pietre și să-l omoare. Și așa să stârpești răul din mijlocul tău și toți Israeliții vor auzi și se vor teme.
\par 22 De se va găsi la cineva vinovăție vrednică de moarte și va fi omorât, spânzurat de copac,
\par 23 Trupul lui să nu rămână peste noapte spânzurat de copac, ci să-l îngropi tot în ziua aceea, căci blestemat este înaintea Domnului tot cel spânzurat pe lemn și să nu spurci pământul tău pe care Domnul Dumnezeul tău ți-l dă moștenire".

\chapter{22}

\par 1 "Când vei vedea boul fratelui tău sau oaia lui rătăcite pe câmp; să nu treci pe lângă ele, ci să le întorci fratelui tău.
\par 2 Dar dacă fratele tău nu va fi aproape de tine sau nu-l cunoști, să le duci la casa ta și să șadă la tine până le va căuta fratele tău și atunci să i le dai.
\par 3 Așa să faci și cu asinul lui, așa să faci și cu haina lui, așa să faci și cu orice lucru pierdut al fratelui tău pe care el îl va pierde și tu îl vei găsi; de la aceasta nu te poți da la o parte.
\par 4 Când vei vedea asinul fratelui tău sau boul lui căzuți în drum, să nu-i lași, ci să-i ridici împreună cu el.
\par 5 Femeia să nu poarte veșminte bărbătești, nici bărbatul să nu îmbrace haine femeiești, că tot cel ce face aceasta, urâciune este înaintea Domnului Dumnezeului tău.
\par 6 Dacă în cale, în vreun copac sau pe pământ, vei găsi cuib de pasăre cu pui sau cu ouă și mama lor va fi șezând pe pui sau pe ouă, să nu iei mama împreună cu puii;
\par 7 Mamei dă-i drumul, iar puii ia-i pentru tine ca să-ți fie bine și să se înmulțească zilele tale.
\par 8 De vei zidi casă nouă, să faci apărătoare pe marginea acoperișului tău, ca să nu aduci sânge asupra casei tale când va cădea cineva de pe ea.
\par 9 Să nu semeni via ta cu două feluri de semințe, ca să nu-ți faci blestemată strângerea semințelor, pe care tu le semeni împreună cu roadele viei tale.
\par 10 Să nu ari cu un bou și cu un asin.
\par 11 Să nu te îmbraci cu haină făcută din două feluri de fire: de lână și de in.
\par 12 Fă-ți ciucuri în cele patru colțuri ale mantiei tale cu care te acoperi.
\par 13 De își va lua cineva femeie și va intra la dânsa,
\par 14 Iar apoi o va urî și va ridica asupra ei învinuiri de lucruri urâte, va împrăștia zvon rău despre ea și va zice: Am luat femeia aceasta și am intrat la ea și n-am găsit la ea feciorie,
\par 15 Atunci tatăl fetei și mama ei să ia și să ducă semnele fecioriei fetei la bătrânii cetății, în poartă;
\par 16 Și tatăl fetei să zică bătrânilor: Am dat pe fiica mea de femeie acestui om și acum el a urât-o,
\par 17 Și iată ridică asupra ei învinuiri de lucruri urâte, zicând: N-am găsit feciorie la fiica ta; dar iată semnele fecioriei fiicei mele. Și să întindă haina înaintea bătrânilor cetății.
\par 18 Atunci bătrânii acelei cetăți să ia pe bărbat și să-l pedepsească;
\par 19 Să pună asupra lui gloabă de o sută de sicli de argint și să-i dea tatălui fetei, pentru că a stârnit zvonuri rele despre o fată israelită; ea însă să-i rămână femeie și el să nu se poată despărți de ea toată viața lui.
\par 20 Iar dacă cele spuse vor fi adevărate și nu se va găsi feciorie la fată,
\par 21 Atunci fata să fie adusă la ușa casei tatălui ei și locuitorii cetății ei să o ucidă cu pietre și să o omoare, pentru că a făcut lucru de rușine în Israel, desfrânându-se în casa tatălui său. Și așa să stârpești răul din mijlocul tău.
\par 22 De se va găsi cineva dormind cu femeie măritată, pe amândoi să-i dați morții: și bărbatul, care a dormit cu femeia și femeia. Și așa să stârpești răul din Israel.
\par 23 De va fi vreo fată tânără, logodită cu bărbat și cineva o va întâlni în cetate și se va culca cu dânsa,
\par 24 Să-i aduceți pe amândoi la poarta cetății aceleia și să-i ucideți cu pietre: pe fată pentru că n-a țipat în cetate, iar pe bărbat pentru că a necinstit pe femeia aproapelui său. Și așa să stârpești răul din mijlocul tău.
\par 25 Dacă vreun bărbat va întâlni la câmp o fată logodită și, prinzând-o, se va culca cu ea, să-l ucideți numai pe bărbatul care s-a culcat cu ea;
\par 26 Iar fetei să nu-i faci nimic. Asupra fetei nu este vină de moarte, căci aceasta este tot una ca și cum cineva s-ar ridica asupra aproapelui său și l-ar omorî;
\par 27 Pentru că el a întâlnit-o în câmp și, deși fata logodită va fi strigat, n-a avut cine s-o scape.
\par 28 De se va întâlni cineva cu o fată nelogodită și o va prinde și se va culca cu ea și vor fi prinși,
\par 29 Atunci cel ce s-a culcat cu ea să dea tatălui fetei cincizeci de sicli de argint, iar ea să-i fie nevastă, pentru că a necinstit-o; toată viața lui să nu se poată despărți de ea.
\par 30 Nimeni să nu ia de soție pe femeia tatălui său și să ridice marginea hainei tatălui său".

\chapter{23}

\par 1 "Scopitul și famenul să nu intre în obștea Domnului.
\par 2 Fiul femeii desfrânate să nu intre în obștea Domnului.
\par 3 Amonitul și Moabitul să nu intre în obștea Domnului; nici al zecelea neam al lor în veci să nu intre în obștea Domnului,
\par 4 Pentru că nu v-au întâmpinat cu pâine și cu apă în cale, când veneați din Egipt, și pentru că ei au plătit împotriva ta pe Valaam, fiul lui Beor, din Petorul Mesopotamiei, ca să vă blesteme.
\par 5 Dar Domnul Dumnezeul tău n-a voit să asculte pe Valaam și a prefăcut Domnul Dumnezeul tău blestemul lui în binecuvântare pentru tine, pentru că Domnul Dumnezeul tău te iubește.
\par 6 Să nu le dorești pace și fericire în toate zilele tale, în veci.
\par 7 Să nu-ți fie scârbă de edomit căci acesta îți este frate. Să nu-ți fie scârbă de egiptean, că ai fost străin în pământul lui.
\par 8 Copiii ce se vor naște acestora în al treilea neam pot intra în obștea Domnului.
\par 9 Când vei ieși cu război asupra dușmanilor tăi, ferește-te de tot ce este rău.
\par 10 De va fi careva din ai tăi necurat de ceea ce i s-a întâmplat noaptea, acela să iasă afară din tabără și să nu intre în tabără,
\par 11 Ci, după ce se va face seară, să-și spele trupul cu apă și după asfințitul soarelui să intre în tabără.
\par 12 Să ai afară din tabără loc și acolo să ieși afară.
\par 13 Afară de uneltele tale, să mai ai și o lopată și, când vei vrea să ieși cu scaunul afară din tabără, să sapi o groapă și apoi să îngropi cu ea necurățeniile tale;
\par 14 Că Domnul Dumnezeul tău umblă prin tabăra ta, ca să te izbăvească și să-ți dea pe vrăjmașii tăi în mâinile tale; de aceea tabăra ta să fie sfântă, ca să nu vadă El la tine ceva de rușine și să Se depărteze de la tine.
\par 15 Să nu dai pe rob în mâinile stăpânului său, când acela va fugi de la Stăpânul său la tine;
\par 16 Lasă-l să trăiască la tine, lasă-l să locuiască în mijlocul vostru, în locul ce-și va alege, în una din cetățile tale unde-i va place; dar să nu-l strâmtorezi.
\par 17 Să nu fie desfrânată din fiicele lui Israel, nici desfrânat din fiii lui Israel.
\par 18 Câștigul de la desfrânată și prețul de pe câine să nu-l duci în casa Domnului Dumnezeului tău pentru împlinirea oricărei făgăduințe, căci și unul și altul sunt urâciune înaintea Domnului Dumnezeului tău.
\par 19 Să nu dai cu camătă fratelui tău nici argint, nici pâine, nici nimic din câte se pot da cu camătă.
\par 20 Celui de alt neam să-i dai cu camătă; iar fratelui tău să nu-i dai cu camătă, ca Domnul Dumnezeul tău să te binecuvânteze întru toate câte se fac de mâinile tale în pământul în care mergi ca să-l iei în stăpânire.
\par 21 De vei da făgăduință Domnului Dumnezeului tău, să nu întârzii a o împlini, căci Domnul Dumnezeul tău o va cere de la tine și păcat vei avea asupra ta.
\par 22 Iar de n-ai dat făgăduință, nu va fi păcat asupra ta.
\par 23 Ceea ce a ieșit din gura ta să păzești și să împlinești făgăduința pe care tu ai făcut-o de bună voie Domnului Dumnezeului tău, și de care ai grăit cu gura ta.
\par 24 De vei intra în via aproapelui tău, poți să mănânci poamă până ce te vei sătura, dar în panerul tău să nu pui.
\par 25 Și când vei intra în grâul aproapelui tău, rupe spice cu mâinile tale, iar secera să n-o pui în ogorul aproapelui tău".

\chapter{24}

\par 1 "De va lua cineva femeie și se va face bărbat ei, dar ea nu va afla bunăvoință în ochii lui, pentru că va găsi el ceva neplăcut la ea, și-i va scrie carte de despărțire, i-o va da la mână și o va slobozi din casa sa,
\par 2 Iar ea va ieși și, ducându-se, se va mărita cu alt bărbat.
\par 3 Dar dacă și acest din urmă bărbat o va urî și-i va scrie carte de despărțire și i-o va da la mână și o va slobozi din casa sa, sau va muri acest din urmă bărbat al ei, care a luat-o de soție,
\par 4 Bărbatul ei cel dintâi care a lăsat-o nu o poate lua iar de soție, după ce a fost întinată, că aceasta este urâciune înaintea Domnului Dumnezeului tău; să nu întinezi pământul, pe care Domnul Dumnezeul tău ti-l dă de moștenire.
\par 5 Dacă cineva și-a luat femeie de curând, să nu se ducă la război și să nu i se pună nici o sarcină; lasă-l să rămână slobod la casa sa timp de un an, să veselească pe femeia sa pe care a luat-o.
\par 6 Nimeni să nu ia zălog piatra de deasupra sau cea de dedesubt a râșniței, că ar însemna că iei zălog însăși viața cuiva.
\par 7 De se va afla că cineva a furat pe vreunul din frații săi, din fiii lui Israel și, făcându-l rob, l-a vândut, să fie omorât tălharul acela și să stârpești răul din mijlocul tău.
\par 8 Fii atent cu tine la boala leprei, păzește foarte bine toată legea care te învață preoții cei din leviți, și împliniți cu sfințenie ceea ce le-am poruncit eu.
\par 9 Adu-ți aminte ce a făcut Domnul Dumnezeul tău Mariamei pe drum, când veneați voi din Egipt.
\par 10 De vei da aproapelui tău ceva împrumut, să nu te duci în casa lui ca să iei zălog de la el;
\par 11 Stai în uliță, și acela căruia i-ai dat împrumut să-ți scoată zălog în uliță.
\par 12 Iar dacă acela va fi om sărac, să nu te culci, având la tine zălogului;
\par 13 Ci să-i întorci zălogul la asfințitul soarelui, ca să se culce în haina sa și să te binecuvânteze; aceasta ți se va socoti ca o faptă bună înaintea Domnului Dumnezeului tău.
\par 14 Să nu nedreptățești pe cel ce muncește cu plată, pe sărac și pe cel lipsit dintre frații tăi sau dintre străinii care sunt în pământul tău și în cetățile tale.
\par 15 Ci să dai plata în aceeași zi și să nu apună soarele înainte de aceasta, că el este sărac și sufletul lui așteaptă această plată; ca să nu strige el asupra ta către Domnul și să nu ai păcat.
\par 16 Părinții să nu fie pedepsiți cu moartea pentru vina copiilor și nici copiii să nu fie pedepsiți cu moartea pentru vina părinților; ci fiecare să fie pedepsit cu moartea pentru păcatul său.
\par 17 Să nu judeci strâmb pe străin, pe orfan și pe văduvă, și văduvei să nu-i iei haina zălog.
\par 18 Adu-ți aminte că și tu ai fost rob în Egipt și Domnul Dumnezeul tău te-a izbăvit de acolo, de aceea îți și poruncesc eu să faci aceasta.
\par 19 Când vei secera holda în țarina ta și vei uita vreun snop în țarină, să nu te întorci să-l iei, ci lasă-l să rămână al străinului, săracului, orfanului și văduvei, ca Domnul Dumnezeul tău să te binecuvânteze întru toate lucrurile mâinilor tale.
\par 20 Când vei scutura măslinul tău, să nu te întorci să culegi rămășițele, ci lasă-le străinului, orfanului și văduvei".
\par 21 Când vei strânge roadele viei tale, să nu aduni rămășițele, ci lasă-le străinului, orfanului și văduvei.
\par 22 Adu-ți aminte că și tu ai fost rob în pământul Egiptului și de aceea îți poruncesc eu să faci aceasta".

\chapter{25}

\par 1 "De va fi neînțelegere între oameni, să fie aduși la judecată și să fie judecați; celui drept să i se dea dreptate, iar cel vinovat să se osândească.
\par 2 Dacă celui vinovat i se va cuveni bătaie, să poruncească judecătorii să fie pus jos și să fie bătut înaintea lor, după măsura vinovăției lui.
\par 3 I se pot da până la patruzeci de lovituri, iar nu mai mult, ca nu cumva fratele tău, din pricina multelor lovituri, să fie schilodit înaintea ochilor tăi.
\par 4 Să nu legi gura boului care treieră.
\par 5 De vor trăi frații împreună și unul din ei va muri, fără să aibă fiu, femeia celui mort să nu se mărite în altă parte după străin, ci cumnatul ei să intre la ea, să și-o ia soție și să trăiască cu ea.
\par 6 Întâiul născut pe care-l va naște ea să poarte numele fratelui lui cel mort, pentru ca numele acestuia să nu se șteargă din Israel.
\par 7 Iar dacă el nu va voi să ia pe cumnata sa, aceasta să se ducă la poarta cetății, înaintea bătrânilor și să zică: Cumnatul meu nu vrea să păstreze numele fratelui său în Israel, nevrând să se căsătorească cu mine.
\par 8 Iar bătrânii cetății lui să-l cheme și să-l sfătuiască și, dacă el se va ridica și va zice: Nu vreau s-o iau,
\par 9 Atunci cumnata lui să se ducă la el acolo, în fața bătrânilor, să-i dezlege sandaua din piciorul lui, să-l scuipe în obraz și să zică: Așa se cuvine omului care nu vrea să zidească fratelui său casă în Israel.
\par 10 Și casa acestuia se va numi în Israel casa desculțului.
\par 11 De se vor bate între dânșii niște bărbați, și femeia unuia din ei se va duce ca să scoată pe bărbatul său din mâna celui ce-l bate și, întinzându-și ea mâna, va apuca pe acesta de părțile lui rușinoase,
\par 12 Să i se taie mâna ei și să nu o cruțe ochiul tău.
\par 13 Să nu ai în săculețul tău două feluri de greutăți pentru cumpănă: unele mai mari și altele mai mici.
\par 14 În casa ta să nu ai două feluri de efă: una mai mare și alta mai mică.
\par 15 Greutățile pentru cumpăna ta să fie adevărate și drepte și efa ta să fie adevărată și dreaptă, ca să se înmulțească zilele tale pe pământul pe care ți-l dă Domnul Dumnezeul tău de moștenire;
\par 16 Că urâciune este înaintea Domnului Dumnezeului tău tot cel ce face strâmbătate.
\par 17 Adu-ți aminte cum s-a purtat cu tine Amalec pe drum, când veneați voi din Egipt,
\par 18 Și cum te-a întâmpinat el în cale și a ucis în urma ta pe toți cei slăbiți, când erai ostenit și obosit, netemându-se de Dumnezeu.
\par 19 Așadar, când Domnul Dumnezeul tău te va liniști de toți vrăjmașii tăi, din toate părțile, în pământul pe care Domnul Dumnezeul tău ți-l dă moștenire ca să-l stăpânești, să ștergi pomenirea lui Amalec de sub cer. Nu uita aceasta".

\chapter{26}

\par 1 Când vei intra în pământul pe care Domnul Dumnezeul tău ți-l dă moștenire, și-l vei lua în stăpânire și te vei așeza pe el,
\par 2 Să iei pârga tuturor roadelor pământului ce vei lua tu din pământul tău pe care Domnul Dumnezeul tău ți-l dă, să le pui în paner și să te duci la locul acela pe care Domnul Dumnezeul tău îl va alege ca să-i fie acolo numele Lui;
\par 3 Să mergi la preotul care va fi în zilele acelea și să-i zici: Astăzi mărturisesc înaintea Domnului Dumnezeului tău că am intrat în acel pământ pentru care Domnul S-a jurat părinților noștri să ni-l dea nouă.
\par 4 Iar când preotul va lua panerul din mâna ta și-l va pune înaintea jertfelnicului Domnului Dumnezeului tău,
\par 5 Tu să răspunzi și să zici înaintea Domnului Dumnezeului tău: Tatăl meu a fost un arameian pribeag, s-a dus în Egipt, s-a așezat acolo cu puținii oameni ai săi și acolo s-a ridicat din el popor mare, puternic și mult la număr.
\par 6 Dar Egiptenii s-au purtat rău cu noi și ne-au strâmtorat și ne-au silit la munci grele.
\par 7 De aceea am strigat noi către Domnul Dumnezeul părinților noștri; iar Domnul a auzit strigarea noastră, a văzut nenorocirea noastră, muncile noastre și împilarea noastră.
\par 8 Și ne-a scos Domnul din Egipt, singur cu puterea Lui cea mare, cu mână tare și cu braț înalt, cu înfricoșare mare, cu semne și minuni;
\par 9 Și ne-a adus în locul acesta și ne-a dat pământul acesta, țara în care curge lapte și miere.
\par 10 Acum iată am adus pârga roadelor din pământul pe care Tu, Doamne, mi l-ai dat din pământul unde curge lapte și miere. Să pui  roadele acelea înaintea Domnului Dumnezeului tău, să te închini înaintea Domnului Dumnezeului tău
\par 11 Și să te veselești de toate bunătățile ce Domnul Dumnezeul tău ți-a dat ție și casei tale; dar să se veselească și levitul și străinul care va fi la tine.
\par 12 Iar când vei osebi toate zeciuielile din roadele pământului tău în anul al treilea, care este anul zeciuielii, și le vei da levitului, străinului, orfanului și văduvei, ca să mănânce aceștia în locașurile tale și să se sature,
\par 13 Atunci să zici înaintea Domnului Dumnezeului tău: Am osebit din casa mea cele sfinte și le-am dat levitului, străinului, orfanului și văduvei, după toate poruncile Tale pe care mi le-ai dat Tu mie; n-am călcat poruncile Tale, nici nu le-am uitat;
\par 14 N-am mâncat din ele în întristarea mea, nici nu le-am osebit în necurățenie, nici n-am dat din ele cu prilejul vreunui mort, ci m-am supus glasului Domnului Dumnezeului meu și am împlinit tot ce mi-ai poruncit Tu.
\par 15 Caută deci din locașul Tău cel sfânt, din ceruri, și binecuvintează pe poporul Tău, Israel, și pământul pe care ni l-ai dat nouă, după cum Te-ai jurat părinților noștri, ca să ne dai pământul în care curge lapte și miere.
\par 16 În ziua aceasta îți poruncește Domnul Dumnezeul tău să împlinești toate hotărârile și rânduielile acestea; să le păzești și să le împlinești din toată inima ta și din tot sufletul tău.
\par 17 Astăzi ai mărturisit tu Domnului că El va fi Dumnezeul tău și că tu vei umbla în căile Lui și vei păzi hotărârile Lui, poruncile Lui și legile Lui și vei asculta glasul Lui.
\par 18 Și Domnul ți-a făgăduit astăzi că tu vei fi poporul Lui adevărat, precum ți-a grăit El, de vei păzi toate poruncile Lui;
\par 19 Și te va pune cu cinstea și cu mărirea și cu faima mai presus de toate popoarele pe care le-a făcut El și vei fi poporul sfânt al Domnului Dumnezeului tău, precum ți-a grăit El".

\chapter{27}

\par 1 Moise, împreună cu bătrânii fiilor lui Israel, a poruncit poporului și a zis: "Toate poruncile pe care vi le poruncesc eu astăzi să le împliniți.
\par 2 Când vei trece peste Iordan în pământul pe care Domnul Dumnezeul tău ți-l dă, să-ți așezi pietre mari și să le văruiești cu var;
\par 3 Și pe pietrele acelea să scrii toate cuvintele acestei legi când vei trece Iordanul, ca să intri în pământul pe care Domnul Dumnezeul tău ți-l dă, în pământul unde curge lapte și miere, după cum ți-a grăit Domnul Dumnezeul părinților tăi.
\par 4 După ce veți fi trecut Iordanul, să puneți pietrele acelea, precum vă poruncesc eu astăzi, pe muntele Ebal și să le văruiți cu var.
\par 5 Să zidești acolo jertfelnic Domnului Dumnezeului tău, jertfelnic făcut din pietre, fără să pui asupra lor fierul.
\par 6 Jertfelnicul Domnului Dumnezeului tău însă să-l faci din pietre întregi și să aduci pe el Domnului Dumnezeului tău ardere de tot.
\par 7 Să mai aduci jertfe de pace; să mănânci și să te saturi acolo și să te veselești înaintea Domnului Dumnezeului tău.
\par 8 Dar să scrii pe pietrele acelea cuvintele legii acesteia foarte lămurit".
\par 9 Moise cu preoții cei din leviți a grăit la tot Israelul și a zis: "Ia aminte și ascultă, Israele: Astăzi te-ai făcut poporul Domnului Dumnezeului tău.
\par 10 Ascultă dar glasul Domnului Dumnezeului tău și plinește toate poruncile Lui și hotărârile Lui pe care ți le spun eu astăzi".
\par 11 În ziua aceea a mai poruncit Moise poporului și a zis:
\par 12 "După ce veți trece Iordanul, semințiile: Simeon, Levi, Iuda, Isahar, Iosif și Veniamin să stea pe muntele Garizim și să binecuvânteze poporul;
\par 13 Iar semințiile: Ruben, Gad, Așer, Zabulon, Dan și Neftali să stea pe muntele Ebal, ca să rostească blestemul.
\par 14 Atunci leviții să strige și să zică cu glas tare tuturor Israeliților:
\par 15 Blestemat să fie cel ce va face idol cioplit sau turnat, lucru de mână de meșter și urâciune înaintea Domnului și-l va pune la loc tainic! La aceasta tot poporul să răspundă și să zică: Amin!
\par 16 Blestemat să fie cel ce va grăi de rău pe tatăl său sau pe mama sa! Și tot poporul să zică: Amin!
\par 17 Blestemat să fie cel ce va muta hotarul aproapelui său! Și tot poporul să zică: Amin!
\par 18 Blestemat să fie cel ce va abate pe orb din drum! Și tot poporul să zică: Amin!
\par 19 Blestemat să fie cel ce va judeca strâmb pe străin, pe orfan și pe văduvă! Și tot poporul să zică: Amin!
\par 20 Blestemat să fie cel ce se va culca cu femeia tatălui său, că a ridicat poala hainei tatălui său! Și tot poporul să zică: Amin!
\par 21 Blestemat să fie cel ce se va culca cu vreun dobitoc! Și tot poporul să zică: Amin!
\par 22 Blestemat să fie cel ce se va culca cu sora sa, fiica tatălui său, sau fiica mamei sale! Și tot poporul să zică: Amin!
\par 23 Blestemat să fie cel ce se va culca cu soacra sa! Și tot poporul să zică: Amin! Blestemat să fie cel ce se va culca cu sora femeii sale! și tot poporul să zică: Amin!
\par 24 Blestemat să fie cel ce va ucide în ascuns pe aproapele său! Și tot poporul să zică: Amin!
\par 25 Blestemat să fie cel ce va lua mită, ca să ucidă suflet și să verse sânge nevinovat! Și tot poporul să zică: Amin!
\par 26 Blestemat să fie tot omul care nu va plini toate cuvintele legii acesteia și nu va urma după ea! Și tot poporul să zică: Amin!"

\chapter{28}

\par 1 "Dacă tu, după ce vei trece peste Iordan, în pământul pe care Domnul Dumnezeul tău îi-l va da, vei asculta glasul Domnului Dumnezeului tău și vei împlini cu băgare de seamă toate poruncile Lui pe care ți le dau astăzi, atunci Domnul Dumnezeul tău te va pune mai presus de toate popoarele pământului.
\par 2 De vei asculta glasul Domnului Dumnezeului tău, vor veni asupra ta toate binecuvântările acestea și se vor împlini asupra ta:
\par 3 Binecuvântat să fii în cetate și binecuvântat să fii în țarină;
\par 4 Binecuvântat să fie rodul pântecelui tău, rodul pământului tău, rodul dobitoacelor tale;
\par 5 Binecuvântate să fie hambarele tale și cămările tale;
\par 6 Binecuvântat să fii la intrarea ta în casă și binecuvântat să fii la ieșirea ta din casă;
\par 7 Să bată Domnul înaintea ta pe vrăjmașii tăi cei ce se vor ridica asupra ta; pe o cale să vină asupra ta și pe șapte căi să fugă de tine;
\par 8 Să-îi trimită Domnul binecuvântare peste grânarele tale și peste tot lucrul mâinilor tale și să te binecuvânteze în pământul pe care Domnul Dumnezeul tău ți-l dă;
\par 9 Să facă Domnul Dumnezeu din tine popor sfânt al Său, precum ți S-a jurat El ție și părinților tăi, dacă vei asculta poruncile Domnului Dumnezeului tău și vei umbla în căile Lui;
\par 10 Vor vedea toate popoarele pământului că porți numele Domnului Dumnezeului tău și se vor teme de tine;
\par 11 Domnul Dumnezeul tău îți va da belșug în toate bunătățile, în rodul pântecelui tău, în rodul dobitoacelor tale și în rodul ogoarelor tale din pământul pe care Domnul S-a jurat părinților tăi să ți-l dea;
\par 12 Domnul îți va deschide comoara Sa cea bună, cerul, ca să dea ploaie pământului tău la vreme și ca să binecuvânteze toate lucrurile mâinilor tale; și vei da împrumut multor popoare, iar tu nu vei lua împrumut; vei domni peste multe popoare, iar acelea nu vor domni peste tine;
\par 13 Domnul Dumnezeul tău te va pune cap iar nu coadă și vei fi numai sus, iar jos nu vei fi, dacă te vei supune poruncilor Domnului Dumnezeului tău, care ți le spun eu astăzi să le ții și să le împlinești
\par 14 Și dacă nu te vei abate de la toate poruncile care ți le poruncesc eu astăzi nici la dreapta nici la stânga, ca să mergeți după alți dumnezei să le slujiți.
\par 15 Iar dacă nu vei asculta glasul Domnului Dumnezeului tău și nu te vei sili să împlinești toate poruncile și hotărârile Lui pe care îi le poruncesc eu astăzi, să vină asupra ta toate blestemele acestea și să te ajungă:
\par 16 Blestemat să fii tu în cetate și blestemat să fii tu în țarină;
\par 17 Blestemate să fie grânarele tale și cămările tale;
\par 18 Blestemat să fie rodul pântecelui tău și rodul pământului tău, rodul vacilor tale și rodul oilor tale;
\par 19 Blestemat să fii tu la intrarea ta în casă și blestemat la ieșirea ta din casă;
\par 20 Să trimită Domnul asupra ta blestem, tulburare și necaz în tot lucrul mâinilor tale pe care te vei apuca să-l faci, până vei fi stârpit și până vei pieri curând, pentru faptele tale rele și pentru că M-ai părăsit;
\par 21 Ba să mai trimită Domnul asupra ta ciumă, până te va stârpi de pe pământul în care mergi ca să-l stăpânești;
\par 22 Să te bată Domnul cu oftică, cu lingoare, cu friguri, cu aprindere, cu secetă, cu vânt rău și cu rugină, și te vor urmări acestea până vei pieri;
\par 23 Cerurile tale, care sunt deasupra capului tău, să se facă aramă și pământul de sub tine fier;
\par 24 În loc de ploaie, Domnul să dea pământului tău praf și pulbere, care să cadă din cer asupra ta până te va pierde și până vei fi prăpădit;
\par 25 Domnul te va da să fii bătut de vrăjmașii tăi; pe un drum să mergi asupra lor și pe șapte drumuri să fugi de ei și să fii împrăștiat prin toate țările pământului;
\par 26 Trupurile tale să fie hrană tuturor păsărilor cerului și fiarelor și nu va fi cine să le alunge;
\par 27 Te va lovi Domnul cu lepra Egiptului, cu trânji, cu râie și cu pecingine, de care să nu te poți vindeca;
\par 28 Să te bată Domnul cu nebunie, cu orbire și cu amorțirea inimii;
\par 29 Pe dibuite să mergi ziua în amiaza mare, cum umblă orbul pipăind pe întuneric, și să te strâmtoreze și să te ocărască în toate zilele, și nimeni să nu te apere;
\par 30 Cu femeie să te logodești și altul să se culce cu ea; casă să zidești, și să nu trăiești în ea; vie să sădești, dar de ea să nu te folosești;
\par 31 Boul tău să fie junghiat sub ochii tăi și să nu-l mănânci tu; asinul să ți-l ia și să nu ți-l mai aducă; oile tale să fie date vrăjmașilor tăi și nimeni să nu te izbăvească;
\par 32 Fiii tăi și fiicele tale să fie date la popor străin; ochii tăi să-i vadă și să se topească în toate zilele de mila lor, dar să nu ai nici o putere în mâinile tale;
\par 33 Roadele pământului tău și toate ostenelile tale să le mănânce un popor pe care tu nu l-ai cunoscut, iar tu să fii numai strâmtorat și chinuit în toate zilele;
\par 34 Din pricina celor ce-ți vor vedea ochii tăi, îți vei ieși din minți;
\par 35 Domnul te va lovi cu lepră rea peste genunchi și peste fluiere și din tălpile picioarelor tale până în creștetul capului tău, de care nu te vei mai putea vindeca;
\par 36 Te va duce Domnul pe tine și pe regele tău, pe care-l vei pune peste tine, la poporul pe care nu l-ai cunoscut nici tu, nici părinții tăi, și acolo vei sluji altor dumnezei de lemn și de piatră;
\par 37 Și vei fi de spaimă, de pomină și de râs la toate popoarele la care te va duce Domnul Dumnezeul tău;
\par 38 Vei semăna multă sămânță în țarină, dar puțină vei culege, pentru că o vor mânca lăcustele;
\par 39 Vii vei sădi și le vei lucra, dar nu le vei culege, nici nu vei bea vin, pentru că le vor mânca viermii;
\par 40 Măslini încă vei avea în toate ținuturile tale, dar cu untdelemn nu te vei unge, pentru că măslinele tale vor cădea;
\par 41 Fii și fiice vei naște, dar nu-i vei avea, pentru că vor fi luați în robie;
\par 42 Toți pomii tăi și roadele pământului le va strica rugina;
\par 43 Străinul cel din mijlocul tău se va înălța peste tine din ce în ce mai sus, iar tu te vei pogorî din ce în ce mai jos;
\par 44 Acela îți va da împrumut, iar tu nu-i vei da lui împrumut; acela va fi cap, iar tu vei fi coadă.
\par 45 Vor veni asupra ta toate blestemele acestea, te vor urmări și te vor ajunge. până vei fi stârpit, pentru că n-ai ascultat glasul Domnului Dumnezeului tău și n-ai păzit poruncile Lui, nici hotărârile Lui pe care ți le-a dat El.
\par 46 Și vor fi ele semn și pecete asupra ta și asupra seminției tale în veci.
\par 47 Pentru că tu n-ai slujit Domnului Dumnezeului tău cu veselia și cu bucuria inimii, când erai îmbelșugat de toate.
\par 48 Vei sluji vrăjmașului tău, pe care-l va trimite asupra ta Domnul Dumnezeul tău, când vei fi în foamete și în sete și în golătate și în tot felul de lipsă; acela va pune pe grumazul tău jug de fier și te va istovi.
\par 49 Trimite-va Domnul asupra ta popor din depărtare, de la marginea pământului; ca un vultur va veni poporul acela a cărui limbă tu nu o vei înțelege;
\par 50 Popor crunt, care nu va da cinste bătrânului și nu va cruța pe cel tânăr.
\par 51 Va mânca acela rodul dobitoacelor tale și rodul pământului tău, până te va nimici, căci nu-ți va lăsa nici pâine, nici vin, nici untdelemn, nici rodul vitelor tale, nici rodul oilor tale, până te va pierde.
\par 52 Te va strâmtora în toate cetățile tale, până ce va dărâma, în tot pământul tău, zidurile tale cele înalte și tari în care nădăjduiești tu și te va împila în toate locașurile tale, în tot pământul tău pe care Domnul Dumnezeul tău ți-l dă.
\par 53 Și vei mânca tu rodul pântecelui tău, carnea fiilor și fiicelor tale, pe care ți-i va fi dat Domnul Dumnezeul tău în timpul împresurării și al strâmtorării cu care te va strâmtora vrăjmașul tău.
\par 54 Bărbatul tău, cel răsfățat și trăit în alintare, va privi cu ochi nemilostivi la fratele său; la soția de la sânul său și la ceilalți copii ai săi, care îți vor rămâne.
\par 55 Și nu va da nici unuia din ei carne din copiii săi, pe care îi va mânca el, căci nu-i va mai rămâne nimic în vremea împresurării cu care te va strâmtora vrăjmașul tău în toate cetățile tale.
\par 56 Femeia ta, trăită în belșug și răsfăț, care nu și-a pus piciorul său în pământ din pricina traiului alintat și îndestulat dinainte, va privi cu ochi nemilostivi pe bărbatul de la sânul ei și la fiul său și la fiica sa,
\par 57 Și nu le va da fătul, care a ieșit din coapsele sale și copiii, pe care i-a născut, pentru că ea, din pricina lipsei de toate, îi va mânca pe ascuns în timpul împresurării și al strâmtorării cu care te va strâmtora vrăjmașul tău în cetățile tale.
\par 58 De nu te vei sili să împlinești toate cuvintele legii acesteia, care sunt scrise în cartea aceasta și nu te vei teme de acest nume slăvit și înfricoșător al Domnului Dumnezeului tău,
\par 59 Atunci Domnul te va bate pe tine și pe urmașii tăi cu plăgi nemaiauzite, cu plăgi mari și nesfârșite și cu boli rele și necurmate;
\par 60 Va aduce asupra ta toate plăgile cele rele ale Egiptului, de care te-ai temut și se vor lipi acelea de tine.
\par 61 Toată boala, toată plaga scrisă și toată cea nescrisă în cartea legii acesteia, o va aduce Domnul asupra ta, până vei fi stârpit.
\par 62 Puțini din voi vor rămâne, deși veți fi fost ca stelele cerului, pentru că n-ați ascultat glasul Domnului Dumnezeului vostru.
\par 63 Cum s-a bucurat Domnul, când v-a făcut bine și v-a înmulțit, tot așa se va bucura Domnul când vă va pierde și vă va stârpi și veți fi aruncați din pământul în care intrați ca să-l stăpâniți.
\par 64 Atunci te va împrăștia Domnul Dumnezeul tău prin toate popoarele și acolo vei sluji altor dumnezei, pe care nu i-ai cunoscut nici tu, nici părinții tăi; vei sluji la lemne și la pietre.
\par 65 Dar și între aceste popoare nu te vei liniști și nu vei avea loc de odihnă pentru piciorul tău, că Domnul îți va da acolo inimă tremurătoare, topirea ochilor și durere sufletului.
\par 66 Viața ta va fi mereu în primejdie înaintea ochilor tăi; vei tremura ziua și noaptea și nu vei fi sigur de viața ta.
\par 67 De tremurul inimii tale, de care vei fi cuprins, și de cele ce vei vedea cu ochii tăi, dimineața vei zice: O, de ar veni seara! Iar seara vei zice: O, de ar veni ziua!
\par 68 Și te va întoarce Domnul în Egipt, în corăbii pe calea aceea de care ți-a zis: "Să nu o mai vezi!"; și acolo vă veți da spre vânzare vrăjmașilor voștri robi și roabe și nu va fi cine să vă cumpere".

\chapter{29}

\par 1 Iată cuvintele legământului ce a poruncit Domnul lui Moise să încheie cu fiii lui Israel în pământul Moabului, afară de legământul pe care l-a încheiat Domnul cu ei în Horeb.
\par 2 Atunci a chemat Moise pe toți fiii lui Israel și le-a zis: "Ați văzut toate câte a făcut Domnul înaintea ochilor voștri, în pământul Egiptului, cu Faraon și cu toate slugile lui și cu tot pământul lui,
\par 3 Acele pedepse mari pe care le-au văzut ochii tăi și acele semne și minuni, mâna cea tare și brațul cel înalt;
\par 4 Dar până în ziua de astăzi nu v-a dat Domnul Dumnezeu minte ca să pricepeți, ochi ca să vedeți și urechi ca să auziți.
\par 5 Patruzeci de ani v-a purtat prin pustie și hainele de pe voi nu s-au învechit, nici încălțămintele voastre nu s-au stricat în picioarele voastre.
\par 6 Pâine n-ați mâncat, vin și sicheră n-ați băut, ca să știți că Eu sunt Domnul Dumnezeul vostru.
\par 7 Când însă ați ajuns la locul acesta, s-a ridicat împotriva voastră Sihon, regele Heșbonului, și Og, regele Vasanului, ca să se lupte cu noi.
\par 8 Dar noi i-am bătut și am cuprins țara lor și am dat-o moștenire semințiilor lui Ruben și Gad și la jumătate din seminția lui Manase.
\par 9 Păziți dar toate cuvintele așezământului acestuia și le împliniți, ca să aveți spor la toate câte veți face.
\par 10 Voi cu toții vă înfățișați astăzi înaintea feței Domnului Dumnezeului vostru: căpeteniile semințiilor voastre, bătrânii voștri, judecătorii voștri, mai marii oștirii voastre și toți Israeliții,
\par 11 Copiii voștri, femeile voastre și străinii tăi care se află în taberele tale, de la tăietorul de lemne până la cărătorul de apă.
\par 12 Ca să închei legământ cu Domnul Dumnezeul tău și să ai parte de legământul făcut prin jurământ pe care Domnul Dumnezeul tău l-a încheiat astăzi cu tine,
\par 13 Ca să te faci astăzi poporul Lui și El să-ți fie Dumnezeu, precum ți-a grăit El și cum S-a jurat părinților tăi: lui Avraam, lui Isaac și lui Iacov.
\par 14 Și nu numai cu voi singuri închei Eu acest legământ și fac acest jurământ,
\par 15 Ci atât cu cei ce stau astăzi aici cu noi înaintea feței Domnului Dumnezeului vostru, cît și cu acei care nu sunt astăzi aici cu noi.
\par 16 Că știți cum am trăit noi în pământul Egiptului și cum am trecut prin mijlocul popoarelor pe la care ați venit,
\par 17 și ați văzut urâciunile lor și idolii lor de lemn și de piatră, de argint și de aur, pe care îi au ele.
\par 18 Să nu fie printre voi bărbat sau femeie, sau neam, sau seminție, a căror inimă să se abată acum de la Domnul Dumnezeul vostru, ca să meargă să slujească dumnezeilor acelor popoare; să nu fie printre voi rădăcină din care să răsară otravă și pelin,
\par 19 Nici astfel de om care, auzind cuvintele blestemului acestuia, s-ac lăuda în inima sa, zicând: "Eu voi fi fericit, cu toate că voi umbla după voința inimii mele", și să piară astfel cel sătul cu cel flămând.
\par 20 Pe unul ca acesta nu-l va ierta Domnul, ci îndată se va aprinde mânia Domnului și iuțimea Lui asupra unui astfel de om și va cădea asupra lui tot blestemul legământului acestuia, care este scris în cartea aceasta a legământului și va șterge Domnul numele lui de sub cer,
\par 21 Și-l va despărți Domnul spre pieire din toate semințiile lui Israel, după toate blestemele legământului, care sunt scrise în cartea aceasta a legii.
\par 22 Rândul de oameni care va urma, copiii voștri care vor fi după voi, străinul care va veni din țară depărtată și toate popoarele, văzând pedepsirea pământului acestuia și bolile cu care îl pustiește Domnul,
\par 23 Văzând pucioasa și sarea și că tot pământul este zgură, încât nici nu se seamănă, nici nu rodește și nu răsare pe el nici un fir de iarbă, ca de pe urma Sodomei, Gomorei, Admei și Țeboimului, pe care le-a stricat Domnul în mânia Sa și în iuțimea Sa,
\par 24 Vor zice: Pentru ce a făcut Domnul așa cu țara aceasta? Cât de mare este aprinderea mâniei Lui!
\par 25 Atunci vor răspunde: Pentru că au părăsit legământul Domnului Dumnezeului părinților lor, pe care Acesta l-a încheiat cu ei, când i-a scos din pământul Egiptului,
\par 26 Și s-au dus și s-au apucat să slujească altor dumnezei și s-au închinat acelor dumnezei pe care ei nu i-au cunoscut și pe care El nu i-a hotărât.
\par 27 De aceea s-a aprins mânia Domnului asupra Zării acesteia și a adus El asupra ei toate blestemele legământului, scrise în această carte a legii,
\par 28 Și i-a lepădat Domnul din pământul lor cu mânie, cu iuțime și cu aprindere mare și i-a aruncat în alt pământ, cum vedem  acum.
\par 29 Cele ascunse sunt ale Domnului Dumnezeului nostru, iar cele descoperite sunt ale noastre și ale fiilor noștri pe veci, ca să plinim toate cuvintele legii acesteia".

\chapter{30}

\par 1 "Când vor veni asupra ta toate cuvintele acestea, binecuvântarea și blestemul, pe care ti le-am spus eu și le vei primi în inima ta în toate popoarele printre care te va împrăștia Domnul Dumnezeul tău,
\par 2 Și te vei întoarce la Domnul Dumnezeul tău și, cum ti-am poruncit eu astăzi, vei asculta glasul Domnului Dumnezeului tău, tu și fiii tăi, din toată inima ta și din tot sufletul tău,
\par 3 Atunci Domnul Dumnezeul tău va întoarce pe robii tăi și se va milostivi asupra ta și iar te va aduna din toate popoarele printre care te-a împrăștiat Domnul Dumnezeul tău.
\par 4 Chiar de ai fi risipit de la o margine a cerului până la cealaltă margine a cerului, și de acolo te va aduna Domnul Dumnezeul tău și te va lua și de acolo,
\par 5 Și te va aduce Domnul Dumnezeul tău în pământul pe care l-au stăpânit părinții tăi și-l vei lua în stăpânire și te va face fericit și te va înmulți mai mult decât pe părinții tăi.
\par 6 Va tăia Domnul împrejur inima ta și inima urmașilor tăi, ca să iubești pe Domnul Dumnezeul tău din toată inima ta și din tot sufletul tău, ca să trăiești.
\par 7 Atunci Domnul Dumnezeul tău va întoarce toate blestemele acestea asupra vrăjmașilor tăi și a celor ce te-au urât și te-au prigonit;
\par 8 Iar tu te vei întoarce și vei asculta glasul Domnului Dumnezeului tău și vei împlini toate poruncile Lui pe care ți le spun astăzi.
\par 9 Domnul Dumnezeul tău îți va da cu prisosință spor la tot lucrul mâinilor tale, la rodul pântecelui tău, la rodul dobitoacelor tale, la rodul pământului tău, că se va bucura Domnul Dumnezeul tău din nou de tine, cum S-a bucurat de părinți tăi, și-ți va face bine,
\par 10 De vei asculta glasul Domnului Dumnezeului tău, păzind și împlinind toate poruncile Lui, hotărârile Lui și legile Lui, și de te vei întoarce la Domnul Dumnezeul tău din toată inima ta și din tot sufletul tău.
\par 11 Căci porunca aceasta care ți-o poruncesc eu astăzi nu este neînțeleasă de tine și nu este departe.
\par 12 Ea nu este în cer, ca să zici: Cine se va sui pentru noi în cer, ca să ne-o aducă și să ne-o dea s-o auzim și s-o facem?
\par 13 Și nu este ea nici peste mare, ca să zici: Cine se va duce pentru noi peste mare, ca să ne-o aducă, să ne facă s-o auzim și s-o împlinim?
\par 14 Ci Cuvântul acesta este foarte aproape de tine; el este în gura ta și în inima ta, ca să-l faci.
\par 15 Iată eu astăzi ți-am pus înainte viața și moartea, binele și răul,
\par 16 Poruncindu-ți astăzi să iubești pe Domnul Dumnezeul tău, să umbli în toate căile Lui și să împlinești poruncile Lui, hotărârile Lui și legile Lui, ca să trăiești și să te înmulțești și să te binecuvânteze Domnul Dumnezeul tău pe pământul pe care îl vei stăpâni.
\par 17 Iar de se va întoarce inima ta și nu vei asculta, ci te vei lăsa ademenit și te vei închina la alți dumnezei și le vei sluji lor,
\par 18 Vă dau de știre astăzi că veți pieri și nu veți trăi mult în pământul pe care Domnul Dumnezeu ți-l dă și pentru a cărui stăpânire treci tu acum Iordanul.
\par 19 Ca martori înaintea voastră iau astăzi cerul și pământul: viață și moarte Zi-am pus eu astăzi înainte, și binecuvântare și blestem. Alege viața ca să trăiești tu și urmașii tăi.
\par 20 Să iubești pe Domnul Dumnezeul tău, să asculți glasul Lui și să te lipești de El; căci în aceasta este viața ta și lungimea zilelor tale, ca să locuiești pe pământul pe care Domnul Dumnezeul tău cu jurământ l-a făgăduit părinților tăi, lui Avraam, lui Isaac și lui Iacov că îl va da lor".

\chapter{31}

\par 1 Atunci s-a dus Moise și a grăit cuvintele acestea tuturor fiilor lui Israel
\par 2 Și le-a zis: "Eu acum sunt de o sută douăzeci de ani și nu mai pot intra și ieși și Domnul mi-a zis: Tu nu vei trece Iordanul acesta,
\par 3 Ci va merge înaintea ta Însuși Domnul Dumnezeul tău și va stârpi El pe popoarele acestea de la fața ta și tu le vei stăpâni; și Iosua va merge înaintea ta, cum a zis Domnul.
\par 4 și va face Domnul cu ei cum a făcut și cu Sihon și cu Og, regii Amoreilor, care erau dincoace de Iordan, și cum a făcut cu pământul acelora pe care i-a pierdut;
\par 5 Îi va da Domnul pe ei vouă și veți face cu ei după poruncile pe care vi le-am spus eu.
\par 6 Fiți tari și curajoși, nu vă temeți, nu vă îngroziți, nici nu vă spăimântați de ei, că Domnul Dumnezeul tău va merge El Însuși cu tine și nu se va depărta de tine, nici nu te va părăsi".
\par 7 Apoi a chemat Moise pe Iosua și înaintea ochilor tuturor Israeliților i-a zis: "Fii tare și curajos, că tu vei intra cu poporul acesta în pământul pe care Domnul S-a jurat părinților lui să i-l dea și tu i-l vei împărți în părți de moștenire.
\par 8 Domnul Însuși va merge înaintea ta; El Însuși va fi cu tine și nu se va depărta de tine, nici te va părăsi; nu te teme, nici nu te spăimânta".
\par 9 Apoi a scris Moise legea aceasta și a dat-o preoților, fiilor leviților, care purtau chivotul legii Domnului, și tuturor bătrânilor fiilor lui Israel.
\par 10 Și le-a poruncit Moise acestora și le-a zis: "După trecerea a șapte ani, în anul iertării, la sărbătoarea corturilor,
\par 11 Când tot Israelul va veni să se înfățișeze înaintea feței Domnului Dumnezeului tău, în locul pe care-l va alege Domnul, să citești legea aceasta înaintea a tot Israelul și în auzul lui;
\par 12 Să aduni poporul, bărbații, femeile, copiii și pe străinii tăi care se vor afla în cetățile tale, ca să audă și să învețe și ca să se teamă de Domnul Dumnezeul vostru și ca să se silească să împlinească toate cuvintele legii acesteia.
\par 13 Fiii lor care nu știu vor auzi și vor învăța a se teme de Domnul Dumnezeul vostru în toate zilele, cît veri trăi pe pământul în care treceți voi peste Iordan ca să-l stăpâniți".
\par 14 După aceea a zis Domnul către Moise: "Iată s-a apropiat clipa în care să mori; cheamă pe Iosua și stați la ușa cortului adunării, că Eu îi voi da povețe!" Și a venit Moise cu Iosua și au stat la ușa cortului adunării.
\par 15 Atunci S-a arătat Domnul în cort, în stâlp de nor, și stâlpul de nor a stat la ușa cortului adunării.
\par 16 Și a zis Domnul către Moise: "Iată, tu te vei odihni cu părinții tăi, iar poporul acesta se va scula și se va desfrâna după dumnezeii străini ai pământului aceluia în care va intra, iar pe Mine Mă va părăsi și va călca legământul Meu, pe care l-am încheiat cu el;
\par 17 Pentru aceasta se va aprinde mânia Mea asupra lui în ziua aceea și-i voi părăsi, Îmi voi ascunde fața de la ei și vor fi omorâți și-i vor ajunge mulțime de necazuri și greutăți și atunci Israel va zice: Aceste necazuri nu m-au ajuns ele oare pentru că nu este Domnul Dumnezeul meu în mijlocul meu?
\par 18 Dar Eu Îmi voi ascunde fața Mea de la el în ziua aceea, pentru toate fărădelegile lui pe care le-a făcut el, întorcându-se la alți dumnezei.
\par 19 Așadar, scrie-ți cuvintele cântării acesteia și învață pe fiii lui Israel și le-o pune în gura lor, pentru ca această cântare să-Mi fie mărturie printre fiii lui Israel.
\par 20 Căci Eu îi voi duce în pământul cel bun, unde curge lapte și miere, după cum M-am jurat părinților lor, și vor mânca, se vor sătura, se vor îngrășa, se vor îndrepta spre alți dumnezei și vor sluji acelora, iar pe Mine Mă vor lepăda și vor călca legământul Meu, pe care l-am dat lor.
\par 21 Dar când îi vor ajunge mulțime de nenorociri și de necazuri, atunci cântarea aceasta va fi mărturie împotriva lor, căci ea nu va pieri din gura lor și din gura urmașilor lor. Cunosc Eu cugetele lor pe care le au ei acum, înainte de a-i duce în pământul cel bun pentru care M-am jurat părinților lor".
\par 22 Și a scris Moise cântarea aceasta în ziua aceea și a spus-o fiilor lui Israel.
\par 23 Iar Domnul a poruncit lui Iosua Navi și i-a zis: "Fii tare și curajos, căci tu vei duce pe fiii lui Israel în pământul pentru care M-am jurat lor, și Eu voi fi cu tine".
\par 24 Când a scris Moise în carte toate cuvintele legii acesteia până la sfârșit,
\par 25 Atunci Moise a poruncit leviților care purtau chivotul legii Domnului
\par 26 Și a zis: "Luați această carte a legii și o puneți de-a dreapta chivotului legii Domnului Dumnezeului vostru și va fi ea acolo mărturie împotriva ta.
\par 27 Că eu cunosc îndărătnicia ta și cerbicia ta; dacă și acum, când trăiesc eu cu voi, sunteți îndărătnici înaintea Domnului, dar cu cît mai mult veți fi după moartea mea?
\par 28 Chemați la mine pe toți bătrânii semințiilor voastre și pe judecătorii voștri și pe căpeteniile voastre și eu voi spune în auzul lor cuvintele acestea și voi chema martori împotriva lor cerul și pământul;
\par 29 Căci știu eu că după moartea mea vă veți răzvrăti și vă veți abate de la calea pe care v-am arătat-o eu; în zilele cele de apoi vă vor ajunge necazuri, pentru că veți face rău înaintea ochilor Domnului Dumnezeu, mîniindu-L cu lucrurile mâinilor voastre".
\par 30 Și a rostit Moise în auzul întregii obști a Israeliților cuvintele cântării acesteia până la sfârșit:

\chapter{32}

\par 1 "Ia aminte, cerule, și voi grăi! Ascultă, pământule, cuvintele gurii mele!
\par 2 Ca ploaia să curgă învățătura mea și graiurile mele să se coboare ca roua, ca bura pe verdeață și ca ploaia repede pe iarbă.
\par 3 Căci numele Domnului voi preamări. Dați slavă Dumnezeului nostru!
\par 4 El este tăria; desăvârșite sunt lucrurile Lui, căci toate căile Lui sunt drepte. Credincios este Dumnezeu și nu este întru El nedreptate; drept și adevărat este El,
\par 5 Iar ei s-au răzvrătit împotriva Lui; ei, după netrebniciile lor, nu sunt fiii Lui, ci neam îndărătnic și ticălos. Cu acestea răsplătiți voi Domnului?
\par 6 Popor nechibzuit și fără de minte, au nu este El tatăl tău, Cel ce te-a zidit, te-a făcut și te-a întemeiat?
\par 7 Adu-ți aminte de zilele cele de demult, cugetă la anii neamurilor trecute! Întreabă pe tatăl tău și-ți va da de știre, întreabă pe bătrâni, și-ți vor spune:
\par 8 Când Cel Preaînalt a împărțit moștenire popoarelor, când a împărțit pe fiii lui Adam, atunci a statornicit hotarele neamurilor după numărul îngerilor lui Dumnezeu;
\par 9 Iar partea Domnului este poporul lui Iacov, Israel e partea lui de moștenire.
\par 10 Găsitu-l-a în pământ pustiu, în pustiu trist și cu urlete sălbatice, și t-a apărat, l-a îngrijit și l-a păzit, ca lumina ochiului Său.
\par 11 Întocmai ca vulturul care îndeamnă la zbor puii săi și se rotește pe deasupra lor, întinzându-și aripile, a luat pe Israel și l-a dus pe penele sale.
\par 12 Domnul l-a povățuit și n-a fost cu el dumnezeu străin.
\par 13 Și l-a așezat pe înălțimile pământului și l-a hrănit cu roada țarinilor. I-a dat să scoată miere din piatră și cu untdelemn din stâncă vârtoasă l-a hrănit;
\par 14 L-a hrănit cu unt de vacă și cu lapte de oi, cu grăsimea mieilor, a berbecilor de Vasan, a țapilor și cu grâu gras; a băut vin, sângele bobițelor de strugure.
\par 15 A mâncat Iacov, s-a îngrășat Israel și s-a făcut îndărătnic; îngrășatu-s-a, îngroșatu-s-a și s-a umplut de grăsime; a părăsit pe Dumnezeu, Cel ce l-a făcut și a disprețuit cetatea mântuirii sale.
\par 16 Întărâtat-au râvna Lui cu dumnezei străini și cu urâciunile lor L-au mâniat;
\par 17 Adus-au jertfe demonilor, și nu lui Dumnezeu, unor dumnezei noi, pe care nu i-au știut, care au venit de la vecinii lor și pe care părinții lor nu i-au cunoscut.
\par 18 Iar pe Apărătorul, Cel ce te-a născut, L-ai uitat și nu ți-ai adus aminte de Dumnezeu, Cel ce te-a zidit.
\par 19 Văzut-a Domnul și S-a mâniat și în mânia Sa a trecut cu vederea pe fiii Săi și pe fiicele Sale,
\par 20 Și a zis: îmi voi ascunde fața Mea de la ei și voi vedea cum va fi sfârșitul lor; căci neam ticălos sunt ei și copii în care nu este credincioșie.
\par 21 Ei M-au întărâtat la gelozie prin cei ce nu sunt Dumnezeu și au aprins mânia Mea prin idolii lor; îi voi întărâta și Eu pe ei printr-un popor care nu e popor, le voi aprinde mânia printr-un neam fără pricepere.
\par 22 Că foc s-a aprins din pricina mâniei Mele: va arde până în fundul locuinței morților, va mânca pământul și roadele lui și va pârjoli temeliile munților.
\par 23 Voi strânge împotriva lor necazuri și voi cheltui asupra lor toate săgețile Mele;
\par 24 Istoviți vor fi de foame și prăpădiți de lingoare și molimă rea; voi trimite asupra lor dinții fiarelor, veninul târâtoarelor din pulbere voi trimite.
\par 25 De din afară îi va pierde sabia, iar prin case groaza, pierzând pe tânăr și pe tânără, pe copilul de țâță și pe bătrânul acoperit de căruntețe.
\par 26 Am zis: ți voi împrăștia și voi șterge pomenirea lor dintre oameni.
\par 27 Dar am amânat aceasta, pentru răutatea vrăjmașilor, ca vrăjmașii lor să nu se mândrească și să zică: Mâna noastră este puternică și toate acestea nu le-a făcut Domnul.
\par 28 Că aceștia sunt oameni care și-au pierdut mintea și n-au nici o pricepere.
\par 29 O, de ar judeca ei și de s-ar gândi la aceasta! De ar pricepe ce are să fie cu ei mai pe urmă:
\par 30 Cum ar putea unul să pună pe fugă o mie, și doi, zece mii, dacă apărătorul lor nu i-ar vinde și Domnul nu i-ar părăsi!
\par 31 Căci apărătorul lor nu este ca Apărătorul nostru și la aceasta chiar vrăjmașii noștri sunt martori.
\par 32 Că via lor este din vița de vie a Sodomei și din șesurile Gomorei; strugurii lor sunt struguri otrăviți și bobițele lor amare;
\par 33 Vinul lor este venin de scorpion și otravă pierzătoare de aspidă.
\par 34 Au nu sunt acestea ascunse la Mine? Și nu sunt ele pecetluite în cămările Mele?
\par 35 A Mea este răzbunarea și răsplătirea când se va poticni piciorul lor; că aproape este ziua pieirii lor și curând vor veni cele gătite pentru ei.
\par 36 Iar Domnul va judeca pe poporul Său și Se va milostivi asupra robilor Săi, când va vedea că a slăbit tăria lor și că nu se mai află nici robi, nici slobozi.
\par 37 Atunci Domnul va zice: Unde sunt dumnezeii lor și tăria în care nădăjduiau ei?
\par 38 Unde sunt cei ce au mâncat grăsimea jertfelor lor și au băut vinul turnărilor lor? Să se scoale, să vă ajute și să vă fie ocrotire.
\par 39 Vedeți, vedeți, dar, că Eu sunt și nu este alt Dumnezeu afară de Mine: Eu omor și înviez, Eu rănesc și tămăduiesc și nimeni nu poate scăpa din mâna Mea!
\par 40 Eu ridic la cer mâna Mea și Mă jur pe dreapta Mea și zic: Viu sunt Eu în veac!
\par 41 Când voi ascuți sabia Mea cea lucitoare și va începe mâna Mea a judeca, Mă voi răzbuna pe vrăjmașii Mei și celor ce Mă urăsc le voi răsplăti.
\par 42 Adăpa-voi săgețile Mele cu sânge și sabia Mea se va sătura de carnea și de sângele celor uciși și robiți și de capetele căpeteniilor vrăjmașului.
\par 43 Veseliți-vă, ceruri, împreună cu El și vă închinați Lui toți îngerii lui Dumnezeu! Veseliți-vă, neamuri, împreună cu poporul Lui și să se întărească toți fiii lui Dumnezeu! Căci El va răzbuna sângele robilor Săi și va răsplăti cu răzbunare vrăjmașilor Săi și celor ce-L urăsc le va răsplăti și va curăți Domnul pământul poporului Său!"
\par 44 În ziua aceea a scris Moise cântarea aceasta și a spus-o fiilor lui Israel. Atunci a venit Moise la popor și a rostit toate cuvintele cântării acesteia în auzul poporului, el și Iosua, fiul lui Navi.
\par 45 Iar după ce a rostit Moise toate cuvintele acestea înaintea a tot Israelul, le-a zis:
\par 46 "Puneți la inima voastră toate cuvintele pe care vi le-am spus eu astăzi și să le lăsați moștenire copiilor voștri, ca să se silească și ei a împlini toate poruncile legii acesteia;
\par 47 Căci acestea nu sunt în deșert date vouă, ci acestea sunt viața voastră și prin acestea veți trăi multă vreme în pământul acela în care treceți acum peste Iordan, ca să-l stăpâniți".
\par 48 Tot în ziua aceea a grăit Domnul cu Moise și a zis:
\par 49 "Suie-te în muntele acesta al Abarimului, în muntele Nebo, care este în pământul Moabului, în fața Ierihonului, și privește asupra Canaanului, pe care-l dau în stăpânirea fiilor lui Israel,
\par 50 Și mori pe munte și te adaugă la poporul tău, cum a murit și Aaron, fratele tău, pe muntele Hor și s-a adăugat la poporul său,
\par 51 Pentru că ați greșit înaintea Mea în mijlocul fiilor lui Israel, la apele Meribei, la Cadeș, în pustiul Sin, și pentru că n-ați arătat sfințenia Mea între fiii lui Israel.
\par 52 Numai de departe vei vedea pământul pe care Eu îl dau fiilor lui Israel, dar de intrat nu vei intra în pământul acela".

\chapter{33}

\par 1 Iată acum și binecuvântarea cu care Moise, omul lui Dumnezeu, a binecuvântat pe fiii lui Israel înainte de moartea sa.
\par 2 Și a zis el: "Venit-a Domnul din Sinai și ni S-a descoperit întru slava Sa în Seir; strălucit-a din Munții Paranului și a ieșit cu mulțime mare de sfinți, având la dreapta focul legii.
\par 3 Cu adevărat El a iubit pe poporul Său. Toți sfinții Lui sunt sub mâna Lui și au căzut la picioarele Lui ca să asculte cuvintele Lui.
\par 4 Moise ne-a dat o lege, moștenire a obștii lui Iacov;
\par 5 El a fost regele lui Israel, când se adunau căpeteniile popoarelor împreună cu semințiile lui Israel.
\par 6 Să trăiască Ruben și să nu moară, și Simeon să nu fie puțin la număr!
\par 7 Iar pentru Iuda a zis acestea: "Ascultă, Doamne, glasul lui Iuda și adu-l la poporul său; cu mâinile sale să se apere și Tu să-i fii ajutor împotriva vrăjmașilor lui".
\par 8 Pentru Levi a zis: "Urimul Tău, Doamne, și Tumimul Tău să fie pentru bărbatul Tău cel sfânt, pe care Tu l-ai încercat la Massa și cu care Tu Te-ai certat la apele Meribei;
\par 9 Care a zis de tatăl său și de mama sa: "Nu i-am văzut", și pe frații săi nu i-a cunoscut și de fiii săi nu știe nimic; căci ei au ținut cuvintele Tale și legământul Tău l-au păzit;
\par 10 Învață pe Iacov legile Tale și pe Israel poruncile Tale; pune tămâie înaintea feței Tale și arderi de tot pe jertfelnicul Tău.
\par 11 Binecuvintează, Doamne, puterea lui și lucrul mâinilor lui fie-ți plăcut; lovește coapsele celar ce se ridică împotriva lui și celor ce-l urăsc, ca să nu se poată împotrivi".
\par 12 Pentru Veniamin a zis: "Iubitul Domnului va sta lângă El fără primejdie și Dumnezeu îl va ocroti în toată vremea și el va odihni pe umerii Lui".
\par 13 Pentru Iosif a zis: "Să binecuvânteze Domnul pământul lui cu darurile cele alese ale cerului, cu roua și cu darurile adâncului celui dedesubt;
\par 14 Cu roade alese din cele pe care le face să crească soarele și cu cele mai bune roade care odrăslesc în fiecare lună;
\par 15 Cu cele mai alese din câte dau munții cei vechi și cu darurile alese ale dealurilor celor veșnice;
\par 16 Și cu darurile cele alese ale pământului și ale celor ce-l umplu. Binecuvântarea Celui ce S-a arătat în rug să vină pe capul lui Iosif, pe creștetul celui dintâi dintre frații lui!
\par 17 Frumusețea lui să fie ca a taurului întâi născut și coarnele lui să fie ca și coarnele bivolului; cu acelea va împunge popoarele toate până la marginile pământului; acestea sunt mulțimile mari ale lui Efraim, acestea sunt miile lui Manase".
\par 18 Pentru Zabulon a zis: "Veselește-te, Zabulon, în căile tale și tu, Isahare, în corturile tale!
\par 19 Chema-vor aceștia poporul pe munte și acolo vor junghia jertfele cele legiuite, căci se hrănesc cu bogăția mării și cu comorile cele ascunse în nisip".
\par 20 Pentru Gad a zis: "Binecuvântat. să fie cel ce a sporit pe Gad, care odihnește ca un pui de leu și sfărâmă și braț și cap.
\par 21 Alesu-și-a el pârga cea bună a țării; acolo fost-a păstrată o moșie de căpetenie; venit-a în fruntea poporului și a împlinit dreptatea Domnului și judecățile lui Israel".
\par 22 Pentru Dan a zis: "Dan este pui de leu, care se aruncă din Vasan".
\par 23 Pentru Neftali a zis: "Neftali, tu cel sătul de bunăvoință și plin de binecuvântarea Domnului, ia marea și partea cea de miazăzi în stăpânire".
\par 24 Pentru Așer a zis: "Binecuvântat să fie Așer între fii, iubit să fie de frații lui și să-și afunde în untdelemn piciorul lui.
\par 25 De fier și de aramă să fie zăvoarele și liniștea ta să țină cît zilele vieții tale.
\par 26 Nimeni, o Israele, nu este ca Dumnezeu, Care să meargă pe ceruri întru ajutorul tău și pe nori întru slava Sa.
\par 27 Dumnezeu este liman din vremi străvechi; căci cu brațul Lui cel veșnic El te sprijină și din fața ta gonind vrăjmașii, zice: "Stârpește-i!"
\par 28 Israel locuiește neprimejduit. Ochiul lui Iacov privește îmbelșugat de pâine și de vin, și cerurile lui picură rouă.
\par 29 Ferice de tine, Israele! Cine este asemenea ție, popor izbăvit de Domnul, scutul și ajutorul tău, sabia și slava ta. Vrăjmașii tăi se vor da înapoi înaintea ta și tu vei călca peste grumajii lor!"

\chapter{34}

\par 1 Atunci s-a suit Moise din șesurile Moabului în Muntele Nebo, pe vârful Fazga, care este în fața Ierihonului, și i-a arătat Domnul tot pământul Galaad până la Dan,
\par 2 Tot pământul lui Neftali, tot pământul lui Efraim și Manase și tot pământul lui Iuda până la marea cea de la asfințit,
\par 3 Partea de la miazăzi a țării, șesul Ierihonului, cetatea Palmierilor, până la Țoar.
\par 4 Și i-a zis Domnul: "Iată pământul pentru care M-am jurat lui Avraam, lui Isaac și lui Iacov, zicând: Seminției tale îl voi da. Te-am învrednicit să-l vezi cu ochii tăi; dar în el nu vei intra!
\par 5 Și a murit Moise, robul lui Dumnezeu, acolo, în pământul Moabului, după Cuvântul Domnului;
\par 6 Și a fost îngropat în vale, în pământul Moabului, în fața Bet-Peorului, dar nimeni nu știe mormântul lui nici până în ziua de astăzi.
\par 7 Și era Moise de o sută douăzeci de ani, când a murit; dar vederea lui nu slăbise și tăria lui nu se împuținase.
\par 8 Și au plâns fiii lui Israel pe Moise, în șesurile Moabului, la Iordan, aproape de Ierihon, treizeci de zile, până s-au împlinit zilele de jelit și de plâns după Moise.
\par 9 Iar Iosua, fiul lui Navi, s-a umplut de duhul înțelepciunii, pentru că își pusese Moise mâinile asupra lui, și i s-au supus fiii lui Israel și au făcut așa după cum le poruncise Domnul prin Moise.
\par 10 De atunci nu s-a mai ridicat în Israel prooroc asemenea lui Moise pe care Dumnezeu să-l fi cunoscut față către față,
\par 11 Nici să săvârșească toate semnele și minunile cu care Domnul l-a trimis în pământul Egiptului asupra lui Faraon și asupra tuturor dregătorilor lui și asupra a tot pământul lui;
\par 12 Nici să facă cu mână tare și cu mari înfricoșări ceea ce a făcut Moise înaintea ochilor a tot Israelul.


\end{document}