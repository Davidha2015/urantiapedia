\begin{document}

\title{Iosua}


\chapter{1}

\par 1 După moartea lui Moise, robul Domnului, a grăit Domnul cu Iosua, fiul lui Navi, slujitorul lui Moise, și a zis:
\par 2 "Moise, robul Meu, a murit. Scoală dar și treci Iordanul tu și tot poporul acesta, în țara pe care o voi da fiilor lui Israel.
\par 3 Tot locul pe care vor călca tălpile picioarelor voastre, îl voi da vouă, cum am spus lui Moise:
\par 4 De la pustie și de la Libanul acesta până la râul cel mare, până la râul Eufratului și până la marea cea mare spre asfințitul soarelui vor fi hotarele voastre.
\par 5 Nimeni nu se va putea împotrivi ție, în toate zilele vieții tale. Precum am fost cu Moise, așa voi fi și cu tine; nu Mă voi depărta de tine și nu te voi părăsi.
\par 6 Fii tare și curajos, că tu vei împărți poporului acestuia, prin sorți, țara pe care M-am jurat părinților lor să le-o dau.
\par 7 Fii dar tare și foarte curajos, ca să păzești și să împlinești toată legea pe care ți-a încredințat-o Moise, robul Meu; să nu te abați de la ea nici la dreapta nici la stânga, ca să pricepi toate câte ai de făcut.
\par 8 Să nu se pogoare cartea legii acesteia de pe buzele tale, ci călăuzește-te de ea ziua și noaptea, ca să plinești întocmai tot ce este scris în ea; atunci vei fi cu izbândă în căile tale și vei păși cu spor.
\par 9 Iată îți poruncesc: Fii tare și curajos, să nu te temi, nici să te spăimântezi, căci Domnul Dumnezeul tău este cu tine pretutindenea, oriunde vei merge".
\par 10 Atunci a poruncit Iosua căpeteniilor poporului și a zis:
\par 11 "Treceți prin tabără și porunciți poporului și-i ziceți: Pregătiți-vă merinde de drum, că după trei zile veți trece peste Iordanul acesta, ca să mergeți și să luați în stăpânire pământul pe care Domnul Dumnezeul părinților voștri are să vi-l dea".
\par 12 Iar seminției lui Ruben și seminției lui Gad și la jumătate din seminția lui Manase, Iosua le-a zis:
\par 13 "Aduceți-vă aminte ce v-a poruncit Moise, sluga Domnului, când v-a zis: Domnul Dumnezeul vostru v-a liniștit și v-a dat pământul acesta.
\par 14 Femeile voastre, copiii voștri și vitele voastre să rămână în pământul pe care vi l-a dat Moise dincoace de Iordan, iar voi toți câți puteți lupta, înarmându-vă, duceți-vă înaintea fraților voștri și le ajutați,
\par 15 Până ce Domnul Dumnezeul vostru va liniști și pe frații voștri, cum va liniștit și pe voi, și până ce vor primi și ei de moștenire pământul pe care Domnul Dumnezeul vostru îl dă lor. Atunci vă veți întoarce în moștenirea voastră și veți stăpâni pământul pe care Moise, sluga Domnului, vi l-a dat dincolo de Iordan, spre răsăritul soarelui".
\par 16 Iar ei au răspuns lui Iosua și au zis: "Toate, oricâte ne vei porunci, vom face și oriunde ne vei trimite, vom merge.
\par 17 Cum am ascultat pe Moise, așa te vom asculta și pe tine, numai să fie Domnul Dumnezeul tău cu tine cum a fost cu Moise.
\par 18 Tot cel ce se va împotrivi poruncilor tale și nu va asculta cuvintele tale, în toate câte vei porunci, să moară. Dar fii tare și curajos!"

\chapter{2}

\par 1 Atunci Iosua, fiul lui Navi, a trimis în taină din Sitim doi tineri să iscodească țara și a zis: "Duceți-vă și cercetați țara și mai ales Ierihonul!" Și s-au dus cei doi tineri și, ajungând la Ierihon, au intrat în casa unei desfrânate, al cărei nume era Rahab, și au rămas să se odihnească acolo.
\par 2 Și s-a dat de știre regelui Ierihonului: "Iată, niște oameni din fiii lui Israel au venit aici în noaptea aceasta, ca să iscodească țara!"
\par 3 Iar regele Ierihonului a trimis la Rahab să i se spună: "Scoate pe oamenii care au intrat în casa ta, în noaptea aceasta, că au venit să iscodească țara".
\par 4 Femeia însă, luând pe cei doi oameni, i-a ascuns și a zis: "Adevărat, au venit la mine niște oameni,
\par 5 Dar în amurg, când se închideau porțile, bărbații au plecat și nu știu unde s-au dus. Alergați după ei și-i veți ajunge".
\par 6 Apoi ea a suit pe cei doi oameni pe acoperiș și i-a ascuns în niște fuioare de in ce se aflau pe acoperișul casei ei.
\par 7 Iar trimișii regelui au alergat după ei pe drumul cel către Iordan, până la vad, și porțile s-au închis îndată ce au ieșit cei ce urmăreau pe iscoade.
\par 8 Înainte însă de a adormi iscoadele femeia s-a suit la dânșii pe acoperiș,
\par 9 Și le-a zis: "Știu că Domnul v-a dat vouă pământul acesta, căci frica voastră a căzut asupra noastră și toți locuitorii pământului acestuia au frică de voi,
\par 10 Pentru că am auzit noi cum a secat Domnul Dumnezeu înaintea voastră Marea Roșie, când ah ieșit din Egipt, și câte ați făcut voi cu cei doi regi ai Amoreilor peste Iordan, cu Sihon și cu Og, pe care i-ați pierdut.
\par 11 Când am auzit noi de acestea, ne-a slăbit inima și în nici unul din noi n-a mai rămas bărbăție în fața voastră, căci Domnul Dumnezeul vostru este Dumnezeu în cer sus și pe pământ jos.
\par 12 Acum dar jurați-mi pe Domnul Dumnezeul vostru că, precum am făcut eu milă cu voi, veți face și voi milă cu casa tatălui meu și dați-mi semn de nădejde,
\par 13 Că veți lăsa cu viață pe tatăl meu, pe mama mea, pe frații mei, pe surorile mele, toată casa mea și veți izbăvi sufletul meu de la moarte".
\par 14 Iar oamenii aceia au zis către dânsa: "Sufletul nostru îl vom pune pentru voi". Și ea a zis: "Când Domnul vă va da cetatea aceasta, să faceți cu mine milă și dreptate".
\par 15 Apoi le-a dat drumul cu o frânghie pe fereastră, căci casa ei era în zidul cetății și ea locuia chiar deasupra zidului.
\par 16 Și le-a zis: "Duceți-vă în munte, ca să nu vă întâlniți cu cei ce vă urmăresc și stați acolo ascunși trei zile, până se vor întoarce cei ce vă urmăresc și apoi vă veți duce în drumul vostru".
\par 17 Iar oamenii aceia au zis către dânsa: "De nu vei face cum îți vom zice, vom fi slobozi de acest jurământ al tău:
\par 18 Iată, când vom intra noi în cetate pe partea aceasta, tu să pui semn această funie roșie la fereastra pe care ne-ai slobozit, iar pe tatăl tău, pe mama ta, pe frații tăi și pe toți cei din familia tatălui tău să-i aduni în casa ta.
\par 19 Și de va ieși cineva afară pe ușa casei tale, sângele aceluia să fie asupra capului lui, și noi vom fi slobozi de acest jurământ al tău; iar cine va fi cu tine în casa ta, sângele aceluia să fie asupra capului nostru, de se va atinge de el mâna cuiva.
\par 20 Dacă însă ne va năpăstui cineva și va descoperi fapta noastră, atunci vom fi slobozi de acest jurământ al tău".
\par 21 Și ea le-a zis: "Să fie cum ai grăit!" Apoi le-a dat drumul și s-au dus, iar ea a legat la fereastră funia cea roșie.
\par 22 Ducându-se oamenii aceia și ajungând în munte, au așteptat acolo trei zile, până ce s-au întors cei ce-i urmăreau, care-i căutaseră pe toate drumurile și nu-i găsiră.
\par 23 Întorcându-se apoi, cei doi tineri s-au coborât din munte, au trecut Iordanul și au venit la Iosua, fiul lui Navi, și i-au povestit tot ce se întâmplase cu ei,
\par 24 Și au zis către Iosua: "Domnul Dumnezeul nostru a dat tot pământul acesta în mâinile noastre și toți cei ce locuiesc în țara aceea tremură de frica noastră".

\chapter{3}

\par 1 Sculându-se apoi Iosua a doua zi dis-de-dimineață, a pornit de la Sitim și a venit până la Iordan, el și toți fiii lui Israel, și au poposit acolo înainte de a-l trece.
\par 2 Iar după trei zile au trecut vestitori prin tabără,
\par 3 Și au poruncit poporului, zicând: "Când veți vedea chivotul legământului Domnului Dumnezeului vostru și preoții voștri și pe leviții cei ce-l duc, atunci să porniți și voi de la locul vostru și să mergeți după el.
\par 4 Iar depărtarea între voi și el să fie ca de două mii de coți; să nu vă apropiați prea mult de el, ca să puteți vedea bine calea pe care aveți să mergeți; căci înainte n-ați mai umblat pe calea aceasta niciodată".
\par 5 Și a mai zis Iosua către popor: "Sfințiți-vă pentru dimineață, căci mâine are să facă Domnul minuni între voi".
\par 6 Iar preoților le-a zis Iosua: "Ridicați chivotul legământului Domnului și mergeți înaintea poporului!" Și au luat preoții chivotul legământului Domnului și au purces înaintea poporului.
\par 7 Atunci a zis Domnul către Iosua: "În ziua aceasta voi începe a te preamări înaintea ochilor tuturor fiilor lui Israel, ca să cunoască ei că, precum am fost cu Moise, așa am să fiu și cu tine.
\par 8 Tu însă să poruncești preoților care poartă chivotul legământului și să zici: "Îndată ce veți intra în mijlocul apelor Iordanului, să vă opriți în Iordan!"
\par 9 Iar Iosua a zis către fiii lui Israel: "Veniți încoace și ascultați cuvintele Domnului Dumnezeului vostru.
\par 10 Din aceasta veți cunoaște că în mijlocul vostru este Dumnezeul cel viu Care va alunga de la voi pe Canaanei, pe Hetei, pe Hevei, pe Ferezei, pe Gherghesei, pe Amorei și pe Iebusei.
\par 11 Iată chivotul legământului Domnului a tot pământul va trece înaintea voastră peste Iordan.
\par 12 Să vă alegeți doisprezece oameni dintre fiii lui Israel, câte un om din fiecare seminție;
\par 13 Și îndată ce tălpile picioarelor preoților care duc Chivotul Domnului, Stăpânul a tot pământul, vor călca în apa Iordanului, apele Iordanului se vor despărți: cele de la vale se vor scurge, iar cele care vin din sus se vor opri ca un perete".
\par 14 Deci poporul a pornit de la corturile sale, ca să treacă Iordanul, iar preoții care duceau chivotul legământului Domnului mergeau înaintea poporului.
\par 15 Îndată ce preoții cei ce duceau chivotul au intrat în Iordan și picioarele preoților, care duceau chivotul, s-au afundat în apa Iordanului,
\par 16 Apa care curgea din sus s-a oprit și s-a făcut perete pe o foarte mare depărtare, până la cetatea Adam, care e lângă Țartan, iar cea care curgea spre marea cea din Arabah, spre Marea Sărată, s-a scurs și a secat, iar poporul a trecut în fala Ierihonului. Atunci Iordanul umplea matca sa și ieșea din toate malurile sale, ca în timpul secerișului grâului.
\par 17 Preoții care duceau chivotul legământului Domnului stăteau ca pe uscat în mijlocul Iordanului, cu picioarele neudate; iar fiii lui Israel au mers ca pe uscat, până ce tot poporul a trecut prin Iordan.

\chapter{4}

\par 1 După ce a isprăvit tot poporul de trecut Iordanul; a grăit Domnul către Iosua și a zis:
\par 2 "Ia din popor doisprezece bărbați, câte un om din fiecare seminție,
\par 3 Și le poruncește: "Luați din mijlocul Iordanului douăsprezece pietre, duceți-le cu voi și le puneți în tabăra voastră, unde aveți să tăbărâți la noapte".
\par 4 Și a chemat Iosua doisprezece bărbați pe care și-i alesese dintre fiii lui Israel, câte un om din fiecare seminție,
\par 5 Și le-a zis: "Mergeți înaintea chivotului Domnului Dumnezeului vostru, în mijlocul Iordanului, și luați de acolo și ridicați pe umerii voștri fiecare câte o piatră, după numărul celor douăsprezece seminții ale lui Israel,
\par 6 Ca să vă fie acestea puse pentru totdeauna ca semn în mijlocul vostru și, când vă vor întreba mâine copiii voștri și vor zice: Ce înseamnă aici pietrele acestea?
\par 7 Atunci veți spune fiilor voștri: Apele Iordanului s-au despărțit înaintea chivotului legământului Domnului a tot pământul, când acesta l-a trecut. Și așa pietrele acestea vor fi pentru fiii lui Israel amintire pentru vecie".
\par 8 Și au făcut fiii lui Israel așa cum le-a poruncit Iosua: după ce au isprăvit fiii lui Israel de trecut Iordanul, au luat douăsprezece pietre din Iordan, precum Domnul poruncise lui Iosua, după numărul semințiilor fiilor lui Israel și le-au dus cu ei în tabără și le-au pus acolo.
\par 9 Iosua însă a pus alte douăsprezece pietre în mijlocul Iordanului, pe locul unde au stat picioarele preoților care duceau chivotul legământului Domnului, și sunt acolo până în ziua de astăzi.
\par 10 Preoții care duceau chivotul Domnului au stat în mijlocul Iordanului până ce a isprăvit Iosua toate câte îi poruncise Domnul să spună poporului, cum poruncise Moise lui Iosua. Și poporul a grăbit să treacă Iordanul.
\par 11 Iar după ce a trecut tot poporul, a trecut și chivotul Domnului și preoții au mers iarăși în fruntea poporului.
\par 12 Atunci au trecut și fiii lui Ruben, fiii lui Gad și jumătate din seminția lui Manase, gata de război, înaintea fiilor lui Israel, cum le poruncise Moise.
\par 13 Ca la patruzeci de mii de oameni înarmați pentru război au trecut înaintea Domnului, ca să lupte împotriva cetății Ierihonului.
\par 14 În ziua aceea a preamărit Domnul pe Iosua înaintea a tot neamul lui Israel și s-au temut de dânsul cât a trăit, cum se temuseră și de Moise.
\par 15 Apoi a grăit Domnul cu Iosua și a zis:
\par 16 "Poruncește preoților care duc chivotul mărturiei Domnului, să iasă din Iordan".
\par 17 Și a poruncit Iosua preoților și a zis: "Ieșiți din Iordan!"
\par 18 Și cum au ieșit din Iordan preoții, cei ce duceau chivotul legământului Domnului, și și-au pus picioarele pe uscat, apa Iordanului a pornit pe albia sa și a curs ca mai înainte, până peste maluri.
\par 19 Poporul a ieșit din Iordan în ziua a zecea a lunii întâi și au tăbărât fiii lui Israel la Ghilgal, în partea de răsărit a Ierihonului.
\par 20 Iar cele douăsprezece pietre pe care le luaseră ei din Iordan, le-a așezat Iosua în Ghilgal.
\par 21 Și a zis fiilor lui Israel: "Când vă vor întreba fiii voștri mâine și vor zice: Ce înseamnă aceste pietre?
\par 22 Să spuneți fiilor voștri: Israel a trecut prin Iordanul acesta, ca pe uscat,
\par 23 Căci Domnul Dumnezeul vostru a secat apele Iordanului înaintea lor, până ce le-au trecut, cum făcuse Domnul Dumnezeul vostru și cu Marea Roșie, pe care a secat-o Domnul Dumnezeul nostru înaintea noastră. până ce am trecut-o,
\par 24 Ca să cunoască toate neamurile pământului că puterea Domnului este mare și ca voi să vă temeți de Domnul Dumnezeul vostru, în toată vremea".

\chapter{5}

\par 1 Când au auzit regii Amoreilor, care locuiau peste Iordan și regii Canaaneilor, care locuiau pe lângă mare, că Domnul Dumnezeu a secat râul Iordanului înaintea fiilor lui Israel până ce l-au trecut, a slăbit inima lor, s-au înspăimântat și nu mai aveau curaj împotriva fiilor lui Israel.
\par 2 În vremea aceea a zis Domnul către Iosua: "Fă-ți cuțite tăioase de cremene și taie împrejur pe fiii lui Israel a doua oară".
\par 3 Și și-a făcut Iosua cuțite ascuțite de cremene și a tăiat împrejur pe fiii lui Israel, la locul ce se numește Dealul Aralot.
\par 4 Iată pricina pentru care Iosua a tăiat împrejur pe fiii lui Israel: Tot poporul de parte bărbătească care ieși se din Egipt, toți cei buni de război, muriseră pe cale, în pustiu, după ieșirea din Egipt;
\par 5 Căci tot poporul care ieșise din Egipt era tăiat împrejur; iar poporul născut pe cale, în pustiu, după ieșirea din Egipt, nu era tăiat împrejur.
\par 6 Căci fiii lui Israel umblaseră prin pustiu patruzeci și doi de ani, până ce a murit tot poporul bun de război, care ieșise din Egipt și care nu ascultase de glasul Domnului. Acestora Domnul se jurase să nu le dea voie să vadă țara pe care El făgăduise cu jurământ să o dea părinților lor, țara în care curge lapte și miere.
\par 7 În locul acestora se ridicaseră fiii lor, pe care i-a tăiat împrejur Iosua, căci nu erau tăiați împrejur, pentru că se născuseră pe cale.
\par 8 După ce poporul s-a tăiat împrejur, a rămas liniștit în tabără, până s-a însănătoșit.
\par 9 Apoi a zis Domnul către Iosua, fiul lui Navi: "În ziua de astăzi am ridicat de pe voi ocara Egiptului", și de aceea se și numește locul acela Ghilgal, până în ziua aceasta.
\par 10 Fiii lui Israel au stat cu tabăra la Ghilgal și au făcut acolo în șesul Ierihonului Paștile în ziua a paisprezecea a lunii, seara.
\par 11 Iar a doua zi de Paști au mâncat din roadele pământului acestuia azime și pâine nouă.
\par 12 În ziua aceea a încetat mana de a mai cădea și de a doua zi, după ce au mâncat din roadele pământului, fiii lui Israel n-au mai avut mană, ci au mâncat anul acela din roadele Zării Canaanului.
\par 13 Aflându-se însă Iosua aproape de Ierihon, a căutat cu ochii săi și iată stătea înaintea lui un om; acela avea în mână o sabie goală. Și apropiindu-se Iosua de dânsul, i-a zis: "De-ai noștri ești sau ești dintre dușmanii noștri?"
\par 14 Iar acela a răspuns: "Eu sunt căpetenia oștirii Domnului și am venit acum!" Atunci Iosua a căzut cu fala la pământ, s-a închinat și a zis către acela: "Stăpâne, ce poruncești slugii tale?"
\par 15 Zis-a către Iosua căpetenia oștirii Domnului: "Scoate-ți încălțămintea din picioare, că locul pe care stai tu acum este sfânt!" Și a făcut Iosua așa.

\chapter{6}

\par 1 Ierihonul însă era încuiat și întărit de frica fiilor lui Israel; nimeni nu intra în el și nimeni nu ieșea.
\par 2 Atunci a zis Domnul către Iosua: "Iată, Eu voi da în mâinile tale Ierihonul, pe regii lui și pe puternicii lui.
\par 3 Duceți-vă împrejurul cetății toți cei buni de război și înconjurați cetatea câteodată pe zi. Aceasta să o faceți șase zile.
\par 4 Șapte preoți să poarte înaintea chivotului șapte trâmbițe din corn de berbec, iar în ziua a șaptea să ocoliți cetatea de șapte ori și preoții să trâmbițeze din trâmbițe.
\par 5 Când vor suna din trâmbița de corn de berbec și când veți auzi sunetul de trâmbiță, atunci tot poporul să strige cu glas tare deodată, și când vor striga ei zidurile cetății se vor prăbuși. Atunci tot poporul să năvălească în cetate, plecând fiecare din partea unde se află".
\par 6 Și a chemat Iosua, fiul lui Navi, pe preoți și le-a zis: "Luați chivotul legământului și șapte preoți să poarte șapte trâmbițe din corn de berbec înaintea chivotului Domnului".
\par 7 Și le-a mai zis acestora să spună poporului: "Mergeți și ocoliți cetatea, iar cei înarmați să meargă înaintea chivotului Domnului!"
\par 8 Îndată ce Iosua a spus acestea poporului, cei șapte preoți, care purtau cele șapte trâmbițe sfinte înaintea Domnului, au pornit și au început a suna tare din trâmbițe și chivotul legământului Domnului mergea în urma lor.
\par 9 Cei înarmați mergeau înaintea preoților care sunau din trâmbițe, iar cei din urmă veneau după chivotul Domnului, trâmbițând în mers din trâmbițe.
\par 10 Poporului însă Iosua i-a poruncit și a zis: "Să nu strigați! Nimeni să nu audă glasul vostru și nici o vorbă să nu iasă din gura voastră până în ziua când vă voi zice eu să strigați; atunci să strigați".
\par 11 Și așa chivotul Domnului a plecat împrejurul cetății și a înconjurat-o o dată; apoi a venit în tabără și a rămas în tabără.
\par 12 A doua zi Iosua s-a sculat dis-de-dimineață și preoții au ridicat chivotul Domnului;
\par 13 Și cei șapte preoți, care purtau cele șapte trâmbițe, mergeau înaintea chivotului Domnului și trâmbițau din trâmbițe; cei înarmați mergeau înaintea lor, iar celălalt popor venea în urma chivotului Domnului și preoții trâmbițau din trâmbițe.
\par 14 Astfel și a doua zi au înconjurat cetatea o dată și s-au întors în tabără. Și au făcut așa șase zile.
\par 15 În ziua a șaptea s-au sculat de dimineață, în revărsatul zorilor și au mers tot așa împrejurul cetății de șapte ori; numai în această zi au înconjurat cetatea de șapte ori.
\par 16 Iar când au sunat preoții a șaptea oară din trâmbițe, Iosua a zis către fiii lui Israel: "Strigați, că v-a dat Domnul cetatea!
\par 17 Cetatea va fi sub blestem și tot ce este în ea e al Domnului puterilor; numai Rahab desfrânata să rămână vie, ea și tot cel ce va fi în casa ei, pentru că ea a ascuns iscoadele pe care le-am trimis noi.
\par 18 Voi însă să vă păziți foarte de tot ce este dat blestemului, ca să nu cădeți și voi sub blestem, dacă ați lua ceva din cele date spre nimicire și pentru ca asupra taberei lui Israel să nu vină blestemul și să-i aducă pieire.
\par 19 Tot aurul și argintul și vasele de aramă și de fier să fie sfințenie a Domnului și să intre în vistieria Domnului".
\par 20 Atunci au trâmbițat preoții din trâmbițe. Și cum a auzit poporul glasul de trâmbiță, a strigat tot poporul împreună cu glas tare și puternic și s-au prăbușit toate zidurile împrejurul cetății până în temelie și a intrat tot poporul în cetate, fiecare din partea unde era, și au luat cetatea.
\par 21 Și au dat junghierii tot ce era în cetate: bărbați și femei și tineri și bătrâni și boi și oi și asini, tot au trecut prin ascuțișul sabiei.
\par 22 Iar celor doi tineri care iscodiseră țara Iosua le-a zis: "Duceți-vă în casa desfrânatei aceleia, scoateți-o de acolo pe ea și pe toți cei ce vor fi cu dânsa, întrucât v-ați jurat ei".
\par 23 Și s-au dus tinerii care iscodiseră cetatea, la casa desfrânatei și au scos pe Rahab desfrânata, pe tatăl ei, pe mama ei, pe frații ei și pe toate rudele ei, i-au scos și i-au pus afară din tabăra Israeliților,
\par 24 Iar cetatea și tot ce era în ea au ars cu foc; numai aurul și argintul și vasele de aramă și de fier le-au dat ca să le ducă Domnului în vistieria casei Domnului.
\par 25 Pe Rahab desfrânata însă, casa tatălui ei și pe toți care erau la dânsa, Iosua i-a lăsat cu viață și trăiește ea în mijlocul lui Israel până în ziua de astăzi, pentru că a ascuns iscoadele, care fuseseră trimise de Iosua să iscodească Ierihonul.
\par 26 În ziua aceea s-a jurat Iosua și a zis: "Blestemat să fie înaintea Domnului tot cel ce se va scula și va zidi cetatea aceasta a Ierihonului: pe fiul său cel întâi-născut să pună temeliile ei, iar porțile ei să le așeze pe fiul său cel mai mic". Și așa a și făcut Ozan din Betel. Acesta pe Abiron, întâiul său născut, a pus temeliile ei, și pe fiul său cel mai mic a așezat porțile ei.
\par 27 Domnul însă era cu Iosua și faima lui s-a răspândit în toată țara.

\chapter{7}

\par 1 Iar fiii lui Israel au făcut păcat mare: au luat din cele date nimicirii; căci Acan, fiul lui Carmi, fiul lui Zabdi, fiul lui Zerah, din tribul lui Iuda, a luat din cele date spre nimicire și s-a aprins mânia Domnului asupra fiilor lui Israel.
\par 2 Din Ierihon Iosua a trimis oameni asupra cetății Ai, care e aproape de Bet-Aven, la răsărit de Betel, și le-a zis: "Duceți-vă de iscodiți țara!" Și s-au dus oamenii aceștia și au iscodit cetatea Ai.
\par 3 Apoi întorcându-se la Iosua, i-au spus: "Să nu meargă tot poporul, ci să meargă numai două sau trei mii de oameni și să lovească cetatea Ai. Nu obosi până acolo tot poporul, pentru că acolo sunt puțini dușmani".
\par 4 Și s-au dus acolo ca la trei mii de oameni, dar aceștia au luat-o la fugă în fața celor din Ai.
\par 5 Și locuitorii din Ai au ucis din ei vreo treizeci și șase de oameni și i-au fugărit de la poartă până la Șebarim, iar pe coasta dealului i-a bătut; din care pricină inima poporului a slăbit și și-au pierdut tot curajul.
\par 6 Atunci Iosua și-a sfâșiat veșmintele sale, a căzut cu fața la pământ înaintea chivotului Domnului și a stat așa până seara și el și bătrânii lui Israel și și-au presărat pulbere pe capetele lor.
\par 7 Atunci a zis Iosua: "O, Doamne Dumnezeule, de ce a trecut robul Tău poporul acesta peste Iordan? Oare, ca să-l dai în mâinile Amoreilor și să ne pierzi? Mai bine rămâneam să locuim de cealaltă parte de Iordan!
\par 8 Ce să zic eu acum, când Israel a fugit înapoi dinaintea vrăjmașilor săi?
\par 9 Auzind Canaaneii și toți locuitorii țării, ne vor împresura și ne vor pierde de pe pământ. Și ce vei face Tu pentru numele Tău cel mare?"
\par 10 Iar Domnul a zis către Iosua: "Scoală, pentru ce ai căzut cu fața la pământ?
\par 11 A păcătuit poporul și a călcat legământul Meu pe care l-am încheiat cu el; au furat din lucrurile date spre nimicire, au mințit și le-au pus printre lucrurile lor.
\par 12 De aceea fiii lui Israel n-au putut sta împotriva vrăjmașilor lor și au fugit înapoi, căci au căzut sub blestem și nu voi mai fi cu voi de nu veți ridica blestemul dintre voi.
\par 13 Scoală dar și sfințește poporul și zi: Sfințiți-vă pentru mâine; căci așa zice Domnul Dumnezeul lui Israel: Blestemul e în mijlocul tău, Israel. De aceea tu nu poți să stai împotriva vrăjmașilor tăi până nu vei îndepărta din mijlocul tău blestemul.
\par 14 Mâine să vă apropiați toți, pe seminții; iar seminția pe care o va arăta Domnul să se apropie pe familii și familia pe care o va arăta Domnul să se apropie pe case; și casa pe care o va arăta Domnul să se apropie om cu om.
\par 15 Iar omul care se va dovedi că a furat lucru dat spre nimicire, să se ardă cu foc și el și toate câte are, pentru că a călcat legământul Domnului și a făcut fărădelege în Israel".
\par 16 Atunci, sculându-se Iosua a doua zi dis-de-dimineață, a poruncit lui Israel să se apropie pe seminții; și a fost arătată seminția lui Iuda.
\par 17 După aceea a poruncit să se apropie seminția lui Iuda pe familii și a fost arătată familia lui Zerah. Și a poruncit să se apropie familia lui Zerah pe case și a fost dată în vileag casa lui Zabdi.
\par 18 Și a poruncit să se apropie casa lui Zabdi om cu om și a fost arătat Acan, fiul lui Carmi, fiul lui Zabdi, fiul lui Zerah, din seminția lui Iuda.
\par 19 Atunci Iosua a zis către Acan: "Fiul meu, dă astăzi slavă Domnului Dumnezeului lui Israel și fă mărturisire înaintea Lui și arată-mi ce-ai făcut. Să nu ascunzi de mine!"
\par 20 Și răspunzând Acan lui Iosua, a zis: "Adevărat, am păcătuit înaintea Domnului Dumnezeului lui Israel și iată ce am făcut:
\par 21 Am văzut printre prăzi o haină frumoasă pestriță și două sute de sicli de argint, un drug de aur, greu de cincizeci de sicli; acestea mi-au plăcut și le-am luat și iată sunt ascunse în pământ, în mijlocul cortului meu și argintul este pus sub elen.
\par 22 Atunci a trimis Iosua câțiva slujitori și aceștia au alergat la cort, în tabără, și toate acestea erau ascunse În cortul lui și argintul era sub ele.
\par 23 Și au luat ei acestea din cort și le-au adus la Iosua și la bătrânii lui Israel și le-au pus înaintea Domnului.
\par 24 Iar Iosua și împreună cu el tot poporul au luat pe Acan, fiul lui Zerah, argintul, haina, drugul cel de aur, pe fiii lui, pe fiicele lui, boii lui, asinii lui și toate oile lui, cortul lui și tot ce avea el și i-au scos pe ei și toate ale lor în valea Acor.
\par 25 Și a zis Iosua: "Pentru că tu ai adus asupra noastră tulburare, să aducă și Domnul necaz asupra ta în ziua aceasta!" și toți Israeliții i-au ucis cu pietre și, după ce i-au ucis cu pietre, i-au ars cu foc.
\par 26 Apoi au ridicat deasupra lor o grămadă mare de pietre, care se vede și astăzi. După aceasta s-a potolit mânia Domnului. De aceea locul acela se numește valea Acor până în ziua de astăzi.

\chapter{8}

\par 1 Atunci a zis Domnul către Iosua: "Nu te teme, nici nu te înspăimânta! Ia cu tine toți bărbații buni de luptă și, sculându-te, du-te la Ai. Iată Eu voi da în mâinile tale pe regele din Ai și pe poporul lui, cetatea și ținutul lui.
\par 2 Și să faci cu Ai și cu regele său tot ceea ce ai făcut cu Ierihonul și cu regele lui. Numai prada lui și dobitoacele lui împărțiți-le între voi. Pune oameni la pândă înapoia cetății".
\par 3 Și s-a sculat Iosua cu tot poporul bun de război, ca  să meargă asupra cetății Ai. Și a ales Iosua treizeci de mii de oameni viteji și i-a trimis noaptea.
\par 4 Acestora le-a poruncit și le-a zis: "Luați seama să pândiți înapoia cetății, să nu vă depărtați tare de cetate și să fiți cu toții gata.
\par 5 Iar eu și toți cei cu mine vom înainta spre cetate. Și când locuitorii din Ai ne vor ieși înainte, ca prima dată, noi vom fugi de dânșii.
\par 6 Și când vor alerga după noi și-i vom depărta de cetate, vor zice: "Iată fug de noi, ca prima dată".
\par 7 Și când vom fugi noi de ei și ei după noi, atunci voi să ieșiți din ascunzătoare și să cuprindeți cetatea, că Domnul Dumnezeu o va da în mâinile voastre.
\par 8 Iar după ce veți prăda cetatea, să-i dați foc; după cuvântul Domnului să faceți. Iată eu vă poruncesc!"
\par 9 Așa i-a trimis Iosua și ei s-au dus și au stat la pândă între Betel și Ai, spre apus de cetatea Ai. Iar Iosua a rămas în noaptea aceea în mijlocul poporului.
\par 10 Și, sculându-se dis-de-dimineață, Iosua a cercetat poporul și s-a dus el și bătrânii lui Israel în fruntea poporului spre Ai.
\par 11 Și tot poporul bun de luptă, care era cu el, a mers, a înaintat și s-a apropiat de cetate pe partea de răsărit, iar pânda era în partea de apus a cetății. Și a așezat tabăra spre miazănoapte de Ai; iar între el și Ai era o vale.
\par 12 El luase ca la treizeci de mii de oameni și-i pusese la pândă între Betel și Ai, în partea de apus a cetății.
\par 13 Iar poporul l-a așezat tot într-o tabără, care se întindea în partea de miazănoapte a cetății, așa încât coada taberei ajungea spre partea de apus a cetății. Și a venit Iosua în noaptea aceea în mijlocul văii.
\par 14 Și când a văzut aceasta, regele din Ai s-a sculat îndată dis-de-dimineață, a ieșit înaintea lui Israel la luptă, el și tot poporul lui, la un loc hotărât, în șes, și nu știa el că e o pândă asupra lui în spatele cetății.
\par 15 Iar Iosua, ca și cum ar fi fost învins, a fugit cu tot poporul pe calea dinspre pustiu.
\par 16 Iar aceia au strigat tot poporul care era în cetate, ca să-i urmărească și, urmărind ei pe fiii lui Israel, s-au depărtat de cetate.
\par 17 Și n-a rămas în Ai și în Betel nici unul care să nu fi ieșit după Israel. Și urmărind ei pe Israel, și-au lăsat cetatea lor deschisă.
\par 18 Atunci Domnul a zis către Iosua: "Întinde mâna ta cu sulița asupra cetății Ai, căci o voi da în mâinile tale și cei de la pândă vor ieși numaidecât de la locul lor!" Și și-a întins Iosua mâna cu sulița spre cetate.
\par 19 Și atunci cei ce ședeau la pândă sculându-se numaidecât de la locul lor, când și-a întins el mâna, au alergat în cetate, au luat-o și i-au dat repede foc.
\par 20 Atunci, uitându-se locuitorii cetății Ai înapoi și văzând fum ridicându-se din cetatea lor spre cer, nu mai aveau încotro fugi, căci poporul care fugea spre pustiu se întorsese împotriva celor ce-i urmăreau;
\par 21 Și Iosua și tot Israelul, văzând că cei ce șezuseră în ascunzătoare luaseră cetatea și că din cetate se urca fum spre cer, s-au întors înapoi și au început să ucidă pe locuitorii din Ai.
\par 22 Iar cei din cetate au ieșit în întâmpinarea lor, așa că Aiții se aflau în mijlocul taberei Israeliților, dintre în care unii veneau dintr-o parte, iar alții din altă parte.
\par 23 Și i-au ucis așa, încât n-a scăpat cu viață nici unul din ei. Iar pe regele din Ai l-au prins viu și l-au adus la de Iosua.
\par 24 După ce fiii lui Israel au ucis pe toți bărbații din Ai, în câmpia dinspre pustiu și dealurile unde aceștia îi urmăriseră, și după ce aceștia au căzut toți până la unul sub ascuțișul sabiei, Israeliții s-au întors cu toții asupra cetății Ai și au lovit-o cu ascuțișul lui sabiei.
\par 25 Iar toți locuitorii din Ai, bărbați și femei, care au căzut în ziua aceea au fost douăsprezece mii.
\par 26 Și Iosua nu și-a lăsat în jos mâna sa, pe care o întinsese cu sulița, până nu a dat pierzării pe toți locuitorii din Ai.
\par 27 Numai vitele și prada cetății le-au împărțit fiii lui Israel între dânșii, după cuvântul Domnului pe care-l spusese Domnul lui Iosua.
\par 28 Și a ars Iosua cetatea Ai cu foc și a făcut-o dărâmătură veșnică și pustietate până în ziua de astăzi.
\par 29 Iar pe regele din Ai l-a spânzurat de un copac unde a stat până seara; iar după asfințitul soarelui a poruncit Iosua de au coborât trupul lui din copac și l-au aruncat la porțile cetății și au ridicat deasupra lui o movilă de pietre, care este până în și ziua de astăzi.
\par 30 Atunci Iosua a înălțat jertfelnic Domnului Dumnezeului lui Israel pe Muntele Ebal,
\par 31 Cum poruncise Moise, sluga Domnului, fiilor lui Israel, și de care este scris în legea lui Moise; jertfelnicul era din pietre întregi, asupra cărora nimeni n-a ridicat unealtă de fier și el au adus pe el ardere de tot Domnului și au făcut jertfe de împăcare.
\par 32 Și a scris Iosua acolo din nou pe pietre legea pe care Moise o scrisese înaintea fiilor lui Israel.
\par 33 Și tot Israelul cu bătrânii lui, cu căpeteniile lui, cu judecătorii  lui, toți băștinașii și veneticii lui au stat de o parte și de alta a chivotului în fața preoților și a leviților care duceau chivotul legământului Domnului, așa că jumătate din ei se aflau pe muntele Garizim, iar cealaltă jumătate pe muntele Ebal, cum poruncise de mai înainte Moise, sluga Domnului, ca să se binecuvânteze poporul lui Israel.
\par 34 După aceasta a citit Iosua toate cuvintele legii: binecuvântările și blestemul, cum era scris în legea lui Moise.
\par 35 Din toate câte poruncise Moise lui Iosua n-a fost nici un cuvânt pe care Iosua să nu-l fi citit în auzul obștii fiilor lui Israel, înaintea bărbaților, a femeilor, a copiilor și a străinilor care mergeau împreună cu Israel.

\chapter{9}

\par 1 Auzind acestea toți regii Amoreilor, cei de peste Iordan, cei din munte și de la șes, cei de pe tot malul mării celei mari și cei din apropierea Libanului: Heteii, Amoreii, Ghergheseii, Canaaneii, Ferezeii, Heveii și Ibuseii,
\par 2 S-au adunat împreună ca să se lupte toți cu Iosua și cu Israel.
\par 3 Iar locuitorii Ghibeonului, auzind ce a făcut Iosua cu Ierihonul și cu Ai,
\par 4 Au pus la cale un vicleșug, căci s-au dus și au strâns merinde de drum și au pus pe asinii lor saci vechi cu pâine și vin în burdufuri vechi, rupte și cârpite
\par 5 Și în picioarele lor încălțăminte și sandale vechi și peticite, iar pe ei haine rele; pâinea lor de drum era uscată, mucedă și sfărâmată.
\par 6 Așa au venit ei la Iosua în tabăra Israeliților de la Ghilgal și au zis către el și către toți Israeliții: "Noi am venit dintr-o țară foarte depărtată; încheiați dar legământ cu noi".
\par 7 Fiii lui Israel însă au zis către Hevei: "Poate că locuiți aproape de noi? Cum să încheiem legământ cu voi?"
\par 8 Iar ei au zis către Iosua: "Noi suntem robii tăi". Iosua le-a zis: "Cine sunteți voi și de unde ați venit?"
\par 9 Și ei au răspuns: "Robii tăi au venit dintr-o țară foarte depărtată, în numele Domnului Dumnezeului tău, că am auzit de numele Lui și de câte a făcut El în Egipt
\par 10 Și de câte a făcut El celor doi regi ai Amoreilor, care erau dincolo de Iordan, lui Sihon, regele Heșbonului și lui Og, regele Vasanului, care locuia în Aștarot și Edrea.
\par 11 Și auzind acestea, bătrânii noștri și toți locuitorii țării noastre ne-au grăit și au zis: Luați-vă merinde de drum și duceți-vă în întâmpinarea lor și le spuneți: Noi suntem robii voștri. Încheiați dar legământ cu noi!
\par 12 Pâinile acestea erau calde în ziua când am plecat să venim la voi, iar acum iată s-au uscat și s-au mucezit.
\par 13 Aceste burdufuri de vin, pe care le-am umplut noi, iată-le s-au învechit; și îmbrăcămintea noastră și încălțămintea noastră s-au tocit de drumul cel foarte lung".
\par 14 Căpeteniile Israeliților au luat din merindele lor, dar n-au întrebat pe Domnul.
\par 15 Și a făcut Iosua pace cu ei și a încheiat cu ei legământ, ca să nu fie uciși, iar căpeteniile obștii s-au legat față de ei cu jurământ.
\par 16 La trei zile însă, după ce au încheiat legământ cu dânșii, au auzit că sunt aproape de ei și că trăiesc în ținuturile cuvenite lor,
\par 17 Căci fiii lui Israel, plecând la drum, a treia zi au ajuns la cetățile lor; iar cetățile lor erau Ghibeonul, Chefira, Beerot și Chiriat-Iearim.
\par 18 Iar Iosua și fiii lui Israel nu i-au ucis, pentru că toate căpeteniile obștii li se juraseră pe Domnul Dumnezeul lui Israel. De aceea toată obștea lui Israel a început a cârti împotriva căpeteniilor.
\par 19 Și toate căpeteniile au zis către întreaga obște: "Noi ne-am jurat lor pe Domnul Dumnezeul lui Israel și de aceea nu ne putem atinge de ei.
\par 20 Dar iată ce le vom face: să-i robim și să-i păstrăm în viață, ca să nu ne ajungă mânia pentru jurământul cu care ne-am jurat lor".
\par 21 Și le-au mai zis căpeteniile: "Lăsați-i să trăiască, dar să taie lemne și să care apă la toată obștea". Și toată obștea a făcut cum au zis căpeteniile.
\par 22 Și i-a chemat Iosua și le-a zis: "Pentru ce m-ați amăgit, spunând: Suntem tare departe de tine; voi însă sunteți dintre cei ce locuiesc în ținuturile cuvenite nouă.
\par 23 De aceea blestemați să fiți! Să nu încetați de a fi în robie, ca tăietori de lemne și cărători de apă pentru casa Dumnezeului meu".
\par 24 Iar ei au răspuns lui Iosua și au zis: "Ni s-a vestit nouă câte a poruncit Domnul Dumnezeul tău lui Moise, slugii Sale, ca să vă dea toată țara aceasta, iar pe noi și pe toți locuitorii pământului acestuia să ne piardă de la fața voastră; și, înfricoșându-ne foarte tare de voi pentru viața noastră, am făcut fapta aceasta.
\par 25 Și acum iată suntem sub mâna voastră; faceți cu noi cum veți socoti și cum vi se pare mai drept".
\par 26 Și a făcut cu ei așa: i-a izbăvit Iosua în ziua aceea din mâinile fiilor lui Israel și nu i-a ucis.
\par 27 Dar din ziua aceea i-a pus Iosua tăietori de lemne și cărători de apă pentru obște și pentru jertfelnicul Domnului. De aceea locuitorii Ghibeonului s-au făcut tăietori de lemne și cărători de apă pentru jertfelnicul lui Dumnezeu până în ziua de astăzi, la locul pe care avea să-l aleagă Domnul.

\chapter{10}

\par 1 Auzind însă Adoni-Țedec, regele Ierusalimului, că Iosua a luat cetatea Ai și a dat-o blestemului și a făcut cu Ai și cu regele său cum făcuse cu Ierihonul și cu regele lui și că locuitorii Ghibeonului de bunăvoie s-au supus lui Iosua și lui Israel și au rămas între ei,
\par 2 S-a spăimântat foarte tare, pentru că cetatea Ghibeonului era cetate mare, ca una ce era dintre cetățile domnești, mai mare decât Ai, și toți locuitorii ei erau oameni viteji.
\par 3 De aceea Adoni-Țedec, regele Ierusalimului, a trimis la Hoham, regele Hebronului, la Piream, regele Iarmutului, la Iafia, regele Lachișului, și la Debir, regele Eglonului, zicând:
\par 4 "Veniți la mine și-mi ajutați să bat Ghibeonul, pentru că s-a supus lui Iosua și fiilor lui Israel".
\par 5 Și acești cinci regi ai Amoreilor: regele Ierusalimului, regele Hebronului, regele Iarmutului, regele Lachișului și regele Eglonului, s-au suit cu tot poporul lor și au tăbărât asupra Ghibeonului și l-au împresurat.
\par 6 Atunci locuitorii Ghibeonului au trimis la Iosua, în tabăra Israeliților de la Ghilgal, zicând: "Să nu-ți iei mâna de deasupra robilor tăi! Vino la noi repede de ne dă ajutor și ne izbăvește, că s-au adunat împotriva noastră toți regii Amoreilor care trăiesc în munți".
\par 7 Și s-a suit Iosua din Ghilgal, el însuși și împreună cu dânsul tot poporul bun de război și toți bărbații viteji.
\par 8 Iar Domnul a zis către Iosua: "Nu te teme de ei, că i-am dat în mâinile tale; nimeni dintre ei nu va putea sta împotriva voastră".
\par 9 Și a năvălit Iosua fără de veste asupra lor, după ce mersese toată noaptea venind din Ghilgal.
\par 10 Și Domnul i-a făcut să se sperie la vederea fiilor lui Israel și i-a învins pe ei Domnul cu înfrângere grea la Ghibeon, și i-a urmărit în calea spre înălțimile Bet-Horon și i-a bătut până la Azeca și până la Macheda.
\par 11 Pe când însă fugeau ei de fiii lui Israel, pe povârnișul Bet-Horon, Domnul a aruncat din cer asupra lor grindină mare, până la Azeca, și cei ce au murit de grindină au fost mai mulți decât cei uciși de fiii lui Israel cu sabia în luptă.
\par 12 În ziua aceea în care Dumnezeu a dat pe Amorei în mâinile lui Israel și când i-a bătut la Ghibeon și au fost zdrobiți înaintea feței fiilor lui Israel, a strigat Iosua către Domnul și a zis înaintea Israeliților: "Stai, soare, deasupra Ghibeonului, și tu, lună, oprește-te deasupra văii Aialon!"
\par 13 Și s-a oprit soarele și luna a stat până ce Dumnezeu a făcut izbândă asupra vrăjmașilor lor. Oare nu de aceea se scrie în Cartea Dreptului: "Soarele a stat în mijlocul cerului și nu s-a grăbit către asfințit aproape toată ziua".
\par 14 Și n-a mai fost nici înainte, nici după aceea, o astfel de zi în care Domnul să asculte așa glasul omului; că Domnul lupta pentru Israel.
\par 15 Apoi Iosua s-a întors cu tot Israelul în tabără, la Ghilgal.
\par 16 Iar cei cinci regi au fugit și s-au ascuns în peștera din Macheda.
\par 17 Când însă i s-a spus lui Iosua că: "Cei cinci regi au fost găsiți ascunși în peșteră, la Macheda", Iosua a zis:
\par 18 "Răsturnați pietre mari pe gura peșterii și puneți la ușa ei oameni ca să-i păzească;
\par 19 Iar voi nu vă opriți acolo, ci goniți din urmă pe vrăjmașii voștri și nimiciți partea din urmă a oștirii lor; să nu-i lăsați să scape în cetățile lor, căci Domnul Dumnezeul vostru i-a dat în mâinile voastre!"
\par 20 Și după ce Iosua și fiii lui Israel i-au zdrobit cu desăvârșire într-o bătălie foarte mare și cei ce au scăpat au fugit în cetăți întărite,
\par 21 S-a întors tot poporul cu izbândă la Iosua, în tabără la Macheda, și n-a rostit nimeni nici un cuvânt împotriva fiilor lui Israel.
\par 22 Atunci Iosua a zis: "Deschideți peștera și scoateți la mine pe cei cinci regi din peșteră!"
\par 23 Și s-a făcut așa: au scos la dânsul pe cei cinci regi din peșteră: pe, regele Ierusalimului, pe regele Hebronului, pe regele Iarmutului, pe regele Lachișului și pe regele Eglonului.
\par 24 Și după ce au scos pe regii aceștia la Iosua, Iosua a chemat pe toți Israeliții și a zis către căpeteniile oștenilor care fuseseră cu el: "Apropiați-vă și puneți-vă picioarele pe grumajii regilor acestora!" Și ei s-au apropiat și și-au pus picioarele pe grumajii lor.
\par 25 Atunci Iosua le-a zis: "Nu vă temeți, nici nu vă spăimântați, ci îmbărbătați-vă și vă întăriți, că așa va face Domnul cu toți vrăjmașii voștri cu care vă veți lupta".
\par 26 După aceea i-a lovit Iosua și i-a ucis și i-a spânzurat pe cinci spânzurători și au stat spânzurați până seara.
\par 27 Iar la asfințitul soarelui a poruncit Iosua de i-au pogorât din spânzurători și i-au aruncat în peștera în care fuseseră ascunși și au prăvălit pietre mari pe gura peșterii și stau acolo până în ziua de azi.
\par 28 Tot în ziua aceea a cuprins Iosua Macheda și a lovit-o cu sabia, pe ea și pe regele ei, și a nimicit pe locuitorii ei și toată suflarea care se afla în ea; pe nimeni n-a cruțat ca să nu moară sau să fugă și a făcut cu regele din Macheda tot așa cum făcuse cu regele Ierihonului.
\par 29 Apoi s-a dus Iosua împreună cu toți Israeliții de la Macheda la Libna și a împresurat-o.
\par 30 Și a dat-o Domnul și pe aceasta în mâinile lui Israel de a luat-o pe ea și pe regele ei și Iosua a trecut-o prin ascuțișul sabiei pe ea și toată suflarea din ea; pe nimeni n-a lăsat în ea, care să nu moară sau să fugă și a făcut cu regele ei cum făcuse cu regele Ierihonului.
\par 31 De la Libna s-a dus Iosua împreună cu toți Israeliții la Lachiș și l-a împresurat, începând războiul cu ei.
\par 32 Și a dat Domnul Lachișul în mâinile lui Israel și l-a luat a doua zi, l-a trecut prin sabie pe el și toată suflarea din el și l-a pierdut, cum făcuse și cu Libna.
\par 33 Atunci s-a ridicat în ajutorul Lachișului Horam, regele din Ghezer. Dar Iosua l-a lovit și pe el și pe poporul lui cu sabia, încât n-a lăsat pe nimeni dintre ai lui care să nu moară sau să fugă.
\par 34 Din Lachiș s-a dus Iosua împreună cu toți Israeliții la Eglon și l-a împresurat, începând războiul împotriva lui.
\par 35 Și Domnul l-a dat în mâinile lui Israel. Și l-a luat în aceeași zi și l-a lovit cu sabia pe el și toată suflarea ce era în el și l-a dat în ziua aceea blestemului, cum făcuse și cu Lachișul.
\par 36 De la Eglon, Iosua împreună cu toți Israeliții s-au dus la Hebron, l-au înconjurat
\par 37 Și l-au lovit cu ascuțișul sabiei pe el și pe regele lui și toate cetățile lui și toată suflarea câtă era în acesta, cum făcuse cu Eglonul; și l-a dat blestemului pe el și toată suflarea ce era în el.
\par 38 După aceea s-a întors Iosua împreună cu toți Israeliții asupra Debirului și l-a împresurat.
\par 39 Și l-a luat pe el, pe regele lui și toate satele lui. Și le-au lovit cu ascuțișul sabiei, le-au nimicit pe ele și toată suflarea ce era în ele și n-a lăsat pe nimeni să scape; cum făcuse cu Hebronul și cu regele lui și cum a făcut cu Libna și regele ei, așa a făcut și cu Debirul și cu regele lui.
\par 40 Apoi a lovit Iosua tot ținutul muntos, Neghebul, câmpia și ținutul mării și pe regii lor și n-a lăsat să scape nimeni, ci a omorât toată suflarea, cum poruncise Domnul Dumnezeul lui Israel.
\par 41 Și i-a bătut Iosua de la Cadeș-Barnea până la Gaza și tot ținutul Goșen până la Ghibeon.
\par 42 Și pe toți regii acestora și ținuturile lor le-a luat Iosua deodată, căci Domnul Dumnezeul lui Israel lupta pentru Israel.
\par 43 După aceea Iosua împreună cu toți Israeliții s-au întors în tabăra de la Ghilgal.

\chapter{11}

\par 1 Cum a auzit de aceasta, Iabin, regele Hațorului, a trimis la Iobab, regele Madonului, la regele Șimronului și la regele Acșafului;
\par 2 La regii cei de la miazănoapte; din munți, de pe podiș, de la miazăzi de Chinerot, în câmpie și în ținutul Dor, la apus;
\par 3 La Canaaneii din răsărit, de pe malul mării; la Amorei, la Hetei, la Ferezei, la Iebuseii din munți și la Heveii de sub Hermon, în pământul Mițpa.
\par 4 Și au ieșit aceștia împreună cu regii lor, popor mult la număr, ca nisipul de la marginea mării, și cai și care de război foarte multe.
\par 5 Și s-au adunat toți regii aceștia și au tăbărât împreună la apele Merom, ca să se lupte cu Israel.
\par 6 Domnul însă a zis: "Nu te teme de fața lor, căci mâine pe vremea aceasta îi voi da pe toți aceia fiilor lui Israel ca să-i ucidă; cailor lor să le tai vinele picioarelor, iar carele să le arzi cu foc".
\par 7 Atunci Iosua împreună cu tot poporul bun de luptă a ieșit fără de veste înaintea lor, la apele Merom și a năvălit în munți asupra lor.
\par 8 Și i-a dat pe ei Domnul în mâinile lui Israel și i-a lovit și i-a urmărit până la Sidonul cel Mare, până la Misrefot-Maim și până în valea Mițpa, la răsărit, și i-a ucis până n-a scăpat nimeni.
\par 9 Și a făcut Iosua cu ei cum îi zisese Domnul: cailor lor le-a tăiat vinele picioarelor și carele lor le-a ars cu foc.
\par 10 Tot atunci întorcându-se, Iosua a luat Hațorul și pe regele lui l-a omorât cu sabia. Hațorul fusese până atunci capul tuturor regatelor acestora.
\par 11 Și a ucis toată suflarea din acesta cu sabia, dând toate pieirii; și n-a rămas nici un suflet, iar Hațorul l-a ars cu foc.
\par 12 Astfel a luat Iosua toate cetățile regatelor acestora și pe toți regii lor i-a ucis cu sabia, dându-i pieirii, cum poruncise Moise, sluga Domnului.
\par 13 Iar cetățile cele întărite nu le-a ars Israel, afară de Hațor, pe care l-a ars Iosua.
\par 14 Toată prada cetăților acestora și toate vitele le-au luat fiii lui Israel pentru ei, iar pe oameni i-au ucis cu sabia, i-au dat blestemului și n-au lăsat din ei nici un suflet.
\par 15 Precum poruncise Domnul lui Moise, sluga Sa, și cum poruncise și Moise lui Iosua, așa a făcut Iosua: nu s-a abătut de la nimic, nici de la un cuvânt din câte poruncise Domnul lui Moise.
\par 16 Și așa a luat Iosua toată țara aceea de sus și tot ținutul Negheb, tot ținutul Goșen și ținuturile de jos, șesul și muntele lui Israel și locurile joase de pe lângă munte,
\par 17 De la Muntele Pele, care se întinde spre Seir, până la Baal-Gad din valea Libanului, la poalele Hermonului; a luat pe toți regii lor și i-a lovit și i-a ucis.
\par 18 Dar Iosua a purtat multă vreme război cu toți regii aceștia.
\par 19 Și n-a fost cetate pe care să n-o fi luat cu fiii lui Israel; toate le-au luat cu război, afară de Ghibeon în care locuiau Heveii;
\par 20 Căci așa a fost de la Domnul, ca să-și învârtoșeze inima lor și să întâmpine pe Israel cu război, ca să fie dați pieirii și ca să nu găsească milă, ci să fie nimiciți, cum poruncise Domnul lui Moise.
\par 21 Apoi a venit Iosua în vremea aceea și a lovit pe toți Anachimii de la munte, din Hebron, din Debir, din Anab, din toți munții lui Iuda și din toți munții lui Israel și i-a dat Iosua nimicirii împreună cu cetățile lor.
\par 22 Și n-a rămas niciunul din Anachimi în pământul fiilor lui Israel, ci numai în Gaza, în Gat și în Așdod au rămas din ei.
\par 23 Așa a luat Iosua tot pământul, cum poruncise Domnul lui Moise și l-a dat Iosua de moștenire lui Israel, împărțindu-l între semințiile lor. Și s-a liniștit pământul de război.

\chapter{12}

\par 1 Iată regii pe care i-au bătut fiii lui Israel și al căror pământ l-au luat tot de moștenire dincolo de Iordan, spre răsăritul soarelui, de la râul Arnon până la muntele Hermon și tot șesul dinspre răsărit:
\par 2 Sihon, regele Amoreilor, care locuia în Heșbon și domnea de la Aroerul cel de pe malul râului Arnon, de la mijlocul râului și peste jumătate din Galaad, până la râul Iaboc, care este hotarul Amoniților,
\par 3 Peste șes până lângă marea Chineretului, spre răsărit și până la marea șesului, până la Marea Sărată, spre răsărit, pe calea către Bet-Ieșimot, iar spre miazăzi peste locurile ce se întindeau pe la poalele muntelui Fazga;
\par 4 Vecinul său Og, regele Vasanului, cel din urmă din Refaim, care locuia în Aștarot și Edrea
\par 5 Și care stăpânea muntele Hermon și Salca și tot Vasanul, până la hotarul Gheșurului și al Maacului și jumătate din Galaad, până la hotarul lui Sihon, regele Heșbonului.
\par 6 Pe aceștia Moise, sluga Domnului, și fiii lui Israel i-au ucis și a dat Moise, sluga Domnului, pământul lor de moștenire semințiilor lui Ruben și Gad și la jumătate din seminția lui Manase.
\par 7 Iată acum și regii din țara Amoreilor, pe care i-a bătut Iosua și fiii lui Israel dincoace de Iordan, spre apus de la Baal-Gad, din valea Libanului, până la Pele, muntele care se întinde spre Seir; și pământul l-a dat Iosua semințiilor lui Israel de moștenire, după cum le-au căzut sorții,
\par 8 La munte sau la loc șes, la câmpie sau la locurile de sub munți, în pustiu și la miazăzi și care fusese al Heteilor, Amoreilor, Canaaneilor, Ferezeilor, Heveilor și Iebuseilor:
\par 9 Un rege al Ierihonului, un rege al cetății Ai, care e aproape de Betel;
\par 10 Un rege al Ierusalimului, un rege al Hebronului;
\par 11 Un rege al Iarmutului, un rege al Lachișului;
\par 12 Un rege al Eglonului, un rege al Ghezerului;
\par 13 Un rege al Debirului, un rege al Ghederului;
\par 14 Un rege al Hormei, un rege al Aradului;
\par 15 Un rege al Libnei, un rege al Adulamului;
\par 16 Un rege al Machedei; un rege al Betelului;
\par 17 Un rege al Tapuahului, un rege al Heferului;
\par 18 Un rege al Afecului, un rege al Șaronului;
\par 19 Un rege al Madonului, un rege al Hațorului;
\par 20 Un rege al Șimron-Meronului, un rege al Acșafului;
\par 21 Un rege al Taanacului, un rege al Meghidonului;
\par 22 Un rege al Chedeșului, un rege al Iocneamului de lângă Carmel;
\par 23 Un rege al Dorului de lângă Nafat-Dor, un rege al Goimului din Ghilgal;
\par 24 Un rege al Tirței. De toți treizeci și unu de regi.

\chapter{13}

\par 1 Fiind Iosua bătrân și înaintat în zile, a zis Domnul către dânsul: "Iată tu ai îmbătrânit și ești în vârstă înaintată, dar pământ de luat în moștenire a mai rămas încă mult.
\par 2 Pământul care a mai rămas de luat în moștenire este acesta: ținuturile Filistenilor și toată țara Gheșurului și a Canaanului.
\par 3 De la Sicor, care este la răsărit de Egipt, până la hotarele Ecronului, la miazănoapte, se socotesc cele cinci căpetenii filistene: Gaza, Așdod, Ascalonul, Gat și Ecronul.
\par 4 Apoi țara Heveilor: tot pământul Canaan și Maara Sidonienilor, de la Teman până la Afec și până la hotarele Amoreilor.
\par 5 De asemenea ținutul filistean Ghebla și tot Libanul, spre răsăritul soarelui, de la Baal-Gad, de la poalele muntelui Hermon până la intrarea Hamatului.
\par 6 Pe toți locuitorii muntelui de la Liban până la Misrefot-Maim, pe toți Sidonienii să-i pierzi de la fața fiilor lui Israel  pământul lor să-l împarți lui Israel prin sorti, cum ti-am poruncit.
\par 7 Așadar la cele nouă seminții și la jumătate din seminția lui Manase, împarte-le moștenire prin sorți pământul acesta: de la Iordan până la marea cea mare de la apus să-l dai lor și marea cea mare să fie hotar.
\par 8 Căci celelalte două seminții: a lui Ruben și Gad și jumătate din seminția lui Manase au primit partea de la Moise, dincolo de Iordan, spre răsăritul soarelui, cum le-a dat Moise sluga Domnului,
\par 9 De la Aroer, care e pe malul râului Arnon, cetatea cea din mijlocul văii și toată câmpia Medeba până la Dibon,
\par 10 Precum și toate cetățile lui Sihon, regele Amoreilor, care a domnit în Heșbon, până la hotarele fiilor lui Amon,
\par 11 Și Galaadul, ținutul Gheșur și al Maacatienilor, tot muntele Hermon și tot Vasanul până la Salca;
\par 12 Tot regatul lui Og al Vasanului, care a domnit în Aștarot și în Edrea. Acesta mai rămăsese din Refaimi, pe care Moise i-a bătut și i-a alungat".
\par 13 Dar fiii lui Israel n-au vrut să piardă pe Gheșureni și pe Maacatieni și până în ziua de astăzi locuiește regele din Gheșur și al Maacatienilor în mijlocul lui Israel.
\par 14 Numai seminției lui Levi nu i-a dat Iosua moștenire, căci jertfele și prinoasele Domnului Dumnezeului lui Israel sunt partea ei, cum i-a zis Domnul.
\par 15 Iată împărțirea pe care a făcut-o Moise fiilor lui Israel, după semințiile lor, în șesurile Moabului, dincolo de Iordan în fața Ierihonului: Seminției fiilor lui Ruben, după familiile ei, Moise i-a dat parte:
\par 16 Hotarele ei cuprindeau cetatea Aroer, care se află pe malul râului Arnon, la mijlocul cursului acestui râu și tot șesul de lângă Medeba;
\par 17 Heșbonul și toate cetățile lui cele din șes; Dibonul, Bamot-Baal și Bet-Baal-Meon;
\par 18 Iahța, Chedemot și Mefaat;
\par 19 Chiriataim, Sibma și Țeret-Hașahar, în muntele Emec;
\par 20 Bet-Peor, locurile de la poalele muntelui Fazga și Bet-Ieșimot;
\par 21 Toate cetățile din șes și toată împărăția lui Sihon, regele Amoreilor, care a domnit la Heșbon și pe care l-a ucis Moise, ca și pe căpeteniile lui Madiam, pe Evi, Rechem, Țur, Hur și Reba, căpeteniile lui Sihon, care locuiau în țara aceea;
\par 22 De asemenea și pe Valaam, fiul lui Beor, vrăjitorul, l-au ucis fiii lui Israel cu sabia, în numărul celor uciși de ei.
\par 23 Hotarul fiilor lui Ruben era Iordanul. Aceasta e partea fiilor lui Ruben după familiile lor, după cetățile și satele lor.
\par 24 De asemenea a dat Moise parte seminției lui Gad, fiilor lui Gad, după familiile lor.
\par 25 În hotarele lor se cuprindea Iezerul și toate cetățile Galaadului și jumătate din țara fiilor lui Amon,
\par 26 Până la Aroer, care e în fața cetății Raba și țara de la Heșbon până la Ramat-Mițpa și Betonim și de la Mahanaim până la hotarele Debirului;
\par 27 În vale i-a dat Bet-Haram, Bet-Nimra, Sucot și Țafon, rămășița regatului lui Sihon, regele Heșbonului. Hotarul lui era Iordanul până la marea Chineret, întinzându-se de la Iordan spre răsărit.
\par 28 Aceasta era partea fiilor lui Gad după familiile lor, cetățile și satele lor.
\par 29 Și a mai dat Moise parte și la jumătate din seminția lui Manase, adică la jumătate din seminția fiilor lui Manase, după familiile lor.
\par 30 În hotarele lor se cuprindea tot Vasanul de la Mahanaim, toată împărăția lui Og, regele Vasanului, și toate sălașurile Iairului celui din Vasan, șaizeci de cetăți.
\par 31 Iar jumătate din Galaad cu Aștarotul și Edrea, orașele nepotului lui Og al Vasanului, au fost date fiilor lui Machir, fiul lui Manase, la jumătate din fiii lui Machir după familiile lor.
\par 32 Iată ce a dat Moise ca parte de moștenire în șesul Moabului, peste Iordan, în fața Ierihonului spre răsărit.
\par 33 Dar seminției lui Levi, Moise nu i-a dat parte, că Însuși Domnul Dumnezeul lui Israel este partea lor, cum le-a grăit El.

\chapter{14}

\par 1 Iată ce moștenire au primit fiii lui Israel în țara Canaan, pe care le-au împărțit-o prin sorți preotul Eleazar și Iosua, fiul lui Navi, și căpeteniile de familii ale semințiilor fiilor lui Israel.
\par 2 Moștenirea aceasta au împărțit-o prin sorți cum poruncise Domnul prin Moise, la cele nouă seminții și la jumătate din seminția lui Manase;
\par 3 Căci la două seminții și la jumătate din seminția lui Manase le dăduse Moise parte peste Iordan. Iar leviților nu le-a dat moștenire între ei;
\par 4 Căci din fiii lui Iosif se ridicaseră două seminții: a lui Manase și a lui Efraim; de aceea nu s-a dat leviților parte de pământ, ci numai cetăți pentru locuit cu împrejurimile lor, pentru vitele și pentru turmele lor.
\par 5 Cum poruncise Domnul prin Moise, așa au și făcut fiii lui Israel, când au împărțit țara.
\par 6 Atunci au venit fiii lui Iuda la Iosua în Ghilgal și a zis către el Caleb, fiul lui Iefone Chenezeul: "Tu știi ce a zis Domnul către Moise, omul lui Dumnezeu, pentru mine și pentru tine, la Cadeș-Barnea.
\par 7 Eu eram de patruzeci de ani, când Moise, sluga lui Dumnezeu, m-a trimis din Cadeș-Barnea să iscodesc țara și eu i-am adus răspuns după dorința lui.
\par 8 Frații mei, care fuseseră cu mine, au umplut de spaimă inima poporului, iar eu am urmat hotărât Domnului Dumnezeului meu.
\par 9 Și s-a jurat Moise în ziua aceea și a zis: "Pământul pe unde a călcat piciorul tău va fi moștenirea ta și a copiilor tăi pe veci, că tu ai urmat hotărât Domnului Dumnezeului nostru".
\par 10 Și iată Domnul m-a ținut viu cum a zis. Au trecut acum patruzeci și cinei de ani de când a spus Domnul lui Moise cuvântul acesta și Israel umbla prin pustiu; și iată acum am optzeci și cinci de ani
\par 11 Și sunt încă tare, ca și atunci când m-a trimis Moise; câtă putere aveam atunci, tot atâta am și acum, ca să ies și să intru în luptă.
\par 12 Așadar îți cer muntele acesta, de care a vorbit Domnul în ziua aceea; căci tu ai auzit în ziua aceea cuvântul acesta acolo. Acum acolo sunt fiii lui Enac, care au cetăți mari și tari, dar de va fi Domnul cu mine, îi voi izgoni, cum mi-a zis Domnul".
\par 13 Și l-a binecuvântat Iosua și a dat Hebronul moștenire lui Caleb, fiul lui Iefone Chenezeul.
\par 14 De aceea a ajuns Hebronul moșia lui Caleb, fiul lui Iefone, până în ziua de azi, pentru că el a urmat întocmai porunca Domnului Dumnezeului lui Israel.
\par 15 Mai înainte Hebronul se numea Chiriat-Arba, capitala fiilor lui Enac. După aceea s-a liniștit țara de război.

\chapter{15}

\par 1 Moșia căzută la sorți pentru seminția fiilor lui Iuda, după familiile lor, se întindea de la hotarele lui Edom din miazăzi și de la pustiul Sin până la Cadeș, spre răsărit.
\par 2 Astfel hotarul lor de miazăzi pornește de la Marea Sărată, de unde pleacă un golf al ei spre miazăzi,
\par 3 Merge spre înălțimea Acravimului, trece prin pustiul Sin și, ridicându-se dinspre miazăzi către Cadeș-Barnea, merge pe la Hețron și, ajungând la Adar, trece pe partea dinspre apus a Cadeșului și se întoarce spre Carcaa;
\par 4 Trece apoi prin Ațmon și urmează înainte până la râul Egiptului și apoi capătul acestui hotar atinge marea. Acesta va fi hotarul vostru de miazăzi.
\par 5 Hotarul de răsărit e toată Marea Sărată până la gurile Iordanului. Iar apoi hotarul de miazănoapte pleacă din golful mării de la gurile Iordanului;
\par 6 De aici se ridică spre Bet-Hogla. Trece pe la miazănoapte de Bet-Araba și merge în sus până la piatra lui Bohan, fiul lui Ruben;
\par 7 Apoi hotarul se urcă spre Debir, din valea Acor și se îndreaptă spre miazănoapte către Ghilgal, care se află în fața Adumimului, pe partea de miazăzi a râului; apoi trece pe la apele En-Șemeș și se prelungește până la En-Roghel.
\par 8 De aici hotarul merge în sus spre valea Ben-Hinom, pe partea de miazăzi de Iebus, care este Ierusalimul; apoi hotarul se ridică spre vârful muntelui, care este spre apus, în fața văii Hinom, la marginea văii Refaim, la miazănoapte.
\par 9 Din vârful muntelui, hotarul se îndreaptă spre izvorul apelor Neftoah și merge spre cetățile din muntele Efron; apoi hotarul cotește spre Baala, care e Chiriat-Iearimul;
\par 10 După aceea hotarul se întoarce de la Baala spre mare și merge spre muntele Seir, trece pe partea de miazănoapte a muntelui Iearim, care e Chesalonul și, pogorându-se către Bet-Șemeș, trece prin Timna;
\par 11 De aici hotarul merge pe partea de miazănoapte a Ecronului și se întoarce spre Șicron, trece prin muntele Baala și ajunge până la Iabneel și apoi se termină hotarul la mare. Hotarul de la apus îl formează Marea cea Mare.
\par 12 Acesta este hotarul pământului fiilor lui Iuda, după familiile lor, din toate părțile.
\par 13 Lui Caleb, fiul lui Iefone, i-a dat Iosua parte intre fiii lui Iuda, cum poruncise Domnul lui Iosua și i-a dat Iosua Chiriat-Arba a tatălui lui Enac, care este Hebronul.
\par 14 Însă Caleb, fiul lui Iefone, a alungat de acolo pe cei trei fii ai lui Enac: pe Șeșai, pe Ahiman și pe Talmai, copiii lui Enac.
\par 15 De aici Caleb a pornit asupra locuitorilor Debirului; numele Debirului era mai înainte Chiriat-Sefer.
\par 16 Și a zis Caleb: "Cel ce va bate Chiriat-Seferul și-l va lua, aceluia îi voi da pe Acsa, fiica mea, de femeie".
\par 17 Și l-a luat Otniel cel tânăr, fiul lui Chenaz, fratele lui Caleb și i-a dat Caleb de femeie pe Acsa, fiica sa.
\par 18 Dar când a trebuit să plece, a fost învățată să ceară de la tatăl său o țarină. Și când a descălecat ea de pe asin, Caleb i-a zis: "Ce vrei?"
\par 19 Iar ea a zis: "Dă-mi binecuvântare. Tu mi-ai dat pământul de la miazăzi; dă-mi și izvoarele de apă!" Și i-a dat izvoarele cele de sus și izvoarele cele de jos.
\par 20 Aceasta este moștenirea fiilor lui Iuda, după familiile lor.
\par 21 Cetățile care se aflau în partea de miazăzi, la marginea seminției fiilor lui Iuda, spre hotarul Edomului, erau: Cabțeel, Eder și Iagur;
\par 22 China, Dimona și Adada;
\par 23 Chedeș, Hațor și Itnan;
\par 24 Zif, Telem și Bealot;
\par 25 Hațor-Hadata, Cheriot-Hețron, adică Hațor;
\par 26 Amam, Șerna și Molada;
\par 27 Hațar-Gada, Heșmon și Bet-Palet;
\par 28 Hațar-Șual, Beerșeba și Biziotia, cu împrejurimile și satele lor;
\par 29 Baala, Iim și Ațem;
\par 30 Eltolad, Chesil și Horma;
\par 31 Țiclag, Madmana și Sansana;
\par 32 Lebaot, Șilhim, Ain și Rimon; de toate douăzeci și nouă de cetăți cu satele lor.
\par 33 Iar la șes erau: Eștaol, Țora, Așna și Gatnam;
\par 34 Zanuah, En-Ganim, Tapuah și Enam;
\par 35 Iarmut, Adulam, Memvra, Soco și Azeca;
\par 36 Șaaraim, Aditaim, Ghedera și Ghederotaim; paisprezece cetăți cu satele lor.
\par 37 Țenan, Hadașa și Migdal-Gad;
\par 38 Dilean, Mițpe și Iocteel;
\par 39 Lachiș, Boțcat și Eglon;
\par 40 Cabon, Lahmas și Chitliș;
\par 41 Ghederot, Bet-Dagon, Naama și Macheda; șaisprezece cetăți cu satele lor.
\par 42 Libna, Eter și Așan;
\par 43 Iftah, Așna și Nețib;
\par 44 Cheila, Aczib, Mareșa și Edom; nouă cetăți cu satele lor.
\par 45 Ecron cu cetățile care țineau de el și cu satele lui;
\par 46 Și de la Ecron spre mare tot ce se află împrejurul Așdodului cu satele lui;
\par 47 Așdodul și cetățile care țineau de el și satele lui; Gaza cu cetățile care țineau de ea și satele ei până ia râul Egiptului și până la Marea cea Mare, care este hotar.
\par 48 Iar în munți erau: Șamir, Iatir și Soco;
\par 49 Dana, Chiriat-Sana, zis și Debir;
\par 50 Anab, Eștemo și Anim;
\par 51 Goșen, Holon și Ghilo: unsprezece cetăți cu satele lor.
\par 52 Anab, Duma și Eșean;
\par 53 Ianum, Bet-Tapuah și Afeca;
\par 54 Humta, Chiriat-Arba, zisă și Hebronul și Țior; nouă cetăți cu satele lor.
\par 55 Maon, Carmel, Zif și Iuta;
\par 56 Izreel, Iocdeam și Zanuah;
\par 57 Cain, Ghibeea și Timna: zece cetăți cu satele lor.
\par 58 Halhul, Bet-Țur și Ghedor;
\par 59 Maarat, Bet-Anot și Eltecon: șase cetăți cu satele lor. Tecoa, Efrata sau Betleemul, Peor, Etam, Culon, Tatam, Sores, Carem, Galem, Betir și Manah: unsprezece cetăți cu satele lor.
\par 60 Chiriat-Baal sau Chiriat-Iearim și Harabah; două cetăți cu satele lor și cu împrejurimile.
\par 61 În pustiu erau: Bet-Araba, Midin și Secaca;
\par 62 Nibșan, Ir-Melah și En-Ghedi: șase cetăți cu satele lor.
\par 63 Dar pe Iebusei, locuitorii Ierusalimului, nu i-au putut alunga fiii lui Iuda și de aceea Iebuseii trăiesc cu fiii lui Iuda în Ierusalim până în ziua de astăzi.

\chapter{16}

\par 1 Apoi au căzut sorții fiilor lui Iosif; hotarul începe la Iordan, lângă Ierihon, merge către apele Ierihonului, având la răsărit pustiul care se întinde de la Ierihon până la Betel
\par 2 Și granița merge spre Luz, trece hotarul Archienilor până la Atarot,
\par 3 Apoi se lasă spre mare, către hotarele lui Iaflet, până la hotarele Bet-Horonului de jos și până la Ghezer și se înfundă la mare.
\par 4 Această parte au primit-o fiii. lui Iosif: Manase și Efraim.
\par 5 Hotarele fiilor lui Efraim, după familiile lor, au fost acestea: hotarul moștenirii lor era la răsărit Atarot-Adar până la Bet-Horonul de Sus și Ghezer;
\par 6 Apoi înainta spre apus pe la miazănoapte de Micmetat, se întorcea la răsărit spre Taanat-Șilo și trecea pe la răsărit de Ianoah.
\par 7 De la Ianoah se pogora la Atarot și la Naarata, atingând Ierihonul și se prelungea până la Iordan.
\par 8 De la Tapuah mergea spre apus, către râulețul Cana, pentru a se sfârși la mare. Aceasta este partea fiilor lui Efraim, după familiile lor.
\par 9 Și fiilor lui Efraim li s-au mai dat cetăți și în partea fiilor lui Manase; toate acele cetăți erau cu satele lor.
\par 10 Dar Efraimiții n-au alungat pe Canaanei, care locuiau în Ghezer; de aceea Canaaneii au trăit între Efraimiți până în ziua de astăzi, plătindu-le bir. În cele din urmă a venit Faraon, regele Egiptului, și a luat cetatea și a ars-o cu foc, și pe Canaanei și pe Ferezei și pe locuitorii Ghezerului i-a ucis și a dat Faraon cetatea de zestre fiicei sale.

\chapter{17}

\par 1 De asemenea a căzut, la sorți, parte și seminției lui Manase, căci acesta era întâiul născut al lui Iosif. Lui Machir întâiul născut al lui Manase, care fusese viteaz în război, i-a căzut Galaadul și Vasanul.
\par 2 Dar le-au căzut, la sorți, părți și celorlalți fii ai lui Manase, după familiile lor: fiilor lui Abiezer, fiilor lui Helec, fiilor lui Asriel, fiilor lui Sichem, fiilor lui Hefer și fiilor lui Șemida. Aceștia sunt fiii lui Manase, fiul lui Iosif de parte bărbătească, după familiile lor.
\par 3 Iar Salfaad, fiul lui Hefer, fiul lui Galaad, fiul lui Machir, fiul lui Manase, n-a avut fii, ci numai fiice, ale căror nume sunt acestea: Mahla, Noa, Hogla, Milca și Tirța.
\par 4 Acestea au venit la preotul Eleazar și la Iosua, fiul lui Navi, și la căpetenii și au zis: "Domnul a poruncit lui Moise să ni se dea parte și nouă între frații noștri". Și li s-a dat parte, între frații tatălui lor.
\par 5 Și a căzut lui Manase zece părți, afară de țara Galaadului și a Vasanului, care erau peste Iordan,
\par 6 Pentru că fiicele fiilor lui Manase au primit părți printre fiii lui, iar rara Galaadului s-a cuvenit celorlalți fii ai lui Manase.
\par 7 Hotarul pământului fiilor lui Manase pleca de la Așer către Micmetat, care e în fața Sichemului; de aici hotarul merge spre dreapta, până la locuitorii En-Tapuahului.
\par 8 Pământul Tapuah a căzut lui Manase; iar cetatea Tapuah, de la hotarul lui Manase, a rămas fiilor lui Veniamin.
\par 9 De aici hotarul se pogoară spre râul Cana, pe malul de miazăzi al râului.
\par 10 Cetățile acestea sunt ale lui Efraim, deși sunt între fiii lui Manase. Hotarul lui Manase se sfârșește la mare, pe partea de miazănoapte a râului. Partea de miazăzi este a lui Efraim, iar cea de la miazănoapte este a lui Manase. Marea însă era hotarul lor la apus. La miazănoapte se mărginea cu Așer, iar spre răsărit cu Isahar.
\par 11 În Isahar și Așer sunt ale lui Manase: Bet-Șean cu locurile care se țin de el, Ibleam cu locurile care se țin de el, locuitorii din Dor și din locurile care lin de el, locuitorii din En-Dor și locurile care țin de el, locuitorii din Taanac și locurile care țin de el, locuitorii Meghidonului și locurile care țin de el, și a treia parte din Nafet cu satele lui.
\par 12 Fiii lui Manase n-au putut alunga pe locuitorii acestor orașe și Canaaneii au rămas să locuiască în țara lui.
\par 13 Când fiii lui Israel au ajuns puternici, atunci Canaaneii au fost făcuți birnici ai lor, dar de alungat nu i-au alungat.
\par 14 Fiii lui Iosif au zis către Iosua: "Pentru ce ne-ai dat o singură parte și un singur sorț, când noi suntem mulți, de vreme ce ne-a binecuvântat așa Domnul?"
\par 15 Și Iosua le-a răspuns: "Dacă sunteți mulți, duceți-vă în păduri și acolo, în țara Ferezeilor și Refaimilor, căutați-vă loc, dacă muntele Efraim vă e strâmt!"
\par 16 Iar fiii lui Iosif au zis: "Muntele nu va rămâne al nostru, pentru că toți Canaaneii care locuiesc în vale au căruțe de fier, atât cei din Bet-Șean și din locurile care țin de ea, cât și cei din șesul Izreel".
\par 17 Dar Iosua a zis către casa lui Iosif, lui Efraim și lui Manase: "Tu ești mult la număr și ai putere multă. Deci nu vei avea numai o parte.
\par 18 Muntele va fi al tău și pădurea. Tu îl vei curăți și va fi al tău până la capătul lui, căci tu vei izgoni pe Canaanei, deși ei au căruțe de fier; deși ei sunt tari, tu îi vei birui".

\chapter{18}

\par 1 Atunci s-a adunat toată obștea fiilor lui Israel la Șilo și au așezat acolo cortul adunării, căci țara fusese supusă de ei.
\par 2 Dar dintre fiii lui Israel mai rămăseseră șapte seminții care nu-și primiseră încă părțile lor.
\par 3 Atunci a zis Iosua către fiii lui Israel: "Oare mult veți rămâne voi nepăsători de a merge să luați moștenire țara pe care v-a dat-o Domnul Dumnezeul părinților voștri?
\par 4 "Dați câte trei oameni de seminție și eu îi voi trimite, iar ei, sculându-se, se vor duce prin țară și o vor descrie, cum trebuie să li se împartă în părți, și apoi vor veni la mine.
\par 5 Să o împartă în șapte părți; Iuda să rămână în partea sa spre miazăzi, iar casa lui Iosif să rămână în partea sa la miazănoapte.
\par 6 Voi însă întocmiți un plan al țării, împărțit în șapte părți și să mi-l aduceți și eu voi arunca sorți aici înaintea Domnului Dumnezeului nostru.
\par 7 Leviții însă n-au parte între voi, căci preoția Domnului este partea lor. Iar Gad, Ruben și jumătate din seminția lui Manase și-au primit partea lor peste Iordan, spre răsărit, pe care le-a dat-o Moise, robul Domnului".
\par 8 Atunci s-au sculat oamenii aceia și s-au dus. Dar Iosua a dat celor ce s-au dus să facă planul țării astfel de poruncă: "Duceți-vă și cutreierați țara, descrieți-o și vă întoarceți la mine, și eu vă voi arunca aici sorți înaintea Domnului, în Șilo".
\par 9 Și ei s-au dus, au cutreierat țara, au cercetat-o și au împărțit-o, după cetățile ei, în șapte părți, după un plan, și apoi au venit la Iosua în Șilo.
\par 10 Și le-a aruncat Iosua sorți în Șilo, înaintea Domnului, și a împărțit Iosua acolo țara fiilor lui Israel în părțile ce li se cuvenea.
\par 11 Întâiul sorț a căzut seminției fiilor lui Veniamin, după familiile lor. Partea lor după sorț se întindea între fiii lui Iuda și fiii lui Iosif.
\par 12 Hotarul lor de miazănoapte se începe de la Iordan și trece pe lângă Ierihon, pe partea de miazănoapte, și se urcă pe muntele de la apus și se sfârșește în pustiul Betaven.
\par 13 De acolo hotarul merge spre Luz, pe partea de miazăzi a Luzului sau a Betelului; apoi hotarul, coborând spre Atarot-Adar, merge către muntele care e spre miazăzi de Bet-Horonul de jos.
\par 14 Apoi hotarul se îndreaptă spre partea mării, pe la rniazăzi de muntele care vine la miazăzi de Bet-Horon sau Chiriat-Iearim, cetatea fiilor lui Iuda. Aceasta este latura de la apus.
\par 15 Iar în partea de miazăzi, de la Chiriat-Iearim, hotarul merge spre mare și ajunge până la izvorul apei Neftoah.
\par 16 Apoi hotarul se coboară spre capătul muntelui celui din fața văii Ben-Hinom, la miazănoapte, și se coboară pe valea Hinom spre partea de miazăzi a Iebusului și merge spre En-Roghel.
\par 17 Apoi se întoarce de la  miazănoapte și merge spre En-Șemeș și înaintează către Ghelilot, care e în fața înălțimii Adumium, și se coboară spre piatra lui Bohan, fiul lui Ruben.
\par 18 Apoi trece pe la miazănoapte de Harabah și se coboară la hotarul Arabei;
\par 19 De acolo hotarul trece pe lângă Bet-Hogla, pe la miazănoapte, și se sfârșește la, golful de miazănoapte al Mării Sărate, la capătul de miazăzi al Iordanului. Acesta e hotarul de miazăzi. Iar spre răsărit hotarul îl formează Iordanul.
\par 20 Aceasta este partea fiilor lui Veniamin cu hotarele ei din toate părțile, după familiile lor.
\par 21 Iar cetățile fiilor lui Veniamin, după familiile lor, sunt acestea: Ierihonul, Bet-Hogla și Emec-Chețiț;
\par 22 Bet-Harabah, Țemaraim și Betel;
\par 23 Avim, Para și Ofra;
\par 24 Chefar-Amonai, Ofni și Gheba: douăsprezece cetăți cu satele lor;
\par 25 Ghibeon, Rama și Beerot;
\par 26 Mițpa, Chefira și Moța;
\par 27 Rechem, Irpeel și Tareala;
\par 28 Țela, Elef și Iebus sau Ierusalimul, Ghibeat și Chiriat: paisprezece cetăți cu satele lor. Aceasta este partea lui Veniamin, după familiile lor.

\chapter{19}

\par 1 Al doilea sorț a căzut lui Simeon, seminției fiilor lui Simeon, după familiile lor; partea lor de moștenire a fost între hotarele părții fiilor lui Iuda.
\par 2 În partea lor se aflau: Beerșeba, Șeba și Molada;
\par 3 Hațar-Șual, Bala și Ațem;
\par 4 Eltolad, Betul și Horma;
\par 5 Țiclag, Bet-Marcabot și Hațar-Susa;
\par 6 Bet-Lebaot și Șaruhen: treisprezece cetăți cu satele lor.
\par 7 Ain, Rimon, Eter și Așan: patru cetăți cu satele lor.
\par 8 Și toate satele care se aflau împrejurul cetăților acestora chiar până la Baalat-Beer-Ramatul de miazăzi. Aceasta este partea de moștenire a fiilor lui Simeon, după familiile lor.
\par 9 Din moșia lui Iuda a fost despărțită partea fiilor lui Simeon. Deoarece partea fiilor lui Iuda era prea mare pentru ei, fiii lui Simeon au primit parte între hotarele moșiei lor.
\par 10 Al treilea sorț a căzut fiilor lui Zabulon, după familiile lor. Hotarul moștenirii lor se întindea până la Sarid,
\par 11 Se ridica la apus până la Mareala și atingea Dabeșetul și pârâul ce curge prin fața Iocneamului.
\par 12 De la Sarid apuca îndărăt și mergea în partea de răsărit, spre răsăritul soarelui până la hotarul ținutului Chislot-Tabor și de aici apuca spre Dabrat și se urca către Iafia;
\par 13 Apoi trecea spre răsărit la Ghet-Hefer, la Ita-Cațin și mergea spre Rimon, Metora și Nea;
\par 14 După aceea hotarul se întorcea de la miazănoapte către Hanaton și se termina în valea Iftah-El.
\par 15 Mai departe: Catat, Nahalal, Șimron, Idala și Betleem: douăsprezece cetăți cu satele lor.
\par 16 Aceasta e partea fiilor lui Zabulon, după familiile lor, și acestea sunt cetățile lor.
\par 17 Al patrulea sorț a căzut lui Isahar, fiilor lui Isahar, după familiile lor.
\par 18 În hotarul lor se cuprindeau: Izreel, Chesulot și Șunem;
\par 19 Hafaraim, Șion și Anaharat;
\par 20 Harabit, Chișion și Ebeț;
\par 21 Remet, En-Ganim, En-Hada și Bet-Pațeț;
\par 22 Și atingea Taborul, Șahațima și Bet-Șemeș și hotarul lor se termina la Iordan: șaisprezece cetăți cu satele lor.
\par 23 Aceasta e partea fiilor lui Isahar, după familiile lor și acestea sunt cetățile lor.
\par 24 Al cincilea sorț a căzut seminției fiilor lui Așer, după familiile lor.
\par 25 Hotarul lor trecea prin Helcat, Hali, Beten și Acșaf,
\par 26 Alamelec, Amead și Mișeal și hotarul lor atingea, spre apus Carmelul și Șihor-Libnat.
\par 27 După aceea, hotarul se întorcea spre răsăritul soarelui la Bet-Dagon și atingea ținutul Zabulon și valea Iftah-El pe la miazănoapte și  intra în hotarul Asatei la Bet-Emec și Neiel și mergea pe partea stângă a Cabulului.
\par 28 Mai departe urmează Abdon, Rehob, Hamon și Cana, până la Sidonul cel Mare.
\par 29 După aceea, hotarul se întorcea spre Rama până la orașul cel întărit al Tirului, apoi spre Hosa și se sfârșește la mare, în târgușorul Aczib.
\par 30 După aceea, urmează: Aco, Afec și Rehob: douăzeci și două de cetăți și satele lor.
\par 31 Aceasta este partea seminției fiilor lui Așer, după familiile lor, și acestea sunt cetățile și satele lor.
\par 32 Al șaselea sorț a căzut fiilor lui Neftali, seminției fiilor lui Neftali, după familiile lor.
\par 33 Hotarul lor mergea de la Helef și de la dumbrava cea din Țaananim către Adami-Necheb și Iabneel, până la Lacum și se sfârșea la Iordan.
\par 34 De aici se întorcea hotarul spre apus, către Asnot-Tabor și de acolo mergea spre Hucoc și atingea ținutul  Zabulonului în partea de miazăzi și ținutul Așer în partea de apus și ținutul lui Iuda la Iordan, spre răsăritul soarelui.
\par 35 Cetăți întărite erau: Țidim, Țer, Hamat, Racat și Chineret,
\par 36 Adama, Rama și Hațor;
\par 37 Chedeș, Edrea și En-Hațor;
\par 38 Ireon, Migdal-El, Horem, Bet-Anat și Bet-Șemeș: nouăsprezece cetăți cu satele lor.
\par 39 Aceasta "este partea seminției fiilor lui Neftali, după familiile lor, și acestea sunt cetățile și satele lor.
\par 40 Seminției fiilor lui Dan, după familiile lor, i-a căzut al șaptelea sorț.
\par 41 În hotarul moștenirii lor se cuprindeau: Țora, Eștaol și Ir-Șemeș;
\par 42 Șaalabin, Aialon și Itla;
\par 43 Elon, Timnata și Ecron;
\par 44 Elteche, Ghibeton și Baalat;
\par 45 Iehud, Bene-Berac și Gat-Rimon;
\par 46 Me-Iarcon și Haracon cu hotarele aproape de Iopi. S-a văzut însă că partea moștenirii fiilor lui Dan e mică pentru ei.
\par 47 Atunci s-au dus fiii lui Dan cu război asupra Leșemului și l-au împresurat, l-au lovit cu sabia și l-au luat moștenire și s-au așezat pe el și l-au numit Leșemul lui Dan, după numele lui Dan, tatăl lor.
\par 48 Aceasta este partea seminției fiilor lui Dan, după familiile lor și acestea sunt cetățile și satele lor.
\par 49 După ce au isprăvit împărțirea țării, prin sorți, fiii lui Israel au dat între ei parte de moștenire lui Iosua, fiul lui Navi.
\par 50 După porunca Domnului, i-au dat lui cetatea Timnat-Serah, pe care a cerut-o el, în muntele lui Efraim. Și a zidit cetate și a locuit în ea.
\par 51 Acestea sunt moșiile pe care Eleazar preotul, Iosua, fiul tui Navi, și căpeteniile familiilor le-au împărțit fiilor lui Israel, prin sorti, în Șilo, înaintea feței Domnului, la intrarea cortului adunării. Și așa s-a isprăvit împărțirea țării.

\chapter{20}

\par 1 Și a zis Domnul către Iosua:
\par 2 "Spune fiilor lui Israel: Faceți-vă orașe de scăpare cum v-am zis Eu prin Moise,
\par 3 Ca să poată scăpa acolo ucigașul care a ucis om din greșeală, fără precugetare; și să fie orașele acestea loc de scăpare pentru cel ce a ucis, ca să nu moară de mâna celui ce răzbună sângele vărsat, înainte de a se înfățișa înaintea obștii la judecată.
\par 4 Și cine va fugi în una din cetățile acestea să stea la poarta cetății și să spună pricina sa în auzul bătrânilor cetății și ei îl vor primi în cetate și-i vor da loc, ca să trăiască la ei.
\par 5 Și când va alerga după el cel ce răzbună sângele, atunci ei nu trebuie să-l dea pe ucigaș în mâinile lui, pentru că el a ucis fără să vrea pe aproapele său, neavând pe dânsul ură cu o zi sau cu două mai înainte.
\par 6 Să trăiască el în cetatea aceasta până ce se va înfățișa înaintea obștii la judecată, până va muri arhiereul care va fi în acele zile. Apoi să se întoarcă ucigașul și să sa ducă în cetatea sa și la casa sa, în cetatea de unde a fugit".
\par 7 Și au rânduit Chedeșul în Galileea, în muntele Neftalimului; Sichemul în muntele Efraim și Chiriat-Arba sau Hebronul, în muntele lui Iuda.
\par 8 Peste Iordan, în fața lerihonului, spre răsărit, au rânduit: Bețerul în pustiu, la șes, în seminția lui Ruben; Ramot în Galaad, în seminția lui Gad; și Golan în Vasan, în seminția lui Manase.
\par 9 Aceste cetăți le-au rânduit pentru toți fiii lui Israel și pentru străinii care trăiesc printre ei, ca să fugă acolo cel ce va ucide om din greșeală, ca să nu moară de mâna celui ce răzbună sângele vărsat, până nu se va înfățișa înaintea obștii la judecată.

\chapter{21}

\par 1 Căpeteniile familiilor Leviților au venit la Eleazar preotul și către Iosua, fiul lui Navi, și la căpeteniile semințiilor fiilor lui Israel,
\par 2 Și au vorbit cu ei în Șilo, în pământul Canaanului și au zis: "Domnul a poruncit prin Moise să ni se dea cetăți pentru locuit și împrejurimile lor pentru vitele noastre".
\par 3 Și au dat fiii lui Israel Leviților din părțile lor, după porunca Domnului, următoarele cetăți cu împrejurimile lor:
\par 4 Și s-au tras sorți pentru familia lui Cahat și Leviții, fiii lui Aaron preotul, au primit prin sorți treisprezece cetăți din seminția lui Iuda și din seminția lui Simeon și din seminția lui Veniamin.
\par 5 Iar celorlalți fii ai lui Cahat le-au căzut la sorți zece cetăți din familiile seminției lui Efraim și din seminția lui Dan și de la jumătate din seminția lui Manase.
\par 6 Fiilor lui Gherșon li s-au cuvenit prin sorți treisprezece cetăți de la familiile seminției lui Isahar și de la seminția lui Așer și de la seminția lui Neftali și de la jumătate din seminția lui Manase, în Vasan.
\par 7 Iar fiilor lui Merari, după familiile lor, le-au căzut la sorți douăsprezece cetăți din seminția lui Ruben și din seminția lui Gad și din seminția lui Zabulon.
\par 8 Și au dat fiii lui Israel cetățile acestea cu împrejurimile lor prin sorți, cum poruncise Domnul prin Moise.
\par 9 Din seminția fiilor lui Iuda și din seminția fiilor lui Veniamin și din seminția fiilor lui Simeon au dat cetățile următoare, care se numesc pe nume:
\par 10 Fiilor lui Aaron din familia lui Cahat dintre fiii lui Levi, fiindcă sorțul lor a fost întâiul,
\par 11 Li s-a dat: Chiriat-Arba, a tatălui lui Enac, sau Hebronul, în munții lui Iuda și locurile dimprejurul ei;
\par 12 Iar țarina cetății acesteia și satele ei s-au dat ca moșie lui Caleb, fiul lui Iefone.
\par 13 Și astfel fiilor lui Aaron preotul li s-a dat cetatea cea de scăpare a ucigașilor, Hebronul, și împrejurimile ei, Libna și împrejurimile ei.
\par 14 Iatirul și împrejurimile lui, Eștemoa și împrejurimile ei;
\par 15 Holonul și împrejurimile lui, Debirul și împrejurimile;
\par 16 Ainul și împrejurimile lui, Iuta și împrejurimile ei, Bet-Semeșul și împrejurimile lui: nouă cetăți din aceste două seminții.
\par 17 Iar din seminția lui Veniamin: Ghibeonul și împrejurimile lui, Gheba cu împrejurimile ei,
\par 18 Anatotul cu împrejurimile lui, Almonul cu împrejurimile lui: patru cetăți.
\par 19 Cetățile care au căzut la sorți pentru preoți, fiii lui Aaron, au fost toate treisprezece cu împrejurimile lor.
\par 20 Iar celorlalți din familiile fiilor lui Cahat, Leviților, le-a căzut la sorți cetăți în pământul lui Efraim.
\par 21 Și li s-a dat cetatea de scăpare pentru ucigași, Sichemul cu împrejurimile lui în muntele lui Efraim, Ghezerul cu împrejurimile lui,
\par 22 Chibțaimul cu împrejurimile lui, Bet-Horonul cu împrejurimile lui: patru cetăți.
\par 23 Din seminția lui Dan le-au căzut: Elteche și împrejurimile lui, Ghibetonul și împrejurimile lui;
\par 24 Aialonul și împrejurimile lui, Gat-Rimonul și împrejurimile lui: patru cetăți.
\par 25 Din jumătatea seminției lui Manase: Taanacul și împrejurimile lui, Gat-Rimonul și împrejurimile lui: două cetăți.
\par 26 Toate cetățile cu împrejurimile lor căzute la sorți celorlalți fii ai lui Cahat au fost zece.
\par 27 Iar fiilor lui Gherșon din familiile Leviților li s-au dat: două cetăți în Vasan, din jumătatea seminției lui Manase, și anume: Golanul, cetatea de scăpare pentru ucigași, cu împrejurimile lui, și Beștra cu împrejurimile ei;
\par 28 Patru cetăți în seminția lui Isahar: Chișionul cu împrejurimile lui, Dabrat cu împrejurimile lui,
\par 29 Iarmutul cu împrejurimile lui și En-Ganimul cu împrejurimile lui;
\par 30 Patru cetăți în seminția lui Așer: Mișalul cu împrejurimile lui, Abdonul cu împrejurimile lui,
\par 31 Helcatul cu împrejurimile lui și Rehobul cu împrejurimile lui;
\par 32 Trei cetăți din seminția lui Neftali: Chedeșul Galileii, cetate pentru scăparea ucigașilor, cu împrejurimile ei, Hamot-Dorul cu împrejurimile lui și Cartanul cu împrejurimile lui.
\par 33 Toate cetățile căzute la sorți fiilor lui Gherșon, după familiile lor, au fost treisprezece cetăți cu împrejurimile lor.
\par 34 Celorlalți leviți din fiii lui Merari li s-au dat patru cetăți din seminția lui Zabulon: Iocneamul cu împrejurimile lui,
\par 35 Carta cu împrejurimile ei, Dimna cu împrejurimile ei și Nahalalul cu împrejurimile lui;
\par 36 Patru cetăți de cealaltă parte de Iordan, în fața Ierihonului, în seminția lui Ruben: Bețerul, cetate de scăpare pentru ucigași, cu împrejurimile lui, în pustiul Miso, Iahța și împrejurimile ei,
\par 37 Chedemotul și împrejurimile lui și Mefaatul cu împrejurimile lui.
\par 38 În seminția lui Gad li s-au dat cetățile de scăpare pentru ucigași: Ramot-Galaadul cu împrejurimile lui, Mahanaimul cu împrejurimile lui,
\par 39 Heșbonul cu împrejurimile lui și Iazerul cu împrejurimile lui.
\par 40 Cetățile ieșite la sorți pentru familiile de Leviți din neamul lui Merari, după familiile lor, au fost douăsprezece.
\par 41 Iar toate cetățile date Leviților între fiii lui Israel au fost patruzeci și opt de cetăți cu împrejurimile lor. Fiecare din aceste cetăți își avea împrejurimile ei de jur împrejur; așa erau toate cetățile acestea.
\par 42 După ce a sfârșit Iosua împărțirea țării prin sorți, fiii lui Israel au dat parte lui Iosua după porunca Domnului; și i-au dat cetatea pe care a cerut-o el: Timnat-Serah, în muntele Efraim, și a zidit Iosua cetatea pe care a cerut-o și a locuit în ea. Și a luat Iosua cuțitele cele de piatră, cele cu care tăiase împrejur pe fiii lui Israel, care se născuseră pe cale în pustiu, căci în pustiu nu fuseseră tăiați împrejur, și le-a pus în Timnat-Serah.
\par 43 Astfel a dat Domnul lui Israel toată țara, pe care jurase să o dea părinților lor și au primit-o ei moștenire și s-au așezat în ea.
\par 44 Și le-a dat Domnul liniște și odihnă din toate părțile, cum jurase părinților lor, și nimeni dintre toți vrăjmașii lor n-a putut sta împotriva lor, ci pe toți vrăjmașii lor i-a dat Domnul în mâinile lor.
\par 45 Și n-a rămas neîmplinit nici un cuvânt din toate cuvintele cele bune pe care le vorbise Domnul casei lui Israel: toate s-au împlinit.

\chapter{22}

\par 1 Atunci Iosua a chemat seminția lui Ruben, a lui Gad și jumătate din a lui Manase și le-a zis:
\par 2 "Voi ați împlinit toate câte v-a poruncit Domnul prin Moise și ați ascultat cuvintele mele întru toate câte v-am poruncit;
\par 3 N-ați lăsat pe frații voștri în toată această îndelungată vreme, până în ziua aceasta și ați împlinit cele ce se cuvenea a împlini după porunca Domnului Dumnezeului vostru.
\par 4 Acum Domnul Dumnezeul vostru a liniștit pe frații voștri, cum le spusese. Întoarceți-vă dar și duceți-vă la corturile voastre, în pământul moștenirii voastre, pe care vi l-a dat Moise, sluga Domnului, peste Iordan.
\par 5 Dar să vă siliți a împlini cu grijă poruncile și legea pe care v-a dat-o Moise, sluga Domnului: de a iubi pe Domnul Dumnezeul vostru, de a umbla în toate căile Lui, de a păzi poruncile Lui, de a vă lipi de El și de a-I sluji Lui din toată inima voastră și din tot sufletul vostru".
\par 6 După aceea Iosua i-a binecuvântat și le-a dat drumul și ei s-au împărțit pe la corturile lor.
\par 7 Unei jumătăți din seminția lui Manase i-a dat Moise parte în Vasan, iar celeilalte jumătăți i-a dat Iosua parte cu frații lui dincoace de Iordan, spre apus. Și când le-a dat drumul Iosua pe la corturile lor și i-a binecuvântat,
\par 8 Atunci le-a spus: "Cu mari bogății vă întoarceți voi pe la corturile voastre, cu mare mulțime de vite și de argint, cu aur, cu aramă și cu fier și cu mare mulțime de haine: să împărțiți dar prada cu frații voștri".
\par 9 Și s-au întors fiii lui Ruben și ai lui Gad și jumătate din seminția lui Manase și au plecat de la fiii lui Israel din Șilo, care e în pământul Canaanului, ca să meargă în țara Galaadului, în pământul moștenirii lor, pe care îl luaseră în stăpânire după porunca Domnului dată prin Moise.
\par 10 Și ajungând în preajma Iordanului ce e în rara Canaan, fiii lui Ruben și fiii lui Gad și jumătate din seminția lui Manase au zidit acolo lângă Iordan jertfelnic, jertfelnic mare la vedere.
\par 11 Și au auzit fiii lui Israel că se zicea: "Iată fiii lui Ruben și fiii lui Gad și jumătate din seminția lui Manase au zidit jertfelnic pe pământul Canaanului, în preajma Iordanului, în fața fiilor lui Israel".
\par 12 Când au auzit acestea fiii lui Israel, s-a adunat toată obștea fiilor lui Israel la Șilo, ca să meargă împotriva lor cu război.
\par 13 Dar fiii lui Israel au trimis mai întâi la fiii lui Ruben, la fiii lui Gad și la jumătate din seminția lui Manase, în țara Galaadului, pe Finees, fiul preotului Eleazar,
\par 14 Împreună cu zece căpetenii, câte o căpetenie de fiecare seminție a lui Israel de dincoace de Iordan; fiecare din aceștia era căpetenie de mie din semințiile lui Israel.
\par 15 Și au venit ei la fiii lui Ruben și la fiii lui Gad și la jumătatea seminției lui Manase în pământul Galaadului și au vorbit cu ei și le-a zis:
\par 16 "Așa grăiește toată obștea Domnului: Ce înseamnă nelegiuirea aceasta pe care ați făcut-o înaintea Domnului Dumnezeului lui Israel, abătându-vă acum de la Domnul Dumnezeul lui Israel, ridicându-vă jertfelnic și sculându-vă acum împotriva Domnului?
\par 17 Nu vă ajunge oare păcatul din Peor, de care nu ne-am spălat nici până în ziua de astăzi și pentru care a fost bătută obștea de Domnul?
\par 18 Și iată astăzi voi vă abateți de la Domnul! Astăzi voi vă sculați împotriva Domnului, iar mâine se va mânia Domnul pe toată obștea lui Israel.
\par 19 Dacă însă pământul moștenirii voastre vi se pare necurat, atunci treceți în pământul moștenirii Domnului, unde se află cortul Domnului, luați-vă parte între noi, dar nu vă ridicați împotriva Domnului, nici împotriva noastră nu vă ridicați, zidindu-vă jertfelnic afară de cel al Domnului Dumnezeului nostru.
\par 20 Oare n-a făcut singur Acan, fiul lui Zerah, nelegiuire, luând din cele date nimicirii, dar mânia a venit asupra a toată obștea fiilor lui Israel? Și oare numai el singur a murit pentru nelegiuire?"
\par 21 Atunci fiii lui Ruben și fiii lui Gad și jumătate din seminția lui Manase, răspunzând la acestea, au zis căpeteniilor lui Israel:
\par 22 "Dumnezeul dumnezeilor este Domnul și Domnul Dumnezeul dumnezeilor știe și să știe și Israel: de ne răzvrătim și ne abatem noi de la Domnul, atunci să nu ne cruțe pe noi Domnul astăzi!
\par 23 Și dacă am ridicat noi un jertfelnic, ca să ne abatem de la Domnul Dumnezeul nostru și ca să aducem pe el arderi de tot și prinos de pâine și ca să săvârșim pe el jertfe de împăcare, atunci Domnul să ne ceară socoteală de aceasta!
\par 24 Dar noi am făcut aceasta de teamă ca nu cumva în viitor fiii voștri să zică fiilor noștri: "Ce aveți voi cu Domnul Dumnezeul lui Israel?
\par 25 Domnul a pus hotar între noi și voi, fiii lui Ruben și fiii lui Gad și jumătatea seminției lui Manase, Iordanul; voi n-aveți deci nici o legătură cu Domnul". Și astfel fiii voștri nu vor îngădui fiilor noștri să se închine Domnului în Șilo.
\par 26 De aceea am zis noi: Să ne facem un jertfelnic, nu pentru arderi de tot nici pentru jertfe,
\par 27 Ci ca să fie el între noi și voi, între urmașii noștri, mărturie că noi putem sluji Domnului cu arderile de tot ale noastre și cu jertfele noastre și cu cele de împăcare ale noastre, și pentru ca în vremurile viitoare să nu zică fiii voștri către fiii noștri: Voi nu aveți nici o legătură cu Domnul.
\par 28 Și ziceam noi: Dacă ni se va zice astfel nouă și urmașilor noștri, atunci vom răspunde: Priviți chipul jertfelnicului Domnului pe care l-au făcut părinții noștri nu pentru arderi de tot și nu pentru jertfe, ci ca să fie mărturie între noi și voi și între fiii noștri și fiii voștri.
\par 29 Să nu se întâmple una ca aceea ca să ne ridicăm noi împotriva Domnului și să ne abatem acum de la Domnul și să facem jertfelnic pentru arderi de tot și pentru prinos de pâine și pentru jertfe, afară de jertfelnicul Domnului Dumnezeu care se află înaintea cortului".
\par 30 Iar preotul Finees și toate căpeteniile obștii și căpeteniile peste miile lui Israel, care erau cu dânsul, auzind cuvintele pe care le-au vorbit fiii lui Ruben și fiii lui Gad și jumătatea seminției lui Manase, au rămas mulțumiți.
\par 31 Și a zis Finees, fiul preotului Eleazar, către fiii lui Ruben și către fiii lui Gad și către jumătatea de seminție a lui Manase: "Astăzi am aflat noi că Domnul este în mijlocul nostru și că voi n-ați făcut prin aceasta o nelegiuire; acum ați izbăvit pe fiii lui Israel din mâna Domnului".
\par 32 Și s-a întors Finees, fiul preotului Eleazar, și căpeteniile de la fiii lui Ruben și de la fiii lui Gad și de la jumătatea de seminție a lui Manase din pământul Galaadului în țara Canaan la fiii lui Israel și le-a adus răspunsul.
\par 33 Și le-a plăcut aceasta fiilor lui Israel și au binecuvântat fiii lui Israel pe Dumnezeu și au zis să nu se mai ridice împotriva lor cu război, ca să pustiiască țara în care locuiau fiii lui Ruben și fiii lui Gad și jumătate din seminția lui Manase.
\par 34 Iar fiii lui Ruben și fiii lui Gad și jumătate din seminția lui Manase au numit jertfelnicul Ed, adică mărturie, căci își ziceau: Acesta este mărturie între noi că Domnul este Dumnezeul nostru.

\chapter{23}

\par 1 Trecând multă vreme, după ce Domnul Dumnezeu a odihnit pe Israel, scutindu-l de toți vrăjmașii lui din toate părțile, Iosua a ajuns bătrân și înaintat în vârstă.
\par 2 Atunci a chemat Iosua pe toți fiii lui Israel; pe bătrânii lor, căpeteniile lor, pe judecătorii lor și pe mai-marii oștilor lor și le-a zis: "Eu am îmbătrânit și sunt înaintat în vârstă;
\par 3 Voi ați văzut ce a făcut Domnul Dumnezeul vostru înaintea feței voastre cu toate aceste popoare, căci Domnul Dumnezeul vostru Însuși S-a luptat pentru voi.
\par 4 Iată eu v-am împărțit prin sorți popoarele acestea ce au mai rămas în moștenirea semințiilor voastre, toate popoarele pe care eu le-am nimicit de la Iordan până la Marea cea Mare de la apusul soarelui.
\par 5 Domnul Dumnezeul vostru Însuși le va alunga de la fața voastră până vor pieri; și va trimite asupra lor fiare sălbatice până le va stârpi pe ele și pe regii lor de la fața voastră, și le va nimici înaintea voastră, ca să primiți de moștenire țara lor, cum v-a grăit Domnul Dumnezeu.
\par 6 De aceea siliți-vă să pliniți întocmai și să păziți cele scrise în cartea legii lui Moise, neabătându-vă de la ea nici la dreapta, nici la stânga.
\par 7 Să nu intrați în legătură cu aceste popoare care au mai rămas printre voi, să nu pomeniți numele dumnezeilor lor, să nu vă plecați înaintea lor, nici să le slujiți, sau să vă închinați lor;
\par 8 Ci vă lipiți de Domnul Dumnezeul vostru, cum ați făcut până în ziua de astăzi.
\par 9 Domnul a alungat de la voi popoarele mari și tari și nimeni nu s-a putut împotrivi până astăzi;
\par 10 Unul din voi a alungat mii, căci Însuși Domnul Dumnezeul vostru S-a luptat pentru voi, cum v-a grăit.
\par 11 De aceea siliți-vă să iubiți pe Domnul Dumnezeul vostru.
\par 12 Iar de vă veți întoarce și vă veți alătura la popoarele acestea rămase și veți intra în înrudire cu ele și veți merge la ele și ele vor veni la voi,
\par 13 Atunci să știți că Domnul Dumnezeul vostru nu va mai alunga de la voi popoarele acestea, ci ele vor fi pentru voi laț și mreajă, bici pentru spinările voastre și spin pentru ochii voștri, până veți fi stârpiți din această țară bună pe care v-a dat-o Domnul Dumnezeul vostru.
\par 14 Iată eu astăzi plec în calea în care merg toți pământenii, iar voi să recunoașteți cu toată inima voastră și cu tot sufletul vostru că n-a rămas zadarnic nici un cuvânt din toate cuvintele bune pe care le-a rostit pentru voi Domnul Dumnezeul vostru: toate s-au împlinit pentru voi și nici un cuvânt n-a rămas neîmplinit.
\par 15 Dar după cum s-a împlinit cu voi tot cuvântul bun pe care l-a grăit Domnul Dumnezeul vostru, tot așa va împlini Domnul asupra voastră și tot cuvântul rău până vă va stârpi din această țară bogată pe care v-a dat-o Domnul Dumnezeul vostru.
\par 16 De veți călca așezământul Domnului Dumnezeului vostru, pe care l-a încheiat El cu voi și vă veți duce să slujiți la alți dumnezei și să vă închinați lor, se va aprinde asupra voastră mânia Domnului și veți pieri curând din țara aceasta bogată pe care v-a dat-o Domnul".

\chapter{24}

\par 1 Apoi a adunat Iosua toate semințiile lui Israel la Sichem și a chemat pe bătrânii lui Israel, pe căpeteniile lui, pe judecătorii lui și pe mai-marii oștirii lui; și s-au înfățișat ei înaintea Domnului.
\par 2 Și a zis Iosua către tot poporul: "Așa zice Domnul Dumnezeul lui Israel: În vechime părinții voștri, Terah, tatăl lui Avraam și tatăl lui Nahor, au trăit dincolo de râu (Eufrat) și slujeau la alți dumnezei.
\par 3 Dar Eu am luat pe părintele vostru Avraam de mână, de dincolo de Eufrat, și l-am povățuit către această țară a Canaanului și am înmulțit sămânța lui și i-am dat pe Isaac.
\par 4 Lui Isaac i-am dat pe Iacov și pe Isav; lui Isav i-am dat muntele Seir de moștenire, iar Iacov și fiii lui s-au pogorât în Egipt și au ajuns acolo popor mare, tare și mult la număr și Egiptenii au început să-i strâmtoreze.
\par 5 Dar am trimis pe Moise și pe Aaron și am lovit Egiptul cu semne pe care le-am făcut Eu acolo și apoi v-am scos pe voi.
\par 6 Eu am scos pe părinții voștri din Egipt și ați venit la Marea Roșie. Atunci Egiptenii au alergat după părinții voștri cu care și cu călăreți până la Marea Roșie.
\par 7 Iar ei au strigat către Domnul și El a pus nor între voi și Egipteni și a adus asupra lor marea care i-a și acoperit. Ochii voștri au văzut ce am făcut Eu în Egipt. După aceea, ați rămas voi multă vreme în pustiu.
\par 8 Apoi v-am dus Eu asupra Amoreilor care locuiau peste Iordan; și ei s-au luptat cu voi, dar Eu i-am dat în mâinile voastre și ați primit de moștenire țara lor și Eu i-am stârpit înaintea voastră.
\par 9 S-a sculat apoi Balac, fiul lui Sefor, regele Moabului, și a pornit cu război asupra lui Israel și a trimis să cheme pe Valaam, fiul lui Beor, ca să vă blesteme;
\par 10 Dar Eu n-am voit să ascult pe Valaam și el v-a binecuvântat și v-am izbăvit din mâinile lui Balac.
\par 11 După aceea ați trecut Iordanul și ați venit la Ierihon. Atunci au început a se lupta cu voi locuitorii Ierihonului, apoi Amoreii, Ferezeii, Canaaneii, Heteii, Ghergheseii, Heveii și Iebuseii, dar Eu i-am dat în mâinile voastre.
\par 12 Trimis-am înaintea voastră viespi care au gonit de la voi pe cei doi regi ai Amoreilor; nu cu sabia ta, nici cu arcul tău ai făcut acestea.
\par 13 Și v-am dat țara cu care nu v-aii ostenit și cetățile pe care nu le-aii zidit și trăiți în ele; din viile și din grădinile de măslini pe care nu le-ați sădit, iată, mâncați roade.
\par 14 Temeți-vă dar de Domnul și-I slujiți Lui cu credincioșie și curățenie. Lepădați dumnezeii cărora au slujit părinții voștri dincolo de râu și în Egipt și slujiți Domnului.
\par 15 Iar dacă nu vă place să slujiți Domnului, atunci alegeți-vă acum cui veți sluji: sau dumnezeilor cărora au slujit părinții voștri cei de peste râu sau dumnezeilor Amoreilor, în țara căl-ora trăiți. Eu însă și casa mea vom sluji Domnului, că sfânt este!"
\par 16 Și a răspuns poporul și a zis: "Departe de noi. gândul să părăsim pe Domnul și să ne apucăm să slujim la alți dumnezei,
\par 17 Căci Domnul este Dumnezeul nostru; El ne-a scos pe noi și pe părinții noștri din țara Egiptului, din casa robiei și a făcut Înaintea ochilor noștri minuni mari și ne-a păzit în toată calea pe care am umblat și printre toate popoarele pe la care am trecut.
\par 18 Domnul a alungat de la noi toate popoarele și pe Amoreii care trăiau în țara aceasta. De aceea și noi vom sluji Domnului, căci El este Dumnezeul nostru!"
\par 19 Și a zis Iosua poporului: "Nu veți putea să slujiți Domnului Dumnezeu, căci El este Dumnezeu sfânt, Dumnezeu zelos și nu va răbda nelegiuirile  voastre, nici păcatele voastre.
\par 20 Dacă voi veți părăsi pe Domnul și veți sluji la dumnezei străini, atunci El va aduce asupra voastră răul și vă va stârpi, după ce v-a făcut bine".
\par 21 Iar poporul a zis către Iosua: "Nu, noi Domnului vom sluji".
\par 22 Iosua însă a zis poporului: "Vă sunteți voi martori că v-aii ales pe Domnul să-I slujiți?" Ei au răspuns: "Suntem martori!"
\par 23 "Așadar, a adăugat Iosua, lepădați dumnezeii străini pe care îi aveți și întoarceți-vă inima către Domnul Dumnezeul lui Israel!"
\par 24 Și a zis poporul către Iosua: "Domnului Dumnezeului nostru vom sluji și glasul Lui vom asculta!"
\par 25 Și a încheiat Iosua cu poporul legământ în ziua aceea și i-a dat legi și porunci în Sichem, înaintea cortului Domnului Dumnezeului lui Israel.
\par 26 Și a scris Iosua cuvintele acestea în cartea legii lui Dumnezeu și a luat o piatră mare și a pus-o acolo sub stejarul care era lângă locașul sfânt al Domnului.
\par 27 Apoi a zis Iosua către tot poporul: "Iată piatra aceasta ne va fi mărturie, căci ea a auzit toate cuvintele Domnului, pe care le-a grăit El cu noi astăzi. Să fie dar ca mărturie împotriva voastră în zilele viitoare, ca să nu mințiți înaintea Domnului Dumnezeului vostru!"
\par 28 Și a dat Iosua drumul poporului și s-a întors fiecare la moștenirea lui.
\par 29 Și a murit după aceea Iosua, fiul lui Navi, robul Domnului, fiind de o sută zece ani.
\par 30 Și l-au îngropat în ținutul moștenirii sale la Timnat-Serah, care e în muntele Efraim, la miazănoapte de muntele Gaaș. Și au pus acolo cu dânsul, în mormântul în care l-au îngropat, cuțitele cele de piatră cu care Iosua a tăiat împrejur pe fiii lui Israel în Ghilgal, când i-a scos pe ei din Egipt, cum poruncise Domnul, și sunt ele acolo până în ziua de astăzi.
\par 31 Israel a slujit Domnului în toate zilele lui Iosua și în toate zilele bătrânilor a căror viață s-a prelungit după Iosua și care văzuseră toate lucrurile Domnului, pe care le făcuse El cu Israel.
\par 32 Oasele lui Iosif, pe care le aduseseră fiii lui Israel din Egipt, le-au îngropat în Sichem, în partea de țarină pe care o cumpărase Iacov de la fiii lui Hemor, tatăl lui Sichem, cu o sută de arginți, și care căzuse de moștenire fiilor lui Iosif.
\par 33 După aceasta a murit și Eleazar, fiul lui Aaron, arhiereul, și l-au îngropat în Ghibeea, orașul lui Finees, fiul lui, care i se dăduse în muntele Efraim. În ziua aceea fiii lui Israel, luând chivotul lui Dumnezeu, l-au dus cu ei, iar Finees a fost preot în locul lui Eleazar, tatăl lui, până ce a murit și a fost îngropat în cetatea sa Ghibeea. Iar fiii lui Israel s-au dus fiecare la locul său și în cetatea sa. Și au început fiii lui Israel a sluji Astartei și lui Aștarot și dumnezeilor popoarelor vecine. De aceea i-a dat Domnul în mâinile lui Eglon, regele Moabului, și i-a stăpânit optsprezece ani.


\end{document}