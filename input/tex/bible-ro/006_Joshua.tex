\begin{document}

\title{Iosua}


\chapter{1}

\par 1 Dupa moartea lui Moise, robul Domnului, a grait Domnul cu Iosua, fiul lui Navi, slujitorul lui Moise, ?i a zis:
\par 2 "Moise, robul Meu, a murit. Scoala dar ?i treci Iordanul tu ?i tot poporul acesta, în ?ara pe care o voi da fiilor lui Israel.
\par 3 Tot locul pe care vor calca talpile picioarelor voastre, îl voi da voua, cum am spus lui Moise:
\par 4 De la pustie ?i de la Libanul acesta pâna la râul cel mare, pâna la râul Eufratului ?i pâna la marea cea mare spre asfin?itul soarelui vor fi hotarele voastre.
\par 5 Nimeni nu se va putea împotrivi ?ie, în toate zilele vie?ii tale. Precum am fost cu Moise, a?a voi fi ?i cu tine; nu Ma voi departa de tine ?i nu te voi parasi.
\par 6 Fii tare ?i curajos, ca tu vei împar?i poporului acestuia, prin sor?i, ?ara pe care M-am jurat parin?ilor lor sa le-o dau.
\par 7 Fii dar tare ?i foarte curajos, ca sa paze?ti ?i sa împline?ti toata legea pe care ?i-a încredin?at-o Moise, robul Meu; sa nu te aba?i de la ea nici la dreapta nici la stânga, ca sa pricepi toate câte ai de facut.
\par 8 Sa nu se pogoare cartea legii acesteia de pe buzele tale, ci calauze?te-te de ea ziua ?i noaptea, ca sa pline?ti întocmai tot ce este scris în ea; atunci vei fi cu izbânda în caile tale ?i vei pa?i cu spor.
\par 9 Iata î?i poruncesc: Fii tare ?i curajos, sa nu te temi, nici sa te spaimântezi, caci Domnul Dumnezeul tau este cu tine pretutindenea, oriunde vei merge".
\par 10 Atunci a poruncit Iosua capeteniilor poporului ?i a zis:
\par 11 "Trece?i prin tabara ?i porunci?i poporului ?i-i zice?i: Pregati?i-va merinde de drum, ca dupa trei zile ve?i trece peste Iordanul acesta, ca sa merge?i ?i sa lua?i în stapânire pamântul pe care Domnul Dumnezeul parin?ilor vo?tri are sa vi-l dea".
\par 12 Iar semin?iei lui Ruben ?i semin?iei lui Gad ?i la jumatate din semin?ia lui Manase, Iosua le-a zis:
\par 13 "Aduce?i-va aminte ce v-a poruncit Moise, sluga Domnului, când v-a zis: Domnul Dumnezeul vostru v-a lini?tit ?i v-a dat pamântul acesta.
\par 14 Femeile voastre, copiii vo?tri ?i vitele voastre sa ramâna în pamântul pe care vi l-a dat Moise dincoace de Iordan, iar voi to?i câ?i pute?i lupta, înarmându-va, duce?i-va înaintea fra?ilor vo?tri ?i le ajuta?i,
\par 15 Pâna ce Domnul Dumnezeul vostru va lini?ti ?i pe fra?ii vo?tri, cum va lini?tit ?i pe voi, ?i pâna ce vor primi ?i ei de mo?tenire pamântul pe care Domnul Dumnezeul vostru îl da lor. Atunci va ve?i întoarce în mo?tenirea voastra ?i ve?i stapâni pamântul pe care Moise, sluga Domnului, vi l-a dat dincolo de Iordan, spre rasaritul soarelui".
\par 16 Iar ei au raspuns lui Iosua ?i au zis: "Toate, oricâte ne vei porunci, vom face ?i oriunde ne vei trimite, vom merge.
\par 17 Cum am ascultat pe Moise, a?a te vom asculta ?i pe tine, numai sa fie Domnul Dumnezeul tau cu tine cum a fost cu Moise.
\par 18 Tot cel ce se va împotrivi poruncilor tale ?i nu va asculta cuvintele tale, în toate câte vei porunci, sa moara. Dar fii tare ?i curajos!"

\chapter{2}

\par 1 Atunci Iosua, fiul lui Navi, a trimis în taina din Sitim doi tineri sa iscodeasca ?ara ?i a zis: "Duce?i-va ?i cerceta?i ?ara ?i mai ales Ierihonul!" ?i s-au dus cei doi tineri ?i, ajungând la Ierihon, au intrat în casa unei desfrânate, al carei nume era Rahab, ?i au ramas sa se odihneasca acolo.
\par 2 ?i s-a dat de ?tire regelui Ierihonului: "Iata, ni?te oameni din fiii lui Israel au venit aici în noaptea aceasta, ca sa iscodeasca ?ara!"
\par 3 Iar regele Ierihonului a trimis la Rahab sa i se spuna: "Scoate pe oamenii care au intrat în casa ta, în noaptea aceasta, ca au venit sa iscodeasca ?ara".
\par 4 Femeia însa, luând pe cei doi oameni, i-a ascuns ?i a zis: "Adevarat, au venit la mine ni?te oameni,
\par 5 Dar în amurg, când se închideau por?ile, barba?ii au plecat ?i nu ?tiu unde s-au dus. Alerga?i dupa ei ?i-i ve?i ajunge".
\par 6 Apoi ea a suit pe cei doi oameni pe acoperi? ?i i-a ascuns în ni?te fuioare de in ce se aflau pe acoperi?ul casei ei.
\par 7 Iar trimi?ii regelui au alergat dupa ei pe drumul cel catre Iordan, pâna la vad, ?i por?ile s-au închis îndata ce au ie?it cei ce urmareau pe iscoade.
\par 8 Înainte însa de a adormi iscoadele femeia s-a suit la dân?ii pe acoperi?,
\par 9 ?i le-a zis: "?tiu ca Domnul v-a dat voua pamântul acesta, caci frica voastra a cazut asupra noastra ?i to?i locuitorii pamântului acestuia au frica de voi,
\par 10 Pentru ca am auzit noi cum a secat Domnul Dumnezeu înaintea voastra Marea Ro?ie, când ah ie?it din Egipt, ?i câte a?i facut voi cu cei doi regi ai Amoreilor peste Iordan, cu Sihon ?i cu Og, pe care i-a?i pierdut.
\par 11 Când am auzit noi de acestea, ne-a slabit inima ?i în nici unul din noi n-a mai ramas barba?ie în fa?a voastra, caci Domnul Dumnezeul vostru este Dumnezeu în cer sus ?i pe pamânt jos.
\par 12 Acum dar jura?i-mi pe Domnul Dumnezeul vostru ca, precum am facut eu mila cu voi, ve?i face ?i voi mila cu casa tatalui meu ?i da?i-mi semn de nadejde,
\par 13 Ca ve?i lasa cu via?a pe tatal meu, pe mama mea, pe fra?ii mei, pe surorile mele, toata casa mea ?i ve?i izbavi sufletul meu de la moarte".
\par 14 Iar oamenii aceia au zis catre dânsa: "Sufletul nostru îl vom pune pentru voi". ?i ea a zis: "Când Domnul va va da cetatea aceasta, sa face?i cu mine mila ?i dreptate".
\par 15 Apoi le-a dat drumul cu o frânghie pe fereastra, caci casa ei era în zidul ceta?ii ?i ea locuia chiar deasupra zidului.
\par 16 ?i le-a zis: "Duce?i-va în munte, ca sa nu va întâlni?i cu cei ce va urmaresc ?i sta?i acolo ascun?i trei zile, pâna se vor întoarce cei ce va urmaresc ?i apoi va ve?i duce în drumul vostru".
\par 17 Iar oamenii aceia au zis catre dânsa: "De nu vei face cum î?i vom zice, vom fi slobozi de acest juramânt al tau:
\par 18 Iata, când vom intra noi în cetate pe partea aceasta, tu sa pui semn aceasta funie ro?ie la fereastra pe care ne-ai slobozit, iar pe tatal tau, pe mama ta, pe fra?ii tai ?i pe to?i cei din familia tatalui tau sa-i aduni în casa ta.
\par 19 ?i de va ie?i cineva afara pe u?a casei tale, sângele aceluia sa fie asupra capului lui, ?i noi vom fi slobozi de acest juramânt al tau; iar cine va fi cu tine în casa ta, sângele aceluia sa fie asupra capului nostru, de se va atinge de el mâna cuiva.
\par 20 Daca însa ne va napastui cineva ?i va descoperi fapta noastra, atunci vom fi slobozi de acest juramânt al tau".
\par 21 ?i ea le-a zis: "Sa fie cum ai grait!" Apoi le-a dat drumul ?i s-au dus, iar ea a legat la fereastra funia cea ro?ie.
\par 22 Ducându-se oamenii aceia ?i ajungând în munte, au a?teptat acolo trei zile, pâna ce s-au întors cei ce-i urmareau, care-i cautasera pe toate drumurile ?i nu-i gasira.
\par 23 Întorcându-se apoi, cei doi tineri s-au coborât din munte, au trecut Iordanul ?i au venit la Iosua, fiul lui Navi, ?i i-au povestit tot ce se întâmplase cu ei,
\par 24 ?i au zis catre Iosua: "Domnul Dumnezeul nostru a dat tot pamântul acesta în mâinile noastre ?i to?i cei ce locuiesc în ?ara aceea tremura de frica noastra".

\chapter{3}

\par 1 Sculându-se apoi Iosua a doua zi dis-de-diminea?a, a pornit de la Sitim ?i a venit pâna la Iordan, el ?i to?i fiii lui Israel, ?i au poposit acolo înainte de a-l trece.
\par 2 Iar dupa trei zile au trecut vestitori prin tabara,
\par 3 ?i au poruncit poporului, zicând: "Când ve?i vedea chivotul legamântului Domnului Dumnezeului vostru ?i preo?ii vo?tri ?i pe levi?ii cei ce-l duc, atunci sa porni?i ?i voi de la locul vostru ?i sa merge?i dupa el.
\par 4 Iar departarea între voi ?i el sa fie ca de doua mii de co?i; sa nu va apropia?i prea mult de el, ca sa pute?i vedea bine calea pe care ave?i sa merge?i; caci înainte n-a?i mai umblat pe calea aceasta niciodata".
\par 5 ?i a mai zis Iosua catre popor: "Sfin?i?i-va pentru diminea?a, caci mâine are sa faca Domnul minuni între voi".
\par 6 Iar preo?ilor le-a zis Iosua: "Ridica?i chivotul legamântului Domnului ?i merge?i înaintea poporului!" ?i au luat preo?ii chivotul legamântului Domnului ?i au purces înaintea poporului.
\par 7 Atunci a zis Domnul catre Iosua: "În ziua aceasta voi începe a te preamari înaintea ochilor tuturor fiilor lui Israel, ca sa cunoasca ei ca, precum am fost cu Moise, a?a am sa fiu ?i cu tine.
\par 8 Tu însa sa porunce?ti preo?ilor care poarta chivotul legamântului ?i sa zici: "Îndata ce ve?i intra în mijlocul apelor Iordanului, sa va opri?i în Iordan!"
\par 9 Iar Iosua a zis catre fiii lui Israel: "Veni?i încoace ?i asculta?i cuvintele Domnului Dumnezeului vostru.
\par 10 Din aceasta ve?i cunoa?te ca în mijlocul vostru este Dumnezeul cel viu Care va alunga de la voi pe Canaanei, pe Hetei, pe Hevei, pe Ferezei, pe Gherghesei, pe Amorei ?i pe Iebusei.
\par 11 Iata chivotul legamântului Domnului a tot pamântul va trece înaintea voastra peste Iordan.
\par 12 Sa va alege?i doisprezece oameni dintre fiii lui Israel, câte un om din fiecare semin?ie;
\par 13 ?i îndata ce talpile picioarelor preo?ilor care duc Chivotul Domnului, Stapânul a tot pamântul, vor calca în apa Iordanului, apele Iordanului se vor despar?i: cele de la vale se vor scurge, iar cele care vin din sus se vor opri ca un perete".
\par 14 Deci poporul a pornit de la corturile sale, ca sa treaca Iordanul, iar preo?ii care duceau chivotul legamântului Domnului mergeau înaintea poporului.
\par 15 Îndata ce preo?ii cei ce duceau chivotul au intrat în Iordan ?i picioarele preo?ilor, care duceau chivotul, s-au afundat în apa Iordanului,
\par 16 Apa care curgea din sus s-a oprit ?i s-a facut perete pe o foarte mare departare, pâna la cetatea Adam, care e lânga ?artan, iar cea care curgea spre marea cea din Arabah, spre Marea Sarata, s-a scurs ?i a secat, iar poporul a trecut în fala Ierihonului. Atunci Iordanul umplea matca sa ?i ie?ea din toate malurile sale, ca în timpul seceri?ului grâului.
\par 17 Preo?ii care duceau chivotul legamântului Domnului stateau ca pe uscat în mijlocul Iordanului, cu picioarele neudate; iar fiii lui Israel au mers ca pe uscat, pâna ce tot poporul a trecut prin Iordan.

\chapter{4}

\par 1 Dupa ce a ispravit tot poporul de trecut Iordanul; a grait Domnul catre Iosua ?i a zis:
\par 2 "Ia din popor doisprezece barba?i, câte un om din fiecare semin?ie,
\par 3 ?i le porunce?te: "Lua?i din mijlocul Iordanului douasprezece pietre, duce?i-le cu voi ?i le pune?i în tabara voastra, unde ave?i sa tabarâ?i la noapte".
\par 4 ?i a chemat Iosua doisprezece barba?i pe care ?i-i alesese dintre fiii lui Israel, câte un om din fiecare semin?ie,
\par 5 ?i le-a zis: "Merge?i înaintea chivotului Domnului Dumnezeului vostru, în mijlocul Iordanului, ?i lua?i de acolo ?i ridica?i pe umerii vo?tri fiecare câte o piatra, dupa numarul celor douasprezece semin?ii ale lui Israel,
\par 6 Ca sa va fie acestea puse pentru totdeauna ca semn în mijlocul vostru ?i, când va vor întreba mâine copiii vo?tri ?i vor zice: Ce înseamna aici pietrele acestea?
\par 7 Atunci ve?i spune fiilor vo?tri: Apele Iordanului s-au despar?it înaintea chivotului legamântului Domnului a tot pamântul, când acesta l-a trecut. ?i a?a pietrele acestea vor fi pentru fiii lui Israel amintire pentru vecie".
\par 8 ?i au facut fiii lui Israel a?a cum le-a poruncit Iosua: dupa ce au ispravit fiii lui Israel de trecut Iordanul, au luat douasprezece pietre din Iordan, precum Domnul poruncise lui Iosua, dupa numarul semin?iilor fiilor lui Israel ?i le-au dus cu ei în tabara ?i le-au pus acolo.
\par 9 Iosua însa a pus alte douasprezece pietre în mijlocul Iordanului, pe locul unde au stat picioarele preo?ilor care duceau chivotul legamântului Domnului, ?i sunt acolo pâna în ziua de astazi.
\par 10 Preo?ii care duceau chivotul Domnului au stat în mijlocul Iordanului pâna ce a ispravit Iosua toate câte îi poruncise Domnul sa spuna poporului, cum poruncise Moise lui Iosua. ?i poporul a grabit sa treaca Iordanul.
\par 11 Iar dupa ce a trecut tot poporul, a trecut ?i chivotul Domnului ?i preo?ii au mers iara?i în fruntea poporului.
\par 12 Atunci au trecut ?i fiii lui Ruben, fiii lui Gad ?i jumatate din semin?ia lui Manase, gata de razboi, înaintea fiilor lui Israel, cum le poruncise Moise.
\par 13 Ca la patruzeci de mii de oameni înarma?i pentru razboi au trecut înaintea Domnului, ca sa lupte împotriva ceta?ii Ierihonului.
\par 14 În ziua aceea a preamarit Domnul pe Iosua înaintea a tot neamul lui Israel ?i s-au temut de dânsul cât a trait, cum se temusera ?i de Moise.
\par 15 Apoi a grait Domnul cu Iosua ?i a zis:
\par 16 "Porunce?te preo?ilor care duc chivotul marturiei Domnului, sa iasa din Iordan".
\par 17 ?i a poruncit Iosua preo?ilor ?i a zis: "Ie?i?i din Iordan!"
\par 18 ?i cum au ie?it din Iordan preo?ii, cei ce duceau chivotul legamântului Domnului, ?i ?i-au pus picioarele pe uscat, apa Iordanului a pornit pe albia sa ?i a curs ca mai înainte, pâna peste maluri.
\par 19 Poporul a ie?it din Iordan în ziua a zecea a lunii întâi ?i au tabarât fiii lui Israel la Ghilgal, în partea de rasarit a Ierihonului.
\par 20 Iar cele douasprezece pietre pe care le luasera ei din Iordan, le-a a?ezat Iosua în Ghilgal.
\par 21 ?i a zis fiilor lui Israel: "Când va vor întreba fiii vo?tri mâine ?i vor zice: Ce înseamna aceste pietre?
\par 22 Sa spune?i fiilor vo?tri: Israel a trecut prin Iordanul acesta, ca pe uscat,
\par 23 Caci Domnul Dumnezeul vostru a secat apele Iordanului înaintea lor, pâna ce le-au trecut, cum facuse Domnul Dumnezeul vostru ?i cu Marea Ro?ie, pe care a secat-o Domnul Dumnezeul nostru înaintea noastra. pâna ce am trecut-o,
\par 24 Ca sa cunoasca toate neamurile pamântului ca puterea Domnului este mare ?i ca voi sa va teme?i de Domnul Dumnezeul vostru, în toata vremea".

\chapter{5}

\par 1 Când au auzit regii Amoreilor, care locuiau peste Iordan ?i regii Canaaneilor, care locuiau pe lânga mare, ca Domnul Dumnezeu a secat râul Iordanului înaintea fiilor lui Israel pâna ce l-au trecut, a slabit inima lor, s-au înspaimântat ?i nu mai aveau curaj împotriva fiilor lui Israel.
\par 2 În vremea aceea a zis Domnul catre Iosua: "Fa-?i cu?ite taioase de cremene ?i taie împrejur pe fiii lui Israel a doua oara".
\par 3 ?i ?i-a facut Iosua cu?ite ascu?ite de cremene ?i a taiat împrejur pe fiii lui Israel, la locul ce se nume?te Dealul Aralot.
\par 4 Iata pricina pentru care Iosua a taiat împrejur pe fiii lui Israel: Tot poporul de parte barbateasca care ie?i se din Egipt, to?i cei buni de razboi, murisera pe cale, în pustiu, dupa ie?irea din Egipt;
\par 5 Caci tot poporul care ie?ise din Egipt era taiat împrejur; iar poporul nascut pe cale, în pustiu, dupa ie?irea din Egipt, nu era taiat împrejur.
\par 6 Caci fiii lui Israel umblasera prin pustiu patruzeci ?i doi de ani, pâna ce a murit tot poporul bun de razboi, care ie?ise din Egipt ?i care nu ascultase de glasul Domnului. Acestora Domnul se jurase sa nu le dea voie sa vada ?ara pe care El fagaduise cu juramânt sa o dea parin?ilor lor, ?ara în care curge lapte ?i miere.
\par 7 În locul acestora se ridicasera fiii lor, pe care i-a taiat împrejur Iosua, caci nu erau taia?i împrejur, pentru ca se nascusera pe cale.
\par 8 Dupa ce poporul s-a taiat împrejur, a ramas lini?tit în tabara, pâna s-a însanato?it.
\par 9 Apoi a zis Domnul catre Iosua, fiul lui Navi: "În ziua de astazi am ridicat de pe voi ocara Egiptului", ?i de aceea se ?i nume?te locul acela Ghilgal, pâna în ziua aceasta.
\par 10 Fiii lui Israel au stat cu tabara la Ghilgal ?i au facut acolo în ?esul Ierihonului Pa?tile în ziua a paisprezecea a lunii, seara.
\par 11 Iar a doua zi de Pa?ti au mâncat din roadele pamântului acestuia azime ?i pâine noua.
\par 12 În ziua aceea a încetat mana de a mai cadea ?i de a doua zi, dupa ce au mâncat din roadele pamântului, fiii lui Israel n-au mai avut mana, ci au mâncat anul acela din roadele Zarii Canaanului.
\par 13 Aflându-se însa Iosua aproape de Ierihon, a cautat cu ochii sai ?i iata statea înaintea lui un om; acela avea în mâna o sabie goala. ?i apropiindu-se Iosua de dânsul, i-a zis: "De-ai no?tri e?ti sau e?ti dintre du?manii no?tri?"
\par 14 Iar acela a raspuns: "Eu sunt capetenia o?tirii Domnului ?i am venit acum!" Atunci Iosua a cazut cu fala la pamânt, s-a închinat ?i a zis catre acela: "Stapâne, ce porunce?ti slugii tale?"
\par 15 Zis-a catre Iosua capetenia o?tirii Domnului: "Scoate-?i încal?amintea din picioare, ca locul pe care stai tu acum este sfânt!" ?i a facut Iosua a?a.

\chapter{6}

\par 1 Ierihonul însa era încuiat ?i întarit de frica fiilor lui Israel; nimeni nu intra în el ?i nimeni nu ie?ea.
\par 2 Atunci a zis Domnul catre Iosua: "Iata, Eu voi da în mâinile tale Ierihonul, pe regii lui ?i pe puternicii lui.
\par 3 Duce?i-va împrejurul ceta?ii to?i cei buni de razboi ?i înconjura?i cetatea câteodata pe zi. Aceasta sa o face?i ?ase zile.
\par 4 ?apte preo?i sa poarte înaintea chivotului ?apte trâmbi?e din corn de berbec, iar în ziua a ?aptea sa ocoli?i cetatea de ?apte ori ?i preo?ii sa trâmbi?eze din trâmbi?e.
\par 5 Când vor suna din trâmbi?a de corn de berbec ?i când ve?i auzi sunetul de trâmbi?a, atunci tot poporul sa strige cu glas tare deodata, ?i când vor striga ei zidurile ceta?ii se vor prabu?i. Atunci tot poporul sa navaleasca în cetate, plecând fiecare din partea unde se afla".
\par 6 ?i a chemat Iosua, fiul lui Navi, pe preo?i ?i le-a zis: "Lua?i chivotul legamântului ?i ?apte preo?i sa poarte ?apte trâmbi?e din corn de berbec înaintea chivotului Domnului".
\par 7 ?i le-a mai zis acestora sa spuna poporului: "Merge?i ?i ocoli?i cetatea, iar cei înarma?i sa mearga înaintea chivotului Domnului!"
\par 8 Îndata ce Iosua a spus acestea poporului, cei ?apte preo?i, care purtau cele ?apte trâmbi?e sfinte înaintea Domnului, au pornit ?i au început a suna tare din trâmbi?e ?i chivotul legamântului Domnului mergea în urma lor.
\par 9 Cei înarma?i mergeau înaintea preo?ilor care sunau din trâmbi?e, iar cei din urma veneau dupa chivotul Domnului, trâmbi?ând în mers din trâmbi?e.
\par 10 Poporului însa Iosua i-a poruncit ?i a zis: "Sa nu striga?i! Nimeni sa nu auda glasul vostru ?i nici o vorba sa nu iasa din gura voastra pâna în ziua când va voi zice eu sa striga?i; atunci sa striga?i".
\par 11 ?i a?a chivotul Domnului a plecat împrejurul ceta?ii ?i a înconjurat-o o data; apoi a venit în tabara ?i a ramas în tabara.
\par 12 A doua zi Iosua s-a sculat dis-de-diminea?a ?i preo?ii au ridicat chivotul Domnului;
\par 13 ?i cei ?apte preo?i, care purtau cele ?apte trâmbi?e, mergeau înaintea chivotului Domnului ?i trâmbi?au din trâmbi?e; cei înarma?i mergeau înaintea lor, iar celalalt popor venea în urma chivotului Domnului ?i preo?ii trâmbi?au din trâmbi?e.
\par 14 Astfel ?i a doua zi au înconjurat cetatea o data ?i s-au întors în tabara. ?i au facut a?a ?ase zile.
\par 15 În ziua a ?aptea s-au sculat de diminea?a, în revarsatul zorilor ?i au mers tot a?a împrejurul ceta?ii de ?apte ori; numai în aceasta zi au înconjurat cetatea de ?apte ori.
\par 16 Iar când au sunat preo?ii a ?aptea oara din trâmbi?e, Iosua a zis catre fiii lui Israel: "Striga?i, ca v-a dat Domnul cetatea!
\par 17 Cetatea va fi sub blestem ?i tot ce este în ea e al Domnului puterilor; numai Rahab desfrânata sa ramâna vie, ea ?i tot cel ce va fi în casa ei, pentru ca ea a ascuns iscoadele pe care le-am trimis noi.
\par 18 Voi însa sa va pazi?i foarte de tot ce este dat blestemului, ca sa nu cade?i ?i voi sub blestem, daca a?i lua ceva din cele date spre nimicire ?i pentru ca asupra taberei lui Israel sa nu vina blestemul ?i sa-i aduca pieire.
\par 19 Tot aurul ?i argintul ?i vasele de arama ?i de fier sa fie sfin?enie a Domnului ?i sa intre în vistieria Domnului".
\par 20 Atunci au trâmbi?at preo?ii din trâmbi?e. ?i cum a auzit poporul glasul de trâmbi?a, a strigat tot poporul împreuna cu glas tare ?i puternic ?i s-au prabu?it toate zidurile împrejurul ceta?ii pâna în temelie ?i a intrat tot poporul în cetate, fiecare din partea unde era, ?i au luat cetatea.
\par 21 ?i au dat junghierii tot ce era în cetate: barba?i ?i femei ?i tineri ?i batrâni ?i boi ?i oi ?i asini, tot au trecut prin ascu?i?ul sabiei.
\par 22 Iar celor doi tineri care iscodisera ?ara Iosua le-a zis: "Duce?i-va în casa desfrânatei aceleia, scoate?i-o de acolo pe ea ?i pe to?i cei ce vor fi cu dânsa, întrucât v-a?i jurat ei".
\par 23 ?i s-au dus tinerii care iscodisera cetatea, la casa desfrânatei ?i au scos pe Rahab desfrânata, pe tatal ei, pe mama ei, pe fra?ii ei ?i pe toate rudele ei, i-au scos ?i i-au pus afara din tabara Israeli?ilor,
\par 24 Iar cetatea ?i tot ce era în ea au ars cu foc; numai aurul ?i argintul ?i vasele de arama ?i de fier le-au dat ca sa le duca Domnului în vistieria casei Domnului.
\par 25 Pe Rahab desfrânata însa, casa tatalui ei ?i pe to?i care erau la dânsa, Iosua i-a lasat cu via?a ?i traie?te ea în mijlocul lui Israel pâna în ziua de astazi, pentru ca a ascuns iscoadele, care fusesera trimise de Iosua sa iscodeasca Ierihonul.
\par 26 În ziua aceea s-a jurat Iosua ?i a zis: "Blestemat sa fie înaintea Domnului tot cel ce se va scula ?i va zidi cetatea aceasta a Ierihonului: pe fiul sau cel întâi-nascut sa puna temeliile ei, iar por?ile ei sa le a?eze pe fiul sau cel mai mic". ?i a?a a ?i facut Ozan din Betel. Acesta pe Abiron, întâiul sau nascut, a pus temeliile ei, ?i pe fiul sau cel mai mic a a?ezat por?ile ei.
\par 27 Domnul însa era cu Iosua ?i faima lui s-a raspândit în toata ?ara.

\chapter{7}

\par 1 Iar fiii lui Israel au facut pacat mare: au luat din cele date nimicirii; caci Acan, fiul lui Carmi, fiul lui Zabdi, fiul lui Zerah, din tribul lui Iuda, a luat din cele date spre nimicire ?i s-a aprins mânia Domnului asupra fiilor lui Israel.
\par 2 Din Ierihon Iosua a trimis oameni asupra ceta?ii Ai, care e aproape de Bet-Aven, la rasarit de Betel, ?i le-a zis: "Duce?i-va de iscodi?i ?ara!" ?i s-au dus oamenii ace?tia ?i au iscodit cetatea Ai.
\par 3 Apoi întorcându-se la Iosua, i-au spus: "Sa nu mearga tot poporul, ci sa mearga numai doua sau trei mii de oameni ?i sa loveasca cetatea Ai. Nu obosi pâna acolo tot poporul, pentru ca acolo sunt pu?ini du?mani".
\par 4 ?i s-au dus acolo ca la trei mii de oameni, dar ace?tia au luat-o la fuga în fa?a celor din Ai.
\par 5 ?i locuitorii din Ai au ucis din ei vreo treizeci ?i ?ase de oameni ?i i-au fugarit de la poarta pâna la ?ebarim, iar pe coasta dealului i-a batut; din care pricina inima poporului a slabit ?i ?i-au pierdut tot curajul.
\par 6 Atunci Iosua ?i-a sfâ?iat ve?mintele sale, a cazut cu fa?a la pamânt înaintea chivotului Domnului ?i a stat a?a pâna seara ?i el ?i batrânii lui Israel ?i ?i-au presarat pulbere pe capetele lor.
\par 7 Atunci a zis Iosua: "O, Doamne Dumnezeule, de ce a trecut robul Tau poporul acesta peste Iordan? Oare, ca sa-l dai în mâinile Amoreilor ?i sa ne pierzi? Mai bine ramâneam sa locuim de cealalta parte de Iordan!
\par 8 Ce sa zic eu acum, când Israel a fugit înapoi dinaintea vrajma?ilor sai?
\par 9 Auzind Canaaneii ?i to?i locuitorii ?arii, ne vor împresura ?i ne vor pierde de pe pamânt. ?i ce vei face Tu pentru numele Tau cel mare?"
\par 10 Iar Domnul a zis catre Iosua: "Scoala, pentru ce ai cazut cu fa?a la pamânt?
\par 11 A pacatuit poporul ?i a calcat legamântul Meu pe care l-am încheiat cu el; au furat din lucrurile date spre nimicire, au min?it ?i le-au pus printre lucrurile lor.
\par 12 De aceea fiii lui Israel n-au putut sta împotriva vrajma?ilor lor ?i au fugit înapoi, caci au cazut sub blestem ?i nu voi mai fi cu voi de nu ve?i ridica blestemul dintre voi.
\par 13 Scoala dar ?i sfin?e?te poporul ?i zi: Sfin?i?i-va pentru mâine; caci a?a zice Domnul Dumnezeul lui Israel: Blestemul e în mijlocul tau, Israel. De aceea tu nu po?i sa stai împotriva vrajma?ilor tai pâna nu vei îndeparta din mijlocul tau blestemul.
\par 14 Mâine sa va apropia?i to?i, pe semin?ii; iar semin?ia pe care o va arata Domnul sa se apropie pe familii ?i familia pe care o va arata Domnul sa se apropie pe case; ?i casa pe care o va arata Domnul sa se apropie om cu om.
\par 15 Iar omul care se va dovedi ca a furat lucru dat spre nimicire, sa se arda cu foc ?i el ?i toate câte are, pentru ca a calcat legamântul Domnului ?i a facut faradelege în Israel".
\par 16 Atunci, sculându-se Iosua a doua zi dis-de-diminea?a, a poruncit lui Israel sa se apropie pe semin?ii; ?i a fost aratata semin?ia lui Iuda.
\par 17 Dupa aceea a poruncit sa se apropie semin?ia lui Iuda pe familii ?i a fost aratata familia lui Zerah. ?i a poruncit sa se apropie familia lui Zerah pe case ?i a fost data în vileag casa lui Zabdi.
\par 18 ?i a poruncit sa se apropie casa lui Zabdi om cu om ?i a fost aratat Acan, fiul lui Carmi, fiul lui Zabdi, fiul lui Zerah, din semin?ia lui Iuda.
\par 19 Atunci Iosua a zis catre Acan: "Fiul meu, da astazi slava Domnului Dumnezeului lui Israel ?i fa marturisire înaintea Lui ?i arata-mi ce-ai facut. Sa nu ascunzi de mine!"
\par 20 ?i raspunzând Acan lui Iosua, a zis: "Adevarat, am pacatuit înaintea Domnului Dumnezeului lui Israel ?i iata ce am facut:
\par 21 Am vazut printre prazi o haina frumoasa pestri?a ?i doua sute de sicli de argint, un drug de aur, greu de cincizeci de sicli; acestea mi-au placut ?i le-am luat ?i iata sunt ascunse în pamânt, în mijlocul cortului meu ?i argintul este pus sub elen.
\par 22 Atunci a trimis Iosua câ?iva slujitori ?i ace?tia au alergat la cort, în tabara, ?i toate acestea erau ascunse În cortul lui ?i argintul era sub ele.
\par 23 ?i au luat ei acestea din cort ?i le-au adus la Iosua ?i la batrânii lui Israel ?i le-au pus înaintea Domnului.
\par 24 Iar Iosua ?i împreuna cu el tot poporul au luat pe Acan, fiul lui Zerah, argintul, haina, drugul cel de aur, pe fiii lui, pe fiicele lui, boii lui, asinii lui ?i toate oile lui, cortul lui ?i tot ce avea el ?i i-au scos pe ei ?i toate ale lor în valea Acor.
\par 25 ?i a zis Iosua: "Pentru ca tu ai adus asupra noastra tulburare, sa aduca ?i Domnul necaz asupra ta în ziua aceasta!" ?i to?i Israeli?ii i-au ucis cu pietre ?i, dupa ce i-au ucis cu pietre, i-au ars cu foc.
\par 26 Apoi au ridicat deasupra lor o gramada mare de pietre, care se vede ?i astazi. Dupa aceasta s-a potolit mânia Domnului. De aceea locul acela se nume?te valea Acor pâna în ziua de astazi.

\chapter{8}

\par 1 Atunci a zis Domnul catre Iosua: "Nu te teme, nici nu te înspaimânta! Ia cu tine to?i barba?ii buni de lupta ?i, sculându-te, du-te la Ai. Iata Eu voi da în mâinile tale pe regele din Ai ?i pe poporul lui, cetatea ?i ?inutul lui.
\par 2 ?i sa faci cu Ai ?i cu regele sau tot ceea ce ai facut cu Ierihonul ?i cu regele lui. Numai prada lui ?i dobitoacele lui împar?i?i-le între voi. Pune oameni la pânda înapoia ceta?ii".
\par 3 ?i s-a sculat Iosua cu tot poporul bun de razboi, ca  sa mearga asupra ceta?ii Ai. ?i a ales Iosua treizeci de mii de oameni viteji ?i i-a trimis noaptea.
\par 4 Acestora le-a poruncit ?i le-a zis: "Lua?i seama sa pândi?i înapoia ceta?ii, sa nu va departa?i tare de cetate ?i sa fi?i cu to?ii gata.
\par 5 Iar eu ?i to?i cei cu mine vom înainta spre cetate. ?i când locuitorii din Ai ne vor ie?i înainte, ca prima data, noi vom fugi de dân?ii.
\par 6 ?i când vor alerga dupa noi ?i-i vom departa de cetate, vor zice: "Iata fug de noi, ca prima data".
\par 7 ?i când vom fugi noi de ei ?i ei dupa noi, atunci voi sa ie?i?i din ascunzatoare ?i sa cuprinde?i cetatea, ca Domnul Dumnezeu o va da în mâinile voastre.
\par 8 Iar dupa ce ve?i prada cetatea, sa-i da?i foc; dupa cuvântul Domnului sa face?i. Iata eu va poruncesc!"
\par 9 A?a i-a trimis Iosua ?i ei s-au dus ?i au stat la pânda între Betel ?i Ai, spre apus de cetatea Ai. Iar Iosua a ramas în noaptea aceea în mijlocul poporului.
\par 10 ?i, sculându-se dis-de-diminea?a, Iosua a cercetat poporul ?i s-a dus el ?i batrânii lui Israel în fruntea poporului spre Ai.
\par 11 ?i tot poporul bun de lupta, care era cu el, a mers, a înaintat ?i s-a apropiat de cetate pe partea de rasarit, iar pânda era în partea de apus a ceta?ii. ?i a a?ezat tabara spre miazanoapte de Ai; iar între el ?i Ai era o vale.
\par 12 El luase ca la treizeci de mii de oameni ?i-i pusese la pânda între Betel ?i Ai, în partea de apus a ceta?ii.
\par 13 Iar poporul l-a a?ezat tot într-o tabara, care se întindea în partea de miazanoapte a ceta?ii, a?a încât coada taberei ajungea spre partea de apus a ceta?ii. ?i a venit Iosua în noaptea aceea în mijlocul vaii.
\par 14 ?i când a vazut aceasta, regele din Ai s-a sculat îndata dis-de-diminea?a, a ie?it înaintea lui Israel la lupta, el ?i tot poporul lui, la un loc hotarât, în ?es, ?i nu ?tia el ca e o pânda asupra lui în spatele ceta?ii.
\par 15 Iar Iosua, ca ?i cum ar fi fost învins, a fugit cu tot poporul pe calea dinspre pustiu.
\par 16 Iar aceia au strigat tot poporul care era în cetate, ca sa-i urmareasca ?i, urmarind ei pe fiii lui Israel, s-au departat de cetate.
\par 17 ?i n-a ramas în Ai ?i în Betel nici unul care sa nu fi ie?it dupa Israel. ?i urmarind ei pe Israel, ?i-au lasat cetatea lor deschisa.
\par 18 Atunci Domnul a zis catre Iosua: "Întinde mâna ta cu suli?a asupra ceta?ii Ai, caci o voi da în mâinile tale ?i cei de la pânda vor ie?i numaidecât de la locul lor!" ?i ?i-a întins Iosua mâna cu suli?a spre cetate.
\par 19 ?i atunci cei ce ?edeau la pânda sculându-se numaidecât de la locul lor, când ?i-a întins el mâna, au alergat în cetate, au luat-o ?i i-au dat repede foc.
\par 20 Atunci, uitându-se locuitorii ceta?ii Ai înapoi ?i vazând fum ridicându-se din cetatea lor spre cer, nu mai aveau încotro fugi, caci poporul care fugea spre pustiu se întorsese împotriva celor ce-i urmareau;
\par 21 ?i Iosua ?i tot Israelul, vazând ca cei ce ?ezusera în ascunzatoare luasera cetatea ?i ca din cetate se urca fum spre cer, s-au întors înapoi ?i au început sa ucida pe locuitorii din Ai.
\par 22 Iar cei din cetate au ie?it în întâmpinarea lor, a?a ca Ai?ii se aflau în mijlocul taberei Israeli?ilor, dintre în care unii veneau dintr-o parte, iar al?ii din alta parte.
\par 23 ?i i-au ucis a?a, încât n-a scapat cu via?a nici unul din ei. Iar pe regele din Ai l-au prins viu ?i l-au adus la de Iosua.
\par 24 Dupa ce fiii lui Israel au ucis pe to?i barba?ii din Ai, în câmpia dinspre pustiu ?i dealurile unde ace?tia îi urmarisera, ?i dupa ce ace?tia au cazut to?i pâna la unul sub ascu?i?ul sabiei, Israeli?ii s-au întors cu to?ii asupra ceta?ii Ai ?i au lovit-o cu ascu?i?ul lui sabiei.
\par 25 Iar to?i locuitorii din Ai, barba?i ?i femei, care au cazut în ziua aceea au fost douasprezece mii.
\par 26 ?i Iosua nu ?i-a lasat în jos mâna sa, pe care o întinsese cu suli?a, pâna nu a dat pierzarii pe to?i locuitorii din Ai.
\par 27 Numai vitele ?i prada ceta?ii le-au împar?it fiii lui Israel între dân?ii, dupa cuvântul Domnului pe care-l spusese Domnul lui Iosua.
\par 28 ?i a ars Iosua cetatea Ai cu foc ?i a facut-o darâmatura ve?nica ?i pustietate pâna în ziua de astazi.
\par 29 Iar pe regele din Ai l-a spânzurat de un copac unde a stat pâna seara; iar dupa asfin?itul soarelui a poruncit Iosua de au coborât trupul lui din copac ?i l-au aruncat la por?ile ceta?ii ?i au ridicat deasupra lui o movila de pietre, care este pâna în ?i ziua de astazi.
\par 30 Atunci Iosua a înal?at jertfelnic Domnului Dumnezeului lui Israel pe Muntele Ebal,
\par 31 Cum poruncise Moise, sluga Domnului, fiilor lui Israel, ?i de care este scris în legea lui Moise; jertfelnicul era din pietre întregi, asupra carora nimeni n-a ridicat unealta de fier ?i el au adus pe el ardere de tot Domnului ?i au facut jertfe de împacare.
\par 32 ?i a scris Iosua acolo din nou pe pietre legea pe care Moise o scrisese înaintea fiilor lui Israel.
\par 33 ?i tot Israelul cu batrânii lui, cu capeteniile lui, cu judecatorii  lui, to?i ba?tina?ii ?i veneticii lui au stat de o parte ?i de alta a chivotului în fa?a preo?ilor ?i a levi?ilor care duceau chivotul legamântului Domnului, a?a ca jumatate din ei se aflau pe muntele Garizim, iar cealalta jumatate pe muntele Ebal, cum poruncise de mai înainte Moise, sluga Domnului, ca sa se binecuvânteze poporul lui Israel.
\par 34 Dupa aceasta a citit Iosua toate cuvintele legii: binecuvântarile ?i blestemul, cum era scris în legea lui Moise.
\par 35 Din toate câte poruncise Moise lui Iosua n-a fost nici un cuvânt pe care Iosua sa nu-l fi citit în auzul ob?tii fiilor lui Israel, înaintea barba?ilor, a femeilor, a copiilor ?i a strainilor care mergeau împreuna cu Israel.

\chapter{9}

\par 1 Auzind acestea to?i regii Amoreilor, cei de peste Iordan, cei din munte ?i de la ?es, cei de pe tot malul marii celei mari ?i cei din apropierea Libanului: Heteii, Amoreii, Ghergheseii, Canaaneii, Ferezeii, Heveii ?i Ibuseii,
\par 2 S-au adunat împreuna ca sa se lupte to?i cu Iosua ?i cu Israel.
\par 3 Iar locuitorii Ghibeonului, auzind ce a facut Iosua cu Ierihonul ?i cu Ai,
\par 4 Au pus la cale un vicle?ug, caci s-au dus ?i au strâns merinde de drum ?i au pus pe asinii lor saci vechi cu pâine ?i vin în burdufuri vechi, rupte ?i cârpite
\par 5 ?i în picioarele lor încal?aminte ?i sandale vechi ?i peticite, iar pe ei haine rele; pâinea lor de drum era uscata, muceda ?i sfarâmata.
\par 6 A?a au venit ei la Iosua în tabara Israeli?ilor de la Ghilgal ?i au zis catre el ?i catre to?i Israeli?ii: "Noi am venit dintr-o ?ara foarte departata; încheia?i dar legamânt cu noi".
\par 7 Fiii lui Israel însa au zis catre Hevei: "Poate ca locui?i aproape de noi? Cum sa încheiem legamânt cu voi?"
\par 8 Iar ei au zis catre Iosua: "Noi suntem robii tai". Iosua le-a zis: "Cine sunte?i voi ?i de unde a?i venit?"
\par 9 ?i ei au raspuns: "Robii tai au venit dintr-o ?ara foarte departata, în numele Domnului Dumnezeului tau, ca am auzit de numele Lui ?i de câte a facut El în Egipt
\par 10 ?i de câte a facut El celor doi regi ai Amoreilor, care erau dincolo de Iordan, lui Sihon, regele He?bonului ?i lui Og, regele Vasanului, care locuia în A?tarot ?i Edrea.
\par 11 ?i auzind acestea, batrânii no?tri ?i to?i locuitorii ?arii noastre ne-au grait ?i au zis: Lua?i-va merinde de drum ?i duce?i-va în întâmpinarea lor ?i le spune?i: Noi suntem robii vo?tri. Încheia?i dar legamânt cu noi!
\par 12 Pâinile acestea erau calde în ziua când am plecat sa venim la voi, iar acum iata s-au uscat ?i s-au mucezit.
\par 13 Aceste burdufuri de vin, pe care le-am umplut noi, iata-le s-au învechit; ?i îmbracamintea noastra ?i încal?amintea noastra s-au tocit de drumul cel foarte lung".
\par 14 Capeteniile Israeli?ilor au luat din merindele lor, dar n-au întrebat pe Domnul.
\par 15 ?i a facut Iosua pace cu ei ?i a încheiat cu ei legamânt, ca sa nu fie uci?i, iar capeteniile ob?tii s-au legat fa?a de ei cu juramânt.
\par 16 La trei zile însa, dupa ce au încheiat legamânt cu dân?ii, au auzit ca sunt aproape de ei ?i ca traiesc în ?inuturile cuvenite lor,
\par 17 Caci fiii lui Israel, plecând la drum, a treia zi au ajuns la ceta?ile lor; iar ceta?ile lor erau Ghibeonul, Chefira, Beerot ?i Chiriat-Iearim.
\par 18 Iar Iosua ?i fiii lui Israel nu i-au ucis, pentru ca toate capeteniile ob?tii li se jurasera pe Domnul Dumnezeul lui Israel. De aceea toata ob?tea lui Israel a început a cârti împotriva capeteniilor.
\par 19 ?i toate capeteniile au zis catre întreaga ob?te: "Noi ne-am jurat lor pe Domnul Dumnezeul lui Israel ?i de aceea nu ne putem atinge de ei.
\par 20 Dar iata ce le vom face: sa-i robim ?i sa-i pastram în via?a, ca sa nu ne ajunga mânia pentru juramântul cu care ne-am jurat lor".
\par 21 ?i le-au mai zis capeteniile: "Lasa?i-i sa traiasca, dar sa taie lemne ?i sa care apa la toata ob?tea". ?i toata ob?tea a facut cum au zis capeteniile.
\par 22 ?i i-a chemat Iosua ?i le-a zis: "Pentru ce m-a?i amagit, spunând: Suntem tare departe de tine; voi însa sunte?i dintre cei ce locuiesc în ?inuturile cuvenite noua.
\par 23 De aceea blestema?i sa fi?i! Sa nu înceta?i de a fi în robie, ca taietori de lemne ?i caratori de apa pentru casa Dumnezeului meu".
\par 24 Iar ei au raspuns lui Iosua ?i au zis: "Ni s-a vestit noua câte a poruncit Domnul Dumnezeul tau lui Moise, slugii Sale, ca sa va dea toata ?ara aceasta, iar pe noi ?i pe to?i locuitorii pamântului acestuia sa ne piarda de la fa?a voastra; ?i, înfrico?ându-ne foarte tare de voi pentru via?a noastra, am facut fapta aceasta.
\par 25 ?i acum iata suntem sub mâna voastra; face?i cu noi cum ve?i socoti ?i cum vi se pare mai drept".
\par 26 ?i a facut cu ei a?a: i-a izbavit Iosua în ziua aceea din mâinile fiilor lui Israel ?i nu i-a ucis.
\par 27 Dar din ziua aceea i-a pus Iosua taietori de lemne ?i caratori de apa pentru ob?te ?i pentru jertfelnicul Domnului. De aceea locuitorii Ghibeonului s-au facut taietori de lemne ?i caratori de apa pentru jertfelnicul lui Dumnezeu pâna în ziua de astazi, la locul pe care avea sa-l aleaga Domnul.

\chapter{10}

\par 1 Auzind însa Adoni-?edec, regele Ierusalimului, ca Iosua a luat cetatea Ai ?i a dat-o blestemului ?i a facut cu Ai ?i cu regele sau cum facuse cu Ierihonul ?i cu regele lui ?i ca locuitorii Ghibeonului de bunavoie s-au supus lui Iosua ?i lui Israel ?i au ramas între ei,
\par 2 S-a spaimântat foarte tare, pentru ca cetatea Ghibeonului era cetate mare, ca una ce era dintre ceta?ile domne?ti, mai mare decât Ai, ?i to?i locuitorii ei erau oameni viteji.
\par 3 De aceea Adoni-?edec, regele Ierusalimului, a trimis la Hoham, regele Hebronului, la Piream, regele Iarmutului, la Iafia, regele Lachi?ului, ?i la Debir, regele Eglonului, zicând:
\par 4 "Veni?i la mine ?i-mi ajuta?i sa bat Ghibeonul, pentru ca s-a supus lui Iosua ?i fiilor lui Israel".
\par 5 ?i ace?ti cinci regi ai Amoreilor: regele Ierusalimului, regele Hebronului, regele Iarmutului, regele Lachi?ului ?i regele Eglonului, s-au suit cu tot poporul lor ?i au tabarât asupra Ghibeonului ?i l-au împresurat.
\par 6 Atunci locuitorii Ghibeonului au trimis la Iosua, în tabara Israeli?ilor de la Ghilgal, zicând: "Sa nu-?i iei mâna de deasupra robilor tai! Vino la noi repede de ne da ajutor ?i ne izbave?te, ca s-au adunat împotriva noastra to?i regii Amoreilor care traiesc în mun?i".
\par 7 ?i s-a suit Iosua din Ghilgal, el însu?i ?i împreuna cu dânsul tot poporul bun de razboi ?i to?i barba?ii viteji.
\par 8 Iar Domnul a zis catre Iosua: "Nu te teme de ei, ca i-am dat în mâinile tale; nimeni dintre ei nu va putea sta împotriva voastra".
\par 9 ?i a navalit Iosua fara de veste asupra lor, dupa ce mersese toata noaptea venind din Ghilgal.
\par 10 ?i Domnul i-a facut sa se sperie la vederea fiilor lui Israel ?i i-a învins pe ei Domnul cu înfrângere grea la Ghibeon, ?i i-a urmarit în calea spre înal?imile Bet-Horon ?i i-a batut pâna la Azeca ?i pâna la Macheda.
\par 11 Pe când însa fugeau ei de fiii lui Israel, pe povârni?ul Bet-Horon, Domnul a aruncat din cer asupra lor grindina mare, pâna la Azeca, ?i cei ce au murit de grindina au fost mai mul?i decât cei uci?i de fiii lui Israel cu sabia în lupta.
\par 12 În ziua aceea în care Dumnezeu a dat pe Amorei în mâinile lui Israel ?i când i-a batut la Ghibeon ?i au fost zdrobi?i înaintea fe?ei fiilor lui Israel, a strigat Iosua catre Domnul ?i a zis înaintea Israeli?ilor: "Stai, soare, deasupra Ghibeonului, ?i tu, luna, opre?te-te deasupra vaii Aialon!"
\par 13 ?i s-a oprit soarele ?i luna a stat pâna ce Dumnezeu a facut izbânda asupra vrajma?ilor lor. Oare nu de aceea se scrie în Cartea Dreptului: "Soarele a stat în mijlocul cerului ?i nu s-a grabit catre asfin?it aproape toata ziua".
\par 14 ?i n-a mai fost nici înainte, nici dupa aceea, o astfel de zi în care Domnul sa asculte a?a glasul omului; ca Domnul lupta pentru Israel.
\par 15 Apoi Iosua s-a întors cu tot Israelul în tabara, la Ghilgal.
\par 16 Iar cei cinci regi au fugit ?i s-au ascuns în pe?tera din Macheda.
\par 17 Când însa i s-a spus lui Iosua ca: "Cei cinci regi au fost gasi?i ascun?i în pe?tera, la Macheda", Iosua a zis:
\par 18 "Rasturna?i pietre mari pe gura pe?terii ?i pune?i la u?a ei oameni ca sa-i pazeasca;
\par 19 Iar voi nu va opri?i acolo, ci goni?i din urma pe vrajma?ii vo?tri ?i nimici?i partea din urma a o?tirii lor; sa nu-i lasa?i sa scape în ceta?ile lor, caci Domnul Dumnezeul vostru i-a dat în mâinile voastre!"
\par 20 ?i dupa ce Iosua ?i fiii lui Israel i-au zdrobit cu desavâr?ire într-o batalie foarte mare ?i cei ce au scapat au fugit în ceta?i întarite,
\par 21 S-a întors tot poporul cu izbânda la Iosua, în tabara la Macheda, ?i n-a rostit nimeni nici un cuvânt împotriva fiilor lui Israel.
\par 22 Atunci Iosua a zis: "Deschide?i pe?tera ?i scoate?i la mine pe cei cinci regi din pe?tera!"
\par 23 ?i s-a facut a?a: au scos la dânsul pe cei cinci regi din pe?tera: pe, regele Ierusalimului, pe regele Hebronului, pe regele Iarmutului, pe regele Lachi?ului ?i pe regele Eglonului.
\par 24 ?i dupa ce au scos pe regii ace?tia la Iosua, Iosua a chemat pe to?i Israeli?ii ?i a zis catre capeteniile o?tenilor care fusesera cu el: "Apropia?i-va ?i pune?i-va picioarele pe grumajii regilor acestora!" ?i ei s-au apropiat ?i ?i-au pus picioarele pe grumajii lor.
\par 25 Atunci Iosua le-a zis: "Nu va teme?i, nici nu va spaimânta?i, ci îmbarbata?i-va ?i va întari?i, ca a?a va face Domnul cu to?i vrajma?ii vo?tri cu care va ve?i lupta".
\par 26 Dupa aceea i-a lovit Iosua ?i i-a ucis ?i i-a spânzurat pe cinci spânzuratori ?i au stat spânzura?i pâna seara.
\par 27 Iar la asfin?itul soarelui a poruncit Iosua de i-au pogorât din spânzuratori ?i i-au aruncat în pe?tera în care fusesera ascun?i ?i au pravalit pietre mari pe gura pe?terii ?i stau acolo pâna în ziua de azi.
\par 28 Tot în ziua aceea a cuprins Iosua Macheda ?i a lovit-o cu sabia, pe ea ?i pe regele ei, ?i a nimicit pe locuitorii ei ?i toata suflarea care se afla în ea; pe nimeni n-a cru?at ca sa nu moara sau sa fuga ?i a facut cu regele din Macheda tot a?a cum facuse cu regele Ierihonului.
\par 29 Apoi s-a dus Iosua împreuna cu to?i Israeli?ii de la Macheda la Libna ?i a împresurat-o.
\par 30 ?i a dat-o Domnul ?i pe aceasta în mâinile lui Israel de a luat-o pe ea ?i pe regele ei ?i Iosua a trecut-o prin ascu?i?ul sabiei pe ea ?i toata suflarea din ea; pe nimeni n-a lasat în ea, care sa nu moara sau sa fuga ?i a facut cu regele ei cum facuse cu regele Ierihonului.
\par 31 De la Libna s-a dus Iosua împreuna cu to?i Israeli?ii la Lachi? ?i l-a împresurat, începând razboiul cu ei.
\par 32 ?i a dat Domnul Lachi?ul în mâinile lui Israel ?i l-a luat a doua zi, l-a trecut prin sabie pe el ?i toata suflarea din el ?i l-a pierdut, cum facuse ?i cu Libna.
\par 33 Atunci s-a ridicat în ajutorul Lachi?ului Horam, regele din Ghezer. Dar Iosua l-a lovit ?i pe el ?i pe poporul lui cu sabia, încât n-a lasat pe nimeni dintre ai lui care sa nu moara sau sa fuga.
\par 34 Din Lachi? s-a dus Iosua împreuna cu to?i Israeli?ii la Eglon ?i l-a împresurat, începând razboiul împotriva lui.
\par 35 ?i Domnul l-a dat în mâinile lui Israel. ?i l-a luat în aceea?i zi ?i l-a lovit cu sabia pe el ?i toata suflarea ce era în el ?i l-a dat în ziua aceea blestemului, cum facuse ?i cu Lachi?ul.
\par 36 De la Eglon, Iosua împreuna cu to?i Israeli?ii s-au dus la Hebron, l-au înconjurat
\par 37 ?i l-au lovit cu ascu?i?ul sabiei pe el ?i pe regele lui ?i toate ceta?ile lui ?i toata suflarea câta era în acesta, cum facuse cu Eglonul; ?i l-a dat blestemului pe el ?i toata suflarea ce era în el.
\par 38 Dupa aceea s-a întors Iosua împreuna cu to?i Israeli?ii asupra Debirului ?i l-a împresurat.
\par 39 ?i l-a luat pe el, pe regele lui ?i toate satele lui. ?i le-au lovit cu ascu?i?ul sabiei, le-au nimicit pe ele ?i toata suflarea ce era în ele ?i n-a lasat pe nimeni sa scape; cum facuse cu Hebronul ?i cu regele lui ?i cum a facut cu Libna ?i regele ei, a?a a facut ?i cu Debirul ?i cu regele lui.
\par 40 Apoi a lovit Iosua tot ?inutul muntos, Neghebul, câmpia ?i ?inutul marii ?i pe regii lor ?i n-a lasat sa scape nimeni, ci a omorât toata suflarea, cum poruncise Domnul Dumnezeul lui Israel.
\par 41 ?i i-a batut Iosua de la Cade?-Barnea pâna la Gaza ?i tot ?inutul Go?en pâna la Ghibeon.
\par 42 ?i pe to?i regii acestora ?i ?inuturile lor le-a luat Iosua deodata, caci Domnul Dumnezeul lui Israel lupta pentru Israel.
\par 43 Dupa aceea Iosua împreuna cu to?i Israeli?ii s-au întors în tabara de la Ghilgal.

\chapter{11}

\par 1 Cum a auzit de aceasta, Iabin, regele Ha?orului, a trimis la Iobab, regele Madonului, la regele ?imronului ?i la regele Ac?afului;
\par 2 La regii cei de la miazanoapte; din mun?i, de pe podi?, de la miazazi de Chinerot, în câmpie ?i în ?inutul Dor, la apus;
\par 3 La Canaaneii din rasarit, de pe malul marii; la Amorei, la Hetei, la Ferezei, la Iebuseii din mun?i ?i la Heveii de sub Hermon, în pamântul Mi?pa.
\par 4 ?i au ie?it ace?tia împreuna cu regii lor, popor mult la numar, ca nisipul de la marginea marii, ?i cai ?i care de razboi foarte multe.
\par 5 ?i s-au adunat to?i regii ace?tia ?i au tabarât împreuna la apele Merom, ca sa se lupte cu Israel.
\par 6 Domnul însa a zis: "Nu te teme de fa?a lor, caci mâine pe vremea aceasta îi voi da pe to?i aceia fiilor lui Israel ca sa-i ucida; cailor lor sa le tai vinele picioarelor, iar carele sa le arzi cu foc".
\par 7 Atunci Iosua împreuna cu tot poporul bun de lupta a ie?it fara de veste înaintea lor, la apele Merom ?i a navalit în mun?i asupra lor.
\par 8 ?i i-a dat pe ei Domnul în mâinile lui Israel ?i i-a lovit ?i i-a urmarit pâna la Sidonul cel Mare, pâna la Misrefot-Maim ?i pâna în valea Mi?pa, la rasarit, ?i i-a ucis pâna n-a scapat nimeni.
\par 9 ?i a facut Iosua cu ei cum îi zisese Domnul: cailor lor le-a taiat vinele picioarelor ?i carele lor le-a ars cu foc.
\par 10 Tot atunci întorcându-se, Iosua a luat Ha?orul ?i pe regele lui l-a omorât cu sabia. Ha?orul fusese pâna atunci capul tuturor regatelor acestora.
\par 11 ?i a ucis toata suflarea din acesta cu sabia, dând toate pieirii; ?i n-a ramas nici un suflet, iar Ha?orul l-a ars cu foc.
\par 12 Astfel a luat Iosua toate ceta?ile regatelor acestora ?i pe to?i regii lor i-a ucis cu sabia, dându-i pieirii, cum poruncise Moise, sluga Domnului.
\par 13 Iar ceta?ile cele întarite nu le-a ars Israel, afara de Ha?or, pe care l-a ars Iosua.
\par 14 Toata prada ceta?ilor acestora ?i toate vitele le-au luat fiii lui Israel pentru ei, iar pe oameni i-au ucis cu sabia, i-au dat blestemului ?i n-au lasat din ei nici un suflet.
\par 15 Precum poruncise Domnul lui Moise, sluga Sa, ?i cum poruncise ?i Moise lui Iosua, a?a a facut Iosua: nu s-a abatut de la nimic, nici de la un cuvânt din câte poruncise Domnul lui Moise.
\par 16 ?i a?a a luat Iosua toata ?ara aceea de sus ?i tot ?inutul Negheb, tot ?inutul Go?en ?i ?inuturile de jos, ?esul ?i muntele lui Israel ?i locurile joase de pe lânga munte,
\par 17 De la Muntele Pele, care se întinde spre Seir, pâna la Baal-Gad din valea Libanului, la poalele Hermonului; a luat pe to?i regii lor ?i i-a lovit ?i i-a ucis.
\par 18 Dar Iosua a purtat multa vreme razboi cu to?i regii ace?tia.
\par 19 ?i n-a fost cetate pe care sa n-o fi luat cu fiii lui Israel; toate le-au luat cu razboi, afara de Ghibeon în care locuiau Heveii;
\par 20 Caci a?a a fost de la Domnul, ca sa-?i învârto?eze inima lor ?i sa întâmpine pe Israel cu razboi, ca sa fie da?i pieirii ?i ca sa nu gaseasca mila, ci sa fie nimici?i, cum poruncise Domnul lui Moise.
\par 21 Apoi a venit Iosua în vremea aceea ?i a lovit pe to?i Anachimii de la munte, din Hebron, din Debir, din Anab, din to?i mun?ii lui Iuda ?i din to?i mun?ii lui Israel ?i i-a dat Iosua nimicirii împreuna cu ceta?ile lor.
\par 22 ?i n-a ramas niciunul din Anachimi în pamântul fiilor lui Israel, ci numai în Gaza, în Gat ?i în A?dod au ramas din ei.
\par 23 A?a a luat Iosua tot pamântul, cum poruncise Domnul lui Moise ?i l-a dat Iosua de mo?tenire lui Israel, împar?indu-l între semin?iile lor. ?i s-a lini?tit pamântul de razboi.

\chapter{12}

\par 1 Iata regii pe care i-au batut fiii lui Israel ?i al caror pamânt l-au luat tot de mo?tenire dincolo de Iordan, spre rasaritul soarelui, de la râul Arnon pâna la muntele Hermon ?i tot ?esul dinspre rasarit:
\par 2 Sihon, regele Amoreilor, care locuia în He?bon ?i domnea de la Aroerul cel de pe malul râului Arnon, de la mijlocul râului ?i peste jumatate din Galaad, pâna la râul Iaboc, care este hotarul Amoni?ilor,
\par 3 Peste ?es pâna lânga marea Chineretului, spre rasarit ?i pâna la marea ?esului, pâna la Marea Sarata, spre rasarit, pe calea catre Bet-Ie?imot, iar spre miazazi peste locurile ce se întindeau pe la poalele muntelui Fazga;
\par 4 Vecinul sau Og, regele Vasanului, cel din urma din Refaim, care locuia în A?tarot ?i Edrea
\par 5 ?i care stapânea muntele Hermon ?i Salca ?i tot Vasanul, pâna la hotarul Ghe?urului ?i al Maacului ?i jumatate din Galaad, pâna la hotarul lui Sihon, regele He?bonului.
\par 6 Pe ace?tia Moise, sluga Domnului, ?i fiii lui Israel i-au ucis ?i a dat Moise, sluga Domnului, pamântul lor de mo?tenire semin?iilor lui Ruben ?i Gad ?i la jumatate din semin?ia lui Manase.
\par 7 Iata acum ?i regii din ?ara Amoreilor, pe care i-a batut Iosua ?i fiii lui Israel dincoace de Iordan, spre apus de la Baal-Gad, din valea Libanului, pâna la Pele, muntele care se întinde spre Seir; ?i pamântul l-a dat Iosua semin?iilor lui Israel de mo?tenire, dupa cum le-au cazut sor?ii,
\par 8 La munte sau la loc ?es, la câmpie sau la locurile de sub mun?i, în pustiu ?i la miazazi ?i care fusese al Heteilor, Amoreilor, Canaaneilor, Ferezeilor, Heveilor ?i Iebuseilor:
\par 9 Un rege al Ierihonului, un rege al ceta?ii Ai, care e aproape de Betel;
\par 10 Un rege al Ierusalimului, un rege al Hebronului;
\par 11 Un rege al Iarmutului, un rege al Lachi?ului;
\par 12 Un rege al Eglonului, un rege al Ghezerului;
\par 13 Un rege al Debirului, un rege al Ghederului;
\par 14 Un rege al Hormei, un rege al Aradului;
\par 15 Un rege al Libnei, un rege al Adulamului;
\par 16 Un rege al Machedei; un rege al Betelului;
\par 17 Un rege al Tapuahului, un rege al Heferului;
\par 18 Un rege al Afecului, un rege al ?aronului;
\par 19 Un rege al Madonului, un rege al Ha?orului;
\par 20 Un rege al ?imron-Meronului, un rege al Ac?afului;
\par 21 Un rege al Taanacului, un rege al Meghidonului;
\par 22 Un rege al Chede?ului, un rege al Iocneamului de lânga Carmel;
\par 23 Un rege al Dorului de lânga Nafat-Dor, un rege al Goimului din Ghilgal;
\par 24 Un rege al Tir?ei. De to?i treizeci ?i unu de regi.

\chapter{13}

\par 1 Fiind Iosua batrân ?i înaintat în zile, a zis Domnul catre dânsul: "Iata tu ai îmbatrânit ?i e?ti în vârsta înaintata, dar pamânt de luat în mo?tenire a mai ramas înca mult.
\par 2 Pamântul care a mai ramas de luat în mo?tenire este acesta: ?inuturile Filistenilor ?i toata ?ara Ghe?urului ?i a Canaanului.
\par 3 De la Sicor, care este la rasarit de Egipt, pâna la hotarele Ecronului, la miazanoapte, se socotesc cele cinci capetenii filistene: Gaza, A?dod, Ascalonul, Gat ?i Ecronul.
\par 4 Apoi ?ara Heveilor: tot pamântul Canaan ?i Maara Sidonienilor, de la Teman pâna la Afec ?i pâna la hotarele Amoreilor.
\par 5 De asemenea ?inutul filistean Ghebla ?i tot Libanul, spre rasaritul soarelui, de la Baal-Gad, de la poalele muntelui Hermon pâna la intrarea Hamatului.
\par 6 Pe to?i locuitorii muntelui de la Liban pâna la Misrefot-Maim, pe to?i Sidonienii sa-i pierzi de la fa?a fiilor lui Israel  pamântul lor sa-l împar?i lui Israel prin sorti, cum ti-am poruncit.
\par 7 A?adar la cele noua semin?ii ?i la jumatate din semin?ia lui Manase, împarte-le mo?tenire prin sor?i pamântul acesta: de la Iordan pâna la marea cea mare de la apus sa-l dai lor ?i marea cea mare sa fie hotar.
\par 8 Caci celelalte doua semin?ii: a lui Ruben ?i Gad ?i jumatate din semin?ia lui Manase au primit partea de la Moise, dincolo de Iordan, spre rasaritul soarelui, cum le-a dat Moise sluga Domnului,
\par 9 De la Aroer, care e pe malul râului Arnon, cetatea cea din mijlocul vaii ?i toata câmpia Medeba pâna la Dibon,
\par 10 Precum ?i toate ceta?ile lui Sihon, regele Amoreilor, care a domnit în He?bon, pâna la hotarele fiilor lui Amon,
\par 11 ?i Galaadul, ?inutul Ghe?ur ?i al Maacatienilor, tot muntele Hermon ?i tot Vasanul pâna la Salca;
\par 12 Tot regatul lui Og al Vasanului, care a domnit în A?tarot ?i în Edrea. Acesta mai ramasese din Refaimi, pe care Moise i-a batut ?i i-a alungat".
\par 13 Dar fiii lui Israel n-au vrut sa piarda pe Ghe?ureni ?i pe Maacatieni ?i pâna în ziua de astazi locuie?te regele din Ghe?ur ?i al Maacatienilor în mijlocul lui Israel.
\par 14 Numai semin?iei lui Levi nu i-a dat Iosua mo?tenire, caci jertfele ?i prinoasele Domnului Dumnezeului lui Israel sunt partea ei, cum i-a zis Domnul.
\par 15 Iata împar?irea pe care a facut-o Moise fiilor lui Israel, dupa semin?iile lor, în ?esurile Moabului, dincolo de Iordan în fa?a Ierihonului: Semin?iei fiilor lui Ruben, dupa familiile ei, Moise i-a dat parte:
\par 16 Hotarele ei cuprindeau cetatea Aroer, care se afla pe malul râului Arnon, la mijlocul cursului acestui râu ?i tot ?esul de lânga Medeba;
\par 17 He?bonul ?i toate ceta?ile lui cele din ?es; Dibonul, Bamot-Baal ?i Bet-Baal-Meon;
\par 18 Iah?a, Chedemot ?i Mefaat;
\par 19 Chiriataim, Sibma ?i ?eret-Ha?ahar, în muntele Emec;
\par 20 Bet-Peor, locurile de la poalele muntelui Fazga ?i Bet-Ie?imot;
\par 21 Toate ceta?ile din ?es ?i toata împara?ia lui Sihon, regele Amoreilor, care a domnit la He?bon ?i pe care l-a ucis Moise, ca ?i pe capeteniile lui Madiam, pe Evi, Rechem, ?ur, Hur ?i Reba, capeteniile lui Sihon, care locuiau în ?ara aceea;
\par 22 De asemenea ?i pe Valaam, fiul lui Beor, vrajitorul, l-au ucis fiii lui Israel cu sabia, în numarul celor uci?i de ei.
\par 23 Hotarul fiilor lui Ruben era Iordanul. Aceasta e partea fiilor lui Ruben dupa familiile lor, dupa ceta?ile ?i satele lor.
\par 24 De asemenea a dat Moise parte semin?iei lui Gad, fiilor lui Gad, dupa familiile lor.
\par 25 În hotarele lor se cuprindea Iezerul ?i toate ceta?ile Galaadului ?i jumatate din ?ara fiilor lui Amon,
\par 26 Pâna la Aroer, care e în fa?a ceta?ii Raba ?i ?ara de la He?bon pâna la Ramat-Mi?pa ?i Betonim ?i de la Mahanaim pâna la hotarele Debirului;
\par 27 În vale i-a dat Bet-Haram, Bet-Nimra, Sucot ?i ?afon, rama?i?a regatului lui Sihon, regele He?bonului. Hotarul lui era Iordanul pâna la marea Chineret, întinzându-se de la Iordan spre rasarit.
\par 28 Aceasta era partea fiilor lui Gad dupa familiile lor, ceta?ile ?i satele lor.
\par 29 ?i a mai dat Moise parte ?i la jumatate din semin?ia lui Manase, adica la jumatate din semin?ia fiilor lui Manase, dupa familiile lor.
\par 30 În hotarele lor se cuprindea tot Vasanul de la Mahanaim, toata împara?ia lui Og, regele Vasanului, ?i toate sala?urile Iairului celui din Vasan, ?aizeci de ceta?i.
\par 31 Iar jumatate din Galaad cu A?tarotul ?i Edrea, ora?ele nepotului lui Og al Vasanului, au fost date fiilor lui Machir, fiul lui Manase, la jumatate din fiii lui Machir dupa familiile lor.
\par 32 Iata ce a dat Moise ca parte de mo?tenire în ?esul Moabului, peste Iordan, în fa?a Ierihonului spre rasarit.
\par 33 Dar semin?iei lui Levi, Moise nu i-a dat parte, ca Însu?i Domnul Dumnezeul lui Israel este partea lor, cum le-a grait El.

\chapter{14}

\par 1 Iata ce mo?tenire au primit fiii lui Israel în ?ara Canaan, pe care le-au împar?it-o prin sor?i preotul Eleazar ?i Iosua, fiul lui Navi, ?i capeteniile de familii ale semin?iilor fiilor lui Israel.
\par 2 Mo?tenirea aceasta au împar?it-o prin sor?i cum poruncise Domnul prin Moise, la cele noua semin?ii ?i la jumatate din semin?ia lui Manase;
\par 3 Caci la doua semin?ii ?i la jumatate din semin?ia lui Manase le daduse Moise parte peste Iordan. Iar levi?ilor nu le-a dat mo?tenire între ei;
\par 4 Caci din fiii lui Iosif se ridicasera doua semin?ii: a lui Manase ?i a lui Efraim; de aceea nu s-a dat levi?ilor parte de pamânt, ci numai ceta?i pentru locuit cu împrejurimile lor, pentru vitele ?i pentru turmele lor.
\par 5 Cum poruncise Domnul prin Moise, a?a au ?i facut fiii lui Israel, când au împar?it ?ara.
\par 6 Atunci au venit fiii lui Iuda la Iosua în Ghilgal ?i a zis catre el Caleb, fiul lui Iefone Chenezeul: "Tu ?tii ce a zis Domnul catre Moise, omul lui Dumnezeu, pentru mine ?i pentru tine, la Cade?-Barnea.
\par 7 Eu eram de patruzeci de ani, când Moise, sluga lui Dumnezeu, m-a trimis din Cade?-Barnea sa iscodesc ?ara ?i eu i-am adus raspuns dupa dorin?a lui.
\par 8 Fra?ii mei, care fusesera cu mine, au umplut de spaima inima poporului, iar eu am urmat hotarât Domnului Dumnezeului meu.
\par 9 ?i s-a jurat Moise în ziua aceea ?i a zis: "Pamântul pe unde a calcat piciorul tau va fi mo?tenirea ta ?i a copiilor tai pe veci, ca tu ai urmat hotarât Domnului Dumnezeului nostru".
\par 10 ?i iata Domnul m-a ?inut viu cum a zis. Au trecut acum patruzeci ?i cinei de ani de când a spus Domnul lui Moise cuvântul acesta ?i Israel umbla prin pustiu; ?i iata acum am optzeci ?i cinci de ani
\par 11 ?i sunt înca tare, ca ?i atunci când m-a trimis Moise; câta putere aveam atunci, tot atâta am ?i acum, ca sa ies ?i sa intru în lupta.
\par 12 A?adar î?i cer muntele acesta, de care a vorbit Domnul în ziua aceea; caci tu ai auzit în ziua aceea cuvântul acesta acolo. Acum acolo sunt fiii lui Enac, care au ceta?i mari ?i tari, dar de va fi Domnul cu mine, îi voi izgoni, cum mi-a zis Domnul".
\par 13 ?i l-a binecuvântat Iosua ?i a dat Hebronul mo?tenire lui Caleb, fiul lui Iefone Chenezeul.
\par 14 De aceea a ajuns Hebronul mo?ia lui Caleb, fiul lui Iefone, pâna în ziua de azi, pentru ca el a urmat întocmai porunca Domnului Dumnezeului lui Israel.
\par 15 Mai înainte Hebronul se numea Chiriat-Arba, capitala fiilor lui Enac. Dupa aceea s-a lini?tit ?ara de razboi.

\chapter{15}

\par 1 Mo?ia cazuta la sor?i pentru semin?ia fiilor lui Iuda, dupa familiile lor, se întindea de la hotarele lui Edom din miazazi ?i de la pustiul Sin pâna la Cade?, spre rasarit.
\par 2 Astfel hotarul lor de miazazi porne?te de la Marea Sarata, de unde pleaca un golf al ei spre miazazi,
\par 3 Merge spre înal?imea Acravimului, trece prin pustiul Sin ?i, ridicându-se dinspre miazazi catre Cade?-Barnea, merge pe la He?ron ?i, ajungând la Adar, trece pe partea dinspre apus a Cade?ului ?i se întoarce spre Carcaa;
\par 4 Trece apoi prin A?mon ?i urmeaza înainte pâna la râul Egiptului ?i apoi capatul acestui hotar atinge marea. Acesta va fi hotarul vostru de miazazi.
\par 5 Hotarul de rasarit e toata Marea Sarata pâna la gurile Iordanului. Iar apoi hotarul de miazanoapte pleaca din golful marii de la gurile Iordanului;
\par 6 De aici se ridica spre Bet-Hogla. Trece pe la miazanoapte de Bet-Araba ?i merge în sus pâna la piatra lui Bohan, fiul lui Ruben;
\par 7 Apoi hotarul se urca spre Debir, din valea Acor ?i se îndreapta spre miazanoapte catre Ghilgal, care se afla în fa?a Adumimului, pe partea de miazazi a râului; apoi trece pe la apele En-?eme? ?i se prelunge?te pâna la En-Roghel.
\par 8 De aici hotarul merge în sus spre valea Ben-Hinom, pe partea de miazazi de Iebus, care este Ierusalimul; apoi hotarul se ridica spre vârful muntelui, care este spre apus, în fa?a vaii Hinom, la marginea vaii Refaim, la miazanoapte.
\par 9 Din vârful muntelui, hotarul se îndreapta spre izvorul apelor Neftoah ?i merge spre ceta?ile din muntele Efron; apoi hotarul cote?te spre Baala, care e Chiriat-Iearimul;
\par 10 Dupa aceea hotarul se întoarce de la Baala spre mare ?i merge spre muntele Seir, trece pe partea de miazanoapte a muntelui Iearim, care e Chesalonul ?i, pogorându-se catre Bet-?eme?, trece prin Timna;
\par 11 De aici hotarul merge pe partea de miazanoapte a Ecronului ?i se întoarce spre ?icron, trece prin muntele Baala ?i ajunge pâna la Iabneel ?i apoi se termina hotarul la mare. Hotarul de la apus îl formeaza Marea cea Mare.
\par 12 Acesta este hotarul pamântului fiilor lui Iuda, dupa familiile lor, din toate par?ile.
\par 13 Lui Caleb, fiul lui Iefone, i-a dat Iosua parte intre fiii lui Iuda, cum poruncise Domnul lui Iosua ?i i-a dat Iosua Chiriat-Arba a tatalui lui Enac, care este Hebronul.
\par 14 Însa Caleb, fiul lui Iefone, a alungat de acolo pe cei trei fii ai lui Enac: pe ?e?ai, pe Ahiman ?i pe Talmai, copiii lui Enac.
\par 15 De aici Caleb a pornit asupra locuitorilor Debirului; numele Debirului era mai înainte Chiriat-Sefer.
\par 16 ?i a zis Caleb: "Cel ce va bate Chiriat-Seferul ?i-l va lua, aceluia îi voi da pe Acsa, fiica mea, de femeie".
\par 17 ?i l-a luat Otniel cel tânar, fiul lui Chenaz, fratele lui Caleb ?i i-a dat Caleb de femeie pe Acsa, fiica sa.
\par 18 Dar când a trebuit sa plece, a fost înva?ata sa ceara de la tatal sau o ?arina. ?i când a descalecat ea de pe asin, Caleb i-a zis: "Ce vrei?"
\par 19 Iar ea a zis: "Da-mi binecuvântare. Tu mi-ai dat pamântul de la miazazi; da-mi ?i izvoarele de apa!" ?i i-a dat izvoarele cele de sus ?i izvoarele cele de jos.
\par 20 Aceasta este mo?tenirea fiilor lui Iuda, dupa familiile lor.
\par 21 Ceta?ile care se aflau în partea de miazazi, la marginea semin?iei fiilor lui Iuda, spre hotarul Edomului, erau: Cab?eel, Eder ?i Iagur;
\par 22 China, Dimona ?i Adada;
\par 23 Chede?, Ha?or ?i Itnan;
\par 24 Zif, Telem ?i Bealot;
\par 25 Ha?or-Hadata, Cheriot-He?ron, adica Ha?or;
\par 26 Amam, ?erna ?i Molada;
\par 27 Ha?ar-Gada, He?mon ?i Bet-Palet;
\par 28 Ha?ar-?ual, Beer?eba ?i Biziotia, cu împrejurimile ?i satele lor;
\par 29 Baala, Iim ?i A?em;
\par 30 Eltolad, Chesil ?i Horma;
\par 31 ?iclag, Madmana ?i Sansana;
\par 32 Lebaot, ?ilhim, Ain ?i Rimon; de toate douazeci ?i noua de ceta?i cu satele lor.
\par 33 Iar la ?es erau: E?taol, ?ora, A?na ?i Gatnam;
\par 34 Zanuah, En-Ganim, Tapuah ?i Enam;
\par 35 Iarmut, Adulam, Memvra, Soco ?i Azeca;
\par 36 ?aaraim, Aditaim, Ghedera ?i Ghederotaim; paisprezece ceta?i cu satele lor.
\par 37 ?enan, Hada?a ?i Migdal-Gad;
\par 38 Dilean, Mi?pe ?i Iocteel;
\par 39 Lachi?, Bo?cat ?i Eglon;
\par 40 Cabon, Lahmas ?i Chitli?;
\par 41 Ghederot, Bet-Dagon, Naama ?i Macheda; ?aisprezece ceta?i cu satele lor.
\par 42 Libna, Eter ?i A?an;
\par 43 Iftah, A?na ?i Ne?ib;
\par 44 Cheila, Aczib, Mare?a ?i Edom; noua ceta?i cu satele lor.
\par 45 Ecron cu ceta?ile care ?ineau de el ?i cu satele lui;
\par 46 ?i de la Ecron spre mare tot ce se afla împrejurul A?dodului cu satele lui;
\par 47 A?dodul ?i ceta?ile care ?ineau de el ?i satele lui; Gaza cu ceta?ile care ?ineau de ea ?i satele ei pâna ia râul Egiptului ?i pâna la Marea cea Mare, care este hotar.
\par 48 Iar în mun?i erau: ?amir, Iatir ?i Soco;
\par 49 Dana, Chiriat-Sana, zis ?i Debir;
\par 50 Anab, E?temo ?i Anim;
\par 51 Go?en, Holon ?i Ghilo: unsprezece ceta?i cu satele lor.
\par 52 Anab, Duma ?i E?ean;
\par 53 Ianum, Bet-Tapuah ?i Afeca;
\par 54 Humta, Chiriat-Arba, zisa ?i Hebronul ?i ?ior; noua ceta?i cu satele lor.
\par 55 Maon, Carmel, Zif ?i Iuta;
\par 56 Izreel, Iocdeam ?i Zanuah;
\par 57 Cain, Ghibeea ?i Timna: zece ceta?i cu satele lor.
\par 58 Halhul, Bet-?ur ?i Ghedor;
\par 59 Maarat, Bet-Anot ?i Eltecon: ?ase ceta?i cu satele lor. Tecoa, Efrata sau Betleemul, Peor, Etam, Culon, Tatam, Sores, Carem, Galem, Betir ?i Manah: unsprezece ceta?i cu satele lor.
\par 60 Chiriat-Baal sau Chiriat-Iearim ?i Harabah; doua ceta?i cu satele lor ?i cu împrejurimile.
\par 61 În pustiu erau: Bet-Araba, Midin ?i Secaca;
\par 62 Nib?an, Ir-Melah ?i En-Ghedi: ?ase ceta?i cu satele lor.
\par 63 Dar pe Iebusei, locuitorii Ierusalimului, nu i-au putut alunga fiii lui Iuda ?i de aceea Iebuseii traiesc cu fiii lui Iuda în Ierusalim pâna în ziua de astazi.

\chapter{16}

\par 1 Apoi au cazut sor?ii fiilor lui Iosif; hotarul începe la Iordan, lânga Ierihon, merge catre apele Ierihonului, având la rasarit pustiul care se întinde de la Ierihon pâna la Betel
\par 2 ?i grani?a merge spre Luz, trece hotarul Archienilor pâna la Atarot,
\par 3 Apoi se lasa spre mare, catre hotarele lui Iaflet, pâna la hotarele Bet-Horonului de jos ?i pâna la Ghezer ?i se înfunda la mare.
\par 4 Aceasta parte au primit-o fiii. lui Iosif: Manase ?i Efraim.
\par 5 Hotarele fiilor lui Efraim, dupa familiile lor, au fost acestea: hotarul mo?tenirii lor era la rasarit Atarot-Adar pâna la Bet-Horonul de Sus ?i Ghezer;
\par 6 Apoi înainta spre apus pe la miazanoapte de Micmetat, se întorcea la rasarit spre Taanat-?ilo ?i trecea pe la rasarit de Ianoah.
\par 7 De la Ianoah se pogora la Atarot ?i la Naarata, atingând Ierihonul ?i se prelungea pâna la Iordan.
\par 8 De la Tapuah mergea spre apus, catre râule?ul Cana, pentru a se sfâr?i la mare. Aceasta este partea fiilor lui Efraim, dupa familiile lor.
\par 9 ?i fiilor lui Efraim li s-au mai dat ceta?i ?i în partea fiilor lui Manase; toate acele ceta?i erau cu satele lor.
\par 10 Dar Efraimi?ii n-au alungat pe Canaanei, care locuiau în Ghezer; de aceea Canaaneii au trait între Efraimi?i pâna în ziua de astazi, platindu-le bir. În cele din urma a venit Faraon, regele Egiptului, ?i a luat cetatea ?i a ars-o cu foc, ?i pe Canaanei ?i pe Ferezei ?i pe locuitorii Ghezerului i-a ucis ?i a dat Faraon cetatea de zestre fiicei sale.

\chapter{17}

\par 1 De asemenea a cazut, la sor?i, parte ?i semin?iei lui Manase, caci acesta era întâiul nascut al lui Iosif. Lui Machir întâiul nascut al lui Manase, care fusese viteaz în razboi, i-a cazut Galaadul ?i Vasanul.
\par 2 Dar le-au cazut, la sor?i, par?i ?i celorlal?i fii ai lui Manase, dupa familiile lor: fiilor lui Abiezer, fiilor lui Helec, fiilor lui Asriel, fiilor lui Sichem, fiilor lui Hefer ?i fiilor lui ?emida. Ace?tia sunt fiii lui Manase, fiul lui Iosif de parte barbateasca, dupa familiile lor.
\par 3 Iar Salfaad, fiul lui Hefer, fiul lui Galaad, fiul lui Machir, fiul lui Manase, n-a avut fii, ci numai fiice, ale caror nume sunt acestea: Mahla, Noa, Hogla, Milca ?i Tir?a.
\par 4 Acestea au venit la preotul Eleazar ?i la Iosua, fiul lui Navi, ?i la capetenii ?i au zis: "Domnul a poruncit lui Moise sa ni se dea parte ?i noua între fra?ii no?tri". ?i li s-a dat parte, între fra?ii tatalui lor.
\par 5 ?i a cazut lui Manase zece par?i, afara de ?ara Galaadului ?i a Vasanului, care erau peste Iordan,
\par 6 Pentru ca fiicele fiilor lui Manase au primit par?i printre fiii lui, iar rara Galaadului s-a cuvenit celorlal?i fii ai lui Manase.
\par 7 Hotarul pamântului fiilor lui Manase pleca de la A?er catre Micmetat, care e în fa?a Sichemului; de aici hotarul merge spre dreapta, pâna la locuitorii En-Tapuahului.
\par 8 Pamântul Tapuah a cazut lui Manase; iar cetatea Tapuah, de la hotarul lui Manase, a ramas fiilor lui Veniamin.
\par 9 De aici hotarul se pogoara spre râul Cana, pe malul de miazazi al râului.
\par 10 Ceta?ile acestea sunt ale lui Efraim, de?i sunt între fiii lui Manase. Hotarul lui Manase se sfâr?e?te la mare, pe partea de miazanoapte a râului. Partea de miazazi este a lui Efraim, iar cea de la miazanoapte este a lui Manase. Marea însa era hotarul lor la apus. La miazanoapte se marginea cu A?er, iar spre rasarit cu Isahar.
\par 11 În Isahar ?i A?er sunt ale lui Manase: Bet-?ean cu locurile care se ?in de el, Ibleam cu locurile care se ?in de el, locuitorii din Dor ?i din locurile care lin de el, locuitorii din En-Dor ?i locurile care ?in de el, locuitorii din Taanac ?i locurile care ?in de el, locuitorii Meghidonului ?i locurile care ?in de el, ?i a treia parte din Nafet cu satele lui.
\par 12 Fiii lui Manase n-au putut alunga pe locuitorii acestor ora?e ?i Canaaneii au ramas sa locuiasca în ?ara lui.
\par 13 Când fiii lui Israel au ajuns puternici, atunci Canaaneii au fost facu?i birnici ai lor, dar de alungat nu i-au alungat.
\par 14 Fiii lui Iosif au zis catre Iosua: "Pentru ce ne-ai dat o singura parte ?i un singur sor?, când noi suntem mul?i, de vreme ce ne-a binecuvântat a?a Domnul?"
\par 15 ?i Iosua le-a raspuns: "Daca sunte?i mul?i, duce?i-va în paduri ?i acolo, în ?ara Ferezeilor ?i Refaimilor, cauta?i-va loc, daca muntele Efraim va e strâmt!"
\par 16 Iar fiii lui Iosif au zis: "Muntele nu va ramâne al nostru, pentru ca to?i Canaaneii care locuiesc în vale au caru?e de fier, atât cei din Bet-?ean ?i din locurile care ?in de ea, cât ?i cei din ?esul Izreel".
\par 17 Dar Iosua a zis catre casa lui Iosif, lui Efraim ?i lui Manase: "Tu e?ti mult la numar ?i ai putere multa. Deci nu vei avea numai o parte.
\par 18 Muntele va fi al tau ?i padurea. Tu îl vei cura?i ?i va fi al tau pâna la capatul lui, caci tu vei izgoni pe Canaanei, de?i ei au caru?e de fier; de?i ei sunt tari, tu îi vei birui".

\chapter{18}

\par 1 Atunci s-a adunat toata ob?tea fiilor lui Israel la ?ilo ?i au a?ezat acolo cortul adunarii, caci ?ara fusese supusa de ei.
\par 2 Dar dintre fiii lui Israel mai ramasesera ?apte semin?ii care nu-?i primisera înca par?ile lor.
\par 3 Atunci a zis Iosua catre fiii lui Israel: "Oare mult ve?i ramâne voi nepasatori de a merge sa lua?i mo?tenire ?ara pe care v-a dat-o Domnul Dumnezeul parin?ilor vo?tri?
\par 4 "Da?i câte trei oameni de semin?ie ?i eu îi voi trimite, iar ei, sculându-se, se vor duce prin ?ara ?i o vor descrie, cum trebuie sa li se împarta în par?i, ?i apoi vor veni la mine.
\par 5 Sa o împarta în ?apte par?i; Iuda sa ramâna în partea sa spre miazazi, iar casa lui Iosif sa ramâna în partea sa la miazanoapte.
\par 6 Voi însa întocmi?i un plan al ?arii, împar?it în ?apte par?i ?i sa mi-l aduce?i ?i eu voi arunca sor?i aici înaintea Domnului Dumnezeului nostru.
\par 7 Levi?ii însa n-au parte între voi, caci preo?ia Domnului este partea lor. Iar Gad, Ruben ?i jumatate din semin?ia lui Manase ?i-au primit partea lor peste Iordan, spre rasarit, pe care le-a dat-o Moise, robul Domnului".
\par 8 Atunci s-au sculat oamenii aceia ?i s-au dus. Dar Iosua a dat celor ce s-au dus sa faca planul ?arii astfel de porunca: "Duce?i-va ?i cutreiera?i ?ara, descrie?i-o ?i va întoarce?i la mine, ?i eu va voi arunca aici sor?i înaintea Domnului, în ?ilo".
\par 9 ?i ei s-au dus, au cutreierat ?ara, au cercetat-o ?i au împar?it-o, dupa ceta?ile ei, în ?apte par?i, dupa un plan, ?i apoi au venit la Iosua în ?ilo.
\par 10 ?i le-a aruncat Iosua sor?i în ?ilo, înaintea Domnului, ?i a împar?it Iosua acolo ?ara fiilor lui Israel în par?ile ce li se cuvenea.
\par 11 Întâiul sor? a cazut semin?iei fiilor lui Veniamin, dupa familiile lor. Partea lor dupa sor? se întindea între fiii lui Iuda ?i fiii lui Iosif.
\par 12 Hotarul lor de miazanoapte se începe de la Iordan ?i trece pe lânga Ierihon, pe partea de miazanoapte, ?i se urca pe muntele de la apus ?i se sfâr?e?te în pustiul Betaven.
\par 13 De acolo hotarul merge spre Luz, pe partea de miazazi a Luzului sau a Betelului; apoi hotarul, coborând spre Atarot-Adar, merge catre muntele care e spre miazazi de Bet-Horonul de jos.
\par 14 Apoi hotarul se îndreapta spre partea marii, pe la rniazazi de muntele care vine la miazazi de Bet-Horon sau Chiriat-Iearim, cetatea fiilor lui Iuda. Aceasta este latura de la apus.
\par 15 Iar în partea de miazazi, de la Chiriat-Iearim, hotarul merge spre mare ?i ajunge pâna la izvorul apei Neftoah.
\par 16 Apoi hotarul se coboara spre capatul muntelui celui din fa?a vaii Ben-Hinom, la miazanoapte, ?i se coboara pe valea Hinom spre partea de miazazi a Iebusului ?i merge spre En-Roghel.
\par 17 Apoi se întoarce de la  miazanoapte ?i merge spre En-?eme? ?i înainteaza catre Ghelilot, care e în fa?a înal?imii Adumium, ?i se coboara spre piatra lui Bohan, fiul lui Ruben.
\par 18 Apoi trece pe la miazanoapte de Harabah ?i se coboara la hotarul Arabei;
\par 19 De acolo hotarul trece pe lânga Bet-Hogla, pe la miazanoapte, ?i se sfâr?e?te la, golful de miazanoapte al Marii Sarate, la capatul de miazazi al Iordanului. Acesta e hotarul de miazazi. Iar spre rasarit hotarul îl formeaza Iordanul.
\par 20 Aceasta este partea fiilor lui Veniamin cu hotarele ei din toate par?ile, dupa familiile lor.
\par 21 Iar ceta?ile fiilor lui Veniamin, dupa familiile lor, sunt acestea: Ierihonul, Bet-Hogla ?i Emec-Che?i?;
\par 22 Bet-Harabah, ?emaraim ?i Betel;
\par 23 Avim, Para ?i Ofra;
\par 24 Chefar-Amonai, Ofni ?i Gheba: douasprezece ceta?i cu satele lor;
\par 25 Ghibeon, Rama ?i Beerot;
\par 26 Mi?pa, Chefira ?i Mo?a;
\par 27 Rechem, Irpeel ?i Tareala;
\par 28 ?ela, Elef ?i Iebus sau Ierusalimul, Ghibeat ?i Chiriat: paisprezece ceta?i cu satele lor. Aceasta este partea lui Veniamin, dupa familiile lor.

\chapter{19}

\par 1 Al doilea sor? a cazut lui Simeon, semin?iei fiilor lui Simeon, dupa familiile lor; partea lor de mo?tenire a fost între hotarele par?ii fiilor lui Iuda.
\par 2 În partea lor se aflau: Beer?eba, ?eba ?i Molada;
\par 3 Ha?ar-?ual, Bala ?i A?em;
\par 4 Eltolad, Betul ?i Horma;
\par 5 ?iclag, Bet-Marcabot ?i Ha?ar-Susa;
\par 6 Bet-Lebaot ?i ?aruhen: treisprezece ceta?i cu satele lor.
\par 7 Ain, Rimon, Eter ?i A?an: patru ceta?i cu satele lor.
\par 8 ?i toate satele care se aflau împrejurul ceta?ilor acestora chiar pâna la Baalat-Beer-Ramatul de miazazi. Aceasta este partea de mo?tenire a fiilor lui Simeon, dupa familiile lor.
\par 9 Din mo?ia lui Iuda a fost despar?ita partea fiilor lui Simeon. Deoarece partea fiilor lui Iuda era prea mare pentru ei, fiii lui Simeon au primit parte între hotarele mo?iei lor.
\par 10 Al treilea sor? a cazut fiilor lui Zabulon, dupa familiile lor. Hotarul mo?tenirii lor se întindea pâna la Sarid,
\par 11 Se ridica la apus pâna la Mareala ?i atingea Dabe?etul ?i pârâul ce curge prin fa?a Iocneamului.
\par 12 De la Sarid apuca îndarat ?i mergea în partea de rasarit, spre rasaritul soarelui pâna la hotarul ?inutului Chislot-Tabor ?i de aici apuca spre Dabrat ?i se urca catre Iafia;
\par 13 Apoi trecea spre rasarit la Ghet-Hefer, la Ita-Ca?in ?i mergea spre Rimon, Metora ?i Nea;
\par 14 Dupa aceea hotarul se întorcea de la miazanoapte catre Hanaton ?i se termina în valea Iftah-El.
\par 15 Mai departe: Catat, Nahalal, ?imron, Idala ?i Betleem: douasprezece ceta?i cu satele lor.
\par 16 Aceasta e partea fiilor lui Zabulon, dupa familiile lor, ?i acestea sunt ceta?ile lor.
\par 17 Al patrulea sor? a cazut lui Isahar, fiilor lui Isahar, dupa familiile lor.
\par 18 În hotarul lor se cuprindeau: Izreel, Chesulot ?i ?unem;
\par 19 Hafaraim, ?ion ?i Anaharat;
\par 20 Harabit, Chi?ion ?i Ebe?;
\par 21 Remet, En-Ganim, En-Hada ?i Bet-Pa?e?;
\par 22 ?i atingea Taborul, ?aha?ima ?i Bet-?eme? ?i hotarul lor se termina la Iordan: ?aisprezece ceta?i cu satele lor.
\par 23 Aceasta e partea fiilor lui Isahar, dupa familiile lor ?i acestea sunt ceta?ile lor.
\par 24 Al cincilea sor? a cazut semin?iei fiilor lui A?er, dupa familiile lor.
\par 25 Hotarul lor trecea prin Helcat, Hali, Beten ?i Ac?af,
\par 26 Alamelec, Amead ?i Mi?eal ?i hotarul lor atingea, spre apus Carmelul ?i ?ihor-Libnat.
\par 27 Dupa aceea, hotarul se întorcea spre rasaritul soarelui la Bet-Dagon ?i atingea ?inutul Zabulon ?i valea Iftah-El pe la miazanoapte ?i  intra în hotarul Asatei la Bet-Emec ?i Neiel ?i mergea pe partea stânga a Cabulului.
\par 28 Mai departe urmeaza Abdon, Rehob, Hamon ?i Cana, pâna la Sidonul cel Mare.
\par 29 Dupa aceea, hotarul se întorcea spre Rama pâna la ora?ul cel întarit al Tirului, apoi spre Hosa ?i se sfâr?e?te la mare, în târgu?orul Aczib.
\par 30 Dupa aceea, urmeaza: Aco, Afec ?i Rehob: douazeci ?i doua de ceta?i ?i satele lor.
\par 31 Aceasta este partea semin?iei fiilor lui A?er, dupa familiile lor, ?i acestea sunt ceta?ile ?i satele lor.
\par 32 Al ?aselea sor? a cazut fiilor lui Neftali, semin?iei fiilor lui Neftali, dupa familiile lor.
\par 33 Hotarul lor mergea de la Helef ?i de la dumbrava cea din ?aananim catre Adami-Necheb ?i Iabneel, pâna la Lacum ?i se sfâr?ea la Iordan.
\par 34 De aici se întorcea hotarul spre apus, catre Asnot-Tabor ?i de acolo mergea spre Hucoc ?i atingea ?inutul  Zabulonului în partea de miazazi ?i ?inutul A?er în partea de apus ?i ?inutul lui Iuda la Iordan, spre rasaritul soarelui.
\par 35 Ceta?i întarite erau: ?idim, ?er, Hamat, Racat ?i Chineret,
\par 36 Adama, Rama ?i Ha?or;
\par 37 Chede?, Edrea ?i En-Ha?or;
\par 38 Ireon, Migdal-El, Horem, Bet-Anat ?i Bet-?eme?: nouasprezece ceta?i cu satele lor.
\par 39 Aceasta "este partea semin?iei fiilor lui Neftali, dupa familiile lor, ?i acestea sunt ceta?ile ?i satele lor.
\par 40 Semin?iei fiilor lui Dan, dupa familiile lor, i-a cazut al ?aptelea sor?.
\par 41 În hotarul mo?tenirii lor se cuprindeau: ?ora, E?taol ?i Ir-?eme?;
\par 42 ?aalabin, Aialon ?i Itla;
\par 43 Elon, Timnata ?i Ecron;
\par 44 Elteche, Ghibeton ?i Baalat;
\par 45 Iehud, Bene-Berac ?i Gat-Rimon;
\par 46 Me-Iarcon ?i Haracon cu hotarele aproape de Iopi. S-a vazut însa ca partea mo?tenirii fiilor lui Dan e mica pentru ei.
\par 47 Atunci s-au dus fiii lui Dan cu razboi asupra Le?emului ?i l-au împresurat, l-au lovit cu sabia ?i l-au luat mo?tenire ?i s-au a?ezat pe el ?i l-au numit Le?emul lui Dan, dupa numele lui Dan, tatal lor.
\par 48 Aceasta este partea semin?iei fiilor lui Dan, dupa familiile lor ?i acestea sunt ceta?ile ?i satele lor.
\par 49 Dupa ce au ispravit împar?irea ?arii, prin sor?i, fiii lui Israel au dat între ei parte de mo?tenire lui Iosua, fiul lui Navi.
\par 50 Dupa porunca Domnului, i-au dat lui cetatea Timnat-Serah, pe care a cerut-o el, în muntele lui Efraim. ?i a zidit cetate ?i a locuit în ea.
\par 51 Acestea sunt mo?iile pe care Eleazar preotul, Iosua, fiul tui Navi, ?i capeteniile familiilor le-au împar?it fiilor lui Israel, prin sorti, în ?ilo, înaintea fe?ei Domnului, la intrarea cortului adunarii. ?i a?a s-a ispravit împar?irea ?arii.

\chapter{20}

\par 1 ?i a zis Domnul catre Iosua:
\par 2 "Spune fiilor lui Israel: Face?i-va ora?e de scapare cum v-am zis Eu prin Moise,
\par 3 Ca sa poata scapa acolo uciga?ul care a ucis om din gre?eala, fara precugetare; ?i sa fie ora?ele acestea loc de scapare pentru cel ce a ucis, ca sa nu moara de mâna celui ce razbuna sângele varsat, înainte de a se înfa?i?a înaintea ob?tii la judecata.
\par 4 ?i cine va fugi în una din ceta?ile acestea sa stea la poarta ceta?ii ?i sa spuna pricina sa în auzul batrânilor ceta?ii ?i ei îl vor primi în cetate ?i-i vor da loc, ca sa traiasca la ei.
\par 5 ?i când va alerga dupa el cel ce razbuna sângele, atunci ei nu trebuie sa-l dea pe uciga? în mâinile lui, pentru ca el a ucis fara sa vrea pe aproapele sau, neavând pe dânsul ura cu o zi sau cu doua mai înainte.
\par 6 Sa traiasca el în cetatea aceasta pâna ce se va înfa?i?a înaintea ob?tii la judecata, pâna va muri arhiereul care va fi în acele zile. Apoi sa se întoarca uciga?ul ?i sa sa duca în cetatea sa ?i la casa sa, în cetatea de unde a fugit".
\par 7 ?i au rânduit Chede?ul în Galileea, în muntele Neftalimului; Sichemul în muntele Efraim ?i Chiriat-Arba sau Hebronul, în muntele lui Iuda.
\par 8 Peste Iordan, în fa?a lerihonului, spre rasarit, au rânduit: Be?erul în pustiu, la ?es, în semin?ia lui Ruben; Ramot în Galaad, în semin?ia lui Gad; ?i Golan în Vasan, în semin?ia lui Manase.
\par 9 Aceste ceta?i le-au rânduit pentru to?i fiii lui Israel ?i pentru strainii care traiesc printre ei, ca sa fuga acolo cel ce va ucide om din gre?eala, ca sa nu moara de mâna celui ce razbuna sângele varsat, pâna nu se va înfa?i?a înaintea ob?tii la judecata.

\chapter{21}

\par 1 Capeteniile familiilor Levi?ilor au venit la Eleazar preotul ?i catre Iosua, fiul lui Navi, ?i la capeteniile semin?iilor fiilor lui Israel,
\par 2 ?i au vorbit cu ei în ?ilo, în pamântul Canaanului ?i au zis: "Domnul a poruncit prin Moise sa ni se dea ceta?i pentru locuit ?i împrejurimile lor pentru vitele noastre".
\par 3 ?i au dat fiii lui Israel Levi?ilor din par?ile lor, dupa porunca Domnului, urmatoarele ceta?i cu împrejurimile lor:
\par 4 ?i s-au tras sor?i pentru familia lui Cahat ?i Levi?ii, fiii lui Aaron preotul, au primit prin sor?i treisprezece ceta?i din semin?ia lui Iuda ?i din semin?ia lui Simeon ?i din semin?ia lui Veniamin.
\par 5 Iar celorlal?i fii ai lui Cahat le-au cazut la sor?i zece ceta?i din familiile semin?iei lui Efraim ?i din semin?ia lui Dan ?i de la jumatate din semin?ia lui Manase.
\par 6 Fiilor lui Gher?on li s-au cuvenit prin sor?i treisprezece ceta?i de la familiile semin?iei lui Isahar ?i de la semin?ia lui A?er ?i de la semin?ia lui Neftali ?i de la jumatate din semin?ia lui Manase, în Vasan.
\par 7 Iar fiilor lui Merari, dupa familiile lor, le-au cazut la sor?i douasprezece ceta?i din semin?ia lui Ruben ?i din semin?ia lui Gad ?i din semin?ia lui Zabulon.
\par 8 ?i au dat fiii lui Israel ceta?ile acestea cu împrejurimile lor prin sor?i, cum poruncise Domnul prin Moise.
\par 9 Din semin?ia fiilor lui Iuda ?i din semin?ia fiilor lui Veniamin ?i din semin?ia fiilor lui Simeon au dat ceta?ile urmatoare, care se numesc pe nume:
\par 10 Fiilor lui Aaron din familia lui Cahat dintre fiii lui Levi, fiindca sor?ul lor a fost întâiul,
\par 11 Li s-a dat: Chiriat-Arba, a tatalui lui Enac, sau Hebronul, în mun?ii lui Iuda ?i locurile dimprejurul ei;
\par 12 Iar ?arina ceta?ii acesteia ?i satele ei s-au dat ca mo?ie lui Caleb, fiul lui Iefone.
\par 13 ?i astfel fiilor lui Aaron preotul li s-a dat cetatea cea de scapare a uciga?ilor, Hebronul, ?i împrejurimile ei, Libna ?i împrejurimile ei.
\par 14 Iatirul ?i împrejurimile lui, E?temoa ?i împrejurimile ei;
\par 15 Holonul ?i împrejurimile lui, Debirul ?i împrejurimile;
\par 16 Ainul ?i împrejurimile lui, Iuta ?i împrejurimile ei, Bet-Seme?ul ?i împrejurimile lui: noua ceta?i din aceste doua semin?ii.
\par 17 Iar din semin?ia lui Veniamin: Ghibeonul ?i împrejurimile lui, Gheba cu împrejurimile ei,
\par 18 Anatotul cu împrejurimile lui, Almonul cu împrejurimile lui: patru ceta?i.
\par 19 Ceta?ile care au cazut la sor?i pentru preo?i, fiii lui Aaron, au fost toate treisprezece cu împrejurimile lor.
\par 20 Iar celorlal?i din familiile fiilor lui Cahat, Levi?ilor, le-a cazut la sor?i ceta?i în pamântul lui Efraim.
\par 21 ?i li s-a dat cetatea de scapare pentru uciga?i, Sichemul cu împrejurimile lui în muntele lui Efraim, Ghezerul cu împrejurimile lui,
\par 22 Chib?aimul cu împrejurimile lui, Bet-Horonul cu împrejurimile lui: patru ceta?i.
\par 23 Din semin?ia lui Dan le-au cazut: Elteche ?i împrejurimile lui, Ghibetonul ?i împrejurimile lui;
\par 24 Aialonul ?i împrejurimile lui, Gat-Rimonul ?i împrejurimile lui: patru ceta?i.
\par 25 Din jumatatea semin?iei lui Manase: Taanacul ?i împrejurimile lui, Gat-Rimonul ?i împrejurimile lui: doua ceta?i.
\par 26 Toate ceta?ile cu împrejurimile lor cazute la sor?i celorlal?i fii ai lui Cahat au fost zece.
\par 27 Iar fiilor lui Gher?on din familiile Levi?ilor li s-au dat: doua ceta?i în Vasan, din jumatatea semin?iei lui Manase, ?i anume: Golanul, cetatea de scapare pentru uciga?i, cu împrejurimile lui, ?i Be?tra cu împrejurimile ei;
\par 28 Patru ceta?i în semin?ia lui Isahar: Chi?ionul cu împrejurimile lui, Dabrat cu împrejurimile lui,
\par 29 Iarmutul cu împrejurimile lui ?i En-Ganimul cu împrejurimile lui;
\par 30 Patru ceta?i în semin?ia lui A?er: Mi?alul cu împrejurimile lui, Abdonul cu împrejurimile lui,
\par 31 Helcatul cu împrejurimile lui ?i Rehobul cu împrejurimile lui;
\par 32 Trei ceta?i din semin?ia lui Neftali: Chede?ul Galileii, cetate pentru scaparea uciga?ilor, cu împrejurimile ei, Hamot-Dorul cu împrejurimile lui ?i Cartanul cu împrejurimile lui.
\par 33 Toate ceta?ile cazute la sor?i fiilor lui Gher?on, dupa familiile lor, au fost treisprezece ceta?i cu împrejurimile lor.
\par 34 Celorlal?i levi?i din fiii lui Merari li s-au dat patru ceta?i din semin?ia lui Zabulon: Iocneamul cu împrejurimile lui,
\par 35 Carta cu împrejurimile ei, Dimna cu împrejurimile ei ?i Nahalalul cu împrejurimile lui;
\par 36 Patru ceta?i de cealalta parte de Iordan, în fa?a Ierihonului, în semin?ia lui Ruben: Be?erul, cetate de scapare pentru uciga?i, cu împrejurimile lui, în pustiul Miso, Iah?a ?i împrejurimile ei,
\par 37 Chedemotul ?i împrejurimile lui ?i Mefaatul cu împrejurimile lui.
\par 38 În semin?ia lui Gad li s-au dat ceta?ile de scapare pentru uciga?i: Ramot-Galaadul cu împrejurimile lui, Mahanaimul cu împrejurimile lui,
\par 39 He?bonul cu împrejurimile lui ?i Iazerul cu împrejurimile lui.
\par 40 Ceta?ile ie?ite la sor?i pentru familiile de Levi?i din neamul lui Merari, dupa familiile lor, au fost douasprezece.
\par 41 Iar toate ceta?ile date Levi?ilor între fiii lui Israel au fost patruzeci ?i opt de ceta?i cu împrejurimile lor. Fiecare din aceste ceta?i î?i avea împrejurimile ei de jur împrejur; a?a erau toate ceta?ile acestea.
\par 42 Dupa ce a sfâr?it Iosua împar?irea ?arii prin sor?i, fiii lui Israel au dat parte lui Iosua dupa porunca Domnului; ?i i-au dat cetatea pe care a cerut-o el: Timnat-Serah, în muntele Efraim, ?i a zidit Iosua cetatea pe care a cerut-o ?i a locuit în ea. ?i a luat Iosua cu?itele cele de piatra, cele cu care taiase împrejur pe fiii lui Israel, care se nascusera pe cale în pustiu, caci în pustiu nu fusesera taia?i împrejur, ?i le-a pus în Timnat-Serah.
\par 43 Astfel a dat Domnul lui Israel toata ?ara, pe care jurase sa o dea parin?ilor lor ?i au primit-o ei mo?tenire ?i s-au a?ezat în ea.
\par 44 ?i le-a dat Domnul lini?te ?i odihna din toate par?ile, cum jurase parin?ilor lor, ?i nimeni dintre to?i vrajma?ii lor n-a putut sta împotriva lor, ci pe to?i vrajma?ii lor i-a dat Domnul în mâinile lor.
\par 45 ?i n-a ramas neîmplinit nici un cuvânt din toate cuvintele cele bune pe care le vorbise Domnul casei lui Israel: toate s-au împlinit.

\chapter{22}

\par 1 Atunci Iosua a chemat semin?ia lui Ruben, a lui Gad ?i jumatate din a lui Manase ?i le-a zis:
\par 2 "Voi a?i împlinit toate câte v-a poruncit Domnul prin Moise ?i a?i ascultat cuvintele mele întru toate câte v-am poruncit;
\par 3 N-a?i lasat pe fra?ii vo?tri în toata aceasta îndelungata vreme, pâna în ziua aceasta ?i a?i împlinit cele ce se cuvenea a împlini dupa porunca Domnului Dumnezeului vostru.
\par 4 Acum Domnul Dumnezeul vostru a lini?tit pe fra?ii vo?tri, cum le spusese. Întoarce?i-va dar ?i duce?i-va la corturile voastre, în pamântul mo?tenirii voastre, pe care vi l-a dat Moise, sluga Domnului, peste Iordan.
\par 5 Dar sa va sili?i a împlini cu grija poruncile ?i legea pe care v-a dat-o Moise, sluga Domnului: de a iubi pe Domnul Dumnezeul vostru, de a umbla în toate caile Lui, de a pazi poruncile Lui, de a va lipi de El ?i de a-I sluji Lui din toata inima voastra ?i din tot sufletul vostru".
\par 6 Dupa aceea Iosua i-a binecuvântat ?i le-a dat drumul ?i ei s-au împar?it pe la corturile lor.
\par 7 Unei jumata?i din semin?ia lui Manase i-a dat Moise parte în Vasan, iar celeilalte jumata?i i-a dat Iosua parte cu fra?ii lui dincoace de Iordan, spre apus. ?i când le-a dat drumul Iosua pe la corturile lor ?i i-a binecuvântat,
\par 8 Atunci le-a spus: "Cu mari boga?ii va întoarce?i voi pe la corturile voastre, cu mare mul?ime de vite ?i de argint, cu aur, cu arama ?i cu fier ?i cu mare mul?ime de haine: sa împar?i?i dar prada cu fra?ii vo?tri".
\par 9 ?i s-au întors fiii lui Ruben ?i ai lui Gad ?i jumatate din semin?ia lui Manase ?i au plecat de la fiii lui Israel din ?ilo, care e în pamântul Canaanului, ca sa mearga în ?ara Galaadului, în pamântul mo?tenirii lor, pe care îl luasera în stapânire dupa porunca Domnului data prin Moise.
\par 10 ?i ajungând în preajma Iordanului ce e în rara Canaan, fiii lui Ruben ?i fiii lui Gad ?i jumatate din semin?ia lui Manase au zidit acolo lânga Iordan jertfelnic, jertfelnic mare la vedere.
\par 11 ?i au auzit fiii lui Israel ca se zicea: "Iata fiii lui Ruben ?i fiii lui Gad ?i jumatate din semin?ia lui Manase au zidit jertfelnic pe pamântul Canaanului, în preajma Iordanului, în fa?a fiilor lui Israel".
\par 12 Când au auzit acestea fiii lui Israel, s-a adunat toata ob?tea fiilor lui Israel la ?ilo, ca sa mearga împotriva lor cu razboi.
\par 13 Dar fiii lui Israel au trimis mai întâi la fiii lui Ruben, la fiii lui Gad ?i la jumatate din semin?ia lui Manase, în ?ara Galaadului, pe Finees, fiul preotului Eleazar,
\par 14 Împreuna cu zece capetenii, câte o capetenie de fiecare semin?ie a lui Israel de dincoace de Iordan; fiecare din ace?tia era capetenie de mie din semin?iile lui Israel.
\par 15 ?i au venit ei la fiii lui Ruben ?i la fiii lui Gad ?i la jumatatea semin?iei lui Manase în pamântul Galaadului ?i au vorbit cu ei ?i le-a zis:
\par 16 "A?a graie?te toata ob?tea Domnului: Ce înseamna nelegiuirea aceasta pe care a?i facut-o înaintea Domnului Dumnezeului lui Israel, abatându-va acum de la Domnul Dumnezeul lui Israel, ridicându-va jertfelnic ?i sculându-va acum împotriva Domnului?
\par 17 Nu va ajunge oare pacatul din Peor, de care nu ne-am spalat nici pâna în ziua de astazi ?i pentru care a fost batuta ob?tea de Domnul?
\par 18 ?i iata astazi voi va abate?i de la Domnul! Astazi voi va scula?i împotriva Domnului, iar mâine se va mânia Domnul pe toata ob?tea lui Israel.
\par 19 Daca însa pamântul mo?tenirii voastre vi se pare necurat, atunci trece?i în pamântul mo?tenirii Domnului, unde se afla cortul Domnului, lua?i-va parte între noi, dar nu va ridica?i împotriva Domnului, nici împotriva noastra nu va ridica?i, zidindu-va jertfelnic afara de cel al Domnului Dumnezeului nostru.
\par 20 Oare n-a facut singur Acan, fiul lui Zerah, nelegiuire, luând din cele date nimicirii, dar mânia a venit asupra a toata ob?tea fiilor lui Israel? ?i oare numai el singur a murit pentru nelegiuire?"
\par 21 Atunci fiii lui Ruben ?i fiii lui Gad ?i jumatate din semin?ia lui Manase, raspunzând la acestea, au zis capeteniilor lui Israel:
\par 22 "Dumnezeul dumnezeilor este Domnul ?i Domnul Dumnezeul dumnezeilor ?tie ?i sa ?tie ?i Israel: de ne razvratim ?i ne abatem noi de la Domnul, atunci sa nu ne cru?e pe noi Domnul astazi!
\par 23 ?i daca am ridicat noi un jertfelnic, ca sa ne abatem de la Domnul Dumnezeul nostru ?i ca sa aducem pe el arderi de tot ?i prinos de pâine ?i ca sa savâr?im pe el jertfe de împacare, atunci Domnul sa ne ceara socoteala de aceasta!
\par 24 Dar noi am facut aceasta de teama ca nu cumva în viitor fiii vo?tri sa zica fiilor no?tri: "Ce ave?i voi cu Domnul Dumnezeul lui Israel?
\par 25 Domnul a pus hotar între noi ?i voi, fiii lui Ruben ?i fiii lui Gad ?i jumatatea semin?iei lui Manase, Iordanul; voi n-ave?i deci nici o legatura cu Domnul". ?i astfel fiii vo?tri nu vor îngadui fiilor no?tri sa se închine Domnului în ?ilo.
\par 26 De aceea am zis noi: Sa ne facem un jertfelnic, nu pentru arderi de tot nici pentru jertfe,
\par 27 Ci ca sa fie el între noi ?i voi, între urma?ii no?tri, marturie ca noi putem sluji Domnului cu arderile de tot ale noastre ?i cu jertfele noastre ?i cu cele de împacare ale noastre, ?i pentru ca în vremurile viitoare sa nu zica fiii vo?tri catre fiii no?tri: Voi nu ave?i nici o legatura cu Domnul.
\par 28 ?i ziceam noi: Daca ni se va zice astfel noua ?i urma?ilor no?tri, atunci vom raspunde: Privi?i chipul jertfelnicului Domnului pe care l-au facut parin?ii no?tri nu pentru arderi de tot ?i nu pentru jertfe, ci ca sa fie marturie între noi ?i voi ?i între fiii no?tri ?i fiii vo?tri.
\par 29 Sa nu se întâmple una ca aceea ca sa ne ridicam noi împotriva Domnului ?i sa ne abatem acum de la Domnul ?i sa facem jertfelnic pentru arderi de tot ?i pentru prinos de pâine ?i pentru jertfe, afara de jertfelnicul Domnului Dumnezeu care se afla înaintea cortului".
\par 30 Iar preotul Finees ?i toate capeteniile ob?tii ?i capeteniile peste miile lui Israel, care erau cu dânsul, auzind cuvintele pe care le-au vorbit fiii lui Ruben ?i fiii lui Gad ?i jumatatea semin?iei lui Manase, au ramas mul?umi?i.
\par 31 ?i a zis Finees, fiul preotului Eleazar, catre fiii lui Ruben ?i catre fiii lui Gad ?i catre jumatatea de semin?ie a lui Manase: "Astazi am aflat noi ca Domnul este în mijlocul nostru ?i ca voi n-a?i facut prin aceasta o nelegiuire; acum a?i izbavit pe fiii lui Israel din mâna Domnului".
\par 32 ?i s-a întors Finees, fiul preotului Eleazar, ?i capeteniile de la fiii lui Ruben ?i de la fiii lui Gad ?i de la jumatatea de semin?ie a lui Manase din pamântul Galaadului în ?ara Canaan la fiii lui Israel ?i le-a adus raspunsul.
\par 33 ?i le-a placut aceasta fiilor lui Israel ?i au binecuvântat fiii lui Israel pe Dumnezeu ?i au zis sa nu se mai ridice împotriva lor cu razboi, ca sa pustiiasca ?ara în care locuiau fiii lui Ruben ?i fiii lui Gad ?i jumatate din semin?ia lui Manase.
\par 34 Iar fiii lui Ruben ?i fiii lui Gad ?i jumatate din semin?ia lui Manase au numit jertfelnicul Ed, adica marturie, caci î?i ziceau: Acesta este marturie între noi ca Domnul este Dumnezeul nostru.

\chapter{23}

\par 1 Trecând multa vreme, dupa ce Domnul Dumnezeu a odihnit pe Israel, scutindu-l de to?i vrajma?ii lui din toate par?ile, Iosua a ajuns batrân ?i înaintat în vârsta.
\par 2 Atunci a chemat Iosua pe to?i fiii lui Israel; pe batrânii lor, capeteniile lor, pe judecatorii lor ?i pe mai-marii o?tilor lor ?i le-a zis: "Eu am îmbatrânit ?i sunt înaintat în vârsta;
\par 3 Voi a?i vazut ce a facut Domnul Dumnezeul vostru înaintea fe?ei voastre cu toate aceste popoare, caci Domnul Dumnezeul vostru Însu?i S-a luptat pentru voi.
\par 4 Iata eu v-am împar?it prin sor?i popoarele acestea ce au mai ramas în mo?tenirea semin?iilor voastre, toate popoarele pe care eu le-am nimicit de la Iordan pâna la Marea cea Mare de la apusul soarelui.
\par 5 Domnul Dumnezeul vostru Însu?i le va alunga de la fa?a voastra pâna vor pieri; ?i va trimite asupra lor fiare salbatice pâna le va stârpi pe ele ?i pe regii lor de la fa?a voastra, ?i le va nimici înaintea voastra, ca sa primi?i de mo?tenire ?ara lor, cum v-a grait Domnul Dumnezeu.
\par 6 De aceea sili?i-va sa plini?i întocmai ?i sa pazi?i cele scrise în cartea legii lui Moise, neabatându-va de la ea nici la dreapta, nici la stânga.
\par 7 Sa nu intra?i în legatura cu aceste popoare care au mai ramas printre voi, sa nu pomeni?i numele dumnezeilor lor, sa nu va pleca?i înaintea lor, nici sa le sluji?i, sau sa va închina?i lor;
\par 8 Ci va lipi?i de Domnul Dumnezeul vostru, cum a?i facut pâna în ziua de astazi.
\par 9 Domnul a alungat de la voi popoarele mari ?i tari ?i nimeni nu s-a putut împotrivi pâna astazi;
\par 10 Unul din voi a alungat mii, caci Însu?i Domnul Dumnezeul vostru S-a luptat pentru voi, cum v-a grait.
\par 11 De aceea sili?i-va sa iubi?i pe Domnul Dumnezeul vostru.
\par 12 Iar de va ve?i întoarce ?i va ve?i alatura la popoarele acestea ramase ?i ve?i intra în înrudire cu ele ?i ve?i merge la ele ?i ele vor veni la voi,
\par 13 Atunci sa ?ti?i ca Domnul Dumnezeul vostru nu va mai alunga de la voi popoarele acestea, ci ele vor fi pentru voi la? ?i mreaja, bici pentru spinarile voastre ?i spin pentru ochii vo?tri, pâna ve?i fi stârpi?i din aceasta ?ara buna pe care v-a dat-o Domnul Dumnezeul vostru.
\par 14 Iata eu astazi plec în calea în care merg to?i pamântenii, iar voi sa recunoa?te?i cu toata inima voastra ?i cu tot sufletul vostru ca n-a ramas zadarnic nici un cuvânt din toate cuvintele bune pe care le-a rostit pentru voi Domnul Dumnezeul vostru: toate s-au împlinit pentru voi ?i nici un cuvânt n-a ramas neîmplinit.
\par 15 Dar dupa cum s-a împlinit cu voi tot cuvântul bun pe care l-a grait Domnul Dumnezeul vostru, tot a?a va împlini Domnul asupra voastra ?i tot cuvântul rau pâna va va stârpi din aceasta ?ara bogata pe care v-a dat-o Domnul Dumnezeul vostru.
\par 16 De ve?i calca a?ezamântul Domnului Dumnezeului vostru, pe care l-a încheiat El cu voi ?i va ve?i duce sa sluji?i la al?i dumnezei ?i sa va închina?i lor, se va aprinde asupra voastra mânia Domnului ?i ve?i pieri curând din ?ara aceasta bogata pe care v-a dat-o Domnul".

\chapter{24}

\par 1 Apoi a adunat Iosua toate semin?iile lui Israel la Sichem ?i a chemat pe batrânii lui Israel, pe capeteniile lui, pe judecatorii lui ?i pe mai-marii o?tirii lui; ?i s-au înfa?i?at ei înaintea Domnului.
\par 2 ?i a zis Iosua catre tot poporul: "A?a zice Domnul Dumnezeul lui Israel: În vechime parin?ii vo?tri, Terah, tatal lui Avraam ?i tatal lui Nahor, au trait dincolo de râu (Eufrat) ?i slujeau la al?i dumnezei.
\par 3 Dar Eu am luat pe parintele vostru Avraam de mâna, de dincolo de Eufrat, ?i l-am pova?uit catre aceasta ?ara a Canaanului ?i am înmul?it samân?a lui ?i i-am dat pe Isaac.
\par 4 Lui Isaac i-am dat pe Iacov ?i pe Isav; lui Isav i-am dat muntele Seir de mo?tenire, iar Iacov ?i fiii lui s-au pogorât în Egipt ?i au ajuns acolo popor mare, tare ?i mult la numar ?i Egiptenii au început sa-i strâmtoreze.
\par 5 Dar am trimis pe Moise ?i pe Aaron ?i am lovit Egiptul cu semne pe care le-am facut Eu acolo ?i apoi v-am scos pe voi.
\par 6 Eu am scos pe parin?ii vo?tri din Egipt ?i a?i venit la Marea Ro?ie. Atunci Egiptenii au alergat dupa parin?ii vo?tri cu care ?i cu calare?i pâna la Marea Ro?ie.
\par 7 Iar ei au strigat catre Domnul ?i El a pus nor între voi ?i Egipteni ?i a adus asupra lor marea care i-a ?i acoperit. Ochii vo?tri au vazut ce am facut Eu în Egipt. Dupa aceea, a?i ramas voi multa vreme în pustiu.
\par 8 Apoi v-am dus Eu asupra Amoreilor care locuiau peste Iordan; ?i ei s-au luptat cu voi, dar Eu i-am dat în mâinile voastre ?i a?i primit de mo?tenire ?ara lor ?i Eu i-am stârpit înaintea voastra.
\par 9 S-a sculat apoi Balac, fiul lui Sefor, regele Moabului, ?i a pornit cu razboi asupra lui Israel ?i a trimis sa cheme pe Valaam, fiul lui Beor, ca sa va blesteme;
\par 10 Dar Eu n-am voit sa ascult pe Valaam ?i el v-a binecuvântat ?i v-am izbavit din mâinile lui Balac.
\par 11 Dupa aceea a?i trecut Iordanul ?i a?i venit la Ierihon. Atunci au început a se lupta cu voi locuitorii Ierihonului, apoi Amoreii, Ferezeii, Canaaneii, Heteii, Ghergheseii, Heveii ?i Iebuseii, dar Eu i-am dat în mâinile voastre.
\par 12 Trimis-am înaintea voastra viespi care au gonit de la voi pe cei doi regi ai Amoreilor; nu cu sabia ta, nici cu arcul tau ai facut acestea.
\par 13 ?i v-am dat ?ara cu care nu v-aii ostenit ?i ceta?ile pe care nu le-aii zidit ?i trai?i în ele; din viile ?i din gradinile de maslini pe care nu le-a?i sadit, iata, mânca?i roade.
\par 14 Teme?i-va dar de Domnul ?i-I sluji?i Lui cu credincio?ie ?i cura?enie. Lepada?i dumnezeii carora au slujit parin?ii vo?tri dincolo de râu ?i în Egipt ?i sluji?i Domnului.
\par 15 Iar daca nu va place sa sluji?i Domnului, atunci alege?i-va acum cui ve?i sluji: sau dumnezeilor carora au slujit parin?ii vo?tri cei de peste râu sau dumnezeilor Amoreilor, în ?ara cal-ora trai?i. Eu însa ?i casa mea vom sluji Domnului, ca sfânt este!"
\par 16 ?i a raspuns poporul ?i a zis: "Departe de noi. gândul sa parasim pe Domnul ?i sa ne apucam sa slujim la al?i dumnezei,
\par 17 Caci Domnul este Dumnezeul nostru; El ne-a scos pe noi ?i pe parin?ii no?tri din ?ara Egiptului, din casa robiei ?i a facut Înaintea ochilor no?tri minuni mari ?i ne-a pazit în toata calea pe care am umblat ?i printre toate popoarele pe la care am trecut.
\par 18 Domnul a alungat de la noi toate popoarele ?i pe Amoreii care traiau în ?ara aceasta. De aceea ?i noi vom sluji Domnului, caci El este Dumnezeul nostru!"
\par 19 ?i a zis Iosua poporului: "Nu ve?i putea sa sluji?i Domnului Dumnezeu, caci El este Dumnezeu sfânt, Dumnezeu zelos ?i nu va rabda nelegiuirile  voastre, nici pacatele voastre.
\par 20 Daca voi ve?i parasi pe Domnul ?i ve?i sluji la dumnezei straini, atunci El va aduce asupra voastra raul ?i va va stârpi, dupa ce v-a facut bine".
\par 21 Iar poporul a zis catre Iosua: "Nu, noi Domnului vom sluji".
\par 22 Iosua însa a zis poporului: "Va sunte?i voi martori ca v-aii ales pe Domnul sa-I sluji?i?" Ei au raspuns: "Suntem martori!"
\par 23 "A?adar, a adaugat Iosua, lepada?i dumnezeii straini pe care îi ave?i ?i întoarce?i-va inima catre Domnul Dumnezeul lui Israel!"
\par 24 ?i a zis poporul catre Iosua: "Domnului Dumnezeului nostru vom sluji ?i glasul Lui vom asculta!"
\par 25 ?i a încheiat Iosua cu poporul legamânt în ziua aceea ?i i-a dat legi ?i porunci în Sichem, înaintea cortului Domnului Dumnezeului lui Israel.
\par 26 ?i a scris Iosua cuvintele acestea în cartea legii lui Dumnezeu ?i a luat o piatra mare ?i a pus-o acolo sub stejarul care era lânga loca?ul sfânt al Domnului.
\par 27 Apoi a zis Iosua catre tot poporul: "Iata piatra aceasta ne va fi marturie, caci ea a auzit toate cuvintele Domnului, pe care le-a grait El cu noi astazi. Sa fie dar ca marturie împotriva voastra în zilele viitoare, ca sa nu min?i?i înaintea Domnului Dumnezeului vostru!"
\par 28 ?i a dat Iosua drumul poporului ?i s-a întors fiecare la mo?tenirea lui.
\par 29 ?i a murit dupa aceea Iosua, fiul lui Navi, robul Domnului, fiind de o suta zece ani.
\par 30 ?i l-au îngropat în ?inutul mo?tenirii sale la Timnat-Serah, care e în muntele Efraim, la miazanoapte de muntele Gaa?. ?i au pus acolo cu dânsul, în mormântul în care l-au îngropat, cu?itele cele de piatra cu care Iosua a taiat împrejur pe fiii lui Israel în Ghilgal, când i-a scos pe ei din Egipt, cum poruncise Domnul, ?i sunt ele acolo pâna în ziua de astazi.
\par 31 Israel a slujit Domnului în toate zilele lui Iosua ?i în toate zilele batrânilor a caror via?a s-a prelungit dupa Iosua ?i care vazusera toate lucrurile Domnului, pe care le facuse El cu Israel.
\par 32 Oasele lui Iosif, pe care le adusesera fiii lui Israel din Egipt, le-au îngropat în Sichem, în partea de ?arina pe care o cumparase Iacov de la fiii lui Hemor, tatal lui Sichem, cu o suta de argin?i, ?i care cazuse de mo?tenire fiilor lui Iosif.
\par 33 Dupa aceasta a murit ?i Eleazar, fiul lui Aaron, arhiereul, ?i l-au îngropat în Ghibeea, ora?ul lui Finees, fiul lui, care i se daduse în muntele Efraim. În ziua aceea fiii lui Israel, luând chivotul lui Dumnezeu, l-au dus cu ei, iar Finees a fost preot în locul lui Eleazar, tatal lui, pâna ce a murit ?i a fost îngropat în cetatea sa Ghibeea. Iar fiii lui Israel s-au dus fiecare la locul sau ?i în cetatea sa. ?i au început fiii lui Israel a sluji Astartei ?i lui A?tarot ?i dumnezeilor popoarelor vecine. De aceea i-a dat Domnul în mâinile lui Eglon, regele Moabului, ?i i-a stapânit optsprezece ani.


\end{document}