\begin{document}

\title{Ruth}


\chapter{1}

\par 1 În zilele acelea, când cârmuiau în Israel judecătorii, s-a întâmplat foamete pe pământ. Atunci un om din Betleemul lui Iuda s-a dus cu femeia sa și cu cei doi feciori ai săi să locuiască în șesul Moabiților.
\par 2 Numele omului aceluia era Elimelec; pe femeia sa o chema Noemina, iar numele celor doi feciori ai lor erau Mahlon și Chilion. Aceștia erau Efrateni din Betleemul lui Iuda și, venind în șesul Moabiților, au rămas acolo.
\par 3 După un timp Elimelec, bărbatul Noeminei, a murit ea a rămas cu cei doi feciori ai săi.
\par 4 Aceștia și-au luat soții dintre moabitence: numele uneia era Orfa, iar numele celeilalte era Rut. Aceștia au trăit acolo ca la zece ani.
\par 5 După aceea au murit amândoi feciorii ei, Mahlon și Chilion și a rămas femeia aceea și fără bărbatul său și fără cei doi feciori ai săi.
\par 6 Atunci s-a hotărât ea cu nurorile sale să se întoarcă din șesul Moabiților, căci auzise ea în șesul Moabiților că Dumnezeu a cercetat pe poporul Său și i-a dat pâine.
\par 7 Deci a plecat ea din locul acela în care trăia, împreună cu cele două nurori ale sale. Dar mergând ele pe cale, pentru a se întoarce în pământul lui Iuda,
\par 8 Noemina a zis către cele două nurori ale sale: "Întoarceți-vă și vă duceți fiecare la casa mamei voastre; și să facă Domnul milă cu voi, cum ați făcut și voi cu cei morți și cu mine!
\par 9 Domnul să vă ajute, ca să vă găsiți adăpost fiecare în casa bărbatului său!" Apoi le-a sărutat; iar ele, începând a se tângui și a plânge,
\par 10 Au zis: "Nu, ci ne vom întoarce împreună la poporul tău!"
\par 11 Noemina însă a zis: "Întoarceți-vă, fiicele mele, de ce să mergeți voi cu mine? Au doară mai am eu feciori în pântecele meu care să vă poată fi bărbați?
\par 12 Întoarceți-vă, fiicele mele, întoarceri-vă, căci eu sunt prea bătrână ca să mă mai mărit. Și chiar de v-aș spune că tot mai am nădejde și chiar dacă la noapte aș avea bărbat și apoi aș naște fii,
\par 13 Ați putea voi oare aștepta până vor crește? Puteți voi oare să întârziați să nu vă măritați? Nu, fiicele mele; mie îmi pare foarte rău de voi, căci mâna Domnului m-a apăsat".
\par 14 Atunci ele din nou și-au ridicat glasul și au început a plânge. Apoi Orfa și-a luat rămas bun de la soacra sa și s-a întors la poporul său, iar Rut a rămas cu ea.
\par 15 Și a zis Noemina către Rut: "Iată cumnata ta s-a întors la poporul său și la dumnezeii săi. Întoarce-te și tu după cumnata ta!"
\par 16 Iar Rut a zis: "Nu mă sili să te părăsesc și să mă duc de la tine; căci unde te vei duce tu, acolo voi merge și eu și unde vei trăi tu, voi trăi și eu; poporul tău va fi poporul meu și Dumnezeul tău va fi Dumnezeul meu;
\par 17 Unde vei muri tu, voi muri și eu și voi fi îngropată acolo. Orice-mi va face Domnul, numai moartea mă va despărți de tine!"
\par 18 Văzând Noemina că este așa de hotărâtă să meargă cu ea, a încetat de a o mai îndemna să se întoarcă.
\par 19 Și au plecat amândouă și au venit la Betleem. Iar dacă au sosit aici, s-a zvonit de ele în toată cetatea și se zicea: "Oare aceasta este Noemina?"
\par 20 Iar ea zicea: "Nu mă mai numiți Noemina, ci numiți-mă Mara, pentru că amărăciune mare mi-a trimis Atotțiitorul.
\par 21 Îndestulată am ieșit eu de aici, iar Domnul m-a întors cu mâinile goale. La ce să mă mai numiți Noemina, când Domnul m-a făcut să sufăr și Atotțiitorul mi-a trimis necaz?"
\par 22 Așa s-a întors Noemina cu nora sa Rut moabiteanca, venind din șesul Moabiților și au intrat în Betleem pe la începutul secerișului orzului.

\chapter{2}

\par 1 Noemina avea rudă după bărbatul său pe un om foarte bogat, din neamul lui Elimelec, al cărui nume era Booz.
\par 2 Și a zis Rut moabiteanca Noeminei: "Mă duc în țarină să adun spice pe urma aceluia la care voi afla trecere". și aceasta a zis către ea: "Du-te, fiica mea!"
\par 3 Și plecând ea, s-a dus în țarină să adune spice de pe urma secerătorilor. Și s-a întâmplat că acea parte de țarină era a lui Booz, din neamul lui Elimelec.
\par 4 Și iată a venit Booz de la Betleem și a zis către secerători: "Domnul să fie cu voi!" Iar aceștia i-au răspuns: "Domnul să te binecuvânteze!"
\par 5 Apoi a zis Booz către sluga sa, care era pusă peste secerători: Cine este această femeie tânără?"
\par 6 Iar sluga care era pusă peste secerători a răspuns și a zis: "Această femeie tânără este moabiteanca aceea care a venit cu Noemina din țara Moabiților.
\par 7 Ea m-a rugat: "Voi culege și voi aduna spice printre snopi pe urma secerătorilor. Și se află aici de azi dimineață și acasă șade foarte puțin".
\par 8 Atunci Booz a zis către Rut: "Ascultă, fiica mea, să nu te duci să strângi în altă țarină și să nu te depărtezi de aici, ci rămâi aici cu slujnicele mele;
\par 9 Să ai înaintea ochilor tăi țarina unde seceră ele și să mergi după ele. Iată am poruncit slugilor mele să nu te atingă. Când vei vrea să bei, mergi și bea de unde beau slugile mele".
\par 10 Și a căzut ea cu fala la pământ și s-a închinat până la pământ și a zis către el: "Cu ce am dobândit eu milă înaintea ta de mă primești, cu toate că sunt străină?"
\par 11 Răspuns-a Booz și i-a zis: "Mie mi s-au spus toate cele ce ai făcut tu cu soacra ta, după moartea bărbatului tău, că ți-ai lăsat pe tatăl tău și pe mama ta și țara ta de naștere și ai venit la poporul pe care nu l-ai cunoscut nici ieri, nici alaltăieri.
\par 12 Să-ți plătească Domnul pentru această faptă a ta și să ai plată deplină de la Domnul Dumnezeul lui Israel, la care ai venit, ca să te adăpostești sub aripile Lui!"
\par 13 Iar ea a zis: "Domnul meu, fie să am milă înaintea ochilor tăi! Tu m-ai mângâiat și ai vorbit după inima roabei tale, deși nu sunt măcar ca una din slujnicele tale!"
\par 14 Atunci Booz a zis către ea: "E vremea prânzului; vino de mănâncă pâine și-ți moaie bucătura în oțet". și a șezut lângă secerători, iar el i-a dat pâine și ea a mâncat și sa săturat și i-a mai și rămas.
\par 15 Apoi s-a sculat și s-a apucat de strâns. Iar Booz a dat poruncă slugilor sale, zicând: "Lăsați-o să adune și printre snopi și să nu o stânjeniți!
\par 16 Ba și din snopi să aruncați și să lăsați pentru ea; lăsați-o să adune și să mănânce; să n-o ocărâți".
\par 17 Și așa a adunat ea în țarină până seara și a bătut cele adunate și i-a ieșit aproape o efă de orz.
\par 18 Și luând aceasta, s-a dus în cetate și soacra sa a văzut ce adunase. Apoi a scos Rut din sin și i-a dat ceea ce-i rămăsese după ce se săturase.
\par 19 Și a zis soacra sa către ea: "Unde ai adunat tu astăzi și unde ai lucrat? Binecuvântat să fie cel ce te-a primit! Și Rut a spus soacrei sale la cine a lucrat și a zis: "Pe omul acela, la care am lucrat astăzi, îl cheamă Booz".
\par 20 Și a zis Noemina către nora sa: "Binecuvântat este el de Domnul, Ca re n-a lipsit de mila Sa nici pe cei vii, nici pe cei morți!" Apoi Noemina a adăugat: "Omul acela e aproape de noi, e una din rudeniile noastre".
\par 21 Și a zis Rut moabiteanca soacrei sale: "El chiar mi-a zis: "Rămâi cu slujnicele mele până când vor isprăvi secerișul meu".
\par 22 A zis Noemina către nora sa Rut: "Este bine, fiica mea, că ai să umbli cu slujnicele lui și nu vei fi stânjenită, ca în altă țarină".
\par 23 Și așa a rămas ea cu slujnicele lui Booz și a adunat spice până când s-a isprăvit secerișul orzului și secerișul griului. Trăia însă la soacra sa.

\chapter{3}

\par 1 După aceea a zis către ea Noemina, soacra sa: "Fiica mea, n-ar fi bine oare să-ți cauți un adăpost, ca să-ți fie bine?
\par 2 Iată Booz, cu ale cărui slujnice ai fost, îmi este rudă și iată el în noaptea aceasta treieră orzul la arie.
\par 3 Spală-te și te unge, îmbracă-ți hainele tale cele bune și du-tț. la arie, dar nu te arăta lui până nu va fi isprăvit de mâncat și de băut.
\par 4 Iar după ce se va culca să doarmă, află locul unde este culcat și fă-ți loc la picioarele lui și te culcă, și el îți va spune ce să faci".
\par 5 Atunci Rut a zis: "Voi face tot ce mi-ai grăit".
\par 6 Ducându-se deci la arie, a făcut toate cum îi poruncise soacra sa.
\par 7 Iar Booz a mâncat, a băut, s-a veselit inima lui și s-a dus de s-a culcat lângă un stog. Iar ea a venit încetișor, și-a făcut loc la picioarele lui și s-a culcat acolo.
\par 8 Pe la miezul nopții însă a tresărit el și s-a ridicat; și iată la picioarele lui o femeie culcată.
\par 9 Și a zis Booz către ea: "Cine ești tu?" Iar ea a zis: "Eu sunt Rut, roaba ta. Întinde-ți aripa ta peste roaba ta, că îmi ești rudă!"
\par 10 Zis-a Booz: "Binecuvântată ești tu de Domnul Dumnezeu, fiica mea! Această de pe urmă faptă bună a ta este încă și mai frumoasă decât celelalte, căci nu te-ai dus să cauți oameni tineri, săraci sau bogați.
\par 11 Deci, fiica mea, nu te teme, îți voi face tot ce ai zis, căci în toate părțile poporului meu se știe că ești femeie vrednică.
\par 12 Adevărat e că îți sunt rudă, dar mai ai o rudă încă și mai aproape decât mine.
\par 13 Rămâi noaptea aceasta aici și mâine, de va vrea acela să te răscumpere, bine, să te răscumpere; iar de nu va vrea să te răscumpere el, te voi răscumpăra eu; viu este Domnul! Dormi aici până mâine!"
\par 14 Și a dormit ea la picioarele lui până dimineața. Apoi s-a sculat înainte de a se fi putut ea cunoaște unul pe altul. Și a zis Booz: "Să nu se știe că a venit femeia la arie!"
\par 15 Iar către ea a zis: "Dezbracă-ți haina ta cea de deasupra și ține-o". Și ea a ținut-o, iar el i-a măsurat șase măsuri de orz și i le-a pus pe umăr și s-a dus în cetate.
\par 16 Atunci a venit Rut la soacra sa și i-a zis: "Ce e, fiica mea?" Și ea i-a povestit tot ce i-a făcut omul acela și a zis:
\par 17 "Aceste șase măsuri de orz mi le-a dat el, zicându-mi: "Să nu te duci la soacra ta cu mâinile goale!"
\par 18 Iar soacra a zis: "Ai răbdare fiica mea, până vei afla cum se va isprăvi lucrul acesta; căci omul acela nu se va liniști până nu va isprăvi chiar astăzi lucrul acesta".

\chapter{4}

\par 1 În ziua aceea a ieșit Booz la poarta cetății și a șezut acolo. Și iată trecea pe acolo ruda de care grăise Booz; și Booz i-a zis: "Vino încoace și șezi aici". Și acela s-a dus și a șezut.
\par 2 Și a luat Booz zece oameni dintre bătrânii cetății și a zis: "Ședeți aici!" și ei au șezut.
\par 3 Apoi a zis Booz către ruda sa: "Noemina, întorcându-se din șesul Moabiților, vinde partea de țarină, cuvenită fratelui nostru Elimelec; și eu m-am hotărât să fac cunoscut auzului tău și să-ți spun: Cumpăr-o în fața celor ce șed aici și în fața bătrânilor poporului meu.
\par 4 De vrei s-o cumperi, cumpăr-o, iar de nu vrei s-o cumperi, spune-mi, ca să știu și eu. Căci afară de tine n-are cine s-o cumpere, iar după tine vin eu". Și acela a zis: "O cumpăr!"
\par 5 Răspuns-a Booz: "De cumperi țarina de la Noemina, atunci trebuie să cumperi și pe Rut moabiteanca, femeia celui mort, și trebuie să o iei de soție, ca să păstrezi numele celui mort în moștenirea lui".
\par 6 Iar ruda aceea a zis: "Nu pot să o iau, ca să nu-mi stric moștenirea mea; ia-o tu, căci eu nu pot să o iau!"
\par 7 Înainte, la facerea unei cumpărături sau a unui schimb, pentru întărirea lucrului, era în Israel obiceiul acesta: unul își descălța sandaua sa și o da celuilalt, care primea dreptul de rudenie mai apropiată și aceasta era mărturie în Israel.
\par 8 Și a zis ruda aceea către Booz: "Cumpăr-o pentru tine!", și și-a descălțat sandaua sa și a dat-o acestuia.
\par 9 Iar Booz a zis către bătrâni și către tot poporul: "Voi sunteți martori astăzi, că eu am cumpărat de la Noemina toate ale lui Elimelec și toate ale lui Chilion și toate ale lui Mahlon.
\par 10 De asemenea și pe Rut moabiteanca, femeia lui Mahlon, o iau de soție, ca să păstrez numele celui mort în moștenirea lui și ca să nu piară numele celui mort dintre frații lui și din poarta locuinței lui; voi astăzi sunteți martori la aceasta".
\par 11 Și tot poporul care era la poartă și bătrânii au zis: "Suntem martori! Să facă Domnul pe femeia care intră în casa ta ca pe Rahila și ca pe Lia, care amândouă au ridicat casă lui Israel. Să câștigi avere în Efrata și numele tău să fie mărit în Betleem.
\par 12 Iar casa ta să fie cum a fost casa lui Fares, pe care l-a născut Tamara lui Iuda, și să se slăvească prin sămânța ce ți-o va da Domnul din această femeie tânără".
\par 13 Și a luat Booz pe Rut și ea s-a făcut soția lui. Și intrând el la ea, Domnul i-a dat ei sarcină și a născut un fiu.
\par 14 Și ziceau femeile către Noemina: "Binecuvântat este Domnul, că nu te-a lăsat fără moștenitor! Slăvit să fie numele lui Israel!
\par 15 Acesta îți va fi bucurie și hrănitor la bătrânețile tale, căci l-a născut nora ta, care este mai bună pentru tine decât șapte fii".
\par 16 Și a luat Noemina pe copilul acesta și l-a purtat în brațele sale și i-a fost doică.
\par 17 Iar vecinele i-au pus nume și au zis: "Noeminei i s-a născut fiu și i s-a pus numele Obed". Acesta este părintele lui Iesei, tatăl lui David.
\par 18 Iată acum spița neamului lui Fares: lui Fares i s-a născut Esron:
\par 19 Lui Esron i s-a născut Aram; lui Aram i s-a născut Aminadab;
\par 20 Lui Aminadab i s-a născut Naason; lui Naason i s-a născut Salmon;
\par 21 Lui Salmon. i s-a născut Booz; lui Booz i s-a născut Obed; lui Obed i s-a născut Iesei;
\par 22 Lui Iesei i s-a născut David.


\end{document}