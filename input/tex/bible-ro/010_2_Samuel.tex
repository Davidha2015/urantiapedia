\begin{document}

\title{2 Samuel}


\chapter{1}

\par 1 Dupa moartea lui Saul, când David se întorcea din razboiul ce-l purtase împotriva Amaleci?ilor ?i dupa ce s-a oprit doua zile în ?iclag,
\par 2 Iata ca a treia zi a venit un om din tabara lui Saul. Acesta avea pe el hainele rupte ?i pe cap ?arâna ?i dupa ce a ajuns la David, a cazut la pamânt ?i i s-a închinat.
\par 3 Iar David i-a zis: "De unde vii tu?" "Am fugit din tabara lui Israel", raspunse acela.
\par 4 "Spune-mi, a zis David, ce s-a întâmplat?" "Poporul, a zis omul, a fugit din lupta ?i mul?ime din popor a cazut ?i a murit, ?i a murit ?i Saul ?i fiul sau Ionatan".
\par 5 "Cum ?tii ca Saul ?i fiul sau Ionatan au murit?" a întrebat David pe tânarul care-i adusese vestea.
\par 6 "Din întâmplare, m-am dus pe muntele Ghelboa, a zis omul ce-i graia, ?i iata Saul statea sprijinit pe suli?a sa, iar carele ?i calare?ii îl ajungeau.
\par 7 Atunci el s-a întors ?i, vazându-ma pe mine, m-a strigat ?i eu am raspuns: "Iata-ma!"
\par 8 ?i el mi-a zis: "Cine e?ti tu?" ?i eu i-am raspuns: "Sunt un amalecit!"
\par 9 Atunci el mi-a zis: "Apropie-te de mine ?i ma ucide, ca durerea mor?ii m-a cuprins ?i sufletul meu este înca tot în mine!"
\par 10 ?i eu m-am apropiat ?i l-am ucis, caci ?tiam ca n-are sa mai traiasca dupa caderea sa. Apoi am luat cununa regeasca de pe capul lui, bra?ara ce era la mâna lui ?i le-am adus aici la stapânul meu".
\par 11 Atunci a apucat David hainele sale ?i le-a rupt; asemenea ?i oamenii cei ce erau cu el ?i-au rupt hainele lor,
\par 12 ?i au plâns ?i s-au tânguit ?i au postit pâna seara dupa Saul ?i dupa fiul sau Ionatan, dupa poporul Domnului ?i dupa casa lui Israel, care cazusera de sabie.
\par 13 Apoi David a zis catre omul care-i spusese acestea: "De unde e?ti tu?" "Eu, a zis acela, sunt fiul unui strain amalecit".
\par 14 ?i David a zis: "Cum nu te-ai temut tu sa-li ridici mâna, ca sa ucizi pe unsul Domnului?"
\par 15 Apoi chemând David pe unul din slujitori, i-a zis: "Vino de-l ucide!" ?i l-a ucis.
\par 16 Iar David a zis catre el: "Sângele tau sa fie pe capul tau, caci buzele tale au marturisit împotriva ta, când ai zis: Eu am ucis pe unsul Domnului".
\par 17 ?i a plâns David pe Saul ?i pe Ionatan, fiul lui, prin aceasta cântare de jale,
\par 18 Pe care a poruncit sa o înve?e fiii lui Iuda, dupa cum este scrisa în cartea dreptului, unde se zice:
\par 19 "Podoaba ta, Israele, a fost doborâta pe înal?imile tale! Cum au cazut vitejii!
\par 20 Nu vesti?i în Gat ?i nu da?i de ?tire pe uli?ele Ascalonului, ca sa nu se bucure fiicele Filistenilor, ca sa nu praznuiasca fiicele celor netaia?i împrejur!
\par 21 Mun?ilor Ghelboa, sa nu mai cada pe voi nici roua, nici ploaie; ?i sa nu mai fie pe voi ?arini roditoare, caci acolo a fost doborât scutul celor razboinici, scutul lui Saul, ca ?i cum n-ar fi fost uns cu untdelemn sfin?it.
\par 22 Arcul lui Ionatan nu se întorcea fara sânge de rani?i, fara grasimea celor puternici, nici sabia lui Saul nu se învârtea zadarnic!
\par 23 Saul ?i Ionatan, cei iubi?i ?i uni?i în via?a lor, nici la moarte nu s-au despar?it. Fost-au mai iu?i decât vulturii ?i mai puternici decât leii!
\par 24 Fiicele lui Israel, plânge?i pe Saul, cel ce v-a îmbracat în purpura cu podoabe ?i v-a pus pe haine podoabe de aur!
\par 25 Cum au cazut vitejii în toiul luptei! Ucis a fost Ionatan pe înal?imile tale, Israele!
\par 26 Frate Ionatane, întristat sunt dupa tine, caci tu mi-ai fost foarte scump ?i iubirea ta a fost pentru mine mai presus de iubirea femeiasca!
\par 27 Cum au cazut cei viteji! Pierit-a arma de razboi!"

\chapter{2}

\par 1 Dupa aceasta David a întrebat pe Domnul ?i a zis: "Sa ma duc, oare, în vreuna din ceta?ile lui Iuda?" "Du-te!", i-a zis Domnul. "Unde sa ma duc?", a întrebat iara?i David. "În Hebron", i s-a raspuns.
\par 2 ?i s-a dus acolo David ?i cele doua femei ale sale, Ahinoam izreeliteanca ?i Abigail carmeliteanca, fosta femeie a lui Nabal.
\par 3 ?i pe oamenii cei ce fusesera cu el i-a adus David pe fiecare cu familia lui ?i s-a a?ezat în cetatea Hebron.
\par 4 Atunci au venit barba?ii lui Iuda ?i au uns acolo pe David rege pentru casa lui Iuda, spunându-i-se lui David ca locuitorii din Iabe?ul Galaadului au îngropat pe Saul.
\par 5 Atunci a trimis David soli la locuitorii din Iabe?ul Galaadului, ca sa le spuna: "Binecuvânta?i sunte?i voi de Domnul, pentru ca a?i aratat mila lui Saul, domnul vostru ?i unsul Domnului, ?i l-a?i îngropat pe el ?i pe fiul lui, Ionatan.
\par 6 Sa va rasplateasca dar Domnul cu mila ?i cu credincio?ie, ?i eu va voi face bine, pentru ca a?i facut aceasta.
\par 7 Sa se întareasca mâinile voastre ?i sa fi?i curajo?i, caci stapânul vostru Saul a murit ?i casa lui luda m-a uns pe mine rege peste voi.
\par 8 Dar Abner, fiul lui Ner, capetenia o?tirilor lui Saul, a luat pe I?bo?et, fiul lui Saul, ?i l-a dus din tabara lui la Mahanaim
\par 9 ?i l-a facut rege peste Galaad, A?er, Izreel, Efrem, Veniamin ?i peste tot Israelul.
\par 10 I?bo?et însa, fiul lui Saul, era ca de patruzeci de ani când s-a facut rege peste Israel ?i a domnit doi ani; iar cu David a ramas numai casa lui Iuda.
\par 11 Tot timpul cât a domnit David în Hebron peste casa lui Iuda, au fost ?apte ani ?i ?ase luni.
\par 12 Atunci Abner, fiul lui Ner, ?i slugile lui I?bo?et, fiul lui Saul, au ie?it din Mahanaim la Ghibeon ?i a ie?it ?i Ioab, fiul ?eruiei, cu slugile lui David ?i s-au întâlnit la iazul Ghibeonului.
\par 13 ?i s-au a?ezat unii de o parte de iaz, iar al?ii de cealalta parte a iazului.
\par 14 ?i a zis Abner catre Ioab: "Sa se scoale tinerii ?i sa joace înaintea noastra!" "Sa se scoale!", a zis Ioab.
\par 15 ?i s-au sculat ?i s-au dus un numar de doisprezece Veniamineni din partea lui I?bo?et, fiul lui Saul, ?i doisprezece dintre slugile lui David.
\par 16 ?i s-au apucat de cap unul pe altul ?i ?i-au înfipt sabia unul altuia în coasta ?i au cazut împreuna. ?i s-a numit locul acela Helcat-Ha?urim, (Locul Sabiilor), care se afla în Ghibeon.
\par 17 Apoi s-a dat în ziua aceea batalia cea mai crâncena ?i Abner cu oamenii lui Israel au fost batu?i de slugile lui David.
\par 18 Acolo se aflau trei feciori ai ?eruiei: Ioab, Abi?ai ?i Asael. Asael însa era sprinten de picior, ca o caprioara de câmp.
\par 19 ?i a alergat Asael dupa Abner ?i l-a urmarit, fara sa se abata nici la dreapta, nici la stânga de pe urmele lui.
\par 20 ?i uitându-se Abner înapoi, a zis: "Asael, tu e?ti oare?" "Eu!" a zis acesta.
\par 21 "Abate-te la dreapta sau la stânga, a zis Abner, ?i alege-?i unul din oameni ?i ia armele lui!"
\par 22 Dar Asael nu a voit sa se lase de el. ?i a zis iara?i Abner catre Asael: "Lasa-te de mine, ca sa nu te dobor la pamânt. Atunci cu ce obraz ma voi arata înaintea lui Ioab, fratele tau? Ce este aceasta? întoarce-te la fratele tau!"
\par 23 Dar acela nu a voit sa se lase. Atunci Abner, întorcându-?i lancea, l-a lovit în pântece ?i lancea a trecut printr-însul ?i el a cazut chiar acolo ?i a murit pe loc. ?i to?i cei ce treceau pe la locul unde cazuse ?i murise Asael se opreau.
\par 24 Ioab ?i Abi?ai înca urmareau pe Abner. Soarele asfin?ise când au sosit ace?tia la dealul Amma ce vine în fa?a Ghiahului, pe calea ce duce spre pustiul Ghibeonului.
\par 25 ?i atunci, adunându-se Veniaminenii împrejurul lui Abner, au alcatuit o o?tire ?i au stat pe vârful unui deal,
\par 26 De unde a strigat Abner catre Ioab ?i a zis: "Oare mereu va sfâ?ia sabia? Dare nu ?tii tu ca urmarile au sa fie dureroase? Pâna când nu vei zice oamenilor sa înceteze de a mai urmari pe fra?ii lor?"
\par 27 "Viu este Dumnezeu, a zis Ioab, daca tu ne-ai fi grait altfel, înca de diminea?a ar fi încetat oamenii mei de a mai urmari pe fra?ii lor".
\par 28 Apoi a sunat Ioab din trâmbi?a ?i tot poporul s-a oprit ?i mai mult n-au mai urmarit pe Israeli?i, încetând lupta.
\par 29 Abner însa ?i oamenii lui au mers pe ?es toata noaptea aceea ?i au trecut Iordanul ?i, strabatând tot Bitronul, au venit la Mahanaim.
\par 30 Ioab însa s-a întors din urmarirea lui Abner ?i a adunat tot poporul ?i au lipsit de la numar, dintre oamenii lui David, nouasprezece in?i, precum ?i Asael.
\par 31 Iar slugile lui David, lovind pe Veniamineni ?i pe oamenii lui Abner, au cazut din ace?tia trei sute ?aizeci de oameni.
\par 32 ?i luând pe Asael, l-au îngropat în mormântul tatalui sau, ce se afla în Betleem. Apoi Ioab cu oamenii sai au mers toata noaptea ?i în revarsatul zorilor au ajuns la Hebron.

\chapter{3}

\par 1 ?i a ?inut multa vreme du?mania între casa lui Saul ?i casa lui David. David însa se întarea mereu, iar casa lui Saul slabea din ce în ce mai mult.
\par 2 Lui David i s-au nascut ?ase fii în Hebron. Întâiul sau nascut a fost Amnon din Ahinoam, izreeliteanca.
\par 3 Al doilea fiu al lui a fost Chileab din Abigail carmeliteanca, fosta femeie a lui Nabal. Al treilea a fost Abesalom, fiul Maachei, fiica lui Talmai, regele Ghe?urului.
\par 4 Al patrulea a fost Adonia, fiul Haghitei. Al cincilea a fost ?efatia, fiul Abitalei.
\par 5 Iar al ?aselea a fost Itream din Egla, femeia lui David. Ace?tia i s-au nascut lui David în Hebron.
\par 6 Pe când era du?manie între casa lui Saul ?i casa lui David, Abner ?inea cu casa lui Saul.
\par 7 Saul avusese o concubina cu numele Ri?pa, fiica lui Aia. Abner a intrat la ea, iar I?bo?et a zis catre Abner: "La ce ai intrat tu la concubina tatalui meu?"
\par 8 Abner însa, mâniindu-se stra?nic de vorbele lui I?bo?et, a zis: "Au doara eu sunt cap de câine? Eu, împotriva casei lui Iuda, am aratat acum mila casei lui Saul, tatal tau, fra?ilor lui ?i prietenilor lui, ?i nu te-am dat în mâinile lui David; iar tu îmi gase?ti acum vina pentru o femeie?
\par 9 A?a ?i a?a sa faca Dumnezeu lui Abner ?i înca ?i mar mult sa-i faca! Precum s-a jurat Domnul lui David, tocmai a?a voi ?i face în ziua aceasta:
\par 10 Voi lua domnia de la casa lui Saul ?i voi pune tronul lui David peste ca?a lui Israel ?i peste Iuda, de la Dan pâna la Beer-?eba".
\par 11 ?i n-a putut I?bo?et sa raspunda lui Abner nimic, caci se temea de el.
\par 12 Abner însa a trimis din partea sa vestitori la David în Hebron, unde se afla el, sa-i zica: "Al cui este pamântul acesta?", ?i sa-i mai zica înca: "Încheie legamânt cu mine ?i mâna mea va fi cu tine, ca sa întoarca la tine pe tot poporul lui Israel!"
\par 13 Iar David a raspuns: "Bine, voi încheia legamânt cu tine; dar te rog un lucru anume: nu vei vedea fa?a mea, daca nu vei aduce cu tine ?i pe Micol, fiica lui Saul, când vei veni sa te vezi cu mine".
\par 14 Apoi a trimis David soli la I?bo?et, fiul lui Saul, sa-i zica: "Da-mi pe femeia mea, Micol, pe care am luat-o de femeie pentru o suta de prepu?uri filistene".
\par 15 ?i a trimis I?bo?et ?i a luat-o de la barbatul ei, de la Paltiel, fiul lui Lai?;
\par 16 ?i s-a dus cu ea ?i barbatul ei ?i a petrecut-o cu plângere pâna la Bahurim. Dar Abner a zis catre el: "Du-te înapoi!" ?i acela s-a întors.
\par 17 Atunci s-a întors Abner catre batrânii lui Israel ?i a zis: "Voi ?i ieri ?i alaltaieri a?i dorit ca David sa fie rege peste voi;
\par 18 Acum face?i aceasta, caci Domnul a zis catre David: Prin mâna robului Meu David voi izbavi poporul Meu Israel din mâna Filistenilor ?i din mâna tuturor vrajma?ilor lui".
\par 19 La fel a grait Abner ?i Veniaminenilor. S-a dus apoi Abner la Hebron, ca sa spuna lui David tot ce dorea Israel ?i toata casa lui Veniamin.
\par 20 ?i a venit Abner la David în Hebron, ?i cu el au venit ?i douazeci de oameni, ?i a facut David ospa? pentru Abner ?i pentru înso?itorii lui.
\par 21 Abner a zis lui David: "Eu ma voi scula ?i ma voi duce ?i voi aduna la regele, stapânul meu, tot poporul lui Israel ca sa faca legamânt cu tine ?i vei fi rege peste to?i, dupa cum dore?te sufletul tau". Iar David a dat drumul lui Abner ?i s-a dus cu pace.
\par 22 ?i iata slugile lui David cu Ioab au venit de la batalie ?i au adus cu ei prada multa. Dar Abner nu mai era cu David în Hebron, caci David îi daduse drumul ?i se dusese cu pace.
\par 23 ?i când Ioab ?i toata o?tirea lui au venit, i s-a spus lui Ioab: "Abner, fiul lui Ner, a venit la rege ?i acesta i-a dat drumul de s-a dus cu pace".
\par 24 Atunci a venit Ioab la rege ?i a zis: "Ce ai facut? Iata a venit la tine Abner; de ce n-ai dat drumul sa plece?
\par 25 Tu ?tii pe Abner, fiul lui Ner; el a venit sa te în?ele ?i sa afle pe unde intri ?i pe unde ie?i ?i sa cunoasca tot ceea ce faci tu".
\par 26 Ie?ind apoi Ioab de la David, a trimis oameni dupa Abner ?i l-au întors ace?tia de la fântâna Sira, fara ?tirea lui David.
\par 27 ?i când Abner s-a întors la Hebron, Ioab l-a bagat pe poarta înauntru, ca ?i cum ar fi vrut sa vorbeasca cu el în taina, ?i acolo l-a lovit în pântece. ?i a murit Abner pentru sângele lui Asael, fratele lui Ioab.
\par 28 Auzind în urma David de aceasta, a zis: "Nevinovat sunt eu ?i regatul meu în veac înaintea Domnului de sângele lui Abner, fiul lui Ner; cada el pe capul lui Ioab ?i peste toata casa tatalui sau;
\par 29 Niciodata sa nu lipseasca din casa lui Ioab cei ce patimesc de scurgere, cei lepro?i, cei ce merg în cârji, cei omorâ?i de sabie ?i cei lipsi?i de pâine".
\par 30 Ioab însa ?i fratele sau Abi?ai ucisesera pe Abner, pentru ca acesta omorâse pe fratele lor Asael în lupta de la Ghibeon.
\par 31 David însa a zis catre Ioab ?i catre to?i oamenii care erau cu el: "Rupe?i-va hainele ?i va încinge?i cu sac ?i jeli?i pe Abner!"
\par 32 Apoi regele David a mers dupa sicriul lui ?i, când a fost îngropat Abner în Hebron, regele a plâns tare la mormântul lui Abner ?i a plâns ?i tot poporul.
\par 33 ?i a zis regele, când plângea pe Abner:
\par 34 "Cum sa moara Abner ca un rau? Mâinile nu ?i-au fost legate, nici picioarele nu-?i erau încatu?ate, ci ai cazut ca cei doborâ?i de tâlhari!"
\par 35 Atunci tot poporul a început sa plânga înca ?i mai tare dupa el. ?i a venit tot poporul sa aduca lui David pâine, când înca era ziua; însa David s-a jurat, zicând: "A?a ?i a?a sa faca Dumnezeu cu mine ?i înca ?i mai mult sa faca, de voi gusta pâine sau altceva înainte de asfin?itul soarelui".
\par 36 ?i a aflat de aceasta tot poporul ?i i-a placut aceasta, cum placea întregului popor tot ceea ce facea regele.
\par 37 ?i a aflat în ziua aceea tot poporul ?i tot Israelul ca nu din pricina regelui s-a savâr?it uciderea lui Abner, fiul lui Ner.
\par 38 ?i a zis regele catre slugile sale: "?ti?i voi oare ca astazi a cazut în Israel un barbat ?i o capetenie mare?
\par 39 Eu astazi sunt înca slab, de?i sunt uns rege; iar oamenii ace?tia, fiii ?eruiei, sunt mai tari decât mine. Sa rasplateasca deci Domnul celui ce face rau dupa rautatea lui!"

\chapter{4}

\par 1 Auzind I?bo?et, fiul lui Saul, ca a murit Abner în Hebron, i-au slabit mâinile ?i tot Israelul s-a tulburat.
\par 2 I?bo?et, fiul lui Saul, avea doua capetenii de o?tire; numele unuia era Baana ?i numele celuilalt era Rechab, feciorii lui Rimon Beeroteanul, din urma?ii lui Veniamin, caci ?i Beerotul se socotea al lui Veniamin.
\par 3 ?i au fugit Beerotenii la Ghitaim ramânând acolo ca straini pâna azi.
\par 4 De la Ionatan, fiul lui Saul, ramasese ura fiu ?chiop. Acesta era de cinci ani, când a venit din Israel vestea despre moartea lui Saul ?i a lui Ionatan, iar doica lui l-a luat ?i a fugit. Dar pe când fugea ea grabita, el a cazut ?i a ramas ?chiop. Numele lui era Mefibo?et.
\par 5 Atunci au plecat Rechab ?i Baana, fiii lui Rimon Beeroteanul, ?i au venit în casa lui I?bo?et chiar în ar?i?a zilei; acesta însa dormea de amiaza în patul sau.
\par 6 Iar portarului casei, care cura?ise grâu, îi venise somn ?i adormise. Atunci Rechab ?i Baana, fratele sau, au intrat în casa, ca ?i-cum ar fi vrut sa ia grâu, ?i l-au lovit pe I?bo?et în stomac ?i apoi au fugit.
\par 7 ?i când intrasera ei în casa, I?bo?et dormea în patul sau, în odaia sa de dormit, ?i ei l-au lovit ?i l-au omorât ?i i-au taiat capul ?i au luat capul lui cu ei ?i au mers prin câmpie toata noaptea
\par 8 ?i au adus capul lui I?bo?et la David, în Hebron, ?i au zis catre rege: "Iata capul lui I?bo?et, fiul lui Saul, du?manul tau, care a cautat sufletul tau. Acum Domnul a razbunat pe domnul meu regele, împotriva lui Saul, vrajma?ul tau, ?i împotriva urma?ilor lui".
\par 9 ?i raspunzând David lui Rechab ?i lui Baana, fratele lui, feciorii lui Rimon Beeroteanul, le-a zis: "Viu este Domnul, Care a izbavit sufletul meu din tot necazul,
\par 10 Ca daca pe cel ce mi-a adus vestea ?i a zis: "Iata a murit Saul ?i Ionatan", ?i se socotea pe sine vestitor de bucurie, eu, în loc sa-l rasplatesc, l-am prins ?i l-am ucis în ?iclag,
\par 11 Apoi acum, când ni?te oameni netrebnici au ucis un om nevinovat în casa lui ?i în patul lui, oare nu voi cere sângele lui din mâinile voastre ?i nu va voi stârpi de pe pamânt?"
\par 12 ?i a poruncit David slugilor ?i i-au ucis, ?i le-au taiat mâinile ?i picioarele ?i le-au spânzurat deasupra iazului din Hebron. Iar capul lui I?bo?et l-au luat ?i l-au îngropat în mormântul lui Abner, în Hebron.

\chapter{5}

\par 1 Atunci au venit toate triburile lui Israel la David în Hebron ?i au zis:
\par 2 "Iata, noi suntem oasele tale ?i carnea ta. Înca de pe când Saul domnea peste noi, tu ai pova?uit pe Israel ?i Domnul a zis catre tine: Tu vei pa?te pe poporul Meu Israel ?i tu vei fi pova?uitorul lui Israel".
\par 3 Au venit to?i batrânii lui Israel la rege în Hebron ?i a încheiat cu ei regele David legamânt în Hebron, înaintea Domnului; ?i au uns pe David rege peste tot Israelul.
\par 4 David însa era ca de treizeci de ani când s-a facut rege ?i a domnit patruzeci de ani.
\par 5 în Hebron a domnit peste Iuda ?apte ani ?i ?ase luni, iar în Ierusalim a domnit treizeci ?i trei de ani peste tot Israelul ?i peste Iuda.
\par 6 Atunci au pornit regele ?i oamenii lui la Ierusalim, împotriva Iebuseilor, locuitorii ?arii aceleia. Dar ace?tia au zis catre David: "Nu vei intra aici, caci te vor goni orbii ?i ?chiopii care zic: David nu va intra aici!"
\par 7 David însa a luat cetatea Sionului; aceasta este cetatea lui David.
\par 8 ?i a zis David în ziua aceea: "Tot cel ce va ucide pe Iebusei sa loveasca cu lancea ?i pe ?chiopii ?i pe orbii care urasc sufletul lui David". De aceea se ?i zice: "Orbul ?i ?chiopul nu vor intra în casa Domnului!"
\par 9 Atunci s-a mutat David în cetate ?i a numit-o cetatea lui David; ?i a facut întarituri de jur împrejur, de la Milo ?i pâna înauntru.
\par 10 ?i a propa?it David ?i s-a înal?at, ?i Domnul Dumnezeul Savaot era cu el.
\par 11 În vremea aceea a trimis Hiram, regele Tirului, soli la David ?i lemn de cedru, tâmplari, pietrari ?i zidari, ?i ace?tia au facut casa lui David.
\par 12 ?i a în?eles David ca Domnul l-a întarit rege peste Israel ?i a înal?at regatul sau din pricina poporului sau Israel.
\par 13 ?i ?i-a mai luat David femei ?i concubine din Ierusalim, dupa ce a venit din Hebron.
\par 14 ?i i s-au mai nascut lui David fii ?i fiice. Iata ?i numele celor ce i s-au nascut în Ierusalim: ?amua ?i ?obab, Natan ?i Solomon;
\par 15 Ibhar ?i Eli?ua, Nefeg ?i Iafia;
\par 16 Eli?ama, Eliada ?i Elifelet; Samae, Iesivat, Natan, Galamaan, Ievaar, Teisus, Elfalat, Naged, Nafec, Ionatan, Leasamis, Baalimat ?i Elifaat.
\par 17 Iar daca au auzit Filistenii ca David a fost uns rege peste Israel, s-au ridicat Filistenii cu to?ii sa caute pe David. Auzind însa David, s-a dus în cetate;
\par 18 Iar Filistenii au venit ?i s-au a?ezat în valea Refaim.
\par 19 ?i a întrebat David pe Domnul, zicând: "Sa ma duc oare împotriva Filistenilor? Îi vei da Tu, oare, în mâinile mele?" "Du-te, a zis Domnul catre David, caci Eu voi da pe Filisteni în mâinile tale!"
\par 20 Atunci s-a dus David la Baal-Pera?im ?i i-a lovit acolo ?i a zis David: "Domnul a maturat pe vrajma?ii mei dinaintea mea, ca ?i cum i-ar fi luat apa". ?i de aceea s-a ?i dat locului aceluia numele de Baal-Pera?im.
\par 21 Filistenii însa ?i-au lasat acolo idolii lor, iar David ?i oamenii sai i-au luat ?i a poruncit sa-i arda cu foc.
\par 22 Dar Filistenii au navalit iara?i ?i s-au a?ezat în valea Refaim.
\par 23 David a întrebat din nou pe Domnul, zicând: "Sa ma duc oare împotriva Filistenilor ?i îi vei da Tu oare în mâinile mele?" "Sa nu ie?i înaintea lor, i-a raspuns El, ci i-ai pe la spate ?i înainteaza spre ei dinspre dumbrava murelor;
\par 24 ?i când vei auzi un zgomot, ca ?i cum ar veni pe vârful arborilor dumbravilor, atunci sa porne?ti, caci atunci a pornit Domnul înaintea ta, ca sa loveasca o?tirea Filistenilor".
\par 25 ?i a facut David cum i-a poruncit Domnul ?i a lovit pe Filisteni de la Ghibeon pâna la Ghezer.

\chapter{6}

\par 1 Dupa aceea a adunat David din nou pe to?i ale?ii sai din Israel, ca la treizeci de mii.
\par 2 ?i, David, cu tot poporul care era cu el, a pornit la Baalat în Iuda, ca sa aduca chivotul Domnului, asupra caruia este chemat numele Domnului Savaot Cel ce ?ade pe heruvimi.
\par 3 ?i punând chivotul Domnului într-un car nou, l-au scos din casa lui Aminadab; iar fiii lui Aminadab, Uza ?i Ahio, duceau carul cel nou.
\par 4 ?i l-au adus cu chivotul Domnului din casa lui Aminadab cea de pe deal ?i Ahio mergea înaintea chivotului Domnului.
\par 5 Iar David ?i to?i fiii lui Israel cântau înaintea Domnului din tot felul de instrumente muzicale de lemn de chiparos, din harpe, din psaltire, din timpane, din fluiere ?i din chimvale.
\par 6 Când însa au ajuns la aria lui Nachon, Uza ?i-a întins mâinile sale spre chivotul Domnului ca sa-l sprijine, ?i s-a apucat de el, caci boii erau gata sa-l rastoarne.
\par 7 Domnul însa s-a mâniat pe Uza ?i l-a lovit Dumnezeu chiar acolo pentru îndrazneala lui ?i a murit el acolo linga chivotul Domnului.
\par 8 Atunci s-a întristat David ca a lovit Domnul pe Uza ?i locul acesta ?i pâna astazi se cheama Perei-Uza.
\par 9 ?i s-a temut David în ziua aceea de Domnul ?i a zis: "Cum va intra chivotul Domnului la mine?"
\par 10 ?i n-a voit David sa duca chivotul Domnului la sine, în cetatea lui David, ci l-au întors în casa lui Obed-Edom Gateanul.
\par 11 ?i a ramas chivotul Domnului în casa lui Obed-Edom Gateanul trei luni ?i a binecuvântat Domnul pe Obed-Edom ?i toata casa lui.
\par 12 Iar când i s-a spus regelui David ?i i s-a zis: "Domnul a binecuvântat casa lui Obed-Edom ?i toate câte erau ale lui pentru chivotul Domnului", atunci s-a dus David ?i a adus cu alai chivotul Domnului din casa lui Obed-Edom în cetatea lui David.
\par 13 Dar când cei ce duceau chivotul Domnului faceau câte ?ase pa?i, el aducea jertfa un vi?el ?i un berbec.
\par 14 ?i David dan?uia cât putea înaintea Domnului ?i era îmbracat cu efod de in.
\par 15 A?a a adus David ?i tot poporul chivotul Domnului cu strigate ?i cu sunete de trâmbi?a.
\par 16 Iar când a intrat chivotul Domnului în cetatea lui David, Micol, fiica lui Saul, se uita pe fereastra ?i, vazând pe regele David sarind ?i jucând înaintea Domnului, l-a dispre?uit în inima sa.
\par 17 ?i au dus chivotul Domnului ?i l-au pus la locul lui, în mijlocul cortului, pe care-l facuse pentru el David; apoi David a adus arderi de tot înaintea Domnului ?i jertfe de împacare.
\par 18 Iar daca a ispravit David de adus arderile de tot ?i jertfele de împacare, a binecuvântat poporul în numele Domnului Savaot.
\par 19 ?i a împar?it la tot poporul, la toata mul?imea lui Israel de la Dan pâna la Beer-?eba, fiecaruia, atât barba?ilor cât ?i femeilor, câte o pâine ?i câte o bucata de carne fripta ?i câte o turta. ?i s-a dus tot poporul, mergând fiecare la casa sa.
\par 20 Dar când s-a întors David ca sa binecuvânteze casa sa, atunci Micol, fiica lui Saul, i-a ie?it întru întâmpinare ?i i-a zis: "Câta cinste ?i-a facut azi regele lui Israel, descoperindu-se înaintea ochilor roabelor ?i robilor sai, cum se descopera un om de nimic".
\par 21 "Înaintea Domnului voi juca, a zis David catre Micol; binecuvântat este Domnul, Cel ce m-a ales pe mine în locul tatalui tau ?i a casei lui întregi, întarindu-ma cârmuitor al poporului Domnului, Israel; cânta-voi ?i voi juca înaintea Domnului.
\par 22 ?i înca ?i mai mult ma voi înjosi ?i voi fi înca ?i mai mic în ochii tai, iar înaintea slujnicilor de care graie?ti tu, voi fi în cinste".
\par 23 ?i Micol, fiica lui Saul, n-a avut copii pâna în ziua mor?ii ei.

\chapter{7}

\par 1 Pe când regele traia în casa sa ?i-l lini?tise Domnul dinspre to?i vrajma?ii sai de primprejur,
\par 2 A zis regele catre proorocul Natan: "Iata, eu locuiesc în casa de cedru, iar chivotul Domnului sta în cort".
\par 3 "Tot ce ai la inima, a zis Natan catre rege, mergi ?i fa, caci Domnul este cu tine!"
\par 4 ?i chiar în noaptea aceea a fost cuvântul Domnului catre Natan, zicând:
\par 5 "Mergi ?i spune robului Meu David: A?a graie?te Domnul: Tu oare ai sa-Mi zide?ti casa pentru locuin?a Mea,
\par 6 Când Eu n-am locuit în casa din timpul în care am scos pe fiii lui Israel din Egipt ?i pâna astazi, ci am trecut din cort în cort?
\par 7 Pe oriunde am umblat cu to?i fiii lui Israel, am spus Eu, oare, macar o vorba cuiva din semin?ii, caruia i-am încredin?at sa pastoreasca poporul Meu Israel ?i am zis Eu oare: Pentru ce nu-Mi face?i casa de cedru?
\par 8 ?i acum a?a sa zici robului Meu David: A?a zice Domnul Savaot: Te-am luat de la stâna, de la oi, ca sa fii pova?uitorul poporului Meu Israel;
\par 9 Am fost cu tine pretutindeni; oriunde ai umblat, am stârpit pe to?i vrajma?ii tai dinaintea fe?ei tale ?i am facut numele tau mare, ca numele celor mari de pe pamânt.
\par 10 Voi tocmi loc pentru poporul Meu, pentru Israel, îl voi înradacina ?i va trai el în pace la locul sau ?i mai mult nu se va mai nelini?ti; oamenii necredincio?i nu-l vor mai strâmtora, ca mai înainte,
\par 11 Pe vremea când puneam judecatori peste poporul Meu Israel. Ba te voi lini?ti ?i pe tine dinspre vrajma?ii tai.
\par 12 Iata Domnul î?i veste?te ca-?i va întari casa, iar când se vor împlini zilele tale ?i vei raposa cu parin?ii tai, atunci voi ridica dupa tine pe urma?ul tau, care va rasari din coapsele tale ?i voi întari stapânirea sa.
\par 13 Acela va zidi casa numelui Meu ?i Eu voi întari scaunul domniei lui în veci.
\par 14 Eu voi fi aceluia tata, iar el Îmi va fi fiu; de va gre?i, îl voi pedepsi Eu cu toiagul barba?ilor ?i cu loviturile fiilor oamenilor,
\par 15 Dar mila Mea nu o voi lua de la el cum am luat-o de la Saul, pe care l-am lepadat înaintea fe?ei tale.
\par 16 Casa ta va fi neclintita, regatul tau va ramâne ve?nic înaintea ta ?i tronul tau va sta în veci".
\par 17 Toate cuvintele acestea ?i toata vedenia aceasta le-a spus Natan lui David.
\par 18 Atunci s-a dus regele David ?i, stând înaintea fe?ei Domnului, a zis: "Cine sunt eu, Doamne Dumnezeul meu, ?i ce este casa mea, de m-ai marit a?a?
\par 19 Ba înca aceasta s-a parut lucru mic în ochii Tai, Doamne Dumnezeul meu, ?i ai mai vestit înca ?i de viitorul casei robului Tau! Este aceasta, oare lucru omenesc, Doamne Dumnezeul Meu?
\par 20 Ce mai poate sa-?i spuna David? Tu ?tii pe robul Tau, Doamne Dumnezeule!
\par 21 Pentru cuvântul Tau ?i dupa inima Ta faci aceasta, descoperind toata marirea aceasta robului Tau.
\par 22 În toate e?ti mare, Doamne Dumnezeule, caci nu este asemenea ?ie ?i nu este Dumnezeu afara de Tine, dupa toate câte am auzit noi cu urechile noastre.
\par 23 Cine este asemenea poporului Tau Israel, singurul popor de pe pamânt, pentru care a venit Dumnezeu, ca sa ?i-I câ?tige de popor, sa-?i preaslaveasca numele Lui ?i sa savâr?easca lucruri mari ?i minunate, înaintea poporului Tau, pe care Tu ?i l-ai câ?tigat de la Egipteni, izgonind popoarele ?i zeii lor?
\par 24 ?i Tu ?i-ai întarit pe poporul Tau Israel, ca popor al Tau pe veci, ?i Tu; Doamne, Te-ai facut Dumnezeul lui.
\par 25 ?i acum, Doamne Dumnezeule, întare?te pe veci cuvântul pe care l-ai rostit despre robul Tau ?i despre casa lui ?i împline?te ceea ce i-ai sortit,
\par 26 Ca sa preaînal?e numele Tau în veci ?i sa se zica: Domnul Savaot este Dumnezeu peste Israel. Casa robului Tau David sa fie tare înaintea fe?ei Tale.
\par 27 De vreme ce Tu, Doamne Savaot, Dumnezeul lui Israel, ai descoperit robului Tau, zicând: "Î?i voi face casa", apoi robul Tau ?i-a gatit inima sa, ca sa se roage ?ie cu aceasta rugaciune.
\par 28 Deci, Doamne Dumnezeul meu, Tu e?ti Dumnezeu ?i cuvintele Tale sunt neschimbate ?i Tu ai vestit robului Tau un astfel de bine.
\par 29 Începe acum ?i binecuvânteaza casa robului Tau, ca sa fie ea ve?nic  înaintea fe?ei Tale, caci Tu, Doamne Dumnezeule, ai vestit aceasta, ?i, prin binecuvântarea Ta, se va face casa robului Tau binecuvântata, ca sa fie înaintea Ta în veci".

\chapter{8}

\par 1 Dupa aceasta David a lovit pe Filisteni ?i i-a supus ?i a luat David Meteg-Haama din mâna Filistenilor.
\par 2 Apoi a batut ?i pe Moabi?i ?i i-a masurat cu funia, punându-i la pamânt; ?i a masurat doua funii spre ucidere, ?i o funie spre cru?are ?i lasare în via?a. Atunci au ajuns Moabi?ii robi lui David ?i birnici.
\par 3 Apoi a batut David pe Hadad-Ezer, fiul lui Rehob, regele din ?oba, pe când acesta mergea ca sa-?i întemeieze din nou domnia sa la râul Eufratului;
\par 4 ?i a luat David de la el o mie ?apte sute de calare?i ?i douazeci de mii de pedestra?i ?i a taiat David vinele la to?i caii de la care, lasând pentru sine din ei numai pentru o suta de care.
\par 5 Atunci au venit Sirienii din Damasc în ajutor lui Hadad-Ezer, regele ?obei; însa David a omorât douazeci ?i doua de mii de Sirieni.
\par 6 ?i a pus David o?ti de paza în Siria Damascului, iar Sirienii au ajuns robi ?i birnici lui David. Domnul însa a pazit pe David pretutindeni unde s-a dus.
\par 7 Atunci a luat David scuturile cele de aur care s-au gasit la robii lui Hadad-Ezer ?i le-a dus la Ierusalim.
\par 8 Pe acestea le-a luat apoi ?i?ac, regele Egiptului, în timpul navalirii lui asupra Ierusalimului, în zilele lui Roboam, fiul lui Solomon. Iar din Tebah ?i Beritai, ceta?ile lui Hadad-Ezer, regele David a luat foarte multa arama.
\par 9 Auzind Tou, regele Hamatului, ca David a batut toata o?tirea lui Hadad-Ezer,
\par 10 A trimis pe Hadoram, fiul sau, la regele David sa-l salute ?i sa-i mul?umeasca, pentru ca s-a razboit cu Hadad-Ezer ?i l-a biruit. Caci Hadad-Ezer se afla în razboi cu Tou. Iar în mâinile lui Hadoram se aflau vase de argint, de aur ?i de arama.
\par 11 Pe acestea înca le-a harazit David Domnului, împreuna cu aurul ?i argintul pe care îl afierosise din cele luate de la toate popoarele supuse: de la Sirieni, Filisteni ?i Amaleci?i ?i din prada de la Hadad-Ezer, fiul lui Rehob, regele ?obei.
\par 12 Astfel ?i-a facut David nume, întorcându-se de la înfrângerea celor optsprezece mii de Sirieni din Valea Sarata.
\par 13 Apoi a pus o?tiri de paza în Idumeea; în toata Idumeea a pus o?tiri de paza ?i to?i Idumeii au ajuns robii lui David.
\par 14 Iar Domnul a pazit pe David pretutindeni pe unde a fost.
\par 15 ?i a domnit David peste tot Israelul, facând judecata ?i dreptate în tot poporul sau.
\par 16 Ioab, fiul lui ?eruia, era capetenia o?tirii, iar Iosafat, fiul lui Ahilud, era cronicar.
\par 17 ?adoc, fiul lui Ahitub, ?i Ahimelec, fiul lui Abiatar, au fost preo?i, iar Seraia a fost dregator.
\par 18 Benaia, fiul lui Iehoiada, a fost capetenie peste Cheretieni ?i Peletieni, iar fiii lui David erau cei dintâi la curte.

\chapter{9}

\par 1 "N-a mai ramas, oare, cineva din casa lui Saul? - zise David. Eu i-a? arata mila din pricina lui Ionatan".
\par 2 În casa lui Saul însa fusese un rob, cu numele ?iba. Pe acesta l-au chemat la David ?i i-a zis regele: "Tu e?ti ?iba?" "Eu, robul tau", a raspuns acesta.
\par 3 "Nu cumva mai este cineva din casa lui Saul? a întrebat regele, ca i-a? arata mila lui Dumnezeu". "Ba este, fiul lui Ionatan, cel ?chiop de picioare", a zis ?iba catre rege.
\par 4 Iar regele zise: "Unde este?" "Iata, a raspuns ?iba regelui, el este în casa lui Machir, fiul lui Amiel, din Lodebar".
\par 5 ?i a trimis regele David de l-au luat de la casa lui Machir, fiul lui Amiel, din Lodebar.
\par 6 ?i a venit Mefibo?et, fiul lui Ionatan, la David ?i, cazând cu fa?a la pamânt, s-a închinat regelui. ?i a zis regele: "Mefibo?et!" "Da, robul tau!" a raspuns acesta.
\par 7 "Nu te teme, a zis regele David, ca eu î?i voi arata mila pentru tatal tau, Ionatan, ?i-?i voi întoarce toate ?arinile lui Saul, bunicul tau, ?i tu vei mânca totdeauna pâine la masa mea".
\par 8 Atunci s-a închinat Mefibo?et ?l a zis: "Ce este robul tau, de ai cautat tu la un asemenea câine mort, cum sunt eu?"
\par 9 Regele însa a chemat pe ?iba, sluga lui Saul, ?i i-a zis: "Toate câte au fost ale lui Saul ?i ale întregii lui case le dau fiului stapânului tau:
\par 10 Deci lucreaza pentru el pamântul, tu cu fiii tai ?i cu robii tai, ?i strânge roadele lui, ca fiul stapânului tau sa aiba pâine de hrana. Mefibo?et, fiul stapânului tau, va mânca totdeauna la masa mea".
\par 11 ?iba avea cincisprezece feciori ?i douazeci de robi. ?i a zis ?iba catre rege: "Tot ce porunce?te regele, stapânul meu, robului sau, robul tau va îndeplini".
\par 12 ?i mânca Mefibo?et la masa lui David, ca unul din copiii regelui. Mefibo?et avea un copil mic cu numele Micha. ?i to?i cei ce traiau în casa lui ?iba erau slugile lui Mefibo?et.
\par 13 Iar Mefibo?et era olog. Traia în Ierusalim ?i mânca totdeauna la masa regelui.

\chapter{10}

\par 1 Trecând câtava vreme, a murit regele Amoni?ilor, iar în locul lui s-a facut rege Hanun, fiul lui.
\par 2 Atunci a zis David: "Voi arata mila lui Hanun, fiul lui Naha?, pentru binefacerea ce mi-a aratat tatal sau". Apoi a trimis David pe slugile sale sa mângâie pe Hanun de moartea tatalui sau. ?i au venit slugile lui David în ?ara Amoni?ilor.
\par 3 Însa capeteniile Amoni?ilor au zis catre Hanun, domnul lor: "Socoti?i, oare, ca David din dragoste catre tatal tau a trimis mângâietori la tine? Nu cumva a trimis David slugile sale la tine ca sa iscodeasca cetatea ?i sa vada ce este în ea ?i apoi s-o darâme?
\par 4 Atunci a luat Hanun pe slugile lui David ?i a ras fiecaruia jumatate de barba ?i le-a taiat hainele pe jumatate, pâna la ?olduri, ?i apoi le-a dat drumul.
\par 5 Când i s-a spus aceasta lui David, acesta a trimis înaintea lor, deoarece erau foarte batjocori?i. ?i a poruncit regele sa li se spuna: "Ramâne?i în Ierihon pâna va vor cre?te barbile ?i atunci va ve?i întoarce".
\par 6 Amoni?ii insa, vazând ca s-au facut nesuferi?i înaintea lui David, au trimis sa tocmeasca cu plata pe Sirienii din Bet-Rehov ?i pe Sirienii din ?oba, douazeci de mii de pedestra?i, de la regele Amalecit din Maacha o mie de oameni ?i din I?tov douasprezece mii de oameni.
\par 7 Când a auzit de aceasta, David a trimis pe Ioab cu toata o?tirea de viteji.
\par 8 ?i ie?ind, Amoni?ii s-au a?ezat în rânduri de lupta la poarta, iar Sirienii din ?oba, din Rehov, din I?tov ?i din Maacha au stat deoparte în câmp.
\par 9 Vazând Ioab ca o?tirea du?mana era a?ezata împotriva lui ?i înainte ?i în urma, a ales o?tenii cei mai de seama ai lui Israel ?i i-a pus în rânduri de lupta împotriva Sirienilor.
\par 10 Iar cealalta parte de oameni a încredin?at-o lui Abi?ai, fratele sau, ca sa-i puna în rânduri de lupta împotriva Amaleci?ilor:
\par 11 Apoi a zis Ioab: "Daca Sirienii ma vor birui pe mine, tu sa ma aju?i, iar daca Amoni?ii te vor birui pe tine, î?i voi veni eu în ajutor.
\par 12 Fii curajos ?i sa ne ?inem cu barba?ie pentru poporul nostru ?i pentru ceta?ile Dumnezeului nostru, ?i Domnul va face ce va binevoi".
\par 13 Dupa aceea a întrat Ioab ?i poporul ce era cu el în lupta cu Sirienii, dar ace?tia au fugit de el.
\par 14 Amoni?ii, vazând ca Sirienii pleaca, au fugit ?i ei de Abi?ai ?i s-au dus în cetate. Întorcându-se Ioab de la Amoni?i, a intrat în Ierusalim.
\par 15 Sirienii însa, vazând ca au fost birui?i de Israeli?i, s-au adunat la un loc.
\par 16 ?i a trimis Hadad-Ezer de au chemat pe Sirienii cei de peste râul Eufrat ?i ace?tia au venit la Helam, iar Sovac, capetenia o?tirii lui Hadad-Ezer, îi conducea.
\par 17 Când s-a spus de aceasta lui David, acesta a adunat pe to?i Israeli?ii ?i, trecând Iordanul, a venit la Helam. Sirienii s-au a?ezat împotriva lui David ?i s-au batut cu el.
\par 18 Dar au fugit Sirienii de Israeli?i ?i David a nimicit Sirienilor ?apte sute de care ?i patruzeci de mii de calare?i; ba a lovit ?i pe capetenia Sovac, care a ?i murit acolo.
\par 19 ?i când regii supu?i lui Hadad-Ezer au vazut ca sunt învin?i de Israeli?i, au încheiat pace cu Israeli?ii ?i s-au supus acestora. Iar Sirienii s-au temut sa mai dea ajutor Amoni?ilor.

\chapter{11}

\par 1 Peste un an, pe vremea când regii pornesc la razboi, David a trimis pe Ioab ?i slugile sale cu el ?i pe to?i Israeli?ii ?i ace?tia au lovit pe Amoni?i ?i au împresurat Raba;
\par 2 Dar David a ramas în Ierusalim. Odata, spre seara, sculându-se David din pat ?i plimbându-se pe acoperi?ul casei domne?ti, a vazut de pe acoperi? o femeie scaldându-se, ?i femeia aceasta era foarte frumoasa.
\par 3 Atunci a trimis David sa se cerceteze cine este acea femeie. ?i i s-a spus ca este Bat?eba, fiica lui Eliam, femeia lui Urie Heteul.
\par 4 Apoi David a trimis slugile sa o ia; ea a venit la el ?i el s-a culcat cu ea. Iar daca s-a cura?it ea de necura?ia ei, s-a întors la casa sa.
\par 5 Femeia aceasta a ramas însarcinata ?i a trimis de s-a vestit lui David, zicând: "Eu sunt însarcinata".
\par 6 Atunci a trimis David sa se zica lui Ioab: "Trimite la mine pe Urie Heteul!" ?i a trimis Ioab pe Urie la David.
\par 7 Venind Urie la David, acesta l-a întrebat de sanatatea lui Ioab, de starea poporului ?i de mersul razboiului.
\par 8 Apoi a zis David catre Urie: "Du-te acasa ?i-?i spala picioarele!" Ie?ind Urie din casa regelui, în urma lui i s-a trimis un dar de la masa regelui.
\par 9 Dar Urie a dormit la poarta casei regelui cu toate slugile stapânului sau ?i nu s-a dus la casa sa.
\par 10 ?i i s-a spus lui David, zicând: "Urie nu s-a dus la casa sa". "Iata, a zis David catre Urie, tu ai venit de pe drum, de ce nu te-ai dus la casa ta?"
\par 11 Iar Urie a zis: "Chivotul Domnului ?i Israel ?i Iuda sunt în corturi; stapânul meu Ioab ?i robii domnului meu sunt în tabara, iar eu sa ma duc la casa mea sa manânc, sa beau ?i sa ma culc cu femeia mea? Ma jur pe via?a ta ?i pe via?a sufletului tau ca nu voi face aceasta".
\par 12 "Ramâi aici ?i ziua aceasta, a zis David lui Urie, iar mâine î?i voi da drumul". ?i a ramas Urie în Ierusalim în ziua aceea pâna a doua zi.
\par 13 ?i l-a chemat David ?i a mâncat Urie înaintea lui ?i a baut ?i David i-a aratat cinste. Dar seara Urie s-a dus sa se culce în patul sau cu robii stapânului sau, iar la casa sa nu s-a dus.
\par 14 Diminea?a David a scris scrisoare lui Ioab ?i a trimis-o pe Urie.
\par 15 În scrisoarea aceea el scria a?a: "Pune?i pe Urie unde va fi lupta mai crâncena ?i retrage?i-va de la el, ca sa fie lovit ?i ucis".
\par 16 De aceea, când Ioab a împresurat cetatea, a pus pe Urie într-un astfel de loc, de care ?tia ca este aparat de oameni viteji.
\par 17 ?i au ie?it oamenii din cetate ?i s-au luptat cu Ioab ?i au cazut câ?iva din popor, din slugile lui David. Acolo a fost ucis ?i Urie Heteul.
\par 18 Atunci a trimis Ioab sa se faca cunoscut lui David tot mersul luptei.
\par 19 ?i a poruncit trimisului ?i i-a zis: "Dupa ce vei povesti regelui tot mersul luptei,
\par 20 ?i vei vedea ca regele se mânie ?i-?i zice: "De ce v-a?i apropiat sa va lupta?i a?a aproape de cetate? Nu ?tia?i voi ca de pe zidurile ceta?ii pot sa arunce în voi?
\par 21 Cine oare a ucis pe Abimelec, fiul lui Ierubaal? Au nu o femeie, care a aruncat în el de pe zid o bucata de râ?ni?a ?i l-a lovit ?i el a murit în Tebe?? De ce v-a?i apropiat a?a tare de cetate?" Atunci sa-i zici: "?i robul tau Urie Heteul a fost lovit ?i a murit".
\par 22 S-a dus deci trimisul lui Ioab la rege în Ierusalim ?i, ajungând, a povestit lui David despre toate, pentru care fusese trimis de Ioab ?i de tot mersul luptei. ?i s-a mâniat David pe Ioab ?i a zis trimisului: "De ce v-a?i apropiat a?a tare de cetate sa va lupta?i? Nu ?tia?i voi oare ca va pot lovi de pe zidurile ceta?ii? Cine a ucis pe Abimelec, fiul lui Ierubaal? Oare nu o femeie care a aruncat în el de pe zid cu o bucata de râ?nita, ?i a murit în Tebe?? De ce v-a?i apropiat a?a tare de zid?"
\par 23 Atunci trimisul a spus lui David: "Acei oameni ne-au rapus pe noi ?i au ie?it asupra noastra în câmp, dar noi i-am alungat pâna la poarta.
\par 24 Atunci au început a sageta arca?ii de pe ziduri asupra robilor tai ?i au murit câ?iva din robii regelui; ?i a murit de asemenea ?i robul tau Urie Heteul".
\par 25 Atunci David a zis: "A?a sa spui lui Ioab: Sa nu te tulbure lucrul acesta, caci sabia o data manânca pe unul, alta data manânca pe altul. Înte?e?te lupta împotriva ceta?ii ?i darâm-o. A?a sa-l încurajezi".
\par 26 ?i auzind femeia lui Urie ca a murit Urie, barbatul ei, a plâns dupa el.
\par 27 Iar daca s-a ispravit vremea plângerii, a trimis David ?i a luat-o în casa sa ?i ea a ajuns femeia lui ?i i-a nascut un fiu. Fapta aceasta, pe care a facut-o David, a fost rea înaintea Domnului.

\chapter{12}

\par 1 Atunci a trimis Domnul pe Natan proorocul la David ?i a venit acela la el ?i i-a zis: "Erau într-o cetate doi oameni: unul bogat ?i altul sarac.
\par 2 Cel bogat avea foarte multe vite mari ?i marunte,
\par 3 Iar cel sarac n-avea decât o singura oi?a, pe care el o cumparase de mica ?i o hranise ?i ea crescuse cu copiii lui. Din pâinea lui mâncase ?i ea ?i se adapase din ulcica lui, la sânul lui dormise ?i era pentru el ca o fiica.
\par 4 Dar iata ca a venit la bogat un calator, ?i gazda nu s-a îndurat sa ia din oile sale sau din vitele sale, ca sa gateasca cina pentru calatorul care venise la el, ci a luat oi?a saracului ?i a gatit-o pe aceea pentru omul care venise la el".
\par 5 Atunci s-a mâniat David cumplit asupra acelui om ?i a zis catre Natan: "Precum este adevarat ca Domnul este viu, tot a?a este de adevarat ca omul care a facut aceasta este vrednic de moarte;
\par 6 Pentru oaie el trebuie sa întoarca împatrit, pentru ca a facut una ca aceasta ?i pentru ca n-a avut mila".
\par 7 Atunci Natan a zis catre David: "Tu e?ti omul care a facut aceasta. A?a zice Domnul Dumnezeul lui Israel: Eu te-am uns rege pentru Israel ?i Eu te-am izbavit din mâna lui Saul,
\par 8 ?i-am dat casa domnului tau ?i femeile domnului tau la sânul tau; ?i-am dat ?ie casa lui Israel ?i a lui Iuda ?i, daca aceasta este pu?in pentru tine, ?i-a? mai adauga.
\par 9 Pentru ce însa ai nesocotit tu cuvântul Domnului, facând rau înaintea ochilor Lui? Pe Urie Heteul tu l-ai lovit cu sabia, pe femeia lui ?i-ai luat-o de so?ie, iar pe el l-ai ucis cu sabia Amoni?ilor.
\par 10 Deci nu se va departa sabia de deasupra casei tale în veac, pentru ca tu M-ai nesocotit pe Mine ?i ai luat pe femeia lui Urie Heteul, ca sa-?i fie nevasta.
\par 11 A?a zice Domnul: Iata Eu voi ridica asupra ta rau chiar din casa ta ?i voi lua pe femeile tale înaintea ochilor tai ?i le voi da aproapelui tau ?i se va culca acela cu femeile tale în vazul soarelui acestuia.
\par 12 Tu ai facut pe ascuns, iar Eu voi face aceasta înaintea a tot Israelul ?i înaintea soarelui". "Am pacatuit înaintea Domnului", a zis David catre Natan.
\par 13 "?i Domnul a ridicat pacatul de deasupra ta, a zis Natan, ?i tu nu vei muri.
\par 14 Dar fiindca tu prin aceasta fapta ai dat vrajma?ilor Domnului pricina sa-L huleasca, de aceea va muri fiul ce ?i se va na?te".
\par 15 Apoi s-a dus Natan la casa sa, iar Domnul a lovit copilul pe care i-l nascuse lui David femeia lui Urie ?i acela s-a îmbolnavit.
\par 16 ?i s-a rugat David Domnului pentru copil, a postit ?i, ducându-se deoparte, a petrecut noaptea întins pe pamânt.
\par 17 Atunci au intrat la el batrânii casei lui ca sa-l ridice de la pamânt, dar el n-î voit ?i nici n-a mâncat pâine cu ei.
\par 18 Dupa ?apte zile a murit copilul ?i slugile lui David se temeau sa-i spuna ca a murit copilul. Caci ei î?i ziceau: "Când copilul era înca viu ?i noi îl mângâiam, el nu ne baga în seama; cum sa-i spunem acum: A murit copilul? Ar putea sa faca vreun rau".
\par 19 Dar vazând David ca slugile sale ?optesc între ele, a priceput ca a murit copilul ?i le-a întrebat: "A murit copilul?" "A murit", i s-a raspuns.
\par 20 Atunci David s-a sculat de la pamânt, s-a spalat, s-a uns ?i ?i-a schimbat hainele ?i s-a dus în casa Domnului ?i s-a rugat. Întorcându-se apoi acasa, a cerut sa i se dea pâine ?i a mâncat.
\par 21 ?i i-au zis slugile: "Ce va sa zica aceasta? Când copilul era înca în via?a, ai postit, ai plâns ?i n-ai dormit; iar dupa ce copilul a murit, te-ai sculat, ai mâncat ?i ai baut?"
\par 22 "Câta vreme copilul era viu, a zis David, am postit ?i am plâns, caci socoteam: Cine ?tie, poate ma va milui Domnul ?i va trai copilul.
\par 23 Dar acum el a murit; de ce sa mai postesc? Îl mai pot eu, oare, întoarce? Eu ma voi duce la el, iar el nu se va mai întoarce la mine".
\par 24 ?i a mângâiat David pe Bat?eba, femeia sa, a intrat la ea, s-a culcat cu ea ?i ea a zamislit ?i a mai nascut un fiu ?i i-a pus numele Solomon. Domnul l-a iubit pe acesta,
\par 25 ?i a trimis pe proorocul Natan, ?i acesta i-a pus numele Iedida, adica iubitul Domnului, cum îi spusese Domnul.
\par 26 Ioab însa lupta împotriva ceta?ii Amoni?ilor, Raba, ?i aproape luase cetatea domneasca.
\par 27 Atunci a trimis Ioab la David sa i se spuna: "Am tabarât asupra ceta?ii Raba ?i am luat cetatea prin apa.
\par 28 Aduna acum celalalt popor ?i vino asupra ceta?ii ?i o ia; caci de o voi lua eu, atunci se va slavi numele meu".
\par 29 Atunci a adunat David tot poporul ?i s-a dus asupra ceta?ii Raba, s-a luptat împotriva ei ?i a luat-o.
\par 30 ?i a luat David de pe capul regelui ei coroana, care era de un talant de aur ?i cu pietre scumpe, ?i a pus-o pe capul sau; a luat ?i foarte multa prada din cetate.
\par 31 Iar pe poporul care se afla în ea l-a scos ?i l-a pus sub fierastrau ?i sub grapa de fier ?i sub securi de fier ?i i-a aruncat în cuptoarele de ars caramida. A?a a facut el cu toate ceta?ile Amoni?ilor. Dupa aceea David s-a întors la Ierusalim cu tot poporul.

\chapter{13}

\par 1 Dupa aceea s-au petrecut urmatoarele: Abesalom, fiul lui David, avea o sora frumoasa, cu numele Tamara. Pe aceasta o iubea Amnon, alt fiu al lui David.
\par 2 ?i s-a chinuit Amnon pâna într-atâta, ca s-a îmbolnavit din pricina surorii sale Tamara, caci aceasta era fecioara ?i lui Amnon i se parea greu sa-i faca ceva.
\par 3 Avea însa Amnon un prieten, anume Ionadab, fiul lui ?ama, fratele lui David.
\par 4 Ionadab era om foarte ?iret. Acesta i-a zis: "Fiul regelui, de ce slabe?ti tu a?a pe fiecare zi? Spune mie!" "Iubesc pe Tamara, sora lui Abesalom, fratele meu", a zis Amnon catre el.
\par 5 "Culca-te în patul tau, i-a zis Ionadab, ?i te fa bolnav; iar când tatal tau va veni sa te cerceteze, sa-i zici: Lasa sa vina Tamara, sora mea, sa ma întareasca cu hrana, pregatind mâncare înaintea ochilor mei, ca sa vad ?i sa manânc din mâinile ei!"
\par 6 ?i s-a culcat Amnon ?i s-a facut bolnav ?i a venit regele sa-l cerceteze. Atunci Amnon a zis catre rege: "Lasa pe Tamara, sora mea, sa vina ?i sa coaca înaintea ochilor mei o turta sau doua ?i sa manânc din mâinile ei".
\par 7 ?i a trimis David la Tamara acasa sa-i spuna: "Du-te acasa la Amnon, fratele tau, ?i-i fa de mâncare!"
\par 8 ?i s-a dus ea acasa la fratele sau Amnon; acesta sta culcat. ?i a luat ea faina, a framântat-o, a facut înaintea ochilor lui turte ?i le-a copt;
\par 9 Apoi a luat tigaia ?i a pus-o înaintea lui, dar el n-a vrut sa manânce. ?i a zis Amnon: "Sa iasa to?i de la mine!"
\par 10 ?i au ie?it de la el to?i oamenii. Apoi Amnon a zis catre Tamara: "Du mâncarea în odaia cea din fund ?i voi mânca acolo din mâinile tale". Atunci a luat Tamara turtele ce le gatise ?i le-a dus lui Amnon, fratele sau, în odaia cea din fund.
\par 11 Dar când le-a pus înaintea lui ca sa manânce, el a apucat-o ?i i-a zis: "Vino ?i te culca cu mine, sora mea!"
\par 12 "Nu, frate, a zis ea, nu ma necinsti, caci aceasta nu se face în Israel; nu face ticalo?ia aceasta!
\par 13 Caci unde ma voi duce eu cu necinstea mea? ?i tu vei fi în Israel cu unul din cei fara de minte. Vorbe?te cu regele ?i el nu se va împotrivi sa ma dea dupa tine".
\par 14 El însa n-a vrut sa asculte cuvintele ei, ci a silit-o ?i s-a culcat cu ea ?i a necinstit-o.
\par 15 Dupa aceea a urât-o Amnon cu ura cea mai mare, a?a încât ura cu care a urât-o el era mai mare decât iubirea pe care o avusese catre ea. ?i i-a zis ei Amnon: "Scoala ?i pleaca!"
\par 16 "Ba nu, frate, i-a zis Tamara, a ma alunga este un rau ?i înca ?i mai mare decât cel dintâi, pe care mi l-ai facut tu mie".
\par 17 Dar el n-a vrut sa o asculte, ci a chemat pe omul sau care-l slujea ?i i-a zis: "Alunga pe aceasta de la mine afara ?i încuie u?a dupa ea".
\par 18 Ea însa era îmbracata cu haina pestri?a, caci astfel de haine purtau pe deasupra fetele regelui care erau fecioare. ?i a scos-o sluga afara ?i a încuiat u?a dupa ea.
\par 19 Iar Tamara ?i-a presarat cenu?a pe capul sau ?i-a rupt haina cea pestri?a, cu care era îmbracata ?i, punându-?i mâinile pe cap, mergea a?a ?i striga.
\par 20 Atunci a zis catre ea Abesalom, fratele ei: "Nu cumva Amnon, fratele tau, a umblat cu tine? Dar taci acum, sora mea, caci el este fratele tau; nu-?i zdrobi inima pentru fapta aceasta". ?i a ?ezut Tamara parasita în casa lui Abesalom, fratele sau.
\par 21 ?i a auzit regele David de toate cele întâmplate ?i s-a mâniat foarte tare, dar n-a stricat inima lui Amnon, fiul sau, caci îl iubea, pentru ca era întâiul sau nascut.
\par 22 Abesalom însa nu graia cu Amnon nici bine, nici rau, caci Abesalom ura pe Amnon, pentru ca acesta necinstise pe Tamara, sora sa.
\par 23 Iar dupa doi ani, pe vremea când tundeau oile lui Abesalom în Baal-Ha?or, care se afla în Efraim, a chemat Abesalom pe to?i fiii regelui.
\par 24 ?i venind Abesalom la rege, a zis: "Iata acum este tunsul oilor la robul tau; deci sa mearga regele ?i slugile sale la robul tau!"
\par 25 Regele însa a zis catre Abesalom: "Ba nu, fiul meu, nu vom merge cu to?ii, ca sa nu te împovaram". Abesalom însa l-a rugat cu mare staruin?a, dar el n-a vrut sa se duca, ci l-a binecuvântat.
\par 26 Atunci Abesalom a zis catre el: "Daca nu, sa mearga cu noi macar Amnon, fratele meu". "De ce sa mearga el cu tine?" a zis regele.
\par 27 Dar staruind Abesalom, regele a dat drumul lui Amnon ?i la to?i fiii regelui sa se duca cu el. ?i a facut Abesalom ospa?, ca ospa?ul unui rege.
\par 28 ?i a mai poruncit Abesalom slugilor sale, zicând: "Lua?i seama, ca îndata ce inima lui Amnon se va veseli de vin ?i când eu voi zice: Lovi?i pe Amnon, sa-l ucide?i ?i sa nu va teme?i; eu va poruncesc aceasta, sa fi?i curajo?i ?i viteji!"
\par 29 ?i au facut slugile lui Abesalom cu Amnon cum le poruncise Abesalom. Atunci s-au sculat fiii regelui cu to?ii ?i, încalecând fiecare pe catârul sau, au fugit.
\par 30 ?i înca pe cale fiind ei, a ajuns la David vestea ca Abesalom a omorât pe coti fiii regelui ?i n-a mai ramas nici unul din ei.
\par 31 Atunci s-a sculat regele ?i ?i-a rupt hainele sale ?i s-a aruncat la pamânt; ?i toate slugile sale, care stateau înaintea sa, ?i-au rupt ve?mintele lor.
\par 32 Atunci Ionadab, fiul lui ?ama, fratele lui David, a zis: "Sa nu creada regele, stapânul meu, ca au omorât pe to?i baie?ii, fiii regelui; numai singur Amnon a murit, caci Abesalom avea aceasta în gând înca din ziua când Amnon a necinstit pe sora sa, Tamara.
\par 33 Deci, regele, stapânul meu, sa nu se tulbure cu gândul ca ar fi murit to?i fiii regelui, caci numai Amnon singur a murit".
\par 34 Atunci a fugit Abesalom, iar omul de straja, ridicându-?i ochii sai, a privit ?i iata popor mult venea pe drumul de pe coasta muntelui. ?i venind straja, a dat de veste regelui, zicând: "Am vazut oameni pe drumul Bahurim, de pe coasta muntelui".
\par 35 Atunci Ionadab a zis catre rege: "Iata vin fiii regelui; cum a zis robul tau a?a ?i este".
\par 36 ?i cum a sfâr?it el vorbele acestea, iata au sosit ?i fiii regelui ?i au ridicat strigat ?i au plâns. ?i a plâns ?i regele însu?i ?i toate slugile lui plângere mare.
\par 37 Iar Abesalom a fugit ?i s-a dus la Talmai, fiul lui Amihud, regele Ghe?urului. ?i a plâns regele David dupa fiul sau în toate zilele.
\par 38 Iar Abesalom, fugind ?i ducându-se în Ghe?ur, a stat acolo trei ani.
\par 39 Dar regele David nu s-a apucat sa urmareasca pe Abesalom, caci se mângâiase de moartea lui Amnon.

\chapter{14}

\par 1 Cunoscând Ioab, fiul ?eruiei, ca inima regelui s-a întors spre Abesalom,
\par 2 A trimis Ioab la Tecoa ?i a luat de acolo o femeie în?eleapta ?i i-a zis: "Fa-te ca e?ti bocitoare, îmbraca-te cu haine de jale, nu te unge cu untdelemn ?i fii ca a femeie care a plâns zile multe dupa un mort;
\par 3 ?i du-te la rege ?i zi catre el a?a ?i a?a". ?i i-a spus Ioab ce anume sa zica.
\par 4 Venind deci femeia cea din Tecoa la rege ?i cazând cu fa?a la pamânt, s-a închinat ?i a zis: "Ajutor, o, rege, ajutor!"
\par 5 "Ce ai?" a zis regele catre ea. "Sunt de mult vaduva, a zis ea, caci mi-a murit barbatul.
\par 6 ?i avea roaba ta doi feciori ?i ace?tia s-au sfadit în ?arina; ?i, neavând cine-i despar?i, a lovit unul din ei pe celalalt ?i l-a omorât.
\par 7 ?i iata s-au sculat toate rudele asupra roabei tale ?i zic: "Da-ne pe uciga?ul fratelui sau sa-l omorâm pentru sufletul fratelui sau pe care l-a pierdut el ?i vom pierde chiar ?i pe mo?tenitorul lui". ?i a?a vor sa stinga ei ?i cea din urma scânteie a mea, ca sa nu mai lase barbatului meu nici nume, nici urma?i pe fa?a pamântului".
\par 8 "Mergi în pace la casa ta, a zis regele catre femeie, caci voi da porunca pentru tine".
\par 9 Dar femeia cea din Tecoa a zis catre rege: "O, rege, stapânul meu, asupra mea sa fie vina ?i asupra casei tatalui meu; iar regele ?i tronul lui este nevinovat".
\par 10 "pe cel ce va fi împotriva ta, a zis regele, sa-l aduci la mine ?i mai mult nu te va mai atinge".
\par 11 "Porunce?te, o, rege, în numele Domnului Dumnezeului tau, a zis ea, ca sa nu se înmul?easca razbunatorii sângelui ?i sa nu piarda pe fiul meu". "Viu este Domnul, a zis regele, nici un fir de par de al fiului tau nu va cadea pe pamânt!"
\par 12 "Îngaduie roabei tale, a zis femeia, sa mai spun o vorba regelui, stapânului meu".
\par 13 "Spune", a zis el. "Pentru ce cuge?i tu a?a împotriva poporului Domnului?, a zis femeia. Rostind cuvântul acesta, regele s-a osândit pe sine însu?i, pentru ca nu aduce înapoi pe izgonitul sau.
\par 14 Noi vom muri ?i vom fi ca apa varsata pe pamânt, care nu se mai poate aduna. Dumnezeu însa nu voie?te sa piarda sufletul ?i se gânde?te cum ar face sa nu lepede de la Sine nici pe cel înlaturat.
\par 15 ?i acum eu am venit sa spun regelui, stapânul meu, cuvintele acestea, pentru ca poporul ma sperie ?i roaba ta a zis: "Am sa graiesc eu cu regele, sa vad nu va face el dupa cuvântul roabei sale?
\par 16 De buna seama regele va asculta ?i va izbavi pe roaba sa din mâna oamenilor care voiesc sa ma piarda împreuna cu fiul meu din mo?tenirea lui Dumnezeu.
\par 17 ?i roaba ta a zis: Sa fie cuvântul regelui, stapânul meu, spre mângâierea mea, caci regele, stapânul meu, este ca îngerul lui Dumnezeu ?i poate ca sa asculte ?i bune ?i rele, ?i Domnul Dumnezeul tau va fi cu tine".
\par 18 ?i raspunzând, regele a zis catre femeie: "Sa nu ascunzi de mine ceea ce am sa te întreb!" "Graie?te, a zis femeia, o, rege, stapânul meu!"
\par 19 "Nu cumva este mâna lui Ioab în tot ce spui tu?" a zis regele. Iar femeia i-a raspuns ?i a zis: "Sa traiasca sufletul tau, o, rege! Nu pot sa ma abat nici la dreapta, nici la stânga de la ceea ce a zis regele, stapânul meu. Adevarat, robul tau Ioab mi-a poruncit ?i el a pus în gura roabei tale toate cuvintele acestea;
\par 20 ?i tot robul tau m-a înva?at ca prin pilda sa dau lucrului aceasta înfa?i?are. Dar regele, stapânul meu, este în?elept, cum este în?elept îngerul lui Dumnezeu, ca sa cunoasca tot ce este pe pamânt".
\par 21 Atunci regele a zis catre Ioab: "Iata, fac lucrul acesta; du-te dar ?i adu înapoi pe baiatul Abesalom".
\par 22 Atunci Ioab a cazut cu fa?a la pamânt ?i s-a închinat ?i a binecuvântat pe rege, zicând: "Acum robul tau cunoa?te ca a aflat bunavoin?a înaintea ochilor tai, o, rege, stapânul meu, de vreme ce regele a facut cum a zis robul tau".
\par 23 ?i sculându-se, Ioab s-a dus în Ghe?ur ?i a adus pe Abesalom la Ierusalim.
\par 24 ?i a zis regele: "Sa se întoarca la casa sa, dar fa?a mea nu o va vedea". ?i s-a întors Abesalom la casa sa; dar fa?a regelui n-a vazut-o.
\par 25 În tot Israelul nu era barbat a?a de frumos ca Abesalom ?i a?a de laudat ca el; din talpile picioarelor ?i pâna în cre?tetul capului nu avea nici o meteahna.
\par 26 Când î?i tundea capul sau - ?i ?i-l tundea în fiecare an, pentru ca-l îngreuia - parul de pe capul lui cântarea doua sute de sicli, dupa cântarul regesc.
\par 27 ?i i s-au nascut lui Abesalom trei baie?i ?i o fata, anume Tamara. Aceasta a fost o femeie frumoasa la chip ?i a ajuns so?ia lui Roboam, fiul lui Solomon ?i i-a nascut pe Abia.
\par 28 Abesalom a ramas în Ierusalim doi ani, dar fa?a regelui n-a vazut-o.
\par 29 ?i a trimis Abesalom dupa Ioab, ca sa-l trimita la rege, dar acesta n-a vrut sa vina la el. ?i a trimis ?i a doua oara ?i acesta tot n-a vrut sa vina.
\par 30 Atunci a zis Abesalom slugilor sale: "Vede?i voi partea de ?arina a lui Ioab, care este lânga a mea ?i unde el are semanat orz? Duce?i-va ?i-i da?i foc!" ?i au ars slugile lui Abesalom acea parte de ?arina cu foc. Deci venind slugile lui Ioab la. acesta cu hainele rupte, au zis: "Slugile lui Abesalom au ars ogorul tau cu foc".
\par 31 Atunci s-a sculat Ioab ?i a venit la Abesalom acasa ?i i-a zis: "Pentru ce slugile tale au ars cu foc ogorul meu?"
\par 32 Iar Abesalom a zis: "Iata, eu am trimis la tine ?i am zis: Vino încoace, ca sa te trimit la rege sa-i zici: Pentru ce am venit din Ghe?ur? Mai bine-mi era sa fi ramas acolo. Vreau sa vad fa?a regelui. De sunt vinovat, atunci ucide-ma".
\par 33 ?i s-a dus Ioab la rege ?i i-a spus aceasta. ?i a chemat regele pe Abesalom ?i a venit acesta la rege ?i, cazând cu fa?a sa la pamânt înaintea regelui, i s-a închinat, iar regele a sarutat. pe Abesalom.

\chapter{15}

\par 1 Dupa aceasta Abesalom ?i-a înjghebat care ?i cai ?i cincizeci de barba?i, care mergeau înaintea sa.
\par 2 ?i se scula Abesalom dis-de-diminea?a, se oprea la poarta lânga cale, ?i, când venea cineva la rege sa se judece pentru vreo pricina, Abesalom îl chema la sine ?i-l întreba: "Din ce cetate e?ti tu?" ?i când acela îi raspundea: "Robul tau este din cutare trib al lui Israel",
\par 3 Atunci Abesalom îi zicea: "Iata pricina ta este buna ?i dreapta, dar la rege n-are cine sa te asculte".
\par 4 ?i mai zicea Abesalom: "O, de m-ar pune pe mine judecator în ?ara aceasta, ar veni la mine oricine ar avea neîn?elegeri ?i judecata ?i eu l-a? judeca drept".
\par 5 ?i de se apropia cineva sa i se închine, el î?i întindea mâna ?i-l îmbra?i?a ?i-l saruta.
\par 6 A?a se purta Abesalom cu tot israelitul care venea pentru judecata la rege ?i a intrat Abesalom la inima Israeli?ilor.
\par 7 Dupa patruzeci de ani de domnie a lui David, a zis Abesalom catre rege: "Ma duc la Hebron sa-mi împlinesc o fagaduin?a, pe care am facut-o Domnului,
\par 8 Caci eu, robul tau, pe când traiam la Ghe?ur în Siria, am facut fagaduin?a aceasta: De ma va întoarce Domnul la Ierusalim, voi aduce jertfa Domnului".
\par 9 ?i i-a zis regele: "Du-te cu pace!" ?i el s-a sculat ?i s-a dus la Hebron.
\par 10 Atunci a trimis Abesalom ?apte fete la toate triburile lui Israel, zicând: "Când ve?i auzi sunetul cornului, sa zice?i: Abesalom s-a facut rege în Hebron".
\par 11 ?i s-au dus cu Abesalom doua sute de oameni din Ierusalim, care fusesera pofti?i de el, dar s-au dus din nevinova?ie, ne?tiind ce este la mijloc.
\par 12 În timpul jertfei, Abesalom a trimis ?i a chemat pe Ahitofel Ghiloneanul, sfetnicul lui David, din cetatea lui, Ghilo. ?i s-a facut razvratire mare ?i curgea poporul ?i se înmul?ea împrejurul lui Abesalom.
\par 13 Deci a venit un vestitor la David ?i a zis: "Inima Israeli?ilor a înclinat în partea lui Abesalom".
\par 14 Iar David a zis catre toate slugile sale, care erau cu el în Ierusalim: "Scula?i-va sa fugim, caci nu vom scapa de Abesalom. Grabi?i-va sa plecam, ca sa nu ne ajunga ?i sa ne prinda, ca sa nu aduca necaz asupra noastra ?i sa strice ceta?ile cu sabia".
\par 15 ?i slugile regelui au zis catre rege: "La tot ce va voi regele, stapânul nostru, noi slugile tale suntem gata".
\par 16 ?i a ie?it regele pe jos ?i dupa el a mers toata casa lui. Regele însa a lasat zece femei din concubinele sale, ca sa pazeasca casa.
\par 17 ?i au plecat regele ?i tot poporul pe jos ?i s-au oprit la Bet-Merhac.
\par 18 Toate slugile lui mergeau pe lânga el, iar to?i Cheretienii ?i to?i Peletienii ?i to?i Gateienii, ca la ?ase sute de oameni, care venisera împreuna cu el din Gat, mergeau înaintea regelui.
\par 19 "Pentru ce mergi ?i tu cu noi? a zis regele catre Itai din Gat. Întoarce-te ?i ramâi cu regele, caci tu e?ti strain ?i ai venit aici din ?ara ta.
\par 20 Ieri ai venit ?i astazi sa te silesc sa mergi cu noi? Eu ma duc unde se va întâmpla. Întoarce-te ?i întoarce ?i pe fra?ii tai cu tine. Domnul sa faca mila ?i dreptate cu tine".
\par 21 "Precum e adevarat ca Domnul este viu, a raspuns Itai regelui, ?i precum este viu regele, stapânul meu, tot a?a este de adevarat ca oriunde va fi regele, stapânul meu, la via?a ?i la moarte, acolo va fi ?i robul tau".
\par 22 "Atunci, a zis regele David catre Itai, vino ?i umbla cu mine". ?i s-a dus Itai din Gat ?i to?i oamenii lui ?i to?i copiii care erau cu el.
\par 23 ?i a plâns toata ?ara cu glas mare ?i tot poporul a trecut pârâul Chedron ?i a trecut ?i regele pârâul Chedron ?i s-a dus tot poporul cu regele pe calea spre pustiu.
\par 24 ?i iata era acolo ?i ?adoc preotul, împreuna cu to?i Levi?ii care duceau chivotul legamântului Domnului din Betar ?i au pus acolo chivotul lui Dumnezeu; iar Abiatar a stat pe un loc înalt pâna ce a ie?it tot poporul din cetate.
\par 25 "Întoarce chivotul lui Dumnezeu în cetate, a zis regele catre ?adoc, ca sa stea la locul lui. De voi afla mila în ochii Domnului, ma va întoarce ?i-mi va da sa-L vad pe El ?i loca?ul Lui.
\par 26 Iar daca El îmi va zice: "Nu mai este bunavoin?a Mea cu tine", atunci iata-ma, faca cu mine ce va binevoi".
\par 27 "În?elegi tu? a mai zis regele catre ?adoc preotul, întoarce-te cu pace în cetate, cu Ahimaa?, fiul tau ?i cu Ionatan, fiul lui Abiatar, amândoi fiii vo?tri.
\par 28 Sa ?ti?i, eu am sa ramân în câmpia din pustiu pâna îmi va veni veste de la voi".
\par 29 Atunci ?adoc ?i Abiatar au întors chivotul lui Dumnezeu în Ierusalim ?i au ramas acolo.
\par 30 Iar David s-a dus în Muntele Eleonului ?i, mergând, plângea; capul îi era acoperit ?i picioarele descul?e. ?i to?i oamenii care erau cu el î?i acoperisera fiecare capul ?i mergeau plângând.
\par 31 Atunci s-a spus lui David: "?i Ahitofel este în numarul razvrati?ilor cu Abesalom". "Doamne, Dumnezeul meu, a zis atunci David, risipe?te planurile lui Ahitofel!"
\par 32 Iar când David a ajuns pe vârful muntelui, unde s-a închinat lui Dumnezeu, iata a venit în întâmpinarea lui Hu?ai Archianul, cel mai bun prieten al lui David. Acesta avea haina sfâ?iata ?i pe cap cenu?a.
\par 33 "De vei merge cu mine, i-a zis David, îmi vei fi o povara.
\par 34 Iar de te vei întoarce în cetate ?i vei zice lui Abesalom: "Rege, fra?ii tai au trecut; a trecut ?i regele, tatal tau, ?i acum eu sunt robul tau; iasa-ma cu via?a. Pâna acum am fast robul tatalui tau, iar acum sunt robul tau". Atunci vei strica planurile lui Ahitofel cele împotriva mea.
\par 35 Iata este acolo cu tine ?adoc ?i Abiatar preo?ii ?i tot cuvântul ce vei auzi la casa regelui sa-l spui preo?ilor ?adoc ?i Abiatar.
\par 36 Acolo sunt ?i cei doi fii ai lor: Ahimaa?, fiul lui ?adoc ?i Ionatan, fiul lui Abiatar. Prin ace?tia sa trimite?i la mine orice veste ve?i auzi".
\par 37 ?i a venit Hu?ai, prietenul lui David, în cetate. Abesalom însa intra atunci în Ierusalim.

\chapter{16}

\par 1 Dupa ce David a trecut pu?in de vârful muntelui, iata îl întâmpina ?iba, sluga lui Mefibo?et, cu o pereche de asini încarca?i, pe care se aflau doua sute de pâini, o suta de legaturi de stafide, o suta de legaturi de smochine ?i un burduf de vin.
\par 2 ?i regele a zis catre ?iba: "Ce sunt acestea?" "Asinii, a raspuns ?iba, sunt pentru rege, ca sa umble, pâinile ?i fructele pentru hrana oamenilor, iar vinul, ca sa bea cei ce vor slabi în pustie".
\par 3 "Unde este fiul stapânului tau?" a întrebat regele. "A ramas în Ierusalim", a raspuns ?iba regelui; caci a zis: "Acum casa lui Israel îmi va întoarce mie domnia tatalui meu".
\par 4 "Ale tale sa fie toate câte are Mefibo?et", a zis regele catre ?iba. "Sa aflu mila în ochii domnului meu, regele", a raspuns ?iba, închinându-se.
\par 5 Iar când a ajuns regele David la Bahurim, ie?ea de acolo un om din neamul casei lui Saul, cu numele de ?imei, fiul lui Ghera. El mergea ?i blestema,
\par 6 Aruncând cu pietre asupra lui David ?i asupra tuturor robilor lui David; Iar poporul tot ?i to?i oamenii de lupta erau la dreapta ?i la stânga regelui.
\par 7 "Pleaca, pleaca, uciga?ule ?i nelegiuitule, zicea ?imei, blestemând pe rege.
\par 8 Domnul a întors asupra ta tot sângele casei lui Saul, în locul caruia te-ai facut tu rege ?i a dat Domnul domnia în mâinile lui Abesalom; fiul tau; ?i iata tu e?ti în necaz, pentru ca e?ti bautor de sânge".
\par 9 "Pentru ce acest câine le?inat blesteama pe domnul meu, rege?, a zis Abi?ai, fiul ?eruiei. Ma duc sa-i iau capul".
\par 10 "Fiii ?eruiei, a zis regele, ce ne prive?te aceasta pe mine ?i pe voi? Lasa?i-l sa blesteme, caci Domnul i-a poruncit sa blesteme pe David. Cine deci poate sa-i zica: "De ce faci tu a?a?"
\par 11 Apoi David a mai zis lui Abi?ai ?i tuturor slugilor sale: "Iata, daca fiul; meu, care a ie?it din coapsele mele, cauta sufletul meu, cu atât mai vârtos fiul unui veniaminean. Lasa?i-l sa blesteme, caci Domnul i-a poruncit.
\par 12 Poate va cauta Domnul la umilirea mea ?i-mi va rasplati cu bine pentru acest blestem al lui".
\par 13 ?i s-a dus David ?i oamenii lui în drumul lor, iar ?imei mergea pe coasta muntelui în preajma lui, mergea ?i blestema, aruncând spre el cu pietre ?i cu praf.
\par 14 Apoi, ajungând regele ?i tot poporul ce era cu el la Aiefim, s-a odihnit acolo.
\par 15 Abesalom însa ?i tot poporul lui Israel au venit în Ierusalim ?i împreuna cu ei a venit ?i Ahitofel.
\par 16 ?i când Hu?ai Archianul, prietenul lui David, a venit la Abesalom ?i i-a zis: "Traiasca regele!"
\par 17 Abesalom a zis catre Hu?ai: "A?a dragoste ai tu catre prietenul tau? De ce nu te-ai dus ?i tu cu prietenul tau?"
\par 18 "Nu, a zis Hu?ai catre Abesalom, eu urmez pe acela pe care l-a ales Domnul ?i acest popor ?i tot Israelul; cu acela sunt eu ?i cu acela ramân.
\par 19 ?i apoi cui am sa slujesc? Oare nu fiului sau? Cum am slujit tatalui tau, a?a am sa-?i slujesc ?i ?ie".
\par 20 "Da?i-mi sfat, a zis Abesalom catre Ahitofel, ce sa facem!"
\par 21 "Intra la concubinele tatalui tau, a raspuns Ahitofel, pe care le-a lasat el sa pazeasca casa sa; ?i vor auzi to?i Israeli?ii ca tu ai ajuns sa fii urât de tatal tau ?i se vor întari mâinile tuturor celor ce sunt cu tine".
\par 22 Atunci au întins pentru Abesalom un cort pe acoperi?ul casei. ?i a intrat Abesalom la concubinele tatalui sau, înaintea ochilor a tot Israelul.
\par 23 Iar sfaturile lui Ahitofel, pe care le dadea el, se socoteau atunci ca ?i cum ar fi cerut cineva pova?a de la Dumnezeu. A?a fusese orice sfat al lui Ahitofel atât pentru David, cât ?i pentru Abesalom.

\chapter{17}

\par 1 "Eu, a zis Ahitofel catre Abesalom, am sa aleg douasprezece mii de oameni ?i ma voi ridica sa ma duc noaptea asta în urmarirea lui David;
\par 2 ?i voi navali asupra lui când va fi ostenit ?i cu mâinile slabanogite, ?i-l voi umple de groaza ?i to?i oamenii care sunt cu el se vor împra?tia ?i voi ucide numai pe rege,
\par 3 Iar pe oameni îi voi întoarce pe to?i la tine. ?i când nu va mai fi unul, al carui suflet îl cau?i tu, atunci tot poporul va fi în pace".
\par 4 ?i a placut vorba aceasta lui Abesalom ?i tuturor batrânilor lui Israel.
\par 5 "Chema?i pe Hu?ai Archianul, a zis Abesalom, sa auzim ce zice el".
\par 6 Atunci a venit Hu?ai la Abesalom ?i Abesalom i-a zis: "Iata ce zice Ahitofel; sa facem oare cum zice el? Iar daca nu, spune-mi tu!"
\par 7 "De data aceasta, a zis Hu?ai catre Abesalom, nu este bun sfatul pe care l-a dat Ahitofel".
\par 8 "Tu cuno?ti pe tatal tau ?i pe oamenii lui, urma mai departe Hu?ai. Ei sunt viteji ?i foarte îndârji?i, ca ursoaica pustiului când i se rapesc puii ?i ca vierul salbatic din câmp. ?i apoi tatal tau este om razboinic; el nu sta sa ramâna cu poporul.
\par 9 Iata acum ei, de buna seama, se ascunde în vreo pe?tera sau în alt loc; ?i de cade cineva la cel dintâi atac asupra lor, se va auzi ?i se va zice: "Au fost înfrân?i oamenii care au urmat lui Abesalom".
\par 10 Atunci pâna ?i cel mai viteaz care are inima ca de leu va cadea cu duhul; caci la tot Israelul este cunoscut cât de viteaz este tatal tau ?i cât de curajo?i sunt cei ce se afla cu el.
\par 11 De aceea eu va sfatuiesc: Sa se adune la tine tot Israelul de la Dan pâna la Beer?eba, la numar tocmai ca nisipul marii, ?i sa mergi tu însu?i în mijlocul lor.
\par 12 Atunci vom merge asupra lui, în orice loc s-ar afla, ?i vom navali asupra lui, cum cade roua pe pamânt, ?i nu-i va mai ramâne pe lânga el nici un om din cei ce se afla cu el.
\par 13 Iar de va intra în vreo cetate, atunci tot Israelul va aduce frânghii la cetatea aceea ?i o vom târî în râu, încât nu va ramâne din ea nici pietricica".
\par 14 "Sfatul lui Hu?ai Archianul, au zis atunci Abesalom ?i tot Israelul, e mai bun decât sfatul lui. Ahitofel". A?a a judecat Domnul sa strice sfatul cel mai bun al lui Ahitofel, ca sa aduca Domnul pieirea asupra lui Abesalom.
\par 15 Apoi Hu?ai a zis catre preo?ii ?adoc ?i Abiatar: "A?a ?i a?a a sfatuit Ahitofel pe Abesalom ?i pe batrânii lui Israel, iar eu i-am sfatuit a?a ?i a?a".
\par 16 Deci trimite?i acum repede sa spuna lui David a?a: Tu, noaptea asta, sa nu ramâi pe câmp în pustiu, ci sa treci mai repede, ca sa nu piara regele ?i oamenii care sunt cu el".
\par 17 În vremea aceasta Ionatan ?i Ahimaa? stateau la En-Roghel. Deci s-a dus o slujnica ?i le-a spus acestora, iar ace?tia s-au dus ?i au vestit pe regele David, caci ei nu se puteau arata în cetate.
\par 18 Dar i-a zarit un tânar ?i a spus lui Abesalom. Ei însa au plecat amândoi repede ?i s-au dus la Bahurim, în casa unui om, în ograda caruia se afla o fântâna, ?i s-au ascuns în ea;
\par 19 Iar femeia omului a luat o patura ?i a întins-o pe gura fântânii, a pus pe ea ni?te grâu pisat, a?a încât nu se vedea nimic.
\par 20 ?i venind slujitorii lui Abesalom în casa la femeie, i-au zis: "Unde sunt Ahimaa? ?i Ionatan?" "Au trecut pârâul", le-a raspuns femeia. ?i i-au cautat oamenii lui Abesalom, dar nu i-au gasit ?i s-au întors la Ierusalim.
\par 21 Iar daca au plecat ace?tia, ei au ie?it din fântâna ?i s-au dus de au spus lui David: "Ridica?i-va ?i trece?i repede apa, caci acestea a sfatuit Ahitofel împotriva voastra".
\par 22 Atunci s-a ridicat David ?i to?i oamenii care erau cu el ?i au trecut Iordanul ?i pâna la ziua n-a ramas niciunul care sa nu fi trecut.
\par 23 Ahitofel însa, vazând ca planul sau n-a fost urmat, a pus ?aua pe asin, a plecat ?i s-a dus la casa sa, în cetatea sa, ?i ?i-a facut testamentul în folosul casei sale, apoi s-a spânzurat ?i a murit ?i a fost înmormântat în cetatea tatalui sau.
\par 24 David a venit dupa aceea la Mahanaim, iar Abesalom a trecut Iordanul ?i tot Israelul era cu el.
\par 25 În locul lui Ioab, Abesalom a pus peste o?tire pe Amasa. Amasa era fiul unui om cu numele Itra, din Israel, care intrase la Abigail, fiica lui Naha?, sora ?eruiei, mama lui Ioab.
\par 26 Israel cu Abesalom ?i-au a?ezat tabara în ?inutul Galaad.
\par 27 Când David a venit la Mahanaim, ?obi, fiul lui Naha?, din "Raba Amoni?ilor, Machir, fiul lui Amiel din Lodebar, ?i Barzilai Galaaditeanul, din Roghelim,
\par 28 Au adus zece paturi pregatite, zece talere, vase de lut, grâu, orz, faina, graun?e prajite, bob, linte ?i pâine;
\par 29 Miere, unt, oi ?i brânza de vaci ?i le-au dat lui David ?i oamenilor care erau cu el, caci zicea: "Poporul este flamând ?i ostenit ?i a suferit de sete în pustie".

\chapter{18}

\par 1 Atunci a numarat David pe oamenii care erau cu el ?i a pus capetenii peste sute ?i capetenii peste mii.
\par 2 ?i a trimis David pe oamenii sai: a treia parte sub comanda lui Ioab, a treia parte sub comanda lui Abi?ai, fiul ?eruiei, fratele lui Ioab, ?i a treia parte sub comanda lui Itai Gateul. ?i a zis regele catre oameni: "?i eu însumi voi merge cu voi".
\par 3 "Sa nu mergi, i-au zis oamenii, ca noi chiar de vom fugi, nu se va ?ine seama de aceasta; ?i chiar de ar muri jumatate din noi, de asemenea nu se va ?ine seama; iar tu singur e?ti cât noi, zece mii. Deci pentru noi este mai bine ca tu sa ne dai ajutor din cetate!"
\par 4 "Ce vi se pare ca este bine, a raspuns regele, aceea voi face". Atunci a stat regele la poarta ?i a ie?it tot poporul rânduit pe sute ?i pe mii.
\par 5 Apoi regele a dat porunca lui Ioab, lui Abi?ai ?i lui Itai ?i le-a zis: "Sa-mi cru?a?i pe baiatul Abesalom!" ?i tot poporul a auzit cum a poruncit regele tuturor capeteniilor pentru Abesalom.
\par 6 Au ie?it deci oamenii la câmp în întâmpinarea Israeli?ilor, ?i s-a dat batalia în padurea lui Efraim.
\par 7 Acolo a fost în ziua aceea batalie mare; poporul israelit a fost înfrânt de robii lui David, cazând uci?i douazeci de mii de oameni.
\par 8 Lupta s-a întins în tot ?inutul acela ?i padurea a mâncat în ziua aceea mai mul?i oameni decât a doborât sabia.
\par 9 Când s-a întâlnit Abesalom cu oamenii lui David, era calare pe un catâr. Când catârul a fugit cu el pe sub cracile unui stejar mare, parul lui Abesalom s-a încurcat în crengile stejarului ?i el a ramas spânzurat în vazduh, iar catârul de sub el s-a dus înainte.
\par 10 Atunci cineva a vazut aceasta ?i a spus lui Ioab, zicând: "Iata, am vazut pe Abesalom spânzurat de un stejar".
\par 11 "Daca l-ai vazut, a zis Ioab catre omul care-i adusese vestea, de ce nu l-ai doborât acolo, la pamânt? ?i-a? fi dat zece sicli de argint ?i o cingatoare!"
\par 12 "De mi-ai fi pus în mâna ?i o mie de sicli de argint, a raspuns acela lui Ioab, nici atunci nu mi-a? fi ridicat mâna asupra fiului regelui, caci în auzul nostru ?i-a poruncit regele ?ie ?i lui Abi?ai ?i lui Itai ?i a zis: "Cru?a?i-mi pe baiatul Abesalom!"
\par 13 ?i daca eu însumi a? fi facut altfel cu primejdia vie?ii mele, aceasta nu s-ar fi putut ascunde de rege ?i tu singur te-ai fi ridicat asupra mea".
\par 14 "N-am la ce sa mai zabovesc cu tine", a zis Ioab. Apoi a luat în mâna trei sage?i ?i le-a înfipt în inima lui Abesalom, care era înca viu în crengile stejarului.
\par 15 Apoi au împresurat pe Abesalom zece tineri care duceau armele lui Ioab ?i, lovindu-l, l-au ucis.
\par 16 Dupa aceea Ioab a sunat din trâmbi?a ?i s-au întors oamenii de la urmarirea lui Israel, caci Ioab a cru?at poporul.
\par 17 Apoi au luat pe Abesalom ?i l-au aruncat acolo în padure într-o groapa adânca ?i au aruncat deasupra lui o gramada mare de pietre. Iar Israeli?ii s-au împra?tiat cu to?ii, ducându-se fiecare la casa sa.
\par 18 Abesalom însa î?i facuse un monument înca de pe când traia, în Valea Regelui; caci î?i zisese: "Eu n-am fiu, ca sa mi se pastreze amintirea!" ?i a dat monumentului numele sau, a?a ca ?i astazi se nume?te el: Monumentul lui Abesalom.
\par 19 Iar Ahimaa?, fiul lui ?adoc, a zis catre Ioab: "Ma duc sa vestesc pe regele ca Domnul prin judecata Sa l-a izbavit din mâinile du?manilor lui".
\par 20 "Astazi, a raspuns Ioab, n-ai sa fii un bun vestitor. Îl vei vesti în alta zi, iar nu astazi, caci a murit fiul regelui".
\par 21 "Du-te, a zis apoi Ioab catre Hu?ai, du-te ?i spune regelui ceea ce ai vazut". Hu?ai, închinându-se înaintea lui Ioab, s-a dus numaidecât,
\par 22 Iar Ahimaa?, fiul lui ?adoc, a zis staruitor catre Ioab: "Fie ce-o fi, dar eu ma duc cu Hu?ai". "De ce sa te duci, fiul meu? a zis Ioab. Nu duci o veste buna!"
\par 23 "Fie ?i a?a, a raspuns Ahimaa?, dar eu tot ma duc". "Du-te", i-a zis Ioab. ?i a apucat Ahimaa? la fuga pe un drum mai de-a dreptul ?i a întrecut pe Hu?ai.
\par 24 David ?edea atunci între cele doua por?i; iar straja se suise pe acoperi?ul por?ii, pe zid, ?i, ridicându-?i ochii, a vazut pe cel ce venea ?i a strigat ?i a zis catre rege: "Iata un om vine în fuga".
\par 25 "Daca este numai unul, a zis regele, atunci ne aduce o veste". Omul însa se apropia din ce în ce mai mult.
\par 26 Atunci straja a mai vazut un om venind în fuga; ?i a strigat straja la portar ?i a zis: "Iata ca mai alearga un om". "?i acela este un vestitor", a zis regele. "Eu vad, a zis straja, ca mersul omului dinainte seamana cu mersul lui Ahimaa?, fiul lui ?adoc".
\par 27 "Acesta este om bun, a zis regele, ?i vine cu veste buna!"
\par 28 ?i a strigat Ahimaa? ?i a zis catre rege: "Pace!" Apoi s-a închinat regelui cu fa?a pâna la pamânt ?i a zis: "Binecuvântat este Domnul Dumnezeul tau, Care a dat în mâinile noastre pe oamenii care î?i ridicasera mâinile lor împotriva regelui, stapânul meu!"
\par 29 "Dar baiatul Abesalom este sanatos?" a întrebat regele. "Am vazut tulburare mare acolo când Ioab, robul regelui, a trimis pe robul tau, a zis Ahimaa?, dar eu nu ?tiu ce era acolo".
\par 30 "Treci ?i ramâi aici", a zis regele. ?i Ahimaa? a trecut ?i a stat acolo.
\par 31 Atunci a sosit ?i Hu?ai. ?i Hu?ai a zis catre rege: "Veste buna aduc regelui, stapânul meu! Domnul ?i-a facut astazi dreptate, izbavindu-te din mâna tuturor celor ce s-au ridicat împotriva ta".
\par 32 "Baiatul Abesalom este oare el  sanatos?", a întrebat regele pe Hu?ai. "Întâmple-se du?manilor regelui, stapânul meu, a raspuns Hu?ai, ?i tuturor celor ce au uneltit rele împotriva ta, ce i s-a întâmplat lui!"
\par 33 Atunci regele s-a tulburat ?i s-a dus în foi?orul de deasupra por?ii ?i a plâns, iar când se ducea, zicea: "O, fiul meu Abesalom, Abesalom, fiul meu! Mai bine muream eu în locul tau! Abesalom, Abesalom, fiul meu!"

\chapter{19}

\par 1 Atunci i s-a spus lui Ioab: "Iata regele plânge dupa Abesalom".
\par 2 Astfel biruin?a din ziua aceea s-a prefacut în plângere pentru tot poporul, caci poporul a auzit chiar în ziua aceea ?i zicea ca regele este întristat dupa fiul sau.
\par 3 Atunci a intrat poporul în cetate pe furi?, cum se furi?eaza oamenii ru?ina?i care au luat-o la fuga în timpul luptei.
\par 4 Regele însa, acoperindu-?i fa?a, striga tare: "Abesalom, Abesalom, fiul meu!"
\par 5 Dar venind Ioab la rege în casa, a zis: "Tu astazi ai umplut de ru?ine pe toate slugile tale care au izbavit acum via?a ta ?i via?a fiilor ?i fiicelor tale, via?a femeilor ?i concubinelor tale.
\par 6 Tu iube?ti pe cei ce te urasc ?i pe cei ce te iubesc îi ura?ti; caci ai aratat astazi ca pentru tine sunt nimic ?i capeteniile ?i slugile; astazi am aflat eu ca de ar fi ramas Abesalom cu via?a, iar noi am fi murit cu to?ii, aceasta ?i-ar fi fost mai placut.
\par 7 Deci, scoala ?i ie?i de graie?te dupa inima slugilor tale. Caci ma jur pe Domnul ca daca nu ie?i, în noaptea aceasta nu-?i va mai ramâne nici un om. ?i aceasta va fi pentru tine cea mai mare din toate nenorocirile care au venit asupra ta din tinere?ea ta ?i pâna acum!"
\par 8 Atunci s-a sculat regele ?i a ?ezut la poarta, iar poporului întreg s-a vestit ca regele sta la poarta. ?i a venit tot poporul în fa?a regelui la poarta, iar Israeli?ii au fugit pe la vetrele lor.
\par 9 Tot poporul din toate triburile lui Israel vorbea ?i zicea: "Regele David ne-a izbavit din mâinile vrajma?ilor no?tri ?i ne-a scapat din mâinile Filistenilor, iar acum el însu?i a fugit din ?ara aceasta, din regatul sau, de frica lui Abesalom.
\par 10 Abesalom insa, pe care noi l-am uns rege pentru noi, a murit în razboi. Pentru ce dar întârziem noi acum a aduce înapoi pe rege?" ?i aceste vorbe au strabatut tot Israelul ?i au ajuns ?i pâna la rege.
\par 11 Atunci regele David a trimis sa se spuna preo?ilor ?adoc ?i Abiatar: "Spune?i batrânilor lui luda: Pentru ce voi?i sa fi?i cei din urma în a aduce înapoi pe rege, când cuvintele a tot Israelul au ajuns pâna la rege, în casa lui?
\par 12 Voi sunte?i fra?ii mei; oasele mele ?i carnea mea voi sunte?i. Pentru ce dar voi?i sa fi?i cei din urma în a aduce pe rege înapoi?
\par 13 Iar lui Amasa sa-i zice?i: "Nu e?ti tu oare osul meu ?i carnea mea? A?a ?i a?a sa-mi faca mie Dumnezeu ?i înca ?i mai rau sa-mi faca, daca tu nu ai sa fii capetenia o?tirii mele pentru totdeauna în locul lui Ioab!"
\par 14 ?i a?a a înduplecat regele inima tuturor Iudeilor ca a unui singur om; iar ace?tia au trimis la rege sa-i spuna: "Întoarce-te tu însu?i ?i toate slugile tale!
\par 15 ?i s-a întors regele ?i a venit la Iordan, iar Iudeii au venit la Ghilgal, ca sa întâmpine pe rege ?i sa-l treaca Iordanul.
\par 16 Atunci ?imei, fiul lui Ghera, un veniaminean din Bahurim, s-a grabit sa iasa cu Iudeii în întâmpinarea regelui David.
\par 17 Acesta avea cu el o mie de oameni veniamineni, ?i pe ?iba, sluga casei lui Saul, cu cei cincisprezece fii ai sai ?i cu douazeci de robi ai sai. Ace?tia au trecut Iordanul înaintea regelui ?i au pregatit pentru rege trecerea Iordanului.
\par 18 Când însa au pornit luntrea, ca sa aduca pe rege ?i casa lui, ca sa-i slujeasca, atunci ?imei, fiul lui Ghera, a cazut cu fa?a la pamânt înaintea regelui, îndata ce acesta a trecut Iordanul,
\par 19 ?i a zis catre rege: "Domnul meu, sa nu-mi socote?ti ca o nelegiuire ?i sa nu pomene?ti ceea ce ?i-a gre?it robul tau în ziua aceea când regele, stapânul meu, a ie?it din Ierusalim ?i sa nu iei în seama, o, rege, aceasta!
\par 20 Caci robul tau ?tie ca a gre?it. ?i iata eu acum am venit cel dintâi din toata casa lui Iosif, ca sa ies în întâmpinarea regelui, stapânul meu".
\par 21 "Se poate oare, a zis Abi?ai, fiul ?eruiei, ca ?imei sa nu moara pentru ca a blestemat pe unsul Domnului?"
\par 22 "Fiii ?eruiei, ce este între mine ?i voi? a zis David, ?i pentru ce va împotrivi?i voi astazi mie? E timpul oare astazi sa se ucida cineva în Israel? Nu vad eu, oare, acum ca sunt rege peste Israel?"
\par 23 Iar lui ?imei i-a zis regele: "Nu, tu nu vei muri!" ?i i s-a jurat regele.
\par 24 Mefibo?et, fiul lui Ionatan, fiul lui Saul, a ie?it în întâmpinarea regelui. Acesta, din ziua în care ie?ise regele ?i pâna în ziua când se întorsese cu pace, nu-?i mai spalase picioarele, nici nu-?i mai taiase unghiile, nici nu-?i îngrijise barba ?i nici hainele nu ?i le mai spalase.
\par 25 ?i când a ie?it din Ierusalim în întâmpinarea regelui, regele i-a zis: "Mefibo?et, pentru ce n-ai mers ?i tu cu mine?"
\par 26 "Stapânul meu, rege, a raspuns acela, sluga mea m-a în?elat, caci eu, robul tau, am zis: "Voi pune ?aua pe asin ?i voi încaleca ?i ma voi duce c; regele, deoarece robul tau este olog.
\par 27 Iar el a clevetit pe robul tau înaintea regelui, stapânul meu. Dar regele, stapânul meu, este ca un înger al lui Dumnezeu. Fa ce binevoie?ti.
\par 28 De?i toata casa tatalui meu a fost vinovata de moarte înaintea domnului meu, regele, totu?i tu ai pus pe robul tau printre cei ce manânca la masa ta. Ce drept am eu oare sa ma jeluiesc înaintea regelui?"
\par 29 "De ce graie?ti tu toate acestea? i-a zis regele. Eu am zis ca tu ?i ?iba sa împar?i?i între voi ?arina!"
\par 30 "Sa ia chiar toata ?arina, a raspuns Mefibo?et. Bine ca s-a întors cu pace domnul meu, regele, la casa sa!"
\par 31 Atunci a venit ?i Barzilai Galaaditeanul din Roghelim ?i a trecut Iordanul cu regele, ca sa-l petreaca pe acesta pâna peste Iordan.
\par 32 Barzilai însa era foarte batrân, ca de optzeci de ani, ?i el ospatase pe rege în timpul ?ederii lui la Mahanaim, pentru ca era om foarte bogat.
\par 33 "Hai cu mine, a zis regele catre Barzilai, ?i te voi ospata ?i eu în Ierusalim".
\par 34 "Mult oare mi-a mai ramas de trait, a raspuns Barzilai, ca sa merg cu regele la Ierusalim?
\par 35 Eu am acum optzeci de ani. Mai pot eu oare osebi binele de rau? ?i va afla oare robul tau gustul celor ce va mânca ?i va bea? Sau voi fi eu în stare sa aud glasul cântarelilor ?i cântare?elor? La ce dar sa fie robul tau o povara pentru domnul meu, regele?
\par 36 Robul tau va mai merge pu?in dincolo de Iordan cu regele. Pentru ce sa-mi rasplateasca regele cu a?a mila?
\par 37 Da voie robului tau sa se întoarca, ca sa moara în cetatea sa, lânga mormântul tatalui meu ?i al mamei mele. Dar iata fiul meu Chimham, robul tau! Sa mearga el cu domnul meu, regele, ?i fa cu el ceea ce binevoie?ti!
\par 38 "Sa mearga cu mine Chimham, a zis regele, ?i voi face cu el ce vrei ?i orice vei voi tu de la mine voi face pentru tine!"
\par 39 ?i a trecut tot poporul Iordanul, de asemenea ?i regele. Apoi a sarutat regele pe Barzilai ?i l-a binecuvântat ?i acesta s-a întors la casa sa.
\par 40 Dupa aceea regele s-a îndreptat spre Ghilgal ?i s-a dus cu el ?i Chimham, ?i tot poporul lui Iuda ?i jumatate din poporul lui Israel a petrecut pe rege.
\par 41 Dar iata tot Israelul a venit la rege ?i a zis catre el: "Pentru ce barba?ii lui Iuda, fra?ii no?tri, te-au rapit ?i au trecut pe rege ?i casa lui ?i pe to?i oamenii lui David peste Iordan?"
\par 42 "Pentru aceea, ca regele este mai apropiat de noi, au raspuns Israeli?ilor to?i barba?ii lui Iuda. ?i de ce sa va supara?i voi pentru aceasta? Am mâncat noi, oare, ceva de la rege, sau am primit daruri de la el, sau ne-a scutit de dari?"
\par 43 ?i Israeli?ii au raspuns barba?ilor lui Iuda: "Noi suntem zece par?i din rege ?i noi suntem înca ?i întâi nascu?i fa?a de voi. Pentru ce dar ne-a?i dispre?uit? Oare nu noi trebuia sa spunem cel dintâi cuvânt pentru întoarcerea regelui?" Dar cuvântul barba?ilor lui Iuda a fost mai puternic decât cuvântul Israeli?ilor.

\chapter{20}

\par 1 Din întâmplare, se afla acolo un om netrebnic, anume ?eba, fiul lui Bicri veniamineanul. Acesta a sunat din trâmbi?a ?i a zis: "Noi n-avem nici o împarta?ire cu David ?i nici o legatura cu fiul lui Iesei! Fiecare la cortul sau, Israele!"
\par 2 Atunci s-au despar?it de David to?i Israeli?ii ?i s-au dus dupa ?eba, fiul lui Bicri. Iudeii însa au ramas de partea regelui lor, de la Iordan, pâna la Ierusalim.
\par 3 Ajungând apoi David la casa sa în Ierusalim, a luat regele pe cele zece concubine pe care le lasase sa aiba în grija casa ?i le-a pus într-o casa deosebita, sub supraveghere ?i le purta de grija, dar nu se ducea la ele. ?i au trait ele acolo pâna la moartea lor, ca vaduve.
\par 4 Apoi a zis David catre Amasa: "În timp de trei zile cheama pe Iudei la mine ?i sa vii ?i tu aici!"
\par 5 ?i s-a dus Amasa sa cheme pe Iudei, dar a zabovit mai multa vreme decât i se daduse.
\par 6 Atunci David a zis lui Abi?ai: "Acum ?eba, fiul lui Bicri, are sa ne faca mai mult rau decât Abesalom. Ia tu slugile stapânului tau ?i urmare?te-l, ca sa nu-?i gaseasca ceta?i întarite ?i sa nu scape din ochii no?tri!"
\par 7 ?i a plecat Abi?ai, urmat de oamenii lui Ioab, de Cheretieni ?i Peletieni ?i de to?i oamenii buni de lupta din Ierusalim, sa urmareasca pe ?eba, fiul lui Bicri.
\par 8 Dar pe când erau ei aproape de piatra cea mare, de lânga Ghibeon, s-a întâlnit cu ei Amasa. Ioab era îmbracat cu hainele sale osta?e?ti ?i încins cu sabia care atârna la ?old în teaca ei, în care intra ?i ie?ea foarte u?or.
\par 9 "E?ti sanatos, frate?" a zis Ioab catre Amasa. Apoi a apucat Ioab pe Amasa cu mâna de barba, ca sa-l sarute.
\par 10 Amasa însa nu s-a ferit de sabia care era în mâna lui Ioab ?i acesta l-a lovit cu ea în pântece ?i i-a varsat maruntaiele jos ?i nu i-a mai dat alta lovitura. ?i Amasa a murit. Apoi Ioab cu fratele sau Abi?ai au alergat dupa ?eba, fiul lui Bicri.
\par 11 Unul din oamenii lui Ioab însa a ramas lânga Amasa ?i striga: "Cine vrea pe Ioab ?i cine este pentru David sa urmeze pe Ioab!"
\par 12 Amasa însa zacea mort în sânge, în mijlocul drumului. Dar vazând omul acela al lui Ioab ca tot poporul se opre?te la trupul lui Amasa, l-a târât pe acesta din drum în câmp ?i a aruncat peste el o haina, deoarece vedea ca orice trecator se apropia de el.
\par 13 Iar dupa ce a fost târât din drum, tot poporul lui Israel s-a dus dupa Ioab, sa urmareasca pe ?eba, fiul lui Bicri.
\par 14 Ioab însa a trecut prin toate triburile israelite pâna la Abel ?i Bet-Maaca ?i prin tot ?inutul Berim ?i to?i locuitorii ceta?ilor s-au adunat ?i au mers dupa el.
\par 15 Venind apoi, au împresurat pe ?eba în Abel ?i în Bet-Maaca ?i au ridicat un val împrejurul ceta?ii; ?i apropiindu-se de zid, to?i cei ce erau cu Ioab se sileau sa sparga zidul.
\par 16 Atunci o femeie în?eleapta a strigat de pe zidul ceta?ii: "Asculta?i, asculta?i, spune?i lui Ioab sa vina încoace, ca am sa-i vorbesc!"
\par 17 Apropiindu-se Ioab, femeia a zis: "Tu e?ti Ioab?" "Eu!" a raspuns Ioab. "Asculta vorba roabei tale!" a zis, femeia. "Ascult!" a raspuns Ioab.
\par 18 "Odinioara era obiceiul, a adaugat ea, sa se spuna: Sa se întrebe în Abel ?i în Dan, daca se mai pastreaza ceea ce au hotarât credincio?ii din Israel
\par 19 Eu sunt din ceta?ile pa?nice ?i credincioase ale lui Israel, ?i tu voie?ti sa strici o cetate ?i înca mama ceta?ilor lui Israel. De ce sa strici tu mo?tenirea Domnului?"
\par 20 "Departe, departe de mine gândul de a strica sau de a darâma! a raspuns Ioab.
\par 21 Lucrul nu era a?a, ci un om din mun?ii lui Efraim, anume ?eba, fiul lui Bicri, ?i-a ridicat mâna asupra regelui David. Da?i-mi-l numai pe el singur ?i ma voi departa de cetate". "Iata, a zis femeia catre Ioab, capul lui î?i va fi aruncat peste zid!"
\par 22 ?i s-a dus femeia cu vorba ei în?eleapta la tot poporul ?i a spus la tot poporul ca sa taie capul lui ?eba, fiul lui Bicri. ?i au taiat capul lui ?eba, fiul lui Bicri, ?i l-au aruncat lui Ioab. Atunci Ioab a sunat din trâmbi?a ?i to?i oamenii s-au departat de cetate ?i s-au dus pe la casele lor. Iar Ioab s-a întors în Ierusalim la rege.
\par 23 ?i era Ioab peste toata o?tirea Israeli?ilor, iar Benaia, fiul lui Iehoiada, era peste Cheretieni ?i Peletieni.
\par 24 Adoram era peste dari; Iosafat, fiul lui Ahilud, era cronicar;
\par 25 Siva era secretar; ?adoc ?i Abiatar erau preo?i.
\par 26 De asemenea ?i Ira din Iair era sfetnic apropiat al lui David.

\chapter{21}

\par 1 În zilele lui David a fost foamete în ?ara trei ani, unul dupa altul. ?i a întrebat David pe Domnul ?i Domnul a zis: "Aceasta este din pricina lui Saul ?i a casei lui cea însetata de sânge, pentru ca el a ucis pe Ghibeoni?i".
\par 2 Atunci regele a chemat pe Ghibeoni?i ?i a vorbit eu ei. Ghibeoni?ii însa nu erau din fiii lui Israel, ci rama?i?e din Amorei, ?i Israeli?ii le facusera juramânt, dar Saul a voit sa-i piarda din râvna pentru fiii lui Israel ?i ai lui Iuda.
\par 3 "Ce sa fac pentru voi, a zis David catre Ghibeoni?i, ?i cu ce sa va împac, ca sa binecuvânta?i mo?tenirea Domnului?"
\par 4 "Noua nu ne trebuie nici aur, nici argint de la Saul sau de la casa lui, au raspuns Ghibeoni?ii, ?i nici nu trebuie sa se ucida cineva din Israel". "Atunci ce voi?i sa fac eu pentru voi?" a întrebat din nou David.
\par 5 "Fiindca acel om, au raspuns ei regelui, ne-a macelarit ?i a vrut sa ne stârpeasca astfel ca sa nu mai fim nici unul în ?inuturile lui Israel,
\par 6 De aceea, da-ne din urma?ii acelui om ?apte barba?i ?i noi îi vom spânzura, înaintea Domnului în Ghibeea lui Saul, alesul Domnului". "Va voi da!", a zis regele.
\par 7 Dar regele a cru?at pe Mefibo?et, fiul lui Ionatan, fiul lui Saul, pentru juramântul pe numele Domnului, care era între ei, între David ?i Ionatan, fiul lui Saul.
\par 8 A luat însa regele pe cei doi fii pe care Ri?pa, fiica lui Aia, îi nascuse lui Saul, Armoni ?i Mefibo?et, ?i pe cei cinci fii pe care Merob, fiica lui Saul, îi nascuse lui Adriel, fiul lui Barzilai din Mehola,
\par 9 ?i i-a dat în mâinile Ghibeoni?ilor; iar ace?tia i-au spânzurat, pe munte, înaintea Domnului; ?i au pierit to?i ?apte împreuna. Ei au fost uci?i în cele dintâi zile ale seceri?ului orzului.
\par 10 Atunci Ri?pa, fiica lui Aia, a luat sac ?i l-a întins pe stânca în chip de cort ?i a ?ezut de la începutul seceri?ului pâna a dat Dumnezeu ploaie din cer ?i n-a lasat sa se atinga de ei ziua pasarile cerului ?i noaptea fiarele câmpului.
\par 11 ?i i s-a spus lui David ce a facut Ri?pa; fiica lui Aia, concubina lui Saul.
\par 12 Atunci David s-a dus ?i a luat oasele lui Saul ?i oasele lui Ionatan, fiul lui, de la locuitorii din Iabe?ul Galaadului, pe care ace?tia le luasera pe ascuns din pia?a Bet-Sanului, unde fusesera spânzurate de Filisteni, când Filistenii ucisesera pe Saul pe muntele Ghilboa.
\par 13 ?i a stramutat el de acolo oasele lui Saul ?i oasele lui Ionatan, fiul lui, ?i a adunat ?i oasele celor ce fusesera spânzura?i;
\par 14 ?i a îngropat oasele lui Saul ?i ale lui Ionatan, fiul lui, ?i oasele celor spânzura?i în ?inutul lui Veniamin, la ?ela, în mormântul lui Chi?, tatal lui Saul. ?i s-a facut tot ce a poruncit regele. ?i dupa aceea S-a milostivit Domnul asupra ?arii.
\par 15 Dar din nou s-a pornit razboi între Filisteni ?i Israeli?i. ?i a ie?it David împreuna cu slugile lui ?i au luptat cu Filistenii. David însa a obosit.
\par 16 Atunci I?bi-Benob, unul din urma?ii lui Rafa, a carui suli?a era de trei sute de sicli de arama, ?i care era încins cu sabie noua, a vrut sa loveasca pe David;
\par 17 Dar i-a ajutat lui David Abi?ai, fiul ?eruiei, ?i a scapat Abi?ai pe David ?i a lovit pe filistean ?i l-a omorât. Atunci oamenii lui David s-au jurat ?i au zis: "Tu sa nu mai ie?i cu noi la razboi, ca sa nu se stinga faclia în Israel".
\par 18 Dupa aceea a fost din nou razboi cu Filistenii la Gob. Atunci Sibecai Hu?atitul a ucis pe Saf, unul din urma?ii lui Rafa.
\par 19 ?i a mai fost o alta batalie la Gob cu Filistenii. Atunci Elhanan, fiul lui Iaare Oreghim din Betleem, a ucis pe Goliat din Gat, a carui coada de suli?a era ca un sul de la razboiul de ?esut.
\par 20 ?i a mai fost înca o batalie în Gat. ?i era acolo un om înalt, care avea câte ?ase degete la mâna ?i la picioare, în total deci douazeci ?i patru, tot din urma?ii lui Rafa.
\par 21 Acesta hulea pe Israeli?i, dar l-a ucis Ionatan, fiul lui ?ama, fratele lui David.
\par 22 To?i ace?ti patru oameni erau din neamul lui Rafa, din Gat, ?i au cazut de mâna lui David ?i a slugilor lui.

\chapter{22}

\par 1 Atunci a cântat David cântarea aceasta Domnului, în ziua când Domnul l-a izbavit de to?i vrajma?ii lui ?i din mâinile lui Saul, ?i a zis:
\par 2 "Domnul este întarirea mea, scaparea mea ?i izbavitorul meu.
\par 3 Dumnezeu este stânca mea cea de scapare, Scutul meu ?i puterea cea mântuitoare, Adapostul meu cel tare ?i scaparea mea! Mântuitorul meu, din necaz m-ai izbavit!
\par 4 Pe Domnul Cel vrednic de lauda L-am chemat ?i de vrajma?ii mei am fost izbavit.
\par 5 Valurile mor?ii ma-mpresurasera ?i eram potopit de ?uvoaiele rauta?ii;
\par 6 Lan?urile iadului ma încatu?asera ?i eram prins în lan?urile mor?ii.
\par 7 Dar în necazul meu am chemat pe Domnul, Strigat am înal?at catre Dumnezeul meu ?i El mi-a auzit glasul din loca?ul Sau ?i strigatul meu a ajuns la urechile Lui.
\par 8 Atunci s-a clatinat ?i s-a cutremurat pamântul, ?i temeliile cerurilor s-au zguduit, Pentru ca Se mâniase Domnul!
\par 9 Fum ie?ea din narile Lui, Din gura Lui ie?ea foc mistuitor ?i carbuni aprin?i ?â?neau.
\par 10 Plecat-a cerurile ?i S-a coborât ?i sub picioarele Lui era negura deasa.
\par 11 ?ezut-a pe heruvimi ?i a zburat, zburat-a pe aripile vântului!
\par 12 Din negura ?i-a facut adapost ?i cort împrejurul Sau; Cu ape întunecoase ?i cu nori negri era înfa?urat.
\par 13 De stralucirea ce-I mergea înainte Se împra?tiau norii, Aruncând grindina ?i carbuni de foc.
\par 14 Atunci Domnul a tunat în ceruri ?i Cel Preaînalt a facut sa rasuna glasul Sau;
\par 15 Slobozit-a sage?ile Sale ?i a împra?tiat pe vrajma?i, Înmul?itu-?i-a fulgerele ?i i-a pus pe fuga.
\par 16 De certarea Ta, Doamne, De suflarea narilor Tale, S-au aratat fundurile marii ?i temeliile lumii s-au descoperit.
\par 17 Tinzându-?i El mâna de sus, m-a apucat ?i m-a scos din ape adânci;
\par 18 Izbavitu-m-a de vrajma?ul meu cel puternic, De cei ce ma urau ?i erau mai tari ca mine:
\par 19 Ace?tia ma împresurasera în ziua necazului, Dar Domnul a fost ajutorul meu;
\par 20 El m-a scos le loc larg, izbavitu-m-a, pentru ca m-a iubit!
\par 21 Domnul mi-a dat dupa nevinova?ia mea, Dupa cura?ia mâinilor mele mi-s platit.
\par 22 Caci am urmat caile Domnului, ?i înaintea Dumnezeului meu m-am pocait.
\par 23 Toate poruncile Lui le-am avut înaintea mea ?i de la legea Lui nu m-am abatut.
\par 24 Fara prihana am fost înaintea Lui ?i de nedreptate m-am pazit.
\par 25 Deci Domnul mi-a platit dupa nevinova?ia mea ?i dupa cura?ia mâinilor mele pe care a cunoscut-o.
\par 26 Caci Tu, Doamne, cu cel bun Te ara?i bun, Cu omul drept Te por?i cu dreptate,
\par 27 Cu cel cuvios, Cuvios e?ti, Iar cu cel îndaratnic, Te por?i dupa îndaratnicia lui.
\par 28 Tu pe poporul smerit îl izbave?ti, Iar ochii cei mândri îi smere?ti.
\par 29 Tu e?ti lumina mea, Doamne! Doamne, lumineaza întunericul meu!
\par 30 Cu Tine ma voi arunca spre o?tirea înarmata, Cu Dumnezeul meu voi doborî zidul.
\par 31 Caile Domnului sunt desavâr?ite, Cuvântul Domnului e lamurit prin foc. ?i scut este El tuturor celor ce nadajduiesc în El;
\par 32 Caci cine este Dumnezeu, daca nu Domnul? ?i cine este aparator, daca nu Dumnezeul nostru?
\par 33 Dumnezeu este Cel ce ma încinge cu putere ?i calea cea dreapta-mi arata;
\par 34 Care da picioarelor mele sprinteneala cerbului, ?i la locurile cele înalte ma a?eaza;
\par 35 Care-mi deprinde mâinile mele la razboi ?i bra?ele mele sa întinda arzul de arama.
\par 36 Doamne, Tu ma aperi cu scutul Tau izbavitor, Cu dreapta Ta ma sprijini ?i înal?i cu bunatatea Ta,
\par 37 Tu large?ti calea sub pa?ii mei ?i picioarele mele nu se poticnesc.
\par 38 Urmari-voi pe vrajma?ii mei ?i-i voi prinde. Nu ma voi întoarce pâna nu-i stârpesc.
\par 39 Îi voi lovi ?i nu se vor putea ?ine. Cadea-vor sub picioarele mele.
\par 40 Tu ma încingi cu putere pentru razboi ?i faci sa cada potrivnicii sub mine
\par 41 Pe vrajma?ii mei Tu-i pui pe fuga înaintea mea ?i pe cei ce ma urasc îi spulberi.
\par 42 Striga-vor, dar nimeni nu-i va scapa, Pe Domnul vor striga, dar nu-i va auzi;
\par 43 Ca praful ce-l ia vântul îi voi pisa ?i-i voi calca în picioare ca tina uli?elor.
\par 44 Tu ma izbave?ti de razvratirea poporului ?i în fruntea neamurilor ma pui.
\par 45 Poporul pe care nu-l cuno?team îmi sluje?te ?i dintr-un cuvânt mi se supune.
\par 46 Fiii celor de alt neam palesc de frica ?i tremura în zidurile ceta?ii lor.
\par 47 Viu este Domnul ?i binecuvântat este numele Lui! Laudat fie Dumnezeu, Cel ce ma izbave?te,
\par 48 Dumnezeu, Cel ce ma razbuna, Care pune popoarele sub stapânirea mea ?i de vrajma?ii mei ma izbave?te.
\par 49 Doamne, Tu ma înal?i peste vrajma?ii mei ?i de omul nedrept ma izbave?ti.
\par 50 De aceea Te voi lauda printre popoare ?i voi da slava numelui Tau,
\par 51 Cel ce da biruin?a stralucita regelui Sau ?i face mila cu unsul Sau, David, ?i cu urma?ii lui din veac în veac".

\chapter{23}

\par 1 Iata cele din urma cuvinte ale lui David: "Cuvintele lui David, fiul lui Iesei, Graiurile barbatului suspus, Ale unsului lui Dumnezeu, celui din Iacov, Ale dulcelui cântare? din Israel:
\par 2 "Duhul  Domnului graie?te prin mine, ?i cuvântul Lui este pe limba mea.
\par 3 Dumnezeul lui Israel a vorbit. Taria lui Israel mi-a zis: Cel ce domne?te între oameni cu dreptate, Cel ce stapâne?te cu temere de Dumnezeu
\par 4 E ca lumina dimine?ii când rasare soarele, E ca diminea?a fara nori, Ca razele dupa ploaie ce fac sa rasara iarba din pamânt.
\par 5 Nu este a?a oare, casa mea la Dumnezeu? Caci legamânt ve?nic a încheiat El cu mine, A?ezamânt pe veci ?i neschimbat, ?i El face sa rasara toata voia ?i nadejdea mea!
\par 6 Iar cei rai sunt ca spinii arunca?i, Care nu se pot lua cu mâna;
\par 7 Ci se iau cu fierul sau cu coada unei suli?i, ?i-n raspântii de foc se ard".
\par 8 Iata acum ?i numele vitejilor lui David: Io?eb-Ba?ebet Ta?chemonitul, unul dintre marile capetenii. Acesta ?i-a ridicat lancea sa asupra a opt sute de oameni deodata ?i i-a ucis.
\par 9 Dupa el vine Eleazar, fiul lui Dodo, fiul lui Ahohi, unul din cei trei viteji care împreuna cu David au înfruntat pe Filisteni, când se adunasera la razboi ?i când Israeli?ii se retrasesera pe înal?imi.
\par 10 Acesta s-a ridicat atunci ?i a lovit în Filisteni pâna ce i-a obosit ?i i s-a lipit mâna de sabia sa. În ziua aceea a dat Domnul biruin?a mare ?i poporul s-a dus dupa el, numai ca sa adune pe cei uci?i.
\par 11 Dupa el vine ?ama, fiul lui Aghe, din Harar. Când Filistenii s-au adunat la Lehi, unde era o ?arina semanata cu linte, ?i când poporul a fugit de Filisteni,
\par 12 Acesta a ramas în ?arina ?i a aparat-o, batând pe Filisteni. Atunci Domnul a dat biruin?a mare.
\par 13 Ace?tia trei mai însemna?i din cele treizeci de capetenii s-au dus ?i au intrat în vremea seceri?ului la David în pe?tera Adulam, când cetele Filistenilor tabarâsera în valea Refaim.
\par 14 David se afla într-un loc întarit, iar o ceata de Filisteni era în Betleem.
\par 15 ?i fiindu-i sete lui David, a zis: "Cine ma va adapa cu apa din fântâna Betleemului cea de la poarta?"
\par 16 Atunci ace?ti trei viteji au strabatut prin tabara Filistenilor, au scos apa din fântâna Betleemului cea de la poarta ?i au adus-o lui David; dar el n-a vrut s-o bea, ci a varsat-o înaintea Domnului, zicând:
\par 17 "Sa ma fereasca Dumnezeu sa fac una ca aceasta! Aceasta nu este, oare, sângele oamenilor, care ?i-au pus via?a în primejdie?" ?i n-a vrut sa bea. Iata ce-au facut ace?ti trei viteji.
\par 18 Abi?ai, fratele lui Ioab, fiul ?eruiei, era întâiul între al?i trei. El a ucis cu suli?a sa trei sute de oameni ?i era în mare cinste între ace?ti trei.
\par 19 Între ace?ti trei el era cel mai de seama ?i capetenie, dar cu cei trei de mai sus nu se asemana.
\par 20 Benaia, fiul lui Iehoiada, un om din Cab?eel, viteaz ?i vestit prin fapte mari, a ucis pe cei doi fii ai lui Ariel Moabitul ?i s-a coborât într-o groapa ?i a ucis un leu pe vreme de iarna.
\par 21 Tot el a ucis un egiptean de o statura falnica. Egipteanul avea suli?a în mâna, iar el s-a dus la acela cu un ba?, i-a smucit suli?a ?i l-a ucis cu ea.
\par 22 Iata ce a facut Benaia, fiul lui Iehoiada, ?i era mult pre?uit de cei trei.
\par 23 El era cel mai de seama între cei treizeci, dar cu cei trei de mai sus nu se asemana. ?i David l-a primit între cei mai de aproape sfetnici ai sai.
\par 24 Asael, fratele lui Ioab, este din numarul celor treizeci, Elhanan, fiul lui Dodo, din Betleem.
\par 25 ?ama Haroditeanul; Elica Haroditeanul;
\par 26 Hele? Paltianul; Ira, fiul lui Iche?, din Tecoa;
\par 27 Abiezer din Anatot; Mebunai din Hu?a.
\par 28 ?almon Ahohitul; Maharai din Netof,
\par 29 Heleb, fiul lui Baana, din Netof; Itai, fiul lui Ribai, din Ghibeea lui Veniamin;
\par 30 Benaia din Piraton; Hidai din Nahale-Gaa?;
\par 31 Abi-Baal din Araba; Azmavet din Bahurim;
\par 32 Eliahba din ?aalbon; Ionatan unul din fiii lui Ia?en;
\par 33 ?ama din Harar; Ahiam, fiul lui ?arar, din Arar;
\par 34 Elifelet, fiul lui Ahasbai, fiul unui Maacatean; Eliam, fiul lui Ahitofel, din Ghilo;
\par 35 He?rai din Carmel; Paarai din Arba;
\par 36 Igal, fiul lui Natan, din ?oba; Bani din Gad.
\par 37 ?elec Amonitul; Naharai din Beerot, arma?ul lui Ioab, fiul ?eruiei.
\par 38 Ira din Ieter; Gareb din Ieter;
\par 39 Urie Heteul. De to?i treizeci ?i ?apte.

\chapter{24}

\par 1 Mânia Domnului s-a aprins iara?i asupra Israeli?ilor, pentru ca cineva din ei îndemnase pe David, zicând: "Mergi de numara pe Israel ?i pe Iuda!"
\par 2 ?i a zis regele catre Ioab, capetenia o?tirii care era cu el: "Cutreiera toate triburile din Israel ?i ale lui Iuda, de la Dan pâna la Beer-?eba, numara poporul, ca sa ?tiu numarul oamenilor".
\par 3 "Domnul Dumnezeul tau, a raspuns Ioab, sa înmul?easca poporul înca pe atâta pe cât este, ba înca ?i de o suta de ori atâta ?i ochii domnului meu, regele, sa vada. ?i pentru ce domnul meu, regele, voie?te acest lucru?"
\par 4 Dar cuvântul regelui, dat lui Ioab ?i capeteniilor o?tirii, a biruit. ?i s-a dus Ioab cu capeteniile o?tirii de la rege sa numere poporul lui Israel.
\par 5 Trecând Iordanul, au poposit la Aroer, în partea dreapta a ceta?ii care este în mijlocul vaii Gad, aproape de Iazer.
\par 6 De acolo au mers în Galaad ?i în pamântul Tahtim-Hod?i, de unde au venit la Dan-Iaan,
\par 7 ?i ocolind Sidonul, au mers la cetatea Tir ?i prin toate ceta?ile Heveilor ?i Canaaneilor ?i au ie?it în partea de miazazi a Iudei, la Beer-?eba.
\par 8 Apoi au strabatut toata ?ara aceasta ?i dupa noua luni ?i douazeci de zile au ajuns la Ierusalim.
\par 9 ?i a dat Ioab regelui cartea cu numaratoarea poporului, din care se vedea ca Israeli?ii erau opt sute de mii de barba?i vârstnici, buni de razboi, iar cei din Iuda cinci sute de mii.
\par 10 Atunci s-a cutremurat inima lui David dupa ce a numarat poporul. ?i a zis David catre Domnul: "Greu am pacatuit eu, facând a?a, ?i acum ma rog înaintea Ta, Doamne, iarta pacatul robului Tau, caci m-am purtat peste masura de nebune?te!"
\par 11 A doua zi diminea?a s-a sculat David. Fusese însa cuvântul Domnului catre Gad proorocul, ca sa spuna lui David viitorul, zicându-i:
\par 12 "Mergi ?i spune lui David: A?a zice Domnul: Î?i arat trei pedepse: alege-?i una din ele, sa vina asupra ta".
\par 13 ?i a venit Gad la David ?i i-a vestit, zicându-i: "Alege-?i ce vrei: foamete în ?ara ta ?apte ani; sa fugi trei luni de vrajma?ii tai ?i ei sa te urmareasca; sau timp de trei zile sa fie ciuma în ?ara ta? Chibzuie?te ?i hotara?te; ce sa spun Celui ce m-a trimis",
\par 14 "E tare greu, a raspuns David lui Gad, dar sa cad mai bine în mâna Domnului, caci mila Lui este mare; numai în mâinile oamenilor sa nu cad!" ?i ?i-a ales David ciuma în vremea seceri?ului grâului.
\par 15 ?i a trimis Domnul ciuma asupra lui Israel de diminea?a pâna la vremea hotarâta. ?i a început molima în popor ?i au murit de la Dan pâna la Beer-?eba ?aptezeci de mii de oameni.
\par 16 ?i ?i-a întins îngerul Domnului mâna asupra Ierusalimului, ca sa-l pustiiasca, dar I S-a facut mila Domnului ?i a zis îngerului care ucidea poporul: "Destul! Opre?te-i acum mâna!" Îngerul Domnului se afla atunci la aria lui Aravna Iebuseul.
\par 17 ?i vazând David pe îngerul care lovea poporul, a zis catre Domnul: "Iata eu am pacatuit! Faradelegea am facut-o eu. Dar aceste oi ce-au facut? Deci îndreapta-?i mâna Ta asupra mea ?i asupra casei tatalui meu!"
\par 18 În ziua aceea Gad a venit la David ?i a zis: "Mergi de ridica jertfelnic Domnului în aria lui Aravna Iebuseul".
\par 19 ?i s-a dus David, cum îi zisese proorocul Gad ?i cum poruncise Domnul.
\par 20 ?i privind Aravna, a vazut pe rege ?i slugile lui venind la el; ?i a ie?it Aravna ?i s-a închinat regelui cu fa?a pâna la pamânt.
\par 21 "La ce a venit domnul meu, regele, la robul tau?" a întrebat Aravna. "Ca sa cumpar de la tine aria, a raspuns David, ?i sa fac acolo jertfelnic Domnului, pentru ca sa înceteze moartea în popor"
\par 22 "Sa ia domnul meu, regele, ?i sa aduca jertfa Domnului ce voie?te, a zis Aravna catre David. Iata boii pentru ardere de tot, iar carele ?i jugurile boilor vor sluji de lemne.
\par 23 Toate acestea le daruiesc regelui. Domnul Dumnezeul tau sa te binecuvânteze!" a adaugat Aravna.
\par 24 "Ba nu, a zis regele catre Aravna, eu am sa-?i platesc pre?ul ?i nu voi aduce Domnului Dumnezeului meu arderi de tot, lucruri luate în darn. ?i a cumparat David aria ?i boii cu cincizeci de sicli de argint.
\par 25 ?i a ridicat David acolo jertfelnic Domnului ?i a adus arderi de tot ?i jertfe de împacare. ?i S-a milostivit Domnul asupra ?arii ?i a încetat moartea în Israel.


\end{document}