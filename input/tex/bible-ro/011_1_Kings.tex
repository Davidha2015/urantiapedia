\begin{document}

\title{1 Kings}

1Ki 1:1  Când regele David a ajuns la adânci batrâne?e, îl acopereau cu haine, însa nu putea sa se încalzeasca.
1Ki 1:2  Atunci slugile lui i-au zis: "Sa se caute pentru domnul nostru rege o fata tânara, care sa stea înaintea regelui, sa-l îngrijeasca ?i sa se culce cu el, ca sa se încalzeasca domnul nostru regele".
1Ki 1:3  ?i i-au cautat în toate ?inuturile lui Israel o fata frumoasa ?i au gasit pe Abi?ag Sunamiteanca ?i au adus-o la rege.
1Ki 1:4  Fata era foarte frumoasa, îngrijea pe rege ?i-l servea; însa ragele n-a cunoscut-o.
1Ki 1:5  Atunci Adonia, fiul Haghitei, s-a falit ?i a zis: "Eu am sa fiu rege". ?i ?i-a gatit care de razboi, calare?i ?i cincizeci de oameni care sa-i alerge înainte.
1Ki 1:6  Caci tatal sau niciodata nu-l oprise ?i nu-i zisese: "Pentru ce faci aceasta?" El însa mai era ?i foarte frumos la chip ?i nascut dupa Abesalom.
1Ki 1:7  ?i s-a în?eles cu Ioab, fiul ?eruiei, ?i cu Abiatar preotul; ?i ace?tia ajutau pe Adonia.
1Ki 1:8  Iar preotul ?adoc ?i Benaia, fiul lui Iehoiada, proorocul Natan, ?imei, Rei ?i vitejii lui David n-au fost de partea lui Adonia.
1Ki 1:9  ?i junghiind Adonia ai, boi ?i vi?ei gra?i la piatra Zohelet, cea de la En-Roghel, a chemat pe ta?i fra?ii sai, fiii regelui, dimpreuna cu to?i barba?ii lui Iuda, care erau în slujba la rege.
1Ki 1:10  Iar pe proorocul Natan, pe Benaia, pe viteji ?i pe Solomon, fratele lui, nu i-a chemat.
1Ki 1:11  Atunci Natan a grait catre Bat?eba, mama lui Solomon, zicând: "Auzit-ai tu ca Adonia, fiul Haghitei, s-a facut rege, iar domnul nostru David nu ?tie de aceasta?
1Ki 1:12  Acum, iata, i?i dau un sfat: Sa-?i scapi via?a ta ?i a fiului tau Solomon.
1Ki 1:13  Du-te ?i intra la regele David ?l-i spune: "Nu tu, oare, domnul meu rage, te-ai jurat catre roaba ta, zicând: Solomon, fiul tau, va fi rege dupa mine: ?i va ?edea pe tronul meu? Pentru ce dar Adonia s-a facut rege?"
1Ki 1:14  ?i iata, când tu înca vei vorbi acolo cu regele, voi intra ?i eu în urma ta, voi adeveri ?i voi întregi cuvintele tale",
1Ki 1:15  Deci a intrat Bat?eba la rege în odaia unde odihnea. ?i iata regele era foarte batrân ?i Abi?ag Sunamiteanca îngrijea de el.
1Ki 1:16  ?i s-a plecat Bat?eba ?i s-a închinat regelui ?i el a întrebat-o: "Ce voie?ti?"
1Ki 1:17  Iar ea i-a spus: "Domnul meu rege, tu te-ai jurat pe Domnul Dumnezeul tau câtre roaba ta, zicând: Solomon, fiul tau, va fi rege dupa mine ?i va ?edea pe tronul meu.
1Ki 1:18  Dar acum iata ca Adonia s-a facut rege ?i tu, domnul meu rege, nu ?tii nimic da aceasta.
1Ki 1:19  Acela a înjunghiat o mul?ime de boi, vi?ei gra?i ?i oi; ?i a poftit pa to?i fiii regelui, pe preotul Abiatar ?i pe Ioab, mai-marele o?tirii; iar pe Solomon, robul tau, nu l-a chemat.
1Ki 1:20  Însa tu e?ti rege, domnul meu. Ochii tuturor Israeli?ilor la tine privesc, ca sa le ara?i cine va ?edea rege pe tronul domnului meu, dupa el.
1Ki 1:21  Altfel, când domnul meu, regele, va raposa cu parin?ii sai, voi fi învinuita eu ?i fiul meu Solomon".
1Ki 1:22  ?i iata, când înca vorbea cu regele, a venit ?i proorocul Natan.
1Ki 1:23  ?i i s-a spus regelui, zicând: "Iata Natan proorocul!" ?i a intrat Natan la rege ?i i s-a închinat regelui cu fa?a pâna la pamânt.
1Ki 1:24  ?i a zis Natan: "Domnul meu rege, ai spus tu oare: Adonia va fi rege ?i va ?edea pe tronul meu dupa mine?
1Ki 1:25  Pentru ca el chiar astazi a plecat ?i a junghiat o mul?ime de boi, vi?ei gra?i ?i oi; ?i a poftit pe to?i fiii regelui, pe mai-marii o?tirii ?i pe Abiatar preotul; ?i, iata, ei manânca ?i beau dinaintea lui ?i zic: "Traiasca regele Adonia!"
1Ki 1:26  Iar pe mine, robul tau, pe preotul  ?adoc, pe Benaia, fiul lui Iehoiada, ?i pe Solomon, robul tau, nu ne-a chemat.
1Ki 1:27  ?i daca cu voia ta, domnul meu rege, s-a facut lucrul acesta, atunci pentru ce tu n-ai descoperit robului tau cine va ?edea rege pe tronul domnului meu, dupa el?"
1Ki 1:28  Dar regele David a raspuns ?i a zis: "Chema?i pe Bat?eba la mine!" ?i a intrat ea la rege ?i a stat înaintea lui.
1Ki 1:29  Atunci regele s-a jurat ?i a zis: "Viu este Domnul, Care mi-a scapat sufletul meu de la orice necaz!
1Ki 1:30  Precum m-am jurat pe Domnul Dumnezeul lui Israel catre tine, zicând: "Solomon, fiul tau, va fi rege dupa mine, ?i va ?edea pe tronul meu în locul meu, a?a voi face chiar astazi!"
1Ki 1:31  ?i s-a plecat Bat?eba cu fa?a pâna la pamânt ?i i s-a închinat regelui ?i i-a zis: "Sa traiasca domnul meu, regele David, În veci!"
1Ki 1:32  Apoi a zis regele David: "Chema?i la mine pe preotul ?adoc, pe proorocul Natan ?i pe Benaia, fiul lui Iehoiada!" ?i ei au intrat la rege.
1Ki 1:33  ?i regele le-a spus: "Sa lua?i pe slugile domnului vostru cu voi ?i sa pune?i pe Solomon, fiul meu, calare pe catârul meu ?i sa-l duce?i pâna la Ghihon,
1Ki 1:34  ?i acolo sa-l unga preotul ?adoc ?i proorocul Natan de rege peste Israel ?i sa suna?i din trâmbi?a ?i sa zice?i: "Sa traiasca regele Solomon!"
1Ki 1:35  Apoi sa-l petrece?i înapoi, ca sa vina ?i sa ?ada pe tronul meu; caci el va fi rege în locul meu ?i lui i-am poruncit sa fie conducatorul lui Israel ?i al lui Iuda".
1Ki 1:36  ?i a raspuns Benaia, fiul lui Iehoiada regelui ?i i-a zis: "Amin! A?a sa întareasca Domnul Dumnezeul domnului meu, regele, cuvântul acesta!
1Ki 1:37  ?i cum Domnul Dumnezeu a fost cu domnul meu, regele, a?a sa fie el ?i cu Solomon ?i sa-i preamareasca tronul lui mai mult decât tronul domnului meu, regele David!"
1Ki 1:38  ?i a?a au plecat preotul ?adoc, proorocul Natan, Benaia, fiul lui Iehoiada, Cheretienii ?i Peletienii ?i au pus pe Solomon calare pe catârul regelui David ?i l-au dus pâna la Ghihon.
1Ki 1:39  ?i a luat preotul ?adoc cornul cu untdelemn sfin?it din cort ?i a uns pe Solomon. ?i s-a sunat din trâmbi?e ?i tot poporul a strigat: "Traiasca regele Solomon! "
1Ki 1:40  Apoi tot poporul l-a petrecut pe Solomon ?i i-a cântat din fluiere ?i bucurie mare era pe popor, încât ?i pamântul se zguduia de strigatele lui.
1Ki 1:41  Atunci a auzit Adonia de aceasta ?i to?i cei chema?i ai lui, tocmai când ispravisera de mâncat; iar Ioab, auzind sunetul de trâmbi?e, a zis: "Ce este acest zgomot de care rasuna cetatea?"
1Ki 1:42  ?i pe când el înca vorbea, iata ca a venit Ionatan, fiul preotului Abiatar. ?i a zis Adonia: "Intra, caci tu e?ti om cinstit ?i aduci veste buna!"
1Ki 1:43  ?i a raspuns Ionatan lui Adonia ?i i-a zis: "Ba chiar veste rea; caci domnul nostru, regele David, a facut rege pe Solomon.
1Ki 1:44  ?i a trimis regele cu el pe ?adoc ?i pe proorocul Natan, pe Benaia, fiul lui Iehoiada, ?i pe Cheretieni ?i Peletieni ?i l-au pus calare pe catârul regelui.
1Ki 1:45  ?i l-au uns preotul ?adoc ?i proorocul Natan rege la Ghihon ?i de acolo s-au întors cu bucurie ?i au pus în mi?care cetatea. Iata zgomotul pe care-l auzi?i".
1Ki 1:46  ?i a ?ezut Solomon pe tronul regatului.
1Ki 1:47  ?i slugile regelui au venit sa-l binecuvânteze pe domnul nostru regele David, zicând: "Sa mareasca Dumnezeul tau numele lui Solomon mai mult decât numele tau ?i sa-i aduca tronul lui la marire mai multa decât tronul tau!" ?i s-a închinat regele în patul lui;
1Ki 1:48  ?i a zis regele a?a: "Binecuvântat sa fie Domnul Dumnezeul lui Israel, Care a facut sa fie astazi din samân?a mea un urma? pe tronul meu ?i sa vad cu ochii mei aceasta!"
1Ki 1:49  Atunci to?i cei chema?i, care erau cu Adonia, s-au spaimântat ?i, sculându-se, s-au dus fiecare în drumul sau.
1Ki 1:50  Iar Adonia, temându-se de Solomon, s-a sculat ?i s-a dus ?i s-a apucat cu mâinile de coarnele jertfelnicului.
1Ki 1:51  ?i i s-a spus lui Solomon, zicând: "Iata, Adonia s-a temut de regele Solomon; caci iata, ca se ?ine cu mâinile de coarnele jertfelnicului, zicând: Sa-mi fagaduiasca astazi cu juramânt regele Solomon ca nu va omorî pe robul sau cu sabia".
1Ki 1:52  ?i a zis Solomon: "Daca el va fi om cinstit, nici un par din capul lui nu va cadea pe pamânt; iar daca va fi om viclean, va muri".
1Ki 1:53  ?i a trimis regele Solomon ?i l-a adus de la jertfelnic cu de-a sila; ?i a venit ?i s-a închinat regelui Solomon. ?i Solomon i-a zis: "Du-te la casa ta!"
1Ki 2:1  Apropiindu-se vremea lui David ca sa moara, a lasat el fiului sau Solomon acest legamânt:
1Ki 2:2  "Iata, eu ma duc pe drumul pe care to?i pamântenii se duc; fii tare ?i sa fii barbat.
1Ki 2:3  Sa paze?ti legamântul Domnului Dumnezeului tau, umblând în caile Lui ?i pazind legile Lui, poruncile Lui, hotarârile Lui ?i a?ezamintele Lui, precum sunt scrise în legea lui Moise, pentru ca sa-?i fie bine-cunoscut tot ce vei face, oriunde ?i ori încotro te vei întoarce;
1Ki 2:4  ?i ca sa Î?i ?ina ?i Domnul cuvântul Sau care l-a grait catre mine, zicând: "Daca fiii tai î?i vor pazi drumul lor, ca sa se poarte cu credincio?ie înaintea Mea, din toata inima ?i din tot sufletul lor, atunci nu va conteni sa fie din tine barbat pe tronul lui Israel.
1Ki 2:5  Dar ?i tu ?tii ce mi-a facut Ioab, fiul ?eruiei, cum s-a purtat cu cele doua capetenii ale o?tirii lui Israel, cu Abner, fiul lui Ner, ?i cu Amasa, fiul lui Ieter, pe care i-a omorât; ?i cum a varsat în timp de pace sânge ca în timp de razboi, mânjind cu sângele cel varsat ca la razboi încingatoarea de la coapsele sale ?i încal?amintea din picioarele sale.
1Ki 2:6  Sa faci dar cu el dupa în?elepciunea ta, ca sa nu se coboare carunte?ea lui cu pace în locuin?a mor?ilor.
1Ki 2:7  Iar fiilor lui Barzilai Galaaditul arata-le mila, ca sa fie cu cei care se hranesc la masa ta, caci ei au venit la mine când am fugit de Abesalom, fratele tau.
1Ki 2:8  Iata ca ai la tine ?i pe ?imei, fiul lui Ghera veniamineanul din Bahurim, care m-a blestemat cu greu blestem, când mergeam la Mahanaim; dar fiindca mi-a ie?it în cale la Iordan, m-am jurat pe Domnul catre el, zicând: Nu te voi mai ucide cu sabia.
1Ki 2:9  Acum însa sa nu-l la?i nepedepsit, caci e?ti barbat în?elept, ?i ?tii ce sa faci cu el, ca sa cobori carunte?ea lui cu sânge în locuin?a mor?ilor".
1Ki 2:10  ?i a raposat David cu parin?ii sai ?i a fost înmormântat în cetatea lui David.
1Ki 2:11  Timpul domniei lui David peste Israel a fost de patruzeci de ani: în Hebron ?apte ani ?i în Ierusalim treizeci ?i trei de ani.
1Ki 2:12  ?i s-a a?ezat Solomon pe tronul lui David, tatal sau, ?i domnia lui s fost foarte stralucita.
1Ki 2:13  Atunci a venit Adonia, fiul Haghitei, la Bat?eba, mama lui Solomon, ?i i s-a închinat. ?i ea i-a zis: "Cu pace î?i este venirea?" Iar el a raspuns: "Cu pace!"
1Ki 2:14  Apoi a zis: "Am sa-?i spun o vorba!" ?i ea a zis: "Spune;!"
1Ki 2:15  ?i a zis el: "Tu ?tii ca domnia era a mea ?i ca tot Israelul privea la mine ca la viitorul lor rege; dar domnia a trecut de la mine ?i i s-a dat fratelui meu, pentru ca de la Domnul i-a fost lui aceasta.
1Ki 2:16  Acum te rog un lucru: sa nu ma nesocote?ti!" ?i ea i-a zis: "Graie?te!"
1Ki 2:17  Iar el a zis: "Te rog vorbe?te cu regele Solomon, caci el va ?ine seama de vorba ta, ca sa-mi dea pe Abi?ag Sunamiteanca de femeie".
1Ki 2:18  ?i a zis Bat?eba: "Bine, voi vorbi cu regele pentru tine".
1Ki 2:19  Deci, a intrat Bat?eba la regele Solomon, ca sa-i vorbeasca pentru Adonia. Regele s-a sculat înaintea ei, i s-a închinat ?i s-a a?ezat pe tronul lui. ?i s-a pus ?i pentru mama regelui un tron; ?i ea a stat de-a dreapta lui
1Ki 2:20  ?i a zis: "Am sa-?i fac o mica rugaminte, sa nu mi-o treci cu vederea!" A zis regele: "Cere, mama mea!"
1Ki 2:21  Ea a zis: "Sa dai pe Abi?ag Sunamiteanca lui Adonia, fratele tau, de femeie!"
1Ki 2:22  Atunci a raspuns regele Solomon ?i a zis mamei sale: "Dar de ce ceri tu pe Abi?ag Sunamiteanca pentru Adonia? Cere atunci pentru el ?i domnia, caci el este fratele meu cel mai mare ?i cu el este preotul Abiatar ?i tot cu el este prieten ?i Ioab, fiul ?eruiei, mai-marele o?tirii!"
1Ki 2:23  Apoi s-a jurat regele Solomon pe Domnul, zicând: "A?a rau sa-mi dea mie Dumnezeu ?i înca ?i altele sa ma ajunga, daca n-a grait Adonia cuvântul acesta împotriva vie?ii mele.
1Ki 2:24  Dar acum, viu este Domnul, Care m-a socotit pe mine vrednic de cinste ?i m-a a?ezat pe tronul lui David, tatal meu, ?i mi-a zidit casa precum a grait el; Adonia chiar astazi va muri".
1Ki 2:25  ?i a trimis regele Solomon pe Benaia, fiul lui Iehoiada, care fara de mila l-a lovit pe acela ?i a murit Adonia în ziua aceea.
1Ki 2:26  Iar lui Abiatar preotul, regele i-a zis: "Sa pleci la Anatot, la mo?ia ta; caci ?i tu e?ti vrednic de moarte; dar acum nu te voi omorî, caci ai purtat chivotul Domnului Dumnezeu înaintea lui David, tatal meu ?i ai suferit ?i tu ce a suferit tatal meu".
1Ki 2:27  ?i l-a îndepartat Solomon pe Abiatar de la preo?ia Domnului, ca sa se împlineasca cuvântul Domnului care l-a grait pentru casa lui Eli în ?ilo.
1Ki 2:28  Vestea aceasta a ajuns pâna la Ioab, fiul ?eruiei, fiindca ?i Ioab se daduse de partea lui Adonia ?i nu de partea lui Solomon ?i a fugit Ioab la cortul Domnului ?i s-a apucat cu mâinile de coarnele jertfelnicului.
1Ki 2:29  ?i i s-a spus regelui Solomon, zicând: "Iata Ioab a fugit la cortul Domnului ?i iata s-a apucat cu mâinile de coarnele jertfelnicului. ?i a trimis Solomon la Ioab sa-i zica: "Ce ?i-am facut de ai fugit la jertfelnic?" ?i a raspuns Ioab: "M-am temut de fa?a ta ?i am fugit la Domnul". ?i a trimis Solomon pe Benaia, fiul lui Iehoiada, zicând: "Mergi, omoara-l ?i-l îngroapa!"
1Ki 2:30  ?i s-a dus Benaia, fiul lui Iehoiada, la Ioab, la cortul Domnului, ?i i-a zis: "A?a zice regele: Sa ie?i!" ?i a zis Ioab: "Nu ies, pentru ca vreau sa mor aici! " ?i s-a întors Benaia, fiul lui Iehoiada ?i a spus de aceasta regelui, zicând: "A?a a zis Ioab ?i a?a mi-a raspuns!"
1Ki 2:31  Iar regele i-a zis: "Du-te ?i fa-i a?a cum a zis; omoara-l ?i-l îngroapa. ?i ia de pe mine ?i de pe casa tatalui meu sângele nevinovat varsat de Ioab.
1Ki 2:32  Sa-i întoarca Domnul pe capul lui sângele nedrepta?ii lui, pentru ca a ucis pe doi barba?i nevinova?i ?i mai buni decât el, omorându-i cu sabia, fara de ?tirea tatalui meu, David: pe Abner, fiul lui Ner, mai-marele o?tirii lui Israel, ?i pe Amasa, fiul lui Ieter, mai marele o?tirii lui Iuda.
1Ki 2:33  Pe capul lui ?i pe capul semin?iei lui în veci sa se întoarca sângele lor; iar David ?i semin?ia lui ?i casa lui ?i tronul lui sa aiba în veci pace de la Domnul!
1Ki 2:34  ?i s-a dus Benaia, fiul lui Iehoiada ?i a lovit pe Ioab ?i l-a omorât; ?i a fost înmormântat la casa lui, în pustiu.
1Ki 2:35  Iar regele Solomon a pus în locul lui peste o?tire pe Benaia, fiul lui Iehoiada. Cârmuirea regatului era însa la Ierusalim; iar mai mare peste preo?i, în locul lui Abiatar, regele a pus pe ?adoc preotul. Solomon, fiul lui David, a domnit peste Israel ?i Iuda, în Ierusalim. ?i a dat Domnul lui Solomon în?elepciune ?i pricepere foarte mare ?i cuno?tin?e multe, ca nisipul de pe ?armul marii. ?i Solomon a avut în?elepciune mai presus de în?elepciunea tuturor fiilor Rasaritului ?i de în?elepciunea tuturor în?elep?ilor Egiptenilor. El a luat pentru sine pe fiica lui Faraon ?i a adus-o în cetatea lui David, pâna ce a terminat de zidit casa sa ?i, mai întâi, templul Domnului ?i zidul dimprejurul Ierusalimului; în ?apte ani a facut acestea ?i le-a ispravit. Solomon a avut ?aptezeci de mii de oameni salahori ?i optzeci de mii de taietori de piatra în munte. El a facut marea ?i postamentele ?i spalatoriile cele mari, stâlpii, fântâna cea din curte ?i marea de arama; ?i a zidit el ceta?uia ?i întariturile ei ?i a împar?it cetatea lui David. Atunci fiica lui Faraon o trecut din cetatea lui David în casa sa pe care i-o zidise el; tot atunci a zidit Solomon zidul dimprejurul ceta?ii. ?i aducea Solomon de trei ori pe an arderi de tot ?i jertfe de pace pe jertfelnicul pe care-l facuse Domnului; ?i savâr?ea la el ?i tamâieri înaintea Domnului. ?i a ispravit zidirea templului. La lucrarile lui Solomon erau trei mii ?apte sute de ispravnici mari care conduceau poporul ce facea lucrul. ?i a zidit el Asurul, Magdinul, Gazerul, Bet-Horonul de Sus ?i Valatul. Dar aceste ceta?i le-a zidit el dupa ce a facut templul Domnului ?i zidul dimprejurul Ierusalimului. Înca din timpul vie?ii sale, David poruncise lui Solomon, zicând: "Iata, ai pe ?imei, fiul lui Ghera, fiul lui Ieminie, din Bahurim; el m-a blestemat cu greu blestem, în ziua când m-am dus la Mahanaim; dar el mi-a ie?it în cale la Iordan ?i eu m-am jurat pe Domnul fa?a de el, zicând: Nu te voi mai ucide eu sabia! Tu însa sa nu-l la?i nepedepsit; caci e?ti barbat în?elept ?i ?tii ce sa faci cu el, ca sa-i cobori carunte?ea lui cu sânge în casa mor?ilor".
1Ki 2:36  ?i trimi?ând regele, a chemat pe ?imei ?i i-a zis: "Fa-?i casa în Ierusalim ?i traie?te aici ?i de aici sa nu mai ie?i.
1Ki 2:37  Dar sa ?tii ca în ziua în care vai ie?i ?i vei trece pârâul Chedron, numaidecât cu moarte vei muri. Sângele tau va fi asupra capului tau!"
1Ki 2:38  ?i a zis ?imei regelui: "Bine, cum a poruncit domnul meu regele, a?a va face robul tau". ?i a trait ?imei în Ierusalim multa vreme.
1Ki 2:39  Iar peste trei ani s-a întâmplat ca au fugit doi robi de la ?imei la Achi?, fiul lui Maaca, regele Gatului. ?i i s-a spus lui ?imei: "Iata, robii tai sunt la Gat".
1Ki 2:40  ?i sculându-se ?imei ?i punând ?aua pe asinul sau, a plecat la Gat, la Achi?, ca sa-?i caute robii sai. ?i s-a întors ?imei ?i a adus robii sai de la Gat.
1Ki 2:41  Dar, spunându-i-se lui Solomon ca ?imei a mers din Ierusalim pâna la Gat ?i ca s-a întors, a trimis regele ?i a chemat pe ?imei ?i i-a zis:
1Ki 2:42  "Nu m-am jurat eu, oare, pe Domnul catre tine ?i nu ?i-am spus eu, oare, înainte, zicând: Sa ?tii ca în ziua în care vei ie?i din Ierusalim ?i te vei duce undeva, numaidecât vai muri? ?i tu mi-ai spus: Bine!
1Ki 2:43  De ce n-ai pazit legea pe care ?i-am dat-o înaintea Domnului cu juramânt?
1Ki 2:44  Apoi a mai zis regele catre ?imei: "Tu ?tii ?i ?tie ?i inima ta tot râul care l-ai facut tatalui meu David; sa se întoarca asupra capului tau rautatea ta!
1Ki 2:45  Iar regele Solomon sa fie binecuvântat ?i tronul lui David sa fie neclintit înaintea Domnului în veci!"
1Ki 2:46  ?i a poruncit regele lui Benaia, fiul lui Iehoiada; ?i el s-a dus ?i a lovit pe ?imei ?i acela a murit.
1Ki 3:1  Dupa ce regatul s-a întarit în mâinile lui Solomon, Solomon s-a înrudit cu Faraon, regele Egiptului, caci a luat pentru el pe fiica lui Faraon ?i a adus-o în cetatea lui David, pâna o terminat de zidit casa sa, templul Domnului ?i zidul dimprejurul Ierusalimului.
1Ki 3:2  Poporul tot mai aducea jertfe pe înal?imi, caci nu era înca zidit templul numelui Domnului pâna în acel timp.
1Ki 3:3  ?i Solomon, care iubea pe Domnul, purtându-se dupa legea tatalui sau, a adus ?i el jertfe ?i tamâieri pe înal?imi.
1Ki 3:4  Deci s-a sculat ?i el ?i s-a dus la Ghibeon, ca sa aduca jertfe acolo, caci acolo era cea mai mare înal?ime. O mie de jertfe pentru ardere de tot a adus Solomon pe acel jertfelnic.
1Ki 3:5  La Ghibeon însa S-a aratat Domnul lui Solomon noaptea în vis ?i a zis: "Cere ce vrei sa-?i dau!"
1Ki 3:6  ?i a zis Solomon: "Tu ai facut cu robul Tau David, tatal meu, mare mila; ?i, pentru ca el s-a purtat cu vrednicie ?i cu dreptate ?i inima curata înaintea Ta, nu ?i-ai luat aceasta mila mare de la el; ?i i-ai daruit fiu pe tronul lui, precum ?i este aceasta astazi.
1Ki 3:7  ?i acum Tu, Doamne Dumnezeul meu, ai pus pe robul Tau rege în locul lui David, tatal meu; însa eu sunt foarte tânar ?i nu ?tiu sa conduc.
1Ki 3:8  ?i robul Tau este în mijlocul poporului Tau pe care l-ai ales, popor nesfâr?it de mare, care din pricina mul?imii lui nu se poate nici socoti, nici numara.
1Ki 3:9  Daruie?te-i dar robului Tau minte priceputa, ca sa asculte ?i sa judece poporul Tau ?i sa deosebeasca ce este bine ?i ca este rau; caci cine poate sa pova?uiasca pe acest popor al Tau, care este nesfâr?it de mare?"
1Ki 3:10  ?i i-a placut Domnului ca Solomon a cerut aceasta.
1Ki 3:11  ?i a zis Dumnezeu: "Deoarece tu ai cerut aceasta ?i n-ai cerut via?a lunga; n-ai cerut boga?ie, n-ai cerut sufletele du?manilor tai, ci ai cerut în?elepciune, ca sa ?tii sa judeci,
1Ki 3:12  Iata Eu voi face dupa cuvântul tau; iata, Eu î?i dau minte în?eleapta ?i priceputa, cum nici unul n-a fost ca tine înaintea ta ?i cum nici nu se va mai ridica dupa tine.
1Ki 3:13  Ba î?i voi da ?i ceea ce tu n-ai cerut: boga?ie ?i slava, a?a încât nici unul dintre regi nu va fi asemenea ?ie, în toate zilele tale.
1Ki 3:14  ?i daca vei umbla pe drumul Meu, ca sa paze?ti legile Mele ?i poruncile Mele, cum a umblat tatal tau David, î?i voi înmul?i ?i zilele tale".
1Ki 3:15  ?i s-a trezit Solomon din somn ?i iata, acesta fusese vis. Apoi s-a sculat ?i a venit la Ierusalim ?i a stat înaintea jertfelnicului celui de dinaintea chivotului cu legea Domnului, care era în Sion, a adus arderi de tot, a savâr?it jertfe de împacare ?i a facut mare ospa? pentru toate slugile sale.
1Ki 3:16  Atunci au venit doua femei desfrânate la rege ?i au stat înaintea lui.
1Ki 3:17  ?i a zis una din femei: "Rogu-ma, domnul meu, noi traim într-o casa; ?i eu am nascut la ea, în casa aceea.
1Ki 3:18  A treia zi dupa ce am nascut eu, a nascut ?i aceasta femeie ?i eram împreuna ?i nu era nimeni strain cu noi în casa, afara de noi amândoua.
1Ki 3:19  Însa noaptea a mu?it fiul acestei femei, caci a adormit peste el.
1Ki 3:20  ?i s-a sculat ea pe la miezul nop?ii ?i mi-a luat pe fiul meu de lânga mine, când eu, roaba ta, dormeam ?i l-a pus la pieptul ci; iar pe fiul ei cel mort l-a pus la pieptul meu.
1Ki 3:21  Diminea?a când m-am sculat ca sa-mi alaptez fiul, iata, el era mort; iar când m-am uitat la el mai bine diminea?a, acesta nu era fiut meu, pe care-l nascusem".
1Ki 3:22  Iar cealalta femeie a zis: "Ba nu, fiul meu e viu, iar fiul tau e mort!" Iar aceasta îi zicea: "Ba nu, fiul tau este mort ?i al meu e viu!" ?i vorbeau ele a?a înaintea regelui.
1Ki 3:23  Atunci regele a zis: "Aceasta zice: Fiul meu este cel viu, iar fiul tau este cel mort; iar aceea zice; Ba nu, fiul tau este cel mort ?i fiul meu aste cel viu".
1Ki 3:24  Apoi a zis Solomon: "Da?i-mi o sabie"; ?i i s-a adus regelui o sabie.
1Ki 3:25  ?i a zis regele: "Taia?i copilul cel viu în doua ?i da?i o jumatate din el uneia ?i o jumatate din el celeilalte!"
1Ki 3:26  ?i a raspuns femeia al carui fiu era viu regelui, - caci i se rupea inima de mila pentru fiul ei: "Rogu-ma, domnul meu, da?i-i ei acest prunc viu ?i nu-l omorâ?i!" Iar cealalta a zis: "Ca sa nu fie nici al meu, nici al ei, taia?i-l!"
1Ki 3:27  ?i regele a zis: "Da?i-i acesteia copilul cel viu, ca aceasta este mama lui!"
1Ki 3:28  ?i a auzit tot Israelul do judecata aceasta pe care a facut-o regele. ?i au început sa se teama de rege, caci vedeau ca în?elepciunea lui Dumnezeu este în el, ca sa faca judecata ?i dreptate.
1Ki 4:1  ?i a fost regele Solomon rege peste tot Israelul.
1Ki 4:2  Iata acum capeteniile pe care le avea el la curtea sa: Azaria, fiul lui ?adoc, preotul;
1Ki 4:3  Elihoref ?i Ahia, fiii lui ?i?a, scriitori; Ioasaf, fiul lui Ahilud, cronicar;
1Ki 4:4  Benaia, fiul lui Iehoiada, capetenie peste o?tire; ?adoc ?i Abiatar, preo?i;
1Ki 4:5  Azaria, fiul lui Natan, capetenie peste ispravnici, iar Zabud, fiul lui Natan, preot ?i prieten al regelui;
1Ki 4:6  Ahi?ar era capetenie peste casa regelui; Eliav, fiul lui Saf, era peste mo?ii ?i Adoniram, fiul lui Abda, era peste dari.
1Ki 4:7  ?i mai avea Solomon doisprezece ispravnici peste tot Israelul, care aduceau alimente pentru rege ?i casa lui; fiecare trebuia sa aduca alimente pe o luna în an.
1Ki 4:8  Iata numele lor: Ben-Hur, peste muntele lui Efraim, singur;
1Ki 4:9  Ben-Decher, peste Maca?, peste ?aalebim, peste Bet-?eme?, peste Elon ?i peste Bet-Hanan;
1Ki 4:10  Ben-Hased, peste Arubot; ?i tot sub el mai era ?i Soco, cum ?i tot pamântul Hefer;
1Ki 4:11  Ben-Abinadab, peste tot Nafat-Dor; Tafat, fiica lui Solomon, era femeia lui;
1Ki 4:12  Baana, fiul lui Ahilud, peste Tanac, peste Meghidon ?i peste tot pamântul Bet-?eanului, care este aproape de ?artan, mai jos de Izreel, de la Bet-?ean pâna la Abel-Mehol ?i chiar pâna dincolo de Iocmeam;
1Ki 4:13  Ben-Gheber, peste Ramot-Galaad; sub el mai erau ?i satele lui Iair, fiul lui Manase, care sunt în Galaad; tot sub el mai era ?i ?inutul Argob, care este în Vasan, ?aizeci de ceta?i mari cu ziduri ?i zavoare de arama;
1Ki 4:14  Ahinadab, fiul lui Ido, peste Mahanaim;
1Ki 4:15  Ahimaa?, care a avut de femeie pe Basemat, fiica lui Solomon, era peste pamântul Neftali;
1Ki 4:16  Baana, fiul lui Hu?ai, peste A?er ?i Bealot;
1Ki 4:17  Iosafat, fiul lui Paruah, peste Isahar;
1Ki 4:18  ?imei, fiul lui Ela, peste Veniamin;
1Ki 4:19  Gheber, fiul lui Urie, peste Galaad, peste ?ara lui Sihon, regele Amoreilor, ?i a lui Og, regele Vasanului. El era singur ispravnic peste aceste pamânturi.
1Ki 4:20  Iuda ?i Israel, care erau nesfâr?it de mul?i la numar, ca nisipul de pe ?armul marii, mâncau, beau ?i se veseleau.
1Ki 4:21  Solomon domnea peste toate regatele de la râul Eufrat pâna la pamântul Filistenilor ?i pâna în hotarul Egiptului. Acestea îi aduceau daruri ?i au slujit lui Solomon în toate zilele vie?ii lui.
1Ki 4:22  Hrana lui Solomon pe fiecare zi era: treizeci de core faina de grâu ?i ?aizeci core de alte preparate de faina;
1Ki 4:23  Zece boi îngra?a?i, douazeci boi din cei care pa?teau iarba ?i o suta de oi, afara de vânatul de cerbi, caprioare, ciute ?i de pasarile îngra?ate;
1Ki 4:24  Caci domnea peste tot pamântul de dincoace de Eufrat, de la Tifsah pâna la Gaza, ?i peste to?i regii de dincoace de Eufrat, ?i era în pace cu toate ?arile de primprejur.
1Ki 4:25  Astfel a trait Iuda ?i Israelul în lini?te, fiecare sub vi?a sa de vie ?i sub smochinul sau, de la Dan pâna la Beer-?eba, în toate zilele lui Solomon.
1Ki 4:26  Solomon avea patruzeci de mii de iesle pentru caii de la carele lui ?i douasprezece mii de calare?i.
1Ki 4:27  ?i acei ispravnici aduceau regelui Solomon tot ce trebuia pentru masa regelui, fiecare în luna lui, ?i nu lasa sa duca lipsa de nimic.
1Ki 4:28  ?i orz ?i paie pentru cai ?i pentru celelalte vite aducea fiecare, când îi era rândul lui, la locul unde se afla regele.
1Ki 4:29  ?i a dat Dumnezeu lui Solomon în?elepciune ?i pricepere foarte mare ?i cuno?tin?e multe, ca nisipul de pe ?armul marii.
1Ki 4:30  ?i era în?elepciunea lui Solomon mai presus de în?elepciunea tuturor fiilor Rasaritului ?i mai presus de toata în?elepciunea Egiptenilor.
1Ki 4:31  El era mai în?elept decât to?i oamenii; mai în?elept mult ?i decât Etan Ezrahiteanul, decât Heman ?i decât Calcol ?i Darda, feciorii lui Mahol; ?i numele lui era în slava la toate popoarele de primprejur.
1Ki 4:32  Solomon a spus trei mii de pilde; ?i cântarile lui au fost o mie ?i cinci.
1Ki 4:33  El a vorbit despre copaci, de la cedrii cei din Liban pâna la isopul de pe ziduri; a vorbit ?i despre animale, despre pasari, despre târâtoare ?i despre pe?ti.
1Ki 4:34  ?i veneau de la toate popoarele, ca sa asculte în?elepciunea lui Solomon, ?i de la to?i regii pamântului care auzeau de în?elepciunea lui.
1Ki 5:1  Atunci a trimis Hiram, regele Tirului, pe slugile sale la Solomon, când a auzit ca l-au uns rege în locul tatalui sau. Caci Hiram fusese prieten cu David toata via?a.
1Ki 5:2  ?i a trimis ?i Solomon la Hiram ca sa-i spuna:
1Ki 5:3  "Tu ?tii ca David, tatal meu, n-a putut sa înal?e casa numelui Domnului Dumnezeului sau, din pricina razboaielor cu popoarele dimprejur, pâna ce Domnul nu le-a supus sub talpa picioarelor lui.
1Ki 5:4  Acum însa Domnul Dumnezeul meu mi-a daruit odihna din toate par?ile; n-am nici potrivnic, nici alte primejdii.
1Ki 5:5  ?i iata eu ma gândesc sa zidesc templu numelui Domnului Dumnezeului meu, dupa cum a grait Domnul catre tatal meu David, zicând: Fiul tau pe care Eu îl voi pune în locul tau pe tron, acela va zidi templu numelui Meu.
1Ki 5:6  A?adar porunce?te sa taie pentru mine cedri din Liban; ?i iata robii mei vor fi împreuna cu robii tai; ?i eu î?i voi da plata pentru robii tai cât vei hotarî tu, ca tu cuno?ti ca la noi nu sunt oameni care sa ?tie a taia lemnele a?a ca Sidonienii".
1Ki 5:7  Când a auzit Hiram cuvintele lui Solomon, s-a bucurat foarte ?i a zis: "Binecuvântat fie astazi Domnul, Care a dat lui David fecior în?elept pentru pova?uirea acestui popor nesfâr?it de mare!
1Ki 5:8  ?i a trimis Hiram la Solomon sa-i spuna: "Am auzit pentru ce ai trimis la mine ?i î?i îndeplinesc toata dorin?a ta pentru lemnul de cedru ?i lemnul de chiparos.
1Ki 5:9  Robii mei le vor scoate din Liban la mare ?i eu cu plutele le voi duce pe mare la locul care ni-l vei hotarî; ?i acolo le voi descarca ?i tu le vei lua; însa ?i tu sa pline?ti dorin?a mea: sa aduci pâine pentru casa mea!"
1Ki 5:10  A dat deci Hiram lui Solomon lemn de cedru ?i lemn de chiparos, toate tocmai dupa dorin?a lui.
1Ki 5:11  Iar Solomon a dat lui Hiram douazeci de mii de core de grâu pentru hrana casei iui ?i douazeci de core de untdelemn de masline curat. Atât îi da Solomon lui Hiram pe fiecare an.
1Ki 5:12  Domnul i-a dat în?elepciune lui Solomon dupa cum i-a fagaduit. ?i a fost pace între Hiram ?i Solomon ?i amândoi între ei au facut legamânt.
1Ki 5:13  ?i a pus regele Solomon o corvoada peste tot Israelul ?i corvoada era de treizeci de mii de oameni.
1Ki 5:14  ?i-i trimetea la Liban, câte zece mii pe luna, cu schimbul: o luna erau la Liban, iar doua luni la casa lor. Iar Adoniram era capetenie mai mare peste ei.
1Ki 5:15  ?i mai avea Solomon înca ?aptezeci de mii de salahori ?i optzeci de mii de oameni taietori de piatra în munte,
1Ki 5:16  Afara de cele trei mii ?i trei sute de capetenii, care erau puse de Solomon, sa supravegheze poporul care facea lucrul.
1Ki 5:17  ?i a poruncit regele sa pregateasca pietre mari, pietre cu ciubuce pentru temelia templului ?i pietre cioplite
1Ki 5:18  ?i le-au lucrat lucratorii lui Solomon, lucratorii lui Hiram ?i lucratorii din Biblos. ?i a?a s-a pregatit lemnul ?i piatra pentru ridicarea templului, timp de trei ani.
1Ki 6:1  Iar în anul patru sute optzeci, dupa ie?irea fiilor lui Israel din Egipt, în al patrulea an al domniei lui Solomon peste Israel, în luna Zif, care este a doua luna a anului, a început el sa zideasca templul Domnului.
1Ki 6:2  Templul, pe care l-a zidit regele Solomon Domnului era lung de ?aizeci de co?i, lat de douazeci ?i înalt de treizeci.
1Ki 6:3  Pridvorul de dinaintea templului era lung de douazeci de co?i, raspunzând cu la?imea templului, ?i lat de zece co?i înaintea templului.
1Ki 6:4  ?i a facut el la acele odai ferestre cu zabrele, largi înauntru ?i strâmte în afara.
1Ki 6:5  ?i a mai facut o cladire lânga zidul templului, cu trei caturi în jurul pere?ilor templului, în jurul Sfintei Sfintelor.
1Ki 6:6  Catul de jos al cladirii era lat de cinci co?i; cel din mijloc lat de ?ase co?i, iar cel de al treilea, lat de ?apte co?i; caci împrejurul templului erau facute prichiciuri de zid, ca zidirea sa nu fie lipita de pere?ii templului.
1Ki 6:7  Când era zidit templul, la zidirea lui au întrebuin?at pietre cioplite, lucrate mai dinainte. A?a ca nici ciocan, nici topor, nici orice alta unealta de fier nu s-au auzit la zidirea lui.
1Ki 6:8  Intrarea la catul de jos al cladirii era pe partea dreapta a templului. Pe scari în spirala se suiau la catul din mijloc, ?i de la catul din mijloc, la catul al treilea.
1Ki 6:9  ?i a zidit el templul ?i l-a terminat ?i a pardosit templul cu scânduri de cedru.
1Ki 6:10  Odailor dimprejurul întregului templu le-a dat înal?ime de câte cinci co?i la fiecare cat ?i erau legate de templu prin grinzi de cedru.
1Ki 6:11  Atunci a fost cuvântul Domnului catre Solomon ?i i-a zis:
1Ki 6:12  "Iata, tu-Mi zide?ti casa; daca te vei purta dupa legile Mele, ?i vei urma dupa hotarârile Mele, ?i vei pazi toate poruncile Mele, lucrând dupa ele, atunci Îmi voi împlini ?i Eu cu tine cuvântul Meu pe care l-am grait catre David, tatal tau:
1Ki 6:13  Voi locui în mijlocul fiilor lui Israel ?i nu voi parasi pe poporul Meu Israel".
1Ki 6:14  ?i a zidit Solomon templul ?i l-a terminat.
1Ki 6:15  ?i a îmbracat pere?ii templului pe dinauntru cu scânduri de cedru; de la pardoseala templului pâna la tavan pe dinauntru l-a îmbracat peste tot cu lemn de cedru; iar pardoseala templului a facut-o din scânduri de chiparos.
1Ki 6:16  ?i a facut în partea din fund a templului o despar?itura de douazeci de co?i lungime ?i a îmbracat pere?ii ?i tavanul casei acestei despar?ituri cu scânduri de cedru; ?i a?a a facut despar?itura pentru Sfânta Sfintelor.
1Ki 6:17  De patruzeci de co?i era despar?itura întâi a templului.
1Ki 6:18  Pe scândurile de cedru dinauntru templului erau facute sculpturi în forma de castrave?i ?i flori de trandafiri îmboboci?i; totul era acoperit cu cedru ?i piatra nu se vedea.
1Ki 6:19  Iar despar?itura din fundul templului el a pregatit-o, ca sa puna acolo chivotul cu legea Domnului.
1Ki 6:20  ?i despar?itura aceasta era lunga de douazeci de co?i, lata de douazeci de co?i ?i înalta de douazeci de co?i; ?i a îmbracat-o cu aur curat; asemenea a îmbracat ?i jertfelnicul cel de cedru.
1Ki 6:21  ?i a îmbracat Solomon templul ?i pe dinauntru cu aur curat; ?i a întins lan?uri de aur pe dinaintea catapetesmei ?i a îmbracat-o cu aur.
1Ki 6:22  Tot templul l-a îmbracat cu aur, tot templul pâna la capat, ?i tot jertfelnicul care este dinaintea altarului l-a îmbracat cu aur.
1Ki 6:23  ?i a facut în Sfânta Sfintelor doi heruvimi de lemn de maslin, înal?i de zece co?i.
1Ki 6:24  O aripa a heruvimului era de cinci co?i ?i cealalta aripa a heruvimului era tot de cinci co?i. Zece co?i erau de la un vârf al aripilor lui pâna la vârful celeilalte aripi.
1Ki 6:25  Tot de zece co?i era ?i celalalt heruvim; amândoi heruvimii aveau aceea?i masura ?i aceea?i înfa?i?are.
1Ki 6:26  Înal?imea unui heruvim era de zece co?i; la fel ?i celalalt heruvim.
1Ki 6:27  ?i a a?ezat el heruvimii la mijloc în partea de la fund a templului. Aripile heruvimilor erau însa întinse; ?i atingea aripa unuia un perete ?i aripa celuilalt heruvim atingea pe celalalt perete. Iar celelalte aripi ale lor se atingeau în mijlocul templului aripa de aripa.
1Ki 6:28  ?i a îmbracat el heruvimii cu aur.
1Ki 6:29  Pe to?i pere?ii templului de jur împrejur, pe dinauntru ?i pe dinafara, a facut chipuri sapate de heruvimi, de copaci, de finici ?i de flori îmbobocite.
1Ki 6:30  ?i a îmbracat cu aur pardoseala în templu, în partea din fund ?i în partea din fa?a.
1Ki 6:31  Pentru intrat în Sfânta Sfintelor, a facut u?i de lemn de maslin care se deschid în doua par?i, cu u?ori în cinci muchii.
1Ki 6:32  Pe cele doua jumata?i ale u?ilor de lemn de maslin, el a facut heruvimi sapa?i, finici ?i flori îmbobocite; ?i le-a îmbracat în aur ?i heruvimii ?i finicii.
1Ki 6:33  La intrarea în templu a facut u?ori din lemn de maslin în patru muchii
1Ki 6:34  ?i doua u?i din lemn de chiparos, fiecare cu câte doua canate. Amândoua jumata?ile unei u?i se învârteau într-o parte ?i în alta ?i amândoua jumata?ile celeilalte u?i de asemenea se; învârteau într-o parte ?i în alta.
1Ki 6:35  ?i a sapat pe ele heruvimi, finici ?i flori îmbobocite, ?i le-a îmbracat cu aur peste sapatura.
1Ki 6:36  ?i a facut de asemenea curtea cea dinauntru din trei rânduri de pietre cioplite ?i dintr-un rând de grinzi de cedru.
1Ki 6:37  În anul al patrulea, în luna Zif, care este a doua luna a anului, a  pus el temelia casei Domnului.
1Ki 6:38  Iar în anul al unsprezecelea, în luna Bul, care este luna a opta, a terminat el templul, cu toate par?ile lor ?i dupa toate rânduielile lor; a?a ca l-a zidit în ?apte ani.
1Ki 7:1  Iar Solomon a zidit ?i a terminat casa sa în treisprezece ani.
1Ki 7:2  Casa aceasta a facut-o de lemn din Liban, lunga de o suta de co?i, larga de cincizeci de co?i ?i înalta de treizeci de co?i, pe patru ?iruri de stâlpi de cedru ?i pe stâlpi erau puse grinzi de cedru.
1Ki 7:3  Iar peste grinzi deasupra era întinsa podeaua da lemn de cedru, care se rezema pe patruzeci ?i cinci de stâlpi, câte cincisprezece în ?ir.
1Ki 7:4  ?i erau trei rânduri ?i fiecare rând avea ferestre a?ezate unele în dreptul altora, a?a ca raspundea fereastra cu fereastra în toate cele trei rânduri.
1Ki 7:5  Toate u?ile ?i u?orii de u?i erau din grinzi patrate, ?i ferestrele la cele trei rânduri fa?a în fa?a unele de altele.
1Ki 7:6  ?i a mai facut un pridvor pe stâlpi, lung da cincizeci de co?i, lat de treizeci de co?i; ?i înaintea lui un pridvor mai mic, au stâlpi ?i trepte în fa?a.
1Ki 7:7  ?i a facut de asemenea pridvor cu tron, de pe care el judeca, numit pridvorul judeca?ii; pe acesta l-a facut ?i l-a îmbracat cu cedru, da la pardoseala pâna la tavan.
1Ki 7:8  La casa sa de locuit, era facuta la fel alta curte ?i tot cu por?i. Asemenea ?i casa fiicei lui Faraon care Solomon o luase de femeie, a facut el un astfel de pridvor.
1Ki 7:9  Toate aceste cladiri au fost facute din pietre alese ?i frumos lucrate, cioplite dupa masura, retezate cu fierastraul înauntru ?i în afara, de la temelie pâna la strea?ina, ?i din afara pâna la curtea cea mare.
1Ki 7:10  La temelie au fost puse de asemenea pietre alese, pietre mari, pietre de zece co?i ?i opt co?i.
1Ki 7:11  ?i deasupra, pietre cioplite, lucrate dupa masura ?i lemn de cedru.
1Ki 7:12  Curtea cea mare avea împrejur o îngraditura din trei rânduri de pietre cioplite ?i un rând de grinzi de cedru. Tot astfel era îngradita ?i curtea dinlauntru a templului Domnului, precum ?i pridvorul templului.
1Ki 7:13  ?i a trimis regele Solomon ?i a luat pe me?terul Hiram din Tir.
1Ki 7:14  Acesta era fiul unei vaduve din semin?ia lui Neftali. Tatal lui, un tirian, era aramar; era ?i Hiram plin de pricepere, cu me?te?ug ?i cu ?tiin?a de a face orice lucru din arama. ?i a venit la regele Solomon ?i a facut tot felul de lucruri.
1Ki 7:15  A turnat pentru pridvor doi stâlpi de arama, fiecare stâlp de optsprezece co?i înal?ime, ?i rotundul fiecarui stâlp ara cât putea sa-l cuprinda o sfoara de doisprezece co?i. Grosimea era da patru degete, iar pe dinauntru era gol.
1Ki 7:16  ?i pentru pus în capetele stâlpilor, a facut doua coroane, turnate din arama; înal?imea unei coroane era de cinci co?i ?i înal?imea celeilalte coroane de cinci co?i;
1Ki 7:17  ?i pentru acoperit coroanele care erau în capetele stâlpilor, a facut el doua re?ele, lucrate împletit, ?nururi în forma de lan?uri: ?apte la o coroana ?i ?apte la cealalta coroana.
1Ki 7:18  ?i, când a facut stâlpii, a facut ?i doua ?iruri de rodii de arama, atârnate împrejur pe marginea re?elelor, ca sa împodobeasca coroana ce era pe vârful stâlpului; la fel a facut ?i celeilalte coroane de la celalalt stâlp.
1Ki 7:19  Coroanele din capetele stâlpilor de la pridvor erau facute în forma cupei florii de crin, de patru co?i la gura;
1Ki 7:20  ?i la coroanele de la amândoi stâlpii erau sus, la încheietura lor, în dreptul marginii re?elei, rodii de arama; ?i ?irurile de rodii dimprejur la fiecare coroana erau de câte doua sute de rodii.
1Ki 7:21  ?i a a?ezat stâlpii la pridvorul templului, punând un stâlp în partea din dreapta ?i dându-i numele Iachin; ?i pe celalalt stâlp în partea stânga, dându-i numele Booz.
1Ki 7:22  ?i pe capul stâlpilor a pus coroanele, facute în forma cupei florii de crin. A?a s-a sfâr?it lucrul stâlpilor.
1Ki 7:23  A mai facut o mare, turnata din arama, de zece co?i de la o margine a ei pâna la cealalta margine, rotunda de jur împrejur; înalta de cinci co?i ?i groasa cât o cuprindea o sfoara de treizeci de co?i.
1Ki 7:24  Pe la gura de jur împrejur avea sculpturi în forma de colocin?i, câte zece la un cot, care împrejmuiau marea din toate par?ile. Chipurile colocin?ilor a?eza?i în doua ?iruri erau turnate odata cu marea dintr-o singura bucata.
1Ki 7:25  Aceasta era a?ezata pe doisprezece boi de arama, din care: trei priveau spre miazanoapte, trei spre apus, trei spre miazazi ?i trei spre rasarit. Marea ?edea pe ei ?i toata partea dinapoi a trupului lor era înauntru.
1Ki 7:26  Grosimea pere?ilor ei era de un lat de mâna; ?i marginile ei, facute ca marginile potirului, semanau cu floarea de crin îmbobocit. ?i încapeau în ea doua mii de baturi (vedre).
1Ki 7:27  A mai facut apoi zece postamente de arama. Lungimea fiecarui postament era de patru co?i, la?imea de patru co?i ?i înal?imea de trei coli.
1Ki 7:28  Înfa?i?area postamentelor era a?a: erau lucrate în tablii, ?i tabliile erau încheiate la unghiuri;
1Ki 7:29  ?i pe tabliile acestea, care erau încheiate la unghiuri, erau sapa?i lei ?i boi ?i heruvimi; asemenea ?i pe încheieturi. Iar deasupra ?i dedesubtul leilor ?i boilor erau închipuite ghirlande de flori.
1Ki 7:30  Fiecare postament avea câte patru ro?i de arama ?i osii de arama. La cele patru col?uri ale lor erau ni?te console, în chipul unor numere; jos sub cupa spalatoarei ?i pe lânga fiecare ghirlanda de flori era o policioara turnata.
1Ki 7:31  Postamentul în partea de deasupra avea înauntru o adâncitura de pus ligheanul, adânca de un cot; gura ei era rotunda ca baza stâlpilor, de un cot ?i jumatate în diametru; ?i împrejurul gurii erau podoabe sapate; iar tabliile ei de pe laturi erau patrate ?i nu rotunde.
1Ki 7:32  Cele patru ro?i erau sub. tablii; ?i osiile ro?ilor erau fixate în postamente; înal?imea fiecarei ro?i era de un cot ?i jumatate.
1Ki 7:33  Forma ro?ilor era aceea?i, ca forma ro?ilor de trasura. Osiile lor, obezile lor, spi?ele lor ?i butucii lor, toate erau turnate.
1Ki 7:34  Cele patru console de la cele patru col?uri ale fiecarui postament erau tot turnate; policioarele erau ie?ite din postament.
1Ki 7:35  Partea de deasupra a postamentului se termina prin o cununa înalta de o jumatate de cot ?i facuta a?a, ca sa se poata pune spalatoarea deasupra; aceasta cu consolele ?i policioarele ei erau turnate din o bucata.
1Ki 7:36  ?i a sapat pe fe?ele consolelor postamentului ?i pe policioarele dintre ele heruvimi, lei ?i finici, pe unde a gasit loc; ?i împrejur a atârnat ghirlande de flori.
1Ki 7:37  A?a a facut el zece postamente: toate aveau aceea?i turnatura, aceea?i masura ?i aceea?i înfa?i?are.
1Ki 7:38  ?i a mai facut zece lighene de arama: în fiecare lighean încapea câte patruzeci de baturi; fiecare lighean era de patru coli ?i fiecare lighean sta pe unul din cele zece postamente.
1Ki 7:39  ?i a a?ezat postamentele cinci în partea dreapta a templului ?i cinci în partea stânga a templului, iar marea a a?ezat-o în partea dreapta a templului în partea de rasarit-miazazi.
1Ki 7:40  ?i a mai facut Hiram caldari, lope?i ?i cupe. ?i a?a a ispravit Hiram toate lucrarile date de Solomon sa le faca pentru templul Domnului:
1Ki 7:41  Cei doi stâlpi cu cele doua capiteluri rotunde a?ezate în capetele stâlpilor; cele doua împletituri care acopereau capetele rotunde ale capitelurilor pe vârful stâlpilor;
1Ki 7:42  Cele patru sute de rodii de arama de la cele doua re?ele; doua ?iruri de rodii la fiecare re?ea, pentru acoperirea celor doua globuri ale coroanelor care erau pe stâlpi;
1Ki 7:43  Cele zece postamente ?i cele zece spalatori de pe postamente;
1Ki 7:44  O mare ?i cei doisprezece boi de sub mare;
1Ki 7:45  Caldarile, lope?ile ?i cupele. Toate lucrurile care le-a facut Hiram regelui Solomon pentru templul Domnului erau de arama ?lefuita.
1Ki 7:46  Regele a pus sa le toarne într-un pamânt clisos din împrejurimile Iordanului, între Sucot ?i ?artan.
1Ki 7:47  ?i a pus Solomon toate aceste lucruri la locul lor. Din pricina mul?imii lor peste masura, greutatea aramei nu se mai ?tia.
1Ki 7:48  ?i a mai facut Solomon toate lucrurile care erau în templul Domnului: jertfelnicul cel de aur ?i masa cea de aur a pâinilor punerii înainte;
1Ki 7:49  Sfe?nice de aur curat: cinci în partea dreapta ?i cinci în partea stinga, înaintea Sfintei Sfintelor, cu florile, candelele ?i mucarile lor, toate de aur;
1Ki 7:50  Lighene, cu?ite, cupe, linguri ?i cadelni?e tot de aur curat; ?i ?â?ânile u?ilor celor din fundul templului de la Sfânta Sfintelor ?i ale u?ilor celor de la templu erau toate de aur.
1Ki 7:51  A?a s-a sfâr?it toata lucrarea pe care a facut-o regele Solomon la templul Domnului. ?i a adus Solomon ?i pe cele afierosite de David, tatal lui: argint ?i aur ?i lucruri, ?i le-a dat în vistieria templului Domnului.
1Ki 8:1  Atunci a adunat Solomon la el în Ierusalim pe batrânii lui Israel, pe capeteniile semin?iilor ?i pe to?i capii de familii ai fiilor lui Israel, ca sa aduca chivotul cu legea Domnului din cetatea lui David, adica din Sion.
1Ki 8:2  ?i s-au adunat la regele Solomon to?i Israeli?ii în zilele sarbatorilor din luna Etanim, care este a ?aptea luna.
1Ki 8:3  Iar dupa ce au venit to?i batrânii lui Israel, au ridicat preo?ii chivotul
1Ki 8:4  ?i au adus chivotul Domnului ?i cortul adunarii ?i toate lucrurile sfinte care au fost în cort; acestea le-au adus preo?ii ?i levi?ii.
1Ki 8:5  Iar regele Solomon împreuna cu toata ob?tea fiilor lui Israel, care se adunasera la el, mergea înaintea chivotului, aducând jertfe vite marunte ?i mari, care nu se puteau socoti ?i nici numara din pricina mul?imii lor.
1Ki 8:6  ?i au bagat preo?ii chivotul cu legea Domnului la locul lui, în Sfânta Sfintelor din templu, sub aripile heruvimilor;
1Ki 8:7  Caci heruvimii î?i aveau aripile întinse peste locul chivotului ?i heruvimii acopereau de sus chivotul ?i pârghiile lui:
1Ki 8:8  Pârghiile însa se împinsesera a?a, încât capetele lor se vedeau din loca?ul sfânt, din fa?a Sfintei Sfintelor, iar de afara nu se zareau; ?i acolo se afla ele pâna în ziua de azi.
1Ki 8:9  În chivot nu era nimic, decât cele doua table de piatra pe care Moise le pusese acolo în Horeb, când Domnul a facut legamânt cu fiii lui Israel, dupa ie?irea lor din pamântul Egiptului.
1Ki 8:10  Când preo?ii au ie?it din loca?ul sfânt, un nor a umplut templul Domnului.
1Ki 8:11  ?i n-au putut preo?ii sa stea la slujba, din pricina norului, caci slava Domnului umpluse templul Domnului.
1Ki 8:12  Atunci Solomon a zis: "Domnul a spus ca binevoie?te sa locuiasca în norul cel întunecos.
1Ki 8:13  Eu ?i-am zidit templul pentru locuin?a, în care Tu sa petreci în veci".
1Ki 8:14  Apoi s-a întors regele cu fa?a spre mul?ime ?i a binecuvântat toata adunarea Israeli?ilor, caci toata adunarea Israeli?ilor sta de fa?a,
1Ki 8:15  ?i a zis: "Binecuvântat fie Domnul Dumnezeul lui Israel, Care a grait cu gura Sa catre David, tatal meu, ceea ce astazi a împlinit cu mâna Sa!
1Ki 8:16  El a zis: Din ziua în care am scos pe poporul Meu Israel din Egipt, nu Mi-am ales cetate în nici una din semin?iile lui Israel, unde sa fie zidita casa în care sa petreaca numele Meu; dar apoi am ales Ierusalimul pentru petrecerea numelui Meu în el ?i am ales pe David ca sa fie peste poporul Meu, Israel.
1Ki 8:17  Lui David, tatal meu, îi intrase la inima sa zideasca casa numelui Domnului Dumnezeului lui Israel;
1Ki 8:18  Însa Domnul a zis câtre David, tatal meu: ?i-ai pus în gând sa zide?ti casa numelui Meu; este bine ca ?i-ai  pus aceasta la inima.
1Ki 8:19  Însa nu tu v ai zidi templul, ci fiul tau care va ie?i din coapsele tale, acela va zidi casa numelui Meu.
1Ki 8:20  ?i a Împlinit Domnul cuvântul Sau care l-a grait. Eu am urmat în locul tatalui meu. David, ?i am ?ezut pe tronul lui Israel, precum Domnul a zis, ?i am zidit templu numelui Domnului Dumnezeului lui Israel;
1Ki 8:21  ?i am pregatit acolo loc pentru chivotul în care se afla legamântul Domnului, facut cu parin?ii no?tri, când i-a scos din pamântul Egiptului".
1Ki 8:22  Apoi a stat Solomon înaintea jertfelnicului Domnului, în fa?a întregii adunari a lui Israel, ?i ?i-a ridicat mâinile la cer ?i a zis:
1Ki 8:23  "Doamne Dumnezeul lui Israel! Nu este Dumnezeu asemenea ?ie, nici în cer sus, nici pe pamânt jos; Tu paze?ti legamântul ?i ai mila de robii Tai care umbla cu toata inima lor înaintea Ta;
1Ki 8:24  Tu ai împlinit ce ai grait catre robul Tau David, tatal meu; caci ce ai grait cu gura Ta, aceea astazi ai împlinit cu mâna Ta.
1Ki 8:25  ?i asum, Doamne Dumnezeul lui Israel, sa împline?ti ceea ce ai grait cu robul Tau David, tatal meu, zicând: "Nu-?i va lipsi niciodata înaintea Mea un urma?, care sa ?ada pe tronul lui Israel, daca fiii tai î?i vor pazi drumul lor, purtându-se a?a cum te-ai purtat tu înaintea Mea!"
1Ki 8:26  ?i acum, Doamne Dumnezeul lui Israel, fa sa se adevereasca cuvântul Tau care l-ai grait cu robul Tau David, tatal meu!
1Ki 8:27  Oare adevarat sa fie ca Domnul va locui cu oamenii pe pamânt? Cerul ?i cerul cerurilor nu Te încap, cu atât mai pu?in acest templu pe care l-am zidit numelui Tau;
1Ki 8:28  Însa cauta la rugaciunea robului Tau ?i la cererea lui, Doamne Dumnezeul meu! Asculta strigarea ?i rugaciunea lui cu care se roaga astazi;
1Ki 8:29  Sa-?i fie ochii Tai deschi?i ziua ?i noaptea la templul acesta, la acest loc, pentru care Tu ai zis: "Numele Meu va fi acolo"; sa ascul?i strigarea ?i rugaciunea cu care robul Tau se va ruga în locul acesta.
1Ki 8:30  Sa ascul?i rugaciunea robului Tau ?i a poporului Tau, Israel, când ei se vor ruga în locul acesta; sa ascul?i din locul ?ederii Tale cel din ceruri, sa ascul?i ?i sa miluie?ti.
1Ki 8:31  Când cineva va gre?i împotriva aproapelui sau ?i i se va cere juramânt ca sa jure ?i pentru juramânt ei vor veni înaintea jertfelnicului Tau la templul acesta,
1Ki 8:32  Atunci Tu sa ascul?i din cer ?i sa faci judecata robilor Tai, sa osânde?ti pe cel vinovat, întorcându-i în capul lui fapta lui, ?i sa scapi pe cel drept, dându-i dupa dreptatea lui!
1Ki 8:33  Când poporul Tau Israel va fi batut de du?man, pentru ca a pacatuit înaintea Ta, ?i ei se vor întoarce la Tine ?i se vor marturisi numelui Tau aducând rugaciuni ?i cereri în acest templu,
1Ki 8:34  Atunci Tu sa ascul?i din cer, sa ier?i pacatul poporului Tau Israel ?i sa-l întorci în pamântul pe care l-ai dat parin?ilor lor!
1Ki 8:35  Când se va încuia cerul ?i nu va fi ploaie, pentru ca ei au pacatuit înaintea Ta, ?i Î?i vor aduce rugaciuni în locul acesta ?i vor marturisi numele Tau ?i se vor întoarce de la pacatul lor, caci Tu i-ai smerit,
1Ki 8:36  Atunci Tu sa ascul?i din cer ?i sa ier?i pacatul robilor Tai ?i al poporului Tau Israel, aratându-le calea cea buna pe care sa mearga, ?i sa trimi?i ploaie pamântului Tau pe care l-ai dat poporului Tau de mo?tenire!
1Ki 8:37  De va fi foamete pe pamânt, de va fi ciuma ?i boala molipsitoare, de va fi vânt dogoritor, uscaciune, lacusta, omida, du?manul de îl va strâmtora în por?ile ceta?ii lui, de va fi orice necaz sau orice boala,
1Ki 8:38  Orice rugaciune, orice cerere care se va face de orice om din tot poporul lui Israel, când ei î?i vor cunoa?te mustrarea cugetului lor ?i î?i vor întinde mâinile lor la templul acesta,
1Ki 8:39  Tu sa ascul?i din cer, din locul ?ederii Tale, ?i sa miluie?ti; sa faci ?i sa dai fiecaruia dupa caile sale, dupa cum Tu cuno?ti inima lui; caci Tu singur ?tii inima tuturor fiilor oamenilor;
1Ki 8:40  Pentru ca sa se teama de Tine toate zilele, cât vor trai pe pamântul pe care l-ai dat parin?ilor no?tri!
1Ki 8:41  Chiar strainul, care nu este din poporul Tau Israel, de va veni pentru numele Tau din pamânt departat,
1Ki 8:42  Caci se va auzi de numele Tau cel mare ?i de mâna Ta cea puternica ?i de bra?ul Tau cel întins, ?i el va veni ?i se va ruga la templul acesta,
1Ki 8:43  Sa-l ascu?i din cer, din locul ?ederii Tale, ?i sa faci tot ceea ce strainul Î?i va cere ?ie, pentru ca sa ?tie toate popoarele pamântului de numele Tau, sa se teama de Tine, cum se teme poporul Tau Israel, ?i sa ?tie ca numele Tau este chemat peste templul acesta pe care eu l-am zidit!
1Ki 8:44  Când poporul Tau va porni cu razboi împotriva du?manului sau, pe drumul pe care-l  vei trimite, ?i se v ruga Domnului, întorcându-se spre cetatea care ?i-ai ales-o ?i spre templul pe care l-am zidit numelui Tau,
1Ki 8:45  Atunci asculta din cer rugaciunea lor ?i sa faci ceea ce le este cu dreptate!
1Ki 8:46  Când ei vor pacatui înaintea Ta, caci nu este om care sa nu pacatuiasca, ?i Tu Te vei mânia pe ei ?i îi vei da du?manilor lor, ?i cei care i-au robit îi vor duce în pamântul du?manului, departe sau aproape,
1Ki 8:47  ?i când ei, în pamântul în care se vor gasi în robie, î?i vor veni în sine ?i se vor întoarce ?i ?i se vor ruga în pamântul celor ce i-au robit, zicând: "Am pacatuit, faradelege am facut, vinova?i suntem",
1Ki 8:48  ?i se vor întoarce catre Tine cu toata inima lor ?i cu tot sufletul lor, în pamântul du?manilor care i-au robit, ?i se vor ruga catre Tine, întorcându-se spre pamântul care l-ai dat parin?ilor lor, spre cetatea care ?i-ai ales-o ?i spre templul pe care l-am zidit numelui Tau,
1Ki 8:49  Atunci sa ascul?i din cer, din locul ?ederii Tale, rugaciunea ?i cererea lor, facându-le ceea ce este cu dreptate.
1Ki 8:50  Sa ier?i poporului Tau ce a pacatuit înaintea Ta ?i toate nelegiuirile lui care le-a facut înaintea Ta ?i sa treze?ti mila catre ei în cei ce i-au robit, pentru ca sa fie milo?i cu ei;
1Ki 8:51  Caci ei sunt poporul Tau ?i mo?tenirea Ta, pe care l-ai scos din Egipt, din cuptorul cel de fier!
1Ki 8:52  Sa-?i fie urechile Tale ?i ochii Tai deschi?i la rugaciunea robului Tau ?i la rugaciunea poporului Tau Israel, pentru ca sa-i auzi totdeauna, când ei vor striga catre Tine,
1Ki 8:53  Ca Tu ?i r-ai ales spre mo?tenire dintre toata popoarele pamântului, precum ai grait prin Moise, robul Tau, când ai scos pe parin?ii no?tri din Egipt, Stapâne Doamne!"
1Ki 8:54  Când Solomon a sfâr?it toata aceasta rugaciune ?i cerere catre Domnul, s-a sculat dinaintea jertfelnicului Domnului, unde statuse îngenunchiat cu mâinile întinse spre cer,
1Ki 8:55  ?i stând în picioare, a binecuvântat toata adunarea Israeli?ilor, zicând:
1Ki 8:56  "Binecuvântat fie Domnul Dumnezeu, Care a dat odihna poporului Sau Israel, precum a grait! Nu a ramas neîmplinit nici un cuvânt din toate bunele Lui cuvinte care le-a grait prin robul Sau Moise.
1Ki 8:57  Sa fie cu noi Domnul Dumnezeul nostru cum a fost El cu parin?ii no?tri ?i sa nu ne lase, parasindu-ne.
1Ki 8:58  Plecând spre El inima noastra, sa umblam pe toate caile Lui ?i sa pazim poruncile, rânduielile ?i legile Lui, pe care le-a poruncit parin?ilor no?tri;
1Ki 8:59  ?i sa fie cuvintele acestea cu care m-am rugat astazi, înaintea Domnului, aproape de Domnul Dumnezeul nostru, ziua ?i noaptea, pentru ca sa faca dreptate robului ?i poporului Sau Israel, din zi în zi,
1Ki 8:60  Pentru ca sa cunoasca toate popoarele ca Domnul este Dumnezeu ?i nu este altul afara de El!
1Ki 8:61  Sa fie inima voastra întreaga la Domnul Dumnezeul nostru, ca sa petrece?i dupa rânduielile Lui ?i sa pazi?i poruncile Lui, ca acum!"
1Ki 8:62  ?i regele împreuna cu to?i Israeli?ii au adus jertfa Domnului.
1Ki 8:63  Pentru jertfa de împacare pe care a adus-o el Domnului, Solomon a adus douazeci ?i doua de mii de vite mari ?i o suta douazeci de mii de vite marunte. A?a a sfin?it regele ?i to?i fiii lui Israel templul Domnului.
1Ki 8:64  Tot în acea zi regele a mai sfin?it ?i mijlocul cur?ii care era înaintea templului Domnului, savâr?ind acolo arderea de tot, darul de pâine ?i grasimea jertfelor de împacare, pentru ca jertfelnicul de arama care se afla înaintea Domnului era mic pentru a încapea arderea de tot, darul de pâine ?i grasimea jertfelor de împacare.
1Ki 8:65  ?i a sarbatorit Solomon ?i sarbatoarea (Corturilor) în acela?i timp împreuna cu tot Israelul, strângându-se adunare mare de la intrarea Hamatului ?i pâna la râul Egiptului, pentru a fi înaintea Domnului Dumnezeului nostru timp de ?apte zile ?i alte ?apte zile, adica paisprezece zile.
1Ki 8:66  în ziua a opta Solomon a dat drumul poporului. ?i au binecuvântat to?i pe rege ?i s-au întors la corturile lor, bucurându-se ?i veselindu-se cu inima pentru tot binele ce l-a facut Domnul robului Sau David ?i poporului Sau Israel.
1Ki 9:1  Dupa ce Solomon a sfâr?it de zidit templul Domnului ?i casa regelui ?i tot ce Solomon a dorit sa faca,
1Ki 9:2  S-a aratat Domnul lui Solomon a doua oara, la Ghibeon,
1Ki 9:3  ?i i-a zis Domnul: "Am auzit rugaciunea ta ?i cererea ta cu care te-ai rugat catre Mine ?i ?i-am îndeplinit toate dupa cererea ta; am sfin?it templul pe care l-ai zidit, ca sa petreaca numele Meu acolo în veci ?i vor fi ochii ?i inima Mea acolo în toate zilele.
1Ki 9:4  Daca tu te vei purta înaintea fe?ei Mele, cum s-a purtat tatal tau David, cu inima curata ?i cu dreptate, împlinind tot ce Eu ?i-am poruncit, ?i vei pazi rânduielile ?i legile Mele,
1Ki 9:5  Atunci voi întari tronul regatului tau peste Israel în veci, precum i-am grait lui David, tatal tau, zicând: Nu vei fi lipsit niciodata de un urma? pe tronul lui Israel.
1Ki 9:6  Iar daca voi ?i fiii vo?tri va ve?i departa de la Mine ?i nu ve?i pazi poruncile Mele ?i rânduielile Mele pe care Eu vi le-am dat ?i va ve?i duce ?i ve?i sluji ?i va ve?i închina la al?i dumnezei,
1Ki 9:7  Atunci Eu voi stârpi pe Israel de pe fa?a pamântului pe care i l-am dat, iar templul pe care l-am sfin?it în numele Meu îl voi lepada de la fa?a Mea ?i Israel va fi de pomina ?i de râs între toate popoarele.
1Ki 9:8  ?i de acest templu înalt, oricine va trece pe lânga el se va îngrozi ?i va fluiera ?i va zice: Pentru ce Domnul a facut a?a cu acest pamânt ?i cu acest templu?
1Ki 9:9  ?i se va zice: Pentru ca au parasit pe Domnul Dumnezeul lor, Care a scos pe parin?ii lor din pamântul Egiptului, din casa robiei ?i au primit în schimb al?i dumnezei ?i s-au închinat lor ?i au slujit lor; pentru aceasta a adus Domnul peste ei toata aceasta nenorocire".
1Ki 9:10  Dupa trecerea celor douazeci de ani în care Solomon a zidit templul Domnului ?i casa regelui,
1Ki 9:11  Pentru care Hiram, regele Tirului, a dat lui Solomon lemn de cedru, lemn de chiparos ?i aur dupa cerin?a lui, regele Solomon a dat lui Hiram douazeci de ceta?i din pamântul Galileii.
1Ki 9:12  ?i a plecat Hiram din Tir ?i s-a dus în Galileea ca sa vada ceta?ile daruite de Solomon ?i nu i-au placut.
1Ki 9:13  ?i a zis: "Ce sunt, fratele meu, ceta?ile acestea care mi le-ai dat?" ?i le-a numit pamântul Cabul, cum se numesc ele pâna în ziua de astazi.
1Ki 9:14  ?i Hiram trimisese regelui Solomon o suta de talan?i de aur.
1Ki 9:15  Iata hotarârea pentru darea pe care a pus-o regele Solomon, ca sa zideasca templul Domnului, casa lui, Milo, zidul Ierusalimului, Ha?orul, Meghido ?i Ghezerul;
1Ki 9:16  Caci Faraon, regele Egiptului, venise ?i luase Ghezerul ?i-l arsese cu foc ?i pe Canaaneii care locuiau în cetate îi ucisese, ?i-l daduse de zestre fiicei sale, femeii lui Solomon.
1Ki 9:17  ?i a zidit Solomon ceta?ile Ghezer, Bet-Horonul de Jos,
1Ki 9:18  Baalat ?i Tadmorul din pustiu,
1Ki 9:19  ?i toate ceta?ile grânare, care le-a avut Solomon ?i ceta?ile pentru carele de razboi, ceta?ile pentru calare?i ?i tot ce a vrut Solomon sa zideasca în Ierusalim, în Liban ?i în pamântul stapânirii sale.
1Ki 9:20  Pe tot poporul care a ramas de la Amorei, Hetei, Ferezei, Canaanei, Hevei, Iebusei ?i Gherghesei, care nu erau dintre fiii lui Israel,
1Ki 9:21  ?i pe fiii acestora, rama?i în ?ara dupa ei ?i pe care fiii lui Israel nu au putut sa-i stapâneasca, Solomon i-a facut lucratori de corvoada pâna în ziua de azi,
1Ki 9:22  Iar pe fiii lui Israel, Solomon nu i-a facut lucratori; pe ei îi avea însa pentru o?tirea lui, pentru slujitorii lui, pentru capitanii lui, pentru capeteniile lui ?i pentru conducatori la carele lui ?i la calare?ii lui.
1Ki 9:23  Iar ispravnicii cei de frunte, de peste lucrarile lui Solomon, cei ce supravegheau poporul care facea lucrul, erau cinci sute cincizeci.
1Ki 9:24  Atunci fiica lui Faraon a trecut din cetatea lui David în casa zidita de Solomon pentru ea. Apoi a zidit el Milo.
1Ki 9:25  Solomon aducea de trei ori pe an arderi de tot ?i jertfe de împacare pe jertfelnicul pe care-l zidise Domnului, savâr?ind tamâiere înaintea Domnului. ?i a terminat el ?i zidirea casei lui.
1Ki 9:26  Regele Solomon a mai facut ?i corabii la E?ion-Gheber, care este lânga Elot, pe malul Marii Ro?ii, în pamântul lui Edom.
1Ki 9:27  ?i a trimis Hiram dintre supu?ii sai corabieri, cunoscatori ai marii, ca sa duca corabiile cu supu?ii lui Solomon.
1Ki 9:28  ?i s-au dus la Ofir ?i au luat de acolo patru sute douazeci de talan?i de aur ?i i-au dus regelui Solomon.
1Ki 10:1  Regina din Saba însa, auzind de slava lui Solomon cea în numele Domnului, a venit sa-i încerce în?elepciunea cu cuvinte greu de în?eles.
1Ki 10:2  Venind ea la Ierusalim cu foarte mare boga?ie, cu camile încarcate cu aromate, cu foarte mult aur ?i pietre scumpe, a mers la Solomon ?i s-a sfatuit cu el pentru tot ce avea ea pe inima.
1Ki 10:3  ?i i-a dezlegat Solomon toate vorbele ei ?i n-a fost vorba adânca pe care sa n-o cunoasca regele ?i sa nu i-o dezlege.
1Ki 10:4  Vazând deci regina din Saba toata în?elepciunea lui Solomon, casa care a zidit-o el,
1Ki 10:5  Bucatele de la masa lui, locuin?a robilor lui, rânduiala slugilor lui, îmbracamintea lor, paharnicii lui ?i arderile de tot ale lui care le aducea în templul Domnului, nu a putut sa se mai stapâneasca
1Ki 10:6  ?i a zis regelui: "Adevarat este ce am auzit eu în ?ara mea de lucrurile tale ?i de în?elepciunea ta;
1Ki 10:7  Însa eu nu credeam vorbele, pâna n-am venit ?i n-am vazut cu ochii mei ?i iata, nici pe jumatate nu mi se spusese; tu ai în?elepciune ?i boga?ie mult mai mare decât am auzit eu.
1Ki 10:8  Ferice de oamenii tai ?i de aceste slugi ale tale, care totdeauna î?i stau înainte ?i asculta în?elepciunea ta!
1Ki 10:9  Binecuvântat sa fie Domnul Dumnezeul tau Care a binevoit sa te puna pe tronul lui Israel! Domnul, din dragostea cea ve?nica a Lui catre Israel, te-a pus rege sa faci judecata ?i dreptate".
1Ki 10:10  ?i a daruit ea regelui o suta douazeci de talan?i de aur ?i o mul?ime de aromate ?i de pietre scumpe; niciodata însa nu i s-a adus atât de multe aromate, câte a daruit regina din Saba regelui Solomon.
1Ki 10:11  Iar corabiile lui Hiram, care aduceau aur de la Ofir, au adus foarte mult lemn ro?u ?l pietre scumpe.
1Ki 10:12  ?i a facut regele din acest lemn ro?u balustrade pentru templul Domnului ?i pentru casa regelui ?i de asemenea chitare ?i harpe pentru cântare?i. Niciodata nu i s-a adus lui atâta lemn ro?u, nici nu s-a mai vazut pâna în ziua de azi.
1Ki 10:13  Iar regele Solomon a dat reginei din Saba tot ce a dorit ?i a cerut, pe lânga ce i-a daruit regele Solomon cu mâna lui. ?i s-a întors ea înapoi la ?ara ei, ea ?i toate slugile ei.
1Ki 10:14  Greutatea aurului care i se aducea pe fiecare an lui Solomon era de ?ase sute ?aizeci de talan?i de aur,
1Ki 10:15  Afara de ce primea el de la aducatorii de marfuri ?i de la negustori, de la to?i regii arabi ?i de la capeteniile ?inuturilor.
1Ki 10:16  ?i a facut regele Solomon doua sute de scuturi de aur ciocanit, câte ?ase sute de sicli pentru fiecare scut,
1Ki 10:17  ?i trei sute de scuturi mai mici tot din aur ciocanit; câte trei mine de aur intra în fiecare scut; ?i le-a pus regele în casa numita Padurea Libanului.
1Ki 10:18  ?i a mai facut regele un tron mare de os de filde?, ferecându-l cu aur curat.
1Ki 10:19  Tronul avea ?ase trepte, iar vârful spetezei tronului era rotund; ?i avea de o parte ?i de alta a locului de ?edere rezematori, pe care stateau doi lei.
1Ki 10:20  ?i mai erau înca doisprezece lei care stateau de o parte ?i de alta a tronului, pe cele ?ase trepte. Asemenea tron nu mai era în nici un regat.
1Ki 10:21  Toate vasele de baut ale regelui Solomon erau de aur; lighenele lui tot de aur ?i toate vasele din casa Padurea Libanului erau de aur curat; de argint nu era nimic facut; argintul nu valora în zilele lui Solomon,
1Ki 10:22  Caci regele avea pe mare corabii care mergeau la Tarsis cu corabiile lui Hiram ?i la trei ani o data veneau corabiile din Tarsis ?i îi aduceau aur, argint, filde?, maimu?e ?i pauni.
1Ki 10:23  Regele Solomon a întrecut pe to?i regii pamântului în boga?ie ?i în în?elepciune.
1Ki 10:24  ?i to?i regii de pe pamânt cautau sa vada pe Solomon, ca sa-i asculte în?elepciunea pe care i-o pusese Dumnezeu în inima lui.
1Ki 10:25  ?i-i aduceau fiecare de la ei, ca dar în fiecare an, vase de aur ?i argint, haine, arme, aromate, cai ?i catâri.
1Ki 10:26  ?i ?i-a adunat Solomon care ?i calare?i; ?i avea el o mie patru sute de care ?i douasprezece mii de calare?i; ?i i-a a?ezat în ceta?ile unde ?inea carele ?i pe lânga rege, în Ierusalim. ?i e! era domn peste to?i regii de la râul Eufrat pâna la pamântul Filistenilor ?i pâna în hotarul Egiptului.
1Ki 10:27  El a facut ca argintul sa fie tot a?a de pre?uit la Ierusalim ca pietrele, iar cedrii, prin mul?imea lor i-a facut sa fie pre?ui?i ca ?i smochinii cei salbatici care cresc prin locuri joase.
1Ki 10:28  Iar caii pentru regele Solomon se aduceau din Egipt ?i din Coa. Negustorii regelui luau cai din Coa (Cheve) cu bani.
1Ki 10:29  Un car din Egipt se cumpara ?i se aducea cu ?ase sute sicli de argint, iar un cal, cu o suta cincizeci de sicli. Tot astfel aduceau ei toate acestea ?i pentru regii Heteilor ?i regii Siriei.
1Ki 11:1  Regele Solomon, în afara de fata lui Faraon, a iubit ?i alte multe femei straine: moabite, amonite, idumeiene, sidoniene, hetite ?i amoriene,
1Ki 11:2  Adica din acele popoare pentru care Domnul zisese fiilor lui Israel: "Sa nu va duceri la ele, nici ele sa nu vina la voi, ca sa nu va întoarca inima voastra spre dumnezeii lor". De acestea s-a lipit Solomon cu dragoste.
1Ki 11:3  ?i a avut el ?apte sute de femei ?i trei sute de concubine; ?i femeile i-au smintit inima lui.
1Ki 11:4  La timpul batrâne?ii lui Solomon, femeile lui i-au întors inima spre al?i dumnezei ?i inima lui nu i-a mai fost deloc întreaga la Domnul Dumnezeul sau, ca inima lui David, tatal sau.
1Ki 11:5  ?i a început Solomon sa slujeasca Astartei, zei?a Sidonienilor, ?i lui Moloh, idolul Amoni?ilor.
1Ki 11:6  Astfel a faptuit Solomon lucruri neplacute înaintea ochilor Domnului ?i nu a urmat cu staruin?a dupa Domnul, cum urmase David, tatal lui.
1Ki 11:7  Atunci a zidit Solomon capi?tea lui Chemo?, idolul Moabi?ilor, pe muntele din fa?a Ierusalimului, ?i capi?tea lui Moloh, idolul Amoni?ilor.
1Ki 11:8  Tot a?a a facut el pentru toate femeile sale straine, care tamâiau ?i aduceau jertfe dumnezeilor lor.
1Ki 11:9  Atunci S-a mâniat Domnul pe Solomon, pentru ca ?i-a abatut inima lui de la Domnul Dumnezeul lui Israel, Care de doua ori i Se aratase,
1Ki 11:10  ?i-i poruncise chiar anume pentru aceasta, ca el sa nu umble dupa al?i dumnezei, ci sa pazeasca ?i sa faca cele ce i-a poruncit Domnul Dumnezeu; dar el n-a împlinit cele ce i-a poruncit Domnul Dumnezeu.
1Ki 11:11  ?i a zis Domnul lui Solomon: "Pentru ca s-au facut aceste lucruri cu tine ?i pentru ca tu nu ai ?inut legamântul ?i rânduielile Mele pe care ?i le-am poruncit, voi rupe regatul tau din mâna ta ?i-l voi da slujitorului tau;
1Ki 11:12  Însa nu voi face aceasta în zilele tale, pentru David, tatal tau, ci din mâna fiului tau îl voi rupe.
1Ki 11:13  ?i nu tot regatul îl voi rupe: o semin?ie o voi da fiului tau, pentru David, robul Meu, ?i pentru Ierusalimul pe care l-am ales".
1Ki 11:14  ?i a ridicat Domnul un vrajma? împotriva lui Solomon, pe Hadad Idumeul, vi?a de rege din Edom.
1Ki 11:15  Caci pe când David era în Idumeea ?i Ioab, mai-marele o?tirii lui, venise ca sa îngroape pe cei uci?i ?i a ucis pe to?i cei de parte barbateasca din Idumeea,
1Ki 11:16  Pentru ca ?ase luni a stat acolo Ioab ?i to?i Israeli?ii, stârpind pe to?i cei de parte barbateasca din Idumeea,
1Ki 11:17  Atunci acest Hadad a fugit în Egipt ?i împreuna cu el ?i vreo câ?iva idumei, care fusesera în slujba tatalui lui. Hadad era pe atunci copil mic.
1Ki 11:18  Plecând din Madian, ei au venit la Paran ?i au luat cu ei oameni din Paran ?i au venit în Egipt la Faraon, regele Egiptului.
1Ki 11:19  ?i s-a dus Hadad la Faraon ?i acesta i-a dat casa, i-a purtat grija de mâncare ?i i-a dat ?i pamânt. În sfâr?it, Hadad a gasit la Faraon mila foarte mare, caci el i-a dat femeie pe sora femeii lui, pe sora reginei Tafnes.
1Ki 11:20  ?i i-a nascut sora Tafnesei fiu lui Hadad pe Ghenubat, pe care Tafnes însu?i l-a crescut în casa lui Faraon. ?i a trait Ghenubat în casa lui Faraon la un loc cu fiii lui Faraon.
1Ki 11:21  Când Hadad a auzit ca David a raposat cu parin?ii lui ?i ca Ioab, mai-marele o?tirii a murit, atunci a zis lui Faraon: "Da-mi drumul, ca vreau sa plec în ?ara mea!"
1Ki 11:22  ?i i-a zis Faraon lui Hadad: "Ce lipse?te ?ie aici la mine, de vrei sa pleci? ?i el a raspuns: "Nimic; însa a?a te rog, sa-mi dai drumul!" ?i s-a întors Hadad în ?ara lui.
1Ki 11:23  Apoi a mai ridicat Dumnezeu împotriva lui Solomon înca un vrajma?, pe Rezon, fiul lui Eliada, care fugise la Hadad-Ezer, regele din ?oba.
1Ki 11:24  Acesta, dupa ce David a zdrobit pe Hadad-Ezer, adunându-?i oameni împrejurul sau, s-a facut capetenia unei bande de ho?i ?i s-a dus la Damasc ?i a locuit acolo ?i s-a facut stapân pe Damasc.
1Ki 11:25  El a du?manit pe Israel în toate zilele lui Solomon ?i, afara de raul ce-l pricinuia Hadad acestuia din urma, Rezon totdeauna a adus vatamare lui Israel ?i s-a facut rege în Siria.
1Ki 11:26  Asemenea ?i Ieroboam, fiul lui Nabat, Efremiteanul din ?ereda, pe a carui mama vaduva o chema ?erua, ?i care era roaba lui Solomon, a ridicat mâna împotriva regelui.
1Ki 11:27  ?i iata împrejurarea în care el ?i-a ridicat mâna împotriva regelui: Pe când Solomon zidea Milo ?i repara stricaciunile de la cetatea lui David, tatal sau,
1Ki 11:28  Se afla acolo ?i Ieroboam, om tare în putere. Solomon, bagând de seama ca acest om tânar ?tie sa faca treaba, l-a pus dregator peste lucratorii de corvoada din casa lui Iosif.
1Ki 11:29  În acel timp i s-a întâmplat lui Ieroboam sa iasa din Ierusalim ?i l-a întâlnit în drum proorocul Ahia din ?ilo, care avea pe el o haina noua. În câmp erau numai ei amândoi.
1Ki 11:30  ?i a luat Ahia de pe el haina cea noua, a rupt-o în douasprezece buca?i
1Ki 11:31  ?i a zis lui Ieroboam: "Ia-?i pentru tine zece buca?i caci a?a graie?te Domnul Dumnezeul lui Israel: Iata, Eu rup regatul din mâna lui Solomon ?i-?i dau ?ie zece semin?ii,
1Ki 11:32  Iar doua semin?ii îi vor ramâne lui, pentru robul Meu David ?i pentru cetatea Ierusalimului, pe care am ales-o dintre toate semin?iile lui Israel.
1Ki 11:33  ?i aceasta o fac, pentru ca ei M-au parasit ?i au început sa se închine Astartei, zei?a Sidonienilor, lui Chemo?, dumnezeul Moabi?ilor, ?i lui Moloh, dumnezeul fiilor Amoni?ilor, ?i n-au umblat în caile Mele, ca sa faca cele placute înaintea ochilor Mei ?i sa pazeasca rânduielile Mele ?i poruncile Mele, cum a facut David, tatal lui Solomon.
1Ki 11:34  Tot regatul nu din mâna lui îl iau, caci îl las pe el sa fie stapân în toate zilele vie?ii lui, pentru David, robul Meu, pe care l-am ales ?i care a pazit poruncile ?i rânduielile Mele;
1Ki 11:35  Ci din mâna fiului lui voi lua regatul ?i-?i voi da din el ?ie zece semin?ii;
1Ki 11:36  Iar fiului lui îi voi da doua semin?ii, pentru ca sa fie ?i pentru David, robul Meu, în toate zilele, o lumina înaintea fe?ei Mele, în cetatea Ierusalimului pe care am ales-o pentru petrecerea numelui Meu acolo.
1Ki 11:37  ?i pe tine te voi alege ?i vei domni peste tot ce-?i dore?te sufletul tau ?i vei fi rege peste Israel.
1Ki 11:38  ?i daca tu vei pazi tot ce-?i poruncesc ?i vei umbla în caile Mele ?i vei face cele placute înaintea ochilor Mei ca sa paze?ti rânduielile Mele ?i poruncile Mele, cum a facut David, robul Meu, atunci Eu voi fi cu tine ?i-?i voi zidi casa tare, cum am zidit lui David ?i î?i voi da Israelul;
1Ki 11:39  ?i voi umili neamul lui David pentru aceasta, însa nu pentru totdeauna".
1Ki 11:40  De aceea Solomon a cautat sa omoare pe Ieroboam; dar Ieroboam s-a sculat ?i a fugit în Egipt, la ?i?ac, regele Egiptului, ?i a trait în Egipt pâna la moartea lui Solomon.
1Ki 11:41  Iar celelalte fapte ale lui Solomon de la sfâr?it ?i tot ce el a mai facut ?i toata în?elepciunea lui sunt scrise în cartea faptelor lui Solomon.
1Ki 11:42  Timpul domniei lui Solomon în Ierusalim peste tot Israelul a fost de patruzeci de ani.
1Ki 11:43  ?i a adormit cu parin?ii lui ?i a fost înmormântat Solomon la un loc cu parin?ii lui, în cetatea lui David, tatal sau; iar în locul lui în Ierusalim s-a facut rege Roboam, fiul sau.
1Ki 12:1  Atunci s-a dus Roboam la Sichem, caci la Sichem venisera to?i Israeli?ii, ca sa-l faca rege.
1Ki 12:2  ?i a auzit ?i Ieroboam, fiul lui Nabat, de aceasta, pe când se afla înca în Egipt, unde fugise de regele Solomon, ?i a venit Ieroboam din Egipt, caci au trimis unii dupa el ?i l-au chemat.
1Ki 12:3  Atunci Ieroboam ?i toata adunarea Israeli?ilor au venit de i-au grait regelui Roboam ?i i-au zis:
1Ki 12:4  "Tatal tau a pus jug greu pe noi; însa u?ureaza-ne munca cea grea a tatalui tau ?i jugul cel greu care l-a pus el pe noi, ?i î?i vom sluji!"
1Ki 12:5  ?i a zis el catre ei: "Duce?i-va ?i sa veni?i poimâine la mine!" ?i a plecat poporul.
1Ki 12:6  Atunci regele Roboam a întrebat pe batrânii care fusesera sfetnici pe lânga Solomon, tatal lui, pe când traia el, ?i le-a zis: "Cum ma sfatui?i sa raspund poporului la cererea aceasta?"
1Ki 12:7  Grait-au lui aceia ?i au zis: "Daca tu vei fi sluga astazi poporului acestuia ?i-i vei sluji, daca le vei face gustul lor ?i le vei vorbi cu blânde?e, atunci ei î?i vor fi robi în toate zilele!
1Ki 12:8  Însa el n-a ?inut seama de sfatul batrânilor pe care i l-au dat, ?i a facut sfat cu cei tineri, crescu?i odata cu el, ?i care erau sfetnicii lui ?i le-a zis:
1Ki 12:9  "Ce ma sfatui?i sa raspund poporului care mi-a zis: U?ureaza jugul pe care l-a pus tatal tau pe noi?"
1Ki 12:10  ?i i-au grait oamenii cei tineri, care crescusera odata cu el ?i erau acum sfetnicii lui, ?i i-au zis: "A?a sa spui poporului acestuia care ?i-a grait ?i ?i-a zis: Tatal tau a pus jug greu pe noi; tu însa u?ureaza-l de pe noi; a?a spune-le: Degetul meu cel mic este mai gros decât mijlocul tatalui meu.
1Ki 12:11  Deci daca tatal meu v-a împovarat cu jug greu, eu ?i mai greu voi face jugul vostru; tatal meu v-a pedepsit cu bice, iar eu va voi pedepsi cu scorpioane!
1Ki 12:12  Atunci a venit Ieroboam ?i tot poporul la Roboam a treia zi, dupa cum poruncise regele, când a zis: "Sa veni?i la mine poimâine!"
1Ki 12:13  ?i a raspuns regele poporului cu asprime ?i n-a ?inut seama de sfatul care i-l dadusera batrânii,
1Ki 12:14  Ci a grait catre el dupa sfatul celor tineri ?i a zis: "Tatal meu a pus jug greu peste voi; eu însa ?i mai greu voi face jugul vostru; tatal meu v-a pedepsit cu bice, eu însa va voi pedepsi cu scorpioane!"
1Ki 12:15  ?i n-a ascultat regele de popor, caci a?a fusese rânduit de Domnul ca sa se împlineasca cuvântul Lui pe care l-a grait Domnul prin Ahia din ?ilo catre Ieroboam, fiul lui Nabat.
1Ki 12:16  Iar când to?i Israeli?ii au vazut ca regele n-a vrut sa-i asculte, a raspuns ?i poporul regelui ?i i-a zis: "Ce parte avem noi cu David? Nici o mo?tenire nu avem noi cu fiul lui Iesei. Pleaca, dar, la corturile tale, Israele! ?i tu, Davide, cunoa?te-?i acum casa ta! ?i s-a împra?tiat Israel la corturile lui.
1Ki 12:17  Acum Roboam domnea numai peste fiii lui Israel care locuiau în ceta?ile lui Iuda.
1Ki 12:18  ?i a trimis regele Roboam pe Adoniram, capetenia cea peste dari, în contra lor; însa to?i Israeli?ii au aruncat cu pietre asupra lui ?i el a murit; iar regele Roboam s-a suit repede într-un car, ca sa fuga la Ierusalim.
1Ki 12:19  ?i astfel s-a rupt Israelul de casa lui David pâna în ziua de astazi.
1Ki 12:20  Când ton Israeli?ii au auzit ca Ieroboam s-a întors din Egipt, au trimis ?i l-au chemat la adunare ?i l-au facut rege peste to?i Israeli?ii. Cu casa lui David n-a ramas nimeni, decât semin?ia lui Iuda.
1Ki 12:21  Roboam, venind la Ierusalim, a adunat din toata casa lui Iuda ?i din semin?ia lui Veniamin o suta ?i optzeci de mii de osta?i, pregati?i pentru razboi, ca sa se lupte cu casa lui Israel ?i sa întoarca regatul iara?i sub stapânirea lui Roboam, fiul lui Solomon.
1Ki 12:22  ?i a fost cuvântul Domnului catre ?emaia, omul lui Dumnezeu, ?i i-a zis:
1Ki 12:23  "Spune-i lui Roboam, fiul lui Solomon, regele lui Iuda, ?i la toata casa lui Iuda ?i a lui Veniamin ?i la celalalt popor:
1Ki 12:24  A?a zice Domnul: Sa nu merge?i, nici sa nu începe?i razboi cu fra?ii vo?tri, fiii lui Israel. Sa va întoarce?i fiecare la casele voastre, caci de la Mine a fost aceasta! ?i au ascultat ei de cuvântul Domnului ?i s-au întors înapoi dupa cuvântul Domnului. ?i a raposat regele Solomon cu parin?ii lui ?i a fost înmormântat la un loc cu parin?ii lui în cetatea lui David, tatal sau; iar în locul lui în Ierusalim s-a facut rege Roboam, fiul sau. Roboam era de patruzeci ?i unu de ani când s-a facut rege ?i ?aptesprezece ani a domnit el în Ierusalim. Pe mama lui o chema Naama ?i a fost fiica lui Ana, fiul lui Naha?, regele fiilor Amoni?ilor. ?i a facut fapte ce nu erau placute înaintea ochilor Domnului ?i nici n-a umblat pe caile lui David, stramo?ul sau. Solomon însa avusese un rob, un om din muntele lui Efraim, pe care-l chema Ieroboam, iar pe mama lui o chema ?erua, femeie desfrânata. ?i l-a pus Solomon între dregatorii cei peste lucratorii de corvoada ai casei lui Iosif. ?i a zidit lui Solomon cetatea Tir?a, din muntele lui Efraim. ?i avea sub el trei sute de care cu cai. El a zidit Milo cu lucratorii de corvoada ai casei lui Efraim; tot el a reparat ?i stricaciunile de la cetatea lui David ?i a fost conducator peste regat. ?i a cautat Solomon sa-l omoare. ?i s-a temut Ieroboam ?i a fugit la; ?i?ac, regele Egiptului. ?i a fost la ?i?ac pâna a murit Solomon. ?i a auzit Ieroboam, pe când se afla înca în Egipt, ca a murit Solomon. A ?optit el la urechile lui ?i?ac, regele Egiptului, zicând: "Da-mi drumul, ca sa ma duc în ?ara mea". ?i i-a zis ?i?ac: Cere ce dore?ti ?i î?i voi da. ?i ?i?ac i-a dat lui Ieroboam de femeie pe Ano, sora cea mai mare a Tafnesei, femeia lui. Aceasta era cea mai mare între fiicele regelui, ?i i-a nascut lui Ieroboam pe Abia, fiul lui. ?i a zis Ieroboam lui ?i?ac: "De buna seama, da-mi drumul, ca am sa ma duc". ?i a plecat Ieroboam din Egipt ?i a venit în pamântul Tir?a, din muntele Efraim. ?i s-a adunat acolo toata semin?ia lui Efraim. ?i a zidit Ieroboam acolo tabere. ?i i s-a îmbolnavit copilul lui de o boala foarte grea. ?i s-a dus Ieroboam sa întrebe pentru copil. ?i a zis catre Ano, femeia lui: "Scoala-te ?i du-te de întreaba pe un om al lui Dumnezeu pentru copil daca el are sa scape cu via?a din boala lui". ?i era un om în ?ilo, pe care-l chema Ahia. Acesta era un barbat ca de ?aizeci de ani ?i cuvântul Domnului era cu el. ?i a zis Ieroboam catre femeia lui: "Scoala-te ?i ia în mâna ta pentru omul lui Dumnezeu pâini ?i turte pentru copiii lui ?i struguri ?i un ulcior cu miere". ?i s-a sculat femeia ?i a luat în mâna ei pâini ?i doua turte, struguri ?i un ulcior cu miere pentru Ahia. ?i omul acesta era batrân ?i nu putea sa vada, caci ochii lui i se uscasera de batrâne?e. ?i s-a sculat ?i a plecat din Tir?a. ?i, pe când venea ea în cetate la Ahia ?ilonitul, a zis Ahia catre fiul lui: "Ie?i pentru întâmpinarea Anoei, femeia lui Ieroboam, ?i-i spune: Intra ?i nu sta, caci a?a zice Domnul: Eu sunt trimis la tine ca sa-?i vestesc lucruri grozave! ?i a intrat Ano la omul lui Dumnezeu. ?i i-a zis Ahia: Pentru ce mi-ai adus pâinile, strugurii ?i ulciorul cu miere? A?a zice Domnul: "Iata, tu vei pleca de la mine, ?i când vei intra în Tir?a, în cetate, î?i vor ie?i locuitorii în întâmpinare ?i î?i vor spune: Copilul a murit. Caci a?a zice Domnul: Iata, Eu voi pierde din ai lui Ieroboam pe tot sufletul de parte barbateasca. Cine va muri din ai lui Ieroboam în cetate, pe acela îl vor linge câinii; iar cine va muri în câmp, pe acela pasarile cerului îl vor mânca ?i copilul va fi mult jelit. Vai, Doamne, caci numai în el din casa lui Ieroboam s-a gasit ceva bun înaintea Domnului". ?i a plecat înapoi femeia când a auzit aceasta. ?i când a intrat în Tir?a, copilul a murit. ?i i-a ie?it în întâmpinare ?ipatul cel de jale. ?i s-a dus Ieroboam la Sichem, cetatea cea din muntele lui Efraim. ?i s-au adunat acolo semin?iile lui Israel. ?i a mers acolo ?i Roboam, fiul lui Solomon. ?i cuvântul Domnului a fost catre ?emaia Enlamitul, zicând: "Ia-?i o haina noua, care n-a fost spalata cu apa, ?i o rupe în douasprezece buca?i ?i i le da lui Ieroboam ?i-i spune: "A?a zice Domnul: Ia-?i pentru tine zece buca?i, ca sa te încingi". ?i a luat Ieroboam, ?i a zis ?emaia: "A?a zice Domnul: Sa fii peste cele zece semin?ii ale lui Israel!" ?i a zis poporul catre Roboam, fiul lui Solomon: "Tatal tau a pus jug greu pe noi ?i ?i-a îngreuiat bucatele mesei lui; tu însa acum u?ureaza-l de pe noi ?i-?i vom sluji ?ie!" ?i a zis Roboam catre popor: "Pâna în trei zile eu va voi da raspunsul". ?i a zis Roboam: "Aduce?i-mi pe cei batrâni, ca sa ma sfatuiesc cu ei pentru cuvântul care voi raspunde poporului în ziua a treia". ?i a grait Roboam la urechile lor pentru cuvântul care i-a trimis poporul la el. ?i au zis batrânii poporului pentru acel cuvânt, care i-a grait poporul. ?i a lepadat Roboam sfatul lor, pentru ca nu se potrivea cu parerea lui. ?i a trimis ?i a adunat pe cei ce crescusera împreuna cu el ?i le-a grait: "Acestea ?i acestea a trimis poporul la mine, ca sa-mi spuna". ?i au zis cei ce crescusera împreuna cu el: "A?a vei grai poporului: Degetul meu cel mic este mai gros decât mijlocul tatalui meu. Tatal meu v-a biciuit cu biciul, iar eu va voi bate cu scorpioane". ?i i-a placut lui Roboam acest cuvânt ?i a raspuns el poporului în felul cum l-au sfatuit cei ce crescusera ?i copilarisera împreuna cu el. ?i a zis tot poporul, fiecare cu vecinul sau pâna într-un om, ?i au strigat to?i cu glas mare ?i au zis: "Nici o parte n-avem noi cu David, nici o mo?tenire cu fiul lui Iesei! Fiecare la corturile sale, Israele! Caci omul acesta n-a fost nici capetenie ?i nici conducator peste regat". ?i s-a împra?tiat tot poporul de la Sichem ?i s-a dus fiecare la cortul lui. Iar Roboam, scapând cu fuga, s-a suit în una din caru?ele lui ?i a venit la Ierusalim. ?i l-au urmat numai semin?ia lui Iuda ?i semin?ia lui Veniamin. Iar când a fost începutul anului, Roboam a adunat pe to?i barba?ii lui Iuda ?i ai lui Veniamin ?i a plecat ca sa se bata cu Ieroboam la Sichem. ?i a fost cuvântul Domnului catre ?emaia, omul lui Dumnezeu, zicând: "Spune-i lui Roboam, regele lui Iuda, ?i la toata casa lui Iuda ?i a lui Veniamin ?i la celalalt popor, zicând: A?a zice Domnul, sa nu merge?i, nici sa începe?i razboi cu fra?ii vo?tri, fiii lui Israel, ci sa va întoarce?i fiecare la corturile voastre, ca de la Mine a fost aceasta!" ?i au ascultat ei de cuvântul Domnului ?i s-au oprit de a mai începe razboiul, dupa cuvântul Domnului.
1Ki 12:25  ?i a zidit Ieroboam Sichemul, pe muntele lui Efraim, ?i s-a a?ezat cu locuin?a lui în el; ?i a plecat de acolo ?i a zidit Penuelul.
1Ki 12:26  ?i a zis Ieroboam în sine: "Regatul poate sa treaca iar la casa lui David.
1Ki 12:27  Daca poporul acesta va mai merge la Ierusalim pentru aducere de jertfa în templul Domnului, atunci inima poporului acestuia are sa se întoarca la domnul lor Roboam, regele lui Iuda; pe mine au sa ma ucida, iar ei au sa se întoarca iara?i sub stapânirea lui Roboam, regele lui Iuda".
1Ki 12:28  ?i sfatuindu-se, regele a facut doi vi?ei de aur ?i a zis poporului: "Nu trebuie sa va mai duce?i la Ierusalim; iata Israele dumnezeii tai, care te-au scos din pamântul Egiptului!"
1Ki 12:29  ?i a pus unul în Betel, iar pe celalalt în Dan.
1Ki 12:30  Însa fapta aceasta a dus la pacat, caci poporul a început sa mearga pentru a se închina unuia din ei, pâna la Dan, ?i a parasit templul Domnului.
1Ki 12:31  ?i a zidit el ?i capi?ti pe înal?imi ?i a facut preo?i, lua?i din popor, care nu erau fii ai lui Levi.
1Ki 12:32  ?i a a?ezat Ieroboam ?i o sarbatoare în luna a opta, în ziua a cincisprezecea a lunii, asemenea cu sarbatoarea care era în Iuda, ?i a adus jertfe pe jertfelnic. Tot a?a a facut el ?i la Betel, ca sa aduca jertfe vi?eilor pe care-i facuse. ?i a a?ezat la Betel preo?i pentru înal?imile facute de el.
1Ki 12:33  ?i a adus jertfe pe jertfelnicul pe care l-a facut la Betel în ziua a cincisprezecea din luna a opta, luna pe care ?i-o alesese el dupa placul lui pentru sarbatorire. ?i a facut sarbatoare pentru fiii lui Israel ?i s-a apropiat de jertfelnic, ca sa savâr?easca tamâiere.
1Ki 13:1  Iata însa un om al lui Dumnezeu a venit, dupa cuvântul Domnului, din Iuda la Betel, în timpul când Ieroboam se afla la jertfelnic, ca sa tamâieze.
1Ki 13:2  ?i a grait cuvântul Domnului înaintea jertfelnicului ?i a zis: "Jertfelnice, jertfelnice, a?a zice Domnul: Iata ca i se va na?te casei lui David un fiu, numele lui, Iosia, ?i va jertfi pe tine pe preo?ii înal?imilor care tamâiaza acum înaintea ta ?i va arde pe tine oase de oameni!"
1Ki 13:3  ?i a aratat în acea zi ?i un semn zicând: "Iata semnul dupa care se va cunoa?te ca Domnul a grait aceasta: Jertfelnicul acesta se va despica, ?i cenu?a care este pe el se va împra?tia!"
1Ki 13:4  Când regele Ieroboam a auzit cuvântul omului lui Dumnezeu pe care l-a grait în gura mare în fala jertfelnicului de la Betel, ?i ?i-a întins mâna lui de la jertfelnic, zicând: "Pune?i mâna pe el!", i-a înlemnit mâna care o întinsese asupra lui ?i nu putea sa o întoarca înapoi.
1Ki 13:5  ?i jertfelnicul s-a despicat ?i cenu?a de pe jertfelnic s-a împra?tiat, dupa semnul care l-a dat omul lui Dumnezeu prin cuvântul Domnului.
1Ki 13:6  Atunci a zis regele Ieroboam catre omul lui Dumnezeu: "Îmblânze?te fa?a Domnului Dumnezeului tau ?i roaga-te pentru mine, ca sa mi se poata întoarce mâna mea la mine". ?i a îmblânzit omul lui Dumnezeu fala Domnului ?i mâna regelui s-a întors la el ?i s-a facut ca ?i înainte.
1Ki 13:7  ?i a grait regele catre omul lui Dumnezeu: "Vino la mine acasa ?i prânze?te cu mine ?i-?i voi da un dar!"
1Ki 13:8  Însa omul lui Dumnezeu a zis regelui: "Macar sa-mi dai tu ?i jumatate din casa ta, eu nu voi merge la tine, nici pâine nu voi mânca ?i nici apa nu voi bea în acest loc.
1Ki 13:9  Caci a?a mi s-a poruncit prin cuvântul Domnului: Sa nu manânci acolo pâine, apa sa nu bei, nici sa te întorci pe drumul pe care te-ai dus!"
1Ki 13:10  ?i a plecat el pe alt drum ?i nu s-a întors înapoi pe drumul pe care venise la Betel.
1Ki 13:11  În Betel traia atunci un prooroc batrân; ?i au venit fiii sai ?i i-au istorisit tot ce a facut omul lui Dumnezeu în acea zi în Betel; de asemenea au istorisit ei tatalui lor ?i cuvintele care le-a grait el regelui.
1Ki 13:12  ?i i-a întrebat tatal lor: "Pe ce drum a apucat el?" ?i fiii lui i-au aratat drumul pe care a apucat omul lui Dumnezeu, care venise din Iuda.
1Ki 13:13  ?i a zis el fiilor sai: "Pune?i-mi ?aua pe asin". ?i i-a pus ?aua pe asin ?i a încalecat pe el.
1Ki 13:14  ?i a plecat dupa omul lui Dumnezeu ?i l-a gasit ?ezând sub un stejar ?i i-a zis: "Tu e?ti omul lui Dumnezeu venit din Iuda?" El a zis: "Da, eu sunt!"
1Ki 13:15  ?i i-a zis lui: "Hai cu mine acasa, ca sa manânci pâine!
1Ki 13:16  Acela a zis: "Nu pot sa ma întorc, nici sa merg la tine, ca eu pâine nu voi mânca ?i nici apa nu voi bea în acest loc la tine,
1Ki 13:17  Caci prin cuvântul Domnului mi s-a spus: "Sa nu manânci pâine, sa nu bei acolo apa ?i nici sa te întorci pe drumul pe care te-ai dus!"
1Ki 13:18  ?i i-a zis el: "?i eu sunt prooroc ca tine; ?i îngerul mi-a grait prin cuvântul Domnului ?i a zis: Întoarce-l la tine acasa, ca sa manânce pâine ?i sa bea apa!"
1Ki 13:19  ?i l-a în?elat, caci s-a întors cu el ?i a mâncat pâine ?i a baut apa în casa lui.
1Ki 13:20  Pe când ei înca ?edeau la masa, cuvântul Domnului a fost catre proorocul cel întors din drum ?i a grait omului lui Dumnezeu venit din Iuda
1Ki 13:21  ?i a zis: "A?a zice Domnul: Fiindca tu nu te-ai supus cuvintelor Domnului ?i nici n-ai pazit porunca data de Domnul Dumnezeul tau,
1Ki 13:22  Ci te-ai întors, ai mâncat pâine ?i ai baut apa în locul de care El ?i-a spus: Sa nu manânci pâine, nici sa bei apa, trupul tau nu va fi îngropat în mormântul parin?ilor tai!",
1Ki 13:23  Iar dupa ce el a mâncat pâine ?i a baut apa, proorocul batrân a pus ?aua pe asinul sau pentru proorocul care fusese întors.
1Ki 13:24  ?i a plecat acela ?i l-a întâlnit un leu în drum ?i l-a omorât. ?i zacea trupul lui, aruncat în drum; iar asinul ?i leul stateau lânga el.
1Ki 13:25  ?i iata ni?te drume?i, care au trecut pe alaturi, i-au vazut trupul aruncat în drum ?i pe leu stând lânga trup. Ace?tia au venit ?i au vestit ceta?ii în care traia proorocul cel batrân.
1Ki 13:26  Proorocul care l-a întors de pe drum, auzind acestea, a zis: "Acesta este omul lui Dumnezeu, acela care nu s-a supus cuvintelor Domnului; Domnul l-a dat leului care l-a sfâ?iat ?i l-a omorât, dupa cuvântul Domnului, pe care l-a grait pentru el".
1Ki 13:27  ?i a zis fiilor sai: "Pune?i ?aua pe asin!" ?i i-au pus ei ?aua pe asin.
1Ki 13:28  ?i a plecat ?i a gasit trupul lui, aruncat în drum, iar asinul ?i leul stateau lânga trup; leul nu mâncase trupul omului lui Dumnezeu ?i nici pe asin nu-l sfâ?iase.
1Ki 13:29  ?i a ridicat proorocul trupul omului lui Dumnezeu ?i l-a pus pe asin ?i l-a adus înapoi. ?i a venit proorocul cel batrân în cetatea sa, ca sa-l plânga ?i sa-l înmormânteze.
1Ki 13:30  ?i i-a pus trupul lui într-un mormânt al sau ?i l-a plâns, zicând: "Vai, vai, fratele meu!"
1Ki 13:31  Iar dupa ce l-a înmormântat, a zis el fiilor sai: "Când eu voi muri, sa ma înmormânta?i ?i pe mine tot în mormântul în care este înmormântat omul lui Dumnezeu; sa pune?i oasele mele lânga oasele lui;
1Ki 13:32  Caci cu adevarat se va împlini cuvântul, care l-a grait el din porunca Domnului pentru jertfelnicul din Betel ?i pentru toate capi?tele înal?imilor, care sunt în ceta?ile Samariei".
1Ki 13:33  Dar Ieroboam nu s-a întors din calea lui cea rea nici dupa întâmplarea aceasta, ci a venit iara?i la calea dinainte ?i a facut din poporul de rând preo?i pentru înal?imi; pe cine vrea, pe acela î?i punea mâna lui ?i-l facea pe acela preot de înal?imi.
1Ki 13:34  Aceasta a dus casa lui Ieroboam la pacat ?i la pieire ?i la stârpirea ei de pe fa?a pamântului.
1Ki 14:1  În acel timp s-a îmbolnavit Abia, fiul lui Ieroboam.
1Ki 14:2  ?i a zis Ieroboam femeii sale: "Scoala-te ?i te schimba de haine, ca sa nu te cunoasca nimeni ca tu e?ti femeia lui Ieroboam, ?i du-te la ?ilo. Acolo este proorocul Ahia; el mi-a spus ca voi fi rege peste acest popor.
1Ki 14:3  Ia cu tine pentru omul lui Dumnezeu: zece pâini, turte ?i un ulcior cu miere ?i du-te la el, ca el î?i va spune ce se va întâmpla cu copilul acesta".
1Ki 14:4  Femeia lui Ieroboam a?a a ?i facut; s-a sculat ?i s-a dus la ?ilo ?i a tras la casa lui Ahia. Acum însa Ahia nu putea sa vada, caci ochii i se întunecasera de batrâne?e.
1Ki 14:5  ?i a zis Domnul lui Ahia: "Iata, vine femeia lui Ieroboam, ca sa te întrebe de fiul ei, ca este bolnav; a?a ?i a?a îi vei spune; ea vine schimbata de haine".
1Ki 14:6  Ahia auzind zgomotul pa?ilor ei, când a intrat pe u?a, a zis: "Intra femeie a lui Ieroboam. Ce gând ai avut de n-ai schimbat hainele? Eu sunt trimis sa-?i vestesc lucruri grozave.
1Ki 14:7  Du-te ?i spune-i lui Ieroboam: A?a zice Domnul Dumnezeul lui Israel: Te-am ridicat din mijlocul poporului de jos ?i te-am pus conducator peste poporul Meu Israel.
1Ki 14:8  ?i am rupt regatul de la casa lui David ?i Zi l-am dat ?ie; iar tu nu te-ai purtat a?a cum s-a purtat robul Meu David, care a pazit poruncile Mele ?i a umblat dupa Mine cu toata inima lui, facând numai lucrurile care sunt placute înaintea ochilor Mei.
1Ki 14:9  Tu însa ai întrecut în rautate pe to?i care au fost înaintea ta, caci te-ai dus ?i ?i-ai facut al?i dumnezei ?i chipuri turnate, ca sa Ma întarâ?i la mânie, iar pe Mine M-ai aruncat înapoia ta.
1Ki 14:10  Pentru aceasta voi aduce necazuri peste casa lui Ieroboam ?i voi stârpi din casa lui Ieroboam pe to?i cei de parte barbateasca, rob sau liber în Israel, ?i voi cura?a casa lui Ieroboam cum se matura gunoiul, de ramâne curat.
1Ki 14:11  Cine va muri dintre cei care sunt ai lui Ieroboam în cetate, pe acela câinii îl vor mânca; cine va muri în câmp, pe acela pasarile cerului îl vor ciuguli, pentru ca a?a a grait Domnul.
1Ki 14:12  Scoala-te dar, ?i du-te la casa ta; ?i îndata ce piciorul tau va calca în cetate, copilul va muri.
1Ki 14:13  ?i-l vor plânge ton Israeli?ii ?i-l vor înmormânta; caci el singur dintre acei care sunt ai lui Ieroboam va fi îngropat în mormânt, fiindca numai în el, din casa lui Ieroboam, s-a gasit ceva bun înaintea Domnului Dumnezeului lui Israel.
1Ki 14:14  ?i î?i va a?eza Domnul un rege peste Israel care va nimici casa lui Ieroboam în acea zi. Dar când? Chiar acum începe necazul.
1Ki 14:15  ?i va lovi Domnul pe Israel ?i va fi el ca trestia, care se clatina în apa, ?i va azvârli pe Israeli?i din acest pamânt bun pe care l-a dat parin?ilor lor ?i-i va spulbera dincolo de Eufrat, pentru ca ei ?i-au facut dumbravi închinate Astartei, întarâtând la mânie pe Domnul.
1Ki 14:16  Domnul va vinde pe Israel pentru pacatele lui Ieroboam pe care el însu?i le-a facut ?i cu care a dus la pacat pe Israel".
1Ki 14:17  Atunci s-a sculat femeia lui Ieroboam ?i a plecat ?i a venit la Tir?a ?i când a pa?it ea cu piciorul peste pragul casei, copilul a murit.
1Ki 14:18  ?i l-au înmormântat ?i l-au jelit to?i Israeli?ii, dupa cuvântul Domnului, încredin?at robului Sau Ahia proorocul.
1Ki 14:19  Celelalte fapte ale lui Ieroboam, cum a purtat razboaie ?i cum a domnit, sunt scrise în cronica regilor lui Israel.
1Ki 14:20  Timpul domniei lui Ieroboam a fost de douazeci ?i doi de ani; ?i a adormit cu parin?ii lui, iar în locul lui s-a facut rege Nadab, fiul lui.
1Ki 14:21  Iar în Iuda domnea Roboam, fiul lui Solomon. Roboam, când s-a facut rege, era ca de patruzeci ?i unu de ani; ?i a domnit ?aptesprezece ani în Ierusalim, în cetatea pe care a ales-o Domnul din toate semin?iile lui Israel, ca sa petreaca numele Lui acolo. Pe mama lui o chema Naama Amonita.
1Ki 14:22  ?i a facut ?i Iuda lucruri foarte rele înaintea ochilor Domnului ?i L-au mâniat cu pacatele lor, pe care le-au facut mai mult decât parin?ii lor;
1Ki 14:23  Caci ?i ace?tia ?i-au facut înal?imi, idoli ?i Astarte, pe orice deal înalt ?i sub orice copac umbros.
1Ki 14:24  ?i erau de asemenea ?i sodomi?i în aceasta ?ara, ?i faceau toate ticalo?iile pagânilor, pe care Domnul îi gonise din fa?a fiilor lui Israel.
1Ki 14:25  Iar în anul al cincilea al domniei lui Roboam, ?i?ac, regele Egiptului, a pornit razboi asupra Ierusalimului,
1Ki 14:26  ?i a luat vistieriile templului Domnului ?i vistieriile casei regelui ?i scuturile cele de aur pe care le luase David din mâna robilor lui Hadad Ezer, regele din ?oba, ?i le adusese în Ierusalim; toate le-a luat ?i a pradat toate scuturile de aur pe care le facuse Solomon.
1Ki 14:27  ?i a facut regele Roboam în locul lor scuturi de arama ?i le-a dat în mâinile capitanilor de garzi, care pazeau intrarea casei regelui.
1Ki 14:28  Numai când regele mergea la templul Domnului, numai atunci garda le purta; ?i pe urma le ducea iara?i în casa de garda.
1Ki 14:29  Celelalte fapte ale lui Roboam ?i tot ce el a facut sunt scrise în cronica regilor lui Iuda.
1Ki 14:30  Între Roboam ?i Ieroboam a fost razboi în toate zilele vie?ii lor.
1Ki 14:31  ?i a adormit cu parin?ii lui ?i a fost înmormântat Roboam la un loc cu parin?ii lui în cetatea lui David. Pe mama lui o chema Naama Amonita. Iar în locul lui s-a facut rege Abia, fiul sau.
1Ki 15:1  Abia s-a facut rege peste Iuda, în anul al optsprezecelea al domniei lui Ieroboam, fiul lui Nabat.
1Ki 15:2  ?i a domnit trei ani în Ierusalim. Pe mama lui o chema Maaca ?i era fiica lui Abesalom.
1Ki 15:3  Acesta a umblat în toate pacatele tatalui sau, pe care tatal sau le facuse mai înainte, ?i inima lui nu i-a fost întreaga la Domnul Dumnezeul lui, cum a fost inima lui David, stramo?ul sau.
1Ki 15:4  Dar din dragostea catre David, Domnul Dumnezeul lui i-a dat o lumina în Ierusalim, caci a înal?at dupa el pe fiul sau ?i a întarit Ierusalimul,
1Ki 15:5  Pentru ca David a facut tot ce a fost placut înaintea ochilor Domnului ?i nu s-a abatut în toate zilele vie?ii lui de la poruncile pe care El i le-a dat, afara de fapta cu Urie Heteul.
1Ki 15:6  ?i a fost razboi între Roboam ?i Ieroboam în toate zilele vie?ii lor.
1Ki 15:7  Celelalte fapte ale lui Abia ?i tot ce a facut el sunt scrise în cartea cronicilor regilor lui Iuda. ?i a fost razboi între Abia ?i Ieroboam.
1Ki 15:8  Apoi a trecut Abia la parin?ii lui ?i l-au înmormântat în cetatea lui David. ?i s-a facut rege Asa, fiul sau.
1Ki 15:9  Asa a început sa domneasca peste Iuda în al douazecilea an al domniei lui Ieroboam, regele lui Israel.
1Ki 15:10  ?i a domnit patruzeci ?i unu de ani în Ierusalim. Pe mama lui o chema Ana, din neamul lui Abesalom.
1Ki 15:11  Asa a facut lucruri placute înaintea ochilor Domnului, ca David, stramo?ul lui;
1Ki 15:12  Caci a izgonit pe desfrâna?i din ?ara ?i a îndepartat to?i idolii pe care îi facusera parin?ii lui;
1Ki 15:13  Ba a lipsit chiar pe mama lui, Ana, de numele de regina, pentru ca ea facuse un chip turnat Astartei. A?a a sfarâmat Asa chipul cel turnat al Astartei arzându-l pe prundul pârâului Chedron.
1Ki 15:14  Dar înal?imile nu le-a stricat. Însa inima lui Asa i-a fost întreaga la Domnul în toate zilele vie?ii sale.
1Ki 15:15  ?i a adus el în templul Domnului lucrurile afierosite de tatal lui ?i lucrurile afierosite de el: argint, aur ?i vase sfinte.
1Ki 15:16  Razboi a fost între Asa ?i Bae?a, regele lui Israel, în toate zilele lor.
1Ki 15:17  Caci Bae?a, regele lui Israel, a venit în contra Iudei ?i a început sa zideasca Rama, pentru ca nimeni sa nu iasa sau sa intre la Asa, regele Iudei.
1Ki 15:18  Atunci Asa a luat tot argintul ?i aurul care se mai afla în vistieriile templului Domnului ?i în vistieriile casei regelui ?i le-a dat în mina slugilor lui; ?i i-a trimis regele Asa ca sa se duca cu ele la Benhadad, fiul lui Tabrimon, fiul lui Hezion, regele Siriei, care traia în Damasc, ?i a zis:
1Ki 15:19  "Legamânt sa fie între mine ?i tine, cum a fost între tatal meu ?i tatal tau; iata, eu î?i trimit ca dar argint ?i aur; ?i tu sa rupi legamântul tau care îl ai cu Bae?a, regele lui Israel, ca sa se retraga acela de la mine!"
1Ki 15:20  ?i a ascultat Benhadad de regele Asa ?i a trimis pe mai-marii o?tirii lui asupra ceta?ilor lui Israel, ?i a batut Ainul, Danul, Abel, Bet-Maaca ?i tot Chineretul, dimpreuna cu tot pamântul lui Neftali.
1Ki 15:21  Iar Bae?a, cum a auzit de aceasta, a încetat sa mai zideasca Rama ?i s-a întors la Tir?a.
1Ki 15:22  Atunci regele Asa a chemat pe to?i Iudeii, fara sa scuteasca pe nimeni, ?i a carat cu ei pietraria ?i lemnaria din Rama, pe care Bae?a le întrebuin?ase la zidire; ?i a zidit regele Asa cu ele Ghibeea lui Veniamin ?i Mi?pa.
1Ki 15:23  Toate celelalte fapte ale lui Asa, ?i toate ostenelile lui, ?i tot ce el a facut ?i ceta?ile pe care le-a zidit sunt scrise în cronica regilor lui Iuda, afara de faptul ca la batrâne?ea lui s fost bolnav de picioare.
1Ki 15:24  Apoi a adormit cu parin?ii lui ?i a fost înmormântat Asa la un loc cu parin?ii lui în cetatea lui David, stramo?ul sau. Iar în locul lui s-a facut rege Iosafat, fiul sau.
1Ki 15:25  Nadab însa, fiul lui Ieroboam a început sa domneasca peste Israeli?i în anul al doilea al lui Asa, regele Iudei, ?i a domnit peste Israel doi ani.
1Ki 15:26  Acesta a savâr?it fapte neplacute înaintea ochilor Domnului, caci a umblat pe drumul tatalui sau, cum ?i în pacatele lui cu care a facut pe Israel sa pacatuiasca.
1Ki 15:27  Împotriva lui a uneltit Bae?a, fiul lui Ahia, din casa lui Isahar; Bae?a l-a omorât la Ghibeton, cetatea Filistenilor, pe când Nadab ?i to?i Israeli?ii împresurasera Ghibetonul.
1Ki 15:28  Bae?a l-a omorât în anul al treilea al lui Asa, regele Iudei, ?i s-a facut rege în locul lui.
1Ki 15:29  Acesta, cum s-a facut rege, a stârpit toata casa lui Ieroboam ?i n-a lasat nici un suflet din neamul lui Ieroboam, pâna ce nu l-a nimicit, dupa cuvântul Domnului, pe care-l graise prin robul Sau Ahia din ?ilo,
1Ki 15:30  Din pricina pacatelor pe care Ieroboam le facuse ?i cu care a facut pe Israel sa pacatuiasca ?i pentru faradelegea cu care el a mâniat pe Domnul Dumnezeul lui Israel.
1Ki 15:31  Celelalte fapte ale lui Nadab ?i tot ce el a facut sunt scrise în cronica regilor lui Israel.
1Ki 15:32  Între Asa ?i Bae?a, regele lui Israel, a fost razboi în toate zilele vie?ii lor.
1Ki 15:33  Bae?a, fiul lui Ahia, a început sa domneasca peste to?i Israeli?ii la Tir?a, în anul al treilea al lui Asa, regele lui Iuda, ?i a domnit douazeci ?i patru de ani.
1Ki 15:34  Acesta a savâr?it fapte rele înaintea ochilor Domnului caci a umblat pe drumul lui Ieroboam ?i în pacatele lui, cu care acesta a dus pe Israel în pacat.
1Ki 16:1  Atunci a fost cuvântul Domnului catre Iehu, fiul lui Hanani, pentru Bae?a, zicând:
1Ki 16:2  "Pentru ca Eu te-am scuturat de ?arâna ?i te-am facut domn peste poporul Meu Israel, iar tu ai umblat pe drumul lui Ieroboam ?i ai dus pe poporul Meu Israelit la pacat, ca sa Ma mânii cu pacatele lor,
1Ki 16:3  Iata, Eu voi darâma de tot casa lui Bae?a ?i casa urma?ilor lui ?i voi face cu casa ta ce am facut cu casa lui Ieroboam, fiul lui Nabat;
1Ki 16:4  Cine va muri din neamul lui Bae?a în cetate, pe acela câinii îl vor mânca, iar cine va muri în câmp, pe acela pasarile cerului îl vor mânca".
1Ki 16:5  Celelalte fapte ale lui Bae?a, tot ce a facut el ?i razboaiele lui, sunt scrise în cronica regilor lui Israel.
1Ki 16:6  ?i a adormit cu parin?ii sai ?i a fost înmormântat Bae?a în Tiria. În locul lui s-a facut rege Ela, fiul sau.
1Ki 16:7  Dar despre Bae?a, despre casa lui ?i despre tot raul ce l-a facut înaintea ochilor Domnului, mâniindu-L cu faptele mâinilor sale ?i urmând casei lui Ieroboam, pentru care acesta a ?i fost stârpit, fusese cuvântul Domnului prin Iehu, fiul lui Hanani.
1Ki 16:8  Ela, fiul lui Bae?a, a început sa domneasca peste Israel în Tir?a, în anul al douazeci ?i ?aselea al lui Asa, regele Iudei, ?i a domnit doi ani.
1Ki 16:9  Împotriva lui a uneltit robul sau Zimri, care era mai mare peste jumatate din carele de razboi.
1Ki 16:10  Când a baut acesta de s-a îmbatat la Tir?a, în casa lui Ar?a, mai marele cur?ii din Tir?a, atunci a intrat Zimri, l-a lovit ?i l-a omorât, în anul al douazeci ?i ?aptelea al lui Asa, regele Iudei, ?i s-a facut rege în locul lui.
1Ki 16:11  Acesta, cum s-a facut rege ?i s-a a?ezat pe tronul lui, a stârpit toata casa lui Bae?a, nelasând nici picior de barbat, nici din rudele lui, nici din prietenii lui.
1Ki 16:12  ?i a stârpit Zimri toata casa lui Bae?a, dupa cuvântul Domnului grait pentru Bae?a, prin Iehu proorocul,
1Ki 16:13  Pentru toate faradelegile lui Bae?a ?i pentru cele ale lui Ela, fiul sau, pe care le-a savâr?it, facând pe Israel sa pacatuiasca, ?i mâniind cu idolii lor pe Domnul Dumnezeul lui Israel.
1Ki 16:14  Celelalte fapte ale lui Ela, tot ce a facut el sunt scrise în cronica regilor lui Israel.
1Ki 16:15  Zimri s-a facut rege în anul al douazeci ?i ?aptelea al lui Asa, regele Iudei, ?i a domnit ?apte zile în Tiria, când poporul se gasea în tabere la Ghibeton, cetatea Filistenilor.
1Ki 16:16  Când poporul care se gasea în tabere a auzit ca Zimri a uneltit ?i a omorât pe rege, to?i Israeli?ii au ales ?i ei rege pe Omri, mai-marele o?tirii lui Israel, tot în aceea?i zi, în tabara.
1Ki 16:17  ?i a plecat Omri cu to?i Israeli?ii de la Ghibeton, împresurând Tiria.
1Ki 16:18  Când a auzit Zimri ca cetatea este luata, s-a dus în odaia din fund a casei domne?ti ?i a dat foc casei domne?ti, în care era, ?i a pierit
1Ki 16:19  În pacatele sale, caci savâr?ise fapte rele înaintea ochilor Domnului, umblând pe calea lui Ieroboam ?i în faradelegile lui, pe care acela le faptuise ?i cu care a dus pe Israel la pacat.
1Ki 16:20  Celelalte fapte ale lui Zimri ?i uneltirile lui pe care le-a urzit sunt scrise în cronica regilor lui Israel.
1Ki 16:21  Atunci s-a împar?it poporul lui Israel în doua: jumatate de popor era pentru Tibni, fiul lui Ghinat, ca sa-l faca rege ?i jumatate pentru Omri.
1Ki 16:22  ?i cei care urmau pe Omri au biruit pe cei care urmau pe Tibni, fiul lui Ghinat. ?i a murit Tibni ?i Ioram, fratele lui, tot în acel timp; în locul lui Tibni s-a facut rege Omri.
1Ki 16:23  Omri a început sa domneasca peste Israel în anul treizeci ?i unu al lui Asa, regele Iudei; ?i a domnit doisprezece ani. ?ase ani a stat în Tir?a.
1Ki 16:24  ?i a cumparat Omri muntele Samariei de la ?emer, cu doi talan?i de argint ?i a zidit pe acest munte o cetate ?i a numit cetatea pe care o zidise Samaria, dupa numele lui ?emer, stapânul muntelui Samariei.
1Ki 16:25  Omri a savâr?it fapte rele înaintea ochilor Domnului ?i a fost mai nelegiuit decât to?i cei dinaintea lui.
1Ki 16:26  El a umblat în totul pe caile lui Ieroboam, fiul lui Nabat ?i întru faradelegile eu care acesta a dus pe Israel la pacat, mâniind pe Domnul Dumnezeul lui Israel cu idolii lor.
1Ki 16:27  Celelalte fapte ale lui Omri, pe care le-a facut ?i vitejia în razboaie sunt scrise în cronica regilor lui Israel.
1Ki 16:28  ?i a adormit Omri cu parin?ii lui ?i a fost înmormântat în Samaria. În locul lui s-a facut rege Ahab, fiul sau.
1Ki 16:29  Ahab, fiul lui Omri, a început sa domneasca peste Israel în anul al treizeci ?i optulea al lui Asa, regele lui Iuda; ?i a domnit Ahab, fiul lui Omri, peste Israel în Samaria douazeci ?i doi de ani.
1Ki 16:30  ?i a savâr?it Ahab, fiul lui Omri, fapte rele înaintea ochilor Domnului, mai mult decât to?i cei ce au fost înaintea lui.
1Ki 16:31  Caci nu i-a fost de ajuns sa cada numai în pacatele lui Ieroboam, fiul lui Nabat; ci daca ?i-a luat de femeie pe Izabela, fiica lui Etbaal, regele Sidonului, a început sa slujeasca lui Baal ?i sa i se închine.
1Ki 16:32  ?i a ridicat pentru Baal un jertfelnic în templul lui Baal, pe care îl zidise în Samaria.
1Ki 16:33  A facut Ahab ?i o A?era (stâlp facut din lemn, sfin?it în cinstea zei?ei Astarte), încât Ahab, mai mult decât to?i regii lui Israel, care au fost înaintea lui, a savâr?it faradelegi, prin care a mâniat pe Domnul Dumnezeul lui Israel ?i ?i-a pierdut suflatul sau.
1Ki 16:34  În zilele lui, Hiel din Betel a zidit Ierihonul; temelia a pus-o pe mormântul lui Abiram, întâiul nascut al lui, iar por?ile le-a pus pe mormântul lui Segub, feciorul cel mai mic al lui, dupa cuvântul Domnului pe care-l graise prin Iosua, fiul lui Navi.
1Ki 17:1  Atunci Ilie Tesviteanul, prooroc din Tesba Galaadului, a zis catre Ahab: "Viu este Domnul Dumnezeul lui Israel, înaintea Caruia slujesc au; în ace?ti ani nu va fi nici roua, nici ploaie decât numai când voi zice eu!"
1Ki 17:2  ?i a zis Domnul catre Ilie:
1Ki 17:3  "Du-te de aici, îndreapta-te spre rasarit ?i te ascunde la pârâul Cherit, care este în fa?a Iordanului.
1Ki 17:4  Apa vei bea din acel pârâu, iar mâncare am poruncit corbilor sa-?i aduca acolo!"
1Ki 17:5  ?i a plecat Ilie ?i a facut dupa cuvântul Domnului; s-a dus ?i a ?ezut la pârâul Cherit, care este în fa?a Iordanului.
1Ki 17:6  Corbii îi aduceau pâine ?i carne diminea?a, pâine ?i carne seara; iar apa bea din pârâu.
1Ki 17:7  Dupa o vreme pârâul a secat, nemaifiind ploaie pe pamânt.
1Ki 17:8  Atunci a fost cuvântul Domnului catre Ilie, zicând:
1Ki 17:9  "Scoala ?i du-te la Sarepta Sidonului ?i ?ezi acolo, caci iata am poruncit unei femei vaduve sa te hraneasca!"
1Ki 17:10  ?i s-a sculat el ?i s-a dus la Sarepta. ?i când a ajuns la por?ile ceta?ii, iata o femeie vaduva aduna vreascuri ?i a chemat-o Ilie ?i i-a zis: "Adu-mi pu?ina apa ca sa beau! "
1Ki 17:11  ?i s-a dus ca sa-i aduca, dar Ilie a strigat-o ?i i-a zis: "Adu-mi ?i o bucata de pâine sa manânc!"
1Ki 17:12  Ea însa a zis: "Viu este Domnul Dumnezeul tau, n-am nici o farâmitura de pâine, ci numai o mâna de faina într-un vas ?i pu?in untdelemn într-un urcior. ?i iata, am adunat câteva vreascuri ?i ma duc sa o gatesc pentru mine ?i pentru fiul meu ?i apoi sa mâncam ?i sa murim!"
1Ki 17:13  Atunci i-a zis Ilie: "Nu te teme, ci du-te ?i fa cum ai zis; dar fa mai întâi de acolo o turta pentru mine ?i adu-mi-o, iar pentru tine ?i pentru fiul tau vei face mai pe urma.
1Ki 17:14  Caci a?a zice Domnul Dumnezeul lui Israel: Faina din vas nu va scadea ?i untdelemnul din urcior nu se va împu?ina pâna în ziua când va da Domnul ploaie pe pamânt!"
1Ki 17:15  ?i s-a dus ea ?i a facut a?a, cum i-a zis Ilie; ?i s-a hranit ea ?i el ?i casa ei o bucata de vreme.
1Ki 17:16  Caci faina din vas n-a scazut ?i untdelemnul din urcior nu s-a împu?inat, dupa cuvântul Domnului, grait prin Ilie.
1Ki 17:17  Dupa aceasta s-a îmbolnavit copilul femeii, stapâna casei, ?i boala lui a fost atât de grea, ca n-a mai ramas suflare într-însul.
1Ki 17:18  ?i a zis ea catre Ilie: "Ce ai avut cu mine, omul lui Dumnezeu? Ai venit la mine ca sa-mi pomene?ti pacatele mele ?i sa-mi omori fiul?"
1Ki 17:19  Iar Ilie a zis: "Da-mi pe fiul tau!" ?i l-a luat din bra?ele ei ?i l-a suit în foi?or unde ?edea el ?i l-a pus pe patul sau.
1Ki 17:20  Apoi a strigat Ilie catre Domnul ?i a zis: "Doamne Dumnezeul meu, oare ?i vaduvei la care locuiesc îi faci rau, omorând pe fiul ei?"
1Ki 17:21  ?i suflând de trei ori peste copil, a strigat catre Domnul ?i a zis: "Doamne Dumnezeul meu, sa se întoarca sufletul acestui copil în el!"
1Ki 17:22  ?i a ascultat Domnul glasul lui Ilie; ?i s-a întors sufletul copilului acestuia în el ?i a înviat.
1Ki 17:23  ?i a luat Ilie copilul ?i s-a coborât cu el din foi?or în casa ?i l-a dat mamei sale ?i a zis Ilie: "Iata copilul tau este viu!"
1Ki 17:24  Atunci a zis femeia catre Ilie: "Acum cunosc ?i eu ca tu e?ti omul lui Dumnezeu ?i cu adevarat cuvântul lui Dumnezeu este în gura ta!"
1Ki 18:1  Dupa trecere de mai multe zile, a fost cuvântul Domnului catre Ilie în anul al treilea, zicând: "Du-te ?i te arata lui Ahab ?i Eu voi da ploaie pe pamânt!"
1Ki 18:2  ?i a plecat Ilie sa se arate lui Ahab, în Samaria fiind foamete mare.
1Ki 18:3  Atunci a chemat Ahab pe Obadia, mai-marele cur?ii ?i Obadia era un om foarte temator de Dumnezeu;
1Ki 18:4  Caci când Izabela a ucis pe proorocii Domnului, Obadia a luat o suta de prooroci ?i i-a ascuns: cincizeci într-o pe?tera ?i cincizeci în alta ?i i-a hranit cu pâine ?i cu apa.
1Ki 18:5  ?i a zis Ahab catre Obadia: "Du-te prin ?ara la toate izvoarele de apa ?i pe la toate pâraiele, poate sa gasim undeva iarba, ca sa ne hranim caii, catârii ?i sa nu prapadim vitele".
1Ki 18:6  Atunci ?i-au împar?it între ei ?ara ca s-o cutreiere: Ahab a apucat singur pe un drum, iar Obadia a apucat singur pe altul.
1Ki 18:7  Când Obadia mergea pe drum, iata i-a ie?it înainte Ilie. Obadia l-a cunoscut ?i a cazut cu fa?a la pamânt, zicând: "Oare tu e?ti domnul meu, Ilie?"
1Ki 18:8  Iar acela i-a zis: "Eu sunt; du-te ?i spune domnului tau: Ilie este aici!"
1Ki 18:9  Obadia însa a zis: "Cu ce am gre?it eu de dai pe robul tau pe mâna lui Ahab, ca sa ma omoare?
1Ki 18:10  Viu este Domnul Dumnezeul tau! Nu este popor ?i împara?ie la care sa nu fi trimis domnul meu ca sa te caute. ?i când i s-a spus ca nu e?ti acolo, a pus pe acea împara?ie ?i pe acel popor sa jure ca nu te-a gasit pe tine.
1Ki 18:11  Iar acum zici: Du-te ?i spune domnului tau: Ilie este aici!
1Ki 18:12  Gând voi pleca de la tine, Duhul Domnului are sa te duca cine ?tie unde ?i daca ma voi duce ?i voi spune lui Ahab de tine ?i apoi el nu te va gasi, ma va ucide pe mine; iar robul tau este temator de Dumnezeu din tinere?ile mele.
1Ki 18:13  Oare nu ?i s-a spus, domnul meu, ce am facut eu când Izabela a ucis pe proorocii Domnului, în ce chip am ascuns o suta de oameni, prooroci ai Domnului: cincizeci de in?i într-o pe?tera ?i cincizeci de in?i în alta ?i acolo i-am hranit cu pâine ?i cu apa?
1Ki 18:14  Tu zici: "Du-te, spune domnului tau: "Ilie este aici". "Vrei sa ma ucida?"
1Ki 18:15  ?i a zis Ilie: "Viu este Domnul Savaot Caruia Îi slujesc. Astazi ma voi arata lui Ahab!"
1Ki 18:16  ?i a plecat Obadia în întâmpinarea lui Ahab ?i i-a spus de aceasta. ?i a venit Ahab în întâmpinarea lui Ilie.
1Ki 18:17  Când a vazut Ahab pe Ilie, i-a zis: "Tu e?ti oare cel ce aduci nenorociri peste Israel?"
1Ki 18:18  Iar Ilie a zis: "Nu eu sunt cel ce aduce nenorocire peste Israel; ci tu ?i casa tatalui tau, pentru ca a?i parasit poruncile Domnului ?i merge?i dupa baali.
1Ki 18:19  Trimite dar acum ?i aduna la mine în muntele Carmel tot Israelul, dimpreuna cu cei patru sute cincizeci de prooroci ai lui Baal ?i cu cei patru sute de prooroci ai A?erei, care manânca la masa Izabelei".
1Ki 18:20  ?i a trimis Ahab pe to?i fiii lui Israel ?i a luat pe to?i proorocii în muntele Carmel.
1Ki 18:21  Atunci s-a apropiat Ilie de tot poporul ?i a zis: "Pâna când ve?i ?chiopata de amândoua picioarele? Daca Domnul este Dumnezeu, urma?i Lui! ?i daca este Baal, urma?i aceluia". Poporul însa n-a raspuns nimic.
1Ki 18:22  ?i a zis Ilie catre popor: "Prooroc al Domnului am ramas numai eu singur, iar prooroci ai lui Baal sunt patru sute cincizeci de oameni ?i prooroci ai A?erei patru sute.
1Ki 18:23  Da?i-ne doi vi?ei; el sa-?i aleaga unul, sa-l taie buca?i ?i sa-l puna pe lemne, dar foc sa nu faca, iar eu voi taia buca?i pe celalalt vi?el ?i-l voi pune pe lemne ?i foc nu voi face.
1Ki 18:24  Apoi voi sa chema?i numele dumnezeului vostru, iar eu voi chema numele Domnului Dumnezeului meu. ?i Dumnezeul Care va raspunde cu foc, Acela este Dumnezeu". ?i a raspuns tot poporul: "Bine ai grait!"
1Ki 18:25  ?i a zis Ilie proorocilor lui Baal: "Sa va alege?i un vi?el ?i sa-l pregati?i voi înainte, caci sunte?i mai mul?i ?i sa chema?i numele dumnezeului vostru, dar foc sa nu face?i".
1Ki 18:26  ?i au luat ei vi?elul care li s-a dat ?i l-au pregatit ?i au chemat numele lui Baal de diminea?a pâna la amiaza, zicând: "Baale, auzi-ne!" Dar n-a fost nici glas, nici raspuns. ?i sareau împrejurul jertfelnicului pe care-l facusera.
1Ki 18:27  Iar pe la amiaza, Ilie a început sa râda de ei ?i zicea: "Striga?i mai tare, caci doar este dumnezeu! Poate sta de vorba cu cineva, sau se îndeletnice?te cu ceva, sau este în calatorie, sau poate doarme; striga?i tare sa se trezeasca!"
1Ki 18:28  ?i ei strigau cu glas mai tare ?i se în?epau dupa obiceiul lor cu sabii ?i cu lanci, pâna ce curgea sânge.
1Ki 18:29  Trecuse acum de amiaza ?i ei s-au zbuciumat mereu pâna la timpul jertfei. Dar n-a fost nici glas, nici raspuns, nici auzire. Atunci a zis Ilie Tesviteanul catre proorocii lui Baal: "Da?i-va acum la o parte, ca sa-mi savâr?esc ?i eu jertfa mea!" ?i s-au dat la o parte.
1Ki 18:30  Atunci a zis Ilie catre popor: "Apropia?i-va de mine!" ?i s-a apropiat tot poporul de el. ?i a facut jertfelnicul Domnului ce fusese darâmat;
1Ki 18:31  A luat Ilie douasprezece pietre, dupa numarul semin?iilor fiilor lui Iacov, catre care a zis Domnul: "Israel va fi numele tau!"
1Ki 18:32  ?i a zidit din pietrele acelea jertfelnicul în numele Domnului, facând împrejurul jertfelnicului ?an? în care încapeau doua masuri de samân?a,
1Ki 18:33  A a?ezat lemnele pe jertfelnic, a taiat vi?elul buca?i ?i le-a pus pe el.
1Ki 18:34  ?i a zis: "Umple?i patru cofe cu apa ?i le turna?i peste jertfa arderii de tot ?i peste lemne!" ?i au facut a?a. Apoi a zis: "Face?i aceasta a doua oara!" ?i au facut la fel a doua oara. ?i a zis: "Face?i a?a ?i a treia oara!"
1Ki 18:35  ?i umbla apa împrejurul jertfelnicului ?i ?an?ul se umpluse de apa.
1Ki 18:36  Iar la vremea jertfei de seara, s-a apropiat Ilie proorocul ?i a strigat la cer ?i a zis: "Doamne Dumnezeul lui Avraam, al lui Isaac ?i al lui Israel! Auzi-ma Doamne, auzi-ma acum cu foc, ca sa cunoasca astazi poporul acesta ca Tu singur e?ti Dumnezeu în Israel ?i ca eu sunt slujitorul Tau.
1Ki 18:37  Auzi-ma, Doamne, auzi-ma, ca sa cunoasca poporul acesta ca Tu Doamne e?ti Dumnezeu ?i ca Tu le întorci inima la Tine!"
1Ki 18:38  ?i s-a pogorât foc de la Domnul ?i a mistuit arderea de tot ?i lemnele ?i pietrele ?i ?arâna ?i a mistuit ?i toata apa care era în ?an?.
1Ki 18:39  ?i tot poporul, când a vazut aceasta, a cazut cu fa?a la pamânt ?i a zis  "Domnul este Dumnezeu, Domnul este Dumnezeu!"
1Ki 18:40  Iar Ilie le-a zis: "Prinde?i pe proorocii lui Baal, ca sa nu scape nici unul din ei!" ?i i-au prins, ?i s-a dus Ilie la pârâul Chi?onului ?i i-a junghiat acolo.
1Ki 18:41  Apoi a zis Ilie catre Ahab: "Du-te de manânca ?i bea, caci se aude vuiet de ploaie!"
1Ki 18:42  ?i a plocat Ahab sa manânce ?i sa bea; iar Ilie s-a suit în vârful Carmelului ?i s-a aplecat ta pamânt pâna a atins genunchii au fa?a sa.
1Ki 18:43  ?i a zis ucenicului sau: "Du-te ?i te uita spre mare!" ?i s-a dus el ?i s-a uitat ?i a zis: "Nu vad nimic!" El i-a zis: "Du-te ?i fa aceasta de ?apte ori".
1Ki 18:44  Iar a ?aptea oara a zis: "Iata se ridica din mare un nor cât o palma". Iar Ilie a zis: "Du-te ?i spune lui Ahab: Înhama ?i fugi, ca sa nu te apuce ploaia!"
1Ki 18:45  ?i pe când grabea el, cerul s-a întunecat de nori ?i s-a pornit vijelie ?i ploaie mare. Iar Ahab, suindu-se în caru?a, s-a dus la Izreel.
1Ki 18:46  Iar mâna Domnului a fost peste Ilie, care, încingându-?i mijlocul, a alergat înaintea lui Ahab, pâna la Izreel.
1Ki 19:1  ?i a spus Ahab Izabelei. tot ce a facut Ilie ?i cum a ucis pe to?i proorocii cu sabia.
1Ki 19:2  Atunci a trimis Izabela un vestitor la Ilie, ca sa-i spuna: "Daca tu e?ti Ilie ?i eu Izabela, a?a ?i a?a sa-mi faca dumnezeii, ba înca ?i mai mult, daca mâine pe vremea aceasta nu voi face cu via?a ta la fel cum ai facut ?i tu cu fiecare din ei!"
1Ki 19:3  Când a auzit Ilie aceasta, s-a sculat ?i a plecat, sa-?i scape via?a sa, ?i a venit la Beer-?eba, care este în Iuda, ?i ?i-a lasat ucenicul acolo,
1Ki 19:4  Iar el s-a dus mai departe în pustiu, cale de o zi, ?i s-a a?ezat sub un ienupar ?i î?i ruga moartea, zicând: "Îmi ajunge acum, Doamne! Ia-mi sufletul ca nu sunt eu mai bun decât parin?ii mei!"
1Ki 19:5  ?i s-a culcat ?i a adormit acolo sub ienupar. ?i iata un înger l-a atins ?i i-a zis: "Scoala de manânca ?i bea!"
1Ki 19:6  ?i a cautat Ilie ?i iata, la capatâiul lui, o turta coapta în vatra ?i un urcior cu apa. ?i a mâncat ?i a baut ?i a adormit iar.
1Ki 19:7  Dar iata îngerul Domnului s-a întors a doua oara, s-a atins de el ?i a zis: "Scoata de manânca ?i bea, ca lunga-?i este calea!"
1Ki 19:8  ?i s-a sculat Ilie ?i a mâncat ?i a batut ?i întarindu-se cu acea mâncare, a mers patruzeci de zile ?i patruzeci  de nop?i, pâna la Horeb, muntele lui Dumnezeu.
1Ki 19:9  ?i a intrat acolo într-o pe?tera ?i a ramas acolo. ?i iata cuvântul Domnului a fost catre el ?i i-a zis: "Ce faci aici, Ilie?"
1Ki 19:10  Iar Ilie a zis: "Cu râvna am râvnit pentru Domnul Dumnezeul Savaot, caci fiii lui Israel au parasit legamântul Tau, au darâmat jertfelnicele Tale ?i pe proorocii Tai i-au ucis cu sabia, ramânând numai eu singur, dar cauta sa ia ?i sufletul meu!"
1Ki 19:11  A zis Domnul: "Ie?i ?i stai pe munte înaintea fe?ei Domnului! Ca iata Domnul va trece; ?i înaintea Lui va fi vijelie naprasnica ce va despica mun?ii ?i va sfarâma stâncile, dar Domnul nu va fi în vijelie. Dupa vijelie va fi cutremur, dar Domnul nu va fi în cutremur;
1Ki 19:12  Dupa cutremur va fi foc, dar nici în foc nu va fi Domnul. Iar dupa foc va fi adiere de vânt lin ?i acolo va fi Domnul".
1Ki 19:13  Auzind aceasta, Ilie ?i-a acoperit fa?a cu mantia lui ?i a ie?it ?i a stat la gura pe?terii. ?i a fost catre el un glas care i-a zis: "Ce faci aici, Ilie?"
1Ki 19:14  Iar el a zis: "Cu râvna am râvnit pentru Domnul Dumnezeul Savaot, ca au parasit fiii lui Israel legamântul Tau, au darâmat jertfelnicele Tale ?i pe proorocii Tai i-au ucis cu sabia; numai eu singur am ramas, dar cauta sa ia ?i sufletul meu!"
1Ki 19:15  ?i a zis Domnul: "Mergi ?i întoarce-te pe calea ta prin pustiu la Damasc ?i, când vei ajunge acolo, sa ungi rege peste Siria pe Hazael;
1Ki 19:16  Pe Iehu, fiul lui Nim?i, sa-l ungi rege peste Israel; iar pe Elisei, fiul lui ?afat din Abel-Mehola, sa-l ungi prooroc în locul tau!
1Ki 19:17  Cine va fugi de sabia lui Hazael, pe acela sa-l omoare Iehu, iar cine va scapa de sabia lui Iehu, pe acela sa-l omoare Elisei.
1Ki 19:18  Eu însa mi-am oprit dintre Israeli?i ?apte mii de barba?i; genunchii tuturor acestora nu s-au plecat înaintea lui Baal ?i buzele tuturor acestora nu l-au sarutat!"
1Ki 19:19  Atunci a plecat Ilie de acolo ?i a gasit pe Elisei, fiul lui ?afat, arând; acesta avea douasprezece perechi de boi la pluguri ?i la perechea a douasprezecea era el însu?i. ?i Ilie a trecut pe lânga el aruncându-i mantia.
1Ki 19:20  Atunci a lasat Elisei boii ?i a alergat dupa Ilie, zicând: "Lasa-ma sa merg sa sarut pe tatal ?i pe mama mea ?i voi veni dupa tine!" Iar el i-a zis: "Du-te ?i vino înapoi, ca ce-am facut e facut!"
1Ki 19:21  Plecând de la el, a luat o pereche de boi pe care i-a junghiat ?i, facând foc cu plugul boilor, a fript carnea lor ?i au împar?it-o la oameni ?i au mâncat-o. Iar el s-a sculat ?i s-a dus dupa Ilie ?i a început sa-i slujeasca.
1Ki 20:1  Benhadad, regele Siriei, ?i-a adunat o?tirea sa; ?i s-au înso?it cu el treizeci ?i doi de regi, cu cai ?i care de razboi. ?i s-a dus ?i a împresurat Samaria ?i s-a razboit împotriva ei.
1Ki 20:2  Atunci a trimis soli la Ahab, regele lui Israel, în cetate ?i a zis:
1Ki 20:3  "A?a zice Benhadad: Argintul tau ?i aurul tau este al meu, femeile tale ?i fiii tai cei frumo?i sunt ai mei!"
1Ki 20:4  Iar regele lui Israel a raspuns ?i a zis: "Dupa cuvântul tau sa fie, domnul meu rege. Eu ?i toate ale mele ale tale sa fie!"
1Ki 20:5  ?i s-au întors solii ?i au zis: "A?a zice Benhadad: Am trimis la tine ca sa-?i spuna: Argintul ?i aurul tau, femeile ?i fiii tai sa mi le dai!
1Ki 20:6  De aceea mâine, pe vremea aceasta, voi trimite robii mei la tine, iar ei î?i vor scotoci casa ta ?i casele slugilor tale ?i pe tot ce este mai scump în ochii tai vor pune mâna ?i vor lua!"
1Ki 20:7  Atunci a chemat regele lui Israel pe to?i batrânii ?arii ?i a zis: "Lua?i seama ?i vede?i ca îmi cauta mereu pricina; când a trimis la mine pentru femeile mele ?i pentru fiii mei ?i pentru argintul meu ?i pentru aurul meu, eu nu l-am respins!"
1Ki 20:8  ?i au zis to?i batrânii ?i tot poporul: "Sa nu-l ascul?i, nici sa nu te învoie?ti!"
1Ki 20:9  ?i a zis el solilor lui Benhadad: "Spune?i domnului meu, regele: Toate lucrurile pentru care ai trimis la robul tau ia început sunt gata sa le fac; dar acest lucru nu pot sa-l fac!" ?i au plecat solii ?i au dus raspunsul.
1Ki 20:10  ?i a trimis Benhadad la el, ca sa-i spuna: "Asta ?i asta sa-mi faca mie dumnezeii ?i a?a sa ma pedepseasca, daca ?arâna Samariei are sa ajunga sa umple pumnii tuturor oamenilor care au sa vina cu mine!"
1Ki 20:11  ?i a raspuns regele lui Israel ?i a zis: "Spune?i sa nu se laude cel ce se încinge ca cel ce se descinge".
1Ki 20:12  Când a primit acest raspuns, Benhadad era la ospa? în corturi, cu ceilal?i regi. ?i a zis robilor sai: "Împresura?i cetatea!" ?i au împresurat-o.
1Ki 20:13  Dar iata un prooroc s-a apropiat de Ahab, regele lui Israel, ?i a zis: "A?a zice Domnul: Vezi toata mul?imea aceasta mare? Iata, Eu ?i-o dau astazi în mâna, ca sa cuno?ti ca Eu sunt Domnul.
1Ki 20:14  Iar Ahab a zis: "Prin cine?" Raspuns-a acela: "A?a zice Domnul: Prin slugile capeteniilor de peste ?inuturi!" ?i a zis Ahab: "Cine va începe lupta?" Iar acela a zis: "Tu!"
1Ki 20:15  Atunci a numarat Ahab slugile capeteniilor de peste ?inuturi, ?i s-au gasit doua sute treizeci ?i doua; dupa ei a numarat ?i tot poporul ?i to?i fiii lui Israel au fost ?apte mii.
1Ki 20:16  ?i au ie?it la razboi pe la amiaza. Iar Benhadad bause pâna se îmbatase în corturi cu cei treizeci ?i doi de regi, care-i dadusera ajutor.
1Ki 20:17  ?i au ie?it la razboi mai întâi slugile capeteniilor de peste ?inuturi. ?i a trimis Benhadad pe unii din ai lui, ?i ace?tia i-au facut cunoscut ca oamenii din Samaria au ie?it la razboi.
1Ki 20:18  Iar el le-a zis: "Daca ei au venit pentru pace, sa-i prinde?i de vii; iar daca au venit pentru razboi, tot de vii sa-i prinde?i!"
1Ki 20:19  Au ie?it, a?adar, la razboi slugile capeteniilor de peste ?inuturi din cetate ?i o?tirea oare era dupa ei ?i au lovit fiecare pe potrivnicul sau;
1Ki 20:20  ?i au fugit Sirienii, iar Israeli?ii îi fugareau din urma; ?i Benhadad a scapat pe un cal cu calare?ii sai.
1Ki 20:21  Atunci, ie?ind regele israelitean, a luat caii ?i carele ?i a facut macel mare în rândul Sirienilor.
1Ki 20:22  ?i s-a apropiat proorocul de regele israelit ?i i-a zis: "Du-te de te întare?te, ca sa ?tii sa bagi de seama ce ai de facut, pentru ca dupa un an regele Siriei are sa se scoale din nou împotriva ta!"
1Ki 20:23  Zis-au slugile regelui sirian catre el: "Dumnezeul lor este Dumnezeul mun?ilor, iar nu Dumnezeul vailor, pentru aceasta au fost ei mai tari decât noi; dar de ne vom bate cu ei în vale, vom fi noi mai tari decât ei.
1Ki 20:24  A?adar, iata ce sa faci: Sa sco?i pe fiecare rege de la locul lui, ?i sa-l înlocuie?ti cu capeteniile din ?inuturi.
1Ki 20:25  ?i sa strângi atâta o?tire, câta o?tire ai pierdut, ?i cai atâ?ia, câ?i ai avut, ?i care atâtea, câte ai avut, ?i sa ne batem cu ei ?i în vale; ?i atunci de buna seama vom fi mai tari decât ei". ?i a ascultat el cuvântul lor ?i a facut a?a.
1Ki 20:26  Dupa trecerea anului, Benhadad a numarat pe Sirieni ?i a mers la Afec, ca sa se bata cu Israel.
1Ki 20:27  Fiii lui Israel însa erau ?i ei cu toate gata de lupta ?i au ie?it în întâmpinarea lor. Fiii lui Israel ?i-au a?ezat tabara înaintea lor, ca doua turme mici de capre, iar Sirienii umpleau pamântul.
1Ki 20:28  Atunci s-a apropiat omul lui Dumnezeu ?i a zis regelui lui Israel: "A?a zice Domnul: Pentru ca Sirienii vorbesc: Domnul este Dumnezeul mun?ilor, iar nu Dumnezeul vailor, iata Eu î?i voi da toata aceasta mul?ime în mâna ta, ca sa cuno?ti ca Eu sunt Domnul!"
1Ki 20:29  ?i au stat taberele una în fa?a celeilalte ?apte zile; iar în ziua a ?aptea s-a început batalia ?i fiii lui Israel au doborât într-o singura zi o suta de mii de pedestra?i sirieni.
1Ki 20:30  Ceilal?i au fugit la Afec, în cetate; acolo a cazut zidul peste douazeci ?i ?apte de mii de oameni, din cei rama?i. Iar Benhadad a fugit în cetate ?i a intrat la curtea domneasca într-o camara dosnica.
1Ki 20:31  Slugile sale însa i-au zis: "Am auzit ca regii casei lui Israel sunt regi milostivi; da-ne voie sa punem sac peste coapsele noastre ?i peste capetele noastre funii, sa mergem la regele israelit; poate ne va cru?a via?a".
1Ki 20:32  ?i s-au încins ei peste coapsele lor cu sac ?i au venit la regele lui Israel ?i i-au zis: "Robul tau Benhadad zice: Cru?a-mi via?a!" Acela însa a zis: "De mai traie?te înca, e fratele meu!"
1Ki 20:33  ?i oamenii ace?tia au luat aceasta ca semn bun ?i repede i-au apucat vorba din gura ?i au zis: "Fratele tau Benhadad traie?te". Iar el a zis: "Duce?i-va de-l aduce?i". ?i a venit Benhadad la el, ?i acesta l-a luat cu el în car.
1Ki 20:34  ?i i-a zis Benhadad: "Ceta?ile care le-a luat tatal meu de la tatal tau, eu ?i le întorc; ?i î?i deschid târg de vânzare în Damasc, cum ?i-a deschis tatal meu în Samaria". ?i a zis Ahab: "Cu acest legamânt ?i eu î?i voi da drumul!" ?i încheind cu el legamânt, i-a dat drumul.
1Ki 20:35  Atunci un om din fiii proorocilor a zis catre un prieten al sau, dupa cuvântul Domnului: "Bate-ma!" Dar acesta nu s-a învoit sa-l bata.
1Ki 20:36  Atunci i-a zis: "Pentru ca nu ascul?i de glasul Domnului, iata, când vei pleca de lânga mine, te va ucide un leu". ?i plecând omul acela de la el, l-a întâlnit un leu ?i l-a ucis pe dânsul.
1Ki 20:37  Iar cel din fiii proorocilor a gasit un alt om ?i i-a zis: "Love?te-ma!" ?i acest om l-a lovit ?i l-a ranit.
1Ki 20:38  ?i s-a dus proorocul în calea regelui, acoperindu-?i ochii cu parul capului sau.
1Ki 20:39  Iar când a trecut regele pe lânga el, a strigat dupa rege ?i a zis: "Robul tau a fost la batalie, ?i iata un. om care statuse la o parte mi-a adus un alt om ?i mi-a zis: Paze?te pe omul acesta; de îl vei scapa, vei raspunde cu sufletul tau pentru sufletul lui sau va trebui sa-l plati?i cu un talant de argint.
1Ki 20:40  Când robul tau se îndeletnicea cu unele treburi, acela a scapat". ?i i-a zis regele lui Israel: "Osânda ta singur ai spus-o".
1Ki 20:41  Atunci el ?i-a dat la o parte parul de peste ochii lui ?i l-a cunoscut regele ca este unul din prooroci.
1Ki 20:42  ?i i-a zis: "A?a zice Domnul: Pentru ca ai dat drumul din mâinile tale unui om blestemat de Mine, sufletul tau va fi în locul sufletului lui ?i poporul tau în locul poporului lui!"
1Ki 20:43  ?i s-a dus regele lui Israel acasa, trist ?i aprins de mânie ?i a venit în Samaria.
1Ki 21:1  Iar dupa aceasta iata ce s-a mai întâmplat: Nabot Izreeliteanul avea în Izreel o vie lânga curtea lui Ahab, regele Samariei.
1Ki 21:2  ?i a grait Ahab cu Nabot ?i a zis: "Sa-mi dai mie via ta, ca sa-mi fac din ea o gradina de verde?uri, caci e aproape de casa mea; iar în locul ei î?i voi da o vie mai buna decât aceasta, sau daca î?i convine mai bine, î?i voi da argint cât pre?uie?te ea".
1Ki 21:3  Nabot însa a zis catre Ahab: "Sa ma fereasca Dumnezeu sa-?i dau eu mo?tenirea parin?ilor mei!"
1Ki 21:4  ?i a venit Ahab acasa trist ?i mânios pentru cuvântul care i-l spusese Nabot Izreeliteanul, zicând: Nu-îi dau mo?tenirea parin?ilor mei. ?i cu duhul tulburat, s-a culcat în patul sau, s-a întors cu fa?a la perete ?i n-a mâncat.
1Ki 21:5  Atunci a intrat Izabela, femeia lui la el ?i i-a zis: "De ce este întristat duhul tau ?i nu manânci?"
1Ki 21:6  Iar el a zis: "Când am vorbit cu Nabot Izreeliteanul ?i i-am zis: Da-mi via ta pe bani, sau daca vrei, sa-îi dau alta vie în locul ei, el a zis: Nu-?i dau via mea, ca este mo?tenirea parin?ilor mei".
1Ki 21:7  A zis Izabela, femeia lui: "Ce cârmuire ar mai fi în Israel, daca tu ai face tot a?a? Scoala, manânca ?i fii cu voie buna, ca via lui Nabot Izreeliteanul ?i-o dau eu!"
1Ki 21:8  Apoi ea a scris scrisori în numele lui Ahab, le-a pecetluit cu inelul lui ?i a trimis aceste scrisori la batrânii ?i la frunta?ii din cetatea lui Nabot, care locuiau cu el acolo.
1Ki 21:9  În scrisori însa ea scria a?a: "Vesti?i tuturor sa posteasca post ?i pune?i pe Nabot înaintea poporului.
1Ki 21:10  ?i aduce?i doi oameni netrebnici sa marturiseasca împotriva lui ?i sa spuna: Ai hulit pe Dumnezeu ?i pe rege; ?i apoi sa-l scoate?i afara ?i sa-l ucide?i cu pietre ?i a?a sa moara".
1Ki 21:11  ?i au facut barba?ii ceta?ii lui Nabot, batrânii ?i frunta?ii care locuiau cu el în cetate, a?a cum le poruncise Izabela ?i cum era scris în scrisorile trimise de ea lor:
1Ki 21:12  Au vestit post tuturor ?i au pus pe Nabot înaintea poporului.
1Ki 21:13  Atunci au ie?it doi oameni ticalo?i. ?i au stat împotriva lui Nabot ?i au marturisit ace?ti oameni rai în fa?a poporului ?i au zis: "Nabot a hulit pe Dumnezeu ?i pe rege". Iar ei l-au scos afara din cetate ?i l-au batut cu pietre ?i a murit.
1Ki 21:14  Apoi au trimis la Izabela sa-i spuna: "Nabot a fost omorât cu pietre".
1Ki 21:15  Auzind Izabela ca Nabot a fost batut eu pietre ?i a murit, a zis catre Ahab: "Scoala ?i ia în stapânire via lui Nabot Izreeliteanul, care n-a vrut sa ?i-o vânda cu bani; ca Nabot nu mai traie?te, ci a murit!"
1Ki 21:16  Când a auzit Ahab ca Nabot Izreeliteanul a fost ucis, ?i-a rupt hainele sale, s-a îmbracat cu sac, ?i apoi s-a sculat Ahab ca sa se duca la via lui Nabot Izreeliteanul ?i sa o ia în stapânire.
1Ki 21:17  Atunci a fost cuvântul Domnului catre Ilie Tesviteanul:
1Ki 21:18  "Scoala ?i ie?i în întâmpinarea lui Ahab, regele lui Israel, care este în Samaria, ca iata acum e la via lui Nabot, unde s-a dus ca sa o ia în stapânire,
1Ki 21:19  ?i spune-i: A?a graie?te Domnul: Ai ucis ?i vrei înca sa intri în mo?tenire? ?i sa-i mai spui: A?a zice Domnul: În locul unde au lins câinii sângele lui Nabot, acolo vor linge câinii ?i sângele tau!",
1Ki 21:20  ?i a zis Ahab catre Ilie: "M-ai aflat, du?manule, ?i aici!" Iar el a zis: "Te-am aflat, caci te-ai încumetat sa savâr?e?ti fapte nelegiuite înaintea ochilor Domnului ?i sa-L min?i.
1Ki 21:21  A?a zice Domnul: "Iata voi aduce peste tine necazuri ?i te voi matura ?i voi stârpi din ai lui Ahab pe cei de parte barbateasca, fie rob, fie slobod, în Israel;
1Ki 21:22  ?i voi face cu casa ?a cum am facut cu casa lui Ieroboam, fiul lui Nabat, ?i cu casa lui Bae?a, fiul lui Ahia, pentru faradelegea cu care M-ai mâniat ?i ai dus pe Israel în pacat".
1Ki 21:23  Asemenea ?i pentru Izabela a grait Domnul: "Câinii vor mânca pe Izabela pe zidul Izreelului.
1Ki 21:24  Cine va muri din ai lui Ahab în cetate, pe acela câinii îl vor mânca, iar cine va muri în câmp, pe acela pasarile cerului îl vor mânca;
1Ki 21:25  Caci n-a fost înca nimeni ca Ahab, care sa se încumete a savâr?i fapte urâte înaintea ochilor Domnului, la care l-a împins Izabela so?ia sa.
1Ki 21:26  S-a purtat rau ca un ticalos, urmând dupa idoli, cum au facut Amoreii pe care Domnul i-a izgonit de la fa?a fiilor lui Israel".
1Ki 21:27  Când a auzit Ahab toate cuvintele acestea, a început sa plânga, ?i-a rupt hainele sale, s-a îmbracat pesta trupul sau cu sac, a postit ?i a dormit în sac ?i a umblat trist.
1Ki 21:28  Atunci a fost cuvântul Domnului catre Ilie Tesviteanul, pentru Ahab, ?i a zis Domnul:
1Ki 21:29  "Vezi cum s-a smerit Ahab înaintea Mea? Fiindca s-a smerit înaintea Mea, de aceea nu voi aduce necazurile în zilele lui, ci în zilele fiului lui voi aduce necazurile peste casa lui!"
1Ki 22:1  Au trecut trei ani fara razboi între Siria ?i Israel.
1Ki 22:2  În al treilea an s-a dus Iosafat, regele Iudei, la regele lui Israel;
1Ki 22:3  ?i a zis regele lui Israel catre slugile sale: "?ti?i voi, oare, ca Ramotul Galaadului este al nostru, ?i noi tacem de atâta vreme ?i nu-l scoatem din mâna regelui Siriei?"
1Ki 22:4  Apoi a zis el lui Iosafat: "Vei merge ?i tu cu mine la razboi împotriva Ramot-Galaadului?" Iar Iosafat a zis catre regele lui Israel: "Cum e?ti tu, a?a sunt ?i eu; cum este poporul tau, a?a este ?i poporul meu; cum sunt caii tai, a?a sunt ?i caii mei!"
1Ki 22:5  ?i a mai zis Iosafat, regele lui Iuda, catre regele lui Israel: "Întreaba dar, astazi, ce zice Domnul?"
1Ki 22:6  ?i a adunat regele lui Israel ca la patru sute de prooroci ?i le-a zis: "Sa merg eu, oare, cu razboi împotriva Ramot-Galaadului sau nu?" ?i ei au zis: "Sa mergi, ca Domnul îl va da în mâinile regelui!"
1Ki 22:7  ?i a zis Iosafat: "Nu mai este oare aici vreun prooroc al Domnului, ca sa întrebam pe Domnul prin el?"
1Ki 22:8  ?i a zis regele lui Israel catre Iosafat: "Mai este un om prin care se poate întreba Domnul, însa eu nu-l iubesc, caci nu prooroce?te de bine pentru mine, ci numai de rau; acesta e Miheia, fiul lui Imla". ?i a zis Iosafat: "Nu vorbi a?a, rege!"
1Ki 22:9  ?i a chemat regele lui Israel pe un famen ?i a zis: "Du-te repede dupa Miheia, fiul lui Imla!"
1Ki 22:10  Apoi regele lui Israel ?i Iosafat, regale Iudei, s-au a?ezat fiecare în tronul sau, îmbraca?i în haine domne?ti, pe locul dinaintea por?ii Samariei, ?i to?i proorocii prooroceau înaintea lor.
1Ki 22:11  Iar Sedechia, fiul lui Chenaana, ?i-a facut ni?te coarne de fier ?i a zis: "A?a zice Domnul: cu acestea vei împunge pe Sirieni pâna ce vor muri".
1Ki 22:12  ?i to?i proorocii au proorocit la fel, zicând: "Sa te duci împotriva Ramot-Galaadului, ca vei izbuti. Domnul îl va da în mina regelui".
1Ki 22:13  Iar trimisul, care s-a dus sa cheme pe Miheia, i-a grait acestuia, zicând: "Iata to?i proorocii proorocesc într-un glas de bine regelui; sa fie dar ?i cuvântul tau asemenea cu cuvântul fiecaruia din ei".
1Ki 22:14  Iar Miheia a zis: "Viu este Domnul! Ce-mi va spune Domnul, aceea voi grai!"
1Ki 22:15  Apoi a venit el a rege ?i regale i-a zis: "Miheia, sa mai mergem noi oare cu razboi împotriva Ramot-Galaadului sau nu?" ?i i-a zis acela: "Du-te, ca vei izbuti. Domnul îl va da în mâna regelui!"
1Ki 22:16  ?i i-a zis regele: "Iar ?i iar te jur, ca sa nu-mi graie?ti nimic, decât ce este adevarat, în numele Domnului".
1Ki 22:17  ?i a zis el: "Iata, vad pe to?i Israeli?ii împra?tia?i prin mun?i, ca oile ce n-au pastor. ?i a zis Domnul: Ei n-au domn, sa se întoarca fiecare cu pace la casa  sa".
1Ki 22:18  Atunci regele lui Israel a zis câtre Iosafat, regele Iudei: "Nu ?i-am spus eu oare ca el nu prooroce?te de bine pentru mine, ci numai de rau?"
1Ki 22:19  Miheia însa a zis: "Nu este a?a. Nu eu graiesc; asculta cuvântul Domnului. Nu este a?a. Am vazut pe Domnul stând pe tronul Sau, ?i toata o?tirea cereasca sta lânga El, la dreapta, ?i la stânga Lui".
1Ki 22:20  ?i a zis Domnul: "Cine ar îndupleca pe Ahab sa mearga în Ramot-Galaad ?i sa piara acolo?" Unul spunea una, ?i altul alta.
1Ki 22:21  Atunci a ie?it un duh ?i a stat înaintea fe?ei Domnului ?i a zis: "Eu îl voi ademeni". Domnul a zis: "Cum?"
1Ki 22:22  Iar acela a zis: "Ma duc ?i ma fac duh mincinos în gura tuturor proorocilor lui". Domnul a zis: "Tu îl vei ademeni ?i vei face aceasta; du-te ?i fa cum ai zis!"
1Ki 22:23  ?i iata cum a îngaduit Domnul duhului celui mincinos sa fie în gura tuturor acestor prooroci ai tai; însa Domnul n-a grait bine de tine!"
1Ki 22:24  Atunci s-a apropiat Sedechia, fiul lui Chenaana, ?i, lovind pe Miheia peste obraz, a zis: "Cum? Au doara s-a departat Duhul Domnului de la mine, ca sa graiasca prin tine?"
1Ki 22:25  Iar Miheia a zis: "Iata, ai sa vezi aceasta în ziua când vei fugi din camara în camara, ca sa te ascunzi".
1Ki 22:26  A zis regele lui Israel: "Lua?i pe Miheia ?i-l duce?i la Amon, capetenia ceta?ii, ?i la Ioa?, fiul regelui,
1Ki 22:27  ?i spune?i: A?a zice regele: Arunca?i-l în temni?a ?i-l hrani?i numai cu pu?ina pâine ?i cu pu?ina apa, pâna ce ma voi întoarce biruitor".
1Ki 22:28  Iar Miheia a zis: "Ca ai sa te întorci biruitor, aceasta n-a grait-o Domnul prin mine". Apoi a zis: "Asculta?i, toate popoarele!"
1Ki 22:29  Dupa aceea a purces regele lui Israel ?i Iosafat, regele Iudei, împreuna asupra Ramot-Galaadului.
1Ki 22:30  ?i a zis regele lui Israel catre Iosafat: "Eu îmi schimb hainele ?i intru în lupta, iar tu îmbraca-?i hainele de rege!" ?i ?i-a schimbat hainele regele lui Israel ?i a intrat în lupta.
1Ki 22:31  Regale sirian însa a poruncit celor treizeci ?i doua de capetenii ale carelor de razboi ?i a zis: "Sa nu va lupta?i nici cu mic, nici cu mare, ci numai eu regele lui Israel".
1Ki 22:32  Capeteniile carelor, vazând pe Iosafat, au crezut ca acesta este cu adevarat regele lui Israel ?i s-au repezit asupra lui, ca sa se lupte cu el. Iosafat însa a strigat.
1Ki 22:33  Atunci capeteniile carelor, vazând ca nu este acesta regele lui Israel, s-au abatut de la el.
1Ki 22:34  Iar un om ?i-a întins arcul ?i a lovit din întâmplare pe regele lui Israel într-o încheietura a plato?ei, ?i acesta a zis carau?ului sau: "Întoarce îndarat ?i ma scoate din oaste, ca sunt ranit".
1Ki 22:35  ?i s-a pornit lupta mare în acea zi ?i regele a stat În carul lui în fala Sirienilor toata ziua, iar seara a murit; ?i a curs sânge din rana sa pe podul carului.
1Ki 22:36  ?i la asfin?itul soarelui s-a dat de veste la toata tabara, zicând: "Sa mearga fiecare la cetatea lui, fiecare la ?inutul lui!"
1Ki 22:37  ?i murind regele, a fost dus ?i l-au înmormântat în Samaria.
1Ki 22:38  ?i au spalat carul lui în iazul Samariei; ?i câinii i-au lins sângele lui Ahab, iar desfrânatele s-au scaldat în spalatura acelui sânge, dupa cuvântul pe care l-a grait Domnul.
1Ki 22:39  Celelalte fapte ale lui Ahab, tot ce a facut el ?i casa cea de filde? pe care a facut-o el ?i toate ceta?ile pe care el le-a zidit sunt scrise în cronica regilor lui Israel.
1Ki 22:40  Adormind cu parin?ii sai, în locul lui Ahab s-a facut rege Ohozia, fiul sau.
1Ki 22:41  Iosafat, fiul lui Asa, a început sa domneasca peste Iuda în anul al patrulea al lui Ahab, regele lui Israel.
1Ki 22:42  Când s-a facut rege, Iosafat era ca da treizeci ?i cinci de ani ?i douazeci ?i cinci de ani a domnit el în Ierusalim. Pe mama lui o chema Azuba ?i era fiica lui ?ilhi.
1Ki 22:43  El a umblat întru totul pe caile lui Asa, tatal sau ?i nu s-a abatut de la ele, savâr?ind fapte placute înaintea ochilor Domnului. Numai înal?imile nu le-a desfiin?at, caci poporul înca savâr?ea jertfe ?i tamâieri pe înal?imi.
1Ki 22:44  Iosafat a facut pace cu regele lui Israel.
1Ki 22:45  Celelalte fapte ale lui Iosafat ?i razboaiele pe care le-a purtat sunt scrise în cronica regilor lui Iuda.
1Ki 22:46  ?i rama?i?a desfrâna?ilor care mai ramasese din zilele tatalui sau Asa, a stârpit-o din ?ara.
1Ki 22:47  În Idumeea pe atunci nu era rege, ci numai un loc?iitor de rege.
1Ki 22:48  Regele Iosafat a facut ?i corabii ca cele de Tarsis, ca sa se duca sa aduca aur de la Ofir; dar n-au putut sa ajunga, sfarâmându-se la E?ion Gheber.
1Ki 22:49  Atunci a zis Ohozia, fiul lui Ahab, catre Iosafat: "Sa se duca ?i slugile mele cu slugile tale cu corabiile". Însa Iosafat n-a vrut.
1Ki 22:50  ?i a adormit cu parin?ii sai în cetatea lui David, stramo?ul sau, ?i a fost îngropat Iosafat cu ei, în cetatea lui David ?i în locul lui s-a facut rege Ioram, fiul sau.
1Ki 22:51  Ohozia, fiul lui Ahab, s-a facut rege peste Israel în Samaria în anul al ?aptesprezecelea al lui Iosafat, regele Iudei; ?i a domnit peste Israel în Samaria doi ani.
1Ki 22:52  El a savâr?it fapte urâte înaintea ochilor Domnului ?i a umblat pe caile tatalui sau ?i pe caile mamei sale ?i pe caile lui Ieroboam, fiul lui Nabat, care a dus pe Israel la pacat;
1Ki 22:53  Caci a slujit lui Baal ?i i s-a închinat lui ?i a mâniat pe Domnul Dumnezeul lui Israel, cum facuse ?i tatal sau.


\end{document}