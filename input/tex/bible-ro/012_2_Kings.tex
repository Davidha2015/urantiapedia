\begin{document}

\title{2 Regi}


\chapter{1}

\par 1 Dupa moartea lui Ahab, Moabul s-a razvratit împotriva lui Israel.
\par 2 Iar Ohozia, cazând printre gratiile foi?orului sau cel din Samaria, s-a îmbolnavit ?i a trimis soli ?i le-a zis: "Duce?i-va ?i întreba?i pe Baal-Zebub, dumnezeul Ecronului: Ma voi însanato?i eu oare din boala aceasta?" ?i ace?tia s-au dus sa întrebe.
\par 3 Atunci îngerul Domnului a zis catre Ilie Tesviteanul: "Scoala ?i ie?i înaintea trimi?ilor regelui Samariei ?i le spune: Au doara în Israel nu este Dumnezeu, de va duce?i sa întreba?i pe Baal-Zebub, dumnezeul Ecronului?
\par 4 De aceea a?a zice Domnul: Din patul în care te-ai suit, nu te vei mai coborî, ci vei muri". ?i s-a dus Ilie ?i le-a spus.
\par 5 Atunci s-au întors solii la Ohozia ?i acesta le-a zis: "De ce v-a?i întors?"
\par 6 Iar ei i-au raspuns: "Ne-a ie?it înainte un om ?i ne-a zis: Întoarce?i-va ?i va duce?i la regele care v-a trimis ?i-i spune?i: A?a zice Domnul: Au doar nu este în Israel Dumnezeu, de trimi?i sa întrebe pe Baal-Zebub, dumnezeul Ecronului? De aceea nu te vei mai coborî din patul în care te-ai suit, ci vei muri".
\par 7 Iar regele le-a zis: "Ce înfa?i?are avea omul acela care v-a ie?it înainte ?i v-a grait cuvintele acestea?"
\par 8 ?i ei i-au raspuns: "Omul acela este paros peste tot ?i încins peste mijloc cu o cingatoare de curea". A zis regele: "Acela este Ilie Tesviteanul".
\par 9 Atunci regele a trimis la el o capetenie peste cincizeci cu cei cincizeci ai lui; ?i acesta a venit la el, când Ilie sta pe vârful muntelui ?i i-a zis: "Omul lui Dumnezeu, regele î?i zice: "Coboara-te!"
\par 10 Iar Ilie a raspuns: "De sunt omul lui Dumnezeu, sa se coboare foc din cer ?i sa te arda pe tine ?i pe cei cincizeci ai tai!" ?i s-a coborât foc din cer ?i l-a mistuit pe el ?i pe cei cincizeci ai lui.
\par 11 Apoi a trimis regele la el alta capetenie cu al?i cincizeci. Acesta i-a zis: "Omul lui Dumnezeu, a?a a zis regele: Coboara-te degraba!"
\par 12 ?i raspunzând, Ilie i-a zis: "De sunt omul lui Dumnezeu, sa se coboare foc din cer ?i sa te arda pe tine ?i pe cei cincizeci ai tai!" ?i s-a coborât focul lui Dumnezeu din cer ?i l-a ars pe el ?i pe cei cincizeci ai lui.
\par 13 ?i a mai trimis regele a treia oara o capetenie cu cincizeci. Dar a treia capetenie, venind ?i cazând în genunchi înaintea lui Ilie, l-a rugat, zicând: "Omul lui Dumnezeu, sa nu fie trecut cu vederea de ochii tai sufletul meu ?i sufletul acestor cincizeci de robi ai tai!
\par 14 Iata s-a coborât foc din cer ?i a mistuit pe cele doua capetenii peste cincizeci ?i pe oamenii lor; acum însa sa nu fie sufletul meu trecut cu vederea de ochii tai!"
\par 15 Atunci îngerul Domnului a zis catre Ilie: "Du-te cu el ?i nu te teme de el!" ?i s-a sculat Ilie ?i s-a dus cu el la rege.
\par 16 ?i a zis catre el: "A?a zice Domnul: De vreme ce tu ai trimis soli sa întrebe pe Baal-Zebub, dumnezeul Ecronului, ca ?i cum în Israel n-ar fi Dumnezeu, ca sa-I ceri cuvântul, de aceea din patul în care te-ai suit nu te vei mai coborî, ci vei muri".
\par 17 ?i apoi a murit Ohozia, dupa cuvântul Domnului pe care l-a rostit Ilie. ?i în locul lui s-a facut rege Ioram, fratele lui Ohozia, în anul al doilea al lui Ioram, fiul lui Iosafat, regele Iudei, caci Ohozia nu avea fecior.
\par 18 Celelalte fapte pe care le-a facut Ohozia sunt scrise în cartea cronicilor regilor lui Israel.

\chapter{2}

\par 1 În vremea când Domnul a vrut sa înal?e pe Ilie în vârtej de vânt la cer Ilie a plecat cu Elisei din Ghilgal.
\par 2 ?i Ilie a zis catre Elisei: "Stai aici, caci Dumnezeu ma trimite la Betel. Iar Elisei a zis: "Cât este de adevarat ca Domnul este viu ?i cum este viu ?i sufletul tau, tot a?a de adevarat este ca nu te voi lasa singur". ?i s-au dus amândoi la Betel.
\par 3 ?i au ie?it fiii proorocilor cei din Betel la Elisei ?i au zis catre el: "?tii oare ca astazi Domnul va sa ridice pe stapânul tau deasupra capului tau?" ?i el a zis: "?tiu ?i eu, dar tace?i!"
\par 4 Atunci Ilie a zis catre el: "Elisei, ramâi aici, caci Domnul ma trimite la Ierihon". Iar Elisei a zis: "Cât este de adevarat ca Domnul este viu ?i viu este ?i sufletul tau, tot a?a de adevarat este ca nu te voi lasa singur!"
\par 5 ?i au venit amândoi la Ierihon. Atunci s-au apropiat fiii proorocilor cei din Ierihon de Elisei ?i i-au zis: "?tii oare ca Domnul ia pe stapânul tau ?i-l înal?a deasupra capului tau?" ?i el a raspuns: "?tiu ?i eu, dar tace?i!"
\par 6 A zis Ilie: "Ramâi aici, caci Domnul ma trimite la Iordan!" Iar Elisei a raspuns: "Cât este de adevarat ca Domnul este viu ?i cum este viu ?i sufletul tau, tot a?a de adevarat este ca nu te voi lasa singur!"
\par 7 ?i s-au dus amândoi; s-au dus ?i cincizeci din fiii proorocilor ?i au stat deoparte în fa?a lor, iar ei amândoi ?edeau lânga Iordan.
\par 8 Atunci, luând Ilie mantia sa ?i strângând-o valatuc, a lovit cu ea apa ?i aceasta s-a strâns la dreapta ?i la stânga ?i au trecut ca pe uscat.
\par 9 Iar dupa ce au trecut, a zis Ilie catre Elisei: "Cere ce sa-?i fac, înainte de a fi luat de la tine". Iar Elisei a zis: "Duhul care este în tine sa fie îndoit în mine!"
\par 10 Raspuns-a Ilie: "Greu lucru ceri! Dar de ma vei vedea când voi fi luat de la tine, va fi a?a; iar de nu ma vei vedea, nu va fi".
\par 11 Pe când mergeau ei a?a pe drum ?i graiau, deodata s-a ivit un car ?i cai de foc ?i, despar?indu-i pe unul de altul, a ridicat pe Ilie în vârtej de vânt la cer.
\par 12 Iar Elisei se uita ?i striga: "Parinte, parinte, carul lui Israel ?i caii lui!" ?i apoi nu l-a mai vazut. ?i apucându-?i hainele le-a sfâ?iat în doua.
\par 13 Apoi, apucând mantia lui Ilie, care cazuse de la acesta, s-a întors înapoi ?i s-a oprit pe malul Iordanului.
\par 14 ?i a luat mantia lui Ilie care cazuse de la acesta ?i a lovit apa cu ea, zicând: "Unde este Domnul Dumnezeul lui Ilie?" ?i lovind, apa s-a tras la dreapta ?i la stânga ?i a trecut Elisei.
\par 15 Iar fiii proorocilor cei din Ierihon, vazându-l de departe, au zis: "Duhul lui Ilie s-a odihnit peste Elisei! ?i au ie?i t înaintea lui ?i i s-au plecat pâna la pamânt, zicându-i:
\par 16 "Iata, la noi, robii tai, se afla cincizeci de oameni voinici; sa se duca sa caute pe stapânul tau, poate l-a dus Duhul Domnului ?i l-a aruncat pe vreun munte sau într-o vale". Iar el a zis: "Sa nu-i trimite?i!"
\par 17 Ei însa au staruit mult pe lânga el ?i el, vazând ca nu poate scapa de ei, le-a zis: "Trimite?i-i!" ?i au trimis ei cincizeci de oameni ?i au cautat trei zile ?i nu l-au gasit;
\par 18 Întorcându-se apoi aceia la el în Ierihon, unde ramasese în vremea aceea, a zis catre ei Elisei: "Nu v-am spus eu sa nu va duce?i?"
\par 19 Iar locuitorii ceta?ii aceleia au zis catre Elisei: "Iata a?ezarea ceta?ii acesteia este buna, dupa cum poate vedea ?i stapânul nostru, dar apa nu este buna ?i pamântul este neroditor".
\par 20 ?i el a zis: "Aduce?i-mi o oala noua ?i pune?i sare în ea!"
\par 21 ?i i-au adus ?i a ie?it el la fântâna de apa ?i, aruncând sarea în ea, a zis: "A?a zice Domnul: Iata am facut apa aceasta sanatoasa ?i nu va mai pricinui nici vatamare, nici moarte, nici nerodire".
\par 22 ?i s-a facut apa curata pâna astazi, dupa cuvântul pe care l-a spus Elisei.
\par 23 De acolo s-a dus el la Betel. ?i cum mergea pe drum, au ie?it din cetate ni?te copii ?i s-au apucat sa râda de al zicând: "Hai, ple?uvule, hai!"
\par 24 Iar el, întorcându-se ?i vazându-i, i-a blestemat cu numele Domnului. Atunci, ie?ind din padure doi ur?i, au sfâ?iat din ei patruzeci ?i doi de copii.
\par 25 De aici Elisei s-a dus la muntele Carmelului, iar de acolo s-a întors în Samaria.

\chapter{3}

\par 1 Ioram, fiul lui Ahab, se facuse rege peste Israel în Samaria, în anul al optsprezecelea al lui Iosafat, regele Iudei, ?i a domnit doisprezece ani.
\par 2 Acesta a facut lucruri netrebnice în ochii Domnului, dar nu a?a cum facuse tatal sau ?i mama sa, caci el a departat stâlpii cu pisanii facu?i în cinstea lui Baal, pe care-î facuse tatal sau.
\par 3 Dar de pacatele lui Ieroboam, fiul lui Nabat, care a dus pe Israel în ratacire, s-a ?inut ?i el ?i nu s-a lasat de ele.
\par 4 Me?a, regele Moabului, era bogat în vite ?i trimitea regelui lui Israel câte o suta de mii de miei ?i câte o suta de mii de berbeci netun?i.
\par 5 Dar când a murit Ahab, regele Moabului s-a razvratit împotriva regelui lui Israel.
\par 6 În vremea aceea a ie?it regele Ioram din Samaria ?i a numarat tot Israelul;
\par 7 Iar dupa aceea s-a dus la Iosafat, regele Iudei, sa-i zica: "Regele Moabului s-a razvratit asupra mea. Vrei sa mergi cu mine la razboi împotriva Moabului?" Iar acesta a zis: "Merg. Cum e?ti tu, a?a sunt ?i eu; cum este poporul tau, a?a este ?i al meu ?i cum sunt caii tai, a?a sunt ?i ai mei!"
\par 8 Apoi a zis: "Pe ce drum sa mergem?" Iar el a raspuns: "Pe calea pustiului Edomului".
\par 9 ?i a plecat regele lui Israel ?i regele Iudei ?i regele Edomului ?i au înconjurat cale de ?apte zile; dar nu era apa pentru o?tire ?i pentru vitele ce veneau în urma.
\par 10 Atunci a zis regele lui Israel: "Ah, iata a chemat Domnul pe ace?ti trei regi ca sa-i dea în mâinile lui Moab".
\par 11 Iar Iosafat a zis: "Nu este oare pe aici vreun prooroc al Domnului, ca sa întrebam pe Domnul prin el?" ?i auzind, unul din slujitorii regelui lui Israel a zis: "Este aici Elisei, fiul lui Safat, care turna apa pe mâini lui Ilie".
\par 12 A zis Iosafat: "El are cuvântul Domnului". ?i s-au dus la el regele lui Israel ?i regele Iudei ?i regele Edomului.
\par 13 Atunci a zis Elisei catre regele lui Israel: "Ce poate fi între mine ?i tine? Du-te la proorocii tatalui tau ?i la proorocii mamei tale!" Iar regele lui Israel a zis catre el: "Ba nu, caci Domnul a chemat aici pe ace?ti trei regi ca sa-i dea în mâinile lui Moab".
\par 14 Iar Elisei a zis: "Pe cât este de adevarat ca Domnul Savaot, Caruia slujesc, este viu, tot a?a este de adevarat ca de nu a? cinsti pe Iosafat, regele Iudei, nici nu m-a? uita la tine ?i nici nu te-a? vedea!
\par 15 Acum însa chema?i-mi un cântare?!" ?i daca a început acesta a cânta, s-a atins mâna Domnului de Elisei
\par 16 ?i acesta a zis: "A?a graie?te Domnul: Face?i în valea aceasta ?an?uri.
\par 17 Caci a?a zice Domnul: Nu ve?i vedea vânt, nici ploaie nu ve?i vedea, dar valea aceasta se va umplea de apa, din care ve?i bea ?i voi ?i vitele voastre cele mici ?i cele mari.
\par 18 Însa acesta este pu?in lucru în ochii Domnului. El ?i pe Moab îl va da în mâinile voastre;
\par 19 ?i ve?i bate toate ceta?ile cele întarite ?i toate ceta?ile însemnate, to?i copacii cei mai buni îi ve?i taia ?i toate izvoarele de apa le ve?i astupa ?i toate ogoarele cele mai bune le ve?i strica cu pietre".
\par 20 Diminea?a însa, când se înal?a darul de pâine, deodata s-a revarsat apa pe drumul dinspre Edom ?i s-a umplut pamântul de apa.
\par 21 ?i când au auzit Moabi?ii ca vin regii sa se bata cu ei, s-au adunat to?i care erau în stare sa poarte arme, ba ?i cei mai batrâni ?i au stat da hotar.
\par 22 Iar diminea?a s-au sculat de noapte ?i, când a stralucit soarele deasupra apei, Moabi?ilor li s-a parut din departare ca apa aceea este ro?ie ca sângele.
\par 23 ?i au zis: "Acela este sânge. Regii aceia s-au luptat între ei ranindu-se unul pe altul. Acum la prada, Moabe!"
\par 24 ?i au venit ei spre tabara israelita. ?i s-au sculat Israeli?ii ?i au început a bate pe Moabi?i ?i ace?tia au fugit de ai, iar ei i-au urmarit mereu ?i au batut pe Moabi?i.
\par 25 Ceta?ile lor le-au darâmat ?i în toate ogoarele cele mai bune au aruncat fiecare cu pietre ?i le-au umplut cu pietre; toate izvoarele de apa le-au astupat ?i to?i copacii cei mai buni i-au taiat; apoi pra?tia?ii au înconjurat Chir-Hare?etul ?i l-au luat ?i n-au lasat decât numai pietrele.
\par 26 Atunci, vazând regele Moabului ca este biruit în razboi, a luat cu sine ?apte sute de oameni, deprin?i la mânuirea sabiei, ca sa patrunda la regele Edomului, dar n-a putut.
\par 27 Deci a luat pe fiul sau cel întâi nascut, care trebuia sa domneasca în locul lui, ?i l-a adus ardere de tot pe zid. Aceasta a pricinuit o mare mânie asupra Israeli?ilor ?i s-au dus de la el, întorcându-se în ?ara lor.

\chapter{4}

\par 1 În vremea aceea o femeie a unuia din fiii proorocilor striga catre Elisei, zicând: "Robul tau, barbatul meu a murit ?i tu ?tii ca robul tau era om cu temere de Domnul. Acum însa iata ca au venit datornicii sa ia. robi pe amândoi fiii mei!"
\par 2 Elisei însa i-a zis: "Ce sa-?i fac? Spune-mi ce ai tu în casa?" Iar ea a raspuns: "Roaba ta n-are în casa nimic, afara de un vas cu untdelemn".
\par 3 A zis Elisei: "Mergi ?i împrumuta vase din alta parte, pe la to?i vecinii tai.
\par 4 Ia vase goale cit de multe ?i apoi intra ?i-?i încuie u?a dupa tine ?i dupa fiii tai; apoi toarna untdelemn în toate vasele acelea ?i pe cele pline da-le la o parte".
\par 5 ?i ducându-se ea de la dânsul a încuiat u?a dupa sine ?i dupa fiii sai; ?i ace?tia îi dadeau vasele ?i ea le umplea.
\par 6 Iar daca s-au umplut vasele, a zis ea catre fiul sau: "Mai da-mi un vas". El însa a zis: "Nu mai sunt vase". ?i untdelemnul a încetat sa curga.
\par 7 ?i venind, ea a spus omului lui Dumnezeu, iar acesta a zis: "Du-te ?i vinde untdelemnul ?i-?i plate?te datoriile tale, iar cu ceea ce va ramâne, cu aceea vei trai tu cu fiii tai".
\par 8 Într-una din zile a venit Elisei la ?unem ?i acolo o femeie bogata l-a poftit la masa ?i dupa aceea, ori de câte ori trecea pe acolo, totdeauna se abatea sa manânce.
\par 9 ?i a zis aceasta catre barbatul sau: "Eu ?tiu ca omul lui Dumnezeu care trece mereu pe aici este sfânt;
\par 10 Sa-i facem dar un mic foi?or sus ?i sa-i punem acolo un pat, o masa, un scaun ?i un sfe?nic ?i, când va veni pe la noi, sa se duca acolo".
\par 11 Venind deci Elisei într-o zi acolo ?i intrând în foi?or, s-a culcat acolo.
\par 12 ?i a zis catre Ghehazi, sluga sa: "Cheama pe ?unamiteanca aceasta!" ?i a chemat-o ?i ea a stat înaintea lui.
\par 13 Apoi a zis lui Ghehazi: "Zi-i: Iata tu te îngrije?ti atâta de noi. Ce sa-?i facem? Nu cumva ai nevoie sa vorbim pentru tine cu regele sau cu capetenia o?tirii?" Iar ea a zis: "Nu, caci traiesc în pace în mijlocul poporului meu".
\par 14 Zis-a iara?i Elisei catre Ghehazi: "Atunci ce sa-i fac?" Iar Ghehazi a raspuns: "Iata, n-are nici un copil ?i barbatul ei este batrân".
\par 15 ?i a zis Elisei: "Cheam-o încoace!" ?i a chemat-o ?i ea a stat în u?a.
\par 16 Iar Elisei a zis: "La anul pe vremea asta vei ?ine în bra?e un fiu". Ea a raspuns: "Nu, omule al lui Dumnezeu ?i stapânul meu, nu amagi pe roaba ta!"
\par 17 Dar femeia aceea a zamislit ?i în anul urmator, chiar pe vremea aceea, a nascut un fiu, dupa cum îi spusese Elisei.
\par 18 Crescând acel copii, s-a dus întruna din zile la tatal sau, la seceratori.
\par 19 ?i a zis catre tatal sau: "Vai, ma doare capul!" Iar acesta a zis catre o sluga: "Du-l la mama lui!"
\par 20 ?i l-a luat ?i l-a dus la mama lui. ?i a stat pe bra?ele ei pâna la amiaza ?i a murit.
\par 21 Atunci ea s-a dus ?i l-a pus în patul omului lui Dumnezeu ?i l-a încuiat acolo ?i a ie?it.
\par 22 Apoi a chemat pe barbatul sau ?i a zis: "Trimite-mi o sluga ?i o asina, caci ma duc pâna la omul lui Dumnezeu ?i ma întorc îndata".
\par 23 Barbatul a zis: "De ce sa te duci la el? Astazi nu este nici luna noua, nici zi de odihna". Ea a zis: "Fii pe pace!"
\par 24 ?i punând ?aua pe asina, a zis catre sluga sa: "Ia-o ?i porne?te, dar sa nu ma opre?ti din mers pâna nu-?i voi spune eu!"
\par 25 ?i pornind, s-a dus la omul lui Dumnezeu, în muntele Carmelului. ?i când a vazut-o omul lui Dumnezeu din departare, a zis catre sluga sa Ghehazi: "Asta este ?unamiteanca aceea!
\par 26 Alearga întru întâmpinarea ei ?i întreab-o: E?ti sanatoasa? ?i barbatul tau e sanatos? ?i copilul tau e sanatos?"
\par 27 ?i ea a raspuns: "Sunt sanato?i!" Iar daca a ajuns pe munte la omul lui Dumnezeu, s-a apucat de picioarele lui. Atunci Ghehazi a venit sa o dea la o parte; dar omul lui Dumnezeu i-a zis: "Las-o, caci este cu sufletul amarât ?i Domnul a ascuns aceasta de mine ?i nu mi-a aratat".
\par 28 Iar ea a zis: "Au cerut-am eu fiu de la domnul meu? N-am zis eu oare, nu amagi pe roaba ta?"
\par 29 Atunci Elisei a zis catre Ghehazi: "Încinge-?i mijlocul tau, ia toiagul meu în mâna ta ?i du-te; de vei întâlni pe cineva, sa nu-i dai buna ziua, sa nu-i raspunzi, ?i sa pui toiagul meu pe fa?a copilului!"
\par 30 Iar mama copilului a zis: "Pe cât este de adevarat ca Domnul este viu, cum este viu ?i sufletul tau, tot a?a este de adevarat ca nu te voi lasa!"
\par 31 Atunci el s-a sculat ?i s-a dus dupa dânsa. Ghehazi însa s-a dus înaintea lor ?i a pus toiagul pe fa?a copilului; dar n-a fost nici glas, nici raspuns. ?i a ie?it Ghehazi întru întâmpinarea lui Elisei ?i i-a spus, zicând: "Copilul nu se treze?te!"
\par 32 Intrând Elisei în casa, a vazut copilul mort, întins în patul sau.
\par 33 ?i dupa ce a intrat, a încuiat u?a dupa sine ?i s-a rugat Domnului.
\par 34 Apoi s-a ridicat ?i s-a culcat peste copil ?i ?i-a pus buzele sale pe buzele lui ?i ochii sai pe ochii lui ?i palmele sale pe palmele lui ?i s-a întins pe el ?i a încalzit trupul copilului.
\par 35 Sculându-se apoi, Elisei s-a plimbat prin foi?or înainte ?i înapoi. Dupa aceea s-a dus ?i s-a întins iar peste copil. ?i a stranutat copilul de ?apte ori ?i ?i-a deschis copilul ochii.
\par 36 Atunci a chemat pe Ghehazi ?i i-a zis: "Cheama pe ?unamiteanca aceea!" ?i acela a chemat-o ?i, venind ea, i-a zis: "Ia-?i copilul!"
\par 37 Iar ea apropiindu-se, a cazut la picioarele lui ?i s-a închinat pâna la pamânt. Apoi ?i-a luat copilul ?i a ie?it.
\par 38 Iar Elisei s-a întors la Ghilgal ?i era foamete în pamântul acela ?i fiii proorocilor ?edeau înaintea lui. ?i a zis Elisei slugii sale: "Pune caldarea cea mare ?i fierbe ceva pentru fiii proorocilor".
\par 39 Iar unul dintr-în?ii, ie?ind la câmp sa adune verde?uri, a gasit o buruiana salbatica, aga?atoare, a cules din ea roade o poala plina ?i, venind, le-a aruncat în caldarea cu fiertura, fara sa le cunoasca.
\par 40 Apoi le-a dat sa manânce. Dar îndata ce au început a mânca, au strigat: "Omul lui Dumnezeu, în caldare este moarte!" ?i n-au mai putut sa manânce.
\par 41 Iar el a zis: "Aduce?i faina!" ?i a presarat-o în caldare ?i a zis catre Ghehazi: "Da oamenilor sa manânce!" ?i n-a mai ramas în caldare nimic vatamator.
\par 42 Tot pe atunci a venit un om din Baal-?ali?a ?i a adus omului lui Dumnezeu pârga de pâine douazeci de pâini de orz ?i graun?e proaspete de grâu într-un sacu?or. Elisei a zis: "Da?i oamenilor sa manânce!"
\par 43 Iar sluga sa a zis: "Ce sa dau eu de aici la o suta de oameni?" Zis-u Elisei: "Da oamenilor sa manânce; caci a?a zice Domnul: Se vor satura ?i va mai ?i ramâne".
\par 44 ?i a dat ?i s-au saturat ?i a mai ramas, dupa cuvântul Domnului.

\chapter{5}

\par 1 Neeman, capetenia o?tirii regelui Siriei, era om de seama ?i cu trecere înaintea stapânului sau, pentru ca prin Domnul daduse biruin?a Sirienilor. Dar acest osta? vrednic ara lepros.
\par 2 O data Sirienii au ie?it în cetate ?i între altele luasera din pamântul lui Israel o copila, care acum slujea la femeia lui Neeman.
\par 3 Aceasta a zis catre stapâna sa: "O, daca stapânul meu s-ar duce la proorocul cel din Samaria, de buna seama s-ar tamadui de lepra".
\par 4 Atunci s-a dus Neeman ?i a spus aceasta stapânului sau: "A?a ?i a?a zice feti?a cea din pamântul lui Israel".
\par 5 Iar regele Siriei a zis catre Neeman: "Scoala ?i du-te ?i voi trimite ?i eu scrisoare regelui lui Israel". ?i s-a dus Neeman, luând cu sine zece talan?i de argint, ?ase mii de sicli de aur ?i zece rânduri de haine;
\par 6 ?i a dus regelui lui Israel scrisoarea în care zicea: "Împreuna cu scrisoarea aceasta, trimit pe Neeman, sluga mea, ca sa cure?i lepra de pe el".
\par 7 Când a citit regele lui Israel scrisoarea, ?i-a rupt hainele sale ?i a zis: "Au doara eu sunt Dumnezeu, ca sa omor ?i sa fac viu, de trimite el la mine, ea sa vindec pe acest om de lepra? Iata acum sa vede?i ?i sa ?ti?i ca el cauta pricina de du?manie împotriva mea".
\par 8 Când însa a auzit Elisei, omul lui Dumnezeu, ca regele lui Israel ?i-a rupt hainele sale, a trimis sa i se spuna regelui: "Pentru ce ji-ai rupt tu hainele tale? Lasa-l sa vina la mine ?i vor cunoa?te ca este prooroc în Israel".
\par 9 ?i a venit Neeman cu caii ?i cu caru?a sa, oprindu-se la poarta casei lui Elisei.
\par 10 Iar Elisei a trimis la el pe sluga sa sa-i zica: "Du-te ?i te scalda de ?apte ori în Iordan, ca ?i se va înnoi trupul tau ?i vei fi curat!"
\par 11 Neeman însa s-a mâniat ?i a plecat, zicând: "Iata, socoteam ca va ie?i el ?i, stând la rugaciune, va chema numele Domnului Dumnezeului sau, î?i va pune mâna pe locul bolnav ?i va cura?i lepra.
\par 12 Au doara Abana ?i Farfar, râurile Damascului, nu sunt ele mai bune decât toate apele lui Israel? Nu puteam eu oare sa ma scald în ele ?i sa ma cura?? ?i a?a s-a întors ?i a plecat mânios.
\par 13 Dar slugile lui, apropiindu-se, i-au grait ?i i-au zis: "Stapâne, daca proorocul ?i-ar fi zis sa faci ceva însemnat, oare n-ai fi facut? Cu atât mai vârtos trebuie sa faci când ?i-a zis numai: Spala-te ?i vei fi curat!"
\par 14 Atunci el s-a coborât ?i s-a cufundat de ?apte ori în Iordan, dupa cuvântul omului lui Dumnezeu, ?i i s-a înnoit trupul ca trupul unui copil mic ?i s-a cura?it.
\par 15 Atunci s-a întors la omul lui Dumnezeu cu to?i cei ce-l înso?eau ?i, venind, a stat înaintea lui ?i a zis: "Iata am cunoscut ca în tot pamântul nu este Dumnezeu decât numai în Israel! Deci, ia un dar de la robul tau!"
\par 16 Iar Elisei a zis: "Pe cât este de adevarat ca Domnul, înaintea Caruia slujesc, este viu, tot atât este de adevarat ca nu voi primi".
\par 17 Acela însa îl silea sa primeasca, dar el n-a vrut. Atunci a zis Neeman: "Daca nu, atunci sa sa dea robului tau pamânt cât pot duce doi catâri, pentru ca de aici înainte robul tau nu va mai aduce arderi de tot ?i jertfe la al?i dumnezei afara de Domnul.
\par 18 Numai iata ce sa ierte Domnul robului tau: Când va merge stapânul meu în templul lui Rimon, ca sa se închine acolo, ?i se va sprijini de mâna mea, ?i ma voi închina ?i eu în templul lui Rimon, atunci, pentru închinarea mea în templul lui Rimon, sa ierte Domnul pe robul tau la asemenea întâmplare".
\par 19 Elisei l-a raspuns: "Mergi în pace!" ?i el s-a dus.
\par 20 Iar daca s-a departat pu?in, Ghehazi, sluga lui Elisei, omul lui Dumnezeu, ?i-a zis: "Iata, stapânul meu n-a vrut sa ia din mâna acestui Neeman Sirianul ceea ce i-a adus. Pe cât de adevarat ca Domnul este viu, a?a este de adevarat ca voi alerga dupa el ?i voi lua de la el ceva".
\par 21 ?i a alergat Ghehazi dupa Neeman, iar Neeman, vazându-l alergând dupa el, s-a coborât din caru?a înaintea lui ?i a zis: "Cu pace vii?"
\par 22 ?i Ghehazi a raspuns: "Cu pace. M-a trimis stapânul meu sa-îi spun: Iata au venit acum la mine din muntele lui Efraim doi tineri din fiii proorocilor; da-le un talant de argint ?i doua rânduri de haine".
\par 23 ?i a zis Neeman: "Ia, rogu-te, doi talan?i de argint". ?i l-a rugat, ?i s legat doi talan?i de argint în doi saci ?i doua rânduri de haine ?i le-a dat la doua slugi care le-a dus înaintea lui.
\par 24 Iar daca au ajuns sub deal, le-a luat din mâinile lor ?i le-a ascuns în casa. Apoi a dat drumul oamenilor de s-au dus.
\par 25 Când însa a venit ?i s-a aratat stapânului sau, Elisei i-a zis: "De unde vii Ghehazi?" ?i El a raspuns: "Robul tau n-a fost nicaieri".
\par 26 Iar Elisei i-a zis: "Au doar inima mea nu te-a întovara?it când omul acela s-a dat jos din caru?a ?i a venit în întâmpinarea ta? Este timpul oare sa iau argint ?i haine, maslini ?i vii, vite mari sau marunte, robi sau roabe?
\par 27 Sa se lipeasca dar lepra lui Neeman de tine ?i de urma?ii tai în veci". ?i a ie?it Ghehazi de la Elisei alb de lepra ca zapada.

\chapter{6}

\par 1 Atunci au zis fiii proorocilor catre Elisei: "Iata, locul unde traim aici la tine e strâmt pentru noi.
\par 2 Sa mergem dar la Iordan ?i sa luam de acolo fiecare câte o bârna ?i sa ne facem locuin?a acolo".
\par 3 ?i el a zis: "Duce?i-va!" Iar unul a zis: "Fa mila ?i mergi ?i tu cu robii tai!" ?i el a zis: "Merg!" ?i s-a dus cu ei.
\par 4 ?i ajungând la Iordan, s-au apucat de taiat copaci.
\par 5 ?i când a pravalit unul o bârna, i-a cazut toporul în apa ?i a strigat acela ?i a zis: "Ah, stapânul meu! Acesta îl luasem împrumut!"
\par 6 A zis omul lui Dumnezeu: "Unde a cazut?" ?i acela i-a aratat locul. Iar Elisei a taiat o bucata de lemn ?i, aruncând-o acolo, a ie?it toporul deasupra. Apoi a zis: "Ia-?i-l!"
\par 7 ?i acela a întins mâna ?i l-a luat.
\par 8 În vremea aceea s-a ridicat regele Siriei cu razboi împotriva Israeli?ilor ?i s-a sfatuit cu slujitorii sai zicând: "Am sa a?ez tabara în cutare sau în cutare loc".
\par 9 ?i a trimis omul lui Dumnezeu la regele lui Israel sa i se spuna: "Paze?te-te de a trece prin locul acela; caci acolo s-au ascuns Sirienii".
\par 10 ?i a trimis regele lui Israel la locul acela de care-i graise omul lui Dumnezeu ?i-i spusese sa se fereasca ?i s-a pazit de el mereu.
\par 11 ?i s-a nelini?tit inima regelui Siriei de întâmplarea aceasta ?i, chemând pe slujitorii sai, a zis: "Spune?i-mi care din ai vo?tri este în legatura cu regele lui Israel?"
\par 12 ?i raspunzând unul din slujitori, a zis: "Nimeni, stapânul meu rege. Dar Elisei proorocul, pe care-l are Israel, spune regelui lui Israel pâna ?i cuvintele ce le graie?ti tu în odaia ta de culcare".
\par 13 A zis regele: "Duce?i-va ?i afla?i unde este, caci am sa trimit sa-l iau". ?i i s-a spus: "Iata, este în Dotan".
\par 14 ?i a trimis acolo cai ?i care ?i mul?ime de o?tire, care au venit noaptea ?i au împresurat cetatea.
\par 15 Iar diminea?a slujitorul omului lui Dumnezeu, sculându-se ?i ie?ind a vazut o?tirea împrejurul ceta?ii, ?i caii ?i carele; ?i a zis slujitorul sau catre el: "Vai, stapânul meu! Ce sa facem?"
\par 16 El însa i-a zis: "Nu te teme, pentru ca cei ce sunt cu noi sunt mai numero?i decât cei ce sunt cu ei".
\par 17 ?i s-a rugat Elisei ?i a zis: "Doamne, deschide-i ochii ca sa vada!" ?i a deschis Domnul ochii slujitorului ?i acesta a vazut ca tot muntele era plin de cai ?i care de foc împrejurul lui Elisei.
\par 18 Iar când au venit asupra lui Sirienii, Elisei s-a rugat Domnului ?i a zis: "Love?te-i cu orbire!" ?i Domnul i-a orbit, dupa cuvântul lui Elisei.
\par 19 Apoi Elisei le-a zis: "Nu este acesta drumul ?i nici cetatea nu este aceasta. Dar veni?i dupa mine ?i eu va voi duce la omul acela pe care-l cauta?i!" ?i i-a dus în Samaria.
\par 20 Iar la intrarea lor în Samaria, Elisei a zis: "Doamne, deschide-ne ochii ca sa vada!" ?i le-a deschis Domnul ochii ?i au vazut ca sunt în Samaria.
\par 21 ?i vazându-i, regele lui Israel a zis catre Elisei: "Sa-i ucid, parinte?"
\par 22 Iar Elisei a zis: "Sa nu-i ucizi. Au doara cu arcul tau ?i cu sabia ta i-ai prins ca sa-i ucizi? Da-le pâine ?i apa, sa manânce ?i sa bea ?i apoi sa se duca la domnul lor".
\par 23 ?i le-a facut masa mare ?i ei au mâncat ?i au baut. Dupa aceea le-a dat drumul ?i s-au dus la stapânul lor. ?i n-au mai venit taberele siriene în ?ara lui Israel.
\par 24 Dupa aceea a adunat Benhadad, regele Siriei, toata o?tirea sa ?i a venit ?i a împresurat Samaria.
\par 25 ?i a fost în Samaria foamete mare în timpul împresurarii lor, încât un cap de asin se vindea cu optzeci de sicli de argint ?i un sfert de capa?âna de ceapa salbatica se vindea cu cinci sicli de argint.
\par 26 ?i trecând odata regele lui Israel pe zid, o femeie, plângând, i-a zis: "Ajuta-ma, domnul meu rege!"
\par 27 ?i regele i-a zis: "De nu te va ajuta Domnul, cu ce te pot ajuta eu? Au doara eu vin de la arie sau de la teascuri?" Apoi i-a zis regele: "Ce ai?" Iar ea a raspuns:
\par 28 "Iata, femeia aceasta mi-a zis: Da pe fiul tau sa-l mâncam astazi, iar pe fiul meu îl vom mânca mâine.
\par 29 ?i a?a am fiert noi pe fiul meu ?i l-am mâncat; ?i a doua zi am zis catre ea: Da acum pe fiul tau sa-l mâncam. Ea însa a ascuns pe fiul sau".
\par 30 ?i auzind regele cuvintele femeii, ?i-a rupt hainele ?i apoi a trecut înainte pe zidul ceta?ii, iar poporul a vazut ca pe dinauntru trupul sau era îmbracat în haina de jale.
\par 31 Apoi a zis regele: "Sa ma pedepseasca Dumnezeu cu toata asprimea daca va ramâne azi capul lui Elisei, fiul lui Safat, pe umeri".
\par 32 Elisei însa ?edea în casa sa ?i batrânii ?edeau împreuna cu el. ?i a trimis regele un om al sau; dar înainte de a veni cel trimis la el, el a zis catre batrâni: "?ti?i voi oare ca acest fiu de uciga? a trimis sa mi se taie capul? Baga?i de seama; când va veni trimisul lui, sa încuia?i u?a ?i sa-l opri?i la u?a. Dar iata ?i zgomotul pa?ilor stapânului sau se aude în urma lui!"
\par 33 ?i înca graind el cu ei, a venit la el trimisul ?i a zis: "Iata, ce necaz a venit de la Domnul! Ce sa mai a?teptam acum de la Domnul?"

\chapter{7}

\par 1 Iar Elisei a zis: "Asculta?i cuvântul Domnului! A?a zice Domnul: Mâine, pe vremea aceasta, în poarta Samariei o masura de faina din cea mai buna va fi un siclu ?i doua masuri de orz tot un siclu".
\par 2 Slujba?ul de al carui bra? se sprijinea regele a raspuns: "Chiar daca Domnul ar deschide ferestrele cerului nici atunci n-ar putea fi una ca aceasta!" Iar Elisei a zis: "Iata a?a ai sa vezi cu ochii tai, dar tu nu vei mânca din acelea!"
\par 3 În vremea aceasta la poarta Samariei se aflau patru oameni lepro?i ?i ziceau unul catre altul: "Ce ?edem noi aici ?i a?teptam moartea?
\par 4 De ne vom hotarî sa ne ducem în cetate, în cetate e foamete ?i vom muri acolo; iar daca vom ?edea aici, tot vom muri. Sa ne ducem mai bine în tabara Sirienilor! De ne vor lasa cu via?a, vom trai, de nu, vom muri".
\par 5 ?i s-au sculat ei în amurg, ca sa se duca în tabara Sirienilor. Dar când au ajuns la marginea taberei Sirienilor, acolo nu mai era nici un om;
\par 6 Caci Domnul facuse în tabara Sirienilor sa se auda zgomot de care ?i nechezat de cai ?i zgomot de o?tire mare. ?i au zis ei unul catre altul: "De buna seama, regele lui. Israel a tocmit sa vina împotriva noastra pe regii Heteilor ?i ai Egiptului".
\par 7 ?i s-au sculat ?i au fugit în amurg ?i ?i-au lasat corturile, caii, asinii lor ?i toata tabara cum era ?i au fugit, ca sa-?i scape via?a.
\par 8 Ajungând deci lepro?ii aceia la marginea taberei, au intrat într-un cort ?i au mâncat ?i au baut ?i au luat de acolo argint ?i aur ?i haine ?i s-au dus de le-au ascuns. ?i s-au mai dus ?i în alt cort ?i au luat ?i de acolo ?i au ascuns.
\par 9 Apoi au zis unul catre altul: "Ceea ce facem, nu facem bine. Ziua aceasta este zi de veste buna. Daca întârziem ?i a?teptam lumina zilei atunci vina va cadea asupra noastra. Hai deci sa vestim casa regelui!"
\par 10 ?i au venit ei ?i au strigat pe portarii ceta?ii ?i le-au povestit, zicând: "Noi am fost în tabara Sirienilor ?i iata acolo nu se vedea nimeni ?i nu se auzea nimic, ci numai cai lega?i ?i corturi cum trebuie sa fie".
\par 11 ?i portarii au strigat ?i au dat de veste la casa regelui.
\par 12 ?i s-a sculat regele noaptea ?i a zis slugilor sale: "Am sa va spun ce fac Sirie cu noi: Ei ?tiu ca noi suferim de foame ?i au ie?it din tabara ?i s-au ascuns în câmp, cugetând a?a: Când vor ie?i ei din cetate, îi vom prinde vii ?i vom navali în cetate".
\par 13 Dar unul din cei ce slujeau înaintea lui a raspuns ?i a zis: "Sa se ia cei cinci cai rama?i care mai sunt în cetate (din toata tabara lui Israel numai atâta mai ramasese; cealalta tabara a lui Israel pierise toata) ?i sa trimitem oameni sa vada".
\par 14 Au luat deci doua perechi de cai înhama?i la caru?e ?i a trimis regele pe urma o?tirii siriene, zicând: "Duce?i-va ?i vede?i!"
\par 15 ?i s-au dus dupa ei pâna la Iordan ?i iata tot drumul era semanat cu haine ?i cu lucruri aruncate de Sirieni în graba lor. ?i s-au întors trimi?ii ?i au spus regelui.
\par 16 Atunci a ie?it poporul ?i a pradat tabara siriana ?i a fost masura de faina din cea mai buna un siclu ?i doua masuri de orz un siclu, dupa cuvântul Domnului.
\par 17 Iar regele a luat pe slujitorul acela de mâna caruia se sprijinea ?i l-a rânduit de paza la poarta ceta?ii; acesta a fost calcat în picioare de popor ?i a murit la poarta, cum zisese omul lui Dumnezeu când venise trimisul regelui la el;
\par 18 Pentru ca atunci când omul lui Dumnezeu a spus regelui a?a: "Mâine pe vremea aceasta în poarta Samariei doua masuri de orz vor fi un siclu ?i o masura de faina din cea mai buna va fi tot un siclu",
\par 19 Atunci acest slujitor a raspuns omului lui Dumnezeu ?i a zis: "Chiar daca Domnul va deschide ferestrele cerului, nici atunci n-ar putea fi una ca aceasta!" Iar Elisei i-a zis: "Vei vedea-o cu ochii tai, dar nu vei mânca din ea!"
\par 20 ?i a?a s-a întâmplat cu el: l-a calcat poporul în poarta ?i a murit.

\chapter{8}

\par 1 În vremea aceea a grait Elisei cu femeia careia îi înviase copilul ?i i-a zis: "Scoala ?i du-te de la casa ta ?i traie?te unde vei putea, caci Domnul a chemat foametea ?i aceea va veni asupra pamântului pentru ?apte ani".
\par 2 ?i s-a sculat femeia aceea ?i a facut dupa cuvântul omului lui Dumnezeu ?i s-a dus ea ?i casa sa ?i a trait în pamântul Filistenilor ?apte ani.
\par 3 Iar dupa trecerea celor ?apte ani s-a întors femeia aceea din pamântul Filistenilor ?i a venit sa roage pe rege pentru casa sa ?i pentru ?arina sa.
\par 4 Tocmai atunci regele vorbea cu Ghehazi, sluga omului lui Dumnezeu, ?i a zis: "Poveste?te-mi tot ce este mai însemnat din câte a facut Elisei!"
\par 5 ?i pe când istorisea el regelui despre copilul înviat de Elisei, a rugat pe rege pentru casa sa ?i pentru ?arina sa. ?i a zis Ghehazi: "Stapânul meu rege, aceasta este chiar femeia aceea ?i el este acel fiul al ei pe care l-a înviat Elisei".
\par 6 ?i a întrebat regele ?i pe femeie ?i i-a povestit ?i ea. Atunci regele i-a dat pe unul de la curte, zicând: "Sa i se întoarca toate câte sunt ale ei ?i toate veniturile ?arinei din ziua când ea a parasit ?arina pâna acum".
\par 7 ?i a venit Elisei în Damasc, când Benhadad, regele Siriei, era bolnav. ?i i s-a spus acestuia ?i i s-a zis: "A venit aici omul lui Dumnezeu".
\par 8 Iar regele a zis catre Hazael: "Ia în mâna ta un dar ?i du-te în întâmpinarea omului lui Dumnezeu ?i întreaba pe Domnul prin el, zicând: Ma voi însanato?i eu, oare, de boala aceasta?"
\par 9 ?i s-a dus Hazael în întâmpinarea lui ?i a luat dar în mâna sa din cele mai bune lucruri din Damasc cât pot duce patruzeci de camile ?i a venit ?i a stat înaintea fe?ei lui ?i a zis: "Benhadad, fiul tau, regele Siriei, m-a trimis la tine sa întreb: Ma voi însanato?i eu, oare, de. boala aceasta?"
\par 10 Iar Elisei i-a zis: "Du-te ?i spune-i: "Te vei însanato?i!" Cu toate acestea, Domnul mi-a descoperit ca el va muri".
\par 11 ?i ?i-a îndreptat Elisei privirile spre Hazael ?i l-a privit mult, apoi a plâns omul lui Dumnezeu.
\par 12 ?i a zis Hazael. "De ce plânge domnul meu?" ?i el a zis: "Pentru ca ?tiu ce rau ai sa faci tu fiilor lui Israel; taria lor o vei da focului, pe tinerii lor cu sabia îi vei ucide, pe copiii lor de sân îi vei omorî ?i pe cele însarcinate ale lor le vei taia".
\par 13 A zis Hazael: "Robul tau este câine ca sa faca asemenea lucru necugetat?" ?i Elisei a zis: "Domnul mi-a aratat în tine pe regele Siriei".
\par 14 ?i s-a dus Hazael de la Elisei ?i a venit la domnul sau ?i acesta i-a zis: "Ce ?i-a grait Elisei?" Iar el a zis: "Mi-a grait ca ai sa te faci sanatos".
\par 15 A doua zi însa a luat o învelitoare, a udat-o cu apa, a pus-o pe fa?a lui ?i a murit regele. ?i în locul lui s-a facut rege Hazael.
\par 16 În anul al cincilea al lui Ioram, fiul lui Ahab, regele lui Israel, în locul lui Iosafat, regele Iudei, s-a facut rege Ioram, fiul lui Iosafat, regele Iudei.
\par 17 Acesta era de treizeci ?i doi de ani când s-a facut rege.
\par 18 El a domnit în Ierusalim opt ani ?i a pa?it pe urma regilor lui Israel, cum se purtase casa lui Ahab, pentru ca fata lui Ahab era so?ia lui Iosafat ?i a facut lucruri rele în ochii Domnului.
\par 19 Cu toate acestea Domnul n-a vrut sa piarda pe Iuda pentru David, robul Sau, deoarece îi fagaduise ca îi va da totdeauna o faclie printre fiii sai.
\par 20 În zilele lui Ioram a ie?it Edom de sub mâna lui Iuda ?i ei ?i-au pus rege.
\par 21 ?i s-a dus Ioram la ?air ?i împreuna cu el s-au dus toate carele lui. Apoi s-a sculat el noaptea ?i a lovit pe Edomi?ii care-l împresurasera, ?i pe capeteniile carelor, dar poporul a fugit în corturile sale.
\par 22 Edom a ie?it de sub mâna lui Iuda ?i a ramas a?a pâna în ziua de astazi.
\par 23 ?i tot atunci a ie?it ?i Libna. Celelalte fapte pe care le-a facut Ioram, sunt scrise în cartea faptelor regilor lui Iuda.
\par 24 ?i a raposat Ioram cu parin?ii sai ?i a fost îngropat cu parin?ii sai în cetatea lui David, iar în locul lui s-a facut rege Ohozia, fiul sau.
\par 25 În anul al doisprezecelea al lui Ioram, fiul lui Ahab, regele lui Israel, s-a facut rege în Iuda Ohozia, fiul lui Ioram, regele Iudei.
\par 26 Ohozia era de douazeci ?i doi de ani când s-a facut rege ?i a domnit un an în Ierusalim. Numele mamei lui era Atalia, fiica lui Omri, regele lui Israel.
\par 27 Ohozia a pa?it pe urmele casei lui Ahab ?i a facut rele în ochii Domnului, ca ?i casa lui Ahab, pentru ca era înrudit cu casa lui Ahab.
\par 28 ?i s-a dus el cu Ioram, fiul lui Ahab, la razboi împotriva lui Hazael, regele Siriei, la Ramot în Galaad. Atunci au ranit Sirienii pe Ioram.
\par 29 ?i regele Ioram s-a întors ca sa se vindece în Izreel de rana ce i-o pricinuisera Sirienii la Ramot când se luptase cu Hazael, regele Siriei. Iar Ohozia, fiul lui Ioram, regele Iudei, a venit sa-l vada pe Ioram, fiul lui Ahab, în Izreel, caci acesta era bolnav.

\chapter{9}

\par 1 Atunci Elisei proorocul a chemat pe unul dintre fiii proorocilor ?i i-a zis: "Încinge?i mijlocul ?i ia acest vas cu untdelemn în mâna ta ?i du-te la Ramot în Galaad.
\par 2 Iar daca vei ajunge acolo, cauta pe Iehu, fiul lui Iosafat, fiul lui Nim?i, ?i, apropiindu-te, porunce?te-i sa iasa dintre fra?ii lui ?i du-l în camara cea mai dinauntru.
\par 3 Apoi ia vasul cu untdelemn ?i toarna pe capul lui ?i zi: A?a zice Domnul: Iata te ung rege peste Israel. Dupa aceea deschide u?a ?i fugi fara zabava".
\par 4 ?i s-a dus tânarul, sluga proorocului, la Ramot în Galaad.
\par 5 ?i ajungând acolo, a vazut pe capeteniile o?tirii ?ezând ?i a zis: "Am un cuvânt catre tine, capetenie!" ?i a zis Iehu: "Catre care din noi to?i?" Iar el a zis: "Catre tine, capetenie!"
\par 6 Atunci s-a sculat acesta ?i a intrat în casa. Iar tânarul a turnat untdelemn pe capul lui ?i i-a zis: "A?a zice Domnul Dumnezeul lui Israel: Te ung rege peste poporul Domnului, peste Israel.
\par 7 Tu vei nimici casa lui Ahab, stapânul tau, de la fa?a Mea ?i vei razbuna asupra Izabelei sângele robilor Mei, proorocii, ?i sângele tuturor slujitorilor Domnului.
\par 8 Toata casa lui Ahab va pieri ?i voi stârpi din ai lui Ahab pe tot cel de parte barbateasca, pe cel rob ?i pe cel slobod în Israel;
\par 9 Voi face casa lui Ahab ca ?i casa lui Ieroboam, fiul lui Nabat ?i ca ?i casa lui Bae?a, fiul lui Ahia.
\par 10 Iar pe Izabela, câinii o vor mânca în câmpia lui Izreel ?i nimeni nu o va îngropa". Apoi tânarul a deschis u?a ?i a fugit.
\par 11 Dupa aceea a ie?it Iehu la slujitorii domnului sau ?i ace?tia i-au zis: "E pace? De ce a venit acest necunoscut la tine?" ?i el i-a zis: "Voi cunoa?te?i pe acest om ?i de ce a venit". Iar ei au zis: "Nu este adevarat!
\par 12 Spune-ne!" ?i el le-a zis: "Iata el mi-a spus: A?a graie?te Domnul: Iata te ung rege peste Israel".
\par 13 Atunci ei s-au grabit sa-?i ia fiecare haina sa ?i i-au a?ternut-o pe trepte, au trâmbi?at ?i au zis: "Iehu s-a facut rege!"
\par 14 S-a sculat deci Iehu; fiul lui Iosafat, fiul lui Nim?i, împotriva lui Ioram. Ioram însa fusese cu to?i Israeli?ii la Ramot în Galaad ?i-l aparase împotriva lui Hazael, regele Siriei.
\par 15 Iar acum regele Ioram se întorsese la Izreel, ca sa se vindece de ranile ce i le pricinuisera Sirienii, când se luptase el cu Hazael, regele Siriei. ?i a zis Iehu: "Daca sunte?i de parerea mea, sa nu iasa nimeni din cetate, ca sa dea de veste la Izreel".
\par 16 Apoi a încalecat Iehu pe cal ?i s-a dus în Izreel, unde zacea Ioram, regele lui Israel ?i se îngrijea de ranile ce i le pricinuisera Sirienii la Ramot, în timpul luptei cu Hazael, regele Siriei, cel tare ?i puternic ?i unde venise Ohozia, regele Iudei, ca sa cerceteze pe Ioram.
\par 17 În turnul din Izreel statea un osta? de straja. Vazând acesta ceata lui Iehu venind, a zis: "Vad o ceata!" Iar Ioram a zis: "Ia un calare? ?i trimite-l în întâmpinare, ca sa întrebe: Cu pace vii?"
\par 18 ?i s-a dus calare?ul calare în întâmpinarea lui ?i a zis: "Cu pace vii?" Iar Iehu a zis: "Ce ai tu cu pacea? Treci în urma mea!" ?i straja a dat de veste, zicând: "A ajuns la ei, dar nu se întoarce":
\par 19 ?i a trimis alt calare? ?i acesta s-a dus la ei ?i a zis: "A?a zice regele: Cu pace vii?" Iar Iehu a zis: "Ce ai tu cu pacea? Treci în urma mea!"
\par 20 ?i a vestit straja, zicând: "A ajuns la ei, dar nu se întoarce. ?i mersul parc-ar fi al lui Iehu, fiul lui Nim?i, pentru ca merge nebune?te".
\par 21 Atunci Ioram a zis: "Înhama!" ?i au înhamat la carul lui: ?i a ie?it Ioram, regele lui Israel ?i Ohozia; regele Iudei, fiecare în carul sau, în întâmpinarea lui Iehu ?i s-au întâlnit cu el în ?arina lui Nabot Izreeliteanul.
\par 22 ?i când a vazut Ioram pe Iehu, a zis: "Cu pace, Iehu? Iar el a zis: "Ce pace, când sunt atâtea desfrânarile Izabelei, mama ta, ?i vrajitoriile ei?"
\par 23 ?i întorcându-se Ioram sa fuga, a zis catre Ohozia: "Vânzare, Ohozia!
\par 24 Iar Iehu ?i-a întins arcul cu mâna sa ?i a lovit pe Ioram între umeri ?i sageata a trecut prin inima lui ?i el a cazut în carul sau.
\par 25 ?i a zis Iehu catre Bidcar, capetenia o?tirii: "Ia-l ?i-l arunca în ?arina lui Nabot Izreeliteanul, caci adu-?i aminte ca atunci când mergeam eu ?i cu tine calari în urma lui Ahab, tatal sau, Domnul a rostit împotriva lui proorocia aceasta:
\par 26 Adevarat, am vazut ieri sângele fiilor lui, zice Domnul, ?i Ma voi razbuna pe tine în ?arina aceasta. Deci ia-l ?i-l arunca în ?arina, dupa cuvântul Domnului".
\par 27 Iar Ohozia, regele lui Iuda, vazând aceasta, a fugit pe drum spre casa ce se afla în gradina. Dar Iehu a alergat dupa el, zicând: "Ucide-?i-l ?i pe el în car!" ?i l-au lovit pe înal?imea Gur, care vine lânga Ibleam. ?i a fugit Ohozia la Moghido ?i a murit acolo.
\par 28 Iar slujitorii lui l-au dus la Ierusalim, l-au îngropat în mormânt cu parin?ii lui, în cetatea lui David.
\par 29 Ohozia se facuse rege în Iuda în anul al unsprezecelea al lui Ioram, fiul lui Ahab.
\par 30 Apoi Iehu a venit în Izreel. Iar Izabela, fiind în?tiin?ata de aceasta, ?i-a uns fa?a, ?i-a împodobit capul ?i privea de la fereastra.
\par 31 Iar când a intrat Iehu pe poarta, ea a zis: "Zimri, uciga?ul stapânului sau, va avea el oare pace?"
\par 32 ?i ridicându-?i el fa?a sa ?i privind spre fereastra, a zis: "Cine, cine este cu mine?" Atunci s-au plecat spre el doi sau trei eunuci;
\par 33 ?i el le-a zis: "Arunca?i-o jos!" ?i au aruncat-o ?i a ?â?nit sângele ei pe zid ?i pe caii care au calcat-o în picioare.
\par 34 Dupa aceea a venit Iehu de a mâncat ?i a baut ?i a zis: "Cauta?i pe ticaloasa aceea ?i îngropa?i-o, caci e fiica de rege!"
\par 35 ?i s-au dus sa o îngroape, dar n-au mai gasit nimic din ea, decât numai ?easta, picioarele ?i palmele mâinilor.
\par 36 ?i s-au întors ?i i-au spus. Iar el a zis: "A?a a fost cuvântul Domnului, rostit prin robul Sau Ilie Tesviteanul, zicând: În ?arina Izreel câinii vor mânca trupul Izabelei;
\par 37 ?i va fi trupul Izabelei în ?arina Izreel ca gunoiul pe ogor, încât nimeni nu va zice: Iata ea e Izabela!"

\chapter{10}

\par 1 Ahab avea în Samaria ?aptezeci de fii. ?i a scris Iehu scrisori ?i le-a trimis la Samaria capeteniilor ceta?ii, batrânilor ?i celor ce cre?teau pe copiii lui Ahab,
\par 2 Zicând: "Când va ajunge scrisoarea aceasta la voi, la care se afla ?i fiii domnului vostru, precum ?i care ?i cai, cetatea întarita ?i arme,
\par 3 Sa alege?i pe cel mai bun ?i mai vrednic dintre fiii domnului vostru ?i sa-l pune?i pe tronul tatalui sau ?i sa va lupta?i pentru casa domnului vostru".
\par 4 Ace?tia însa s-au speriat cumplit ?i au zis: "Iata, doi regi nu i-au putut sta înainte; cum dar vom sta noi?"
\par 5 ?i capetenia casei domne?ti, capeteniile ceta?ii, batrânii ?i cei ce cre?teau pe copiii regelui au trimis la Iehu sa-i spuna: "Noi suntem robii tai ?i ce ne vei zice, aceea vom face; nu vom pune pe nimeni rege, ci fa ceea ce-?i place".
\par 6 ?i le-a scris Iehu scrisoare a doua oara ?i le-a zis: "De sunte?i ai mei ?i de va supune?i cuvântului meu, atunci ridica?i capetele fiilor domnului vostru ?i veni?i la mine în Izreel, mâine pe vremea aceasta". Fiii regelui erau în numar de ?aptezeci ?i-i cre?teau oameni de vaza ai ceta?ii.
\par 7 Când a ajuns la ei scrisoarea, ei au luat pe fiii regelui ?i i-au junghiat pe to?i cei ?aptezeci, au pus capetele în panere ?i le-au trimis la el în Izreel.
\par 8 ?i venind trimisul, i-a adus ?tire ?i a zis: "S-au adus capetele fiilor regelui". Iar el a zis: "A?eza?i-le pâna diminea?a în doua gramezi la poarta".
\par 9 ?i ie?ind el diminea?a a stat ?i a zis catre tot poporul: "Voi nu sunte?i vinova?i. Iata eu m-am sculat împotriva stapânului meu ?i l-am ucis.
\par 10 Dar pe ace?tia cine i-a omorât? Sa ?ti?i deci ca nimic din ceea ce a spus Domnul împotriva casei lui Ahab n-a ramas neîmplinit. Domnul a facut ce zisese prin robul Sau Ilie".
\par 11 ?i a omorât Iehu pe to?i cei ce ramasesera din casa lui Ahab în Izreel ?i pe to?i cei mari ai lui ?i pe cei de aproape ai lui ?i pe preo?ii lui, încât n-a scapat nici unul.
\par 12 Apoi sculându-se, a plecat sa mearga la Samaria. Dar în drumul sau, ajungând la Bet-Eched, adica la Colibele Pastorilor,
\par 13 Iehu a întâlnit pe fra?ii lui Ohozia, regele Iudei, ?i le-a zis: "Cine sunte?i voi?" Iar ei au raspuns: "Noi suntem fra?ii lui Ohozia ?i ne ducem sa aflam de sanatatea fiilor regelui ?i a fiilor reginei".
\par 14 ?i el a zis: "Prinde-?i-i de vii!" ?i i-au prins de vii ?i i-au junghiat la fântâna de lânga Colibele Pastorilor. Aceia erau patruzeci ?i doi de oameni ?i n-a mai ramas nici unul din ei.
\par 15 Apoi plecând de acolo, s-a întâlnit cu Ionadab, fiul lui Recab, care venea în întâmpinarea lui, ?i l-a salutat ?i a zis: "Inima ta este ea oare cum este inima mea catre inima ta?" Iar Ionadab a zis: "Da". "De este a?a, da-mi mâna!" ?i i-a dat mâna ?i l-a ridicat la el în car,
\par 16 Zicând: "Hai cu mine, ca sa vezi râvna mea pentru Domnul". ?i l-a a?ezat în car.
\par 17 ?i ajungând în Samaria, a ucis pe to?i cei ce ramasesera din ai lui Ahab în Samaria ?i a?a a stârpit neamul lui cu totul, dupa cuvântul Domnului ce-l rostise prin Ilie.
\par 18 Apoi a adunat Iehu tot poporul ?i a zis: "Ahab a slujit pu?in lui Baal; Iehu însa îi va sluji mai mult.
\par 19 Deci chema?i la mine pe to?i proorocii lui Baal, pe to?i slujitorii lui ?i pe to?i preo?ii lui, ?i nimeni sa nu lipseasca, pentru ca am sa fac o jertfa mare lui Baal. Tot cel ce va lipsi nu va ramâne cu via?a". Iehu însa a facut aceasta cu gând viclean, ca sa stârpeasca pe slujitorii lui Baal.
\par 20 ?i a zis Iehu: "Vesti?i o zi de sarbatoare în cinstea lui Baal!"
\par 21 ?i au vestit, iar Iehu a trimis în tot Israelul de au venit to?i slujitorii lui Baal ?i n-a ramas nici unul care sa nu fi venit. Dupa aceea au intrat în templul lui Baal ?i s-a umplut templul de la un capat la celalalt.
\par 22 Atunci a zis Iehu catre pastratorul ve?mintelor: "Adu ve?minte pentru to?i slujitorii lui Baal". ?i acela le-a adus ve?minte.
\par 23 Apoi a intrat ?i Iehu cu Ionadab, fiul lui Recab, în templul lui Baal ?i a zis slujitorilor lui Baal: "Cauta?i ?i vede?i nu cumva se afla printre voi careva din slujitorii Domnului, pentru ca aici trebuie sa fie numai slujitorii lui Baal singuri".
\par 24 Apoi s-au apropiat ei sa faca jertfele ?i arderile de tot. Iar Iehu a pus afara optzeci de oameni ?i le-a zis: "Sufletul aceluia caruia îi va scapa careva din oamenii pe care vi-i voi da în mâini, va fi în locul celui scapat".
\par 25 Iar dupa ce s-a ispravit arderea de tot, Iehu a zis catre o?tenii sai ?i catre capeteniile lor: "Duce?i-va ?i-i ucide?i ?i sa nu scape nici unul". O?tenii ?i capeteniile i-au lovit cu ascu?i?ul sabiei ?i i-au aruncat acolo.
\par 26 Apoi s-au dus în cetate unde era capi?tea lui Baal, au scos idolii din capi?tea lui Baal ?i i-au ars;
\par 27 ?i au sfarâmat chipul cel cioplit al lui Baal ?i au darâmat capi?tea lui Baal ?i au facut din ea loc de necura?enii pâna în ziua de astazi.
\par 28 Astfel a stârpit Iehu pe Baal din pamântul lui Israel;
\par 29 Iar în ce prive?te pacatele lui Ieroboam, feciorul lui Nabat, care a dus pe Israel în ispita, de la acestea nu s-a departat nici Iehu; nu s-a lepadat de vi?eii de aur, din Betel ?i Dan.
\par 30 ?i a zis Domnul catre Iehu: "Pentru ca tu cu placere ai facut ceea ce era drept în ochii Mei ?i ai îndeplinit împotriva casei lui Ahab tot ceea ce aveam la inima Mea, feciorii tai pâna la al patrulea neam vor ?edea pe tronul lui Israel".
\par 31 Dar Iehu nu s-a silit sa umble din toata inima dupa legea Domnului Dumnezeului lui Israel ?i nu s-a abatut de la pacatele lui Ieroboam, care a dus pe Israel în ispita.
\par 32 În zilele acelea a început Domnul sa taie par?i din pamântul lui Israel ?i Hazael a lovit hotarele lui Israel,
\par 33 Pustiind la rasarit de Iordan tot pamântul Galaadului, al lui Gad, al lui Ruben ?i al lui Manase, începând de la Aroer, care vine lânga Arnon, ?i Galaadul ?i Vasanul.
\par 34 Celelalte fapte ale lui Iehu, tot ce a facut el, precum ?i vitejiile lui sunt scrise în cartea faptelor regilor lui Israel.
\par 35 ?i a raposat Iehu cu parin?ii sai în Samaria, iar în locul lui s-a facut rege Ioahaz, fiul sau.
\par 36 Iar timpul domniei lui Iehu peste Israel, în Samaria, a fost de douazeci ?i opt de ani.

\chapter{11}

\par 1 Atalia, mama lui Ohozia, vazând ca fiul sau a murit, s-a sculat ?i a stârpit tot neamul regesc.
\par 2 Dar Io?eba, fiica regelui Ioram, sora lui Ohozia, a furat pe Ioa?, fiul lui Ohozia, dintre fii regelui care trebuiau uci?i ?i l-a dus pe ascuns în odaia de dormit, împreuna cu doica lui, ?i l-a ascuns de Atalia ?i n-a fost ucis.
\par 3 Acesta a stat ascuns împreuna cu ea în templul Domnului ?ase ani, în care timp a domnit peste ?ara Atalia.
\par 4 Iar în anul al ?aptelea a trimis Iehoiada de au luat suta?i din garda ?i din o?tire ?i i-au adus la el în casa Domnului. Aici a facut cu ei legamânt , luând de la ei juramânt în templul Domnului ?i apoi le-a aratat pe fiul regelui,
\par 5 ?i le-a dat ordin, zicând: "Iata ce sa face?i! A treia parte din voi, din cei ce veni?i în ziua de odihna, ve?i face straja la casa domneasca,
\par 6 A treia parte la poarta Sur ?i a treia parte la poarta garzii, strajuind casa, ca sa nu fie vreo vatamare;
\par 7 Iar doua par?i din voi, din cei pleca?i în ziua de odihna, ve?i face straja la templul Domnului pentru rege.
\par 8 ?i ve?i înconjura pe rege din toate par?ile, având fiecare arma sa în mâna; cine va intra în rânduri, acela sa fie ucis; ?i ve?i fi pe lânga rege ?i când va ie?i ?i când va intra".
\par 9 ?i au facut osta?ii tot ce le-a poruncit preotul Iehoiada. A luat fiecare pe oamenii sai cei ce intrau de servici ?i cei ce ie?eau din servici în ziua de odihna ?i au venit la preotul Iehoiada.
\par 10 Iar preotul a împar?it suta?ilor suli?ele ?i scuturile regelui David care erau în templul Domnului.
\par 11 ?i s-au a?ezat o?tenii, fiecare cu arma în mâna, împrejurul regelui, de la dreapta templului pâna la stânga lui ?i pâna la jertfelnic.
\par 12 Apoi au scos pe fiul regelui, au pus pe capul lui coroana ?i podoabele rege?ti  ?i astfel l-au uns ?i l-au facut rege, batând din palme ?i strigând: "Traiasca regele!"
\par 13 Auzind Atalia glasul poporului ce striga, s-a dus în templul Domnului.
\par 14 Pe când se uita ea, iata regele statea la locul de sus, dupa obicei, iar lânga rege stateau cântare?ii ?i trâmbi?a?ii ?i tot poporul ?arii se veselea ?i suna din trâmbi?e. Atunci Atalia ?i-a rupt ve?mintele sale ?i a strigat: "Vânzare, vânzare!"
\par 15 Dar preotul Iehoiada a dat porunca suta?ilor, care cârmuiau o?tirea, ?i le-a zis: "Scoate?i-o din rânduri ?i pe cel ce se va duce dupa ea ucide?i-l cu sabia", caci preotul zisese: "Sa nu o ucide?i în templul Domnului".
\par 16 Deci i-au facut loc de a trecut pe poarta cailor, spre casa domneasca, ?i au ucis-o acolo.
\par 17 ?i a încheiat Iehoiada legamânt între Domnul ?i între rege ?i popor, ca acesta sa fie poporul Domnului, precum ?i între rege ?i popor.
\par 18 ?i s-a dus tot poporul ?arii în capi?tea lui Baal de a stricat jertfelnicele lui ?i chipurile lui le-au sfarâmat cu totul, iar pe Matan, preotul lui Baal, l-a ucis înaintea jertfelnicului, ?i preotul Iehoada a a?ezat straja în templul Domnului.
\par 19 Apoi a luat pe suta?i cu garda ?i o?tirea ?i tot poporul ?arii ?i au petrecut pe rege din templul Domnului ?i au venit pe drumul ce trece pe poarta garzii, la casa regelui ?i s-a urcat pe tronul regesc.
\par 20 Apoi s-a vestit poporului ?arii ?i cetatea s-a lini?tit. Pe Atalia au omorât-o cu sabia în casa domneasca.
\par 21 Ioa? era de ?apte ani când a fost facut rege.

\chapter{12}

\par 1 Astfel s-a facut Ioa? rege în anul al ?aptelea al lui Iehu ?i a domnit în Ierusalim patruzeci de ani.
\par 2 Ioa? a facut fapte placute în ochii Domnului în toate zilele sale, cât l-a pova?uit preotul Iehoiada;
\par 3 Numai înal?imile nu le-a desfiin?at, caci poporul tot mai aducea jertfe ?i tamâieri pe înal?imi.
\par 4 Deci a zis Ioa? catre preo?i: "Tot argintul daruit, care se aduce în templul Domnului: argintul de la trecatori, argintul adus pentru rascumpararea sufletului, dupa pre?uirea hotarâta, ?i tot argintul cât îl lasa pe cineva inima sa-l aduca în templul Domnului,
\par 5 Sa-l ia preo?ii pentru ei, fiecare de la cunoscutul sau, ?i sa repare stricaciunile casei oriunde s-ar afla".
\par 6 Însa pâna la anul al douazeci ?i treilea al regelui Ioa? preo?ii n-au reparat stricaciunile templului Domnului.
\par 7 De aceea regele Ioa? a chemat pe preotul Iehoiada ?i pe ceilal?i preo?i ?i le-a zis: "De ce nu repara?i stricaciunile templului Domnului? Deci sa nu mai lua?i de acum argintul de la cunoscu?ii vo?tri, ci sa-l da?i pentru repararea stricaciunilor templului Domnului".
\par 8 ?i s-au învoit preo?ii sa nu mai ia argintul pe care-l va da poporul pentru repararea stricaciunilor templului Domnului.
\par 9 Atunci a luat preotul Iehoiada o lada, i-a facut o deschizatura în partea ei de deasupra ?â a pus-o lânga jertfelnic, în partea dreapta, pe unde intra poporul în templul Domnului. ?i preo?ii care stateau de paza la prag puneau acolo tot argintul ce se aducea în templul Domnului.
\par 10 ?i când vedeau ca s-a strâns argint mult în lada, venea vistiernicul regelui ?i arhiereul ?i scoteau argintul gasit în templul Domnului ?i-l legau în saci.
\par 11 Argintul a?a socotit îl dadeau în primirea celor rândui?i sa faca lucrarile la templul Domnului, iar ace?tia îl cheltuiau cu dulgherii ?i me?te?ugarii care lucrau la templul Domnului,
\par 12 Cu zidarii ?i pietrarii, precum ?i cu cumpararea lemnului ?i a pietrelor de cioplit, cu repararea stricaciunilor la templul Domnului ?i cu tot ce trebuia pentru între?inerea templului Domnului.
\par 13 Dar din argintul adus în templul Domnului nu s-au facut pentru templul Domnului vase de argint, cu?ite, cupe pentru turnare, trâmbi?e, nici tot felul de vase de aur ?i de argint.
\par 14 Ci argintul care intra în templul Domnului se dadea lucratorilor ca sa savâr?easca lucrul ?i sa-l întrebuin?eze la repararea templului Domnului.
\par 15 ?i nu se cerea socoteala de la oamenii carora li se încredin?a argintul ca sa-l împarta celor ce faceau lucrarile, pentru ca ace?tia se purtau cinstit.
\par 16 Însa argintul ce se platea pe jertfa pentru vina ?i argintul ce se platea pe jertfa pentru pacat nu se ducea în templul Domnului, ci acesta era al preo?ilor.
\par 17 Atunci s-a ridicat Hazael, regele Siriei, ?i a pornit cu razboi împotriva ceta?ii Gat ?i a luat-o. ?i ?i-a pus Hazael în gând sa mearga ?i asupra Ierusalimului.
\par 18 Însa Ioa?, regele Iudei, a luat toate lucrurile daruite, pe care le daruisera templului Domnului Iosafat, Ioram ?i Ohozia, parin?ii lui, regii Iudei, precum ?i cele ce erau daruite de el ?i tot aurul ce s-a gasit în vistieria templului Domnului ?i a casei rege?ti ?i le-a trimis lui Hazael, regele Siriei, ?i acesta s-a retras din Ierusalim.
\par 19 Celelalte despre Ioa? ?i despre toate cele ce a facut el sunt scrise în cartea faptelor regilor lui Iuda.
\par 20 În urma s-au sculat slugile lui ?i au facut razvratire împotriva lui ?i au ucis pe Ioa?, în casa Milo, pe calea spre Sela.
\par 21 ?i l-au ucis slugile sale Iozacar, fiul lui ?imeat, ?i Iozabad, fiul lui ?omer. ?i el a murit ?i l-au îngropat cu parin?ii lui în cetatea lui David. ?i în locul lui s-a facut rege Amasia, fiul sau.

\chapter{13}

\par 1 În anul al douazeci ?i treilea al lui Ioa?, fiul lui Ohozia, regele lui Iuda, s-a facut rege peste Israel în Samaria Ioahaz, fiul lui Iehu, ?i a domnit ?aptesprezece ani.
\par 2 Acesta a facut rele în ochii Domnului ?i a umblat în pacatele lui Ieroboam, fiul lui Nabat, care a dus pe Israel în ispita ?i nu s-a lasat de ele.
\par 3 De aceea s-a aprins mânia Domnului asupra lui Israel ?i l-a dat pentru totdeauna în mâna lui Hazael, regele Siriei, ?i în mâna lui Benhadad, fiul lui Hazael.
\par 4 Atunci s-a rugat Ioahaz Domnului ?i Domnul l-a auzit, pentru ca vedea necazul Izraeli?ilor ?i cum îi strâmtora regele Siriei.
\par 5 ?i a dat Domnul Israeli?ilor izbavitor ?i au ie?it de sub mâna Sirienilor ?i au trait fiii lui Israel în ceta?ile lor ca ?i mai înainte.
\par 6 Dar tot nu s-au departat de pacatele casei lui Ieroboam, care a dus pe Israel în ispita, ci au umblat în ele ?i A?era a ramas mai departe în Samaria,
\par 7 Lui Ioahaz nu-i ramasese din o?tire decât numai cincizeci de calare?i, zece care ?i zece mii de pedestra?i, caci îi nimicise regele Siriei ?i-i prefacuse în pulbere de calcat cu picioarele.
\par 8 Celelalte ?tiri despre Ioahaz, despre vitejiile lui ?i despre tot ceea ce a facut el, sunt scrise în cartea Cronicilor regilor lui Israel.
\par 9 Apoi a raposat Ioahaz cu parin?ii sai ?i l-au îngropat în Samaria, iar în locul lui s-a facut rege Ioa?.
\par 10 În anul al treizeci ?i ?aptelea al lui Ioa?, regele Iudei, s-a facut rege peste Israel în Samaria Ioa?, fiul lui Ioahaz, domnind ?aisprezece ani.
\par 11 ?i a facut fapte netrebnice înaintea Domnului ?i nu s-a abatut de la toate pacatele lui Ieroboam, fiul lui Nabat, care a dus pe Israel în ispita, ci a umblat în ele.
\par 12 Celelalte ?tiri despre Ioa?, tot ceea ce a facut el, ispravile lui, razboiul avut cu Amasia, regele Iudei, sunt scrise în cartea Cronicilor regilor lui Israel.
\par 13 Apoi a adormit Ioa? cu parin?ii sai, iar pe scaunul lui s-a suit Ieroboam. Ioa? a fost îngropat în Samaria, lânga regii lui Israel.
\par 14 În vremea lui Ioa? s-a îmbolnavit Elisei de o boala de care apoi a ?i murit. Atunci a venit la el Ioa?, regele lui Israel ?i, plângând deasupra lui, a zis: "Parinte, parinte, carul lui Israel ?i calare?ul lui!"
\par 15 Iar Elisei a zis catre el: "Ia un arc ?i ni?te sage?i! "
\par 16 ?i a luat regele un arc ?i ni?te sage?i. Apoi Elisei a zis catre regele lui Israel: "Pune mâna pe arc!" ?i a pus regele mâna pe arc, ?i ?i-a pus ?i Elisei mâinile sale pe mâinile regelui,
\par 17 ?i a zis: "Deschide fereastra dinspre rasarit!" ?i regele a deschis-o. ?i a zis Elisei: "Sageteaza!". ?i a sagetat regele. Elisei a zis: "Aceasta este sageata izbavirii ce vine de la Domnul ?i sageata izbavirii de Sirieni, caci vei bate pe Sirieni cu desavâr?ire, la Afec".
\par 18 Apoi iara?i a zis Elisei: "Ia ni?te sage?i!" ?i regele a luat. ?i a zis Elisei catre regele lui Israel: "Love?te în pamânt!" ?i a lovit de trei ori ?i s-a oprit.
\par 19 Iar omul lui Dumnezeu s-a mâniat pe el, ?i a zis: "Trebuia sa love?ti de cinci sau de ?apte ori, caci atunci ai fi batut cu desavâr?ire pe Sirieni; acum însa vei bate pe Sirieni numai de trei ori".
\par 20 Apoi a murit Elisei ?i l-au îngropat, iar în anul urmator au intrat în ?ara cete de Moabi?i.
\par 21 Dar iata, odata, când îngropau un mort, s-a întâmplat ca cei ce-l îngropau sa vada una din aceste cete ?i, speriindu-se, au aruncat mortul în mormântul lui Elisei. Cazând acela, s-a atins de oasele lui Elisei ?i a înviat ?i s-a sculat pe picioarele sale.
\par 22 Iar Hazael, regele Siriei, a strâmtorat pe Israeli?i în toate zilele lui Ioahaz.
\par 23 Domnul însa S-a milostivit asupra lor de i-a iertat ?i S-a întors spre ei pentru legamântul Sau cel încheiat cu Avraam, Isaac ?i Iacov ?i n-a voit sa-i piarda, nici nu i-a lepadat de la fa?a Sa, pâna astazi.
\par 24 Dupa ce a murit Hazael, regele Siriei, în locul lui a fost facut rege Benhadad, fiul sau.
\par 25 Atunci Ioa?, fiul lui Ioahaz, a luat înapoi din mâinile lui Benhadad, fiul lui Hazael, ceta?ile pe care le luase acela cu razboi din mâinile tatalui sau, Ioahaz. De trei ori l-a batut Ioa? ?i a întors ceta?ile lui Israel.

\chapter{14}

\par 1 În anul al doilea al lui Ioa?, fiul lui Ioahaz, regele lui Israel, s-a facut rege în Iuda, Amasia, fiul lui Ioa?, regele Iudei.
\par 2 Amasia era de douazeci ?i cinci de arii când a fost facut rege ?i a domnit în Ierusalim douazeci ?i noua de ani. Numele mamei sale era Ioadin ?i era din Ierusalim.
\par 3 Acesta a facut fapte placute în ochii Domnului, dar nu ca stramo?ul sau David; s-a purtat însa în toate ca tatal sau Ioa?.
\par 4 Numai înal?imile n-au fost înlaturate, caci poporul tot mai savâr?ea jertfe ?i tamâieri pe înal?imi.
\par 5 Când a pus bine mâna pe domnie, Amasia a ucis slugile sale care ucisesera pe tatal sau;
\par 6 Dar pe copiii uciga?ilor nu i-a ucis, de vreme ce în cartea legii lui Moise, prin care porunce?te Domnul, este scris: "Parin?ii nu trebuie sa fie pedepsi?i cu moarte pentru copii, nici copiii nu trebuie sa fie pedepsi?i cu moarte pentru parin?i, ci fiecare pentru vina sa trebuie sa fie pedepsit cu moarte!"
\par 7 Tot el a batut zece mii de Edomi?i în Valea Sarata ?i a luat ?ilo prin lupta ?i i-a pus numele Iocteel, pe care-l poarta pâna în ziua de astazi.
\par 8 Atunci a trimis Amasia soli la Ioa?, regele lui Israel, fiul lui Ioahaz, fiul lui Iehu, zicând: "Vino sa ne vedem la fa?a".
\par 9 Iar Ioa?, regele lui Israel, a trimis la Amasia, regele Iudei, sa i se zica: "Spânul cel din Liban a trimis la cedrul Libanului sa-i zica: Da-?i fata dupa fiul meu! ?i trecând fiarele salbatice din Liban au calcat spinul.
\par 10 Tu ai batut pe Edomi?i ?i s-a înal?at inima ta! Bucura-te de biruin?a, dar ?ezi acasa la tine! De ce sa te cer?i spre raul tau? Vei cadea ?i cu tine va cadea Iuda!"
\par 11 Amasia însa n-a ascultat. Atunci s-a sculat Ioa?, regele lui Israel, ?i s-a vazut fa?a catre fa?a, el ?i Amasia, regele lui Iuda, la Bet-?eme? în Iuda.
\par 12 Dar au fost batu?i Iudeii de Israeli?i ?i au fugit în corturile lor.
\par 13 Iar pe Amasia, regele Iudei, fiul lui Ioa?, fiul lui Ohozia, l-a prins Ioa?, regele lui Israel, în Bet-?eme? ?i s-a dus Ioa? la Ierusalim ?i a stricat zidul Ierusalimului de la por?ile lui Efraim pâna la por?ile din col?, pe o întindere de patru sute de coji.
\par 14 ?i au luat tot aurul ?i argintul ?i toate vasele câte s-au gasit la templul Domnului ?i în vistieria casei regelui ?i ostateci ?i s-au întors în Samaria.
\par 15 Celelalte ?tiri despre Ioa?, faptele ?i vitejiile lui ?i cum s-a luptat el cu Amasia, regele Iudei, sunt scrise în cartea Cronicilor regilor lui Israel.
\par 16 În urma a adormit Ioa? cu parin?ii sai ?i a fost îngropat în Samaria cu regii lui Israel; iar în locul lui a fost facut rege Ieroboam, fiul sau.
\par 17 Amasia, fiul lui Ioa?, regele Iudei, a trait cincisprezece ani dupa moartea lui Ioa?, fiul lui Ioahaz, regele lui Israel.
\par 18 Celelalte fapte ale lui Amasia sunt scrise în cartea Cronicilor regilor lui Iuda.
\par 19 Facându-se însa la Ierusalim rasvratire împotriva lui, a fugit în Lachi?, ?i s-a trimis dupa el în Lachi? ?i l-au ucis acolo,
\par 20 ?i l-au adus pe cai ?i l-au îngropat în Ierusalim, cu parin?ii sai, în cetatea lui David.
\par 21 Dupa aceea tot poporul iudeu a luat pe Azaria, care era atunci numai de ?aisprezece ani, ?i l-a facut rege în locul tatalui sau Amasia.
\par 22 El a zidit Elatul ?i l-a întors la Iuda, dupa ce regele raposase cu parin?ii sai.
\par 23 În anul al cincisprezecelea al lui Amasia, fiul lui Ioa?, regele Iudei, a fost facut rege în Samaria Ieroboam, fiul lui Ioa?, regele lui Israel.
\par 24 ?i a domnit patruzeci ?i unu de ani ?i a facut fapte netrebnice în ochii Domnului; neabatându-se de la toate pacatele lui Ieroboam, fiul lui Nabat, care a dus pe Israel în ispita.
\par 25 Acesta a a?ezat din nou vechiul hotar al lui Israel de la intrarea în Hamat pâna la marea Araba, dupa cuvântul Domnului Dumnezeului lui Israel, rostit prin robul Sau Iona, fiul lui Amitai, proorocul cel din Gat-Hefer,
\par 26 Ca Domnul vazuse necazul foarte amar al lui Israel cel strâmtorat, lipsit ?i parasit ?i nu era cine sa-l ajute.
\par 27 ?i n-a vrut Domnul sa stârpeasca numele Israeli?ilor de sub cer, ci i-a izbavit prin mâna lui Ieroboam, fiul lui Ioa?.
\par 28 Celelalte ?tiri despre Ieroboam, despre toate cele ce a facut el, ?i cum a întors Damascul ?i Hamatul lui Iuda în Israel sunt scrise în cartea Cronicilor regilor lui Israel.
\par 29 Apoi a adormit Ieroboam, cu parin?ii sai, cu regii lui Israel, iar în locul lui a fost facut rege Zaharia, fiul sau.

\chapter{15}

\par 1 În anul al douazeci ?i ?aptelea al lui Ieroboam, regele lui Israel, s-a facut rege Azaria, fiul lui Amasia, regele lui Iuda.
\par 2 Acesta era de ?aisprezece ani când S-a facut rege ?i a domnit în Ierusalim cincizeci ?i doi de ani. Numele mamei sale era Iecolia din Ierusalim.
\par 3 Azaria a facut fapte placute în ochii Domnului, purtându-se în toate ca tatal sau Amasia.
\par 4 Numai înal?imile nu le-a departat, caci poporul tot mai savâr?ea jertfe ?i tamâieri pe înal?imi.
\par 5 Dar a lovit Domnul pe rege, ?i acesta a fost lepros pâna în ziua mor?ii sale ?i a trait într-o casa osebita, în vreme ce Iotam, fiul regelui, era în casa domneasca, judecând poporul ?arii.
\par 6 Celelalte ?tiri despre Azaria ?i tot ce a facut el sunt scrise în cartea Cronicilor regilor lui Iuda.
\par 7 În urma a raposat Azaria cu parin?ii sai ?i l-au îngropat cu parin?ii lui în cetatea lui David, iar în locul lui S-a facut rege Iotam, fiul sau.
\par 8 În anul al treizeci ?i optulea al lui Azaria, regele Iudei, s-a facut rege peste Israel, în Samaria, Zaharia, fiul lui Ieroboam, ?i a domnit ?ase luni.
\par 9 Acesta a facut lucruri netrebnice în ochii Domnului, cum facusera ?i parin?ii lui, caci nu s-a lasat de pacatele lui Ieroboam, fiul lui Nabat, care a dus pe Israel în ispita.
\par 10 Dar împotriva lui a facut razvratire ?alum, fiul lui Iabe?, ?i l-a lovit înaintea poporului ?i l-a ucis ?i s-a facut rege în locul lui.
\par 11 Celelalte ?tiri despre Zaharia sunt scrise în cartea Cronicilor regilor lui Israel.
\par 12 ?i astfel s-a împlinit cuvântul Domnului, cel rostit catre Iehu, când a zis: "Fiii tai pâna la al patrulea neam vor ?edea pe tronul lui Israel".
\par 13 ?alum, fiul lui Iabe?, s-a facut rege în anul al treizeci ?i noualea al lui Azaria, regele Iudei, ?i a domnit o luna în Samaria.
\par 14 Atunci s-a dus Menahem, fiul lui Gadi, din Tir?a ?i, ajungând în Samaria, a lovit pe ?alum, fiul lui Iabe?, în Samaria ?i l-a ucis ?i s-a facut rege în locul lui.
\par 15 Alte ?tiri despre ?alum ?i despre razvratirea pe care a facut-o el sunt scrise în cartea Cronicilor regilor lui Israel.
\par 16 Atunci Menahem a batut Tapuahul ?i pe to?i cei ce erau în el ?i în hotarele lui, începând din Tir?a, pentru ca nu ?i-a deschis por?ile, ?i a taiat pe toate femeile însarcinate de acolo.
\par 17 Apoi Menahem, fiul lui Gadi, s-a facut rege peste Israel în anul al treizeci ?i noualea al lui Azaria, regele Iudei, ?i a domnit în Samaria zece ani;
\par 18 ?i a facut fapte rele în ochii Domnului, nelasându-se în toate zilele sale de pacatele lui Ieroboam, fiul lui Nabat, care a dus pe Israel în ispita.
\par 19 Atunci a venit Ful (Tiglatfalasar III), regele Asiriei, asupra ?arii israelite. Însa Menahem a dat lui Ful o mie de talan?i de argint, ca sa-l sus?ina ?i sa întareasca domnia în mâinile lui.
\par 20 Apoi Menahem a scos argintul acesta de la Israeli?i, de la to?i cei boga?i, câte cincizeci de sicli de argint de fiecare om, ca sa-i dea regelui Asiriei. ?i regele Asiriei s-a întors ?i n-a ramas acolo în ?ara.
\par 21 Celelalte ?tiri despre Menahem, tot ce a facut el, sunt scrise în cartea Cronicilor regilor lui Israel.
\par 22 În urma a adormit Menahem cu parin?ii sai ?i în locul lui s-a facut rege Pecahia, feciorul sau.
\par 23 Pecahia, feciorul lui Menahem, s-a facut rege peste Israel în Samaria în anul al cincizecilea al lui Azaria, regele Iudei ?i a domnit doi ani.
\par 24 ?i a facut fapte rele în ochii Domnului, nelasându-se de pacatele lui Ieroboam, fiul lui Nabat, care a dus pe Israel în ispita.
\par 25 Dar împotriva lui s-a razvratit Pecah, fiul lui Remalia, cel al treilea în rang dupa el, împreuna cu Argob ?i cu Arie, care aveau cu ei cincizeci de barba?i din Galaad ?i l-au lovit în Samaria, chiar în turnul casei regale, l-au omorât ?i s-a facut rege În locul lui.
\par 26 Celelalte ?tiri despre Pecahia, tot ce a facut el, sunt scrise în cartea Cronicilor regilor lui Israel.
\par 27 În anul al cincizeci ?i doilea al lui Azaria, regele Iudei, s-a facut rege peste Israel, în Samaria, Pecah, fiul lui Remalia ?i a domnit douazeci de ani.
\par 28 ?i a facut lucruri rele în ochii Domnului, nelasându-se de pacatele lui Ieroboam, fiul lui Nabat, care a dus pe Israel în ispita.
\par 29 În zilele lui Pecah, regele lui Israel, a venit Tiglatfalasar, regele Asiriei, ?i a luat Ionul, Abel-Bet-Maaca, Ianoah, Chede?, Ha?or, Galaadul, Galileea ?i tot pamântul lui Neftali ?i pe locuitori i-a stramutat în Asiria.
\par 30 Atunci Osea, fiul lui Ela, a facut o uneltire împotriva lui Pecah, fiul lui Remalia ?i l-a lovit ?i l-a ucis ?i s-a facut rege în locul lui, în anul al douazecilea al lui Ioatam, fiul lui Azaria.
\par 31 Celelalte ?tiri despre Pecah ?i despre tot ce a facut el sunt scrise în cartea Cronicilor regilor lui Israel.
\par 32 În anul al doilea al lui Pecah, fiul lui Remalia, regele lui Israel, s-a facut rege Ioatam, fiul lui Azaria, regele Iudei.
\par 33 Acesta era de douazeci ?i cinci de ani când s-a facut rege ?i a domnit în Ierusalim ?aisprezece ani. Numele mamei lui era Ieru?a, fiica lui ?adoc.
\par 34 ?i a facut el lucruri placute în ochii Domnului. Cum s-a purtat tatal sau Azaria a?a s-a purtat ?i el în toate.
\par 35 Numai înal?imile nu le-a înlaturat, caci poporul tot mai savâr?ea jertfe ?i tamâieri pe înal?imi. Tot el a facut poarta de sus la templul Domnului.
\par 36 Iar celelalte ?tiri despre Ioatam ?i tot ce a facut el, sunt scrise în cartea Cronicilor regilor lui Iuda.
\par 37 În zilele acelea a început Domnul a trimite asupra Iudei pe Ra?on, regele Siriei, ?i pe Pecah, fiul lui Remalia.
\par 38 Ioatam a raposat cu parin?ii sai ?i a fost îngropat cu ei în cetatea lui David, stramo?ul sau, iar în locul lui a fost facut rege Ahaz, fiul sau.

\chapter{16}

\par 1 În anul al ?aptesprezecelea al lui Pecah, fiul lui Remalia, a fost facut rege Ahaz, fiul lui Ioatam, regele lui Iuda.
\par 2 Ahaz era de douazeci de ani când s-a facut rege ?i a domnit în Ierusalim ?aisprezece ani, dar n-a facut fapte placute în ochii Domnului Dumnezeului sau, ca David, stramo?ul sau,
\par 3 Ci a pa?it pe urmele regilor lui Israel ?i chiar ?i pe fiul sau l-a trecut prin foc, urmând urâciunile popoarelor pe care le alungase Domnul de la fa?a fiilor lui Israel;
\par 4 ?i a adus jertfe ?i tamâieri pe înal?imi ?i dealuri ?i sub tot pomul umbros.
\par 5 Atunci s-au dus Ra?on, regele Siriei, ?i Pecah, fiul lui Remalia, regele lui Israel, asupra Ierusalimului, ca sa-l cuprinda ?i au ?inut pe Ahaz împresurat, dar nu l-au putut birui.
\par 6 În vremea aceea Ra?on, regele Siriei, a întors Elatul la Siria ?i a izgonit pe Iudei din Elat; apoi au intrat în Elat Edomi?ii ?i traiesc acolo pâna în ziua de astazi.
\par 7 Atunci a trimis Ahaz soli la Tiglatfalasar, regele Asiriei, sa-i spuna: "Robul tau ?i fiul tau sunt eu; vino ?i ma apara de mâna regelui Siriei ?i de mâna regelui lui Israel, care s-au ridicat împotriva mea!"
\par 8 Cu acel prilej a luat Ahaz argintul ?i aurul care s-a gasit în templul Domnului ?i în vistieria casei domne?ti ?i l-a trimis dar regelui Asiriei.
\par 9 Deci l-a ascultat regele Asiriei ?i s-a dus regele Asiriei la Damasc, unde a prins pe Ra?on ?i pe locuitorii lui i-a stramutat în Chir, iar pe Ra?on l-a ucis.
\par 10 Atunci a ie?it regele Ahaz întru întâmpinarea lui Tiglatfalasar, regele Asiriei, la Damasc; ?i a vazut jertfelnicul cel din Damasc ?i a trimis regele Ahaz lui Urie preotul chipul jertfelnicului ?i planul alcatuirii lui.
\par 11 Iar preotul Urie a facut un jertfelnic dupa planul ce i-l trimisese regele Ahaz din Damasc; a?a a facut preotul Urie pâna a venit regele de la Damasc.
\par 12 Iar daca a venit regele de la Damasc ?i a vazut jertfelnicul, s-a apropiat regele de jertfelnic,
\par 13 A adus jertfa pe el ?i a ars arderea de tot a sa cu darul de pâine; a savâr?it turnarea sa ?i l-a stropit cu sângele jertfei de împacare.
\par 14 Iar jertfelnicul cel de arama, care era înaintea Domnului, l-a mutat din fa?a templului, dintre jertfelnicul cel nou ?i templul Domnului, ?i l-a pus în partea dinspre miazanoapte a acestuia.
\par 15 Apoi regele Ahaz a dat porunca preotului Urie, zicând: "Pe jertfelnicul cel mare sa arzi arderea de tot cea de diminea?a ?i darul de pâine cel de seara, arderea de tot a regelui ?i darul lui de pâine, arderea de tot din partea întregului popor al ?arii ?i darul lui cel de pâine, turnarea lui ?i cu tot sângele arderii de tot ?i cu tot sângele jertfei sa-l strope?ti, iar jertfelnicul cel de arama va ramâne pâna voi vedea ce este de facut cu el".
\par 16 ?i a facut preotul Urie a?a cum a poruncit regele Ahaz.
\par 17 Dupa aceea a desprins regele Ahaz pervazurile postamentelor, a luat bazinele de pe ele ?i a luat ?i marea de pe boii cei de arama, care erau sub ea ?i a a?ezat-o pe o temelie de piatra;
\par 18 A desfiin?at de asemenea pridvorul acoperit numit al zilei de odihna, care fusese facut la templul Domnului ?i pridvorul din afara al templului Domnului, care se numea al regelui, numai ca sa placa regelui Asiriei.
\par 19 Celelalte ?tiri despre Ahaz ?i cele ce a facut el sunt scrise în cartea Cronicilor regilor lui Iuda.
\par 20 Apoi a raposat Ahaz cu parin?ii sai în cetatea lui David; iar în locul lui a fost facut rege Iezechia, fiul sau.

\chapter{17}

\par 1 În anul al doisprezecelea al lui Ahaz, regele Iudei, a fost facut rege în Samaria peste Israel, Osea, fiul lui Ela ?i a domnit noua ani.
\par 2 El a facut lucruri rele în ochii Domnului, dar nu ca regii lui Israel, care au fost înainte de el.
\par 3 Împotriva lui s-a ridicat Salmanasar, regele Asiriei, ?i a ajuns Osea supusul acestuia ?i-i platea bir.
\par 4 Dar regele Asiriei a sim?it necredincio?ia lui Osea, caci acesta trimisese robi la So, regele Egiptului, ?i nu platise bir regelui Asiriei în fiecare an. De aceea regele Asiriei l-a luat legat ?i l-a aruncat în închisoare.
\par 5 Apoi regele Asiriei a navalit asupra ?arii întregi ?i a mers ?i la Samaria ?i a ?inut-o împresurata trei ani.
\par 6 Iar în anul al zecelea al lui Osea, regele Asiriei a luat Samaria ?i a stramutat pe Israeli?i în Asiria ?i i-a a?ezat în Halach ?i în Habor, lânga râul Gozan, în ceta?ile Mediei.
\par 7 Când fiii lui Israel au început a pacatui înaintea Domnului Dumnezeului lor, Care îi scosese din. pamântul Egiptului ?i de sub mâna lui Faraon, regele Egiptului, ?i s-au apucat sa cinsteasca dumnezeii altora;
\par 8 Când au început ei sa se poarte dupa obiceiurile popoarelor pe care le alungase Domnul de la fa?a fiilor lui Israel ?i dupa obiceiurile regilor lui Israel, facând cum faceau ace?tia;
\par 9 Când au început fiii lui Israel a face fapte neplacute Domnului Dumnezeului lor, zidindu-?i înal?imi prin toate târgurile lor, de la turnul de straja, pâna la cetatea întarita,
\par 10 ?i a?ezând idolii ?i chipurile Astartei pe tot dealul înalt ?i sub tot pomul umbros;
\par 11 ?i când s-au apucat sa savâr?easca tamâieri pe toate înal?imile, ca popoarele pe care le izgonise Domnul de la ei ?i sa faca fapte urâte, care mâniasera pe Domnul
\par 12 Slujind idolilor de care Domnul le zisese: "Sa nu face?i aceasta",
\par 13 Atunci Domnul a dat marturie împotriva lui Israel ?i a lui Iuda prin to?i proorocii Sai, prin to?i vazatorii, zicând: "Întoarce?i-va din caile voastre cele rele ?i pazi?i poruncile Mele, a?ezamintele Mele ?i toata înva?atura pe care Eu am dat-o parin?ilor vo?tri ?i pe care v-am dat-o ?i voua prin prooroci, robii Mei".
\par 14 Dar ei n-au ascultat, ci ?i-au învârto?at cerbicia, ca ?i parin?ii lor care nu crezusera în Domnul Dumnezeul lor
\par 15 ?i au dispre?uit poruncile Lui ?i legamântul Lui, pe care-l încheiase El cu parin?ii lor ?i descoperirile Lui, cu care El îi de?teptase ?i au umblat dupa idoli ?i au ajuns netrebnici, purtându-se ca popoarele cele dimprejur de care Domnul le zisese sa nu se poarte ca ele;
\par 16 ?i au parasit toate poruncile Domnului Dumnezeului lor ?i au facut chipurile turnate a doi vi?ei ?i au a?ezat A?ere ?i s-au închinat la toata o?tirea cerului ?i au slujit lui Baal;
\par 17 ?i au trecut pe fiii lor ?i pe fiicele lor prin foc, au ghicit ?i au vrajit ?i s-au apucat sa faca lucruri netrebnice în ochii Domnului ?i sa-L mânie.
\par 18 Atunci S-a mâniat Domnul tare pe Israeli?i ?i i-a lepadat de la fa?a Sa, ?i n-a mai ramas decât semin?ia lui Iuda.
\par 19 Dar nici Iuda n-a pazit poruncile Domnului Dumnezeului sau ?i s-a purtat dupa obiceiurile Israeli?ilor, cum Se purtau ace?tia.
\par 20 ?i ?i-a întors Domnul fa?a de la to?i urma?ii lui Israel ?i i-a smerit dându-i în mâinile jefuitorilor ?i în sfâr?it i-a lepadat de la fa?a Sa.
\par 21 Caci Israeli?ii se dezbinasera de la casa lui David ?i facusera rege pe Ieroboam, fiul lui Nabat, Ieroboam a abatut pe Israeli?i de la Domnul ?i i-a bagat în pacat mare.
\par 22 ?i au umblat fiii lui Israel în toate pacatele lui Ieroboam, câte facuse acesta ?i nu s-a departat de la ele, pâna când n-a lepadat Domnul pe Israel de la fa?a Sa, cum zisese prin to?i proorocii, robii Sai.
\par 23 ?i a fost stramutat Israel din pamântul sau în Asiria, unde se afla pâna în ziua de astazi.
\par 24 Dupa aceea regele Asiriei a adunat oameni din Babilon, din Cuta, din Ava, din Hamat ?i din Sefarvaim ?i i-a a?ezat prin ceta?ile Samariei în locul fiilor lui Israel. Ace?tia au stapânit Samaria ?i au început a locui prin ceta?ile ei.
\par 25 Dar fiindca la începutul vie?uirii lor acolo ei nu cinsteau pe Domnul, de aceea Domnul a trimis asupra lor lei care-i omorau.
\par 26 Atunci s-a spus regelui Asiriei, zicând: "Popoarele pe care tu le-ai stramutat ?i le-ai a?ezat prin ceta?ile Samariei nu cunosc legea Dumnezeului acelei ?ari ?i de aceea El trimite asupra lor lei ?i iata ace?tia le omoara, pentru ca ele nu cunosc legea Dumnezeului acelei ?ari".
\par 27 Iar regele Asiriei a poruncit ?i a zis: "Trimite?i acolo pe unul din preo?ii pe care i-a?i adus de acolo, ca sa se duca sa traiasca acolo ?i sa-i înve?e legea Dumnezeului acelei ?ari".
\par 28 Atunci a venit unul din preo?ii cei ce fusesera adu?i din Samaria ?i a locuit în Betel ?i i-a înva?at cum sa cinsteasca pe Domnul.
\par 29 Afara de acestea, fiecare popor ?i-a mai facut ?i dumnezeii sai ?i i-a pus în capi?tile de pe înal?imi pe care le facusera Samarinenii; fiecare popor în cetatea sa în care traia.
\par 30 Caci Babilonenii ?i-au facut pe Sucot-Benot, Cutienii ?i-au facut pe Nergal, Hamatienii ?i-au facut pe A?ima,
\par 31 Aveenii ?i-au facut pe Nivhaz ?i Tartac, iar Sefarvaimii î?i ardeau pe fiii ?i pe fiicele lor cu foc în cinstea lui Adramelec ?i Anamelec, zeii lor.
\par 32 Cinsteau însa ?i pe Domnul ?i ?i-au facut dintre ei preo?i pentru înal?imi ?i ace?tia slujeau la ei, în capi?tile de pe înal?imi.
\par 33 Dar ei cinsteau pe Domnul ?i slujeau zeilor sai dupa obiceiul popoarelor din mijlocul carora fusesera adu?i.
\par 34 A?a urmeaza ei pâna în ziua de astazi, dupa obiceiurile lor cele de la început; de Domnul nu se tem ?i nu urmeaza dupa a?ezamintele, dupa rânduielile, dupa legea ?i dupa poruncile pe care le-a poruncit Domnul fiilor lui Iacov, caruia îi daduse numele de Israel ?i cu ai carui urma?i încheiase legamânt ?i le poruncise a?a: "Sa nu cinsti?i pe dumnezeii altora, nici sa va închina?i lor;
\par 35 Sa nu le sluji?i nici sa le aduce?i jertfe;
\par 36 Ci sa cinsti?i pe Domnul, Care v-a scos din pamântul Egiptului cu putere mare ?i cu bra? înalt; pe Acesta sa-L cinsti?i ?i Lui sa va închina?i ?i sa-I aduceri jertfe;
\par 37 Sili?i-va în toate zilele sa împlini?i rânduielile, a?ezamintele, legea ?i poruncile pe care vi le-a scris El, iar pe zeii altora sa nu-i cinsti?i;
\par 38 Legamântul pe care l-am încheiat cu voi sa nu-l uita?i ?i pe zeii altora sa nu-i cinsti?i;
\par 39 Ci sa cinsti?i numai pe Domnul Dumnezeul vostru ?i El va va izbavi din mâna tuturor vrajma?ilor vo?tri".
\par 40 Dar ei n-au ascultat, ci au urmat obiceiurile celor de mai înainte.
\par 41 Astfel popoarele acestea cinsteau pe Domnul, dar slujeau ?i idolilor lor. Ba ?i copiii lor ?i copiii copiilor lor pâna în ziua de astazi urmeaza tot a?a cum au urmat ?i parin?ii lor.

\chapter{18}

\par 1 În anul al treilea al lui Osea, fiul lui Ela, regele lui Israel, s-a facut rege Iezechia, fiul lui Ahaz, regele Iudei.
\par 2 Acesta era de douazeci ?i cinci de ani când s-a facut rege ?i a domnit douazeci ?i noua de ani în Ierusalim. Numele mamei sale era Abia, fiica lui Zaharia.
\par 3 Acesta a facut fapte placute în ochii Domnului în toate, cum facuse ?i David, tatal sau, ca el a desfiin?at înal?imile, a sfarâmat stâlpii cu pisanii idole?ti, A?erele,
\par 4 ?i a stricat ?arpele cel de arama, pe care-l facuse Moise; chiar pâna în zilele acelea fiii lui Israel îl tamâiau ?i-l numeau Nehu?tan.
\par 5 ?i a nadajduit el în Domnul Dumnezeu. Ca el n-a mai fost altul între to?i regii lui Iuda, nici înainte, nici dupa;
\par 6 Caci s-a lipit el de Domnul ?i nu s-a abatut de la El, ci a pazit poruncile Lui, cum poruncise Domnul lui Moise.
\par 7 De aceea Domnul a fost cu el ?i tot ce a facut Iezechia, facea cu chibzuin?a. ?i s-a departat el de regele Asiriei ?i n-a mai slujit aceluia.
\par 8 Apoi a batut pe Filisteni pâna la Gaza, precum ?i în cuprinsul ei, de la turnul de paza pâna la cetatea întarita.
\par 9 în anul al patrulea al regelui Iezechia, adica în anul al ?aptelea al lui Osea, fiul lui Ela, regele lui Israel, s-a dus Salmanasar, regele Asiriei, asupra Samariei ?i a împresurat-o ?i dupa trei ani a luat-o.
\par 10 În anul al ?aptelea al lui Iezechia, adica în anul al zecelea al lui Osea, regele lui Israel, a fost luata Samaria.
\par 11 Atunci regele Asiriei a stramutat pe Israeli?i în Asiria ?i i-a a?ezat în Halach, în Habor, lânga râul Gozan, în ceta?ile Mediei,
\par 12 Pentru ca n-au ascultat glasul Domnului Dumnezeului lor ?i au calcat legamântul Lui; tot ceea ce a poruncit Moise, robul Domnului, ei nici n-au ascultat, nici n-au facut.
\par 13 Iar în anul al paisprezecelea al regelui Iezechia s-a dus Senaherib, regele Asiriei, asupra tuturor ceta?ilor întarite ale lui Iuda ?i le-a luat.
\par 14 Atunci a trimis Iezechia, regele Iudei, la regele Asiriei în Lachi?, ca sa-i zica: "Vinovat sunt! Du-te de la mine, caci ceea ce vei pune asupra mea, voi plati!" ?i a pus regele Asiriei asupra lui Iezechia, un bir de trei sute de talanii de argint ?i treizeci de talan?i de aur.
\par 15 ?i a dat Iezechia tot argintul ce s-a gasit în templul Domnului ?i în vistieria casei domne?ti.
\par 16 În vremea aceea a luat Iezechia aurul de pe u?ile templului Domnului ?i de pe stâlpii cei vechi pe care-i aurise însu?i Iezechia. ?i l-a dat regelui Asiriei.
\par 17 ?i a trimis regele Asiriei din Lachi? pe Tartan, pe Rabsaris ?i pe Rab?ache cu o?tire mare asupra Ierusalimului ?i, sosind, s-au oprit la canalul iazului de sus, care se afla lânga drumul ce merge spre ?arina nalbitorului.
\par 18 ?i chemând aceia pe rege, au ie?it la ei Eliachim, fiul lui Hilchia, capetenia cur?ii domne?ti, ?ebna scriitorul ?i Ioah cronicarul, fiul lui Asaf.
\par 19 Atunci a zis catre ei Rab?ache: "Spune?i lui Iezechia: A?a zice regele cel mare, regele Asiriei: Ce fel de nadejde este aceea în care te reazemi? Tu ai spus numai vorbe goale; pentru razboi însa trebuie pricepere ?i putere.
\par 20 Acum însa în cine nadajduie?ti tu, de te-ai razvratit împotriva mea?
\par 21 Iata, tu soco?i sa te reazemi pe Egipt, pe acea trestie frânta care de se sprijine?te cineva în ea îi intra în mâna ?i i-o sparge. A?a este Faraon, regele Egiptului, pentru to?i cei ce nadajduiesc în el.
\par 22 Iar de-mi ve?i zice: Noi nadajduim în Domnul Dumnezeul nostru! Apoi în acela, oare, ale carui înal?imi ?i jertfelnice le-a stricat Iezechia ?i a zis lui Iuda ?i Ierusalimului: Numai înaintea acestui jertfelnic sa va închina?i, care este în Ierusalim?
\par 23 Deci intra în legatura cu stapânul meu, regele Asiriei, ?i eu î?i voi da doua mii de cai; po?i tu oare sa gase?ti calare?i pentru ei?
\par 24 Cum vei birui tu macar o singura capetenie dintre cele mai mici slugi ale stapânului meu? Ai nadejde în Egipt, pentru care ?i pentru calare?i?
\par 25 Pe lânga aceasta, au doara eu fara voia Domnului am venit la locul acesta ca sa-l stric? Domnul mi-a zis: Du-te asupra ?arii acesteia ?i o strica!"
\par 26 Iar Eliachim, fiul lui Hilchia, ?abna ?i Ioah au zis catre Rab?ache: "Vorbe?te cu robii tai în limba aramaica, pentru ca noi în?elegem, ?i nu grai cu noi în limba iudaica, în auzul poporului, care sta pe zid!"
\par 27 Zis-a Rab?ache catre ei: "Au doara stapânul meu m-a trimis sa spun aceste cuvinte numai stapânului ?arii ?i ?ie? Nu, ci ?i poporului care sta pe zid ?i care va ajunge sa-?i manânce murdaria ?i sa-?i bea udul cu voi!"
\par 28 Apoi  s-a sculat Rab?ache ?i a strigat cu glas tare în limba iudaica ?i a spus aceste cuvinte: "Asculta?i cuvântul regelui celui mare, regele Asiriei!
\par 29 A?a zice regele: Sa nu va în?ele pe voi Iezechia, caci nu poate sa va izbaveasca din mâna mea,
\par 30 ?i sa nu va încurajeze Iezechia cu Domnul, zicând: Ne va izbavi Domnul ?i cetatea aceasta nu va fi data în mâinile regelui Asiriei.
\par 31 Sa nu asculta?i pe Iezechia, caci a?a zice regele Asiriei: Împaca?i-va cu mine ?i ie?i?i la mine; sa-?i manânce fiecare rodul vi?ei sale de vie ?i al smochinului sau ?i sa bea fiecare apa din fântâna sa, pâna nu vin sa va iau într-o ?ara la fel cu ?ara voastra,
\par 32 În ?ara pâinii ?i a vinului, în ?ara fructelor ?i a viilor, în ?ara smochinelor ?i mierei, ?i nu ve?i muri, ci ve?i trai. Deci nu asculta?i pe Iezechia, care va amage?te, zicând: Domnul ne va izbavi.
\par 33 Dumnezeii popoarelor au izbavit ei oare fiecare ?ara sa din mâna regelui Asiriei?
\par 34 Unde sunt dumnezeii Hamatului ?i ai Arpadului? Unde sunt dumnezeii Sefarvaimului, Inei ?i Hevei? Au scapat oare Samaria din mâna mea?
\par 35 Care din dumnezeii ?arilor acestora a izbavit ?ara sa din mâna mea? A?adar va izbavi Domnul Ierusalimul din mâna mea?"
\par 36 Poporul însa a tacut ?i nu i-a raspuns nici un cuvânt, pentru ca porunca regelui era sa nu-i raspunda.
\par 37 Dupa aceea a venit Eliachim, fiul lui Hilchia, capetenia cur?ii domne?ti, ?ebna scriitorul ?i Ioah cronicarul, fiul lui Asaf, la Iezechia, cu hainele sfâ?iate ?i i-au spus vorbele lui Rab?ache.

\chapter{19}

\par 1 Auzind acestea, regele Iezechia ?i-a rupt hainele sale, s-a îmbracat cu sac ?i s-a dus în templul Domnului.
\par 2 Atunci a trimis pe Eliachim, capetenia cur?ii domne?ti, pe ?ebna scriitorul ?i pe preo?ii cei mai mari, îmbraca?i în sac, la Isaia proorocul, fiul lui Amos,
\par 3 ?i ace?tia au zis catre el: "A?a zice Iezechia: Zi de necaz, de pedeapsa ?i de ru?ine este ziua aceasta, caci pruncii sunt aproape sa iasa din pântecele mamelor ?i ele nu pot na?te.
\par 4 Poate va auzi Domnul Dumnezeul tau toate cuvintele lui Rab?ache, pe care l-a trimis regele Asiriei, stapânul lui, sa defaime pe Dumnezeul cel viu ?i sa-L huleasca cu vorbele pe care le-a auzit Domnul Dumnezeul tau. Adu dar rugaciune pentru cei ce au ramas ?i se afla printre cei vii!"
\par 5 ?i au venit slugile regelui Iezechia la Isaia,
\par 6 Iar Isaia le-a zis: "A?a sa zice?i domnului vostru: A?a graie?te Domnul: Nu te teme de vorbele ce le-ai auzit ?i cu care M-au hulit pe Mine slugile regelui Asiriei.
\par 7 Caci iata voi trimite în el duh ?i va auzi o veste ?i se va întoarce în ?ara sa, iar acolo îl voi lovi cu sabia".
\par 8 Atunci s-a întors Rab?ache ?i a gasit pe regele Asiriei razboindu-se împotriva Libnei, caci auzise ca acesta a plecat din Lachi?.
\par 9 ?i a auzit el de Tirhaca, regele Etiopiei, caci i s-a spus: "Iata vine sa se lupte cu tine". ?i din nou a trimis soli la Iezechia sa-i spuna:
\par 10 "A?a sa zice?i lui Iezechia, regele Iudei: Sa nu te în?ele Dumnezeul tau, în Care nadajduie?ti tu, gândind: Nu va fi dat Ierusalimul în mâinile regelui Asiriei!
\par 11 Doar tu ai auzit ce a facut regele Asiriei cu toate ?arile, aruncând asupra lor blestem. ?i tu ai sa ramâi?
\par 12 Dumnezeii popoarelor pe care le-au ruinat parin?ii mei le-au izbavit ei oare?
\par 13 Unde este regele Hamatului, sau regele Arpadului, sau regele ceta?ii Sefarvaimului, Inei ?i Hevei?"
\par 14 ?i a luat Iezechia scrisoarea din mâna solilor ?i a citit-o; apoi s-a dus în templul Domnului ?i a deschis-o Iezechia înaintea felei Domnului.
\par 15 ?i s-a rugat Iezechia înaintea fe?ei Domnului ?i a zis: "Doamne Dumnezeul lui Israel, Cel ce ?ezi pe heruvimi, numai Tu singur e?ti Dumnezeul tuturor regatelor pamântului, Tu ai facut cerul ?i pamântul!
\par 16 Pleaca-?i, Doamne, urechea Ta ?i ma auzi! Deschide-?i, Doamne, ochii Tai ?i vezi ?i auzi cuvintele lui Senaherib, cel ce a trimis sa Te huleasca pe Tine, Dumnezeul cel viu!
\par 17 Adevarat, o, Doamne, regii Asiriei au pustiit popoarele ?i ?arile lor, au aruncat dumnezeii acestora în foc, dar aceia nu erau dumnezei, ci lucruri de mâini omene?ti, lemn ?i piatra, ?i de aceea i-a ?i nimicit pe ei.
\par 18 ?i acum, Doamne Dumnezeul nostru, izbave?te-ne din mâna lui,
\par 19 ?i vor afla toate regatele pamântului ca numai Tu, Doamne, e?ti Dumnezeu! "
\par 20 Atunci a trimis Isaia, fiul lui Amos, la Iezechia sa-i spuna: "A?a zice Domnul Dumnezeul lui Israel: Cele pentru care te-ai rugat tu Mie împotriva lui Senaherib, regele Asiriei, le-am auzit.
\par 21 Iata cuvântul pe care l-a rostit Domnul pentru el: Te dispre?uie?te ?i râde de tine fecioara, fiica Sionului, ?i clatina din cap, în urma ta, fiica Ierusalimului.
\par 22 Pe cine însa ai mustrat ?i ai hulit tu? ?i asupra cui ?i-ai ridicat tu glasul ?i ?i-ai înal?at a?a de sus ochii tai? Asupra Sfântului lui Israel.
\par 23 Prin trimi?ii tai tu ai înfruntat pe Domnul ?i ai zis: Cu mul?imea carelor mele m-am urcat pe înal?imea mun?ilor, pe coastele Libanului, ?i am taiat cedrii cei falnici ?i chiparo?ii cei minuna?i ai lui, ?i am ajuns la cea din urma adapostire a lui, la gradina lui plina de pomi;
\par 24 ?i am sapat ?i am baut apa straina ?i cu talpile picioarelor mele voi seca râurile Egiptului.
\par 25 N-ai auzit tu, oare, ca aceasta Eu am facut-o de demult, din zilele cele de demult am hotarât-o? Acum însa am îndeplinit-o prin aceea ca tu pustie?ti ?i prefaci ceta?ile în darâmaturi,
\par 26 Iar locuitorii lor au ajuns neputincio?i, tremura ?i ramân ru?ina?i; au ajuns ca iarba câmpului ?i ca verdea?a cea frageda, ca buruienile de pe acoperi?ul casei ?i ca ni?te fire de grâu uscate înainte de a da în spic.
\par 27 De ?ezi, de intri, ori de ie?i, Eu toate le ?tiu ?i ?tiu ?i obraznicia ta fa?a de Mine.
\par 28 Pentru obraznicia ta cea fa?a de Mine ?i pentru ca trufia ta a ajuns pâna la urechile Mele, Îmi voi pune veriga Mea în narile tale ?i în buzele tale belciugul Meu ?i te voi întoarce pe acela?i drum pe care ai venit.
\par 29 Iar pentru tine, Iezechia, iata semn: Anul acesta ve?i mânca din cele ce vor cre?te din semin?ele scuturate; în anul al doilea ve?i mânca din cele ce vor cre?te de la sine, iar în anul al treilea ve?i semana ?i veri secera, ve?i sadi vii ?i ve?i mânca roadele lor.
\par 30 Ceea ce nu se va strica în casa lui Iuda, ceea ce va  ramâne, va da radacini în jos, iar în sus va aduce rod, caci din Ierusalim vor ie?i câ?iva rama?i ?i din Sion, câ?iva izbavi?i.
\par 31 Râvna Domnului Savaot va face aceasta.
\par 32 De aceea a?a zice Domnul de regele Asiriei: Nu va intra în cetatea aceasta, nici va arunca sage?i încoace; nu se va apropia de ea cu scut, nici va face întarituri de ?an?uri împotriva ei.
\par 33 Pe drumul pe care a venit, se va întoarce ?i în cetatea aceasta nu va intra, zice Domnul;
\par 34 Caci Eu voi pazi cetatea aceasta, ca sa o izbavesc pentru Mine ?i pentru David, robul Meu".
\par 35 În noaptea aceea s-a întâmplat ca a ie?it îngerul Domnului ?i a lovit în tabara Asirienilor o suta optzeci ?i cinci de mii, ?i când s-au sculat diminea?a, iata erau peste tot numai trupuri moarte.
\par 36 Atunci Senaherib, regele Asiriei, sculându-se, a plecat ?i s-a întors ?i a locuit în Ninive.
\par 37 Dar pe când se ruga el în casa lui Nisroc, zeul sau, l-au ucis cu sabia fiii sai Adramelec ?i ?are?er ?i au fugit în pamântul Ararat, iar în locul lui s-a facut rege Asarhadon, fiul lui.

\chapter{20}

\par 1 În zilele acelea s-a îmbolnavit Iezechia de moarte ?i a venit la el Isaia proorocul, fiul lui Amos, ?i i-a zis: "A?a graie?te Domnul: Fa testament pentru casa ta, caci nu te vei mai însanato?i, ci vei muri!"
\par 2 Atunci s-a întors Iezechia cu fa?a la perete ?i s-a rugat Domnului,
\par 3 Zicând: "O, Doamne, adu-?i aminte ca am umblat înaintea fe?ei Tale cu credin?a ?i cu inima dreapta ?i am facut cele placute în ochii Tai!" ?i a plâns Iezechia tare.
\par 4 Isaia însa nu plecase înca din cetate, când a fost cuvântul Domnului catre el ?i i-a zis:
\par 5 "Întoarce-te ?i spune lui Iezechia, stapânul poporului Meu: A?a zice Domnul Dumnezeul lui David, stramo?ul tau: Am auzit rugaciunea ta ?i am vazut lacrimile tale; te vei vindeca ?i a treia zi te vei duce în templul Domnului;
\par 6 ?i voi mai adauga la zilele tale cincisprezece ani ?i din mâna regelui Asiriei te voi izbavi pe tine ?i cetatea aceasta o voi apara pentru Mine ?i pentru David, robul Meu!"
\par 7 ?i a zis Isaia: "Lua?i o turta de smochine! ?i au luat o turta de smochine ?i au pus-o pe rana ?i s-a însanato?it Iezechia.
\par 8 ?i a zis catre Isaia: "Care este semnul ca Domnul ma va vindeca ?i ca ma voi duce a treia zi în templul Domnului?"
\par 9 Iar Isaia a zis: "Iata semn de la Domnul ca-?i va împlini Domnul cuvântul pe care l-a rostit: Vrei sa treaca umbra la ceasul de soare cu zece linii înainte sau sa se dea cu zece linii înapoi?"
\par 10 Iezechia a zis: "E u?or ca umbra sa se mi?te cu zece linii înainte. Nu, ci sa se dea umbra cu zece linii înapoi".
\par 11 ?i a strigat Isaia proorocul catre Domnul ?i s-a dat înapoi cu zece linii.
\par 12 În vremea aceea Merodac Baladan, fiul lui Baladan, regele Babilonului, a trimis scrisoare ?i dar lui Iezechia caci auzise ca Iezechia a fost bolnav.
\par 13 Ascultând Iezechia pe trimi?i, le-a aratat camarile sale, argintul, aurul, aromatele, mirurile cele scumpe ?i toata casa sa de arme ?i tot ce se afla în vistieriile sale; ?i nu a ramas nici un lucru din casa sa ?i din toata stapânirea sa pe care sa nu-l fi aratat lor Iezechia.
\par 14 Venind însa Isaia proorocul la regele Iezechia, a zis catre el: "Ce au zis oamenii ace?tia ?i de unde au venit la tine?" Iezechia a raspuns: "Dintr-o ?ara departata, au venit din Babilon".
\par 15 Isaia a zis: "Ce au vazut ei în casa ta?" ?i Iezechia a zis: "Tot ce este în casa mea au vazut ?i n-a ramas nici un lucru din vistieriile mele pe care sa nu-l fi vazut".
\par 16 Atunci Isaia a zis: "Asculta cuvântul Domnului: Iata vor veni zile când vor fi luate toate câte sunt în casa ta ?i ce-au adunat parin?ii tai pâna în ziua aceasta ?i vor fi duse la Babilon. Nimic nu va ramâne, zice Domnul.
\par 17 Din fiii tai care vor rasari din tine ?i pe care îi vei na?te tu,
\par 18 Se vor lua ?i vor fi eunuci în palatul regelui Babilonului".
\par 19 Iezechia a raspuns lui Isaia: "Bun este cuvântul Domnului pe care l-ai rostit tu!" Apoi a adaugat: "Sa fie pace ?i lini?te în zilele mele!"
\par 20 Celelalte fapte ale lui Iezechia, luptele lui ?i cum ca el a facut iazul ?i canalul pentru adus apa în cetate, sunt scrise în cartea faptelor regilor lui Iuda.
\par 21 Apoi a raposat Iezechia cu parin?ii sai ?i în locul lui s-a facut rege Manase, fiul sau.

\chapter{21}

\par 1 Manase era de doisprezece ani când s-a facut rege ?i a domnit în Ierusalim cincizeci ?i cinei de ani. Numele mamei lui era Hef?ibah.
\par 2 Acesta a facut lucruri netrebnice înaintea Domnului, urmând urâciunile pagânilor pe care-i izgonise Domnul de la fa?a fiilor lui Israel.
\par 3 El a facut din nou înal?imile pe care le stricase tatal sau Iezechia ?i a a?ezat jertfelnice pentru Baal; a facut A?ere, cum facuse ?i Ahab, regele lui Israel, ?i s-a închinat la toata o?tirea cereasca, slujind acesteia.
\par 4 Apoi a zidit jertfelnice chiar ?i în templul Domnului, de care zisese Domnul: "În Ierusalim voi pune numele Meu!"
\par 5 ?i a facut jertfelnice la toata o?tirea cerului în amândoua curiile templului Domnului;
\par 6 A trecut pe fiul sau prin foc, a ghicit, a vrajit, a adus oameni care se îndeletniceau cu chemarea mor?ilor ?i vrajitori ?i a facut ?i alte multe lucruri urâte Domnului, ca sa-L mânie.
\par 7 Dupa aceea chipul A?erei pe care îl facuse l-a a?ezat în casa despre care Domnul îi zisese lui David ?i lui Solomon, fiul lui: "În casa aceasta ?i în Ierusalim, pe care l-am ales din toate semin?iile lui Israel, voi pune numele Meu pe vecie;
\par 8 ?i nu voi mai da sa calce picior de israelit afara din ?ara pe care am dat-o parin?ilor lor, de se vor sili sa se poarte potrivit cu toate cele ce Eu le-am poruncit ?i cu toata legea care le-a dat-o robul Meu Moise".
\par 9 Dar ei n-au ascultat, ci i-a ratacit Manase pâna într-atâta, încât ei s-au purtat mai rau decât acele popoare pe care Domnul le stârpise de la fa?a fiilor lui Israel.
\par 10 Atunci Domnul a grait prin prooroci, robii Sai, ?i a zis:
\par 11 "Pentru ca Manase, regele Iudei, a facut astfel de urâciuni, mai rele decât tot ce au facut Amoreii care au fost înainte de el, ?i a vârât pe Iuda în pacat cu idolii lui,
\par 12 De aceea a?a zice Domnul Dumnezeul lui Israel: Iata Eu voi aduce a?a rau asupra Ierusalimului ?i asupra lui Iuda, încât celui ce va auzi îi vor ?iui amândoua urechile;
\par 13 ?i voi întinde peste Ierusalim frânghia de masurat a Samariei ?i cumpana casei lui Ahab, ?i voi ?terge Ierusalimul, a?a cum se ?terge un vas ?i se pune apoi cu gura în jos;
\par 14 ?i voi lepada rama?ita mo?tenirii Mele ?i-i voi da în mâinile vrajma?ilor lor, ?i vor fi de prada ?i de jaf pentru to?i prietenii lor,
\par 15 Pentru ca au facut lucruri netrebnice înaintea ochilor Mei ?i M-au mâniat din ziua aceea când parin?ii lor au ie?it din Egipt ?i pâna în ziua aceasta".
\par 16 Mai mult înca, Manase, pe lânga pacatul sau de a fi dus pe Iuda în ispita, a varsat ?i foarte mult sânge nevinovat, încât a mânjit Ierusalimul de la o margine la alta.
\par 17 Celelalte ?tiri despre Manase ?i despre toate câte a facut el ?i despre pacatele lui, în ce anume a pacatuit, sunt scrise în cartea faptelor regilor lui Iuda.
\par 18 Apoi a raposat Manase cu parin?ii sai ?i a fost îngropat în gradina de lânga casa lui, în gradina lui Uza, iar în locul lui s-a facut rege Amon, fiul sau.
\par 19 Amon era de douazeci ?i doi de ani când s-a facut rege ?i a domnit doi ani în Ierusalim. Numele mamei lui era Me?ulemet, fiica lui Haru? din Iotba.
\par 20 ?i acesta a facut lucruri netrebnice în ochii Domnului, cum facuse ?i Manase, tatal sau;
\par 21 A umblat întocmai pe aceea?i Cale pe care umblase ?i tatal sau, slujind idolilor carora slujise ?i tatal sau ?i închinându-se lor.
\par 22 A parasit pe Domnul Dumnezeul parin?ilor sai ?i n-a umblat în calea Domnului.
\par 23 Dar slugile lui Amon s-au razvratit împotriva lui ?i au ucis pe rege în casa lui.
\par 24 Poporul însa a ucis pe to?i cei ce luasera parte la razvratire împotriva regelui Amon ?i a pus rege în locul lui pe Iosia, fiul lui.
\par 25 Celelalte ?tiri despre Amon ?i despre cele ce a facut el sunt scrise în cartea faptelor regilor lui Iuda.
\par 26 Amon a fost îngropat în gropni?a lui, în gradina lui Uza, iar în locul lui s-a facut rege Iosia, fiul lui.

\chapter{22}

\par 1 Iosia era de opt ani când s-a facut rege ?i a domnit treizeci ?i unu de ani în Ierusalim; numele mamei lui era Iedida, fiica lui Adaia din Bo?cat.
\par 2 Iosia a facut fapte placute înaintea Domnului ?i a umblat în toate pe calea lui David, stramo?ul sau, neabatându-se nici la dreapta, nici la stânga.
\par 3 În anul al optsprezecelea al regelui Iosia, regele a trimis pe scriitorul ?afan, fiul lui A?alia, fiul lui Me?ulam, în templul Domnului, zicându-i:
\par 4 "Du-te la Hilchia, arhiereul, ca sa socoteasca argintul adus în templul Domnului, pe care l-au strâns de la popor cei ce stau de straja la prag,
\par 5 ?i sa-l dea în mâna celor pu?i sa faca lucrarile la templul Domnului, iar ace?tia sa-l dea celor ce lucreaza ?i repara stricaciunile lui:
\par 6 Dulgherilor, pietrarilor ?i zidarilor, ?i la cumpararea lemnului ?i a pietrelor cioplite pentru repararea templului;
\par 7 Însa sa nu le cere?i socoteala de argintul ce s-a dat în mâna lor, pentru ca se poarta cinstit".
\par 8 Iar Hilchia arhiereul a zis catre ?afan scriitorul: "Am gasit în templul Domnului cartea legii". Apoi Hilchia a dat lui ?afan cartea ?i el a citit-o.
\par 9 ?i venind ?afan scriitorul la rege, a adus raspuns regelui ?i a zis: "Robii tai au luat argintul ce s-a gasit în casa ?i l-au dat în mâinile celor pu?i sa faca lucrarile la templul Domnului".
\par 10 ?i a mai adus ?afan la cuno?tin?a regelui ?i acestea, zicând: "Preotul Hilchia mi-a dat o carte". ?i a citit-o ?afan înaintea regelui.
\par 11 Auzind regele cuvintele car?ii legii, ?i-a sfâ?iat hainele sale.
\par 12 Apoi regele a poruncit preotului Hilchia, lui Ahicam, fiul lui ?afan, lui Acbor, fiul lui Miheia, lui ?afan scriitorul ?i lui Asaia, sluga regelui:
\par 13 "Merge?i ?i întreba?i pe Domnul pentru mine ?i pentru popor ?i pentru tot Iuda despre cuvintele acestei car?i gasite, caci mare este mânia Domnului ce s-a aprins asupra noastra, pentru ca parin?ii no?tri n-au ascultat cuvintele car?ii acesteia, ca sa se poarte dupa cele ce ni s-a poruncit".
\par 14 Atunci s-a dus Hilchia preotul, Ahicam, Acbor, ?afan ?i Asaia la prooroci?a Hulda, so?ia lui ?alum, fiul lui ?icva, fiul lui Harhas, pastratorul ve?mintelor, care locuia în Ierusalim, în despar?itura a doua, ?i au grait cu ea.
\par 15 Iar ea le-a zis: "A?a graie?te Domnul Dumnezeul lui Israel: Spune?i omului care v-a trimis la mine:
\par 16 A?a zice Domnul: Voi aduce rau asupra locului acestuia ?i asupra locuitorilor lui, dupa toate cuvintele car?ii pe care a citit-o regele Iudei.
\par 17 Pentru ca M-au parasit ?i tamâiaza pe al?i dumnezei, ca sa ma a?â?e cu toate lucrurile mâinilor lor, s-a aprins mânia Mea asupra locului acestuia ?i nu se va stinge.
\par 18 Iar regelui lui Iuda care v-a trimis sa întreba?i pe Domnul, spune?i-i: A?a zice Domnul Dumnezeul lui Israel despre cuvintele pe care tu le-ai auzit:
\par 19 Deoarece s-a muiat inima ta ?i tu te-ai smerit înaintea Domnului, când ai auzit ce am grait Eu asupra locului acestuia ?i asupra locuitorilor lui, ca vor fi ?inta groazei ?i a blestemului, ?i ji-ai rupt ve?mintele ?i ai plâns înaintea Mea, de aceea ?i Eu te-am auzit, zice Domnul.
\par 20 De aceea iata te voi adauga la parin?ii tai ?i vei fi pus în gropni?a ta cu pace; ?i nu vor vedea ochii tai toate acele nenorociri pe care le voi aduce asupra locului acestuia". ?i s-a adus regelui raspunsul acesta.

\chapter{23}

\par 1 Atunci a trimis regele sa fie chema?i to?i batrânii lui Iuda ?i ai Ierusalimului.
\par 2 Apoi s-a dus regele în templul Domnului ?i to?i Iudeii ?i to?i locuitorii Ierusalimului au mers cu el, ?i preo?ii ?i proorocii ?i tot poporul de la mic pâna la mare ?i au citit în auzul lor toate cuvintele car?ii legamântului, ce s-a gasit în templul Domnului.
\par 3 Dupa aceea a stat regele pe un loc înalt ?i a încheiat înaintea Domnului legamânt ca sa urmeze Domnului ?i sa pazeasca poruncile Lui, descoperirile Lui ?i legiuirile Lui cu toata inima sa ?i cu tot sufletul, ca sa împlineasca cuvintele legamântului acestuia, scrise în cartea aceasta.
\par 4 Apoi regele a poruncit lui Hilchia arhiereul, preo?ilor de mâna a doua ?i celor ce stateau de straja la prag, sa scoata din templul Domnului toate lucrurile facute pentru Baal, pentru Astarte ?i pentru toata o?tirea cerului ?i sa le arda afara din Ierusalim, în valea Chedronului, iar cenu?a lor sa o duca la Betel.
\par 5 A izgonit dupa aceea pe preo?ii idolilor pe care-i pusesera regii Iudei, ca sa faca tamâieri pe înal?imi, în ceta?ile Iudei ?i în împrejurimile Ierusalimului, ?i care tamâiau pe Baal, soarele, luna, stelele ?i toata o?tirea cerului.
\par 6 Atunci au scos A?era din templul Domnului afara din Ierusalim, la pârâul Chedron, au ars-o la pârâul Chedron ?i au facut-o praf; ?i praful l-au aruncat asupra cimitirului ob?tesc al poporului.
\par 7 Apoi au darâmat casele de desfrâu care se aflau lânga templul Domnului, unde femeile ?ineau ve?minte pentru Astarte;
\par 8 Au scos pe to?i slujitorii idolilor din ceta?ile lui Iuda ?i au spurcat înal?imile pe care ei savâr?eau tamâieri, de la Gheba pâna la Beer-?eba, stricând înal?imile de la por?i: cea care se afla la poarta lui Iosua, capetenia ceta?ii, ?i cea care se afla în partea stânga a por?ilor ceta?ii.
\par 9 De atunci slujitorii înal?imilor nu aduceau jertfe pe jertfelnicul Domnului celui din Ierusalim, ci mâncau numai azime cu fra?ii lor.
\par 10 Dupa aceea regele a spurcat locurile de jertfa din valea fiilor lui Hinom, ca nimeni sa nu mai treaca pe fiul sau sau pe fiica sa prin foc lui Moloh.
\par 11 A nimicit caii pe care regii lui Israel îi a?ezasera în cinstea soarelui înaintea intrarii templului Domnului, aproape de locuin?a eunucului Netan-Melec cea din Parvarim, iar carul soarelui l-a ars.
\par 12 Jertfelnicele cele de pe acoperi?ul foi?orului lui Ahaz, pe care le facusera regii Iudei, ?i jertfelnicele pe care le facuse Manase în amândoua cur?ile templului Domnului, le-a darâmat regele ?i le-a luat de acolo, aruncând molozul lor în pârâul Chedronului.
\par 13 Apoi a spurcat regele cele doua înal?imi din fa?a Ierusalimului, din dreapta Muntelui Maslinilor, pe care le facuse Solomon, regele lui Israel, pentru Astarte, idolul Sidonului, pentru Chemo?, idolul Moabului, ?i pentru Milcom, idolul Amoni?ilor.
\par 14 A sfarâmat stâlpii idole?ti ?i a taiat A?erele ?i locul lor l-a umplut cu oase omene?ti.
\par 15 De asemenea ?i jertfelnicul cel din Betel ?i înal?imea facuta de Ieroboam, fiul lui Nabat, care a dus pe Israel în pacat, le-a stricat regele Iosia ?i a ars înal?imea aceasta ?i a facut-o praf; ?i a mai ars ?i pe A?era.
\par 16 ?i întorcând capul, Iosia a vazut mormintele ce erau acolo pe munte, ?i a trimis de a luat oasele din morminte ?i le-a ars pe jertfelnic ?i l-a spurcat dupa cuvântul Domnului, rostit de omul lui Dumnezeu care prezisese întâmplarea aceasta, când statea Ieroboam în timpul unei sarbatori înaintea jertfelnicului. Dupa aceea, întorcându-se, Iosia a ridicat ochii la mormântul omului lui Dumnezeu care prezisese întâmplarea aceasta,
\par 17 ?i a zis: "Ce mormânt este acesta, pe care-l vad eu?" Iar locuitorii ceta?ii i-au raspuns: "Acesta este mormântul omului lui Dumnezeu care a venit din Iuda ?i a prevestit cele ce tu faci cu jertfelnicul din Betel".
\par 18 Iosia a zis: "Lasa?i-l în pace; nimeni sa nu atinga oasele lui". ?i au pastrat oasele lui ?i oasele proorocului care venise din Samaria.
\par 19 De asemenea a darâmat Iosia ?i toate capi?tile înal?imilor din ceta?ile samarinene, pe care le facusera regii lui Israel, mâniind pe Domnul, ?i a facut cu ele ceea ce facuse ?i în Betel;
\par 20 ?i a junghiat pe jertfelnice pe to?i preo?ii înal?imilor care erau acolo; a ars oase omene?ti pe jertfelnice ?i apoi s-a întors în Ierusalim.
\par 21 Dupa aceea a poruncit regele la tot poporul ?i a zis: "Savâr?i?i Pa?tile Domnului Dumnezeului vostru, dupa cum este scris în aceasta carte a legii!"
\par 22 Pentru ca nu se mai savâr?ise astfel de Pa?ti din zilele Judecatorilor, care judecasera pe Israel, în tot timpul regilor lui Iuda ?i al regilor lui Israel;
\par 23 Iar în anul al optsprezecelea al regelui Iosia s-au savâr?it aceste Pa?ti ale Domnului, în Ierusalim.
\par 24 A mai nimicit regele Iosia pe cei ce se îndeletniceau cu chemarea mor?ilor, pe vrajitori, pe terafimii, idolii ?i toate urâciunile care se ivisera în pamântul lui Iuda ?i în Ierusalim, ca sa împlineasca cuvintele legii, scrise în cartea pe care o gasise Hilchia preotul în templul Domnului.
\par 25 Asemenea lui Iosia n-a mai fost rege înainte de el, care sa se fi întors la Domnul cu toata inima sa, cu toate puterile sale ?i cu tot sufletul sau, dupa toata legea lui Moise, dar nici dupa el nu s-a mai ridicat altul asemenea lui.
\par 26 Cu toate acestea Domnul n-a schimbat marea iu?ime a mâniei Sale, cu care se aprinsese mânia Sa asupra lui Iuda, din pricina tuturor relelor pe care le facuse Manase ca sa-L mânie.
\par 27 ?i a zis Domnul: "?i pe Iuda îl voi lepada de la fa?a Mea, cum am lepadat pe Israel; ?i voi lepada ora?ul acesta, Ierusalimul, pe care l-am ales ?i casa despre care am zis: "Acolo va fi numele Meu".
\par 28 Celelalte ?tiri despre Iosia ?i despre toate câte a facut el sunt scrise în cartea Cronicilor regilor lui Iuda.
\par 29 În zilele lui s-a dus Faraonul Neco, regele Egiptului, împotriva regelui Asiriei, la râul Eufratului. Atunci a ie?it Iosia în întâmpinarea lui, iar acela când l-a vazut l-a omorât în Meghido.
\par 30 Iar robii lui l-au luat mort din Meghido ?i l-au dus la Ierusalim de l-au îngropat în gropni?a lui. Apoi a luat poporul ?arii pe Ioahaz, fiul lui Iosia, l-au uns ?i l-au facut rege în locul tatalui lui.
\par 31 Ioahaz era de douazeci ?i trei de ani când s-a facut rege ?i a domnit în Ierusalim trei luni. Numele mamei lui era Hamutal, fiica lui Ieremia din Libna.
\par 32 Ioahaz a facut lucruri netrebnice în ochii Domnului, întocmai cum facusera parin?ii lui.
\par 33 Dar l-a legat Faraonul Neco în Ribla, în ?ara Hamat, ca sa nu mai domneasca în Ierusalim, ?i a pus pe ?ara un bir de o suta de talan?i de argint ?i o suta de talan?i de aur.
\par 34 Apoi Faraonul Neco a pus rege pe Eliachim, fiul lui Iosia, în locul lui Iosia, tatal lui, dar i-a schimbat numele în Ioiachim; pe Ioahaz l-a luat ?i l-a dus în Egipt, unde a murit.
\par 35 Ioiachim a dat lui Faraon aurul ?i argintul; ?i a pre?uit el ?i pamântul, ca sa se dea argint dupa porunca lui Faraon; ?i a cerut la fiecare din poporul ?arii sa aduca dupa pre?uirea sa aur ?i argint ca sa dea Faraonului Neco.
\par 36 Ioiachim însa era de douazeci ?i cinci de ani când s-a facut rege ?i a domnit în Ierusalim unsprezece ani. Numele mamei lui era Zebuda, fiica lui Pedaia, din Ruma.
\par 37 Acesta a facut rele înaintea Domnului, cum facusera ?i parin?ii lui.

\chapter{24}

\par 1 În zilele lui Ioiachim a venit cu razboi Nabucodonosor, regele Babilonului, ?i Ioiachim a ajuns supusul lui timp de trei ani, dar apoi s-a rasculat împotriva lui.
\par 2 Atunci Domnul a trimis asupra lui cete de Caldei, cete de Sirieni, cete de Moabi?i ?i cete de Amoni?i; ?i le-a trimis asupra lui Iuda, ca sa-l piarda, dupa cuvântul Domnului pe care l-a rostit prin robii Sai, proorocii.
\par 3 Aceasta s-a facut cu Iuda numai din porunca Domnului, ca sa fie lepadat de la fa?a Lui pentru pacatele lui Manase ?i pentru tot ce facuse acesta;
\par 4 ?i pentru sângele nevinovat pe care-l varsase el, umplând tot Ierusalimul, Domnul n-a vrut sa-l ierte.
\par 5 Celelalte ?tiri despre Ioiachim ?i despre tot ce a facut el sunt scrise în cartea Cronicilor regilor lui Iuda.
\par 6 Ioiachim a raposat cu parin?ii sai, iar în locul lui s-a facut rege Iehonia, fiul sau.
\par 7 Regele Egiptului n-a mai ie?it din ?ara sa, pentru ca regele Babilonului a luat regelui Egiptului tot ce avea acesta de la râul Egiptului pâna la râul Eufratului.
\par 8 Iehonia era de optsprezece ani când s-a facut rege ?i a domnit în Ierusalim trei luni. Numele mamei lui era Nehu?ta, fiica lui Elnatan, din Ierusalim.
\par 9 El a facut lucruri netrebnice în ochii Domnului, în toate, a?a cum facuse tatal sau.
\par 10 În vremea aceea slujitorii lui Nabucodonosor, regele Babilonului, au venit asupra Ierusalimului ?i au împresurat cetatea. Iar dupa ce slugile lui au înconjurat cetatea, a venit ?i regele Nabucodonosor.
\par 11 ?i a ie?it Iehonia, regele lui Iuda, la regele Babilonului, împreuna cu mama sa, cu slujitorii sai, cu capeteniile ?i eunucii lui,
\par 12 ?i l-a luat rob regele Babilonului în al optulea an al domniei sale;
\par 13 ?i a scos toate comorile templului Domnului ?i comorile casei domne?ti, ?i a sfarâmat, dupa cum spusese Domnul, toate vasele cele de aur pe care le facuse Solomon, regele lui Israel, pentru templul Domnului, ?i ?a dus în robie tot Ierusalimul,
\par 14 Pe to?i frunta?ii ?i pe to?i oamenii viteji, aproape zece mii de robi, cu to?i dulgherii ?i fierarii, ?i n-a ramas nimeni decât numai poporul ?arii cel sarac.
\par 15 ?i a dus ?i pe Iehonia la Babilon; de asemenea au dus robi din Ierusalim la Babilon pe mama ?i femeile regelui, pe eunucii lui ?i pe puternicii ?arii;
\par 16 Toata o?tirea în numar de ?apte mii, teslarii, fierarii în numar de o mie, ?i to?i oamenii vârstnici ?i buni de o?tire i-a dus regele Babilonului robi la Babilon.
\par 17 Atunci regele Babilonului a pus rege în locul lui Iehonia, pe Matania, unchiul lui Iehonia, schimbându-i numele în Sedechia.
\par 18 Sedechia era de douazeci ?i unu de ani când s-a facut rege ?i a domnit în Ierusalim unsprezece ani. Numele mamei lui era Hamutal, fiica lui Ieremia din Libna.
\par 19 ?i a facut ?i acesta lucruri netrebnice în ochii Domnului, în toate, a?a cum facuse ?i Iehonia.
\par 20 ?i mânia Domnului era peste Ierusalim ?i peste Iuda atât de mare, încât i-a lepadat de la fa?a Sa. ?i s-a lepadat ?i Sedechia de regele Babilonului.

\chapter{25}

\par 1 Iar în anul al noualea al domniei lui Sedechia, în luna a zecea, în ziua a zecea a lunii, a venit Nabucodonosor, regele Babilonului, cu toata o?tirea sa asupra Ierusalimului ?i l-a împresurat ?i a facut împrejurul lui întarituri.
\par 2 ?i a stat cetatea împresurata pâna în anul al unsprezecelea al regelui Sedechia.
\par 3 Iar în anul al unsprszecelea al regelui Sedechia, în ziua a noua a lunii a patra, era mare foamete în cetate ?i poporul ?arii nu mai avea pâine.
\par 4 Atunci cetatea a fost luata ?i to?i osta?ii au fugit noaptea pe calea por?ilor care se aflau între doua ziduri, lânga gradina regelui. Caldeii însa stateau împrejurul ceta?ii; ?i a ie?it ?i regele pe calea ce duce în câmpie.
\par 5 Dar a alergat o?tirea Caldeilor dupa rege ?i l-au ajuns în ?esul Ierihonului; iar o?tirea lui a fugit toata de la el.
\par 6 ?i au luat pe rege ?i l-au dus la regele Babilonului, în Ribla, ?i l-au supus judeca?ii.
\par 7 ?i au junghiat pe fiii regelui înaintea ochilor lui, iar lui Sedechia i-au scos ochii, l-au pus în lan?uri ?i l-au dus la Babilon.
\par 8 Iar în luna a cincea, în ziua a ?aptea a lunii, adica în anul al nouasprezecelea al lui Nabucodonosor, regele Babilonului, Nebuzaradan, capetenia garzii, slujitorul regelui Babilonului, a venit la Ierusalim
\par 9 ?i a ars templul Domnului, casa regelui ?i toate casele din Ierusalim; toate casele cele mari le-a ars cu foc.
\par 10 Iar o?tirea Caldeilor care era cu comandantul garzii a darâmat ?i zidurile cele dimprejurul Ierusalimului.
\par 11 Apoi Nebuzaradan, capetenia garzii, a stramutat la Babilon ?i celalalt popor care mai ramasese în Ierusalim, pe cei ce se predasera regelui Babilonului ?i rama?i?a poporului de rând.
\par 12 Numai pu?ini din poporul sarac al ?arii au fost lasa?i de capetenia garzii, sa lucreze viile ?i ogoarele.
\par 13 Caldeii au stricat ?i stâlpii cei de arama care erau în templul Domnului, postamentele, marea cea de arama din templul Domnului ?i arama lor au dus-o în Babilon.
\par 14 Caldarile, lope?ile, cu?itele, lingurile ?i toate vasele de arama, care se întrebuin?au la slujba, le-au luat.
\par 15 ?i a mai luat capetenia garzii cadelni?ele, cupele ?i tot ce era de aur ?i ce era de argint.
\par 16 Au luat cei doi stâlpi, marea ?i postamentele pe care le facuse Solomon pentru templul Domnului. Arama din toate aceste lucruri nu se mai putea cântari.
\par 17 Un singur stâlp era înalt de optsprezece co?i; coroana lui era de arama ?i înal?imea ei era de trei co?i; pere?ii ei ?i împletiturile dimprejurul ei, toate erau de arama. Asemenea era ?i al doilea stâlp cu coroana lui.
\par 18 Apoi a mai luat capetenia garzii pe Seraia arhiereul, pe ?efania, al doilea preot ?i pe al?i trei care stateau de straja la prag.
\par 19 Iar din cetate a luat un eunuc, care era capetenie peste o?teni ?i cinci oameni, care stateau înaintea regelui ?i care acum se aflau în cetate; pe capetenia cea mare a o?tirii, care înscria la oaste poporul ?i ?aizeci de oameni din poporul ?arii, care se aflau în cetate.
\par 20 Pe ace?tia i-a luat Nebuzaradan, capetenia garzii ?i i-a dus la regele Babilonului, în Ribla.
\par 21 Iar regele Babilonului i-a lovit ?i i-a ucis la Ribla, în ?inutul Hamat. A?a a fost stramutat Iuda din ?ara lui.
\par 22 Iar peste poporul care a ramas în ?ara Iudei ?i pe care l-a lasat Nabucodonosor, regele Babilonului, a pus capetenie pe Ghedalia, fiul lui Ahican, fiul lui ?afan.
\par 23 Când au auzit toate capeteniile ?i osta?ii lor ca regele Babilonului a pus capetenie pe Ghedalia, au venit la Ghedalia în Mi?pa: Ismael, fiul lui Netania, Iohanan, fiul lui Careah, Seraia, fiul lui Tanhumet, din Netofa, ?i Iaazania, fiul lui Maacati, ei împreuna cu oamenii lor.
\par 24 ?i a jurat Ghedalia acestora ?i oamenilor lor ?i le-a zis: "Nu va temeri a fi supu?ii Caldeilor. A?eza?i-va în ?ara aceasta ?i sluji?i regelui Babilonului ?i va fi bine!"
\par 25 Dar în luna a ?aptea a venit Ismael, fiul lui Netania al lui Eli?ama din neamul regesc, cu zece oameni ?i a lovit pe Ghedalia ?i acesta a murit ?i a lovit ?i pe Iudeii ?i pe Caldeii care erau cu el în Mi?pa.
\par 26 Atunci s-a sculat tot poporul, de la mic pâna la mare, cu capeteniile o?tirii ?i s-au dus în, Egipt, pentru ca se temeau de Caldei.
\par 27 În anul al treizeci ?i ?aptelea de la stramutarea lui Ioiachim, regele lui Iuda, în luna a douasprezecea, în ziua a douazeci ?i ?aptea a acestei luni, Evil-Merodac, regele Babilonului, în anul urcarii sale pe tron, a scos pe Ioiachim, regele Iudei, din temni?a,
\par 28 A vorbit cu el prietenos ?i a pus tronul lui mai sus de tronurile regilor care erau la el în Babilon;
\par 29 I-a schimbat hainele lui de temni?a ?i Ioiachim a mâncat totdeauna la masa regelui, în toate zilele vierii lui.
\par 30 Cele trebuitoare hranei lui i-au fost date neîncetat de rege, zi cu zi, cât a trait el.


\end{document}