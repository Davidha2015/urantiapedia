\begin{document}

\title{2 Regi}


\chapter{1}

\par 1 După moartea lui Ahab, Moabul s-a răzvrătit împotriva lui Israel.
\par 2 Iar Ohozia, căzând printre gratiile foișorului său cel din Samaria, s-a îmbolnăvit și a trimis soli și le-a zis: "Duceți-vă și întrebați pe Baal-Zebub, dumnezeul Ecronului: Mă voi însănătoși eu oare din boala aceasta?" Și aceștia s-au dus să întrebe.
\par 3 Atunci îngerul Domnului a zis către Ilie Tesviteanul: "Scoală și ieși înaintea trimișilor regelui Samariei și le spune: Au doară în Israel nu este Dumnezeu, de vă duceți să întrebați pe Baal-Zebub, dumnezeul Ecronului?
\par 4 De aceea așa zice Domnul: Din patul în care te-ai suit, nu te vei mai coborî, ci vei muri". Și s-a dus Ilie și le-a spus.
\par 5 Atunci s-au întors solii la Ohozia și acesta le-a zis: "De ce v-ați întors?"
\par 6 Iar ei i-au răspuns: "Ne-a ieșit înainte un om și ne-a zis: Întoarceți-vă și vă duceți la regele care v-a trimis și-i spuneți: Așa zice Domnul: Au doar nu este în Israel Dumnezeu, de trimiți să întrebe pe Baal-Zebub, dumnezeul Ecronului? De aceea nu te vei mai coborî din patul în care te-ai suit, ci vei muri".
\par 7 Iar regele le-a zis: "Ce înfățișare avea omul acela care v-a ieșit înainte și v-a grăit cuvintele acestea?"
\par 8 Și ei i-au răspuns: "Omul acela este păros peste tot și încins peste mijloc cu o cingătoare de curea". A zis regele: "Acela este Ilie Tesviteanul".
\par 9 Atunci regele a trimis la el o căpetenie peste cincizeci cu cei cincizeci ai lui; și acesta a venit la el, când Ilie sta pe vârful muntelui și i-a zis: "Omul lui Dumnezeu, regele îți zice: "Coboară-te!"
\par 10 Iar Ilie a răspuns: "De sunt omul lui Dumnezeu, să se coboare foc din cer și să te ardă pe tine și pe cei cincizeci ai tăi!" Și s-a coborât foc din cer și l-a mistuit pe el și pe cei cincizeci ai lui.
\par 11 Apoi a trimis regele la el altă căpetenie cu alți cincizeci. Acesta i-a zis: "Omul lui Dumnezeu, așa a zis regele: Coboară-te degrabă!"
\par 12 Și răspunzând, Ilie i-a zis: "De sunt omul lui Dumnezeu, să se coboare foc din cer și să te ardă pe tine și pe cei cincizeci ai tăi!" Și s-a coborât focul lui Dumnezeu din cer și l-a ars pe el și pe cei cincizeci ai lui.
\par 13 Și a mai trimis regele a treia oară o căpetenie cu cincizeci. Dar a treia căpetenie, venind și căzând în genunchi înaintea lui Ilie, l-a rugat, zicând: "Omul lui Dumnezeu, să nu fie trecut cu vederea de ochii tăi sufletul meu și sufletul acestor cincizeci de robi ai tăi!
\par 14 Iată s-a coborât foc din cer și a mistuit pe cele două căpetenii peste cincizeci și pe oamenii lor; acum însă să nu fie sufletul meu trecut cu vederea de ochii tăi!"
\par 15 Atunci îngerul Domnului a zis către Ilie: "Du-te cu el și nu te teme de el!" Și s-a sculat Ilie și s-a dus cu el la rege.
\par 16 Și a zis către el: "Așa zice Domnul: De vreme ce tu ai trimis soli să întrebe pe Baal-Zebub, dumnezeul Ecronului, ca și cum în Israel n-ar fi Dumnezeu, ca să-I ceri cuvântul, de aceea din patul în care te-ai suit nu te vei mai coborî, ci vei muri".
\par 17 Și apoi a murit Ohozia, după cuvântul Domnului pe care l-a rostit Ilie. Și în locul lui s-a făcut rege Ioram, fratele lui Ohozia, în anul al doilea al lui Ioram, fiul lui Iosafat, regele Iudei, căci Ohozia nu avea fecior.
\par 18 Celelalte fapte pe care le-a făcut Ohozia sunt scrise în cartea cronicilor regilor lui Israel.

\chapter{2}

\par 1 În vremea când Domnul a vrut să înalțe pe Ilie în vârtej de vânt la cer Ilie a plecat cu Elisei din Ghilgal.
\par 2 Și Ilie a zis către Elisei: "Stai aici, căci Dumnezeu mă trimite la Betel. Iar Elisei a zis: "Cât este de adevărat că Domnul este viu și cum este viu și sufletul tău, tot așa de adevărat este că nu te voi lăsa singur". și s-au dus amândoi la Betel.
\par 3 Și au ieșit fiii proorocilor cei din Betel la Elisei și au zis către el: "Știi oare că astăzi Domnul va să ridice pe stăpânul tău deasupra capului tău?" Și el a zis: "Știu și eu, dar tăceți!"
\par 4 Atunci Ilie a zis către el: "Elisei, rămâi aici, căci Domnul mă trimite la Ierihon". Iar Elisei a zis: "Cât este de adevărat că Domnul este viu și viu este și sufletul tău, tot așa de adevărat este că nu te voi lăsa singur!"
\par 5 Și au venit amândoi la Ierihon. Atunci s-au apropiat fiii proorocilor cei din Ierihon de Elisei și i-au zis: "Știi oare că Domnul ia pe stăpânul tău și-l înalță deasupra capului tău?" Și el a răspuns: "Știu și eu, dar tăceți!"
\par 6 A zis Ilie: "Rămâi aici, căci Domnul mă trimite la Iordan!" Iar Elisei a răspuns: "Cât este de adevărat că Domnul este viu și cum este viu și sufletul tău, tot așa de adevărat este că nu te voi lăsa singur!"
\par 7 Și s-au dus amândoi; s-au dus și cincizeci din fiii proorocilor și au stat deoparte în fața lor, iar ei amândoi ședeau lângă Iordan.
\par 8 Atunci, luând Ilie mantia sa și strângând-o vălătuc, a lovit cu ea apa și aceasta s-a strâns la dreapta și la stânga și au trecut ca pe uscat.
\par 9 Iar după ce au trecut, a zis Ilie către Elisei: "Cere ce să-ți fac, înainte de a fi luat de la tine". Iar Elisei a zis: "Duhul care este în tine să fie îndoit în mine!"
\par 10 Răspuns-a Ilie: "Greu lucru ceri! Dar de mă vei vedea când voi fi luat de la tine, va fi așa; iar de nu mă vei vedea, nu va fi".
\par 11 Pe când mergeau ei așa pe drum și grăiau, deodată s-a ivit un car și cai de foc și, despărțindu-i pe unul de altul, a ridicat pe Ilie în vârtej de vânt la cer.
\par 12 Iar Elisei se uita și striga: "Părinte, părinte, carul lui Israel și caii lui!" Și apoi nu l-a mai văzut. Și apucându-și hainele le-a sfâșiat în două.
\par 13 Apoi, apucând mantia lui Ilie, care căzuse de la acesta, s-a întors înapoi și s-a oprit pe malul Iordanului.
\par 14 Și a luat mantia lui Ilie care căzuse de la acesta și a lovit apa cu ea, zicând: "Unde este Domnul Dumnezeul lui Ilie?" Și lovind, apa s-a tras la dreapta și la stânga și a trecut Elisei.
\par 15 Iar fiii proorocilor cei din Ierihon, văzându-l de departe, au zis: "Duhul lui Ilie s-a odihnit peste Elisei! Și au ieși t înaintea lui și i s-au plecat până la pământ, zicându-i:
\par 16 "Iată, la noi, robii tăi, se află cincizeci de oameni voinici; să se ducă să caute pe stăpânul tău, poate l-a dus Duhul Domnului și l-a aruncat pe vreun munte sau într-o vale". Iar el a zis: "Să nu-i trimiteți!"
\par 17 Ei însă au stăruit mult pe lângă el și el, văzând că nu poate scăpa de ei, le-a zis: "Trimiteți-i!" Și au trimis ei cincizeci de oameni și au căutat trei zile și nu l-au găsit;
\par 18 Întorcându-se apoi aceia la el în Ierihon, unde rămăsese în vremea aceea, a zis către ei Elisei: "Nu v-am spus eu să nu vă duceți?"
\par 19 Iar locuitorii cetății aceleia au zis către Elisei: "Iată așezarea cetății acesteia este bună, după cum poate vedea și stăpânul nostru, dar apa nu este bună și pământul este neroditor".
\par 20 Și el a zis: "Aduceți-mi o oală nouă și puneți sare în ea!"
\par 21 Și i-au adus și a ieșit el la fântâna de apă și, aruncând sarea în ea, a zis: "Așa zice Domnul: Iată am făcut apa aceasta sănătoasă și nu va mai pricinui nici vătămare, nici moarte, nici nerodire".
\par 22 Și s-a făcut apa curată până astăzi, după cuvântul pe care l-a spus Elisei.
\par 23 De acolo s-a dus el la Betel. Și cum mergea pe drum, au ieșit din cetate niște copii și s-au apucat să râdă de al zicând: "Hai, pleșuvule, hai!"
\par 24 Iar el, întorcându-se și văzându-i, i-a blestemat cu numele Domnului. Atunci, ieșind din pădure doi urși, au sfâșiat din ei patruzeci și doi de copii.
\par 25 De aici Elisei s-a dus la muntele Carmelului, iar de acolo s-a întors în Samaria.

\chapter{3}

\par 1 Ioram, fiul lui Ahab, se făcuse rege peste Israel în Samaria, în anul al optsprezecelea al lui Iosafat, regele Iudei, și a domnit doisprezece ani.
\par 2 Acesta a făcut lucruri netrebnice în ochii Domnului, dar nu așa cum făcuse tatăl său și mama sa, căci el a depărtat stâlpii cu pisanii făcuți în cinstea lui Baal, pe care-î făcuse tatăl său.
\par 3 Dar de păcatele lui Ieroboam, fiul lui Nabat, care a dus pe Israel în rătăcire, s-a ținut și el și nu s-a lăsat de ele.
\par 4 Meșa, regele Moabului, era bogat în vite și trimitea regelui lui Israel câte o sută de mii de miei și câte o sută de mii de berbeci netunși.
\par 5 Dar când a murit Ahab, regele Moabului s-a răzvrătit împotriva regelui lui Israel.
\par 6 În vremea aceea a ieșit regele Ioram din Samaria și a numărat tot Israelul;
\par 7 Iar după aceea s-a dus la Iosafat, regele Iudei, să-i zică: "Regele Moabului s-a răzvrătit asupra mea. Vrei să mergi cu mine la război împotriva Moabului?" Iar acesta a zis: "Merg. Cum ești tu, așa sunt și eu; cum este poporul tău, așa este și al meu și cum sunt caii tăi, așa sunt și ai mei!"
\par 8 Apoi a zis: "Pe ce drum să mergem?" Iar el a răspuns: "Pe calea pustiului Edomului".
\par 9 Și a plecat regele lui Israel și regele Iudei și regele Edomului și au înconjurat cale de șapte zile; dar nu era apă pentru oștire și pentru vitele ce veneau în urmă.
\par 10 Atunci a zis regele lui Israel: "Ah, iată a chemat Domnul pe acești trei regi ca să-i dea în mâinile lui Moab".
\par 11 Iar Iosafat a zis: "Nu este oare pe aici vreun prooroc al Domnului, ca să întrebăm pe Domnul prin el?" Și auzind, unul din slujitorii regelui lui Israel a zis: "Este aici Elisei, fiul lui Safat, care turna apă pe mâini lui Ilie".
\par 12 A zis Iosafat: "El are cuvântul Domnului". Și s-au dus la el regele lui Israel și regele Iudei și regele Edomului.
\par 13 Atunci a zis Elisei către regele lui Israel: "Ce poate fi între mine și tine? Du-te la proorocii tatălui tău și la proorocii mamei tale!" Iar regele lui Israel a zis către el: "Ba nu, căci Domnul a chemat aici pe acești trei regi ca să-i dea în mâinile lui Moab".
\par 14 Iar Elisei a zis: "Pe cât este de adevărat că Domnul Savaot, Căruia slujesc, este viu, tot așa este de adevărat că de nu aș cinsti pe Iosafat, regele Iudei, nici nu m-aș uita la tine și nici nu te-aș vedea!
\par 15 Acum însă chemați-mi un cântăreț!" și dacă a început acesta a cânta, s-a atins mâna Domnului de Elisei
\par 16 Și acesta a zis: "Așa grăiește Domnul: Faceți în valea aceasta șanțuri.
\par 17 Căci așa zice Domnul: Nu veți vedea vânt, nici ploaie nu veți vedea, dar valea aceasta se va umplea de apă, din care veți bea și voi și vitele voastre cele mici și cele mari.
\par 18 Însă acesta este puțin lucru în ochii Domnului. El și pe Moab îl va da în mâinile voastre;
\par 19 Și veți bate toate cetățile cele întărite și toate cetățile însemnate, toți copacii cei mai buni îi veți tăia și toate izvoarele de apă le veți astupa și toate ogoarele cele mai bune le veți strica cu pietre".
\par 20 Dimineața însă, când se înălța darul de pâine, deodată s-a revărsat apă pe drumul dinspre Edom și s-a umplut pământul de apă.
\par 21 Și când au auzit Moabiții că vin regii să se bată cu ei, s-au adunat toți care erau în stare să poarte arme, ba și cei mai bătrâni și au stat da hotar.
\par 22 Iar dimineața s-au sculat de noapte și, când a strălucit soarele deasupra apei, Moabiților li s-a părut din depărtare că apa aceea este roșie ca sângele.
\par 23 Și au zis: "Acela este sânge. Regii aceia s-au luptat între ei rănindu-se unul pe altul. Acum la pradă, Moabe!"
\par 24 Și au venit ei spre tabăra israelită. Și s-au sculat Israeliții și au început a bate pe Moabiți și aceștia au fugit de ai, iar ei i-au urmărit mereu și au bătut pe Moabiți.
\par 25 Cetățile lor le-au dărâmat și în toate ogoarele cele mai bune au aruncat fiecare cu pietre și le-au umplut cu pietre; toate izvoarele de apă le-au astupat și toți copacii cei mai buni i-au tăiat; apoi prăștiașii au înconjurat Chir-Hareșetul și l-au luat și n-au lăsat decât numai pietrele.
\par 26 Atunci, văzând regele Moabului că este biruit în război, a luat cu sine șapte sute de oameni, deprinși la mânuirea sabiei, ca să pătrundă la regele Edomului, dar n-a putut.
\par 27 Deci a luat pe fiul său cel întâi născut, care trebuia să domnească în locul lui, și l-a adus ardere de tot pe zid. Aceasta a pricinuit o mare mânie asupra Israeliților și s-au dus de la el, întorcându-se în țara lor.

\chapter{4}

\par 1 În vremea aceea o femeie a unuia din fiii proorocilor striga către Elisei, zicând: "Robul tău, bărbatul meu a murit și tu știi că robul tău era om cu temere de Domnul. Acum însă iată că au venit datornicii să ia. robi pe amândoi fiii mei!"
\par 2 Elisei însă i-a zis: "Ce să-ți fac? Spune-mi ce ai tu în casă?" Iar ea a răspuns: "Roaba ta n-are în casă nimic, afară de un vas cu untdelemn".
\par 3 A zis Elisei: "Mergi și împrumută vase din altă parte, pe la toți vecinii tăi.
\par 4 Ia vase goale cit de multe și apoi intră și-ți încuie ușa după tine și după fiii tăi; apoi toarnă untdelemn în toate vasele acelea și pe cele pline dă-le la o parte".
\par 5 Și ducându-se ea de la dânsul a încuiat ușa după sine și după fiii săi; și aceștia îi dădeau vasele și ea le umplea.
\par 6 Iar dacă s-au umplut vasele, a zis ea către fiul său: "Mai dă-mi un vas". El însă a zis: "Nu mai sunt vase". Și untdelemnul a încetat să curgă.
\par 7 Și venind, ea a spus omului lui Dumnezeu, iar acesta a zis: "Du-te și vinde untdelemnul și-ți plătește datoriile tale, iar cu ceea ce va rămâne, cu aceea vei trăi tu cu fiii tăi".
\par 8 Într-una din zile a venit Elisei la Șunem și acolo o femeie bogată l-a poftit la masă și după aceea, ori de câte ori trecea pe acolo, totdeauna se abătea să mănânce.
\par 9 Și a zis aceasta către bărbatul său: "Eu știu că omul lui Dumnezeu care trece mereu pe aici este sfânt;
\par 10 Să-i facem dar un mic foișor sus și să-i punem acolo un pat, o masă, un scaun și un sfeșnic și, când va veni pe la noi, să se ducă acolo".
\par 11 Venind deci Elisei într-o zi acolo și intrând în foișor, s-a culcat acolo.
\par 12 Și a zis către Ghehazi, sluga sa: "Cheamă pe Șunamiteanca aceasta!" Și a chemat-o și ea a stat înaintea lui.
\par 13 Apoi a zis lui Ghehazi: "Zi-i: Iată tu te îngrijești atâta de noi. Ce să-ți facem? Nu cumva ai nevoie să vorbim pentru tine cu regele sau cu căpetenia oștirii?" Iar ea a zis: "Nu, căci trăiesc în pace în mijlocul poporului meu".
\par 14 Zis-a iarăși Elisei către Ghehazi: "Atunci ce să-i fac?" Iar Ghehazi a răspuns: "Iată, n-are nici un copil și bărbatul ei este bătrân".
\par 15 Și a zis Elisei: "Cheam-o încoace!" Și a chemat-o și ea a stat în ușă.
\par 16 Iar Elisei a zis: "La anul pe vremea asta vei ține în brațe un fiu". Ea a răspuns: "Nu, omule al lui Dumnezeu și stăpânul meu, nu amăgi pe roaba ta!"
\par 17 Dar femeia aceea a zămislit și în anul următor, chiar pe vremea aceea, a născut un fiu, după cum îi spusese Elisei.
\par 18 Crescând acel copii, s-a dus întruna din zile la tatăl său, la secerători.
\par 19 Și a zis către tatăl său: "Vai, mă doare capul!" Iar acesta a zis către o slugă: "Du-l la mama lui!"
\par 20 Și l-a luat și l-a dus la mama lui. Și a stat pe brațele ei până la amiază și a murit.
\par 21 Atunci ea s-a dus și l-a pus în patul omului lui Dumnezeu și l-a încuiat acolo și a ieșit.
\par 22 Apoi a chemat pe bărbatul său și a zis: "Trimite-mi o slugă și o asină, căci mă duc până la omul lui Dumnezeu și mă întorc îndată".
\par 23 Bărbatul a zis: "De ce să te duci la el? Astăzi nu este nici lună nouă, nici zi de odihnă". Ea a zis: "Fii pe pace!"
\par 24 Și punând șaua pe asină, a zis către sluga sa: "Ia-o și pornește, dar să nu mă oprești din mers până nu-ți voi spune eu!"
\par 25 Și pornind, s-a dus la omul lui Dumnezeu, în muntele Carmelului. Și când a văzut-o omul lui Dumnezeu din depărtare, a zis către sluga sa Ghehazi: "Asta este șunamiteanca aceea!
\par 26 Aleargă întru întâmpinarea ei și întreab-o: Ești sănătoasă? Și bărbatul tău e sănătos? Și copilul tău e sănătos?"
\par 27 Și ea a răspuns: "Sunt sănătoși!" Iar dacă a ajuns pe munte la omul lui Dumnezeu, s-a apucat de picioarele lui. Atunci Ghehazi a venit să o dea la o parte; dar omul lui Dumnezeu i-a zis: "Las-o, căci este cu sufletul amărât și Domnul a ascuns aceasta de mine și nu mi-a arătat".
\par 28 Iar ea a zis: "Au cerut-am eu fiu de la domnul meu? N-am zis eu oare, nu amăgi pe roaba ta?"
\par 29 Atunci Elisei a zis către Ghehazi: "Încinge-ți mijlocul tău, ia toiagul meu în mâna ta și du-te; de vei întâlni pe cineva, să nu-i dai bună ziua, să nu-i răspunzi, și să pui toiagul meu pe fața copilului!"
\par 30 Iar mama copilului a zis: "Pe cât este de adevărat că Domnul este viu, cum este viu și sufletul tău, tot așa este de adevărat că nu te voi lăsa!"
\par 31 Atunci el s-a sculat și s-a dus după dânsa. Ghehazi însă s-a dus înaintea lor și a pus toiagul pe fața copilului; dar n-a fost nici glas, nici răspuns. Și a ieșit Ghehazi întru întâmpinarea lui Elisei și i-a spus, zicând: "Copilul nu se trezește!"
\par 32 Intrând Elisei în casă, a văzut copilul mort, întins în patul său.
\par 33 Și după ce a intrat, a încuiat ușa după sine și s-a rugat Domnului.
\par 34 Apoi s-a ridicat și s-a culcat peste copil și și-a pus buzele sale pe buzele lui și ochii săi pe ochii lui și palmele sale pe palmele lui și s-a întins pe el și a încălzit trupul copilului.
\par 35 Sculându-se apoi, Elisei s-a plimbat prin foișor înainte și înapoi. După aceea s-a dus și s-a întins iar peste copil. și a strănutat copilul de șapte ori și și-a deschis copilul ochii.
\par 36 Atunci a chemat pe Ghehazi și i-a zis: "Cheamă pe șunamiteanca aceea!" Și acela a chemat-o și, venind ea, i-a zis: "Ia-ți copilul!"
\par 37 Iar ea apropiindu-se, a căzut la picioarele lui și s-a închinat până la pământ. Apoi și-a luat copilul și a ieșit.
\par 38 Iar Elisei s-a întors la Ghilgal și era foamete în pământul acela și fiii proorocilor ședeau înaintea lui. Și a zis Elisei slugii sale: "Pune căldarea cea mare și fierbe ceva pentru fiii proorocilor".
\par 39 Iar unul dintr-înșii, ieșind la câmp să adune verdețuri, a găsit o buruiană sălbatică, agățătoare, a cules din ea roade o poală plină și, venind, le-a aruncat în căldarea cu fiertură, fără să le cunoască.
\par 40 Apoi le-a dat să mănânce. Dar îndată ce au început a mânca, au strigat: "Omul lui Dumnezeu, în căldare este moarte!" și n-au mai putut să mănânce.
\par 41 Iar el a zis: "Aduceți făină!" Și a presărat-o în căldare și a zis către Ghehazi: "Dă oamenilor să mănânce!" și n-a mai rămas în căldare nimic vătămător.
\par 42 Tot pe atunci a venit un om din Baal-Șalișa și a adus omului lui Dumnezeu pârgă de pâine douăzeci de pâini de orz și grăunțe proaspete de grâu într-un săcușor. Elisei a zis: "Dați oamenilor să mănânce!"
\par 43 Iar sluga sa a zis: "Ce să dau eu de aici la o sută de oameni?" Zis-u Elisei: "Dă oamenilor să mănânce; căci așa zice Domnul: Se vor sătura și va mai și rămâne".
\par 44 Și a dat și s-au săturat și a mai rămas, după cuvântul Domnului.

\chapter{5}

\par 1 Neeman, căpetenia oștirii regelui Siriei, era om de seamă și cu trecere înaintea stăpânului său, pentru că prin Domnul dăduse biruință Sirienilor. Dar acest ostaș vrednic ara lepros.
\par 2 O dată Sirienii au ieșit în cetate și între altele luaseră din pământul lui Israel o copilă, care acum slujea la femeia lui Neeman.
\par 3 Aceasta a zis către stăpâna sa: "O, dacă stăpânul meu s-ar duce la proorocul cel din Samaria, de bună seamă s-ar tămădui de lepră".
\par 4 Atunci s-a dus Neeman și a spus aceasta stăpânului său: "Așa și așa zice fetița cea din pământul lui Israel".
\par 5 Iar regele Siriei a zis către Neeman: "Scoală și du-te și voi trimite și eu scrisoare regelui lui Israel". Și s-a dus Neeman, luând cu sine zece talanți de argint, șase mii de sicli de aur și zece rânduri de haine;
\par 6 Și a dus regelui lui Israel scrisoarea în care zicea: "Împreună cu scrisoarea aceasta, trimit pe Neeman, sluga mea, ca să cureți lepra de pe el".
\par 7 Când a citit regele lui Israel scrisoarea, și-a rupt hainele sale și a zis: "Au doară eu sunt Dumnezeu, ca să omor și să fac viu, de trimite el la mine, ea să vindec pe acest om de lepră? Iată acum să vedeți și să știți că el caută pricină de dușmănie împotriva mea".
\par 8 Când însă a auzit Elisei, omul lui Dumnezeu, că regele lui Israel și-a rupt hainele sale, a trimis să i se spună regelui: "Pentru ce ji-ai rupt tu hainele tale? Lasă-l să vină la mine și vor cunoaște că este prooroc în Israel".
\par 9 Și a venit Neeman cu caii și cu căruța sa, oprindu-se la poarta casei lui Elisei.
\par 10 Iar Elisei a trimis la el pe sluga sa să-i zică: "Du-te și te scaldă de șapte ori în Iordan, că ți se va înnoi trupul tău și vei fi curat!"
\par 11 Neeman însă s-a mâniat și a plecat, zicând: "Iată, socoteam că va ieși el și, stând la rugăciune, va chema numele Domnului Dumnezeului său, își va pune mâna pe locul bolnav și va curăți lepra.
\par 12 Au doară Abana și Farfar, râurile Damascului, nu sunt ele mai bune decât toate apele lui Israel? Nu puteam eu oare să mă scald în ele și să mă curăț? Și așa s-a întors și a plecat mânios.
\par 13 Dar slugile lui, apropiindu-se, i-au grăit și i-au zis: "Stăpâne, dacă proorocul ți-ar fi zis să faci ceva însemnat, oare n-ai fi făcut? Cu atât mai vârtos trebuie să faci când ți-a zis numai: Spală-te și vei fi curat!"
\par 14 Atunci el s-a coborât și s-a cufundat de șapte ori în Iordan, după cuvântul omului lui Dumnezeu, și i s-a înnoit trupul ca trupul unui copil mic și s-a curățit.
\par 15 Atunci s-a întors la omul lui Dumnezeu cu toți cei ce-l însoțeau și, venind, a stat înaintea lui și a zis: "Iată am cunoscut că în tot pământul nu este Dumnezeu decât numai în Israel! Deci, ia un dar de la robul tău!"
\par 16 Iar Elisei a zis: "Pe cât este de adevărat că Domnul, înaintea Căruia slujesc, este viu, tot atât este de adevărat că nu voi primi".
\par 17 Acela însă îl silea să primească, dar el n-a vrut. Atunci a zis Neeman: "Dacă nu, atunci să sa dea robului tău pământ cât pot duce doi catâri, pentru că de aici înainte robul tău nu va mai aduce arderi de tot și jertfe la alți dumnezei afară de Domnul.
\par 18 Numai iată ce să ierte Domnul robului tău: Când va merge stăpânul meu în templul lui Rimon, ca să se închine acolo, și se va sprijini de mâna mea, și mă voi închina și eu în templul lui Rimon, atunci, pentru închinarea mea în templul lui Rimon, să ierte Domnul pe robul tău la asemenea întâmplare".
\par 19 Elisei l-a răspuns: "Mergi în pace!" Și el s-a dus.
\par 20 Iar dacă s-a depărtat puțin, Ghehazi, sluga lui Elisei, omul lui Dumnezeu, și-a zis: "Iată, stăpânul meu n-a vrut să ia din mâna acestui Neeman Sirianul ceea ce i-a adus. Pe cât de adevărat că Domnul este viu, așa este de adevărat că voi alerga după el și voi lua de la el ceva".
\par 21 Și a alergat Ghehazi după Neeman, iar Neeman, văzându-l alergând după el, s-a coborât din căruță înaintea lui și a zis: "Cu pace vii?"
\par 22 Și Ghehazi a răspuns: "Cu pace. M-a trimis stăpânul meu să-îi spun: Iată au venit acum la mine din muntele lui Efraim doi tineri din fiii proorocilor; dă-le un talant de argint și două rânduri de haine".
\par 23 Și a zis Neeman: "Ia, rogu-te, doi talanți de argint". Și l-a rugat, și s legat doi talanți de argint în doi saci și două rânduri de haine și le-a dat la două slugi care le-a dus înaintea lui.
\par 24 Iar dacă au ajuns sub deal, le-a luat din mâinile lor și le-a ascuns în casă. Apoi a dat drumul oamenilor de s-au dus.
\par 25 Când însă a venit și s-a arătat stăpânului său, Elisei i-a zis: "De unde vii Ghehazi?" Și El a răspuns: "Robul tău n-a fost nicăieri".
\par 26 Iar Elisei i-a zis: "Au doar inima mea nu te-a întovărășit când omul acela s-a dat jos din căruță și a venit în întâmpinarea ta? Este timpul oare să iau argint și haine, măslini și vii, vite mari sau mărunte, robi sau roabe?
\par 27 Să se lipească dar lepra lui Neeman de tine și de urmașii tăi în veci". Și a ieșit Ghehazi de la Elisei alb de lepră ca zăpada.

\chapter{6}

\par 1 Atunci au zis fiii proorocilor către Elisei: "Iată, locul unde trăim aici la tine e strâmt pentru noi.
\par 2 Să mergem dar la Iordan și să luăm de acolo fiecare câte o bârnă și să ne facem locuință acolo".
\par 3 Și el a zis: "Duceți-vă!" Iar unul a zis: "Fă milă și mergi și tu cu robii tăi!" Și el a zis: "Merg!" Și s-a dus cu ei.
\par 4 Și ajungând la Iordan, s-au apucat de tăiat copaci.
\par 5 Și când a prăvălit unul o bârnă, i-a căzut toporul în apă și a strigat acela și a zis: "Ah, stăpânul meu! Acesta îl luasem împrumut!"
\par 6 A zis omul lui Dumnezeu: "Unde a căzut?" Și acela i-a arătat locul. Iar Elisei a tăiat o bucată de lemn și, aruncând-o acolo, a ieșit toporul deasupra. Apoi a zis: "Ia-ți-l!"
\par 7 Și acela a întins mâna și l-a luat.
\par 8 În vremea aceea s-a ridicat regele Siriei cu război împotriva Israeliților și s-a sfătuit cu slujitorii săi zicând: "Am să așez tabăra în cutare sau în cutare loc".
\par 9 Și a trimis omul lui Dumnezeu la regele lui Israel să i se spună: "Păzește-te de a trece prin locul acela; căci acolo s-au ascuns Sirienii".
\par 10 Și a trimis regele lui Israel la locul acela de care-i grăise omul lui Dumnezeu și-i spusese să se ferească și s-a păzit de el mereu.
\par 11 Și s-a neliniștit inima regelui Siriei de întâmplarea aceasta și, chemând pe slujitorii săi, a zis: "Spuneți-mi care din ai voștri este în legătură cu regele lui Israel?"
\par 12 Și răspunzând unul din slujitori, a zis: "Nimeni, stăpânul meu rege. Dar Elisei proorocul, pe care-l are Israel, spune regelui lui Israel până și cuvintele ce le grăiești tu în odaia ta de culcare".
\par 13 A zis regele: "Duceți-vă și aflați unde este, căci am să trimit să-l iau". Și i s-a spus: "Iată, este în Dotan".
\par 14 Și a trimis acolo cai și care și mulțime de oștire, care au venit noaptea și au împresurat cetatea.
\par 15 Iar dimineața slujitorul omului lui Dumnezeu, sculându-se și ieșind a văzut oștirea împrejurul cetății, Și caii și carele; și a zis slujitorul său către el: "Vai, stăpânul meu! Ce să facem?"
\par 16 El însă i-a zis: "Nu te teme, pentru că cei ce sunt cu noi sunt mai numeroși decât cei ce sunt cu ei".
\par 17 Și s-a rugat Elisei și a zis: "Doamne, deschide-i ochii ca să vadă!" Și a deschis Domnul ochii slujitorului și acesta a văzut că tot muntele era plin de cai și care de foc împrejurul lui Elisei.
\par 18 Iar când au venit asupra lui Sirienii, Elisei s-a rugat Domnului și a zis: "Lovește-i cu orbire!" Și Domnul i-a orbit, după cuvântul lui Elisei.
\par 19 Apoi Elisei le-a zis: "Nu este acesta drumul și nici cetatea nu este aceasta. Dar veniți după mine și eu vă voi duce la omul acela pe care-l căutați!" Și i-a dus în Samaria.
\par 20 Iar la intrarea lor în Samaria, Elisei a zis: "Doamne, deschide-ne ochii ca să vadă!" Și le-a deschis Domnul ochii și au văzut că sunt în Samaria.
\par 21 Și văzându-i, regele lui Israel a zis către Elisei: "Să-i ucid, părinte?"
\par 22 Iar Elisei a zis: "Să nu-i ucizi. Au doară cu arcul tău și cu sabia ta i-ai prins ca să-i ucizi? Dă-le pâine și apă, să mănânce și să bea și apoi să se ducă la domnul lor".
\par 23 Și le-a făcut masă mare și ei au mâncat și au băut. După aceea le-a dat drumul și s-au dus la stăpânul lor. Și n-au mai venit taberele siriene în țara lui Israel.
\par 24 După aceea a adunat Benhadad, regele Siriei, toată oștirea sa și a venit și a împresurat Samaria.
\par 25 Și a fost în Samaria foamete mare în timpul împresurării lor, încât un cap de asin se vindea cu optzeci de sicli de argint și un sfert de căpățână de ceapă sălbatică se vindea cu cinci sicli de argint.
\par 26 Și trecând odată regele lui Israel pe zid, o femeie, plângând, i-a zis: "Ajută-mă, domnul meu rege!"
\par 27 Și regele i-a zis: "De nu te va ajuta Domnul, cu ce te pot ajuta eu? Au doară eu vin de la arie sau de la teascuri?" Apoi i-a zis regele: "Ce ai?" Iar ea a răspuns:
\par 28 "Iată, femeia aceasta mi-a zis: Dă pe fiul tău să-l mâncăm astăzi, iar pe fiul meu îl vom mânca mâine.
\par 29 Și așa am fiert noi pe fiul meu și l-am mâncat; și a doua zi am zis către ea: Dă acum pe fiul tău să-l mâncăm. Ea însă a ascuns pe fiul său".
\par 30 Și auzind regele cuvintele femeii, și-a rupt hainele și apoi a trecut înainte pe zidul cetății, iar poporul a văzut că pe dinăuntru trupul său era îmbrăcat în haină de jale.
\par 31 Apoi a zis regele: "Să mă pedepsească Dumnezeu cu toată asprimea dacă va rămâne azi capul lui Elisei, fiul lui Safat, pe umeri".
\par 32 Elisei însă ședea în casa sa și bătrânii ședeau împreună cu el. Și a trimis regele un om al său; dar înainte de a veni cel trimis la el, el a zis către bătrâni: "Știți voi oare că acest fiu de ucigaș a trimis să mi se taie capul? Băgați de seamă; când va veni trimisul lui, să încuiați ușa și să-l opriți la ușă. Dar iată și zgomotul pașilor stăpânului său se aude în urma lui!"
\par 33 Și încă grăind el cu ei, a venit la el trimisul și a zis: "Iată, ce necaz a venit de la Domnul! Ce să mai așteptăm acum de la Domnul?"

\chapter{7}

\par 1 Iar Elisei a zis: "Ascultați cuvântul Domnului! Așa zice Domnul: Mâine, pe vremea aceasta, în poarta Samariei o măsură de făină din cea mai bună va fi un siclu și două măsuri de orz tot un siclu".
\par 2 Slujbașul de al cărui braț se sprijinea regele a răspuns: "Chiar dacă Domnul ar deschide ferestrele cerului nici atunci n-ar putea fi una ca aceasta!" Iar Elisei a zis: "Iată așa ai să vezi cu ochii tăi, dar tu nu vei mânca din acelea!"
\par 3 În vremea aceasta la poarta Samariei se aflau patru oameni leproși și ziceau unul către altul: "Ce ședem noi aici și așteptăm moartea?
\par 4 De ne vom hotărî să ne ducem în cetate, în cetate e foamete și vom muri acolo; iar dacă vom ședea aici, tot vom muri. Să ne ducem mai bine în tabăra Sirienilor! De ne vor lăsa cu viață, vom trăi, de nu, vom muri".
\par 5 Și s-au sculat ei în amurg, ca să se ducă în tabăra Sirienilor. Dar când au ajuns la marginea taberei Sirienilor, acolo nu mai era nici un om;
\par 6 Căci Domnul făcuse în tabăra Sirienilor să se audă zgomot de care și nechezat de cai și zgomot de oștire mare. Și au zis ei unul către altul: "De bună seamă, regele lui. Israel a tocmit să vină împotriva noastră pe regii Heteilor și ai Egiptului".
\par 7 Și s-au sculat și au fugit în amurg și și-au lăsat corturile, caii, asinii lor și toată tabăra cum era și au fugit, ca să-și scape viața.
\par 8 Ajungând deci leproșii aceia la marginea taberei, au intrat într-un cort și au mâncat și au băut și au luat de acolo argint și aur și haine și s-au dus de le-au ascuns. Și s-au mai dus și în alt cort și au luat și de acolo și au ascuns.
\par 9 Apoi au zis unul către altul: "Ceea ce facem, nu facem bine. Ziua aceasta este zi de veste bună. Dacă întârziem și așteptăm lumina zilei atunci vina va cădea asupra noastră. Hai deci să vestim casa regelui!"
\par 10 Și au venit ei și au strigat pe portarii cetății și le-au povestit, zicând: "Noi am fost în tabăra Sirienilor și iată acolo nu se vedea nimeni și nu se auzea nimic, ci numai cai legați și corturi cum trebuie să fie".
\par 11 Și portarii au strigat și au dat de veste la casa regelui.
\par 12 Și s-a sculat regele noaptea și a zis slugilor sale: "Am să vă spun ce fac Sirie cu noi: Ei știu că noi suferim de foame și au ieșit din tabără și s-au ascuns în câmp, cugetând așa: Când vor ieși ei din cetate, îi vom prinde vii și vom năvăli în cetate".
\par 13 Dar unul din cei ce slujeau înaintea lui a răspuns și a zis: "Să se ia cei cinci cai rămași care mai sunt în cetate (din toată tabăra lui Israel numai atâta mai rămăsese; cealaltă tabără a lui Israel pierise toată) și să trimitem oameni să vadă".
\par 14 Au luat deci două perechi de cai înhămați la căruțe și a trimis regele pe urma oștirii siriene, zicând: "Duceți-vă și vedeți!"
\par 15 Și s-au dus după ei până la Iordan și iată tot drumul era semănat cu haine și cu lucruri aruncate de Sirieni în graba lor. și s-au întors trimișii și au spus regelui.
\par 16 Atunci a ieșit poporul și a prădat tabăra siriană și a fost măsura de făină din cea mai bună un siclu și două măsuri de orz un siclu, după cuvântul Domnului.
\par 17 Iar regele a luat pe slujitorul acela de mâna căruia se sprijinea și l-a rânduit de pază la poarta cetății; acesta a fost călcat în picioare de popor și a murit la poartă, cum zisese omul lui Dumnezeu când venise trimisul regelui la el;
\par 18 Pentru că atunci când omul lui Dumnezeu a spus regelui așa: "Mâine pe vremea aceasta în poarta Samariei două măsuri de orz vor fi un siclu și o măsură de făină din cea mai bună va fi tot un siclu",
\par 19 Atunci acest slujitor a răspuns omului lui Dumnezeu și a zis: "Chiar dacă Domnul va deschide ferestrele cerului, nici atunci n-ar putea fi una ca aceasta!" Iar Elisei i-a zis: "Vei vedea-o cu ochii tăi, dar nu vei mânca din ea!"
\par 20 Și așa s-a întâmplat cu el: l-a călcat poporul în poartă și a murit.

\chapter{8}

\par 1 În vremea aceea a grăit Elisei cu femeia căreia îi înviase copilul și i-a zis: "Scoală și du-te de la casa ta și trăiește unde vei putea, căci Domnul a chemat foametea și aceea va veni asupra pământului pentru șapte ani".
\par 2 Și s-a sculat femeia aceea și a făcut după cuvântul omului lui Dumnezeu și s-a dus ea și casa sa și a trăit în pământul Filistenilor șapte ani.
\par 3 Iar după trecerea celor șapte ani s-a întors femeia aceea din pământul Filistenilor și a venit să roage pe rege pentru casa sa și pentru țarina sa.
\par 4 Tocmai atunci regele vorbea cu Ghehazi, sluga omului lui Dumnezeu, și a zis: "Povestește-mi tot ce este mai însemnat din câte a făcut Elisei!"
\par 5 Și pe când istorisea el regelui despre copilul înviat de Elisei, a rugat pe rege pentru casa sa și pentru țarina sa. Și a zis Ghehazi: "Stăpânul meu rege, aceasta este chiar femeia aceea și el este acel fiul al ei pe care l-a înviat Elisei".
\par 6 Și a întrebat regele și pe femeie și i-a povestit și ea. Atunci regele i-a dat pe unul de la curte, zicând: "Să i se întoarcă toate câte sunt ale ei și toate veniturile țarinei din ziua când ea a părăsit țarina până acum".
\par 7 Și a venit Elisei în Damasc, când Benhadad, regele Siriei, era bolnav. Și i s-a spus acestuia și i s-a zis: "A venit aici omul lui Dumnezeu".
\par 8 Iar regele a zis către Hazael: "Ia în mâna ta un dar și du-te în întâmpinarea omului lui Dumnezeu și întreabă pe Domnul prin el, zicând: Mă voi însănătoși eu, oare, de boala aceasta?"
\par 9 Și s-a dus Hazael în întâmpinarea lui și a luat dar în mâna sa din cele mai bune lucruri din Damasc cât pot duce patruzeci de cămile și a venit și a stat înaintea feței lui și a zis: "Benhadad, fiul tău, regele Siriei, m-a trimis la tine să întreb: Mă voi însănătoși eu, oare, de. boala aceasta?"
\par 10 Iar Elisei i-a zis: "Du-te și spune-i: "Te vei însănătoși!" Cu toate acestea, Domnul mi-a descoperit că el va muri".
\par 11 Și și-a îndreptat Elisei privirile spre Hazael și l-a privit mult, apoi a plâns omul lui Dumnezeu.
\par 12 Și a zis Hazael. "De ce plânge domnul meu?" Și el a zis: "Pentru că știu ce rău ai să faci tu fiilor lui Israel; tăria lor o vei da focului, pe tinerii lor cu sabia îi vei ucide, pe copiii lor de sân îi vei omorî și pe cele însărcinate ale lor le vei tăia".
\par 13 A zis Hazael: "Robul tău este câine ca să facă asemenea lucru necugetat?" Și Elisei a zis: "Domnul mi-a arătat în tine pe regele Siriei".
\par 14 Și s-a dus Hazael de la Elisei și a venit la domnul său și acesta i-a zis: "Ce ți-a grăit Elisei?" Iar el a zis: "Mi-a grăit că ai să te faci sănătos".
\par 15 A doua zi însă a luat o învelitoare, a udat-o cu apă, a pus-o pe fața lui și a murit regele. Și în locul lui s-a făcut rege Hazael.
\par 16 În anul al cincilea al lui Ioram, fiul lui Ahab, regele lui Israel, în locul lui Iosafat, regele Iudei, s-a făcut rege Ioram, fiul lui Iosafat, regele Iudei.
\par 17 Acesta era de treizeci și doi de ani când s-a făcut rege.
\par 18 El a domnit în Ierusalim opt ani și a pășit pe urma regilor lui Israel, cum se purtase casa lui Ahab, pentru că fata lui Ahab era soția lui Iosafat și a făcut lucruri rele în ochii Domnului.
\par 19 Cu toate acestea Domnul n-a vrut să piardă pe Iuda pentru David, robul Său, deoarece îi făgăduise că îi va da totdeauna o făclie printre fiii săi.
\par 20 În zilele lui Ioram a ieșit Edom de sub mâna lui Iuda și ei și-au pus rege.
\par 21 Și s-a dus Ioram la Țair și împreună cu el s-au dus toate carele lui. Apoi s-a sculat el noaptea și a lovit pe Edomiții care-l împresuraseră, și pe căpeteniile carelor, dar poporul a fugit în corturile sale.
\par 22 Edom a ieșit de sub mâna lui Iuda și a rămas așa până în ziua de astăzi.
\par 23 Și tot atunci a ieșit și Libna. Celelalte fapte pe care le-a făcut Ioram, sunt scrise în cartea faptelor regilor lui Iuda.
\par 24 Și a răposat Ioram cu părinții săi și a fost îngropat cu părinții săi în cetatea lui David, iar în locul lui s-a făcut rege Ohozia, fiul său.
\par 25 În anul al doisprezecelea al lui Ioram, fiul lui Ahab, regele lui Israel, s-a făcut rege în Iuda Ohozia, fiul lui Ioram, regele Iudei.
\par 26 Ohozia era de douăzeci și doi de ani când s-a făcut rege și a domnit un an în Ierusalim. Numele mamei lui era Atalia, fiica lui Omri, regele lui Israel.
\par 27 Ohozia a pășit pe urmele casei lui Ahab și a făcut rele în ochii Domnului, ca și casa lui Ahab, pentru că era înrudit cu casa lui Ahab.
\par 28 Și s-a dus el cu Ioram, fiul lui Ahab, la război împotriva lui Hazael, regele Siriei, la Ramot în Galaad. Atunci au rănit Sirienii pe Ioram.
\par 29 Și regele Ioram s-a întors ca să se vindece în Izreel de rana ce i-o pricinuiseră Sirienii la Ramot când se luptase cu Hazael, regele Siriei. Iar Ohozia, fiul lui Ioram, regele Iudei, a venit să-l vadă pe Ioram, fiul lui Ahab, în Izreel, căci acesta era bolnav.

\chapter{9}

\par 1 Atunci Elisei proorocul a chemat pe unul dintre fiii proorocilor și i-a zis: "Încingeți mijlocul și ia acest vas cu untdelemn în mâna ta și du-te la Ramot în Galaad.
\par 2 Iar dacă vei ajunge acolo, caută pe Iehu, fiul lui Iosafat, fiul lui Nimși, și, apropiindu-te, poruncește-i să iasă dintre frații lui și du-l în cămara cea mai dinăuntru.
\par 3 Apoi ia vasul cu untdelemn și toarnă pe capul lui și zi: Așa zice Domnul: Iată te ung rege peste Israel. După aceea deschide ușa și fugi fără zăbavă".
\par 4 Și s-a dus tânărul, sluga proorocului, la Ramot în Galaad.
\par 5 Și ajungând acolo, a văzut pe căpeteniile oștirii șezând și a zis: "Am un cuvânt către tine, căpetenie!" Și a zis Iehu: "Către care din noi toți?" Iar el a zis: "Către tine, căpetenie!"
\par 6 Atunci s-a sculat acesta și a intrat în casă. Iar tânărul a turnat untdelemn pe capul lui și i-a zis: "Așa zice Domnul Dumnezeul lui Israel: Te ung rege peste poporul Domnului, peste Israel.
\par 7 Tu vei nimici casa lui Ahab, stăpânul tău, de la fața Mea și vei răzbuna asupra Izabelei sângele robilor Mei, proorocii, și sângele tuturor slujitorilor Domnului.
\par 8 Toată casa lui Ahab va pieri și voi stârpi din ai lui Ahab pe tot cel de parte bărbătească, pe cel rob și pe cel slobod în Israel;
\par 9 Voi face casa lui Ahab ca și casa lui Ieroboam, fiul lui Nabat și ca și casa lui Baeșa, fiul lui Ahia.
\par 10 Iar pe Izabela, câinii o vor mânca în câmpia lui Izreel și nimeni nu o va îngropa". Apoi tânărul a deschis ușa și a fugit.
\par 11 După aceea a ieșit Iehu la slujitorii domnului său și aceștia i-au zis: "E pace? De ce a venit acest necunoscut la tine?" Și el i-a zis: "Voi cunoașteți pe acest om și de ce a venit". Iar ei au zis: "Nu este adevărat!
\par 12 Spune-ne!" Și el le-a zis: "Iată el mi-a spus: Așa grăiește Domnul: Iată te ung rege peste Israel".
\par 13 Atunci ei s-au grăbit să-și ia fiecare haina sa și i-au așternut-o pe trepte, au trâmbițat și au zis: "Iehu s-a făcut rege!"
\par 14 S-a sculat deci Iehu; fiul lui Iosafat, fiul lui Nimși, împotriva lui Ioram. Ioram însă fusese cu toți Israeliții la Ramot în Galaad și-l apărase împotriva lui Hazael, regele Siriei.
\par 15 Iar acum regele Ioram se întorsese la Izreel, ca să se vindece de rănile ce i le pricinuiseră Sirienii, când se luptase el cu Hazael, regele Siriei. Și a zis Iehu: "Dacă sunteți de părerea mea, să nu iasă nimeni din cetate, ca să dea de veste la Izreel".
\par 16 Apoi a încălecat Iehu pe cal și s-a dus în Izreel, unde zăcea Ioram, regele lui Israel și se îngrijea de rănile ce i le pricinuiseră Sirienii la Ramot, în timpul luptei cu Hazael, regele Siriei, cel tare și puternic și unde venise Ohozia, regele Iudei, ca să cerceteze pe Ioram.
\par 17 În turnul din Izreel stătea un ostaș de strajă. Văzând acesta ceata lui Iehu venind, a zis: "Văd o ceată!" Iar Ioram a zis: "Ia un călăreț și trimite-l în întâmpinare, ca să întrebe: Cu pace vii?"
\par 18 Și s-a dus călărețul călare în întâmpinarea lui și a zis: "Cu pace vii?" Iar Iehu a zis: "Ce ai tu cu pacea? Treci în urma mea!" și straja a dat de veste, zicând: "A ajuns la ei, dar nu se întoarce":
\par 19 Și a trimis alt călăreț și acesta s-a dus la ei și a zis: "Așa zice regele: Cu pace vii?" Iar Iehu a zis: "Ce ai tu cu pacea? Treci în urma mea!"
\par 20 Și a vestit straja, zicând: "A ajuns la ei, dar nu se întoarce. Și mersul parc-ar fi al lui Iehu, fiul lui Nimși, pentru că merge nebunește".
\par 21 Atunci Ioram a zis: "Înhamă!" Și au înhămat la carul lui: Și a ieșit Ioram, regele lui Israel și Ohozia; regele Iudei, fiecare în carul său, în întâmpinarea lui Iehu și s-au întâlnit cu el în țarina lui Nabot Izreeliteanul.
\par 22 Și când a văzut Ioram pe Iehu, a zis: "Cu pace, Iehu? Iar el a zis: "Ce pace, când sunt atâtea desfrânările Izabelei, mama ta, și vrăjitoriile ei?"
\par 23 Și întorcându-se Ioram să fugă, a zis către Ohozia: "Vânzare, Ohozia!
\par 24 Iar Iehu și-a întins arcul cu mâna sa și a lovit pe Ioram între umeri și săgeata a trecut prin inima lui și el a căzut în carul său.
\par 25 Și a zis Iehu către Bidcar, căpetenia oștirii: "Ia-l și-l aruncă în țarina lui Nabot Izreeliteanul, căci adu-ți aminte că atunci când mergeam eu și cu tine călări în urma lui Ahab, tatăl său, Domnul a rostit împotriva lui proorocia aceasta:
\par 26 Adevărat, am văzut ieri sângele fiilor lui, zice Domnul, și Mă voi răzbuna pe tine în țarina aceasta. Deci ia-l și-l aruncă în țarină, după cuvântul Domnului".
\par 27 Iar Ohozia, regele lui Iuda, văzând aceasta, a fugit pe drum spre casa ce se afla în grădină. Dar Iehu a alergat după el, zicând: "Ucide-ți-l și pe el în car!" Și l-au lovit pe înălțimea Gur, care vine lângă Ibleam. Și a fugit Ohozia la Moghido și a murit acolo.
\par 28 Iar slujitorii lui l-au dus la Ierusalim, l-au îngropat în mormânt cu părinții lui, în cetatea lui David.
\par 29 Ohozia se făcuse rege în Iuda în anul al unsprezecelea al lui Ioram, fiul lui Ahab.
\par 30 Apoi Iehu a venit în Izreel. Iar Izabela, fiind înștiințată de aceasta, și-a uns fața, și-a împodobit capul și privea de la fereastră.
\par 31 Iar când a intrat Iehu pe poartă, ea a zis: "Zimri, ucigașul stăpânului său, va avea el oare pace?"
\par 32 Și ridicându-și el fața sa și privind spre fereastră, a zis: "Cine, cine este cu mine?" Atunci s-au plecat spre el doi sau trei eunuci;
\par 33 Și el le-a zis: "Aruncați-o jos!" Și au aruncat-o și a țâșnit sângele ei pe zid și pe caii care au călcat-o în picioare.
\par 34 După aceea a venit Iehu de a mâncat și a băut și a zis: "Căutați pe ticăloasa aceea și îngropați-o, căci e fiică de rege!"
\par 35 Și s-au dus să o îngroape, dar n-au mai găsit nimic din ea, decât numai țeasta, picioarele și palmele mâinilor.
\par 36 Și s-au întors și i-au spus. Iar el a zis: "Așa a fost cuvântul Domnului, rostit prin robul Său Ilie Tesviteanul, zicând: În țarina Izreel câinii vor mânca trupul Izabelei;
\par 37 Și va fi trupul Izabelei în țarina Izreel ca gunoiul pe ogor, încât nimeni nu va zice: Iată ea e Izabela!"

\chapter{10}

\par 1 Ahab avea în Samaria șaptezeci de fii. Și a scris Iehu scrisori și le-a trimis la Samaria căpeteniilor cetății, bătrânilor și celor ce creșteau pe copiii lui Ahab,
\par 2 Zicând: "Când va ajunge scrisoarea aceasta la voi, la care se află și fiii domnului vostru, precum și care și cai, cetatea întărită și arme,
\par 3 Să alegeți pe cel mai bun și mai vrednic dintre fiii domnului vostru și să-l puneți pe tronul tatălui său și să vă luptați pentru casa domnului vostru".
\par 4 Aceștia însă s-au speriat cumplit și au zis: "Iată, doi regi nu i-au putut sta înainte; cum dar vom sta noi?"
\par 5 Și căpetenia casei domnești, căpeteniile cetății, bătrânii și cei ce creșteau pe copiii regelui au trimis la Iehu să-i spună: "Noi suntem robii tăi și ce ne vei zice, aceea vom face; nu vom pune pe nimeni rege, ci fă ceea ce-ți place".
\par 6 Și le-a scris Iehu scrisoare a doua oară și le-a zis: "De sunteți ai mei și de vă supuneți cuvântului meu, atunci ridicați capetele fiilor domnului vostru și veniți la mine în Izreel, mâine pe vremea aceasta". Fiii regelui erau în număr de șaptezeci și-i creșteau oameni de vază ai cetății.
\par 7 Când a ajuns la ei scrisoarea, ei au luat pe fiii regelui și i-au junghiat pe toți cei șaptezeci, au pus capetele în panere și le-au trimis la el în Izreel.
\par 8 Și venind trimisul, i-a adus știre și a zis: "S-au adus capetele fiilor regelui". Iar el a zis: "Așezați-le până dimineață în două grămezi la poartă".
\par 9 Și ieșind el dimineața a stat și a zis către tot poporul: "Voi nu sunteți vinovați. Iată eu m-am sculat împotriva stăpânului meu și l-am ucis.
\par 10 Dar pe aceștia cine i-a omorât? Să știți deci că nimic din ceea ce a spus Domnul împotriva casei lui Ahab n-a rămas neîmplinit. Domnul a făcut ce zisese prin robul Său Ilie".
\par 11 Și a omorât Iehu pe toți cei ce rămăseseră din casa lui Ahab în Izreel și pe toți cei mari ai lui și pe cei de aproape ai lui și pe preoții lui, încât n-a scăpat nici unul.
\par 12 Apoi sculându-se, a plecat să meargă la Samaria. Dar în drumul său, ajungând la Bet-Eched, adică la Colibele Păstorilor,
\par 13 Iehu a întâlnit pe frații lui Ohozia, regele Iudei, și le-a zis: "Cine sunteți voi?" Iar ei au răspuns: "Noi suntem frații lui Ohozia și ne ducem să aflăm de sănătatea fiilor regelui și a fiilor reginei".
\par 14 Și el a zis: "Prinde-ți-i de vii!" Și i-au prins de vii și i-au junghiat la fântâna de lângă Colibele Păstorilor. Aceia erau patruzeci și doi de oameni și n-a mai rămas nici unul din ei.
\par 15 Apoi plecând de acolo, s-a întâlnit cu Ionadab, fiul lui Recab, care venea în întâmpinarea lui, și l-a salutat și a zis: "Inima ta este ea oare cum este inima mea către inima ta?" Iar Ionadab a zis: "Da". "De este așa, dă-mi mâna!" Și i-a dat mâna și l-a ridicat la el în car,
\par 16 Zicând: "Hai cu mine, ca să vezi râvna mea pentru Domnul". Și l-a așezat în car.
\par 17 Și ajungând în Samaria, a ucis pe toți cei ce rămăseseră din ai lui Ahab în Samaria și așa a stârpit neamul lui cu totul, după cuvântul Domnului ce-l rostise prin Ilie.
\par 18 Apoi a adunat Iehu tot poporul și a zis: "Ahab a slujit puțin lui Baal; Iehu însă îi va sluji mai mult.
\par 19 Deci chemați la mine pe toți proorocii lui Baal, pe toți slujitorii lui și pe toți preoții lui, și nimeni să nu lipsească, pentru că am să fac o jertfă mare lui Baal. Tot cel ce va lipsi nu va rămâne cu viață". Iehu însă a făcut aceasta cu gând viclean, ca să stârpească pe slujitorii lui Baal.
\par 20 Și a zis Iehu: "Vestiți o zi de sărbătoare în cinstea lui Baal!"
\par 21 Și au vestit, iar Iehu a trimis în tot Israelul de au venit toți slujitorii lui Baal și n-a rămas nici unul care să nu fi venit. După aceea au intrat în templul lui Baal și s-a umplut templul de la un capăt la celălalt.
\par 22 Atunci a zis Iehu către păstrătorul veșmintelor: "Adu veșminte pentru toți slujitorii lui Baal". Și acela le-a adus veșminte.
\par 23 Apoi a intrat și Iehu cu Ionadab, fiul lui Recab, în templul lui Baal și a zis slujitorilor lui Baal: "Căutați și vedeți nu cumva se află printre voi careva din slujitorii Domnului, pentru că aici trebuie să fie numai slujitorii lui Baal singuri".
\par 24 Apoi s-au apropiat ei să facă jertfele și arderile de tot. Iar Iehu a pus afară optzeci de oameni și le-a zis: "Sufletul aceluia căruia îi va scăpa careva din oamenii pe care vi-i voi da în mâini, va fi în locul celui scăpat".
\par 25 Iar după ce s-a isprăvit arderea de tot, Iehu a zis către oștenii săi și către căpeteniile lor: "Duceți-vă și-i ucideți și să nu scape nici unul". Oștenii și căpeteniile i-au lovit cu ascuțișul sabiei și i-au aruncat acolo.
\par 26 Apoi s-au dus în cetate unde era capiștea lui Baal, au scos idolii din capiștea lui Baal și i-au ars;
\par 27 Și au sfărâmat chipul cel cioplit al lui Baal și au dărâmat capiștea lui Baal și au făcut din ea loc de necurățenii până în ziua de astăzi.
\par 28 Astfel a stârpit Iehu pe Baal din pământul lui Israel;
\par 29 Iar în ce privește păcatele lui Ieroboam, feciorul lui Nabat, care a dus pe Israel în ispită, de la acestea nu s-a depărtat nici Iehu; nu s-a lepădat de vițeii de aur, din Betel și Dan.
\par 30 Și a zis Domnul către Iehu: "Pentru că tu cu plăcere ai făcut ceea ce era drept în ochii Mei și ai îndeplinit împotriva casei lui Ahab tot ceea ce aveam la inima Mea, feciorii tăi până la al patrulea neam vor ședea pe tronul lui Israel".
\par 31 Dar Iehu nu s-a silit să umble din toată inima după legea Domnului Dumnezeului lui Israel și nu s-a abătut de la păcatele lui Ieroboam, care a dus pe Israel în ispită.
\par 32 În zilele acelea a început Domnul să taie părți din pământul lui Israel și Hazael a lovit hotarele lui Israel,
\par 33 Pustiind la răsărit de Iordan tot pământul Galaadului, al lui Gad, al lui Ruben și al lui Manase, începând de la Aroer, care vine lângă Arnon, și Galaadul și Vasanul.
\par 34 Celelalte fapte ale lui Iehu, tot ce a făcut el, precum și vitejiile lui sunt scrise în cartea faptelor regilor lui Israel.
\par 35 Și a răposat Iehu cu părinții săi în Samaria, iar în locul lui s-a făcut rege Ioahaz, fiul său.
\par 36 Iar timpul domniei lui Iehu peste Israel, în Samaria, a fost de douăzeci și opt de ani.

\chapter{11}

\par 1 Atalia, mama lui Ohozia, văzând că fiul său a murit, s-a sculat și a stârpit tot neamul regesc.
\par 2 Dar Ioșeba, fiica regelui Ioram, sora lui Ohozia, a furat pe Ioaș, fiul lui Ohozia, dintre fii regelui care trebuiau uciși și l-a dus pe ascuns în odaia de dormit, împreună cu doica lui, și l-a ascuns de Atalia și n-a fost ucis.
\par 3 Acesta a stat ascuns împreună cu ea în templul Domnului șase ani, în care timp a domnit peste țară Atalia.
\par 4 Iar în anul al șaptelea a trimis Iehoiada de au luat sutași din gardă și din oștire și i-au adus la el în casa Domnului. Aici a făcut cu ei legământ , luând de la ei jurământ în templul Domnului și apoi le-a arătat pe fiul regelui,
\par 5 Și le-a dat ordin, zicând: "Iată ce să faceți! A treia parte din voi, din cei ce veniți în ziua de odihnă, veți face strajă la casa domnească,
\par 6 A treia parte la poarta Sur și a treia parte la poarta gărzii, străjuind casa, ca să nu fie vreo vătămare;
\par 7 Iar două părți din voi, din cei plecați în ziua de odihnă, veți face strajă la templul Domnului pentru rege.
\par 8 Și veți înconjura pe rege din toate părțile, având fiecare arma sa în mână; cine va intra în rânduri, acela să fie ucis; și veți fi pe lângă rege și când va ieși și când va intra".
\par 9 Și au făcut ostașii tot ce le-a poruncit preotul Iehoiada. A luat fiecare pe oamenii săi cei ce intrau de servici și cei ce ieșeau din servici în ziua de odihnă și au venit la preotul Iehoiada.
\par 10 Iar preotul a împărțit sutașilor sulițele și scuturile regelui David care erau în templul Domnului.
\par 11 Și s-au așezat oștenii, fiecare cu arma în mână, împrejurul regelui, de la dreapta templului până la stânga lui și până la jertfelnic.
\par 12 Apoi au scos pe fiul regelui, au pus pe capul lui coroana și podoabele regești  și astfel l-au uns și l-au făcut rege, bătând din palme și strigând: "Trăiască regele!"
\par 13 Auzind Atalia glasul poporului ce striga, s-a dus în templul Domnului.
\par 14 Pe când se uita ea, iată regele stătea la locul de sus, după obicei, iar lângă rege stăteau cântăreții și trâmbițașii și tot poporul țării se veselea și suna din trâmbițe. Atunci Atalia și-a rupt veșmintele sale și a strigat: "Vânzare, vânzare!"
\par 15 Dar preotul Iehoiada a dat poruncă sutașilor, care cârmuiau oștirea, și le-a zis: "Scoateți-o din rânduri și pe cel ce se va duce după ea ucideți-l cu sabia", căci preotul zisese: "Să nu o ucideți în templul Domnului".
\par 16 Deci i-au făcut loc de a trecut pe poarta cailor, spre casa domnească, și au ucis-o acolo.
\par 17 Și a încheiat Iehoiada legământ între Domnul și între rege și popor, ca acesta să fie poporul Domnului, precum și între rege și popor.
\par 18 Și s-a dus tot poporul țării în capiștea lui Baal de a stricat jertfelnicele lui și chipurile lui le-au sfărâmat cu totul, iar pe Matan, preotul lui Baal, l-a ucis înaintea jertfelnicului, și preotul Iehoada a așezat strajă în templul Domnului.
\par 19 Apoi a luat pe sutași cu garda și oștirea și tot poporul țării și au petrecut pe rege din templul Domnului și au venit pe drumul ce trece pe poarta gărzii, la casa regelui și s-a urcat pe tronul regesc.
\par 20 Apoi s-a vestit poporului țării și cetatea s-a liniștit. Pe Atalia au omorât-o cu sabia în casa domnească.
\par 21 Ioaș era de șapte ani când a fost făcut rege.

\chapter{12}

\par 1 Astfel s-a făcut Ioaș rege în anul al șaptelea al lui Iehu și a domnit în Ierusalim patruzeci de ani.
\par 2 Ioaș a făcut fapte plăcute în ochii Domnului în toate zilele sale, cât l-a povățuit preotul Iehoiada;
\par 3 Numai înălțimile nu le-a desființat, căci poporul tot mai aducea jertfe și tămâieri pe înălțimi.
\par 4 Deci a zis Ioaș către preoți: "Tot argintul dăruit, care se aduce în templul Domnului: argintul de la trecători, argintul adus pentru răscumpărarea sufletului, după prețuirea hotărâtă, și tot argintul cât îl lasă pe cineva inima să-l aducă în templul Domnului,
\par 5 Să-l ia preoții pentru ei, fiecare de la cunoscutul său, și să repare stricăciunile casei oriunde s-ar afla".
\par 6 Însă până la anul al douăzeci și treilea al regelui Ioaș preoții n-au reparat stricăciunile templului Domnului.
\par 7 De aceea regele Ioaș a chemat pe preotul Iehoiada și pe ceilalți preoți și le-a zis: "De ce nu reparați stricăciunile templului Domnului? Deci să nu mai luați de acum argintul de la cunoscuții voștri, ci să-l dați pentru repararea stricăciunilor templului Domnului".
\par 8 Și s-au învoit preoții să nu mai ia argintul pe care-l va da poporul pentru repararea stricăciunilor templului Domnului.
\par 9 Atunci a luat preotul Iehoiada o ladă, i-a făcut o deschizătură în partea ei de deasupra șâ a pus-o lângă jertfelnic, în partea dreaptă, pe unde intra poporul în templul Domnului. Și preoții care stăteau de pază la prag puneau acolo tot argintul ce se aducea în templul Domnului.
\par 10 Și când vedeau că s-a strâns argint mult în ladă, venea vistiernicul regelui și arhiereul și scoteau argintul găsit în templul Domnului și-l legau în saci.
\par 11 Argintul așa socotit îl dădeau în primirea celor rânduiți să facă lucrările la templul Domnului, iar aceștia îl cheltuiau cu dulgherii și meșteșugarii care lucrau la templul Domnului,
\par 12 Cu zidarii și pietrarii, precum și cu cumpărarea lemnului și a pietrelor de cioplit, cu repararea stricăciunilor la templul Domnului și cu tot ce trebuia pentru întreținerea templului Domnului.
\par 13 Dar din argintul adus în templul Domnului nu s-au făcut pentru templul Domnului vase de argint, cuțite, cupe pentru turnare, trâmbițe, nici tot felul de vase de aur și de argint.
\par 14 Ci argintul care intra în templul Domnului se dădea lucrătorilor ca să săvârșească lucrul și să-l întrebuințeze la repararea templului Domnului.
\par 15 Și nu se cerea socoteală de la oamenii cărora li se încredința argintul ca să-l împartă celor ce făceau lucrările, pentru că aceștia se purtau cinstit.
\par 16 Însă argintul ce se plătea pe jertfa pentru vină și argintul ce se plătea pe jertfa pentru păcat nu se ducea în templul Domnului, ci acesta era al preoților.
\par 17 Atunci s-a ridicat Hazael, regele Siriei, și a pornit cu război împotriva cetății Gat și a luat-o. Și și-a pus Hazael în gând să meargă și asupra Ierusalimului.
\par 18 Însă Ioaș, regele Iudei, a luat toate lucrurile dăruite, pe care le dăruiseră templului Domnului Iosafat, Ioram și Ohozia, părinții lui, regii Iudei, precum și cele ce erau dăruite de el și tot aurul ce s-a găsit în vistieria templului Domnului și a casei regești și le-a trimis lui Hazael, regele Siriei, și acesta s-a retras din Ierusalim.
\par 19 Celelalte despre Ioaș și despre toate cele ce a făcut el sunt scrise în cartea faptelor regilor lui Iuda.
\par 20 În urmă s-au sculat slugile lui și au făcut răzvrătire împotriva lui și au ucis pe Ioaș, în casa Milo, pe calea spre Sela.
\par 21 Și l-au ucis slugile sale Iozacar, fiul lui Șimeat, și Iozabad, fiul lui Șomer. Și el a murit și l-au îngropat cu părinții lui în cetatea lui David. Și în locul lui s-a făcut rege Amasia, fiul său.

\chapter{13}

\par 1 În anul al douăzeci și treilea al lui Ioaș, fiul lui Ohozia, regele lui Iuda, s-a făcut rege peste Israel în Samaria Ioahaz, fiul lui Iehu, și a domnit șaptesprezece ani.
\par 2 Acesta a făcut rele în ochii Domnului și a umblat în păcatele lui Ieroboam, fiul lui Nabat, care a dus pe Israel în ispită și nu s-a lăsat de ele.
\par 3 De aceea s-a aprins mânia Domnului asupra lui Israel și l-a dat pentru totdeauna în mâna lui Hazael, regele Siriei, și în mâna lui Benhadad, fiul lui Hazael.
\par 4 Atunci s-a rugat Ioahaz Domnului și Domnul l-a auzit, pentru că vedea necazul Izraeliților și cum îi strâmtora regele Siriei.
\par 5 Și a dat Domnul Israeliților izbăvitor și au ieșit de sub mâna Sirienilor și au trăit fiii lui Israel în cetățile lor ca și mai înainte.
\par 6 Dar tot nu s-au depărtat de păcatele casei lui Ieroboam, care a dus pe Israel în ispită, ci au umblat în ele și Așera a rămas mai departe în Samaria,
\par 7 Lui Ioahaz nu-i rămăsese din oștire decât numai cincizeci de călăreți, zece care și zece mii de pedestrași, căci îi nimicise regele Siriei și-i prefăcuse în pulbere de călcat cu picioarele.
\par 8 Celelalte știri despre Ioahaz, despre vitejiile lui și despre tot ceea ce a făcut el, sunt scrise în cartea Cronicilor regilor lui Israel.
\par 9 Apoi a răposat Ioahaz cu părinții săi și l-au îngropat în Samaria, iar în locul lui s-a făcut rege Ioaș.
\par 10 În anul al treizeci și șaptelea al lui Ioaș, regele Iudei, s-a făcut rege peste Israel în Samaria Ioaș, fiul lui Ioahaz, domnind șaisprezece ani.
\par 11 Și a făcut fapte netrebnice înaintea Domnului și nu s-a abătut de la toate păcatele lui Ieroboam, fiul lui Nabat, care a dus pe Israel în ispită, ci a umblat în ele.
\par 12 Celelalte știri despre Ioaș, tot ceea ce a făcut el, isprăvile lui, războiul avut cu Amasia, regele Iudei, sunt scrise în cartea Cronicilor regilor lui Israel.
\par 13 Apoi a adormit Ioaș cu părinții săi, iar pe scaunul lui s-a suit Ieroboam. Ioaș a fost îngropat în Samaria, lângă regii lui Israel.
\par 14 În vremea lui Ioaș s-a îmbolnăvit Elisei de o boală de care apoi a și murit. Atunci a venit la el Ioaș, regele lui Israel și, plângând deasupra lui, a zis: "Părinte, părinte, carul lui Israel și călărețul lui!"
\par 15 Iar Elisei a zis către el: "Ia un arc și niște săgeți! "
\par 16 Și a luat regele un arc și niște săgeți. Apoi Elisei a zis către regele lui Israel: "Pune mâna pe arc!" și a pus regele mâna pe arc, și și-a pus și Elisei mâinile sale pe mâinile regelui,
\par 17 Și a zis: "Deschide fereastra dinspre răsărit!" Și regele a deschis-o. Și a zis Elisei: "Săgetează!". și a săgetat regele. Elisei a zis: "Aceasta este săgeata izbăvirii ce vine de la Domnul și săgeata izbăvirii de Sirieni, căci vei bate pe Sirieni cu desăvârșire, la Afec".
\par 18 Apoi iarăși a zis Elisei: "Ia niște săgeți!" și regele a luat. și a zis Elisei către regele lui Israel: "Lovește în pământ!" Și a lovit de trei ori și s-a oprit.
\par 19 Iar omul lui Dumnezeu s-a mâniat pe el, și a zis: "Trebuia să lovești de cinci sau de șapte ori, căci atunci ai fi bătut cu desăvârșire pe Sirieni; acum însă vei bate pe Sirieni numai de trei ori".
\par 20 Apoi a murit Elisei și l-au îngropat, iar în anul următor au intrat în țară cete de Moabiți.
\par 21 Dar iată, odată, când îngropau un mort, s-a întâmplat ca cei ce-l îngropau să vadă una din aceste cete și, speriindu-se, au aruncat mortul în mormântul lui Elisei. Căzând acela, s-a atins de oasele lui Elisei și a înviat și s-a sculat pe picioarele sale.
\par 22 Iar Hazael, regele Siriei, a strâmtorat pe Israeliți în toate zilele lui Ioahaz.
\par 23 Domnul însă S-a milostivit asupra lor de i-a iertat și S-a întors spre ei pentru legământul Său cel încheiat cu Avraam, Isaac și Iacov și n-a voit să-i piardă, nici nu i-a lepădat de la fața Sa, până astăzi.
\par 24 După ce a murit Hazael, regele Siriei, în locul lui a fost făcut rege Benhadad, fiul său.
\par 25 Atunci Ioaș, fiul lui Ioahaz, a luat înapoi din mâinile lui Benhadad, fiul lui Hazael, cetățile pe care le luase acela cu război din mâinile tatălui său, Ioahaz. De trei ori l-a bătut Ioaș și a întors cetățile lui Israel.

\chapter{14}

\par 1 În anul al doilea al lui Ioaș, fiul lui Ioahaz, regele lui Israel, s-a făcut rege în Iuda, Amasia, fiul lui Ioaș, regele Iudei.
\par 2 Amasia era de douăzeci și cinci de arii când a fost făcut rege și a domnit în Ierusalim douăzeci și nouă de ani. Numele mamei sale era Ioadin și era din Ierusalim.
\par 3 Acesta a făcut fapte plăcute în ochii Domnului, dar nu ca strămoșul său David; s-a purtat însă în toate ca tatăl său Ioaș.
\par 4 Numai înălțimile n-au fost înlăturate, căci poporul tot mai săvârșea jertfe și tămâieri pe înălțimi.
\par 5 Când a pus bine mâna pe domnie, Amasia a ucis slugile sale care uciseseră pe tatăl său;
\par 6 Dar pe copiii ucigașilor nu i-a ucis, de vreme ce în cartea legii lui Moise, prin care poruncește Domnul, este scris: "Părinții nu trebuie să fie pedepsiți cu moarte pentru copii, nici copiii nu trebuie să fie pedepsiți cu moarte pentru părinți, ci fiecare pentru vina sa trebuie să fie pedepsit cu moarte!"
\par 7 Tot el a bătut zece mii de Edomiți în Valea Sărată și a luat Șilo prin luptă și i-a pus numele Iocteel, pe care-l poartă până în ziua de astăzi.
\par 8 Atunci a trimis Amasia soli la Ioaș, regele lui Israel, fiul lui Ioahaz, fiul lui Iehu, zicând: "Vino să ne vedem la față".
\par 9 Iar Ioaș, regele lui Israel, a trimis la Amasia, regele Iudei, să i se zică: "Spânul cel din Liban a trimis la cedrul Libanului să-i zică: Dă-ți fata după fiul meu! Și trecând fiarele sălbatice din Liban au călcat spinul.
\par 10 Tu ai bătut pe Edomiți și s-a înălțat inima ta! Bucură-te de biruință, dar șezi acasă la tine! De ce să te cerți spre răul tău? Vei cădea și cu tine va cădea Iuda!"
\par 11 Amasia însă n-a ascultat. Atunci s-a sculat Ioaș, regele lui Israel, și s-a văzut față către față, el și Amasia, regele lui Iuda, la Bet-Șemeș în Iuda.
\par 12 Dar au fost bătuți Iudeii de Israeliți și au fugit în corturile lor.
\par 13 Iar pe Amasia, regele Iudei, fiul lui Ioaș, fiul lui Ohozia, l-a prins Ioaș, regele lui Israel, în Bet-Șemeș și s-a dus Ioaș la Ierusalim și a stricat zidul Ierusalimului de la porțile lui Efraim până la porțile din colț, pe o întindere de patru sute de coji.
\par 14 Și au luat tot aurul și argintul și toate vasele câte s-au găsit la templul Domnului și în vistieria casei regelui și ostateci și s-au întors în Samaria.
\par 15 Celelalte știri despre Ioaș, faptele și vitejiile lui și cum s-a luptat el cu Amasia, regele Iudei, sunt scrise în cartea Cronicilor regilor lui Israel.
\par 16 În urmă a adormit Ioaș cu părinții săi și a fost îngropat în Samaria cu regii lui Israel; iar în locul lui a fost făcut rege Ieroboam, fiul său.
\par 17 Amasia, fiul lui Ioaș, regele Iudei, a trăit cincisprezece ani după moartea lui Ioaș, fiul lui Ioahaz, regele lui Israel.
\par 18 Celelalte fapte ale lui Amasia sunt scrise în cartea Cronicilor regilor lui Iuda.
\par 19 Făcându-se însă la Ierusalim răsvrătire împotriva lui, a fugit în Lachiș, și s-a trimis după el în Lachiș și l-au ucis acolo,
\par 20 Și l-au adus pe cai și l-au îngropat în Ierusalim, cu părinții săi, în cetatea lui David.
\par 21 După aceea tot poporul iudeu a luat pe Azaria, care era atunci numai de șaisprezece ani, și l-a făcut rege în locul tatălui său Amasia.
\par 22 El a zidit Elatul și l-a întors la Iuda, după ce regele răposase cu părinții săi.
\par 23 În anul al cincisprezecelea al lui Amasia, fiul lui Ioaș, regele Iudei, a fost făcut rege în Samaria Ieroboam, fiul lui Ioaș, regele lui Israel.
\par 24 Și a domnit patruzeci și unu de ani și a făcut fapte netrebnice în ochii Domnului; neabătându-se de la toate păcatele lui Ieroboam, fiul lui Nabat, care a dus pe Israel în ispită.
\par 25 Acesta a așezat din nou vechiul hotar al lui Israel de la intrarea în Hamat până la marea Araba, după cuvântul Domnului Dumnezeului lui Israel, rostit prin robul Său Iona, fiul lui Amitai, proorocul cel din Gat-Hefer,
\par 26 Că Domnul văzuse necazul foarte amar al lui Israel cel strâmtorat, lipsit și părăsit și nu era cine să-l ajute.
\par 27 Și n-a vrut Domnul să stârpească numele Israeliților de sub cer, ci i-a izbăvit prin mâna lui Ieroboam, fiul lui Ioaș.
\par 28 Celelalte știri despre Ieroboam, despre toate cele ce a făcut el, și cum a întors Damascul și Hamatul lui Iuda în Israel sunt scrise în cartea Cronicilor regilor lui Israel.
\par 29 Apoi a adormit Ieroboam, cu părinții săi, cu regii lui Israel, iar în locul lui a fost făcut rege Zaharia, fiul său.

\chapter{15}

\par 1 În anul al douăzeci și șaptelea al lui Ieroboam, regele lui Israel, s-a făcut rege Azaria, fiul lui Amasia, regele lui Iuda.
\par 2 Acesta era de șaisprezece ani când S-a făcut rege și a domnit în Ierusalim cincizeci și doi de ani. Numele mamei sale era Iecolia din Ierusalim.
\par 3 Azaria a făcut fapte plăcute în ochii Domnului, purtându-se în toate ca tatăl său Amasia.
\par 4 Numai înălțimile nu le-a depărtat, căci poporul tot mai săvârșea jertfe ți tămâieri pe înălțimi.
\par 5 Dar a lovit Domnul pe rege, și acesta a fost lepros până în ziua morții sale și a trăit într-o casă osebită, în vreme ce Iotam, fiul regelui, era în casa domnească, judecând poporul țării.
\par 6 Celelalte știri despre Azaria și tot ce a făcut el sunt scrise în cartea Cronicilor regilor lui Iuda.
\par 7 În urmă a răposat Azaria cu părinții săi și l-au îngropat cu părinții lui în cetatea lui David, iar în locul lui S-a făcut rege Iotam, fiul său.
\par 8 În anul al treizeci și optulea al lui Azaria, regele Iudei, s-a făcut rege peste Israel, în Samaria, Zaharia, fiul lui Ieroboam, și a domnit șase luni.
\par 9 Acesta a făcut lucruri netrebnice în ochii Domnului, cum făcuseră și părinții lui, căci nu s-a lăsat de păcatele lui Ieroboam, fiul lui Nabat, care a dus pe Israel în ispită.
\par 10 Dar împotriva lui a făcut răzvrătire Șalum, fiul lui Iabeș, și l-a lovit înaintea poporului și l-a ucis și s-a făcut rege în locul lui.
\par 11 Celelalte știri despre Zaharia sunt scrise în cartea Cronicilor regilor lui Israel.
\par 12 Și astfel s-a împlinit cuvântul Domnului, cel rostit către Iehu, când a zis: "Fiii tăi până la al patrulea neam vor ședea pe tronul lui Israel".
\par 13 Șalum, fiul lui Iabeș, s-a făcut rege în anul al treizeci și nouălea al lui Azaria, regele Iudei, și a domnit o lună în Samaria.
\par 14 Atunci s-a dus Menahem, fiul lui Gadi, din Tirța și, ajungând în Samaria, a lovit pe Șalum, fiul lui Iabeș, în Samaria și l-a ucis și s-a făcut rege în locul lui.
\par 15 Alte știri despre Șalum și despre răzvrătirea pe care a făcut-o el sunt scrise în cartea Cronicilor regilor lui Israel.
\par 16 Atunci Menahem a bătut Tapuahul și pe toți cei ce erau în el și în hotarele lui, începând din Tirța, pentru că nu și-a deschis porțile, și a tăiat pe toate femeile însărcinate de acolo.
\par 17 Apoi Menahem, fiul lui Gadi, s-a făcut rege peste Israel în anul al treizeci și nouălea al lui Azaria, regele Iudei, și a domnit în Samaria zece ani;
\par 18 Și a făcut fapte rele în ochii Domnului, nelăsându-se în toate zilele sale de păcatele lui Ieroboam, fiul lui Nabat, care a dus pe Israel în ispită.
\par 19 Atunci a venit Ful (Tiglatfalasar III), regele Asiriei, asupra țării israelite. Însă Menahem a dat lui Ful o mie de talanți de argint, ca să-l susțină și să întărească domnia în mâinile lui.
\par 20 Apoi Menahem a scos argintul acesta de la Israeliți, de la toți cei bogați, câte cincizeci de sicli de argint de fiecare om, ca să-i dea regelui Asiriei. Și regele Asiriei s-a întors și n-a rămas acolo în țară.
\par 21 Celelalte știri despre Menahem, tot ce a făcut el, sunt scrise în cartea Cronicilor regilor lui Israel.
\par 22 În urmă a adormit Menahem cu părinții săi și în locul lui s-a făcut rege Pecahia, feciorul său.
\par 23 Pecahia, feciorul lui Menahem, s-a făcut rege peste Israel în Samaria în anul al cincizecilea al lui Azaria, regele Iudei și a domnit doi ani.
\par 24 Și a făcut fapte rele în ochii Domnului, nelăsându-se de păcatele lui Ieroboam, fiul lui Nabat, care a dus pe Israel în ispită.
\par 25 Dar împotriva lui s-a răzvrătit Pecah, fiul lui Remalia, cel al treilea în rang după el, împreună cu Argob și cu Arie, care aveau cu ei cincizeci de bărbați din Galaad și l-au lovit în Samaria, chiar în turnul casei regale, l-au omorât și s-a făcut rege În locul lui.
\par 26 Celelalte știri despre Pecahia, tot ce a făcut el, sunt scrise în cartea Cronicilor regilor lui Israel.
\par 27 În anul al cincizeci și doilea al lui Azaria, regele Iudei, s-a făcut rege peste Israel, în Samaria, Pecah, fiul lui Remalia și a domnit douăzeci de ani.
\par 28 Și a făcut lucruri rele în ochii Domnului, nelăsându-se de păcatele lui Ieroboam, fiul lui Nabat, care a dus pe Israel în ispită.
\par 29 În zilele lui Pecah, regele lui Israel, a venit Tiglatfalasar, regele Asiriei, și a luat Ionul, Abel-Bet-Maaca, Ianoah, Chedeș, Hațor, Galaadul, Galileea și tot pământul lui Neftali și pe locuitori i-a strămutat în Asiria.
\par 30 Atunci Osea, fiul lui Ela, a făcut o uneltire împotriva lui Pecah, fiul lui Remalia și l-a lovit și l-a ucis și s-a făcut rege în locul lui, în anul al douăzecilea al lui Ioatam, fiul lui Azaria.
\par 31 Celelalte știri despre Pecah și despre tot ce a făcut el sunt scrise în cartea Cronicilor regilor lui Israel.
\par 32 În anul al doilea al lui Pecah, fiul lui Remalia, regele lui Israel, s-a făcut rege Ioatam, fiul lui Azaria, regele Iudei.
\par 33 Acesta era de douăzeci și cinci de ani când s-a făcut rege și a domnit în Ierusalim șaisprezece ani. Numele mamei lui era Ierușa, fiica lui Țadoc.
\par 34 Și a făcut el lucruri plăcute în ochii Domnului. Cum s-a purtat tatăl său Azaria așa s-a purtat și el în toate.
\par 35 Numai înălțimile nu le-a înlăturat, căci poporul tot mai săvârșea jertfe și tămâieri pe înălțimi. Tot el a făcut poarta de sus la templul Domnului.
\par 36 Iar celelalte știri despre Ioatam și tot ce a făcut el, sunt scrise în cartea Cronicilor regilor lui Iuda.
\par 37 În zilele acelea a început Domnul a trimite asupra Iudei pe Rațon, regele Siriei, și pe Pecah, fiul lui Remalia.
\par 38 Ioatam a răposat cu părinții săi și a fost îngropat cu ei în cetatea lui David, strămoșul său, iar în locul lui a fost făcut rege Ahaz, fiul său.

\chapter{16}

\par 1 În anul al șaptesprezecelea al lui Pecah, fiul lui Remalia, a fost făcut rege Ahaz, fiul lui Ioatam, regele lui Iuda.
\par 2 Ahaz era de douăzeci de ani când s-a făcut rege și a domnit în Ierusalim șaisprezece ani, dar n-a făcut fapte plăcute în ochii Domnului Dumnezeului său, ca David, strămoșul său,
\par 3 Ci a pășit pe urmele regilor lui Israel și chiar și pe fiul său l-a trecut prin foc, urmând urâciunile popoarelor pe care le alungase Domnul de la fața fiilor lui Israel;
\par 4 Și a adus jertfe și tămâieri pe înălțimi și dealuri și sub tot pomul umbros.
\par 5 Atunci s-au dus Rațon, regele Siriei, și Pecah, fiul lui Remalia, regele lui Israel, asupra Ierusalimului, ca să-l cuprindă și au ținut pe Ahaz împresurat, dar nu l-au putut birui.
\par 6 În vremea aceea Rațon, regele Siriei, a întors Elatul la Siria și a izgonit pe Iudei din Elat; apoi au intrat în Elat Edomiții și trăiesc acolo până în ziua de astăzi.
\par 7 Atunci a trimis Ahaz soli la Tiglatfalasar, regele Asiriei, să-i spună: "Robul tău și fiul tău sunt eu; vino și mă apără de mâna regelui Siriei și de mâna regelui lui Israel, care s-au ridicat împotriva mea!"
\par 8 Cu acel prilej a luat Ahaz argintul și aurul care s-a găsit în templul Domnului și în vistieria casei domnești și l-a trimis dar regelui Asiriei.
\par 9 Deci l-a ascultat regele Asiriei și s-a dus regele Asiriei la Damasc, unde a prins pe Rațon și pe locuitorii lui i-a strămutat în Chir, iar pe Rațon l-a ucis.
\par 10 Atunci a ieșit regele Ahaz întru întâmpinarea lui Tiglatfalasar, regele Asiriei, la Damasc; și a văzut jertfelnicul cel din Damasc și a trimis regele Ahaz lui Urie preotul chipul jertfelnicului și planul alcătuirii lui.
\par 11 Iar preotul Urie a făcut un jertfelnic după planul ce i-l trimisese regele Ahaz din Damasc; așa a făcut preotul Urie până a venit regele de la Damasc.
\par 12 Iar dacă a venit regele de la Damasc și a văzut jertfelnicul, s-a apropiat regele de jertfelnic,
\par 13 A adus jertfă pe el și a ars arderea de tot a sa cu darul de pâine; a săvârșit turnarea sa și l-a stropit cu sângele jertfei de împăcare.
\par 14 Iar jertfelnicul cel de aramă, care era înaintea Domnului, l-a mutat din fața templului, dintre jertfelnicul cel nou și templul Domnului, și l-a pus în partea dinspre miazănoapte a acestuia.
\par 15 Apoi regele Ahaz a dat poruncă preotului Urie, zicând: "Pe jertfelnicul cel mare să arzi arderea de tot cea de dimineață și darul de pâine cel de seară, arderea de tot a regelui și darul lui de pâine, arderea de tot din partea întregului popor al țării și darul lui cel de pâine, turnarea lui și cu tot sângele arderii de tot și cu tot sângele jertfei să-l stropești, iar jertfelnicul cel de aramă va rămâne până voi vedea ce este de făcut cu el".
\par 16 Și a făcut preotul Urie așa cum a poruncit regele Ahaz.
\par 17 După aceea a desprins regele Ahaz pervazurile postamentelor, a luat bazinele de pe ele și a luat și marea de pe boii cei de aramă, care erau sub ea și a așezat-o pe o temelie de piatră;
\par 18 A desființat de asemenea pridvorul acoperit numit al zilei de odihnă, care fusese făcut la templul Domnului și pridvorul din afară al templului Domnului, care se numea al regelui, numai ca să placă regelui Asiriei.
\par 19 Celelalte știri despre Ahaz și cele ce a făcut el sunt scrise în cartea Cronicilor regilor lui Iuda.
\par 20 Apoi a răposat Ahaz cu părinții săi în cetatea lui David; iar în locul lui a fost făcut rege Iezechia, fiul său.

\chapter{17}

\par 1 În anul al doisprezecelea al lui Ahaz, regele Iudei, a fost făcut rege în Samaria peste Israel, Osea, fiul lui Ela și a domnit nouă ani.
\par 2 El a făcut lucruri rele în ochii Domnului, dar nu ca regii lui Israel, care au fost înainte de el.
\par 3 Împotriva lui s-a ridicat Salmanasar, regele Asiriei, și a ajuns Osea supusul acestuia și-i plătea bir.
\par 4 Dar regele Asiriei a simțit necredincioșia lui Osea, căci acesta trimisese robi la So, regele Egiptului, și nu plătise bir regelui Asiriei în fiecare an. De aceea regele Asiriei l-a luat legat și l-a aruncat în închisoare.
\par 5 Apoi regele Asiriei a năvălit asupra țării întregi și a mers și la Samaria și a ținut-o împresurată trei ani.
\par 6 Iar în anul al zecelea al lui Osea, regele Asiriei a luat Samaria și a strămutat pe Israeliți în Asiria și i-a așezat în Halach și în Habor, lângă râul Gozan, în cetățile Mediei.
\par 7 Când fiii lui Israel au început a păcătui înaintea Domnului Dumnezeului lor, Care îi scosese din. pământul Egiptului și de sub mâna lui Faraon, regele Egiptului, și s-au apucat să cinstească dumnezeii altora;
\par 8 Când au început ei să se poarte după obiceiurile popoarelor pe care le alungase Domnul de la fața fiilor lui Israel și după obiceiurile regilor lui Israel, făcând cum făceau aceștia;
\par 9 Când au început fiii lui Israel a face fapte neplăcute Domnului Dumnezeului lor, zidindu-și înălțimi prin toate târgurile lor, de la turnul de strajă, până la cetatea întărită,
\par 10 Și așezând idolii și chipurile Astartei pe tot dealul înalt și sub tot pomul umbros;
\par 11 Și când s-au apucat să săvârșească tămâieri pe toate înălțimile, ca popoarele pe care le izgonise Domnul de la ei și să facă fapte urâte, care mâniaseră pe Domnul
\par 12 Slujind idolilor de care Domnul le zisese: "Să nu faceți aceasta",
\par 13 Atunci Domnul a dat mărturie împotriva lui Israel și a lui Iuda prin toți proorocii Săi, prin toți văzătorii, zicând: "Întoarceți-vă din căile voastre cele rele și păziți poruncile Mele, așezămintele Mele și toată învățătura pe care Eu am dat-o părinților voștri și pe care v-am dat-o și vouă prin prooroci, robii Mei".
\par 14 Dar ei n-au ascultat, ci și-au învârtoșat cerbicia, ca și părinții lor care nu crezuseră în Domnul Dumnezeul lor
\par 15 Și au disprețuit poruncile Lui și legământul Lui, pe care-l încheiase El cu părinții lor și descoperirile Lui, cu care El îi deșteptase și au umblat după idoli și au ajuns netrebnici, purtându-se ca popoarele cele dimprejur de care Domnul le zisese să nu se poarte ca ele;
\par 16 Și au părăsit toate poruncile Domnului Dumnezeului lor și au făcut chipurile turnate a doi viței și au așezat Așere și s-au închinat la toată oștirea cerului și au slujit lui Baal;
\par 17 Și au trecut pe fiii lor și pe fiicele lor prin foc, au ghicit și au vrăjit și s-au apucat să facă lucruri netrebnice în ochii Domnului și să-L mânie.
\par 18 Atunci S-a mâniat Domnul tare pe Israeliți și i-a lepădat de la fața Sa, și n-a mai rămas decât seminția lui Iuda.
\par 19 Dar nici Iuda n-a păzit poruncile Domnului Dumnezeului său și s-a purtat după obiceiurile Israeliților, cum Se purtau aceștia.
\par 20 Și Și-a întors Domnul fața de la toți urmașii lui Israel și i-a smerit dându-i în mâinile jefuitorilor și în sfârșit i-a lepădat de la fața Sa.
\par 21 Căci Israeliții se dezbinaseră de la casa lui David și făcuseră rege pe Ieroboam, fiul lui Nabat, Ieroboam a abătut pe Israeliți de la Domnul și i-a băgat în păcat mare.
\par 22 Și au umblat fiii lui Israel în toate păcatele lui Ieroboam, câte făcuse acesta și nu s-a depărtat de la ele, până când n-a lepădat Domnul pe Israel de la fața Sa, cum zisese prin toți proorocii, robii Săi.
\par 23 Și a fost strămutat Israel din pământul său în Asiria, unde se află până în ziua de astăzi.
\par 24 După aceea regele Asiriei a adunat oameni din Babilon, din Cuta, din Ava, din Hamat și din Sefarvaim și i-a așezat prin cetățile Samariei în locul fiilor lui Israel. Aceștia au stăpânit Samaria și au început a locui prin cetățile ei.
\par 25 Dar fiindcă la începutul viețuirii lor acolo ei nu cinsteau pe Domnul, de aceea Domnul a trimis asupra lor lei care-i omorau.
\par 26 Atunci s-a spus regelui Asiriei, zicând: "Popoarele pe care tu le-ai strămutat și le-ai așezat prin cetățile Samariei nu cunosc legea Dumnezeului acelei țări și de aceea El trimite asupra lor lei și iată aceștia le omoară, pentru că ele nu cunosc legea Dumnezeului acelei țări".
\par 27 Iar regele Asiriei a poruncit și a zis: "Trimiteți acolo pe unul din preoții pe care i-ați adus de acolo, ca să se ducă să trăiască acolo și să-i învețe legea Dumnezeului acelei țări".
\par 28 Atunci a venit unul din preoții cei ce fuseseră aduși din Samaria și a locuit în Betel și i-a învățat cum să cinstească pe Domnul.
\par 29 Afară de acestea, fiecare popor și-a mai făcut și dumnezeii săi și i-a pus în capiștile de pe înălțimi pe care le făcuseră Samarinenii; fiecare popor în cetatea sa în care trăia.
\par 30 Căci Babilonenii și-au făcut pe Sucot-Benot, Cutienii și-au făcut pe Nergal, Hamatienii și-au făcut pe Așima,
\par 31 Aveenii și-au făcut pe Nivhaz și Tartac, iar Sefarvaimii își ardeau pe fiii și pe fiicele lor cu foc în cinstea lui Adramelec și Anamelec, zeii lor.
\par 32 Cinsteau însă și pe Domnul și și-au făcut dintre ei preoți pentru înălțimi și aceștia slujeau la ei, în capiștile de pe înălțimi.
\par 33 Dar ei cinsteau pe Domnul și slujeau zeilor săi după obiceiul popoarelor din mijlocul cărora fuseseră aduși.
\par 34 Așa urmează ei până în ziua de astăzi, după obiceiurile lor cele de la început; de Domnul nu se tem și nu urmează după așezămintele, după rânduielile, după legea și după poruncile pe care le-a poruncit Domnul fiilor lui Iacov, căruia îi dăduse numele de Israel și cu ai cărui urmași încheiase legământ și le poruncise așa: "Să nu cinstiți pe dumnezeii altora, nici să vă închinați lor;
\par 35 Să nu le slujiți nici să le aduceți jertfe;
\par 36 Ci să cinstiți pe Domnul, Care v-a scos din pământul Egiptului cu putere mare și cu braț înalt; pe Acesta să-L cinstiți și Lui să vă închinați și să-I aduceri jertfe;
\par 37 Siliți-vă în toate zilele să împliniți rânduielile, așezămintele, legea și poruncile pe care vi le-a scris El, iar pe zeii altora să nu-i cinstiți;
\par 38 Legământul pe care l-am încheiat cu voi să nu-l uitați și pe zeii altora să nu-i cinstiți;
\par 39 Ci să cinstiți numai pe Domnul Dumnezeul vostru și El vă va izbăvi din mâna tuturor vrăjmașilor voștri".
\par 40 Dar ei n-au ascultat, ci au urmat obiceiurile celor de mai înainte.
\par 41 Astfel popoarele acestea cinsteau pe Domnul, dar slujeau și idolilor lor. Ba și copiii lor și copiii copiilor lor până în ziua de astăzi urmează tot așa cum au urmat și părinții lor.

\chapter{18}

\par 1 În anul al treilea al lui Osea, fiul lui Ela, regele lui Israel, s-a făcut rege Iezechia, fiul lui Ahaz, regele Iudei.
\par 2 Acesta era de douăzeci și cinci de ani când s-a făcut rege și a domnit douăzeci și nouă de ani în Ierusalim. Numele mamei sale era Abia, fiica lui Zaharia.
\par 3 Acesta a făcut fapte plăcute în ochii Domnului în toate, cum făcuse și David, tatăl său, că el a desființat înălțimile, a sfărâmat stâlpii cu pisanii idolești, Așerele,
\par 4 Și a stricat șarpele cel de aramă, pe care-l făcuse Moise; chiar până în zilele acelea fiii lui Israel îl tămâiau și-l numeau Nehuștan.
\par 5 Și a nădăjduit el în Domnul Dumnezeu. Ca el n-a mai fost altul între toți regii lui Iuda, nici înainte, nici după;
\par 6 Căci s-a lipit el de Domnul și nu s-a abătut de la El, ci a păzit poruncile Lui, cum poruncise Domnul lui Moise.
\par 7 De aceea Domnul a fost cu el și tot ce a făcut Iezechia, făcea cu chibzuință. Și s-a depărtat el de regele Asiriei și n-a mai slujit aceluia.
\par 8 Apoi a bătut pe Filisteni până la Gaza, precum și în cuprinsul ei, de la turnul de pază până la cetatea întărită.
\par 9 în anul al patrulea al regelui Iezechia, adică în anul al șaptelea al lui Osea, fiul lui Ela, regele lui Israel, s-a dus Salmanasar, regele Asiriei, asupra Samariei și a împresurat-o și după trei ani a luat-o.
\par 10 În anul al șaptelea al lui Iezechia, adică în anul al zecelea al lui Osea, regele lui Israel, a fost luată Samaria.
\par 11 Atunci regele Asiriei a strămutat pe Israeliți în Asiria și i-a așezat în Halach, în Habor, lângă râul Gozan, în cetățile Mediei,
\par 12 Pentru că n-au ascultat glasul Domnului Dumnezeului lor și au călcat legământul Lui; tot ceea ce a poruncit Moise, robul Domnului, ei nici n-au ascultat, nici n-au făcut.
\par 13 Iar în anul al paisprezecelea al regelui Iezechia s-a dus Senaherib, regele Asiriei, asupra tuturor cetăților întărite ale lui Iuda și le-a luat.
\par 14 Atunci a trimis Iezechia, regele Iudei, la regele Asiriei în Lachiș, ca să-i zică: "Vinovat sunt! Du-te de la mine, căci ceea ce vei pune asupra mea, voi plăti!" Și a pus regele Asiriei asupra lui Iezechia, un bir de trei sute de talanii de argint și treizeci de talanți de aur.
\par 15 Și a dat Iezechia tot argintul ce s-a găsit în templul Domnului și în vistieria casei domnești.
\par 16 În vremea aceea a luat Iezechia aurul de pe ușile templului Domnului și de pe stâlpii cei vechi pe care-i aurise însuși Iezechia. și l-a dat regelui Asiriei.
\par 17 Și a trimis regele Asiriei din Lachiș pe Tartan, pe Rabsaris și pe Rabșache cu oștire mare asupra Ierusalimului și, sosind, s-au oprit la canalul iazului de sus, care se află lângă drumul ce merge spre țarina nălbitorului.
\par 18 Și chemând aceia pe rege, au ieșit la ei Eliachim, fiul lui Hilchia, căpetenia curții domnești, Șebna scriitorul și Ioah cronicarul, fiul lui Asaf.
\par 19 Atunci a zis către ei Rabșache: "Spuneți lui Iezechia: Așa zice regele cel mare, regele Asiriei: Ce fel de nădejde este aceea în care te reazemi? Tu ai spus numai vorbe goale; pentru război însă trebuie pricepere și putere.
\par 20 Acum însă în cine nădăjduiești tu, de te-ai răzvrătit împotriva mea?
\par 21 Iată, tu socoți să te reazemi pe Egipt, pe acea trestie frântă care de se sprijinește cineva în ea îi intră în mână și i-o sparge. Așa este Faraon, regele Egiptului, pentru toți cei ce nădăjduiesc în el.
\par 22 Iar de-mi veți zice: Noi nădăjduim în Domnul Dumnezeul nostru! Apoi în acela, oare, ale cărui înălțimi și jertfelnice le-a stricat Iezechia și a zis lui Iuda și Ierusalimului: Numai înaintea acestui jertfelnic să vă închinați, care este în Ierusalim?
\par 23 Deci intră în legătură cu stăpânul meu, regele Asiriei, și eu îți voi da două mii de cai; poți tu oare să găsești călăreți pentru ei?
\par 24 Cum vei birui tu măcar o singură căpetenie dintre cele mai mici slugi ale stăpânului meu? Ai nădejde în Egipt, pentru care și pentru călăreți?
\par 25 Pe lângă aceasta, au doară eu fără voia Domnului am venit la locul acesta ca să-l stric? Domnul mi-a zis: Du-te asupra țării acesteia și o strică!"
\par 26 Iar Eliachim, fiul lui Hilchia, Șabna și Ioah au zis către Rabșache: "Vorbește cu robii tăi în limba aramaică, pentru că noi înțelegem, și nu grăi cu noi în limba iudaică, în auzul poporului, care stă pe zid!"
\par 27 Zis-a Rabșache către ei: "Au doară stăpânul meu m-a trimis să spun aceste cuvinte numai stăpânului țării și ție? Nu, ci și poporului care stă pe zid și care va ajunge să-și mănânce murdăria și să-și bea udul cu voi!"
\par 28 Apoi  s-a sculat Rabșache și a strigat cu glas tare în limba iudaică și a spus aceste cuvinte: "Ascultați cuvântul regelui celui mare, regele Asiriei!
\par 29 Așa zice regele: Să nu vă înșele pe voi Iezechia, căci nu poate să vă izbăvească din mâna mea,
\par 30 Și să nu vă încurajeze Iezechia cu Domnul, zicând: Ne va izbăvi Domnul și cetatea aceasta nu va fi dată în mâinile regelui Asiriei.
\par 31 Să nu ascultați pe Iezechia, căci așa zice regele Asiriei: Împăcați-vă cu mine și ieșiți la mine; să-și mănânce fiecare rodul viței sale de vie și al smochinului său și să bea fiecare apă din fântâna sa, până nu vin să vă iau într-o țară la fel cu țara voastră,
\par 32 În țara pâinii și a vinului, în țara fructelor și a viilor, în țara smochinelor și mierei, și nu veți muri, ci veți trăi. Deci nu ascultați pe Iezechia, care vă amăgește, zicând: Domnul ne va izbăvi.
\par 33 Dumnezeii popoarelor au izbăvit ei oare fiecare țara sa din mâna regelui Asiriei?
\par 34 Unde sunt dumnezeii Hamatului și ai Arpadului? Unde sunt dumnezeii Sefarvaimului, Inei și Hevei? Au scăpat oare Samaria din mâna mea?
\par 35 Care din dumnezeii țărilor acestora a izbăvit țara sa din mâna mea? Așadar va izbăvi Domnul Ierusalimul din mâna mea?"
\par 36 Poporul însă a tăcut și nu i-a răspuns nici un cuvânt, pentru că porunca regelui era să nu-i răspundă.
\par 37 După aceea a venit Eliachim, fiul lui Hilchia, căpetenia curții domnești, Șebna scriitorul și Ioah cronicarul, fiul lui Asaf, la Iezechia, cu hainele sfâșiate și i-au spus vorbele lui Rabșache.

\chapter{19}

\par 1 Auzind acestea, regele Iezechia și-a rupt hainele sale, s-a îmbrăcat cu sac și s-a dus în templul Domnului.
\par 2 Atunci a trimis pe Eliachim, căpetenia curții domnești, pe Șebna scriitorul și pe preoții cei mai mari, îmbrăcați în sac, la Isaia proorocul, fiul lui Amos,
\par 3 Și aceștia au zis către el: "Așa zice Iezechia: Zi de necaz, de pedeapsă și de rușine este ziua aceasta, căci pruncii sunt aproape să iasă din pântecele mamelor și ele nu pot naște.
\par 4 Poate va auzi Domnul Dumnezeul tău toate cuvintele lui Rabșache, pe care l-a trimis regele Asiriei, stăpânul lui, să defaime pe Dumnezeul cel viu și să-L hulească cu vorbele pe care le-a auzit Domnul Dumnezeul tău. Adu dar rugăciune pentru cei ce au rămas și se află printre cei vii!"
\par 5 Și au venit slugile regelui Iezechia la Isaia,
\par 6 Iar Isaia le-a zis: "Așa să ziceți domnului vostru: Așa grăiește Domnul: Nu te teme de vorbele ce le-ai auzit și cu care M-au hulit pe Mine slugile regelui Asiriei.
\par 7 Căci iată voi trimite în el duh și va auzi o veste și se va întoarce în țara sa, iar acolo îl voi lovi cu sabia".
\par 8 Atunci s-a întors Rabșache și a găsit pe regele Asiriei războindu-se împotriva Libnei, căci auzise că acesta a plecat din Lachiș.
\par 9 Și a auzit el de Tirhaca, regele Etiopiei, căci i s-a spus: "Iată vine să se lupte cu tine". Și din nou a trimis soli la Iezechia să-i spună:
\par 10 "Așa să ziceți lui Iezechia, regele Iudei: Să nu te înșele Dumnezeul tău, în Care nădăjduiești tu, gândind: Nu va fi dat Ierusalimul în mâinile regelui Asiriei!
\par 11 Doar tu ai auzit ce a făcut regele Asiriei cu toate țările, aruncând asupra lor blestem. Și tu ai să rămâi?
\par 12 Dumnezeii popoarelor pe care le-au ruinat părinții mei le-au izbăvit ei oare?
\par 13 Unde este regele Hamatului, sau regele Arpadului, sau regele cetății Sefarvaimului, Inei și Hevei?"
\par 14 Și a luat Iezechia scrisoarea din mâna solilor și a citit-o; apoi s-a dus în templul Domnului și a deschis-o Iezechia înaintea felei Domnului.
\par 15 Și s-a rugat Iezechia înaintea feței Domnului și a zis: "Doamne Dumnezeul lui Israel, Cel ce șezi pe heruvimi, numai Tu singur ești Dumnezeul tuturor regatelor pământului, Tu ai făcut cerul și pământul!
\par 16 Pleacă-ți, Doamne, urechea Ta și mă auzi! Deschide-ți, Doamne, ochii Tăi și vezi și auzi cuvintele lui Senaherib, cel ce a trimis să Te hulească pe Tine, Dumnezeul cel viu!
\par 17 Adevărat, o, Doamne, regii Asiriei au pustiit popoarele și țările lor, au aruncat dumnezeii acestora în foc, dar aceia nu erau dumnezei, ci lucruri de mâini omenești, lemn și piatră, și de aceea i-a și nimicit pe ei.
\par 18 Și acum, Doamne Dumnezeul nostru, izbăvește-ne din mâna lui,
\par 19 Și vor afla toate regatele pământului că numai Tu, Doamne, ești Dumnezeu! "
\par 20 Atunci a trimis Isaia, fiul lui Amos, la Iezechia să-i spună: "Așa zice Domnul Dumnezeul lui Israel: Cele pentru care te-ai rugat tu Mie împotriva lui Senaherib, regele Asiriei, le-am auzit.
\par 21 Iată cuvântul pe care l-a rostit Domnul pentru el: Te disprețuiește și râde de tine fecioara, fiica Sionului, și clatină din cap, în urma ta, fiica Ierusalimului.
\par 22 Pe cine însă ai mustrat și ai hulit tu? Și asupra cui ți-ai ridicat tu glasul și ți-ai înălțat așa de sus ochii tăi? Asupra Sfântului lui Israel.
\par 23 Prin trimișii tăi tu ai înfruntat pe Domnul și ai zis: Cu mulțimea carelor mele m-am urcat pe înălțimea munților, pe coastele Libanului, și am tăiat cedrii cei falnici și chiparoșii cei minunați ai lui, și am ajuns la cea din urmă adăpostire a lui, la grădina lui plină de pomi;
\par 24 Și am săpat și am băut apă străină și cu tălpile picioarelor mele voi seca râurile Egiptului.
\par 25 N-ai auzit tu, oare, că aceasta Eu am făcut-o de demult, din zilele cele de demult am hotărât-o? Acum însă am îndeplinit-o prin aceea că tu pustiești și prefaci cetățile în dărâmături,
\par 26 Iar locuitorii lor au ajuns neputincioși, tremură și rămân rușinați; au ajuns ca iarba câmpului și ca verdeața cea fragedă, ca buruienile de pe acoperișul casei și ca niște fire de grâu uscate înainte de a da în spic.
\par 27 De șezi, de intri, ori de ieși, Eu toate le știu și știu și obrăznicia ta față de Mine.
\par 28 Pentru obrăznicia ta cea față de Mine și pentru că trufia ta a ajuns până la urechile Mele, Îmi voi pune veriga Mea în nările tale și în buzele tale belciugul Meu și te voi întoarce pe același drum pe care ai venit.
\par 29 Iar pentru tine, Iezechia, iată semn: Anul acesta veți mânca din cele ce vor crește din semințele scuturate; în anul al doilea veți mânca din cele ce vor crește de la sine, iar în anul al treilea veți semăna și veri secera, veți sădi vii și veți mânca roadele lor.
\par 30 Ceea ce nu se va strica în casa lui Iuda, ceea ce va  rămâne, va da rădăcini în jos, iar în sus va aduce rod, căci din Ierusalim vor ieși câțiva rămași și din Sion, câțiva izbăviți.
\par 31 Râvna Domnului Savaot va face aceasta.
\par 32 De aceea așa zice Domnul de regele Asiriei: Nu va intra în cetatea aceasta, nici va arunca săgeți încoace; nu se va apropia de ea cu scut, nici va face întărituri de șanțuri împotriva ei.
\par 33 Pe drumul pe care a venit, se va întoarce și în cetatea aceasta nu va intra, zice Domnul;
\par 34 Căci Eu voi păzi cetatea aceasta, ca să o izbăvesc pentru Mine și pentru David, robul Meu".
\par 35 În noaptea aceea s-a întâmplat că a ieșit îngerul Domnului și a lovit în tabăra Asirienilor o sută optzeci și cinci de mii, și când s-au sculat dimineața, iată erau peste tot numai trupuri moarte.
\par 36 Atunci Senaherib, regele Asiriei, sculându-se, a plecat și s-a întors și a locuit în Ninive.
\par 37 Dar pe când se ruga el în casa lui Nisroc, zeul său, l-au ucis cu sabia fiii săi Adramelec și Șarețer și au fugit în pământul Ararat, iar în locul lui s-a făcut rege Asarhadon, fiul lui.

\chapter{20}

\par 1 În zilele acelea s-a îmbolnăvit Iezechia de moarte și a venit la el Isaia proorocul, fiul lui Amos, și i-a zis: "Așa grăiește Domnul: Fă testament pentru casa ta, căci nu te vei mai însănătoși, ci vei muri!"
\par 2 Atunci s-a întors Iezechia cu fața la perete și s-a rugat Domnului,
\par 3 Zicând: "O, Doamne, adu-ți aminte că am umblat înaintea feței Tale cu credință și cu inimă dreaptă și am făcut cele plăcute în ochii Tăi!" Și a plâns Iezechia tare.
\par 4 Isaia însă nu plecase încă din cetate, când a fost cuvântul Domnului către el și i-a zis:
\par 5 "Întoarce-te și spune lui Iezechia, stăpânul poporului Meu: Așa zice Domnul Dumnezeul lui David, strămoșul tău: Am auzit rugăciunea ta și am văzut lacrimile tale; te vei vindeca și a treia zi te vei duce în templul Domnului;
\par 6 Și voi mai adăuga la zilele tale cincisprezece ani și din mâna regelui Asiriei te voi izbăvi pe tine și cetatea aceasta o voi apăra pentru Mine și pentru David, robul Meu!"
\par 7 Și a zis Isaia: "Luați o turtă de smochine! Și au luat o turtă de smochine și au pus-o pe rană și s-a însănătoșit Iezechia.
\par 8 Și a zis către Isaia: "Care este semnul că Domnul mă va vindeca și că mă voi duce a treia zi în templul Domnului?"
\par 9 Iar Isaia a zis: "Iată semn de la Domnul că-și va împlini Domnul cuvântul pe care l-a rostit: Vrei să treacă umbra la ceasul de soare cu zece linii înainte sau să se dea cu zece linii înapoi?"
\par 10 Iezechia a zis: "E ușor ca umbra să se miște cu zece linii înainte. Nu, ci să se dea umbra cu zece linii înapoi".
\par 11 Și a strigat Isaia proorocul către Domnul și s-a dat înapoi cu zece linii.
\par 12 În vremea aceea Merodac Baladan, fiul lui Baladan, regele Babilonului, a trimis scrisoare și dar lui Iezechia căci auzise că Iezechia a fost bolnav.
\par 13 Ascultând Iezechia pe trimiși, le-a arătat cămările sale, argintul, aurul, aromatele, mirurile cele scumpe și toată casa sa de arme și tot ce se afla în vistieriile sale; și nu a rămas nici un lucru din casa sa și din toată stăpânirea sa pe care să nu-l fi arătat lor Iezechia.
\par 14 Venind însă Isaia proorocul la regele Iezechia, a zis către el: "Ce au zis oamenii aceștia și de unde au venit la tine?" Iezechia a răspuns: "Dintr-o țară depărtată, au venit din Babilon".
\par 15 Isaia a zis: "Ce au văzut ei în casa ta?" Și Iezechia a zis: "Tot ce este în casa mea au văzut și n-a rămas nici un lucru din vistieriile mele pe care să nu-l fi văzut".
\par 16 Atunci Isaia a zis: "Ascultă cuvântul Domnului: Iată vor veni zile când vor fi luate toate câte sunt în casa ta și ce-au adunat părinții tăi până în ziua aceasta și vor fi duse la Babilon. Nimic nu va rămâne, zice Domnul.
\par 17 Din fiii tăi care vor răsări din tine și pe care îi vei naște tu,
\par 18 Se vor lua și vor fi eunuci în palatul regelui Babilonului".
\par 19 Iezechia a răspuns lui Isaia: "Bun este cuvântul Domnului pe care l-ai rostit tu!" Apoi a adăugat: "Să fie pace și liniște în zilele mele!"
\par 20 Celelalte fapte ale lui Iezechia, luptele lui și cum că el a făcut iazul și canalul pentru adus apă în cetate, sunt scrise în cartea faptelor regilor lui Iuda.
\par 21 Apoi a răposat Iezechia cu părinții săi și în locul lui s-a făcut rege Manase, fiul său.

\chapter{21}

\par 1 Manase era de doisprezece ani când s-a făcut rege și a domnit în Ierusalim cincizeci și cinei de ani. Numele mamei lui era Hefțibah.
\par 2 Acesta a făcut lucruri netrebnice înaintea Domnului, urmând urâciunile păgânilor pe care-i izgonise Domnul de la fața fiilor lui Israel.
\par 3 El a făcut din nou înălțimile pe care le stricase tatăl său Iezechia și a așezat jertfelnice pentru Baal; a făcut Așere, cum făcuse și Ahab, regele lui Israel, și s-a închinat la toată oștirea cerească, slujind acesteia.
\par 4 Apoi a zidit jertfelnice chiar și în templul Domnului, de care zisese Domnul: "În Ierusalim voi pune numele Meu!"
\par 5 Și a făcut jertfelnice la toată oștirea cerului în amândouă curiile templului Domnului;
\par 6 A trecut pe fiul său prin foc, a ghicit, a vrăjit, a adus oameni care se îndeletniceau cu chemarea morților și vrăjitori și a făcut și alte multe lucruri urâte Domnului, ca să-L mânie.
\par 7 După aceea chipul Așerei pe care îl făcuse l-a așezat în casa despre care Domnul îi zisese lui David și lui Solomon, fiul lui: "În casa aceasta și în Ierusalim, pe care l-am ales din toate semințiile lui Israel, voi pune numele Meu pe vecie;
\par 8 Și nu voi mai da să calce picior de israelit afară din țara pe care am dat-o părinților lor, de se vor sili să se poarte potrivit cu toate cele ce Eu le-am poruncit și cu toată legea care le-a dat-o robul Meu Moise".
\par 9 Dar ei n-au ascultat, ci i-a rătăcit Manase până într-atâta, încât ei s-au purtat mai rău decât acele popoare pe care Domnul le stârpise de la fața fiilor lui Israel.
\par 10 Atunci Domnul a grăit prin prooroci, robii Săi, și a zis:
\par 11 "Pentru că Manase, regele Iudei, a făcut astfel de urâciuni, mai rele decât tot ce au făcut Amoreii care au fost înainte de el, și a vârât pe Iuda în păcat cu idolii lui,
\par 12 De aceea așa zice Domnul Dumnezeul lui Israel: Iată Eu voi aduce așa rău asupra Ierusalimului și asupra lui Iuda, încât celui ce va auzi îi vor țiui amândouă urechile;
\par 13 Și voi întinde peste Ierusalim frânghia de măsurat a Samariei și cumpăna casei lui Ahab, și voi șterge Ierusalimul, așa cum se șterge un vas și se pune apoi cu gura în jos;
\par 14 Și voi lepăda rămășita moștenirii Mele și-i voi da în mâinile vrăjmașilor lor, și vor fi de pradă și de jaf pentru toți prietenii lor,
\par 15 Pentru că au făcut lucruri netrebnice înaintea ochilor Mei și M-au mâniat din ziua aceea când părinții lor au ieșit din Egipt și până în ziua aceasta".
\par 16 Mai mult încă, Manase, pe lângă păcatul său de a fi dus pe Iuda în ispită, a vărsat și foarte mult sânge nevinovat, încât a mânjit Ierusalimul de la o margine la alta.
\par 17 Celelalte știri despre Manase și despre toate câte a făcut el și despre păcatele lui, în ce anume a păcătuit, sunt scrise în cartea faptelor regilor lui Iuda.
\par 18 Apoi a răposat Manase cu părinții săi și a fost îngropat în grădina de lângă casa lui, în grădina lui Uza, iar în locul lui s-a făcut rege Amon, fiul său.
\par 19 Amon era de douăzeci și doi de ani când s-a făcut rege și a domnit doi ani în Ierusalim. Numele mamei lui era Meșulemet, fiica lui Haruț din Iotba.
\par 20 Și acesta a făcut lucruri netrebnice în ochii Domnului, cum făcuse și Manase, tatăl său;
\par 21 A umblat întocmai pe aceeași Cale pe care umblase și tatăl său, slujind idolilor cărora slujise și tatăl său și închinându-se lor.
\par 22 A părăsit pe Domnul Dumnezeul părinților săi și n-a umblat în calea Domnului.
\par 23 Dar slugile lui Amon s-au răzvrătit împotriva lui și au ucis pe rege în casa lui.
\par 24 Poporul însă a ucis pe toți cei ce luaseră parte la răzvrătire împotriva regelui Amon și a pus rege în locul lui pe Iosia, fiul lui.
\par 25 Celelalte știri despre Amon și despre cele ce a făcut el sunt scrise în cartea faptelor regilor lui Iuda.
\par 26 Amon a fost îngropat în gropnița lui, în grădina lui Uza, iar în locul lui s-a făcut rege Iosia, fiul lui.

\chapter{22}

\par 1 Iosia era de opt ani când s-a făcut rege și a domnit treizeci și unu de ani în Ierusalim; numele mamei lui era Iedida, fiica lui Adaia din Boțcat.
\par 2 Iosia a făcut fapte plăcute înaintea Domnului și a umblat în toate pe calea lui David, strămoșul său, neabătându-se nici la dreapta, nici la stânga.
\par 3 În anul al optsprezecelea al regelui Iosia, regele a trimis pe scriitorul Șafan, fiul lui Ațalia, fiul lui Meșulam, în templul Domnului, zicându-i:
\par 4 "Du-te la Hilchia, arhiereul, ca să socotească argintul adus în templul Domnului, pe care l-au strâns de la popor cei ce stau de strajă la prag,
\par 5 Și să-l dea în mâna celor puși să facă lucrările la templul Domnului, iar aceștia să-l dea celor ce lucrează și repară stricăciunile lui:
\par 6 Dulgherilor, pietrarilor și zidarilor, și la cumpărarea lemnului și a pietrelor cioplite pentru repararea templului;
\par 7 Însă să nu le cereți socoteală de argintul ce s-a dat în mâna lor, pentru că se poartă cinstit".
\par 8 Iar Hilchia arhiereul a zis către Șafan scriitorul: "Am găsit în templul Domnului cartea legii". Apoi Hilchia a dat lui Șafan cartea și el a citit-o.
\par 9 Și venind Șafan scriitorul la rege, a adus răspuns regelui și a zis: "Robii tăi au luat argintul ce s-a găsit în casă și l-au dat în mâinile celor puși să facă lucrările la templul Domnului".
\par 10 Și a mai adus Șafan la cunoștința regelui și acestea, zicând: "Preotul Hilchia mi-a dat o carte". Și a citit-o Șafan înaintea regelui.
\par 11 Auzind regele cuvintele cărții legii, și-a sfâșiat hainele sale.
\par 12 Apoi regele a poruncit preotului Hilchia, lui Ahicam, fiul lui Șafan, lui Acbor, fiul lui Miheia, lui Șafan scriitorul și lui Asaia, sluga regelui:
\par 13 "Mergeți și întrebați pe Domnul pentru mine și pentru popor și pentru tot Iuda despre cuvintele acestei cărți găsite, căci mare este mânia Domnului ce s-a aprins asupra noastră, pentru că părinții noștri n-au ascultat cuvintele cărții acesteia, ca să se poarte după cele ce ni s-a poruncit".
\par 14 Atunci s-a dus Hilchia preotul, Ahicam, Acbor, Șafan și Asaia la proorocița Hulda, soția lui Șalum, fiul lui Țicva, fiul lui Harhas, păstrătorul veșmintelor, care locuia în Ierusalim, în despărțitura a doua, și au grăit cu ea.
\par 15 Iar ea le-a zis: "Așa grăiește Domnul Dumnezeul lui Israel: Spuneți omului care v-a trimis la mine:
\par 16 Așa zice Domnul: Voi aduce rău asupra locului acestuia și asupra locuitorilor lui, după toate cuvintele cărții pe care a citit-o regele Iudei.
\par 17 Pentru că M-au părăsit și tămâiază pe alți dumnezei, ca să mă ațâțe cu toate lucrurile mâinilor lor, s-a aprins mânia Mea asupra locului acestuia și nu se va stinge.
\par 18 Iar regelui lui Iuda care v-a trimis să întrebați pe Domnul, spuneți-i: Așa zice Domnul Dumnezeul lui Israel despre cuvintele pe care tu le-ai auzit:
\par 19 Deoarece s-a muiat inima ta și tu te-ai smerit înaintea Domnului, când ai auzit ce am grăit Eu asupra locului acestuia și asupra locuitorilor lui, că vor fi ținta groazei și a blestemului, și ji-ai rupt veșmintele și ai plâns înaintea Mea, de aceea și Eu te-am auzit, zice Domnul.
\par 20 De aceea iată te voi adăuga la părinții tăi și vei fi pus în gropnița ta cu pace; și nu vor vedea ochii tăi toate acele nenorociri pe care le voi aduce asupra locului acestuia". Și s-a adus regelui răspunsul acesta.

\chapter{23}

\par 1 Atunci a trimis regele să fie chemați toți bătrânii lui Iuda și ai Ierusalimului.
\par 2 Apoi s-a dus regele în templul Domnului și toți Iudeii și toți locuitorii Ierusalimului au mers cu el, și preoții și proorocii și tot poporul de la mic până la mare și au citit în auzul lor toate cuvintele cărții legământului, ce s-a găsit în templul Domnului.
\par 3 După aceea a stat regele pe un loc înalt și a încheiat înaintea Domnului legământ ca să urmeze Domnului și să păzească poruncile Lui, descoperirile Lui și legiuirile Lui cu toată inima sa și cu tot sufletul, ca să împlinească cuvintele legământului acestuia, scrise în cartea aceasta.
\par 4 Apoi regele a poruncit lui Hilchia arhiereul, preoților de mâna a doua și celor ce stăteau de strajă la prag, să scoată din templul Domnului toate lucrurile făcute pentru Baal, pentru Astarte și pentru toată oștirea cerului și să le ardă afară din Ierusalim, în valea Chedronului, iar cenușa lor să o ducă la Betel.
\par 5 A izgonit după aceea pe preoții idolilor pe care-i puseseră regii Iudei, ca să facă tămâieri pe înălțimi, în cetățile Iudei și în împrejurimile Ierusalimului, și care tămâiau pe Baal, soarele, luna, stelele și toată oștirea cerului.
\par 6 Atunci au scos Așera din templul Domnului afară din Ierusalim, la pârâul Chedron, au ars-o la pârâul Chedron și au făcut-o praf; și praful l-au aruncat asupra cimitirului obștesc al poporului.
\par 7 Apoi au dărâmat casele de desfrâu care se aflau lângă templul Domnului, unde femeile țineau veșminte pentru Astarte;
\par 8 Au scos pe toți slujitorii idolilor din cetățile lui Iuda și au spurcat înălțimile pe care ei săvârșeau tămâieri, de la Gheba până la Beer-Șeba, stricând înălțimile de la porți: cea care se afla la poarta lui Iosua, căpetenia cetății, și cea care se afla în partea stângă a porților cetății.
\par 9 De atunci slujitorii înălțimilor nu aduceau jertfe pe jertfelnicul Domnului celui din Ierusalim, ci mâncau numai azime cu frații lor.
\par 10 După aceea regele a spurcat locurile de jertfă din valea fiilor lui Hinom, ca nimeni să nu mai treacă pe fiul său sau pe fiica sa prin foc lui Moloh.
\par 11 A nimicit caii pe care regii lui Israel îi așezaseră în cinstea soarelui înaintea intrării templului Domnului, aproape de locuința eunucului Netan-Melec cea din Parvarim, iar carul soarelui l-a ars.
\par 12 Jertfelnicele cele de pe acoperișul foișorului lui Ahaz, pe care le făcuseră regii Iudei, și jertfelnicele pe care le făcuse Manase în amândouă curțile templului Domnului, le-a dărâmat regele și le-a luat de acolo, aruncând molozul lor în pârâul Chedronului.
\par 13 Apoi a spurcat regele cele două înălțimi din fața Ierusalimului, din dreapta Muntelui Măslinilor, pe care le făcuse Solomon, regele lui Israel, pentru Astarte, idolul Sidonului, pentru Chemoș, idolul Moabului, și pentru Milcom, idolul Amoniților.
\par 14 A sfărâmat stâlpii idolești și a tăiat Așerele și locul lor l-a umplut cu oase omenești.
\par 15 De asemenea și jertfelnicul cel din Betel și înălțimea făcută de Ieroboam, fiul lui Nabat, care a dus pe Israel în păcat, le-a stricat regele Iosia și a ars înălțimea aceasta și a făcut-o praf; și a mai ars și pe Așera.
\par 16 Și întorcând capul, Iosia a văzut mormintele ce erau acolo pe munte, și a trimis de a luat oasele din morminte și le-a ars pe jertfelnic și l-a spurcat după cuvântul Domnului, rostit de omul lui Dumnezeu care prezisese întâmplarea aceasta, când stătea Ieroboam în timpul unei sărbători înaintea jertfelnicului. După aceea, întorcându-se, Iosia a ridicat ochii la mormântul omului lui Dumnezeu care prezisese întâmplarea aceasta,
\par 17 Și a zis: "Ce mormânt este acesta, pe care-l văd eu?" Iar locuitorii cetății i-au răspuns: "Acesta este mormântul omului lui Dumnezeu care a venit din Iuda și a prevestit cele ce tu faci cu jertfelnicul din Betel".
\par 18 Iosia a zis: "Lăsați-l în pace; nimeni să nu atingă oasele lui". Și au păstrat oasele lui și oasele proorocului care venise din Samaria.
\par 19 De asemenea a dărâmat Iosia și toate capiștile înălțimilor din cetățile samarinene, pe care le făcuseră regii lui Israel, mâniind pe Domnul, și a făcut cu ele ceea ce făcuse și în Betel;
\par 20 Și a junghiat pe jertfelnice pe toți preoții înălțimilor care erau acolo; a ars oase omenești pe jertfelnice și apoi s-a întors în Ierusalim.
\par 21 După aceea a poruncit regele la tot poporul și a zis: "Săvârșiți Paștile Domnului Dumnezeului vostru, după cum este scris în această carte a legii!"
\par 22 Pentru că nu se mai săvârșise astfel de Paști din zilele Judecătorilor, care judecaseră pe Israel, în tot timpul regilor lui Iuda și al regilor lui Israel;
\par 23 Iar în anul al optsprezecelea al regelui Iosia s-au săvârșit aceste Paști ale Domnului, în Ierusalim.
\par 24 A mai nimicit regele Iosia pe cei ce se îndeletniceau cu chemarea morților, pe vrăjitori, pe terafimii, idolii și toate urâciunile care se iviseră în pământul lui Iuda și în Ierusalim, ca să împlinească cuvintele legii, scrise în cartea pe care o găsise Hilchia preotul în templul Domnului.
\par 25 Asemenea lui Iosia n-a mai fost rege înainte de el, care să se fi întors la Domnul cu toată inima sa, cu toate puterile sale și cu tot sufletul său, după toată legea lui Moise, dar nici după el nu s-a mai ridicat altul asemenea lui.
\par 26 Cu toate acestea Domnul n-a schimbat marea iuțime a mâniei Sale, cu care se aprinsese mânia Sa asupra lui Iuda, din pricina tuturor relelor pe care le făcuse Manase ca să-L mânie.
\par 27 Și a zis Domnul: "Și pe Iuda îl voi lepăda de la fața Mea, cum am lepădat pe Israel; și voi lepăda orașul acesta, Ierusalimul, pe care l-am ales și casa despre care am zis: "Acolo va fi numele Meu".
\par 28 Celelalte știri despre Iosia și despre toate câte a făcut el sunt scrise în cartea Cronicilor regilor lui Iuda.
\par 29 În zilele lui s-a dus Faraonul Neco, regele Egiptului, împotriva regelui Asiriei, la râul Eufratului. Atunci a ieșit Iosia în întâmpinarea lui, iar acela când l-a văzut l-a omorât în Meghido.
\par 30 Iar robii lui l-au luat mort din Meghido și l-au dus la Ierusalim de l-au îngropat în gropnița lui. Apoi a luat poporul țării pe Ioahaz, fiul lui Iosia, l-au uns și l-au făcut rege în locul tatălui lui.
\par 31 Ioahaz era de douăzeci și trei de ani când s-a făcut rege și a domnit în Ierusalim trei luni. Numele mamei lui era Hamutal, fiica lui Ieremia din Libna.
\par 32 Ioahaz a făcut lucruri netrebnice în ochii Domnului, întocmai cum făcuseră părinții lui.
\par 33 Dar l-a legat Faraonul Neco în Ribla, în țara Hamat, ca să nu mai domnească în Ierusalim, și a pus pe țară un bir de o sută de talanți de argint și o sută de talanți de aur.
\par 34 Apoi Faraonul Neco a pus rege pe Eliachim, fiul lui Iosia, în locul lui Iosia, tatăl lui, dar i-a schimbat numele în Ioiachim; pe Ioahaz l-a luat și l-a dus în Egipt, unde a murit.
\par 35 Ioiachim a dat lui Faraon aurul și argintul; și a prețuit el și pământul, ca să se dea argint după porunca lui Faraon; și a cerut la fiecare din poporul țării să aducă după prețuirea sa aur și argint ca să dea Faraonului Neco.
\par 36 Ioiachim însă era de douăzeci și cinci de ani când s-a făcut rege și a domnit în Ierusalim unsprezece ani. Numele mamei lui era Zebuda, fiica lui Pedaia, din Ruma.
\par 37 Acesta a făcut rele înaintea Domnului, cum făcuseră și părinții lui.

\chapter{24}

\par 1 În zilele lui Ioiachim a venit cu război Nabucodonosor, regele Babilonului, și Ioiachim a ajuns supusul lui timp de trei ani, dar apoi s-a răsculat împotriva lui.
\par 2 Atunci Domnul a trimis asupra lui cete de Caldei, cete de Sirieni, cete de Moabiți și cete de Amoniți; și le-a trimis asupra lui Iuda, ca să-l piardă, după cuvântul Domnului pe care l-a rostit prin robii Săi, proorocii.
\par 3 Aceasta s-a făcut cu Iuda numai din porunca Domnului, ca să fie lepădat de la fața Lui pentru păcatele lui Manase și pentru tot ce făcuse acesta;
\par 4 Și pentru sângele nevinovat pe care-l vărsase el, umplând tot Ierusalimul, Domnul n-a vrut să-l ierte.
\par 5 Celelalte știri despre Ioiachim și despre tot ce a făcut el sunt scrise în cartea Cronicilor regilor lui Iuda.
\par 6 Ioiachim a răposat cu părinții săi, iar în locul lui s-a făcut rege Iehonia, fiul său.
\par 7 Regele Egiptului n-a mai ieșit din țara sa, pentru că regele Babilonului a luat regelui Egiptului tot ce avea acesta de la râul Egiptului până la râul Eufratului.
\par 8 Iehonia era de optsprezece ani când s-a făcut rege și a domnit în Ierusalim trei luni. Numele mamei lui era Nehușta, fiica lui Elnatan, din Ierusalim.
\par 9 El a făcut lucruri netrebnice în ochii Domnului, în toate, așa cum făcuse tatăl său.
\par 10 În vremea aceea slujitorii lui Nabucodonosor, regele Babilonului, au venit asupra Ierusalimului și au împresurat cetatea. Iar după ce slugile lui au înconjurat cetatea, a venit și regele Nabucodonosor.
\par 11 Și a ieșit Iehonia, regele lui Iuda, la regele Babilonului, împreună cu mama sa, cu slujitorii săi, cu căpeteniile și eunucii lui,
\par 12 Și l-a luat rob regele Babilonului în al optulea an al domniei sale;
\par 13 Și a scos toate comorile templului Domnului și comorile casei domnești, și a sfărâmat, după cum spusese Domnul, toate vasele cele de aur pe care le făcuse Solomon, regele lui Israel, pentru templul Domnului, și ța dus în robie tot Ierusalimul,
\par 14 Pe toți fruntașii și pe toți oamenii viteji, aproape zece mii de robi, cu toți dulgherii și fierarii, și n-a rămas nimeni decât numai poporul țării cel sărac.
\par 15 Și a dus și pe Iehonia la Babilon; de asemenea au dus robi din Ierusalim la Babilon pe mama și femeile regelui, pe eunucii lui și pe puternicii țării;
\par 16 Toată oștirea în număr de șapte mii, teslarii, fierarii în număr de o mie, și toți oamenii vârstnici și buni de oștire i-a dus regele Babilonului robi la Babilon.
\par 17 Atunci regele Babilonului a pus rege în locul lui Iehonia, pe Matania, unchiul lui Iehonia, schimbându-i numele în Sedechia.
\par 18 Sedechia era de douăzeci și unu de ani când s-a făcut rege și a domnit în Ierusalim unsprezece ani. Numele mamei lui era Hamutal, fiica lui Ieremia din Libna.
\par 19 Și a făcut și acesta lucruri netrebnice în ochii Domnului, în toate, așa cum făcuse și Iehonia.
\par 20 Și mânia Domnului era peste Ierusalim și peste Iuda atât de mare, încât i-a lepădat de la fața Sa. Și s-a lepădat și Sedechia de regele Babilonului.

\chapter{25}

\par 1 Iar în anul al nouălea al domniei lui Sedechia, în luna a zecea, în ziua a zecea a lunii, a venit Nabucodonosor, regele Babilonului, cu toată oștirea sa asupra Ierusalimului și l-a împresurat și a făcut împrejurul lui întărituri.
\par 2 Și a stat cetatea împresurată până în anul al unsprezecelea al regelui Sedechia.
\par 3 Iar în anul al unsprszecelea al regelui Sedechia, în ziua a noua a lunii a patra, era mare foamete în cetate și poporul țării nu mai avea pâine.
\par 4 Atunci cetatea a fost luată și toți ostașii au fugit noaptea pe calea porților care se aflau între două ziduri, lângă grădina regelui. Caldeii însă stăteau împrejurul cetății; și a ieșit și regele pe calea ce duce în câmpie.
\par 5 Dar a alergat oștirea Caldeilor după rege și l-au ajuns în șesul Ierihonului; iar oștirea lui a fugit toată de la el.
\par 6 Și au luat pe rege și l-au dus la regele Babilonului, în Ribla, și l-au supus judecății.
\par 7 Și au junghiat pe fiii regelui înaintea ochilor lui, iar lui Sedechia i-au scos ochii, l-au pus în lanțuri și l-au dus la Babilon.
\par 8 Iar în luna a cincea, în ziua a șaptea a lunii, adică în anul al nouăsprezecelea al lui Nabucodonosor, regele Babilonului, Nebuzaradan, căpetenia gărzii, slujitorul regelui Babilonului, a venit la Ierusalim
\par 9 Și a ars templul Domnului, casa regelui și toate casele din Ierusalim; toate casele cele mari le-a ars cu foc.
\par 10 Iar oștirea Caldeilor care era cu comandantul gărzii a dărâmat și zidurile cele dimprejurul Ierusalimului.
\par 11 Apoi Nebuzaradan, căpetenia gărzii, a strămutat la Babilon și celălalt popor care mai rămăsese în Ierusalim, pe cei ce se predaseră regelui Babilonului și rămășița poporului de rând.
\par 12 Numai puțini din poporul sărac al țării au fost lăsați de căpetenia gărzii, să lucreze viile și ogoarele.
\par 13 Caldeii au stricat și stâlpii cei de aramă care erau în templul Domnului, postamentele, marea cea de aramă din templul Domnului și arama lor au dus-o în Babilon.
\par 14 Căldările, lopețile, cuțitele, lingurile și toate vasele de aramă, care se întrebuințau la slujbă, le-au luat.
\par 15 Și a mai luat căpetenia gărzii cădelnițele, cupele și tot ce era de aur și ce era de argint.
\par 16 Au luat cei doi stâlpi, marea și postamentele pe care le făcuse Solomon pentru templul Domnului. Arama din toate aceste lucruri nu se mai putea cântări.
\par 17 Un singur stâlp era înalt de optsprezece coți; coroana lui era de aramă și înălțimea ei era de trei coți; pereții ei și împletiturile dimprejurul ei, toate erau de aramă. Asemenea era și al doilea stâlp cu coroana lui.
\par 18 Apoi a mai luat căpetenia gărzii pe Seraia arhiereul, pe Țefania, al doilea preot și pe alți trei care stăteau de strajă la prag.
\par 19 Iar din cetate a luat un eunuc, care era căpetenie peste oșteni și cinci oameni, care stăteau înaintea regelui și care acum se aflau în cetate; pe căpetenia cea mare a oștirii, care înscria la oaste poporul și șaizeci de oameni din poporul țării, care se aflau în cetate.
\par 20 Pe aceștia i-a luat Nebuzaradan, căpetenia gărzii și i-a dus la regele Babilonului, în Ribla.
\par 21 Iar regele Babilonului i-a lovit și i-a ucis la Ribla, în ținutul Hamat. Așa a fost strămutat Iuda din țara lui.
\par 22 Iar peste poporul care a rămas în țara Iudei și pe care l-a lăsat Nabucodonosor, regele Babilonului, a pus căpetenie pe Ghedalia, fiul lui Ahican, fiul lui Șafan.
\par 23 Când au auzit toate căpeteniile și ostașii lor că regele Babilonului a pus căpetenie pe Ghedalia, au venit la Ghedalia în Mițpa: Ismael, fiul lui Netania, Iohanan, fiul lui Careah, Seraia, fiul lui Tanhumet, din Netofa, și Iaazania, fiul lui Maacati, ei împreună cu oamenii lor.
\par 24 Și a jurat Ghedalia acestora și oamenilor lor și le-a zis: "Nu vă temeri a fi supușii Caldeilor. Așezați-vă în țara aceasta și slujiți regelui Babilonului și va fi bine!"
\par 25 Dar în luna a șaptea a venit Ismael, fiul lui Netania al lui Elișama din neamul regesc, cu zece oameni și a lovit pe Ghedalia și acesta a murit și a lovit și pe Iudeii Și pe Caldeii care erau cu el în Mițpa.
\par 26 Atunci s-a sculat tot poporul, de la mic până la mare, cu căpeteniile oștirii și s-au dus în, Egipt, pentru că se temeau de Caldei.
\par 27 În anul al treizeci și șaptelea de la strămutarea lui Ioiachim, regele lui Iuda, în luna a douăsprezecea, în ziua a douăzeci și șaptea a acestei luni, Evil-Merodac, regele Babilonului, în anul urcării sale pe tron, a scos pe Ioiachim, regele Iudei, din temniță,
\par 28 A vorbit cu el prietenos și a pus tronul lui mai sus de tronurile regilor care erau la el în Babilon;
\par 29 I-a schimbat hainele lui de temniță și Ioiachim a mâncat totdeauna la masa regelui, în toate zilele vierii lui.
\par 30 Cele trebuitoare hranei lui i-au fost date neîncetat de rege, zi cu zi, cât a trăit el.


\end{document}