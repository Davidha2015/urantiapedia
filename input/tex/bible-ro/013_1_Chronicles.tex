\begin{document}

\title{1 Cronici}


\chapter{1}

\par 1 Adam, Set, Enos;
\par 2 Chenan, Mahalaleel, Iared;
\par 3 Enoh, Matusalem, Lameh;
\par 4 Noe, Sem, Ham ?i Iafet.
\par 5 Fiii lui Iafet: Gomer, Magog, Madai, Iavan, Eli?a, Tubal, Me?ec ?i Tiras.
\par 6 Fiii lui Gomer: A?chenaz, Rifat ?i Togarma.
\par 7 Fiii lui Iavan: Eli?a, Tar?i?, Chitim ?i Dodanim.
\par 8 Fiii lui Ham: Cu?, Mi?raim, Put ?i Canaan.
\par 9 Fiii lui Cu?: Seba, Havila, Savta Rama ?i Sabteca. Fiii lui Rama: ?eba ?i Dedan.
\par 10 Lui Cu? i s-a mai nascut de asemenea ?i Nimrod. Acesta a început sa fie puternic pe pamânt.
\par 11 Lui Mi?raim i s-a nascut: Ludim, Anamim, Lehabim, Naftuhim,
\par 12 Patrusim, Casluhim, din care se trag Filistenii ?i Caftorim.
\par 13 Lui Canaan i s-au nascut: Sidon, întâiul sau nascut ?i Het,
\par 14 Iebuseu, Amoreu, Ghergheseu,
\par 15 Heveu, Archeu, Sineu,
\par 16 Arvadeu, ?emareu ?i Hamateu.
\par 17 Fiii lui Sem: Elam, Asur, Arpaxad, Lud ?i Aram. Fiii lui Aram: Ut, Hul, Gheter ?i Me?ec.
\par 18 Lui Arpaxad i s-a nascut Cainan, lui Cainan i s-a nascut ?elah, lui ?elah i s-a nascut Eber.
\par 19 Lui Eber i s-au nascut doi fii: numele unuia era Peleg, pentru ca în zilele lui s-a împar?it ?ara; iar numele fratelui sau era Ioctan.
\par 20 Lui Ioctan i s-au nascut: Almodad, ?elef, Ha?armavet, Iarah,
\par 21 Hadoram, Uzal, Dicla,
\par 22 Ebal, Abimael, ?eba,
\par 23 Ofir, Havila ?i Iobab. To?i ace?tia sunt fiii lui Ioctan.
\par 24 Iar fiii lui Sim sunt: Arpaxad, Cainan, Selah,
\par 25 Eber, Peleg, Reu,
\par 26 Serug, Nahor, Terah
\par 27 ?i Avram, adica Avraam.
\par 28 Fiii lui Avraam sunt Isaac ?i Ismael.
\par 29 Iata spi?a neamului lor: Nebaiot, întâiul nascut al lui Ismael, apoi: Chedar, Adbeel, Mibsam,
\par 30 Mi?ma, Duma, Ma?a, Hadad, Tema,
\par 31 Ietur, Nafi? ?i Chedma. Ace?tia sunt fiii lui Ismael.
\par 32 Fiii Cheturei, ?iitoarea lui Avraam. Ea a nascut pe Zimran, Ioc?an, Medan, Madian, I?bac ?i ?uah. Fiii lui Ioc?an sunt: ?eba ?i Dedan. Fiii lui Dedan sunt: Raguel, Navdeel, A?urim, Letu?im ?i Leumim.
\par 33 Fiii lui Madian sunt: Efa, Efer, Enoh, Abida ?i Eldaa. To?i ace?tia sunt fiii Cheturei.
\par 34 Lui Avraam i s-a nascut Isaac. Fiii lui Isaac sunt: Isav ?i Israel.
\par 35 Fiii lui Isav sunt: Elifaz, Raguel, Ieu?, Ialam ?i Core.
\par 36 Fiii lui Elifaz sunt: Teman, Omar, ?efi, Gatam, Chenaz; iar Temna, concubina lui Elifaz, i-a nascut pe Amalec.
\par 37 Fiii lui Raguel sunt: Nahat, Zerah, ?ama ?i Miza.
\par 38 Fiii lui Seir sunt: Lotan, ?obal, ?ibeon, Ana, Di?on, E?er ?i Di?an.
\par 39 Fiii lui Lotan sunt: Hori ?i Heman; iar sora lui Lotan se numea Timna.
\par 40 Fiii lui ?obal sunt: Alvan, Manahat, Ebal, ?efo ?i Onam. Fiii lui ?ibeon sunt: Aia ?i Ana.
\par 41 Fiii lui Ana sunt: Di?on ?i Olibama; fiii lui Di?on sunt: Hemdan, E?ban, Itran ?i Cheran.
\par 42 Fiii lui E?er sunt: Bilhan, Zaavan ?i Acan; fiii lui Di?an sunt: U? ?i Aran.
\par 43 Ace?tia sunt regii care au domnit în pamântul Edom, înainte de a se ridica rege, peste fiii lui Israel, Bela, fiul lui Beor, cetatea caruia se numea Dinhaba.
\par 44 Murind Bela, dupa el a fost facut rege Iobab, fiul lui Zerah din Bo?ra.
\par 45 Dupa moartea lui Iobab s-a facut rege Hu?am, în ?ara Temani?ilor.
\par 46 Murind Hu?am, s-a facut rege dupa el Hadad, fiul lui Bedad, care a lovit pe Madiani?i în câmpia Moabului. Ora?ul lui se numea Avit.
\par 47 Murind Hadad, s-a facut rege dupa el ?amla, din Masreca;
\par 48 Murind ?amla, s-a facut rege dupa el ?aul, din Rehobotul cel de lânga râu.
\par 49 Murind ?aul, s-a facut rege dupa el Baal-Hanan, fiul lui Acbor.
\par 50 Murind Baal-Hanan, s-a facut rege dupa el Hadad. Numele ceta?ii lui era Pau, iar numele femeii lui era Mehetabeel, fiica lui Matred, fiica lui Mezahab.
\par 51 Murind Hadad, au urmat capetenii peste Edom: capetenia Timna, capetenia Alia, capetenia Ietet,
\par 52 Capetenia Oholibama, capetenia Ela, capetenia Pinon,
\par 53 Capetenia Chenaz, capetenia Teman, capetenia Mib?ar,
\par 54 Capetenia Magdiel, capetenia Iram. Acestea sunt capeteniile Edomului.

\chapter{2}

\par 1 Iata acum fiii lui Israel: Ruben, Simeon, Levi, Iuda, Isahar, Zabulon,
\par 2 Dan, Iosif, Veniamin, Neftali, Gad ?i A?er.
\par 3 Fiii lui Iuda sunt: Ir, Onan ?i ?ela. Ace?ti trei i s-au nascut lui din fata unui canaanit anume ?ua. Ir, întâiul nascut al lui Iuda, a fost rau în ochii Domnului ?i l-a omorât.
\par 4 Tamara, nora lui Iuda, i-a nascut acestuia pe Fares ?i pe Zara. A?a ca, de to?ii, fiii lui Iuda au fost cinci.
\par 5 Fiii lui Fares sunt He?ron ?i Hamul.
\par 6 Fiii lui Zara sunt: Zimri, Etan, Heman, Calcol ?i Darda; cinci de to?i.
\par 7 Fiul lui Carmi este Acar, care a adus nenorocire asupra lui Israel, calcând juramântul.
\par 8 Fiul lui Etan este Azaria.
\par 9 Fiii lui He?ron care i s-a nascut sunt: Ierahmeel, Ram ?i Chelubai (Caleb).
\par 10 Lui Ram însa i s-a nascut Aminadab; lui Aminadab i s-a nascut Naason, capetenia fiilor lui Iuda.
\par 11 Lui Naason i s-a nascut Salmon, lui Salmon i s-a nascut Booz.
\par 12 Lui Booz i s-a nascut Obed, lui Obed i s-a nascut Iesei.
\par 13 Lui Iesei i s-a nascut Eliab, întâiul sau nascut, apoi al doilea, Aminadab, al treilea, ?ama,
\par 14 Al patrulea, Natanael, al cincilea, Radai,
\par 15 Al ?aselea, O?em ?i al ?aptelea, David.
\par 16 Surorile lor au fost ?eruia ?i Abigail. Fiii ?eruiei au fost trei: Abi?ai, Ioab ?i Asael.
\par 17 Abigail a nascut pe Amasa; iar tatal lui Amasa este Ieter Ismaelitul.
\par 18 Caleb, fiul lui He?ron, a avut de la Azuba, femeia sa, ?i de la Ieriot urmatorii copii: Ie?er, ?obab ?i Ardon.
\par 19 Murind însa Azuba, Caleb ?i-a luat de femeie pe Efrata ?i aceasta i-a nascut pe Hur.
\par 20 Lui Hur i s-a nascut Urie; lui Urie i s-a nascut Be?aleel.
\par 21 Dupa aceea He?ron a intrat la fata lui Machir, tatal lui Galaad; ?i a luat-o, fiind de ?aizeci de ani ?i ea i-a nascut fiu pe Segub.
\par 22 Lui Segub i s-a nascut Iair ?i avea el atunci douazeci ?i trei de ceta?i în pamântul Galaadului.
\par 23 Dar Ghe?urenii ?i Sirienii le-au luat sala?urile lui Iair cu Chenatul ?i ceta?ile care ?ineau de el, în numar de ?aizeci. Toate aceste ceta?i erau ale fiilor lui Machir, tatal lui Galaad.
\par 24 Dupa ce a murit He?ron, Caleb a intrat la Efrata, femeia lui He?ron, tatal sau, care a nascut pe A?ur, tatal lui Tecoa.
\par 25 Fiii lui Ierahmeel, întâiul nascut al lui He?ron, sunt: întâiul nascut Ram, dupa el Vuna, Oren, O?em ?i Ahia.
\par 26 Ierahmeel a mai avut ?i alta femeie, cu numele Atara; aceasta este mama lui Onan.
\par 27 Fiii lui Ram, întâiul nascut al lui Ierahmeel, sunt: Maa?, Iamin ?i Echer.
\par 28 Fiii lui Onan au fost: ?amai ?i Iada. Fiii lui ?amai au fost: Nadab ?i Abi?ur.
\par 29 Numele femeii lui Abi?ur era Abihail ?i aceasta i-a nascut pe Ahban ?i pe Molid.
\par 30 Fiii lui Nadab au fost: Seled ?i Efraim. Dar Seled a murit fara copii.
\par 31 Fiul lui Efraim a fost I?ei, iar fiul lui I?ei a fost ?e?an; iar fiul lui ?e?an a fost Ahlai.
\par 32 Fiii lui Iada, fratele lui ?amai, au fost Ieter ?i Ionatan. Ieter a murit fara copii.
\par 33 Fiii lui Ionatan au fost: Pelet ?i Zaza. Ace?tia sunt fiii lui Ierahmeel.
\par 34 ?e?an n-a avut fii, ci numai fiice. ?e?an avea un rob egiptean, cu numele Iarha.
\par 35 ?e?an a dat pe o fata a sa lui Iarha, robul sau, de femeie ?i ea a nascut pe Atai.
\par 36 Atai a avut de fiu pe Natan, iar lui Natan i s-a nascut Zabad.
\par 37 Lui Zabad i s-a nascut Eflal, iar lui Eflal i s-a nascut Obed.
\par 38 Lui Obed i s-a nascut Iehu, iar lui Iehu i s-a nascut Azaria.
\par 39 Lui Azaria i s-a nascut Hele?, iar lui Hele? i s-a nascut Eleasa.
\par 40 Lui Eleasa i s-a nascut Sismai, iar lui Sismai i s-a nascut ?alum.
\par 41 ?alum a avut de fiu pe Iecamia, iar Iecamia pe Eli?ama.
\par 42 Fiul lui Caleb, fratele lui Ierahmeel, era Me?a, întâiul sau nascut, tatal lui Zif. Acesta a avut ca fiu pe Mare?a, tatal lui Hebron.
\par 43 Fiii lui Hebron sunt: Core, Tapuah, Rechem ?i ?ema.
\par 44 Lui ?ema i s-a nascut Raham tatal lui Iorchean, iar lui Rechem i s-a nascut ?amai.
\par 45 Fiul lui ?amai a fost Maon, iar Maon este tatal lui Bet-?ur.
\par 46 ?i Efa, concubina lui Caleb, a nascut pe Haran, Mo?a ?i Gazez; iar Haran a fost tatal lui Gazez.
\par 47 Fiii lui Iahdai sunt: Reghem, Iotan, Ghe?an, Pelet, Efa ?i ?aaf.
\par 48 Concubina lui Caleb, Maaca, a nascut pe ?eber ?i pe Tirhana;
\par 49 Tot ea a nascut pe ?aaf, tatal Madmanei, pe ?eva, tatal Macbenei ?i tatal Ghibeii. Fiica lui Caleb este Acsa.
\par 50 Ace?tia au fost fiii lui Caleb. Fiul lui Hur, întâiul nascut al Efratei a fost ?obal, tatal lui Chiriat-Iearim;
\par 51 Salma, tatal lui Betleem; Haref, tatal lui Betgader.
\par 52 ?obal, tatal Chiriat-Iearimului a avut fii pe Haroe, Ha?i ?i Hamenuhot.
\par 53 Familiile Chiriat-Iearimului sunt: Itrienii, Putienii, ?umatienii ?i Mi?raenii. Din acestea se trag ?oreenii ?i E?tauleenii.
\par 54 Fiii lui Salma sunt: Betleem, Netofati?ii, Atrot-Bet-Ioab, jumatate din Manahteni, ?oareni,
\par 55 Familiile Soferi?ilor, care traiau în Iabe?, Tirati?ii, ?imati?ii, Sucati?ii. Ace?tia sunt Chineenii, care se trag din Hamat, tatal casei lui Recab.

\chapter{3}

\par 1 Fiii lui David, care i s-au nascut în Hebron, au fost: întâiul nascut Amnon din Ahinoama Izreeliteanca; al doilea, Daniel, din Abigail Carmeliteanca.
\par 2 Al treilea, Abesalom, fiul Maacai, fata lui Talmai, regele din Ghe?ur; al patrulea, Adonia, fiul Haghitei;
\par 3 Al cincilea, ?efatia din Abitala; al ?aselea, Itrean din Egla, femeia sa.
\par 4 Ace?ti ?ase i s-au nascut în Hebron. În Hebron David a domnit ?apte ani ?i ?ase luni, iar în Ierusalim a domnit treizeci ?i trei de ani.
\par 5 Iata ?i cei ce i s-au nascut în Ierusalim: ?imea, ?obab, Natan ?i Solomon, patru, din Bat?eba, fiica lui Amiel.
\par 6 Ibhar, Eli?ama, Elifelet,
\par 7 Nogah, Nefeg, Iafia,
\par 8 Eli?ama, Eliada ?i Elifelet; în total noua.
\par 9 Ace?tia sunt to?i fiii lui David, afara de cei de la ?iitoare. Iar sora lor era Tamara.
\par 10 Fiul lui Solomon este Roboam; fiul acestuia este Abia, iar al acestuia, Asa, iar al lui Asa este Iosafat.
\par 11 Fiul acestuia este Ioram, al acestuia este Ahazia ?i al acestuia este Ioa?.
\par 12 Fiul lui este Amasia, al acestuia este Azaria, iar al acestuia este Ioatam.
\par 13 Fiul acestuia este Ahaz, al acestuia este Iezechia, iar al acestuia este Manase;
\par 14 Fiul acestuia este Amon, iar al acestuia este Iosia.
\par 15 Fiii lui Iosia au fost: întâiul nascut Iohanan, al doilea Ioiachim, al treilea Sedechia ?i al patrulea ?alum.
\par 16 Fiii lui Ioiachim au fost: Iehonia, fiul lui; Sedechia, fiul lui.
\par 17 Fiii lui Iehonia, cel dus în robie, au fost: Salatiel,
\par 18 Malchiram, Pedaia, ?ena?ar, Iecamia, Ho?ama ?i Nedabia.
\par 19 Iar fiii lui Pedaia au fost: Zorobabel ?i ?imei. Iar fiii lui Zorobabel au fost: Me?ulam ?i Hanania, ?i sora lor ?elomit.
\par 20 Fiii lui Me?ulam: Ha?uba, Ohel, Berechia, Hasadia ?i Iu?ab-Hesed.
\par 21 Fiii lui Hanania au fost Pelatia ?i Isaia; fiul acestuia a fost Refaia, al acestuia a fost Arnan, al acestuia a fost Obadia, iar al acestuia ?ecania.
\par 22 Fiii lui ?ecania au fost ?ase: ?emaia, Hatu?, Igheal, Bariah, Nearia ?i ?afat.
\par 23 Fiii lui Nearia au fost trei: Elioenai, Iezechia ?i Azricam.
\par 24 Fiii lui Elioenai au fost ?apte: Hodavia, Elia?ib, Pelaia, Acub, Iohanan, Delaia ?i Anani.

\chapter{4}

\par 1 Fiii lui Iuda au fost: Fares, He?ron, Carmi, Hur ?i ?obal.
\par 2 Reaia, fiul lui ?obal, a avut fiu pe Iahat; lui Iahat i s-a nascut Ahumai ?i Lahad. Din el se trag familiile ?oreenilor.
\par 3 Fiii lui Etam sunt: Izreel, I?ma ?i Idba?, ?i sora lor cu numele Ha?lelponi.
\par 4 Panuel, tatal lui Ghedor ?i Ezer, tatal lui Hu?a sunt fiii lui Hur, întâiul nascut din Efrata ?i tatal lui Betleem.
\par 5 A?hur, tatal lui Tecoa, a avut doua femei: pe Helea ?i Naara.
\par 6 Naara i-a nascut pe Ahuzam, Hefer, Temni ?i Aha?tari. Ace?tia sunt fiii Naarei.
\par 7 Iar fiii Helei sunt: ?eret, ?ohar Etna ?i Co?.
\par 8 Lui Co? i s-au nascut: Anub, ?obeba, Iahe? ?i familiile lui Aharhel, fiul lui Harum.
\par 9 Iabe? a fost mai însemnat decât fra?ii sai. Mama lui i-a dat numele de Iabe?, zicând: "Cu durere l-am nascut".
\par 10 ?i a strigat Iabe? catre Dumnezeul lui Israel ?i a zis: "O, de m-ai binecuvânta Tu cu binecuvântare, de ai largi hotarele mele ?i de ar fi mâna Ta cu mine, pazindu-ma de rele, ca sa nu fiu omorât!..." Atunci Dumnezeu i-a trimis ceea ce a dorit el.
\par 11 ?i lui Chelub, fratele lui ?uha, i s-a nascut Mehir. Acesta e tatal lui E?ton.
\par 12 Lui E?ton i s-au nascut Bet-Rafa, Paseah ?i Techina, tatal ceta?ii Naha?; ace?tia sunt locuitorii din Recab.
\par 13 Fiii lui Chenaz sunt Otniel ?i Seraia. Fiii lui Otniel au fost Hatat ?i Meonotai.
\par 14 Lui Meonotai i s-a nascut Ofra. Lui Seraia i s-a nascut Ioab, stramo?ul lui Ghehara?im, numi?i a?a pentru ca ei erau dulgheri.
\par 15 Fiii lui Caleb, fiul lui Iefoni, au fost: Ir, Ela ?i Naam. Fiul lui Ela a fost Chenaz.
\par 16 Fiii lui Iehaleleel au fost: Zif, Zifa, Tiria ?i Asareel.
\par 17 Fiii lui Ezra sunt: Ieter, Mered, Efer ?i Ialon; iar lui Ieter i s-au nascut Miriam, ?amai ?i I?bah, tatal lui E?temoa.
\par 18 Femeia acestuia, Iehudia, a nascut pe Iered, tatal lui Ghedor, pe Heber, tatal lui Soco, ?i pe Iecutiel, tatal lui Zanoah. Ace?tia sunt fiii Bitiei, fata lui Faraon, pe care a luat-o Mered.
\par 19 Fiii femeii acestuia, Hodia, sora lui Naham, tatal Cheilei, sunt: Garmi ?i E?temoa Maacateanul.
\par 20 Fiii lui Simeon sunt: Amnon, Rina, Benhanan ?i Tilon. Fiii lui I?i sunt Zohet ?i Benzohet.
\par 21 Fiii lui ?ela, fiul lui Iuda, sunt: Er, tatal lui Leca, Laeda, tatal lui Mare?a, ?i familiile din casa lui A?beia, care lucrau visonul,
\par 22 Iochim ?i locuitorii din Cozeba; Ioa? ?i Saraf, care au stapânit asupra Moabului ?i Ia?ubi-Lehem. Dar acestea sunt întâmplari mai vechi.
\par 23 Ace?tia erau olari ?i traiau la gradini ?i la livezi ?i prin ceta?i; ei traiau acolo la rege ca sa-i lucreze lui.
\par 24 Fiii lui Simeon au fost: Nemuel, Iamin, Iarib, Zerah ?i Saul.
\par 25 Fiul lui Saul a fost ?alum, fiul acestuia a fost Mibsam, iar al acestuia a fost Mi?ma.
\par 26 Fiii lui Mi?ma au fost: Hamuel, fiul lui; fiul acestuia a fost Zacur, iar al acestuia a fost ?imei.
\par 27 ?imei a avut ?aisprezece fii ?i ?ase fete, iar fra?ii lui au avut pu?ini copii ?i tot neamul lor n-a fost a?a de numeros ca neamul fiilor lui Iuda.
\par 28 Ei traiau în Beer-?eba, Molada ?i Ha?ar-?ual,
\par 29 În Bilha, E?em, Tolad,
\par 30 Betuel, Horma, ?iclag,
\par 31 În Bet-Marcabot, Ha?ar-Susim, Bet-Birei ?i ?aaraim. Iata ceta?ile lor dinainte de domnia lui David cu satele lor.
\par 32 ?i mai aveau: Etam, Ain, Rimon, Tochen ?i A?an, cinci ceta?i,
\par 33 Cu toate satele lor, care se aflau împrejurul acestor ceta?i pâna la Baal. Iata locurile lor de locuin?a ?i spi?a neamului lor:
\par 34 Me?obab, Iamlec ?i Io?a, fiul lui Amasia
\par 35 Ioil ?i Iehu, fiul lui Io?ibia, fiul lui Seraia, fiul lui Asiel;
\par 36 Elioenai, Iaacoba, Ie?ohaia, Asaia, Adiel, Ie?imiel, Benaia,
\par 37 Ziza, fiul lui ?ifei, fiul lui Alon, fiul lui Iedaia, fiul lui ?imri, fiul lui ?emaia.
\par 38 Ace?ti numi?i mai sus au fost capetenii neamurilor lor, iar casa tatalui lor s-a împar?it în multe ramuri.
\par 39 Ei s-au întins pâna în partea Gherarei ?i pâna în partea de rasarit a vaii Gai, ca sa gaseasca pa?uni pentru turmele lor;
\par 40 ?i au gasit pa?uni grase ?i bune ?i pamânt larg, lini?tit ?i lipsit de primejdii, pentru ca înainte de ei au trait acolo numai pu?ini Hami?i.
\par 41 ?i au venit ace?tia, care sunt scri?i pe nume, în zilele lui Iezechia, regele Iudei, ?i au batut pe nomazi ?i pe cei a?eza?i, care se aflau acolo ?i i-au nimicit pentru totdeauna ?i s-au a?ezat în locul lor, caci acolo se aflau pa?uni pentru turmele lor.
\par 42 Dar din ei, din fiii lui Simeon, s-au dus catre muntele Seir cinci sute de oameni, în frunte cu Pelatia, Nearia, Refaia ?i Uziel, fiii lui I?ei,
\par 43 ?i au batut rama?i?a de Amaleci?i, ce se mai gasea acolo, ?i traiesc acolo pâna în ziua de astazi.

\chapter{5}

\par 1 Fiii lui Ruben, întâiul nascut al lui Israel, caci el a fost nascut întâi, dar pentru ca a întinat el patul tatalui sau, întâietatea lui a fost data fiilor lui Iosif, fiul lui Israel, ca sa nu se mai înscrie ei (fiii lui Ruben) ca întâi nascu?i;
\par 2 Caci Iuda era cel mai puternic dintre fra?ii sai ?i pova?uitorul e din el, dar întâietatea a trecut la Iosif.
\par 3 Fiii lui Ruben, întâiul nascut al lui Israel, au fost: Enoh, Palu, He?ron ?i Carmi.
\par 4 Fiii lui Ioil au fost: ?emaia, fiul lui, fiul acestuia a fost Gog, iar al acestuia a fost ?imei;
\par 5 Fiul acestuia a fost Mihea, al acestuia a fost Reaia, iar al acestuia a fost Baal;
\par 6 Fiul acestuia a fost Beera, pe care l-a dus în robie Tiglatfalasar, regele Asiriei. El era capetenia Rubeni?ilor.
\par 7 ?i fra?ii lui, dupa familiile lor, dupa spira neamului lor, au fost: cel mai însemnat Ioil, apoi Zaharia,
\par 8 ?i Bela, fiul lui Azaz, fiul lui ?ema, fiul lui Ioil. El locuia în Aroer pâna la Nebo ?i Baal-Meon.
\par 9 Iar spre rasarit a locuit el pâna la marginea pustiului, care pleaca de la râul Eufrat, pentru ca turmele lui erau foarte multe în ?inutul Galaadului.
\par 10 În zilele lui Saul au purtat ei razboi cu Agarenii, care au cazut în mâinile lor, ?i au locuit în corturi, în toata latura de rasarit a Galaadului.
\par 11 Fiii lui Gad traiau în fa?a lor, în ?ara Vasanului, pâna la Salca.
\par 12 În Vasan, cel mai de seama era Ioil. ?afam era al doilea, apoi venea Iaenai ?i ?afat.
\par 13 Fra?ii lor cu familiile lor erau în numar de ?apte: Micael, Me?ulam, ?eba, Iorai, Iacan, Zia ?i Eber.
\par 14 Iata fiii lui Abihail, fiul lui Huri, fiul lui Iaroah, fiul lui Galaad, fiul lui Micael, fiul lui Ie?i?ai, fiul lui Iahdo, fiul lui Buz.
\par 15 Ahi, fiul lui Abdiel, fiul lui Guni, era capul neamului sau.
\par 16 Ei traiau în Galaad, în Vasan ?i în ceta?ile care ?ineau de el în toate împrejurimile ?irionului, pâna la capatul lor.
\par 17 Ei cu to?ii au fost numara?i în zilele lui Ioatam, regele Iudei, ?i în zilele lui Ieroboam, regele lui Israel.
\par 18 Urma?ii lui Ruben, ai lui Gad ?i o jumatate din semin?ia lui Manase aveau oameni razboinici, barba?i care purtau scut ?i sabie, care trageau cu arcul ?i deprin?i la lupta, patruzeci ?i patru de mii ?apte sute ?aizeci care ie?eau la razboi.
\par 19 ?i s-au luptat ei cu Agarenii, cu Ietur, cu Nafis ?i cu Nodab.
\par 20 Dar li s-a dat ajutor contra acelora ?i au fost da?i în mâna lor Agarenii ?i toate ale lor, pentru ca ei în vremea luptei au strigat catre Dumnezeu ?i El i-a auzit, pentru ca ei nadajduiau în El.
\par 21 Atunci au luat ei turmele acelora: cincizeci de mii de camile, doua sute cincizeci de mii de oi ?i capre, doua mii de asini ?i o suta de mii de oameni,
\par 22 Pentru ca mul?i au cazut uci?i, caci lupta aceasta a fost de la Dumnezeu. ?i au locuit ei în locul acelora pâna la ducerea în robie.
\par 23 Urma?ii jumata?ii din tribul lui Manase au trait în acel pamânt de la Vasan pâna la Baal-Ermon ?i Senir ?i pâna la muntele Hermon, ?i erau mul?i la numar.
\par 24 Iata capii de familie cei mai însemna?i ai lor: Efer, I?ei, Eliel, Azriel, Ieremia, Hodavia ?i Iahdiel, barba?i puternici, barba?i vesti?i, capetenii ale neamurilor lor.
\par 25 Dar când ei au gre?it împotriva Dumnezeului parin?ilor lor ?i au început sa se desfrâneze dupa dumnezeii popoarelor pamântului aceluia, pe care le stârpise Dumnezeu de la fa?a lor,
\par 26 Atunci Dumnezeul lui Israel a întarâtat duhul lui Ful, regele Asiriei, adica al lui Tiglatfalasar, regele Asiriei ?i acesta a stramutat pe Rubeni?i ?i pe Gadi?i ?i jumatate din tribul lui Manase ?i i-a dus în Halach, Habor, Hara ?i la râul Gozan, unde sunt pâna astazi.

\chapter{6}

\par 1 Fiii lui Levi sunt: Gher?om, Cahat ?i Merari.
\par 2 Fiii lui Cahat sunt: Amram, I?har, Hebron ?i Uziel.
\par 3 Copiii lui Amram sunt: Aaron, Moise ?i Mariam. Fiii lui Aaron sunt: Nadab, Abiud, Eleazar ?i Itamar.
\par 4 Lui Eleazar i s-a nascut Finees, lui Finees i s-a nascut Abi?ua;
\par 5 Lui Abi?ua i s-a nascut Buchi, iar lui Buchi i s-a nascut Uzi;
\par 6 Lui Uzi i s-a nascut Zerahia, iar lui Zerahia i s-a nascut Meraiot.
\par 7 Lui Meraiot i s-a nascut Amaria, iar lui Amaria i s-a nascut Ahitub;
\par 8 Lui Ahitub i s-a nascut ?adoc, iar lui ?adoc i s-a nascut Ahimaa?;
\par 9 Lui Ahimaa? i s-a nascut Azaria, iar lui Azaria i s-a nascut Iohanan;
\par 10 Lui Iohanan. i s-a nascut Azaria; acesta e acela care a fost preot la templul zidit de Solomon în Ierusalim.
\par 11 Lui Azaria i s-a nascut Amaria, iar lui Amaria i s-a nascut Ahitub;
\par 12 Lui Ahitub i s-a nascut ?adoc, iar lui ?adoc i s-a nascut ?alum;
\par 13 Lui ?alum i s-a nascut Hilchia, iar lui Hilchia i s-a nascut Azaria;
\par 14 Lui Azaria i s-a nascut Seraia, iar lui Seraia i s-a nascut Iehosadac.
\par 15 Iehosadac a mers în robie când Domnul a stramutat pe cei din Iuda ?i pe cei din Ierusalim prin mâna lui Nabucodonosor.
\par 16 Astfel fiii lui Levi au fost: Gher?om, Cahat ?i Merari.
\par 17 Iata numele fiilor lui Gher?om: Libni ?i ?imei.
\par 18 Fiii lui Cahat au fost: Amram, I?har, Hebron ?i Uziel.
\par 19 Fiii lui Merari au fost: Mahli ?i Mu?i. Iata urma?ii lui Levi dupa neamurile lor.
\par 20 Gher?om a avut pe Libni, fiul lui; pe Iahat, fiul lui, ?i pe Zima, fiul lui;
\par 21 Pe Ioah, fiul lui; pe Ido, fiul lui; pe Zerah, fiul lui, ?i pe Ieatrai, fiul lui.
\par 22 Fiii lui Cahat au fost: Aminadab, fiul lui; Core, fiul lui, ?i Asir, fiul lui;
\par 23 Elcana, fiul lui; Ebiasaf, fiul lui, ?i Asir, fiul lui;
\par 24 Tahat, fiul lui, Uriel, fiul lui; Uzia, fiul lui, ?i Saul, fiul lui.
\par 25 Fiii lui Elcana sunt: Amasai ?i Ahimot,
\par 26 Elcana, fiul lui; ?ofai fiul lui, ?i Nahat, fiul lui,
\par 27 Eliab, fiul lui; Ieroham, fiul lui; Elcana, fiul lui; Samuel, fiul lui.
\par 28 Fiii lui Samuel au fost: întâiul nascut Ioil, al doilea, Abia.
\par 29 Fiii lui Merari au fost: Mahli, Libni, fiul lui; ?imei, fiul lui; Uza, fiul lui;
\par 30 ?imea, fiul lui; Aghia, fiul lui, ?i Asaia, fiul lui.
\par 31 Iata cei pe care David i-a pus capetenii peste cântare?i în casa Domnului, în timpul când a a?ezat în ea chivotul legii,
\par 32 Care au servit de cântare?i înaintea cortului adunarii, pâna când Solomon a zidit templul Domnului în Ierusalim, ?i care fusesera rândui?i la slujba lor dupa rânduiala lor;
\par 33 Iata pe cei care au fost rândui?i cu fiii lor: din fiii lui Cahat: Heman cântare?ul, fiul lui Ioil, fiul lui Samuel,
\par 34 Fiul lui Elcana, fiul lui Ieroham, fiul lui Eliel, fiul lui Toah,
\par 35 Fiul lui ?uf, fiul lui Elcana, fiul lui Mahat, fiul lui Amasai,
\par 36 Fiul lui Elcana, fiul lui Ioil, fiul lui Azaria, fiul lui ?efania,
\par 37 Fiul lui Tahat, fiul lui Asir, fiul lui Abiasaf, fiul lui Core,
\par 38 Fiul lui I?har, fiul lui Cahat, fiul lui Levi, fiul lui Israel.
\par 39 ?i fratele sau Asaf, care statea în partea dreapta a lui, adica Asaf, fiul lui Berechia, fiul lui ?imea,
\par 40 Fiul lui Micael, fiul lui Baaseia, fiul lui Malchia,
\par 41 Fiul lui Etni, fiul lui Zerah, fiul lui Adaia,
\par 42 Fiul lui Etan, fiul lui Zima, fiul lui ?imei,
\par 43 Fiul lui Iahat, fiul lui Gher?om, fiul lui Levi.
\par 44 Iar din fiii lui Merari, fra?ii lor, au fost în partea stânga: Etan, fiul lui Chi?i, fiul lui Abdi, fiul lui Maluc,
\par 45 Fiul lui Ha?abia, fiul lui Amasia, fiul lui Hilchia,
\par 46 Fiul lui Am?i, fiul lui Bani, fiul lui ?emer,
\par 47 Fiul lui Mahli, fiul lui Mu?i, fiul lui Merari, fiul lui Levi.
\par 48 Fra?ii lor levi?i erau rândui?i la tot felul de slujbe, la casa Domnului.
\par 49 Iar Aaron ?i fiii lor ardeau pe jertfelnic arderi de tot ?i tamâie pe altarul tamâierii, savâr?ind toate slujbele sfinte în Sfânta Sfintelor ?i pentru ispa?irea lui Israel, în toate, cum poruncise robul lui Dumnezeu Moise.
\par 50 Iata fiii lui Aaron: Eleazar, fiul lui; Finees, fiul lui; Abi?ua, fiul lui;
\par 51 Buchi, fiul lui; Uzi, fiul lui; Zerahia, fiul lui;
\par 52 Meraiot, fiul lui; Amaria, fiul lui; Ahitub, fiul lui;
\par 53 ?adoc, fiul lui; Ahimaa?, fiul lui.
\par 54 Iata locuin?ele lor dupa satele lor în hotarele lor: fiilor lui Aaron din familia lui Cahat, dupa cum le-a cazut sor?ul,
\par 55 Li s-au dat Hebronul, în pamântul lui Iuda ?i împrejurimile lui,
\par 56 Iar ?arinile acestei ceta?i ?i satele ei s-au dat lui Caleb, fiul lui Iefonie.
\par 57 Fiilor lui Aaron li s-au dat de asemenea ora?ele de scapare: Hebron ?i Libna ?i împrejurimile lor, Iatir ?i E?temoa ?i ?inuturile lor,
\par 58 Hilenul (Holonul) ?i pa?unile lui, Debirul ?i pa?unile lui;
\par 59 A?anul (Ainul) ?i împrejurimile lui, Bet?eme?ul ?i împrejurimile lui;
\par 60 Iar de la tribul lui Veniamin: Gheba ?i pa?unile ei, Alemetul (Almonul) ?i împrejurimile lui, Anatotul ?i ?inuturile lui; ceta?ile familiilor lor erau de toate treisprezece ceta?i.
\par 61 Celorlal?i fii ai lui Cahat, din familiile acestui trib, li s-au dat, dupa sor?i, zece ceta?i din hotarele jumata?ii tribului lui Manase.
\par 62 Fiilor lui Gher?om, dupa familiile lor, li s-au dat treisprezece ceta?i din tribul lui Isahar, din tribul lui A?er, din tribul lui Neftali ?i din tribul lui Manase, în Vasan.
\par 63 Fiilor lui Merari, dupa familiile lor, li s-au dat prin sor?i douasprezece ceta?i din tribul lui Ruben, din tribul lui Gad ?i din tribul lui Zabulon.
\par 64 A?a au dat fiii lui Israel Levi?ilor ceta?i cu împrejurimile lor.
\par 65 Li s-au dat prin sor?i din tribul fiilor lui Iuda, din tribul fiilor lui Simeon ?i din tribul fiilor lui Veniamin acele ceta?i pe care ei le-au numit pe nume.
\par 66 Iar unora din familiile fiilor lui Cahat li s-au dat ceta?i din tribul lui Efraim.
\par 67 Li s-au dat ceta?ile de scapare: Sichemul ?i împrejurimile lui, pe muntele Efraim ?i Ghezerul cu împrejurimile lui;
\par 68 Iocmeamul (Chib?oimul) cu împrejurimile lui ?i Bethoronul cu împrejurimile lui;
\par 69 Aialonul cu împrejurimile lui ?i Gat-Rimonul cu împrejurimile lui.
\par 70 Din jumatatea tribului lui Manase li s-au dat: Anerul cu împrejurimile lui, Bileanul cu împrejurimile lui. Acestea sunt locuin?ele pentru ceilal?i fii ai lui Cahat.
\par 71 Fiilor lui Gher?om din familiile din jumatatea tribului lui Manase li s-au dat Golanul în Vasan cu împrejurimile lui ?i A?tarotul cu împrejurimile lui.
\par 72 Din tribul lui Isahar li s-au dat Chede?ul (Chi?ionul) cu împrejurimile lui, Dobratul cu împrejurimile lui,
\par 73 Ramotul cu împrejurimile lui ?i Anemul cu împrejurimile lui.
\par 74 Din tribul lui A?er li s-au dat: Ma?alul cu împrejurimile lui ?i Abdonul cu împrejurimile lui;
\par 75 Hucocul (Helcotul) cu ?inutul lui ?i Rehobul cu ?inutul lui;
\par 76 Din tribul lui Neftali li s-au dat Chede?ul în Galileea, cu ?inutul lui, Hamonul cu ?inutul lui ?i Chiriataimul cu ?inutul lui.
\par 77 Iar celorlal?i fii ai lui Merari li s-au dat: din tribul lui Zabulon, Rimonul cu ?inutul lui ?i Taborul cu ?inutul lui,
\par 78 Iar dincolo de Iordan, în fa?a Ierihonului, la rasarit de Iordan, li s-a dat în tribul lui Ruben: Be?erul, în pustiu, cu ?inutul lui, ?i Iah?a cu ?inutul ei,
\par 79 Chedemotul cu împrejurimile lui ?i Mefaatul cu ?inutul lui.
\par 80 Din tribul lui Gad li s-au dat: Ramotul în Galaad cu ?inutul lui ?i Mahanaimul cu ?inutul lui;
\par 81 He?bonul ?i Iazerul cu ?inuturile lor.

\chapter{7}

\par 1 Fiii lui Isahar au fost patru: Tola, Pua, Ia?ub ?i ?imron.
\par 2 Fiii lui Tola au fost: Uzi, Refaia, Ieriil, Iahmai, Ibsam ?i, Samuel; ace?tia sunt cei mai de seama în neamul lui Tola, oameni razboinici în neamul lor; numarul lor în zilele lui David era douazeci ?i doua de mii ?i ?ase sute.
\par 3 Fiul lui Uzi a fost Izrahia; iar fiii lui Izrahia au fost: Micael, Obadia, Ioil ?i I?ia, de to?i cinci. To?i ace?tia sunt capetenii.
\par 4 Ei, dupa familiile lor ?i dupa neamurile lor, aveau o?tire de treizeci ?i ?ase de mii de oameni, pentru ca ei au avut multe femei ?i mul?i copii.
\par 5 Iar fra?ii lor, în toate neamurile lui Isahar, aveau oameni de lupta optzeci ?i ?apte de mii, înscri?i în tabli?ele cele cu spi?a neamului.
\par 6 Veniamin a avut trei: pe Bela, Becher ?i Iediael (A?bel).
\par 7 Fiii lui Bela au fost cinci: E?bon, Uzi, Uziel, Ierimot ?i Iri, to?i capetenii de familii, oameni razboinici. În tabli?ele cu spi?a neamului sunt înscri?i douazeci ?i doua de mii treizeci ?i patru.
\par 8 Fiii lui Becher au fost: Zemira, Ioa?, Eliezer, Elioenai, Omri, Ieremot, Abia, Anatot ?i Alemet; to?i ace?tia sunt fiii lui Becher.
\par 9 În tabli?ele cu spi?a neamului sunt înscri?i din ace?tia, dupa familiile ?i dupa neamurile lor, oameni razboinici douazeci de mii ?i doua sute.
\par 10 Fiul lui Iediael (A?bel) a fost Bilhan. Fiii lui Bilhan au fost: Ieu?, Veniamin, Ehud, Chenaana, Zetan, Tar?i? ?i Ahi?ahar.
\par 11 To?i ace?ti fii ai lui Iediael (A?bel), au fost capi de familie, oameni razboinici; ?aptesprezece mii ?i doua sute erau în stare de a ie?i la razboi.
\par 12 ?upim ?i Hupim erau fiii lui Ir, iar Hu?im era fiul lui Aher.
\par 13 Fiii lui Neftali au fost: Iah?iel, Guni, Ie?er ?i ?alum (Silem), copiii Bilhei.
\par 14 Fiii lui Manase au fost: Asriel, pe care l-a nascut concubina sa arameana; tot aceasta a nascut pe Machir, tatal lui Galaad.
\par 15 Machir ?i-a luat de femeie pe sora lui Hupim ?i a lui ?upim, al carei nume era Maaca. Numele fiului al doilea a fost Salfaad. Salfaad a avut numai fete.
\par 16 Maaca, femeia lui Machir, a nascut un fiu ?i i-a pus numele Pere?, iar numele fratelui lui era ?ere?. Fiii acestuia au fost Ulam ?i Rechem.
\par 17 Fiul lui Ulam a fost Bedan. Ace?tia sunt fiii lui Galaad, fiul lui Machir, fiul lui Manase.
\par 18 Sora sa, Molechet, a nascut pe I?hod, pe Abiezer ?i pe Mahla.
\par 19 Fiii lui ?emida au fost: Ahian, ?echem, Lichi ?i Aniam.
\par 20 Fiii lui Efraim au fost: ?utelah, Bered, fiul lui; Tahat, fiul lui; Eleadah, fiul lui ?i Tahat, fiul lui;
\par 21 Zabad, fiul lui; ?utelah, fiul lui; Ezer ?i Elead. Pe ace?tia i-au ucis locuitorii din Gat, ba?tina?ii jarii aceleia, pentru ca ei se dusesera sa le apuce turmele lor.
\par 22 Dupa ei a plâns Efraim, tatal lor, zile multe ?i au venit fra?ii lui sa-l mângâie.
\par 23 Apoi a intrat el la femeia sa ?i ea a zamislit ?i a nascut un fiu ?i el i-a pus numele Beria, pentru ca nenorocirea se atinsese de casa lui.
\par 24 ?i a avut el ?i o fata: ?eera. Aceasta a zidit Bethoronul de jos ?i de sus ?i Uzen-?eera.
\par 25 Refah, fiul sau, ?i Re?ef, fiul sau; Telah, fiul sau, ?i Tahan, fiul sau.
\par 26 Ladan, fiul sau; Amiud, fiul sau, ?i Eli?ama, fiul sau.
\par 27 Non, fiul sau; Iosua, fiul sau.
\par 28 Mo?iile lor ?i locurile lor de locuit au fost: Betelul ?i ceta?ile care ?ineau de el, spre rasarit Naaranul, spre apus Ghezerul ?i ceta?ile care ?ineau de el; Sichemul ?i ceta?ile care ?ineau de el pâna la Gaza ?i ceta?ile ce ?ineau de aceasta.
\par 29 Iar din partea fiilor lui Manase: Bet-?eanul ?i ceta?ile ce ?ineau de el, Taanacul ?i ceta?ile ce ?ineau de el, Meghidonul ?i ceta?ile ce ?ineau de el, Dorul ?i ceta?ile ce ?ineau de el. în ele locuiau fiii lui Iosif, fiul lui Israel.
\par 30 Fiii lui A?er erau: Imna, I?va, I?vi ?i Beria ?i sora lor Serah.
\par 31 Fiii lui Beria au fost: Heber ?i Malchiel. Acesta e tatal lui Birzait.
\par 32 Heber a avut fii pe Iaflet, ?emer ?i Hotam ?i pe sora lor ?ua.
\par 33 Fiii lui Iaflet au fost: Pasac, Bimhal ?i A?vat. Ace?tia sunt fiii lui Iaflet.
\par 34 Fiii lui ?emer au fost: Ahi, Rohga, Huba ?i Aram.
\par 35 Fiii lui Helem, fratele lui, au fost: ?ofah, Imna, ?ele? ?i Amal.
\par 36 Fiii lui ?ofah au fost: Suah, Harnefer, ?ual, Beri, Imra,
\par 37 Be?er, Hod, ?ama, ?il?a, Itran ?i Beera.
\par 38 Fiii lui Ieter au fost: Iefune, Pispa ?i Ara.
\par 39 Fiii nascu?i din Ula au fost: Arah, Haniel ?i Ri?ia.
\par 40 To?i ace?tia sunt fiii lui A?er, capi de familie, oameni ale?i, razboinici, capetenii de mâna întâi. În tabli?ele lor cu spi?a neamului sunt înscri?i în o?tire pentru razboi un numar de douazeci ?i ?ase de mii de oameni.

\chapter{8}

\par 1 Lui Veniamin i s-a nascut Bela, întâiul sau nascut; al doilea, A?bel, al treilea, Ahrah,
\par 2 Al patrulea, Noha ?i al cincilea, Rafa.
\par 3 Fiii lui Bela au fost: Adar, Ghera, Abiud,
\par 4 Abi?ua, Naaman, Ahoah,
\par 5 Ghera, ?efufan ?i Huram.
\par 6 Iata fiii lui Ehud, care au fost capi de familii, care au trait în Gheba ?i au fost stramuta?i în Manahat:
\par 7 Naaman, Ahia ?i Ghera, care i-a stramutat, a nascut pe Uza ?i pe Ahiud.
\par 8 Lui ?aharaim i s-au nascut copii în ?ara Moabi?ilor, dupa ce a dat drumul femeilor sale, Hu?im ?i Baara.
\par 9 I s-au nascut din Hode?a, femeia sa: Iobab, ?ibia, Me?a ?i Malcam;
\par 10 Ieu?, ?achia ?i Mirma. Ace?tia sunt fiii lui, capi de familie.
\par 11 Din Hu?im i s-au nascut Abitub ?i Elpaal.
\par 12 Fiii lui Elpaal au fost: Eber, Mi?eam ?i ?emed, care au zidit Ono ?i Lodul ?i ceta?ile ce ?ineau de el,
\par 13 Precum ?i Beria ?i Sema. - Ace?tia au fost capii familiilor locuitorilor Aialonului; ei au alungat pe locuitorii din Gat. -
\par 14 Ahio, ?a?ac, ?i Iremot;
\par 15 Zebadia, Arad ?i Eder;
\par 16 Micael, I?pa ?i Ioha, fiii lui Beria.
\par 17 Zecadia, Me?ulam, Hizchi ?i Heber,
\par 18 Ie?mere, Izliah ?i Iobab, fiii lui Elpaal.
\par 19 Iachim, Zicri ?i Zabdi,
\par 20 Elienai, ?iltai ?i Eliel,
\par 21 Adaia, Beraia ?i ?imrat, fiii lui ?imei.
\par 22 I?pan, Eber ?i Eliel,
\par 23 Abdon, Zicri ?i Hanan,
\par 24 Hanania, Elam ?i Antotia,
\par 25 Ifdia ?i Fanuil, fiii lui ?i?ac.
\par 26 ?am?erai, ?eharia ?i Atalia,
\par 27 Iaare?ia, Elia ?i Zicri, fiii lui Ieroham.
\par 28 Acestea sunt capeteniile familiilor mai de seama ale neamului lor. Ei au locuit în Ierusalim.
\par 29 În Ghibeon a trait Ieguel, tatal Ghibeoni?ilor. Numele femeii lui a fost Maaca
\par 30 ?i fiul lui, întâiul nascut, Abdon, dupa el ?ur, Chi?, Baal, Nadab, Ner,
\par 31 Ghedeor, Ahio, Zecher ?i Miclot;
\par 32 Lui Miclot i s-a nascut ?imea. ?i au trait ei împreuna cu fra?ii lor în Ierusalim.
\par 33 Lui Ner i s-a nascut Chi?; lui Chi? i s-a nascut Saul; lui Saul i s-a nascut Ionatan, Melchi?ua, Aminadab ?i E?baal.
\par 34 Fiul lui Ionatan a fost Meribaal; (Mefibo?et); a avut de fiu pe Mica.
\par 35 Fiii lui Mica au fost: Piton, Melec, Tarea ?i Ahaz.
\par 36 Lui Ahaz i s-a nascut Iehoada, lui Iehoada i s-a nascut Alemet, Asmavet ?i Zimri; lui Zimri i s-a nascut Mo?a;
\par 37 Lui Mo?a i s-a nascut Binea; Rafa, fiul lui; Eleasa, fiul lui; A?el, fiul lui.
\par 38 A?el a avut ?ase feciori ?i iata numele lor: Azricam, Bocru, Ismael, ?earia, Obadia ?i Hanan. To?i ace?tia sunt fiii lui A?el.
\par 39 Fiii lui E?ec, fratele sau, au fost: Ulam, întâiul sau nascut; al doilea, Ieu?; al treilea Elifelet.
\par 40 Fiii lui Ulam au fost oameni razboinici, tragatori din arc, având mul?i copii ?i nepo?i pâna la o suta cincizeci. To?i ace?tia sunt din fiii lui Veniamin.

\chapter{9}

\par 1 A?a au fost numara?i dupa neamurile lor to?i Israeli?ii ?i iata sunt înscri?i în cartea regilor lui Israel. Iar Iudeii, pentru faradelegile lor, au fost du?i în Babilon.
\par 2 Cei dintâi locuitori care au trait în pamânturile lor, prin ora?ele lui Israel, au fost Israeli?ii, preo?ii, levi?ii ?i cei afierosi?i templului.
\par 3 În Ierusalim au trait unii din fiii lui Iuda, din fiii lui Veniamin ?i din fiii lui Efraim ?i ai lui Manase.
\par 4 Utai, fiul lui Amihud, fiul lui Omri, fiul lui Imri, fiul lui Bani, din fiii lui Fares, fiul lui Iuda;
\par 5 Din fiii lui ?iloni: Asaia, întâiul nascut, ?i fiii acestuia.
\par 6 Din fiii lui Zerah: Ieuel ?i fra?ii lui, ?ase sute nouazeci;
\par 7 Din fiii lui Veniamin: Salu, fiul lui Me?ulam, fiul lui Hodavia, fiul lui Asenua
\par 8 ?i Ibneia, fiul lui Ieroham, ?i Ela, fiul lui Uzi, fiul lui Micri, ?i Me?ulam, fiul lui ?efatia, fiul lui Reguel, fiul lui Ibneia
\par 9 ?i fra?ii lor, dupa neamurile lor: noua sute cincizeci ?i ?ase. Ace?ti barba?i erau capi de familie în neamul lor.
\par 10 Iar dintre preo?i: Iedaia, Ioiarib, Iachin
\par 11 ?i Azaria, fiul lui Hilchia, fiul lui Me?ulam, fiul lui ?adoc, fiul lui Meraiot, fiul lui Ahitub, capetenia în casa lui Dumnezeu;
\par 12 Adaia, fiul lui Ieroham, fiul lui Pa?hur, fiul lui Malchia; ?i Maesai, fiul lui Adiel, fiul lui Iahzera, fiul lui Me?ulam, fiul lui Me?ilemit, fiul lui Imer.
\par 13 ?i fra?ii lor, capi în familiile lor: o mie ?apte sute ?aizeci, barba?i de isprava la slujbele din casa Domnului.
\par 14 Iar din levi?i: ?emaia, fiul lui Ha?ub, fiul lui Azricam, fiul lui Ha?abia; ace?tia sunt din fiii lui Merari.
\par 15 Bacbacar, Here?, Galal ?i Matania, fiul lui Mica, fiul lui Zicri, fiul lui Asaf,
\par 16 Obadia, fiul lui ?emaia, fiul lui Galal, fiul lui Iedutun; Berechia, fiul lui Asa, fiul lui Elcana, care locuia în satele Netofati?ilor.
\par 17 Iar din portari: ?alum, Acub, Talmon ?i Ahiman ?i fra?ii lor; ?alum era capetenie.
\par 18 Ace?ti portari fac straja fiilor levi?ilor pâna astazi la por?ile regale cele de la rasarit.
\par 19 ?alum, fiul lui Core, fiul lui Ebiasaf, fiul lui Corah, ?i fra?ii lui cei din neamul lui Corah, dupa datoria slujbei lor, aveau paza cortului, iar parin?ii lor pazeau intrarea în tabara Domnului.
\par 20 Finees, fiul lui Eleazar, fusese înainte capetenie peste ei ?i Domnul era cu el.
\par 21 Zaharia, fiul lui Me?elemia, era portar la u?a cortului adunarii.
\par 22 To?i cei ale?i ca portari la praguri erau doua sute doisprezece. Ei erau înscri?i în registre dupa a?ezarile lor. Pe ei îi pusese David ?i Samuel înainte-vazatorul, pentru credincio?ia lor.
\par 23 ?i ei ?i fiii lor ?ineau straja la u?ile casei Domnului, la casa cortului.
\par 24 În patru laturi se aflau portari: la rasarit, la apus, la miazanoapte ?i la miazazi.
\par 25 Iar fra?ii lor traiau în sala?urile lor, venind la ei din timp în timp, pentru ?apte zile.
\par 26 Aceste patru capetenii de portari levi?i erau de încredere ?i tot ei aveau grija casei Domnului ?i vistieriei ei.
\par 27 Împrejurul casei lui Dumnezeu ei petreceau ?i noaptea, caci asupra lor era lasata paza ?i trebuia sa deschida în fiecare diminea?a u?ile.
\par 28 Unii din ei erau pu?i de paza la vasele de slujba, a?a ca ei cu numar le primeau ?i cu numar le dadeau.
\par 29 Altora din ei le era încredin?ata cealalta zestre ?i toate lucrurile trebuitoare pentru cele sfinte: faina cea mai buna, vinul, untdelemnul ?i tamâia cea mirositoare.
\par 30 Iar dintre fiii preo?ilor unii pregateau mir cu aromate.
\par 31 Matatia levitul, care era întâiul nascut al lui ?alum, fiul lui Core, era pus sa aiba grija de coptul aluaturilor în tigai.
\par 32 Unora din fra?ii lor, din fiii lui Cahat, le era încredin?ata pregatirea pâinilor punerii înainte, ca sa le puna în fiecare zi de odihna.
\par 33 Iar cântare?ii, cei mai de seama din neamul Levi?ilor, erau liberi de ocupa?ii în camarile templului, pentru ca ziua ?i noaptea erau îndatora?i sa se îndeletniceasca cu cântarea.
\par 34 Ace?tia erau cei mai de seama între familiile levi?ilor ?i locuiau în Ierusalim.
\par 35 În Ghibeon locuiau: tatal Ghibeoni?ilor, Ieguel, a carui femeie se numea Maaca;
\par 36 ?i fiul lui, întâiul nascut, se numea Abdon; dupa el venea ?ur, Chi?, Baal, Ner, Nadab,
\par 37 Ghedor, Ahio, Zaharia ?i Miclot.
\par 38 Lui Miclot i s-a nascut ?imeam. ?i ei traiau lânga fra?ii lor, în Ierusalim, împreuna cu fra?ii lor.
\par 39 Lui Ner i s-a nascut Chi?; lui Chi? i s-a nascut Saul, lui Saul i s-a nascut Ionatan, Melchi?ua, Aminadab ?i E?baal.
\par 40 Fiul lui Ionatan a fost Meribaal; lui Meribaal i s-a nascut Mica.
\par 41 Fiii lui Mica au fost: Piton, Melec, Tareia ?i Ahaz.
\par 42 Lui Ahaz i s-a nascut Iara; lui Iara i s-au nascut Alemet, Asmavet ?i Zimri; lui Zimri i s-a nascut Mo?a.
\par 43 Lui Mo?a i s-a nascut Binea; Refaia, fiul lui; Eleasa, fiul lui; A?el, fiul lui.
\par 44 Lui A?el i s-au nascut ?ase fii ?i iata numele lor: Azricam, Bocru, Ismael, ?earia, Obadia ?i Hanan. Ace?tia sunt fiii lui A?el.

\chapter{10}

\par 1 În vremea aceea Filistenii s-au ridicat cu razboi asupra lui Israel ?i Israeli?ii au fugit de Filisteni ?i au cazut birui?i pe muntele Ghilboa.
\par 2 Atunci au alergat Filistenii dupa Saul ?i dupa fiii lui ?i au ucis Filistenii pe Ionatan, pe Aminadab ?i pe Melchi?ua, fiii lui Saul.
\par 3 Iar lupta împotriva lui Saul s-a înte?it ?i arca?ii au tras asupra lui, a?a ca el a fost ranit de sage?i.
\par 4 Saul a zis purtatorului sau de arme: "Scoate sabia ta ?i ma strapunge cu ea, ca sa nu vina ace?ti netaia?i împrejur ?i sa-?i bata joc de mine". Dar purtatorul de arme nu s-a hotarât la aceasta, pentru ca se speriase foarte tare. Atunci Saul a luat sabia ?i s-a aruncat în ea.
\par 5 Vazând purtatorul de arme ca Saul a murit, s-a aruncat ?i el în sabia sa.
\par 6 A?a a murit Saul cu cei trei fii ai lui ?i toata casa a murit împreuna cu el.
\par 7 Când au vazut Israeli?ii, care erau în vale, ca fug to?i ?i ca Saul ?i fiii lui au murit, au lasat ceta?ile lor ?i au fugit în toate par?ile, iar Filistenii au venit ?i s-au a?ezat în ele.
\par 8 A doua zi au venit Filistenii sa ridice pe cei uci?i ?i, gasind pe Saul ?i pe fiii lui cazu?i pe muntele Ghilboa,
\par 9 L-au dezbracat ?i i-au luat capul ?i armele ?i au trimis prin ?ara Filistenilor, sa se vesteasca aceasta înaintea idolilor lor ?i înaintea poporului.
\par 10 Armele lui le-au pus în templul zeilor lor, iar capul lui l-au spânzurat în templul lui Dagon.
\par 11 Atunci auzind tot Iabe?ul Galaadului ce au facut Filistenii cu Saul,
\par 12 S-au ridicat to?i oamenii de lupta, au luat trupul lui Saul ?i trupurile fiilor lui, le-au dus în Iabe? ?i au îngropat oasele lor sub un stejar în Iabe? ?i au postit ?apte zile.
\par 13 A?a a murit Saul pentru nelegiuirea sa pe care o facuse el înaintea Domnului, pentru ca n-a pazit cuvântul Domnului ?i pentru ca a întrebat ?i a cercetat o vrajitoare
\par 14 ?i nu a cercetat pe Domnul. De aceea a ?i fost el omorât ?i domnia a fost data lui David, fiul lui Iesei.

\chapter{11}

\par 1 Dupa aceea s-au adunat to?i Israeli?ii la David în Hebron zicând: "Iata noi suntem oasele tale ?i carnea ta.
\par 2 ?i mai înainte, când Saul era înca rege, tu ai pova?uit pe Israel la razboi ?i l-ai adus teafar înapoi ?i Domnul Dumnezeul tau ?i-a spus: Tu vei pa?te pe poporul Meu Israel ?i tu vei fi pova?uitorul poporului Meu Israel".
\par 3 ?i au venit toate capeteniile lui Israel la rege în Hebron ?i a încheiat cu ei David legamânt în Hebron înaintea fe?ei Domnului; ?i au uns pe David de rege peste Israel, dupa cuvântul Domnului care fusese prin Samuel.
\par 4 Apoi s-a dus David ?i tot Israelul la Ierusalim, adica la Iebus. Acolo însa erau Iebuseii, locuitorii ?arii aceleia.
\par 5 ?i au zis locuitorii Iebusului catre David: "Nu vei intra aici!" Dar David a luat cetatea Sionului. Aceasta este cetatea lui David.
\par 6 Apoi a zis David: "Cine va lovi cel dintâi pe Iebuseu, acela va fi cap ?i capetenie peste o?tire". ?i s-a sculat înainte de to?i Ioab, fiul ?eruiei, ?i s-a facut capetenie.
\par 7 David a locuit în cetatea aceea ?i ea s-a ?i numit cetatea lui David.
\par 8 El a zidit cetatea împrejur, începând de la Milo, iar Ioab a reînnoit celelalte par?i ale ceta?ii.
\par 9 Dupa aceea a propa?it David ?i s-a ridicat din ce în ce mai mult ?i Domnul Savaot era cu. el.
\par 10 Iata cei mai de seama dintre puternicii lui David, care s-au luptat tare împreuna cu el, în domnia lui, împreuna cu tot Israelul, ca sa întareasca domnia lui asupra lui Israel, dupa cuvântul Domnului.
\par 11 ?i iata numarul vitejilor pe care i-a avut David: Ia?obeam (Io?eb-Ba?ebet), fiul lui Hacmoni, cel mai de seama între cei treizeci; el ?i-a ridicat suli?a asupra a trei sute de oameni ?i i-a ucis dintr-o data.
\par 12 Dupa el vine Eleazar, fiul lui Dodo Ahohitul, unul din cei trei viteji.
\par 13 Acesta a fost cu David la Pasdamim, unde se adunasera Filistenii pentru razboi. Acolo, parte din câmp era semanat cu orz ?i Israeli?ii au fugit de Filisteni;
\par 14 Dar ei au stat în mijlocul câmpului, l-au aparat ?i au înfrânt pe Filisteni ?i le-a daruit Domnul biruin?a mare.
\par 15 Trei din cele treizeci de capetenii s-au coborât pe stânca la David, în pe?tera Adulam, când tabara Filistenilor era a?ezata în valea Refaim.
\par 16 David atunci era la loc întarit, iar o?tirea de întarire a Filistenilor era atunci în Betleem.
\par 17 ?i a dorit David ?i a zis: "Cine ma va adapa cu apa din fântâna Betleemului care este la poarta?"
\par 18 Atunci ace?ti trei au strabatut prin tabara Filistenilor, au scos apa din fântâna Betleemului cea de la poarta ?i au luat-o ?i au dus-o lui David. Dar David n-a voit sa o bea ?i a varsat-o înaintea Domnului,
\par 19 Zicând: "Sa ma fereasca Dumnezeu sa fac eu aceasta! A? putea sa beau eu sângele celor care s-au dus acolo cu primejduirea vie?ii lor? Caci cu primejduirea vie?ii lor au adus-o!" ?i n-a vrut sa o bea. Iata ce au facut ace?ti trei viteji.
\par 20 ?i Abi?ai, fratele lui Ioab, era capetenia celor trei; el a ranit deodata cu suli?a trei sute de oameni ?i era vestit între cei trei.
\par 21 El era mai stralucit decât cei treizeci ?i le era capetenie, dar cu ceilal?i trei nu era deopotriva.
\par 22 Benaia, fiul lui Iehoiada, barbat viteaz, mare dupa fapte, era din Cab?eel; el a ucis doi fii ai lui Ariel Moabitul ?i s-a coborât într-o groapa ?i a ucis un leu pe o vreme cu zapada.
\par 23 Tot el a ucis un egiptean, un om cu statura de cinci co?i, care avea în mâna o suli?a, ca un sul de la razboiul de ?esut; el s-a dus la el cu toiagul, i-a smuls suli?a din mâna ?i l-a ucis cu suli?a lui.
\par 24 Iata ce a facut Benaia, fiul lui Iehoiada, care era în cinste la cei trei viteji.
\par 25 El era mai vestit decât cei treizeci, dar cu cei trei nu era deopotriva ?i David l-a pus cel mai de aproape împlinitor al poruncilor sale.
\par 26 Iar dintre o?tenii cei mai de seama erau: Asael, fratele lui Ioab; Elhanan, fiul lui Dodo, din Betleem;
\par 27 ?amot din Haror; Hele? din Pelon;
\par 28 Ira, fiul lui Iche? din Tecoa; Abiezer din Anatot;
\par 29 Sibecai Hu?ateul, Ilai din Ahoh;
\par 30 Maharai din Netofat, Heled, fiul lui Baana din Netofat,
\par 31 Itai, fiul lui Ribai, din Ghibeea lui Veniamin; Benaia din Piraton
\par 32 Hurai din Nahali-Gaa?; Abiel din Araba;
\par 33 Azmavet din Bahurim; Eliahba din ?aalbon;
\par 34 Fiii lui Ha?em din Ghizon: Ionatan, fiul lui ?aghi din Harar;
\par 35 Ahiam, fiul lui Sacar din Harar; Elifelet, fiul lui Uri.
\par 36 Hefer din Mechera, Ahia din Pelon,
\par 37 He?ro din Carmel, Naarai, fiul lui Ezbai;
\par 38 Ioil, fratele lui Natan; Mibhar, fiul lui Hagri;
\par 39 ?elec Amonitul; Nahrai din Beerot, purtatorul de arme al lui Ioab, fiul ?eruiei,
\par 40 Ira din Iatir, Gareb din Iatir;
\par 41 Urie Heteul; Zabad, fiul lui Ahlai;
\par 42 Adina, fiul lui ?iza Rubenitul, capetenia Rubeni?ilor care avea sub el treizeci de in?i,
\par 43 Hanan, fiul lui Maaca; Iosafat din Mitni,
\par 44 Uzia din A?tarot; ?ama ?i Iehiel, fiii lui Hotam din Aroer,
\par 45 Iediael, fiul lui ?imri ?i Ioha, fratele lui Ti?itul,
\par 46 Eliel din Mahavim, Ieribai ?i Io?avia, fiii lui Elnaam ?i Itma Moabitul,
\par 47 Eliel, Obed ?i Iaasiel din Me?oba.

\chapter{12}

\par 1 Iata pe cei care au mai mers la David în ?iclag, pe când statea el înca ascuns de Saul, fiul lui Chi?. Ace?tia erau dintre vitejii care ajutasera la lupta.
\par 2 Ei erau arca?i, aruncau pietre ?i cu dreapta ?i cu stânga ?i din arcuri trageau cu sage?i ?i faceau parte dintre Veniamineni, fra?ii lui Saul, ?i anume:
\par 3 Capetenia Ahiezer, apoi Ioa?, fiii lui ?emaa din Ghibeea, Ieziel ?i Pelet, fiii lui Azmavet, Beraca ?i Iehu din Anatot;
\par 4 I?maia Ghibeoneanul, un viteaz dintre cei treizeci ?i capetenie peste treizeci; Ieremia, Iahaziel, Iohanan ?i Iozabad din Ghedera;
\par 5 Eluzai, Ierimot, Bealia, ?emaria ?i ?efatia Harifianul;
\par 6 Elcana, I?ia, Azareel, Ioezer ?i Ia?obeam, Corei?i;
\par 7 Ioela ?i Zebadia, fiii lui Ieroham din Ghedor.
\par 8 Din Gaditeni au trecut la David, în cetatea din pustie, oameni curajo?i, razboinici ?i înarma?i cu scut ?i suli?a, cu fa?a lor ca fa?a leului ?i iu?i ca ?i caprioarele din mun?i ?i anume:
\par 9 Capetenia Ezer, al doilea Obadia ?i al treilea Eliab;
\par 10 Al patrulea Ma?mana, al cincilea Ieremia,
\par 11 Al ?aselea Atai, al ?aptelea Eliel,
\par 12 Al optulea Iohanan ?i al noualea Elzabad,
\par 13 Al zecelea Ieremia ?i al unsprezecelea Macbanai.
\par 14 Ace?tia sunt din fiii lui Gad ?i erau capetenii în o?tire: cei mai mici, peste sute ?i cei mai mari, peste mii.
\par 15 Ace?tia au trecut Iordanul în luna întâi, când el iese din matca sa, ?i au alungat pe to?i cei ce locuiau pe vai, spre rasarit ?i apus.
\par 16 Au mai venit de asemenea ?i dintre fiii lui Veniamin ?i ai lui Iuda în cetate, la David.
\par 17 Iar David a ie?it în întâmpinarea lor ?i le-a zis: "De a?i venit cu pace, ca sa-mi ajuta?i, atunci sa fie în mine ?i în voi o singura inima; iar de a?i venit ca prin vicle?ug sa ma da?i vrajma?ilor mei, atunci, cum nu este prihana în mâinile mele, va vedea ?i va judeca Dumnezeul parin?ilor no?tri".
\par 18 Atunci a cuprins Duhul pe Amasai, capetenia celor treizeci ?i a zis: "Suntem cu tine, Davide, ?i pacea sa fie cu tine, fiul lui Iesei! Pace ?ie ?i pace celor ce-?i ajuta, ca î?i ajuta Dumnezeul tau". ?i i-a primit David ?i i-a pus în capul o?tirii.
\par 19 ?i din tribul lui Manase au trecut unii la David, când mergea el cu Filistenii la razboi contra lui Saul, dar nu l-au ajutat, pentru ca conducatorii Filistenilor, sfatuindu-se, l-au trimis înapoi, zicând: "Pentru primejduirea capului nostru, el va trece la domnul sau Saul".
\par 20 Dupa ce s-a întors el la ?iclag, au trecut la dânsul din ai lui Manase: Adnah, Iozabad, Iediael, Micael, Iozabad, Elihu ?i ?iltai, capetenii peste mii în Manase.
\par 21 Ace?tia au ajutat lui David contra navalitorilor, caci to?i ace?tia erau oameni viteji ?i capetenii în o?tire.
\par 22 Astfel în fiecare zi veneau lui David în ajutor pâna întru atât, încât tabara lui ajunsese mare, ca o tabara a lui Dumnezeu.
\par 23 Iata acum numarul capeteniilor de o?tire, care au venit la David în Hebron, ca sa-i încredin?eze domnia lui Saul, dupa cuvântul Domnului:
\par 24 Fii de-ai lui Iuda care purtau scut ?i suli?a erau ?ase mii opt sute, gata de lupta;
\par 25 Din fiii lui Simeon erau ?apte mii o suta, oameni viteji de o?tire;
\par 26 Din fiii lui Levi, patru mii ?ase sute;
\par 27 Iehoiada, capetenie din neamul lui Aaron, ?i cu el trei mii ?apte sute,
\par 28 ?i ?adoc, un tânar voinic cu neamurile lui douazeci ?i doua de capetenii.
\par 29 Din fiii lui Veniamin, fra?ii lui Saul, au venit trei mii, dar înca mul?i din ei se ?ineau de casa lui Saul;
\par 30 Din fiii lui Efraim, douazeci de mii opt sute de oameni viteji, oameni cunoscu?i în neamul lor;
\par 31 Din jumatate din semin?ia lui Manase, optsprezece mii care au fost chema?i pe nume ca sa mearga sa faca rege pe David;
\par 32 Dintre fiii lui Isahar au venit oameni în?elep?i care ?tiau ce ?i când trebuie sa faca Israel; ace?tia erau doua sute capetenii ?i to?i fra?ii lor urmau sfatul lor;
\par 33 Din tribul lui Zabulon au venit oameni gata de lupta ?i înarma?i cu tot felul de arme, în numar de cincizeci de mii, în ordine ?i într-un suflet;
\par 34 Din tribul lui Neftali, o mie de capetenii ?i cu ei treizeci ?i ?apte de mii cu scuturi ?i cu suli?e;
\par 35 Din tribul lui Dan au venit douazeci ?i opt de mii ?ase sute oameni gata de lupta;
\par 36 Din A?er au venit o?teni, gata de lupta, patruzeci de mii;
\par 37 De peste Iordan, din tribul lui Ruben, al lui Gad ?i din jumatate de trib al lui Manase, au venit o suta douazeci de mii cu tot felul de arme de lupta.
\par 38 To?i ace?ti o?teni, gata de lupta ?i cu toata inima, au venit la Hebron sa faca rege pe David peste Israel. Dar ?i to?i ceilal?i Israeli?i erau într-un cuget pentru a fi facut rege David.
\par 39 ?i au ramas acolo la David trei zile ?i au mâncat ?i au baut, pentru ca fra?ii lor pregatisera toate pentru ei.
\par 40 ?i apoi chiar vecinii lor, pâna chiar ?i Isahar, Zabulon ?i Neftali, adusesera toate de ale mâncarii pe asini, pe camile, pe catâri ?i cu carele de boi: faina, smochine ?i stafide, vin, untdelemn ?i vite mari ?i marunte, mul?ime multa, pentru ca bucurie mare era peste Israel.

\chapter{13}

\par 1 Atunci David s-a sfatuit cu capeteniile cele peste mii, cu suta?ii ?i cu toate capeteniile,
\par 2 ?i a zis David catre toata adunarea Israeli?ilor: "Daca binevoi?i voi ?i daca este voia Domnului Dumnezeului nostru, sa trimitem pretutindeni la ceilal?i fra?i ai no?tri, în toata ?ara lui Israel ?i totodata ?i la preo?i ?i levi?i prin ora?ele ?i prin satele lor, ca sa Se adune la noi;
\par 3 ?i sa aducem la noi chivotul Dumnezeului nostru, pentru ca în zilele lui Saul ne-am îndreptat spre el".
\par 4 Atunci toata adunarea a zis: "A?a sa fie", pentru ca lucrul acesta s-a parut drept înaintea a tot poporul.
\par 5 Astfel a adunat David pe to?i Israeli?ii, de la ?ihorul egiptean pâna la intrarea Hamatului, ca sa stramute chivotul Domnului din Chiriat-Iearim.
\par 6 Atunci s-a dus David ?i tot Israelul la Chiriat-Iearim, ce este în Iuda, ca sa stramute de acolo chivotul lui Dumnezeu, înaintea caruia se cheama numele Domnului Celui care sade pe heruvimi.
\par 7 ?i au adus chivotul lui Dumnezeu într-un car nou din casa lui Abinadab; iar carul îl conduceau Uza ?i Ahia.
\par 8 David însa ?i to?i Israeli?ii jucau înaintea lui Dumnezeu cât puteau cu cântari din gura, din chitara, din psaltirion, din timpane ?i ?imbale ?i din trâmbi?e;
\par 9 Dar când au ajuns la aria lui Chidon, Uza ?i-a întins mâna, ca sa sprijine chivotul, caci boii erau sa-l rastoarne.
\par 10 Atunci S-a mâniat Domnul pe Uza ?i l-a lovit, pentru ca ?i-a întins mâna spre chivot; ?i el a murit acolo pe loc înaintea lui Dumnezeu.
\par 11 ?i s-a întristat David ca Domnul lovise pe Uza ?i a numit locul acela Pere?-Uza ?i a?a se nume?te pâna azi.
\par 12 ?i s-a temut David de Dumnezeu în ziua aceea ?i a zis: "Cum voi duce la mine chivotul lui Dumnezeu?"
\par 13 De aceea nu a dus David chivotul la sine, în cetatea lui David, ci l-a întors la casa lui Obed-Edom.
\par 14 ?i a ramas chivotul Domnului la Obed-Edom, în casa lui, trei luni ?i a binecuvântat Domnul casa lui Obed-Edom ?i toate ale lui.

\chapter{14}

\par 1 În vremea aceea a trimis Hiram, regele Tirului, soli la David ?i lemne de cedru ?i pietrari ?i dulgheri, ca sa-i ridice casa.
\par 2 Când a aflat David ca l-a întarit Domnul rege peste Israel, ca domnia lui a fost înal?ata sus pentru poporul sau Israel,
\par 3 ?i-a luat înca alte femei din Ierusalim ?i i s-au mai nascut lui David fii ?i fiice.
\par 4 Iata numele celor ce i s-au nascut în Ierusalim: ?amua, ?obab, Natan ?i Solomon;
\par 5 Ibhar, Eli?ua ?i Elifelet;
\par 6 Nogah, Nefeg ?i Iafia;
\par 7 Eli?ama, Beeliada ?i Elifelet.
\par 8 Auzind însa Filistenii ca David a fost uns rege peste tot Israelul, s-au ridicat ei to?i sa caute pe David. ?i auzind David de aceasta, a ie?it înaintea lor.
\par 9 Iar Filistenii au venit ?i s-au a?ezat în valea Refaim.
\par 10 Atunci a întrebat David pe Dumnezeu, zicând: "Sa merg eu oare contra Filistenilor ?i-i vei da Tu în mâna mea?" ?i Domnul i-a raspuns: "Mergi, ca îi voi da în mâna ta!"
\par 11 ?i s-au dus ei în Baal-Pera?im ?i acolo i-a lovit David; apoi David a zis: "Dumnezeu a zdrobit pe vrajma?i cu mâna mea, ca pe o surpatura de apa". De aceea au ?i dat vaii aceleia numele de Baal- Pera?im.
\par 12 Filistenii ?i-au lasat acolo zeii, iar David i-a adunat ?i i-a ars cu foc.
\par 13 ?i au venit iara?i Filistenii ?i au tabarât în vale.
\par 14 Iar David a întrebat din nou pe Dumnezeu, iar Dumnezeu i-a zis: "Nu te duce de-a dreptul asupra lor; abate-te de la ei ?i te du asupra lor pe la dumbrava duzilor;
\par 15 ?i când vei auzi zgomot ca de pa?i prin vârful duzilor, atunci sa intri în lupta, caci a ie?it Dumnezeu înaintea ta, ca sa bata tabara Filistenilor".
\par 16 ?i a facut David cum îi poruncise Dumnezeu; ?i a lovit tabara Filistenilor de la Ghibeon pâna la Ghezer.
\par 17 Atunci a rasunat numele lui David prin toate ?arile dimprejur ?i l-a facut Domnul înfrico?ator pentru toate popoarele vecine.

\chapter{15}

\par 1 David ?i-a facut apoi case în cetatea lui, a pregatit loc pentru chivotul lui Dumnezeu ?i a facut pentru el un cort.
\par 2 Atunci David a zis: "Nimeni, afara de levi?i, nu trebuie sa poarte chivotul lui Dumnezeu ?i sa-I slujeasca Lui în veci".
\par 3 Atunci a adunat David pe to?i Israeli?ii la Ierusalim, ca sa duca chivotul lui Dumnezeu la locul lui, pe care i-l pregatise el.
\par 4 A chemat deci David pe fiii lui Aaron ?i pe levi?i ?i anume:
\par 5 Din urma?ii lui Cahat a chemat pe capetenia Uriel ?i pe fra?ii lui, o suta douazeci;
\par 6 Din urma?ii lui Merari a chemat pe capetenia Asaia ?i pe fra?ii lui, doua sute douazeci de oameni;
\par 7 Din urma?ii lui Gher?om a chemat pe capetenia Ioil ?i pe fra?ii lui, o suta treizeci de oameni;
\par 8 Din urma?ii lui Eli?afan a chemat pe capetenia ?emaia ?i pe fra?ii lui, doua sute de oameni;
\par 9 Din urma?ii lui Hebron a chemat pe capetenia Eliel ?i pe fra?ii lui, optzeci de oameni;
\par 10 Din urma?ii lui Uziel a chemat pe capetenia Aminadab ?i pe fra?ii lui, o suta douazeci de oameni.
\par 11 Apoi a chemat David pe preo?ii ?adoc ?i Abiatar ?i pe levi?ii Uriel, Asaia, Ioil, ?emaia, Eliel ?i Aminadab,
\par 12 ?i le-a zis: "Voi, capeteniile familiilor levite, sfin?i?i-va voi ?i fra?ii vo?tri ?i aduce?i chivotul Domnului Dumnezeului lui Israel la locul pe care l-am pregatit eu pentru el.
\par 13 Deoarece înainte n-a?i facut voi aceasta, Domnul Dumnezeul nostru ne-a lovit, pentru ca nu L-am cautat cum se cuvine".
\par 14 Atunci s-au sfin?it preo?ii ?i levi?ii, ca sa aduca chivotul Domnului Dumnezeului lui Israel.
\par 15 ?i au adus fiii levi?ilor chivotul Domnului, cum poruncise Moise dupa cuvântul Domnului, pe pârghii; pe umeri, iar nu cu caru?a.
\par 16 Apoi a poruncit David capeteniilor levi?ilor sa puna pe fra?ii lor cântare?i cu instrumente muzicale, cu psaltirioane, ca sa vesteasca cu glas tare de bucurie.
\par 17 Ace?tia au pus pe levi?ii: Heman, fiul lui Ioil, iar din fra?ii lui, pe Asaf, fiul lui Berechia. Din urma?ii lui Merari, fra?ii lor, au pus pe Etan, fiul lui Cu?aia.
\par 18 Iar pe fra?ii lor din a doua spi?a: Zaharia, fiul lui Iaaziel, ?emiramot, Iehiel, Uni, Eliab, Benaia, Maaseia, Matitia, Elifelehu, Micneia, Obed-Edom ?i Ieiel, i-au pus portari.
\par 19 Heman, Asaf ?i Etan cântau puternic din ?imbale de arama;
\par 20 Zaharia, Iaaziel, ?emiramot, Iehiel, Uni, Eliab, Maaseia ?i Benaia cântau din psaltirioanele cu sunete sub?iri.
\par 21 Matitia însa, Elifelehu, Micneia, Obed-Edom, Ieiel ?i Azazia faceau începutul cu harpele cu câte opt coarde.
\par 22 Iar Chenaia, capetenia levi?ilor, era dascal de cântari, pentru ca era iscusit în acestea.
\par 23 Berechia ?i Elcana erau u?ieri la chivot.
\par 24 Preo?ii ?ebania, Iosafat, Natanael, Amasai, Zaharia, Benaia ?i Eliezer trâmbi?au din trâmbi?e înaintea chivotului lui Dumnezeu. Obed-Edom ?i Iehia erau u?ieri la chivot.
\par 25 Astfel s-au dus David cu batrânii lui Israel ?i capeteniile cele peste mii sa aduca chivotul Domnului din casa lui Obed-Edom cu veselie.
\par 26 ?i dupa ce a ajutat Dumnezeu levi?ilor sa aduca chivotul Domnului, atunci au junghiat pentru jertfe ?apte vi?ei ?i ?apte berbeci.
\par 27 David era îmbracat cu ve?minte de vison, asemenea erau îmbraca?i ?i to?i levi?ii care aduceau chivotul ?i cântare?ii ?i Chenania, capetenia muzican?ilor ?i cântarelilor. David însa mai avea pe el ?i un efod de in.
\par 28 A?a tot Israelul a adus chivotul legamântului Domnului cu strigate de bucurie, cu sunete de corn, de trâmbi?e, de ?imbale ?i de harpe.
\par 29 Când chivotul legamântului Domnului a intrat în cetatea lui David, Micol, fata lui Saul, privea de la fereastra ?i, vazând pe regele David jucând ?i veselindu-se, l-a dispre?uit în inima sa.

\chapter{16}

\par 1 Astfel au adus chivotul lui Dumnezeu ?i l-au a?ezat în mijlocul cortului pe care-l facuse David pentru el ?i au înal?at lui Dumnezeu arderi de tot ?i jertfe de pace.
\par 2 Dupa ce David a ispravit de adus arderile de tot ?i jertfele de pace, a binecuvântat poporul în numele Domnului,
\par 3 ?i a împar?it tuturor Israeli?ilor, femei ?i barba?i, câte o pâine ?i câte o buca?ica de carne ?i câte o turta de struguri.
\par 4 Apoi a pus la slujba înaintea chivotului Domnului din levi?i, ca sa preaslaveasca, sa mul?umeasca ?i sa preaînal?e pe Domnul Dumnezeul lui Israel, ?i anume:
\par 5 Pe Asaf, capetenie; al doilea dupa el a pus pe Zaharia; apoi urmau Uziel, ?emiramot, Iehiel, Matitia, Eliab, Benaia, Obed-Edom ?i Ieiel cu psaltirioane ?i harpe, iar Asaf cânta din ?imbale.
\par 6 A pus de asemenea pe preo?ii Benaia ?i Oziel sa sune necontenit din trâmbi?e înaintea chivotului legamântului lui Dumnezeu.
\par 7 În aceasta zi David, pentru întâia oara, a dat, prin Asaf ?i fra?ii lui, urmatorul psalm de lauda Domnului:
\par 8 "Lauda?i pe Domnul ?i chema?i numele Lui; vesti?i între neamuri lucrurile Lui!
\par 9 Cânta?i, cânta?i în cinstea Lui! Spune?i toate minunile Lui!
\par 10 Lauda?i-va cu numele Lui cel sfânt! Sa se bucure inima celor ce-L cauta pe El!
\par 11 Alerga?i la Domnul ?i la ajutorul Lui; cauta?i pururea fa?a Lui!
\par 12 Neamul lui Israel, sluga Lui, fiii lui Iacov, ale?ii Lui,
\par 13 Aduce?i-va aminte de minunile Lui, de semnele Lui ?i de judeca?ile gurii Lui!
\par 14 Caci El este Domnul Dumnezeul nostru ?i dreptatea Lui este peste tot pamântul.
\par 15 Aduce?i-va aminte de a?ezamântul Lui, de fagaduin?a data pentru mii de neamuri.
\par 16 De legamântul facut cu Avraam ?i de juramântul Sau catre Isaac,
\par 17 Juramânt pus ca o lege pentru Iacov, ?i ca un legamânt ve?nic pentru Israel,
\par 18 Zicând: ?ie-?i voi da pamântul Canaan, ca partea voastra de mo?tenire.
\par 19 Ei atunci erau pu?ini la numar ?i neînsemna?i, ?i straini în ?ara aceasta.
\par 20 ?i treceau de la popor la popor ?i dintr-o împara?ie la alta împara?ie.
\par 21 Dar El n-a lasat pe nimeni sa-i apese, ?i a pedepsit regi pentru ei, zicând:
\par 22 Nu va atinge?i de un?ii Mei ?i proorocilor Mei sa nu le face?i rau.
\par 23 Cânta?i Domnului tot pamântul, binevesti?i din zi în zi izbavirea Lui!
\par 24 Vestiri pagânilor slava Lui, spune?i la toate popoarele minunile Lui!
\par 25 Ca mare este Domnul ?i vrednic de lauda ?i mai înfrico?at decât to?i dumnezeii.
\par 26 Ca to?i dumnezeii pagânilor sunt nimic, iar Domnul a facut cerurile.
\par 27 Înaintea Lui este stralucire ?i mare?ie, putere ?i bucurie în loca?ul Lui cel sfânt.
\par 28 Da?i Domnului, neamuri pagâne, da?i Domnului slava ?i cinste!
\par 29 Da?i Domnului slava pentru numele Lui; aduce?i-va darul, merge?i înaintea fe?ei Lui, închina?i-va Domnului în podoabele sfin?eniei Lui!
\par 30 Sa tremure înaintea Lui tot pamântul, ca El a întemeiat lumea ?i nu se va clatina.
\par 31 Sa se bucure cerurile ?i sa praznuiasca pamântul, iar printre popoare sa se spuna: Domnul este Împarat!
\par 32 Sa se zguduie marea ?i toate cele din ea; câmpia ?i toate cele de pe ea sa se veseleasca!
\par 33 Sa dan?uiasca împreuna to?i copacii dumbravii înaintea fe?ei Domnului, ca vine sa judece pamântul.
\par 34 Lauda?i pe Domnul, ca în veac este mila Lui!
\par 35 Zice?i: Izbave?te-ne pe noi, Dumnezeule, Izbavitorul nostru! Aduna-ne ?i ne izbave?te de prin popoare, ca sa slavim sfânt numele Tau ?i sa ne laudam cu slava Ta!
\par 36 Binecuvântat fie Domnul Dumnezeul lui Israel din veac în veac!" ?i tot poporul a zis: "Amin! Aliluia!"
\par 37 ?i a lasat David acolo, înaintea chivotului legamântului Domnului, pe Asaf ?i pe fra?ii lui, ca sa slujeasca ei înaintea chivotului neîncetat, în fiecare zi;
\par 38 Pe Obed-Edom ?i pe fra?ii lui, ?aizeci ?i opt de oameni; pe Obed-Edom, fiul lui Iedutun ?i pe Hosa, i-a lasat u?ieri.
\par 39 Iar pe preotul ?adoc ?i pe ceilal?i preo?i, fra?ii sai, i-a pus înaintea loca?ului Domnului, cel de pe înal?imea din Ghibeon,
\par 40 Ca sa înal?e arderi de tot Domnului pe jertfelnicul arderilor de tot neîncetat, diminea?a ?i seara, pentru toate cele scrise în legea Domnului, pe care El le-a poruncit lui Israel.
\par 41 ?i cu ei a lasat pe Heman, pe Iedutun ?i pe ceilal?i ale?i, care au fost numi?i pe nume, ca sa slaveasca pe Domnul, ca în veac este mila Lui.
\par 42 Împreuna cu ei, Heman ?i Iedutun preaslaveau pe Dumnezeu, cântând din trâmbi?e ?i felurite instrumente muzicale; pe fiii lui Iedutun i-a pus la poarta.
\par 43 Apoi s-a dus tot poporul, fiecare la casa sa. De asemenea s-a întors ?i David, ca sa binecuvânteze casa sa.

\chapter{17}

\par 1 Când David locuia în casa sa, a zis el catre Natan proorocul: "Iata eu traiesc în casa de cedru, iar chivotul legamântului Domnului este în cort".
\par 2 Iar Natan a zis catre David: "Fa tot ce ai la inima, ca Dumnezeu este cu tine!"
\par 3 Dar în aceea?i noapte a fost cuvântul Domnului catre Natan ?i i-a zis:
\par 4 "Mergi ?i spune robului Meu David: A?a zice Domnul: Nu tu ai sa-Mi zide?ti Mie casa de locuit,
\par 5 Caci Eu n-am locuit în casa din ziua în care am scos pe fiii lui Israel ?i pâna astazi, ci am umblat din cort în cort ?i din loca? în loca?.
\par 6 Oriunde am mers Eu cu tot Israelul, spus-am Eu oare macar un cuvânt cuiva din judecatorii lui Israel, carora le-am poruncit sa pastoreasca pe poporul Meu, pentru ce nu-Mi zide?ti Mie casa de cedru?
\par 7 ?i acum a?a sa spui robului Meu David: Eu te-am luat de la turma de oi, ca sa fii conducatorul poporului Meu Israel;
\par 8 ?i am fost cu tine pretutindeni, oriunde ai umblat; am stârpit pe to?i vrajma?ii tai înaintea fe?ei tale ?i am facut numele tau ca numele celor puternici ai pamântului.
\par 9 Am rânduit loc pentru poporul Meu Israel ?i l-am înradacinat, ?i va trai el în pace la locul sau, ?i nu va mai fi nelini?tit ?i necredincio?ii nu-l vor mai strâmtora ca altadata,
\par 10 Ca în zilele acelea, când puneam judecatori peste poporul Meu Israel; dar am supus pe to?i vrajma?ii tai ?i ?i-am vestit ca Domnul ?i-a pregatit ?ie casa.
\par 11 Când se vor împlini zilele tale ?i vei trece la parin?ii tai, atunci Eu voi ridica pe urma?ul tau dupa tine, ?i voi întari domnia ta.
\par 12 Acela Îmi va zidi Mie casa ?i voi întemeia tronul lui pe veci.
\par 13 Eu îi voi fi tata ?i el Îmi va fi fiu ?i mila Mea nu o voi lua de la el, cum am luat-o de la cel ce a fost înaintea ta.
\par 14 Îl voi pune pe acela în casa Mea ?i în împara?ia Mea pe veci ?i tronul lui în veci va fi tare".
\par 15 Toate cuvintele ?i toata vedenia aceasta le-a spus Natan lui David.
\par 16 Atunci a venit regele David ?i a stat înaintea fe?ei Domnului ?i a zis: "Cine sunt eu, Doamne Dumnezeule, ?i ce este casa mea, de m-ai înal?at a?a?
\par 17 Dar ?i aceasta s-a parut înca pu?in în ochii Tai, Dumnezeule, caci iata veste?ti despre casa robului Tau în viitor ?i prive?ti la mine, ca la un om mare, Doamne Dumnezeule!
\par 18 Ce mai poate adauga David înaintea Ta pentru marirea robului Tau? Tu cuno?ti pe robul Tau.
\par 19 Doamne, pentru robul Tau, dupa inima Ta, faci toate aceste lucruri mari, ca sa ara?i toata marirea.
\par 20 Doamne, nu este altul asemenea ?ie, ?i nu este Dumnezeu afara de Tine, dupa câte am auzit noi.
\par 21 ?i nu este înca alt popor pe pamânt ca poporul Tau Israel, pe care l-a calauzit Dumnezeu, ca sa-l rascumpere Sie?i de popor, sa-?i faca nume mare ?i stralucit, izgonind popoarele de la fa?a poporului Tau, pe care l-ai izbavit din Egipt.
\par 22 Tu ai facut pe poporul Tau, Israel, poporul Tau pe veci ?i Tu, Doamne, Te-ai facut Dumnezeul lui.
\par 23 A?adar, Doamne, cuvântul pe care l-ai grait Tu acum despre robul Tau ?i despre casa lui, întare?te-l pe veci, ?i fa cum ai zis Tu.
\par 24 Sa ramâna ?i sa se preamareasca numele Tau în veci, ca sa se zica ca Domnul Savaot, Dumnezeul lui Israel, este Dumnezeu peste Israel, ?i casa robului Tau David sa fie tare înaintea fe?ei Tale.
\par 25 Caci Tu, Dumnezeul meu, ai descoperit robului Tau ca-i vei zidi casa, de aceea robul Tau a ?i îndraznit sa se roage înaintea Ta.
\par 26 ?i acum, Doamne, Tu e?ti adevaratul Dumnezeu ?i Tu ai vestit despre robul Tau astfel de lucruri bune.
\par 27 Începe dar a binecuvânta casa robului Tau, ca sa fie ea ve?nica înaintea fe?ei Tale. Caci daca Tu, Doamne, o vei binecuvânta, binecuvântata va fi ea în veci".

\chapter{18}

\par 1 Dupa aceasta a lovit David pe Filisteni ?i i-a supus ?i a luat cetatea Gat ?i ora?ele ce ?ineau de ea din mâinile Filistenilor.
\par 2 A lovit el de asemenea ?i pe Moabi?i ?i au ajuns Moabi?ii robii lui David, platindu-i tribut.
\par 3 Apoi a lovit David pe Hadadezer, regele ?obei, la Hamat, când mergea acela sa-?i întareasca stapânirea la râul Eufrat.
\par 4 ?i a luat David de la el o mie de care de razboi, ?apte mii de calare?i ?i douazeci de mii de pedestra?i. ?i a stricat David toate carele, neoprind din ele decât numai o suta.
\par 5 Sirienii din Damasc ar fi venit în ajutor lui Hadadezer, regele ?obei, dar David a batut douazeci ?i doua de mii de Sirieni.
\par 6 Apoi a a?ezat David o?tire de paza în Siria Damascului ?i au ajuns Sirienii robii lui David, platindu-i tribut. Domnul a ajutat lui David pretutindeni, oriunde a mers el.
\par 7 Atunci a luat David scuturile de aur, care erau la robii lui Hadadezer, ?i le-a adus în Ierusalim.
\par 8 Iar din Tibhat ?i din Cun, ceta?ile lui Hadadezer, a luat David foarte multa arama. Din arama aceasta a facut Solomon marea cea de arama, stâlpii ?i vasele cele de arama.
\par 9 Auzind Tou, regele din Hamat, ca David a batut toata armata lui Hadadezer, regele ?obei,
\par 10 A trimis pe Hadoram, fiul sau, la regele David, sa-l salute ?i sa-i mul?umeasca, pentru ca s-a razboit cu Hadadezer ?i l-a batut, caci Tou era în razboi cu Hadadezer, ?i a trimis cu el tot felul de vase de aur, de argint ?i de arama.
\par 11 Regele David a închinat aceste vase Domnului, împreuna cu aurul ?i argintul pe care-l luase el de la toate popoarele: de la Edomi?i, Moabi?i, Amoni?i, Filisteni ?i Amaleci?i.
\par 12 ?i Abi?ai, fiul ?eruiei, a batut optsprezece mii de Edomi?i în Valea Sarata.
\par 13 ?i a pus în Edom oaste de paza ?i s-au facut Edomi?ii robii lui David, caci Domnul ajuta lui David oriunde mergea.
\par 14 ?i a domnit David peste tot Israelul ?i a facut judecata ?i dreptate la tot poporul sau.
\par 15 Ioab, fiul ?eruiei, era comandantul o?tirii, iar Iosafat, fiul lui Ahilud, era cronicar.
\par 16 ?adoc, fiul lui Ahitub, ?i Ahimelec, fiul lui Abiatar, au fost preo?i, iar ?ausa (Serais) a fost secretar.
\par 17 Benaia, fiul lui Iehoiada, era capetenie peste Cheretieni ?i Peletieni, iar fiii lui David erau cei întâi pe lânga rege,

\chapter{19}

\par 1 Dupa aceasta a murit Naha?, regele Amoni?ilor, ?i în locul lui s-a facut rege fiul sau.
\par 2 Atunci David a zis: "Am sa arat mila lui Hanun, fiul lui Naha?, pentru binefacerea ce mi-a aratat tatal sau". ?i a trimis David soli sa-l mângâie pentru pierderea tatalui sau. Au mers deci solii lui David în ?ara Amoni?ilor, ca sa-l mângâie pe Hanun.
\par 3 Însa capeteniile Amoni?ilor au zis catre Hanun: "Socote?ti tu, oare, ca David din dragoste pentru tatal tau a trimis la tine mângâietori? Nu cumva au venit slugile lui la tine, ca sa iscodeasca ?i sa vada ?ara ?i apoi sa o pustiiasca "
\par 4 Atunci a prins Hanun pe trimi?ii lui David ?i i-a ras ?i le-a taiat hainele pe jumatate, pâna la coapsa, ?i a?a le-a dat drumul ?i ei au plecat.
\par 5 Spunându-se lui David de pa?ania oamenilor acelora, a trimis el în întâmpinarea lor, ca erau tare batjocori?i, ?i li s-a zis: "Ramâne?i în Ierihon pâna va vor cre?te barbile ?i atunci va ve?i întoarce acasa".
\par 6 Dupa ce Amoni?ii au vazut ca au ajuns urâ?i lui David, au trimis Hanun ?i Amoni?ii o mie de talan?i de argint, ca sa ia în solda lor ni?te care de razboi ?i calare?i din Siria Mesopotamiei, Siria Maacai ?i din ?oba.
\par 7 ?i au luat în solda lor treizeci ?i doua de mii de care ?i pe regele Maacai cu poporul lui, care au venit ?i au tabarât înaintea Medebei. Iar Amoni?ii s-au adunat din ceta?ile lor ?i au ie?it la razboi.
\par 8 Când David a auzit de acestea, a trimis pe Ioab cu toata o?tirea de viteji.
\par 9 Atunci au înaintat Amoni?ii ?i s-au a?ezat în linie de bataie la por?ile ceta?ii, iar regii care venisera erau la o parte în câmp.
\par 10 Ioab, vazând ca are a lupta pe doua laturi, una în fa?a ?i alta în spate, a ales o?teni din to?i cei mai de seama din Israel ?i i-a rânduit contra Sirienilor.
\par 11 Iar cealalta parte de popor a încredin?at-o lui Abi?ai, fratele sau, ca sa se îndrepte contra Amoni?ilor.
\par 12 Apoi a zis: "Daca Sirienii vor fi mai tari decât mine, sa-mi vii tu în ajutor, iar daca Amoni?ii vor fi mai tari decât tine, î?i voi veni eu în ajutor.
\par 13 Fii curajos ?i sa stam cu tarie pentru apararea poporului nostru ?i pentru ceta?ile Dumnezeului nostru, ?i Domnul sa faca ce va binevoi".
\par 14 ?i a intrat Ioab ?i oamenii ce erau cu el în lupta cu Sirienii, dar ace?tia au fugit de el.
\par 15 Amoni?ii însa, vazând ca Sirienii fug, au fugit ?i ei de Abi?ai, fratele lui Ioab, ?i s-au dus în cetate. Atunci Ioab a venit la Ierusalim.
\par 16 Sirienii, vazând ca sunt batu?i de Israeli?i, au trimis soli ?i au scos pe Sirienii care erau dincolo de râu, iar ?ofac, capetenia lui Hadadezer, îi conducea.
\par 17 Când s-a spus aceasta lui David, el a adunat pe to?i Israeli?ii, a trecut Iordanul ?i, venind asupra acelora, s-a a?ezat în linie de bataie în fa?a lor. ?i a intrat David în lupta cu Sirienii ?i ace?tia s-au luptat cu el.
\par 18 Dar curând Sirienii au fugit de Israeli?i, iar David, a nimicit Sirienilor ?apte mii de care ?i patruzeci de mii de pedestra?i, ?i pe ?ofac, comandantul o?tirii, I-a ucis.
\par 19 Când au vazut slugile lui Hadadezer ca sunt birui?i de Israeli?i, au încheiat pace cu David ?i s-au supus. ?i n-au mai voit Sirienii sa mai ajute pe Amoni?i.

\chapter{20}

\par 1 Peste un an, pe vremea când regii ies la razboi, a scos Ioab o?tirea ?i a început sa pustiiasca ?ara Amoni?ilor ?i a venit ?i a înconjurat Raba. Însa David a ramas în Ierusalim. Ioab a cucerit Raba ?i a darâmat-o.
\par 2 ?i a luat David coroana regelui lor de pe capul lui ?i s-a aflat ca are aur în greutate de un talant ?i erau pe ea ?i pietre scumpe; ?i s-a pus coroana aceasta pe capul lui David. ?i au fost scoase din cetatea aceea ?i foarte multe prazi.
\par 3 Iar poporul care era în ea a fost scos ?i omorât cu fierastraie, cu ciocane de fier ?i cu securi. A?a a facut David cu toate ora?ele Amoni?ilor ?i apoi s-a întors el ?i tot poporul la Ierusalim.
\par 4 Dupa aceea s-a început razboiul cu Filistenii la Ghezer. Atunci Sibecai Hu?atitul a batut pe Sipai, unul din urma?ii Refaimilor, ?i s-au supus ?i ei.
\par 5 Apoi iar a fost razboi cu Filistenii. Dar Elhanan, fiul lui Iair, a lovit pe Lahmi, fratele lui Goliat Gateul; coada suli?ei lui era ca a sulului de la razboiul de ?esut.
\par 6 ?i a mai fost o lupta la Gat. Acolo era un om înalt care avea câte ?ase degete la mâini ?i la picioare, adica de toate douazeci ?i patru. ?i acesta era tot din urma?ii Refaimilor.
\par 7 El batjocorea pe Israel, dar Ionatan, fiul lui ?ama, fratele lui David, l-a ucis.
\par 8 Ace?tia erau oameni nascu?i din Refaimi în Gat ?i au cazut de mâna lui David ?i de mâna oamenilor lui.

\chapter{21}

\par 1 Atunci s-a sculat Satana împotriva lui Israel ?i a îndemnat pe David sa faca numaratoarea Israeli?ilor.
\par 2 Deci a zis David catre Ioab ?i catre capeteniile poporului: "Merge?i ?i numara?i pe Israeli?i de la Beer-?eba pâna la Dan ?i-mi aduce?i raspuns ca sa ?tiu numarul lor!"
\par 3 Ioab însa a zis: "Sa înmul?easca Domnul pe poporul Sau de o suta de ori mai mult decât este el acum! Au doara nu sunt ei to?i, o, rege, domnul meu, robii stapânului meu? Pentru ce dar cere aceasta domnul meu? Oare pentru a se scoate asta ca o vina lui Israel?"
\par 4 Dar cuvântul regesc biruind pe Ioab, s-a dus acesta de a cutreierat tot Israelul ?i venind la Ierusalim,
\par 5 A dat Ioab lui David catagrafia înscrierii poporului ?i s-au aflat în tot Israelul un milion ?i o suta de mii de barba?i destoinici de razboi, iar în Iuda, patru sute ?aptezeci de mii în stare de a lua parte la razboi.
\par 6 Pe levi?i însa ?i pe Veniamineni el nu i-a numarat împreuna cu ei, pentru ca cuvântul regelui nu placuse lui Ioab.
\par 7 Lucrul acesta n-a fost placut nici înaintea lui Dumnezeu ?i de aceea a lovit El pe Israel.
\par 8 Atunci a zis David catre Dumnezeu: "Am gre?it mult, facând aceasta; iarta dar vina robului Tau, ca m-am purtat cu totul nepriceput".
\par 9 Iar Domnul a grait cu Gad, proorocul lui David ?i i-a zis:
\par 10 "Mergi ?i spune lui David: A?a zice Domnul: Î?i pun înainte trei pedepse; alege-?i una din ele ?i o voi trimite asupra ta".
\par 11 A venit deci Gad la David ?i i-a zis: "A?a graie?te Domnul, alege:
\par 12 Sau trei ani de foamete, sau trei luni sa fii tu urmarit de vrajma?ii tai ?i sabia du?manilor sa ajunga pâna la tine, sau trei zile sabia Domnului ?i molima sa fie pe pamânt ?i îngerul Domnului sa pustiiasca în toate hotarele lui Israel. Vezi acum ce trebuie sa raspund Celui ce m-a trimis cu acest cuvânt".
\par 13 ?i a raspuns David lui Gad: "Sunt într-o mare nelini?te! Sa cad mai bine în mâinile Domnului, caci îndurarile Lui sunt foarte multe, dar sa nu cad în mâinile oamenilor".
\par 14 Atunci a trimis Domnul molima asupra lui Israel ?i au murit ?aptezeci de mii de Israeli?i.
\par 15 ?i a trimis Dumnezeu îngerul la Ierusalim ca sa-l piarda. ?i când a început el sa pustiiasca, a vazut Domnul ?i I s-a facut mila pentru aceasta nenorocire ?i a zis catre îngerul pierzator: "Destul! De acum lasa-?i mâinile în jos!" Îngerul Domnului statea atunci deasupra ariei lui Ornan (Aravna) Iebuseul.
\par 16 Atunci ridicându-?i David ochii sai, a vazut pe îngerul Domnului stând între pamânt ?i cer cu sabia goala în mâna sa, întinsa asupra Ierusalimului; ?i a cazut David ?i batrânii cu fe?ele la pamânt, îmbraca?i în sac.
\par 17 ?i a zis David catre Dumnezeu: "Oare nu eu am poruncit sa se numere poporul? Eu am gre?it, eu am facut rau; dar aceste oi ce au facut? Doamne Dumnezeul meu, sa vina mâna Ta asupra mea, asupra casei tatalui meu, iar nu asupra poporului Tau, ca sa-l piarda pe el!"
\par 18 Iar îngerul Domnului a zis lui Gad proorocul sa spuna lui David: "Sa se suie David ?i sa ridice un jertfelnic Domnului în aria lui Ornan Iebuseul".
\par 19 ?i s-a dus David, dupa cuvântul lui Gad pe care i-l graise în numele Domnului.
\par 20 Ornan, întorcându-se, a vazut îngerul, ?i cei trei fii ai lui s-au ascuns împreuna cu el; în vremea aceea Ornan treiera.
\par 21 A venit deci David la Ornan; iar Ornan, privind ?i vazând pe David, a ie?it din arie ?i s-a plecat înaintea lui David cu fa?a pâna la pamânt.
\par 22 Atunci David a zis catre Ornan: "Da-mi un loc în arie, ca am sa fac pe el jertfelnic Domnului; da-mi-l cu pre?ul cât costa, ca sa înceteze prapadul în popor".
\par 23 Iar Ornan a zis lui David: "Ia-?i! Faca domnul meu regele ce binevoie?te! Iata eu dau ?i boi pentru ardere de tot ?i uneltele de treierat ca lemne pentru foc ?i grâu pentru prinos. Toate acestea le dau în dar!"
\par 24 Regele David a zis catre Ornan: "Nu, eu voiesc sa cumpar de la tine cu pre? adevarat, caci nu ma voi apuca sa aduc avutul tau Domnului ?i nu voi aduce ardere de tot vite luate în dar".
\par 25 ?i a dat David lui Ornan pentru acest loc ?ase sute de sicli de aur.
\par 26 Apoi a facut David acolo jertfelnic Domnului ?i a înal?at ardere de tot ?i jertfe de împacare ?i a chemat pe Domnul ?i Dumnezeu l-a auzit trimi?ând foc din cer pe altarul arderii de tot.
\par 27 Atunci a zis Domnul catre înger: "Pune-?i sabia în teaca".
\par 28 În vremea aceasta, vazând David ca Domnul l-a auzit în aria lui Ornan Iebuseul, a adus acolo jertfa.
\par 29 Cortul Domnului însa, pe care-l facuse Moise în pustiu ?i altarul arderii de tot se aflau în vremea aceea pe înal?imea de la Ghibeon.
\par 30 ?i nu s-a putut duce David acolo, ca sa întrebe pe Domnul, pentru ca era îngrozit de sabia îngerului Domnului.

\chapter{22}

\par 1 Apoi a zis David: "Iata templul Domnului Dumnezeu ?i iata jertfelnicul arderilor de tot al lui Israel!"
\par 2 ?i a poruncit David sa aduca pe strainii din pamântul lui Israel ?i i-a pus cioplitori de piatra ca sa ciopleasca pietre pentru zidirea templului Domnului.
\par 3 Apoi a pregatit David fier foarte mult pentru piroane la ferecarea u?ilor ?i pentru legaturi; arama atât de multa încât nu se mai putea cântari;
\par 4 Lemne de cedru, nemasurat, pentru ca Sidonienii ?i Tirienii adusesera lui David foarte mult lemn de cedru.
\par 5 Dupa aceea a zis David: "Solomon, fiul meu, e tânar ?i cu pu?ina putere, iar templul care are a se zidi pentru Domnul trebuie sa fie foarte mare?, spre slava ?i podoaba a toata lumea; deci voi pregati eu toate pentru el". ?i a pregatit David multe pâna la moartea sa.
\par 6 Chemând apoi pe fiul sau Solomon, i-a poruncit sa ridice templul Domnului Dumnezeului lui Israel.
\par 7 ?i a zis David lui Solomon: "Fiul meu, eu am avut la inima sa zidesc templu în numele Domnului Dumnezeului meu;
\par 8 Dar a fost catre mine cuvântul Domnului ?i a zis: Tu ai varsat mult sânge ?i ai purtat razboaie mari; nu se cuvine sa zide?ti tu casa numelui Meu, pentru ca ai varsat mult sânge pe pamânt înaintea fe?ei Mele.
\par 9 Iata însa ?ie ?i se va na?te un fiu; acela va fi pa?nic ?i Eu îi voi da lini?te din partea tuturor vrajma?ilor dimprejur; de aceea numele lui va fi Solomon. ?i voi da lui Israel în zilele lui pace ?i lini?te.
\par 10 El va zidi templu numelui Meu ?i el Îmi va fi fiu, iar Eu îi voi fi tata ?i voi întari tronul domniei lui peste Israel pe veci.
\par 11 ?i acum, fiul meu, sa fie Domnul cu tine, ca sa ai spor ?i sa zide?ti templul Domnului Dumnezeului tau, cum a vorbit El despre tine.
\par 12 Sa-?i dea ?ie Domnul minte ?i în?elepciune ?i sa te puna peste Israel! Sa paze?ti legile Domnului Dumnezeului tau!
\par 13 Atunci vei fi cu spor, de te vei sili sa împline?ti a?ezamintele ?i legile pe care le-a dat Domnul lui Moise pentru Israel. Fii tare ?i curajos, nu te teme ?i nu te deznadajdui.
\par 14 ?i iata eu din saracia mea am pregatit pentru templul Domnului o suta de mii de talan?i de aur ?i un milion de talan?i de argint, iar fier ?i arama fara numar, pentru ca acestea sunt foarte multe; ?i lemn ?i piatra de asemenea am pregatit ?i sa mai adaugi ?i tu la acestea.
\par 15 Ai mul?ime de lucratori ?i cioplitori de piatra, sculptori, dulgheri ?i tot felul de oameni pricepu?i la tot felul de lucruri.
\par 16 Aur, argint, arama ?i fier ai cât nu se pot cântari; începe ?i fa! Domnul va fi cu tine".
\par 17 Apoi a poruncit David tuturor mai-marilor lui Israel sa ajute lui Solomon, fiul sau, zicând:
\par 18 "Au nu este cu voi Domnul Dumnezeul vostru, Care v-a daruit pace din toate par?ile, pentru ca El a dat în mâinile mele pe locuitorii ?arii ?i s-a supus ?ara înaintea Domnului ?i înaintea poporului Sau?
\par 19 A?adar îndrepta?i-va inima ?i sufletul vostru, ca sa caute pe Domnul Dumnezeul vostru! Scula?i-va ?i zidi?i loca? sfânt Domnului Dumnezeu, ca sa muta?i chivotul legamântului Domnului ?i vasele sfinte ale lui Dumnezeu în loca?ul zidit numelui Domnului".

\chapter{23}

\par 1 Îmbatrânind David ?i saturându-se de via?a, a facut rege peste Israel pe fiul sau Solomon.
\par 2 A adunat pe toate capeteniile lui Israel ?i pe preo?i ?i pe levi?i.
\par 3 ?i au fost numara?i levi?ii, de la treizeci de ani în sus ?i s-a aflat numarul lor, numara?i pe cap, treizeci ?i opt de mii de barba?i.
\par 4 Dintre ei au fost rândui?i pentru lucru în templul Domnului douazeci ?i patru de mii; ?ase mii sa fie scriitori ?i judecatori,
\par 5 Patru mii sa fie portari ?i patru mii sa laude pe Domnul cu instrumentele muzicale pe care el le facuse pentru aceasta.
\par 6 ?i i-a împar?it David în cete, care sa faca de rând, dupa fiii lui Levi: Gher?on, Cahat ?i Merari.
\par 7 Din Gher?oni?i erau Laedan ?i ?imei.
\par 8 Fiii lui Laedan au fost trei: Iehiel, capetenie, Zetam ?i Ioil.
\par 9 Fiii lui ?imei au fost trei: ?elomit, Haziel ?i Haran. Ace?tia sunt capeteniile familiilor lui Laedan.
\par 10 Tot fii ai lui ?imei au mai fost: Iahat, Ziza, Ieu? ?i Beraia. Ace?ti patru sunt tot fii ai lui ?imei.
\par 11 Iahat a fost capetenie; Ziza era al doilea; Ieu? ?i Beraia au avut pu?ini copii ?i de aceea ei au fost socoti?i la un loc la casa tatalui lor.
\par 12 Fiii lui Cahat au fost patru: Amram, I?har, Hebron ?i Uziel.
\par 13 Fiii lui Amram au fost: Aaron ?i Moise. Aaron a fost ales ca sa fie sfin?it pentru Sfânta Sfintelor, el ?i fiii lui pe veci, pentru a savâr?i tamâierea înaintea fe?ei Domnului, ca sa-I slujeasca Lui ?i sa binecuvânteze numele Lui în veci.
\par 14 Iar Moise, omul lui Dumnezeu ?i fiii lui au fost numara?i la tribul lui Levi.
\par 15 Fiii lui Moise au fost: Gher?om ?i Eliezer.
\par 16 Fiul lui Gher?om a fost ?ebuel, capetenia.
\par 17 Fiul lui Eliezer a fost: Rehabia, capetenia. Eliezer n-a mai avut al?i copii. Rehabia însa a avut foarte mul?i copii.
\par 18 Fiul lui I?har a fost ?elomit, capetenia.
\par 19 Fiii lui Hebron au fost: întâiul Ieria, al doilea Amaria, al treilea Iahaziel ?i al patrulea Iecameam.
\par 20 Fiii lui Uziel au fost: întâiul Mica ?i al doilea I?ia.
\par 21 Fiii lui Merari au fost: Mahli ?i Mu?i; fiii lui Mahli au fost: Eleazar ?i Chi?.
\par 22 Eleazar însa a murit ?i n-a avut feciori, ci numai fete ?i le-au luat de so?ii fiii lui Chi?, verii lor.
\par 23 Fiii lui Mu?i au fost trei: Mahli, Eder ?i Ieremot.
\par 24 Ace?tia sunt fiii lui Levi, dupa casele parin?ilor lor, adica a capilor de familie, dupa numaratoarea lor pe nume ?i pe capete, care faceau slujba la templul Domnului de la douazeci de ani în sus.
\par 25 Caci David a zis: "Domnul Dumnezeul lui Israel a dat lini?te poporului Sau ?i l-a a?ezat în Ierusalim pe veci,
\par 26 ?i levi?ii nu vor mai duce cortul ?i tot felul de lucruri ale lui pentru slujbele din el".
\par 27 De aceea, dupa cele din urma porunci ale lui David, au fost numara?i levi?ii de la douazeci de ani în sus,
\par 28 Ca sa fie pe lânga fiii lui Aaron, pentru a sluji la templul Domnului, în curte ?i în camerele din jur, pentru cura?irea a tot ceea ce este sfânt ?i pentru savâr?irea slujbelor în templul lui Dumnezeu,
\par 29 Pentru a îngriji de pâinile punerii înainte, de faina de grâu pentru prinosul de pâine ?i azime, pentru cele de copt, de fript ?i de toata masura ?i cântarul,
\par 30 ?i pentru a sta diminea?a ?i seara sa laude ?i sa slaveasca pe Domnul,
\par 31 ?i la toate arderile de tot aduse Domnului sâmbata, la luna noua ?i la sarbatori, dupa numar, cum este scris pentru ele, sa fie totdeauna înaintea Domnului,
\par 32 ?i ca sa pazeasca chivotul legii ?i loca?ul sfânt ?i pe fiii lui Aaron, fra?ii lor, la slujbele din templul Domnului.

\chapter{24}

\par 1 Iata acum cetele în care au fost împar?i?i fiii lui Aaron: Fiii lui Aaron au fost: Nadab, Abiud, Eleazar ?i Itamar.
\par 2 Nadab ?i Abiud au murit înainte de tatal lor, iar copii n-au avut ?i de aceea au preo?it numai pe Eleazar ?i Itamar.
\par 3 David i-a împar?it astfel: Pe ?adoc, unul din fiii lui Eleazar ?i pe Ahimelec, unul din fiii lui Itamar, i-a pus sa faca slujbele cu rândul.
\par 4 S-au gasit însa între fiii lui Eleazar mai multe capetenii decât între fiii lui Itamar ?i i-a împar?it astfel: Din fiii lui Eleazar ?aisprezece capi de familii, iar din fiii lui Itamar opt.
\par 5 ?i i-a împar?it prin sor?i, pentru ca cei mai însemna?i în loca?ul sfânt ?i înaintea lui erau dintre fiii lui Eleazar ?i dintre fiii lui Itamar.
\par 6 ?i i-a înscris ?emaia, fiul lui Natanael, scriitor din levi?i, înaintea fe?ei regelui ?i a capeteniilor, înaintea preo?ilor ?adoc ?i Ahimelec, fiul lui Abiatar ?i înaintea capilor de familie ai preo?ilor ?i levi?ilor, luând prin tragere la sor?i o familie din neamul lui Eleazar ?i apoi una din neamul lui Itamar.
\par 7 Întâiul sor? a cazut lui Iehoiarib, al doilea lui Iedaia,
\par 8 Al treilea lui Harim, al patrulea lui Seorim,
\par 9 Al cincilea lui Malchia, al ?aselea lui Miiamin,
\par 10 Al ?aptelea lui Haco?, al optulea lui Abia,
\par 11 Al noualea lui Ie?ua, al zecelea lui ?ecania,
\par 12 Al unsprezecelea lui Elia?ib, al doisprezecelea lui Iachim,
\par 13 Al treisprezecelea lui Hupa, al paisprezecelea lui I?baal,
\par 14 Al cincisprezecelea lui Bilga, al ?aisprezecelea lui Imer,
\par 15 Al ?aptesprezecelea lui Hezir, al optsprezecelea lui Hapi?e?,
\par 16 Al nouasprezecelea lui Petahia, al douazecilea lui Iezechiel,
\par 17 Al douazeci ?i unulea lui Iachin, al douazeci ?i doilea lui Gamul,
\par 18 Al douazeci ?i treilea lui Delaia ?i al douazeci ?i patrulea lui Maazia.
\par 19 Aceasta era în?irarea lor la slujba, cum trebuia sa vina în templul Domnului, dupa rânduiala lor data prin Aaron, tatal lor, cum poruncise acestuia Domnul Dumnezeul lui Israel.
\par 20 Ceilal?i fiii ai lui Levi au fost împar?i?i astfel: Din fiii lui Amram: ?ubael; din fiii lui ?ubael: Iehdia;
\par 21 Din fiii lui Rehabia, întâiul era I?ia;
\par 22 Din ai lui I?har, ?elomot; din ai lui ?elomot era Iahat;
\par 23 Din ai lui Hebron întâiul era Ieria, al doilea, Amaria, al treilea, Iahaziel, al patrulea, Iecameam.
\par 24 Din ai lui Uziel era Mica; din ai lui Mica era ?amir.
\par 25 Fratele lui Mica a fost I?ia; din fiii lui I?ia era Zaharia.
\par 26 Fiii lui Merari au fost: Mahli ?i Mu?i; din fiii lui Iaazia a fost Beno;
\par 27 Din fiii lui Merari, dupa Iaazia, au fost: Beno, ?oham, Zacur ?i Ibri.
\par 28 Mahli a avut pe Eleazar, iar acesta n-a avut fii.
\par 29 Chi? a avut pe Ierahmeel.
\par 30 Fiii lui Mu?i au fost: Mahli, Eder ?i Ierimot. Ace?tia sunt fiii levi?ilor, dupa familii.
\par 31 Au aruncat ?i ei sor?i la fel ca ?i fra?ii lor, fiii lui Aaron, înaintea fe?ei regelui David, a lui ?adoc ?i Ahimelec, ?i a capilor familiilor preo?e?ti ?i levite, fara sa se faca deosebire între cei batrâni ?i cei tineri.

\chapter{25}

\par 1 Apoi David ?i capeteniile o?tirii au împar?it la slujba pe fiii lui Asaf, ai lui Heman ?i ai lui Iedutun, ca sa prooroceasca acompania?i de harfe, alaute ?i chimvale.
\par 2 Din fiii lui Asaf au fost rândui?i la slujbele lor ace?tia: Zacur, Iosif, Netania ?i A?arela, fiii lui Asaf, sub conducerea lui Asaf, care cântau dupa porunca regelui.
\par 3 Din ai lui Iedutun au fost rândui?i fiii lui Iedutun: Ghedalia, ?eri, Isaia, ?imei, Ha?abia ?i Matitia; ei erau ?ase sub conducerea tatalui lor Iedutun, care cânta din chitara spre slava ?i lauda Domnului.
\par 4 Din ai lui Heman au fost rândui?i fiii lui Heman: Buchia, Matania, Uziel, ?ebuel, Ierimot, Hanania, Hanani, Eliata, Ghidalti, Romamti-Ezer, Io?beca?a, Maloti, Hotir ?i Mahaziot.
\par 5 To?i ace?tia sunt fiii lui Heman, care era vazatorul regelui, dupa cuvintele lui Dumnezeu, ca sa mareasca slava Lui. ?i i-a dat Dumnezeu lui Heman paisprezece fii ?i trei fete.
\par 6 To?i ace?tia cântau sub conducerea tatalui lor în templul Domnului din chimvale, psaltirioane ?i harpe, la slujbele din templul Domnului, dupa aratarile lui David sau ale lui Asaf, Iedutun ?i Heman.
\par 7 Iar numarul lor, cu al fra?ilor lor care înva?asera sa cânte înaintea Domnului, ?i al tuturor care ?tiau acest lucru era doua sute optzeci ?i opt.
\par 8 Au aruncat ?i ei sor?i pentru rândul la slujba, mic cu mare, dascal ?i ucenic deopotriva.
\par 9 Întâiul sor? a cazut pentru Iosif, fiul lui Asaf, cu fra?ii ?i fiii lui; ei erau doisprezece; al doilea, lui Ghedalia cu fra?ii ?i fiii lui, ei erau doisprezece;
\par 10 Al treilea, lui Zacur cu fra?ii lui ?i cu fiii lui; ei erau doisprezece.
\par 11 Al patrulea, lui I?ri cu fiii ?i cu fra?ii lui; ei erau doisprezece.
\par 12 Al cincilea, lui Netania cu fiii ?i fra?ii lui; ei erau doisprezece.
\par 13 Al ?aselea, lui Buchia cu fiii ?i fra?ii lui; ei erau doisprezece.
\par 14 Al ?aptelea lui Ie?arela cu fiii ?i fra?ii lui; ei erau doisprezece.
\par 15 Al optulea, lui Isaia cu fiii ?i fra?ii lui; ei erau doisprezece.
\par 16 Al noualea, lui Matania cu fiii ?i fra?ii lui; ei erau doisprezece.
\par 17 Al zecelea, lui ?imei cu fiii ?i cu fra?ii lui; ei erau doisprezece.
\par 18 Al unsprezecelea, lui Azareel cu fiii ?i fra?ii lui; ei erau doisprezece.
\par 19 Al doisprezecelea, lui Ha?abia cu fiii ?i fra?ii lui; ei erau doisprezece.
\par 20 Al treisprezecelea, lui ?ebuel cu fiii ?i fra?ii lui; ei erau doisprezece.
\par 21 Al paisprezecelea, lui Matitia cu fiii ?i fra?ii lui; ei erau doisprezece.
\par 22 Al cincisprezecelea, lui Ierimot cu fiii ?i fra?ii lui; ei erau doisprezece.
\par 23 Al ?aisprezecelea, lui Hanania cu fiii ?i fra?ii lui; ei erau doisprezece.
\par 24 Al ?aptesprezecelea, lui Io?beca?a cu fiii ?i cu fra?ii lui; ei erau doisprezece.
\par 25 Al optsprezecelea, lui Hanani cu fiii ?i fra?ii lui; ace?tia erau doisprezece.
\par 26 Al nouasprezecelea, lui Maloti cu fiii ?i fra?ii lui; ace?tia erau doisprezece.
\par 27 Al douazecilea, lui Eliata cu fiii ?i fra?ii lui; ei erau doisprezece.
\par 28 Al douazeci ?i unulea, lui Hotir cu fiii ?i fra?ii lui; ei erau doisprezece.
\par 29 Al douazeci ?i doilea, lui Ghidalti cu fiii ?i fra?ii lui; ei erau doisprezece.
\par 30 Al douazeci ?i treilea, lui Mahaziot cu fiii ?i fra?ii lui; ace?tia erau doisprezece.
\par 31 Al douazeci ?i patrulea, lui Romamti-Ezer cu fiii ?i fra?ii lui; ace?tia erau doisprezece.

\chapter{26}

\par 1 Iata acum împar?irea portarilor. Din fiii lui Core: Me?elemia, fiul lui Core, unul din fiii lui Asaf.
\par 2 Fiii lui Me?elemia au fost: întâiul nascut Zaharia, al doilea Iediael, al treilea Zebadia, al patrulea Iatniel,
\par 3 Al cincilea Elam, al ?aselea Iohanan, al ?aptelea Elihoenai.
\par 4 Fiii lui Obed-Edom au fost: Întâiul nascut ?emaia, al doilea Iehozabad, al treilea Ioah, al patrulea Sacar, al cincilea Natanael,
\par 5 Al ?aselea Amiel, al ?aptelea Isahar, al optulea Peultai, pentru ca Dumnezeu l-a binecuvântat.
\par 6 Fiului sau ?emaia i s-a nascut de asemenea fii, care au fost capetenii în neamul lor, pentru ca au fost oameni puternici.
\par 7 Fiii lui ?emaia au fost: Otni, Rafael, Obed ?i Elzabad; fra?ii lui, oameni puternici, au fost: Elihu, Semachia ?i Isbacom.
\par 8 To?i ace?tia sunt dintre fiii lui Obed-Edom; ei, fiii lor ?i fra?ii lor, au fost oameni sârguincio?i ?i la slujba pricepu?i; au fost ?aizeci ?i doi din Obed-Edom.
\par 9 Me?elemia a avut optsprezece fii ?i fra?i, oameni vrednici.
\par 10 Hosa, unul din fiii lui Merari, a avut fii pe ?imri, capetenie, de?i n-a fost întâiul nascut, dar tatal sau l-a pus capetenie;
\par 11 Al doilea Hilchia, al treilea Tebalia, al patrulea Zaharia; to?i fiii ?i fra?ii lui Hosa au fost treisprezece.
\par 12 A?a a fost împar?irea portarilor, dupa capii de familie, vrednici de slujba, împreuna cu fra?ii lor, ca sa slujeasca la templul Domnului.
\par 13 ?i au aruncat ?i ei sor?i, mare ?i mic, dupa familiile lor, pentru fiecare poarta.
\par 14 ?i pentru poarta dinspre rasarit a cazut sor?ul lui ?elemia; ?i lui Zaharia, fiul lui, sfetnic în?elept, i s-a aruncat sor? ?i i-a cazut sor? pentru poarta de miazanoapte.
\par 15 Lui Obed-Edom i-a cazut poarta dinspre miazazi; iar fiilor lui le-a cazut sor?ul pentru camari.
\par 16 Lui ?upim ?i Hosa le-a cazut pentru cea dinspre apus, la por?ile ?elechet, unde drumul urca ?i unde sunt straji fa?a în fa?a.
\par 17 Spre rasarit câte ?ase levi?i, spre miazanoapte câte patru, spre miazazi câte patru, iar la camari câte doi.
\par 18 Spre apus, în fa?a pridvorului la drum, câte patru, iar la pridvor câte doi.
\par 19 Acestea sunt cetele de portari din fiii lui Core ?i din fiii lui Merari.
\par 20 Iar al?ii dintre levi?i, fra?ii lor, pazeau vistieria templului lui Dumnezeu ?i vistieria lucrurilor sfinte,
\par 21 ?i anume: Fiii lui Laedan, fiul lui Gher?on, Capeteniile familiilor din Laedan Gher?onitul: Iehiel,
\par 22 ?i fiii lui Iehiel: Zetam ?i Ioil, fratele lui, care pazeau vistieria templului lui Dumnezeu,
\par 23 Împreuna cu urma?ii lui Amram I?har, Hebron, Uziel;
\par 24 ?ebuel, fiul lui Gher?on, fiul lui Moise, era strajuitor de capetenie al vistieriilor.
\par 25 Fratele sau Eleazar avea fiu pe Rehabia; acesta a avut fiu pe Isaia; acesta a avut fiu pe Ioram; acesta a avut fiu pe Zicri, iar acesta a avut fiu pe ?elomit.
\par 26 ?elomit ?i fra?ii lui privegheau asupra tuturor vistieriilor lucrurilor sfinte care le harazise regele David, capeteniile familiilor,  capeteniile peste mii ?i peste sute ?i capeteniile de o?tire.
\par 27 Din cuceriri ?i prazi ei afierosisera pentru între?inerea templului Domnului
\par 28 ?i tot ce afierosise Samuel proorocul ?i Saul, fiul lui Chi?, Abner, fiul lui Ner, ?i Ioab, fiul ?eruiei; toate cele afierosite erau în grija lui ?elomit ?i a fra?ilor lui.
\par 29 Din neamul lui I?har, Hanania ?i fiii lui erau rândui?i la slujbele din afara ale Israeli?ilor, ca scriitori ?i judecatori.
\par 30 Din neamul lui Hebron, Ha?abia ?i fiii lui, oameni curajo?i, în numar de o mie ?apte sute, aveau supravegherea asupra lui Israel de cealalta parte de Iordan, spre apus, pentru tot felul de slujbe ale Domnului ?i ale regelui.
\par 31 În neamul Hebroni?ilor, Ieria era capetenia Hebroni?ilor, în neamul ?i familiile lor. În anul al patruzecilea al domniei lui David ei au fost numara?i ?i s-au gasit între ei barba?i curajo?i în Iazerul Galaadului.
\par 32 ?i fra?ii lui, oameni vrednici, în numar de doua mii ?apte sute erau capi de familie. Pe ace?tia i-a pus regele David peste triburile lui Ruben ?i Gad ?i peste jumatate din semin?ia lui Manase, pentru toate treburile lui Dumnezeu ?i ale regelui.

\chapter{27}

\par 1 Iata fiii lui Israel, dupa numarul lor, capii de familie, capeteniile peste mii, peste sute ?i cârmuitorii care, împar?i?i în cete, slujeau regelui la tot cuvântul, ducându-se ?i venind în fiecare luna, în toate lunile anului. În fiecare ceata erau câte douazeci ?i patru de mii.
\par 2 Peste ceata întâi, pentru luna întâi, era capetenia Ia?obeam, fiul lui Zabdiel; în ceata lui erau douazeci ?i patru de mii;
\par 3 El era din fiii lui Fares, mai-mare peste toate capeteniile de o?tire în luna întâi.
\par 4 Peste ceata din luna a doua era Dodai Ahohitul; în ceata lui se afla ?i capetenia Miclot; ?i în ceata lui erau douazeci ?i patru de mii.
\par 5 A treia mare capetenie de o?tire, pentru luna a treia, era Benaia, fiul lui Iehoiada preotul; ?i în ceata lui erau douazeci ?i patru de mii.
\par 6 Acest Benaia era unul dintre cei treizeci de viteji ?i capetenie peste ei; ?i în ceata lui se afla Amizabad, fiul sau.
\par 7 A patra capetenie, pentru luna a patra, era Asael, fratele lui Ioab, ?i dupa el era Zebadia, fiul sau; ?i în ceata lui erau douazeci ?i patru de mii;
\par 8 A cincea capetenie, pentru luna a cincea, era ?amhut Izrahitul; ?i în ceata lui erau douazeci ?i patru de mii.
\par 9 A ?asea capetenie, pentru luna a ?asea, era Ira, fiul lui Iche? Tecoanul; ?i în ceata lui erau douazeci ?i patru de mii.
\par 10 A ?aptea capetenie, pentru luna a ?aptea, era Hele? Peloninul, din fiii lui Efraim; ?i în ceata lui erau douazeci ?i patru de mii.
\par 11 A opta capetenie, pentru luna a opta, era Sibecai Hu?atitul, din semin?ia lui Zarah; ?i în ceata lui erau douazeci ?i patru de mii.
\par 12 A noua capetenie, pentru luna a noua, era Abiezer Anatoteanul, din fiii lui Veniamin; ?i în ceata lui erau douazeci ?i patru de mii.
\par 13 A zecea capetenie, pentru luna a zecea, era Maherai din Netofat, din familia lui Zara; ?i în ceata lui erau douazeci ?i patru de mii.
\par 14 A unsprezecea capetenie, pentru luna a unsprezecea, era Benaia din Piraton, din fiii lui Efraim; ?i în ceata lui erau douazeci ?i patru de mii.
\par 15 A douasprezecea capetenie, pentru luna a douasprezecea, era Heldai din Netofat, din urma?ii lui Otniel; ?i în ceata lui erau douazeci ?i patru de mii.
\par 16 Iar peste triburile lui Israel capetenii înalte erau: la Rubeni?i, Eliezer, fiul lui Zicri; la Simeon, ?efatia, fiul lui Maaca;
\par 17 La levi?i era Ha?abia, fiul lui Chemuel; la Aaron era ?adoc;
\par 18 La Iuda era Elihu, din fra?ii lui David; la Isahar era Omri, fiul lui Micael;
\par 19 La Zabulon era I?maia, fiul lui Obadia; la Neftali era Ierimot, fiul lui Azriel;
\par 20 La fiii lui Efraim era Hoseia, fiul lui Azazia; la jumatatea de trib a lui Manase era Ioil, fiul lui Pedaia;
\par 21 La jumatatea de trib al lui Manase din Galaad, era Ido, fiul lui Zaharia; la Veniamin era Iaasiel, fiul lui Abner;
\par 22 La Dan era Azareel, fiul lui Ieroham. Iata capeteniile triburilor lui Israel.
\par 23 David n-a facut numaratoarea acelora, care erau de la douazeci de ani în jos, pentru ca Domnul zisese ca El va înmul?i pe Israel ca stelele cerului.
\par 24 Ioab, fiul ?eruiei, începuse sa faca numaratoarea, dar nu o sfâr?ise. ?i pentru aceasta a venit mânia lui Dumnezeu asupra lui Israel ?i numaratoarea aceea n-a intrat în cronica regelui David.
\par 25 Peste vistieriile regale era Azmavet, fiul lui Adiel, iar peste depozitele de provizii de la câmp, de prin ceta?i ?i de prin sate ?i turnuri era Ionatan, fiul lui Uzia.
\par 26 Peste cei ce se îndeletniceau cu lucrul câmpului, cu agricultura, era Ezri, fiul lui Chelub.
\par 27 Peste vii era ?imei din Rama, iar peste depozitele de vin din vii era Zabdi, fiul lui ?ifmi.
\par 28 Peste livezile de maslini ?i de smochini din câmpie era Baal-Hanan din Gheder, iar peste depozitele de untdelemn era Ioa?.
\par 29 Peste vitele mari care pa?teau în ?aron era ?itrai Ha?aroneanul; iar peste cele din vai, ?afat, fiul lui Adlai.
\par 30 Peste camile era Obil Ismaelitul; peste asini era Iehdia Meroneanul.
\par 31 Peste oi ?i capre era Iaziz Agariteanul. To?i ace?tia erau capetenii peste averea regelui David.
\par 32 Ionatan, unchiul lui David, era sfetnic, om în?elept ?i scriitor; Iehiel, fiul lui Hacmoni, era pe lânga fiii regelui.
\par 33 Ahitofel era sfetnicul regelui; Hu?ai Architul era prietenul regelui.
\par 34 Iar dupa Ahitofel a fost Iehoiada, fiul lui Benaia ?i Abiatar, iar Ioab era capetenia o?tirii pe lânga rege.

\chapter{28}

\par 1 Apoi a adunat David la Ierusalim pe toate capeteniile lui Israel, pe mai marii triburilor, capeteniile cetelor care slujeau regelui, capeteniile peste mii, peste sute, îngrijitorii mo?iilor ?i turmelor regelui, pe fiii sai cu eunucii, capeteniile o?tirii ?i pe to?i vitejii lui.
\par 2 ?i sculându-se, regele David a zis: "Asculta?i-ma, fra?ilor ?i poporul meu! Am avut la inima mea gând sa zidesc loca? de odihna pentru chivotul legamântului Domnului ?i a?ternut picioarelor Dumnezeului nostru ?i cele de trebuin?a pentru zidire le-am pregatit.
\par 3 Dar Dumnezeu mi-a zis: Sa nu zide?ti loca? numelui Meu, pentru ca tu e?ti om razboinic ?i ai varsat sânge.
\par 4 Cu toate acestea m-a ales Domnul Dumnezeul lui Israel din toata casa tatalui meu, ca sa fiu rege peste Israel în veci, pentru ca pe Iuda l-a  ales El domn, iar din casa lui Iuda a ales casa tatalui meu ?i din casa tatalui meu ?i dintre fiii tatalui meu a binevoit a ma pune pe mine rege peste tot Israelul;
\par 5 Iar din to?i fiii mei - caci mul?i fii mi-a dat Domnul - a ales pe Solomon, fiul meu, sa ?ada pe tronul regatului Domnului, peste Israel.
\par 6 ?i mi-a zis: Solomon, fiul tau, va zidi loca?ul Meu ?i cur?ile Mele, pentru ca Mi l-am ales pe el de fiu ?i Eu îi voi fi lui tata.
\par 7 ?i voi întari domnia lui pe veci, daca va fi tare în împlinirea poruncilor Mele ?i a a?ezamintelor Mele, ca pâna astazi.
\par 8 ?i acum înaintea ochilor a tot Israelul, a adunarii Domnului ?i în auzul Dumnezeului nostru va graiesc: Pazi?i ?i ?ine?i toate poruncile Domnului Dumnezeului vostru, ca sa stapâni?i tot pamântul cel bun ?i sa-l lasa?i dupa voi mo?tenire copiilor vo?tri pe veci.
\par 9 ?i tu, Solomon, fiul meu, cunoa?te pe Dumnezeul tatalui tau ?i Îi sluje?te din toata inima ?i din tot sufletul, caci Domnul cerceteaza toate inimile ?i cunoa?te toata mi?carea gândurilor. De Îl vei cauta pe El, Îl vei gasi, iar de Îl vei parasi ?i El te va parasi.
\par 10 Baga de seama însa, de vreme ce Domnul te-a ales sa zide?ti loca? sfin?eniei Lui, fii tare ?i fa ceea ce a rânduit El".
\par 11 ?i a dat David lui Solomon, fiul sau, planul pridvorului ?i al caselor lui, al camarilor lui, al odailor celor mari de primire, al odailor celor mai dinauntru de odihna ?i al casei chivotului legii.
\par 12 A dat de asemenea planul tuturor celor ce avea el în gândul sau: Planul cur?ii templului Domnului, al tuturor camarilor dimprejur, al vistieriilor lucrurilor sfinte
\par 13 Al încaperilor preo?ilor ?i levi?ilor, al tuturor slujitorilor din templul Domnului ?i al tuturor vaselor sfinte din templul Domnului,
\par 14 Al lucrurilor de aur, cu aratarea greuta?ii, al tuturor vaselor de slujba, al tuturor lucrurilor de argint, cu aratarea greuta?ii lor ?i al tuturor celorlalte vase de slujba.
\par 15 Apoi i-a dat aurul pentru sfe?nicele ?i pentru candelele de aur ale lor, cu însemnarea greuta?ii fiecaruia din sfe?nice ?i din candelele lui; de asemenea ?i argintul pentru sfe?nicele de argint, cu însemnarea greuta?ii fiecaruia din sfe?nice ?i din candelele lui, potrivit cu menirea de slujba a fiecaruia din sfe?nice;
\par 16 ?i aurul pentru mesele pâinilor punerii înainte, cu însemnarea greuta?ii fiecareia din mesele de aur ?i argintul pentru mesele de argint.
\par 17 I-a dat aurul pentru furculi?ele, castroanele ?i cupele cele de aur curat ?i pentru vasele de aur, cu însemnarea greuta?ii fiecarui vas ?i argintul pentru vasele de argint, cu însemnarea greuta?ii fiecarui vas,
\par 18 Precum ?i aurul pentru jertfelnicul tamâierii, turnat din aur, cu însemnarea greuta?ii. I-a dat modelul carului divin, al heruvimilor de aur, cu aripile întinse pentru acoperirea chivotului legamântului Domnului.
\par 19 "Toate acestea sunt în scrisoarea insuflata de la Domnul - a zis David - cum m-a luminat El pentru toate lucrarile zidirii".
\par 20 Apoi a zis David catre fiul sau Solomon: "Fii tare ?i curajos ?i pa?e?te la lucru, nu te teme, nici te speria, caci Domnul Dumnezeu, Dumnezeul meu, este cu tine. El nu se va departa de tine, nici nu te va parasi, pâna nu vei ispravi toata lucrarea ce se cere la templul Domnului.
\par 21 Iata ?i cetele de preo?i ?i de levi?i pentru toate slujbele cele de la templul lui Dumnezeu. Ai de asemenea oameni sârguincio?i pentru orice lucru ?i iscusi?i pentru orice lucrare; ?i capeteniile ?i tot poporul sunt gata sa împlineasca toate poruncile tale".

\chapter{29}

\par 1 Dupa aceea a zis regele David catre toata adunarea: "Solomon, fiul meu, singurul pe care l-a ales Dumnezeu, este tânar ?i cu pu?ina putere, iar lucrul acesta este mare, fiindca nu este pentru om zidirea aceasta, ci pentru Domnul Dumnezeu.
\par 2 Din toate puterile am pregatit eu pentru templul lui Dumnezeu mult aur pentru lucrurile cele de aur, argint pentru cele de argint, arama pentru cele de arama, fier pentru cele de fier, lemn pentru cele de lemn, pietre de onix ?i pietre pentru încrustat, pietre frumoase de diferite culori ?i tot felul de pietre scumpe ?i multa marmura.
\par 3 Mai mult! Din dragoste pentru templul Dumnezeului meu, dau înca tot ce am eu aur ?i argint, templului Dumnezeului meu, afara de ceea ce am pregatit eu pentru templul cel sfânt,
\par 4 ?i anume: trei mii de talan?i de aur, aur de ofir, ?apte mii de talan?i de argint curat, pentru captu?irea pere?ilor în templu,
\par 5 Pentru orice lucru de aur, pentru tot lucrul de argint ?i pentru tot lucrul de mâna de me?ter. ?i cine vrea sa vina astazi cu mâinile pline la Domnul?"
\par 6 Au început atunci sa aduca jertfa capii de familii ?i capeteniile triburilor, capeteniile peste mii ?i peste sute ?i capeteniile cele peste averea regelui.
\par 7 ?i au dat pentru zidirea templului lui Dumnezeu cinci mii de talan?i ?i zece mii de drahme aur, argint zece mii de talan?i, arama optsprezece mii talan?i ?i fier o suta de mii de talan?i.
\par 8 ?i cei care aveau pietre scumpe, le-au dat ?i pe acelea în vistieria templului Domnului, prin mâinile lui Iehiel Gher?onitul.
\par 9 ?i s-a bucurat poporul de râvna lor, pentru ca din toata inima au jertfit Domnului. De asemenea s-a bucurat foarte mult ?i regele David.
\par 10 Atunci a slavit David pe Domnul înaintea a toata adunarea ?i a zis: "Binecuvântat e?ti Tu, Doamne Dumnezeul lui Israel, Tatal nostru, din veac ?i pâna în veac.
\par 11 A Ta este, Doamne, mare?ia ?i puterea ?i slava ?i biruin?a ?i stralucirea; toate câte sunt în cer ?i pe pamânt sunt ale Tale; a Ta este, Doamne, împara?ia ?i Tu e?ti mai presus de toate, ca unul ce împara?e?ti peste toate.
\par 12 Boga?ia ?i slava sunt de la fa?a Ta ?i Tu domne?ti peste toate; în mâna Ta este taria ?i puterea ?i în puterea Ta sta sa mare?ti ?i sa întare?ti toate.
\par 13 ?i acum dar, Dumnezeul nostru, Te slavim pe Tine ?i laudam preaslavit numele Tau.
\par 14 Ca cine sunt eu ?i cine este poporul meu, încât sa avem putin?a de a face asemenea jertfe? Dar de la Tine sunt toate ?i cele primite din mâna Ta ?i le-am dat ?ie;
\par 15 Caci calatori suntem noi înaintea Ta ?i pribegi, ca to?i parin?ii no?tri; ca umbra sunt zilele noastre pe pamânt ?i nimic nu este statornic.
\par 16 Doamne Dumnezeul nostru, toata aceasta mul?ime de lucruri, pe care am pregatit-o noi pentru a zidi templu sfânt numelui Tau, din mâna Ta le avem ?i ale Tale sunt toate.
\par 17 ?tiu, Dumnezeul meu, ca ispite?ti inimile ?i iube?ti cura?enia inimii! Eu din inima curata am jertfit toate ?i vad acum ca ?i poporul Tau, care se afla aici, cu bucurie Î?i jertfe?te ?ie.
\par 18 Doamne Dumnezeul lui Avraam ?i al lui Isaac ?i al lui Iacov, parin?ii no?tri, paze?te acestea în veci, aceasta aplecare a gândurilor inimii poporului Tau ?i îndreapta inimile lor catre Tine.
\par 19 Iar lui Solomon, fiul meu, daruie?te-i inima dreapta ca sa pazeasca poruncile Tale, descoperirile Tale ?i legile Tale ?i sa împlineasca toate acestea ?i sa înal?e cladirea pentru care am facut pregatire".
\par 20 Apoi a zis David catre toata adunarea: "Binecuvânta?i pe Domnul Dumnezeul nostru!" ?i toata adunarea a binecuvântat pe Domnul Dumnezeul parin?ilor sai ?i a cazut ?i s-a închinat Domnului ?i regelui.
\par 21 Apoi au adus Domnului jertfe ?i au înal?at arderi de tot Domnului, a doua zi dupa aceasta: o mie de miei, o mie de berbeci ?i o mie de vitei cu turnarile lor ?i o mul?ime de jertfe de la tot Israelul.
\par 22 ?i au mâncat ?i au baut înaintea Domnului în ziua aceea cu mare bucurie; iar în alt rând au facut rege pe Solomon, fiul lui David, ?i l-au uns înaintea Domnului ca rege, iar pe ?adoc arhiereu.
\par 23 ?i s-a urcat Solomon pe tronul Domnului, ca rege, în locul lui David, tatal sau, ?i a avut spor ?i tot Israelul s-a supus lui.
\par 24 S-au supus lui Solomon de asemenea toate capeteniile ?i puternicii, precum ?i to?i fiii lui David.
\par 25 Iar Domnul a marit pe Solomon în ochii a tot Israelul ?i i-a daruit domnie slavita, cum nu mai avusese înainte de el nici unul din regii lui Israel.
\par 26 David, fiul lui Iesei, a domnit peste tot Israelul.
\par 27 Timpul domniei lui peste Israel a fost patruzeci de ani: în Hebron a domnit el ?apte ani, iar în Ierusalim a domnit treizeci ?i trei.
\par 28 ?i a murit la adânci batrâne?e, satul de via?a, de boga?ie ?i de slava, iar în locul lui s-a facut rege Solomon, fiul sau.
\par 29 Faptele lui David, cele dintâi ?i cele de pe urma, sunt scrise în însemnarile lui Samuel vazatorul ?i în însemnarile lui Natan proorocul ?i în însemnarile lui Gad vazatorul,
\par 30 Precum ?i toata domnia lui ?i barba?ia lui ?i întâmplarile ce s-au petrecut cu el ?i cu Israel ?i cu toate împara?iile pamântului.


\end{document}