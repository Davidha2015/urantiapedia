\begin{document}

\title{Ezra}


\chapter{1}

\par 1 În anul întâi al domniei lui Cirus, regele Perșilor, ca să se împlinească cuvântul Domnului, cel grăit prin gura lui Ieremia, a deșteptat Domnul duhul lui Cirus, regele Perșilor și acesta a poruncit să se facă știut în tot regatul său, prin grai și prin scris, acestea:
\par 2 "Așa grăiește Cirus, regele Perșilor: Toate regatele pământului mi le-a dat mie Domnul Dumnezeul cerului și mi-a poruncit să-I fac locaș la Ierusalim în Iuda.
\par 3 Așadar, aceia dintre voi, din tot poporul Lui, care voiesc - fie cu ei Dumnezeul lor - să se ducă la Ierusalim în Iuda și să zidească templul Domnului Dumnezeului lui Israel, a Acelui Dumnezeu Care este în Ierusalim;
\par 4 Și tot celui rămas în toate locurile, oriunde ar trăi, să-i ajute locuitorii locului aceluia cu argint și cu aur și cu altă avere și cu vite, cu daruri de bună voie pentru templul lui Dumnezeu care este la Ierusalim".
\par 5 Atunci s-au ridicat căpeteniile semințiilor lui Iuda și ale lui Veniamin, și preoții și leviții și toți aceia cărora Dumnezeu le deșteptase duhul să se ducă să înalțe templul Domnului care este la Ierusalim.
\par 6 Și toți vecinii lor i-au ajutat cu vase de argint, cu aur și cu altă avuție și cu vite și cu lucruri scumpe și cu tot felul de daruri de bună voie pentru templu.
\par 7 Iar regele Cirus a scos vasele templului Domnului, pe care Nabucodonosor le luase din Ierusalim și le pusese în casa dumnezeului său;
\par 8 Pe acestea le-a scos Cirus, regele Perșilor, prin mâna lui Mitridat, vistiernicul, care le-a dat pe seama lui Șeșbațar, căpetenia lui Iuda.
\par 9 Și numărul lor era: treizeci de vase de aur, o mie de vase de argint, douăzeci și nouă de cuțite,
\par 10 Treizeci de cupe de aur, patru sute zece cupe de argint de mâna a doua, o mie alte felurite vase;
\par 11 Iar de toate erau cinci mii patru sute de vase de argint și de aur. Toate acestea le-a luat cu el Șeșbațar, când au plecat robii cei ce voiau să se întoarcă din Babilon la Ierusalim.

\chapter{2}

\par 1 Iată fiii țării, dintre robii strămutați din Iuda, pe care Nabucodonosor, regele Babilonului, i-a dus la Babilon, care s-au întors și au venit la Ierusalim și în Iuda, fiecare în orașul său, cu Zorobabel,
\par 2 Iosua, Neemia, Seraia, Reelaia, Nahamani, Mardoheu, Bilșan, Mispar, Bigvai, Rehum și Baana. Numărul oamenilor poporului israelit care s-au întors a fost acesta:
\par 3 Fiii lui Fares: două mii o sută șaptezeci și doi;
\par 4 Fiii lui Șefatia: trei sute șaptezeci și doi;
\par 5 Fiii lui Arah: șapte sute șaptezeci și cinci;
\par 6 Fiii lui Pahat-Moab, din fiii lui Iosua și Ioab: două mii opt sute doisprezece;
\par 7 Fiii lui Elam: o mie două sute cincizeci și patru;
\par 8 Fiii lui Zatu: nouă sute patruzeci și cinci;
\par 9 Fiii lui Zacai: șapte sute șaizeci;
\par 10 Fiii lui Bani: șase sute patruzeci și doi;
\par 11 Fiii lui Bebai: șase sute douăzeci și trei;
\par 12 Fiii lui Azgad: o mie două sute douăzeci și doi;
\par 13 Fiii lui Adonicam: șase sute șaizeci și șase;
\par 14 Fiii lui Bigvai: două mii cincizeci și șase;
\par 15 Fiii lui Adin: patru sute cincizeci și patru;
\par 16 Fiii lui Ater, din casa lui Iezechia: nouăzeci și opt;
\par 17 Fiii lui Bețai: trei sute douăzeci și trei;
\par 18 Fiii lui Iora: o sută doisprezece;
\par 19 Fiii lui Hașum: două sute douăzeci și trei;
\par 20 Fiii lui Ghibar: nouăzeci și cinci;
\par 21 Oamenii din Betleem: o sută douăzeci și trei;
\par 22 Oamenii din Netofa: cincizeci și șase;
\par 23 Oamenii din Anatot: o sută douăzeci și opt;
\par 24 Oamenii din Betazmavet: patruzeci și doi;
\par 25 Din Chiriat-Iearim, Chefira și Beerot: șapte sute patruzeci și trei;
\par 26 Din Rama și Gheba: șase sute douăzeci și unu;
\par 27 Oamenii din Micmas: o sută douăzeci și doi;
\par 28 Oamenii din Betel și din Ai: două sute douăzeci și trei;
\par 29 Oamenii din Nebo: cincizeci și doi;
\par 30 Oamenii din Magbiș: o sută cincizeci și șase;
\par 31 Fiii celuilalt Elam: o mie două sute cincizeci și patru;
\par 32 Fiii lui Harim: trei sute douăzeci;
\par 33 Oamenii din Lod, Hadid și Ono: șapte sute douăzeci și cinci;
\par 34 Oamenii din Ierihon: trei sute patruzeci și cinci;
\par 35 Fiii lui Senaa: trei mii șase sute treizeci;
\par 36 Preoți: fiii lui Iedaia, din casa lui Iosua: nouă sute șaptezeci și trei;
\par 37 Fiii lui Imer: o mie cincizeci și doi;
\par 38 Fiii lui Pașhur: o mie două sute patruzeci și șapte;
\par 39 Fiii lui Harim: o mie șaptesprezece;
\par 40 Leviți: fiii lui Iosua și Cadmiel, din fiii lui Hodavia: șaptezeci și patru;
\par 41 Cântăreți: fiii lui Asaf: o sută douăzeci și opt;
\par 42 Fiii portarilor: fiii lui Șalum, fiii lui Ater, fiii lui Talmon, fiii lui Acuv, fiii lui Hatita, fiii lui Șobai, cu toții la un loc: o sută treizeci și nouă;
\par 43 Cei încredințați templului: fiii lui Țiha, fiii lui Hasufa, fiii lui Tabaot,
\par 44 Fiii lui Cheros, fiii lui Siaha, fiii lui Fadon;
\par 45 Fiii lui Lebana, fiii lui Hagaba, fiii lui Acub;
\par 46 Fiii lui Hagab, fiii lui Șamlai, fiii lui Hanan;
\par 47 Fiii lui Ghidel, fiii lui Gahar, fiii lui Reaia;
\par 48 Fiii lui Rețin, fiii lui Necoda;
\par 49 Fiii lui Gazam, fiii lui Uza, fiii lui Paseah;
\par 50 Fiii lui Besai, fiii lui Asna, fiii lui Meunim;
\par 51 Fiii lui Nefișim, fiii lui Bacbuc;
\par 52 Fiii lui Hacufa, fiii lui Harhur, fiii lui Bațlut, fiii lui Mechida, fiii lui Harșa;
\par 53 Fiii lui Barcos, fiii lui Sisera, fiii lui Tamah;
\par 54 Fiii lui Nețiah, fiii lui Hatifa.
\par 55 Fiii robilor lui Solomon: fiii lui Sotai, fiii lui Soferet, fiii lui Peruda;
\par 56 Fiii lui Iaala, fiii lui Darcon, fiii lui Ghidel;
\par 57 Fiii lui Șefatia, fiii lui Hatil, fiii lui Pocheret-Hațebaim, fiii lui Ami.
\par 58 Cei încredințați templului și fiii robilor lui Solomon erau cu toții trei sute nouăzeci și doi;
\par 59 Iar cei ce au ieșit din Telmelah, din Telharșa și din Cherub-Adan-Imer, care n-au putut să-și arate seminția și neamul lor, ca să arate că sunt din Israel, au fost:
\par 60 Fiii lui Delaia, fiii lui Tobie și fiii lui Necoda: șase sute cincizeci și doi.
\par 61 Iar din neamul preoților: fiii lui Hobaia, fiii lui Hacoț și fiii lui Barzilai care și-a luat femeie din fiicele lui Barzilai Galaaditul al cărui nume l-a adoptat.
\par 62 Și-au căutat cartea spiței neamului lor și n-au găsit-o și de aceea au fost îndepărtați de la preoție.
\par 63 Și Tirșata le-a zis să nu mănânce din cele sfinte, până nu se va ridica preot cu Urim și Tumim.
\par 64 Deci toată adunarea la un loc se alcătuia din patruzeci și două de mii trei sute șaizeci de oameni,
\par 65 Afară de robii lor și de roabele lor, care erau în număr de șapte mii trei sute treizeci și șapte; și mai erau cu ei două sute de cântăreți și cântărețe;
\par 66 Și aveau șapte sute treizeci și șase de cai,
\par 67 Două sute patruzeci și cinci de catâri, patru sute treizeci și cinci de cămile și șase mii șapte sute douăzeci de asini.
\par 68 Și ajungând unele din căpeteniile semințiilor la templul Domnului care este la Ierusalim, au dăruit de bunăvoie pentru templul lui Dumnezeu, ca să fie ridicat din nou pe temeliile lui.
\par 69 Și aceștia din prisosul lor au dăruit în casa obștească pentru începerea lucrărilor: șaizeci și una de mii de drahme de aur, cinci mii de mine de argint și o sută de veșminte preoțești.
\par 70 Și au început să locuiască preoții și leviții și poporul și cântăreții și portarii și cei încredințați templului în orașele lor și tot Israelul s-a așezat în cetățile sale.

\chapter{3}

\par 1 Iar când a venit luna a șaptea și când fiii lui Israel erau prin orașele lor, s-a strâns poporul ca un singur om la Ierusalim;
\par 2 Și s-a sculat Iosua, fiul lui Ioțadac, și frații lui, preoții și Zorobabel, fiul lui Salatiel, și frații lui și au făcut jertfelnic Dumnezeului lui Israel, ca să aducă pe el arderi de tot, cum se scrie în legea lui Moise, omul lui Dumnezeu.
\par 3 Și au așezat jertfelnicul pe temeliile lui, deși se temeau de popoarele străine, și au început să aducă pe el arderi de tot Domnului, arderi de tot dimineața și seara.
\par 4 Și au săvârșit sărbătoarea corturilor, după cum este scris, cu ardere de tot zilnică, după numărul hotărât și după rânduiala fiecărei zile.
\par 5 După aceea au început să săvârșească arderea de tot cea obișnuită, și cea de la lună nouă, și cea de la sărbătorile închinate Domnului, și jertfele pe care le aducea Domnului fiecare de bună voie.
\par 6 Chiar din ziua întâi a lunii a șaptea au început să aducă Domnului arderi de tot. Iar temelia templului Domnului nu se pusese încă.
\par 7 Și au început a da argint pietrarilor și dulgherilor, iar Sidonienilor și Tirienilor bucate și băutură și untdelemn, ca să aducă pe mare la Iafa lemn din Liban, după învoirea dată lor de Cirus, regele Perșilor.
\par 8 Și în anul al doilea după sosirea lor la templul lui Dumnezeu din Ierusalim, în luna a doua, Zorobabel, fiul lui Salatiel, și Iosua, fiul lui Ioțadac, și ceilalți frați ai lor, preoții și leviții, și toți cei ce veniseră din robie la Ierusalim au făcut început și au pus pe leviții de la douăzeci de ani în sus să supravegheze lucrările templului Domnului.
\par 9 Și așa Iosua cu fiii și frații lui, și Cadmiel cu fiii lui, fiii lui Iuda, precum și fiii lui Henadad cu fiii lor și cu frații lor, leviții, au stat să supravegheze pe cei ce lucrau la templul lui Dumnezeu.
\par 10 Și când ziditorii puneau temelia templului Domnului, atunci preoții, îmbrăcați în veșmintele lor și cu trâmbițe, și leviții, fiii lui Asaf, cu chimvale, au fost puși să laude pe Domnul, după rânduiala lui David, regele lui Israel.
\par 11 Și au început ei să cânte pe rând "lăudați" și "slăviți pe Domnul, că este bun, că în veac este mila Lui spre Israel". Și tot poporul striga cu glas mare, slăvind pe Domnul, pentru că a ajutat să se pună temeliile templului Domnului.
\par 12 În vremea aceasta mulți din preoți și din leviți și din capii de familie și din bătrânii care văzuseră vechiul templu, văzând punerea temeliei acestui templu, plângeau cu hohote, mulți însă cântau tare de bucurie;
\par 13 Însă poporul nu putea osebi strigătele de bucurie de bocetele de plâns ale mulțimii, pentru că poporul striga tare și glasul se auzea departe.

\chapter{4}

\par 1 Auzind vrăjmașii lui Iuda și ai lui Veniamin că cei ce s-au întors din robie zidesc templu Domnului Dumnezeului lui Israel,
\par 2 Au venit la Zorobabel și au zis către el: "Să zidim și noi cu voi, pentru că și noi, ca și voi, căutăm pe Dumnezeul vostru și Lui Îi aducem jertfe încă din zilele lui Asarhadon, regele Asiriei, care ne-a adus aici".
\par 3 Iar Zorobabel, Iosua și celelalte căpetenii ale semințiilor lui Israel le-au zis: "Nu se cuvine să zidiți împreună cu noi templu Dumnezeului nostru, ci numai noi singuri vom zidi templu Domnului Dumnezeului lui Israel, precum ne-a poruncit Cirus, regele Perșilor".
\par 4 Atunci poporul jării aceleia a început să descurajeze poporul lui Iuda și să-l împiedice de la zidire,
\par 5 Cumpărând contra lor pe sfetnicii regelui, ca să zădărnicească planul lor în toate zilele lui Cirus, regele Perșilor, până în zilele lui Darie, regale Perșilor.
\par 6 Și sub domnia lui Ahașveroș, pe la începutul domniei acestuia, au scris plângere împotriva locuitorilor lui Iuda și ai Ierusalimului.
\par 7 Și în zilele lui Artaxerxe, Bișlam, Mitridat, Tabeel și ceilalți tovarăși ai lor au scris lui Artaxerxe, regele Perșilor. Și scrisoarea a fost scrisă cu slove aramaice și în limba aramaică.
\par 8 Și sfetnicul Rehum cu scriitorul Șimșai încă au scris către regele Artaxerxe următoarea scrisoare împotriva Ierusalimului:
\par 9 "Atunci Rehum, cârmuitorul, și Șimșai scriitorul și ceilalți tovarăși ai lor: Dineii și Arfarsateii, Tarpeleii, Afarseii, Erecii, Babilonienii, Suzienii, Dehaveii,
\par 10 Elamiții și celelalte popoare, pe care strălucitul și marele Asurbanipal le-a strămutat și le-a așezat în cetățile Samariei și în celelalte cetăți de peste râu, scriu către regele Artaxerxe...".
\par 11 Iată copia de pe scrisoarea ce au trimis către regele Artaxerxe: "Slugile tale, oamenii de dincolo de Eufrat...
\par 12 Cunoscut să fie regelui că Iudeii care au plecat de la tine și au venit la noi la Ierusalim rezidesc cetatea cea rea și răzvrătită și fac ziduri, și temeliile le-au și isprăvit.
\par 13 Și să mai știe regele că, dacă cetatea aceasta se va zidi și zidurile ei se vor face din nou, atunci Iudeii nu vor plăti nici bir, nici dări, nici vamă și pagube se vor face vistieriei regale.
\par 14 Și fiindcă noi mâncăm sare de la curtea regelui și nu putem suferi să vedem pe rege păgubit, de aceea dăm de știre regelui:
\par 15 Să se caute în cartea faptelor părinților tăi și în cartea faptelor vei găsi și vei afla că cetatea aceasta este cetate răzvrătită și primejdioasă pentru regi și ținuturi și că din vechime s-au petrecut în ea abateri, din care pricină a și fost pustiită cetatea aceasta.
\par 16 De aceea noi înștiințăm pe rege că, dacă cetatea aceasta se va isprăvi de zidit și zidurile ei se vor face, atunci nu vei mai avea stăpânire peste râu".
\par 17 Iar regele a trimis răspunsul următor lui Rehum cârmuitorul și lui Șimșai scriitorul și celorlalți tovarăși ai lor, care locuiesc în Samaria și în celelalte cetăți de peste râu.
\par 18 "Pace... Scrisoarea ce mi-ați trimis a fost citită cu luare-aminte înaintea noastră,
\par 19 Și s-a dat din partea noastră poruncă de s-a cercetat și s-a aflat că cetatea aceea de mult s-a răzvrătit împotriva regilor ți că s-a făcut în ea tulburări și răscoale;
\par 20 Și că au fost în Ierusalim regi puternici care au stăpânit toată latura cea de peste râu și cărora li s-au plătit bir și vamă.
\par 21 Așadar, poruncă dați ca oamenii aceia să înceteze de a mai lucra și ca cetatea aceea să nu se mai zidească, până nu va veni poruncă de la mine;
\par 22 Și să fiți cu luare-aminte, ca să nu vă scape ceva nebăgat în seamă în treburile acestea. De ce îngăduiți înmulțirea lucrărilor vătămătoare în paguba regelui?"
\par 23 Îndată ce s-a citit scrisoarea aceasta a regelui Artaxerxe, înaintea lui Rehum și a lui Șimșai scriitorul și a tovarășilor lor, aceștia au trimis îndată la Ierusalim și cu puterea armelor au oprit lucrările Iudeilor.
\par 24 Atunci s-au oprit lucrările la templul lui Dumnezeu cel din Ierusalim, și oprirea aceasta a ținut până în anul al doilea al domniei lui Darie, regele Perșilor.

\chapter{5}

\par 1 Dar proorocul Agheu și proorocul Zaharia, fiul lui Ido, au grăit Iudeilor celor din Ierusalim și din Iuda cuvinte proorocești în numele Dumnezeului lui Israel.
\par 2 Atunci s-au sculat Zorobabel, fiul lui Salatiel, și Iosua, fiul lui Ioțadac, și au început a zidi templul lui Dumnezeu cel din Ierusalim, fiind cu ei proorocii lui Dumnezeu, care îi întăreau.
\par 3 În vremea aceea au venit la el Tatnai, cârmuitorul ținuturilor de peste fluviu, și Șetar-Boznai și tovarășii lor, și le-au grăit așa: "Cine v-a dat vouă învoire să zidiți casa aceasta și să isprăviți zidurile acestea?"
\par 4 Atunci noi le-am spus numele acelor oameni care zideau casa aceasta.
\par 5 Dar ochiul Dumnezeului lor era spre căpeteniile iudaice, și aceia nu i-au mustrat până nu au adus lucrurile la cunoștința lui Darie și până n-a venit dezlegare în chestiunea aceasta.
\par 6 Iată cuprinsul scrisorii pe care Tatnai, cârmuitorul ținuturilor de peste Eufrat, și Șetar-Boznai cu tovarășii lor, cu Arfarsateii cei de peste fluviu, au trimis-o regelui Darie,
\par 7 Și iată ce era scris în înștiințarea trimisă lui:
\par 8 "Pace întru toate regelui Darie! Cunoscut fie regelui că noi am fost în ținutul lui Iuda, la templul Dumnezeului celui mare, și am văzut că el se zidește din pietre mari și se pune și lemn în zid și lucrarea aceasta merge repede și au spor la mână.
\par 9 Atunci am întrebat noi pe căpetenii și le-am zis așa: Cine v-a dat vouă învoire să zidiți casa aceasta și zidurile acestea să le isprăviți?
\par 10 Pe lângă aceasta i-am mai întrebat și de numele acelora, ca să-ți dăm de veste și să-ți seriem numele acelor oameni, care sunt mai însemnați la ei.
\par 11 Și ei mi-au răspuns eu vorbele acestea: Noi suntem robii Dumnezeului cerului și al pământului și zidim templul care a fost făcut cu mult înainte de aceasta și pe care un mare rege al lui Israel l-a zidit și l-a isprăvit.
\par 12 Când însă părinții noștri au mâniat pe Dumnezeul cerului, Acesta i-a dat în mâna lui Nabucodonosor, regele Babilonului, care a dărâmat templul acesta și pe popor l-a strămutat la Babilon.
\par 13 Iar în anul întâi al lui Cirus, regele Babilonului, regele Cirus a dat învoire să se zidească acest templu al lui Dumnezeu;
\par 14 Chiar și vasele de aur și de argint ale templului lui Dumnezeu, pe care Nabucodonosor le luase din templul Ierusalimului și le dusese în capiștea din Babilon, le-a scos regele Cirus din capiștea Babilonului și le-a dat în seama lui Șeșbațar, pe care l-a numit el cârmuitor și i-a zis:
\par 15 Ia vasele acestea, mergi și pune-le în templul din Ierusalim, și templul lui Dumnezeu să se zidească pe locul lui vechi.
\par 16 Atunci Șeșbațar acela a venit și a pus temelia templului lui Dumnezeu din Ierusalim, șț de atunci și până astăzi se zidește el și nu s-a isprăvit.
\par 17 Deci, dacă binevoiește regele, să se caute în casa vistieriei regale de acolo din Babilon și să se vadă dacă în adevăr regele Cirus a dat învoire să se zidească acest templu al lui Dumnezeu în Ierusalim și să ni se trimită știre care este voința regelui în această privință".

\chapter{6}

\par 1 Atunci regele Darie a dat poruncă să se cerceteze la locul unde se păstrau actele și unde se ținea vistieria la Babilon.
\par 2 Și s-a găsit în palatul din Ecbatana, care este în ținutul Mediei, un sul de pergament, în care era scrisă această amintire:
\par 3 "În anul întâi al regelui Cirus, regele Cirus a dat această poruncă pentru templul lui Dumnezeu cel din Ierusalim: Să se zidească templul pe locul acela unde se aduc jertfele și să i se pună temelii sănătoase! Înălțimea lui să fie de șaizeci de coți și lărgimea lui tot de șaizeci de coți.
\par 4 Să se pună trei rânduri de pietre și un rând de lemn, iar cheltuielile să se dea din casa regelui.
\par 5 Chiar și vasele de aur și de argint ale templului lui Dumnezeu, pe care Nabucodonosor le-a luat din templul din Ierusalim și le-a dus la Babilon să se înapoieze și să se ducă în templul din Ierusalim și să se pună fiecare la locul lui în templul lui Dumnezeu".
\par 6 "Deci, Tatnai, cârmuitor al ținuturilor de peste fluviu, și Șetar-Boznai cu tovarășii voștri, cu Arfarsateii cei de peste fluviu, depărtați-vă de acolo și nu opriți lucrările la acel templu al lui Dumnezeu;
\par 7 Lasă pe cârmuitorul iudeu și pe căpeteniile iudaice să zidească acel templu al lui Dumnezeu pe locul lui.
\par 8 Iar din partea mea se dă poruncă cu privire la cele cu care voi trebuie să ajutați acelor căpetenii iudaice la zidirea acelui templu al lui Dumnezeu: Luați numaidecât din birul ținuturilor de peste fluviu și dați oamenilor acelora, ca lucrul să nu se oprească; și ce trebuie pentru arderile de tot ale Dumnezeului ceresc:
\par 9 Viței, sau berbeci, sau miei, precum și grâu, sare, vin și untdelemn, să li se dea, cum vor zice preoții cei din Ierusalim,
\par 10 Ca ei să aducă jertfă plăcută Dumnezeului ceresc și să se roage pentru sănătatea regelui și a fiilor lui.
\par 11 Tot de mine poruncă se mai dă că dacă vreun om va schimba această hotărâre, atunci se va scoate o bârnă din casa lui și va fi acela ridicat și pironit pe ea, iar casa lui se va preface pentru aceasta în dărâmătură.
\par 12 Și Dumnezeu al Cărui nume locuiește acolo să doboare pe tot regele și poporul care și-ar întinde mâna sa ca să schimbe aceasta în paguba acestui templu al lui Dumnezeu din Ierusalim. Eu, Darie, am dat porunca aceasta; să fie întocmai adusă la îndeplinire!"
\par 13 Atunci Tatnai, cârmuitorul ținuturilor de peste fluviu, și Șetar-Boznai cu tovarășii lor au făcut întocmai așa, cum poruncise regele Darie.
\par 14 Și căpeteniile Iudeilor au zidit și au sporit, după proorocia lui Agheu proorocul și a lui Zaharia, fiul lui Ido. Și cu voia Dumnezeului lui Israel și a lui Cirus și Darie și Artaxerxe, regii Perșilor, l-au zidit și l-au isprăvit.
\par 15 Și s-a isprăvit templul acesta în ziua a treia a lunii Adar, în al șaselea an al domniei regelui Darie.
\par 16 Și fiii lui Israel, preoții și ceilalți, care se întorseseră din robie, au săvârșit cu bucurie sfințirea acestui templu al lui Dumnezeu.
\par 17 La sfințirea acestui templu al lui Dumnezeu s-au adus: o sută de boi, două sute de berbeci, patru sute de miei și doisprezece țapi, jertfă de iertarea păcatelor pentru tot Israelul, după numărul semințiilor lui Israel.
\par 18 Și au fost puși preoții cu rândul și leviții tot cu rândul să slujească lui Dumnezeu în Ierusalim, cum era scris în cartea lui Moise.
\par 19 Și cei ce se întorseseră din robie au săvârșit Paștile în ziua a paisprezecea a lunii întâi,
\par 20 Pentru că se curățiseră preoții și leviții și cu toții până la unul erau curați; și au junghiat mielul Paștilor pentru toți cei ce se întorseseră din robie, pentru frații lor preoți și pentru ei înșiși.
\par 21 Și au mâncat fiii lui Israel, cei ce se întorseseră din robie și cei ce se despărțiseră cu ei de necurățenia popoarelor țării, ca să caute pe Domnul Dumnezeul lui Israel;
\par 22 Și au prăznuit sărbătoarea azimelor șapte zile cu bucurie, pentru că îi înveselise Domnul și întorsese spre ei inima regelui Asiriei, ca să le întărească mâinile la zidirea templului Domnului Dumnezeului lui Israel.

\chapter{7}

\par 1 După întâmplările acestea, sub domnia lui Artaxerxe, regele Perșilor, Ezdra, fiul lui Seraia, fiul lui Azaria, fiul lui Hilchia,
\par 2 Fiul lui Șalum, fiul lui Țadoc, fiul lui Ahitub,
\par 3 Fiul lui Amaria, fiul lui Azaria, fiul lui Meraiot,
\par 4 Fiul lui Zerahia, fiul lui Uzi,
\par 5 Fiul lui Buchi, fiul lui Abișua, fiul lui Finees, fiul lui Eleazar, fiul lui Aaron arhiereul,
\par 6 Acest Ezdra a ieșit din Babilon; și era el cărturar iscusit și cunoscător al legii lui Moise, pe care o dăduse Domnul Dumnezeul lui Israel. Iar regele i-a dat lui tot ce a dorit, pentru că mâna Dumnezeului său era peste el.
\par 7 Și împreună cu el au plecat la Ierusalim și unii din fiii lui Israel și preoți și leviți și cântăreți și portari și cei încredințați templului, în anul al șaptelea al domniei lui Artaxerxe.
\par 8 Și a venit el la Ierusalim tot în al șaptelea an al regelui, în luna a cincea;
\par 9 Căci în ziua întâi a lunii întâi a fost plecarea lui din Babilon, iar în ziua întâi a lunii a cincea a ajuns la Ierusalim, pentru că mâna binefăcătoare a lui Dumnezeu era peste el;
\par 10 Căci Ezdra se hotărâse cu toată inima să învețe legea Domnului și s-o împlinească și să învețe pe Israel legea și dreptatea.
\par 11 Iată acum și cuprinsul scrisorii pe care Artaxerxe a dat-o lui Ezdra, preotul și cărturarul, care propovăduise în Israel cuvintele poruncilor Domnului și ale legilor Lui:
\par 12 "Artaxerxe, regele regilor, către Ezdra, preotul, învățătorul legii Dumnezeului ceresc Celui desăvârșit...
\par 13 S-a dat de mine poruncă, ca în regatul meu toți aceia din poporul lui Israel și din preoții lui și din leviții lui, care doresc să se ducă la Ierusalim, să meargă cu tine.
\par 14 Fiindcă tu ești trimis de rege și de cei șapte sfetnici ai lui, ca să cercetezi Iuda și Ierusalimul după legea Dumnezeului tău, pe care o ai în mâna ta,
\par 15 Și să duci argintul și aurul pe care regele și sfetnicii lui l-au jertfit Dumnezeului lui Israel a Căruia locuință este în Ierusalim
\par 16 Și tot aurul și argintul pe care-l vei aduna tu din toată țara Babilonului, împreună cu toate darurile de bună voie de la popor și preoți, pe care le vor jertfi ei pentru templul Dumnezeului lor, cel din Ierusalim.
\par 17 De aceea cumpără numaidecât cu banii aceștia, boi, berbeci, miei și daruri de pâine cât trebuie, și turnări pentru ei și du-le la jertfelnicul templului Dumnezeului vostru cel din Ierusalim.
\par 18 Și ce veți crede voi și frații voștri că este bine să faceți cu celălalt argint și aur, aceea să faceți după voia Dumnezeului vostru.
\par 19 Și vasele ce ți s-au dat ție pentru slujbele templului Dumnezeului tău, pune-le înaintea Dumnezeului Ierusalimului.
\par 20 Și alte lucruri de trebuință pentru templul Dumnezeului tău, ce vei crede tu că trebuie, dă-le din casa vistieriilor regești.
\par 21 Din partea mea, a regelui Artaxerxe, se dă tuturor păstrătorilor vistieriilor de peste râu porunca aceasta: Tot ce va cere de la voi preotul Ezdra, învățătorul legii Dumnezeului ceresc, să-i dați numaidecât;
\par 22 Argint până la o sută de talanți, grâu până la o sută de core, vin până la o sută de baturi și tot până la o sută de baturi de untdelemn; iar sare, fără măsură.
\par 23 Tot ce s-a poruncit de Dumnezeul ceresc trebuie să se facă cu îngrijire pentru templul Dumnezeului celui ceresc. Băgați de seamă să nu-și întindă cineva mâna asupra templului Dumnezeului celui ceresc, ca să nu fie mânia Lui asupra regatului, a regelui și a fiilor lui.
\par 24 Și vă dăm știre că nici asupra unuia din preoți, sau leviți, sau cântăreți, sau portari, sau cei încredințați templului, sau slujitori ai acestui templu al lui Dumnezeu să nu se pună nici bir, nici dare, nici vamă.
\par 25 Iar tu, Ezdra, după înțelepciunea Dumnezeului tău, care este în mina ta, să pui cârmuitori și judecători și să judece aceia tot poporul cel de peste fluviu, pe toți cei ce știu legea Dumnezeului tău, iar pe cei ce nu o știu, să-i învățați.
\par 26 Și cine nu va împlini legea Dumnezeului tău și legea regelui, asupra aceluia să se facă îndată judecată și să se osândească sau la moarte, sau la izgonire, sau la amendă, sau la închidere în temniță".
\par 27 "Binecuvântat este Domnul Dumnezeul părinților noștri Care a pus în inima regelui gândul să împodobească templul Domnului cel din Ierusalim și a atras asupra mea mila regelui și a sfetnicilor lui și a tuturor dregătorilor celor puternici ai regelui!
\par 28 Atunci eu m-am întărit, căci mâna Domnului Dumnezeului meu era peste mine, și am adunat pe căpeteniile lui Israel, ca să meargă cu mine".

\chapter{8}

\par 1 "Iată capii de familie și spița neamului acelora care au plecat cu mine din Babilon, în timpul domniei regelui Artaxerxe:
\par 2 Gherșom din fiii lui Finees; Daniel din fiii lui Itamar; Hatuș din fiii lui David;
\par 3 Zaharia din fiii lui Șecania, care se trăgea din fiii lui Fares, și împreună cu el o sută cincizeci de suflete, parte bărbătească, scrise în spița neamului;
\par 4 Elioenai, fiul lui Zerahia, din neamul lui Pahat-Moab împreună cu două sute de suflete, parte bărbătească.
\par 5 Șecania, fiul lui Iahaziel, din urmașii lui Zatu, cu trei sute de suflete, parte bărbătească.
\par 6 Ebed, fiul lui Ionatan, din urmașii lui Adin, cu cincizeci de suflete, parte bărbătească.
\par 7 Isaia, fiul lui Atalia, din urmașii lui Elam, cu șaptezeci de oameni.
\par 8 Zebadia, fiul lui Mihail, din urmașii lui Șefatia, cu optzeci de oameni.
\par 9 Obadia, fiul lui Iehiel, din urmașii lui Ioab, cu două sute optsprezece oameni.
\par 10 Șelomit, fiul lui Iosifia, din urmașii lui Lani, cu o sută șaizeci de oameni.
\par 11 Zaharia, fiul lui Bebai, din urmașii lui Bebai, cu douăzeci și opt de oameni.
\par 12 Iohanan, fiul lui Hacatan, din urmașii lui Azgad, cu o sută zece oameni.
\par 13 Și cei din urmă din fiii lui Adonicam, ale căror nume erau: Ielifelet, Ieiel și Șemaia cu șaizeci de oameni.
\par 14 Utai și Zabud, din fiii lui Bigvai, cu șaptezeci de oameni.
\par 15 Pe aceștia i-am adunat eu la râul ce curge prin Ahava și am poposit acolo trei zile; iar când am cercetat eu poporul și pe preoți, n-am găsit acolo pe nimeni din fiii lui Levi.
\par 16 Și am trimis să cheme pe Eleazar, Ariel, Șemaia, Elnatan, Iariv, Elnatan, Natan, Zaharia și Meșulam, care erau căpetenii, și pe Ioarib și Elnatan, care erau învățători,
\par 17 Și le-am dat acestora însărcinare către Ido, care era căpetenie în ținutul Casifia, și le-am pus în gura lor ce să grăiască cu Ido și cu frații lui, și cu cei încredințați templului din ținutul Casifia, ca să ne aducă slujitori pentru templul Dumnezeului nostru.
\par 18 Pentru că mâna binefăcătoare a Dumnezeului nostru era peste noi, ne-au adus ei un om înțelept din fiii lui Mahli, fiul lui Levi, fiul lui Israel, anume pe Șerevia, și pe fiii acestuia și pe frații lui în număr de optsprezece;
\par 19 Și ne-au mai adus pe Hașabia și pe Isaia din fiii lui Merari, împreună cu frații lor și cu fiii lor, douăzeci de oameni;
\par 20 Și dintre cei încredințați templului pe care i-a dat David și dregătorii lui în slujba leviților, ne-a adus două sute douăzeci de inși; aceștia toți erau numiți pe nume.
\par 21 Și acolo, la râul Ahava, am rânduit post, ca să ne smerim înaintea feței Dumnezeului nostru și să cerem de la El călătorie bună pentru noi și pentru copiii noștri și pentru toată avuția noastră,
\par 22 Căci îmi fusese rușine să cer de la rege oștire și călăreți, ca să ne păzească de vrăjmași în cale, că noi, când am grăit cu regele, am zis: "Mâna Dumnezeului nostru este binefăcătoare pentru toți cei ce aleargă la El, iar asupra tuturor celor ce-L părăsesc este puterea Lui și mânia Lui!"
\par 23 Și așa am postit noi și am rugat pentru aceasta pe Dumnezeul nostru, și El ne-a auzit.
\par 24 Și am luat din cei ce erau mai mari peste preoți doisprezece oameni: pe Șerevia și pe Hașabia și împreună cu ei pe cei zece frați ai lor.
\par 25 Și le-am dat lor cu cântarul aurul și argintul și vasele și tot ce se dăruise pentru templul Dumnezeului nostru, ce dăruise regele și sfetnicii lui și dregătorii lui și toți Israeliții care se aflau acolo.
\par 26 Acestea le-am dat în mâna lor, cântărite: argint, șase sute cincizeci de talanți, vase de argint, ca la o sută de talanți, aur o sută de talanți,
\par 27 Cupe de aur, douăzeci, de o mie de drahme una, și două vase de aramă din cea mai bună, lucitoare, care se prețuiește ca și aurul.
\par 28 Și le-am zis: "Voi sunteți sfințiții Domnului și vasele sunt sfințite, iar argintul și aurul sunt darurile cele de bună voie Domnului Dumnezeului părinților voștri!
\par 29 Vegheați și păziți acestea, până le veți da cu cântarul mai-marilor preoților, leviților și căpeteniilor semințiilor lui Israel la Ierusalim, în camerele templului Domnului".
\par 30 Și au primit preoții și leviții aurul și argintul și vasele cântărite ca să le ducă la Ierusalim, în templul Dumnezeului nostru.
\par 31 După aceea am plecat noi de la râul Ahava în ziua a douăsprezecea a lunii întâi, ca să mergem la Ierusalim; și mâna Dumnezeului nostru a fost cu noi și ne-a scăpat din mâna vrăjmașului și de cei ce ne pândeau în cale.
\par 32 Și am venit la Ierusalim și am rămas acolo trei zile,
\par 33 Iar a patra zi am dat cu cântarul argintul și aurul și vasele la templul Dumnezeului nostru, în mâna lui Meremot preotul, fiul lui Urie, împreună și lui Eleazar, fiul lui Finees, precum și lui Iozabat, fiul lui Iosua, și lui Noadia, fiul lui Binui, leviții.
\par 34 Toate le-am dat cântărit și numărat și toate cele cântărite s-au scris în același timp.
\par 35 Și cei veniți din robie au adus ardere de tot Dumnezeului lui Israel, doisprezece viței pentru tot Israelul, douăzeci și șase de berbeci, șaptezeci și șapte de miei și doisprezece țapi, jertfă pentru păcat; toate acestea le-au adus ardere de tot Domnului.
\par 36 Și am dat poruncile regelui satrapilor și guvernatorilor de peste râu, și aceștia au arătat cinste poporului și templului lui Dumnezeu".

\chapter{9}

\par 1 "După isprăvirea acestora, au venit la mine căpeteniile și au zis: "Poporul lui Israel și preoții și leviții nu s-au deosebit de popoarele cele de alt neam și de urâciunile lor, adică de Canaanei, Hetei, Ferezei, Iebusei, Amoniți, Moabiți, Egipteni și Amorei,
\par 2 Pentru că au luat pe fiicele acelora soții pentru ei și pentru feciorii lor și s-a amestecat sămânța cea sfântă cu popoarele cele de alt neam, ba încă mâna celor mai însemnați și mai de frunte a fost cea dintâi în această nelegiuire".
\par 3 Auzind cuvântul acesta, mi-am rupt haina cea de deasupra și cea de dedesubt și mi-am smuls părul din capul meu și din barba mea și am căzut de mâhnire.
\par 4 Atunci s-au adunat la mine toți cei ce se temeau de cuvintele Dumnezeului lui Israel, din pricina nelegiuirii celor veniți din robie, și eu am stat în întristare până la jertfa cea de seară.
\par 5 Iar la vremea jertfei de seară m-am sculat din locul tânguirii mele și cu hainele rupte de deasupra și de dedesubt am căzut în genunchi și mi-am întins mâinile către Domnul Dumnezeul meu și am zis:
\par 6 "Dumnezeul meu, mă rușinez și mă tem să-mi ridic fața către Tine, Dumnezeul meu, pentru că fărădelegile noastre au trecut peste cap și vina noastră s-a mărit până la cer.
\par 7 Din zilele părinților noștri și până astăzi suntem în mare vinovăție, și pentru fărădelegile noastre am fost dați noi, și regii noștri, și preoții noștri, în mâinile regilor celor de alt neam, și în sabie și în robie, și pradă și în rușine, cum suntem și astăzi.
\par 8 Și iată, după puțină vreme, ni s-a dat îndurare de la Domnul Dumnezeul nostru și El ne-a lăsat pe câțiva să scăpăm și ne-a ajutat să ne așezăm în locul cel sfânt al Lui și ne-a dăruit să ne înviorăm puțin din robia noastră.
\par 9 Noi robi suntem, dar nici în robie nu ne-a părăsit pe noi Dumnezeul nostru și a îndreptat El spre noi mila regilor Perșilor ca să ne lase să înviem, să zidim templu Dumnezeului nostru, să-l scoatem din dărâmăturile lui și ne-a dat întărire în Iuda și în Ierusalim.
\par 10 Dar acum, după toate acestea, ce vom zice noi, Dumnezeul nostru? Căci ne-am abătut de la poruncile Tale,
\par 11 Pe care le-ai dat Tu prin prooroci, robii Tăi, și ai zis: Pământul în care vă duceți voi ca să-l stăpâniți este pământ necurat, căci este spurcat de necurățenia popoarelor celor de alt neam și de urâciunile lor, cu care ele l-au umplut de la un capăt la altul.
\par 12 Deci pe fetele voastre să nu le dați după feciorii lor și pe fetele lor să nu le luați pentru feciorii voștri și pacea lor și bunurile lor să nu le căutați în veci, ca să vă întăriți și să vă hrăniți cu bunătățile pământului aceluia și să-l lăsați moștenire veșnică fiilor voștri.
\par 13 După toate cele ce ne-au ajuns pe noi pentru faptele noastre cele rele și pentru marea noastră vinovăție și pentru că Tu, Dumnezeule, nu Te-ai purtat cu noi după măsura nelegiuirilor noastre și ne-ai dat și izbăvirea aceasta,
\par 14 Au doară iarăși vom călca poruncile Tale și vom intra în legătură de rudenie cu aceste popoare ticăloase? Nu Te vei mânia Tu oare atâta, încât să ne stârpești și să nu mai rămână nici unul și să nu mai fie nici o izbăvire?
\par 15 Doamne Dumnezeul lui Israel, drept ești Tu, pentru că am scăpat noi până în ziua de astăzi; și iată noi și astăzi suntem în nelegiuirile noastre; deci astfel fiind, n-ar trebui să stăm înaintea feței Tale".

\chapter{10}

\par 1 Pe când se ruga astfel Ezdra și se mărturisea, plângând și îngenunchind înaintea templului lui Dumnezeu, s-a strâns la el o mare mulțime de Israeliți, bărbați și femei și copii, pentru că și poporul a plâns foarte mult.
\par 2 Și a grăit Șecania, fiul lui Iehiel, care era din urmașii lui Elam, și a zis către Ezdra: "Noi am făcut nelegiuire înaintea Dumnezeului nostru, când ne-am luat femei de alt neam din popoarele pământului acestuia, dar mai este încă o nădejde pentru Israel în lucrul acesta;
\par 3 Să încheiem acum legământ cu Dumnezeul nostru că, după sfatul stăpânului meu și al celor ce cinstesc poruncile Dumnezeului nostru, să dăm drumul tuturor femeilor și copiilor născuți cu ele, ca să fim după lege.
\par 4 Scoală deci, că aceasta este treaba ta, îmbărbătează-te și lucrează, că noi suntem cu tine".
\par 5 Și s-a sculat Ezdra și a pus pe căpeteniile preoților și ale leviților și pe tot Israelul să facă jurământ că vor face așa. Și ei au făcut jurământ.
\par 6 Și după ce s-a sculat, Ezdra s-a dus de la templul lui Dumnezeu la locuința lui Iohanan, fiul lui Eliașib și, ajungând acolo, n-a mâncat pâine, nici, apă n-a băut, căci plângea pentru nelegiuirea celor din robie.
\par 7 Și a făcut cunoscut în Iuda și în Ierusalim tuturor celor ce fuseseră în robie să se adune la Ierusalim;
\par 8 Și cel ce nu va veni până în trei zile, pe averea aceluia, după sfatul căpeteniilor și al bătrânilor, se va pune blestem, iar el însuși va fi îndepărtat din obștea celor ce fuseseră în robie.
\par 9 Și s-au adunat toți locuitorii Iudei și ai ținutului lui Veniamin la Ierusalim în trei zile. și aceasta era în luna a noua, în ziua a douăzecea a lunii acesteia. Și s-a așezat tot poporul în piața de la templul lui Dumnezeu, tremurând atât pentru păcatul acesta, cât și din pricina ploilor.
\par 10 Și s-a sculat Ezdra preotul și le-a zis: "Voi ați făcut păcat, luându-vă femei de neam străin și cu aceasta ați mărit vina lui Israel.
\par 11 Așadar, pocăiți-vă de păcatul acesta înaintea Domnului Dumnezeului părinților voștri și faceți voia Lui și depărtați-vă de popoarele pământului acestuia și de femeile celor de alt neam".
\par 12 Și răspunzând toată adunarea, a zis cu glas tare: "Cum zici tu, așa vom face!
\par 13 Însă poporul este mult la număr și acum este timp ploios și nu putem sta afară. Și apoi și treaba aceasta nu este de-o zi ori de două, pentru că mulți din noi am săvârșit acest păcat.
\par 14 Deci să rămână căpeteniile noastre pentru întreaga obște și toți cei din orașele noastre care și-au luat femei străine să vină aici la vremea hotărâtă și împreună cu ei să vină și căpeteniile fiecărui oraș și judecătorii lui, până se va potoli de la noi mânia cea arzătoare a Dumnezeului nostru, care s-a pornit pentru păcatul acesta".
\par 15 Atunci Ionatan, fiul lui Asael și Iahzeia, fiul lui Ticva, au fost puși pentru lucrul acesta; iar leviții Meșulam și Șabetai erau ajutoarele lor.
\par 16 Cei ce se întorseseră din robie au făcut așa. Și Ezdra preotul a rânduit la treaba aceasta și pe căpeteniile semințiilor din fiecare seminție și i-a numit pe nume. Și au făcut ei sfat în ziua întâi a lunii a zecea, ca să cerceteze lucrul acesta,
\par 17 Și au isprăvit cercetarea tuturor celor ce-și luaseră femei de alt neam, în ziua întâi a lunii întâi.
\par 18 Din fiii preoților care-și luaseră femei străine, s-au găsit: Maaseia, Eliezer, Iariv și Ghedalia, din fiii lui Iosua, al lui Ioțadac și frații lui;
\par 19 Și aceștia și-au dat mâinile că vor da drumul femeilor lor și că vor aduce jertfă un berbec pentru vina lor.
\par 20 Și s-au mai găsit: Hanani și Zebadia, din fiii lui Imer;
\par 21 Maaseia, Ilie, Șemaia, Iehiel și Uzia, din fiii lui Harim;
\par 22 Elioenai, Maaseia, Ismael, Natanael, Iozabad și Eleasa, din fiii lui Pașhur;
\par 23 Iozabad, Șimei, Chelaia, zis și Chelita, Petahia, Iuda și Eliezer, din leviți;
\par 24 Eliașib, din cântăreți; Șalum, Telem și Uri, din portari.
\par 25 Iar din Israeliți: Ramia, Izia, Malchia, Miamin, Eleazar, Malchia și Benaia, din fiii lui Fares;
\par 26 Matania, Zaharia, Iehiel, Abdie, Iremot și Ilie, din fiii lui Elam;
\par 27 Elioenai, Eliașib, Matania, Ieremot, Zabad și Aziza, din fiii lui Zatu;
\par 28 Iohanan, Hanania, Zabai și Atlai, din fiii lui Bebai;
\par 29 Meșulam, Maluc, Adaia, Iașub, Șeal și Ieramot, din fiii lui Bani;
\par 30 Adna, Chelal, Benaia, Maaseia, Matania, Binui, Manase și Bețaleel, din fiii lui Pahat-Moab;
\par 31 Eliezer, Ișia, Malchia, Șemaia,
\par 32 Simeon, Veniamin, Maluc și Șemaria, din fiii lui Harim;
\par 33 Matnai, Matata, Zabad, Elifelet, Ieremai, Manase și Șimei, din fiii lui Hașun;
\par 34 Iar din fiii lui Bani: Maadai, Amram, Ioel,
\par 35 Benaia, Bedia, Cheluhu,
\par 36 Vania, Meremot, Eliașib,
\par 37 Matania, Matnai, Iaașai,
\par 38 Bani, Binui, Șimei,
\par 39 Șelemia, Natan, Adaia,
\par 40 Macnadbai, Șașai, Șarai,
\par 41 Azariel, Șelemiahu, Șemaria,
\par 42 Șalum, Amaria și Iosif;
\par 43 Și în sfârșit din fiii lui Nebo: Ieiel, Matitia, Zabad, Zebina, Iadai, Ioel și Benaia.
\par 44 Toți aceștia își luaseră femei străine și unele din aceste femei le născuseră copii.


\end{document}