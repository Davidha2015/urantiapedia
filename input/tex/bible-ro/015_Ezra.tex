\begin{document}

\title{Ezra}

Ezr 1:1  În anul întâi al domniei lui Cirus, regele Per?ilor, ca sa se împlineasca cuvântul Domnului, cel grait prin gura lui Ieremia, a de?teptat Domnul duhul lui Cirus, regele Per?ilor ?i acesta a poruncit sa se faca ?tiut în tot regatul sau, prin grai ?i prin scris, acestea:
Ezr 1:2  "A?a graie?te Cirus, regele Per?ilor: Toate regatele pamântului mi le-a dat mie Domnul Dumnezeul cerului ?i mi-a poruncit sa-I fac loca? la Ierusalim în Iuda.
Ezr 1:3  A?adar, aceia dintre voi, din tot poporul Lui, care voiesc - fie cu ei Dumnezeul lor - sa se duca la Ierusalim în Iuda ?i sa zideasca templul Domnului Dumnezeului lui Israel, a Acelui Dumnezeu Care este în Ierusalim;
Ezr 1:4  ?i tot celui ramas în toate locurile, oriunde ar trai, sa-i ajute locuitorii locului aceluia cu argint ?i cu aur ?i cu alta avere ?i cu vite, cu daruri de buna voie pentru templul lui Dumnezeu care este la Ierusalim".
Ezr 1:5  Atunci s-au ridicat capeteniile semin?iilor lui Iuda ?i ale lui Veniamin, ?i preo?ii ?i levi?ii ?i to?i aceia carora Dumnezeu le de?teptase duhul sa se duca sa înal?e templul Domnului care este la Ierusalim.
Ezr 1:6  ?i to?i vecinii lor i-au ajutat cu vase de argint, cu aur ?i cu alta avu?ie ?i cu vite ?i cu lucruri scumpe ?i cu tot felul de daruri de buna voie pentru templu.
Ezr 1:7  Iar regele Cirus a scos vasele templului Domnului, pe care Nabucodonosor le luase din Ierusalim ?i le pusese în casa dumnezeului sau;
Ezr 1:8  Pe acestea le-a scos Cirus, regele Per?ilor, prin mâna lui Mitridat, vistiernicul, care le-a dat pe seama lui ?e?ba?ar, capetenia lui Iuda.
Ezr 1:9  ?i numarul lor era: treizeci de vase de aur, o mie de vase de argint, douazeci ?i noua de cu?ite,
Ezr 1:10  Treizeci de cupe de aur, patru sute zece cupe de argint de mâna a doua, o mie alte felurite vase;
Ezr 1:11  Iar de toate erau cinci mii patru sute de vase de argint ?i de aur. Toate acestea le-a luat cu el ?e?ba?ar, când au plecat robii cei ce voiau sa se întoarca din Babilon la Ierusalim.
Ezr 2:1  Iata fiii ?arii, dintre robii stramuta?i din Iuda, pe care Nabucodonosor, regele Babilonului, i-a dus la Babilon, care s-au întors ?i au venit la Ierusalim ?i în Iuda, fiecare în ora?ul sau, cu Zorobabel,
Ezr 2:2  Iosua, Neemia, Seraia, Reelaia, Nahamani, Mardoheu, Bil?an, Mispar, Bigvai, Rehum ?i Baana. Numarul oamenilor poporului israelit care s-au întors a fost acesta:
Ezr 2:3  Fiii lui Fares: doua mii o suta ?aptezeci ?i doi;
Ezr 2:4  Fiii lui ?efatia: trei sute ?aptezeci ?i doi;
Ezr 2:5  Fiii lui Arah: ?apte sute ?aptezeci ?i cinci;
Ezr 2:6  Fiii lui Pahat-Moab, din fiii lui Iosua ?i Ioab: doua mii opt sute doisprezece;
Ezr 2:7  Fiii lui Elam: o mie doua sute cincizeci ?i patru;
Ezr 2:8  Fiii lui Zatu: noua sute patruzeci ?i cinci;
Ezr 2:9  Fiii lui Zacai: ?apte sute ?aizeci;
Ezr 2:10  Fiii lui Bani: ?ase sute patruzeci ?i doi;
Ezr 2:11  Fiii lui Bebai: ?ase sute douazeci ?i trei;
Ezr 2:12  Fiii lui Azgad: o mie doua sute douazeci ?i doi;
Ezr 2:13  Fiii lui Adonicam: ?ase sute ?aizeci ?i ?ase;
Ezr 2:14  Fiii lui Bigvai: doua mii cincizeci ?i ?ase;
Ezr 2:15  Fiii lui Adin: patru sute cincizeci ?i patru;
Ezr 2:16  Fiii lui Ater, din casa lui Iezechia: nouazeci ?i opt;
Ezr 2:17  Fiii lui Be?ai: trei sute douazeci ?i trei;
Ezr 2:18  Fiii lui Iora: o suta doisprezece;
Ezr 2:19  Fiii lui Ha?um: doua sute douazeci ?i trei;
Ezr 2:20  Fiii lui Ghibar: nouazeci ?i cinci;
Ezr 2:21  Oamenii din Betleem: o suta douazeci ?i trei;
Ezr 2:22  Oamenii din Netofa: cincizeci ?i ?ase;
Ezr 2:23  Oamenii din Anatot: o suta douazeci ?i opt;
Ezr 2:24  Oamenii din Betazmavet: patruzeci ?i doi;
Ezr 2:25  Din Chiriat-Iearim, Chefira ?i Beerot: ?apte sute patruzeci ?i trei;
Ezr 2:26  Din Rama ?i Gheba: ?ase sute douazeci ?i unu;
Ezr 2:27  Oamenii din Micmas: o suta douazeci ?i doi;
Ezr 2:28  Oamenii din Betel ?i din Ai: doua sute douazeci ?i trei;
Ezr 2:29  Oamenii din Nebo: cincizeci ?i doi;
Ezr 2:30  Oamenii din Magbi?: o suta cincizeci ?i ?ase;
Ezr 2:31  Fiii celuilalt Elam: o mie doua sute cincizeci ?i patru;
Ezr 2:32  Fiii lui Harim: trei sute douazeci;
Ezr 2:33  Oamenii din Lod, Hadid ?i Ono: ?apte sute douazeci ?i cinci;
Ezr 2:34  Oamenii din Ierihon: trei sute patruzeci ?i cinci;
Ezr 2:35  Fiii lui Senaa: trei mii ?ase sute treizeci;
Ezr 2:36  Preo?i: fiii lui Iedaia, din casa lui Iosua: noua sute ?aptezeci ?i trei;
Ezr 2:37  Fiii lui Imer: o mie cincizeci ?i doi;
Ezr 2:38  Fiii lui Pa?hur: o mie doua sute patruzeci ?i ?apte;
Ezr 2:39  Fiii lui Harim: o mie ?aptesprezece;
Ezr 2:40  Levi?i: fiii lui Iosua ?i Cadmiel, din fiii lui Hodavia: ?aptezeci ?i patru;
Ezr 2:41  Cântare?i: fiii lui Asaf: o suta douazeci ?i opt;
Ezr 2:42  Fiii portarilor: fiii lui ?alum, fiii lui Ater, fiii lui Talmon, fiii lui Acuv, fiii lui Hatita, fiii lui ?obai, cu to?ii la un loc: o suta treizeci ?i noua;
Ezr 2:43  Cei încredin?a?i templului: fiii lui ?iha, fiii lui Hasufa, fiii lui Tabaot,
Ezr 2:44  Fiii lui Cheros, fiii lui Siaha, fiii lui Fadon;
Ezr 2:45  Fiii lui Lebana, fiii lui Hagaba, fiii lui Acub;
Ezr 2:46  Fiii lui Hagab, fiii lui ?amlai, fiii lui Hanan;
Ezr 2:47  Fiii lui Ghidel, fiii lui Gahar, fiii lui Reaia;
Ezr 2:48  Fiii lui Re?in, fiii lui Necoda;
Ezr 2:49  Fiii lui Gazam, fiii lui Uza, fiii lui Paseah;
Ezr 2:50  Fiii lui Besai, fiii lui Asna, fiii lui Meunim;
Ezr 2:51  Fiii lui Nefi?im, fiii lui Bacbuc;
Ezr 2:52  Fiii lui Hacufa, fiii lui Harhur, fiii lui Ba?lut, fiii lui Mechida, fiii lui Har?a;
Ezr 2:53  Fiii lui Barcos, fiii lui Sisera, fiii lui Tamah;
Ezr 2:54  Fiii lui Ne?iah, fiii lui Hatifa.
Ezr 2:55  Fiii robilor lui Solomon: fiii lui Sotai, fiii lui Soferet, fiii lui Peruda;
Ezr 2:56  Fiii lui Iaala, fiii lui Darcon, fiii lui Ghidel;
Ezr 2:57  Fiii lui ?efatia, fiii lui Hatil, fiii lui Pocheret-Ha?ebaim, fiii lui Ami.
Ezr 2:58  Cei încredin?a?i templului ?i fiii robilor lui Solomon erau cu to?ii trei sute nouazeci ?i doi;
Ezr 2:59  Iar cei ce au ie?it din Telmelah, din Telhar?a ?i din Cherub-Adan-Imer, care n-au putut sa-?i arate semin?ia ?i neamul lor, ca sa arate ca sunt din Israel, au fost:
Ezr 2:60  Fiii lui Delaia, fiii lui Tobie ?i fiii lui Necoda: ?ase sute cincizeci ?i doi.
Ezr 2:61  Iar din neamul preo?ilor: fiii lui Hobaia, fiii lui Haco? ?i fiii lui Barzilai care ?i-a luat femeie din fiicele lui Barzilai Galaaditul al carui nume l-a adoptat.
Ezr 2:62  ?i-au cautat cartea spi?ei neamului lor ?i n-au gasit-o ?i de aceea au fost îndeparta?i de la preo?ie.
Ezr 2:63  ?i Tir?ata le-a zis sa nu manânce din cele sfinte, pâna nu se va ridica preot cu Urim ?i Tumim.
Ezr 2:64  Deci toata adunarea la un loc se alcatuia din patruzeci ?i doua de mii trei sute ?aizeci de oameni,
Ezr 2:65  Afara de robii lor ?i de roabele lor, care erau în numar de ?apte mii trei sute treizeci ?i ?apte; ?i mai erau cu ei doua sute de cântare?i ?i cântare?e;
Ezr 2:66  ?i aveau ?apte sute treizeci ?i ?ase de cai,
Ezr 2:67  Doua sute patruzeci ?i cinci de catâri, patru sute treizeci ?i cinci de camile ?i ?ase mii ?apte sute douazeci de asini.
Ezr 2:68  ?i ajungând unele din capeteniile semin?iilor la templul Domnului care este la Ierusalim, au daruit de bunavoie pentru templul lui Dumnezeu, ca sa fie ridicat din nou pe temeliile lui.
Ezr 2:69  ?i ace?tia din prisosul lor au daruit în casa ob?teasca pentru începerea lucrarilor: ?aizeci ?i una de mii de drahme de aur, cinci mii de mine de argint ?i o suta de ve?minte preo?e?ti.
Ezr 2:70  ?i au început sa locuiasca preo?ii ?i levi?ii ?i poporul ?i cântare?ii ?i portarii ?i cei încredin?a?i templului în ora?ele lor ?i tot Israelul s-a a?ezat în ceta?ile sale.
Ezr 3:1  Iar când a venit luna a ?aptea ?i când fiii lui Israel erau prin ora?ele lor, s-a strâns poporul ca un singur om la Ierusalim;
Ezr 3:2  ?i s-a sculat Iosua, fiul lui Io?adac, ?i fra?ii lui, preo?ii ?i Zorobabel, fiul lui Salatiel, ?i fra?ii lui ?i au facut jertfelnic Dumnezeului lui Israel, ca sa aduca pe el arderi de tot, cum se scrie în legea lui Moise, omul lui Dumnezeu.
Ezr 3:3  ?i au a?ezat jertfelnicul pe temeliile lui, de?i se temeau de popoarele straine, ?i au început sa aduca pe el arderi de tot Domnului, arderi de tot diminea?a ?i seara.
Ezr 3:4  ?i au savâr?it sarbatoarea corturilor, dupa cum este scris, cu ardere de tot zilnica, dupa numarul hotarât ?i dupa rânduiala fiecarei zile.
Ezr 3:5  Dupa aceea au început sa savâr?easca arderea de tot cea obi?nuita, ?i cea de la luna noua, ?i cea de la sarbatorile închinate Domnului, ?i jertfele pe care le aducea Domnului fiecare de buna voie.
Ezr 3:6  Chiar din ziua întâi a lunii a ?aptea au început sa aduca Domnului arderi de tot. Iar temelia templului Domnului nu se pusese înca.
Ezr 3:7  ?i au început a da argint pietrarilor ?i dulgherilor, iar Sidonienilor ?i Tirienilor bucate ?i bautura ?i untdelemn, ca sa aduca pe mare la Iafa lemn din Liban, dupa învoirea data lor de Cirus, regele Per?ilor.
Ezr 3:8  ?i în anul al doilea dupa sosirea lor la templul lui Dumnezeu din Ierusalim, în luna a doua, Zorobabel, fiul lui Salatiel, ?i Iosua, fiul lui Io?adac, ?i ceilal?i fra?i ai lor, preo?ii ?i levi?ii, ?i to?i cei ce venisera din robie la Ierusalim au facut început ?i au pus pe levi?ii de la douazeci de ani în sus sa supravegheze lucrarile templului Domnului.
Ezr 3:9  ?i a?a Iosua cu fiii ?i fra?ii lui, ?i Cadmiel cu fiii lui, fiii lui Iuda, precum ?i fiii lui Henadad cu fiii lor ?i cu fra?ii lor, levi?ii, au stat sa supravegheze pe cei ce lucrau la templul lui Dumnezeu.
Ezr 3:10  ?i când ziditorii puneau temelia templului Domnului, atunci preo?ii, îmbraca?i în ve?mintele lor ?i cu trâmbi?e, ?i levi?ii, fiii lui Asaf, cu chimvale, au fost pu?i sa laude pe Domnul, dupa rânduiala lui David, regele lui Israel.
Ezr 3:11  ?i au început ei sa cânte pe rând "lauda?i" ?i "slavi?i pe Domnul, ca este bun, ca în veac este mila Lui spre Israel". ?i tot poporul striga cu glas mare, slavind pe Domnul, pentru ca a ajutat sa se puna temeliile templului Domnului.
Ezr 3:12  În vremea aceasta mul?i din preo?i ?i din levi?i ?i din capii de familie ?i din batrânii care vazusera vechiul templu, vazând punerea temeliei acestui templu, plângeau cu hohote, mul?i însa cântau tare de bucurie;
Ezr 3:13  Însa poporul nu putea osebi strigatele de bucurie de bocetele de plâns ale mul?imii, pentru ca poporul striga tare ?i glasul se auzea departe.
Ezr 4:1  Auzind vrajma?ii lui Iuda ?i ai lui Veniamin ca cei ce s-au întors din robie zidesc templu Domnului Dumnezeului lui Israel,
Ezr 4:2  Au venit la Zorobabel ?i au zis catre el: "Sa zidim ?i noi cu voi, pentru ca ?i noi, ca ?i voi, cautam pe Dumnezeul vostru ?i Lui Îi aducem jertfe înca din zilele lui Asarhadon, regele Asiriei, care ne-a adus aici".
Ezr 4:3  Iar Zorobabel, Iosua ?i celelalte capetenii ale semin?iilor lui Israel le-au zis: "Nu se cuvine sa zidi?i împreuna cu noi templu Dumnezeului nostru, ci numai noi singuri vom zidi templu Domnului Dumnezeului lui Israel, precum ne-a poruncit Cirus, regele Per?ilor".
Ezr 4:4  Atunci poporul jarii aceleia a început sa descurajeze poporul lui Iuda ?i sa-l împiedice de la zidire,
Ezr 4:5  Cumparând contra lor pe sfetnicii regelui, ca sa zadarniceasca planul lor în toate zilele lui Cirus, regele Per?ilor, pâna în zilele lui Darie, regale Per?ilor.
Ezr 4:6  ?i sub domnia lui Aha?vero?, pe la începutul domniei acestuia, au scris plângere împotriva locuitorilor lui Iuda ?i ai Ierusalimului.
Ezr 4:7  ?i în zilele lui Artaxerxe, Bi?lam, Mitridat, Tabeel ?i ceilal?i tovara?i ai lor au scris lui Artaxerxe, regele Per?ilor. ?i scrisoarea a fost scrisa cu slove aramaice ?i în limba aramaica.
Ezr 4:8  ?i sfetnicul Rehum cu scriitorul ?im?ai înca au scris catre regele Artaxerxe urmatoarea scrisoare împotriva Ierusalimului:
Ezr 4:9  "Atunci Rehum, cârmuitorul, ?i ?im?ai scriitorul ?i ceilal?i tovara?i ai lor: Dineii ?i Arfarsateii, Tarpeleii, Afarseii, Erecii, Babilonienii, Suzienii, Dehaveii,
Ezr 4:10  Elami?ii ?i celelalte popoare, pe care stralucitul ?i marele Asurbanipal le-a stramutat ?i le-a a?ezat în ceta?ile Samariei ?i în celelalte ceta?i de peste râu, scriu catre regele Artaxerxe...".
Ezr 4:11  Iata copia de pe scrisoarea ce au trimis catre regele Artaxerxe: "Slugile tale, oamenii de dincolo de Eufrat...
Ezr 4:12  Cunoscut sa fie regelui ca Iudeii care au plecat de la tine ?i au venit la noi la Ierusalim rezidesc cetatea cea rea ?i razvratita ?i fac ziduri, ?i temeliile le-au ?i ispravit.
Ezr 4:13  ?i sa mai ?tie regele ca, daca cetatea aceasta se va zidi ?i zidurile ei se vor face din nou, atunci Iudeii nu vor plati nici bir, nici dari, nici vama ?i pagube se vor face vistieriei regale.
Ezr 4:14  ?i fiindca noi mâncam sare de la curtea regelui ?i nu putem suferi sa vedem pe rege pagubit, de aceea dam de ?tire regelui:
Ezr 4:15  Sa se caute în cartea faptelor parin?ilor tai ?i în cartea faptelor vei gasi ?i vei afla ca cetatea aceasta este cetate razvratita ?i primejdioasa pentru regi ?i ?inuturi ?i ca din vechime s-au petrecut în ea abateri, din care pricina a ?i fost pustiita cetatea aceasta.
Ezr 4:16  De aceea noi în?tiin?am pe rege ca, daca cetatea aceasta se va ispravi de zidit ?i zidurile ei se vor face, atunci nu vei mai avea stapânire peste râu".
Ezr 4:17  Iar regele a trimis raspunsul urmator lui Rehum cârmuitorul ?i lui ?im?ai scriitorul ?i celorlal?i tovara?i ai lor, care locuiesc în Samaria ?i în celelalte ceta?i de peste râu.
Ezr 4:18  "Pace... Scrisoarea ce mi-a?i trimis a fost citita cu luare-aminte înaintea noastra,
Ezr 4:19  ?i s-a dat din partea noastra porunca de s-a cercetat ?i s-a aflat ca cetatea aceea de mult s-a razvratit împotriva regilor ?i ca s-a facut în ea tulburari ?i rascoale;
Ezr 4:20  ?i ca au fost în Ierusalim regi puternici care au stapânit toata latura cea de peste râu ?i carora li s-au platit bir ?i vama.
Ezr 4:21  A?adar, porunca da?i ca oamenii aceia sa înceteze de a mai lucra ?i ca cetatea aceea sa nu se mai zideasca, pâna nu va veni porunca de la mine;
Ezr 4:22  ?i sa fi?i cu luare-aminte, ca sa nu va scape ceva nebagat în seama în treburile acestea. De ce îngadui?i înmul?irea lucrarilor vatamatoare în paguba regelui?"
Ezr 4:23  Îndata ce s-a citit scrisoarea aceasta a regelui Artaxerxe, înaintea lui Rehum ?i a lui ?im?ai scriitorul ?i a tovara?ilor lor, ace?tia au trimis îndata la Ierusalim ?i cu puterea armelor au oprit lucrarile Iudeilor.
Ezr 4:24  Atunci s-au oprit lucrarile la templul lui Dumnezeu cel din Ierusalim, ?i oprirea aceasta a ?inut pâna în anul al doilea al domniei lui Darie, regele Per?ilor.
Ezr 5:1  Dar proorocul Agheu ?i proorocul Zaharia, fiul lui Ido, au grait Iudeilor celor din Ierusalim ?i din Iuda cuvinte prooroce?ti în numele Dumnezeului lui Israel.
Ezr 5:2  Atunci s-au sculat Zorobabel, fiul lui Salatiel, ?i Iosua, fiul lui Io?adac, ?i au început a zidi templul lui Dumnezeu cel din Ierusalim, fiind cu ei proorocii lui Dumnezeu, care îi întareau.
Ezr 5:3  În vremea aceea au venit la el Tatnai, cârmuitorul ?inuturilor de peste fluviu, ?i ?etar-Boznai ?i tovara?ii lor, ?i le-au grait a?a: "Cine v-a dat voua învoire sa zidi?i casa aceasta ?i sa ispravi?i zidurile acestea?"
Ezr 5:4  Atunci noi le-am spus numele acelor oameni care zideau casa aceasta.
Ezr 5:5  Dar ochiul Dumnezeului lor era spre capeteniile iudaice, ?i aceia nu i-au mustrat pâna nu au adus lucrurile la cuno?tin?a lui Darie ?i pâna n-a venit dezlegare în chestiunea aceasta.
Ezr 5:6  Iata cuprinsul scrisorii pe care Tatnai, cârmuitorul ?inuturilor de peste Eufrat, ?i ?etar-Boznai cu tovara?ii lor, cu Arfarsateii cei de peste fluviu, au trimis-o regelui Darie,
Ezr 5:7  ?i iata ce era scris în în?tiin?area trimisa lui:
Ezr 5:8  "Pace întru toate regelui Darie! Cunoscut fie regelui ca noi am fost în ?inutul lui Iuda, la templul Dumnezeului celui mare, ?i am vazut ca el se zide?te din pietre mari ?i se pune ?i lemn în zid ?i lucrarea aceasta merge repede ?i au spor la mâna.
Ezr 5:9  Atunci am întrebat noi pe capetenii ?i le-am zis a?a: Cine v-a dat voua învoire sa zidi?i casa aceasta ?i zidurile acestea sa le ispravi?i?
Ezr 5:10  Pe lânga aceasta i-am mai întrebat ?i de numele acelora, ca sa-?i dam de veste ?i sa-?i seriem numele acelor oameni, care sunt mai însemna?i la ei.
Ezr 5:11  ?i ei mi-au raspuns eu vorbele acestea: Noi suntem robii Dumnezeului cerului ?i al pamântului ?i zidim templul care a fost facut cu mult înainte de aceasta ?i pe care un mare rege al lui Israel l-a zidit ?i l-a ispravit.
Ezr 5:12  Când însa parin?ii no?tri au mâniat pe Dumnezeul cerului, Acesta i-a dat în mâna lui Nabucodonosor, regele Babilonului, care a darâmat templul acesta ?i pe popor l-a stramutat la Babilon.
Ezr 5:13  Iar în anul întâi al lui Cirus, regele Babilonului, regele Cirus a dat învoire sa se zideasca acest templu al lui Dumnezeu;
Ezr 5:14  Chiar ?i vasele de aur ?i de argint ale templului lui Dumnezeu, pe care Nabucodonosor le luase din templul Ierusalimului ?i le dusese în capi?tea din Babilon, le-a scos regele Cirus din capi?tea Babilonului ?i le-a dat în seama lui ?e?ba?ar, pe care l-a numit el cârmuitor ?i i-a zis:
Ezr 5:15  Ia vasele acestea, mergi ?i pune-le în templul din Ierusalim, ?i templul lui Dumnezeu sa se zideasca pe locul lui vechi.
Ezr 5:16  Atunci ?e?ba?ar acela a venit ?i a pus temelia templului lui Dumnezeu din Ierusalim, ?? de atunci ?i pâna astazi se zide?te el ?i nu s-a ispravit.
Ezr 5:17  Deci, daca binevoie?te regele, sa se caute în casa vistieriei regale de acolo din Babilon ?i sa se vada daca în adevar regele Cirus a dat învoire sa se zideasca acest templu al lui Dumnezeu în Ierusalim ?i sa ni se trimita ?tire care este voin?a regelui în aceasta privin?a".
Ezr 6:1  Atunci regele Darie a dat porunca sa se cerceteze la locul unde se pastrau actele ?i unde se ?inea vistieria la Babilon.
Ezr 6:2  ?i s-a gasit în palatul din Ecbatana, care este în ?inutul Mediei, un sul de pergament, în care era scrisa aceasta amintire:
Ezr 6:3  "În anul întâi al regelui Cirus, regele Cirus a dat aceasta porunca pentru templul lui Dumnezeu cel din Ierusalim: Sa se zideasca templul pe locul acela unde se aduc jertfele ?i sa i se puna temelii sanatoase! Înal?imea lui sa fie de ?aizeci de co?i ?i largimea lui tot de ?aizeci de co?i.
Ezr 6:4  Sa se puna trei rânduri de pietre ?i un rând de lemn, iar cheltuielile sa se dea din casa regelui.
Ezr 6:5  Chiar ?i vasele de aur ?i de argint ale templului lui Dumnezeu, pe care Nabucodonosor le-a luat din templul din Ierusalim ?i le-a dus la Babilon sa se înapoieze ?i sa se duca în templul din Ierusalim ?i sa se puna fiecare la locul lui în templul lui Dumnezeu".
Ezr 6:6  "Deci, Tatnai, cârmuitor al ?inuturilor de peste fluviu, ?i ?etar-Boznai cu tovara?ii vo?tri, cu Arfarsateii cei de peste fluviu, departa?i-va de acolo ?i nu opri?i lucrarile la acel templu al lui Dumnezeu;
Ezr 6:7  Lasa pe cârmuitorul iudeu ?i pe capeteniile iudaice sa zideasca acel templu al lui Dumnezeu pe locul lui.
Ezr 6:8  Iar din partea mea se da porunca cu privire la cele cu care voi trebuie sa ajuta?i acelor capetenii iudaice la zidirea acelui templu al lui Dumnezeu: Lua?i numaidecât din birul ?inuturilor de peste fluviu ?i da?i oamenilor acelora, ca lucrul sa nu se opreasca; ?i ce trebuie pentru arderile de tot ale Dumnezeului ceresc:
Ezr 6:9  Vi?ei, sau berbeci, sau miei, precum ?i grâu, sare, vin ?i untdelemn, sa li se dea, cum vor zice preo?ii cei din Ierusalim,
Ezr 6:10  Ca ei sa aduca jertfa placuta Dumnezeului ceresc ?i sa se roage pentru sanatatea regelui ?i a fiilor lui.
Ezr 6:11  Tot de mine porunca se mai da ca daca vreun om va schimba aceasta hotarâre, atunci se va scoate o bârna din casa lui ?i va fi acela ridicat ?i pironit pe ea, iar casa lui se va preface pentru aceasta în darâmatura.
Ezr 6:12  ?i Dumnezeu al Carui nume locuie?te acolo sa doboare pe tot regele ?i poporul care ?i-ar întinde mâna sa ca sa schimbe aceasta în paguba acestui templu al lui Dumnezeu din Ierusalim. Eu, Darie, am dat porunca aceasta; sa fie întocmai adusa la îndeplinire!"
Ezr 6:13  Atunci Tatnai, cârmuitorul ?inuturilor de peste fluviu, ?i ?etar-Boznai cu tovara?ii lor au facut întocmai a?a, cum poruncise regele Darie.
Ezr 6:14  ?i capeteniile Iudeilor au zidit ?i au sporit, dupa proorocia lui Agheu proorocul ?i a lui Zaharia, fiul lui Ido. ?i cu voia Dumnezeului lui Israel ?i a lui Cirus ?i Darie ?i Artaxerxe, regii Per?ilor, l-au zidit ?i l-au ispravit.
Ezr 6:15  ?i s-a ispravit templul acesta în ziua a treia a lunii Adar, în al ?aselea an al domniei regelui Darie.
Ezr 6:16  ?i fiii lui Israel, preo?ii ?i ceilal?i, care se întorsesera din robie, au savâr?it cu bucurie sfin?irea acestui templu al lui Dumnezeu.
Ezr 6:17  La sfin?irea acestui templu al lui Dumnezeu s-au adus: o suta de boi, doua sute de berbeci, patru sute de miei ?i doisprezece ?api, jertfa de iertarea pacatelor pentru tot Israelul, dupa numarul semin?iilor lui Israel.
Ezr 6:18  ?i au fost pu?i preo?ii cu rândul ?i levi?ii tot cu rândul sa slujeasca lui Dumnezeu în Ierusalim, cum era scris în cartea lui Moise.
Ezr 6:19  ?i cei ce se întorsesera din robie au savâr?it Pa?tile în ziua a paisprezecea a lunii întâi,
Ezr 6:20  Pentru ca se cura?isera preo?ii ?i levi?ii ?i cu to?ii pâna la unul erau cura?i; ?i au junghiat mielul Pa?tilor pentru to?i cei ce se întorsesera din robie, pentru fra?ii lor preo?i ?i pentru ei în?i?i.
Ezr 6:21  ?i au mâncat fiii lui Israel, cei ce se întorsesera din robie ?i cei ce se despar?isera cu ei de necura?enia popoarelor ?arii, ca sa caute pe Domnul Dumnezeul lui Israel;
Ezr 6:22  ?i au praznuit sarbatoarea azimelor ?apte zile cu bucurie, pentru ca îi înveselise Domnul ?i întorsese spre ei inima regelui Asiriei, ca sa le întareasca mâinile la zidirea templului Domnului Dumnezeului lui Israel.
Ezr 7:1  Dupa întâmplarile acestea, sub domnia lui Artaxerxe, regele Per?ilor, Ezdra, fiul lui Seraia, fiul lui Azaria, fiul lui Hilchia,
Ezr 7:2  Fiul lui ?alum, fiul lui ?adoc, fiul lui Ahitub,
Ezr 7:3  Fiul lui Amaria, fiul lui Azaria, fiul lui Meraiot,
Ezr 7:4  Fiul lui Zerahia, fiul lui Uzi,
Ezr 7:5  Fiul lui Buchi, fiul lui Abi?ua, fiul lui Finees, fiul lui Eleazar, fiul lui Aaron arhiereul,
Ezr 7:6  Acest Ezdra a ie?it din Babilon; ?i era el carturar iscusit ?i cunoscator al legii lui Moise, pe care o daduse Domnul Dumnezeul lui Israel. Iar regele i-a dat lui tot ce a dorit, pentru ca mâna Dumnezeului sau era peste el.
Ezr 7:7  ?i împreuna cu el au plecat la Ierusalim ?i unii din fiii lui Israel ?i preo?i ?i levi?i ?i cântare?i ?i portari ?i cei încredin?a?i templului, în anul al ?aptelea al domniei lui Artaxerxe.
Ezr 7:8  ?i a venit el la Ierusalim tot în al ?aptelea an al regelui, în luna a cincea;
Ezr 7:9  Caci în ziua întâi a lunii întâi a fost plecarea lui din Babilon, iar în ziua întâi a lunii a cincea a ajuns la Ierusalim, pentru ca mâna binefacatoare a lui Dumnezeu era peste el;
Ezr 7:10  Caci Ezdra se hotarâse cu toata inima sa înve?e legea Domnului ?i s-o împlineasca ?i sa înve?e pe Israel legea ?i dreptatea.
Ezr 7:11  Iata acum ?i cuprinsul scrisorii pe care Artaxerxe a dat-o lui Ezdra, preotul ?i carturarul, care propovaduise în Israel cuvintele poruncilor Domnului ?i ale legilor Lui:
Ezr 7:12  "Artaxerxe, regele regilor, catre Ezdra, preotul, înva?atorul legii Dumnezeului ceresc Celui desavâr?it...
Ezr 7:13  S-a dat de mine porunca, ca în regatul meu to?i aceia din poporul lui Israel ?i din preo?ii lui ?i din levi?ii lui, care doresc sa se duca la Ierusalim, sa mearga cu tine.
Ezr 7:14  Fiindca tu e?ti trimis de rege ?i de cei ?apte sfetnici ai lui, ca sa cercetezi Iuda ?i Ierusalimul dupa legea Dumnezeului tau, pe care o ai în mâna ta,
Ezr 7:15  ?i sa duci argintul ?i aurul pe care regele ?i sfetnicii lui l-au jertfit Dumnezeului lui Israel a Caruia locuin?a este în Ierusalim
Ezr 7:16  ?i tot aurul ?i argintul pe care-l vei aduna tu din toata ?ara Babilonului, împreuna cu toate darurile de buna voie de la popor ?i preo?i, pe care le vor jertfi ei pentru templul Dumnezeului lor, cel din Ierusalim.
Ezr 7:17  De aceea cumpara numaidecât cu banii ace?tia, boi, berbeci, miei ?i daruri de pâine cât trebuie, ?i turnari pentru ei ?i du-le la jertfelnicul templului Dumnezeului vostru cel din Ierusalim.
Ezr 7:18  ?i ce ve?i crede voi ?i fra?ii vo?tri ca este bine sa face?i cu celalalt argint ?i aur, aceea sa face?i dupa voia Dumnezeului vostru.
Ezr 7:19  ?i vasele ce ?i s-au dat ?ie pentru slujbele templului Dumnezeului tau, pune-le înaintea Dumnezeului Ierusalimului.
Ezr 7:20  ?i alte lucruri de trebuin?a pentru templul Dumnezeului tau, ce vei crede tu ca trebuie, da-le din casa vistieriilor rege?ti.
Ezr 7:21  Din partea mea, a regelui Artaxerxe, se da tuturor pastratorilor vistieriilor de peste râu porunca aceasta: Tot ce va cere de la voi preotul Ezdra, înva?atorul legii Dumnezeului ceresc, sa-i da?i numaidecât;
Ezr 7:22  Argint pâna la o suta de talan?i, grâu pâna la o suta de core, vin pâna la o suta de baturi ?i tot pâna la o suta de baturi de untdelemn; iar sare, fara masura.
Ezr 7:23  Tot ce s-a poruncit de Dumnezeul ceresc trebuie sa se faca cu îngrijire pentru templul Dumnezeului celui ceresc. Baga?i de seama sa nu-?i întinda cineva mâna asupra templului Dumnezeului celui ceresc, ca sa nu fie mânia Lui asupra regatului, a regelui ?i a fiilor lui.
Ezr 7:24  ?i va dam ?tire ca nici asupra unuia din preo?i, sau levi?i, sau cântare?i, sau portari, sau cei încredin?a?i templului, sau slujitori ai acestui templu al lui Dumnezeu sa nu se puna nici bir, nici dare, nici vama.
Ezr 7:25  Iar tu, Ezdra, dupa în?elepciunea Dumnezeului tau, care este în mina ta, sa pui cârmuitori ?i judecatori ?i sa judece aceia tot poporul cel de peste fluviu, pe to?i cei ce ?tiu legea Dumnezeului tau, iar pe cei ce nu o ?tiu, sa-i înva?a?i.
Ezr 7:26  ?i cine nu va împlini legea Dumnezeului tau ?i legea regelui, asupra aceluia sa se faca îndata judecata ?i sa se osândeasca sau la moarte, sau la izgonire, sau la amenda, sau la închidere în temni?a".
Ezr 7:27  "Binecuvântat este Domnul Dumnezeul parin?ilor no?tri Care a pus în inima regelui gândul sa împodobeasca templul Domnului cel din Ierusalim ?i a atras asupra mea mila regelui ?i a sfetnicilor lui ?i a tuturor dregatorilor celor puternici ai regelui!
Ezr 7:28  Atunci eu m-am întarit, caci mâna Domnului Dumnezeului meu era peste mine, ?i am adunat pe capeteniile lui Israel, ca sa mearga cu mine".
Ezr 8:1  "Iata capii de familie ?i spi?a neamului acelora care au plecat cu mine din Babilon, în timpul domniei regelui Artaxerxe:
Ezr 8:2  Gher?om din fiii lui Finees; Daniel din fiii lui Itamar; Hatu? din fiii lui David;
Ezr 8:3  Zaharia din fiii lui ?ecania, care se tragea din fiii lui Fares, ?i împreuna cu el o suta cincizeci de suflete, parte barbateasca, scrise în spi?a neamului;
Ezr 8:4  Elioenai, fiul lui Zerahia, din neamul lui Pahat-Moab împreuna cu doua sute de suflete, parte barbateasca.
Ezr 8:5  ?ecania, fiul lui Iahaziel, din urma?ii lui Zatu, cu trei sute de suflete, parte barbateasca.
Ezr 8:6  Ebed, fiul lui Ionatan, din urma?ii lui Adin, cu cincizeci de suflete, parte barbateasca.
Ezr 8:7  Isaia, fiul lui Atalia, din urma?ii lui Elam, cu ?aptezeci de oameni.
Ezr 8:8  Zebadia, fiul lui Mihail, din urma?ii lui ?efatia, cu optzeci de oameni.
Ezr 8:9  Obadia, fiul lui Iehiel, din urma?ii lui Ioab, cu doua sute optsprezece oameni.
Ezr 8:10  ?elomit, fiul lui Iosifia, din urma?ii lui Lani, cu o suta ?aizeci de oameni.
Ezr 8:11  Zaharia, fiul lui Bebai, din urma?ii lui Bebai, cu douazeci ?i opt de oameni.
Ezr 8:12  Iohanan, fiul lui Hacatan, din urma?ii lui Azgad, cu o suta zece oameni.
Ezr 8:13  ?i cei din urma din fiii lui Adonicam, ale caror nume erau: Ielifelet, Ieiel ?i ?emaia cu ?aizeci de oameni.
Ezr 8:14  Utai ?i Zabud, din fiii lui Bigvai, cu ?aptezeci de oameni.
Ezr 8:15  Pe ace?tia i-am adunat eu la râul ce curge prin Ahava ?i am poposit acolo trei zile; iar când am cercetat eu poporul ?i pe preo?i, n-am gasit acolo pe nimeni din fiii lui Levi.
Ezr 8:16  ?i am trimis sa cheme pe Eleazar, Ariel, ?emaia, Elnatan, Iariv, Elnatan, Natan, Zaharia ?i Me?ulam, care erau capetenii, ?i pe Ioarib ?i Elnatan, care erau înva?atori,
Ezr 8:17  ?i le-am dat acestora însarcinare catre Ido, care era capetenie în ?inutul Casifia, ?i le-am pus în gura lor ce sa graiasca cu Ido ?i cu fra?ii lui, ?i cu cei încredin?a?i templului din ?inutul Casifia, ca sa ne aduca slujitori pentru templul Dumnezeului nostru.
Ezr 8:18  Pentru ca mâna binefacatoare a Dumnezeului nostru era peste noi, ne-au adus ei un om în?elept din fiii lui Mahli, fiul lui Levi, fiul lui Israel, anume pe ?erevia, ?i pe fiii acestuia ?i pe fra?ii lui în numar de optsprezece;
Ezr 8:19  ?i ne-au mai adus pe Ha?abia ?i pe Isaia din fiii lui Merari, împreuna cu fra?ii lor ?i cu fiii lor, douazeci de oameni;
Ezr 8:20  ?i dintre cei încredin?a?i templului pe care i-a dat David ?i dregatorii lui în slujba levi?ilor, ne-a adus doua sute douazeci de in?i; ace?tia to?i erau numi?i pe nume.
Ezr 8:21  ?i acolo, la râul Ahava, am rânduit post, ca sa ne smerim înaintea fe?ei Dumnezeului nostru ?i sa cerem de la El calatorie buna pentru noi ?i pentru copiii no?tri ?i pentru toata avu?ia noastra,
Ezr 8:22  Caci îmi fusese ru?ine sa cer de la rege o?tire ?i calare?i, ca sa ne pazeasca de vrajma?i în cale, ca noi, când am grait cu regele, am zis: "Mâna Dumnezeului nostru este binefacatoare pentru to?i cei ce alearga la El, iar asupra tuturor celor ce-L parasesc este puterea Lui ?i mânia Lui!"
Ezr 8:23  ?i a?a am postit noi ?i am rugat pentru aceasta pe Dumnezeul nostru, ?i El ne-a auzit.
Ezr 8:24  ?i am luat din cei ce erau mai mari peste preo?i doisprezece oameni: pe ?erevia ?i pe Ha?abia ?i împreuna cu ei pe cei zece fra?i ai lor.
Ezr 8:25  ?i le-am dat lor cu cântarul aurul ?i argintul ?i vasele ?i tot ce se daruise pentru templul Dumnezeului nostru, ce daruise regele ?i sfetnicii lui ?i dregatorii lui ?i to?i Israeli?ii care se aflau acolo.
Ezr 8:26  Acestea le-am dat în mâna lor, cântarite: argint, ?ase sute cincizeci de talan?i, vase de argint, ca la o suta de talan?i, aur o suta de talan?i,
Ezr 8:27  Cupe de aur, douazeci, de o mie de drahme una, ?i doua vase de arama din cea mai buna, lucitoare, care se pre?uie?te ca ?i aurul.
Ezr 8:28  ?i le-am zis: "Voi sunte?i sfin?i?ii Domnului ?i vasele sunt sfin?ite, iar argintul ?i aurul sunt darurile cele de buna voie Domnului Dumnezeului parin?ilor vo?tri!
Ezr 8:29  Veghea?i ?i pazi?i acestea, pâna le ve?i da cu cântarul mai-marilor preo?ilor, levi?ilor ?i capeteniilor semin?iilor lui Israel la Ierusalim, în camerele templului Domnului".
Ezr 8:30  ?i au primit preo?ii ?i levi?ii aurul ?i argintul ?i vasele cântarite ca sa le duca la Ierusalim, în templul Dumnezeului nostru.
Ezr 8:31  Dupa aceea am plecat noi de la râul Ahava în ziua a douasprezecea a lunii întâi, ca sa mergem la Ierusalim; ?i mâna Dumnezeului nostru a fost cu noi ?i ne-a scapat din mâna vrajma?ului ?i de cei ce ne pândeau în cale.
Ezr 8:32  ?i am venit la Ierusalim ?i am ramas acolo trei zile,
Ezr 8:33  Iar a patra zi am dat cu cântarul argintul ?i aurul ?i vasele la templul Dumnezeului nostru, în mâna lui Meremot preotul, fiul lui Urie, împreuna ?i lui Eleazar, fiul lui Finees, precum ?i lui Iozabat, fiul lui Iosua, ?i lui Noadia, fiul lui Binui, levi?ii.
Ezr 8:34  Toate le-am dat cântarit ?i numarat ?i toate cele cântarite s-au scris în acela?i timp.
Ezr 8:35  ?i cei veni?i din robie au adus ardere de tot Dumnezeului lui Israel, doisprezece vi?ei pentru tot Israelul, douazeci ?i ?ase de berbeci, ?aptezeci ?i ?apte de miei ?i doisprezece ?api, jertfa pentru pacat; toate acestea le-au adus ardere de tot Domnului.
Ezr 8:36  ?i am dat poruncile regelui satrapilor ?i guvernatorilor de peste râu, ?i ace?tia au aratat cinste poporului ?i templului lui Dumnezeu".
Ezr 9:1  "Dupa ispravirea acestora, au venit la mine capeteniile ?i au zis: "Poporul lui Israel ?i preo?ii ?i levi?ii nu s-au deosebit de popoarele cele de alt neam ?i de urâciunile lor, adica de Canaanei, Hetei, Ferezei, Iebusei, Amoni?i, Moabi?i, Egipteni ?i Amorei,
Ezr 9:2  Pentru ca au luat pe fiicele acelora so?ii pentru ei ?i pentru feciorii lor ?i s-a amestecat samân?a cea sfânta cu popoarele cele de alt neam, ba înca mâna celor mai însemna?i ?i mai de frunte a fost cea dintâi în aceasta nelegiuire".
Ezr 9:3  Auzind cuvântul acesta, mi-am rupt haina cea de deasupra ?i cea de dedesubt ?i mi-am smuls parul din capul meu ?i din barba mea ?i am cazut de mâhnire.
Ezr 9:4  Atunci s-au adunat la mine to?i cei ce se temeau de cuvintele Dumnezeului lui Israel, din pricina nelegiuirii celor veni?i din robie, ?i eu am stat în întristare pâna la jertfa cea de seara.
Ezr 9:5  Iar la vremea jertfei de seara m-am sculat din locul tânguirii mele ?i cu hainele rupte de deasupra ?i de dedesubt am cazut în genunchi ?i mi-am întins mâinile catre Domnul Dumnezeul meu ?i am zis:
Ezr 9:6  "Dumnezeul meu, ma ru?inez ?i ma tem sa-mi ridic fa?a catre Tine, Dumnezeul meu, pentru ca faradelegile noastre au trecut peste cap ?i vina noastra s-a marit pâna la cer.
Ezr 9:7  Din zilele parin?ilor no?tri ?i pâna astazi suntem în mare vinova?ie, ?i pentru faradelegile noastre am fost da?i noi, ?i regii no?tri, ?i preo?ii no?tri, în mâinile regilor celor de alt neam, ?i în sabie ?i în robie, ?i prada ?i în ru?ine, cum suntem ?i astazi.
Ezr 9:8  ?i iata, dupa pu?ina vreme, ni s-a dat îndurare de la Domnul Dumnezeul nostru ?i El ne-a lasat pe câ?iva sa scapam ?i ne-a ajutat sa ne a?ezam în locul cel sfânt al Lui ?i ne-a daruit sa ne învioram pu?in din robia noastra.
Ezr 9:9  Noi robi suntem, dar nici în robie nu ne-a parasit pe noi Dumnezeul nostru ?i a îndreptat El spre noi mila regilor Per?ilor ca sa ne lase sa înviem, sa zidim templu Dumnezeului nostru, sa-l scoatem din darâmaturile lui ?i ne-a dat întarire în Iuda ?i în Ierusalim.
Ezr 9:10  Dar acum, dupa toate acestea, ce vom zice noi, Dumnezeul nostru? Caci ne-am abatut de la poruncile Tale,
Ezr 9:11  Pe care le-ai dat Tu prin prooroci, robii Tai, ?i ai zis: Pamântul în care va duce?i voi ca sa-l stapâni?i este pamânt necurat, caci este spurcat de necura?enia popoarelor celor de alt neam ?i de urâciunile lor, cu care ele l-au umplut de la un capat la altul.
Ezr 9:12  Deci pe fetele voastre sa nu le da?i dupa feciorii lor ?i pe fetele lor sa nu le lua?i pentru feciorii vo?tri ?i pacea lor ?i bunurile lor sa nu le cauta?i în veci, ca sa va întari?i ?i sa va hrani?i cu bunata?ile pamântului aceluia ?i sa-l lasa?i mo?tenire ve?nica fiilor vo?tri.
Ezr 9:13  Dupa toate cele ce ne-au ajuns pe noi pentru faptele noastre cele rele ?i pentru marea noastra vinova?ie ?i pentru ca Tu, Dumnezeule, nu Te-ai purtat cu noi dupa masura nelegiuirilor noastre ?i ne-ai dat ?i izbavirea aceasta,
Ezr 9:14  Au doara iara?i vom calca poruncile Tale ?i vom intra în legatura de rudenie cu aceste popoare ticaloase? Nu Te vei mânia Tu oare atâta, încât sa ne stârpe?ti ?i sa nu mai ramâna nici unul ?i sa nu mai fie nici o izbavire?
Ezr 9:15  Doamne Dumnezeul lui Israel, drept e?ti Tu, pentru ca am scapat noi pâna în ziua de astazi; ?i iata noi ?i astazi suntem în nelegiuirile noastre; deci astfel fiind, n-ar trebui sa stam înaintea fe?ei Tale".
Ezr 10:1  Pe când se ruga astfel Ezdra ?i se marturisea, plângând ?i îngenunchind înaintea templului lui Dumnezeu, s-a strâns la el o mare mul?ime de Israeli?i, barba?i ?i femei ?i copii, pentru ca ?i poporul a plâns foarte mult.
Ezr 10:2  ?i a grait ?ecania, fiul lui Iehiel, care era din urma?ii lui Elam, ?i a zis catre Ezdra: "Noi am facut nelegiuire înaintea Dumnezeului nostru, când ne-am luat femei de alt neam din popoarele pamântului acestuia, dar mai este înca o nadejde pentru Israel în lucrul acesta;
Ezr 10:3  Sa încheiem acum legamânt cu Dumnezeul nostru ca, dupa sfatul stapânului meu ?i al celor ce cinstesc poruncile Dumnezeului nostru, sa dam drumul tuturor femeilor ?i copiilor nascu?i cu ele, ca sa fim dupa lege.
Ezr 10:4  Scoala deci, ca aceasta este treaba ta, îmbarbateaza-te ?i lucreaza, ca noi suntem cu tine".
Ezr 10:5  ?i s-a sculat Ezdra ?i a pus pe capeteniile preo?ilor ?i ale levi?ilor ?i pe tot Israelul sa faca juramânt ca vor face a?a. ?i ei au facut juramânt.
Ezr 10:6  ?i dupa ce s-a sculat, Ezdra s-a dus de la templul lui Dumnezeu la locuin?a lui Iohanan, fiul lui Elia?ib ?i, ajungând acolo, n-a mâncat pâine, nici, apa n-a baut, caci plângea pentru nelegiuirea celor din robie.
Ezr 10:7  ?i a facut cunoscut în Iuda ?i în Ierusalim tuturor celor ce fusesera în robie sa se adune la Ierusalim;
Ezr 10:8  ?i cel ce nu va veni pâna în trei zile, pe averea aceluia, dupa sfatul capeteniilor ?i al batrânilor, se va pune blestem, iar el însu?i va fi îndepartat din ob?tea celor ce fusesera în robie.
Ezr 10:9  ?i s-au adunat to?i locuitorii Iudei ?i ai ?inutului lui Veniamin la Ierusalim în trei zile. ?i aceasta era în luna a noua, în ziua a douazecea a lunii acesteia. ?i s-a a?ezat tot poporul în pia?a de la templul lui Dumnezeu, tremurând atât pentru pacatul acesta, cât ?i din pricina ploilor.
Ezr 10:10  ?i s-a sculat Ezdra preotul ?i le-a zis: "Voi a?i facut pacat, luându-va femei de neam strain ?i cu aceasta a?i marit vina lui Israel.
Ezr 10:11  A?adar, pocai?i-va de pacatul acesta înaintea Domnului Dumnezeului parin?ilor vo?tri ?i face?i voia Lui ?i departa?i-va de popoarele pamântului acestuia ?i de femeile celor de alt neam".
Ezr 10:12  ?i raspunzând toata adunarea, a zis cu glas tare: "Cum zici tu, a?a vom face!
Ezr 10:13  Însa poporul este mult la numar ?i acum este timp ploios ?i nu putem sta afara. ?i apoi ?i treaba aceasta nu este de-o zi ori de doua, pentru ca mul?i din noi am savâr?it acest pacat.
Ezr 10:14  Deci sa ramâna capeteniile noastre pentru întreaga ob?te ?i to?i cei din ora?ele noastre care ?i-au luat femei straine sa vina aici la vremea hotarâta ?i împreuna cu ei sa vina ?i capeteniile fiecarui ora? ?i judecatorii lui, pâna se va potoli de la noi mânia cea arzatoare a Dumnezeului nostru, care s-a pornit pentru pacatul acesta".
Ezr 10:15  Atunci Ionatan, fiul lui Asael ?i Iahzeia, fiul lui Ticva, au fost pu?i pentru lucrul acesta; iar levi?ii Me?ulam ?i ?abetai erau ajutoarele lor.
Ezr 10:16  Cei ce se întorsesera din robie au facut a?a. ?i Ezdra preotul a rânduit la treaba aceasta ?i pe capeteniile semin?iilor din fiecare semin?ie ?i i-a numit pe nume. ?i au facut ei sfat în ziua întâi a lunii a zecea, ca sa cerceteze lucrul acesta,
Ezr 10:17  ?i au ispravit cercetarea tuturor celor ce-?i luasera femei de alt neam, în ziua întâi a lunii întâi.
Ezr 10:18  Din fiii preo?ilor care-?i luasera femei straine, s-au gasit: Maaseia, Eliezer, Iariv ?i Ghedalia, din fiii lui Iosua, al lui Io?adac ?i fra?ii lui;
Ezr 10:19  ?i ace?tia ?i-au dat mâinile ca vor da drumul femeilor lor ?i ca vor aduce jertfa un berbec pentru vina lor.
Ezr 10:20  ?i s-au mai gasit: Hanani ?i Zebadia, din fiii lui Imer;
Ezr 10:21  Maaseia, Ilie, ?emaia, Iehiel ?i Uzia, din fiii lui Harim;
Ezr 10:22  Elioenai, Maaseia, Ismael, Natanael, Iozabad ?i Eleasa, din fiii lui Pa?hur;
Ezr 10:23  Iozabad, ?imei, Chelaia, zis ?i Chelita, Petahia, Iuda ?i Eliezer, din levi?i;
Ezr 10:24  Elia?ib, din cântare?i; ?alum, Telem ?i Uri, din portari.
Ezr 10:25  Iar din Israeli?i: Ramia, Izia, Malchia, Miamin, Eleazar, Malchia ?i Benaia, din fiii lui Fares;
Ezr 10:26  Matania, Zaharia, Iehiel, Abdie, Iremot ?i Ilie, din fiii lui Elam;
Ezr 10:27  Elioenai, Elia?ib, Matania, Ieremot, Zabad ?i Aziza, din fiii lui Zatu;
Ezr 10:28  Iohanan, Hanania, Zabai ?i Atlai, din fiii lui Bebai;
Ezr 10:29  Me?ulam, Maluc, Adaia, Ia?ub, ?eal ?i Ieramot, din fiii lui Bani;
Ezr 10:30  Adna, Chelal, Benaia, Maaseia, Matania, Binui, Manase ?i Be?aleel, din fiii lui Pahat-Moab;
Ezr 10:31  Eliezer, I?ia, Malchia, ?emaia,
Ezr 10:32  Simeon, Veniamin, Maluc ?i ?emaria, din fiii lui Harim;
Ezr 10:33  Matnai, Matata, Zabad, Elifelet, Ieremai, Manase ?i ?imei, din fiii lui Ha?un;
Ezr 10:34  Iar din fiii lui Bani: Maadai, Amram, Ioel,
Ezr 10:35  Benaia, Bedia, Cheluhu,
Ezr 10:36  Vania, Meremot, Elia?ib,
Ezr 10:37  Matania, Matnai, Iaa?ai,
Ezr 10:38  Bani, Binui, ?imei,
Ezr 10:39  ?elemia, Natan, Adaia,
Ezr 10:40  Macnadbai, ?a?ai, ?arai,
Ezr 10:41  Azariel, ?elemiahu, ?emaria,
Ezr 10:42  ?alum, Amaria ?i Iosif;
Ezr 10:43  ?i în sfâr?it din fiii lui Nebo: Ieiel, Matitia, Zabad, Zebina, Iadai, Ioel ?i Benaia.
Ezr 10:44  To?i ace?tia î?i luasera femei straine ?i unele din aceste femei le nascusera copii.


\end{document}