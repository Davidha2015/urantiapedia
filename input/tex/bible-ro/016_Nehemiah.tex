\begin{document}

\title{Nehemiah}

Neh 1:1  Cuvintele lui Neemia, fiul lui Hacalia. "În luna Chislev, în anul al douazecilea al regelui Artaxerxe, mi s-a întâmplat sa fiu în capitala Suza.
Neh 1:2  Atunci a venit din Iuda Hanani, unul din fra?ii mei ?i al?i câ?iva oameni; ?i i-am întrebat despre cei ee scapasera din robie ?i ramasesera în Iuda ?i despre Ierusalim.
Neh 1:3  "Cei ce au scapat din robie ?i au ramas - îmi spusera ei - sunt acolo, în ?ara lor, în mare necaz ?i înjosire; iar zidurile Ierusalimului sunt darâmate ?i por?ile lui arse".
Neh 1:4  Auzind eu cuvintele acestea, am început sa plâng ?i am fost întristat câteva zile, am postit ?i m-am rugat înaintea Dumnezeului ceresc, zicând:
Neh 1:5  "Doamne, Dumnezeule al cerurilor, Dumnezeul cel mare ?i înfrico?ator, Care paze?ti legamântul Tau ?i e?ti milostiv cu cei ce Te iubesc ?i pazesc poruncile Tale,
Neh 1:6  Sa fie urechile Tale cu luare-aminte ?i ochii Tai deschi?i, ca sa auzi rugaciunea robului Tau, cu care ma rog eu acum ziua ?i noaptea înaintea Ta, pentru fiii lui Israel, robii Tai, marturisind pacatele fiilor lui Israel, cu care am pacatuit înaintea Ta ?i eu ?i casa tatalui meu.
Neh 1:7  Noi Te-am mâniat ?i n-am pazit poruncile, legile ?i orânduielile pe care le-ai dat Tu lui Moise, robul Tau.
Neh 1:8  Adu-?i aminte însa de cuvântul pe care l-ai spus robului Tau Moise, când ai zis: "De ve?i pacatui, va voi împra?tia printre popoare;
Neh 1:9  Iar când va ve?i întoarce la Mine ?i ve?i pazi poruncile Mele ?i le ve?i împlini, atunci, de a?i fi izgoni?i chiar ?i la marginea cerului, ?i de acolo va voi aduna ?i va voi aduce la locul pe care l-am ales, ca sa-Mi pun numele Meu acolo".
Neh 1:10  Ace?tia însa sunt robii Tai ?i poporul Tau, pe care Tu l-ai rascumparat cu puterea Ta ?i cu mâna Ta cea puternica.
Neh 1:11  Rogu-Te dar, o, Doamne, sa fie urechile Tale cu luare-aminte la rugaciunea robului Tau ?i la rugaciunea robilor Tai, carora le place sa se teama de numele Tau, ?i ajuta robului Tau acum ?i-i da sa dobândeasca mila la omul acesta. Caci eu eram paharnicul regelui".
Neh 2:1  "În luna Nisan, în anul al douazecilea al regelui Artaxerxe, având vinul în seama mea, am luat vin ?i am dat regelui ?i niciodata, mi se pare, nu m-am aratat trist înaintea lui.
Neh 2:2  Dar regele mi-a zis: "De ce este trista fa?a ta? De bolnav, nu e?ti bolnav! Se vede ca ai vre-o întristare la inima!"
Neh 2:3  ?i m-am speriat stra?nic ?i am raspuns regelui: "în veci sa traiasca regele! Cum sa nu fie trista fa?a mea, când cetatea, casa mormintelor parin?ilor mei, este pustiita ?i por?ile ei arse cu foc!"
Neh 2:4  "?i ce dore?ti tu?" - întreba regele. Eu însa, dupa ce m-am rugat Dumnezeului ceresc, am zis catre rege:
Neh 2:5  "De binevoie?te regele ?i de are robul tau trecere la tine, atunci trimite-ma în Iuda, la cetatea unde sunt mormintele parin?ilor mei, ca sa o zidesc".
Neh 2:6  Iar regele ?i regina, care ?edea lânga el, mi-au zis: "Cât timp are sa dureze calatoria ta ?i când ai sa te întorci?" ?i dupa ce am aratat cât timp am sa lipsesc, regele a binevoit sa ma trimita în Iuda.
Neh 2:7  La plecare însa am zis catre rege: "De binevoie?te regele, sa-mi dea scrisori catre guvernatorii de peste fluviu, ca sa-mi dea drumul sa pot ajunge în Iuda,
Neh 2:8  Precum ?i o scrisoare catre Asaf, pazitorul padurilor regelui, ca sa-mi dea lemn pentru por?ile ceta?ii, cele dinspre templul Domnului, ?i pentru zidul ceta?ii ?i pentru casa mea de locuit". ?i mi-a dat regele scrisori, pentru ca mâna binefacatoare a Dumnezeului meu era peste mine.
Neh 2:9  ?i am mers la guvernatorii de peste fluviu ?i le-am dat scrisorile regelui. Regele însa trimisese cu mine capetenii militare cu calare?i.
Neh 2:10  Dar când au auzit de aceasta Sanbalat Horonitul ?i Tobie, robul amonit, le-a parut foarte rau ca a. venit un om sa se îngrijeasca de binele fiilor lui Israel.
Neh 2:11  Iar dupa ce am ajuns la Ierusalim ?i am stat acolo trei zile,
Neh 2:12  M-am sculat noaptea cu pu?ini oameni ai mei sa cercetez cetatea. Nu spusesem însa nimanui ce-mi daduse Dumnezeu în gând sa fac pentru Ierusalim ?i nu se afla acolo la mine nici o vita, decât numai aceea pe care încalecasem eu.
Neh 2:13  ?i am ie?it atunci noaptea pe Poarta Vaii ?i m-am îndreptat spre izvorul Dragonului ?i spre Poarta Gunoiului ?i am cercetat zidurile Ierusalimului cele stricate ?i por?ile lui cele arse.
Neh 2:14  ?i m-am apropiat de poarta Izvorului ?i de iazul Regelui, dar nu era ioc sa treaca dobitocul care ma purta.
Neh 2:15  ?i de aceea m-am suit înapoi pe vale ?i am cercetat din nou zidurile ?i, intrând tot pe Poarta Vaii, m-am înapoiat.
Neh 2:16  Capeteniile insa nu ?tiau unde fusesem eu ?i ce facusem. Pâna atunci eu nu spusesem nimic nici Iudeilor, nici preo?ilor, nici celor mai de seama, nici capeteniilor, nici celorlal?i care se îndeletniceau cu felurite lucrari.
Neh 2:17  Atunci însa le-am zis: "Voi vede?i în ce stare rea ne aflam: Ierusalimul este darâmat ?i por?ile lui mistuite de foc. Hai sa zidim zidurile Ierusalimului ?i nu vom mai fi de batjocura!"
Neh 2:18  ?i le-am istorisit cum mâna binefacatoare a lui Dumnezeu fusese peste mine ?i cum mi-a vorbit regele. Atunci ei au zis: "Hai sa zidim!" ?i s-au întarit ei în aceasta hotarâre buna.
Neh 2:19  Dar auzind despre aceasta, Sanbalat Horonitul, Tobie, robul amonit ?i Ghe?em Arabul au râs de noi ?i cu dispre? ne-au zis: "Ce v-a?i apucat sa face?i voi aici? Nu cumva gândi?i sa va razvrati?i împotriva regelui?"
Neh 2:20  "Dumnezeul cel ceresc ne va ajuta, le raspunsei eu. Noi, slujitorii Lui, ne vom apuca de zidit, iar voi n-ave?i nici parte, nici drept, nici pomenire în Ierusalim".
Neh 3:1  "Atunci s-a ridicat Elia?ib, preotul cel mare, cu fra?ii sai preo?i ?i au zidit Poarta Oilor ?i, punându-i canaturile, au sfin?it-o; tot ei au reparat ?i zidul de la turnul Mea pâna la turnul Hananeel ?i l-au sfin?it.
Neh 3:2  Lânga Elia?ib au zidit Ierihonenii, iar lânga ace?tia a zidit Zacur, fiul lui Imri.
Neh 3:3  Fiii lui Senaa au zidit Poarta Pe?tilor ?i au acoperit-o, punându-i canaturile cu încuietorile ?i zavoarele trebuitoare.
Neh 3:4  De la ace?tia înainte a reparat Meremot, fiul lui Urie, fiul lui Haco?; lânga acesta a reparat Me?ulam, fiul lui Berechia, fiul lui Me?ezabeel; de la ace?tia înainte a reparat ?adoc, fiul lui Baana.
Neh 3:5  Alaturi de ace?tia au reparat cei din Tecoa, ai caror frunta?i de altfel nu ?i-au plecat grumazul sa lucreze pentru Domnul lor.
Neh 3:6  Ioiada, fiul lui Paseah ?i Me?ulam, fiul lui Besodia, au reparat Poarta Veche ?i, acoperind-o, i-au pus canaturile cu încuietorile ?i zavoarele trebuitoare.
Neh 3:7  De la ei înainte au lucrat Melatia Ghibeonitul ?i Iadon din Meronot, precum ?i oamenii din Ghibeon ?i din Mi?pa, supu?i stapânirii guvernatorilor de dincolo de fluviu.
Neh 3:8  Alaturi de ace?tia a reparat Uziel, fiul lui Harhaia Argintarul, iar de la acesta înainte a reparat Hanania, fiul lui Rocheim. ?i întarira Ierusalimul pâna la zidul cel lat.
Neh 3:9  De la ei înainte a reparat Refaia, fiul lui Hur, capetenia unei jumata?i din ?inutul Ierusalimului.
Neh 3:10  Lânga acesta a reparat, în fa?a casei lui, Iedaia, fiul lui Harumaf, iar lânga el a reparat Hatu?, fiul lui Ha?abneia.
Neh 3:11  O alta parte de zid a fost reparata de Malchia, fiul lui Harim ?i de Ha?ub, fiul lui Pahat-Moab; tot ei au reparat ?i turnul Cuptoarelor.
Neh 3:12  De la ei înainte au lucrat, cu fiicele sale, ?alum, fiul lui Halohe?, capetenia celeilalte jumata?i a ?inutului Ierusalimului.
Neh 3:13  Hanun ?i locuitorii din Zanoah au reparat Poarta Vaii. Ei au zidit-o ?i i-au pus canaturile cu încuietorile ?i zavoarele. Tot ei au facut mai bine de o mie de co?i de zid, pâna la Poarta Gunoiului.
Neh 3:14  Malchia, fiul lui Recab, capetenia ?inutului Bet-Hacherem, a reparat Poarta Gunoiului; i-a pus canaturile, cu încuietorile ?i zavoarele necesare.
Neh 3:15  ?alum, fiul lui Col-Hoze, capetenia ?inutului Mi?pa, a reparat Poarta Izvorului, a zidit-o ?i i-a pus canaturile cu încuietorile ?i zavoarele trebuitoare. El a mai facut zidul de la scaldatoarea Siloam, de lânga gradina regelui, pâna la scara ce coboara din cetatea lui David.
Neh 3:16  De la el înainte a reparat Neemia, fiul lui Azbuc, capetenia unei jumata?i din ?inutul Bet-?ur, pâna în fa?a mormintelor lui David ?i pâna la iazul cel sapat ?i pâna la Casa Vitejilor.
Neh 3:17  De la el înainte au reparat levi?ii: Rehum, fiul lui Bani; lânga acesta a reparat Ha?abia, capetenia unei jumata?i din ?inutul Cheila, din partea acestui ?inut.
Neh 3:18  Mai departe au reparat fra?ii lui: Bavai, fiul lui Henanad, capetenia celeilalte jumata?i din ?inutul Cheila,
Neh 3:19  ?i lânga el Ezer, fiul lui Iosua, capetenia din Mi?pa, a reparat alta parte de zid din fa?a scarii armelor pâna la col?.
Neh 3:20  De la el înainte Baruc, fiul lui Zabai, a reparat cu multa sârguin?a o alta parte de zid, de la col? pâna la poarta casei lui Elia?ib, preotul cel mare.
Neh 3:21  Dupa el a reparat o alta parte de zid Meremot, fiul lui Urie, fiul lui Haco?, de la poarta casei lui Elia?ib pâna la capatul casei lui Elia?ib.
Neh 3:22  Mai departe au reparat preo?ii din împrejurimile Ierusalimului.
Neh 3:23  Iar lânga ei, Veniamin ?i Ha?ub au reparat zidul în dreptul casei lor. Lânga ei Azaria, fiul lui Maaseia, fiul lui Anania, a reparat zidul lânga casa sa.
Neh 3:24  Lânga acesta, Binui, fiul lui Henadad a reparat alta bucata de zid, de la casa lui Azaria pâna în dreptul col?ului.
Neh 3:25  Palal, fiul lui Uzai, a reparat în dreptul unghiului ?i al turnului, care se ridica deasupra casei de sus a regelui cea din apropierea cur?ii închisorii. Lânga acesta a reparat Pedaia, fiul lui Paro?.
Neh 3:26  Iar cei încredin?a?i templului, care erau pe deal, au reparat zidul pâna în dreptul Por?ii Apelor, spre rasarit ?i pâna la turnul cel înalt.
Neh 3:27  De la ei înainte cei din Tecoa au reparat o parte, în fa?a turnului celui mare ?i înalt pâna la zidul lui Ofel.
Neh 3:28  De la Poarta Cailor în sus au reparat preo?ii, fiecare în dreptul casei sale.
Neh 3:29  De la ei înainte a reparat ?adoc, fiul lui Imer, dinaintea casei sale, iar dupa el a reparat ?emaia, fiul lui ?ecania, strajerul Por?ii Rasaritului.
Neh 3:30  Dupa el Hanania, fiul lui ?elemia ?i Hanun, al ?aselea fiu al lui ?alaf, au reparat alta parte de zid. Dupa ei Me?ulam, fiul lui Berechia, a reparat în dreptul locuin?ei lui.
Neh 3:31  Apoi Malchia faurarul a reparat pâna la casa celor încredin?a?i templului ?i a negustorilor, în dreptul por?ii lui Mifcad ?i pâna la foi?orul din col?.
Neh 3:32  Iar argintarii ?i negustorii au dres zidul de la foi?orul din col? ?i pâna la Poarta Oilor".
Neh 4:1  "Când a auzit Sanbalat, ca noi refacem zidurile Ierusalimului, s-a mâniat ?i s-a tulburat tare ?i a batjocorit pe Iudei;
Neh 4:2  ?i de fa?a cu fra?ii sai ?i de fa?a cu osta?ii samarineni a zis: "Ce fac ace?ti Iudei neputincio?i? Se poate oare sa li se îngaduie aceasta? Se poate oare sa aduca jertfe? Se poate oare sa li se îngaduie aceasta? Se poate oare sa dezgroape pietrele din movilele de moloz ?i înca arse?"
Neh 4:3  Iar Tobie Amonitul de lânga el a zis: "Lasa-i sa zideasca, fiindca are sa vina o vulpe ?i are sa le strice zidul lor cel de piatra!"
Neh 4:4  Auzi, Dumnezeul nostru, în ce dispre? suntem noi ?i întoarce ocara lor asupra copiilor lor ?i da-i dispre?ului în ?ara robiei.
Neh 4:5  ?i nelegiuirile lor sa nu le acoperi, nici pacatul lor sa nu se ?tearga înaintea fe?ei Tale, pentru ca au amarât pe cei ce zideau!
Neh 4:6  ?i noi totu?i am lucrat înainte la zid ?i l-am încheiat peste tot pâna la jumatate; ?i poporului nu-i lipsea râvna de a lucra.
Neh 4:7  Dar Sanbalat, Tobie, Arabii, Amoni?ii ?i cei din A?dod, auzind ca se înal?a zidurile Ierusalimului, ca sparturile lui au început sa se astupe, s-au mâniat foarte tare;
Neh 4:8  ?i s-au sfatuit împreuna sa vina cu razboi ?i sa darâme Ierusalimul.
Neh 4:9  Atunci noi ne-am rugat Dumnezeului nostru ?i am pus straja, ca sa ne pazeasca de ei ziua ?i noaptea.
Neh 4:10  Iudeii însa au zis: "Puterile salahorilor au slabit ?i moloz e foarte mult; noi nu mai suntem în stare sa facem zidul".
Neh 4:11  Iar du?manii no?tri graiau: "Nici n-au sa afle, nici n-au sa vada nimic pâna când vom ajunge în mijlocul lor; îi vom ucide ?i vom face sa înceteze lucrul".
Neh 4:12  ?i dupa ce au venit Iudeii care locuiau lânga ei ?i ne-au spus de vreo zece ori, din toate par?ile, ca du?manii no?tri au sa navaleasca asupra noastra,
Neh 4:13  Atunci în par?ile de jos ale ceta?ii, dupa ziduri, am a?ezat poporul dupa semin?ii, cu sabii ?i cu lanci ?i cu arcuri.
Neh 4:14  Apoi am cercetat toate ?i, sculându-ma, am zis celor mai de seama ?i capeteniilor ?i celuilalt popor: "Nu va teme?i de ei! Aduce?i-va aminte de Domnul cel mare ?i înfrico?ator ?i lupta?i-va pentru fra?ii vo?tri, pentru fiii vo?tri, pentru fiicele voastre, pentru femeile voastre ?i pentru casele voastre!"
Neh 4:15  Când au auzit du?manii no?tri ca ne este cunoscut gândul lor, atunci Dumnezeu a risipit planul lor; iar noi ne-am întors cu to?ii la ziduri, apucându-ne fiecare de lucrul nostru.
Neh 4:16  Din acea zi jumatate din tinerii mei se îndeletniceau cu lucrul, iar jumatate din ei stateau gata de aparare cu lanci, cu scuturi, cu arcuri ?i cu plato?e, iar îndaratul lor se aflau capeteniile a toata casa lui Iuda.
Neh 4:17  Cei ce lucrau la zid ?i cei ce carau, cu o mâna lucrau, iar cu cealalta ?ineau lancea.
Neh 4:18  Fiecare din cei ce zideau erau încin?i peste coapsele lor cu sabia ?i a?a lucrau; iar lânga mine se afla cel ce suna din trâmbi?a.
Neh 4:19  ?i eu am zis celor mai însemna?i ?i capeteniilor ?i poporului de rând: "Lucrul este mult ?i greu ?i noi suntem risipi?i pe ziduri ?i departa?i unii de al?ii.
Neh 4:20  De aceea sa alerga?i la mine acolo unde ve?i auzi sunetul trâmbi?ei ?i Dumnezeul nostru se va lupta pentru noi".
Neh 4:21  ?i a?a am lucrat noi, iar jumatate statea cu lancea în mâna de la rasaritul soarelui pâna seara la ie?irea stelelor.
Neh 4:22  Afara de aceasta, tot atunci am mai zis poporului sa ramâna to?i în Ierusalim cu robii lor, strajuind în timpul nop?ii, iar în timpul zilei lucrând.
Neh 4:23  ?i nici eu, nici fra?ii mei, nici slugile mele, nici strajerii care ma înso?isera, nu s-au dezbracat de haine ?i fiecare pâna ?i la apa mergea cu sabia în mâna".
Neh 5:1  Atunci s-a facut murmur mare în popor ?i între femei împotriva fra?ilor lor Iudei.
Neh 5:2  Erau unii care ziceau: "Noi cu fiii ?i cu fiicele noastre suntem mul?i, sa ni se dea pâine sa ne hranim ?i sa traim".
Neh 5:3  Al?ii ziceau: "Ne-am pus zalog viile ?i ogoarele ?i casele noastre, ca sa ne luam pâine sa ne astâmparam foamea".
Neh 5:4  Iar al?ii ziceau: "Noi luam bani cu camata ca sa platim bir regelui, zalogindu-ne viile ?i ogoarele noastre.
Neh 5:5  ?i noi avem acela?i trup ca ?i fra?ii vo?tri ?i fiii vo?tri sunt tot a?a cum sunt ?i fiii lor; dar iata ca noi suntem nevoi?i sa ne dam fiii ?i fiicele robi, ba unele din fiicele noastre se ?i afla în robie. N-avem la îndemâna nici un mijloc de rascumparare. ?i viile noastre ?i ogoarele noastre sunt în stapânirea altora".
Neh 5:6  Când am auzit eu murmurul lor ?i astfel de vorbe m-am mâniat foarte tare.
Neh 5:7  Inima mea s-a tulburat ?i am dojenit aspru pe cei mai însemna?i ?i pe capetenii. "Voi lua?i mita de la fra?ii vo?tri", le-am zis eu. ?i am facut împotriva lor o adunare mare,
Neh 5:8  ?i le-am zis: "Noi, pe cât ne-au ajutat puterile, am rascumparat fra?i de ai no?tri, Iudei, care fusesera vându?i pagânilor; iar voi vinde?i pe fra?ii vo?tri". Ei însa taceau ?i nu gaseau raspuns.
Neh 5:9  "Nu face?i bine - am zis eu mai departe. Nu vi se cuvine oare sa umbla?i în frica lui Dumnezeu, ca sa înlatura?i ocara neamurilor vrajma?e noua?
Neh 5:10  ?i eu, fra?ii mei ?i slujitorii mei, le-am dat împrumut ?i argint ?i pâine, dar n-am mai cerut înapoi.
Neh 5:11  Întoarce?i-le dar chiar acum ogoarele, viile, maslinii ?i casele lor ?i dobânda argintului, pâinii, vinului ?i untdelemnului, pentru care le-a?i dat împrumut".
Neh 5:12  "Se vor întoarce ?i nu vom mai cere de la ei - zisera ei. Vom face a?a cum zici tu". Atunci am chemat preo?ii ?i le-am poruncit sa le ia juramânt ca vor face a?a.
Neh 5:13  Iar eu mi-am scuturat haina ?i am zis: "A?a sa scuture Dumnezeu ?i de casa ?i de avere pe fiecare om care nu-?i va ?ine cuvântul sau ?i a?a sa fie el scuturat ?i gol. "Amin" - zise toata adunarea. Atunci am preaslavit pe Dumnezeu ?i poporul ?i-a împlinit cuvântul.
Neh 5:14  ?i din ziua aceea în care eu am fost numit guvernator al lor în ?ara lui Iuda, din anul al douazecilea pâna în anul al treizeci ?i doilea al regelui Artaxerxe, timp de doisprezece ani eu ?i fra?ii mei n-am mâncat pâinea de conducator.
Neh 5:15  Iar conducatorii de mai înainte, apasau poporul ?i luau de la el pâine ?i vin, pe lânga cei patruzeci de sicli de argint; ba pâna ?i slugile lor apasau poporul. Eu însa n-am facut a?a, pentru ca ma temeam de Dumnezeu.
Neh 5:16  în vremea aceasta eu am mers înainte cu lucrul la zidurile acestea ?i nici o ?arina n-am cumparat ?i to?i slujba?ii mei se adunau acolo la lucru.
Neh 5:17  Eu aveam la masa mea pâna la o suta cincizeci de oameni, Iudei ?i capetenii, afara de cei ce veneau la noi de pe la popoarele dimprejur.
Neh 5:18  ?i iata ce se gatea în fiecare zi: un bou, ?ase oi îngra?ate ?i pasari; ?i la fiecare zece zile se consuma mul?ime de vin de tot felul. Cu toate acestea pâinea de conducator n-am cerut-o, deoarece poporul acesta avea de dus o munca grea.
Neh 5:19  Dumnezeu sa primeasca tot ce am facut eu spre binele poporului acestuia".
Neh 6:1  Canaturile por?ilor înca nu le pusesem, când a auzit Sanbalat, Tobie, Ghe?em Arabul ?i ceilal?i du?mani ai no?tri ca eu am reparat zidul ?i nu a mai ramas nici o spartura în el.
Neh 6:2  Atunci au trimis Sanbalat ?i Ghe?em la mine sa mi se spuna: "Vino sa ne întâlnim împreuna intr-unul din satele din Valea Ono!" Caci ei pusesera la cale sa-mi faca rau.
Neh 6:3  Dar eu am trimis la ei robi sa le spuna: "Am de facut un lucru mare ?i nu pot sa vin, caci lucrul ar înceta, daca l-a? lasa ?i a? veni la voi".
Neh 6:4  De patru ori au trimis ei la mine cu aceea?i poftire ?i eu le-am raspuns la fel.
Neh 6:5  Atunci a trimis la mine Sanbalat a cincea oara pe slujitorul sau care avea în mâna sa o scrisoare deschisa.
Neh 6:6  În ea se scria: "Umbla zvonul printre popoare ?i spune ?i Ghe?em, cum ca tu ?i Iudeii v-a?i gândit sa va rascula?i, pentru care scop ?i zide?ti zidul ?i vrei sa fii regele lor, dupa acelea?i zvonuri;
Neh 6:7  ?i ai fi pus ?i prooroci, ca sa te vesteasca în Ierusalim ?i sa-?i zica regele Iudei, ?i astfel de zvonuri vor ajunge pâna la rege. Vino deci sa ne sfatuim împreuna".
Neh 6:8  Eu însa am trimis sa i se spuna: "Nimic de felul celor ce spui tu n-a fost. Acestea le-ai nascocit tu din capul tau".
Neh 6:9  Caci to?i ne înfrico?eaza socotind ca ne vom lasa de lucru ?i zidul nu se va mai face. Dar eu înca ?i mai mult ma încurajam la lucru ?i ziceam: "Întare?te-ma ?i acum, Dumnezeule!"
Neh 6:10  ?i am mers eu în casa lui ?emaia, fiul lui Delaia, fiul lui Mehetabeel, ?i el, încuindu-se, a zis: "Sa ne adunam în templul lui Dumnezeu, în mijlocul lui, ?i sa încuiem u?ile, caci au sa vina sa te ucida, ?i au sa vina noaptea sa te ucida".
Neh 6:11  "Se poate oare sa fuga un astfel de om, ca mine? - am raspuns eu. Se poate oare ca unul ca mine sa intre în templu, pentru a ramâne cu via?a? Nu merg! "
Neh 6:12  ?i am cunoscut ca nu Dumnezeu îl trimisese, ci el graia ca un prooroc împotriva mea, pentru ca îl cumparasera Tobie ?i Sanbalat.
Neh 6:13  ?i fusese cumparat ca eu sa ma sperii ?i sa fac a?a ?i sa gre?esc, ca apoi sa aiba lumea parere urâta despre mine, ?i sa fiu prigonit pentru aceasta scadere.
Neh 6:14  Pomene?te, Dumnezeul meu, pe Tobie ?i pe Sanbalat dupa aceste fapte ale lor; de asemenea ?i pe prooroci?a Noadia ?i pe ceilal?i prooroci care au vrut sa ma sperie!
Neh 6:15  Zidul a fost ispravit în ziua a douazeci ?i cincea a lunii Elul, adica în cincizeci ?i doua de zile.
Neh 6:16  Când au auzit de aceasta to?i du?manii no?tri ?i au vazut aceasta toate popoarele cele dimprejurul nostru, s-au smerit foarte mult ?i au cunoscut ca acest lucru a fost facut de Dumnezeul nostru.
Neh 6:17  Pe lânga aceasta, în zilele acelea, cei mai de seama ai Iudei au scris multe scrisori, pe care le-au trimis la Tobie, iar ei primeau scrisori de-ale lui Tobie;
Neh 6:18  Caci mul?i din Iuda erau cu el în legatura întarita prin juramânt, caci el era ginerele lui ?ecania, fiul lui Arah, iar fiul sau Iohanan luase de nevasta pe fiica lui Me?ulam, fiul lui Berechia.
Neh 6:19  Ba îmi vorbeau ei ?i de bunatatea lui ?i vorbele mele ajungeau la el. ?i Tobie trimitea scrisori, ca sa ma sperie.
Neh 7:1  Dupa ce s-a terminat zidul ?i am pus canaturile ?i au fost a?eza?i la slujba lor portarii ?i cântare?ii ?i levi?ii,
Neh 7:2  Atunci am poruncit fratelui meu Hanani ?i lui Anania, capetenia ceta?ii Ierusalimului, caci el cu mult mai mult decât al?ii era om credincios ?i temator de Dumnezeu,
Neh 7:3  ?i le-am zis: "Sa nu se deschida por?ile Ierusalimului pâna nu rasare soarele, ?i sa le închida ?i sa le încuie în fa?a voastra". ?i am pus strajeri pe locuitorii Ierusalimului, fiecare când îi venea rândul ?i fiecare în fa?a casei sale.
Neh 7:4  Cetatea însa era întinsa ?i mare, iar popor era pu?in în ea ?i case nu se zidisera din nou.
Neh 7:5  ?i mi-a dat Dumnezeul meu gând sa aduc pe cei mai mari, pe capetenii ?i poporul, ca sa fac o numaratoare, dupa spi?ele neamului lor, ?i am gasit cartea cu spi?ele de neam ale celor ce venisera întâia oara, ?i într-însa erau scrise urmatoarele:
Neh 7:6  Iata locuitorii ?arii care au plecat din robie, unde-i dusese Nabucodonosor, regele Babilonului, ?i s-au întors la Ierusalim ?i în Iuda, a?ezându-se fiecare în cetatea sa,
Neh 7:7  Împreuna venind cu Zorobabel: Iosua, Neemia, Azaria, Raamia, Nahamani, Mardoheu, Bil?an, Misperet, Bigvai, Nehum ?i Baana. Numarul poporului lui Israel:
Neh 7:8  Fiii lui Fares, doua mii o suta ?aptezeci ?i doi;
Neh 7:9  Fiii lui ?efatia, trei sute ?aptezeci ?i doi;
Neh 7:10  Fiii lui Arah, ?ase sute cincizeci ?i doi;
Neh 7:11  Fiii lui Pahat-Moab, din urma?ii lui Iosua ?i Ioab, doua mii opt sute optsprezece;
Neh 7:12  Fiii lui Elam, o mie doua sute cincizeci ?i patru;
Neh 7:13  Fiii lui Zatu, opt sute patruzeci ?i cinci;
Neh 7:14  Fiii lui Zacai, ?apte sute ?aizeci;
Neh 7:15  Fiii lui Binui, ?ase sute patruzeci ?i opt;
Neh 7:16  Fiii lui Bebai, ?ase sute douazeci ?i opt;
Neh 7:17  Fiii lui Azgad, doua mii trei sute douazeci ?i doi;
Neh 7:18  Fiii lui Adonicam, ?ase sute ?aizeci ?i ?apte;
Neh 7:19  Fiii lui Bigvai, doua mii ?aizeci ?i ?apte;
Neh 7:20  Fiii lui Adin, ?ase sute cincizeci ?i cinci;
Neh 7:21  Fiii lui Ater, din familia lui Iezechia, nouazeci ?i opt;
Neh 7:22  Fiii lui Ha?um, trei sute douazeci ?i opt;
Neh 7:23  Fiii lui Be?ai, trei sute douazeci ?i patru;
Neh 7:24  Fiii lui Harif, o suta doisprezece;
Neh 7:25  Ghibeoni?ii, nouazeci ?i cinci;
Neh 7:26  Oameni din Betleem ?i Netofa, o suta optzeci ?i opt;
Neh 7:27  Oameni din Anatot, o suta douazeci ?i opt;
Neh 7:28  Oameni din Bet-Azmavet, patruzeci ?i doi;
Neh 7:29  Oameni din Chiriat-Iearim, din Chefira ?i din Beerot, ?apte sute patruzeci ?i trei;
Neh 7:30  Oameni din Rama ?i din Gheba, ?ase sute douazeci ?i unu;
Neh 7:31  Oameni din Micmas, o suta douazeci ?i doi;
Neh 7:32  Oameni din Betel ?i din Ai, o suta douazeci ?i trei;
Neh 7:33  Oameni din celalalt Nebo, cincizeci ?i doi;
Neh 7:34  Fiii celuilalt Elam, o mie doua sute cincizeci ?i patru;
Neh 7:35  Fiii lui Harim, trei sute douazeci;
Neh 7:36  Ierihoneni, trei sute patruzeci ?i cinci;
Neh 7:37  Lodieni, Hadidieni ?i oameni din Ono, ?apte sute douazeci ?i unu.
Neh 7:38  Fiii lui Senaa, trei mii noua sute treizeci.
Neh 7:39  Preo?ii: fiii lui Iedaia, din casa lui Iosif, noua sute ?aptezeci ?i trei;
Neh 7:40  Fiii lui Imer, o mie cincizeci ?i doi;
Neh 7:41  Fiii lui Pa?hur, o mie doua sute patruzeci ?i ?apte;
Neh 7:42  Fiii lui Harim, o mie ?aptesprezece;
Neh 7:43  Levi?ii: fiii lui Iosua ?i ai lui Cadmiel, din casa lui Hodavia, ?aptezeci ?i patru.
Neh 7:44  Cântare?ii: fiii lui Asaf, o suta patruzeci ?i opt.
Neh 7:45  Portarii: fiii lui ?alum, fiii lui Ater, fiii lui Talmon, fiii lui Acuv, fiii lui Hatita, fiii lui ?obai, o suta treizeci ?i opt.
Neh 7:46  Cei încredin?a?i templului: fiii lui ?iha, fiii lui Hasufa, fiii lui Tabaot,
Neh 7:47  Fiii lui Cheros, fiii lui Sia, fiii lui Padon,
Neh 7:48  Fiii lui Lebana, fiii lui Hagaba, fiii lui ?almai,
Neh 7:49  Fiii lui Hanan, fiii lui Ghidel, fiii lui Gahar,
Neh 7:50  Fiii lui Reaia, fiii lui Re?in, fiii lui Necoda,
Neh 7:51  Fiii lui Gazam, fiii lui Uza, fiii lui Paseah,
Neh 7:52  Fiii lui Besai, fiii lui Meunim, fiii lui Nefi?im,
Neh 7:53  Fiii lui Bacbuc, fiii lui Hacufa, fiii lui Harhur,
Neh 7:54  Fiii lui Ba?lit, fiii lui Mehida, fiii lui Har?a,
Neh 7:55  Fiii lui Barcos, fiii lui Sisera, fiii lui Tamah,
Neh 7:56  Fiii lui Ne?iah, fiii lui Hatifa.
Neh 7:57  Fiii robilor lui Solomon: fiii lui Sotai, fiii lui Soferet, fiii lui Perida;
Neh 7:58  Fiii lui Iaala, fiii lui Darcon, fiii lui Ghidel;
Neh 7:59  Fiii lui ?efatia, fiii lui Hatil, fiii lui Pocheret-Ha?ebaim, fiii lui Amon.
Neh 7:60  Netineii ?i fiii robilor lui Solomon în total erau trei sute nouazeci ?i doi.
Neh 7:61  Iata acum ?i cei ce au plecat din Tel-Melah, din Tel-Har?a, din Cherub-Adon ?i din Imer, ?i care n-au putut sa-?i arate nici casa din care se trag, nici semin?ia, pentru a dovedi ca sunt Israeli?i:
Neh 7:62  Fiii lui Delaia, fiii lui Tobie, fiii lui Necoda, ?ase sute patruzeci ?i doi.
Neh 7:63  ?i dintre preo?i: fiii lui Hobaia, fiii lui Haco?, fiii lui Barzilai, care a luat de so?ie una din fiicele lui Barzilai Galaaditul ?i a primit numele acesteia.
Neh 7:64  Ei ?i-au cautat spi?a neamului lor, dar nu ?i-au gasit-o, ?i au fost sco?i din preo?ie;
Neh 7:65  ?i guvernatorul le-a zis sa nu manânce din lucrurile de mare sfin?enie pâna nu se va ridica un preot cu Urim ?i Tumim.
Neh 7:66  Adunarea întreaga a fost de patruzeci ?i doua de mii trei sute ?aizeci de suflete,
Neh 7:67  Afara de robi ?i roabe, care erau în numar de ?apte mii trei sute treizeci ?i ?apte. Printre ei se mai aflau doua sute patruzeci ?i cinci de cântare?i ?i cântare?e.
Neh 7:68  Ei aveau ?apte sute treizeci ?i ?ase de cai, doua sute patruzeci ?i cinci de catâri,
Neh 7:69  Patru sute treizeci ?i cinci camile ?i ?ase mii ?apte sute douazeci asini.
Neh 7:70  Unii capi de familie au facut daruri pentru lucru. Tir?ata a dat la vistierie o mie de drahme de aur, cincizeci de cupe ?i cinci sute treizeci ve?minte preo?e?ti.
Neh 7:71  Capeteniile caselor au dat în vistieria lucrarilor douazeci de mii de drahme de aur ?i doua mii doua sute mine de argint. Celalalt popor a dat douazeci de mii de drahme de aur ?i doua mii de mine de argint ?i ?aizeci ?i ?apte de ve?minte preo?e?ti.
Neh 7:72  Preo?ii ?i levi?ii, portarii, cântare?ii ?i oamenii din popor, cei încredin?a?i templului ?i tot Israelul s-au a?ezat în ceta?ile lor.
Neh 7:73  ?i când a sosit luna a ?aptea, fiii lui Israel se aflau prin ceta?ile lor.
Neh 8:1  Atunci s-a adunat tot poporul ca un singur om în pia?a cea din fa?a Por?ii Apelor ?i a zis lui Ezdra carturarul sa aduca el cartea legii lui Moise pe care a dat-o Domnul lui Israel.
Neh 8:2  ?i carturarul Ezdra a adus legea înaintea adunarii, care era alcatuita din barba?i ?i din femei ?i din to?i acei ce erau în stare sa în?eleaga. Aceasta s-a facut în ziua întâi a lunii a ?aptea.
Neh 8:3  ?i a citit Ezdra din carte de diminea?a pâna la amiaza în pia?a cea de dinaintea Por?ii Apelor, înaintea barba?ilor ?i femeilor ?i a celor ce erau în stare sa priceapa ?i tot poporul a fost cu luare-aminte la citirea car?ii legii.
Neh 8:4  Ezdra carturarul statea sus pe un pode? de lemn, facut anume pentru acest prilej. Alaturi de el, la dreapta lui, stateau Matitia, ?ema, Anaia, Urie, Hilchia ?i Maaseia; iar la stânga lui se aflau Pedaia, Misael, Malchia, Ha?um, Ha?badana, Zaharia ?i Me?ulam.
Neh 8:5  ?i a deschis Ezdra cartea înaintea ochilor a tot poporul, caci el se afla mai sus decât tot poporul ?i când a deschis el cartea, tot poporul statea în picioare în pia?a.
Neh 8:6  Mai întâi Ezdra a binecuvântat pe Domnul, marele Dumnezeu ?i tot poporul, ridicându-?i mâinile, a raspuns: Amin, Amin! ?i plecându-se, s-a închinat înaintea Domnului pâna la pamânt.
Neh 8:7  Apoi Iosua, Bani, ?erebia, Iamin, Acub, ?abetai, Hodia, Maaseia, Chelita, Azaria, Iozabad, Hanan, Pelaia, ?i levi?ii au tâlcuit poporului legea, stând fiecare la locul lui.
Neh 8:8  Ei citeau lamurit buca?i din cartea legii lui Dumnezeu, iar buca?ile citite le lamureau ?i poporul în?elegea cele ce se citeau.
Neh 8:9  Atunci Neemia, guvernatorul, Ezdra preotul ?i carturarul, ?i levi?ii care înva?au poporul au zis catre tot poporul: "Ziua aceasta este închinata Domnului Dumnezeului vostru, sa nu fi?i tri?ti, nici sa plânge?i! Caci tot poporul plângea, auzind cuvintele legii.
Neh 8:10  Ci merge?i de mânca?i carne grasa ?i be?i vin dulce - adaugara ei - ?i trimite?i parte ?i celor ce n-au nimic gatit, caci ziua aceasta este sfin?ita Domnului nostru. Nu fi?i tri?ti, caci bucuria Domnului va fi puterea voastra".
Neh 8:11  ?i au lini?tit levi?ii tot poporul, zicând: "Tace?i, caci ziua aceasta este sfânta. Nu fi?i tri?ti!"
Neh 8:12  ?i tot poporul s-a dus sa manânce ?i sa bea ?i sa trimita par?i celor ce nu aveau ?i sa faca veselie mare; caci în?elesesera ei cuvintele ce li se tâlcuisera.
Neh 8:13  A doua zi, capii familiilor din tot poporul, preo?ii ?i levi?ii se strânsesera împrejurul lui Ezdra carturarul, ca sa asculte talmacirea cuvintelor legii.
Neh 8:14  ?i au gasit ei scris în legea pe care Domnul a dat-o lui Moise, ca fiii lui Israel în timpul sarbatorilor din luna a ?aptea trebuie sa locuiasca în corturi.
Neh 8:15  ?i au vestit aceasta în toate ceta?ile lor ?i în Ierusalim, zicând: "Duce?i-va de cauta?i, în munte, crengi de maslin salbatic, de mirt, de finici ?i de tufari ?i face?i corturi cum este scris".
Neh 8:16  Atunci poporul s-a dus sa caute crengi ?i a facut fiecare cort pe acoperi?ul casei lor, în curtea sa, în curtea templului Domnului ?i în pia?a de dinaintea Por?ii Apelor ?i în pia?a de dinaintea Por?ii lui Efraim.
Neh 8:17  Toata ob?tea celor ce se întorsesera din robie ?i-a facut corturi ?i a stat în ele. Din zilele lui Iosua, fiul lui Navi, pâna în ziua aceasta, fiii lui Israel nu mai fusesera a?a. ?i a fost bucurie foarte mare.
Neh 8:18  ?i s-a citit din cartea legii Domnului în fiecare zi, din cea dintâi zi pâna la cea din urma. ?i au ?inut sarbatoarea ?apte zile iar a opta zi au facut o adunare sarbatoreasca, dupa rânduiala.
Neh 9:1  În ziua de douazeci ?i patru a acestei luni s-au adunat to?i fiii lui Israel, îmbraca?i cu sac ?i cu capetele presarate cu cenu?a, ca sa posteasca.
Neh 9:2  ?i osebindu-se cei ce erau din neamul lui Israel de to?i cei de alt neam, au venit de ?i-au marturisit pacatele lor ?i faradelegile parin?ilor lor.
Neh 9:3  ?i dupa ce s-au a?ezat la locurile lor, li s-a citit din cartea legii Domnului Dumnezeului lor un sfert de zi, iar alt sfert de zi ?i-au marturisit pacatele lor ?i s-au închinat Domnului Dumnezeului lor.
Neh 9:4  Apoi Iosua, Binui, Cadmiel, ?ebania, Buni, ?erebia, Baani ?i Chenani s-au urcat pe pode?ul levi?ilor ?i au strigat cu glas mare catre Domnul Dumnezeul lor;
Neh 9:5  ?i levi?ii Iosua, Cadmiel, Bani, Ha?abneia, ?erebia, Hodia, ?ebania ?i Petahia au zis: "Scula?i-va ?i binecuvânta?i pe Domnul Dumnezeul vostru din veac în veac!" "Dumnezeule - a zis Ezdra - slaveasca-se numele Tau cel slavit ?i mai presus de urice lauda ?i slavire!
Neh 9:6  Numai Tu e?ti Domn ?i numai Tu ai facut cerurile, cerurile cerurilor ?i toata o?tirea lor, pamântul ?i toate cele de pe el, marile ?i toate cele ce se cuprind în ele; Tu dai via?a la toate ?i ?ie se închina o?tirea cerurilor.
Neh 9:7  Tu, Doamne Dumnezeule ai ales pe Avram, l-ai scos din Urul Caldeii ?i i-ai dat numele de Avraam.
Neh 9:8  Tu ai gasit ca inima lui e credincioasa înaintea Ta, Tu ai facut legamânt cu el ?i Tu ai fagaduit sa dai urma?ilor lui ?ara Canaanului, a Heteilor, a Amoreilor, a Ferezeilor, a Iebuseilor ?i a Ghergheseilor; ?i Tu ?i-ai ?inut cuvântul, pentru ca Tu e?ti drept.
Neh 9:9  Tu ai vazut necazul parin?ilor no?tri în Egipt ?i ai auzit strigatele lor la Marea Ro?ie.
Neh 9:10  Tu ai facut semne ?i minuni înaintea lui Faraon, împotriva tuturor slugilor lui ?i împotriva întregului popor din ?ara lui, pentru ca Tu ai vazut cu câta rautate s-au purtat ei cu parin?ii no?tri ?i ?i-ai facut nume pâna în ziua de astazi.
Neh 9:11  Tu ai despar?it marea înaintea parin?ilor no?tri ?i au trecut prin mijlocul marii ca pe uscat, dar i-ai cufundat în adânc, cum se cufunda o piatra în apa, pe cei ce-i urmareau.
Neh 9:12  Tu i-ai pova?uit ziua cu stâlp de nor ?i noaptea cu stâlp de foc ?i le-ai luminat calea pe care aveau sa mearga.
Neh 9:13  Tu Te-ai pogorât pe muntele Sinai, le-ai grait din înal?imea cerurilor ?i le-ai dat porunci drepte; legi ale adevarului, înva?aturi ?i orânduiri minunate.
Neh 9:14  Tu le-ai aratat odihna Ta cea sfânta ?i le-ai scris prin Moise, sluga Ta, porunci, rânduieli ?i lege.
Neh 9:15  Tu le-ai dat din înal?imea cerului pâine, când au flamânzit ?i le-ai scos apa din piatra, când au însetat, ?i le-ai zis sa intre ?i sa mo?teneasca ?ara pe care cu juramânt le-ai fagaduit-o.
Neh 9:16  Dar parin?ii no?tri s-au îndaratnicit ?i ?i-au învârto?at cerbicia lor; n-au ascultat poruncile Tale, nici s-au supus ?i au uitat minunile pe care le-ai facut pentru ei.
Neh 9:17  Învârto?atu-?i-au cerbicia lor ?i în razvratirea lor ?i-au ales o capetenie, ca sa se întoarca în robia lor; dar Tu, fiind Dumnezeu iubitor ?i iertator, negrabnic la mânie ?i bogat în mila ?i îndurare, nu i-ai parasit.
Neh 9:18  Chiar când ?i-au facut un vi?el turnat ?i au zis: "Iata dumnezeul tau care te-a scos din Egipt!" ?i s-au dedat la hule mari împotriva Ta,
Neh 9:19  În nemarginita Ta îndurare, nu i-ai parasit în pustiu ?i stâlpul de nor n-a încetat sa-i calauzeasca ziua în calea lor, nici stâlpul de foc sa le lumineze noaptea drumul ce faceau.
Neh 9:20  Trimisu-le-ai Duhul Tau cel bun, ca sa-i în?elep?easca; nu ai lipsit gura lor de mana Ta ?i în setea lor le-ai dat apa.
Neh 9:21  Timp de patruzeci de ani i-ai hranit în pustie ?i nimic nu le-a lipsit; hainele lor nu s-au învechit, nici încal?amintele lor nu s-au rupt.
Neh 9:22  Le-ai dat regate ?i popoare le-ai împar?it ?i au pus stapânire pe ?ara lui Sihon, regele He?bonului ?i pe pamântul lui Og, regele Vasanului.
Neh 9:23  Tu ai înmul?it pe fiii lui ca stelele cerului ?i i-ai dus în ?ara de care ai spus parin?ilor lor ca o vor mo?teni.
Neh 9:24  ?i fiii lor au intrat ?i au mo?tenit ?ara aceea; ai supus înaintea lor pe locuitorii pamântului aceluia, pe Canaanei ?i i-ai dat pe ace?tia în mâna lor cu regele ?i cu tot poporul ba?tina?, ca sa le faca ce vor vrea.
Neh 9:25  ?i s-au facut ei stapâni peste ceta?i tari ?i peste pamântul roditor, peste case pline de toate bunata?ile, fântâni sapate în piatra, vii, maslini ?i pomi roditori din bel?ug; ?i au mâncat ?i s-au saturat ?i s-au îngra?at traind în desfatari prin bunatatea Ta.
Neh 9:26  Dar ei s-au ridicat ?i s-au razvratit împotriva Ta; au aruncat legea Ta la spate; pe proorocii Tai care-i îndemnau sa se întoarca la Tine i-au ucis; ?i ?i-au adus hule mari.
Neh 9:27  Atunci Tu i-ai dat în mâinile vrajma?ilor lor, care i-au apasat. Dar în vremea necazului lor au strigat catre Tine ?i Tu i-ai auzit din înal?imea cerurilor ?i, în mila Ta cea mare, le-ai trimis izbavitori ca sa-i izbaveasca din mâinile vrajma?ilor lor.
Neh 9:28  Iar daca s-au odihnit, au început iar sa faca rau înaintea Ta. Atunci Tu i-ai dat din nou în mâna vrajma?ilor lor ca sa-i stapâneasca. ?i ei din nou au strigat catre Tine ?i Tu i-ai auzit din înal?imea cerurilor ?i, în mila Ta cea mare, i-ai izbavit de multe ori.
Neh 9:29  I-ai pova?uit sa se întoarca la legea Ta, dar ei s-au îndaratnicit ?i n-au ascultat poruncile Tale, ci au pacatuit împotriva poruncilor care dau via?a celui ce le împline?te ?i ?i-au îndârjit spinarea lor ?i cerbicia lor ?i-au învârto?at-o ?i nu s-au supus.
Neh 9:30  Tu însa, a?teptând întoarcerea lor, i-ai îngaduit mul?i ani ?i le-ai de?teptat luarea-aminte prin Duhul Tau ?i prin proorocii Tai, dar ei nu ?i-au plecat urechea. Atunci i-ai dat în mâna popoarelor straine.
Neh 9:31  Dar, în mila Ta cea mare, nu i-ai stârpit, nici nu i-ai parasit, caci Tu e?ti un Dumnezeu milos ?i îndurat.
Neh 9:32  ?i acum, Dumnezeul nostru, Dumnezeul cel mare, puternic ?i înfrico?ator, Cel ce ?ii legamântul Tau ?i faci mila, nu soco?i ca pu?in lucru suferin?ele ce am îndurat noi, regii no?tri, capeteniile noastre, preo?ii no?tri, proorocii no?tri, parin?ii no?tri ?i tot poporul Tau din vremea regilor Asiriei pâna în ziua aceasta.
Neh 9:33  Tu e?ti drept în toate câte ne-au ajuns, caci Tu ai fost credincios, iar noi am facut rau.
Neh 9:34  Regii no?tri, capeteniile noastre, preo?ii no?tri ?i parin?ii no?tri n-au pazit legea Ta ?i n-au luat aminte nici la poruncile Tale, nici la îndemnurile ce le-ai dat.
Neh 9:35  Cât au fost în regatul lor în mijlocul binefacerilor Tale, într-o ?ara larga ?i roditoare pe care le-ai dat-o Tu, ei nu ?i-au slujit ?ie, nici nu s-au întors de la faptele lor cele urâte.
Neh 9:36  ?i astazi iata suntem robi ?i ?ara ce Tu ai dat-o parin?ilor no?tri, ca sa se bucure de roadele ei ?i de bunata?ile ei,
Neh 9:37  Ea î?i înmul?e?te astazi roadele pentru regii carora ne-ai supus pentru pacatele noastre. Ei domnesc peste noi ?i peste vitele noastre, dupa bunul lor plac ?i noi ne aflam în necaz mare.
Neh 9:38  Pentru toate acestea facem legamânt pe care ?i noi în?ine cu iscalitura, ?i capeteniile, levi?ii ?i preo?ii no?tri îl întaresc cu pecetea".
Neh 10:1  Iata acum ?i cei care ?i-au pus pecetea lor: Neemia guvernatorul, fiul lui Hacalia;
Neh 10:2  Preo?ii: Sedechia, Seraia, Azaria, Ieremia,
Neh 10:3  Pa?hur, Amaria, Malchia,
Neh 10:4  Hatu?, ?ebania, Maluc,
Neh 10:5  Harim, Meremot, Obadia,
Neh 10:6  Daniel, Ghineton, Baruc,
Neh 10:7  Me?ulam, Abia, Miamin,
Neh 10:8  Maazia, Bilgai, ?emaia.
Neh 10:9  Levi?ii: Iosua, fiul lui Azania, Binui, din fiii lui Henadad, Cadmiel;
Neh 10:10  ?i fra?ii lor: ?ebania, Hodia, Chelita, Pelaia, Hanan,
Neh 10:11  Mica, Rehob, Ha?abia,
Neh 10:12  Zacur, ?erebia, ?ebania,
Neh 10:13  Hodia, Bani, Beninu.
Neh 10:14  Capeteniile poporului: Paro?, Pahat-Moab, Elam, Zatu, Bani,
Neh 10:15  Buni, Azgad, Bebai,
Neh 10:16  Adonia, Bigvai, Adin,
Neh 10:17  Ater, Iezechia, Azur,
Neh 10:18  Hodia, Ha?um, Be?ai,
Neh 10:19  Harif, Anatot, Nebai,
Neh 10:20  Magpia?, Me?ulam, Hezir,
Neh 10:21  Me?ezabeel, ?adoc, Iadua,
Neh 10:22  Pelatia, Hanan, Anaia,
Neh 10:23  Hozea, Hanania, Ha?ub,
Neh 10:24  Halohe?, Pileha, ?obec,
Neh 10:25  Rehum, Ha?abna, Maaseia,
Neh 10:26  Ahia, Hanan, Anan,
Neh 10:27  Maluc, Harim, Baana.
Neh 10:28  Celalalt popor: preo?i, levi?i, portari, cântare?i, cei încredin?a?i templului ?i to?i cei ce s-au despar?it de popoarele straine, ca sa urmeze legea lui Dumnezeu, femeile lor, baie?ii lor ?i fetele lor, to?i cei ce erau în stare sa priceapa ?i sa în?eleaga
Neh 10:29  S-au unit cu fra?ii ?i cu capeteniile lor ?i au fagaduit cu blestem ?i juramânt ?i au jurat sa se poarte dupa legea lui Dumnezeu, data prin Moise, sluga lui Dumnezeu, sa pazeasca ?i sa împlineasca toate poruncile Domnului, Stapânul nostru, hotarârile ?i legile Lui.
Neh 10:30  Am fagaduit sa nu dam fetele noastre popoarelor ?arii, nici sa luam fetele lor pentru feciorii no?tri;
Neh 10:31  Sa nu cumparam nimic în ziua de odihna, în zilele de sarbatoare de la popoarele din ?ara, când ne vor aduce în acele zile de vânzare marfuri sau orice bucate; iar în anul al ?aptelea sa iertam toate datoriile.
Neh 10:32  Am luat de asemenea îndatorirea sa dam a treia parte de siclu, pe an, pentru slujbele templului Dumnezeului nostru.
Neh 10:33  Pentru pâinile punerii înainte, pentru jertfa cea necontenita, pentru arderile de tot necontenite, care se fac în zile de odihna, la luna noua ?i la sarbatori, pentru lucrurile sfinte, pentru jertfa de iertare, spre cura?irea lui Israel, ?i pentru tot ce se savâr?e?te în templul Dumnezeului nostru.
Neh 10:34  Apoi au tras la sorti preo?ii, levi?ii ?i poporul, dupa casele noastre parinte?ti, ca sa se ?tie cine ?i când trebuie sa aduca în fiecare an, la anumita vreme, lemne în templul Dumnezeului nostru, spre a se arde pe altarul Domnului Dumnezeului nostru, dupa cum este scris în lege.
Neh 10:35  Am mai hotarât sa mai ducem în fiecare an la templul Domnului pârga din roadele pamântului nostru ?i pârga din roadele tuturor pomilor;
Neh 10:36  Sa aducem la templul Dumnezeului nostru, preo?ilor care fac slujbele în templul Dumnezeului nostru, pe întâiul nascut dintre fiii no?tri ?i dintre dobitoacele noastre, cum este scris în lege, ?i pe întâiul-nascut al vitelor noastre mari ?i marunte;
Neh 10:37  Sa aducem la preo?i, în vistieria templului Dumnezeului nostru, pârga din aluatul nostru ?i ofrandele noastre de fructe din tot pomul, din vin ?i untdelemn, ?i sa dam dijma pentru pamântul nostru levi?ilor, care au dreptul sa o ia în toate ceta?ile de pe pamânturile lucrate de noi.
Neh 10:38  Un preot, fiu al lui Aaron, va fi cu levi?ii, când ace?tia vor lua zeciuiala; ?i din zeciuiala luata, levi?ii vor duce zeciuiala în templul Dumnezeului nostru, în odaile vistieriei.
Neh 10:39  Caci în aceste odai atât fiii lui Israel, cât ?i fiii levi?ilor trebuie sa aduca prinoase de pâine, de vin ?i de untdelemn. Acolo sunt casele sfinte ?i stau preo?ii care slujesc ?i portarii ?i cântare?ii. Astfel noi nu vom lasa în parasire templul Dumnezeului nostru.
Neh 11:1  Capeteniile poporului s-au a?ezat în Ierusalim, iar celalalt popor a tras sor?i, ca a zecea parte din ei sa locuiasca în sfânta cetate a Ierusalimului, iar celelalte noua par?i sa ramâna în celelalte ceta?i.
Neh 11:2  ?i a binecuvântat poporul pe to?i cei ce s-au învoit de buna voie sa se a?eze în Ierusalim.
Neh 11:3  Iata capeteniile ?arii care s-au a?ezat în Ierusalim ?i în ceta?ile lui Iuda, locuind fiecare pe mo?ia sa, în cetatea sa: Israeli?ii, preo?ii ?i levi?ii, cei încredin?a?i templului ?i urma?ii slujitorilor lui Solomon.
Neh 11:4  În Ierusalim s-au a?ezat din fiii lui Iuda ?i din fiii lui Veniamin. Din fiii lui Iuda s-au a?ezat: Ataia, fiul lui Uzia, fiul lui Zaharia, fiul lui Amaria, fiul lui ?efatia, fiul lui Mahalaleel din fiii lui Fares,
Neh 11:5  ?i Maaseia, fiul lui Baruc, fiul lui Col-Hoze, fiul lui Hazaia, fiul lui Adaia, fiul lui Ioiarib, fiul lui Zaharia, fiul lui ?iloni.
Neh 11:6  Fiii lui Fares care s-au a?ezat în Ierusalim erau în total patru sute ?aizeci ?i opt de oameni de oaste.
Neh 11:7  Iata acum fiii lui Veniamin: Salu, fiul lui Me?ulam, fiul lui Ioed, fiul lui Pedaia, fiul lui Colaia, fiul lui Maaseia, fiul lui Itiel, fiul lui Isaia,
Neh 11:8  Apoi Gabai ?i ?alai, noua sute douazeci ?i opt.
Neh 11:9  Ioil, fiul lui Zicri, era capetenia lor, iar Iuda, fiul lui Senua, era a doua capetenie a ceta?ii.
Neh 11:10  Din preo?i: Iedaia, fiul lui Ioiarib, ?i Iachin,
Neh 11:11  Seraia, fiul lui Hilchia, fiul lui Me?ulam, fiul lui ?adoc, fiul lui Meraiot, fiul lui Ahitub, capetenia templului Domnului;
Neh 11:12  ?i fra?ii lor, care se îndeletnicesc cu slujba în templul Domnului, opt sute douazeci ?i doi; Adaia, fiul lui Ieroham, fiul lui Pelalia, fiul lui Am?i, fiul lui Zaharia, fiul lui Pa?hur, fiul lui Malchia,
Neh 11:13  ?i fra?ii sai, capeteniile caselor parinte?ti, doua sute patruzeci ?i doi; ?i Amasai, fiul lui Azareel, fiul lui Ahzai, fiul lui Me?ilemot, fiul lui Imer,
Neh 11:14  ?i fra?ii lor, barba?i de oaste, o suta douazeci ?i opt; Zabdiel, fiul lui Ghedolim, era capetenia lor.
Neh 11:15  Dintre levi?i: ?emaia, fiul lui Ha?ub, fiul lui Azricam, fiul lui Ha?abia, fiul lui Buni,
Neh 11:16  ?abetai ?i Iozabad, capetenii ale levi?ilor pentru treburile din afara ale templului lui Dumnezeu;
Neh 11:17  Matania, fiul lui Mica, fiul lui Zabdi, fiul lui Asaf, capetenia care începea laudele în timpul rugaciunii, ?i Bacbuchia, al doilea între fra?ii sai ?i Abda, fiul lui ?amua, fiul lui Galal, fiul lui Iedutun.
Neh 11:18  To?i levi?ii în cetatea sfânta erau doua sute optzeci ?i patru.
Neh 11:19  Portarii: Acub, Talmon ?i fra?ii lor, pazitorii por?ilor, o suta ?aptezeci ?i doi.
Neh 11:20  Ceilal?i Israeli?i, preo?i ?i levi?i, se a?ezara în toate ceta?ile lui Iuda, fiecare la mo?ia sa.
Neh 11:21  Cei încredin?a?i templului se a?ezara pe colina ?i aveau de capetenie pe ?iha ?i Ghi?pa.
Neh 11:22  Capetenia levi?ilor la Ierusalim era Uzi, fiul lui Bani, fiul lui Ha?abia, fiul lui Matania, fiul lui Mica, dintre fiii lui Asaf, cântare?i însarcina?i sa faca slujba la templul lui Dumnezeu;
Neh 11:23  Caci pentru cântareli era o porunca din partea regelui, prin care li se hotara o plata anumita pe fiecare zi.
Neh 11:24  Petahia, fiul lui Me?ezabeel din fiii lui Zerah, fiul lui Iuda, era împuternicit de rege pentru toate nevoile poporului.
Neh 11:25  Iar dintre cei ce traiesc în sate pe ?arinile lor, fii de ai lui Iuda, s-au a?ezat la Kiriat-Arba ?i în satele ce ?ineau de el; la Dibon ?i în satele ce ?ineau de el; la Iecab?eel ?i în satele lui;
Neh 11:26  La Ie?ua, la Molada, la Bet-Pelet;
Neh 11:27  La Ha?ar-?ual, la Beer-?eba ?i în satele ce ?ineau de ea;
Neh 11:28  La ?iclag, la Mecona ?i în satele ce ?ineau de ea;
Neh 11:29  La En-Rimon, la ?orea, la Iarmut;
Neh 11:30  La Zanoah, la Adulam ?i în satele ce ?ineau de el; la Lachi? ?i în ?arinile ce ?ineau de el; la Azeca ?i în satele ce ?ineau de el; s-au a?ezat ei de la Beer-?eba pâna la Ghe-Hinom.
Neh 11:31  Fiii lui Veniamin s-au a?ezat, începând de la Gheba: în Micmas, în Aia, în Betel ?i în satele lui;
Neh 11:32  În Anatot, în Nob, în Anania,
Neh 11:33  În Ha?or, în Rama, în Ghitaim,
Neh 11:34  În Hadid, în ?eboim, în Nebalat,
Neh 11:35  În Lod, în Ono ?i în Ghehara?im.
Neh 11:36  Erau ?i levi?i care aveau ?arini în Veniamin, cu toate ca erau socoti?i în partea lui Iuda.
Neh 12:1  Iata preo?ii ?i levi?ii care au venit cu Zorobabel, fiul lui Salatiel, ?i cu Iosua: Seraia, Ieremia, Ezdra,
Neh 12:2  Amaria, Maluc, Hatu?,
Neh 12:3  ?ecania, Rehum, Meremot,
Neh 12:4  Ido, Ghineton, Abia,
Neh 12:5  Miiamin, Moadia, Bilga,
Neh 12:6  ?emaia, Ioiarib, Iedaia,
Neh 12:7  Salu, Amoc, Hilchia, Iedaia. Ace?tia au fost capeteniile preo?ilor ?i ale fra?ilor lor în vremea lui Iosua.
Neh 12:8  Levi?ii au fost: Iosua, Binui, Cadmiel, ?erebia, Iuda, Matania, care împreuna cu fra?ii sai conducea pe cântareli în timpul slujbelor de lauda;
Neh 12:9  Bacbuchia ?i Uni, care împreuna cu fra?ii lor cântau de cealalta parte.
Neh 12:10  Iosua a nascut pe Ioachim, Ioachim a nascut pe Elia?ib, Elia?ib a nascut pe Ioiada;
Neh 12:11  Ioiada a nascut pe Iohanan, Iohanan a nascut pe Iadua.
Neh 12:12  În zilele lui Ioachim, preo?i, capetenii de familii, erau: Meraia pentru casa lui Seraia; Hanania pentru casa lui Ieremia;
Neh 12:13  Me?ulam pentru casa lui Ezdra; Iohanan pentru casa lui Amaria;
Neh 12:14  Ionatan pentru casa lui Maluc, Iosif pentru casa lui ?ecania;
Neh 12:15  Adna pentru casa lui Harim; Helcai pentru casa lui Meremot;
Neh 12:16  Zaharia pentru casa lui Ido; Me?ulam pentru casa lui Ghineton;
Neh 12:17  Zicri pentru casa lui Abia; Piltai pentru casa lui Miiamin ?i Moadia;
Neh 12:18  ?amua pentru casa lui Bilga; Ionatan pentru casa lui ?emaia;
Neh 12:19  Matanai pentru casa lui Ioiarib; Uzi pentru casa lui Iedaia;
Neh 12:20  Calai pentru casa lui Salai; Eber pentru casa lui Amoc;
Neh 12:21  Ha?abia pentru casa lui Hilchia; Natanael pentru casa lui Iedaia.
Neh 12:22  În zilele lui Elia?ib, ale lui Ioiada, ale lui Iohanan ?i ale lui Iadua, levi?ii, capetenii de case, au fost înscri?i cu preo?ii pâna la domnia lui Darie Persanul;
Neh 12:23  Fiii lui Levi, capetenii de familie, au fost înscri?i în cartea Cronicilor pâna în timpul lui Iohanan, nepot de fiu al lui Elia?ib;
Neh 12:24  Capeteniile levi?ilor erau deci: Ha?abia, ?erebia ?i Iosua, fiul lui Cadmiel; iar fra?ii lor alcatuiau ceata a doua pentru a lauda ?i a slavi numele lui Dumnezeu, dupa rânduiala lui David, omul lui Dumnezeu, raspunzând o ceata celeilalte.
Neh 12:25  Matania, Bacbuchia ?i Obadia, Me?ulam, Talmon ?i Acub aveau slujba de portari ?i strajuiau când se aduna poporul înaintea por?ii.
Neh 12:26  Ace?tia au trait pe vremea lui Ioachim, fiul lui Iosua, fiul lui Io?adac ?i pe timpul lui Neemia guvernatorul ?i al preotului ?i carturarului Ezdra.
Neh 12:27  La sfin?irea zidului Ierusalimului înca s-au chemat levi?ii din toate locurile, poruncindu-li-se sa vina la Ierusalim pentru savâr?irea sfin?irii ?i pentru sarbatoarea vesela cu doxologii ?i cântari, în sunetul chimvalelor, lirelor ?i harfelor.
Neh 12:28  Cei din neamul cântarelilor s-au adunat din împrejurimile Ierusalimului ?i din ceta?ile netofatiene,
Neh 12:29  Din Bet-Ghilgal, din ?esurile Ghebei, din Azmavet, caci cântare?ii î?i facusera satele în apropiere de Ierusalim.
Neh 12:30  ?i preo?ii ?i levi?ii s-au cura?it ?i au sfin?it poporul, por?ile ?i zidurile.
Neh 12:31  ?i am suit eu capeteniile lui Iuda pe ziduri ?i am pus doua coruri mari pentru cântarea de lauda; unul din ele s-a a?ezat în partea dreapta a zidului, spre poarta Gunoiului.
Neh 12:32  În urma lor mergeau Ho?aia ?i jumatate din capeteniile lui Iuda:
Neh 12:33  Azaria, Ezdra ?i Me?ulam;
Neh 12:34  Iuda, Veniamin, ?emaia ?i Ieremia;
Neh 12:35  Iar dintre fiii preo?ilor mergeau cu trâmbi?e: Zaharia, fiul lui Ionatan, fiul lui ?emaia, fiul lui Matania, fiul lui Mica, fiul lui Zacur, fiul lui Asaf,
Neh 12:36  ?i fra?ii lui: ?emaia, Azareel, Milalai, Ghilalai, Maai, Natanael, Iuda ?i Hanani cu instrumentele muzicale ale lui David, omul lui Dumnezeu, iar înaintea lor se afla carturarul Ezdra.
Neh 12:37  La Poarta Izvorului, în fa?a lor, s-au urcat pe treptele scarii de la cetatea lui David, care ducea pe zidul de deasupra casei lui David, pâna la Poarta Apelor, spre rasarit.
Neh 12:38  Al doilea cor s-a îndreptat în cealalta parte, ?i dupa el am mers eu ?i jumatate din popor ?i am înaintat pe zid de la turnul Cuptoarelor ?i pâna la zidul cel lat,
Neh 12:39  ?i de la Poarta Efraim, am trecut pe lânga poarta veche ?i Poarta Pe?tilor, pe lânga turnul lui Hananeel, pe lânga turnul Mea spre Poarta Oilor ?i ne-am oprit la Poarta închisorii.
Neh 12:40  Apoi amândoua corurile au mers la templul lui Dumnezeu. ?i am facut acela?i lucru ?i noi, eu ?i capeteniile care erau cu mine
Neh 12:41  ?i preo?ii: Eliachim, Maaseia, Miniamin, Mica, Elioenai, Zaharia, Hanania cu trâmbi?e,
Neh 12:42  Maaseia, ?emaia, Eleazar, Uzi, Iohanan, Malchia, Elam ?i Ezer. Cântare?ii cântau sub conducerea lui Izrahia.
Neh 12:43  ?i s-au adus în ziua aceea mul?ime de jertfe ?i s-au veselit, caci Dumnezeu daduse poporului pricina mare de bucurie. ?i femeile ?i copiii s-au veselit, iar strigatele lor de bucurie din Ierusalim se auzeau pâna departe.
Neh 12:44  În aceea?i zi au fost pu?i oameni supraveghetori la camarile unde aveau sa se aduca prinoasele, pârga ?i zeciuielile, ?i au fost împuternici?i sa adune din ?arinile din jurul ceta?ilor par?ile hotarâte prin lege pentru preo?i ?i levi?i. ?i Iudeii s-au bucurat când au vazut pe preo?i ?i pe levi?i la slujba lor,
Neh 12:45  Îndeplinind toate cele pentru slujba dumnezeiasca ?i pentru cura?ire. Cântare?ii ?i portarii ?i-au îndeplinit datoria lor dupa porunca lui David ?i a lui Solomon, fiul lui;
Neh 12:46  Înca de demult, din zilele lui David ?i Asaf se hotarâsera capeteniile cântare?ilor ?i cântarile cele de lauda ?i de mul?umire în cinstea lui Dumnezeu.
Neh 12:47  În zilele lui Zorobabel ?i Neemia, tot Israelul dadea parte cântare?ilor ?i portarilor pentru fiecare zi; ?i dadeau levi?ilor darurile sfin?ite, ?i levi?ii le dadeau de asemenea darurile sfin?ite urma?ilor lui Aaron.
Neh 13:1  În acea zi s-a citit din cartea lui Moise în auzul poporului ?i s-a gasit scris în ea ca Amonitul ?i Moabitul nu pot fi primi?i niciodata în adunarea lui Dumnezeu,
Neh 13:2  Pentru ca ei n-au întâmpinat pe fiii lui Israel cu pâine ?i apa ?i pentru ca au tocmit împotriva lor pe Valaam, ca sa-i blesteme, dar Dumnezeul nostru a schimbat blestemul în binecuvântare.
Neh 13:3  ?i auzind aceasta lege, a despar?it Israel tot ce era strain.
Neh 13:4  Dar înainte de aceasta, preotul Elia?ib, mai-marele peste camarile templului Dumnezeului nostru, ?i ruda cu Tobie,
Neh 13:5  Rânduise pentru acesta o odaie mare, unde mai înainte se puneau prinoasele, tamâia, vasele sfinte, zeciuiala de pâine, de vin ?i de untdelemn, ce era hotarâta pentru levi?i, cântare?i, portari, ?i ceea ce se cuvenea preo?ilor.
Neh 13:6  Eu nu eram la Ierusalim când s-au facut toate acestea, pentru ca ma întorsesem la rege, în anul al treizeci ?i doilea al domniei lui Artaxerxe, regele Babilonului.
Neh 13:7  ?i la sfâr?itul acestui an, am capatat de la rege învoirea de a veni la Ierusalim, ?i am aflat de raul ce a facut Elia?ib, dând lui Tobie o camara în curtea templului lui Dumnezeu.
Neh 13:8  ?i mi-a parut foarte rau de acestea ?i am aruncat afara din camara toate lucrurile care erau ale lui Tobie, dând porunca sa fie sfin?ite camarile.
Neh 13:9  ?i am pus din nou lucrurile templului lui Dumnezeu, prinoasele ?i tamâia.
Neh 13:10  Am mai în?eles de asemenea ca par?ile ce se cuveneau levi?ilor nu le fusesera date ?i ca levi?ii ?i cântare?ii, însarcina?i sa faca slujba, fugisera pe la ?arinile lor.
Neh 13:11  De aceea am mustrat pe capetenii ?i am zis: "Pentru ce a fost parasit templul lui Dumnezeu?" ?i i-am adunat pe levi?i ?i pe cântare?i ?i i-am pus la locurile lor.
Neh 13:12  Atunci to?i Iudeii au început a aduce la camara zeciuiala de pâine, de vin ?i de untdelemn;
Neh 13:13  ?i eu am încredin?at supravegherea camarilor lui ?elemia preotul ?i lui ?adoc carturarul, ?i lui Pedaia, unul dintre levi?i, ?i le-am dat ca ajutor pe Hanan, fiul lui Zacur, fiul lui Matania, caci ace?tia se bucurau de mare cinste. ?i tot ei au fost împuternici?i sa faca împar?ire fra?ilor lor.
Neh 13:14  Pomene?te-ma, Dumnezeul meu, pentru aceasta ?i nu uita faptele mele evlavioase facute pentru templul Dumnezeului meu ?i pentru slujbele din el!
Neh 13:15  În zilele acelea am vazut în Iuda oameni care calcau în teascuri în ziua odihnei, aduceau snopi ?i încarcau pe asini vin, struguri, smochine ?i tot felul de greuta?i ?i le duceau în ziua odihnei la Ierusalim. ?i le-am facut mustrare aspra chiar atunci, când vindeau lucruri de mâncare.
Neh 13:16  De asemenea se aflau Tirieni, a?eza?i în Ierusalim, care aduceau pe?te ?i tot felul de marfuri ?i le vindeau, în ziua odihnei, Iudeilor în Ierusalim.
Neh 13:17  ?i am mustrat pe mai-marii Iudeilor ?i le-am zis: "Ce înseamna aceste fapte urâte, pe care le face?i voi, pângarind ziua odihnei?
Neh 13:18  Oare nu a?a au facut parin?ii vo?tri ?i nu din pricina aceasta a adus Dumnezeul nostru aceste necazuri asupra voastra ?i asupra acestei ceta?i? ?i voi atrage?i din nou mânia Lui asupra lui Israel, necinstind ziua odihnei!"
Neh 13:19  Apoi am poruncit sa se închida toate por?ile Ierusalimului în ajunul zilei de odihna, când începe a se întuneca, ?i sa nu se mai deschida decât dupa ea. ?i am pus câte unul din slujitorii mei la por?i, ca sa împiedice intrarea de marfuri în ziua odihnei.
Neh 13:20  Atunci negu?atori ?i vânzatori de tot felul de lucruri au ramas câteodata sau de doua ori afara din Ierusalim.
Neh 13:21  Dar eu i-am mustrat aspru, zicându-le: "pentru ce petrece?i voi noaptea înaintea zidurilor? De ve?i mai face aceasta, voi pune mâna pe voi!" De atunci ei n-au mai venit în ziua odihnei.
Neh 13:22  ?i am mai poruncit eu levi?ilor sa se cure?e ?i sa vina sa pazeasca por?ile, ca sa sfin?easca ziua odihnei. Adu-?i aminte de mine, Dumnezeul meu, pentru acestea ?i ma ocrote?te dupa mare mila Ta!
Neh 13:23  Tot atunci am vazut eu Iudei, luându-?i femei a?dodiene, amonite ?i moabite.
Neh 13:24  Jumatate din copiii lor vorbeau limba a?dodiana ?i nu ?tiau sa vorbeasca evreie?te; nu ?tiau decât limba unuia sau altuia din acele popoare.
Neh 13:25  Eu i-am mustrat aspru ?i pe ace?tia ?i i-am blestemat; ba pe unii i-am lovit, le-am smuls parul ?i i-am jurat pe numele lui Dumnezeu, zicând: "Sa nu va da?i fetele dupa feciorii lor ?i sa nu lua?i pe fetele lor, nici pentru fiii vo?tri, nici pentru voi.
Neh 13:26  Oare nu a?a a pacatuit Solomon, regele lui Israel? Nu se afla rege ca el la nici un popor ?i el era iubit de Dumnezeu ?i Domnul îl pusese rege peste tot Israelul; dar femeile cele de alt neam l-au atras ?i pe el în pacat.
Neh 13:27  Se poate oare sa aud eu de voi ca face?i acest rau mare ?i pacatui?i înaintea lui Dumnezeu, luându-va femei de alt neam?"
Neh 13:28  Unul din fiii lui Ioiada, fiul lui Elia?ib, preotul cel mare, era ginerele lui Sanbalat Horonitul; dar eu l-am alungat de la mine.
Neh 13:29  Adu-?i aminte de ei, Dumnezeul meu, caci au spurcat preo?ia ?i legamântul preo?esc ?i levit!
Neh 13:30  Astfel i-am cura?it eu de to?i strainii ?i am pus rânduiala preo?ilor ?i levi?ilor, fiecaruia dupa slujba lui,
Neh 13:31  ?i deasemenea pentru aducerea lemnelor la vremea hotarâta, precum ?i a prinoaselor de pârga. Adu-?i aminte de mine, Dumnezeul meu, spre binele meu!


\end{document}