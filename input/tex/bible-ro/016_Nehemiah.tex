\begin{document}

\title{Neemia}


\chapter{1}

\par 1 Cuvintele lui Neemia, fiul lui Hacalia. "În luna Chislev, în anul al douăzecilea al regelui Artaxerxe, mi s-a întâmplat să fiu în capitala Suza.
\par 2 Atunci a venit din Iuda Hanani, unul din frații mei și alți câțiva oameni; și i-am întrebat despre cei ee scăpaseră din robie și rămăseseră în Iuda și despre Ierusalim.
\par 3 "Cei ce au scăpat din robie și au rămas - îmi spuseră ei - sunt acolo, în țara lor, în mare necaz și înjosire; iar zidurile Ierusalimului sunt dărâmate și porțile lui arse".
\par 4 Auzind eu cuvintele acestea, am început să plâng și am fost întristat câteva zile, am postit și m-am rugat înaintea Dumnezeului ceresc, zicând:
\par 5 "Doamne, Dumnezeule al cerurilor, Dumnezeul cel mare și înfricoșător, Care păzești legământul Tău și ești milostiv cu cei ce Te iubesc și păzesc poruncile Tale,
\par 6 Să fie urechile Tale cu luare-aminte și ochii Tăi deschiși, ca să auzi rugăciunea robului Tău, cu care mă rog eu acum ziua și noaptea înaintea Ta, pentru fiii lui Israel, robii Tăi, mărturisind păcatele fiilor lui Israel, cu care am păcătuit înaintea Ta și eu și casa tatălui meu.
\par 7 Noi Te-am mâniat și n-am păzit poruncile, legile și orânduielile pe care le-ai dat Tu lui Moise, robul Tău.
\par 8 Adu-Ți aminte însă de cuvântul pe care l-ai spus robului Tău Moise, când ai zis: "De veți păcătui, vă voi împrăștia printre popoare;
\par 9 Iar când vă veți întoarce la Mine și veți păzi poruncile Mele și le veți împlini, atunci, de ați fi izgoniți chiar și la marginea cerului, și de acolo vă voi aduna și vă voi aduce la locul pe care l-am ales, ca să-Mi pun numele Meu acolo".
\par 10 Aceștia însă sunt robii Tăi și poporul Tău, pe care Tu l-ai răscumpărat cu puterea Ta și cu mâna Ta cea puternică.
\par 11 Rogu-Te dar, o, Doamne, să fie urechile Tale cu luare-aminte la rugăciunea robului Tău și la rugăciunea robilor Tăi, cărora le place să se teamă de numele Tău, și ajută robului Tău acum și-i dă să dobândească milă la omul acesta. Căci eu eram paharnicul regelui".

\chapter{2}

\par 1 "În luna Nisan, în anul al douăzecilea al regelui Artaxerxe, având vinul în seama mea, am luat vin și am dat regelui și niciodată, mi se pare, nu m-am arătat trist înaintea lui.
\par 2 Dar regele mi-a zis: "De ce este tristă fața ta? De bolnav, nu ești bolnav! Se vede că ai vre-o întristare la inimă!"
\par 3 Și m-am speriat strașnic și am răspuns regelui: "în veci să trăiască regele! Cum să nu fie tristă fața mea, când cetatea, casa mormintelor părinților mei, este pustiită și porțile ei arse cu foc!"
\par 4 "Și ce dorești tu?" - întrebă regele. Eu însă, după ce m-am rugat Dumnezeului ceresc, am zis către rege:
\par 5 "De binevoiește regele și de are robul tău trecere la tine, atunci trimite-mă în Iuda, la cetatea unde sunt mormintele părinților mei, ca să o zidesc".
\par 6 Iar regele și regina, care ședea lângă el, mi-au zis: "Cât timp are să dureze călătoria ta și când ai să te întorci?" Și după ce am arătat cât timp am să lipsesc, regele a binevoit să mă trimită în Iuda.
\par 7 La plecare însă am zis către rege: "De binevoiește regele, să-mi dea scrisori către guvernatorii de peste fluviu, ca să-mi dea drumul să pot ajunge în Iuda,
\par 8 Precum și o scrisoare către Asaf, păzitorul pădurilor regelui, ca să-mi dea lemn pentru porțile cetății, cele dinspre templul Domnului, și pentru zidul cetății și pentru casa mea de locuit". Și mi-a dat regele scrisori, pentru că mâna binefăcătoare a Dumnezeului meu era peste mine.
\par 9 Și am mers la guvernatorii de peste fluviu și le-am dat scrisorile regelui. Regele însă trimisese cu mine căpetenii militare cu călăreți.
\par 10 Dar când au auzit de aceasta Sanbalat Horonitul și Tobie, robul amonit, le-a părut foarte rău că a. venit un om să se îngrijească de binele fiilor lui Israel.
\par 11 Iar după ce am ajuns la Ierusalim și am stat acolo trei zile,
\par 12 M-am sculat noaptea cu puțini oameni ai mei să cercetez cetatea. Nu spusesem însă nimănui ce-mi dăduse Dumnezeu în gând să fac pentru Ierusalim și nu se afla acolo la mine nici o vită, decât numai aceea pe care încălecasem eu.
\par 13 Și am ieșit atunci noaptea pe Poarta Văii și m-am îndreptat spre izvorul Dragonului și spre Poarta Gunoiului și am cercetat zidurile Ierusalimului cele stricate și porțile lui cele arse.
\par 14 Și m-am apropiat de poarta Izvorului și de iazul Regelui, dar nu era ioc să treacă dobitocul care mă purta.
\par 15 Și de aceea m-am suit înapoi pe vale și am cercetat din nou zidurile și, intrând tot pe Poarta Văii, m-am înapoiat.
\par 16 Căpeteniile insă nu știau unde fusesem eu și ce făcusem. Până atunci eu nu spusesem nimic nici Iudeilor, nici preoților, nici celor mai de seamă, nici căpeteniilor, nici celorlalți care se îndeletniceau cu felurite lucrări.
\par 17 Atunci însă le-am zis: "Voi vedeți în ce stare rea ne aflăm: Ierusalimul este dărâmat și porțile lui mistuite de foc. Hai să zidim zidurile Ierusalimului și nu vom mai fi de batjocură!"
\par 18 Și le-am istorisit cum mâna binefăcătoare a lui Dumnezeu fusese peste mine și cum mi-a vorbit regele. Atunci ei au zis: "Hai să zidim!" Și s-au întărit ei în această hotărâre bună.
\par 19 Dar auzind despre aceasta, Sanbalat Horonitul, Tobie, robul amonit și Gheșem Arabul au râs de noi și cu dispreț ne-au zis: "Ce v-ați apucat să faceți voi aici? Nu cumva gândiți să vă răzvrătiți împotriva regelui?"
\par 20 "Dumnezeul cel ceresc ne va ajuta, le răspunsei eu. Noi, slujitorii Lui, ne vom apuca de zidit, iar voi n-aveți nici parte, nici drept, nici pomenire în Ierusalim".

\chapter{3}

\par 1 "Atunci s-a ridicat Eliașib, preotul cel mare, cu frații săi preoți și au zidit Poarta Oilor și, punându-i canaturile, au sfințit-o; tot ei au reparat și zidul de la turnul Mea până la turnul Hananeel și l-au sfințit.
\par 2 Lângă Eliașib au zidit Ierihonenii, iar lângă aceștia a zidit Zacur, fiul lui Imri.
\par 3 Fiii lui Senaa au zidit Poarta Peștilor și au acoperit-o, punându-i canaturile cu încuietorile și zăvoarele trebuitoare.
\par 4 De la aceștia înainte a reparat Meremot, fiul lui Urie, fiul lui Hacoț; lângă acesta a reparat Meșulam, fiul lui Berechia, fiul lui Meșezabeel; de la aceștia înainte a reparat Țadoc, fiul lui Baana.
\par 5 Alături de aceștia au reparat cei din Tecoa, ai căror fruntași de altfel nu și-au plecat grumazul să lucreze pentru Domnul lor.
\par 6 Ioiada, fiul lui Paseah și Meșulam, fiul lui Besodia, au reparat Poarta Veche și, acoperind-o, i-au pus canaturile cu încuietorile și zăvoarele trebuitoare.
\par 7 De la ei înainte au lucrat Melatia Ghibeonitul și Iadon din Meronot, precum și oamenii din Ghibeon și din Mițpa, supuși stăpânirii guvernatorilor de dincolo de fluviu.
\par 8 Alături de aceștia a reparat Uziel, fiul lui Harhaia Argintarul, iar de la acesta înainte a reparat Hanania, fiul lui Rocheim. Și întăriră Ierusalimul până la zidul cel lat.
\par 9 De la ei înainte a reparat Refaia, fiul lui Hur, căpetenia unei jumătăți din ținutul Ierusalimului.
\par 10 Lângă acesta a reparat, în fața casei lui, Iedaia, fiul lui Harumaf, iar lângă el a reparat Hatuș, fiul lui Hașabneia.
\par 11 O altă parte de zid a fost reparată de Malchia, fiul lui Harim și de Hașub, fiul lui Pahat-Moab; tot ei au reparat și turnul Cuptoarelor.
\par 12 De la ei înainte au lucrat, cu fiicele sale, Șalum, fiul lui Haloheș, căpetenia celeilalte jumătăți a ținutului Ierusalimului.
\par 13 Hanun și locuitorii din Zanoah au reparat Poarta Văii. Ei au zidit-o și i-au pus canaturile cu încuietorile și zăvoarele. Tot ei au făcut mai bine de o mie de coți de zid, până la Poarta Gunoiului.
\par 14 Malchia, fiul lui Recab, căpetenia ținutului Bet-Hacherem, a reparat Poarta Gunoiului; i-a pus canaturile, cu încuietorile și zăvoarele necesare.
\par 15 Șalum, fiul lui Col-Hoze, căpetenia ținutului Mițpa, a reparat Poarta Izvorului, a zidit-o și i-a pus canaturile cu încuietorile și zăvoarele trebuitoare. El a mai făcut zidul de la scăldătoarea Siloam, de lângă grădina regelui, până la scara ce coboară din cetatea lui David.
\par 16 De la el înainte a reparat Neemia, fiul lui Azbuc, căpetenia unei jumătăți din ținutul Bet-Țur, până în fața mormintelor lui David și până la iazul cel săpat și până la Casa Vitejilor.
\par 17 De la el înainte au reparat leviții: Rehum, fiul lui Bani; lângă acesta a reparat Hașabia, căpetenia unei jumătăți din ținutul Cheila, din partea acestui ținut.
\par 18 Mai departe au reparat frații lui: Bavai, fiul lui Henanad, căpetenia celeilalte jumătăți din ținutul Cheila,
\par 19 Și lângă el Ezer, fiul lui Iosua, căpetenia din Mițpa, a reparat altă parte de zid din fața scării armelor până la colț.
\par 20 De la el înainte Baruc, fiul lui Zabai, a reparat cu multă sârguință o altă parte de zid, de la colț până la poarta casei lui Eliașib, preotul cel mare.
\par 21 După el a reparat o altă parte de zid Meremot, fiul lui Urie, fiul lui Hacoț, de la poarta casei lui Eliașib până la capătul casei lui Eliașib.
\par 22 Mai departe au reparat preoții din împrejurimile Ierusalimului.
\par 23 Iar lângă ei, Veniamin și Hașub au reparat zidul în dreptul casei lor. Lângă ei Azaria, fiul lui Maaseia, fiul lui Anania, a reparat zidul lângă casa sa.
\par 24 Lângă acesta, Binui, fiul lui Henadad a reparat altă bucată de zid, de la casa lui Azaria până în dreptul colțului.
\par 25 Palal, fiul lui Uzai, a reparat în dreptul unghiului și al turnului, care se ridică deasupra casei de sus a regelui cea din apropierea curții închisorii. Lângă acesta a reparat Pedaia, fiul lui Paroș.
\par 26 Iar cei încredințați templului, care erau pe deal, au reparat zidul până în dreptul Porții Apelor, spre răsărit și până la turnul cel înalt.
\par 27 De la ei înainte cei din Tecoa au reparat o parte, în fața turnului celui mare și înalt până la zidul lui Ofel.
\par 28 De la Poarta Cailor în sus au reparat preoții, fiecare în dreptul casei sale.
\par 29 De la ei înainte a reparat Țadoc, fiul lui Imer, dinaintea casei sale, iar după el a reparat Șemaia, fiul lui Șecania, străjerul Porții Răsăritului.
\par 30 După el Hanania, fiul lui Șelemia și Hanun, al șaselea fiu al lui țalaf, au reparat altă parte de zid. După ei Meșulam, fiul lui Berechia, a reparat în dreptul locuinței lui.
\par 31 Apoi Malchia făurarul a reparat până la casa celor încredințați templului și a negustorilor, în dreptul porții lui Mifcad și până la foișorul din colț.
\par 32 Iar argintarii și negustorii au dres zidul de la foișorul din colț și până la Poarta Oilor".

\chapter{4}

\par 1 "Când a auzit Sanbalat, că noi refacem zidurile Ierusalimului, s-a mâniat și s-a tulburat tare și a batjocorit pe Iudei;
\par 2 Și de față cu frații săi și de față cu ostașii samarineni a zis: "Ce fac acești Iudei neputincioși? Se poate oare să li se îngăduie aceasta? Se poate oare să aducă jertfe? Se poate oare să li se îngăduie aceasta? Se poate oare să dezgroape pietrele din movilele de moloz și încă arse?"
\par 3 Iar Tobie Amonitul de lângă el a zis: "Lasă-i să zidească, fiindcă are să vină o vulpe și are să le strice zidul lor cel de piatră!"
\par 4 Auzi, Dumnezeul nostru, în ce dispreț suntem noi și întoarce ocara lor asupra copiilor lor și dă-i disprețului în țara robiei.
\par 5 Și nelegiuirile lor să nu le acoperi, nici păcatul lor să nu se șteargă înaintea feței Tale, pentru că au amărât pe cei ce zideau!
\par 6 Și noi totuși am lucrat înainte la zid și l-am încheiat peste tot până la jumătate; și poporului nu-i lipsea râvnă de a lucra.
\par 7 Dar Sanbalat, Tobie, Arabii, Amoniții și cei din Așdod, auzind că se înalță zidurile Ierusalimului, că spărturile lui au început să se astupe, s-au mâniat foarte tare;
\par 8 Și s-au sfătuit împreună să vină cu război și să dărâme Ierusalimul.
\par 9 Atunci noi ne-am rugat Dumnezeului nostru și am pus strajă, ca să ne păzească de ei ziua și noaptea.
\par 10 Iudeii însă au zis: "Puterile salahorilor au slăbit și moloz e foarte mult; noi nu mai suntem în stare să facem zidul".
\par 11 Iar dușmanii noștri grăiau: "Nici n-au să afle, nici n-au să vadă nimic până când vom ajunge în mijlocul lor; îi vom ucide și vom face să înceteze lucrul".
\par 12 Și după ce au venit Iudeii care locuiau lângă ei și ne-au spus de vreo zece ori, din toate părțile, că dușmanii noștri au să năvălească asupra noastră,
\par 13 Atunci în părțile de jos ale cetății, după ziduri, am așezat poporul după seminții, cu săbii și cu lănci și cu arcuri.
\par 14 Apoi am cercetat toate și, sculându-mă, am zis celor mai de seamă și căpeteniilor și celuilalt popor: "Nu vă temeți de ei! Aduceți-vă aminte de Domnul cel mare și înfricoșător și luptați-vă pentru frații voștri, pentru fiii voștri, pentru fiicele voastre, pentru femeile voastre și pentru casele voastre!"
\par 15 Când au auzit dușmanii noștri că ne este cunoscut gândul lor, atunci Dumnezeu a risipit planul lor; iar noi ne-am întors cu toții la ziduri, apucându-ne fiecare de lucrul nostru.
\par 16 Din acea zi jumătate din tinerii mei se îndeletniceau cu lucrul, iar jumătate din ei stăteau gata de apărare cu lănci, cu scuturi, cu arcuri și cu platoșe, iar îndărătul lor se aflau căpeteniile a toată casa lui Iuda.
\par 17 Cei ce lucrau la zid și cei ce cărau, cu o mână lucrau, iar cu cealaltă țineau lancea.
\par 18 Fiecare din cei ce zideau erau încinși peste coapsele lor cu sabia și așa lucrau; iar lângă mine se afla cel ce suna din trâmbiță.
\par 19 Și eu am zis celor mai însemnați și căpeteniilor și poporului de rând: "Lucrul este mult și greu și noi suntem risipiți pe ziduri și depărtați unii de alții.
\par 20 De aceea să alergați la mine acolo unde veți auzi sunetul trâmbiței și Dumnezeul nostru se va lupta pentru noi".
\par 21 Și așa am lucrat noi, iar jumătate stătea cu lancea în mână de la răsăritul soarelui până seara la ieșirea stelelor.
\par 22 Afară de aceasta, tot atunci am mai zis poporului să rămână toți în Ierusalim cu robii lor, străjuind în timpul nopții, iar în timpul zilei lucrând.
\par 23 Și nici eu, nici frații mei, nici slugile mele, nici străjerii care mă însoțiseră, nu s-au dezbrăcat de haine și fiecare până și la apă mergea cu sabia în mână".

\chapter{5}

\par 1 Atunci s-a făcut murmur mare în popor și între femei împotriva fraților lor Iudei.
\par 2 Erau unii care ziceau: "Noi cu fiii și cu fiicele noastre suntem mulți, să ni se dea pâine să ne hrănim și să trăim".
\par 3 Alții ziceau: "Ne-am pus zălog viile și ogoarele și casele noastre, ca să ne luăm pâine să ne astâmpărăm foamea".
\par 4 Iar alții ziceau: "Noi luăm bani cu camătă ca să plătim bir regelui, zălogindu-ne viile și ogoarele noastre.
\par 5 Și noi avem același trup ca și frații voștri și fiii voștri sunt tot așa cum sunt și fiii lor; dar iată că noi suntem nevoiți să ne dăm fiii și fiicele robi, ba unele din fiicele noastre se și află în robie. N-avem la îndemână nici un mijloc de răscumpărare. Și viile noastre și ogoarele noastre sunt în stăpânirea altora".
\par 6 Când am auzit eu murmurul lor și astfel de vorbe m-am mâniat foarte tare.
\par 7 Inima mea s-a tulburat și am dojenit aspru pe cei mai însemnați și pe căpetenii. "Voi luați mită de la frații voștri", le-am zis eu. Și am făcut împotriva lor o adunare mare,
\par 8 Și le-am zis: "Noi, pe cât ne-au ajutat puterile, am răscumpărat frați de ai noștri, Iudei, care fuseseră vânduți păgânilor; iar voi vindeți pe frații voștri". Ei însă tăceau și nu găseau răspuns.
\par 9 "Nu faceți bine - am zis eu mai departe. Nu vi se cuvine oare să umblați în frica lui Dumnezeu, ca să înlăturați ocara neamurilor vrăjmașe nouă?
\par 10 Și eu, frații mei și slujitorii mei, le-am dat împrumut și argint și pâine, dar n-am mai cerut înapoi.
\par 11 Întoarceți-le dar chiar acum ogoarele, viile, măslinii și casele lor și dobânda argintului, pâinii, vinului și untdelemnului, pentru care le-ați dat împrumut".
\par 12 "Se vor întoarce și nu vom mai cere de la ei - ziseră ei. Vom face așa cum zici tu". Atunci am chemat preoții și le-am poruncit să le ia jurământ că vor face așa.
\par 13 Iar eu mi-am scuturat haina și am zis: "Așa să scuture Dumnezeu și de casă și de avere pe fiecare om care nu-și va ține cuvântul său și așa să fie el scuturat și gol. "Amin" - zise toată adunarea. Atunci am preaslăvit pe Dumnezeu și poporul și-a împlinit cuvântul.
\par 14 și din ziua aceea în care eu am fost numit guvernator al lor în țara lui Iuda, din anul al douăzecilea până în anul al treizeci și doilea al regelui Artaxerxe, timp de doisprezece ani eu și frații mei n-am mâncat pâinea de conducător.
\par 15 Iar conducătorii de mai înainte, apăsau poporul și luau de la el pâine și vin, pe lângă cei patruzeci de sicli de argint; ba până și slugile lor apăsau poporul. Eu însă n-am făcut așa, pentru că mă temeam de Dumnezeu.
\par 16 în vremea aceasta eu am mers înainte cu lucrul la zidurile acestea și nici o țarină n-am cumpărat și toți slujbașii mei se adunau acolo la lucru.
\par 17 Eu aveam la masa mea până la o sută cincizeci de oameni, Iudei și căpetenii, afară de cei ce veneau la noi de pe la popoarele dimprejur.
\par 18 Și iată ce se gătea în fiecare zi: un bou, șase oi îngrășate și păsări; și la fiecare zece zile se consuma mulțime de vin de tot felul. Cu toate acestea pâinea de conducător n-am cerut-o, deoarece poporul acesta avea de dus o muncă grea.
\par 19 Dumnezeu să primească tot ce am făcut eu spre binele poporului acestuia".

\chapter{6}

\par 1 Canaturile porților încă nu le pusesem, când a auzit Sanbalat, Tobie, Gheșem Arabul și ceilalți dușmani ai noștri că eu am reparat zidul și nu a mai rămas nici o spărtură în el.
\par 2 Atunci au trimis Sanbalat și Gheșem la mine să mi se spună: "Vino să ne întâlnim împreună intr-unul din satele din Valea Ono!" Căci ei puseseră la cale să-mi facă rău.
\par 3 Dar eu am trimis la ei robi să le spună: "Am de făcut un lucru mare și nu pot să vin, căci lucrul ar înceta, dacă l-aș lăsa și aș veni la voi".
\par 4 De patru ori au trimis ei la mine cu aceeași poftire și eu le-am răspuns la fel.
\par 5 Atunci a trimis la mine Sanbalat a cincea oară pe slujitorul său care avea în mâna sa o scrisoare deschisă.
\par 6 În ea se scria: "Umblă zvonul printre popoare și spune și Gheșem, cum că tu și Iudeii v-ați gândit să vă răsculați, pentru care scop și zidești zidul și vrei să fii regele lor, după aceleași zvonuri;
\par 7 Și ai fi pus și prooroci, ca să te vestească în Ierusalim și să-ți zică regele Iudei, și astfel de zvonuri vor ajunge până la rege. Vino deci să ne sfătuim împreună".
\par 8 Eu însă am trimis să i se spună: "Nimic de felul celor ce spui tu n-a fost. Acestea le-ai născocit tu din capul tău".
\par 9 Căci toți ne înfricoșează socotind că ne vom lăsa de lucru și zidul nu se va mai face. Dar eu încă și mai mult mă încurajam la lucru și ziceam: "Întărește-mă și acum, Dumnezeule!"
\par 10 Și am mers eu în casa lui Șemaia, fiul lui Delaia, fiul lui Mehetabeel, și el, încuindu-se, a zis: "Să ne adunăm în templul lui Dumnezeu, în mijlocul lui, și să încuiem ușile, căci au să vină să te ucidă, și au să vină noaptea să te ucidă".
\par 11 "Se poate oare să fugă un astfel de om, ca mine? - am răspuns eu. Se poate oare ca unul ca mine să intre în templu, pentru a rămâne cu viață? Nu merg! "
\par 12 Și am cunoscut că nu Dumnezeu îl trimisese, ci el grăia ca un prooroc împotriva mea, pentru că îl cumpăraseră Tobie și Sanbalat.
\par 13 Și fusese cumpărat ca eu să mă sperii și să fac așa Și să greșesc, ca apoi să aibă lumea părere urâtă despre mine, și să fiu prigonit pentru această scădere.
\par 14 Pomenește, Dumnezeul meu, pe Tobie și pe Sanbalat după aceste fapte ale lor; de asemenea și pe proorocița Noadia și pe ceilalți prooroci care au vrut să mă sperie!
\par 15 Zidul a fost isprăvit în ziua a douăzeci și cincea a lunii Elul, adică în cincizeci și două de zile.
\par 16 Când au auzit de aceasta toți dușmanii noștri și au văzut aceasta toate popoarele cele dimprejurul nostru, s-au smerit foarte mult și au cunoscut că acest lucru a fost făcut de Dumnezeul nostru.
\par 17 Pe lângă aceasta, în zilele acelea, cei mai de seamă ai Iudei au scris multe scrisori, pe care le-au trimis la Tobie, iar ei primeau scrisori de-ale lui Tobie;
\par 18 Căci mulți din Iuda erau cu el în legătură întărită prin jurământ, căci el era ginerele lui Șecania, fiul lui Arah, iar fiul său Iohanan luase de nevastă pe fiica lui Meșulam, fiul lui Berechia.
\par 19 Ba îmi vorbeau ei și de bunătatea lui și vorbele mele ajungeau la el. Și Tobie trimitea scrisori, ca să mă sperie.

\chapter{7}

\par 1 După ce s-a terminat zidul și am pus canaturile și au fost așezați la slujba lor portarii și cântăreții și leviții,
\par 2 Atunci am poruncit fratelui meu Hanani și lui Anania, căpetenia cetății Ierusalimului, căci el cu mult mai mult decât alții era om credincios și temător de Dumnezeu,
\par 3 Și le-am zis: "Să nu se deschidă porțile Ierusalimului până nu răsare soarele, și să le închidă și să le încuie în fața voastră". Și am pus străjeri pe locuitorii Ierusalimului, fiecare când îi venea rândul și fiecare în fața casei sale.
\par 4 Cetatea însă era întinsă și mare, iar popor era puțin în ea și case nu se zidiseră din nou.
\par 5 Și mi-a dat Dumnezeul meu gând să aduc pe cei mai mari, pe căpetenii și poporul, ca să fac o numărătoare, după spițele neamului lor, și am găsit cartea cu spițele de neam ale celor ce veniseră întâia oară, și într-însa erau scrise următoarele:
\par 6 Iată locuitorii țării care au plecat din robie, unde-i dusese Nabucodonosor, regele Babilonului, și s-au întors la Ierusalim și în Iuda, așezându-se fiecare în cetatea sa,
\par 7 Împreună venind cu Zorobabel: Iosua, Neemia, Azaria, Raamia, Nahamani, Mardoheu, Bilșan, Misperet, Bigvai, Nehum și Baana. Numărul poporului lui Israel:
\par 8 Fiii lui Fares, două mii o sută șaptezeci și doi;
\par 9 Fiii lui Șefatia, trei sute șaptezeci și doi;
\par 10 Fiii lui Arah, șase sute cincizeci și doi;
\par 11 Fiii lui Pahat-Moab, din urmașii lui Iosua și Ioab, două mii opt sute optsprezece;
\par 12 Fiii lui Elam, o mie două sute cincizeci și patru;
\par 13 Fiii lui Zatu, opt sute patruzeci și cinci;
\par 14 Fiii lui Zacai, șapte sute șaizeci;
\par 15 Fiii lui Binui, șase sute patruzeci și opt;
\par 16 Fiii lui Bebai, șase sute douăzeci și opt;
\par 17 Fiii lui Azgad, două mii trei sute douăzeci și doi;
\par 18 Fiii lui Adonicam, șase sute șaizeci și șapte;
\par 19 Fiii lui Bigvai, două mii șaizeci și șapte;
\par 20 Fiii lui Adin, șase sute cincizeci și cinci;
\par 21 Fiii lui Ater, din familia lui Iezechia, nouăzeci și opt;
\par 22 Fiii lui Hașum, trei sute douăzeci și opt;
\par 23 Fiii lui Bețai, trei sute douăzeci și patru;
\par 24 Fiii lui Harif, o sută doisprezece;
\par 25 Ghibeoniții, nouăzeci și cinci;
\par 26 Oameni din Betleem și Netofa, o sută optzeci și opt;
\par 27 Oameni din Anatot, o sută douăzeci și opt;
\par 28 Oameni din Bet-Azmavet, patruzeci și doi;
\par 29 Oameni din Chiriat-Iearim, din Chefira și din Beerot, șapte sute patruzeci și trei;
\par 30 Oameni din Rama și din Gheba, șase sute douăzeci și unu;
\par 31 Oameni din Micmas, o sută douăzeci și doi;
\par 32 Oameni din Betel și din Ai, o sută douăzeci și trei;
\par 33 Oameni din celălalt Nebo, cincizeci și doi;
\par 34 Fiii celuilalt Elam, o mie două sute cincizeci și patru;
\par 35 Fiii lui Harim, trei sute douăzeci;
\par 36 Ierihoneni, trei sute patruzeci și cinci;
\par 37 Lodieni, Hadidieni și oameni din Ono, șapte sute douăzeci și unu.
\par 38 Fiii lui Senaa, trei mii nouă sute treizeci.
\par 39 Preoții: fiii lui Iedaia, din casa lui Iosif, nouă sute șaptezeci și trei;
\par 40 Fiii lui Imer, o mie cincizeci și doi;
\par 41 Fiii lui Pașhur, o mie două sute patruzeci și șapte;
\par 42 Fiii lui Harim, o mie șaptesprezece;
\par 43 Leviții: fiii lui Iosua și ai lui Cadmiel, din casa lui Hodavia, șaptezeci și patru.
\par 44 Cântăreții: fiii lui Asaf, o sută patruzeci și opt.
\par 45 Portarii: fiii lui Șalum, fiii lui Ater, fiii lui Talmon, fiii lui Acuv, fiii lui Hatita, fiii lui Șobai, o sută treizeci și opt.
\par 46 Cei încredințați templului: fiii lui Țiha, fiii lui Hasufa, fiii lui Tabaot,
\par 47 Fiii lui Cheros, fiii lui Sia, fiii lui Padon,
\par 48 Fiii lui Lebana, fiii lui Hagaba, fiii lui Șalmai,
\par 49 Fiii lui Hanan, fiii lui Ghidel, fiii lui Gahar,
\par 50 Fiii lui Reaia, fiii lui Rețin, fiii lui Necoda,
\par 51 Fiii lui Gazam, fiii lui Uza, fiii lui Paseah,
\par 52 Fiii lui Besai, fiii lui Meunim, fiii lui Nefișim,
\par 53 Fiii lui Bacbuc, fiii lui Hacufa, fiii lui Harhur,
\par 54 Fiii lui Bațlit, fiii lui Mehida, fiii lui Harșa,
\par 55 Fiii lui Barcos, fiii lui Sisera, fiii lui Tamah,
\par 56 Fiii lui Nețiah, fiii lui Hatifa.
\par 57 Fiii robilor lui Solomon: fiii lui Sotai, fiii lui Soferet, fiii lui Perida;
\par 58 Fiii lui Iaala, fiii lui Darcon, fiii lui Ghidel;
\par 59 Fiii lui Șefatia, fiii lui Hatil, fiii lui Pocheret-Hațebaim, fiii lui Amon.
\par 60 Netineii și fiii robilor lui Solomon în total erau trei sute nouăzeci și doi.
\par 61 Iată acum și cei ce au plecat din Tel-Melah, din Tel-Harșa, din Cherub-Adon și din Imer, și care n-au putut să-și arate nici casa din care se trag, nici seminția, pentru a dovedi că sunt Israeliți:
\par 62 Fiii lui Delaia, fiii lui Tobie, fiii lui Necoda, șase sute patruzeci și doi.
\par 63 Și dintre preoți: fiii lui Hobaia, fiii lui Hacoț, fiii lui Barzilai, care a luat de soție una din fiicele lui Barzilai Galaaditul și a primit numele acesteia.
\par 64 Ei și-au căutat spița neamului lor, dar nu și-au găsit-o, și au fost scoși din preoție;
\par 65 Și guvernatorul le-a zis să nu mănânce din lucrurile de mare sfințenie până nu se va ridica un preot cu Urim și Tumim.
\par 66 Adunarea întreagă a fost de patruzeci și două de mii trei sute șaizeci de suflete,
\par 67 Afară de robi și roabe, care erau în număr de șapte mii trei sute treizeci și șapte. Printre ei se mai aflau două sute patruzeci și cinci de cântăreți și cântărețe.
\par 68 Ei aveau șapte sute treizeci și șase de cai, două sute patruzeci și cinci de catâri,
\par 69 Patru sute treizeci și cinci cămile și șase mii șapte sute douăzeci asini.
\par 70 Unii capi de familie au făcut daruri pentru lucru. Tirșata a dat la vistierie o mie de drahme de aur, cincizeci de cupe și cinci sute treizeci veșminte preoțești.
\par 71 Căpeteniile caselor au dat în vistieria lucrărilor douăzeci de mii de drahme de aur și două mii două sute mine de argint. Celălalt popor a dat douăzeci de mii de drahme de aur și două mii de mine de argint și șaizeci și șapte de veșminte preoțești.
\par 72 Preoții și leviții, portarii, cântăreții și oamenii din popor, cei încredințați templului și tot Israelul s-au așezat în cetățile lor.
\par 73 Și când a sosit luna a șaptea, fiii lui Israel se aflau prin cetățile lor.

\chapter{8}

\par 1 Atunci s-a adunat tot poporul ca un singur om în piața cea din fața Porții Apelor și a zis lui Ezdra cărturarul să aducă el cartea legii lui Moise pe care a dat-o Domnul lui Israel.
\par 2 Și cărturarul Ezdra a adus legea înaintea adunării, care era alcătuită din bărbați și din femei și din toți acei ce erau în stare să înțeleagă. Aceasta s-a făcut în ziua întâi a lunii a șaptea.
\par 3 Și a citit Ezdra din carte de dimineață până la amiază în piața cea de dinaintea Porții Apelor, înaintea bărbaților și femeilor și a celor ce erau în stare să priceapă și tot poporul a fost cu luare-aminte la citirea cărții legii.
\par 4 Ezdra cărturarul stătea sus pe un podeț de lemn, făcut anume pentru acest prilej. Alături de el, la dreapta lui, stăteau Matitia, Șema, Anaia, Urie, Hilchia și Maaseia; iar la stânga lui se aflau Pedaia, Misael, Malchia, Hașum, Hașbadana, Zaharia și Meșulam.
\par 5 Și a deschis Ezdra cartea înaintea ochilor a tot poporul, căci el se afla mai sus decât tot poporul și când a deschis el cartea, tot poporul stătea în picioare în piață.
\par 6 Mai întâi Ezdra a binecuvântat pe Domnul, marele Dumnezeu și tot poporul, ridicându-și mâinile, a răspuns: Amin, Amin! Și plecându-se, s-a închinat înaintea Domnului până la pământ.
\par 7 Apoi Iosua, Bani, Șerebia, Iamin, Acub, Șabetai, Hodia, Maaseia, Chelita, Azaria, Iozabad, Hanan, Pelaia, și leviții au tâlcuit poporului legea, stând fiecare la locul lui.
\par 8 Ei citeau lămurit bucăți din cartea legii lui Dumnezeu, iar bucățile citite le lămureau și poporul înțelegea cele ce se citeau.
\par 9 Atunci Neemia, guvernatorul, Ezdra preotul și cărturarul, și leviții care învățau poporul au zis către tot poporul: "Ziua aceasta este închinată Domnului Dumnezeului vostru, să nu fiți triști, nici să plângeți! Căci tot poporul plângea, auzind cuvintele legii.
\par 10 Ci mergeți de mâncați carne grasă și beți vin dulce - adăugară ei - și trimiteți parte și celor ce n-au nimic gătit, căci ziua aceasta este sfințită Domnului nostru. Nu fiți triști, căci bucuria Domnului va fi puterea voastră".
\par 11 Și au liniștit leviții tot poporul, zicând: "Tăceți, căci ziua aceasta este sfântă. Nu fiți triști!"
\par 12 Și tot poporul s-a dus să mănânce și să bea și să trimită părți celor ce nu aveau și să facă veselie mare; căci înțeleseseră ei cuvintele ce li se tâlcuiseră.
\par 13 A doua zi, capii familiilor din tot poporul, preoții și leviții se strânseseră împrejurul lui Ezdra cărturarul, ca să asculte tălmăcirea cuvintelor legii.
\par 14 Și au găsit ei scris în legea pe care Domnul a dat-o lui Moise, că fiii lui Israel în timpul sărbătorilor din luna a șaptea trebuie să locuiască în corturi.
\par 15 Și au vestit aceasta în toate cetățile lor și în Ierusalim, zicând: "Duceți-vă de căutați, în munte, crengi de măslin sălbatic, de mirt, de finici și de tufari și faceți corturi cum este scris".
\par 16 Atunci poporul s-a dus să caute crengi și a făcut fiecare cort pe acoperișul casei lor, în curtea sa, în curtea templului Domnului și în piața de dinaintea Porții Apelor și în piața de dinaintea Porții lui Efraim.
\par 17 Toată obștea celor ce se întorseseră din robie și-a făcut corturi și a stat în ele. Din zilele lui Iosua, fiul lui Navi, până în ziua aceasta, fiii lui Israel nu mai fuseseră așa. Și a fost bucurie foarte mare.
\par 18 Și s-a citit din cartea legii Domnului în fiecare zi, din cea dintâi zi până la cea din urmă. Și au ținut sărbătoarea șapte zile iar a opta zi au făcut o adunare sărbătorească, după rânduială.

\chapter{9}

\par 1 În ziua de douăzeci și patru a acestei luni s-au adunat toți fiii lui Israel, îmbrăcați cu sac și cu capetele presărate cu cenușă, ca să postească.
\par 2 Și osebindu-se cei ce erau din neamul lui Israel de toți cei de alt neam, au venit de și-au mărturisit păcatele lor și fărădelegile părinților lor.
\par 3 Și după ce s-au așezat la locurile lor, li s-a citit din cartea legii Domnului Dumnezeului lor un sfert de zi, iar alt sfert de zi și-au mărturisit păcatele lor și s-au închinat Domnului Dumnezeului lor.
\par 4 Apoi Iosua, Binui, Cadmiel, Șebania, Buni, Șerebia, Baani și Chenani s-au urcat pe podețul leviților și au strigat cu glas mare către Domnul Dumnezeul lor;
\par 5 Și leviții Iosua, Cadmiel, Bani, Hașabneia, Șerebia, Hodia, Șebania și Petahia au zis: "Sculați-vă și binecuvântați pe Domnul Dumnezeul vostru din veac în veac!" "Dumnezeule - a zis Ezdra - slăvească-se numele Tău cel slăvit și mai presus de urice laudă și slăvire!
\par 6 Numai Tu ești Domn și numai Tu ai făcut cerurile, cerurile cerurilor și toată oștirea lor, pământul și toate cele de pe el, mările și toate cele ce se cuprind în ele; Tu dai viață la toate și ție se închină oștirea cerurilor.
\par 7 Tu, Doamne Dumnezeule ai ales pe Avram, l-ai scos din Urul Caldeii și i-ai dat numele de Avraam.
\par 8 Tu ai găsit că inima lui e credincioasă înaintea Ta, Tu ai făcut legământ cu el și Tu ai făgăduit să dai urmașilor lui țara Canaanului, a Heteilor, a Amoreilor, a Ferezeilor, a Iebuseilor și a Ghergheseilor; și Tu Ți-ai ținut cuvântul, pentru că Tu ești drept.
\par 9 Tu ai văzut necazul părinților noștri în Egipt și ai auzit strigătele lor la Marea Roșie.
\par 10 Tu ai făcut semne și minuni înaintea lui Faraon, împotriva tuturor slugilor lui și împotriva întregului popor din țara lui, pentru că Tu ai văzut cu câtă răutate s-au purtat ei cu părinții noștri și ți-ai făcut nume până în ziua de astăzi.
\par 11 Tu ai despărțit marea înaintea părinților noștri și au trecut prin mijlocul mării ca pe uscat, dar i-ai cufundat în adânc, cum se cufundă o piatră în apă, pe cei ce-i urmăreau.
\par 12 Tu i-ai povățuit ziua cu stâlp de nor și noaptea cu stâlp de foc și le-ai luminat calea pe care aveau să meargă.
\par 13 Tu Te-ai pogorât pe muntele Sinai, le-ai grăit din înălțimea cerurilor și le-ai dat porunci drepte; legi ale adevărului, învățături și orânduiri minunate.
\par 14 Tu le-ai arătat odihna Ta cea sfântă și le-ai scris prin Moise, sluga Ta, porunci, rânduieli și lege.
\par 15 Tu le-ai dat din înălțimea cerului pâine, când au flămânzit și le-ai scos apă din piatră, când au însetat, și le-ai zis să intre și să moștenească țara pe care cu jurământ le-ai făgăduit-o.
\par 16 Dar părinții noștri s-au îndărătnicit și și-au învârtoșat cerbicia lor; n-au ascultat poruncile Tale, nici s-au supus și au uitat minunile pe care le-ai făcut pentru ei.
\par 17 Învârtoșatu-și-au cerbicia lor și în răzvrătirea lor și-au ales o căpetenie, ca să se întoarcă în robia lor; dar Tu, fiind Dumnezeu iubitor și iertător, negrabnic la mânie și bogat în milă și îndurare, nu i-ai părăsit.
\par 18 Chiar când și-au făcut un vițel turnat și au zis: "Iată dumnezeul tău care te-a scos din Egipt!" și s-au dedat la hule mari împotriva Ta,
\par 19 În nemărginita Ta îndurare, nu i-ai părăsit în pustiu și stâlpul de nor n-a încetat să-i călăuzească ziua în calea lor, nici stâlpul de foc să le lumineze noaptea drumul ce făceau.
\par 20 Trimisu-le-ai Duhul Tău cel bun, ca să-i înțelepțească; nu ai lipsit gura lor de mana Ta și în setea lor le-ai dat apă.
\par 21 Timp de patruzeci de ani i-ai hrănit în pustie și nimic nu le-a lipsit; hainele lor nu s-au învechit, nici încălțămintele lor nu s-au rupt.
\par 22 Le-ai dat regate și popoare le-ai împărțit și au pus stăpânire pe țara lui Sihon, regele Heșbonului și pe pământul lui Og, regele Vasanului.
\par 23 Tu ai înmulțit pe fiii lui ca stelele cerului și i-ai dus în țara de care ai spus părinților lor că o vor moșteni.
\par 24 Și fiii lor au intrat și au moștenit țara aceea; ai supus înaintea lor pe locuitorii pământului aceluia, pe Canaanei și i-ai dat pe aceștia în mâna lor cu regele și cu tot poporul băștinaș, ca să le facă ce vor vrea.
\par 25 Și s-au făcut ei stăpâni peste cetăți tari și peste pământul roditor, peste case pline de toate bunătățile, fântâni săpate în piatră, vii, măslini și pomi roditori din belșug; și au mâncat și s-au săturat și s-au îngrășat trăind în desfătări prin bunătatea Ta.
\par 26 Dar ei s-au ridicat și s-au răzvrătit împotriva Ta; au aruncat legea Ta la spate; pe proorocii Tăi care-i îndemnau să se întoarcă la Tine i-au ucis; și ți-au adus hule mari.
\par 27 Atunci Tu i-ai dat în mâinile vrăjmașilor lor, care i-au apăsat. Dar în vremea necazului lor au strigat către Tine și Tu i-ai auzit din înălțimea cerurilor și, în mila Ta cea mare, le-ai trimis izbăvitori ca să-i izbăvească din mâinile vrăjmașilor lor.
\par 28 Iar dacă s-au odihnit, au început iar să facă rău înaintea Ta. Atunci Tu i-ai dat din nou în mâna vrăjmașilor lor ca să-i stăpânească. Și ei din nou au strigat către Tine și Tu i-ai auzit din înălțimea cerurilor și, în mila Ta cea mare, i-ai izbăvit de multe ori.
\par 29 I-ai povățuit să se întoarcă la legea Ta, dar ei s-au îndărătnicit și n-au ascultat poruncile Tale, ci au păcătuit împotriva poruncilor care dau viață celui ce le împlinește și și-au îndârjit spinarea lor și cerbicia lor și-au învârtoșat-o și nu s-au supus.
\par 30 Tu însă, așteptând întoarcerea lor, i-ai îngăduit mulți ani și le-ai deșteptat luarea-aminte prin Duhul Tău și prin proorocii Tăi, dar ei nu și-au plecat urechea. Atunci i-ai dat în mâna popoarelor străine.
\par 31 Dar, în mila Ta cea mare, nu i-ai stârpit, nici nu i-ai părăsit, căci Tu ești un Dumnezeu milos și îndurat.
\par 32 Și acum, Dumnezeul nostru, Dumnezeul cel mare, puternic și înfricoșător, Cel ce ții legământul Tău și faci milă, nu socoți ca puțin lucru suferințele ce am îndurat noi, regii noștri, căpeteniile noastre, preoții noștri, proorocii noștri, părinții noștri și tot poporul Tău din vremea regilor Asiriei până în ziua aceasta.
\par 33 Tu ești drept în toate câte ne-au ajuns, căci Tu ai fost credincios, iar noi am făcut rău.
\par 34 Regii noștri, căpeteniile noastre, preoții noștri și părinții noștri n-au păzit legea Ta și n-au luat aminte nici la poruncile Tale, nici la îndemnurile ce le-ai dat.
\par 35 Cât au fost în regatul lor în mijlocul binefacerilor Tale, într-o țară largă și roditoare pe care le-ai dat-o Tu, ei nu ți-au slujit ție, nici nu s-au întors de la faptele lor cele urâte.
\par 36 Și astăzi iată suntem robi și țara ce Tu ai dat-o părinților noștri, ca să se bucure de roadele ei și de bunătățile ei,
\par 37 Ea își înmulțește astăzi roadele pentru regii cărora ne-ai supus pentru păcatele noastre. Ei domnesc peste noi și peste vitele noastre, după bunul lor plac și noi ne aflăm în necaz mare.
\par 38 Pentru toate acestea facem legământ pe care și noi înșine cu iscălitură, și căpeteniile, leviții și preoții noștri îl întăresc cu pecetea".

\chapter{10}

\par 1 Iată acum și cei care și-au pus pecetea lor: Neemia guvernatorul, fiul lui Hacalia;
\par 2 Preoții: Sedechia, Seraia, Azaria, Ieremia,
\par 3 Pașhur, Amaria, Malchia,
\par 4 Hatuș, Șebania, Maluc,
\par 5 Harim, Meremot, Obadia,
\par 6 Daniel, Ghineton, Baruc,
\par 7 Meșulam, Abia, Miamin,
\par 8 Maazia, Bilgai, Șemaia.
\par 9 Leviții: Iosua, fiul lui Azania, Binui, din fiii lui Henadad, Cadmiel;
\par 10 Și frații lor: Șebania, Hodia, Chelita, Pelaia, Hanan,
\par 11 Mica, Rehob, Hașabia,
\par 12 Zacur, Șerebia, Șebania,
\par 13 Hodia, Bani, Beninu.
\par 14 Căpeteniile poporului: Paroș, Pahat-Moab, Elam, Zatu, Bani,
\par 15 Buni, Azgad, Bebai,
\par 16 Adonia, Bigvai, Adin,
\par 17 Ater, Iezechia, Azur,
\par 18 Hodia, Hașum, Bețai,
\par 19 Harif, Anatot, Nebai,
\par 20 Magpiaș, Meșulam, Hezir,
\par 21 Meșezabeel, Țadoc, Iadua,
\par 22 Pelatia, Hanan, Anaia,
\par 23 Hozea, Hanania, Hașub,
\par 24 Haloheș, Pileha, Șobec,
\par 25 Rehum, Hașabna, Maaseia,
\par 26 Ahia, Hanan, Anan,
\par 27 Maluc, Harim, Baana.
\par 28 Celălalt popor: preoți, leviți, portari, cântăreți, cei încredințați templului și toți cei ce s-au despărțit de popoarele străine, ca să urmeze legea lui Dumnezeu, femeile lor, băieții lor și fetele lor, toți cei ce erau în stare să priceapă și să înțeleagă
\par 29 S-au unit cu frații și cu căpeteniile lor și au făgăduit cu blestem și jurământ și au jurat să se poarte după legea lui Dumnezeu, dată prin Moise, sluga lui Dumnezeu, să păzească și să împlinească toate poruncile Domnului, Stăpânul nostru, hotărârile și legile Lui.
\par 30 Am făgăduit să nu dăm fetele noastre popoarelor țării, nici să luăm fetele lor pentru feciorii noștri;
\par 31 Să nu cumpărăm nimic în ziua de odihnă, în zilele de sărbătoare de la popoarele din țară, când ne vor aduce în acele zile de vânzare mărfuri sau orice bucate; iar în anul al șaptelea să iertăm toate datoriile.
\par 32 Am luat de asemenea îndatorirea să dăm a treia parte de siclu, pe an, pentru slujbele templului Dumnezeului nostru.
\par 33 Pentru pâinile punerii înainte, pentru jertfa cea necontenită, pentru arderile de tot necontenite, care se fac în zile de odihnă, la lună nouă și la sărbători, pentru lucrurile sfinte, pentru jertfa de iertare, spre curățirea lui Israel, și pentru tot ce se săvârșește în templul Dumnezeului nostru.
\par 34 Apoi au tras la sorti preoții, leviții și poporul, după casele noastre părintești, ca să se știe cine și când trebuie să aducă în fiecare an, la anumită vreme, lemne în templul Dumnezeului nostru, spre a se arde pe altarul Domnului Dumnezeului nostru, după cum este scris în lege.
\par 35 Am mai hotărât să mai ducem în fiecare an la templul Domnului pârga din roadele pământului nostru și pârga din roadele tuturor pomilor;
\par 36 Să aducem la templul Dumnezeului nostru, preoților care fac slujbele în templul Dumnezeului nostru, pe întâiul născut dintre fiii noștri și dintre dobitoacele noastre, cum este scris în lege, și pe întâiul-născut al vitelor noastre mari și mărunte;
\par 37 Să aducem la preoți, în vistieria templului Dumnezeului nostru, pârga din aluatul nostru și ofrandele noastre de fructe din tot pomul, din vin și untdelemn, și să dăm dijmă pentru pământul nostru leviților, care au dreptul să o ia în toate cetățile de pe pământurile lucrate de noi.
\par 38 Un preot, fiu al lui Aaron, va fi cu leviții, când aceștia vor lua zeciuiala; și din zeciuiala luată, leviții vor duce zeciuială în templul Dumnezeului nostru, în odăile vistieriei.
\par 39 Căci în aceste odăi atât fiii lui Israel, cât și fiii leviților trebuie să aducă prinoase de pâine, de vin și de untdelemn. Acolo sunt casele sfinte și stau preoții care slujesc și portarii și cântăreții. Astfel noi nu vom lăsa în părăsire templul Dumnezeului nostru.

\chapter{11}

\par 1 Căpeteniile poporului s-au așezat în Ierusalim, iar celălalt popor a tras sorți, ca a zecea parte din ei să locuiască în sfânta cetate a Ierusalimului, iar celelalte nouă părți să rămână în celelalte cetăți.
\par 2 Și a binecuvântat poporul pe toți cei ce s-au învoit de bună voie să se așeze în Ierusalim.
\par 3 Iată căpeteniile țării care s-au așezat în Ierusalim și în cetățile lui Iuda, locuind fiecare pe moșia sa, în cetatea sa: Israeliții, preoții și leviții, cei încredințați templului și urmașii slujitorilor lui Solomon.
\par 4 În Ierusalim s-au așezat din fiii lui Iuda și din fiii lui Veniamin. Din fiii lui Iuda s-au așezat: Ataia, fiul lui Uzia, fiul lui Zaharia, fiul lui Amaria, fiul lui Șefatia, fiul lui Mahalaleel din fiii lui Fares,
\par 5 Și Maaseia, fiul lui Baruc, fiul lui Col-Hoze, fiul lui Hazaia, fiul lui Adaia, fiul lui Ioiarib, fiul lui Zaharia, fiul lui Șiloni.
\par 6 Fiii lui Fares care s-au așezat în Ierusalim erau în total patru sute șaizeci și opt de oameni de oaste.
\par 7 Iată acum fiii lui Veniamin: Salu, fiul lui Meșulam, fiul lui Ioed, fiul lui Pedaia, fiul lui Colaia, fiul lui Maaseia, fiul lui Itiel, fiul lui Isaia,
\par 8 Apoi Gabai și Șalai, nouă sute douăzeci și opt.
\par 9 Ioil, fiul lui Zicri, era căpetenia lor, iar Iuda, fiul lui Senua, era a doua căpetenie a cetății.
\par 10 Din preoți: Iedaia, fiul lui Ioiarib, și Iachin,
\par 11 Seraia, fiul lui Hilchia, fiul lui Meșulam, fiul lui Țadoc, fiul lui Meraiot, fiul lui Ahitub, căpetenia templului Domnului;
\par 12 Și frații lor, care se îndeletnicesc cu slujba în templul Domnului, opt sute douăzeci și doi; Adaia, fiul lui Ieroham, fiul lui Pelalia, fiul lui Amți, fiul lui Zaharia, fiul lui Pașhur, fiul lui Malchia,
\par 13 Și frații săi, căpeteniile caselor părintești, două sute patruzeci și doi; și Amasai, fiul lui Azareel, fiul lui Ahzai, fiul lui Meșilemot, fiul lui Imer,
\par 14 Și frații lor, bărbați de oaste, o sută douăzeci și opt; Zabdiel, fiul lui Ghedolim, era căpetenia lor.
\par 15 Dintre leviți: Șemaia, fiul lui Hașub, fiul lui Azricam, fiul lui Hașabia, fiul lui Buni,
\par 16 Șabetai și Iozabad, căpetenii ale leviților pentru treburile din afară ale templului lui Dumnezeu;
\par 17 Matania, fiul lui Mica, fiul lui Zabdi, fiul lui Asaf, căpetenia care începea laudele în timpul rugăciunii, și Bacbuchia, al doilea între frații săi și Abda, fiul lui Șamua, fiul lui Galal, fiul lui Iedutun.
\par 18 Toți leviții în cetatea sfântă erau două sute optzeci și patru.
\par 19 Portarii: Acub, Talmon și frații lor, păzitorii porților, o sută șaptezeci și doi.
\par 20 Ceilalți Israeliți, preoți și leviți, se așezară în toate cetățile lui Iuda, fiecare la moșia sa.
\par 21 Cei încredințați templului se așezară pe colină și aveau de căpetenie pe Țiha și Ghișpa.
\par 22 Căpetenia leviților la Ierusalim era Uzi, fiul lui Bani, fiul lui Hașabia, fiul lui Matania, fiul lui Mica, dintre fiii lui Asaf, cântăreți însărcinați să facă slujbă la templul lui Dumnezeu;
\par 23 Căci pentru cântăreli era o poruncă din partea regelui, prin care li se hotăra o plată anumită pe fiecare zi.
\par 24 Petahia, fiul lui Meșezabeel din fiii lui Zerah, fiul lui Iuda, era împuternicit de rege pentru toate nevoile poporului.
\par 25 Iar dintre cei ce trăiesc în sate pe țarinile lor, fii de ai lui Iuda, s-au așezat la Kiriat-Arba și în satele ce țineau de el; la Dibon și în satele ce țineau de el; la Iecabțeel și în satele lui;
\par 26 La Ieșua, la Molada, la Bet-Pelet;
\par 27 La Hațar-Șual, la Beer-Șeba și în satele ce țineau de ea;
\par 28 La Țiclag, la Mecona și în satele ce țineau de ea;
\par 29 La En-Rimon, la Țorea, la Iarmut;
\par 30 La Zanoah, la Adulam și în satele ce țineau de el; la Lachiș și în țarinile ce țineau de el; la Azeca și în satele ce țineau de el; s-au așezat ei de la Beer-Șeba până la Ghe-Hinom.
\par 31 Fiii lui Veniamin s-au așezat, începând de la Gheba: în Micmas, în Aia, în Betel și în satele lui;
\par 32 În Anatot, în Nob, în Anania,
\par 33 În Hațor, în Rama, în Ghitaim,
\par 34 În Hadid, în Țeboim, în Nebalat,
\par 35 În Lod, în Ono și în Gheharașim.
\par 36 Erau și leviți care aveau țarini în Veniamin, cu toate că erau socotiți în partea lui Iuda.

\chapter{12}

\par 1 Iată preoții și leviții care au venit cu Zorobabel, fiul lui Salatiel, și cu Iosua: Seraia, Ieremia, Ezdra,
\par 2 Amaria, Maluc, Hatuș,
\par 3 Șecania, Rehum, Meremot,
\par 4 Ido, Ghineton, Abia,
\par 5 Miiamin, Moadia, Bilga,
\par 6 Șemaia, Ioiarib, Iedaia,
\par 7 Salu, Amoc, Hilchia, Iedaia. Aceștia au fost căpeteniile preoților și ale fraților lor în vremea lui Iosua.
\par 8 Leviții au fost: Iosua, Binui, Cadmiel, Șerebia, Iuda, Matania, care împreună cu frații săi conducea pe cântăreli în timpul slujbelor de laudă;
\par 9 Bacbuchia și Uni, care împreună cu frații lor cântau de cealaltă parte.
\par 10 Iosua a născut pe Ioachim, Ioachim a născut pe Eliașib, Eliașib a născut pe Ioiada;
\par 11 Ioiada a născut pe Iohanan, Iohanan a născut pe Iadua.
\par 12 În zilele lui Ioachim, preoți, căpetenii de familii, erau: Meraia pentru casa lui Seraia; Hanania pentru casa lui Ieremia;
\par 13 Meșulam pentru casa lui Ezdra; Iohanan pentru casa lui Amaria;
\par 14 Ionatan pentru casa lui Maluc, Iosif pentru casa lui Șecania;
\par 15 Adna pentru casa lui Harim; Helcai pentru casa lui Meremot;
\par 16 Zaharia pentru casa lui Ido; Meșulam pentru casa lui Ghineton;
\par 17 Zicri pentru casa lui Abia; Piltai pentru casa lui Miiamin și Moadia;
\par 18 Șamua pentru casa lui Bilga; Ionatan pentru casa lui Șemaia;
\par 19 Matanai pentru casa lui Ioiarib; Uzi pentru casa lui Iedaia;
\par 20 Calai pentru casa lui Salai; Eber pentru casa lui Amoc;
\par 21 Hașabia pentru casa lui Hilchia; Natanael pentru casa lui Iedaia.
\par 22 În zilele lui Eliașib, ale lui Ioiada, ale lui Iohanan și ale lui Iadua, leviții, căpetenii de case, au fost înscriși cu preoții până la domnia lui Darie Persanul;
\par 23 Fiii lui Levi, căpetenii de familie, au fost înscriși în cartea Cronicilor până în timpul lui Iohanan, nepot de fiu al lui Eliașib;
\par 24 Căpeteniile leviților erau deci: Hașabia, Șerebia și Iosua, fiul lui Cadmiel; iar frații lor alcătuiau ceata a doua pentru a lăuda și a slăvi numele lui Dumnezeu, după rânduiala lui David, omul lui Dumnezeu, răspunzând o ceată celeilalte.
\par 25 Matania, Bacbuchia și Obadia, Meșulam, Talmon și Acub aveau slujba de portari și străjuiau când se aduna poporul înaintea porții.
\par 26 Aceștia au trăit pe vremea lui Ioachim, fiul lui Iosua, fiul lui Ioțadac și pe timpul lui Neemia guvernatorul și al preotului și cărturarului Ezdra.
\par 27 La sfințirea zidului Ierusalimului încă s-au chemat leviții din toate locurile, poruncindu-li-se să vină la Ierusalim pentru săvârșirea sfințirii și pentru sărbătoarea veselă cu doxologii și cântări, în sunetul chimvalelor, lirelor și harfelor.
\par 28 Cei din neamul cântărelilor s-au adunat din împrejurimile Ierusalimului și din cetățile netofatiene,
\par 29 Din Bet-Ghilgal, din șesurile Ghebei, din Azmavet, căci cântăreții își făcuseră satele în apropiere de Ierusalim.
\par 30 Și preoții și leviții s-au curățit și au sfințit poporul, porțile și zidurile.
\par 31 Și am suit eu căpeteniile lui Iuda pe ziduri și am pus două coruri mari pentru cântarea de laudă; unul din ele s-a așezat în partea dreaptă a zidului, spre poarta Gunoiului.
\par 32 În urma lor mergeau Hoșaia și jumătate din căpeteniile lui Iuda:
\par 33 Azaria, Ezdra și Meșulam;
\par 34 Iuda, Veniamin, Șemaia și Ieremia;
\par 35 Iar dintre fiii preoților mergeau cu trâmbițe: Zaharia, fiul lui Ionatan, fiul lui Șemaia, fiul lui Matania, fiul lui Mica, fiul lui Zacur, fiul lui Asaf,
\par 36 Și frații lui: Șemaia, Azareel, Milalai, Ghilalai, Maai, Natanael, Iuda și Hanani cu instrumentele muzicale ale lui David, omul lui Dumnezeu, iar înaintea lor se afla cărturarul Ezdra.
\par 37 La Poarta Izvorului, în fața lor, s-au urcat pe treptele scării de la cetatea lui David, care ducea pe zidul de deasupra casei lui David, până la Poarta Apelor, spre răsărit.
\par 38 Al doilea cor s-a îndreptat în cealaltă parte, și după el am mers eu și jumătate din popor și am înaintat pe zid de la turnul Cuptoarelor și până la zidul cel lat,
\par 39 Și de la Poarta Efraim, am trecut pe lângă poarta veche și Poarta Peștilor, pe lângă turnul lui Hananeel, pe lângă turnul Mea spre Poarta Oilor și ne-am oprit la Poarta închisorii.
\par 40 Apoi amândouă corurile au mers la templul lui Dumnezeu. Și am făcut același lucru și noi, eu și căpeteniile care erau cu mine
\par 41 Și preoții: Eliachim, Maaseia, Miniamin, Mica, Elioenai, Zaharia, Hanania cu trâmbițe,
\par 42 Maaseia, Șemaia, Eleazar, Uzi, Iohanan, Malchia, Elam și Ezer. Cântăreții cântau sub conducerea lui Izrahia.
\par 43 Și s-au adus în ziua aceea mulțime de jertfe și s-au veselit, căci Dumnezeu dăduse poporului pricină mare de bucurie. Și femeile și copiii s-au veselit, iar strigătele lor de bucurie din Ierusalim se auzeau până departe.
\par 44 În aceeași zi au fost puși oameni supraveghetori la cămările unde aveau să se aducă prinoasele, pârga și zeciuielile, și au fost împuterniciți să adune din țarinile din jurul cetăților părțile hotărâte prin lege pentru preoți și leviți. Și Iudeii s-au bucurat când au văzut pe preoți și pe leviți la slujba lor,
\par 45 Îndeplinind toate cele pentru slujba dumnezeiască și pentru curățire. Cântăreții și portarii și-au îndeplinit datoria lor după porunca lui David și a lui Solomon, fiul lui;
\par 46 Încă de demult, din zilele lui David și Asaf se hotărâseră căpeteniile cântăreților și cântările cele de laudă și de mulțumire în cinstea lui Dumnezeu.
\par 47 În zilele lui Zorobabel și Neemia, tot Israelul dădea parte cântăreților și portarilor pentru fiecare zi; și dădeau leviților darurile sfințite, și leviții le dădeau de asemenea darurile sfințite urmașilor lui Aaron.

\chapter{13}

\par 1 În acea zi s-a citit din cartea lui Moise în auzul poporului și s-a găsit scris în ea că Amonitul și Moabitul nu pot fi primiți niciodată în adunarea lui Dumnezeu,
\par 2 Pentru că ei n-au întâmpinat pe fiii lui Israel cu pâine și apă și pentru că au tocmit împotriva lor pe Valaam, ca să-i blesteme, dar Dumnezeul nostru a schimbat blestemul în binecuvântare.
\par 3 Și auzind această lege, a despărțit Israel tot ce era străin.
\par 4 Dar înainte de aceasta, preotul Eliașib, mai-marele peste cămările templului Dumnezeului nostru, și rudă cu Tobie,
\par 5 Rânduise pentru acesta o odaie mare, unde mai înainte se puneau prinoasele, tămâia, vasele sfinte, zeciuiala de pâine, de vin și de untdelemn, ce era hotărâtă pentru leviți, cântăreți, portari, și ceea ce se cuvenea preoților.
\par 6 Eu nu eram la Ierusalim când s-au făcut toate acestea, pentru că mă întorsesem la rege, în anul al treizeci și doilea al domniei lui Artaxerxe, regele Babilonului.
\par 7 Și la sfârșitul acestui an, am căpătat de la rege învoirea de a veni la Ierusalim, și am aflat de răul ce a făcut Eliașib, dând lui Tobie o cămară în curtea templului lui Dumnezeu.
\par 8 Și mi-a părut foarte rău de acestea și am aruncat afară din cămară toate lucrurile care erau ale lui Tobie, dând poruncă să fie sfințite cămările.
\par 9 Și am pus din nou lucrurile templului lui Dumnezeu, prinoasele și tămâia.
\par 10 Am mai înțeles de asemenea că părțile ce se cuveneau leviților nu le fuseseră date și că leviții și cântăreții, însărcinați să facă slujbă, fugiseră pe la țarinile lor.
\par 11 De aceea am mustrat pe căpetenii și am zis: "Pentru ce a fost părăsit templul lui Dumnezeu?" Și i-am adunat pe leviți și pe cântăreți și i-am pus la locurile lor.
\par 12 Atunci toți Iudeii au început a aduce la cămară zeciuială de pâine, de vin și de untdelemn;
\par 13 Și eu am încredințat supravegherea cămărilor lui Șelemia preotul și lui Țadoc cărturarul, și lui Pedaia, unul dintre leviți, și le-am dat ca ajutor pe Hanan, fiul lui Zacur, fiul lui Matania, căci aceștia se bucurau de mare cinste. Și tot ei au fost împuterniciți să facă împărțire fraților lor.
\par 14 Pomenește-mă, Dumnezeul meu, pentru aceasta și nu uita faptele mele evlavioase făcute pentru templul Dumnezeului meu și pentru slujbele din el!
\par 15 În zilele acelea am văzut în Iuda oameni care călcau în teascuri în ziua odihnei, aduceau snopi și încărcau pe asini vin, struguri, smochine și tot felul de greutăți și le duceau în ziua odihnei la Ierusalim. Și le-am făcut mustrare aspră chiar atunci, când vindeau lucruri de mâncare.
\par 16 De asemenea se aflau Tirieni, așezați în Ierusalim, care aduceau pește și tot felul de mărfuri și le vindeau, în ziua odihnei, Iudeilor în Ierusalim.
\par 17 Și am mustrat pe mai-marii Iudeilor și le-am zis: "Ce înseamnă aceste fapte urâte, pe care le faceți voi, pângărind ziua odihnei?
\par 18 Oare nu așa au făcut părinții voștri și nu din pricina aceasta a adus Dumnezeul nostru aceste necazuri asupra voastră și asupra acestei cetăți? Și voi atrageți din nou mânia Lui asupra lui Israel, necinstind ziua odihnei!"
\par 19 Apoi am poruncit să se închidă toate porțile Ierusalimului în ajunul zilei de odihnă, când începe a se întuneca, și să nu se mai deschidă decât după ea. Și am pus câte unul din slujitorii mei la porți, ca să împiedice intrarea de mărfuri în ziua odihnei.
\par 20 Atunci neguțători și vânzători de tot felul de lucruri au rămas câteodată sau de două ori afară din Ierusalim.
\par 21 Dar eu i-am mustrat aspru, zicându-le: "pentru ce petreceți voi noaptea înaintea zidurilor? De veți mai face aceasta, voi pune mâna pe voi!" De atunci ei n-au mai venit în ziua odihnei.
\par 22 Și am mai poruncit eu leviților să se curețe și să vină să păzească porțile, ca să sfințească ziua odihnei. Adu-ți aminte de mine, Dumnezeul meu, pentru acestea și mă ocrotește după mare mila Ta!
\par 23 Tot atunci am văzut eu Iudei, luându-și femei așdodiene, amonite și moabite.
\par 24 Jumătate din copiii lor vorbeau limba așdodiană și nu știau să vorbească evreiește; nu știau decât limba unuia sau altuia din acele popoare.
\par 25 Eu i-am mustrat aspru și pe aceștia și i-am blestemat; ba pe unii i-am lovit, le-am smuls părul și i-am jurat pe numele lui Dumnezeu, zicând: "Să nu vă dați fetele după feciorii lor și să nu luați pe fetele lor, nici pentru fiii voștri, nici pentru voi.
\par 26 Oare nu așa a păcătuit Solomon, regele lui Israel? Nu se afla rege ca el la nici un popor și el era iubit de Dumnezeu și Domnul îl pusese rege peste tot Israelul; dar femeile cele de alt neam l-au atras și pe el în păcat.
\par 27 Se poate oare să aud eu de voi că faceți acest rău mare și păcătuiți înaintea lui Dumnezeu, luându-vă femei de alt neam?"
\par 28 Unul din fiii lui Ioiada, fiul lui Eliașib, preotul cel mare, era ginerele lui Sanbalat Horonitul; dar eu l-am alungat de la mine.
\par 29 Adu-Ți aminte de ei, Dumnezeul meu, căci au spurcat preoția și legământul preoțesc și levit!
\par 30 Astfel i-am curățit eu de toți străinii și am pus rânduială preoților și leviților, fiecăruia după slujba lui,
\par 31 Și deasemenea pentru aducerea lemnelor la vremea hotărâtă, precum și a prinoaselor de pârgă. Adu-Ți aminte de mine, Dumnezeul meu, spre binele meu!


\end{document}