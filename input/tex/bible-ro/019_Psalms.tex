\begin{document}

\title{Psalms}

Psa 1:1  (Un psalm al lui David, nescris deasupra la evrei.) Fericit barbatul, care n-a umblat în sfatul necredincio?ilor ?i în calea pacato?ilor nu a stat ?i pe scaunul hulitorilor n-a ?ezut;
Psa 1:2  Ci în legea Domnului e voia lui ?i la legea Lui va cugeta ziua ?i noaptea.
Psa 1:3  ?i va fi ca un pom rasadit lânga izvoarele apelor, care rodul sau va da la vremea sa ?i frunza lui nu va cadea ?i toate câte va face vor spori:
Psa 1:4  Nu sunt a?a necredincio?ii, nu sunt a?a! Ci ca praful ce-l spulbera vântul de pe fa?a pamântului.
Psa 1:5  De aceea nu se vor ridica necredincio?ii la judecata, nici pacato?ii în sfatul drep?ilor.
Psa 1:6  Ca ?tie Domnul calea drep?ilor, iar calea necredincio?ilor va pieri.
Psa 2:1  (Un psalm al lui David, nescris deasupra la evrei.) Pentru ce s-au întarâtat neamurile ?i popoarele au cugetat de?ertaciuni?
Psa 2:2  S-au ridicat împara?ii pamântului ?i capeteniile s-au adunat împreuna împotriva Domnului ?i a unsului Sau, zicând:
Psa 2:3  "Sa rupem legaturile lor ?i sa lepadam de la noi jugul lor".
Psa 2:4  Cel ce locuie?te în ceruri va râde de dân?ii ?i Domnul îi va batjocori pe ei!
Psa 2:5  Atunci va grai catre ei întru urgia Lui ?i întru mânia Lui îi va tulbura pe ei;
Psa 2:6  Iar Eu sunt pus împarat de El peste Sion, muntele cel sfânt al Lui, vestind porunca Domnului.
Psa 2:7  Domnul a zis catre Mine: "Fiul Meu e?ti Tu, Eu astazi Te-am nascut!
Psa 2:8  Cere de la Mine ?i-?i voi da neamurile mo?tenirea Ta ?i stapânirea Ta, marginile pamântului.
Psa 2:9  Le vei pa?te pe ele cu toiag de fier; ca pe vasul olarului le vei zdrobi!"
Psa 2:10  ?i acum împara?i, în?elege?i! Înva?a?i-va to?i, care judeca?i pamântul!
Psa 2:11  Sluji?i Domnului cu frica ?i va bucura?i de El cu cutremur.
Psa 2:12  Lua?i înva?atura, ca nu cumva sa Se mânie Domnul ?i sa pieri?i din calea cea dreapta, când se va aprinde degrab mânia Lui! Ferici?i to?i cei ce nadajduiesc în El.
Psa 3:1  (Un psalm al lui David, când a fugit din fa?a lui Avesalom, fiul sau.) Doamne, cât s-au înmul?it cei ce ma necajesc! Mul?i se scoala asupra mea;
Psa 3:2  Mul?i zic sufletului meu: "Nu este mântuire lui, întru Dumnezeul lui! "
Psa 3:3  Iar Tu, Doamne, sprijinitorul meu e?ti, slava mea ?i Cel ce înal?i capul meu.
Psa 3:4  Cu glasul meu catre Domnul am strigat ?i m-a auzit din muntele cel sfânt al Lui.
Psa 3:5  Eu m-am culcat ?i am adormit; sculatu-m-am, ca Domnul ma va sprijini.
Psa 3:6  Nu ma voi teme de mii de popoare, care împrejur ma împresoara.
Psa 3:7  Scoala, Doamne, mântuie?te-ma, Dumnezeul meu, ca Tu ai batut pe to?i cei ce ma vrajma?esc în de?ert; din?ii pacato?ilor ai zdrobit.
Psa 3:8  A Domnului este mântuirea ?i peste poporul Tau, binecuvântarea Ta.
Psa 4:1  (Un psalm al lui David; mai-marelui cântare?ilor pentru instrumente cu coarde.) Când Te-am chemat, m-ai auzit, Dumnezeul drepta?ii mele! Întru necaz m-ai desfatat! Milostive?te-Te spre mine ?i asculta rugaciunea mea.
Psa 4:2  Fiii oamenilor, pâna când grei la inima? Pentru ce iubi?i de?ertaciunea ?i cauta?i minciuna?
Psa 4:3  Sa ?ti?i ca minunat a facut Domnul pe cel cuvios al Sau; Domnul ma va auzi când voi striga catre Dânsul.
Psa 4:4  Mânia?i-va, dar nu gre?i?i; de cele ce zice?i în inimile voastre, întru a?ternuturile voastre, va cai?i.
Psa 4:5  Jertfi?i jertfa drepta?ii ?i nadajdui?i în Domnul.
Psa 4:6  Mul?i zic: "Cine ne va arata noua cele bune?" Dar s-a însemnat peste noi lumina fe?ei Tale, Doamne!
Psa 4:7  Dat-ai veselie în inima mea, mai mare decât veselia pentru rodul lor de grâu, de vin ?i de untdelemn ce s-a înmul?it.
Psa 4:8  Cu pace, a?a ma voi culca ?i voi adormi, ca Tu, Doamne, îndeosebi întru nadejde m-ai a?ezat.
Psa 5:1  (Un psalm al lui David; mai-marelui cântare?ilor, pentru instrumente de suflat.) Graiurile mele asculta-le, Doamne! În?elege strigarea mea!
Psa 5:2  Ia aminte la glasul rugaciunii mele, Împaratul meu ?i Dumnezeul meu, caci catre Tine, ma voi ruga, Doamne!
Psa 5:3  Diminea?a vei auzi glasul meu; diminea?a voi sta înaintea Ta ?i ma vei vedea.
Psa 5:4  Ca Tu e?ti Dumnezeu, Care nu voie?ti faradelegea, nici nu va locui lânga Tine cel ce viclene?te.
Psa 5:5  Nu vor sta calcatorii de lege în preajma ochilor Tai. Urât-ai pe to?i cei ce lucreaza fara de lege.
Psa 5:6  Pierde-vei pe to?i cei ce graiesc minciuna; pe uciga? ?i pe viclean îl ura?te Domnul.
Psa 5:7  Iar eu, întru mul?imea milei Tale, voi intra în casa Ta, închina-ma-voi spre sfânt loca?ul Tau, întru frica Ta.
Psa 5:8  Doamne, pova?uie?te-ma întru dreptatea Ta din pricina du?manilor mei! Îndrepteaza înaintea mea calea Ta.
Psa 5:9  Ca nu este în gura lor adevar, inima lor este de?arta; groapa deschisa grumazul lor, cu limbile lor viclenesc.
Psa 5:10  Judeca-i pe ei, Dumnezeule; sa cada din sfaturile lor; dupa mul?imea nelegiuirilor lor, alunga-i pe ei, ca Te-au amarât, Doamne,
Psa 5:11  ?i sa se veseleasca to?i cei ce nadajduiesc întru Tine; în veac se vor bucura ?i le vei fi lor sala? ?i se vor lauda cu Tine to?i cei ce iubesc numele Tau.
Psa 5:12  Ca Tu vei binecuvânta pe cel drept, Doamne, caci cu arma bunei voiri ne-ai încununat pe noi.
Psa 6:1  (Un psalm al lui David; mai-marelui cântare?ilor, pentru instrumente cu coarde.) Doamne, nu cu mânia Ta sa ma mustri pe mine, nici cu urgia Ta sa ma cer?i.
Psa 6:2  Miluie?te-ma, Doamne, ca neputincios sunt; vindeca-ma, Doamne, ca s-au tulburat oasele mele;
Psa 6:3  ?i sufletul meu s-a tulburat foarte ?i Tu, Doamne, pâna când?
Psa 6:4  Întoarce-Te, Doamne; izbave?te sufletul meu, mântuie?te-ma, pentru mila Ta.
Psa 6:5  Ca nu este întru moarte cel ce Te pomene?te pe Tine. ?i în iad cine Te va lauda pe Tine?
Psa 6:6  Ostenit-am întru suspinul meu, spala-voi în fiecare noapte patul meu, cu lacrimile mele a?ternutul meu voi uda.
Psa 6:7  Tulburatu-s-a de suparare ochiul meu, îmbatrânit-am între to?i vrajma?ii mei.
Psa 6:8  Departa?i-va de la mine to?i cei ce lucra?i faradelegea, ca a auzit Domnul glasul plângerii mele.
Psa 6:9  Auzit-a Domnul cererea mea, Domnul rugaciunea mea a primit.
Psa 6:10  Sa se ru?ineze ?i sa se tulbure foarte to?i vrajma?ii mei; sa se întoarca ?i sa se ru?ineze foarte degrab.
Psa 7:1  (Un psalm al lui David, pe care l-a cântat Domnului, pentru cuvintele lui Husi, fiul lui Iemeni.) Doamne, Dumnezeul meu, în Tine am nadajduit. Mântuie?te-ma de to?i cei ce ma prigonesc ?i ma izbave?te,
Psa 7:2  Ca nu cumva sa rapeasca sufletul meu ca un leu, nefiind cine sa ma izbaveasca, nici cine sa ma mântuiasca.
Psa 7:3  Doamne, Dumnezeul meu, de am facut aceasta, de este nedreptate în mâinile mele,
Psa 7:4  De am rasplatit cu rau celor ce mi-au facut mie rau ?i de am jefuit pe vrajma?ii mei fara temei,
Psa 7:5  Sa prigoneasca vrajma?ul sufletul meu ?i sa-l prinda, sa calce la pamânt via?a mea ?i marirea mea în ?arâna sa o a?eze.
Psa 7:6  Scoala-Te, Doamne, întru mânia Ta, înal?a-Te pâna la hotarele vrajma?ilor mei; scoala-Te, Doamne, Dumnezeul meu, cu porunca cu care ai poruncit
Psa 7:7  ?i adunare de popoare Te va înconjura ?i peste ea la înal?ime Te întoarce.
Psa 7:8  Domnul va judeca pe popoare; judeca-ma, Doamne, dupa dreptatea mea ?i dupa nevinova?ia mea.
Psa 7:9  Sfâr?easca-se rautatea pacato?ilor ?i întare?te pe cel drept, Cel ce cerci inimile ?i rarunchii, Dumnezeule drepte.
Psa 7:10  Ajutorul meu de la Dumnezeu, Cel ce mântuie?te pe cei drep?i la inima.
Psa 7:11  Dumnezeu este judecator drept, tare ?i îndelung-rabdator ?i nu se mânie în fiecare zi.
Psa 7:12  De nu va ve?i întoarce, sabia Sa o va luci, arcul Sau l-a încordat ?i l-a pregatit
Psa 7:13  ?i în el a gatit unelte de moarte; sage?ile Lui pentru cei ce ard le-a lucrat.
Psa 7:14  Iata a poftit nedreptatea, a zamislit silnicia ?i a nascut nelegiuirea.
Psa 7:15  Groapa a sapat ?i a adâncit-o ?i va cadea în groapa pe care a facut-o.
Psa 7:16  Sa se întoarca nedreptatea lui pe capul lui ?i pe cre?tetul lui silnicia lui sa se coboare.
Psa 7:17  Lauda-voi pe Domnul dupa dreptatea Lui ?i voi cânta numele Domnului Celui Preaînalt.
Psa 8:1  (Un psalm al lui David; mai-marelui cântare?ilor, pentru ghitith.) Doamne, Dumnezeul nostru, cât de minunat este numele Tau în tot pamântul! Ca s-a înal?at slava Ta, mai presus de ceruri.
Psa 8:2  Din gura pruncilor ?i a celor ce sug ai savâr?it lauda, pentru vrajma?ii Tai, ca sa amu?e?ti pe vrajma? ?i pe razbunator.
Psa 8:3  Când privesc cerurile, lucrul mâinilor Tale, luna ?i stelele pe care Tu le-ai întemeiat, îmi zic:
Psa 8:4  Ce este omul ca-?i aminte?ti de el? Sau fiul omului, ca-l cercetezi pe el?
Psa 8:5  Mic?oratu-l-ai pe dânsul cu pu?in fa?a de îngeri, cu marire ?i cu cinste l-ai încununat pe el.
Psa 8:6  Pusu-l-ai pe dânsul peste lucrul mâinilor Tale, toate le-ai supus sub picioarele lui.
Psa 8:7  Oile ?i boii, toate; înca ?i dobitoacele câmpului;
Psa 8:8  Pasarile cerului ?i pe?tii marii, cele ce strabat cararile marilor.
Psa 8:9  Doamne, Dumnezeul nostru, cât de minunat este numele Tau în tot pamântul!
Psa 9:1  (Un psalm al lui David; mai-marelui cântare?ilor, pentru cele ascunse ale fiului.) Lauda-Te-voi, Doamne, din toata inima mea, spune-voi toate minunile Tale.
Psa 9:2  Veseli-ma-voi ?i ma voi bucura de Tine; cânta-voi numele Tau, Preaînalte.
Psa 9:3  Când se vor întoarce vrajma?ii mei înapoi, slabi-vor ?i vor pieri de la fa?a Ta!
Psa 9:4  Ca ai facut judecata mea ?i dreptatea mea; ?ezut-ai pe scaun, Cel ce judeci cu dreptate.
Psa 9:5  Certat-ai neamurile ?i au pierit nelegiui?ii; stins-ai numele lor în veac ?i în veacul veacului.
Psa 9:6  Vrajma?ului i-au lipsit de tot sabiile ?i ceta?ile i le-ai sfarâmat; pierit-a pomenirea lor în sunet.
Psa 9:7  Iar Domnul ramâne în veac; gatit-a scaunul Lui de judecata
Psa 9:8  ?i El va judeca lumea; cu dreptate va judeca popoarele.
Psa 9:9  ?i a fost Domnul scapare saracului, ajutor la vreme potrivita în necazuri.
Psa 9:10  Sa nadajduiasca în Tine cei ce cunosc numele Tau, ca n-ai parasit pe cei ce Te cauta pe Tine, Doamne!
Psa 9:11  Cânta?i Domnului, Celui ce locuie?te în Sion, vesti?i între neamuri faptele Lui.
Psa 9:12  Ca Cel ce razbuna sângele lor ?i-a adus aminte. N-a uitat strigatul saracilor.
Psa 9:13  Miluie?te-ma, Doamne! Vezi smerenia mea, de catre vrajma?ii mei, Cel ce ma înal?i din por?ile mor?ii,
Psa 9:14  Ca sa vestesc toate laudele Tale, în por?ile fiicei Sionului; veseli-ma-voi de mântuirea Ta!
Psa 9:15  Cazut-au neamurile în groapa pe care au facut-o; în cursa aceasta, pe care au ascuns-o, s-a prins piciorul lor.
Psa 9:16  Se cunoa?te Domnul când face judecata! Întru faptele mâinilor lui s-a prins pacatosul.
Psa 9:17  Sa se întoarca pacato?ii în iad; toate neamurile care uita pe Dumnezeu.
Psa 9:18  Ca nu pâna în sfâr?it va fi uitat saracul, iar rabdarea saracilor în veac nu va pieri.
Psa 9:19  Scoala-Te, Doamne, sa nu se întareasca omul; sa fie judecate neamurile înaintea Ta!
Psa 9:20  Pune, Doamne, legiuitor peste ele, ca sa cunoasca neamurile ca oameni sunt.
Psa 10:1  Pentru ce, Doamne, stai departe? Pentru ce treci cu vederea la vreme de necaz?
Psa 10:2  Când se mândre?te necredinciosul, se aprinde saracul; se prind în sfaturile pe care le gândesc.
Psa 10:3  Ca se lauda pacatosul cu poftele sufletului lui, iar cel ce face strâmbatate, pe sine se binecuvinteaza.
Psa 10:4  Întarâtat-a cel pacatos pe Domnul, dupa mul?imea mâniei lui; nu-L va cauta; nu este Dumnezeu înaintea lui.
Psa 10:5  Spurcate sunt caile lui în toata vremea; lepadate sunt judeca?ile Tale de la fa?a lui, peste to?i vrajma?ii lui va stapâni.
Psa 10:6  Ca a zis întru inima sa: Nu ma voi clinti din neam în neam, rau nu-mi va fi.
Psa 10:7  Gura lui e plina de blestem, de amaraciune ?i de vicle?ug; sub limba lui osteneala ?i durere.
Psa 10:8  Sta la pânda în ascuns cu cei boga?i ca sa ucida pe cel nevinovat; ochii lui spre cel sarac privesc.
Psa 10:9  Pânde?te din ascunzi?, ca leul din culcu?ul sau; pânde?te ca sa apuce pe sarac, pânde?te pe sarac ca sa-l traga la el.
Psa 10:10  În lan?ul lui îl va smeri; se va pleca ?i va cadea asupra lui, când va stapâni pe cei saraci.
Psa 10:11  Ca a zis în inima lui: "Uitat-a Dumnezeu! Întors-a fa?a Lui, ca sa nu vada pâna în sfâr?it!"
Psa 10:12  Scoala-Te, Doamne, Dumnezeul meu, înal?a-se mâna Ta, nu uita pe saracii Tai pâna în sfâr?it!
Psa 10:13  Pentru ce a mâniat necredinciosul pe Dumnezeu? Ca a zis în inima lui: Domnul nu va cerceta!
Psa 10:14  Vezi, pentru ca Tu prive?ti la necazuri ?i la durere, ca sa le iei în mâinile Tale; caci în Tine se încrede saracul, iar orfanului Tu i-ai fost ajutor.
Psa 10:15  Zdrobe?te bra?ul celui pacatos ?i rau, pacatul lui va fi cautat ?i nu se va afla.
Psa 10:16  Împara?i-va Domnul în veac ?i în veacul veacului! Pieri?i neamuri din pamântul Lui.
Psa 10:17  Dorin?a saracilor a auzit-o Domnul; la râvna inimii lor a luat aminte urechea Ta.
Psa 10:18  Judeca pe sarac ?i pe smerit, ca sa nu se mai mândreasca omul pe pamânt.
Psa 11:1  (Un psalm al lui David; mai-marelui cântare?ilor.) În Domnul am nadajduit. Cum ve?i zice sufletului meu: "Muta-te în mun?i, ca o pasare?"
Psa 11:2  Ca iata pacato?ii au încordat arcul, au gatit sage?i în tolba, ca sa sageteze în întuneric pe cei drep?i la inima.
Psa 11:3  Ca au surpat ceea ce ai a?ezat; dar dreptul ce a facut?
Psa 11:4  Domnul este în loca?ul cel sfânt al Sau, Domnul în cer are scaunul Sau. Ochii Lui spre sarac privesc, genele Lui cerceteaza pe fiii oamenilor.
Psa 11:5  Domnul cerceteaza pe cel drept ?i pe cel necredincios; iar pe cel ce iube?te nedreptatea îl ura?te sufletul Sau.
Psa 11:6  Va ploua peste pacato?i la?uri, foc ?i pucioasa; iar suflare de vifor este partea paharului lor.
Psa 11:7  Ca drept este Domnul ?i dreptatea a iubit ?i fa?a Lui spre cel drept prive?te.
Psa 12:1  (Un psalm al lui David; mai-marelui cântare?ilor, pentru instrumente cu opt coarde.) Mântuie?te-ma, Doamne, ca a lipsit cel cuvios, ca s-a împu?inat adevarul de la fiii oamenilor.
Psa 12:2  De?ertaciuni a grait fiecare catre aproapele sau, buze viclene în inima ?i în inima rele au grait.
Psa 12:3  Pierde-va Domnul toate buzele cele viclene ?i limba cea plina de mândrie.
Psa 12:4  Pe cei ce au zis: "Cu limba noastra ne vom mari, caci buzele noastre la noi sunt; cine ne este Domn?"
Psa 12:5  Pentru necazul saracilor ?i suspinul nenoroci?ilor, acum Ma voi scula, zice Domnul; le voi aduce lor mântuirea ?i le voi vorbi pe fa?a.
Psa 12:6  Cuvintele Domnului, cuvinte curate, argint lamurit în foc, cura?at de pamânt, cura?at de ?apte ori.
Psa 12:7  Tu, Doamne, ne vei pazi ?i ne vei feri de neamul acesta în veac.
Psa 12:8  Caci, atunci când se ridica sus oamenii de nimic, nelegiui?ii mi?una pretutindeni.
Psa 13:1  (Un psalm al lui David; mai-marelui cântare?ilor.) Pâna când, Doamne, ma vei uita pâna în sfâr?it? Pâna când vei întoarce fa?a Ta de la mine?
Psa 13:2  Pâna când voi gramadi gânduri în sufletul meu, durere în inima mea ziua ?i noaptea? Pâna când se va înal?a vrajma?ul meu împotriva mea?
Psa 13:3  Cauta, auzi-ma, Doamne, Dumnezeul meu, lumineaza ochii mei, ca nu cumva sa adorm întru moarte,
Psa 13:4  Ca nu cumva sa zica vrajma?ul meu: "Întaritu-m-am asupra lui". Cei ce ma necajesc se vor bucura de ma voi clatina.
Psa 13:5  Iar eu spre mila Ta am nadajduit; bucura-se-va inima mea de mântuirea Ta;
Psa 13:6  Cânta-voi Domnului, Celui ce mi-a facut bine ?i voi cânta numele Domnului Celui Preaînalt.
Psa 14:1  (Un psalm al lui David; mai-marelui cântare?ilor.) Zis-a cel nebun în inima sa: "Nu este Dumnezeu!" Stricatu-s-au oamenii ?i urâ?i s-au facut întru îndeletnicirile lor. Nu este cel ce face bunatate, nu este pâna la unul.
Psa 14:2  Domnul din cer a privit peste fiii oamenilor, sa vada de este cel ce în?elege, sau cel ce cauta pe Dumnezeu.
Psa 14:3  To?i s-au abatut, împreuna netrebnici s-au facut; nu este cel ce face bunatate, nu este pâna la unul.
Psa 14:4  Oare, nu se vor în?elep?i to?i cei ce lucreaza faradelegea? Cei ce manânca pe poporul Meu ca pâinea, pe Domnul nu L-au chemat.
Psa 14:5  Acolo s-au temut de frica, unde nu era frica, ca Dumnezeu este cu neamul drep?ilor.
Psa 14:6  Saracul nadajduie?te în Domnul ?i voi a?i râs de nadejdea lui, zicând: Cine va da din Sion mântuire lui Israel?
Psa 14:7  Dar când va întoarce Domnul pe cei robi?i ai poporului Sau, bucura-se-va Iacob ?i se va veseli Israel.
Psa 15:1  (Un psalm al lui David.) Doamne, cine va locui în loca?ul Tau ?i cine se va sala?lui în muntele cel sfânt al Tau?
Psa 15:2  Cel ce umbla fara prihana ?i face dreptate, cel ce are adevarul în inima sa,
Psa 15:3  Cel ce n-a viclenit cu limba, nici n-a facut rau împotriva vecinului sau ?i ocara n-a rostit împotriva aproapelui sau.
Psa 15:4  Defaimat sa fie înaintea Lui "el ce viclene?te, iar pe cei ce se tem de Domnul îi slave?te; cel ce se jura aproapelui sau ?i nu se leapada,
Psa 15:5  Argintul sau nu l-a dat cu camata ?i daruri împotriva celor nevinova?i n-a luat. Cel ce face acestea nu se va clatina în veac.
Psa 16:1  (Un psalm al lui David.) Paze?te-ma, Doamne, ca spre Tine am nadajduit.
Psa 16:2  Zis-am Domnului: "Domnul meu e?ti Tu, ca bunata?ile mele nu-?i trebuie".
Psa 16:3  Prin sfin?ii care sunt pe pamântul Lui minunata a facut Domnul toata voia întru ei.
Psa 16:4  Înmul?itu-s-au slabiciunile celor ce alearga dupa al?i dumnezei. Nu voi lua parte la adunarile lor cu jertfe de sânge, nici nu voi pomeni numele lor pe buzele mele.
Psa 16:5  Domnul este partea mo?tenirii mele ?i a paharului meu. Tu e?ti Cel care îmi a?ezi mie iara?i mo?tenirea mea.
Psa 16:6  Sor?ii mi-au cazut între cei puternici, ca mo?tenirea mea este puternica.
Psa 16:7  Binecuvânta-voi pe Domnul, Cel ce m-a în?elep?it; la aceasta ?i noaptea ma îndeamna inima mea.
Psa 16:8  Vazut-am mai înainte pe Domnul înaintea mea pururea, ca de-a dreapta mea este ca sa nu ma clatin.
Psa 16:9  Pentru aceasta s-a veselit inima mea ?i s-a bucurat limba mea, dar înca ?i trupul meu va sala?lui întru nadejde.
Psa 16:10  Ca nu vei lasa sufletul meu în iad, nici nu vei da pe cel cuvios al Tau sa vada stricaciunea.
Psa 16:11  Cunoscute mi-ai facut caile vie?ii; umplea-ma-vei de veselie cu fa?a Ta, ?i la dreapta Ta de frumuse?i ve?nice ma vei satura.
Psa 17:1  (O rugaciune a lui David.) Auzi, Doamne, dreptatea mea, ia aminte cererea mea, asculta rugaciunea mea, din buze fara de viclenie.
Psa 17:2  De la fa?a Ta judecata mea sa iasa, ochii mei sa vada cele drepte.
Psa 17:3  Cercetat-ai inima mea, noaptea ai cercetat-o; cu foc m-ai lamurit, dar nu s-a aflat întru mine nedreptate.
Psa 17:4  Ca sa nu graiasca gura mea lucruri omene?ti, pentru cuvintele buzelor Tale eu am pazit cai aspre.
Psa 17:5  Îndreapta picioarele mele în cararile Tale, ca sa nu ?ovaie pa?ii mei.
Psa 17:6  Eu am strigat, ca m-ai auzit Dumnezeule; pleaca urechea Ta catre mine ?i auzi cuvintele mele.
Psa 17:7  Minunate fa milele Tale, Cel ce mântuie?ti pe cei ce nadajduiesc în Tine de cei ce stau împotriva dreptei Tale.
Psa 17:8  Paze?te-ma, Doamne, ca pe lumina ochilor; cu acoperamântul aripilor Tale acopera-ma
Psa 17:9  De fa?a necredincio?ilor care ma necajesc pe mine. Vrajma?ii mei sufletul meu l-au cuprins;
Psa 17:10  Cu grasime inima lor ?i-au încuiat, gura lor a grait mândrie.
Psa 17:11  Izgonindu-ma acum m-au înconjurat, ochii lor ?i-au a?intit ca sa ma plece la pamânt.
Psa 17:12  Apucatu-m-au ca un leu gata de prada, ca un pui de leu ce locuie?te în ascunzi?uri.
Psa 17:13  Scoala-Te, Doamne, întâmpina-i pe ei ?i împiedica-i! Izbave?te sufletul meu de cel necredincios, cu sabia Ta.
Psa 17:14  Doamne, desparte-ma de oamenii acestei lumi, ce-?i iau partea în via?a, caci s-a umplut pântecele lor de bunata?ile Tale; saturatu-s-au fiii lor ?i au lasat rama?i?ele pruncilor.
Psa 17:15  Iar eu întru dreptate ma voi arata fe?ei Tale, satura-ma-voi când se va arata slava Ta.
Psa 18:1  (Mai-marelui cântare?ilor; un psalm al lui David, sluga Domnului, care a grait Domnului cuvintele cântarii acesteia, în ziua în care l-a izbavit pe el Domnul din mâna tuturor vrajma?ilor lui ?i din mâna lui Saul. ?i a zis:) Iubi-Te-voi Doamne, vârtutea mea.
Psa 18:2  Domnul este întarirea mea ?i scaparea mea ?i izbavitorul meu, Dumnezeul meu, ajutorul meu ?i voi nadajdui spre Dânsul, aparatorul meu ?i puterea mântuirii mele ?i sprijinitorul meu.
Psa 18:3  Laudând voi chema pe Domnul ?i de vrajma?ii mei ma voi izbavi.
Psa 18:4  Cuprinsu-m-au durerile mor?ii ?i râurile faradelegii m-au tulburat.
Psa 18:5  Durerile iadului m-au înconjurat; întâmpinatu-m-au la?urile mor?ii.
Psa 18:6  ?i când ma necajeau am chemat pe Domnul ?i catre Dumnezeul meu am strigat. Auzit-a din loca?ul Lui cel sfânt glasul meu ?i strigarea mea, înaintea Lui, va intra în urechile Lui.
Psa 18:7  ?i s-a clatinat ?i s-a cutremurat pamântul ?i temeliile mun?ilor s-au tulburat ?i s-au clatinat ca S-a mâniat pe ele Dumnezeu.
Psa 18:8  Întru mânia Lui fum s-a ridicat ?i para de foc de la fa?a Lui, carbuni aprin?i de la El.
Psa 18:9  ?i a plecat cerurile ?i S-a pogorât ?i negura era sub picioarele Lui.
Psa 18:10  ?i S-a suit pe heruvimi ?i a zburat; zburat-a pe aripile vântului.
Psa 18:11  ?i ?i-a pus întunericul acoperamânt, împrejurul Lui cortul Lui, apa întunecoasa în norii vazduhului.
Psa 18:12  De stralucirea fe?ei Lui norii au fugit, glasul Lui prin grindina ?i carbuni de foc.
Psa 18:13  ?i a tunat din cer Domnul ?i Cel Preaînalt a dat glasul Sau.
Psa 18:14  Trimis-a sage?i ?i i-a risipit pe ei, ?i fulgere a înmul?it ?i i-a tulburat pe ei.
Psa 18:15  ?i s-au aratat izvoarele apelor ?i s-au descoperit temeliile lumii, de certarea Ta, Doamne, de suflarea Duhului mâniei Tale.
Psa 18:16  Trimis-a din înal?ime ?i m-a luat, ridicatu-m-a din ape multe.
Psa 18:17  Izbave?te-ma de vrajma?ii mei cei tari ?i de cei ce ma urasc pe mine, ca s-au întarit mai mult decât mine.
Psa 18:18  Întâmpinatu-m-au ei în ziua necazului meu, dar Domnul a fost întarirea mea
Psa 18:19  ?i m-a scos la loc larg, m-a izbavit, ca m-a voit.
Psa 18:20  ?i îmi va rasplati mie Domnul dupa dreptatea mea, ?i dupa cura?ia mâinilor mele îmi va rasplati mie,
Psa 18:21  Ca am pazit caile Domnului ?i n-am fost necredincios Dumnezeului meu.
Psa 18:22  Ca toate judeca?ile Lui sunt înaintea mea, ?i îndreptarile Lui nu s-au departat de la mine.
Psa 18:23  ?i voi fi fara prihana cu Dânsul, ?i ma voi pazi de faradelegea mea.
Psa 18:24  ?i îmi va rasplati mie Domnul dupa dreptatea mea, ?i dupa cura?ia mâinilor mele înaintea ochilor Lui.
Psa 18:25  Cu cel cuvios, cuvios vei fi; ?i cu omul nevinovat, nevinovat vei fi.
Psa 18:26  ?i cu cel ales, ales vei fi; ?i cu cel îndaratnic Te vei îndaratnici.
Psa 18:27  Ca Tu pe poporul cel smerit îl vei mântui, ?i ochii mândrilor îi vei smeri.
Psa 18:28  Ca Tu vei aprinde faclia mea, Doamne; Dumnezeul meu, vei lumina întunericul meu.
Psa 18:29  Caci cu Tine ma voi izbavi de ispita, ?i cu Dumnezeul meu voi trece zidul.
Psa 18:30  Dumnezeul meu, fara prihana este calea Lui, cuvintele Domnului în foc lamurite; scut este tuturor celor ce nadajduiesc în El.
Psa 18:31  Ca cine este Dumnezeu afara de Domnul? ?i cine este Dumnezeu afara de Dumnezeul nostru?
Psa 18:32  Dumnezeu, Cel ce ma încinge cu putere, ?i a pus fara prihana calea mea.
Psa 18:33  Cel ce face picioarele mele ca ale cerbului ?i peste cele înalte ma pune.
Psa 18:34  Cel ce întare?ti mâinile mele în vreme de razboi, ?i ai pus arc de arama în bra?ele mele.
Psa 18:35  ?i mi-ai dat mie scutul mântuirii mele ?i dreapta Ta m-a sprijinit. ?i certarea Ta m-a îndreptat pâna în sfâr?it, ?i certarea Ta însa?i ma va înva?a.
Psa 18:36  Largit-ai pa?ii mei sub mine, ?i n-au slabit picioarele mele.
Psa 18:37  Urmari-voi pe vrajma?ii mei ?i-i voi prinde pe dân?ii ?i nu ma voi întoarce pâna ce se vor sfâr?i.
Psa 18:38  Îi voi zdrobi pe ei ?i nu vor putea sa stea, cadea-vor sub picioarele mele.
Psa 18:39  ?i m-ai încins cu putere spre razboi ?i ai împiedicat pe to?i cei ce se sculau împotriva mea.
Psa 18:40  ?i pe vrajma?ii mei i-ai facut sa fuga, iar pe cei ce ma urasc pe mine i-ai nimicit.
Psa 18:41  Strigat-au catre Domnul, ?i nu era cel ce mântuie?te; ?i nu i-a auzit pe ei.
Psa 18:42  ?i-i voi sfarâma pe ei ca praful în fa?a vântului, ca tina uli?elor îi voi zdrobi pe ei.
Psa 18:43  Izbave?te-ma de razvratirile poporului; pusu-m-ai capetenie neamurilor.
Psa 18:44  Poporul pe care nu l-am cunoscut mi-a slujit mie. Cu auzul urechii m-a auzit.
Psa 18:45  Fiii straini m-au min?it pe mine. Fiii straini au îmbatrânit ?i au ?chiopatat din cararile lor.
Psa 18:46  Viu este Domnul ?i binecuvântat este Dumnezeul meu, ?i sa se înal?e Dumnezeul mântuirii mele.
Psa 18:47  Dumnezeule, Cel ce mi-ai dat izbânda ?i mi-ai supus popoarele; Izbavitorul meu de vrajma?ii ceâ furio?i,
Psa 18:48  De la cei ce se ridica împotriva mea, înal?a-ma, de omul nedrept izbave?te-ma.
Psa 18:49  Pentru aceasta Te voi lauda între neamuri, Doamne, ?i numele Tau îl voi cânta.
Psa 18:50  Cel ce mare?ti mântuirea împaratului Tau ?i faci mila unsului Tau, lui David ?i semin?iei lui pâna în veac.
Psa 19:1  (Un psalm al lui David; mai-marelui cântare?ilor.) Cerurile spun slava lui Dumnezeu ?i facerea mâinilor Lui o veste?te taria.
Psa 19:2  Ziua zilei spune cuvânt, ?i noaptea nop?ii veste?te ?tiin?a.
Psa 19:3  Nu sunt graiuri, nici cuvinte, ale caror glasuri sa nu se auda.
Psa 19:4  În tot pamântul a ie?it vestirea lor, ?i la marginile lumii cuvintele lor.
Psa 19:5  În soare ?i-a pus loca?ul sau; ?i el este ca un mire ce iese din camara sa. Bucura-se-va ca un uria?, care alearga drumul lui.
Psa 19:6  De la marginea cerului ie?irea lui, ?i oprirea lui pâna la marginea cerului; ?i nu este cine sa se ascunda de caldura lui.
Psa 19:7  Legea Domnului este fara prihana, întoarce sufletele; marturia Domnului este credincioasa, în?elep?e?te pruncii;
Psa 19:8  Judeca?ile Domnului sunt drepte, veselesc inima; porunca Domnului este stralucitoare, lumineaza ochii.
Psa 19:9  Frica de Domnul este curata, ramâne în veacul veacului. Judeca?ile Domnului sunt adevarate, toate îndrepta?ite.
Psa 19:10  Dorite sunt mai mult decât aurul, ?i decât piatra cea de mare pre?; ?i mai dulci decât mierea ?i fagurele.
Psa 19:11  Ca robul Tau le paze?te pe ele, ?i rasplatire multa are.
Psa 19:12  Gre?elile cine le va pricepe? De cele ascunse ale mele cura?e?te-ma
Psa 19:13  ?i de cele straine fere?te pe robul Tau; de nu ma vor stapâni, atunci fara prihana voi fi ?i ma voi cura?i de pacat mare.
Psa 19:14  ?i vor bineplacea cuvintele gurii mele ?i cugetul inimii mele înaintea Ta pururea; Doamne, Ajutorul meu ?i Izbavitorul meu.
Psa 20:1  (Un psalm al lui David; mai-marelui cântare?ilor.) Sa te auda Domnul în ziua necazului ?i sa te apere numele Dumnezeului lui Iacob.
Psa 20:2  Trimita ?ie ajutor din loca?ul Sau cel sfânt ?i din Sion sa te sprijineasca pe tine.
Psa 20:3  Pomeneasca toata jertfa ta ?i arderea de tot a ta bineplacuta sa-I fie.
Psa 20:4  Dea ?ie Domnul dupa inima ta ?i tot sfatul tau sa-l plineasca.
Psa 20:5  Bucura-ne-vom de mântuirea ta ?i întru numele Dumnezeului nostru ne vom mari. Plineasca Domnul toate cererile tale.
Psa 20:6  Acum am cunoscut ca a mântuit Domnul pe unsul Sau, cu puterea dreptei Sale. Îl va auzi pe dânsul din cerul cel sfânt al Lui.
Psa 20:7  Unii se lauda cu caru?ele lor, al?ii cu caii lor, iar noi ne laudam cu numele Domnului Dumnezeului nostru.
Psa 20:8  Ace?tia s-au împiedicat ?i au cazut, iar noi ne-am sculat ?i ne-am îndreptat.
Psa 20:9  Doamne, mântuie?te pe împaratul ?i ne auzi pe noi, în orice zi Te vom chema.
Psa 21:1  (Un psalm al lui David; mai-marelui cântare?ilor.) Doamne, întru puterea Ta se va veseli împaratul ?i întru mântuirea Ta se va bucura foarte.
Psa 21:2  Dupa dorirea inimii lui i-ai dat lui, ?i de voia buzelor lui nu l-ai lipsit pe el.
Psa 21:3  Ca l-ai întâmpinat pe el cu binecuvântarile bunata?ii, pus-ai pe capul lui cununa de piatra scumpa.
Psa 21:4  Via?a a cerut de la Tine ?i i-ai dat lui lungime de zile în veacul veacului.
Psa 21:5  Mare este slava lui întru mântuirea Ta, slava ?i mare cuviin?a vei pune peste el.
Psa 21:6  Ca îi vei da lui binecuvântare în veacul veacului, îl vei veseli pe dânsul întru bucurie cu fa?a Ta.
Psa 21:7  Ca împaratul nadajduie?te în Domnul ?i întru mila Celui Preaînalt nu se va clinti.
Psa 21:8  Afla-se mâna Ta peste to?i vrajma?ii Tai, dreapta Ta sa afle pe to?i cei ce Te urasc pe Tine.
Psa 21:9  Îi vei pune pe ei ca un cuptor de foc în vremea aratarii Tale; Domnul întru mânia Sa îi va tulbura pe ei, ?i-i va mânca pe ei focul.
Psa 21:10  Rodul lor de pe pamânt îl vei pierde ?i samân?a lor dintre fiii oamenilor.
Psa 21:11  Ca au gândit rele împotriva Ta, au cugetat sfaturi care nu vor putea sa stea.
Psa 21:12  Ca îi vei pune pe ei pe fuga ?i cu arcul Tau vei ?inti capul lor.
Psa 21:13  Înal?a-Te, Doamne, întru taria Ta, cânta-vom ?i vom lauda puterile Tale.
Psa 22:1  (Un psalm al lui David; mai-marelui cântare?ilor, pentru sprijinul cel de diminea?a.) Dumnezeul meu, Dumnezeul meu, ia aminte la mine, pentru ce m-ai parasit? Departe sunt de mântuirea mea cuvintele gre?elilor mele.
Psa 22:2  Dumnezeul meu, striga-voi ziua ?i nu vei auzi, ?i noaptea ?i nu Te vei gândi la mine.
Psa 22:3  Iar Tu întru cele sfinte locuie?ti, lauda lui Israel.
Psa 22:4  În Tine au nadajduit parin?ii no?tri, nadajduit-au în Tine ?i i-ai izbavit pe ei.
Psa 22:5  Catre Tine au strigat ?i s-au mântuit, în Tine au nadajduit ?i nu s-au ru?inat.
Psa 22:6  Iar eu sunt vierme ?i nu om, ocara oamenilor ?i defaimarea poporului.
Psa 22:7  To?i cei ce m-au vazut m-au batjocorit, grait-au cu buzele, clatinat-au capul zicând:
Psa 22:8  "Nadajduit-a spre Domnul, izbaveasca-l pe el, mântuiasca-l pe el, ca-l voie?te pe el".
Psa 22:9  Ca Tu e?ti Cel ce m-ai scos din pântece, nadejdea mea, de la sânul maicii mele.
Psa 22:10  Spre Tine m-am aruncat de la na?tere, din pântecele maicii mele Dumnezeul meu e?ti Tu.
Psa 22:11  Nu Te departa de la mine, ca necazul este aproape, ?i nu este cine sa-mi ajute.
Psa 22:12  Înconjuratu-m-au vi?ei mul?i, tauri gra?i m-au împresurat.
Psa 22:13  Deschis-au asupra mea gura lor, ca un leu ce rape?te ?i racne?te.
Psa 22:14  Ca apa m-am varsat ?i s-au risipit toate oasele mele. Facutu-s-a inima mea ca ceara ce se tope?te în mijlocul pântecelui meu.
Psa 22:15  Uscatu-s-a ca un vas de lut taria mea, ?i limba mea s-a lipit de cerul gurii mele ?i în ?arâna mor?ii m-ai coborât.
Psa 22:16  Ca m-au înconjurat câini mul?i, adunarea celor vicleni m-a împresurat. Strapuns-au mâinile mele ?i picioarele mele.
Psa 22:17  Numarat-au toate oasele mele, iar ei priveau ?i se uitau la mine.
Psa 22:18  Împar?it-au hainele mele loru?i ?i pentru cama?a mea au aruncat sor?i.
Psa 22:19  Iar Tu, Doamne, nu departa ajutorul Tau de la mine, spre sprijinul meu ia aminte.
Psa 22:20  Izbave?te de sabie sufletul meu ?i din gheara câinelui via?a mea.
Psa 22:21  Izbave?te-ma din gura leului ?i din coarnele taurilor smerenia mea.
Psa 22:22  Spune-voi numele Tau fra?ilor mei; în mijlocul adunarii Te voi lauda, zicând:
Psa 22:23  Cei ce va teme?i de Domnul, lauda?i-L pe El, toata semin?ia lui Iacob slavi?i-L pe El! Sa se teama de Dânsul toata semin?ia lui Israel.
Psa 22:24  Ca n-a defaimat, nici n-a lepadat ruga saracului, nici n-a întors fa?a Lui de la mine ?i când am strigat catre Dânsul, m-a auzit.
Psa 22:25  De la Tine este lauda mea în adunare mare, rugaciunile mele le voi face înaintea celor ce se tem de El.
Psa 22:26  Mânca-vor saracii ?i se vor satura ?i vor lauda pe Domnul, iar cei ce-L cauta pe Dânsul vii vor fi inimile lor în veacul veacului.
Psa 22:27  Î?i vor aduce aminte ?i se vor întoarce la Domnul toate marginile pamântului. ?i se vor închina înaintea Lui toate semin?iile neamurilor.
Psa 22:28  Ca a Domnului este împara?ia ?i El stapâne?te peste neamuri.
Psa 22:29  Mâncat-au ?i s-au închinat to?i gra?ii pamântului, înaintea Lui vor cadea to?i cei ce se coboara în pamânt.
Psa 22:30  ?i sufletul meu în El viaza, ?i semin?ia mea va sluji Lui. Se va vesti Domnului neamul ce va sa vina.
Psa 22:31  ?i vor vesti dreptatea Lui poporului ce se va na?te ?i ce a facut Domnul.
Psa 23:1  (Un psalm al lui David.) Domnul ma pa?te ?i nimic nu-mi va lipsi.
Psa 23:2  La loc de pa?une, acolo m-a sala?luit; la apa odihnei m-a hranit.
Psa 23:3  Sufletul meu l-a întors, pova?uitu-m-a pe caile drepta?ii, pentru numele Lui.
Psa 23:4  Ca de voi ?i umbla în mijlocul mor?ii, nu ma voi teme de rele; ca Tu cu mine e?ti. Toiagul Tau ?i varga Ta, acestea m-au mângâiat.
Psa 23:5  Gatit-ai masa înaintea mea, împotriva celor ce ma necajesc; uns-ai cu untdelemn capul meu ?i paharul Tau este adapându-ma ca un puternic.
Psa 23:6  ?i mila Ta ma va urma în toate zilele vie?ii mele, ca sa locuiesc în casa Domnului, întru lungime de zile.
Psa 24:1  (Un psalm al lui David; al uneia din sâmbete.) Al Domnului este pamântul ?i plinirea lui; lumea ?i to?i cei ce locuiesc în ea.
Psa 24:2  Acesta pe mari l-a întemeiat pe el ?i pe râuri l-a a?ezat pe el.
Psa 24:3  Cine se va sui în muntele Domnului ?i cine va sta în locul cel sfânt al Lui?
Psa 24:4  Cel nevinovat cu mâinile ?i curat cu inima, care n-a luat în de?ert sufletul sau ?i nu s-a jurat cu vicle?ug aproapelui sau.
Psa 24:5  Acesta va lua binecuvântare de la Domnul ?i milostenie de la Dumnezeu, Mântuitorul sau.
Psa 24:6  Acesta este neamul celor ce-L cauta pe Domnul, al celor ce cauta fa?a Dumnezeului lui Iacob.
Psa 24:7  Ridica?i, capetenii, por?ile voastre ?i va ridica?i por?ile cele ve?nice ?i va intra Împaratul slavei.
Psa 24:8  Cine este acesta Împaratul slavei? Domnul Cel tare ?i puternic, Domnul Cel tare în razboi.
Psa 24:9  Ridica?i, capetenii, por?ile voastre ?i va ridica?i por?ile cele ve?nice ?i va intra Împaratul slavei.
Psa 24:10  Cine este acesta Împaratul slavei? Domnul puterilor, Acesta este Împaratul slavei.
Psa 25:1  (Un psalm al lui David.) Catre Tine, Doamne, am ridicat sufletul meu, Dumnezeul meu.
Psa 25:2  Spre Tine am nadajduit, sa nu fiu ru?inat în veac, nici sa râda de mine vrajma?ii mei.
Psa 25:3  Pentru ca to?i cei ce Te a?teapta pe Tine nu se vor ru?ina; sa se ru?ineze to?i cei ce fac faradelegi în de?ert.
Psa 25:4  Caile Tale, Doamne, arata-mi, ?i cararile Tale ma înva?a.
Psa 25:5  Îndrepteaza-ma spre adevarul Tau ?i ma înva?a, ca Tu e?ti Dumnezeu, Mântuitorul meu, ?i pe Tine Te-am a?teptat toata ziua.
Psa 25:6  Adu-?i aminte de îndurarile ?i milele Tale, Doamne, ca din veac sunt.
Psa 25:7  Pacatele tinere?ilor mele ?i ale ne?tiin?ei mele nu le pomeni. Dupa mila Ta pomene?te-ma Tu, pentru bunatatea Ta, Doamne.
Psa 25:8  Bun ?i drept este Domnul, pentru aceasta lege va pune celor ce gre?esc în cale.
Psa 25:9  Îndrepta-va pe cei blânzi la judecata, înva?a-va pe cei blânzi caile Sale.
Psa 25:10  Toate caile Domnului sunt mila ?i adevar pentru cei ce cauta a?ezamântul Lui ?i marturiile Lui.
Psa 25:11  Pentru numele Tau, Doamne, cura?e?te pacatul meu ca mult este.
Psa 25:12  Cine este omul cel ce se teme de Domnul? Lege va pune lui în calea pe care a ales-o.
Psa 25:13  Sufletul lui întru bunata?i se va sala?lui ?i semin?ia lui va mo?teni pamântul.
Psa 25:14  Domnul este întarirea celor ce se tem de El, a?ezamântul Lui îl va arata lor.
Psa 25:15  Ochii mei sunt pururea spre Domnul ca El va scoate din la? picioarele mele.
Psa 25:16  Cauta spre mine ?i ma miluie?te, ca parasit ?i sarac sunt eu.
Psa 25:17  Necazurile inimii mele s-au înmul?it; din nevoile mele scoate-ma.
Psa 25:18  Vezi smerenia mea ?i osteneala mea ?i-mi iarta toate pacatele mele.
Psa 25:19  Vezi pe vrajma?ii mei ca s-au înmul?it ?i cu ura nedreapta m-au urât.
Psa 25:20  Paze?te sufletul meu ?i ma izbave?te, ca sa nu ma ru?inez ca am nadajduit în Tine.
Psa 25:21  Cei fara rautate ?i cei drep?i s-au lipit de mine, ca Te-am a?teptat, Doamne.
Psa 25:22  Izbave?te, Dumnezeule, pe Israel din toate necazurile lui.
Psa 26:1  (Un psalm al lui David.) Judeca-ma, Doamne, ca eu întru nerautate am umblat ?i în Domnul nadajduind, nu voi slabi.
Psa 26:2  Cerceteaza-ma, Doamne, ?i ma cearca; aprinde rarunchii ?i inima mea.
Psa 26:3  Ca mila Ta este înaintea ochilor mei ?i bine mi-a placut adevarul Tau.
Psa 26:4  Nu am ?ezut în adunarea de?ertaciunii ?i cu calcatorii de lege nu voi intra.
Psa 26:5  Urât-am adunarea celor ce viclenesc ?i cu cei necredincio?i nu voi ?edea.
Psa 26:6  Spala-voi întru cele nevinovate mâinile mele ?i voi înconjura jertfelnicul Tau, Doamne,
Psa 26:7  Ca sa aud glasul laudei Tale ?i sa spun toate minunile Tale.
Psa 26:8  Doamne, iubit-am bunacuviin?a casei Tale ?i locul loca?ului slavei Tale.
Psa 26:9  Sa nu pierzi cu cei necredincio?i sufletul meu ?i cu varsatorii de sânge via?a mea,
Psa 26:10  Întru ale caror mâini sunt faradelegi ?i dreapta carora e plina de daruri.
Psa 26:11  Iar eu întru nerautatea mea am umblat; izbave?te-ma, Doamne, ?i ma miluie?te,
Psa 26:12  Caci piciorul meu a stat întru dreptate; întru adunari Te voi binecuvânta, Doamne.
Psa 27:1  (Un psalm al lui David; mai înainte de ungere.) Domnul este luminarea mea ?i mântuirea mea; de cine ma voi teme? Domnul este aparatorul vie?ii mele; de cine ma voi înfrico?a?
Psa 27:2  Când se vor apropia de mine cei ce îmi fac rau, ca sa manânce trupul meu; cei ce ma necajesc ?i vrajma?ii mei, aceia au slabit ?i au cazut.
Psa 27:3  De s-ar rândui împotriva mea o?tire, nu se va înfrico?a inima mea; de s-ar ridica împotriva mea razboi, eu în El nadajduiesc.
Psa 27:4  Una am cerut de la Domnul, pe aceasta o voi cauta: sa locuiesc în casa Domnului în toate zilele vie?ii mele, ca sa vad frumuse?ea Domnului ?i sa cercetez loca?ul Lui.
Psa 27:5  Ca Domnul m-a ascuns în cortul Lui în ziua necazurilor mele; m-a acoperit în locul cel ascuns al cortului Lui; pe piatra m-a înal?at.
Psa 27:6  ?i acum iata, a înal?at capul meu peste vrajma?ii mei. Înconjurat-am ?i am jertfit în cortul Lui jertfa de lauda. Îl voi lauda ?i voi cânta Domnului.
Psa 27:7  Auzi, Doamne, glasul meu cu care am strigat; miluie?te-ma ?i ma asculta.
Psa 27:8  ?ie a zis inima mea: Pe Domnul voi cauta. Te-a cautat fa?a mea; fa?a Ta, Doamne, voi cauta.
Psa 27:9  Sa nu-?i întorci fa?a Ta de la mine ?i sa nu Te aba?i întru mânie de la robul Tau; ajutorul meu fii, sa nu ma lepezi pe mine ?i sa nu ma la?i, Dumnezeule, Mântuitorul meu.
Psa 27:10  Ca tatal meu ?i mama mea m-au parasit, dar Domnul m-a luat.
Psa 27:11  Lege pune-mi mie, Doamne, în calea Ta ?i ma îndrepteaza pe cararea dreapta, din pricina vrajma?ilor mei.
Psa 27:12  Nu ma da pe mine pe mâna celor ce ma necajesc, ca s-au ridicat împotriva mea martori nedrep?i ?i nedreptatea a min?it sie?i.
Psa 27:13  Cred ca voi vedea bunata?ile Domnului, în pamântul celor vii.
Psa 27:14  A?teapta pe Domnul, îmbarbateaza-te ?i sa se întareasca inima ta ?i a?teapta pe Domnul.
Psa 28:1  (Un psalm al lui David.) Catre Tine, Doamne, am strigat, Dumnezeul meu, ia aminte! Ca de nu ma vei auzi, ma voi asemana cu cei care se coboara în groapa.
Psa 28:2  Asculta glasul rugaciunii mele când ma rog catre Tine, când ridic mâinile mele catre loca?ul Tau cel sfânt.
Psa 28:3  Sa nu tragi cu cei pacato?i sufletul meu, ?i cu cei ce lucreaza nedreptate sa nu ma pierzi, cu cei ce graiesc pace catre aproapele lor, dar cele rele sunt în inimile lor.
Psa 28:4  Da-le lor dupa faptele lor ?i dupa vicle?ugul gândurilor lor. Dupa lucrul mâinilor lor, da-le lor; rasplate?te-i dupa faptele lor,
Psa 28:5  Ca n-au în?eles lucrurile Domnului ?i faptele mâinilor Lui; îi vei darâma ?i nu-i vei zidi.
Psa 28:6  Binecuvântat este Domnul ca a auzit glasul rugaciunii mele.
Psa 28:7  Domnul este ajutorul ?i aparatorul meu, în El a nadajduit inima mea ?i mi-a ajutat. ?i a înflorit trupul meu ?i de bunavoia mea Îl voi lauda pe El.
Psa 28:8  Domnul este întarirea poporului Sau ?i aparator mântuirilor unsului Sau.
Psa 28:9  Mântuie?te poporul Tau ?i binecuvinteaza mo?tenirea Ta; pa?te-i pe ei ?i-i ridica pâna în veac.
Psa 29:1  (Un psalm al lui David; la scoaterea Cortului.) Aduce?i Domnului, fii ai lui Dumnezeu, aduce?i Domnului mieii oilor, aduce?i Domnului slava ?i cinste;
Psa 29:2  Aduce?i Domnului slava numelui Sau; închina?i-va Domnului în curtea cea sfânta a Lui.
Psa 29:3  Glasul Domnului peste ape; Dumnezeul slavei a tunat; Domnul peste ape multe.
Psa 29:4  Glasul Domnului întru tarie, glasul Domnului întru mare cuviin?a;
Psa 29:5  Glasul Domnului cel ce sfarâma cedrii ?i va zdrobi Domnul cedrii Libanului;
Psa 29:6  El face sa sara Libanul ca un vi?el; iar Ermonul ca un pui de gazela.
Psa 29:7  Glasul Domnului, cel ce varsa para focului.
Psa 29:8  Glasul Domnului, cel ce cutremura pustiul ?i va cutremura Domnul pustiul Cade?ului.
Psa 29:9  Glasul Domnului dezleaga pântecele cerboaicelor, glasul Domnului despoaie cedrii ?i în loca?ul Lui, fiecare va spune: Slava!
Psa 29:10  Domnul va împara?i peste potop ?i va ?edea Domnul Împarat în veac.
Psa 29:11  Domnul tarie poporului Sau va da, Domnul va binecuvânta pe poporul Sau cu pace.
Psa 30:1  (Un psalm al lui David; mai-marelui cântare?ilor pentru sfin?irea casei.) Te voi înal?a, Doamne, ca m-ai ridicat ?i n-ai veselit pe vrajma?ii mei împotriva mea.
Psa 30:2  Doamne, Dumnezeul meu, strigat-am catre Tine ?i m-ai vindecat.
Psa 30:3  Doamne, scos-ai din iad sufletul meu, mântuitu-m-ai de cei ce se coboara în groapa.
Psa 30:4  Cânta?i Domnului cei cuvio?i ai Lui ?i lauda?i pomenirea sfin?eniei Lui.
Psa 30:5  Ca iu?ime este întru mânia Lui ?i via?a întru voia Lui; seara se va sala?lui plângerea, iar diminea?a bucuria.
Psa 30:6  Iar eu am zis întru îndestularea mea: "Nu ma voi clatina în veac!"
Psa 30:7  Doamne, întru voia Ta, dat-ai frumuse?ii mele putere; dar când ?i-ai întors fa?a Ta, eu m-am tulburat.
Psa 30:8  Catre Tine, Doamne, voi striga ?i catre Dumnezeul meu ma voi ruga.
Psa 30:9  Ce folos ai de sângele meu de ma cobor în stricaciune? Oare, Te va lauda pe Tine ?arâna, sau va vesti adevarul Tau?
Psa 30:10  Auzit-a Domnul ?i m-a miluit; Domnul a fost ajutorul meu!
Psa 30:11  Schimbat-ai plângerea mea întru bucurie, rupt-ai sacul meu ?i m-ai încins cu veselie.
Psa 30:12  Ca slava mea sa-?i cânte ?ie ?i sa nu ma mâhnesc; Doamne, Dumnezeul meu, în veac Te voi lauda!
Psa 31:1  (Un psalm al lui David; mai-marelui cântare?ilor, pentru uimire.) Spre Tine, Doamne, am nadajduit, sa nu fiu ru?inat în veac. Întru îndreptarea Ta izbave?te-ma ?i ma scoate.
Psa 31:2  Pleaca spre mine urechea Ta, grabe?te de ma scoate. Fii mie Dumnezeu aparator ?i casa de scapare ca sa ma mântuie?ti.
Psa 31:3  Ca puterea mea ?i scaparea mea e?ti Tu ?i pentru numele Tau ma vei pova?ui ?i ma vei hrani.
Psa 31:4  Scoate-ma-vei din cursa aceasta pe care mi-au ascuns-o mie, ca Tu e?ti aparatorul meu.
Psa 31:5  În mâinile Tale îmi voi da duhul meu; izbavitu-m-ai, Doamne, Dumnezeul adevarului.
Psa 31:6  Urât-ai pe cei ce pazesc de?ertaciuni în zadar, iar eu spre Domnul am nadajduit.
Psa 31:7  Bucura-ma-voi ?i ma voi veseli de mila Ta, ca ai cautat spre smerenia mea, mântuit-ai din nevoi sufletul meu
Psa 31:8  ?i nu m-ai lasat în mâinile vrajma?ului; pus-ai în loc desfatat picioarele mele.
Psa 31:9  Miluie?te-ma, Doamne, ca ma necajesc; tulburatu-s-a de mânie ochiul meu, sufletul meu ?i inima mea.
Psa 31:10  Ca s-a stins întru durere via?a mea ?i anii mei în suspinuri; slabit-a întru saracie taria mea ?i oasele mele s-au tulburat.
Psa 31:11  La to?i vrajma?ii mei m-am facut de ocara ?i vecinilor mei foarte, ?i frica cunoscu?ilor mei. Cei ce ma vedeau afara fugeau de mine.
Psa 31:12  Uitat am fost ca un mort din inima lor, ajuns-am ca un vas stricat.
Psa 31:13  Ca am auzit ocara multora din cei ce locuiesc împrejur, când se adunau ei împreuna împotriva mea; ca sa ia sufletul meu s-au sfatuit.
Psa 31:14  Iar eu catre Tine am nadajduit, Doamne, zis-am: "Tu e?ti Dumnezeul meu!"
Psa 31:15  În mâinile Tale, soarta mea, izbave?te-ma din mâna vrajma?ilor mei ?i de cei ce ma prigonesc.
Psa 31:16  Arata fa?a Ta peste robul Tau, mântuie?te-ma cu mila Ta!
Psa 31:17  Doamne, sa nu fiu ru?inat, ca Te-am chemat pe Tine; sa se ru?ineze necredincio?ii ?i sa se coboare în iad.
Psa 31:18  Mute sa fie buzele cele viclene, care graiesc împotriva dreptului faradelege, cu mândrie ?i cu defaimare.
Psa 31:19  Cât este de mare mul?imea bunata?ii Tale, Doamne, pe care ai gatit-o celor ce se tem de Tine, pe care ai facut-o celor ce nadajduiesc în Tine, înaintea fiilor oamenilor!
Psa 31:20  Ascunde-i-vei pe dân?ii cu acoperamântul fe?ei Tale de tulburarea oamenilor. Acoperi-i-vei pe ei în cortul Tau de împotrivirea limbilor.
Psa 31:21  Binecuvântat este Domnul, ca minunata a fost mila Sa, în cetate întarita.
Psa 31:22  Iar eu am zis întru uimirea mea: Lepadat sunt de la fa?a ochilor Tai. Pentru aceasta ai auzit glasul rugaciunii mele când am strigat catre Tine.
Psa 31:23  Iubi?i pe Domnul to?i cuvio?ii Lui ca adevarul cauta Domnul ?i rasplate?te celor ce se mândresc, cu prisosin?a.
Psa 31:24  Îmbarbata?i-va ?i sa se întareasca inima voastra, to?i cei ce nadajdui?i în Domnul.
Psa 32:1  (Un psalm al lui David; pentru pricepere.) Ferici?i carora s-au iertat faradelegile ?i carora s-au acoperit pacatele.
Psa 32:2  Fericit barbatul, caruia nu-i va socoti Domnul pacatul, nici nu este în gura lui vicle?ug.
Psa 32:3  Ca am tacut, îmbatrânit-au oasele mele, când strigam toata ziua.
Psa 32:4  Ca ziua ?i noaptea s-a îngreunat peste mine mâna Ta ?i am cazut în suferin?a când ghimpele Tau ma împungea.
Psa 32:5  Pacatul meu l-am cunoscut ?i faradelegea mea n-am ascuns-o, împotriva mea. Zis-am: "Marturisi-voi faradelegea mea Domnului"; ?i Tu ai iertat nelegiuirea pacatului meu.
Psa 32:6  Pentru aceasta se va ruga catre Tine tot cuviosul la vreme potrivita, iar potop de ape multe de el nu se va apropia.
Psa 32:7  Tu e?ti scaparea mea din necazul ce ma cuprinde, bucuria mea; izbave?te-ma de cei ce m-au înconjurat.
Psa 32:8  În?elep?i-te-voi ?i te voi îndrepta pe calea aceasta, pe care vei merge; a?inti-voi spre tine ochii Mei.
Psa 32:9  Nu fi ca un cal ?i ca un catâr, la care nu este pricepere; cu zabala ?i cu frâu falcile lor voi strânge ca sa nu se apropie de tine.
Psa 32:10  Multe sunt bataile pacatosului; iar pe cel ce nadajduie?te în Domnul, mila îl va înconjura.
Psa 32:11  Veseli?i-va în Domnul ?i va bucura?i, drep?ilor, ?i va lauda?i to?i cei drep?i la inima.
Psa 33:1  (Un psalm al lui David, nescris deasupra la evrei.) Bucura?i-va, drep?ilor; celor drep?i li se cuvine lauda.
Psa 33:2  Lauda?i pe Domnul în alauta, în psaltire cu zece strune cânta?i-I Lui.
Psa 33:3  Cânta?i-I Lui cântare noua, cânta?i-I frumos, cu strigat de bucurie.
Psa 33:4  Ca drept este cuvântul Domnului ?i toate lucrurile Lui întru credin?a.
Psa 33:5  Iube?te milostenia ?i judecata, Domnul; de mila Domnului plin este pamântul.
Psa 33:6  Cu cuvântul Domnului cerurile s-au întarit ?i cu duhul gurii Lui toata puterea lor.
Psa 33:7  Adunat-a ca într-un burduf apele marii, pus-a în vistierii adâncurile.
Psa 33:8  Sa se teama de Domnul tot pamântul ?i de El sa se cutremure to?i locuitorii lumii.
Psa 33:9  Ca El a zis ?i s-au facut, El a poruncit ?i s-au zidit.
Psa 33:10  Domnul risipe?te sfaturile neamurilor, leapada gândurile popoarelor ?i defaima sfaturile capeteniilor.
Psa 33:11  Iar sfatul Domnului ramâne în veac, gândurile inimii Lui, din neam în neam.
Psa 33:12  Fericit este neamul caruia Domnul este Dumnezeul lui, poporul pe care l-a ales de mo?tenire Lui.
Psa 33:13  Din cer a privit Domnul, vazut-a pe to?i fiii oamenilor.
Psa 33:14  Din loca?ul Sau, cel gata, privit-a spre to?i cei ce locuiesc pamântul.
Psa 33:15  Cel ce a zidit îndeosebi inimile lor, Cel ce pricepe toate lucrurile lor.
Psa 33:16  Nu se mântuie?te împaratul cu o?tire multa ?i uria?ul nu se va mântui cu mul?imea tariei lui.
Psa 33:17  Mincinos este calul spre scapare ?i cu mul?imea puterii lui nu te va izbavi.
Psa 33:18  Iata ochii Domnului spre cei ce se tem de Dânsul, spre cei ce nadajduiesc în mila Lui.
Psa 33:19  Ca sa izbaveasca de moarte sufletele lor ?i sa-i hraneasca pe ei în foamete.
Psa 33:20  ?i sufletul nostru a?teapta pe Domnul, ca ajutorul ?i aparatorul nostru este.
Psa 33:21  Ca în El se va veseli inima noastra ?i în numele cel sfânt al Lui am nadajduit.
Psa 33:22  Fie, Doamne, mila Ta spre noi, precum am nadajduit ?i noi întru Tine.
Psa 34:1  (Un psalm al lui David, când ?i-a schimbat fa?a sa înaintea lui Abimelec ?i i-a dat drumul ?i s-a dus.) Bine voi cuvânta pe Domnul în toata vremea, pururea lauda Lui în gura mea.
Psa 34:2  În Domnul se va lauda sufletul meu; sa auda cei blânzi ?i sa se veseleasca.
Psa 34:3  Slavi?i pe Domnul împreuna cu mine ?i sa înal?am numele Lui împreuna.
Psa 34:4  Cautat-am pe Domnul ?i m-a auzit ?i din toate necazurile mele m-a izbavit.
Psa 34:5  Apropia?i-va de El ?i va lumina?i; ?i fe?ele voastre sa nu se ru?ineze.
Psa 34:6  Saracul acesta a strigat ?i Domnul l-a auzit pe el ?i din toate necazurile lui l-a izbavit.
Psa 34:7  Strajui-va îngerul Domnului împrejurul celor ce se tem de El ?i-i va izbavi pe ei.
Psa 34:8  Gusta?i ?i vede?i ca bun este Domnul; fericit barbatul care nadajduie?te în El.
Psa 34:9  Teme?i-va de Domnul to?i sfin?ii Lui, ca n-au lipsa cei ce se tem de El.
Psa 34:10  Boga?ii au saracit ?i au flamânzit, iar cei ce-L cauta pe Domnul, nu se vor lipsi de tot binele.
Psa 34:11  Veni?i fiilor, asculta?i-ma pe mine, frica Domnului va voi înva?a pe voi;
Psa 34:12  Cine este omul cel ce voie?te via?a, care iube?te sa vada zile bune?
Psa 34:13  Opre?te-?i limba de la rau ?i buzele tale sa nu graiasca vicle?ug.
Psa 34:14  Fere?te-te de rau ?i fa bine, cauta pacea ?i o urmeaza pe ea.
Psa 34:15  Ochii Domnului spre cei drep?i ?i urechile Lui spre rugaciunea lor.
Psa 34:16  Iar fa?a Domnului spre cei ce fac rele, ca sa piara de pe pamânt pomenirea lor.
Psa 34:17  Strigat-au drep?ii ?i Domnul i-a auzit ?i din toate necazurile lor i-a izbavit.
Psa 34:18  Aproape este Domnul de cei umili?i la inima ?i pe cei smeri?i cu duhul îi va mântui.
Psa 34:19  Multe sunt necazurile drep?ilor ?i din toate acelea îi va izbavi pe ei Domnul.
Psa 34:20  Domnul paze?te toate oasele lor, nici unul din ele nu se va zdrobi.
Psa 34:21  Moartea pacato?ilor este cumplita ?i cei ce urasc pe cel drept vor gre?i.
Psa 34:22  Mântui-va Domnul sufletele robilor Sai ?i nu vor gre?i to?i cei ce nadajduiesc în El.
Psa 35:1  (Un psalm al lui David.) Judeca, Doamne, pe cei ce-mi fac mie strâmbatate; lupta împotriva celor ce se lupta cu mine;
Psa 35:2  Apuca arma ?i pavaza ?i scoala-Te întru ajutorul meu;
Psa 35:3  Scoate sabia ?i închide calea celor ce ma prigonesc; spune sufletului meu: "Mântuirea ta sunt Eu!"
Psa 35:4  Sa fie ru?ina?i ?i înfrunta?i cei ce cauta sufletul meu; sa se întoarca înapoi ?i sa se ru?ineze cei ce gândesc rau de mine.
Psa 35:5  Sa fie ca praful în fa?a vântului ?i îngerul Domnului sa-i necajeasca.
Psa 35:6  Sa fie calea lor întuneric ?i alunecare ?i îngerul Domnului sa-i prigoneasca.
Psa 35:7  Ca în zadar au ascuns de mine groapa la?ului lor, în de?ert au ocarât sufletul meu.
Psa 35:8  Sa vina asupra lor la?ul pe care nu-l cunosc ?i cursa pe care au ascuns-o sa-i prinda pe ei; ?i chiar în la?ul lor sa cada.
Psa 35:9  Iar sufletul meu sa se bucure de Domnul, sa se veseleasca de mântuirea lui.
Psa 35:10  Toate oasele mele vor zice: Doamne, cine este asemenea ?ie? Cel ce izbave?te pe sarac din mâna celor mai tari decât el ?i pe sarac ?i pe sarman de cei ce-l rapesc pe el.
Psa 35:11  S-au sculat martori nedrep?i ?i de cele ce nu ?tiam m-au întrebat.
Psa 35:12  Rasplatit-au mie rele pentru bune ?i au vlaguit sufletul meu.
Psa 35:13  Iar eu, când ma suparau ei, m-am îmbracat cu sac ?i am smerit cu post sufletul meu ?i rugaciunea mea în sânul meu se va întoarce.
Psa 35:14  Ca ?i cu un vecin, ca ?i cu un frate al nostru a?a de bine m-am purtat; ca ?i cum a? fi jelit ?i m-a? fi întristat, a?a m-am smerit.
Psa 35:15  Dar împotriva mea s-au veselit ?i s-au adunat; adunatu-s-au împotriva mea cu batai ?i n-am ?tiut; risipi?i au fost ?i nu s-au cait.
Psa 35:16  M-au ispitit, cu batjocura m-au batjocorit, au scrâ?nit împotriva mea cu din?ii lor.
Psa 35:17  Doamne, când vei vedea? Întoarce sufletul meu de la fapta lor cea rea, de la lei, via?a mea.
Psa 35:18  Lauda-Te-voi, Doamne, în adunare mare, întru popor numeros Te voi lauda.
Psa 35:19  Sa nu se bucure de. mine cei ce ma du?manesc pe nedrept, cei ce ma urasc în zadar ?i fac semn cu ochii.
Psa 35:20  Ca mie de pace îmi graiau ?i asupra mea vicle?uguri gândeau.
Psa 35:21  Largitu-?i-au împotriva mea gura lor; zis-au: "Bine, bine, vazut-au ochii no?tri".
Psa 35:22  Vazut-ai, Doamne, sa nu taci; Doamne, nu Te departa de la mine.
Psa 35:23  Scoala-Te, Doamne ?i ia aminte spre judecata mea, Dumnezeul meu ?i Domnul meu, spre pricina mea.
Psa 35:24  Judeca-ma dupa dreptatea Ta, Doamne Dumnezeul meu ?i sa nu se bucure de mine.
Psa 35:25  Sa nu zica întru inimile lor: "Bine, bine, sufletului nostru", nici sa zica: "L-am înghi?it pe el".
Psa 35:26  Sa fie ru?ina?i ?i înfrunta?i deodata cei ce se bucura de necazurile mele; sa se îmbrace cu ru?ine ?i ocara cei ce se lauda împotriva mea.
Psa 35:27  Sa se bucure ?i sa se veseleasca cei ce voiesc dreptatea mea ?i sa spuna pururea: "Slavit sa fie Domnul, Cel ce voie?te pacea robului Sau!"
Psa 35:28  ?i limba mea va grai dreptatea Ta, în toata ziua, lauda Ta.
Psa 36:1  (Un psalm al lui David, robul Domnului; mai-marelui cântare?ilor.) Necredin?a calcatorului de lege spune inimii mele, ca nu este într-însul frica de Dumnezeu.
Psa 36:2  El singur se amage?te în ochii sai, când zice ca ar fi urmarind faradelegea ?i ar fi urând-o.
Psa 36:3  Graiurile gurii lui faradelege ?i vicle?ug; n-a vrut sa priceapa ca sa faca bine.
Psa 36:4  Faradelege a gândit în a?ternutul sau, în toata calea cea buna n-a stat ?i rautatea n-a urât.
Psa 36:5  Doamne, în cer este mila Ta ?i adevarul Tau pâna la nori.
Psa 36:6  Dreptatea Ta ca mun?ii lui Dumnezeu, judeca?ile Tale adânc mare; oameni ?i dobitoace vei izbavi Doamne.
Psa 36:7  Ca ai înmul?it mila Ta, Dumnezeule, iar fiii oamenilor în umbra aripilor Tale vor nadajdui.
Psa 36:8  Satura-se-vor din grasimea casei Tale ?i din izvorul desfatarii Tale îi vei adapa pe ei.
Psa 36:9  Ca la Tine este izvorul vie?ii, întru lumina Ta vom vedea lumina.
Psa 36:10  Tinde mila Ta celor ce Te cunosc pe Tine ?i dreptatea Ta celor drep?i la inima.
Psa 36:11  Sa nu vina peste mine picior de mândrie ?i mâna pacato?ilor sa nu ma clatine.
Psa 36:12  Acolo au cazut to?i cei ce lucreaza faradelegea; izgoni?i au fost ?i nu vor putea sa stea.
Psa 37:1  (Un psalm al lui David.) Nu râvni la cei ce viclenesc, nici nu urma pe cei ce fac faradelegea.
Psa 37:2  Caci ca iarba curând se vor usca ?i ca verdea?a ierbii degrab se vor trece.
Psa 37:3  Nadajduie?te în Domnul ?i fa bunatate ?i locuie?te pamântul ?i hrane?te-te cu boga?ia lui.
Psa 37:4  Desfateaza-te în Domnul ?i î?i va împlini ?ie cererile inimii tale.
Psa 37:5  Descopera Domnului calea ta ?i nadajduie?te în El ?i El va împlini.
Psa 37:6  ?i va scoate ca lumina dreptatea ta ?i judecata ca lumina de amiaza.
Psa 37:7  Supune-te Domnului ?i roaga-L pe El; nu râvni dupa cel ce spore?te în calea sa, dupa omul care face nelegiuirea.
Psa 37:8  Parase?te mânia ?i lasa iu?imea; nu cauta sa viclene?ti.
Psa 37:9  Ca cei ce viclenesc de tot vor pieri; iar cei ce a?teapta pe Domnul vor mo?teni pamântul.
Psa 37:10  ?i înca pu?in ?i nu va mai fi pacatosul ?i vei cauta locul lui ?i nu-l vei afla.
Psa 37:11  Iar cei blânzi vor mo?teni pamântul ?i se vor desfata de mul?imea pacii.
Psa 37:12  Pândi-va pacatosul pe cel drept ?i va scrâ?ni asupra lui, cu din?ii sai.
Psa 37:13  Iar Domnul va râde de el, ca mai înainte vede ca va veni ziua lui.
Psa 37:14  Sabia au scos pacato?ii, întins-au arcul lor ca sa doboare pe sarac ?i pe sarman, ca sa junghie pe cei drep?i la inima.
Psa 37:15  Sabia lor sa intre în inima lor ?i arcurile lor sa se frânga.
Psa 37:16  Mai bun este pu?inul celui drept, decât boga?ia multa a pacato?ilor.
Psa 37:17  Ca bra?ele pacato?ilor se vor zdrobi, iar Domnul întare?te pe cei drep?i.
Psa 37:18  Cunoa?te Domnul caile celor fara prihana ?i mo?tenirea lor în veac va fi.
Psa 37:19  Nu se vor ru?ina în vremea cea rea ?i în zilele de foamete se vor satura.
Psa 37:20  Ca pacato?ii vor pieri, iar vrajma?ii Domnului, îndata ce s-au marit ?i s-au înal?at, s-au stins, ca fumul au pierit.
Psa 37:21  Împrumuta pacatosul ?i nu da înapoi, iar dreptul se îndura ?i da.
Psa 37:22  Ca cei ce-L binecuvânteaza pe El vor mo?teni pamântul, iar cei ce-L blesteama pe El, de tot vor pieri.
Psa 37:23  De la Domnul pa?ii omului se îndrepteaza ?i calea lui o va voi foarte.
Psa 37:24  Când va cadea, nu se va zdruncina, ca Domnul întare?te mâna lui.
Psa 37:25  Tânar am fost ?i am îmbatrânit ?i n-am vazut pe cel drept parasit, nici semin?ia lui cerând pâine;
Psa 37:26  Toata ziua dreptul miluie?te ?i împrumuta ?i semin?ia lui binecuvântata va fi.
Psa 37:27  Fere?te-te de rau ?i fa binele ?i vei trai în veacul veacului.
Psa 37:28  Ca Domnul iube?te judecata ?i nu va parasi pe cei cuvio?i ai Sai; în veac vor fi pazi?i. Iar cei fara de lege vor fi izgoni?i ?i semin?ia necredincio?ilor va fi stârpita.
Psa 37:29  Iar drep?ii vor mo?teni pamântul ?i vor locui în veacul veacului pe el.
Psa 37:30  Gura dreptului va deprinde în?elepciunea ?i limba lui va grai judecata.
Psa 37:31  Legea Dumnezeului sau în inima lui ?i nu se vor poticni pa?ii lui.
Psa 37:32  Pânde?te pacatosul pe cel drept ?i cauta sa-l omoare pe el;
Psa 37:33  Iar Domnul nu-l va lasa pe el, în mâinile lui, nici nu-l va osândi, când se va judeca cu el.
Psa 37:34  A?teapta pe Domnul ?i paze?te calea Lui! ?i te va înva?a pe tine ca sa mo?tene?ti pamântul; când vor pieri pacato?ii vei vedea.
Psa 37:35  Vazut-am pe cel necredincios falindu-se ?i înal?ându-se ca cedrii Libanului.
Psa 37:36  ?i am trecut ?i iata nu era ?i l-am cautat pe el ?i nu s-a aflat locul lui.
Psa 37:37  Paze?te nerautatea ?i cauta dreptatea, ca urma?i are omul facator de pace.
Psa 37:38  Iar cei fara de lege vor pieri deodata ?i urma?ii necredincio?ilor vor fi stârpi?i.
Psa 37:39  Iar mântuirea drep?ilor de la Domnul, ca aparatorul lor este în vreme de necaz.
Psa 37:40  ?i-i va ajuta pe ei Domnul ?i-i va izbavi pe ei ?i-i va scoate pe ei din mâna pacato?ilor ?i-i va mântui pe ei ca au nadajduit în El.
Psa 38:1  (Un psalm al lui David; pentru pomenirea sâmbetei.) Doamne, nu cu mânia Ta sa ma mustri pe mine, nici cu iu?imea Ta sa ma cer?i.
Psa 38:2  Ca sage?ile Tale s-au înfipt în mine ?i ai întarit peste mine mâna Ta.
Psa 38:3  Nu este vindecare în trupul meu de la fa?a mâniei Tale; nu este pace în oasele mele de la fa?a pacatelor mele.
Psa 38:4  Ca faradelegile mele au covâr?it capul meu, ca o sarcina grea apasat-au peste mine.
Psa 38:5  Împu?itu-s-au ?i au putrezit ranile mele, de la fa?a nebuniei mele.
Psa 38:6  Chinuitu-m-am ?i m-am gârbovit pâna în sfâr?it, toata ziua mâhnindu-ma umblam.
Psa 38:7  Ca ?alele mele s-au umplut de ocari ?i nu este vindecare în trupul meu.
Psa 38:8  Necajitu-m-am ?i m-am smerit foarte; racnit-am din suspinarea inimii mele.
Psa 38:9  Doamne, înaintea Ta este toata dorirea mea ?i suspinul meu de la Tine nu s-a ascuns.
Psa 38:10  Inima mea s-a tulburat, parasitu-m-a taria mea ?i lumina ochilor mei ?i aceasta nu este cu mine.
Psa 38:11  Prietenii mei ?i vecinii mei în preajma mea s-au apropiat ?i au ?ezut; ?i cei de aproape ai mei departe au stat.
Psa 38:12  ?i se sileau cei ce cautau sufletul meu ?i cei ce cautau cele rele mie graiau de?ertaciuni ?i vicle?uguri toata ziua cugetau.
Psa 38:13  Iar eu ca un surd nu auzeam ?i ca un mut ce nu-?i deschide gura sa.
Psa 38:14  ?i m-am facut ca un om ce nu aude ?i nu are în gura lui mustrari.
Psa 38:15  Ca spre Tine, Doamne, am nadajduit; Tu ma vei auzi, Doamne, Dumnezeul meu,
Psa 38:16  Ca am zis, ca nu cumva sa se bucure de mine vrajma?ii mei; ?i când s-au clatinat picioarele mele, împotriva mea s-au seme?it.
Psa 38:17  Ca eu spre batai gata sunt ?i durerea mea înaintea mea este pururea.
Psa 38:18  Ca faradelegea mea eu o voi vesti  ?i ma voi îngriji pentru pacatul meu;
Psa 38:19  Iar vrajma?ii mei traiesc ?i s-au întarit mai mult decât mine ?i s-au înmul?it cei ce ma urasc pe nedrept.
Psa 38:20  Cei ce îmi rasplatesc rele pentru bune, ma defaimau, ca urmam bunatatea.
Psa 38:21  Nu ma lasa, Doamne Dumnezeul meu, nu Te departa de la mine;
Psa 38:22  Ia aminte spre ajutorul meu, Doamne al mântuirii mele.
Psa 39:1  (Întru sfâr?it, lui Iditum, o cântare a lui David.) Zis-am: "Pazi-voi caile mele, ca sa nu pacatuiesc eu cu limba mea; pus-am gurii mele paza, când a stat pacatosul împotriva mea".
Psa 39:2  Amu?it-am ?i m-am smerit ?i nici de bine n-am grait ?i durerea mea s-a înnoit.
Psa 39:3  Înfierbântatu-s-a inima mea înauntrul meu ?i în cugetul meu se va aprinde foc.
Psa 39:4  Grait-am cu limba mea: "Fa-mi cunoscut, Doamne, sfâr?itul meu, ?i numarul zilelor mele care este, ca sa ?tiu ce-mi lipse?te".
Psa 39:5  Iata, cu palma ai masurat zilele mele ?i statul meu ca nimic înaintea Ta. Dar toate sunt de?ertaciuni; tot omul ce viaza.
Psa 39:6  De?i ca o umbra trece omul, dar în zadar se tulbura. Strânge comori ?i nu ?tie cui le aduna pe ele.
Psa 39:7  ?i acum cine este rabdarea mea? Oare, nu Domnul? ?i statul meu de la Tine este.
Psa 39:8  De toate faradelegile mele izbave?te-ma; ocara celui fara de minte nu ma da.
Psa 39:9  Amu?it-am ?i n-am deschis gura mea, ca Tu e?ti Cel ce m-ai facut pe mine.
Psa 39:10  Departeaza de la mine bataile Tale. De taria mâinii Tale, eu m-am sfâr?it.
Psa 39:11  Cu mustrari pentru faradelege ai pedepsit pe om ?i ai sub?iat ca pânza de paianjen sufletul sau; dar în de?ert se tulbura tot pamânteanul.
Psa 39:12  Auzi rugaciunea mea, Doamne, ?i cererea mea ascult-o; lacrimile mele sa nu le treci, caci strain sunt eu la Tine ?i strain ca to?i parin?ii mei.
Psa 39:13  Lasa-ma ca sa ma odihnesc, mai înainte de a ma duce ?i de a nu mai fi.
Psa 40:1  (Întru sfâr?it, un psalm al lui David.) A?teptând am a?teptat pe Domnul ?i S-a plecat spre mine.
Psa 40:2  A auzit rugaciunea mea. M-a scos din groapa ticalo?iei ?i din tina noroiului ?i a pus pe piatra picioarele mele ?i a îndreptat pa?ii mei.
Psa 40:3  ?i a pus în gura mea cântare noua, cântare Dumnezeului nostru; vedea-vor mul?i ?i se vor teme ?i vor nadajdui în Domnul.
Psa 40:4  Fericit barbatul, a carui nadejde este numele Domnului ?i n-a privit la de?ertaciuni ?i la nebunii mincinoase.
Psa 40:5  Multe ai facut Tu, Doamne, Dumnezeul meu, minunile Tale, ?i nu este cine sa se asemene gândurilor Tale; vestit-am ?i am grait: înmul?itu-s-au peste numar.
Psa 40:6  Jertfa ?i prinos n-ai voit, dar trup mi-ai întocmit; ardere de tot ?i jertfa pentru pacat n-ai cerut.
Psa 40:7  Atunci am zis: "Iata vin! În capul car?ii este scris despre mine.
Psa 40:8  Ca sa fac voia Ta, Dumnezeul meu, am voit ?i legea Ta înauntru inimii mele".
Psa 40:9  Bine am vestit dreptate în adunare mare; iata buzele mele nu le voi opri; Doamne, Tu ai cunoscut.
Psa 40:10  Dreptatea Ta n-am ascuns-o în inima mea, adevarul Tau ?i mântuirea Ta am spus. N-am ascuns mila Ta ?i adevarul Tau în adunare mare.
Psa 40:11  Iar Tu, Doamne, sa nu departezi îndurarile Tale de la mine, mila Ta ?i adevarul Tau pururea sa ma sprijineasca.
Psa 40:12  Ca m-au împresurat rele, carora nu este numar; ajunsu-m-au faradelegile mele ?i n-am putut sa vad; înmul?itu-s-au mai mult decât perii capului meu ?i inima mea m-a parasit.
Psa 40:13  Binevoie?te, Doamne, ca sa ma izbave?ti; Doamne, spre ajutorul meu ia aminte.
Psa 40:14  Sa fie ru?ina?i ?i înfrunta?i deodata cei ce cauta sa ia sufletul meu. Sa se întoarca înapoi ?i sa se ru?ineze cei ce-mi voiesc mie rele;
Psa 40:15  Sa fie ru?ina?i îndata cei ce-mi zic mie: "Bine, bine".
Psa 40:16  Sa se bucure ?i sa se veseleasca de Tine, to?i cei ce Te cauta pe Tine, Doamne, ?i sa zica pururea cei ce iubesc mântuirea Ta: "Slavit sa fie Domnul!"
Psa 40:17  Iar eu sarac sunt ?i sarman; Domnul se va îngriji de mine. Ajutorul meu ?i aparatorul meu e?ti Tu; Dumnezeul meu nu zabovi.
Psa 41:1  (Întru sfâr?it, un psalm al lui David.) Fericit cel care cauta la sarac ?i la sarman; în ziua cea rea îl va izbavi pe el Domnul.
Psa 41:2  Domnul sa-l pazeasca pe el ?i sa-l vieze ?i sa-l fericeasca pe pamânt ?i sa nu-l dea în mâinile vrajma?ilor lui.
Psa 41:3  Domnul sa-l ajute pe el pe patul durerii lui; în a?ternutul bolii lui sa-l întareasca pe el.
Psa 41:4  Eu am zis: "Doamne, miluie?te-ma; vindeca sufletul meu, ca am gre?it ?ie".
Psa 41:5  Vrajma?ii mei m-au grait de rau zicând: "Când va muri ?i va pieri numele lui?"
Psa 41:6  Iar de venea cineva sa ma vada, minciuni graia; inima lui aduna faradelege sie?i, ie?ea afara ?i graia.
Psa 41:7  Împreuna împotriva mea ?opteau to?i vrajma?ii mei; împotriva mea gândeau de mine rele.
Psa 41:8  Cuvânt nelegiuit spuneau împotriva mea, zicând: "Nu zace, oare? Nu se va mai scula!"
Psa 41:9  Chiar omul cu care eram în pace, în care am nadajduit, care a mâncat pâinea mea, a ridicat împotriva mea calcâiul.
Psa 41:10  Iar Tu, Doamne, miluie?te-ma ?i ma scoala ?i voi rasplati lor.
Psa 41:11  Întru aceasta am cunoscut ca m-ai voit, ca nu se va bucura vrajma?ul meu de mine.
Psa 41:12  Iar pe mine pentru nerautatea mea m-ai sprijinit ?i m-ai întarit înaintea Ta, în veac.
Psa 41:13  Binecuvântat este Domnul Dumnezeul lui Israel din veac ?i pâna în veac. Amin. Amin.
Psa 42:1  (Un psalm al lui David, spre în?elep?ire fiilor lui Core.) În ce chip dore?te cerbul izvoarele apelor, a?a Te dore?te sufletul meu pe Tine, Dumnezeule.
Psa 42:2  Însetat-a sufletul meu de Dumnezeul cel viu; când voi veni ?i ma voi arata fe?ei lui Dumnezeu?
Psa 42:3  Facutu-mi-s-au lacrimile mele pâine ziua ?i noaptea, când mi se zicea mie în toate zilele: "Unde este Dumnezeul tau?"
Psa 42:4  De acestea mi-am adus aminte cu revarsare de inima, când treceam cu mul?ime mare spre casa lui Dumnezeu, în glas de bucurie ?i de lauda ?i în sunet de sarbatoare.
Psa 42:5  Pentru ce e?ti mâhnit, suflete al meu, ?i pentru ce ma tulburi? Nadajduie?te în Dumnezeu, ca-L voi lauda pe El; mântuirea fe?ei mele este Dumnezeul meu.
Psa 42:6  În mine sufletul meu s-a tulburat; pentru aceasta îmi voi aduce aminte de Tine, din pamântul Iordanului ?i al Ermonului, din muntele cel mic.
Psa 42:7  Adânc pe adânc cheama în glasul caderilor apelor Tale. Toate talazurile ?i valurile Tale peste mine au trecut.
Psa 42:8  Ziua va porunci Domnul milei Sale, iar noaptea cântare Lui de la mine.
Psa 42:9  Rugaciunea Dumnezeului vie?ii mele, spune-voi lui Dumnezeu: "Sprijinitorul meu e?ti Tu, pentru ce m-ai uitat? Pentru ce umblu mâhnit când ma necaje?te vrajma?ul meu?"
Psa 42:10  Când se sfarâmau oasele mele ma ocarau asupritorii mei. Când îmi ziceau mie în toate zilele: "Unde este Dumnezeul tau?"
Psa 42:11  Pentru ce e?ti mâhnit, suflete al meu, ?i pentru ce ma tulburi? Nadajduie?te în Dumnezeu, ca-L voi lauda pe El; mântuirea fe?ei mele este Dumnezeul meu.
Psa 43:1  (Un psalm al lui David, nescris deasupra la evrei.) Judeca-ma, Dumnezeule, ?i apara dreptatea mea de neamul necuvios, de omul nedrept ?i viclean, ?i izbave?te-ma.
Psa 43:2  Ca Tu e?ti, Dumnezeule, întarirea mea; pentru ce m-ai lepadat? Pentru ce umblu mâhnit când ma necaje?te vrajma?ul meu?
Psa 43:3  Trimite lumina Ta ?i adevarul Tau; acestea m-au pova?uit ?i m-au condus la muntele cel sfânt al Tau ?i la loca?urile Tale.
Psa 43:4  ?i voi intra la jertfelnicul lui Dumnezeu, la Dumnezeul Cel ce vesele?te tinere?ile mele; lauda-Te-voi în alauta, Dumnezeule, Dumnezeul meu.
Psa 43:5  Pentru ce e?ti mâhnit, suflete al meu, ?i pentru ce ma tulburi? Nadajduie?te în Dumnezeu ca-L voi lauda pe El; mântuirea fe?ei mele este Dumnezeul meu.
Psa 44:1  (Întru sfâr?it, fiilor lui Core, spre în?elegere.) Dumnezeule, cu urechile noastre am auzit, parin?ii no?tri ne-au spus noua lucrul pe care l-ai facut în zilele lor, în zilele cele de demult.
Psa 44:2  Mâna Ta popoare a nimicit, iar pe parin?i i-ai sadit; batut-ai popoare, iar pe ei i-ai înmul?it.
Psa 44:3  Ca nu cu sabia lor au mo?tenit pamântul ?i bra?ul lor nu i-a izbavit pe ei, ci dreapta Ta ?i bra?ul Tau ?i luminarea fe?ei Tale, ca bine ai voit întru ei.
Psa 44:4  Tu e?ti Însu?i Împaratul meu ?i Dumnezeul meu, Cel ce porunce?ti mântuirea lui Iacob;
Psa 44:5  Cu Tine pe vrajma?ii no?tri îi vom lovi ?i cu numele Tau vom nimici pe cei ce se scoala asupra noastra.
Psa 44:6  Pentru ca nu în arcul meu voi nadajdui ?i sabia mea nu ma va mântui.
Psa 44:7  Ca ne-ai izbavit pe noi de cei ce ne necajesc pe noi ?i pe cei ce ne urasc pe noi i-ai ru?inat.
Psa 44:8  Cu Dumnezeu ne vom lauda toata ziua ?i numele Tau îl vom lauda în veac.
Psa 44:9  Iar acum ne-ai lepadat ?i ne-ai ru?inat pe noi ?i nu vei ie?i cu o?tirile noastre;
Psa 44:10  Întorsu-ne-ai pe noi înapoi de la du?manii no?tri ?i cei ce ne urasc pe noi ne-au jefuit.
Psa 44:11  Datu-ne-ai pe noi ca oi de mâncare ?i întru neamuri ne-ai risipit;
Psa 44:12  Vândut-ai pe poporul Tau fara de pre? ?i nu l-ai pre?uit când l-ai vândut.
Psa 44:13  Pusu-ne-ai pe noi ocara vecinilor no?tri, batjocura ?i râs celor dimprejurul nostru;
Psa 44:14  Pusu-ne-ai pe noi pilda catre neamuri, clatinare de cap între popoare.
Psa 44:15  Toata ziua înfruntarea mea înaintea mea este ?i ru?inea obrazului meu m-a acoperit,
Psa 44:16  De catre glasul celui ce ocara?te ?i clevete?te, de catre fa?a vrajma?ului ?i prigonitorului.
Psa 44:17  Acestea toate au venit peste noi ?i nu Te-am uitat ?i n-am calcat legamântul Tau
Psa 44:18  ?i nu s-a dat înapoi inima noastra; iar pa?ii no?tri nu s-au abatut de la calea Ta,
Psa 44:19  Ca ne-ai smerit pe noi în loc de durere ?i ne-a acoperit pe noi umbra mor?ii.
Psa 44:20  De am fi uitat numele Dumnezeului nostru ?i am fi întins mâinile noastre spre dumnezeu strain,
Psa 44:21  Oare, Dumnezeu n-ar fi cercetat acestea? Ca El ?tie ascunzi?urile inimii.
Psa 44:22  Ca pentru Tine suntem uci?i toata ziua, socoti?i am fost ca ni?te oi de junghiere.
Psa 44:23  De?teapta-Te, pentru ce dormi, Doamne? Scoala-Te ?i nu ne lepada pâna în sfâr?it.
Psa 44:24  Pentru ce întorci fa?a Ta? Ui?i de saracia noastra ?i de necazul nostru?
Psa 44:25  Ca s-a plecat în ?arâna sufletul nostru, lipitu-s-a de pamânt pântecele nostru.
Psa 44:26  Scoala-Te, Doamne, ajuta-ne noua ?i ne izbave?te pe noi, pentru numele Tau.
Psa 45:1  (Un psalm al lui David, pentru cei ce se vor schimba. Fiilor lui Core, spre în?elegere, cântare pentru cel iubit.) Cuvânt bun raspuns-a inima mea; grai-voi cântarea mea Împaratului. Limba mea este trestie de scriitor ce scrie iscusit.
Psa 45:2  Împodobit e?ti cu frumuse?ea mai mult decât fiii oamenilor; revarsatu-s-a har pe buzele tale. Pentru aceasta te-a binecuvântat pe tine Dumnezeu, în veac.
Psa 45:3  Încinge-te cu sabia ta peste coapsa ta, puternice,
Psa 45:4  Cu frumuse?ea ta ?i cu stralucirea ta. Încordeaza-?i arcul, propa?e?te ?i împara?e?te, pentru adevar, blânde?e ?i dreptate, ?i te va pova?ui minunat dreapta ta.
Psa 45:5  Sage?ile tale ascu?ite sunt puternice în inima du?manilor împaratului; popoarele sub tine vor cadea.
Psa 45:6  Scaunul Tau, Dumnezeule, în veacul veacului, toiag de dreptate toiagul împara?iei Tale.
Psa 45:7  Iubit-ai dreptatea ?i ai urât faradelegea; pentru aceasta Te-a uns pe Tine, Dumnezeul Tau, cu untdelemnul bucuriei, mai mult decât pe parta?ii Tai.
Psa 45:8  Smirna ?i aloea îmbalsameaza ve?mintele Tale; din palate de filde? cântari de alauta Te veselesc; fiice de împara?i întru cinstea Ta;
Psa 45:9  Statut-a împarateasa de-a dreapta Ta, îmbracata în haina aurita ?i prea înfrumuse?ata.
Psa 45:10  Asculta fiica ?i vezi ?i pleaca urechea ta ?i uita poporul tau ?i casa parintelui tau,
Psa 45:11  Ca a poftit Împaratul frumuse?ea ta, ca El este Domnul tau.
Psa 45:12  ?i se vor închina Lui fiicele Tirului cu daruri, fe?ei Tale se vor ruga mai-marii poporului.
Psa 45:13  Toata slava fiicei Împaratului este înauntru, îmbracata cu ?esaturi de aur ?i prea înfrumuse?ata.
Psa 45:14  Aduce-se-vor Împaratului fecioare în urma ei, prietenele ei se vor aduce ?ie.
Psa 45:15  Aduce-se-vor întru veselie ?i bucurie, aduce-se-vor în palatul Împaratului.
Psa 45:16  În locul parin?ilor tai s-au nascut ?ie fii; pune-i-vei pe ei capetenii peste tot pamântul.
Psa 45:17  Pomeni-vor numele tau în tot neamul; pentru aceasta popoarele te vor lauda în veac ?i în veacul veacului.
Psa 46:1  (Un psalm al lui David, fiilor lui Core, pentru cele ascunse.) Dumnezeu este scaparea ?i puterea noastra, ajutor întru necazurile ce ne împresoara.
Psa 46:2  Pentru aceasta nu ne vom teme când se va cutremura pamântul ?i se vor muta mun?ii în inima marilor.
Psa 46:3  Venit-au ?i s-au tulburat apele lor, cutremuratu-s-au mun?ii de taria lui.
Psa 46:4  Apele râurilor veselesc cetatea lui Dumnezeu; Cel Preaînalt a sfin?it loca?ul Lui.
Psa 46:5  Dumnezeu este în mijlocul ceta?ii, nu se va clatina; o va ajuta Dumnezeu dis-de-diminea?a.
Psa 46:6  Tulburatu-s-au neamurile, plecatu-s-au împara?iile; dat-a Cel Preaînalt glasul Lui, cutremuratu-s-a pamântul.
Psa 46:7  Domnul puterilor cu noi, sprijinitorul nostru, Dumnezeul lui Iacob.
Psa 46:8  Veni?i ?i vede?i lucrurile lui Dumnezeu, minunile pe care le-a pus Domnul pe pamânt.
Psa 46:9  Pune-va capat razboaielor pâna la marginile pamântului, arcul va sfarâma ?i va frânge arma, iar pavezele în foc le va arde.
Psa 46:10  Opri?i-va ?i cunoa?te?i ca Eu sunt Dumnezeu, înal?a-Ma-voi pe pamânt.
Psa 46:11  Domnul puterilor cu noi, sprijinitorul nostru, Dumnezeul lui Iacob.
Psa 47:1  (Mai-marelui cântare?ilor; un psalm pentru fiii lui Core.) Toate popoarele bate?i din palme, striga?i lui Dumnezeu cu glas de bucurie.
Psa 47:2  Ca Domnul este Preaînalt, înfrico?ator, Împarat mare peste tot pamântul.
Psa 47:3  Supusu-ne-a noua popoare ?i neamuri sub picioarele noastre;
Psa 47:4  Alesu-ne-a noua mo?tenirea Sa, frumuse?ea lui Iacob, pe care a iubit-o.
Psa 47:5  Suitu-S-a Dumnezeu întru strigare, Domnul în glas de trâmbi?a.
Psa 47:6  Cânta?i Dumnezeului nostru, cânta?i; cânta?i Împaratului nostru, cânta?i.
Psa 47:7  Ca Împarat a tot pamântul este Dumnezeu; cânta?i cu în?elegere.
Psa 47:8  Împara?it-a Dumnezeu peste neamuri, Dumnezeu ?ade pe tronul cel sfânt al Sau.
Psa 47:9  Mai-marii popoarelor s-au adunat cu poporul Dumnezeului lui Avraam, ca ai lui Dumnezeu sunt puternicii pamântului; El S-a înal?at foarte.
Psa 48:1  (O cântare de psalm, pentru fiii lui Core, pentru a doua zi a sâmbetei.) Mare este Domnul ?i laudat foarte în cetatea Dumnezeului nostru, în muntele cel sfânt al Lui;
Psa 48:2  Bine întemeiata spre bucuria întregului pamânt. Muntele Sionului, coastele de miazanoapte, cetatea Împaratului Celui mare.
Psa 48:3  Dumnezeu în palatele ei se cunoa?te, când o apara pe ea.
Psa 48:4  Ca iata împara?ii s-au adunat, strânsu-s-au împreuna.
Psa 48:5  Ace?tia vazând-o a?a, s-au minunat, s-au tulburat, s-au cutremurat;
Psa 48:6  Cutremur i-a cuprins pe ei acolo; dureri ca ale celei ce na?te.
Psa 48:7  Cu vânt puternic va sfarâma corabiile Tarsisului.
Psa 48:8  Precum am auzit, a?a am ?i vazut, în cetatea Domnului puterilor, în cetatea Dumnezeului nostru.
Psa 48:9  Dumnezeu a întemeiat-o pe ea în veac. Primit-am, Dumnezeule, mila Ta, în mijlocul loca?ului Tau.
Psa 48:10  Dupa numele Tau, Dumnezeule, a?a ?i lauda Ta peste marginile pamântului; dreapta Ta este plina de dreptate.
Psa 48:11  Sa se veseleasca Muntele Sionului, sa se bucure fiicele lui Iuda pentru judeca?ile Tale, Doamne.
Psa 48:12  Înconjura?i Sionul ?i-l cuprinde?i pe el, povesti?i despre turnurile lui.
Psa 48:13  Pune?i-va inimile voastre întru puterea lui ?i strabate?i palatele lui, ca sa povesti?i neamului ce vine,
Psa 48:14  Ca Acesta este Dumnezeu, Dumnezeul nostru în veac ?i în veacul veacului; El ne va pa?te pe noi în veci.
Psa 49:1  (Mai-marelui cântare?ilor; un psalm pentru fiii lui Core.) Auzi?i acestea toate neamurile, asculta?i to?i cei ce locui?i în lume:
Psa 49:2  Pamântenii ?i fiii oamenilor, împreuna bogatul ?i saracul.
Psa 49:3  Gura mea va grai în?elepciune ?i cugetul inimii mele pricepere.
Psa 49:4  Pleca-voi spre pilda urechea mea, tâlcui-voi în sunet de psaltire gândul meu.
Psa 49:5  Pentru ce sa ma tem în ziua cea rea, când ma va înconjura faradelegea vrajma?ilor mei?
Psa 49:6  Ei se încred în puterea lor ?i cu mul?imea boga?iei lor se lauda.
Psa 49:7  Nimeni însa nu poate sa scape de la moarte, nici sa plateasca lui Dumnezeu pre? de rascumparare,
Psa 49:8  Ca rascumpararea sufletului e prea scumpa ?i niciodata nu se va putea face,
Psa 49:9  Ca sa ramâna cineva pe totdeauna viu ?i sa nu vada niciodata moartea.
Psa 49:10  Fiecare vede ca în?elep?ii mor, cum mor ?i cei neîn?elep?i ?i nebunii, ?i lasa altora boga?ia lor.
Psa 49:11  Mormântul lor va fi casa lor în veac, loca?urile lor din neam în neam, de?i numit-au cu numele lor pamânturile lor.
Psa 49:12  ?i omul, în cinste fiind, n-a priceput; alaturatu-s-a dobitoacelor celor fara de minte ?i s-a asemanat lor.
Psa 49:13  Aceasta cale le este sminteala lor ?i celor ce vor gasi de bune spusele lor.
Psa 49:14  Ca ni?te oi în iad sunt pu?i, moartea îi va pa?te pe ei. ?i-i vor stapâni pe ei cei drep?i ?i ajutorul ce-l nadajduiau din slava lor, se va învechi în iad.
Psa 49:15  Dar Dumnezeu va izbavi sufletul meu din mâna iadului, când ma va apuca.
Psa 49:16  Sa nu te temi când se îmboga?e?te omul ?i când se înmul?e?te slava casei lui.
Psa 49:17  Ca la moarte el nu va lua nimic, nici nu se va coborî cu el slava lui.
Psa 49:18  Chiar daca sufletul lui se va binecuvânta în via?a lui ?i te va lauda când îi vei face bine,
Psa 49:19  Totu?i intra-va pâna la neamul parin?ilor lui ?i în veac nu va vedea lumina.
Psa 49:20  Omul în cinste fiind n-a priceput; alaturatu-s-a dobitoacelor celor fara de minte ?i s-a asemanat lor.
Psa 50:1  (Un psalm al lui Asaf.) Dumnezeul dumnezeilor, Domnul a grait ?i a chemat pamântul,
Psa 50:2  De la rasaritul soarelui pâna la apus. Din Sion este stralucirea frumuse?ii Lui.
Psa 50:3  Dumnezeu stralucit va veni, Dumnezeul nostru, ?i nu va tacea. Foc înaintea Lui va arde ?i împrejurul Lui vifor mare.
Psa 50:4  Chema-va cerul de sus ?i pamântul, ca sa judece pe poporul Sau.
Psa 50:5  Aduna?i-I Lui pe cuvio?ii Lui, pe cei ce au facut legamânt cu El pentru jertfe.
Psa 50:6  ?i vor vesti cerurile dreptatea Lui, ca Dumnezeu judecator este.
Psa 50:7  "Asculta, poporul Meu ?i-?i voi grai ?ie, Israele!... ?i voi marturisi ?ie: Dumnezeu, Dumnezeul tau sunt Eu.
Psa 50:8  Nu pentru jertfele tale te voi mustra, iar arderile de tot ale tale înaintea Mea sunt pururea.
Psa 50:9  Nu voi primi din casa ta vi?ei, nici din turmele tale ?api,
Psa 50:10  Ca ale Mele sunt toate fiarele câmpului, dobitoacele din mun?i ?i boii.
Psa 50:11  Cunoscut-am toate pasarile cerului ?i frumuse?ea ?arinii cu Mine este.
Psa 50:12  De voi flamânzi, nu-?i voi spune ?ie, ca a Mea este lumea ?i plinirea ei.
Psa 50:13  Oare, voi mânca carne de taur, sau sânge de ?api voi bea?
Psa 50:14  Jertfe?te lui Dumnezeu jertfa de lauda ?i împline?te Celui Preaînalt fagaduin?ele tale.
Psa 50:15  ?i Ma cheama pe Mine în ziua necazului ?i te voi izbavi ?i Ma vei preaslavi".
Psa 50:16  Iar pacatosului i-a zis Dumnezeu: "Pentru ce tu istorise?ti drepta?ile Mele ?i iei legamântul Meu în gura ta?
Psa 50:17  Tu ai urât înva?atura ?i ai lepadat cuvintele Mele înapoia ta.
Psa 50:18  De vedeai furul, alergai cu el ?i cu cel desfrânat partea ta puneai.
Psa 50:19  Gura ta a înmul?it rautate ?i limba ta a împletit vicle?ug.
Psa 50:20  ?ezând împotriva fratelui tau cleveteai ?i împotriva fiului maicii tale ai pus sminteala.
Psa 50:21  Acestea ai facut ?i am tacut, ai cugetat faradelegea, ca voi fi asemenea ?ie; mustra-te-voi ?i voi pune înaintea fe?ei tale pacatele tale.
Psa 50:22  În?elege?i dar acestea cei ce uita?i pe Dumnezeu, ca nu cumva sa va rapeasca ?i sa nu fie cel ce izbave?te.
Psa 50:23  Jertfa de lauda Ma va slavi ?i acolo este calea în care voi arata lui mântuirea Mea".
Psa 51:1  (Mai-marelui cântare?ilor; un psalm al lui David, când a venit profetul Natan, ca sa-l mustre pentru femeia lui Urie.) Miluie?te-ma, Dumnezeule, dupa mare mila Ta ?i dupa mul?imea îndurarilor Tale, ?terge faradelegea mea.
Psa 51:2  Mai vârtos ma spala de faradelegea mea ?i de pacatul meu ma cura?e?te.
Psa 51:3  Ca faradelegea mea eu o cunosc ?i pacatul meu înaintea mea este pururea.
Psa 51:4  ?ie unuia am gre?it ?i rau înaintea Ta am facut, a?a încât drept e?ti Tu întru cuvintele Tale ?i biruitor când vei judeca Tu.
Psa 51:5  Ca iata întru faradelegi m-am zamislit ?i în pacate m-a nascut maica mea.
Psa 51:6  Ca iata adevarul ai iubit; cele nearatate ?i cele ascunse ale în?elepciunii Tale, mi-ai aratat mie.
Psa 51:7  Stropi-ma-vei cu isop ?i ma voi cura?i; spala-ma-vei ?i mai vârtos decât zapada ma voi albi.
Psa 51:8  Auzului meu vei da bucurie ?i veselie; bucura-se-vor oasele mele cele smerite.
Psa 51:9  Întoarce fa?a Ta de la pacatele mele ?i toate faradelegile mele ?terge-le.
Psa 51:10  Inima curata zide?te întru mine, Dumnezeule ?i duh drept înnoie?te întru cele dinlauntru ale mele.
Psa 51:11  Nu ma lepada de la fa?a Ta ?i Duhul Tau cel sfânt nu-l lua de la mine.
Psa 51:12  Da-mi mie bucuria mântuirii Tale ?i cu duh stapânitor ma întare?te.
Psa 51:13  Înva?a-voi pe cei fara de lege caile Tale ?i cei necredincio?i la Tine se vor întoarce.
Psa 51:14  Izbave?te-ma de varsarea de sânge, Dumnezeule, Dumnezeul mântuirii mele; bucura-se-va limba mea de dreptatea Ta.
Psa 51:15  Doamne, buzele mele vei deschide ?i gura mea va vesti lauda Ta.
Psa 51:16  Ca de ai fi voit jertfa, ?i-a? fi dat; arderile de tot nu le vei binevoi.
Psa 51:17  Jertfa lui Dumnezeu: duhul umilit; inima înfrânta ?i smerita Dumnezeu nu o va urgisi.
Psa 51:18  Fa bine, Doamne, întru buna voirea Ta, Sionului, ?i sa se zideasca zidurile Ierusalimului.
Psa 51:19  Atunci vei binevoi jertfa drepta?ii, prinosul ?i arderile de tot; atunci vor pune pe altarul Tau vi?ei.
Psa 52:1  (Mai-marelui cântare?ilor; un psalm al lui David, când a venit Doig Idumeul ?i a vestit lui Saul ?i i-a zis: venit-a David în casa lui Abimelec.) Ce te fale?ti întru rautate, puternice?
Psa 52:2  Faradelege toata ziua, nedreptate a vorbit limba ta; ca un brici ascu?it a facut vicle?ug.
Psa 52:3  Iubit-ai rautatea mai mult decât bunatatea, nedreptatea mai mult decât a grai dreptatea.
Psa 52:4  Iubit-ai toate cuvintele pierzarii, limba vicleana!
Psa 52:5  Pentru aceasta Dumnezeu te va doborî pâna în sfâr?it, te va smulge ?i te va muta din loca?ul tau ?i radacina ta din pamântul celor vii.
Psa 52:6  Vedea-vor drep?ii ?i se vor teme ?i de el vor râde ?i vor zice: "Iata omul care nu ?i-a pus pe Dumnezeu ajutorul lui,
Psa 52:7  Ci a nadajduit în mul?imea boga?iei sale ?i s-a întarit întru de?ertaciunea sa".
Psa 52:8  Dar eu, ca un maslin roditor în casa lui Dumnezeu, am nadajduit în mila lui Dumnezeu, în veac ?i în veacul veacului.
Psa 52:9  Slavi-Te-voi în veac, ca ai facut aceasta ?i voi a?tepta numele Tau, ca bun este înaintea cuvio?ilor Tai.
Psa 53:1  (Un psalm al lui David; mai-marelui cântare?ilor, pentru maelet, un instrument muzical; un psalm al priceperii.) Zis-a cel nebun întru inima sa: "Nu este Dumnezeu!" Stricatu-s-au ?i urâ?i s-au facut întru faradelegi. Nu este cel ce face bine.
Psa 53:2  Domnul din cer a privit peste fiii oamenilor, sa vada de este cel ce în?elege, sau cel ce cauta pe Dumnezeu.
Psa 53:3  To?i s-au abatut, împreuna netrebnici s-au facut; nu este cel ce face bine, nu este pâna la unul.
Psa 53:4  Oare, nu vor cunoa?te, to?i cei ce lucreaza faradelegea? Cei ce manânca pe poporul Meu cum manânca pâinea,
Psa 53:5  Pe Domnul nu L-au chemat. Acolo s-au temut de frica unde nu era frica, ca Dumnezeu a risipit oasele celor ce plac oamenilor; ru?inatu-s-au, ca Dumnezeu i-a urgisit pe ei.
Psa 53:6  Cine va da din Sion mântuirea lui Israel? Când va întoarce Domnul pe cei robi?i ai poporului Sau, bucura-se-va Iacob ?i se va veseli Israel.
Psa 54:1  (Un psalm al lui David; mai-marelui cântare?ilor, pentru instrumente cu coarde; un psalm al priceperii. Când au venit zefeii ?i au zis lui Saul: Iata, oare, nu s-a ascuns David la noi?) Dumnezeule, întru numele Tau mântuie?te-ma ?i întru puterea Ta ma judeca.
Psa 54:2  Dumnezeule, auzi rugaciunea mea, ia aminte cuvintele gurii mele!
Psa 54:3  Ca strainii s-au ridicat împotriva mea ?i cei tari au cautat sufletul meu ?i n-au pus pe Dumnezeu înaintea lor.
Psa 54:4  Dar iata, Dumnezeu ajuta mie ?i Domnul este sprijinul sufletului meu.
Psa 54:5  Întoarce-va cele rele vrajma?ilor mei; cu adevarul Tau îi vei pierde pe ei.
Psa 54:6  De bunavoie voi jertfi ?ie; lauda-voi numele Tau, Doamne, ca este bun,
Psa 54:7  Ca din tot necazul m-ai izbavit ?i spre vrajma?ii mei a privit ochiul meu.
Psa 55:1  (Un psalm al lui David; mai-marelui cântare?ilor, pentru instrumente cu coarde; un psalm al priceperii.) Auzi, Dumnezeule, rugaciunea mea ?i nu trece cu vederea ruga mea.
Psa 55:2  Ia aminte spre mine ?i ma asculta; mâhnitu-m-am întru nelini?tea mea ?i m-am tulburat de glasul vrajma?ului ?i de necazul pacatosului.
Psa 55:3  Ca a abatut asupra mea faradelege ?i întru mânie m-a vrajma?it.
Psa 55:4  Inima mea s-a tulburat întru mine ?i frica mor?ii a cazut peste mine;
Psa 55:5  Teama ?i cutremur au venit asupra mea ?i m-a acoperit întunericul.
Psa 55:6  ?i am zis: Cine-mi va da mie aripi ca de porumbel, ca sa zbor ?i sa ma odihnesc?
Psa 55:7  Iata m-a? îndeparta fugind ?i m-a? sala?lui în pustiu.
Psa 55:8  A?teptat-am pe Dumnezeu, Cel ce ma mântuie?te de pu?inatatea sufletului ?i de vifor.
Psa 55:9  Nimice?te-i, Doamne ?i împarte limbile lor, ca am vazut faradelege ?i dezbinare în cetate.
Psa 55:10  Ziua ?i noaptea o va înconjura pe ea peste zidurile ei; faradelege ?i osteneala în mijlocul ei ?i nedreptate;
Psa 55:11  ?i n-a lipsit din uli?ele ei camata ?i vicle?ug.
Psa 55:12  Ca de m-ar fi ocarât vrajma?ul, a? fi rabdat; ?i daca cel ce ma ura?te s-ar fi falit împotriva mea, m-a? fi ascuns de el.
Psa 55:13  Iar tu, omule, asemenea mie, capetenia mea ?i cunoscutul meu,
Psa 55:14  Care împreuna cu mine te-ai îndulcit de mâncari, în casa lui Dumnezeu am umblat în acela?i gând!
Psa 55:15  Sa vina moartea peste ei ?i sa se coboare în iad de vii, caci vicle?ug este în loca?urile lor, în mijlocul lor.
Psa 55:16  Iar eu, catre Dumnezeu am strigat, ?i Domnul m-a auzit pe mine.
Psa 55:17  Seara ?i diminea?a ?i la amiezi spune-voi, voi vesti, ?i va auzi glasul meu.
Psa 55:18  Izbavi-va cu pace sufletul meu de cei ce se apropie de mine, ca mul?i erau împotriva mea.
Psa 55:19  Auzi-va Dumnezeu ?i-i va smeri pe ei, Cel ce este mai înainte de veci.
Psa 55:20  Ca nu este în ei îndreptare ?i nu s-au temut de Dumnezeu. Întins-au mâinile împotriva alia?ilor lor, întinat-au legamântul Lui. Risipi?i au fost de mânia fe?ei Lui ?i s-au apropiat inimile lor;
Psa 55:21  Muiatu-s-au cuvintele lor mai mult decât untdelemnul, dar ele sunt sage?i.
Psa 55:22  Arunca spre Domnul grija ta ?i El te va hrani; nu va da în veac clatinare dreptului.
Psa 55:23  Iar Tu, Dumnezeule, îi vei coborî pe ei în groapa stricaciunii. Barba?ii varsatori de sânge ?i vicleni nu vor ajunge la jumatatea zilelor lor; iar eu voi nadajdui spre Tine, Doamne.
Psa 56:1  (Un psalm al lui David; mai-marelui cântare?ilor, pentru poporul ce se departase de la cele sfinte. Când l-au prins pe el cei de alt neam în Gat.) Mântuie?te-ma, Doamne, ca m-a necajit omul; toata ziua razboindu-se m-a necajit.
Psa 56:2  Calcatu-m-au vrajma?ii mei toata ziua, ca mul?i sunt cei ce se lupta cu mine, din înal?ime.
Psa 56:3  Ziua când ma voi teme, voi nadajdui în Tine.
Psa 56:4  În Dumnezeu voi lauda toate cuvintele mele toata ziua; în Dumnezeu am nadajduit, nu ma voi teme: Ce-mi va face mie omul?
Psa 56:5  Toata ziua cuvintele mele au urât, împotriva mea toate gândurile lor sunt spre rau.
Psa 56:6  Locui-vor lânga mine ?i se vor ascunde; ei vor pazi calcâiul meu, ca ?i cum ar cauta sufletul meu.
Psa 56:7  Pentru nimic nu-i vei mântui pe ei; în mânie popoare vei sfarâma, Dumnezeule.
Psa 56:8  Via?a mea am spus-o ?ie; pune lacrimile mele înaintea Ta, dupa fagaduin?a Ta.
Psa 56:9  Întoarce-se-vor vrajma?ii mei înapoi, în orice zi Te voi chema. Iata, am cunoscut ca Dumnezeul meu e?ti Tu.
Psa 56:10  În Dumnezeu voi lauda graiul, în Dumnezeu voi lauda cuvântul;
Psa 56:11  În Dumnezeu am nadajduit, nu ma voi teme: Ce-mi va face mie omul?
Psa 56:12  În mine sunt, Dumnezeule,  fagaduin?ele pe care le voi aduce laudei Tale,
Psa 56:13  Ca ai izbavit sufletul meu de la moarte, picioarele mele de alunecare, ca bine sa plac înaintea lui Dumnezeu, în lumina celor vii.
Psa 57:1  (Un psalm al lui David; mai-marelui cântare?ilor. Sa nu strici! Când a fugit de la fa?a lui Saul în pe?tera.) Miluie?te-ma, Dumnezeule, miluie?te-ma, ca spre Tine a nadajduit sufletul meu ?i în umbra aripilor Tale voi nadajdui, pâna ce va trece faradelegea.
Psa 57:2  Striga-voi catre Dumnezeul Cel Preaînalt, Dumnezeul Care mi-a facut bine.
Psa 57:3  Trimis-a din cer ?i m-a mântuit, dat-a spre ocara pe cei ce ma necajesc pe mine.
Psa 57:4  Trimis-a Dumnezeu mila Sa ?i adevarul Sau ?i a izbavit sufletul meu din mijlocul puilor de lei. Adormit-am tulburat. Fiii oamenilor, din?ii lor sunt arme ?i sage?i ?i limba lor sabie ascu?ita.
Psa 57:5  Înal?a-Te peste ceruri, Dumnezeule, ?i peste tot pamântul slava Ta!
Psa 57:6  Curse au gatit sub picioarele mele ?i au împilat sufletul meu; sapat-au înaintea mea groapa ?i au cazut în ea.
Psa 57:7  Gata este inima mea, Dumnezeule, gata este inima mea! Cânta-voi ?i voi lauda slava Ta.
Psa 57:8  De?teapta-te marirea mea! De?teapta-te psaltire ?i alauta! De?tepta-ma-voi diminea?a.
Psa 57:9  Lauda-Te-voi între popoare, Doamne, cânta-voi ?ie între neamuri,
Psa 57:10  Ca s-a marit pâna la cer mila Ta ?i pâna la nori adevarul Tau.
Psa 57:11  Înal?a-Te peste ceruri, Dumnezeule, ?i peste tot pamântul slava Ta!
Psa 58:1  (Un psalm al lui David; mai-marelui cântare?ilor. Sa nu strici!) De grai?i într-adevar dreptate, drept judeca?i, fii ai oamenilor.
Psa 58:2  Pentru ca în inima faradelege lucra?i pe pamânt, nedreptate mâinile voastre împletesc.
Psa 58:3  Înstrainatu-s-au pacato?ii de la na?tere, ratacit-au din pântece, grait-au minciuni.
Psa 58:4  Mânia lor dupa asemanarea ?arpelui, ca a unei vipere surde, care-?i astupa urechile ei,
Psa 58:5  Care nu va auzi glasul descântatoarelor, al vrajitorului care vraje?te cu iscusin?a.
Psa 58:6  Dumnezeu va zdrobi din?ii lor în gura lor; maselele leilor le-a sfarâmat Domnul.
Psa 58:7  De nimic se vor face, ca apa care trece; întinde-va arcul Sau pâna ce vor slabi.
Psa 58:8  Ca ceara ce se tope?te vor fi nimici?i; a cazut foc peste ei ?i n-au vazut soarele.
Psa 58:9  Înainte ca spinii vo?tri sa se aprinda, ca pe ni?te vii întru mânie îi va înghi?i pe ei.
Psa 58:10  Veseli-se-va dreptul când va vedea razbunarea împotriva necredincio?ilor; mâinile lui le va spala în sângele pacatosului.
Psa 58:11  ?i va zice omul: "Da, este rasplata pentru cel drept! Da, este Dumnezeu Care îi judeca pe ei în via?a!
Psa 59:1  (Un psalm al lui David; mai-marelui cântare?ilor. Când a trimis Saul ?i a pazit casa lui, ca sa-l omoare pe el.) Scoate-ma de la vrajma?ii mei, Dumnezeule, ?i de cei ce se scoala împotriva mea, izbave?te-ma.
Psa 59:2  Izbave?te-ma de cei ce lucreaza faradelegea ?i de barba?ii varsarilor de sânge ma izbave?te.
Psa 59:3  Ca iata au vânat sufletul meu, statut-au împotriva mea cei tari. Nici faradelegea ?i nici pacatul meu nu sunt pricina, Doamne. Fara de nelegiuire am alergat ?i m-am îndreptat spre Tine;
Psa 59:4  Scoala-Te, întru întâmpinarea mea ?i vezi. ?i Tu, Doamne, Dumnezeul puterilor, Dumnezeul lui Israel,
Psa 59:5  Ia aminte sa cercetezi toate neamurile; sa nu Te milostive?ti de to?i cei ce lucreaza faradelegea.
Psa 59:6  Întoarce-se-vor spre seara ?i vor flamânzi ca un câine ?i vor înconjura cetatea.
Psa 59:7  Iata, vor striga cu gura lor ?i sabie în buzele lor, ca cine i-a auzit?
Psa 59:8  ?i Tu, Doamne, vei râde de ei, vei face de nimic toate neamurile.
Psa 59:9  Puterea mea în Tine o voi pazi, ca Tu, Dumnezeule, sprijinitorul meu e?ti. Dumnezeul meu, mila Ta ma va întâmpina;
Psa 59:10  Dumnezeu îmi va arata înfrângerea du?manilor mei.
Psa 59:11  Sa nu-i omori pe ei, ca nu cumva sa uite legea Ta; risipe?te-i pe ei cu puterea Ta ?i doboara-i pe ei, aparatorul meu, Doamne.
Psa 59:12  Pacatul gurii lor, cuvântul buzelor lor, sa se prinda întru mândria lor ?i de blestemul ?i minciuna lor se va duce vestea.
Psa 59:13  Nimice?te-i, întru mânia Ta nimice?te-i, ca sa nu mai fie! ?i vor cunoa?te ca Dumnezeu stapâne?te pe Iacob ?i marginile pamântului.
Psa 59:14  Întoarce-se-vor spre seara ?i vor flamânzi ca un câine ?i vor înconjura cetatea.
Psa 59:15  Ei se vor risipi sa manânce; iar de nu se vor satura, vor murmura.
Psa 59:16  Iar eu voi lauda puterea Ta ?i ma voi bucura diminea?a de mila Ta. Ca Te-ai facut sprijinitorul meu ?i scaparea mea în ziua necazului meu.
Psa 59:17  Ajutorul meu e?ti, ?ie-?i voi cânta, ca Tu, Dumnezeule, sprijinitorul meu e?ti, Dumnezeul meu, mila mea.
Psa 60:1  (Un psalm al lui David; mai-marelui cântare?ilor. Pentru cei ce se vor schimba, spre înva?atura. Când a ars Mesopotamia Siriei ?i Sova Siriei ?i s-a întors Iacob ?i a batut pe Edom în Valea Sarii, douasprezece mii.) Dumnezeule, lepadatu-ne-ai pe noi ?i ne-ai doborât; mâniatu-Te-ai ?i Te-ai milostivit spre noi.
Psa 60:2  Cutremurat-ai pamântul ?i l-ai tulburat pe el; vindeca sfarâmaturile lui, ca s-a cutremurat.
Psa 60:3  Aratat-ai poporului Tau asprime, adapatu-ne-ai pe noi cu vinul umilin?ei.
Psa 60:4  Dat-ai celor ce se tem de Tine semn ca sa fuga de la fa?a arcului;
Psa 60:5  Ca sa se izbaveasca cei iubi?i ai Tai. Mântuie?te-ma cu dreapta Ta ?i ma auzi.
Psa 60:6  Dumnezeu a grait în locul cel sfânt al Sau: "Bucura-Ma-voi ?i voi împar?i Sichemul ?i valea Sucot o voi masura.
Psa 60:7  Al Meu este Galaadul ?i al Meu este Manase ?i Efraim, taria capului Meu,
Psa 60:8  Iuda împaratul Meu; Moab vasul nadejdii Mele. Spre Idumeea voi întinde încal?amintea Mea; Mie cei de alt neam Mi s-au supus".
Psa 60:9  Cine ma va duce la cetatea întarita? Cine ma va pova?ui pâna la Idumeea?
Psa 60:10  Oare, nu Tu, Dumnezeule, Cel ce ne-ai lepadat pe noi? Oare, nu vei ie?i Dumnezeule, cu o?tirile noastre?
Psa 60:11  Da-ne noua ajutor, ca sa ne sco?i din necaz, ca de?arta este izbavirea de la om.
Psa 60:12  Cu Dumnezeu vom birui ?i El va nimici pe cei ce ne necajesc pe noi.
Psa 61:1  (Un psalm al lui David; mai-marelui cântare?ilor, pentru instrumente cu coarde.) Auzi, Dumnezeule, cererea mea, ia aminte la rugaciunea mea!
Psa 61:2  De la marginile pamântului catre Tine am strigat; când s-a mâhnit inima mea, pe piatra m-ai înal?at.
Psa 61:3  Pova?uitu-m-ai, ca ai fost nadejdea mea, turn de tarie în fa?a vrajma?ului.
Psa 61:4  Locui-voi în loca?ul Tau în veci; acoperi-ma-voi cu acoperamântul aripilor Tale,
Psa 61:5  Ca Tu, Dumnezeule, ai auzit rugaciunile mele; dat-ai mo?tenire celor ce se tem de numele Tau.
Psa 61:6  Zile la zilele împaratului adauga, anii lui din neam în neam.
Psa 61:7  Ramâne-va în veac înaintea lui Dumnezeu; mila ?i adevarul va pazi.
Psa 61:8  A?a voi cânta numelui Tau în veacul veacului, ca sa împlinesc fagaduin?ele mele zi de zi.
Psa 62:1  (Un psalm al lui David; mai-marelui cântare?ilor. Pentru Iditum.) Oare nu lui Dumnezeu se va supune sufletul meu? Ca de la El este mântuirea mea;
Psa 62:2  Pentru ca El este Dumnezeul meu, Mântuitorul meu ?i Sprijinitorul meu; nu ma voi clatina mai mult.
Psa 62:3  Pâna când va ridica?i asupra omului? Cauta?i to?i a-l doborî, socotindu-l ca un zid povârnit ?i ca un gard surpat!
Psa 62:4  S-au sfatuit sa doboare cinstea mea, alergat-au cu minciuna; cu gura lor ma binecuvântau ?i cu inima lor ma blestemau.
Psa 62:5  Dar lui Dumnezeu supune-te, suflete al meu, ca de la El vine rabdarea mea;
Psa 62:6  Ca El este Dumnezeul meu ?i Mântuitorul meu, Sprijinitorul meu; nu ma voi stramuta.
Psa 62:7  În Dumnezeu este mântuirea mea ?i slava mea; Dumnezeu este ajutorul meu ?i nadejdea mea este în Dumnezeu.
Psa 62:8  Nadajdui?i în El toata adunarea poporului; revarsa?i înaintea Lui inimile voastre, ca El este ajutorul nostru.
Psa 62:9  Dar de?ertaciune sunt fiii oamenilor, mincino?i sunt fiii oamenilor; în balan?a, to?i împreuna sunt de?ertaciune.
Psa 62:10  Nu nadajdui?i spre nedreptate ?i spre jefuire nu pofti?i; boga?ia de ar curge nu va lipi?i inima de ea.
Psa 62:11  O data a grait Dumnezeu, aceste doua lucruri am auzit: ca puterea este a lui Dumnezeu
Psa 62:12  ?i a Ta, Doamne, este mila; ca Tu vei rasplati fiecaruia dupa faptele lui.
Psa 63:1  (Un psalm al lui David, când a fost în pustiul Iudeii.) Dumnezeule, Dumnezeul meu, pe Tine Te caut dis-de-diminea?a.
Psa 63:2  Însetat-a de Tine sufletul meu, suspinat-a dupa Tine trupul meu,
Psa 63:3  În pamânt pustiu ?i neumblat ?i fara de apa. A?a în locul cel sfânt m-am aratat ?ie, ca sa vad puterea Ta ?i slava Ta.
Psa 63:4  Ca mai buna este mila Ta decât via?a; buzele mele Te vor lauda.
Psa 63:5  A?a Te voi binecuvânta în via?a mea ?i în numele Tau voi ridica mâinile mele.
Psa 63:6  Ca de seu ?i de grasime sa se sature sufletul meu ?i cu buze de bucurie Te va lauda gura mea.
Psa 63:7  De mi-am adus aminte de Tine în a?ternutul meu, în dimine?i am cugetat la Tine, ca ai fost ajutorul meu
Psa 63:8  ?i întru acoperamântul aripilor Tale ma voi bucura. Lipitu-s-a sufletul meu de Tine ?i pe mine m-a sprijinit dreapta Ta.
Psa 63:9  Iar ei în de?ert au cautat sufletul meu, intra-vor în cele mai de jos ale pamântului;
Psa 63:10  Da-se-vor în mâinile sabiei, par?i vulpilor vor fi.
Psa 63:11  Iar împaratul se va veseli de Dumnezeu; lauda-se-va tot cel ce se jura întru El, ca s-a astupat gura celor ce graiesc nedrepta?i.
Psa 64:1  (Un psalm al lui David; mai-marelui cântare?ilor.) Auzi, Dumnezeule, glasul meu, când ma rog ?ie; de la frica vrajma?ului scoate sufletul meu.
Psa 64:2  Acopera-ma de adunarea celor ce viclenesc, de mul?imea celor ce lucreaza faradelege,
Psa 64:3  Care ?i-au ascu?it ca sabia limbile lor ?i ca ni?te sage?i arunca vorbele lor veninoase ca sa sageteze din ascunzi?uri pe cel nevinovat.
Psa 64:4  Fara de veste îl vor sageta pe el ?i nu se vor teme. Întaritu-s-au în gânduri rele.
Psa 64:5  Grait-au ca sa ascunda curse; spus-au: "Cine ne va vedea pe noi?"
Psa 64:6  Iscodit-au faradelegi ?i au pierit când le iscodeau, ca sa patrunda înlauntrul omului ?i în adâncimea inimii lui.
Psa 64:7  Dar Dumnezeu îi va lovi cu sageata ?i fara de veste îi va rani, ca ei singuri se vor rani cu limbile lor.
Psa 64:8  Tulburatu-s-au to?i cei ce i-au vazut pe ei; ?i s-a temut tot omul.
Psa 64:9  ?i au vestit lucrurile lui Dumnezeu ?i faptele Lui le-au în?eles.
Psa 64:10  Veseli-se-va cel drept de Domnul ?i va nadajdui în El ?i se vor lauda to?i cei drep?i la inima.
Psa 65:1  (Un psalm al lui David; mai-marelui cântare?ilor. Cântarea lui Ieremia ?i a lui Iezechiel din timpul ?ederii în pamânt strain, când avea sa iasa din Babilon.) ?ie ?i se cuvine cântare, Dumnezeule, în Sion ?i ?ie ?i se va împlini fagaduin?a în Ierusalim.
Psa 65:2  Auzi rugaciunea mea, catre Tine tot trupul va veni.
Psa 65:3  Cuvintele celor fara de lege ne-au biruit pe noi ?i nelegiuirile noastre Tu le vei cura?i.
Psa 65:4  Fericit este cel pe care l-ai ales ?i l-ai primit; locui-va în cur?ile Tale.
Psa 65:5  Umplea-ne-vom de bunata?ile casei Tale; sfânt este loca?ul Tau, minunat în dreptate.
Psa 65:6  Auzi-ne pe noi, Dumnezeule, Mântuitorul nostru, nadejdea tuturor marginilor pamântului ?i a celor de pe mare departe;
Psa 65:7  Cel ce gate?ti mun?ii cu taria Ta, încins fiind cu putere; Cel ce tulburi adâncul marii ?i vuietul valurilor ei.
Psa 65:8  Tulbura-se-vor neamurile ?i se vor spaimânta cei ce locuiesc marginile, de semnele Lui; ie?irile dimine?ii ?i ale serii le vei veseli.
Psa 65:9  Cercetat-ai pamântul ?i l-ai adapat pe el, boga?iile lui le-ai înmul?it; râul lui Dumnezeu s-a umplut de apa; gatit-ai hrana lor, ca a?a este gatirea Ta.
Psa 65:10  Adapa brazdele lui, înmul?e?te roadele lui ?i se vor bucura de picaturi de ploaie, rasarind.
Psa 65:11  Vei binecuvânta cununa anului bunata?ii Tale ?i câmpiile Tale se vor umple de roade grase.
Psa 65:12  Îngra?a-se-vor pa?unile pustiei ?i cu bucurie dealurile se vor încinge.
Psa 65:13  Îmbracatu-s-au pa?unile cu oi ?i vaile vor înmul?i grâul; vor striga ?i vor cânta.
Psa 66:1  (Un psalm al lui David; mai-marelui cântare?ilor. O cântare a învierii.) Striga?i lui Dumnezeu tot pamântul.
Psa 66:2  Cânta?i numele Lui; da?i slava laudei Lui.
Psa 66:3  Zice?i lui Dumnezeu: Cât sunt de înfrico?atoare lucrurile Tale! Pentru mul?imea puterii Tale, Te vor lingu?i vrajma?ii Tai.
Psa 66:4  Tot pamântul sa se închine ?ie ?i sa cânte ?ie, sa cânte numelui Tau.
Psa 66:5  Veni?i ?i vede?i lucrurile lui Dumnezeu, înfrico?ator în sfaturi mai mult decât fiii oamenilor.
Psa 66:6  Cel ce preface marea în uscat, prin râu vor trece cu piciorul. Acolo ne vom veseli de El,
Psa 66:7  De Cel ce stapâne?te cu puterea Sa veacul. Ochii Lui spre neamuri privesc; cei ce se razvratesc, sa nu se înal?e întru sine.
Psa 66:8  Binecuvânta?i neamuri pe Dumnezeul nostru ?i face?i sa se auda glasul laudei Lui,
Psa 66:9  Care a dat sufletului meu via?a ?i n-a lasat sa se clatine picioarele mele.
Psa 66:10  Ca ne-ai cercetat pe noi, Dumnezeule, cu foc ne-ai lamurit pe noi, precum se lamure?te argintul.
Psa 66:11  Prinsu-ne-ai pe noi în cursa; pus-ai necazuri pe umarul nostru;
Psa 66:12  Ridicat-ai oameni pe capetele noastre, trecut-am prin foc ?i prin apa ?i ne-ai scos la odihna.
Psa 66:13  Intra-voi în casa Ta cu arderi de tot, împlini-voi ?ie fagaduin?ele mele,
Psa 66:14  Pe care le-au rostit buzele mele ?i le-a grait gura mea, întru necazul meu.
Psa 66:15  Arderi de tot grase voi aduce ?ie, cu tamâie ?i berbeci; Î?i voi jertfi boi ?i ?api.
Psa 66:16  Veni?i de auzi?i to?i cei ce va teme?i de Dumnezeu ?i va voi povesti câte a facut El sufletului meu.
Psa 66:17  Catre Dânsul cu gura mea am strigat ?i L-am laudat cu gura mea.
Psa 66:18  Nedreptate de am avut în inima mea sa nu ma auda Domnul.
Psa 66:19  Pentru aceasta m-a auzit Dumnezeu; luat-a aminte glasul rugaciunii mele.
Psa 66:20  Binecuvântat este Dumnezeu, Care n-a departat rugaciunea mea ?i mila Lui de la mine.
Psa 67:1  (Un psalm al lui David; mai-marelui cântare?ilor. Pentru instrumente cu coarde.) Dumnezeule, milostive?te-Te spre noi ?i ne binecuvinteaza, lumineaza fa?a Ta spre noi ?i ne miluie?te!
Psa 67:2  Ca sa cunoa?tem pe pamânt calea Ta, în toate neamurile mântuirea Ta.
Psa 67:3  Lauda-Te-vor pe Tine popoarele, Dumnezeule, lauda-Te-vor pe Tine popoarele toate!
Psa 67:4  Veseleasca-se ?i sa se bucure neamurile, ca vei judeca popoarele cu dreptate ?i neamurile pe pamânt le vei povatui.
Psa 67:5  Lauda-Te-vor pe Tine popoarele Dumnezeule, lauda-Te-vor pe Tine popoarele toate. Pamântul ?i-a dat rodul sau.
Psa 67:6  Binecuvinteaza-ne pe noi, Dumnezeule, Dumnezeul nostru.
Psa 67:7  Binecuvinteaza-ne pe noi, Dumnezeule, ?i sa se teama de Tine toate marginile pamântului.
Psa 68:1  (Un psalm al lui David; mai-marelui cântare?ilor.) Sa se scoale Dumnezeu ?i sa se risipeasca vrajma?ii Lui ?i sa fuga de la fa?a Lui cei ce-L urasc pe El.
Psa 68:2  Precum se stinge fumul, sa se stinga; cum se tope?te ceara de fa?a focului, a?a sa piara pacato?ii de la fa?a lui Dumnezeu,
Psa 68:3  Iar drep?ii sa se bucure ?i sa se veseleasca înaintea lui Dumnezeu, sa se desfateze în veselie.
Psa 68:4  Cânta?i lui Dumnezeu, cânta?i numelui Lui, gati?i calea Celui ce strabate pustia, Domnul este numele Lui,
Psa 68:5  ?i va bucura?i înaintea Lui. Sa se tulbure de fa?a Lui, a Parintelui orfanilor ?i a Judecatorului vaduvelor.
Psa 68:6  Dumnezeu este în locul cel sfânt al Lui; Dumnezeu a?aza pe cei singuratici în casa, scoate cu vitejie pe cei lega?i în obezi, la fel pe cei amarâ?i, pe cei ce locuiesc în morminte.
Psa 68:7  Dumnezeule, când mergeai Tu înaintea poporului Tau, când treceai Tu prin pustiu,
Psa 68:8  Pamântul s-a cutremurat ?i cerurile s-au topit ?i Sinaiul s-a clatinat de la fa?a Dumnezeului lui Israel.
Psa 68:9  Ploaie de bunavoie vei osebi, Dumnezeule, mo?tenirii Tale. Ea a slabit, dar Tu ai întarit-o.
Psa 68:10  Vieta?ile Tale locuiesc în ea; întru bunatatea Ta, Dumnezeule, purtat-ai grija de cel sarac.
Psa 68:11  Domnul va da cuvântul celor ce vestesc cu putere multa.
Psa 68:12  Împaratul puterilor, poporului iubit va împar?i prazile.
Psa 68:13  Daca ve?i dormi în mijlocul mo?tenirilor voastre, aripile voastre argintate vor fi ca ale porumbi?ei ?i spatele vostru va straluci ca aurul.
Psa 68:14  Când Împaratul Cel ceresc va împra?tia pe regi în ?ara Sa, ei vor fi albi ca zapada pe Selmon.
Psa 68:15  Munte al lui Dumnezeu este muntele Vasan, munte de piscuri este muntele Vasan.
Psa 68:16  Pentru ce, mun?i cu piscuri, pizmui?i muntele în care a binevoit Dumnezeu sa locuiasca în el, pentru ca va locui în el pâna la sfâr?it?
Psa 68:17  Carele lui Dumnezeu sunt mii de mii; mii sunt cei ce se bucura de ele. Domnul în mijlocul lor, pe Sinai, în loca?ul Sau cel sfânt.
Psa 68:18  Suitu-Te-ai la înal?ime, robit-ai mul?ime, luat-ai daruri de la oameni, chiar ?i cu cei razvrati?i îngadui?i au fost sa locuiasca.
Psa 68:19  Domnul Dumnezeu este binecuvântat, binecuvântat este Dumnezeu zi de zi; sa sporeasca între noi, Dumnezeul mântuirii noastre.
Psa 68:20  Dumnezeul nostru este Dumnezeul mântuirii ?i ale Domnului Dumnezeu sunt ie?irile mor?ii.
Psa 68:21  Dar Dumnezeu va sfarâma capetele vrajma?ilor Sai, cre?tetul parului celor ce umbla întru gre?elile lor.
Psa 68:22  Zis-a Domnul: "Din Vasan îl voi întoarce, întoarce-voi pe vrajma?ii tai din adâncurile marii,
Psa 68:23  Pentru ca sa se afunde piciorul tau în sângele lor ?i limba câinilor tai în sângele vrajma?ilor tai.
Psa 68:24  Vazut-am, Dumnezeule, intrarea Ta, vazut-am intrarea Dumnezeului ?i Împaratului meu în loca?ul cel sfânt:
Psa 68:25  Înainte mergeau capeteniile, dupa ei cei ce cântau din strune, în mijloc fecioarele batând din timpane ?i zicând:
Psa 68:26  "În adunari binecuvânta?i pe Dumnezeu, pe Domnul din izvoarele lui Israel!"
Psa 68:27  Acolo era Veniamin cel mai tânar, în uimire; capeteniile lui Iuda, pova?uitorii lor, capeteniile Zabulonului, capeteniile Neftalimului ?i ziceau:
Psa 68:28  "Porunce?te, Dumnezeule, puterii Tale; întare?te Dumnezeule aceasta lucrare pe care ai facut-o noua.
Psa 68:29  Pentru loca?ul Tau, din Ierusalim, Î?i vor aduce împara?ii daruri.
Psa 68:30  Cearta fiarele din trestii, cearta taurii aduna?i împotriva junincilor popoarelor, ca sa nu fie departa?i cei care au fost încerca?i ca argintul.
Psa 68:31  Risipe?te neamurile cele ce voiesc razboaie". Veni-vor soli din Egipt; Etiopia va întinde mai înainte la Dumnezeu mâna ei, zicând:
Psa 68:32  "Împara?iile pamântului cânta?i lui Dumnezeu, cânta?i Domnului.
Psa 68:33  Cânta?i Dumnezeului Celui ce S-a suit peste cerul cerului, spre rasarit; iata va da glasul Sau, glas de putere.
Psa 68:34  Da?i slava lui Dumnezeu! Peste Israel mare?ia Lui ?i puterea Lui în nori.
Psa 68:35  Minunat este Dumnezeu întru sfin?ii Lui, Dumnezeul lui Israel; Însu?i va da putere ?i întarire poporului Sau. Binecuvântat este Dumnezeu".
Psa 69:1  (Un psalm al lui David; mai-marelui cântare?ilor. Pentru cei ce se vor schimba.) Mântuie?te-ma, Dumnezeule, ca au intrat ape pâna la sufletul meu.
Psa 69:2  Afundatu-m-am în noroiul adâncului, care nu are fund; intrat-am în adâncurile marii ?i furtuna m-a potopit.
Psa 69:3  Ostenit-am strigând, amor?it-a gâtlejul meu, slabit-au ochii mei nadajduind spre Dumnezeul meu.
Psa 69:4  Înmul?itu-s-au mai mult decât perii capului meu cei ce ma urasc pe mine în zadar. Întaritu-s-au vrajma?ii mei, cei ce ma prigonesc pe nedrept; cele ce n-am rapit, pe acelea le-am platit.
Psa 69:5  Dumnezeule, Tu ai cunoscut nepriceperea mea ?i gre?elile mele de la Tine nu s-au ascuns.
Psa 69:6  Sa nu fie ru?ina?i, din pricina mea, cei ce Te a?teapta pe Tine, Doamne, Doamne al puterilor, nici înfrunta?i pentru mine, cei ce Te cauta pe Tine, Dumnezeul lui Israel.
Psa 69:7  Ca pentru Tine am suferit ocara, acoperit-a batjocura obrazul meu.
Psa 69:8  Înstrainat am fost de fra?ii mei ?i strain fiilor maicii mele,
Psa 69:9  Ca râvna casei Tale m-a mâncat ?i ocarile celor ce Te ocarasc pe Tine au cazut asupra mea.
Psa 69:10  ?i mi-am smerit cu post sufletul meu ?i mi-a fost spre ocara mie.
Psa 69:11  ?i m-am îmbracat cu sac ?i am ajuns pentru ei de batjocura.
Psa 69:12  Împotriva mea graiau cei ce ?edeau în por?i ?i despre mine cântau cei ce beau vin.
Psa 69:13  Iar eu întru rugaciunea mea catre Tine, Doamne, am strigat la timp bine-placut. Dumnezeule, întru mul?imea milei Tale auzi-ma, întru adevarul milei Tale.
Psa 69:14  Mântuie?te-ma din noroi, ca sa nu ma afund; izbave?te-ma de cei ce ma urasc ?i din adâncul apelor,
Psa 69:15  Ca sa nu ma înece vâltoarea apei, nici sa ma înghita adâncul, nici sa-?i închida peste mine adâncul gura lui.
Psa 69:16  Auzi-ma, Doamne, ca buna este mila Ta; dupa mul?imea îndurarilor Tale cauta spre mine.
Psa 69:17  Sa nu-?i întorci fa?a Ta de la credinciosul Tau, când ma necajesc. Degraba ma auzi.
Psa 69:18  Ia aminte la sufletul meu ?i-l mântuie?te pe el; din mâinile vrajma?ilor mei izbave?te-ma,
Psa 69:19  Ca Tu cuno?ti ocara mea ?i ru?inea mea ?i înfruntarea mea; înaintea Ta sunt to?i cei ce ma necajesc.
Psa 69:20  Zdrobit a fost sufletul meu de ocari ?i necaz, ?i am a?teptat pe cel ce m-ar milui ?i nu era ?i pe cei ce m-ar mângâia ?i nu i-am aflat.
Psa 69:21  ?i mi-au dat spre mâncarea mea fiere ?i în setea mea m-au adapat cu o?et.
Psa 69:22  Faca-se masa lor înaintea lor cursa, rasplatire ?i sminteala;
Psa 69:23  Sa se întunece ochii lor, ca sa nu vada ?i spinarea lor pururea o gârbove?te;
Psa 69:24  Varsa peste ei urgia Ta ?i mânia urgiei Tale sa-i cuprinda pe ei;
Psa 69:25  Faca-se curtea lor pustie ?i în loca?urile lor sa nu fie locuitori;
Psa 69:26  Ca pe care Tu l-ai batut, ei l-au prigonit ?i au înmul?it durerea ranilor lui.
Psa 69:27  Adauga faradelege la faradelegea lor ?i sa nu intre întru dreptatea Ta;
Psa 69:28  ?ter?i sa fie din cartea celor vii ?i cu cei drep?i sa nu se scrie.
Psa 69:29  Sarac ?i îndurerat sunt eu; mântuirea Ta, Dumnezeule, sa ma sprijineasca!
Psa 69:30  Lauda-voi numele Dumnezeului meu cu cântare ?i-L voi preamari pe El cu lauda;
Psa 69:31  ?i-I va placea lui Dumnezeu mai mult decât vi?elul tânar, caruia îi cresc coarne ?i unghii.
Psa 69:32  Sa râda saracii ?i sa se veseleasca; cânta?i lui Dumnezeu ?i viu va fi sufletul vostru!
Psa 69:33  Ca a auzit pe cei saraci Domnul ?i pe cei fereca?i în obezi ai Sai nu i-a urgisit.
Psa 69:34  Sa-L laude pe El cerurile ?i pamântul, marea ?i toate câte se târasc în ea.
Psa 69:35  Ca Dumnezeu va mântui Sionul ?i se vor zidi ceta?ile lui Iuda ?i vor locui acolo ?i-l vor mo?teni pe el;
Psa 69:36  ?i semin?ia credincio?ilor Lui îl va stapâni pe el ?i cei ce iubesc numele Lui vor locui în el.
Psa 70:1  (Un psalm al lui David; mai-marelui cântare?ilor. Spre aducere aminte. Ca sa ma mântuiasca Domnul.) Dumnezeule, spre ajutorul meu ia aminte! Doamne, sa-mi aju?i mie grabe?te-Te!
Psa 70:2  Sa se ru?ineze ?i sa se înfrunte cei ce cauta sufletul meu; sa se întoarca înapoi ?i sa se ru?ineze cei ce-mi voiesc mie rele;
Psa 70:3  Întoarca-se îndata ru?ina?i cei ce-mi graiesc mie: "Bine, bine!"
Psa 70:4  Sa se bucure ?i sa se veseleasca de Tine to?i cei ce Te cauta pe Tine, Dumnezeule, ?i sa zica pururea cei ce iubesc mântuirea Ta: "Slavit sa fie Domnul!"
Psa 70:5  Iar eu sarac sunt ?i sarman, Dumnezeule, ajuta-ma! Ajutorul meu ?i Izbavitorul meu e?ti Tu, Doamne, nu zabovi!
Psa 71:1  (Un psalm al lui David; fiilor lui Ionadav ?i ai celor ce s-au robit mai întâi.) Spre Tine, Doamne, am nadajduit, sa nu fiu ru?inat în veac.
Psa 71:2  Întru dreptatea Ta, izbave?te-ma ?i ma scoate, pleaca urechea Ta catre mine ?i ma mântuie?te.
Psa 71:3  Fii mie Dumnezeu aparator ?i loc întarit, ca sa ma mântuie?ti, ca întarirea ?i scaparea mea e?ti Tu.
Psa 71:4  Dumnezeul meu, izbave?te-ma din mâna pacatosului, din mâna calcatorului de lege ?i a celui ce face strâmbatate,
Psa 71:5  Ca Tu e?ti a?teptarea mea, Doamne; Domnul este nadejdea mea din tinere?ile mele.
Psa 71:6  Întru Tine m-am întarit din pântece; din pântecele maicii mele Tu e?ti acoperitorul meu; întru Tine este lauda mea pururea.
Psa 71:7  Ca o minune m-am facut multora, iar Tu e?ti ajutorul meu cel tare.
Psa 71:8  Sa se umple gura mea de lauda Ta, ca sa laud slava Ta, toata ziua mare cuviin?a Ta.
Psa 71:9  Nu ma lepada la vremea batrâne?ilor; când va lipsi taria mea, sa nu ma la?i pe mine.
Psa 71:10  Ca au zis vrajma?ii mei mie ?i cei ce pazesc sufletul meu s-au sfatuit împreuna,
Psa 71:11  Zicând: Dumnezeu l-a parasit pe el, urmari?i-l ?i-l prinde?i pe el, ca nu este cel ce izbave?te.
Psa 71:12  Dumnezeule, nu Te departa de la mine; Dumnezeul meu, spre ajutorul meu ia aminte!
Psa 71:13  Sa se ru?ineze ?i sa piara cei ce defaimeaza sufletul meu, sa se îmbrace cu ru?ine ?i înfruntare cei ce cauta sa-mi faca rau.
Psa 71:14  Iar eu pururea voi nadajdui spre Tine ?i voi înmul?i lauda Ta.
Psa 71:15  Gura mea va vesti dreptatea Ta, toata ziua mântuirea Ta, al caror nume nu-l cunosc.
Psa 71:16  Intra-voi întru puterea Domnului; Doamne, îmi voi aduce aminte numai de dreptatea Ta.
Psa 71:17  Dumnezeule, m-ai înva?at din tinere?ile mele ?i eu ?i astazi vestesc minunile Tale.
Psa 71:18  Pâna la batrâne?e ?i carunte?e, Dumnezeule, sa nu ma parase?ti, ca sa vestesc bra?ul Tau la tot neamul ce va sa vina,
Psa 71:19  Puterea Ta ?i dreptatea Ta, Dumnezeule, pâna la cele înalte, mare?iile pe care le-ai facut. Dumnezeule, cine este asemenea ?ie?
Psa 71:20  Multe necazuri ?i rele ai trimis asupra mea, dar întorcându-Te mi-ai dat via?a ?i din adâncurile pamântului iara?i m-ai scos.
Psa 71:21  Înmul?it-ai spre mine marirea Ta ?i întorcându-Te m-ai mângâiat ?i din adâncurile pamântului iara?i m-ai scos.
Psa 71:22  Ca eu voi lauda cu instrumente de cântare adevarul Tau, Dumnezeule, cânta-voi ?ie din alauta, Sfântul lui Israel.
Psa 71:23  Bucura-se-vor buzele mele când voi cânta ?ie ?i sufletul meu pe care l-ai mântuit.
Psa 71:24  Înca ?i limba mea toata ziua va rosti dreptatea Ta, când vor fi ru?ina?i ?i înfrunta?i cei ce cauta sa-mi faca rau.
Psa 72:1  (Despre Solomon.) Dumnezeule, judecata Ta da-o împaratului ?i dreptatea Ta fiului împaratului,
Psa 72:2  Ca sa judece pe poporul Tau cu dreptate ?i pe saracii Tai cu judecata.
Psa 72:3  Sa aduca mun?ii pace poporului Tau ?i dealurile dreptate.
Psa 72:4  Judeca-va pe saracii poporului ?i va milui pe fiii saracilor ?i va umili pe clevetitor.
Psa 72:5  ?i se vor teme de Tine cât va fi soarele ?i cât va fi luna din neam în neam.
Psa 72:6  Pogorâ-se-va ca ploaia pe lâna ?i ca picaturile ce cad pe pamânt.
Psa 72:7  Rasari-va în zilele lui dreptatea ?i mul?imea pacii, cât va fi luna.
Psa 72:8  ?i va domni de la o mare pâna la alta ?i de la râu pâna la marginile lumii.
Psa 72:9  Înaintea lui vor îngenunchia etiopienii ?i vrajma?ii lui ?arâna vor linge.
Psa 72:10  Împara?ii Tarsisului ?i insulele daruri vor aduce, împara?ii arabilor ?i ai reginei Saba prinoase vor aduce.
Psa 72:11  ?i se vor închina lui to?i împara?ii pamântului, toate neamurile vor sluji lui.
Psa 72:12  Ca a izbavit pe sarac din mâna celui puternic ?i pe sarmanul care n-avea ajutor.
Psa 72:13  Va avea mila de sarac ?i de sarman ?i sufletele saracilor va mântui;
Psa 72:14  De camata ?i de asuprire va scapa sufletele lor ?i scump va fi numele lor înaintea lui.
Psa 72:15  ?i va fi viu ?i se va da lui din aurul Arabiei ?i se vor ruga pentru el pururea; toata ziua îl vor binecuvânta pe el.
Psa 72:16  Fi-va bel?ug de pâine pe pamânt pâna-n vârful mun?ilor; pomii roditori se vor înal?a ca cedrii Libanului; ?i vor înflori cei din cetate ca iarba pamântului.
Psa 72:17  Numele lui va dainui pe vecie; cât va fi soarele va fi pomenit numele lui. Se vor binecuvânta întru el toate semin?iile pamântului, toate neamurile îl vor ferici pe el.
Psa 72:18  Binecuvântat este Domnul Dumnezeu, Dumnezeul lui Israel, singurul Care face minuni.
Psa 72:19  Binecuvântat este numele slavei Lui în veac ?i în veacul veacului.
Psa 72:20  Tot pamântul se va umple de slava Lui. Amin. Amin.
Psa 73:1  (Un psalm al lui Asaf.) Cât de bun este Dumnezeu cu Israel, cu cei drep?i la inima.
Psa 73:2  Iar mie, pu?in a fost de nu mi-au alunecat picioarele, pu?in a fost de nu s-au poticnit pa?ii mei.
Psa 73:3  Ca am pizmuit pe cei fara de lege, când vedeam pacea pacato?ilor.
Psa 73:4  Ca n-au necazuri pâna la moartea lor ?i tari sunt când lovesc ei.
Psa 73:5  De osteneli omene?ti n-au parte ?i cu oamenii nu sunt biciui?i.
Psa 73:6  Pentru aceea îi stapâne?te pe ei mândria ?i se îmbraca cu nedreptatea ?i silnicia.
Psa 73:7  Din rautatea lor iese nedreptatea ?i cugetele inimii lor ies la iveala.
Psa 73:8  Gândesc ?i vorbesc cu vicle?ug, nedreptate graiesc de sus.
Psa 73:9  Pâna la cer ridica gura lor ?i cu limba lor strabat pamântul.
Psa 73:10  Pentru aceasta poporul meu se ia dupa ei ?i gase?te ca ei sunt plini de zile bune
Psa 73:11  ?i zice: "Cum? ?tie aceasta Dumnezeu? Are cuno?tin?a Cel Preaînalt?
Psa 73:12  Iata, ace?tia sunt pacato?i ?i sunt îndestula?i. Ve?nic sunt boga?i".
Psa 73:13  Iar eu am zis: "Deci, în de?ert am fost drept la inima ?i mi-am spalat întru cele nevinovate mâinile mele,
Psa 73:14  Ca am fost lovit toata ziua ?i mustrat în fiecare diminea?a".
Psa 73:15  Daca a? fi grait a?a, iata a? fi calcat legamântul neamului fiilor Tai.
Psa 73:16  ?i ma framântam sa pricep aceasta, dar anevoios lucru este înaintea mea.
Psa 73:17  Pâna ce am intrat în loca?ul cel sfânt al lui Dumnezeu ?i am în?eles sfâr?itul celor rai:
Psa 73:18  Într-adevar pe drumuri viclene i-ai pus pe ei ?i i-ai doborât când se înal?au.
Psa 73:19  Cât de iute i-ai pustiit pe ei! S-au stins, au pierit din pricina nelegiuirii lor.
Psa 73:20  Ca visul celui ce se de?teapta, Doamne, în cetatea Ta chipul lor de nimic l-ai facut.
Psa 73:21  De aceea s-a bucurat inima mea ?i rarunchii mei s-au potolit.
Psa 73:22  Ca eram fara de minte ?i nu ?tiam; ca un dobitoc eram înaintea Ta.
Psa 73:23  Dar eu sunt pururea cu Tine. Apucatu-m-ai de mâna mea cea dreapta.
Psa 73:24  Cu sfatul Tau m-ai pova?uit ?i cu slava m-ai primit.
Psa 73:25  Ca pe cine am eu în cer afara de Tine? ?i afara de Tine, ce am dorit pe pamânt?
Psa 73:26  Stinsu-s-a inima mea ?i trupul meu, Dumnezeul inimii mele ?i partea mea, Dumnezeule, în veac.
Psa 73:27  Ca iata cei ce se departeaza de Tine vor pieri; nimicit-ai pe tot cel ce se leapada de Tine.
Psa 73:28  Iar mie a ma lipi de Dumnezeu bine este, a pune în Domnul nadejdea mea, ca sa vestesc toate laudele Tale în por?ile fiicei Sionului.
Psa 74:1  (Un psalm al lui Asaf. Al priceperii.) Pentru ce m-ai lepadat, Dumnezeule, pâna în sfâr?it? Aprinsu-s-a inima Ta peste oile pa?unii Tale.
Psa 74:2  Adu-?i aminte de poporul Tau, pe care l-ai câ?tigat de la început. Izbavit-ai toiagul mo?tenirii Tale, muntele Sionului, acesta în care ai locuit.
Psa 74:3  Ridica mâinile Tale împotriva mândriilor lor, pâna la sfâr?it, ca rau a facut vrajma?ul în locul cel sfânt al Tau.
Psa 74:4  ?i s-au falit cei ce Te urasc pe Tine în mijlocul locului de praznuire al Tau, pus-au semnele lor drept semne;
Psa 74:5  Sfarâmat-au intrarea cea de deasupra.
Psa 74:6  Ca în codru cu topoarele au taiat u?ile loca?ului Tau, cu topoare ?i ciocane l-au sfarâmat.
Psa 74:7  Ars-au cu foc loca?ul cel sfânt al Tau, pâna la pamânt; spurcat-au locul numelui Tau.
Psa 74:8  Zis-au în inima lor împreuna cu neamul lor: "Veni?i sa ardem toate locurile de praznuire ale lui Dumnezeu de pe pamânt".
Psa 74:9  Semnele noastre nu le-am vazut; nu mai este profet ?i pe noi nu ne va mai cunoa?te.
Psa 74:10  Pâna când, Dumnezeule, Te va ocarî vrajma?ul, pâna când va huli potrivnicul numele Tau, pâna în sfâr?it?
Psa 74:11  Pentru ce întorci mâna Ta ?i dreapta Ta din sânul Tau, pâna în sfâr?it?
Psa 74:12  Dar Dumnezeu, Împaratul nostru înainte de veac, a facut mântuire în mijlocul pamântului.
Psa 74:13  Tu ai despar?it, cu puterea Ta, marea; Tu ai zdrobit capetele balaurilor din apa;
Psa 74:14  Tu ai sfarâmat capul balaurului; datu-l-ai pe el mâncare popoarelor pustiului.
Psa 74:15  Tu ai deschis izvoare ?i pâraie; Tu ai secat râurile Itanului.
Psa 74:16  A Ta este ziua ?i a Ta este noaptea. Tu ai întocmit lumina ?i soarele.
Psa 74:17  Tu ai facut toate marginile pamântului; vara ?i primavara Tu le-ai zidit.
Psa 74:18  Adu-?i aminte de aceasta: Vrajma?ul a ocarât pe Domnul ?i poporul cel fara de minte a hulit numele Tau.
Psa 74:19  Sa nu dai fiarelor sufletul ce Te lauda pe Tine; sufletele saracilor Tai sa nu le ui?i pâna în sfâr?it.
Psa 74:20  Cauta spre legamântul Tau, ca s-au umplut ascunzi?urile pamântului de locuin?ele faradelegilor.
Psa 74:21  Sa nu se întoarca ru?inat cel umilit; saracul ?i sarmanul ?a laude numele Tau.
Psa 74:22  Scoala-Te, Dumnezeule, apara pricina Ta; adu-?i aminte de ocara de fiecare zi, cu care Te necinste?te cel fara de minte.
Psa 74:23  Nu uita strigatul vrajma?ilor Tai! Razvratirea celor ce Te urasc pe Tine se urca pururea spre Tine.
Psa 75:1  (Un psalm al lui Asaf; mai-marelui cântare?ilor. Sa nu strici!) Lauda-Te-vom pe Tine, Dumnezeule, lauda-Te-vom ?i vom chema numele Tau.
Psa 75:2  Voi spune toate minunile Tale. "Când va fi vremea, zice Domnul, cu dreptate voi judeca.
Psa 75:3  Cutremuratu-s-a pamântul ?i to?i cei ce locuiesc pe el; Eu am întarit stâlpii lui".
Psa 75:4  ?i am zis celor fara de lege: "Nu face?i faradelege!" ?i pacato?ilor: "Nu înal?a?i fruntea!
Psa 75:5  Nu ridica?i la înal?ime fruntea voastra, sa nu grai?i nedreptate împotriva lui Dumnezeu".
Psa 75:6  Ca nici de la rasarit, nici de la apus, nici din mun?ii pustiei, nu vine ajutorul;
Psa 75:7  Ci Dumnezeu este judecatorul; pe unul îl smere?te ?i pe altul îl înal?a.
Psa 75:8  Paharul este în mâna Domnului, plin cu vin curat bine-mirositor, ?i îl trece de la unul la altul, dar drojdia lui nu s-a varsat; din ea vor bea to?i pacato?ii pamântului.
Psa 75:9  Iar eu ma voi bucura în veac, cânta-voi Dumnezeului lui Iacob.
Psa 75:10  ?i toate frun?ile pacato?ilor voi zdrobi ?i se va înal?a fruntea dreptului.
Psa 76:1  (Un psalm al lui Asaf; mai-marelui cântare?ilor. Pentru instrumentele cu coarde. Cântare catre asirieni.) Cunoscut este în Iudeea Dumnezeu; în Israel mare este numele Lui.
Psa 76:2  Ca s-a facut în Ierusalim locul Lui ?i loca?ul Lui în Sion.
Psa 76:3  Acolo a zdrobit taria arcurilor, arma ?i sabia ?i razboiul.
Psa 76:4  Tu luminezi minunat din mun?ii cei ve?nici.
Psa 76:5  Tulburatu-s-au to?i cei nepricepu?i la inima, dormit-au somnul lor ?i to?i cei razboinici nu ?i-au mai gasit mâinile.
Psa 76:6  De certarea Ta, Dumnezeule al lui Iacob, au încremenit calare?ii pe cai.
Psa 76:7  Tu înfrico?ator e?ti ?i cine va sta împotriva mâniei Tale?
Psa 76:8  Din cer ai facut sa se auda judecata; pamântul s-a temut ?i s-a lini?tit,
Psa 76:9  Când s-a ridicat la judecata Dumnezeu, ca sa mântuiasca pe to?i blânzii pamântului.
Psa 76:10  Ca gândul omului Te va lauda ?i amintirea gândului Te va praznui.
Psa 76:11  Face?i fagaduin?e ?i le împlini?i Domnului Dumnezeului vostru. To?i cei dimprejurul Lui vor aduce daruri
Psa 76:12  Celui înfrico?ator ?i Celui ce ia duhurile capeteniilor, Celui înfrico?ator împara?ilor pamântului.
Psa 77:1  (Un psalm al lui Asaf; mai-marelui cântare?ilor. Pentru Iditum.) Cu glasul meu catre Domnul am strigat, cu glasul meu catre Dumnezeu ?i a cautat spre mine.
Psa 77:2  în ziua necazului meu pe Dumnezeu am cautat; chiar ?i noaptea mâinile mele stau întinse înaintea Lui ?i n-am slabit; sufletul n-a vrut sa se mângâie.
Psa 77:3  Adusu-mi-am aminte de Dumnezeu ?i m-am cutremurat; gândit-am ?i a slabit duhul meu.
Psa 77:4  Ochii mei au luat-o înainte, treji; tulburatu-m-am ?i n-am grait.
Psa 77:5  Gândit-am la zilele cele de demult ?i de anii cei ve?nici mi-am adus aminte ?i cugetam;
Psa 77:6  Noaptea în inima mea gândeam ?i se framânta duhul meu zicând:
Psa 77:7  Oare, în veci ma va lepada Domnul ?i nu va mai binevoi în mine?
Psa 77:8  Oare, pâna în sfâr?it ma va lipsi de mila Lui, din neam în neam?
Psa 77:9  Oare, va uita sa Se milostiveasca Dumnezeu? Sau va închide în mâinile Lui îndurarile Sale?
Psa 77:10  ?i am zis: Acum am început sa în?eleg; aceasta este schimbarea dreptei Celui Preaînalt.
Psa 77:11  Adusu-mi-am aminte de lucrurile Domnului ?i-mi voi aduce aminte de minunile Tale, dintru început.
Psa 77:12  ?i voi cugeta la toate lucrurile Tale ?i la faptele Tale ma voi gândi.
Psa 77:13  Dumnezeule, în sfin?enie este calea Ta. Cine este Dumnezeu mare ca Dumnezeul nostru? Tu e?ti Dumnezeu, Care faci minuni!
Psa 77:14  Cunoscuta ai facut între popoare puterea Ta.
Psa 77:15  Izbavit-ai cu bra?ul Tau poporul Tau, pe fiii lui Iacob ?i ai lui Iosif.
Psa 77:16  Vazutu-Te-au apele, Dumnezeule, vazutu-Te-au apele ?i s-au spaimântat ?i s-au tulburat adâncurile.
Psa 77:17  Glas au dat norii ca sage?ile Tale trec.
Psa 77:18  Glasul tunetului Tau în vârtej, luminat-au fulgerele Tale lumea, clatinatu-s-a ?i s-a cutremurat pamântul.
Psa 77:19  În mare este calea Ta ?i cararile Tale în ape multe ?i urmele Tale nu se vor cunoa?te.
Psa 77:20  Pova?uit-ai ca pe ni?te oi pe poporul Tau, cu mâna lui Moise ?i a lui Aaron.
Psa 78:1  (Un psalm al lui Asaf. Al în?elegerii.) Lua?i aminte, poporul meu, la legea mea, pleca?i urechile voastre spre graiurile gurii mele.
Psa 78:2  Deschide-voi în pilde gura mea, spune-voi cele ce au fost dintru început,
Psa 78:3  Câte am auzit ?i am cunoscut ?i câte parin?ii no?tri ne-au înva?at.
Psa 78:4  Nu s-au ascuns de la fiii lor, din neam în neam, vestind laudele Domnului ?i puterile Lui ?i minunile pe care le-a facut.
Psa 78:5  ?i a ridicat marturie în Iacob ?i lege a pus în Israel. Câte a poruncit parin?ilor no?tri ca sa le arate pe ele fiilor lor, ca sa le cunoasca neamul ce va sa vina,
Psa 78:6  Fiii ce se vor na?te ?i se vor ridica, ?i le vor vesti fiilor lor,
Psa 78:7  Ca sa-?i puna în Dumnezeu nadejdea lor ?i sa nu uite binefacerile lui Dumnezeu ?i poruncile Lui sa le ?ina,
Psa 78:8  Ca sa nu fie ca parin?ii lor neam îndaratnic ?i razvratit, neam care nu ?i-a îndreptat inima sa ?i nu ?i-a încredin?at lui Dumnezeu duhul sau.
Psa 78:9  Fiii lui Efraim, arca?i înarma?i, întors-au spatele, în zi de razboi.
Psa 78:10  N-au pazit legamântul lui Dumnezeu ?i în legea Lui n-au vrut sa umble.
Psa 78:11  ?i au uitat facerile Lui de bine ?i minunile Lui, pe care le-a aratat lor,
Psa 78:12  Minunile pe care le-a facut înaintea parin?ilor lor, în pamântul Egiptului, în câmpia Taneos.
Psa 78:13  Despicat-a marea ?i i-a trecut pe ei; statut-au apele ca un zid;
Psa 78:14  Pova?uitu-i-a pe ei cu nor ziua ?i toata noaptea cu lumina de foc;
Psa 78:15  Despicat-a piatra în pustie ?i i-a adapat pe ei cu boga?ie de apa.
Psa 78:16  Scos-a apa din piatra ?i au curs apele ca ni?te râuri.
Psa 78:17  Dar ei înca au gre?it înaintea Lui, amarât-au pe Cel Preaînalt, în loc fara de apa.
Psa 78:18  ?i au ispitit pe Dumnezeu în inimile lor, cerând mâncare sufletelor lor.
Psa 78:19  ?i au grait împotriva lui Dumnezeu ?i au zis: "Va putea, oare, Dumnezeu sa gateasca masa în pustiu?"
Psa 78:20  - Pentru ca a lovit piatra ?i au curs ape ?i pâraiele s-au umplut de apa. - "Oare, va putea da ?i pâine, sau va putea întinde masa poporului Sau?"
Psa 78:21  Pentru aceasta a auzit Domnul ?i S-a mâniat ?i foc s-a aprins peste Iacob ?i mânie s-a suit peste Israel.
Psa 78:22  Caci n-au crezut în Dumnezeu, nici n-au nadajduit în izbavirea Lui.
Psa 78:23  ?i a poruncit norilor de deasupra ?i u?ile cerului le-a deschis
Psa 78:24  ?i a plouat peste ei mana de mâncare ?i pâine cereasca le-a dat lor.
Psa 78:25  Pâine îngereasca a mâncat omul; bucate le-a trimis lor din destul.
Psa 78:26  Poruncit-a El, din cer, vânt dinspre rasarit ?i a adus cu puterea Lui vânt dinspre miazazi.
Psa 78:27  ?i a plouat peste ei ca pulberea carnuri ?i ca nisipul marii pasari zburatoare.
Psa 78:28  ?i au cazut în mijlocul taberei lor, împrejurul corturilor lor.
Psa 78:29  ?i au mâncat ?i s-au saturat foarte ?i pofta lor ?i-au împlinit-o.
Psa 78:30  Nimic nu le lipsea din cele ce pofteau ?i mâncarea le era înca în gura lor,
Psa 78:31  Când mânia lui Dumnezeu s-a ridicat peste ei ?i a ucis pe cei satui ai lor ?i pe cei ale?i ai lui Israel i-a doborât.
Psa 78:32  Cu toate acestea înca au mai pacatuit ?i n-au crezut în minunile Lui.
Psa 78:33  ?i s-au stins în de?ertaciune zilele lor ?i anii lor degraba.
Psa 78:34  Când îi ucidea pe ei, Îl cautau ?i se întorceau ?i reveneau la Dumnezeu.
Psa 78:35  ?i ?i-au adus aminte ca Dumnezeu este ajutorul lor ?i Dumnezeul Cel Preaînalt este izbavitorul lor.
Psa 78:36  Dar L-au în?elat pe El, cu gura lor ?i cu limba lor L-au min?it.
Psa 78:37  În inima lor n-au fost drep?i cu El, nici n-au crezut în legamântul Lui.
Psa 78:38  Iar El este îndurator, va cura?i pacatele ?i nu-i va nimici. Î?i va întoarce de multe ori mânia Lui ?i nu va aprinde toata urgia Lui.
Psa 78:39  ?i-a adus aminte ca trup sunt ei, suflare ce trece ?i nu se mai întoarce.
Psa 78:40  De câte ori L-au amarât în pustiu, L-au mâniat în pamânt fara de apa?
Psa 78:41  ?i s-au întors ?i au ispitit pe Dumnezeu ?i pe Sfântul lui Israel L-au întarâtat.
Psa 78:42  Nu ?i-au adus aminte de bra?ul Lui, de ziua în care i-a izbavit pe ei din mâna asupritorului.
Psa 78:43  Ca a facut în Egipt semnele Lui ?i minunile Lui în câmpia Taneos:
Psa 78:44  El a prefacut în sânge râurile lor ?i apele lor, ca sa nu bea.
Psa 78:45  El a trimis asupra lor tauni ?i i-a mâncat pe ei; ?i broa?te ?i i-a prapadit pe ei.
Psa 78:46  Dat-a stricaciunii rodul lor ?i ostenelile lor, lacustelor.
Psa 78:47  Batut-a cu grindina via lor ?i duzii lor cu piatra.
Psa 78:48  Dat-a grindinii dobitoacele lor ?i averea lor focului.
Psa 78:49  Trimis-a asupra lor urgia mâniei Lui; mânie, urgie ?i necaz trimis-a prin îngeri nimicitori.
Psa 78:50  Facut-a cale mâniei Lui; n-a cru?at de moarte sufletele lor ?i dobitoacele lor mor?ii le-a dat.
Psa 78:51  Lovit-a pe to?i cei întâi-nascu?i din Egipt, pârga ostenelilor lor, în loca?urile lui Ham.
Psa 78:52  Ridicat-a ca pe ni?te oi pe poporul Sau ?i i-a dus pe ei, ca pe o turma, în pustiu.
Psa 78:53  Pova?uitu-i-a pe ei cu nadejde ?i nu s-au înfrico?at ?i pe vrajma?ii lor i-a acoperit marea.
Psa 78:54  Dusu-i-a pe ei la hotarul sfin?eniei Lui, muntele pe care l-a dobândit dreapta Lui.
Psa 78:55  Izgonit-a dinaintea lor neamuri ?i le-a dat lor prin sor?i pamântul de mo?tenire; ?i a a?ezat în corturile lor semin?iile lui Israel.
Psa 78:56  Dar ei au ispitit ?i au amarât pe Dumnezeul Cel Preaînalt ?i poruncile Lui nu le-au pazit.
Psa 78:57  ?i s-au întors ?i au calcat legamântul ca ?i parin?ii lor, întorsu-s-au ca un arc strâmb.
Psa 78:58  ?i L-au mâniat pe El cu înal?imile lor ?i cu idolii lor L-au întarâtat pe El.
Psa 78:59  Auzit-a Dumnezeu ?i S-a mâniat ?i a urgisit foarte pe Israel.
Psa 78:60  A lepadat cortul Sau din ?ilo, loca?ul Lui, în care a locuit printre oameni.
Psa 78:61  ?i a robit taria lor ?i frumuse?ea lor a dat-o în mâinile vrajma?ului.
Psa 78:62  ?i a dat sabiei pe poporul Sau ?i de mo?tenirea Lui n-a ?inut seama.
Psa 78:63  Pe tinerii lor i-a mistuit focul ?i fecioarele lor n-au fost înconjurate cu cinste.
Psa 78:64  Preo?ii lor de sabie au cazut ?i vaduvele lor nu vor plânge.
Psa 78:65  ?i S-a de?teptat Domnul ca cel ce doarme, ca un viteaz ame?it de vin,
Psa 78:66  ?i a lovit din spate pe vrajma?ii Sai; ocara ve?nica le-a dat lor.
Psa 78:67  ?i a lepadat cortul lui Iosif ?i semin?ia lui Efraim n-a ales-o;
Psa 78:68  Ci a ales semin?ia lui Iuda, Muntele Sion, pe care l-a iubit.
Psa 78:69  ?i a zidit loca?ul Sau cel sfânt, ca înal?imea cerului; pe pamânt l-a întemeiat pe el în veac.
Psa 78:70  ?i a ales pe David robul Sau ?i l-a luat pe el de la turmele oilor.
Psa 78:71  De lânga oile ce nasc l-a luat pe el, ca sa pasca pe Iacob, poporul Sau, ?i pe Israel, mo?tenirea Sa.
Psa 78:72  ?i i-a pascut pe ei întru nerautatea inimii lui ?i în priceperea mâinii lui i-a pova?uit pe ei.
Psa 79:1  (Un psalm al lui Asaf.) Dumnezeule, intrat-au neamurile în mo?tenirea Ta, pângarit-au loca?ul Tau cel sfânt, facut-au din Ierusalim ruina.
Psa 79:2  Pus-au cadavrele robilor Tai mâncare pasarilor cerului, trupurile celor cuvio?i ai Tai, fiarelor pamântului.
Psa 79:3  Varsat-au sângele lor ca apa împrejurul Ierusalimului ?i nu era cine sa-i îngroape.
Psa 79:4  Facutu-ne-am ocara vecinilor no?tri, batjocura ?i râs celor dimprejurul nostru.
Psa 79:5  Pâna când, Doamne, Te vei mânia pâna în sfâr?it? Pâna când se va aprinde ca focul mânia Ta?
Psa 79:6  Varsa mânia Ta peste neamurile care nu Te cunosc ?i peste împara?iile care n-au chemat numele Tau.
Psa 79:7  Ca au mâncat pe Iacob ?i locul lui l-au pustiit.
Psa 79:8  Sa nu pomene?ti faradelegile noastre cele de demult; degraba sa ne întâmpine pe noi îndurarile Tale, ca am saracit foarte.
Psa 79:9  Ajuta-ne noua, Dumnezeule, Mântuitorul nostru, pentru slava numelui Tau; Doamne, izbave?te-ne pe noi ?i cura?e?te pacatele noastre pentru numele Tau.
Psa 79:10  Ca nu cumva sa zica neamurile: "Unde este Dumnezeul lor?" Sa se cunoasca între neamuri, înaintea ochilor no?tri,
Psa 79:11  Razbunarea sângelui varsat, al robilor Tai; sa intre înaintea Ta suspinul celor fereca?i. Dupa mare?ia bra?ului Tau, paze?te pe fiii celor omorâ?i.
Psa 79:12  Rasplate?te vecinilor no?tri de ?apte ori, în sânul lor, ocara lor cu care Te-au ocarât pe Tine, Doamne.
Psa 79:13  Iar noi, poporul Tau ?i oile pa?unii Tale, marturisi-ne-vom ?ie, în veac, din neam în neam vom vesti lauda Ta.
Psa 80:1  (Un psalm al lui Asaf; mai-marelui cântare?ilor. Pentru cei ce se vor schimba. Psalm pentru asirieni.) Cel ce pa?ti pe Israel, ia aminte! Cel ce pova?uie?ti ca pe o oaie pe Iosif,
Psa 80:2  Cel ce ?ezi pe heruvimi, arata-Te înaintea lui Efraim ?i Veniamin ?i Manase. De?teapta puterea Ta ?i vino sa ne mântuie?ti pe noi.
Psa 80:3  Dumnezeule, întoarce-ne pe noi ?i arata fa?a Ta, ?i ne vom mântui!
Psa 80:4  Doamne, Dumnezeul puterilor, pâna când Te vei mânia de ruga robilor Tai?
Psa 80:5  Ne vei hrani pe noi cu pâine de lacrimi ?i ne vei adapa cu lacrimi, peste masura?
Psa 80:6  Pusu-ne-ai în cearta cu vecinii no?tri ?i vrajma?ii no?tri ne-au batjocorit pe noi.
Psa 80:7  Doamne, Dumnezeul puterilor, întoarce-ne pe noi ?i arata fa?a Ta ?i ne vom mântui.
Psa 80:8  Via din Egipt ai mutat-o; izgonit-ai neamuri ?i ai rasadit-o pe ea.
Psa 80:9  Cale ai facut înaintea ei ?i ai rasadit radacinile ei ?i s-a umplut pamântul.
Psa 80:10  Umbra ei ?i mladi?ele ei au acoperit cedrii lui Dumnezeu.
Psa 80:11  Întins-a vi?ele ei pâna la mare ?i pâna la râu lastarele ei.
Psa 80:12  Pentru ce ai darâmat gardul ei ?i o culeg pe ea to?i cei ce trec pe cale?
Psa 80:13  A stricat-o pe ea mistre?ul din padure ?i porcul salbatic a pascut-o pe ea.
Psa 80:14  Dumnezeul puterilor, întoarce-Te dar, cauta din cer ?i vezi ?i cerceteaza via aceasta,
Psa 80:15  ?i o desavâr?e?te pe ea, pe care a sadit-o dreapta Ta, ?i pe fiul omului pe care l-ai întarit ?ie.
Psa 80:16  Arsa a fost în foc ?i smulsa, dar de cercetarea fe?ei Tale ei vor pieri.
Psa 80:17  Sa fie mâna Ta peste barbatul dreptei Tale ?i peste fiul omului pe care l-ai întarit ?ie.
Psa 80:18  ?i nu ne vom departa de Tine; ne vei da via?a ?i numele Tau vom chema.
Psa 80:19  Doamne, Dumnezeul puterilor, întoarce-ne pe noi ?i arata fa?a Ta ?i ne vom mântui.
Psa 81:1  (Un psalm al lui Asaf; mai marelui cântare?ilor. Pentru ghitith.) Bucura?i-va de Dumnezeu, ajutorul nostru; striga?i Dumnezeului lui Iacob!
Psa 81:2  Cânta?i psalmi ?i bate?i din timpane; cânta?i dulce din psaltire ?i din alauta!
Psa 81:3  Suna?i din trâmbi?a, la luna noua, în ziua cea binevestita a sarbatorii noastre!
Psa 81:4  Ca porunca pentru Israel este ?i orânduire a Dumnezeului lui Iacob.
Psa 81:5  Marturie a pus în Iosif, când a ie?it din pamântul Egiptului, ?i a auzit limba pe care n-o ?tia:
Psa 81:6  "Luat-am sarcina de pe umerii lui, ca mâinile lui au robit la co?uri.
Psa 81:7  Întru necaz M-ai chemat ?i te-am izbavit, te-am auzit în mijlocul furtunii ?i te-am cercat la apa certarii.
Psa 81:8  Asculta, poporul Meu, ?i-?i voi marturisi ?ie, Israele: De Ma vei asculta pe Mine,
Psa 81:9  Nu vei avea alt Dumnezeu, nici nu te vei închina la dumnezeu strain,
Psa 81:10  Ca Eu sunt Domnul Dumnezeul tau, Cel ce te-am scos din pamântul Egiptului. Deschide gura ?i o voi umple pe ea.
Psa 81:11  Dar na ascultat poporul Meu glasul Meu ?i Israel n-a cautat la Mine.
Psa 81:12  ?i i-am lasat sa umble dupa dorin?ele inimilor lor ?i au mers dupa cugetele lor.
Psa 81:13  De M-ar fi ascultat poporul Meu, de ar fi umblat Israel în caile Mele,
Psa 81:14  I-a? fi supus de tot pe vrajma?ii lor ?i a? fi pus mâna Mea pe asupritorii lor.
Psa 81:15  Vrajma?ii Domnului L-au min?it pe El, dar le va veni timpul lor, în veac.
Psa 81:16  Ca Domnul i-a hranit pe ei din griul cel mai ales ?i cu miere din stânca i-a saturat pe ei".
Psa 82:1  (Un psalm al lui Asaf.) Dumnezeu a stat în dumnezeiasca adunare ?i în mijlocul dumnezeilor va judeca.
Psa 82:2  Pâna când ve?i judeca cu nedreptate ?i la fe?ele pacato?ilor ve?i cauta?
Psa 82:3  Judeca?i drept pe orfan ?i pe sarac ?i face?i dreptate celui smerit, celui sarman.
Psa 82:4  Mântui?i pe cel sarman ?i pe cel sarac; din mina pacatosului, izbavi?i-i.
Psa 82:5  Dar ei n-au cunoscut, nici n-au priceput, ci în întuneric umbla; stricase-vor toate rânduielile pamântului.
Psa 82:6  Eu am zis: "Dumnezei sunte?i ?i to?i fii ai Celui Preaînalt".
Psa 82:7  Dar voi ca ni?te oameni muri?i ?i ca unul din capetenii cade?i.
Psa 82:8  Scoala-Te, Dumnezeule, judeca pamântul, ca toate neamurile sunt ale Tale.
Psa 83:1  (Un psalm al lui Asaf.) Dumnezeule, cine se va asemana ?ie? Sa nu taci, nici sa Te lini?te?ti, Dumnezeule!
Psa 83:2  Ca iata, vrajma?ii Tai s-au întarâtat ?i cei ce Te urasc au ridicat capul.
Psa 83:3  Împotriva poporului Tau au lucrat cu vicle?ug ?i s-au sfatuit împotriva sfin?ilor Tai.
Psa 83:4  Zis-au: "Veni?i sa-i pierdem pe ei dintre neamuri ?i sa nu se mai pomeneasca numele lui Israel".
Psa 83:5  Ca s-au sfatuit într-un gând împotriva lui. Împotriva Ta legamânt au facut:
Psa 83:6  Loca?urile Idumeilor ?i Ismaelitenii, Moabul ?i Agarenii,
Psa 83:7  Gheval ?i Amon ?i Amalic ?i cei de alt neam, cu cei ce locuiesc în Tir.
Psa 83:8  Ca ?i Asur a venit împreuna cu ei, ajutat-au fiilor lui Lot.
Psa 83:9  Fa-le lor ca lui Madian ?i lui Sisara ?i ca lui Iavin, la râul Chi?on.
Psa 83:10  Pierit-au în Endor; facutu-s-au ca gunoiul pe pamânt.
Psa 83:11  Pune pe capeteniile lor ca pe Oriv ?i Zev ?i Zevel ?i Salmana, pe toate capeteniile lor,
Psa 83:12  Care au zis: "Sa mo?tenim noi jertfelnicul lui Dumnezeu".
Psa 83:13  Dumnezeul meu, pune-i pe ei ca o roata, ca trestia în fa?a vântului,
Psa 83:14  Ca focul care arde padurea, ca vapaia care arde mun?ii,
Psa 83:15  A?a alunga-i pe ei, în viforul Tau ?i în urgia Ta.
Psa 83:16  Umple fe?ele lor de ocara ?i vor cauta fa?a Ta, Doamne.
Psa 83:17  Sa se ru?ineze ?i sa se tulbure în veacul veacului ?i sa fie înfrunta?i ?i sa piara.
Psa 83:18  ?i sa cunoasca ei ca numele Tau este Domnul. Tu singur e?ti Cel Preaînalt peste tot pamântul.
Psa 84:1  (Mai-marelui cântare?ilor; pentru ghitith, fiilor lui Core.) Cât de iubite sunt loca?urile Tale, Doamne al puterilor!
Psa 84:2  Dore?te ?i se sfâr?e?te sufletul meu dupa cur?ile Domnului; inima mea ?i trupul meu s-au bucurat de Dumnezeul cel viu.
Psa 84:3  Ca pasarea ?i-a aflat casa ?i turtureaua cuib, unde-?i va pune puii sai: altarele Tale, Doamne al puterilor, Împaratul meu ?i Dumnezeul meu.
Psa 84:4  Ferici?i sunt cei ce locuiesc în casa Ta; în vecii vecilor Te vor lauda.
Psa 84:5  Fericit este barbatul al carui ajutor este de la Tine, Doamne; sui?uri în inima sa a pus,
Psa 84:6  În valea plângerii, în locul care i-a fost pus. Ca binecuvântare va da Cel ce pune lege,
Psa 84:7  Merge-vor din putere în putere, arata-Se-va Dumnezeul dumnezeilor în Sion.
Psa 84:8  Doamne, Dumnezeul puterilor, auzi rugaciunea mea! Asculta, Dumnezeul lui Iacob!
Psa 84:9  Aparatorul nostru, vezi Dumnezeule ?i cauta spre fa?a unsului Tau!
Psa 84:10  Ca mai buna este o zi în cur?ile Tale decât mii. Ales-am a fi lepadat în casa lui Dumnezeu, mai bine, decât a locui în loca?urile pacato?ilor.
Psa 84:11  Ca mila ?i adevarul iube?te Domnul; Dumnezeu har ?i slava va da. Dumnezeu nu va lipsi de bunata?i pe cei ce umbla întru nerautate.
Psa 84:12  Doamne al puterilor, fericit este omul cel ce nadajduie?te întru Tine.
Psa 85:1  (Mai-marelui cântare?ilor; fiilor lui Core.) Bine ai voit, Doamne, pamântului Tau, întors-ai robimea lui Iacob.
Psa 85:2  Iertat-ai faradelegile poporului Tau, acoperit-ai toate pacatele lor.
Psa 85:3  Potolit-ai toata mânia Ta; întorsu-Te-ai de catre iu?imea mâniei Tale.
Psa 85:4  Întoarce-ne pe noi, Dumnezeul mântuirii noastre ?i-?i întoarce mânia Ta de la noi.
Psa 85:5  Oare, în veci Te vei mânia pe noi? Sau vei întinde mânia Ta din neam în neam?
Psa 85:6  Dumnezeule, Tu întorcându-Te, ne vei darui via?a ?i poporul Tau se va veseli de Tine.
Psa 85:7  Arata-ne noua, Doamne, mila Ta ?i mântuirea Ta da-ne-o noua.
Psa 85:8  Auzi-voi ce va grai întru mine Domnul Dumnezeu; ca va grai pace peste poporul Sau ?i peste cuvio?ii Sai ?i peste cei ce î?i întorc inima spre Dânsul.
Psa 85:9  Dar mântuirea Lui aproape este de cei ce se tem de Dânsul, ca sa se sala?luiasca slava în pamântul nostru.
Psa 85:10  Mila ?i adevarul s-au întâmpinat, dreptatea ?i pacea s-au sarutat.
Psa 85:11  Adevarul din pamânt a rasarit ?i dreptatea din cer a privit.
Psa 85:12  Ca Domnul va da bunatate ?i pamântul nostru î?i va da rodul sau;
Psa 85:13  Dreptatea înaintea Lui va merge ?i va pune pe cale pa?ii Sai.
Psa 86:1  (Un psalm al lui David; o rugaciune.) Pleaca, Doamne, urechea Ta ?i ma auzi, ca sarac ?i necajit sunt eu.
Psa 86:2  Paze?te sufletul meu, caci cuvios sunt; mântuie?te, Dumnezeul meu, pe robul Tau, pe cel ce nadajduie?te în Tine.
Psa 86:3  Miluie?te-ma, Doamne, ca spre Tine voi striga toata ziua.
Psa 86:4  Vesele?te sufletul robului Tau, ca spre Tine, Doamne, am ridicat sufletul meu.
Psa 86:5  Ca Tu, Doamne, bun ?i blând e?ti ?i mult-milostiv tuturor celor ce Te cheama pe Tine.
Psa 86:6  Asculta, Doamne, rugaciunea mea ?i ia aminte la glasul cererii mele.
Psa 86:7  În ziua necazului meu am strigat catre Tine, ca m-ai auzit.
Psa 86:8  Nu este asemenea ?ie între dumnezei, Doamne ?i nici fapte nu sunt ca faptele Tale.
Psa 86:9  Veni-vor toate neamurile pe care le-ai facut ?i se vor închina înaintea Ta, Doamne ?i vor slavi numele Tau.
Psa 86:10  Ca mare e?ti Tu, Cel ce faci minuni, Tu e?ti singurul Dumnezeu.
Psa 86:11  Pova?uie?te-ma, Doamne, pe calea Ta ?i voi merge întru adevarul Tau; veseleasca-se inima mea, ca sa se teama de numele Tau.
Psa 86:12  Lauda-Te-voi, Doamne, Dumnezeul meu, cu toata inima mea ?i voi slavi numele Tau în veac.
Psa 86:13  Ca mare este mila Ta spre mine ?i ai izbavit sufletul meu din iadul cel mai de jos.
Psa 86:14  Dumnezeule, calcatorii de lege s-au sculat asupra mea ?i adunarea celor tari a cautat sufletul meu ?i nu Te-au pus pe Tine înaintea lor.
Psa 86:15  Dar Tu, Doamne, Dumnezeu îndurat ?i milostiv e?ti; îndelung-rabdator ?i mult-milostiv ?i adevarat.
Psa 86:16  Cauta spre mine ?i ma miluie?te, da taria Ta slugii Tale ?i mântuie?te pe fiul slujnicei Tale.
Psa 86:17  Fa cu mine semn spre bine, ca sa vada cei ce ma urasc ?i sa se ru?ineze, ca Tu, Doamne, m-ai ajutat ?i m-ai mângâiat.
Psa 87:1  (O cântare fiilor lui Core.) Temelia Sionului pe mun?ii cei sfin?i.
Psa 87:2  Domnul iube?te por?ile Sionului, mai mult decât toate loca?urile lui Iacob.
Psa 87:3  Lucruri marite s-au grait despre tine, cetatea lui Dumnezeu.
Psa 87:4  Îmi voi aduce aminte de Raav ?i de Babilon, între cei ce ma cunosc pe mine; ?i iata cei de alt neam ?i Tirul ?i poporul etiopienilor, ace?tia acolo s-au nascut.
Psa 87:5  Mama va zice Sionului omul ?i om s-a nascut în el ?i Însu?i Cel Preaînalt l-a întemeiat pe el.
Psa 87:6  Domnul va povesti în cartea popoarelor ?i a capeteniilor acestora, ce s-au nascut în el.
Psa 87:7  Ca în Tine este loca?ul tuturor celor ce se veselesc.
Psa 88:1  (Un psalm fiilor lui Core; mai-marelui cântare?ilor, pentru maelet. Ca sa raspunda; spre pricepere, lui Etam Ezrahitul.) Doamne, Dumnezeul mântuirii mele, ziua am strigat ?i noaptea înaintea Ta.
Psa 88:2  Sa ajunga înaintea Ta rugaciunea mea; pleaca urechea Ta spre ruga mea, Doamne,
Psa 88:3  Ca s-a umplut de rele sufletul meu ?i via?a mea de iad s-a apropiat.
Psa 88:4  Socotit am fost cu cei ce se coboara în groapa; ajuns-am ca un om neajutorat, între cei mor?i slobod.
Psa 88:5  Ca ni?te oameni rani?i ce dorm în mormânt, de care nu ?i-ai mai adus aminte ?i care au fost lepada?i de la mâna Ta.
Psa 88:6  Pusu-m-au în groapa cea mai de jos, întru cele întunecate ?i în umbra mor?ii.
Psa 88:7  Asupra mea s-a întarâtat mânia Ta ?i toate valurile Tale le-ai adus spre mine.
Psa 88:8  Departat-ai pe cunoscu?ii mei de la mine, ajuns-am urâciune lor.
Psa 88:9  Închis am fost ?i n-am putut ie?i. Ochii mei au slabit de suferin?a. Strigat-am catre Tine, Doamne, toata ziua, întins-am catre Tine mâinile mele.
Psa 88:10  Oare, mor?ilor vei face minuni? Sau cei morii se vor scula ?i Te vor lauda pe Tine?
Psa 88:11  Oare, va spune cineva în mormânt mila Ta ?i adevarul Tau în locul pierzarii?
Psa 88:12  Oare, se vor cunoa?te întru întuneric minunile Tale ?i dreptatea Ta în pamânt uitat?
Psa 88:13  Iar eu catre Tine, Doamne, am strigat ?i diminea?a rugaciunea mea Te va întâmpina.
Psa 88:14  Pentru ce Doamne, lepezi sufletul meu ?i întorci fa?a Ta de la mine?
Psa 88:15  Sarac sunt eu ?i în osteneli din tinere?ile mele, înal?at am fost, dar m-am smerit ?i m-am mâhnit.
Psa 88:16  Peste mine au trecut mâniile Tale ?i înfrico?arile Tale m-au tulburat.
Psa 88:17  Înconjuratu-m-au ca apa toata ziua ?i m-au cuprins deodata.
Psa 88:18  Departat-ai de la mine pe prieten ?i pe vecin, iar pe cunoscu?ii mei de ticalo?ia mea.
Psa 89:1  (Un psalm spre priceperea lui Etam Ezrahitul.) Milele Tale, Doamne, în veac le voi cânta. Din neam în neam voi vesti adevarul Tau cu gura mea,
Psa 89:2  Ca ai zis: "În veac mila se va zidi, în ceruri se va întari adevarul Tau.
Psa 89:3  Facut-am legamânt cu ale?ii Mei, juratu-M-am lui David, robul Meu:
Psa 89:4  Pâna în veac voi întari semin?ia ta ?i voi zidi din neam în neam scaunul tau".
Psa 89:5  Lauda-vor cerurile minunile Tale, Doamne ?i adevarul Tau, în adunarea sfin?ilor.
Psa 89:6  Ca cine va fi asemenea Domnului în nori ?i cine se va asemana cu Domnul între fiii lui Dumnezeu?
Psa 89:7  Dumnezeul Cel Preamarit în sfatul sfin?ilor, mare ?i înfrico?ator este peste cei dimprejurul Lui.
Psa 89:8  Doamne, Dumnezeul puterilor, cine este asemenea ?ie? Tare e?ti, Doamne, ?i adevarul Tau împrejurul Tau.
Psa 89:9  Tu stapâne?ti puterea marii ?i mi?carea valurilor ei Tu o potole?ti.
Psa 89:10  Tu ai smerit ca pe un ranit pe cel mândru; cu bra?ul puterii Tale ai risipit pe vrajma?ii Tai.
Psa 89:11  Ale Tale sunt cerurile ?i al Tau este pamântul; lumea ?i plinirea ei Tu le-ai întemeiat.
Psa 89:12  Miazanoapte ?i miazazi Tu ai zidit; Taborul ?i Ermonul în numele Tau se vor bucura.
Psa 89:13  Bra?ul Tau este cu putere. Sa se întareasca mâna Ta, sa se înal?e dreapta Ta.
Psa 89:14  Dreptatea ?i judecata sunt temelia scaunului Tau. Mila ?i adevarul vor merge înaintea fe?ei Tale.
Psa 89:15  Fericit este poporul care cunoa?te strigat de bucurie; Doamne, în lumina fe?ei Tale vor merge
Psa 89:16  ?i în numele Tau se vor bucura toata ziua ?i întru dreptatea Ta se vor înal?a.
Psa 89:17  Ca lauda puterii lor Tu e?ti ?i întru buna vrerea Ta se va înal?a puterea noastra.
Psa 89:18  Ca al Domnului este sprijinul ?i al Sfântului lui Israel, Împaratului nostru.
Psa 89:19  Atunci ai grait în vedenii cuvio?ilor Tai ?i ai zis: Dat-am ajutor celui puternic, înal?at-am pe cel ales din poporul Meu.
Psa 89:20  Aflat-am pe David, robul Meu; cu untdelemnul cel sfânt al Meu l-am uns pe el;
Psa 89:21  Pentru ca mâna Mea îl va ajuta ?i bra?ul Meu îl va întari.
Psa 89:22  Nici un vrajma? nu va izbuti împotriva lui ?i fiul faradelegii nu-i va mai face rau.
Psa 89:23  ?i voi taia pe vrajma?ii sai de la fa?a lui ?i pe cei ce-l urasc pe el îi voi înfrânge.
Psa 89:24  ?i adevarul Meu ?i mila Mea cu el vor fi ?i în numele Meu se va înal?a puterea lui.
Psa 89:25  ?i voi pune peste mare mâna lui ?i peste râuri dreapta lui;
Psa 89:26  Acesta Ma va chema: Tatal meu e?ti Tu, Dumnezeul meu ?i sprijinitorul mântuirii mele.
Psa 89:27  ?i îl voi face pe el întâi-nascut, mai înalt decât împara?ii pamântului;
Psa 89:28  În veac îi voi pastra mila Mea ?i legamântul Meu credincios îi va fi.
Psa 89:29  ?i voi pune în veacul veacului semin?ia lui ?i scaunul lui ca zilele cerului;
Psa 89:30  De vor parasi fiii lui legea Mea ?i dupa rânduielile Mele nu vor umbla,
Psa 89:31  De vor nesocoti drepta?ile Mele ?i poruncile Mele nu vor pazi,
Psa 89:32  Cerceta-voi cu toiag faradelegile lor ?i cu batai pacatele lor.
Psa 89:33  Iar mila Mea nu o voi departa de la el, nici nu voi face strâmbatate întru adevarul Meu,
Psa 89:34  Nici nu voi rupe legamântul Meu ?i cele ce ies din buzele Mele nu le voi schimba.
Psa 89:35  O data m-am jurat pe sfin?enia Mea: Oare, voi min?i pe David?
Psa 89:36  Semin?ia lui în veac va ramâne ?i scaunul lui ca soarele înaintea Mea
Psa 89:37  ?i ca luna întocmita în veac ?i martor credincios în cer.
Psa 89:38  Dar Tu ai lepadat, ai defaimat ?i ai aruncat pe unsul Tau.
Psa 89:39  Stricat-ai legamântul robului Tau, batjocorit-ai pe pamânt sfin?enia lui.
Psa 89:40  Doborât-ai toate gardurile lui, facut-ai întariturile lui ruina.
Psa 89:41  Jefuitu-l-au pe el to?i cei ce treceau pe cale, ajuns-a ocara vecinilor sai.
Psa 89:42  Înal?at-ai dreapta vrajma?ilor lui, veselit-ai pe to?i du?manii lui.
Psa 89:43  Luat-ai puterea sabiei lui ?i nu l-ai ajutat în vreme de razboi.
Psa 89:44  Nimicit-ai cura?enia lui ?i scaunul lui la pamânt l-ai doborât.
Psa 89:45  Mic?orat-ai zilele vie?ii lui, umplutu-l-ai de ru?ine.
Psa 89:46  Pâna când, Doamne, Te vei întoarce? Pâna când se va aprinde ca focul mânia Ta?
Psa 89:47  Adu-?i aminte de mine; oare, în de?ert ai zidit pe to?i fiii oamenilor?
Psa 89:48  Cine este omul ca sa traiasca ?i sa nu vada moartea ?i sa-?i izbaveasca sufletul sau din mâna iadului?
Psa 89:49  Unde sunt milele Tale cele de demult, Doamne, pe care le-ai jurat lui David, întru adevarul Tau?
Psa 89:50  Adu-?i aminte, Doamne, de ocara robilor Tai, pe care o port în sânul lneu, de la multe neamuri.
Psa 89:51  Adu-?i aminte, Doamne, de ocara cu care m-au ocarât vrajma?ii Tai, cu care au ocarât pa?ii unsului Tau.
Psa 89:52  Binecuvântat este Domnul în veci. Amin. Amin.
Psa 90:1  (Rugaciunea lui Moise, omul lui Dumnezeu.) Doamne, scapare Te-ai facut noua în neam ?i în neam.
Psa 90:2  Mai înainte de ce s-au facut mun?ii ?i s-a zidit pamântul ?i lumea, din veac ?i pâna în veac e?ti Tu.
Psa 90:3  Nu întoarce pe om întru smerenie, Tu, care ai zis: "Întoarce?i-va, fii ai oamenilor",
Psa 90:4  Ca o mie de ani înaintea ochilor Tai sunt ca ziua de ieri, care a trecut ?i ca straja nop?ii.
Psa 90:5  Nimicnicie vor fi anii lor; diminea?a ca iarba va trece.
Psa 90:6  Diminea?a va înflori ?i va trece, seara va cadea, se va întari ?i se va usca.
Psa 90:7  Ca ne-am sfâr?it de urgia Ta ?i de mânia Ta ne-am tulburat.
Psa 90:8  Pus-ai faradelegile noastre înaintea Ta, gre?elile noastre ascunse, la lumina fe?ei Tale.
Psa 90:9  Ca toate zilele noastre s-au împu?inat ?i în mânia Ta ne-am stins.
Psa 90:10  Anii no?tri s-au socotit ca pânza unui paianjen; zilele anilor no?tri sunt ?aptezeci de ani; iar de vor fi în putere optzeci de ani ?i ce este mai mult decât ace?tia osteneala ?i durere; ca trece via?a noastra ?i ne vom duce.
Psa 90:11  Cine cunoa?te puterea urgiei Tale ?i cine masoara mânia Ta, dupa temerea de Tine?
Psa 90:12  Înva?a-ne sa socotim bine zilele noastre, ca sa ne îndreptam inimile spre în?elepciune.
Psa 90:13  Întoarce-Te, Doamne! Pâna când vei sta departe? Mângâie pe robii Tai!
Psa 90:14  Umplutu-ne-am diminea?a de mila Ta ?i ne-am bucurat ?i ne-am veselit în toate zilele vie?ii noastre.
Psa 90:15  Veselitu-ne-am pentru zilele în care ne-ai smerit, pentru anii în care am vazut rele.
Psa 90:16  Cauta spre robii Tai ?i spre lucrurile Tale ?i îndrepteaza pe fiii lor.
Psa 90:17  ?i sa fie lumina Domnului Dumnezeului nostru peste noi ?i lucrurile mâinilor noastre le îndrepteaza.
Psa 91:1  (Un psalm al lui David.) Cel ce locuie?te în ajutorul Celui Preaînalt, întru acoperamântul Dumnezeului cerului se va sala?lui.
Psa 91:2  Va zice Domnului: "Sprijinitorul meu e?ti ?i scaparea mea; Dumnezeul meu, voi nadajdui spre Dânsul".
Psa 91:3  Ca El te va izbavi din cursa vânatorilor ?i de cuvântul tulburator.
Psa 91:4  Cu spatele te va umbri pe tine ?i sub aripile Lui vei nadajdui; ca o arma te va înconjura adevarul Lui.
Psa 91:5  Nu te vei teme de frica de noapte, de sageata ce zboara ziua,
Psa 91:6  De lucrul ce umbla în întuneric, de molima ce bântuie întru amiaza.
Psa 91:7  Cadea-vor dinspre latura ta o mie ?i zece mii de-a dreapta ta, dar de tine nu se vor apropia.
Psa 91:8  Însa cu ochii tai vei privi ?i rasplatirea pacato?ilor vei vedea.
Psa 91:9  Pentru ca pe Domnul, nadejdea mea, pe Cel Preaînalt L-ai pus scapare ?ie.
Psa 91:10  Nu vor veni catre tine rele ?i bataie nu se va apropia de loca?ul tau.
Psa 91:11  Ca îngerilor Sai va porunci pentru tine ca sa te pazeasca în toate caile tale.
Psa 91:12  Pe mâini te vor înal?a ca nu cumva sa împiedici de piatra piciorul tau.
Psa 91:13  Peste aspida ?i vasilisc vei pa?i ?i vei calca peste leu ?i peste balaur.
Psa 91:14  "Ca spre Mine a nadajduit ?i-l voi izbavi pe el, zice Domnul; îl voi acoperi pe el ca a cunoscut numele Meu.
Psa 91:15  Striga-va catre Mine ?i-l voi auzi pe el; cu dânsul sunt în necaz ?i-l voi scoate pe el ?i-l voi slavi.
Psa 91:16  Cu lungime de zile îl voi umple pe el, ?i-i voi arata lui mântuirea Mea".
Psa 92:1  (Un psalm pentru ziua sâmbetei.) Bine este a lauda pe Domnul ?i a cânta numele Tau, Preaînalte,
Psa 92:2  A vesti diminea?a mila Ta ?i adevarul Tau în toata noaptea,
Psa 92:3  În psaltire cu zece strune, cu cântece din alauta.
Psa 92:4  Ca m-ai veselit, Doamne, întru fapturile Tale ?i întru lucrurile mâinilor Tale ma voi bucura.
Psa 92:5  Cât s-au marit lucrurile Tale, Doamne, adânci cu totul sunt gândurile Tale!
Psa 92:6  Barbatul nepriceput nu va cunoa?te ?i cel neîn?elegator nu va în?elege acestea,
Psa 92:7  Când rasar pacato?ii ca iarba ?i se ivesc to?i cei ce fac faradelegea,
Psa 92:8  Ca sa piara în veacul veacului. Iar Tu, Preaînalt e?ti în veac, Doamne.
Psa 92:9  Ca iata vrajma?ii Tai, Doamne, vor pieri ?i se vor risipi to?i cei ce lucreaza faradelegea.
Psa 92:10  ?i se va înal?a puterea mea ca a inorogului ?i batrâne?ile mele unse din bel?ug.
Psa 92:11  ?i a privit ochiul meu catre vrajma?ii mei ?i pe cei vicleni, ce se ridica împotriva mea, îi va auzi urechea mea.
Psa 92:12  Dreptul ca finicul va înflori ?i ca cedrul cel din Liban se va înmul?i.
Psa 92:13  Rasadi?i fiind în casa Domnului, în cur?ile Dumnezeului nostru vor înflori.
Psa 92:14  Înca întru batrâne?e unse se vor înmul?i ?i bine vie?uind vor fi, ca sa vesteasca:
Psa 92:15  Drept este Domnul Dumnezeul nostru ?i nu este nedreptate întru Dânsul.
Psa 93:1  (O cântare de lauda a lui David. Pentru ziua dinaintea sâmbetei, când s-a locuit pamântul.) Domnul a împara?it, întru podoaba S-a îmbracat; îmbracatu-S-a Domnul întru putere ?i S-a încins, pentru ca a întarit lumea care nu se va clinti.
Psa 93:2  Gata este scaunul Tau de atunci, din veac e?ti Tu.
Psa 93:3  Ridicat-au râurile, Doamne, ridicat-au râurile glasurile lor,
Psa 93:4  Ridicat-au râurile valurile lor; dar mai mult decât glasul apelor clocotitoare, mai mult decât zbuciumul neasemuit al marii, minunat este, întru cele înalte, Domnul.
Psa 93:5  Marturiile Tale s-au adeverit foarte. Casei Tale se cuvine sfin?enie, Doamne, întru lungime de zile.
Psa 94:1  (Un psalm al lui David. Pentru ziua a patra a sâmbetei.) Dumnezeul razbunarilor, Domnul, Dumnezeul razbunarilor cu îndrazneala a grait.
Psa 94:2  Înal?a-Te Cel ce judeci pamântul, rasplate?te rasplatirea celor mândri.
Psa 94:3  Pâna când pacato?ii, Doamne, pâna când pacato?ii se vor fali?
Psa 94:4  Pâna când vor spune ?i vor grai nedreptate; grai-vor to?i cei ce lucreaza faradelegea?
Psa 94:5  Pe poporul Tau, Doamne, l-au asuprit ?i mo?tenirea Ta au apasat-o.
Psa 94:6  Pe vaduva ?i pe sarac au ucis ?i pe orfani i-au omorât.
Psa 94:7  ?i au zis: "Nu va vedea Domnul, nici nu va pricepe Dumnezeul lui Iacob".
Psa 94:8  În?elege?i, dar, cei neîn?elep?i din popor, ?i cei nebuni, în?elep?i?i-va odata!
Psa 94:9  Cel ce a sadit urechea, oare, nu aude? Cel ce a zidit ochiul, oare, nu prive?te?
Psa 94:10  Cel ce pedepse?te neamurile, oare, nu va certa? Cel ce înva?a pe om cuno?tin?a,
Psa 94:11  Domnul, cunoa?te gândurile oamenilor, ca sunt de?arte.
Psa 94:12  Fericit este omul pe care îl vei certa, Doamne, ?i din legea Ta îl vei înva?a pe el,
Psa 94:13  Ca sa-l lini?te?ti pe el în zile rele, pâna ce se va sapa groapa pacatosului.
Psa 94:14  Ca nu va lepada Domnul pe poporul Sau ?i mo?tenirea Sa nu o va parasi,
Psa 94:15  Pâna ce dreptatea se va întoarce la judecata ?i to?i cei cu inima curata, care se ?in de dânsa.
Psa 94:16  Cine se va ridica cu mine împotriva celor ce viclenesc ?i cine va sta împreuna cu mine împotriva celor ce lucreaza faradelegea?
Psa 94:17  Ca de nu mi-ar fi ajutat mie Domnul, pu?in de nu s-ar fi sala?luit în iad sufletul meu.
Psa 94:18  Când am zis: "S-a clatinat piciorul meu", mila Ta, Doamne, mi-a ajutat mie.
Psa 94:19  Doamne, când s-au înmul?it durerile mele în inima mea, mângâierile Tale au veselit sufletul meu.
Psa 94:20  Nu va sta împreuna cu Tine scaunul faradelegii, cel ce face asuprire împotriva legii.
Psa 94:21  Ei vor prinde în cursa sufletul dreptului ?i sânge nevinovat vor osândi.
Psa 94:22  Dar Domnul mi-a fost mie scapare ?i Dumnezeul meu ajutorul nadejdii mele;
Psa 94:23  Le va rasplati lor Domnul dupa faradelegea lor ?i dupa rautatea lor îi va pierde pe ei Domnul Dumnezeul nostru.
Psa 95:1  (O cântare de lauda a lui David.) Veni?i sa ne bucuram de Domnul ?i sa strigam lui Dumnezeu, Mântuitorului nostru.
Psa 95:2  Sa întâmpinam fa?a Lui întru lauda ?i în psalmi sa-I strigam Lui,
Psa 95:3  Ca Dumnezeu mare este Domnul ?i Împarat mare peste tot pamântul.
Psa 95:4  Ca în mâna Lui sunt marginile pamântului ?i înal?imile mun?ilor ale Lui sunt.
Psa 95:5  Ca a Lui este marea ?i El a facut-o pe ea, ?i uscatul mâinile Lui l-au zidit.
Psa 95:6  Veni?i sa ne închinam ?i sa cadem înaintea Lui ?i sa plângem înaintea Domnului, Celui ce ne-a facut pe noi.
Psa 95:7  Ca El este Dumnezeul nostru ?i noi poporul pa?unii Lui ?i oile mâinii Lui.
Psa 95:8  O, de I-a?i auzi glasul care zice: "Sa nu va învârto?a?i inimile voastre, ca în timpul cercetarii, ca în ziua ispitirii în pustiu,
Psa 95:9  Unde M-au ispitit parin?ii vo?tri, M-au ispitit ?i au vazut lucrurile Mele.
Psa 95:10  Patruzeci de ani am urât neamul acesta ?i am zis: "Pururea ratacesc cu inima".
Psa 95:11  ?i ei n-au cunoscut caile Mele, ca M-am jurat întru mânia Mea: "Nu vor intra întru odihna Mea".
Psa 96:1  (O cântare a lui David, când s-a zidit casa, dupa robie. La Evrei fara titlu.) Cânta?i Domnului cântare noua, cânta?i Domnului tot pamântul.
Psa 96:2  Cânta?i Domnului, binecuvânta?i numele Lui, binevesti?i din zi în zi mântuirea Lui.
Psa 96:3  Vesti?i între neamuri slava Lui, între toate popoarele minunile Lui;
Psa 96:4  Ca mare este Domnul ?i laudat foarte, înfrico?ator este; mai presus decât to?i dumnezeii.
Psa 96:5  Ca to?i dumnezeii neamurilor sunt idoli; iar Domnul cerurile a facut.
Psa 96:6  Lauda ?i frumuse?e este înaintea Lui, sfin?enie ?i mare?ie în loca?ul cel sfânt al Lui.
Psa 96:7  Aduce?i Domnului, semin?iile popoarelor, aduce?i Domnului slava ?i cinste; aduce?i Domnului slava numelui Lui.
Psa 96:8  Aduce?i jertfe ?i intra?i în cur?ile Lui. Închina?i-va Domnului în curtea cea sfânta a Lui.
Psa 96:9  Sa tremure de fa?a Lui tot pamântul. Spune?i între neamuri ca Domnul a împara?it,
Psa 96:10  Pentru ca a întarit lumea care nu se va clinti; judeca-va popoare întru dreptate.
Psa 96:11  Sa se veseleasca cerurile ?i sa se bucure pamântul, sa se zguduie marea ?i toate cele ce sunt întru ea; sa se bucure câmpiile ?i toate cele ce sunt pe ele.
Psa 96:12  Atunci se vor bucura to?i copacii padurii, de fa?a Domnului, ca vine, vine sa judece pamântul.
Psa 96:13  Judeca-va lumea întru dreptate ?i popoarele întru adevarul Sau.
Psa 97:1  (Un psalm al lui David, când s-a întarit domnia lui.) Domnul împara?e?te! Sa se bucure pamântul, sa se veseleasca insule multe.
Psa 97:2  Nor ?i negura împrejurul Lui, dreptatea ?i judecata este temelia neamului Lui.
Psa 97:3  Foc înaintea Lui va merge ?i va arde împrejur pe vrajma?ii Lui.
Psa 97:4  Luminat-au fulgerele Lui lumea; vazut-a ?i s-a cutremurat pamântul.
Psa 97:5  Mun?ii ca ceara s-au topit de fa?a Domnului, de fa?a Domnului a tot pamântul.
Psa 97:6  Vestit-au cerurile dreptatea Lui ?i au vazut toate popoarele slava Lui.
Psa 97:7  Sa se ru?ineze to?i cei ce se închina chipurilor cioplite ?i se lauda cu idolii lor. Închina?i-va Lui to?i îngerii Lui;
Psa 97:8  Auzit-a ?i s-a veselit Sionul ?i s-au bucurat fiicele Iudeii, pentru judeca?ile Tale, Doamne.
Psa 97:9  Ca Tu e?ti Domnul Cel Preaînalt peste tot pamântul; înal?atu-Te-ai foarte, mai presus decât to?i dumnezeii.
Psa 97:10  Cei ce iubi?i pe Domnul, urâ?i raul; Domnul paze?te sufletele cuvio?ilor Lui; din mâna pacatosului îi va izbavi pe ei.
Psa 97:11  Lumina a rasarit dreptului ?i celor drep?i cu inima, veselie.
Psa 97:12  Veseli?i-va, drep?ilor, în Domnul ?i lauda?i pomenirea sfin?eniei Lui!
Psa 98:1  (Un psalm al lui David.) Cânta?i Domnului cântare noua, ca lucruri minunate a facut Domnul. Mântuitu-l-a pe el dreapta Lui ?i bra?ul cel sfânt al Lui.
Psa 98:2  Cunoscuta a facut Domnul mântuirea Sa; înaintea neamurilor a descoperit dreptatea Sa.
Psa 98:3  Pomenit-a mila Sa lui Iacob ?i adevarul Sau casei lui Israel; vazut-au toate marginile pamântului mântuirea Dumnezeului nostru.
Psa 98:4  Striga?i lui Dumnezeu tot pamântul; cânta?i ?i va bucura?i ?i cânta?i.
Psa 98:5  Cânta?i Domnului cu alauta, cu alauta ?i în sunet de psaltire;
Psa 98:6  Cu trâmbi?e ?i în sunet de corn, striga?i înaintea Împaratului ?i Domnului.
Psa 98:7  Sa se zguduie marea ?i plinirea ei, lumea ?i cei ce locuiesc în ea.
Psa 98:8  Râurile vor bate din palme, deodata, mun?ii se vor bucura de fa?a Domnului, ca vine, vine sa judece pamântul.
Psa 98:9  Judeca-va lumea cu dreptate ?i popoarele cu nepartinire.
Psa 99:1  (Un psalm al lui David.) Domnul împara?e?te: sa tremure popoarele; ?ade pe heruvimi: sa se cutremure pamântul.
Psa 99:2  Domnul în Sion este mare ?i înalt peste toate popoarele.
Psa 99:3  Sa se laude numele Tau cel mare, ca înfrico?ator ?i sfânt este. ?i cinstea împaratului iube?te dreptatea.
Psa 99:4  Tu ai întemeiat dreptatea; judecata ?i dreptate în Iacob, Tu ai facut.
Psa 99:5  Înal?a?i pe Domnul Dumnezeul nostru ?i va închina?i a?ternutului picioarelor Lui, ca sfânt este!
Psa 99:6  Moise ?i Aaron, între preo?ii Lui ?i Samuel între cei ce cheama numele Lui. Chemat-au pe Domnul ?i El i-a auzit pe ei,
Psa 99:7  În stâlp de nor graia catre ei; caci pazeau marturiile Lui ?i poruncile pe care le-a dat lor.
Psa 99:8  Doamne, Dumnezeul nostru, Tu i-ai auzit pe ei; Dumnezeule, Tu Te-ai milostivit de ei ?i ai rasplatit toate faptele lor.
Psa 99:9  Înal?a?i pe Domnul Dumnezeul nostru ?i va închina?i în Muntele cel sfânt al Lui, ca sfânt este Domnul, Dumnezeul nostru!
Psa 100:1  (Un psalm al lui David; spre lauda.) Striga?i Domnului tot pamântul,
Psa 100:2  Sluji?i Domnului cu veselie, intra?i înaintea Lui cu bucurie.
Psa 100:3  Cunoa?te?i ca Domnul, El este Dumnezeul nostru; El ne-a facut pe noi ?i nu noi.
Psa 100:4  Iar noi poporul Lui ?i oile pa?unii Lui. Intra?i pe por?ile Lui cu lauda ?i în cur?ile Lui cu cântari lauda?i-L pe El.
Psa 100:5  Cânta?i numele Lui! Ca bun este Domnul; în veac este mila Lui ?i din neam în neam adevarul Lui.
Psa 101:1  (Un psalm al lui David.) Mila ?i judecata Ta voi cânta ?ie, Doamne.
Psa 101:2  Cânta-voi ?i voi merge, cu pricepere, în cale fara prihana. Când vei veni la mine? Umblat-am întru nerautatea inimii mele, în casa mea.
Psa 101:3  N-am pus înaintea ochilor mei lucru nelegiuit; pe calcatorii de lege i-am urât.
Psa 101:4  Nu s-a lipit de mine inima îndaratnica; pe cel rau, care se departa de mine, nu l-am cunoscut.
Psa 101:5  Pe cel ce clevetea în ascuns pe vecinul sau, pe acela l-am izgonit. Cu cel mândru cu ochiul ?i nesa?ios cu inima, cu acela n-am mâncat.
Psa 101:6  Ochii mei sunt peste credincio?ii pamântului, ca sa ?ada ei împreuna cu mine. Cel ce umbla pe cale fara prihana, acela îmi slujea.
Psa 101:7  Nu va locui în casa mea cel mândru; cel ce graie?te nedrepta?i nu va sta înaintea ochilor mei.
Psa 101:8  În dimine?i voi judeca pe to?i pacato?ii pamântului, ca sa nimicesc din cetatea Domnului pe to?i cei ce lucreaza faradelegea.
Psa 102:1  (Rugaciunea unui sarac mâhnit, care-?i îndrepteaza înaintea Domnului ruga sa.) Doamne, auzi rugaciunea mea, ?i strigarea mea la Tine sa ajunga!
Psa 102:2  Sa nu întorci fa?a Ta de la mine; în orice zi ma necajesc, pleaca spre mine urechea Ta! În orice zi Te voi chema, degraba auzi-ma!
Psa 102:3  Ca s-au stins ca fumul zilele mele ?i oasele mele ca uscaciunea s-au facut.
Psa 102:4  Ranita este inima mea ?i s-a uscat ca iarba; ca am uitat sa-mi manânc pâinea mea.
Psa 102:5  De glasul suspinului meu, osul meu s-a lipit de carnea mea.
Psa 102:6  Asemanatu-m-am cu pelicanul din pustiu; ajuns-am ca bufni?a din darâmaturi.
Psa 102:7  Privegheat-am ?i am ajuns ca o pasare singuratica pe acoperi?.
Psa 102:8  Toata ziua m-au ocarât vrajma?ii mei ?i cei ce ma laudau, împotriva mea se jurau.
Psa 102:9  Ca cenu?a am mâncat, în loc de pâine, ?i bautura mea cu plângere am amestecat-o,
Psa 102:10  Din pricina urgiei Tale ?i a mâniei Tale; ca ridicându-ma eu, m-ai surpat.
Psa 102:11  Zilele mele ca umbra s-au plecat ?i eu ca iarba m-am uscat.
Psa 102:12  Iar Tu, Doamne, în veac ramâi ?i pomenirea Ta din neam în neam.
Psa 102:13  Sculându-Te, vei milui Sionul, ca vremea este sa-l miluie?ti pe el, ca a venit vremea.
Psa 102:14  Ca au iubit robii Tai pietrele lui ?i de ?arâna lui le va fi mila.
Psa 102:15  ?i se vor teme neamurile de numele Domnului ?i to?i împara?ii pamântului de slava Ta.
Psa 102:16  Ca va zidi Domnul Sionul ?i se va arata întru slava Sa.
Psa 102:17  Cautat-a spre rugaciunea celor smeri?i ?i n-a dispre?uit cererea lor.
Psa 102:18  Sa se scrie acestea pentru neamul ce va sa vina ?i poporul ce se zide?te va lauda pe Domnul;
Psa 102:19  Ca a privit din înal?imea cea sfânta a Lui, Domnul din cer pe pamânt a privit,
Psa 102:20  Ca sa auda suspinul celor fereca?i, sa dezlege pe fiii celor omorâ?i,
Psa 102:21  Sa vesteasca în Sion numele Domnului ?i lauda Lui în Ierusalim,
Psa 102:22  Când se vor aduna popoarele împreuna ?i împara?iile ca sa slujeasca Domnului.
Psa 102:23  Zis-am catre Dumnezeu în calea tariei Lui: Veste?te-mi pu?inatatea zilelor mele.
Psa 102:24  Nu ma lua la jumatatea zilelor mele, ca anii Tai, Doamne, sunt din neam în neam.
Psa 102:25  Dintru început Tu, Doamne, pamântul l-ai întemeiat ?i lucrul mâinilor Tale, sunt cerurile.
Psa 102:26  Acelea vor pieri, iar Tu vei ramâne ?i to?i ca o haina se vor învechi ?i ca un ve?mânt îi vei schimba ?i se vor schimba.
Psa 102:27  Dar Tu acela?i e?ti ?i anii Tai nu se vor împu?ina.
Psa 102:28  Fiii robilor Tai vor locui pamântul lor ?i semin?ia lor în veac va propa?i.
Psa 103:1  (Un psalm al lui David.) Binecuvinteaza, suflete al meu, pe Domnul ?i toate cele dinlauntrul meu, numele cel sfânt al Lui.
Psa 103:2  Binecuvinteaza, suflete al meu, pe Domnul ?i nu uita toate rasplatirile Lui.
Psa 103:3  Pe Cel ce cura?e?te toate faradelegile tale, pe Cel ce vindeca toate bolile tale;
Psa 103:4  Pe Cel ce izbave?te din stricaciune via?a ta, pe Cel ce te încununeaza cu mila ?i cu îndurari;
Psa 103:5  Pe Cel ce umple de bunata?i pofta ta; înnoi-se-vor ca ale vulturului tinere?ile tale.
Psa 103:6  Cel ce face milostenie, Domnul, ?i judecata tuturor celor ce li se face strâmbatate.
Psa 103:7  Cunoscute a facut caile Sale lui Moise, fiilor lui Israel voile Sale.
Psa 103:8  Îndurat ?i milostiv este Domnul, îndelung-rabdator ?i mult-milostiv.
Psa 103:9  Nu pâna în sfâr?it se va iu?i, nici în veac se va mânia.
Psa 103:10  Nu dupa pacatele noastre a facut noua, nici dupa faradelegile noastre a rasplatit noua,
Psa 103:11  Ci cât este departe cerul de pamânt, atât este de mare mila Lui, spre cei ce se tem de El.
Psa 103:12  Pe cât sunt de departe rasariturile de la apusuri, departat-a de la noi faradelegile noastre.
Psa 103:13  În ce chip miluie?te tatal pe fii, a?a a miluit Domnul pe cei ce se tem de El;
Psa 103:14  Ca El a cunoscut zidirea noastra, adusu-?i-a aminte ca ?arâna suntem.
Psa 103:15  Omul ca iarba, zilele lui ca floarea câmpului; a?a va înflori.
Psa 103:16  Ca vânt a trecut peste el ?i nu va mai fi ?i nu se va mai cunoa?te înca locul sau.
Psa 103:17  Iar mila Domnului din veac în veac spre cei ce se tem de Dânsul,
Psa 103:18  ?i dreptatea Lui spre fiii fiilor, spre cei ce pazesc legamântul Lui
Psa 103:19  ?i î?i aduc aminte de poruncile Lui, ca sa le faca pe ele. Domnul în cer a gatit scaunul Sau ?i împara?ia Lui peste to?i stapâne?te.
Psa 103:20  Binecuvânta?i pe Domnul to?i îngerii Lui, cei tari la vârtute, care face?i cuvântul Lui ?i auzi?i glasul cuvintelor Lui.
Psa 103:21  Binecuvânta?i pe Domnul toate puterile Lui, slugile Lui, care face?i voia Lui.
Psa 103:22  Binecuvânta?i pe Domnul toate lucrurile Lui; în tot locul stapânirii Lui, binecuvinteaza suflete al meu pe Domnul.
Psa 104:1  (Un psalm al lui David.) Binecuvinteaza, suflete al meu, pe Domnul! Doamne, Dumnezeul meu, maritu-Te-ai foarte.
Psa 104:2  Întru stralucire ?i în mare podoaba Te-ai îmbracat; Cel ce Te îmbraci cu lumina ca ?i cu o haina;
Psa 104:3  Cel ce întinzi cerul ca un cort; Cel ce acoperi cu ape cele mai de deasupra ale lui; Cel ce pui norii suirea Ta; Cel ce umbli peste aripile vânturilor;
Psa 104:4  Cel ce faci pe îngerii Tai duhuri ?i pe slugile Tale para de foc;
Psa 104:5  Cel ce ai întemeiat pamântul pe întarirea lui ?i nu se va clatina în veacul veacului.
Psa 104:6  Adâncul ca o haina este îmbracamintea lui; peste mun?i vor sta ape.
Psa 104:7  De certarea Ta vor fugi, de glasul tunetului Tau se vor înfrico?a.
Psa 104:8  Se suie mun?i ?i se coboara vai, în locul în care le-ai întemeiat pe ele.
Psa 104:9  Hotar ai pus, pe care nu-l vor trece ?i nici nu se vor întoarce sa acopere pamântul.
Psa 104:10  Cel ce trimi?i izvoare în vai, prin mijlocul mun?ilor vor trece ape;
Psa 104:11  Adapa-se-vor toate fiarele câmpului, asinii salbatici setea î?i vor potoli.
Psa 104:12  Peste acelea pasarile cerului vor locui; din mijlocul stâncilor vor da glas.
Psa 104:13  Cel ce adapi mun?ii din cele mai de deasupra ale Tale, din rodul lucrurilor Tale se va satura pamântul.
Psa 104:14  Cel ce rasari iarba dobitoacelor ?i verdea?a spre slujba oamenilor;
Psa 104:15  Ca sa scoata pâine din pamânt ?i vinul vesele?te inima omului;
Psa 104:16  Ca sa veseleasca fa?a cu untdelemn ?i pâinea inima omului o întare?te.
Psa 104:17  Satura-se-vor copacii câmpului, cedrii Libanului pe care i-ai sadit; acolo pasarile î?i vor face cuib.
Psa 104:18  Loca?ul cocostârcului în chiparo?i. Mun?ii cei înal?i adapost cerbilor stâncile scapare iepurilor.
Psa 104:19  Facut-ai luna spre vremi, soarele ?i-a cunoscut apusul sau.
Psa 104:20  Pus-ai întuneric ?i s-a facut noapte, când vor ie?i toate fiarele padurii;
Psa 104:21  Puii leilor mugesc ca sa apuce ?i sa ceara de la Dumnezeu mâncarea lor.
Psa 104:22  Rasarit-a soarele ?i s-au adunat ?i în culcu?urile lor se vor culca.
Psa 104:23  Ie?i-va omul la lucrul sau ?i la lucrarea sa pâna seara.
Psa 104:24  Cât s-au marit lucrurile Tale, Doamne, toate cu în?elepciune le-ai facut! Umplutu-s-a pamântul de zidirea Ta.
Psa 104:25  Marea aceasta este mare ?i larga; acolo se gasesc târâtoare, carora nu este numar, vieta?i mici ?i mari.
Psa 104:26  Acolo corabiile umbla; balaurul acesta pe care l-ai zidit, ca sa se joace în ea.
Psa 104:27  Toate catre Tine a?teapta ca sa le dai lor hrana la buna vreme.
Psa 104:28  Dându-le Tu lor, vor aduna, deschizând Tu mâna Ta, toate se vor umple de bunata?i;
Psa 104:29  Dar întorcându-?i Tu fa?a Ta, se vor tulbura; lua-vei duhul lor ?i se vor sfâr?i ?i în ?arâna se vor întoarce.
Psa 104:30  Trimite-vei duhul Tau ?i se vor zidi ?i vei înnoi fa?a pamântului.
Psa 104:31  Fie slava Domnului în veac! Veseli-se-va Domnul de lucrurile Sale.
Psa 104:32  Cel ce cauta spre pamânt ?i-l face pe el de se cutremura; Cel ce se atinge de mun?i ?i fumega.
Psa 104:33  Cânta-voi Domnului în via?a mea, cânta-voi Dumnezeului meu cât voi fi.
Psa 104:34  Placute sa-I fie Lui cuvintele mele, iar eu ma voi veseli de Domnul.
Psa 104:35  Piara pacato?ii de pe pamânt ?i cei fara de lege, ca sa nu mai fie. Binecuvinteaza, suflete al meu, pe Domnul.
Psa 105:1  Aliluia! Lauda?i pe Domnul ?i chema?i numele Lui; vesti?i între neamuri lucrurile Lui.
Psa 105:2  Cânta?i-I ?i-L lauda?i pe El; spune?i toate minunile Lui.
Psa 105:3  Lauda?i-va cu numele cel sfânt al Lui; veseleasca-se inima celor ce cauta pe Domnul.
Psa 105:4  Cauta?i pe Domnul ?i va întari?i; cauta?i fala Lui, pururea.
Psa 105:5  Aduce?i-va aminte de minunile Lui, pe care le-a facut; de minunile Lui ?i de judeca?ile gurii Lui.
Psa 105:6  Semin?ia lui Avraam, robii Lui, fiii lui Iacob, ale?ii Lui.
Psa 105:7  Acesta este Domnul Dumnezeul nostru, în tot pamântul judeca?ile Lui.
Psa 105:8  Adusu-?i-a aminte în veac de legamântul Lui, de cuvântul pe care l-a poruncit într-o mie de neamuri,
Psa 105:9  Pe care l-a încheiat cu Avraam ?i de juramântul Sau lui Isaac.
Psa 105:10  ?i l-a pus pe el lui Iacob, spre porunca, ?i lui Israel legatura ve?nica,
Psa 105:11  Zicând: "?ie î?i voi da pamântul Canaan, partea mo?tenirii tale".
Psa 105:12  Atunci când erau ei pu?ini la numar ?i straini în pamântul lor
Psa 105:13  ?i au trecut de la un neam la altul, de la o împara?ie la un alt popor,
Psa 105:14  N-a lasat om sa le faca strâmbatate ?i a certat pentru ei pe împara?i, zicându-le:
Psa 105:15  "Nu va atinge?i de un?ii Mei ?i nu vicleni?i împotriva profe?ilor Mei".
Psa 105:16  ?i a chemat foamete pe pamânt ?i a sfarâmat paiul de grâu.
Psa 105:17  Trimis-a înaintea lor om; rob a fost rânduit Iosif.
Psa 105:18  Smeritu-l-au, punând în obezi picioarele lui; prin fier a trecut sufletul lui,
Psa 105:19  Pâna ce a venit cuvântul Lui. Cuvântul Domnului l-a aprins pe el;
Psa 105:20  Trimis-a împaratul ?i l-a slobozit, capetenia poporului ?i l-a liberat pe el.
Psa 105:21  Pusu-l-a pe el domn casei lui ?i capetenie peste toata avu?ia lui,
Psa 105:22  Ca sa înve?e pe capeteniile lui, ca pe sine însu?i ?i pe batrânii lui sa-i în?elep?easca.
Psa 105:23  ?i a intrat Israel în Egipt ?i Iacob a locuit ca strain, în pamântul lui Ham.
Psa 105:24  ?i a înmul?it pe poporul lui foarte ?i l-a întarit pe el mai mult decât pe vrajma?ii lui.
Psa 105:25  Întors-a inima lor, ca sa urasca pe poporul Sau, ca sa vicleneasca împotriva robilor Sai.
Psa 105:26  Trimis-a pe Moise robul Sau, pe Aaron, pe care l-a ales.
Psa 105:27  Pus-a în ei cuvintele semnelor ?i minunilor Lui în pamântul lui Ham.
Psa 105:28  Trimis-a întuneric ?i i-a întunecat, caci au amarât cuvintele Lui;
Psa 105:29  Prefacut-a apele lor în sânge ?i a omorât pe?tii lor;
Psa 105:30  Scos-a pamântul lor broa?te în camarile împara?ilor lor.
Psa 105:31  Zis-a ?i a venit musca câineasca ?i mul?ime de mu?te în toate hotarele lor.
Psa 105:32  Pus-a în ploile lor grindina, foc arzator în pamântul lor;
Psa 105:33  ?i a batut viile lor ?i smochinii lor ?i a sfarâmat pomii hotarelor lor.
Psa 105:34  Zis-a ?i a venit lacusta ?i omida fara numar.
Psa 105:35  ?i a mâncat toata iarba în pamântul lor ?i a mâncat rodul pamântului lor,
Psa 105:36  ?i a batut pe to?i întâi-nascu?ii din pamântul lor, pârga întregii lor osteneli.
Psa 105:37  ?i i-a scos pe ei cu argint ?i cu aur ?i nu era în semin?iile lor bolnav.
Psa 105:38  Veselitu-s-a Egiptul la ie?irea lor, ca frica de ei îi cuprinsese.
Psa 105:39  Întins-a nor spre acoperirea lor ?i foc ca sa le lumineze noaptea.
Psa 105:40  Cerut-au ?i au venit prepeli?e ?i cu pâine cereasca i-a saturat pe ei.
Psa 105:41  Despicat-a piatra ?i au curs ape ?i au curs râuri în pamânt fara de apa.
Psa 105:42  Ca ?i-a adus aminte de cuvântul cel sfânt al Lui, spus lui Avraam, robul Lui.
Psa 105:43  ?i a scos pe poporul Sau, întru bucurie ?i pe cei ale?i ai Sai, întru veselie.
Psa 105:44  ?i le-a dat lor ?arile neamurilor ?i ostenelile popoarelor au mo?tenit,
Psa 105:45  Ca sa pazeasca drepta?ile Lui ?i legea Lui s-o ?ina.
Psa 106:1  Aliluia! Lauda?i pe Domnul ca este bun, ca în veac este mila Lui.
Psa 106:2  Cine va grai puterile Domnului ?i cine va face auzite toate laudele Lui?
Psa 106:3  Ferici?i cei ce pazesc judecata ?i fac dreptate în toata vremea.
Psa 106:4  Adu-?i aminte de noi, Doamne, întru bunavoin?a Ta fa?a de poporul Tau; cerceteaza-ne pe noi cu mântuirea Ta,
Psa 106:5  Ca sa vedem întru bunata?i pe ale?ii Tai, sa ne bucuram de veselia poporului Tau ?i sa ne laudam cu mo?tenirea Ta.
Psa 106:6  Pacatuit-am ca ?i parin?ii no?tri, nelegiuit-am, facut-am strâmbatate.
Psa 106:7  Parin?ii no?tri în Egipt n-au în?eles minunile Tale, nu ?i-au adus aminte de mul?imea milei Tale ?i Te-au amarât când s-au suit la Marea Ro?ie.
Psa 106:8  Dar i-a mântuit pe ei pentru numele Sau, ca sa faca cunoscuta puterea Lui.
Psa 106:9  El a certat Marea Ro?ie ?i a secat-o ?i i-a condus pe ei prin  adâncul marii ca prin pustiu;
Psa 106:10  El i-a scos pe ei din mâna celor ce-i urau ?i i-a izbavit pe ei din mâna vrajma?ului;
Psa 106:11  ?i a acoperit apa pe cei ce-i asupreau pe ei, nici unul din ei n-a ramas.
Psa 106:12  ?i au crezut în cuvintele Lui ?i au cântat lauda Lui;
Psa 106:13  Dar degrab au uitat lucrurile Lui ?i n-au suferit sfatul Lui;
Psa 106:14  Ci au fost cuprin?i de mare pofta, în pustiu, ?i au ispitit pe Dumnezeu, în loc fara de apa.
Psa 106:15  ?i le-a împlinit cererea lor ?i a saturat sufletele lor.
Psa 106:16  ?i au mâniat pe Moise în tabara ?i pe Aaron, sfântul Domnului.
Psa 106:17  S-a deschis pamântul ?i a înghi?it pe Datan ?i a acoperit adunarea lui Abiron.
Psa 106:18  ?i s-a aprins foc în adunarea lor, vapaie a ars pe pacato?i.
Psa 106:19  ?i au facut vi?ei în Horeb ?i s-au închinat idolului.
Psa 106:20  ?i au schimbat slava Lui întru asemanare de vi?el, care manânca iarba.
Psa 106:21  Au uitat pe Dumnezeu, Care i-a izbavit pe ei, Care a facut lucruri mari în Egipt,
Psa 106:22  Lucruri minunate în pamântul lui Ham ?i înfrico?atoare în Marea Ro?ie.
Psa 106:23  Atunci a zis sa-i piarda pe dân?ii, ?i i-ar fi pierdut, daca Moise, alesul Lui, n-ar fi stat înaintea fe?ei Lui ca sa întoarca mânia Lui ?i sa nu-i piarda.
Psa 106:24  Apoi ei au dispre?uit pamântul cel dorit ?i n-au crezut în cuvântul Lui,
Psa 106:25  Ci au cârtit în corturile lor ?i n-au ascultat glasul Domnului.
Psa 106:26  Atunci El a ridicat mâna Sa asupra lor, ca sa-i doboare pe ei în pustiu
Psa 106:27  ?i sa doboare samân?a lor întru neamuri ?i sa-i risipeasca pe ei în toate par?ile.
Psa 106:28  Au jertfit lui Baal-Peor ?i au mâncat jertfele mor?ilor
Psa 106:29  ?i L-au întarâtat pe El cu faptele lor ?i au murit mul?i dintre ei.
Psa 106:30  Dar a stat Finees ?i L-a îmblânzit ?i a încetat bataia
Psa 106:31  ?i i s-a socotit lui întru dreptate, din neam în neam pâna în veac.
Psa 106:32  Apoi L-au mâniat pe El la apa certarii ?i Moise a suferit pentru ei,
Psa 106:33  Ca au amarât duhul lui ?i a grait nesocotit cu buzele lui.
Psa 106:34  N-au nimicit neamurile de care le-a pomenit Domnul;
Psa 106:35  Ci s-au amestecat cu neamurile ?i au deprins lucrurile lor
Psa 106:36  ?i au slujit idolilor lor ?i s-au smintit.
Psa 106:37  ?i-au jertfit pe fiii lor ?i pe fetele lor idolilor,
Psa 106:38  Au varsat sânge nevinovat, sângele fiilor lor ?i al fetelor lor, pe care i-au jertfit idolilor din Canaan ?i s-a spurcat pamântul de sânge.
Psa 106:39  S-au pângarit de lucrurile lor ?i s-au desfrânat cu faptele lor.
Psa 106:40  Atunci S-a aprins de mânie Domnul împotriva poporului Sau ?i a urât mo?tenirea Sa
Psa 106:41  ?i i-a dat pe ei în mâinile neamurilor ?i i-au stapânit pe ei cei ce-i urau pe ei.
Psa 106:42  Vrajma?ii lor i-au asuprit pe ei ?i au fost neferici?i sub mâinile lor.
Psa 106:43  De multe ori Domnul i-a izbavit pe ei, dar ei L-au amarât pe El cu sfatul lor ?i i-a umilit pentru faradelegile lor.
Psa 106:44  Dar Domnul i-a vazut când se necajeau ei, a auzit rugaciunea lor,
Psa 106:45  ?i ?i-a adus aminte de legamântul Lui ?i S-a cait dupa mul?imea milei Sale;
Psa 106:46  ?i le-a dat sa gaseasca mila înaintea celor ce i-au robit pe ei.
Psa 106:47  Izbave?te-ne, Doamne Dumnezeul nostru, ?i ne aduna din neamuri, ca sa laudam numele cel sfânt al Tau ?i sa ne falim cu lauda Ta.
Psa 106:48  Binecuvântat este Domnul Dumnezeul lui Israel, din veac ?i pâna în veac. Tot poporul sa zica: Amin. Amin.
Psa 107:1  Lauda?i pe Domnul ca este bun, ca în veac este mila Lui.
Psa 107:2  Sa spuna cei izbavi?i de Domnul, pe care i-a izbavit din mâna vrajma?ului.
Psa 107:3  Din ?ari i-a adunat pe ei, de la rasarit ?i de la apus, de la miazanoapte ?i de la miazazi.
Psa 107:4  Ratacit-au în pustie, în pamânt fara de apa ?i cale spre cetatea de locuit n-au gasit.
Psa 107:5  Erau flamânzi ?i înseta?i; sufletul lor într-în?ii se sfâr?ea;
Psa 107:6  Dar au strigat catre Domnul în necazurile lor ?i din nevoile lor i-a izbavit pe ei
Psa 107:7  ?i i-a pova?uit pe cale dreapta, ca sa mearga spre cetatea de locuit.
Psa 107:8  Laudat sa fie Domnul pentru milele Lui, pentru minunile Lui, pe care le-a facut fiilor oamenilor.
Psa 107:9  Ca a saturat suflet însetat ?i suflet flamând a umplut de bunata?i.
Psa 107:10  ?edeau în întuneric ?i în umbra mor?ii; erau fereca?i de saracie ?i de fier,
Psa 107:11  Pentru ca au amarât cuvintele Domnului ?i sfatul Celui Preaînalt au întarâtat.
Psa 107:12  El a umilit întru osteneli inima lor; slabit-au ?i nu era cine sa le ajute;
Psa 107:13  Dar au strigat catre Domnul în necazurile lor ?i din nevoile lor i-a izbavit pe ei.
Psa 107:14  ?i i-a scos pe ei din întuneric ?i din umbra mor?ii ?i legaturile lor le-a rupt.
Psa 107:15  Laudat sa fie Domnul pentru milele Lui, pentru minunile Lui, pe care le-a facut fiilor oamenilor!
Psa 107:16  Ca a sfarâmat por?i de arama ?i zavoare de fier a frânt
Psa 107:17  ?i i-a ajutat sa iasa din calea faradelegii lor, caci pentru faradelegile lor au fost umili?i.
Psa 107:18  Urât-a sufletul lor orice mâncare ?i s-au apropiat de por?ile mor?ii.
Psa 107:19  Dar au strigat catre Domnul în necazurile lor ?i din nevoile lor i-a izbavit.
Psa 107:20  Trimis-a cuvântul Sau ?i i-a vindecat pe ei ?i i-a izbavit pe ei din stricaciunile lor.
Psa 107:21  Laudat sa fie Domnul pentru milele Lui, pentru minunile Lui, pe care le-a facut fiilor oamenilor!
Psa 107:22  ?i sa-I jertfeasca Lui jertfa de lauda ?i sa vesteasca lucrurile Lui, în bucurie.
Psa 107:23  Cei ce se coboara la mare în corabii, cei ce-?i fac lucrarea lor în ape multe,
Psa 107:24  Aceia au vazut lucrurile Domnului ?i minunile Lui întru adânc.
Psa 107:25  El a zis ?i s-a pornit vânt furtunos ?i s-au înal?at valurile marii.
Psa 107:26  Se urcau pâna la ceruri ?i se coborau pâna în adâncuri, iar sufletul lor întru primejdii încremenea.
Psa 107:27  Se tulburau ?i se clatinau ca un om beat ?i toata priceperea lor a pierit.
Psa 107:28  Dar au strigat catre Domnul în necazurile lor ?i din nevoile lor ia izbavit
Psa 107:29  ?i i-a poruncit furtunii ?i s-a lini?tit ?i au tacut valurile marii.
Psa 107:30  ?i s-au veselit ei, ca s-au lini?tit valurile ?i Domnul i-a pova?uit pe ei la limanul dorit de ei.
Psa 107:31  Laudat sa fie Domnul pentru milele Lui, pentru minunile Lui, pe care le-a facut fiilor oamenilor!
Psa 107:32  Înal?aii-L pe El în adunarea poporului ?i în scaunul batrânilor lauda?i-L pe El,
Psa 107:33  Prefacut-a râurile în pamânt pustiu, izvoarele de apa în pamânt însetat
Psa 107:34  ?i pamântul cel roditor în pamânt sarat, din pricina celor ce locuiesc pe el.
Psa 107:35  Prefacut-a pustiul în iezer de ape, iar pamântul cel fara de apa în izvoare de ape,
Psa 107:36  ?i a a?ezat acolo pe cei flamânzi ?i au zidit cetate de locuit
Psa 107:37  ?i au semanat ?arine ?i au sadit vii ?i au strâns bel?ug de roade
Psa 107:38  ?i i-a binecuvântat pe ei ?i s-au înmul?it foarte ?i vitele lor nu le-a împu?inat.
Psa 107:39  ?i iara?i au fost împu?ina?i ?i chinui?i de apasarea necazurilor ?i a durerii.
Psa 107:40  Aruncat-a dispre? asupra capeteniilor lor ?i i-a ratacit pe ei în loc neumblat ?i fara de cale.
Psa 107:41  Dar pe sarac l-a izbavit de saracie ?i i-a pus pe ei ca pe ni?te oi de mo?tenire.
Psa 107:42  Vedea-vor drep?ii ?i se vor veseli ?i toata faradelegea î?i va astupa gura ei.
Psa 107:43  Cine este în?elept va pazi acestea ?i va pricepe milele Domnului.
Psa 108:1  (Un psalm al lui David.) Gata este inima mea, Dumnezeule, gata este inima mea; cânta-voi ?i voi lauda întru inima mea.
Psa 108:2  De?teapta-te slava mea! De?teapta-te psaltire ?i alauta! De?tepta-ma-voi diminea?a.
Psa 108:3  Lauda-Te-voi între popoare, Doamne, cânta-voi ?ie între neamuri,
Psa 108:4  Ca mai mare decât cerurile este mila Ta ?i pâna la nori adevarul Tau.
Psa 108:5  Înal?a-Te peste ceruri, Dumnezeule, ?i peste tot pamântul slava Ta, ca sa se izbaveasca cei placu?i ai Tai.
Psa 108:6  Mântuie?te-ma cu dreapta Ta ?i ma auzi. Dumnezeu a grait în locul cel sfânt al Lui:
Psa 108:7  Înal?a-Ma-voi ?i voi împar?i Sichemul ?i Valea Sucot o voi masura.
Psa 108:8  Al Meu este Galaad ?i al Meu este Manase ?i Efraim sprijinul capului Meu,
Psa 108:9  Iuda, legiuitorul Meu; Moab, vas al spalarii Mele; spre Idumeea voi arunca încal?amintea Mea. Mie cei de alt neam Mi s-au supus".
Psa 108:10  Cine ma va duce la cetatea întarita? Cine ma va pova?ui pâna în Idumeea?
Psa 108:11  Oare, nu Tu, Dumnezeule, Cel ce ne-ai lepadat pe noi? Oare, nu vei ie?i, Dumnezeule, cu o?tirile noastre?
Psa 108:12  Da-ne noua ajutor, ca sa ie?im din necaz, ca de?arta este izbavirea cea de la oameni.
Psa 108:13  Cu Dumnezeu vom birui ?i El va nimici pe vrajma?ii no?tri.
Psa 109:1  (Un psalm al lui David; mai-marelui cântare?ilor.) Dumnezeule, lauda mea n-o ?ine sub tacere.
Psa 109:2  Ca gura pacatosului ?i gura vicleanului asupra mea s-au deschis.
Psa 109:3  Grait-au împotriva mea cu limba vicleana ?i cu cuvinte de ura m-au înconjurat ?i s-au luptat cu mine în zadar.
Psa 109:4  În loc sa ma iubeasca, ma cleveteau, iar eu ma rugam.
Psa 109:5  Pus-au împotriva mea rele în loc de bune ?i ura în locul iubirii mele.
Psa 109:6  Pune peste dânsul pe cel pacatos ?i diavolul sa stea de-a dreapta lui.
Psa 109:7  Când se va judeca sa iasa osândit, iar rugaciunea lui sa se prefaca în pacat.
Psa 109:8  Sa fie zilele lui pu?ine ?i dregatoria lui sa o ia altul;
Psa 109:9  Sa ajunga copiii lui orfani ?i femeia lui vaduva;
Psa 109:10  Sa fie stramuta?i copiii lui ?i sa cer?easca; sa fie sco?i din cur?ile caselor lor;
Psa 109:11  Sa smulga camatarul toata averea lui; sa rapeasca strainii ostenelile lui;
Psa 109:12  Sa nu aiba sprijinitor ?i nici orfanii lui miluitor;
Psa 109:13  Sa piara copiii lui ?i într-un neam sa se stinga numele lui;
Psa 109:14  Sa se pomeneasca faradelegea parin?ilor lui înaintea Domnului ?i pacatul maicii lui sa nu se ?tearga;
Psa 109:15  Sa fie înaintea Domnului pururea ?i sa piara de pe pamânt pomenirea lui, pentru ca nu ?i-a adus aminte sa faca mila.
Psa 109:16  ?i a prigonit pe cel sarman, pe cel sarac ?i pe cel smerit cu inima, ca sa-l omoare.
Psa 109:17  ?i a iubit blestemul ?i va veni asupra lui; ?i n-a voit binecuvântarea ?i se va îndeparta de la el.
Psa 109:18  ?i s-a îmbracat cu blestemul ca ?i cu o haina ?i a intrat ca apa înlauntrul lui ?i ca untdelemnul în oasele lui.
Psa 109:19  Sa-i fie lui ca o haina cu care se îmbraca ?i ca un brâu cu care pururea se încinge.
Psa 109:20  Aceasta sa fie rasplata celor ce ma clevetesc pe mine înaintea Domnului ?i graiesc rele împotriva sufletului meu.
Psa 109:21  Dar Tu, Doamne, fa cu mine mila, pentru numele Tau, ca buna este mila Ta.
Psa 109:22  Izbave?te-ma, ca sarac ?i sarman sunt eu ?i inima mea s-a tulburat înlauntrul meu.
Psa 109:23  Ca umbra ce se înclina m-am trecut; ca bataia de aripi a lacustelor tremur.
Psa 109:24  Genunchii mei au slabit de post ?i trupul meu s-a istovit de lipsa untdelemnului
Psa 109:25  ?i eu am ajuns lor ocara. M-au vazut ?i au clatinat cu capetele lor.
Psa 109:26  Ajuta-ma, Doamne Dumnezeul meu, mântuie?te-ma, dupa mila Ta,
Psa 109:27  ?i sa cunoasca ei ca mâna Ta este aceasta ?i Tu, Doamne, ai facut-o pe ea.
Psa 109:28  Ei vor blestema ?i Tu vei binecuvânta. Cei ce se scoala împotriva mea sa se ru?ineze, iar robul Tau sa se veseleasca.
Psa 109:29  Sa se îmbrace cei ce ma clevetesc pe mine cu ocara ?i cu ru?inea lor ca ?i cu un ve?mânt sa se înveleasca.
Psa 109:30  Lauda-voi pe Domnul foarte cu gura mea ?i în mijlocul multora Îl voi preaslavi pe El,
Psa 109:31  Ca a stat de-a dreapta saracului, ca sa izbaveasca sufletul lui de cei ce-l prigonesc.
Psa 110:1  (Un psalm al lui David.) Zis-a Domnul Domnului Meu: "?ezi de-a dreapta Mea, pâna ce voi pune pe vrajma?ii Tai a?ternut picioarelor Tale".
Psa 110:2  Toiagul puterii Tale ?i-l va trimite Domnul din Sion, zicând: "Stapâne?te în mijlocul vrajma?ilor Tai.
Psa 110:3  Cu Tine este poporul Tau în ziua puterii Tale, întru stralucirile sfin?ilor Tai. Din pântece mai înainte de luceafar Te-am nascut".
Psa 110:4  Juratu-S-a Domnul ?i nu-I va parea rau: "Tu e?ti preot în veac, dupa rânduiala lui Melchisedec".
Psa 110:5  Domnul este de-a dreapta Ta; sfarâmat-a în ziua mâniei Sale împara?i.
Psa 110:6  Judeca-va între neamuri; va umple totul de ruini; va zdrobi capetele multora pe pamânt.
Psa 110:7  Din pârâu pe cale va bea; pentru aceasta va înal?a capul.
Psa 111:1  (Un psalm scris pe vremea lui Neemia.) Lauda-Te-voi, Doamne, cu toata inima mea, în sfatul celor drep?i ?i în adunare.
Psa 111:2  Mari sunt lucrurile Domnului ?i potrivite tuturor voilor Lui.
Psa 111:3  Lauda ?i mare?ie este lucrul Lui ?i dreptatea Lui ramâne în veacul veacului.
Psa 111:4  Pomenire a facut de minunile Sale. Milostiv ?i îndurat este Domnul.
Psa 111:5  Hrana a dat celor ce se tem de Dânsul; aduce?i-va aminte în veac de legamântul Lui.
Psa 111:6  Taria lucrurilor Sale a vestit-o poporului Sau, ca sa le dea lor mo?tenirea neamurilor.
Psa 111:7  Lucrurile mâinilor Lui adevar ?i judecata. Adevarate sunt toate poruncile Lui,
Psa 111:8  Întarite în veacul veacului, facute în adevar ?i dreptate.
Psa 111:9  Izbavire a trimis poporului Sau; poruncit-a în veac legamântul Sau; sfânt ?i înfrico?ator este numele Lui.
Psa 111:10  Începutul în?elepciunii este frica de Domnul; în?elegere buna este tuturor celor ce o fac pe ea. Lauda Lui ramâne în veacul veacului.
Psa 112:1  (Un psalm scris pe vremea lui Neemia.) Fericit barbatul care se teme de Domnul; întru poruncile Lui va voi foarte.
Psa 112:2  Puternica va fi pe pamânt semin?ia Lui; neamul drep?ilor se va binecuvânta.
Psa 112:3  Slava ?i boga?ie în casa lui ?i dreptatea lui ramâne în veacul veacului.
Psa 112:4  Rasarit-a în întuneric lumina drep?ilor, Cel milostiv, îndurat ?i drept.
Psa 112:5  Bun este barbatul care se îndura ?i împrumuta; î?i rânduie?te vorbele sale cu judecata, ca în veac nu se va clinti.
Psa 112:6  Întru pomenire ve?nica va fi dreptul; de vorbire de rau nu se va teme.
Psa 112:7  Gata este inima lui a nadajdui în Domnul; întarita este inima lui, nu se va teme, pâna ce va ajunge sa dispre?uiasca pe vrajma?ii sai.
Psa 112:8  Risipit-a, dat-a saracilor; dreptatea lui ramâne în veacul veacului.
Psa 112:9  Puterea lui se va înal?a întru slava.
Psa 112:10  Pacatosul va vedea ?i se va mânia, va scrâ?ni din din?i ?i se va topi. Pofta pacato?ilor va pieri.
Psa 113:1  Aliluia! Lauda?i, tineri, pe Domnul, lauda?i numele Domnului.
Psa 113:2  Fie numele Domnului binecuvântat de acum ?i pâna în veac.
Psa 113:3  De la rasaritul soarelui pâna la apus, laudat este numele Domnului.
Psa 113:4  Înalt este peste toate neamurile Domnul, peste ceruri este slava Lui.
Psa 113:5  Cine este ca Domnul Dumnezeul nostru, Cel ce locuie?te întru cele înalte
Psa 113:6  ?i spre cele smerite prive?te, În cer ?i pe pamânt?
Psa 113:7  Cel ce scoate din pulbere pe cel sarac ?i ridica din gunoi pe cel sarman,
Psa 113:8  Ca sa-l a?eze cu cei mari, cu cei mari ai poporului Sau.
Psa 113:9  Cel ce face sa locuiasca cea stearpa în casa, ca o mama ce se bucura de fii.
Psa 114:1  La ie?irea lui Israel din Egipt, a casei lui Iacob dintr-un popor barbar,
Psa 114:2  Ajuns-a Iuda sfin?irea Lui, Israel stapânirea Lui.
Psa 114:3  Marea a vazut ?i a fugit, Iordanul s-a întors înapoi.
Psa 114:4  Mun?ii au saltat ca berbecii ?i dealurile ca mieii oilor.
Psa 114:5  Ce-?i este ?ie, mare, ca ai fugit? ?i ?ie Iordane, ca te-ai întors înapoi?
Psa 114:6  Mun?ilor, ca a?i saltat ca berbecii ?i dealurilor, ca mieii oilor?
Psa 114:7  De fa?a Domnului s-a cutremurat pamântul, de fa?a Dumnezeului lui Iacob,
Psa 114:8  Care a prefacut stânca în iezer, iar piatra în izvoare de apa.
Psa 115:1  Nu noua, Doamne, nu noua, ci numelui Tau se cuvine slava, pentru mila Ta ?i pentru adevarul Tau,
Psa 115:2  Ca nu cumva sa zica neamurile: "Unde este Dumnezeul lor?"
Psa 115:3  Dar Dumnezeul nostru e în cer; în cer ?i pe pamânt toate câte a voit a facut.
Psa 115:4  Idolii neamurilor sunt argint ?i aur, lucruri de mâini omene?ti:
Psa 115:5  Gura au ?i nu vor grai; ochi au ?i nu vor vedea;
Psa 115:6  Urechi au ?i nu vor auzi; nari au ?i nu vor mirosi;
Psa 115:7  Mâini au ?i nu vor pipai; picioare au ?i nu vor umbla, nu vor glasui cu gâtlejul lor.
Psa 115:8  Asemenea lor sa fie cei ce-i fac pe ei ?i to?i cei ce se încred în ei.
Psa 115:9  Casa lui Israel a nadajduit în Domnul; ajutorul lor ?i aparatorul lor este.
Psa 115:10  Casa lui Aaron a nadajduit în Domnul; ajutorul lor ?i aparatorul lor este.
Psa 115:11  Cei ce se tem de Domnul au nadajduit în Domnul; ajutorul lor ?i aparatorul lor este.
Psa 115:12  Domnul ?i-a adus aminte de noi ?i ne-a binecuvântat pe noi; a binecuvântat casa lui Israel, a binecuvântat casa lui Aaron,
Psa 115:13  A binecuvântat pe cei ce se tem de Domnul, pe cei mici împreuna cu cei mari.
Psa 115:14  Sporeasca-va Domnul pe voi, pe voi ?i pe copiii vo?tri!
Psa 115:15  Binecuvânta?i sa fi?i de Domnul, Cel ce a facut cerul ?i pamântul.
Psa 115:16  Cerul cerului este al Domnului, iar pamântul l-a dat fiilor oamenilor.
Psa 115:17  Nu mor?ii Te vor lauda pe Tine, Doamne, nici to?i cei ce se coboara în iad,
Psa 115:18  Ci noi, cei vii, vom binecuvânta pe Domnul de acum ?i pâna în veac.
Psa 116:1  Aliluia! Iubit-am pe Domnul, ca a auzit glasul rugaciunii mele,
Psa 116:2  Ca a plecat urechea Lui spre mine ?i în zilele mele Îl voi chema.
Psa 116:3  Cuprinsu-m-au durerile mor?ii, primejdiile iadului m-au gasit; necaz ?i durere am aflat
Psa 116:4  ?i numele Domnului am chemat: "O, Doamne, izbave?te sufletul meu!"
Psa 116:5  Milostiv este Domnul ?i drept ?i Dumnezeul nostru miluie?te.
Psa 116:6  Cel ce paze?te pe prunci este Domnul; umilit am fost ?i m-am izbavit.
Psa 116:7  Întoarce-te, suflete al meu, la odihna ta, ca Domnul ?i-a facut ?ie bine;
Psa 116:8  Ca a scos sufletul meu din moarte, ochii mei din lacrimi ?i picioarele mele de la cadere.
Psa 116:9  Bine voi placea înaintea Domnului, în pamântul celor vii.
Psa 116:10  Crezut-am, pentru aceea am grait, iar eu m-am smerit foarte.
Psa 116:11  Eu am zis întru uimirea mea: "Tot omul este mincinos!"
Psa 116:12  Ce voi rasplati Domnului pentru toate câte mi-a dat mie?
Psa 116:13  Paharul mântuirii voi lua ?i numele Domnului voi chema.
Psa 116:14  Fagaduin?ele mele le voi plini Domnului, înaintea a tot poporului Sau.
Psa 116:15  Scumpa este înaintea Domnului moartea cuvio?ilor Lui.
Psa 116:16  O, Doamne, eu sunt robul Tau, eu sunt robul Tau ?i fiul roabei Tale; rupt-ai legaturile mele.
Psa 116:17  ?ie-?i voi aduce jertfa de lauda ?i numele Domnului voi chema.
Psa 116:18  Fagaduin?ele mele le voi plini Domnului, înaintea a tot poporului Lui,
Psa 116:19  În cur?ile casei Domnului, în mijlocul tau, Ierusalime.
Psa 117:1  Aliluia! Lauda?i pe Domnul toate neamurile; lauda?i-L pe El toate popoarele;
Psa 117:2  Ca s-a întarit mila Lui peste noi ?i adevarul Domnului ramâne în veac.
Psa 118:1  Aliluia! Lauda?i pe Domnul ca este bun, ca în veac este mila Lui.
Psa 118:2  Sa zica, dar, casa lui Israel, ca este bun, ca în veac este mila Lui.
Psa 118:3  Sa zica, dar, casa lui Aaron, ca este bun, ca în veac este mila Lui.
Psa 118:4  Sa zica, dar, to?i cei ce se tem de Domnul, ca este bun, ca în veac este mila Lui.
Psa 118:5  În necaz am chemat pe Domnul ?i m-a auzit ?i m-a scos întru desfatare.
Psa 118:6  Domnul este ajutorul meu, nu ma voi teme de ce-mi va face mie omul.
Psa 118:7  Domnul este ajutorul meu ?i eu voi privi cu bucurie pe vrajma?ii mei.
Psa 118:8  Mai bine este a Te încrede în Domnul, decât a Te încrede în om.
Psa 118:9  Mai bine este a nadajdui în Domnul, decât a nadajdui în capetenii.
Psa 118:10  Toate neamurile m-au înconjurat ?i în numele Domnului i-am înfrânt pe ei.
Psa 118:11  Înconjurând m-au înconjurat ?i în numele Domnului i-am înfrânt pe ei.
Psa 118:12  Înconjuratu-m-au ca albinele fagurele, dar s-au stins ca focul de spini ?i în numele Domnului i-am înfrânt pe ei.
Psa 118:13  Împingându-ma m-au împins sa cad, dar Domnul m-a sprijinit.
Psa 118:14  Taria mea ?i lauda mea este Domnul ?i mi-a fost mie spre izbavire.
Psa 118:15  Glas de bucurie ?i de izbavire în corturile drep?ilor: "Dreapta Domnului a facut putere,
Psa 118:16  Dreapta Domnului m-a înal?at, dreapta Domnului a facut putere!"
Psa 118:17  Nu voi muri, ci voi fi viu ?i voi povesti lucrurile Domnului.
Psa 118:18  Certând m-a certat Domnul, dar mor?ii nu m-a dat.
Psa 118:19  Deschide?i-mi mie por?ile drepta?ii, intrând în ele voi lauda pe Domnul.
Psa 118:20  Aceasta este poarta Domnului; drep?ii vor intra prin ea.
Psa 118:21  Te voi lauda, ca m-ai auzit ?i ai fost mie spre izbavire.
Psa 118:22  Piatra pe care n-au bagat-o în seama ziditorii, aceasta s-a facut în capul unghiului.
Psa 118:23  De la Domnul s-a facut aceasta ?i minunata este în ochii no?tri.
Psa 118:24  Aceasta este ziua pe care a facut-o Domnul, sa ne bucuram ?i sa ne veselim întru ea.
Psa 118:25  O, Doamne, mântuie?te! O, Doamne, spore?te!
Psa 118:26  Binecuvântat este cel ce vine întru numele Domnului; binecuvântatu-v-am pe voi, din casa Domnului.
Psa 118:27  Dumnezeu este Domnul ?i S-a aratat noua. Tocmi?i sarbatoare cu ramuri umbroase, pâna la coarnele altarului.
Psa 118:28  Dumnezeul meu e?ti Tu ?i Te voi lauda; Dumnezeul meu e?ti Tu ?i Te voi înal?a. Te voi lauda ca m-ai auzit ?i ai fost mie spre mântuire.
Psa 118:29  Lauda?i pe Domnul ca este bun, ca în veac este mila Lui.
Psa 119:1  Aliluia! Ferici?i cei fara prihana în cale, care umbla în legea Domnului.
Psa 119:2  Ferici?i cei ce pazesc poruncile Lui ?i-L cauta cu toata inima lor,
Psa 119:3  Ca n-au umblat în caile Lui cei ce lucreaza faradelegea.
Psa 119:4  Tu ai poruncit ca poruncile Tale sa fie pazite foarte.
Psa 119:5  O, de s-ar îndrepta caile mele, ca sa pazesc poruncile Tale!
Psa 119:6  Atunci nu ma voi ru?ina când voi cauta spre toate poruncile Tale.
Psa 119:7  Lauda-Te-voi întru îndreptarea inimii, ca sa înva? judeca?ile drepta?ii Tale.
Psa 119:8  Îndreptarile Tale voi pazi; nu ma parasi pâna în sfâr?it.
Psa 119:9  Prin ce î?i va îndrepta tânarul calea sa? Prin pazirea cuvintelor Tale.
Psa 119:10  Cu toata inima Te-am cautat pe Tine; sa nu ma lepezi de la poruncile Tale.
Psa 119:11  În inima mea am ascuns cuvintele Tale, ca sa nu gre?esc ?ie.
Psa 119:12  Binecuvântat e?ti, Doamne, înva?a-ma îndreptarile Tale.
Psa 119:13  Cu buzele am rostit toate judeca?ile gurii Tale.
Psa 119:14  În calea marturiilor Tale m-am desfatat ca de toata boga?ia.
Psa 119:15  La poruncile Tale voi cugeta ?i voi cunoa?te caile Tale.
Psa 119:16  La îndreptarile Tale voi cugeta ?i nu voi uita cuvintele Tale.
Psa 119:17  Rasplate?te robului Tau! Voi trai ?i voi pazi poruncile Tale.
Psa 119:18  Deschide ochii mei ?i voi cunoa?te minunile din legea Ta.
Psa 119:19  Strain sunt eu pe pamânt, sa nu ascunzi de la mine poruncile Tale.
Psa 119:20  Aprins e sufletul meu de dorirea judeca?ilor Tale, în toata vremea.
Psa 119:21  Certat-ai pe cai mândri; blestema?i sunt cei ce se abat de la poruncile Tale.
Psa 119:22  Ia de la mine ocara ?i defaimarea, ca marturiile Tale am pazit.
Psa 119:23  Pentru ca au ?ezut capeteniile ?i pe mine ma cleveteau, iar robul Tau cugeta la îndreptarile Tale.
Psa 119:24  Ca marturiile Tale sunt cugetarea mea, iar îndreptarile Tale, sfatul meu.
Psa 119:25  Lipitu-s-a de pamânt sufletul meu; viaza-ma, dupa cuvântul Tau.
Psa 119:26  Vestit-am caile mele ?i m-ai auzit; înva?a-ma îndreptarile Tale.
Psa 119:27  Fa sa în?eleg calea îndreptarilor Tale ?i voi cugeta la minunile Tale.
Psa 119:28  Istovitu-s-a sufletul meu de suparare; întare?te-ma întru cuvintele Tale.
Psa 119:29  Departeaza de la mine calea nedrepta?ii ?i cu legea Ta ma miluie?te.
Psa 119:30  Calea adevarului am ales ?i judeca?ile Tale nu le-am uitat.
Psa 119:31  Lipitu-m-am de marturiile Tale, Doamne, sa nu ma ru?inezi.
Psa 119:32  Pe calea poruncilor Tale am alergat când ai largit inima mea.
Psa 119:33  Lege pune mie, Doamne, calea îndreptarilor Tale ?i o voi pazi pururea.
Psa 119:34  În?elep?e?te-ma ?i voi cauta legea Ta ?i o voi pazi cu toata inima mea.
Psa 119:35  Pova?uie?te-ma pe cararea poruncilor Tale, ca aceasta am voit.
Psa 119:36  Pleaca inima mea la marturiile Tale ?i nu la lacomie.
Psa 119:37  Întoarce ochii mei ca sa nu vada de?ertaciunea; în calea Ta viaza-ma.
Psa 119:38  Împline?te robului Tau cuvântul Tau, care este pentru cei ce se tem de Tine.
Psa 119:39  Îndeparteaza ocara, de care ma tem, caci judeca?ile Tale sunt bune.
Psa 119:40  Iata, am dorit poruncile Tale; întru dreptatea Ta viaza-ma.
Psa 119:41  Sa vina peste mine mila Ta, Doamne, mântuirea Ta, dupa cuvântul Tau,
Psa 119:42  ?i voi raspunde cuvânt celor ce ma ocarasc, ca am nadajduit în cuvintele Tale.
Psa 119:43  Sa nu îndepartezi din gura mea cuvântul adevarului, pâna în sfâr?it, ca întru judeca?ile Tale am nadajduit,
Psa 119:44  ?i voi pazi legea Ta pururea, în veac ?i în veacul veacului.
Psa 119:45  Am umblat întru largime, ca poruncile Tale am cautat.
Psa 119:46  Am vorbit despre marturiile Tale, înaintea împara?ilor, ?i nu m-am ru?inat.
Psa 119:47  Am cugetat la poruncile Tale pe care le-am iubit foarte.
Psa 119:48  Am ridicat mâinile mele la poruncile Tale, pe care le-am iubit ?i am cugetat la îndreptarile Tale.
Psa 119:49  Adu-?i aminte de cuvântul Tau, catre robul Tau, întru care mi-ai dat nadejde.
Psa 119:50  Aceasta m-a mângâiat întru smerenia mea, ca cuvântul Tau m-a viat.
Psa 119:51  Cei mândri m-au batjocorit peste masura, dar de la legea Ta nu m-am abatut.
Psa 119:52  Adusu-mi-am aminte de judeca?ile Tale cele din veac, Doamne, ?i m-am mângâiat.
Psa 119:53  Mâhnire m-a cuprins din pricina pacato?ilor, care parasesc legea Ta.
Psa 119:54  Cântate erau de mine îndreptarile Tale, în locul pribegiei mele.
Psa 119:55  Adusu-mi-am aminte de numele Tau, Doamne, ?i am pazit legea Ta.
Psa 119:56  Aceasta s-a facut mie, ca îndreptarile Tale am cautat.
Psa 119:57  Partea mea e?ti, Doamne, zis-am sa pazesc legea Ta.
Psa 119:58  Rugatu-m-am fe?ei Tale, din toata inima mea, miluie?te-ma dupa cuvântul Tau.
Psa 119:59  Cugetat-am la caile Tale ?i am întors picioarele mele la marturiile Tale.
Psa 119:60  Gata am fost ?i nu m-am tulburat sa pazesc poruncile Tale.
Psa 119:61  Funiile pacato?ilor s-au înfa?urat împrejurul meu, dar legea Ta n-am uitat.
Psa 119:62  La miezul nop?ii m-am sculat ca sa Te laud pe Tine, pentru judeca?ile drepta?ii Tale.
Psa 119:63  Parta? sunt cu to?i cei ce se tem de Tine ?i pazesc poruncile Tale.
Psa 119:64  De mila Ta, Doamne, este plin pamântul; îndreptarile Tale ma înva?a.
Psa 119:65  Bunatate ai facut cu robul Tau, Doamne, dupa cuvântul Tau.
Psa 119:66  Înva?a-ma bunatatea, înva?atura ?i cuno?tin?a, ca în poruncile Tale am crezut.
Psa 119:67  Mai înainte de a fi umilit, am gre?it; pentru aceasta cuvântul Tau am pazit.
Psa 119:68  Bun e?ti Tu, Doamne, ?i întru bunatatea Ta, înva?a-ma îndreptarile Tale.
Psa 119:69  Înmul?itu-s-a asupra mea nedreptatea celor mândri, iar eu cu toata inima mea voi cerceta poruncile Tale.
Psa 119:70  Închegatu-s-a ca grasimea inima lor, iar eu cu legea Ta m-am desfatat.
Psa 119:71  Bine este mie ca m-ai smerit, ca sa înva? îndreptarile Tale.
Psa 119:72  Buna-mi este mie legea gurii Tale, mai mult decât mii de comori de aur ?i argint.
Psa 119:73  Mâinile Tale m-au facut ?i m-au zidit, în?elep?e?te-ma ?i voi înva?a poruncile Tale.
Psa 119:74  Cei ce se tem de Tine ma vor vedea ?i se vor veseli, ca în cuvintele Tale am nadajduit.
Psa 119:75  Cunoscut-am, Doamne, ca drepte sunt judeca?ile Tale ?i întru adevar m-ai smerit.
Psa 119:76  Faca-se dar, mila Ta, ca sa ma mângâie, dupa cuvântul Tau, catre robul Tau.
Psa 119:77  Sa vina peste mine îndurarile Tale ?i voi trai, ca legea Ta cugetarea mea este.
Psa 119:78  Sa se ru?ineze cei mândri, ca pe nedrept m-au nedrepta?it; iar eu voi cugeta la poruncile Tale.
Psa 119:79  Sa se întoarca spre mine cei ce se tem de Tine ?i cei ce cunosc marturiile Tale.
Psa 119:80  Sa fie inima mea fara prihana întru îndreptarile Tale, ca sa nu ma ru?inez.
Psa 119:81  Se tope?te sufletul meu dupa mântuirea Ta; în cuvântul Tau am nadajduit.
Psa 119:82  Sfâr?itu-s-au ochii mei dupa cuvântul Tau, zicând: "Când ma vei mângâia?"
Psa 119:83  Ca m-am facut ca un foale la fum, dar îndreptarile Tale nu le-am uitat.
Psa 119:84  Câte sunt zilele robului Tau? Când vei judeca pe cei ce ma prigonesc?
Psa 119:85  Spusu-mi-au calcatorii de lege de?ertaciuni, dar nu sunt ca legea Ta, Doamne.
Psa 119:86  Toate poruncile Tale sunt adevar; pe nedrept m-au prigonit. Ajuta-ma!
Psa 119:87  Pu?in a fost de nu m-am sfâr?it pe pamânt, dar eu n-am parasit poruncile Tale.
Psa 119:88  Dupa mila Ta viaza-ma ?i voi pazi marturiile gurii mele.
Psa 119:89  În veac, Doamne, cuvântul Tau ramâne în cer;
Psa 119:90  În neam ?i în neam adevarul Tau. Întemeiat-ai pamântul ?i ramâne.
Psa 119:91  Dupa rânduiala Ta ramâne ziua, ca toate sunt slujitoare ?ie.
Psa 119:92  De n-ar fi fost legea Ta gândirea mea, atunci a? fi pierit întru necazul meu.
Psa 119:93  În veac nu voi uita îndreptarile Tale, ca într-însele m-ai viat, Doamne.
Psa 119:94  Al Tau sunt eu, mântuie?te-ma, ca îndreptarile Tale am cautat.
Psa 119:95  Pe mine m-au a?teptat pacato?ii ca sa ma piarda. Marturiile Tale am priceput.
Psa 119:96  La tot lucrul desavâr?it am vazut sfâr?it, dar porunca Ta este fara de sfâr?it.
Psa 119:97  Ca am iubit legea Ta, Doamne, ea toata ziua cugetarea mea este.
Psa 119:98  Mai mult decât pe vrajma?ii mei mai în?elep?it cu porunca Ta, ca în veac a mea este.
Psa 119:99  Mai mult decât înva?atorii mei am priceput, ca la marturiile Tale gândirea mea este.
Psa 119:100  Mai mult decât batrânii am în?eles, ca poruncile Tale am cautat.
Psa 119:101  De la toata calea cea rea mi-am oprit picioarele mele, ca sa pazesc cuvintele Tale.
Psa 119:102  De la judeca?ile Tale nu m-am abatut, ca Tu ai pus mie lege.
Psa 119:103  Cât sunt de dulci limbii mele, cuvintele Tale, mai mult decât mierea, în gura mea!
Psa 119:104  Din poruncile Tale m-am facut priceput; pentru aceasta am urât toata calea nedrepta?ii.
Psa 119:105  Faclie picioarelor mele este legea Ta ?i lumina cararilor mele.
Psa 119:106  Juratu-m-am ?i m-am hotarât sa pazesc judeca?ile drepta?ii Tale.
Psa 119:107  Umilit am fost pâna în sfâr?it: Doamne, viaza-ma, dupa cuvântul Tau.
Psa 119:108  Cele de bunavoie ale gurii mele binevoie?te-le Doamne, ?i judeca?ile Tale ma înva?a.
Psa 119:109  Sufletul meu în mâinile Tale este pururea ?i legea Ta n-am uitat.
Psa 119:110  Pusu-mi-au pacato?ii cursa mie, dar de la poruncile Tale n-am ratacit.
Psa 119:111  Mo?tenit-am marturiile Tale în veac, ca bucurie inimii mele sunt ele.
Psa 119:112  Plecat-am inima mea ca sa fac îndreptarile Tale în veac spre rasplatire.
Psa 119:113  Pe calcatorii de lege am urât ?i legea Ta am iubit.
Psa 119:114  Ajutorul meu ?i sprijinitorul meu e?ti Tu, în cuvântul Tau am nadajduit.
Psa 119:115  Departa?i-va de la mine cei ce vicleni?i ?i voi cerceta poruncile Dumnezeului meu.
Psa 119:116  Apara-ma, dupa cuvântul Tau, ?i ma viaza ?i sa nu-mi dai de ru?ine a?teptarea mea.
Psa 119:117  Ajuta-ma ?i ma voi mântui ?i voi cugeta la îndreptarile Tale, pururea.
Psa 119:118  Defaimat-ai pe to?i cei ce se departeaza de la îndreptarile Tale, pentru ca nedrept este gândul lor.
Psa 119:119  Socotit-am calcatori de lege pe to?i pacato?ii pamântului; pentru aceasta am iubit marturiile Tale, pururea.
Psa 119:120  Strapunge cu frica Ta trupul meu, ca de judeca?ile Tale m-am temut.
Psa 119:121  Facut-am judecata ?i dreptate; nu ma da pe mâna celor ce-mi fac strâmbatate.
Psa 119:122  Prime?te pe robul Tau în bunatate, ca sa nu ma cleveteasca cei mândri.
Psa 119:123  Sfâr?itu-sau ochii mei dupa mântuirea Ta ?i dupa cuvântul drepta?ii Tale.
Psa 119:124  Fa cu robul Tau, dupa mila Ta, ?i îndreptarile Tale ma înva?a.
Psa 119:125  Robul Tau sunt eu; în?elep?e?te-ma ?i voi cunoa?te marturiile Tale.
Psa 119:126  Vremea este sa lucreze Domnul, ca oamenii au stricat legea Ta.
Psa 119:127  Pentru aceasta am iubit poruncile Tale, mai mult decât aurul ?i topazul.
Psa 119:128  Pentru aceasta spre toate poruncile Tale m-am îndreptat, toata calea nedreapta am urât.
Psa 119:129  Minunate sunt marturiile Tale, pentru aceasta le-a cercetat pe ele sufletul meu.
Psa 119:130  Aratarea cuvintelor Tale lumineaza ?i în?elep?e?te pe prunci.
Psa 119:131  Gura mea am deschis ?i am aflat, ca de poruncile Tale am dorit.
Psa 119:132  Cauta spre mine ?i ma miluie?te, dupa judecata Ta, fala de cei ce iubesc numele Tau.
Psa 119:133  Pa?ii mei îndrepteaza-i dupa cuvântul Tau, ?i sa nu ma stapâneasca nici o faradelege.
Psa 119:134  Izbave?te-ma de clevetirea oamenilor ?i voi pazi poruncile Tale.
Psa 119:135  Fa?a Ta arat-o robului Tau ?i ma înva?a poruncile Tale.
Psa 119:136  Izvoare de apa s-au coborât din ochii mei, pentru ca n-am pazit legea Ta.
Psa 119:137  Drept e?ti, Doamne, ?i drepte sunt judeca?ile Tale.
Psa 119:138  Poruncit-ai cu dreptate marturiile Tale ?i cu tot adevarul.
Psa 119:139  Topitu-m-a râvna casei Tale, ca au uitat cuvintele Tale vrajma?ii mei.
Psa 119:140  Lamurit cu foc este cuvântul Tau foarte ?i robul Tau l-a iubit pe el.
Psa 119:141  Tânar sunt eu ?i defaimat, dar îndreptarile Tale nu le-am uitat.
Psa 119:142  Dreptatea Ta este dreptate în veac ?i legea Ta adevarul.
Psa 119:143  Necazuri ?i nevoi au dat peste mine, dar poruncile Tale sunt gândirea mea.
Psa 119:144  Drepte sunt marturiile Tale, în veac; în?elep?e?te-ma ?i voi fi viu.
Psa 119:145  Strigat-am cu toata inima mea: Auzi-ma, Doamne! Îndreptarile Tale voi cauta.
Psa 119:146  Strigat-am catre Tine, mântuie?te-ma, ?i voi pazi marturiile Tale.
Psa 119:147  Din zori m-am sculat ?i am strigat; întru cuvintele Tale am nadajduit.
Psa 119:148  Deschis-am ochii mei dis-de-diminea?a, ca sa cuget la cuvintele Tale.
Psa 119:149  Glasul meu auzi-l, Doamne, dupa mila Ta; dupa judecata Ta ma viaza.
Psa 119:150  Apropiatu-s-au cei ce ma prigonesc cu faradelege, dar de la legea Ta s-au îndepartat.
Psa 119:151  Aproape e?ti Tu, Doamne, ?i toate poruncile Tale sunt adevarul.
Psa 119:152  Din început am cunoscut, din marturiile Tale, ca în veac le-ai întemeiat pe ele.
Psa 119:153  Vezi smerenia mea ?i ma scoate, ca legea Ta n-am uitat.
Psa 119:154  Judeca pricina mea ?i ma izbave?te; dupa cuvântul Tau, fa-ma viu.
Psa 119:155  Departe de pacato?i este mântuirea, ca îndreptarile Tale n-au cautat.
Psa 119:156  Îndurarile Tale multe sunt Doamne; dupa judecata Ta ma viaza.
Psa 119:157  Mul?i sunt cei ce ma prigonesc ?i ma necajesc, dar de la marturiile Tale nu m-am abatut.
Psa 119:158  Vazut-am pe cei nepricepu?i ?i ma sfâr?eam, ca n-au pazit cuvintele Tale.
Psa 119:159  Vezi ca poruncile Tale am iubit, Doamne; întru mila Ta ma viaza.
Psa 119:160  Începutul cuvintelor Tale este adevarul ?i ve?nice toate judeca?ile drepta?ii Tale.
Psa 119:161  Capeteniile m-au prigonit în zadar; iar de cuvintele Tale s-a înfrico?at inima mea.
Psa 119:162  Bucura-ma-voi de cuvintele Tale, ca cel ce a aflat comoara mare.
Psa 119:163  Nedreptatea am urât ?i am dispre?uit, iar legea Ta am iubit.
Psa 119:164  De ?apte ori pe zi Te-am laudat pentru judeca?ile drepta?ii Tale.
Psa 119:165  Pace multa au cei ce iubesc legea Ta ?i nu se smintesc.
Psa 119:166  A?teptat-am mântuirea Ta, Doamne, ?i poruncile Tale am iubit.
Psa 119:167  Pazit-a sufletul meu marturiile Tale ?i le-a iubit foarte.
Psa 119:168  Pazit-am poruncile Tale ?i marturiile Tale, ca toate caile mele înaintea Ta sunt, Doamne.
Psa 119:169  Sa se apropie rugaciunea mea înaintea Ta, Doamne; dupa cuvântul Tau ma în?elep?e?te.
Psa 119:170  Sa ajunga cererea mea înaintea Ta, Doamne; dupa cuvântul Tau ma izbave?te.
Psa 119:171  Sa raspândeasca buzele mele lauda, ca m-ai înva?at îndreptarile Tale.
Psa 119:172  Rosti-va limba mea cuvintele Tale, ca toate poruncile Tale sunt drepte.
Psa 119:173  Mina Ta sa ma izbaveasca, ca poruncile Tale am ales.
Psa 119:174  Dorit-am mântuirea Ta, Doamne, ?i legea Ta cugetarea mea este.
Psa 119:175  Viu va fi sufletul meu ?i Te va lauda ?i judeca?ile Tale îmi vor ajuta mie.
Psa 119:176  Ratacit-am ca o oaie pierduta; cauta pe robul Tau, ca poruncile Tale nu le-am uitat.
Psa 120:1  (O cântare a treptelor.) Catre Domnul am strigat când m-am necajit ?i m-a auzit.
Psa 120:2  Doamne, izbave?te sufletul meu de buzele nedrepte ?i de limba vicleana.
Psa 120:3  Ce se va da ?ie ?i ce vei câ?tiga de la limba vicleana?
Psa 120:4  Sage?i ascu?ite cu carbuni aprin?i trase de Cel puternic.
Psa 120:5  Vai mie, ca pribegia mea s-a prelungit, ca locuiesc în corturile lui Chedar!
Psa 120:6  Mult a pribegit sufletul meu.
Psa 120:7  Cu cei ce urau pacea, facator de pace eram; când graiam lor, se luptau cu mine în zadar.
Psa 121:1  (O cântare a treptelor.) Ridicat-am ochii mei la mun?i, de unde va veni ajutorul meu.
Psa 121:2  Ajutorul meu de la Domnul, Cel ce a facut cerul ?i pamântul.
Psa 121:3  Nu va lasa sa se clatine piciorul tau, nici nu va dormita Cel ce paze?te.
Psa 121:4  Iata, nu va dormita, nici nu va adormi Cel ce paze?te pe Israel.
Psa 121:5  Domnul te va pazi pe tine, Domnul este acoperamântul tau, de-a dreapta ta.
Psa 121:6  Ziua soarele nu te va arde, nici luna noaptea.
Psa 121:7  Domnul te va pazi pe tine de tot raul; pazi-va sufletul tau.
Psa 121:8  Domnul va pazi intrarea ta ie?irea ta de acum ?i pâna în veac.
Psa 122:1  (O cântare a treptelor.) Veselitu-m-am de cei ce mi-au zis mie: "În casa Domnului vom merge!"
Psa 122:2  Stateau picioarele noastre în cur?ile tale, Ierusalime!
Psa 122:3  Ierusalimul, cel ce este zidit ca o cetate, ale carei por?i sunt strâns-unite.
Psa 122:4  Ca acolo s-au suit semin?iile, semin?iile Domnului, dupa legea lui Israel, ca sa laude numele Domnului.
Psa 122:5  Ca acolo s-au a?ezat scaunele la judecata, scaunele pentru casa lui David.
Psa 122:6  Ruga?i-va pentru pacea Ierusalimului ?i pentru îndestularea celor ce te iubesc pe tine.
Psa 122:7  Sa fie pace în întariturile tale ?i îndestulare în turnurile tale.
Psa 122:8  Pentru fra?ii mei ?i pentru vecinii mei graiam despre tine pace.
Psa 122:9  Pentru casa Domnului Dumnezeului nostru am dorit cele bune ?ie.
Psa 123:1  (O cântare a treptelor.) Catre Tine, Cel ce locuie?ti în cer, am ridicat ochii mei.
Psa 123:2  Iata, precum sunt ochii robilor la mâinile stapânilor lor, precum sunt ochii slujnicei la mâinile stapânei sale, a?a sunt ochii no?tri catre Domnul Dumnezeul nostru, pâna ce Se va milostivi spre noi.
Psa 123:3  Miluie?te-ne pe noi, Doamne, miluie?te-ne pe noi, ca mult ne-am saturat de defaimare,
Psa 123:4  Ca prea mult s-a saturat sufletul nostru de ocara celor îndestula?i ?i de defaimarea celor mândri.
Psa 124:1  (Un psalm al lui David. O cântare a treptelor.) De n-ar fi fost Domnul cu noi, sa spuna Israel!
Psa 124:2  De n-ar fi fost Domnul cu noi, când s-au ridicat oamenii împotriva noastra,
Psa 124:3  De vii ne-ar fi înghi?it pe noi, când s-a aprins mânia lor împotriva noastra.
Psa 124:4  Apa ne-ar fi înecat pe noi, ?uvoi ar fi trecut peste sufletul nostru.
Psa 124:5  Atunci ar fi trecut peste sufletul nostru valuri înspaimântatoare.
Psa 124:6  Binecuvântat este Domnul, Care nu ne-a dat pe noi spre vânare din?ilor lor.
Psa 124:7  Sufletul nostru a scapat ca o pasare din cursa vânatorilor; cursa s-a sfarâmat ?i noi ne-am izbavit.
Psa 124:8  Ajutorul nostru este în numele Domnului, Cel ce a facut cerul ?i pamântul.
Psa 125:1  (O cântare a treptelor.) Cei ce se încred în Domnul sunt ca muntele Sionului; nu se va clatina în veac cel ce locuie?te în Ierusalim.
Psa 125:2  Mun?i sunt împrejurul lui ?i Domnul împrejurul poporului Sau, de acum ?i pâna în veac.
Psa 125:3  Ca nu va lasa Domnul toiagul pacato?ilor peste soarta drep?ilor, ca sa nu-?i întinda drep?ii întru faradelegi mâinile lor.
Psa 125:4  Fa bine, Doamne, celor buni ?i celor drep?i cu inima;
Psa 125:5  Iar pe cei ce se abat pe cai nedrepte, Domnul îi va duce cu cei ce lucreaza faradelegea. Pace peste Israel!
Psa 126:1  (O cântare a treptelor.) Când a întors Domnul robia Sionului ne-am umplut de mângâiere.
Psa 126:2  Atunci s-a umplut de bucurie gura noastra ?i limba noastra de veselie; atunci se zicea între neamuri: "Mari lucruri a facut Domnul cu ei!"
Psa 126:3  Mari lucruri a facut Domnul cu noi: ne-a umplut de bucurie.
Psa 126:4  Întoarce, Doamne, robia noastra, cum întorci pâraiele spre miazazi.
Psa 126:5  Cei ce seamana cu lacrimi, cu bucurie vor secera.
Psa 126:6  Mergând mergeau ?i plângeau, aruncând semin?ele lor, dar venind vor veni cu bucurie, ridicând snopii lor.
Psa 127:1  (Un psalm al lui Solomon. O cântare a treptelor.) De n-ar zidi Domnul casa, în zadar s-ar osteni cei ce o zidesc; de n-ar pazi Domnul cetatea, în zadar ar priveghea cel ce o paze?te.
Psa 127:2  în zadar va scula?i dis-de-diminea?a, în zadar va culca?i târziu, voi care mânca?i pâinea durerii, daca nu v-ar da Domnul somn, iubi?i ai Sai.
Psa 127:3  Iata, fiii sunt mo?tenirea Domnului, rasplata rodului pântecelui.
Psa 127:4  Precum sunt sage?ile în mâna celui viteaz, a?a sunt copiii parin?ilor tineri.
Psa 127:5  Fericit este omul care-?i va umple casa de copii; nu se va ru?ina când va grai cu vrajma?ii sai în poarta.
Psa 128:1  (O cântare a treptelor.) Ferici?i to?i cei ce se tem de Domnul, care umbla în caile Lui.
Psa 128:2  Rodul muncii mâinilor tale vei mânca. Fericit e?ti; bine-?i va fi!
Psa 128:3  Femeia ta ca o vie roditoare, în laturile casei tale; fiii tai ca ni?te vlastare tinere de maslin, împrejurul mesei tale.
Psa 128:4  Iata, a?a se va binecuvânta omul, cel ce se teme de Domnul.
Psa 128:5  "Te va binecuvânta Domnul din Sion ?i vei vedea bunata?ile Ierusalimului în toate zilele vie?ii tale.
Psa 128:6  ?i vei vedea pe fiii fiilor tai. Pace peste Israel!
Psa 129:1  (O cântare a treptelor.) De multe ori s-au luptat cu mine din tinere?ile mele, sa spuna Israel!
Psa 129:2  De multe ori s-au luptat cu mine, din tinere?ile mele ?i nu m-au biruit.
Psa 129:3  Spatele mi-au lovit pacato?ii, întins-au nelegiuirea lor;
Psa 129:4  Dar Domnul Cel drept a taiat grumajii pacato?ilor.
Psa 129:5  Sa se ru?ineze ?i sa se întoarca înapoi to?i cei ce urasc Sionul.
Psa 129:6  Faca-se ca iarba de pe acoperi?uri, care, mai înainte de a fi smulsa, s-a uscat,
Psa 129:7  Din care nu ?i-a umplut mâna lui seceratorul ?i sânul sau, cel ce aduna snopii,
Psa 129:8  ?i trecatorii nu vor zice: "Binecuvântarea Domnului fie peste voi!" sau: "Va binecuvântam în numele Domnului!"
Psa 130:1  (O cântare a treptelor.) Dintru adâncuri am strigat catre Tine; Doamne! Doamne, auzi glasul meu!
Psa 130:2  Fie urechile Tale cu luare-aminte la glasul rugaciunii mele.
Psa 130:3  De Te vei uita la faradelegi, Doamne, Doamne, cine va suferi?
Psa 130:4  Ca la Tine este milostivirea.
Psa 130:5  Pentru numele Tau, Te-am a?teptat, Doamne; a?teptat-a sufletul meu spre cuvântul Tau,
Psa 130:6  Nadajduit-a sufletul meu în Domnul, din straja dimine?ii pâna în noapte. Din straja dimine?ii sa nadajduiasca Israel spre Domnul.
Psa 130:7  Ca la Domnul este mila ?i multa mântuire la El
Psa 130:8  ?i El va izbavi pe Israel din toate faradelegile lui.
Psa 131:1  (Un psalm al lui David. O cântare a treptelor.) Doamne, nu s-a mândrit inima mea, nici nu s-au înal?at ochii mei, nici n-am umblat dupa lucruri mari, nici dupa lucruri mai presus de mine,
Psa 131:2  Dimpotriva, mi-am smerit ?i mi-am domolit sufletul meu, ca un prunc în?arcat de mama lui, ca rasplata a sufletului meu.
Psa 131:3  Sa nadajduiasca Israel în Domnul, de acum ?i pâna în veac!
Psa 132:1  (O cântare a treptelor.) Adu-?i aminte, Doamne, de David ?i de toate blânde?ile lui.
Psa 132:2  Cum s-a jurat Domnului ?i a fagaduit Dumnezeului lui Iacob:
Psa 132:3  Nu voi intra în loca?ul casei mele, nu ma voi sui pe patul meu de odihna,
Psa 132:4  Nu voi da somn ochilor mei ?i genelor mele dormitare ?i odihna tâmplelor mele,
Psa 132:5  Pâna ce nu voi afla loc Domnului, loca? Dumnezeului lui Iacob.
Psa 132:6  Iata am auzit de chivotul legii; în Efrata l-am gasit, în ?arina lui Iaar.
Psa 132:7  Intra-vom în loca?urile Lui, închina-ne-vom la locul unde au stat picioarele Lui.
Psa 132:8  Scoala-Te, Doamne, întru odihna Ta, Tu ?i chivotul sfin?irii Tale.
Psa 132:9  Preo?ii Tai se vor îmbraca cu dreptate ?i cuvio?ii Tai se vor bucura.
Psa 132:10  Din pricina lui David, robul Tau, sa nu întorci fa?a unsului Tau.
Psa 132:11  Juratu-S-a Domnul lui David adevarul ?i nu-l va lepada: "Din rodul pântecelui tau voi pune pe scaunul tau,
Psa 132:12  De vor pazi fiii tai legamântul Meu ?i marturiile acestea ale Mele, în care îi voi înva?a pe ei ?i fiii lor vor ?edea pâna în veac pe scaunul tau".
Psa 132:13  Ca a ales Domnul Sionul ?i l-a ales ca locuin?a Lui.
Psa 132:14  "Aceasta este odihna Mea în veacul veacului. Aici voi locui, ca l-am ales pe el.
Psa 132:15  Roadele lui le voi binecuvânta foarte; pe saracii lui îi voi satura cu pâine.
Psa 132:16  Pe preo?ii lui îi voi îmbraca cu izbavire ?i cuvio?ii lui cu bucurie se vor bucura.
Psa 132:17  Acolo voi face sa rasara puterea lui David, gatit-am faclie unsului Meu.
Psa 132:18  Pe vrajma?ii lui îi voi îmbraca cu ru?ine, iar pe dânsul va înflori sfin?enia Mea".
Psa 133:1  (Un psalm al lui David. O cântare a treptelor.) Iata acum ce este bun ?i ce este frumos, decât numai a locui fra?ii împreuna!
Psa 133:2  Aceasta este ca mirul pe cap, care se coboara pe barba, pe barba lui Aaron, care se coboara pe marginea ve?mintelor lui.
Psa 133:3  Aceasta este ca roua Ermonului, ce se coboara pe mun?ii Sionului, ca unde este unire acolo a poruncit Domnul binecuvântarea ?i via?a pâna în veac.
Psa 134:1  (O cântare a treptelor.) Iata acum binecuvânta?i pe Domnul toate slugile Domnului, care sta?i în casa Domnului, în cur?ile casei Dumnezeului nostru.
Psa 134:2  Noaptea ridica?i mâinile voastre spre cele sfinte ?i binecuvânta?i pe Domnul.
Psa 134:3  Te va binecuvânta Domnul din Sion, Cel ce a facut cerul ?i pamântul.
Psa 135:1  Aliluia! Lauda?i numele Domnului, lauda?i slugi pe Domnul,
Psa 135:2  Cei ce sta?i în casa Domnului, în cur?ile Dumnezeului nostru.
Psa 135:3  Lauda?i pe Domnul, ca este bun Domnul; cânta?i numele Lui, ca este bun.
Psa 135:4  Ca pe Iacob ?i l-a ales Domnul, pe Israel spre mo?tenire Lui.
Psa 135:5  Ca eu am cunoscut ca este mare Domnul ?i Domnul nostru peste to?i dumnezeii.
Psa 135:6  Toate câte a vrut Domnul a facut în cer ?i pe pamânt, în mari ?i în toate adâncurile.
Psa 135:7  A ridicat nori de la marginea pamântului; fulgerele spre ploaie le-a facut; El scoate vânturile din vistieriile Sale.
Psa 135:8  El a batut pe cei întâi-nascu?i ai Egiptului, de la om pâna la dobitoc.
Psa 135:9  Trimis-a semne ?i minuni în mijlocul tau, Egipte, lui Faraon ?i tuturor robilor lui.
Psa 135:10  El a batut neamuri multe ?i a ucis împara?i puternici:
Psa 135:11  Pe Sihon împaratul Amoreilor ?i pe Og împaratul Vasanului ?i toate stapânirile Canaanului.
Psa 135:12  ?i a dat pamântul lor mo?tenire, mo?tenire lui Israel, poporului Sau.
Psa 135:13  Doamne, numele Tau este în veac ?i, pomenirea Ta în neam ?i în neam.
Psa 135:14  Ca va judeca Domnul pe poporul Sau ?i de slugile Sale se va milostivi.
Psa 135:15  Idolii neamurilor sunt argint ?i aur, lucruri facute de mâini omene?ti.
Psa 135:16  Gura au ?i nu vor grai, ochi au ?i nu vor vedea.
Psa 135:17  Urechi au ?i nu vor auzi, ca nu este duh în gura lor.
Psa 135:18  Asemenea lor sa fie to?i cei care îi fac pe ei ?i to?i cei ce se încred în ei.
Psa 135:19  Casa lui Israel, binecuvânta?i pe Domnul; casa lui Aaron, binecuvânta?i pe Domnul;
Psa 135:20  Casa lui Levi, binecuvânta?i pe Domnul; cei ce va teme?i de Domnul, binecuvânta?i pe Domnul.
Psa 135:21  Binecuvântat este Domnul din Sion, Cel ce locuie?te în Ierusalim.
Psa 136:1  Aliluia! Lauda?i pe Domnul ca este bun, ca în veac este mila Lui.
Psa 136:2  Lauda?i pe Dumnezeul dumnezeilor, ca în veac este mila Lui.
Psa 136:3  Lauda?i pe Domnul domnilor, ca în veac este mila Lui.
Psa 136:4  Singurul Care face minuni mari, ca în veac este mila Lui.
Psa 136:5  Cel ce a facut cerul cu pricepere, ca în veac este mila Lui.
Psa 136:6  Cel ce a întarit pamântul pe ape, ca în veac este mila Lui.
Psa 136:7  Cel ce a facut luminatorii cei mari, ca în veac este mila Lui.
Psa 136:8  Soarele, spre stapânirea zilei, ca în veac este mila Lui.
Psa 136:9  Luna ?i stelele spre stapânirea nop?ii, ca în veac este mila Lui.
Psa 136:10  Cel ce a batut Egiptul cu cei întâi-nascu?i ai lor, ca în veac este mila Lui.
Psa 136:11  ?i a scos pe Israel din mijlocul lor, ca în veac este mila Lui.
Psa 136:12  Cu mâna tare ?i cu bra? înalt, ca în veac este mila Lui.
Psa 136:13  Cel ce a despar?it Marea Ro?ie în doua; ca în veac este mila Lui.
Psa 136:14  ?i a trecut pe Israel prin mijlocul ei, ca în veac este mila Lui.
Psa 136:15  ?i a rasturnat pe Faraon ?i o?tirea lui în Marea Ro?ie, ca în veac este mila Lui.
Psa 136:16  Cel ce a trecut pe poporul Lui prin pustiu, ca în veac este mila Lui.
Psa 136:17  Cel ce a batut împara?i mari, ca în veac este mila Lui.
Psa 136:18  Cel ce a omorât împara?i tari, ca în veac este mila Lui.
Psa 136:19  Pe Sihon împaratul Amoreilor, ca în veac este mila Lui.
Psa 136:20  ?i pe Og împaratul Vasanului, ca în veac este mila Lui.
Psa 136:21  ?i le-a dat pamântul lor mo?tenire, ca în veac este mila Lui.
Psa 136:22  Mo?tenire lui Israel, robul Lui, ca în veac este mila Lui.
Psa 136:23  Ca în smerenia noastra ?i-a adus aminte de noi Domnul, ca în veac este mila Lui.
Psa 136:24  ?i ne-a izbavit pe noi de vrajma?ii no?tri, ca în veac este mila Lui.
Psa 136:25  Cel ce da hrana la tot trupul, ca în veac este mila Lui.
Psa 136:26  Lauda?i pe Dumnezeul cerului, ca în veac este mila Lui.
Psa 137:1  (Un psalm al lui David.) La râul Babilonului, acolo am ?ezut ?i am plâns, când ne-am adus aminte de Sion.
Psa 137:2  În salcii, în mijlocul lor, am atârnat harpele noastre.
Psa 137:3  Ca acolo cei ce ne-au robit pe noi ne-au cerut noua cântare, zicând: "Cânta?i-ne noua din cântarile Sionului!"
Psa 137:4  Cum sa cântam cântarea Domnului în pamânt strain?
Psa 137:5  De te voi uita, Ierusalime, uitata sa fie dreapta mea!
Psa 137:6  Sa se lipeasca limba mea de grumazul meu, de nu-mi voi aduce aminte de tine, de nu voi pune înainte Ierusalimul, ca început al bucuriei mele.
Psa 137:7  Adu-?i aminte, Doamne, de fiii lui Edom, în ziua darâmarii Ierusalimului, când ziceau: "Strica?i-l, strica?i-l pâna la temeliile lui!"
Psa 137:8  Fiica Babilonului, ticaloasa! Fericit este cel ce-?i va rasplati ?ie fapta ta pe care ai facut-o noua.
Psa 137:9  Fericit este cel ce va apuca ?i va lovi pruncii tai de piatra.
Psa 138:1  (Un psalm al lui David.) Lauda-Te-voi, Doamne, cu toata inima mea, ca ai auzit cuvintele gurii mele ?i înaintea îngerilor Î?i voi cânta.
Psa 138:2  Închina-ma-voi în loca?ul Tau cel sfânt ?i voi lauda numele Tau, pentru mila Ta ?i adevarul Tau; ca ai marit peste tot numele cel sfânt al Tau.
Psa 138:3  În orice zi Te voi chema, degraba ma auzi! Spore?te în sufletul meu puterea.
Psa 138:4  Sa Te laude pe Tine, Doamne, to?i împara?ii pamântului, când vor auzi toate graiurile gurii Tale,
Psa 138:5  ?i sa cânte în caile Domnului, ca mare este slava Domnului.
Psa 138:6  Ca înalt este Domnul ?i spre cele smerite prive?te ?i pe cele înalte de departe le cunoa?te.
Psa 138:7  De ajung la necaz, viaza-ma! Împotriva vrajma?ilor mei ai întins mâna Ta ?i m-a izbavit dreapta Ta.
Psa 138:8  Domnul le va plati lor pentru mine. Doamne, mila Ta este în veac; lucrurile mâinilor Tale nu le trece cu vederea.
Psa 139:1  (Un psalm al lui David; mai-marelui cântare?ilor.) Doamne, cercetatu-m-ai ?i m-ai cunoscut.
Psa 139:2  Tu ai cunoscut ?ederea mea ?i scularea mea; Tu ai priceput gândurile mele de departe.
Psa 139:3  Cararea mea ?i firul vie?ii mele Tu le-ai cercetat ?i toate caile mele mai dinainte le-ai vazut.
Psa 139:4  Ca înca nu este cuvânt pe limba mea
Psa 139:5  ?i iata, Doamne, Tu le-ai cunoscut pe toate ?i pe cele din urma ?i pe cele de demult; Tu m-ai zidit ?i ai pus peste mine mâna Ta.
Psa 139:6  Minunata este ?tiin?a Ta, mai presus de mine; este înalta ?i n-o pot ajunge.
Psa 139:7  Unde ma voi duce de la Duhul Tau ?i de la fa?a Ta unde voi fugi?
Psa 139:8  De ma voi sui în cer, Tu acolo e?ti. De ma voi coborî în iad, de fa?a e?ti.
Psa 139:9  De voi lua aripile mele de diminea?a ?i de ma voi a?eza la marginile marii
Psa 139:10  ?i acolo mâna Ta ma va pova?ui ?i ma va ?ine dreapta Ta.
Psa 139:11  ?i am zis: "Poate întunericul ma va acoperi ?i se va face noapte lumina dimprejurul meu".
Psa 139:12  Dar întunericul nu este întuneric la Tine ?i noaptea ca ziua va lumina. Cum este întunericul ei, a?a este ?i lumina ei.
Psa 139:13  Ca Tu ai zidit rarunchii mei, Doamne, Tu m-ai alcatuit în pântecele maicii mele.
Psa 139:14  Te voi lauda, ca sunt o faptura a?a de minunata. Minunate sunt lucrurile Tale ?i sufletul meu le cunoa?te foarte.
Psa 139:15  Nu sunt ascunse de Tine oasele mele, pe care le-ai facut întru ascuns, nici fiin?a mea pe care ai urzit-o ca în cele mai de jos ale pamântului.
Psa 139:16  Cele nelucrate ale mele le-au cunoscut ochii Tai ?i în cartea Ta toate se vor scrie; zi de zi se vor savâr?i ?i nici una din ele nu va fi nescrisa.
Psa 139:17  Iar eu am cinstit foarte pe prietenii Tai, Dumnezeule, ?i foarte s-a întarit stapânirea lor.
Psa 139:18  ?i-i voi numara pe ei, ?i mai mult decât nisipul se vor înmul?i. M-am sculat ?i înca sunt cu Tine.
Psa 139:19  O, de ai ucide pe pacato?i, Dumnezeule! Barba?i varsatori de sânge, departa?i-va de la mine!
Psa 139:20  Ace?tia Te graiesc de rau, Doamne, ?i vrajma?ii Î?i hulesc numele.
Psa 139:21  Oare, nu pe cei ce Te urasc pe Tine, Doamne, am urât ?i asupra vrajma?ilor Tai m-am mâhnit?
Psa 139:22  Cu ura desavâr?ita i-am urât pe ei ?i mi s-au facut du?mani.
Psa 139:23  Cerceteaza-ma, Doamne, ?i cunoa?te inima mea; încearca-ma ?i cunoa?te cararile mele
Psa 139:24  ?i vezi de este calea faradelegii în mine ?i ma îndrepteaza pe calea cea ve?nica.
Psa 140:1  (Un psalm al lui David; mai-marelui cântare?ilor.) Scoate-ma, Doamne, de la omul viclean ?i de omul nedrept ma izbave?te,
Psa 140:2  Care gândeau nedreptate în inima, toata ziua îmi duceau razboi.
Psa 140:3  Ascu?it-au limba lor ca a ?arpelui; venin de aspida sub buzele lor.
Psa 140:4  Paze?te-ma, Doamne, de mâna pacatosului; scoate-ma de la oamenii nedrep?i, care au gândit sa împiedice pa?ii mei.
Psa 140:5  Pusu-mi-au cei mândri cursa mie ?i funii; curse au întins picioarelor mele; pe carare mi-au pus pietre de poticneala.
Psa 140:6  Zis-am Domnului: "Dumnezeul meu e?ti Tu, asculta, Doamne, glasul rugaciunii mele".
Psa 140:7  Doamne, Doamne, puterea mântuirii mele, umbrit-ai capul meu în zi de razboi.
Psa 140:8  Sa nu ma dai pe mine, Doamne, din pricina poftei mele, pe mina pacatosului; viclenit-au împotriva mea; sa nu ma parase?ti, ca sa nu se trufeasca.
Psa 140:9  Vârful la?ului lor, osteneala buzelor lor sa-i acopere pe ei!
Psa 140:10  Sa cada peste ei carbuni aprin?i; în foc arunca-i pe ei, în necazuri, pe care sa nu le poata rabda.
Psa 140:11  Barbatul limbut nu se va îndrepta pe pamânt; pe omul nedrept rautatea îl va duce la pieire.
Psa 140:12  ?tiu ca Domnul va face judecata celui sarac ?i dreptate celor sarmani;
Psa 140:13  Iar drep?ii vor lauda numele Tau ?i vor locui cei drep?i în fa?a Ta.
Psa 141:1  (Un psalm al lui David.) Doamne, strigat-am catre Tine, auzi-ma; ia aminte la glasul rugaciunii mele, când strig catre Tine.
Psa 141:2  Sa se îndrepteze rugaciunea mea ca tamâia înaintea Ta; ridicarea mâinilor mele, jertfa de seara.
Psa 141:3  Pune Doamne, straja gurii mele ?i u?a de îngradire, împrejurul buzelor mele.
Psa 141:4  Sa nu aba?i inima mea spre cuvinte de vicle?ug, ca sa-mi dezvinova?esc pacatele mele; iar cu oamenii cei care fac faradelege nu ma voi înso?i cu ale?ii lor.
Psa 141:5  Certa-ma-va dreptul cu mila ?i ma va mustra, iar untdelemnul pacato?ilor sa nu unga capul meu; ca înca ?i rugaciunea mea este împotriva vrerilor lor.
Psa 141:6  Prabu?easca-se de pe stânca judecatorii lor. Auzi-se-vor graiurile mele ca s-au îndulcit,
Psa 141:7  Ca o brazda de pamânt s-au rupt pe pamânt, risipitu-s-au oasele lor lânga iad.
Psa 141:8  Caci catre Tine, Doamne, Doamne, ochii mei, spre Tine am nadajduit, sa nu iei sufletul meu.
Psa 141:9  Paze?te-ma de cursa care mi-au pus mie ?i de smintelile celor ce fac faradelege.
Psa 141:10  Cadea-vor în mreaja lor pacato?ii, ferit sunt eu pâna ce voi trece.
Psa 142:1  (Un psalm al lui David: al priceperii; când era el în pe?tera. O rugaciune.) Cu glasul meu catre Domnul am strigat, cu glasul meu catre Domnul m-am rugat.
Psa 142:2  Varsa-voi înaintea Lui rugaciunea mea, necazul meu înaintea Lui voi spune.
Psa 142:3  Când lipsea dintru mine duhul meu, Tu ai cunoscut cararile mele. În calea aceasta în care am umblat, ascuns-au cursa mie.
Psa 142:4  Luat-am seama de-a dreapta ?i am privit ?i nu era cine sa ma cunoasca. Pierit-a fuga de la mine ?i nu este cel ce cauta sufletul meu.
Psa 142:5  Strigat-am catre Tine, Doamne, zis-am: "Tu e?ti nadejdea mea, partea mea e?ti în pamântul celor vii".
Psa 142:6  Ia aminte la rugaciunea mea, ca m-am smerit foarte. Izbave?te-ma de cei ce ma prigonesc, ca s-au întarit mai mult decât mine.
Psa 142:7  Scoate din temni?a sufletul meu, ca sa laude numele Tau, Doamne. Pe mine ma a?teapta drep?ii, pâna ce-mi vei rasplati mie.
Psa 143:1  (Un psalm al lui David: al priceperii; când îl prigonea pe el fiul lui.) Doamne, auzi rugaciunea mea, asculta cererea mea, întru credincio?ia Ta, auzi-ma, întru dreptatea Ta.
Psa 143:2  Sa nu intri la judecata cu robul Tau, ca nimeni din cei vii nu-i drept înaintea Ta.
Psa 143:3  Vrajma?ul prigone?te sufletul meu ?i via?a mea o calca în picioare; facutu-m-a sa locuiesc în întuneric ca mor?ii cei din veacuri.
Psa 143:4  Mâhnit e duhul în mine ?i inima mea încremenita înlauntrul meu.
Psa 143:5  Adusu-mi-am aminte de zilele cele de demult; cugetat-am la toate lucrurile Tale, la faptele mâinilor Tale m-am gândit.
Psa 143:6  Întins-am catre Tine mâinile mele, sufletul meu ca un pamânt înseto?at.
Psa 143:7  Degrab auzi-ma, Doamne, ca a slabit duhul meu. Nu-?i întoarce fala Ta de la mine, ca sa nu ma aseman celor ce se coboara în mormânt.
Psa 143:8  Fa sa aud diminea?a mila Ta, ca la Tine îmi este nadejdea. Arata-mi calea pe care voi merge, ca la Tine am ridicat sufletul meu.
Psa 143:9  Scapa-ma de vrajma?ii mei, ca la Tine alerg, Doamne.
Psa 143:10  Înva?a-ma sa fac voia Ta, ca Tu e?ti Dumnezeul meu. Duhul Tau cel bun sa ma pova?uiasca la pamântul drepta?ii.
Psa 143:11  Pentru numele Tau, Doamne, daruie?te-mi via?a. Întru dreptatea Ta scoate din necaz sufletul meu.
Psa 143:12  Fa bunatate de stârpe?te pe vrajma?ii mei ?i pierde pe to?i cei ce necajesc sufletul meu, ca eu sunt robul Tau.
Psa 144:1  (Un psalm al lui David împotriva lui Goliat.) Binecuvântat este Domnul Dumnezeul meu, Cel ce înva?a mâinile mele la lupta ?i degetele mele la razboi.
Psa 144:2  Mila mea ?i Scaparea mea, Sprijinitorul meu ?i Izbavitorul meu, Aparatorul meu, ?i în El am nadajduit, Cel ce supune pe poporul meu sub mine.
Psa 144:3  Doamne, ce este omul ca Te-ai facut cunoscut lui, sau fiul omului ca-l socote?ti pe el?
Psa 144:4  Omul cu de?ertaciunea se aseamana; zilele lui ca umbra trec.
Psa 144:5  Doamne, pleaca cerurile ?i Te pogoara, atinge-Te de mun?i ?i fa-i sa fumege.
Psa 144:6  Cu fulger fulgera-i ?i-i risipe?te! Trimite sage?ile Tale ?i tulbura-i!
Psa 144:7  Trimite mâna Ta dintru înal?ime; scoate-ma ?i ma izbave?te de ape multe, din mâna strainilor,
Psa 144:8  A caror gura a grait de?ertaciune ?i dreapta lor e dreapta nedrepta?ii.
Psa 144:9  Dumnezeule, cântare noua Î?i voi cânta ?ie; în psaltire cu zece strune Î?i voi cânta ?ie,
Psa 144:10  Celui ce dai mântuire împara?ilor, Celui ce izbave?ti pe David, robul Tau, din robia cea cumplita.
Psa 144:11  Izbave?te-ma ?i ma scoate din mâna strainilor, a caror gura a grait de?ertaciune ?i dreapta lor e dreapta nedrepta?ii,
Psa 144:12  Ai caror fii sunt ca ni?te odrasle tinere, crescute în tinere?ile lor; fiicele lor înfrumuse?ate ?i împodobite ca chipurile templului.
Psa 144:13  Camarile lor pline, varsându-se din una în alta. Oile lor cu mul?i miei, umplând drumurile când ies;
Psa 144:14  Boii lor sunt gra?i. Nu este gard cazut, nici spartura, nici strigare în uli?ele lor.
Psa 144:15  Au fericit pe poporul care are aceste bunata?i. Dar fericit este poporul acela care are pe Domnul ca Dumnezeu al sau.
Psa 145:1  (Un psalm de lauda al lui David.) Înal?a-Te-voi Dumnezeul meu, Împaratul meu ?i voi binecuvânta numele Tau în veac ?i în veacul veacului.
Psa 145:2  În toate zilele Te voi binecuvânta ?i voi lauda numele Tau în veac ?i în veacul veacului.
Psa 145:3  Mare este Domnul ?i laudat foarte ?i mare?ia Lui nu are sfâr?it.
Psa 145:4  Neam ?i neam vor lauda lucrurile Tale ?i puterea Ta o vor vesti.
Psa 145:5  Mare?ia slavei sfin?eniei Tale vor grai ?i minunile Tale vor istorisi
Psa 145:6  ?i puterea lucrurilor Tale înfrico?atoare vor spune ?i slava Ta vor povesti.
Psa 145:7  Pomenirea mul?imii bunata?ii Tale vor vesti ?i de dreptatea Ta se vor bucura.
Psa 145:8  Îndurat ?i milostiv este Domnul, îndelung-rabdator ?i mult-milostiv.
Psa 145:9  Bun este Domnul cu to?i ?i îndurarile Lui peste toate lucrurile Lui.
Psa 145:10  Sa Te laude pe Tine, Doamne, toate lucrurile Tale ?i cuvio?ii Tai sa Te binecuvânteze.
Psa 145:11  Slava împara?iei Tale vor spune ?i de puterea Ta vor grai.
Psa 145:12  Ca sa se faca fiilor oamenilor cunoscuta puterea Ta ?i slava mare?iei împara?iei Tale.
Psa 145:13  Împara?ia Ta este împara?ia tuturor veacurilor, iar stapânirea Ta din neam în neam. Credincios este Domnul întru cuvintele Sale ?i cuvios întru toate lucrurile Sale.
Psa 145:14  Domnul sprijina pe to?i cei ce cad ?i îndreapta pe to?i cei gârbovi?i.
Psa 145:15  Ochii tuturor spre Tine nadajduiesc ?i Tu le dai lor hrana la buna vreme.
Psa 145:16  Deschizi Tu mâna Ta ?i de bunavoin?a saturi pe to?i cei vii.
Psa 145:17  Drept este Domnul în toate caile Lui ?i cuvios în toate lucrurile Lui.
Psa 145:18  Aproape este Domnul de to?i cei ce-L cheama pe El, de to?i cei ce-L cheama pe El întru adevar.
Psa 145:19  Voia celor ce se tem de El o va face ?i rugaciunea lor o va auzi ?i-i va mântui pe dân?ii.
Psa 145:20  Domnul paze?te pe to?i cei ce-L iubesc pe El ?i pe to?i pacato?ii îi va pierde.
Psa 145:21  Lauda Domnului va grai gura mea ?i sa binecuvinteze tot trupul  numele cel sfânt al Lui, zn veac ?i în veacul veacului.
Psa 146:1  (Un psalm al lui Agheu ?i al lui Zaharia.) Aliluia! Lauda, suflete al meu, pe Domnul.
Psa 146:2  Lauda-voi pe Domnul în via?a mea, cânta-voi Dumnezeului meu cât voi trai.
Psa 146:3  Nu va încrede?i în cei puternici, în fiii oamenilor, în care nu este izbavire.
Psa 146:4  Ie?i-va duhul lor ?i se vor întoarce în pamânt. În ziua aceea vor pieri toate gândurile lor.
Psa 146:5  Fericit cel ce are ajutor pe Dumnezeul lui Iacob, nadejdea lui, în Domnul Dumnezeul lui,
Psa 146:6  Cel ce a facut cerul ?i pamântul, marea ?i toate cele din ele; Cel ce paze?te adevarul în veac;
Psa 146:7  Cel ce face judecata celor napastui?i, Cel ce da hrana celor flamânzi. Domnul dezleaga pe cei fereca?i în obezi;
Psa 146:8  Domnul îndreapta pe cei gârbovi?i, Domnul în?elep?e?te orbii, Domnul iube?te pe cei drep?i;
Psa 146:9  Domnul paze?te pe cei straini; pe orfani ?i pe vaduva va sprijini ?i calea pacato?ilor o va pierde.
Psa 146:10  Împara?i-va Domnul în veac, Dumnezeul tau, Sioane, în neam ?i în neam.
Psa 147:1  (Un psalm al lui Agheu ?i al lui Zaharia.) Aliluia! Lauda?i pe Domnul, ca bine este a cânta; Dumnezeului nostru placuta Îi este cântarea.
Psa 147:2  Când va zidi Ierusalimul, Domnul va aduna ?i pe cei risipi?i ai lui Israel;
Psa 147:3  Cel ce vindeca pe cei zdrobi?i cu inima ?i leaga ranile lor
Psa 147:4  Cel ce numara mul?imea stelelor ?i da tuturor numele lor.
Psa 147:5  Mare este Domnul nostru ?l mare este taria Lui ?i priceperea Lui nu are hotar.
Psa 147:6  Domnul înal?a pe cei blânzi ?i smere?te pe cei pacato?i pâna la pamânt.
Psa 147:7  Cânta?i Domnului cu cântare de lauda; cânta?i Dumnezeului nostru în alauta;
Psa 147:8  Celui ce îmbraca cerul cu nori, Celui ce gate?te pamântului ploaie, Celui ce rasare în mun?i iarba ?i verdea?a spre slujba oamenilor;
Psa 147:9  Celui ce da animalelor hrana lor ?i puilor de corb, care Îl cheama pe El.
Psa 147:10  Nu în puterea calului este voia Lui, nici în cel iute la picior bunavoin?a Lui.
Psa 147:11  Bunavoin?a Domnului este în cei ce se tem de El ?i în cei ce nadajduiesc în mila Lui.
Psa 147:12  Lauda, Ierusalime, pe Domnul, lauda pe Dumnezeul tau, Sioane,
Psa 147:13  Ca a întarit stâlpii por?ilor tale, a binecuvântat pe fiii tai, în tine.
Psa 147:14  Cel ce pune la hotarele tale pace ?i cu fruntea grâului te-a saturat,
Psa 147:15  Cel ce trimite cuvântul Sau pamântului; repede alearga cuvântul Lui;
Psa 147:16  Cel ce da zapada ca lâna, Cel ce presara negura ca cenu?a,
Psa 147:17  Cel ce arunca ghea?a Lui, ca buca?elele de pâine; gerul Lui cine-l va suferi?
Psa 147:18  Va trimite cuvântul Lui ?i le va topi; va sufla Duhul Lui ?i vor curge apele.
Psa 147:19  Cel ce veste?te cuvântul Sau lui Iacob, îndreptarile ?i judeca?ile Sale lui Israel.
Psa 147:20  N-a facut a?a oricarui neam ?i judeca?ile Sale nu le-a aratat lor.
Psa 148:1  (Un psalm al lui Agheu ?i al lui Zaharia.) Aliluia! Lauda?i pe Domnul din ceruri, lauda?i-L pe El întru cele înalte.
Psa 148:2  Lauda?i-L pe El to?i îngerii Lui, lauda?i-L pe El toate puterile Lui.
Psa 148:3  Lauda?i-L pe El soarele ?i luna, lauda?i-L pe El toate stelele ?i lumina.
Psa 148:4  Lauda?i-L pe El cerurile cerurilor ?i apa cea mai presus de ceruri,
Psa 148:5  Sa laude numele Domnului, ca El a zis ?i s-au facut, El a poruncit ?i s-au zidit.
Psa 148:6  Pusu-le-ai pe ele în veac ?i în veacul veacului; lege le-a pus ?i nu o vor trece.
Psa 148:7  Lauda?i pe Domnul to?i cei de pe pamânt, balaurii ?i toate adâncurile;
Psa 148:8  Focul, grindina, zapada, ghea?a, viforul, toate îndeplini?i cuvântul Lui;
Psa 148:9  Mun?ii ?i toate dealurile, pomii cei roditori ?i to?i cedrii;
Psa 148:10  Fiarele ?i toate animalele, târâtoarele ?i pasarile cele zburatoare;
Psa 148:11  Împara?ii pamântului ?i toate popoarele, capeteniile ?i to?i judecatorii pamântului;
Psa 148:12  Tinerii ?i fecioarele, batrânii cu tinerii,
Psa 148:13  Sa laude numele Domnului, ca numai numele Lui s-a înal?at. Lauda Lui pe pamânt ?i în cer.
Psa 148:14  ?i va înal?a puterea poporului Lui. Cântare tuturor cuvio?ilor Lui, fiilor lui Israel, poporului ce se apropie de El.
Psa 149:1  Aliluia! Cânta?i Domnului cântare noua, lauda Lui în adunarea celor cuvio?i.
Psa 149:2  Sa se veseleasca Israel de Cel ce l-a facut pe el ?i fiii Sionului sa se bucure de Împaratul lor.
Psa 149:3  Sa laude numele Lui în hora; în timpane ?i în psaltire sa-I cânte Lui.
Psa 149:4  Ca iube?te Domnul poporul Sau ?i va înva?a pe cei blânzi ?i-i va izbavi.
Psa 149:5  Se vor lauda cuvio?ii întru slava ?i se vor bucura în a?ternuturile lor.
Psa 149:6  Laudele Domnului în gura lor ?i sabii cu doua tai?uri în mâinile lor,
Psa 149:7  Ca sa se razbune pe neamuri ?i sa pedepseasca pe popoare,
Psa 149:8  Ca sa lege pe împara?ii lor în obezi ?i pe cei slavi?i ai lor în catu?e de fier,
Psa 149:9  Ca sa faca între dân?ii judecata scrisa. Slava aceasta este a tuturor cuvio?ilor Sai.
Psa 150:1  Aliluia! Lauda?i pe Domnul întru sfin?ii Lui; lauda?i-L pe El întru taria puterii Lui.
Psa 150:2  Lauda?i-L pe El întru puterile Lui; lauda?i-L pe El dupa mul?imea slavei Lui.
Psa 150:3  Lauda?i-L pe El în glas de trâmbi?a; lauda?i-L pe El în psaltire ?i în alauta.
Psa 150:4  Lauda?i-L pe El în timpane ?i în hora; lauda?i-L pe El în strune ?i organe.
Psa 150:5  Lauda?i-L pe El în chimvale bine rasunatoare; lauda?i-L pe El în chimvale de strigare.
Psa 150:6  Toata suflarea sa laude pe Domnul!


\end{document}