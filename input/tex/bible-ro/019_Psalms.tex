\begin{document}

\title{Psalmii}


\chapter{1}

\par 1 (Un psalm al lui David, nescris deasupra la evrei.) Fericit barbatul, care n-a umblat în sfatul necredincio?ilor ?i în calea pacato?ilor nu a stat ?i pe scaunul hulitorilor n-a ?ezut;
\par 2 Ci în legea Domnului e voia lui ?i la legea Lui va cugeta ziua ?i noaptea.
\par 3 ?i va fi ca un pom rasadit lânga izvoarele apelor, care rodul sau va da la vremea sa ?i frunza lui nu va cadea ?i toate câte va face vor spori:
\par 4 Nu sunt a?a necredincio?ii, nu sunt a?a! Ci ca praful ce-l spulbera vântul de pe fa?a pamântului.
\par 5 De aceea nu se vor ridica necredincio?ii la judecata, nici pacato?ii în sfatul drep?ilor.
\par 6 Ca ?tie Domnul calea drep?ilor, iar calea necredincio?ilor va pieri.

\chapter{2}

\par 1 (Un psalm al lui David, nescris deasupra la evrei.) Pentru ce s-au întarâtat neamurile ?i popoarele au cugetat de?ertaciuni?
\par 2 S-au ridicat împara?ii pamântului ?i capeteniile s-au adunat împreuna împotriva Domnului ?i a unsului Sau, zicând:
\par 3 "Sa rupem legaturile lor ?i sa lepadam de la noi jugul lor".
\par 4 Cel ce locuie?te în ceruri va râde de dân?ii ?i Domnul îi va batjocori pe ei!
\par 5 Atunci va grai catre ei întru urgia Lui ?i întru mânia Lui îi va tulbura pe ei;
\par 6 Iar Eu sunt pus împarat de El peste Sion, muntele cel sfânt al Lui, vestind porunca Domnului.
\par 7 Domnul a zis catre Mine: "Fiul Meu e?ti Tu, Eu astazi Te-am nascut!
\par 8 Cere de la Mine ?i-?i voi da neamurile mo?tenirea Ta ?i stapânirea Ta, marginile pamântului.
\par 9 Le vei pa?te pe ele cu toiag de fier; ca pe vasul olarului le vei zdrobi!"
\par 10 ?i acum împara?i, în?elege?i! Înva?a?i-va to?i, care judeca?i pamântul!
\par 11 Sluji?i Domnului cu frica ?i va bucura?i de El cu cutremur.
\par 12 Lua?i înva?atura, ca nu cumva sa Se mânie Domnul ?i sa pieri?i din calea cea dreapta, când se va aprinde degrab mânia Lui! Ferici?i to?i cei ce nadajduiesc în El.

\chapter{3}

\par 1 (Un psalm al lui David, când a fugit din fa?a lui Avesalom, fiul sau.) Doamne, cât s-au înmul?it cei ce ma necajesc! Mul?i se scoala asupra mea;
\par 2 Mul?i zic sufletului meu: "Nu este mântuire lui, întru Dumnezeul lui! "
\par 3 Iar Tu, Doamne, sprijinitorul meu e?ti, slava mea ?i Cel ce înal?i capul meu.
\par 4 Cu glasul meu catre Domnul am strigat ?i m-a auzit din muntele cel sfânt al Lui.
\par 5 Eu m-am culcat ?i am adormit; sculatu-m-am, ca Domnul ma va sprijini.
\par 6 Nu ma voi teme de mii de popoare, care împrejur ma împresoara.
\par 7 Scoala, Doamne, mântuie?te-ma, Dumnezeul meu, ca Tu ai batut pe to?i cei ce ma vrajma?esc în de?ert; din?ii pacato?ilor ai zdrobit.
\par 8 A Domnului este mântuirea ?i peste poporul Tau, binecuvântarea Ta.

\chapter{4}

\par 1 (Un psalm al lui David; mai-marelui cântare?ilor pentru instrumente cu coarde.) Când Te-am chemat, m-ai auzit, Dumnezeul drepta?ii mele! Întru necaz m-ai desfatat! Milostive?te-Te spre mine ?i asculta rugaciunea mea.
\par 2 Fiii oamenilor, pâna când grei la inima? Pentru ce iubi?i de?ertaciunea ?i cauta?i minciuna?
\par 3 Sa ?ti?i ca minunat a facut Domnul pe cel cuvios al Sau; Domnul ma va auzi când voi striga catre Dânsul.
\par 4 Mânia?i-va, dar nu gre?i?i; de cele ce zice?i în inimile voastre, întru a?ternuturile voastre, va cai?i.
\par 5 Jertfi?i jertfa drepta?ii ?i nadajdui?i în Domnul.
\par 6 Mul?i zic: "Cine ne va arata noua cele bune?" Dar s-a însemnat peste noi lumina fe?ei Tale, Doamne!
\par 7 Dat-ai veselie în inima mea, mai mare decât veselia pentru rodul lor de grâu, de vin ?i de untdelemn ce s-a înmul?it.
\par 8 Cu pace, a?a ma voi culca ?i voi adormi, ca Tu, Doamne, îndeosebi întru nadejde m-ai a?ezat.

\chapter{5}

\par 1 (Un psalm al lui David; mai-marelui cântare?ilor, pentru instrumente de suflat.) Graiurile mele asculta-le, Doamne! În?elege strigarea mea!
\par 2 Ia aminte la glasul rugaciunii mele, Împaratul meu ?i Dumnezeul meu, caci catre Tine, ma voi ruga, Doamne!
\par 3 Diminea?a vei auzi glasul meu; diminea?a voi sta înaintea Ta ?i ma vei vedea.
\par 4 Ca Tu e?ti Dumnezeu, Care nu voie?ti faradelegea, nici nu va locui lânga Tine cel ce viclene?te.
\par 5 Nu vor sta calcatorii de lege în preajma ochilor Tai. Urât-ai pe to?i cei ce lucreaza fara de lege.
\par 6 Pierde-vei pe to?i cei ce graiesc minciuna; pe uciga? ?i pe viclean îl ura?te Domnul.
\par 7 Iar eu, întru mul?imea milei Tale, voi intra în casa Ta, închina-ma-voi spre sfânt loca?ul Tau, întru frica Ta.
\par 8 Doamne, pova?uie?te-ma întru dreptatea Ta din pricina du?manilor mei! Îndrepteaza înaintea mea calea Ta.
\par 9 Ca nu este în gura lor adevar, inima lor este de?arta; groapa deschisa grumazul lor, cu limbile lor viclenesc.
\par 10 Judeca-i pe ei, Dumnezeule; sa cada din sfaturile lor; dupa mul?imea nelegiuirilor lor, alunga-i pe ei, ca Te-au amarât, Doamne,
\par 11 ?i sa se veseleasca to?i cei ce nadajduiesc întru Tine; în veac se vor bucura ?i le vei fi lor sala? ?i se vor lauda cu Tine to?i cei ce iubesc numele Tau.
\par 12 Ca Tu vei binecuvânta pe cel drept, Doamne, caci cu arma bunei voiri ne-ai încununat pe noi.

\chapter{6}

\par 1 (Un psalm al lui David; mai-marelui cântare?ilor, pentru instrumente cu coarde.) Doamne, nu cu mânia Ta sa ma mustri pe mine, nici cu urgia Ta sa ma cer?i.
\par 2 Miluie?te-ma, Doamne, ca neputincios sunt; vindeca-ma, Doamne, ca s-au tulburat oasele mele;
\par 3 ?i sufletul meu s-a tulburat foarte ?i Tu, Doamne, pâna când?
\par 4 Întoarce-Te, Doamne; izbave?te sufletul meu, mântuie?te-ma, pentru mila Ta.
\par 5 Ca nu este întru moarte cel ce Te pomene?te pe Tine. ?i în iad cine Te va lauda pe Tine?
\par 6 Ostenit-am întru suspinul meu, spala-voi în fiecare noapte patul meu, cu lacrimile mele a?ternutul meu voi uda.
\par 7 Tulburatu-s-a de suparare ochiul meu, îmbatrânit-am între to?i vrajma?ii mei.
\par 8 Departa?i-va de la mine to?i cei ce lucra?i faradelegea, ca a auzit Domnul glasul plângerii mele.
\par 9 Auzit-a Domnul cererea mea, Domnul rugaciunea mea a primit.
\par 10 Sa se ru?ineze ?i sa se tulbure foarte to?i vrajma?ii mei; sa se întoarca ?i sa se ru?ineze foarte degrab.

\chapter{7}

\par 1 (Un psalm al lui David, pe care l-a cântat Domnului, pentru cuvintele lui Husi, fiul lui Iemeni.) Doamne, Dumnezeul meu, în Tine am nadajduit. Mântuie?te-ma de to?i cei ce ma prigonesc ?i ma izbave?te,
\par 2 Ca nu cumva sa rapeasca sufletul meu ca un leu, nefiind cine sa ma izbaveasca, nici cine sa ma mântuiasca.
\par 3 Doamne, Dumnezeul meu, de am facut aceasta, de este nedreptate în mâinile mele,
\par 4 De am rasplatit cu rau celor ce mi-au facut mie rau ?i de am jefuit pe vrajma?ii mei fara temei,
\par 5 Sa prigoneasca vrajma?ul sufletul meu ?i sa-l prinda, sa calce la pamânt via?a mea ?i marirea mea în ?arâna sa o a?eze.
\par 6 Scoala-Te, Doamne, întru mânia Ta, înal?a-Te pâna la hotarele vrajma?ilor mei; scoala-Te, Doamne, Dumnezeul meu, cu porunca cu care ai poruncit
\par 7 ?i adunare de popoare Te va înconjura ?i peste ea la înal?ime Te întoarce.
\par 8 Domnul va judeca pe popoare; judeca-ma, Doamne, dupa dreptatea mea ?i dupa nevinova?ia mea.
\par 9 Sfâr?easca-se rautatea pacato?ilor ?i întare?te pe cel drept, Cel ce cerci inimile ?i rarunchii, Dumnezeule drepte.
\par 10 Ajutorul meu de la Dumnezeu, Cel ce mântuie?te pe cei drep?i la inima.
\par 11 Dumnezeu este judecator drept, tare ?i îndelung-rabdator ?i nu se mânie în fiecare zi.
\par 12 De nu va ve?i întoarce, sabia Sa o va luci, arcul Sau l-a încordat ?i l-a pregatit
\par 13 ?i în el a gatit unelte de moarte; sage?ile Lui pentru cei ce ard le-a lucrat.
\par 14 Iata a poftit nedreptatea, a zamislit silnicia ?i a nascut nelegiuirea.
\par 15 Groapa a sapat ?i a adâncit-o ?i va cadea în groapa pe care a facut-o.
\par 16 Sa se întoarca nedreptatea lui pe capul lui ?i pe cre?tetul lui silnicia lui sa se coboare.
\par 17 Lauda-voi pe Domnul dupa dreptatea Lui ?i voi cânta numele Domnului Celui Preaînalt.

\chapter{8}

\par 1 (Un psalm al lui David; mai-marelui cântare?ilor, pentru ghitith.) Doamne, Dumnezeul nostru, cât de minunat este numele Tau în tot pamântul! Ca s-a înal?at slava Ta, mai presus de ceruri.
\par 2 Din gura pruncilor ?i a celor ce sug ai savâr?it lauda, pentru vrajma?ii Tai, ca sa amu?e?ti pe vrajma? ?i pe razbunator.
\par 3 Când privesc cerurile, lucrul mâinilor Tale, luna ?i stelele pe care Tu le-ai întemeiat, îmi zic:
\par 4 Ce este omul ca-?i aminte?ti de el? Sau fiul omului, ca-l cercetezi pe el?
\par 5 Mic?oratu-l-ai pe dânsul cu pu?in fa?a de îngeri, cu marire ?i cu cinste l-ai încununat pe el.
\par 6 Pusu-l-ai pe dânsul peste lucrul mâinilor Tale, toate le-ai supus sub picioarele lui.
\par 7 Oile ?i boii, toate; înca ?i dobitoacele câmpului;
\par 8 Pasarile cerului ?i pe?tii marii, cele ce strabat cararile marilor.
\par 9 Doamne, Dumnezeul nostru, cât de minunat este numele Tau în tot pamântul!

\chapter{9}

\par 1 (Un psalm al lui David; mai-marelui cântare?ilor, pentru cele ascunse ale fiului.) Lauda-Te-voi, Doamne, din toata inima mea, spune-voi toate minunile Tale.
\par 2 Veseli-ma-voi ?i ma voi bucura de Tine; cânta-voi numele Tau, Preaînalte.
\par 3 Când se vor întoarce vrajma?ii mei înapoi, slabi-vor ?i vor pieri de la fa?a Ta!
\par 4 Ca ai facut judecata mea ?i dreptatea mea; ?ezut-ai pe scaun, Cel ce judeci cu dreptate.
\par 5 Certat-ai neamurile ?i au pierit nelegiui?ii; stins-ai numele lor în veac ?i în veacul veacului.
\par 6 Vrajma?ului i-au lipsit de tot sabiile ?i ceta?ile i le-ai sfarâmat; pierit-a pomenirea lor în sunet.
\par 7 Iar Domnul ramâne în veac; gatit-a scaunul Lui de judecata
\par 8 ?i El va judeca lumea; cu dreptate va judeca popoarele.
\par 9 ?i a fost Domnul scapare saracului, ajutor la vreme potrivita în necazuri.
\par 10 Sa nadajduiasca în Tine cei ce cunosc numele Tau, ca n-ai parasit pe cei ce Te cauta pe Tine, Doamne!
\par 11 Cânta?i Domnului, Celui ce locuie?te în Sion, vesti?i între neamuri faptele Lui.
\par 12 Ca Cel ce razbuna sângele lor ?i-a adus aminte. N-a uitat strigatul saracilor.
\par 13 Miluie?te-ma, Doamne! Vezi smerenia mea, de catre vrajma?ii mei, Cel ce ma înal?i din por?ile mor?ii,
\par 14 Ca sa vestesc toate laudele Tale, în por?ile fiicei Sionului; veseli-ma-voi de mântuirea Ta!
\par 15 Cazut-au neamurile în groapa pe care au facut-o; în cursa aceasta, pe care au ascuns-o, s-a prins piciorul lor.
\par 16 Se cunoa?te Domnul când face judecata! Întru faptele mâinilor lui s-a prins pacatosul.
\par 17 Sa se întoarca pacato?ii în iad; toate neamurile care uita pe Dumnezeu.
\par 18 Ca nu pâna în sfâr?it va fi uitat saracul, iar rabdarea saracilor în veac nu va pieri.
\par 19 Scoala-Te, Doamne, sa nu se întareasca omul; sa fie judecate neamurile înaintea Ta!
\par 20 Pune, Doamne, legiuitor peste ele, ca sa cunoasca neamurile ca oameni sunt.

\chapter{10}

\par 1 Pentru ce, Doamne, stai departe? Pentru ce treci cu vederea la vreme de necaz?
\par 2 Când se mândre?te necredinciosul, se aprinde saracul; se prind în sfaturile pe care le gândesc.
\par 3 Ca se lauda pacatosul cu poftele sufletului lui, iar cel ce face strâmbatate, pe sine se binecuvinteaza.
\par 4 Întarâtat-a cel pacatos pe Domnul, dupa mul?imea mâniei lui; nu-L va cauta; nu este Dumnezeu înaintea lui.
\par 5 Spurcate sunt caile lui în toata vremea; lepadate sunt judeca?ile Tale de la fa?a lui, peste to?i vrajma?ii lui va stapâni.
\par 6 Ca a zis întru inima sa: Nu ma voi clinti din neam în neam, rau nu-mi va fi.
\par 7 Gura lui e plina de blestem, de amaraciune ?i de vicle?ug; sub limba lui osteneala ?i durere.
\par 8 Sta la pânda în ascuns cu cei boga?i ca sa ucida pe cel nevinovat; ochii lui spre cel sarac privesc.
\par 9 Pânde?te din ascunzi?, ca leul din culcu?ul sau; pânde?te ca sa apuce pe sarac, pânde?te pe sarac ca sa-l traga la el.
\par 10 În lan?ul lui îl va smeri; se va pleca ?i va cadea asupra lui, când va stapâni pe cei saraci.
\par 11 Ca a zis în inima lui: "Uitat-a Dumnezeu! Întors-a fa?a Lui, ca sa nu vada pâna în sfâr?it!"
\par 12 Scoala-Te, Doamne, Dumnezeul meu, înal?a-se mâna Ta, nu uita pe saracii Tai pâna în sfâr?it!
\par 13 Pentru ce a mâniat necredinciosul pe Dumnezeu? Ca a zis în inima lui: Domnul nu va cerceta!
\par 14 Vezi, pentru ca Tu prive?ti la necazuri ?i la durere, ca sa le iei în mâinile Tale; caci în Tine se încrede saracul, iar orfanului Tu i-ai fost ajutor.
\par 15 Zdrobe?te bra?ul celui pacatos ?i rau, pacatul lui va fi cautat ?i nu se va afla.
\par 16 Împara?i-va Domnul în veac ?i în veacul veacului! Pieri?i neamuri din pamântul Lui.
\par 17 Dorin?a saracilor a auzit-o Domnul; la râvna inimii lor a luat aminte urechea Ta.
\par 18 Judeca pe sarac ?i pe smerit, ca sa nu se mai mândreasca omul pe pamânt.

\chapter{11}

\par 1 (Un psalm al lui David; mai-marelui cântare?ilor.) În Domnul am nadajduit. Cum ve?i zice sufletului meu: "Muta-te în mun?i, ca o pasare?"
\par 2 Ca iata pacato?ii au încordat arcul, au gatit sage?i în tolba, ca sa sageteze în întuneric pe cei drep?i la inima.
\par 3 Ca au surpat ceea ce ai a?ezat; dar dreptul ce a facut?
\par 4 Domnul este în loca?ul cel sfânt al Sau, Domnul în cer are scaunul Sau. Ochii Lui spre sarac privesc, genele Lui cerceteaza pe fiii oamenilor.
\par 5 Domnul cerceteaza pe cel drept ?i pe cel necredincios; iar pe cel ce iube?te nedreptatea îl ura?te sufletul Sau.
\par 6 Va ploua peste pacato?i la?uri, foc ?i pucioasa; iar suflare de vifor este partea paharului lor.
\par 7 Ca drept este Domnul ?i dreptatea a iubit ?i fa?a Lui spre cel drept prive?te.

\chapter{12}

\par 1 (Un psalm al lui David; mai-marelui cântare?ilor, pentru instrumente cu opt coarde.) Mântuie?te-ma, Doamne, ca a lipsit cel cuvios, ca s-a împu?inat adevarul de la fiii oamenilor.
\par 2 De?ertaciuni a grait fiecare catre aproapele sau, buze viclene în inima ?i în inima rele au grait.
\par 3 Pierde-va Domnul toate buzele cele viclene ?i limba cea plina de mândrie.
\par 4 Pe cei ce au zis: "Cu limba noastra ne vom mari, caci buzele noastre la noi sunt; cine ne este Domn?"
\par 5 Pentru necazul saracilor ?i suspinul nenoroci?ilor, acum Ma voi scula, zice Domnul; le voi aduce lor mântuirea ?i le voi vorbi pe fa?a.
\par 6 Cuvintele Domnului, cuvinte curate, argint lamurit în foc, cura?at de pamânt, cura?at de ?apte ori.
\par 7 Tu, Doamne, ne vei pazi ?i ne vei feri de neamul acesta în veac.
\par 8 Caci, atunci când se ridica sus oamenii de nimic, nelegiui?ii mi?una pretutindeni.

\chapter{13}

\par 1 (Un psalm al lui David; mai-marelui cântare?ilor.) Pâna când, Doamne, ma vei uita pâna în sfâr?it? Pâna când vei întoarce fa?a Ta de la mine?
\par 2 Pâna când voi gramadi gânduri în sufletul meu, durere în inima mea ziua ?i noaptea? Pâna când se va înal?a vrajma?ul meu împotriva mea?
\par 3 Cauta, auzi-ma, Doamne, Dumnezeul meu, lumineaza ochii mei, ca nu cumva sa adorm întru moarte,
\par 4 Ca nu cumva sa zica vrajma?ul meu: "Întaritu-m-am asupra lui". Cei ce ma necajesc se vor bucura de ma voi clatina.
\par 5 Iar eu spre mila Ta am nadajduit; bucura-se-va inima mea de mântuirea Ta;
\par 6 Cânta-voi Domnului, Celui ce mi-a facut bine ?i voi cânta numele Domnului Celui Preaînalt.

\chapter{14}

\par 1 (Un psalm al lui David; mai-marelui cântare?ilor.) Zis-a cel nebun în inima sa: "Nu este Dumnezeu!" Stricatu-s-au oamenii ?i urâ?i s-au facut întru îndeletnicirile lor. Nu este cel ce face bunatate, nu este pâna la unul.
\par 2 Domnul din cer a privit peste fiii oamenilor, sa vada de este cel ce în?elege, sau cel ce cauta pe Dumnezeu.
\par 3 To?i s-au abatut, împreuna netrebnici s-au facut; nu este cel ce face bunatate, nu este pâna la unul.
\par 4 Oare, nu se vor în?elep?i to?i cei ce lucreaza faradelegea? Cei ce manânca pe poporul Meu ca pâinea, pe Domnul nu L-au chemat.
\par 5 Acolo s-au temut de frica, unde nu era frica, ca Dumnezeu este cu neamul drep?ilor.
\par 6 Saracul nadajduie?te în Domnul ?i voi a?i râs de nadejdea lui, zicând: Cine va da din Sion mântuire lui Israel?
\par 7 Dar când va întoarce Domnul pe cei robi?i ai poporului Sau, bucura-se-va Iacob ?i se va veseli Israel.

\chapter{15}

\par 1 (Un psalm al lui David.) Doamne, cine va locui în loca?ul Tau ?i cine se va sala?lui în muntele cel sfânt al Tau?
\par 2 Cel ce umbla fara prihana ?i face dreptate, cel ce are adevarul în inima sa,
\par 3 Cel ce n-a viclenit cu limba, nici n-a facut rau împotriva vecinului sau ?i ocara n-a rostit împotriva aproapelui sau.
\par 4 Defaimat sa fie înaintea Lui "el ce viclene?te, iar pe cei ce se tem de Domnul îi slave?te; cel ce se jura aproapelui sau ?i nu se leapada,
\par 5 Argintul sau nu l-a dat cu camata ?i daruri împotriva celor nevinova?i n-a luat. Cel ce face acestea nu se va clatina în veac.

\chapter{16}

\par 1 (Un psalm al lui David.) Paze?te-ma, Doamne, ca spre Tine am nadajduit.
\par 2 Zis-am Domnului: "Domnul meu e?ti Tu, ca bunata?ile mele nu-?i trebuie".
\par 3 Prin sfin?ii care sunt pe pamântul Lui minunata a facut Domnul toata voia întru ei.
\par 4 Înmul?itu-s-au slabiciunile celor ce alearga dupa al?i dumnezei. Nu voi lua parte la adunarile lor cu jertfe de sânge, nici nu voi pomeni numele lor pe buzele mele.
\par 5 Domnul este partea mo?tenirii mele ?i a paharului meu. Tu e?ti Cel care îmi a?ezi mie iara?i mo?tenirea mea.
\par 6 Sor?ii mi-au cazut între cei puternici, ca mo?tenirea mea este puternica.
\par 7 Binecuvânta-voi pe Domnul, Cel ce m-a în?elep?it; la aceasta ?i noaptea ma îndeamna inima mea.
\par 8 Vazut-am mai înainte pe Domnul înaintea mea pururea, ca de-a dreapta mea este ca sa nu ma clatin.
\par 9 Pentru aceasta s-a veselit inima mea ?i s-a bucurat limba mea, dar înca ?i trupul meu va sala?lui întru nadejde.
\par 10 Ca nu vei lasa sufletul meu în iad, nici nu vei da pe cel cuvios al Tau sa vada stricaciunea.
\par 11 Cunoscute mi-ai facut caile vie?ii; umplea-ma-vei de veselie cu fa?a Ta, ?i la dreapta Ta de frumuse?i ve?nice ma vei satura.

\chapter{17}

\par 1 (O rugaciune a lui David.) Auzi, Doamne, dreptatea mea, ia aminte cererea mea, asculta rugaciunea mea, din buze fara de viclenie.
\par 2 De la fa?a Ta judecata mea sa iasa, ochii mei sa vada cele drepte.
\par 3 Cercetat-ai inima mea, noaptea ai cercetat-o; cu foc m-ai lamurit, dar nu s-a aflat întru mine nedreptate.
\par 4 Ca sa nu graiasca gura mea lucruri omene?ti, pentru cuvintele buzelor Tale eu am pazit cai aspre.
\par 5 Îndreapta picioarele mele în cararile Tale, ca sa nu ?ovaie pa?ii mei.
\par 6 Eu am strigat, ca m-ai auzit Dumnezeule; pleaca urechea Ta catre mine ?i auzi cuvintele mele.
\par 7 Minunate fa milele Tale, Cel ce mântuie?ti pe cei ce nadajduiesc în Tine de cei ce stau împotriva dreptei Tale.
\par 8 Paze?te-ma, Doamne, ca pe lumina ochilor; cu acoperamântul aripilor Tale acopera-ma
\par 9 De fa?a necredincio?ilor care ma necajesc pe mine. Vrajma?ii mei sufletul meu l-au cuprins;
\par 10 Cu grasime inima lor ?i-au încuiat, gura lor a grait mândrie.
\par 11 Izgonindu-ma acum m-au înconjurat, ochii lor ?i-au a?intit ca sa ma plece la pamânt.
\par 12 Apucatu-m-au ca un leu gata de prada, ca un pui de leu ce locuie?te în ascunzi?uri.
\par 13 Scoala-Te, Doamne, întâmpina-i pe ei ?i împiedica-i! Izbave?te sufletul meu de cel necredincios, cu sabia Ta.
\par 14 Doamne, desparte-ma de oamenii acestei lumi, ce-?i iau partea în via?a, caci s-a umplut pântecele lor de bunata?ile Tale; saturatu-s-au fiii lor ?i au lasat rama?i?ele pruncilor.
\par 15 Iar eu întru dreptate ma voi arata fe?ei Tale, satura-ma-voi când se va arata slava Ta.

\chapter{18}

\par 1 (Mai-marelui cântare?ilor; un psalm al lui David, sluga Domnului, care a grait Domnului cuvintele cântarii acesteia, în ziua în care l-a izbavit pe el Domnul din mâna tuturor vrajma?ilor lui ?i din mâna lui Saul. ?i a zis:) Iubi-Te-voi Doamne, vârtutea mea.
\par 2 Domnul este întarirea mea ?i scaparea mea ?i izbavitorul meu, Dumnezeul meu, ajutorul meu ?i voi nadajdui spre Dânsul, aparatorul meu ?i puterea mântuirii mele ?i sprijinitorul meu.
\par 3 Laudând voi chema pe Domnul ?i de vrajma?ii mei ma voi izbavi.
\par 4 Cuprinsu-m-au durerile mor?ii ?i râurile faradelegii m-au tulburat.
\par 5 Durerile iadului m-au înconjurat; întâmpinatu-m-au la?urile mor?ii.
\par 6 ?i când ma necajeau am chemat pe Domnul ?i catre Dumnezeul meu am strigat. Auzit-a din loca?ul Lui cel sfânt glasul meu ?i strigarea mea, înaintea Lui, va intra în urechile Lui.
\par 7 ?i s-a clatinat ?i s-a cutremurat pamântul ?i temeliile mun?ilor s-au tulburat ?i s-au clatinat ca S-a mâniat pe ele Dumnezeu.
\par 8 Întru mânia Lui fum s-a ridicat ?i para de foc de la fa?a Lui, carbuni aprin?i de la El.
\par 9 ?i a plecat cerurile ?i S-a pogorât ?i negura era sub picioarele Lui.
\par 10 ?i S-a suit pe heruvimi ?i a zburat; zburat-a pe aripile vântului.
\par 11 ?i ?i-a pus întunericul acoperamânt, împrejurul Lui cortul Lui, apa întunecoasa în norii vazduhului.
\par 12 De stralucirea fe?ei Lui norii au fugit, glasul Lui prin grindina ?i carbuni de foc.
\par 13 ?i a tunat din cer Domnul ?i Cel Preaînalt a dat glasul Sau.
\par 14 Trimis-a sage?i ?i i-a risipit pe ei, ?i fulgere a înmul?it ?i i-a tulburat pe ei.
\par 15 ?i s-au aratat izvoarele apelor ?i s-au descoperit temeliile lumii, de certarea Ta, Doamne, de suflarea Duhului mâniei Tale.
\par 16 Trimis-a din înal?ime ?i m-a luat, ridicatu-m-a din ape multe.
\par 17 Izbave?te-ma de vrajma?ii mei cei tari ?i de cei ce ma urasc pe mine, ca s-au întarit mai mult decât mine.
\par 18 Întâmpinatu-m-au ei în ziua necazului meu, dar Domnul a fost întarirea mea
\par 19 ?i m-a scos la loc larg, m-a izbavit, ca m-a voit.
\par 20 ?i îmi va rasplati mie Domnul dupa dreptatea mea, ?i dupa cura?ia mâinilor mele îmi va rasplati mie,
\par 21 Ca am pazit caile Domnului ?i n-am fost necredincios Dumnezeului meu.
\par 22 Ca toate judeca?ile Lui sunt înaintea mea, ?i îndreptarile Lui nu s-au departat de la mine.
\par 23 ?i voi fi fara prihana cu Dânsul, ?i ma voi pazi de faradelegea mea.
\par 24 ?i îmi va rasplati mie Domnul dupa dreptatea mea, ?i dupa cura?ia mâinilor mele înaintea ochilor Lui.
\par 25 Cu cel cuvios, cuvios vei fi; ?i cu omul nevinovat, nevinovat vei fi.
\par 26 ?i cu cel ales, ales vei fi; ?i cu cel îndaratnic Te vei îndaratnici.
\par 27 Ca Tu pe poporul cel smerit îl vei mântui, ?i ochii mândrilor îi vei smeri.
\par 28 Ca Tu vei aprinde faclia mea, Doamne; Dumnezeul meu, vei lumina întunericul meu.
\par 29 Caci cu Tine ma voi izbavi de ispita, ?i cu Dumnezeul meu voi trece zidul.
\par 30 Dumnezeul meu, fara prihana este calea Lui, cuvintele Domnului în foc lamurite; scut este tuturor celor ce nadajduiesc în El.
\par 31 Ca cine este Dumnezeu afara de Domnul? ?i cine este Dumnezeu afara de Dumnezeul nostru?
\par 32 Dumnezeu, Cel ce ma încinge cu putere, ?i a pus fara prihana calea mea.
\par 33 Cel ce face picioarele mele ca ale cerbului ?i peste cele înalte ma pune.
\par 34 Cel ce întare?ti mâinile mele în vreme de razboi, ?i ai pus arc de arama în bra?ele mele.
\par 35 ?i mi-ai dat mie scutul mântuirii mele ?i dreapta Ta m-a sprijinit. ?i certarea Ta m-a îndreptat pâna în sfâr?it, ?i certarea Ta însa?i ma va înva?a.
\par 36 Largit-ai pa?ii mei sub mine, ?i n-au slabit picioarele mele.
\par 37 Urmari-voi pe vrajma?ii mei ?i-i voi prinde pe dân?ii ?i nu ma voi întoarce pâna ce se vor sfâr?i.
\par 38 Îi voi zdrobi pe ei ?i nu vor putea sa stea, cadea-vor sub picioarele mele.
\par 39 ?i m-ai încins cu putere spre razboi ?i ai împiedicat pe to?i cei ce se sculau împotriva mea.
\par 40 ?i pe vrajma?ii mei i-ai facut sa fuga, iar pe cei ce ma urasc pe mine i-ai nimicit.
\par 41 Strigat-au catre Domnul, ?i nu era cel ce mântuie?te; ?i nu i-a auzit pe ei.
\par 42 ?i-i voi sfarâma pe ei ca praful în fa?a vântului, ca tina uli?elor îi voi zdrobi pe ei.
\par 43 Izbave?te-ma de razvratirile poporului; pusu-m-ai capetenie neamurilor.
\par 44 Poporul pe care nu l-am cunoscut mi-a slujit mie. Cu auzul urechii m-a auzit.
\par 45 Fiii straini m-au min?it pe mine. Fiii straini au îmbatrânit ?i au ?chiopatat din cararile lor.
\par 46 Viu este Domnul ?i binecuvântat este Dumnezeul meu, ?i sa se înal?e Dumnezeul mântuirii mele.
\par 47 Dumnezeule, Cel ce mi-ai dat izbânda ?i mi-ai supus popoarele; Izbavitorul meu de vrajma?ii ceâ furio?i,
\par 48 De la cei ce se ridica împotriva mea, înal?a-ma, de omul nedrept izbave?te-ma.
\par 49 Pentru aceasta Te voi lauda între neamuri, Doamne, ?i numele Tau îl voi cânta.
\par 50 Cel ce mare?ti mântuirea împaratului Tau ?i faci mila unsului Tau, lui David ?i semin?iei lui pâna în veac.

\chapter{19}

\par 1 (Un psalm al lui David; mai-marelui cântare?ilor.) Cerurile spun slava lui Dumnezeu ?i facerea mâinilor Lui o veste?te taria.
\par 2 Ziua zilei spune cuvânt, ?i noaptea nop?ii veste?te ?tiin?a.
\par 3 Nu sunt graiuri, nici cuvinte, ale caror glasuri sa nu se auda.
\par 4 În tot pamântul a ie?it vestirea lor, ?i la marginile lumii cuvintele lor.
\par 5 În soare ?i-a pus loca?ul sau; ?i el este ca un mire ce iese din camara sa. Bucura-se-va ca un uria?, care alearga drumul lui.
\par 6 De la marginea cerului ie?irea lui, ?i oprirea lui pâna la marginea cerului; ?i nu este cine sa se ascunda de caldura lui.
\par 7 Legea Domnului este fara prihana, întoarce sufletele; marturia Domnului este credincioasa, în?elep?e?te pruncii;
\par 8 Judeca?ile Domnului sunt drepte, veselesc inima; porunca Domnului este stralucitoare, lumineaza ochii.
\par 9 Frica de Domnul este curata, ramâne în veacul veacului. Judeca?ile Domnului sunt adevarate, toate îndrepta?ite.
\par 10 Dorite sunt mai mult decât aurul, ?i decât piatra cea de mare pre?; ?i mai dulci decât mierea ?i fagurele.
\par 11 Ca robul Tau le paze?te pe ele, ?i rasplatire multa are.
\par 12 Gre?elile cine le va pricepe? De cele ascunse ale mele cura?e?te-ma
\par 13 ?i de cele straine fere?te pe robul Tau; de nu ma vor stapâni, atunci fara prihana voi fi ?i ma voi cura?i de pacat mare.
\par 14 ?i vor bineplacea cuvintele gurii mele ?i cugetul inimii mele înaintea Ta pururea; Doamne, Ajutorul meu ?i Izbavitorul meu.

\chapter{20}

\par 1 (Un psalm al lui David; mai-marelui cântare?ilor.) Sa te auda Domnul în ziua necazului ?i sa te apere numele Dumnezeului lui Iacob.
\par 2 Trimita ?ie ajutor din loca?ul Sau cel sfânt ?i din Sion sa te sprijineasca pe tine.
\par 3 Pomeneasca toata jertfa ta ?i arderea de tot a ta bineplacuta sa-I fie.
\par 4 Dea ?ie Domnul dupa inima ta ?i tot sfatul tau sa-l plineasca.
\par 5 Bucura-ne-vom de mântuirea ta ?i întru numele Dumnezeului nostru ne vom mari. Plineasca Domnul toate cererile tale.
\par 6 Acum am cunoscut ca a mântuit Domnul pe unsul Sau, cu puterea dreptei Sale. Îl va auzi pe dânsul din cerul cel sfânt al Lui.
\par 7 Unii se lauda cu caru?ele lor, al?ii cu caii lor, iar noi ne laudam cu numele Domnului Dumnezeului nostru.
\par 8 Ace?tia s-au împiedicat ?i au cazut, iar noi ne-am sculat ?i ne-am îndreptat.
\par 9 Doamne, mântuie?te pe împaratul ?i ne auzi pe noi, în orice zi Te vom chema.

\chapter{21}

\par 1 (Un psalm al lui David; mai-marelui cântare?ilor.) Doamne, întru puterea Ta se va veseli împaratul ?i întru mântuirea Ta se va bucura foarte.
\par 2 Dupa dorirea inimii lui i-ai dat lui, ?i de voia buzelor lui nu l-ai lipsit pe el.
\par 3 Ca l-ai întâmpinat pe el cu binecuvântarile bunata?ii, pus-ai pe capul lui cununa de piatra scumpa.
\par 4 Via?a a cerut de la Tine ?i i-ai dat lui lungime de zile în veacul veacului.
\par 5 Mare este slava lui întru mântuirea Ta, slava ?i mare cuviin?a vei pune peste el.
\par 6 Ca îi vei da lui binecuvântare în veacul veacului, îl vei veseli pe dânsul întru bucurie cu fa?a Ta.
\par 7 Ca împaratul nadajduie?te în Domnul ?i întru mila Celui Preaînalt nu se va clinti.
\par 8 Afla-se mâna Ta peste to?i vrajma?ii Tai, dreapta Ta sa afle pe to?i cei ce Te urasc pe Tine.
\par 9 Îi vei pune pe ei ca un cuptor de foc în vremea aratarii Tale; Domnul întru mânia Sa îi va tulbura pe ei, ?i-i va mânca pe ei focul.
\par 10 Rodul lor de pe pamânt îl vei pierde ?i samân?a lor dintre fiii oamenilor.
\par 11 Ca au gândit rele împotriva Ta, au cugetat sfaturi care nu vor putea sa stea.
\par 12 Ca îi vei pune pe ei pe fuga ?i cu arcul Tau vei ?inti capul lor.
\par 13 Înal?a-Te, Doamne, întru taria Ta, cânta-vom ?i vom lauda puterile Tale.

\chapter{22}

\par 1 (Un psalm al lui David; mai-marelui cântare?ilor, pentru sprijinul cel de diminea?a.) Dumnezeul meu, Dumnezeul meu, ia aminte la mine, pentru ce m-ai parasit? Departe sunt de mântuirea mea cuvintele gre?elilor mele.
\par 2 Dumnezeul meu, striga-voi ziua ?i nu vei auzi, ?i noaptea ?i nu Te vei gândi la mine.
\par 3 Iar Tu întru cele sfinte locuie?ti, lauda lui Israel.
\par 4 În Tine au nadajduit parin?ii no?tri, nadajduit-au în Tine ?i i-ai izbavit pe ei.
\par 5 Catre Tine au strigat ?i s-au mântuit, în Tine au nadajduit ?i nu s-au ru?inat.
\par 6 Iar eu sunt vierme ?i nu om, ocara oamenilor ?i defaimarea poporului.
\par 7 To?i cei ce m-au vazut m-au batjocorit, grait-au cu buzele, clatinat-au capul zicând:
\par 8 "Nadajduit-a spre Domnul, izbaveasca-l pe el, mântuiasca-l pe el, ca-l voie?te pe el".
\par 9 Ca Tu e?ti Cel ce m-ai scos din pântece, nadejdea mea, de la sânul maicii mele.
\par 10 Spre Tine m-am aruncat de la na?tere, din pântecele maicii mele Dumnezeul meu e?ti Tu.
\par 11 Nu Te departa de la mine, ca necazul este aproape, ?i nu este cine sa-mi ajute.
\par 12 Înconjuratu-m-au vi?ei mul?i, tauri gra?i m-au împresurat.
\par 13 Deschis-au asupra mea gura lor, ca un leu ce rape?te ?i racne?te.
\par 14 Ca apa m-am varsat ?i s-au risipit toate oasele mele. Facutu-s-a inima mea ca ceara ce se tope?te în mijlocul pântecelui meu.
\par 15 Uscatu-s-a ca un vas de lut taria mea, ?i limba mea s-a lipit de cerul gurii mele ?i în ?arâna mor?ii m-ai coborât.
\par 16 Ca m-au înconjurat câini mul?i, adunarea celor vicleni m-a împresurat. Strapuns-au mâinile mele ?i picioarele mele.
\par 17 Numarat-au toate oasele mele, iar ei priveau ?i se uitau la mine.
\par 18 Împar?it-au hainele mele loru?i ?i pentru cama?a mea au aruncat sor?i.
\par 19 Iar Tu, Doamne, nu departa ajutorul Tau de la mine, spre sprijinul meu ia aminte.
\par 20 Izbave?te de sabie sufletul meu ?i din gheara câinelui via?a mea.
\par 21 Izbave?te-ma din gura leului ?i din coarnele taurilor smerenia mea.
\par 22 Spune-voi numele Tau fra?ilor mei; în mijlocul adunarii Te voi lauda, zicând:
\par 23 Cei ce va teme?i de Domnul, lauda?i-L pe El, toata semin?ia lui Iacob slavi?i-L pe El! Sa se teama de Dânsul toata semin?ia lui Israel.
\par 24 Ca n-a defaimat, nici n-a lepadat ruga saracului, nici n-a întors fa?a Lui de la mine ?i când am strigat catre Dânsul, m-a auzit.
\par 25 De la Tine este lauda mea în adunare mare, rugaciunile mele le voi face înaintea celor ce se tem de El.
\par 26 Mânca-vor saracii ?i se vor satura ?i vor lauda pe Domnul, iar cei ce-L cauta pe Dânsul vii vor fi inimile lor în veacul veacului.
\par 27 Î?i vor aduce aminte ?i se vor întoarce la Domnul toate marginile pamântului. ?i se vor închina înaintea Lui toate semin?iile neamurilor.
\par 28 Ca a Domnului este împara?ia ?i El stapâne?te peste neamuri.
\par 29 Mâncat-au ?i s-au închinat to?i gra?ii pamântului, înaintea Lui vor cadea to?i cei ce se coboara în pamânt.
\par 30 ?i sufletul meu în El viaza, ?i semin?ia mea va sluji Lui. Se va vesti Domnului neamul ce va sa vina.
\par 31 ?i vor vesti dreptatea Lui poporului ce se va na?te ?i ce a facut Domnul.

\chapter{23}

\par 1 (Un psalm al lui David.) Domnul ma pa?te ?i nimic nu-mi va lipsi.
\par 2 La loc de pa?une, acolo m-a sala?luit; la apa odihnei m-a hranit.
\par 3 Sufletul meu l-a întors, pova?uitu-m-a pe caile drepta?ii, pentru numele Lui.
\par 4 Ca de voi ?i umbla în mijlocul mor?ii, nu ma voi teme de rele; ca Tu cu mine e?ti. Toiagul Tau ?i varga Ta, acestea m-au mângâiat.
\par 5 Gatit-ai masa înaintea mea, împotriva celor ce ma necajesc; uns-ai cu untdelemn capul meu ?i paharul Tau este adapându-ma ca un puternic.
\par 6 ?i mila Ta ma va urma în toate zilele vie?ii mele, ca sa locuiesc în casa Domnului, întru lungime de zile.

\chapter{24}

\par 1 (Un psalm al lui David; al uneia din sâmbete.) Al Domnului este pamântul ?i plinirea lui; lumea ?i to?i cei ce locuiesc în ea.
\par 2 Acesta pe mari l-a întemeiat pe el ?i pe râuri l-a a?ezat pe el.
\par 3 Cine se va sui în muntele Domnului ?i cine va sta în locul cel sfânt al Lui?
\par 4 Cel nevinovat cu mâinile ?i curat cu inima, care n-a luat în de?ert sufletul sau ?i nu s-a jurat cu vicle?ug aproapelui sau.
\par 5 Acesta va lua binecuvântare de la Domnul ?i milostenie de la Dumnezeu, Mântuitorul sau.
\par 6 Acesta este neamul celor ce-L cauta pe Domnul, al celor ce cauta fa?a Dumnezeului lui Iacob.
\par 7 Ridica?i, capetenii, por?ile voastre ?i va ridica?i por?ile cele ve?nice ?i va intra Împaratul slavei.
\par 8 Cine este acesta Împaratul slavei? Domnul Cel tare ?i puternic, Domnul Cel tare în razboi.
\par 9 Ridica?i, capetenii, por?ile voastre ?i va ridica?i por?ile cele ve?nice ?i va intra Împaratul slavei.
\par 10 Cine este acesta Împaratul slavei? Domnul puterilor, Acesta este Împaratul slavei.

\chapter{25}

\par 1 (Un psalm al lui David.) Catre Tine, Doamne, am ridicat sufletul meu, Dumnezeul meu.
\par 2 Spre Tine am nadajduit, sa nu fiu ru?inat în veac, nici sa râda de mine vrajma?ii mei.
\par 3 Pentru ca to?i cei ce Te a?teapta pe Tine nu se vor ru?ina; sa se ru?ineze to?i cei ce fac faradelegi în de?ert.
\par 4 Caile Tale, Doamne, arata-mi, ?i cararile Tale ma înva?a.
\par 5 Îndrepteaza-ma spre adevarul Tau ?i ma înva?a, ca Tu e?ti Dumnezeu, Mântuitorul meu, ?i pe Tine Te-am a?teptat toata ziua.
\par 6 Adu-?i aminte de îndurarile ?i milele Tale, Doamne, ca din veac sunt.
\par 7 Pacatele tinere?ilor mele ?i ale ne?tiin?ei mele nu le pomeni. Dupa mila Ta pomene?te-ma Tu, pentru bunatatea Ta, Doamne.
\par 8 Bun ?i drept este Domnul, pentru aceasta lege va pune celor ce gre?esc în cale.
\par 9 Îndrepta-va pe cei blânzi la judecata, înva?a-va pe cei blânzi caile Sale.
\par 10 Toate caile Domnului sunt mila ?i adevar pentru cei ce cauta a?ezamântul Lui ?i marturiile Lui.
\par 11 Pentru numele Tau, Doamne, cura?e?te pacatul meu ca mult este.
\par 12 Cine este omul cel ce se teme de Domnul? Lege va pune lui în calea pe care a ales-o.
\par 13 Sufletul lui întru bunata?i se va sala?lui ?i semin?ia lui va mo?teni pamântul.
\par 14 Domnul este întarirea celor ce se tem de El, a?ezamântul Lui îl va arata lor.
\par 15 Ochii mei sunt pururea spre Domnul ca El va scoate din la? picioarele mele.
\par 16 Cauta spre mine ?i ma miluie?te, ca parasit ?i sarac sunt eu.
\par 17 Necazurile inimii mele s-au înmul?it; din nevoile mele scoate-ma.
\par 18 Vezi smerenia mea ?i osteneala mea ?i-mi iarta toate pacatele mele.
\par 19 Vezi pe vrajma?ii mei ca s-au înmul?it ?i cu ura nedreapta m-au urât.
\par 20 Paze?te sufletul meu ?i ma izbave?te, ca sa nu ma ru?inez ca am nadajduit în Tine.
\par 21 Cei fara rautate ?i cei drep?i s-au lipit de mine, ca Te-am a?teptat, Doamne.
\par 22 Izbave?te, Dumnezeule, pe Israel din toate necazurile lui.

\chapter{26}

\par 1 (Un psalm al lui David.) Judeca-ma, Doamne, ca eu întru nerautate am umblat ?i în Domnul nadajduind, nu voi slabi.
\par 2 Cerceteaza-ma, Doamne, ?i ma cearca; aprinde rarunchii ?i inima mea.
\par 3 Ca mila Ta este înaintea ochilor mei ?i bine mi-a placut adevarul Tau.
\par 4 Nu am ?ezut în adunarea de?ertaciunii ?i cu calcatorii de lege nu voi intra.
\par 5 Urât-am adunarea celor ce viclenesc ?i cu cei necredincio?i nu voi ?edea.
\par 6 Spala-voi întru cele nevinovate mâinile mele ?i voi înconjura jertfelnicul Tau, Doamne,
\par 7 Ca sa aud glasul laudei Tale ?i sa spun toate minunile Tale.
\par 8 Doamne, iubit-am bunacuviin?a casei Tale ?i locul loca?ului slavei Tale.
\par 9 Sa nu pierzi cu cei necredincio?i sufletul meu ?i cu varsatorii de sânge via?a mea,
\par 10 Întru ale caror mâini sunt faradelegi ?i dreapta carora e plina de daruri.
\par 11 Iar eu întru nerautatea mea am umblat; izbave?te-ma, Doamne, ?i ma miluie?te,
\par 12 Caci piciorul meu a stat întru dreptate; întru adunari Te voi binecuvânta, Doamne.

\chapter{27}

\par 1 (Un psalm al lui David; mai înainte de ungere.) Domnul este luminarea mea ?i mântuirea mea; de cine ma voi teme? Domnul este aparatorul vie?ii mele; de cine ma voi înfrico?a?
\par 2 Când se vor apropia de mine cei ce îmi fac rau, ca sa manânce trupul meu; cei ce ma necajesc ?i vrajma?ii mei, aceia au slabit ?i au cazut.
\par 3 De s-ar rândui împotriva mea o?tire, nu se va înfrico?a inima mea; de s-ar ridica împotriva mea razboi, eu în El nadajduiesc.
\par 4 Una am cerut de la Domnul, pe aceasta o voi cauta: sa locuiesc în casa Domnului în toate zilele vie?ii mele, ca sa vad frumuse?ea Domnului ?i sa cercetez loca?ul Lui.
\par 5 Ca Domnul m-a ascuns în cortul Lui în ziua necazurilor mele; m-a acoperit în locul cel ascuns al cortului Lui; pe piatra m-a înal?at.
\par 6 ?i acum iata, a înal?at capul meu peste vrajma?ii mei. Înconjurat-am ?i am jertfit în cortul Lui jertfa de lauda. Îl voi lauda ?i voi cânta Domnului.
\par 7 Auzi, Doamne, glasul meu cu care am strigat; miluie?te-ma ?i ma asculta.
\par 8 ?ie a zis inima mea: Pe Domnul voi cauta. Te-a cautat fa?a mea; fa?a Ta, Doamne, voi cauta.
\par 9 Sa nu-?i întorci fa?a Ta de la mine ?i sa nu Te aba?i întru mânie de la robul Tau; ajutorul meu fii, sa nu ma lepezi pe mine ?i sa nu ma la?i, Dumnezeule, Mântuitorul meu.
\par 10 Ca tatal meu ?i mama mea m-au parasit, dar Domnul m-a luat.
\par 11 Lege pune-mi mie, Doamne, în calea Ta ?i ma îndrepteaza pe cararea dreapta, din pricina vrajma?ilor mei.
\par 12 Nu ma da pe mine pe mâna celor ce ma necajesc, ca s-au ridicat împotriva mea martori nedrep?i ?i nedreptatea a min?it sie?i.
\par 13 Cred ca voi vedea bunata?ile Domnului, în pamântul celor vii.
\par 14 A?teapta pe Domnul, îmbarbateaza-te ?i sa se întareasca inima ta ?i a?teapta pe Domnul.

\chapter{28}

\par 1 (Un psalm al lui David.) Catre Tine, Doamne, am strigat, Dumnezeul meu, ia aminte! Ca de nu ma vei auzi, ma voi asemana cu cei care se coboara în groapa.
\par 2 Asculta glasul rugaciunii mele când ma rog catre Tine, când ridic mâinile mele catre loca?ul Tau cel sfânt.
\par 3 Sa nu tragi cu cei pacato?i sufletul meu, ?i cu cei ce lucreaza nedreptate sa nu ma pierzi, cu cei ce graiesc pace catre aproapele lor, dar cele rele sunt în inimile lor.
\par 4 Da-le lor dupa faptele lor ?i dupa vicle?ugul gândurilor lor. Dupa lucrul mâinilor lor, da-le lor; rasplate?te-i dupa faptele lor,
\par 5 Ca n-au în?eles lucrurile Domnului ?i faptele mâinilor Lui; îi vei darâma ?i nu-i vei zidi.
\par 6 Binecuvântat este Domnul ca a auzit glasul rugaciunii mele.
\par 7 Domnul este ajutorul ?i aparatorul meu, în El a nadajduit inima mea ?i mi-a ajutat. ?i a înflorit trupul meu ?i de bunavoia mea Îl voi lauda pe El.
\par 8 Domnul este întarirea poporului Sau ?i aparator mântuirilor unsului Sau.
\par 9 Mântuie?te poporul Tau ?i binecuvinteaza mo?tenirea Ta; pa?te-i pe ei ?i-i ridica pâna în veac.

\chapter{29}

\par 1 (Un psalm al lui David; la scoaterea Cortului.) Aduce?i Domnului, fii ai lui Dumnezeu, aduce?i Domnului mieii oilor, aduce?i Domnului slava ?i cinste;
\par 2 Aduce?i Domnului slava numelui Sau; închina?i-va Domnului în curtea cea sfânta a Lui.
\par 3 Glasul Domnului peste ape; Dumnezeul slavei a tunat; Domnul peste ape multe.
\par 4 Glasul Domnului întru tarie, glasul Domnului întru mare cuviin?a;
\par 5 Glasul Domnului cel ce sfarâma cedrii ?i va zdrobi Domnul cedrii Libanului;
\par 6 El face sa sara Libanul ca un vi?el; iar Ermonul ca un pui de gazela.
\par 7 Glasul Domnului, cel ce varsa para focului.
\par 8 Glasul Domnului, cel ce cutremura pustiul ?i va cutremura Domnul pustiul Cade?ului.
\par 9 Glasul Domnului dezleaga pântecele cerboaicelor, glasul Domnului despoaie cedrii ?i în loca?ul Lui, fiecare va spune: Slava!
\par 10 Domnul va împara?i peste potop ?i va ?edea Domnul Împarat în veac.
\par 11 Domnul tarie poporului Sau va da, Domnul va binecuvânta pe poporul Sau cu pace.

\chapter{30}

\par 1 (Un psalm al lui David; mai-marelui cântare?ilor pentru sfin?irea casei.) Te voi înal?a, Doamne, ca m-ai ridicat ?i n-ai veselit pe vrajma?ii mei împotriva mea.
\par 2 Doamne, Dumnezeul meu, strigat-am catre Tine ?i m-ai vindecat.
\par 3 Doamne, scos-ai din iad sufletul meu, mântuitu-m-ai de cei ce se coboara în groapa.
\par 4 Cânta?i Domnului cei cuvio?i ai Lui ?i lauda?i pomenirea sfin?eniei Lui.
\par 5 Ca iu?ime este întru mânia Lui ?i via?a întru voia Lui; seara se va sala?lui plângerea, iar diminea?a bucuria.
\par 6 Iar eu am zis întru îndestularea mea: "Nu ma voi clatina în veac!"
\par 7 Doamne, întru voia Ta, dat-ai frumuse?ii mele putere; dar când ?i-ai întors fa?a Ta, eu m-am tulburat.
\par 8 Catre Tine, Doamne, voi striga ?i catre Dumnezeul meu ma voi ruga.
\par 9 Ce folos ai de sângele meu de ma cobor în stricaciune? Oare, Te va lauda pe Tine ?arâna, sau va vesti adevarul Tau?
\par 10 Auzit-a Domnul ?i m-a miluit; Domnul a fost ajutorul meu!
\par 11 Schimbat-ai plângerea mea întru bucurie, rupt-ai sacul meu ?i m-ai încins cu veselie.
\par 12 Ca slava mea sa-?i cânte ?ie ?i sa nu ma mâhnesc; Doamne, Dumnezeul meu, în veac Te voi lauda!

\chapter{31}

\par 1 (Un psalm al lui David; mai-marelui cântare?ilor, pentru uimire.) Spre Tine, Doamne, am nadajduit, sa nu fiu ru?inat în veac. Întru îndreptarea Ta izbave?te-ma ?i ma scoate.
\par 2 Pleaca spre mine urechea Ta, grabe?te de ma scoate. Fii mie Dumnezeu aparator ?i casa de scapare ca sa ma mântuie?ti.
\par 3 Ca puterea mea ?i scaparea mea e?ti Tu ?i pentru numele Tau ma vei pova?ui ?i ma vei hrani.
\par 4 Scoate-ma-vei din cursa aceasta pe care mi-au ascuns-o mie, ca Tu e?ti aparatorul meu.
\par 5 În mâinile Tale îmi voi da duhul meu; izbavitu-m-ai, Doamne, Dumnezeul adevarului.
\par 6 Urât-ai pe cei ce pazesc de?ertaciuni în zadar, iar eu spre Domnul am nadajduit.
\par 7 Bucura-ma-voi ?i ma voi veseli de mila Ta, ca ai cautat spre smerenia mea, mântuit-ai din nevoi sufletul meu
\par 8 ?i nu m-ai lasat în mâinile vrajma?ului; pus-ai în loc desfatat picioarele mele.
\par 9 Miluie?te-ma, Doamne, ca ma necajesc; tulburatu-s-a de mânie ochiul meu, sufletul meu ?i inima mea.
\par 10 Ca s-a stins întru durere via?a mea ?i anii mei în suspinuri; slabit-a întru saracie taria mea ?i oasele mele s-au tulburat.
\par 11 La to?i vrajma?ii mei m-am facut de ocara ?i vecinilor mei foarte, ?i frica cunoscu?ilor mei. Cei ce ma vedeau afara fugeau de mine.
\par 12 Uitat am fost ca un mort din inima lor, ajuns-am ca un vas stricat.
\par 13 Ca am auzit ocara multora din cei ce locuiesc împrejur, când se adunau ei împreuna împotriva mea; ca sa ia sufletul meu s-au sfatuit.
\par 14 Iar eu catre Tine am nadajduit, Doamne, zis-am: "Tu e?ti Dumnezeul meu!"
\par 15 În mâinile Tale, soarta mea, izbave?te-ma din mâna vrajma?ilor mei ?i de cei ce ma prigonesc.
\par 16 Arata fa?a Ta peste robul Tau, mântuie?te-ma cu mila Ta!
\par 17 Doamne, sa nu fiu ru?inat, ca Te-am chemat pe Tine; sa se ru?ineze necredincio?ii ?i sa se coboare în iad.
\par 18 Mute sa fie buzele cele viclene, care graiesc împotriva dreptului faradelege, cu mândrie ?i cu defaimare.
\par 19 Cât este de mare mul?imea bunata?ii Tale, Doamne, pe care ai gatit-o celor ce se tem de Tine, pe care ai facut-o celor ce nadajduiesc în Tine, înaintea fiilor oamenilor!
\par 20 Ascunde-i-vei pe dân?ii cu acoperamântul fe?ei Tale de tulburarea oamenilor. Acoperi-i-vei pe ei în cortul Tau de împotrivirea limbilor.
\par 21 Binecuvântat este Domnul, ca minunata a fost mila Sa, în cetate întarita.
\par 22 Iar eu am zis întru uimirea mea: Lepadat sunt de la fa?a ochilor Tai. Pentru aceasta ai auzit glasul rugaciunii mele când am strigat catre Tine.
\par 23 Iubi?i pe Domnul to?i cuvio?ii Lui ca adevarul cauta Domnul ?i rasplate?te celor ce se mândresc, cu prisosin?a.
\par 24 Îmbarbata?i-va ?i sa se întareasca inima voastra, to?i cei ce nadajdui?i în Domnul.

\chapter{32}

\par 1 (Un psalm al lui David; pentru pricepere.) Ferici?i carora s-au iertat faradelegile ?i carora s-au acoperit pacatele.
\par 2 Fericit barbatul, caruia nu-i va socoti Domnul pacatul, nici nu este în gura lui vicle?ug.
\par 3 Ca am tacut, îmbatrânit-au oasele mele, când strigam toata ziua.
\par 4 Ca ziua ?i noaptea s-a îngreunat peste mine mâna Ta ?i am cazut în suferin?a când ghimpele Tau ma împungea.
\par 5 Pacatul meu l-am cunoscut ?i faradelegea mea n-am ascuns-o, împotriva mea. Zis-am: "Marturisi-voi faradelegea mea Domnului"; ?i Tu ai iertat nelegiuirea pacatului meu.
\par 6 Pentru aceasta se va ruga catre Tine tot cuviosul la vreme potrivita, iar potop de ape multe de el nu se va apropia.
\par 7 Tu e?ti scaparea mea din necazul ce ma cuprinde, bucuria mea; izbave?te-ma de cei ce m-au înconjurat.
\par 8 În?elep?i-te-voi ?i te voi îndrepta pe calea aceasta, pe care vei merge; a?inti-voi spre tine ochii Mei.
\par 9 Nu fi ca un cal ?i ca un catâr, la care nu este pricepere; cu zabala ?i cu frâu falcile lor voi strânge ca sa nu se apropie de tine.
\par 10 Multe sunt bataile pacatosului; iar pe cel ce nadajduie?te în Domnul, mila îl va înconjura.
\par 11 Veseli?i-va în Domnul ?i va bucura?i, drep?ilor, ?i va lauda?i to?i cei drep?i la inima.

\chapter{33}

\par 1 (Un psalm al lui David, nescris deasupra la evrei.) Bucura?i-va, drep?ilor; celor drep?i li se cuvine lauda.
\par 2 Lauda?i pe Domnul în alauta, în psaltire cu zece strune cânta?i-I Lui.
\par 3 Cânta?i-I Lui cântare noua, cânta?i-I frumos, cu strigat de bucurie.
\par 4 Ca drept este cuvântul Domnului ?i toate lucrurile Lui întru credin?a.
\par 5 Iube?te milostenia ?i judecata, Domnul; de mila Domnului plin este pamântul.
\par 6 Cu cuvântul Domnului cerurile s-au întarit ?i cu duhul gurii Lui toata puterea lor.
\par 7 Adunat-a ca într-un burduf apele marii, pus-a în vistierii adâncurile.
\par 8 Sa se teama de Domnul tot pamântul ?i de El sa se cutremure to?i locuitorii lumii.
\par 9 Ca El a zis ?i s-au facut, El a poruncit ?i s-au zidit.
\par 10 Domnul risipe?te sfaturile neamurilor, leapada gândurile popoarelor ?i defaima sfaturile capeteniilor.
\par 11 Iar sfatul Domnului ramâne în veac, gândurile inimii Lui, din neam în neam.
\par 12 Fericit este neamul caruia Domnul este Dumnezeul lui, poporul pe care l-a ales de mo?tenire Lui.
\par 13 Din cer a privit Domnul, vazut-a pe to?i fiii oamenilor.
\par 14 Din loca?ul Sau, cel gata, privit-a spre to?i cei ce locuiesc pamântul.
\par 15 Cel ce a zidit îndeosebi inimile lor, Cel ce pricepe toate lucrurile lor.
\par 16 Nu se mântuie?te împaratul cu o?tire multa ?i uria?ul nu se va mântui cu mul?imea tariei lui.
\par 17 Mincinos este calul spre scapare ?i cu mul?imea puterii lui nu te va izbavi.
\par 18 Iata ochii Domnului spre cei ce se tem de Dânsul, spre cei ce nadajduiesc în mila Lui.
\par 19 Ca sa izbaveasca de moarte sufletele lor ?i sa-i hraneasca pe ei în foamete.
\par 20 ?i sufletul nostru a?teapta pe Domnul, ca ajutorul ?i aparatorul nostru este.
\par 21 Ca în El se va veseli inima noastra ?i în numele cel sfânt al Lui am nadajduit.
\par 22 Fie, Doamne, mila Ta spre noi, precum am nadajduit ?i noi întru Tine.

\chapter{34}

\par 1 (Un psalm al lui David, când ?i-a schimbat fa?a sa înaintea lui Abimelec ?i i-a dat drumul ?i s-a dus.) Bine voi cuvânta pe Domnul în toata vremea, pururea lauda Lui în gura mea.
\par 2 În Domnul se va lauda sufletul meu; sa auda cei blânzi ?i sa se veseleasca.
\par 3 Slavi?i pe Domnul împreuna cu mine ?i sa înal?am numele Lui împreuna.
\par 4 Cautat-am pe Domnul ?i m-a auzit ?i din toate necazurile mele m-a izbavit.
\par 5 Apropia?i-va de El ?i va lumina?i; ?i fe?ele voastre sa nu se ru?ineze.
\par 6 Saracul acesta a strigat ?i Domnul l-a auzit pe el ?i din toate necazurile lui l-a izbavit.
\par 7 Strajui-va îngerul Domnului împrejurul celor ce se tem de El ?i-i va izbavi pe ei.
\par 8 Gusta?i ?i vede?i ca bun este Domnul; fericit barbatul care nadajduie?te în El.
\par 9 Teme?i-va de Domnul to?i sfin?ii Lui, ca n-au lipsa cei ce se tem de El.
\par 10 Boga?ii au saracit ?i au flamânzit, iar cei ce-L cauta pe Domnul, nu se vor lipsi de tot binele.
\par 11 Veni?i fiilor, asculta?i-ma pe mine, frica Domnului va voi înva?a pe voi;
\par 12 Cine este omul cel ce voie?te via?a, care iube?te sa vada zile bune?
\par 13 Opre?te-?i limba de la rau ?i buzele tale sa nu graiasca vicle?ug.
\par 14 Fere?te-te de rau ?i fa bine, cauta pacea ?i o urmeaza pe ea.
\par 15 Ochii Domnului spre cei drep?i ?i urechile Lui spre rugaciunea lor.
\par 16 Iar fa?a Domnului spre cei ce fac rele, ca sa piara de pe pamânt pomenirea lor.
\par 17 Strigat-au drep?ii ?i Domnul i-a auzit ?i din toate necazurile lor i-a izbavit.
\par 18 Aproape este Domnul de cei umili?i la inima ?i pe cei smeri?i cu duhul îi va mântui.
\par 19 Multe sunt necazurile drep?ilor ?i din toate acelea îi va izbavi pe ei Domnul.
\par 20 Domnul paze?te toate oasele lor, nici unul din ele nu se va zdrobi.
\par 21 Moartea pacato?ilor este cumplita ?i cei ce urasc pe cel drept vor gre?i.
\par 22 Mântui-va Domnul sufletele robilor Sai ?i nu vor gre?i to?i cei ce nadajduiesc în El.

\chapter{35}

\par 1 (Un psalm al lui David.) Judeca, Doamne, pe cei ce-mi fac mie strâmbatate; lupta împotriva celor ce se lupta cu mine;
\par 2 Apuca arma ?i pavaza ?i scoala-Te întru ajutorul meu;
\par 3 Scoate sabia ?i închide calea celor ce ma prigonesc; spune sufletului meu: "Mântuirea ta sunt Eu!"
\par 4 Sa fie ru?ina?i ?i înfrunta?i cei ce cauta sufletul meu; sa se întoarca înapoi ?i sa se ru?ineze cei ce gândesc rau de mine.
\par 5 Sa fie ca praful în fa?a vântului ?i îngerul Domnului sa-i necajeasca.
\par 6 Sa fie calea lor întuneric ?i alunecare ?i îngerul Domnului sa-i prigoneasca.
\par 7 Ca în zadar au ascuns de mine groapa la?ului lor, în de?ert au ocarât sufletul meu.
\par 8 Sa vina asupra lor la?ul pe care nu-l cunosc ?i cursa pe care au ascuns-o sa-i prinda pe ei; ?i chiar în la?ul lor sa cada.
\par 9 Iar sufletul meu sa se bucure de Domnul, sa se veseleasca de mântuirea lui.
\par 10 Toate oasele mele vor zice: Doamne, cine este asemenea ?ie? Cel ce izbave?te pe sarac din mâna celor mai tari decât el ?i pe sarac ?i pe sarman de cei ce-l rapesc pe el.
\par 11 S-au sculat martori nedrep?i ?i de cele ce nu ?tiam m-au întrebat.
\par 12 Rasplatit-au mie rele pentru bune ?i au vlaguit sufletul meu.
\par 13 Iar eu, când ma suparau ei, m-am îmbracat cu sac ?i am smerit cu post sufletul meu ?i rugaciunea mea în sânul meu se va întoarce.
\par 14 Ca ?i cu un vecin, ca ?i cu un frate al nostru a?a de bine m-am purtat; ca ?i cum a? fi jelit ?i m-a? fi întristat, a?a m-am smerit.
\par 15 Dar împotriva mea s-au veselit ?i s-au adunat; adunatu-s-au împotriva mea cu batai ?i n-am ?tiut; risipi?i au fost ?i nu s-au cait.
\par 16 M-au ispitit, cu batjocura m-au batjocorit, au scrâ?nit împotriva mea cu din?ii lor.
\par 17 Doamne, când vei vedea? Întoarce sufletul meu de la fapta lor cea rea, de la lei, via?a mea.
\par 18 Lauda-Te-voi, Doamne, în adunare mare, întru popor numeros Te voi lauda.
\par 19 Sa nu se bucure de. mine cei ce ma du?manesc pe nedrept, cei ce ma urasc în zadar ?i fac semn cu ochii.
\par 20 Ca mie de pace îmi graiau ?i asupra mea vicle?uguri gândeau.
\par 21 Largitu-?i-au împotriva mea gura lor; zis-au: "Bine, bine, vazut-au ochii no?tri".
\par 22 Vazut-ai, Doamne, sa nu taci; Doamne, nu Te departa de la mine.
\par 23 Scoala-Te, Doamne ?i ia aminte spre judecata mea, Dumnezeul meu ?i Domnul meu, spre pricina mea.
\par 24 Judeca-ma dupa dreptatea Ta, Doamne Dumnezeul meu ?i sa nu se bucure de mine.
\par 25 Sa nu zica întru inimile lor: "Bine, bine, sufletului nostru", nici sa zica: "L-am înghi?it pe el".
\par 26 Sa fie ru?ina?i ?i înfrunta?i deodata cei ce se bucura de necazurile mele; sa se îmbrace cu ru?ine ?i ocara cei ce se lauda împotriva mea.
\par 27 Sa se bucure ?i sa se veseleasca cei ce voiesc dreptatea mea ?i sa spuna pururea: "Slavit sa fie Domnul, Cel ce voie?te pacea robului Sau!"
\par 28 ?i limba mea va grai dreptatea Ta, în toata ziua, lauda Ta.

\chapter{36}

\par 1 (Un psalm al lui David, robul Domnului; mai-marelui cântare?ilor.) Necredin?a calcatorului de lege spune inimii mele, ca nu este într-însul frica de Dumnezeu.
\par 2 El singur se amage?te în ochii sai, când zice ca ar fi urmarind faradelegea ?i ar fi urând-o.
\par 3 Graiurile gurii lui faradelege ?i vicle?ug; n-a vrut sa priceapa ca sa faca bine.
\par 4 Faradelege a gândit în a?ternutul sau, în toata calea cea buna n-a stat ?i rautatea n-a urât.
\par 5 Doamne, în cer este mila Ta ?i adevarul Tau pâna la nori.
\par 6 Dreptatea Ta ca mun?ii lui Dumnezeu, judeca?ile Tale adânc mare; oameni ?i dobitoace vei izbavi Doamne.
\par 7 Ca ai înmul?it mila Ta, Dumnezeule, iar fiii oamenilor în umbra aripilor Tale vor nadajdui.
\par 8 Satura-se-vor din grasimea casei Tale ?i din izvorul desfatarii Tale îi vei adapa pe ei.
\par 9 Ca la Tine este izvorul vie?ii, întru lumina Ta vom vedea lumina.
\par 10 Tinde mila Ta celor ce Te cunosc pe Tine ?i dreptatea Ta celor drep?i la inima.
\par 11 Sa nu vina peste mine picior de mândrie ?i mâna pacato?ilor sa nu ma clatine.
\par 12 Acolo au cazut to?i cei ce lucreaza faradelegea; izgoni?i au fost ?i nu vor putea sa stea.

\chapter{37}

\par 1 (Un psalm al lui David.) Nu râvni la cei ce viclenesc, nici nu urma pe cei ce fac faradelegea.
\par 2 Caci ca iarba curând se vor usca ?i ca verdea?a ierbii degrab se vor trece.
\par 3 Nadajduie?te în Domnul ?i fa bunatate ?i locuie?te pamântul ?i hrane?te-te cu boga?ia lui.
\par 4 Desfateaza-te în Domnul ?i î?i va împlini ?ie cererile inimii tale.
\par 5 Descopera Domnului calea ta ?i nadajduie?te în El ?i El va împlini.
\par 6 ?i va scoate ca lumina dreptatea ta ?i judecata ca lumina de amiaza.
\par 7 Supune-te Domnului ?i roaga-L pe El; nu râvni dupa cel ce spore?te în calea sa, dupa omul care face nelegiuirea.
\par 8 Parase?te mânia ?i lasa iu?imea; nu cauta sa viclene?ti.
\par 9 Ca cei ce viclenesc de tot vor pieri; iar cei ce a?teapta pe Domnul vor mo?teni pamântul.
\par 10 ?i înca pu?in ?i nu va mai fi pacatosul ?i vei cauta locul lui ?i nu-l vei afla.
\par 11 Iar cei blânzi vor mo?teni pamântul ?i se vor desfata de mul?imea pacii.
\par 12 Pândi-va pacatosul pe cel drept ?i va scrâ?ni asupra lui, cu din?ii sai.
\par 13 Iar Domnul va râde de el, ca mai înainte vede ca va veni ziua lui.
\par 14 Sabia au scos pacato?ii, întins-au arcul lor ca sa doboare pe sarac ?i pe sarman, ca sa junghie pe cei drep?i la inima.
\par 15 Sabia lor sa intre în inima lor ?i arcurile lor sa se frânga.
\par 16 Mai bun este pu?inul celui drept, decât boga?ia multa a pacato?ilor.
\par 17 Ca bra?ele pacato?ilor se vor zdrobi, iar Domnul întare?te pe cei drep?i.
\par 18 Cunoa?te Domnul caile celor fara prihana ?i mo?tenirea lor în veac va fi.
\par 19 Nu se vor ru?ina în vremea cea rea ?i în zilele de foamete se vor satura.
\par 20 Ca pacato?ii vor pieri, iar vrajma?ii Domnului, îndata ce s-au marit ?i s-au înal?at, s-au stins, ca fumul au pierit.
\par 21 Împrumuta pacatosul ?i nu da înapoi, iar dreptul se îndura ?i da.
\par 22 Ca cei ce-L binecuvânteaza pe El vor mo?teni pamântul, iar cei ce-L blesteama pe El, de tot vor pieri.
\par 23 De la Domnul pa?ii omului se îndrepteaza ?i calea lui o va voi foarte.
\par 24 Când va cadea, nu se va zdruncina, ca Domnul întare?te mâna lui.
\par 25 Tânar am fost ?i am îmbatrânit ?i n-am vazut pe cel drept parasit, nici semin?ia lui cerând pâine;
\par 26 Toata ziua dreptul miluie?te ?i împrumuta ?i semin?ia lui binecuvântata va fi.
\par 27 Fere?te-te de rau ?i fa binele ?i vei trai în veacul veacului.
\par 28 Ca Domnul iube?te judecata ?i nu va parasi pe cei cuvio?i ai Sai; în veac vor fi pazi?i. Iar cei fara de lege vor fi izgoni?i ?i semin?ia necredincio?ilor va fi stârpita.
\par 29 Iar drep?ii vor mo?teni pamântul ?i vor locui în veacul veacului pe el.
\par 30 Gura dreptului va deprinde în?elepciunea ?i limba lui va grai judecata.
\par 31 Legea Dumnezeului sau în inima lui ?i nu se vor poticni pa?ii lui.
\par 32 Pânde?te pacatosul pe cel drept ?i cauta sa-l omoare pe el;
\par 33 Iar Domnul nu-l va lasa pe el, în mâinile lui, nici nu-l va osândi, când se va judeca cu el.
\par 34 A?teapta pe Domnul ?i paze?te calea Lui! ?i te va înva?a pe tine ca sa mo?tene?ti pamântul; când vor pieri pacato?ii vei vedea.
\par 35 Vazut-am pe cel necredincios falindu-se ?i înal?ându-se ca cedrii Libanului.
\par 36 ?i am trecut ?i iata nu era ?i l-am cautat pe el ?i nu s-a aflat locul lui.
\par 37 Paze?te nerautatea ?i cauta dreptatea, ca urma?i are omul facator de pace.
\par 38 Iar cei fara de lege vor pieri deodata ?i urma?ii necredincio?ilor vor fi stârpi?i.
\par 39 Iar mântuirea drep?ilor de la Domnul, ca aparatorul lor este în vreme de necaz.
\par 40 ?i-i va ajuta pe ei Domnul ?i-i va izbavi pe ei ?i-i va scoate pe ei din mâna pacato?ilor ?i-i va mântui pe ei ca au nadajduit în El.

\chapter{38}

\par 1 (Un psalm al lui David; pentru pomenirea sâmbetei.) Doamne, nu cu mânia Ta sa ma mustri pe mine, nici cu iu?imea Ta sa ma cer?i.
\par 2 Ca sage?ile Tale s-au înfipt în mine ?i ai întarit peste mine mâna Ta.
\par 3 Nu este vindecare în trupul meu de la fa?a mâniei Tale; nu este pace în oasele mele de la fa?a pacatelor mele.
\par 4 Ca faradelegile mele au covâr?it capul meu, ca o sarcina grea apasat-au peste mine.
\par 5 Împu?itu-s-au ?i au putrezit ranile mele, de la fa?a nebuniei mele.
\par 6 Chinuitu-m-am ?i m-am gârbovit pâna în sfâr?it, toata ziua mâhnindu-ma umblam.
\par 7 Ca ?alele mele s-au umplut de ocari ?i nu este vindecare în trupul meu.
\par 8 Necajitu-m-am ?i m-am smerit foarte; racnit-am din suspinarea inimii mele.
\par 9 Doamne, înaintea Ta este toata dorirea mea ?i suspinul meu de la Tine nu s-a ascuns.
\par 10 Inima mea s-a tulburat, parasitu-m-a taria mea ?i lumina ochilor mei ?i aceasta nu este cu mine.
\par 11 Prietenii mei ?i vecinii mei în preajma mea s-au apropiat ?i au ?ezut; ?i cei de aproape ai mei departe au stat.
\par 12 ?i se sileau cei ce cautau sufletul meu ?i cei ce cautau cele rele mie graiau de?ertaciuni ?i vicle?uguri toata ziua cugetau.
\par 13 Iar eu ca un surd nu auzeam ?i ca un mut ce nu-?i deschide gura sa.
\par 14 ?i m-am facut ca un om ce nu aude ?i nu are în gura lui mustrari.
\par 15 Ca spre Tine, Doamne, am nadajduit; Tu ma vei auzi, Doamne, Dumnezeul meu,
\par 16 Ca am zis, ca nu cumva sa se bucure de mine vrajma?ii mei; ?i când s-au clatinat picioarele mele, împotriva mea s-au seme?it.
\par 17 Ca eu spre batai gata sunt ?i durerea mea înaintea mea este pururea.
\par 18 Ca faradelegea mea eu o voi vesti  ?i ma voi îngriji pentru pacatul meu;
\par 19 Iar vrajma?ii mei traiesc ?i s-au întarit mai mult decât mine ?i s-au înmul?it cei ce ma urasc pe nedrept.
\par 20 Cei ce îmi rasplatesc rele pentru bune, ma defaimau, ca urmam bunatatea.
\par 21 Nu ma lasa, Doamne Dumnezeul meu, nu Te departa de la mine;
\par 22 Ia aminte spre ajutorul meu, Doamne al mântuirii mele.

\chapter{39}

\par 1 (Întru sfâr?it, lui Iditum, o cântare a lui David.) Zis-am: "Pazi-voi caile mele, ca sa nu pacatuiesc eu cu limba mea; pus-am gurii mele paza, când a stat pacatosul împotriva mea".
\par 2 Amu?it-am ?i m-am smerit ?i nici de bine n-am grait ?i durerea mea s-a înnoit.
\par 3 Înfierbântatu-s-a inima mea înauntrul meu ?i în cugetul meu se va aprinde foc.
\par 4 Grait-am cu limba mea: "Fa-mi cunoscut, Doamne, sfâr?itul meu, ?i numarul zilelor mele care este, ca sa ?tiu ce-mi lipse?te".
\par 5 Iata, cu palma ai masurat zilele mele ?i statul meu ca nimic înaintea Ta. Dar toate sunt de?ertaciuni; tot omul ce viaza.
\par 6 De?i ca o umbra trece omul, dar în zadar se tulbura. Strânge comori ?i nu ?tie cui le aduna pe ele.
\par 7 ?i acum cine este rabdarea mea? Oare, nu Domnul? ?i statul meu de la Tine este.
\par 8 De toate faradelegile mele izbave?te-ma; ocara celui fara de minte nu ma da.
\par 9 Amu?it-am ?i n-am deschis gura mea, ca Tu e?ti Cel ce m-ai facut pe mine.
\par 10 Departeaza de la mine bataile Tale. De taria mâinii Tale, eu m-am sfâr?it.
\par 11 Cu mustrari pentru faradelege ai pedepsit pe om ?i ai sub?iat ca pânza de paianjen sufletul sau; dar în de?ert se tulbura tot pamânteanul.
\par 12 Auzi rugaciunea mea, Doamne, ?i cererea mea ascult-o; lacrimile mele sa nu le treci, caci strain sunt eu la Tine ?i strain ca to?i parin?ii mei.
\par 13 Lasa-ma ca sa ma odihnesc, mai înainte de a ma duce ?i de a nu mai fi.

\chapter{40}

\par 1 (Întru sfâr?it, un psalm al lui David.) A?teptând am a?teptat pe Domnul ?i S-a plecat spre mine.
\par 2 A auzit rugaciunea mea. M-a scos din groapa ticalo?iei ?i din tina noroiului ?i a pus pe piatra picioarele mele ?i a îndreptat pa?ii mei.
\par 3 ?i a pus în gura mea cântare noua, cântare Dumnezeului nostru; vedea-vor mul?i ?i se vor teme ?i vor nadajdui în Domnul.
\par 4 Fericit barbatul, a carui nadejde este numele Domnului ?i n-a privit la de?ertaciuni ?i la nebunii mincinoase.
\par 5 Multe ai facut Tu, Doamne, Dumnezeul meu, minunile Tale, ?i nu este cine sa se asemene gândurilor Tale; vestit-am ?i am grait: înmul?itu-s-au peste numar.
\par 6 Jertfa ?i prinos n-ai voit, dar trup mi-ai întocmit; ardere de tot ?i jertfa pentru pacat n-ai cerut.
\par 7 Atunci am zis: "Iata vin! În capul car?ii este scris despre mine.
\par 8 Ca sa fac voia Ta, Dumnezeul meu, am voit ?i legea Ta înauntru inimii mele".
\par 9 Bine am vestit dreptate în adunare mare; iata buzele mele nu le voi opri; Doamne, Tu ai cunoscut.
\par 10 Dreptatea Ta n-am ascuns-o în inima mea, adevarul Tau ?i mântuirea Ta am spus. N-am ascuns mila Ta ?i adevarul Tau în adunare mare.
\par 11 Iar Tu, Doamne, sa nu departezi îndurarile Tale de la mine, mila Ta ?i adevarul Tau pururea sa ma sprijineasca.
\par 12 Ca m-au împresurat rele, carora nu este numar; ajunsu-m-au faradelegile mele ?i n-am putut sa vad; înmul?itu-s-au mai mult decât perii capului meu ?i inima mea m-a parasit.
\par 13 Binevoie?te, Doamne, ca sa ma izbave?ti; Doamne, spre ajutorul meu ia aminte.
\par 14 Sa fie ru?ina?i ?i înfrunta?i deodata cei ce cauta sa ia sufletul meu. Sa se întoarca înapoi ?i sa se ru?ineze cei ce-mi voiesc mie rele;
\par 15 Sa fie ru?ina?i îndata cei ce-mi zic mie: "Bine, bine".
\par 16 Sa se bucure ?i sa se veseleasca de Tine, to?i cei ce Te cauta pe Tine, Doamne, ?i sa zica pururea cei ce iubesc mântuirea Ta: "Slavit sa fie Domnul!"
\par 17 Iar eu sarac sunt ?i sarman; Domnul se va îngriji de mine. Ajutorul meu ?i aparatorul meu e?ti Tu; Dumnezeul meu nu zabovi.

\chapter{41}

\par 1 (Întru sfâr?it, un psalm al lui David.) Fericit cel care cauta la sarac ?i la sarman; în ziua cea rea îl va izbavi pe el Domnul.
\par 2 Domnul sa-l pazeasca pe el ?i sa-l vieze ?i sa-l fericeasca pe pamânt ?i sa nu-l dea în mâinile vrajma?ilor lui.
\par 3 Domnul sa-l ajute pe el pe patul durerii lui; în a?ternutul bolii lui sa-l întareasca pe el.
\par 4 Eu am zis: "Doamne, miluie?te-ma; vindeca sufletul meu, ca am gre?it ?ie".
\par 5 Vrajma?ii mei m-au grait de rau zicând: "Când va muri ?i va pieri numele lui?"
\par 6 Iar de venea cineva sa ma vada, minciuni graia; inima lui aduna faradelege sie?i, ie?ea afara ?i graia.
\par 7 Împreuna împotriva mea ?opteau to?i vrajma?ii mei; împotriva mea gândeau de mine rele.
\par 8 Cuvânt nelegiuit spuneau împotriva mea, zicând: "Nu zace, oare? Nu se va mai scula!"
\par 9 Chiar omul cu care eram în pace, în care am nadajduit, care a mâncat pâinea mea, a ridicat împotriva mea calcâiul.
\par 10 Iar Tu, Doamne, miluie?te-ma ?i ma scoala ?i voi rasplati lor.
\par 11 Întru aceasta am cunoscut ca m-ai voit, ca nu se va bucura vrajma?ul meu de mine.
\par 12 Iar pe mine pentru nerautatea mea m-ai sprijinit ?i m-ai întarit înaintea Ta, în veac.
\par 13 Binecuvântat este Domnul Dumnezeul lui Israel din veac ?i pâna în veac. Amin. Amin.

\chapter{42}

\par 1 (Un psalm al lui David, spre în?elep?ire fiilor lui Core.) În ce chip dore?te cerbul izvoarele apelor, a?a Te dore?te sufletul meu pe Tine, Dumnezeule.
\par 2 Însetat-a sufletul meu de Dumnezeul cel viu; când voi veni ?i ma voi arata fe?ei lui Dumnezeu?
\par 3 Facutu-mi-s-au lacrimile mele pâine ziua ?i noaptea, când mi se zicea mie în toate zilele: "Unde este Dumnezeul tau?"
\par 4 De acestea mi-am adus aminte cu revarsare de inima, când treceam cu mul?ime mare spre casa lui Dumnezeu, în glas de bucurie ?i de lauda ?i în sunet de sarbatoare.
\par 5 Pentru ce e?ti mâhnit, suflete al meu, ?i pentru ce ma tulburi? Nadajduie?te în Dumnezeu, ca-L voi lauda pe El; mântuirea fe?ei mele este Dumnezeul meu.
\par 6 În mine sufletul meu s-a tulburat; pentru aceasta îmi voi aduce aminte de Tine, din pamântul Iordanului ?i al Ermonului, din muntele cel mic.
\par 7 Adânc pe adânc cheama în glasul caderilor apelor Tale. Toate talazurile ?i valurile Tale peste mine au trecut.
\par 8 Ziua va porunci Domnul milei Sale, iar noaptea cântare Lui de la mine.
\par 9 Rugaciunea Dumnezeului vie?ii mele, spune-voi lui Dumnezeu: "Sprijinitorul meu e?ti Tu, pentru ce m-ai uitat? Pentru ce umblu mâhnit când ma necaje?te vrajma?ul meu?"
\par 10 Când se sfarâmau oasele mele ma ocarau asupritorii mei. Când îmi ziceau mie în toate zilele: "Unde este Dumnezeul tau?"
\par 11 Pentru ce e?ti mâhnit, suflete al meu, ?i pentru ce ma tulburi? Nadajduie?te în Dumnezeu, ca-L voi lauda pe El; mântuirea fe?ei mele este Dumnezeul meu.

\chapter{43}

\par 1 (Un psalm al lui David, nescris deasupra la evrei.) Judeca-ma, Dumnezeule, ?i apara dreptatea mea de neamul necuvios, de omul nedrept ?i viclean, ?i izbave?te-ma.
\par 2 Ca Tu e?ti, Dumnezeule, întarirea mea; pentru ce m-ai lepadat? Pentru ce umblu mâhnit când ma necaje?te vrajma?ul meu?
\par 3 Trimite lumina Ta ?i adevarul Tau; acestea m-au pova?uit ?i m-au condus la muntele cel sfânt al Tau ?i la loca?urile Tale.
\par 4 ?i voi intra la jertfelnicul lui Dumnezeu, la Dumnezeul Cel ce vesele?te tinere?ile mele; lauda-Te-voi în alauta, Dumnezeule, Dumnezeul meu.
\par 5 Pentru ce e?ti mâhnit, suflete al meu, ?i pentru ce ma tulburi? Nadajduie?te în Dumnezeu ca-L voi lauda pe El; mântuirea fe?ei mele este Dumnezeul meu.

\chapter{44}

\par 1 (Întru sfâr?it, fiilor lui Core, spre în?elegere.) Dumnezeule, cu urechile noastre am auzit, parin?ii no?tri ne-au spus noua lucrul pe care l-ai facut în zilele lor, în zilele cele de demult.
\par 2 Mâna Ta popoare a nimicit, iar pe parin?i i-ai sadit; batut-ai popoare, iar pe ei i-ai înmul?it.
\par 3 Ca nu cu sabia lor au mo?tenit pamântul ?i bra?ul lor nu i-a izbavit pe ei, ci dreapta Ta ?i bra?ul Tau ?i luminarea fe?ei Tale, ca bine ai voit întru ei.
\par 4 Tu e?ti Însu?i Împaratul meu ?i Dumnezeul meu, Cel ce porunce?ti mântuirea lui Iacob;
\par 5 Cu Tine pe vrajma?ii no?tri îi vom lovi ?i cu numele Tau vom nimici pe cei ce se scoala asupra noastra.
\par 6 Pentru ca nu în arcul meu voi nadajdui ?i sabia mea nu ma va mântui.
\par 7 Ca ne-ai izbavit pe noi de cei ce ne necajesc pe noi ?i pe cei ce ne urasc pe noi i-ai ru?inat.
\par 8 Cu Dumnezeu ne vom lauda toata ziua ?i numele Tau îl vom lauda în veac.
\par 9 Iar acum ne-ai lepadat ?i ne-ai ru?inat pe noi ?i nu vei ie?i cu o?tirile noastre;
\par 10 Întorsu-ne-ai pe noi înapoi de la du?manii no?tri ?i cei ce ne urasc pe noi ne-au jefuit.
\par 11 Datu-ne-ai pe noi ca oi de mâncare ?i întru neamuri ne-ai risipit;
\par 12 Vândut-ai pe poporul Tau fara de pre? ?i nu l-ai pre?uit când l-ai vândut.
\par 13 Pusu-ne-ai pe noi ocara vecinilor no?tri, batjocura ?i râs celor dimprejurul nostru;
\par 14 Pusu-ne-ai pe noi pilda catre neamuri, clatinare de cap între popoare.
\par 15 Toata ziua înfruntarea mea înaintea mea este ?i ru?inea obrazului meu m-a acoperit,
\par 16 De catre glasul celui ce ocara?te ?i clevete?te, de catre fa?a vrajma?ului ?i prigonitorului.
\par 17 Acestea toate au venit peste noi ?i nu Te-am uitat ?i n-am calcat legamântul Tau
\par 18 ?i nu s-a dat înapoi inima noastra; iar pa?ii no?tri nu s-au abatut de la calea Ta,
\par 19 Ca ne-ai smerit pe noi în loc de durere ?i ne-a acoperit pe noi umbra mor?ii.
\par 20 De am fi uitat numele Dumnezeului nostru ?i am fi întins mâinile noastre spre dumnezeu strain,
\par 21 Oare, Dumnezeu n-ar fi cercetat acestea? Ca El ?tie ascunzi?urile inimii.
\par 22 Ca pentru Tine suntem uci?i toata ziua, socoti?i am fost ca ni?te oi de junghiere.
\par 23 De?teapta-Te, pentru ce dormi, Doamne? Scoala-Te ?i nu ne lepada pâna în sfâr?it.
\par 24 Pentru ce întorci fa?a Ta? Ui?i de saracia noastra ?i de necazul nostru?
\par 25 Ca s-a plecat în ?arâna sufletul nostru, lipitu-s-a de pamânt pântecele nostru.
\par 26 Scoala-Te, Doamne, ajuta-ne noua ?i ne izbave?te pe noi, pentru numele Tau.

\chapter{45}

\par 1 (Un psalm al lui David, pentru cei ce se vor schimba. Fiilor lui Core, spre în?elegere, cântare pentru cel iubit.) Cuvânt bun raspuns-a inima mea; grai-voi cântarea mea Împaratului. Limba mea este trestie de scriitor ce scrie iscusit.
\par 2 Împodobit e?ti cu frumuse?ea mai mult decât fiii oamenilor; revarsatu-s-a har pe buzele tale. Pentru aceasta te-a binecuvântat pe tine Dumnezeu, în veac.
\par 3 Încinge-te cu sabia ta peste coapsa ta, puternice,
\par 4 Cu frumuse?ea ta ?i cu stralucirea ta. Încordeaza-?i arcul, propa?e?te ?i împara?e?te, pentru adevar, blânde?e ?i dreptate, ?i te va pova?ui minunat dreapta ta.
\par 5 Sage?ile tale ascu?ite sunt puternice în inima du?manilor împaratului; popoarele sub tine vor cadea.
\par 6 Scaunul Tau, Dumnezeule, în veacul veacului, toiag de dreptate toiagul împara?iei Tale.
\par 7 Iubit-ai dreptatea ?i ai urât faradelegea; pentru aceasta Te-a uns pe Tine, Dumnezeul Tau, cu untdelemnul bucuriei, mai mult decât pe parta?ii Tai.
\par 8 Smirna ?i aloea îmbalsameaza ve?mintele Tale; din palate de filde? cântari de alauta Te veselesc; fiice de împara?i întru cinstea Ta;
\par 9 Statut-a împarateasa de-a dreapta Ta, îmbracata în haina aurita ?i prea înfrumuse?ata.
\par 10 Asculta fiica ?i vezi ?i pleaca urechea ta ?i uita poporul tau ?i casa parintelui tau,
\par 11 Ca a poftit Împaratul frumuse?ea ta, ca El este Domnul tau.
\par 12 ?i se vor închina Lui fiicele Tirului cu daruri, fe?ei Tale se vor ruga mai-marii poporului.
\par 13 Toata slava fiicei Împaratului este înauntru, îmbracata cu ?esaturi de aur ?i prea înfrumuse?ata.
\par 14 Aduce-se-vor Împaratului fecioare în urma ei, prietenele ei se vor aduce ?ie.
\par 15 Aduce-se-vor întru veselie ?i bucurie, aduce-se-vor în palatul Împaratului.
\par 16 În locul parin?ilor tai s-au nascut ?ie fii; pune-i-vei pe ei capetenii peste tot pamântul.
\par 17 Pomeni-vor numele tau în tot neamul; pentru aceasta popoarele te vor lauda în veac ?i în veacul veacului.

\chapter{46}

\par 1 (Un psalm al lui David, fiilor lui Core, pentru cele ascunse.) Dumnezeu este scaparea ?i puterea noastra, ajutor întru necazurile ce ne împresoara.
\par 2 Pentru aceasta nu ne vom teme când se va cutremura pamântul ?i se vor muta mun?ii în inima marilor.
\par 3 Venit-au ?i s-au tulburat apele lor, cutremuratu-s-au mun?ii de taria lui.
\par 4 Apele râurilor veselesc cetatea lui Dumnezeu; Cel Preaînalt a sfin?it loca?ul Lui.
\par 5 Dumnezeu este în mijlocul ceta?ii, nu se va clatina; o va ajuta Dumnezeu dis-de-diminea?a.
\par 6 Tulburatu-s-au neamurile, plecatu-s-au împara?iile; dat-a Cel Preaînalt glasul Lui, cutremuratu-s-a pamântul.
\par 7 Domnul puterilor cu noi, sprijinitorul nostru, Dumnezeul lui Iacob.
\par 8 Veni?i ?i vede?i lucrurile lui Dumnezeu, minunile pe care le-a pus Domnul pe pamânt.
\par 9 Pune-va capat razboaielor pâna la marginile pamântului, arcul va sfarâma ?i va frânge arma, iar pavezele în foc le va arde.
\par 10 Opri?i-va ?i cunoa?te?i ca Eu sunt Dumnezeu, înal?a-Ma-voi pe pamânt.
\par 11 Domnul puterilor cu noi, sprijinitorul nostru, Dumnezeul lui Iacob.

\chapter{47}

\par 1 (Mai-marelui cântare?ilor; un psalm pentru fiii lui Core.) Toate popoarele bate?i din palme, striga?i lui Dumnezeu cu glas de bucurie.
\par 2 Ca Domnul este Preaînalt, înfrico?ator, Împarat mare peste tot pamântul.
\par 3 Supusu-ne-a noua popoare ?i neamuri sub picioarele noastre;
\par 4 Alesu-ne-a noua mo?tenirea Sa, frumuse?ea lui Iacob, pe care a iubit-o.
\par 5 Suitu-S-a Dumnezeu întru strigare, Domnul în glas de trâmbi?a.
\par 6 Cânta?i Dumnezeului nostru, cânta?i; cânta?i Împaratului nostru, cânta?i.
\par 7 Ca Împarat a tot pamântul este Dumnezeu; cânta?i cu în?elegere.
\par 8 Împara?it-a Dumnezeu peste neamuri, Dumnezeu ?ade pe tronul cel sfânt al Sau.
\par 9 Mai-marii popoarelor s-au adunat cu poporul Dumnezeului lui Avraam, ca ai lui Dumnezeu sunt puternicii pamântului; El S-a înal?at foarte.

\chapter{48}

\par 1 (O cântare de psalm, pentru fiii lui Core, pentru a doua zi a sâmbetei.) Mare este Domnul ?i laudat foarte în cetatea Dumnezeului nostru, în muntele cel sfânt al Lui;
\par 2 Bine întemeiata spre bucuria întregului pamânt. Muntele Sionului, coastele de miazanoapte, cetatea Împaratului Celui mare.
\par 3 Dumnezeu în palatele ei se cunoa?te, când o apara pe ea.
\par 4 Ca iata împara?ii s-au adunat, strânsu-s-au împreuna.
\par 5 Ace?tia vazând-o a?a, s-au minunat, s-au tulburat, s-au cutremurat;
\par 6 Cutremur i-a cuprins pe ei acolo; dureri ca ale celei ce na?te.
\par 7 Cu vânt puternic va sfarâma corabiile Tarsisului.
\par 8 Precum am auzit, a?a am ?i vazut, în cetatea Domnului puterilor, în cetatea Dumnezeului nostru.
\par 9 Dumnezeu a întemeiat-o pe ea în veac. Primit-am, Dumnezeule, mila Ta, în mijlocul loca?ului Tau.
\par 10 Dupa numele Tau, Dumnezeule, a?a ?i lauda Ta peste marginile pamântului; dreapta Ta este plina de dreptate.
\par 11 Sa se veseleasca Muntele Sionului, sa se bucure fiicele lui Iuda pentru judeca?ile Tale, Doamne.
\par 12 Înconjura?i Sionul ?i-l cuprinde?i pe el, povesti?i despre turnurile lui.
\par 13 Pune?i-va inimile voastre întru puterea lui ?i strabate?i palatele lui, ca sa povesti?i neamului ce vine,
\par 14 Ca Acesta este Dumnezeu, Dumnezeul nostru în veac ?i în veacul veacului; El ne va pa?te pe noi în veci.

\chapter{49}

\par 1 (Mai-marelui cântare?ilor; un psalm pentru fiii lui Core.) Auzi?i acestea toate neamurile, asculta?i to?i cei ce locui?i în lume:
\par 2 Pamântenii ?i fiii oamenilor, împreuna bogatul ?i saracul.
\par 3 Gura mea va grai în?elepciune ?i cugetul inimii mele pricepere.
\par 4 Pleca-voi spre pilda urechea mea, tâlcui-voi în sunet de psaltire gândul meu.
\par 5 Pentru ce sa ma tem în ziua cea rea, când ma va înconjura faradelegea vrajma?ilor mei?
\par 6 Ei se încred în puterea lor ?i cu mul?imea boga?iei lor se lauda.
\par 7 Nimeni însa nu poate sa scape de la moarte, nici sa plateasca lui Dumnezeu pre? de rascumparare,
\par 8 Ca rascumpararea sufletului e prea scumpa ?i niciodata nu se va putea face,
\par 9 Ca sa ramâna cineva pe totdeauna viu ?i sa nu vada niciodata moartea.
\par 10 Fiecare vede ca în?elep?ii mor, cum mor ?i cei neîn?elep?i ?i nebunii, ?i lasa altora boga?ia lor.
\par 11 Mormântul lor va fi casa lor în veac, loca?urile lor din neam în neam, de?i numit-au cu numele lor pamânturile lor.
\par 12 ?i omul, în cinste fiind, n-a priceput; alaturatu-s-a dobitoacelor celor fara de minte ?i s-a asemanat lor.
\par 13 Aceasta cale le este sminteala lor ?i celor ce vor gasi de bune spusele lor.
\par 14 Ca ni?te oi în iad sunt pu?i, moartea îi va pa?te pe ei. ?i-i vor stapâni pe ei cei drep?i ?i ajutorul ce-l nadajduiau din slava lor, se va învechi în iad.
\par 15 Dar Dumnezeu va izbavi sufletul meu din mâna iadului, când ma va apuca.
\par 16 Sa nu te temi când se îmboga?e?te omul ?i când se înmul?e?te slava casei lui.
\par 17 Ca la moarte el nu va lua nimic, nici nu se va coborî cu el slava lui.
\par 18 Chiar daca sufletul lui se va binecuvânta în via?a lui ?i te va lauda când îi vei face bine,
\par 19 Totu?i intra-va pâna la neamul parin?ilor lui ?i în veac nu va vedea lumina.
\par 20 Omul în cinste fiind n-a priceput; alaturatu-s-a dobitoacelor celor fara de minte ?i s-a asemanat lor.

\chapter{50}

\par 1 (Un psalm al lui Asaf.) Dumnezeul dumnezeilor, Domnul a grait ?i a chemat pamântul,
\par 2 De la rasaritul soarelui pâna la apus. Din Sion este stralucirea frumuse?ii Lui.
\par 3 Dumnezeu stralucit va veni, Dumnezeul nostru, ?i nu va tacea. Foc înaintea Lui va arde ?i împrejurul Lui vifor mare.
\par 4 Chema-va cerul de sus ?i pamântul, ca sa judece pe poporul Sau.
\par 5 Aduna?i-I Lui pe cuvio?ii Lui, pe cei ce au facut legamânt cu El pentru jertfe.
\par 6 ?i vor vesti cerurile dreptatea Lui, ca Dumnezeu judecator este.
\par 7 "Asculta, poporul Meu ?i-?i voi grai ?ie, Israele!... ?i voi marturisi ?ie: Dumnezeu, Dumnezeul tau sunt Eu.
\par 8 Nu pentru jertfele tale te voi mustra, iar arderile de tot ale tale înaintea Mea sunt pururea.
\par 9 Nu voi primi din casa ta vi?ei, nici din turmele tale ?api,
\par 10 Ca ale Mele sunt toate fiarele câmpului, dobitoacele din mun?i ?i boii.
\par 11 Cunoscut-am toate pasarile cerului ?i frumuse?ea ?arinii cu Mine este.
\par 12 De voi flamânzi, nu-?i voi spune ?ie, ca a Mea este lumea ?i plinirea ei.
\par 13 Oare, voi mânca carne de taur, sau sânge de ?api voi bea?
\par 14 Jertfe?te lui Dumnezeu jertfa de lauda ?i împline?te Celui Preaînalt fagaduin?ele tale.
\par 15 ?i Ma cheama pe Mine în ziua necazului ?i te voi izbavi ?i Ma vei preaslavi".
\par 16 Iar pacatosului i-a zis Dumnezeu: "Pentru ce tu istorise?ti drepta?ile Mele ?i iei legamântul Meu în gura ta?
\par 17 Tu ai urât înva?atura ?i ai lepadat cuvintele Mele înapoia ta.
\par 18 De vedeai furul, alergai cu el ?i cu cel desfrânat partea ta puneai.
\par 19 Gura ta a înmul?it rautate ?i limba ta a împletit vicle?ug.
\par 20 ?ezând împotriva fratelui tau cleveteai ?i împotriva fiului maicii tale ai pus sminteala.
\par 21 Acestea ai facut ?i am tacut, ai cugetat faradelegea, ca voi fi asemenea ?ie; mustra-te-voi ?i voi pune înaintea fe?ei tale pacatele tale.
\par 22 În?elege?i dar acestea cei ce uita?i pe Dumnezeu, ca nu cumva sa va rapeasca ?i sa nu fie cel ce izbave?te.
\par 23 Jertfa de lauda Ma va slavi ?i acolo este calea în care voi arata lui mântuirea Mea".

\chapter{51}

\par 1 (Mai-marelui cântare?ilor; un psalm al lui David, când a venit profetul Natan, ca sa-l mustre pentru femeia lui Urie.) Miluie?te-ma, Dumnezeule, dupa mare mila Ta ?i dupa mul?imea îndurarilor Tale, ?terge faradelegea mea.
\par 2 Mai vârtos ma spala de faradelegea mea ?i de pacatul meu ma cura?e?te.
\par 3 Ca faradelegea mea eu o cunosc ?i pacatul meu înaintea mea este pururea.
\par 4 ?ie unuia am gre?it ?i rau înaintea Ta am facut, a?a încât drept e?ti Tu întru cuvintele Tale ?i biruitor când vei judeca Tu.
\par 5 Ca iata întru faradelegi m-am zamislit ?i în pacate m-a nascut maica mea.
\par 6 Ca iata adevarul ai iubit; cele nearatate ?i cele ascunse ale în?elepciunii Tale, mi-ai aratat mie.
\par 7 Stropi-ma-vei cu isop ?i ma voi cura?i; spala-ma-vei ?i mai vârtos decât zapada ma voi albi.
\par 8 Auzului meu vei da bucurie ?i veselie; bucura-se-vor oasele mele cele smerite.
\par 9 Întoarce fa?a Ta de la pacatele mele ?i toate faradelegile mele ?terge-le.
\par 10 Inima curata zide?te întru mine, Dumnezeule ?i duh drept înnoie?te întru cele dinlauntru ale mele.
\par 11 Nu ma lepada de la fa?a Ta ?i Duhul Tau cel sfânt nu-l lua de la mine.
\par 12 Da-mi mie bucuria mântuirii Tale ?i cu duh stapânitor ma întare?te.
\par 13 Înva?a-voi pe cei fara de lege caile Tale ?i cei necredincio?i la Tine se vor întoarce.
\par 14 Izbave?te-ma de varsarea de sânge, Dumnezeule, Dumnezeul mântuirii mele; bucura-se-va limba mea de dreptatea Ta.
\par 15 Doamne, buzele mele vei deschide ?i gura mea va vesti lauda Ta.
\par 16 Ca de ai fi voit jertfa, ?i-a? fi dat; arderile de tot nu le vei binevoi.
\par 17 Jertfa lui Dumnezeu: duhul umilit; inima înfrânta ?i smerita Dumnezeu nu o va urgisi.
\par 18 Fa bine, Doamne, întru buna voirea Ta, Sionului, ?i sa se zideasca zidurile Ierusalimului.
\par 19 Atunci vei binevoi jertfa drepta?ii, prinosul ?i arderile de tot; atunci vor pune pe altarul Tau vi?ei.

\chapter{52}

\par 1 (Mai-marelui cântare?ilor; un psalm al lui David, când a venit Doig Idumeul ?i a vestit lui Saul ?i i-a zis: venit-a David în casa lui Abimelec.) Ce te fale?ti întru rautate, puternice?
\par 2 Faradelege toata ziua, nedreptate a vorbit limba ta; ca un brici ascu?it a facut vicle?ug.
\par 3 Iubit-ai rautatea mai mult decât bunatatea, nedreptatea mai mult decât a grai dreptatea.
\par 4 Iubit-ai toate cuvintele pierzarii, limba vicleana!
\par 5 Pentru aceasta Dumnezeu te va doborî pâna în sfâr?it, te va smulge ?i te va muta din loca?ul tau ?i radacina ta din pamântul celor vii.
\par 6 Vedea-vor drep?ii ?i se vor teme ?i de el vor râde ?i vor zice: "Iata omul care nu ?i-a pus pe Dumnezeu ajutorul lui,
\par 7 Ci a nadajduit în mul?imea boga?iei sale ?i s-a întarit întru de?ertaciunea sa".
\par 8 Dar eu, ca un maslin roditor în casa lui Dumnezeu, am nadajduit în mila lui Dumnezeu, în veac ?i în veacul veacului.
\par 9 Slavi-Te-voi în veac, ca ai facut aceasta ?i voi a?tepta numele Tau, ca bun este înaintea cuvio?ilor Tai.

\chapter{53}

\par 1 (Un psalm al lui David; mai-marelui cântare?ilor, pentru maelet, un instrument muzical; un psalm al priceperii.) Zis-a cel nebun întru inima sa: "Nu este Dumnezeu!" Stricatu-s-au ?i urâ?i s-au facut întru faradelegi. Nu este cel ce face bine.
\par 2 Domnul din cer a privit peste fiii oamenilor, sa vada de este cel ce în?elege, sau cel ce cauta pe Dumnezeu.
\par 3 To?i s-au abatut, împreuna netrebnici s-au facut; nu este cel ce face bine, nu este pâna la unul.
\par 4 Oare, nu vor cunoa?te, to?i cei ce lucreaza faradelegea? Cei ce manânca pe poporul Meu cum manânca pâinea,
\par 5 Pe Domnul nu L-au chemat. Acolo s-au temut de frica unde nu era frica, ca Dumnezeu a risipit oasele celor ce plac oamenilor; ru?inatu-s-au, ca Dumnezeu i-a urgisit pe ei.
\par 6 Cine va da din Sion mântuirea lui Israel? Când va întoarce Domnul pe cei robi?i ai poporului Sau, bucura-se-va Iacob ?i se va veseli Israel.

\chapter{54}

\par 1 (Un psalm al lui David; mai-marelui cântare?ilor, pentru instrumente cu coarde; un psalm al priceperii. Când au venit zefeii ?i au zis lui Saul: Iata, oare, nu s-a ascuns David la noi?) Dumnezeule, întru numele Tau mântuie?te-ma ?i întru puterea Ta ma judeca.
\par 2 Dumnezeule, auzi rugaciunea mea, ia aminte cuvintele gurii mele!
\par 3 Ca strainii s-au ridicat împotriva mea ?i cei tari au cautat sufletul meu ?i n-au pus pe Dumnezeu înaintea lor.
\par 4 Dar iata, Dumnezeu ajuta mie ?i Domnul este sprijinul sufletului meu.
\par 5 Întoarce-va cele rele vrajma?ilor mei; cu adevarul Tau îi vei pierde pe ei.
\par 6 De bunavoie voi jertfi ?ie; lauda-voi numele Tau, Doamne, ca este bun,
\par 7 Ca din tot necazul m-ai izbavit ?i spre vrajma?ii mei a privit ochiul meu.

\chapter{55}

\par 1 (Un psalm al lui David; mai-marelui cântare?ilor, pentru instrumente cu coarde; un psalm al priceperii.) Auzi, Dumnezeule, rugaciunea mea ?i nu trece cu vederea ruga mea.
\par 2 Ia aminte spre mine ?i ma asculta; mâhnitu-m-am întru nelini?tea mea ?i m-am tulburat de glasul vrajma?ului ?i de necazul pacatosului.
\par 3 Ca a abatut asupra mea faradelege ?i întru mânie m-a vrajma?it.
\par 4 Inima mea s-a tulburat întru mine ?i frica mor?ii a cazut peste mine;
\par 5 Teama ?i cutremur au venit asupra mea ?i m-a acoperit întunericul.
\par 6 ?i am zis: Cine-mi va da mie aripi ca de porumbel, ca sa zbor ?i sa ma odihnesc?
\par 7 Iata m-a? îndeparta fugind ?i m-a? sala?lui în pustiu.
\par 8 A?teptat-am pe Dumnezeu, Cel ce ma mântuie?te de pu?inatatea sufletului ?i de vifor.
\par 9 Nimice?te-i, Doamne ?i împarte limbile lor, ca am vazut faradelege ?i dezbinare în cetate.
\par 10 Ziua ?i noaptea o va înconjura pe ea peste zidurile ei; faradelege ?i osteneala în mijlocul ei ?i nedreptate;
\par 11 ?i n-a lipsit din uli?ele ei camata ?i vicle?ug.
\par 12 Ca de m-ar fi ocarât vrajma?ul, a? fi rabdat; ?i daca cel ce ma ura?te s-ar fi falit împotriva mea, m-a? fi ascuns de el.
\par 13 Iar tu, omule, asemenea mie, capetenia mea ?i cunoscutul meu,
\par 14 Care împreuna cu mine te-ai îndulcit de mâncari, în casa lui Dumnezeu am umblat în acela?i gând!
\par 15 Sa vina moartea peste ei ?i sa se coboare în iad de vii, caci vicle?ug este în loca?urile lor, în mijlocul lor.
\par 16 Iar eu, catre Dumnezeu am strigat, ?i Domnul m-a auzit pe mine.
\par 17 Seara ?i diminea?a ?i la amiezi spune-voi, voi vesti, ?i va auzi glasul meu.
\par 18 Izbavi-va cu pace sufletul meu de cei ce se apropie de mine, ca mul?i erau împotriva mea.
\par 19 Auzi-va Dumnezeu ?i-i va smeri pe ei, Cel ce este mai înainte de veci.
\par 20 Ca nu este în ei îndreptare ?i nu s-au temut de Dumnezeu. Întins-au mâinile împotriva alia?ilor lor, întinat-au legamântul Lui. Risipi?i au fost de mânia fe?ei Lui ?i s-au apropiat inimile lor;
\par 21 Muiatu-s-au cuvintele lor mai mult decât untdelemnul, dar ele sunt sage?i.
\par 22 Arunca spre Domnul grija ta ?i El te va hrani; nu va da în veac clatinare dreptului.
\par 23 Iar Tu, Dumnezeule, îi vei coborî pe ei în groapa stricaciunii. Barba?ii varsatori de sânge ?i vicleni nu vor ajunge la jumatatea zilelor lor; iar eu voi nadajdui spre Tine, Doamne.

\chapter{56}

\par 1 (Un psalm al lui David; mai-marelui cântare?ilor, pentru poporul ce se departase de la cele sfinte. Când l-au prins pe el cei de alt neam în Gat.) Mântuie?te-ma, Doamne, ca m-a necajit omul; toata ziua razboindu-se m-a necajit.
\par 2 Calcatu-m-au vrajma?ii mei toata ziua, ca mul?i sunt cei ce se lupta cu mine, din înal?ime.
\par 3 Ziua când ma voi teme, voi nadajdui în Tine.
\par 4 În Dumnezeu voi lauda toate cuvintele mele toata ziua; în Dumnezeu am nadajduit, nu ma voi teme: Ce-mi va face mie omul?
\par 5 Toata ziua cuvintele mele au urât, împotriva mea toate gândurile lor sunt spre rau.
\par 6 Locui-vor lânga mine ?i se vor ascunde; ei vor pazi calcâiul meu, ca ?i cum ar cauta sufletul meu.
\par 7 Pentru nimic nu-i vei mântui pe ei; în mânie popoare vei sfarâma, Dumnezeule.
\par 8 Via?a mea am spus-o ?ie; pune lacrimile mele înaintea Ta, dupa fagaduin?a Ta.
\par 9 Întoarce-se-vor vrajma?ii mei înapoi, în orice zi Te voi chema. Iata, am cunoscut ca Dumnezeul meu e?ti Tu.
\par 10 În Dumnezeu voi lauda graiul, în Dumnezeu voi lauda cuvântul;
\par 11 În Dumnezeu am nadajduit, nu ma voi teme: Ce-mi va face mie omul?
\par 12 În mine sunt, Dumnezeule,  fagaduin?ele pe care le voi aduce laudei Tale,
\par 13 Ca ai izbavit sufletul meu de la moarte, picioarele mele de alunecare, ca bine sa plac înaintea lui Dumnezeu, în lumina celor vii.

\chapter{57}

\par 1 (Un psalm al lui David; mai-marelui cântare?ilor. Sa nu strici! Când a fugit de la fa?a lui Saul în pe?tera.) Miluie?te-ma, Dumnezeule, miluie?te-ma, ca spre Tine a nadajduit sufletul meu ?i în umbra aripilor Tale voi nadajdui, pâna ce va trece faradelegea.
\par 2 Striga-voi catre Dumnezeul Cel Preaînalt, Dumnezeul Care mi-a facut bine.
\par 3 Trimis-a din cer ?i m-a mântuit, dat-a spre ocara pe cei ce ma necajesc pe mine.
\par 4 Trimis-a Dumnezeu mila Sa ?i adevarul Sau ?i a izbavit sufletul meu din mijlocul puilor de lei. Adormit-am tulburat. Fiii oamenilor, din?ii lor sunt arme ?i sage?i ?i limba lor sabie ascu?ita.
\par 5 Înal?a-Te peste ceruri, Dumnezeule, ?i peste tot pamântul slava Ta!
\par 6 Curse au gatit sub picioarele mele ?i au împilat sufletul meu; sapat-au înaintea mea groapa ?i au cazut în ea.
\par 7 Gata este inima mea, Dumnezeule, gata este inima mea! Cânta-voi ?i voi lauda slava Ta.
\par 8 De?teapta-te marirea mea! De?teapta-te psaltire ?i alauta! De?tepta-ma-voi diminea?a.
\par 9 Lauda-Te-voi între popoare, Doamne, cânta-voi ?ie între neamuri,
\par 10 Ca s-a marit pâna la cer mila Ta ?i pâna la nori adevarul Tau.
\par 11 Înal?a-Te peste ceruri, Dumnezeule, ?i peste tot pamântul slava Ta!

\chapter{58}

\par 1 (Un psalm al lui David; mai-marelui cântare?ilor. Sa nu strici!) De grai?i într-adevar dreptate, drept judeca?i, fii ai oamenilor.
\par 2 Pentru ca în inima faradelege lucra?i pe pamânt, nedreptate mâinile voastre împletesc.
\par 3 Înstrainatu-s-au pacato?ii de la na?tere, ratacit-au din pântece, grait-au minciuni.
\par 4 Mânia lor dupa asemanarea ?arpelui, ca a unei vipere surde, care-?i astupa urechile ei,
\par 5 Care nu va auzi glasul descântatoarelor, al vrajitorului care vraje?te cu iscusin?a.
\par 6 Dumnezeu va zdrobi din?ii lor în gura lor; maselele leilor le-a sfarâmat Domnul.
\par 7 De nimic se vor face, ca apa care trece; întinde-va arcul Sau pâna ce vor slabi.
\par 8 Ca ceara ce se tope?te vor fi nimici?i; a cazut foc peste ei ?i n-au vazut soarele.
\par 9 Înainte ca spinii vo?tri sa se aprinda, ca pe ni?te vii întru mânie îi va înghi?i pe ei.
\par 10 Veseli-se-va dreptul când va vedea razbunarea împotriva necredincio?ilor; mâinile lui le va spala în sângele pacatosului.
\par 11 ?i va zice omul: "Da, este rasplata pentru cel drept! Da, este Dumnezeu Care îi judeca pe ei în via?a!

\chapter{59}

\par 1 (Un psalm al lui David; mai-marelui cântare?ilor. Când a trimis Saul ?i a pazit casa lui, ca sa-l omoare pe el.) Scoate-ma de la vrajma?ii mei, Dumnezeule, ?i de cei ce se scoala împotriva mea, izbave?te-ma.
\par 2 Izbave?te-ma de cei ce lucreaza faradelegea ?i de barba?ii varsarilor de sânge ma izbave?te.
\par 3 Ca iata au vânat sufletul meu, statut-au împotriva mea cei tari. Nici faradelegea ?i nici pacatul meu nu sunt pricina, Doamne. Fara de nelegiuire am alergat ?i m-am îndreptat spre Tine;
\par 4 Scoala-Te, întru întâmpinarea mea ?i vezi. ?i Tu, Doamne, Dumnezeul puterilor, Dumnezeul lui Israel,
\par 5 Ia aminte sa cercetezi toate neamurile; sa nu Te milostive?ti de to?i cei ce lucreaza faradelegea.
\par 6 Întoarce-se-vor spre seara ?i vor flamânzi ca un câine ?i vor înconjura cetatea.
\par 7 Iata, vor striga cu gura lor ?i sabie în buzele lor, ca cine i-a auzit?
\par 8 ?i Tu, Doamne, vei râde de ei, vei face de nimic toate neamurile.
\par 9 Puterea mea în Tine o voi pazi, ca Tu, Dumnezeule, sprijinitorul meu e?ti. Dumnezeul meu, mila Ta ma va întâmpina;
\par 10 Dumnezeu îmi va arata înfrângerea du?manilor mei.
\par 11 Sa nu-i omori pe ei, ca nu cumva sa uite legea Ta; risipe?te-i pe ei cu puterea Ta ?i doboara-i pe ei, aparatorul meu, Doamne.
\par 12 Pacatul gurii lor, cuvântul buzelor lor, sa se prinda întru mândria lor ?i de blestemul ?i minciuna lor se va duce vestea.
\par 13 Nimice?te-i, întru mânia Ta nimice?te-i, ca sa nu mai fie! ?i vor cunoa?te ca Dumnezeu stapâne?te pe Iacob ?i marginile pamântului.
\par 14 Întoarce-se-vor spre seara ?i vor flamânzi ca un câine ?i vor înconjura cetatea.
\par 15 Ei se vor risipi sa manânce; iar de nu se vor satura, vor murmura.
\par 16 Iar eu voi lauda puterea Ta ?i ma voi bucura diminea?a de mila Ta. Ca Te-ai facut sprijinitorul meu ?i scaparea mea în ziua necazului meu.
\par 17 Ajutorul meu e?ti, ?ie-?i voi cânta, ca Tu, Dumnezeule, sprijinitorul meu e?ti, Dumnezeul meu, mila mea.

\chapter{60}

\par 1 (Un psalm al lui David; mai-marelui cântare?ilor. Pentru cei ce se vor schimba, spre înva?atura. Când a ars Mesopotamia Siriei ?i Sova Siriei ?i s-a întors Iacob ?i a batut pe Edom în Valea Sarii, douasprezece mii.) Dumnezeule, lepadatu-ne-ai pe noi ?i ne-ai doborât; mâniatu-Te-ai ?i Te-ai milostivit spre noi.
\par 2 Cutremurat-ai pamântul ?i l-ai tulburat pe el; vindeca sfarâmaturile lui, ca s-a cutremurat.
\par 3 Aratat-ai poporului Tau asprime, adapatu-ne-ai pe noi cu vinul umilin?ei.
\par 4 Dat-ai celor ce se tem de Tine semn ca sa fuga de la fa?a arcului;
\par 5 Ca sa se izbaveasca cei iubi?i ai Tai. Mântuie?te-ma cu dreapta Ta ?i ma auzi.
\par 6 Dumnezeu a grait în locul cel sfânt al Sau: "Bucura-Ma-voi ?i voi împar?i Sichemul ?i valea Sucot o voi masura.
\par 7 Al Meu este Galaadul ?i al Meu este Manase ?i Efraim, taria capului Meu,
\par 8 Iuda împaratul Meu; Moab vasul nadejdii Mele. Spre Idumeea voi întinde încal?amintea Mea; Mie cei de alt neam Mi s-au supus".
\par 9 Cine ma va duce la cetatea întarita? Cine ma va pova?ui pâna la Idumeea?
\par 10 Oare, nu Tu, Dumnezeule, Cel ce ne-ai lepadat pe noi? Oare, nu vei ie?i Dumnezeule, cu o?tirile noastre?
\par 11 Da-ne noua ajutor, ca sa ne sco?i din necaz, ca de?arta este izbavirea de la om.
\par 12 Cu Dumnezeu vom birui ?i El va nimici pe cei ce ne necajesc pe noi.

\chapter{61}

\par 1 (Un psalm al lui David; mai-marelui cântare?ilor, pentru instrumente cu coarde.) Auzi, Dumnezeule, cererea mea, ia aminte la rugaciunea mea!
\par 2 De la marginile pamântului catre Tine am strigat; când s-a mâhnit inima mea, pe piatra m-ai înal?at.
\par 3 Pova?uitu-m-ai, ca ai fost nadejdea mea, turn de tarie în fa?a vrajma?ului.
\par 4 Locui-voi în loca?ul Tau în veci; acoperi-ma-voi cu acoperamântul aripilor Tale,
\par 5 Ca Tu, Dumnezeule, ai auzit rugaciunile mele; dat-ai mo?tenire celor ce se tem de numele Tau.
\par 6 Zile la zilele împaratului adauga, anii lui din neam în neam.
\par 7 Ramâne-va în veac înaintea lui Dumnezeu; mila ?i adevarul va pazi.
\par 8 A?a voi cânta numelui Tau în veacul veacului, ca sa împlinesc fagaduin?ele mele zi de zi.

\chapter{62}

\par 1 (Un psalm al lui David; mai-marelui cântare?ilor. Pentru Iditum.) Oare nu lui Dumnezeu se va supune sufletul meu? Ca de la El este mântuirea mea;
\par 2 Pentru ca El este Dumnezeul meu, Mântuitorul meu ?i Sprijinitorul meu; nu ma voi clatina mai mult.
\par 3 Pâna când va ridica?i asupra omului? Cauta?i to?i a-l doborî, socotindu-l ca un zid povârnit ?i ca un gard surpat!
\par 4 S-au sfatuit sa doboare cinstea mea, alergat-au cu minciuna; cu gura lor ma binecuvântau ?i cu inima lor ma blestemau.
\par 5 Dar lui Dumnezeu supune-te, suflete al meu, ca de la El vine rabdarea mea;
\par 6 Ca El este Dumnezeul meu ?i Mântuitorul meu, Sprijinitorul meu; nu ma voi stramuta.
\par 7 În Dumnezeu este mântuirea mea ?i slava mea; Dumnezeu este ajutorul meu ?i nadejdea mea este în Dumnezeu.
\par 8 Nadajdui?i în El toata adunarea poporului; revarsa?i înaintea Lui inimile voastre, ca El este ajutorul nostru.
\par 9 Dar de?ertaciune sunt fiii oamenilor, mincino?i sunt fiii oamenilor; în balan?a, to?i împreuna sunt de?ertaciune.
\par 10 Nu nadajdui?i spre nedreptate ?i spre jefuire nu pofti?i; boga?ia de ar curge nu va lipi?i inima de ea.
\par 11 O data a grait Dumnezeu, aceste doua lucruri am auzit: ca puterea este a lui Dumnezeu
\par 12 ?i a Ta, Doamne, este mila; ca Tu vei rasplati fiecaruia dupa faptele lui.

\chapter{63}

\par 1 (Un psalm al lui David, când a fost în pustiul Iudeii.) Dumnezeule, Dumnezeul meu, pe Tine Te caut dis-de-diminea?a.
\par 2 Însetat-a de Tine sufletul meu, suspinat-a dupa Tine trupul meu,
\par 3 În pamânt pustiu ?i neumblat ?i fara de apa. A?a în locul cel sfânt m-am aratat ?ie, ca sa vad puterea Ta ?i slava Ta.
\par 4 Ca mai buna este mila Ta decât via?a; buzele mele Te vor lauda.
\par 5 A?a Te voi binecuvânta în via?a mea ?i în numele Tau voi ridica mâinile mele.
\par 6 Ca de seu ?i de grasime sa se sature sufletul meu ?i cu buze de bucurie Te va lauda gura mea.
\par 7 De mi-am adus aminte de Tine în a?ternutul meu, în dimine?i am cugetat la Tine, ca ai fost ajutorul meu
\par 8 ?i întru acoperamântul aripilor Tale ma voi bucura. Lipitu-s-a sufletul meu de Tine ?i pe mine m-a sprijinit dreapta Ta.
\par 9 Iar ei în de?ert au cautat sufletul meu, intra-vor în cele mai de jos ale pamântului;
\par 10 Da-se-vor în mâinile sabiei, par?i vulpilor vor fi.
\par 11 Iar împaratul se va veseli de Dumnezeu; lauda-se-va tot cel ce se jura întru El, ca s-a astupat gura celor ce graiesc nedrepta?i.

\chapter{64}

\par 1 (Un psalm al lui David; mai-marelui cântare?ilor.) Auzi, Dumnezeule, glasul meu, când ma rog ?ie; de la frica vrajma?ului scoate sufletul meu.
\par 2 Acopera-ma de adunarea celor ce viclenesc, de mul?imea celor ce lucreaza faradelege,
\par 3 Care ?i-au ascu?it ca sabia limbile lor ?i ca ni?te sage?i arunca vorbele lor veninoase ca sa sageteze din ascunzi?uri pe cel nevinovat.
\par 4 Fara de veste îl vor sageta pe el ?i nu se vor teme. Întaritu-s-au în gânduri rele.
\par 5 Grait-au ca sa ascunda curse; spus-au: "Cine ne va vedea pe noi?"
\par 6 Iscodit-au faradelegi ?i au pierit când le iscodeau, ca sa patrunda înlauntrul omului ?i în adâncimea inimii lui.
\par 7 Dar Dumnezeu îi va lovi cu sageata ?i fara de veste îi va rani, ca ei singuri se vor rani cu limbile lor.
\par 8 Tulburatu-s-au to?i cei ce i-au vazut pe ei; ?i s-a temut tot omul.
\par 9 ?i au vestit lucrurile lui Dumnezeu ?i faptele Lui le-au în?eles.
\par 10 Veseli-se-va cel drept de Domnul ?i va nadajdui în El ?i se vor lauda to?i cei drep?i la inima.

\chapter{65}

\par 1 (Un psalm al lui David; mai-marelui cântare?ilor. Cântarea lui Ieremia ?i a lui Iezechiel din timpul ?ederii în pamânt strain, când avea sa iasa din Babilon.) ?ie ?i se cuvine cântare, Dumnezeule, în Sion ?i ?ie ?i se va împlini fagaduin?a în Ierusalim.
\par 2 Auzi rugaciunea mea, catre Tine tot trupul va veni.
\par 3 Cuvintele celor fara de lege ne-au biruit pe noi ?i nelegiuirile noastre Tu le vei cura?i.
\par 4 Fericit este cel pe care l-ai ales ?i l-ai primit; locui-va în cur?ile Tale.
\par 5 Umplea-ne-vom de bunata?ile casei Tale; sfânt este loca?ul Tau, minunat în dreptate.
\par 6 Auzi-ne pe noi, Dumnezeule, Mântuitorul nostru, nadejdea tuturor marginilor pamântului ?i a celor de pe mare departe;
\par 7 Cel ce gate?ti mun?ii cu taria Ta, încins fiind cu putere; Cel ce tulburi adâncul marii ?i vuietul valurilor ei.
\par 8 Tulbura-se-vor neamurile ?i se vor spaimânta cei ce locuiesc marginile, de semnele Lui; ie?irile dimine?ii ?i ale serii le vei veseli.
\par 9 Cercetat-ai pamântul ?i l-ai adapat pe el, boga?iile lui le-ai înmul?it; râul lui Dumnezeu s-a umplut de apa; gatit-ai hrana lor, ca a?a este gatirea Ta.
\par 10 Adapa brazdele lui, înmul?e?te roadele lui ?i se vor bucura de picaturi de ploaie, rasarind.
\par 11 Vei binecuvânta cununa anului bunata?ii Tale ?i câmpiile Tale se vor umple de roade grase.
\par 12 Îngra?a-se-vor pa?unile pustiei ?i cu bucurie dealurile se vor încinge.
\par 13 Îmbracatu-s-au pa?unile cu oi ?i vaile vor înmul?i grâul; vor striga ?i vor cânta.

\chapter{66}

\par 1 (Un psalm al lui David; mai-marelui cântare?ilor. O cântare a învierii.) Striga?i lui Dumnezeu tot pamântul.
\par 2 Cânta?i numele Lui; da?i slava laudei Lui.
\par 3 Zice?i lui Dumnezeu: Cât sunt de înfrico?atoare lucrurile Tale! Pentru mul?imea puterii Tale, Te vor lingu?i vrajma?ii Tai.
\par 4 Tot pamântul sa se închine ?ie ?i sa cânte ?ie, sa cânte numelui Tau.
\par 5 Veni?i ?i vede?i lucrurile lui Dumnezeu, înfrico?ator în sfaturi mai mult decât fiii oamenilor.
\par 6 Cel ce preface marea în uscat, prin râu vor trece cu piciorul. Acolo ne vom veseli de El,
\par 7 De Cel ce stapâne?te cu puterea Sa veacul. Ochii Lui spre neamuri privesc; cei ce se razvratesc, sa nu se înal?e întru sine.
\par 8 Binecuvânta?i neamuri pe Dumnezeul nostru ?i face?i sa se auda glasul laudei Lui,
\par 9 Care a dat sufletului meu via?a ?i n-a lasat sa se clatine picioarele mele.
\par 10 Ca ne-ai cercetat pe noi, Dumnezeule, cu foc ne-ai lamurit pe noi, precum se lamure?te argintul.
\par 11 Prinsu-ne-ai pe noi în cursa; pus-ai necazuri pe umarul nostru;
\par 12 Ridicat-ai oameni pe capetele noastre, trecut-am prin foc ?i prin apa ?i ne-ai scos la odihna.
\par 13 Intra-voi în casa Ta cu arderi de tot, împlini-voi ?ie fagaduin?ele mele,
\par 14 Pe care le-au rostit buzele mele ?i le-a grait gura mea, întru necazul meu.
\par 15 Arderi de tot grase voi aduce ?ie, cu tamâie ?i berbeci; Î?i voi jertfi boi ?i ?api.
\par 16 Veni?i de auzi?i to?i cei ce va teme?i de Dumnezeu ?i va voi povesti câte a facut El sufletului meu.
\par 17 Catre Dânsul cu gura mea am strigat ?i L-am laudat cu gura mea.
\par 18 Nedreptate de am avut în inima mea sa nu ma auda Domnul.
\par 19 Pentru aceasta m-a auzit Dumnezeu; luat-a aminte glasul rugaciunii mele.
\par 20 Binecuvântat este Dumnezeu, Care n-a departat rugaciunea mea ?i mila Lui de la mine.

\chapter{67}

\par 1 (Un psalm al lui David; mai-marelui cântare?ilor. Pentru instrumente cu coarde.) Dumnezeule, milostive?te-Te spre noi ?i ne binecuvinteaza, lumineaza fa?a Ta spre noi ?i ne miluie?te!
\par 2 Ca sa cunoa?tem pe pamânt calea Ta, în toate neamurile mântuirea Ta.
\par 3 Lauda-Te-vor pe Tine popoarele, Dumnezeule, lauda-Te-vor pe Tine popoarele toate!
\par 4 Veseleasca-se ?i sa se bucure neamurile, ca vei judeca popoarele cu dreptate ?i neamurile pe pamânt le vei povatui.
\par 5 Lauda-Te-vor pe Tine popoarele Dumnezeule, lauda-Te-vor pe Tine popoarele toate. Pamântul ?i-a dat rodul sau.
\par 6 Binecuvinteaza-ne pe noi, Dumnezeule, Dumnezeul nostru.
\par 7 Binecuvinteaza-ne pe noi, Dumnezeule, ?i sa se teama de Tine toate marginile pamântului.

\chapter{68}

\par 1 (Un psalm al lui David; mai-marelui cântare?ilor.) Sa se scoale Dumnezeu ?i sa se risipeasca vrajma?ii Lui ?i sa fuga de la fa?a Lui cei ce-L urasc pe El.
\par 2 Precum se stinge fumul, sa se stinga; cum se tope?te ceara de fa?a focului, a?a sa piara pacato?ii de la fa?a lui Dumnezeu,
\par 3 Iar drep?ii sa se bucure ?i sa se veseleasca înaintea lui Dumnezeu, sa se desfateze în veselie.
\par 4 Cânta?i lui Dumnezeu, cânta?i numelui Lui, gati?i calea Celui ce strabate pustia, Domnul este numele Lui,
\par 5 ?i va bucura?i înaintea Lui. Sa se tulbure de fa?a Lui, a Parintelui orfanilor ?i a Judecatorului vaduvelor.
\par 6 Dumnezeu este în locul cel sfânt al Lui; Dumnezeu a?aza pe cei singuratici în casa, scoate cu vitejie pe cei lega?i în obezi, la fel pe cei amarâ?i, pe cei ce locuiesc în morminte.
\par 7 Dumnezeule, când mergeai Tu înaintea poporului Tau, când treceai Tu prin pustiu,
\par 8 Pamântul s-a cutremurat ?i cerurile s-au topit ?i Sinaiul s-a clatinat de la fa?a Dumnezeului lui Israel.
\par 9 Ploaie de bunavoie vei osebi, Dumnezeule, mo?tenirii Tale. Ea a slabit, dar Tu ai întarit-o.
\par 10 Vieta?ile Tale locuiesc în ea; întru bunatatea Ta, Dumnezeule, purtat-ai grija de cel sarac.
\par 11 Domnul va da cuvântul celor ce vestesc cu putere multa.
\par 12 Împaratul puterilor, poporului iubit va împar?i prazile.
\par 13 Daca ve?i dormi în mijlocul mo?tenirilor voastre, aripile voastre argintate vor fi ca ale porumbi?ei ?i spatele vostru va straluci ca aurul.
\par 14 Când Împaratul Cel ceresc va împra?tia pe regi în ?ara Sa, ei vor fi albi ca zapada pe Selmon.
\par 15 Munte al lui Dumnezeu este muntele Vasan, munte de piscuri este muntele Vasan.
\par 16 Pentru ce, mun?i cu piscuri, pizmui?i muntele în care a binevoit Dumnezeu sa locuiasca în el, pentru ca va locui în el pâna la sfâr?it?
\par 17 Carele lui Dumnezeu sunt mii de mii; mii sunt cei ce se bucura de ele. Domnul în mijlocul lor, pe Sinai, în loca?ul Sau cel sfânt.
\par 18 Suitu-Te-ai la înal?ime, robit-ai mul?ime, luat-ai daruri de la oameni, chiar ?i cu cei razvrati?i îngadui?i au fost sa locuiasca.
\par 19 Domnul Dumnezeu este binecuvântat, binecuvântat este Dumnezeu zi de zi; sa sporeasca între noi, Dumnezeul mântuirii noastre.
\par 20 Dumnezeul nostru este Dumnezeul mântuirii ?i ale Domnului Dumnezeu sunt ie?irile mor?ii.
\par 21 Dar Dumnezeu va sfarâma capetele vrajma?ilor Sai, cre?tetul parului celor ce umbla întru gre?elile lor.
\par 22 Zis-a Domnul: "Din Vasan îl voi întoarce, întoarce-voi pe vrajma?ii tai din adâncurile marii,
\par 23 Pentru ca sa se afunde piciorul tau în sângele lor ?i limba câinilor tai în sângele vrajma?ilor tai.
\par 24 Vazut-am, Dumnezeule, intrarea Ta, vazut-am intrarea Dumnezeului ?i Împaratului meu în loca?ul cel sfânt:
\par 25 Înainte mergeau capeteniile, dupa ei cei ce cântau din strune, în mijloc fecioarele batând din timpane ?i zicând:
\par 26 "În adunari binecuvânta?i pe Dumnezeu, pe Domnul din izvoarele lui Israel!"
\par 27 Acolo era Veniamin cel mai tânar, în uimire; capeteniile lui Iuda, pova?uitorii lor, capeteniile Zabulonului, capeteniile Neftalimului ?i ziceau:
\par 28 "Porunce?te, Dumnezeule, puterii Tale; întare?te Dumnezeule aceasta lucrare pe care ai facut-o noua.
\par 29 Pentru loca?ul Tau, din Ierusalim, Î?i vor aduce împara?ii daruri.
\par 30 Cearta fiarele din trestii, cearta taurii aduna?i împotriva junincilor popoarelor, ca sa nu fie departa?i cei care au fost încerca?i ca argintul.
\par 31 Risipe?te neamurile cele ce voiesc razboaie". Veni-vor soli din Egipt; Etiopia va întinde mai înainte la Dumnezeu mâna ei, zicând:
\par 32 "Împara?iile pamântului cânta?i lui Dumnezeu, cânta?i Domnului.
\par 33 Cânta?i Dumnezeului Celui ce S-a suit peste cerul cerului, spre rasarit; iata va da glasul Sau, glas de putere.
\par 34 Da?i slava lui Dumnezeu! Peste Israel mare?ia Lui ?i puterea Lui în nori.
\par 35 Minunat este Dumnezeu întru sfin?ii Lui, Dumnezeul lui Israel; Însu?i va da putere ?i întarire poporului Sau. Binecuvântat este Dumnezeu".

\chapter{69}

\par 1 (Un psalm al lui David; mai-marelui cântare?ilor. Pentru cei ce se vor schimba.) Mântuie?te-ma, Dumnezeule, ca au intrat ape pâna la sufletul meu.
\par 2 Afundatu-m-am în noroiul adâncului, care nu are fund; intrat-am în adâncurile marii ?i furtuna m-a potopit.
\par 3 Ostenit-am strigând, amor?it-a gâtlejul meu, slabit-au ochii mei nadajduind spre Dumnezeul meu.
\par 4 Înmul?itu-s-au mai mult decât perii capului meu cei ce ma urasc pe mine în zadar. Întaritu-s-au vrajma?ii mei, cei ce ma prigonesc pe nedrept; cele ce n-am rapit, pe acelea le-am platit.
\par 5 Dumnezeule, Tu ai cunoscut nepriceperea mea ?i gre?elile mele de la Tine nu s-au ascuns.
\par 6 Sa nu fie ru?ina?i, din pricina mea, cei ce Te a?teapta pe Tine, Doamne, Doamne al puterilor, nici înfrunta?i pentru mine, cei ce Te cauta pe Tine, Dumnezeul lui Israel.
\par 7 Ca pentru Tine am suferit ocara, acoperit-a batjocura obrazul meu.
\par 8 Înstrainat am fost de fra?ii mei ?i strain fiilor maicii mele,
\par 9 Ca râvna casei Tale m-a mâncat ?i ocarile celor ce Te ocarasc pe Tine au cazut asupra mea.
\par 10 ?i mi-am smerit cu post sufletul meu ?i mi-a fost spre ocara mie.
\par 11 ?i m-am îmbracat cu sac ?i am ajuns pentru ei de batjocura.
\par 12 Împotriva mea graiau cei ce ?edeau în por?i ?i despre mine cântau cei ce beau vin.
\par 13 Iar eu întru rugaciunea mea catre Tine, Doamne, am strigat la timp bine-placut. Dumnezeule, întru mul?imea milei Tale auzi-ma, întru adevarul milei Tale.
\par 14 Mântuie?te-ma din noroi, ca sa nu ma afund; izbave?te-ma de cei ce ma urasc ?i din adâncul apelor,
\par 15 Ca sa nu ma înece vâltoarea apei, nici sa ma înghita adâncul, nici sa-?i închida peste mine adâncul gura lui.
\par 16 Auzi-ma, Doamne, ca buna este mila Ta; dupa mul?imea îndurarilor Tale cauta spre mine.
\par 17 Sa nu-?i întorci fa?a Ta de la credinciosul Tau, când ma necajesc. Degraba ma auzi.
\par 18 Ia aminte la sufletul meu ?i-l mântuie?te pe el; din mâinile vrajma?ilor mei izbave?te-ma,
\par 19 Ca Tu cuno?ti ocara mea ?i ru?inea mea ?i înfruntarea mea; înaintea Ta sunt to?i cei ce ma necajesc.
\par 20 Zdrobit a fost sufletul meu de ocari ?i necaz, ?i am a?teptat pe cel ce m-ar milui ?i nu era ?i pe cei ce m-ar mângâia ?i nu i-am aflat.
\par 21 ?i mi-au dat spre mâncarea mea fiere ?i în setea mea m-au adapat cu o?et.
\par 22 Faca-se masa lor înaintea lor cursa, rasplatire ?i sminteala;
\par 23 Sa se întunece ochii lor, ca sa nu vada ?i spinarea lor pururea o gârbove?te;
\par 24 Varsa peste ei urgia Ta ?i mânia urgiei Tale sa-i cuprinda pe ei;
\par 25 Faca-se curtea lor pustie ?i în loca?urile lor sa nu fie locuitori;
\par 26 Ca pe care Tu l-ai batut, ei l-au prigonit ?i au înmul?it durerea ranilor lui.
\par 27 Adauga faradelege la faradelegea lor ?i sa nu intre întru dreptatea Ta;
\par 28 ?ter?i sa fie din cartea celor vii ?i cu cei drep?i sa nu se scrie.
\par 29 Sarac ?i îndurerat sunt eu; mântuirea Ta, Dumnezeule, sa ma sprijineasca!
\par 30 Lauda-voi numele Dumnezeului meu cu cântare ?i-L voi preamari pe El cu lauda;
\par 31 ?i-I va placea lui Dumnezeu mai mult decât vi?elul tânar, caruia îi cresc coarne ?i unghii.
\par 32 Sa râda saracii ?i sa se veseleasca; cânta?i lui Dumnezeu ?i viu va fi sufletul vostru!
\par 33 Ca a auzit pe cei saraci Domnul ?i pe cei fereca?i în obezi ai Sai nu i-a urgisit.
\par 34 Sa-L laude pe El cerurile ?i pamântul, marea ?i toate câte se târasc în ea.
\par 35 Ca Dumnezeu va mântui Sionul ?i se vor zidi ceta?ile lui Iuda ?i vor locui acolo ?i-l vor mo?teni pe el;
\par 36 ?i semin?ia credincio?ilor Lui îl va stapâni pe el ?i cei ce iubesc numele Lui vor locui în el.

\chapter{70}

\par 1 (Un psalm al lui David; mai-marelui cântare?ilor. Spre aducere aminte. Ca sa ma mântuiasca Domnul.) Dumnezeule, spre ajutorul meu ia aminte! Doamne, sa-mi aju?i mie grabe?te-Te!
\par 2 Sa se ru?ineze ?i sa se înfrunte cei ce cauta sufletul meu; sa se întoarca înapoi ?i sa se ru?ineze cei ce-mi voiesc mie rele;
\par 3 Întoarca-se îndata ru?ina?i cei ce-mi graiesc mie: "Bine, bine!"
\par 4 Sa se bucure ?i sa se veseleasca de Tine to?i cei ce Te cauta pe Tine, Dumnezeule, ?i sa zica pururea cei ce iubesc mântuirea Ta: "Slavit sa fie Domnul!"
\par 5 Iar eu sarac sunt ?i sarman, Dumnezeule, ajuta-ma! Ajutorul meu ?i Izbavitorul meu e?ti Tu, Doamne, nu zabovi!

\chapter{71}

\par 1 (Un psalm al lui David; fiilor lui Ionadav ?i ai celor ce s-au robit mai întâi.) Spre Tine, Doamne, am nadajduit, sa nu fiu ru?inat în veac.
\par 2 Întru dreptatea Ta, izbave?te-ma ?i ma scoate, pleaca urechea Ta catre mine ?i ma mântuie?te.
\par 3 Fii mie Dumnezeu aparator ?i loc întarit, ca sa ma mântuie?ti, ca întarirea ?i scaparea mea e?ti Tu.
\par 4 Dumnezeul meu, izbave?te-ma din mâna pacatosului, din mâna calcatorului de lege ?i a celui ce face strâmbatate,
\par 5 Ca Tu e?ti a?teptarea mea, Doamne; Domnul este nadejdea mea din tinere?ile mele.
\par 6 Întru Tine m-am întarit din pântece; din pântecele maicii mele Tu e?ti acoperitorul meu; întru Tine este lauda mea pururea.
\par 7 Ca o minune m-am facut multora, iar Tu e?ti ajutorul meu cel tare.
\par 8 Sa se umple gura mea de lauda Ta, ca sa laud slava Ta, toata ziua mare cuviin?a Ta.
\par 9 Nu ma lepada la vremea batrâne?ilor; când va lipsi taria mea, sa nu ma la?i pe mine.
\par 10 Ca au zis vrajma?ii mei mie ?i cei ce pazesc sufletul meu s-au sfatuit împreuna,
\par 11 Zicând: Dumnezeu l-a parasit pe el, urmari?i-l ?i-l prinde?i pe el, ca nu este cel ce izbave?te.
\par 12 Dumnezeule, nu Te departa de la mine; Dumnezeul meu, spre ajutorul meu ia aminte!
\par 13 Sa se ru?ineze ?i sa piara cei ce defaimeaza sufletul meu, sa se îmbrace cu ru?ine ?i înfruntare cei ce cauta sa-mi faca rau.
\par 14 Iar eu pururea voi nadajdui spre Tine ?i voi înmul?i lauda Ta.
\par 15 Gura mea va vesti dreptatea Ta, toata ziua mântuirea Ta, al caror nume nu-l cunosc.
\par 16 Intra-voi întru puterea Domnului; Doamne, îmi voi aduce aminte numai de dreptatea Ta.
\par 17 Dumnezeule, m-ai înva?at din tinere?ile mele ?i eu ?i astazi vestesc minunile Tale.
\par 18 Pâna la batrâne?e ?i carunte?e, Dumnezeule, sa nu ma parase?ti, ca sa vestesc bra?ul Tau la tot neamul ce va sa vina,
\par 19 Puterea Ta ?i dreptatea Ta, Dumnezeule, pâna la cele înalte, mare?iile pe care le-ai facut. Dumnezeule, cine este asemenea ?ie?
\par 20 Multe necazuri ?i rele ai trimis asupra mea, dar întorcându-Te mi-ai dat via?a ?i din adâncurile pamântului iara?i m-ai scos.
\par 21 Înmul?it-ai spre mine marirea Ta ?i întorcându-Te m-ai mângâiat ?i din adâncurile pamântului iara?i m-ai scos.
\par 22 Ca eu voi lauda cu instrumente de cântare adevarul Tau, Dumnezeule, cânta-voi ?ie din alauta, Sfântul lui Israel.
\par 23 Bucura-se-vor buzele mele când voi cânta ?ie ?i sufletul meu pe care l-ai mântuit.
\par 24 Înca ?i limba mea toata ziua va rosti dreptatea Ta, când vor fi ru?ina?i ?i înfrunta?i cei ce cauta sa-mi faca rau.

\chapter{72}

\par 1 (Despre Solomon.) Dumnezeule, judecata Ta da-o împaratului ?i dreptatea Ta fiului împaratului,
\par 2 Ca sa judece pe poporul Tau cu dreptate ?i pe saracii Tai cu judecata.
\par 3 Sa aduca mun?ii pace poporului Tau ?i dealurile dreptate.
\par 4 Judeca-va pe saracii poporului ?i va milui pe fiii saracilor ?i va umili pe clevetitor.
\par 5 ?i se vor teme de Tine cât va fi soarele ?i cât va fi luna din neam în neam.
\par 6 Pogorâ-se-va ca ploaia pe lâna ?i ca picaturile ce cad pe pamânt.
\par 7 Rasari-va în zilele lui dreptatea ?i mul?imea pacii, cât va fi luna.
\par 8 ?i va domni de la o mare pâna la alta ?i de la râu pâna la marginile lumii.
\par 9 Înaintea lui vor îngenunchia etiopienii ?i vrajma?ii lui ?arâna vor linge.
\par 10 Împara?ii Tarsisului ?i insulele daruri vor aduce, împara?ii arabilor ?i ai reginei Saba prinoase vor aduce.
\par 11 ?i se vor închina lui to?i împara?ii pamântului, toate neamurile vor sluji lui.
\par 12 Ca a izbavit pe sarac din mâna celui puternic ?i pe sarmanul care n-avea ajutor.
\par 13 Va avea mila de sarac ?i de sarman ?i sufletele saracilor va mântui;
\par 14 De camata ?i de asuprire va scapa sufletele lor ?i scump va fi numele lor înaintea lui.
\par 15 ?i va fi viu ?i se va da lui din aurul Arabiei ?i se vor ruga pentru el pururea; toata ziua îl vor binecuvânta pe el.
\par 16 Fi-va bel?ug de pâine pe pamânt pâna-n vârful mun?ilor; pomii roditori se vor înal?a ca cedrii Libanului; ?i vor înflori cei din cetate ca iarba pamântului.
\par 17 Numele lui va dainui pe vecie; cât va fi soarele va fi pomenit numele lui. Se vor binecuvânta întru el toate semin?iile pamântului, toate neamurile îl vor ferici pe el.
\par 18 Binecuvântat este Domnul Dumnezeu, Dumnezeul lui Israel, singurul Care face minuni.
\par 19 Binecuvântat este numele slavei Lui în veac ?i în veacul veacului.
\par 20 Tot pamântul se va umple de slava Lui. Amin. Amin.

\chapter{73}

\par 1 (Un psalm al lui Asaf.) Cât de bun este Dumnezeu cu Israel, cu cei drep?i la inima.
\par 2 Iar mie, pu?in a fost de nu mi-au alunecat picioarele, pu?in a fost de nu s-au poticnit pa?ii mei.
\par 3 Ca am pizmuit pe cei fara de lege, când vedeam pacea pacato?ilor.
\par 4 Ca n-au necazuri pâna la moartea lor ?i tari sunt când lovesc ei.
\par 5 De osteneli omene?ti n-au parte ?i cu oamenii nu sunt biciui?i.
\par 6 Pentru aceea îi stapâne?te pe ei mândria ?i se îmbraca cu nedreptatea ?i silnicia.
\par 7 Din rautatea lor iese nedreptatea ?i cugetele inimii lor ies la iveala.
\par 8 Gândesc ?i vorbesc cu vicle?ug, nedreptate graiesc de sus.
\par 9 Pâna la cer ridica gura lor ?i cu limba lor strabat pamântul.
\par 10 Pentru aceasta poporul meu se ia dupa ei ?i gase?te ca ei sunt plini de zile bune
\par 11 ?i zice: "Cum? ?tie aceasta Dumnezeu? Are cuno?tin?a Cel Preaînalt?
\par 12 Iata, ace?tia sunt pacato?i ?i sunt îndestula?i. Ve?nic sunt boga?i".
\par 13 Iar eu am zis: "Deci, în de?ert am fost drept la inima ?i mi-am spalat întru cele nevinovate mâinile mele,
\par 14 Ca am fost lovit toata ziua ?i mustrat în fiecare diminea?a".
\par 15 Daca a? fi grait a?a, iata a? fi calcat legamântul neamului fiilor Tai.
\par 16 ?i ma framântam sa pricep aceasta, dar anevoios lucru este înaintea mea.
\par 17 Pâna ce am intrat în loca?ul cel sfânt al lui Dumnezeu ?i am în?eles sfâr?itul celor rai:
\par 18 Într-adevar pe drumuri viclene i-ai pus pe ei ?i i-ai doborât când se înal?au.
\par 19 Cât de iute i-ai pustiit pe ei! S-au stins, au pierit din pricina nelegiuirii lor.
\par 20 Ca visul celui ce se de?teapta, Doamne, în cetatea Ta chipul lor de nimic l-ai facut.
\par 21 De aceea s-a bucurat inima mea ?i rarunchii mei s-au potolit.
\par 22 Ca eram fara de minte ?i nu ?tiam; ca un dobitoc eram înaintea Ta.
\par 23 Dar eu sunt pururea cu Tine. Apucatu-m-ai de mâna mea cea dreapta.
\par 24 Cu sfatul Tau m-ai pova?uit ?i cu slava m-ai primit.
\par 25 Ca pe cine am eu în cer afara de Tine? ?i afara de Tine, ce am dorit pe pamânt?
\par 26 Stinsu-s-a inima mea ?i trupul meu, Dumnezeul inimii mele ?i partea mea, Dumnezeule, în veac.
\par 27 Ca iata cei ce se departeaza de Tine vor pieri; nimicit-ai pe tot cel ce se leapada de Tine.
\par 28 Iar mie a ma lipi de Dumnezeu bine este, a pune în Domnul nadejdea mea, ca sa vestesc toate laudele Tale în por?ile fiicei Sionului.

\chapter{74}

\par 1 (Un psalm al lui Asaf. Al priceperii.) Pentru ce m-ai lepadat, Dumnezeule, pâna în sfâr?it? Aprinsu-s-a inima Ta peste oile pa?unii Tale.
\par 2 Adu-?i aminte de poporul Tau, pe care l-ai câ?tigat de la început. Izbavit-ai toiagul mo?tenirii Tale, muntele Sionului, acesta în care ai locuit.
\par 3 Ridica mâinile Tale împotriva mândriilor lor, pâna la sfâr?it, ca rau a facut vrajma?ul în locul cel sfânt al Tau.
\par 4 ?i s-au falit cei ce Te urasc pe Tine în mijlocul locului de praznuire al Tau, pus-au semnele lor drept semne;
\par 5 Sfarâmat-au intrarea cea de deasupra.
\par 6 Ca în codru cu topoarele au taiat u?ile loca?ului Tau, cu topoare ?i ciocane l-au sfarâmat.
\par 7 Ars-au cu foc loca?ul cel sfânt al Tau, pâna la pamânt; spurcat-au locul numelui Tau.
\par 8 Zis-au în inima lor împreuna cu neamul lor: "Veni?i sa ardem toate locurile de praznuire ale lui Dumnezeu de pe pamânt".
\par 9 Semnele noastre nu le-am vazut; nu mai este profet ?i pe noi nu ne va mai cunoa?te.
\par 10 Pâna când, Dumnezeule, Te va ocarî vrajma?ul, pâna când va huli potrivnicul numele Tau, pâna în sfâr?it?
\par 11 Pentru ce întorci mâna Ta ?i dreapta Ta din sânul Tau, pâna în sfâr?it?
\par 12 Dar Dumnezeu, Împaratul nostru înainte de veac, a facut mântuire în mijlocul pamântului.
\par 13 Tu ai despar?it, cu puterea Ta, marea; Tu ai zdrobit capetele balaurilor din apa;
\par 14 Tu ai sfarâmat capul balaurului; datu-l-ai pe el mâncare popoarelor pustiului.
\par 15 Tu ai deschis izvoare ?i pâraie; Tu ai secat râurile Itanului.
\par 16 A Ta este ziua ?i a Ta este noaptea. Tu ai întocmit lumina ?i soarele.
\par 17 Tu ai facut toate marginile pamântului; vara ?i primavara Tu le-ai zidit.
\par 18 Adu-?i aminte de aceasta: Vrajma?ul a ocarât pe Domnul ?i poporul cel fara de minte a hulit numele Tau.
\par 19 Sa nu dai fiarelor sufletul ce Te lauda pe Tine; sufletele saracilor Tai sa nu le ui?i pâna în sfâr?it.
\par 20 Cauta spre legamântul Tau, ca s-au umplut ascunzi?urile pamântului de locuin?ele faradelegilor.
\par 21 Sa nu se întoarca ru?inat cel umilit; saracul ?i sarmanul ?a laude numele Tau.
\par 22 Scoala-Te, Dumnezeule, apara pricina Ta; adu-?i aminte de ocara de fiecare zi, cu care Te necinste?te cel fara de minte.
\par 23 Nu uita strigatul vrajma?ilor Tai! Razvratirea celor ce Te urasc pe Tine se urca pururea spre Tine.

\chapter{75}

\par 1 (Un psalm al lui Asaf; mai-marelui cântare?ilor. Sa nu strici!) Lauda-Te-vom pe Tine, Dumnezeule, lauda-Te-vom ?i vom chema numele Tau.
\par 2 Voi spune toate minunile Tale. "Când va fi vremea, zice Domnul, cu dreptate voi judeca.
\par 3 Cutremuratu-s-a pamântul ?i to?i cei ce locuiesc pe el; Eu am întarit stâlpii lui".
\par 4 ?i am zis celor fara de lege: "Nu face?i faradelege!" ?i pacato?ilor: "Nu înal?a?i fruntea!
\par 5 Nu ridica?i la înal?ime fruntea voastra, sa nu grai?i nedreptate împotriva lui Dumnezeu".
\par 6 Ca nici de la rasarit, nici de la apus, nici din mun?ii pustiei, nu vine ajutorul;
\par 7 Ci Dumnezeu este judecatorul; pe unul îl smere?te ?i pe altul îl înal?a.
\par 8 Paharul este în mâna Domnului, plin cu vin curat bine-mirositor, ?i îl trece de la unul la altul, dar drojdia lui nu s-a varsat; din ea vor bea to?i pacato?ii pamântului.
\par 9 Iar eu ma voi bucura în veac, cânta-voi Dumnezeului lui Iacob.
\par 10 ?i toate frun?ile pacato?ilor voi zdrobi ?i se va înal?a fruntea dreptului.

\chapter{76}

\par 1 (Un psalm al lui Asaf; mai-marelui cântare?ilor. Pentru instrumentele cu coarde. Cântare catre asirieni.) Cunoscut este în Iudeea Dumnezeu; în Israel mare este numele Lui.
\par 2 Ca s-a facut în Ierusalim locul Lui ?i loca?ul Lui în Sion.
\par 3 Acolo a zdrobit taria arcurilor, arma ?i sabia ?i razboiul.
\par 4 Tu luminezi minunat din mun?ii cei ve?nici.
\par 5 Tulburatu-s-au to?i cei nepricepu?i la inima, dormit-au somnul lor ?i to?i cei razboinici nu ?i-au mai gasit mâinile.
\par 6 De certarea Ta, Dumnezeule al lui Iacob, au încremenit calare?ii pe cai.
\par 7 Tu înfrico?ator e?ti ?i cine va sta împotriva mâniei Tale?
\par 8 Din cer ai facut sa se auda judecata; pamântul s-a temut ?i s-a lini?tit,
\par 9 Când s-a ridicat la judecata Dumnezeu, ca sa mântuiasca pe to?i blânzii pamântului.
\par 10 Ca gândul omului Te va lauda ?i amintirea gândului Te va praznui.
\par 11 Face?i fagaduin?e ?i le împlini?i Domnului Dumnezeului vostru. To?i cei dimprejurul Lui vor aduce daruri
\par 12 Celui înfrico?ator ?i Celui ce ia duhurile capeteniilor, Celui înfrico?ator împara?ilor pamântului.

\chapter{77}

\par 1 (Un psalm al lui Asaf; mai-marelui cântare?ilor. Pentru Iditum.) Cu glasul meu catre Domnul am strigat, cu glasul meu catre Dumnezeu ?i a cautat spre mine.
\par 2 în ziua necazului meu pe Dumnezeu am cautat; chiar ?i noaptea mâinile mele stau întinse înaintea Lui ?i n-am slabit; sufletul n-a vrut sa se mângâie.
\par 3 Adusu-mi-am aminte de Dumnezeu ?i m-am cutremurat; gândit-am ?i a slabit duhul meu.
\par 4 Ochii mei au luat-o înainte, treji; tulburatu-m-am ?i n-am grait.
\par 5 Gândit-am la zilele cele de demult ?i de anii cei ve?nici mi-am adus aminte ?i cugetam;
\par 6 Noaptea în inima mea gândeam ?i se framânta duhul meu zicând:
\par 7 Oare, în veci ma va lepada Domnul ?i nu va mai binevoi în mine?
\par 8 Oare, pâna în sfâr?it ma va lipsi de mila Lui, din neam în neam?
\par 9 Oare, va uita sa Se milostiveasca Dumnezeu? Sau va închide în mâinile Lui îndurarile Sale?
\par 10 ?i am zis: Acum am început sa în?eleg; aceasta este schimbarea dreptei Celui Preaînalt.
\par 11 Adusu-mi-am aminte de lucrurile Domnului ?i-mi voi aduce aminte de minunile Tale, dintru început.
\par 12 ?i voi cugeta la toate lucrurile Tale ?i la faptele Tale ma voi gândi.
\par 13 Dumnezeule, în sfin?enie este calea Ta. Cine este Dumnezeu mare ca Dumnezeul nostru? Tu e?ti Dumnezeu, Care faci minuni!
\par 14 Cunoscuta ai facut între popoare puterea Ta.
\par 15 Izbavit-ai cu bra?ul Tau poporul Tau, pe fiii lui Iacob ?i ai lui Iosif.
\par 16 Vazutu-Te-au apele, Dumnezeule, vazutu-Te-au apele ?i s-au spaimântat ?i s-au tulburat adâncurile.
\par 17 Glas au dat norii ca sage?ile Tale trec.
\par 18 Glasul tunetului Tau în vârtej, luminat-au fulgerele Tale lumea, clatinatu-s-a ?i s-a cutremurat pamântul.
\par 19 În mare este calea Ta ?i cararile Tale în ape multe ?i urmele Tale nu se vor cunoa?te.
\par 20 Pova?uit-ai ca pe ni?te oi pe poporul Tau, cu mâna lui Moise ?i a lui Aaron.

\chapter{78}

\par 1 (Un psalm al lui Asaf. Al în?elegerii.) Lua?i aminte, poporul meu, la legea mea, pleca?i urechile voastre spre graiurile gurii mele.
\par 2 Deschide-voi în pilde gura mea, spune-voi cele ce au fost dintru început,
\par 3 Câte am auzit ?i am cunoscut ?i câte parin?ii no?tri ne-au înva?at.
\par 4 Nu s-au ascuns de la fiii lor, din neam în neam, vestind laudele Domnului ?i puterile Lui ?i minunile pe care le-a facut.
\par 5 ?i a ridicat marturie în Iacob ?i lege a pus în Israel. Câte a poruncit parin?ilor no?tri ca sa le arate pe ele fiilor lor, ca sa le cunoasca neamul ce va sa vina,
\par 6 Fiii ce se vor na?te ?i se vor ridica, ?i le vor vesti fiilor lor,
\par 7 Ca sa-?i puna în Dumnezeu nadejdea lor ?i sa nu uite binefacerile lui Dumnezeu ?i poruncile Lui sa le ?ina,
\par 8 Ca sa nu fie ca parin?ii lor neam îndaratnic ?i razvratit, neam care nu ?i-a îndreptat inima sa ?i nu ?i-a încredin?at lui Dumnezeu duhul sau.
\par 9 Fiii lui Efraim, arca?i înarma?i, întors-au spatele, în zi de razboi.
\par 10 N-au pazit legamântul lui Dumnezeu ?i în legea Lui n-au vrut sa umble.
\par 11 ?i au uitat facerile Lui de bine ?i minunile Lui, pe care le-a aratat lor,
\par 12 Minunile pe care le-a facut înaintea parin?ilor lor, în pamântul Egiptului, în câmpia Taneos.
\par 13 Despicat-a marea ?i i-a trecut pe ei; statut-au apele ca un zid;
\par 14 Pova?uitu-i-a pe ei cu nor ziua ?i toata noaptea cu lumina de foc;
\par 15 Despicat-a piatra în pustie ?i i-a adapat pe ei cu boga?ie de apa.
\par 16 Scos-a apa din piatra ?i au curs apele ca ni?te râuri.
\par 17 Dar ei înca au gre?it înaintea Lui, amarât-au pe Cel Preaînalt, în loc fara de apa.
\par 18 ?i au ispitit pe Dumnezeu în inimile lor, cerând mâncare sufletelor lor.
\par 19 ?i au grait împotriva lui Dumnezeu ?i au zis: "Va putea, oare, Dumnezeu sa gateasca masa în pustiu?"
\par 20 - Pentru ca a lovit piatra ?i au curs ape ?i pâraiele s-au umplut de apa. - "Oare, va putea da ?i pâine, sau va putea întinde masa poporului Sau?"
\par 21 Pentru aceasta a auzit Domnul ?i S-a mâniat ?i foc s-a aprins peste Iacob ?i mânie s-a suit peste Israel.
\par 22 Caci n-au crezut în Dumnezeu, nici n-au nadajduit în izbavirea Lui.
\par 23 ?i a poruncit norilor de deasupra ?i u?ile cerului le-a deschis
\par 24 ?i a plouat peste ei mana de mâncare ?i pâine cereasca le-a dat lor.
\par 25 Pâine îngereasca a mâncat omul; bucate le-a trimis lor din destul.
\par 26 Poruncit-a El, din cer, vânt dinspre rasarit ?i a adus cu puterea Lui vânt dinspre miazazi.
\par 27 ?i a plouat peste ei ca pulberea carnuri ?i ca nisipul marii pasari zburatoare.
\par 28 ?i au cazut în mijlocul taberei lor, împrejurul corturilor lor.
\par 29 ?i au mâncat ?i s-au saturat foarte ?i pofta lor ?i-au împlinit-o.
\par 30 Nimic nu le lipsea din cele ce pofteau ?i mâncarea le era înca în gura lor,
\par 31 Când mânia lui Dumnezeu s-a ridicat peste ei ?i a ucis pe cei satui ai lor ?i pe cei ale?i ai lui Israel i-a doborât.
\par 32 Cu toate acestea înca au mai pacatuit ?i n-au crezut în minunile Lui.
\par 33 ?i s-au stins în de?ertaciune zilele lor ?i anii lor degraba.
\par 34 Când îi ucidea pe ei, Îl cautau ?i se întorceau ?i reveneau la Dumnezeu.
\par 35 ?i ?i-au adus aminte ca Dumnezeu este ajutorul lor ?i Dumnezeul Cel Preaînalt este izbavitorul lor.
\par 36 Dar L-au în?elat pe El, cu gura lor ?i cu limba lor L-au min?it.
\par 37 În inima lor n-au fost drep?i cu El, nici n-au crezut în legamântul Lui.
\par 38 Iar El este îndurator, va cura?i pacatele ?i nu-i va nimici. Î?i va întoarce de multe ori mânia Lui ?i nu va aprinde toata urgia Lui.
\par 39 ?i-a adus aminte ca trup sunt ei, suflare ce trece ?i nu se mai întoarce.
\par 40 De câte ori L-au amarât în pustiu, L-au mâniat în pamânt fara de apa?
\par 41 ?i s-au întors ?i au ispitit pe Dumnezeu ?i pe Sfântul lui Israel L-au întarâtat.
\par 42 Nu ?i-au adus aminte de bra?ul Lui, de ziua în care i-a izbavit pe ei din mâna asupritorului.
\par 43 Ca a facut în Egipt semnele Lui ?i minunile Lui în câmpia Taneos:
\par 44 El a prefacut în sânge râurile lor ?i apele lor, ca sa nu bea.
\par 45 El a trimis asupra lor tauni ?i i-a mâncat pe ei; ?i broa?te ?i i-a prapadit pe ei.
\par 46 Dat-a stricaciunii rodul lor ?i ostenelile lor, lacustelor.
\par 47 Batut-a cu grindina via lor ?i duzii lor cu piatra.
\par 48 Dat-a grindinii dobitoacele lor ?i averea lor focului.
\par 49 Trimis-a asupra lor urgia mâniei Lui; mânie, urgie ?i necaz trimis-a prin îngeri nimicitori.
\par 50 Facut-a cale mâniei Lui; n-a cru?at de moarte sufletele lor ?i dobitoacele lor mor?ii le-a dat.
\par 51 Lovit-a pe to?i cei întâi-nascu?i din Egipt, pârga ostenelilor lor, în loca?urile lui Ham.
\par 52 Ridicat-a ca pe ni?te oi pe poporul Sau ?i i-a dus pe ei, ca pe o turma, în pustiu.
\par 53 Pova?uitu-i-a pe ei cu nadejde ?i nu s-au înfrico?at ?i pe vrajma?ii lor i-a acoperit marea.
\par 54 Dusu-i-a pe ei la hotarul sfin?eniei Lui, muntele pe care l-a dobândit dreapta Lui.
\par 55 Izgonit-a dinaintea lor neamuri ?i le-a dat lor prin sor?i pamântul de mo?tenire; ?i a a?ezat în corturile lor semin?iile lui Israel.
\par 56 Dar ei au ispitit ?i au amarât pe Dumnezeul Cel Preaînalt ?i poruncile Lui nu le-au pazit.
\par 57 ?i s-au întors ?i au calcat legamântul ca ?i parin?ii lor, întorsu-s-au ca un arc strâmb.
\par 58 ?i L-au mâniat pe El cu înal?imile lor ?i cu idolii lor L-au întarâtat pe El.
\par 59 Auzit-a Dumnezeu ?i S-a mâniat ?i a urgisit foarte pe Israel.
\par 60 A lepadat cortul Sau din ?ilo, loca?ul Lui, în care a locuit printre oameni.
\par 61 ?i a robit taria lor ?i frumuse?ea lor a dat-o în mâinile vrajma?ului.
\par 62 ?i a dat sabiei pe poporul Sau ?i de mo?tenirea Lui n-a ?inut seama.
\par 63 Pe tinerii lor i-a mistuit focul ?i fecioarele lor n-au fost înconjurate cu cinste.
\par 64 Preo?ii lor de sabie au cazut ?i vaduvele lor nu vor plânge.
\par 65 ?i S-a de?teptat Domnul ca cel ce doarme, ca un viteaz ame?it de vin,
\par 66 ?i a lovit din spate pe vrajma?ii Sai; ocara ve?nica le-a dat lor.
\par 67 ?i a lepadat cortul lui Iosif ?i semin?ia lui Efraim n-a ales-o;
\par 68 Ci a ales semin?ia lui Iuda, Muntele Sion, pe care l-a iubit.
\par 69 ?i a zidit loca?ul Sau cel sfânt, ca înal?imea cerului; pe pamânt l-a întemeiat pe el în veac.
\par 70 ?i a ales pe David robul Sau ?i l-a luat pe el de la turmele oilor.
\par 71 De lânga oile ce nasc l-a luat pe el, ca sa pasca pe Iacob, poporul Sau, ?i pe Israel, mo?tenirea Sa.
\par 72 ?i i-a pascut pe ei întru nerautatea inimii lui ?i în priceperea mâinii lui i-a pova?uit pe ei.

\chapter{79}

\par 1 (Un psalm al lui Asaf.) Dumnezeule, intrat-au neamurile în mo?tenirea Ta, pângarit-au loca?ul Tau cel sfânt, facut-au din Ierusalim ruina.
\par 2 Pus-au cadavrele robilor Tai mâncare pasarilor cerului, trupurile celor cuvio?i ai Tai, fiarelor pamântului.
\par 3 Varsat-au sângele lor ca apa împrejurul Ierusalimului ?i nu era cine sa-i îngroape.
\par 4 Facutu-ne-am ocara vecinilor no?tri, batjocura ?i râs celor dimprejurul nostru.
\par 5 Pâna când, Doamne, Te vei mânia pâna în sfâr?it? Pâna când se va aprinde ca focul mânia Ta?
\par 6 Varsa mânia Ta peste neamurile care nu Te cunosc ?i peste împara?iile care n-au chemat numele Tau.
\par 7 Ca au mâncat pe Iacob ?i locul lui l-au pustiit.
\par 8 Sa nu pomene?ti faradelegile noastre cele de demult; degraba sa ne întâmpine pe noi îndurarile Tale, ca am saracit foarte.
\par 9 Ajuta-ne noua, Dumnezeule, Mântuitorul nostru, pentru slava numelui Tau; Doamne, izbave?te-ne pe noi ?i cura?e?te pacatele noastre pentru numele Tau.
\par 10 Ca nu cumva sa zica neamurile: "Unde este Dumnezeul lor?" Sa se cunoasca între neamuri, înaintea ochilor no?tri,
\par 11 Razbunarea sângelui varsat, al robilor Tai; sa intre înaintea Ta suspinul celor fereca?i. Dupa mare?ia bra?ului Tau, paze?te pe fiii celor omorâ?i.
\par 12 Rasplate?te vecinilor no?tri de ?apte ori, în sânul lor, ocara lor cu care Te-au ocarât pe Tine, Doamne.
\par 13 Iar noi, poporul Tau ?i oile pa?unii Tale, marturisi-ne-vom ?ie, în veac, din neam în neam vom vesti lauda Ta.

\chapter{80}

\par 1 (Un psalm al lui Asaf; mai-marelui cântare?ilor. Pentru cei ce se vor schimba. Psalm pentru asirieni.) Cel ce pa?ti pe Israel, ia aminte! Cel ce pova?uie?ti ca pe o oaie pe Iosif,
\par 2 Cel ce ?ezi pe heruvimi, arata-Te înaintea lui Efraim ?i Veniamin ?i Manase. De?teapta puterea Ta ?i vino sa ne mântuie?ti pe noi.
\par 3 Dumnezeule, întoarce-ne pe noi ?i arata fa?a Ta, ?i ne vom mântui!
\par 4 Doamne, Dumnezeul puterilor, pâna când Te vei mânia de ruga robilor Tai?
\par 5 Ne vei hrani pe noi cu pâine de lacrimi ?i ne vei adapa cu lacrimi, peste masura?
\par 6 Pusu-ne-ai în cearta cu vecinii no?tri ?i vrajma?ii no?tri ne-au batjocorit pe noi.
\par 7 Doamne, Dumnezeul puterilor, întoarce-ne pe noi ?i arata fa?a Ta ?i ne vom mântui.
\par 8 Via din Egipt ai mutat-o; izgonit-ai neamuri ?i ai rasadit-o pe ea.
\par 9 Cale ai facut înaintea ei ?i ai rasadit radacinile ei ?i s-a umplut pamântul.
\par 10 Umbra ei ?i mladi?ele ei au acoperit cedrii lui Dumnezeu.
\par 11 Întins-a vi?ele ei pâna la mare ?i pâna la râu lastarele ei.
\par 12 Pentru ce ai darâmat gardul ei ?i o culeg pe ea to?i cei ce trec pe cale?
\par 13 A stricat-o pe ea mistre?ul din padure ?i porcul salbatic a pascut-o pe ea.
\par 14 Dumnezeul puterilor, întoarce-Te dar, cauta din cer ?i vezi ?i cerceteaza via aceasta,
\par 15 ?i o desavâr?e?te pe ea, pe care a sadit-o dreapta Ta, ?i pe fiul omului pe care l-ai întarit ?ie.
\par 16 Arsa a fost în foc ?i smulsa, dar de cercetarea fe?ei Tale ei vor pieri.
\par 17 Sa fie mâna Ta peste barbatul dreptei Tale ?i peste fiul omului pe care l-ai întarit ?ie.
\par 18 ?i nu ne vom departa de Tine; ne vei da via?a ?i numele Tau vom chema.
\par 19 Doamne, Dumnezeul puterilor, întoarce-ne pe noi ?i arata fa?a Ta ?i ne vom mântui.

\chapter{81}

\par 1 (Un psalm al lui Asaf; mai marelui cântare?ilor. Pentru ghitith.) Bucura?i-va de Dumnezeu, ajutorul nostru; striga?i Dumnezeului lui Iacob!
\par 2 Cânta?i psalmi ?i bate?i din timpane; cânta?i dulce din psaltire ?i din alauta!
\par 3 Suna?i din trâmbi?a, la luna noua, în ziua cea binevestita a sarbatorii noastre!
\par 4 Ca porunca pentru Israel este ?i orânduire a Dumnezeului lui Iacob.
\par 5 Marturie a pus în Iosif, când a ie?it din pamântul Egiptului, ?i a auzit limba pe care n-o ?tia:
\par 6 "Luat-am sarcina de pe umerii lui, ca mâinile lui au robit la co?uri.
\par 7 Întru necaz M-ai chemat ?i te-am izbavit, te-am auzit în mijlocul furtunii ?i te-am cercat la apa certarii.
\par 8 Asculta, poporul Meu, ?i-?i voi marturisi ?ie, Israele: De Ma vei asculta pe Mine,
\par 9 Nu vei avea alt Dumnezeu, nici nu te vei închina la dumnezeu strain,
\par 10 Ca Eu sunt Domnul Dumnezeul tau, Cel ce te-am scos din pamântul Egiptului. Deschide gura ?i o voi umple pe ea.
\par 11 Dar na ascultat poporul Meu glasul Meu ?i Israel n-a cautat la Mine.
\par 12 ?i i-am lasat sa umble dupa dorin?ele inimilor lor ?i au mers dupa cugetele lor.
\par 13 De M-ar fi ascultat poporul Meu, de ar fi umblat Israel în caile Mele,
\par 14 I-a? fi supus de tot pe vrajma?ii lor ?i a? fi pus mâna Mea pe asupritorii lor.
\par 15 Vrajma?ii Domnului L-au min?it pe El, dar le va veni timpul lor, în veac.
\par 16 Ca Domnul i-a hranit pe ei din griul cel mai ales ?i cu miere din stânca i-a saturat pe ei".

\chapter{82}

\par 1 (Un psalm al lui Asaf.) Dumnezeu a stat în dumnezeiasca adunare ?i în mijlocul dumnezeilor va judeca.
\par 2 Pâna când ve?i judeca cu nedreptate ?i la fe?ele pacato?ilor ve?i cauta?
\par 3 Judeca?i drept pe orfan ?i pe sarac ?i face?i dreptate celui smerit, celui sarman.
\par 4 Mântui?i pe cel sarman ?i pe cel sarac; din mina pacatosului, izbavi?i-i.
\par 5 Dar ei n-au cunoscut, nici n-au priceput, ci în întuneric umbla; stricase-vor toate rânduielile pamântului.
\par 6 Eu am zis: "Dumnezei sunte?i ?i to?i fii ai Celui Preaînalt".
\par 7 Dar voi ca ni?te oameni muri?i ?i ca unul din capetenii cade?i.
\par 8 Scoala-Te, Dumnezeule, judeca pamântul, ca toate neamurile sunt ale Tale.

\chapter{83}

\par 1 (Un psalm al lui Asaf.) Dumnezeule, cine se va asemana ?ie? Sa nu taci, nici sa Te lini?te?ti, Dumnezeule!
\par 2 Ca iata, vrajma?ii Tai s-au întarâtat ?i cei ce Te urasc au ridicat capul.
\par 3 Împotriva poporului Tau au lucrat cu vicle?ug ?i s-au sfatuit împotriva sfin?ilor Tai.
\par 4 Zis-au: "Veni?i sa-i pierdem pe ei dintre neamuri ?i sa nu se mai pomeneasca numele lui Israel".
\par 5 Ca s-au sfatuit într-un gând împotriva lui. Împotriva Ta legamânt au facut:
\par 6 Loca?urile Idumeilor ?i Ismaelitenii, Moabul ?i Agarenii,
\par 7 Gheval ?i Amon ?i Amalic ?i cei de alt neam, cu cei ce locuiesc în Tir.
\par 8 Ca ?i Asur a venit împreuna cu ei, ajutat-au fiilor lui Lot.
\par 9 Fa-le lor ca lui Madian ?i lui Sisara ?i ca lui Iavin, la râul Chi?on.
\par 10 Pierit-au în Endor; facutu-s-au ca gunoiul pe pamânt.
\par 11 Pune pe capeteniile lor ca pe Oriv ?i Zev ?i Zevel ?i Salmana, pe toate capeteniile lor,
\par 12 Care au zis: "Sa mo?tenim noi jertfelnicul lui Dumnezeu".
\par 13 Dumnezeul meu, pune-i pe ei ca o roata, ca trestia în fa?a vântului,
\par 14 Ca focul care arde padurea, ca vapaia care arde mun?ii,
\par 15 A?a alunga-i pe ei, în viforul Tau ?i în urgia Ta.
\par 16 Umple fe?ele lor de ocara ?i vor cauta fa?a Ta, Doamne.
\par 17 Sa se ru?ineze ?i sa se tulbure în veacul veacului ?i sa fie înfrunta?i ?i sa piara.
\par 18 ?i sa cunoasca ei ca numele Tau este Domnul. Tu singur e?ti Cel Preaînalt peste tot pamântul.

\chapter{84}

\par 1 (Mai-marelui cântare?ilor; pentru ghitith, fiilor lui Core.) Cât de iubite sunt loca?urile Tale, Doamne al puterilor!
\par 2 Dore?te ?i se sfâr?e?te sufletul meu dupa cur?ile Domnului; inima mea ?i trupul meu s-au bucurat de Dumnezeul cel viu.
\par 3 Ca pasarea ?i-a aflat casa ?i turtureaua cuib, unde-?i va pune puii sai: altarele Tale, Doamne al puterilor, Împaratul meu ?i Dumnezeul meu.
\par 4 Ferici?i sunt cei ce locuiesc în casa Ta; în vecii vecilor Te vor lauda.
\par 5 Fericit este barbatul al carui ajutor este de la Tine, Doamne; sui?uri în inima sa a pus,
\par 6 În valea plângerii, în locul care i-a fost pus. Ca binecuvântare va da Cel ce pune lege,
\par 7 Merge-vor din putere în putere, arata-Se-va Dumnezeul dumnezeilor în Sion.
\par 8 Doamne, Dumnezeul puterilor, auzi rugaciunea mea! Asculta, Dumnezeul lui Iacob!
\par 9 Aparatorul nostru, vezi Dumnezeule ?i cauta spre fa?a unsului Tau!
\par 10 Ca mai buna este o zi în cur?ile Tale decât mii. Ales-am a fi lepadat în casa lui Dumnezeu, mai bine, decât a locui în loca?urile pacato?ilor.
\par 11 Ca mila ?i adevarul iube?te Domnul; Dumnezeu har ?i slava va da. Dumnezeu nu va lipsi de bunata?i pe cei ce umbla întru nerautate.
\par 12 Doamne al puterilor, fericit este omul cel ce nadajduie?te întru Tine.

\chapter{85}

\par 1 (Mai-marelui cântare?ilor; fiilor lui Core.) Bine ai voit, Doamne, pamântului Tau, întors-ai robimea lui Iacob.
\par 2 Iertat-ai faradelegile poporului Tau, acoperit-ai toate pacatele lor.
\par 3 Potolit-ai toata mânia Ta; întorsu-Te-ai de catre iu?imea mâniei Tale.
\par 4 Întoarce-ne pe noi, Dumnezeul mântuirii noastre ?i-?i întoarce mânia Ta de la noi.
\par 5 Oare, în veci Te vei mânia pe noi? Sau vei întinde mânia Ta din neam în neam?
\par 6 Dumnezeule, Tu întorcându-Te, ne vei darui via?a ?i poporul Tau se va veseli de Tine.
\par 7 Arata-ne noua, Doamne, mila Ta ?i mântuirea Ta da-ne-o noua.
\par 8 Auzi-voi ce va grai întru mine Domnul Dumnezeu; ca va grai pace peste poporul Sau ?i peste cuvio?ii Sai ?i peste cei ce î?i întorc inima spre Dânsul.
\par 9 Dar mântuirea Lui aproape este de cei ce se tem de Dânsul, ca sa se sala?luiasca slava în pamântul nostru.
\par 10 Mila ?i adevarul s-au întâmpinat, dreptatea ?i pacea s-au sarutat.
\par 11 Adevarul din pamânt a rasarit ?i dreptatea din cer a privit.
\par 12 Ca Domnul va da bunatate ?i pamântul nostru î?i va da rodul sau;
\par 13 Dreptatea înaintea Lui va merge ?i va pune pe cale pa?ii Sai.

\chapter{86}

\par 1 (Un psalm al lui David; o rugaciune.) Pleaca, Doamne, urechea Ta ?i ma auzi, ca sarac ?i necajit sunt eu.
\par 2 Paze?te sufletul meu, caci cuvios sunt; mântuie?te, Dumnezeul meu, pe robul Tau, pe cel ce nadajduie?te în Tine.
\par 3 Miluie?te-ma, Doamne, ca spre Tine voi striga toata ziua.
\par 4 Vesele?te sufletul robului Tau, ca spre Tine, Doamne, am ridicat sufletul meu.
\par 5 Ca Tu, Doamne, bun ?i blând e?ti ?i mult-milostiv tuturor celor ce Te cheama pe Tine.
\par 6 Asculta, Doamne, rugaciunea mea ?i ia aminte la glasul cererii mele.
\par 7 În ziua necazului meu am strigat catre Tine, ca m-ai auzit.
\par 8 Nu este asemenea ?ie între dumnezei, Doamne ?i nici fapte nu sunt ca faptele Tale.
\par 9 Veni-vor toate neamurile pe care le-ai facut ?i se vor închina înaintea Ta, Doamne ?i vor slavi numele Tau.
\par 10 Ca mare e?ti Tu, Cel ce faci minuni, Tu e?ti singurul Dumnezeu.
\par 11 Pova?uie?te-ma, Doamne, pe calea Ta ?i voi merge întru adevarul Tau; veseleasca-se inima mea, ca sa se teama de numele Tau.
\par 12 Lauda-Te-voi, Doamne, Dumnezeul meu, cu toata inima mea ?i voi slavi numele Tau în veac.
\par 13 Ca mare este mila Ta spre mine ?i ai izbavit sufletul meu din iadul cel mai de jos.
\par 14 Dumnezeule, calcatorii de lege s-au sculat asupra mea ?i adunarea celor tari a cautat sufletul meu ?i nu Te-au pus pe Tine înaintea lor.
\par 15 Dar Tu, Doamne, Dumnezeu îndurat ?i milostiv e?ti; îndelung-rabdator ?i mult-milostiv ?i adevarat.
\par 16 Cauta spre mine ?i ma miluie?te, da taria Ta slugii Tale ?i mântuie?te pe fiul slujnicei Tale.
\par 17 Fa cu mine semn spre bine, ca sa vada cei ce ma urasc ?i sa se ru?ineze, ca Tu, Doamne, m-ai ajutat ?i m-ai mângâiat.

\chapter{87}

\par 1 (O cântare fiilor lui Core.) Temelia Sionului pe mun?ii cei sfin?i.
\par 2 Domnul iube?te por?ile Sionului, mai mult decât toate loca?urile lui Iacob.
\par 3 Lucruri marite s-au grait despre tine, cetatea lui Dumnezeu.
\par 4 Îmi voi aduce aminte de Raav ?i de Babilon, între cei ce ma cunosc pe mine; ?i iata cei de alt neam ?i Tirul ?i poporul etiopienilor, ace?tia acolo s-au nascut.
\par 5 Mama va zice Sionului omul ?i om s-a nascut în el ?i Însu?i Cel Preaînalt l-a întemeiat pe el.
\par 6 Domnul va povesti în cartea popoarelor ?i a capeteniilor acestora, ce s-au nascut în el.
\par 7 Ca în Tine este loca?ul tuturor celor ce se veselesc.

\chapter{88}

\par 1 (Un psalm fiilor lui Core; mai-marelui cântare?ilor, pentru maelet. Ca sa raspunda; spre pricepere, lui Etam Ezrahitul.) Doamne, Dumnezeul mântuirii mele, ziua am strigat ?i noaptea înaintea Ta.
\par 2 Sa ajunga înaintea Ta rugaciunea mea; pleaca urechea Ta spre ruga mea, Doamne,
\par 3 Ca s-a umplut de rele sufletul meu ?i via?a mea de iad s-a apropiat.
\par 4 Socotit am fost cu cei ce se coboara în groapa; ajuns-am ca un om neajutorat, între cei mor?i slobod.
\par 5 Ca ni?te oameni rani?i ce dorm în mormânt, de care nu ?i-ai mai adus aminte ?i care au fost lepada?i de la mâna Ta.
\par 6 Pusu-m-au în groapa cea mai de jos, întru cele întunecate ?i în umbra mor?ii.
\par 7 Asupra mea s-a întarâtat mânia Ta ?i toate valurile Tale le-ai adus spre mine.
\par 8 Departat-ai pe cunoscu?ii mei de la mine, ajuns-am urâciune lor.
\par 9 Închis am fost ?i n-am putut ie?i. Ochii mei au slabit de suferin?a. Strigat-am catre Tine, Doamne, toata ziua, întins-am catre Tine mâinile mele.
\par 10 Oare, mor?ilor vei face minuni? Sau cei morii se vor scula ?i Te vor lauda pe Tine?
\par 11 Oare, va spune cineva în mormânt mila Ta ?i adevarul Tau în locul pierzarii?
\par 12 Oare, se vor cunoa?te întru întuneric minunile Tale ?i dreptatea Ta în pamânt uitat?
\par 13 Iar eu catre Tine, Doamne, am strigat ?i diminea?a rugaciunea mea Te va întâmpina.
\par 14 Pentru ce Doamne, lepezi sufletul meu ?i întorci fa?a Ta de la mine?
\par 15 Sarac sunt eu ?i în osteneli din tinere?ile mele, înal?at am fost, dar m-am smerit ?i m-am mâhnit.
\par 16 Peste mine au trecut mâniile Tale ?i înfrico?arile Tale m-au tulburat.
\par 17 Înconjuratu-m-au ca apa toata ziua ?i m-au cuprins deodata.
\par 18 Departat-ai de la mine pe prieten ?i pe vecin, iar pe cunoscu?ii mei de ticalo?ia mea.

\chapter{89}

\par 1 (Un psalm spre priceperea lui Etam Ezrahitul.) Milele Tale, Doamne, în veac le voi cânta. Din neam în neam voi vesti adevarul Tau cu gura mea,
\par 2 Ca ai zis: "În veac mila se va zidi, în ceruri se va întari adevarul Tau.
\par 3 Facut-am legamânt cu ale?ii Mei, juratu-M-am lui David, robul Meu:
\par 4 Pâna în veac voi întari semin?ia ta ?i voi zidi din neam în neam scaunul tau".
\par 5 Lauda-vor cerurile minunile Tale, Doamne ?i adevarul Tau, în adunarea sfin?ilor.
\par 6 Ca cine va fi asemenea Domnului în nori ?i cine se va asemana cu Domnul între fiii lui Dumnezeu?
\par 7 Dumnezeul Cel Preamarit în sfatul sfin?ilor, mare ?i înfrico?ator este peste cei dimprejurul Lui.
\par 8 Doamne, Dumnezeul puterilor, cine este asemenea ?ie? Tare e?ti, Doamne, ?i adevarul Tau împrejurul Tau.
\par 9 Tu stapâne?ti puterea marii ?i mi?carea valurilor ei Tu o potole?ti.
\par 10 Tu ai smerit ca pe un ranit pe cel mândru; cu bra?ul puterii Tale ai risipit pe vrajma?ii Tai.
\par 11 Ale Tale sunt cerurile ?i al Tau este pamântul; lumea ?i plinirea ei Tu le-ai întemeiat.
\par 12 Miazanoapte ?i miazazi Tu ai zidit; Taborul ?i Ermonul în numele Tau se vor bucura.
\par 13 Bra?ul Tau este cu putere. Sa se întareasca mâna Ta, sa se înal?e dreapta Ta.
\par 14 Dreptatea ?i judecata sunt temelia scaunului Tau. Mila ?i adevarul vor merge înaintea fe?ei Tale.
\par 15 Fericit este poporul care cunoa?te strigat de bucurie; Doamne, în lumina fe?ei Tale vor merge
\par 16 ?i în numele Tau se vor bucura toata ziua ?i întru dreptatea Ta se vor înal?a.
\par 17 Ca lauda puterii lor Tu e?ti ?i întru buna vrerea Ta se va înal?a puterea noastra.
\par 18 Ca al Domnului este sprijinul ?i al Sfântului lui Israel, Împaratului nostru.
\par 19 Atunci ai grait în vedenii cuvio?ilor Tai ?i ai zis: Dat-am ajutor celui puternic, înal?at-am pe cel ales din poporul Meu.
\par 20 Aflat-am pe David, robul Meu; cu untdelemnul cel sfânt al Meu l-am uns pe el;
\par 21 Pentru ca mâna Mea îl va ajuta ?i bra?ul Meu îl va întari.
\par 22 Nici un vrajma? nu va izbuti împotriva lui ?i fiul faradelegii nu-i va mai face rau.
\par 23 ?i voi taia pe vrajma?ii sai de la fa?a lui ?i pe cei ce-l urasc pe el îi voi înfrânge.
\par 24 ?i adevarul Meu ?i mila Mea cu el vor fi ?i în numele Meu se va înal?a puterea lui.
\par 25 ?i voi pune peste mare mâna lui ?i peste râuri dreapta lui;
\par 26 Acesta Ma va chema: Tatal meu e?ti Tu, Dumnezeul meu ?i sprijinitorul mântuirii mele.
\par 27 ?i îl voi face pe el întâi-nascut, mai înalt decât împara?ii pamântului;
\par 28 În veac îi voi pastra mila Mea ?i legamântul Meu credincios îi va fi.
\par 29 ?i voi pune în veacul veacului semin?ia lui ?i scaunul lui ca zilele cerului;
\par 30 De vor parasi fiii lui legea Mea ?i dupa rânduielile Mele nu vor umbla,
\par 31 De vor nesocoti drepta?ile Mele ?i poruncile Mele nu vor pazi,
\par 32 Cerceta-voi cu toiag faradelegile lor ?i cu batai pacatele lor.
\par 33 Iar mila Mea nu o voi departa de la el, nici nu voi face strâmbatate întru adevarul Meu,
\par 34 Nici nu voi rupe legamântul Meu ?i cele ce ies din buzele Mele nu le voi schimba.
\par 35 O data m-am jurat pe sfin?enia Mea: Oare, voi min?i pe David?
\par 36 Semin?ia lui în veac va ramâne ?i scaunul lui ca soarele înaintea Mea
\par 37 ?i ca luna întocmita în veac ?i martor credincios în cer.
\par 38 Dar Tu ai lepadat, ai defaimat ?i ai aruncat pe unsul Tau.
\par 39 Stricat-ai legamântul robului Tau, batjocorit-ai pe pamânt sfin?enia lui.
\par 40 Doborât-ai toate gardurile lui, facut-ai întariturile lui ruina.
\par 41 Jefuitu-l-au pe el to?i cei ce treceau pe cale, ajuns-a ocara vecinilor sai.
\par 42 Înal?at-ai dreapta vrajma?ilor lui, veselit-ai pe to?i du?manii lui.
\par 43 Luat-ai puterea sabiei lui ?i nu l-ai ajutat în vreme de razboi.
\par 44 Nimicit-ai cura?enia lui ?i scaunul lui la pamânt l-ai doborât.
\par 45 Mic?orat-ai zilele vie?ii lui, umplutu-l-ai de ru?ine.
\par 46 Pâna când, Doamne, Te vei întoarce? Pâna când se va aprinde ca focul mânia Ta?
\par 47 Adu-?i aminte de mine; oare, în de?ert ai zidit pe to?i fiii oamenilor?
\par 48 Cine este omul ca sa traiasca ?i sa nu vada moartea ?i sa-?i izbaveasca sufletul sau din mâna iadului?
\par 49 Unde sunt milele Tale cele de demult, Doamne, pe care le-ai jurat lui David, întru adevarul Tau?
\par 50 Adu-?i aminte, Doamne, de ocara robilor Tai, pe care o port în sânul lneu, de la multe neamuri.
\par 51 Adu-?i aminte, Doamne, de ocara cu care m-au ocarât vrajma?ii Tai, cu care au ocarât pa?ii unsului Tau.
\par 52 Binecuvântat este Domnul în veci. Amin. Amin.

\chapter{90}

\par 1 (Rugaciunea lui Moise, omul lui Dumnezeu.) Doamne, scapare Te-ai facut noua în neam ?i în neam.
\par 2 Mai înainte de ce s-au facut mun?ii ?i s-a zidit pamântul ?i lumea, din veac ?i pâna în veac e?ti Tu.
\par 3 Nu întoarce pe om întru smerenie, Tu, care ai zis: "Întoarce?i-va, fii ai oamenilor",
\par 4 Ca o mie de ani înaintea ochilor Tai sunt ca ziua de ieri, care a trecut ?i ca straja nop?ii.
\par 5 Nimicnicie vor fi anii lor; diminea?a ca iarba va trece.
\par 6 Diminea?a va înflori ?i va trece, seara va cadea, se va întari ?i se va usca.
\par 7 Ca ne-am sfâr?it de urgia Ta ?i de mânia Ta ne-am tulburat.
\par 8 Pus-ai faradelegile noastre înaintea Ta, gre?elile noastre ascunse, la lumina fe?ei Tale.
\par 9 Ca toate zilele noastre s-au împu?inat ?i în mânia Ta ne-am stins.
\par 10 Anii no?tri s-au socotit ca pânza unui paianjen; zilele anilor no?tri sunt ?aptezeci de ani; iar de vor fi în putere optzeci de ani ?i ce este mai mult decât ace?tia osteneala ?i durere; ca trece via?a noastra ?i ne vom duce.
\par 11 Cine cunoa?te puterea urgiei Tale ?i cine masoara mânia Ta, dupa temerea de Tine?
\par 12 Înva?a-ne sa socotim bine zilele noastre, ca sa ne îndreptam inimile spre în?elepciune.
\par 13 Întoarce-Te, Doamne! Pâna când vei sta departe? Mângâie pe robii Tai!
\par 14 Umplutu-ne-am diminea?a de mila Ta ?i ne-am bucurat ?i ne-am veselit în toate zilele vie?ii noastre.
\par 15 Veselitu-ne-am pentru zilele în care ne-ai smerit, pentru anii în care am vazut rele.
\par 16 Cauta spre robii Tai ?i spre lucrurile Tale ?i îndrepteaza pe fiii lor.
\par 17 ?i sa fie lumina Domnului Dumnezeului nostru peste noi ?i lucrurile mâinilor noastre le îndrepteaza.

\chapter{91}

\par 1 (Un psalm al lui David.) Cel ce locuie?te în ajutorul Celui Preaînalt, întru acoperamântul Dumnezeului cerului se va sala?lui.
\par 2 Va zice Domnului: "Sprijinitorul meu e?ti ?i scaparea mea; Dumnezeul meu, voi nadajdui spre Dânsul".
\par 3 Ca El te va izbavi din cursa vânatorilor ?i de cuvântul tulburator.
\par 4 Cu spatele te va umbri pe tine ?i sub aripile Lui vei nadajdui; ca o arma te va înconjura adevarul Lui.
\par 5 Nu te vei teme de frica de noapte, de sageata ce zboara ziua,
\par 6 De lucrul ce umbla în întuneric, de molima ce bântuie întru amiaza.
\par 7 Cadea-vor dinspre latura ta o mie ?i zece mii de-a dreapta ta, dar de tine nu se vor apropia.
\par 8 Însa cu ochii tai vei privi ?i rasplatirea pacato?ilor vei vedea.
\par 9 Pentru ca pe Domnul, nadejdea mea, pe Cel Preaînalt L-ai pus scapare ?ie.
\par 10 Nu vor veni catre tine rele ?i bataie nu se va apropia de loca?ul tau.
\par 11 Ca îngerilor Sai va porunci pentru tine ca sa te pazeasca în toate caile tale.
\par 12 Pe mâini te vor înal?a ca nu cumva sa împiedici de piatra piciorul tau.
\par 13 Peste aspida ?i vasilisc vei pa?i ?i vei calca peste leu ?i peste balaur.
\par 14 "Ca spre Mine a nadajduit ?i-l voi izbavi pe el, zice Domnul; îl voi acoperi pe el ca a cunoscut numele Meu.
\par 15 Striga-va catre Mine ?i-l voi auzi pe el; cu dânsul sunt în necaz ?i-l voi scoate pe el ?i-l voi slavi.
\par 16 Cu lungime de zile îl voi umple pe el, ?i-i voi arata lui mântuirea Mea".

\chapter{92}

\par 1 (Un psalm pentru ziua sâmbetei.) Bine este a lauda pe Domnul ?i a cânta numele Tau, Preaînalte,
\par 2 A vesti diminea?a mila Ta ?i adevarul Tau în toata noaptea,
\par 3 În psaltire cu zece strune, cu cântece din alauta.
\par 4 Ca m-ai veselit, Doamne, întru fapturile Tale ?i întru lucrurile mâinilor Tale ma voi bucura.
\par 5 Cât s-au marit lucrurile Tale, Doamne, adânci cu totul sunt gândurile Tale!
\par 6 Barbatul nepriceput nu va cunoa?te ?i cel neîn?elegator nu va în?elege acestea,
\par 7 Când rasar pacato?ii ca iarba ?i se ivesc to?i cei ce fac faradelegea,
\par 8 Ca sa piara în veacul veacului. Iar Tu, Preaînalt e?ti în veac, Doamne.
\par 9 Ca iata vrajma?ii Tai, Doamne, vor pieri ?i se vor risipi to?i cei ce lucreaza faradelegea.
\par 10 ?i se va înal?a puterea mea ca a inorogului ?i batrâne?ile mele unse din bel?ug.
\par 11 ?i a privit ochiul meu catre vrajma?ii mei ?i pe cei vicleni, ce se ridica împotriva mea, îi va auzi urechea mea.
\par 12 Dreptul ca finicul va înflori ?i ca cedrul cel din Liban se va înmul?i.
\par 13 Rasadi?i fiind în casa Domnului, în cur?ile Dumnezeului nostru vor înflori.
\par 14 Înca întru batrâne?e unse se vor înmul?i ?i bine vie?uind vor fi, ca sa vesteasca:
\par 15 Drept este Domnul Dumnezeul nostru ?i nu este nedreptate întru Dânsul.

\chapter{93}

\par 1 (O cântare de lauda a lui David. Pentru ziua dinaintea sâmbetei, când s-a locuit pamântul.) Domnul a împara?it, întru podoaba S-a îmbracat; îmbracatu-S-a Domnul întru putere ?i S-a încins, pentru ca a întarit lumea care nu se va clinti.
\par 2 Gata este scaunul Tau de atunci, din veac e?ti Tu.
\par 3 Ridicat-au râurile, Doamne, ridicat-au râurile glasurile lor,
\par 4 Ridicat-au râurile valurile lor; dar mai mult decât glasul apelor clocotitoare, mai mult decât zbuciumul neasemuit al marii, minunat este, întru cele înalte, Domnul.
\par 5 Marturiile Tale s-au adeverit foarte. Casei Tale se cuvine sfin?enie, Doamne, întru lungime de zile.

\chapter{94}

\par 1 (Un psalm al lui David. Pentru ziua a patra a sâmbetei.) Dumnezeul razbunarilor, Domnul, Dumnezeul razbunarilor cu îndrazneala a grait.
\par 2 Înal?a-Te Cel ce judeci pamântul, rasplate?te rasplatirea celor mândri.
\par 3 Pâna când pacato?ii, Doamne, pâna când pacato?ii se vor fali?
\par 4 Pâna când vor spune ?i vor grai nedreptate; grai-vor to?i cei ce lucreaza faradelegea?
\par 5 Pe poporul Tau, Doamne, l-au asuprit ?i mo?tenirea Ta au apasat-o.
\par 6 Pe vaduva ?i pe sarac au ucis ?i pe orfani i-au omorât.
\par 7 ?i au zis: "Nu va vedea Domnul, nici nu va pricepe Dumnezeul lui Iacob".
\par 8 În?elege?i, dar, cei neîn?elep?i din popor, ?i cei nebuni, în?elep?i?i-va odata!
\par 9 Cel ce a sadit urechea, oare, nu aude? Cel ce a zidit ochiul, oare, nu prive?te?
\par 10 Cel ce pedepse?te neamurile, oare, nu va certa? Cel ce înva?a pe om cuno?tin?a,
\par 11 Domnul, cunoa?te gândurile oamenilor, ca sunt de?arte.
\par 12 Fericit este omul pe care îl vei certa, Doamne, ?i din legea Ta îl vei înva?a pe el,
\par 13 Ca sa-l lini?te?ti pe el în zile rele, pâna ce se va sapa groapa pacatosului.
\par 14 Ca nu va lepada Domnul pe poporul Sau ?i mo?tenirea Sa nu o va parasi,
\par 15 Pâna ce dreptatea se va întoarce la judecata ?i to?i cei cu inima curata, care se ?in de dânsa.
\par 16 Cine se va ridica cu mine împotriva celor ce viclenesc ?i cine va sta împreuna cu mine împotriva celor ce lucreaza faradelegea?
\par 17 Ca de nu mi-ar fi ajutat mie Domnul, pu?in de nu s-ar fi sala?luit în iad sufletul meu.
\par 18 Când am zis: "S-a clatinat piciorul meu", mila Ta, Doamne, mi-a ajutat mie.
\par 19 Doamne, când s-au înmul?it durerile mele în inima mea, mângâierile Tale au veselit sufletul meu.
\par 20 Nu va sta împreuna cu Tine scaunul faradelegii, cel ce face asuprire împotriva legii.
\par 21 Ei vor prinde în cursa sufletul dreptului ?i sânge nevinovat vor osândi.
\par 22 Dar Domnul mi-a fost mie scapare ?i Dumnezeul meu ajutorul nadejdii mele;
\par 23 Le va rasplati lor Domnul dupa faradelegea lor ?i dupa rautatea lor îi va pierde pe ei Domnul Dumnezeul nostru.

\chapter{95}

\par 1 (O cântare de lauda a lui David.) Veni?i sa ne bucuram de Domnul ?i sa strigam lui Dumnezeu, Mântuitorului nostru.
\par 2 Sa întâmpinam fa?a Lui întru lauda ?i în psalmi sa-I strigam Lui,
\par 3 Ca Dumnezeu mare este Domnul ?i Împarat mare peste tot pamântul.
\par 4 Ca în mâna Lui sunt marginile pamântului ?i înal?imile mun?ilor ale Lui sunt.
\par 5 Ca a Lui este marea ?i El a facut-o pe ea, ?i uscatul mâinile Lui l-au zidit.
\par 6 Veni?i sa ne închinam ?i sa cadem înaintea Lui ?i sa plângem înaintea Domnului, Celui ce ne-a facut pe noi.
\par 7 Ca El este Dumnezeul nostru ?i noi poporul pa?unii Lui ?i oile mâinii Lui.
\par 8 O, de I-a?i auzi glasul care zice: "Sa nu va învârto?a?i inimile voastre, ca în timpul cercetarii, ca în ziua ispitirii în pustiu,
\par 9 Unde M-au ispitit parin?ii vo?tri, M-au ispitit ?i au vazut lucrurile Mele.
\par 10 Patruzeci de ani am urât neamul acesta ?i am zis: "Pururea ratacesc cu inima".
\par 11 ?i ei n-au cunoscut caile Mele, ca M-am jurat întru mânia Mea: "Nu vor intra întru odihna Mea".

\chapter{96}

\par 1 (O cântare a lui David, când s-a zidit casa, dupa robie. La Evrei fara titlu.) Cânta?i Domnului cântare noua, cânta?i Domnului tot pamântul.
\par 2 Cânta?i Domnului, binecuvânta?i numele Lui, binevesti?i din zi în zi mântuirea Lui.
\par 3 Vesti?i între neamuri slava Lui, între toate popoarele minunile Lui;
\par 4 Ca mare este Domnul ?i laudat foarte, înfrico?ator este; mai presus decât to?i dumnezeii.
\par 5 Ca to?i dumnezeii neamurilor sunt idoli; iar Domnul cerurile a facut.
\par 6 Lauda ?i frumuse?e este înaintea Lui, sfin?enie ?i mare?ie în loca?ul cel sfânt al Lui.
\par 7 Aduce?i Domnului, semin?iile popoarelor, aduce?i Domnului slava ?i cinste; aduce?i Domnului slava numelui Lui.
\par 8 Aduce?i jertfe ?i intra?i în cur?ile Lui. Închina?i-va Domnului în curtea cea sfânta a Lui.
\par 9 Sa tremure de fa?a Lui tot pamântul. Spune?i între neamuri ca Domnul a împara?it,
\par 10 Pentru ca a întarit lumea care nu se va clinti; judeca-va popoare întru dreptate.
\par 11 Sa se veseleasca cerurile ?i sa se bucure pamântul, sa se zguduie marea ?i toate cele ce sunt întru ea; sa se bucure câmpiile ?i toate cele ce sunt pe ele.
\par 12 Atunci se vor bucura to?i copacii padurii, de fa?a Domnului, ca vine, vine sa judece pamântul.
\par 13 Judeca-va lumea întru dreptate ?i popoarele întru adevarul Sau.

\chapter{97}

\par 1 (Un psalm al lui David, când s-a întarit domnia lui.) Domnul împara?e?te! Sa se bucure pamântul, sa se veseleasca insule multe.
\par 2 Nor ?i negura împrejurul Lui, dreptatea ?i judecata este temelia neamului Lui.
\par 3 Foc înaintea Lui va merge ?i va arde împrejur pe vrajma?ii Lui.
\par 4 Luminat-au fulgerele Lui lumea; vazut-a ?i s-a cutremurat pamântul.
\par 5 Mun?ii ca ceara s-au topit de fa?a Domnului, de fa?a Domnului a tot pamântul.
\par 6 Vestit-au cerurile dreptatea Lui ?i au vazut toate popoarele slava Lui.
\par 7 Sa se ru?ineze to?i cei ce se închina chipurilor cioplite ?i se lauda cu idolii lor. Închina?i-va Lui to?i îngerii Lui;
\par 8 Auzit-a ?i s-a veselit Sionul ?i s-au bucurat fiicele Iudeii, pentru judeca?ile Tale, Doamne.
\par 9 Ca Tu e?ti Domnul Cel Preaînalt peste tot pamântul; înal?atu-Te-ai foarte, mai presus decât to?i dumnezeii.
\par 10 Cei ce iubi?i pe Domnul, urâ?i raul; Domnul paze?te sufletele cuvio?ilor Lui; din mâna pacatosului îi va izbavi pe ei.
\par 11 Lumina a rasarit dreptului ?i celor drep?i cu inima, veselie.
\par 12 Veseli?i-va, drep?ilor, în Domnul ?i lauda?i pomenirea sfin?eniei Lui!

\chapter{98}

\par 1 (Un psalm al lui David.) Cânta?i Domnului cântare noua, ca lucruri minunate a facut Domnul. Mântuitu-l-a pe el dreapta Lui ?i bra?ul cel sfânt al Lui.
\par 2 Cunoscuta a facut Domnul mântuirea Sa; înaintea neamurilor a descoperit dreptatea Sa.
\par 3 Pomenit-a mila Sa lui Iacob ?i adevarul Sau casei lui Israel; vazut-au toate marginile pamântului mântuirea Dumnezeului nostru.
\par 4 Striga?i lui Dumnezeu tot pamântul; cânta?i ?i va bucura?i ?i cânta?i.
\par 5 Cânta?i Domnului cu alauta, cu alauta ?i în sunet de psaltire;
\par 6 Cu trâmbi?e ?i în sunet de corn, striga?i înaintea Împaratului ?i Domnului.
\par 7 Sa se zguduie marea ?i plinirea ei, lumea ?i cei ce locuiesc în ea.
\par 8 Râurile vor bate din palme, deodata, mun?ii se vor bucura de fa?a Domnului, ca vine, vine sa judece pamântul.
\par 9 Judeca-va lumea cu dreptate ?i popoarele cu nepartinire.

\chapter{99}

\par 1 (Un psalm al lui David.) Domnul împara?e?te: sa tremure popoarele; ?ade pe heruvimi: sa se cutremure pamântul.
\par 2 Domnul în Sion este mare ?i înalt peste toate popoarele.
\par 3 Sa se laude numele Tau cel mare, ca înfrico?ator ?i sfânt este. ?i cinstea împaratului iube?te dreptatea.
\par 4 Tu ai întemeiat dreptatea; judecata ?i dreptate în Iacob, Tu ai facut.
\par 5 Înal?a?i pe Domnul Dumnezeul nostru ?i va închina?i a?ternutului picioarelor Lui, ca sfânt este!
\par 6 Moise ?i Aaron, între preo?ii Lui ?i Samuel între cei ce cheama numele Lui. Chemat-au pe Domnul ?i El i-a auzit pe ei,
\par 7 În stâlp de nor graia catre ei; caci pazeau marturiile Lui ?i poruncile pe care le-a dat lor.
\par 8 Doamne, Dumnezeul nostru, Tu i-ai auzit pe ei; Dumnezeule, Tu Te-ai milostivit de ei ?i ai rasplatit toate faptele lor.
\par 9 Înal?a?i pe Domnul Dumnezeul nostru ?i va închina?i în Muntele cel sfânt al Lui, ca sfânt este Domnul, Dumnezeul nostru!

\chapter{100}

\par 1 (Un psalm al lui David; spre lauda.) Striga?i Domnului tot pamântul,
\par 2 Sluji?i Domnului cu veselie, intra?i înaintea Lui cu bucurie.
\par 3 Cunoa?te?i ca Domnul, El este Dumnezeul nostru; El ne-a facut pe noi ?i nu noi.
\par 4 Iar noi poporul Lui ?i oile pa?unii Lui. Intra?i pe por?ile Lui cu lauda ?i în cur?ile Lui cu cântari lauda?i-L pe El.
\par 5 Cânta?i numele Lui! Ca bun este Domnul; în veac este mila Lui ?i din neam în neam adevarul Lui.

\chapter{101}

\par 1 (Un psalm al lui David.) Mila ?i judecata Ta voi cânta ?ie, Doamne.
\par 2 Cânta-voi ?i voi merge, cu pricepere, în cale fara prihana. Când vei veni la mine? Umblat-am întru nerautatea inimii mele, în casa mea.
\par 3 N-am pus înaintea ochilor mei lucru nelegiuit; pe calcatorii de lege i-am urât.
\par 4 Nu s-a lipit de mine inima îndaratnica; pe cel rau, care se departa de mine, nu l-am cunoscut.
\par 5 Pe cel ce clevetea în ascuns pe vecinul sau, pe acela l-am izgonit. Cu cel mândru cu ochiul ?i nesa?ios cu inima, cu acela n-am mâncat.
\par 6 Ochii mei sunt peste credincio?ii pamântului, ca sa ?ada ei împreuna cu mine. Cel ce umbla pe cale fara prihana, acela îmi slujea.
\par 7 Nu va locui în casa mea cel mândru; cel ce graie?te nedrepta?i nu va sta înaintea ochilor mei.
\par 8 În dimine?i voi judeca pe to?i pacato?ii pamântului, ca sa nimicesc din cetatea Domnului pe to?i cei ce lucreaza faradelegea.

\chapter{102}

\par 1 (Rugaciunea unui sarac mâhnit, care-?i îndrepteaza înaintea Domnului ruga sa.) Doamne, auzi rugaciunea mea, ?i strigarea mea la Tine sa ajunga!
\par 2 Sa nu întorci fa?a Ta de la mine; în orice zi ma necajesc, pleaca spre mine urechea Ta! În orice zi Te voi chema, degraba auzi-ma!
\par 3 Ca s-au stins ca fumul zilele mele ?i oasele mele ca uscaciunea s-au facut.
\par 4 Ranita este inima mea ?i s-a uscat ca iarba; ca am uitat sa-mi manânc pâinea mea.
\par 5 De glasul suspinului meu, osul meu s-a lipit de carnea mea.
\par 6 Asemanatu-m-am cu pelicanul din pustiu; ajuns-am ca bufni?a din darâmaturi.
\par 7 Privegheat-am ?i am ajuns ca o pasare singuratica pe acoperi?.
\par 8 Toata ziua m-au ocarât vrajma?ii mei ?i cei ce ma laudau, împotriva mea se jurau.
\par 9 Ca cenu?a am mâncat, în loc de pâine, ?i bautura mea cu plângere am amestecat-o,
\par 10 Din pricina urgiei Tale ?i a mâniei Tale; ca ridicându-ma eu, m-ai surpat.
\par 11 Zilele mele ca umbra s-au plecat ?i eu ca iarba m-am uscat.
\par 12 Iar Tu, Doamne, în veac ramâi ?i pomenirea Ta din neam în neam.
\par 13 Sculându-Te, vei milui Sionul, ca vremea este sa-l miluie?ti pe el, ca a venit vremea.
\par 14 Ca au iubit robii Tai pietrele lui ?i de ?arâna lui le va fi mila.
\par 15 ?i se vor teme neamurile de numele Domnului ?i to?i împara?ii pamântului de slava Ta.
\par 16 Ca va zidi Domnul Sionul ?i se va arata întru slava Sa.
\par 17 Cautat-a spre rugaciunea celor smeri?i ?i n-a dispre?uit cererea lor.
\par 18 Sa se scrie acestea pentru neamul ce va sa vina ?i poporul ce se zide?te va lauda pe Domnul;
\par 19 Ca a privit din înal?imea cea sfânta a Lui, Domnul din cer pe pamânt a privit,
\par 20 Ca sa auda suspinul celor fereca?i, sa dezlege pe fiii celor omorâ?i,
\par 21 Sa vesteasca în Sion numele Domnului ?i lauda Lui în Ierusalim,
\par 22 Când se vor aduna popoarele împreuna ?i împara?iile ca sa slujeasca Domnului.
\par 23 Zis-am catre Dumnezeu în calea tariei Lui: Veste?te-mi pu?inatatea zilelor mele.
\par 24 Nu ma lua la jumatatea zilelor mele, ca anii Tai, Doamne, sunt din neam în neam.
\par 25 Dintru început Tu, Doamne, pamântul l-ai întemeiat ?i lucrul mâinilor Tale, sunt cerurile.
\par 26 Acelea vor pieri, iar Tu vei ramâne ?i to?i ca o haina se vor învechi ?i ca un ve?mânt îi vei schimba ?i se vor schimba.
\par 27 Dar Tu acela?i e?ti ?i anii Tai nu se vor împu?ina.
\par 28 Fiii robilor Tai vor locui pamântul lor ?i semin?ia lor în veac va propa?i.

\chapter{103}

\par 1 (Un psalm al lui David.) Binecuvinteaza, suflete al meu, pe Domnul ?i toate cele dinlauntrul meu, numele cel sfânt al Lui.
\par 2 Binecuvinteaza, suflete al meu, pe Domnul ?i nu uita toate rasplatirile Lui.
\par 3 Pe Cel ce cura?e?te toate faradelegile tale, pe Cel ce vindeca toate bolile tale;
\par 4 Pe Cel ce izbave?te din stricaciune via?a ta, pe Cel ce te încununeaza cu mila ?i cu îndurari;
\par 5 Pe Cel ce umple de bunata?i pofta ta; înnoi-se-vor ca ale vulturului tinere?ile tale.
\par 6 Cel ce face milostenie, Domnul, ?i judecata tuturor celor ce li se face strâmbatate.
\par 7 Cunoscute a facut caile Sale lui Moise, fiilor lui Israel voile Sale.
\par 8 Îndurat ?i milostiv este Domnul, îndelung-rabdator ?i mult-milostiv.
\par 9 Nu pâna în sfâr?it se va iu?i, nici în veac se va mânia.
\par 10 Nu dupa pacatele noastre a facut noua, nici dupa faradelegile noastre a rasplatit noua,
\par 11 Ci cât este departe cerul de pamânt, atât este de mare mila Lui, spre cei ce se tem de El.
\par 12 Pe cât sunt de departe rasariturile de la apusuri, departat-a de la noi faradelegile noastre.
\par 13 În ce chip miluie?te tatal pe fii, a?a a miluit Domnul pe cei ce se tem de El;
\par 14 Ca El a cunoscut zidirea noastra, adusu-?i-a aminte ca ?arâna suntem.
\par 15 Omul ca iarba, zilele lui ca floarea câmpului; a?a va înflori.
\par 16 Ca vânt a trecut peste el ?i nu va mai fi ?i nu se va mai cunoa?te înca locul sau.
\par 17 Iar mila Domnului din veac în veac spre cei ce se tem de Dânsul,
\par 18 ?i dreptatea Lui spre fiii fiilor, spre cei ce pazesc legamântul Lui
\par 19 ?i î?i aduc aminte de poruncile Lui, ca sa le faca pe ele. Domnul în cer a gatit scaunul Sau ?i împara?ia Lui peste to?i stapâne?te.
\par 20 Binecuvânta?i pe Domnul to?i îngerii Lui, cei tari la vârtute, care face?i cuvântul Lui ?i auzi?i glasul cuvintelor Lui.
\par 21 Binecuvânta?i pe Domnul toate puterile Lui, slugile Lui, care face?i voia Lui.
\par 22 Binecuvânta?i pe Domnul toate lucrurile Lui; în tot locul stapânirii Lui, binecuvinteaza suflete al meu pe Domnul.

\chapter{104}

\par 1 (Un psalm al lui David.) Binecuvinteaza, suflete al meu, pe Domnul! Doamne, Dumnezeul meu, maritu-Te-ai foarte.
\par 2 Întru stralucire ?i în mare podoaba Te-ai îmbracat; Cel ce Te îmbraci cu lumina ca ?i cu o haina;
\par 3 Cel ce întinzi cerul ca un cort; Cel ce acoperi cu ape cele mai de deasupra ale lui; Cel ce pui norii suirea Ta; Cel ce umbli peste aripile vânturilor;
\par 4 Cel ce faci pe îngerii Tai duhuri ?i pe slugile Tale para de foc;
\par 5 Cel ce ai întemeiat pamântul pe întarirea lui ?i nu se va clatina în veacul veacului.
\par 6 Adâncul ca o haina este îmbracamintea lui; peste mun?i vor sta ape.
\par 7 De certarea Ta vor fugi, de glasul tunetului Tau se vor înfrico?a.
\par 8 Se suie mun?i ?i se coboara vai, în locul în care le-ai întemeiat pe ele.
\par 9 Hotar ai pus, pe care nu-l vor trece ?i nici nu se vor întoarce sa acopere pamântul.
\par 10 Cel ce trimi?i izvoare în vai, prin mijlocul mun?ilor vor trece ape;
\par 11 Adapa-se-vor toate fiarele câmpului, asinii salbatici setea î?i vor potoli.
\par 12 Peste acelea pasarile cerului vor locui; din mijlocul stâncilor vor da glas.
\par 13 Cel ce adapi mun?ii din cele mai de deasupra ale Tale, din rodul lucrurilor Tale se va satura pamântul.
\par 14 Cel ce rasari iarba dobitoacelor ?i verdea?a spre slujba oamenilor;
\par 15 Ca sa scoata pâine din pamânt ?i vinul vesele?te inima omului;
\par 16 Ca sa veseleasca fa?a cu untdelemn ?i pâinea inima omului o întare?te.
\par 17 Satura-se-vor copacii câmpului, cedrii Libanului pe care i-ai sadit; acolo pasarile î?i vor face cuib.
\par 18 Loca?ul cocostârcului în chiparo?i. Mun?ii cei înal?i adapost cerbilor stâncile scapare iepurilor.
\par 19 Facut-ai luna spre vremi, soarele ?i-a cunoscut apusul sau.
\par 20 Pus-ai întuneric ?i s-a facut noapte, când vor ie?i toate fiarele padurii;
\par 21 Puii leilor mugesc ca sa apuce ?i sa ceara de la Dumnezeu mâncarea lor.
\par 22 Rasarit-a soarele ?i s-au adunat ?i în culcu?urile lor se vor culca.
\par 23 Ie?i-va omul la lucrul sau ?i la lucrarea sa pâna seara.
\par 24 Cât s-au marit lucrurile Tale, Doamne, toate cu în?elepciune le-ai facut! Umplutu-s-a pamântul de zidirea Ta.
\par 25 Marea aceasta este mare ?i larga; acolo se gasesc târâtoare, carora nu este numar, vieta?i mici ?i mari.
\par 26 Acolo corabiile umbla; balaurul acesta pe care l-ai zidit, ca sa se joace în ea.
\par 27 Toate catre Tine a?teapta ca sa le dai lor hrana la buna vreme.
\par 28 Dându-le Tu lor, vor aduna, deschizând Tu mâna Ta, toate se vor umple de bunata?i;
\par 29 Dar întorcându-?i Tu fa?a Ta, se vor tulbura; lua-vei duhul lor ?i se vor sfâr?i ?i în ?arâna se vor întoarce.
\par 30 Trimite-vei duhul Tau ?i se vor zidi ?i vei înnoi fa?a pamântului.
\par 31 Fie slava Domnului în veac! Veseli-se-va Domnul de lucrurile Sale.
\par 32 Cel ce cauta spre pamânt ?i-l face pe el de se cutremura; Cel ce se atinge de mun?i ?i fumega.
\par 33 Cânta-voi Domnului în via?a mea, cânta-voi Dumnezeului meu cât voi fi.
\par 34 Placute sa-I fie Lui cuvintele mele, iar eu ma voi veseli de Domnul.
\par 35 Piara pacato?ii de pe pamânt ?i cei fara de lege, ca sa nu mai fie. Binecuvinteaza, suflete al meu, pe Domnul.

\chapter{105}

\par 1 Aliluia! Lauda?i pe Domnul ?i chema?i numele Lui; vesti?i între neamuri lucrurile Lui.
\par 2 Cânta?i-I ?i-L lauda?i pe El; spune?i toate minunile Lui.
\par 3 Lauda?i-va cu numele cel sfânt al Lui; veseleasca-se inima celor ce cauta pe Domnul.
\par 4 Cauta?i pe Domnul ?i va întari?i; cauta?i fala Lui, pururea.
\par 5 Aduce?i-va aminte de minunile Lui, pe care le-a facut; de minunile Lui ?i de judeca?ile gurii Lui.
\par 6 Semin?ia lui Avraam, robii Lui, fiii lui Iacob, ale?ii Lui.
\par 7 Acesta este Domnul Dumnezeul nostru, în tot pamântul judeca?ile Lui.
\par 8 Adusu-?i-a aminte în veac de legamântul Lui, de cuvântul pe care l-a poruncit într-o mie de neamuri,
\par 9 Pe care l-a încheiat cu Avraam ?i de juramântul Sau lui Isaac.
\par 10 ?i l-a pus pe el lui Iacob, spre porunca, ?i lui Israel legatura ve?nica,
\par 11 Zicând: "?ie î?i voi da pamântul Canaan, partea mo?tenirii tale".
\par 12 Atunci când erau ei pu?ini la numar ?i straini în pamântul lor
\par 13 ?i au trecut de la un neam la altul, de la o împara?ie la un alt popor,
\par 14 N-a lasat om sa le faca strâmbatate ?i a certat pentru ei pe împara?i, zicându-le:
\par 15 "Nu va atinge?i de un?ii Mei ?i nu vicleni?i împotriva profe?ilor Mei".
\par 16 ?i a chemat foamete pe pamânt ?i a sfarâmat paiul de grâu.
\par 17 Trimis-a înaintea lor om; rob a fost rânduit Iosif.
\par 18 Smeritu-l-au, punând în obezi picioarele lui; prin fier a trecut sufletul lui,
\par 19 Pâna ce a venit cuvântul Lui. Cuvântul Domnului l-a aprins pe el;
\par 20 Trimis-a împaratul ?i l-a slobozit, capetenia poporului ?i l-a liberat pe el.
\par 21 Pusu-l-a pe el domn casei lui ?i capetenie peste toata avu?ia lui,
\par 22 Ca sa înve?e pe capeteniile lui, ca pe sine însu?i ?i pe batrânii lui sa-i în?elep?easca.
\par 23 ?i a intrat Israel în Egipt ?i Iacob a locuit ca strain, în pamântul lui Ham.
\par 24 ?i a înmul?it pe poporul lui foarte ?i l-a întarit pe el mai mult decât pe vrajma?ii lui.
\par 25 Întors-a inima lor, ca sa urasca pe poporul Sau, ca sa vicleneasca împotriva robilor Sai.
\par 26 Trimis-a pe Moise robul Sau, pe Aaron, pe care l-a ales.
\par 27 Pus-a în ei cuvintele semnelor ?i minunilor Lui în pamântul lui Ham.
\par 28 Trimis-a întuneric ?i i-a întunecat, caci au amarât cuvintele Lui;
\par 29 Prefacut-a apele lor în sânge ?i a omorât pe?tii lor;
\par 30 Scos-a pamântul lor broa?te în camarile împara?ilor lor.
\par 31 Zis-a ?i a venit musca câineasca ?i mul?ime de mu?te în toate hotarele lor.
\par 32 Pus-a în ploile lor grindina, foc arzator în pamântul lor;
\par 33 ?i a batut viile lor ?i smochinii lor ?i a sfarâmat pomii hotarelor lor.
\par 34 Zis-a ?i a venit lacusta ?i omida fara numar.
\par 35 ?i a mâncat toata iarba în pamântul lor ?i a mâncat rodul pamântului lor,
\par 36 ?i a batut pe to?i întâi-nascu?ii din pamântul lor, pârga întregii lor osteneli.
\par 37 ?i i-a scos pe ei cu argint ?i cu aur ?i nu era în semin?iile lor bolnav.
\par 38 Veselitu-s-a Egiptul la ie?irea lor, ca frica de ei îi cuprinsese.
\par 39 Întins-a nor spre acoperirea lor ?i foc ca sa le lumineze noaptea.
\par 40 Cerut-au ?i au venit prepeli?e ?i cu pâine cereasca i-a saturat pe ei.
\par 41 Despicat-a piatra ?i au curs ape ?i au curs râuri în pamânt fara de apa.
\par 42 Ca ?i-a adus aminte de cuvântul cel sfânt al Lui, spus lui Avraam, robul Lui.
\par 43 ?i a scos pe poporul Sau, întru bucurie ?i pe cei ale?i ai Sai, întru veselie.
\par 44 ?i le-a dat lor ?arile neamurilor ?i ostenelile popoarelor au mo?tenit,
\par 45 Ca sa pazeasca drepta?ile Lui ?i legea Lui s-o ?ina.

\chapter{106}

\par 1 Aliluia! Lauda?i pe Domnul ca este bun, ca în veac este mila Lui.
\par 2 Cine va grai puterile Domnului ?i cine va face auzite toate laudele Lui?
\par 3 Ferici?i cei ce pazesc judecata ?i fac dreptate în toata vremea.
\par 4 Adu-?i aminte de noi, Doamne, întru bunavoin?a Ta fa?a de poporul Tau; cerceteaza-ne pe noi cu mântuirea Ta,
\par 5 Ca sa vedem întru bunata?i pe ale?ii Tai, sa ne bucuram de veselia poporului Tau ?i sa ne laudam cu mo?tenirea Ta.
\par 6 Pacatuit-am ca ?i parin?ii no?tri, nelegiuit-am, facut-am strâmbatate.
\par 7 Parin?ii no?tri în Egipt n-au în?eles minunile Tale, nu ?i-au adus aminte de mul?imea milei Tale ?i Te-au amarât când s-au suit la Marea Ro?ie.
\par 8 Dar i-a mântuit pe ei pentru numele Sau, ca sa faca cunoscuta puterea Lui.
\par 9 El a certat Marea Ro?ie ?i a secat-o ?i i-a condus pe ei prin  adâncul marii ca prin pustiu;
\par 10 El i-a scos pe ei din mâna celor ce-i urau ?i i-a izbavit pe ei din mâna vrajma?ului;
\par 11 ?i a acoperit apa pe cei ce-i asupreau pe ei, nici unul din ei n-a ramas.
\par 12 ?i au crezut în cuvintele Lui ?i au cântat lauda Lui;
\par 13 Dar degrab au uitat lucrurile Lui ?i n-au suferit sfatul Lui;
\par 14 Ci au fost cuprin?i de mare pofta, în pustiu, ?i au ispitit pe Dumnezeu, în loc fara de apa.
\par 15 ?i le-a împlinit cererea lor ?i a saturat sufletele lor.
\par 16 ?i au mâniat pe Moise în tabara ?i pe Aaron, sfântul Domnului.
\par 17 S-a deschis pamântul ?i a înghi?it pe Datan ?i a acoperit adunarea lui Abiron.
\par 18 ?i s-a aprins foc în adunarea lor, vapaie a ars pe pacato?i.
\par 19 ?i au facut vi?ei în Horeb ?i s-au închinat idolului.
\par 20 ?i au schimbat slava Lui întru asemanare de vi?el, care manânca iarba.
\par 21 Au uitat pe Dumnezeu, Care i-a izbavit pe ei, Care a facut lucruri mari în Egipt,
\par 22 Lucruri minunate în pamântul lui Ham ?i înfrico?atoare în Marea Ro?ie.
\par 23 Atunci a zis sa-i piarda pe dân?ii, ?i i-ar fi pierdut, daca Moise, alesul Lui, n-ar fi stat înaintea fe?ei Lui ca sa întoarca mânia Lui ?i sa nu-i piarda.
\par 24 Apoi ei au dispre?uit pamântul cel dorit ?i n-au crezut în cuvântul Lui,
\par 25 Ci au cârtit în corturile lor ?i n-au ascultat glasul Domnului.
\par 26 Atunci El a ridicat mâna Sa asupra lor, ca sa-i doboare pe ei în pustiu
\par 27 ?i sa doboare samân?a lor întru neamuri ?i sa-i risipeasca pe ei în toate par?ile.
\par 28 Au jertfit lui Baal-Peor ?i au mâncat jertfele mor?ilor
\par 29 ?i L-au întarâtat pe El cu faptele lor ?i au murit mul?i dintre ei.
\par 30 Dar a stat Finees ?i L-a îmblânzit ?i a încetat bataia
\par 31 ?i i s-a socotit lui întru dreptate, din neam în neam pâna în veac.
\par 32 Apoi L-au mâniat pe El la apa certarii ?i Moise a suferit pentru ei,
\par 33 Ca au amarât duhul lui ?i a grait nesocotit cu buzele lui.
\par 34 N-au nimicit neamurile de care le-a pomenit Domnul;
\par 35 Ci s-au amestecat cu neamurile ?i au deprins lucrurile lor
\par 36 ?i au slujit idolilor lor ?i s-au smintit.
\par 37 ?i-au jertfit pe fiii lor ?i pe fetele lor idolilor,
\par 38 Au varsat sânge nevinovat, sângele fiilor lor ?i al fetelor lor, pe care i-au jertfit idolilor din Canaan ?i s-a spurcat pamântul de sânge.
\par 39 S-au pângarit de lucrurile lor ?i s-au desfrânat cu faptele lor.
\par 40 Atunci S-a aprins de mânie Domnul împotriva poporului Sau ?i a urât mo?tenirea Sa
\par 41 ?i i-a dat pe ei în mâinile neamurilor ?i i-au stapânit pe ei cei ce-i urau pe ei.
\par 42 Vrajma?ii lor i-au asuprit pe ei ?i au fost neferici?i sub mâinile lor.
\par 43 De multe ori Domnul i-a izbavit pe ei, dar ei L-au amarât pe El cu sfatul lor ?i i-a umilit pentru faradelegile lor.
\par 44 Dar Domnul i-a vazut când se necajeau ei, a auzit rugaciunea lor,
\par 45 ?i ?i-a adus aminte de legamântul Lui ?i S-a cait dupa mul?imea milei Sale;
\par 46 ?i le-a dat sa gaseasca mila înaintea celor ce i-au robit pe ei.
\par 47 Izbave?te-ne, Doamne Dumnezeul nostru, ?i ne aduna din neamuri, ca sa laudam numele cel sfânt al Tau ?i sa ne falim cu lauda Ta.
\par 48 Binecuvântat este Domnul Dumnezeul lui Israel, din veac ?i pâna în veac. Tot poporul sa zica: Amin. Amin.

\chapter{107}

\par 1 Lauda?i pe Domnul ca este bun, ca în veac este mila Lui.
\par 2 Sa spuna cei izbavi?i de Domnul, pe care i-a izbavit din mâna vrajma?ului.
\par 3 Din ?ari i-a adunat pe ei, de la rasarit ?i de la apus, de la miazanoapte ?i de la miazazi.
\par 4 Ratacit-au în pustie, în pamânt fara de apa ?i cale spre cetatea de locuit n-au gasit.
\par 5 Erau flamânzi ?i înseta?i; sufletul lor într-în?ii se sfâr?ea;
\par 6 Dar au strigat catre Domnul în necazurile lor ?i din nevoile lor i-a izbavit pe ei
\par 7 ?i i-a pova?uit pe cale dreapta, ca sa mearga spre cetatea de locuit.
\par 8 Laudat sa fie Domnul pentru milele Lui, pentru minunile Lui, pe care le-a facut fiilor oamenilor.
\par 9 Ca a saturat suflet însetat ?i suflet flamând a umplut de bunata?i.
\par 10 ?edeau în întuneric ?i în umbra mor?ii; erau fereca?i de saracie ?i de fier,
\par 11 Pentru ca au amarât cuvintele Domnului ?i sfatul Celui Preaînalt au întarâtat.
\par 12 El a umilit întru osteneli inima lor; slabit-au ?i nu era cine sa le ajute;
\par 13 Dar au strigat catre Domnul în necazurile lor ?i din nevoile lor i-a izbavit pe ei.
\par 14 ?i i-a scos pe ei din întuneric ?i din umbra mor?ii ?i legaturile lor le-a rupt.
\par 15 Laudat sa fie Domnul pentru milele Lui, pentru minunile Lui, pe care le-a facut fiilor oamenilor!
\par 16 Ca a sfarâmat por?i de arama ?i zavoare de fier a frânt
\par 17 ?i i-a ajutat sa iasa din calea faradelegii lor, caci pentru faradelegile lor au fost umili?i.
\par 18 Urât-a sufletul lor orice mâncare ?i s-au apropiat de por?ile mor?ii.
\par 19 Dar au strigat catre Domnul în necazurile lor ?i din nevoile lor i-a izbavit.
\par 20 Trimis-a cuvântul Sau ?i i-a vindecat pe ei ?i i-a izbavit pe ei din stricaciunile lor.
\par 21 Laudat sa fie Domnul pentru milele Lui, pentru minunile Lui, pe care le-a facut fiilor oamenilor!
\par 22 ?i sa-I jertfeasca Lui jertfa de lauda ?i sa vesteasca lucrurile Lui, în bucurie.
\par 23 Cei ce se coboara la mare în corabii, cei ce-?i fac lucrarea lor în ape multe,
\par 24 Aceia au vazut lucrurile Domnului ?i minunile Lui întru adânc.
\par 25 El a zis ?i s-a pornit vânt furtunos ?i s-au înal?at valurile marii.
\par 26 Se urcau pâna la ceruri ?i se coborau pâna în adâncuri, iar sufletul lor întru primejdii încremenea.
\par 27 Se tulburau ?i se clatinau ca un om beat ?i toata priceperea lor a pierit.
\par 28 Dar au strigat catre Domnul în necazurile lor ?i din nevoile lor ia izbavit
\par 29 ?i i-a poruncit furtunii ?i s-a lini?tit ?i au tacut valurile marii.
\par 30 ?i s-au veselit ei, ca s-au lini?tit valurile ?i Domnul i-a pova?uit pe ei la limanul dorit de ei.
\par 31 Laudat sa fie Domnul pentru milele Lui, pentru minunile Lui, pe care le-a facut fiilor oamenilor!
\par 32 Înal?aii-L pe El în adunarea poporului ?i în scaunul batrânilor lauda?i-L pe El,
\par 33 Prefacut-a râurile în pamânt pustiu, izvoarele de apa în pamânt însetat
\par 34 ?i pamântul cel roditor în pamânt sarat, din pricina celor ce locuiesc pe el.
\par 35 Prefacut-a pustiul în iezer de ape, iar pamântul cel fara de apa în izvoare de ape,
\par 36 ?i a a?ezat acolo pe cei flamânzi ?i au zidit cetate de locuit
\par 37 ?i au semanat ?arine ?i au sadit vii ?i au strâns bel?ug de roade
\par 38 ?i i-a binecuvântat pe ei ?i s-au înmul?it foarte ?i vitele lor nu le-a împu?inat.
\par 39 ?i iara?i au fost împu?ina?i ?i chinui?i de apasarea necazurilor ?i a durerii.
\par 40 Aruncat-a dispre? asupra capeteniilor lor ?i i-a ratacit pe ei în loc neumblat ?i fara de cale.
\par 41 Dar pe sarac l-a izbavit de saracie ?i i-a pus pe ei ca pe ni?te oi de mo?tenire.
\par 42 Vedea-vor drep?ii ?i se vor veseli ?i toata faradelegea î?i va astupa gura ei.
\par 43 Cine este în?elept va pazi acestea ?i va pricepe milele Domnului.

\chapter{108}

\par 1 (Un psalm al lui David.) Gata este inima mea, Dumnezeule, gata este inima mea; cânta-voi ?i voi lauda întru inima mea.
\par 2 De?teapta-te slava mea! De?teapta-te psaltire ?i alauta! De?tepta-ma-voi diminea?a.
\par 3 Lauda-Te-voi între popoare, Doamne, cânta-voi ?ie între neamuri,
\par 4 Ca mai mare decât cerurile este mila Ta ?i pâna la nori adevarul Tau.
\par 5 Înal?a-Te peste ceruri, Dumnezeule, ?i peste tot pamântul slava Ta, ca sa se izbaveasca cei placu?i ai Tai.
\par 6 Mântuie?te-ma cu dreapta Ta ?i ma auzi. Dumnezeu a grait în locul cel sfânt al Lui:
\par 7 Înal?a-Ma-voi ?i voi împar?i Sichemul ?i Valea Sucot o voi masura.
\par 8 Al Meu este Galaad ?i al Meu este Manase ?i Efraim sprijinul capului Meu,
\par 9 Iuda, legiuitorul Meu; Moab, vas al spalarii Mele; spre Idumeea voi arunca încal?amintea Mea. Mie cei de alt neam Mi s-au supus".
\par 10 Cine ma va duce la cetatea întarita? Cine ma va pova?ui pâna în Idumeea?
\par 11 Oare, nu Tu, Dumnezeule, Cel ce ne-ai lepadat pe noi? Oare, nu vei ie?i, Dumnezeule, cu o?tirile noastre?
\par 12 Da-ne noua ajutor, ca sa ie?im din necaz, ca de?arta este izbavirea cea de la oameni.
\par 13 Cu Dumnezeu vom birui ?i El va nimici pe vrajma?ii no?tri.

\chapter{109}

\par 1 (Un psalm al lui David; mai-marelui cântare?ilor.) Dumnezeule, lauda mea n-o ?ine sub tacere.
\par 2 Ca gura pacatosului ?i gura vicleanului asupra mea s-au deschis.
\par 3 Grait-au împotriva mea cu limba vicleana ?i cu cuvinte de ura m-au înconjurat ?i s-au luptat cu mine în zadar.
\par 4 În loc sa ma iubeasca, ma cleveteau, iar eu ma rugam.
\par 5 Pus-au împotriva mea rele în loc de bune ?i ura în locul iubirii mele.
\par 6 Pune peste dânsul pe cel pacatos ?i diavolul sa stea de-a dreapta lui.
\par 7 Când se va judeca sa iasa osândit, iar rugaciunea lui sa se prefaca în pacat.
\par 8 Sa fie zilele lui pu?ine ?i dregatoria lui sa o ia altul;
\par 9 Sa ajunga copiii lui orfani ?i femeia lui vaduva;
\par 10 Sa fie stramuta?i copiii lui ?i sa cer?easca; sa fie sco?i din cur?ile caselor lor;
\par 11 Sa smulga camatarul toata averea lui; sa rapeasca strainii ostenelile lui;
\par 12 Sa nu aiba sprijinitor ?i nici orfanii lui miluitor;
\par 13 Sa piara copiii lui ?i într-un neam sa se stinga numele lui;
\par 14 Sa se pomeneasca faradelegea parin?ilor lui înaintea Domnului ?i pacatul maicii lui sa nu se ?tearga;
\par 15 Sa fie înaintea Domnului pururea ?i sa piara de pe pamânt pomenirea lui, pentru ca nu ?i-a adus aminte sa faca mila.
\par 16 ?i a prigonit pe cel sarman, pe cel sarac ?i pe cel smerit cu inima, ca sa-l omoare.
\par 17 ?i a iubit blestemul ?i va veni asupra lui; ?i n-a voit binecuvântarea ?i se va îndeparta de la el.
\par 18 ?i s-a îmbracat cu blestemul ca ?i cu o haina ?i a intrat ca apa înlauntrul lui ?i ca untdelemnul în oasele lui.
\par 19 Sa-i fie lui ca o haina cu care se îmbraca ?i ca un brâu cu care pururea se încinge.
\par 20 Aceasta sa fie rasplata celor ce ma clevetesc pe mine înaintea Domnului ?i graiesc rele împotriva sufletului meu.
\par 21 Dar Tu, Doamne, fa cu mine mila, pentru numele Tau, ca buna este mila Ta.
\par 22 Izbave?te-ma, ca sarac ?i sarman sunt eu ?i inima mea s-a tulburat înlauntrul meu.
\par 23 Ca umbra ce se înclina m-am trecut; ca bataia de aripi a lacustelor tremur.
\par 24 Genunchii mei au slabit de post ?i trupul meu s-a istovit de lipsa untdelemnului
\par 25 ?i eu am ajuns lor ocara. M-au vazut ?i au clatinat cu capetele lor.
\par 26 Ajuta-ma, Doamne Dumnezeul meu, mântuie?te-ma, dupa mila Ta,
\par 27 ?i sa cunoasca ei ca mâna Ta este aceasta ?i Tu, Doamne, ai facut-o pe ea.
\par 28 Ei vor blestema ?i Tu vei binecuvânta. Cei ce se scoala împotriva mea sa se ru?ineze, iar robul Tau sa se veseleasca.
\par 29 Sa se îmbrace cei ce ma clevetesc pe mine cu ocara ?i cu ru?inea lor ca ?i cu un ve?mânt sa se înveleasca.
\par 30 Lauda-voi pe Domnul foarte cu gura mea ?i în mijlocul multora Îl voi preaslavi pe El,
\par 31 Ca a stat de-a dreapta saracului, ca sa izbaveasca sufletul lui de cei ce-l prigonesc.

\chapter{110}

\par 1 (Un psalm al lui David.) Zis-a Domnul Domnului Meu: "?ezi de-a dreapta Mea, pâna ce voi pune pe vrajma?ii Tai a?ternut picioarelor Tale".
\par 2 Toiagul puterii Tale ?i-l va trimite Domnul din Sion, zicând: "Stapâne?te în mijlocul vrajma?ilor Tai.
\par 3 Cu Tine este poporul Tau în ziua puterii Tale, întru stralucirile sfin?ilor Tai. Din pântece mai înainte de luceafar Te-am nascut".
\par 4 Juratu-S-a Domnul ?i nu-I va parea rau: "Tu e?ti preot în veac, dupa rânduiala lui Melchisedec".
\par 5 Domnul este de-a dreapta Ta; sfarâmat-a în ziua mâniei Sale împara?i.
\par 6 Judeca-va între neamuri; va umple totul de ruini; va zdrobi capetele multora pe pamânt.
\par 7 Din pârâu pe cale va bea; pentru aceasta va înal?a capul.

\chapter{111}

\par 1 (Un psalm scris pe vremea lui Neemia.) Lauda-Te-voi, Doamne, cu toata inima mea, în sfatul celor drep?i ?i în adunare.
\par 2 Mari sunt lucrurile Domnului ?i potrivite tuturor voilor Lui.
\par 3 Lauda ?i mare?ie este lucrul Lui ?i dreptatea Lui ramâne în veacul veacului.
\par 4 Pomenire a facut de minunile Sale. Milostiv ?i îndurat este Domnul.
\par 5 Hrana a dat celor ce se tem de Dânsul; aduce?i-va aminte în veac de legamântul Lui.
\par 6 Taria lucrurilor Sale a vestit-o poporului Sau, ca sa le dea lor mo?tenirea neamurilor.
\par 7 Lucrurile mâinilor Lui adevar ?i judecata. Adevarate sunt toate poruncile Lui,
\par 8 Întarite în veacul veacului, facute în adevar ?i dreptate.
\par 9 Izbavire a trimis poporului Sau; poruncit-a în veac legamântul Sau; sfânt ?i înfrico?ator este numele Lui.
\par 10 Începutul în?elepciunii este frica de Domnul; în?elegere buna este tuturor celor ce o fac pe ea. Lauda Lui ramâne în veacul veacului.

\chapter{112}

\par 1 (Un psalm scris pe vremea lui Neemia.) Fericit barbatul care se teme de Domnul; întru poruncile Lui va voi foarte.
\par 2 Puternica va fi pe pamânt semin?ia Lui; neamul drep?ilor se va binecuvânta.
\par 3 Slava ?i boga?ie în casa lui ?i dreptatea lui ramâne în veacul veacului.
\par 4 Rasarit-a în întuneric lumina drep?ilor, Cel milostiv, îndurat ?i drept.
\par 5 Bun este barbatul care se îndura ?i împrumuta; î?i rânduie?te vorbele sale cu judecata, ca în veac nu se va clinti.
\par 6 Întru pomenire ve?nica va fi dreptul; de vorbire de rau nu se va teme.
\par 7 Gata este inima lui a nadajdui în Domnul; întarita este inima lui, nu se va teme, pâna ce va ajunge sa dispre?uiasca pe vrajma?ii sai.
\par 8 Risipit-a, dat-a saracilor; dreptatea lui ramâne în veacul veacului.
\par 9 Puterea lui se va înal?a întru slava.
\par 10 Pacatosul va vedea ?i se va mânia, va scrâ?ni din din?i ?i se va topi. Pofta pacato?ilor va pieri.

\chapter{113}

\par 1 Aliluia! Lauda?i, tineri, pe Domnul, lauda?i numele Domnului.
\par 2 Fie numele Domnului binecuvântat de acum ?i pâna în veac.
\par 3 De la rasaritul soarelui pâna la apus, laudat este numele Domnului.
\par 4 Înalt este peste toate neamurile Domnul, peste ceruri este slava Lui.
\par 5 Cine este ca Domnul Dumnezeul nostru, Cel ce locuie?te întru cele înalte
\par 6 ?i spre cele smerite prive?te, În cer ?i pe pamânt?
\par 7 Cel ce scoate din pulbere pe cel sarac ?i ridica din gunoi pe cel sarman,
\par 8 Ca sa-l a?eze cu cei mari, cu cei mari ai poporului Sau.
\par 9 Cel ce face sa locuiasca cea stearpa în casa, ca o mama ce se bucura de fii.

\chapter{114}

\par 1 La ie?irea lui Israel din Egipt, a casei lui Iacob dintr-un popor barbar,
\par 2 Ajuns-a Iuda sfin?irea Lui, Israel stapânirea Lui.
\par 3 Marea a vazut ?i a fugit, Iordanul s-a întors înapoi.
\par 4 Mun?ii au saltat ca berbecii ?i dealurile ca mieii oilor.
\par 5 Ce-?i este ?ie, mare, ca ai fugit? ?i ?ie Iordane, ca te-ai întors înapoi?
\par 6 Mun?ilor, ca a?i saltat ca berbecii ?i dealurilor, ca mieii oilor?
\par 7 De fa?a Domnului s-a cutremurat pamântul, de fa?a Dumnezeului lui Iacob,
\par 8 Care a prefacut stânca în iezer, iar piatra în izvoare de apa.

\chapter{115}

\par 1 Nu noua, Doamne, nu noua, ci numelui Tau se cuvine slava, pentru mila Ta ?i pentru adevarul Tau,
\par 2 Ca nu cumva sa zica neamurile: "Unde este Dumnezeul lor?"
\par 3 Dar Dumnezeul nostru e în cer; în cer ?i pe pamânt toate câte a voit a facut.
\par 4 Idolii neamurilor sunt argint ?i aur, lucruri de mâini omene?ti:
\par 5 Gura au ?i nu vor grai; ochi au ?i nu vor vedea;
\par 6 Urechi au ?i nu vor auzi; nari au ?i nu vor mirosi;
\par 7 Mâini au ?i nu vor pipai; picioare au ?i nu vor umbla, nu vor glasui cu gâtlejul lor.
\par 8 Asemenea lor sa fie cei ce-i fac pe ei ?i to?i cei ce se încred în ei.
\par 9 Casa lui Israel a nadajduit în Domnul; ajutorul lor ?i aparatorul lor este.
\par 10 Casa lui Aaron a nadajduit în Domnul; ajutorul lor ?i aparatorul lor este.
\par 11 Cei ce se tem de Domnul au nadajduit în Domnul; ajutorul lor ?i aparatorul lor este.
\par 12 Domnul ?i-a adus aminte de noi ?i ne-a binecuvântat pe noi; a binecuvântat casa lui Israel, a binecuvântat casa lui Aaron,
\par 13 A binecuvântat pe cei ce se tem de Domnul, pe cei mici împreuna cu cei mari.
\par 14 Sporeasca-va Domnul pe voi, pe voi ?i pe copiii vo?tri!
\par 15 Binecuvânta?i sa fi?i de Domnul, Cel ce a facut cerul ?i pamântul.
\par 16 Cerul cerului este al Domnului, iar pamântul l-a dat fiilor oamenilor.
\par 17 Nu mor?ii Te vor lauda pe Tine, Doamne, nici to?i cei ce se coboara în iad,
\par 18 Ci noi, cei vii, vom binecuvânta pe Domnul de acum ?i pâna în veac.

\chapter{116}

\par 1 Aliluia! Iubit-am pe Domnul, ca a auzit glasul rugaciunii mele,
\par 2 Ca a plecat urechea Lui spre mine ?i în zilele mele Îl voi chema.
\par 3 Cuprinsu-m-au durerile mor?ii, primejdiile iadului m-au gasit; necaz ?i durere am aflat
\par 4 ?i numele Domnului am chemat: "O, Doamne, izbave?te sufletul meu!"
\par 5 Milostiv este Domnul ?i drept ?i Dumnezeul nostru miluie?te.
\par 6 Cel ce paze?te pe prunci este Domnul; umilit am fost ?i m-am izbavit.
\par 7 Întoarce-te, suflete al meu, la odihna ta, ca Domnul ?i-a facut ?ie bine;
\par 8 Ca a scos sufletul meu din moarte, ochii mei din lacrimi ?i picioarele mele de la cadere.
\par 9 Bine voi placea înaintea Domnului, în pamântul celor vii.
\par 10 Crezut-am, pentru aceea am grait, iar eu m-am smerit foarte.
\par 11 Eu am zis întru uimirea mea: "Tot omul este mincinos!"
\par 12 Ce voi rasplati Domnului pentru toate câte mi-a dat mie?
\par 13 Paharul mântuirii voi lua ?i numele Domnului voi chema.
\par 14 Fagaduin?ele mele le voi plini Domnului, înaintea a tot poporului Sau.
\par 15 Scumpa este înaintea Domnului moartea cuvio?ilor Lui.
\par 16 O, Doamne, eu sunt robul Tau, eu sunt robul Tau ?i fiul roabei Tale; rupt-ai legaturile mele.
\par 17 ?ie-?i voi aduce jertfa de lauda ?i numele Domnului voi chema.
\par 18 Fagaduin?ele mele le voi plini Domnului, înaintea a tot poporului Lui,
\par 19 În cur?ile casei Domnului, în mijlocul tau, Ierusalime.

\chapter{117}

\par 1 Aliluia! Lauda?i pe Domnul toate neamurile; lauda?i-L pe El toate popoarele;
\par 2 Ca s-a întarit mila Lui peste noi ?i adevarul Domnului ramâne în veac.

\chapter{118}

\par 1 Aliluia! Lauda?i pe Domnul ca este bun, ca în veac este mila Lui.
\par 2 Sa zica, dar, casa lui Israel, ca este bun, ca în veac este mila Lui.
\par 3 Sa zica, dar, casa lui Aaron, ca este bun, ca în veac este mila Lui.
\par 4 Sa zica, dar, to?i cei ce se tem de Domnul, ca este bun, ca în veac este mila Lui.
\par 5 În necaz am chemat pe Domnul ?i m-a auzit ?i m-a scos întru desfatare.
\par 6 Domnul este ajutorul meu, nu ma voi teme de ce-mi va face mie omul.
\par 7 Domnul este ajutorul meu ?i eu voi privi cu bucurie pe vrajma?ii mei.
\par 8 Mai bine este a Te încrede în Domnul, decât a Te încrede în om.
\par 9 Mai bine este a nadajdui în Domnul, decât a nadajdui în capetenii.
\par 10 Toate neamurile m-au înconjurat ?i în numele Domnului i-am înfrânt pe ei.
\par 11 Înconjurând m-au înconjurat ?i în numele Domnului i-am înfrânt pe ei.
\par 12 Înconjuratu-m-au ca albinele fagurele, dar s-au stins ca focul de spini ?i în numele Domnului i-am înfrânt pe ei.
\par 13 Împingându-ma m-au împins sa cad, dar Domnul m-a sprijinit.
\par 14 Taria mea ?i lauda mea este Domnul ?i mi-a fost mie spre izbavire.
\par 15 Glas de bucurie ?i de izbavire în corturile drep?ilor: "Dreapta Domnului a facut putere,
\par 16 Dreapta Domnului m-a înal?at, dreapta Domnului a facut putere!"
\par 17 Nu voi muri, ci voi fi viu ?i voi povesti lucrurile Domnului.
\par 18 Certând m-a certat Domnul, dar mor?ii nu m-a dat.
\par 19 Deschide?i-mi mie por?ile drepta?ii, intrând în ele voi lauda pe Domnul.
\par 20 Aceasta este poarta Domnului; drep?ii vor intra prin ea.
\par 21 Te voi lauda, ca m-ai auzit ?i ai fost mie spre izbavire.
\par 22 Piatra pe care n-au bagat-o în seama ziditorii, aceasta s-a facut în capul unghiului.
\par 23 De la Domnul s-a facut aceasta ?i minunata este în ochii no?tri.
\par 24 Aceasta este ziua pe care a facut-o Domnul, sa ne bucuram ?i sa ne veselim întru ea.
\par 25 O, Doamne, mântuie?te! O, Doamne, spore?te!
\par 26 Binecuvântat este cel ce vine întru numele Domnului; binecuvântatu-v-am pe voi, din casa Domnului.
\par 27 Dumnezeu este Domnul ?i S-a aratat noua. Tocmi?i sarbatoare cu ramuri umbroase, pâna la coarnele altarului.
\par 28 Dumnezeul meu e?ti Tu ?i Te voi lauda; Dumnezeul meu e?ti Tu ?i Te voi înal?a. Te voi lauda ca m-ai auzit ?i ai fost mie spre mântuire.
\par 29 Lauda?i pe Domnul ca este bun, ca în veac este mila Lui.

\chapter{119}

\par 1 Aliluia! Ferici?i cei fara prihana în cale, care umbla în legea Domnului.
\par 2 Ferici?i cei ce pazesc poruncile Lui ?i-L cauta cu toata inima lor,
\par 3 Ca n-au umblat în caile Lui cei ce lucreaza faradelegea.
\par 4 Tu ai poruncit ca poruncile Tale sa fie pazite foarte.
\par 5 O, de s-ar îndrepta caile mele, ca sa pazesc poruncile Tale!
\par 6 Atunci nu ma voi ru?ina când voi cauta spre toate poruncile Tale.
\par 7 Lauda-Te-voi întru îndreptarea inimii, ca sa înva? judeca?ile drepta?ii Tale.
\par 8 Îndreptarile Tale voi pazi; nu ma parasi pâna în sfâr?it.
\par 9 Prin ce î?i va îndrepta tânarul calea sa? Prin pazirea cuvintelor Tale.
\par 10 Cu toata inima Te-am cautat pe Tine; sa nu ma lepezi de la poruncile Tale.
\par 11 În inima mea am ascuns cuvintele Tale, ca sa nu gre?esc ?ie.
\par 12 Binecuvântat e?ti, Doamne, înva?a-ma îndreptarile Tale.
\par 13 Cu buzele am rostit toate judeca?ile gurii Tale.
\par 14 În calea marturiilor Tale m-am desfatat ca de toata boga?ia.
\par 15 La poruncile Tale voi cugeta ?i voi cunoa?te caile Tale.
\par 16 La îndreptarile Tale voi cugeta ?i nu voi uita cuvintele Tale.
\par 17 Rasplate?te robului Tau! Voi trai ?i voi pazi poruncile Tale.
\par 18 Deschide ochii mei ?i voi cunoa?te minunile din legea Ta.
\par 19 Strain sunt eu pe pamânt, sa nu ascunzi de la mine poruncile Tale.
\par 20 Aprins e sufletul meu de dorirea judeca?ilor Tale, în toata vremea.
\par 21 Certat-ai pe cai mândri; blestema?i sunt cei ce se abat de la poruncile Tale.
\par 22 Ia de la mine ocara ?i defaimarea, ca marturiile Tale am pazit.
\par 23 Pentru ca au ?ezut capeteniile ?i pe mine ma cleveteau, iar robul Tau cugeta la îndreptarile Tale.
\par 24 Ca marturiile Tale sunt cugetarea mea, iar îndreptarile Tale, sfatul meu.
\par 25 Lipitu-s-a de pamânt sufletul meu; viaza-ma, dupa cuvântul Tau.
\par 26 Vestit-am caile mele ?i m-ai auzit; înva?a-ma îndreptarile Tale.
\par 27 Fa sa în?eleg calea îndreptarilor Tale ?i voi cugeta la minunile Tale.
\par 28 Istovitu-s-a sufletul meu de suparare; întare?te-ma întru cuvintele Tale.
\par 29 Departeaza de la mine calea nedrepta?ii ?i cu legea Ta ma miluie?te.
\par 30 Calea adevarului am ales ?i judeca?ile Tale nu le-am uitat.
\par 31 Lipitu-m-am de marturiile Tale, Doamne, sa nu ma ru?inezi.
\par 32 Pe calea poruncilor Tale am alergat când ai largit inima mea.
\par 33 Lege pune mie, Doamne, calea îndreptarilor Tale ?i o voi pazi pururea.
\par 34 În?elep?e?te-ma ?i voi cauta legea Ta ?i o voi pazi cu toata inima mea.
\par 35 Pova?uie?te-ma pe cararea poruncilor Tale, ca aceasta am voit.
\par 36 Pleaca inima mea la marturiile Tale ?i nu la lacomie.
\par 37 Întoarce ochii mei ca sa nu vada de?ertaciunea; în calea Ta viaza-ma.
\par 38 Împline?te robului Tau cuvântul Tau, care este pentru cei ce se tem de Tine.
\par 39 Îndeparteaza ocara, de care ma tem, caci judeca?ile Tale sunt bune.
\par 40 Iata, am dorit poruncile Tale; întru dreptatea Ta viaza-ma.
\par 41 Sa vina peste mine mila Ta, Doamne, mântuirea Ta, dupa cuvântul Tau,
\par 42 ?i voi raspunde cuvânt celor ce ma ocarasc, ca am nadajduit în cuvintele Tale.
\par 43 Sa nu îndepartezi din gura mea cuvântul adevarului, pâna în sfâr?it, ca întru judeca?ile Tale am nadajduit,
\par 44 ?i voi pazi legea Ta pururea, în veac ?i în veacul veacului.
\par 45 Am umblat întru largime, ca poruncile Tale am cautat.
\par 46 Am vorbit despre marturiile Tale, înaintea împara?ilor, ?i nu m-am ru?inat.
\par 47 Am cugetat la poruncile Tale pe care le-am iubit foarte.
\par 48 Am ridicat mâinile mele la poruncile Tale, pe care le-am iubit ?i am cugetat la îndreptarile Tale.
\par 49 Adu-?i aminte de cuvântul Tau, catre robul Tau, întru care mi-ai dat nadejde.
\par 50 Aceasta m-a mângâiat întru smerenia mea, ca cuvântul Tau m-a viat.
\par 51 Cei mândri m-au batjocorit peste masura, dar de la legea Ta nu m-am abatut.
\par 52 Adusu-mi-am aminte de judeca?ile Tale cele din veac, Doamne, ?i m-am mângâiat.
\par 53 Mâhnire m-a cuprins din pricina pacato?ilor, care parasesc legea Ta.
\par 54 Cântate erau de mine îndreptarile Tale, în locul pribegiei mele.
\par 55 Adusu-mi-am aminte de numele Tau, Doamne, ?i am pazit legea Ta.
\par 56 Aceasta s-a facut mie, ca îndreptarile Tale am cautat.
\par 57 Partea mea e?ti, Doamne, zis-am sa pazesc legea Ta.
\par 58 Rugatu-m-am fe?ei Tale, din toata inima mea, miluie?te-ma dupa cuvântul Tau.
\par 59 Cugetat-am la caile Tale ?i am întors picioarele mele la marturiile Tale.
\par 60 Gata am fost ?i nu m-am tulburat sa pazesc poruncile Tale.
\par 61 Funiile pacato?ilor s-au înfa?urat împrejurul meu, dar legea Ta n-am uitat.
\par 62 La miezul nop?ii m-am sculat ca sa Te laud pe Tine, pentru judeca?ile drepta?ii Tale.
\par 63 Parta? sunt cu to?i cei ce se tem de Tine ?i pazesc poruncile Tale.
\par 64 De mila Ta, Doamne, este plin pamântul; îndreptarile Tale ma înva?a.
\par 65 Bunatate ai facut cu robul Tau, Doamne, dupa cuvântul Tau.
\par 66 Înva?a-ma bunatatea, înva?atura ?i cuno?tin?a, ca în poruncile Tale am crezut.
\par 67 Mai înainte de a fi umilit, am gre?it; pentru aceasta cuvântul Tau am pazit.
\par 68 Bun e?ti Tu, Doamne, ?i întru bunatatea Ta, înva?a-ma îndreptarile Tale.
\par 69 Înmul?itu-s-a asupra mea nedreptatea celor mândri, iar eu cu toata inima mea voi cerceta poruncile Tale.
\par 70 Închegatu-s-a ca grasimea inima lor, iar eu cu legea Ta m-am desfatat.
\par 71 Bine este mie ca m-ai smerit, ca sa înva? îndreptarile Tale.
\par 72 Buna-mi este mie legea gurii Tale, mai mult decât mii de comori de aur ?i argint.
\par 73 Mâinile Tale m-au facut ?i m-au zidit, în?elep?e?te-ma ?i voi înva?a poruncile Tale.
\par 74 Cei ce se tem de Tine ma vor vedea ?i se vor veseli, ca în cuvintele Tale am nadajduit.
\par 75 Cunoscut-am, Doamne, ca drepte sunt judeca?ile Tale ?i întru adevar m-ai smerit.
\par 76 Faca-se dar, mila Ta, ca sa ma mângâie, dupa cuvântul Tau, catre robul Tau.
\par 77 Sa vina peste mine îndurarile Tale ?i voi trai, ca legea Ta cugetarea mea este.
\par 78 Sa se ru?ineze cei mândri, ca pe nedrept m-au nedrepta?it; iar eu voi cugeta la poruncile Tale.
\par 79 Sa se întoarca spre mine cei ce se tem de Tine ?i cei ce cunosc marturiile Tale.
\par 80 Sa fie inima mea fara prihana întru îndreptarile Tale, ca sa nu ma ru?inez.
\par 81 Se tope?te sufletul meu dupa mântuirea Ta; în cuvântul Tau am nadajduit.
\par 82 Sfâr?itu-s-au ochii mei dupa cuvântul Tau, zicând: "Când ma vei mângâia?"
\par 83 Ca m-am facut ca un foale la fum, dar îndreptarile Tale nu le-am uitat.
\par 84 Câte sunt zilele robului Tau? Când vei judeca pe cei ce ma prigonesc?
\par 85 Spusu-mi-au calcatorii de lege de?ertaciuni, dar nu sunt ca legea Ta, Doamne.
\par 86 Toate poruncile Tale sunt adevar; pe nedrept m-au prigonit. Ajuta-ma!
\par 87 Pu?in a fost de nu m-am sfâr?it pe pamânt, dar eu n-am parasit poruncile Tale.
\par 88 Dupa mila Ta viaza-ma ?i voi pazi marturiile gurii mele.
\par 89 În veac, Doamne, cuvântul Tau ramâne în cer;
\par 90 În neam ?i în neam adevarul Tau. Întemeiat-ai pamântul ?i ramâne.
\par 91 Dupa rânduiala Ta ramâne ziua, ca toate sunt slujitoare ?ie.
\par 92 De n-ar fi fost legea Ta gândirea mea, atunci a? fi pierit întru necazul meu.
\par 93 În veac nu voi uita îndreptarile Tale, ca într-însele m-ai viat, Doamne.
\par 94 Al Tau sunt eu, mântuie?te-ma, ca îndreptarile Tale am cautat.
\par 95 Pe mine m-au a?teptat pacato?ii ca sa ma piarda. Marturiile Tale am priceput.
\par 96 La tot lucrul desavâr?it am vazut sfâr?it, dar porunca Ta este fara de sfâr?it.
\par 97 Ca am iubit legea Ta, Doamne, ea toata ziua cugetarea mea este.
\par 98 Mai mult decât pe vrajma?ii mei mai în?elep?it cu porunca Ta, ca în veac a mea este.
\par 99 Mai mult decât înva?atorii mei am priceput, ca la marturiile Tale gândirea mea este.
\par 100 Mai mult decât batrânii am în?eles, ca poruncile Tale am cautat.
\par 101 De la toata calea cea rea mi-am oprit picioarele mele, ca sa pazesc cuvintele Tale.
\par 102 De la judeca?ile Tale nu m-am abatut, ca Tu ai pus mie lege.
\par 103 Cât sunt de dulci limbii mele, cuvintele Tale, mai mult decât mierea, în gura mea!
\par 104 Din poruncile Tale m-am facut priceput; pentru aceasta am urât toata calea nedrepta?ii.
\par 105 Faclie picioarelor mele este legea Ta ?i lumina cararilor mele.
\par 106 Juratu-m-am ?i m-am hotarât sa pazesc judeca?ile drepta?ii Tale.
\par 107 Umilit am fost pâna în sfâr?it: Doamne, viaza-ma, dupa cuvântul Tau.
\par 108 Cele de bunavoie ale gurii mele binevoie?te-le Doamne, ?i judeca?ile Tale ma înva?a.
\par 109 Sufletul meu în mâinile Tale este pururea ?i legea Ta n-am uitat.
\par 110 Pusu-mi-au pacato?ii cursa mie, dar de la poruncile Tale n-am ratacit.
\par 111 Mo?tenit-am marturiile Tale în veac, ca bucurie inimii mele sunt ele.
\par 112 Plecat-am inima mea ca sa fac îndreptarile Tale în veac spre rasplatire.
\par 113 Pe calcatorii de lege am urât ?i legea Ta am iubit.
\par 114 Ajutorul meu ?i sprijinitorul meu e?ti Tu, în cuvântul Tau am nadajduit.
\par 115 Departa?i-va de la mine cei ce vicleni?i ?i voi cerceta poruncile Dumnezeului meu.
\par 116 Apara-ma, dupa cuvântul Tau, ?i ma viaza ?i sa nu-mi dai de ru?ine a?teptarea mea.
\par 117 Ajuta-ma ?i ma voi mântui ?i voi cugeta la îndreptarile Tale, pururea.
\par 118 Defaimat-ai pe to?i cei ce se departeaza de la îndreptarile Tale, pentru ca nedrept este gândul lor.
\par 119 Socotit-am calcatori de lege pe to?i pacato?ii pamântului; pentru aceasta am iubit marturiile Tale, pururea.
\par 120 Strapunge cu frica Ta trupul meu, ca de judeca?ile Tale m-am temut.
\par 121 Facut-am judecata ?i dreptate; nu ma da pe mâna celor ce-mi fac strâmbatate.
\par 122 Prime?te pe robul Tau în bunatate, ca sa nu ma cleveteasca cei mândri.
\par 123 Sfâr?itu-sau ochii mei dupa mântuirea Ta ?i dupa cuvântul drepta?ii Tale.
\par 124 Fa cu robul Tau, dupa mila Ta, ?i îndreptarile Tale ma înva?a.
\par 125 Robul Tau sunt eu; în?elep?e?te-ma ?i voi cunoa?te marturiile Tale.
\par 126 Vremea este sa lucreze Domnul, ca oamenii au stricat legea Ta.
\par 127 Pentru aceasta am iubit poruncile Tale, mai mult decât aurul ?i topazul.
\par 128 Pentru aceasta spre toate poruncile Tale m-am îndreptat, toata calea nedreapta am urât.
\par 129 Minunate sunt marturiile Tale, pentru aceasta le-a cercetat pe ele sufletul meu.
\par 130 Aratarea cuvintelor Tale lumineaza ?i în?elep?e?te pe prunci.
\par 131 Gura mea am deschis ?i am aflat, ca de poruncile Tale am dorit.
\par 132 Cauta spre mine ?i ma miluie?te, dupa judecata Ta, fala de cei ce iubesc numele Tau.
\par 133 Pa?ii mei îndrepteaza-i dupa cuvântul Tau, ?i sa nu ma stapâneasca nici o faradelege.
\par 134 Izbave?te-ma de clevetirea oamenilor ?i voi pazi poruncile Tale.
\par 135 Fa?a Ta arat-o robului Tau ?i ma înva?a poruncile Tale.
\par 136 Izvoare de apa s-au coborât din ochii mei, pentru ca n-am pazit legea Ta.
\par 137 Drept e?ti, Doamne, ?i drepte sunt judeca?ile Tale.
\par 138 Poruncit-ai cu dreptate marturiile Tale ?i cu tot adevarul.
\par 139 Topitu-m-a râvna casei Tale, ca au uitat cuvintele Tale vrajma?ii mei.
\par 140 Lamurit cu foc este cuvântul Tau foarte ?i robul Tau l-a iubit pe el.
\par 141 Tânar sunt eu ?i defaimat, dar îndreptarile Tale nu le-am uitat.
\par 142 Dreptatea Ta este dreptate în veac ?i legea Ta adevarul.
\par 143 Necazuri ?i nevoi au dat peste mine, dar poruncile Tale sunt gândirea mea.
\par 144 Drepte sunt marturiile Tale, în veac; în?elep?e?te-ma ?i voi fi viu.
\par 145 Strigat-am cu toata inima mea: Auzi-ma, Doamne! Îndreptarile Tale voi cauta.
\par 146 Strigat-am catre Tine, mântuie?te-ma, ?i voi pazi marturiile Tale.
\par 147 Din zori m-am sculat ?i am strigat; întru cuvintele Tale am nadajduit.
\par 148 Deschis-am ochii mei dis-de-diminea?a, ca sa cuget la cuvintele Tale.
\par 149 Glasul meu auzi-l, Doamne, dupa mila Ta; dupa judecata Ta ma viaza.
\par 150 Apropiatu-s-au cei ce ma prigonesc cu faradelege, dar de la legea Ta s-au îndepartat.
\par 151 Aproape e?ti Tu, Doamne, ?i toate poruncile Tale sunt adevarul.
\par 152 Din început am cunoscut, din marturiile Tale, ca în veac le-ai întemeiat pe ele.
\par 153 Vezi smerenia mea ?i ma scoate, ca legea Ta n-am uitat.
\par 154 Judeca pricina mea ?i ma izbave?te; dupa cuvântul Tau, fa-ma viu.
\par 155 Departe de pacato?i este mântuirea, ca îndreptarile Tale n-au cautat.
\par 156 Îndurarile Tale multe sunt Doamne; dupa judecata Ta ma viaza.
\par 157 Mul?i sunt cei ce ma prigonesc ?i ma necajesc, dar de la marturiile Tale nu m-am abatut.
\par 158 Vazut-am pe cei nepricepu?i ?i ma sfâr?eam, ca n-au pazit cuvintele Tale.
\par 159 Vezi ca poruncile Tale am iubit, Doamne; întru mila Ta ma viaza.
\par 160 Începutul cuvintelor Tale este adevarul ?i ve?nice toate judeca?ile drepta?ii Tale.
\par 161 Capeteniile m-au prigonit în zadar; iar de cuvintele Tale s-a înfrico?at inima mea.
\par 162 Bucura-ma-voi de cuvintele Tale, ca cel ce a aflat comoara mare.
\par 163 Nedreptatea am urât ?i am dispre?uit, iar legea Ta am iubit.
\par 164 De ?apte ori pe zi Te-am laudat pentru judeca?ile drepta?ii Tale.
\par 165 Pace multa au cei ce iubesc legea Ta ?i nu se smintesc.
\par 166 A?teptat-am mântuirea Ta, Doamne, ?i poruncile Tale am iubit.
\par 167 Pazit-a sufletul meu marturiile Tale ?i le-a iubit foarte.
\par 168 Pazit-am poruncile Tale ?i marturiile Tale, ca toate caile mele înaintea Ta sunt, Doamne.
\par 169 Sa se apropie rugaciunea mea înaintea Ta, Doamne; dupa cuvântul Tau ma în?elep?e?te.
\par 170 Sa ajunga cererea mea înaintea Ta, Doamne; dupa cuvântul Tau ma izbave?te.
\par 171 Sa raspândeasca buzele mele lauda, ca m-ai înva?at îndreptarile Tale.
\par 172 Rosti-va limba mea cuvintele Tale, ca toate poruncile Tale sunt drepte.
\par 173 Mina Ta sa ma izbaveasca, ca poruncile Tale am ales.
\par 174 Dorit-am mântuirea Ta, Doamne, ?i legea Ta cugetarea mea este.
\par 175 Viu va fi sufletul meu ?i Te va lauda ?i judeca?ile Tale îmi vor ajuta mie.
\par 176 Ratacit-am ca o oaie pierduta; cauta pe robul Tau, ca poruncile Tale nu le-am uitat.

\chapter{120}

\par 1 (O cântare a treptelor.) Catre Domnul am strigat când m-am necajit ?i m-a auzit.
\par 2 Doamne, izbave?te sufletul meu de buzele nedrepte ?i de limba vicleana.
\par 3 Ce se va da ?ie ?i ce vei câ?tiga de la limba vicleana?
\par 4 Sage?i ascu?ite cu carbuni aprin?i trase de Cel puternic.
\par 5 Vai mie, ca pribegia mea s-a prelungit, ca locuiesc în corturile lui Chedar!
\par 6 Mult a pribegit sufletul meu.
\par 7 Cu cei ce urau pacea, facator de pace eram; când graiam lor, se luptau cu mine în zadar.

\chapter{121}

\par 1 (O cântare a treptelor.) Ridicat-am ochii mei la mun?i, de unde va veni ajutorul meu.
\par 2 Ajutorul meu de la Domnul, Cel ce a facut cerul ?i pamântul.
\par 3 Nu va lasa sa se clatine piciorul tau, nici nu va dormita Cel ce paze?te.
\par 4 Iata, nu va dormita, nici nu va adormi Cel ce paze?te pe Israel.
\par 5 Domnul te va pazi pe tine, Domnul este acoperamântul tau, de-a dreapta ta.
\par 6 Ziua soarele nu te va arde, nici luna noaptea.
\par 7 Domnul te va pazi pe tine de tot raul; pazi-va sufletul tau.
\par 8 Domnul va pazi intrarea ta ie?irea ta de acum ?i pâna în veac.

\chapter{122}

\par 1 (O cântare a treptelor.) Veselitu-m-am de cei ce mi-au zis mie: "În casa Domnului vom merge!"
\par 2 Stateau picioarele noastre în cur?ile tale, Ierusalime!
\par 3 Ierusalimul, cel ce este zidit ca o cetate, ale carei por?i sunt strâns-unite.
\par 4 Ca acolo s-au suit semin?iile, semin?iile Domnului, dupa legea lui Israel, ca sa laude numele Domnului.
\par 5 Ca acolo s-au a?ezat scaunele la judecata, scaunele pentru casa lui David.
\par 6 Ruga?i-va pentru pacea Ierusalimului ?i pentru îndestularea celor ce te iubesc pe tine.
\par 7 Sa fie pace în întariturile tale ?i îndestulare în turnurile tale.
\par 8 Pentru fra?ii mei ?i pentru vecinii mei graiam despre tine pace.
\par 9 Pentru casa Domnului Dumnezeului nostru am dorit cele bune ?ie.

\chapter{123}

\par 1 (O cântare a treptelor.) Catre Tine, Cel ce locuie?ti în cer, am ridicat ochii mei.
\par 2 Iata, precum sunt ochii robilor la mâinile stapânilor lor, precum sunt ochii slujnicei la mâinile stapânei sale, a?a sunt ochii no?tri catre Domnul Dumnezeul nostru, pâna ce Se va milostivi spre noi.
\par 3 Miluie?te-ne pe noi, Doamne, miluie?te-ne pe noi, ca mult ne-am saturat de defaimare,
\par 4 Ca prea mult s-a saturat sufletul nostru de ocara celor îndestula?i ?i de defaimarea celor mândri.

\chapter{124}

\par 1 (Un psalm al lui David. O cântare a treptelor.) De n-ar fi fost Domnul cu noi, sa spuna Israel!
\par 2 De n-ar fi fost Domnul cu noi, când s-au ridicat oamenii împotriva noastra,
\par 3 De vii ne-ar fi înghi?it pe noi, când s-a aprins mânia lor împotriva noastra.
\par 4 Apa ne-ar fi înecat pe noi, ?uvoi ar fi trecut peste sufletul nostru.
\par 5 Atunci ar fi trecut peste sufletul nostru valuri înspaimântatoare.
\par 6 Binecuvântat este Domnul, Care nu ne-a dat pe noi spre vânare din?ilor lor.
\par 7 Sufletul nostru a scapat ca o pasare din cursa vânatorilor; cursa s-a sfarâmat ?i noi ne-am izbavit.
\par 8 Ajutorul nostru este în numele Domnului, Cel ce a facut cerul ?i pamântul.

\chapter{125}

\par 1 (O cântare a treptelor.) Cei ce se încred în Domnul sunt ca muntele Sionului; nu se va clatina în veac cel ce locuie?te în Ierusalim.
\par 2 Mun?i sunt împrejurul lui ?i Domnul împrejurul poporului Sau, de acum ?i pâna în veac.
\par 3 Ca nu va lasa Domnul toiagul pacato?ilor peste soarta drep?ilor, ca sa nu-?i întinda drep?ii întru faradelegi mâinile lor.
\par 4 Fa bine, Doamne, celor buni ?i celor drep?i cu inima;
\par 5 Iar pe cei ce se abat pe cai nedrepte, Domnul îi va duce cu cei ce lucreaza faradelegea. Pace peste Israel!

\chapter{126}

\par 1 (O cântare a treptelor.) Când a întors Domnul robia Sionului ne-am umplut de mângâiere.
\par 2 Atunci s-a umplut de bucurie gura noastra ?i limba noastra de veselie; atunci se zicea între neamuri: "Mari lucruri a facut Domnul cu ei!"
\par 3 Mari lucruri a facut Domnul cu noi: ne-a umplut de bucurie.
\par 4 Întoarce, Doamne, robia noastra, cum întorci pâraiele spre miazazi.
\par 5 Cei ce seamana cu lacrimi, cu bucurie vor secera.
\par 6 Mergând mergeau ?i plângeau, aruncând semin?ele lor, dar venind vor veni cu bucurie, ridicând snopii lor.

\chapter{127}

\par 1 (Un psalm al lui Solomon. O cântare a treptelor.) De n-ar zidi Domnul casa, în zadar s-ar osteni cei ce o zidesc; de n-ar pazi Domnul cetatea, în zadar ar priveghea cel ce o paze?te.
\par 2 în zadar va scula?i dis-de-diminea?a, în zadar va culca?i târziu, voi care mânca?i pâinea durerii, daca nu v-ar da Domnul somn, iubi?i ai Sai.
\par 3 Iata, fiii sunt mo?tenirea Domnului, rasplata rodului pântecelui.
\par 4 Precum sunt sage?ile în mâna celui viteaz, a?a sunt copiii parin?ilor tineri.
\par 5 Fericit este omul care-?i va umple casa de copii; nu se va ru?ina când va grai cu vrajma?ii sai în poarta.

\chapter{128}

\par 1 (O cântare a treptelor.) Ferici?i to?i cei ce se tem de Domnul, care umbla în caile Lui.
\par 2 Rodul muncii mâinilor tale vei mânca. Fericit e?ti; bine-?i va fi!
\par 3 Femeia ta ca o vie roditoare, în laturile casei tale; fiii tai ca ni?te vlastare tinere de maslin, împrejurul mesei tale.
\par 4 Iata, a?a se va binecuvânta omul, cel ce se teme de Domnul.
\par 5 "Te va binecuvânta Domnul din Sion ?i vei vedea bunata?ile Ierusalimului în toate zilele vie?ii tale.
\par 6 ?i vei vedea pe fiii fiilor tai. Pace peste Israel!

\chapter{129}

\par 1 (O cântare a treptelor.) De multe ori s-au luptat cu mine din tinere?ile mele, sa spuna Israel!
\par 2 De multe ori s-au luptat cu mine, din tinere?ile mele ?i nu m-au biruit.
\par 3 Spatele mi-au lovit pacato?ii, întins-au nelegiuirea lor;
\par 4 Dar Domnul Cel drept a taiat grumajii pacato?ilor.
\par 5 Sa se ru?ineze ?i sa se întoarca înapoi to?i cei ce urasc Sionul.
\par 6 Faca-se ca iarba de pe acoperi?uri, care, mai înainte de a fi smulsa, s-a uscat,
\par 7 Din care nu ?i-a umplut mâna lui seceratorul ?i sânul sau, cel ce aduna snopii,
\par 8 ?i trecatorii nu vor zice: "Binecuvântarea Domnului fie peste voi!" sau: "Va binecuvântam în numele Domnului!"

\chapter{130}

\par 1 (O cântare a treptelor.) Dintru adâncuri am strigat catre Tine; Doamne! Doamne, auzi glasul meu!
\par 2 Fie urechile Tale cu luare-aminte la glasul rugaciunii mele.
\par 3 De Te vei uita la faradelegi, Doamne, Doamne, cine va suferi?
\par 4 Ca la Tine este milostivirea.
\par 5 Pentru numele Tau, Te-am a?teptat, Doamne; a?teptat-a sufletul meu spre cuvântul Tau,
\par 6 Nadajduit-a sufletul meu în Domnul, din straja dimine?ii pâna în noapte. Din straja dimine?ii sa nadajduiasca Israel spre Domnul.
\par 7 Ca la Domnul este mila ?i multa mântuire la El
\par 8 ?i El va izbavi pe Israel din toate faradelegile lui.

\chapter{131}

\par 1 (Un psalm al lui David. O cântare a treptelor.) Doamne, nu s-a mândrit inima mea, nici nu s-au înal?at ochii mei, nici n-am umblat dupa lucruri mari, nici dupa lucruri mai presus de mine,
\par 2 Dimpotriva, mi-am smerit ?i mi-am domolit sufletul meu, ca un prunc în?arcat de mama lui, ca rasplata a sufletului meu.
\par 3 Sa nadajduiasca Israel în Domnul, de acum ?i pâna în veac!

\chapter{132}

\par 1 (O cântare a treptelor.) Adu-?i aminte, Doamne, de David ?i de toate blânde?ile lui.
\par 2 Cum s-a jurat Domnului ?i a fagaduit Dumnezeului lui Iacob:
\par 3 Nu voi intra în loca?ul casei mele, nu ma voi sui pe patul meu de odihna,
\par 4 Nu voi da somn ochilor mei ?i genelor mele dormitare ?i odihna tâmplelor mele,
\par 5 Pâna ce nu voi afla loc Domnului, loca? Dumnezeului lui Iacob.
\par 6 Iata am auzit de chivotul legii; în Efrata l-am gasit, în ?arina lui Iaar.
\par 7 Intra-vom în loca?urile Lui, închina-ne-vom la locul unde au stat picioarele Lui.
\par 8 Scoala-Te, Doamne, întru odihna Ta, Tu ?i chivotul sfin?irii Tale.
\par 9 Preo?ii Tai se vor îmbraca cu dreptate ?i cuvio?ii Tai se vor bucura.
\par 10 Din pricina lui David, robul Tau, sa nu întorci fa?a unsului Tau.
\par 11 Juratu-S-a Domnul lui David adevarul ?i nu-l va lepada: "Din rodul pântecelui tau voi pune pe scaunul tau,
\par 12 De vor pazi fiii tai legamântul Meu ?i marturiile acestea ale Mele, în care îi voi înva?a pe ei ?i fiii lor vor ?edea pâna în veac pe scaunul tau".
\par 13 Ca a ales Domnul Sionul ?i l-a ales ca locuin?a Lui.
\par 14 "Aceasta este odihna Mea în veacul veacului. Aici voi locui, ca l-am ales pe el.
\par 15 Roadele lui le voi binecuvânta foarte; pe saracii lui îi voi satura cu pâine.
\par 16 Pe preo?ii lui îi voi îmbraca cu izbavire ?i cuvio?ii lui cu bucurie se vor bucura.
\par 17 Acolo voi face sa rasara puterea lui David, gatit-am faclie unsului Meu.
\par 18 Pe vrajma?ii lui îi voi îmbraca cu ru?ine, iar pe dânsul va înflori sfin?enia Mea".

\chapter{133}

\par 1 (Un psalm al lui David. O cântare a treptelor.) Iata acum ce este bun ?i ce este frumos, decât numai a locui fra?ii împreuna!
\par 2 Aceasta este ca mirul pe cap, care se coboara pe barba, pe barba lui Aaron, care se coboara pe marginea ve?mintelor lui.
\par 3 Aceasta este ca roua Ermonului, ce se coboara pe mun?ii Sionului, ca unde este unire acolo a poruncit Domnul binecuvântarea ?i via?a pâna în veac.

\chapter{134}

\par 1 (O cântare a treptelor.) Iata acum binecuvânta?i pe Domnul toate slugile Domnului, care sta?i în casa Domnului, în cur?ile casei Dumnezeului nostru.
\par 2 Noaptea ridica?i mâinile voastre spre cele sfinte ?i binecuvânta?i pe Domnul.
\par 3 Te va binecuvânta Domnul din Sion, Cel ce a facut cerul ?i pamântul.

\chapter{135}

\par 1 Aliluia! Lauda?i numele Domnului, lauda?i slugi pe Domnul,
\par 2 Cei ce sta?i în casa Domnului, în cur?ile Dumnezeului nostru.
\par 3 Lauda?i pe Domnul, ca este bun Domnul; cânta?i numele Lui, ca este bun.
\par 4 Ca pe Iacob ?i l-a ales Domnul, pe Israel spre mo?tenire Lui.
\par 5 Ca eu am cunoscut ca este mare Domnul ?i Domnul nostru peste to?i dumnezeii.
\par 6 Toate câte a vrut Domnul a facut în cer ?i pe pamânt, în mari ?i în toate adâncurile.
\par 7 A ridicat nori de la marginea pamântului; fulgerele spre ploaie le-a facut; El scoate vânturile din vistieriile Sale.
\par 8 El a batut pe cei întâi-nascu?i ai Egiptului, de la om pâna la dobitoc.
\par 9 Trimis-a semne ?i minuni în mijlocul tau, Egipte, lui Faraon ?i tuturor robilor lui.
\par 10 El a batut neamuri multe ?i a ucis împara?i puternici:
\par 11 Pe Sihon împaratul Amoreilor ?i pe Og împaratul Vasanului ?i toate stapânirile Canaanului.
\par 12 ?i a dat pamântul lor mo?tenire, mo?tenire lui Israel, poporului Sau.
\par 13 Doamne, numele Tau este în veac ?i, pomenirea Ta în neam ?i în neam.
\par 14 Ca va judeca Domnul pe poporul Sau ?i de slugile Sale se va milostivi.
\par 15 Idolii neamurilor sunt argint ?i aur, lucruri facute de mâini omene?ti.
\par 16 Gura au ?i nu vor grai, ochi au ?i nu vor vedea.
\par 17 Urechi au ?i nu vor auzi, ca nu este duh în gura lor.
\par 18 Asemenea lor sa fie to?i cei care îi fac pe ei ?i to?i cei ce se încred în ei.
\par 19 Casa lui Israel, binecuvânta?i pe Domnul; casa lui Aaron, binecuvânta?i pe Domnul;
\par 20 Casa lui Levi, binecuvânta?i pe Domnul; cei ce va teme?i de Domnul, binecuvânta?i pe Domnul.
\par 21 Binecuvântat este Domnul din Sion, Cel ce locuie?te în Ierusalim.

\chapter{136}

\par 1 Aliluia! Lauda?i pe Domnul ca este bun, ca în veac este mila Lui.
\par 2 Lauda?i pe Dumnezeul dumnezeilor, ca în veac este mila Lui.
\par 3 Lauda?i pe Domnul domnilor, ca în veac este mila Lui.
\par 4 Singurul Care face minuni mari, ca în veac este mila Lui.
\par 5 Cel ce a facut cerul cu pricepere, ca în veac este mila Lui.
\par 6 Cel ce a întarit pamântul pe ape, ca în veac este mila Lui.
\par 7 Cel ce a facut luminatorii cei mari, ca în veac este mila Lui.
\par 8 Soarele, spre stapânirea zilei, ca în veac este mila Lui.
\par 9 Luna ?i stelele spre stapânirea nop?ii, ca în veac este mila Lui.
\par 10 Cel ce a batut Egiptul cu cei întâi-nascu?i ai lor, ca în veac este mila Lui.
\par 11 ?i a scos pe Israel din mijlocul lor, ca în veac este mila Lui.
\par 12 Cu mâna tare ?i cu bra? înalt, ca în veac este mila Lui.
\par 13 Cel ce a despar?it Marea Ro?ie în doua; ca în veac este mila Lui.
\par 14 ?i a trecut pe Israel prin mijlocul ei, ca în veac este mila Lui.
\par 15 ?i a rasturnat pe Faraon ?i o?tirea lui în Marea Ro?ie, ca în veac este mila Lui.
\par 16 Cel ce a trecut pe poporul Lui prin pustiu, ca în veac este mila Lui.
\par 17 Cel ce a batut împara?i mari, ca în veac este mila Lui.
\par 18 Cel ce a omorât împara?i tari, ca în veac este mila Lui.
\par 19 Pe Sihon împaratul Amoreilor, ca în veac este mila Lui.
\par 20 ?i pe Og împaratul Vasanului, ca în veac este mila Lui.
\par 21 ?i le-a dat pamântul lor mo?tenire, ca în veac este mila Lui.
\par 22 Mo?tenire lui Israel, robul Lui, ca în veac este mila Lui.
\par 23 Ca în smerenia noastra ?i-a adus aminte de noi Domnul, ca în veac este mila Lui.
\par 24 ?i ne-a izbavit pe noi de vrajma?ii no?tri, ca în veac este mila Lui.
\par 25 Cel ce da hrana la tot trupul, ca în veac este mila Lui.
\par 26 Lauda?i pe Dumnezeul cerului, ca în veac este mila Lui.

\chapter{137}

\par 1 (Un psalm al lui David.) La râul Babilonului, acolo am ?ezut ?i am plâns, când ne-am adus aminte de Sion.
\par 2 În salcii, în mijlocul lor, am atârnat harpele noastre.
\par 3 Ca acolo cei ce ne-au robit pe noi ne-au cerut noua cântare, zicând: "Cânta?i-ne noua din cântarile Sionului!"
\par 4 Cum sa cântam cântarea Domnului în pamânt strain?
\par 5 De te voi uita, Ierusalime, uitata sa fie dreapta mea!
\par 6 Sa se lipeasca limba mea de grumazul meu, de nu-mi voi aduce aminte de tine, de nu voi pune înainte Ierusalimul, ca început al bucuriei mele.
\par 7 Adu-?i aminte, Doamne, de fiii lui Edom, în ziua darâmarii Ierusalimului, când ziceau: "Strica?i-l, strica?i-l pâna la temeliile lui!"
\par 8 Fiica Babilonului, ticaloasa! Fericit este cel ce-?i va rasplati ?ie fapta ta pe care ai facut-o noua.
\par 9 Fericit este cel ce va apuca ?i va lovi pruncii tai de piatra.

\chapter{138}

\par 1 (Un psalm al lui David.) Lauda-Te-voi, Doamne, cu toata inima mea, ca ai auzit cuvintele gurii mele ?i înaintea îngerilor Î?i voi cânta.
\par 2 Închina-ma-voi în loca?ul Tau cel sfânt ?i voi lauda numele Tau, pentru mila Ta ?i adevarul Tau; ca ai marit peste tot numele cel sfânt al Tau.
\par 3 În orice zi Te voi chema, degraba ma auzi! Spore?te în sufletul meu puterea.
\par 4 Sa Te laude pe Tine, Doamne, to?i împara?ii pamântului, când vor auzi toate graiurile gurii Tale,
\par 5 ?i sa cânte în caile Domnului, ca mare este slava Domnului.
\par 6 Ca înalt este Domnul ?i spre cele smerite prive?te ?i pe cele înalte de departe le cunoa?te.
\par 7 De ajung la necaz, viaza-ma! Împotriva vrajma?ilor mei ai întins mâna Ta ?i m-a izbavit dreapta Ta.
\par 8 Domnul le va plati lor pentru mine. Doamne, mila Ta este în veac; lucrurile mâinilor Tale nu le trece cu vederea.

\chapter{139}

\par 1 (Un psalm al lui David; mai-marelui cântare?ilor.) Doamne, cercetatu-m-ai ?i m-ai cunoscut.
\par 2 Tu ai cunoscut ?ederea mea ?i scularea mea; Tu ai priceput gândurile mele de departe.
\par 3 Cararea mea ?i firul vie?ii mele Tu le-ai cercetat ?i toate caile mele mai dinainte le-ai vazut.
\par 4 Ca înca nu este cuvânt pe limba mea
\par 5 ?i iata, Doamne, Tu le-ai cunoscut pe toate ?i pe cele din urma ?i pe cele de demult; Tu m-ai zidit ?i ai pus peste mine mâna Ta.
\par 6 Minunata este ?tiin?a Ta, mai presus de mine; este înalta ?i n-o pot ajunge.
\par 7 Unde ma voi duce de la Duhul Tau ?i de la fa?a Ta unde voi fugi?
\par 8 De ma voi sui în cer, Tu acolo e?ti. De ma voi coborî în iad, de fa?a e?ti.
\par 9 De voi lua aripile mele de diminea?a ?i de ma voi a?eza la marginile marii
\par 10 ?i acolo mâna Ta ma va pova?ui ?i ma va ?ine dreapta Ta.
\par 11 ?i am zis: "Poate întunericul ma va acoperi ?i se va face noapte lumina dimprejurul meu".
\par 12 Dar întunericul nu este întuneric la Tine ?i noaptea ca ziua va lumina. Cum este întunericul ei, a?a este ?i lumina ei.
\par 13 Ca Tu ai zidit rarunchii mei, Doamne, Tu m-ai alcatuit în pântecele maicii mele.
\par 14 Te voi lauda, ca sunt o faptura a?a de minunata. Minunate sunt lucrurile Tale ?i sufletul meu le cunoa?te foarte.
\par 15 Nu sunt ascunse de Tine oasele mele, pe care le-ai facut întru ascuns, nici fiin?a mea pe care ai urzit-o ca în cele mai de jos ale pamântului.
\par 16 Cele nelucrate ale mele le-au cunoscut ochii Tai ?i în cartea Ta toate se vor scrie; zi de zi se vor savâr?i ?i nici una din ele nu va fi nescrisa.
\par 17 Iar eu am cinstit foarte pe prietenii Tai, Dumnezeule, ?i foarte s-a întarit stapânirea lor.
\par 18 ?i-i voi numara pe ei, ?i mai mult decât nisipul se vor înmul?i. M-am sculat ?i înca sunt cu Tine.
\par 19 O, de ai ucide pe pacato?i, Dumnezeule! Barba?i varsatori de sânge, departa?i-va de la mine!
\par 20 Ace?tia Te graiesc de rau, Doamne, ?i vrajma?ii Î?i hulesc numele.
\par 21 Oare, nu pe cei ce Te urasc pe Tine, Doamne, am urât ?i asupra vrajma?ilor Tai m-am mâhnit?
\par 22 Cu ura desavâr?ita i-am urât pe ei ?i mi s-au facut du?mani.
\par 23 Cerceteaza-ma, Doamne, ?i cunoa?te inima mea; încearca-ma ?i cunoa?te cararile mele
\par 24 ?i vezi de este calea faradelegii în mine ?i ma îndrepteaza pe calea cea ve?nica.

\chapter{140}

\par 1 (Un psalm al lui David; mai-marelui cântare?ilor.) Scoate-ma, Doamne, de la omul viclean ?i de omul nedrept ma izbave?te,
\par 2 Care gândeau nedreptate în inima, toata ziua îmi duceau razboi.
\par 3 Ascu?it-au limba lor ca a ?arpelui; venin de aspida sub buzele lor.
\par 4 Paze?te-ma, Doamne, de mâna pacatosului; scoate-ma de la oamenii nedrep?i, care au gândit sa împiedice pa?ii mei.
\par 5 Pusu-mi-au cei mândri cursa mie ?i funii; curse au întins picioarelor mele; pe carare mi-au pus pietre de poticneala.
\par 6 Zis-am Domnului: "Dumnezeul meu e?ti Tu, asculta, Doamne, glasul rugaciunii mele".
\par 7 Doamne, Doamne, puterea mântuirii mele, umbrit-ai capul meu în zi de razboi.
\par 8 Sa nu ma dai pe mine, Doamne, din pricina poftei mele, pe mina pacatosului; viclenit-au împotriva mea; sa nu ma parase?ti, ca sa nu se trufeasca.
\par 9 Vârful la?ului lor, osteneala buzelor lor sa-i acopere pe ei!
\par 10 Sa cada peste ei carbuni aprin?i; în foc arunca-i pe ei, în necazuri, pe care sa nu le poata rabda.
\par 11 Barbatul limbut nu se va îndrepta pe pamânt; pe omul nedrept rautatea îl va duce la pieire.
\par 12 ?tiu ca Domnul va face judecata celui sarac ?i dreptate celor sarmani;
\par 13 Iar drep?ii vor lauda numele Tau ?i vor locui cei drep?i în fa?a Ta.

\chapter{141}

\par 1 (Un psalm al lui David.) Doamne, strigat-am catre Tine, auzi-ma; ia aminte la glasul rugaciunii mele, când strig catre Tine.
\par 2 Sa se îndrepteze rugaciunea mea ca tamâia înaintea Ta; ridicarea mâinilor mele, jertfa de seara.
\par 3 Pune Doamne, straja gurii mele ?i u?a de îngradire, împrejurul buzelor mele.
\par 4 Sa nu aba?i inima mea spre cuvinte de vicle?ug, ca sa-mi dezvinova?esc pacatele mele; iar cu oamenii cei care fac faradelege nu ma voi înso?i cu ale?ii lor.
\par 5 Certa-ma-va dreptul cu mila ?i ma va mustra, iar untdelemnul pacato?ilor sa nu unga capul meu; ca înca ?i rugaciunea mea este împotriva vrerilor lor.
\par 6 Prabu?easca-se de pe stânca judecatorii lor. Auzi-se-vor graiurile mele ca s-au îndulcit,
\par 7 Ca o brazda de pamânt s-au rupt pe pamânt, risipitu-s-au oasele lor lânga iad.
\par 8 Caci catre Tine, Doamne, Doamne, ochii mei, spre Tine am nadajduit, sa nu iei sufletul meu.
\par 9 Paze?te-ma de cursa care mi-au pus mie ?i de smintelile celor ce fac faradelege.
\par 10 Cadea-vor în mreaja lor pacato?ii, ferit sunt eu pâna ce voi trece.

\chapter{142}

\par 1 (Un psalm al lui David: al priceperii; când era el în pe?tera. O rugaciune.) Cu glasul meu catre Domnul am strigat, cu glasul meu catre Domnul m-am rugat.
\par 2 Varsa-voi înaintea Lui rugaciunea mea, necazul meu înaintea Lui voi spune.
\par 3 Când lipsea dintru mine duhul meu, Tu ai cunoscut cararile mele. În calea aceasta în care am umblat, ascuns-au cursa mie.
\par 4 Luat-am seama de-a dreapta ?i am privit ?i nu era cine sa ma cunoasca. Pierit-a fuga de la mine ?i nu este cel ce cauta sufletul meu.
\par 5 Strigat-am catre Tine, Doamne, zis-am: "Tu e?ti nadejdea mea, partea mea e?ti în pamântul celor vii".
\par 6 Ia aminte la rugaciunea mea, ca m-am smerit foarte. Izbave?te-ma de cei ce ma prigonesc, ca s-au întarit mai mult decât mine.
\par 7 Scoate din temni?a sufletul meu, ca sa laude numele Tau, Doamne. Pe mine ma a?teapta drep?ii, pâna ce-mi vei rasplati mie.

\chapter{143}

\par 1 (Un psalm al lui David: al priceperii; când îl prigonea pe el fiul lui.) Doamne, auzi rugaciunea mea, asculta cererea mea, întru credincio?ia Ta, auzi-ma, întru dreptatea Ta.
\par 2 Sa nu intri la judecata cu robul Tau, ca nimeni din cei vii nu-i drept înaintea Ta.
\par 3 Vrajma?ul prigone?te sufletul meu ?i via?a mea o calca în picioare; facutu-m-a sa locuiesc în întuneric ca mor?ii cei din veacuri.
\par 4 Mâhnit e duhul în mine ?i inima mea încremenita înlauntrul meu.
\par 5 Adusu-mi-am aminte de zilele cele de demult; cugetat-am la toate lucrurile Tale, la faptele mâinilor Tale m-am gândit.
\par 6 Întins-am catre Tine mâinile mele, sufletul meu ca un pamânt înseto?at.
\par 7 Degrab auzi-ma, Doamne, ca a slabit duhul meu. Nu-?i întoarce fala Ta de la mine, ca sa nu ma aseman celor ce se coboara în mormânt.
\par 8 Fa sa aud diminea?a mila Ta, ca la Tine îmi este nadejdea. Arata-mi calea pe care voi merge, ca la Tine am ridicat sufletul meu.
\par 9 Scapa-ma de vrajma?ii mei, ca la Tine alerg, Doamne.
\par 10 Înva?a-ma sa fac voia Ta, ca Tu e?ti Dumnezeul meu. Duhul Tau cel bun sa ma pova?uiasca la pamântul drepta?ii.
\par 11 Pentru numele Tau, Doamne, daruie?te-mi via?a. Întru dreptatea Ta scoate din necaz sufletul meu.
\par 12 Fa bunatate de stârpe?te pe vrajma?ii mei ?i pierde pe to?i cei ce necajesc sufletul meu, ca eu sunt robul Tau.

\chapter{144}

\par 1 (Un psalm al lui David împotriva lui Goliat.) Binecuvântat este Domnul Dumnezeul meu, Cel ce înva?a mâinile mele la lupta ?i degetele mele la razboi.
\par 2 Mila mea ?i Scaparea mea, Sprijinitorul meu ?i Izbavitorul meu, Aparatorul meu, ?i în El am nadajduit, Cel ce supune pe poporul meu sub mine.
\par 3 Doamne, ce este omul ca Te-ai facut cunoscut lui, sau fiul omului ca-l socote?ti pe el?
\par 4 Omul cu de?ertaciunea se aseamana; zilele lui ca umbra trec.
\par 5 Doamne, pleaca cerurile ?i Te pogoara, atinge-Te de mun?i ?i fa-i sa fumege.
\par 6 Cu fulger fulgera-i ?i-i risipe?te! Trimite sage?ile Tale ?i tulbura-i!
\par 7 Trimite mâna Ta dintru înal?ime; scoate-ma ?i ma izbave?te de ape multe, din mâna strainilor,
\par 8 A caror gura a grait de?ertaciune ?i dreapta lor e dreapta nedrepta?ii.
\par 9 Dumnezeule, cântare noua Î?i voi cânta ?ie; în psaltire cu zece strune Î?i voi cânta ?ie,
\par 10 Celui ce dai mântuire împara?ilor, Celui ce izbave?ti pe David, robul Tau, din robia cea cumplita.
\par 11 Izbave?te-ma ?i ma scoate din mâna strainilor, a caror gura a grait de?ertaciune ?i dreapta lor e dreapta nedrepta?ii,
\par 12 Ai caror fii sunt ca ni?te odrasle tinere, crescute în tinere?ile lor; fiicele lor înfrumuse?ate ?i împodobite ca chipurile templului.
\par 13 Camarile lor pline, varsându-se din una în alta. Oile lor cu mul?i miei, umplând drumurile când ies;
\par 14 Boii lor sunt gra?i. Nu este gard cazut, nici spartura, nici strigare în uli?ele lor.
\par 15 Au fericit pe poporul care are aceste bunata?i. Dar fericit este poporul acela care are pe Domnul ca Dumnezeu al sau.

\chapter{145}

\par 1 (Un psalm de lauda al lui David.) Înal?a-Te-voi Dumnezeul meu, Împaratul meu ?i voi binecuvânta numele Tau în veac ?i în veacul veacului.
\par 2 În toate zilele Te voi binecuvânta ?i voi lauda numele Tau în veac ?i în veacul veacului.
\par 3 Mare este Domnul ?i laudat foarte ?i mare?ia Lui nu are sfâr?it.
\par 4 Neam ?i neam vor lauda lucrurile Tale ?i puterea Ta o vor vesti.
\par 5 Mare?ia slavei sfin?eniei Tale vor grai ?i minunile Tale vor istorisi
\par 6 ?i puterea lucrurilor Tale înfrico?atoare vor spune ?i slava Ta vor povesti.
\par 7 Pomenirea mul?imii bunata?ii Tale vor vesti ?i de dreptatea Ta se vor bucura.
\par 8 Îndurat ?i milostiv este Domnul, îndelung-rabdator ?i mult-milostiv.
\par 9 Bun este Domnul cu to?i ?i îndurarile Lui peste toate lucrurile Lui.
\par 10 Sa Te laude pe Tine, Doamne, toate lucrurile Tale ?i cuvio?ii Tai sa Te binecuvânteze.
\par 11 Slava împara?iei Tale vor spune ?i de puterea Ta vor grai.
\par 12 Ca sa se faca fiilor oamenilor cunoscuta puterea Ta ?i slava mare?iei împara?iei Tale.
\par 13 Împara?ia Ta este împara?ia tuturor veacurilor, iar stapânirea Ta din neam în neam. Credincios este Domnul întru cuvintele Sale ?i cuvios întru toate lucrurile Sale.
\par 14 Domnul sprijina pe to?i cei ce cad ?i îndreapta pe to?i cei gârbovi?i.
\par 15 Ochii tuturor spre Tine nadajduiesc ?i Tu le dai lor hrana la buna vreme.
\par 16 Deschizi Tu mâna Ta ?i de bunavoin?a saturi pe to?i cei vii.
\par 17 Drept este Domnul în toate caile Lui ?i cuvios în toate lucrurile Lui.
\par 18 Aproape este Domnul de to?i cei ce-L cheama pe El, de to?i cei ce-L cheama pe El întru adevar.
\par 19 Voia celor ce se tem de El o va face ?i rugaciunea lor o va auzi ?i-i va mântui pe dân?ii.
\par 20 Domnul paze?te pe to?i cei ce-L iubesc pe El ?i pe to?i pacato?ii îi va pierde.
\par 21 Lauda Domnului va grai gura mea ?i sa binecuvinteze tot trupul  numele cel sfânt al Lui, zn veac ?i în veacul veacului.

\chapter{146}

\par 1 (Un psalm al lui Agheu ?i al lui Zaharia.) Aliluia! Lauda, suflete al meu, pe Domnul.
\par 2 Lauda-voi pe Domnul în via?a mea, cânta-voi Dumnezeului meu cât voi trai.
\par 3 Nu va încrede?i în cei puternici, în fiii oamenilor, în care nu este izbavire.
\par 4 Ie?i-va duhul lor ?i se vor întoarce în pamânt. În ziua aceea vor pieri toate gândurile lor.
\par 5 Fericit cel ce are ajutor pe Dumnezeul lui Iacob, nadejdea lui, în Domnul Dumnezeul lui,
\par 6 Cel ce a facut cerul ?i pamântul, marea ?i toate cele din ele; Cel ce paze?te adevarul în veac;
\par 7 Cel ce face judecata celor napastui?i, Cel ce da hrana celor flamânzi. Domnul dezleaga pe cei fereca?i în obezi;
\par 8 Domnul îndreapta pe cei gârbovi?i, Domnul în?elep?e?te orbii, Domnul iube?te pe cei drep?i;
\par 9 Domnul paze?te pe cei straini; pe orfani ?i pe vaduva va sprijini ?i calea pacato?ilor o va pierde.
\par 10 Împara?i-va Domnul în veac, Dumnezeul tau, Sioane, în neam ?i în neam.

\chapter{147}

\par 1 (Un psalm al lui Agheu ?i al lui Zaharia.) Aliluia! Lauda?i pe Domnul, ca bine este a cânta; Dumnezeului nostru placuta Îi este cântarea.
\par 2 Când va zidi Ierusalimul, Domnul va aduna ?i pe cei risipi?i ai lui Israel;
\par 3 Cel ce vindeca pe cei zdrobi?i cu inima ?i leaga ranile lor
\par 4 Cel ce numara mul?imea stelelor ?i da tuturor numele lor.
\par 5 Mare este Domnul nostru ?l mare este taria Lui ?i priceperea Lui nu are hotar.
\par 6 Domnul înal?a pe cei blânzi ?i smere?te pe cei pacato?i pâna la pamânt.
\par 7 Cânta?i Domnului cu cântare de lauda; cânta?i Dumnezeului nostru în alauta;
\par 8 Celui ce îmbraca cerul cu nori, Celui ce gate?te pamântului ploaie, Celui ce rasare în mun?i iarba ?i verdea?a spre slujba oamenilor;
\par 9 Celui ce da animalelor hrana lor ?i puilor de corb, care Îl cheama pe El.
\par 10 Nu în puterea calului este voia Lui, nici în cel iute la picior bunavoin?a Lui.
\par 11 Bunavoin?a Domnului este în cei ce se tem de El ?i în cei ce nadajduiesc în mila Lui.
\par 12 Lauda, Ierusalime, pe Domnul, lauda pe Dumnezeul tau, Sioane,
\par 13 Ca a întarit stâlpii por?ilor tale, a binecuvântat pe fiii tai, în tine.
\par 14 Cel ce pune la hotarele tale pace ?i cu fruntea grâului te-a saturat,
\par 15 Cel ce trimite cuvântul Sau pamântului; repede alearga cuvântul Lui;
\par 16 Cel ce da zapada ca lâna, Cel ce presara negura ca cenu?a,
\par 17 Cel ce arunca ghea?a Lui, ca buca?elele de pâine; gerul Lui cine-l va suferi?
\par 18 Va trimite cuvântul Lui ?i le va topi; va sufla Duhul Lui ?i vor curge apele.
\par 19 Cel ce veste?te cuvântul Sau lui Iacob, îndreptarile ?i judeca?ile Sale lui Israel.
\par 20 N-a facut a?a oricarui neam ?i judeca?ile Sale nu le-a aratat lor.

\chapter{148}

\par 1 (Un psalm al lui Agheu ?i al lui Zaharia.) Aliluia! Lauda?i pe Domnul din ceruri, lauda?i-L pe El întru cele înalte.
\par 2 Lauda?i-L pe El to?i îngerii Lui, lauda?i-L pe El toate puterile Lui.
\par 3 Lauda?i-L pe El soarele ?i luna, lauda?i-L pe El toate stelele ?i lumina.
\par 4 Lauda?i-L pe El cerurile cerurilor ?i apa cea mai presus de ceruri,
\par 5 Sa laude numele Domnului, ca El a zis ?i s-au facut, El a poruncit ?i s-au zidit.
\par 6 Pusu-le-ai pe ele în veac ?i în veacul veacului; lege le-a pus ?i nu o vor trece.
\par 7 Lauda?i pe Domnul to?i cei de pe pamânt, balaurii ?i toate adâncurile;
\par 8 Focul, grindina, zapada, ghea?a, viforul, toate îndeplini?i cuvântul Lui;
\par 9 Mun?ii ?i toate dealurile, pomii cei roditori ?i to?i cedrii;
\par 10 Fiarele ?i toate animalele, târâtoarele ?i pasarile cele zburatoare;
\par 11 Împara?ii pamântului ?i toate popoarele, capeteniile ?i to?i judecatorii pamântului;
\par 12 Tinerii ?i fecioarele, batrânii cu tinerii,
\par 13 Sa laude numele Domnului, ca numai numele Lui s-a înal?at. Lauda Lui pe pamânt ?i în cer.
\par 14 ?i va înal?a puterea poporului Lui. Cântare tuturor cuvio?ilor Lui, fiilor lui Israel, poporului ce se apropie de El.

\chapter{149}

\par 1 Aliluia! Cânta?i Domnului cântare noua, lauda Lui în adunarea celor cuvio?i.
\par 2 Sa se veseleasca Israel de Cel ce l-a facut pe el ?i fiii Sionului sa se bucure de Împaratul lor.
\par 3 Sa laude numele Lui în hora; în timpane ?i în psaltire sa-I cânte Lui.
\par 4 Ca iube?te Domnul poporul Sau ?i va înva?a pe cei blânzi ?i-i va izbavi.
\par 5 Se vor lauda cuvio?ii întru slava ?i se vor bucura în a?ternuturile lor.
\par 6 Laudele Domnului în gura lor ?i sabii cu doua tai?uri în mâinile lor,
\par 7 Ca sa se razbune pe neamuri ?i sa pedepseasca pe popoare,
\par 8 Ca sa lege pe împara?ii lor în obezi ?i pe cei slavi?i ai lor în catu?e de fier,
\par 9 Ca sa faca între dân?ii judecata scrisa. Slava aceasta este a tuturor cuvio?ilor Sai.

\chapter{150}

\par 1 Aliluia! Lauda?i pe Domnul întru sfin?ii Lui; lauda?i-L pe El întru taria puterii Lui.
\par 2 Lauda?i-L pe El întru puterile Lui; lauda?i-L pe El dupa mul?imea slavei Lui.
\par 3 Lauda?i-L pe El în glas de trâmbi?a; lauda?i-L pe El în psaltire ?i în alauta.
\par 4 Lauda?i-L pe El în timpane ?i în hora; lauda?i-L pe El în strune ?i organe.
\par 5 Lauda?i-L pe El în chimvale bine rasunatoare; lauda?i-L pe El în chimvale de strigare.
\par 6 Toata suflarea sa laude pe Domnul!


\end{document}