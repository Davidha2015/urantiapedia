\begin{document}

\title{Proverbele}


\chapter{1}

\par 1 Pildele lui Solomon, fiul lui David,
\par 2 Folositoare pentru cunoașterea înțelepciunii și a stăpânirii de sine,
\par 3 Pentru înțelegerea cuvintelor adânci, pentru dobândirea unei îndrumări bune, pentru dreptate, pentru dreapta judecată și nepărtinire,
\par 4 Pentru a prilejui celor fără gând rău o judecată isteață, omului tânăr cunoștință și bună cugetare.
\par 5 Să ia aminte cel înțelept și își va spori știința, iar cel priceput va dobândi iscusința de a se purta,
\par 6 Pătrunzând cu mintea pildele și înțelesurile adânci, graiurile celor înțelepți și tâlcuirea lor nepătrunsă.
\par 7 Frica de Dumnezeu este începutul înțelepciunii; cei fără minte disprețuiesc înțelepciunea și stăpânirea de sine.
\par 8 Ascultă, fiul meu, învățătura tatălui tău și nu lepăda îndrumările maicii tale.
\par 9 Căci ele sunt ca o cunună pe capul tău Și ca o salbă împrejurul gâtului tău.
\par 10 Fiul meu, de voiesc păcătoșii să te ademenească, nu te învoi,
\par 11 Dacă-ți spun: "Vino cu noi, să ne punem la pândă, ca să vărsăm sânge, să întindem curse fără cuvânt celui neprihănit,
\par 12 Să-i înghițim de vii ca locuința morților, și întregi, ca pe cei ce se coboară în mormânt.
\par 13 Să punem stăpânire pe tot felul de lucruri scumpe, să ne umplem de pradă casele noastre,
\par 14 Fii părtaș la obștea noastră, o singură pungă fi-va pentru toți!"
\par 15 Fiul meu, nu te întovărăși cu ei pe cale; abate piciorul tău din cărarea lor,
\par 16 Căci picioarele lor aleargă numai la rău, iar ei zoresc să verse sânge.
\par 17 Zadarnic se întind curse în văzul păsărilor!
\par 18 Căci ei întind curse tocmai împotriva sângelui lor, și sufletului lor își întind ei lațuri.
\par 19 Aceasta este soarta celor lacomi de câștig; lăcomia le aduce pierderea vieții.
\par 20 Înțelepciunea strigă pe uliță și în piele își ridică glasul său.
\par 21 Ea propovăduiește la răspântiile zgomotoase; înaintea porților cetății își spune cuvântul:
\par 22 "Până când, proștilor, veți iubi prostia? Până când, nebunilor, veți iubi nebunia? Și voi, neștiutorilor, până când veți urî știința?
\par 23 Întoarceți-vă iarăși la mustrarea mea și iată eu voi turna peste voi duhul meu și vă voi vesti cuvintele mele.
\par 24 Chematu-v-am, dar voi n-ați luat aminte! Întinsu-mi-am mâna și n-a fost cine să ia seama!
\par 25 Ci ați lepădat toate sfaturile mele și mustrările mele nu le-ați primit.
\par 26 De aceea și eu voi râde de pieirea voastră și mă voi bucura când va veni groaza peste voi,
\par 27 Când va veni peste voi necazul ca furtuna și când nenorocirea ca vijelia vă va cuprinde.
\par 28 Atunci mă vor chema, dar eu nu voi răspunde; din zori mă vor căuta și nu mă vor afla,
\par 29 Pentru că ei au urât știința și frica de Dumnezeu n-au ales-o,
\par 30 Fiindcă n-au luat aminte la sfaturile mele și cercetarea mea au disprețuit-o.
\par 31 Mânca-vor din rodul căii lor și de sfaturile lor sătura-se-vor,
\par 32 Căci îndărătnicia omoară pe cei proști și nepăsarea pierde pe cei fără minte;
\par 33 Iar cel ce mă ascultă va trăi în pace și liniște și de rele nu se va teme".

\chapter{2}

\par 1 Fiul meu, de vei primi povețele mele și sfaturile mele de le vei păstra,
\par 2 Plecându-ți urechea la înțelepciune și înclinând inima ta spre bună chibzuială,
\par 3 Dacă vei chema prevederea și spre buna-cugetare îți vei îndrepta glasul tău,
\par 4 Dacă o vei căuta întocmai ca pe argint și o vei săpa ca și pe o comoară,
\par 5 Atunci vei pricepe temerea de Domnul și vei dobândi cunoștința de Dumnezeu,
\par 6 Căci Domnul dă înțelepciune; din gura Lui izvorăște știința și prevederea;
\par 7 El păstrează mântuirea pentru oamenii cei drepți; El este scut pentru cei ce umblă în calea desăvârșirii;
\par 8 El păzește căile dreptății și pe cărarea celor cuvioși ai Lui stă de veghe.
\par 9 Atunci tu vei înțelege dreptatea și buna judecată, calea cea dreaptă și toate potecile binelui.
\par 10 Când înțelepciunea se va sui la inima ta și știința va desfăta sufletul tău,
\par 11 Când buna chibzuială va veghea peste tine și înțelegerea te va păzi,
\par 12 Atunci tu vei fi izbăvit de calea celui rău și de omul care grăiește minciună,
\par 13 De cei ce părăsesc căile cele drepte, ca să umble pe drumuri întunecoase,
\par 14 De cei ce se bucură când fac rău și se veselesc când umblă pe poteci întortochiate,
\par 15 Ale căror cărări sunt strâmbe și rătăcesc pe căi piezișe.
\par 16 Atunci tu vei scăpa de femeia care este a altuia, de străina ale cărei cuvinte sunt ademenitoare,
\par 17 Care lasă pe tovarășul ei din tinerețe și uită de legământul Dumnezeului ei,
\par 18 Căci ea se pleacă împreună cu casa ei spre moarte și drumul ei duce în iad;
\par 19 Nimeni din cei ce se duc la ea nu se mai întoarce și niciunul nu mai află cărările vieții.
\par 20 Drept aceea mergi pe calea oamenilor celor buni și păzește cărările celor drepți,
\par 21 Căci cei drepți vor locui pământul și cei fără de prihană vor sălășlui pe el;
\par 22 Iar cei fără de lege vor fi nimiciți de pe pământ și cei necredincioși vor fi smulși de pe el.

\chapter{3}

\par 1 Fiul meu, nu uita învățătura mea și inima ta să păzească sfaturile mele,
\par 2 Căci lungime de zile și ani de viață și propășire îi se vor adăuga.
\par 3 Mila și adevărul să nu te părăsească; leagă-le împrejurul gâtului tău, scrie-le pe tabla inimii tale;
\par 4 Atunci vei afla har și bunăvoință înaintea lui Dumnezeu și a oamenilor.
\par 5 Pune-ți nădejdea în Domnul din toată inima ta și nu te bizui pe priceperea ta.
\par 6 Pe toate căile tale gândește la Dânsul și El îți va netezi toate cărările tale.
\par 7 Nu fii înțelept în ochii tăi; teme-te de Dumnezeu și fugi de rău;
\par 8 Aceasta va fi sănătate pentru trupul tău și o înviorare pentru oasele tale.
\par 9 Cinstește pe Domnul din averea ta și din pârga tuturor roadelor tale.
\par 10 Atunci jitnițele tale se vor umple de grâu și mustul va da afară din teascurile tale.
\par 11 Fiul meu, nu disprețui certarea Domnului și nu simți scârbă pentru mustrările Lui,
\par 12 Căci Domnul ceartă pe cel pe care-l iubește și ca un părinte pedepsește pe feciorul care îi este drag.
\par 13 Fericit este omul care a aflat înțelepciunea și bărbatul care a dobândit pricepere,
\par 14 Căci dobândirea ei este mai scumpă decât argintul și prețul ei mai mare decât al celui mai curat aur.
\par 15 Ea este mai prețioasă decât pietrele scumpe; nici un rău nu i se poate împotrivi și e bine-cunoscută tuturor celor ce se apropie de ea; nimic din cele dorite de tine nu se aseamănă cu ea.
\par 16 Viață lungă este în dreapta ei, iar în stânga ei, bogăție și slavă; din gura ei iese dreptatea; legea și mila pe limbă le poartă.
\par 17 Căile ei sunt plăcute și toate cărările ei sunt căile păcii.
\par 18 Pom al vierii este ea pentru cei ce o stăpânesc, iar cei care se sprijină pe ea sunt fericiți.
\par 19 Prin înțelepciune, Domnul a întemeiat pământul, iar prin înțelegere a întărit cerurile.
\par 20 Prin știința Sa a deschis adâncurile și norii picură rouă.
\par 21 Fiul meu, să nu se depărteze acestea dinaintea ochilor tăi; păstrează înțelepciunea și buna chibzuială,
\par 22 Căci ele sunt viața sufletului tău și podoabă pentru gâtul tău.
\par 23 Atunci tu vei merge fără teamă pe calea ta și piciorul tău nu se va poticni.
\par 24 De te culci, nu-ți va fi teamă, iar de adormi, somnul tău va fi dulce.
\par 25 Să nu te temi de frica fără veste și nici de vreo năvală a celor păcătoși,
\par 26 Că Domnul este nădejdea ta și va feri piciorul tău de cursă.
\par 27 Nu zăbovi a face bine celui ce are nevoie, când ai putința să-i ajuți.
\par 28 Nu spune aproapelui tău: "Du-te și vino, mâine îți voi da!", când poți să-i dai acum.
\par 29 Nu pune la cale răul împotriva aproapelui tău, când el locuiește fără grijă lângă tine.
\par 30 Nu te certa cu nimeni fără pricină, de vreme ce nu ți-a făcut nici un rău.
\par 31 Nu râvni să fii ca omul silnic și nu alege nici una din căile lui;
\par 32 Căci omul cu gând rău este urât de Domnul, iar de cei drepți El este mai aproape.
\par 33 Domnul blesteamă casa celui fără de lege și binecuvântează adăposturile celor drepți.
\par 34 De cei batjocoritori El râde, iar celor smeriți le dă har.
\par 35 Cei înțelepți vor moșteni mărirea, iar cei nebuni vor avea parte de ocară.

\chapter{4}

\par 1 Ascultați, fiilor, învățătura tatălui și luați aminte să cunoașteți buna chibzuială,
\par 2 Căci eu vă dau învățătură bună: Nu părăsiți povața mea.
\par 3 Căci și eu am fost fecior la tatăl meu, singur, și cu duioșie iubit la mama mea
\par 4 Și el mă învăța și-mi zicea: "Inima ta să păstreze cuvintele inimii mele, păzește poruncile mele și vei fi viu.
\par 5 Adună înțelepciune, dobândește pricepere! Nu le uita și nu te depărta de la cuvintele gurii mele!
\par 6 Nu o lepăda și ea te va păzi; iubește-o și ea va sta de veghe.
\par 7 Iată începutul înțelepciunii: Agonisește înțelepciunea și cu prețul a tot ce ai, capătă priceperea.
\par 8 Prețuiește-o mult și ea te va înălța; ea te va ridica în slăvi dacă o vei îmbrățișa.
\par 9 Ea va pune cunună de daruri pe capul tău și te va împodobi cu diademă de mare cinste.
\par 10 Ascultă, fiul meu și primește cuvintele mele și anii vieții tale se vor înmulți.
\par 11 Eu te voi învăța calea înțelepciunii și te voi purta pe căile dreptății.
\par 12 Când vei merge, pașii tăi nu vor șovăi și, chiar de vei alerga, nu te vei poticni.
\par 13 Ține cu tărie învățătura și nu o părăsi, păzește-o căci ea este viața ta.
\par 14 Nu apuca pe calea celor fără de lege și nu păși pe drumul celor răi.
\par 15 Ocolește-o și nu merge pe ea, treci pe alăturea și du-te mai departe;
\par 16 Căci ei nu dorm până nu făptuiesc rău și nu-i mai prinde somnul până nu fac pe cineva să cadă.
\par 17 Căci ei se hrănesc din pâine agonisită prin fărădelege și beau vin dobândit prin asuprire.
\par 18 Calea drepților e ca zarea dimineții ce se mărește mereu până se face ziua mare;
\par 19 Iar calea celor fără de lege e ca întunericul și ei nici nu bănuiesc de ce se pot împiedica.
\par 20 Fiul meu, ia aminte la graiurile mele; la povețele mele pleacă-ți urechea ta!
\par 21 Nu le scăpa din ochi, păstrează-le înlăuntrul inimii tale,
\par 22 Căci ele sunt viață pentru cei ce le pun în faptă și doctorie pentru tot trupul omenesc.
\par 23 Păzește-ți inima mai mult decât orice, căci din ea țâșnește viața.
\par 24 Leapădă din gura ta orice cuvinte cu înțeles sucit, alungă de pe buzele tale viclenia.
\par 25 Ochii tăi să privească drept înainte și genele tale drept înainte să caute.
\par 26 Fii cu luare aminte la calea picioarelor tale și toate cărările tale să fie bine chibzuite.
\par 27 Nu te abate nici la dreapta, nici la stânga, ține piciorul tău departe de rău. Căci cărările drepte le păzește Domnul, iar cele strâmbe sunt căi rele. El va face drepte căile tale și mergerea ta o va face să fie în pace.

\chapter{5}

\par 1 Fiul meu, ia aminte la înțelepciunea mea și la sfatul meu cel bun pleacă urechea ta,
\par 2 Ca să-ți poți păstra judecata și ca buzele tale să păzească știința.
\par 3 Nu te uita la femeia lingușitoare, căci buzele celei străine picură miere și cerul gurii sale e mai alunecător decât untdelemnul,
\par 4 Dar la sfârșit ea este mai amară decât pelinul, mai tăioasă decât o sabie cu două ascuțișuri.
\par 5 Picioarele ei coboară către moarte; pașii ei duc de-a dreptul în împărăția morții.
\par 6 Ea nu ia seama la calea vieții, pașii ei merg în neștire, nici ea nu știe unde.
\par 7 Și acum, fiul meu, ascultă-mă și nu te îndepărta de la cuvintele gurii mele.
\par 8 Ferește-ți calea ta de ea și nu te apropia de ușa casei ei,
\par 9 Ca să nu dai vârtutea ta altora și anii tăi unuia fără de milă;
\par 10 Ca străinii să nu se îndestuleze de strădania ta și ostenelile tale să nu treacă în casa altuia;
\par 11 Ca să nu suspini la sfârșit, când trupul tău și carnea ta vor fi fără de vlagă,
\par 12 Și să zici: "Pentru ce am urât povața și de ce inima mea a urgisit certarea?
\par 13 De ce nu am ascultat de îndemnul dascălilor mei și spre cei ce mă învățau n-am plecat urechea mea?
\par 14 Puțin a trebuit să nu mă nenorocesc, în plină adunare și în mijlocul obștei".
\par 15 Bea apă din puțul tău și din pârâiașele care curg din izvorul tău.
\par 16 Să nu se risipească izvoarele tale pe uliță, nici pâraiele tale prin piețe.
\par 17 Să fie numai pentru tine singur, iar nu pentru străinii care sunt cu tine!
\par 18 Binecuvântat să fie izvorul tău și să te mângâi cu femeia ta din tinerețe.
\par 19 Cerboaică preaiubită și gazelă plină de farmec să-ți fie ea; dragostea de ea să te îmbete totdeauna și iubirea ei să te desfăteze.
\par 20 Pentru ce, fiul meu, să te momească femeie străină și tu să îmbrățișezi sânul unei necunoscute?
\par 21 Căci cărările omului sunt înaintea Domnului și El ia seama la toate căile lui.
\par 22 Cel fără de lege este prins în lațurile fărădelegilor lui și de funiile păcatelor lui este înfășurat.
\par 23 El va muri în păcatele lui și de mulțimea nebuniei lui va pieri.

\chapter{6}

\par 1 Fiul meu, dacă te-ai pus chezaș pentru prietenul tău, dacă ai dat mâna pentru altul,
\par 2 Atunci te-ai prins prin făgăduieli ieșite din gura ta și te-ai legat prin cuvintele gurii tale.
\par 3 Fă dar, fiul meu, aceasta: o, izbăvește-te. Și fiindcă ai căzut în mâinile aproapelui tău, du-te și cazi la picioarele aproapelui tău și-l roagă;
\par 4 Nu da somn ochilor tăi, nici dormitare genelor tale.
\par 5 Și te izbăvește ca o căprioară din mâna vânătorului și ca o pasăre din mâna păsărarului.
\par 6 Du-te, leneșule, la furnică și vezi munca ei și prinde minte!
\par 7 Ea, care nu are nici mai-mare peste ea, nici îndrumător, nici sfătuitor,
\par 8 Își pregătește de cu vară hrana ei și își strânge la seceriș mâncare. Sau mergi la albină și vezi cât e de harnică și ce lucrare iscusită săvârșește. Munca ei o folosesc spre sănătate și regii și oamenii de rând. Ea e iubită și lăudată de toți, deși e slabă în putere, dar e minunată cu iscusința.
\par 9 Până când, leneșule, vei mai sta culcat? Când te vei scula din somnul tău?
\par 10 "Puțin somn, încă puțină ațipire, puțin să mai stau în pat cu mâinile încrucișate!"
\par 11 Iată vine sărăcia ca un trecător și nevoia te prinde ca un tâlhar. Dar dacă nu vei lenevi, atunci va veni secerișul tău ca un izvor, iar lipsa va fi departe de tine.
\par 12 Omul de nimic, omul necinstit și viclean umblă cu minciuna pe buze.
\par 13 Face cu ochiul, dă din picioare, face semne cu degetele.
\par 14 în inima lui e vicleșug, pururea se gândește la rău și seamănă gâlceavă.
\par 15 Pentru aceasta fără de veste va veni peste el prăpădul, nimicit va fi dintr-o dată și fără leac.
\par 16 Șase sunt lucrurile pe care le urăște Domnul, ba chiar șapte de care se scârbește cugetul Său:
\par 17 Ochii mândri, limba mincinoasă, mâinile care varsă sânge nevinovat,
\par 18 Inima care plănuiește gânduri viclene, picioare grabnice să alerge spre rău,
\par 19 Martorul mincinos care spune minciuni și cel care seamănă vrajbă între frați.
\par 20 Păzește, fiule, povața tatălui tău și nu lepăda îndemnul maicii tale.
\par 21 Leagă-le la inima ta, pururea atârnă-le de gâtul tău.
\par 22 Ele te vor conduce când vei vrea să mergi; în vremea somnului te vor păzi, iar când te vei deștepta vor grăi cu tine.
\par 23 Că povața este un sfeșnic bun și legea o lumină, iar îndemnurile care dau învățătură sunt calea vieții.
\par 24 Ele te vor păzi de femeia vicleană, de limba cea ademenitoare a celei străine.
\par 25 Nu dori frumusețea ei întru inima ta și să nu te vâneze cu genele ei.
\par 26 Că femeia desfrânată umblă după o bucată de pâine, pe când femeia-soție dorește un suflet de mare preț.
\par 27 Oare poate pune cineva foc în sânul lui, fără ca veșmintele lui să nu ardă?
\par 28 Sau va merge cineva pe cărbuni fără să i se frigă tălpile?
\par 29 Așa este cu cel ce se duce la femeia aproapelui său: nimeni din cei ce se ating de ea nu va rămâne nepedepsit.
\par 30 Nimeni nu disprețuiește un hoț pentru că a furat ca să-și astâmpere foamea;
\par 31 Dar când a fost prins, el dă înapoi înșeptit, întoarce tot ceea ce are în casa lui.
\par 32 Cel ce se desfrânează însă cu o femeie este lipsit de minte, se pierde pe el însuși făcând astfel;
\par 33 El nu dobândește decât bătaie, iar ocara lui niciodată nu se șterge.
\par 34 Pizma trezește mânia omului defăimat și el este fără milă în ziua răzbunării;
\par 35 El nu se uită la nici un preț de răscumpărare, și chiar când îi vei spori darurile, tot nu se îmblânzește.

\chapter{7}

\par 1 Fiul meu; păzește spusele mele și îndrumările mele ascunde-le la tine.
\par 2 Păstrează sfaturile mele ca să rămâi în viață și orânduielile mele ca lumina ochilor tăi.
\par 3 Leagă-le pe degetele tale, scrie-le pe tabla inimii tale!
\par 4 Spune înțelepciunii: "Tu ești sora mea!", și numește priceperea prietena ta,
\par 5 Ca ea să te păzească de femeia străină, de femeia altuia, ale cărei cuvinte sunt ademenitoare.
\par 6 Odată stam la fereastra casei mele și priveam printre gratii,
\par 7 Și am zărit printre cei lipsiți de minte, am văzut un tânăr fără pricepere.
\par 8 El trecea pe uliță pe lângă colțul casei ei și se îndrepta către locuința ei.
\par 9 Era în amurgul serii unei zile, când se lasă umbra și întunericul nopții.
\par 10 Și iată o femeie îl întâmpină, având înfățișare de desfrânată și cu prefăcătorie în inimă;
\par 11 Aprigă și de neținut în frâu, picioarele ei nu se mai odihneau în casă;
\par 12 Când în casă, când afară, stând la pândă lângă orice colț.
\par 13 Ea îl apucă și-l sărută și cu o căutătură obraznică îi zise:
\par 14 "Trebuia să aduc jertfe de pace; astăzi am împlinit făgăduințele mele;
\par 15 Pentru aceasta am ieșit în întâmpinarea ta, ca să te caut și iată că te-am găsit.
\par 16 Cu scoarțe am gătit patul meu, cu așternuturi de in din Egipt,
\par 17 Cu miresme am stropit patul meu, cu mir, aloe și chinamon.
\par 18 Vino, să ne îmbătăm de iubire până dimineață, să ne cufundăm în desfătări de dragoste,
\par 19 Că bărbatul meu nu este acasă, plecat-a la drum departe,
\par 20 Luat-a cu dânsul o pungă cu bani și se va întoarce acasă la lună plină!"
\par 21 Ea îl ademeni prin mulțimea cuvintelor ei și-l smulse prin graiurile ademenitoare ale buzelor sale;
\par 22 El începu să meargă dintr-o dată după ea, ca un bou la junghiere și ca un cerb care se zorește spre capcană,
\par 23 Până când o săgeată îi străpunge ficatul; după cum o pasăre grăbește spre laț și nu-și dă seama că acolo își sfârșește viața.
\par 24 Și acum, fiule, ascultă-mă și ia aminte la cuvintele gurii mele!
\par 25 Inima ta să nu se plece spre căile ei și nu te rătăci pe potecile ei,
\par 26 Căci ea a rănit pe mulți și pe foarte mulți i-a omorât.
\par 27 Casa ei sunt căile iadului, care duc la cămările morții.

\chapter{8}

\par 1 Oare înțelepciunea nu strigă ea și priceperea nu-și ridică glasul său?
\par 2 Pe vârfurile cele mai înalte, pe cale, la răspântiile drumurilor stă,
\par 3 Pe lângă porți, în împrejurimile cetății, la intrarea porților, strigă tare:
\par 4 "Către voi, oamenilor, se îndreaptă strigătul meu și glasul meu către voi, fii ai oamenilor.
\par 5 Voi, cei simpli, învățați cumințenia și voi, cei nebuni, înțelepțiți-vă!
\par 6 Ascultați, căci voi spune lucruri mărețe și buzele mele se deschid pentru a înălța ceea ce este drept;
\par 7 Căci gura mea grăiește adevărul și buzele mele se dezgustă de fărădelege.
\par 8 Toate graiurile gurii mele sunt întru dreptate, în ele nu este nimic sucit și fără rost;
\par 9 Toate sunt lămurite pentru cel priceput și drepte pentru cei ce au aflat știința.
\par 10 Luați învățătura mea mai degrabă decât argintul și știința mai mult decât aurul cel mai curat,
\par 11 Căci înțelepciunea este mai bună decât pietrele prețioase și nici lucrurile cele mai prețioase nu au valoarea ei.
\par 12 Eu, înțelepciunea, locuiesc împreună cu prevederea și stăpânesc știința și buna-chibzuială.
\par 13 Frica de Dumnezeu este urgisirea răului. Mândria și obrăznicia, calea răutății și gura cea aprigă le urăsc eu.
\par 14 Al meu este sfatul și buna-chibzuială, eu sunt priceperea, a mea este puterea.
\par 15 Prin mine împărățesc împărații și principii rânduiesc dreptatea.
\par 16 Prin mine cârmuiesc dregătorii și mai-marii sunt judecătorii pământului.
\par 17 Eu iubesc pe cei ce mă iubesc și cei ce mă caută mă găsesc.
\par 18 Cu mine este bogăția și mărirea, averea vrednică de cinste și dreptatea.
\par 19 Rodul meu e mai bun decât aurul și decât aurul cel mai curat, și ceea ce vine de la mine este mai de preț decât argintul lămurit.
\par 20 Merg pe calea dreptății, în mijlocul căilor judecății drepte,
\par 21 Ca să dau celor ce mă iubesc bogății și să le umplu cămările lor.
\par 22 Domnul m-a zidit la începutul lucrărilor Lui; înainte de lucrările Lui cele mai de demult.
\par 23 Eu am fost din veac întemeiată de la început, înainte de a se fi făcut pământul.
\par 24 Nu era adâncul atunci când am fost născută, nici chiar izvoare încărcate cu apă.
\par 25 Înainte de a fi fost întemeiați munții și înaintea văilor eu am luat ființă.
\par 26 Când încă nu era făcut pământul, nici câmpiile, nici cel dintâi fir de praf din lume,
\par 27 Când El a întemeiat cerurile eu eram acolo; când El a tras bolta cerului peste fața adâncului,
\par 28 Când a întărit norii sus și izvoarele adâncului curgeau din belșug,
\par 29 Când El a pus hotar mării, ca apele să nu mai treacă peste țărmuri și când El a așezat temeliile pământului,
\par 30 Atunci eu eram ca un copil mic alături de El, veselindu-mă în fiecare zi și desfătându-mă fără încetare în fața Lui;
\par 31 Dezmierdându-mă pe rotundul pământului Lui și găsindu-mi plăcerea printre fiii oamenilor.
\par 32 Și acum, fiilor, ascultați-mă! Fericiți sunt cei ce păzesc căile mele!
\par 33 Ascultați învățătura, ca să ajungeți înțelepți, și nu o lepădați.
\par 34 Fericit este omul care ascultă de mine și veghează în fiecare zi la porțile mele și cel ce străjuiește lângă pragul casei mele!
\par 35 Cel ce mă află, a aflat viața și dobândește har de la Domnul;
\par 36 Iar cel ce păcătuiește împotriva mea își păgubește viața lui. Toți cei ce mă urăsc pe mine iubesc moartea".

\chapter{9}

\par 1 Înțelepciunea și-a zidit casă rezemată pe șapte stâlpi,
\par 2 A înjunghiat vite pentru ospăț, a pregătit vinul cu mirodenii și a întins masa sa.
\par 3 Ea a trimis slujnicele sale să strige pe vârfurile dealurilor cetății:
\par 4 "Cine este neînțelept să intre la mine!" Și celor lipsiți de buna-chibzuială le zice:
\par 5 "Veniți și mâncați din pâinea mea și beți din vinul pe care eu l-am amestecat cu mirodenii.
\par 6 Părăsiți neînțelepciunea ca să rămâneți cu viață și umblați pe calea cea dreaptă a priceperii!"
\par 7 Cel ce ceartă pe batjocoritor își atrage disprețul, și cel ce dojenește pe cel fără de lege își atrage ocara.
\par 8 Nu certa pe cel batjocoritor ca să nu te urască; dojenește pe cel înțelept, și el te va iubi.
\par 9 Dă sfat celui înțelept, și el se va face și mai înțelept; învață pe cel drept, și el își va spori știința lui.
\par 10 Începutul înțelepciunii este frica de Dumnezeu și priceperea este știința Celui Sfânt.
\par 11 Căci prin Domnul se vor înmulți zilele tale și se vor adăuga ție ani de viață.
\par 12 Dacă tu ești înțelept, ești înțelept pentru tine, și dacă ești batjocoritor, singur vei purta ponosul.
\par 13 Nebunia este o femeie gălăgioasă, proastă și care nu știe nimic.
\par 14 Ea stă la ușa casei sale, pe un scaun înalt și strigă,
\par 15 Ca să poftească pe cei ce trec pe drum și pe cei ce merg pe calea lor fără să se abată:
\par 16 "Cine este neînțelept să intre la mine!" Și celui lipsit de buna-chibzuială îi zice:
\par 17 "Apa furată e mai plăcută și pâinea mâncată pe furiș are gust mai bun".
\par 18 Și omul nu știe că acolo sunt numai umbre, iar cei pe care îi poftește nebunia se află de mult în adâncurile șeolului (locuința morților).

\chapter{10}

\par 1 Pildele lui Solomon. Fiul înțelept înveselește pe tatăl său, iar cel nebun este supărarea maicii lui.
\par 2 Nu sunt de nici un folos comorile dobândite prin fărădelege; numai dreptatea scapă de la moarte.
\par 3 Domnul nu lasă să piară de foame sufletul celui drept; însă el respinge lăcomia celor fără de lege.
\par 4 Mâna leneșilor pricinuiește sărăcie, iar mâna celor înțelepți adună avuții.
\par 5 Cel ce adună în timpul verii este un om prevăzător, iar cel care doarme în vremea secerișului este de ocară.
\par 6 Binecuvântarea Domnului vine pe capul celui drept, iar ocara acoperă fața celor fără de lege.
\par 7 Pomenirea celui drept este spre binecuvântare, iar numele celor nelegiuiți va fi blestemat.
\par 8 Cel cu inimă înțeleaptă primește sfaturile, iar cel nebun grăiește vorbe spre pieirea lui.
\par 9 Cel ce umblă întru neprihănire umblă pe cale sigură, iar cel ce umblă pe căi lăturalnice va fi dat de gol.
\par 10 Cel ce clipește din ochi va fi pricină de supărare; iar cel care ceartă cu inimă bună așază pacea.
\par 11 Izvor de viață este gura celui drept, dar gura celor fără de lege, izvor de nedreptate.
\par 12 Ura aduce ceartă, iar dragostea acopere toate cusururile.
\par 13 Pe buzele omului priceput se află înțelepciunea; toiagul este pentru spatele celui lipsit de chibzuință.
\par 14 Cei înțelepți ascund știința, iar gura celui fără de socotință este o nenorocire apropiată.
\par 15 Avuția este pentru cel bogat o cetate întărită; nenorocirea celor sărmani este sărăcia lor.
\par 16 Agonisita celui drept este spre viață; roadele celui fără de lege spre păcat;
\par 17 Cel ce păzește învățătura apucă pe calea vieții, iar cel ce leapădă certarea rătăcește.
\par 18 Cel care ascunde ura are buze mincinoase; cel ce răspândește defăimarea este un nebun.
\par 19 Mulțimea cuvintelor nu scutește de păcătuire, iar cel ce-și ține buzele lui este un om înțelept.
\par 20 Limba omului drept este argint curat, dar inima celor fără de lege este lucru de puțin preț.
\par 21 Buzele celui drept călăuzesc pe mulți oameni, iar cei nebuni mor din pricină că nu sunt pricepuți.
\par 22 Numai binecuvântarea Domnului îmbogățește, iar truda zadarnică nu aduce spor.
\par 23 Ca o pricină de bucurie este pentru nebun săvârșirea unei fapte rușinoase; la fel este cu înțelepciunea pentru omul priceput.
\par 24 De ceea ce se teme cel nelegiuit nu scapă, iar cererea celor drepți (Dumnezeu) o împlinește.
\par 25 Precum trece furtuna, așa piere și cel fără de lege, iar dreptul este ca o temelie neclintită.
\par 26 Precum este oțetul pentru dinți și fumul pentru ochi, așa este omul leneș pentru cei ce-l pun la treabă.
\par 27 Frica de Dumnezeu lungește zilele (omului), iar anii celor fără de lege sunt puțini.
\par 28 Nădejdea celor drepți este numai bucurie, iar nădejdea celor păcătoși sfârșește în rău.
\par 29 Calea Domnului este o întăritură pentru cel desăvârșit și o prăbușire pentru cei ce săvârșesc fărădelegi.
\par 30 Niciodată cel drept nu se va clătina, iar cei nelegiuiți nu vor locui pământul.
\par 31 Gura celui drept rodește înțelepciune, iar limba urzitoare de rele aduce pierzare.
\par 32 Buzele celui drept cunosc bunăvoirea, iar gura păcătoșilor strâmbătatea.

\chapter{11}

\par 1 Cântarul strâmb este urgisit de Domnul, și cântărirea dreaptă este plăcerea Lui.
\par 2 Dacă vine mândria, va veni și ocara, iar înțelepciunea este cu cei smeriți.
\par 3 Neprihănirea poartă pe cei drepți, iar strâmbătatea prăpădește pe cei vicleni.
\par 4 La nimic nu folosește bogăția în ziua mâniei; numai dreptatea izbăvește de moarte.
\par 5 Dreptatea netezește calea celui fără prihană, iar cel fără de lege va cădea prin fărădelegea lui.
\par 6 Dreptatea izbăvește pe cei drepți, iar cei vicleni vor fi prinși prin pofta lor.
\par 7 La moartea omului drept rămâne nădejdea, iar la moartea celui păcătos piere nădejdea.
\par 8 Dreptul scapă din strâmtorare, și cel fără de lege îi ia locul.
\par 9 Făptuitorul de rele prăbușește cu gura pe aproapele lui, iar prin știința celor drepți va fi mântuit.
\par 10 De propășirea celor drepți cetatea se bucură, iar când pier cei fără de lege ea tresaltă de veselie.
\par 11 Prin binecuvântarea oamenilor drepți cetatea merge înainte, iar prin gura celor nelegiuiți ajunge ruină.
\par 12 Cel nepriceput urgisește pe aproapele lui, iar omul cu bună-chibzuială tace.
\par 13 Grăitorul de rele dă pe față lucruri de taină, iar omul cu duhul cumpănit le ține ascunse.
\par 14 Unde lipsește cârmuirea, poporul cade; izbăvirea stă în mulțimea sfetnicilor.
\par 15 Celui ce se pune chezaș pentru un străin îi merge rău; cel ce nu se pune chezaș stă la adăpost.
\par 16 Femeia cu purtare bună agonisește cinstire, iar cea care urăște cinstea e o rușine. Nu leneșii ci silitorii agonisesc avere.
\par 17 Omul milostiv își face bine sufletului său, pe când cel fără milă își chinuiește trupul său.
\par 18 Cel nelegiuit capătă un câștig înșelător, iar cel ce seamănă dreptatea, o răsplată adevărată.
\par 19 Cel ce umblă după dreptate ajunge la viață, iar cel ce fuge după rău, la moarte.
\par 20 Pe cei cu inima vicleană îi urgisește Domnul; plăcerea Lui este spre cei fără prihană.
\par 21 Încetul cu încetul păcătosul nu va rămâne nepedepsit, iar neamul celor drepți va fi mântuit.
\par 22 Inel de aur în râtul porcului, așa este femeia frumoasă și fără minte.
\par 23 Dorința celor drepți este bine; nădejdea celor fără de lege este mânia lui Dumnezeu.
\par 24 Unul dă mereu și se îmbogățește, altul se zgârcește afară din cale și sărăcește.
\par 25 Cel ce binecuvintează va fi îndestulat, iar cel ce blesteamă va fi blestemat.
\par 26 Cel ce ține grâul este blestemat de popor, iar binecuvântarea (se revarsă) peste capul celui ce îl vinde.
\par 27 Cel ce caută binele dobândește bunăvoința Domnului, iar cel ce umblă după rău va da peste el.
\par 28 Cel ce-și pune nădejdea în bogăția lui se veștejește, iar cei drepți ca frunzișul odrăslesc.
\par 29 Cine își tulbură casa lui culege vânt, iar cel nebun va fi sluga celui înțelept.
\par 30 Rodul dreptății este un pom al vieții, iar silnicia nimicește viața.
\par 31 Dacă cel drept este răsplătit pe pământ, cu cât mai mult cel nelegiuit și păcătos!

\chapter{12}

\par 1 Cel ce iubește învățătura iubește știința, iar cel ce urăște certarea este nebun.
\par 2 Cel bun dobândește har de la Domnul, iar pe omul viclean îl osândește Domnul.
\par 3 Omul nu se întărește întru fărădelegea lui; rădăcina celor drepți nu se va clătina niciodată.
\par 4 Femeia virtuoasă este o cunună pentru bărbatul ei, iar femeia fără cinste este un cariu în oasele lui.
\par 5 Socotelile celor drepți sunt dreptatea, iar punerile la cale ale celor nelegiuiți înșelăciunea.
\par 6 Graiurile celor nelegiuiți sunt curse de moarte, iar gura celor drepți îi scapă pe ei din primejdie.
\par 7 Cei fără de lege numai cât se întorc și nu mai sunt, dar casa drepților dăinuiește de-a pururi.
\par 8 Omul aste prețuit după priceperea lui, iar cel nepriceput este urgisit.
\par 9 Mai mult prețuiește un om smerit dar harnic, decât unul mândru dar lipsit de pâine.
\par 10 Cel drept are milă de vite, iar inima celui rău este fără îndurare.
\par 11 Cel ce muncește ogorul său se satură de pâine, iar cel ce umblă după deșertăciuni este om lipsit de minte.
\par 12 Nelegiuitul poftește prada celor răi, dar rădăcina celor drepți dă rodul său.
\par 13 Prin păcatul buzelor se prinde în laț păcătosul, iar dreptul (prin dreptatea lui) scapă din strâmtorare.
\par 14 Din rodul gurii sale se satur; de cele bune omul, și fiecăruia i se răsplătește după faptele lui.
\par 15 Calea celui nebun este dreaptă în ochii lui, iar cel înțelept ascultă de sfat.
\par 16 Nebunul dă pe față îndată mânia lui, iar omul prevăzător își ascunde ocara.
\par 17 Cel ce spune adevărul vestește dreptatea, iar martorul mincinos umblă cu înșelăciunea.
\par 18 Cei nechibzuiți la vorbă sunt ca împunsăturile de sabie, pe când limba celor înțelepți aduce tămăduire.
\par 19 Buzele care spun adevărul vor dăinui totdeauna, iar limba grăitoare de minciună numai pentru o clipă.
\par 20 Înșelăciunea este în inima celor ce gândesc rău, iar bucuria pentru cei ce dau sfaturi de pace.
\par 21 Nici o nenorocire nu se întâmplă celui drept, pe când cei nelegiuiți sunt covârșiți de rele.
\par 22 Buzele cele grăitoare de minciună sunt urâciune înaintea Domnului, iar cei ce făptuiesc după adevăr sunt plăcerea Lui.
\par 23 Omul înțelept își ascunde știința, pe când inima celor nebuni propovăduiește nebunia.
\par 24 Mâna celor silitori va stăpâni, iar cea lăsătoare va fi birnică.
\par 25 Supărarea se abate asupra omului, dar numai un cuvânt bun îl bucură.
\par 26 Dreptul cercetează cu de-amănuntul pe prietenul său; calea celor nelegiuiți duce la rătăcire.
\par 27 Leneșul nu-și frige nici vânatul lui; cea mai scumpă comoară pentru om este munca.
\par 28 Pe cărarea dreptății este viața și pe calea pe care ea o însemnează; nemurirea, iar calea nebuniei duce la moarte.

\chapter{13}

\par 1 Fiul înțelept ascultă de învățătura tatălui său, iar cel batjocoritor nici de mustrare.
\par 2 Din rodul gurii sale omul mănâncă binele; pofta celor vicleni este silnicia.
\par 3 Cine își păzește gura își păzește sufletul său; cel ce deschide prea tare buzele o face spre pieirea lui.
\par 4 Sufletul celui leneș poftește, însă în zadar. Numai sufletul celor silitori este îndestulat.
\par 5 Dreptul urăște cuvintele mincinoase; ticălosul aduce numai rușine și ocară.
\par 6 Dreptatea păzește calea omului fără prihană, iar fărădelegea e pricina ruinii celui păcătos.
\par 7 Unii se dau drept bogați și n-au nimic, alții trec drept săraci, cu toate că au multe averi.
\par 8 Bogăția cuiva slujește la răscumpărarea lui; cel sărac nu se teme nici chiar de amenințare.
\par 9 Lumina celor drepți luminează, pe când sfeșnicul celor fără de lege se stinge.
\par 10 Mândria nu dă prilej decât la ceartă, înțelepciunea se află numai la cei ce primesc sfaturi.
\par 11 Bogăția adunată în grabă se împuținează, numai cel ce-o adună pe încetul o înmulțește.
\par 12 Așteptarea prea îndelungată îmbolnăvește inima, iar dorința împlinită este pom al vieții.
\par 13 Cel ce nu ia în seamă cuvântul (lui Dumnezeu) este dat pierzării, iar cel ce se teme de porunca Lui este răsplătit.
\par 14 Învățătura celui înțelept este izvor de viață, ca să putem scăpa de cursele morții.
\par 15 Buna înțelegere rodește har; calea celor vicleni este spre pieirea lor.
\par 16 Orice om înțelept lucrează cu chibzuință, numai cel nebun își desfășoară nebunia.
\par 17 Un sol ticălos cade în nenorocire, iar unul credincios aduce alinare.
\par 18 De sărăcie și de rușine are parte cel ce nesocotește certarea, iar cel ce primește mustrarea va fi cinstit.
\par 19 Dorința împlinită mulțumește sufletul, iar depărtarea de rău este urâciune pentru cei nebuni.
\par 20 Cel ce se însoțește cu cei înțelepți ajunge înțelept, iar cel ce se întovărășește cu cei nebuni se face rău.
\par 21 Nenorocirea urmărește pe cei păcătoși, iar fericirea răsplătește pe cei drepți.
\par 22 Omul bun lasă moștenirea sa nepoților săi, iar averea celui păcătos este sortită pentru cei drepți.
\par 23 Arătura în țelină, făcută de cei săraci, dă hrană din belșug, insă averea se pierde din pricina nedreptății.
\par 24 Cine cruță toiagul său își urăște copilul, iar cel care îl iubește îl ceartă la vreme.
\par 25 Dreptul mănâncă și își îndestulează sufletul său, iar pântecele celor fără de lege duce lipsă.

\chapter{14}

\par 1 Femeile înțelepte zidesc casa, iar cele nebune o dărâmă cu mâna lor.
\par 2 Cel ce umblă întru dreptate se teme de Domnul, iar cel ce umblă pe căi strâmbe îl disprețuiește.
\par 3 În gura celui nebun este varga mândriei lui; buzele pe cei înțelepți îi păzesc.
\par 4 Unde nu sunt boi, staulul este gol, însă puterea boilor aduce mult folos.
\par 5 Martorul care grăiește adevărul nu minte, iar martorul mincinos spune numai minciuni.
\par 6 Batjocoritorul caută înțelepciunea și nu o găsește, iar pentru cel priceput știința este ușoară.
\par 7 Fugi dinaintea omului fără de minte, căci tu știi că nu este nici o știință pe buzele lui.
\par 8 Înțelepciunea omului chibzuit este de a-și înțelege calea lui; iar nebunia celor neînțelepți este înșelăciune.
\par 9 Nebunul își bate joc de jertfa pentru păcat, însă între oamenii drepți este bună înțelegere.
\par 10 Inima cunoaște amărăciunile sale, iar un străin nu poate împărți bucuriile ei.
\par 11 Casa celor fără de lege va fi distrusă, iar cortul celor drepți va înflori.
\par 12 Unele căi par drepte în ochii omului, dar sfârșitul lor sunt căile morții.
\par 13 Chiar când râdem, inima se întristează; bucuria se sfârșește prin plângere.
\par 14 Nelegiuitul se va sătura de căile sale și omul bun de roadele sale.
\par 15 Omul simplu crede toate vorbele; omul înțelept veghează pașii săi.
\par 16 Înțeleptul se teme și se ferește de rău, iar cel fără de minte își iese din fire și se simte la adăpost.
\par 17 Cel iute la mânie săvârșește nebunii, iar cel cumpănit se stăpânește.
\par 18 Cei nepricepuți au parte de nebunie, pe când cei înțelepți sunt încununați cu știință.
\par 19 Cei răi se pleacă înaintea celor buni și cei nelegiuiți stau la porțile celor drepți.
\par 20 Săracul este disprețuit chiar și de prietenul lui, pe când prietenii celui bogat sunt fără de număr.
\par 21 Cel care nu bagă în seamă pe prietenul său săvârșește un păcat; iar cel ce se îndură de sărmani e fericit.
\par 22 Cu adevărat rătăcesc cei ce plănuiesc fărădelegea, iar cei ce cugetă la lucruri bune au parte de milostivire și de adevăr.
\par 23 Orice osteneală duce la îndestulare, iar cuvintele fără rost la lipsă.
\par 24 Bogăția este o cunună pentru cei înțelepți; iar coroana color nebuni este nebunia.
\par 25 Martorul drept scapă suflete, iar cel viclean spune numai minciuni.
\par 26 Întru frica lui Dumnezeu este nădejdea celui tare; fiii lui afla-vor (acolo) un liman.
\par 27 Frica de Dumnezeu este un izvor de viață, ca să putem scăpa de cursele morții.
\par 28 Strălucirea unui rege se sprijină pe mulțimea poporului, iar lipsa de supuși este pieirea prințului.
\par 29 Cel încet la mânie este bogat în înțelepciune, iar cel ce se mânie degrabă își dă pe față nebunia.
\par 30 O inimă fără patimă este viața trupului, pe când pornirea pătimașă este ca un cariu în oase.
\par 31 Cel care apasă pe cel sărman defaimă pe Ziditorul lui, dar cel ce are milă de sărac Îl cinstește.
\par 32 Cel fără de lege este răsturnat de răutatea lui, iar cel drept găsește scăpare în neprihănirea lui.
\par 33 Înțelepciunea sălășluiește în inima celui înțelept, iar în inima celor nebuni nu se arată.
\par 34 Dreptatea înalță un popor, în vreme ce păcatul este ocara popoarelor.
\par 35 Bunăvoința regelui este pentru sluga înțeleaptă, iar mânia lui pentru cel ce îi face rușine.

\chapter{15}

\par 1 Un răspuns blând domolește mânia, iar un cuvânt aspru ațâță mânia.
\par 2 Limba celor înțelepți picură știință, țar gura celor nebuni revarsă prostie.
\par 3 Ochii Domnului sunt pretutindeni, veghind asupra celor buni și asupra celor răi.
\par 4 Limba dulce este pom al vieții, iar limba vicleană zdrobește inima.
\par 5 Nebunul nu ia în seamă învățătura tatălui său, iar cine trage folos din certare se face mai înțelept.
\par 6 În casa celui drept sunt comori fără de număr; în câștigul celui fără de lege este tulburare.
\par 7 Buzele celor înțelepți răspândesc știința, dar inima celor nebuni nu.
\par 8 Jertfa celor fără de lege este urâciune înaintea Domnului, iar rugăciunea celor drepți este plăcerea Lui.
\par 9 Calea celui nelegiuit este urâciune înaintea Domnului, dar El iubește pe cel ce umblă după dreptate.
\par 10 O certare aspră capătă ,el ce părăsește cărarea; cel ce urgisește mustrarea va muri.
\par 11 Iadul și adâncul sunt cunoscute Domnului, cu atât mai vârtos inimile fiilor oamenilor.
\par 12 Celui batjocoritor nu-i place dojana; de aceea el nu se îndreaptă spre cei înțelepți.
\par 13 O inimă veselă înseninează fața, iar când inima e tristă și duhul e fără de curaj.
\par 14 Inima înțeleaptă caută știința, iar gura celor nebuni se simte mulțumită cu nebunia.
\par 15 Toate zilele celui sărac sunt rele, dar inima mulțumită este un ospăț necurmat.
\par 16 Mai bine puțin întru frica lui Dumnezeu, decât vistierie mare și tulburare (multă).
\par 17 Mai mult face o mâncare de verdețuri și cu dragoste, decât un bou îngrășat și cu ură.
\par 18 Omul mânios ațâță cearta, pe când cel domol liniștește aprinderea.
\par 19 Calea celui leneș e ca un gard de spini, iar calea celui silitor e netedă.
\par 20 Fiul înțelept bucură pe tatăl său, iar fiul nebun nu bagă în seamă pe maica lui.
\par 21 Nebunia este o bucurie pentru omul fără minte, iar cel înțelept merge pe calea dreaptă.
\par 22 Punerile la cale nu se înfăptuiesc unde lipsește chibzuirea, dar ele își iau ființă cu mulți sfătuitori.
\par 23 Omul se bucură pentru un răspuns bun ieșit din gura lui și cât e de bună vorba spusă la locul ei!
\par 24 Înțeleptul merge pe cărarea vieții ce duce în sus, ca să ocolească drumul iadului care merge în jos.
\par 25 Domnul prăbușește casa celor mândri și întărește hotarul văduvei.
\par 26 Gândurile cele rele sunt urâciune înaintea Domnului, iar cuvintele frumoase sunt curate (în ochii Lui).
\par 27 Cel ce umblă după câștig nedrept își surpă casa lui, iar cel ce urăște mita va trăi.
\par 28 Inima celui drept chibzuiește ce să răspundă, iar gura celor nelegiuiți împrăștie răutăți.
\par 29 Domnul se ține departe de cei nelegiuiți, dar ascultă rugăciunea celor drepți.
\par 30 O privire binevoitoare înveselește inima și o veste bună întărește oasele.
\par 31 Urechea care ascultă o dojană folositoare vieții își are locașul printre cei înțelepți.
\par 32 Cel ce leapădă mustrarea își urgisește sufletul său, iar cel ce ia aminte la dojană dobândește înțelepciune.
\par 33 Frica de Dumnezeu este învățătură și înțelepciune, iar smerenia trece înaintea măririi.

\chapter{16}

\par 1 În putere stă omului să plăsmuiască planuri în inimă, dar răspunsul limbii vine de la Domnul.
\par 2 Toate căile omului sunt curate în ochii lui, dar numai Domnul este Cel ce cercetează duhul.
\par 3 Înfățișează Domnului lucrările tale și gândurile tale vor izbuti.
\par 4 Pe toate le-a făcut Domnul fiecare cu țelul său, la fel și pe nelegiuit pentru ziua nenorocirii.
\par 5 Toată inima semeață este urâciune înaintea Domnului; hotărât, ea nu va rămâne nepedepsită.
\par 6 Prin iubire și credință se ispășește păcatul și prin frica de Dumnezeu te ferești de rele.
\par 7 Când căile omului sunt plăcute înaintea Domnului, chiar și pe vrăjmașii lui îi silește la pace.
\par 8 Mai degrabă puțin și cu dreptate, decât agoniseală multă cu strâmbătate.
\par 9 Inima omului gândește la calea lui, dar numai Domnul poartă pașii lui.
\par 10 Hotărâri dumnezeiești sunt pe buzele împăratului; la darea hotărârii să nu se înșele gura lui!
\par 11 Cântarul și cumpăna dreaptă sunt de la Domnul; toate greutățile de cântărit sunt lucrarea Lui.
\par 12 Urâciune sunt regii care făptuiesc fărădelegea, fiindcă numai întru dreptate se întărește tronul.
\par 13 Buzele grăitoare de dreptate sunt plăcute regilor și iubesc pe cel ce spune drept.
\par 14 Întărâtarea regelui este ca vestitorii morții, dar omul înțelept o domolește.
\par 15 Seninătatea feței regelui dă viață și bunăvoința lui este ca un nor de ploaie de primăvară.
\par 16 Dobândirea înțelepciunii este mai bună decât aurul, iar câștigarea priceperii este mai de preț decât argintul.
\par 17 Calea celor drepți este ferirea de rău; numai acela care ia aminte la mersul lui își păzește sufletul său.
\par 18 Înaintea prăbușirii merge trufia și semeția înaintea căderii.
\par 19 Mai bine să fii smerit cu cei smeriți decât să împarți prada cu cei mândri.
\par 20 Cel care ia aminte la cuvânt află fericirea, iar cel ce se încrede în Domnul este fericit.
\par 21 Cel ce este înțelept se cheamă priceput; dulceața cuvintelor de pe buzele lui înmulțește știința.
\par 22 Înțelepciunea este un izvor de viață pentru cine o are; pedeapsa celui nebun e nebunia.
\par 23 Inima celui înțelept dă înțelepciune gurii lui și pe buzele sale sporește știința.
\par 24 Cuvintele frumoase sunt un fagure de miere, dulceață pentru suflet și tămăduire pentru oase.
\par 25 Multe căi i se par bune omului, dar la capătul lor încep căile morții.
\par 26 Foamea îndeamnă pe lucrător la muncă, fiindcă gura lui îl silește.
\par 27 Omul viclean pregătește nenorocirea și pe buzele lui este ca un foc arzător.
\par 28 Omul cu gând rău ațâță ceartă și defăimătorul desparte pț prieteni.
\par 29 Omul asupritor amăgește pe prietenul său și îl îndreaptă pe un drum care nu este bun.
\par 30 Cel care închide din ochi urzește viclenii; cine își mușcă buzele a și săvârșit răul.
\par 31 Bătrânețea este o cunună strălucită; ea se află mergând pe calea cuvioșiei.
\par 32 Cel încet la mânie e mai de preț decât un viteaz, iar cel ce își stăpânește duhul este mai grețuit decât cuceritorul unei cetăți.
\par 33 Se aruncă sorții în pulpana hainei, dar hotărârea toată vine de la Domnul.

\chapter{17}

\par 1 Mai bună este o bucată de pâine uscată în pace, decât o casă plină cu carne de jertfe, dar cu vrajbă.
\par 2 Un slujitor înțelept e mai presus decât un fiu aducător de rușine; acela va împărți moștenirea eu frații.
\par 3 în topitoare se lămurește argintul și în cuptor aurul, dar cel ce încearcă inimile este Domnul.
\par 4 Făcătorul de rele ia aminte la buzele nedrepte, mincinosul pleacă urechea la limba cea rea.
\par 5 Cel ce își bate joc de sărac defaimă pe Ziditorul lui; cel ce se bucură de o nenorocire nu va rămâne nepedepsit.
\par 6 Cununa bătrânilor sunt nepoții, iar mărirea fiilor sunt părinții lor.
\par 7 Nebunului nu-i sunt dragi cuvintele alese, cu atât mai mult unui om de seamă vorbele mincinoase.
\par 8 Piatră nestemată este darul în ochii celui ce-l are; oriunde se întoarce (totul) îi merge în plin.
\par 9 Cel ce acoperă un păcat caută prietenie, iar cine scoate la iveală un lucru (uitat) desparte pe prieteni.
\par 10 Certarea înrâurește mai adânc pe omul înțelept, decât o sută de lovituri pe cel nebun.
\par 11 Omul rău ațâță răzvrătirea; pentru aceasta un sol aprig va fi trimis împotriva lui.
\par 12 Mai degrabă să întâlnești o ursoaică lipsită de puii ei, decât un nebun în nebunia lui.
\par 13 Cel ce răsplătește cu rău pentru bine nu va vedea depărtându-se nenorocirea din casa lui.
\par 14 Începutul unei certe e ca slobozirea apei (dintr-un iezer); înainte de a se aprinde, dă-te la o parte!
\par 15 Cel care achită pe cel vinovat și cel ce osândește pe cel drept, amândoi sunt urâciune înaintea Domnului.
\par 16 Ce folos aduc banii în mâna celui nebun? Ar putea dobândi înțelepciune, dar nu are pricepere.
\par 17 Prietenul iubește în orice vreme, iar în nenorocire el e ca un frate.
\par 18 Omul fără pricepere se prinde prin dărnicia mâinii lui; el se pune chezaș pentru aproapele lui.
\par 19 Cine iubește certurile iubește păcatul; cel ce ridică glasul își iubește ruina.
\par 20 Cel ce are o inimă vicleană nu află fericirea și cel ce are o limbă șireată dă peste necaz.
\par 21 Cel ce dă naștere unui nebun va avea o mare supărare și părintele nu are nici o bucurie pentru un fiu sărit din minte.
\par 22 O inimă veselă este un leac minunat, pe când un duh fără curaj usucă oasele.
\par 23 Nelegiuitul primește daruri scoase (pe ascuns) din sân, ca să abată cărările dreptății.
\par 24 Omul priceput are înaintea ochilor lui înțelepciunea, iar ochii celui nebun se uită la capătul pământului.
\par 25 Feciorul nebun este necaz pentru tatăl lui și amărăciune pentru maica lui.
\par 26 Nu se cuvine să pui la plată pe omul drept, nici să osândești pe cei nevinovați din pricină că umblă drept.
\par 27 Cel ce își stăpânește cuvintele sale are știință și cel ce își ține cumpătul este un om priceput.
\par 28 Chiar și nebunul, când tace, trece drept înțelept; când închide gura este asemenea unui om cuminte.

\chapter{18}

\par 1 Cel ce se rine deoparte caută să-și mulțumească pornirea pătimașă; împotriva oricărui sfat înțelept se pornește.
\par 2 Celui nebun nu-i place înțelepciunea, ci darea pe față a gândurilor lui.
\par 3 Când vine cel nelegiuit, vine și defăimarea și o dată cu rușinea și batjocura.
\par 4 Vorbele (ieșite) din gura omului sunt ape fără fund; izvorul înțelepciunii este un șuvoi care dă peste maluri.
\par 5 Nu este bine să cauți la fața celui fără de lege și să nu faci dreptate celui drept la judecată.
\par 6 Buzele celui nebun duc la ceartă și gura lui dă naștere la ocări!
\par 7 Gura celui nebun este prăbușirea lui și buzele lui sunt un laț pentru sufletul lui.
\par 8 Vorbele defăimătorului sunt ca niște mâncări alese; ele coboară în cămările pântecelui.
\par 9 Omul lăsător pentru lucrul lui e frate cu cel care dărâmă.
\par 10 Turn puternic este numele Domnului; cel drept la el își află scăparea și este la adăpost.
\par 11 Averea celui bogat este o cetate tare pentru el, iar în închipuirea lui ca un zid înalt.
\par 12 Înaintea prăbușirii vine trufia inimii, iar înaintea măririi merge umilința.
\par 13 Cel ce răspunde la vorbă înainte de a fi ascultat-o este nebun și încurcat la minte.
\par 14 Curajul omului îl întărește în vreme de suferință, iar pe un om lipsit de bărbăție, cine-l va ridica?
\par 15 O inimă pricepută dobândește știință și urechea celor înțelepți umblă după iscusință.
\par 16 Darul adus de un om îi lărgește (calea lui) și-l poartă înaintea celor mari.
\par 17 Pârâtul pare că are dreptate în pricina sa, iar când vine pârâșul, atunci se ia la cercetare.
\par 18 Sorțul face să înceteze sfada și hotărăște între cei puternici.
\par 19 Frate ajutat de frate este ca o cetate tare și înaltă și are putere ca o împărăție întemeiată.
\par 20 Din rodul gurii omului se satură pântecele lui; din ceea ce dă buzele lui se îndestulează.
\par 21 În puterea limbii este viața și moartea și cei ce o iubesc mănâncă din rodul ei.
\par 22 Cel ce găsește o femeie bună află un lucru de mare preț și dobândește dar de la Dumnezeu.
\par 23 Săracul vorbește rugător, iar cel bogat răspunde cu îndrăzneală.
\par 24 Sunt prieteni aducători de nenorocire; dar este și câte un prieten mai apropiat decât un frate.

\chapter{19}

\par 1 Mai de preț este săracul care umblă întru neprihănirea lui, decât un bogat cu buze viclene și nebun.
\par 2 Neștiința sufletului nu este bună iar cel ce umblă repede dă greș.
\par 3 Nebunia omului dărâmă calea lui și inima lui se mânie împotriva Domnului.
\par 4 Bogăția strânge prieteni fără de număr, iar săracul se desparte chiar de prietenul său.
\par 5 Martorul mincinos nu va rămâne nepedepsit și cel ce spune lucruri neadevărate nu va scăpa.
\par 6 Mulți sunt cei ce lingușesc pe un om darnic și toți sunt prieteni ai celui ce dă daruri.
\par 7 Toți frații celui sărac îl urăsc; cât de mult prietenii lui se depărtează de el! El caută vorbe (mângâietoare), dar nu le află.
\par 8 Cel ce dobândește înțelepciune iubește sufletul său și cel ce ține cu tărie la pricepere află fericirea.
\par 9 Martorul mincinos nu rămâne nepedepsit și cel ce spune lucruri neadevărate se va prăbuși.
\par 10 Nu-i stă bine celui fără de minte să trăiască în desfătări, cu atât mai puțin unui rob să conducă peste căpetenii.
\par 11 Înțelepciunea domolește mânia omului și faima lui este iertarea greșelilor.
\par 12 Furia unui rege e ca răcnetul unui leu, iar bunăvoința lui este ca roua pe iarbă.
\par 13 Un fiu neascultător este nenorocirea tatălui lui, iar certurile unei femei un jgheab care curge întruna.
\par 14 O casă și o avere sunt moștenire de la părinți, iar o femeie înțeleaptă este un dar de la Dumnezeu.
\par 15 Lenea te face să cazi în toropeală; sufletului trândav îi va fi foame.
\par 16 Cel ce ia seama la poruncă își păstrează sufletul său, iar cel ce disprețuiește cuvântul (Domnului) va muri.
\par 17 Cel ce are milă de sărman împrumută Domnului și El îi va răsplăti fapta lui cea bună.
\par 18 Pedepsește pe feciorul tău, cât mai este nădejde (de îndreptare), dar nu ajunge până acolo ca să-l omori.
\par 19 Omul aprig la mânie trebuie să fie pedepsit; dacă îl cruți o dată, trebuie să începi din nou.
\par 20 Ascultă sfatul și primește învățătura, ca să fii înțelept toată viața ta.
\par 21 Multe puneri la cale frământă inima omului, dar numai sfatul Domnului se împlinește.
\par 22 Omul se face plăcut prin mărinimia lui; mai de preț este un sărac bun decât un om mincinos.
\par 23 Frica de Dumnezeu duce la viață și ne îndestulăm fără să fim loviți de nenorocire.
\par 24 Leneșul întinde mâna în blid și nu are putere s-o ducă la gură.
\par 25 Lovește pe cel ce batjocorește și cel fără de minte va deveni înțelept; mustră pe cel înțelept și el va pricepe știința.
\par 26 Cel ce se poartă rău cu tatăl său și alungă (din casă) pe mama sa, este fiu aducător de ocară și de rușine.
\par 27 Încetează, fiul meu, să asculți ademenirea și să te lași îndepărtat de învățăturile înțelepte.
\par 28 Martorul de nimic își bate joc de dreptate și gura celor fără de lege înghite nelegiuirea.
\par 29 Pentru batjocoritori sunt gata toiege; loviturile sunt pentru spinarea celor nebuni.

\chapter{20}

\par 1 Un ocărâtor este vinul, un zurbagiu băutura îmbătătoare și oricine se lasă ademenit nu este înțelept.
\par 2 Groaza pe care o insuflă regele este ca răcnetul leului; cel ce îl întărâtă păcătuiește împotriva sa însuși.
\par 3 Este o mare însușire pentru om să se stăpânească de la ceartă și tot nebunul se întărâtă.
\par 4 Toamna leneșul nu lucrează, iar când vine să culeagă rodul la seceriș, nimic nu află.
\par 5 O apă adâncă este sfatul în inima omului, iar omul deștept știe s-o scoată.
\par 6 Mulți oameni se laudă cu mărinimia, dar un prieten adevărat cine-l află?
\par 7 Omul drept umblă pe calea lui fără prihană; fericiți sunt copiii care vin după el!
\par 8 Un rege care stă pe scaunul de judecată deosebește cu ochii lui orice faptă rea.
\par 9 Cine poate spune: Curățit-am inima mea; sunt curat de păcat?
\par 10 Două feluri de greutăți de cântărit și de măsurat sunt urâciune înaintea Domnului.
\par 11 Copilul se dă pe față din lucrările lui, dacă purtarea lui este fără prihană și dreaptă.
\par 12 Urechea care aude și ochiul care vede, pe amândouă le-a zidit Domnul.
\par 13 Nu iubi somnul, ca să nu ajungi sărac; ține ochii deschiși, numai așa vei fi îndestulat de pâine.
\par 14 "Rău, rău!", zice cumpărătorul, iar după ce pleacă se laudă.
\par 15 Chiar dacă ai aur și pietre prețioase, dar o podoabă fără seamăn sunt buzele chibzuite.
\par 16 Ia-i haina și fiindcă s-a pus chezaș pentru altul în locul celor străini, ia-l zălog.
\par 17 Bună e la gust pâinea agonisită cu înșelăciune, dar după aceea gura se umple de pietricele.
\par 18 Planurile se întăresc prin sfaturi; luptă-te cu luare aminte.
\par 19 Cine trădează taina umblă ca un defăimător și nu te întovărăși cu cel ce are mereu buzele deschise.
\par 20 Cel ce blesteamă pe tatăl său și pe mama sa stinge sfeșnicul în mijlocul întunericului.
\par 21 O moștenire repede câștigată de la început, la urmă va fi fără binecuvântare.
\par 22 Nu spune: "Vreau să răsplătesc cu rău!" Nădăjduiește în Domnul și El iți va veni în ajutor.
\par 23 Greutățile nedrepte pentru cântărire sunt urâciune înaintea Domnului și cântarele înșelătoare nu sunt decât un lucru rău.
\par 24 De Domnul sunt hotărâți pașii omului, căci cum ar putea omul să priceapă calea lui?
\par 25 O cursă este pentru om să afierosească Domnului ceva în grabă și după ce a făgăduit să-i pară rău.
\par 26 Un rege înțelept simte pe cei fără de lege și lasă să treacă roata peste ei.
\par 27 Sufletul omului este un sfeșnic de la Domnul; el cercetează toate cămările trupului.
\par 28 Iubirea și credința păzesc pe rege și prin iubire își sprijină tronul său.
\par 29 Faima celor tineri este puterea lor și podoaba celor bătrâni părul lor cărunt.
\par 30 Rănile sângeroase sunt un leac pentru cel răufăcător și lovituri care pătrund până înlăuntrul trupului.

\chapter{21}

\par 1 Asemenea unui curs de apă este inima regelui în mâna Domnului, pe care îl îndreaptă încotro vrea.
\par 2 Toată calea omului este dreaptă în ochii lui, dar numai Domnul cântărește inimile.
\par 3 Făptuirea dreptății și a judecății este mai de preț pentru Domnul decât jertfa sângeroasă.
\par 4 Ochii semeți și inima îngâmfată sunt sfeșnicul păcătoșilor. Aceasta nu este decât păcat.
\par 5 Chibzuiala omului silitor duce numai la câștig, iar cel ce se zorește ajunge la pagubă.
\par 6 Comorile dobândite cu limbă mincinoasă sunt deșertăciune trecătoare și lațuri ale morții.
\par 7 Silnicia celor fără de lege se ține după ei, căci nu voiesc să înfăptuiască dreptatea.
\par 8 Calea celui răufăcător e sucită; cel nevinovat lucrează drept.
\par 9 Mai degrabă să locuiești într-un colț pe acoperiș, decât cu o femeie certăreață și într-o casă mare.
\par 10 Sufletul celui fără de lege poftește răutatea, iar aproapele lui nu află milă în ochii lui.
\par 11 Când cel batjocoritor e pedepsit, cel fără minte se înțelepțește și când cel înțelept este dojenit, el câștigă în știință.
\par 12 Dreptul ia aminte la casa celui nelegiuit. Dumnezeu prăbușește pe cei fără de lege în nenorocire.
\par 13 Cine își astupă urechea la strigătul celui sărman și el, când va striga, nu i se va răspunde.
\par 14 Un dar făcut într-ascuns potolește mânia și un plocon scos din sân, o mânie puternică.
\par 15 Dreptul tresaltă de bucurie când poate să pună în faptă dreptatea, iar spaima e pentru cei ce săvârșesc fărădelegea.
\par 16 Un om care rătăcește de pe drumul înțelepciunii, se va odihni curând în adunarea celor morți.
\par 17 Cel ce iubește veselia va duce lipsă; cel căruia îi place vinul și miresmele nu se îmbogățește.
\par 18 Nelegiuitul slujește ca preț de răscumpărare pentru cel drept și vicleanul pentru cel fără prihană.
\par 19 Mai bine să locuiești în pustiu decât cu o femeie certăreață și supărăcioasă.
\par 20 Comori de preț și untdelemn (se găsesc) în casa celui înțelept, dar omul cel nebun le risipește.
\par 21 Cel ce umblă în calea dreptății și a milei află viață, dreptate și mărire.
\par 22 Înțeleptul ia cu luptă cruntă cetatea vitejilor și răstoarnă întăriturile în care ei își puneau nădejdea.
\par 23 Cel ce-și păzește gura și limba lui își păzește sufletul lui de primejdie.
\par 24 Cine este semeț și îngâmfat se cheamă batjocoritor; acela se poartă cu prisos de trufie.
\par 25 Pofta celui leneș îl omoară, căci mâinile nu voiesc să lucreze.
\par 26 Mereu cel fără de lege poftește, iar cel drept dă și nu se zgârcește.
\par 27 Jertfa celor nelegiuiți este urâciune pentru Domnul, mai cu seamă când o aduc pentru o faptă rușinoasă.
\par 28 Martorul mincinos va pieri, iar omul care ascultă va putea vorbi totdeauna.
\par 29 Răufăcătorul are o privire nerușinată, iar omul cel drept își ia aminte la purtarea lui.
\par 30 Nu este nici înțelepciune, nici pricepere și nici sfat care să aibă putere înaintea Domnului.
\par 31 Calul este gata pentru ziua de război, însă biruința vine de la Domnul.

\chapter{22}

\par 1 Un nume (bun) este mai de preț decât bogăția; cinstea este mai prețioasă decât argintul și decât aurul.
\par 2 Bogatul și săracul se întâlnesc unul cu altul; dar Cine i-a făcut este Domnul.
\par 3 Cel iscusit vede nenorocirea și se ascunde, cei simpli trec mai departe și suferă.
\par 4 Rodul umilinței și a temerii de Dumnezeu sunt: bogăția, mărirea și viața.
\par 5 Mărăcini și curse sunt în calea celui viclean; cel ce își ferește sufletul lui sa dă la o parte de ele.
\par 6 Deprinde pe tânăr cu purtarea pe care trebuie s-o aibă; chiar când va îmbătrâni nu se va abate de la ea.
\par 7 Bogatul stăpânește pe cei săraci, și cel ce împrumută este slujitor celui de la care se împrumută.
\par 8 Cel ce seamănă nedreptatea seceră nenorocire, iar toiagul mâniei lui îl va bate pe el.
\par 9 Omul blând va fi binecuvântat, căci din pâinea ui dă celui sărac.
\par 10 Alungă pe cel batjocoritor și cearta va lua sfârșit și pricina și defăimarea vor înceta.
\par 11 Cel ce iubește curăția inimii și ale cărui buze sunt ( pline) de vorbe alese are de prieten pe conducător.
\par 12 Ochii Domnului păzesc știința și dărâmă cuvintele celui fără de lege.
\par 13 Cel leneș pune pricini și zice: "Afară este un leu, aș putea să fiu sugrumat în mijlocul ulițelor".
\par 14 O groapă fără fund este gura femeilor străine; cel ce este lovit de mânia Domnului cade în ea.
\par 15 Dacă nebunia se pripășește în inima celui tânăr, numai varga certării o va îndepărta de el.
\par 16 Dacă împilezi pe un sărac, îți înmulțești averea, dacă dai unui bogat sărăcești.
\par 17 Pleacă urechea ta și ascultă cuvintele celor iscusiți și inima ta îndreapt-o spre știința mea.
\par 18 Este plăcut dacă tu le păstrezi înlăuntrul tău. O, de-ar sta toate pe buzele tale!
\par 19 Pentru a-ți pune nădejdea în Domnul, vreau să-ți dau învățătură astăzi.
\par 20 Oare nu ți-am așezat în scris în nenumărate rânduri sfaturi și învățături,
\par 21 Ca să-ți fac cunoscut credincioșia cuvintelor adevărate și să răspunzi prin cuvinte de bună credință, celor ce te întreabă?
\par 22 Nu jefui pe sărac, pentru că el e sărac și nu asupri pe cel nenorocit la poarta (cetății),
\par 23 Căci Domnul va apăra pricina lor și va ridica viața celor care îi vor fi jefuit.
\par 24 Nu te întovărăși cu omul mânios și cu cel înfierbântat de furie să n-ai nici un amestec,
\par 25 Ca să nu te deprinzi pe calea lui și să-ți întinzi o cursă pentru viața ta.
\par 26 Nu fi dintre aceia care dau mâna, care se pun chezași pentru datorii.
\par 27 Dacă nu ai cu ce plăti, pentru ce te învoiești ca să ți se ia și patul de sub tine?
\par 28 Nu muta hotarul străvechi pe care l-au însemnat părinții tăi.
\par 29 Vezi tu un om dibaci la lucrul lui? El va sta înaintea conducătorilor și nu înaintea oamenilor de rând.

\chapter{23}

\par 1 Când stai la masă cu un dregător, ia seama pe cine ai înaintea ta;
\par 2 Pune-ți un cuțit la gât, dacă tu ești lacom.
\par 3 Nu pofti bucatele lui, căci sunt mâncări înșelătoare.
\par 4 Nu te osteni să ajungi bogat; nu-ți pune iscusința ta în aceasta.
\par 5 Oare vrei să te uiți cu ochii cum ea se risipește? Căci bogăția face aripi ca un vultur care se înalță către cer.
\par 6 Nu mânca pâinea celui ce se uită cu ochi răi și nu pofti bucatele lui,
\par 7 Căci el îți numără bucățile din gură. "Mănâncă și bea!", îți va spune, dar inima lui nu e pentru tine.
\par 8 Bucata pe care ai mâncat-o o vei da afară din tine, iar tu ți-ai risipit (zadarnic) vorbele tale alese.
\par 9 Nu grăi la urechea celui nebun, căci el nu va băga de seamă iscusința graiurilor tale.
\par 10 Nu muta hotarul văduvei și nu încălca ogorul celor orfani,
\par 11 Căci Ocrotitorul lor e tare și El va apăra pricina lor împotriva ta.
\par 12 Silește la învățătură inima ta și urechea ta la cuvinte iscusite.
\par 13 Nu cruța pe feciorul tău de pedeapsă; chiar dacă îl lovești cu varga, nu moare.
\par 14 Tu îl bați cu toiagul, dar scapi sufletul lui din împărăția morții.
\par 15 Fiul meu, dacă inima ta e plină de înțelepciune și inima mea se va bucura.
\par 16 Rărunchii mei vor tresări de bucurie, când buzele tale vor grăi ceea ce este drept.
\par 17 Să nu râvnească inima ta la cei păcătoși, ci totdeauna să rămână la frica de Domnul,
\par 18 Căci dacă o vei păzi pe ea, mai ai și tu un viitor și nădejdea ta nu se va pierde.
\par 19 Ascultă fiul meu, și te înțelepțește și îndreaptă inima ta pe calea cea dreaptă.
\par 20 Nu fi printre cei ce se îmbată de vin și printre cei ce își desfrânează trupul lor,
\par 21 Căci bețivul și desfrânatul sărăcesc, iar dormitul mereu te face să porți zdrențe.
\par 22 Ascultă pe tatăl tău care te-a născut și nu disprețui pe mama ta când ea a ajuns bătrână.
\par 23 Adună adevăr și nu-l vinde, înțelepciune și învățătură și bună chibzuială.
\par 24 Tatăl celui drept tresaltă de bucurie și cel ce a dat naștere unui înțelept se bucură de el.
\par 25 Să se bucure tatăl și mama ta și să salte de veselie cea care te-a născut!
\par 26 Dă-mi, fiule, mie inima ta, și ochii tăi să simtă plăcere pentru căile mele,
\par 27 Căci femeia desfrânată este o groapă adâncă și cea străină un puț strâmt.
\par 28 Pentru aceasta ea stă ca un hoț la pândă și sporește printre oameni numărul celor înșelați de ea.
\par 29 Pentru cine sunt suspinele, pentru cine văicărelile, pentru cine gâlcevile, pentru cine plânsetele, pentru cine rănile fără pricină, pentru cine ochii întristați?
\par 30 Pentru cei ce zăbovesc pe lângă vin, pentru cei ce vin să guste băuturi cu mirodenii.
\par 31 Nu te uita la vin cum este el de roșu, cum scânteiază în cupă și cum alunecă pe gât,
\par 32 Căci la urmă el ca un șarpe mușcă și ca o viperă împroașcă venin.
\par 33 Dacă ochii tăi vor privi la femei străine și gura ta va grăi lucruri meșteșugite,
\par 34 Vei fi ca unul care stă culcat în mijlocul mării, ca unul care a adormit pe vârful unui catarg.
\par 35 "M-au lovit... Nu m-a durut! M-au bătut... Nu știu nimic! Când mă voi deștepta din somn, voi cere iarăși vin".

\chapter{24}

\par 1 Nu râvni la oamenii răi și nu pofti să fii în tovărășia lor,
\par 2 Căci inima lor pune la cale lucruri silnice și buzele lor grăiesc cele nelegiuite.
\par 3 Prin înțelepciune se ridică o casă, prin bună chibzuială se întărește
\par 4 Și prin știință se umplu cămările ei de tot felul de avuție scumpă și plăcută.
\par 5 Mai puternic este un înțelept decât un voinic și cel priceput decât unul plin de putere.
\par 6 Cu oricâtă dibăcie te vei război, biruința se dobândește cu mulți sfătuitori.
\par 7 Peste măsură de înaltă este înțelepciunea pentru omul nebun; când stă la poarta (cetății) el nu deschide gura.
\par 8 Cel ce-și pune în gând să facă rău se cheamă un mare răufăcător.
\par 9 Gândul celui nebun nu este decât păcat; batjocoritorul este urgia oamenilor.
\par 10 Dacă te arăți slab în ziua strâmtorării, puterea ta nu este decât slăbiciune.
\par 11 Izbăvește pe cei ce sunt târâți la moarte și pe cei ce se duc clătinându-se la junghiere scapă-i!
\par 12 Dacă vrei să spui: "Iată n-am știut nimic!", oare Cel ce cântărește inimile nu pătrunde cu privirea și Cel ce veghează peste sufletul tău nu știe și nu va răsplăti omului după faptele lui?
\par 13 Fiul meu, mănâncă miere, căci e bună și un fagure de miere este dulce gurii tale.
\par 14 Să știi că înțelepciunea este la fel pentru sufletul tău; dacă o dobândești, ai un viitor, iar nădejdea ta nu este pierdută.
\par 15 Nu pândi, nelegiuitule, casa celui drept și nu tulbura locașul lui,
\par 16 Căci dacă cel drept cade de șapte ori și tot se scoală, cei fără de lege se poticnesc în nenorocire.
\par 17 Nu te bucura când cade vrăjmașul tău și, când se poticnește, să nu se veselească inima ta,
\par 18 Ca nu cumva să vadă Domnul și să fie neplăcut în ochii Lui și să nu întoarcă mânia Sa de la el (spre tine).
\par 19 Nu te aprinde împotriva răufăcătorului și nu-ți întărâta râvna împotriva celor fără de lege.
\par 20 Căci cel ce face rău nu propășește, și sfeșnicul celor nelegiuiți se va stinge.
\par 21 Fiul meu, teme-te de Domnul și de rege și cu cei ce se răzvrătesc nu lega prietenie,
\par 22 Că fără de veste va veni nenorocirea și cine poate să cunoască sfârșitul lor năprasnic?
\par 23 Și aceste (proverbe) sunt ale înțelepților: Nu e bine ca la judecată să cauți la fața oamenilor.
\par 24 Pe cel ce zice celui fără de lege: "Tu ești drept!", popoarele îl blesteamă și neamurile îl afurisesc;
\par 25 Dar celor care îl ceartă Cum se cuvine le merge bine și peste ei vine binecuvântarea și fericirea.
\par 26 Buzele sărută pe cei ce dau răspunsuri drepte.
\par 27 Rânduiește-ți lucrul tău afară și adu-l la îndeplinire pe câmpul tău, apoi îți vei ridica o casă.
\par 28 Nu fi martor mincinos împotriva prietenului tău și nu fi pricina (unei hotărâri nedrepte), cu buzele tale.
\par 29 Nu spune: "Precum mi-a făcut așa îi voi face și eu lui; voi răsplăti omului după faptele lui".
\par 30 Am trecut pe ogorul unui leneș și pe la via unui om lipsit de minte,
\par 31 Și iată spinii creșteau în toate locurile, mărăcinii o acopereau cu totul, iar zidul de pietre se prăbușise.
\par 32 Atunci m-am uitat Și m-am frământat în inima mea, am privit cu luare aminte și am tras o învățătură:
\par 33 "Încă puțin somn, încă puțină ațipeală, încă puțin să mai stau cu mâinile în sân ca să dorm..."
\par 34 Și sărăcia va veni peste tine ca un călător și lipsa ca un om înarmat.

\chapter{25}

\par 1 Și acestea sunt pildele lui Solomon, pe care le-au adunat oamenii lui Iezechia, regele lui Iuda.
\par 2 Slava lui Dumnezeu este să ascundă lucrurile, iar mărirea regilor e să le cerceteze cu de-amănuntul.
\par 3 Precum înălțimea cerului și adâncul pământului sunt lucruri nepătrunse, tot așa și inima regilor.
\par 4 Curăță argintul de zgură și turnătorul va face din el un vas ales.
\par 5 Dă la o parte pe cel fără de lege din fața mai-marelui și tronul lui se va întări prin dreptate.
\par 6 Nu te făli înaintea cârmuitorului și nu sta în locul hotărât pentru cei mari,
\par 7 Căci mai degrabă să ți se zică: "Suie aici!" decât să te umilească în fața stăpânului. Ceea ce au văzut ochii tăi,
\par 8 Nu aduce grabnic spre dispută, căci ce ai să faci după aceea când aproapele tău te va da de rușine?
\par 9 Ceartă-te cu aproapele tău, dar taina altuia să nu o dai pe față,
\par 10 Ca nu cumva cel ce o aude să nu te defaime și să nu dărâme (pentru totdeauna) faima ta.
\par 11 Ca merele de aur pe poliți de argint, așa este cuvântul spus la locul lui.
\par 12 Inel de aur și podoabe de aur de mult preț este povățuitorul înțelept la urechea ascultătoare.
\par 13 Precum este răcoarea zăpezii în vremea secerișului, așa este solul credincios pentru cei ce-l trimit; el bucură sufletul stăpânului său.
\par 14 Precum sunt norii și vântul fără ploaie, așa este omul care se laudă cu darul pe care niciodată nu-l dă.
\par 15 Prin răbdare se poate îndupleca un om mânios și o limbă dulce înmoaie oase.
\par 16 Dacă ai găsit miere, mănâncă atât cit îți trebuie, ca nu cumva să te saturi și s-o verși.
\par 17 Pune rar piciorul în casa prietenului tău, ca nu cumva să se sature de tine și să te urască.
\par 18 Un ciocan, o sabie și o săgeată ascuțită este omul care dă mărturie mincinoasă împotriva aproapelui său.
\par 19 Dinte rău și picior șovăitor este cel fără credință în vreme de nevoie.
\par 20 Ca atunci când dezbraci haina ne vreme friguroasă, sau torni oțet pe silitră, așa este cântarea pentru o inimă întristată.
\par 21 De flămânzește vrăjmașul tău, dă-i să mănânce pâine și dacă însetează, adapă-l eu apă,
\par 22 Că numai așa îi îngrămădești cărbuni aprinși pe capul lui și Domnul îți v a răsplăti ție.
\par 23 Vântul de la miazănoapte aduce ploaie și limba clevetitoare aduce o față mâhnită.
\par 24 Mai bine să sălășluiești într-un colț de acoperiș, decât să trăiești cu o femeie certăreață într-o casă mare.
\par 25 Precum e apa rece pentru un suflet însetat, așa e vestea bună dintr-o țară depărtată.
\par 26 Ca un izvor tulbure și stricat, așa este dreptul care șovăie în fața celui nelegiuit.
\par 27 Precum celui care mănâncă multă miere nu-i merge bine, tot așa și celui care se lasă copleșit de cuvinte de laudă.
\par 28 Asemenea unei cetăți cu o spărtură și fără zid, așa este omul căruia îi lipsește stăpânirea de sine.

\chapter{26}

\par 1 Precum este zăpada în timpul verii și ploaia la seceriș, așa nu-i place celui nebun cinstea.
\par 2 Precum vrabia zboară și rândunica se înalță în văzduh, tot așa blestemul fără pricină nu nimerește.
\par 3 Biciul este bun pentru cal, frâul pentru măgar, iar varga pentru spatele celor nebuni.
\par 4 Nu răspunde nebunului după nebunia lui, ca să nu te asemeni și tu cu el.
\par 5 Răspunde nebunului după nebunia lui, ca să nu se creadă înțelept în ochii lui.
\par 6 Cel ce încredințează solia în mâna celui nebun își taie picioarele și bea nedreptate.
\par 7 Precum nu poate să se folosească slăbănogul de picioarele sale, tot așa nici cei nebuni de cuvintele cele înțelepte.
\par 8 Ca și când pui o piatră în praștie, așa este cel ce dă cinste unui nebun.
\par 9 Precum un ghimpe intră în mâna unui bețiv, tot așa sunt cuvintele înțelepte în gura celor păcătoși.
\par 10 Ca un arcaș care rănește pe toți, așa este cel ce se pune chezaș pentru cel nebun și pentru cei ce trec pe cale.
\par 11 Ca un câine care se întoarce unde a vărsat, așa este omul nebun care se întoarce la nebunia lui.
\par 12 Dacă vezi un om care se crede înțelept în ochii lui, să nădăjduiești mai mult de la un nebun decât de la el.
\par 13 Leneșul zice: "Pe drum trece un leu, un leu pe ulițe!"
\par 14 Precum ușa se sucește în țâțână, tot așa și leneșul în patul lui.
\par 15 Leneșul bagă mâna în blid, dar cu mare greutate o duce la gură.
\par 16 Leneșul se crede înțelept în ochii lui, mai mult decât șapte sfetnici înțelepți.
\par 17 Ca cel ce prinde un câine de urechi, așa este cel ce se vâră într-o ceartă în care nu este amestecat.
\par 18 Ca unul care aruncă săgeți arzătoare, lănci, săgeți și moarte,
\par 19 Așa e omul care înșală pe prietenul său și zice: "Da, am glumit!"
\par 20 Când nu mai sunt lemne se stinge focul și dacă nu mai este nici un defăimător se potolește cearta.
\par 21 Cărbunii slujesc pentru căldură, lemnele pentru foc, iar omul certăreț pentru a ațâța cearta.
\par 22 Vorbele celui defăimător sunt ca bucatele gustoase; ele se duc în adâncul măruntaielor.
\par 23 Spoială de argint care îmbracă un vas de lut, așa sunt buzele mieroase și o inimă rea.
\par 24 Cu buzele sale se preface cel ce urăște, iar înlăuntrul lui nutrește înșelăciune;
\par 25 Când își schimbă glasul, să nu-l crezi, căci șapte urâciuni sunt în inima lui.
\par 26 Cineva poate să-și ascundă ura lui prin prefăcătorie, dar în adunare răutatea lui se dă pe față.
\par 27 Cine sapă groapa (altuia) cade singur în ea și cel ce rostogolește o piatră se prăvălește (tot) peste el.
\par 28 Limba mincinoasă urăște adevărul și gura lingușitorilor pricinuiește prăbușirea.

\chapter{27}

\par 1 Nu te lăuda cu ziua de mâine, că nu știi la ce poate da naștere.
\par 2 Să te laude altul și nu gura ta, un străin și nu buzele tale.
\par 3 Piatra este grea și nisipul cu anevoie de ridicat; însă furia nebunului este mai grea decât amândouă.
\par 4 Întărâtarea este crudă și mânia aprigă, dar tăria pizmei cine o va putea îndura?
\par 5 Mai mult prețuiește o dojană pe față decât o dragoste ascunsă.
\par 6 De bună credință sunt rănile pricinuite de un prieten, iar sărutările celui ce te urăște sunt viclene.
\par 7 Sătulul calcă mierea în picioare, iar flămândului tot ce este amar (i se pare) dulce.
\par 8 Ca o pasăre gonită din cuibul ei, așa este omul izgonit din casa sa.
\par 9 Untdelemnul și miresmele înveselesc inima, dar tot așa de dulci sunt sfaturile unui prieten care pornesc din suflet.
\par 10 Pe prietenul tău și pe prietenul tatălui tău nu-i părăsi; în casa fratelui tău nu intra în ziua restriștii tale. Mai bun e un vecin aproape de tine, decât un frate departe.
\par 11 Fii înțelept, fiul meu, și bucură inima mea, ca să pot răspunde celui ce mă clevetește.
\par 12 Înțeleptul vede nenorocirea și se ascunde, cei proști dau peste ea și îndură necaz.
\par 13 Ia-i haina căci s-a pus chezaș pentru altul și cere-i zălog din pricina celor străini.
\par 14 Celui ce binecuvântează pe prietenul său cu glas mare dis-de-dimineață, i se socotește ca un blestem.
\par 15 Un jgheab care curge în vreme de ploaie și o femeie arțăgoasă sunt la fel;
\par 16 Cel care vrea s-o oprească oprește vânt și mâna lui cea dreaptă parcă ar ține în ea untdelemn.
\par 17 Fierul cu fier se ascute și un om ascute mânia altui om.
\par 18 Cel ce păzește un smochin mănâncă din rodul lui, iar cel ce păzește pe stăpânul său va fi răsplătit cu cinste.
\par 19 Precum nu se aseamănă față cu față, tot așa inima unui om cu inima altuia.
\par 20 Iadul și adâncul nu se pot sătura, tot așa și inima omului e de nesăturat.
\par 21 În topitoare se lămurește argintul și în cuptor aurul, iar omul se prețuiește după numele cel bun.
\par 22 Chiar dacă vei pisa în piuliță cu pilugul pe cel nebun, întocmai ca pe boabe, tot nu-l vei despărți de nebunia lui.
\par 23 Sârguiește-te să-ți cunoști oile tale și ia seama la turma ta,
\par 24 Că bunăstarea nu dăinuiește de-a pururi și nici bogăția din neam în neam.
\par 25 Când iarba s-a trecut și pășunea s-a isprăvit și finul de pe dealuri s-a strâns,
\par 26 Tu ai miei pentru îmbrăcămintea ta și țapi ea să plătești pășunea;
\par 27 Și laptele de capră îl ai cu îndestulare, pentru hrana casei, și merinde pentru slujnicele tale.

\chapter{28}

\par 1 Cel nelegiuit fuge fără ca nimeni să-l urmărească, iar dreptul stă ca un pui de leu fără grijă.
\par 2 Din pricina greșelilor unui om silnic se ivesc certuri, iar omul iscusit le stinge.
\par 3 Un om bogat, care asuprește pe cei săraci, e ca ploaia care trântește tot la pământ, iar pâinea nu se face.
\par 4 Cei ce părăsesc legea ridică în slăvi pe păcătoși, iar cei ce o păzesc se aprind împotriva lor.
\par 5 Oamenii răi nu pricep nimic din ceea ce e drept, iar cei ce caută pe Domnul înțeleg tot.
\par 6 Mai de preț e săracul care umblă întru neprihănirea lui, decât cel prefăcut în căile lui, chiar dacă e bogat.
\par 7 Cel ce păzește legea este un fiu înțelept, iar cel ce se întovărășește cu clevetitorii face rușine tatălui său.
\par 8 Cel ce își sporește averea lui, prin dobândă și prin camătă, adună pentru cel ce are milă de săraci.
\par 9 Cel ce își oprește urechea de la ascultarea legii, chiar rugăciunea lui e urâciune.
\par 10 Cel ce rătăcește pe cei drepți pe o cale rea va cădea în groapa (pe care a săpat-o); cei fără prihană vor fi fericiți.
\par 11 Omul bogat este înțelept în ochii lui, dar cel sărac și priceput îl dovedește cu mintea.
\par 12 Când drepții biruiesc e mare sărbătoare, iar când cei fără de lege ies la iveală, oamenii se ascund.
\par 13 Cel ce își ascunde păcatele lui nu propășește, iar cel ce le mărturisește și se lasă de ele va fi miluit.
\par 14 Fericit este omul care se teme totdeauna, iar cel ce își învârtoșează inima lui va cădea în nenorocire.
\par 15 Leu care răcnește și urs flămând este cel rău care stăpânește peste un popor sărac.
\par 16 Stăpânitorul cel lipsit de venituri este mare asupritor; cel ce urăște câștigul (nedrept) va trăi multă vreme.
\par 17 Un om pe care îl îngreuiază sângele unui ucis fuge până la groapă; nimeni să nu-l oprească!
\par 18 Cel ce umblă fără prihană va fi mântuit, iar cine apucă pe căi strâmbe va cădea într-o groapă.
\par 19 Cel ce lucrează pământul lui se va îndestula de pâine, iar cel ce umblă după lucruri de nimic se va sătura de sărăcie.
\par 20 Omul credincios va fi încărcat de binecuvântări, iar cine zorește să ajungă bogat nu va rămâne nepedepsit.
\par 21 Nu este bine să te uiți la fața omului, căci pentru o bucată de pâine cineva poate să greșească.
\par 22 Omul lacom se grăbește să se îmbogățească, dar nu gândește că lipsa va veni peste el.
\par 23 Cel care ceartă pe un om va avea mai multă mulțumire decât cel care-l lingușește.
\par 24 Cine despoaie pe tatăl său și pe mama sa și zice: "Nu-i păcat!" este tovarăș cu făcătorul de rele.
\par 25 Omul lacom ațâță cearta, iar cel ce nădăjduiește în Domnul va fi îndestulat.
\par 26 Cel ce își pune nădejdea în inima lui este un nebun, iar cel ce se conduce după înțelepciune, acela va fi mântuit.
\par 27 Cine dă la cel sărac nu duce lipsă; iar cine își acoperă ochii lui va fi mult blestemat.
\par 28 Când nelegiuiții ies la iveală, oamenii se ascund, iar când ei pier, se înmulțesc cei drepți.

\chapter{29}

\par 1 Un om pedepsit îndelung și tare la cerbice va fi intr-o clipă zdrobit și fără vindecare.
\par 2 Când drepții domnesc se bucură poporul și, când stăpânesc cei fără de lege, suspină.
\par 3 Cine iubește înțelepciunea bucură pe tatăl său, iar cine umblă cu desfrânatele își prăpădește averea.
\par 4 Un conducător prin dreptate face să propășească țara, iar cel ce pune dări grele o ruinează.
\par 5 Omul care lingușește pe prietenul său întinde cursă pașilor lui.
\par 6 Pe calea celui rău este întins un laț, dar dreptul trebuie să fugă și să salte de bucurie.
\par 7 Omul drept se îngrijește de pricina celor sărmani; celui fără de lege nu-i pasă de ei.
\par 8 Batjocoritorii răscoală cetatea, iar cei înțelepți potolesc mânia.
\par 9 Când un înțelept se ceartă cu un nebun, fie că se supără, fie că râde, nu-și pierde cumpătul.
\par 10 Oamenii setoși de sânge urăsc pe cel fără prihană, iar cei drepți ocrotesc viața lui.
\par 11 Nebunul face să izbucnească pornirea lui pătimașă, iar înțeleptul își înfrânează mânia.
\par 12 Când un conducător ascultă de cuvinte mincinoase, toți slujitorii săi sunt răi.
\par 13 Săracul și cu cel ce asuprește pe cei săraci se întâlnesc; Cel ce luminează ochii amândurora este Domnul.
\par 14 Un conducător care judecă cu dreptate pe cei săraci își întărește scaunul lui pe veci.
\par 15 Varga și certarea aduc înțelepciune, iar tânărul care este lăsat (în voia apucăturilor lui) face rușine maicii sale.
\par 16 Când cei fără de lege domnesc se înmulțesc răutățile, iar drepții vor vedea (cu bucurie) prăbușirea lor.
\par 17 Mustră pe fiul tău și el iți va fi odihnă și îți va face plăcere sufletului tău.
\par 18 Fără vedenie de prooroc poporul e fără stăpân, dar fericit este cel care păzește legea!
\par 19 Sluga nu se îndreaptă numai cu povețe, fiindcă, deși pricepe, însă nu ascultă.
\par 20 Dacă vezi un om care se zorește la vorbă, atunci pentru un nebun e mai multă nădejde decât pentru el.
\par 21 Dacă (vreun stăpân) dezmiardă din copilărie pe robul său, acesta ajunge la sfârșit să se creadă fiu.
\par 22 Un om mânios ațâță cearta și cel aprig săvârșește multe păcate.
\par 23 Mândria umilește pe om, iar de cinste are parte cel smerit.
\par 24 Cel ce împarte cu hoțul își urăște sufletul lui, fiindcă aude blestemul, dar nu zice nimic.
\par 25 Teama de oameni duce la căderea în cursă, dar cel ce nădăjduiește în Domnul stă la adăpost.
\par 26 Mulți caută fața stăpânitorului, dar dreptatea omului vine de la Domnul.
\par 27 Omul nedrept este urâciune pentru cei drepți, iar cel drept este o urâciune pentru cei răi.

\chapter{30}

\par 1 Cuvintele lui Agur, fiul lui Iache din Massa. Acest om a zis: "Sunt ostenit, Dumnezeule, sunt obosit, Doamne, sunt sleit de puteri!
\par 2 Căci sunt tare prost, ca să mă pot socoti ca om și nu am pricepere (care ar putea să fie vrednică) de un om.
\par 3 Nici n-am învățat înțelepciunea și nici știința celor sfinți nu o cunosc.
\par 4 Cine s-a suit în ceruri și iarăși s-a pogorât, cine a adunat vântul în mâinile lui? Cine a legat apele în haina lui? Cine a întărit toate marginile pământului? Care este numele lui și care este numele fiului său? Spune dacă știi!
\par 5 Toate cuvintele lui Dumnezeu sunt lămurite, scut este El pentru cei ce caută la El scăparea.
\par 6 Nu adăuga nimic la cuvintele Lui, ca să nu te tragă la socoteală și să fii găsit de minciună!
\par 7 Două lucruri cer de la Tine, nu mă respinge înainte de a muri:
\par 8 Prefăcătoria și cuvântul mincinos îndepărtează-le de la mine; sărăcie și bogăție nu-mi da, ci dă-mi pâinea care-mi este de trebuință,
\par 9 Ca nu cumva, săturându-mă, să mă lepăd de Tine și să zic: "Cine este Domnul?" Ca nu cumva, sărăcind, să mă apuc de furat și să defaim numele Dumnezeului meu.
\par 10 Nu grăi de rău pe slugă la stăpânul său, ca nu cumva să te blesteme și să te silească să-ți ceri iertare.
\par 11 Este câte un neam de oameni care blesteamă pe tatăl său și nu binecuvântează pe maica sa;
\par 12 Un neam căruia i se pare că e fără prihană în ochii lui și care nu este curățit de necurăția lui;
\par 13 Un neam... O, cum ridică ochii lui sus și cit se înalță de sus genele lui!
\par 14 Un neam ai cărui dinți sunt ca săbiile și ai căror colți sunt cuțite, ca să mănânce pe cei sărmani de pe pământ și pe cei săraci dintre oameni.
\par 15 Lipitoarea are două fiice care zic: "Dă-mi, dă-mi!" Trei lucruri nu se pot sătura, ba și al patrulea care nu zice niciodată: "Destul!" și anume:
\par 16 Locuința morților, pântecele sterp, pământul care nu e sătul de apă și focul care nu zice niciodată: "Destul!"
\par 17 Ochiul care își bate joc de părintele său și nu ia în seamă ascultarea (ce este dator) maicii sale, să-l scoată corbii care sălășluiesc lingă un curs de apă, iar puii de vultur să-l mănânce.
\par 18 Trei lucruri mi se par minunate, ba chiar patru, pe care nu le pot pricepe:
\par 19 Calea vulturului pe cer, urma șarpelui pe stâncă, mersul corăbiei în mijlocul mării și calea omului la o fecioară.
\par 20 Așa este purtarea unei femei desfrânate: ea mănâncă și își șterge gura și zice: "N-am făcut nimic rău"
\par 21 Pentru trei lucruri se cutremură pământul, ba chiar pentru patru nu poate să rabde:
\par 22 Pentru robul care ajunge rege, pentru nebunul care se satură de pâine,
\par 23 Pentru o femeie disprețuită când ea se mărită și pentru o slugă care moștenește pe stăpâna sa.
\par 24 Patru sunt animalele cele mai mici de pe pământ și care sunt cele mai înțelepte:
\par 25 Furnicile, neam fără putere, care își agonisesc vara hrana lor;
\par 26 Dihorii, neam slab, care-și clădesc în stânci locașul lor;
\par 27 Lăcustele care nu au rege și totuși ies toate în stoluri;
\par 28 Șopârla care se poate prinde cu mâna și care pătrunde în palatele regilor.
\par 29 Trei ființe au înfățișare frumoasă, ba patru, care au un mers măreț:
\par 30 Leul, viteazul printre dobitoace, care nu dă înapoi în fața nimănui;
\par 31 Cocoșul cel ager, țapul și regele căruia nimeni nu-i poate sta împotrivă.
\par 32 De ești așa de nebun ca să te lași mânat de mânie, bate-te cu mâna peste gură.
\par 33 Bătutul laptelui dă untul, lovitura peste nas face să țâșnească sângele, iar întărâtarea mâniei duce la ceartă.

\chapter{31}

\par 1 Cuvintele lui Lemuel, regele din Massa, cu care mama sa îl învăța:
\par 2 Fiul meu, rodul pântecelui meu, feciorul făgăduințelor mele, cu ce pot eu să te îndemn?
\par 3 Nu da puterea ta femeilor și căile taie celor care pierd pe regi.
\par 4 Nu se cuvine regilor, o, Lemuel, nu se cuvine regilor să bea vin și conducătorii băuturi îmbătătoare,
\par 5 Ca nu cumva bând să uite legea și să judece strâmb pe toți sărmanii.
\par 6 Dați băutură îmbătătoare celui ce este gata să piară și vin celui cu amărăciune în suflet,
\par 7 Ca să bea și să uite sărăcia și să nu-și mai aducă aminte de chinul lui.
\par 8 Deschide gura ta pentru cel mut și pentru pricina tuturor părăsiților.
\par 9 Deschide gura ta, judecă drept și fă dreptate celui sărac și năpăstuit.
\par 10 Cine poate găsi o femeie virtuoasă? Prețul ei întrece mărgeanul.
\par 11 Într-însa se încrede inima soțului ei, iar câștigul nu-i va lipsi niciodată.
\par 12 Ea îi face bine și nu rău în tot timpul vieții sale:
\par 13 Ea caută lână și cânepă și lucrează voios cu mâna sa.
\par 14 Ea se aseamănă cu corabia unui neguțător care de departe aduce hrana ei.
\par 15 Ea se scoală dis-de-dimineață și împarte hrana în casa ei și dă porunci slujnicelor.
\par 16 Gândește să cumpere o țarină și o dobândește; din osteneala palmelor sale sădește vie.
\par 17 Ea își încinge cu putere coapsele sale și își întărește brațele sale.
\par 18 Ea simte că bun e câștigul ei; sfeșnicul ei nu se stinge nici noaptea.
\par 19 Ea pune mâna pe furcă și cu degetele sale apucă fusul.
\par 20 Ea întinde mâna spre cel sărman și brațul ei spre cel necăjit.
\par 21 N-are teamă pentru cei ai casei sale în vreme de iarnă, căci toți din casă sunt îmbrăcați în haine stacojii.
\par 22 Ea își face scoarțe; hainele ei sunt de vison și de porfiră.
\par 23 Cinstit este bărbatul ei la porțile cetății, când stă la sfat cu bătrânii țării.
\par 24 Ea face cămăși și le vinde, și brâie dă neguțătorilor.
\par 25 Tărie și farmec este haina ei și ea râde zilei de mâine.
\par 26 Gura și-o deschide cu înțelepciune și sfaturi pline de dragoste sunt pe limba ei.
\par 27 Ea veghează la propășirea casei sale și pâine, fără să lucreze; ea nu mănâncă.
\par 28 Feciorii săi vin și o fericesc, iar soțul ei o laudă:
\par 29 "Multe fete s-au dovedit harnice, dar tu le-ai întrecut pe toate!"
\par 30 Înșelător este farmecul și deșartă este frumusețea; femeia care se teme de Domnul trebuie lăudată!
\par 31 Să se bucure de rodul mâinilor sale, și la porțile cetății hărnicia ei să fie dată ca pildă!


\end{document}