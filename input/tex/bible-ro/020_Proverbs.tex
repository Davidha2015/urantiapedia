\begin{document}

\title{Proverbele}


\chapter{1}

\par 1 Pildele lui Solomon, fiul lui David,
\par 2 Folositoare pentru cunoa?terea în?elepciunii ?i a stapânirii de sine,
\par 3 Pentru în?elegerea cuvintelor adânci, pentru dobândirea unei îndrumari bune, pentru dreptate, pentru dreapta judecata ?i nepartinire,
\par 4 Pentru a prilejui celor fara gând rau o judecata istea?a, omului tânar cuno?tin?a ?i buna cugetare.
\par 5 Sa ia aminte cel în?elept ?i î?i va spori ?tiin?a, iar cel priceput va dobândi iscusin?a de a se purta,
\par 6 Patrunzând cu mintea pildele ?i în?elesurile adânci, graiurile celor în?elep?i ?i tâlcuirea lor nepatrunsa.
\par 7 Frica de Dumnezeu este începutul în?elepciunii; cei fara minte dispre?uiesc în?elepciunea ?i stapânirea de sine.
\par 8 Asculta, fiul meu, înva?atura tatalui tau ?i nu lepada îndrumarile maicii tale.
\par 9 Caci ele sunt ca o cununa pe capul tau ?i ca o salba împrejurul gâtului tau.
\par 10 Fiul meu, de voiesc pacato?ii sa te ademeneasca, nu te învoi,
\par 11 Daca-?i spun: "Vino cu noi, sa ne punem la pânda, ca sa varsam sânge, sa întindem curse fara cuvânt celui neprihanit,
\par 12 Sa-i înghi?im de vii ca locuin?a mor?ilor, ?i întregi, ca pe cei ce se coboara în mormânt.
\par 13 Sa punem stapânire pe tot felul de lucruri scumpe, sa ne umplem de prada casele noastre,
\par 14 Fii parta? la ob?tea noastra, o singura punga fi-va pentru to?i!"
\par 15 Fiul meu, nu te întovara?i cu ei pe cale; abate piciorul tau din cararea lor,
\par 16 Caci picioarele lor alearga numai la rau, iar ei zoresc sa verse sânge.
\par 17 Zadarnic se întind curse în vazul pasarilor!
\par 18 Caci ei întind curse tocmai împotriva sângelui lor, ?i sufletului lor î?i întind ei la?uri.
\par 19 Aceasta este soarta celor lacomi de câ?tig; lacomia le aduce pierderea vie?ii.
\par 20 În?elepciunea striga pe uli?a ?i în piele î?i ridica glasul sau.
\par 21 Ea propovaduie?te la raspântiile zgomotoase; înaintea por?ilor ceta?ii î?i spune cuvântul:
\par 22 "Pâna când, pro?tilor, ve?i iubi prostia? Pâna când, nebunilor, ve?i iubi nebunia? ?i voi, ne?tiutorilor, pâna când ve?i urî ?tiin?a?
\par 23 Întoarce?i-va iara?i la mustrarea mea ?i iata eu voi turna peste voi duhul meu ?i va voi vesti cuvintele mele.
\par 24 Chematu-v-am, dar voi n-a?i luat aminte! Întinsu-mi-am mâna ?i n-a fost cine sa ia seama!
\par 25 Ci a?i lepadat toate sfaturile mele ?i mustrarile mele nu le-a?i primit.
\par 26 De aceea ?i eu voi râde de pieirea voastra ?i ma voi bucura când va veni groaza peste voi,
\par 27 Când va veni peste voi necazul ca furtuna ?i când nenorocirea ca vijelia va va cuprinde.
\par 28 Atunci ma vor chema, dar eu nu voi raspunde; din zori ma vor cauta ?i nu ma vor afla,
\par 29 Pentru ca ei au urât ?tiin?a ?i frica de Dumnezeu n-au ales-o,
\par 30 Fiindca n-au luat aminte la sfaturile mele ?i cercetarea mea au dispre?uit-o.
\par 31 Mânca-vor din rodul caii lor ?i de sfaturile lor satura-se-vor,
\par 32 Caci îndaratnicia omoara pe cei pro?ti ?i nepasarea pierde pe cei fara minte;
\par 33 Iar cel ce ma asculta va trai în pace ?i lini?te ?i de rele nu se va teme".

\chapter{2}

\par 1 Fiul meu, de vei primi pove?ele mele ?i sfaturile mele de le vei pastra,
\par 2 Plecându-?i urechea la în?elepciune ?i înclinând inima ta spre buna chibzuiala,
\par 3 Daca vei chema prevederea ?i spre buna-cugetare î?i vei îndrepta glasul tau,
\par 4 Daca o vei cauta întocmai ca pe argint ?i o vei sapa ca ?i pe o comoara,
\par 5 Atunci vei pricepe temerea de Domnul ?i vei dobândi cuno?tin?a de Dumnezeu,
\par 6 Caci Domnul da în?elepciune; din gura Lui izvora?te ?tiin?a ?i prevederea;
\par 7 El pastreaza mântuirea pentru oamenii cei drep?i; El este scut pentru cei ce umbla în calea desavâr?irii;
\par 8 El paze?te caile drepta?ii ?i pe cararea celor cuvio?i ai Lui sta de veghe.
\par 9 Atunci tu vei în?elege dreptatea ?i buna judecata, calea cea dreapta ?i toate potecile binelui.
\par 10 Când în?elepciunea se va sui la inima ta ?i ?tiin?a va desfata sufletul tau,
\par 11 Când buna chibzuiala va veghea peste tine ?i în?elegerea te va pazi,
\par 12 Atunci tu vei fi izbavit de calea celui rau ?i de omul care graie?te minciuna,
\par 13 De cei ce parasesc caile cele drepte, ca sa umble pe drumuri întunecoase,
\par 14 De cei ce se bucura când fac rau ?i se veselesc când umbla pe poteci întortochiate,
\par 15 Ale caror carari sunt strâmbe ?i ratacesc pe cai piezi?e.
\par 16 Atunci tu vei scapa de femeia care este a altuia, de straina ale carei cuvinte sunt ademenitoare,
\par 17 Care lasa pe tovara?ul ei din tinere?e ?i uita de legamântul Dumnezeului ei,
\par 18 Caci ea se pleaca împreuna cu casa ei spre moarte ?i drumul ei duce în iad;
\par 19 Nimeni din cei ce se duc la ea nu se mai întoarce ?i niciunul nu mai afla cararile vie?ii.
\par 20 Drept aceea mergi pe calea oamenilor celor buni ?i paze?te cararile celor drep?i,
\par 21 Caci cei drep?i vor locui pamântul ?i cei fara de prihana vor sala?lui pe el;
\par 22 Iar cei fara de lege vor fi nimici?i de pe pamânt ?i cei necredincio?i vor fi smul?i de pe el.

\chapter{3}

\par 1 Fiul meu, nu uita înva?atura mea ?i inima ta sa pazeasca sfaturile mele,
\par 2 Caci lungime de zile ?i ani de via?a ?i propa?ire îi se vor adauga.
\par 3 Mila ?i adevarul sa nu te paraseasca; leaga-le împrejurul gâtului tau, scrie-le pe tabla inimii tale;
\par 4 Atunci vei afla har ?i bunavoin?a înaintea lui Dumnezeu ?i a oamenilor.
\par 5 Pune-?i nadejdea în Domnul din toata inima ta ?i nu te bizui pe priceperea ta.
\par 6 Pe toate caile tale gânde?te la Dânsul ?i El î?i va netezi toate cararile tale.
\par 7 Nu fii în?elept în ochii tai; teme-te de Dumnezeu ?i fugi de rau;
\par 8 Aceasta va fi sanatate pentru trupul tau ?i o înviorare pentru oasele tale.
\par 9 Cinste?te pe Domnul din averea ta ?i din pârga tuturor roadelor tale.
\par 10 Atunci jitni?ele tale se vor umple de grâu ?i mustul va da afara din teascurile tale.
\par 11 Fiul meu, nu dispre?ui certarea Domnului ?i nu sim?i scârba pentru mustrarile Lui,
\par 12 Caci Domnul cearta pe cel pe care-l iube?te ?i ca un parinte pedepse?te pe feciorul care îi este drag.
\par 13 Fericit este omul care a aflat în?elepciunea ?i barbatul care a dobândit pricepere,
\par 14 Caci dobândirea ei este mai scumpa decât argintul ?i pre?ul ei mai mare decât al celui mai curat aur.
\par 15 Ea este mai pre?ioasa decât pietrele scumpe; nici un rau nu i se poate împotrivi ?i e bine-cunoscuta tuturor celor ce se apropie de ea; nimic din cele dorite de tine nu se aseamana cu ea.
\par 16 Via?a lunga este în dreapta ei, iar în stânga ei, boga?ie ?i slava; din gura ei iese dreptatea; legea ?i mila pe limba le poarta.
\par 17 Caile ei sunt placute ?i toate cararile ei sunt caile pacii.
\par 18 Pom al vierii este ea pentru cei ce o stapânesc, iar cei care se sprijina pe ea sunt ferici?i.
\par 19 Prin în?elepciune, Domnul a întemeiat pamântul, iar prin în?elegere a întarit cerurile.
\par 20 Prin ?tiin?a Sa a deschis adâncurile ?i norii picura roua.
\par 21 Fiul meu, sa nu se departeze acestea dinaintea ochilor tai; pastreaza în?elepciunea ?i buna chibzuiala,
\par 22 Caci ele sunt via?a sufletului tau ?i podoaba pentru gâtul tau.
\par 23 Atunci tu vei merge fara teama pe calea ta ?i piciorul tau nu se va poticni.
\par 24 De te culci, nu-?i va fi teama, iar de adormi, somnul tau va fi dulce.
\par 25 Sa nu te temi de frica fara veste ?i nici de vreo navala a celor pacato?i,
\par 26 Ca Domnul este nadejdea ta ?i va feri piciorul tau de cursa.
\par 27 Nu zabovi a face bine celui ce are nevoie, când ai putin?a sa-i aju?i.
\par 28 Nu spune aproapelui tau: "Du-te ?i vino, mâine î?i voi da!", când po?i sa-i dai acum.
\par 29 Nu pune la cale raul împotriva aproapelui tau, când el locuie?te fara grija lânga tine.
\par 30 Nu te certa cu nimeni fara pricina, de vreme ce nu ?i-a facut nici un rau.
\par 31 Nu râvni sa fii ca omul silnic ?i nu alege nici una din caile lui;
\par 32 Caci omul cu gând rau este urât de Domnul, iar de cei drep?i El este mai aproape.
\par 33 Domnul blesteama casa celui fara de lege ?i binecuvânteaza adaposturile celor drep?i.
\par 34 De cei batjocoritori El râde, iar celor smeri?i le da har.
\par 35 Cei în?elep?i vor mo?teni marirea, iar cei nebuni vor avea parte de ocara.

\chapter{4}

\par 1 Asculta?i, fiilor, înva?atura tatalui ?i lua?i aminte sa cunoa?te?i buna chibzuiala,
\par 2 Caci eu va dau înva?atura buna: Nu parasi?i pova?a mea.
\par 3 Caci ?i eu am fost fecior la tatal meu, singur, ?i cu duio?ie iubit la mama mea
\par 4 ?i el ma înva?a ?i-mi zicea: "Inima ta sa pastreze cuvintele inimii mele, paze?te poruncile mele ?i vei fi viu.
\par 5 Aduna în?elepciune, dobânde?te pricepere! Nu le uita ?i nu te departa de la cuvintele gurii mele!
\par 6 Nu o lepada ?i ea te va pazi; iube?te-o ?i ea va sta de veghe.
\par 7 Iata începutul în?elepciunii: Agonise?te în?elepciunea ?i cu pre?ul a tot ce ai, capata priceperea.
\par 8 Pre?uie?te-o mult ?i ea te va înal?a; ea te va ridica în slavi daca o vei îmbra?i?a.
\par 9 Ea va pune cununa de daruri pe capul tau ?i te va împodobi cu diadema de mare cinste.
\par 10 Asculta, fiul meu ?i prime?te cuvintele mele ?i anii vie?ii tale se vor înmul?i.
\par 11 Eu te voi înva?a calea în?elepciunii ?i te voi purta pe caile drepta?ii.
\par 12 Când vei merge, pa?ii tai nu vor ?ovai ?i, chiar de vei alerga, nu te vei poticni.
\par 13 ?ine cu tarie înva?atura ?i nu o parasi, paze?te-o caci ea este via?a ta.
\par 14 Nu apuca pe calea celor fara de lege ?i nu pa?i pe drumul celor rai.
\par 15 Ocole?te-o ?i nu merge pe ea, treci pe alaturea ?i du-te mai departe;
\par 16 Caci ei nu dorm pâna nu faptuiesc rau ?i nu-i mai prinde somnul pâna nu fac pe cineva sa cada.
\par 17 Caci ei se hranesc din pâine agonisita prin faradelege ?i beau vin dobândit prin asuprire.
\par 18 Calea drep?ilor e ca zarea dimine?ii ce se mare?te mereu pâna se face ziua mare;
\par 19 Iar calea celor fara de lege e ca întunericul ?i ei nici nu banuiesc de ce se pot împiedica.
\par 20 Fiul meu, ia aminte la graiurile mele; la pove?ele mele pleaca-?i urechea ta!
\par 21 Nu le scapa din ochi, pastreaza-le înlauntrul inimii tale,
\par 22 Caci ele sunt via?a pentru cei ce le pun în fapta ?i doctorie pentru tot trupul omenesc.
\par 23 Paze?te-?i inima mai mult decât orice, caci din ea ?â?ne?te via?a.
\par 24 Leapada din gura ta orice cuvinte cu în?eles sucit, alunga de pe buzele tale viclenia.
\par 25 Ochii tai sa priveasca drept înainte ?i genele tale drept înainte sa caute.
\par 26 Fii cu luare aminte la calea picioarelor tale ?i toate cararile tale sa fie bine chibzuite.
\par 27 Nu te abate nici la dreapta, nici la stânga, ?ine piciorul tau departe de rau. Caci cararile drepte le paze?te Domnul, iar cele strâmbe sunt cai rele. El va face drepte caile tale ?i mergerea ta o va face sa fie în pace.

\chapter{5}

\par 1 Fiul meu, ia aminte la în?elepciunea mea ?i la sfatul meu cel bun pleaca urechea ta,
\par 2 Ca sa-?i po?i pastra judecata ?i ca buzele tale sa pazeasca ?tiin?a.
\par 3 Nu te uita la femeia lingu?itoare, caci buzele celei straine picura miere ?i cerul gurii sale e mai alunecator decât untdelemnul,
\par 4 Dar la sfâr?it ea este mai amara decât pelinul, mai taioasa decât o sabie cu doua ascu?i?uri.
\par 5 Picioarele ei coboara catre moarte; pa?ii ei duc de-a dreptul în împara?ia mor?ii.
\par 6 Ea nu ia seama la calea vie?ii, pa?ii ei merg în ne?tire, nici ea nu ?tie unde.
\par 7 ?i acum, fiul meu, asculta-ma ?i nu te îndeparta de la cuvintele gurii mele.
\par 8 Fere?te-?i calea ta de ea ?i nu te apropia de u?a casei ei,
\par 9 Ca sa nu dai vârtutea ta altora ?i anii tai unuia fara de mila;
\par 10 Ca strainii sa nu se îndestuleze de stradania ta ?i ostenelile tale sa nu treaca în casa altuia;
\par 11 Ca sa nu suspini la sfâr?it, când trupul tau ?i carnea ta vor fi fara de vlaga,
\par 12 ?i sa zici: "Pentru ce am urât pova?a ?i de ce inima mea a urgisit certarea?
\par 13 De ce nu am ascultat de îndemnul dascalilor mei ?i spre cei ce ma înva?au n-am plecat urechea mea?
\par 14 Pu?in a trebuit sa nu ma nenorocesc, în plina adunare ?i în mijlocul ob?tei".
\par 15 Bea apa din pu?ul tau ?i din pârâia?ele care curg din izvorul tau.
\par 16 Sa nu se risipeasca izvoarele tale pe uli?a, nici pâraiele tale prin pie?e.
\par 17 Sa fie numai pentru tine singur, iar nu pentru strainii care sunt cu tine!
\par 18 Binecuvântat sa fie izvorul tau ?i sa te mângâi cu femeia ta din tinere?e.
\par 19 Cerboaica preaiubita ?i gazela plina de farmec sa-?i fie ea; dragostea de ea sa te îmbete totdeauna ?i iubirea ei sa te desfateze.
\par 20 Pentru ce, fiul meu, sa te momeasca femeie straina ?i tu sa îmbra?i?ezi sânul unei necunoscute?
\par 21 Caci cararile omului sunt înaintea Domnului ?i El ia seama la toate caile lui.
\par 22 Cel fara de lege este prins în la?urile faradelegilor lui ?i de funiile pacatelor lui este înfa?urat.
\par 23 El va muri în pacatele lui ?i de mul?imea nebuniei lui va pieri.

\chapter{6}

\par 1 Fiul meu, daca te-ai pus cheza? pentru prietenul tau, daca ai dat mâna pentru altul,
\par 2 Atunci te-ai prins prin fagaduieli ie?ite din gura ta ?i te-ai legat prin cuvintele gurii tale.
\par 3 Fa dar, fiul meu, aceasta: o, izbave?te-te. ?i fiindca ai cazut în mâinile aproapelui tau, du-te ?i cazi la picioarele aproapelui tau ?i-l roaga;
\par 4 Nu da somn ochilor tai, nici dormitare genelor tale.
\par 5 ?i te izbave?te ca o caprioara din mâna vânatorului ?i ca o pasare din mâna pasararului.
\par 6 Du-te, lene?ule, la furnica ?i vezi munca ei ?i prinde minte!
\par 7 Ea, care nu are nici mai-mare peste ea, nici îndrumator, nici sfatuitor,
\par 8 Î?i pregate?te de cu vara hrana ei ?i î?i strânge la seceri? mâncare. Sau mergi la albina ?i vezi cât e de harnica ?i ce lucrare iscusita savâr?e?te. Munca ei o folosesc spre sanatate ?i regii ?i oamenii de rând. Ea e iubita ?i laudata de to?i, de?i e slaba în putere, dar e minunata cu iscusin?a.
\par 9 Pâna când, lene?ule, vei mai sta culcat? Când te vei scula din somnul tau?
\par 10 "Pu?in somn, înca pu?ina a?ipire, pu?in sa mai stau în pat cu mâinile încruci?ate!"
\par 11 Iata vine saracia ca un trecator ?i nevoia te prinde ca un tâlhar. Dar daca nu vei lenevi, atunci va veni seceri?ul tau ca un izvor, iar lipsa va fi departe de tine.
\par 12 Omul de nimic, omul necinstit ?i viclean umbla cu minciuna pe buze.
\par 13 Face cu ochiul, da din picioare, face semne cu degetele.
\par 14 în inima lui e vicle?ug, pururea se gânde?te la rau ?i seamana gâlceava.
\par 15 Pentru aceasta fara de veste va veni peste el prapadul, nimicit va fi dintr-o data ?i fara leac.
\par 16 ?ase sunt lucrurile pe care le ura?te Domnul, ba chiar ?apte de care se scârbe?te cugetul Sau:
\par 17 Ochii mândri, limba mincinoasa, mâinile care varsa sânge nevinovat,
\par 18 Inima care planuie?te gânduri viclene, picioare grabnice sa alerge spre rau,
\par 19 Martorul mincinos care spune minciuni ?i cel care seamana vrajba între fra?i.
\par 20 Paze?te, fiule, pova?a tatalui tau ?i nu lepada îndemnul maicii tale.
\par 21 Leaga-le la inima ta, pururea atârna-le de gâtul tau.
\par 22 Ele te vor conduce când vei vrea sa mergi; în vremea somnului te vor pazi, iar când te vei de?tepta vor grai cu tine.
\par 23 Ca pova?a este un sfe?nic bun ?i legea o lumina, iar îndemnurile care dau înva?atura sunt calea vie?ii.
\par 24 Ele te vor pazi de femeia vicleana, de limba cea ademenitoare a celei straine.
\par 25 Nu dori frumuse?ea ei întru inima ta ?i sa nu te vâneze cu genele ei.
\par 26 Ca femeia desfrânata umbla dupa o bucata de pâine, pe când femeia-so?ie dore?te un suflet de mare pre?.
\par 27 Oare poate pune cineva foc în sânul lui, fara ca ve?mintele lui sa nu arda?
\par 28 Sau va merge cineva pe carbuni fara sa i se friga talpile?
\par 29 A?a este cu cel ce se duce la femeia aproapelui sau: nimeni din cei ce se ating de ea nu va ramâne nepedepsit.
\par 30 Nimeni nu dispre?uie?te un ho? pentru ca a furat ca sa-?i astâmpere foamea;
\par 31 Dar când a fost prins, el da înapoi în?eptit, întoarce tot ceea ce are în casa lui.
\par 32 Cel ce se desfrâneaza însa cu o femeie este lipsit de minte, se pierde pe el însu?i facând astfel;
\par 33 El nu dobânde?te decât bataie, iar ocara lui niciodata nu se ?terge.
\par 34 Pizma treze?te mânia omului defaimat ?i el este fara mila în ziua razbunarii;
\par 35 El nu se uita la nici un pre? de rascumparare, ?i chiar când îi vei spori darurile, tot nu se îmblânze?te.

\chapter{7}

\par 1 Fiul meu; paze?te spusele mele ?i îndrumarile mele ascunde-le la tine.
\par 2 Pastreaza sfaturile mele ca sa ramâi în via?a ?i orânduielile mele ca lumina ochilor tai.
\par 3 Leaga-le pe degetele tale, scrie-le pe tabla inimii tale!
\par 4 Spune în?elepciunii: "Tu e?ti sora mea!", ?i nume?te priceperea prietena ta,
\par 5 Ca ea sa te pazeasca de femeia straina, de femeia altuia, ale carei cuvinte sunt ademenitoare.
\par 6 Odata stam la fereastra casei mele ?i priveam printre gratii,
\par 7 ?i am zarit printre cei lipsi?i de minte, am vazut un tânar fara pricepere.
\par 8 El trecea pe uli?a pe lânga col?ul casei ei ?i se îndrepta catre locuin?a ei.
\par 9 Era în amurgul serii unei zile, când se lasa umbra ?i întunericul nop?ii.
\par 10 ?i iata o femeie îl întâmpina, având înfa?i?are de desfrânata ?i cu prefacatorie în inima;
\par 11 Apriga ?i de ne?inut în frâu, picioarele ei nu se mai odihneau în casa;
\par 12 Când în casa, când afara, stând la pânda lânga orice col?.
\par 13 Ea îl apuca ?i-l saruta ?i cu o cautatura obraznica îi zise:
\par 14 "Trebuia sa aduc jertfe de pace; astazi am împlinit fagaduin?ele mele;
\par 15 Pentru aceasta am ie?it în întâmpinarea ta, ca sa te caut ?i iata ca te-am gasit.
\par 16 Cu scoar?e am gatit patul meu, cu a?ternuturi de in din Egipt,
\par 17 Cu miresme am stropit patul meu, cu mir, aloe ?i chinamon.
\par 18 Vino, sa ne îmbatam de iubire pâna diminea?a, sa ne cufundam în desfatari de dragoste,
\par 19 Ca barbatul meu nu este acasa, plecat-a la drum departe,
\par 20 Luat-a cu dânsul o punga cu bani ?i se va întoarce acasa la luna plina!"
\par 21 Ea îl ademeni prin mul?imea cuvintelor ei ?i-l smulse prin graiurile ademenitoare ale buzelor sale;
\par 22 El începu sa mearga dintr-o data dupa ea, ca un bou la junghiere ?i ca un cerb care se zore?te spre capcana,
\par 23 Pâna când o sageata îi strapunge ficatul; dupa cum o pasare grabe?te spre la? ?i nu-?i da seama ca acolo î?i sfâr?e?te via?a.
\par 24 ?i acum, fiule, asculta-ma ?i ia aminte la cuvintele gurii mele!
\par 25 Inima ta sa nu se plece spre caile ei ?i nu te rataci pe potecile ei,
\par 26 Caci ea a ranit pe mul?i ?i pe foarte mul?i i-a omorât.
\par 27 Casa ei sunt caile iadului, care duc la camarile mor?ii.

\chapter{8}

\par 1 Oare în?elepciunea nu striga ea ?i priceperea nu-?i ridica glasul sau?
\par 2 Pe vârfurile cele mai înalte, pe cale, la raspântiile drumurilor sta,
\par 3 Pe lânga por?i, în împrejurimile ceta?ii, la intrarea por?ilor, striga tare:
\par 4 "Catre voi, oamenilor, se îndreapta strigatul meu ?i glasul meu catre voi, fii ai oamenilor.
\par 5 Voi, cei simpli, înva?a?i cumin?enia ?i voi, cei nebuni, în?elep?i?i-va!
\par 6 Asculta?i, caci voi spune lucruri mare?e ?i buzele mele se deschid pentru a înal?a ceea ce este drept;
\par 7 Caci gura mea graie?te adevarul ?i buzele mele se dezgusta de faradelege.
\par 8 Toate graiurile gurii mele sunt întru dreptate, în ele nu este nimic sucit ?i fara rost;
\par 9 Toate sunt lamurite pentru cel priceput ?i drepte pentru cei ce au aflat ?tiin?a.
\par 10 Lua?i înva?atura mea mai degraba decât argintul ?i ?tiin?a mai mult decât aurul cel mai curat,
\par 11 Caci în?elepciunea este mai buna decât pietrele pre?ioase ?i nici lucrurile cele mai pre?ioase nu au valoarea ei.
\par 12 Eu, în?elepciunea, locuiesc împreuna cu prevederea ?i stapânesc ?tiin?a ?i buna-chibzuiala.
\par 13 Frica de Dumnezeu este urgisirea raului. Mândria ?i obraznicia, calea rauta?ii ?i gura cea apriga le urasc eu.
\par 14 Al meu este sfatul ?i buna-chibzuiala, eu sunt priceperea, a mea este puterea.
\par 15 Prin mine împara?esc împara?ii ?i principii rânduiesc dreptatea.
\par 16 Prin mine cârmuiesc dregatorii ?i mai-marii sunt judecatorii pamântului.
\par 17 Eu iubesc pe cei ce ma iubesc ?i cei ce ma cauta ma gasesc.
\par 18 Cu mine este boga?ia ?i marirea, averea vrednica de cinste ?i dreptatea.
\par 19 Rodul meu e mai bun decât aurul ?i decât aurul cel mai curat, ?i ceea ce vine de la mine este mai de pre? decât argintul lamurit.
\par 20 Merg pe calea drepta?ii, în mijlocul cailor judeca?ii drepte,
\par 21 Ca sa dau celor ce ma iubesc boga?ii ?i sa le umplu camarile lor.
\par 22 Domnul m-a zidit la începutul lucrarilor Lui; înainte de lucrarile Lui cele mai de demult.
\par 23 Eu am fost din veac întemeiata de la început, înainte de a se fi facut pamântul.
\par 24 Nu era adâncul atunci când am fost nascuta, nici chiar izvoare încarcate cu apa.
\par 25 Înainte de a fi fost întemeia?i mun?ii ?i înaintea vailor eu am luat fiin?a.
\par 26 Când înca nu era facut pamântul, nici câmpiile, nici cel dintâi fir de praf din lume,
\par 27 Când El a întemeiat cerurile eu eram acolo; când El a tras bolta cerului peste fa?a adâncului,
\par 28 Când a întarit norii sus ?i izvoarele adâncului curgeau din bel?ug,
\par 29 Când El a pus hotar marii, ca apele sa nu mai treaca peste ?armuri ?i când El a a?ezat temeliile pamântului,
\par 30 Atunci eu eram ca un copil mic alaturi de El, veselindu-ma în fiecare zi ?i desfatându-ma fara încetare în fa?a Lui;
\par 31 Dezmierdându-ma pe rotundul pamântului Lui ?i gasindu-mi placerea printre fiii oamenilor.
\par 32 ?i acum, fiilor, asculta?i-ma! Ferici?i sunt cei ce pazesc caile mele!
\par 33 Asculta?i înva?atura, ca sa ajunge?i în?elep?i, ?i nu o lepada?i.
\par 34 Fericit este omul care asculta de mine ?i vegheaza în fiecare zi la por?ile mele ?i cel ce strajuie?te lânga pragul casei mele!
\par 35 Cel ce ma afla, a aflat via?a ?i dobânde?te har de la Domnul;
\par 36 Iar cel ce pacatuie?te împotriva mea î?i pagube?te via?a lui. To?i cei ce ma urasc pe mine iubesc moartea".

\chapter{9}

\par 1 În?elepciunea ?i-a zidit casa rezemata pe ?apte stâlpi,
\par 2 A înjunghiat vite pentru ospa?, a pregatit vinul cu mirodenii ?i a întins masa sa.
\par 3 Ea a trimis slujnicele sale sa strige pe vârfurile dealurilor ceta?ii:
\par 4 "Cine este neîn?elept sa intre la mine!" ?i celor lipsi?i de buna-chibzuiala le zice:
\par 5 "Veni?i ?i mânca?i din pâinea mea ?i be?i din vinul pe care eu l-am amestecat cu mirodenii.
\par 6 Parasi?i neîn?elepciunea ca sa ramâne?i cu via?a ?i umbla?i pe calea cea dreapta a priceperii!"
\par 7 Cel ce cearta pe batjocoritor î?i atrage dispre?ul, ?i cel ce dojene?te pe cel fara de lege î?i atrage ocara.
\par 8 Nu certa pe cel batjocoritor ca sa nu te urasca; dojene?te pe cel în?elept, ?i el te va iubi.
\par 9 Da sfat celui în?elept, ?i el se va face ?i mai în?elept; înva?a pe cel drept, ?i el î?i va spori ?tiin?a lui.
\par 10 Începutul în?elepciunii este frica de Dumnezeu ?i priceperea este ?tiin?a Celui Sfânt.
\par 11 Caci prin Domnul se vor înmul?i zilele tale ?i se vor adauga ?ie ani de via?a.
\par 12 Daca tu e?ti în?elept, e?ti în?elept pentru tine, ?i daca e?ti batjocoritor, singur vei purta ponosul.
\par 13 Nebunia este o femeie galagioasa, proasta ?i care nu ?tie nimic.
\par 14 Ea sta la u?a casei sale, pe un scaun înalt ?i striga,
\par 15 Ca sa pofteasca pe cei ce trec pe drum ?i pe cei ce merg pe calea lor fara sa se abata:
\par 16 "Cine este neîn?elept sa intre la mine!" ?i celui lipsit de buna-chibzuiala îi zice:
\par 17 "Apa furata e mai placuta ?i pâinea mâncata pe furi? are gust mai bun".
\par 18 ?i omul nu ?tie ca acolo sunt numai umbre, iar cei pe care îi pofte?te nebunia se afla de mult în adâncurile ?eolului (locuin?a mor?ilor).

\chapter{10}

\par 1 Pildele lui Solomon. Fiul în?elept învesele?te pe tatal sau, iar cel nebun este supararea maicii lui.
\par 2 Nu sunt de nici un folos comorile dobândite prin faradelege; numai dreptatea scapa de la moarte.
\par 3 Domnul nu lasa sa piara de foame sufletul celui drept; însa el respinge lacomia celor fara de lege.
\par 4 Mâna lene?ilor pricinuie?te saracie, iar mâna celor în?elep?i aduna avu?ii.
\par 5 Cel ce aduna în timpul verii este un om prevazator, iar cel care doarme în vremea seceri?ului este de ocara.
\par 6 Binecuvântarea Domnului vine pe capul celui drept, iar ocara acopera fa?a celor fara de lege.
\par 7 Pomenirea celui drept este spre binecuvântare, iar numele celor nelegiui?i va fi blestemat.
\par 8 Cel cu inima în?eleapta prime?te sfaturile, iar cel nebun graie?te vorbe spre pieirea lui.
\par 9 Cel ce umbla întru neprihanire umbla pe cale sigura, iar cel ce umbla pe cai laturalnice va fi dat de gol.
\par 10 Cel ce clipe?te din ochi va fi pricina de suparare; iar cel care cearta cu inima buna a?aza pacea.
\par 11 Izvor de via?a este gura celui drept, dar gura celor fara de lege, izvor de nedreptate.
\par 12 Ura aduce cearta, iar dragostea acopere toate cusururile.
\par 13 Pe buzele omului priceput se afla în?elepciunea; toiagul este pentru spatele celui lipsit de chibzuin?a.
\par 14 Cei în?elep?i ascund ?tiin?a, iar gura celui fara de socotin?a este o nenorocire apropiata.
\par 15 Avu?ia este pentru cel bogat o cetate întarita; nenorocirea celor sarmani este saracia lor.
\par 16 Agonisita celui drept este spre via?a; roadele celui fara de lege spre pacat;
\par 17 Cel ce paze?te înva?atura apuca pe calea vie?ii, iar cel ce leapada certarea ratace?te.
\par 18 Cel care ascunde ura are buze mincinoase; cel ce raspânde?te defaimarea este un nebun.
\par 19 Mul?imea cuvintelor nu scute?te de pacatuire, iar cel ce-?i ?ine buzele lui este un om în?elept.
\par 20 Limba omului drept este argint curat, dar inima celor fara de lege este lucru de pu?in pre?.
\par 21 Buzele celui drept calauzesc pe mul?i oameni, iar cei nebuni mor din pricina ca nu sunt pricepu?i.
\par 22 Numai binecuvântarea Domnului îmboga?e?te, iar truda zadarnica nu aduce spor.
\par 23 Ca o pricina de bucurie este pentru nebun savâr?irea unei fapte ru?inoase; la fel este cu în?elepciunea pentru omul priceput.
\par 24 De ceea ce se teme cel nelegiuit nu scapa, iar cererea celor drep?i (Dumnezeu) o împline?te.
\par 25 Precum trece furtuna, a?a piere ?i cel fara de lege, iar dreptul este ca o temelie neclintita.
\par 26 Precum este o?etul pentru din?i ?i fumul pentru ochi, a?a este omul lene? pentru cei ce-l pun la treaba.
\par 27 Frica de Dumnezeu lunge?te zilele (omului), iar anii celor fara de lege sunt pu?ini.
\par 28 Nadejdea celor drep?i este numai bucurie, iar nadejdea celor pacato?i sfâr?e?te în rau.
\par 29 Calea Domnului este o întaritura pentru cel desavâr?it ?i o prabu?ire pentru cei ce savâr?esc faradelegi.
\par 30 Niciodata cel drept nu se va clatina, iar cei nelegiui?i nu vor locui pamântul.
\par 31 Gura celui drept rode?te în?elepciune, iar limba urzitoare de rele aduce pierzare.
\par 32 Buzele celui drept cunosc bunavoirea, iar gura pacato?ilor strâmbatatea.

\chapter{11}

\par 1 Cântarul strâmb este urgisit de Domnul, ?i cântarirea dreapta este placerea Lui.
\par 2 Daca vine mândria, va veni ?i ocara, iar în?elepciunea este cu cei smeri?i.
\par 3 Neprihanirea poarta pe cei drep?i, iar strâmbatatea prapade?te pe cei vicleni.
\par 4 La nimic nu folose?te boga?ia în ziua mâniei; numai dreptatea izbave?te de moarte.
\par 5 Dreptatea neteze?te calea celui fara prihana, iar cel fara de lege va cadea prin faradelegea lui.
\par 6 Dreptatea izbave?te pe cei drep?i, iar cei vicleni vor fi prin?i prin pofta lor.
\par 7 La moartea omului drept ramâne nadejdea, iar la moartea celui pacatos piere nadejdea.
\par 8 Dreptul scapa din strâmtorare, ?i cel fara de lege îi ia locul.
\par 9 Faptuitorul de rele prabu?e?te cu gura pe aproapele lui, iar prin ?tiin?a celor drep?i va fi mântuit.
\par 10 De propa?irea celor drep?i cetatea se bucura, iar când pier cei fara de lege ea tresalta de veselie.
\par 11 Prin binecuvântarea oamenilor drep?i cetatea merge înainte, iar prin gura celor nelegiui?i ajunge ruina.
\par 12 Cel nepriceput urgise?te pe aproapele lui, iar omul cu buna-chibzuiala tace.
\par 13 Graitorul de rele da pe fa?a lucruri de taina, iar omul cu duhul cumpanit le ?ine ascunse.
\par 14 Unde lipse?te cârmuirea, poporul cade; izbavirea sta în mul?imea sfetnicilor.
\par 15 Celui ce se pune cheza? pentru un strain îi merge rau; cel ce nu se pune cheza? sta la adapost.
\par 16 Femeia cu purtare buna agonise?te cinstire, iar cea care ura?te cinstea e o ru?ine. Nu lene?ii ci silitorii agonisesc avere.
\par 17 Omul milostiv î?i face bine sufletului sau, pe când cel fara mila î?i chinuie?te trupul sau.
\par 18 Cel nelegiuit capata un câ?tig în?elator, iar cel ce seamana dreptatea, o rasplata adevarata.
\par 19 Cel ce umbla dupa dreptate ajunge la via?a, iar cel ce fuge dupa rau, la moarte.
\par 20 Pe cei cu inima vicleana îi urgise?te Domnul; placerea Lui este spre cei fara prihana.
\par 21 Încetul cu încetul pacatosul nu va ramâne nepedepsit, iar neamul celor drep?i va fi mântuit.
\par 22 Inel de aur în râtul porcului, a?a este femeia frumoasa ?i fara minte.
\par 23 Dorin?a celor drep?i este bine; nadejdea celor fara de lege este mânia lui Dumnezeu.
\par 24 Unul da mereu ?i se îmboga?e?te, altul se zgârce?te afara din cale ?i sarace?te.
\par 25 Cel ce binecuvinteaza va fi îndestulat, iar cel ce blesteama va fi blestemat.
\par 26 Cel ce ?ine grâul este blestemat de popor, iar binecuvântarea (se revarsa) peste capul celui ce îl vinde.
\par 27 Cel ce cauta binele dobânde?te bunavoin?a Domnului, iar cel ce umbla dupa rau va da peste el.
\par 28 Cel ce-?i pune nadejdea în boga?ia lui se ve?teje?te, iar cei drep?i ca frunzi?ul odraslesc.
\par 29 Cine î?i tulbura casa lui culege vânt, iar cel nebun va fi sluga celui în?elept.
\par 30 Rodul drepta?ii este un pom al vie?ii, iar silnicia nimice?te via?a.
\par 31 Daca cel drept este rasplatit pe pamânt, cu cât mai mult cel nelegiuit ?i pacatos!

\chapter{12}

\par 1 Cel ce iube?te înva?atura iube?te ?tiin?a, iar cel ce ura?te certarea este nebun.
\par 2 Cel bun dobânde?te har de la Domnul, iar pe omul viclean îl osânde?te Domnul.
\par 3 Omul nu se întare?te întru faradelegea lui; radacina celor drep?i nu se va clatina niciodata.
\par 4 Femeia virtuoasa este o cununa pentru barbatul ei, iar femeia fara cinste este un cariu în oasele lui.
\par 5 Socotelile celor drep?i sunt dreptatea, iar punerile la cale ale celor nelegiui?i în?elaciunea.
\par 6 Graiurile celor nelegiui?i sunt curse de moarte, iar gura celor drep?i îi scapa pe ei din primejdie.
\par 7 Cei fara de lege numai cât se întorc ?i nu mai sunt, dar casa drep?ilor dainuie?te de-a pururi.
\par 8 Omul aste pre?uit dupa priceperea lui, iar cel nepriceput este urgisit.
\par 9 Mai mult pre?uie?te un om smerit dar harnic, decât unul mândru dar lipsit de pâine.
\par 10 Cel drept are mila de vite, iar inima celui rau este fara îndurare.
\par 11 Cel ce munce?te ogorul sau se satura de pâine, iar cel ce umbla dupa de?ertaciuni este om lipsit de minte.
\par 12 Nelegiuitul pofte?te prada celor rai, dar radacina celor drep?i da rodul sau.
\par 13 Prin pacatul buzelor se prinde în la? pacatosul, iar dreptul (prin dreptatea lui) scapa din strâmtorare.
\par 14 Din rodul gurii sale se satur; de cele bune omul, ?i fiecaruia i se rasplate?te dupa faptele lui.
\par 15 Calea celui nebun este dreapta în ochii lui, iar cel în?elept asculta de sfat.
\par 16 Nebunul da pe fa?a îndata mânia lui, iar omul prevazator î?i ascunde ocara.
\par 17 Cel ce spune adevarul veste?te dreptatea, iar martorul mincinos umbla cu în?elaciunea.
\par 18 Cei nechibzui?i la vorba sunt ca împunsaturile de sabie, pe când limba celor în?elep?i aduce tamaduire.
\par 19 Buzele care spun adevarul vor dainui totdeauna, iar limba graitoare de minciuna numai pentru o clipa.
\par 20 În?elaciunea este în inima celor ce gândesc rau, iar bucuria pentru cei ce dau sfaturi de pace.
\par 21 Nici o nenorocire nu se întâmpla celui drept, pe când cei nelegiui?i sunt covâr?i?i de rele.
\par 22 Buzele cele graitoare de minciuna sunt urâciune înaintea Domnului, iar cei ce faptuiesc dupa adevar sunt placerea Lui.
\par 23 Omul în?elept î?i ascunde ?tiin?a, pe când inima celor nebuni propovaduie?te nebunia.
\par 24 Mâna celor silitori va stapâni, iar cea lasatoare va fi birnica.
\par 25 Supararea se abate asupra omului, dar numai un cuvânt bun îl bucura.
\par 26 Dreptul cerceteaza cu de-amanuntul pe prietenul sau; calea celor nelegiui?i duce la ratacire.
\par 27 Lene?ul nu-?i frige nici vânatul lui; cea mai scumpa comoara pentru om este munca.
\par 28 Pe cararea drepta?ii este via?a ?i pe calea pe care ea o însemneaza; nemurirea, iar calea nebuniei duce la moarte.

\chapter{13}

\par 1 Fiul în?elept asculta de înva?atura tatalui sau, iar cel batjocoritor nici de mustrare.
\par 2 Din rodul gurii sale omul manânca binele; pofta celor vicleni este silnicia.
\par 3 Cine î?i paze?te gura î?i paze?te sufletul sau; cel ce deschide prea tare buzele o face spre pieirea lui.
\par 4 Sufletul celui lene? pofte?te, însa în zadar. Numai sufletul celor silitori este îndestulat.
\par 5 Dreptul ura?te cuvintele mincinoase; ticalosul aduce numai ru?ine ?i ocara.
\par 6 Dreptatea paze?te calea omului fara prihana, iar faradelegea e pricina ruinii celui pacatos.
\par 7 Unii se dau drept boga?i ?i n-au nimic, al?ii trec drept saraci, cu toate ca au multe averi.
\par 8 Boga?ia cuiva sluje?te la rascumpararea lui; cel sarac nu se teme nici chiar de amenin?are.
\par 9 Lumina celor drep?i lumineaza, pe când sfe?nicul celor fara de lege se stinge.
\par 10 Mândria nu da prilej decât la cearta, în?elepciunea se afla numai la cei ce primesc sfaturi.
\par 11 Boga?ia adunata în graba se împu?ineaza, numai cel ce-o aduna pe încetul o înmul?e?te.
\par 12 A?teptarea prea îndelungata îmbolnave?te inima, iar dorin?a împlinita este pom al vie?ii.
\par 13 Cel ce nu ia în seama cuvântul (lui Dumnezeu) este dat pierzarii, iar cel ce se teme de porunca Lui este rasplatit.
\par 14 Înva?atura celui în?elept este izvor de via?a, ca sa putem scapa de cursele mor?ii.
\par 15 Buna în?elegere rode?te har; calea celor vicleni este spre pieirea lor.
\par 16 Orice om în?elept lucreaza cu chibzuin?a, numai cel nebun î?i desfa?oara nebunia.
\par 17 Un sol ticalos cade în nenorocire, iar unul credincios aduce alinare.
\par 18 De saracie ?i de ru?ine are parte cel ce nesocote?te certarea, iar cel ce prime?te mustrarea va fi cinstit.
\par 19 Dorin?a împlinita mul?ume?te sufletul, iar departarea de rau este urâciune pentru cei nebuni.
\par 20 Cel ce se înso?e?te cu cei în?elep?i ajunge în?elept, iar cel ce se întovara?e?te cu cei nebuni se face rau.
\par 21 Nenorocirea urmare?te pe cei pacato?i, iar fericirea rasplate?te pe cei drep?i.
\par 22 Omul bun lasa mo?tenirea sa nepo?ilor sai, iar averea celui pacatos este sortita pentru cei drep?i.
\par 23 Aratura în ?elina, facuta de cei saraci, da hrana din bel?ug, insa averea se pierde din pricina nedrepta?ii.
\par 24 Cine cru?a toiagul sau î?i ura?te copilul, iar cel care îl iube?te îl cearta la vreme.
\par 25 Dreptul manânca ?i î?i îndestuleaza sufletul sau, iar pântecele celor fara de lege duce lipsa.

\chapter{14}

\par 1 Femeile în?elepte zidesc casa, iar cele nebune o darâma cu mâna lor.
\par 2 Cel ce umbla întru dreptate se teme de Domnul, iar cel ce umbla pe cai strâmbe îl dispre?uie?te.
\par 3 În gura celui nebun este varga mândriei lui; buzele pe cei în?elep?i îi pazesc.
\par 4 Unde nu sunt boi, staulul este gol, însa puterea boilor aduce mult folos.
\par 5 Martorul care graie?te adevarul nu minte, iar martorul mincinos spune numai minciuni.
\par 6 Batjocoritorul cauta în?elepciunea ?i nu o gase?te, iar pentru cel priceput ?tiin?a este u?oara.
\par 7 Fugi dinaintea omului fara de minte, caci tu ?tii ca nu este nici o ?tiin?a pe buzele lui.
\par 8 În?elepciunea omului chibzuit este de a-?i în?elege calea lui; iar nebunia celor neîn?elep?i este în?elaciune.
\par 9 Nebunul î?i bate joc de jertfa pentru pacat, însa între oamenii drep?i este buna în?elegere.
\par 10 Inima cunoa?te amaraciunile sale, iar un strain nu poate împar?i bucuriile ei.
\par 11 Casa celor fara de lege va fi distrusa, iar cortul celor drep?i va înflori.
\par 12 Unele cai par drepte în ochii omului, dar sfâr?itul lor sunt caile mor?ii.
\par 13 Chiar când râdem, inima se întristeaza; bucuria se sfâr?e?te prin plângere.
\par 14 Nelegiuitul se va satura de caile sale ?i omul bun de roadele sale.
\par 15 Omul simplu crede toate vorbele; omul în?elept vegheaza pa?ii sai.
\par 16 În?eleptul se teme ?i se fere?te de rau, iar cel fara de minte î?i iese din fire ?i se simte la adapost.
\par 17 Cel iute la mânie savâr?e?te nebunii, iar cel cumpanit se stapâne?te.
\par 18 Cei nepricepu?i au parte de nebunie, pe când cei în?elep?i sunt încununa?i cu ?tiin?a.
\par 19 Cei rai se pleaca înaintea celor buni ?i cei nelegiui?i stau la por?ile celor drep?i.
\par 20 Saracul este dispre?uit chiar ?i de prietenul lui, pe când prietenii celui bogat sunt fara de numar.
\par 21 Cel care nu baga în seama pe prietenul sau savâr?e?te un pacat; iar cel ce se îndura de sarmani e fericit.
\par 22 Cu adevarat ratacesc cei ce planuiesc faradelegea, iar cei ce cugeta la lucruri bune au parte de milostivire ?i de adevar.
\par 23 Orice osteneala duce la îndestulare, iar cuvintele fara rost la lipsa.
\par 24 Boga?ia este o cununa pentru cei în?elep?i; iar coroana color nebuni este nebunia.
\par 25 Martorul drept scapa suflete, iar cel viclean spune numai minciuni.
\par 26 Întru frica lui Dumnezeu este nadejdea celui tare; fiii lui afla-vor (acolo) un liman.
\par 27 Frica de Dumnezeu este un izvor de via?a, ca sa putem scapa de cursele mor?ii.
\par 28 Stralucirea unui rege se sprijina pe mul?imea poporului, iar lipsa de supu?i este pieirea prin?ului.
\par 29 Cel încet la mânie este bogat în în?elepciune, iar cel ce se mânie degraba î?i da pe fa?a nebunia.
\par 30 O inima fara patima este via?a trupului, pe când pornirea patima?a este ca un cariu în oase.
\par 31 Cel care apasa pe cel sarman defaima pe Ziditorul lui, dar cel ce are mila de sarac Îl cinste?te.
\par 32 Cel fara de lege este rasturnat de rautatea lui, iar cel drept gase?te scapare în neprihanirea lui.
\par 33 În?elepciunea sala?luie?te în inima celui în?elept, iar în inima celor nebuni nu se arata.
\par 34 Dreptatea înal?a un popor, în vreme ce pacatul este ocara popoarelor.
\par 35 Bunavoin?a regelui este pentru sluga în?eleapta, iar mânia lui pentru cel ce îi face ru?ine.

\chapter{15}

\par 1 Un raspuns blând domole?te mânia, iar un cuvânt aspru a?â?a mânia.
\par 2 Limba celor în?elep?i picura ?tiin?a, ?ar gura celor nebuni revarsa prostie.
\par 3 Ochii Domnului sunt pretutindeni, veghind asupra celor buni ?i asupra celor rai.
\par 4 Limba dulce este pom al vie?ii, iar limba vicleana zdrobe?te inima.
\par 5 Nebunul nu ia în seama înva?atura tatalui sau, iar cine trage folos din certare se face mai în?elept.
\par 6 În casa celui drept sunt comori fara de numar; în câ?tigul celui fara de lege este tulburare.
\par 7 Buzele celor în?elep?i raspândesc ?tiin?a, dar inima celor nebuni nu.
\par 8 Jertfa celor fara de lege este urâciune înaintea Domnului, iar rugaciunea celor drep?i este placerea Lui.
\par 9 Calea celui nelegiuit este urâciune înaintea Domnului, dar El iube?te pe cel ce umbla dupa dreptate.
\par 10 O certare aspra capata ,el ce parase?te cararea; cel ce urgise?te mustrarea va muri.
\par 11 Iadul ?i adâncul sunt cunoscute Domnului, cu atât mai vârtos inimile fiilor oamenilor.
\par 12 Celui batjocoritor nu-i place dojana; de aceea el nu se îndreapta spre cei în?elep?i.
\par 13 O inima vesela însenineaza fa?a, iar când inima e trista ?i duhul e fara de curaj.
\par 14 Inima în?eleapta cauta ?tiin?a, iar gura celor nebuni se simte mul?umita cu nebunia.
\par 15 Toate zilele celui sarac sunt rele, dar inima mul?umita este un ospa? necurmat.
\par 16 Mai bine pu?in întru frica lui Dumnezeu, decât vistierie mare ?i tulburare (multa).
\par 17 Mai mult face o mâncare de verde?uri ?i cu dragoste, decât un bou îngra?at ?i cu ura.
\par 18 Omul mânios a?â?a cearta, pe când cel domol lini?te?te aprinderea.
\par 19 Calea celui lene? e ca un gard de spini, iar calea celui silitor e neteda.
\par 20 Fiul în?elept bucura pe tatal sau, iar fiul nebun nu baga în seama pe maica lui.
\par 21 Nebunia este o bucurie pentru omul fara minte, iar cel în?elept merge pe calea dreapta.
\par 22 Punerile la cale nu se înfaptuiesc unde lipse?te chibzuirea, dar ele î?i iau fiin?a cu mul?i sfatuitori.
\par 23 Omul se bucura pentru un raspuns bun ie?it din gura lui ?i cât e de buna vorba spusa la locul ei!
\par 24 În?eleptul merge pe cararea vie?ii ce duce în sus, ca sa ocoleasca drumul iadului care merge în jos.
\par 25 Domnul prabu?e?te casa celor mândri ?i întare?te hotarul vaduvei.
\par 26 Gândurile cele rele sunt urâciune înaintea Domnului, iar cuvintele frumoase sunt curate (în ochii Lui).
\par 27 Cel ce umbla dupa câ?tig nedrept î?i surpa casa lui, iar cel ce ura?te mita va trai.
\par 28 Inima celui drept chibzuie?te ce sa raspunda, iar gura celor nelegiui?i împra?tie rauta?i.
\par 29 Domnul se ?ine departe de cei nelegiui?i, dar asculta rugaciunea celor drep?i.
\par 30 O privire binevoitoare învesele?te inima ?i o veste buna întare?te oasele.
\par 31 Urechea care asculta o dojana folositoare vie?ii î?i are loca?ul printre cei în?elep?i.
\par 32 Cel ce leapada mustrarea î?i urgise?te sufletul sau, iar cel ce ia aminte la dojana dobânde?te în?elepciune.
\par 33 Frica de Dumnezeu este înva?atura ?i în?elepciune, iar smerenia trece înaintea maririi.

\chapter{16}

\par 1 În putere sta omului sa plasmuiasca planuri în inima, dar raspunsul limbii vine de la Domnul.
\par 2 Toate caile omului sunt curate în ochii lui, dar numai Domnul este Cel ce cerceteaza duhul.
\par 3 Înfa?i?eaza Domnului lucrarile tale ?i gândurile tale vor izbuti.
\par 4 Pe toate le-a facut Domnul fiecare cu ?elul sau, la fel ?i pe nelegiuit pentru ziua nenorocirii.
\par 5 Toata inima semea?a este urâciune înaintea Domnului; hotarât, ea nu va ramâne nepedepsita.
\par 6 Prin iubire ?i credin?a se ispa?e?te pacatul ?i prin frica de Dumnezeu te fere?ti de rele.
\par 7 Când caile omului sunt placute înaintea Domnului, chiar ?i pe vrajma?ii lui îi sile?te la pace.
\par 8 Mai degraba pu?in ?i cu dreptate, decât agoniseala multa cu strâmbatate.
\par 9 Inima omului gânde?te la calea lui, dar numai Domnul poarta pa?ii lui.
\par 10 Hotarâri dumnezeie?ti sunt pe buzele împaratului; la darea hotarârii sa nu se în?ele gura lui!
\par 11 Cântarul ?i cumpana dreapta sunt de la Domnul; toate greuta?ile de cântarit sunt lucrarea Lui.
\par 12 Urâciune sunt regii care faptuiesc faradelegea, fiindca numai întru dreptate se întare?te tronul.
\par 13 Buzele graitoare de dreptate sunt placute regilor ?i iubesc pe cel ce spune drept.
\par 14 Întarâtarea regelui este ca vestitorii mor?ii, dar omul în?elept o domole?te.
\par 15 Seninatatea fe?ei regelui da via?a ?i bunavoin?a lui este ca un nor de ploaie de primavara.
\par 16 Dobândirea în?elepciunii este mai buna decât aurul, iar câ?tigarea priceperii este mai de pre? decât argintul.
\par 17 Calea celor drep?i este ferirea de rau; numai acela care ia aminte la mersul lui î?i paze?te sufletul sau.
\par 18 Înaintea prabu?irii merge trufia ?i seme?ia înaintea caderii.
\par 19 Mai bine sa fii smerit cu cei smeri?i decât sa împar?i prada cu cei mândri.
\par 20 Cel care ia aminte la cuvânt afla fericirea, iar cel ce se încrede în Domnul este fericit.
\par 21 Cel ce este în?elept se cheama priceput; dulcea?a cuvintelor de pe buzele lui înmul?e?te ?tiin?a.
\par 22 În?elepciunea este un izvor de via?a pentru cine o are; pedeapsa celui nebun e nebunia.
\par 23 Inima celui în?elept da în?elepciune gurii lui ?i pe buzele sale spore?te ?tiin?a.
\par 24 Cuvintele frumoase sunt un fagure de miere, dulcea?a pentru suflet ?i tamaduire pentru oase.
\par 25 Multe cai i se par bune omului, dar la capatul lor încep caile mor?ii.
\par 26 Foamea îndeamna pe lucrator la munca, fiindca gura lui îl sile?te.
\par 27 Omul viclean pregate?te nenorocirea ?i pe buzele lui este ca un foc arzator.
\par 28 Omul cu gând rau a?â?a cearta ?i defaimatorul desparte p? prieteni.
\par 29 Omul asupritor amage?te pe prietenul sau ?i îl îndreapta pe un drum care nu este bun.
\par 30 Cel care închide din ochi urze?te viclenii; cine î?i mu?ca buzele a ?i savâr?it raul.
\par 31 Batrâne?ea este o cununa stralucita; ea se afla mergând pe calea cuvio?iei.
\par 32 Cel încet la mânie e mai de pre? decât un viteaz, iar cel ce î?i stapâne?te duhul este mai gre?uit decât cuceritorul unei ceta?i.
\par 33 Se arunca sor?ii în pulpana hainei, dar hotarârea toata vine de la Domnul.

\chapter{17}

\par 1 Mai buna este o bucata de pâine uscata în pace, decât o casa plina cu carne de jertfe, dar cu vrajba.
\par 2 Un slujitor în?elept e mai presus decât un fiu aducator de ru?ine; acela va împar?i mo?tenirea eu fra?ii.
\par 3 în topitoare se lamure?te argintul ?i în cuptor aurul, dar cel ce încearca inimile este Domnul.
\par 4 Facatorul de rele ia aminte la buzele nedrepte, mincinosul pleaca urechea la limba cea rea.
\par 5 Cel ce î?i bate joc de sarac defaima pe Ziditorul lui; cel ce se bucura de o nenorocire nu va ramâne nepedepsit.
\par 6 Cununa batrânilor sunt nepo?ii, iar marirea fiilor sunt parin?ii lor.
\par 7 Nebunului nu-i sunt dragi cuvintele alese, cu atât mai mult unui om de seama vorbele mincinoase.
\par 8 Piatra nestemata este darul în ochii celui ce-l are; oriunde se întoarce (totul) îi merge în plin.
\par 9 Cel ce acopera un pacat cauta prietenie, iar cine scoate la iveala un lucru (uitat) desparte pe prieteni.
\par 10 Certarea înrâure?te mai adânc pe omul în?elept, decât o suta de lovituri pe cel nebun.
\par 11 Omul rau a?â?a razvratirea; pentru aceasta un sol aprig va fi trimis împotriva lui.
\par 12 Mai degraba sa întâlne?ti o ursoaica lipsita de puii ei, decât un nebun în nebunia lui.
\par 13 Cel ce rasplate?te cu rau pentru bine nu va vedea departându-se nenorocirea din casa lui.
\par 14 Începutul unei certe e ca slobozirea apei (dintr-un iezer); înainte de a se aprinde, da-te la o parte!
\par 15 Cel care achita pe cel vinovat ?i cel ce osânde?te pe cel drept, amândoi sunt urâciune înaintea Domnului.
\par 16 Ce folos aduc banii în mâna celui nebun? Ar putea dobândi în?elepciune, dar nu are pricepere.
\par 17 Prietenul iube?te în orice vreme, iar în nenorocire el e ca un frate.
\par 18 Omul fara pricepere se prinde prin darnicia mâinii lui; el se pune cheza? pentru aproapele lui.
\par 19 Cine iube?te certurile iube?te pacatul; cel ce ridica glasul î?i iube?te ruina.
\par 20 Cel ce are o inima vicleana nu afla fericirea ?i cel ce are o limba ?ireata da peste necaz.
\par 21 Cel ce da na?tere unui nebun va avea o mare suparare ?i parintele nu are nici o bucurie pentru un fiu sarit din minte.
\par 22 O inima vesela este un leac minunat, pe când un duh fara curaj usuca oasele.
\par 23 Nelegiuitul prime?te daruri scoase (pe ascuns) din sân, ca sa abata cararile drepta?ii.
\par 24 Omul priceput are înaintea ochilor lui în?elepciunea, iar ochii celui nebun se uita la capatul pamântului.
\par 25 Feciorul nebun este necaz pentru tatal lui ?i amaraciune pentru maica lui.
\par 26 Nu se cuvine sa pui la plata pe omul drept, nici sa osânde?ti pe cei nevinova?i din pricina ca umbla drept.
\par 27 Cel ce î?i stapâne?te cuvintele sale are ?tiin?a ?i cel ce î?i ?ine cumpatul este un om priceput.
\par 28 Chiar ?i nebunul, când tace, trece drept în?elept; când închide gura este asemenea unui om cuminte.

\chapter{18}

\par 1 Cel ce se rine deoparte cauta sa-?i mul?umeasca pornirea patima?a; împotriva oricarui sfat în?elept se porne?te.
\par 2 Celui nebun nu-i place în?elepciunea, ci darea pe fa?a a gândurilor lui.
\par 3 Când vine cel nelegiuit, vine ?i defaimarea ?i o data cu ru?inea ?i batjocura.
\par 4 Vorbele (ie?ite) din gura omului sunt ape fara fund; izvorul în?elepciunii este un ?uvoi care da peste maluri.
\par 5 Nu este bine sa cau?i la fa?a celui fara de lege ?i sa nu faci dreptate celui drept la judecata.
\par 6 Buzele celui nebun duc la cearta ?i gura lui da na?tere la ocari!
\par 7 Gura celui nebun este prabu?irea lui ?i buzele lui sunt un la? pentru sufletul lui.
\par 8 Vorbele defaimatorului sunt ca ni?te mâncari alese; ele coboara în camarile pântecelui.
\par 9 Omul lasator pentru lucrul lui e frate cu cel care darâma.
\par 10 Turn puternic este numele Domnului; cel drept la el î?i afla scaparea ?i este la adapost.
\par 11 Averea celui bogat este o cetate tare pentru el, iar în închipuirea lui ca un zid înalt.
\par 12 Înaintea prabu?irii vine trufia inimii, iar înaintea maririi merge umilin?a.
\par 13 Cel ce raspunde la vorba înainte de a fi ascultat-o este nebun ?i încurcat la minte.
\par 14 Curajul omului îl întare?te în vreme de suferin?a, iar pe un om lipsit de barba?ie, cine-l va ridica?
\par 15 O inima priceputa dobânde?te ?tiin?a ?i urechea celor în?elep?i umbla dupa iscusin?a.
\par 16 Darul adus de un om îi large?te (calea lui) ?i-l poarta înaintea celor mari.
\par 17 Pârâtul pare ca are dreptate în pricina sa, iar când vine pârâ?ul, atunci se ia la cercetare.
\par 18 Sor?ul face sa înceteze sfada ?i hotara?te între cei puternici.
\par 19 Frate ajutat de frate este ca o cetate tare ?i înalta ?i are putere ca o împara?ie întemeiata.
\par 20 Din rodul gurii omului se satura pântecele lui; din ceea ce da buzele lui se îndestuleaza.
\par 21 În puterea limbii este via?a ?i moartea ?i cei ce o iubesc manânca din rodul ei.
\par 22 Cel ce gase?te o femeie buna afla un lucru de mare pre? ?i dobânde?te dar de la Dumnezeu.
\par 23 Saracul vorbe?te rugator, iar cel bogat raspunde cu îndrazneala.
\par 24 Sunt prieteni aducatori de nenorocire; dar este ?i câte un prieten mai apropiat decât un frate.

\chapter{19}

\par 1 Mai de pre? este saracul care umbla întru neprihanirea lui, decât un bogat cu buze viclene ?i nebun.
\par 2 Ne?tiin?a sufletului nu este buna iar cel ce umbla repede da gre?.
\par 3 Nebunia omului darâma calea lui ?i inima lui se mânie împotriva Domnului.
\par 4 Boga?ia strânge prieteni fara de numar, iar saracul se desparte chiar de prietenul sau.
\par 5 Martorul mincinos nu va ramâne nepedepsit ?i cel ce spune lucruri neadevarate nu va scapa.
\par 6 Mul?i sunt cei ce lingu?esc pe un om darnic ?i to?i sunt prieteni ai celui ce da daruri.
\par 7 To?i fra?ii celui sarac îl urasc; cât de mult prietenii lui se departeaza de el! El cauta vorbe (mângâietoare), dar nu le afla.
\par 8 Cel ce dobânde?te în?elepciune iube?te sufletul sau ?i cel ce ?ine cu tarie la pricepere afla fericirea.
\par 9 Martorul mincinos nu ramâne nepedepsit ?i cel ce spune lucruri neadevarate se va prabu?i.
\par 10 Nu-i sta bine celui fara de minte sa traiasca în desfatari, cu atât mai pu?in unui rob sa conduca peste capetenii.
\par 11 În?elepciunea domole?te mânia omului ?i faima lui este iertarea gre?elilor.
\par 12 Furia unui rege e ca racnetul unui leu, iar bunavoin?a lui este ca roua pe iarba.
\par 13 Un fiu neascultator este nenorocirea tatalui lui, iar certurile unei femei un jgheab care curge întruna.
\par 14 O casa ?i o avere sunt mo?tenire de la parin?i, iar o femeie în?eleapta este un dar de la Dumnezeu.
\par 15 Lenea te face sa cazi în toropeala; sufletului trândav îi va fi foame.
\par 16 Cel ce ia seama la porunca î?i pastreaza sufletul sau, iar cel ce dispre?uie?te cuvântul (Domnului) va muri.
\par 17 Cel ce are mila de sarman împrumuta Domnului ?i El îi va rasplati fapta lui cea buna.
\par 18 Pedepse?te pe feciorul tau, cât mai este nadejde (de îndreptare), dar nu ajunge pâna acolo ca sa-l omori.
\par 19 Omul aprig la mânie trebuie sa fie pedepsit; daca îl cru?i o data, trebuie sa începi din nou.
\par 20 Asculta sfatul ?i prime?te înva?atura, ca sa fii în?elept toata via?a ta.
\par 21 Multe puneri la cale framânta inima omului, dar numai sfatul Domnului se împline?te.
\par 22 Omul se face placut prin marinimia lui; mai de pre? este un sarac bun decât un om mincinos.
\par 23 Frica de Dumnezeu duce la via?a ?i ne îndestulam fara sa fim lovi?i de nenorocire.
\par 24 Lene?ul întinde mâna în blid ?i nu are putere s-o duca la gura.
\par 25 Love?te pe cel ce batjocore?te ?i cel fara de minte va deveni în?elept; mustra pe cel în?elept ?i el va pricepe ?tiin?a.
\par 26 Cel ce se poarta rau cu tatal sau ?i alunga (din casa) pe mama sa, este fiu aducator de ocara ?i de ru?ine.
\par 27 Înceteaza, fiul meu, sa ascul?i ademenirea ?i sa te la?i îndepartat de înva?aturile în?elepte.
\par 28 Martorul de nimic î?i bate joc de dreptate ?i gura celor fara de lege înghite nelegiuirea.
\par 29 Pentru batjocoritori sunt gata toiege; loviturile sunt pentru spinarea celor nebuni.

\chapter{20}

\par 1 Un ocarâtor este vinul, un zurbagiu bautura îmbatatoare ?i oricine se lasa ademenit nu este în?elept.
\par 2 Groaza pe care o insufla regele este ca racnetul leului; cel ce îl întarâta pacatuie?te împotriva sa însu?i.
\par 3 Este o mare însu?ire pentru om sa se stapâneasca de la cearta ?i tot nebunul se întarâta.
\par 4 Toamna lene?ul nu lucreaza, iar când vine sa culeaga rodul la seceri?, nimic nu afla.
\par 5 O apa adânca este sfatul în inima omului, iar omul de?tept ?tie s-o scoata.
\par 6 Mul?i oameni se lauda cu marinimia, dar un prieten adevarat cine-l afla?
\par 7 Omul drept umbla pe calea lui fara prihana; ferici?i sunt copiii care vin dupa el!
\par 8 Un rege care sta pe scaunul de judecata deosebe?te cu ochii lui orice fapta rea.
\par 9 Cine poate spune: Cura?it-am inima mea; sunt curat de pacat?
\par 10 Doua feluri de greuta?i de cântarit ?i de masurat sunt urâciune înaintea Domnului.
\par 11 Copilul se da pe fa?a din lucrarile lui, daca purtarea lui este fara prihana ?i dreapta.
\par 12 Urechea care aude ?i ochiul care vede, pe amândoua le-a zidit Domnul.
\par 13 Nu iubi somnul, ca sa nu ajungi sarac; ?ine ochii deschi?i, numai a?a vei fi îndestulat de pâine.
\par 14 "Rau, rau!", zice cumparatorul, iar dupa ce pleaca se lauda.
\par 15 Chiar daca ai aur ?i pietre pre?ioase, dar o podoaba fara seaman sunt buzele chibzuite.
\par 16 Ia-i haina ?i fiindca s-a pus cheza? pentru altul în locul celor straini, ia-l zalog.
\par 17 Buna e la gust pâinea agonisita cu în?elaciune, dar dupa aceea gura se umple de pietricele.
\par 18 Planurile se întaresc prin sfaturi; lupta-te cu luare aminte.
\par 19 Cine tradeaza taina umbla ca un defaimator ?i nu te întovara?i cu cel ce are mereu buzele deschise.
\par 20 Cel ce blesteama pe tatal sau ?i pe mama sa stinge sfe?nicul în mijlocul întunericului.
\par 21 O mo?tenire repede câ?tigata de la început, la urma va fi fara binecuvântare.
\par 22 Nu spune: "Vreau sa rasplatesc cu rau!" Nadajduie?te în Domnul ?i El i?i va veni în ajutor.
\par 23 Greuta?ile nedrepte pentru cântarire sunt urâciune înaintea Domnului ?i cântarele în?elatoare nu sunt decât un lucru rau.
\par 24 De Domnul sunt hotarâ?i pa?ii omului, caci cum ar putea omul sa priceapa calea lui?
\par 25 O cursa este pentru om sa afieroseasca Domnului ceva în graba ?i dupa ce a fagaduit sa-i para rau.
\par 26 Un rege în?elept simte pe cei fara de lege ?i lasa sa treaca roata peste ei.
\par 27 Sufletul omului este un sfe?nic de la Domnul; el cerceteaza toate camarile trupului.
\par 28 Iubirea ?i credin?a pazesc pe rege ?i prin iubire î?i sprijina tronul sau.
\par 29 Faima celor tineri este puterea lor ?i podoaba celor batrâni parul lor carunt.
\par 30 Ranile sângeroase sunt un leac pentru cel raufacator ?i lovituri care patrund pâna înlauntrul trupului.

\chapter{21}

\par 1 Asemenea unui curs de apa este inima regelui în mâna Domnului, pe care îl îndreapta încotro vrea.
\par 2 Toata calea omului este dreapta în ochii lui, dar numai Domnul cântare?te inimile.
\par 3 Faptuirea drepta?ii ?i a judeca?ii este mai de pre? pentru Domnul decât jertfa sângeroasa.
\par 4 Ochii seme?i ?i inima îngâmfata sunt sfe?nicul pacato?ilor. Aceasta nu este decât pacat.
\par 5 Chibzuiala omului silitor duce numai la câ?tig, iar cel ce se zore?te ajunge la paguba.
\par 6 Comorile dobândite cu limba mincinoasa sunt de?ertaciune trecatoare ?i la?uri ale mor?ii.
\par 7 Silnicia celor fara de lege se ?ine dupa ei, caci nu voiesc sa înfaptuiasca dreptatea.
\par 8 Calea celui raufacator e sucita; cel nevinovat lucreaza drept.
\par 9 Mai degraba sa locuie?ti într-un col? pe acoperi?, decât cu o femeie certarea?a ?i într-o casa mare.
\par 10 Sufletul celui fara de lege pofte?te rautatea, iar aproapele lui nu afla mila în ochii lui.
\par 11 Când cel batjocoritor e pedepsit, cel fara minte se în?elep?e?te ?i când cel în?elept este dojenit, el câ?tiga în ?tiin?a.
\par 12 Dreptul ia aminte la casa celui nelegiuit. Dumnezeu prabu?e?te pe cei fara de lege în nenorocire.
\par 13 Cine î?i astupa urechea la strigatul celui sarman ?i el, când va striga, nu i se va raspunde.
\par 14 Un dar facut într-ascuns potole?te mânia ?i un plocon scos din sân, o mânie puternica.
\par 15 Dreptul tresalta de bucurie când poate sa puna în fapta dreptatea, iar spaima e pentru cei ce savâr?esc faradelegea.
\par 16 Un om care ratace?te de pe drumul în?elepciunii, se va odihni curând în adunarea celor mor?i.
\par 17 Cel ce iube?te veselia va duce lipsa; cel caruia îi place vinul ?i miresmele nu se îmboga?e?te.
\par 18 Nelegiuitul sluje?te ca pre? de rascumparare pentru cel drept ?i vicleanul pentru cel fara prihana.
\par 19 Mai bine sa locuie?ti în pustiu decât cu o femeie certarea?a ?i suparacioasa.
\par 20 Comori de pre? ?i untdelemn (se gasesc) în casa celui în?elept, dar omul cel nebun le risipe?te.
\par 21 Cel ce umbla în calea drepta?ii ?i a milei afla via?a, dreptate ?i marire.
\par 22 În?eleptul ia cu lupta crunta cetatea vitejilor ?i rastoarna întariturile în care ei î?i puneau nadejdea.
\par 23 Cel ce-?i paze?te gura ?i limba lui î?i paze?te sufletul lui de primejdie.
\par 24 Cine este seme? ?i îngâmfat se cheama batjocoritor; acela se poarta cu prisos de trufie.
\par 25 Pofta celui lene? îl omoara, caci mâinile nu voiesc sa lucreze.
\par 26 Mereu cel fara de lege pofte?te, iar cel drept da ?i nu se zgârce?te.
\par 27 Jertfa celor nelegiui?i este urâciune pentru Domnul, mai cu seama când o aduc pentru o fapta ru?inoasa.
\par 28 Martorul mincinos va pieri, iar omul care asculta va putea vorbi totdeauna.
\par 29 Raufacatorul are o privire neru?inata, iar omul cel drept î?i ia aminte la purtarea lui.
\par 30 Nu este nici în?elepciune, nici pricepere ?i nici sfat care sa aiba putere înaintea Domnului.
\par 31 Calul este gata pentru ziua de razboi, însa biruin?a vine de la Domnul.

\chapter{22}

\par 1 Un nume (bun) este mai de pre? decât boga?ia; cinstea este mai pre?ioasa decât argintul ?i decât aurul.
\par 2 Bogatul ?i saracul se întâlnesc unul cu altul; dar Cine i-a facut este Domnul.
\par 3 Cel iscusit vede nenorocirea ?i se ascunde, cei simpli trec mai departe ?i sufera.
\par 4 Rodul umilin?ei ?i a temerii de Dumnezeu sunt: boga?ia, marirea ?i via?a.
\par 5 Maracini ?i curse sunt în calea celui viclean; cel ce î?i fere?te sufletul lui sa da la o parte de ele.
\par 6 Deprinde pe tânar cu purtarea pe care trebuie s-o aiba; chiar când va îmbatrâni nu se va abate de la ea.
\par 7 Bogatul stapâne?te pe cei saraci, ?i cel ce împrumuta este slujitor celui de la care se împrumuta.
\par 8 Cel ce seamana nedreptatea secera nenorocire, iar toiagul mâniei lui îl va bate pe el.
\par 9 Omul blând va fi binecuvântat, caci din pâinea ui da celui sarac.
\par 10 Alunga pe cel batjocoritor ?i cearta va lua sfâr?it ?i pricina ?i defaimarea vor înceta.
\par 11 Cel ce iube?te cura?ia inimii ?i ale carui buze sunt ( pline) de vorbe alese are de prieten pe conducator.
\par 12 Ochii Domnului pazesc ?tiin?a ?i darâma cuvintele celui fara de lege.
\par 13 Cel lene? pune pricini ?i zice: "Afara este un leu, a? putea sa fiu sugrumat în mijlocul uli?elor".
\par 14 O groapa fara fund este gura femeilor straine; cel ce este lovit de mânia Domnului cade în ea.
\par 15 Daca nebunia se pripa?e?te în inima celui tânar, numai varga certarii o va îndeparta de el.
\par 16 Daca împilezi pe un sarac, î?i înmul?e?ti averea, daca dai unui bogat sarace?ti.
\par 17 Pleaca urechea ta ?i asculta cuvintele celor iscusi?i ?i inima ta îndreapt-o spre ?tiin?a mea.
\par 18 Este placut daca tu le pastrezi înlauntrul tau. O, de-ar sta toate pe buzele tale!
\par 19 Pentru a-?i pune nadejdea în Domnul, vreau sa-?i dau înva?atura astazi.
\par 20 Oare nu ?i-am a?ezat în scris în nenumarate rânduri sfaturi ?i înva?aturi,
\par 21 Ca sa-?i fac cunoscut credincio?ia cuvintelor adevarate ?i sa raspunzi prin cuvinte de buna credin?a, celor ce te întreaba?
\par 22 Nu jefui pe sarac, pentru ca el e sarac ?i nu asupri pe cel nenorocit la poarta (ceta?ii),
\par 23 Caci Domnul va apara pricina lor ?i va ridica via?a celor care îi vor fi jefuit.
\par 24 Nu te întovara?i cu omul mânios ?i cu cel înfierbântat de furie sa n-ai nici un amestec,
\par 25 Ca sa nu te deprinzi pe calea lui ?i sa-?i întinzi o cursa pentru via?a ta.
\par 26 Nu fi dintre aceia care dau mâna, care se pun cheza?i pentru datorii.
\par 27 Daca nu ai cu ce plati, pentru ce te învoie?ti ca sa ?i se ia ?i patul de sub tine?
\par 28 Nu muta hotarul stravechi pe care l-au însemnat parin?ii tai.
\par 29 Vezi tu un om dibaci la lucrul lui? El va sta înaintea conducatorilor ?i nu înaintea oamenilor de rând.

\chapter{23}

\par 1 Când stai la masa cu un dregator, ia seama pe cine ai înaintea ta;
\par 2 Pune-?i un cu?it la gât, daca tu e?ti lacom.
\par 3 Nu pofti bucatele lui, caci sunt mâncari în?elatoare.
\par 4 Nu te osteni sa ajungi bogat; nu-?i pune iscusin?a ta în aceasta.
\par 5 Oare vrei sa te ui?i cu ochii cum ea se risipe?te? Caci boga?ia face aripi ca un vultur care se înal?a catre cer.
\par 6 Nu mânca pâinea celui ce se uita cu ochi rai ?i nu pofti bucatele lui,
\par 7 Caci el î?i numara buca?ile din gura. "Manânca ?i bea!", î?i va spune, dar inima lui nu e pentru tine.
\par 8 Bucata pe care ai mâncat-o o vei da afara din tine, iar tu ?i-ai risipit (zadarnic) vorbele tale alese.
\par 9 Nu grai la urechea celui nebun, caci el nu va baga de seama iscusin?a graiurilor tale.
\par 10 Nu muta hotarul vaduvei ?i nu încalca ogorul celor orfani,
\par 11 Caci Ocrotitorul lor e tare ?i El va apara pricina lor împotriva ta.
\par 12 Sile?te la înva?atura inima ta ?i urechea ta la cuvinte iscusite.
\par 13 Nu cru?a pe feciorul tau de pedeapsa; chiar daca îl love?ti cu varga, nu moare.
\par 14 Tu îl ba?i cu toiagul, dar scapi sufletul lui din împara?ia mor?ii.
\par 15 Fiul meu, daca inima ta e plina de în?elepciune ?i inima mea se va bucura.
\par 16 Rarunchii mei vor tresari de bucurie, când buzele tale vor grai ceea ce este drept.
\par 17 Sa nu râvneasca inima ta la cei pacato?i, ci totdeauna sa ramâna la frica de Domnul,
\par 18 Caci daca o vei pazi pe ea, mai ai ?i tu un viitor ?i nadejdea ta nu se va pierde.
\par 19 Asculta fiul meu, ?i te în?elep?e?te ?i îndreapta inima ta pe calea cea dreapta.
\par 20 Nu fi printre cei ce se îmbata de vin ?i printre cei ce î?i desfrâneaza trupul lor,
\par 21 Caci be?ivul ?i desfrânatul saracesc, iar dormitul mereu te face sa por?i zdren?e.
\par 22 Asculta pe tatal tau care te-a nascut ?i nu dispre?ui pe mama ta când ea a ajuns batrâna.
\par 23 Aduna adevar ?i nu-l vinde, în?elepciune ?i înva?atura ?i buna chibzuiala.
\par 24 Tatal celui drept tresalta de bucurie ?i cel ce a dat na?tere unui în?elept se bucura de el.
\par 25 Sa se bucure tatal ?i mama ta ?i sa salte de veselie cea care te-a nascut!
\par 26 Da-mi, fiule, mie inima ta, ?i ochii tai sa simta placere pentru caile mele,
\par 27 Caci femeia desfrânata este o groapa adânca ?i cea straina un pu? strâmt.
\par 28 Pentru aceasta ea sta ca un ho? la pânda ?i spore?te printre oameni numarul celor în?ela?i de ea.
\par 29 Pentru cine sunt suspinele, pentru cine vaicarelile, pentru cine gâlcevile, pentru cine plânsetele, pentru cine ranile fara pricina, pentru cine ochii întrista?i?
\par 30 Pentru cei ce zabovesc pe lânga vin, pentru cei ce vin sa guste bauturi cu mirodenii.
\par 31 Nu te uita la vin cum este el de ro?u, cum scânteiaza în cupa ?i cum aluneca pe gât,
\par 32 Caci la urma el ca un ?arpe mu?ca ?i ca o vipera împroa?ca venin.
\par 33 Daca ochii tai vor privi la femei straine ?i gura ta va grai lucruri me?te?ugite,
\par 34 Vei fi ca unul care sta culcat în mijlocul marii, ca unul care a adormit pe vârful unui catarg.
\par 35 "M-au lovit... Nu m-a durut! M-au batut... Nu ?tiu nimic! Când ma voi de?tepta din somn, voi cere iara?i vin".

\chapter{24}

\par 1 Nu râvni la oamenii rai ?i nu pofti sa fii în tovara?ia lor,
\par 2 Caci inima lor pune la cale lucruri silnice ?i buzele lor graiesc cele nelegiuite.
\par 3 Prin în?elepciune se ridica o casa, prin buna chibzuiala se întare?te
\par 4 ?i prin ?tiin?a se umplu camarile ei de tot felul de avu?ie scumpa ?i placuta.
\par 5 Mai puternic este un în?elept decât un voinic ?i cel priceput decât unul plin de putere.
\par 6 Cu oricâta dibacie te vei razboi, biruin?a se dobânde?te cu mul?i sfatuitori.
\par 7 Peste masura de înalta este în?elepciunea pentru omul nebun; când sta la poarta (ceta?ii) el nu deschide gura.
\par 8 Cel ce-?i pune în gând sa faca rau se cheama un mare raufacator.
\par 9 Gândul celui nebun nu este decât pacat; batjocoritorul este urgia oamenilor.
\par 10 Daca te ara?i slab în ziua strâmtorarii, puterea ta nu este decât slabiciune.
\par 11 Izbave?te pe cei ce sunt târâ?i la moarte ?i pe cei ce se duc clatinându-se la junghiere scapa-i!
\par 12 Daca vrei sa spui: "Iata n-am ?tiut nimic!", oare Cel ce cântare?te inimile nu patrunde cu privirea ?i Cel ce vegheaza peste sufletul tau nu ?tie ?i nu va rasplati omului dupa faptele lui?
\par 13 Fiul meu, manânca miere, caci e buna ?i un fagure de miere este dulce gurii tale.
\par 14 Sa ?tii ca în?elepciunea este la fel pentru sufletul tau; daca o dobânde?ti, ai un viitor, iar nadejdea ta nu este pierduta.
\par 15 Nu pândi, nelegiuitule, casa celui drept ?i nu tulbura loca?ul lui,
\par 16 Caci daca cel drept cade de ?apte ori ?i tot se scoala, cei fara de lege se poticnesc în nenorocire.
\par 17 Nu te bucura când cade vrajma?ul tau ?i, când se poticne?te, sa nu se veseleasca inima ta,
\par 18 Ca nu cumva sa vada Domnul ?i sa fie neplacut în ochii Lui ?i sa nu întoarca mânia Sa de la el (spre tine).
\par 19 Nu te aprinde împotriva raufacatorului ?i nu-?i întarâta râvna împotriva celor fara de lege.
\par 20 Caci cel ce face rau nu propa?e?te, ?i sfe?nicul celor nelegiui?i se va stinge.
\par 21 Fiul meu, teme-te de Domnul ?i de rege ?i cu cei ce se razvratesc nu lega prietenie,
\par 22 Ca fara de veste va veni nenorocirea ?i cine poate sa cunoasca sfâr?itul lor naprasnic?
\par 23 ?i aceste (proverbe) sunt ale în?elep?ilor: Nu e bine ca la judecata sa cau?i la fa?a oamenilor.
\par 24 Pe cel ce zice celui fara de lege: "Tu e?ti drept!", popoarele îl blesteama ?i neamurile îl afurisesc;
\par 25 Dar celor care îl cearta Cum se cuvine le merge bine ?i peste ei vine binecuvântarea ?i fericirea.
\par 26 Buzele saruta pe cei ce dau raspunsuri drepte.
\par 27 Rânduie?te-?i lucrul tau afara ?i adu-l la îndeplinire pe câmpul tau, apoi î?i vei ridica o casa.
\par 28 Nu fi martor mincinos împotriva prietenului tau ?i nu fi pricina (unei hotarâri nedrepte), cu buzele tale.
\par 29 Nu spune: "Precum mi-a facut a?a îi voi face ?i eu lui; voi rasplati omului dupa faptele lui".
\par 30 Am trecut pe ogorul unui lene? ?i pe la via unui om lipsit de minte,
\par 31 ?i iata spinii cre?teau în toate locurile, maracinii o acopereau cu totul, iar zidul de pietre se prabu?ise.
\par 32 Atunci m-am uitat ?i m-am framântat în inima mea, am privit cu luare aminte ?i am tras o înva?atura:
\par 33 "Înca pu?in somn, înca pu?ina a?ipeala, înca pu?in sa mai stau cu mâinile în sân ca sa dorm..."
\par 34 ?i saracia va veni peste tine ca un calator ?i lipsa ca un om înarmat.

\chapter{25}

\par 1 ?i acestea sunt pildele lui Solomon, pe care le-au adunat oamenii lui Iezechia, regele lui Iuda.
\par 2 Slava lui Dumnezeu este sa ascunda lucrurile, iar marirea regilor e sa le cerceteze cu de-amanuntul.
\par 3 Precum înal?imea cerului ?i adâncul pamântului sunt lucruri nepatrunse, tot a?a ?i inima regilor.
\par 4 Cura?a argintul de zgura ?i turnatorul va face din el un vas ales.
\par 5 Da la o parte pe cel fara de lege din fa?a mai-marelui ?i tronul lui se va întari prin dreptate.
\par 6 Nu te fali înaintea cârmuitorului ?i nu sta în locul hotarât pentru cei mari,
\par 7 Caci mai degraba sa ?i se zica: "Suie aici!" decât sa te umileasca în fa?a stapânului. Ceea ce au vazut ochii tai,
\par 8 Nu aduce grabnic spre disputa, caci ce ai sa faci dupa aceea când aproapele tau te va da de ru?ine?
\par 9 Cearta-te cu aproapele tau, dar taina altuia sa nu o dai pe fa?a,
\par 10 Ca nu cumva cel ce o aude sa nu te defaime ?i sa nu darâme (pentru totdeauna) faima ta.
\par 11 Ca merele de aur pe poli?i de argint, a?a este cuvântul spus la locul lui.
\par 12 Inel de aur ?i podoabe de aur de mult pre? este pova?uitorul în?elept la urechea ascultatoare.
\par 13 Precum este racoarea zapezii în vremea seceri?ului, a?a este solul credincios pentru cei ce-l trimit; el bucura sufletul stapânului sau.
\par 14 Precum sunt norii ?i vântul fara ploaie, a?a este omul care se lauda cu darul pe care niciodata nu-l da.
\par 15 Prin rabdare se poate îndupleca un om mânios ?i o limba dulce înmoaie oase.
\par 16 Daca ai gasit miere, manânca atât cit î?i trebuie, ca nu cumva sa te saturi ?i s-o ver?i.
\par 17 Pune rar piciorul în casa prietenului tau, ca nu cumva sa se sature de tine ?i sa te urasca.
\par 18 Un ciocan, o sabie ?i o sageata ascu?ita este omul care da marturie mincinoasa împotriva aproapelui sau.
\par 19 Dinte rau ?i picior ?ovaitor este cel fara credin?a în vreme de nevoie.
\par 20 Ca atunci când dezbraci haina ne vreme friguroasa, sau torni o?et pe silitra, a?a este cântarea pentru o inima întristata.
\par 21 De flamânze?te vrajma?ul tau, da-i sa manânce pâine ?i daca înseteaza, adapa-l eu apa,
\par 22 Ca numai a?a îi îngramade?ti carbuni aprin?i pe capul lui ?i Domnul î?i v a rasplati ?ie.
\par 23 Vântul de la miazanoapte aduce ploaie ?i limba clevetitoare aduce o fa?a mâhnita.
\par 24 Mai bine sa sala?luie?ti într-un col? de acoperi?, decât sa traie?ti cu o femeie certarea?a într-o casa mare.
\par 25 Precum e apa rece pentru un suflet însetat, a?a e vestea buna dintr-o ?ara departata.
\par 26 Ca un izvor tulbure ?i stricat, a?a este dreptul care ?ovaie în fa?a celui nelegiuit.
\par 27 Precum celui care manânca multa miere nu-i merge bine, tot a?a ?i celui care se lasa cople?it de cuvinte de lauda.
\par 28 Asemenea unei ceta?i cu o spartura ?i fara zid, a?a este omul caruia îi lipse?te stapânirea de sine.

\chapter{26}

\par 1 Precum este zapada în timpul verii ?i ploaia la seceri?, a?a nu-i place celui nebun cinstea.
\par 2 Precum vrabia zboara ?i rândunica se înal?a în vazduh, tot a?a blestemul fara pricina nu nimere?te.
\par 3 Biciul este bun pentru cal, frâul pentru magar, iar varga pentru spatele celor nebuni.
\par 4 Nu raspunde nebunului dupa nebunia lui, ca sa nu te asemeni ?i tu cu el.
\par 5 Raspunde nebunului dupa nebunia lui, ca sa nu se creada în?elept în ochii lui.
\par 6 Cel ce încredin?eaza solia în mâna celui nebun î?i taie picioarele ?i bea nedreptate.
\par 7 Precum nu poate sa se foloseasca slabanogul de picioarele sale, tot a?a nici cei nebuni de cuvintele cele în?elepte.
\par 8 Ca ?i când pui o piatra în pra?tie, a?a este cel ce da cinste unui nebun.
\par 9 Precum un ghimpe intra în mâna unui be?iv, tot a?a sunt cuvintele în?elepte în gura celor pacato?i.
\par 10 Ca un arca? care rane?te pe to?i, a?a este cel ce se pune cheza? pentru cel nebun ?i pentru cei ce trec pe cale.
\par 11 Ca un câine care se întoarce unde a varsat, a?a este omul nebun care se întoarce la nebunia lui.
\par 12 Daca vezi un om care se crede în?elept în ochii lui, sa nadajduie?ti mai mult de la un nebun decât de la el.
\par 13 Lene?ul zice: "Pe drum trece un leu, un leu pe uli?e!"
\par 14 Precum u?a se suce?te în ?â?âna, tot a?a ?i lene?ul în patul lui.
\par 15 Lene?ul baga mâna în blid, dar cu mare greutate o duce la gura.
\par 16 Lene?ul se crede în?elept în ochii lui, mai mult decât ?apte sfetnici în?elep?i.
\par 17 Ca cel ce prinde un câine de urechi, a?a este cel ce se vâra într-o cearta în care nu este amestecat.
\par 18 Ca unul care arunca sage?i arzatoare, lanci, sage?i ?i moarte,
\par 19 A?a e omul care în?ala pe prietenul sau ?i zice: "Da, am glumit!"
\par 20 Când nu mai sunt lemne se stinge focul ?i daca nu mai este nici un defaimator se potole?te cearta.
\par 21 Carbunii slujesc pentru caldura, lemnele pentru foc, iar omul certare? pentru a a?â?a cearta.
\par 22 Vorbele celui defaimator sunt ca bucatele gustoase; ele se duc în adâncul maruntaielor.
\par 23 Spoiala de argint care îmbraca un vas de lut, a?a sunt buzele mieroase ?i o inima rea.
\par 24 Cu buzele sale se preface cel ce ura?te, iar înlauntrul lui nutre?te în?elaciune;
\par 25 Când î?i schimba glasul, sa nu-l crezi, caci ?apte urâciuni sunt în inima lui.
\par 26 Cineva poate sa-?i ascunda ura lui prin prefacatorie, dar în adunare rautatea lui se da pe fa?a.
\par 27 Cine sapa groapa (altuia) cade singur în ea ?i cel ce rostogole?te o piatra se pravale?te (tot) peste el.
\par 28 Limba mincinoasa ura?te adevarul ?i gura lingu?itorilor pricinuie?te prabu?irea.

\chapter{27}

\par 1 Nu te lauda cu ziua de mâine, ca nu ?tii la ce poate da na?tere.
\par 2 Sa te laude altul ?i nu gura ta, un strain ?i nu buzele tale.
\par 3 Piatra este grea ?i nisipul cu anevoie de ridicat; însa furia nebunului este mai grea decât amândoua.
\par 4 Întarâtarea este cruda ?i mânia apriga, dar taria pizmei cine o va putea îndura?
\par 5 Mai mult pre?uie?te o dojana pe fa?a decât o dragoste ascunsa.
\par 6 De buna credin?a sunt ranile pricinuite de un prieten, iar sarutarile celui ce te ura?te sunt viclene.
\par 7 Satulul calca mierea în picioare, iar flamândului tot ce este amar (i se pare) dulce.
\par 8 Ca o pasare gonita din cuibul ei, a?a este omul izgonit din casa sa.
\par 9 Untdelemnul ?i miresmele înveselesc inima, dar tot a?a de dulci sunt sfaturile unui prieten care pornesc din suflet.
\par 10 Pe prietenul tau ?i pe prietenul tatalui tau nu-i parasi; în casa fratelui tau nu intra în ziua restri?tii tale. Mai bun e un vecin aproape de tine, decât un frate departe.
\par 11 Fii în?elept, fiul meu, ?i bucura inima mea, ca sa pot raspunde celui ce ma clevete?te.
\par 12 În?eleptul vede nenorocirea ?i se ascunde, cei pro?ti dau peste ea ?i îndura necaz.
\par 13 Ia-i haina caci s-a pus cheza? pentru altul ?i cere-i zalog din pricina celor straini.
\par 14 Celui ce binecuvânteaza pe prietenul sau cu glas mare dis-de-diminea?a, i se socote?te ca un blestem.
\par 15 Un jgheab care curge în vreme de ploaie ?i o femeie ar?agoasa sunt la fel;
\par 16 Cel care vrea s-o opreasca opre?te vânt ?i mâna lui cea dreapta parca ar ?ine în ea untdelemn.
\par 17 Fierul cu fier se ascute ?i un om ascute mânia altui om.
\par 18 Cel ce paze?te un smochin manânca din rodul lui, iar cel ce paze?te pe stapânul sau va fi rasplatit cu cinste.
\par 19 Precum nu se aseamana fa?a cu fa?a, tot a?a inima unui om cu inima altuia.
\par 20 Iadul ?i adâncul nu se pot satura, tot a?a ?i inima omului e de nesaturat.
\par 21 În topitoare se lamure?te argintul ?i în cuptor aurul, iar omul se pre?uie?te dupa numele cel bun.
\par 22 Chiar daca vei pisa în piuli?a cu pilugul pe cel nebun, întocmai ca pe boabe, tot nu-l vei despar?i de nebunia lui.
\par 23 Sârguie?te-te sa-?i cuno?ti oile tale ?i ia seama la turma ta,
\par 24 Ca bunastarea nu dainuie?te de-a pururi ?i nici boga?ia din neam în neam.
\par 25 Când iarba s-a trecut ?i pa?unea s-a ispravit ?i finul de pe dealuri s-a strâns,
\par 26 Tu ai miei pentru îmbracamintea ta ?i ?api ea sa plate?ti pa?unea;
\par 27 ?i laptele de capra îl ai cu îndestulare, pentru hrana casei, ?i merinde pentru slujnicele tale.

\chapter{28}

\par 1 Cel nelegiuit fuge fara ca nimeni sa-l urmareasca, iar dreptul sta ca un pui de leu fara grija.
\par 2 Din pricina gre?elilor unui om silnic se ivesc certuri, iar omul iscusit le stinge.
\par 3 Un om bogat, care asupre?te pe cei saraci, e ca ploaia care trânte?te tot la pamânt, iar pâinea nu se face.
\par 4 Cei ce parasesc legea ridica în slavi pe pacato?i, iar cei ce o pazesc se aprind împotriva lor.
\par 5 Oamenii rai nu pricep nimic din ceea ce e drept, iar cei ce cauta pe Domnul în?eleg tot.
\par 6 Mai de pre? e saracul care umbla întru neprihanirea lui, decât cel prefacut în caile lui, chiar daca e bogat.
\par 7 Cel ce paze?te legea este un fiu în?elept, iar cel ce se întovara?e?te cu clevetitorii face ru?ine tatalui sau.
\par 8 Cel ce î?i spore?te averea lui, prin dobânda ?i prin camata, aduna pentru cel ce are mila de saraci.
\par 9 Cel ce î?i opre?te urechea de la ascultarea legii, chiar rugaciunea lui e urâciune.
\par 10 Cel ce ratace?te pe cei drep?i pe o cale rea va cadea în groapa (pe care a sapat-o); cei fara prihana vor fi ferici?i.
\par 11 Omul bogat este în?elept în ochii lui, dar cel sarac ?i priceput îl dovede?te cu mintea.
\par 12 Când drep?ii biruiesc e mare sarbatoare, iar când cei fara de lege ies la iveala, oamenii se ascund.
\par 13 Cel ce î?i ascunde pacatele lui nu propa?e?te, iar cel ce le marturise?te ?i se lasa de ele va fi miluit.
\par 14 Fericit este omul care se teme totdeauna, iar cel ce î?i învârto?eaza inima lui va cadea în nenorocire.
\par 15 Leu care racne?te ?i urs flamând este cel rau care stapâne?te peste un popor sarac.
\par 16 Stapânitorul cel lipsit de venituri este mare asupritor; cel ce ura?te câ?tigul (nedrept) va trai multa vreme.
\par 17 Un om pe care îl îngreuiaza sângele unui ucis fuge pâna la groapa; nimeni sa nu-l opreasca!
\par 18 Cel ce umbla fara prihana va fi mântuit, iar cine apuca pe cai strâmbe va cadea într-o groapa.
\par 19 Cel ce lucreaza pamântul lui se va îndestula de pâine, iar cel ce umbla dupa lucruri de nimic se va satura de saracie.
\par 20 Omul credincios va fi încarcat de binecuvântari, iar cine zore?te sa ajunga bogat nu va ramâne nepedepsit.
\par 21 Nu este bine sa te ui?i la fa?a omului, caci pentru o bucata de pâine cineva poate sa gre?easca.
\par 22 Omul lacom se grabe?te sa se îmboga?easca, dar nu gânde?te ca lipsa va veni peste el.
\par 23 Cel care cearta pe un om va avea mai multa mul?umire decât cel care-l lingu?e?te.
\par 24 Cine despoaie pe tatal sau ?i pe mama sa ?i zice: "Nu-i pacat!" este tovara? cu facatorul de rele.
\par 25 Omul lacom a?â?a cearta, iar cel ce nadajduie?te în Domnul va fi îndestulat.
\par 26 Cel ce î?i pune nadejdea în inima lui este un nebun, iar cel ce se conduce dupa în?elepciune, acela va fi mântuit.
\par 27 Cine da la cel sarac nu duce lipsa; iar cine î?i acopera ochii lui va fi mult blestemat.
\par 28 Când nelegiui?ii ies la iveala, oamenii se ascund, iar când ei pier, se înmul?esc cei drep?i.

\chapter{29}

\par 1 Un om pedepsit îndelung ?i tare la cerbice va fi intr-o clipa zdrobit ?i fara vindecare.
\par 2 Când drep?ii domnesc se bucura poporul ?i, când stapânesc cei fara de lege, suspina.
\par 3 Cine iube?te în?elepciunea bucura pe tatal sau, iar cine umbla cu desfrânatele î?i prapade?te averea.
\par 4 Un conducator prin dreptate face sa propa?easca ?ara, iar cel ce pune dari grele o ruineaza.
\par 5 Omul care lingu?e?te pe prietenul sau întinde cursa pa?ilor lui.
\par 6 Pe calea celui rau este întins un la?, dar dreptul trebuie sa fuga ?i sa salte de bucurie.
\par 7 Omul drept se îngrije?te de pricina celor sarmani; celui fara de lege nu-i pasa de ei.
\par 8 Batjocoritorii rascoala cetatea, iar cei în?elep?i potolesc mânia.
\par 9 Când un în?elept se cearta cu un nebun, fie ca se supara, fie ca râde, nu-?i pierde cumpatul.
\par 10 Oamenii seto?i de sânge urasc pe cel fara prihana, iar cei drep?i ocrotesc via?a lui.
\par 11 Nebunul face sa izbucneasca pornirea lui patima?a, iar în?eleptul î?i înfrâneaza mânia.
\par 12 Când un conducator asculta de cuvinte mincinoase, to?i slujitorii sai sunt rai.
\par 13 Saracul ?i cu cel ce asupre?te pe cei saraci se întâlnesc; Cel ce lumineaza ochii amândurora este Domnul.
\par 14 Un conducator care judeca cu dreptate pe cei saraci î?i întare?te scaunul lui pe veci.
\par 15 Varga ?i certarea aduc în?elepciune, iar tânarul care este lasat (în voia apucaturilor lui) face ru?ine maicii sale.
\par 16 Când cei fara de lege domnesc se înmul?esc rauta?ile, iar drep?ii vor vedea (cu bucurie) prabu?irea lor.
\par 17 Mustra pe fiul tau ?i el i?i va fi odihna ?i î?i va face placere sufletului tau.
\par 18 Fara vedenie de prooroc poporul e fara stapân, dar fericit este cel care paze?te legea!
\par 19 Sluga nu se îndreapta numai cu pove?e, fiindca, de?i pricepe, însa nu asculta.
\par 20 Daca vezi un om care se zore?te la vorba, atunci pentru un nebun e mai multa nadejde decât pentru el.
\par 21 Daca (vreun stapân) dezmiarda din copilarie pe robul sau, acesta ajunge la sfâr?it sa se creada fiu.
\par 22 Un om mânios a?â?a cearta ?i cel aprig savâr?e?te multe pacate.
\par 23 Mândria umile?te pe om, iar de cinste are parte cel smerit.
\par 24 Cel ce împarte cu ho?ul î?i ura?te sufletul lui, fiindca aude blestemul, dar nu zice nimic.
\par 25 Teama de oameni duce la caderea în cursa, dar cel ce nadajduie?te în Domnul sta la adapost.
\par 26 Mul?i cauta fa?a stapânitorului, dar dreptatea omului vine de la Domnul.
\par 27 Omul nedrept este urâciune pentru cei drep?i, iar cel drept este o urâciune pentru cei rai.

\chapter{30}

\par 1 Cuvintele lui Agur, fiul lui Iache din Massa. Acest om a zis: "Sunt ostenit, Dumnezeule, sunt obosit, Doamne, sunt sleit de puteri!
\par 2 Caci sunt tare prost, ca sa ma pot socoti ca om ?i nu am pricepere (care ar putea sa fie vrednica) de un om.
\par 3 Nici n-am înva?at în?elepciunea ?i nici ?tiin?a celor sfin?i nu o cunosc.
\par 4 Cine s-a suit în ceruri ?i iara?i s-a pogorât, cine a adunat vântul în mâinile lui? Cine a legat apele în haina lui? Cine a întarit toate marginile pamântului? Care este numele lui ?i care este numele fiului sau? Spune daca ?tii!
\par 5 Toate cuvintele lui Dumnezeu sunt lamurite, scut este El pentru cei ce cauta la El scaparea.
\par 6 Nu adauga nimic la cuvintele Lui, ca sa nu te traga la socoteala ?i sa fii gasit de minciuna!
\par 7 Doua lucruri cer de la Tine, nu ma respinge înainte de a muri:
\par 8 Prefacatoria ?i cuvântul mincinos îndeparteaza-le de la mine; saracie ?i boga?ie nu-mi da, ci da-mi pâinea care-mi este de trebuin?a,
\par 9 Ca nu cumva, saturându-ma, sa ma lepad de Tine ?i sa zic: "Cine este Domnul?" Ca nu cumva, saracind, sa ma apuc de furat ?i sa defaim numele Dumnezeului meu.
\par 10 Nu grai de rau pe sluga la stapânul sau, ca nu cumva sa te blesteme ?i sa te sileasca sa-?i ceri iertare.
\par 11 Este câte un neam de oameni care blesteama pe tatal sau ?i nu binecuvânteaza pe maica sa;
\par 12 Un neam caruia i se pare ca e fara prihana în ochii lui ?i care nu este cura?it de necura?ia lui;
\par 13 Un neam... O, cum ridica ochii lui sus ?i cit se înal?a de sus genele lui!
\par 14 Un neam ai carui din?i sunt ca sabiile ?i ai caror col?i sunt cu?ite, ca sa manânce pe cei sarmani de pe pamânt ?i pe cei saraci dintre oameni.
\par 15 Lipitoarea are doua fiice care zic: "Da-mi, da-mi!" Trei lucruri nu se pot satura, ba ?i al patrulea care nu zice niciodata: "Destul!" ?i anume:
\par 16 Locuin?a mor?ilor, pântecele sterp, pamântul care nu e satul de apa ?i focul care nu zice niciodata: "Destul!"
\par 17 Ochiul care î?i bate joc de parintele sau ?i nu ia în seama ascultarea (ce este dator) maicii sale, sa-l scoata corbii care sala?luiesc linga un curs de apa, iar puii de vultur sa-l manânce.
\par 18 Trei lucruri mi se par minunate, ba chiar patru, pe care nu le pot pricepe:
\par 19 Calea vulturului pe cer, urma ?arpelui pe stânca, mersul corabiei în mijlocul marii ?i calea omului la o fecioara.
\par 20 A?a este purtarea unei femei desfrânate: ea manânca ?i î?i ?terge gura ?i zice: "N-am facut nimic rau"
\par 21 Pentru trei lucruri se cutremura pamântul, ba chiar pentru patru nu poate sa rabde:
\par 22 Pentru robul care ajunge rege, pentru nebunul care se satura de pâine,
\par 23 Pentru o femeie dispre?uita când ea se marita ?i pentru o sluga care mo?tene?te pe stapâna sa.
\par 24 Patru sunt animalele cele mai mici de pe pamânt ?i care sunt cele mai în?elepte:
\par 25 Furnicile, neam fara putere, care î?i agonisesc vara hrana lor;
\par 26 Dihorii, neam slab, care-?i cladesc în stânci loca?ul lor;
\par 27 Lacustele care nu au rege ?i totu?i ies toate în stoluri;
\par 28 ?opârla care se poate prinde cu mâna ?i care patrunde în palatele regilor.
\par 29 Trei fiin?e au înfa?i?are frumoasa, ba patru, care au un mers mare?:
\par 30 Leul, viteazul printre dobitoace, care nu da înapoi în fa?a nimanui;
\par 31 Coco?ul cel ager, ?apul ?i regele caruia nimeni nu-i poate sta împotriva.
\par 32 De e?ti a?a de nebun ca sa te la?i mânat de mânie, bate-te cu mâna peste gura.
\par 33 Batutul laptelui da untul, lovitura peste nas face sa ?â?neasca sângele, iar întarâtarea mâniei duce la cearta.

\chapter{31}

\par 1 Cuvintele lui Lemuel, regele din Massa, cu care mama sa îl înva?a:
\par 2 Fiul meu, rodul pântecelui meu, feciorul fagaduin?elor mele, cu ce pot eu sa te îndemn?
\par 3 Nu da puterea ta femeilor ?i caile taie celor care pierd pe regi.
\par 4 Nu se cuvine regilor, o, Lemuel, nu se cuvine regilor sa bea vin ?i conducatorii bauturi îmbatatoare,
\par 5 Ca nu cumva bând sa uite legea ?i sa judece strâmb pe to?i sarmanii.
\par 6 Da?i bautura îmbatatoare celui ce este gata sa piara ?i vin celui cu amaraciune în suflet,
\par 7 Ca sa bea ?i sa uite saracia ?i sa nu-?i mai aduca aminte de chinul lui.
\par 8 Deschide gura ta pentru cel mut ?i pentru pricina tuturor parasi?ilor.
\par 9 Deschide gura ta, judeca drept ?i fa dreptate celui sarac ?i napastuit.
\par 10 Cine poate gasi o femeie virtuoasa? Pre?ul ei întrece margeanul.
\par 11 Într-însa se încrede inima so?ului ei, iar câ?tigul nu-i va lipsi niciodata.
\par 12 Ea îi face bine ?i nu rau în tot timpul vie?ii sale:
\par 13 Ea cauta lâna ?i cânepa ?i lucreaza voios cu mâna sa.
\par 14 Ea se aseamana cu corabia unui negu?ator care de departe aduce hrana ei.
\par 15 Ea se scoala dis-de-diminea?a ?i împarte hrana în casa ei ?i da porunci slujnicelor.
\par 16 Gânde?te sa cumpere o ?arina ?i o dobânde?te; din osteneala palmelor sale sade?te vie.
\par 17 Ea î?i încinge cu putere coapsele sale ?i î?i întare?te bra?ele sale.
\par 18 Ea simte ca bun e câ?tigul ei; sfe?nicul ei nu se stinge nici noaptea.
\par 19 Ea pune mâna pe furca ?i cu degetele sale apuca fusul.
\par 20 Ea întinde mâna spre cel sarman ?i bra?ul ei spre cel necajit.
\par 21 N-are teama pentru cei ai casei sale în vreme de iarna, caci to?i din casa sunt îmbraca?i în haine stacojii.
\par 22 Ea î?i face scoar?e; hainele ei sunt de vison ?i de porfira.
\par 23 Cinstit este barbatul ei la por?ile ceta?ii, când sta la sfat cu batrânii ?arii.
\par 24 Ea face cama?i ?i le vinde, ?i brâie da negu?atorilor.
\par 25 Tarie ?i farmec este haina ei ?i ea râde zilei de mâine.
\par 26 Gura ?i-o deschide cu în?elepciune ?i sfaturi pline de dragoste sunt pe limba ei.
\par 27 Ea vegheaza la propa?irea casei sale ?i pâine, fara sa lucreze; ea nu manânca.
\par 28 Feciorii sai vin ?i o fericesc, iar so?ul ei o lauda:
\par 29 "Multe fete s-au dovedit harnice, dar tu le-ai întrecut pe toate!"
\par 30 În?elator este farmecul ?i de?arta este frumuse?ea; femeia care se teme de Domnul trebuie laudata!
\par 31 Sa se bucure de rodul mâinilor sale, ?i la por?ile ceta?ii harnicia ei sa fie data ca pilda!


\end{document}