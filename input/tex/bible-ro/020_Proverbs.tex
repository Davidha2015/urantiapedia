\begin{document}

\title{Proverbs}

Pro 1:1  Pildele lui Solomon, fiul lui David,
Pro 1:2  Folositoare pentru cunoa?terea în?elepciunii ?i a stapânirii de sine,
Pro 1:3  Pentru în?elegerea cuvintelor adânci, pentru dobândirea unei îndrumari bune, pentru dreptate, pentru dreapta judecata ?i nepartinire,
Pro 1:4  Pentru a prilejui celor fara gând rau o judecata istea?a, omului tânar cuno?tin?a ?i buna cugetare.
Pro 1:5  Sa ia aminte cel în?elept ?i î?i va spori ?tiin?a, iar cel priceput va dobândi iscusin?a de a se purta,
Pro 1:6  Patrunzând cu mintea pildele ?i în?elesurile adânci, graiurile celor în?elep?i ?i tâlcuirea lor nepatrunsa.
Pro 1:7  Frica de Dumnezeu este începutul în?elepciunii; cei fara minte dispre?uiesc în?elepciunea ?i stapânirea de sine.
Pro 1:8  Asculta, fiul meu, înva?atura tatalui tau ?i nu lepada îndrumarile maicii tale.
Pro 1:9  Caci ele sunt ca o cununa pe capul tau ?i ca o salba împrejurul gâtului tau.
Pro 1:10  Fiul meu, de voiesc pacato?ii sa te ademeneasca, nu te învoi,
Pro 1:11  Daca-?i spun: "Vino cu noi, sa ne punem la pânda, ca sa varsam sânge, sa întindem curse fara cuvânt celui neprihanit,
Pro 1:12  Sa-i înghi?im de vii ca locuin?a mor?ilor, ?i întregi, ca pe cei ce se coboara în mormânt.
Pro 1:13  Sa punem stapânire pe tot felul de lucruri scumpe, sa ne umplem de prada casele noastre,
Pro 1:14  Fii parta? la ob?tea noastra, o singura punga fi-va pentru to?i!"
Pro 1:15  Fiul meu, nu te întovara?i cu ei pe cale; abate piciorul tau din cararea lor,
Pro 1:16  Caci picioarele lor alearga numai la rau, iar ei zoresc sa verse sânge.
Pro 1:17  Zadarnic se întind curse în vazul pasarilor!
Pro 1:18  Caci ei întind curse tocmai împotriva sângelui lor, ?i sufletului lor î?i întind ei la?uri.
Pro 1:19  Aceasta este soarta celor lacomi de câ?tig; lacomia le aduce pierderea vie?ii.
Pro 1:20  În?elepciunea striga pe uli?a ?i în piele î?i ridica glasul sau.
Pro 1:21  Ea propovaduie?te la raspântiile zgomotoase; înaintea por?ilor ceta?ii î?i spune cuvântul:
Pro 1:22  "Pâna când, pro?tilor, ve?i iubi prostia? Pâna când, nebunilor, ve?i iubi nebunia? ?i voi, ne?tiutorilor, pâna când ve?i urî ?tiin?a?
Pro 1:23  Întoarce?i-va iara?i la mustrarea mea ?i iata eu voi turna peste voi duhul meu ?i va voi vesti cuvintele mele.
Pro 1:24  Chematu-v-am, dar voi n-a?i luat aminte! Întinsu-mi-am mâna ?i n-a fost cine sa ia seama!
Pro 1:25  Ci a?i lepadat toate sfaturile mele ?i mustrarile mele nu le-a?i primit.
Pro 1:26  De aceea ?i eu voi râde de pieirea voastra ?i ma voi bucura când va veni groaza peste voi,
Pro 1:27  Când va veni peste voi necazul ca furtuna ?i când nenorocirea ca vijelia va va cuprinde.
Pro 1:28  Atunci ma vor chema, dar eu nu voi raspunde; din zori ma vor cauta ?i nu ma vor afla,
Pro 1:29  Pentru ca ei au urât ?tiin?a ?i frica de Dumnezeu n-au ales-o,
Pro 1:30  Fiindca n-au luat aminte la sfaturile mele ?i cercetarea mea au dispre?uit-o.
Pro 1:31  Mânca-vor din rodul caii lor ?i de sfaturile lor satura-se-vor,
Pro 1:32  Caci îndaratnicia omoara pe cei pro?ti ?i nepasarea pierde pe cei fara minte;
Pro 1:33  Iar cel ce ma asculta va trai în pace ?i lini?te ?i de rele nu se va teme".
Pro 2:1  Fiul meu, de vei primi pove?ele mele ?i sfaturile mele de le vei pastra,
Pro 2:2  Plecându-?i urechea la în?elepciune ?i înclinând inima ta spre buna chibzuiala,
Pro 2:3  Daca vei chema prevederea ?i spre buna-cugetare î?i vei îndrepta glasul tau,
Pro 2:4  Daca o vei cauta întocmai ca pe argint ?i o vei sapa ca ?i pe o comoara,
Pro 2:5  Atunci vei pricepe temerea de Domnul ?i vei dobândi cuno?tin?a de Dumnezeu,
Pro 2:6  Caci Domnul da în?elepciune; din gura Lui izvora?te ?tiin?a ?i prevederea;
Pro 2:7  El pastreaza mântuirea pentru oamenii cei drep?i; El este scut pentru cei ce umbla în calea desavâr?irii;
Pro 2:8  El paze?te caile drepta?ii ?i pe cararea celor cuvio?i ai Lui sta de veghe.
Pro 2:9  Atunci tu vei în?elege dreptatea ?i buna judecata, calea cea dreapta ?i toate potecile binelui.
Pro 2:10  Când în?elepciunea se va sui la inima ta ?i ?tiin?a va desfata sufletul tau,
Pro 2:11  Când buna chibzuiala va veghea peste tine ?i în?elegerea te va pazi,
Pro 2:12  Atunci tu vei fi izbavit de calea celui rau ?i de omul care graie?te minciuna,
Pro 2:13  De cei ce parasesc caile cele drepte, ca sa umble pe drumuri întunecoase,
Pro 2:14  De cei ce se bucura când fac rau ?i se veselesc când umbla pe poteci întortochiate,
Pro 2:15  Ale caror carari sunt strâmbe ?i ratacesc pe cai piezi?e.
Pro 2:16  Atunci tu vei scapa de femeia care este a altuia, de straina ale carei cuvinte sunt ademenitoare,
Pro 2:17  Care lasa pe tovara?ul ei din tinere?e ?i uita de legamântul Dumnezeului ei,
Pro 2:18  Caci ea se pleaca împreuna cu casa ei spre moarte ?i drumul ei duce în iad;
Pro 2:19  Nimeni din cei ce se duc la ea nu se mai întoarce ?i niciunul nu mai afla cararile vie?ii.
Pro 2:20  Drept aceea mergi pe calea oamenilor celor buni ?i paze?te cararile celor drep?i,
Pro 2:21  Caci cei drep?i vor locui pamântul ?i cei fara de prihana vor sala?lui pe el;
Pro 2:22  Iar cei fara de lege vor fi nimici?i de pe pamânt ?i cei necredincio?i vor fi smul?i de pe el.
Pro 3:1  Fiul meu, nu uita înva?atura mea ?i inima ta sa pazeasca sfaturile mele,
Pro 3:2  Caci lungime de zile ?i ani de via?a ?i propa?ire îi se vor adauga.
Pro 3:3  Mila ?i adevarul sa nu te paraseasca; leaga-le împrejurul gâtului tau, scrie-le pe tabla inimii tale;
Pro 3:4  Atunci vei afla har ?i bunavoin?a înaintea lui Dumnezeu ?i a oamenilor.
Pro 3:5  Pune-?i nadejdea în Domnul din toata inima ta ?i nu te bizui pe priceperea ta.
Pro 3:6  Pe toate caile tale gânde?te la Dânsul ?i El î?i va netezi toate cararile tale.
Pro 3:7  Nu fii în?elept în ochii tai; teme-te de Dumnezeu ?i fugi de rau;
Pro 3:8  Aceasta va fi sanatate pentru trupul tau ?i o înviorare pentru oasele tale.
Pro 3:9  Cinste?te pe Domnul din averea ta ?i din pârga tuturor roadelor tale.
Pro 3:10  Atunci jitni?ele tale se vor umple de grâu ?i mustul va da afara din teascurile tale.
Pro 3:11  Fiul meu, nu dispre?ui certarea Domnului ?i nu sim?i scârba pentru mustrarile Lui,
Pro 3:12  Caci Domnul cearta pe cel pe care-l iube?te ?i ca un parinte pedepse?te pe feciorul care îi este drag.
Pro 3:13  Fericit este omul care a aflat în?elepciunea ?i barbatul care a dobândit pricepere,
Pro 3:14  Caci dobândirea ei este mai scumpa decât argintul ?i pre?ul ei mai mare decât al celui mai curat aur.
Pro 3:15  Ea este mai pre?ioasa decât pietrele scumpe; nici un rau nu i se poate împotrivi ?i e bine-cunoscuta tuturor celor ce se apropie de ea; nimic din cele dorite de tine nu se aseamana cu ea.
Pro 3:16  Via?a lunga este în dreapta ei, iar în stânga ei, boga?ie ?i slava; din gura ei iese dreptatea; legea ?i mila pe limba le poarta.
Pro 3:17  Caile ei sunt placute ?i toate cararile ei sunt caile pacii.
Pro 3:18  Pom al vierii este ea pentru cei ce o stapânesc, iar cei care se sprijina pe ea sunt ferici?i.
Pro 3:19  Prin în?elepciune, Domnul a întemeiat pamântul, iar prin în?elegere a întarit cerurile.
Pro 3:20  Prin ?tiin?a Sa a deschis adâncurile ?i norii picura roua.
Pro 3:21  Fiul meu, sa nu se departeze acestea dinaintea ochilor tai; pastreaza în?elepciunea ?i buna chibzuiala,
Pro 3:22  Caci ele sunt via?a sufletului tau ?i podoaba pentru gâtul tau.
Pro 3:23  Atunci tu vei merge fara teama pe calea ta ?i piciorul tau nu se va poticni.
Pro 3:24  De te culci, nu-?i va fi teama, iar de adormi, somnul tau va fi dulce.
Pro 3:25  Sa nu te temi de frica fara veste ?i nici de vreo navala a celor pacato?i,
Pro 3:26  Ca Domnul este nadejdea ta ?i va feri piciorul tau de cursa.
Pro 3:27  Nu zabovi a face bine celui ce are nevoie, când ai putin?a sa-i aju?i.
Pro 3:28  Nu spune aproapelui tau: "Du-te ?i vino, mâine î?i voi da!", când po?i sa-i dai acum.
Pro 3:29  Nu pune la cale raul împotriva aproapelui tau, când el locuie?te fara grija lânga tine.
Pro 3:30  Nu te certa cu nimeni fara pricina, de vreme ce nu ?i-a facut nici un rau.
Pro 3:31  Nu râvni sa fii ca omul silnic ?i nu alege nici una din caile lui;
Pro 3:32  Caci omul cu gând rau este urât de Domnul, iar de cei drep?i El este mai aproape.
Pro 3:33  Domnul blesteama casa celui fara de lege ?i binecuvânteaza adaposturile celor drep?i.
Pro 3:34  De cei batjocoritori El râde, iar celor smeri?i le da har.
Pro 3:35  Cei în?elep?i vor mo?teni marirea, iar cei nebuni vor avea parte de ocara.
Pro 4:1  Asculta?i, fiilor, înva?atura tatalui ?i lua?i aminte sa cunoa?te?i buna chibzuiala,
Pro 4:2  Caci eu va dau înva?atura buna: Nu parasi?i pova?a mea.
Pro 4:3  Caci ?i eu am fost fecior la tatal meu, singur, ?i cu duio?ie iubit la mama mea
Pro 4:4  ?i el ma înva?a ?i-mi zicea: "Inima ta sa pastreze cuvintele inimii mele, paze?te poruncile mele ?i vei fi viu.
Pro 4:5  Aduna în?elepciune, dobânde?te pricepere! Nu le uita ?i nu te departa de la cuvintele gurii mele!
Pro 4:6  Nu o lepada ?i ea te va pazi; iube?te-o ?i ea va sta de veghe.
Pro 4:7  Iata începutul în?elepciunii: Agonise?te în?elepciunea ?i cu pre?ul a tot ce ai, capata priceperea.
Pro 4:8  Pre?uie?te-o mult ?i ea te va înal?a; ea te va ridica în slavi daca o vei îmbra?i?a.
Pro 4:9  Ea va pune cununa de daruri pe capul tau ?i te va împodobi cu diadema de mare cinste.
Pro 4:10  Asculta, fiul meu ?i prime?te cuvintele mele ?i anii vie?ii tale se vor înmul?i.
Pro 4:11  Eu te voi înva?a calea în?elepciunii ?i te voi purta pe caile drepta?ii.
Pro 4:12  Când vei merge, pa?ii tai nu vor ?ovai ?i, chiar de vei alerga, nu te vei poticni.
Pro 4:13  ?ine cu tarie înva?atura ?i nu o parasi, paze?te-o caci ea este via?a ta.
Pro 4:14  Nu apuca pe calea celor fara de lege ?i nu pa?i pe drumul celor rai.
Pro 4:15  Ocole?te-o ?i nu merge pe ea, treci pe alaturea ?i du-te mai departe;
Pro 4:16  Caci ei nu dorm pâna nu faptuiesc rau ?i nu-i mai prinde somnul pâna nu fac pe cineva sa cada.
Pro 4:17  Caci ei se hranesc din pâine agonisita prin faradelege ?i beau vin dobândit prin asuprire.
Pro 4:18  Calea drep?ilor e ca zarea dimine?ii ce se mare?te mereu pâna se face ziua mare;
Pro 4:19  Iar calea celor fara de lege e ca întunericul ?i ei nici nu banuiesc de ce se pot împiedica.
Pro 4:20  Fiul meu, ia aminte la graiurile mele; la pove?ele mele pleaca-?i urechea ta!
Pro 4:21  Nu le scapa din ochi, pastreaza-le înlauntrul inimii tale,
Pro 4:22  Caci ele sunt via?a pentru cei ce le pun în fapta ?i doctorie pentru tot trupul omenesc.
Pro 4:23  Paze?te-?i inima mai mult decât orice, caci din ea ?â?ne?te via?a.
Pro 4:24  Leapada din gura ta orice cuvinte cu în?eles sucit, alunga de pe buzele tale viclenia.
Pro 4:25  Ochii tai sa priveasca drept înainte ?i genele tale drept înainte sa caute.
Pro 4:26  Fii cu luare aminte la calea picioarelor tale ?i toate cararile tale sa fie bine chibzuite.
Pro 4:27  Nu te abate nici la dreapta, nici la stânga, ?ine piciorul tau departe de rau. Caci cararile drepte le paze?te Domnul, iar cele strâmbe sunt cai rele. El va face drepte caile tale ?i mergerea ta o va face sa fie în pace.
Pro 5:1  Fiul meu, ia aminte la în?elepciunea mea ?i la sfatul meu cel bun pleaca urechea ta,
Pro 5:2  Ca sa-?i po?i pastra judecata ?i ca buzele tale sa pazeasca ?tiin?a.
Pro 5:3  Nu te uita la femeia lingu?itoare, caci buzele celei straine picura miere ?i cerul gurii sale e mai alunecator decât untdelemnul,
Pro 5:4  Dar la sfâr?it ea este mai amara decât pelinul, mai taioasa decât o sabie cu doua ascu?i?uri.
Pro 5:5  Picioarele ei coboara catre moarte; pa?ii ei duc de-a dreptul în împara?ia mor?ii.
Pro 5:6  Ea nu ia seama la calea vie?ii, pa?ii ei merg în ne?tire, nici ea nu ?tie unde.
Pro 5:7  ?i acum, fiul meu, asculta-ma ?i nu te îndeparta de la cuvintele gurii mele.
Pro 5:8  Fere?te-?i calea ta de ea ?i nu te apropia de u?a casei ei,
Pro 5:9  Ca sa nu dai vârtutea ta altora ?i anii tai unuia fara de mila;
Pro 5:10  Ca strainii sa nu se îndestuleze de stradania ta ?i ostenelile tale sa nu treaca în casa altuia;
Pro 5:11  Ca sa nu suspini la sfâr?it, când trupul tau ?i carnea ta vor fi fara de vlaga,
Pro 5:12  ?i sa zici: "Pentru ce am urât pova?a ?i de ce inima mea a urgisit certarea?
Pro 5:13  De ce nu am ascultat de îndemnul dascalilor mei ?i spre cei ce ma înva?au n-am plecat urechea mea?
Pro 5:14  Pu?in a trebuit sa nu ma nenorocesc, în plina adunare ?i în mijlocul ob?tei".
Pro 5:15  Bea apa din pu?ul tau ?i din pârâia?ele care curg din izvorul tau.
Pro 5:16  Sa nu se risipeasca izvoarele tale pe uli?a, nici pâraiele tale prin pie?e.
Pro 5:17  Sa fie numai pentru tine singur, iar nu pentru strainii care sunt cu tine!
Pro 5:18  Binecuvântat sa fie izvorul tau ?i sa te mângâi cu femeia ta din tinere?e.
Pro 5:19  Cerboaica preaiubita ?i gazela plina de farmec sa-?i fie ea; dragostea de ea sa te îmbete totdeauna ?i iubirea ei sa te desfateze.
Pro 5:20  Pentru ce, fiul meu, sa te momeasca femeie straina ?i tu sa îmbra?i?ezi sânul unei necunoscute?
Pro 5:21  Caci cararile omului sunt înaintea Domnului ?i El ia seama la toate caile lui.
Pro 5:22  Cel fara de lege este prins în la?urile faradelegilor lui ?i de funiile pacatelor lui este înfa?urat.
Pro 5:23  El va muri în pacatele lui ?i de mul?imea nebuniei lui va pieri.
Pro 6:1  Fiul meu, daca te-ai pus cheza? pentru prietenul tau, daca ai dat mâna pentru altul,
Pro 6:2  Atunci te-ai prins prin fagaduieli ie?ite din gura ta ?i te-ai legat prin cuvintele gurii tale.
Pro 6:3  Fa dar, fiul meu, aceasta: o, izbave?te-te. ?i fiindca ai cazut în mâinile aproapelui tau, du-te ?i cazi la picioarele aproapelui tau ?i-l roaga;
Pro 6:4  Nu da somn ochilor tai, nici dormitare genelor tale.
Pro 6:5  ?i te izbave?te ca o caprioara din mâna vânatorului ?i ca o pasare din mâna pasararului.
Pro 6:6  Du-te, lene?ule, la furnica ?i vezi munca ei ?i prinde minte!
Pro 6:7  Ea, care nu are nici mai-mare peste ea, nici îndrumator, nici sfatuitor,
Pro 6:8  Î?i pregate?te de cu vara hrana ei ?i î?i strânge la seceri? mâncare. Sau mergi la albina ?i vezi cât e de harnica ?i ce lucrare iscusita savâr?e?te. Munca ei o folosesc spre sanatate ?i regii ?i oamenii de rând. Ea e iubita ?i laudata de to?i, de?i e slaba în putere, dar e minunata cu iscusin?a.
Pro 6:9  Pâna când, lene?ule, vei mai sta culcat? Când te vei scula din somnul tau?
Pro 6:10  "Pu?in somn, înca pu?ina a?ipire, pu?in sa mai stau în pat cu mâinile încruci?ate!"
Pro 6:11  Iata vine saracia ca un trecator ?i nevoia te prinde ca un tâlhar. Dar daca nu vei lenevi, atunci va veni seceri?ul tau ca un izvor, iar lipsa va fi departe de tine.
Pro 6:12  Omul de nimic, omul necinstit ?i viclean umbla cu minciuna pe buze.
Pro 6:13  Face cu ochiul, da din picioare, face semne cu degetele.
Pro 6:14  în inima lui e vicle?ug, pururea se gânde?te la rau ?i seamana gâlceava.
Pro 6:15  Pentru aceasta fara de veste va veni peste el prapadul, nimicit va fi dintr-o data ?i fara leac.
Pro 6:16  ?ase sunt lucrurile pe care le ura?te Domnul, ba chiar ?apte de care se scârbe?te cugetul Sau:
Pro 6:17  Ochii mândri, limba mincinoasa, mâinile care varsa sânge nevinovat,
Pro 6:18  Inima care planuie?te gânduri viclene, picioare grabnice sa alerge spre rau,
Pro 6:19  Martorul mincinos care spune minciuni ?i cel care seamana vrajba între fra?i.
Pro 6:20  Paze?te, fiule, pova?a tatalui tau ?i nu lepada îndemnul maicii tale.
Pro 6:21  Leaga-le la inima ta, pururea atârna-le de gâtul tau.
Pro 6:22  Ele te vor conduce când vei vrea sa mergi; în vremea somnului te vor pazi, iar când te vei de?tepta vor grai cu tine.
Pro 6:23  Ca pova?a este un sfe?nic bun ?i legea o lumina, iar îndemnurile care dau înva?atura sunt calea vie?ii.
Pro 6:24  Ele te vor pazi de femeia vicleana, de limba cea ademenitoare a celei straine.
Pro 6:25  Nu dori frumuse?ea ei întru inima ta ?i sa nu te vâneze cu genele ei.
Pro 6:26  Ca femeia desfrânata umbla dupa o bucata de pâine, pe când femeia-so?ie dore?te un suflet de mare pre?.
Pro 6:27  Oare poate pune cineva foc în sânul lui, fara ca ve?mintele lui sa nu arda?
Pro 6:28  Sau va merge cineva pe carbuni fara sa i se friga talpile?
Pro 6:29  A?a este cu cel ce se duce la femeia aproapelui sau: nimeni din cei ce se ating de ea nu va ramâne nepedepsit.
Pro 6:30  Nimeni nu dispre?uie?te un ho? pentru ca a furat ca sa-?i astâmpere foamea;
Pro 6:31  Dar când a fost prins, el da înapoi în?eptit, întoarce tot ceea ce are în casa lui.
Pro 6:32  Cel ce se desfrâneaza însa cu o femeie este lipsit de minte, se pierde pe el însu?i facând astfel;
Pro 6:33  El nu dobânde?te decât bataie, iar ocara lui niciodata nu se ?terge.
Pro 6:34  Pizma treze?te mânia omului defaimat ?i el este fara mila în ziua razbunarii;
Pro 6:35  El nu se uita la nici un pre? de rascumparare, ?i chiar când îi vei spori darurile, tot nu se îmblânze?te.
Pro 7:1  Fiul meu; paze?te spusele mele ?i îndrumarile mele ascunde-le la tine.
Pro 7:2  Pastreaza sfaturile mele ca sa ramâi în via?a ?i orânduielile mele ca lumina ochilor tai.
Pro 7:3  Leaga-le pe degetele tale, scrie-le pe tabla inimii tale!
Pro 7:4  Spune în?elepciunii: "Tu e?ti sora mea!", ?i nume?te priceperea prietena ta,
Pro 7:5  Ca ea sa te pazeasca de femeia straina, de femeia altuia, ale carei cuvinte sunt ademenitoare.
Pro 7:6  Odata stam la fereastra casei mele ?i priveam printre gratii,
Pro 7:7  ?i am zarit printre cei lipsi?i de minte, am vazut un tânar fara pricepere.
Pro 7:8  El trecea pe uli?a pe lânga col?ul casei ei ?i se îndrepta catre locuin?a ei.
Pro 7:9  Era în amurgul serii unei zile, când se lasa umbra ?i întunericul nop?ii.
Pro 7:10  ?i iata o femeie îl întâmpina, având înfa?i?are de desfrânata ?i cu prefacatorie în inima;
Pro 7:11  Apriga ?i de ne?inut în frâu, picioarele ei nu se mai odihneau în casa;
Pro 7:12  Când în casa, când afara, stând la pânda lânga orice col?.
Pro 7:13  Ea îl apuca ?i-l saruta ?i cu o cautatura obraznica îi zise:
Pro 7:14  "Trebuia sa aduc jertfe de pace; astazi am împlinit fagaduin?ele mele;
Pro 7:15  Pentru aceasta am ie?it în întâmpinarea ta, ca sa te caut ?i iata ca te-am gasit.
Pro 7:16  Cu scoar?e am gatit patul meu, cu a?ternuturi de in din Egipt,
Pro 7:17  Cu miresme am stropit patul meu, cu mir, aloe ?i chinamon.
Pro 7:18  Vino, sa ne îmbatam de iubire pâna diminea?a, sa ne cufundam în desfatari de dragoste,
Pro 7:19  Ca barbatul meu nu este acasa, plecat-a la drum departe,
Pro 7:20  Luat-a cu dânsul o punga cu bani ?i se va întoarce acasa la luna plina!"
Pro 7:21  Ea îl ademeni prin mul?imea cuvintelor ei ?i-l smulse prin graiurile ademenitoare ale buzelor sale;
Pro 7:22  El începu sa mearga dintr-o data dupa ea, ca un bou la junghiere ?i ca un cerb care se zore?te spre capcana,
Pro 7:23  Pâna când o sageata îi strapunge ficatul; dupa cum o pasare grabe?te spre la? ?i nu-?i da seama ca acolo î?i sfâr?e?te via?a.
Pro 7:24  ?i acum, fiule, asculta-ma ?i ia aminte la cuvintele gurii mele!
Pro 7:25  Inima ta sa nu se plece spre caile ei ?i nu te rataci pe potecile ei,
Pro 7:26  Caci ea a ranit pe mul?i ?i pe foarte mul?i i-a omorât.
Pro 7:27  Casa ei sunt caile iadului, care duc la camarile mor?ii.
Pro 8:1  Oare în?elepciunea nu striga ea ?i priceperea nu-?i ridica glasul sau?
Pro 8:2  Pe vârfurile cele mai înalte, pe cale, la raspântiile drumurilor sta,
Pro 8:3  Pe lânga por?i, în împrejurimile ceta?ii, la intrarea por?ilor, striga tare:
Pro 8:4  "Catre voi, oamenilor, se îndreapta strigatul meu ?i glasul meu catre voi, fii ai oamenilor.
Pro 8:5  Voi, cei simpli, înva?a?i cumin?enia ?i voi, cei nebuni, în?elep?i?i-va!
Pro 8:6  Asculta?i, caci voi spune lucruri mare?e ?i buzele mele se deschid pentru a înal?a ceea ce este drept;
Pro 8:7  Caci gura mea graie?te adevarul ?i buzele mele se dezgusta de faradelege.
Pro 8:8  Toate graiurile gurii mele sunt întru dreptate, în ele nu este nimic sucit ?i fara rost;
Pro 8:9  Toate sunt lamurite pentru cel priceput ?i drepte pentru cei ce au aflat ?tiin?a.
Pro 8:10  Lua?i înva?atura mea mai degraba decât argintul ?i ?tiin?a mai mult decât aurul cel mai curat,
Pro 8:11  Caci în?elepciunea este mai buna decât pietrele pre?ioase ?i nici lucrurile cele mai pre?ioase nu au valoarea ei.
Pro 8:12  Eu, în?elepciunea, locuiesc împreuna cu prevederea ?i stapânesc ?tiin?a ?i buna-chibzuiala.
Pro 8:13  Frica de Dumnezeu este urgisirea raului. Mândria ?i obraznicia, calea rauta?ii ?i gura cea apriga le urasc eu.
Pro 8:14  Al meu este sfatul ?i buna-chibzuiala, eu sunt priceperea, a mea este puterea.
Pro 8:15  Prin mine împara?esc împara?ii ?i principii rânduiesc dreptatea.
Pro 8:16  Prin mine cârmuiesc dregatorii ?i mai-marii sunt judecatorii pamântului.
Pro 8:17  Eu iubesc pe cei ce ma iubesc ?i cei ce ma cauta ma gasesc.
Pro 8:18  Cu mine este boga?ia ?i marirea, averea vrednica de cinste ?i dreptatea.
Pro 8:19  Rodul meu e mai bun decât aurul ?i decât aurul cel mai curat, ?i ceea ce vine de la mine este mai de pre? decât argintul lamurit.
Pro 8:20  Merg pe calea drepta?ii, în mijlocul cailor judeca?ii drepte,
Pro 8:21  Ca sa dau celor ce ma iubesc boga?ii ?i sa le umplu camarile lor.
Pro 8:22  Domnul m-a zidit la începutul lucrarilor Lui; înainte de lucrarile Lui cele mai de demult.
Pro 8:23  Eu am fost din veac întemeiata de la început, înainte de a se fi facut pamântul.
Pro 8:24  Nu era adâncul atunci când am fost nascuta, nici chiar izvoare încarcate cu apa.
Pro 8:25  Înainte de a fi fost întemeia?i mun?ii ?i înaintea vailor eu am luat fiin?a.
Pro 8:26  Când înca nu era facut pamântul, nici câmpiile, nici cel dintâi fir de praf din lume,
Pro 8:27  Când El a întemeiat cerurile eu eram acolo; când El a tras bolta cerului peste fa?a adâncului,
Pro 8:28  Când a întarit norii sus ?i izvoarele adâncului curgeau din bel?ug,
Pro 8:29  Când El a pus hotar marii, ca apele sa nu mai treaca peste ?armuri ?i când El a a?ezat temeliile pamântului,
Pro 8:30  Atunci eu eram ca un copil mic alaturi de El, veselindu-ma în fiecare zi ?i desfatându-ma fara încetare în fa?a Lui;
Pro 8:31  Dezmierdându-ma pe rotundul pamântului Lui ?i gasindu-mi placerea printre fiii oamenilor.
Pro 8:32  ?i acum, fiilor, asculta?i-ma! Ferici?i sunt cei ce pazesc caile mele!
Pro 8:33  Asculta?i înva?atura, ca sa ajunge?i în?elep?i, ?i nu o lepada?i.
Pro 8:34  Fericit este omul care asculta de mine ?i vegheaza în fiecare zi la por?ile mele ?i cel ce strajuie?te lânga pragul casei mele!
Pro 8:35  Cel ce ma afla, a aflat via?a ?i dobânde?te har de la Domnul;
Pro 8:36  Iar cel ce pacatuie?te împotriva mea î?i pagube?te via?a lui. To?i cei ce ma urasc pe mine iubesc moartea".
Pro 9:1  În?elepciunea ?i-a zidit casa rezemata pe ?apte stâlpi,
Pro 9:2  A înjunghiat vite pentru ospa?, a pregatit vinul cu mirodenii ?i a întins masa sa.
Pro 9:3  Ea a trimis slujnicele sale sa strige pe vârfurile dealurilor ceta?ii:
Pro 9:4  "Cine este neîn?elept sa intre la mine!" ?i celor lipsi?i de buna-chibzuiala le zice:
Pro 9:5  "Veni?i ?i mânca?i din pâinea mea ?i be?i din vinul pe care eu l-am amestecat cu mirodenii.
Pro 9:6  Parasi?i neîn?elepciunea ca sa ramâne?i cu via?a ?i umbla?i pe calea cea dreapta a priceperii!"
Pro 9:7  Cel ce cearta pe batjocoritor î?i atrage dispre?ul, ?i cel ce dojene?te pe cel fara de lege î?i atrage ocara.
Pro 9:8  Nu certa pe cel batjocoritor ca sa nu te urasca; dojene?te pe cel în?elept, ?i el te va iubi.
Pro 9:9  Da sfat celui în?elept, ?i el se va face ?i mai în?elept; înva?a pe cel drept, ?i el î?i va spori ?tiin?a lui.
Pro 9:10  Începutul în?elepciunii este frica de Dumnezeu ?i priceperea este ?tiin?a Celui Sfânt.
Pro 9:11  Caci prin Domnul se vor înmul?i zilele tale ?i se vor adauga ?ie ani de via?a.
Pro 9:12  Daca tu e?ti în?elept, e?ti în?elept pentru tine, ?i daca e?ti batjocoritor, singur vei purta ponosul.
Pro 9:13  Nebunia este o femeie galagioasa, proasta ?i care nu ?tie nimic.
Pro 9:14  Ea sta la u?a casei sale, pe un scaun înalt ?i striga,
Pro 9:15  Ca sa pofteasca pe cei ce trec pe drum ?i pe cei ce merg pe calea lor fara sa se abata:
Pro 9:16  "Cine este neîn?elept sa intre la mine!" ?i celui lipsit de buna-chibzuiala îi zice:
Pro 9:17  "Apa furata e mai placuta ?i pâinea mâncata pe furi? are gust mai bun".
Pro 9:18  ?i omul nu ?tie ca acolo sunt numai umbre, iar cei pe care îi pofte?te nebunia se afla de mult în adâncurile ?eolului (locuin?a mor?ilor).
Pro 10:1  Pildele lui Solomon. Fiul în?elept învesele?te pe tatal sau, iar cel nebun este supararea maicii lui.
Pro 10:2  Nu sunt de nici un folos comorile dobândite prin faradelege; numai dreptatea scapa de la moarte.
Pro 10:3  Domnul nu lasa sa piara de foame sufletul celui drept; însa el respinge lacomia celor fara de lege.
Pro 10:4  Mâna lene?ilor pricinuie?te saracie, iar mâna celor în?elep?i aduna avu?ii.
Pro 10:5  Cel ce aduna în timpul verii este un om prevazator, iar cel care doarme în vremea seceri?ului este de ocara.
Pro 10:6  Binecuvântarea Domnului vine pe capul celui drept, iar ocara acopera fa?a celor fara de lege.
Pro 10:7  Pomenirea celui drept este spre binecuvântare, iar numele celor nelegiui?i va fi blestemat.
Pro 10:8  Cel cu inima în?eleapta prime?te sfaturile, iar cel nebun graie?te vorbe spre pieirea lui.
Pro 10:9  Cel ce umbla întru neprihanire umbla pe cale sigura, iar cel ce umbla pe cai laturalnice va fi dat de gol.
Pro 10:10  Cel ce clipe?te din ochi va fi pricina de suparare; iar cel care cearta cu inima buna a?aza pacea.
Pro 10:11  Izvor de via?a este gura celui drept, dar gura celor fara de lege, izvor de nedreptate.
Pro 10:12  Ura aduce cearta, iar dragostea acopere toate cusururile.
Pro 10:13  Pe buzele omului priceput se afla în?elepciunea; toiagul este pentru spatele celui lipsit de chibzuin?a.
Pro 10:14  Cei în?elep?i ascund ?tiin?a, iar gura celui fara de socotin?a este o nenorocire apropiata.
Pro 10:15  Avu?ia este pentru cel bogat o cetate întarita; nenorocirea celor sarmani este saracia lor.
Pro 10:16  Agonisita celui drept este spre via?a; roadele celui fara de lege spre pacat;
Pro 10:17  Cel ce paze?te înva?atura apuca pe calea vie?ii, iar cel ce leapada certarea ratace?te.
Pro 10:18  Cel care ascunde ura are buze mincinoase; cel ce raspânde?te defaimarea este un nebun.
Pro 10:19  Mul?imea cuvintelor nu scute?te de pacatuire, iar cel ce-?i ?ine buzele lui este un om în?elept.
Pro 10:20  Limba omului drept este argint curat, dar inima celor fara de lege este lucru de pu?in pre?.
Pro 10:21  Buzele celui drept calauzesc pe mul?i oameni, iar cei nebuni mor din pricina ca nu sunt pricepu?i.
Pro 10:22  Numai binecuvântarea Domnului îmboga?e?te, iar truda zadarnica nu aduce spor.
Pro 10:23  Ca o pricina de bucurie este pentru nebun savâr?irea unei fapte ru?inoase; la fel este cu în?elepciunea pentru omul priceput.
Pro 10:24  De ceea ce se teme cel nelegiuit nu scapa, iar cererea celor drep?i (Dumnezeu) o împline?te.
Pro 10:25  Precum trece furtuna, a?a piere ?i cel fara de lege, iar dreptul este ca o temelie neclintita.
Pro 10:26  Precum este o?etul pentru din?i ?i fumul pentru ochi, a?a este omul lene? pentru cei ce-l pun la treaba.
Pro 10:27  Frica de Dumnezeu lunge?te zilele (omului), iar anii celor fara de lege sunt pu?ini.
Pro 10:28  Nadejdea celor drep?i este numai bucurie, iar nadejdea celor pacato?i sfâr?e?te în rau.
Pro 10:29  Calea Domnului este o întaritura pentru cel desavâr?it ?i o prabu?ire pentru cei ce savâr?esc faradelegi.
Pro 10:30  Niciodata cel drept nu se va clatina, iar cei nelegiui?i nu vor locui pamântul.
Pro 10:31  Gura celui drept rode?te în?elepciune, iar limba urzitoare de rele aduce pierzare.
Pro 10:32  Buzele celui drept cunosc bunavoirea, iar gura pacato?ilor strâmbatatea.
Pro 11:1  Cântarul strâmb este urgisit de Domnul, ?i cântarirea dreapta este placerea Lui.
Pro 11:2  Daca vine mândria, va veni ?i ocara, iar în?elepciunea este cu cei smeri?i.
Pro 11:3  Neprihanirea poarta pe cei drep?i, iar strâmbatatea prapade?te pe cei vicleni.
Pro 11:4  La nimic nu folose?te boga?ia în ziua mâniei; numai dreptatea izbave?te de moarte.
Pro 11:5  Dreptatea neteze?te calea celui fara prihana, iar cel fara de lege va cadea prin faradelegea lui.
Pro 11:6  Dreptatea izbave?te pe cei drep?i, iar cei vicleni vor fi prin?i prin pofta lor.
Pro 11:7  La moartea omului drept ramâne nadejdea, iar la moartea celui pacatos piere nadejdea.
Pro 11:8  Dreptul scapa din strâmtorare, ?i cel fara de lege îi ia locul.
Pro 11:9  Faptuitorul de rele prabu?e?te cu gura pe aproapele lui, iar prin ?tiin?a celor drep?i va fi mântuit.
Pro 11:10  De propa?irea celor drep?i cetatea se bucura, iar când pier cei fara de lege ea tresalta de veselie.
Pro 11:11  Prin binecuvântarea oamenilor drep?i cetatea merge înainte, iar prin gura celor nelegiui?i ajunge ruina.
Pro 11:12  Cel nepriceput urgise?te pe aproapele lui, iar omul cu buna-chibzuiala tace.
Pro 11:13  Graitorul de rele da pe fa?a lucruri de taina, iar omul cu duhul cumpanit le ?ine ascunse.
Pro 11:14  Unde lipse?te cârmuirea, poporul cade; izbavirea sta în mul?imea sfetnicilor.
Pro 11:15  Celui ce se pune cheza? pentru un strain îi merge rau; cel ce nu se pune cheza? sta la adapost.
Pro 11:16  Femeia cu purtare buna agonise?te cinstire, iar cea care ura?te cinstea e o ru?ine. Nu lene?ii ci silitorii agonisesc avere.
Pro 11:17  Omul milostiv î?i face bine sufletului sau, pe când cel fara mila î?i chinuie?te trupul sau.
Pro 11:18  Cel nelegiuit capata un câ?tig în?elator, iar cel ce seamana dreptatea, o rasplata adevarata.
Pro 11:19  Cel ce umbla dupa dreptate ajunge la via?a, iar cel ce fuge dupa rau, la moarte.
Pro 11:20  Pe cei cu inima vicleana îi urgise?te Domnul; placerea Lui este spre cei fara prihana.
Pro 11:21  Încetul cu încetul pacatosul nu va ramâne nepedepsit, iar neamul celor drep?i va fi mântuit.
Pro 11:22  Inel de aur în râtul porcului, a?a este femeia frumoasa ?i fara minte.
Pro 11:23  Dorin?a celor drep?i este bine; nadejdea celor fara de lege este mânia lui Dumnezeu.
Pro 11:24  Unul da mereu ?i se îmboga?e?te, altul se zgârce?te afara din cale ?i sarace?te.
Pro 11:25  Cel ce binecuvinteaza va fi îndestulat, iar cel ce blesteama va fi blestemat.
Pro 11:26  Cel ce ?ine grâul este blestemat de popor, iar binecuvântarea (se revarsa) peste capul celui ce îl vinde.
Pro 11:27  Cel ce cauta binele dobânde?te bunavoin?a Domnului, iar cel ce umbla dupa rau va da peste el.
Pro 11:28  Cel ce-?i pune nadejdea în boga?ia lui se ve?teje?te, iar cei drep?i ca frunzi?ul odraslesc.
Pro 11:29  Cine î?i tulbura casa lui culege vânt, iar cel nebun va fi sluga celui în?elept.
Pro 11:30  Rodul drepta?ii este un pom al vie?ii, iar silnicia nimice?te via?a.
Pro 11:31  Daca cel drept este rasplatit pe pamânt, cu cât mai mult cel nelegiuit ?i pacatos!
Pro 12:1  Cel ce iube?te înva?atura iube?te ?tiin?a, iar cel ce ura?te certarea este nebun.
Pro 12:2  Cel bun dobânde?te har de la Domnul, iar pe omul viclean îl osânde?te Domnul.
Pro 12:3  Omul nu se întare?te întru faradelegea lui; radacina celor drep?i nu se va clatina niciodata.
Pro 12:4  Femeia virtuoasa este o cununa pentru barbatul ei, iar femeia fara cinste este un cariu în oasele lui.
Pro 12:5  Socotelile celor drep?i sunt dreptatea, iar punerile la cale ale celor nelegiui?i în?elaciunea.
Pro 12:6  Graiurile celor nelegiui?i sunt curse de moarte, iar gura celor drep?i îi scapa pe ei din primejdie.
Pro 12:7  Cei fara de lege numai cât se întorc ?i nu mai sunt, dar casa drep?ilor dainuie?te de-a pururi.
Pro 12:8  Omul aste pre?uit dupa priceperea lui, iar cel nepriceput este urgisit.
Pro 12:9  Mai mult pre?uie?te un om smerit dar harnic, decât unul mândru dar lipsit de pâine.
Pro 12:10  Cel drept are mila de vite, iar inima celui rau este fara îndurare.
Pro 12:11  Cel ce munce?te ogorul sau se satura de pâine, iar cel ce umbla dupa de?ertaciuni este om lipsit de minte.
Pro 12:12  Nelegiuitul pofte?te prada celor rai, dar radacina celor drep?i da rodul sau.
Pro 12:13  Prin pacatul buzelor se prinde în la? pacatosul, iar dreptul (prin dreptatea lui) scapa din strâmtorare.
Pro 12:14  Din rodul gurii sale se satur; de cele bune omul, ?i fiecaruia i se rasplate?te dupa faptele lui.
Pro 12:15  Calea celui nebun este dreapta în ochii lui, iar cel în?elept asculta de sfat.
Pro 12:16  Nebunul da pe fa?a îndata mânia lui, iar omul prevazator î?i ascunde ocara.
Pro 12:17  Cel ce spune adevarul veste?te dreptatea, iar martorul mincinos umbla cu în?elaciunea.
Pro 12:18  Cei nechibzui?i la vorba sunt ca împunsaturile de sabie, pe când limba celor în?elep?i aduce tamaduire.
Pro 12:19  Buzele care spun adevarul vor dainui totdeauna, iar limba graitoare de minciuna numai pentru o clipa.
Pro 12:20  În?elaciunea este în inima celor ce gândesc rau, iar bucuria pentru cei ce dau sfaturi de pace.
Pro 12:21  Nici o nenorocire nu se întâmpla celui drept, pe când cei nelegiui?i sunt covâr?i?i de rele.
Pro 12:22  Buzele cele graitoare de minciuna sunt urâciune înaintea Domnului, iar cei ce faptuiesc dupa adevar sunt placerea Lui.
Pro 12:23  Omul în?elept î?i ascunde ?tiin?a, pe când inima celor nebuni propovaduie?te nebunia.
Pro 12:24  Mâna celor silitori va stapâni, iar cea lasatoare va fi birnica.
Pro 12:25  Supararea se abate asupra omului, dar numai un cuvânt bun îl bucura.
Pro 12:26  Dreptul cerceteaza cu de-amanuntul pe prietenul sau; calea celor nelegiui?i duce la ratacire.
Pro 12:27  Lene?ul nu-?i frige nici vânatul lui; cea mai scumpa comoara pentru om este munca.
Pro 12:28  Pe cararea drepta?ii este via?a ?i pe calea pe care ea o însemneaza; nemurirea, iar calea nebuniei duce la moarte.
Pro 13:1  Fiul în?elept asculta de înva?atura tatalui sau, iar cel batjocoritor nici de mustrare.
Pro 13:2  Din rodul gurii sale omul manânca binele; pofta celor vicleni este silnicia.
Pro 13:3  Cine î?i paze?te gura î?i paze?te sufletul sau; cel ce deschide prea tare buzele o face spre pieirea lui.
Pro 13:4  Sufletul celui lene? pofte?te, însa în zadar. Numai sufletul celor silitori este îndestulat.
Pro 13:5  Dreptul ura?te cuvintele mincinoase; ticalosul aduce numai ru?ine ?i ocara.
Pro 13:6  Dreptatea paze?te calea omului fara prihana, iar faradelegea e pricina ruinii celui pacatos.
Pro 13:7  Unii se dau drept boga?i ?i n-au nimic, al?ii trec drept saraci, cu toate ca au multe averi.
Pro 13:8  Boga?ia cuiva sluje?te la rascumpararea lui; cel sarac nu se teme nici chiar de amenin?are.
Pro 13:9  Lumina celor drep?i lumineaza, pe când sfe?nicul celor fara de lege se stinge.
Pro 13:10  Mândria nu da prilej decât la cearta, în?elepciunea se afla numai la cei ce primesc sfaturi.
Pro 13:11  Boga?ia adunata în graba se împu?ineaza, numai cel ce-o aduna pe încetul o înmul?e?te.
Pro 13:12  A?teptarea prea îndelungata îmbolnave?te inima, iar dorin?a împlinita este pom al vie?ii.
Pro 13:13  Cel ce nu ia în seama cuvântul (lui Dumnezeu) este dat pierzarii, iar cel ce se teme de porunca Lui este rasplatit.
Pro 13:14  Înva?atura celui în?elept este izvor de via?a, ca sa putem scapa de cursele mor?ii.
Pro 13:15  Buna în?elegere rode?te har; calea celor vicleni este spre pieirea lor.
Pro 13:16  Orice om în?elept lucreaza cu chibzuin?a, numai cel nebun î?i desfa?oara nebunia.
Pro 13:17  Un sol ticalos cade în nenorocire, iar unul credincios aduce alinare.
Pro 13:18  De saracie ?i de ru?ine are parte cel ce nesocote?te certarea, iar cel ce prime?te mustrarea va fi cinstit.
Pro 13:19  Dorin?a împlinita mul?ume?te sufletul, iar departarea de rau este urâciune pentru cei nebuni.
Pro 13:20  Cel ce se înso?e?te cu cei în?elep?i ajunge în?elept, iar cel ce se întovara?e?te cu cei nebuni se face rau.
Pro 13:21  Nenorocirea urmare?te pe cei pacato?i, iar fericirea rasplate?te pe cei drep?i.
Pro 13:22  Omul bun lasa mo?tenirea sa nepo?ilor sai, iar averea celui pacatos este sortita pentru cei drep?i.
Pro 13:23  Aratura în ?elina, facuta de cei saraci, da hrana din bel?ug, insa averea se pierde din pricina nedrepta?ii.
Pro 13:24  Cine cru?a toiagul sau î?i ura?te copilul, iar cel care îl iube?te îl cearta la vreme.
Pro 13:25  Dreptul manânca ?i î?i îndestuleaza sufletul sau, iar pântecele celor fara de lege duce lipsa.
Pro 14:1  Femeile în?elepte zidesc casa, iar cele nebune o darâma cu mâna lor.
Pro 14:2  Cel ce umbla întru dreptate se teme de Domnul, iar cel ce umbla pe cai strâmbe îl dispre?uie?te.
Pro 14:3  În gura celui nebun este varga mândriei lui; buzele pe cei în?elep?i îi pazesc.
Pro 14:4  Unde nu sunt boi, staulul este gol, însa puterea boilor aduce mult folos.
Pro 14:5  Martorul care graie?te adevarul nu minte, iar martorul mincinos spune numai minciuni.
Pro 14:6  Batjocoritorul cauta în?elepciunea ?i nu o gase?te, iar pentru cel priceput ?tiin?a este u?oara.
Pro 14:7  Fugi dinaintea omului fara de minte, caci tu ?tii ca nu este nici o ?tiin?a pe buzele lui.
Pro 14:8  În?elepciunea omului chibzuit este de a-?i în?elege calea lui; iar nebunia celor neîn?elep?i este în?elaciune.
Pro 14:9  Nebunul î?i bate joc de jertfa pentru pacat, însa între oamenii drep?i este buna în?elegere.
Pro 14:10  Inima cunoa?te amaraciunile sale, iar un strain nu poate împar?i bucuriile ei.
Pro 14:11  Casa celor fara de lege va fi distrusa, iar cortul celor drep?i va înflori.
Pro 14:12  Unele cai par drepte în ochii omului, dar sfâr?itul lor sunt caile mor?ii.
Pro 14:13  Chiar când râdem, inima se întristeaza; bucuria se sfâr?e?te prin plângere.
Pro 14:14  Nelegiuitul se va satura de caile sale ?i omul bun de roadele sale.
Pro 14:15  Omul simplu crede toate vorbele; omul în?elept vegheaza pa?ii sai.
Pro 14:16  În?eleptul se teme ?i se fere?te de rau, iar cel fara de minte î?i iese din fire ?i se simte la adapost.
Pro 14:17  Cel iute la mânie savâr?e?te nebunii, iar cel cumpanit se stapâne?te.
Pro 14:18  Cei nepricepu?i au parte de nebunie, pe când cei în?elep?i sunt încununa?i cu ?tiin?a.
Pro 14:19  Cei rai se pleaca înaintea celor buni ?i cei nelegiui?i stau la por?ile celor drep?i.
Pro 14:20  Saracul este dispre?uit chiar ?i de prietenul lui, pe când prietenii celui bogat sunt fara de numar.
Pro 14:21  Cel care nu baga în seama pe prietenul sau savâr?e?te un pacat; iar cel ce se îndura de sarmani e fericit.
Pro 14:22  Cu adevarat ratacesc cei ce planuiesc faradelegea, iar cei ce cugeta la lucruri bune au parte de milostivire ?i de adevar.
Pro 14:23  Orice osteneala duce la îndestulare, iar cuvintele fara rost la lipsa.
Pro 14:24  Boga?ia este o cununa pentru cei în?elep?i; iar coroana color nebuni este nebunia.
Pro 14:25  Martorul drept scapa suflete, iar cel viclean spune numai minciuni.
Pro 14:26  Întru frica lui Dumnezeu este nadejdea celui tare; fiii lui afla-vor (acolo) un liman.
Pro 14:27  Frica de Dumnezeu este un izvor de via?a, ca sa putem scapa de cursele mor?ii.
Pro 14:28  Stralucirea unui rege se sprijina pe mul?imea poporului, iar lipsa de supu?i este pieirea prin?ului.
Pro 14:29  Cel încet la mânie este bogat în în?elepciune, iar cel ce se mânie degraba î?i da pe fa?a nebunia.
Pro 14:30  O inima fara patima este via?a trupului, pe când pornirea patima?a este ca un cariu în oase.
Pro 14:31  Cel care apasa pe cel sarman defaima pe Ziditorul lui, dar cel ce are mila de sarac Îl cinste?te.
Pro 14:32  Cel fara de lege este rasturnat de rautatea lui, iar cel drept gase?te scapare în neprihanirea lui.
Pro 14:33  În?elepciunea sala?luie?te în inima celui în?elept, iar în inima celor nebuni nu se arata.
Pro 14:34  Dreptatea înal?a un popor, în vreme ce pacatul este ocara popoarelor.
Pro 14:35  Bunavoin?a regelui este pentru sluga în?eleapta, iar mânia lui pentru cel ce îi face ru?ine.
Pro 15:1  Un raspuns blând domole?te mânia, iar un cuvânt aspru a?â?a mânia.
Pro 15:2  Limba celor în?elep?i picura ?tiin?a, ?ar gura celor nebuni revarsa prostie.
Pro 15:3  Ochii Domnului sunt pretutindeni, veghind asupra celor buni ?i asupra celor rai.
Pro 15:4  Limba dulce este pom al vie?ii, iar limba vicleana zdrobe?te inima.
Pro 15:5  Nebunul nu ia în seama înva?atura tatalui sau, iar cine trage folos din certare se face mai în?elept.
Pro 15:6  În casa celui drept sunt comori fara de numar; în câ?tigul celui fara de lege este tulburare.
Pro 15:7  Buzele celor în?elep?i raspândesc ?tiin?a, dar inima celor nebuni nu.
Pro 15:8  Jertfa celor fara de lege este urâciune înaintea Domnului, iar rugaciunea celor drep?i este placerea Lui.
Pro 15:9  Calea celui nelegiuit este urâciune înaintea Domnului, dar El iube?te pe cel ce umbla dupa dreptate.
Pro 15:10  O certare aspra capata ,el ce parase?te cararea; cel ce urgise?te mustrarea va muri.
Pro 15:11  Iadul ?i adâncul sunt cunoscute Domnului, cu atât mai vârtos inimile fiilor oamenilor.
Pro 15:12  Celui batjocoritor nu-i place dojana; de aceea el nu se îndreapta spre cei în?elep?i.
Pro 15:13  O inima vesela însenineaza fa?a, iar când inima e trista ?i duhul e fara de curaj.
Pro 15:14  Inima în?eleapta cauta ?tiin?a, iar gura celor nebuni se simte mul?umita cu nebunia.
Pro 15:15  Toate zilele celui sarac sunt rele, dar inima mul?umita este un ospa? necurmat.
Pro 15:16  Mai bine pu?in întru frica lui Dumnezeu, decât vistierie mare ?i tulburare (multa).
Pro 15:17  Mai mult face o mâncare de verde?uri ?i cu dragoste, decât un bou îngra?at ?i cu ura.
Pro 15:18  Omul mânios a?â?a cearta, pe când cel domol lini?te?te aprinderea.
Pro 15:19  Calea celui lene? e ca un gard de spini, iar calea celui silitor e neteda.
Pro 15:20  Fiul în?elept bucura pe tatal sau, iar fiul nebun nu baga în seama pe maica lui.
Pro 15:21  Nebunia este o bucurie pentru omul fara minte, iar cel în?elept merge pe calea dreapta.
Pro 15:22  Punerile la cale nu se înfaptuiesc unde lipse?te chibzuirea, dar ele î?i iau fiin?a cu mul?i sfatuitori.
Pro 15:23  Omul se bucura pentru un raspuns bun ie?it din gura lui ?i cât e de buna vorba spusa la locul ei!
Pro 15:24  În?eleptul merge pe cararea vie?ii ce duce în sus, ca sa ocoleasca drumul iadului care merge în jos.
Pro 15:25  Domnul prabu?e?te casa celor mândri ?i întare?te hotarul vaduvei.
Pro 15:26  Gândurile cele rele sunt urâciune înaintea Domnului, iar cuvintele frumoase sunt curate (în ochii Lui).
Pro 15:27  Cel ce umbla dupa câ?tig nedrept î?i surpa casa lui, iar cel ce ura?te mita va trai.
Pro 15:28  Inima celui drept chibzuie?te ce sa raspunda, iar gura celor nelegiui?i împra?tie rauta?i.
Pro 15:29  Domnul se ?ine departe de cei nelegiui?i, dar asculta rugaciunea celor drep?i.
Pro 15:30  O privire binevoitoare învesele?te inima ?i o veste buna întare?te oasele.
Pro 15:31  Urechea care asculta o dojana folositoare vie?ii î?i are loca?ul printre cei în?elep?i.
Pro 15:32  Cel ce leapada mustrarea î?i urgise?te sufletul sau, iar cel ce ia aminte la dojana dobânde?te în?elepciune.
Pro 15:33  Frica de Dumnezeu este înva?atura ?i în?elepciune, iar smerenia trece înaintea maririi.
Pro 16:1  În putere sta omului sa plasmuiasca planuri în inima, dar raspunsul limbii vine de la Domnul.
Pro 16:2  Toate caile omului sunt curate în ochii lui, dar numai Domnul este Cel ce cerceteaza duhul.
Pro 16:3  Înfa?i?eaza Domnului lucrarile tale ?i gândurile tale vor izbuti.
Pro 16:4  Pe toate le-a facut Domnul fiecare cu ?elul sau, la fel ?i pe nelegiuit pentru ziua nenorocirii.
Pro 16:5  Toata inima semea?a este urâciune înaintea Domnului; hotarât, ea nu va ramâne nepedepsita.
Pro 16:6  Prin iubire ?i credin?a se ispa?e?te pacatul ?i prin frica de Dumnezeu te fere?ti de rele.
Pro 16:7  Când caile omului sunt placute înaintea Domnului, chiar ?i pe vrajma?ii lui îi sile?te la pace.
Pro 16:8  Mai degraba pu?in ?i cu dreptate, decât agoniseala multa cu strâmbatate.
Pro 16:9  Inima omului gânde?te la calea lui, dar numai Domnul poarta pa?ii lui.
Pro 16:10  Hotarâri dumnezeie?ti sunt pe buzele împaratului; la darea hotarârii sa nu se în?ele gura lui!
Pro 16:11  Cântarul ?i cumpana dreapta sunt de la Domnul; toate greuta?ile de cântarit sunt lucrarea Lui.
Pro 16:12  Urâciune sunt regii care faptuiesc faradelegea, fiindca numai întru dreptate se întare?te tronul.
Pro 16:13  Buzele graitoare de dreptate sunt placute regilor ?i iubesc pe cel ce spune drept.
Pro 16:14  Întarâtarea regelui este ca vestitorii mor?ii, dar omul în?elept o domole?te.
Pro 16:15  Seninatatea fe?ei regelui da via?a ?i bunavoin?a lui este ca un nor de ploaie de primavara.
Pro 16:16  Dobândirea în?elepciunii este mai buna decât aurul, iar câ?tigarea priceperii este mai de pre? decât argintul.
Pro 16:17  Calea celor drep?i este ferirea de rau; numai acela care ia aminte la mersul lui î?i paze?te sufletul sau.
Pro 16:18  Înaintea prabu?irii merge trufia ?i seme?ia înaintea caderii.
Pro 16:19  Mai bine sa fii smerit cu cei smeri?i decât sa împar?i prada cu cei mândri.
Pro 16:20  Cel care ia aminte la cuvânt afla fericirea, iar cel ce se încrede în Domnul este fericit.
Pro 16:21  Cel ce este în?elept se cheama priceput; dulcea?a cuvintelor de pe buzele lui înmul?e?te ?tiin?a.
Pro 16:22  În?elepciunea este un izvor de via?a pentru cine o are; pedeapsa celui nebun e nebunia.
Pro 16:23  Inima celui în?elept da în?elepciune gurii lui ?i pe buzele sale spore?te ?tiin?a.
Pro 16:24  Cuvintele frumoase sunt un fagure de miere, dulcea?a pentru suflet ?i tamaduire pentru oase.
Pro 16:25  Multe cai i se par bune omului, dar la capatul lor încep caile mor?ii.
Pro 16:26  Foamea îndeamna pe lucrator la munca, fiindca gura lui îl sile?te.
Pro 16:27  Omul viclean pregate?te nenorocirea ?i pe buzele lui este ca un foc arzator.
Pro 16:28  Omul cu gând rau a?â?a cearta ?i defaimatorul desparte p? prieteni.
Pro 16:29  Omul asupritor amage?te pe prietenul sau ?i îl îndreapta pe un drum care nu este bun.
Pro 16:30  Cel care închide din ochi urze?te viclenii; cine î?i mu?ca buzele a ?i savâr?it raul.
Pro 16:31  Batrâne?ea este o cununa stralucita; ea se afla mergând pe calea cuvio?iei.
Pro 16:32  Cel încet la mânie e mai de pre? decât un viteaz, iar cel ce î?i stapâne?te duhul este mai gre?uit decât cuceritorul unei ceta?i.
Pro 16:33  Se arunca sor?ii în pulpana hainei, dar hotarârea toata vine de la Domnul.
Pro 17:1  Mai buna este o bucata de pâine uscata în pace, decât o casa plina cu carne de jertfe, dar cu vrajba.
Pro 17:2  Un slujitor în?elept e mai presus decât un fiu aducator de ru?ine; acela va împar?i mo?tenirea eu fra?ii.
Pro 17:3  în topitoare se lamure?te argintul ?i în cuptor aurul, dar cel ce încearca inimile este Domnul.
Pro 17:4  Facatorul de rele ia aminte la buzele nedrepte, mincinosul pleaca urechea la limba cea rea.
Pro 17:5  Cel ce î?i bate joc de sarac defaima pe Ziditorul lui; cel ce se bucura de o nenorocire nu va ramâne nepedepsit.
Pro 17:6  Cununa batrânilor sunt nepo?ii, iar marirea fiilor sunt parin?ii lor.
Pro 17:7  Nebunului nu-i sunt dragi cuvintele alese, cu atât mai mult unui om de seama vorbele mincinoase.
Pro 17:8  Piatra nestemata este darul în ochii celui ce-l are; oriunde se întoarce (totul) îi merge în plin.
Pro 17:9  Cel ce acopera un pacat cauta prietenie, iar cine scoate la iveala un lucru (uitat) desparte pe prieteni.
Pro 17:10  Certarea înrâure?te mai adânc pe omul în?elept, decât o suta de lovituri pe cel nebun.
Pro 17:11  Omul rau a?â?a razvratirea; pentru aceasta un sol aprig va fi trimis împotriva lui.
Pro 17:12  Mai degraba sa întâlne?ti o ursoaica lipsita de puii ei, decât un nebun în nebunia lui.
Pro 17:13  Cel ce rasplate?te cu rau pentru bine nu va vedea departându-se nenorocirea din casa lui.
Pro 17:14  Începutul unei certe e ca slobozirea apei (dintr-un iezer); înainte de a se aprinde, da-te la o parte!
Pro 17:15  Cel care achita pe cel vinovat ?i cel ce osânde?te pe cel drept, amândoi sunt urâciune înaintea Domnului.
Pro 17:16  Ce folos aduc banii în mâna celui nebun? Ar putea dobândi în?elepciune, dar nu are pricepere.
Pro 17:17  Prietenul iube?te în orice vreme, iar în nenorocire el e ca un frate.
Pro 17:18  Omul fara pricepere se prinde prin darnicia mâinii lui; el se pune cheza? pentru aproapele lui.
Pro 17:19  Cine iube?te certurile iube?te pacatul; cel ce ridica glasul î?i iube?te ruina.
Pro 17:20  Cel ce are o inima vicleana nu afla fericirea ?i cel ce are o limba ?ireata da peste necaz.
Pro 17:21  Cel ce da na?tere unui nebun va avea o mare suparare ?i parintele nu are nici o bucurie pentru un fiu sarit din minte.
Pro 17:22  O inima vesela este un leac minunat, pe când un duh fara curaj usuca oasele.
Pro 17:23  Nelegiuitul prime?te daruri scoase (pe ascuns) din sân, ca sa abata cararile drepta?ii.
Pro 17:24  Omul priceput are înaintea ochilor lui în?elepciunea, iar ochii celui nebun se uita la capatul pamântului.
Pro 17:25  Feciorul nebun este necaz pentru tatal lui ?i amaraciune pentru maica lui.
Pro 17:26  Nu se cuvine sa pui la plata pe omul drept, nici sa osânde?ti pe cei nevinova?i din pricina ca umbla drept.
Pro 17:27  Cel ce î?i stapâne?te cuvintele sale are ?tiin?a ?i cel ce î?i ?ine cumpatul este un om priceput.
Pro 17:28  Chiar ?i nebunul, când tace, trece drept în?elept; când închide gura este asemenea unui om cuminte.
Pro 18:1  Cel ce se rine deoparte cauta sa-?i mul?umeasca pornirea patima?a; împotriva oricarui sfat în?elept se porne?te.
Pro 18:2  Celui nebun nu-i place în?elepciunea, ci darea pe fa?a a gândurilor lui.
Pro 18:3  Când vine cel nelegiuit, vine ?i defaimarea ?i o data cu ru?inea ?i batjocura.
Pro 18:4  Vorbele (ie?ite) din gura omului sunt ape fara fund; izvorul în?elepciunii este un ?uvoi care da peste maluri.
Pro 18:5  Nu este bine sa cau?i la fa?a celui fara de lege ?i sa nu faci dreptate celui drept la judecata.
Pro 18:6  Buzele celui nebun duc la cearta ?i gura lui da na?tere la ocari!
Pro 18:7  Gura celui nebun este prabu?irea lui ?i buzele lui sunt un la? pentru sufletul lui.
Pro 18:8  Vorbele defaimatorului sunt ca ni?te mâncari alese; ele coboara în camarile pântecelui.
Pro 18:9  Omul lasator pentru lucrul lui e frate cu cel care darâma.
Pro 18:10  Turn puternic este numele Domnului; cel drept la el î?i afla scaparea ?i este la adapost.
Pro 18:11  Averea celui bogat este o cetate tare pentru el, iar în închipuirea lui ca un zid înalt.
Pro 18:12  Înaintea prabu?irii vine trufia inimii, iar înaintea maririi merge umilin?a.
Pro 18:13  Cel ce raspunde la vorba înainte de a fi ascultat-o este nebun ?i încurcat la minte.
Pro 18:14  Curajul omului îl întare?te în vreme de suferin?a, iar pe un om lipsit de barba?ie, cine-l va ridica?
Pro 18:15  O inima priceputa dobânde?te ?tiin?a ?i urechea celor în?elep?i umbla dupa iscusin?a.
Pro 18:16  Darul adus de un om îi large?te (calea lui) ?i-l poarta înaintea celor mari.
Pro 18:17  Pârâtul pare ca are dreptate în pricina sa, iar când vine pârâ?ul, atunci se ia la cercetare.
Pro 18:18  Sor?ul face sa înceteze sfada ?i hotara?te între cei puternici.
Pro 18:19  Frate ajutat de frate este ca o cetate tare ?i înalta ?i are putere ca o împara?ie întemeiata.
Pro 18:20  Din rodul gurii omului se satura pântecele lui; din ceea ce da buzele lui se îndestuleaza.
Pro 18:21  În puterea limbii este via?a ?i moartea ?i cei ce o iubesc manânca din rodul ei.
Pro 18:22  Cel ce gase?te o femeie buna afla un lucru de mare pre? ?i dobânde?te dar de la Dumnezeu.
Pro 18:23  Saracul vorbe?te rugator, iar cel bogat raspunde cu îndrazneala.
Pro 18:24  Sunt prieteni aducatori de nenorocire; dar este ?i câte un prieten mai apropiat decât un frate.
Pro 19:1  Mai de pre? este saracul care umbla întru neprihanirea lui, decât un bogat cu buze viclene ?i nebun.
Pro 19:2  Ne?tiin?a sufletului nu este buna iar cel ce umbla repede da gre?.
Pro 19:3  Nebunia omului darâma calea lui ?i inima lui se mânie împotriva Domnului.
Pro 19:4  Boga?ia strânge prieteni fara de numar, iar saracul se desparte chiar de prietenul sau.
Pro 19:5  Martorul mincinos nu va ramâne nepedepsit ?i cel ce spune lucruri neadevarate nu va scapa.
Pro 19:6  Mul?i sunt cei ce lingu?esc pe un om darnic ?i to?i sunt prieteni ai celui ce da daruri.
Pro 19:7  To?i fra?ii celui sarac îl urasc; cât de mult prietenii lui se departeaza de el! El cauta vorbe (mângâietoare), dar nu le afla.
Pro 19:8  Cel ce dobânde?te în?elepciune iube?te sufletul sau ?i cel ce ?ine cu tarie la pricepere afla fericirea.
Pro 19:9  Martorul mincinos nu ramâne nepedepsit ?i cel ce spune lucruri neadevarate se va prabu?i.
Pro 19:10  Nu-i sta bine celui fara de minte sa traiasca în desfatari, cu atât mai pu?in unui rob sa conduca peste capetenii.
Pro 19:11  În?elepciunea domole?te mânia omului ?i faima lui este iertarea gre?elilor.
Pro 19:12  Furia unui rege e ca racnetul unui leu, iar bunavoin?a lui este ca roua pe iarba.
Pro 19:13  Un fiu neascultator este nenorocirea tatalui lui, iar certurile unei femei un jgheab care curge întruna.
Pro 19:14  O casa ?i o avere sunt mo?tenire de la parin?i, iar o femeie în?eleapta este un dar de la Dumnezeu.
Pro 19:15  Lenea te face sa cazi în toropeala; sufletului trândav îi va fi foame.
Pro 19:16  Cel ce ia seama la porunca î?i pastreaza sufletul sau, iar cel ce dispre?uie?te cuvântul (Domnului) va muri.
Pro 19:17  Cel ce are mila de sarman împrumuta Domnului ?i El îi va rasplati fapta lui cea buna.
Pro 19:18  Pedepse?te pe feciorul tau, cât mai este nadejde (de îndreptare), dar nu ajunge pâna acolo ca sa-l omori.
Pro 19:19  Omul aprig la mânie trebuie sa fie pedepsit; daca îl cru?i o data, trebuie sa începi din nou.
Pro 19:20  Asculta sfatul ?i prime?te înva?atura, ca sa fii în?elept toata via?a ta.
Pro 19:21  Multe puneri la cale framânta inima omului, dar numai sfatul Domnului se împline?te.
Pro 19:22  Omul se face placut prin marinimia lui; mai de pre? este un sarac bun decât un om mincinos.
Pro 19:23  Frica de Dumnezeu duce la via?a ?i ne îndestulam fara sa fim lovi?i de nenorocire.
Pro 19:24  Lene?ul întinde mâna în blid ?i nu are putere s-o duca la gura.
Pro 19:25  Love?te pe cel ce batjocore?te ?i cel fara de minte va deveni în?elept; mustra pe cel în?elept ?i el va pricepe ?tiin?a.
Pro 19:26  Cel ce se poarta rau cu tatal sau ?i alunga (din casa) pe mama sa, este fiu aducator de ocara ?i de ru?ine.
Pro 19:27  Înceteaza, fiul meu, sa ascul?i ademenirea ?i sa te la?i îndepartat de înva?aturile în?elepte.
Pro 19:28  Martorul de nimic î?i bate joc de dreptate ?i gura celor fara de lege înghite nelegiuirea.
Pro 19:29  Pentru batjocoritori sunt gata toiege; loviturile sunt pentru spinarea celor nebuni.
Pro 20:1  Un ocarâtor este vinul, un zurbagiu bautura îmbatatoare ?i oricine se lasa ademenit nu este în?elept.
Pro 20:2  Groaza pe care o insufla regele este ca racnetul leului; cel ce îl întarâta pacatuie?te împotriva sa însu?i.
Pro 20:3  Este o mare însu?ire pentru om sa se stapâneasca de la cearta ?i tot nebunul se întarâta.
Pro 20:4  Toamna lene?ul nu lucreaza, iar când vine sa culeaga rodul la seceri?, nimic nu afla.
Pro 20:5  O apa adânca este sfatul în inima omului, iar omul de?tept ?tie s-o scoata.
Pro 20:6  Mul?i oameni se lauda cu marinimia, dar un prieten adevarat cine-l afla?
Pro 20:7  Omul drept umbla pe calea lui fara prihana; ferici?i sunt copiii care vin dupa el!
Pro 20:8  Un rege care sta pe scaunul de judecata deosebe?te cu ochii lui orice fapta rea.
Pro 20:9  Cine poate spune: Cura?it-am inima mea; sunt curat de pacat?
Pro 20:10  Doua feluri de greuta?i de cântarit ?i de masurat sunt urâciune înaintea Domnului.
Pro 20:11  Copilul se da pe fa?a din lucrarile lui, daca purtarea lui este fara prihana ?i dreapta.
Pro 20:12  Urechea care aude ?i ochiul care vede, pe amândoua le-a zidit Domnul.
Pro 20:13  Nu iubi somnul, ca sa nu ajungi sarac; ?ine ochii deschi?i, numai a?a vei fi îndestulat de pâine.
Pro 20:14  "Rau, rau!", zice cumparatorul, iar dupa ce pleaca se lauda.
Pro 20:15  Chiar daca ai aur ?i pietre pre?ioase, dar o podoaba fara seaman sunt buzele chibzuite.
Pro 20:16  Ia-i haina ?i fiindca s-a pus cheza? pentru altul în locul celor straini, ia-l zalog.
Pro 20:17  Buna e la gust pâinea agonisita cu în?elaciune, dar dupa aceea gura se umple de pietricele.
Pro 20:18  Planurile se întaresc prin sfaturi; lupta-te cu luare aminte.
Pro 20:19  Cine tradeaza taina umbla ca un defaimator ?i nu te întovara?i cu cel ce are mereu buzele deschise.
Pro 20:20  Cel ce blesteama pe tatal sau ?i pe mama sa stinge sfe?nicul în mijlocul întunericului.
Pro 20:21  O mo?tenire repede câ?tigata de la început, la urma va fi fara binecuvântare.
Pro 20:22  Nu spune: "Vreau sa rasplatesc cu rau!" Nadajduie?te în Domnul ?i El i?i va veni în ajutor.
Pro 20:23  Greuta?ile nedrepte pentru cântarire sunt urâciune înaintea Domnului ?i cântarele în?elatoare nu sunt decât un lucru rau.
Pro 20:24  De Domnul sunt hotarâ?i pa?ii omului, caci cum ar putea omul sa priceapa calea lui?
Pro 20:25  O cursa este pentru om sa afieroseasca Domnului ceva în graba ?i dupa ce a fagaduit sa-i para rau.
Pro 20:26  Un rege în?elept simte pe cei fara de lege ?i lasa sa treaca roata peste ei.
Pro 20:27  Sufletul omului este un sfe?nic de la Domnul; el cerceteaza toate camarile trupului.
Pro 20:28  Iubirea ?i credin?a pazesc pe rege ?i prin iubire î?i sprijina tronul sau.
Pro 20:29  Faima celor tineri este puterea lor ?i podoaba celor batrâni parul lor carunt.
Pro 20:30  Ranile sângeroase sunt un leac pentru cel raufacator ?i lovituri care patrund pâna înlauntrul trupului.
Pro 21:1  Asemenea unui curs de apa este inima regelui în mâna Domnului, pe care îl îndreapta încotro vrea.
Pro 21:2  Toata calea omului este dreapta în ochii lui, dar numai Domnul cântare?te inimile.
Pro 21:3  Faptuirea drepta?ii ?i a judeca?ii este mai de pre? pentru Domnul decât jertfa sângeroasa.
Pro 21:4  Ochii seme?i ?i inima îngâmfata sunt sfe?nicul pacato?ilor. Aceasta nu este decât pacat.
Pro 21:5  Chibzuiala omului silitor duce numai la câ?tig, iar cel ce se zore?te ajunge la paguba.
Pro 21:6  Comorile dobândite cu limba mincinoasa sunt de?ertaciune trecatoare ?i la?uri ale mor?ii.
Pro 21:7  Silnicia celor fara de lege se ?ine dupa ei, caci nu voiesc sa înfaptuiasca dreptatea.
Pro 21:8  Calea celui raufacator e sucita; cel nevinovat lucreaza drept.
Pro 21:9  Mai degraba sa locuie?ti într-un col? pe acoperi?, decât cu o femeie certarea?a ?i într-o casa mare.
Pro 21:10  Sufletul celui fara de lege pofte?te rautatea, iar aproapele lui nu afla mila în ochii lui.
Pro 21:11  Când cel batjocoritor e pedepsit, cel fara minte se în?elep?e?te ?i când cel în?elept este dojenit, el câ?tiga în ?tiin?a.
Pro 21:12  Dreptul ia aminte la casa celui nelegiuit. Dumnezeu prabu?e?te pe cei fara de lege în nenorocire.
Pro 21:13  Cine î?i astupa urechea la strigatul celui sarman ?i el, când va striga, nu i se va raspunde.
Pro 21:14  Un dar facut într-ascuns potole?te mânia ?i un plocon scos din sân, o mânie puternica.
Pro 21:15  Dreptul tresalta de bucurie când poate sa puna în fapta dreptatea, iar spaima e pentru cei ce savâr?esc faradelegea.
Pro 21:16  Un om care ratace?te de pe drumul în?elepciunii, se va odihni curând în adunarea celor mor?i.
Pro 21:17  Cel ce iube?te veselia va duce lipsa; cel caruia îi place vinul ?i miresmele nu se îmboga?e?te.
Pro 21:18  Nelegiuitul sluje?te ca pre? de rascumparare pentru cel drept ?i vicleanul pentru cel fara prihana.
Pro 21:19  Mai bine sa locuie?ti în pustiu decât cu o femeie certarea?a ?i suparacioasa.
Pro 21:20  Comori de pre? ?i untdelemn (se gasesc) în casa celui în?elept, dar omul cel nebun le risipe?te.
Pro 21:21  Cel ce umbla în calea drepta?ii ?i a milei afla via?a, dreptate ?i marire.
Pro 21:22  În?eleptul ia cu lupta crunta cetatea vitejilor ?i rastoarna întariturile în care ei î?i puneau nadejdea.
Pro 21:23  Cel ce-?i paze?te gura ?i limba lui î?i paze?te sufletul lui de primejdie.
Pro 21:24  Cine este seme? ?i îngâmfat se cheama batjocoritor; acela se poarta cu prisos de trufie.
Pro 21:25  Pofta celui lene? îl omoara, caci mâinile nu voiesc sa lucreze.
Pro 21:26  Mereu cel fara de lege pofte?te, iar cel drept da ?i nu se zgârce?te.
Pro 21:27  Jertfa celor nelegiui?i este urâciune pentru Domnul, mai cu seama când o aduc pentru o fapta ru?inoasa.
Pro 21:28  Martorul mincinos va pieri, iar omul care asculta va putea vorbi totdeauna.
Pro 21:29  Raufacatorul are o privire neru?inata, iar omul cel drept î?i ia aminte la purtarea lui.
Pro 21:30  Nu este nici în?elepciune, nici pricepere ?i nici sfat care sa aiba putere înaintea Domnului.
Pro 21:31  Calul este gata pentru ziua de razboi, însa biruin?a vine de la Domnul.
Pro 22:1  Un nume (bun) este mai de pre? decât boga?ia; cinstea este mai pre?ioasa decât argintul ?i decât aurul.
Pro 22:2  Bogatul ?i saracul se întâlnesc unul cu altul; dar Cine i-a facut este Domnul.
Pro 22:3  Cel iscusit vede nenorocirea ?i se ascunde, cei simpli trec mai departe ?i sufera.
Pro 22:4  Rodul umilin?ei ?i a temerii de Dumnezeu sunt: boga?ia, marirea ?i via?a.
Pro 22:5  Maracini ?i curse sunt în calea celui viclean; cel ce î?i fere?te sufletul lui sa da la o parte de ele.
Pro 22:6  Deprinde pe tânar cu purtarea pe care trebuie s-o aiba; chiar când va îmbatrâni nu se va abate de la ea.
Pro 22:7  Bogatul stapâne?te pe cei saraci, ?i cel ce împrumuta este slujitor celui de la care se împrumuta.
Pro 22:8  Cel ce seamana nedreptatea secera nenorocire, iar toiagul mâniei lui îl va bate pe el.
Pro 22:9  Omul blând va fi binecuvântat, caci din pâinea ui da celui sarac.
Pro 22:10  Alunga pe cel batjocoritor ?i cearta va lua sfâr?it ?i pricina ?i defaimarea vor înceta.
Pro 22:11  Cel ce iube?te cura?ia inimii ?i ale carui buze sunt ( pline) de vorbe alese are de prieten pe conducator.
Pro 22:12  Ochii Domnului pazesc ?tiin?a ?i darâma cuvintele celui fara de lege.
Pro 22:13  Cel lene? pune pricini ?i zice: "Afara este un leu, a? putea sa fiu sugrumat în mijlocul uli?elor".
Pro 22:14  O groapa fara fund este gura femeilor straine; cel ce este lovit de mânia Domnului cade în ea.
Pro 22:15  Daca nebunia se pripa?e?te în inima celui tânar, numai varga certarii o va îndeparta de el.
Pro 22:16  Daca împilezi pe un sarac, î?i înmul?e?ti averea, daca dai unui bogat sarace?ti.
Pro 22:17  Pleaca urechea ta ?i asculta cuvintele celor iscusi?i ?i inima ta îndreapt-o spre ?tiin?a mea.
Pro 22:18  Este placut daca tu le pastrezi înlauntrul tau. O, de-ar sta toate pe buzele tale!
Pro 22:19  Pentru a-?i pune nadejdea în Domnul, vreau sa-?i dau înva?atura astazi.
Pro 22:20  Oare nu ?i-am a?ezat în scris în nenumarate rânduri sfaturi ?i înva?aturi,
Pro 22:21  Ca sa-?i fac cunoscut credincio?ia cuvintelor adevarate ?i sa raspunzi prin cuvinte de buna credin?a, celor ce te întreaba?
Pro 22:22  Nu jefui pe sarac, pentru ca el e sarac ?i nu asupri pe cel nenorocit la poarta (ceta?ii),
Pro 22:23  Caci Domnul va apara pricina lor ?i va ridica via?a celor care îi vor fi jefuit.
Pro 22:24  Nu te întovara?i cu omul mânios ?i cu cel înfierbântat de furie sa n-ai nici un amestec,
Pro 22:25  Ca sa nu te deprinzi pe calea lui ?i sa-?i întinzi o cursa pentru via?a ta.
Pro 22:26  Nu fi dintre aceia care dau mâna, care se pun cheza?i pentru datorii.
Pro 22:27  Daca nu ai cu ce plati, pentru ce te învoie?ti ca sa ?i se ia ?i patul de sub tine?
Pro 22:28  Nu muta hotarul stravechi pe care l-au însemnat parin?ii tai.
Pro 22:29  Vezi tu un om dibaci la lucrul lui? El va sta înaintea conducatorilor ?i nu înaintea oamenilor de rând.
Pro 23:1  Când stai la masa cu un dregator, ia seama pe cine ai înaintea ta;
Pro 23:2  Pune-?i un cu?it la gât, daca tu e?ti lacom.
Pro 23:3  Nu pofti bucatele lui, caci sunt mâncari în?elatoare.
Pro 23:4  Nu te osteni sa ajungi bogat; nu-?i pune iscusin?a ta în aceasta.
Pro 23:5  Oare vrei sa te ui?i cu ochii cum ea se risipe?te? Caci boga?ia face aripi ca un vultur care se înal?a catre cer.
Pro 23:6  Nu mânca pâinea celui ce se uita cu ochi rai ?i nu pofti bucatele lui,
Pro 23:7  Caci el î?i numara buca?ile din gura. "Manânca ?i bea!", î?i va spune, dar inima lui nu e pentru tine.
Pro 23:8  Bucata pe care ai mâncat-o o vei da afara din tine, iar tu ?i-ai risipit (zadarnic) vorbele tale alese.
Pro 23:9  Nu grai la urechea celui nebun, caci el nu va baga de seama iscusin?a graiurilor tale.
Pro 23:10  Nu muta hotarul vaduvei ?i nu încalca ogorul celor orfani,
Pro 23:11  Caci Ocrotitorul lor e tare ?i El va apara pricina lor împotriva ta.
Pro 23:12  Sile?te la înva?atura inima ta ?i urechea ta la cuvinte iscusite.
Pro 23:13  Nu cru?a pe feciorul tau de pedeapsa; chiar daca îl love?ti cu varga, nu moare.
Pro 23:14  Tu îl ba?i cu toiagul, dar scapi sufletul lui din împara?ia mor?ii.
Pro 23:15  Fiul meu, daca inima ta e plina de în?elepciune ?i inima mea se va bucura.
Pro 23:16  Rarunchii mei vor tresari de bucurie, când buzele tale vor grai ceea ce este drept.
Pro 23:17  Sa nu râvneasca inima ta la cei pacato?i, ci totdeauna sa ramâna la frica de Domnul,
Pro 23:18  Caci daca o vei pazi pe ea, mai ai ?i tu un viitor ?i nadejdea ta nu se va pierde.
Pro 23:19  Asculta fiul meu, ?i te în?elep?e?te ?i îndreapta inima ta pe calea cea dreapta.
Pro 23:20  Nu fi printre cei ce se îmbata de vin ?i printre cei ce î?i desfrâneaza trupul lor,
Pro 23:21  Caci be?ivul ?i desfrânatul saracesc, iar dormitul mereu te face sa por?i zdren?e.
Pro 23:22  Asculta pe tatal tau care te-a nascut ?i nu dispre?ui pe mama ta când ea a ajuns batrâna.
Pro 23:23  Aduna adevar ?i nu-l vinde, în?elepciune ?i înva?atura ?i buna chibzuiala.
Pro 23:24  Tatal celui drept tresalta de bucurie ?i cel ce a dat na?tere unui în?elept se bucura de el.
Pro 23:25  Sa se bucure tatal ?i mama ta ?i sa salte de veselie cea care te-a nascut!
Pro 23:26  Da-mi, fiule, mie inima ta, ?i ochii tai sa simta placere pentru caile mele,
Pro 23:27  Caci femeia desfrânata este o groapa adânca ?i cea straina un pu? strâmt.
Pro 23:28  Pentru aceasta ea sta ca un ho? la pânda ?i spore?te printre oameni numarul celor în?ela?i de ea.
Pro 23:29  Pentru cine sunt suspinele, pentru cine vaicarelile, pentru cine gâlcevile, pentru cine plânsetele, pentru cine ranile fara pricina, pentru cine ochii întrista?i?
Pro 23:30  Pentru cei ce zabovesc pe lânga vin, pentru cei ce vin sa guste bauturi cu mirodenii.
Pro 23:31  Nu te uita la vin cum este el de ro?u, cum scânteiaza în cupa ?i cum aluneca pe gât,
Pro 23:32  Caci la urma el ca un ?arpe mu?ca ?i ca o vipera împroa?ca venin.
Pro 23:33  Daca ochii tai vor privi la femei straine ?i gura ta va grai lucruri me?te?ugite,
Pro 23:34  Vei fi ca unul care sta culcat în mijlocul marii, ca unul care a adormit pe vârful unui catarg.
Pro 23:35  "M-au lovit... Nu m-a durut! M-au batut... Nu ?tiu nimic! Când ma voi de?tepta din somn, voi cere iara?i vin".
Pro 24:1  Nu râvni la oamenii rai ?i nu pofti sa fii în tovara?ia lor,
Pro 24:2  Caci inima lor pune la cale lucruri silnice ?i buzele lor graiesc cele nelegiuite.
Pro 24:3  Prin în?elepciune se ridica o casa, prin buna chibzuiala se întare?te
Pro 24:4  ?i prin ?tiin?a se umplu camarile ei de tot felul de avu?ie scumpa ?i placuta.
Pro 24:5  Mai puternic este un în?elept decât un voinic ?i cel priceput decât unul plin de putere.
Pro 24:6  Cu oricâta dibacie te vei razboi, biruin?a se dobânde?te cu mul?i sfatuitori.
Pro 24:7  Peste masura de înalta este în?elepciunea pentru omul nebun; când sta la poarta (ceta?ii) el nu deschide gura.
Pro 24:8  Cel ce-?i pune în gând sa faca rau se cheama un mare raufacator.
Pro 24:9  Gândul celui nebun nu este decât pacat; batjocoritorul este urgia oamenilor.
Pro 24:10  Daca te ara?i slab în ziua strâmtorarii, puterea ta nu este decât slabiciune.
Pro 24:11  Izbave?te pe cei ce sunt târâ?i la moarte ?i pe cei ce se duc clatinându-se la junghiere scapa-i!
Pro 24:12  Daca vrei sa spui: "Iata n-am ?tiut nimic!", oare Cel ce cântare?te inimile nu patrunde cu privirea ?i Cel ce vegheaza peste sufletul tau nu ?tie ?i nu va rasplati omului dupa faptele lui?
Pro 24:13  Fiul meu, manânca miere, caci e buna ?i un fagure de miere este dulce gurii tale.
Pro 24:14  Sa ?tii ca în?elepciunea este la fel pentru sufletul tau; daca o dobânde?ti, ai un viitor, iar nadejdea ta nu este pierduta.
Pro 24:15  Nu pândi, nelegiuitule, casa celui drept ?i nu tulbura loca?ul lui,
Pro 24:16  Caci daca cel drept cade de ?apte ori ?i tot se scoala, cei fara de lege se poticnesc în nenorocire.
Pro 24:17  Nu te bucura când cade vrajma?ul tau ?i, când se poticne?te, sa nu se veseleasca inima ta,
Pro 24:18  Ca nu cumva sa vada Domnul ?i sa fie neplacut în ochii Lui ?i sa nu întoarca mânia Sa de la el (spre tine).
Pro 24:19  Nu te aprinde împotriva raufacatorului ?i nu-?i întarâta râvna împotriva celor fara de lege.
Pro 24:20  Caci cel ce face rau nu propa?e?te, ?i sfe?nicul celor nelegiui?i se va stinge.
Pro 24:21  Fiul meu, teme-te de Domnul ?i de rege ?i cu cei ce se razvratesc nu lega prietenie,
Pro 24:22  Ca fara de veste va veni nenorocirea ?i cine poate sa cunoasca sfâr?itul lor naprasnic?
Pro 24:23  ?i aceste (proverbe) sunt ale în?elep?ilor: Nu e bine ca la judecata sa cau?i la fa?a oamenilor.
Pro 24:24  Pe cel ce zice celui fara de lege: "Tu e?ti drept!", popoarele îl blesteama ?i neamurile îl afurisesc;
Pro 24:25  Dar celor care îl cearta Cum se cuvine le merge bine ?i peste ei vine binecuvântarea ?i fericirea.
Pro 24:26  Buzele saruta pe cei ce dau raspunsuri drepte.
Pro 24:27  Rânduie?te-?i lucrul tau afara ?i adu-l la îndeplinire pe câmpul tau, apoi î?i vei ridica o casa.
Pro 24:28  Nu fi martor mincinos împotriva prietenului tau ?i nu fi pricina (unei hotarâri nedrepte), cu buzele tale.
Pro 24:29  Nu spune: "Precum mi-a facut a?a îi voi face ?i eu lui; voi rasplati omului dupa faptele lui".
Pro 24:30  Am trecut pe ogorul unui lene? ?i pe la via unui om lipsit de minte,
Pro 24:31  ?i iata spinii cre?teau în toate locurile, maracinii o acopereau cu totul, iar zidul de pietre se prabu?ise.
Pro 24:32  Atunci m-am uitat ?i m-am framântat în inima mea, am privit cu luare aminte ?i am tras o înva?atura:
Pro 24:33  "Înca pu?in somn, înca pu?ina a?ipeala, înca pu?in sa mai stau cu mâinile în sân ca sa dorm..."
Pro 24:34  ?i saracia va veni peste tine ca un calator ?i lipsa ca un om înarmat.
Pro 25:1  ?i acestea sunt pildele lui Solomon, pe care le-au adunat oamenii lui Iezechia, regele lui Iuda.
Pro 25:2  Slava lui Dumnezeu este sa ascunda lucrurile, iar marirea regilor e sa le cerceteze cu de-amanuntul.
Pro 25:3  Precum înal?imea cerului ?i adâncul pamântului sunt lucruri nepatrunse, tot a?a ?i inima regilor.
Pro 25:4  Cura?a argintul de zgura ?i turnatorul va face din el un vas ales.
Pro 25:5  Da la o parte pe cel fara de lege din fa?a mai-marelui ?i tronul lui se va întari prin dreptate.
Pro 25:6  Nu te fali înaintea cârmuitorului ?i nu sta în locul hotarât pentru cei mari,
Pro 25:7  Caci mai degraba sa ?i se zica: "Suie aici!" decât sa te umileasca în fa?a stapânului. Ceea ce au vazut ochii tai,
Pro 25:8  Nu aduce grabnic spre disputa, caci ce ai sa faci dupa aceea când aproapele tau te va da de ru?ine?
Pro 25:9  Cearta-te cu aproapele tau, dar taina altuia sa nu o dai pe fa?a,
Pro 25:10  Ca nu cumva cel ce o aude sa nu te defaime ?i sa nu darâme (pentru totdeauna) faima ta.
Pro 25:11  Ca merele de aur pe poli?i de argint, a?a este cuvântul spus la locul lui.
Pro 25:12  Inel de aur ?i podoabe de aur de mult pre? este pova?uitorul în?elept la urechea ascultatoare.
Pro 25:13  Precum este racoarea zapezii în vremea seceri?ului, a?a este solul credincios pentru cei ce-l trimit; el bucura sufletul stapânului sau.
Pro 25:14  Precum sunt norii ?i vântul fara ploaie, a?a este omul care se lauda cu darul pe care niciodata nu-l da.
Pro 25:15  Prin rabdare se poate îndupleca un om mânios ?i o limba dulce înmoaie oase.
Pro 25:16  Daca ai gasit miere, manânca atât cit î?i trebuie, ca nu cumva sa te saturi ?i s-o ver?i.
Pro 25:17  Pune rar piciorul în casa prietenului tau, ca nu cumva sa se sature de tine ?i sa te urasca.
Pro 25:18  Un ciocan, o sabie ?i o sageata ascu?ita este omul care da marturie mincinoasa împotriva aproapelui sau.
Pro 25:19  Dinte rau ?i picior ?ovaitor este cel fara credin?a în vreme de nevoie.
Pro 25:20  Ca atunci când dezbraci haina ne vreme friguroasa, sau torni o?et pe silitra, a?a este cântarea pentru o inima întristata.
Pro 25:21  De flamânze?te vrajma?ul tau, da-i sa manânce pâine ?i daca înseteaza, adapa-l eu apa,
Pro 25:22  Ca numai a?a îi îngramade?ti carbuni aprin?i pe capul lui ?i Domnul î?i v a rasplati ?ie.
Pro 25:23  Vântul de la miazanoapte aduce ploaie ?i limba clevetitoare aduce o fa?a mâhnita.
Pro 25:24  Mai bine sa sala?luie?ti într-un col? de acoperi?, decât sa traie?ti cu o femeie certarea?a într-o casa mare.
Pro 25:25  Precum e apa rece pentru un suflet însetat, a?a e vestea buna dintr-o ?ara departata.
Pro 25:26  Ca un izvor tulbure ?i stricat, a?a este dreptul care ?ovaie în fa?a celui nelegiuit.
Pro 25:27  Precum celui care manânca multa miere nu-i merge bine, tot a?a ?i celui care se lasa cople?it de cuvinte de lauda.
Pro 25:28  Asemenea unei ceta?i cu o spartura ?i fara zid, a?a este omul caruia îi lipse?te stapânirea de sine.
Pro 26:1  Precum este zapada în timpul verii ?i ploaia la seceri?, a?a nu-i place celui nebun cinstea.
Pro 26:2  Precum vrabia zboara ?i rândunica se înal?a în vazduh, tot a?a blestemul fara pricina nu nimere?te.
Pro 26:3  Biciul este bun pentru cal, frâul pentru magar, iar varga pentru spatele celor nebuni.
Pro 26:4  Nu raspunde nebunului dupa nebunia lui, ca sa nu te asemeni ?i tu cu el.
Pro 26:5  Raspunde nebunului dupa nebunia lui, ca sa nu se creada în?elept în ochii lui.
Pro 26:6  Cel ce încredin?eaza solia în mâna celui nebun î?i taie picioarele ?i bea nedreptate.
Pro 26:7  Precum nu poate sa se foloseasca slabanogul de picioarele sale, tot a?a nici cei nebuni de cuvintele cele în?elepte.
Pro 26:8  Ca ?i când pui o piatra în pra?tie, a?a este cel ce da cinste unui nebun.
Pro 26:9  Precum un ghimpe intra în mâna unui be?iv, tot a?a sunt cuvintele în?elepte în gura celor pacato?i.
Pro 26:10  Ca un arca? care rane?te pe to?i, a?a este cel ce se pune cheza? pentru cel nebun ?i pentru cei ce trec pe cale.
Pro 26:11  Ca un câine care se întoarce unde a varsat, a?a este omul nebun care se întoarce la nebunia lui.
Pro 26:12  Daca vezi un om care se crede în?elept în ochii lui, sa nadajduie?ti mai mult de la un nebun decât de la el.
Pro 26:13  Lene?ul zice: "Pe drum trece un leu, un leu pe uli?e!"
Pro 26:14  Precum u?a se suce?te în ?â?âna, tot a?a ?i lene?ul în patul lui.
Pro 26:15  Lene?ul baga mâna în blid, dar cu mare greutate o duce la gura.
Pro 26:16  Lene?ul se crede în?elept în ochii lui, mai mult decât ?apte sfetnici în?elep?i.
Pro 26:17  Ca cel ce prinde un câine de urechi, a?a este cel ce se vâra într-o cearta în care nu este amestecat.
Pro 26:18  Ca unul care arunca sage?i arzatoare, lanci, sage?i ?i moarte,
Pro 26:19  A?a e omul care în?ala pe prietenul sau ?i zice: "Da, am glumit!"
Pro 26:20  Când nu mai sunt lemne se stinge focul ?i daca nu mai este nici un defaimator se potole?te cearta.
Pro 26:21  Carbunii slujesc pentru caldura, lemnele pentru foc, iar omul certare? pentru a a?â?a cearta.
Pro 26:22  Vorbele celui defaimator sunt ca bucatele gustoase; ele se duc în adâncul maruntaielor.
Pro 26:23  Spoiala de argint care îmbraca un vas de lut, a?a sunt buzele mieroase ?i o inima rea.
Pro 26:24  Cu buzele sale se preface cel ce ura?te, iar înlauntrul lui nutre?te în?elaciune;
Pro 26:25  Când î?i schimba glasul, sa nu-l crezi, caci ?apte urâciuni sunt în inima lui.
Pro 26:26  Cineva poate sa-?i ascunda ura lui prin prefacatorie, dar în adunare rautatea lui se da pe fa?a.
Pro 26:27  Cine sapa groapa (altuia) cade singur în ea ?i cel ce rostogole?te o piatra se pravale?te (tot) peste el.
Pro 26:28  Limba mincinoasa ura?te adevarul ?i gura lingu?itorilor pricinuie?te prabu?irea.
Pro 27:1  Nu te lauda cu ziua de mâine, ca nu ?tii la ce poate da na?tere.
Pro 27:2  Sa te laude altul ?i nu gura ta, un strain ?i nu buzele tale.
Pro 27:3  Piatra este grea ?i nisipul cu anevoie de ridicat; însa furia nebunului este mai grea decât amândoua.
Pro 27:4  Întarâtarea este cruda ?i mânia apriga, dar taria pizmei cine o va putea îndura?
Pro 27:5  Mai mult pre?uie?te o dojana pe fa?a decât o dragoste ascunsa.
Pro 27:6  De buna credin?a sunt ranile pricinuite de un prieten, iar sarutarile celui ce te ura?te sunt viclene.
Pro 27:7  Satulul calca mierea în picioare, iar flamândului tot ce este amar (i se pare) dulce.
Pro 27:8  Ca o pasare gonita din cuibul ei, a?a este omul izgonit din casa sa.
Pro 27:9  Untdelemnul ?i miresmele înveselesc inima, dar tot a?a de dulci sunt sfaturile unui prieten care pornesc din suflet.
Pro 27:10  Pe prietenul tau ?i pe prietenul tatalui tau nu-i parasi; în casa fratelui tau nu intra în ziua restri?tii tale. Mai bun e un vecin aproape de tine, decât un frate departe.
Pro 27:11  Fii în?elept, fiul meu, ?i bucura inima mea, ca sa pot raspunde celui ce ma clevete?te.
Pro 27:12  În?eleptul vede nenorocirea ?i se ascunde, cei pro?ti dau peste ea ?i îndura necaz.
Pro 27:13  Ia-i haina caci s-a pus cheza? pentru altul ?i cere-i zalog din pricina celor straini.
Pro 27:14  Celui ce binecuvânteaza pe prietenul sau cu glas mare dis-de-diminea?a, i se socote?te ca un blestem.
Pro 27:15  Un jgheab care curge în vreme de ploaie ?i o femeie ar?agoasa sunt la fel;
Pro 27:16  Cel care vrea s-o opreasca opre?te vânt ?i mâna lui cea dreapta parca ar ?ine în ea untdelemn.
Pro 27:17  Fierul cu fier se ascute ?i un om ascute mânia altui om.
Pro 27:18  Cel ce paze?te un smochin manânca din rodul lui, iar cel ce paze?te pe stapânul sau va fi rasplatit cu cinste.
Pro 27:19  Precum nu se aseamana fa?a cu fa?a, tot a?a inima unui om cu inima altuia.
Pro 27:20  Iadul ?i adâncul nu se pot satura, tot a?a ?i inima omului e de nesaturat.
Pro 27:21  În topitoare se lamure?te argintul ?i în cuptor aurul, iar omul se pre?uie?te dupa numele cel bun.
Pro 27:22  Chiar daca vei pisa în piuli?a cu pilugul pe cel nebun, întocmai ca pe boabe, tot nu-l vei despar?i de nebunia lui.
Pro 27:23  Sârguie?te-te sa-?i cuno?ti oile tale ?i ia seama la turma ta,
Pro 27:24  Ca bunastarea nu dainuie?te de-a pururi ?i nici boga?ia din neam în neam.
Pro 27:25  Când iarba s-a trecut ?i pa?unea s-a ispravit ?i finul de pe dealuri s-a strâns,
Pro 27:26  Tu ai miei pentru îmbracamintea ta ?i ?api ea sa plate?ti pa?unea;
Pro 27:27  ?i laptele de capra îl ai cu îndestulare, pentru hrana casei, ?i merinde pentru slujnicele tale.
Pro 28:1  Cel nelegiuit fuge fara ca nimeni sa-l urmareasca, iar dreptul sta ca un pui de leu fara grija.
Pro 28:2  Din pricina gre?elilor unui om silnic se ivesc certuri, iar omul iscusit le stinge.
Pro 28:3  Un om bogat, care asupre?te pe cei saraci, e ca ploaia care trânte?te tot la pamânt, iar pâinea nu se face.
Pro 28:4  Cei ce parasesc legea ridica în slavi pe pacato?i, iar cei ce o pazesc se aprind împotriva lor.
Pro 28:5  Oamenii rai nu pricep nimic din ceea ce e drept, iar cei ce cauta pe Domnul în?eleg tot.
Pro 28:6  Mai de pre? e saracul care umbla întru neprihanirea lui, decât cel prefacut în caile lui, chiar daca e bogat.
Pro 28:7  Cel ce paze?te legea este un fiu în?elept, iar cel ce se întovara?e?te cu clevetitorii face ru?ine tatalui sau.
Pro 28:8  Cel ce î?i spore?te averea lui, prin dobânda ?i prin camata, aduna pentru cel ce are mila de saraci.
Pro 28:9  Cel ce î?i opre?te urechea de la ascultarea legii, chiar rugaciunea lui e urâciune.
Pro 28:10  Cel ce ratace?te pe cei drep?i pe o cale rea va cadea în groapa (pe care a sapat-o); cei fara prihana vor fi ferici?i.
Pro 28:11  Omul bogat este în?elept în ochii lui, dar cel sarac ?i priceput îl dovede?te cu mintea.
Pro 28:12  Când drep?ii biruiesc e mare sarbatoare, iar când cei fara de lege ies la iveala, oamenii se ascund.
Pro 28:13  Cel ce î?i ascunde pacatele lui nu propa?e?te, iar cel ce le marturise?te ?i se lasa de ele va fi miluit.
Pro 28:14  Fericit este omul care se teme totdeauna, iar cel ce î?i învârto?eaza inima lui va cadea în nenorocire.
Pro 28:15  Leu care racne?te ?i urs flamând este cel rau care stapâne?te peste un popor sarac.
Pro 28:16  Stapânitorul cel lipsit de venituri este mare asupritor; cel ce ura?te câ?tigul (nedrept) va trai multa vreme.
Pro 28:17  Un om pe care îl îngreuiaza sângele unui ucis fuge pâna la groapa; nimeni sa nu-l opreasca!
Pro 28:18  Cel ce umbla fara prihana va fi mântuit, iar cine apuca pe cai strâmbe va cadea într-o groapa.
Pro 28:19  Cel ce lucreaza pamântul lui se va îndestula de pâine, iar cel ce umbla dupa lucruri de nimic se va satura de saracie.
Pro 28:20  Omul credincios va fi încarcat de binecuvântari, iar cine zore?te sa ajunga bogat nu va ramâne nepedepsit.
Pro 28:21  Nu este bine sa te ui?i la fa?a omului, caci pentru o bucata de pâine cineva poate sa gre?easca.
Pro 28:22  Omul lacom se grabe?te sa se îmboga?easca, dar nu gânde?te ca lipsa va veni peste el.
Pro 28:23  Cel care cearta pe un om va avea mai multa mul?umire decât cel care-l lingu?e?te.
Pro 28:24  Cine despoaie pe tatal sau ?i pe mama sa ?i zice: "Nu-i pacat!" este tovara? cu facatorul de rele.
Pro 28:25  Omul lacom a?â?a cearta, iar cel ce nadajduie?te în Domnul va fi îndestulat.
Pro 28:26  Cel ce î?i pune nadejdea în inima lui este un nebun, iar cel ce se conduce dupa în?elepciune, acela va fi mântuit.
Pro 28:27  Cine da la cel sarac nu duce lipsa; iar cine î?i acopera ochii lui va fi mult blestemat.
Pro 28:28  Când nelegiui?ii ies la iveala, oamenii se ascund, iar când ei pier, se înmul?esc cei drep?i.
Pro 29:1  Un om pedepsit îndelung ?i tare la cerbice va fi intr-o clipa zdrobit ?i fara vindecare.
Pro 29:2  Când drep?ii domnesc se bucura poporul ?i, când stapânesc cei fara de lege, suspina.
Pro 29:3  Cine iube?te în?elepciunea bucura pe tatal sau, iar cine umbla cu desfrânatele î?i prapade?te averea.
Pro 29:4  Un conducator prin dreptate face sa propa?easca ?ara, iar cel ce pune dari grele o ruineaza.
Pro 29:5  Omul care lingu?e?te pe prietenul sau întinde cursa pa?ilor lui.
Pro 29:6  Pe calea celui rau este întins un la?, dar dreptul trebuie sa fuga ?i sa salte de bucurie.
Pro 29:7  Omul drept se îngrije?te de pricina celor sarmani; celui fara de lege nu-i pasa de ei.
Pro 29:8  Batjocoritorii rascoala cetatea, iar cei în?elep?i potolesc mânia.
Pro 29:9  Când un în?elept se cearta cu un nebun, fie ca se supara, fie ca râde, nu-?i pierde cumpatul.
Pro 29:10  Oamenii seto?i de sânge urasc pe cel fara prihana, iar cei drep?i ocrotesc via?a lui.
Pro 29:11  Nebunul face sa izbucneasca pornirea lui patima?a, iar în?eleptul î?i înfrâneaza mânia.
Pro 29:12  Când un conducator asculta de cuvinte mincinoase, to?i slujitorii sai sunt rai.
Pro 29:13  Saracul ?i cu cel ce asupre?te pe cei saraci se întâlnesc; Cel ce lumineaza ochii amândurora este Domnul.
Pro 29:14  Un conducator care judeca cu dreptate pe cei saraci î?i întare?te scaunul lui pe veci.
Pro 29:15  Varga ?i certarea aduc în?elepciune, iar tânarul care este lasat (în voia apucaturilor lui) face ru?ine maicii sale.
Pro 29:16  Când cei fara de lege domnesc se înmul?esc rauta?ile, iar drep?ii vor vedea (cu bucurie) prabu?irea lor.
Pro 29:17  Mustra pe fiul tau ?i el i?i va fi odihna ?i î?i va face placere sufletului tau.
Pro 29:18  Fara vedenie de prooroc poporul e fara stapân, dar fericit este cel care paze?te legea!
Pro 29:19  Sluga nu se îndreapta numai cu pove?e, fiindca, de?i pricepe, însa nu asculta.
Pro 29:20  Daca vezi un om care se zore?te la vorba, atunci pentru un nebun e mai multa nadejde decât pentru el.
Pro 29:21  Daca (vreun stapân) dezmiarda din copilarie pe robul sau, acesta ajunge la sfâr?it sa se creada fiu.
Pro 29:22  Un om mânios a?â?a cearta ?i cel aprig savâr?e?te multe pacate.
Pro 29:23  Mândria umile?te pe om, iar de cinste are parte cel smerit.
Pro 29:24  Cel ce împarte cu ho?ul î?i ura?te sufletul lui, fiindca aude blestemul, dar nu zice nimic.
Pro 29:25  Teama de oameni duce la caderea în cursa, dar cel ce nadajduie?te în Domnul sta la adapost.
Pro 29:26  Mul?i cauta fa?a stapânitorului, dar dreptatea omului vine de la Domnul.
Pro 29:27  Omul nedrept este urâciune pentru cei drep?i, iar cel drept este o urâciune pentru cei rai.
Pro 30:1  Cuvintele lui Agur, fiul lui Iache din Massa. Acest om a zis: "Sunt ostenit, Dumnezeule, sunt obosit, Doamne, sunt sleit de puteri!
Pro 30:2  Caci sunt tare prost, ca sa ma pot socoti ca om ?i nu am pricepere (care ar putea sa fie vrednica) de un om.
Pro 30:3  Nici n-am înva?at în?elepciunea ?i nici ?tiin?a celor sfin?i nu o cunosc.
Pro 30:4  Cine s-a suit în ceruri ?i iara?i s-a pogorât, cine a adunat vântul în mâinile lui? Cine a legat apele în haina lui? Cine a întarit toate marginile pamântului? Care este numele lui ?i care este numele fiului sau? Spune daca ?tii!
Pro 30:5  Toate cuvintele lui Dumnezeu sunt lamurite, scut este El pentru cei ce cauta la El scaparea.
Pro 30:6  Nu adauga nimic la cuvintele Lui, ca sa nu te traga la socoteala ?i sa fii gasit de minciuna!
Pro 30:7  Doua lucruri cer de la Tine, nu ma respinge înainte de a muri:
Pro 30:8  Prefacatoria ?i cuvântul mincinos îndeparteaza-le de la mine; saracie ?i boga?ie nu-mi da, ci da-mi pâinea care-mi este de trebuin?a,
Pro 30:9  Ca nu cumva, saturându-ma, sa ma lepad de Tine ?i sa zic: "Cine este Domnul?" Ca nu cumva, saracind, sa ma apuc de furat ?i sa defaim numele Dumnezeului meu.
Pro 30:10  Nu grai de rau pe sluga la stapânul sau, ca nu cumva sa te blesteme ?i sa te sileasca sa-?i ceri iertare.
Pro 30:11  Este câte un neam de oameni care blesteama pe tatal sau ?i nu binecuvânteaza pe maica sa;
Pro 30:12  Un neam caruia i se pare ca e fara prihana în ochii lui ?i care nu este cura?it de necura?ia lui;
Pro 30:13  Un neam... O, cum ridica ochii lui sus ?i cit se înal?a de sus genele lui!
Pro 30:14  Un neam ai carui din?i sunt ca sabiile ?i ai caror col?i sunt cu?ite, ca sa manânce pe cei sarmani de pe pamânt ?i pe cei saraci dintre oameni.
Pro 30:15  Lipitoarea are doua fiice care zic: "Da-mi, da-mi!" Trei lucruri nu se pot satura, ba ?i al patrulea care nu zice niciodata: "Destul!" ?i anume:
Pro 30:16  Locuin?a mor?ilor, pântecele sterp, pamântul care nu e satul de apa ?i focul care nu zice niciodata: "Destul!"
Pro 30:17  Ochiul care î?i bate joc de parintele sau ?i nu ia în seama ascultarea (ce este dator) maicii sale, sa-l scoata corbii care sala?luiesc linga un curs de apa, iar puii de vultur sa-l manânce.
Pro 30:18  Trei lucruri mi se par minunate, ba chiar patru, pe care nu le pot pricepe:
Pro 30:19  Calea vulturului pe cer, urma ?arpelui pe stânca, mersul corabiei în mijlocul marii ?i calea omului la o fecioara.
Pro 30:20  A?a este purtarea unei femei desfrânate: ea manânca ?i î?i ?terge gura ?i zice: "N-am facut nimic rau"
Pro 30:21  Pentru trei lucruri se cutremura pamântul, ba chiar pentru patru nu poate sa rabde:
Pro 30:22  Pentru robul care ajunge rege, pentru nebunul care se satura de pâine,
Pro 30:23  Pentru o femeie dispre?uita când ea se marita ?i pentru o sluga care mo?tene?te pe stapâna sa.
Pro 30:24  Patru sunt animalele cele mai mici de pe pamânt ?i care sunt cele mai în?elepte:
Pro 30:25  Furnicile, neam fara putere, care î?i agonisesc vara hrana lor;
Pro 30:26  Dihorii, neam slab, care-?i cladesc în stânci loca?ul lor;
Pro 30:27  Lacustele care nu au rege ?i totu?i ies toate în stoluri;
Pro 30:28  ?opârla care se poate prinde cu mâna ?i care patrunde în palatele regilor.
Pro 30:29  Trei fiin?e au înfa?i?are frumoasa, ba patru, care au un mers mare?:
Pro 30:30  Leul, viteazul printre dobitoace, care nu da înapoi în fa?a nimanui;
Pro 30:31  Coco?ul cel ager, ?apul ?i regele caruia nimeni nu-i poate sta împotriva.
Pro 30:32  De e?ti a?a de nebun ca sa te la?i mânat de mânie, bate-te cu mâna peste gura.
Pro 30:33  Batutul laptelui da untul, lovitura peste nas face sa ?â?neasca sângele, iar întarâtarea mâniei duce la cearta.
Pro 31:1  Cuvintele lui Lemuel, regele din Massa, cu care mama sa îl înva?a:
Pro 31:2  Fiul meu, rodul pântecelui meu, feciorul fagaduin?elor mele, cu ce pot eu sa te îndemn?
Pro 31:3  Nu da puterea ta femeilor ?i caile taie celor care pierd pe regi.
Pro 31:4  Nu se cuvine regilor, o, Lemuel, nu se cuvine regilor sa bea vin ?i conducatorii bauturi îmbatatoare,
Pro 31:5  Ca nu cumva bând sa uite legea ?i sa judece strâmb pe to?i sarmanii.
Pro 31:6  Da?i bautura îmbatatoare celui ce este gata sa piara ?i vin celui cu amaraciune în suflet,
Pro 31:7  Ca sa bea ?i sa uite saracia ?i sa nu-?i mai aduca aminte de chinul lui.
Pro 31:8  Deschide gura ta pentru cel mut ?i pentru pricina tuturor parasi?ilor.
Pro 31:9  Deschide gura ta, judeca drept ?i fa dreptate celui sarac ?i napastuit.
Pro 31:10  Cine poate gasi o femeie virtuoasa? Pre?ul ei întrece margeanul.
Pro 31:11  Într-însa se încrede inima so?ului ei, iar câ?tigul nu-i va lipsi niciodata.
Pro 31:12  Ea îi face bine ?i nu rau în tot timpul vie?ii sale:
Pro 31:13  Ea cauta lâna ?i cânepa ?i lucreaza voios cu mâna sa.
Pro 31:14  Ea se aseamana cu corabia unui negu?ator care de departe aduce hrana ei.
Pro 31:15  Ea se scoala dis-de-diminea?a ?i împarte hrana în casa ei ?i da porunci slujnicelor.
Pro 31:16  Gânde?te sa cumpere o ?arina ?i o dobânde?te; din osteneala palmelor sale sade?te vie.
Pro 31:17  Ea î?i încinge cu putere coapsele sale ?i î?i întare?te bra?ele sale.
Pro 31:18  Ea simte ca bun e câ?tigul ei; sfe?nicul ei nu se stinge nici noaptea.
Pro 31:19  Ea pune mâna pe furca ?i cu degetele sale apuca fusul.
Pro 31:20  Ea întinde mâna spre cel sarman ?i bra?ul ei spre cel necajit.
Pro 31:21  N-are teama pentru cei ai casei sale în vreme de iarna, caci to?i din casa sunt îmbraca?i în haine stacojii.
Pro 31:22  Ea î?i face scoar?e; hainele ei sunt de vison ?i de porfira.
Pro 31:23  Cinstit este barbatul ei la por?ile ceta?ii, când sta la sfat cu batrânii ?arii.
Pro 31:24  Ea face cama?i ?i le vinde, ?i brâie da negu?atorilor.
Pro 31:25  Tarie ?i farmec este haina ei ?i ea râde zilei de mâine.
Pro 31:26  Gura ?i-o deschide cu în?elepciune ?i sfaturi pline de dragoste sunt pe limba ei.
Pro 31:27  Ea vegheaza la propa?irea casei sale ?i pâine, fara sa lucreze; ea nu manânca.
Pro 31:28  Feciorii sai vin ?i o fericesc, iar so?ul ei o lauda:
Pro 31:29  "Multe fete s-au dovedit harnice, dar tu le-ai întrecut pe toate!"
Pro 31:30  În?elator este farmecul ?i de?arta este frumuse?ea; femeia care se teme de Domnul trebuie laudata!
Pro 31:31  Sa se bucure de rodul mâinilor sale, ?i la por?ile ceta?ii harnicia ei sa fie data ca pilda!


\end{document}