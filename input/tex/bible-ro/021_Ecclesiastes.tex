\begin{document}

\title{Ecclesiastes}

Ecc 1:1  Cuvintele Ecclesiastului, fiul lui David, rege în Ierusalim.
Ecc 1:2  De?ertaciunea de?ertaciunilor, zice Ecclesiastul, de?ertaciunea de?ertaciunilor, toate sânt de?ertaciuni!
Ecc 1:3  Ce folos are omul din toata truda lui cu care se trude?te sub soare?
Ecc 1:4  Un neam trece ?i altul vine, dar pamântul ramâne totdeauna!
Ecc 1:5  Soarele rasare, soarele apune ?i zore?te catre locul lui ca sa rasara iara?i.
Ecc 1:6  Vântul sufla catre miazazi, vântul se întoarce catre miazanoapte ?i, facând roate-roate, el trece neîncetat prin cercurile sale.
Ecc 1:7  Toate fluviile curg în mare, dar marea nu se umple, caci ele se întorc din nou la locul din care au plecat.
Ecc 1:8  Toate lucrurile se zbuciuma mai mult decât poate omul sa o spuna: ochiul nu se satura de câte vede ?i urechea nu se umple de câte aude.
Ecc 1:9  Ceea ce a mai fost, aceea va mai fi, ?i ceea ce s-a întâmplat se va mai petrece, caci nu este nimic nou sub soare.
Ecc 1:10  Daca este vreun lucru despre care sa se spuna: "Iata ceva nou!" aceasta a fost în vremurile stravechi, de dinaintea noastra.
Ecc 1:11  Nu ne aducem aminte despre cei ce au fost înainte, ?i tot a?a despre cei ce vor veni pe urma; nici o pomenire nu va fi la urma?ii lor.
Ecc 1:12  Eu  Ecclesiastul  am fost regele lui Israel în Ierusalim.
Ecc 1:13  ?i m-am sârguit în inima mea sa cercetez ?i sa iau aminte cu în?elepciune la tot ceea ce se petrece sub cer. Acesta este un chin cumplit pe care Dumnezeu l-a dat fiilor oamenilor, ca sa se chinuiasca întru el.
Ecc 1:14  M-am uitat cu luare aminte la toate lucrarile care se fac sub soare ?i iata: totul este de?ertaciune ?i vânare de vânt.
Ecc 1:15  Ceea ce este strâmb nu se poate îndrepta ?i ceea ce lipse?te nu se poate numara.
Ecc 1:16  Grait-am în inima mea: Cu adevarat am adunat ?i am strâns în?elepciune - mai mult decât to?i cei care au fost înaintea mea în Ierusalim caci inima mea a avut cu bel?ug în?elepciune ?i ?tiind.
Ecc 1:17  ?i mi-am silit inima ca sa patrund în?elepciunea ?i ?tiin?a, nebunia ?i prostia, dar am în?eles ca ?i aceasta este vânare de vânt,
Ecc 1:18  Ca unde este multa în?elepciune este ?i multa amaraciune, ?i cel ce î?i înmul?e?te ?tiin?a î?i spore?te suferin?a.
Ecc 2:1  Graiam inimii mele: "Vino sa te ispitesc cu veselia ?i sa te fac sa gu?ti placerea!" ?i iata ca ?i aceasta este de?ertaciune.
Ecc 2:2  "E nebunie!" am zis despre râs. ?i despre veselie: "La ce poate sa foloseasca!"
Ecc 2:3  Am cugetat apoi în inima mea, sa desfatez trupul meu cu vin - pe când cugetul meu umbla dupa în?elepciune ?i cerceta nebunia - pâna ce voi vedea ceea ce este bun pentru fiii oamenilor sa faca sub cer în vremea vie?ii lor.
Ecc 2:4  Am început lucrari mari: am zidit case, am sadit vii,
Ecc 2:5  Am facut gradini ?i parcuri ?i am sadit în ele tot felul de pomi roditori;
Ecc 2:6  Mi-am facut iazuri, ca sa pot uda din ele o dumbrava unde cre?teau copacii;
Ecc 2:7  Am cumparat robi ?i roabe ?i am avut feciori nascu?i în casa, asemenea ?i turme de vite, ?i oi fara de numar, mai mult decât to?i cei care au fost înaintea mea în Ierusalim.
Ecc 2:8  Am strâns aur ?i argint ?i numar mare de regi ?i de satrapi; am adus cântare?i ?i cântare?e ?i desfatarea fiilor anului mi-am agonisit: o prin?esa ?i alte prin?ese.
Ecc 2:9  Am fost mare ?i am întrecut pe to?i cei ce au trait înaintea mea în Ierusalim ?i în?elepciunea a ramas cu mine.
Ecc 2:10  ?i tot ceea ce doreau ochii mei nu am dat la o parte ?i n-am oprit inima mea de la nici o veselie, caci inima mea s-a bucurat de toata osteneala mea, ?i aceasta mi-a fost partea din toata munca mea.
Ecc 2:11  Apoi m-am uitat cu luare aminte la toate lucrurile pe care le-au facut mâinile mele ?i la truda cu care m-am trudit ca sa le savâr?esc ?i iata, totul este de?ertaciune ?i vânare de vânt ?i fara nici un folos sub soare.
Ecc 2:12  ?i mi-am întors privirea sa vad în?elepciunea, nebunia ?i prostia. Caci ce poate sa faca un om de rând peste ceea ce a facut un rege?
Ecc 2:13  Atunci m-am încredin?at ca în?elepciunea are întâietate asupra nebuniei tot atât cât are lumina asupra întunericului.
Ecc 2:14  În?eleptul are ochii în cap, iar nebunul merge întru întuneric. Dar am cunoscut ?i eu ca aceea?i soarta vor avea to?i.
Ecc 2:15  Deci am zis în inima mea: "Aceea?i soarta ca ?i cel nebun avea-voi ?i eu; atunci la ce îmi folose?te în?elepciunea?" ?i am zis în mintea mea ca ?i aceasta este de?ertaciune.
Ecc 2:16  Caci pomenirea celui în?elept ca ?i a celui nebun nu este ve?nica, fiindca în zilele ce vor veni amândoi vor fi uita?i; atunci în?eleptul moare ca ?i nebunul.
Ecc 2:17  Drept aceea am urât via?a, caci rele sunt cele ce se fac sub soare; ?i totul este de?ertaciune ?i vânare de vânt.
Ecc 2:18  ?i am urât toata munca pe care am muncit-o sub soare, fiindca voi lasa-o omului care va veni dupa mine.
Ecc 2:19  ?i cine ?tie daca el va fi în?elept sau nebun! ?i el va face ce va gasi cu cale din tot lucrul cu care m-am trudit ?i m-am chibzuit sub soare! ?i aceasta este de?ertaciune!
Ecc 2:20  ?i am început sa ma las deznadejdii pentru toata munca cea de sub soare,
Ecc 2:21  Caci un om care a pus în lucrul lui în?elepciune ?i ?tiin?a ?i a avut izbânda, îl împarte cu cel care n-a lucrat. ?i aceasta aste de?ertaciune ?i un rau nespus de mare.
Ecc 2:22  Oare, ce-i ramâne omului din toata munca lui ?i din grija inimii lui cu care s-a trudit sub soare?
Ecc 2:23  Toate zilele lui nu sunt decât suferin?a ?i îndeletnicirea lui nu-i decât necaz; nici chiar noaptea n-are odihna inima lui. ?i aceasta este de?ertaciune!
Ecc 2:24  Nimic nu este mai bun pentru om decât sa manânce ?i sa bea ?i sa-?i desfateze sufletul cu mul?umirea din munca sa. ?i am vazut ca ?i aceasta vine numai din mâna lui Dumnezeu.
Ecc 2:25  Cine poate oare sa manânce ?i sa bea fara sa mul?umeasca Lui?
Ecc 2:26  Omului care este bun înaintea lui Dumnezeu, Dumnezeu îi da în?elepciune, ?tiin?a ?i bucurie, iar pacatosului îi da sarcina sa adune ?i sa strânga pentru a da celui ce este bun în fa?a lui Dumnezeu. ?i aceasta este de?ertaciune ?i vinare de vânt!
Ecc 3:1  Pentru orice lucru este o clipa prielnica ?i vreme pentru orice îndeletnicire de sub cer.
Ecc 3:2  Vreme este sa te na?ti ?i vreme sa mori; vreme este sa sade?ti ?i vreme sa smulgi ceea ce ai sadit.
Ecc 3:3  Vreme este sa rane?ti ?i vreme sa tamaduie?ti; vreme este sa darâmi ?i vreme sa zide?ti.
Ecc 3:4  Vreme este sa plângi ?i vreme sa râzi; vreme este sa jele?ti ?i vreme sa dan?uie?ti.
Ecc 3:5  Vreme este sa arunci pietre ?i vreme sa le strângi; vreme este sa îmbra?i?ezi ?i vreme este sa fugi de îmbra?i?are.
Ecc 3:6  Vreme este sa agonise?ti ?i vreme sa prapade?ti; vreme este sa pastrezi ?i vreme sa arunci.
Ecc 3:7  Vreme este sa rupi ?i vreme sa co?i; vreme este sa taci ?i vreme sa graie?ti.
Ecc 3:8  Vreme este sa iube?ti ?i vreme sa ura?ti. Este vreme de razboi ?i vreme de pace.
Ecc 3:9  Care este folosul celui ce lucreaza întru osteneala pe care o ia asupra-?i?
Ecc 3:10  Am vazut zbuciumul pe care l-a dat Dumnezeu fiilor oamenilor, ca sa se zbuciume.
Ecc 3:11  Toate le-a facut Dumnezeu frumoase ?i la timpul lor; El a pus în inima lor ?i ve?nicia, dar fara ca omul sa poata în?elege lucrarea pe care o face Dumnezeu, de la început pâna la sfâr?it.
Ecc 3:12  Atunci mi-am dat cu socoteala ca nu este fericire decât sa te bucuri ?i sa traie?ti bine în timpul vie?ii tale.
Ecc 3:13  Drept aceea daca un om manânca ?i bea ?i traie?te bine de pe urma muncii lui, acesta este un dar de la Dumnezeu.
Ecc 3:14  Atunci mi-am dat seama ca tot ceea ce a facut Dumnezeu va ?ine în veac de veac ?i nimic nu se poate adauga, nici nu se poate mic?ora ?i ca Dumnezeu lucreaza a?a ca sa ne temem de fa?a Lui.
Ecc 3:15  Ceea ce este a mai fost ?i ceea ce va mai fi a fost în alte vremuri; ?i Dumnezeu cheama iara?i aceea ce a lasat sa treaca.
Ecc 3:16  Dar am mai vazut sub soare ca în locul drepta?ii este faradelegea ?i în locul celui cucernic, cel nelegiuit.
Ecc 3:17  ?i am gândit în inima mea: "Dumnezeu va judeca pe cel drept ca ?i pe cel nelegiuit", caci este vreme pentru orice punere la cale ?i pentru orice fapta.
Ecc 3:18  ?i am zis iar în inima mea despre fiii oamenilor: "Dumnezeu a orânduit sa-i încerce, ca ei sa-?i dea seama ca nu sânt decât dobitoace".
Ecc 3:19  Caci soarta omului ?i soarta dobitocului este aceea?i: precum moare unul, moare ?i celalalt ?i to?i au un singur duh de via?a, iar omul nu are nimic mai mult decât dobitocul. ?i totul este de?ertaciune!
Ecc 3:20  Amândoi merg în acela?i loc: amândoi au ie?it din pulbere ?i amândoi în pulbere se întorc.
Ecc 3:21  Cine ?tie daca duhul omului se urca în sus ?i duhul dobitocului se coboara în jos catre pamânt?
Ecc 3:22  ?i mi-am dat seama ca nimic nu este mai de pre? pentru om decât sa se bucure de lucrurile sale, ca aceasta este partea lui, fiindca cine îi va da lui putere sa mai vada ceea ce se va întâmpla în urma lui?
Ecc 4:1  ?i iara?i am luat aminte la toate silniciile care se savâr?esc sub soare. ?i iata lacrimile celor apasa?i ?i nimeni nu era care sa-i mângâie, iar în mâna celor silnici toata asuprirea ?i nici un mângâietor nu se gasea!
Ecc 4:2  ?i am fericit pe cei ce au murit în vremi stravechi mai mult decât pe cei vii care sânt acum în via?a.
Ecc 4:3  Iar mai fericit ?i decât unii ?i decât al?ii este cel ce n-a venit pe lume, cel care n-a vazut faptele cele rele care se savâr?esc sub soare.
Ecc 4:4  ?i am vazut ca toata stradania ?i toata izbânda omului la lucru nu este decât pizma unuia fa?a de altul. ?i aceasta este de?ertaciune ?i vânare de vânt!
Ecc 4:5  Nebunul sta cu mâinile în sân ?i î?i manânca singur timpul zicând:
Ecc 4:6  "Mai de pre? este un pumn plin de odihna decât doi pumni plini de truda ?i de vânare de vânt".
Ecc 4:7  ?i iara?i am vazut o nepotrivire sub soare:
Ecc 4:8  Este câte un om care este stingher ?i care nu are nici copil, nici frate ?i totu?i lucrul nu-l mai sfâr?e?te ?i ochii lui nu se mai satura de boga?ie. Dar vine o vreme când zice: "Pentru cine m-am trudit ?i am lipsit sufletul meu de traiul cel bun?" ?i aceasta este de?ertaciune ?i rea îndeletnicire.
Ecc 4:9  Mai ferici?i sânt doi laolalta decât unul, fiindca au rasplata buna pentru munca lor;
Ecc 4:10  Caci daca unul cade, îl scoala tovara?ul lui. Dar vai de cel singur care cade ?i nu este cel de-al doilea ca sa-l ridice!
Ecc 4:11  Asemenea când doi se culca se încalzesc, iar unul cum s-ar putea încalzi?
Ecc 4:12  ?i daca unul este luat fara de veste, cel de-al doilea sare pentru el; caci sfoara pusa în trei nu se rupe degraba.
Ecc 4:13  Mai de pre? este un copil sarman ?i în?elept decât un rege batrân ?i nebun, care nu mai este în stare sa asculte de sfaturi;
Ecc 4:14  Caci el poate sa iasa din închisoare ca sa domneasca, de?i s-a nascut sarac în ?ara celuilalt.
Ecc 4:15  Vazut-am pe to?i cei vii care merg sub soare îmbulzindu-se lânga tânarul care va sta în locul regelui ca mo?tenitor.
Ecc 4:16  ?i nu se mai sfâr?ea poporul în fruntea caruia era; totu?i urma?ii lui nu se vor bucura de el. ?i aceasta este de?ertaciune ?i vânare de vânt.
Ecc 5:1  Ia seama la picioarele tale când te duci în templul Domnului. Daca te apropii sa ascul?i este mai bine, decât sa aduci jertfa nebunilor, caci ei nu ?tiu decât sa faca rau.
Ecc 5:2  Nu te grabi sa deschizi gura ta ?i inima ta sa nu se pripeasca sa scoata o vorba înaintea lui Dumnezeu, ca Dumnezeu este în ceruri, iar tu pe pamânt; pentru aceasta sa fie cuvintele tale pu?ine.
Ecc 5:3  Visurile vin din multele griji, iar glasul celui nebun din mul?imea de vorbe.
Ecc 5:4  Daca ai facut un juramânt lui Dumnezeu, nu pierde din vedere sa-l împline?ti, ca nebunii nu au nici o trecere; tu însa împline?te ce ai fagaduit.
Ecc 5:5  Mai bine sa nu fagaduie?ti decât sa nu împline?ti ce ai fagaduit.
Ecc 5:6  Nu îngadui gurii tale sa traga spre pacat trupul tau ?i înaintea trimisului lui Dumnezeu nu spune: "A fost o ratacire!" Pentru ce sa Se mânie Dumnezeu de cuvântul tau ?i sa nimiceasca lucrul mâinilor tale?
Ecc 5:7  Caci din mul?imea grijilor se nasc visele ?i de?ertaciunile din prea multe cuvinte. De aceea, teme-te de Dumnezeu!
Ecc 5:8  Daca vezi asuprirea celui sarac ?i obijduirea dreptului ?i a drepta?ii în cetate, nu te mira de lucrul acesta, caci peste cel mare este unul mai mare, iar Cel Atotputernic vegheaza peste to?i.
Ecc 5:9  Totu?i este un folos pentru ?ara ?i anume: un rege care sa poarte grija muncii pamântului.
Ecc 5:10  Cine iube?te banii nu se va satura de bani, iar cel ce iube?te boga?ia nu va avea parte de rodul ei. ?i aceasta este de?ertaciune!
Ecc 5:11  Când se înmul?esc averile, sporesc ?i cei ce le manânca ?i ce folos are stapânul lor ca numai le vede?
Ecc 5:12  Dulce este somnul lucratorului, fie ca manânca mult, fie ca manânca pu?in, dar bel?ugul bogatului nu-i da ragaz sa doarma.
Ecc 5:13  Este un rau cumplit pe care l-am vazut sub soare: boga?ii puse la o parte de stapânul lor pentru a lui nenorocire.
Ecc 5:14  ?i daca boga?ia se pierde dintr-o întâmplare nenorocita ?i el are un copil, acestuia nu-i ramâne nimic în mina.
Ecc 5:15  Precum a ie?it din pântecele maicii sale, gol se va duce, a?a cum a venit, ?i pentru munca lui el nu va primi nimic, ca sa poata lua în mâna lui.
Ecc 5:16  ?i aceasta este o întâmplare nenorocita, ca sa se duca a?a cum a venit; ?i ce folos ca i-a fost munca în vânt?
Ecc 5:17  Mai mult înca, toata via?a lui este întuneric ?i suparare, necaz peste fire ?i boala ?i durere!
Ecc 5:18  Cu adevarat iata ceea ce am vazut ca este bine ?i frumos: sa manânce ?i sa bea ?i sa traiasca omul bine din tot lucrul cu care se trude?te sub soare în vremea vie?ii daruite lui de Dumnezeu, caci aceasta este partea lui.
Ecc 5:19  ?i ori de câte ori Dumnezeu da omului boga?ii ?i bunuri ?i îi îngaduie sa manânce ?i sa-?i ia partea lui ?i sa se bucure de munca lui, acesta este un dar de la Dumnezeu;
Ecc 5:20  Caci el nu se gânde?te prea mult la zilele vie?ii lui, fiindca Dumnezeu îl sine prins cu bucuria inimii lui.
Ecc 6:1  Este un rau pe care l-am vazut sub soare ?i care apasa greu asupra omului;
Ecc 6:2  Omului caruia Dumnezeu i-a dat averi ?i bunuri, iar sufletului lui nu-i lipse?te nimic din ceea ce ar putea sa doreasca, Dumnezeu nu-i îngaduie însa sa se bucure de ele, ci un strain le va mânca. Iata o de?ertaciune ?i un rau nespus de mare!
Ecc 6:3  Daca un om ar fi sa aiba o suta de fii ?i sa traiasca mul?i ani ?i numeroase sa fie zilele anilor sai, daca nu s-a saturat sufletul lui de bine ?i el nu are loc de îngropare zic: "Chiar ?i fatul lepadat e mai fericit decât el!"
Ecc 6:4  A venit în zadar ?i se duce în întuneric ?i în întuneric numele lui va fi învaluit;
Ecc 6:5  Nici n-a vazut, nici n-a cunoscut soarele, ?i fatul lepadat a avut mai multa odihna decât omul acesta.
Ecc 6:6  ?i daca ar fi trait de doua ori câte o mie de ani ?i nu s-a bucurat de fericire, oare nu to?i se duc în acela?i loc?
Ecc 6:7  Toata munca omului este pentru gura lui ?i cu toate acestea pofta lui nu e astâmparata.
Ecc 6:8  Caci ce are în?eleptul mai mult decât nebunul? Ce folos are saracul care ?tie sa se poarte înaintea celor vii?
Ecc 6:9  Mai bine sa te ui?i cu ochii decât sa pribege?ti cu dorin?a. ?i aceasta este de?ertaciune ?i vinare de vânt!
Ecc 6:10  La tot ce î?i ia fiin?a i s-a hotarât numele de mai înainte; se ?tie ce va fi omul; el nu poate sa intre în pricina cu cel ce este mai tare decât el.
Ecc 6:11  Cu cit se spun mai multe cuvinte, cu atât este mai multa de?ertaciune. Ce folos trage omul?
Ecc 6:12  Caci cine ?tie ce este de folos pentru om în via?a, în vremea zilelor sale de nimicnicie pe care le trece asemenea unei umbre? ?i cine va spune mai dinainte omului ce va fi dupa el sub soare?
Ecc 7:1  Mai de pre? este un nume bun decât untdelemnul cel binemirositor ?i ziua mor?ii decât ziua na?terii.
Ecc 7:2  Mai bine este sa mergi în casa de plâns, decât sa te duci în casa de ospa?, caci acolo se vede sfâr?itul omului ?i cine îl vede pune la inima.
Ecc 7:3  Mai bun este necazul decât râsul, caci întristarea fe?ei este buna pentru inima.
Ecc 7:4  Inima celor în?elep?i este în casa cea cu tânguire, iar inima celor nebuni în casa veseliei.
Ecc 7:5  Mai degraba sa auzi certarea unui în?elept, decât sa ascul?i cântecul celor nebuni;
Ecc 7:6  Ca precum este pârâitul spinilor sub caldare, tot a?a este ?i râsul celui nebun. ?i aceasta este de?ertaciune!...
Ecc 7:7  Caci asuprirea poate sa faca nebun pe un în?elept, ?i mita strica inima.
Ecc 7:8  Mai bun este sfâr?itul unui lucru decât începutul lui; mai de pre? este un duh rabdator decât un duh seme?.
Ecc 7:9  Nu te grabi sa te întarâ?i întru duhul tau, pentru ca mânia  sala?luie?te în sânul celor nebuni.
Ecc 7:10  Nu spune niciodata: "Cum se face ca zilele cele de altadata au fost mai bune decât acestea?" Caci nu din în?elepciune întrebi una ca aceasta.
Ecc 7:11  În?elepciunea este tot atât ?ie buna ca ?i o mo?tenire de folos celor ce vad soarele,
Ecc 7:12  Ca a?a este ocrotirea în?elepciunii ?i a banului; dar folosul ?tiin?ei este ca în?elepciunea ?ine cu via?a pe stapânul ei.
Ecc 7:13  Socote?te cu mintea faptele lui Dumnezeu; cine poate sa îndrepte ceea ce El a strâmbat?
Ecc 7:14  În zi de fericire fii bucuros, iar în zi de nenorocire gânde?te-te ca Dumnezeu a facut ?i pe una ?i pe cealalta, a?a ca omul sa nu descopere nimic din cele viitoare.
Ecc 7:15  Aceste doua lucruri le-am vazut eu în zilele nimicniciei mele: este câte un drept, care piere întru dreptatea lui, ?i este câte un nelegiuit care traie?te mereu în rautatea lui.
Ecc 7:16  Nu fii drept peste masura ?i nu te arata prea în?elept! Pentru ce vrei sa te nimice?ti?
Ecc 7:17  Nu fii nelegiuit pâna la sfâr?it ?i nu fii nici nebun; de ce sa mori înainte de timpul tau?
Ecc 7:18  Este bine sa te ?ii de una ?i de cealalta sa nu te desfaci, ca cine se teme de Dumnezeu scapa din toate.
Ecc 7:19  În?elepciunea da în?eleptului mai multa putere decât au zece nerozi într-o cetate.
Ecc 7:20  Caci nu este om drept pe pamânt care sa faca binele ?i sa nu pacatuiasca.
Ecc 7:21  Nu lua aminte la toate vorbele pe care cineva le spune, ca nu cumva sa auzi ca sluga ta te graie?te de rau;
Ecc 7:22  Caci inima ta singura ?tie de câte ori ?i tu ai defaimat pe al?ii.
Ecc 7:23  Toate acestea le-am încercat prin în?elepciune ?i am zis: "Vreau sa fiu în?elept!" dar în?elepciunea a ramas departe de mine.
Ecc 7:24  Ceea ce a fost este departe, ?i adânc, adânc! Cine poate acum sa-i dea de în?eles?
Ecc 7:25  ?i eu m-am silit ?i inima mea a cercetat ?i a urmarit ?tiin?a ?i în?elepciunea ?i buna chibzuiala ?i mi-am dat seama ca rautatea este o nebunie, iar prostia este zanatica rautate.
Ecc 7:26  ?i am gasit femeia mai amara decât moartea, pentru ca ea este o cursa, inima ei este un la? ?i mâinile ei sânt catu?e. Cel ce este bun înaintea lui Dumnezeu scapa, iar pacatosul este prins.
Ecc 7:27  Iata ce am aflat, zice Ecclesiastul, ?i înca ?i altele ca sa pot descoperi în?elesul,
Ecc 7:28  Pe care sufletul meu l-a cautat ?i nu l-a aflat. Am gasit însa un om la o mie, dar n-am gasit nici o femeie din toate câte sânt.
Ecc 7:29  Dar iata numai ce am gasit: Dumnezeu a facut pe om drept, iar oamenii nascocesc multe vicle?uguri.
Ecc 8:1  Cine este ca în?eleptul ?i cine poate sa ?tie ca el tâlcuirea lucrurilor? În?elepciunea unui om îi lumineaza fa?a ?i asprimea fe?ei lui i se schimba.
Ecc 8:2  Asculta de porunca regelui din pricina juramântului facut lui Dumnezeu.
Ecc 8:3  Nu te grabi sa te departezi de fa?a lui. Nu starui în lucrul cel rau, ca ceea ce voie?te face.
Ecc 8:4  Cuvântul regelui este hotarâtor ?i cine poate sa-i spuna: "Ce faci?"
Ecc 8:5  Celui ce asculta porunca nu i se va întâmpla nimic rau, ca inima unui om în?elept va cunoa?te timpul ?i judecata.
Ecc 8:6  Pentru orice lucru este un timp ?i o judecata; ca mare este nenorocirea care apasa asupra omului,
Ecc 8:7  De vreme ce el nu poate sa ?tie mai dinainte ceea ce se va întâmpla. Oare, cine îi va da de ?tire ceea ce va fi cu el?
Ecc 8:8  Omul nu este stapân pe duhul sau de via?a, ca sa-l poata opri; la fel nu este stapân pe ziua mor?ii ?i în aceasta lupta nu încape amânare. Nelegiuirea nu va scapa pe cel care o savâr?e?te.
Ecc 8:9  Am vazut toate aceste lucruri ?i mi-am sârguit inima mea spre tot lucrul care se face sub soare într-o vreme când omul stapâne?te pe altul spre nenorocirea lui.
Ecc 8:10  ?i am vazut pacato?i în mare cinste purta?i la locul de odihna, pe când cei ce lucrasera drept au fost izgoni?i de la locul cel sfânt ?i au fost uita?i în cetate. ?i aceasta este de?ertaciune!
Ecc 8:11  Din pricina ca hotarârea pentru pedepsirea rauta?ii nu este îndeplinita de îndata, pentru aceasta se umple de rautate inima oamenilor ca sa faca rau.
Ecc 8:12  Dar de?i pacatosul face rau de o suta de ori ?i î?i lunge?te zilele, eu ?tiu ca fericirea este a acelora care se tem de Dumnezeu ?i se sfiesc în fa?a Lui.
Ecc 8:13  Fericirea nu va fi pentru cel fara de lege, care, asemenea umbrei, nu-?i va lungi via?a, fiindca el nu se teme de Domnul.
Ecc 8:14  Este înca o nepotrivire care se petrece pe pamânt, adica: sânt drep?i carora li se rasplate?te ca dupa faptele celor nelegiui?i ?i sânt pacato?i carora li se rasplate?te ca dupa faptele celor drep?i. Am zis ca ?i aceasta este înca o de?ertaciune!
Ecc 8:15  ?i am ridicat în slavi veselia, caci nu este nimic mai bun pentru om sub soare, decât sa manânce, sa bea ?i sa se veseleasca. Numai de l-ar întovara?i la lucrul lui în tot timpul vie?ii pe care Dumnezeu i-o da sub soare!
Ecc 8:16  Când mi-am îndreptat inima ca sa cunosc în?elepciunea ?i sa patrund care este menirea omului pe pamânt, caci nici zi, nici noapte ochii lui nu vad somnul,
Ecc 8:17  Atunci mi-am dat seama, privind lucrarea lui Dumnezeu, ca omul nu poate sa în?eleaga toate câte se fac sub soare, dar se ostene?te cautându-le, fara sa le dea de rost; iar daca în?eleptul crede ca le cunoa?te, el nu poate sa le patrunda.
Ecc 9:1  Cu adevarat toate acestea le-am pus la inima ?i inima mea le-a vazut: ca cei drep?i ?i cei în?elep?i împreuna cu toate faptele lor sânt în mina lui Dumnezeu. Nici macar iubirea, nici ura nu o cunoa?te omul; ci totul este de?ertaciune înaintea oamenilor,
Ecc 9:2  Caci to?i au aceea?i soarta: cel drept ca ?i cel pacatos, cel bun ca ?i cel rau, cel curat ea ?i cel necurat, cel ce aduce jertfa ca ?i cel ce nu aduce, cel bun ca ?i cel rau, cel ce jura ca ?i cel ce cinste?te juramântul.
Ecc 9:3  Este un mare rau în tot ceea ce se face sub soare, caci to?i au aceea?i soarta ?i pe lânga aceasta inima oamenilor este plina de rautate ?i nebunia în inima lor dainuie?te toata via?a lor ?i se duc în acela?i loc cu cei mor?i;
Ecc 9:4  Oare cine va ramâne în via?a? Pentru to?i cei vii este o nadejde, caci un câine viu este mai de pre? decât un câine mort.
Ecc 9:5  Cei vii ?tiu ca vor muri, dar cei mor?i nu ?tiu nimic ?i parte de rasplata nu mai au, caci numele lor a fost uitat.
Ecc 9:6  ?i dragostea lor, ura lor ?i pizma lor a pierit de mult ?i nu se vor mai bucura niciodata de ceea ce se face sub soare.
Ecc 9:7  Du-te ?i manânca cu bucurie pâinea ta ?i bea cu inima buna vinul tau, pentru ca Dumnezeu este îndurator pentru faptele tale.
Ecc 9:8  Toata vremea ve?mintele tale sa fie albe ?i untdelemnul sa nu lipseasca de pe capul tau!
Ecc 9:9  Bucura-te de via?a cu femeia pe care o iube?ti în toate zilele vie?ii tale celei de?arte, pe care ?i-a harazit-o Dumnezeu sub soare; caci aceasta este partea ta în via?a ?i în mijlocul trudei cu care te ostene?ti sub soare.
Ecc 9:10  Tot ceea ce mâna ta prinde sa savâr?easca, fa cu hotarâre, caci în locuin?a mor?ilor în care te vei duce nu se afla nici fapta, nici punere la cale, nici ?tiin?a, nici în?elepciune.
Ecc 9:11  ?i iara?i am vazut sub soare ca izbânda în alergare nu este a celor iu?i ?i biruin?a a celor viteji, ?i pâinea a celor în?elep?i, nici boga?ia a celor pricepu?i, nici faima pentru cei înva?a?i, caci timpul ?i întâmplarea întâmpina pe to?i.
Ecc 9:12  Ca omul nu ?tie nici macar vremea lui: întocmai ea ?i pe?tii care sânt prin?i în vicleanul navod, întocmai ca ?i pasarile în la?, a?a sânt prin?i fara de veste oamenii în vremea de restri?te, când vine dintr-odata peste ei.
Ecc 9:13  ?i am mai vazut sub soare acest fapt de în?elepciune care mi s-a parut într-adevar mare:
Ecc 9:14  A fost odata o cetate mica, locuita de oameni pu?ini ?i împotriva ei a pornit un rege vestit, care a împresurat-o ?i a ridicat de jur împrejur întarituri puternice.
Ecc 9:15  Într-însa se afla un sarac în?elept care a scapat cetatea prin în?elepciunea lui ?i nimeni nu-?i mai aduce aminte de acest om sarman.
Ecc 9:16  ?i am zis: "Mai buna este în?elepciunea decât puterea; dar în?elepciunea celui sarac este urgisita, ?i cuvintele lui nu sânt luate în seama".
Ecc 9:17  Vorbele celui în?elept spuse domol sunt mai ascultate decât strigatul unui stapân între nebuni.
Ecc 9:18  În?elepciunea este mai de pre? decât armele de lupta; dar o singura gre?eala nimice?te mult bine.
Ecc 10:1  O musca moarta strica amestecul de untdelemn al celui ce pregate?te miresme; pu?ina nebunie strica pre?ul la multa în?elepciune.
Ecc 10:2  Inima celui în?elept este la dreapta lui, iar inima celui nebun la stânga.
Ecc 10:3  La fel celui nebun când merge pe drum îi lipse?te dreapta pricepere, iar toata lumea zice: "E nebun!"
Ecc 10:4  Daca mânia stapânitorului se ridica împotriva ta, nu te clinti din locul tau. Caci firea domoala înlatura mari neajunsuri.
Ecc 10:5  Am mai luat seama la înca un rau de sub soare, ca o gre?eala care porne?te de la stapânitor:
Ecc 10:6  Nebunul este ridicat la dregatorii înalte, iar cei vrednici stau în locuri de jos.
Ecc 10:7  Am vazut robi calari pe cai ?i capetenii mergând ca robii pe jos.
Ecc 10:8  Cel ce sapa o groapa poate sa cada în ea ?i cel ce darâma un zid poate fi mu?cat de ?arpe.
Ecc 10:9  Cel ce sfarâma pietre se poate rani cu ele, iar cel ce despica lemne se primejduie?te.
Ecc 10:10  Daca o unealta s-a tocit ?i nu a fost ascu?ita, trebuie sa îndoim puterile, dar în?elepciunea are ca parte izbânda.
Ecc 10:11  Daca ?arpele mu?ca înainte de a fi descântat, atunci descântatorul nu are nici un folos.
Ecc 10:12  Graiurile gurii celui în?elept sânt har, iar buzele celui nebun îl doboara.
Ecc 10:13  Începutul cuvintelor gurii lui este prostia, iar sfâr?itul graiului lui nebunie curata.
Ecc 10:14  Nebunul spore?te vorbele. Omul nu ?tie ce va fi dupa el, caci cine îi va spune ce se va întâmpla dupa el?
Ecc 10:15  Munca obose?te pe cel nebun; cine nu ?tie drumul nu poate sa se duca în cetate.
Ecc 10:16  Vai de tine, ?ara, care ai un copil rege ?i capeteniile tale manânca dis-de-diminea?a!
Ecc 10:17  Fericita e?ti tu, ?ara, care ai rege un fecior de neam mare ?i capeteniile tale manânca la vreme, ca to?i oamenii, ?i nu se dau la bautura.
Ecc 10:18  Din pricina lenei, grinzile casei se lasa, iar când stai cu mâinile în sin, apa picura în casa.
Ecc 10:19  Ospe?ele se fac pentru a gusta placerea; vinul învesele?te via?a ?i banii raspund la toate.
Ecc 10:20  Chiar în gândul tau nu blestema pe rege ?i în camara unde dormi nu defaima pe cel puternic; caci pasarile cerului pot sa duca un cuvânt ?i neamul celor înaripate sa dea vorba ta pe fa?a.
Ecc 11:1  Arunca pâinea ta pe apa, caci o vei afla dupa multe zile.
Ecc 11:2  Împarte o bucata în ?apte ?i chiar în opt, caci nu ?tii ce nenorocire se poate întâmpla în ?ara.
Ecc 11:3  Când norii se umplu de ploaie, ei se de?arta pe pamânt. ?i daca un capac cade la miazazi sau la miazanoapte, unde a cazut, acolo ramâne.
Ecc 11:4  Cel ce paze?te vântul nu seamana, ?i cel ce se uita la nori nu secera.
Ecc 11:5  Dupa cum nu ?tii care este calea vântului, cum se întocmesc oasele în pântecele maicii, tot a?a nu cuno?ti lucrarea lui Dumnezeu, care face toate.
Ecc 11:6  Dis-de-diminea?a seamana samân?a ?i pâna seara nu odihni mâna ta, caci nu ?tii care va izbuti, aceasta sau aceea, sau daca amândoua sânt deopotriva de bune.
Ecc 11:7  ?i lumina este dulce ?i placut este ochilor sa priveasca soarele.
Ecc 11:8  Chiar daca ar trai mul?i ani, omul sa se bucure de toate ?i sa-?i aduca aminte de zilele cele din întuneric, caci multe vor fi. Tot ce se întâmpla este de?ertaciune.
Ecc 11:9  Bucura-te, omule, cât e?ti tânar ?i inima ta sa fie vesela în zilele tinere?ii tale ?i mergi în caile inimii tale ?i dupa ce-?i arata ochii tai, dar sa ?tii ca, pentru toate acestea, Dumnezeu te va aduce la judecata Sa.
Ecc 11:10  Alunga necazul din inima ta ?i departeaza suferin?ele de trupul tau, caci copilaria ?i tinere?ea sânt de?ertaciune.
Ecc 12:1  Adu-?i aminte de Ziditorul tau în zilele tinere?ii tale, înainte ca sa vina zilele de restri?te ?i sa se apropie anii despre care vei zice: "N-am nici o placere de ei ..."
Ecc 12:2  Înainte ca sa se întunece soarele ?i lumina ?i luna ?i stelele ?i ca norii sa mai vina dupa ploaie;
Ecc 12:3  Atunci este vremea când strajerii casei tremura ?i oamenii cei tari se încovoaie la pamânt ?i cele ce macina nu mai lucreaza, caci sânt pu?ine la numar ?i privitoarele de la ferestre se întuneca;
Ecc 12:4  ?i se închid por?ile care dau spre uli?a ?i se domole?te huruitul morii ?i te scoli la ciripitul de diminea?a al pasarii ?i se potolesc toate cântare?ele;
Ecc 12:5  ?i te temi sa mai urci colina ?i spaimele pândesc în cale ?i capul se face alb ca floarea de migdal ?i lacusta sprintena se face grea ?i to?i mugurii s-au deschis, fiindca omul merge la loca?ul sau de veci ?i bocitoarele dau târcoale pe uli?a;
Ecc 12:6  Mai înainte ca sa se rupa funia de argint ?i sa se sparga vasul de aur ?i sa se strice ulciorul la izvor ?i sa se sfarâme roata fântânii,
Ecc 12:7  ?i ca pulberea sa se întoarca în pamânt cum a fost, iar sufletul sa se întoarca la Dumnezeu, Care l-a dat.
Ecc 12:8  De?ertaciunea de?ertaciunilor, a zis Ecclesiastul, toate sânt de?ertaciuni!
Ecc 12:9  ?i pe lânga ca Ecclesiastul a fost un în?elept, el a mai înva?at pe popor cuno?tin?a ?i a cercetat ?i a privit cu luare aminte ?i a cules multe pilde.
Ecc 12:10  Ecclesiastul s-a straduit sa gaseasca sfaturi folositoare ?i îndrumari adevarate ?i sa le scrie întocmai.
Ecc 12:11  Cuvintele celor în?elep?i sânt ca boldurile de îmboldit dobitoacele ?i ca ni?te cuie înfipte ?i ascu?ite ?i sânt date de un Pastor.
Ecc 12:12  ?i peste toate acestea, fiul meu, sa fii cu luare aminte: scrisul de car?i este fara sfâr?it, iar înva?atura multa este oboseala pentru trup.
Ecc 12:13  Iata pe scurt; tot ceea ce ai auzit aceasta este: Teme-te de Dumnezeu ?i paze?te poruncile Lui! Acesta este lucru cuvenit fiecarui om.
Ecc 12:14  Caci Dumnezeu va judeca toate faptele ascunse, fie bune, fie rele.


\end{document}