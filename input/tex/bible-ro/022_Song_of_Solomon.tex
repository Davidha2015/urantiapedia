\begin{document}

\title{Song of Solomon}

Son 1:1  Cântarea cântarilor, a lui Solomon.
Son 1:2  Saruta-ma cu sarutarile gurii tale, ca sarutarile tale sunt mai bune ca vinul.
Son 1:3  Miresmele tale sunt balsam mirositor, mir varsat este numele tau; de aceea fecioarele te iubesc.
Son 1:4  Rape?te-ma, ia-ma cu tine! Hai sa fugim! - Regele m-a dus în camarile sale: ne vom veseli ?i ne vom bucura de tine. Î?i vom preamari dragostea mai mult decât vinul. Cine te iube?te, dupa dreptate te iube?te!
Son 1:5  Neagra sunt, fete din Ierusalim, dar frumoasa, ca sala?urile lui Chedar, ca ?i corturile lui Solomon.
Son 1:6  Nu va uita?i ca sunt negricioasa ca doar soarele m-a ars. Fiii maicii mele s-au mâniat, trimisu-m-au sa le pazesc viile, dar via mea nu mi-am pazit-o!
Son 1:7  Spune-mi dar, iubitul meu, unde-?i pa?ti tu oile? Unde popose?ti tu la amiaza? De ce oare sa ratacesc zadarnic pe la turmele tovara?ilor tai?
Son 1:8  Daca nu ?tii unde, tu cea mai frumoasa între femei, ?ine atunci mereu urmele oilor, pa?te-?i mieii în preajma colibelor, iezii în preajma ciobanilor!
Son 1:9  Cu un telegar înhamat la carul lui Faraon te aseaman eu, draga mea!
Son 1:10  Frumo?i se vad obrajii tai, a?eza?i intre cercei ?i gâtul tau împodobit cu ?ire de margaritare.
Son 1:11  Fauri-vom pentru tine lan?i?oare aurite, cu picaturi ?i crestaturi de argint.
Son 1:12  Cât regele a stat la masa, nardul meu a revarsat mireasma.
Son 1:13  Perini?a de mirt este iubitul meu, care se ascunde între sânii mei.
Son 1:14  Iubitul meu e ciorchinele de ienupar, din Enghedi, de la vii cules.
Son 1:15  Cât de frumoasa e?ti tu, draga mea! Cât de frumoasa e?ti! Ochi de porumbi?a sunt ochii tai.
Son 1:16  Cât de frumos e?ti dragul meu! ?i cît de dragala? e?ti tu! Paji?tea de iarba verde ne este al nostru pat.
Son 1:17  Cedrii ne sunt acoperi? sala?luirii ?i adapost ne sunt chiparo?ii.
Son 2:1  Eu sunt narcisul din câmpie, sunt crinul de prin vâlcele.
Son 2:2  Cum este crinul între spini, a?a este draga mea între fete.
Son 2:3  Cum este marul intre copaci, a?a este dragul meu printre flacai. Sa stau la umbra marului îmi place, dulce este rodul lui în gura mea!
Son 2:4  El m-a dus în casa de ospa? ?i sus drept steag era iubirea.
Son 2:5  Întari?i-ma cu vin, cu mere racori?i-ma, ca sunt bolnava de iubire.
Son 2:6  Stânga sa este sub cap la mine ?i cu dreapta lui ma cuprinde.
Son 2:7  Va jur, fete din Ierusalim, pe cerboaice, pe gazelele din câmp, nu trezi?i pe draga mea; pâna nu-i va fi ei voia!
Son 2:8  Auzi glasul celui drag! Iata-l vine, saltând peste coline, trecând din munte-n munte.
Son 2:9  Ca o gazela e iubitul meu sau e ca un pui de cerb; iata-l la noi pe prispa, iata-l prive?te pe fereastra, printre gratii iata-l se uita.
Son 2:10  ?i începe sa-mi vorbeasca: Scoala, draga mea, ?i vino!
Son 2:11  Iarna a trecut, ploaia a încetat.
Son 2:12  Flori pe câmp s-au aratat ?i a sosit vremea cântarii, în ?arina glas de turturea se aude.
Son 2:13  Smochinii î?i dezvelesc mugurii ?i florile de vie vazduhul parfumeaza. Scoala, draga mea, ?i vino!
Son 2:14  Porumbi?a mea, ce-n crapaturi de stânca, la loc prapastios te-ascunzi, arata-?i fa?a ta! Lasa-ma sa-?i aud glasul! Ca glasul tau e dulce ?i fa?a ta placuta.
Son 2:15  Prinde?i vulpile, prinde?i puii lor, ele ne strica viile, ca via noastra e acum în floare.
Son 2:16  Iubitul meu este al meu ?i eu sunt a lui. El printre crini î?i pa?te mieii.
Son 2:17  Pâna nu se racore?te ziua, pâna nu se-ntinde umbra serii, vino, dragul meu, saltând ca o caprioara, ca un pui de cerb, peste mun?ii ce ne despart.
Son 3:1  Noaptea-n pat l-am cautat pe dragul sufletului meu, l-am cautat, dar, iata, nu l-am mai aflat.
Son 3:2  Scula-ma-voi, mi-am zis, ?i-n târg voi alerga, pe uli?e, prin pie?e, amanun?it voi cauta pe dragul sufletului meu. L-am cautat, nu l-am mai aflat.
Son 3:3  Întâlnitu-m-am cu paznicii, cei ce târgul strajuiesc; "N-a?i vazut, zic eu, pe dragul sufletului meu?"
Son 3:4  Dar abia m-am despar?it de ei ?i iata, eu l-am gasit, pe cel iubit; apucatu-l-am atunci ?i nu l-am mai lasat, pâna nu l-am dus la mama mea, pâna nu l-am dus în casa ei.
Son 3:5  Va jur, fete din Ierusalim, pe cerboaice, pe gazelele din câmp, nu trezi?i pe draga mea, pâna nu-i va fi ei voia!
Son 3:6  Cine este aceea care se ridica din pustiu, ca un stâlp de fum, par-c-ar arde smirna ?i tamâie, par-c-ar arde miresme iscusit gatite?
Son 3:7  Iat-o, este ea, lectica lui Solomon, înconjurata de ?aizeci de voinici, viteji falnici din Israel.
Son 3:8  To?i sunt înarma?i, la razboi deprin?i. Fiecare poarta sabie la ?old, pentru orice întâmplare ?i frica de noapte.
Son 3:9  Regele Solomon ?i-a facut tron de nunta din lemn de cedru din Liban.
Son 3:10  Stâlpii lui sunt de argint, pere?ii de aur curat, patul e de purpura, iar acoperi?ul este de scumpe alesaturi; darul dragostei alese a fetelor din Ierusalim.
Son 3:11  Ie?i?i, fetele Sionului, privi?i pe Solomon încoronat, cum a lui maica l-a încununat, în ziua sarbatoririi nun?ii lui, în ziua bucuriei inimii lui!
Son 4:1  Cât de frumoasa e?ti tu, draga mea, cît de frumoasa e?ti! Ochi de porumbi?a ai, umbriri de negrele-?i sprâncene, parul tau turma de capre pare, ce din mun?i, din Galaad coboara.
Son 4:2  Din?ii tai par turma de oi tunse, ce ies din scaldatoare facând doua ?iruri strânse ?i neavând nici o ?tirbitura.
Son 4:3  Cordelu?e purpurii sunt ale tale buze ?i gura ta-i încântatoare. Doua jumata?i de rodii par obrajii tai sub valul tau cel straveziu.
Son 4:4  Gâtul tau e turnul lui David, menit sa fie casa de arme: mii de scuturi atârna acolo ?i tot scuturi de viteji.
Son 4:5  Cei doi sâni ai tai par doi pui de caprioara, doi iezi care pasc printre crini.
Son 4:6  Pâna nu se racore?te ziua, pâna nu se-ntinde umbra serii, voi veni la tine, colina de mirt, voi veni la tine, munte de tamâie.
Son 4:7  Cât de frumoasa e?ti tu, draga mea, ?i fara nici o pata.
Son 4:8  Vino din Liban, mireasa mea, vino din Liban cu mine! Degraba coboara din Amana, din Senir ?i din Hermon, din culcu?ul leilor ?i din mun?i cu leoparzi!
Son 4:9  Sora mea, mireasa mea, tu mi-ai robit inima numai c-o privire a ta ?i cu colanu-?i de la sân.
Son 4:10  Cât de dulce, când dezmierzi, e?ti tu sora mea mireasa; ?i mai dulce decât vinul este mângâierea ta. ?i mireasma ta placuta este mai presus de orice mir.
Son 4:11  Ale tale buze miere izvorasc, iubito, miere curge, lapte curge, de sub limba ta; mirosul îmbracamintei tale e mireasma de Liban.
Son 4:12  E?ti gradina încuiata, sora mea, mireasa mea, fântâna acoperita ?i izvor pecetluit.
Son 4:13  Vlastarele tale cladesc un paradis de rodii cu fructe dulci ?i minunate, având pe margini arbu?ti care revarsa miresme:
Son 4:14  Nard, ?ofran ?i scor?i?oara cu trestie mirositoare, cu felurime de copaci, ce tamâie lacrimeaza, cu mirt ?i cu aloe ?i cu arbu?ti mirositori.
Son 4:15  În gradina-i o fântâna, un izvor de apa vie ?i pâraie din Libar.
Son 4:16  Scoala vânt de miazanoapte, vino vânt de miazazi, sufla?i prin gradina mea ?i miresmele-i stârni?i; iar iubitul meu sa vina, în gradina sa sa intre ?i din roadele ei scumpe sa culeaga, sa manânce.
Son 5:1  Venit-am în gradina mea, sora mea, mireasa mea! Strâns-am miruri aromate, miere am mâncat din faguri, vin ?i lapte am baut. Mânca?i ?i be?i, prieteni, fi?i be?i de dragoste, iubi?ii mei!
Son 5:2  De dormit dormeam, dar inima-mi veghea. Auzi glasul celui drag! El la u?a batând zice: Deschide-mi, surioara, deschide-mi, iubita mea, porumbi?a mea, curata mea, capul îmi este plin de roua ?i parul ud de vlaga nop?ii.
Son 5:3  Haina eu mi-am dezbracat, cum s-o-mbrac eu iar? Picioarele mi le-am spalat, cum sa le murdaresc eu iar?
Son 5:4  Iubitul mâna pe fereastra a întins ?i inima mi-a tresarit.
Son 5:5  Iute sa-i deschid m-am ridicat, din mâna mir mi-a picurat, mir din degete mi-a curs pe închizatoarea u?ii.
Son 5:6  Celui drag eu i-am deschis, dar iubitul meu plecase; sufletu-mi încremenise, când cel drag mie-mi vorbise; iata eu l-am cautat, dar de-aflat nu l-am aflat; pe nume l-am tot strigat, dar raspuns nu mi s-a dat.
Son 5:7  Întâlnitu-m-au strajerii, cei ce târgul strajuiesc, m-au izbit ?i m-au ranit ?i valul mi l-au luat, cei ce zidul îl pazesc.
Son 5:8  Fete din Ierusalim, va jur: De-ntâlni?i pe dragul meu, ce sa-i spune?i oare lui? Ca-s bolnava de iubire.
Son 5:9  Ce are iubitul tau mai mult ca al?ii, o tu, cea mai frumoasa-ntre femei? Cu cît iubitul tau e mai ales ca al?i iubi?i, ca sa ne rogi a?a cu juramânt?
Son 5:10  Iubitul meu e alb ?i rumen, ?i între zeci de mii este întâiul.
Son 5:11  Capul lui, aur curat; parul lui, par ondulat, negru-nchis, pana de corb.
Son 5:12  Ochii lui sunt porumbei, ce în lapte trupu-?i scalda, la izvor stând mul?umi?i.
Son 5:13  Trandafir mirositor sunt obrajii lui, strat de ierburi aromate. Iar buzele lui, la fel cu crinii ro?ii, în mir mirositor sunt scaldate.
Son 5:14  Braiele-i sunt drugi de aur cu topaze împodobite; pieptul lui e scut de filde? cu safire ferecat.
Son 5:15  Stâlpi de marmura sunt picioarele lui, pe temei de aur a?ezate. Înfa?i?area lui e ca Libanul ?i e mare? ca cedrul.
Son 5:16  Gura lui e negrait de dulce ?i totul este în el fermecator; iata cum este al meu iubit, fiice din Ierusalim, iata cum este al meu mire!
Son 6:1  Unde s-a dus iubitul tau, cea mai frumoasa-ntre femei? Unde-a plecat al tau iubit, ca sa-l cutam ?i noi cu tine?
Son 6:2  Iubitul meu în gradina lui s-a dus, în straturi d-aromate pline, sa-?i pasca turma acolo ?i crini frumo?i s-adune.
Son 6:3  Eu a iubitului meu sunt ?i el este al meu, el printre crini î?i pa?te iezii.
Son 6:4  Frumoasa e?ti, iubita mea, frumoasa e?ti ca Tir?a ?i ca Ierusalimul draga, dar ca ?i oastea în razboi temuta.
Son 6:5  Întoarce-?i ochii de la mine, ca ei de tot ma scot din fire.
Son 6:6  Parul tau turme de capre pare, ce din mun?i, din Galaad coboara. Din?ii tai par turma de oi tunse, ce din scaldatoare ies, facând doua ?iruri strânse ?i neavând nici o ?tirbitura.
Son 6:7  Doua jumata?i de rodii par obrajii tai, sub valul tau cel straveziu.
Son 6:8  Solomon are ?aizeci de regine ?i optzeci de concubine, iar fecioare socoteala cine le-o mai ?ine!
Son 6:9  Dar ea e numai una, porumbi?a mea, curata mea; una-i ea la a ei mama, singura nascuta în casa. Fetele când au vazut-o, laude i-au înal?at, iar reginele ?i concubinele osanale i-au cântat.
Son 6:10  Cine-i aceasta, ziceau ele, care ca zarea straluce?te ?i ca luna-i de frumoasa, ca soarele-i de luminoasa ?i ca oastea de razboi temuta?
Son 6:11  La gradina nucilor m-am dus, ca sa vad verdea?a vaii, daca a dat vita de vie ?i daca merii au înflorit.
Son 6:12  ?i nu ?tiu cum s-a petrecut, ca a mea inima m-a dus la o?tirea de razboi a viteazului meu neam.
Son 6:13  Întoarce-te, Sulamita! Întoarce-te, fa?a sa ?i-o privim! Ce privi?i la Sulamita, ca la hora din Mahanaim?
Son 7:1  Cât de frumoase sunt, domni?a, picioarele tale în sandale! Rotunda-i coapsa ta, ca un colan, de me?ter iscusit lucrat.
Son 7:2  Sânul tau e cupa rotunjita, pururea de vin tamâios plina; trupul tau e snop de grâu, încins frumos cu crini din câmp.
Son 7:3  Cei doi sâni ai tai par doi pui de caprioara, par doi pui gemeni ai unei gazele.
Son 7:4  Gâtul tau e stâlp de filde?; ochii tai sunt parca iezerele din He?bon, de la poarta Bat-Rabim. Nasul tau este ca turnul din Liban, ce prive?te spre Damasc.
Son 7:5  Capul tau este mare? cum e Carmelul, iar parul ?i-e de purpura; cu ale lui mândre ?uvi?e ?ii un rege în robie.
Son 7:6  Cât de frumoasa e?ti ?i atragatoare, prin dragala?ia ta, iubito!
Son 7:7  Ca finicul e?ti de zvelta ?i sânii tai par struguri atârna?i în vie.
Son 7:8  În finic eu m-a? sui - ziceam eu - ?i de-ale lui crengi m-a? apuca, sânii tai mi-ar fi drept struguri, suflul gurii tale ca mirosul de mere.
Son 7:9  Sarutarea ta mai dulce-ar fi ca vinul, ce-ar curge din bel?ug spre-al tau iubit, ale lui buze-nflacarate potolind.
Son 7:10  Eu sunt a lui, a celui drag. El dorul meu îl poarta.
Son 7:11  Hai, iubitul meu, la câmp, hai la ?ara sa petrecem!
Son 7:12  Mâine hai la vie sa vedem daca a dat vi?a de vie, merii de-au înmugurit, de-s aproape de-nflorit. ?i acolo î?i voi da dezmierdarile mele.
Son 7:13  Mandragorele miresme varsa ?i la noi acasa sunt multe fructe vechi ?i noi pe care, iubitul meu, pentru tine le-am pastrat.
Son 8:1  O, de mi-ai fi fost tu fratele meu ?i sa fi supt la sinul mamei mele, atunci pe uli?a de te-ntâlneam, cu drag, prelung te sarutam ?i nimeni nu-ndraznea osândei sa ma dea.
Son 8:2  Te-a? fi luat ?i-n casa mamei te-a? fi dus, în casa celei ce m-a nascut; tu graiuri dulci mi-ai fi spus ?i eu cu drag ?i-a? fi dat vin bun ?i must de rodii.
Son 8:3  Stânga sa este sub capul meu ?i cu dreapta-i ma cuprinde.
Son 8:4  Fete din Ierusalim, va jur pe cerboaicele ?i gazele din câmp, nu trezi?i pe draga mea pâna nu-i va fi ei voia!
Son 8:5  Cine se înal?a din pustiu, sprijinita de-al sau drag? Sub marul acesta am trezit iubirea ta, aici unde te-a nascut ?i ?i-a dat lumina zilei mama ta.
Son 8:6  Ca pecete pe sânul tau ma poarta, poarta-ma pe mâna ta ca pe o bra?ara! Ca iubirea ca moartea e de tare ?i ca iadul de grozava este gelozia. Sage?ile ei sunt sage?i de foc ?i flacara ei ca fulgerul din cer.
Son 8:7  Marea nu poate stinge dragostea, nici râurile s-o potoleasca; de-ar da cineva pentru iubire toate comorile casei sale, cu dispre? ar fi respins acela.
Son 8:8  Avem o mica surioara, care sâni nu are înca. Ce-am face cu sora noastra, când ea ar fi pe?ita?
Son 8:9  Zid de piatra de-ar fi ea, coroana de argint i-am face; iar u?a dac-ar fi, cu lemn de cedru am captu?i-o.
Son 8:10  Zid sunt eu acum ?i sânii mei sunt turnuri; drept aceea în ochii lui eu am aflat pacea.
Son 8:11  Solomon avea o vie pe coasta Baal-Hamon; el a dat via lucratorilor, s-o lucreze ?i sa-i dea fiecare la rod o mie de sicli de argint.
Son 8:12  Via mea este la mine acasa; mia de sicli sa fie a ta, Solomoane, ?i doua sute numai pentru cei ce pazesc roadele ei!
Son 8:13  O, tu, ce în gradini sala?luie?ti, prietenii vor sa-?i asculte glasul; fa-ma sa-l aud ?i eu cu ei!
Son 8:14  Fugi degrab, iubitul meu, sprinten ca o caprioara fii, fii ca puiul cel de cerb, peste mun?ii cei îmbalsama?i!


\end{document}