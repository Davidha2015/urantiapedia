\begin{document}

\title{Isaiah}

Isa 1:1  Vedenia lui Isaia, fiul lui Amos, pe care a vazut-o despre Iuda ?i Ierusalim, în vremea lui Ozia, Iotam, Ahaz ?i Iezechia, regii lui Iuda.
Isa 1:2  Asculta, cerule, ?i ia aminte, pamântule, ca Domnul graie?te: Hranit-am feciori ?i i-am crescut, dar ei s-au razvratit împotriva Mea.
Isa 1:3  Boul î?i cunoa?te stapânul ?i asinul ieslea domnului sau, dar Israel nu Ma cunoa?te; poporul Meu nu Ma pricepe.
Isa 1:4  Vai ?ie neam pacatos, popor împovarat de nedreptate, soi rau, fii ai pieirii! Ei au parasit pe Domnul, tagaduit-au pe Sfântul lui Israel, întorsu-I-au spatele.
Isa 1:5  Pe unde sa mai fi?i lovi?i voi, cei ce mereu va razvrati?i? Tot capul va este numai rani ?i toata inima slabanogita.
Isa 1:6  Din cre?tet pâna în talpile picioarelor nu-i nici un loc sanatos; totul este numai plagi, vânatai ?i rani pline de puroi, necura?ate, nemuiate cu untdelemn ?i nelegate.
Isa 1:7  ?ara voastra este pustiita, ceta?ile voastre arse cu foc, ?arinile voastre le manânca strainii înaintea ochilor vo?tri, totul este pustiit, ca la nimicirea Sodomei.
Isa 1:8  Sionul ajuns-a ca o coliba într-o vie, ca o coverca într-o bostanarie, ca o cetate împresurata!
Isa 1:9  Daca Domnul Savaot nu ne-ar fi lasat o rama?ira, am fi ajuns ca Sodoma ?i ne-am fi asemanat cu Gomora.
Isa 1:10  Asculta?i cuvântul Domnului, voi conducatori ai Sodomei, lua?i aminte la înva?atura Domnului, voi popor al Gomorei!
Isa 1:11  Ce-mi folose?te mul?imea jertfelor voastre?, zice Domnul. M-am saturat de arderile de tot cu berbeci ?i de grasimea vi?eilor gra?i ?i nu mai vreau sânge de tauri, de miei ?i de ?api!
Isa 1:12  Când venea?i sa le aduce?i, cine vi le ceruse? Nu mai calca?i în curtea templului Meu!
Isa 1:13  Nu mai aduce?i daruri zadarnice! Tamâierile Îmi sunt dezgustatoare; lunile noi, zilele de odihna ?i adunarile de la sarbatori nu le mai pot suferi. Însa?i praznuirea voastra e nelegiuire!
Isa 1:14  Urasc lunile noi ?i sarbatorile voastre sunt pentru Mine o povara. Ajunge!
Isa 1:15  Când ridica?i mâinile voastre catre Mine, Eu Îmi întorc ochii aiurea, ?i când înmul?i?i rugaciunile voastre, nu le ascult. Mâinile voastre sunt pline de sânge; spala?i-va, cura?i?i-va!
Isa 1:16  Nu mai face?i rau înaintea ochilor Mei. Înceta?i odata!
Isa 1:17  Înva?a?i sa face?i binele, cauta?i dreptatea, ajuta?i pe cel apasat, face?i dreptate orfanului, apara?i pe vaduva!
Isa 1:18  Veni?i sa ne judecam, zice Domnul. De vor fi pacatele voastre cum e cârmâzul, ca zapada le voi albi, ?i de vor fi ca purpura, ca lâna alba le voi face.
Isa 1:19  De ve?i vrea ?i de Ma ve?i asculta, bunata?ile pamântului ve?i mânca.
Isa 1:20  Iar de nu ve?i vrea ?i nu Ma ve?i asculta, atunci sabia va va mânca, caci gura Domnului graie?te.
Isa 1:21  Cum a ajuns ca o desfrânata cetatea cea credincioasa ?i plina de dreptate? Dreptatea locuia în ea, iar acum este plina de uciga?i.
Isa 1:22  Argintul tau s-a prefacut în zgura ?i vinul tau este amestecat cu apa;
Isa 1:23  Mai-marii tai sunt razvrati?i ?i parta?i cu ho?ii; to?i iubesc darurile ?i umbla dupa rasplata. Ei nu judeca orfanul, iar pricina vaduvei nu ajunge pâna la ei.
Isa 1:24  Pentru aceasta zice Domnul, Dumnezeul Savaot, puternicul lui Israel: Razbuna-Ma-voi împotriva asupritorilor Mei ?i Ma voi întarâta cu razbunare asupra vrajma?ilor Mei!
Isa 1:25  Voi întoarce mâna Mea împotriva ta ?i te voi cura?i de toata zgura ta, ca în cuptor.
Isa 1:26  Voi întoarce judecatorii tai sa judece ca la început ?i sfetnicii tai ca odinioara. Dupa aceasta te vei putea numi iara?i cetate dreapta, ora? credincios.
Isa 1:27  Sionul va fi rascumparat prin judecata ?i locuitorii sai care se vor întoarce la credin?a, prin dreptate.
Isa 1:28  Domnul va zdrobi pe cei razvrati?i, iar cei ce au parasit pe Domnul vor fi nimici?i.
Isa 1:29  Ei vor fi ru?ina?i pentru dumbravile sfinte pe care le-au îndragit ?i se vor ro?i la fa?a din pricina gradinilor pe care le-au ales.
Isa 1:30  Vor fi ca un stejar ale carui frunze cad ?i ca o gradina fara nici un strop de apa.
Isa 1:31  Cel puternic va fi ca puzderiile de câl?i ?i faptele lui ca o leasa de maracini. ?i aceia ?i aceasta vor arde laolalta ?i nimeni nu va putea sa-i stinga.
Isa 2:1  Vedenia pe care a vazut-o Isaia, fiul lui Amos, pentru Iuda ?i Ierusalim.
Isa 2:2  Fi-va în vremurile cele de pe urma, ca muntele templului Domnului va fi întarit peste vârfurile mun?ilor ?i se va ridica pe deasupra dealurilor. ?i toate popoarele vor curge într-acolo.
Isa 2:3  Multe popoare vor veni ?i vor zice: "Veni?i sa ne suim în muntele Domnului, în casa Dumnezeului lui Iacov, ca El sa ne înve?e caile Sale ?i sa mergem pe cararile Sale". Caci din Sion va ie?i legea ?i cuvântul lui Dumnezeu din Ierusalim.
Isa 2:4  El va judeca neamurile ?i la popoare fara de numar va da legile Sale. Preface-vor sabiile în fiare de pluguri ?i lancile lor în cosoare. Nici un neam nu va mai ridica sabia împotriva altuia ?i nu vor mai înva?a razboiul.
Isa 2:5  Voi, cei din casa lui Iacov, veni?i sa umblam în lumina Domnului!
Isa 2:6  Tu ai lepadat neamul Tau, casa lui Iacov. Ea este plina de vrajitori ?i de magi ca Filistenii ?i ea face legamânt cu cei de alt neam.
Isa 2:7  Pamântul ei este plin de aur ?i de argint, de comori fara numar, pamântul ei este plin de cai ?i de caru?e fara sfâr?it.
Isa 2:8  Pamântul ei este plin de idoli ?i locuitorii se închina la lucrul mâinilor lor, înaintea celor facu?i de degetele lor.
Isa 2:9  ?i omul va fi umilit ?i muritorul va fi pogorât ?i Tu nu-i vei ierta!
Isa 2:10  Intra?i în crapaturile stâncilor ?i ascunde?i-va în pulbere, din pricina fricii de Dumnezeu, de stralucirea slavei Lui.
Isa 2:11  Ochii celui mândru vor fi smeri?i, mândria celor de rând va fi pogorâta ?i numai Domnul în ziua aceea va fi ridicat în slavi,
Isa 2:12  Ca Domnul Savaot va avea ziua Lui, se va ridica împotriva a tot ceea ce este mândru ?i seme? ?i-l va pogorî.
Isa 2:13  Împotriva tuturor cedrilor Libanului ?i stejarilor celor înal?i ai Vasanului.
Isa 2:14  Împotriva tuturor mun?ilor înal?i ?i colinelor celor mândre.
Isa 2:15  Împotriva tuturor turnurilor ridicate sus ?i zidurilor întarite.
Isa 2:16  Împotriva tuturor corabiilor Tarsisului ?i lucrurilor de pre?.
Isa 2:17  Mândria omului va fi pogorâta ?i seme?ia celor muritori va fi smerita; în ziua aceea numai Domnul va fi înalt;
Isa 2:18  ?i to?i idolii vor pieri.
Isa 2:19  Iar oamenii vor intra în scorburile stâncilor, în prapastiile ?i în crapaturile pamântului, de frica Domnului ?i de stralucirea slavei Lui, când va veni El ca sa loveasca pamântul.
Isa 2:20  În ziua aceea idolii de argint ?i de aur, pe care omul i-a facut ca sa li se închine, vor fi parasi?i ca sa fie sala? ?obolanilor ?i liliecilor.
Isa 2:21  Iar el va intra în crapaturile stâncilor ?i în prapastiile mun?ilor, de frica Domnului ?i de stralucirea slavei Lui, când va veni El ca sa loveasca pamântul.
Isa 2:22  Nu mai nadajdui?i în omul cel muritor, în narile caruia nu este decât o suflare! Oare, ce putere are el?
Isa 3:1  Iata, Domnul Dumnezeu Savaot va lua din Ierusalim ?i din Iuda orice sprijin ?i orice ajutor, orice hrana de pâine ?i orice sprijin de apa,
Isa 3:2  Pe viteaz ?i pe omul de lupta, pe judecator ?i pe prooroc, pe prezicator ?i pe batrân;
Isa 3:3  Pe capetenia peste cincizeci, pe sfatuitor, pe în?elept, pe fermecator ?i pe ghicitor.
Isa 3:4  Voi pune baie?i capetenii peste ei ?i copiii vor domni peste aceia.
Isa 3:5  În popor se vor strâmtora unul pe altul ?i fiecare va împila pe aproapele sau; cel tânar se va purta obraznic cu cel batrân ?i cel de neam rau cu cel de neam bun.
Isa 3:6  Atunci va alerga omul la fratele sau, în casa tatalui sau, ?i-i va zice: Tu mai ai o haina, fii capetenie peste noi ?i sa fie aceste darâmaturi sub mâna ta.
Isa 3:7  Iar acela cu juramânt se va lepada, zicând: "Nu pot vindeca ranile poporului! N-am în casa mea nici pâine, nici haina, nu ma face?i conducator peste popor!"
Isa 3:8  Ierusalimul va ajunge darâmatura ?i Iuda este gata sa cada, caci limba lor ?i gândurile lor sunt împotriva Domnului ?i dispre?uiesc privirea slavei Lui.
Isa 3:9  Înfa?i?area lor marturise?te împotriva lor, caci ei î?i vadesc pacatele lor ca Sodoma, în loc sa fie ascunda. Vai de ei! Caci ei ?i-au facut loru?i rau...
Isa 3:10  Fericit este omul drept, ca el va mânca din rodul lucrurilor lui.
Isa 3:11  Vai de cel rau, ca rautatea este a lui ?i va fi judecat dupa faptele lui.
Isa 3:12  Poporul meu este asuprit de ni?te copii, ?i femeile domnesc peste el. Poporul meu! Cei care te conduc te ratacesc ?i te abat de la calea pe care tu mergi.
Isa 3:13  Domnul Se ridica la judecata ?i sta ca sa judece pe poporul Sau.
Isa 3:14  Domnul intra la judecata cu batrânii ?i cârmuitorii poporului Sau ?i zice: "Voi, voi a?i pustiit via Mea ?i prada luata de la cei sarmani se afla în casele voastre.
Isa 3:15  Pentru ce a?i zdrobit pe poporul Meu ?i a?i sfarâmat fa?a celor sarmani?" zice Domnul Dumnezeu Savaot.
Isa 3:16  ?i mai zice Domnul: "Pentru ca fiicele Sionului sunt atât de mândre ?i umbla cu capul pe sus ?i cu priviri obraznice, cu pa?i domoli, cu zanganit de inele la picioarele lor,
Isa 3:17  Domnul va ple?uvi cre?tetul capului fiicelor Sionului, Domnul va descoperi goliciunea lor".
Isa 3:18  În ziua aceea va lua Domnul toate podoabele: inele, sori, luni?e,
Isa 3:19  Cercei, bra?ari, valuri,
Isa 3:20  Cununi, lan?i?oare, cingatoare, miresme, talismane,
Isa 3:21  Inele, verigi de nas,
Isa 3:22  Ve?minte de sarbatoare, mantii, ?aluri, pungi,
Isa 3:23  Oglinzi, pânzeturi sub?iri, turbane ?i tunici.
Isa 3:24  Atunci va fi în loc de miresme, putreziciune, ?i în loc de cingatori, frânghie, în loc de cârlion?i facu?i cu fierul, ple?uvie, în loc de ve?mânt pre?ios, zdren?e, ?i în loc de frumuse?e: pecete de robie.
Isa 3:25  Locuitorii Sionului vor cadea de sabie ?i vitejii lui în razboaie.
Isa 3:26  Por?ile fiicei Sionului vor scâr?âi ?i se vor jeli ?i, jefuita, ea va ?edea despuiata pe pamânt.
Isa 4:1  În ziua aceea ?apte femei se vor certa pentru un singur om, zicând: "Noi vom mânca pâinea noastra ?i vom purta ve?mintele noastre. Nu cerem altceva decât sa purtam numele tau. Ridica ocara noastra!"
Isa 4:2  În ziua aceea se va arata mladi?a Domnului în podoaba ?i în slava ?i roadele pamântului în marire ?i în cinste pentru aceia din Israel care vor fi scapat.
Isa 4:3  Rama?i?a Sionului ?i cei ce vor fi scapat cu via?a din Ierusalim se vor chema sfin?i ?i oricine va fi înscris sa traiasca în Ierusalim.
Isa 4:4  Când Domnul va fi spalat necura?enia fiicelor Sionului ?i va fi ?ters faradelegile din mijlocul lui prin duhul drepta?ii ?i al nimicirii,
Isa 4:5  Domnul va veni pe Muntele Sionului ?i în adunarile Sale ca un nor ?i ca un fum ziua, iar noaptea ca un foc stralucitor ?i ca o vapaie. Ca peste tot locul slava Domnului va fi acoperamânt:
Isa 4:6  Fi-va în timpul zilei cort, care sa-l apere de caldura ?i sa-l adaposteasca de vreme rea ?i de ploaie.
Isa 5:1  Vreau sa cânt pentru prietenul meu cântecul lui de dragoste pentru via lui. Prietenul meu avea o vie pe o coasta manoasa.
Isa 5:2  El a sapat-o, a cura?it-o de pietre ?i a sadit-o cu vi?a de bun soi. Ridicat-a în mijlocul ei un turn, sapat-a ?i un teasc. ?i avea nadejde ca va face struguri, dar ea a facut agurida.
Isa 5:3  ?i acum voi, locuitori ai Ierusalimului ?i barba?i ai lui Iuda, fi?i judecatori intre mine ?i via mea.
Isa 5:4  Ce se putea face pentru via mea ?i n-am facut eu? Pentru ce atunci când nadajduiam sa-mi rodeasca struguri, mi-a rodit agurida?
Isa 5:5  Acum va voi face sa ?ti?i cum ma voi purta cu via mea: Strica-voi gardul ei ?i ea va fi pustiita, darâma-voi zidul ei ?i va fi calcata în picioare.
Isa 5:6  ?i o voi pustii! Nu va mai fi taiata, nici sapata ?i o vor napadi spinii ?i balariile. De asemenea ?i norilor le voi da porunca sa nu-?i mai verse ploaia peste ea.
Isa 5:7  Dar via Domnului Savaot este casa lui Israel, iar oamenii din Iuda sunt sadirea Sa draga. El nadajduia ca acesta sa fie un popor fara pacate, dar iata-l plin de sânge. Nadajduit-a sa-I rodeasca dreptate, dar iata: razvratire.
Isa 5:8  Vai voua care cladi?i casa lânga casa ?i gramadi?i ?arini lânga ?arini pâna nu mai ramâne nici un loc, ca sa fi?i numai voi stapânitori în ?ara!
Isa 5:9  Urechile mele au auzit de asemenea acest juramânt al Domnului Savaot: "Jur ca aceste case multe, mari ?i frumoase, vor fi pustii ?i nimeni nu va mai locui în ele.
Isa 5:10  Zece pogoane de vie vor rodi un bat, ?i un homer de samân?a, numai o efa".
Isa 5:11  Vai de cei ce dis-de-diminea?a alearga dupa bauturi îmbatatoare; vai de cei ce pâna târziu seara se înfierbânta cu vin!
Isa 5:12  Cei care doresc, la ospe?ele lor, chitara, harpa, toba, flaut ?i vin ei nu iau în seama faptele Domnului ?i nu vad lucrurile mâinilor Sale.
Isa 5:13  Pentru aceasta poporul meu va fi dus în robie fara sa bage de seama, mai-marii sai vor fi doborâ?i de foame, iar gloata se va usca de sete!
Isa 5:14  De aceea ?i iadul ?i-a marit de doua ori lacomia lui, cascat-a gura sa peste masura; acolo se vor coborî marirea Sionului ?i gloatele sale, chiotele de veselie...
Isa 5:15  Omul cel muritor va fi smerit ?i umilit ?i ochii celor mândri vor fi pogorâ?i.
Isa 5:16  Dar Domnul Savaot este mare prin judecata Sa ?i Dumnezeul cel sfânt este sfânt prin dreptatea Sa.
Isa 5:17  Oile vor pa?te în voie, iar strainii se vor hrani în locurile manoase, lasate de cei boga?i.
Isa 5:18  Vai de cei ce î?i atrag pedeapsa ca ?i cu ni?te frânghii ?i plata pacatului ca ?i cu ni?te ?treanguri,
Isa 5:19  Caci ei zic: "Grabeasca Domnul sa-?i faca lucrul Sau curând, ca sa vedem ?i sa se plineasca planul Sfântului lui Israel, ca sa-l cunoa?tem".
Isa 5:20  Vai de cei ce zic raului bine ?i binelui rau; care numesc lumina întuneric ?i întunericul lumina; care socotesc amarul dulce ?i dulcele amar!
Isa 5:21  Vai de cei care sunt în?elep?i în ochii lor ?i pricepu?i dupa gândurile lor!
Isa 5:22  Vai de cei viteji la baut vin ?i me?teri la facut bauturi îmbatatoare!
Isa 5:23  Vai de cei ce dau dreptate celui nelegiuit pentru mita ?i lipsesc de dreptate pe cel drept!
Isa 5:24  Pentru aceasta, dupa cum paiele sunt mistuite de foc ?i iarba uscata de flacari, a?a radacina lor va fi topita ca pleava ?i floarea lor va fi spulberata precum este cenu?a, caci au calcat legea Domnului Savaot ?i au nesocotit cuvântul Sfântului lui Israel!
Isa 5:25  De aceea, mânia Domnului s-a aprins împotriva poporului Sau! El întinde mâna Sa spre el, îl love?te ?i mun?ii se clatina. Cadavrele lor sunt ca gunoiul pe cale. Cu toate acestea mânia Lui nu se domole?te ?i mâna Lui sta mereu întinsa
Isa 5:26  ?i va ridica steagul pentru un popor de departe ?i îl va chema de la capatul pamântului. Iata-l ca se zore?te ?i vine.
Isa 5:27  Nimeni din ai lui nu va obosi, nici va boli, nu va dormita, nici va adormi; nimeni nu-?i va descinge brâul ?i nici cureaua încal?amintei lui nu se va rupe.
Isa 5:28  Sage?ile lor sunt ascu?ite ?i arcurile lor gata sa traga. Copitele cailor sunt ca ?i cremenea cea tare, ro?ile caru?elor sunt ca o furtuna.
Isa 5:29  Strigatul, strigat de leu, racnesc ca puii de leu, mugesc ?i apuca prada, ?i nimeni nu roate s-o scape.
Isa 5:30  În vremea aceea fi-va împotriva lui un vuiet ca vuietul marii. To?i vor arunca privirea spre pamânt ?i iata: întuneric ?i strâmtorare; lumina se va întuneca întocmai ca o noapte, fara sa se mai iveasca zorile!
Isa 6:1  În anul mor?ii regelui Ozia, am vazut pe Domnul stând pe un scaun înalt ?i mare? ?i poalele hainelor Lui umpleau templul.
Isa 6:2  Serafimi stateau înaintea Lui, fiecare având câte ?ase aripi: cu doua î?i acopereau fe?ele, cu doua picioarele, iar cu doua zburau
Isa 6:3  ?i strigau unul catre altul, zicând: "Sfânt, sfânt, sfânt este Domnul Savaot, plin este tot pamântul de slava Lui!"
Isa 6:4  Din pricina acestor strigate, por?ile se zguduiau din ?â?ânele lor, iar templul s-a umplut de fum.
Isa 6:5  ?i am zis: "Vai mie, ca sunt pierdut! Sunt om cu buze spurcate ?i locuiesc în mijlocul unui popor cu buze necurate. ?i pe Domnul Savaot L-am vazut cu ochii mei!"
Isa 6:6  Atunci unul dintre serafimi a zburat spre mine, având în mâna sa un carbune, pe care îl luase cu cle?tele de pe jertfelnic.
Isa 6:7  ?i l-a apropiat de gura mea ?i a zis: "Iata s-a atins de buzele tale ?i va ?terge toate pacatele tale, ?i faradelegile tale le va cura?i".
Isa 6:8  ?i am auzit glasul Domnului care zicea: "Pe cine îl voi trimite ?i cine va merge pentru Noi?" ?i am raspuns: "Iata-ma, trimite-ma pe mine!"
Isa 6:9  ?i El a zis: "Du-te ?i spune poporului acestuia: Cu auzul ve?i auzi ?i nu ve?i în?elege ?i, uitându-va, va ve?i uita, dar nu ve?i vedea.
Isa 6:10  Ca s-a învârto?at inima poporului acestuia ?i cu urechile sale greu a auzit ?i ochii sai i-a închis, ca nu cumva sa vada cu ochii ?i cu urechile sa auda ?i cu inima sa în?eleaga ?i sa se întoarca la Mine ?i sa-l vindec".
Isa 6:11  ?i am zis: "Pâna când, Doamne!" Atunci El mi-a raspuns: "Pâna când ceta?ile vor fi pustiite ?i vor ramâne fara locuitori, ?i casele fara oameni ?i pamântul pustiu;
Isa 6:12  Pâna când Domnul va izgoni pe oameni ?i pustiirea va fi mare în mijlocul acestei ?ari.
Isa 6:13  ?i daca va ramâne înca unul din zece, ?i acela va fi harazit focului, ca ?i terebintul ?i stejarul, ale caror trunchiuri sunt trântite la pamânt. Din butucul ramas va lastari o mladi?a sfânta".
Isa 7:1  ?i a fost în zilele lui Ahaz, fiul lui Iotam, fiul lui Ozia, regele lui Iuda, ca s-a suit Re?in, regele Siriei, împreuna cu Pecah, fiul lui Remalia, regele lui Israel, ca sa cuprinda Ierusalimul. ?i n-a izbutit ca sa-l cuprinda.
Isa 7:2  Atunci a venit cineva sa dea de ?tire casei lui David, zicând: "Armata Sirienilor a tabarât în Efraim". ?i inima regelui ?i a poporului tremura de spaima în ziua aceea, precum tremura copacii padurii din pricina vântului.
Isa 7:3  ?i a grait Domnul catre Isaia, zicând: "Ie?i întru întâmpinarea lui Ahaz, tu ?i ?ear-Ia?ub, fiul tau, la capatul canalului lacului celui de sus, pe drumul ?arinii nalbitorului,
Isa 7:4  ?i îi vei zice: "Ia aminte, fii lini?tit ?i nu te teme ?i inima ta sa nu se slabeasca din pricina acestor doi taciuni care fumega: de iu?imea mâniei lui Re?in ?i a lui Aram ?i a fiului lui Remalia
Isa 7:5  De vreme ce Aram a hotarât pustiirea ta, împreuna cu Efraim ?i cu fiul Remaliei, zicând:
Isa 7:6  "Sa ne suim în Iuda, sa-l speriem, sa ne facem stapâni pe el ?i sa punem rege peste el pe feciorul lui Tabeel".
Isa 7:7  A?a zice Domnul Dumnezeu: "Aceasta nu va fi, nici nu se va împlini!
Isa 7:8  Caci  capetenia Aramului este Damascul ?i mai mare peste Damasc este Re?in. - Mai sunt înca ?aizeci ?i cinci de ani ?i Efraim va pieri din rândul popoarelor.
Isa 7:9  ?i capitala lui Efraim este Samaria ?i mai mare peste Samaria este feciorul lui Remalia. Daca nu crede?i, ve?i fi zdrobi?i!"
Isa 7:10  ?i Isaia mai grai catre Ahaz:
Isa 7:11  "Cere un semn de la Domnul Dumnezeul tau, în adâncurile iadului sau în înal?imile cele de sus".
Isa 7:12  ?i a spus Ahaz: "Nu voi cere ?i nu voi ispiti pe Domnul!"
Isa 7:13  ?i a zis Isaia: "Asculta?i voi cei din casa lui David! Nu va ajunge sa obosi?i pe oameni, de veni?i sa obosi?i ?i pe Dumnezeul meu?
Isa 7:14  Pentru aceasta Domnul meu va va da un semn: Iata, Fecioara va lua în pântece ?i va na?te fiu ?i vor chema numele lui Emanuel.
Isa 7:15  El se va hrani cu lapte ?i cu miere pâna în vremea când va ?ti sa arunce raul ?i sa aleaga binele.
Isa 7:16  Ca înainte ca fiul acesta sa ?tie sa dea la o parte raul ?i sa aleaga binele, pamântul de care î?i este teama, din pricina celor doi regi, va fi pustiit.
Isa 7:17  Dar Domnul va aduce peste tine, peste poporul tau ?i peste casa tatalui tau, vremuri care n-au mai venit de când Efraim s-a desfacut de Iuda; va aduce pe regele Asiriei.
Isa 7:18  ?i va fi ca în ziua aceea Domnul va chema mu?tele care se afla la capatul Nilului - fluviul Egiptului - ?i albinele din pamântul Asiriei;
Isa 7:19  ?i vor veni ?i se vor a?eza cu toate în vaile cele prapastioase ?i în crapaturile stâncilor ?i în toate tufi?urile ?i în toate ?inuturile nelocuite.
Isa 7:20  În vremea aceea, va rade Domnul cu un brici, luat de împrumut de dincolo de Eufrat, pe regele Asiriei, capul, parul de pe trup ?i îi va smulge ?i barba.
Isa 7:21  În vremea aceea, cine va hrani o vaca ?i doua oi
Isa 7:22  Va avea bel?ug de unt din pricina mul?imii laptelui ?i cei ce vor fi ramas în ?ara se vor hrani cu smântâna ?i cu miere.
Isa 7:23  În ziua aceea, unde era un loc de o mie de butuci pe pre? de o mie de sicli, va fi plin de spini ?i de balarii.
Isa 7:24  Acolo oamenii vor intra înarma?i cu arcuri ?i cu sage?i, caci toata ?ara va fi plina de spini ?i de balarii.
Isa 7:25  ?i în to?i mun?ii care erau cura?a?i cu sapaliga, tu nu te vei duce, de frica spinilor ?i a balariilor. Acolo se va da drumul boilor ?i oilor, ca sa calce pamântul.
Isa 8:1  ?i a zis Domnul catre mine: "Ia o carte mare ?i scrie deasupra ei cu slove omene?ti: "Maher-?alal-Ha?-Baz" (grabnic-prada-apropiat-jaf).
Isa 8:2  Adu-Mi martori credincio?i pe Urie preotul ?i pe Zaharia, fiul lui Ieberechia".
Isa 8:3  Atunci m-am apropiat de prooroci?a ?i a luat în pântece ?i a nascut un fiu. ?i a zis Domnul catre mine: "Pune-i numele Maher-?alal-Ha?-Baz".
Isa 8:4  Caci înainte ca baiatul sa zica: "tata ?i mama!", toata boga?ia Damascului ?i prada Samariei vor fi duse înaintea regelui Asiriei".
Isa 8:5  ?i mi-a mai grait Domnul astfel:
Isa 8:6  "Fiindca poporul acesta a nesocotit apele Siloamului, care curg lin, ?i a tremurat înaintea lui Re?in, feciorul Remaliei,
Isa 8:7  Iata acum ca Domnul va aduce peste ei apele cele mari ?i furioase ale Eufratului: pe regele Asiriei ?i toata stralucirea lui. Ele vor trece peste toate zagazurile ?i vor da afara peste malurile lui.
Isa 8:8  ?i se va revarsa în Iuda, îl va îneca ?i îl va umple de apa, va ajunge pâna la gât ?i cu revarsarile lui întinse va acoperi toata ?ara.
Isa 8:9  Cu noi este Dumnezeu, în?elege?i neamuri ?i va pleca?i. Auzi?i pâna la marginile pamântului, cei puternici pleca?i-va. De va ve?i întari, iara?i ve?i fi birui?i.
Isa 8:10  ?i orice sfat ve?i sfatui, îl va risipi Domnul, ?i cuvântul pe care îl ve?i grai nu va ramâne întru voi, caci cu noi este Dumnezeu!"
Isa 8:11  A?a îmi zicea mie Domnul, ?inând peste mine mâna Sa cea tare ?i însuflându-mi sa nu umblu pe caile acestui popor. Apoi mi-a zis:
Isa 8:12  "Nu numi?i uneltire tot ceea ce poporul acesta socote?te uneltire, ?i nu va teme?i, nici nu va înfrico?a?i de ceea ce se tem ei.
Isa 8:13  Numai pe Domnul Savaot socoti?i-L sfânt, de El sa va teme?i ?i sa va înfrico?a?i.
Isa 8:14  El va fi pentru voi piatra de încercare ?i stânca de poticnire pentru cele doua case ale lui Israel, cursa ?i la? pentru cei ce locuiesc în Ierusalim.
Isa 8:15  ?i mul?i se vor poticni, vor cadea ?i se vor sfarâma, vor fi prin?i în cursa ?i vor fi du?i în robie!"
Isa 8:16  Voi strânge laolalta aceasta marturie ?i voi sigila aceasta înva?atura pentru ucenicii mei.
Isa 8:17  Voi a?tepta deci pe Domnul, Care î?i ascunde fa?a Sa de la casa lui Iacov ?i voi nadajdui întru El.
Isa 8:18  Iata eu ?i pruncii pe care mi i-a dat Dumnezeu spre semne ?i minuni în Israel, din partea Domnului Savaot, Care locuie?te în Muntele Sionului.
Isa 8:19  ?i când va vor zice: "Întreba?i pe cei ce cheama mor?ii ?i ghicitorii care ?optesc ?i bolborosesc", sa le raspunde?i: "Nu se cuvine oare poporului sa alerge la Dumnezeul sau? Sa întrebe oare pe mor?i pentru soarta celor vii?"
Isa 8:20  Întreba?i legea ?i descoperirea! De nu va vor vorbi asemenea cuvântului acesta, atunci nu-i lumina în ei.
Isa 8:21  Vor rataci pe pamânt flamânzi ?i cumplit apasa?i, ?i în vremea foametei î?i vor arata col?ii ?i vor huli pe regele lor ?i pe Dumnezeul lor.
Isa 8:22  Apoi î?i vor întoarce privirea spre pamânt ?i iata ca acolo va fi strâmtorare, întuneric ?i scârba ?i nevoie! Dar noaptea va fi alungata!
Isa 9:1  Caci nu va mai fi întuneric pentru ?ara care era în nevoie. În vremurile de dedemult el a supus pamântul Zabulonului ?i ?inutul lui Neftali; în vremurile cele de pe urma el va acoperi de slava calea marii, celalalt ?arm al Iordanului, Galileea neamurilor.
Isa 9:2  Poporul care locuia întru întuneric va vedea lumina mare ?i voi cei ce locuia?i în latura umbrei mor?ii lumina va straluci peste voi.
Isa 9:3  Tu vei înmul?i poporul ?i vei spori bucuria lui. El se va veseli înaintea Ta, cum se bucura oamenii în timpul seceri?ului ?i se veselesc la împar?irea prazilor.
Isa 9:4  Caci jugul ce-l apasa, ?i toiagul ce-l love?te, ?i nuiaua ce-l asupre?te, Tu le vei sfarâma, ca în zilele lui Madian.
Isa 9:5  Încal?amintea cea zgomotoasa de om razboinic ?i haina cea stropita de sânge vor fi aruncate în foc ?i mistuite în flacari!
Isa 9:6  Caci Prunc s-a nascut noua, un Fiu s-a dat noua, a Carui stapânire e pe umarul Lui ?i se cheama numele Lui: Înger de mare sfat, Sfetnic minunat, Dumnezeu tare, biruitor, Domn al pacii, Parinte al veacului ce va sa fie.
Isa 9:7  ?i mare va fi stapânirea Lui ?i pacea Lui nu va avea hotar. Va împara?i pe tronul ?i peste împara?ia lui David, ca s-o întareasca ?i s-o întemeieze prin judecata ?i prin dreptate, de acum ?i pâna-n veac. Râvna Domnului Savaot va face aceasta.
Isa 9:8  Cuvânt va trimite Domnul peste Iacob, ?i el se va pogorî peste Israel.
Isa 9:9  Ca sa ?tie tot poporul, Efraim ?i locuitorii Samariei, care întru mândria lor ?i întru seme?ia inimii lor zic:
Isa 9:10  "Caramizile au cazut, sa zidim cu piatra cioplita; smochinii au fost taia?i, sa punem cedri în locul lor!"
Isa 9:11  Ridica-va Domnul împotriva lui pe vrajma?ii lui Re?in ?i pe du?manii lui îi va înarma:
Isa 9:12  Pe Sirienii de la rasarit ?i pe Filistenii de la asfin?it; ?i vor mânca ace?tia pe Israel cu toata gura. Cu toate acestea mânia Lui nu se va potoli ?i mâna Lui tot întinsa va fi;
Isa 9:13  Dar poporul nu se va întoarce la Cel care îl lovise ?i nu va cauta pe Domnul Savaot.
Isa 9:14  ?i Domnul va taia din Israel, într-o singura zi, capul ?i coada, ramura de finic ?i trestia.
Isa 9:15  Batrânii ?i capeteniile sunt capul; proorocul ?i înva?atorul mincinos sunt coada.
Isa 9:16  Capeteniile acestui popor îl duc în ratacire ?i cei condu?i de ei vor pieri.
Isa 9:17  Pentru aceasta, Domnul nu se bucura de cei tineri ?i de orfanii lui ?i de vaduve nu-i este mila, fiindca to?i sunt nelegiui?i ?i rai ?i gura lor graie?te vorbe nesocotite. Pentru toate acestea, mânia Lui nu se va potoli ?i mâna Lui mereu întinsa va fi.
Isa 9:18  Ca faradelegea arde ca focul, care mistuie spinii ?i balariile uscate; el arde tot maracini?ul padurii, iar fumul se înal?a în rotocoale.
Isa 9:19  Din pricina iu?imii mâniei Domnului Savaot, pamântul va fi ca un jeratic, iar poporul va ajunge prada focului. Nimeni nu va cru?a pe vecinul sau.
Isa 9:20  Jefui-vor la dreapta ?i vor ramâne flamânzi; la stânga vor mânca ?i nu se vor satura; fiecare va mânca din carnea aproapelui sau:
Isa 9:21  Manase pe Efraim, Efraim pe Manase, ?i amândoi sunt împotriva lui Iuda. Pe lânga toate acestea, mânia Lui nu se va potoli ?i bra?ul Lui mereu întins va fi.
Isa 10:1  Vai de cei ce fac legi nedrepte ?i de cei ce scriu hotarâri silnice
Isa 10:2  Ca sa îndeparteze pe cei slabi de la judecata ?i sa rapeasca dreptatea sarmanilor poporului Meu, ca sa faca din vaduve prada lor ?i sa jefuiasca pe cei orfani!
Isa 10:3  Dar ce ve?i face voi în ziua pedepsirii ?i cum va ve?i feri de furtuna ce vine de departe? Catre cine ve?i fugi ca sa fi?i ajuta?i ?i unde ve?i lasa comorile voastre?
Isa 10:4  Fara mine vor merge cu frun?ile plecate printre robi ?i vor cadea printre cei uci?i ?i totu?i mânia Lui nu se va potoli ?i mâna Lui mereu întinsa va fi.
Isa 10:5  O, Asirie, varga mâniei Mele ?i toiagul urgiei Mele!
Isa 10:6  Împotriva unui neam fara de lege o voi trimite ?i împotriva unui popor al urgiei Mele o voi îndrepta, ca sa-l prade ?i sa-l jefuiasca ?i sa-l calce ca pe tina uli?elor.
Isa 10:7  Dar ea n-are aceea?i judecata ?i inima ei nu simte la fel; sa prade este în inima ei ?i sa nimiceasca neamuri fara numar!
Isa 10:8  Caci ea zice: "Oare mai-marii mei nu sunt ei laolalta regi?
Isa 10:9  Calno oare nu este ca ?i Carchemi?ul? ?i Hamatul ca Arpadul ?i Samaria ca Damascul?"
Isa 10:10  Cum a cuprins mâna Mea împara?iile idolilor, unde idolii erau mai numero?i decât în Ierusalim ?i în Samaria;
Isa 10:11  Precum am facut Samariei ?i idolilor ei, tot a?a voi face ?i Ierusalimului ?i chipurilor lui!
Isa 10:12  ?i când Domnul va sfâr?i tot lucrul Lui în muntele Sionului ?i în Ierusalim, atunci va pedepsi pe regele Asiriei pentru graiul cel mândru din inima lui ?i pentru seme?ia privirilor lui,
Isa 10:13  Ca a zis: "Prin puterea mâinii mele am facut aceasta ?i prin în?elepciunea mea; caci sunt priceput! Trecut-am peste grani?ele popoarelor, jefuit-am comorile lor ?i ca un atotputernic am dat jos de pe tron pe conducatori.
Isa 10:14  Mâna mea a apucat ca pe un cuib boga?iile popoarelor ?i, precum iei oua parasite, tot a?a am cuprins eu tot pamântul. ?i nimeni n-a scuturat aripile, n-a deschis ciocul ?i nici n-a scos vreun ?ipat!
Isa 10:15  Oare securea este ea marea?a fa?a de cel ce o ridica sau ferastraul se înal?a împotriva celui ce-l mânuie?te? Ca ?i cum varga ar da avânt celui care o ridica ?i toiagul ar însufle?i bra?ul care îl duce!
Isa 10:16  De aceea Domnul Dumnezeu Savaot va trimite prapadul în aceasta voinica o?tire asiriana ?i sanatatea lor o vor mistui frigurile ca un pârjol.
Isa 10:17  ?i lumina lui Israel se va face foc ?i Sfântul sau o flacara ?i va arde ?i va mistui spinii ?i balariile uscate, într-o singura zi!
Isa 10:18  ?i stralucirea padurii lui ?i a livezii lui va fi nimicita de sus ?i pâna jos.
Isa 10:19  Copacii rama?i din padurea lui vor fi a?a de pu?ini la numar, încât ?i un copil va putea sa-i numere.
Isa 10:20  În vremea aceea rama?i?a lui Iuda ?i cei scapa?i din casa lui Iacov nu se vor mai sprijini pe cel ce i-a lovit, ci se vor sprijini, cu credin?a, pe Dumnezeu, Sfântul lui Israel.
Isa 10:21  O rama?i?a din Iacov se va întoarce la Dumnezeul cel puternic.
Isa 10:22  Chiar daca poporul tau, Israele, ?ar fi ca nisipul marii, numai o rama?i?a se va întoarce. Nimicirea este hotarâta de dreptatea cea nemasurata.
Isa 10:23  Aceasta porunca de nimicire, Domnul Dumnezeu Savaot o va împlini în tot cuprinsul ?arii.
Isa 10:24  Pentru aceasta, a?a zice Domnul Dumnezeu Savaot: "Poporul Meu, care locuie?te în Sion, nu te teme de Asiria, care te love?te cu toiagul pe care îl ridica asupra ta, ca altadata Egiptul.
Isa 10:25  Dar, peste pu?ina vreme, urgia va înceta ?i mânia Mea îi va nimici".
Isa 10:26  Domnul Savaot ridica-va asupra lor un bici, ca atunci când a batut pe Madian la stânca Oreb ?i Î?i va întinde toiagul Sau spre mare ?i-l va ridica precum l-a ridicat asupra Egiptenilor.
Isa 10:27  În vremea aceea va ridica povara de pe umerii tai ?i jugul de pe grumajii tai.
Isa 10:28  Vine din latura Rimonului ?i ajunge la Aiat, trece la Migron, la Micmas lasa poverile sale de drum.
Isa 10:29  Ei au trecut pasul ?i noaptea au ramas la Gheba. Rama este înspaimântata, Ghibeea lui Saul a luat-o la fuga.
Isa 10:30  Urla fiica a lui Galim, ia aminte Lai?a, raspunde-i tu, Anatot.
Isa 10:31  Madmena se împra?tie, locuitorii din Ghebim au luat-o la fuga.
Isa 10:32  O zi va sta la Nob, amenin?a cu mâna muntele Sionului ?i colina Ierusalimului!
Isa 10:33  Iata ca Domnul Dumnezeu Savaot frânge crengile dintr-o lovitura naprasnica: vârfurile sunt taiate ?i crengile de sus date jos.
Isa 10:34  Desi?ul padurii cade sub lovituri de unelte de fier, cedrii Libanului se prabu?esc la pamânt.
Isa 11:1  O Mladi?a va ie?i din tulpina lui Iesei ?i un Lastar din radacinile lui va da.
Isa 11:2  ?i Se va odihni peste El Duhul lui Dumnezeu, duhul în?elepciunii ?i al în?elegerii, duhul sfatului ?i al tariei, duhul cuno?tin?ei ?i al bunei-credin?e.
Isa 11:3  ?i-L va umple pe El duhul temerii de Dumnezeu. ?i va judeca nu dupa înfa?i?area cea din afara ?i nici nu va da hotarârea Sa dupa cele ce se zvonesc,
Isa 11:4  Ci va judeca pe cei saraci întru dreptate ?i dupa lege va mustra pe sarmanii din ?ara. Pe cel aprig îl va bate cu toiagul gurii Lui ?i cu suflarea buzelor Lui va omorî pe cel fara de lege.
Isa 11:5  Dreptatea va fi ca o cingatoare pentru rarunchii Lui ?i credincio?ia ca un brâu pentru coapsele Lui.
Isa 11:6  Atunci lupul va locui laolalta cu mielul ?i leopardul se va culca lânga caprioara; ?i vi?elul ?i puiul de leu vor mânca împreuna ?i un copil îi va pa?te.
Isa 11:7  Juninca se va duce la pascut împreuna cu ursoaica ?i puii lor vor sala?lui la un loc, iar leul ca ?i boul va mânca paie;
Isa 11:8  Pruncul de ?â?a se va juca lânga culcu?ul viperei ?i în vizuina ?arpelui otravitor copilul abia în?arcat î?i va întinde mâna.
Isa 11:9  Nu va fi nici o nenorocire ?i nici un prapad în tot muntele Meu cel sfânt! Ca tot pamântul este plin de cuno?tin?a ?i de temerea de Dumnezeu, precum marea este umpluta de ape!
Isa 11:10  ?i în vremea aceea, Mladi?a cea din radacina lui Iesei, va fi ca un steag pentru popoare; pe Ea o vor cauta neamurile ?i sala?ul Ei va fi plin de slava.
Isa 11:11  în ziua aceea, Domnul va ridica din nou mâna Sa ca sa rascumpere rama?i?a poporului Sau dintre robii din Asiria ?i din Egipt, din Patros, din Etiopia, din Elam, din Babilon, din Hamat ?i din insulele marii.
Isa 11:12  El va ridica steag pentru neamuri ?i va aduna pe cei risipi?i ai lui Israel ?i va strânge la un loc pe cei împra?tia?i ai lui Iuda din cele patru col?uri ale pamântului.
Isa 11:13  Atunci pizma lui Efraim va înceta ?i du?manii lui Iuda vor fi zdrobi?i. Efraim nu va mai pizmui pe Iuda ?i Iuda nu va mai fi vrajma?ul lui Efraim.
Isa 11:14  Ci se vor avânta în latura Filistenilor la apus ?i vor jefui împreuna pe feciorii rasaritului; asupra Edomului ?i Moabului î?i vor întinde mâna lor, ?i copiii lui Amon vor asculta de ei.
Isa 11:15  Domnul va seca limba de mare a Egiptului ?i mâna Lui va amenin?a groaznic Eufratul, ?i-l va împar?i în ?apte râuri ?i se va putea trece cu piciorul.
Isa 11:16  Atunci se va croi un drum pentru rama?i?a din poporul Sau, pentru cei scapa?i din robia Asiriei, precum s-a întâmplat altadata cu Israel, în ziua când el a ie?it din Egipt.
Isa 12:1  ?i tu vei zice în ziua aceea: Lauda-Te-voi, Doamne, ca de?i pornit împotriva mea, mânia Ta s-a întors de la mine ?i m-ai miluit.
Isa 12:2  Iata Dumnezeul cel tare al mântuirii mele; nadajdui-voi întru El ?i nu ma voi înfrico?a, ca izvorul puterii mele ?i cântarea mea de lauda este Domnul Dumnezeu ?i izbavirea mea.
Isa 12:3  Ve?i scoate apa cu veselie din izvoarele mântuirii
Isa 12:4  ?i ve?i zice în ziua aceea: "Lauda?i pe Domnul, chema?i numele Lui, vesti?i printre neamuri lucrarile Lui, da?i de ?tire ca înalt este numele Lui!
Isa 12:5  Cânta?i în strune pe Domnul, caci El a facut fapte stralucite! Sa ?tie aceasta tot pamântul!
Isa 12:6  Salta?i ?i va veseli?i locuitori ai Sionului, caci mare este în mijlocul vostru Sfântul lui Israel!"
Isa 13:1  Proorocia despre Babilon pe care a vazut-o Isaia, fiul lui Amos.
Isa 13:2  Pe un munte ple?uv înal?a?i steag, striga?i catre ei, face?i semn cu mâna, ca sa intre pe poarta asupritorilor.
Isa 13:3  "Eu am poruncit sfintei Mele o?tiri, zice Domnul, chemat-am pe vitejii mâniei Mele, pe cei ce se veselesc de slava Mea".
Isa 13:4  Asculta?i acest zgomot surd în mun?i, vuiet al unui neam fara de numar; auzi?i aceasta zarva de împara?ii, de neamuri adunate; Domnul Savaot cerceteaza o?tirea gata de lupta.
Isa 13:5  Ele vin dintr-un ?inut departat, de la capatul cerului; vine Domnul ?i uneltele mâniei Lui, ca sa nimiceasca tot pamântul.
Isa 13:6  Striga?i, ca aproape este ziua Domnului, ea vine ca o pustiire de la Cel Atotputernic.
Isa 13:7  Drept aceea, toate bra?ele vor fi neputincioase ?i inima omului se va topi de frica.
Isa 13:8  Vor fi cuprin?i de spaima, vor vedea naluci ?i durerile îi vor cuprinde; zvârcolise-vor în dureri ca femeia gata sa nasca. Se vor privi unul pe altul cu groaza, iar fe?ele lor vor fi ro?ii ca flacara.
Isa 13:9  Iata ziua Domnului, ea vine apriga, mânioasa ?i întarâtata la mânie ca sa pustiiasca pamântul ?i sa stârpeasca pe pacato?i de pe el.
Isa 13:10  Luceferii de pe cer ?i gramezile de stele nu-?i vor mai da lumina lor; soarele se va întuneca în rasaritul lui ?i luna nu va mai straluci.
Isa 13:11  Atunci voi pedepsi lumea pentru faradelegile ei ?i pe cei nelegiui?i pentru pacatele lor. Voi smeri mândria celor îngâmfa?i ?i obraznicia celor cruzi o voi arunca la pamânt.
Isa 13:12  Voi face ca oamenii sa fie mai rari decât aurul cel mai scump, mai cauta?i decât aurul de Ofir.
Isa 13:13  Pentru aceasta voi prabu?i cerurile; ?i pamântul se va clatina  din locul lui, din pricina mâniei Domnului Savaot, în ziua iu?imii mâniei Lui.
Isa 13:14  Atunci, ca o gazela sperioasa ?i ca o turma pe care nimeni nu poate s-o adune, fiecare se va întoarce la poporul sau ?i fiecare va fugi în pamântul sau.
Isa 13:15  Oricine va fi aflat va fi strapuns ?i oricare va fi prins va cadea de sabie.
Isa 13:16  Copiii lor vor fi zdrobi?i înaintea ochilor lor, casele lor vor fi jefuite ?i femeile lor necinstite.
Isa 13:17  Iata ca Eu ridic asupra lor pe Mezi, care nu pun pre? pe argint ?i care nu se lacomesc pentru aur.
Isa 13:18  Arcurile oamenilor de lupta vor doborî pe cei tineri. De roada pântecelui nu se vor milostivi, ?i pentru copii ochii lor nu vor sim?i nici o mila.
Isa 13:19  Atunci Babilonul, podoaba împara?iilor, cununa mândriei Caldeilor, fi-va ca Sodoma ?i ca Gomora, pe care Dumnezeu le-a nimicit.
Isa 13:20  Nu va mai fi locuit în veci ?i din neam în neam. Arabii nu vor mai înfige acolo corturi ?i nici ciobanii nu-?i vor mai face târle în latura aceea.
Isa 13:21  Ci numai animale salbatice se vor sala?lui într-însul, ?i bufni?ele vor locui prin case, stru?ii î?i vor face cuiburi acolo ?i oameni cu chip de ?ap vor juca în acel loc.
Isa 13:22  ?acalii vor urla în palatele lor ?i lupii în casele lor de petrecere. Vremea este aproape sa soseasca ?i zilele ei nu vor zabovi!
Isa 14:1  Dar Domnul Se va milostivi de Iacov ?i va alege înca o data pe Israel ?i îl va statornici în pamântul lui. Cei straini se vor alatura lor ?i se vor uni cu casa lui Iacov.
Isa 14:2  Pe popoare le va lua ?i le va duce la ei, iar casa lui Israel le va avea în pamântul Domnului ca robi ?i roabe. Ei vor duce în robie pe cei care i-au dus în robie ?i vor stapâni peste apasatorii lor.
Isa 14:3  Iar în ziua în care Domnul te va odihni de osteneli, de chinuri ?i de amarnica ta robie în care ai fost,
Isa 14:4  Tu vei cânta cântecul acesta de ocara împotriva împaratului Babilonului ?i vei zice: "Cum s-a sfâr?it cu asupritorul ?i cum a încetat chinul nostru!
Isa 14:5  Domnul a zdrobit toiagul celor fara de lege, sceptrul railor apasatori!
Isa 14:6  Iata pe cel care lovea popoarele fara încetare cu mânia lui ?i care în întarâtarea lui punea neamurile sub stapânirea lui, supunându-le fara cru?are!
Isa 14:7  Tot pamântul este în pace ?i se odihne?te; to?i izbucnesc în cântece de veselie.
Isa 14:8  Pâna ?i chiparo?ii împreuna cu cedrii cei din Liban se bucura de caderea ta: "De când tu te-ai prabu?it, nimeni nu se mai suie la noi ca sa ne doboare!"
Isa 14:9  ?eolul (iadul) se mi?ca în adâncurile sale, ca sa iasa întru întâmpinarea ta. Pentru tine el de?teapta umbrele, pe to?i stapânitorii pamântului; el ridica de pe jil?urile lor pe to?i împara?ii pamântului.
Isa 14:10  To?i iau cuvântul ?i î?i zic: "?i tu e?ti slab ca noi ?i te asemeni noua".
Isa 14:11  În iad s-a pogorât marirea ta în cântecul harfelor tale. Sub tine se vor a?terne viermii ?i viermii vor fi acoperamântul tau.
Isa 14:12  Cum ai cazut tu din ceruri, stea stralucitoare, fecior al dimine?ii! Cum ai fost aruncat la pamânt, tu, biruitor de neamuri!
Isa 14:13  Tu care ziceai în cugetul tau: "Ridica-ma-voi în ceruri ?i mai presus de stelele Dumnezeului celui puternic voi a?eza jil?ul meu! În muntele cel sfânt voi pune sala?ul meu, în fundurile laturei celei de miazanoapte.
Isa 14:14  Sui-ma-voi deasupra norilor ?i asemenea cu Cel Preaînalt voi fi".
Isa 14:15  ?i acum, tu te pogori în iad, în cele mai de jos ale adâncului!
Isa 14:16  Cei ce te vad î?i întorc privirea în spre tine ?i se uita cu luare aminte zicând: "Oare acesta este omul de care tremura pamântul ?i împara?iile se cutremurau?"
Isa 14:17  Oare acesta este cel ce prefacea lumea în pustiu ?i ceta?ile le dobora ?i nu da drumul robilor sai?"
Isa 14:18  To?i împara?ii popoarelor se odihnesc cu cinste, fiecare în loca?ul sau.
Isa 14:19  ?i numai tu e?ti azvârlit departe de mormântul tau, ca o ramura fara de pre?, ca rama?i?ele celor care au fost uci?i cu lovituri de sabie, zvârli?i pe pietre de mormânt, ca un hoit calcat în picioare.
Isa 14:20  Tu nu te vei pogorî în mormânt, caci tu ai pustiit pamântul tau ?i pe poporul tau l-ai ucis! Niciodata nu se va mai vorbi despre neamul celor rai!
Isa 14:21  Pregati?i macelul feciorilor, din pricina faradelegilor parin?ilor lor, ca nu cumva sa se ridice ?i sa cucereasca pamântul ?i sa umple de ruine fa?a a tot pamântul.
Isa 14:22  "Eu Ma voi scula împotriva lor, zice Domnul Savaot, ?i voi nimici numele Babilonului ?i pe cei care au mai ramas: ?i mugurii ?i mladi?ele, zice Domnul.
Isa 14:23  Acolo va stapâni ariciul ?i va fi mla?tina ?i îl voi matura cu matura nimicirii", zice Domnul Savaot.
Isa 14:24  Juratu-S-a Domnul Savaot ?i a zis: "Cum am hotarât, a?a va fi, precum M-am sfatuit, a?a se va întâmpla!
Isa 14:25  Sfarâma-voi Asiria în pamântul Meu ?i o voi calca în picioare pe mun?ii Mei. ?i robii vor fi libera?i de jugul lor ?i umerii de povara lor".
Isa 14:26  Iata hotarârea pentru tot pamântul, iata mâna întinsa peste toate neamurile!
Isa 14:27  Daca Domnul Savaot a hotarât, cine îl va putea împiedica? ?i daca mâna Lui sta întinsa, cine o va întoarce la loc?
Isa 14:28  În anul mor?ii lui Ahaz, fost-a aceasta proorocie:
Isa 14:29  "Nu te veseli, toata ?ara Filistenilor, fiindca a fost zdrobit toiagul care te lovea. Caci din radacina ?arpelui va ie?i o vipera ?i din urma?ii lui un ?arpe zburator.
Isa 14:30  Cei sarmani vor pa?te pe pa?unile Mele, iar cei saraci vor fi fara de grija. Voi face sa moara de foame neamul tau, iar pe cei ce vor ramâne din tine îi voi ucide.
Isa 14:31  Tu, poarta, urla! ?i tu, cetate, ?ipa! Cutremura-te tu, ?ara a Filistenilor, toata! Ca din partea de miazanoapte vine un fum ?i ?irurile vrajma?ilor sunt strânse".
Isa 14:32  ?i ce se va raspunde în ziua aceea celor trimi?i dintre popoare? Ca "Domnul a întemeiat Sionul, limanul celor îndurera?i din poporul Lui".
Isa 15:1  Prins fara de veste în vreme de noapte Ar-Moabul a fost pustiit. Luat fara de veste noaptea, Chir-Moabul a fost nimicit.
Isa 15:2  Poporul se urca la templul de la Dibon, la locurile înalte, ca sa plânga pe Nebo ?i la Medeba, Moabul se tânguie?te. Toate capetele sunt rase, toate barbile sunt taiate.
Isa 15:3  Pe uli?ele lui to?i ies îmbraca?i în sac; pe acoperi?uri, în pie?e, to?i se jelesc ?i izbucnesc în plâns.
Isa 15:4  He?bonul ?i Eleale bocesc, iar glasul lor pâna la Iaha? se aude. Chiar ?i razboinicii Moabului se vaita ?i sufletul le este cuprins de groaza.
Isa 15:5  Din adâncul inimii, Moabul striga; fugarii lui sosesc pâna la ?oar, pâna la Eglat-?eli?ia. Coasta Luhitului to?i o urca plângând; pe drumul de la Horonaim scot strigate de deznadejde,
Isa 15:6  Ca apele de la Nimrim au secat, iarba s-a uscat, iarba verde nu mai este, verdea?a a pierit.
Isa 15:7  De aceea ei î?i fac provizii ?i duc bunurile lor dincolo de pârâul Salciilor.
Isa 15:8  ?ipetele au facut înconjurul Moabului, vaietele sale au ajuns pâna la Eglaim, jeluirea lui pâna la Beer-Elim, ca apele Dimonului sunt pline de sânge!
Isa 15:9  Asupra Dimonului voi trimite iara?i nenorociri; pentru cei ce au scapat din Moab cât ?i pentru cei rama?i în ?ara voi trimite lei.
Isa 16:1  "Trimite?i miei stapânitorului ?arii, trimite?i-i din Petra, prin pustiu, la muntele fiicei Sionului".
Isa 16:2  Ca o pasare care fuge sperioasa din cuibul ei, ca un cuib risipit, a?a sunt fiicele Moabului la vadurile Arnonului.
Isa 16:3  "Da un sfat, da o hotarâre, întinde umbra ta, ca noaptea, în miezul zilei, ascunde pe cei du?i în robie, nu descoperi pe cei fugari!
Isa 16:4  Adaposte?te la tine pe to?i robii Moabului, sa le fii acoperitor în fa?a pustiitorului, pâna când navala va fi trecut, prapadul va lua sfâr?it ?i vrajma?ul va lasa ?ara în pace.
Isa 16:5  Jil?ul lui se va întari prin milostivire ?i pe el va ?edea de-a pururi în cortul lui David un judecator aparator al pricinei drepte ?i râvnitor drepta?ii.
Isa 16:6  Am auzit de seme?ia Moabului, ca foarte mândru este; am auzit de obraznicia, de mândria, de trufia ?i de graiurile lui de?arte".
Isa 16:7  Pentru aceasta, Moabi?ii se tânguiesc pentru Moab, ton împreuna se bocesc! Ei suspina pentru turtele de struguri de la Chir-Hareset, înmarmuri?i.
Isa 16:8  Câmpiile He?bonului au saracit, asemenea ?i via de la Sibma; stapânitorul popoarelor a distrus cele mai bune vi?e ale ei, care se întinsesera pâna la Iazer ?i acoperisera pustiul; lastarii lor se întinsesera ?i trecusera marea.
Isa 16:9  Pentru aceasta plâng împreuna cu Iazerul pentru via din Sibma. Va ud cu lacrimile mele pe voi, He?bon ?i Eleale, ca nu se mai aud acolo strigatele vesele din timpul seceri?ului ?i al culesului viilor.
Isa 16:10  Nici bucurie, nici veselie prin gradini, iar prin vii nici cântece, nici chiote! Nimeni nu mai da vinul la teasc, strigatul calcatorului a încetat.
Isa 16:11  Pentru aceasta launtrul meu se zbuciuma pentru Moab ca o harfa ?i inima mea pentru Chir-Hares.
Isa 16:12  Iata ca Moabul este vazut urcând obosit pe locurile înalte, intra în templul sau sa se roage, dar nu dobânde?te nimic.
Isa 16:13  Aceasta este proorocia pe care a grait-o Domnul odinioara pentru Moab.
Isa 16:14  Iar acum Domnul a zis a?a: "Peste trei ani, socoti?i ca anii unui simbria?, marirea Moabului se va mic?ora cu vuiet mare ?i va ramâne mica ?i slaba, fara nici o putere".
Isa 17:1  Proorocie împotriva Damascului: "Damascul este scos din numarul ceta?ilor ?i a ramas o gramada de ruine.
Isa 17:2  Ceta?ile Aroerului sunt pustiite pentru vecie; ele sunt bune de pascut turmele, care se culca acolo ?i nimeni nu le gone?te.
Isa 17:3  Nici cetate întarita pentru Efraim ?i nici împara?ie la Damasc. Tot a?a va fi cu rama?i?a Siriei ?i cu marirea ei, precum a fost cu fiii lui Israel, zice Domnul Savaot.
Isa 17:4  ?i va fi în ziua aceea ca marirea lui Iacob se va împu?ina ?i acest trup gras se va usca.
Isa 17:5  Va fi atunci ca pe urma seceratorului ce secera holda, când mâna lui aduna spice ?i cum e când oamenii aduna spice în valea Refaim;
Isa 17:6  Vor ramâne pe urma câteva roade, ca la scuturatul maslinului, doua-trei masline pe vârf, patru-cinci pe ramuri", zice Domnul Dumnezeul lui Israel.
Isa 17:7  În ziua aceea, omul î?i va întoarce privirea catre Ziditorul sau ?i ochii lui catre Sfântul lui Israel se vor întoarce.
Isa 17:8  ?i nu va mai privi catre jertfelnice, lucrurile mâinilor lui, ?i nu se va mai uita la faptura degetelor lui, la Astartele ?i statuile ridicate soarelui.
Isa 17:9  În vremea aceea, ceta?ile sale întarite vor fi parasite ca ale Amoreilor ?i Heveilor, lasate înaintea fiilor lui Israel ?i vor ramâne pustii.
Isa 17:10  Caci tu ai uitat pe Dumnezeul izbavirii tale ?i de Stânca scaparii tale nu ?i-ai adus aminte. Iata pentru ce tu întemeiezi gradini lui Adonis ?i acolo sade?ti vie pentru un dumnezeu strain.
Isa 17:11  În ziua când o sade?ti, tu vezi ca se ridica ?i a doua zi are flori; dar de culesul roadelor nu te bucuri în ziua nenorocirii ?i durerea este fara leac.
Isa 17:12  Ah! Aceasta zarva de popoare este ca vuietul de ape multe, acest zgomot de neamuri este ca zgomotul de ape mari;
Isa 17:13  El le amenin?a ?i ele fug departe, gonite ca pleava pe care vânturatorii o vântura în vânt ?i ca vârtejul de pulbere în vreme de furtuna.
Isa 17:14  În vremea serii, atunci e ceasul spaimei, iar mai înainte de a se face ziua, ei nu mai sunt. Iata partea, partea jefuitorilor no?tri ?i soarta celor ce ne-au pradat pe noi.
Isa 18:1  Vai ?ie, ?ara în care se aude zanganit de arme ?i care e?ti dincolo de fluviile Etiopiei!
Isa 18:2  Tu, care trimi?i soli pe Nil în barci de papura pe întinsele ape. Merge?i voi, soli iu?i, catre un neam de statura înalta ?i cu pielea lucie, departe catre un popor de temut, popor plin de putere ?i viteaz, a carui ?ara este strabatuta de fluvii.
Isa 18:3  Voi, to?i locuitori ai lumii ?i care stapâni?i pamântul! Când ve?i vedea înal?ându-se steagul deasupra mun?ilor, privi?i! ?i când va suna trâmbi?a, asculta?i!
Isa 18:4  Ca a?a zice Domnul catre mine: "Privesc lini?tit din loca?ul Meu, întocmai ca adierea fierbinte a verii la lumina soarelui, ca norul de roua în zaduful seceri?ului.
Isa 18:5  Caci înainte de cules, dupa ce florile s-au scuturat ?i mugurii s-au prefacut în ciorchini cop?i, vi?ele vor fi taiate cu cosoarele, ramurile vor fi luate, smulse vor fi.
Isa 18:6  Toate vor fi lasate vulturilor de munte ?i fiarelor pamântului; vulturii vor petrece acolo vara, iar fiarele câmpului iarna.
Isa 18:7  În vremea aceea, se vor aduce daruri de la neamul de statura înalta ?i cu pielea lucie, de la poporul de temut cel de departe, de la poporul cel plin de putere ?i viteaz, a carui ?ara este strabatuta de fluvii, catre locul numelui Domnului Savaot, muntele Sionului".
Isa 19:1  Iata Domnul vine pe nor u?or ?i ajunge în Egipt. Idolii Egiptului tremura înaintea fe?ei Lui ?i inima Egiptenilor se tope?te în ei.
Isa 19:2  Voi întarâta pe Egipteni unii împotriva altora ?i se vor razboi frate cu frate ?i prieten cu prieten, cetate cu cetate, împara?ie cu împara?ie.
Isa 19:3  Egiptul î?i va pierde mintea ?i voi încurca iste?imea lui ?i vor merge ei sa întrebe pe idoli ?i pe vrajitori, pe fermecatori ?i pe ghicitori.
Isa 19:4  ?i voi da Egiptul în mâna unui stapânitor crud ?i un împarat puternic îl va stapâni, zice Domnul Dumnezeu Savaot.
Isa 19:5  Apele marii se vor sfâr?i ?i fluviul va seca ?i se va usca de tot.
Isa 19:6  Canalele se vor preface în ape statatoare. Râurile Egiptului vor scadea ?i se vor usca, papura ?i trestia se vor ve?teji.
Isa 19:7  Lunca Nilului ?i toata verdea?a de pe malurile lui se vor usca, vor cadea ?i nu vor mai fi!
Isa 19:8  Pescarii vor suspina ?i se vor tângui; to?i cei care arunca undi?a în Nil, cei care arunca navodul pe fa?a apelor, vor fi deznadajdui?i.
Isa 19:9  Cei care lucreaza inul vor fi nedumeri?i ?i pieptanatoarele ?i ?esatorii vor fi în mare încurcatura.
Isa 19:10  ?esatorii vor fi tulbura?i ?i to?i lucratorii, în întristare mare.
Isa 19:11  Mai-marii ?oanului au ajuns nebuni, sfatuitorii cei în?elep?i ai lui Faraon dau sfaturi fara de minte! Cum îndrazni?i voi sa zice?i lui Faraon: "Eu sunt ucenicul celor în?elep?i, al regilor de altadata?"
Isa 19:12  Unde sunt oare în?elep?ii tai? Sa te vesteasca ?i sa-?i dea de ?tire ceea ce a hotarât Domnul Savaot împotriva Egiptului.
Isa 19:13  Mai-marii ?oanului au ajuns nebuni, mai-marii Nofului ?i-au pierdut mintea ?i capeteniile semin?iilor duc Egiptul pe cai gre?ite.
Isa 19:14  Domnul a aruncat peste ei un duh de zapaceala; în orice fapta a lor ei ratacesc Egiptul ?i nu-?i dau seama, cum nu-?i da seama be?ivul când varsa.
Isa 19:15  ?i nu va fi nici un lucru în Egipt cu rostul lui: nici cap, nici coada, nici început, nici sfâr?it.
Isa 19:16  În ziua aceea, Egiptenii vor fi ca femeile fricoase ?i tremuratoare, din pricina amenin?arii mâinii Domnului Savaot pe care o va ridica peste ei.
Isa 19:17  Atunci pamântul lui Iuda va fi pentru Egipt înfrico?are mare; ori de câte ori i se va aminti numele; Egiptul va tremura, din pricina hotarârii luate împotriva lui de Domnul Savaot.
Isa 19:18  În vremea aceea, vor fi numai cinci ceta?i în pamântul Egiptului care vor grai limba Canaanului ?i vor jura în numele Domnului Savaot; una se va numi "Cetatea Soarelui".
Isa 19:19  În ziua aceea, va fi un jertfelnic în mijlocul pamântului Egiptului ?i un stâlp de pomenire la hotarul lui, pentru Domnul.
Isa 19:20  Acesta va fi un semn ?i o marturie pentru Domnul Savaot în pamântul Egiptului. Când vor striga catre Domnul în strâmtorarile lor, atunci El le va trimite un mântuitor ?i un razbunator oare-i va mântui.
Isa 19:21  Domnul se va face ?tiut în Egipt ?i Egiptenii vor cunoa?te pe Domnul în ziua aceea. ?i vor aduce arderi de tot ?i prinoase ?i vor face fagaduin?e Domnului ?i le vor împlini.
Isa 19:22  ?i Domnul va bate Egiptul, îl va lovi ?i apoi îl va vindeca. ?i ei se vor întoarce la Domnul ?i El se va îndupleca ?i îi va tamadui.
Isa 19:23  În vremea aceea, va fi un drum din Egipt în Asiria ?i Asiria va merge în Egipt ?i Egiptul în Asiria ?i Egiptenii ?i Asirienii vor sluji pe Domnul.
Isa 19:24  În ziua aceea, Israel va fi al treilea în legamântul cu Egiptul ?i cu Asiria, ca o binecuvântare în mijlocul pamântului,
Isa 19:25  Binecuvântare a Domnului Savaot, Care zice: "Binecuvântat sa fie poporul Meu, Egiptul ?i Asiria, lucrul mâinilor Mele ?i Israel, mo?tenirea Mea!"
Isa 20:1  În anul în care Tartan a venit la A?dod, trimis de Sargon, regele Asiriei, ?i a împresurat A?dodul ?i l-a cuprins,
Isa 20:2  În vremea aceea a grait Domnul prin gura lui Isaia, fiul lui Amos, zicând: "Du-te ?i dezbraca sacul de pe coapsele tale ?i descal?a încal?amintele tale". ?i a facut a?a ?i mergea gol ?i descul?.
Isa 20:3  ?i a zis Domnul: "Precum a umblat robul Meu Isaia gol ?i descul? vreme de trei ani, ca semn ?i prevestire pentru Egipt ?i pentru Etiopia,
Isa 20:4  Astfel va aduce regele Asiriei robi din Egipt ?i surghiuni?i din Etiopia, tineri ?i batrâni, goi ?i descul?i ?i cu spatele descoperit, spre ru?inea Egiptului.
Isa 20:5  ?i cei care se bizuiau pe Etiopia ?i erau mândri cu Egiptul vor fi cuprin?i de teama ?i de ru?ine.
Isa 20:6  Locuitorii acestui ?inut vor zice în ziua aceea: "Iata pe cine ne bizuim, catre care vrem sa fugim sa cautam ajutor ?i scapare dinaintea regelui Asiriei! ?i acum cum vom scapa?"
Isa 21:1  Ca furtuna care vine de la miazanoapte, aceasta vine din pustiu, dintr-un ?inut înfrico?ator.
Isa 21:2  O vedenie grozava mi s-a descoperit: jefuitorul jefuie?te ?i pustiitorul pustie?te. Avânta-te, Elame! împresura?i pe Mezi, n-ave?i nici o mila!
Isa 21:3  De aceea inima mea s-a umplut de nelini?te, dureri m-apuca, ca durerile unei femei care este gata sa nasca. înspaimântat cum sunt, nu mai aud; tulburat, nici ca mai vad;
Isa 21:4  Duhul meu ratace?te, frica da navala pe?te mine. Noaptea care atât îmi placea ma umple de groaza!
Isa 21:5  Masa este pusa, a?ternuturile întinse, to?i manânca ?i beau. Voi, capetenii, scula?i-va, prinde?i scutul!
Isa 21:6  Ca a?a zice Domnul catre mine: "Du-te ?i pune pe cineva de straja, care sa-Mi dea de ?tire despre ceea ce va vedea!
Isa 21:7  Daca va vedea calare?i, doi câte doi pe cai, calare?i pe asini, pe camile, sa se uite cu bagare de seama, cu mare bagare de seama".
Isa 21:8  ?i el a strigat ca un leu: "Stau de straja, Doamne, neîncetat toata ziua ?i la locul meu de veghe în fiecare noapte.
Isa 21:9  ?i iata ca sose?te calarime, calare?i doi câte doi". ?i el a vorbit ?i a zis: "A cazut, a cazut Babilonul ?i toate chipurile cioplite ale idolilor lui stau sfarâmate la pamânt!"
Isa 21:10  O, poporul meu, fecior al ariei mele, batut cum se bate grâul, ceea ce am auzit de la Domnul Savaot, Dumnezeul lui Israel, ti le dau de ?tire!
Isa 21:11  Proorocie despre Edom. Cineva striga din Seir catre mine: "Strajerule, cât a trecut din noapte? Strajerule, cât mai este pâna trece noaptea?
Isa 21:12  ?i strajerul raspunde: "Diminea?a se apropie, dar este înca noapte. De voi?i, întreba?i, întoarce?i-va ?i veni?i iara?i".
Isa 21:13  Proorocie despre Arabia. Într-o padure de stepa petrece?i noaptea, voi, caravane din Dedan!
Isa 21:14  Aduce?i apa celor înseta?i, voi, locuitori ai ?inutului Tema, întâmpina?i cu pâine pe cei fugari,
Isa 21:15  Ca ei au fugit dinaintea sabiei, din fa?a sabiei scoase din teaca, de arcul întins ?i de grozaviile razboiului!
Isa 21:16  Ca iata ce mi-a spus Domnul: "Înca un an, ca anii unui simbria?, ?i toata stralucirea lui Chedar se duce.
Isa 21:17  Vitejii arca?i ai fiilor lui Chedar se vor împu?ina; ca Domnul Dumnezeul lui Israel a grait".
Isa 22:1  Proorocia despre valea vedeniei. Ce ai tu ca tot poporul tau s-a urcat pe acoperi?uri,
Isa 22:2  Tu, cetate zgomotoasa, cetate plina de zarva ?i de chiote de veselie? Rani?ii tai nu sunt rani?i de sabie ?i n-au murit în lupta.
Isa 22:3  Mai-marii tai au fugit laolalta, au fost lua?i robi nu cu puterea arcului; to?i vitejii tai de lupta prin?i au fost cu to?ii, când ei fugeau departe.
Isa 22:4  De aceea va zic: "Departa?i-va de mine ?i lasa?i-ma sa plâng amar, nu va îmbulzi?i sa ma mângâia?i pentru nenorocirea fiicei poporului meu.
Isa 22:5  Ca este o zi de tulburare, de zdrobire, de uluire de la Domnul Dumnezeu Savaot în valea vedeniei, prabu?ire de zid ?i ?ipetele celor ce fug înspre mun?i!
Isa 22:6  Elamul a luat tolba de sage?i, Aramul a încalecat pe cal ?i Chirul a scos pavaza!
Isa 22:7  Vaile tale mare?e sunt pline de care ?i calare?i, tabarâ?i la por?ile tale;
Isa 22:8  Valul va fi ridicat de pe Iuda! ?i voi ve?i privi în ziua aceea gramezile de arme din casa cea din padure.
Isa 22:9  Sparturile zidurilor ceta?ii lui David sunt fara numar, voi le vede?i. Aduna?i apele din iazul cel mai de jos,
Isa 22:10  Numara?i casele cele din Ierusalim, darâma?i-le ca sa întari?i zidul.
Isa 22:11  Un iaz mai mare face?i între cele doua ziduri, ca sa strânge?i apa din iazul cel mai de demult. Dar voi nu lua?i aminte la Cel care a facut toate acestea, la Cel care le-a pregatit de demult. Voi nu-L vede?i!
Isa 22:12  ?i în ziua aceea ne va îndemna Domnul Dumnezeu Savaot sa plângem, sa suspinam, sa ne radem capul ?i sa ne încingem cu sac.
Isa 22:13  Iata bucuria ?i veselia, boi taia?i ?i oi junghiate; to?i manânca din carne ?i beau vin: "Sa mâncam ?i sa bem, ca mâine vom muri!"
Isa 22:14  Domnul Savaot a descoperit urechilor mele: Acest pacat nu va va fi iertat nici pâna la moarte, zice Domnul Dumnezeu Savaot.
Isa 22:15  Împotriva lui ?ebna, mai-marele palatului, iata ce spune Domnul Dumnezeu Savaot: "Du-te la acest dregator,
Isa 22:16  Care î?i sapa mormânt pe un loc înalt, care î?i pregate?te loca? în stânca ?i zi-i: "Ce ai tu ?i cine e?ti tu de-?i sapi aici mormânt?
Isa 22:17  Iata ca Domnul te azvârle, dintr-o singura aruncatura, te strânge cu o singura strângere.
Isa 22:18  El te înfa?ura ?i te rostogole?te ca pe un ghem pe un câmp întins. Acolo tu vei muri; acolo vor merge carele tale mare?e, tu, ru?inea palatului stapânului tau.
Isa 22:19  El î?i va lua slujba ta ?i te va lipsi de dregatoria ta.
Isa 22:20  ?i în ziua aceea voi chema pe sluga mea, pe Eliachim, feciorul lui Hilchia,
Isa 22:21  ?i îl voi îmbraca cu ve?mintele tale, îl voi încinge cu brâul tau ?i-i voi da în mâna dregatoria ta. El va fi tata pentru cei ce locuiesc în Ierusalim ?i pentru casa lui Iuda.
Isa 22:22  ?i îi voi pune pe umeri cheile casei lui David ?i daca el va deschide, nimeni nu va închide, ?i daca el va închide, nimeni nu va deschide.
Isa 22:23  ?i îl voi înfige ca pe un cui într-un loc de nadejde ?i va fi scaun de cinste pentru casa tatalui sau.
Isa 22:24  Pe el se va rezema toata slava casei tatalui sau, fii ?i nepo?i; toate vasele cele mai mici de la cani ?i pâna la marile lighene.
Isa 22:25  În ziua aceea, zice Domnul Savaot, cuiul înfipt într-un loc tare se va slabi; se va smulge ?i va cadea ?i povara atârnata de el va fi nimicita, ca a?a a grait Domnul!"
Isa 23:1  Tângui?i-va voi, corabii ale Tarsisului, caci limanul vostru a fost nimicit. Venitu-le-a aceasta ?tire din ?ara Chitim.
Isa 23:2  Amu?i?i voi, locuitori ai coastei pe care o umpleau negu?atorii din Sidon care strabateau marea!
Isa 23:3  Veniturile lui erau grâul Nilului, seceri?ul din valea lui, adus pe ape mari; el era târgul neamurilor.
Isa 23:4  Ru?ineaza-te, Sidonule, ca marea î?i zice: "Tu n-ai avut dureri de mama, tu n-ai nascut ?i nici n-ai crescut baie?i ?i nici n-ai ridicat fete".
Isa 23:5  Când Egiptul va prinde de veste, va tremura la auzul nenorocirilor Tirului.
Isa 23:6  Trece?i în Tarsis, boci?i-va, voi, locuitori de pe ?armuri!
Isa 23:7  Aceasta este, oare, cetatea voastra de petrecere, a carei obâr?ie se urca în vremuri vechi ?i care î?i calauzea pa?ii spre sala?uri departate?
Isa 23:8  Cine a poruncit acest lucru împotriva Tirului cel încercat, ai carui negu?atori erau prin?i ?i ai carui vânzatori erau cei mari ai pamântului?
Isa 23:9  Domnul Savaot a hotarât aceasta, ca sa ve?tejeasca mândria a tot ce straluce?te, sa smereasca pe to?i cei mari ai pamântului.
Isa 23:10  Treci ?i du-te în pamântul tau, tu fiica a Tarsisului, caci portul tau nu mai este.
Isa 23:11  El a întins mâna spre mare, a doborât regatele. Domnul a hotarât împotriva lui Canaan ruina întarituri lor lui.
Isa 23:12  El a zis: "Nu tresalta de bucurie, tu, fecioara necinstita a Sidonului! Scoala-te ?i du-te la Chitim, dar nici acolo nu vei avea odihna!"
Isa 23:13  Iata ?ara Caldeilor! Acest popor nu sunt Asirienii; El a dat-o prada fiarelor de câmp. Ei ?i-au înal?at turnuri, au darâmat palate, facut-au totul o ruina.
Isa 23:14  Boci?i-va voi, corabii ale Tarsisului, caci portul vostru a fost darâmat.
Isa 23:15  ?i va fi în ziua aceea ca Tirul va fi uitat ?aptezeci de ani, ca în zilele unui singur rege, ?i la sfâr?itul celor ?aptezeci de ani Tirul va fi a?a cum se afla în cântecul desfrânatei:
Isa 23:16  "Ia chitara, da ocol ceta?ii, tu, desfrânata! Cânta cât mai bine, reia cântarile ca lumea sa-?i aduca aminte de tine!"
Isa 23:17  ?i dupa  cei ?aptezeci de ani, Domnul va cerceta iara?i cetatea Tirului ?i ea va reîncepe sa primeasca pre?ul desfrâului ei. Ea se va desfrâna pentru toate regatele lumii de pe fa?a pamântului.
Isa 23:18  Dar tot câ?tigul, toate foloasele ei vor fi afierosite Domnului ?i nu vor fi adunate, nici puse la pastrare; ci câ?tigul va fi pentru cei ce locuiesc înaintea Domnului, ca sa aiba hrana din bel?ug ?i haine stralucite.
Isa 24:1  Iata Domnul pustie?te pamântul ?i îl preface în de?ert, rastoarna fa?a lui ?i împra?tie pe locuitori.
Isa 24:2  ?i preotului i se întâmpla ca ?i poporului, stapânului ca ?i robului, slugii ca ?i stapânei sale; vânzatorului ca ?i cumparatorului, celui care da cu împrumut ca ?i celui care se împrumuta, datornicului ca ?i cel caruia îi este dator.
Isa 24:3  Pamântul va fi pustiit, el va fi jefuit, ca Domnul a grait cuvântul acesta.
Isa 24:4  Pamântul este în chin ?i sleit, lumea tânje?te ?i se istove?te, cerul împreuna cu pamântul vor pieri.
Isa 24:5  Pamântul este pângarit sub locuitorii lui, caci ei au calcat legea, au înfrânt orânduiala ?i legamântul stricatu-l-au pe veci!
Isa 24:6  Pentru aceasta, blestemul mistuie pamântul ?i locuitorii îndura pedeapsa lor; drept aceea cei ce locuiesc pe pamânt sunt mistui?i, iar oamenii rama?i sunt pu?ini la numar!
Isa 24:7  Via tânje?te, vi?ele sale sunt firave, cei cu inima vesela suspina.
Isa 24:8  Glasul cel plin de veselie al lirei a încetat, chiotele zgomotoase nu mai sunt, încetat-a glasul harpei.
Isa 24:9  La cântec nu se mai bea, ?i amar este vinul pentru bautor.
Isa 24:10  Cetatea pustiita este în ruina, intrarea fiecarei case este închisa.
Isa 24:11  Pe uli?a lumea striga: "Nici un strop de vin!" Nu mai este bucurie, veselia este izgonita de pe pamânt.
Isa 24:12  În cetate au ramas numai darâmaturi, por?i sfarâmate ?i stricate.
Isa 24:13  A?a se va întâmpla în mijlocul acestui ?inut, înauntrul popoarelor, ca ?i când se scutura maslinii ?i ca pe urma culesului viei.
Isa 24:14  Aceia înal?a glasul ?i cânta, preaslavesc marirea Domnului la apus.
Isa 24:15  Pentru aceasta, în insule se preaslave?te Domnul, în insulele marii numele Domnului Dumnezeului lui Israel.
Isa 24:16  De la marginile pamântului auzim cântând: "Slava celui drept!" ?i eu am zis: "Vai de cei fara de lege, care lucreaza, departându-se de lege!"
Isa 24:17  Groaza, la? ?i groapa pentru voi, locuitori ai pamântului!
Isa 24:18  Cel care va fugi de groaza va cadea în groapa, cel care va scapa din mijlocul gropii se va prinde în la?! Zagazurile cele de sus se vor deschide ?i temeliile pamântului se vor clatina.
Isa 24:19  Pamântul se sfarâma, pamântul sare în buca?i, se clatina pamântul.
Isa 24:20  Pamântul se mi?ca încoace ?i încolo ca un om be?iv, se da în sus ?i în jos ca un scrânciob; pacatele apasa asupra lui, ca sa nu se mai scoale!
Isa 24:21  ?i în ziua aceea Domnul va cerceta cu asprime, acolo sus, o?tirea cea de sus ?i pe pamânt pe regii pamântului.
Isa 24:22  ?i ca robii vor fi închi?i într-o închisoare sub pamânt ?i dupa multe zile vor fi cerceta?i.
Isa 24:23  Luna va fi ro?ie, iar soarele va pierde din lumina lui, caci Domnul Savaot va fi rege ?i glava Lui va straluci înaintea batrânilor în muntele Sionului ?i în Ierusalim!
Isa 25:1  Doamne Dumnezeul meu, pe Tine Te voi înal?a, lauda-voi numele Tau, ca Tu ai facut lucruri minunate; planurile Tale de mult întocmite sunt adevarate ?i statornice.
Isa 25:2  Ca Tu ai facut din cetate o gramada de pietre ?i din cetatea cea întarita o darâmatura. Cetatea celor fara de lege nu mai este cetate, zidita nu va mai fi în veci.
Isa 25:3  Pentru aceasta un popor tare Te va preaslavi, cetatea puternicelor neamuri de Tine se va teme.
Isa 25:4  Fost-ai scapare pentru cel sarman, adapost pentru cel ce era în strâmtorare, liman în vremi vijelioase, umbra în vreme de caldura. Caci suflarea celor apasatori este ca furtuna de iarna
Isa 25:5  ?i ca ar?i?a soarelui într-un pamânt uscat. Ai potolit zarva celor nelegiui?i. Precum se potole?te caldura la umbra unui nor, a?a se va domoli cântecul de biruin?a al stapânitorilor silnici.
Isa 25:6  ?i Domnul Savaot va pregati în muntele acesta pentru toate popoarele un ospa? de carnuri grase, un ospa? cu vinuri bune, carnuri grase cu maduva, vinuri bune, limpezite!
Isa 25:7  ?i în muntele acesta El va da la o parte valul care învaluie toate popoarele ?i perdeaua care acopera toate neamurile.
Isa 25:8  El va înlatura moartea pe vecie! ?i Domnul Dumnezeu va ?terge lacrimile de pe toate fe?ele ?i ru?inea poporului Sau o va îndeparta de pe pamânt, caci Domnul a grait!
Isa 25:9  ?i se va zice în ziua aceea: Iata Dumnezeul nostru în Care nadajduiam ca sa fim mântui?i. Iata Domnul, în Care am nadajduit, sa ne bucuram ?i sa ne veselim de mântuirea Lui,
Isa 25:10  Ca mâna Domnului se va odihni pe acest munte. Moabul însa va fi calcat în picioare pe locul lui, ca ni?te paie în groapa cu gunoi.
Isa 25:11  ?i va întinde mâinile sale, precum înotatorul le întinde ca sa înoate. Dar Domnul va zdrobi mândria lui ?i silin?ele mâinilor lui.
Isa 25:12  Întariturile lui mare?e ?i înalte le va nimici, le va rasturna ?i la pamânt le va culca, în ?arâna.
Isa 26:1  În ziua aceea se va cânta cântarea aceasta în pamântul lui Iuda: "Avem o cetate întarita. Domnul ne vine într-ajutor cu ziduri ?i întarituri.
Isa 26:2  Deschide?i por?ile, ca sa intre un neam drept care paze?te credincio?ia!
Isa 26:3  Nadejde neclintita, Tu ne vei pastra pacea noastra, ca întru Tine ne punem nadejdea.
Isa 26:4  Încrede?i-va în Domnul pururea, caci Domnul Dumnezeu este stânca veacurilor.
Isa 26:5  Ca El a coborât pe cei ce locuiau pe înal?ime, cetatea cea mândra El a supus-o pâna la pamânt, a culcat-o în pulbere.
Isa 26:6  Ea este calcata în picioare, în picioarele saracilor, sub pa?ii obijdui?ilor!
Isa 26:7  Calea celui drept este dreapta; Tu neteze?ti drumul drept al celui drept.
Isa 26:8  Pe calea judeca?ilor Tale, Doamne, noi Te a?teptam; numele Tau ?i amintirea Ta erau nadejdea sufletului nostru.
Isa 26:9  Sufletul meu Te-a dorit în vreme de noapte, duhul meu nazuie?te spre Tine; caci când îndreptarile Tale vor fi pe pamânt, cei ce locuiesc lumea vor înva?a ce este dreptatea.
Isa 26:10  Daca de cel fara de lege ne este mila, el nu mai înva?a ce este dreptatea ?i în pamântul celor sfin?i va savâr?i strâmbatatea. Sa nu mai fie pe pamânt cei fara de lege ?i sa nu mai vada slava Celui Preaînalt.
Isa 26:11  Doamne, mâna Ta era ridicata, dar ei n-au vazut-o! Vor vedea râvna Ta pentru poporul Tau ?i se vor ru?ina. ?i focul harazit vrajma?ilor Tai îi va mânca!
Isa 26:12  Doamne, revarsa pacea peste noi, caci toate lucrurile noastre, pentru noi le-ai facut!
Isa 26:13  Doamne, Dumnezeul nostru, am avut peste noi ?i al?i stapâni afara de Tine, dar noi ne vom aduce aminte numai de numele Tau!
Isa 26:14  Mor?ii nu vor mai trai ?i umbrele nu vor învia, fiindca Tu le-ai pedepsit ?i le-ai nimicit ?i ai ?ters pâna ?i numele lor.
Isa 26:15  Înmul?e?te poporul, Doamne, înmul?e?te poporul ?i arata-Te mare, large?te din nou toate hotarele ?arii!
Isa 26:16  Doamne, pe Tine Te-au cautat ei în vreme de restri?te, catre Tine am strigat în scârba noastra, când Tu ne pedepseai.
Isa 26:17  Ca femeia însarcinata ?i gata sa nasca prunc, care se zvârcole?te ?i striga în durerea ei, a?a am fost noi, Doamne, cu to?ii în fa?a Ta!
Isa 26:18  Zamislit-am, dureri de facere am avut ?i am nascut vânt! Mântuire ?arii noi n-am dat ?i în lume nu s-au nascut locuitorii ei!
Isa 26:19  Mor?ii Tai vor trai ?i trupurile lor vor învia! De?tepta?i-va, cânta?i de bucurie, voi cei ce sala?lui?i în pulbere! Caci roua Ta este roua de lumina ?i din sânul pamântului umbrele vor învia.
Isa 26:20  Du-te, poporul meu, intra în camarile tale ?i închide u?a dupa tine; ascunde-te pu?ine clipe, pâna când mânia va fi trecut!
Isa 26:21  Ca iata Domnul va ie?i din loca?ul Sau, ca sa pedepseasca faradelegile locuitorilor pamântului. Pamântul va arata sângele pe care l-a supt ?i nu va mai ascunde pe uciga?ii lui".
Isa 27:1  În ziua aceea Domnul se va napusti cu sabia Sa grea, mare ?i puternica, asupra leviatanului, a ?arpelui care fuge, asupra leviatanului, a ?arpelui încolacit, ?i va omorî balaurul cel din Nil.
Isa 27:2  ?i se va zice în ziua aceea: "Vie cu vin bun, cânta!
Isa 27:3  Eu, Domnul, sunt strajerul ei, în fiecare clipa Eu o ud, ca frunzele ei sa nu cada. Zi ?i noapte Eu o pazesc;
Isa 27:4  Nu sunt mâniat de fel pe ea. Dar daca voi gasi maracini ?i spini, voi porni razboi împotriva lor ?i-i voi arde pe to?i.
Isa 27:5  Sau mai bine sa caute ocrotirea Mea ?i cu Mine sa faca pace, ?i cu Mine sa fie în pace!...".
Isa 27:6  Dar într-o zi Iacov va prinde radacini, Israel va înflori, va rodi ?i cu roadele sale lumea o va umple.
Isa 27:7  L-a lovit oare Domnul cum l-au lovit cei ce l-au lovit, sau i-a omorât El cum au facut uciga?ii lui?
Isa 27:8  Cu izgonire, cu robie pedepsitu-i-a ?i i-a maturat cu suflarea Lui naprasnica de vânt de rasarit.
Isa 27:9  A?a a fost ispa?ita faradelegea lui Iacov, ?i acesta este rodul iertarii pacatului sau. El a sfarâmat în buca?i toate pietrele jertfelnicului, ca ni?te pietre de var; dumbravile Astartei ?i stâlpii soarelui nu se vor mai ridica.
Isa 27:10  Cetatea cea întarita a ramas singura, un loc parasit ?i neumblat ca un pustiu. Acolo pa?te vi?elul, în ea î?i are sala?ul ?i îi manânca mladi?ele.
Isa 27:11  Când crengile se usuca, se rup ?i cad, femeile vin ?i le dau foc. Acesta este un popor fara de minte ?i nici Ziditorul lui nu Se milostive?te de el ?i nici Facatorul lui nu are mila de el.
Isa 27:12  ?i în ziua aceea, Domnul va aduna roade de la Eufrat ?i pâna la râul Egiptului; ?i voi ve?i fi cule?i unul câte unul, feciori ai lui Israel!
Isa 27:13  În vremea aceea, trâmbi?a cea mare va trâmbi?a ?i cei care se pierdusera în pamântul Asiriei ?i cei ce se risipisera în ?ara Egiptului vor veni ?i se vor închina Domnului, în muntele cel sfânt, în Ierusalim.
Isa 28:1  Vai de cununa mândriei be?ivilor din Efraim; vai de floarea ve?teda din podoaba lor, care sta pe culmea de deasupra vaii celei manoase a celor be?i de vin!
Isa 28:2  Iata un om tare ?i puternic vine de la Domnul: ca un potop de grindina, ca o vijelie nimicitoare, ca o navala de apa potopitoare o va rasturna la pamânt.
Isa 28:3  ?i va fi calcata în picioare cununa cea îngâmfata a be?ivilor din Efraim;
Isa 28:4  Iar floarea cea ve?tejita din stralucita sa gateala care straluce?te pe culmea de deasupra vaii celei manoase, va fi ca o smochina timpurie ?i înainte de vreme; cine o vede o ia ?i o manânca!
Isa 28:5  În ziua aceea Domnul Savaot va fi o cununa stralucitoare ?i o stralucita gateala pentru cei ce au mai ramas din popor,
Isa 28:6  Duh de dreptate pentru cei ce stau la judecata cu dreptate ?i tarie pentru cei ce se lupta la por?i.
Isa 28:7  Dar ?i ace?tia se clatina de vin ?i ratacesc drumul din pricina bauturilor tari; preotul ?i proorocul se poticnesc de bautura, sunt birui?i de vin, au ame?eli din pricina bauturilor tari, în vedenii se în?ala, în hotarâri ?ovaiesc.
Isa 28:8  Toate mesele sunt pline de varsaturi, nici un loc curat nu mai este.
Isa 28:9  Dar totu?i zic: "Pe cine vrea acesta sa înve?e cu vedenia? ?i pe cine vrea el cu propovaduirea sa în?elep?easca? Au doar pe cei în?arca?i sau pe cei abia departa?i de la sânul mamei lor?
Isa 28:10  Caci ?av la?av, ?av la?av, cav lacav, cav lacav, zeher ?am, zeher ?am, (porunca peste porunca, porunca peste porunca, regula peste regula, regula peste regula, când pe aici, când pe acolo!)
Isa 28:11  De aceea într-o limba straina ?i stâlcita se va grai poporului acestuia,
Isa 28:12  Caruia i se spunea: "Iata odihna, sa se odihneasca cel care este obosit; iata u?urarea, dar el n-a vrut sa asculte".
Isa 28:13  ?i cuvântul Domnului va fi pentru ei: ?av la?av, ?av la?av, cav lacav, cav lacav, zeher ?am, zeher ?am, (porunca peste porunca, porunca peste porunca, regula peste regula, regula peste regula, când pe aici, când pe acolo) ca sa mearga ?i sa cada peste cap, sa se sfarâme ?i în cursa sa fie prin?i!
Isa 28:14  Pentru aceasta, asculta?i cuvântul Domnului, voi, oameni de râs, îndrumatori ai poporului celui din Ierusalim!
Isa 28:15  Voi zice?i: Noi am facut legamânt cu moartea ?i cu iadul (?eolul) învoiala; urgia va trece fara sa ne atinga, caci ne-am facut din minciuna un adapost ?i din viclenie un liman!
Isa 28:16  Pentru aceasta a?a zice Dumnezeu: "Pus-am în Sion o piatra, o piatra de încercare, piatra din capul unghiului, de mare pre?, bine pusa în temelie; cel care se va bizui pe ea, nu se va clatina!
Isa 28:17  ?i voi face judecata dreptar ?i dreptatea cumpana. ?i grindina va lua la vale adapostul minciunii ?i potop de ape va peste locul ei de scapare!
Isa 28:18  ?i legamântul vostru cu moartea va fi stricat ?i în?elegerea voastra cu iadul (?eolul) va fi desfacuta. Când urgia va trece, va va zdrobi,
Isa 28:19  Ori de câte ori va trece, va va apuca! Caci ea va trece în fiecare diminea?a, ziua ?i noaptea, ?i nu va fi decât groaza pentru a pricepe descoperirea!
Isa 28:20  Patul acesta va fi scurt ?i nu te vei putea întinde, iar a?ternutul lui prea scurt, ca sa te învele?ti".
Isa 28:21  Ca Domnul se va ridica precum altadata în muntele Pera?im ?i se va întarâta ca în valea Ghibeonului ca sa savâr?easca fapta Lui, fapta Lui ciudata, sa împlineasca lucrul Lui, lucrul Lui minunat.
Isa 28:22  Deci nu va mai bate?i joc, ca legaturile voastre sa nu se strânga, ca am auzit de la Domnul Dumnezeu Savaot ca nimicirea este hotarâta sa fie pentru toata ?ara!
Isa 28:23  Lua?i aminte ?i asculta?i; fi?i cu luare aminte ?i asculta?i graiul meu!
Isa 28:24  Oare în fiecare zi plugarul ara, seamana, desfunda pamântul ?i îl grapeaza?
Isa 28:25  Nu vine el apoi, dupa ce i-a netezit fa?a, sa arunce în brazde chimenul, sa puna grâul, orzul ?i alacul pe margini?
Isa 28:26  Dumnezeul lui îl înva?a ?i da aceste rânduieli.
Isa 28:27  Meiul nu este calcat sub copita cailor ?i tavalugul nu trece peste chimen; ci meiul cu un bal este batut ?i chimenul cu o nuia.
Isa 28:28  Grâul este treierat, dar nu sfarâmat. Peste el trece un tavalug purtat de cai ?i îl scutura din spice.
Isa 28:29  ?i aceasta vine de la Domnul Savaot. Minunat este sfatul Lui ?i mare purtarea Lui de grija!
Isa 29:1  Vai ?ie, Ariele, Ariele, cetate în care a trait David! Treaca an de an, ?irul de praznice sa se sfâr?easca!
Isa 29:2  Apoi voi împresura Arielul ?i el va plânge ?i va geme! Cetatea va fi ca un Ariel pentru Mine.
Isa 29:3  Ca David voi tabarî asupra ta, te voi înconjura cu valuri ?i voi ridica întarituri împotriva ta.
Isa 29:4  Vei fi doborât la pamânt ?i de acolo se va auzi glasul tau; graiul tau din ?arâna se va auzi; glasul tau va fi ca al unei naluci ce iese din pamânt ?i din praf spusele tale ca un murmur vor parea.
Isa 29:5  Mul?imea vrajma?ilor tai va fi ca pulberea marunta, ceata asupritorilor ca pleava care zboara. Dar aceasta se va petrece într-o clipa.
Isa 29:6  Domnul Savaot te va cerceta cu tunet, cutremur ?i zgomot mare, uragan ?i vijelie ?i flacari de foc mistuitor!
Isa 29:7  ?i ca un vis, ca o vedenie de noapte va fi mul?imea de popoare luptatoare împotriva lui Ariel, care se vor razboi cu el, cu cetatea lui ?i de jur împrejur o vor strânge.
Isa 29:8  Dupa cum cel flamând viseaza ca manânca ?i se treze?te tot cu stomacul gol, ?i dupa cum cel însetat viseaza ca bea ?i se treze?te istovit ?i tot însetat, tot a?a se va întâmpla cu mul?imea de popoare care vor merge împotriva muntelui Sion!
Isa 29:9  Sta?i încremeni?i ?i înmarmuri?i, fi?i orbi ?i orbi ramâne?i! Îmbata?i-va, dar nu de vin; clatina?i-va, dar nu de bautura!
Isa 29:10  Ca Domnul a turnat peste voi un duh de toropeala. El a închis ochii vo?tri, profe?ilor, ?i capetele voastre, vazatorilor, le-a acoperit cu val.
Isa 29:11  Drept aceea orice descoperire este pentru voi ca graiurile dintr-o carte pecetluita. Daca le dai cuiva care ?tie carte ?i-i zici: "Cite?te!" el i?i raspunde: "Nu pot, caci ea este pecetluita!"
Isa 29:12  ?i daca o dai cuiva care nu ?tie carte ?i-i zici: "Cite?te!", el î?i va raspunde: "Nu ?tiu carte!"
Isa 29:13  ?i a zis Domnul: "De aceea poporul acesta se apropie de Mine cu gura ?i cu buzele Ma cinste?te, dar cu inima este departe, caci închinarea înaintea Mea nu este decât o rânduiala omeneasca înva?ata de la oameni.
Isa 29:14  De aceea voi face pentru poporul acesta minuni fara seaman. În?elepciunea celor în?elep?i se va pierde ?i iste?imea celor iste?i va pieri.
Isa 29:15  Vai de cei ce ascund lui Dumnezeu taina planurilor lor, ca faptele lor sa se faca la întuneric! Vai de cei care zic: "Cine ne vede? Cine ne ?tie?"
Isa 29:16  Ce stricaciune! Oare olarul poate fi socotit drept lut? Lucrul poate oare zice despre lucrator: "Nu m-a facut el!" Vasul zice oare despre olar: "El nu pricepe?"
Isa 29:17  Înca pu?ina vreme ?i Libanul se va schimba în gradina, ?i gradina va fi socotita padure.
Isa 29:18  În vremea aceea, cei surzi vor auzi cuvintele car?ii ?i ochii celor orbi vor vedea fara umbra ?i fara întuneric.
Isa 29:19  Cei smeri?i se vor bucura întru Domnul ?i cei saraci se vor veseli de Sfântul lui Israel.
Isa 29:20  Ca apasatorul nu va mai fi, cel batjocoritor va pieri, distru?i vor fi cei ce pândeau sa faca rau,
Isa 29:21  Cei care gaseau vina oricui, pentru un cuvânt în fa?a lumii întind cursa judecatorului ?i pentru nimic rapesc dreptul celui cinstit.
Isa 29:22  Pentru aceasta, Domnul, Care a rascumparat pe Avraam a?a zice catre casa lui Iacov: "De aici încolo, nu se va mai ru?ina Iacov ?i fa?a lui nu se va mai îngalbeni.
Isa 29:23  ?i atunci când vor vedea lucrul mâinilor Mele în mijlocul lor, sfin?i-vor numele Meu, vor chema sfânt pe Sfântul lui Iacov ?i se vor teme de Dumnezeul lui Israel.
Isa 29:24  Cei rataci?i cu duhul vor capata în?elepciune ?i cei cârtitori înva?atura".
Isa 30:1  Vai de feciorii razvrati?i, zice Domnul, vai de cei ce fac planuri fara Mine, care fac legaminte ce nu sunt în Duhul Meu, ca sa gramadeasca pacate peste pacate.
Isa 30:2  Ei iau calea Egiptului, fara sa fi întrebat gura Mea, sa cer?easca de la Faraon ajutor ?i la umbra Egiptului sa se adaposteasca.
Isa 30:3  Pentru aceasta sprijinul lui Faraon va fi pentru voi ru?ine ?i râs adapostul la umbra lui.
Isa 30:4  De?i capeteniile lui sunt la ?oan ?i pâna la Hanes ajung trimi?ii lui,
Isa 30:5  To?i sunt nelini?ti?i de acest popor, care nu le va fi de folos, care nu le va da nici un ajutor, ci numai nedumerire ?i ocara.
Isa 30:6  Proorocie despre fiarele de la miazazi: Printr-o ?ara de strâmtorare ?i îngrijorare, cu lei ?i leoaice mugitoare, napârci ?i ?erpi zburatori, ei duc pe magari avu?iile lor ?i pe camile comorile lor, catre un popor care nu le folose?te la nimic.
Isa 30:7  Caci ajutorul Egiptului este de?ertaciune ?i nimic, pentru aceea l-am numit Rahab cel adormit.
Isa 30:8  Acum, du-te" scrie acestea pe o tabla ?i trece-le într-o carte, ca sa fie pentru mai târziu marturie ve?nica.
Isa 30:9  Pentru ca ei sunt un popor de razvrati?i, feciori mincino?i, care nu voiesc sa asculte de legea Domnului,
Isa 30:10  Care zic proorocilor: "Voi nu vede?i!" ?i vazatorilor: "Nu ne prooroci?i pedepse, ci spune?i-ne lucruri magulitoare, prooroci?i-ne închipuiri amagitoare!
Isa 30:11  Da?i-va la o parte din cale, nu ne împiedica?i în drum, lua?i din fa?a noastra pe Sfântul lui Israel!"
Isa 30:12  Pentru aceasta zice Sfântul lui Israel: "Fiindca voi a?i dispre?uit cuvântul acesta ?i v-a?i încrezut în nedreptate ?i minciuna ?i a?i nadajduit numai în ele,
Isa 30:13  Iata cum va fi pacatul vostru: ca o spartura într-un zid înalt, care dintr-o data ?i pe nea?teptate se prabu?e?te;
Isa 30:14  Ca un vas de lut, care este a?a de spart ?i zdrobit fara de mila, încât între cioburile lui nu se afla macar unul cu care sa iei foc din vatra sau sa sco?i apa din fântâna".
Isa 30:15  Ca a?a zice Domnul Dumnezeu, Sfântul lui Israel: "Daca va întoarce?i ?i sunte?i în buna pace, va ve?i izbavi; lini?tea ?i nadejdea sunt vârtutea voastra". Dar voi n-a?i vrut sa asculta?i,
Isa 30:16  Ci a?i zis: "Nu! Noi vom fugi calari pe cai!" A?a, fugi?i! "Vom încaleca pe cai iu?i ca vântul!" Ei bine, ve?i fi urmari?i ?i mai repede!
Isa 30:17  O mie vor fugi de amenin?area unuia ?i când va vor amenin?a cinci, to?i ve?i fugi, pâna când ve?i ramâne ca un stâlp pe vârful muntelui ?i ca un steag pe vârf de deal.
Isa 30:18  Pentru aceasta Domnul a?teapta sa Se milostiveasca spre voi, de aceea El Se ridica sa aiba mila de voi. Ca Domnul este Dumnezeu al drepta?ii; ferici?i sunt cei care nadajduiesc în El!
Isa 30:19  Popor din Sion, care locuie?ti în Ierusalim, nu plânge! El se va milostivi la glasul strigatului tau ?i te va auzi degraba!
Isa 30:20  Când Domnul î?i va fi dat ?ie pâinea îngrijorarii ?i apa strâmtorarii, ?i cei ce te înva?a nu se vor mai ascunde, ci ochii tai vor vedea pe dascalii tai
Isa 30:21  ?i urechile tale vor auzi cuvântul celor ce te calauzesc pe tine, zicând: "Iata calea, merge?i pe ea!", fie ca a?i merge la dreapta sau la stânga,
Isa 30:22  Atunci argintul care acopera idolii îl ve?i gasi spurcat ?i aurul care împodobe?te chipurile turnate, ca necurat îl ve?i arunca, zicând: "Afara de aici!"
Isa 30:23  ?i El î?i va da ploaie pentru semanatura ta pe care vei fi semanat-a pe pamânt ?i pâinea pe care o va rodi pamântul va fi gustoasa ?i hranitoare. Turmele tale vor pa?te în ziua aceea pe paji?ti întinse.
Isa 30:24  ?i boii ?i asinii care lucreaza pamântul, vor mânca nutre? dat cu sare, cu lopata ?i cu bani?a vânturat.
Isa 30:25  Atunci pe orice munte înalt ?i pe orice deal mare, vor fi râule?e ?i pâraie de apa, în ziua macelului groaznic, când turnurile vor cadea.
Isa 30:26  ?i luna va straluci ca soarele, iar soarele va straluci de ?apte ori mai mult, ca lumina a ?apte zile, în ziua când Domnul va lega rana poporului Sau ?i va tamadui vânataile de pe trupul lui.
Isa 30:27  Iata numele Domnului Care vine de departe, mânie înfocata ?i nor greu; buzele Sale sunt pline de urgie ?i limba Lui e foc mistuitor!
Isa 30:28  Duhul Lui ca un ?uvoi revarsat care ajunge pâna la gât, ca sa cearna pe neamuri cu sita nimicirii.
Isa 30:29  Voi ve?i cânta atunci, ca în noaptea cea de praznic, cu bucurie în inimi, în sunetul de flaut, ca sa merge?i în muntele Domnului, vârtutea lui Israel.
Isa 30:30  ?i Domnul va face sa rasune glasul Sau mare? ?i va pravali bra?ul Sau în aprinderea mâniei Sale, în mijlocul unui foc mistuitor, al vijeliei ?i al potopului de ape ?i grindina.
Isa 30:31  La glasul Domnului va tremura Asiria; cu toiagul Sau o va lovi.
Isa 30:32  La fiecare lovitura pe care Domnul i-o va da cu toiagul cel de mustrare, sunete de toba, de harpa ?i de joc vor izbucni. în cântece ?i Domnul va lupta împotriva ei cu mina ridicata.
Isa 30:33  Un jertfelnic de multa vreme este pregatit, hotarât pentru Moloh. Pus-a un rug mare ?i larg, paiele ?i lemnele sunt din bel?ug. Suflarea Domnului îl va aprinde ca un ?uvoi de pucioasa.
Isa 31:1  Vai de cei ce se coboara în Egipt dupa ajutor ?i se bizuie pe caii lor ?i î?i pun nadejdea în mul?imea carelor ?i în puterea calare?ilor, dar nu-?i a?intesc privirea catre Sfântul lui Israel ?i nu cauta pe Domnul.
Isa 31:2  Dar El este în?elept, El face sa vina nenorocirea ?i nu Î?i ia înapoi cuvintele. El Se ridica împotriva casei celor fara de lege ?i împotriva ajutorului celor care savâr?esc nedreptatea.
Isa 31:3  Egipteanul este om, nu Dumnezeu, caii lui sunt carne ?i nu duh. Când Domnul Î?i va întinde mâna Lui, ocrotitorul se va împiedica ?i ocrotitul va cadea, iar amândoi împreuna vor pieri.
Isa 31:4  Ca iata ce mi-a grait Domnul: "Precum leul ?i puiul de leu racnesc asupra prazii ?i împotriva lor se aduna toata ceata de pastori ?i nu se înfioara de strigatele lor, nici nu se tulbura de mul?imea lor, tot astfel Domnul Savaot Se va pogorî sa se razboiasca pe muntele Sionului ?i pe colina lui. ?i du?manii se vor risipi to?i,
Isa 31:5  Ca pasarile care zboara. A?a Domnul Savaot va ocroti Ierusalimul, îl va acoperi, îl va mântui, îl va cru?a, îl va libera".
Isa 31:6  Întoarce?i-va catre Acela de Care adâncul va desparte, copii ai lui Israel!
Isa 31:7  În vremea aceea fiecare din voi ve?i da la o parte idolii de argint ?i cei de aur pe care i-aii facut cu mâinile voastre cele pacatoase.
Isa 31:8  ?i Asiria va cadea în sabie care nu este omeneasca, va fi nimicita nu de sabia unui muritor. Ea o va lua la fuga în fa?a sabiei, iar tinerii vor fi du?i în robie!
Isa 31:9  De frica întaritura ei va fi nimicita, iar capeteniile vor fugi din jurul steagului, zice Domnul, a Carui vapaie este în Sion ?i cuptorul în Ierusalim!
Isa 32:1  Iata ca un rege va stapâni prin dreptate ?i capeteniile vor cârmui cu dreptate.
Isa 32:2  Fiecare va fi ca un adapost împotriva vântului, ca un liman împotriva vijeliei, ca pâraiele de apa într-un pamânt uscat ?i ca umbra unei stânci înalte într-un ?inut însetat.
Isa 32:3  Ochii celor care vad nu vor fi închi?i ?i urechile celor care aud vor lua aminte.
Isa 32:4  Inima celor u?uratici va judeca sanatos ?i limba celor gângavi va grai iute ?i deslu?it.
Isa 32:5  Nebunului nu i se va mai zice ca e de neam bun ?i celui viclean ca e mare la suflet.
Isa 32:6  Ca nebunul graie?te nebunii ?i inima lui gânde?te raul ca sa savâr?easca nelegiuiri, sa rosteasca cuvinte mincinoase împotriva Domnului, sa lase nemâncat pe cel flamând ?i celor înseta?i sa nu le dea sa bea.
Isa 32:7  Uneltele celui mi?el sunt ticaloase, el plasmuie?te uneltiri ca sa piarda pe cei smeri?i prin cuvinte mincinoase, pe cel sarac care-?i cauta dreptate.
Isa 32:8  Omul de vi?a buna sfatuie?te cele cuviincioase ?i staruie?te în cuviin?a lui.
Isa 32:9  Femei fara de grija, scula?i-va ?i asculta?i glasul meu! Fecioare încrezatoare, lua?i aminte la graiul meu!
Isa 32:10  Într-un an ?i câteva zile ve?i tremura, voi încrezatoarelor, culesul va fi trecut ?i strânsul nu se va mai face!
Isa 32:11  Tremura?i, nepasatoarelor, înfiora?i-va, încrezatoarelor, scoate?i îmbracamintea, dezbraca?i-va, încinge?i-va peste mijloc cu haine de jale.
Isa 32:12  Bate?i-va în piept ?i plânge?i pentru ?arinele cele frumoase, ?i rodnicia viilor.
Isa 32:13  Pe pamântul poporului meu vor cre?te spini ?i ciulini, ba ?i în toate casele de petrecere ale veselei ceta?i.
Isa 32:14  Palatul va fi pustiu, cetatea cea zgomotoasa, parasita, colina ?i turnul de straja, pustiite, prefacute pe vecie în vizuini, loc de zburdare pentru asini ?i pa?une pentru turme,
Isa 32:15  Pâna când se va turna din Duhul cel de sus ?i pustiul va fi ca o gradina cu pomi ?i gradina socotita ca o padure.
Isa 32:16  Atunci judecata va locui în de?ert ?i dreptatea va sala?lui în gradina cea cu pomi.
Isa 32:17  Pacea va fi lucrul drepta?ii, roada drepta?ii va fi lini?tea ?i nadejdea în veci de veci.
Isa 32:18  Atunci poporul meu va locui într-un loc de pace, în sala?uri de nadejde ?i în adaposturi fara grija.
Isa 32:19  Padurea va cadea de grindina, iar cetatea va fi supusa.
Isa 32:20  Ferici?i sunte?i voi, care semana?i ?i nu lega?i nici boul, nici asinul!
Isa 33:1  Vai ?ie, pustiitorule, care n-ai fost pustiit ?i ?ie, jefuitorule, care n-ai fost înca jefuit. Când vei sfâr?i de pustiit, vei fi pustiit, când vei fi jefuit din destul, vei fi jefuit ?i tu.
Isa 33:2  Doamne, miluie?te-ne, ca întru Tine am nadajduit, fii ajutorul nostru în fiecare diminea?a ?i izbavirea noastra în vremuri de strâmtorare!
Isa 33:3  La glasul tunetului Tau neamurile vor fugi; când Te ridici Tu, popoarele se vor risipi.
Isa 33:4  ?i vor aduna prada voastra, cum aduna lacustele; arunca-se-vor asupra ei, cum se arunca lacustele.
Isa 33:5  Domnul este mare, El locuie?te în înal?ime; Sionul este plin de judecata ?i de dreptate.
Isa 33:6  Ocrotirea Domnului în aceste vremuri va fi pentru Sion comoara de fericire; în?elepciunea, ?tiin?a ?i temerea de Dumnezeu sunt avu?ia lui.
Isa 33:7  Iata ca locuitorii din Ariel striga pe uli?e, solii pentru pace plâng amar.
Isa 33:8  Drumurile sunt pustii, nici un trecator pe cale. El strica legamântul, nesocote?te ceta?ile, nu mai ?ine seama de nimeni.
Isa 33:9  ?ara plânge ?i tânje?te, Libanul este tulburat ?i ofilit. ?aronul a ajuns ca un pustiu, Basanul ?i Carmelul î?i scutura frunzi?ul lor.
Isa 33:10  "Acum Ma voi scula, zice Domnul, acum Ma voi ridica, acum Ma voi înal?a!"
Isa 33:11  Zamislit-a?i fin ?i a?i nascut paie, suflarea voastra e foc care va va mistui,
Isa 33:12  Popoarele vor fi prefacute în cenu?a ca spinii taia?i ?i mistui?i de foc!
Isa 33:13  Voi cei de departe, auzi?i ce am facut, ?i voi cei de aproape, cunoa?te?i puterea Mea!
Isa 33:14  Pacato?ii vor tremura în Sion ?i pe cei fara de lege fierul îi va cuprinde: "Care din noi poate sa îndure focul mistuitor, care din noi poate sa stea pe jarul cel de veci?"
Isa 33:15  Omul cel drept în calea sa ?i cel ce graie?te cuvinte de cinste, care da la o parte câ?tigul cel nedrept, cel ce mâinile înapoi le trage ?i mita nu prime?te, care-?i astupa urechile când aude faradelegi ?i î?i pune val pe ochi ca sa nu mai vada raul,
Isa 33:16  Acela va locui pe înal?imi, ?i stâncile cele tari vor fi cetatea lui; pâine i se va da ?i apa nu-i va lipsi.
Isa 33:17  Ochii tai vor privi pe rege în toata frumuse?ea lui ?i o ?ara îndepartata vor vedea.
Isa 33:18  Inima ta î?i va aduce aminte de aceste vremuri de groaza, zicând: "Unde este scriitorul, unde este vistiernicul, unde este strajuitorul cel din turnuri?
Isa 33:19  Atunci nu vei mai vedea pe poporul acesta îndrazne?, acest neam cu vorbe încâlcite, pe care nu-l în?elegem, care bâlbâie o limba care nu se poate pricepe.
Isa 33:20  Prive?te Sionul, cetatea sarbatorilor noastre; ochii tai sa vada Ierusalimul, loc de lini?te, cort bine înfipt, ai carui ?aru?i nu se pot scoate, ale carui frânghii nu se pot desface.
Isa 33:21  Domnul este pentru noi aici în toata slava Sa; El ?ine loc pentru noi de fluvii, de largi canaluri, pe care nici o barca cu vâsle nu trece, pe care nici o corabie mare nu merge.
Isa 33:22  Domnul este Judecatorul nostru, Domnul este Capetenia noastra, Domnul este Împaratul nostru, El ne va izbavi!
Isa 33:23  Frânghiile tale sunt dezlegate, ele nu mai sprijina catargul, nici nu mai întind pânzele. Atunci se va împar?i o mare prada ?i ?chiopii vor avea parte de ea.
Isa 33:24  Nimeni dintre locuitorii Sionului nu va zice: "Sunt bolnav!" Poporul care-l locuie?te va dobândi iertarea pacatelor.
Isa 34:1  Apropia?i-va, voi neamuri, ?i asculta?i, ?i voi popoare, lua?i aminte; sa asculte pamântul ?i cei ce-l locuiesc, lumea cu toate fapturile ei.
Isa 34:2  Ca Domnul este mâniat asupra popoarelor, cu urgie împotriva o?tirii lor. El le nimice?te ?i le da la junghiere;
Isa 34:3  Mor?ii lor vor fi arunca?i pe câmp, cadavrele lor greu vor mirosi ?i prin mun?i vor ?erpui pârâia?e din sângele lor.
Isa 34:4  Toata o?tirea cerului se va topi, cerurile se vor strânge ca un sul de hârtie ?i toata o?tirea lor va cadea cum cad frunzele de vi?a ?i cele de smochin.
Isa 34:5  Ca s-a îmbatat de mânie în ceruri sabia Domnului ?i iata ca asupra lui Edom coboara, asupra poporului harazit pedepsei.
Isa 34:6  Sabia Domnului este plina de sânge, acoperita de grasime, de sânge de miei ?i de ?api, de grasimea rarunchilor de berbeci. Ca Domnul face jertfa la Bo?ra ?i mare junghiere în ?ara lui Edom.
Isa 34:7  Bivolii cad împreuna cu ei, ?i boii cu taurii. ?i pamântul se îmbata de sângele lor ?i pulberea de grasime este plina.
Isa 34:8  Caci aceasta este ziua de razbunare a Domnului, an de rasplatire pentru pricina Sionului!
Isa 34:9  Râurile în pacura se vor preface ?i pulberea în pucioasa. Pamântul lui va fi pucioasa arzatoare,
Isa 34:10  Zi ?i noapte. Niciodata nu se va mai stinge în veci de veci ?i din neam în neam se va înal?a vapaia ?i fumul lui. Pe veci el va ramâne pustiu ?i nimeni pe acolo nu va trece.
Isa 34:11  Pelicanul ?i ariciul vor fi stapânii lui, bufni?a ?i corbul, locuitorii lui. ?i pe deasupra, Domnul va întinde peste el frânghia nimicirii ?i cumpana pustiirii. ?i oameni cu chip de ?ap într-însul vor locui ?i de vi?a buna socoti?i vor fi.
Isa 34:12  Nu se va pomeni acolo de nici un regat ?i to?i prin?ii lui vor fi nimici?i.
Isa 34:13  În palatele lui vor cre?te spini, iar în turnurile darâmate maracini ?i urgie. Acolo va fi sala?ul ?acalilor ?i adapostul stru?ilor.
Isa 34:14  Câini ?i pisici salbatice se vor pripa?i pe acolo ?i fapturi omene?ti cu chip de ?ap se vor strânge (fara numar). Acolo vor zabovi naluci ce umbla noaptea ?i în acele locuri î?i vor gasi odihna.
Isa 34:15  Acolo î?i va face ?arpele cuibul, ?i va depune oua în el, va cloci ?i va scoate pui. Acolo se vor strânge vulturii de prada ?i în acele locuri se vor gasi cu to?ii.
Isa 34:16  Cerceta?i cartea Domnului ?i citi?i, ca nimic din acestea nu lipse?te. Caci gura Domnului a poruncit ?i suflarea Lui le-a adunat.
Isa 34:17  El singur a aruncat sor?ii ?i mâna Lui le-a împar?it pamântul cu funia. Pentru totdeauna ei le vor stapâni ?i în el vor locui din neam în neam.
Isa 35:1  Vesele?te-te pustiu însetat, sa se bucure pustiul; ca ?i crinul sa înfloreasca.
Isa 35:2  ?i va înflori ?i se va bucura pustiul Iordanului ?i marirea Libanului se va da lui ?i cinstea Carmelului; ?i poporul meu va vedea slava Domnului, stralucirea Dumnezeului nostru.
Isa 35:3  Întari?i-va voi, mâini slabe ?i prinde?i putere genunchi slabanogi.
Isa 35:4  Zice?i celor slabi la inima ?i la cuget: "Întari?i-va ?i nu va teme?i. Iata Dumnezeul nostru! Cu judecata rasplate?te ?i va rasplati; El va veni ?i ne va mântui".
Isa 35:5  Atunci se vor deschide ochii celor orbi ?i urechile celor surzi vor auzi.
Isa 35:6  Atunci va sari ?chiopul ca cerbul ?i limpede va fi limba gângavilor; ca izvoare de apa vor curge în pustiu ?i pâraie în pamânt însetat.
Isa 35:7  Pamântul cel fara de apa se va preface în bal?i ?i ?inutul cel însetat va fi izvor de apa. Acolo va fi veselia pasarilor, iarba, trestie ?i bal?i.
Isa 35:8  Acolo va fi cale curata ?i cale sfânta se va chema ?i nu va trece pe acolo nimeni necurat ?i nici nu va fi acolo cale întinata. Chiar ?i cei fara de minte vor merge pe dânsa ?i nu se vor rataci.
Isa 35:9  ?i nu va fi acolo leu, nici fiare cumplite nu se vor sui pe ea ?i nici nu se vor afla acolo; ci vor merge pe dânsa cei mântui?i ?i cei rascumpara?i de Domnul se vor întoarce.
Isa 35:10  ?i vor veni în Sion în chiote de bucurie ?i veselia cea ve?nica va încununa capul lor. Lauda ?i bucuria ?i veselia îi vor ajunge pe ace?tia ?i vor fugi durerea, întristarea ?i suspinarea.
Isa 36:1  În anul al paisprezecelea al domniei lui Iezechia, Sanherib, regele Asiriei, a pornit cu razboi împotriva ceta?ilor celor întarite ale lui Iuda ?i le-a cuprins.
Isa 36:2  ?i regele Asiriei a trimis pe Rab?ache cu mare o?tire din Lachi? la Ierusalim catre regele Iezechia. Rab?ache a tabarât lânga canalul de apa al iazului de sus, pe drumul catre ?arina nalbitorului.
Isa 36:3  Atunci a ie?it întru întâmpinarea lui Eliachim, feciorul lui Hilchia, capetenia casei regelui, ?i ?ebna scriitorul ?i Ioah cronicarul, feciorul lui Asaf.
Isa 36:4  ?i Rab?ache a zis catre ei: "Spune?i lui Iezechia: A?a zice regele cel mare, regele Asiriei: De unde vine încrederea aceasta pe care te bizui?
Isa 36:5  Crezi tu ca vorbele goale slujesc drept sfat ?i tarie în lupta? În cine ?i-ai pus nadejdea, de te-ai razvratit împotriva mea?
Isa 36:6  Ah, ?tiu! ?i-ai pus nadejdea în Egipt; ai luat ca ocrotitor aceasta trestie frânta, care sparge ?i intra în mâna oricui se sprijina de ea. A?a este Faraon pentru to?i cei ce se încred în el!
Isa 36:7  ?i daca voi îmi zice?i: "În Domnul, Dumnezeul nostru, ne-am pus nadejdea noastra", oare nu este Acesta Dumnezeul pentru Care Iezechia a oprit închinarea pe dealurile înalte ?i altarele Lui le-a nimicit, zicând catre Iuda ?i Ierusalim: "Voi va ve?i închina numai înaintea acestui jertfelnic?"
Isa 36:8  ?i acum fa acest legamânt cu stapânul meu, regele Asiriei, ?i eu î?i voi da ?ie doua mii de cai, numai sa ai tot atâ?i calare?i care sa-i încalece.
Isa 36:9  Cum ai putea tu sa nu iei în seama pe unul din cei mai mici slujitori ai stapânului meu? Dar tu te duci în Egipt pentru cai ?i pentru calare?i.
Isa 36:10  ?i crezi tu ca fara voia Domnului m-am suit eu în aceasta ?ara ca sa o pustiesc? Domnul mi-a spus: "Suie-te în ?inutul acesta ?i-l pustie?te!"
Isa 36:11  Atunci Eliachim, ?ebna ?i Ioah raspunsera lui Rab?ache: "Graie?te robilor tai în graiul arameian, ca noi îl în?elegem, ?i nu ne grai în limba iudaica în auzul poporului care este pe ziduri!"
Isa 36:12  ?i a zis Rab?ache: "Catre stapânul tau ?i catre tine m-a trimis stapânul meu ca sa graiesc cuvintele acestea? Oare nu catre oamenii care stau pe ziduri ?i curând vor fi sili?i  sa î?i manânce cu voi excrementele ?i sa î?i bea urina?"
Isa 36:13  ?i a stat Rab?ache ?i a strigat cu glas mare în limba iudaica ?i a zis: "Asculta?i cuvintele marelui rege, regele Asiriei!
Isa 36:14  Ca iata ce va spune regele: "Iezechia sa nu va în?ele pe voi, caci el nu va va putea scapa;
Isa 36:15  ?i Iezechia sa nu va faca sa nadajdui?i în Domnul, zicând: "Domnul ne va izbavi ?i nu va da cetatea aceasta în mâna regelui Asiriei!"
Isa 36:16  Nu asculta?i pe Iezechia, ca iata ce zice regele Asiriei: "Face?i pace cu mine ?i fi?i supu?ii mei, ?i fiecare va mânca din via ?i din smochinul sau ?i va bea apa din pu?ul sau,
Isa 36:17  Pâna ce voi veni ca sa va duc într-o ?ara ca a voastra, ?ara de grâu ?i de must, de pâine ?i de vii.
Isa 36:18  ?i Iezechia sa nu în?ele credin?a voastra, zicând: "Domnul ne va scapa!" Oare dumnezeii neamurilor au scapat fiecare ?ara lui din mâna regelui Asiriei?
Isa 36:19  Unde sunt dumnezeii Hamatului ?i Arpadului ?i ai Samariei? Au scapat ei oare Samaria din mâinile mele?
Isa 36:20  Care din to?i dumnezeii ?arilor acestora au scapat ?ara lor din mina mea, ca Domnul Dumnezeul vostru sa elibereze Ierusalimul din mina mea?"
Isa 36:21  ?i ei au tacut ?i nimic nu i-au raspuns, pentru ca era porunca regelui care spunea: "Sa nu-i raspunde?i!"
Isa 36:22  Atunci Eliachim, feciorul lui Hilchia, mai-marele peste casa regelui, ?i ?ebna scriitorul ?i Ioah cronicarul, feciorul lui Asaf, au venit la Iezechia ?i, rupându-?i hainele, i-au facut cunoscut cuvintele lui Rab?ache.
Isa 37:1  ?i când a auzit regele Iezechia cuvintele acestea, ?i-a rupt ve?mintele, s-a îmbracat în sac ?i a intrat în templul Domnului.
Isa 37:2  ?i a trimis pe Eliachim, cel de peste casa sa, ?i pe ?ebna scriitorul ?i pe cei mai batrâni dintre preo?i, îmbraca?i în sac, catre proorocul Isaia, fiul lui Amos.
Isa 37:3  ?i au zis catre dânsul: "Ziua de astazi este zi de strâmtorare, de pedeapsa ?i de ru?ine; caci pruncii sunt gata a ie?i din pântecele maicii lor, dar ele nu au putere sa-i nasca!
Isa 37:4  Poate Domnul Dumnezeul tau a auzit cuvintele lui Rab?ache, pe care le-a trimis regele Asiriei, stapânul sau, ca sa faca de ocara pe Dumnezeul cel viu, ?i Domnul Dumnezeul tau poate îl va pedepsi pentru cuvintele pe care le-a auzit. Înal?a dar o rugaciune pentru rama?i?a care se mai afla!"
Isa 37:5  ?i au intrat robii regelui Iezechia la proorocul Isaia.
Isa 37:6  ?i le-a zis Isaia: "A?a ve?i raspunde stapânului vostru: A?a graie?te Domnul Dumnezeu: Nu te teme de cuvintele pângaritoare pe care le-ai auzit din partea slujitorilor regelui Asiriei.
Isa 37:7  Iata, voi pune în el un astfel de duh, ca va primi o veste ?i se va întoarce în ?ara lui ?i acolo va cadea în ascu?i?ul sabiei".
Isa 37:8  ?i s-a întors Rab?ache ?i a aflat pe regele Asiriei tabarât la Libna, caci i se spusese ca a plecat din Lachi?.
Isa 37:9  Atunci (regele Asiriei) a aflat ca Tirhaca, regele Etiopiei, pornise împotriva lui ?i iara?i a trimis soli catre Iezechia, zicând:
Isa 37:10  "A?a ve?i zice lui Iezechia, regele lui Iuda: Sa nu te încrezi în Dumnezeul tau, ?i sa nu te amage?ti, zicând: Ierusalimul nu va fi dat în mâinile regelui Asiriei.
Isa 37:11  Tu ai aflat ceea ce au facut regii Asiriei tuturor ?arilor, cum le-au nimicit ?i numai tu ai scapat!
Isa 37:12  Oare dumnezeii lor au izbavit popoarele pe care le-au distrus parin?ii mei: Gozanul, Haranul, Re?eful ?i pe fiii lui Eden din Telasar?
Isa 37:13  Unde este regele Hamatului, al Arpadului, cel al ceta?ii Sefarvaim, al Henei ?i al Ivei?"
Isa 37:14  Atunci Iezechia a luat scrisoarea din mina trimi?ilor ?i a citit-o. Apoi el a intrat în templul Domnului ?i a întins-o desfacuta înaintea Domnului.
Isa 37:15  ?i s-a rugat Iezechia catre Domnul, zicând:
Isa 37:16  "Doamne Savaot, Dumnezeul lui Israel, Care stai pe heruvimi, numai Tu singur e?ti Dumnezeu al tuturor regatelor de pe pamânt. Tu ai facut cerul ?i pamântul.
Isa 37:17  Pleaca, Doamne, urechea Ta ?i deschide, Doamne, ochii Tai ?i vezi ?i ia aminte la cuvintele lui Sanherib, pe care le-a trimis ca sa faca de batjocura pe Dumnezeul cel viu.
Isa 37:18  Cu adevarat, Doamne, regii Asiriei au nimicit toate neamurile ?i ?arile lor;
Isa 37:19  ?i pe dumnezeii lor i-au ars cu foc, ca ei nu sunt dumnezei, ci lucruri de mâini omene?ti: lemn ?i piatra; pentru aceea ei i-au nimicit.
Isa 37:20  ?i acum, Doamne, Dumnezeul nostru, izbave?te-ne din mâna lui ca sa ?tie toate împara?iile pamântului ca Tu singur e?ti Domnul nostru!"
Isa 37:21  ?i a trimis Isaia, fiul lui Amos, catre Iezechia, zicând: "A?a zice Domnul Dumnezeul lui Israel, catre Care te-ai rugat cu privire la Sanherib, regele Asiriei;
Isa 37:22  Iata hotarârea pe care a rostit-o împotriva lui: "Te dispre?uie?te ?i î?i bate joc de tine fecioara, fiica Sionului; în spatele tau clatina din cap fiica Ierusalimului!
Isa 37:23  Pe cine ai pângarit ?i îi facut de râs ?i împotriva cui ai ridicat glasul ?i sus ai înal?at ochii tai? împotriva Sfântului lui Israel!
Isa 37:24  Prin mina servilor tai ai hulit pe Domnul meu ?i ai zis: Cu carele mele multe voi urca pe vârfurile mun?ilor, pe cele mai înalte piscuri ale Libanului! Voi taia cedrii cei falnici ?i cei mai de seama dintre chiparo?i ?i voi ajunge pe cele mai înalte culmi cu paduri dese.
Isa 37:25  Ca eu sunt cel ce am sapat fântâni ?i am baut apa ?i am secat sub pa?ii mei toate pâraiele Egiptului!
Isa 37:26  Oare nu auzi tu? Din vremi stravechi am pregatit aceasta; din veac le-am hotarât ?i acum le aduc la îndeplinire! Tu voiai sa prefaci în ruina ceta?ile cele întarite.
Isa 37:27  Cei ce locuiau în ele erau fara putere, înspaimânta?i ?i ului?i. Ca iarba câmpului erau ei, ca frageda verdea?a, ca iarba de pe acoperi?uri înainte ca paiul ei sa fi fost crescut.
Isa 37:28  ?tiu când te scoli ?i când te culci, toate faptele tale Îmi sunt cunoscute.
Isa 37:29  Întarâtarea ta împotriva Mea, trufia ta au ajuns pâna la urechile Mele. De aceea voi pune belciug în narile tale ?i frâul Meu buzelor tale ?i te voi întoarce pe calea pe care ai venit!
Isa 37:30  ?i pentru tine, acesta va fi semnul: anul acesta mânca?i din pâinea ce cre?te pe ogoare, în anul al doilea, din ceea ce cre?te de la sine, iar în al treilea an semana?i, secera?i, sadi?i vii ?i mânca?i din roadele lor.
Isa 37:31  ?i rama?i?a care va fi scapat din casa lui Iuda î?i va înfige radacini în jos ?i va face roade în sus.
Isa 37:32  Ca din Ierusalim va ie?i o rama?ita ?i din muntele Sionului cei scapa?i cu via?a. Râvna Domnului Savaot va face aceasta.
Isa 37:33  Pentru aceasta, a?a zice Domnul catre regele Asiriei: Nu va intra în aceasta cetate ?i nu va arunca nici o sageata. Nu va porni împotriva ei cu scut ?i nu o va înconjura cu valuri.
Isa 37:34  Pe calea pe care a venit se va întoarce ?i nu va intra în cetatea aceasta, zice Domnul.
Isa 37:35  Apara-voi cetatea aceasta ?i o voi scapa pentru Mine ?i pentru David, sluga Mea!"
Isa 37:36  ?i a ie?it îngerul Domnului ?i a batut în tabara Asiriei o suta ?i cincizeci de mii; iar diminea?a, la sculare, to?i erau mor?i.
Isa 37:37  Atunci Sanherib, regele Asiriei, a ridicat tabara ?i a plecat ?i s-a oprit la Ninive.
Isa 37:38  ?i pe când el se închina în templul lui Nisroc, dumnezeul sau, Adramelec ?i ?areser, feciorii lui, l-au lovit cu sabia ?i au fugit în ?inutul Ararat. Iar în locul lui, a domnit fiul sau Asarhadon.
Isa 38:1  În vremea aceea Iezechia s-a îmbolnavit de moarte. ?i a intrat la el Isaia, fiul lui Amos, ?i i-a zis: "A?a graie?te Domnul: Pune rânduiala în casa ta, ca nu vei mai trai, ci vei muri".
Isa 38:2  Atunci s-a întors Iezechia cu fa?a la perete ?i s-a rugat Domnului:
Isa 38:3  "O, Doamne! Adu-?i aminte ca am umblat înaintea Ta întru credincio?ie ?i cu inima curata, savâr?ind ceea ce este placut înaintea ochilor Tai!" ?i a izbucnit Iezechia în hohote de plâns.
Isa 38:4  ?i a fost cuvântul Domnului catre Isaia, zicând:
Isa 38:5  "Du-te ?i spune lui Iezechia: A?a graie?te Domnul Dumnezeul lui David, tatal tau: Ascultat-am rugaciunea ta, vazut-am lacrimile tale, iata voi adauga la via?a ta înca cincisprezece ani
Isa 38:6  ?i din mâna regelui Asiriei te voi izbavi pe tine ?i cetatea aceasta ?i o voi ocroti".
Isa 38:7  ?i iata semnul care ?i se va da fie de la Domnul ca El Î?i va împlini cuvântul Sau:
Isa 38:8  "Iata voi întoarce umbra cu atâtea linii pe care soarele le-a strabatut pe ceasornicul lui Ahaz, sa zic cu zece linii". ?i soarele s-a dat înapoi cu zece linii pe care el le strabatuse.
Isa 38:9  Rugaciunea lui Iezechia, regele lui Iuda, când a cazut bolnav ?i s-a tamaduit de boala lui:
Isa 38:10  "Atunci eu am zis: Ma duc la amiaza zilelor mele, la por?ile locuin?ei mor?ilor voi fi ?inut pentru restul anilor mei.
Isa 38:11  Nu voi mai vedea pe Domnul în pamântul celor vii; ?i nu voi mai privi pe nimeni dintre locuitorii lumii.
Isa 38:12  Casa mea este smulsa ?i dusa departe de mine, ca o coliba de ciobani. Îmi simt firul vie?ii taiat ca de un ?esator care m-ar rupe din ?esatura lui. De diminea?a pâna seara, Tu ai sfâr?it cu mine.
Isa 38:13  Strig pâna diminea?a. Ca un leu (boala) îmi sfarâma oasele mele! De diminea?a pâna seara, Tu ai sfâr?it cu mine.
Isa 38:14  ?ip cumplit ca o rândunica, gem ca o porumbi?a. Ochii mei slabesc, uitându-se în sus. Doamne, sunt în mare cumpana, nu ma lasa!
Isa 38:15  Ce sa mai graiesc! El mi-a dat de ?tire ?i a facut! Sfâr?i-voi firul vie?ii mele, aducându-mi aminte de amaraciunea sufletului meu!
Isa 38:16  Doamne, prin îndurarea Ta se bucura omul de via?a, prin ea mai am ?i eu suflare; Tu ma tamaduie?ti ?i-mi dai iara?i via?a!
Isa 38:17  Iata ca boala mea se schimba în sanatate. Tu ai pazit via?a mea de adâncul mistuitor! Tu ai aruncat înapoia Ta toate pacatele mele!
Isa 38:18  Ca locuin?a mor?ilor nu Te va lauda ?i moartea nu Te va preaslavi; cei ce se coboara în mormânt nu mai nadajduiesc în credincio?ia Ta.
Isa 38:19  Cel viu, cel viu Te lauda, ca mine astazi; parin?ii înva?a pe copiii lor credincio?ia Ta.
Isa 38:20  Domnul sa ne mântuiasca ?i vom cânta din harpa în toate zilele vie?ii noastre înaintea templului Domnului!"
Isa 38:21  ?i Isaia a adus o turta de smochine ?i a pus-o deasupra bubei ?i Iezechia s-a vindecat.
Isa 38:22  ?i Iezechia a întrebat: "Dupa care semn voi ?ti ca voi intra în templul Domnului?"
Isa 39:1  În vremea aceea, Merodac-Baladan, fiul lui Baladan, regele Babilonului, a trimis scrisori ?i un dar lui Iezechia, auzind ca a fost bolnav ?i s-a facut sanatos.
Isa 39:2  ?i s-a bucurat pentru ele Iezechia ?i a aratat solilor vistieria, argintul, aurul, miresmele ?i untdelemnul cel bun ?i toata strânsura lui de arme ?i tot ceea ce se afla în camarile lui. ?i n-a ramas nimic în casa lui ?i în tot cuprinsul stapânirii lui pe care Iezechia sa nu-l fi aratat.
Isa 39:3  Atunci a zis proorocul Isaia catre regele Iezechia: "Ce au zis oamenii ace?tia ?i de unde au venit ei la tine?" ?i a raspuns Iezechia: "Au venit dintr-o ?ara departata, din Babilon!"
Isa 39:4  ?i a mai întrebat: "Ce au vazut în casa ta?" ?i a zis Iezechia: "Au vazut toate câte sunt în casa mea; ?i n-a ramas nimic în vistieriile mele pe care sa nu-l fi aratat".
Isa 39:5  ?i a zis Isaia catre Iezechia: "Asculta ceea ce graie?te Domnul Savaot!
Isa 39:6  Iata vin zile, când tot ceea ce au agonisit parin?ii tai pâna astazi va fi dus în Babilon ?i nu va ramâne nimic, a?a zice Domnul.
Isa 39:7  ?i din feciorii care vor ie?i din tine ?i îi vei na?te, vor lua. ?i vor fi eunuci la curtea regelui din Babilon".
Isa 39:8  ?i a zis Iezechia catre Isaia: "Bun este cuvântul Domnului pe care l-ai grait!" Caci, se gândea el: "Va fi pace ?i lini?te în timpul vie?ii mele!"
Isa 40:1  "Mângâia?i, mângâia?i pe poporul Meu", zice Dumnezeul vostru.
Isa 40:2  "Da?i curaj Ierusalimului ?i striga?i-i ca munca de rob a luat sfâr?it, faradelegea sa a fost ispa?ita ?i ca a luat pedeapsa îndoita din mâna Domnului pentru pacatele sale".
Isa 40:3  Un glas striga: "În pustiu gati?i calea Domnului, drepte face?i în loc neumblat cararile Dumnezeului nostru.
Isa 40:4  Toata valea sa se umple ?i tot muntele ?i dealul sa se plece; ?i sa fie cele strâmbe, drepte ?i cele col?uroase, cai netede.
Isa 40:5  ?i se va arata slava Domnului ?i tot trupul o va vedea caci gura Domnului a grait".
Isa 40:6  Un glas zice: "Striga!" ?i eu zic: "Ce sa strig?" Tot trupul este ca iarba ?i toata marirea lui, ca floarea câmpului!
Isa 40:7  Se usuca iarba, floarea se ve?teje?te, ca Duhul Domnului a trecut pe deasupra. Poporul este ca iarba.
Isa 40:8  Iarba se usuca ?i floarea se ve?teje?te, dar cuvântul Dumnezeului nostru ramâne în veac!
Isa 40:9  Suie-te pe munte înalt, cel ce bineveste?ti Ierusalimului, ridica glasul tau cu putere, cel ce bineveste?ti Ierusalimului, înal?a glasul ?i nu te teme, zi ceta?ilor lui Iuda: "Iata Dumnezeul vostru!"
Isa 40:10  Ca Domnul Dumnezeu vine cu putere ?i bra?ul Lui supune tot. Iata ca pre?ul biruin?ei Lui este cu El ?i rodul izbânzii merge înaintea Lui.
Isa 40:11  El va pa?te turma Sa ca un Pastor ?i cu bra?ul Sau o va aduna. Pe miei îi va purta la sânul Sau ?i de cele ce alapteaza va avea grija.
Isa 40:12  Cine a masurat apele cu pumnul ?i cine a masurat pamântul cu cotul? Cine a pus pulberea pamântului în bani?a ?i cine a cântarit mun?ii ?i vaile cu cântarul?
Isa 40:13  Cine a cautat în adânc Duhul Domnului ?i cine L-a sfatuit pe El?
Isa 40:14  De la cine a luat El sfat ca sa judece bine ?i sa înve?e caile drepta?ii, sa înve?e ?tiin?a ?i calea în?elepciunii sa o cunoasca?
Isa 40:15  Iata, neamurile sunt ca o picatura de apa pe marginea unei gale?i, ca un fir de pulbere într-un cântar. Iata insulele care cântaresc cât un fir de praf.
Isa 40:16  Libanul nu ajunge pentru aprinderea focului ?i dobitoacele pentru arderi de tot.
Isa 40:17  Toate popoarele sunt ca o nimica înaintea Lui; ele pre?uiesc înaintea Lui cât o suflare.
Isa 40:18  Cu cine ve?i asemana voi pe Cel Preaputernic ?i unde ve?i gasi altul asemenea Lui?
Isa 40:19  Chipul cel turnat este turnat de un faurar, argintarul îl îmbraca cu aur ?i-l înfrumuse?eaza cu lan?i?oare.
Isa 40:20  Saracul, care nu poate oferi mult, alege un lemn care nu putreze?te; î?i cauta un me?ter iscusit ca sa faca un idol care sa nu se clatine.
Isa 40:21  Nu ?ti?i voi, oare, n-a?i auzit, nu vi s-a spus oare de la început, n-a?i în?eles voi ce va înva?a întemeierea lumii?
Isa 40:22  El sta în scaun deasupra cercului pamântului; pe locuitori îi vede ca pe lacuste; El întinde cerul ca un val u?or ?i îl desface ca un cort de locuit.
Isa 40:23  El preface în nimic pe capetenii; pe judecatorii pamântului îi nimice?te.
Isa 40:24  Abia sunt sadi?i, abia sunt semana?i, abia a prins radacini tulpina lor în pamânt; El sufla peste ele ?i le usuca ?i vijelia le spulbera ca pe pleava.
Isa 40:25  Cu cine Ma asemana?i voi ca sa-i fiu asemenea?, zice Sfântul.
Isa 40:26  Ridica?i ochii în sus ?i privi?i: Cine le-a zidit pe toate acestea? - Cel ce scoate o?tirea lor cu numar ?i pe toate pe nume le cheama! Celui Atotputernic ?i cu mare vârtute nici una nu-I scapa!
Isa 40:27  Pentru ce zici tu, Iacove, pentru ce graie?ti, Israele: "Calea mea este ascunsa Domnului, dreptul meu este trecut cu vederea de Dumnezeul meu?"
Isa 40:28  Nu ?tii tu, sau n-ai aflat tu ca Domnul este Dumnezeu ve?nic, Care a facut marginile pamântului, Care nu obose?te ?i nici nu Î?i sleie?te puterea? Ca în?elepciunea Lui este nemarginit de adânca?
Isa 40:29  El da tarie celui obosit ?i celui slab îi da putere mare.
Isa 40:30  Cei tineri se obosesc, î?i risipesc puterile ?i vitejii luptatori vor putea sa se clatine;
Isa 40:31  Dar cei ce nadajduiesc întru Domnul vor înnoi puterea  lor, le vor cre?te aripi ca ale vulturului; vor alerga ?i nu-?i vor slei puterea, vor merge ?i nu se vor obosi.
Isa 41:1  Tace?i înaintea Mea, ostroave, ?i asculta?i-Ma; popoarele sa-?i împrospateze puterea, sa vina mai lânga Mine ?i sa graiasca; apoi sa intram la judecata!
Isa 41:2  Cine a ridicat din Rasarit pe acela pe care biruin?a îl întâmpina pas cu pas? Cine i-a dat în stapânire neamuri ?i i-a supus regi? Cu sabia lui în pulbere îi preface, ?i cu arcul îi risipe?te ca pe pleava cea marunta.
Isa 41:3  El îi urmare?te ?i trece în pace pe cai pe unde n-au mai calcat picioarele lui.
Isa 41:4  Cine a facut aceasta ?i cine a pus-o la cale? Cel ce dintru început cheama neamurile; Eu, Domnul Care sunt cel dintâi ?i voi fi cu cei din urma.
Isa 41:5  Ostroavele Îl vad ?i sunt cuprinse de spaima, marginile pamântului tremura, se apropie, vin, intra la judecata!
Isa 41:6  Fiecare se ajuta unul pe altul ?i-?i zic: "Curaj!"
Isa 41:7  Turnatorul îmbarbateaza pe argintar ?i cel ce bate aurul, pe cel care bate pe nicovala, zicând: "Îmbinarea este buna". ?i ?intuie?te idolul în cuie ca sa nu se clatine.
Isa 41:8  Dar tu, Israele, sluga Mea, Iacove, pe care te-am ales, samân?a lui Avraam, iubitul Meu!...
Isa 41:9  Pe tine care te-am smuls din cele mai departate margini ale pamântului ?i te-am chemat din cele mai departate col?uri, ?i ?i-am zis: Tu e?ti robul Meu, pe tine te-am ales ?i nu te-am lepadat;
Isa 41:10  Nu te teme, ca Eu sunt cu tine, nu privi cu îngrijorare, ca Eu sunt Dumnezeul tau. Eu î?i dau tarie ?i te ocrotesc ?i dreapta Mea cea tare te va sprijini.
Isa 41:11  Iata ca se vor ru?ina ?i de ocara se vor face to?i cei ce sunt aprin?i împotriva ta; to?i vor fi nimici?i ?i vor pieri cei ce se fac vrajma?i ai tai!
Isa 41:12  Cauta-vei ?i nu vei gasi pe cei ce te urasc pe tine ?i ca o nimica vor fi cei ce vor sa se lupte cu tine.
Isa 41:13  Ca Eu sunt Dumnezeul tau, Eu întaresc dreapta ta ?i î?i zic ?ie: "Nu te teme, caci Eu sunt ajutorul tau!"
Isa 41:14  Nu-îi fie frica, vierme al lui Iacov, viermi?or al lui Israel, Eu sunt ajutorul tau, zice Domnul, Mântuitorul tau ?i Sfântul lui Israel.
Isa 41:15  Iata voi face din tine o grapa cu din?i, ascu?ita ?i noua. Vei merge peste mun?i ?i-i vei preface în pulbere ?i vaile în pleava marunta.
Isa 41:16  Tu le vei vântura, vântul le va lua ?i vijelia le va risipi. Iar tu te vei bucura întru Domnul ?i întru Sfântul lui Israel te vei preamari!
Isa 41:17  Cei saraci ?i lipsi?i cauta apa, dar nu o gasesc; limba lor este uscata de sete, Eu, Domnul lor, îi voi auzi; Eu, Dumnezeul lui Israel, nu-i voi parasi!
Isa 41:18  Pe dealuri înalte voi da drumul la râuri ?i la izvoare în mijlocul vailor, pustiul îl voi preface în iaz ?i pamântul uscat în pâraie de apa!
Isa 41:19  Sadi-voi în pustiu: cedri, salcâmi, mir?i ?i maslini ?i în lacuri neumblate: chiparo?i, platani ?i ienuperi laolalta,
Isa 41:20  Ca sa vada ?i sa-?i dea seama, sa cerceteze ?i sa priceapa cu to?ii ca mâna Domnului a facut acestea ?i ca Sfântul lui Israel le-a zidit!
Isa 41:21  Veni?i ?i va apara?i pricina voastra, zice Domnul; apropia?i-va cu dovezile voastre, zice regele lui Iacov.
Isa 41:22  Sa se apropie ?i sa ne spuna mai dinainte ceea ce va fi! Vremea cea straveche, a?a cum ne-au dat de ?tire, cu de-amanuntul o vom cerceta ?i viitorul pe care-l proorocesc vom vedea ce este.
Isa 41:23  Vesti]i cele ce vor fi în vremile mai de pe urma, ca sa ?tim ca sunte?i dumnezei! Haidem! Bine sau rau, face]i ceva ca sa ne putem încerca puterea!
Isa 41:24  Dar iata ca lucrarea voastra este nimic ?i nimic sânte]i ?i voi, urâciune este a va alege!
Isa 41:25  De la miazanoapte l-am chemat ca sa vina, de la rasarit l-am chemat pe nume. El a calcat în picioare pe satrapi ca pe noroi, cum calca olarul lutul.
Isa 41:26  Cine l-a descoperit odinioara ca sa-l ?tim ?i cu mult înainte ca sa zicem: "Este adevarat?" Dar nimeni n-a descoperit nimic, nimeni n-a vestit nimic ?i nimeni n-a auzit cuvintele voastre.
Isa 41:27  Eu Cel dintâi am zis Sionului: "Iata-i, iata-i!" ?i Ierusalimului am adus veste noua.
Isa 41:28  Privesc ?i nu este nimeni; printre ei nu se afla nici un profet. Eu îi întreb: "De unde vine el?" Dar ei nu raspund nimic!
Isa 41:29  Drept aceea, to?i sunt nimic, lucrarile lor de?ertaciune, idolii lor sunt vânare de vânt!
Isa 42:1  Iata Sluga Mea pe Care o sprijin, Alesul Meu, întru Care binevoie?te sufletul Meu. Pus-am peste El Duhul Meu ?i El va propovadui popoarelor legea Mea.
Isa 42:2  Nu va striga, nici nu va grai tare, ?i în pie?e nu se va auzi glasul Lui.
Isa 42:3  Trestia frânta nu o va zdrobi ?i fe?tila ce fumega nu o va stinge. El va propovadui legea Mea cu credincio?ie;
Isa 42:4  El nu va fi nici obosit, nici sleit de puteri, pâna ce nu va fi a?ezat legea pe pamânt; caci înva?atura Lui toate ?inuturile o a?teapta.
Isa 42:5  A?a graie?te Domnul cel Atotputernic, Care a facut cerurile ?i le-a întins, Care a întarit pamântul ?i cele de pe el, Care a dat suflare poporului de pe el ?i duh celor ce umbla pe întinsul lui:
Isa 42:6  "Eu, Domnul, Te-am chemat întru dreptatea Mea ?i Te-am luat de mâna ?i Te-am ocrotit ?i Te-am dat ca legamânt al poporului Meu, spre luminarea neamurilor;
Isa 42:7  Ca sa deschizi ochii celor orbi, sa sco?i din temni?a pe cei robi?i ?i din adâncul închisorii pe cei ce locuiesc întru întuneric.
Isa 42:8  Eu sunt Domnul ?i acesta este numele Meu. Nu voi da nimanui slava Mea ?i nici chipurilor cioplite cinstirea Mea".
Isa 42:9  Cele proorocite altadata s-au împlinit ?i altele mai noi va vestesc; înainte ca sa ia fiin?a vi le dau de ?tire.
Isa 42:10  Cânta?i Domnului cântare noua, cânta?i în strune laudele Lui pâna la marginile pamântului! Marea sa se zbuciume cu tot ce este în ea, ostroavele ?i locuitorii lor!
Isa 42:11  Pustiul ?i ceta?ile lui sa înal?e glas, ?i satele în care are sala? Chedar! Locuitorii din Sela sa chiuie de veselie; ?i din vârfurile mun?ilor sa strige de bucurie!
Isa 42:12  Sa preaslaveasca pe Domnul ?i lauda Lui s-o vesteasca în departatele ostroave.
Isa 42:13  Domnul iese ca un viteaz, ca un razboinic Î?i aprinde râvna Lui; striga puternic, un strigat de razboi. Împotriva vrajma?ilor Lui El lupta ca un viteaz!
Isa 42:14  Tacut-am multa vreme, stat-am lini?tit ?i mi-am stapânit tacerea; acum, ca o femeie care na?te, voi suspina, voi striga ?i voi rasufla!
Isa 42:15  Mun?ii îi voi pustii, la fel ?i dealurile, toata verdea]a lor o voi usca; pâraiele le voi preface în vai uscate ?i bal?ile fara apa le voi lasa.
Isa 42:16  Îndrepta-voi pe cei orbi pe drumuri pe care nu le cunosc, pe poteci ne?tiute îi voi pova?ui; întunericul îl voi preface înaintea lor în lumina ?i povârni?urile în câmpii întinse. Acestea sunt fagaduin?ele Mele pe care le voi împlini ?i cu vederea nu le voi trece.
Isa 42:17  Sa dea înapoi ?i sa se ru?ineze cei ce î?i pun nadejdea în idoli, cei ce zic chipurilor turnate: "Voi sunte?i dumnezeii no?tri!
Isa 42:18  Surzilor, auzi?i; orbilor, privi?i, vede?i!
Isa 42:19  Cine este orb, fara numai sluga Mea? Cine este surd ca trimisul Meu? Cine este orb ca cel de un neam cu Mine ?i surd ca Slujitorul Domnului?
Isa 42:20  Tu multe ai vazut fara sa te ui?i cu luare-aminte, urechile ti-au fost deschise, dar n-ai auzit.
Isa 42:21  Binevoit-a Domnul întru dreptatea Lui ca legea Lui s-o faca mare ?i marea?a.
Isa 42:22  Dar poporul Lui este jefuit ?i pustiit; închi?i to?i în pe?teri ?i ascun?i în temni?e. Prada?i au fost ?i nimeni nu i-a scapat, jefui?i ?i nimeni n-a zis: "Da?i înapoi!"
Isa 42:23  Cine dintre voi va pleca urechea la acestea, va fi cu luare-aminte ?i va asculta la cele ce vor sa fie?
Isa 42:24  Cine a dat pe Iacov jafului ?i pe Israel jefuitorilor? Oare nu Domnul, împotriva Caruia noi am pacatuit, ale Carui cai n-am voit sa le urmam ?i a Carui lege n-am ascultat-o?
Isa 42:25  El a varsat asupra lor iu?imea mâniei Lui ?i furiile razboiului. Vapaia i-a cuprins ?i n-au priceput; ar?i au fost ?i n-au luat seama.
Isa 43:1  ?i acum a?a zice Domnul, Ziditorul tau, Iacove, ?i Creatorul tau, Israele: "Nu te teme, caci Eu te-am rascumparat ?i te-am chemat pe nume, al Meu e?ti!
Isa 43:2  Daca tu vei trece prin ape, Eu sunt cu tine ?i în valuri tu nu vei fi înecat. Daca vei trece prin foc, nu vei fi ars ?i flacarile nu te vor mistui.
Isa 43:3  Ca Eu sunt Domnul Dumnezeul tau, Sfântul lui Israel, Mântuitorul. Eu dau Egiptul pre? de rascumparare pentru tine, Etiopia ?i Saba în locul tau;
Isa 43:4  Fiindca tu e?ti de pre? în ochii Mei ?i de cinste ?i te iubesc; voi da neamurile în locul tau ?i popoarele în locul sufletului tau.
Isa 43:5  Nu te teme, ca Eu sunt cu tine! De la rasarit voi aduce semin?ia ta ?i de la apus te voi strânge pe tine.
Isa 43:6  Voi zice catre miazanoapte: "Da-Mi-i" ?i catre miazazi: "Nu-i opri!" Aduce?i pe fiii Mei din ?inuturi departate ?i pe fiicele Mele de la marginile pamântului;
Isa 43:7  Pe to?i acei care poarta numele Meu ?i pentru slava Mea i-am creat, i-am zidit ?i i-am pregatit!
Isa 43:8  Sa vina poporul cel orb care are ochi ?i cel surd care are urechi!
Isa 43:9  Neamurile toate laolalta sa se adune ?i sa se strânga popoarele! Care dintre ele ne-au dat de ?tire aceasta ?i ne-au facut proorocii? Sa-?i aduca martorii ?i sa dovedeasca, sa auda to?i ?i sa zica: "Adevarat!"
Isa 43:10  Voi sunte?i martorii Mei, zice Domnul, ?i Sluga pe care am ales-o, ca sa ?ti?i, sa crede?i ?i sa pricepe?i ca Eu sunt: înainte de Mine n-a fost Dumnezeu ?i nici dupa Mine nu va mai fi!
Isa 43:11  Eu, Eu sunt Domnul ?i nu este izbavitor afara de Mine!
Isa 43:12  Eu sunt Cel ce am vestit, Cel ce am izbavit ?i Cel ce am prezis ?i nu sunt strain la voi. Voi sunte?i martorii Mei, zice Domnul.
Isa 43:13  Eu sunt Dumnezeu din ve?nicie ?i de aici încolo Eu sunt! Nimeni nu poate sa iasa de sub puterea Mea ?i ceea ce fac Eu, cine poate strica?"
Isa 43:14  A?a zice Domnul, Izbavitorul vostru, Sfântul lui Israel: "Pentru voi trimit prapad la Babilon, ca sa-i pun pe to?i pe fuga, pe ace?ti Caldei a?a de mândri pe corabiile lor.
Isa 43:15  Eu sunt Domnul, Sfântul vostru, Ziditorul lui Israel, Împaratul vostru!"
Isa 43:16  A?a zice Domnul, Cel ce croie?te drum pe mare ?i carare pe întinsele ape;
Isa 43:17  Cel care scoate carele de razboi ?i caii, o?tirea ?i capeteniile, ca sa se culce la pamânt ?i sa nu se mai scoale ?i sa se stinga ca o fe?tila de opai?;
Isa 43:18  Nu va mai aminti?i de întâmplarile trecute ?i nu mai lua?i în seama lucrurile de altadata".
Isa 43:19  Iata ca Eu fac un lucru nou, el da muguri; nu-l vede?i voi oare? Croi-voi în de?ert o cale, în loc uscat izvoare de apa.
Isa 43:20  Pe Tine Te vor preaslavi fiarele câmpului, ?acalii ?i stru?ii, ca Tu ai izvorât apa în pustiu, ?uvoaie de apa în pamânt neumblat, ca sa adapi pe poporul Meu cel ales;
Isa 43:21  Poporul pe care l-am facut pentru Mine, ca sa Ma preaslaveasca întru laude.
Isa 43:22  Dar tu nu M-ai chemat, Iacove, ?i tu nu te-ai ostenit pentru Mine, Israele!
Isa 43:23  Tu nu Mi-ai junghiat nici macar o oaie ca ardere de tot ?i cu jertfa sângeroasa tu nu M-ai preaslavit. Eu nu te-am suparat cerând prinoase ?i nu te-am împovarat cu jertfe de tamâie.
Isa 43:24  Tu n-ai cumparat pe bani miresme pentru Mine ?i de grasimea jertfelor tale tu nu M-ai saturat, ci M-ai cople?it cu pacatele tale ?i cu faradelegile tale tu M-ai chinuit.
Isa 43:25  Eu, Eu sunt Acel Care ?terge pacatele tale ?i nu t?i mai aduce aminte de faradelegile tale.
Isa 43:26  Adu-Mi aminte ca sa judecam împreuna, fa tu însu?i socoteala ca sa te dezvinova?e?ti:
Isa 43:27  Tatal tau dintâi a pacatuit ?i urma?ii tai ?i-au batut joc de Mine;
Isa 43:28  Capeteniile tale au pângarit altarul Meu. Pentru aceasta am dat pe Iacov pierzarii ?i pe Israel spre bataie de joc!"
Isa 44:1  ?i acum asculta Iacove, sluga Mea ?i Israele, pe care te-am ales!
Isa 44:2  A?a zice Domnul, Facatorul ?i Ziditorul tau din pântecele maicii tale, ?i Ocrotitorul tau: "Nu te teme, sluga Mea Iacov ?i tu Israele, pe care te-am ales.
Isa 44:3  Ca Eu voi varsa apa peste pamântul însetat ?i pâraie de apa în ?inut uscat. Varsa-voi din Duhul Meu peste odrasla ta ?i binecuvântarea Mea peste mladi?ele tale.
Isa 44:4  ?i vor odrasli ca iarba pe malul pâraielor ?i ca paji?tile de-a lungul apelor curgatoare!"
Isa 44:5  Unul va zice: "Eu sunt al Domnului!" Altul se va numi cu numele lui Iacov. Unul va scrie cu mâna lui: "Sunt al Domnului" ?i va vrea sa-?i dea numele de Israel!
Isa 44:6  A?a zise Domnul, Regele lui Israel ?i Izbavitorul sau, Domnul Savaot: "Eu sunt Cel dintâi ?i Cel de pe urma ?i nu este alt dumnezeu afara de Mine!
Isa 44:7  Cine este ca Mine sa vina lânga Mine, sa graiasca, sa prooroceasca ?i sa se masoare cu Mine! Cine a vestit de la început viitorul? Ceea ce se va întâmpla, cine poate sa le prevesteasca?
Isa 44:8  Nu va teme?i, nici nu va spaimânta?i! N-am aratat Eu odinioara ?i n-am vestit, când v-am luat pe voi de martori? Este oare un alt dumnezeu afara de Mine? Este un alt adapost ca Mine?
Isa 44:9  To?i facatorii de idoli nu sunt nimic ?i cele mai alese lucrari ale lor nu slujesc la nimic. Martorii lor nu vad nimic ?i, spre ru?inea lor, nici nu în?eleg nimic.
Isa 44:10  Cine a facut un dumnezeu ?i a turnat un idol fara sa traga un folos din aceasta?
Isa 44:11  Iata to?i cinstitorii lor se ru?ineaza; me?te?ugarii nu sunt decât oameni! Sa se adune to?i ?i sa se apropie! Ei tremura laolalta ?i simt mare ru?ine!
Isa 44:12  Fierarul ascute o dalta ?i da chip lucrului sau cu carbuni aprin?i. Alcatuie?te idolul cu lovituri de ciocan ?i-i da chip cu puterea bra?ului sau. Lui îi este foame ?i, sleit de puteri, el rabda de sete ?i este tare obosit.
Isa 44:13  Lemnarul întinde sfoara, face un semn cu plumbul. El lucreaza cu sculele lui ?i masoara cu compasul. El face lucrul lui dupa chipul unui om, dupa frumoasa înfa?i?are a unui pamântean, ca sa fie a?ezat într-o casa.
Isa 44:14  El ?i-a taiat un cedru, sau a luat chiparos sau stejar, pe care ?i-i alesese dintre copacii padurii, sau a plantat un cedru pe care ploaia l-a facut sa creasca.
Isa 44:15  Omul se sluje?te de ei pentru aprins focul ?i îi ia sa se încalzeasca. El îi arde ca sa coaca pâinea, ba mai mult, tot din el face ?i un dumnezeu ?i se închina la el, face un idol pe care îl cinste?te.
Isa 44:16  El a ars jumatate din lemne, a fript pe jeratic carne pe care o manânca ?i se satura. Se mai încalze?te ?i zice: "Mi-e cald! Simt vapaia lui!"
Isa 44:17  ?i cu ce a ramas, el face un dumnezeu, un idol pe care îl cinste?te ?i caruia i se închina ?i caruia se roaga zicând: "Izbave?te-ma, ca tu e?ti dumnezeul meu!"
Isa 44:18  Ei nu-?i dau seama ?i nici nu pricep ca ochii lor sunt închi?i ?i nu pot sa vada ?i inima lor este împietrita ?i nu pot sa în?eleaga.
Isa 44:19  Cu toate acestea el nu-?i face socoteala în inima sa, ca este simplu ?i fara pricepere, ?i nu zice: "Jumatate l-am pus pe foc ?i am copt pâine, pe carbuni am fript carne ?i am mâncat-o, iar cu cealalta jumatate care a mai ramas, voi face un idol urâcios ?i ma voi închina la un trunchi de copac".
Isa 44:20  El se hrane?te cu naluci, inima lui în?elata l-a dus la ratacire. El nu-?i mântuie?te sufletul sau ?i nu zice: "Oare ce am eu în mina mea nu este o momeala?"
Isa 44:21  Adu-?i aminte despre aceasta, Iacove, Israele, ca tu e?ti sluga Mea! Te-am facut sa-Mi fii Mie sluga, Israele, Eu nu te voi uita!
Isa 44:22  Risipit-am pacatele tale ca pe un nor ?i faradelegile tale ca pe o negura. Întoarce-te catre Mine, ca Eu te-am mântuit!
Isa 44:23  Cânta?i voi ceruri, ca Domnul a facut aceasta; rasuna?i adâncuri ale pamântului; mun?ilor, tresalta?i de bucurie, voi to?i copacii padurii, cânta?i, ca a rascumparat Domnul pe Iacov ?i în Israel ?i-a dat pe fa?a slava Sa!
Isa 44:24  A?a graie?te Domnul, Izbavitorul tau ?i Care te-a zidit din sânul maicii tale: "Eu sunt Domnul, Care a zidit lumea; singur am facut cerurile, Eu am întarit pamântul, ?i cine Mi-a fost într-ajutor?
Isa 44:25  Eu zadarnicesc semnele mincino?ilor ?i pe ghicitori îi fac sa fie nebuni, ru?inez pe cei în?elep?i ?i în?elepciunea lor o prefac în nebunie.
Isa 44:26  Eu sunt Domnul Care întare?te cuvântul slugilor Mele ?i împline?te sfatul trimi?ilor Mei. Eu am zis Ierusalimului: "Va fi locuit" ?i ceta?ilor lui Iuda: "Zidite vor fi". ?i Eu le voi ridica din darâmaturi!
Isa 44:27  ?i adâncului i-am zis: "Seaca!" Iata, î?i voi lasa râurile fara apa.
Isa 44:28  ?i am zis despre Cirus: "El este pastorul Meu, el va împlini toate voile Mele". ?i despre Ierusalim am zis: "Sa fie rezidit ?i templul sa fie ridicat din temelii!"
Isa 45:1  A?a zice Domnul unsului Sau Cirus, pe care îl ?ine de mâna lui cea dreapta, ca sa doboare neamurile înaintea lui ?i ca sa dezlege cingatorile regilor, sa deschida por?ile înaintea lui ?i ca ele sa nu mai fie închise:
Isa 45:2  "Eu voi merge înaintea ta ?i drumurile cele muntoase le voi netezi, voi zdrobi por?ile cele de arama ?i zavoarele cele de fier le voi sfarâma.
Isa 45:3  ?i î?i voi da ?ie vistierii ascunse, boga?ii îngropate în pamânt, ca sa ?tii ca Eu sunt Domnul Cel Care te-a chemat pe nume, Eu sunt Dumnezeul lui Israel.
Isa 45:4  Pentru sluga Mea Iacov ?i pentru Israel, alesul Meu, te-am chemat pe nume ?i un nume de cinste i-am dat fara ca tu sa Ma ?tii.
Isa 45:5  Eu sunt Domnul ?i nimeni altul! Afara de Mine nu este Dumnezeu. Eu te-am încins fara ca tu sa Ma cuno?ti.
Isa 45:6  Ca sa se ?tie de la rasarit ?i pâna la apus ca nu este nimic afara de Mine! Eu sunt Domnul ?i nimeni altul!
Isa 45:7  Eu întocmesc lumina ?i dau chip întunericului, Cel ce sala?luie?te pacea ?i restri?tei îi lasa cale: Eu sunt Domnul Care fac toate acestea.
Isa 45:8  Picura?i roua de sus, voi ceruri, ?i norii sa reverse în ploaie dreptatea! Pamântul sa se deschida ?i sa odrasleasca mântuirea ?i dreptatea sa dea mladi?e laolalta: Eu, Domnul, am zidit toate acestea!
Isa 45:9  Vai de cel ce se cearta cu Ziditorul sau, ciob printre hârburile de pamânt! Oare lutul zice olarului: "Ce faci tu?" ?i lucrul catre me?ter: "Tu nu e?ti iscusit!"
Isa 45:10  Vai de cel ce zice catre parinte: "Pentru ce dai na?tere?" ?i femeii: "Pentru ce ai copii?"
Isa 45:11  A?a zice Domnul, Sfântul lui Israel ?i Ziditorul sau: "Îndrazni?i voi oare sa Ma întreba?i despre cele viitoare ?i sa da?i porunca lucrului mâinilor Mele?
Isa 45:12  Eu am facut pamântul ?i omul de pe el Eu l-am zidit. Eu cu mâinile am întins cerurile ?i la toata o?tirea lor Eu îi dau porunca.
Isa 45:13  Eu l-am ridicat întru dreptatea Mea ?i toate caile lui le voi netezi. El va zidi cetatea Mea ?i va libera pe robii Mei, fara rascumparare ?i fara daruri", zice Domnul Savaot.
Isa 45:14  A?a zice Domnul: "Boga?iile Egiptului ?i câ?tigurile Etiopiei ?i ale Sabeenilor celor înal?i la stat vor trece la tine ?i ai tai vor fi; în lan?uri i?i vor sluji ?ie ?i vor cadea înaintea ta ?i rugându-se ?ie vor zice: "Numai tu ai un Dumnezeu tare, ?i nu este alt dumnezeu afara de El.
Isa 45:15  Cu adevarat Tu e?ti Dumnezeu ascuns, Dumnezeul lui Israel Cel izbavitor!
Isa 45:16  Cei care se aprindeau împotriva Ta vor fi ru?ina?i ?i umili?i, facatorii de idoli se vor face de râs.
Isa 45:17  Israel va fi izbavit de Domnul cu mântuire ve?nica. Voi nu ve?i fi ru?ina?i, nici umili?i în vecii vecilor!"
Isa 45:18  Ca a?a zice Domnul, Care a facut cerurile, Dumnezeu, Care a întocmit pamântul, l-a facut ?i l-a întarit; ?i nu în de?ert l-a facut, ci ca sa fie locuit: "Eu sunt Domnul ?i nu este altul!"
Isa 45:19  N-am grait acestea într-ascuns, undeva în vreun col? întunecos al pamântului; ?i n-am zis fara rost neamului lui Iacov: "Cauta?i-Ma!" Eu sunt Domnul Cel ce graie?te drept ?i spune adevarul!
Isa 45:20  Aduna?i-va, veni?i, apropia?i-va laolalta, cei rama?i cu via?a dintre neamuri! Nu î?i dau seama de nimic cei ce duc dupa ei un idol de lemn ?i se închina unui dumnezeu care nu poate izbavi!
Isa 45:21  Grai?i, apropia?i-va ?i sfatui?i-va unul cu altul! Cine a vestit aceasta, cine altadata a dat de ?tire? Oare nu Eu Domnul? Nu este alt dumnezeu afara de Mine. Dumnezeu drept ?i izbavitor nu este altul decât Mine!
Isa 45:22  Întoarce?i-va catre Mine ?i ve?i fi mântui?i, voi cei ce locui?i toate ?inuturile cele mai îndepartate ale pamântului! Ca Eu sunt Dumnezeu tare ?i nu este altul!
Isa 45:23  Am jurat pe Mine Însumi! Din gura Mea iese dreptatea ?i nu-Mi întorc cuvântul; înaintea Mea tot genunchiul se va pleca; pe Mine jura-va toata limba
Isa 45:24  ?i va zice: "Numai în Domnul este dreptatea ?i virtutea! Catre Dânsul vor veni ?i înfrunta?i vor fi cei ce sunt întarâta?i împotriva Lui.
Isa 45:25  Întru Domnul se vor îndrepta ?i va fi preaslavita toata semin?ia lui Israel!"
Isa 46:1  Bel se prabu?e?te, Nebo este rasturnat, chipurile lor a?ezate pe vite ?i pe dobitoace. Chipurile, pe care voi acum le purta?i, sunt încarcate ?i au ajuns o povara pentru vitele trudite.
Isa 46:2  Idolii cad, se prabu?esc laolalta, nu pot sa izbaveasca pe cei care îi poarta; ei în?i?i sunt du?i în robie.
Isa 46:3  "Asculta?i voi, cei din casa lui Iacov ?i to?i cei care a?i mai ramas din casa lui Israel, pe care v-am purtat din sânul maicii voastre, de care am avut grija de la na?terea voastra.
Isa 46:4  Pâna la batrâne?ea voastra Eu sunt Acela?i, pâna la adâncile voastre caruntele Eu va voi ocroti. Precum am facut în trecut, Ma leg înaintea voastra ca va voi ocroti ?i va voi izbavi ?i în viitor.
Isa 46:5  Cu cine Ma ve?i pune alaturi ?i Ma ve?i face egal, cu cine Ma ve?i asemana, ca sa fim deopotriva?
Isa 46:6  Ei scot aurul din pungile lor ?i argintul în cântar îl cântaresc; platesc un argintar ca sa le faca un chip de dumnezeu, apoi se închina lui ?i îl cinstesc.
Isa 46:7  Îl poarta pe umeri, îl duc, îl pun jos ?i el sta fara sa se clinteasca din locul sau. El nu raspunde celui care striga catre el ?i din primejdii nu-l scapa.
Isa 46:8  Aminti?i-va de aceasta ?i înva?a?i-va minte, pacato?ilor!
Isa 46:9  Aduce?i-va aminte de vremurile stravechi, de la obâr?ia lor, ca Eu sunt Dumnezeu ?i nu este un altul. Eu sunt Dumnezeu ?i nu este nimeni asemenea Mie!
Isa 46:10  De la început Eu vestesc sfâr?itul ?i mai dinainte ceea ce are sa se întâmple. ?i zic: Planul Meu va dainui ?i toata voia Mea o voi face!
Isa 46:11  De la rasarit chem o pasare de prada, dintr-un ?inut departat un om care sa împlineasca planul Meu. Am vorbit, voi împlini; am hotarât, voi înfaptui!
Isa 46:12  Asculta?i-Ma voi, oameni cu inima împietrita, voi cei care sta?i departe de mântuirea voastra!
Isa 46:13  Apropia-voi mântuirea Mea, caci ea nu este departe ?i izbavirea Mea nu va zabovi. Atunci voi pune mântuirea Mea în Sion ?i slava Mea pentru Israel!"
Isa 47:1  "Coboara-te ?i ?ezi în ?arâna, fecioara, fiica Babilonului, stai pe pamânt fara tron, fiica a Caldeilor, ca nu te va mai numi nimeni pe tine ginga?a ?i placuta!
Isa 47:2  Învârte?te la râ?ni?a ?i macina faina, da-?i la o parte valul tau, ridica-?i ve?mântul tau, ramâi cu picioarele goale ?i treci râurile!
Isa 47:3  Goliciunea ta sa se descopere, sa se vada ru?inea ta. Ma voi razbuna ?i nu voi cru?a pe nimeni",
Isa 47:4  Zice Izbavitorul nostru; Domnul Savaot este numele Lui, Sfântul lui Israel!
Isa 47:5  "Stai tacuta ?i mai la întuneric, fiica a Caldeilor, nimeni nu te va mai chema pe tine stapâna regatelor".
Isa 47:6  Mâniat am fost pe poporul Meu, pângarit-am mo?tenirea Mea ?i am dat-o în mâna ta. Dar tu n-ai avut mila ?i asupra batrânului ai apasat cu jug greu.
Isa 47:7  ?i tu î?i închipuiai: "Fi-voi pe veci stapâna!", dar niciodata n-ai cugetat ?i de sfâr?it nu ti-ai adus aminte!
Isa 47:8  ?i acum asculta, tu cea în placeri crescuta, care stapâneai fara de grija ?i ziceai în inima ta: "Nimeni alta nu este ca mine! Nu voi ramâne vaduva ?i nu voi ?ti ce este lipsa de copii!"
Isa 47:9  ?i aceste doua într-o clipa, în aceea?i zi, vor da peste tine: lipsa de copii ?i vaduvia; ?i te vor cople?i cu toata mul?imea ?i puterea fermecatoriilor ?i vrajitoriilor tale!
Isa 47:10  Întru faradelegile tale tu nadajduiai ?i ziceai: "Nimeni nu ma vede!" În?elepciunea ta ?i ?tiin?a ta te-au amagit astfel, ca ziceai în inima ta: "Eu ?i nimeni alta nu este ca mine!"
Isa 47:11  Drept aceea va veni peste tine o nenorocire pe care tu nu vei ?ti sa o înlaturi cu frumuse?ea ta ?i te va cople?i nenorocirea pe care tu nu o vei putea ocoli ?i pe nea?teptate va da peste tine pieirea, fara sa fi avut vreme s-o preveste?ti!
Isa 47:12  Pastreaza pentru tine fermecatoriile tale ?i vrajitoriile cu care te-ai trudit din tinere?e, poate î?i vor sluji, poate vei insufla temere!
Isa 47:13  Tu te-ai obosit întrebând pe atâ?ia sfatuitori! Sa iasa la iveala ?i sa te izbaveasca acei care masoara cerul ?i iscodesc stelele; care în fiecare luna noua spun ceea ce se va întâmpla.
Isa 47:14  Iata-i ca pleava pe care o mistuie focul, a?a vor ajunge ei ?i de puterea flacarilor via?a lor nu vor putea s-o scape caci nu va fi jeratic la care sa se încalzeasca, nici vatra ca sa stea dinaintea ei.
Isa 47:15  A?a se va întâmpla cu aceia pe care te-ai ostenit sa-i întrebi ?i cu care ai facut nego? din vremea tinere?ii tale. Fiecare va pleca la ale sale ?i nimeni nu va fi sa te scape".
Isa 48:1  Asculta?i aceasta, voi, cei din casa lui Iacov, care purta?i numele lui Israel, voi cei ie?i?i din samân?a lui Iuda, care va jura?i pe numele Domnului ?i va lauda?i cu Dumnezeul lui Israel, dar nu întru credincio?ie ?i dreptate.
Isa 48:2  Caci voi purta?i numele ceta?ii celei sfinte ?i va bizui?i pe Dumnezeul lui Israel, al Carui nume este Domnul Savaot.
Isa 48:3  "Vestit-am din vremuri stravechi cele ce aveau sa se întâmple; din gura Mea au ie?it ?i Eu le-am dat de ?tire; pe data le-am facut ?i ce s-au întâmplat,
Isa 48:4  Fiindca Eu ?tiu ca tu e?ti tare la cerbice ca un drug de fier ?i fruntea î?i este de arama.
Isa 48:5  ?i-am prezis acestea înainte ca sa se întâmple ?i auzite ?i le-am facut ca sa nu zici: "Idolul meu le-a facut, chipul cel cioplit ?i turnat le-a hotarât!"
Isa 48:6  Tu ai auzit; prive?te acum toate acestea! De ce nu marturise?ti? De aci înainte î?i voi împarta?i lucruri noi, ascunse, pe care nu le ?tiai.
Isa 48:7  Ele sunt zidite acum ?i nu de atunci; înainte de ziua aceasta tu n-ai auzit nimic despre ele ca sa nu zici: "Iata, eu le ?tiam!"
Isa 48:8  Nu, tu n-ai auzit ?i nici n-ai ?tiut, atunci urechea ta nu era deschisa; ca Eu ?tiu ca tu e?ti necredincios ?i ca din pântecele maicii tale ai fost numit razvratit.
Isa 48:9  Pentru numele Meu, Îmi opresc mânia ?i pentru slava Mea o potolesc, ca sa nu te nimicesc.
Isa 48:10  Iata ca te-am lamurit în foc ?i n-am aflat argint, te-am încercat în cuptorul nenorocirii.
Isa 48:11  Pentru Mine, ?i numai pentru Mine o fac; oare cum voi îngadui ca numele Meu sa fie pângarit? Nimanui nu voi da slava Mea!
Isa 48:12  Asculta, Iacove, ?i tu Israele, pe care te-am chemat. Eu sunt Cel dintâi ?i Cel de pe urma.
Isa 48:13  Mâna Mea a întemeiat pamântul ?i dreapta Mea a desfa?urat cerurile. Eu le chem ?i iata ele stau de fa?a.
Isa 48:14  Aduna?i-va to?i ?i asculta?i! Care din voi a prezis aceste lucruri? Cel pe care Domnul îl iube?te va împlini voia Lui împotriva Babilonului ?i împotriva semin?iei Caldeilor.
Isa 48:15  Eu, Eu am grait ?i l-am chemat, l-am adus ?i l-am facut sa propa?easca în calea lui.
Isa 48:16  Apropia?i-va de Mine ?i asculta?i acestea: De la început Eu n-am grait întru ascuns, de când se întâmpla aceste lucruri Eu sunt de fa?a". ?i acum, Domnul Dumnezeu ma trimite cu Duhul Sau!
Isa 48:17  A?a graie?te Domnul, Izbavitorul tau, Sfântul lui Israel: "Eu sunt Domnul Dumnezeul tau Care te înva?a spre folosul tau ?i te duce pe calea pe care trebuie sa mergi.
Isa 48:18  Daca ai fi luat aminte la poruncile Mele, fericirea ta ar fi fost asemenea unui râu ?i dreptatea ta ca valurile marii.
Isa 48:19  ?i va fi semin?ia ta ca nisipul marii ?i odraslele pântecelui tau ca pulberea pamântului. Nimic nu va nimici, nici nu va ?terge numele tau înaintea Mea!
Isa 48:20  Ie?i?i din Babilon, fugi?i din Caldeea cu cântece de veselie! Vesti?i, face?i cunoscuta ?tirea, duce?i-o pâna la marginile pamântului! Zice?i: Domnul rascumpara pe sluga Sa Iacov.
Isa 48:21  ?i nu vor suferi de sete în pustiul unde El îi duce; El le izvora?te apa din stânca. El despica stânca ?i apa ?â?ne?te!
Isa 48:22  Nu este pace, zice Domnul, pentru cei fara de lege!"
Isa 49:1  Asculta?i, ostroave, lua?i aminte, popoare departate! Domnul M-a chemat de la na?terea Mea, din pântecele maicii Mele Mi-a spus pe nume.
Isa 49:2  Facut-a din gura Mea sabie ascu?ita; ascunsu-M-a la umbra mâinii Sale. Facut-a din Mine sageata ascu?ita ?i în tolba Sa de o parte M-a pus,
Isa 49:3  ?i Mi-a zis Mie: "Tu e?ti sluga Mea, Israel, întru care Eu Ma voi preaslavi!"
Isa 49:4  Dar Eu Îmi spuneam: "În de?ert M-am trudit, în zadar ?i pentru nimic Mi-am prapadit puterea Mea!" Partea ce Mi se cuvine Mie este la Domnul ?i rasplata Mea la Dumnezeul Meu.
Isa 49:5  ?i acum Domnul Cel Care M-a zidit din pântecele maicii Mele ca sa-l slujesc Lui ?i sa întorc pe Iacov catre El ?i sa strâng la un loc pe Israel - caci a?a am fost Eu cinstit în ochii Domnului ?i Dumnezeul Meu fost-a puterea Mea, -
Isa 49:6  Mi-a zis: "Pu?in lucru este sa fii sluga Mea ca sa aduci la loc semin?iile lui Iacov ?i sa întorci pe cei ce-au scapat dintre ai lui Israel. Te voi face Lumina popoarelor ca sa duci mântuirea Mea pâna la marginile pamântului!"
Isa 49:7  A?a graie?te Rascumparatorul ?i Sfântul lui, Israel catre Cel dispre?uit ?i catre urâciunea neamurilor, Sluga tiranilor: "Regi Te vor vedea ?i se vor ridica, capetenii se vor închina pentru Domnul cel credincios ?i pentru Sfântul lui Israel, Cel Care Te-a ales!"
Isa 49:8  A?a graie?te Domnul: "În vremea milostivirii Te voi asculta ?i în vremea mântuirii Te voi ajuta. Te-am facut ?i Te-am hotarât Legamânt al poporului, ca sa a?ezi rânduiala în ?ara ?i sa dai fiecaruia mo?tenirile nimicite!"
Isa 49:9  Ca sa zici celor robi?i: "Ie?i?i!" ?i celor care sunt în întuneric: "Veni?i la lumina!" Ei vor pa?te oriunde pe calea lor ?i pe toate povârni?urile va fi pa?unea lor.
Isa 49:10  Nu le va fi nici foame, nici sete, soarele ?i vântul cel arzator nu-i va atinge, ca Cel Care se va milostivi de ei va fi Pova?uitorul lor ?i îi va îndrepta catre izvoare de apa.
Isa 49:11  Voi preface to?i mun?ii Mei în drumuri ?i cararile Mele vor fi bine gatite.
Isa 49:12  Iata ca unii vin din ?inuturi departate, de la miazanoapte, de la apus, iar al?ii din ?ara Sinim.
Isa 49:13  Salta?i, ceruri, de bucurie ?i tu, pamântule, bucura-te; mun?ilor, chioti?i de veselie, ca Domnul a mângâiat pe poporul Sau ?i de cei în necaz ai Lui S-a milostivit.
Isa 49:14  Sionul zicea: "Domnul m-a parasit ?i Stapânul meu m-a uitat!"
Isa 49:15  Oare femeia uita pe pruncul ei ?i de rodul pântecelui ei n-are ea mila? Chiar când ea îl va uita, Eu nu te voi uita pe tine.
Isa 49:16  Iata, te-am însemnat în palmele Mele; zidurile tale sunt totdeauna înaintea ochilor Mei!
Isa 49:17  Cei ce te vor ridica din ruini alearga catre tine ?i cei ce te-au pustiit fug departe de tine.
Isa 49:18  Ridica ochii tai de jur împrejur ?i vezi: to?i se aduna, to?i vin la tine. Viu sunt Eu, zice Domnul, tu te vei îmbraca întru ei ca într-un ve?mânt de podoaba ?i te vei încinge cu ei ca o mireasa.
Isa 49:19  Caci locurile tale pustii, ruinele tale ?i ?ara ta pustiita vor fi prea strâmte pentru locuitorii tai, iar pustiitorii tai vor fi departe.
Isa 49:20  ?i vor mai grai la urechile tale feciorii tai de care tu erai lipsita: "?inutul este prea strâmt pentru mine, fa-mi loc sa stau ?i eu!"
Isa 49:21  Atunci tu vei zice în inima ta: "Cine mi i-a nascut pe ace?tia? Pierdusem copiii mei ?i eram stearpa, dusa în robie ?i gonita; dar pe ace?tia cine i-a nascut? Iata ca ramasesem singura! Dar ace?tia de unde vin?"
Isa 49:22  A?a zice Domnul Dumnezeu: "Iata voi ridica mâna Mea catre neamuri ?i catre popoare voi înal?a steagul meu. Ele vor aduce pe feciorii tai pe bra?e ?i pe fiicele tale pe umeri le vor purta.
Isa 49:23  Regi te vor cre?te ?i prin?ese te vor alapta. Cu fa?a la pamânt se vor închina înaintea ta ?i vor linge pulberea de pe picioarele tale. Atunci tu vei ?ti ca Eu sunt Domnul, Care nu ru?ineaza pe cei ce î?i pun nadejdea în El!
Isa 49:24  Oare poate sa i se ia celui viteaz prada ?i celui puternic sa i se smulga din mâna cei robiri?
Isa 49:25  Da! zice Domnul: "Chiar robii unui viteaz i se vor lua, prada unui razboinic îi va scapa; Eu Ma voi razboi cu potrivnicii tai ?i pe fiii tai Eu îi voi scapa!
Isa 49:26  ?i pe asupritorii tai îi voi face sa-?i manânce carnea lor ?i sa se îmbete de sângele lor ca de vin. Atunci toata faptura va ?ti ca Eu sunt Domnul, Mântuitorul tau ?i Rascumparatorul tau, viteazul lui Iacov!"
Isa 50:1  A?a zice Domnul: Unde este cartea de despar?ire cu care am alungat pe mama voastra? Sau care este datornicul Meu, caruia Eu v-am vândut? Pentru ca numai pentru faradelegile voastre a?i fost vându?i ?i pentru pacatele voastre am alungat pe mama voastra.
Isa 50:2  Pentru ce când veneam nu gaseam pe nimeni ?i când strigam nimeni nu raspundea? Oare mâna Mea este prea scurta, ca sa rascumpere, sau nu am destula putere, ca sa izbavesc? Prin certarea Mea sec marea ?i râurile le prefac în pustiu; pe?tii din ele mor, ca nu mai este apa ?i se sfâr?esc de sete.
Isa 50:3  Eu îmbrac cerul cu zabranic ?i îl acopar cu un ve?mânt de jale.
Isa 50:4  Domnul Dumnezeu Mi-a dat Mie limba de ucenic, ca sa ?tiu sa graiesc celor deznadajdui?i. În fiecare diminea?a El de?teapta, treze?te urechea Mea, ca sa ascult ca un ucenic.
Isa 50:5  Domnul Dumnezeu Mi-a deschis urechea, dar Eu nu M-am împotrivit ?i nici nu M-am dat înapoi.
Isa 50:6  Spatele l-am dat spre batai ?i obrajii mei spre palmuiri, ?i fa?a Mea nu am ferit-o de ru?inea scuiparilor.
Isa 50:7  ?i Domnul Dumnezeu Mi-a venit în ajutor ?i n-am fost facut de ocara. De aceea am ?i întarit fa?a Mea ca o cremene, caci ?tiam ca nu voi fi facut de ocara.
Isa 50:8  Aparatorul Meu este aproape. Cine se judeca cu Mine? Sa ne masuram împreuna! Cine este potrivnicul Meu? Sa se apropie!
Isa 50:9  Iata, Domnul Dumnezeu Îmi este întru ajutor; cine Ma va osândi? Iata, ca un ve?mânt vechi to?i se vor prapadi ?i molia îi va mânca!
Isa 50:10  Cine din voi se teme de Domnul sa asculte glasul Slugii Sale! Cel care umbla în întuneric ?i fara lumina sa nadajduiasca întru numele Domnului ?i sa se bizuie pe Dumnezeul lui!
Isa 50:11  Voi to?i, care aprinde?i focul ?i pregati?i sage?i arzatoare, arunca?i-va în focul sage?ilor voastre pe care l-a?i aprins! Din mâna Mea vi se întâmpla una ca aceasta; pe patul durerii ve?i fi culca?i!
Isa 51:1  Asculta?i-Ma pe Mine, voi care umbla?i dupa dreptate, voi care cauta?i pe Domnul! Privi?i la stânca din care a?i fost taia?i ?i catre cariera de piatra din care a?i fost sco?i.
Isa 51:2  Privi?i pe Avraam, tatal vostru, ?i la Sarra cea care în dureri v-a nascut. Ca pe el singur l-am chemat, l-am binecuvântat ?i l-am înmul?it.
Isa 51:3  Iar Domnul va mângâia Sionul, ?i darâmaturilor lui le va da nadejde. El va preface pustiul lui în rai ?i pamântul lui neroditor în gradina Domnului; bucurie ?i veselie va fi acolo, mul?umiri ?i cântari de lauda!
Isa 51:4  Ia aminte la Mine, poporul Meu, ?i voi, neamuri, fi?i cu urechea la Mine, ca de la Mine va veni înva?atura ?i legea Mea va fi lumina popoarelor.
Isa 51:5  Dreptatea Mea este aproape, vine mântuirea Mea ?i bra?ul Meu va face dreptate popoarelor, întru Mine vor nadajdui ?inuturile cele departate, ca de la bra?ul Meu a?teapta scaparea.
Isa 51:6  Ridica?i la ceruri ochii vo?tri ?i privi?i jos pamântul; cerurile vor trece ca un fum ?i pamântul ca o haina se va învechi; locuitorii vor muri ca mu?tele, mântuirea Mea va dainui în veac ?i în veac ?i dreptatea Mea nu va avea sfâr?it.
Isa 51:7  Asculta?i-Ma pe Mine, voi, cunoscatori ai drepta?ii, popor care e?ti cu legea Mea în inima! Nu te teme de ocara oamenilor ?i de batjocura lor sa nu te înfrico?ezi. La fel ca pe un ve?mânt îi va mânca molia ?i ca pe lâna viermii îi vor mistui.
Isa 51:8  Dar dreptatea Mea va ramâne în veac ?i mântuirea Mea din neam în neam.
Isa 51:9  Ridica-te, scoala-te, îmbraca-te cu tarie, bra? al Domnului! Înal?a-te ca odinioara, ca în veacurile trecute! N-ai zdrobit Tu pe Rahab ?i n-ai spintecat Tu balaurul?
Isa 51:10  Nu e?ti Tu, oare, Cel ce ai secat marea ?i apele adâncului celui fara fund, Cel ce adâncimile marii le-ai prefacut în carare larga pentru cei rascumpara?i ai Tai?
Isa 51:11  ?i astfel cei mântui?i ai Domnului se vor întoarce ?i vor veni în Sion, în cântari de biruin?a ?i o bucurie ve?nica va încununa capul lor. Bucuria ?i veselia vor veni peste ei, iar durerea, întristarea ?i suspinarea se vor departa de la ei.
Isa 51:12  Eu, Eu sunt Cel ce da nadejde! Cine e?ti tu, ca sa te temi de un muritor ?i de un om de rând care trece ca iarba?
Isa 51:13  ?i sa dai uitarii pe Domnul, Ziditorul tau, Care a întins cerurile ?i a întemeiat pamântul? Sa te înfrico?ezi mereu, în fiecare zi, de urgia asupritorului care umbla sa te piarda? Unde este oare urgia asupritorului?
Isa 51:14  Curând cel ferecat în catu?e va fi dezlegat ?i nu va muri în temni?a ?i de pâine nu va duce lipsa.
Isa 51:15  Eu sunt Domnul Dumnezeul Care stârnesc marea ?i face sa mugeasca valurile ei; Domnul Savaot este numele Lui.
Isa 51:16  Pune-voi cuvintele Mele în, gura ta ?i la umbra mâinii Mele te voi acoperi, ca sa întind cerurile, sa întemeiez pamântul ?i sa zic Sionului: "Tu e?ti poporul Meu!"
Isa 51:17  Treze?te-te, treze?te-te, scoala-te, Ierusalime, tu care ai baut din mâna Domnului paharul urgiei Lui; potirul ame?elii l-ai baut ?i l-ai sorbit.
Isa 51:18  Nici unul din to?i copiii pe care i-a nascut nu este care sa-l fi calauzit. Nimeni nu l-a ?inut de mâna din to?i feciorii pe care i-a crescut!
Isa 51:19  Aceste doua nenorociri te-au lovit: Cine te va plânge? Pustiirea ?i darâmarea, foametea ?i sabia. Cine te va mângâia?
Isa 51:20  Feciorii tai zac fara vlaga în col?urile uli?elor, ca o antilopa prinsa în cursa, be?i de urgia Domnului, de certarea Dumnezeului tau.
Isa 51:21  Drept aceea, ia aminte, sarmana cetate, ame?ita, dar nu de vin!
Isa 51:22  A?a graie?te Stapânul, Domnul Dumnezeul tau, Care Se lupta pentru poporul Sau: "Iata Eu iau din mâna ta paharul ame?elii, cupa mâniei Mele, ?i tu nu o vei mai bea!
Isa 51:23  ?i o voi pune în mâna asupritorilor tai, în mâna celor ce te-au supus, care î?i ziceau: Pleaca-te la pamânt ca sa trecem peste tine! ?i tu faceai spatele tau ca un pamânt ?i ca o cale pentru trecatori!"
Isa 52:1  Treze?te-te, treze?te-te, îmbraca-te cu puterea ta, Sioane, înve?mânteaza-te în haine de sarbatoare, Ierusalime, cetate sfânta! Ca nu va mai intra în tine cel netaiat împrejur ?i cel necurat!
Isa 52:2  Scutura-te de pulbere, scoala-te, Ierusalime robit, dezleaga funiile de pe grumazul tau, robita fiica a Sionului?
Isa 52:3  Caci iata ce spune Domnul: "Fara pre? a?i fost vându?i ?i fara argint ve?i fi rascumpara?i".
Isa 52:4  Ca a?a zice Domnul Dumnezeu: "Poporul Meu a coborât odinioara în Egipt ca sa aiba sala?, apoi Asiria l-a împilat fara cuvânt.
Isa 52:5  ?i acum ce sa fac Eu, zice Domnul, când poporul Meu a fost luat pe nedrept? Stapânitorii lui striga în semn de biruin?a, zice Domnul, iar numele Meu, mereu, cât ?ine ziua, este defaimat.
Isa 52:6  Drept aceea poporul va cunoa?te numele Meu, el va în?elege în ziua aceea ca Eu sunt Cel Care graie?te: Iata-Ma!"
Isa 52:7  Cât de frumoase sunt pe mun?i picioarele trimisului care veste?te pacea, a solului de veste buna, care da de ?tire mântuirea, care zice Sionului: Dumnezeul tau este împarat!
Isa 52:8  To?i strajerii tai ridica glas ?i laolalta striga de bucurie, ca ei vad cu ochii când Domnul Se întoarce în Sion.
Isa 52:9  Izbucni?i în chiote de veselie, darâmaturi ale Ierusalimului, ca Domnul mângâie pe poporul Sau, rascumparat-a Ierusalimul.
Isa 52:10  Descoperit-a Domnul bra?ul Sau cel sfânt în ochii tuturor popoarelor ?i toate marginile cele îndepartate ale pamântului vor vedea mântuirea Dumnezeului nostru, zicând:
Isa 52:11  "Pleca?i, pleca?i, ie?i?i de acolo! Nu va atinge?i de lucru spurcat! Ie?i?i, cura?i?i-va, voi cei care purta?i vasele Domnului!
Isa 52:12  Dar nu ve?i ie?i îngramadindu-va ?i nu ve?i pleca fugind, ca înaintea voastra merge Domnul ?i în urma voastra tot El, Dumnezeul lui Israel!"
Isa 52:13  Iata ca Sluga Mea va propa?i, Se va sui, mare Se va face ?i Se va înal?a pe culmile slavei!
Isa 52:14  Precum mul?i s-au spaimântat de El - a?a de schimonosita li era înfa?i?area Lui, ?i chipul Lui atât de fara asemanare cu oamenii, -
Isa 52:15  Tot a?a va fi pricina de uimire pentru multe popoare; înaintea Lui regii vor închide gura, ca acum vad ceea ce nu li s-a spus, ?i în?eleg ceea ce n-au auzit.
Isa 53:1  Cine va crede ceea ce noi am auzit ?i bra?ul Domnului cui se va descoperi?
Isa 53:2  Crescut-a înaintea Lui ca o odrasla, ?i ca o radacina în pamânt uscat; nu avea nici chip, nici frumuse?e, ca sa ne uitam la El, ?i nici o înfa?i?are, ca sa ne fie drag.
Isa 53:3  Dispre?uit era ?i cel din urma dintre oameni; om al durerilor ?i cunoscator al suferin?ei, unul înaintea caruia sa-?i acoperi fa?a; dispre?uit ?i nebagat în seama.
Isa 53:4  Dar El a luat asupra-?i durerile noastre ?i cu suferin?ele noastre S-a împovarat. ?i noi Îl socoteam pedepsit, batut ?i chinuit de Dumnezeu,
Isa 53:5  Dar El fusese strapuns pentru pacatele noastre ?i zdrobit pentru faradelegile noastre. El a fost pedepsit pentru mântuirea noastra ?i prin ranile Lui noi to?i ne-am vindecat.
Isa 53:6  To?i umblam rataci?i ca ni?te oi, fiecare pe calea noastra, ?i Domnul a facut sa cada asupra Lui faradelegile noastre ale tuturor.
Isa 53:7  Chinuit a fost, dar S-a supus ?i nu ?i-a deschis gura Sa; ca un miel spre junghiere s-a adus ?i ca o oaie fara de glas înaintea celor ce o tund, a?a nu ?i-a deschis gura Sa.
Isa 53:8  Întru smerenia Lui judecata Lui s-a ridicat ?i neamul Lui cine îl va spune? Ca s-a luat de pe pamânt via?a Lui! Pentru faradelegile poporului Meu a fost adus spre moarte.
Isa 53:9  Mormântul Lui a fost pus lânga cei fara de lege ?i cu cei facatori de rele, dupa moartea Lui, cu toate ca nu savâr?ise nici o nedreptate ?i nici în?elaciune nu fusese în gura Lui.
Isa 53:10  Dar a fost voia Domnului sa-L zdrobeasca prin suferin?a. ?i fiindca ?i-a dat via?a ca jertfa pentru pacat, va vedea pe urma?ii Sai, î?i va lungi via?a ?i lucrul Domnului în mâna Lui va propa?i.
Isa 53:11  Scapat de chinurile sufletului Sau, va vedea rodul ostenelilor Sale ?i de mul?umire Se va satura. Prin suferin?ele Lui, Dreptul, Sluga Mea, va îndrepta pe mul?i, ?i faradelegile lor le va lua asupra Sa.
Isa 53:12  Pentru aceasta Îi voi da partea Sa printre cei mari ?i cu cei puternici va împar?i prada, ca rasplata ca ?i-a dat sufletul Sau spre moarte ?i cu cei facatori de rele a fost numarat. Ca El a purtat faradelegile multora ?i pentru cei pacato?i ?i-a dat via?a.
Isa 54:1  Vesele?te-te, cea stearpa, care nu na?teai, da glas ?i striga tu care nu te-ai zvârcolit în dureri de na?tere, caci mai mul?i sunt fiii celei parasite, decât ai celei cu barbat, zice Domnul.
Isa 54:2  Large?te locul cortului tau ?i acoperamântul sala?ului tau întinde-l, nu cru?a nimic! Lunge?te funiile ?i întare?te ?aru?ii!
Isa 54:3  Caci tu te vei la?i la dreapta ?i la stânga ?i semin?ia ta va cuceri neamurile ?i ceta?ile cele pustiite le va umple de oameni.
Isa 54:4  Nu te înfrico?a, caci nu vei ramâne de ocara; nu te ru?ina, caci nu vei avea de ce sa te ru?inezi; ca tu vei uita ru?inea tinere?ii tale ?i de ocara vaduviei tale nu-?i vei mai aduce aminte.
Isa 54:5  Caci barbatul tau este Facatorul tau, ?i numele Lui: Domnul Savaot ?i Rascumparatorul tau este Sfântul lui Israel: "Dumnezeul a tot pamântul" se cheama!
Isa 54:6  Ca pe o femeie parasita ?i cu inima întristata te cheama Domnul; ca pe so?ia din tinere?e care a fost alungata; zice Dumnezeul tau.
Isa 54:7  O clipa te-am parasit, dar cu mari îndurari te iau lânga Mine.
Isa 54:8  Într-o izbucnire de mânie, pentru o clipa Mi-am întors fa?a de la tine, dar în îndurarea Mea cea ve?nica Ma voi milostivi de tine, zice Rascumparatorul tau, Domnul.
Isa 54:9  ?i va fi ca în vremea lui Noe, când M-am jurat ca apele potopului nu se vor mai raspândi pe pamânt; tot a?a Ma jur acum sa nu Ma mai mânii împotriva ta ?i sa nu te mai cert.
Isa 54:10  Mun?ii pot sa se mute din loc ?i colinele sa se clatine, dar milostivirea Mea nu se va departa de la tine ?i legamântul Meu de pace nu se va zdruncina, zice Domnul, Care are mila de tine.
Isa 54:11  Sarmana, lovita de vijelie ?i fara mângâiere! Iata, zidurile tale le voi împodobi cu pietre scumpe ?i voi pune temelia ta pe safire.
Isa 54:12  ?i-?i voi face crestele zidurilor de rubin ?i por?ile tale de cristal, iar împrejmuirea de pietre nestemate.
Isa 54:13  To?i copiii tai vor fi ucenici ai Domnului ?i se vor bucura de mare fericire.
Isa 54:14  ?i vei fi întemeiata pe dreptate: departeaza silnicia, caci nu ai de ce sa te mai temi; leapada ?i groaza, caci nu se va mai apropia de tine.
Isa 54:15  Daca cineva va mai da navala, nu este pornita de la Mine, ?i cine se har?uie?te cu tine va cadea în lupta împotriva ta!
Isa 54:16  Iata Eu am facut pe me?terul, care sufla în focul de carbuni ?i faure?te arma cu me?te?ugul lui, dar Eu am lasat ?i pe cel ce trebuie s-o nimiceasca.
Isa 54:17  Orice arma faurita împotriva ta nu va izbuti ?i orice limba oare se ridica la judecata cu tine osândita va fi. Aceasta este mo?tenirea slugilor Domnului ?i dreptatea care vine de la Mine, zice Domnul.
Isa 55:1  Cei ce sunte?i înseta?i, merge?i la apa, ?i cei care nu ave?i argint, merge?i de cumpara?i ?i mânca?i, merge?i ?i cumpara?i fara de argint ?i fara pre? vin ?i grasime.
Isa 55:2  Pentru ce cheltui?i argintul vostru pentru un lucru care nu hrane?te ?i câ?tigul muncii voastre pentru ceva care nu va satura? Asculta?i-Ma pe Mine ?i ve?i mânca cele bune ?i întru bunata?i se va desfata sufletul vostru.
Isa 55:3  Lua?i aminte cu urechile voastre ?i merge?i pe caile Mele. Asculta?i-Ma pe Mine ?i viu va fi sufletul vostru. Voi face cu voi legamânt ve?nic, dându-va îndurarile Mele cele fagaduite lui David.
Isa 55:4  Iata l-am facut marturie popoarelor, capetenie ?i stapânitor peste neamuri.
Isa 55:5  Iata, tu vei chema popoare pe care nu le ?tii ?i popoare care pe tine nu te-au cunoscut vor alerga la tine, pentru Domnul Dumnezeul tau ?i pentru Sfântul lui Israel, caci El te preamare?te.
Isa 55:6  Cauta?i pe Domnul cât Îl pute?i gasi, striga?i catre Dânsul cât El este aproape de voi.
Isa 55:7  Cel rau sa lase calea lui ?i omul cel nelegiuit vicleniile lui ?i sa se întoarca spre Domnul, caci El Se va milostivi de dânsul, ?i catre Dumnezeul nostru cel mult iertator.
Isa 55:8  Caci gândurile Mele nu sunt ca gândurile voastre ?i caile Mele ca ale voastre, zice Domnul.
Isa 55:9  ?i cât de departe sunt cerurile de la pamânt, a?a de departe sunt caile Mele de caile voastre ?i cugetele Mele de cugetele voastre.
Isa 55:10  Precum se coboara ploaia ?i zapada din cer ?i nu se mai întoarce pâna nu adapa pamântul ?i-l face de rasare ?i rode?te ?i da samân?a semanatorului ?i pâine spre mâncare,
Isa 55:11  Asa va fi cuvântul Meu care iese din gura Mea; el nu se întoarce catre Mine fara sa dea rod, ci el face voia Mea ?i î?i îndepline?te rostul lui.
Isa 55:12  ?i voi cu veselie ve?i ie?i ?i în pace ve?i fi calauzi?i; mun?ii ?i colinele vor izbucni în strigate de veselie înaintea voastra ?i to?i copacii câmpului vor bate din palme!
Isa 55:13  În locul spinilor va cre?te Chiparosul ?i în locul urzicii va cre?te mirtul. A Domnului fi-va slava, spre ve?nica ?i nepieritoare pomenire.
Isa 56:1  A?a zice Domnul: "Pazi?i dreptatea ?i face?i lucruri drepte, ca în curând va veni mântuirea Mea ?i dreptatea Mea se va descoperi.
Isa 56:2  Fericit este omul care savâr?e?te acestea ?i care ?ine la ele: Paze?te ziua de odihna ca sa nu fie pângarita ?i î?i fere?te mâna lui ca sa nu faptuiasca nici un rau.
Isa 56:3  ?i sa nu zica cel de alt neam, care s-a alipit de Domnul: "Domnul ma va despar?i de poporul Sau!" ?i famenul sa nu zica: "Iata eu sunt un copac uscat!"
Isa 56:4  Pentru ca a?a zice Domnul catre fameni: Celor care pazesc zilele Mele de odihna ?i aleg ceea ce Îmi este placut Mie ?i staruie în legamântul Meu,
Isa 56:5  Le voi da în casa Mea ?i înauntrul zidurilor Mele un nume ?i un loc mai de pre? decât fii ?i fiice; le voi da un nume ve?nic ?i nepieritor.
Isa 56:6  ?i pentru strainii alipi?i de Domnul ca sa slujeasca ?i sa iubeasca numele Domnului ?i sa fie slujitorii Sai, to?i câ?i pazesc ziua de odihna ca sa nu fie pângarita ?i staruie în legamântul Meu,
Isa 56:7  Pe ace?tia îi voi aduce în muntele cel sfânt al Meu ?i îi voi bucura în loca?ul Meu de rugaciune. Arderile lor de tot ?i jertfele lor vor fi primite pe altarul Meu; caci templul Meu, loca? de rugaciune se va chema pentru toate popoarele!"
Isa 56:8  Acestea sunt zisele Domnului, Care aduna pe cei risipi?i ai lui Israel: "La cei aduna?i voi mai aduna ?i al?ii!"
Isa 56:9  Fiare ale câmpului, veni?i, mânca?i ?i voi, toate animalele padurii!
Isa 56:10  Strajerii Mei sunt orbi cu to?ii, ei nu în?eleg nimic. To?i sunt câini mu?i care nu pot sa latre. Ei viseaza, stau tolani?i ?i le place sa doarma.
Isa 56:11  Ace?tia sunt câini hrapare?i care nu se mai satura; sunt pastorii care nu pricep nimic. To?i umbla în caile lor ?i se silesc pentru câ?tigul lor.
Isa 56:12  "Veni?i, zic ei, voi aduce vin, bea-vom bauturi îmbatatoare! ?i mâine va fi, ca astazi, mare zi de veselie".
Isa 57:1  Dreptul piere ?i nimeni nu ia aminte; se duc oamenii cinsti?i ?t nimanui nu-i pasa ca din pricina rauta?ii a pierit cel drept.
Isa 57:2  El intra în pace în groapa. Cel care umbla pe calea cea dreapta se odihne?te în sala?urile sale.
Isa 57:3  Dar voi, feciori de vrajitoare, neam de strica?i ?i desfrâna?i, apropia?i-va.
Isa 57:4  De cine va bate?i joc? La cine va strâmba?i ?i scoate?i limba? Nu sunte?i voi copii pacato?i, neam de mincino?i?
Isa 57:5  Arde?i de pofta pe lânga idolii de sub orice copac verde ?i jertfi?i pe fii în albia râurilor ?i în pe?teri.
Isa 57:6  Pietrele cele lucioase ale râurilor sunt partea ta! Iata, iata sor?ul tau! Lor le aduci jertfa cu turnare ?i prinoase! Pot Eu sa fiu mul?umit de aceasta?
Isa 57:7  Pe un munte înalt ?i ridicat î?i a?ezi patul tau ?i acolo te urci ca sa aduci jertfa ta!
Isa 57:8  Dupa u?a, în dosul u?orilor, ai pus amintirea ta; ?i departe de Mine tu desfaci patul tau, te urci ?i îl mai large?ti, faci legamânt cu ei, î?i place sa te culci cu ei...
Isa 57:9  Tu alergi dupa Melec cu untdelemn ?i cu miresme multe; tu trimi?i solii tai departe, ?i te cobori pâna la locuin?a mor?ilor.
Isa 57:10  Calatoria ta cea lunga te obose?te, dar nu zici: "Nu-mi mai trebuie!" Tu gase?ti insa puteri noi, pentru aceasta tu nu te dai batut!
Isa 57:11  De cine î?i era frica? De cine te temeai ca sa Ma mânii pe Mine, sa nu-?i mai aduci aminte ?i nici sa nu-?i mai pese? Fiindca n-am deschis gura ?i am închis ochii, tu nu te-ai temut de Mine!
Isa 57:12  Eu î?i voi face ?tiuta dreptatea ta, caci lucrurile tale nu slujesc la nimic.
Isa 57:13  Când tu vei striga, sa te izbaveasca idolii tai! Pe to?i îi va duce vântul ?i o suflare îi va face nevazu?i! Dar cel care î?i pune nadejdea în Mine va mo?teni pamântul ?i va stapâni în muntele cel sfânt.
Isa 57:14  ?i li se va zice: Gati?i, gati?i, face?i drum, da?i la o parte orice piedica din calea poporului Meu.
Isa 57:15  Ca a?a zice Domnul, a Carui locuin?a este ve?nica ?i al Carui nume este sfânt: Sala?luiesc într-un loc înalt ?i sfânt ?i sunt cu cei smeri?i ?i înfrân?i, ca sa înviorez pe cei cu duhul umilit ?i sa îmbarbatez pe cei cu inima frânta.
Isa 57:16  Caci nu vreau sa cert totdeauna ?i sa starui în mânie, caci înaintea Mea ar cadea în nesim?ire duhul ?i sufletele pe care le-am creat.
Isa 57:17  Pentru faradelegea sa, M-am întarâtat o clipa ?i, stând ascuns, l-am lovit întru mânia Mea. ?i el, razvratit, mergea pe calea inimii sale!
Isa 57:18  Am vazut caile sale ?i îl voi vindeca, îl voi pova?ui, îl voi odihni ?i îl voi mângâia.
Isa 57:19  ?i cei care îl jeleau vor izbucni în cântari de mul?umire. Pace, pace celor de aproape ?i celor de departe, zice Domnul, ?i Eu îl voi tamadui.
Isa 57:20  Cei fara de lege sunt ca marea cea înviforata, care nu se poate astâmpara ?i valurile ei scormonesc tina ?i namol.
Isa 57:21  Cei fara de lege n-au pace, zice Domnul.
Isa 58:1  Striga din toate puterile ?i nu te opri, da drumul glasului sa sune ca o trâmbi?a, veste?te poporului Meu pacatele sale ?i casei lui Iacov faradelegile sale.
Isa 58:2  În fiecare zi Ma cauta, pentru ca ei voiesc sa ?tie caile Mele ca un popor care faptuie?te dreptatea ?i de la legea Dumnezeului sau nu se abate. Ei Ma întreaba despre legile drepta?ii ?i doresc sa se apropie de Dumnezeu.
Isa 58:3  Pentru ce sa postim, daca Tu nu vezi? La ce sa ne smerim sufletul nostru, daca Tu nu iei aminte? Da, în zi de post, voi va vede?i de treburile voastre ?i asupri?i pe to?i lucratorii vo?tri.
Isa 58:4  Voi posti?i ca sa va certa?i ?i sa va sfadi?i ?i sa bate?i furio?i cu pumnul; nu posti?i cum se cuvine zilei aceleia, ca glasul vostru sa se auda sus.
Isa 58:5  Este oare acesta un post care Îmi place, o zi în care omul î?i smere?te sufletul sau? Sa-?i plece capul ca o trestie, sa se culce pe sac ?i în cenu?a, oare acesta se cheama post, zi placuta Domnului?
Isa 58:6  Nu ?ti?i voi postul care Îmi place? - zice Domnul. Rupe?i lan?urile nedrepta?ii, dezlega?i legaturile jugului, da?i drumul celor asupri?i ?i sfarâma?i jugul lor.
Isa 58:7  Împarte pâinea ta cu cel flamând, adaposte?te în casa pe cel sarman, pe cel gol îmbraca-l ?i nu te ascunde de cel de un neam cu tine.
Isa 58:8  Atunci lumina ta va rasari ca zorile ?i tamaduirea ta se va grabi. Dreptatea ta va merge înaintea ta, iar în urma ta slava lui Dumnezeu.
Isa 58:9  Atunci vei striga ?i Domnul te va auzi; la strigatul tau El va zice: Iata-ma! Daca tu îndepartezi din mijlocul tau asuprirea, amenin?area cu mâna ?i cuvântul de cârtire,
Isa 58:10  Daca dai pâinea ta celui flamând ?i tu saturi sufletul amarât, lumina ta va rasari în întuneric ?i bezna ta va fi ca miezul zilei.
Isa 58:11  Domnul te va calauzi necontenit ?i în pustiu va satura sufletul tau. El va da tarie oaselor tale ?i vei fi ca o gradina adapata, ca un izvor de apa vie, care nu seaca niciodata.
Isa 58:12  Pe vechile tale ruine se vor înal?a cladiri noi, vei ridica din nou temeliile strabune ?i vei fi numit dregator de sparturi ?i înnoitor de drumuri, ca ?ara sa poata fi locuita.
Isa 58:13  Daca î?i vei opri piciorul tau în ziua de odihna ?i nu-?i vei mai vedea de treburile tale în ziua Mea cea sfânta, ci vei socoti ziua de odihna ca desfatare ?i vrednica de cinste, ca sfin?ita de Domnul, ?i vei cinsti-o, fara sa mai umbli, fara sa te mai îndeletnice?ti cu treburile tale ?i fara sa mai vorbe?ti de?ertaciuni,
Isa 58:14  Atunci vei afla desfatarea ta în Domnul. Eu te voi purta în car de biruin?a pe culmile cele mai înalte ale ?arii ?i te voi bucura de mo?tenirea tatalui tau Iacov, caci gura Domnului a grait acestea.
Isa 59:1  Iata, mâna Domnului nu este prea scurta ca sa nu poata sa izbaveasca, ?i urechea Lui prea tare ca sa nu auda.
Isa 59:2  Ci nelegiuirile voastre au pus despar?ire intre voi ?i Dumnezeul vostru ?i pacatele voastre L-au facut sa-?i ascunda fa?a ca sa nu va auda.
Isa 59:3  Pentru ca mâinile voastre sunt întinate cu sânge ?i degetele voastre cu nelegiuiri; buzele voastre graiesc cuvinte mincinoase ?i limba voastra, strâmbatate.
Isa 59:4  Nimeni nu cheama în sprijinul sau dreptatea ?i cu cinste nici un judecator nu hotara?te; ci to?i î?i pun nadejdea în lucruri de?arte ?i în vorbe fara rost: zamislesc silnicia ?i nasc pacatul.
Isa 59:5  Clocesc oua de ?arpe ?i urzesc pânza de paianjen: cine manânca din ouale lor moare, iar din cele sparte ies napârci.
Isa 59:6  Din pânza lor ve?minte nu se pot face ?i cu lucrul facut de mâna lor nu se acopera, caci lucrul lor este lucru rau; în mâinile lor sunt numai fapte silnice.
Isa 59:7  Picioarele lor alearga spre rau, grabnice sa verse sânge nevinovat; cugetele lor sunt cugete viclene; în calea lor sala?luiesc pustiirea ?i prapadul.
Isa 59:8  Nu cunosc drumul pacii ?i pe urmele lor nu este nici o dreptate; cararile lor sunt întortocheate ?i cine porne?te pe ele nu ?tie de pace.
Isa 59:9  Pentru aceasta, judecata este departe de noi ?i dreptatea nu ne ajunge. Noi a?teptam lumina, dar iata întunericul; a?teptam revarsatul zorilor, dar umblam în bezna.
Isa 59:10  Umblam bâjbâind, ca orbii pe lânga zid; ca ?i cei fara ochi bâjbâim mereu, ne poticnim în miezul zilei ca ?i pe înserate; între oamenii în putere suntem ca ni?te mor?i.
Isa 59:11  Mormaim to?i ca ur?ii, ne vaitam ca porumbi?a, a?teptam judecata, dar ea nu vine; mântuirea, dar ea este departe de noi.
Isa 59:12  Ca pacatele noastre s-au înmul?it înaintea Ta ?i nelegiuirile sunt marturie împotriva noastra; faradelegile noastre sunt de fa?a ?i faptele noastre nelegiuite le ?tim:
Isa 59:13  Necredin?a ?i tagada Domnului, caderea de la credin?a în Dumnezeu, grairea minciunii ?i razvratirea, nascocirea ?i cugetarea la lucruri viclene.
Isa 59:14  ?i lasata la o parte este judecata, iar dreptatea sta departe; adevarul se poticne?te în pie?e ?i fapta cinstita nu mai are loc.
Isa 59:15  Adevarul nu mai este ?i cel ce se da la o parte din calea rauta?ii este sfarâmat. ?i Dumnezeul nostru a vazut ?i S-a mâniat ca nu mai este dreptate.
Isa 59:16  ?i a vazut ca nu este nici un om ?i S-a mirat ca nimeni nu mijloce?te. Atunci bra?ul Lui I-a venit în ajutor, ?i dreptatea Sa a fost sprijinul Sau.
Isa 59:17  S-a îmbracat cu dreptatea ca ?i cu o plato?a ?i a pus pe capul Sau coiful izbavirii; S-a îmbracat cu razbunarea ca ?i cu o haina ?i S-a înfa?urat în râvna Sa ca ?i într-o mantie.
Isa 59:18  Dupa fapta ?i rasplata: urgie împotriva vrajma?ilor ?i rasplata dupa fapta împotrivitorilor Lui; ?inuturilor celor de departe, rasplata cuvenita.
Isa 59:19  Cei de la apus se vor teme de numele Domnului ?i cei de la rasarit, de slava Lui; ca va veni ca un ?uvoi îngust pe care Duhul Domnului îl mâna.
Isa 59:20  "Pentru Sion El va veni ca un Mântuitor, pentru cei din Iacov care se vor cai de pacatele lor", zice Domnul.
Isa 59:21  Iata, acesta este legamântul Meu cu ei, zice Domnul: "Duhul Meu, Care odihne?te peste tine, ?i cuvintele Mele pe care le-am pus în gura ta, sa nu se departeze din gura ta, nici din gura urma?ilor tai ?i nici din gura urma?ilor urma?ilor tai, zice Domnul, de acum ?i pâna în veac!"
Isa 60:1  Lumineaza-te, lumineaza-te, Ierusalime, ca vine lumina ta, ?i slava Domnului peste tine a rasarit!
Isa 60:2  Caci iata întunericul acopera pamântul, ?i bezna, popoarele; iar peste tine rasare Domnul, ?i slava Lui straluce?te peste tine.
Isa 60:3  ?i vor umbla regi întru lumina ta ?i neamuri întru stralucirea ta.
Isa 60:4  Ridica împrejur ochii tai ?i vezi, ca to?i se aduna ?i se îndreapta catre tine. Fiii tai vin de departe ?i fiicele tale sunt aduse pe umeri.
Isa 60:5  Atunci vei vedea, vei straluci ?i va bate tare inima ta ?i se va largi, caci catre tine se va îndrepta boga?ia marii ?i avu?iile popoarelor catre tine vor curge.
Isa 60:6  Caravane de camile te vor acoperi, ?i dromadere din Madian ?i Efa. Toate sosesc din ?eba, încarcate cu aur ?i cu tamâie, cântând laudele Domnului.
Isa 60:7  Toate turmele Chedarului la tine se vor aduna, berbecii din Nebaiot te vor sluji pe tine ?i ca o jertfa bineplacuta se vor urca pe jertfelnicul Meu, ?i templul rugaciunii Mele se va slavi.
Isa 60:8  Cine zboara ca norii ?i ca porumbi?a spre sala?ul ei?
Isa 60:9  Caci pentru Mine se aduna corabiile, în frunte cu cele din Tarsis, ca sa aduca de departe pe feciorii tai; aurul ?i argintul lor pentru numele Dumnezeului tau ?i pentru Sfântul lui Israel, Care te preamare?te.
Isa 60:10  Feciorii de neam strain zidi-vor zidurile tale ?i regii lor în slujba ta vor fi, ca întru mânia Mea te-am lovit ?i în îndurarea Mea M-am milostivit de tine.
Isa 60:11  Por?ile tale mereu vor fi în laturi, zi ?i noapte vor ramâne deschise, ca sa se care la tine boga?iile neamurilor, iar regii lor în fruntea lor vor fi.
Isa 60:12  Caci neamul ?i regatul care nu vor sluji ?ie vor pieri ?i neamurile acelea vor fi nimicite.
Isa 60:13  Marirea Libanului, chiparosul, ulmul ?i meri?orul la tine vor veni, cu to?ii laolalta, ca sa împodobeasca loca?ul cel sfânt al Meu, ?i Eu voi slavi locul unde se odihnesc picioarele Mele.
Isa 60:14  ?i feciorii asupritorilor tai smeri?i la tine vor veni ?i se vor închina la picioarele tale to?i cei ce te-au urât ?i pe tine te vor numi: cetatea Domnului, Sionul Sfântului lui Israel.
Isa 60:15  Din parasita ?i defaimata ce erai pe veci, voi face din tine mândria veacurilor, bucurie din neam în neam.
Isa 60:16  Tu vei suge laptele neamurilor ?i vei mânca bunata?ile regilor. ?i vei ?ti ca Eu, Domnul, sunt Mântuitorul tau, ca Cel puternic al lui Iacov este Rascumparatorul tau.
Isa 60:17  În loc de arama î?i voi aduce aur, în loc de fier, argint, în loc de lemn, arama ?i în loc de pietre, fier. ?i voi pune judecator al tau pacea ?i stapânitor peste tine dreptatea.
Isa 60:18  ?i nu se va mai auzi de silnicie în ?ara ta, de pustiire ?i de ruina în hotarele tale. Zidurile tale le vei numi mântuire ?i por?ile tale lauda.
Isa 60:19  Nu vei mai avea soarele ca lumina în timpul zilei ?i stralucirea lunii nu te va mai lumina; ci Domnul va fi pentru tine o lumina ve?nica ?i Dumnezeul tau va fi slava ta.
Isa 60:20  Soarele tau nu va mai asfin?i ?i luna nu va mai descre?te; ca Domnul va fi pentru tine lumina ve?nica ?i zilele întristarii tale se vor sfâr?i.
Isa 60:21  În poporul tau vor fi numai drep?i ?i vor stapâni ?ara pentru totdeauna; vlastar pe care l-am sadit Eu, lucrul mâinilor Mele facut spre slava Mea.
Isa 60:22  Cel mai mic va fi cât o mie, cel mai neînsemnat va fi cât un neam puternic: Eu, Domnul, am hotarât acestea ?i la vreme voi fi împlinitorul lor.
Isa 61:1  Duhul Domnului este peste Mine, ca Domnul M-a uns sa binevestesc saracilor, M-a trimis sa vindec pe cei cu inima zdrobita, sa propovaduiesc celor robi?i slobozire ?i celor prin?i în razboi libertate;
Isa 61:2  Sa dau de ?tire un an de milostivire al Domnului ?i o zi de razbunare a Dumnezeului nostru;
Isa 61:3  Sa mângâi pe cei întrista?i; celor ce jelesc Sionul, sa le pun pe cap cununa în loc de cenu?a, untdelemn de bucurie în loc de ve?minte de doliu, slava în loc de deznadejde. Ei vor fi numi?i: stejari ai drepta?ii, sad al Domnului spre slavirea Lui.
Isa 61:4  Ei vor zidi pe vechile ruine, vor ridica darâmaturile de altadata, vor reface ceta?ile distruse, pustiite din neam în neam.
Isa 61:5  Oameni de neam strain vor veni ?i va vor pa?te turmele, feciori din alt neam vor fi plugarii ?i vierii vo?tri.
Isa 61:6  ?i voi, voi ve?i fi numi?i preo?i ai Domnului, slujitori ai Dumnezeului nostru. Bunata?ile popoarelor, voi le ve?i mânca ?i cu averile lor voi va ve?i mândri.
Isa 61:7  Fiindca ocara lor era îndoita, batjocura ?i scuipari erau partea lor, pentru aceasta îndoit în pamântul lor vor mo?teni ?i de slava cea de-a pururi ei se vor bucura!
Isa 61:8  Ca Eu sunt Domnul, Care iubesc dreptatea ?i urasc rapirile nedrepte. Eu le voi da cu credincio?ie plata lor ?i legamânt ve?nic cu ei voi încheia.
Isa 61:9  Cu nume mare va fi neamul lor între neamuri ?i urma?ii lor printre popoare. To?i cei ce îi vor vedea vor da marturie ca ei sunt un neam binecuvântat de Domnul.
Isa 61:10  Bucura-Ma-voi întru Domnul, salta-va de veselie sufletul Meu întru Dumnezeul Meu, ca M-a îmbracat cu haina mântuirii, cu ve?mântul veseliei M-a acoperit. Ca unui mire Mi-a pus Mie cununa ?i ca pe o mireasa M-a împodobit cu podoaba.
Isa 61:11  Ca pamântul care rasare ierburi, ?i ca o gradina în care samân?a încol?e?te, a?a Domnul Dumnezeu va face dreptatea sa rasara, ?i înaintea tuturor neamurilor preaslavirea Sa.
Isa 62:1  Pentru Sion nu voi tacea ?i pentru Ierusalim nu voi avea odihna pâna ce dreptatea lui nu va ie?i ca lumina ?i mântuirea lui nu va arde ca o flacara.
Isa 62:2  Atunci neamurile vor vedea dreptatea ta ?i to?i regii slava ta ?i te vor chema pe tine cu nume nou, pe care îl va rosti gura Domnului.
Isa 62:3  ?i tu vei fi ca o cununa de marire în mâna Domnului ?i ca o diadema regala în mâna Dumnezeului tau.
Isa 62:4  ?i nu ?i se va mai zice ?ie: "Alungata", ?i ?arii tale: "Pustiita", ci tu te vei chema: "Întru tine am binevoit" ?i ?ara ta: "Cea cu barbat", ca Domnul a binevoit întru tine ?i pamântul tau va avea un so?.
Isa 62:5  ?i în ce chip se însore?te flacaul cu fecioara, Cel ce te-a zidit Se va înso?i cu tine, ?i în ce chip mirele se vesele?te de mireasa, a?a Se va veseli de tine Dumnezeul tau!
Isa 62:6  Pe zidurile tale, Ierusalime, Eu pun strajeri, care nici zi, nici noapte nu vor tacea! Voi, care aduce?i aminte Domnului de fagaduin?ele Lui, sa n-ave?i odihna!
Isa 62:7  ?i sa nu-I da?i ragaz pâna ce nu va a?eza din nou Ierusalimul, ca sa faca din el lauda pamântului.
Isa 62:8  Juratu-S-a Domnul pe dreapta Lui ?i pe bra?ul Lui cel tare: "Nu voi mai da de aici înainte grâul tau spre hrana vrajma?ilor tai, ?i cei de neam strain, nu vor bea mustul tau, rodul muncii tale.
Isa 62:9  Ci numai cei ce îl vor fi adunat îl vor mânca ?i vor lauda pe Domnul, ?i cei care vor fi facut culesul vor bea vinul în cur?ile templului Meu cel sfânt!"
Isa 62:10  Intra?i, întra?i pe por?i! Gati?i cale poporului, gati?i, gati?i-i drum, cura?i?i-l de pietre, înal?a?i un steag peste neamuri!
Isa 62:11  Iata, Domnul veste?te acestea pâna la marginile pamântului: "Zice?i fiicei Sionului: Mântuitorul tau vine! El vine cu plata, ?i rasplatirile merg înaintea Lui!"
Isa 62:12  ?i ei se vor chema: "Popor sfânt, rascumpara?i ai Domnului ?i ?ie ?i se va zice: "Cea cautata", "Cetatea neparasita!"
Isa 63:1  Cine este Cel ce vine împurpurat, cu ve?mintele Sale mai ro?ii decât ale celui ce culege la vie, cu podoaba în îmbracamintea Lui ?i mândru de bel?ugul puterii Lui? "Eu sunt Acela al Carui cuvânt este dreptatea ?i puternic este sa rascumpere!"
Isa 63:2  Pentru ce ai îmbracamintea ro?ie ?i ve?mântul Tau este ro?u ca al celui care calca în teasc?
Isa 63:3  "Singur am calcat în teasc ?i dintre popoare nimeni nu era cu Mine; ?i i-am calcat în mânia Mea, i-am strivit în urgia Mea, încât sângele lor a râ?nit pe ve?mântul Meu, ?i Mi-am patat toate hainele Mele.
Isa 63:4  Caci o zi de razbunare era sortita în inima Mea ?i anul rascumpararii sosise.
Isa 63:5  Priveam în jur: nici un ajutor! Ma cuprindea mirarea: nici un sprijin! Atunci bra?ul Meu M-a ajutat ?i urgia Mea sprijin Mi-a fost.
Isa 63:6  În mânia Mea am calcat în picioare popoare ?i le-am zdrobit în urgia Mea ?i Sângele lor l-am împra?tiat pe pamânt".
Isa 63:7  Voi pomeni îndurarile Domnului, faptele Lui minunate, dupa tot ce a facut Domnul pentru noi ?i pentru marea bunatate pe care El ne-a marturisit-o în milostivirea Sa ?i dupa mul?imea milelor Sale.
Isa 63:8  ?i a zis: "Cu adevarat ei sunt poporul Meu, fii care nu vor fi necredincio?i!"
Isa 63:9  ?i El le-a fost izbavitor în toate strâmtorarile lor. ?i n-a fost un trimis ?i nici un înger, ci fa?a Lui i-a mântuit. Întru iubirea Lui ?i întru îndurarea Lui El i-a rascumparat, i-a ridicat ?i i-a purtat în toata vremea de demult.
Isa 63:10  Dar ei s-au razvratit ?i au amarât Duhul Lui cel sfânt, din ?are pricina El S-a facut împotrivitorul lor ?i s-a razboit împotriva lor.
Isa 63:11  Atunci ei ?i-au adus aminte de vremurile trecute, de sluga Sa Moise: Unde este Cel ce a scos din mare pe pastorul ?i turma Sa? Unde este Cel ce a pus în mijlocul ei Duhul Sau cel sfânt?
Isa 63:12  Cel Care a calauzit dreapta lui Moise cu bra?ul Sau slavit? Cel Care a despicat apele înaintea lor ca sa-?i faca un nume ve?nic?
Isa 63:13  Care i-a calauzit prin adâncurile marii, ca pe un cal în pustiu ?i ei nu s-au poticnit?
Isa 63:14  Ca dobitoacele care coboara la ?es, a?a Duhul Domnului îi aducea la odihna. Astfel ai pova?uit Tu pe poporul Tau, ca sa-?i faci un nume slavit.
Isa 63:15  Prive?te din ceruri ?i vezi, din loca?ul Tau cel sfânt ?i stralucit: Unde este râvna ?i puterea Ta nesfâr?ita, zbuciumul launtrului Tau ?i milostivirile Tale?
Isa 63:16  Pentru mine, acestea au încetat! Dar Tu e?ti Parintele nostru! Avraam nu ?tie nimic, Israel nu ne cunoa?te. Tu, Doamne, e?ti Tatal nostru, Mântuitorul rostru: acesta este numele Tau de totdeauna.
Isa 63:17  Pentru ce, Doamne, ne-ai lasat sa ratacim departe de caile Tale ?i ne-ai învârto?at inimile noastre ca sa nu ne temem de Tine? Întoarce-Te pentru robii Tai, pentru semin?iile mo?tenirii Tale.
Isa 63:18  Pentru ce au pa?it cei nelegiui?i în templul Tau ?i vrajma?ii no?tri au calcat în picioare altarul Tau?
Isa 63:19  Am ajuns ca unii peste care Tu de multa vreme nu mai stapâne?ti ?i care nu mai sunt chema?i cu numele Tau.
Isa 64:1  Daca ai rupe cerurile ?i Te-ai pogorî, mun?ii s-ar cutremura!
Isa 64:2  Ca un foc care arde vreascurile, ca o vâlvataie care fierbe apa în clocot, fa pe vrajma?ii Tai sa cunoasca numele Tau ?i sa tremure înaintea Ta neamurile, vazându-Te
Isa 64:3  Facând minuni nea?teptate,
Isa 64:4  Despre care niciodata nu s-a auzit graind. Nici urechea n-a auzit, nici ochiul n-a vazut un dumnezeu, afara de Tine, care ar savâr?i unele ca acestea pentru cei ce nadajduiesc în el.
Isa 64:5  Tu Te duci întru întâmpinarea celor ce savâr?esc faptele drepta?ii ?i î?i aduc aminte de caile Tale. Iata, Tu Te-ai pornit cu mânie ?i noi eram vinova?i prin necredin?a ?i faradelegea, noastra!
Isa 64:6  To?i am ajuns ca necura?ii ?i toate faptele drepta?ii noastre ca un ve?mânt întinat. Noi to?i am cazut ca frunzele uscate ?i faradelegile noastre ne luau ca vântul.
Isa 64:7  Nimeni nu chema numele Tau ?i nici unul nu se de?tepta ca sa se întareasca întru Tine. Caci Tu ai ascuns fa?a Ta de la noi ?i ne-ai lasat în voia faradelegilor noastre.
Isa 64:8  ?i acum, Doamne, Tu e?ti Tatal nostru, noi suntem lutul ?i Tu olarul, to?i lucrul mâinilor Tale suntem!
Isa 64:9  O, Doamne! Nu Te mânia pe noi foarte ?i nu-?i aduce aminte la nesfâr?it de faradelegea noastra! Prive?te, caci noi to?i suntem poporul Tau!
Isa 64:10  Ceta?ile Tale sfinte au ajuns pustii, Sionul este ca un de?ert, Ierusalimul ca un loc pustiit!
Isa 64:11  Templul nostru sfânt ?i marit în care Te-au preaslavit parin?ii no?tri a ajuns prada focului ?i toate cele scumpe noua, darâmaturi!
Isa 64:12  Po?i Tu oare sa Te stapâne?ti, sa taci, Doamne, ?i sa ne întristezi atât de mult?
Isa 65:1  Cautat am fost de cei ce nu întrebau de Mine, gasit am fost de cei ce nu Ma cautau. ?i am zis: "Iata-Ma, iata-Ma aici, la un neam care nu chema numele Meu!
Isa 65:2  Tins-am mâinile Mele în toata vremea catre un popor razvratit, care mergea pe cai silnice, dupa cugetele sale,
Isa 65:3  Oameni care întarâtau fara încetare fa?a Mea jertfind în gradini ?i pe lespezile acoperi?ului ardeau miresme;
Isa 65:4  Stateau în morminte ?i mâncau în crapaturi de stânca, mâncau carne de porc, ale caror vase erau pline de mâncaruri spurcate
Isa 65:5  ?i care ziceau: "Da-te înapoi, nu te apropia de mine, caci eu sunt sfânt fa?a de tine!" - Ace?tia sunt ca un fum care se urca în narile Mele, o vapaie care arde fara sfâr?it.
Isa 65:6  Iata este scris înaintea Mea: "Nu voi tacea pâna ce nu voi rasplati
Isa 65:7  Faradelegile voastre ?i faradelegile parin?ilor vo?tri laolalta, - zice Domnul - ale celor care au adus jertfa de tamâie pe mun?i ?i pe dealuri ?i au râs de Mine. Eu îi voi rasplati dupa faptele lor ?i cu masura plina.
Isa 65:8  A?a zice Domnul: "Ca atunci când gase?ti must într-un strugure ?i zici: "Nu-l rupe, ca în acesta se afla o binecuvântare", tot astfel voi face ?i cu slujitorii Mei; Ma voi feri sa stric tot!
Isa 65:9  ?i voi face sa rasara din Iacov o odrasla ?i din Iuda un mo?tenitor peste mun?ii Mei; ?i cei ale?i ai Mei li vor stapâni ?i slujitorii Mei vor locui acolo.
Isa 65:10  ?i ?aronul va ajunge pa?une pentru turme ?i Acorul, ima? pentru vite, pentru poporul Meu care M-a cautat pe Mine.
Isa 65:11  ?i voi, cei ce a?i parasit pe Domnul, care a?i uitat de muntele Meu cel sfânt, care întinde?i masa pentru dumnezeul Gad ?i umple?i o cupa pentru Meni;
Isa 65:12  Pe to?i va voi da în ascu?i?ul sabiei ?i junghierii va ve?i pleca, pentru ca am strigat catre voi ?i nu Mi-a?i raspuns, am grait ?i nu M-a?i auzit, ci a?i facut cele rele în ochii Mei ?i ceea ce nu am binevoit a?i ales".
Isa 65:13  Pentru aceasta, a?a zice Domnul Dumnezeu: "Iata, slugile Mele vor mânca ?i voua va va fi foame, vor bea ?i voi ve?i fi înseta?i, se vor bucura, iar voi ve?i fi înfrunta?i!
Isa 65:14  Iata slugile Mele vor salta de veselie, iar voi ve?i striga de întristata ce va va fi inima, ?i de frânt ce va va fi sufletul ve?i urla!"
Isa 65:15  ?i ve?i lasa numele vostru ale?ilor Mei spre blestem: "Domnul Dumnezeu sa te ucida!... Dar slujitorii Mei vor fi numi?i cu alt nume.
Isa 65:16  Cine se va binecuvânta pe pamânt se va binecuvânta de Dumnezeul adevarului, ?i cel ce se va jura pe pamânt se va jura pe Dumnezeul adevarului; ca nenorocirile din vremurile de demult au fost uitate ?i ei stau departe de ochii Mei.
Isa 65:17  Pentru ca Eu voi face ceruri noi ?i pamânt nou. Nimeni nu-?i va mai aduce aminte de vremurile trecute ?i nimanui nu-i vor mai veni în minte,
Isa 65:18  Ci se vor bucura ?i se vor veseli de ceea ce Eu voi fi facut, caci iata întemeiez Ierusalimul pentru bucurie ?i poporul lui pentru desfatare.
Isa 65:19  ?i Ma voi bucura de Ierusalim ?i Ma voi veseli de poporul Meu ?i nu se va mai auzi în acesta nici plâns, nici ?ipat.
Isa 65:20  Sa nu mai fie acolo copii care mor în floarea vârstei ?i nici batrâni care nu ajung la capatul vie?ii lor! A?a ca cine va muri la o suta de ani va fi tânar ?i cine nu o va ajunge va fi blestemat.
Isa 65:21  ?i ei vor zidi case ?i vor locui ?i vor sadi vii ?i din rodul lor vor mânca.
Isa 65:22  Dar nu vor cladi ca altul sa locuiasca, nici nu vor sadi ca altul sa manânce. Într-adevar vârsta poporului Meu va fi ca vârsta copacilor, ?i cei ale?i ai Mei se vor bucura de osteneala mâinilor lor.
Isa 65:23  Nu se vor trudi în zadar ?i nu vor na?te feciori pentru moarte fara de vreme, ca ei vor fi un neam binecuvântat de Domnul ?i împreuna cu ei ?i odraslele lor.
Isa 65:24  ?i înainte de a Ma chema pe Mine, Eu le voi raspunde, ?i graind ei înca, Eu îi voi fi ascultat.
Isa 65:25  Lupul va pa?te la un loc cu mielul, leul va mânca paie ca boul ?i ?arpele cu ?arâna se va hrani. Nimic rau ?i vatamator nu va fi în muntele Meu cel sfânt", zice Domnul.
Isa 66:1  A?a zice Domnul: "Cerul este scaunul Meu ?i pamântul a?ternut picioarelor Mele! Ce fel de casa Îmi ve?i zidi voi ?i ce loc de odihna pentru Mine?"
Isa 66:2  "Toate acestea mâna Mea le-a facut ?i sunt ale Mele, zice Domnul. Spre unii ca ace?tia Îmi îndrept privirea Mea: spre cei smeri?i, cu duhul umilit ?i care tremura la cuvântul Meu!
Isa 66:3  Cel ce junghie un bou ?i în acela?i timp omoara un om, cel ce jertfe?te o oaie ?i în acela?i timp rupe gâtul unui câine, cel ce aduce prinos ?i în acela?i timp aduce sânge de porc, cel ce aduce jertfa de tamâie ?i în acela?i timp se închina la idoli, - to?i ace?tia ?i-au ales cai nelegiuite ?i în urâciunile lor traie?te sufletul lor.
Isa 66:4  Pentru aceasta ?i Eu voi alege pentru ei soarta cea rea ?i cele ce îi înfrico?eaza le voi aduce peste ei; ca am strigat ?i nu Mi-au raspuns, am grait ?i nu M-au auzit, au facut faradelegi înaintea ochilor Mei ?i ceea ce Eu nu binevoiesc întru Mine, au ales.
Isa 66:5  Lua?i aminte la cuvântul Domnului, voi care tremura?i de El! Iata ce graiesc fra?ii vo?tri care va urasc ?i va prigonesc pentru numele Meu: "Sa-?i arate Domnul slava Sa ?i noi sa o vedem din bucuria voastra!" Dar ei se vor face de ocara.
Isa 66:6  Un glas! Un vuiet din cetate! Un glas din templu! Este glasul Domnului! El rasplate?te vrajma?ilor Sai dupa faptele lor.
Isa 66:7  Înainte de a se zvârcoli în dureri de na?tere, ea a nascut; înainte de a sim?i chinul, ea a nascut un fiu.
Isa 66:8  Cine a auzit sau cine a vazut unele ca acestea? Oare o ?ara se na?te într-o singura zi ?i un popor dintr-odata? Abia au apucat-o durerile na?terii ?i fiica Sionului a ?i nascut fii!
Isa 66:9  Oare, Eu voi deschide pântecele fara sa-l la? sa nasca? Zice Domnul. Sau Eu, Cel ce fac sa nasca, îl voi închide?
Isa 66:10  Bucura-te, Ierusalime, ?i voi, cei care îl iubi?i, salta?i de veselie. Fiii în culmea veseliei, voi cei care îl plângea?i!
Isa 66:11  Astfel ca voi sa fiii alapta?i ?i sa va satura?i la pieptul mângâierilor sale, sa sorbi?i ?i sa va desfata?i la sânul slavei sale!
Isa 66:12  Acestea zice Domnul: "Varsa-voi pacea peste el ca un râu ?i slava popoarelor ca un ?uvoi ie?it din albia lui. Pruncii lui vor fi purta?i în bra?e ?i dezmierda?i pe genunchi.
Isa 66:13  Dupa cum mama î?i mângâie pe fiul ei ?i Eu va voi mângâia pe voi, ?i voi ve?i fi mângâia?i în Ierusalim.
Isa 66:14  Când ve?i vedea, inima voastra va tresari de bucurie ?i oasele voastre vor odrasli ca iarba. ?i mâna Domnului se va arata slujitorilor Sai, iar urgia, peste vrajma?ii Sai.
Isa 66:15  Caci Domnul vine în vapaie ?i carele Lui sunt ca o vijelie, ca sa dezlan?uie cu fierbin?eala mânia Lui ?i certarea Lui cu vapai de foc.
Isa 66:16  Domnul va judeca cu foc ?i cu sabie pe tot omul ?i mul?i vor fi cei ce vor cadea de bataia Domnului!
Isa 66:17  Faptele ?i gândurile celor ce se sfin?esc ?i se cura?esc pentru închinaciunile din gradini, într-un loc ascuns, în mijlocul unei adunari de ucenici, ale celor care manânca din carnea de porc, mâncaruri scârnave ?i ?oareci, vor fi zadarnicite, zice Domnul.
Isa 66:18  Dar Eu vin ca sa strâng la un loc popoarele ?i toate limbile. Ele vor veni ?i vor vedea slava Mea,
Isa 66:19  ?i le voi da un semn. ?i pe cei scapa?i cu via?a îi voi trimite catre popoarele din Tarsis, Put, Lud, Me?ec, Ro?, Tubal, Iavan, catre ?inuturile cele mai departate care n-au auzit despre Mine ?i nu au vazut slava Mea. ?i la aceste neamuri vor vesti slava Mea.
Isa 66:20  ?i din toate neamurile vor aduce pe fra?ii vo?tri prinos Domnului: pe cai, în caru?e, pe paturi, pe catâri ?i pe camile, pâna la muntele cel sfânt al Meu, la Ierusalim, zice Domnul, precum fiii lui Israel aduc prinoase în vase curate pentru templul Domnului.
Isa 66:21  ?i din ei voi lua preo?i ?i levi?i, zice Domnul.
Isa 66:22  Într-adevar, precum cerul cel nou ?i pamântul cel nou pe care le voi face, zice Domnul, vor ramânea înaintea Mea, a?a va dainui totdeauna semin?ia voastra ?i numele vostru.
Isa 66:23  ?i din luna noua în luna noua ?i din zi de odihna în zi de odihna vor veni to?i ?i se vor închina înaintea Mea, zice Domnul.
Isa 66:24  ?i când vor ie?i, vor vedea trupurile moarte ale celor care s-au razvratit împotriva Mea, ca viermele lor nu va muri ?i focul lor nu se va stinge. ?i ei vor fi o sperietoare pentru to?i.


\end{document}