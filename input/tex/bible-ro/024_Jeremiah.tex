\begin{document}

\title{Ieremia}


\chapter{1}

\par 1 Cuvintele lui Ieremia, fiul lui Hilchia, dintre preo?ii de la Anatot, din pamântul lui Veniamin,
\par 2 Catre care a fost cuvântul Domnului, în zilele lui Iosia, fiul lui Amon, regele lui Iuda, în anul al treisprezecelea al domniei acestuia,
\par 3 Precum ?i în zilele lui Ioiachim, fiul lui Iosia, regele lui Iuda, pâna în luna a cincea din anul al unsprezecelea al lui Sedechia, fiul lui Iosia, regele lui Iuda, adica pâna la robirea Ierusalimului.
\par 4 Fost-a cuvântul Domnului catre mine ?i mi-a zis:
\par 5 "Înainte de a te fi zamislit în pântece, te-am cunoscut, ?i înainte de a ie?i din pântece, te-am sfin?it ?i te-am rânduit prooroc pentru popoare".
\par 6 Iar eu am raspuns: "O, Doamne, Dumnezeule, eu nu ?tiu sa vorbesc, pentru ca sunt înca tânar".
\par 7 Domnul însa mi-a zis: "Sa nu zici: Sunt înca tânar; caci la câ?i te voi trimite, la to?i vei merge ?i tot ce-?i voi porunci vei spune.
\par 8 Sa nu te temi de dân?ii, caci Eu sunt cu tine, ca sa te izbavesc", zice Domnul.
\par 9 ?i Domnul mi-a întins mâna, mi-a atins gura ?i mi-a zis: "Iata am pus cuvintele Mele în gura ta!
\par 10 Iata te-am pus în ziua aceasta peste popoare ?i peste regate, ca sa smulgi ?i sa arunci la pamânt, sa pierzi ?i sa darâmi, sa zide?ti ?i sa sade?ti!"
\par 11 Apoi a fost cuvântul Domnului catre mine ?i mi-a zis: "Ieremia, ce vezi tu?" ?i eu am raspuns: "Vad un toiag din lemn de migdal".
\par 12 Zisu-mi-a Domnul: "Tu ai vazut bine, ca Eu priveghez asupra cuvântului Meu, ca sa-l împlinesc!"
\par 13 ?i iar a fost cuvântul Domnului catre mine ?i mi-a zis: "Ce vezi tu?" ?i eu am raspuns: "Vad un cazan clocotind, care e gata sa se verse dinspre miazanoapte".
\par 14 Iar Domnul mi-a zis: De la miazanoapte se va deschide nenorocire asupra tuturor locuitorilor ?arii acesteia.
\par 15 Ca iata voi chema toate popoarele ?arilor de la miazanoapte, zice Domnul, ?i vor veni acelea ?i î?i vor pune fiecare din ele scaunul sau la poarta de intrare a Ierusalimului, pe lânga toate zidurile lui ?i în toate ceta?ile lui Iuda;
\par 16 ?i voi rosti asupra lor judecata Mea pentru toate faradelegile lor, pentru ca M-au parasit, au aprins tamâie înaintea altor dumnezei ?i s-au închinat la lucrurile mâinilor lor.
\par 17 Dar tu, încinge-?i coapsele tale ?i scoala de le spune tot ce-?i voi porunci. Nu tremura înaintea lor, fiindca nu vreau sa tremuri înaintea lor.
\par 18 Ca iata, Eu te-am facut astazi cetate întarita, stâlp de fier ?i zid de arama înaintea acestei ?ari întregi: înaintea regilor lui Iuda, înaintea capeteniilor ei, înaintea preo?ilor ei ?i înaintea poporului ?arii.
\par 19 Ei se vor lupta împotriva ta, dar nu te vor birui ca Eu sunt cu tine, ca sa te izbavesc", zice Domnul.

\chapter{2}

\par 1 Fost-a cuvântul Domnului catre mine ?i a zis:
\par 2 "Mergi ?i striga la urechile fiicei Ierusalimului ?i zi: "A?a graie?te Domnul: Mi-am adus aminte de prietenia cea din tinere?ea ta, de iubirea de pe când erai mireasa ?i mi-ai urmat în pustiu, în pamântul cel nesemanat.
\par 3 Atunci Israel era sfin?enia Domnului ?i pârga roadelor lui; to?i câ?i mâncau din ea se faceau vinova?i ?i nenorocirea venea asupra lor", zice Domnul.
\par 4 Casa lui Iacov ?i toate semin?iile casei lui Israel, asculta?i cuvântul Domnului! A?a zice Domnul:
\par 5 "Ce nedreptate au gasit în Mine parin?ii vo?tri, de s-au departat de Mine ?i s-au dus dupa de?ertaciune ?i au devenit ei în?i?i de?ertaciune?
\par 6 În loc sa zica: Unde este Domnul, Cel ce ne-a scos din pamântul Egiptului ?i ne-a pova?uit prin pustiu, prin pamântul cel gol ?i nelocuit, prin pamântul cel sec, prin pamântul umbrei mor?ii, prin care nimeni nu umblase ?i unde nu locuia om?
\par 7 Eu v-am dus în pamânt roditor, ca sa va hrani?i cu roadele lui ?i cu bunata?ile lui; voi însa a?i intrat ?i a?i spurcat pamântul Meu ?i mo?tenirea Mea a?i facut-o urâciune.
\par 8 Preo?ii n-au zis: Unde este Domnul? Înva?atorii legii nu M-au cunoscut; pastorii au lepadat credin?a ?i proorocii au proorocit în numele lui Baal ?i s-au dus dupa cei ce nu-i pot ajuta.
\par 9 De aceea la judecata voi grai împotriva voastra, zice Domnul, ?i împotriva nepo?ilor vo?tri voi cere osânda!
\par 10 Sa va duce?i în insulele Chitim ?i sa vede?i; trimite?i în Chedar ?i cerceta?i cu de-amanuntul ?i afla?i:
\par 11 Fost-a, oare, acolo ceva de felul acesta? Schimbatu-?i-a oare vreun popor dumnezeii sai, de?i aceia nu sunt dumnezei? Poporul Meu însa ?i-a schimbat slava cu ceea ce nu-l poate ajuta.
\par 12 Mira?i-va de acestea, ceruri; cutremura?i-va, îngrozi?i-va, zice Domnul!
\par 13 Ca doua rele a facut poporul Meu: pe Mine, izvorul apei celei vii, M-au parasit, ?i ?i-au sapat fântâni sparte, care nu pot ?ine apa.
\par 14 Au doara rob sau fiu de rob e Israel? Pentru ce dar s-a facut el prada?
\par 15 Mugit-au asupra lui puii de leu, ridicatu-?i-au glasul lor ?i au facut pustiu ?ara lui; ceta?ile lui sunt fara locuitori.
\par 16 Chiar ?i locuitorii din Nof ?i cei din Tahpanhes ?i-au ras capul, Israele!
\par 17 Oare nu ?i-ai pricinuit tu singur aceasta, parasind pe Domnul Dumnezeul tau, când te pova?uia?
\par 18 ?i acum pentru ce ai luat drumul Egiptului, ca sa bei apa din Nil? ?i pentru ce ?i-ai luat drumul spre Asiria, ca sa bei apa din râul ei?
\par 19 Lepadarea ta de credin?a te va pedepsi ?i rautatea ta te va mustra. În?elege ?i vezi cât e de rau ?i de amar de a parasi pe Domnul Dumnezeul tau ?i de a nu mai avea nici o teama de Mine, zice Domnul Dumnezeul puterilor.
\par 20 Ca în vechime am sfarâmat jugul tau ?i am rupt catu?ele tale, ?i tu ai zis: "Nu voi sluji la idoli, ?i totu?i pe tot dealul înalt ?i sub tot pomul umbros ai facut desfrânare.
\par 21 Eu te-am sadit ca pe o vi?a de soi, ca pe cea mai curata samân?a; cum dar Mi te-ai prefacut în ramura salbatica de vi?a straina?
\par 22 Chiar de te-ai spala cu silitra ?i chiar daca te-ai freca cu le?ie, tot patat e?ti în nedrepta?ile tale fa?a de Mine, zice Domnul Dumnezeu.
\par 23 Cum po?i tu sa zici: Nu m-am întinat ?i n-am umblat dupa Baal? Prive?te la purtarea ta din vale ?i afla ce ai facut tu, camila zburdalnica, tu care cutreieri toate drumurile,
\par 24 Asina salbatica, deprinsa în pustiu, care în aprinderea poftei ei, soarbe aerul! Cine o va putea împiedica sa-?i împlineasca pofta? To?i cei ce o cauta nu se vor osteni, ca în luna ei o vor gasi. Când i s-a zis:
\par 25 Întoarce-?i piciorul de la calea strâmba ?i gâtul nu-l deprinde a înseta, ea a raspuns: Nu, zadarnic! Caci iubesc dumnezei straini ?i merg dupa ei.
\par 26 Cum furul când este prins se ru?ineaza, a?a se va ru?ina ?i casa lui Israel ?i poporul ?i regii lui ?i capeteniile lui ?i proorocii lui ?i preo?ii lui, caci au zis lemnului:
\par 27 "Tu e?ti tatal meu!" ?i pietrei i-au grait: "Tu m-ai nascut!", ?i nu ?i-au întors spre Mine fa?a, ci spatele, iar la vreme de nevoie vor zice: "Scoala ?i ne izbave?te!"
\par 28 Dar unde-?i sunt, Iudo, dumnezeii care ?i i-ai facut? Sa se scoale ?i sa te izbaveasca la vreme de necaz, daca pot! Caci câte ceta?i ai, atâ?ia sunt ?i dumnezeii tai.
\par 29 Pentru ce va certa?i cu Mine? To?i v-a?i purtat cu necredincio?ie ?i a?i pacatuit împotriva Mea, zice Domnul.
\par 30 În zadar am batut pe copiii vo?tri, ca n-au primit înva?atura; pe proorocii vo?tri i-a mâncat sabia voastra, ca un leu pierzator, ?i nu v-a?i temut".
\par 31 Asculta, poporule, cuvântul Domnului care zice: "Pustiu am fost Eu oare pentru Israel? Sau ?ara întunericului am fost? Pentru ce dar poporul Meu zice: Noi în?ine ne suntem stapâni ?i nu mai venim la Tine?
\par 32 Au doara uita fata podoaba sa ?i mireasa gateala sa? Poporul Meu însa M-a uitat de zile nenumarate.
\par 33 Fiica Ierusalimului, cât de iscusita î?i e?ti tu în caile tale ca sa cau?i iubirea! Ba pentru aceasta ?i la nelegiuiri ?i-ai deprins caile tale,
\par 34 ?i chiar ?i în poalele hainei tale se afla sângele saracilor nevinova?i, pe care nu i-ai prins spargând zidul, ?i totu?i zici:
\par 35 "De vreme ce sunt nevinovata, mânia Lui de buna seama se va abate de la mine". Pentru ca zici: "N-am gre?it", de aceea iata Eu cu tine ma voi judeca.
\par 36 De ce atâta graba ca sa-?i schimbi calea? Vei fi ru?inata de Egipt, cum ai fost ru?inata de Asiria.
\par 37 ?i de acolo vei ie?i cu mâinile pe cap, pentru ca a lepadat Domnul pe cei în care tu ?i-ai pus nadejdea, pe cei cu care tu nu vei avea izbânda".

\chapter{3}

\par 1 ?i a mai spus: "Daca un barbat î?i lasa femeia, ?i ea se duce de la el ?i se face so?ie altui barbat, mai poate ea oare sa se întoarca la el? Prin aceasta nu s-ar întina în adevar, oare, ?ara aceea?" ?i tu te-ai desfrânat cu mul?i iubi?i ?i vrei sa te întorci la Mine? zice Domnul.
\par 2 Ridica-?i ochii spre înal?imi ?i prive?te: Unde oare nu s-au desfrânat aceia cu tine? ?ezut-ai pentru ei lânga cale, ca arabul în pustiu, ?i ai spurcat ?ara cu desfrânarea ta ?i cu vicle?ugul tau.
\par 3 De aceea ploile de toamna au lipsit ?i la fel ?i cele de primavara, dar Tu ai avut frunte de desfrânata ?i nu te-ai ru?inat.
\par 4 ?i acum strigi catre Mine: "Tatal meu, Tu ai fost pova?uitorul tinere?ilor mele!
\par 5 Oare pentru totdeauna va fi El mânios ?i oare ve?nic va pastra în Sine mânia?" Iata ce ai zis, dar de facut faci rele ?i spore?ti în acelea.
\par 6 Zisu-mi-a Domnul în zilele lui Iosia: "Vazut-ai ce-a facut Israel, aceasta fiica necredincioasa? A umblat pe to?i mun?ii înal?i ?i pe sub tot copacul umbros ?i s-a desfrânat pe acolo.
\par 7 Dupa ce a facut toate acestea, i-am zis: "Întoarce-te la Mine!" Dar nu s-a întors. ?i a vazut acestea Iuda, sora sa cea necredincioasa.
\par 8 ?i de?i am lasat pe fiica lui Israel cea necredincioasa pentru atâtea fapte de desfrânare ?i i-am dat carte de despar?ire, am vazut ca necredincioasa ei sora, Iuda, nu s-a temut, ci s-a dus ?i ea sa se desfrâneze.
\par 9 ?i prin neru?inarea desfrânarilor ei a pângarit ?ara ?i s-a desfrânat cu pietrele ?i cu lemnele.
\par 10 Peste toate acestea Iuda, necredincioasa sora a fiicei lui Israel, nu s-a întors la Mine din toata inima sa, ci numai din prefacatorie", zice Domnul.
\par 11 Apoi iara?i mi-a zis Domnul: "Fiica lui Israel cea necredincioasa s-a dovedit ca e mai dreapta decât Iuda cea lepadata de Dumnezeu.
\par 12 Mergi de veste?te cuvintele acestea la miazanoapte ?i zi: Întoarce-te, necredincioasa fiica a lui Israel, zice Domnul, ca nu voi varsa asupra voastra mânia Mea, pentru ca sunt milostiv ?i nu Ma voi mânia pe veci, zice Domnul.
\par 13 Recunoa?te-?i însa vinova?ia ta, caci te-ai abatut de la Domnul Dumnezeul tau ?i te-ai desfrânat cu dumnezei straini sub tot arborele umbros ?i glasul Meu nu l-ai ascultat, zice Domnul.
\par 14 Întoarce?i-va, voi copii cazu?i de la credin?a, zice Domnul, ca M-am unit cu voi ?i va voi lua câte unul de cetate ?i câte doi de semin?ie ?i va voi aduce în Sion.
\par 15 Apoi va voi da pastori dupa inima Mea, care va vor pastori cu ?tiin?a ?i pricepere.
\par 16 Când va ve?i înmul?i ?i ve?i ajunge mult roditori pe pamânt, în zilele acelea, zice Domnul, nu se va mai zice: "O, chivotul a?ezamântului Domnului". Nimeni nu se va mai gândi la el, nimeni nu-?i va mai aduce aminte de el, nimanui nu-i va mai parea rau de el, nimeni nu va mai face altul.
\par 17 În vremea aceea Ierusalimul se va numi tronul Domnului ?i toate popoarele se vor aduna acolo pentru numele Domnului ?i nu se vor mai purta dupa îndaratnicia inimii lor celei rele.
\par 18 În zilele acelea va veni casa lui Iuda la casa lui Israel ?i se vor duce împreuna din ?ara de la miazanoapte, în ?ara pe care am dat-o Eu mo?tenire parin?ilor vo?tri.
\par 19 Eu Mi-am zis: Cum sa te pun pe tine în numarul fiilor ?i sa-?i dau ?ara cea placuta, care este mo?tenirea cea mai frumoasa a mul?imii poporului? Dar iara?i Mi-am zis: Tu Ma vei numi Tata al tau ?i nu te vei mai departa de Mine.
\par 20 Însa tocmai cum femeia necredincioasa în?ala pe iubitul sau, a?a ?i voi, casa lui Israel, v-a?i purtat cu în?elaciune fa?a de Mine, zice Domnul.
\par 21 Glas se aude pe înal?imi, s-aude plânsul jalnic al fiilor lui Israel, care se tânguiesc pentru ca ?i-au stricat caile lor ?i au uitat pe Domnul Dumnezeul lor.
\par 22 Întoarce?i-va, copii razvrati?i, ?i Eu voi vindeca neascultarea voastra! Zice?i: Iata venim la Tine, ca Tu e?ti Domnul Dumnezeul nostru.
\par 23 Cu adevarat în de?ert ne-am pus nadejdea în dealuri ?i în mul?imea mun?ilor; cu adevarat, în Domnul Dumnezeul nostru este mântuirea lui Israel.
\par 24 Din tinere?ea noastra aceasta urâciune a mâncat ostenelile parin?ilor no?tri: oile lor, boii lor, fiii lor ?i fiicele lor.
\par 25 Iar noi zacem în ru?inea noastra ?i ocara noastra ne acopera, pentru ca am pacatuit înaintea Domnului Dumnezeului nostru, ?i noi ?i parin?ii no?tri din tinere?ea noastra ?i pâna în ziua de astazi ?i n-am ascultat glasul Domnului Dumnezeului nostru".

\chapter{4}

\par 1 De vrei sa te întorci, Israele, zice Domnul, întoarce-te la Mine ?i, de vei departa urâciunile de la fa?a Mea, nu vei mai rataci.
\par 2 Daca tu vei jura: "Viu este Domnul!", în adevar, în judecata ?i în dreptate, neamurile se vor binecuvânta ?i se vor lauda în El.
\par 3 Caci a?a zice Domnul catre barba?ia lui Iuda ?i ai Ierusalimului: "Ara?i-va ogoare noi ?i nu mai semana?i prin spini!
\par 4 Barba?i ai lui Iuda ?i locuitori ai Ierusalimului, taia?i-va împrejur pentru Domnul ?i lepada?i învârto?area inimii voastre, ca nu cumva sa izbucneasca mânia Mea ca focul ?i sa arda nestinsa din pricina rauta?ii faptelor voastre.
\par 5 Spune?i acestea în Iuda ?i le vesti?i în Ierusalim! Grai?i ?i trâmbi?a?i cu trâmbi?a prin ?ara! Striga?i tare ?i zice?i:
\par 6 "Aduna?i-va ?i sa intram în cetatea cea întarita!" Înal?a?i steagul spre Sion, fugi?i ?i nu va opri?i, ca iata aduc de la miazanoapte necaz ?i nevoie mare!
\par 7 Iata, iese leul din desi?ul sau ?i pierzatorul popoarelor se apropie; plecat-a din locul sau, ca sa pustiiasca pamântul tau; ceta?ile tale vor fi stricate ?i fara locuitori.
\par 8 De aceea încinge?i-va cu sac, plânge?i ?i va tângui?i, ca iu?imea mâniei Domnului nu se va abate de la voi.
\par 9 În ziua aceea, zice Domnul, va Încremeni inima regelui ?i inima capeteniilor; preo?ii se vor îngrozi ?i proorocii se vor mira".
\par 10 Atunci eu am zis: "O, Doamne Dumnezeule, amagit-ai Tu oare pe poporul acesta ?i Ierusalimul, când ai zis: Ve?i avea pace, ?i iata sabia a ajuns pâna la suflet?"
\par 11 În vremea aceea se va zice poporului acestuia ?i Ierusalimului: "Iata vine vânt arzator din mun?ii cei pustii asupra fiicei poporului Meu ?i vine nu pentru a vântura, nici pentru a cura?i grâul;
\par 12 Dar va veni dintr-acolo de la Mine vânt mai puternic decât acesta ?i voi rosti judecata Mea asupra lor.
\par 13 Iata, se va ridica, cum se ridica norii; caru?ele lui vor fi ca furtuna ?i caii lui mai iu?i decât vulturii". Vai de noi, caci vom fi prapadi?i!
\par 14 Ierusalime, spala raul din inima ta, ca sa te izbave?ti! Pâna când se vor sala?lui în tine cugete necredincioase?
\par 15 Ca iata se aude glas de la Dan ?i vestea pieirii din mun?ii lui Efraim:
\par 16 "Spune?i popoarelor ?i vesti?i Ierusalimului ca din ?ara departata vin împresuratori ?i scot strigate împotriva ceta?ilor lui Iuda".
\par 17 Ca paznicii câmpului, a?a l-au înconjurat pe Israel de jur împrejur, pentru ca el s-a razvratit împotriva Mea, zice Domnul.
\par 18 Caile tale ?i faptele tale, Israele, ?i-au pricinuit acestea; din pricina necredincio?ii tale fi-a venit acest amar, care a strabatut pâna la inima ta".
\par 19 Inima mea! Inima mea! Ma doare inima pâna în adânc! Tulburatu-s-a inima mea în mine ?i nu pot tacea, ca tu, suflete al meu, auzi glasul trâmbi?ei, auzi strigatul de razboi.
\par 20 Nenorocire peste nenorocire: tot pamântul se pustie?te ?i fara de veste mi s-au stricat corturile ?i într-o clipeala sala?urile mele.
\par 21 Oare mult îmi este dat sa vad steagul ?i sa aud sunetul trâmbi?ei?
\par 22 ?i toate acestea sunt numai pentru ca poporul Meu e fara minte ?i nu Ma cunoa?te, sunt copii nepricepu?i ?i n-au în?elegere; sunt pricepu?i numai la rele, iar binele nu ?tiu sa-l faca.
\par 23 Ma uit peste ?ara ?i iata este ruinata ?i pustie;
\par 24 Caut la ceruri ?i iata nu este lumina pe ele; privesc la mun?i ?i iata ca ei tremura ?i dealurile toate se clatina.
\par 25 Ma uit ?i iata nu este nici un om ?i toate pasarile cerului au fugit.
\par 26 Ma uit ?i iata Carmelul este o pustietate ?i toate ceta?ile lui sunt arse cu foc de la fa?a Domnului ?i au pierit de la fa?a mâniei Lui.
\par 27 Ca a?a a zis Domnul: "Toata ?ara va fi pustiita, dar nu o voi nimici de tot.
\par 28 Va plânge de aceasta pamântul ?i cerurile sus se vor întuneca, pentru ca Eu am zis, Eu am hotarât ?i nu Ma voi cai, nici Ma voi întoarce de la aceasta.
\par 29 De strigatele calare?ilor ?i ale arca?ilor ?ara întreaga este pusa pe fuga; to?i vor fugi în padurile cele dese ?i se vor sui pe stânci; toate ceta?ile vor fi parasite ?i nici un locuitor nu va mai fi în acestea.
\par 30 ?i tu, pustiito, ce vei face? Chiar când te-ai îmbraca în purpura, chiar daca te-ai gati cu podoabe de aur ?i ?i-ai sulemeni ochii cu dresuri, în zadar te-ai face frumoasa, ca iubi?ii tai te dispre?uiesc ?i vor numai via?a.
\par 31 Când aud glas ca de femeie ce na?te, aud geamat ca al uneia ce na?te pentru întâia oara: este glasul fiicei Sionului; ea geme ?i întinde mâna, zicând: "O, vai de mine, mi se istove?te sufletul înaintea uciga?ilor!"

\chapter{5}

\par 1 "Cutreiera?i uli?ele Ierusalimului, uita?i-va, cerceta?i ?i cauta?i prin pie?ele lui: nu cumva ve?i gasi vreun om, macar unul, care paze?te dreptatea ?i cauta adevarul?
\par 2 Caci Eu a? cru?a Ierusalimul. Chiar când ei zic: "Viu este Domnul", ei jura mincinos.
\par 3 O, Doamne, ochii Tai nu privesc ei oare la adevar? Tu îi ba?i ?i ei nu simt durerea; Tu îi pierzi ?i ei nu vor sa ia înva?atura; ?i-au facut obrazul mai vârtos ca piatra ?i nu vor sa se întoarca.
\par 4 ?i mi-am zis: "Poate ca ace?tia sunt ni?te bie?i nenoroci?i! Sunt ni?te pro?ti, pentru ca nu cunosc calea Domnului, legea Dumnezeului lor.
\par 5 Voi merge deci la cei mari ?i voi grai cu ace?tia, ca ei ?tiu calea Domnului; legea Dumnezeului lor". Dar ?i ace?tia cu to?ii au sfarâmat jugul ?i au rupt catu?ele.
\par 6 De aceea îi va lovi leul din padure ?i lupul din pustiu îi va rapi; leopardul le va fi pazitor ceta?ilor lor; care din ei va ie?i va fi sfâ?iat, ca s-au înmul?it faradelegile lor ?i lepadarile de credin?a au sporit.
\par 7 Cum, adica, sa te iert, Ierusalime, pentru aceasta? Fiii tai M-au parasit ?i se jura pe dumnezei care n-au fiin?a. Eu i-am saturat, iar ei au facut desfrânare, umblând în grup prin casele desfrânatelor.
\par 8 Ei sunt cai îngra?a?i ?i fiecare din ei necheaza dupa femeia aproapelui sau.
\par 9 E cu putin?a sa nu pedepsesc aceasta, zice Domnul, ?i Duhul Meu sa nu se razbune asupra unui popor ca acesta?
\par 10 Sui?i-va pe zidurile lui ?i le darâma?i, dar nu de tot, ci strica?i numai crestele lor, pentru ca acestea nu sunt ale Domnului;
\par 11 Caci casa lui Israel ?i casa lui Iuda s-au purtat fa?a de Mine cu multa necredin?a, zice Domnul.
\par 12 Au tagaduit pe Domnul ?i au zis: "El nu este ?i nenorocirea nu va veni asupra noastra; ?i nu vom vedea nici sabie, nici foamete!
\par 13 Proorocii sunt vânt ?i nu este în ace?tia cuvântul Domnului. De aceasta ?i ei sa aiba parte!"
\par 14 De aceea, a?a zice Domnul Dumnezeul puterilor: "Pentru ca voi grai?i asemenea vorbe, iata voi face cuvintele Mele foc în gura ta, iar pe poporul acesta îl voi face lemne ?-l va mistui focul acesta.
\par 15 Casa lui Israel, iata voi aduce asupra voastra un neam de departe, zice Domnul, un popor puternic, un popor vechi, un popor a carui limba tu nu o ?tii ?i nu vei în?elege ce graie?te el.
\par 16 Tolba lui e ca un mormânt deschis ?i ai lui to?i sunt viteji;
\par 17 ?i vor mânca aceia seceri?ul tau ?i pâinea ta; vor mânca pe fiii tai ?i pe fiicele tale; vor mânca oile tale ?i boii tai; vor mânca strugurii tai ?i smochinele tale ?i vor trece prin sabie ceta?ile tale cele întarite în care tu te încrezi.
\par 18 Dar nici în zilele acelea nu va voi pierde cu totul, zice Domnul.
\par 19 ?i de ve?i zice: Pentru ce ne face Domnul Dumnezeul nostru toate acestea? Atunci sa ?i se raspunda: Pentru ca M-a?i parasit pe Mine ?i a?i slujit la dumnezei straini, în ?ara voastra, de aceea ve?i sluji la dumnezei straini într-o ?ara care nu este a voastra.
\par 20 Spune?i aceasta în casa lui Iacov, vesti?i-o în Iuda ?i zice?i:
\par 21 Asculta?i acestea, popor nebun ?i fara inima! Ei au ochi ?i nu vad, urechi au, dar nu aud.
\par 22 Au doar nu va teme?i de Mine, zice Domnul, ?i nu tremura?i înaintea Mea? Eu am pus nisipul hotar împrejurul marii ?i hotar ve?nic, peste care nu se va trece. De?i valurile ei se înfurie, nu pot sa-l biruiasca ?i, de?i ele se arunca, nu pot sa-l treaca.
\par 23 Dar poporul acesta are inima îndârjita ?i razvratita:
\par 24 Razvratitu-s-au ?i s-au dus ?i n-au zis în inima lor: "Sa ne temem de Domnul Dumnezeul nostru, Care ne da la vreme ploaie timpurie ?i târzie ?i ne pastreaza saptamânile hotarâte ale culesului".
\par 25 Faradelegile voastre au schimbat aceasta ?i pacatele voastre au departat acest bine de la voi.
\par 26 Ca se afla necredincio?i prin poporul Meu, care pândesc ca prinzatorii de pasari, se ascund la pamânt, pun curse ?i prind pe oameni.
\par 27 Cum sunt cote?ele pline de pasari, a?a sunt casele lor pline de în?elatorie;
\par 28 Prin aceasta s-au ridicat ?i s-au îmboga?it ei, s-au facut gra?i ?i cu pielea lucioasa ?i în rele au trecut orice masura;
\par 29 Nu fac dreptate nimanui, nici chiar orfanului, nu dau dreptate saracului ?i huzuresc. E cu putin?a oare sa nu pedepsesc acestea ?i sa nu Ma razbun asupra unui popor ca acesta? - zice Domnul.
\par 30 Lucruri înspaimântatoare se petrec în ?ara aceasta:
\par 31 Proorocii profe?esc minciuni, preo?ii înva?a ca ?i ei, ?i poporului Meu îi place aceasta. Dar la urma ce ve?i face?"

\chapter{6}

\par 1 Fugi?i, fiii lui Veniamin, fugi?i din Ierusalim, trâmbi?a?i cu trâmbi?a în Tecoa ?i da?i semne prin focuri la Bethacherem, ca iata se ive?te de la miazanoapte o nenorocire ?i zdrobire mare!
\par 2 Pierde-voi pe fiica Sionului cea frumoasa ?i ginga?a.
\par 3 Pastorii vor veni la ea cu turmele lor, î?i vor întinde corturile împrejurul ei ?i va pa?te fiecare partea sa.
\par 4 Pregati?i razboi împotriva ei! Scula?i-va ?i haide?i spre miazazi! Vai! Ziua este spre sfâr?it ?i iata se lasa umbrele serii.
\par 5 Dar scula?i-va ?i hai sa mergem noaptea ?i sa stricam palatele ei.
\par 6 Caci a?a zice Domnul Savaot: "Taia?i copaci ?i face?i val împotriva Ierusalimului; aceasta cetate trebuie pedepsita, pentru ca în ea se afla numai nedreptate.
\par 7 Cum arunca izvorul apa din sine, a?a arunca ?i ea din sine rautate; în ea se aude împilare ?i jaf ?i pururea se vad înaintea fe?ei Mele dureri ?i rani.
\par 8 În?elep?e?te-te, Ierusalime, ca sa nu se departeze sufletul Meu de la tine ?i ca sa nu te fac pustietate ?i pamânt nelocuit".
\par 9 A?a zice Domnul Savaot: "Pâna la sfâr?it  se vor culege rama?i?ele lui Israel, cum se culege via; lucreaza cu mâna ta ?i î?i umple panerul, ca ?i culegatorul de struguri.
\par 10 Cu cine sa vorbesc ?i cui sa vestesc, ca sa auda? Ca iata urechea lor este netaiata împrejur ?i nu pot sa ia aminte; ?i iata, cuvântul Domnului a ajuns de râs la ei ?i nu gasesc în el nici o placere.
\par 11 De aceea sunt plin de mânia Domnului ?i n-o mai pot ?ine în mine; o voi varsa deci asupra copiilor pe uli?e ?i asupra adunarii tinerilor, ca vor fi lua?i ?i barbat ?i femeie, ?i cel în vârsta ?i cel încarcat de zile;
\par 12 ?i casele lor vor trece la al?ii; tot a?a ?i ?arinele ?i femeile. Pentru ca voi întinde mâna Mea asupra locuitorilor ?arii acesteia, zice Domnul,
\par 13 Pentru ca fiecare din ei, de la mic pâna la mare, este robit de lacomie ?i, de la prooroc pâna la preot, to?i se poarta mincinos.
\par 14 Ei leaga ranile poporului meu cu nepasare ?i zic: "Pace! Pace!" ?i numai pace nu este!
\par 15 Dar se ru?ineaza ei, oare, când fac urâciuni? Nu, nu se ru?ineaza deloc, nici ro?esc. De aceea vor cadea printre cei cazu?i ?i se vor prabu?i în ziua în care îi voi pedepsi, zice Domnul.
\par 16 A?a zice Domnul: "Opri?i-va de la caile voastre! Privi?i ?i întreba?i de caile celor de demult; de calea cea buna ?i merge?i pe dânsa ?i ve?i afla odihna sufletelor voastre.
\par 17 Pus-am pazitori peste voi ?i am zis: Asculta?i sunetul trâmbi?ei!
\par 18 Iar ei au zis: Nu vom asculta! A?adar, asculta, poporule, ?i afla, adunare, ce are sa se întâmple cu ace?tia.
\par 19 Asculta, pamântule: Iata voi aduce asupra acestui popor o nenorocire, rodul cugetelor lor, ca n-au ascultat cuvintele Mele ?i legea Mea au lepadat-o.
\par 20 La ce Îmi este buna tamâia care vine din ?eba ?i scor?i?oara din ?ara departata? Arderile de tot ale voastre nu le voiesc ?i jertfele voastre Îmi sunt neplacute".
\par 21 De  aceea a?a zice Domnul: "Iata pun înaintea poporului acestuia piedici ?i se vor poticni de ele deodata ?i parin?ii ?i copiii, vecinul ?i prietenii lui, ?i vor pieri".
\par 22 A?a zice Domnul: "Iata vine un popor din ?ara de la miazanoapte, un popor mare se ridica de la marginile pamântului ?i ai sai ?in în mâna arcul ?i suli?a.
\par 23 ?i sunt cruzi ?i neîndura?i; glasul lor muge?te ca marea ?i vin pe cai, gata sa lupte ca un singur om, împotriva ta, fiica Sionului".
\par 24 Noi am auzit de ei ?i ne-au slabit mâinile de spaima; ne-au cuprins groaza ?i dureri ca ale femeii ce na?te.
\par 25 Sa nu ie?i?i la câmp, nici la drum sa nu plcca?i, caci sabia du?manilor ?i groaza sunt pretutindeni.
\par 26 Fiica poporului Meu, încinge-te cu sac ?i-?i presara cenu?a pe cap; tânguie?te-te ca dupa moartea singurului tau fiu! Plângi amar, ca fara de veste va veni pierzatorul asupra voastra!
\par 27 Turn te-am pus în mijlocul poporului Meu ?i stâlp, ca sa afli ?i sa urmare?ti drumul lor.
\par 28 Ace?tia cu to?ii sunt razvrati?i, îndârji?i ?i semanatori de clevetiri; sunt arama ?i fier, to?i sunt ni?te strica?i.
\par 29 Foalele s-au ars, plumbul s-a topit de foc ?i turnatorul în zadar a topit, ca cei rai nu s-au ales.
\par 30 Argint de lepadat se vor numi, ca Domnul i-a lepadat".

\chapter{7}

\par 1 Cuvântul ce a fost de la Domnul catre Ieremia: "Stai în u?a templului Domnului ?i roste?te acolo cuvântul acesta ?i zi:
\par 2 Asculta?i cuvântul Domnului, to?i barba?ii lui Iuda, care intra?i pe aceasta poarta ca sa va închina?i Domnului!
\par 3 A?a zice Domnul Savaot, Dumnezeul lui Israel: Îndrepta?i-va caile ?i faptele voastre ?i va voi lasa sa trai?i în locul acesta!
\par 4 Nu va încrede?i în cuvintele mincinoase care zic: "Acesta este templul Domnului, templul Domnului, templul Domnului".
\par 5 Iar daca va ve?i îndrepta cu totul caile ?i faptele voastre, daca ve?i face judecata cu dreptate între om ?i pârâ?ul lui,
\par 6 Daca nu ve?i strâmtora pe strain, pe orfan ?i pe vaduva, nu ve?i varsa sânge nevinovat în locul acesta ?i nu ve?i merge dupa al?i dumnezei spre pieirea voastra,
\par 7 Atunci va voi lasa sa trai?i în locul acesta ?i pe pamântul acesta, pe care l-am dat parin?ilor vo?tri din neam în neam.
\par 8 Iata, va încrede?i în cuvinte mincinoase, care nu va vor aduce folos.
\par 9 Cum? Voi fura?i, ucide?i ?i face?i adulter; jura?i mincinos, tamâia?i pe Baal ?i umbla?i dupa al?i dumnezei, pe care nu-i cunoa?te?i,
\par 10 ?i apoi veni?i sa va înfa?i?a?i înaintea Mea în templul Meu, asupra caruia s-a chemat numele Meu, ?i zice?i: "Suntem izbavi?i", ca apoi sa face?i iar toate ticalo?iile acelea?
\par 11 Templul acesta, asupra caruia s-a chemat numele Meu, n-a ajuns el oare, în ochii vo?tri pe?tera de tâlhari? Iata, Eu am vazut aceasta, zice Domnul.
\par 12 Mergeri deci la locul Meu din Silo, unde facusem altadata sa locuiasca numele Meu, ?i vede?i ce am facut Eu cu el, pentru necredin?a poporului Meu Israel!
\par 13 ?i acum, de vreme ce a?i facut toate faptele acestea, zice Domnul, ?i Eu v-am grait dis-de-diminea?a ?i n-a?i ascultat, v-am chemat ?i n-a?i raspuns,
\par 14 De aceea ?i cu templul acesta, asupra caruia s-a chemat numele Meu ?i în care voi va pune?i încrederea, ?i cu locul pe care vi l-am dat voua ?i parin?ilor vo?tri, voi face tot a?a, cum am facut cu ?ilo;
\par 15 Va voi lepada de la fa?a Mea, cum am lepadat pe to?i fra?ii vo?tri, toata semin?ia lui Efraim.
\par 16 Tu însa nu te ruga pentru acest popor ?i nu înal?a rugaciune ?i cerere pentru dân?ii, nici nu mijloci înaintea Mea, ca nu te voi asculta.
\par 17 Nu vezi tu ce fac ei prin ceta?ile lui Iuda ?i pe uli?ele Ierusalimului?
\par 18 Copiii aduna lemne, iar parin?ii a?â?a focul ?i femeile framânta aluatul ca sa faca turte pentru zei?a cerului ?i sa savâr?easca turnari în cinstea altor dumnezei, ca sa Ma raneasca pe Mine.
\par 19 Dar oare pe Mine Ma ranesc ei - zice Domnul - ?i nu pe ei în?i?i, spre ru?inea lor?"
\par 20 De aceea, a?a zice Domnul Dumnezeu: Iata se revarsa mânia Mea asupra locului acestuia, asupra oamenilor ?i dobitoacelor, asupra copacilor, asupra ?arinii ?i asupra roadelor pamântului, ?i se vor aprinde ?i nu se vor mai stinge".
\par 21 A?a zice Domnul Savaot, Dumnezeul lui Israel: "Arderile de tot ale voastre adauga?i-le la jertfele voastre ?i mânca?i carne;
\par 22 Ca parin?ilor vo?tri nu le-am vorbit ?i nu le-am dat porunca în ziua aceea, în care i-am scos din pamântul Egiptului, pentru arderea de tot ?i pentru jertfa;
\par 23 Ci iata porunca pe care ?i-am dat-o: Sa asculta?i glasul Meu, ?i Eu voi fi Dumnezeul vostru, iar voi Îmi ve?i fi poporul Meu, ?i sa umbla?i pe toata calea pe care va poruncesc Eu, ea sa va fie bine.
\par 24 Dar ei n-au ascultat glasul Meu ?i nu ?i-au plecat urechea lor, ci au trait dupa pofta ?i îndaratnicia inimii lor rele ?i s-au întors cu spatele catre Mine, iar nu cu fa?a.
\par 25 Din ziua când parin?ii vo?tri au ie?it din pamântul Egiptului ?i pâna în ziua aceasta am trimis la voi pe to?i robii Mei - proorocii - ?i i-am trimis în fiecare zi dis-de-diminea?a;
\par 26 Dar ei nu M-au ascultat ?i nu ?i-au plecat urechea lor, ci ?i-au învârto?at cerbicia ?i s-au purtat mai rau decât parin?ii lor.
\par 27 ?i când le vei vorbi cuvintele acestea, ei nu te vor asculta; ?i când îi vei chema, nu-?i vor raspunde.
\par 28 Atunci sa le zici: Iata un popor care nu asculta glasul Domnului Dumnezeului sau ?i nu prime?te înva?atura! Credin?a nu mai este; ea a disparut din gura lor.
\par 29 Tunde-?i parul tau ?i-l arunca, ?i ridica plângere pe mun?i, ca a lepadat Domnul ?i a parasit pe neamul care ?i-a atras mânia Sa.
\par 30 Ca fiii lui Iuda fac rele înaintea ochilor Mei, zice Domnul, ?i ?i-au pus urâciunile lor în templul asupra caruia s-a chemat numele Meu ca sa-l pângareasca;
\par 31 ?i au zidit locurile înalte la Tofet, în valea fiilor lui Hinom, ca sa-?i arda fiii ?i fiicele cu foc, ceea ce Eu nu le-am poruncit ?i ceea ce Mie nu Mi-a trecut prin minte.
\par 32 De aceea, iata vin zile, zice Domnul, când locul acesta nu se va mai chema Tofet ?i valea fiilor lui Hinom, ci valea uciderii, ?i în Tofet se vor face înmormântari din pricina lipsei de loc.
\par 33 ?i vor fi trupurile poporului acestuia hrana pasarilor cerului ?i fiarelor pamântului, ?i nu va fi cine sa le alunge.
\par 34 ?i voi curma în ceta?ile lui Iuda ?i pe uli?ele Ierusalimului glasul de bucurie ?i glasul de veselie, glasul mirelui ?i glasul miresei, pentru ca pamântul acesta va fi pustiu".

\chapter{8}

\par 1 "În vremea aceea, zice Domnul, oasele regilor lui Iuda ?i oasele capeteniilor lui, oasele preo?ilor ?i proorocilor ?i oasele locuitorilor Ierusalimului vor fi aruncate din mormintele lor,
\par 2 ?i vor fi aruncate înaintea soarelui ?i a lunii ?i înaintea întregii o?tiri cere?ti, pe care ei le-au iubit ?i carora au slujit ?i pe urma carora au umblat, pe care le-au cautat ?i carora s-au închinat. Nimeni nu le va aduna, nici le va îngropa, ci vor zacea ca gunoiul pe pamânt.
\par 3 ?i to?i ceilal?i care vor ramâne din acest neam rau vor dori moartea în locul vie?ii, în toate locurile, pe unde îi voi izgoni, zice Domnul.
\par 4 Sa le mai spui de asemenea: A?a zice Domnul: Oare cei ce cad nu se mai scoala, ?i cei ce ratacesc drumul nu se mai întorc?
\par 5 De ce dar poporul acesta, Ierusalime, staruie în ratacire? De ce ?ine tare la minciuna ?i nu vrea sa se întoarca?
\par 6 Privit-am ?i am ascultat, ?i iata-i nu spun adevarul, nu se caiesc de necredin?a lor ?i nimeni nu zice: "Vai, ce am facut?" Fiecare se întoarce la calea sa, ca un cal ce se arunca în batalie.
\par 7 Pâna ?i barza î?i ?tie vremea sa hotarâta sub cer; ?i turturica ?i rândunica ?i cocorul iau aminte la timpul când trebuie sa vina;
\par 8 Iar poporul Meu nu cunoa?te hotarârea Domnului. Cum pute?i voi sa zice?i: "Suntem în?elep?i ?i avem legea Domnului?" Caci iata, pana cea mincinoasa a carturarilor a prefacut-o în minciuna.
\par 9 S-au facut de ocara în?elep?ii, au turbat ?i s-au prins în curse; iata, au lepadat cuvântul Domnului ?i atunci unde este în?elepciunea lor?
\par 10 De aceea, pe femeile lor le voi da altora ?i ogoarele lor le voi trece altor stapânitori, pentru ca ei cu to?ii, de la mic pâna la mare, se dedau la jaf, ?i de la prooroc pâna la preot, to?i în?ala,
\par 11 ?i leaga rana fiicei poporului Meu cu nepasare, zicând: "Pace, pace!" ?i pace nu este.
\par 12 Se ru?ineaza ei, oare, când fac ticalo?ii? Nu se ru?ineaza deloc, nici nu ro?esc. De aceea vor cadea printre cei ce cad ?i se vor prabu?i când îi voi pedepsi, zice Domnul.
\par 13 Îi voi culege cu totul, zice Domnul, ?i nu va ramâne nici o bobi?a pe vi?a ?i nici o smochina în smochin, ?i va cadea ?i frunza ?i, ceea ce le-am dat Eu, se va duce de la ei".
\par 14 De ce ?edem noi oare? Aduna?i-va ?i haide?i în ceta?ile cele întarite, ca sa pierim acolo; ca Domnul Dumnezeul nostru ne-a hotarât la pieire ?i ne da sa bem apa cu fiere, pentru ca am gre?it înaintea Domnului.
\par 15 A?teptam pace ?i iata nu este nici un bine; a?teptam timpul vindecarii, ?i iata groaza.
\par 16 De la Dan se aude ropotul cailor ?i de nechezatul puternic al armasarilor se cutremura tot pamântul; iata vin ?i vor pustii pamântul ?i tot ce este pe el, cetatea ?i locuitorii ei.
\par 17 "Caci iata voi trimite asupra voastra ?erpi ?i scorpii, împotriva carora nu este descântec, ?i va vor mu?ca, zice Domnul.
\par 18 Când Ma voi mângâia Eu de scârba Mea? Mi s-a amarât inima în Mine.
\par 19 Iata, aud plânsul fiicei poporului Meu din ?ara departata, zicând: "Oare nu mai este Domnul în Sion? Oare acesta nu mai are asupra sa pe regele sau?" - "De ce M-au împins ei oare la mânie cu idolii lor cei straini ?i de nimic?" - zice Domnul.
\par 20 Seceri?ul a trecut, vara este pe sfâr?ite ?i noi tot nu suntem izbavi?i:
\par 21 De durerea fiicei poporului meu sunt îndurerat, umblu posomorât ?i groaza m-a cuprins.
\par 22 Au doara nu mai este balsam în Galaad? Au doara nu mai este acolo doctor? De ce dar nu se vindeca fiica poporului Meu?

\chapter{9}

\par 1 O, cine va da capului meu apa ?i ochilor mei izvoare de lacrimi, ca sa plâng ziua ?i noaptea pe cei lovi?i ai fiicei poporului meu?
\par 2 O, de mi-ar da cineva un adapost de calatori în pustiu, a? parasi pe poporul meu ?i m-a? duce de la ei, caci eu to?ii sunt ni?te desfrâna?i ?i ceata de defaimatori!
\par 3 "Ca un arc î?i încordeaza limbile lor pentru minciuna ?i se întaresc pe pamânt prin nedreptate, ca trec de la rau la mai rau ?i pe Mine nu Ma cunosc, zice Domnul.
\par 4 Pazi?i-va fiecare de prietenul vostru ?i nu va încrede?i în nici unul din fra?ii vo?tri, ca fiecare frate pune piedica celuilalt ?i fiecare prieten împra?tie clevetiri.
\par 5 Fiecare în?ala pe prietenul sau ?i nu spune adevarul; ?i-au deprins limba la minciuna ?i viclenesc pâna obosesc.
\par 6 Tu traie?ti în mijlocul viclenilor ?i ei din pricina vicleniei nu Ma cunosc pe Mine", zice Domnul.
\par 7 De aceea, a?a zice Domnul Savaot: "Îi voi topi ?i-i voi încerca; oare ce altceva pot sa fac Eu cu fiica poporului Meu?
\par 8 Limba lor este sageata ucigatoare ?i graie?te viclenii; cu buzele lor graiesc prietenos catre aproapele lor, iar în inima ei fauresc catu?e.
\par 9 E cu putin?a, oare, sa nu-i pedepsesc pentru aceasta, zice Domnul, ?i sufletul Meu sa nu se razbune pe un popor ca acesta?
\par 10 Pentru mun?i voi ridica plângere ?i bocet ?i pentru pa?unile pustiului ma voi tângui, pentru ca vor fi arse, ?i nimeni nu va mai umbla pe acolo ?i nu se va mai auzi glasul turmelor; de la pasarile cerului ?i pâna la animale toate s-au împra?tiat ?i s-au dus!
\par 11 ?i voi face Ierusalimul o movila de pietre ?i sala? ?acalilor, iar ceta?ile lui Iuda le voi face pustietate fara locuitori".
\par 12 Cine este în?eleptul, care sa priceapa, ?i cui a grait gura Domnului, ca sa spuna pentru ce a pierit ?ara ?i a fost arsa ca un pustiu, încât nimeni sa nu mai treaca prin ea?
\par 13 ?i Domnul a raspuns: "Pentru ca au parasit Legea Mea, pe care le-am pus-o Eu, ?i n-au ascultat glasul Meu, nici nu s-au purtat cum le poruncea acel glas,
\par 14 Ci au umblat dupa îndaratnicia inimii lor ?i în urma lui Baal, cum i-au înva?at parin?ii lor".
\par 15 De aceea, a?a zice Domnul Savaot, Dumnezeul lui Israel: "Iata, îi voi hrani cu pelin ?i le voi da sa bea apa cu fiere, ?i-i voi risipi printre popoarele
\par 16 Pe care nu le-au cunoscut nici ei, nici parin?ii lor, ?i voi trimite pe urma lor sabie pâna îi voi pierde".
\par 17 A?a zice Domnul Savaot: "Gândi?i-va ?i chema?i bocitoare ca sa boceasca; trimite?i la cele mai iste?e în asemenea lucru, ca sa vina!"
\par 18 Sa se grabeasca dar a înal?a o cântare de jale pentru noi, ca sa curga lacrimi din ochii no?tri ?i din genele noastre sa curga apa.
\par 19 Caci glas de plângere se aude din Sion, zicând: "Cât suntem de prapadi?i ?i cât de cumplit batjocori?i, ca trebuie sa parasim ?ara, pentru ca locuin?ele noastre au fost darâmate!
\par 20 A?adar, femei, asculta?i cuvântul Domnului ?i sa ia aminte urechea voastra la cuvântul gurii Lui! înva?a?i pe fiicele voastre a boci ?i una pe alta sa se înve?e cântece de bocet,
\par 21 Ca iata moartea intra pe ferestrele noastre ?i da navala în casele noastre, ca sa piarda pe copiii din uli?a ?i pe tinerii din pie?e;
\par 22 ?i vor cadea trupurile oamenilor ca gunoiul aruncat pe ogor ?i ca snopii în urma seceratorului, ?i nu va fi cine sa-i strânga".
\par 23 A?a zice Domnul: "Sa nu se laude cel în?elept cu în?elepciunea sa, sa nu se laude cel puternic cu puterea sa, nici cel bogat sa nu se laude cu boga?ia sa;
\par 24 Ci de se lauda cineva, sa se laude numai cu aceea ca pricepe; ?i Ma cunoa?te ca Eu sunt Domnul, Cel ce fac mila ?i judecata ?i dreptate pe pamânt, caci numai aceasta este placut înaintea Mea, zice Domnul.
\par 25 Iata, vin zile, zice Domnul, când voi cerceta pe to?i cei taia?i ?i netaia?i împrejur:
\par 26 Egiptul ?i Iuda, Edomul ?i pe fiii lui Amon, Moabul ?i pe to?i locuitorii pustiului, care-?i tund parul împrejurul frun?ii lor, caci toate aceste popoare sunt netaiate împrejur, iar casa lui Israel toata este cu inima netaiata împrejur".

\chapter{10}

\par 1 Casa lui Israel, asculta?i cuvântul ce vi-l graie?te Domnul!
\par 2 A?a zice Domnul: "Nu deprinde?i caile neamurilor ?i nu va îngrozi?i de semnele cerului, de care se îngrozesc neamurile;
\par 3 Ca datinile neamurilor sunt de?ertaciune: aceia taie un lemn din padure ?i mâinile me?terului îl lucreaza cu toporul;
\par 4 Apoi îl îmbraca cu argint ?i cu aur, îl întaresc cu cuie ?i cu ciocanul, ca sa nu se clatine. A?a sunt dumnezeii neamurilor.
\par 5 Stau ca ni?te sperietori într-o gradina cu pepeni ?i nu graiesc, sunt purta?i pentru ca nu pot merge. Nu va teme?i de ei, ca nu pot face rau, dar nici bine nu sunt în stare sa faca".
\par 6 Nimeni nu este ca Tine, Doamne! Mare e?ti Tu ?i mare este puterea numelui Tau!
\par 7 Cine nu se va teme de Tine, Împarate al neamurilor? Numai ?ie unuia se cuvine aceasta, pentru ca printre to?i în?elep?ii neamului ?i în toate regatele lor nu este nimeni asemenea ?ie.
\par 8 Dumnezeii neamurilor, to?i pâna la unul, sunt fara minte ?i fara pricepere; lemne lipsite de orice în?elegere;
\par 9 Argint prefacut în foi ?i adus din Tarsis; aur din Ofir; lucruri de me?ter ?i de mâna de turnator; haina de pe ei este de iacint ?i purpura: iara?i lucru de oameni iscusi?i.
\par 10 Iar Domnul este adevaratul Dumnezeu, este Dumnezeu viu ?i împarat ve?nic; de mânia Lui tremura pamântul ?i neamurile nu pot suferi urgia Lui.
\par 11 A?adar sa zice?i neamurilor: "Dumnezeii, care n-au facut cerul ?i pamântul, vor pieri de pe pamânt ?i de sub ceruri;
\par 12 Iar Domnul a facut cerul cu puterea Sa, a întarit lumea cu în?elepciunea Sa ?i cu priceperea Sa a întins cerurile.
\par 13 La glasul Lui freamata apele în ceruri ?i El ridica norii de la marginile pamântului, faure?te fulgerele în mijlocul ploii ?i scoate vânturile din vistieriile Sale.
\par 14 Atunci se vede cât este de ne?tiutor omul, cu toata ?tiin?a lui, ?i orice argintar se ru?ineaza de idolul sau, caci chipul turnat de el nu este decât minciuna; n-are nici o suflare în el.
\par 15 Aceasta este de?ertaciune adevarata, rodul ratacirii, ?i la vremea pedepsei va pieri.
\par 16 Iar soarta lui Iacov nu este ca a neamurilor, ca Dumnezeul lui este facatorul a toate, iar Israel este toiagul mo?tenirii Lui; ?i numele Lui este Domnul Savaot.
\par 17 O, tu, care e?ti împresurata, strânge-?i de pe pamânt avu?ia, ca a?a zice Domnul:
\par 18 "Iata de data aceasta voi arunca pe locuitorii ?arii acesteia ?i-i voi mâna la lac strâmt, ca sa fie prin?i".
\par 19 Atunci vei grai: "Vai mie din pricina ranii mele! Rana mea e dureroasa; însa îmi zic: Aceasta este nenorocirea mea, dar o voi îndura!
\par 20 Cortul îmi este pustiit ?i toate funiile lui sunt rupte; fiii mei m-au parasit ?i nu mai sunt, ?i n-are cine sa-mi întinda cortul ?i sa-mi ridice pânzele.
\par 21 Pentru ca pastorii ?i-au ie?it din minte ?i n-au cautat pe Domnul ?i de aceea s-au ?i purtat ei nebune?te ?i toata turma ei s-a risipit".
\par 22 Iata, vine vuiet ?i zgomot mare din partea de miazanoapte ca sa pustiiasca ceta?ile lui Iuda ?i sa le faca locuin?a ?acalilor.
\par 23 ?tiu, Doamne, ca nu este în voia omului calea lui, nici în voia celui ce merge, putin?a sa-?i îndrepte pa?ii sai.
\par 24 Pedepse?te-ma, Doamne, dar dupa dreptate ?i nu întru mânia Ta, ca sa nu ma mic?orezi.
\par 25 Varsa-?i iu?imea Ta asupra neamurilor care nu Te cunosc ?i asupra popoarelor care nu cheama numele Tau, ca acelea au mâncat pe Iacov, l-au mistuit ?i l-au stins ?i locuin?ele lui le-au pustiit.

\chapter{11}

\par 1 Cuvântul ce a fost de la Domnul catre Ieremia:
\par 2 "Asculta cuvintele legamântului acestuia ?i spune barba?ilor lui Iuda ?i celor ce locuiesc în Ierusalim ?i le zi:
\par 3 A?a zice Domnul Dumnezeul lui Israel: Blestemat sa fie omul care nu asculta cuvintele acestui legamânt,
\par 4 Pe care l-am dat Eu parin?ilor vo?tri când i-am scos din pamântul Egiptului ?i din cuptorul cel de fier ?i le-am zis: Asculta?i glasul Meu ?i face?i tot ce v-am poruncit, ?i ve?i fi poporul Meu, iar Eu voi fi Dumnezeul vostru,
\par 5 Ca sa împlinesc juramântul cu care M-am jurat parin?ilor vo?tri, ca le voi da o ?ara în care curge lapte ?i miere, cum este astazi". ?i eu am raspuns: "A?a, Doamne!"
\par 6 Iar Domnul mi-a zis iara?i: "Veste?te toate cuvintele acestea în ceta?ile lui Iuda ?i pe uli?ele Ierusalimului ?i zi: asculta?i cuvintele lgamântului acestuia ?i le împlini?i!
\par 7 Caci am în?tiin?at necontenit pe parin?ii vo?tri, din ziua în care i-am scos din ?ara Egiptului ?i pâna astazi; i-am în?tiin?at fara încetare ?i le-am zis: Asculta?i glasul Meu!
\par 8 Dar ei n-au ascultat ?i nu ?i-au plecat urechea, ci au umblat fiecare dupa inima sa cea rea ?i îndaratnica. De aceea am adus asupra lor toate cele spuse în legamântul acesta pe care l-am poruncit sa-l ?ina, ?i nu l-au ?inut".
\par 9 ?i iara?i mi-a zis Domnul: "Între barba?ii lui Iuda ?i locuitorii Ierusalimului se pune la cale razvratire;
\par 10 Ei s-au întors iara?i la faradelegile stramo?ilor lor, care n-au voit sa asculte cuvintele Mele ?i au mers dupa dumnezei straini, slujind acelora. Casa lui Israel ?i casa lui Iuda au calcat legamântul Meu, pe care l-am încheiat cu parin?ii lor.
\par 11 De aceea, a?a zice Domnul: Iata voi aduce asupra lor nenorociri, de care nu se vor putea izbavi ?i, când vor striga catre Mine, nu-i voi auzi.
\par 12 Atunci ceta?ile lui Iuda ?i locuitorii Ierusalimului vor alerga ?i vor striga catre dumnezeii pe care-i tamâiaza, dar aceia nu le vor ajuta în vremea nenorocirii lor.
\par 13 Caci câte ceta?i ai, tot atâ?ia dumnezei ai, Iuda, ?i câte uli?i sunt în Ierusalim, tot atâtea sunt ?i jertfelnicele celui de ru?ine, jertfelnicele pentru tamâierea lui Baal.
\par 14 Tu însa nu te ruga pentru poporul acesta ?i nu înal?a pentru e! rugaciune ?i cereri, caci nu voi auzi când vor striga catre Mine în vremea nenorocirii lor.
\par 15 Ce cau?i, iubitul Meu popor, în templul Meu, când în acesta se savâr?esc atâtea netrebnicii? Jertfele nu-li vor ajuta când, facând rau, te bucuri.
\par 16 Maslin verde, împodobit cu roade frumoase, te-a numit Domnul. Iar acum în zgomotul cumplitei tulburari a aprins foc împrejurul lui ?i ramurile lui s-au stricat.
\par 17 Domnul Savaot, Cel ce te-a sadit, a hotarât asupra ta nenorocirea pentru rautatea pe care casa lui Israel ?i casa lui Iuda ?i-au pricinuit-o singure, împingându-Ma la mânie prin tamâierea lui Baal".
\par 18 Domnul mi-a descoperit ca sa ?tiu ?i mi-a aratat faptele lor.
\par 19 Eu insa, ca un miel blând, dus la junghiere, nici nu ?tiam ca ei urzesc gânduri rele împotriva mea, zicând: "Sa-i punem lemn otravit în mâncarea lui ?i sa-l smulgem de pe pamântul celor vii, pentru ca nici numele sa nu i se mai pomeneasca".
\par 20 Dar Tu, Doamne al puterilor, Judecatorul cel drept, Care cercetezi inimile ?i rarunchii, da-mi sa vad razbunarea Ta asupra lor, pentru ca ?i-am încredin?at pricina mea.
\par 21 Pentru aceasta, a?a zice Domnul despre oamenii din Anatot care cauta sufletul meu ?i zic: "Nu mai prooroci în numele Domnului, ca sa nu mori de mâinile noastre!"
\par 22 De aceea, a?a zice Domnul Savaot: "Iata, îi voi pedepsi ?i tinerii lor vor muri de sabie, iar fiii lor ?i fiicele lor vor muri de foame;
\par 23 Nu va scapa niciunul, din ei, caci voi aduce nenorocirea asupra oamenilor din Anatot, în anul când îi voi pedepsi.

\chapter{12}

\par 1 De voi intra la judecata cu Tine, Doamne, dreptatea va fi de partea Ta, ?i totu?i, despre dreptate vreau sa graiesc cu Tine: "Pentru ce calea necredincio?ilor este cu izbânda ?i pentru ce to?i calcatorii de lege sunt în fericire?
\par 2 Tu i-ai sadit ?i ei au prins radacini, au crescut ?i au facut roade; Tu e?ti aproape numai de buzele lor, iar de inima lor e?ti departe.
\par 3 Pe mine însa ma cuno?ti, Doamne, ma vezi ?i cercetezi daca inima mea este cu Tine. Osebe?te-i dara, ca pe ni?te oi de junghiat, ?i pregate?te-i pentru ziua junghierii.
\par 4 Pâna când va jeli ?ara ?i iarba de prin toate ?arinile se va usca? Dobitoacele ?i pasarile pier pentru necredin?a locuitorilor ei, ca ace?tia zic: "Domnul nu vede caile noastre!"
\par 5 "Daca alergând cu cei ce merg pe jos, ai ostenit, cum te vei lua la întrecere cu caii? ?i daca nu e?ti în siguran?a intr-o ?ara pa?nica, ce vei face în ha?i?urile Iordanului?
\par 6 Ca ?i fra?ii tai ?i casa tatalui tau se poarta necredincios cu tine ?i striga tare în urma ta. ?i chiar când î?i graiesc bine nu te încrede în ei", zice Domnul.
\par 7 "Parasit-am casa Mea ?i mo?tenirea Mea am lasat-o; dat-am pe iubita sufletului Meu în mâinile vrajma?ilor ei.
\par 8 Facutu-s-a mo?tenirea Mea pentru Mine ca un leu din padure, ridicându-?i glasul împotriva Mea, ?i de aceea am urât-o.
\par 9 Mo?tenirea Mea s-a facut pentru Mine ca o pasare de prada pestri?a, asupra careia au navalit din toate par?ile celelalte pasari de prada. Strânge?i-va ?i va duce?i toate fiarele câmpului, duce?i-va ?i o mânca?i.
\par 10 Mul?ime de pastori au calcat via Mea, calcat-au cu picioarele lor partea Mea; partea Mea cea iubita au facut-o de?ert neroditor.
\par 11 Pustietate au facut-o, ?i ea, în pustiirea sa, plânge înaintea Mea. Toata ?ara e pustiita, pentru ca pe nici un om nu-l doare inima de aceasta.
\par 12 Pe toate dealurile din pustiu au venit pradatori, ca sabia Domnului mistuie totul de la o margine la alta a ?arii ?i nici un muritor n-are pace.
\par 13 Semanat-au grâu ?i au secerat spini! Muncit-au ?i n-au avut nici un folos! Ru?ina?i-va dar de asemenea venituri ale voastre, pe care le ave?i din pricina mâniei celei aprinse a Domnului!"
\par 14 A?a zice Domnul despre to?i vecinii mei cei rai, care napadesc asupra par?ii pe care El a dat-o de mo?tenire poporului Sau Israel: "Iata, îi voi smulge din ?ara lor ?i casa lui Iuda o voi înlatura din mijlocul lor.
\par 15 Dar dupa ce îi voi smulge, iara?i ,i voi întoarce ?i-i voi milui ?i voi aduce pe fiecare la ?arina sa ?i la ogorul sau.
\par 16 ?i daca vor înva?a ei caile poporului Meu, ca sa jure pe numele Meu, zicând: "Viu este Domnul", cum au înva?at pe poporul Meu sa jure pe Baal, atunci vor fi a?eza?i în mijlocul poporului Meu.
\par 17 Iar de nu vor asculta, atunci îi voi dezradacina ?i voi pierde cu totul poporul acesta", zice Domnul.

\chapter{13}

\par 1 A?a mi-a grait Domnul: "Mergi ?i-?i cumpara un brâu de in ?i-l încinge peste mijlocul tau, dar în apa sa nu-l bagi!"
\par 2 ?i am cumparat brâul, dupa cuvântul Domnului, ?i l-am încins peste coapsele mele.
\par 3 Apoi a fost cuvântul Domnului iara?i catre mine ?i mi-a zis:
\par 4 "Ia brâul, pe care l-ai cumparat ?i care este peste coapsele tale, ?i, sculându-te, du-te la Eufrat ?i-l ascunde acolo în crapatura unei stânci!"
\par 5 ?i m-am dus ?i l-am ascuns la Eufrat, cum îmi poruncise Domnul.
\par 6 Iar dupa ce au trecut mai multe zile, mi-a zis Domnul: "Scoala, du-te la Eufrat ?i ia de acolo brâul, pe care ti-am poruncit sa-l ascunzi acolo!"
\par 7 ?i m-am dus la Eufrat, am sapat ?i am luat brâul din locul unde-l ascunsesem; dar iata brâul se stricase ?i nu mai era bun de nimic.
\par 8 Atunci a fost cuvântul Domnului catre mine:
\par 9 "Iata ce zice Domnul: A?a voi sfarâma mândria lui Iuda ?i trufia cea mare a Ierusalimului.
\par 10 Acest popor rau, care nu vrea sa asculte cuvintele Mele, ci traie?te dupa îndaratnicia inimii lui, merge pe urmele altor dumnezei ?i se închina ?i sluje?te lor, va fi ca brâul acesta care nu este bun de nimic.
\par 11 Ca precum e brâul aproape de coapsele omului, a?a am apropiat Eu de Mine toata casa lui Israel ?i toata casa lui Iuda, zice Domnul, ca sa-Mi fie poporul Meu, numele Meu, lauda Mea ?i slava Mea, dar ei n-au ascultat.
\par 12 De aceea, spune-le cuvântul acesta: A?a zice Domnul Dumnezeul lui Israel: Tot urciorul se umple de vin. ?i ei to?i vor zice: Au doara noi nu ?tim ca tot urciorul se umple de vin?
\par 13 Iar tu sa le spui: A?a zice Domnul: Iata, Eu voi umple cu vin pâna la îmbatare pe to?i locuitorii ?arii acesteia, pe regii care ?ed pe scaunul lui David, pe preo?i, pe prooroci ?i, pe to?i locuitorii Ierusalimului,
\par 14 ?i-i voi zdrobi pe unii de al?ii, pe parin?i ?i pe fii laolalta, zice Domnul. Nu-i voi cru?a ?i nu-i voi milui, nici Îmi va fi mila ca sa-i pierd.
\par 15 Asculta?i ?i lua?i aminte! Nu fi?i mândri, caci Domnul graie?te:
\par 16 Da?i slava Domnului Dumnezeului vostru, pâna nu vine întunericul ?i pâna nu se lovesc picioarele voastre de mun?ii nop?ii. Voi ve?i a?tepta lumina, dar El o va preface în umbra mor?ii, o va preface în negura adânca.
\par 17 Iar daca nu asculta?i acestea, atunci sufletul meu va plânge în locuri tainice mândria voastra ?i va plânge amar ?i ochii mei vor varsa lacrimi, pentru ca turma Domnului va fi dusa în robie.
\par 18 Spune regelui ?i reginei: Smeri?i-va ?i ?ede?i mai jos, deoarece a cazut de pe capul vostru cununa slavei voastre!
\par 19 Ceta?ile din Negheb sunt închise ?i n-are cine le deschide. Iuda tot va fi dus în robie; în întregime va fi dus în robie.
\par 20 Ridica?i-va ochii vo?tri ?i privi?i pe cei ce vin de la miazanoapte! Unde este turma ce ?i s-a dat, turma ta cea frumoasa?
\par 21 Fiica Sionului, ce vei zice tu când te vor cerceta, ca biruitori, cei pe care i-ai obi?nuit sa-?i fie prieteni apropia?i? Nu te vor cuprinde durerile ca pe femeia care na?te?
\par 22 Iar de vei zice în inima ta: "Pentru ce au venit peste mine acestea?" ?i se va raspunde: Pentru mul?imea nelegiuirilor tale ?i s-au desfacut poalele ?i ?i s-au dezgolit picioarele tale.
\par 23 Poate oare sa-?i schimbe etiopianul pielea sa ?i leopardul petele sale? A?a ?i voi! Pute?i oare sa face?i bine, când sunte?i deprin?i a face rau?
\par 24 De aceea Eu va voi spulbera ca pleava împra?tiata de vântul pustiului.
\par 25 Fiica Sionului, iata soarta ta, plata razvratirii tale, masurata ?ie de Mine, zice Domnul, pentru ca M-ai uitat ?i ?i-ai pus încrederea în minciuna.
\par 26 De aceea ?i se vor da poalele peste cap, ca sa se dezgoleasca ru?inea ta.
\par 27 Vazut-am desfrânarea ta ?i strigatele tale de placere, netrebniciile tale ?i urâciunile tale de pe dealuri ?i din câmp. Vai ?ie, Ierusalime, tu e?ti necurat! Dar pâna când...?"

\chapter{14}

\par 1 Cuvântul Domnului care a fost catre Ieremia în vremea secetei.
\par 2 Plânge Iuda ?i por?ile Ierusalimului au cazut ?i stau înnegrite pe pamânt, ?i strigat se ridica din Ierusalim.
\par 3 Cei mari trimit pe cei mici dupa apa; ace?tia merg la fântâni, dar nu gasesc apa, ?i se întorc înapoi cu vasele goale; de aceea se ru?ineaza, ro?esc ?i î?i acopar capetele.
\par 4 Ogoarele au crapat, pentru ca n-a fost ploaie pe pamânt; de aceea plugarii î?i acopera capetele ?i sunt tulbura?i.
\par 5 Pâna ?i cerboaica na?te în câmp ?i î?i parase?te puii, pentru ca nu este iarba.
\par 6 ?i asinii salbatici stau în locuri înalte ?i înghit aer, ca ?acalii, ?i ochii li s-au înfundat, pentru ca n-au iarba.
\par 7 De?i faradelegile noastre marturisesc împotriva noastra, dar Tu, Doamne, fa mila cu noi pentru numele Tau! Mare este abaterea noastra ?i am pacatuit înaintea Ta.
\par 8 Nadejdea lui Israel ?i Izbavitorul lui la vreme de strâmtorare, pentru ce e?ti ca un strain în ?ara aceasta ?i ca un trecator care se opre?te pentru o noapte?
\par 9 Pentru ce e?ti Tu ca un om fara vlaga ?i ca un razboinic care nu poate ajuta? Totu?i Tu, Doamne, e?ti în mijlocul nostru ?i numele Tau este chemat asupra noastra: Nu ne lasa!
\par 10 A?a zice Domnul catre poporul acesta: "Pentru ca le place sa rataceasca ?i nu-?i cru?a picioarele, de aceea Domnul nu mai gase?te placere în ei; pomene?te acum faradelegile lor ?i nume?te pacatele lor".
\par 11 Apoi Domnul mi-a spus: "Tu sa nu te rogi pentru poporul acesta spre binele lui.
\par 12 De vor posti, nu voi auzi strigarea lor; de vor aduce arderi de tot ?i prinoase, nu voi primi, ci cu sabie, cu foamete ?i cu molima îi voi pierde".
\par 13 Atunci am zis: "Doamne Dumnezeule! Iata ce le graiesc proorocii: Nu ve?i vedea sabie ?i foamete nu va fi la voi, ci voi da în locul acesta pace necontenita".
\par 14 Iar Domnul mi-a raspuns: "Proorocii proorocesc lucruri mincinoase în numele Meu; Eu nu i-am trimis, nici nu le-am dat porunca ?i nici nu le-am grait; ci ei va vestesc vedenii mincinoase, proorociri de?arte ?i închipuiri ale inimii lor".
\par 15 De aceea, a?a zice Domnul despre prooroci: "Ei proorocesc în numele Meu, dar Eu nu i-am trimis; ei zic: "Sabie ?i foamete nu va fi în ?ara aceasta", dar de sabie ?i de foamete vor pieri ace?ti prooroci ?i poporul caruia au proorocit ei.
\par 16 Va fi risipit pe uli?ele Ierusalimului de foamete ?i de sabie, ?i nu va avea cine sa-i îngroape pe ei ?i pe femeile lor, pe fiii lor ?i pe fiicele lor, ca Eu voi varsa asupra lor rautatea lor.
\par 17 ?i sa le mai spui cuvântul acesta: Ochii Mei varsa lacrimi ziua ?i noaptea ?i nu se opresc! Caci cu bataie mare a fost batuta fecioara, fiica poporului Meu, ?i cu lovitura grea.
\par 18 De ies în câmp, iata numai oameni uci?i cu sabia! De intru în cetate, iata numai oameni istovi?i de foame! Chiar ?i proorocul ?i preotul ratacesc prin ?ara fara sa ?tie unde merg".
\par 19 Atunci am zis iara?i: "Lepadat-ai Tu oare cu totul pe Iuda? Au doar Te-a dezgustat cu totul Sionul? Pentru ce ne-ai lovit a?a, încât nu mai avem leac? A?teptam pace, dar iata nu vine nimic bun! A?teptam vremea vindecarii, ?i iata grozavie!
\par 20 Marturisim, Doamne, necredin?a noastra ?i faradelegile parin?ilor no?tri, ca am pacatuit înaintea Ta.
\par 21 Nu ne lepada pe noi pentru numele Tau! Nu necinsti tronul slavei Tale! Adu-?i aminte ?i nu strica legamântul Tau cu noi!
\par 22 Sunt oare printre dumnezeii de?er?i ai neamurilor datatori de ploaie? Sau poate oare cerul singur sa verse apa? Au nu e?ti Tu, Doamne Dumnezeule, Cel ce dai ploaie? În Tine nadajduim, ca Tu faci toate!"

\chapter{15}

\par 1 ?i mi-a zis iara?i Domnul: "Chiar Moise ?i Samuel de ar sta înaintea Mea, sufletul Meu tot nu s-ar îndupleca spre poporul acesta. Izgone?te-i de la fa?a Mea, ca sa se duca.
\par 2 Iar de-?i vor zice: "Unde sa ne ducem?" sa le spui: A?a zice Domnul: "Cel rânduit pentru moarte sa se duca la moarte, cel pentru sabie, la sabie, cel pentru foamete, la foamete, ?i cel pentru robie, în robie!
\par 3 Voi trimite asupra lor patru feluri de pedepse, zice Domnul: sabia, ca sa-i taie, câinii, ca sa-i sfâ?ie, pasarile cerului ?i fiarele câmpului, ca sa-i piarda.
\par 4 ?i-i voi da spre munca la toate regatele pamântului, pentru Manase, fiul lui Iezechia, regele lui Iuda, ?i pentru cele ce a facut el în Ierusalim.
\par 5 Oare cui îi va parea rau de tine, Ierusalime, ?i cine-?i va arata mila sau cine va veni la tine ?i te va întreba de sanatate?
\par 6 Pentru ca tu M-ai lasat, zice Domnul, ?i te-ai întors înapoi, de aceea-Mi voi întinde mâna Mea asupra ta ?i te voi pierde, ca M-am saturat miluindu-te.
\par 7 Cu vânturatoarea îi voi vântura la por?ile ?arii, îi voi lipsi de copii ?i voi pierde pe poporul Meu, dar tot nu se vor întoarce din caile lor.
\par 8 Vaduvele lor sunt mai multe decât nisipul marii. Voi aduce un pustiitor în plina amiaza asupra lor, asupra mamelor celor tinere, ?i va cadea asupra lor, fara de veste, frica ?i groaza.
\par 9 Cea care nascuse ?apte copii zace în neputin?a, î?i da duhul ?i-i apune soarele înca ziua fiind; este ru?inata ?i ocarâta. Pe cei rama?i îi voi da sabiei înaintea ochilor vrajma?ilor lor", zice Domnul.
\par 10 Vai de mine, mama, ca m-ai nascut sa fiu om de cearta ?i de pricina pentru toata ?ara! Nimanui n-am dat cu dobânda ?i nici mie nu mi-a dat nimeni cu dobânda, ?i tot ma blestema to?i.
\par 11 Zis-a Domnul: "Da, te voi întari pentru binele tau; singur voi conduce pe vrajma?ul tau sa te roage la vreme de nenorocire ?i de restri?te.
\par 12 Poate cineva sa rupa fierul, fierul de la miazanoapte ?i arama?
\par 13 Averea ta ?i comorile tale le voi da prada fara plata, pentru toate pacatele tale, în toate hotarele tale;
\par 14 ?i le voi trimite cu vrajma?ii tai într-o ?ara pe care tu n-o cuno?ti, ca s-a aprins focul mâniei Mele ?i va arde peste voi".
\par 15 O, Doamne, Tu ?tii toate! Adu-?i aminte de mine, cerceteaza-ma ?i ma razbuna împotriva prigonitorilor mei! Nu ma pierde dupa îndelungata Ta rabdare, ?tiind ca pentru Tine sufar ocara.
\par 16 Aflat-am cuvintele Tale ?i le-am sorbit ?i cuvântul Tau a fost bucurie ?i veselie pentru inima mea, ca s-a chemat asupra mea numele Tau, Doamne Dumnezeul puterilor.
\par 17 În adunarea celor ce râd n-am ?ezut, nici m-am veselit, ci am stat singur sub mâna Ta ce apasa asupra mea, ca Tu ma umpluse?i de mânie.
\par 18 De ce este a?a de grea boala mea ?i de ce rana mea este a?a de greu de vindecat, încât nu sufera doctorie? Oare vei fi Tu pentru mine ca un izvor amagitor ?i ca o apa în?elatoare?
\par 19 ?i la acestea Domnul mi-a raspuns a?a: "De te vei întoarce, Eu te voi aduce la starea cea dintâi ?i vei sta înaintea fe?ei Mele; daca tu vei deosebi lucrul de pre? de cel fara de pre?, vei fi ca gura Mea; ?i nu te vei întoarce la ei, ci ei se vor întoarce la tine;
\par 20 ?i te voi face pentru poporul acesta zid tare de arama. Vor lupta împotriva ta, dar nu te vor birui, pentru ca Eu sunt cu tine, ca sa te scap ?i sa te izbavesc, zice Domnul.
\par 21 ?i te voi scapa din mâna celor rai; te voi izbavi din mina asupritorilor".

\chapter{16}

\par 1 ?i a fost cuvântul Domnului catre mine:
\par 2 "Sa nu-îi iei femeie ?i sa nu ai nici fii, nici fiice în locul acesta,
\par 3 Deoarece a?a graie?te Domnul despre fiii ?i fiicele care se vor na?te în locul acesta, despre mamele care îi vor na?te ?i despre parin?ii care îi vor face pe pamântul acesta:
\par 4 Vor muri de moarte grea ?i nu vor fi nici boci?i, nici îngropa?i, ci vor fi ca gunoiul pe fa?a pamântului; vor fi pierdu?i prin sabie ?i foamete ?i trupurile lor vor fi mâncarea pasarilor cerului ?i fiarelor pamântului".
\par 5 ?i iara?i a zis Domnul: "Sa nu intri în casa celor ce jelesc ?i sa nu te duci sa plângi ?i sa jele?ti cu ei, caci am luat de la poporul acesta pacea Mea, mila ?i parerea de rau, zice Domnul.
\par 6 ?i vor muri cei mari ?i cei mici în pamântul acesta, ?i nu vor fi îngropa?i, ?i dupa ei nimeni nu va plânge, nimeni nu-?i va face taieturi, nici se va tunde pentru ei.
\par 7 Nu se va frânge pentru ei pâine de jale ca mângâiere pentru cel mort; ?i nu li se va da cupa mângâierii ca sa bea dupa tatal lor ?i dupa mama lor.
\par 8 De asemenea sa nu intri în casa ospa?ului, ca sa ?ezi cu ei sa manânci ?i sa bei, ca a?a zice Domnul Savaot, Dumnezeul lui Israel:
\par 9 Iata, voi curma în locul acesta chiar în zilele voastre ?i sub ochii vo?tri glasul bucuriei ?i glasul de veselie, glasul mirelui ?i glasul miresei.
\par 10 Când vei spune poporului acestuia toate acestea ?i când ei î?i vor zice: "Pentru ce a rostit Domnul asupra noastra aceasta mare nenorocire? Care este nedreptatea noastra ?i care este pacatul cu care am pacatuit noi înaintea Domnului Dumnezeului nostru?"
\par 11 Atunci sa le spui: Pentru ca parin?ii vo?tri M-au parasit, zice Domnul, ?i s-au dus dupa al?i dumnezei, au slujit acelora ?i li s-au închinat, iar pe Mine M-au parasit ?i legea Mea n-au pazit-o.
\par 12 Dar voi face?i înca ?i mai rau decât parin?ii vo?tri, ?i trai?i fiecare dupa inima voastra cea rea ?i îndaratnica ?i nu Ma asculta?i.
\par 13 De aceea va voi arunca din ?ara aceasta într-o rara pe care n-a?i cunoscut-o nici voi, nici parin?ii vo?tri, ?i ve?i sluji acolo ziua ?i noaptea la al?i dumnezei, ca Eu nu va voi arata îndurare.
\par 14 Pentru ca vin zile, zice Domnul, când nu se va mai zice: "Viu este Domnul, Care a scos pe fiii lui Israel din ?ara Egiptului",
\par 15 Ci, "Viu este Domnul, Care a scos pe fiii lui Israel din ?ara cea de la miazanoapte ?i din toate ?arile în care-i izgonise", ca îi voi întoarce în ?ara pe care le-am daruit-o parin?ilor lor.
\par 16 Iata, voi trimite mul?ime de pescari, zice Domnul, ?i-i vor pescui; iar apoi voi trimite mul?ime de vânatori, ?i-i vor vâna de prin to?i mun?ii, de pe toate dealurile ?i de prin crapaturile stâncilor.
\par 17 Pentru ca ochii Mei sunt asupra tuturor cailor lor, care nu sunt ascunse de la fa?a Mea, ?i nedreptatea lor nu se ascunde privirilor Mele.
\par 18 ?i le voi rasplati mai întâi pentru nedreptatea lor ?i pentru îndoitul lor pacat, pentru ca au spurcat ?ara Mea eu stârvurile grozaviilor lor ?i cu urâciunile lor au umplut mo?tenirea Mea".
\par 19 Doamne, puterea mea, taria mea ?i scaparea mea la vreme de necaz, la Tine vor veni popoarele de la marginile pamântului ?i vor zice: "Numai minciuna au mo?tenit parin?ii vo?tri, idoli de?er?i, care nu sunt de nici un folos!"
\par 20 "Poate oare omul sa-?i faca dumnezei, care, de altfel, nici nu sunt dumnezei?
\par 21 De aceea iata le voi arata acum, le voi arata mâna Mea ?i puterea Mea ?i vor afla ca numele Meu este Domnul!"

\chapter{17}

\par 1 "Pacatul lui Iuda este scris cu condei de fier, cu vârf de diamant este sapat pe lespedea inimii lor ?i pe coarnele jertfelnicelor lor.
\par 2 Ca de copiii lor î?i aduc aminte de jertfelnicele lor ?i de locurile de jertfa de pe sub copacii verzi ?i de pe vârfurile dealurilor.
\par 3 Ierusalime, muntele Meu, câmpul, avu?ia ?i toate comorile tale le voi da spre prada; voi da ?i locurile tale înalte din toate hotarele tale, pentru pacatele tale.
\par 4 ?i tu prin tine însu?i te vei lipsi de mo?tenirea ta, pe care ?i-am dat-o Eu ?i te voi da în robie vrajma?ilor tai, în ?ara pe care tu n-o ?tii, pentru ca ai aprins focul mâniei Mele ?i în veci va arde".
\par 5 A?a zice Domnul: "Blestemat fie omul care se încrede în om ?i î?i face sprijin din trup omenesc ?i a carui inima se departeaza de Domnul.
\par 6 Acela va fi ca ierburile pustiului ?i nu va vedea când va veni binele, ci va locui în locurile arse ale pustiului, în pamânt neroditor ?i nelocuit.
\par 7 Binecuvântat fie omul care nadajduie?te în Domnul ?i a carui nadejde este Domnul,
\par 8 Deoarece acesta va fi ca pomul sadit lânga ape, care-?i întinde radacinile pe lânga râu ?i nu ?tie când vine ar?i?a; frunzele lui sunt verzi, la timp de seceta nu se teme ?i nu înceteaza a rodi.
\par 9 Inima omului este mai vicleana decât orice ?i foarte stricata! Cine o va cunoa?te!
\par 10 Eu, Domnul, patrund inima ?i încerc rarunchii, ca sa rasplatesc fiecaruia dupa caile lui ?i dupa roada faptelor lui.
\par 11 Prepeli?a cloce?te ouale pe care nu le-a ouat; a?a este ?i cel ce câ?tiga avu?ie nedreapta, o lasa la jumatatea zilelor sale ?i la sfâr?itul sau se va trezi ca este un nebun".
\par 12 Tronul slavei, înal?at de la început, este locul sfin?irii noastre.
\par 13 Tu, Doamne, e?ti nadejdea lui Israel! To?i cei ce Te parasesc se vor ru?ina, ca Tu ai zis: "Cei ce se departeaza de Mine vor fi scri?i pe pulbere, pentru ca au parasit pe Domnul, izvorul apei celei vii".
\par 14 Vindeca-ma, Doamne, ?i voi fi vindecat; mântuie?te-ma ?i voi fi mântuit, caci Tu e?ti lauda mea!
\par 15 Iata acestea-mi zic ei: "Unde este cuvântul Domnului? Sa vina!"
\par 16 Eu nu te-am îndemnat totu?i la mai rau, nici n-am dorit ziua nenorocirii. Tu ?tii acestea; ?i ce a ie?it din gura mea e descoperit înaintea fe?ei Tale.
\par 17 Nu fi pricina de groaza pentru mine, ca Tu e?ti nadejdea mea în ziua strâmtorarii.
\par 18 Sa nu ma ru?inez eu, ci sa se ru?ineze apasatorii mei; ei sa tremure, dar sa nu tremur eu! Adu asupra lor ziua necazului ?i zdrobe?te-i cu zdrobire îndoita.
\par 19 A?a mi-a zis Domnul: "Du-te ?i stai la poarta fiilor poporului, pe care intra regii lui Iuda, ?i pe care ies ei, ?i la toate por?ile Ierusalimului,
\par 20 ?i le spune: Regii lui Iuda ?i tot Iuda ?i to?i locuitorii Ierusalimului, care intra?i pe por?ile acestea, asculta?i cuvântul acesta:
\par 21 A?a graie?te Domnul: "Pazi?i-va sufletele ?i nu duce?i sarcini în ziua de odihna, nici le baga?i pe por?ile Ierusalimului;
\par 22 Nu scoate?i sarcini din casele voastre în ziua odihnei ?i nu va îndeletnici?i cu nici un fel de munca, ci sfin?i?i ziua odihnei, a?a cum am poruncit Eu parin?ilor vo?tri,
\par 23 Care n-au ascultat ?i nu ?i-au plecat urechea, ci ?i-au învârto?at neascultarea, ca sa nu îndeplineasca ?i sa nu ia înva?atura.
\par 24 De Ma ve?i asculta, zice Domnul, ?i nu ve?i aduce sarcini pe por?ile acestei ceta?i în ziua odihnei; ci ve?i sfin?i ziua odihnei ?i nu va ve?i îndeletnici în acea zi cu nici o munca,
\par 25 Atunci pe por?ile acestei ceta?i vor intra regi care ?ed pe scaunul lui David, în alai de care ?i de cai, ei ?i capeteniile o?tirii lor, oamenii din Iuda ?i locuitorii Ierusalimului; ?i cetatea aceasta în veac va fi locuita.
\par 26 ?i vor veni din ceta?ile lui Iuda, din împrejurimile Ierusalimului ?i din pamântul lui Veniamin, din ?es, din mun?i ?i de la miazazi, ?i vor aduce arderi de tot ?i jertfe, prinoase ?i tamâie ?i jertfe de mul?umire în templul Domnului.
\par 27 Iar de nu Ma ve?i asculta ca sa sfin?i?i ziua odihnei ?i sa nu duce?i sarcini, când intra?i pe por?ile Ierusalimului în ziua odihnei, voi aprinde foc la por?ile lui ?i va arde palatele Ierusalimului ?i nu se va stinge".

\chapter{18}

\par 1 Cuvântul care a fost de la Domnul catre Ieremia ?i i-a zis:
\par 2 "Scoala ?i intra în casa olarului ?i acolo î?i voi vesti cuvintele Mele!"
\par 3 ?i am intrat eu în casa olarului ?i iata acesta lucra cu roata
\par 4 ?i vasul pe care-l facea olarul din lut s-a stricat în mâna lui; dar olarul a facut dintr-însul alt vas, cum a crezut ca-i mai bine sa-l faca.
\par 5 ?i a fost cuvântul Domnului iara?i catre mine ?i mi-a zis: "Casa lui Israel, oare nu pot sa fac ?i Eu cu voi, ca olarul acesta, zice Domnul?
\par 6 Iata, ce este lutul în mâna olarului, aceea sunte?i ?i voi în mâna Mea, casa lui Israel!
\par 7 Daca voi zice cândva despre un popor, sau despre un rege, ca-l voi dezradacina, îl voi sfarâma ?i-l voi pierde;
\par 8 ?i daca poporul acela, despre care am zis Eu acestea, se va întoarce de la faptele lui cele rele, atunci voi îndeparta raul ce gândeam sa-i fac.
\par 9 Sau daca voi zice despre un popor sau despre un rege ca-l voi întocmi ?i-l voi întari,
\par 10 ?i daca acela va face rele înaintea ochilor Mei ?i nu va asculta de glasul Meu, atunci voi schimba binele cu care voiam sa-l fericesc.
\par 11 Spune deci barba?ilor lui Iuda ?i locuitorilor Ierusalimului: A?a zice Domnul: Iata, Eu va gatesc rele ?i uneltiri împotriva voastra. A?adar sa se întoarca fiecare de la calea lui cea rea; îndrepta?i-va caile ?i purtarile voastre!
\par 12 Dar ei zic: "Este zadarnic! Noi vom trai dupa gândul nostru ?i ne vom purta fiecare dupa învârto?area inimii noastre celei rele".
\par 13 De aceea, a?a zice Domnul: "Întreba?i popoarele: Auzit-a oare cineva asemenea lucru? Lucruri peste masura de urâcioase a facut fecioara lui Israel.
\par 14 Parase?te oare zapada Libanului stânca muntelui? Ori seaca apele ce vin de departe ?i sunt reci ?i curgatoare?
\par 15 Poporul Meu însa M-a parasit. Tamâiaza idoli, s-a poticnit în caile sale ?i a parasit caile cele vechi, ca sa umble pe poteci ?i pe drumuri nebatatorite, ca sa-?i faca ?ara grozavie ?i batjocura ve?nica,
\par 16 Încât tot cel ce va trece prin aceasta sa se mire ?i sa clatine din cap.
\par 17 Îi voi spulbera înaintea vrajma?ilor ca vântul cel de la rasarit, ?i nu fa?a, ci spatele îl voi întoarce spre ei în ziua necazului lor".
\par 18 Zis-au ei: "Veni?i sa uneltim împotriva lui Ieremia, ca nu va pieri legea din mâna preotului, nici sfatul de la în?elept, nici cuvântul (lui Dumnezeu) de la prooroc. Veni?i sa-l biruim cu limba ?i sa nu luam aminte la cuvintele lui!
\par 19 Ia aminte la mine, Doamne, ?i auzi glasul potrivnicilor mei!
\par 20 Se cuvine oare a rasplati cu rau pentru bine? Ei însa sapa groapa sufletului meu. Adu-?i aminte ca stau înaintea fe?ei Tale, ca sa graiesc bine de ei ?i ca sa abat de la ei mânia Ta.
\par 21 Da dar pe fiii lor la foamete ?i pune-i sub sabie! Femeile lor sa fie fara copii ?i vaduve, barba?ii lor sa fie lovi?i de moarte ?i tinerii lor sa fie uci?i cu sabia. în razboi.
\par 22 Bocete sa se auda prin casele lor, când fara de veste vei aduce o?tiri asupra lor, ca sapa groapa ca sa ma prinda ?i pe ascuns au întins curse pentru picioarele mele.
\par 23 Dar Tu, Doamne, ?tii tot ce uneltesc ei împotriva mea ca sa ma omoare! Nu ierta nedreptatea lor ?i pacatul lor nu-l ?terge dinaintea fe?ei Tale! Doboara-i înaintea Ta ?i lucreaza împotriva lor la vremea mâniei Tale!

\chapter{19}

\par 1 A?a a zis Domnul: "Mergi ?i cumpara o oala de lut de la olar; ia cu tine pe cei mai batrâni din popor ?i din capeteniile preo?ilor ?i ie?i în valea Ben-Hinom, care este la Poarta Olariei.
\par 2 Acolo sa roste?ti cuvintele pe care ti le voi spune ?i sa zici:
\par 3 "Regi ai lui Iuda ?i locuitori ai Ierusalimului, asculta?i cuvântul Domnului! A?a graie?te Domnul Savaot, Dumnezeul lui Israel: Iata voi aduce a?a strâmtorare asupra lacului acestuia, încât, auzind cineva, sa-i ?iuie urechile,
\par 4 Pentru ca M-au parasit, au înstrainat lacul acesta ?i tamâiaza pe el al?i dumnezei, pe care nu i-au cunoscut nici ei, nici parin?ii lor ?i nici regii lui Iuda; au umplut locul acesta de sângele nevinova?ilor ?i au facut înal?imi pentru Baal, ca sa arda pe fiii lor cu foc.
\par 5 Ei aduc ardere de tot pentru Baal, ceea ce Eu nu le-am poruncit, nici le-am grait ?i ceea ce nici prin minte nu Mi-a trecut.
\par 6 De aceea iata vin zile, zice Domnul, când locul acesta nu se va mai chema Tofet sau valea fiilor lui Hinom, ci valea uciderii.
\par 7 Voi pierde cu desavâr?ire pe Iuda ?i Ierusalimul în locul acesta ?i-i voi rapune cu sabia înaintea vrajma?ilor lor ?i cu mâna celor ce cauta sufletul lor, ?i voi da trupurile lor spre hrana pasarilor cerului ?i fiarelor pamântului.
\par 8 Voi face cetatea aceasta pustie ?i batjocura. Tot cel ce va trece prin ea se va mira ?i va fluiera a pustiu, vazând toate ranile ei.
\par 9 ?i-i voi ospata cu carnea fiilor lor ?i cu carnea fiicelor lor; ?i va mânca fiecare carnea aproapelui sau, fiind în împresurare ?i în strâmtorare, când îi vor strâmtora vrajma?ii lor ?i cei ce vor sa le ia via?a.
\par 10 Apoi sa spargi oala înaintea ochilor barba?ilor acelora, care vor merge cu tine, ?i sa le zici:
\par 11 "Iata ce zice Domnul Savaot: Voi sfarâma poporul acesta ?i cetatea aceasta, a?a cum am sfarâmat vasul olarului, care nu mai poate fi facut la loc, ?i-i vor îngropa în Tofet din pricina lipsei de loc pentru îngropare.
\par 12 A?a voi face cu locul acesta, zice Domnul, ?i cu locuitorii lui, ?i voi face cetatea aceasta ca Tofetul.
\par 13 ?i casele Ierusalimului ?i casele regilor lui Iuda vor fi necurate, ca Tofetul, pentru ca pe acoperi?ul tuturor caselor s-aduce tamâie întregii o?tiri cere?ti ?i se savâr?esc turnari în cinstea dumnezeilor straini".
\par 14 ?i s-a întors Ieremia din Tofet, unde-l trimisese Domnul sa prooroceasca, ?i a stat în curtea templului Domnului ?i a zis catre tot poporul:
\par 15 "A?a zice Domnul Savaot, Dumnezeul lui Israel: Iata voi aduce asupra ceta?ii acesteia ?i asupra celorlalte ceta?i toate nenorocirile pe care le-am rostit împotriva ei, pentru ca ?i-au învârto?at inima ?i nu asculta cuvintele Mele".

\chapter{20}

\par 1 Auzind Pa?hurr, fiul preotului Imer, care era supraveghetor în templul Domnului, ca Ieremia a profe?it aceste cuvinte,
\par 2 A lovit Pa?hurr pe Ieremia proorocul ?i l-a aruncat în temni?a care se afla la poarta cea de sus a lui Veniamin, în templul Domnului.
\par 3 Dar a doua zi Pa?hurr a dat drumul lui Ieremia din temni?a ?i Ieremia i-a zis: "Domnul nu te mai nume?te Pa?hurr (Noroc din toate par?ile),  ci Magor Misabib (Spaima din toate par?ile),
\par 4 Pentru ca a?a zice Domnul: Iata, te voi face groaza ?i pentru tine însu?i ?i pentru to?i prietenii tai; ace?tia vor cadea de sabia vrajma?ilor lor ?i ochii tai vor vedea aceasta. ?i pe tot Iuda îl voi da în mâinile regelui Babilonului, care-i va duce la Babilon ?i-i va lovi cu sabia.
\par 5 ?i toata boga?ia ceta?ii acesteia, toata agonisita ei, toate cele de pre? ale ei ?i toate comorile regilor lui Iuda le voi da în mâinile vrajma?ilor lor, care, jefuindu-le, le vor lua ?i le vor duce la Babilon.
\par 6 Iar tu, Pa?hurr, cu to?i cei ce traiesc în casa ta, va ve?i duce în robie ?i, mergând la Babilon, vei muri acolo ?i acolo vei fi îngropat ?i tu ?i to?i prietenii tai, carora tu le-ai proorocit mincinos.
\par 7 Doamne, Tu m-ai aprins ?i iata sunt înflacarat; Tu e?ti mai tare decât mine ?i ai biruit, iar eu în toate zilele sunt batjocorit ?i fiecare î?i bate joc de mine;
\par 8 Ca de când vorbesc, sco?ând strigate împotriva silniciei ?i rostind pustiirea, cuvântul Domnului s-a prefacut în ocara pentru mine ?i în batjocura zilnica.
\par 9 De aceea mi-am zis: "Nu voi mai pomeni de El ?i nu voi mai grai, în numele Lui!" Dar iata era în inima mea ceva, ca un fel de foc aprins, închis în oasele mele, ?i eu ma sileam sa-l înfrânez ?i n-am putut;
\par 10 Ca am auzit ocari de la mul?i ?i amenin?ari din toate par?ile, zicând: "Pârâ?i-l ?i-l vom pârî ?i noi!" To?i cei ce traiau în pace cu mine, ma pândesc sa vada nu cumva ma voi poticni, ?i ziceau: "Poate va cadea ?i-l vom birui ?i ne vom razbuna pe el!"
\par 11 Dar Domnul este cu mine, ca un aparator puternic. De aceea prigonitorii mei se vor poticni ?i nu vor birui; se vor face de ru?ine, pentru ca n-au izbutit; ocara lor va fi ve?nica ?i niciodata nu se va uita.
\par 12 Doamne al puterilor, Cel ce cercetezi cu dreptate ?i patrunzi rarunchii ?i inimile, fa-ma sa vad razbunarea Ta asupra lor, ca ?ie ?i-am încredin?at pricina mea!
\par 13 Cânta?i Domnului! Lauda?i pe Domnul, caci El izbave?te sufletul celui împilat din mâna facatorilor de rele.
\par 14 Blestemata fie ziua în care m-am nascut, ziua în care m-a nascut maica-mea sa nu fie binecuvântata.
\par 15 Blestemat fie omul care a adus tatalui meu veste ?i a zis: "?i s-a nascut fiu", ?i s-a bucurat mult de aceasta.
\par 16 Întâmple-se omului aceluia ceea ce s-a întâmplat ceta?ilor pe care le-a darâmat Domnul ?i nu le-a cru?at! Sa auda el diminea?a bocet ?i la amiaza tânguire mare,
\par 17 Ca nu m-a ucis chiar din pântece, ca mama mea sa-mi fi fost mormânt ?i pântecele ei sa fi ramas ve?nic însarcinat.
\par 18 Pentru ce am ie?it eu din pântece, ca sa vad suferin?a ?i durere ?i ca zilele mele sa se sfâr?easca în ru?ine?"

\chapter{21}

\par 1 Cuvântul care a fost de la Domnul catre Ieremia, când a trimis regele Sedechia la el pe Pa?hurr, fiul lui Malchia, ?i pe Sofonie, fiul preotului Maaseia, ca sa-i zica:
\par 2 "Întreaba pentru noi pe Domnul, ca Nabucodonosor, regele Babilonului, ne face razboi. Poate ca va face Domnul cu noi ceva în felul minunilor Lui, ca sa se departeze de la noi".
\par 3 Iar Ieremia le-a raspuns: "A?a sa spune?i lui Sedechia:
\par 4 Domnul Dumnezeul lui Israel a?a zice: "Iata, voi întoarce înapoi armele de razboi, care sunt în mâinile voastre ?i cu care va lupta?i cu regele Babilonului ?i cu Caldeii, care va impresara pe din afara de ziduri, ?i le voi aduna în mijlocul ceta?ii acesteia;
\par 5 ?i voi lupta ?i Eu Însumi împotriva voastra cu mâna întinsa ?i cu bra? puternic, cu mânie, cu urgie ?i cu multa furie.
\par 6 ?i voi lovi pe cei ce traiesc în cetatea aceasta de la oameni pâna la animale ?i vor muri de ciuma cumplita.
\par 7 Iar dupa aceea, zice Domnul, pe Sedechia, regele lui Iuda, pe slugile lui, pe popor ?i pe cei ce au ramas în cetatea aceasta pe urma ciumei, a sabiei ?i a foametei, îi voi da în mâinile lui Nabucodonosor, regele Babilonului, ?i în mâinile vrajma?ilor lor ?i în mâinile celor ce vor sa le ia via?a; ?i acela îi va lovi cu ascu?i?ul sabiei ?i nu-i va cru?a, nici se va îndura sa-i miluiasca.
\par 8 Iar poporului acestuia spune-i: A?a zice Domnul: Iata, va pun înainte calea vie?ii ?i calea mor?ii:
\par 9 Cine va ramâne în cetatea aceasta, acela va muri de sabie ?i de foamete ?i de ciuma; iar cine va ie?i ?i se va preda Caldeilor, care va împresoara, acela va trai ?i va fi luat ca prada,
\par 10 Ca Eu Mi-am întors fa?a împotriva ceta?ii acesteia, zice Domnul, în rau, nu în bine; ?i va fi data în mâinile regelui Babilonului, care o va arde cu foc.
\par 11 Iar casei regelui Sedechia sa-i spui: Asculta?i cuvântul Domnului!
\par 12 Casa lui David, a?a zice Domnul: Face?i judecata dis-de-diminea?a ?i scapa?i pe cel asuprit din mâna asupritorului, pentru ca sa nu izbucneasca mânia Mea ca focul ?i pentru ca sa nu se aprinda din pricina faptelor voastre cele rele, a?a încât nimeni sa n-o stinga.
\par 13 Cetate a vaii ?i stânca din câmp, iata sunt împotriva ta, zice Domnul! O, voi, care zice?i: "Cine se va ridica împotriva noastra ?i cine va intra în sala?ul nostru?"
\par 14 Iata, sunt împotriva voastra! Dar Eu va voi pedepsi dupa roadele faptelor voastre, zice Domnul, ?i voi aprinde foc în padurea voastra ?i voi mistui totul împrejurul ei".

\chapter{22}

\par 1 A?a a zis Domnul: "Coboara-te în casa regelui lui Iuda ?i roste?te cuvintele acestea ?i zi:
\par 2 Regele lui Iuda, cel ce ?ezi pe scaunul lui David, asculta cuvântul Domnului ?i tu ?i slugile tale ?i poporul tau, care intra?i pe por?ile acestea!
\par 3 A?a zice Domnul: Face?i judecata ?i dreptate ?i scoate?i pe cel asuprit din mâna asupritorului, nu asupri?i ?i nu împila?i pe strain, pe orfan ?i pe vaduva ?i sânge nevinovat sa nu varsa?i în locul acesta!
\par 4 Caci, de ve?i împlini cuvântul acesta, vor intra pe por?ile casei acesteia regii cei ce ?ed în locul lui David, pe scaunul lui, ?i umbla în care ?i pe cai, vor intra ?i ei ?i slugile lor ?i poporul lor.
\par 5 Iar de nu ve?i asculta cuvintele acestea, Ma jur pe Mine Însumi, zice Domnul, ca aceasta casa va ajunge pustie.
\par 6 De aceea, a?a zice Domnul casei regelui lui Iuda: Galaad Îmi e?ti ?i vârf de Liban, dar te voi face pustietate ?i ceta?i nelocuite.
\par 7 ?i voi gati împotriva ta pierzatori, care vor avea fiecare arma sa ?i var taia cei mai frumo?i cedri ai tai ?i-i vor arunca în foc.
\par 8 Multe popoare vor trece prin cetatea aceasta ?i vor zice unii catre al?ii: "Pentru ce a facut Domnul a?a cu aceasta cetate mare?"
\par 9 ?i li se va raspunde: "Pentru ca locuitorii ei au parasit legamântul Domnului Dumnezeului lor, s-au închinat la al?i dumnezei ?i au slujit acelora".
\par 10 Nu plânge?i dupa mort ?i nu-l boci?i, ci plânge?i amar dupa cel dus în robie, ca acela nu se va mai întoarce ?i nu î?i va mai vedea ?ara sa de na?tere;
\par 11 Caci a?a zice Domnul despre ?alum, fiul lui Iosia, regele lui Iuda, care a domnit dupa tatal sau, Iosia, ?i care a ie?it din locul acesta: "Nu se va mai întoarce acolo,
\par 12 Ci va muri în locul acela unde a fost dus rob ?i nu va mai vedea pamântul acesta.
\par 13 Vai de cel care î?i zide?te casa din nedreptate ?i î?i face încaperi din faradelegi, care sile?te pe aproapele sau sa-i lucreze degeaba ?i nu-i da plata lui,
\par 14 ?i care zice: "Am sa-mi fac casa mare ?i odai încapatoare, am sa fac ferestre, am sa le captu?esc cu cedru ?i am sa le vopsesc cu ro?u!
\par 15 Ai ajuns tu, oare, rege ca sa te fale?ti cu palate cladite din lemn de cedru? Tatal tau n-a mâncat oare ?i n-a baut? Israel a facut judecata ?i dreptate ?i de aceea i-a fost bine.
\par 16 El a judecat pricina saracului ?i a nenorocitului ?i de aceea i-a fost bine. Oare nu aceasta înseamna a Ma cunoa?te pe Mine? - zice Domnul.
\par 17 Inima ta însa ?i ochii tai cauta numai la lucrul tau ?i la varsarea sângelui nevinovat; cauta sa faca numai împilare ?i silnicie.
\par 18 De aceea, a?a zice Domnul despre Ioiachim, fiul lui Iosia, regele lui Iuda: "Nu-l vor plânge, zicând: "Vai, fratele meu!" sau: "Vai, sora mea!" Nu-l vor plânge, zicând: "Vai, doamne!" sau: "Vai, cinstea ta!"
\par 19 Ci el va fi îngropat ca un asin; îl vor târî ?i-l vor arunca departe peste por?ile Ierusalimului.
\par 20 Urca-te pe Liban ?i striga, înal?a-?i glasul de pe Vasan ?i striga de pe Abarim, ca s-au zdrobit to?i prietenii tai.
\par 21 În vremea propa?irii tale Eu ?i-am grait, dar tu ai zis: "Nu ascult!" A?a a fost purtarea ta chiar din tinere?ea ta ?i n-ai ascultat glasul Meu.
\par 22 Pe to?i pastorii tai îi va împra?tia vântul, iar prietenii tai se vor duce în robie. Atunci vei fi ru?inat ?i batjocorit pentru toate faptele tale cele rele.
\par 23 O, cel ce locuie?ti pe Liban ?i-?i faci cuibul în cedri, cum vei geme tu, când te vor ajunge chinurile ca durerile femeii ce na?te!
\par 24 Precum e adevarat ca Eu traiesc, a zis Domnul, tot a?a este de adevarat ca daca Iehonia, fiul lui Ioiachim, regele lui Iuda, ar fi inel la mâna Mea cea dreapta, apoi ?i de acolo l-a? smulge.
\par 25 ?i te voi da în mâinile celor ce vor sa-?i ia via?a ?i în mâinile celor de care tu te temi, în mâinile lui Nabucodonosor, regele Babilonului, ?i în mâinile Caldeilor;
\par 26 ?i te voi arunca pe tine ?i pe mama ta, care te-a nascut, în .ara straina, unde nu v-a?i nascut
\par 27 ?i ve?i muri acolo; iar în pamântul unde va dori sufletul vostru sa se întoarca, nu va ve?i mai întoarce".
\par 28 Au doara acest om, Iehonia, este faptura blestemata ?i lepadata? Sau este vas netrebnic? Pentru ce sunt arunca?i, el ?i neamul lui, ?i înca într-o ?ara pe care ei n-o ?tiau?
\par 29 O, ?ara, ?ara, ?ara, asculta cuvântul Domnului!
\par 30 A?a zice Domnul: "Scrie?i pe omul acesta ca lipit de copii, ca om nenorocit în zilele sale, pentru ca nimeni din neamul lui nu va mai ?edea pe tronul lui David ?i sa domneasca peste Iuda!"

\chapter{23}

\par 1 "Vai de parin?ii care pierd ?i împra?tie oile turmei Mele, zice Domnul.
\par 2 De aceea, a?a zice Domnul Dumnezeul lui Israel catre pastorii care pasc pe poporul Meu: A?i risipit oile Mele ?i le-a?i împra?tiat ?i nu le-a?i pazit; de aceea, va voi pedepsi pentru faptele voastre cele rele, zice Domnul.
\par 3 ?i voi aduna rama?i?ele turmei Mele din toate ?arile prin care le-am risipit, le voi întoarce la staulele lor ?i vor na?te ?i se vor înmul?i.
\par 4 Voi pune peste acestea pastori, care le vor pa?te ?i ele nu se vor mai teme, nici se vor mai speria, nici se vor mai pierde, zice Domnul.
\par 5 Iata vin zile, zice Domnul, când voi ridica lui David Odrasla dreapta ?i va ajunge rege ?i va domni cu în?elepciune; va face judecata ?i dreptate pe pamânt.
\par 6 În zilele Lui, Iuda va fi izbavit ?i Israel va trai în lini?te; iata numele cu care-L voi numi: "Domnul-dreptatea-noastra!"
\par 7 De aceea, vor veni zilele, zice Domnul, când nu se va mai zice: "Viu este Domnul, Care a scos pe fiii lui Israel din ?ara Egiptului",
\par 8 Ci: "Viu este Domnul, Care a scos ?i a adus neamul casei lui Israel din ?ara de la miazanoapte ?i din toate ?arile în care îi risipise ?i vor trai în pamântul acesta".
\par 9 Despre prooroci: "Mi se sfâ?ie inima în mine ?i toate oasele mi se cutremura; sunt ca un am beat, ca omul biruit de vin, pentru Domnul ?i pentru cuvintele Lui cele sfinte;
\par 10 Pentru ca s-a umplut ?ara de desfrâna?i ?i pentru ca plânge ?ara sub blestem; uscatu-s-au pa?unile în stepa; ?inta alergaturii lor este faradelegea ?i ?ara lor este nedreptatea.
\par 11 Ca profetul ?i preotul sunt necredincio?i; ?i pâna ?i în casa Mea am gasit rautatea lor, zice Domnul.
\par 12 De aceea, calea lor le va fi ca locurile lunecoase în întuneric; vor fi îmbrânci?i ?i vor cadea acolo, caci voi aduce asupra lor nenorocirea în anul când îi voi pedepsi, zice Domnul.
\par 13 ?i în proorocii Samariei am vazut nebunie, caci au profe?it în numele lui Baal ?i au dus în ratacire pe poporul Meu Israel.
\par 14 Dar în proorocii Ierusalimului am vazut grozavii: ace?tia fac desfrânare ?i umbla cu minciuni, ajuta mâinile facatorilor de rele, ca nimeni sa nu se întoarca de la necredin?a sa; to?i sunt pentru Mine ca Sodoma, ?i locuitorii lui ca Gomora.
\par 15 De aceea, a?a zice Domnul Savaot despre prooroci: Iata, îi voi hrani cu pelin ?i le voi da sa bea apa cu fiere, ca de la profe?ii Ierusalimului s-a întins necredin?a în toata ?ara.
\par 16 A?a zice Domnul Savaot: "Nu asculta?i cuvintele proorocilor, care va profe?esc, ca va în?eala, povestindu-va închipuirile inimii lor, ?i nimic din cele ale Domnului.
\par 17 Necontenit graiesc ei celor ce Ma dispre?uiesc: "Domnul, a zis ca va fi pace peste voi". ?i tuturor celor care urmeaza inima lor învârto?ata le zic: "Nici un rau nu va veni asupra voastra!"
\par 18 Ca cine a luat parte la sfatul Domnului, ca sa vada ?i sa auda cuvântul Lui? Sau cine a luat aminte la cuvântul Lui ca sa-l vesteasca?
\par 19 Iata vine furtuna Domnului cu iu?ime, vine furtuna mare ?i va cadea peste capetele necredincio?ilor.
\par 20 Mânia Domnului nu se va potoli pâna nu va împlini ?i va înfaptui planurile inimii Sale, ?i în zilele ce vin ve?i pricepe aceasta lamurit.
\par 21 N-am trimis Eu pe proorocii ace?tia, ci au alergat ei singuri; Eu nu le-am spus, ci au profe?it ei de la ei.
\par 22 Au luat ei parte la sfatul Meu? Atunci sa vesteasca cuvintele Mele poporului Meu, sa faca sa se întoarca oamenii de la calea lor rea ?i de la faptele lor cele rele.
\par 23 Au doara Eu numai de aproape sunt Dumnezeu, zice Domnul, iar de departe nu mai sunt Dumnezeu?
\par 24 Poate oare omul sa se ascunda în loc tainic, unde sa nu-l vad Eu, zice Domnul? Au nu umplu Eu cerul ?i pamântul, zice Domnul?
\par 25 Am auzit ce zic proorocii care profe?esc minciuna în numele Meu. Ei zic: "Am visat, am vazut în vis".
\par 26 Pâna când proorocii ace?tia vor spune minciuni ?i vor vesti în?elatoria inimii lor?
\par 27 Cred ei cu visele lor, pe care ?i le povestesc unul altuia, sa faca uitat, poporului Meu, numele Meu, a?a cum au uitat parin?ii lor numele Meu pentru Baal?
\par 28 Proorocul care a vazut vis, povesteasca-l ca vis, iar cel ce are cuvântul Meu, acela sa spuna cuvântul Meu adevarat. Ce legatura poate fi între pleava ?i grauntele de grâu curat, zice Domnul?
\par 29 Cuvântul Meu nu este el, oare, ca un foc, zice Domnul, ca un ciocan care sfarâma stânca?
\par 30 De aceea iata Eu, zice Domnul, sunt împotriva proorocilor care fura cuvântul Meu unul de la altul.
\par 31 Deoarece Eu, zice Domnul, sunt împotriva proorocilor care vorbesc cu limba lor, dar zic: "El a spus".
\par 32 Tot Eu, zice Domnul, sunt împotriva proorocilor care spun visuri mincinoase, care le povestesc pe acestea ?i duc pe poporul Meu la ratacire cu amagirile lor ?i cu lingu?irile lor, de?i Eu nu i-am trimis, nici le-am poruncit; ei nu aduc nici un folos poporului acestuia, zice Domnul.
\par 33 Deci, de te va întreba poporul acesta, sau vreun prooroc, sau vreun preot: "Care este amenin?area Domnului?" Sa le spui: "Ce amenin?are? Am sa va înlatur", zice Domnul.
\par 34 Iar daca un prooroc sau preot, sau poporul va zice: "Este amenin?area Domnului", voi pedepsi pe omul acela ?i casa lui.
\par 35 A?a sa zice?i unul catre altul ?i frate catre frate: "Ce-a raspuns Domnul?" sau: "Ce-a zis Domnul?"
\par 36 Iar cuvântul acesta: "Amenin?are de la Domnul", de-acum sa nu-l mai întrebuin?a?i, ca amenin?are va fi unui astfel de om cuvântul lui, pentru ca strica?i cuvintele Domnului celui viu, cuvintele Domnului Savaot, cuvintele Dumnezeului nostru.
\par 37 A?a sa zici proorocului: "Ce a raspuns Domnul?", sau: "Ce a zis Domnul?"
\par 38 ?i de ve?i mai zice: "Amenin?are de la Domnul", apoi a?a zice Domnul: Pentru ca spune?i acea vorba: "Amenin?are de la Domnul" când Eu am trimis sa vi se spuna: Nu mai zice?i "Amenin?are de la Domnul",
\par 39 De aceea, iata va voi uita cu totul ?i va voi parasi, ?i cetatea aceasta, pe care v-am dat-o voua ?i parin?ilor vo?tri, o voi lepada de la fa?a Mea,
\par 40 ?i voi pune asupra voastra ocara ?i necinste ve?nica care nu se vor uita".

\chapter{24}

\par 1 Dupa ce Nabucodonosor, regele Babilonului, a luat din Ierusalim robi pe Iehonia, fiul lui Ioiachim, regele lui Iuda ?i pe capeteniile lui Iuda cu lemnarii ?i cu fierarii ?i i-a dus la Babilon, mi-a aratat Domnul vedenia aceasta: Iata erau doua co?uri cu smochine, a?ezate înaintea templului Domnului:
\par 2 Un co? era cu smochine foarte bune, cum sunt smochinele varatice, iar celalalt co? era cu smochine foarte rele, care, de rele ce erau, nici nu se puteau mânca.
\par 3 Atunci mi-a zis Domnul: "Ce vezi tu, Ieremia?" ?i eu am zis: "Smochine! Smochinele cele bune sunt foarte bune, iar cele rele sunt foarte rele, încât nu se pot mânca de rele ce sunt".
\par 4 ?i a fost iara?i cuvântul Domnului catre mine ?i mi-a zis:
\par 5 "A?a zice Domnul Dumnezeul lui Israel: Ca pe aceste smochine bune, a?a voi privi cu bunavoin?a pe cei ai lui Iuda, du?i în robie, pe care i-am trimis din locul acesta în ?ara Caldeilor.
\par 6 ?i-Mi întorc ochii Mei asupra lor, spre binele lor, ?i-i voi întoarce în ?ara aceasta ?i nu-i voi nimici, ei-i voi zidi; nu-i vai smulge, ci-i voi sadi;
\par 7 Le voi da inima ca sa Ma cunoasca pe Mine ca Eu sunt Domnul ?i ei vor fi poporul Meu, iar Eu voi fi Dumnezeul lor, ca se vor întoarce la Mine cu toata inima lor".
\par 8 Despre smochinele rele care, de rele ce sunt, nu se pot mânca, zice Domnul: A?a voi face sa ajunga Sedechia, regele lui Iuda, capeteniile lui ?i restul Ierusalimului, cei care au ramas în ?ara aceasta ?i cei care traiesc în ?ara Egiptului;
\par 9 Îl voi face pricina de groaza, o nenorocire pentru toate regatele pamântului, ca sa fie de ocara ?i de pilda, de batjocura ?i blestem prin toate locurile pe unde-i voi izgoni.
\par 10 ?i voi trimite asupra lor sabie, foamete ?i ciuma, pâna vor pieri din ?ara pe care le-o dadusem lor ?i parin?ilor lor".

\chapter{25}

\par 1 Cuvântul care a fost catre Ieremia pentru tot poporul iudeu, în anul al patrulea al lui Ioiachim, fiut lui Iosia, regele lui Iuda sau în anul întâi al lui Nabucodonosor, regele Babilonului
\par 2 Pe care cuvânt l-a rostit proorocul Ieremia catre tot poporul iudeu ?i catre to?i locuitorii Ierusalimului:
\par 3 "De la anul al treisprezecelea al lui Iosia, fiul lui Amon, regele lui Iuda, ?i pâna în ziua aceasta - timp de douazeci ?i trei de ani - a fost cuvântul Domnului catre mine. ?i eu de diminea?a pâna seara v-am vorbit, dar voi n-a?i ascultat.
\par 4 Trimis-a Domnul la noi necontenit pe to?i robii Sai prooroci ?i voi n-a?i ascultat, nici v-a?i plecat urechea ca sa-i asculta?i.
\par 5 Vi s-a zis: Sa se întoarca fiecare de la calea sa cea rea ?i de la faptele sale cele urâte, ?i veri trai în ?ara pe care Domnul a dat-o voua ?i parin?ilor vo?tri din veac în veac.
\par 6 ?i sa nu umbla?i dupa al?i dumnezei, ca sa le sluji?i ?i sa va închina?i lor, ?i sa nu Ma mânia?i cu faptele mâinilor voastre, ?i nu va voi face nici un fel de rau.
\par 7 Voi însa nu M-a?i ascultat, zice Domnul, ci M-a?i mâniat eu faptele mâinilor voastre, spre raul vostru.
\par 8 De aceea, a?a zice Domnul Savaot: Pentru ca n-a?i ascultat cuvintele Mele,
\par 9 Iata, voi trimite sa aduca toate neamurile de la miazanoapte, zice Domnul; ?i voi trimite la Nabucodonosor, regele Babilonului, robul Meu, ca sa le aduca împotriva acestei ?ari ?i a locuitorilor ei ?i împotriva tuturor neamurilor dimprejur pe care le voi pustii, 1e voi înspaimânta, le voi face de râs ?i de ocara ve?nica.
\par 10 Nu vor mai putea avea glas de bucurie ?i glas de veselie, glasul mirelui ?i glasul miresei, uruitul pietrelor de moara ?i lumina sfe?nicului.
\par 11 Toata ?ara aceasta va fi pustiita ?i jefuita; popoarele acestea vor sluji regelui Babilonului ?aptezeci de ani.
\par 12 Iar când se vor împlini ?aptezeci de ani, voi pedepsi pe regele Babilonului ?i pe poporul acela, zice Domnul, pentru necredin?a lor, ?i ?ara Caldeilor o voi pedepsi ?i o voi face pustie pentru totdeauna.
\par 13 Voi împlini asupra ?arii acesteia toate cuvintele Mele pe care le-am rostit împotriva ei; tot ce-i scris în cartea aceasta ?i ceea ce Ieremia a proorocit împotriva tuturor neamurilor.
\par 14 Pentru ca ?i pe ele le vor robi multe popoare ?i regi mari ?i le voi rasplati dupa purtarea lor ?i dupa faptele mâinilor lor.
\par 15 Ca a?a mi-a zis Domnul Dumnezeul lui Israel: "Ia din mâna Mea cupa aceasta cu vinul urgiei ?i adapa cu ea toate popoarele la care te voi trimite;
\par 16 Acelea vor bea ?i se vor clatina ?i vor înnebuni la vederea sabiei pe care o voi trimite asupra lor!"
\par 17 ?i am luat cupa din mâna Domnului ?i am dat sa bea tuturor neamurilor, la care m-a trimis Domnul.
\par 18 Ierusalimului ?i ceta?ilor lui Iuda, regilor lui ?i capeteniilor lui, spre pustiire ?i groaza, spre batjocura ?i blestem, precum se ?i vede astazi;
\par 19 Lui Faraon, regele Egiptului, slujitorilor lui, capeteniilor lui ?i întregului popor al lui;
\par 20 La toata Arabia, tuturor regilor ?arii U?, tuturor regilor ?arii Filistenilor, Ascalonului, Gazei, Ecronului ?i rama?i?elor din A?dod;
\par 21 Edomului, Moabului ?i fiilor lui Amon;
\par 22 Tuturor regilor Tirului, tuturor regilor Sidonului ?i regilor insulelor care sunt dincolo de mare:
\par 23 Dedanului ?i Temei, Buzului ?i tuturor care-?i rad tâmplele;
\par 24 Tuturor regilor Arabiei ?i tuturor regilor popoarelor amestecate, care locuiesc în pustiu;
\par 25 Tuturor regilor Zimrei, tuturor regilor Elamului ?i tuturor regilor Mediei;
\par 26 Tuturor regilor de la miazanoapte, de aproape sau de departe, unora ?i altora, ?i tuturor regatelor lumii, care se afla pe fa?a pamântului; iar regele ?i?acului va bea dupa ei.
\par 27 ?i sa le zici: "A?a zice Domnul Savaot, Dumnezeul lui Israel: Be?i ?i va îmbata?i, varsa?i ?i cade?i ?i nu va ridica?i la vederea sabiei pe care o trimit Eu asupra voastra!"
\par 28 Iar de nu vor vrea sa ia cupa din mâna ta, ca sa bea, sa le zici:
\par 29 "A?a graie?te Domnul Savaot: Trebuie sa be?i numaidecât, pentru ca încep sa aduc nenorocire asupra ceta?ii acesteia, peste care s-a chemat numele Meu. Sa ramâne?i voi nepedepsi?i? Nu, nu ve?i ramâne nepedepsi?i, caci voi chema sabie asupra tuturor celor ce locuiesc pe pamânt", zice Domnul Savaot.
\par 30 De aceea prooroce?te asupra lor toate cuvintele acestea ?i le zi: "Domnul va striga cu putere de sus ?i din loca?ul sfin?eniei Sale Î?i va ridica glasul Sau; va striga cu putere împotriva pa?unii Sale; va striga, ca cei ce calca în teasc, tuturor celor ce locuiesc pe pamânt.
\par 31 Zgomotul va ajunge pâna la marginea pamântului, caci Domnul deschide procesul neamurilor; intra la judecata cu tot trupul ?i pe cei nelegiui?i îi va da sabiei", zice Domnul.
\par 32 A?a zice Domnul Savaot: "Iata nenorocirea se întinde de la neam la neam ?i vijelie mare se ridica de la marginile pamântului.
\par 33 ?i în ziua aceea, cei lovi?i de Domnul vor zacea de la un capat la celalalt al pamântului ?i nu vor fi boci?i, nici nu vor fi aduna?i ?i îngropa?i, ci vor sta ca gunoiul pe fa?a pamântului:
\par 34 Plânge?i, pastori, ?i suspina?i ?i tavali?i-va în cenu?a, stapâni ai turmei, ca s-au împlinit zilele, ca sa fiii junghia?i ?i împra?tia?i.
\par 35 Atunci ve?i cadea ca berbecii ale?i. Nu va fi adapost pentru pastori, nici scapare pentru stapânii turmei.
\par 36 Asculta?i strigatul pastorilor ?i urletul stapânilor turmei, ca Domnul a pustiit pa?unea lor.
\par 37 ?i câmpiile cele pa?nice sunt pustiite de urgia mâniei Domnului.
\par 38 Parasitu-?i-a el loca?ul sau, ca un leu, ?i ?ara lor s-a pustiit de urgia pustiitorului ?i de mânia aprinsa a Domnului".

\chapter{26}

\par 1 La începutul domniei lui Ioiachim, fiul lui Iosia, a fost cuvântul acesta de la Domnul:
\par 2 "A?a zice Domnul: Stai în curtea templului Domnului ?i graie?te tuturor ceta?ilor lui Iuda, care vin la închinare în templul Domnului, toate cuvintele ce î?i voi porunci sa le graie?ti, sa nu la?i nici un cuvânt.
\par 3 Poate vor asculta ?i se vor întoarce de la calea cea rea ?i atunci Îmi va parea rau de nenorocirea pe care aveam de gând sa le-o fac din cauza faptelor lor rele.
\par 4 ?i sa le spui: A?a zice Domnul: "Daca nu Ma ve?i asculta sa umbla?i dupa legea Mea, pe care v-am dat-o,
\par 5 ?i sa lua?i aminte la cuvintele robilor Mei prooroci, pe care-i trimit la voi, pe care-i trimit dis-de-diminea?a, ?i voi nu-i asculta?i,
\par 6 Atunci voi face cu templul acesta ceea ce am facut cu ?ilo, iar cetatea aceasta o voi da spre blestem tuturor popoarelor pamântului".
\par 7 Preo?ii ?i proorocii ?i tot poporul au ascultat pe Ieremia, când a grait el aceste cuvinte în templul Domnului.
\par 8 Iar dupa ce a spus Ieremia tot ce-i poruncise Domnul sa spuna la tot poporul, preo?ii, proorocii ?i tot poporul l-au prins ?i i-au zis: "Tu trebuie sa mori!
\par 9 De ce prooroce?ti în numele Domnului ?i zici: Templul acesta va fi ca ?ilo ?i cetatea aceasta se va pustii ?i va ramâne fara locuitori?" ?i s-a adunat tot poporul împotriva lui Ieremia la templul Domnului.
\par 10 ?i când au auzit acestea, capeteniile lui Iuda au venit din casa regelui la templul Domnului ?i au ?ezut la intrare în poarta cea noua a templului.
\par 11 Atunci preo?ii ?i proorocii au zis catre capetenii ?i catre tot poporul a?a: "Osânda cu moarte se cuvine acestui om, pentru ca prooroce?te împotriva ceta?ii acesteia, cum a?i auzit ?i voi cu urechile voastre!"
\par 12 Iar Ieremia a zis catre toate capeteniile ?i catre tot poporul: "Domnul m-a trimis sa proorocesc împotriva templului acestuia ?i împotriva ceta?ii acesteia cuvintele pe care le-a?i auzit.
\par 13 Îndrepta?i-va dar caile voastre ?i faptele voastre ?i supune?i-va glasului Domnului Dumnezeului vostru, ?i Domnului Îi va parea rau de nenorocirea pe care o rostise împotriva voastra.
\par 14 Iar cât despre mine, iata sunt în mâinile voastre, face?i cu mine ce vi se pare bun ?i drept!
\par 15 Dar sa ?ti?i bine ca, de ma ve?i omorî, ve?i aduce asupra voastra, asupra ceta?ii acesteia ?i asupra locuitorilor ei sânge nevinovat, caci cu adevarat Domnul m-a trimis la voi sa va spun în urechile voastre toate cuvintele acelea".
\par 16 Atunci capeteniile ?i tot poporul au zis catre preo?i ?i catre prooroci: "Omul acesta nu este vrednic de osânda cu moarte, pentru ca ne-a grait în numele Domnului Dumnezeului nostru".
\par 17 Atunci s-au ridicat unii din batrânii poporului ?i au zis catre toata adunarea poporului:
\par 18 "Miheia din More?et a proorocit în zilele lui Iezechia, regele lui Iuda, ?i a zis catre tot poporul iudeu: a?a zice Domnul Savaot: "Sionul va fi arat ca un ogor, Ierusalimul va ajunge o movila de darâmaturi ?i muntele templului acestuia va fi un deal împadurit".
\par 19 Omorâtu-l-a oare pentru aceasta Iezechia, regele lui Iuda ?i tot Iuda? Au nu s-a temut el de Domnul ?i nu s-a rugat Domnului, ca sa-I para rau de nenorocirea pe care o rostise împotriva lor? ?i noi sa ne împovaram sufletele noastre cu o nelegiuire a?a de mare!
\par 20 De asemenea a mai proorocit în numele Domnului un oarecare Urie, fiul lui ?emaia, din Chiriat-Iearim, ?i a proorocit împotriva ceta?ii acesteia ?i împotriva ?arii acesteia tocmai cu acelea?i cuvinte, ca ?i Ieremia.
\par 21 ?i când au auzit cuvintele lui regele Ioiachim ?i to?i curtenii lui ?i toate capeteniile, a cautat regele sa-l omoare. ?i auzind de aceasta, Urie s-a temut ?i a fugit ?i a trecut în Egipt.
\par 22 Dar regele Ioiachim a trimis ?i în Egipt oameni ?i anume: Pe Elnatan, fiul lui Acbor, ?i pe al?ii împreuna cu ace?tia.
\par 23 ?i au adus pe Urie din Egipt ?i l-au înfa?i?at la regele Ioiachim, ?i acesta l-a ucis cu sabia ?i a aruncat trupul lui acolo unde erau mormintele oamenilor de rând".
\par 24 Dar Ahicam, fiul lui ?afan, ocrotea pe Ieremia, ca sa nu fie dat în mâna poporului spre ucidere.

\chapter{27}

\par 1 La începutul domniei lui Ioiachim, fiul lui Iosia, re?ele lui Iuda, a fost de la Domnul catre Ieremia acest cuvânt:
\par 2 "A?a mi-a zis Domnul: Fa-?i catu?e ?i jug ?i le pune pe grumaz;
\par 3 Trimite la fel regelui iui Iuda ?i regelui Moabului, regelui fiilor lui Amon, regelui Tirului ?i regelui Sidonului, prin solii care au venit în Ierusalim la Sedechia, regele Ierusalimului;
\par 4 ?i porunce?te-le sa spuna stapânilor lor: Iata ce zice Domnul Savaot, Dumnezeul lui Israel, a?a sa spune?i stapânilor vo?tri:
\par 5 "Eu am facut pamântul ?i pe om ?i vie?uitoarele cele de pe fa?a pamântului cu puterea Mea cea mare ?i cu bra?ul Meu cel puternic ?i l-am dat cui am vrut.
\par 6 ?i acum voi da toate ?arile acestea în mâna lui Nabucodonosor, regele Babilonului, robul Meu, ba ?i fiarele câmpului le voi da lui spre slujba;
\par 7 Toate popoarele vor sluji lui ?i fiului lui ?i fiului fiului lui, pâna când îi va veni vremea ?i lui ?i ?arii lui; îi vor sluji popoare multe ?i regi mari.
\par 8 Daca vreun popor sau vreun regat nu va voi sa-i slujeasca lui Nabucodonosor, regele Babilonului, ?i nu-?i va pleca grumazul sau sub jugul regelui Babilonului, pe acel popor îl voi pedepsi cu sabie, cu foamete ?i cu ciuma, pâna-l voi stârpi cu mâna lui, zice Domnul.
\par 9 Iar voi sa nu asculta?i pe proorocii vo?tri, pe ghicitorii vo?tri, pe visatorii vo?tri, pe vrajitorii vo?tri ?i pe-ai vo?trii cititori de stele care va zic: "Nu ve?i sluji regelui Babilonului",
\par 10 Ca ace?tia va prezic minciuna, ca sa va departeze din ?ara voastra ?i ca Eu sa va izgonesc, ca sa pieri?i.
\par 11 Iar pe poporul care-?i va pleca grumazul sub jugul regelui Babilonului ?i i se va supune, îl voi lasa în pamântul sau ?i-l va lucra ?i va trai pe el", zice Domnul.
\par 12 ?i lui Sedechia, regele lui Iuda, i-am grait toate cuvintele acestea ?i am zis: "pleca?i-va grumazul sub jugul regelui Babilonului ?i slu?i?i-i lui ?i poporului. lui ?i ve?i trai.
\par 13 De ce sa mori tu ?i poporul tau de sabie, de foamete ?i de ciuma, cum a zis Domnul de poporul acela, care nu va slu?i regelui Babilonului?
\par 14 ?i sa nu asculta?i cuvintele proorocilor care va zic:
\par 15 "Nu ve?i sluji regelui Babilonului, ca minciuna va prezic ?i nici nu i-am trimis Eu, zice Domnul; ei va profe?esc mincinos în numele Meu, ca sa va izgonesc ?i sa pieri?i ?i voi ?i proorocii vo?tri, care va proorocesc".
\par 16 Preo?ilor ?i poporului întreg le-am grait acestea: "A?a zice Domnul: Nu asculta?i cuvintele proorocilor vo?tri care va proorocesc ?i va zic: Iata, curând vor fi întoarse de la Babilon vasele templului Domnului, ca minciuna va proorocesc.
\par 17 Nu-i asculta?i, ci sluji?i regelui Babilonului ?i ve?i trai. De ce sa duce?i cetatea aceasta la pustiire?
\par 18 Iar daca sunt ei prooroci ?i daca au ei cuvântul Domnului, atunci sa mijloceasca înaintea Domnului Savaot ca vasele ramase în templul Domnului ?i în casa regelui lui Iuda ?i în Ierusalim sa nu treaca la Babilon.
\par 19 Ca a?a zice Domnul Savaot de stâlpi, de baia cea de arama, de postamente ?i de celelalte lucruri care au ramas în cetatea aceasta,
\par 20 ?i pe care Nabucodonosor, regele Babilonului, nu le-a luat când a dus din Ierusalim la Babilon pe Iehonia, fiul lui Ioiachim, regele lui Iuda, ?i pe to?i îi însemna?i din Iuda ?i din Ierusalim;
\par 21 A?a zice Domnul Savaot, Dumnezeul lui Israel, despre vasele care au ramas în templul Domnului ?i în casa regelui lui Iuda ?i în Ierusalim:
\par 22 "Vor fi duse ?i acelea la Babilon ?i vor ramâne acolo pâna în ziua când le voi cauta Eu ?i le voi scoate ?i le voi întoarce la locul acesta, zice Domnul".

\chapter{28}

\par 1 Tot în anul acela, la începutul domniei lui Sedechia, regele lui Iuda, adica în anul al patrulea, în luna a cincea, Anania, fiul lui Azur, proorocul cel din Gabon, mi-a grait în templul Domnului, înaintea ochilor preo?ilor ?i a tot poporul ?i mi-a zis:
\par 2 "A?a zice Domnul Savaot, Dumnezeul lui Israel: Voi sfarâma jugul regelui Babilonului;
\par 3 Peste doi ani voi întoarce în locul acesta toate vasele templului Domnului, pe care Nabucodonosor, regele Babilonului, le-a luat din acest loc ?i le-a dus la Babilon.
\par 4 Voi întoarce la locul acesta ?i pe Iehonia, fiul lui Ioiachim, regele lui Iuda, ?i pe to?i robii iudei, care au mers la Babilon, zice Domnul, caci voi sfarâma jugul regelui Babilonului".
\par 5 Atunci proorocul Ieremia a grait catre Anania proorocul, înaintea ochilor preo?ilor ?i înaintea ochilor a tot poporul, ce statea în templul Domnului, ?i a zis Ieremia proorocul:
\par 6 "A?a sa fie ?i sa faca aceasta Domnul! Împlineasca Domnul cuvintele tale, pe care le-ai rostit tu pentru întoarcerea din Babilon a vaselor templului Domnului ?i a tuturor robilor la locul acesta.
\par 7 Dar asculta cuvântul acesta pe care ?i-l spun eu în auzul tau ?i în auzul a tot poporul:
\par 8 Proorocii care au fost demult înaintea mea ?i înaintea ta au prevestit multor ?ari ?i regate puternice razboi, strâmtorare ?i molima.
\par 9 Când însa vreun prooroc a prevestit pace, atunci numai a?a a fost luat el ca prooroc, cu adevarat trimis de Domnul, daca s-a împlinit cuvântul acelui prooroc".
\par 10 Iar proorocul Anania a luat jugul de pe grumazul lui Ieremia proorocul ?i l-a sfarâmat;
\par 11 ?i a zis Anania înaintea ochilor a tot poporul cuvintele acestea: "Iata ce zice Domnul: A?a voi sfarâma jugul lui Nabucodonosor, regele Babilonului, peste doi ani, luându-l de pe grumazul tuturor popoarelor". ?i Ieremia s-a dus în drumul sau.
\par 12 Dupa ce proorocul Anania a sfarâmat jugul de pe grumazul proorocului Ieremia, a fost cuvântul Domnului catre Ieremia ?i i-a zis:
\par 13 "Mergi ?i spune lui Anania: A?a zice Domnul: Tu ai sfarâmat un jug de lemn ?i voi face în locul lui altui de fier,
\par 14 Ca a?a zice Domnul Savaot, Dumnezeul lui Israel: Jug de fier voi pune pe grumazul tuturor acestor popoare, ca sa munceasca ele la Nabucodonosor, regele Babilonului, ?i ca sa-i slujeasca, ba ?i fiarele câmpului le voi da lui!"
\par 15 ?i a mai zis proorocul Ieremia catre Anania proorocul: "Asculta, Anania: Domnul nu te-a trimis ?i tu dai poporului acestuia încredere în minciuna.
\par 16 De aceea a?a zice Domnul: Iata te voi arunca de pe fa?a pamântului; chiar în anul acesta vei muri, pentru ca ai grait împotriva Domnului".
\par 17 ?i a murit Anania proorocul chiar în anul acela, în luna a ?aptea.

\chapter{29}

\par 1 Iata cuvintele scrisorii pe care Ieremia proorocul a trimis-o din Ierusalim catre rama?i?a batrânilor din robie, preo?ilor ?i proorocilor ?i catre tot poporul pe care Nabucodonosor i-a dus din Ierusalim la Babilon,
\par 2 Dupa ce a ie?it din Ierusalim regele Iehonia ?i regina ?i eunucii ?i capeteniile lui Iuda ?i ai Ierusalimului ?i lemnarii ?i fierarii,
\par 3 Prin Eleasa, fiul lui ?afan, ?i prin Ghemaria, fiul lui Hilchia, pe care Sedechia, regele lui Iuda, i-a trimis la Babilon catre Nabucodonosor, regele Babilonului:
\par 4 "A?a zice Domnul Savaot, Dumnezeul lui Israel, catre to?i robii pe care i-am stramutat din Ierusalim la Babilon: Zidi?i casa ?i trai?i;
\par 5 Face?i gradini ?i mânca?i roadele lor!
\par 6 Lua?i femei ?i na?te?i fii ?i fiice! Fiilor vo?tri lua?i-le so?ii, iar pe fiicele voastre marita?i-le, ca sa nasca fii ?i fiice, ?i sa nu va împu?ina?i acolo, ci înmul?iri-va!
\par 7 Cauta?i binele ?arii în care v-am dus robi ?i ruga?i-va Domnului pentru ea, ca de propa?irea ei atârna ?i fericirea voastra!
\par 8 Pentru ca a?a zice Domnul Savaot, Dumnezeul lui Israel: Sa nu va lasa?i amagi?i de proorocii vo?tri ?i de ghicitorii vo?tri care sunt în mijlocul vostru, ?i sa nu asculta?i visele voastre, pe care le ve?i visa,
\par 9 Caci va proorocesc minciuna ?i Eu nu i-am trimis, zice Domnul!
\par 10 Pentru ca Domnul zice: Când vi se vor împlini în Babilon ?aptezeci de ani, atunci va voi cerceta ?i voi împlini cuvântul Meu cel bun pentru voi, ca sa va întoarce?i la locul acesta.
\par 11 Pentru ca numai Eu ?tiu gândul ce-l am pentru voi, zice Domnul, gând bun, nu rau, ca sa va dau viitorul ?i nadejdea.
\par 12 ?i ve?i striga catre Mine ?i ve?i veni ?i va ve?i ruga Mie, ?i Eu va voi auzi;
\par 13 ?i Ma ve?i cauta ?i Ma ve?i gasi, daca Ma ve?i cauta cu toata inima voastra.
\par 14 ?i voi fi gasit de voi, zice Domnul, ?i va voi întoarce din robie ?i va voi strânge din toate popoarele ?i din toate locurile, de pe unde v-am izgonit, zice Domnul, ?i va voi întoarce din locul de unde v-am dus.
\par 15 Voi zice?i: "Domnul ne-a ridicat prooroci în Babilon".
\par 16 A?a zice Domnul de regele care ?ade pe tronul lui David ?i de tot poporul care traie?te în cetatea aceasta ?i de fra?ii vo?tri care n-au fost du?i cu voi în robie.
\par 17 A?a zice Domnul de ace?tia: Iata, voi trimite asupra lor sabie, foamete ?i ciuma ?i-i voi face ca smochinele cele stricate care, de rele ce sunt, nu se pot mânca,
\par 18 ?i-i voi urmari cu sabie, cu foamete ?i cu ciuma, ?i-i voi da spre chin în toate regatele pamântului, spre blestem ?i grozavie, spre râs ?i batjocura în fa?a tuturor popoarelor la care-i voi izgoni,
\par 19 Pentru ca n-au ascultat cuvintele Mele, zice Domnul, de?i le-am trimis dis-de-diminea?a pe slujitorii Mei, proorocii; zice Domnul.
\par 20 Iar voi to?i cei ce sunte?i în robie, pe care v-am trimis din Ierusalim la Babilon, asculta?i cuvântul Domnului.
\par 21 A?a zice Domnul Savaot, Dumnezeul lui Israel despre Ahab, fiul lui Colaia ?i despre Sedechia, fiul lui Maaseia, care va proorocesc voua minciuna în numele Meu: pe ace?tia îi voi da în mâinile lui Nabucodonosor, regele Babilonului, ?i acela îi va omorî înaintea ochilor vo?tri.
\par 22 ?i se va obi?nui dupa ei. printre to?i robii lui Iuda din Babilon a se blestema astfel: "Sa-i faca Domnul cum a facut lui Sedechia ?i lui Ahab", pe care i-a fript regele Babilonului pe jeratic,
\par 23 Pentru ca au facut ticalo?ie în Israel: se desfrânau cu femeile aproapelui lor ?i spuneau minciuni în numele Meu, ceea ce Eu nu le-am poruncit; Eu ?tiu acestea ?i sunt martor, zice Domnul.
\par 24 Iar lui ?emaia Nehelamitul spune-i:
\par 25 A?a zice Domnul Savaot, Dumnezeul lui Israel: Pentru ca ai trimis scrisori în numele tau catre tot poporul cel din Ierusalim ?i catre preotul Sofonie, fiul lui Maaseia, ?i catre to?i preo?ii ?i ai scris:
\par 26 "Domnul te-a pus preot în locul preotului Iehoiada, ca sa fii supraveghetor în templul Domnului peste tot omul nebun ?i peste tot omul ce prooroce?te, ?i ca sa pui pe unul ca acesta în temni?a ?i în catu?e:
\par 27 Pentru ce dar nu opre?ti pe Ieremia din Anatot de a mai prooroci acolo la voi?
\par 28 Ca acesta ?i în Babilon a trimis sa mi se spuna: "Robia va fi lunga: zidi?i case ?i locui?i în ele, sadi?i gradini ?i mânca?i roadele lor".
\par 29 Preotul Sofonie a citit aceasta scrisoare în auzul proorocului Ieremia,
\par 30 ?i cuvântul Domnului a fost catre Ieremia ?i a zis:
\par 31 "Trimite la to?i cei stramuta?i sa li se spuna: A?a zice Domnul de ?emaia Nehelamitul: Pentru ca ?emaia prooroce?te la voi, dar Eu nu l-am trimis, ?i va da încredere într-o minciuna,
\par 32 De aceea a?a zice Domnul: Iata Eu voi pedepsi pe ?emaia Nehelamitul ?i neamul lui; nu va fi din el om care sa locuiasca în mijlocul poporului acestuia, iar el nu va vedea binele acela, pe care Eu îl voi face poporului Meu, zice Domnul, pentru ca a grait împotriva Domnului".

\chapter{30}

\par 1 Cuvântul Domnului care a fost catre Ieremia:
\par 2 "A?a zice Domnul Dumnezeul lui Israel: Scrie-?i într-o carte toate cuvintele pe care ?i le-am grait Eu,
\par 3 Ca iata vin zile, zice Domnul când voi întoarce din robie pe poporul Meu, pe Israel ?i pe Iuda, zice Domnul, ?i-i voi aduce iara?i în pamântul acela, pe care l-am dat parin?ilor lor, ?i-l vor stapâni".
\par 4 Iata cuvintele acelea pe care le-a spus Domnul despre Israel ?i despre Iuda:
\par 5 A?a a zis Domnul: "Glas de tulburare, glas de groaza auzim, ?i nu glas de pace.
\par 6 Întreba?i ?i vede?i daca na?te vreun barbat? De ce vad Eu pe to?i barba?ii cu mâinile pe ?olduri, ca o femeie care na?te, ?i fe?ele tuturor palide?
\par 7 O, vai! Mare este ziua aceea! Asemenea ei n-a mai fost alta! Acela este timp de mare strâmtorare pentru Iacov, dar el va fi izbavit din ea.
\par 8 ?i în ziua aceea, zice Domnul Savaot, voi sfarâma jugul de pe grumazul lor ?i lan?urile lor le voi rupe.
\par 9 ?i nu vor mai sluji strainilor, ci vor sluji Domnului Dumnezeului lor ?i lui David, regele lor, pe care îl voi pune iara?i pe tron.
\par 10 ?i tu robul Meu Iacov, nu te teme, zice Domnul, nici nu te înspaimânta, Israele, ca iata Eu te voi scapa din ?ara cea departata ?i neamul tau îl voi aduce din ?ara robiei tale. ?i se va întoarce Iacov ?i va trai în pace ?i în lini?te ?i nimeni nu-l va mai îngrozi.
\par 11 Caci Eu sunt cu tine, zice Domnul, ca sa te izbavesc; voi pierde de fot toate popoarele, printre care te-am risipit pe tine, iar pe tine nu te voi pierde; te voi pedepsi dupa dreptate ?i nepedepsit nu te voi lasa,
\par 12 Ca a?a zice Domnul: Rana ta e de nevindecat ?i plaga ta e dureroasa.
\par 13 Nimeni nu se îngrije?te de pricina ta, ca sa-?i vindece rana, ?i leac vindecator nu este pentru tine.
\par 14 To?i prietenii tai te-au uitat ?i nu te mai cauta; ca Eu te-am lovit ca pe un vrajma?, cu pedeapsa cruda, pentru mul?imea faradelegilor tale, pentru ca se înmul?isera pacatele tale.
\par 15 De ce te plângi de rana ta? Durerea ta este de nevindecat? Dupa mul?imea faradelegilor tale ?i-am facut aceasta, pentru ca se înmul?isera pacatele tale.
\par 16 Dar to?i cei ce te manânca vor fi mânca?i ?i to?i vrajma?ii tai vor merge ?i ei to?i în robie ?i to?i jefuitorii tai vor fi jefui?i.
\par 17 ?i î?i voi lega ranile ?i te voi vindeca, zice Domnul. Ca ei te numesc: "Cel izgonit", "Sionul, de care nu mai întreaba nimeni!"
\par 18 A?a zice Domnul: "Iata voi restatornici corturile lui Iacov ?i sala?urile lui le voi milui; cetatea va fi zidita iar pe dealul sau ?i templul se va zidi ca mai înainte.
\par 19 Se vor înal?a din ele mul?umiri ?i glas de bucurie; îi voi înmul?i ?i nu vor mai fi împu?ina?i, îi voi încununa cu glorie ?i nu vor mai fi înjosi?i.
\par 20 Fiii lui Iacov vor fi ca mai înainte, adunarea lui va sta înaintea Mea ?i Eu voi pedepsi pe to?i asupritorii lui.
\par 21 Capetenia lui va fi din el ?i stapânul lui se va ridica din mijlocul lui; îl voi apropia ?i se va apropia de Mine: ca cine va îndrazni sa se apropie singur de Mine? - zice Domnul.
\par 22 Voi ve?i fi poporul Meu, ?i Eu voi fi Dumnezeul vostru.
\par 23 Iata vijelie salbatica vine de la Domnul, vijelie groaznica, ?i va cadea pe capul necredincio?ilor.
\par 24 Mânia cea aprinsa a Domnului nu se va abate pâna ce nu se va sfâr?i ?i nu se va împlini gândul inimii Lui. În zilele cele de apoi ve?i pricepe aceasta".

\chapter{31}

\par 1 "În vremea aceea Eu voi fi Dumnezeul tuturor semin?iilor lui Israel, iar ele vor fi poporul Meu", zice Domnul.
\par 2 A?a zice Domnul: "Poporul care a scapat de sabie a aflat mila în pustiu. Israel a ajuns la locul sau de odihna".
\par 3 Atunci mi S-a aratat Domnul din departare ?i mi-a zis: "Cu iubire ve?nica te-am iubit ?i de aceea Mi-am întins spre tine bunavoin?a.
\par 4 Din nou te voi pune în rânduiala, fecioara lui Israel, din nou te voi înfrumuse?a cu timpanele tale ?i vei ie?i la hora celor ce se veselesc;
\par 5 Din nou vei lua viile de pe dealurile Samariei, ?i vierii care le vor lucra se vor folosi ei singuri de ele.
\par 6 Ca va veni vremea când strajile de pe muntele Efraim vor striga: "Scula?i-va sa ne suim în Sion, la Domnul Dumnezeul nostru!"
\par 7 Caci a?a zice Domnul: "Striga?i de bucurie pentru Iacov ?i striga?i înaintea capeteniei popoarelor, vesti?i; lauda?i ?i zice?i: Doamne, izbave?te pe poporul Tau, rama?i?a lui Israel!
\par 8 Iata, îi voi aduce din ?ara cea de la miazanoapte ?i-i voi aduna de la marginile pamântului. Orbul ?i ?chiopul, cea însarcinata ?i cea care na?te vor fi împreuna cu ei; se va întoarce aici mare mul?ime.
\par 9 Cu lacrimi au plecat ?i îi voi întoarce cu mângâiere, îi voi întoarce la izvoarele apelor pe cale neteda, pe care nu se vor poticni, ca Eu sunt parintele lui Israel ?i Efraim este întâiul Meu nascut".
\par 10 Asculta?i, popoare, cuvântul Domnului, vesti?i ?arii departate ?i zice?i: Cel ce a împra?tiat pe Israel, Acela îl va aduna ?i-l va pazi, ca pastorul turma sa, ca va rascumpara Domnul pe Iacov,
\par 11 ?i-l va izbavi din mâna celui ce a fost mai tare decât el.
\par 12 ?i vor veni ?i vor praznui pe înal?imile Sionului ?i vor curge bunata?ile Domnului: grâu ?i vin ?i untdelemn, miei ?i boi. Sufletul lor va fi ca o gradina bine-udata ?i ei nu vor mai tânji.
\par 13 Atunci fecioara se va veseli la hora, tineri ?i batrâni vor fi ferici?i. Voi schimba întristarea lor în veselie ?i-i voi mângâia dupa întristarea lor ?i-i voi bucura.
\par 14 Voi hrani sufletele preo?ilor cu grasime ?i poporul Meu se va satura din bunata?ile Mele", zice Domnul.
\par 15 A?a zice Domnul: "Glas se aude în Rama, bocet ?i plângere amara. Rahila î?i plânge copiii ?i nu vrea sa se mângâie de copiii sai, pentru ca nu mai sunt".
\par 16 A?a zice Domnul: Înfrâneaza-?i glasul de la bocet ?i ochii de la lacrimi, ca vei avea plata pentru munca ta, zice Domnul, ?i ei se vor întoarce din ?ara du?mana;
\par 17 Este nadejde pentru viitorul tau, zice Domnul, ca fiii tai se vor întoarce în hotarele lor.
\par 18 Aud pe Efraim plângând ?i zicând: "Tu m-ai pedepsit ?i sunt pedepsit ca un junc neînva?at. Întoarce-ma ?i ma voi întoarce, ca Tu e?ti Domnul Dumnezeul meu!
\par 19 Dupa ce m-am întors, m-am cait, ?i când am luat cuno?tin?a, m-am batut în piept; am fost ru?inat ?i tulburat am fost, pentru ca am purtat ocara tinere?ii mele".
\par 20 Dar Efraim nu este  feciorul Meu scump, un copil atât de alintat? Atunci când vorbesc de el, totdeauna cu dragoste Mi-amintesc de el; pentru el Mi se mi?ca inima ?i voi avea mila de el, zice Domnul.
\par 21 Pune-?i semne pe linga drum, pune-?i stâlpi, fii atenta la drum, la calea pe care te-ai dus; întoarce-te, fecioara lui Israel, întoarce-te în aceste ceta?i ale tale.
\par 22 Pâna când vei rataci, fiica neascultatoare? Pentru ca Domnul va face pe pamânt lucru nou: femeia va cauta pe barbatul ei.
\par 23 A?a zice Domnul Savaot, Dumnezeul lui Israel: "De acum înainte, când voi întoarce robia lor, vor grai în pamântul lui Iuda ?i în ceta?ile lui cuvintele acestea: "Loca? al drepta?ii ?i munte sfânt, Domnul sa te binecuvânteze!"
\par 24 ?i se vor a?eza în aceasta ?ara Iuda ?i toate ceta?ile lui, plugarii ?i cei ce umbla cu turmele.
\par 25 Ca voi adapa sufletul obosit ?i voi satura toata inima amarâta".
\par 26 Atunci m-am de?teptat ?i am privit, ?i somnul meu a fost lini?tit.
\par 27 "Iata vin zile, zice Domnul, când voi semana în casa lui Israel ?i în casa lui Iuda o samân?a de om ?i samân?a de vite;
\par 28 Precum am privegheat asupra lor ca sa-i smulg ?i sa-i zdrobesc, ca sa-i risipesc, sa-i vatam ?i sa-i pierd, a?a voi priveghea asupra lor, ca sa-i zidesc ?i sa-i sadesc, zice Domnul.
\par 29 În zilele acelea nu vor mai zice: "Parin?ii au mâncat agurida ?i copiilor li s-au strepezit din?ii".
\par 30 Ci fiecare va muri pentru faradelegea sa; cine va mânca agurida, aceluia i se vor strepezi din?ii.
\par 31 Iata vin zile, zice Domnul, când voi încheia cu casa lui Israel ?i cu casa lui Iuda legamânt nou.
\par 32 Însa nu ca legamântul pe care l-am încheiat cu parin?ii lor în ziua când i-am luat de mâna, ca sa-i scot din pamântul Egiptului. Acel legamânt ei l-au calcat, de?i Eu am ramas în legatura cu ei, zice Domnul.
\par 33 Dar iata legamântul pe care-l voi încheia cu casa lui Israel, dupa zilele acela, zice Domnul: Voi pune legea Mea înauntrul lor ?i pe inimile lor voi scrie ?i le voi fi Dumnezeu, iar ei Îmi vor fi popor.
\par 34 ?i nu se vor mai înva?a unul pe altul ?i frate pe frate, zicând: "Cunoa?te?i pe Domnul" ca to?i de la sine Ma vor cunoa?te, de la mic pâna la mare, zice Domnul, pentru ca Eu voi ierta faradelegile lor ?i pacatele lor nu le voi mai pomeni".
\par 35 A?a zice Domnul, Cel ce a dat soarele ca sa lumineze ziua ?i a pus legi lunii ?i stelelor, ca sa lumineze noaptea; Cel ce tulbura marea, de-i mugesc valurile, ?i numele Lui este Domnul Savaot:
\par 36 "Când aceste legi vor înceta sa mai aiba putere înaintea Mea, zice Domnul, atunci ?i semin?ia lui Israel va înceta pentru totdeauna sa mai fie popor înaintea Mea.
\par 37 A?a zice Domnul: Daca cerul poate fi masurat sus ?i temeliile pamântului cercetate jos, atunci ?i Eu voi lepada toata semin?ia lui Israel pentru toate câte au facut ei, zice Domnul.
\par 38 Iata vin zile - zice Domnul când cetatea se va zidi întru slava Domnului de la turnul Hananeel pâna la Poarta Col?ului,
\par 39 ?i funia de masurat pamântul va înainta de-a dreptul pâna la muntele Gareb ?i se va întoarce spre Goa.
\par 40 ?i toata valea trupurilor moarte ?i a cenu?ii ?i tot câmpul pâna la pârâul Chedron, pâna la col?ul porcilor cailor, spre rasarit, vor fi locuri sfin?ite pentru Domnul ?i nu vor mai fi niciodata nici pustiite, nici nimicite".

\chapter{32}

\par 1 Cuvântul care a fost de la Domnul catre Ieremia, în anul al zecelea al lui Sedechia, regele lui Iuda. Anul acesta era anul al optsprezecelea al lui Nabucodonosor.
\par 2 Atunci o?tirea regelui Babilonului înconjurase Ierusalimul ?i proorocul Ieremia era închis în curtea temni?ei, care se afla lânga casa regelui lui Iuda.
\par 3 Acolo îl închisese regele Sedechia. zicând: "De ce prooroce?ti ?i zici: A?a graie?te Domnul: Iata, voi da cetatea în mâinile regelui Babilonului, ?i acesta o va lua;
\par 4 Sedechia, regele lui Iuda, nu va scapa din mâinile Caldeilor, ci va fi dat fara îndoiala în mâinile regelui Babilonului ?i va grai cu el gura catre gura; ochii lui var vedea ochii aceluia,
\par 5 ?i acela va duce pe Sedechia la Babilon, unde va ?i sta el pâna-l voi cerceta, zice Domnul. Daca va ve?i lupta cu Caldeii, nu ve?i avea izbânda!"
\par 6 Atunci Ieremia a zis: "A?a a fost cuvântul Domnului catre mine:
\par 7 Iata vine Hanameel, fiut lui ?alum, unchiul tau, vine la tine ca sa-îi zica: "Cumpara pentru tine ?arina mea cea din Anatot, pentru ca dupa dreptul de înrudire se cuvine sa o cumperi tu".
\par 8 ?i a venit la mine în curtea garzii Hanameel, fiul unchiului meu, dupa cuvântul Domnului, ?i mi-a zis: "Cumpara ?arina mea cea din Anatot, din pamântul lui Veniamin, ca al tau este dreptul de mo?tenire ?i al tau este dreptul de rascumparare. Cumpar-o pentru tine!"
\par 9 Atunci am cunoscut ca acesta fusese cuvântul Domnului ?i am cumparat de la Hanameel, fiul unchiului meu, ?arina cea din Anatot ?i i-am cântarit ?apte sicli de argint ?i zece argin?i;
\par 10 Apoi am scris zapisul ?i l-am întarit cu pecetea, am chemat la aceasta martori ?i am cântarit argintul cu cântarul.
\par 11 Am luat atât zapisul de cumparare cel pecetluit dupa lege ?i rânduiala, cât ?i pe cel deschis;
\par 12 ?i am dat acest zapis de cumparare lui Baruh, fiul lui Neria, fiul lui Maasia, în fa?a lui Hanameel, fiul unchiului meu, ?i în fa?a martorilor care iscalisera acest zapis de cumparare, ?i în fa?a tuturor Iudeilor, care se aflau în curtea garzii,
\par 13 ?i în fa?a tuturor am poruncit lui Baruh:
\par 14 A?a zice Domnul Savaot, Dumnezeul lui Israel: "Ia zapisele acestea, - zapisul de cumparare care este pecetluit ?i zapisul acesta deschis - ?i le pune într-un vas de lut, ca sa stea acolo zile multe.
\par 15 Ca a?a zice Domnul Savant, Dumnezeul lui Israel: Casele, ?arinile ?i viile vor fi din nou cumparate în ?ara aceasta".
\par 16 Dând eu zapisul de cumparare lui Baruh, fiul lui Neria, m-am rugat Domnului ?i am zis:
\par 17 "O, Doamne Dumnezeule, Tu ai facut cerul ?i pamântul cu puterea Ta cea mare ?i cu bra? înalt ?i pentru Tine nimic nu este cu neputin?a!
\par 18 Tu ara?i mila la mii ?i pedepse?ti faradelegile parin?ilor în sânul copiilor lor dupa ei; Tu e?ti Dumnezeu cel mare ?i puternic, al Carui nume este Domnul Savaot;
\par 19 Cel mare în sfat ?i puternic în faptele Tale, ai Carui ochi sunt deschi?i asupra tuturor cailor fiilor oamenilor, ca sa dai fiecaruia dupa caile lui ?i dupa roadele faptelor lui;
\par 20 Tu ai savâr?it semne ?i minuni în ?ara Egiptului ?i savâr?e?ti ?i astazi în Israel ?i printre oameni,
\par 21 ?i ?i-ai facut nume, ca ?i astazi, ?i ai scos pe poporul Tau Israel din pamântul Egiptului prin semne ?i minuni, cu mâna tare ?i cu bra? înalt, în groaza mare,
\par 22 ?i le-ai dat lor ?ara aceasta, în care curge lapte ?i miere, pe care o fagaduise?i cu juramânt parin?ilor lor, ca le-o vei da;
\par 23 ?i ei au intrat ?i au pus stapânire pe ea, dar n-au ascultat glasul Tau ca sa se poarte dupa legea Ta ?i n-au facut ceea ce le-ai poruncit Tu sa faca. De aceea ai adus asupra lor toate aceste nenorociri.
\par 24 Iata valurile de pamânt se întind pâna la cetate ca sa fie luata! ?i cetatea prin sabie, foamete ?i ciuma se da în mâinile Caldeilor, care lupta împotriva ei; ceea ce ai zis Tu, aceea se ?i împline?te. ?i Tu vezi aceasta.
\par 25 Dar Tu, Doamne Dumnezeule, mi-ai zis: Cumpara-?i o ?arina cu argint ?i cheama martori, tocmai când cetatea se da în mâinile Caldeilor".
\par 26 Atunci a fost cuvântul Domnului catre Ieremia ?i a zis:
\par 27 "Iata, Eu sunt Domnul Dumnezeu a tot trupul! Este oare ceva cu neputin?a la Mine?"
\par 28 De aceea a?a zice Domnul: "Iata voi da cetatea aceasta în mâinile Caldeilor ?i în mâinile lui Nabucodonosor, regele Babilonului, ?i acesta o va lua.
\par 29 ?i vor intra Caldeii, care împresoara cetatea aceasta, vor da foc ceta?ii ?i o vor arde cu foc pe ea ?i casele pe ale caror acoperi?uri s-au adus tamâieri lui Baal ?i jertfe cu turnare în cinstea dumnezeilor straini, ca sa Ma mânie pe Mine.
\par 30 Ca fiii lui Israel ?i fiii lui Iuda au facut numai rau înaintea ochilor Mei din tinere?ile lor; fiii lui Israel M-au mâniat necontenit cu faptele mâinilor lor, zice Domnul.
\par 31 Cetatea aceasta, chiar din ziua zidirii sale ?i pâna astazi, pare ca a fost facuta pentru mânia Mea ?i pentru urgia Mea, ca s-o lepad de la fala Mea,
\par 32 Din pricina raului fiilor lui Israel ?i al fiilor lui Iuda, pe care l-au facut pentru mânierea Mea, ei ?i regii lor, capeteniile lor, preo?ii lor ?i proorocii lor, barba?ii lui Iuda ?i locuitorii Ierusalimului.
\par 33 ?i Mi-au întors spatele, iar nu fa?a ?i, când i-am înva?at, dis-de-diminea?a i-am înva?at ?i ei n-au voit sa primeasca înva?atura;
\par 34 În templul asupra caruia s-a chemat numele Meu au pus urâciunile lor, întinându-l.
\par 35 Au facut locuri înalte lui Baal în valea fiilor lui Hinom, ca sa treaca prin foc pe fiii lor ?i pe fiicele lor în cinstea lui Moloh, ceea ce Eu nu le-am poruncit, ?i nici prin minte nu Mi-a trecut ca ei vor face aceasta urâciune, ducând în pacat pe Iuda".
\par 36 ?i acum, a?a vorbe?te Domnul Dumnezeul lui Israel despre cetatea aceasta, de care zice?i: "Ea se va da în mâinile regelui Babilonului prin sabie, foamete ?i boala ciumei":
\par 37 "Iata îi voi aduna din toate ?arile, prin care i-am împra?tiat în mânia Mea ?i cu iu?imea Mea ?i în marea Mea întarâtare, ?i îi voi întoarce la locul acesta ?i le voi da via?a fara de primejdie.
\par 38 Ei vor fi poporul Meu ?i Eu le voi fi Dumnezeu.
\par 39 ?i le voi da o inima ?i o cale, ca sa se teama de Mine în toate zilele vie?ii, spre binele dar ?i spre binele copiilor lor dupa ei.
\par 40 Voi încheia cu ei legamânt ve?nic, dupa care Eu nu Ma voi mai întoarce de la ei, ci le voi face bine ?i voi pune frica Mea în inima lor, ca ei sa nu se mai abata de la Mine.
\par 41 Ma voi bucura sa le fac bine ?i-i voi sadi tare în pamântul acesta din toata inima Mea ?i din tot sufletul Meu".
\par 42 Ca a?a zice Domnul: "Dupa cum am adus asupra poporului acestuia acest mare rau, a?a voi aduce asupra lor tot binele pe care l-am rostit pentru ei.
\par 43 ?i vor cumpara ?arina în ?ara aceasta, de care zice?i: "Aceasta este pustietate fara oameni ?i fara vite ?i este data în mâna Caldeilor".
\par 44 Vor cumpara ?arini pe argint ?i le vor scrie în zapis ?i-l vor pecetlui ?i vor chema martori; cumpara-vor ?arine în pamântul lui Veniamin ?i în împrejurimile Ierusalimului, prin ceta?ile lui Iuda ?i prin ceta?ile din mun?i, prin ceta?ile din câmp, prin ceta?ile de la miazazi, ca voi întoarce pe robii lor", zice Domnul.

\chapter{33}

\par 1 Fost-a cuvântul Domnului catre Ieremia a doua oara, când era el tot închis în curtea garzii, ?i a zis:
\par 2 "A?a graie?te Domnul, Cel ce a facut pamântul. Domnul Cel ce l-a zidit ?i l-a întarit, al Carui nume este Domnul:
\par 3 Striga catre Mine, ca Eu î?i voi raspunde ?i i?i voi arata lucruri mari ?i nepatrunse pe care tu nu le ?tii.
\par 4 Ca a?a zice Domnul Dumnezeul lui Israel despre casele acestei ceta?i ?i despre casele regilor lui Iuda, care se strica pentru a fi facute întarituri împotriva împresurarii ?i sabiei
\par 5 Caldeilor veni?i sa lupte ?i sa le umple cu trupurile oamenilor, pe care-i lovesc cu mânia Mea ?i cu urgia Mea ?i pentru ale caror faradelegi Mi-am ascuns fa?a de la cetatea aceasta.
\par 6 Le voi lega ranile ca sa-i vindec ?i le voi descoperi bel?ug de pace ?i de adevar;
\par 7 Voi întoarce aici pe robii lui Iuda ?i pe robii lui Israel ?i-i voi a?eza ca la început;
\par 8 Îi voi cura?i de necredin?a lor, cu care au gre?it ei înaintea Mea ?i le voi ierta toate faradelegile lor, cu care au pacatuit ei înaintea Mea ?i au cazut de la Mine.
\par 9 Ierusalimul va fi pentru Mine nume de bucurie, lauda ?i cinste în fa?a tuturor popoarelor pamântului, care vor auzi de toate bunata?ile ce i le voi face ?i se vor mira, se vor cutremura de toate binefacerile ?i de toata starea cea buna cu care-l voi învrednici".
\par 10 A?a zice Domnul: "În locul acesta de care voi zice?i ca este pustiu ?i lipsit de oameni ?i de vite, precum ?i în ceta?ile lui Iuda, ?i pe uli?ele Ierusalimului, care sunt pustii, fara oameni, fara locuitori ?i fara vite,
\par 11 Iar se va auzi glas de bucurie, glas de veselie, glas de mire ?i glas de mireasa, glasul celor ce zic: "Slavi?i pe Domnul Savaot, ca bun este Domnul, ca în veac este mila Lui" ?i glasul celor ce aduc jertfa de mul?umire în templul Domnului; caci voi întoarce robii ?arii acesteia la starea cea de altadata", zice Domnul.
\par 12 A?a zice Domnul Savaot: "În ?ara aceasta, care este pustie, fara locuitori ?i fara vite, ?i în toate ceta?ile ei iara?i vor fi loca?uri de pastori, care-?i vor odihni turmele lor.
\par 13 În ceta?ile cele de munte, în ceta?ile cele de la ?es ?i în ceta?ile cele de la miazazi, în pamântul lui Veniamin, în împrejurimile Ierusalimului ?i în ceta?ile lui Iuda, iara?i se vor perinda turme pe sub mina celui ce numara, zice Domnul.
\par 14 Iata vin zilele când voi împlini acel cuvânt bun pe care l-am rostit Eu pentru casa lui Israel ?i pentru casa lui Iuda, zice Domnul.
\par 15 În zilele acelea ?i în vremea aceea voi ridica lui David Odrasla dreapta ?i aceea va face judecata ?i dreptate pe pamânt.
\par 16 În zilele acelea Iuda va fi izbavit ?i Ierusalimul va trai fara primejdie ?i Odraslei aceleia I se va pune numele: "Domnul - dreptatea noastra"
\par 17 Ca a?a zice Domnul: "Nu va lipsi lui David barbat care sa ?ada pe scaunul casei lui Israel.
\par 18 ?i preo?ii-levi?i nu vor avea lipsa de barbat care sa stea înaintea fe?ei Mele ?i sa aduca în toate zilele arderi de tot, sa aprinda tamâie ?i sa savâr?easca jertfe".
\par 19 ?i a mai fost cuvântul Domnului catre Ieremia, zicând:
\par 20 "A?a zice Domnul: De po?i strica legamântul Meu cel pentru ziua ?i legamântul Meu cel pentru noapte ?i sa faci ea ziua ?i noaptea sa nu mai vina ta vremea lor, atunci poate se va strica ?i legamântul Meu cel încheiat ou robul Meu David,
\par 21 Ca sa nu mai aiba el fiu care sa domneasca pe scaunul lui; de asemenea ?i cel încheiat cu levi?ii-preo?i, slujitorii Mei.
\par 22 Precum e nenumarata o?tirea cereasca ?i nenumarat nisipul marii, a?a voi înmul?i neamul lui David, robul Meu, ?i al levi?ilor celor ce-Mi slujesc Mie".
\par 23 Fost-a iara?i cuvântul Domnului catre Ieremia, zicând:
\par 24 Nu vezi tu oare ca poporul acesta zice: "Domnul a lepadat cele doua semin?ii pe care le alesese El?" ?i ei dispre?uiesc poporul Meu, ca ?i cum acesta n-ar fi popor în ochii lor.
\par 25 A?a zice Domnul: Daca legamântul Meu cel pentru ziua ?i pentru noapte ?i rânduiala cerului ?i a pamântului nu le-am întarit Eu,
\par 26 Atunci ?i neamul lui Iacov ?i al lui David, robul Meu, îl voi lepada ?i nu voi mai lua stapânitori din neamul lui pentru semin?ia lui Avraam ?i a lui Isaac ?i a lui Iacov, caci voi aduce înapoi pe prin?ii lor de razboi ?i-i voi milui".

\chapter{34}

\par 1 Cuvântul care a fost de la Domnul catre Ieremia, când Nabucodonosor, regele Babilonului, ?i toata o?tirea lui ?i toate regatele pamântului, supuse sub mâna lui, ?i toate popoarele au venit cu razboi împotriva Ierusalimului ?i împotriva tuturor ceta?ilor lui luda, zicând:
\par 2 "A?a zice Domnul Dumnezeul lui Israel: Mergi ?i graie?te lui Sedechia, regele lui Iuda, ?i-i spune: A?a zice Domnul: lata voi da cetatea aceasta în mâinile regelui Babilonului, ?i acesta o va arde cu foc.
\par 3 Nici tu nu vei scapa din mina lui, ci vei fi cu adevarat luat ?i dat în mâinile lui; ochii tai vor vedea ochii regelui Babilonului ?i buzele lui vor grai buzelor tale ?i vei merge la Babilon.
\par 4 Dar asculta cuvântul Domnului, Sedechia, rege al lui Iuda! A?a zice Domnul de tine: Nu vei muri de sabie, ci vei muri în pace.
\par 5 ?i precum s-a aprins tamâie pentru parin?ii tai ?i pentru ceilal?i regi, care au fost înainte de tine, la înmormântarea lor, a?a se va aprinde ?i pentru tine ?i te vor plânge, zicând: "Vai, doamne!" Ca Eu am rostit cuvântul acesta", zice Domnul.
\par 6 ?i a spus Ieremia proorocul toate cuvintele acestea lui Sedechia, regele lui Iuda, în Ierusalim.
\par 7 În vremea aceea o?tirea regelui Babilonului se razboia împotriva Ierusalimului ?i împotriva tuturor ceta?ilor lui Iuda, care mai ramasesera, ?i anume împotriva Lachi?ului ?i Azecai, deoarece din ceta?ile lui Iuda numai acestea mai ramasesera ceta?i întarite.
\par 8 Cuvântul care a fost de la Domnul catre Ieremia, dupa ce regele Sedechia a încheiat legamânt cu tot poporul ce era în Ierusalim,
\par 9 Ca sa dea slobozenie, a?a încât fiecare sa dea drumul robului sau ?i roabei sale, evreu ?i evreica, ?i ca nimeni sa nu mai ?ina în robie evreu, frate al sau.
\par 10 ?i s-au supus to?i cei mari ?i tot poporul, care au încheiat legamântul, ca sa dea fiecare drumul robului sau ?i roabei sale, ca sa nu-i mai ?ina pe: viitor în robie, ?i au ascultat ?i s-au supus.
\par 11 Dar dupa aceea, razgândindu-se, au început sa ia înapoi pe robii ?i pe roabele carora le dadusera drumul ?i-i silira sa le fie robi ?i roabe.
\par 12 Dar a fost cuvântul Domnului catre Ieremia, caruia i s-a zis din partea Domnului:
\par 13 "A?a zice Domnul Dumnezeul lui Israel: Eu am încheiat un legamânt cu parin?ii vo?tri, când i-am scos din pamântul Egiptului ?i din casa robiei, ?i am zis:
\par 14 La sfâr?itul anului al ?aptelea sa dea drumul fiecare din voi fratelui sau , evreu, care ?i s-a vândut ?ie; sa-?i lucreze el ?ase ani, iar dupa aceea sa-i dai drumul sa fie slobod. Dar parin?ii vo?tri nu m-au ascultat ?i nu ?i-au plecat urechea lor.
\par 15 Acum voi v-a?i întors ?i a?i facut cum e drept în ochii Mei, dând fiecare slobozenie aproapelui sau, ?i a?i încheiat înaintea Mea legamânt în templul asupra caruia s-a chemat numele Meu.
\par 16 Dar apoi v-a?i razgândit ?i a?i necinstit numele Meu ?i a?i întors fiecare pe robul vostru ?i pe roaba voastra carora le dadusera?i drumul sa se duca unde le place, ?i-i sili?i sa va fie robi ?i roabe.
\par 17 De aceea, a?a zice Domnul: Voi nu v-a?i supus Mie, ca sa da?i fiecare drumul aproapelui vostru; pentru aceea iata, Eu, zice Domnul, voi da libertate sabiei asupra voastra, ciumei ?i foametei ?i va voi face "pricina" de groaza pentru toate regatele pamântului.
\par 18 Pe cei ce au calcat legamântul Meu ?i pe cei ce nu s-au ?inut de cuvintele legamântului pe care l-am încheiat înaintea fe?ei Mele, despicând în doua vi?elul ?i trecând printre cele doua buca?i ale lui,
\par 19 Pe cei mari ai lui Iuda ?i pe cei mari ai Ierusalimului, pe eunuci, pe preo?i ?i pe tot poporul ?arii, care au trecut printre buca?ile vi?elului despicat,
\par 20 Îi voi da în mâinile vrajma?ilor ?i în mâinile celor ce vor sa le ia via?a, iar trupurile lor vor fi hrana pasarilor cerului ?i fiarelor pamântului.
\par 21 Pe Sedechia, regele lui Iuda, ?i pe cei mari ai lui îi voi da în mâinile vrajma?ilor lor ?i în mâinile celor ce vor sa le ia via?a, în mâinile o?tirii regelui Babilonului, care s-a retras de la voi.
\par 22 Iata Eu voi da porunca, zice Domnul, ?i-i voi întoarce la cetatea aceasta ?i vor navali asupra ei ?i o vor lua ?i o vor arde cu foc, ?i ceta?ile lui Iuda le vor face pustiu nelocuit".

\chapter{35}

\par 1 Cuvântul care a fost de la Domnul catre Ieremia, în zilele lui Ioiachim, fiul lui Iosia, regele lui Iuda:
\par 2 "Mergi în casa Recabi?ilor ?i vorbe?te cu ei ?i-i adu în templul Domnului, într-o camara, ?i da-le sa bea vin".
\par 3 Atunci am luat eu pe Iaazania, fiul lui Ieremia, fiul lui Haba?inia, pe fra?ii lui, pe to?i fiii lui ?i toata casa Recabi?ilor
\par 4 ?i i-am adus în templul Domnului, în camara fiilor lui Hanan, fiul lui Igdalia, omul lui Dumnezeu, care e lânga camara dregatorilor, deasupra camerei lui Maaseia, fiul lui ?alum, pazitorul pragului.
\par 5 Apoi am pus înaintea fiilor casei Recabi?ilor cupe pline cu vin ?i pahare ?i le-am zis: "Be?i vin!"
\par 6 Dar ei au zis: "Noi nu bem vin, pentru ca Ionadab, fiul lui Recab, tatal nostru, ne-a dat porunca, zicând: Sa nu beli vin nici voi, nici fiii vo?tri în veac!
\par 7 Nici case sa nu zidi?i, nici semin?e sa nu semana?i, nici vii sa nu sadi?i, nici sa ave?i, ci sa trai?i în corturi în toate zilele vie?ii voastre, ca sa trai?i vreme îndelungata pe pamântul în care sunte?i calatori.
\par 8 ?i noi am ascultat glasul lui Ionadab, fiul lui Recab, tatal nostru, întru toate câte ne-a poruncit el, ca sa nu bem vin în toate zilele noastre nici noi, nici femeile noastre, nici fiii no?tri, nici fiicele noastre,
\par 9 ?i sa nu zidim case ca locuin?e pentru noi ?i sa nu avem nici vii; nici ?arini, nici semanaturi,
\par 10 Ci sa traim în corturi. Întru toate ascultam ?i facem toate câte ne-a poruncit Ionadab, stramo?ul nostru.
\par 11 Când însa a venit Nabucodonosor, regele Babilonului, împotriva acestei ?ari, noi am zis: Hai sa intram în Ierusalim dinaintea o?tirilor Caldeilor ?i a o?tirilor Sirienilor! ?i iata acum traim în Ierusalim".
\par 12 Atunci a fost cuvântul Domnului catre Ieremia ?i a zis:
\par 13 "A?a zice Domnul Savaot, Dumnezeul lui Israel: Mergi ?i spune barba?ilor lui Iuda ?i locuitorilor Ierusalimului: Se poate oare sa nu lua?i voi înva?atura din aceasta ?i sa nu asculta?i cuvântul Meu?" - zice Domnul.
\par 14 Cuvintele lui Ionadab, fiul lui Recab, pe care le-a spus fiilor sai de a nu bea vin, se împlinesc, ?i ei nu beau pâna în ziua de astazi, pentru ca se supun celor rânduite de tatal lor; iar Eu necontenit v-am vorbit, v-am vorbit dis-de-diminea?a ?i voi nu M-a?i ascultat.
\par 15 Trimis-am la voi pe to?i proorocii, robii Mei, i-am trimis dis-de-diminea?a ?i am zis: Sa se întoarca fiecare de la calea lui cea rea ?i sa va îndrepta?i purtarile, sa nu merge?i dupa al?i dumnezei, ca sa le sluji?i, ?i ve?i trai în pamântul acesta pe care l-am dat voua ?i parin?ilor vo?tri; dar voi nu v-a?i plecat urechea ?i nu M-a?i ascultat.
\par 16 Deoarece fiii lui Ionadab, fiul lui Recab, împlinesc porunca pe care le-a dat-o tatal lor, iar poporul Meu nu Ma asculta,
\par 17 De aceea a?a zice Domnul Dumnezeu Savaot, Dumnezeul lui Israel: Iata Eu voi aduce asupra lui Iuda ?i a tuturor locuitorilor Ierusalimului tot raul acela pe care l-am rostit asupra lor, pentru ca Eu le-am vorbit ?i ei n-au ascultat, i-am chemat ?i ei n-au raspuns".
\par 18 Iar casei Recabi?ilor, Ieremia i-a grait: "A?a zice Domnul Savaot, Dumnezeul lui Israel: Pentru ca voi a?i ascultat porunca lui Ionadab, tatal vostru, ?i pazi?i toate poruncile lui ?i în toate face?i cum v-a poruncit el,
\par 19 De aceea, a?a zice Domnul Savaot, Dumnezeul lui Israel: Nu va lipsi lui Ionadab, fiul lui Recab, barbatul care sa stea înaintea felei Mele în toate zilele".

\chapter{36}

\par 1 În anul al patrulea al lui Ioiachim, fiul lui Iosia, regele lui Iuda, a fost de la Domnul catre Ieremia cuvântul acesta:
\par 2 "Ia-?i un sul de hârtie ?i scrie pe el toate cuvintele pe care ?i le-am grait Eu despre Israel, despre Iuda ?i despre toate popoarele, din ziua de când am început a-?i grai, din zilele lui Iosia ?i pâna în ziua de astazi;
\par 3 Ca poate va auzi casa lui Iuda tot raul ce Mi-am pus în gând sa i-l fac, ca sa se întoarca fiecare de la calea sa cea rea, pentru ca Eu sa le iert nedrepta?ile lor ?i pacatele lor".
\par 4 Atunci a chemat Ieremia pe Baruh, fiul lui Neria, ?i dupa spusa lui Ieremia a scris Baruh pe hârtia sulului toate cuvintele Domnului pe care le graise El aceluia.
\par 5 ?i a poruncit Ieremia lui Baruh ?i a zis: "Eu sunt închis ?i nu ma pot duce în templul Domnului.
\par 6 Deci, du-te tu ?i cuvintele Domnului din gura mea scrise de tine în sul, cite?te-le în templul Domnului, în ziua postului, în auzul poporului; cite?te-le de asemenea ?i în auzul tuturor acelora din Iuda, care sunt veni?i de prin toate ceta?ile;
\par 7 Poate vor înal?a ruga smerita înaintea fe?ei Domnului ?i se va întoarce fiecare de la calea sa cea rea, caci mare este mânia ?i supararea pe care a aratat-o Domnul asupra poporului acestuia".
\par 8 ?i a facut Baruh, fiul lui Neria, tot ce i-a poruncit Ieremia, citind în templul Domnului cuvintele Domnului cele scrise în sul.
\par 9 În anul al cincilea al lui Ioiachim, fiul lui Iosia, regele lui Iuda, în luna a opta, a vestit post înaintea fe?ei Domnului pentru tot poporul din Ierusalim ?i pentru tot poporul care venise la Ierusalim de prin ora?ele lui Iuda.
\par 10 ?i cuvintele lui Ieremia, cele scrise în sul, le-a citit Baruh în templul Domnului, în camera lui Ghemaria, fiul lui ?afan, scriitorul, în curtea de sus, la intrarea de la poarta cea noua a templului Domnului, în auzul poporului.
\par 11 Iar Miheia, fiul lui Ghemaria, fiul lui ?afan, a auzit toate cuvintele Domnului cele scrise în sul.
\par 12 ?i s-a coborât în casa regelui, în camera secretarului regelui ?i iata acolo stateau to?i dregatorii: Eli?ama, secretarul regelui, Delaia, fiul lui ?emaia, Elnatan, fiul lui Acbor, Ghemaria, fiul lui ?afan, Sedechia, fiul lui Hanania ?i to?i ceilal?i dregatori.
\par 13 ?i le-a povestit acestora Miheia toate cuvintele ce le auzise, când a citit Baruh cartea în auzul poporului.
\par 14 Atunci dregatorii au trimis la Baruh pe Iehudi, fiul lui Netania, fiul lui ?elemia, fiul lui Cu?i, ca sa-i spuna: "Cartea pe care tu ai citit-o în auzul poporului, sa o iei în mâna ta ?i sa vii".
\par 15 Atunci Baruh, fiul lui Neria, a luat cartea în mâna sa ?i a venit la ei. Iar ei i-au zis: "?ezi ?i ne cite?te în auz!" ?i a citit Baruh în auzul lor.
\par 16 Iar daca au ascultat ei toate cuvintele, s-au uitat cu groaza unii la al?ii ?i au zis catre Baruh: "Vom spune numaidecât regelui toate cuvintele acestea!"
\par 17 ?i apoi au întrebat pe Baruh: "Spune-ne insa cum ai scris tu din gura lui toate cuvintele acestea?"
\par 18 ?i Baruh le-a spus: "El mia rostit cu gura sa toate cuvintele acestea, iar eu le-am scris cu cerneala în sulul acesta".
\par 19 Atunci au zis dregatorii catre Baruh: "Du-te ?i te ascunde ?i tu ?i Ieremia, ca nimeni sa nu ?tie unde sunte?i!"
\par 20 Apoi ei s-au dus la rege în palat, iar cartea au lasat-o în odaia lui Eli?ama, secretarul regelui, ?i au spus în auzul regelui toate cuvintele acestea.
\par 21 ?i a trimis regele pe Iehudi sa aduca îndata cartea, ?i acesta a luat-o din camera lui Eli?ama, secretarul regelui; ?i a citit-o Iehudi în auzul regelui ?i în auzul tuturor dregatorilor care stateau lânga rege.
\par 22 În vremea aceea, în luna a noua, regele ?edea în palatul de iarna ?i înaintea lui ardea o tava cu jeratic.
\par 23 Dupa ce a citit Iehudi trei sau patru coloane, regele a taiat cu cu?ita?ul secretarului cartea ?i a aruncat-o în focul de pe tava, nimicind-o toata.
\par 24 ?i nu s-au temut, nici ?i-au sfâ?iat hainele lor, nici regele, nici to?i slujitorii lui, care au auzit toate cuvintele acestea.
\par 25 De?i Elnatan, Delaia ?i Ghemaria au rugat pe rege sa nu arda cartea, el însa nu i-a ascultat.
\par 26 ?i a poruncit regele lui Ierahmeel, fiul regelui, ?i lui Seraia, fiul lui Azirel, ?i lui ?elemia, fiul lui Abdeel, sa prinda pe scriitorul Baruh ?i pe Ieremia proorocul. Dar Domnul i-a ascuns.
\par 27 Atunci a fost cuvântul Domnului catre Ieremia, dupa ce regele a ars cartea ?i cuvintele pe care Baruh le scrisese din gura lui Ieremia, ?i i-a zis:
\par 28 "Ia-?i alt sul de hârtie ?i scrie în el toate cuvintele de mai înainte, care au fost în celalalt sul, pe care l-a ars Ioiachim, regele lui Iuda, iar regelui lui Iuda, Ioiachim, sa-i spui:
\par 29 A?a zice Domnul: Tu ai ars cartea aceasta, zicând: Pentru ce ai scris în ea: Va veni îndata regele Babilonului ?i va pustii ?ara aceasta ?i va pierde oamenii ?i dobitoacele din ea?
\par 30 De aceea, a?a zice Domnul despre Ioiachim, regele lui Iuda: Nu va mai fi din el urma? care sa ?ada pe scaunul lui David; trupul lui va fi aruncat în ar?i?a zilei ?i în frigul nop?ii
\par 31 ?i-l voi pedepsi pe el ?i neamul lui ?i slugile lui pentru nedreptatea lor; voi aduce asupra lor ?i asupra locuitorilor Ierusalimului ?i asupra oamenilor lui Iuda tot raul pe care l-am rostit asupra lor ?i ei n-au ascultat".
\par 32 ?i a luat Ieremia alt sul de hârtie ?i l-a dat lui Baruh scriitorul, fiul lui Neria, ?i acesta a scris în el, din gura lui Ieremia, toate cuvintele din sulul ce-l aruncase în foc Ioiachim, regele lui Iuda, ?i a mai adaugat la ele multe cuvinte asemenea acelora.

\chapter{37}

\par 1 În locul lui Iehonia, fiul lui Ioiachim, a fost facut rege Sedechia, fiul lui Iosia, pe care  Nabucodonosor, regele Babilonului, l-a pus rege în ?ara lui Iuda.
\par 2 Dar nici el, nici slujitorii lui ?i nici poporul ?arii n-au ascultat cuvintele Domnului, cele graite prin proorocul Ieremia.
\par 3 Regele Sedechia a trimis pe Iucal, fiul lui ?elemia, ?i pe Sofonie preotul, fiul lui Maaseia, la Ieremia proorocul, ca sa-i zica: "Roaga-te pentru noi Domnului Dumnezeului nostru!"
\par 4 Caci pe atunci Ieremia intra ?i umbla înca slobod prin popor, pentru ca nu-l aruncasera înca în închisoare.
\par 5 O?tirea lui Faraon a ie?it din Egipt, iar Caldeii, care împresurasera Ierusalimul, auzind vestea aceasta, s-au retras de la Ierusalim.
\par 6 Atunci a fost cuvântul Domnului catre Ieremia proorocul ?i a zis:
\par 7 "A?a graie?te Domnul, Dumnezeul lui Israel: Regelui lui Iuda, care v-a trimis la Mine sa Ma întreba?i, a?a sa-i spune?i: Iata, o?tirea lui Faraon care v-a veni în ajutor se va întoarce în ?ara sa, în Egipt,
\par 8 Iar Caldeii vor veni din nou ?i vor lupta împotriva ceta?ii acesteia, o vor lua ?i o vor arde cu foc.
\par 9 A?a zice Domnul: Nu va în?ela?i pe voi în?iva, zicând: "Fara îndoiala se vor duce de la noi Caldeii" ca nu se vor duce.
\par 10 Chiar daca a?i, sfarâma cu totul o?tirea Caldeilor, care lupta împotriva voastra, ?i ar ramâne la ei numai rani?i, apoi ?i aceia s-ar ridica fiecare din cortul sau ?i ar arde cetatea aceasta cu foc".
\par 11 În vremea când o?tirea Caldeilor s-a retras de la Ierusalim din pricina o?tirii egiptene,
\par 12 Ieremia a plecat din Ierusalim, ca sa se duca în ?ara lui. Veniamin, sa descurce ni?te treburi de mo?tenire cu cei de acolo.
\par 13 Dar când se afla el la poarta lui Veniamin, capetenia garzii, care era acolo ?i care se numea Ireia, fiul lui ?elemia, fiul lui Hanania, a prins pe Ieremia proorocul ?i i-a zis: "Tu vrei sa fugi la Caldei!"
\par 14 "Aceasta este minciuna; eu nu vreau sa fug la Caldei", a zis Ieremia. Dar fara sa-l asculte, Ireia l-a arestat pe Ieremia ?i l-a dus la dregatori.
\par 15 ?i s-au mâniat dregatorii pe Ieremia, l-au batut ?i l-au închis în temni?a, în casa lui Ionatan scriitorul, pentru ca o facusera temni?a.
\par 16 ?i dupa ce a intrat Ieremia în temni?a, în beci, ?i a stat acolo Ieremia zile multe,
\par 17 Regele Sedechia a trimis ?i i-a adus. ?i l-a întrebat regele în casa sa în taina ?i i-a zis: "N-ai oare cuvânt de la Domnul?" Iar Ieremia a zis: "Ba am!" ?i a adaugat: "Tu vei fi dat în mâinile regelui Babilonului!"
\par 18 ?i a mai zis Ieremia catre regele Sedechia: "Cu ce am gre?it eu înaintea ta, înaintea slujitorilor tai ?i înaintea poporului acestuia, de m-a?i bagat în temni?a?
\par 19 ?i unde sunt proorocii vo?tri care v-au proorocit ?i au zis: "Regele Babilonului nu va veni împotriva voastra ?i împotriva pamântului acestuia?"
\par 20 ?i acum asculta, domnul meu rege, sa aiba trecere cererea mea înaintea ta ?i sa nu ma mai întorc în casa lui Ionatan scriitorul, ca sa nu mor acolo".
\par 21 ?i a dat regele Sedechia porunca sa închida pe Ieremia în curtea temni?ei ?i i s-a dat câte o bucata de pâine pe zi din uli?a pitarilor, pâna s-a ispravit pâinea în cetate. ?i a?a a ramas Ieremia în curtea temni?ei.

\chapter{38}

\par 1 ?i ?efatia, fiul lui Matan, ?i Ghedalia, fiul lui Pashor, Iucal, fiul lui ?elemia ?i Pa?hurr, fiul lui Malchia, au auzit cuvintele pe care le-a rostit Ieremia catre tot poporul, zicând:
\par 2 "A?a zice Domnul: Cine va ramâne în aceasta cetate va muri de sabie, de foame ?i de boala ciumei; iar cel ce se va duce la Caldei va trai, va avea ca prada via?a lui ?i va ramâne viu. "
\par 3 A?a zice Domnul: Cetatea aceasta va fi data fara îndoiala în mâinile o?tirilor regelui Babilonului, ?i o vor lua".
\par 4 Atunci capeteniile au zis catre rege: "Omul acesta sa fie dat mor?ii, pentru ca el descurajeaza pe luptatorii care au ramas în cetatea aceasta ?i pe tot poporul, spunându-le asemenea cuvinte. Caci omul acesta nu dore?te propa?irea poporului sau, ci nenorocirea".
\par 5 Iar regele Sedechia a zis: "Iata e în mâinile voastre, caci regele nu poate sa faca nimic împotriva voastra".
\par 6 Atunci au luat pe Ieremia ?i l-au aruncat în groapa lui Malchia, fiul regelui, care era în curtea temni?ii, coborându-l în ea cu funii. în groapa aceea nu era apa, ci numai noroi ?i s-a afundat Ieremia în noroi.
\par 7 Ebed-Melec, Etiopianul, unul din eunucii care se aflau în casa regelui, auzind ca Ieremia a fost aruncat în groapa - regele atunci ?edea la poarta lui Veniamin -
\par 8 A ie?it din palatul regal ?i a zis catre rege:
\par 9 "Domnul meu rege, rau au facut oamenii ace?tia care s-au purtat a?a cu Ieremia proorocul, pe care l-au aruncat în groapa. El va muri acolo de foame, pentru ca nu mai este pâine în cetate".
\par 10 Atunci regele a dat porunca lui Ebed-Melec, Etiopianul, zicând: "Ia de aici treizeci de oameni ?i scoate pe Ieremia proorocul din groapa, pâna nu moare".
\par 11 ?i a luat Ebed-Melec cu sine oamenii ?i a intrat în palatul regal, în ve?mântarie, ?i a luat de acolo buca?i de haine vechi ?i rupturi ?i le-a dat drumul cu frânghia în groapa la Ieremia.
\par 12 ?i a zis Ebed-Melec, Etiopianul, catre Ieremia: "Pune aceste cârpe ?i rupturi vechi, ce ?i s-au aruncat, la subsuorile tale, sub frânghie"; ?i a facut Ieremia a?a.
\par 13 ?i au tras pe Ieremia din groapa; ?i a ramas Ieremia în curtea temni?ii.
\par 14 Atunci regele Sedechia a trimis ?i a chemat pe Ieremia proorocul la sine, la u?a a treia a templului Domnului; ?i a zis regele catre Ieremia: "Am sa te întreb ceva, dar sa nu ascunzi nimic de mine".
\par 15 Iar Ieremia a zis catre Sedechia: "Daca-?i voi descoperi, nu ma vei da oare mor?ii? ?i de-?i voi da sfat, îl vei asculta tu oare?"
\par 16 Zis-a regele Sedechia catre Ieremia: "Viu este Domnul, Care ne-a dat aceasta via?a, nu te voi da la moarte, nici în mâinile acestor oameni, care vor sa-?i ia via?a, nu te voi da".
\par 17 Atunci a zis Ieremia catre Sedechia: "A?a zice Domnul Dumnezeul Savaot, Dumnezeul lui Israel: Daca tu vei ie?i la capeteniile regelui Babilonului, vei scapa cu via?a ?i cetatea aceasta nu va fi arsa cu foc ?i vei trai ?i tu ?i casa ta;
\par 18 Iar de nu vei ie?i la capeteniile regelui Babilonului, cetatea aceasta va fi data în mâinile Caldeilor, care o vor arde cu foc, ?i nici tu nu vei scapa din mâinile lor".
\par 19 Iar regele Sedechia a zis catre Ieremia: "Ma tem de Iudeii care au trecut la Caldei, ca nu cumva sa nu fiu dat de Caldei pe mâna lor ?i ca nu cumva aceia sa-?i bata, joc de mine".
\par 20 Zis-a Ieremia: "Nu te vor da! Asculta glasul Domnului în cele ce-?i graiesc eu ?i bine-?i va fi ?i sufletul tau va fi viu.
\par 21 Iar daca tu nu vei vrea sa ie?i, atunci iata cuvântul pe care mi l-a descoperit Domnul:
\par 22 Toate femeile care au ramas în casa regelui lui Iuda var fi duse la capeteniile regelui Babilonului ?i acelea vor zice: Te-au amagit ?i te-au în?elat bunii tai prieteni; piciorul tau s-a afundat în noroi ?i ace?tia au fugit de tine;
\par 23 ?i toate femeile tale ?i copiii tai vor fi da?i Caldeilor ?i nici tu nu vei scapa din mâinile lor, caci vei fi prins de mâna regelui Babilonului ?i aceasta cetate va fi arsa".
\par 24 Zis-e Sedechia catre Ieremia: "Nimeni nu trebuie sa ?tie cuvintele acestea ?i atunci tu nu vei muri.
\par 25 Iar de vor auzi capeteniile ca eu am grait cu tine ?i de vor veni la tine ?i-?i vor zice: "Spune-ne ce ai vorbit cu regele? Nu ascunde de noi ?i nu te vom da la moarte! ?i ce ?i-e vorbit regele?"
\par 26 Atunci sa le spui: Am prezentat înaintea regelui cererea mea, ca sa nu ma trimita înapoi în casa lui Ionatan ?i sa mor acolo".
\par 27 Atunci au venit to?i dregatorii la Ieremia ?i l-au întrebat, iar el le-a raspuns întocmai cum îi poruncise regele, ?i aceia, tacând, l-au lasat, pentru ca n-au aflat nimic.
\par 28 ?i a ramas Ieremia în curtea temni?ii pâna în ziua în care a fost luat Ierusalimul, caci Ierusalimul a fost luat.

\chapter{39}

\par 1 În anul al noualea al lui Sedechia, regele lui Iuda, în luna a zecea, Nabucodonosor, regele Babilonului, a venit cu toata o?tirea sa la Ierusalim ?i l-a împresurat.
\par 2 Iar în anul al unsprezecelea al lui Sedechia, în luna a patra, în ziua a noua a lunii acesteia, cetatea a fost luata.
\par 3 ?i au intrat în ea ?i s-au a?ezat la por?ile de la mijloc toate capeteniile regelui Babilonului: Nergal-?are?er, Samgar-Nebu, Sarsehim, capetenia eunucilor, Nergal-Sare?er, capetenia vrajitorilor, ?i toate celelalte capetenii ale regelui Babilonului.
\par 4 Când Sedechia, regele lui Iuda, ?i to?i oamenii de oaste i-au vazut, au fugit ?i au ie?it noaptea din cetate prin gradina regelui, pe poarta dintre cele doua ziduri, ?i au apucat pe calea ?esului.
\par 5 Dar o?tirea Caldeilor a alergat dupa ei ?i a ajuns pe Sedechia în ?esul Ierihonului, l-au prins ?i l-au dus la Nabucodonosor, regele Babilonului, la Ribla, în ?ara Hamat, unde acesta a rostit judecata asupra lui.
\par 6 Atunci regele Babilonului a junghiat pe fiii lui Sedechia în Ribla, înaintea ochilor acestuia, ?i pe to?i dregatorii lui Iuda i-a junghiat regele Babilonului.
\par 7 Iar lui Sedechia i-a scos ochii ?i l-a pus în catu?e, ca sa-l duca la Babilon.
\par 8 Casa regelui ?i casele poporului le-au ars Caldeii cu foc, iar zidurile Ierusalimului le-au darâmat.
\par 9 Restul poporului, care ramasese în cetate, pe fugarii, care trecusera la el ?i pe celalalt popor ce mai ramasese, Nebuzaradan, capetenia garzii regelui, i-a dus robi în Babilon.
\par 10 Pe cei saraci din popor, care nu aveau nimic, Nebuzaradan, capetenia garzii, i-a lasat în pamântul lui Iuda ?i tot atunci le-a dat viile ?i ?arinile.
\par 11 Iar cu privire la Ieremia, Nabucodonosor, regele Babilonului, a dat lui Nebuzaradan, capetenia garzii, porunca aceasta:
\par 12 "Ia-l ?i sa ai purtare de grija pentru el; sa nu-i faci nici un rau, ci sa te por?i cu el a?a cum î?i va zice el!"
\par 13 Nebuzaradan, capetenia garzii, Nebuzaradan, capetenia eunucilor, Nergal-?are?er, capetenia vrajitorilor, ?i toate capeteniile regelui Babilonului
\par 14 Au trimis ?i au luat pe Ieremia din curtea temni?ii ?i l-au încredin?at lui Godolia, fiul lui Ahicam, fiul lui ?afan, ca sa-l duca acasa. A?a a ramas el sa locuiasca în mijlocul poporului.
\par 15 Atunci a fost cuvântul Domnului catre Ieremia, pe când era el înca ?inut în curtea temni?ii, ?i i-a zis:
\par 16 "Mergi ?i spune lui Ebed-Melec, Etiopianul: A?a zice Domnul Savaot, Dumnezeul lui Israel: Iata, Eu voi împlini cuvintele Mele despre cetatea aceasta spre raul, iar nu spre binele ei, ?i se vor împlini ele în ziua aceea înaintea ochilor tai.
\par 17 Dar pe tine te voi izbavi, zice Domnul, ?i nu vei fi dat în mâinile oamenilor de care te temi.
\par 18 Te voi izbavi ?i nu vei cadea în sabie; sufletul tau va ramâne la tine ca ?i când ai fi dobândit o prada, pentru ca tu ti-ai pus nadejdea în Mine", zice Domnul.

\chapter{40}

\par 1 Cuvântul care a fost de la Domnul catre Ieremia, dupa ce Nebuzaradan, capetenia garzii, i-a dat drumul din Rama, unde fusese gasit ferecat în lan?uri în mijlocul celorlal?i robi ai Ierusalimului ?i ai lui Iuda, care au fost stramuta?i la Babilon.
\par 2 Atunci a luat capetenia garzii pe Ieremia ?i i-a zis: "Domnul Dumnezeul tau a rostit aceasta nenorocire asupra locului acestuia,
\par 3 ?i acum a adus-o peste el ?i a facut ceea ce zisese, pentru ca a?i pacatuit înaintea Domnului ?i n-a?i ascultat glasul Lui, de aceea v-a ?i ajuns pe voi aceasta.
\par 4 Deci, iata eu î?i dau drumul din lan?urile ce le ai la mâini! De vrei sa mergi cu mine la Babilon, vino, ?i eu voi avea grija de tine, iar de nu ai placere sa mergi cu mine la Babilon, ramâi. Iata, toata ?ara e înaintea ta, unde vrei ?i unde-?i place, acolo du-te!"
\par 5 Înainte de a ie?i el, Nebuzaradan i-a zis: "Du-te la Godolia, fiul lui Ahicam, fiul lui ?afan, pe care regele Babilonului l-a pus capetenie peste cetatea Ierusalimului, ?i ramâi cu dânsul în mijlocul poporului; sau du-te unde-?i place ?ie sa te duci!" ?i capetenia garzii i-a dat merinde ?i daruri ?i l-a slobozit.
\par 6 Deci a venit Ieremia la Godolia, fiul lui Ahicam, la Mi?pa, ?i a trait cu el în mijlocul poporului care ramasese în ?ara.
\par 7 Auzind toate capeteniile o?tirilor, care erau în câmp cu oamenii lor, ca regele Babilonului î pus pe Godolia, fiul lui Ahicam, capetenie peste ?ara ?i i-a încredin?at lui barba?ii, femeile ?i copiii ?i pe aceia din saracii ?arii, care n-au fost stramuta?i la Babilon,
\par 8 Au venit la Godolia, în Mi?pa, Ismael, fiul lui Netania, Iohanan ?i Ionatan, fiii lui Carea, Seraia, fiul lui Tanhumet, ?i fiii lui Efai Netofitul ?i Iezania, fiul lui Maacat, ei ?i oamenii lor.
\par 9 Iar Godolia, fiul lui Ahicam, fiul lui ?afan, li s-a jurat lor ?i oamenilor lor, zicând: "Nu va teme?i a sluji Caldeilor; ramâne?i în ?ara, sluji?i regelui Babilonului ?i va va fi bine.
\par 10 Iar eu voi ramâne în Mi?pa, ca sa mijlocesc înaintea Caldeilor, care vor veni la noi. Voi însa strânge?i vinul ?i fructele verii ?i untdelemnul ?i le pune?i în vasele voastre ?i trai?i în ceta?ile voastre, în care va afla?i".
\par 11 De asemenea ?i to?i Iudeii care se aflau în Moab, la Amoni?i, în Idumeea ?i prin toate ?arile, au auzit ca regele Babilonului a lasat o parte din Iudei ?i a pus peste ei pe Godolia, fiul lui Ahicam, fiul lui ?afan.
\par 12 ?i s-au întors to?i ace?ti Iudei de prin toate locurile pe unde fusesera împra?tia?i ?i au venit în pamântul lui Iuda, la Godolia, în Mi?pa, ?i au adunat vin ?i fructe multe foarte.
\par 13 Iar Iohanan, fiul lui Carea, ?i toate capeteniile o?tirii, care erau în câmp, au venit la Godolia, în Mi?pa, ?i au zis catre dânsul:
\par 14 "?tii tu oare ca Baalis, regele Amoni?ilor, a trimis pe Ismael, fiul lui Netania, ca sa te ucida?" Dar Godolia, fiul lui Ahicam, nu i-a crezut.
\par 15 Atunci Iohanan, fiul lui Carea, a spus lui Godolia în taina, la Mi?pa: "Da-mi voie sa ma duc sa ucid pe Ismael, fiul lui Netania, ?i nimeni nu va afla. De ce sa-l la?i sa te ucida el pe tine ?i sa se risipeasca to?i cei din Iuda, care s-au adunat la tine ?i sa piara rama?i?a lui Iuda?"
\par 16 Însa Godolia, fiul lui Ahicam, a zis catre Iohanan, fiul lui Carea: "Sa nu faci aceasta, caci tu graie?ti neadevar despre Ismael!"

\chapter{41}

\par 1 Prin luna a ?aptea, Ismael, fiul lui Netania, fiul lui Eli?ama, din neam regesc, a venit cu dregatorii regelui ?i cu zece oameni la Godolia, fiul lui Ahicam, în Mi?pa, ?i au mâncat acolo pâine împreuna.
\par 2 Apoi s-a sculat Ismael, fiul lui Netania, ?i cei zece oameni care erau cu el, ?i au lovit cu sabia pe Godolia, fiul lui Ahicam, fiul lui ?afan, ?i au ucis pe acela pe care regele Babilonului îl pusese capetenie peste ?ara.
\par 3 De asemenea a ucis Ismael ?i pe to?i Iudeii care erau cu Godolia, în Mi?pa, precum ?i pe Caldeii o?teni care se aflau acolo.
\par 4 Iar a doua zi dupa uciderea lui Godolia, pe când nu ?tia nimeni de aceasta,
\par 5 Au venit din Sihem, din ?ilo ?i din Samaria optzeci de oameni cu barbile rase ?i cu hainele sfâ?iate ?i cu taieturi pe trup, aducând daruri ?i tamâie în mâinile lor pentru jertfa în templul Domnului.
\par 6 Iar Ismael, fiul lui Netania, din Mi?pa, a ie?it întru întâmpinarea lor; ?i mergând el plângând, s-a întâlnit cu ei ?i le-a zis: "Duce?i-va la Godolia, fiul lui Ahicam".
\par 7 ?i dupa ce au ajuns ei în mijlocul ceta?ii, Ismael, fiul lui Netania, i-a ucis ?i i-a aruncat într-o groapa, ajutat de oamenii care erau cu el.
\par 8 Dar s-au gasit printre ace?tia zece oameni care au zis lui Ismael: "Nu ne ucide, caci noi avem agoniseli de grâu, de orz, de untdelemn ?i de miere, ascunse la câmp". ?i el s-a oprit ?i nu i-a omorât pe aceia împreuna cu ceilal?i fra?i ai lor.
\par 9 Groapa însa în care a aruncat Ismael toate trupurile oamenilor, pe care el i-a ucis pentru Godolia, era chiar aceea pe care o facuse regele Asa, temându-se de Bae?a, regele lui Israel. Pe aceasta a umplut-o Ismael, fiul lui Netania, cu cei uci?i.
\par 10 Apoi Ismael a luat c, prizonieri tot restul de popor, ce era în Mi?pa, pe fiicele regelui ?i tot poporul ce mai ramasese în Mi?pa, pe care Nebuzaradan, capetenia garzii, îl încredin?ase lui Godolia, fiul lui Ahicam; ?i luându-i pe ace?tia, Ismael, fiul lui Netania, s-a îndreptat spre fiii lui Amon.
\par 11 Dar când Iohanan, fiul lui Carea, ?i to?i capitanii o?tirii, care erau cu el, au auzit de nelegiuirile ce savâr?ise Ismael, fiul lui Netania, au luat to?i oamenii lor ?i s-au dus sa se bata cu Ismael, fiul lui Netania,
\par 12 ?i I-au ajuns la iazul cel mare din Gabaon.
\par 13 Când tot poporul ce era cu Ismael a vazut pe Iohanan, fiul lui Carea, ?i pe toate capeteniile o?tirii care erau cu dânsul, s-au bucurat.
\par 14 ?i s-a întors tot poporul pe care Ismael îl ducea în robie din Mi?pa ?i, plecând, s-a dus la Iohanan, fiul lui Carea,
\par 15 Iar Ismael, fiul lui Netania, a fugit de Iohanan cu opt oameni ?i s-a dus la fiii lui Amon.
\par 16 Atunci Iohanan, fiul lui Carea, ?i toate capeteniile o?tirii, care erau cu el, au luat din Mi?pa tot poporul ce ramasese ?i pe care el îl scapase de Ismael, fiul lui Netania, dupa ce acesta ucisese pe Godolia, fiul lui Ahicam - barba?ii, o?tenii, femeile, copiii ?i eunucii pe care îi scosese din Gabaon,
\par 17 ?i s-a dus ?i s-a oprit în satul Chimham, lânga Betleem,
\par 18 Ca apoi sa plece în Egipt de frica Caldeilor, caci se temea de ei, pentru ca Ismael, fiul lui Netania, ucisese pe Godolia, fiul lui Ahicam, pe care regele Babilonului îl pusese capetenie peste ?ara.

\chapter{42}

\par 1 Atunci toate capeteniile o?tirii ?i Iohanan, fiul lui Carea, ?i Azaria, fiul lui Ho?aia, ?i tot poporul, de la mic pâna la mare, au venit ?i au zis catre Ieremia proorocul:
\par 2 "Sa aiba primire rugamintea noastra înaintea ta! Roaga-te Domnului Dumnezeului tau pentru noi, pentru to?i care au ramas, ca din mul?i, pu?ini au ramas, cum ne vezi cu ochii tai;
\par 3 Roaga-te ca Domnul Dumnezeul tau sa ne arate calea pe care sa mergem ?i ce sa facem".
\par 4 Zis-a catre ei Ieremia proorocul: "Ascult ?i iata ma voi ruga Domnului Dumnezeului vostru, dupa cuvântul vostru, ?i tot ce va va raspunde Domnul va voi spune ?i nu voi ascunde de voi nici un cuvânt".
\par 5 Iar ei au zis catre Ieremia: "Domnul sa fie martor împotriva noastra, credincios ?i adevarat, daca nu vom face întocmai dupa cuvântul pe care Domnul Dumnezeul tau ?i-l va fi trimis pentru noi.
\par 6 Fie bine, fie rau, noi vom asculta de glasul Domnului Dumnezeului nostru, catre Care te trimitem, ca sa fim ferici?i ascultând de glasul Domnului Dumnezeului nostru".
\par 7 Dupa trecerea a zece zile, a fost cuvântul Domnului catre Ieremia.
\par 8 ?i a chemat acesta la el pe Iohanan, fiul lui Carea, ?i pe toate capeteniile o?tirii, care erau cu el, ?i tot poporul, de la mic pâna la mare, ?i le-a zis:
\par 9 "A?a zice Domnul Dumnezeul lui Israel catre Care m-a?i trimis, ca sa aduc înaintea Lui rugamintea voastra:
\par 10 De ve?i ramâne în ?ara aceasta, Eu va voi zidi ?i nu va voi mai darâma, va voi sadi ?i nu va voi mai smulge, caci Îmi pare rau de raul pe care vi l-am facut.
\par 11 Nu va teme?i de regele Babilonului, de care va cutremura?i; nu va teme?i de el, zice Domnul, caci Eu sunt cu voi, ca sa va scap ?i sa va izbavesc din mâna lui.
\par 12 ?i va voi arata mila ?i el se va milostivi spre voi ?i va va întoarce în pamântul vostru.
\par 13 Iar de ve?i zice: "Nu vrem sa traim în ?ara aceasta" ?i de nu ve?i asculta glasul Domnului Dumnezeului vostru ?i ve?i zice:
\par 14 "Ba noi ne ducem în ?ara Egiptului, unde razboi nu vom vedea ?i glas de trâmbi?a nu vom auzi ?i foame nu vom duce, ?i acolo vom trai",
\par 15 Apoi asculta?i cuvântul Domnului, voi, care a?i mai ramas din Iuda. A?a zice Domnul Savaot, Dumnezeul lui Israel: "Daca voi va întoarce?i cu hotarâre nestramutata fa?a voastra, ca sa va duce?i în Egipt ?i va ve?i duce sa trai?i acolo,
\par 16 Sabia de care va teme?i va va ajunge acolo, în ?ara Egiptului, ?i foamea de care va îngrozi?i va va înso?i pa?ii vo?tri acolo în Egipt ?i acolo ve?i muri.
\par 17 To?i cei ce î?i întorc fa?a lor, ca sa se duca în Egipt ?i sa traiasca acolo; vor muri de sabie, de foame ?i de molima ?i nici unul din ei nu va ramâne ?i nu va scapa de nenorocirea aceea pe care o voi aduce asupra lor.
\par 18 Caci a?a zice Domnul Savaot, Dumnezeul lui Israel: Cum s-a varsat mânia Mea ?i iu?imea Mea asupra locuitorilor Ierusalimului, a?a se va varsa iu?imea Mea ?i asupra voastra, când va ve?i duce în Egipt ?i ve?i fi blestem ?i grozavie, ocara ?i râs, ?i nu ve?i mai vedea locul acesta.
\par 19 Voua, cei rama?i ai lui Iuda, va zice Domnul: Nu va duce?i în Egipt. Sa ?ti?i bine ca Eu astazi v-am spus lucrul acesta în chip solemn.
\par 20 Va face?i rau voua în?iva, trimi?ându-ma la Domnul Dumnezeul vostru, zicând: "Roaga-te pentru noi Domnului Dumnezeului nostru, ?i tot ce va zice Domnul Dumnezeul nostru sa ne spui ?i noi vom face".
\par 21 A?adar eu v-am spus acum, dar voi n-a?i ascultat glasul Domnului Dumnezeului vostru ?i toate cu câte m-a trimis El la voi.
\par 22 Deci, sa ?ti?i ca ve?i muri de sabie, de foamete ?i de ciuma în locul acela, unde voi?i sa va duce?i ca sa trai?i".

\chapter{43}

\par 1 Dupa ce Ieremia a spus întregului popor toate cuvintele Domnului Dumnezeului lor, toate cuvintele acelea cu care Domnul Dumnezeul lor îl trimisese la ei,
\par 2 Atunci Azaria, fiul lui Ho?aia, Iohanan, fiul lui Carea, ?i to?i oamenii cei îngâmfa?i au zis catre Ieremia: "Neadevar spui, nu te-a trimis Domnul Dumnezeul nostru sa ne spui: "Nu va duce?i în Egipt ca sa trai?i acolo!"
\par 3 Ci Baruh, fiul lui Neria, te a?â?a împotriva noastra, ca sa ne dea în mâinile Caldeilor, sa ne omoare, sau sa ne duca robi la Babilon".
\par 4 Apoi Iohanan, fiul lui Carea, toate capeteniile o?tirii ?i tot poporul n-au ascultat glasul Domnului, ca sa ramâna în pamântul lui Iuda.
\par 5 ?i Iohanan, fiul lui Carea, ?i toate capeteniile o?tirii au luat pe to?i cei care ramasesera din Iuda ?i care se întorsesera din toate neamurile pe unde fusesera izgoni?i, ca sa traiasca în pamântul lui Iuda,
\par 6 Barba?i, femei ?i copii, fetele regelui ?i pe to?i aceia pe care Nebuzaradan, capetenia garzii, îi lasase cu Godolia, fiul lui Ahicam, fiul lui ?afan, pe Ieremia proorocul ?i pe Baruh, fiul lui Neria,
\par 7 ?i s-au dus în ?ara Egiptului, ca n-au ascultat glasul Domnului ?i au mers pâna la Tahpanhes.
\par 8 Iar la Tahpanhes a fost cuvântul Domnului catre Ieremia, zicând:
\par 9 "Ia în mâinile tale ni?te pietre mari ?i le ascunde în lut framântat, la intrarea casei lui Faraon, în Tahpanhes, înaintea ochilor Iudeilor ?i sa le spui:
\par 10 "A?a graie?te Domnul Savaot, Dumnezeul lui Israel: Iata, Eu voi trimite ?i voi lua pe Nabucodonosor. regele Babilonului, robul Meu, ?i voi a?eza tronul lui pe aceste pietre. ascunse de Mine, ?i î?i va întinde el pe ele cortul sau cel minunat.
\par 11 El va veni ?i va lovi ?ara Egiptului; cel rânduit spre moarte va fi dat mor?ii, cel rânduit pentru robie va merge în robie ?i cel rânduit spre sabie va cadea de sabie.
\par 12 Voi aprinde foc în capi?tile dumnezeilor Egiptului; Nabucodonosor le va arde pe acelea, pe idoli îi va distruge cu totul ?i se va îmbraca cu ?ara Egiptului, cum se îmbraca pastorul cu haina sa; ?i va ie?i de acolo nesuparat;
\par 13 El va sfarâma stâlpii templului Soarelui de la On ?i capi?tile dumnezeilor Egiptului le va arde cu foc".

\chapter{44}

\par 1 Cuvântul ce a fost catre Ieremia pentru to?i Iudeii care traiau în ?ara Egiptului ?i care se a?ezasera în Migdol, în Tahpanhes, în Nof ?i în ?ara Patros:
\par 2 "A?a zice Domnul Savaot, Dumnezeul lui Israel: A?i vazut toata nenorocirea pe care am adus-o Eu asupra Ierusalimului ?i asupra tuturor ceta?ilor lui Iuda; iata acelea sunt acum pustii ?i nimeni nu mai locuie?te în ele,
\par 3 Pentru necredin?a lor, pe care au savâr?it-o ele, mâniindu-Ma ?i mergând sa tamâieze ?i sa slujeasca altor dumnezei, pe care nu i-au cunoscut nici ei, nici voi, nici. parin?ii vo?tri.
\par 4 Trimis-am la voi necontenit pe to?i slujitorii Mei, proorocii; i-am trimis dis-de-diminea?a, ca sa va spuna: Nu face?i acest lucru urâcios, pe care Eu îl urasc.
\par 5 Dar ei n-au ascultat ?i nu ?i-au plecat urechea lor ca sa se întoarca de la necredin?a ?i sa nu tamâieze altor dumnezei.
\par 6 De aceea s-a revarsat urgia Mea ?i mânia Mea ?i s-a aprins în ceta?ile lui Iuda ?i pe uli?ele Ierusalimului, ?i s-au prefacut acelea în ruine ?i în pustiu, cum vede?i astazi.
\par 7 Iar acum a?a zice Domnul Dumnezeu Savaot, Dumnezeul lui Israel: Pentru ce face?i voi acest rau mare sufletelor voastre, pierzând din cuprinsul lui Iuda, barba?i, Femei, copii ?i prunci ca sa nu lasa?i dupa voi rama?i?a,
\par 8 Mâniindu-Ma prin lucrul mâinilor voastre, prin tamâierea altor dumnezei în pamântul Egiptului, unde a?i venit sa trai?i, ca sa va pierde?i pe voi în?iva ?i sa ajunge?i blestem ?i defaimare înaintea tuturor neamurilor pamântului?
\par 9 Au doar a?i uitat faradelegile parin?ilor vo?tri, faradelegile regilor lui Iuda ?i cele ale femeilor lor, faradelegile voastre ?i cele ale femeilor voastre, savâr?ite în ?ara lui Iuda ?i pe uli?ele Ierusalimului?
\par 10 Ei nu s-au smerit nici astazi ?i nu se tem ?i nu se poarta dupa legea Mea ?i dupa poruncile Mele, pe care vi le-am dat voua ?i parin?ilor vo?tri.
\par 11 De aceea, a?a zice Domnul Savaot, Dumnezeul lui Israel: Iata Eu voi întoarce fa?a Mea împotriva voastra ?i spre nimicirea întregului Iuda.
\par 12 ?i voi lua pe Iudeii care au ramas ?i care ?i-au întors fa?a lor ?i s-au dus în rara Egiptului, ca sa traiasca acolo; to?i aceia vor fi nimici?i, vor cadea în ?ara Egiptului; de foamete ?i de sabie vor fi nimici?i; vor muri de sabie ?i de foame de la mare pâna la mic ?i vor ajunge de blestem, de ocara, de groaza ?i de râs.
\par 13 Voi pedepsi pe cei ce locuiesc în ?ara Egiptului, precum am pedepsit Ierusalimul, cu sabie, cu foamete ?i cu ciuma,
\par 14 ?i nimeni nu va scapa ?i nu va ramâne din rama?i?a Iudeilor care au venit în ?ara Egiptului, ca sa traiasca acolo ?i apoi sa se întoarca în ?ara lui Iuda, unde doresc ei din tot sufletul sa se întoarca, ca sa traiasca acolo. Nimeni nu se va întoarce, decât numai aceia care vor fugi de aici".
\par 15 Atunci au raspuns lui Ieremia to?i barba?ii care ?tiau ca femeile lor tamâiaza pe al?i dumnezei, ?i toate femeile care se aflau acolo în numar mare, ?i tot poporul, care locuia în pamântul Egiptului, în Patros, ?i au zis:
\par 16 "Cuvântul pe care tu l-ai grait în numele Domnului, nu voim sa-l ascultam de la tine;
\par 17 Dar vom face tot ce am fagaduit cu gura noastra, vom tamâia pe zei?a cerului ?i vom savâr?i pentru ea jertfe cu turnare, cum am mai facut ?i noi ?i parin?ii no?tri, regii no?tri ?i mai-marii no?tri în ceta?ile lui Iuda ?i pe uli?ele Ierusalimului, ca atunci eram satui ?i ferici?i ?i n-am vazut nenorociri.
\par 18 Iar de când am încetat de a mai tamâia pe zei?a cerului ?i a-i savâr?i jertfe cu turnare, suferim toate lipsurile ?i pierim de sabie ?i de foame".
\par 19 ?i femeile au adaugat: "Când tamâiam noi zei?a cerului ?i-i savâr?eam jertfe cu turnare, oare fara ?tirea barba?ilor no?tri îi faceam noi turte cu chipul ei ?i-i savâr?eam jertfe cu turnare?"
\par 20 Atunci Ieremia a zis catre tot poporul, catre barba?i ?i catre femei, ?i catre oamenii care îi raspundeau, astfel:
\par 21 "Oare nu aceasta tamâiere, pe care o savâr?ea?i în ceta?ile lui Iuda ?i pe uli?ele Ierusalimului ?i voi ?i parin?ii vo?tri, regii vo?tri ?i mai-marii vo?tri ?i ai poporului ?arii, o aminte?te Domnul? Oare nu ea I s-a suit la inima?
\par 22 Domnul n-a mai putut suferi faptele voastre cele rele ?i urâciunile pe care le facea?i, ?i de aceea s-a ?i prefacut ?ara voastra în pustietate, lucru de spaima ?i de blestem ?i fara locuitori, cum vede?i acum.
\par 23 Deoarece voi, savâr?ind acea tamâiere, a?i gre?it înaintea Domnului ?i n-a?i ascultat glasul Domnului, nu v-a?i purtat dupa legea Lui, dupa poruncile Lui ?i dupa rânduiala Lui, de aceea v-a ajuns nenorocirea aceasta, cum vede?i acum".
\par 24 ?i a mai zis Ieremia catre tot poporul ?i catre toate femeile: "Asculta?i cuvântul Domnului, voi, to?i Iudeii, cei ce sunte?i în ?ara Egiptului:
\par 25 A?a zice Domnul Savaot, Dumnezeul lui Israel: Voi ?i femeile voastre, ceea ce a?i grait cu gurile voastre aceea a?i facut ?i cu mâinile, ?i mai ?i zicea?i: Vom îndeplini întocmai fagaduin?ele pe care le-am facut, ca sa tamâiem pe zei?a cerului ?i sa-i savâr?im jertfe cu turnare". Voi va ?ine?i bine de faradelegile voastre ?i îndeplini?i întocmai acele fagaduin?e.
\par 26 De aceea asculta?i cuvântul Domnului: Iata Eu M-am jurat pe numele Meu cel mare, zice Domnul, ca nu va mai fi rostit numele Meu de gura vreunui iudeu în toata ?ara Egiptului. Nimeni nu va zice: Viu este Domnul Dumnezeu!
\par 27 Iata Eu voi veghea asupra lor spre pieire, iar nu spre bine; ?i to?i oamenii din Iuda, care sunt în ?ara Egiptului vor pieri de sabie ?i de foame, pâna se vor stinge de tot.
\par 28 Numai un mic numar, care vor fugi de sabie, se va întoarce din ?ara Egiptului în ?ara lui Iuda, ?i vor ?ti to?i cei rama?i din Iuda, care au venit în ?ara Egiptului ca sa traiasca acolo, al cui cuvânt se va împlini: al Meu sau al lor.
\par 29 Iata semnul ce vi-l dau, zice Domnul, ca va voi pedepsi în locul acesta, ca sa ?ti?i ca se va împlini cuvântul Meu cel pentru voi spre pieirea voastra.
\par 30 Asa zice Domnul: Iata Eu voi da pe Faraonul Hofra, regele Egiptului, în mâinile vrajma?ilor lui, cum am dat pe Sedechia, regele lui Iuda, în mâinile lui Nabucodonosor, regele Babilonului, vrajma?ul lui, care voia sa-i ia via?a".

\chapter{45}

\par 1 Cuvântul pe care l-a spus proorocul Ieremia lui Baruh, fiul lui Neria, când a scris el cuvintele acestea din gura lui Ieremia în carte, în anul al patrulea al domniei lui Ioiachim, fiul lui Iosia, regele lui Iuda:
\par 2 "Asa zice Domnul Dumnezeul lui Israel catre tine, Baruh:
\par 3 Tu zici: "Vai de mine, ca Domnul a adaugat durere la boala mea; am slabit suspinând ?i nu-mi gasesc lini?tea!"
\par 4 Spune-i, zice Domnul: Iata, ceea ce am zidit, voi darâma, ?i ce am sadit, voi smulge, - adica toata ?ara aceasta.
\par 5 Iar tu ceri pentru tine lucru mare. Nu cere, deoarece Eu voi aduce nenorocire asupra a tot trupul, zice Domnul, iar ?ie, drept prada, î?i voi lasa via?a ta, în toate locurile, oriunde vei merge".

\chapter{46}

\par 1 Cuvântul Domnului care a fost catre Ieremia proorocul, privitor la neamuri:
\par 2 Asupra Egiptului, împotriva o?tirii Faraonului Neco, regele Egiptului, care se afla în Carchemi?, lânga fluviul Eufratului ?i pe care a zdrobit-o Nabucodonosor, regele Babilonului, în anul al patrulea al lui Ioiachim, fiul lui Iosia, regele lui Iuda:
\par 3 "Gati?i scuturile ?i suli?ele ?i pa?i?i la lupta!
\par 4 Calare?i, în?eua?i caii ?i încaleca?i! Pune?i-va coifurile, ascu?i?i suli?ele ?i îmbraca?i-va în zale!
\par 5 Dar ce vad Eu? Aceia s-au înspaimântat ?i s-au întors înapoi, cei puternici ai lor au fost zdrobi?i ?i fug fara a se uita înapoi. Pretutindeni e groaza, zice Domnul.
\par 6 Cel iute de picior nu va scapa cu fuga, nici se va mântui cel puternic; la miazanoapte, pe râul Eufratului, se vor poticni ?i vor cadea.
\par 7 Cine este acela, care se înal?a ca Nilul ?i î?i învaluie apele ca un fluviu?
\par 8 Egiptul se înal?a ca Nilul ?i ca un fluviu î?i învaluie apele sale ?i zice: Ridica-ma-voi ?i voi acoperi pamântul, pierde-voi ceta?ile ?i locuitorii lor.
\par 9 Încaleca?i pe cai ?i va arunca?i în caru?e ?i porni?i, puternici Etiopieni ?i Libieni înarma?i cu scut, ?i voi Lidieni, ?inând arcurile ?i încordându-le,
\par 10 Caci ziua aceasta e zi de razbunare la Domnul Dumnezeu, ca sa Se razbune pe vrajma?ii Sai, ?i sabia va mânca, se va satura ?i se va îmbata de sângele lor; ?i aceasta va fi jertfa Domnului Dumnezeului Savaot în ?ara cea de la miazanoapte, la râul Eufratului.
\par 11 Fecioara, fiica Egiptului, mergi în Galaad ?i ia-?i balsam; în zadar vei spori leacurile tale, caci nu mai este vindecare pentru tine.
\par 12 Auzit-au popoarele de ru?inea ta ?i bocetul tau a umplut pamântul, caci cel puternic s-a lovit cu cel puternic ?i au cazut amândoi împreuna".
\par 13 Cuvântul pe care l-a grait Domnul proorocului Ieremia despre navalirea lui Nabucodonosor, regele Babilonului, ca sa loveasca pamântul Egiptului:
\par 14 "Vesti?i în Egipt ?i da?i ?tire în Migdol; spune?i în Nof ?i în Tahpanhes, ?i zice?i: Stai ?i te gate?te, caci sabia nimice?te în jurul tau!
\par 15 Ce? Apis a fugit? Cel puternic al tau a cazut? Da, fiindca Domnul l-a lovit.
\par 16 El înmul?e?te pe cei care se clatina ?i cad unii peste al?ii, fiecare dintre ei zicând: Sa ne sculam ?i sa ne întoarcem la poporul nostru, în pamântul patriei noastre, departe de sabia ucigatoare.
\par 17 Iar lui Faraon, regele Egiptului, da?i-i nume: "Zarva care scapa clipa hotarâtoare!"
\par 18 Viu sunt Eu, zice Împaratul, al Carui nume este Domnul Savaot, pe cât e de adevarat ca Taborul este în rândul mun?ilor ?i Carmelul lânga mare, tot atât e de adevarat ca el va veni.
\par 19 Fecioara, locuitoarea Egiptului, gate?te-?i cele trebuitoare pentru bejenie, caci Noful va fi pustiit, darapanat ?i fara locuitori.
\par 20 Egiptul este o juninca preafrumoasa peste care s-a napustit un taune de la miazanoapte.
\par 21 ?i simbria?ii lui sunt în mijlocul lui ca ni?te tauri îngra?a?i; ace?tia s-au întors înapoi, au fugit to?i ?i n-au ?inut piept, pentru ca a venit asupra lor ziua pieirii lor, timpul pedepsirii lor.
\par 22 Glasul lor se aude ca ?uierul ?arpelui, vin cu o?tire ?i tabarasc asupra lui cu topoarele, ca taietorii de lemne;
\par 23 Taie padurea lui, zice Domnul, caci sunt nenumara?i, sunt mai mul?i decât lacustele ?i nu mai au numar.
\par 24 Fiica Egiptului batjocorita e data pe mâna poporului de la miazanoapte.
\par 25 Domnul Savaot, Dumnezeul lui Israel, zice: "Iata, Eu voi pedepsi pe Amon, care se afla în No, pe Faraon ?i Egiptul, pe dumnezeii ?i pe regii lui, pe Faraon ?i pe cei ce-?i pun nadejdea în el.
\par 26 ?i-i voi da în mâinile celor ce cauta via?a lor, în mâinile lui Nabucodonosor, regele Babilonului, ?i în mâinile slujitorilor lui. Dar dupa aceea Egiptul va fi iara?i locuit, ca în zilele de altadata, zice Domnul.
\par 27 Iar tu, Iacove, robul Meu, nu te teme ?i nu te înspaimânta Israele! Caci iata, Eu te voi elibera din ?ara departata ?i neamul tau îl voi aduce din ?ara robiei lui; Iacov se va întoarce ?i va trai lini?tit ?i pa?nic ?i nimeni nu-l va înspaimânta.
\par 28 Iacove, robul Meu, nu te teme, zice Domnul, caci Eu sunt cu tine ?i voi pierde toate neamurile printre care te-am împra?tiat, iar pe tine nu te voi pierde, ci numai te voi pedepsi cu masura, caci nepedepsit nu te voi lasa".

\chapter{47}

\par 1 Cuvântul Domnului care a fost catre proorocul Ieremia pentru Filisteni, înainte de a lovi Faraon Gaza.
\par 2 A?a zice Domnul: "Iata se ridica ape de la miazanoapte ?i cresc ca un fluviu ie?it din matca, îneaca ?ara ?i tot ce cuprinde ea, ceta?ile ?i locuitorii lor. Atunci oamenii vor striga ?i vor plânge to?i locuitorii ?arii,
\par 3 La ropotul zgomotos al copitelor puternicilor lui cai, la vuietul carelor lui ?i la huruitul ro?ilor lui. Parin?ii nu se vor mai uita la copiii lor, pentru ca mâinile lor vor fi încremenite
\par 4 De groaza zilei aceleia, care vine sa piarda pe to?i Filistenii ?i sa rapeasca Tirului ?i Sidonului orice nadejde de ajutor, caci va pustii Domnul pe Filisteni, rama?i?a insulei Caftor.
\par 5 Gaza a ple?uvit ?i va pieri Ascalonul, cu rama?i?a vailor lui". Pâna când î?i vei face taieturi de jale în piele?
\par 6 Pâna când vei taia, sabia Domnului? Pâna când nu te vei lini?ti? Întoarce-te în teaca ta, opre?te-te ?i te lini?te?te!
\par 7 Dar cum sa te lini?te?ti, când ?i-a dat Domnul porunca împotriva Ascalonului ?i împotriva ?armului marii? Într-acolo te-a îndreptat El".

\chapter{48}

\par 1 Asupra Moabului, a?a zice Domnul Savaot, Dumnezeul lui Israel: "Vai de Nebo, ca e pustiit! Chiriataimul este acoperit de ru?ine ?i luat. Cetatea este acoperita de ru?ine ?i înrobita!
\par 2 Nu mai este slava Moabului! În He?bon se urzesc rele împotriva lui, zicând: "Sa mergem ?i sa-l ?tergem dintre neamuri!" ?i tu, Madmen, vei pieri! Sabia vine în urma ta!
\par 3 Strigate se aud din Horonaim, caci e pustiire ?i darâmare grozava.
\par 4 Moab e zdrobit ?i copiii lui au ridicat bocet.
\par 5 Pe sui?ul de la Luhit se ridica plânsete peste plânsete ?i pe povârni?ul din Horonaim se aud strigate de nenorocire:
\par 6 Fugi?i! Scapa?i-va via?a ?i fi?i asemenea asinului salbatic din pustiu.
\par 7 ?i fiindca te-ai încrezut în lucrarile ?i în vistieriile tale vei fi luat ?i tu. Chemo?ul va merge în robie cu preo?ii ?i cu mai-marii sai, to?i împreuna.
\par 8 Va veni pustiitorul asupra fiecarei ceta?i ?i cetatea nu Va ramâne nedarâmata; va pieri valea ?i ?esul ?i se va pustii, precum a zis Domnul.
\par 9 Da?i un mormânt lui Moab, caci el va fi cu totul pustiit. Ceta?ile lui vor fi prefacute în pustiu ?i nu vor mai avea locuitori.
\par 10 Blestemat sa fie tot cel ce face lucrurile Domnului cu nebagare de seama ?i blestemat fie tot cel ce opre?te sabia lui de la sânge!
\par 11 Moab din tinere?ea lui a fost lini?tit, se odihnea pe drojdia sa ?i n-a fost trecut din vas în vas, nici în robie n-a fost. De aceea a ramas în el gustul sau ?i mirosul sau nu s-a schimbat.
\par 12 De aceea iata vin zile, zice Domnul, când voi trimite la el pritocitori, care îl vor pritoci ?i vor strica vasele ?i urcioarele lui le vor sparge.
\par 13 ?i va fi ru?inat Moab pentru Chemo?, dupa cum casa lui Israel a fost ru?inata pentru Betel, nadejdea ei.
\par 14 Cum pute?i voi sa zice?i: "Noi suntem oameni viteji ?i tari pentru razboi?"
\par 15 Moab este pustiit, ceta?ile lui ard ?i tinerii lui ale?i s-au dus la junghiere, zice Împaratul al Carui nume este Domnul Savaot.
\par 16 Aproape este pieirea Moabului ?i nenorocirea lui vine în graba mare.
\par 17 jeli?i-l to?i vecinii lui ?i to?i cei ce cunoa?te?i numele lui zice?i: Cum s-a sfarâmat toiagul puterii, sceptrul slavei!
\par 18 Fecioara, locuitoarea Dibonului, coboara-te din înal?ime ?i ?ezi în pamânt ars de soare, caci pustiitorul Moabului vine la tine ?i va darâma întariturile tale.
\par 19 Stai la drum ?i prive?te, locuitoarea Aroerului, ?i întreaba pe cel care fuge ?i pe cel scapat: Ce s-a întâmplat?
\par 20 Ru?inat este Moabul, caci este biruit. Plânge?i ?i striga?i, da?i de veste în Amon ca Moabul este pustiit.
\par 21 Judecata a venit împotriva ?arii din ?es, împotriva Holonului ?i a Iah?ei, împotriva Mefaatului ?i a Dibonului,
\par 22 Împotriva lui Nebo ?i a Bet-Diblataimului.
\par 23 Împotriva Chiriataimului ?i a Bet-Gamului, împotriva Bet-Meonului ?i a Cheriotului,
\par 24 Împotriva Bosrei ?i împotriva tuturor ceta?ilor ?arii Moabului, de aproape ?i de departe.
\par 25 S-a taiat cornul  Moabului ?i bra?ul lui este zdrobit, zice Domnul.
\par 26 Îmbata?i-l, caci s-a ridicat împotriva Domnului. Tavaleasca-se Moabul în varsatura sa ?i de râs sa fie.
\par 27 N-a fost Israel de râsul tau? Oare a fost el prins printre tâlhari, de clatinai din cap ori de câte ori vorbeai cu el?
\par 28 Parasi?i ceta?ile ?i trai?i pe stânci, locuitori ai Moabului, ?i ve?i fi ca porumbeii care-?i fac cuiburile pe la intrarea pe?terilor.
\par 29 Auzit-am de mândria Moabului, de mândria lui cea nemasurata, de trufia lui ?i de îngâmfarea lui, de înfumurarea lui ?i de seme?ia inimii lui.
\par 30 Eu cunosc îndrazneala lui, zice Domnul, laudaro?eniile lui, vorbele goale ?i faptele de?arte.
\par 31 De aceea voi plânge pe Moab ?i voi striga pentru Moabul întreg; voi suspina dupa oamenii din Chir-Heres.
\par 32 ?i te voi plânge pe tine, vie din Sibma, cu mai mult plânset decât Iazerul; ramurile tale s-au întins peste mare, ajuns-au pâna la Iazer, pustiitorul a navalit asupra roadelor tale celor varatice ?i asupra strugurilor cop?i.
\par 33 Bucuria ?i veselia au fost luate din Carmel ?i din rara Moabului; voi seca vinul din teascuri ?i nimeni nu va mai calca teascul cu strigate de veselie; va fi strigat de razboi ?i nu strigat de bucurie.
\par 34 De la He?bon pâna la Eleale ?i Iaha?, de la ?oar pâna la Horonaim ?i pâna la Eglat-?eli?ia vor ridica glas de tânguire, caci apele Nimrimului vor seca.
\par 35 Voi stârpi din Moab, zice Domnul, pe cei ce aduc jertfe pe locuri înalte ?i pe cei ce tamâiaza pe dumnezeii lui.
\par 36 De aceea inima Mea geme pentru Moab, ca un fluier; geme inima Mea ca un fluier pentru oamenii din Chir-Heres, caci au pierit boga?iile adunate de ei;
\par 37 Fiecare î?i are capul ras, fiecare î?i are barba tunsa; to?i au taieturi pe mâini ?i peste coapse sac.
\par 38 Pe toate acoperi?urile Moabului ?i în pie?ele lui nu este decât o jale, caci am sfarâmat Moabul, ca pe un vas de aruncat, zice Domnul.
\par 39 Cum a fost zdrobit!, vor zice plângând. Cum s-a acoperit Moabul de ru?ine, întorcând spatele! ?i va fi Moabul de râs ?i de groaza pentru to?i cei ce-l înconjoara;
\par 40 Caci a?a zice Domnul: Iata, vrajma?ul ca un vultur va zbura ?i-?i va întinde aripile sale deasupra Moabului.
\par 41 Ora?ele vor fi luate ?i ceta?ile cucerite; inima vitejilor Moabi?i va fi în ziua aceea ca inima unei femei chinuita de durerile na?terii.
\par 42 Moabul va fi ?ters din numarul popoarelor, pentru ca s-a ridicat împotriva Domnului.
\par 43 Groaza, groapa ?i la? sunt pentru tine, locuitorule al Moabului, a zis Domnul.
\par 44 Cel ce va scapa de groaza va cadea în groapa; ?i cel ce va scapa de groapa va cadea în la?, caci Eu voi aduce asupra lui, asupra Moabului, anul pedepsei lui, zice Domnul.
\par 45 Fugarii obosi?i s-au oprit la umbra He?bonului, dar a ie?it foc din He?bon, ?i din inima Sihonului flacari, ?i va mistui tâmplele lui Moab ?i cre?tetul capului fiilor razvratirii.
\par 46 Vai ?ie, Moab! Pierit-a poporul din Chemo?, ca fiii tai sunt lua?i în robie ?i fiicele tale sunt robite.
\par 47 Dar în zilele cele de apoi voi întoarce pe Moab din robie", zice Domnul. Pâna aici e judecata lui Moab.

\chapter{49}

\par 1 Asupra fiilor lui Amon a?a graie?te Domnul: "Au doara Israel n-are fii? Au doara el n-are mo?tenitor? Pentru ce, dar, Milcom a pus stapânire pe Gad ?i poporul lui traie?te în ceta?ile acestuia?
\par 2 De aceea, iata vin zile, zice Domnul, când în Raba, cetatea fiilor lui Amon, se va auzi strigat de razboi, ?i aceasta va fi prefacuta în movila de darâmaturi; ceta?ile ei vor fi arse cu foc ?i Israel va stapâni pe cei ce-l stapâneau, zice Domnul.
\par 3 Plângi He?bon, ca s-a pustiit Ai! Striga?i, voi fiice din Raba, încinge?i-va cu sac, plânge?i ?i rataci?i prin livezi, caci Milcom va merge în robie cu preo?ii ?i cu mai-marii sai, to?i împreuna.
\par 4 "Fiica nepasatoare, tu te lauzi cu Valea ta, tu te încrezi în comorile tale, zicând: "Cine va îndrazni sa vina împotriva mea?"
\par 5 Iata Eu voi aduce asupra ta groaza din toate par?ile, zice Domnul Dumnezeul Savaot ?i ve?i fi fugari?i care încotro ?i nimeni nu va aduna pe fugari.
\par 6 Dar dupa aceea voi întoarce din robie pe fiii lui Amon", zice Domnul.
\par 7 Iar asupra Edomului a?a graie?te Domnul Savaot: "Au doara nu mai este în?elepciune în Teman? Au doara lipse?te sfatul la cei în?elep?i?
\par 8 Au doara a secat în?elepciunea lor? Fugi?i, întorcând spatele! Ascunde?i-va în pe?teri, locuitori din Dedan, caci voi aduce nenorocire peste Isav, la vremea pedepsirii lui.
\par 9 Daca ar fi venit la tine culegatorii de struguri, de buna seama ar fi lasat pu?ine bobi?e neculese. De ar fi venit ho?ii noaptea, ar fi rapit cât le-ar fi fost de trebuin?a.
\par 10 Eu însa voi jefui pe Isav pâna la piele, voi descoperi ascunzatorile lui, ?i el nu se va putea ascunde. Stârpit va fi neamul lui, fra?ii lui ?i vecinii lui, ?i el nu va mai fi.
\par 11 Lasa?i orfanii, ca Eu voi ?ine via?a lor, ?i vaduvele tale sa creada în Mine;
\par 12 Caci a?a graie?te Domnul: Iata, celor ce nu le-a fost dat sa bea paharul, îl vor bea negre?it. Au doara numai tu vei ramâne nepedepsit? Nu, nu vei ramâne nepedepsit, ci vei bea negre?it paharul.
\par 13 Caci Ma jur pe Mine Însumi, zice Domnul, ca Bo?ra va fi o groaza ?i o ru?ine, un de?ert ?i un blestem, ?i ceta?ile ei vor ajunge ruine ve?nice.
\par 14 Auzit-am veste de la Domnul ?i sol s-a trimis la popoare sa spuna: "Aduna?i-va ?i merge?i împotriva ei ?i ridica?i-va pentru lupta".
\par 15 Iata Eu te voi face mic între popoare.
\par 16 Starea ta groaznica ?i seme?ia inimii tale te-au adeverit pe tine, cel ce locuie?ti în crapaturile stâncilor ?i care te-ai a?ezat pe vârfurile dealurilor! De?i ?i-ai împletit cuibul tau sus, ca un vultur, ?i de acolo te voi arunca, zice Domnul.
\par 17 Edom va fi o groaza; to?i cei ce vor trece pe lânga el vor fi uimi?i, vor fluiera, vazând toate ranile lui.
\par 18 Cum au fost aruncate Sodoma, Gomora ?i ceta?ile vecine cu ele, zice Domnul, tot a?a ?i acolo nu va trai nici om, nici fiu de om nu se va opri în el.
\par 19 Iata-l, se înal?a ca un leu de la obâr?iile Iordanului spre sala?urile întarite; dar Eu îi voi sili sa plece cu grabire din Edom ?i cine va fi ales, pe acela îl voi pune peste acesta. Caci cine este asemenea Mie ?i ce pastor se va împotrivi Mie?
\par 20 Asculta?i dar hotarârea Domnului pe care a luat-o El asupra lui Edom ?i planurile pe care El le-a gândit împotriva locuitorilor din Teman: "Da, chiar oile cele mai plapânde vor fi târâte; la vederea lor pa?unea lor se va înfiora de spaima.
\par 21 La vuietul caderii lor se va cutremura pamântul ?i rasunetul strigatului lor se va auzi pâna la Marea Ro?ie.
\par 22 Iata-l se înal?a, ca un vultur, ?i zboara ?i î?i întinde aripile sale deasupra Bo?rei, ?i inima razboinicilor din Edom va fi în ziua aceea ca inima unei femei ce na?te".
\par 23 Asupra Damascului: "Hamatul ?i Arpadul sunt tulburate, caci au primit o veste rea. Inima lor se tope?te de frica; este o mare în furtuna care nu se poate potoli.
\par 24 Temutu-s-a Damascul ?i a luat-o la fuga; de frica l-au cuprins dureri ?i chinuri, ca pe femeia ce na?te.
\par 25 Cum n-a scapat cetatea slavei, cetatea bucuriei mele!
\par 26 Deci vor cadea tinerii lui pe uli?ele lui ?i to?i vor pieri în ziua aceea, zice Domnul Savaot.
\par 27 Voi aprinde foc în zidurile Damascului ?i el va mistui palatele lui Benhadad".
\par 28 Asupra Chedarului ?i asupra regatelor Ha?orului, pe care le-a lovit Nabucodonosor, regele Babilonului, a?a graie?te Domnul: "Scula?i-va ?i ie?i?i înaintea lui Chedar ?i pustii?i pe fiii Rasaritului!
\par 29 Corturile lor ?i turmele lor sa fie luate, ?esaturile lor ?i toate lucrurile lor ?i camilele lor sa fie luate. ?i sa li se strige: Groaza din toate par?ile!
\par 30 Fugi?i, duce?i-va repede, ascunde?i-va în prapastii, locuitori ai Ha?orului, zice Domnul, caci Nabucodonosor, regele Babilonului, a luat hotarâre asupra voastra, ?i a urzit un plan împotriva voastra.
\par 31 Scula?i-va ?i pa?i?i împotriva poporului celui pa?nic, care traie?te fara grija, zice Domnul; acela n-are nici u?i, nici zavoare ?i locuie?te singur.
\par 32 Camilele lor vor fi date prazii ?i mul?imea turmelor lor rapirii ?i-i voi împra?tia în toate vânturile pe ace?tia care-?i tund parul de pe tâmple ?i din toate par?ile lui voi aduce asupra lor pieirea, zice Domnul.
\par 33 Ha?orul va fi sala?ul ?acalilor ?i pustietate ve?nica; nu va trai acolo om, nici fiu de om nu se va opri acolo".
\par 34 Cuvântul Domnului care a fost catre Ieremia proorocul împotriva Elamului, la începutul domniei lui Sedechia, regele lui Iuda:
\par 35 "A?a zice Domnul Savaot: Iata voi sfarâma arcul Elamului, puterea lui de capetenie,
\par 36 ?i voi aduce asupra Elamului patru vânturi din cele patru margini ale cerului ?i-l voi împra?tia în toate vânturile acestea ?i nu va fi neam la care sa nu ajunga Elami?i izgoni?i;
\par 37 Voi lovi pe Elami?i cu frica înaintea vrajma?ilor lor ?i înaintea celor care vor sa le ia via?a; asupra lor voi aduce nenorociri, mânia Mea aprinsa, zice Domnul, ?i în urma lor voi trimite sabie, pâna îi voi stârpi.
\par 38 Voi pune tronul Meu în Elam ?i voi stârpi de acolo pe rege ?i pe dregatori, zice Domnul.
\par 39 Dar în zilele cele de apoi, voi întoarce pe Elam din robie", zice Domnul.

\chapter{50}

\par 1 Cuvântul pe care l-a rostit Domnul despre Babilon ?i despre ?ara Caldeilor prin Ieremia proorocul:
\par 2 "Vesti?i ?i face?i cunoscut între popoare, ridica?i steag, spune?i ?i nu tainui?i, ci zice?i: Babilonul e luat, Bel e ru?inat, Merodah e zdrobit, chipurile lui cele cioplite sunt batjocorite ?i idolii lui sfarâma?i.
\par 3 Caci de la miazanoapte s-a ridicat asupra lui poporul care va preface pamântul în pustiu ?i nimeni nu va locui acolo, nici om, nici dobitoc; to?i se vor ridica ?i se vor duce.
\par 4 În zilele acelea ?i în vremea aceea, zice Domnul, vor veni fiii lui Israel, ?i împreuna cu ei vor merge ?i fiii lui Iuda ?i, plângând, voi cauta pe Domnul Dumnezeul lor;
\par 5 Vor întreba de calea Sionului ?i, întorcându-?i fe?ele spre el, vor zice: Veni?i ?i va uni?i cu Domnul prin legamânt ve?nic, care nu se va uita.
\par 6 Poporul Meu a fost ca oile cele pierdute; pastorii lor le-au abatut din cale ?i le-au împra?tiat prin mun?i; ratacit-au ele din deal în munte ?i ?i-au uitat staulul lor.
\par 7 To?i cei ce le gaseau le mâncau ?i du?manii lor ziceau: Nu suntem noi de vina, pentru ca ele au pacatuit împotriva Domnului, Loca?ul neprihanirii, împotriva Domnului, Nadejdea parin?ilor lor.
\par 8 Fugi?i din Babilon ?i pleca?i din ?ara Caldeilor ?i ve?i fi ca berbecii înaintea turmelor de oi.
\par 9 Caci iata, voi ridica ?i voi aduce împotriva Babilonului o mul?ime de neamuri mari din ?ara de la miazanoapte; acelea se vor în?ira împotriva lui ?i el va fi luat. Sage?ile lor sunt ca ale arca?ului iscusit, nu cad în zadar.
\par 10 Caldeea va fi prada lor ?i pustiitorii ei se vor satura, zice Domnul.
\par 11 "Caci voi v-a?i bucurat ?i v-a?i veselit când a?i rapit mo?tenirea Mea; zburda?i ca vi?eii pe paji?te, necheza?i ca armasarii!
\par 12 Mama voastra va fi în ru?ine mare ?i aceea care v-a nascut va ro?i. Iata, ca ea va fi cea mai de, pe urma dintre neamuri: un pustiu, un pamânt uscat ?i fara apa.
\par 13 De mânia Domnului ?ara lor va ajunge nelocuita ?i toata va fi pustiu. Tot cel ce va trece prin Babilon se va mira ?i va fluiera, vazând toate ranile lui.
\par 14 A?eza?i-va în rânduiala de bataie împrejurul Babilonului. Cei ce încorda?i arcul, trage?i în el, nu cru?a?i sage?ile, caci el a pacatuit împotriva Domnului.
\par 15 Ridica?i strigat de razboi împotriva lui din toate par?ile; el întinde mâna; cazut-au întariturile lui ?i zidurile lui s-au prabu?it, caci aceasta este rasplata Domnului! Razbuna?i-va pe el!
\par 16 Cum s-a purtat el, a?a purta?i-va ?i voi! Stârpi?i din Babilon pe cel ce seamana ?i pe cel ce lucreaza cu secera în vremea seceri?ului! De frica sabiei ucigatoare, sa se întoarca fiecare la poporul sau ?i fiecare sa fuga în ?ara sa.
\par 17 Pe Israel, turma cea risipita leii l-au prigonit; mai înainte l-a mâncat regele Asiriei, iar acum în urma Nabucodonosor, regele Babilonului, i-a zdrobit oasele.
\par 18 De aceea, a?a zice Domnul Savaot, Dumnezeul lui Israel: Iata, Eu voi pedepsi pe regele Babilonului ?i ?ara lui, cum am pedepsit ?i pe regele Asiriei,
\par 19 ?i voi întoarce pe Israel la pa?unea lui ?i va pa?te el pe Carmel ?i în Vasan; sufletul lui se va satura în muntele lui Efraim ?i în Galaad.
\par 20 În zilele acelea ?i în vremea aceea se va cauta nedreptatea lui Israel, zice Domnul, ?i nu se va afla, se vor cauta ?i pacatele lui Iuda ?i nu se vor gasi, caci voi ierta pe aceia pe care îi voi lasa în via?a.
\par 21 Ridica-te împotriva ?arii Merataim, împotriva ?arii Razvratirii ?i împotriva locuitorilor din Pecod, ?ara pedepsirii! Pustie?te ?i nimice?te, zice Domnul, ?i fa tot ce ?i-am poruncit!
\par 22 Strigate de razboi se aud în toata ?ara ?i prapadul este mare!
\par 23 Cum s-a sfarâmat ?i s-a zdrobit ciocanul lumii întregi! Cum a ajuns Babilonul de plâns pe pamânt!
\par 24 Întins-am curse pentru tine ?i te-ai prins, Babilonule, fara sa te a?tep?i. Gasit ai fost ?i prins, pentru ca te-ai ridicat împotriva Domnului.
\par 25 Domnul ?i-a deschis vistieria Sa ?i a luat din ea vasele mâniei Sale, pentru ca Domnul Dumnezeul Savaot are de lucru în ?ara Caldeilor.
\par 26 Alerga?i împotriva ei din toate par?ile, deschide?i jitni?ele ei, calca?i-o ca pe ni?te snopi, nimici?i-o de tot, ca sa nu mai ramâna nimic din ea!
\par 27 Ucide?i to?i boii ei! Sa mearga la junghiere! Vai lor, caci a venit ziua lor ?i vremea pedepsirii lor!
\par 28 Se aude glasul celor ce fug ?i al celor care scapa din ?ara Babilonului, ca sa vesteasca în Sion razbunarea Domnului Dumnezeului nostru, razbunarea cea pentru templul Sau!
\par 29 Chema?i împotriva Babilonului sagetatori! To?i cei ce încorda?i arcul, a?eza?i-va tabara împrejurul lui, ca nimeni sa nu scape din el! Rasplati?i-i dupa faptele lui! Cum s-a purtat el, a?a sa va purta?i ?i voi cu el, caci el s-a ridicat împotriva Domnului, împotriva Sfântului lui Israel.
\par 30 De aceea vor cadea tinerii lui pe uli?ele lui ?i to?i o?tenii lui vor fi pierdu?i în ziua aceea, zice Domnul.
\par 31 Iata, Eu sunt împotriva ta, "trufa?ule", zice Domnul Savaot, caci a venit ziua ta ?i timpul pedepsirii tale.
\par 32 ?i se va împiedica "trufa?ule" ?i va cadea ?i nimeni nu-l va ridica. Voi aprinde foc în ceta?ile lui ?i acesta va mistui toate împrejurul lui.
\par 33 A?a zice Domnul Savaot: Apasa?i sunt fiii lui Israel, ca ?i fiii lui Iuda, ?i to?i cei ce i-au robit îi ?in tare ?i nu vor sa le dea drumul.
\par 34 Dar Rascumparatorul lor este puternic ?i numele Lui este Domnul Savaot. Acesta va apara pricina lor, ca sa lini?teasca ?ara ?i sa faca pe locuitorii Babilonului sa tremure.
\par 35 Sabie împotriva Caldeilor, zice Domnul, ?i împotriva locuitorilor Babilonului ?i a capeteniilor lui ?i a în?elep?ilor lui.
\par 36 Sabie împotriva proorocilor minciunii, ca ei sa-?i piarda mintea! Sabie împotriva o?tenilor lui: sa se teama!
\par 37 Sabie împotriva cailor ?i a carelor lui ?i împotriva a toata mul?imea de oameni din el: sa fie ca ni?te femei! Sabie împotriva comorilor lui: sa fie jefuite!
\par 38 Uscaciune peste apele lui: sa sece! Fiindca aceasta este o ?ara de idoli ?i au înnebunit cu idolii lor.
\par 39 Acolo se vor a?eza fiarele pustiului cu ?acalii ?i vor trai în ea stru?ii; în veci nu va mai fi locuita ?i din neam în neam nu vor locui acolo oameni.
\par 40 Cum au fost aruncate de Domnul Sodoma ?i Gomora ?i ceta?ile vecine cu ele, zice Domnul, a?a ?i aici nici un om nu va trai, nici fiu de om nu va poposi acolo.
\par 41 Iata vine de la miazanoapte un popor mare ?i regi mul?i se ridica de la marginile pamântului.
\par 42 ?i ?in în mâna arc ?i suli?a; ace?tia sunt cruzi ?i nemilostivi ?i glasul lor e zgomotos ca marea, ?i vin pe cai, ca unii care sunt gata sa lupte cu tine, fiica Babilonului!
\par 43 Auzit-a regele Babilonului veste despre ei ?i au început sa-i tremure mâinile; l-a cuprins întristarea ?i dureri ca ale femeii ce na?te.
\par 44 Iata ca un leu se  ridica din tufi?urile Iordanului spre pa?unea totdeauna verde! Dar Eu îl voi face sa plece cu grabire de acolo ?i voi a?eza acolo pe cel pe care l-am ales. Caci cine este asemenea Mie? ?i cine-Mi va cere socoteala? ?i ce pastor se va împotrivi Mie?
\par 45 Asculta?i dar hotarârea Domnului pe care a luat-o El împotriva Babilonului ?i planurile pe care le-a facut El împotriva ?arii Caldeilor! Da, vor fi târâ?i ca oile cele marunte; la vederea lor pa?unea lor se va înfiora de spaima.
\par 46 La vuietul luarii Babilonului tremura pamântul ?i un strigat se întinde printre neamuri".

\chapter{51}

\par 1 A?a zice Domnul: "Iata, voi ridica împotriva Babilonului ?i a locuitorilor ?arii Caldeii un duh nimicitor.
\par 2 ?i voi trimite la Babilon vânturatori, care-l vor vântura ?i vor pustii ?ara lui, caci în ziua nenorocirii vor navali asupra lui din toate par?ile.
\par 3 Arca?ul sa-?i încordeze arcul împotriva arca?ului ?i împotriva celui ce se lauda cu plato?ele sale, ?i sa nu cru?e pe tinerii lui! Nimici?i toata o?tirea lui!
\par 4 Sa cada rani?i de moarte în ?ara Caldeilor ?i strapun?i pe strazile Babilonului.
\par 5 Caci Iuda ?i Israel n-au ramas vaduvi de Dumnezeul lor, de Domnul Savaot, ?i ?ara Caldeilor e plina de pacate înaintea Sfântului lui Israel.
\par 6 Fugi?i din Babilon, ?i fiecare sa-?i scape via?a, ca sa nu pieri?i pentru faradelegile lui, caci acesta este timpul razbunarii pentru Domnul, caci El îi va da rasplata.
\par 7 Babilonul a fost în mâna Domnului cupa de aur, care a îmbatat tot pamântul; baut-au popoarele din vinul ei ?i au înnebunit.
\par 8 Cazut-a fara de veste Babilonul, ?i s-a zdrobit. Plânge?i-l ?i aduce?i balsam pentru ranile lui, ca poate se va vindeca!
\par 9 Am voit sa vindecam Babilonul, dar nu s-a vindecat! Lasa?i-l ?i haide?i sa mergem fiecare în ?ara noastra, pentru ca osânda lui s-a ridicat pâna la nori ?i a ajuns pâna la cer!
\par 10 Domnul a scos la lumina dreptatea noastra! Veni?i sa vestim în Sion fapta Domnului Dumnezeului nostru!
\par 11 Ascu?i?i sage?ile ?i va umple?i tolbele! Ca Domnul a trezit duhul regilor Mediei, pentru ca vrea sa nimiceasca Babilonul. Aceasta este razbunarea Domnului, razbunarea pentru templul Sau.
\par 12 Ridica?i steagul împotriva zidurilor Babilonului, întari?i paza, pune?i strajeri, întinde?i curse, caci Domnul a facut un plan ?i a împlinit ceea ce a rostit împotriva locuitorilor Babilonului.
\par 13 O, tu, cel ce locuie?ti lânga apele cele mari ?i e?ti plin de comori, venit-a sfâr?itul tau ?i masura lacomiei tale ?i s-a umplut!
\par 14 Domnul Savaot S-a jurat pe Sine Însu?i ?i a zis: "Adevarul graiesc, ca te voi umple de oameni, ca de lacuste, ?i ei vor ridica strigat de biruin?a".
\par 15 El a facut pamântul cu puterea Sa, a întemeiat lumea cu în?elepciunea Sa ?i cu mintea Sa a întins cerurile.
\par 16 La glasul Lui freamata apele în ceruri ?i El ridica nori de la marginile pamântului; faure?te fulgerele în mijlocul ploii ?i scoate vânturile din vistieriile Sale.
\par 17 Tot omul ratace?te în ?tiin?a sa ?i orice argintar se ru?ineaza de idolul sau, caci chipurile turnate de el nu sunt decât minciuna; n-au nici o suflare în ele.
\par 18 Aceasta este de?ertaciune adevarata, fapta ratacita ?i în vremea pedepsirii lor vor pieri.
\par 19 Dar soarta lui Iacov nu e ca a lor, pentru ca Dumnezeul lui este facatorul a toate, ?i Israel este toiagul mo?tenirii Lui, ?i numele Lui este Domnul Savaot.
\par 20 "Ciocan Îmi e?ti tu ?i arma de razboi. Cu tine am lovit popoare ?i cu tine am stricat regate.
\par 21 Cu tine am lovit cal ?i calare? ?i cu tine am sfarâmat carul ?i pe conducatorul lui.
\par 22 Cu tine am lovit barbat ?i femeie, cu tine am batut tânar ?i batrân ?i cu tine am lovit fecior ?i fecioara.
\par 23 Cu tine am lovit pastor ?i turma, cu tine am batut pe plugar ?i boii lui ?i cu tine am lovit pe mai-marii ?inuturilor ?i pe capeteniile ceta?ilor.
\par 24 ?i voi rasplati Babilonului ?i tuturor locuitorilor Caldeii pentru toate relele ce au facut ei Sionului sub ochii vo?tri", zice Domnul.
\par 25 "Iata, Eu sunt împotriva ta, munte al nimicirii, care ai pustiit tot pamântul; voi întinde împotriva ta mâna Mea, te voi arunca jos de pe stânci ?i te voi face munte dogorit de soare, zice Domnul!
\par 26 Nimeni nu va lua din tine pietre de pus în capul unghiului, nici pietre de temelie, ci ve?nic vei fi pustiu", zice Domnul.
\par 27 "Ridica?i steag pe pamânt, trâmbi?a?i cu trâmbi?a printre neamuri, pregati?i neamurile împotriva lui, chema?i împotriva lui regatele: Ararat, Mini ?i A?chenaz, pune?i capetenie împotriva lui ?i aduce?i cai ca lacusta cea grozava.
\par 28 Înarma?i împotriva lui neamurile ?i pe regii Mediei, capeteniile provinciilor ei ?i pe toate capeteniile ceta?ilor ei ?i toata ?ara de sub stapânirea lor.
\par 29 Pamântul se cutremura ?i se zbuciuma, caci se împline?te împotriva Babilonului planul Domnului de a face pamântul Babilonului pustietate fara locuitori.
\par 30 Cei puternici ai Babilonului au încetat lupta, ?ed în întariturile lor; secatuitu-s-a puterea lor ?i au ajuns ca femeile; locuin?ele lor sunt arse ?i zavoarele sfarâmate.
\par 31 Sol dupa sol, vestitor dupa vestitor alearga sa dea de ?tire regelui Babilonului ca cetatea sa e cuprinsa din toate par?ile,
\par 32 Ca vadurile sunt luate, bal?ile cu stuf arse ?i osta?ii lovi?i de spaima".
\par 33 Ca a?a zice Domnul Savaot, Dumnezeul lui Israel: "Fiica Babilonului e asemenea unei arii în timpul treieratului: înca pu?in ?i vine vremea seceri?ului".
\par 34 "Mâncatu-m-a ?i m-a ros Nabucodonosor, regele Babilonului; facutu-m-a vas gol; înghi?itu-m-a ca un dragon; umplutu-?i-a pântecele cu bunata?ile mele ?i m-a aruncat.
\par 35 Ocara mea ?i carnea mea cea sfâ?iata sa fie asupra Babilonului", sa zica ceea ce locuie?te în Sion - "?i sângele meu sa fie asupra locuitorilor Caldeii", sa zica Ierusalimul.
\par 36 De aceea, a?a zice Domnul: "Iata Eu iau apararea pricinii tale ?i te voi razbuna ?i voi seca marea lui ?i canalele lui le voi usca.
\par 37 Babilonul va fi o movila de darâmaturi, adapost pentru ?acali, groaza ?i batjocura fara locuitori.
\par 38 Ei to?i vor mugi ca ni?te lei ?i vor mârâi ca ni?te pui de leu.
\par 39 În vremea aprinderii lor le voi face ospa? ?i-i voi adapa, ca sa se veseleasca ?i sa doarma somnul de veci ?i sa nu se mai trezeasca, zice Domnul.
\par 40 Îi voi coborî ca pe ni?te miei la junghiere, ca pe ni?te berbeci ?i ?api.
\par 41 Cum a fost luat ?isac! Cum a fost cucerit acela a carui slava umplea tot pamântul! Cum a ajuns Babilonul spaima printre neamuri!
\par 42 S-a ridicat marea împotriva Babilonului ?i acesta e acoperit de mul?imea valurilor.
\par 43 Ceta?ile lui au ajuns pustii, pamântul lui sec, ?inut unde nu locuie?te nici un om ?i pe unde nu trece fiu de om.
\par 44 Voi pedepsi pe Bel în Babilon ?i voi smulge din gura lui cele înghi?ite de el; popoarele nu se vor mai îngramadi spre el de acum înainte. Dar zidurile Babilonului au ?i cazut.
\par 45 Ie?i din mijlocul lor, poporul Meu, ?i fiecare sa-?i scape via?a de flacara mâniei Domnului.
\par 46 Sa nu slabeasca inima voastra ?i sa nu va teme?i de zvonul care se va auzi în ?ara; caci anul acesta va veni un zvon, iar în anul urmator altul ?i pe pamânt va fi silnicie; tiran peste tiran se va scula.
\par 47 De aceea iata vin zile când voi pedepsi pe idolii Babilonului ?i toata ?ara lui va fi ru?inata, to?i cei lovi?i ai lui vor cadea în mijlocul lui.
\par 48 Atunci cerul ?i pamântul ?i tot ce cuprind ele vor striga de bucurie împotriva Babilonului, caci vin asupra lui pustiitorii de la miazanoapte, zice Domnul.
\par 49 Cum Babilonul a dobândit ?i a zdrobit pe Israeli?i, a?a vor fi doborâ?i ?i zdrobi?i în Babilon locuitorii pamântului lui.
\par 50 Cei scapa?i de sabie, pleca?i, nu va opri?i, aduce?i-va aminte din departare de Domnul ?i sa-?i gaseasca Ierusalimul loc în inima voastra.
\par 51 Ne ru?inam când auzeam ocara, ?i necinstea acoperea obrazul nostru, când strainii au venit în locul cel sfânt al templului Domnului;
\par 52 Dar în schimb iata vin zile, zice Domnul, când voi pedepsi idolii lui ?i tot pamântul lui va fi plin de gemetele celor ce sunt uci?i.
\par 53 De s-ar ridica Babilonul pâna la cer ?i de ?i-ar întari întru înal?ime cetatea sa, tot vor veni din porunca Mea pustiitorii, zice Domnul.
\par 54 Asculta?i ?ipatul care se ridica din Babilon ?i uria?a trosnitura din Caldeea!
\par 55 Ca Domnul va pustii Babilonul ?i va pune capat glasului celui mândru al lui. Suna-vor valurile lor, ca apele cele mari, rasuna-va glasul lor.
\par 56 Caci pustiitorul va veni asupra lui, asupra Babilonului, ?i aparatorii lui vor fi lua?i ?i arcurile lor vor fi sfarâmate, ca Domnul Dumnezeul rasplatirilor va da fiecaruia plata cuvenita.
\par 57 Voi îmbata pe mai-marii lui ?i pe în?elep?ii iui, pe capeteniile ?inuturilor lui ?i pe capeteniile ceta?ilor lui ?i pe osta?ii lui ?i vor dormi somnul de veci ?i nu se vor mai trezi, zice Domnul, al Carui nume e Domnul Savaot".
\par 58 A?a zice Domnul Savaot: "Zidurile cele groase ale Babilonului le voi darâma pâna la temelie ?i por?ile lui cele înalte vor fi arse cu foc. A?adar în de?ert s-au trudit popoarele ?i neamurile au muncit pentru foc".
\par 59 Cuvântul pe care proorocul Ieremia l-a încredin?at lui Seraia, fiului Neria, fiul lui Maaseia, când acesta a plecat la Babilon cu Sedechia, regele lui Iuda, în anul al patrulea al domniei lui; Seraia era mare camara?.
\par 60 Atunci a scris Ieremia într-o carte toate nenorocirile care trebuia sa vina asupra Babilonului.
\par 61 ?i a zis Ieremia catre Seraia: "Când vei ajunge la Babilon, cauta sa cite?ti toate cuvintele acestea ?i sa zici:
\par 62 "Doamne, Tu ai grait de locul acesta ca-l vei pierde a?a încât sa nu ramâna într-însul nici om, nici animal, ci sa fie pustietate ve?nica".
\par 63 ?i dupa ce vei ispravi de citit cartea aceasta, leaga o piatra de ea ?i arunc-o în mijlocul Eufratului ?i zi:
\par 64 "A?a se va cufunda Babilonul ?i nu se va mai ridica din acea nenorocire pe care o voi aduce asupra lui ?i se va istovi". Pâna aici este vorbirea lui Ieremia.

\chapter{52}

\par 1 Sedechia era de douazeci ?i unu de ani când a început sa domneasca ?i a domnit în Ierusalim unsprezece ani. Numele mamei sale era Hamutal, fata lui Ieremia din Libna.
\par 2 El a facut rele în ochii Domnului, precum facuse ?i Ioiachim.
\par 3 De aceea a venit mânia Domnului asupra Ierusalimului ?i a lui Iuda pâna într-atât încât i-a lepadat de la fa?a Sa, ?i Sedechia a fost dat jos de regele Babilonului.
\par 4 Era în al noualea an al domniei lui, în luna a zecea, în ziua a zecea a lunii acesteia, când a venit Nabucodonosor, regele Babilonului, cu toata o?tirea sa împotriva Ierusalimului, l-a înconjurat ?i a facut împrejurul lui valuri de pamânt.
\par 5 Cetatea a stat împresurata pâna în anul al unsprezecelea al regelui Sedechia.
\par 6 Iar în luna a patra, în ziua a noua a lunii acesteia, s-a întarit foametea în cetate, ?i poporul ?arii nu mai avea pâine.
\par 7 Atunci s-a facut o spartura în cetate, pe unde au ie?it to?i o?tenii ?i au fugit din cetate noaptea pe por?ile ce se aflau între cele doua ziduri de lânga gradina regelui; iar Caldeii erau împrejurul ceta?ii.
\par 8 ?i a alergat o?tirea Caldeilor dupa rege ?i a ajuns pe Sedechia în ?esurile Ierihonului. Atunci toata o?tirea lui a fugit de la el.
\par 9 Deci au luat pe rege ?i l-au dus la regele Babilonului în Ribla, în ?inutul Hamat, unde acesta l-a judecat.
\par 10 Regele Babilonului a junghiat pe fiii lui Sedechia înaintea ochilor acestuia; a junghiat de asemenea în Ribla ?i pe to?i cei mari din Iuda.
\par 11 Iar lui Sedechia i-a scos ochii ?i a poruncit sa-l încatu?eze cu catu?e de arama; ?i l-a dus regele Babilonului la Babilon ?i l-a pus în casa cea de paza, unde l-a ?inut pâna în ziua mor?ii lui.
\par 12 în luna a cincea, în ziua a zecea a lunii acesteia, în anul al nouasprezecelea al regelui Nabucodonosor, regele Babilonului, a venit Nebuzaradan, capetenia garzii, care statea înaintea regelui Babilonului, la Ierusalim ?i a ars templul Domnului,
\par 13 Casa regelui ?i toate casele din Ierusalim; toate casele cele mari le-a ars cu foc;
\par 14 Iar o?tirea Caldeilor, care era cu capetenia garzii, a darâmat toate zidurile dimprejurul Ierusalimului.
\par 15 Nebuzaradan, capetenia garzii, a stramutat pe cei saraci din popor ?i tot poporul care ramasese în cetate ?i pe cei care se predasera regelui Babilonului ?i toata rama?i?a de popor.
\par 16 ?i numai pu?ini din poporul sarac al ?arii au fost lasa?i de Nebuzaradan, capetenia garzii, ca lucratori pentru vii ?i ogoare.
\par 17 Caldeii au sfarâmat stâlpii cei de arama, care se aflau în templul Domnului, postamentele ?i marea de arama care se afla în templul Domnului, ?i toata arama lor au dus-o la Babilon.
\par 18 Au luat lighenele, lope?ile, cu?itele ?i castroanele, lingurile ?i toate vasele de arama, care erau întrebuin?ate la slujbele dumnezeie?ti;
\par 19 ?i capetenia garzii a mai luat vasele ?i cle?tele, cazanele ?i candelele, ca?uile ?i cupele - tot ce era de aur ?i ce era de argint.
\par 20 De asemenea au luat cei doi stâlpi, marea ?i cei doisprezece boi de arama, care serveau de postament ?i pe care regele Solomon îi facuse pentru templul Domnului. În toate acestea era atâta arama, cât nu se putea cântari.
\par 21 Fiecare stâlp din ace?tia era de optsprezece co?i în înal?ime ?i o sfoara de doisprezece co?i îl putea cuprinde împrejur, iar grosimea pere?ilor lor era de patru degete, caci înauntru nu erau plini.
\par 22 Coroana unui stâlp era de arama ?i înal?imea ei era de cinci co?i. ?i pere?ii ei ?i rodiile dimprejur erau toate de arama. Asemenea coroana cu rodii era ?i la celalalt stâlp.
\par 23 De jur împrejur erau nouazeci ?i ?ase de rodii; tot a?a ?i la celalalt stâlp împrejurul coroanei lui erau rodii.
\par 24 Capetenia garzii a luat ?i pe Seraia arhiereul ?i pe Sofonie, preotul al doilea, ?i pe trei strajeri ai pragurilor.
\par 25 Din cetate a luat un eunuc, care era capetenie peste o?tiri, ?i ?apte oameni, care stateau înaintea regelui ?i care se aflau în cetate; au mai luat pe secretarul capeteniei o?tirii, care înscrisese la o?tire pe poporul ?arii, precum ?i ?aizeci de oameni din poporul ?arii, care s-au gasit în cetate.
\par 26 Pe ace?tia i-a luat Nebuzaradan, capetenia garzii, ?i i-a dus la Nabucodonosor, regele Babilonului, în Ribla.
\par 27 ?i i a lovit pe ei regele Babilonului ?i i-a omorât în Ribla cea din ?ara Hamat. A?a a fost stramutat Iuda din ?ara sa.
\par 28 Iata acum poporul pe care l-a stramutat Nabucodonosor: în luna a ?aptea, trei mii douazeci ?i trei de oameni;
\par 29 În al optsprezecelea an al lui Nabucodonosor au fost stramuta?i din Ierusalim opt sute treizeci ?i doua de suflete;
\par 30 În anul al douazeci ?i treilea al lui Nabucodonosor, Nebuzaradan, capetenia garzii, a stramutat din Iudei ?apte sute patruzeci ?i cinci de suflete: în total patru mii ?ase sute de suflete.
\par 31 În anul al treizeci ?i ?aptelea dupa stramutarea lui Ioiachim, regele lui Iuda, în luna a douasprezecea, în douazeci ?i cinci ale lunii, Evil-Merodac, regele Babilonului, în anul întâi al domniei lui, s-a îndurat de Ioiachim, regele lui Iuda, ?i l-a scos din închisoare;
\par 32 A vorbit cu el prietene?te ?i a pus scaunul lui mai sus decât al altor regi care erau la el în Babilon.
\par 33 A schimbat hainele lui de închisoare ?i Ioiachim a mâncat întotdeauna la masa regelui în toate zilele lui.
\par 34 Hrana lui i s-a dat de la rege zilnic, pâna la moartea sa, în toate zilele vie?ii sale.


\end{document}