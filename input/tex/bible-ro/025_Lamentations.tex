\begin{document}

\title{Plângerile lui Ieremia}


\chapter{1}

\par 1 O, cum a ramas pustie cetatea cea cu mult popor! Cum a ajuns ca o vaduva cea mai de frunte dintre neamuri; doamna ceta?ilor a ajuns birnica.
\par 2 Noaptea plânge întruna cu lacrimi pe obraz ?i dintre to?i câ?i o iubeau, nici unul n-o mai mângâie; to?i prietenii au devenit du?mani.
\par 3 Iuda s-a dus în robie, la suferin?a ?i la munca grea; sala?luie?te printre neamuri ?i nu-?i afla odihna. To?i asupritorii lui l-au prins la strâmtorare.
\par 4 Toate caile Sionului sunt pline de jale ?i nimeni nu mai vine la sarbatoare. Toate por?ile (ceta?ii) sunt pustii, preo?ii ei suspina; fecioarele sunt deznadajduite ?i ea este plina de amar.
\par 5 Vrajma?ii ei sunt biruitori, du?manii ei sunt cu voie buna; caci Domnul a umilit-o din pricina multelor ei pacate, iar feciorii ei au plecat în robie înaintea asupritorului.
\par 6 A?a ?i-a irosit fiica Sionului toata stralucirea! Capeteniile ei sunt asemenea cerbilor care nu afla pa?une ?i fug slei?i de puteri dinaintea urmaritorului.
\par 7 Ierusalimul î?i aduce aminte de zilele ticalo?iei lui ?i ale ratacirii lui, de toate stralucirile pe care le-a avut în stravechile vremuri. Acum însa, când poporul a cazut în mâna vrajma?ului ?i când nimeni nu-i poate veni în ajutor, du?manii lui se uita la el ?i râd de prabu?irea lui.
\par 8 Ierusalimul a pacatuit de moarte, pentru aceasta a ajuns de spaima; to?i cei ce-l cinsteau nu-l mai iau în seama, caci au vazut goliciunea lui, iar el suspina ?i î?i întoarce fa?a.
\par 9 Necura?ia lui e lipita de poala hainelor lui caci la sfâr?itul lui el nu s-a gândit. El s-a prabu?it în chip uluitor ?i n-are pe nimeni sa-l mângâie! "Vezi, Doamne, necazul meu, caci vrajma?ul biruie?te".
\par 10 Du?manii au întins mâna spre toate vistieriile lui. El a vazut neamuri intrând în templul sau, neamuri carora le-ai dat porunca: "Sa nu intre în ob?tea ta!"
\par 11 Tot poporul Tau suspina cautând pâine, ?i î?i dau odoarele lor pentru mâncare, ca sa-?i ?ina via?a. Vezi, Doamne, ?i ia aminte cum am ajuns de ocara!
\par 12 O, voi trecatorilor, privi?i ?i vede?i daca este vreo durere ca aceea care ma cople?e?te ?i cu care Domnul m-a umplut de necaz în ziua întaririi mâniei Lui.
\par 13 Foc a trimis de sus peste oasele mele ?i m-a smerit, picioarelor mele le-a întins cursa, ?i m-a facut sa dau înapoi; pustiitu-m-a cu totul, iar eu toata ziua bolesc.
\par 14 Jugul pacatelor mele mi-a fost legat de gât de catre mâna Lui; strânse ca într-un manunchi, ele atârna de grumazul meu; El a facut sa se destrame puterea mea ?i m-a dat în mâna celor carora nu puteam sa ma împotrivesc.
\par 15 Domnul a spulberat pe to?i voinicii din mijlocul meu, El a chemat oaste împotriva mea, ca sa sfarâme pe voinicii mei. Stapânul a toate a strivit ca în teasc pe fecioara, fiica lui Iuda.
\par 16 Pentru aceasta eu plâng mereu, din ochii mei izvorasc lacrimi, caci departe de mine este Mângâietorul, Cel ce-mi îmbarbata inima. Feciorii mei cu to?ii au fost da?i pieirii, caci du?manul a avut biruin?a.
\par 17 Sionul întinde mâinile sale ?i nimeni nu-l mângâie! Domnul a dat porunca tuturor vrajma?ilor lui Iacov ca sa-l împresoare. Ajuns-a Ierusalimul înaintea ochilor lor ca un lucru spurcat.
\par 18 Drept este Domnul, caci împotriva poruncilor Lui m-am razvratit. Lua?i aminte, voi, toate popoarele, ?i vede?i necazul meu: fecioarele mele ?i flacaii mei au fost du?i în robie.
\par 19 Strigat-am catre iubi?ii mei, dar ei m-au în?elat; preo?ii mei ?i batrânii mei au pierit în cetate, când cautau hrana ca sa-?i ?ina via?a.
\par 20 Vezi, Doamne, cât sunt de strâmtorat, launtrul meu arde! Inima mea se zbuciuma în trupul meu, pentru ca m-am razvratit foarte. Afara sabia secera pe feciorii mei, iar înauntru, moartea.
\par 21 To?i aud suspinul meu, dar nimeni nu ma mângâie! to?i du?manii, aflând de nenorocirea mea, se bucura ca ai facut a?a. Sa vina peste ei ziua pe care ai fagaduit-o ?i sa ajunga ?i ei ca mine!
\par 22 Toata faradelegea lor sa vina înaintea Ta ?i sa le faci lor precum mi-ai facut mie, pentru toate pacatele mele! Caci suspinele mele sunt fara de numar, iar inima mea bole?te!

\chapter{2}

\par 1 O, cum a acoperit cu nori Domnul întru mânia Lui pe fiica Sionului! Din cer a aruncat pe pamânt mare?ia lui Israel ?i în ziua mâniei Sale nu ?i-a adus aminte de a?ternutul picioarelor Sale.
\par 2 Domnul a nimicit fara mila toate sala?ele lui Iacov; întru întarâtarea urgiei Lui a doborât la pamânt întariturile fiicei lui Iuda; le-a facut una cu pamântul, a pângarit regatul ?i capeteniile lui.
\par 3 Întru aprinderea mâniei Lui a zdrobit toata puterea lui Israel; înaintea du?manului ?i-a tras dreapta înapoi. El a aprins pe Iacov cu un foc arzator care prapade?te de jur împrejur.
\par 4 El a încordat arcul Sau ca un du?man, dreapta Sa a stat gata ca a unui vrajma? ?i a ucis tot ce desfata ochiul în cortul fiicei Sionului; varsat-a ca un foc mânia Lui.
\par 5 Stapânul S-a aratat ca un du?man nimicind Israelul; i-a darâmat toate palatele ?i ceta?ile întarite ?i asupra fiicei lui Iuda a adus multa suparare.
\par 6 Prabu?it-a la pamânt ca pe o dumbrava cortul lui, stricat-a locul de sarbatoare. Domnul a facut sa se uite zilele de odihna în Sion, dispre?uind, în vapaia mâniei Lui, pe rege ?i pe preot.
\par 7 Dispre?uit-a Domnul jertfelnicul Sau ?i S-a îndepartat de loca?ul Sau cel sfânt; dat-a zidurile palatelor Sale în mâna du?manilor care au strigat în templul Domnului ca în zilele de sarbatoare.
\par 8 Gasit-a Domnul cu cale sa surpe zidurile fiicei Sionului; întins-a funia ?i n-a tras înapoi mâna Sa, pâna nu le-a nimicit. El a întins jalea peste ziduri ?i întarituri, ce stau laolalta darapanate.
\par 9 Por?ile lui s-au afundat în pamânt, zavoarele lor El le-a sfarâmat; regele lui ?i capeteniile sunt pribegi printre neamuri. Lege nu mai au, chiar ?i profe?ii nu mai primesc vedenii de la Domnul.
\par 10 Stau la pamânt ?i tac batrânii fiicei Sionului; pe capul lor ?i-au pus ?arâna ?i s-au încins cu sac; la pamânt î?i pleaca fecioarele Ierusalimului capetele lor.
\par 11 Ochii mei se sfâr?esc de plâns, launtrul meu arde ca vapaia, maruntaiele mele fierb ?i fierea mi se varsa pe pamânt din pricina zdrobirii fiicei neamului meu, când copiii ?i pruncii stau slei?i de putere în pie?ele ceta?ii,
\par 12 Zicând mamei lor: "Unde este pâine, unde este vin?" Ei cad slei?i de putere ca doborâ?i de sabie pe pie?ele ceta?ii ?i î?i dau sufletul la sânul maicii lor.
\par 13 Cu cine te voi asemana, cu cine a? putea sa te pun alaturi, o, fiica a Ierusalimului! Cu cine te-a? pune fa?a în fa?a, ca sa te pot mângâia, o, fecioara, fiica a Sionului? Caci ne?armurita ca marea este naruirea ta! Cine ar putea sa te tamaduiasca?
\par 14 Profe?ii tai au avut pentru tine vedenii zadarnice ?i aratari ?i nu ?i-au dat pe fa?a faradelegea ta, ca sa-?i schimbe calea ta, ci ?i-au aratat vedenii în?elatoare ?i aducatoare de pieire.
\par 15 Bat spre tine din palme to?i cei ce trec pe cale, fluiera ?i clatina din cap pentru fiica Ierusalimului: "Aceasta este, oare, cetatea pe care o numeau cununa frumuse?ii, bucuria a tot pamântul?"
\par 16 Catre tine to?i vrajma?ii tai casca gura lor, fluiera, scrâ?nesc din din?i, zicând: "Am nimicit-o. Da, aceasta este ziua pe care noi o a?teptam; am aflat-o ?i o vedem".
\par 17 Împlinit-a Domnul hotarârea Sa, adus-a la îndeplinire cuvântul Sau, spus din zilele stravechi; prabu?it-a fara mila, bucurat-a pe vrajma?ul tau, înal?at-a puterea apasatorilor tai.
\par 18 Striga catre Domnul, tu fecioara, fiica a Sionului! Sa curga lacrimile tale ca un ?uvoi, zi ?i noapte; nu înceta, ochiul tau sa nu zaboveasca!
\par 19 Scoala, jele?te-te în timpul nop?ii, la început do straja; varsa-?i inima ta ca apa înaintea fe?ei Celui Atotstapânitor. Catre El ridica mâna ta pentru via?a pruncilor tai, care se prapadesc de foame la col?ul tuturor uli?elor.
\par 20 Vezi, o, Doamne, ?i prive?te cui ai facut aceasta! Sa manânce femeile rodul pântecelui lor, copiii pe care îi poarta în bra?e? Sa fie uci?i în templu, Doamne, preotul ?i profetul?
\par 21 Stau culca?i la pamânt pe uli?e tânar ?i batrân. Fecioarele ?i flacaii mei de sabie au cazut; Tu i-ai ucis în ziua mâniei Tale, jertfitu-i-ai fara de mila.
\par 22 Chemat-ai ca la sarbatoare pe to?i cei ce au sala? în jurul meu. ?i în ziua mâniei Domnului n-a scapat, nici n-a ramas vreunul; pe cei care i-am purtat în bra?e ?i i-am facut mari, mi i-a nimicit du?manul.

\chapter{3}

\par 1 Eu sunt omul care am vazut nenorocirea sub varga aprinderii Lui.
\par 2 El m-a purtat ?i m-a dus în întuneric ?i în bezna.
\par 3 Da, împotriva mea întoarce ?i iar întoarce în toata vremea mâna Sa.
\par 4 Mistuit-a trupul meu ?i pielea mea, zdrobit-a oasele mele;
\par 5 A ridicat zid împotriva mea ?i m-a înconjurat de venin ?i de zbucium,
\par 6 Mutându-ma în împara?ia morii, ca pe mor?ii cei din veac.
\par 7 M-a împrejmuit cu zid ?i n-am pe unde sa ies, îngreuiat-a lan?urile mele;
\par 8 Chiar când strig ?i racnesc, rugaciunea mea nu se aude;
\par 9 El a astupat cararile mele cu piatra ?i a întortochiat potecile mele.
\par 10 El a ajuns pentru mine ca un urs la pânda, ca un leu în ascunzatoare.
\par 11 Ratacit-a caile mele, m-a sfâ?iat ?i m-a pustiit;
\par 12 A încordat arcul Sau ?i m-a a?ezat ca ?inta sage?ii Sale,
\par 13 Trimi?ând în rarunchii mei pe fiii tolbei Sale.
\par 14 Facutu-m-am de râs fa?a de poporul meu, cântecul lor de batjocura în fiecare zi.
\par 15 El m-a saturat de amaraciuni, adapatu-m-a cu pelin.
\par 16 A zdrobit de piatra din?ii mei ?i m-a afundat în cenu?a.
\par 17 Tu ai rapit pacea sufletului meu, uitat-am fericirea
\par 18 ?i am zis: "S-a dus puterea vie?ii mele ?i nadejdea mea în Domnul".
\par 19 Adu-?i aminte de nevoia ?i necazul meu, de pelin ?i otrava!
\par 20 Sa-?i aduci aminte ca împovarat este în mine sufletul meu.
\par 21 Aceasta voi pune-o la inima, de aceea voi nadajdui:
\par 22 Milele Domnului nu s-au sfâr?it, milostivirile Lui nu înceteaza.
\par 23 În fiecare diminea?a sunt altele, credincio?ia Ta este mare!
\par 24 "Partea mea este Domnul, a zis sufletul meu, de aceea voi nadajdui în El".
\par 25 Bun este Domnul cu cei ce se încred în El, pentru omul care Îl cauta.
\par 26 Bine este sa a?tep?i în tacere ajutorul Domnului.
\par 27 Bine este omului sa poarte un jug din tinere?ile lui.
\par 28 Sa stea la o parte în tacere, daca Domnul îi da porunca!
\par 29 Sa atinga pulberea cu buzele lui; poate mai este nadejde!
\par 30 Sa dea obrazul lui spre lovire ?i sa se sature de ocara!
\par 31 Caci Domnul nu arunca pe oameni pentru totdeauna;
\par 32 Ci El pedepse?te ?i are mila dupa mul?imea milelor Lui.
\par 33 Ca nu de buna voie umile?te ?i pedepse?te pe fiii oamenilor.
\par 34 Când calcam în picioare pe to?i robii pamântului,
\par 35 Când calcam dreptatea omului înaintea fe?ei Celui Preaînalt,
\par 36 Când nu dam dreptate cuiva în pricina lui, oare Stapânul a toate nu vede?
\par 37 Cine este Cel ce a grait ?i s-a facut, fara numai Domnul, Care a poruncit?
\par 38 Nu iese oare din gura Celui Preaînalt binele ?i raul?
\par 39 De ce suspina omul toata via?a, fiecare pentru pacatul lui?
\par 40 Sa cercetam caile noastre, luând aminte ?i întorcându-ne la Domnul!
\par 41 Sa ridicam inimile ?i mâinile noastre la Domnul din cer!
\par 42 Noi am pacatuit ?i ne-am razvratit ?i Tu ne-ai iertat.
\par 43 Tu Te-ai învesmântat cu mânie ?i ne-ai urmarit; Tu ai ucis fara mila;
\par 44 Tu Te-ai ascuns în nori, ca sa nu strabata rugaciunea la Tine;
\par 45 Tu ai facut din mine o maturatura ?i un gunoi, în mijlocul popoarelor.
\par 46 To?i du?manii no?tri au deschis gura împotriva noastra;
\par 47 De spaima ?i de groapa am avut parte, de pustiire ?i de ruina.
\par 48 ?uvoaie de apa lacrimeaza ochiul meu, din pricina prapadului fiicei poporului meu.
\par 49 Ochiul meu varsa lacrimi fara încetare, caci nu este u?urare,
\par 50 Pâna sa se uite în jos ?i sa priveasca Domnul din ceruri.
\par 51 Ochiul meu ma doare din pricina fiicelor ceta?ii mele.
\par 52 Ca pe o pasare m-au vânat fara cuvânt vrajma?ii mei,
\par 53 Au vrut sa nimiceasca în groapa via?a mea, ?i au aruncat cu pietre în mine.
\par 54 Ape navaleau peste capul meu ?i cugetam: "Sunt pierdut!"
\par 55 Chemat-am numele Tau, Doamne, din groapa cea mai dedesubt.
\par 56 Tu ai auzit glasul meu: "Nu astupa urechea la suspinul ?i strigatul meu".
\par 57 Tu erai aproape în ziua când Te-am strigat ?i ai zis: "Nu-?i fie frica!"
\par 58 O, Doamne, Tu ai judecat pricina mea, Tu ai izbavit via?a mea!
\par 59 Vazut-ai, Doamne, apasarea mea, ajuta-mi ?i-mi fa dreptate!
\par 60 Tu ai vazut toata razbunarea lor, toate uneltirile lor împotriva mea;
\par 61 O, Doamne, Tu ai auzit ocarile lor, toate chibzuielile lor împotriva mea,
\par 62 Graiurile potrivnicilor mei ?i gândul lor ascuns împotriva mea.
\par 63 Prive?te: de stau sau de se scoala, eu sunt de râsul lor!
\par 64 Rasplate?te-le, Doamne, dupa faptele mâinilor lor,
\par 65 Da-le învârto?are inimii, blestemul Tau sa fie pentru ei!
\par 66 Urmare?te-i cu mânie ?i nimice?te-i sub cerurile Tale, Doamne!

\chapter{4}

\par 1 O, cum s-a întunecat aurul, ?i cel mai curat aur ?i-a schimbat fa?a; pietrele nestemate varsate au fost la col?ul tuturor uli?elor!
\par 2 Feciorii Sionului, cei mai de seama altadata, cântari?i cu aur, cum au ajuns sa fie socoti?i ca vasele de lut, lucru de mâna de olar!
\par 3 Chiar ?i ?acalii î?i dau sânul, ca puii lor sa suga, dar fiica poporului meu ajuns-a cruda, ca stru?ii în pustiu.
\par 4 Din pricina setei lipitu-s-a limba sugaciului de cerul gurii lui; copiii cer pâine, dar nimeni nu le-o întinde.
\par 5 Cei care mâncau odinioara mâncaruri alese cad de foame pe uli?e; cei care au fost crescu?i în purpura stau trânti?i în gunoi.
\par 6 Vina fiicei poporului meu a fost mai mare decât a Sodomei, prabu?ita într-o clipa, nu de mâna omeneasca.
\par 7 Capeteniile ei erau mai stralucitoare decât zapada, mai albe decât laptele; trupul lor era mai ro?u decât margeanul, ca safirul era înfa?i?area lor.
\par 8 Chipul lor a ajuns mai negru decât funinginea, pe uli?e nu-i po?i cunoa?te; pielea lor s-a zbârcit pe oase, s-a uscat ca o a?chie de lemn.
\par 9 Mai ferici?i au fost cei care au cazut de sabie, decât cei mor?i de foame, care se prapadesc încet, doborâ?i de lipsa roadelor de pe câmp.
\par 10 Femeile, de?i miloase, au fiert cu mâinile lor copiii ?i i-au mâncat în vremea caderii fiicei poporului meu.
\par 11 Sfâr?it-a Domnul mânia, varsat-a pe deplin urgia aprinderii Lui; ?i în Sion a aprins un fac care l-a mistuit.
\par 12 Nici n-ar fi putut sa creada regii pamântului ?i to?i locuitorii lumii ca vrajma?ul ?i apasatorul ar putea sa intre pe por?ile Ierusalimului!
\par 13 Dar s-a întâmplat, din pricina pacatelor profe?ilor (mincino?i) ?i a faradelegilor preo?ilor, care au varsat în mijlocul lui sângele celor drep?i.
\par 14 Pe uli?e rataceau pata?i de sânge, ?i nimeni nu se atingea de hainele lor.
\par 15 "Pazi?i-va! Un necurat!" Striga lumea dupa ei. "Fugi?i, la o parte, nu-i atinge?i!" ?i daca mai voiesc sa rataceasca undeva - se zicea printre neamuri - n-ar trebui sa ramâna aici!
\par 16 Fa?a plina de mânie a Domnului i-a risipit pe ei. Pe preo?i nimeni nu-i mai lua în seama, de batrâni nu se îndura.
\par 17 ?i ochii no?tri se sting de suparare, a?teptând zadarnic un ajutor! Din turnul nostru ne-am uitat departe spre un popor al carui ajutor nu vine.
\par 18 ?i pândeau pa?ii no?tri ca sa nu umblam prin pie?ele noastre. Sfâr?itul nostru se apropia, sosise!
\par 19 Prigonitorii no?tri erau mai iu?i decât vulturii de pe cer; umblau dupa noi prin mun?i, ne pândeau în pustiu.
\par 20 Suflarea vie?ii noastre, unsul Domnului, a fost prins în groapa lor - acela despre care noi ziceam: "La umbra lui vom vie?ui printre popoare".
\par 21 Bucura-te ?i te vesele?te, fiica Edomului, tu care locuie?ti în pamântul U?; ?i la tine va veni cupa; vei bea ?i te vei lasa goala.
\par 22 Faradelegea ta, o, fiica a Sionului, s-a sfâr?it; la fel robia. Dar ?ie î?i cerceteaza pacatele, o, fiica a Edomului, ?i da pe fa?a faradelegile tale!

\chapter{5}

\par 1 Adu-?i aminte, Doamne, de cele întâmplate, ?i vezi ocara noastra!
\par 2 Mo?tenirea ?i casele noastre au cazut în mâna celor straini, de alt neam.
\par 3 Am ajuns orfani, fara de tata, mamele noastre sunt vaduve.
\par 4 Bem apa noastra cu bani, lemnele noastre le primim cu plata.
\par 5 Pe grumajii no?tri stau prigonitorii ?i, de?i n-avem puteri, nu ne dau ragaz.
\par 6 Întindem mâna catre Egipt ?i Asiria ca sa ne sature de pâine.
\par 7 Parin?ii no?tri au gre?it ?i nu mai sunt, dar noi purtam faradelegile lor.
\par 8 Slugi ne stapânesc ?i nimeni nu vine sa ne scoata din mina lor.
\par 9 Cu primejdia vie?ii noastre ne agonisim pâinea, în fa?a sabiei care ne amenin?a în pustiu.
\par 10 Pielea noastra s-a înnegrit ca un cuptor de vapaia foametei.
\par 11 Ei au batjocorit femeile în Sion, fecioarele din ceta?ile lui Iuda.
\par 12 Capeteniile au fost spânzurate de mâna lor, fe?ele batrânilor nu au mai fost luate în seama.
\par 13 Flacaii au învârtit la râ?ni?a ?i tinerii s-au poticnit carând lemne.
\par 14 Batrânii nu mai stau la poarta, cei tineri nu mai cânta din alaute.
\par 15 S-a dus veselia inimii noastre, jocul nostru s-a schimbat în plâns.
\par 16 Cazut-a cununa de pe capul nostru; vai noua, ca am pacatuit!
\par 17 Pentru aceasta inima noastra tânje?te ?i ochii s-au întunecat.
\par 18 Muntele Sionului a ramas pustiu ?i pe el se plimba vulpile.
\par 19 Tu, Doamne, împara?e?ti în veci ?i scaunul Tau în neam de neam!
\par 20 Pentru ce vrei sa ne ui?i, sa ne parase?ti atât de multa vreme?
\par 21 Întoarce-Te catre noi ?i ne vom întoarce; înnoie?te zilele noastre ca în vremea cea de demult!
\par 22 Sau Tu ne-ai urgisit ?i Te-ai mâniat pe noi, fara masura?


\end{document}