\begin{document}

\title{Ezekiel}

Eze 1:1  În anul al treizecilea, în ziua a cincea a lunii a patra, când ma aflam între robi, la râul Chebar, mi s-au deschis cerurile ?i am vazut ni?te vedenii dumnezeie?ti.
Eze 1:2  În ziua a cincea a lunii a patra, în anul al cincilea ai robirii regelui Ioiachim,
Eze 1:3  A fost cuvântul Domnului catre mine, preotul Iezechiel, fiul lui Buzi, preotul, la râul Chebar, în ?ara Caldeilor. Acolo a fost peste mine mâna Domnului.
Eze 1:4  Eu priveam ?i iata venea dinspre miazanoapte un vânt vijelios, un nor mare ?i un val de foc, care raspândea în toate par?ile raze stralucitoare; iar în mijlocul focului stralucea ca un metal în vapaie.
Eze 1:5  ?i în mijloc am vazut ceva ca patru fiare, a caror înfa?i?are semana cu chipul omenesc.
Eze 1:6  Fiecare din ele avea patru fe?e ?i fiecare din ele avea patru aripi.
Eze 1:7  Picioarele lor erau drepte, iar copitele picioarelor lor erau cum sunt copitele picioarelor de vi?el ?i scânteiau ca arama stralucitoare, iar aripile lor erau sprintene.
Eze 1:8  De cele patru par?i ele aveau sub aripi mâini de om ?i toate patru î?i aveau fe?ele lor ?i aripile lor.
Eze 1:9  Aripile lor se atingeau una de alta, ?i când mergeau, fiarele nu se întorceau, ci fiecare mergea drept înainte.
Eze 1:10  Fe?ele lor? - Toate patru aveau câte o fa?a de om înainte, toate patru aveau câte o fa?a de leu la dreapta, toate patru aveau câte o fa?a de bou la stânga ?i toate patru mai aveau ?i câte o fa?a de vultur în spate.
Eze 1:11  Fe?ele lor ?i aripile lor erau despar?ite în partea de sus, ?i, la fiecare, doua din aripi erau întinse, iar doua le acopereau trupul.
Eze 1:12  Fiecare fiara mergea drept înainte ?i mergea încotro îi dadea duhul sa mearga ?i în mersul sau nu se întorcea.
Eze 1:13  Înfa?i?area acestor fiare se asemana cu înfa?i?area carbunilor aprin?i, cu înfa?i?area unor faclii aprinse; printre fiare curgea foc, iar din foc ?â?neau raze ?i fulgere.
Eze 1:14  Fiarele alergau înainte ?i înapoi iute ca fulgerul.
Eze 1:15  Când ma uitam eu la fiare, iata am vazut jos, lânga aceste fiare, câte o roata la fiecare din cele patru fe?e ale lor.
Eze 1:16  Aceste ro?i, dupa înfa?i?area lor, parca erau de crisolit, iar dupa faptura toate aveau aceea?i înfa?i?are. ?i dupa alcatuirea ?i dupa faptura lor ele parca erau vârâte una în alta.
Eze 1:17  Ele înaintau în toate cele patru par?i, ?i în timpul mersului nu se întorceau.
Eze 1:18  Obezile lor formau un cerc larg ?i de o înal?ime înfrico?atoare ?i aceste obezi la toate patru erau pline de ochi de jur împrejur.
Eze 1:19  Când mergeau fiarele, mergeau ?i ro?ile de lânga ele, ?i când se ridicau fiarele de la pamânt, se ridicau ?i ro?ile.
Eze 1:20  Ele mergeau încotro le da duhul sa mearga ?i ro?ile se ridicau împreuna, caci duh de via?a era ?i în ro?i.
Eze 1:21  Când mergeau acelea, mergeau ?i acestea, ?i când acelea se opreau, se opreau ?i acestea; iar când acelea se ridicau de la pamânt, atunci împreuna cu ele se ridicau ?i ro?ile, pentru ca duh de via?a era ?i în ro?i.
Eze 1:22  Deasupra capetelor fiarelor se vedea un fel de bolta, întinsa sus, deasupra capetelor lor, care semana cu cristalul cel mai curat;
Eze 1:23  Iar sub bolta aceasta erau întinse aripile fiarelor una spre alta, ?i fiecare fiara mai avea câte doua aripi, care le acopereau trupurile;
Eze 1:24  Când mergeau fiarele, auzeam fâlfâitul aripilor lor, ca un vuiet de ape mari, ca glasul Celui Atotputernic zgomot stra?nic, ca zgomotul dintr-un lagar osta?esc; iar când ele se opreau, î?i lasau aripile în jos.
Eze 1:25  Dupa ce fiarele se opreau ?i î?i lasau aripile în jos, zgomotul se auzea înca sub bolta ce se întindea deasupra capetelor lor.
Eze 1:26  Pe bolta de deasupra capetelor fiarelor era ceva care semana cu un tron ?i la înfa?i?are era ca piatra de safir; iar sus pe acest tron era ca un chip de om.
Eze 1:27  ?i am mai vazut ceva, ca un metal înro?it în foc, ca ni?te foc, sub care se afla acel chip de om ?i care lumina împrejur; de la coapsele acelui chip de om în sus ?i de la coapsele chipului aceluia în jos se vedea un fel de foc, un fel de lumina stralucitoare care-l împresura de jur împrejur.
Eze 1:28  Cum este curcubeul ce se afla pe cer la vreme de ploaie, a?a era înfa?i?area acelei lumini stralucitoare care-l înconjura. Astfel era chipul slavei Domnului. ?i când am vazut eu aceasta, am cazut cu fa?a la pamânt.
Eze 2:1  Atunci am auzit glasul Unuia care mi-a zis: "Fiul omului, scoala în picioare, ca am sa-?i vorbesc!"
Eze 2:2  ?i cum mi-a zis Acela vorbele acestea, a intrat Duhul în mine ?i m-a ridicat în picioare, ?i am ascultat pe Cel ce-mi vorbea.
Eze 2:3  Acela mi-a zis: "Fiul omului, am sa te trimit la fiii lui Israel, la ace?ti oameni neascultatori, care s-au razvratit împotriva Mea; ei, ca ?i parin?ii lor, pacatuiesc înaintea Mea pâna în ziua de astazi.
Eze 2:4  Ace?ti fii sunt neru?ina?i ?i cu inima împietrita; la ei te trimit Eu, ca sa le zici: A?a graie?te Domnul!
Eze 2:5  Ori te-or asculta, ori nu te-or asculta - caci sunt un neam îndaratnic - sa ?tie însa ca este între ei un prooroc.
Eze 2:6  Dar tu, fiul omului, sa nu te temi de ei ?i de cuvintele lor sa nu te sperii, de?i ei vor fi pentru tine spini ?i ciulini, ?i ai sa traie?ti între ei, ca între scorpii; sa nu te temi de cuvintele lor ?i de fa?a lor sa nu te sperii, de?i sunt un neam îndaratnic,
Eze 2:7  Ci sa le spui cuvintele Mele, ori te-or asculta, ori nu te-or asculta, caci sunt un neam de razvrati?i.
Eze 2:8  Tu însa, fiul omului, asculta ce voiesc sa-?i spun! Nu fi îndaratnic, ca acest neam de razvrati?i! Deschide-?i gura ?i manânca ceea ce am sa-?i dau!"
Eze 2:9  ?i privind eu, am vazut o mâna întinsa spre mine ?i în ea o hârtie strânsa sul;
Eze 2:10  ?i a desfa?urat-o înaintea mea, ?i am vazut ca era scrisa ?i pe o parte ?i pe alta: plângere, tânguire ?i jale era scris în ea.
Eze 3:1  Apoi mi-a zis: "Fiul omului, manânca ceea ce ai dinainte, manânca aceasta hârtie ?i mergi de graie?te casei lui Israel!"
Eze 3:2  Atunci eu mi-am deschis gura ?i Acela mi-a dat sa manânc cartea aceea,
Eze 3:3  ?i mi-a zis: "Fiul omului, hrane?te-?i pântecele ?i-?i satura launtrul tau cu aceasta carte pe care ?i-o dau Eu!" ?i eu am mâncat-o ?i era în gura mea dulce ca mierea.
Eze 3:4  Apoi Acela mi-a zis: "Fiul omului, scoala ?i mergi la casa lui Israel ?i le spune cuvintele Mele;
Eze 3:5  Caci nu e?ti trimis la un popor cu grai necunoscut ?i cu limba neîn?eleasa, ci catre casa lui Israel;
Eze 3:6  Nici la mai multe popoare cu grai necunoscut ?i cu limba neîn?eleasa, ale caror cuvinte nu le-ai pricepe. ?i chiar la unele ca acestea de te-a? trimite, ele tot te-ar asculta;
Eze 3:7  Casa lui Israel însa nu va vrea sa te asculte, pentru ca nu vrea sa Ma asculte pe Mine, ca toata casa lui Israel are fruntea încruntata ?i inima împietrita.
Eze 3:8  Pentru aceasta voi împietri chipul tau ca ?i chipurile lor ?i fruntea ta întocmai ca frun?ile lor;
Eze 3:9  Voi face fruntea ta ca diamantul, mai tare decât stânca. Sa nu te temi de ei ?i de fa?a lor sa nu te sperii, caci ei sunt un neam de razvrati?i.
Eze 3:10  Fiul omului - îmi zice iar Acela, - prime?te în inima ta ?i asculta cu urechile tale toate cuvintele ce am sa-?i vorbesc.
Eze 3:11  Scoala ?i mergi la cei du?i în robie, la fiii poporului tau ?i, ori te-or asculta, ori nu te-or asculta, tu graie?te-le ?i le spune: A?a zice Domnul Dumnezeu".
Eze 3:12  Atunci m-a ridicat Duhul ?i am auzit la spatele meu un glas mare ca de tunet care zicea: "Binecuvântata fie slava Domnului în locul unde sala?luie?te El!"
Eze 3:13  ?i am mai auzit zgomotul fiarelor care bateau din aripi ?i huruitul ro?ilor de lânga ele ?i bubuit puternic de tunet.
Eze 3:14  ?i Duhul m-a luat ?i m-a dus. ?i am mers eu amarât ?i cu sufletul întristat, dar mâna Domnului lucra puternic asupra mea.
Eze 3:15  Apoi am sosit în Tel-Aviv, la robii care locuiau aproape de râul Chebar, ?i m-am oprit acolo unde traiau ei ?i am stat între ei ?apte zile în uimire.
Eze 3:16  Iar dupa ce s-au împlinit cele ?apte zile, a fost catre mine cuvântul Domnului ?i mi-a zis:
Eze 3:17  "Fiul omului! Iata, te-am pus strajer casei lui Israel; vei asculta deci cuvântul ce-Mi va ie?i din gura ?i-l vei vesti ca din partea Mea.
Eze 3:18  De voi zice celui rau: Vei muri! ?i tu nu-l vei în?tiin?a, nici nu-i vei grai, pentru a abate pe cel rau de la calea lui cea rea, ca sa traiasca, cel rau va pieri în nelegiuirea sa ?i Eu voi cere sângele lui din mâna ta.
Eze 3:19  Iar daca tu vei în?tiin?a pe cel rau ?i el nu se va întoarce de la rautatea lui ?i de la calea sa cea rea, acela va pieri de pacatul sau, iar tu î?i vei mântui sufletul tau.
Eze 3:20  De asemenea, când cel drept se va abate de la dreptatea sa ?i va savâr?i raul când voi pune înaintea lui o cursa ?i va muri, daca tu nu l-ai în?tiin?at, acela va muri pentru pacatul sau, iar faptele lui de dreptate, pe care le-a facut el, nu i se vor pomeni, ?i Eu voi cere sângele lui din mâinile tale.
Eze 3:21  Iar daca tu vei în?tiin?a pe cel drept sa nu pacatuiasca ?i el nu va pacatui, atunci va fi ?i el viu, pentru ca a fost în?tiin?at ?i î?i vei mântui ?i tu sufletul tau".
Eze 3:22  Apoi a fost iara?i acolo mâna Domnului peste mine ?i mi-a zis Domnul: "Scoala-te ?i ie?i la câmp, ca am sa-?i vorbesc acolo".
Eze 3:23  Atunci m-am sculat, am ie?it la câmp, ?i iata mi s-a aratat acolo slava Domnului, pe care o vazusem la râul Chebar, ?i am cazut cu fa?a la pamânt.
Eze 3:24  Dar a intrat în mine Duhul ?i m-a ridicat în picioare, iar Domnul mi-a grait ?i mi-a zis: "Mergi ?i te închide în casa ta!
Eze 3:25  Fiul omului, iata se vor pune asupra ta frânghii cu care vei fi legat, ca sa nu mai ie?i în mijlocul lor.
Eze 3:26  ?i limba ta o voi lipi de cerul gurii tale, ca sa fii mut ?i sa nu-i mai po?i mustra, ca ace?tia sunt un neam razvratit.
Eze 3:27  Dar când î?i voi grai, voi deschide gura ta ?i tu le vei zice: A?a graie?te Domnul Dumnezeu! Cine va vrea sa asculte, sa asculte, ?i cine nu va vrea sa asculte, sa nu asculte, caci sunt un neam îndaratnic".
Eze 4:1  "?i tu, fiul omului, ia-?i o caramida ?i pune-o înaintea ta ?i sapa pe ea o cetate, Ierusalimul.
Eze 4:2  Apoi rânduie?te împresurare împotriva ei, ridica întarituri ?i val ?i rânduie?te tabara împotriva ei; de jur împrejurul ei a?aza berbeci de spart zidul.
Eze 4:3  Apoi ia o tabla de fier ?i o pune ca un zid de fier între tine ?i cetate ?i întoarce-?i fa?a spre ea, ca ?i cum ea ar fi împresurata, iar tu cel care o împresori. Aceasta va fi un semn pentru casa lui Israel.
Eze 4:4  Dupa aceea sa te culci pe partea stânga, punând pe ea nelegiuirile casei lui Israel, ?i vei purta nelegiuirile ei atâtea zile, cât vei sta culcat pe partea stânga;
Eze 4:5  Ca Eu am sa-?i numar atâtea zile câ?i ani au ?inut nelegiuirile ei, adica ai sa por?i tu nelegiuirile casei lui Israel trei sute nouazeci de zile.
Eze 4:6  Iar dupa ce vei împlini aceste zile, sa te culci pe partea dreapta ?i vei purta nelegiuirile casei lui Iuda timp de patruzeci de zile, câte o zi de fiecare an ce ?i-am hotarât Eu.
Eze 4:7  Sa-?i îndrep?i fa?a ta ?i mâna ta cea dreapta dezgolita spre Ierusalimul împresurat ?i sa prooroce?ti împotriva lui.
Eze 4:8  ?i iata, Eu voi pune pe tine legaturi, ca sa nu te po?i întoarce de pe o parte pe alta, pâna nu vei împlini zilele împresurarii tale.
Eze 4:9  Ia-?i grâu ?i orz, bob ?i linte, mei ?i orzoaica, toarna-le într-un vas ?i-?i fa din ele atâtea pâini, câte zile ai sa ?ezi culcat pe partea stânga; ca trei sute nouazeci de zile ai sa manânci din ele.
Eze 4:10  Hrana ta, cu care te vei hrani, sa o manânci cu masura, câte douazeci de sicli pe zi, dar s-o manânci din timp în timp.
Eze 4:11  ?i apa sa o bei cu masura, câte a ?asea parte de hin pe zi ?i tot din timp în timp.
Eze 4:12  Pâinile le vei gati ca turtele de orz ?i le vei coace înaintea ochilor lor cu necura?enie de om".
Eze 4:13  Apoi Domnul a mai zis: "A?a î?i vor mânca fiii lui Israel pâinea lor, necurata, printre popoarele la care îi voi izgoni".
Eze 4:14  Atunci am zis: "O, Doamne Dumnezeule, sufletul meu niciodata nu s-a întinat; din tinere?ile mele ?i pâna acum niciodata n-am mâncat dintr-un animal mort sau sfâ?iat ?i carne necurata n-a intrat în gura mea".
Eze 4:15  Iar El mi-a raspuns: "Iata, Eu î?i dau, în loc de necura?enie de om, baliga de bou ?i pe ea vei gati pâinea ta".
Eze 4:16  Apoi mi-a zis: "Fiul omului, iata voi trimite foamete în Ierusalim ?i oamenii vor mânca pâinea cu masura ?i cu întristare ?i tot cu masura ?i amaraciune vor bea ?i apa,
Eze 4:17  Caci va fi la ei lipsa de pâine ?i de apa; se vor uita cu groaza unul la altul ?i vor pieri pentru nelegiuirile lor".
Eze 5:1  "Iar tu, fiul omului, ia-?i un cu?it taios, ia-?i un brici de barbierit ?i-l trece pe capul tau ?i pe barba ta; apoi ia o cumpana cu talgere ?i împarte parul în trei par?i:
Eze 5:2  O treime sa o arzi cu foc în mijlocul ceta?ii, dupa ce se vor împlini zilele împresurarii; o alta treime s-o iei ?i s-o toci cu cu?itul în împrejurimile ceta?ii; iar cealalta treime s-o spulberi în vânt, ?i eu voi trage sabia în urma lor.
Eze 5:3  Sa opre?ti însa din parul acela pu?ine fire ?i sa le legi în poala hainei tale.
Eze 5:4  Apoi sa iei câteva fire, sa le arunci în foc ?i sa le arzi. Din acestea va izbucni foc împotriva întregii case a lui Israel.
Eze 5:5  Apoi vei zice catre casa lui Israel: A?a graie?te Domnul Dumnezeu: Acesta este Ierusalimul, pe care Eu l-am pus în mijlocul neamurilor ?i al ?arilor dimprejur.
Eze 5:6  Dar el s-a razvratit împotriva hotarârilor Mele mai mult decât neamurile ?i împotriva legilor Mele mai mult decât ?arile care îl înconjoara, caci el a lepadat legile Mele ?i poruncile Mele nu le-a urmat.
Eze 5:7  De aceea a?a zice Domnul Dumnezeu: Pentru ca tu ai înmul?it nelegiuirile mai mult decât neamurile dimprejurul tau ?i poruncile Mele nu le urmezi, nici nu împline?ti legile Mele, ba nici dupa legile neamurilor celor dimprejurul tau nu te por?i,
Eze 5:8  De aceea a?a zice Domnul Dumnezeu: Iata, Eu vin împotriva ta ?i în mijlocul tau voi rosti osânda ta, în fala neamurilor.
Eze 5:9  Pentru toate spurcaciunile tale î?i voi face ceea ce niciodata n-am facut pâna acum ?i nici nu voi mai face niciodata.
Eze 5:10  De aceea parin?ii î?i vor mânca pe copii în mijlocul tau ?i copiii î?i vor mânca parin?ii lor; voi aduce asupra ta osânda ?i toate rama?i?ele tale le voi spulbera în toate vânturile,
Eze 5:11  Pentru ca tu ai spurcat loca?ul Meu cel sfânt cu to?i idolii tai ?i cu toate ticalo?iile tale, de aceea zice Domnul Dumnezeu: Precum este adevarat ca Eu sunt viu, tot a?a este de adevarat ca te voi mic?ora, ochiul Meu nu te va cru?a ?i nici nu te va milui.
Eze 5:12  O treime din locuitorii tai vor muri de ciuma ?i vor pieri de foame în mijlocul tau; o treime din ei vor cadea de sabie în împrejurimile tale; ?i cealalta treime o voi împra?tia în toate vânturile ?i voi trage sabia în urma lor.
Eze 5:13  A?a-Mi voi împlini mânia, Îmi voi potoli urgia Mea cu ei ?i Ma voi razbuna; ?i când se va savâr?i urgia Mea asupra lor, vor cunoa?te ca Eu, Domnul, am grait în râvna Mea.
Eze 5:14  Pustietate te voi face ?i de ocara între popoarele cele dimprejurul tau ?i în fa?a tuturor trecatorilor.
Eze 5:15  Vei fi de râs ?i de batjocura, de pilda ?i de groaza la popoarele cele dimprejurul tau, când voi savâr?i asupra ta judeca?ile Mele cu mânie, cu urgie ?i cu pedepse aspre: Eu, Domnul, am zis aceasta.
Eze 5:16  Când voi slobozi împotriva voastra sage?ile cele ucigatoare ale foametei, care vor semana moartea, când le voi slobozi spre pieirea voastra ?i voi întari foametea între voi ?i voi zdrobi tot paiul de grâu,
Eze 5:17  Eu voi trimite împotriva voastra foametea ?i fiare rele care te vor lipsi de copii; ciuma ?i sânge vor trece peste tine ?i sabie voi aduce împotriva ta: Eu, Domnul, graiesc acestea".
Eze 6:1  Fost-a catre mine cuvântul Domnului ?i mi-a zis:
Eze 6:2  "Fiul omului, întoarce-?i fa?a ta spre mun?ii lui Israel, prooroce?te împotriva lor ?i zi:
Eze 6:3  Mun?i ai lui Israel, asculta?i cuvântul Domnului Dumnezeu! A?a graie?te Domnul Dumnezeu mun?ilor ?i dealurilor, ?esurilor ?i vailor: Iata, Eu voi aduce asupra voastra sabie ?i voi darâma înal?imile voastre;
Eze 6:4  Altarele voastre vor fi stricate, idolii ridica?i de voi în cinstea soarelui vor fi sfarâma?i ?i pe oamenii vo?tri îi voi face sa cada mor?i înaintea idolilor vo?tri;
Eze 6:5  Trupurile fiilor lui Israel le voi pune înaintea idolilor lor ?i oasele lor le voi risipi împrejurul altarelor lor.
Eze 6:6  În toate a?ezarile voastre cele locuite, ceta?ile vor fi nimicite ?i locurile înalte ale voastre pustiite, altarele voastre vor fi stricate ?i pustiite, idolii vo?tri vor fi sfarâma?i ?i nimici?i, stâlpii vo?tri închina?i soarelui vor fi sfarâma?i ?i lucrarile voastre vor fi ruinate.
Eze 6:7  Cei uci?i vor cadea între voi ?i ve?i cunoa?te ca Eu sunt Domnul.
Eze 6:8  Dar Eu voi lasa o rama?i?a din voi, care va scapa de sabie printre celelalte neamuri, când ve?i fi risipi?i prin ?ari.
Eze 6:9  Aceia din voi, care vor scapa, î?i vor aduce aminte de Mine printre neamurile unde vor fi du?i în robie, pentru ca voi umili inima lor cea desfrânata, care s-a abatut de la Mine, ?i ochii lor, care s-au desfrânat cu idolii, se vor scârbi de ei în?i?i, din pricina acelor rele pe care le-au facut ?i a tuturor ticalo?iilor lor;
Eze 6:10  ?i vor cunoa?te ca Eu sunt Domnul ?i ca nu în zadar i-am amenin?at ca le voi trimite toate aceste rele".
Eze 6:11  A?a graie?te Domnul Dumnezeu: "Love?te palma de palma, bate cu piciorul ?i zi: Vai de casa lui Israel, ca are sa cada de sabie, de foamete ?i de ciuma, pentru toate nelegiuirile sale!
Eze 6:12  Cel ce va fi departe va muri de ciuma; cel ce va fi aproape va muri de sabie, iar cel ce va scapa de acestea ?i va ramâne va muri de foame. A?a-Mi voi potoli mânia Mea împotriva lor.
Eze 6:13  Când uci?ii vor zacea printre idolii lor ?i împrejurul altarelor lor, pe tot locul înalt ?i pe toate vârfurile de munte, sub tot pomul verde, sub tot stejarul umbros ?i în tot locul, unde aduceau tamâie mirositoare tuturor idolilor lor, ve?i cunoa?te ca Eu sunt Domnul.
Eze 6:14  ?i-Mi voi întinde mâna asupra lor ?i voi face ?ara peste tot unde locuiesc ei mai pustie ?i mai de?arta decât pustiul Diblat, ?i vor cunoa?te ca Eu sunt Domnul".
Eze 7:1  Fost-a catre mine cuvântul Domnului ?i mi-a zis:
Eze 7:2  "?i tu, fiul omului, spune: A?a graie?te Domnul Dumnezeul ?arii lui Israel: Vine, vine sfâr?itul asupra celor patru laturi ale pamântului.
Eze 7:3  Iata î?i vine sfâr?itul! Trimite-voi împotriva ta mânia Mea ?i te voi judeca dupa caile tale ?i dupa toate ticalo?iile tale te voi pedepsi.
Eze 7:4  Ochiul Meu nu te va cru?a, nici nu te va milui, ci-?i voi rasplati dupa caile tale, ticalo?iile tale vor fi peste tine ?i vei cunoa?te ca Eu sunt Domnul".
Eze 7:5  A?a graie?te Domnul Dumnezeu: "Iata vine nenorocire peste nenorocire:
Eze 7:6  Sfâr?itul! A venit sfâr?itul! E lânga tine! Iata-l a sosit!
Eze 7:7  A sosit nenorocire asupra ta, locuitor al ?arii! Vine vremea, vine ziua tulburarii, iar nu a chiuiturilor de bucurie prin mun?i.
Eze 7:8  Iata vin acum îndata sa vars asupra ta urgia Mea, mânia Mea sa o potolesc asupra ta ?i te voi judeca dupa caile tale ?i dupa toate ticalo?iile tale te voi pedepsi.
Eze 7:9  Ochiul Meu nu te va cru?a, nici nu te va milui, ci î?i vor plati dupa caile tale; ticalo?iile tale vor fi peste tine ?i vei cunoa?te ca Eu sunt Domnul Care love?te.
Eze 7:10  Iata ziua! Iat-o ca vine! Rândul tau a venit! Vine urgia! Toiagul înflore?te! Trufia se deschide,
Eze 7:11  ?i se ridica silnicia ca sa slujeasca de toiag pentru rautate; nimic nu va ramâne din ei, nici din boga?iile lor, nici din petrecerile lor, nici din stralucirea lor!
Eze 7:12  Vine vremea, a sosit ziua! Cel ce cumpara samân?a sa nu se bucure ?i cel ce vinde sa nu plânga, caci mânia vine peste toata mul?imea lor!
Eze 7:13  Caci cel ce vinde nu se va mai întoarce la cele vândute, macar de ar ?i ramânea printre cei vii, caci proorocia îndreptata împotriva a toata mul?imea lor nu va fi schimbata ?i nici nu-?i va întari via?a sa prin nelegiuirile sale.
Eze 7:14  Suna din trâmbi?e ?i totul e gata, dar nimeni nu va merge la razboi, pentru ca mânia Mea este împotriva a toata mul?imea lor.
Eze 7:15  Afara va fi sabie, iar înauntru ciuma ?i foamete. Cel din câmp va muri de sabie ?i pe cel din cetate îl va mânca ciuma ?i foametea.
Eze 7:16  Iar care vor fugi, vor scapa ?i vor fi prin mun?i ca ni?te porumbei rataci?i. To?i vor geme, fiecare pentru nelegiuirea sa.
Eze 7:17  La to?i le vor tremura mâinile, ?i picioarele tuturor se vor muia ca apa.
Eze 7:18  Atunci cu sac se vor îmbraca ?i groaza îi va cuprinde; to?i vor avea ru?inea pe fe?e ?i pe cap ple?uvia.
Eze 7:19  Argintul ?i-l vor arunca pe uli?e ?i vor dispre?ui aurul; argintul lor ?i aurul lor nu-i vor putea scapa în ziua urgiei Domnului; cu acestea ei nu vor putea sa-?i sature sufletele, nici sa-?i umple pântecele lor, pentru ca acestea au fost pricina a nelegiuirilor lor.
Eze 7:20  Prin gateli frumoase ei le-au prefacut pe acestea în mândrie ?i tot din acestea au facut ei chipurile cele ru?inoase ale idolilor lor. De aceea le voi face pe acestea necurate pentru ei.
Eze 7:21  ?i le voi da prada în mâinile strainilor ?i de jaf nelegiui?ilor pamântului, care le vor spurca.
Eze 7:22  Îmi voi întoarce fa?a de la ei ?i jefuitorii vor întina loca?ul Meu cel sfânt, pentru ca vor intra în el ?i-l vor spurca.
Eze 7:23  Pregate?te lan?uri, ca ?ara aceasta e mânjita de nelegiui?i cu sânge, iar cetatea e plina de silnicii.
Eze 7:24  De aceea vai aduce pe cei mai rai dintre neamuri ca sa puna stapânire pe casele lor. Voi pune capat trufiei celor puternici ?i cele sfinte ale lor vor fi întinate.
Eze 7:25  Iata vine pieirea ?i var cauta pacea, dar nu o vor gasi.
Eze 7:26  Va veni nenorocire peste nenorocire ?i zvon peste zvon, ?i oamenii vor cere vedenii de la prooroc; dar preotului îi va lipsi cuno?tin?a legii ?i batrânului sfatul.
Eze 7:27  Regele va plânge, capetenia va fi cuprinsa de groaza, iar mâinile poporului ?arii vor tremura. Ma voi purta cu ei dupa purtarile lor ?i dupa judeca?ile lor îi voi judeca ?i vor cunoa?te ca Eu sunt Domnul".
Eze 8:1  În anul al ?aselea de la robirea regelui Ioiachim, în cinci ale lunii a ?asea, pe când ?edeam eu în casa mea ?i batrânii lui Iuda ?edeau înaintea mea, s-a lasat peste mine mâna Domnului Dumnezeu.
Eze 8:2  ?i privind eu, am vazut un chip ca de om, de foc parca; ?i parca de la brâul lui în jos era foc, iar de la brâul lui în sus era o stralucire, ca de metal în vapaie.
Eze 8:3  ?i a întins, parca, acela un fel de mâna, ?i m-a apucat de parul capului meu ?i m-a ridicat Duhul între pamânt ?i cer, ?i m-a dus, în vedenii dumnezeie?ti, la Ierusalim, la intrarea por?ii dinauntru, îndreptata spre miazanoapte, unde era a?ezat idolul geloziei care stârne?te gelozia.
Eze 8:4  ?i iata acolo era slava Dumnezeului lui Israel asemenea aceleia pe care o vazusem eu în câmp.
Eze 8:5  Atunci mi-a zis Domnul: "Fiul omului, ridica-?i ochii spre miazanoapte!" ?i mi-am ridicat ochii spre miazanoapte ?i iata acel idol al geloziei era la u?a dinspre miazanoapte a altarului, la intrare.
Eze 8:6  ?i mi-a zis Domnul: "Fiul omului, vezi ce fac ei? Vezi tu ce urâciuni mari face casa lui Israel aici, ca sa Ma îndepartez de loca?ul Meu cel sfânt? Dar întoarce-te ?i urâciuni ?i mai mari vei vedea!"
Eze 8:7  Apoi m-a dus pe poarta în curte ?i privind, am vazut o spartura în perete.
Eze 8:8  ?i mi-a zis Domnul: "Fiul omului, sapa în perete!" ?i am sapat în perete ?i iata am dat de un fel de u?a.
Eze 8:9  ?i mi-a zis Domnul: "Intra ?i vezi urâciunile cele dezgustatoare pe care le fac ace?tia aici".
Eze 8:10  ?i am intrat ?i am privit ?i iata erau acolo tot felul de chipuri de târâtoare, de animale necurate ?i de tot felul de idoli de ai casei lui Israel, zugravi?i pe pere?i de jur împrejur.
Eze 8:11  Înaintea lor stateau ?aptezeci de barba?i din batrânii casei lui Israel, având în mijloc pe Iaazania, fiul lui ?afan; fiecare din ei avea în mâini câte o cadelni?a ?i un nor gros de fum de tamâie se ridica în sus.
Eze 8:12  ?i mi-a zis Domnul: "Fiul omului, vezi ce fac batrânii casei lui Israel la întuneric, stând fiecare în camara sa plina de chipuri? Ca î?i zic: Domnul nu ne vede! A parasit ?ara Sa".
Eze 8:13  Apoi mi-a zis Domnul: "Întoarce-te ?i vei vedea urâciuni înca ?i mai mari, pe care le fac ei".
Eze 8:14  ?i m-a dus la u?a cea dinspre miazanoapte a templului Domnului, ?i iata acolo ?edeau ni?te femei, care plângeau pe Tamuz.
Eze 8:15  ?i mi-a zis Domnul: "Vezi, fiul omului? Întoarce-te ?i vei vedea înca ?i mai mari urâciuni!"
Eze 8:16  Apoi m-a dus în curtea cea dinauntru a templului Domnului ?i iata la u?a templului Domnului, între pridvor ?i jertfelnic, stateau vreo douazeci ?i cinci de oameni cu spatele spre templul Domnului, iar cu fe?ele spre rasarit ?i se închinau spre rasarit la soare.
Eze 8:17  ?i mi-a zis Domnul: "Vezi, fiul omului? Nu i-a ajuns casei lui Iuda sa-?i faca astfel de urâciuni, ca acele pe care le fac ace?tia aici, ci au umplut ?i ?ara de necredin?a, îndoit mâniindu-Ma. Iata ei apropie ramuri de narile lor.
Eze 8:18  De aceea ?i Eu voi lucra cu urgie; ochiul Meu nu-i va cru?a, ?i Eu nu Ma voi îndura. Chiar de ar striga ei cu glas mare la urechile Mele, nu-i voi auzi".
Eze 9:1  Apoi a rasunat la urechile mele un glas mare ?i a zis: "Apropia?i-va pedepsitorii ceta?ii, având fiecare în mâna unealta de nimicire!"
Eze 9:2  ?i iata dinspre poarta de sus, care da spre miazanoapte, veneau ?ase barba?i, având fiecare în mâna unealta sa ucigatoare; ?i între ei se afla unul, îmbracat cu haina de in, care avea la brâu unelte de scris. Ace?tia, venind, s-au oprit lânga jertfelnicul cel de arama.
Eze 9:3  Atunci slava Dumnezeului lui Israel s-a pogorât de pe heruvimul pe care se afla la pragul casei. ?i a chemat Domnul pe omul cel îmbracat în haina de in, care avea la brâu unelte de scris,
Eze 9:4  ?i i-a zis Domnul: Treci prin mijlocul ceta?ii, prin Ierusalim, ?i însemneaza cu semnul crucii (litera "tau" care în alfabetul vechi grec avea forma unei cruci) pe frunte, pe oamenii care gem ?i care plâng din cauza multor ticalo?ii care se savâr?esc în mijlocul lui".
Eze 9:5  Iar celorlal?i le-a zis în auzul meu: "Merge?i dupa el prin cetate ?i lovi?i! Sa nu ave?i nici o mila ?i ochiul vostru sa fie necru?ator!
Eze 9:6  Ucide?i ?i nimici?i pe batrâni, tineri, fecioare, copii, femei, dar sa nu va atinge?i de nici un om, care are pe frunte semnul "+"! ?i sa începe?i cu locul Meu cel sfânt!" ?i au început ei cu batrânii, care erau înaintea templului.
Eze 9:7  Apoi le-a zis: "Întina?i templul, umple?i cur?ile cu uci?i ?i ie?i?i!" ?i au ie?it ?i au început sa ucida prin cetate.
Eze 9:8  ?i dupa ce i-au ucis, iar eu am ramas, am cazut cu fa?a la pamânt ?i, strigând, am zis: "O, Doamne, Dumnezeule, vei pierde oare tot ce a mai ramas din Israel, varsându-?i mânia asupra Ierusalimului?"
Eze 9:9  Iar El mi-a raspuns: "Nelegiuirea casei lui Israel ?i a lui Iuda este mare, foarte mare ?i ?ara aceasta e mânjita cu sânge, iar cetatea e plina de nedreptate, ca ei zic: "A parasit Domnul ?ara aceasta, Domnul nu mai vede!
Eze 9:10  De aceea ochiul Meu nu-i va cru?a ?i Eu nu Ma voi îndura ?i voi întoarce purtarea lor asupra capului lor".
Eze 9:11  ?i iata omul cel îmbracat cu haina de in, care avea la brâu uneltele de scris, a dat raspuns ?i a zis: "Am facut cum mi-ai poruncit".
Eze 10:1  Privind eu atunci, am vazut pe bolta, care era deasupra capetelor heruvimilor, ceva asemanator la înfa?i?are cu un tron de rege, ca piatra de safir.
Eze 10:2  ?i a zis Domnul catre omul cel îmbracat în haina de in: "Intra între ro?ile cele de sub heruvimi ?i umple-?i pumnii de carbuni aprin?i, pe care-i vei lua dintre heruvimi, ?i-i arunca asupra ceta?ii!" ?i el a intrat acolo înaintea ochilor mei;
Eze 10:3  ?i când a intrat omul acela, heruvimii stateau în partea dreapta a casei ?i un nor umplea curtea cea dinauntru.
Eze 10:4  Atunci s-a ridicat slava Domnului de pe heruvimi spre pragul templului ?i templul s-a umplut de nori, iar curtea s-a umplut de stralucirea slavei Domnului.
Eze 10:5  Freamatul aripilor heruvimilor se auzea pâna ?i în curtea de afara, ca glasul când vorbe?te Dumnezeu Atotputernicul.
Eze 10:6  Când a dat Domnul porunca omului celui îmbracat cu haina de in ?i i-a zis: "Ia foc dintre ro?ile cele dintre heruvimi", ?i când a intrat el ?i a stat lânga ro?i,
Eze 10:7  Atunci unul din heruvimi ?i-a întins mâna în focul cel dintre heruvimi, a luat ?i a dat în pumni celui îmbracat în haina de in, iar el, luându-l, a ie?it.
Eze 10:8  ?i la heruvimi sub aripi se vedeau un fel de mâini omene?ti.
Eze 10:9  Cautând eu atunci, am vazut lânga heruvimi patru ro?i, câte o roata lânga fiecare heruvim; ro?ile acestea erau la vedere ca ?i crisolitul.
Eze 10:10  Dupa faptura erau toate la fel ?i parca erau vârâte una în alta.
Eze 10:11  Când ele mergeau, mergeau în toate patru par?ile, ?i în vremea mersului nu se întorceau, ci încotro era îndreptat capul heruvimului, într-acolo mergeau, ?i în vremea mersului nu se întorceau.
Eze 10:12  Tot trupul heruvimilor, spatele lor, mâinile lor ?i aripile lor erau pline de ochi; asemenea ?i ro?ile, toate cele patru ro?i de jur împrejur...
Eze 10:13  Ro?ilor acestora, dupa cum am auzit eu, li s-a zis: "Galgal" (Vijelie).
Eze 10:14  Fiecare din fiin?ele acestea avea patru fe?e: fa?a întâi era fa?a de heruvim, fa?a a doua era fa?a de om, cea de a treia era fa?a de leu ?i cea de a patra era fa?a de vultur.
Eze 10:15  Atunci heruvimii s-au ridicat; ei erau acelea?i fiare pe care le vazusem la râul Chebar.
Eze 10:16  Când mergeau heruvimii, mergeau ?i ro?ile pe lânga ei, iar când heruvimii î?i ridicau aripile ca sa se ridice de la pamânt, nici ro?ile nu se despar?eau, ci erau împreuna cu ei.
Eze 10:17  Când aceia stateau, stateau ?i acestea; iar când se ridicau aceia, ?i acestea se ridicau, pentru ca duhul fiarelor era ?i în ele.
Eze 10:18  Atunci slava Domnului s-a dus de la prag ?i s-a a?ezat pe heruvimi.
Eze 10:19  Iar heruvimii, întinzând aripile, s-au ridicat de la pamânt înaintea ochilor mei ?i s-au dus înso?i?i de ro?i ?i s-au oprit la poarta cea dinspre rasarit a templului Domnului, ?i slava Dumnezeului lui Israel era deasupra lor.
Eze 10:20  Ace?tia erau acelea?i fiare, pe care le vazusem eu la picioarele Dumnezeului lui Israel, la râul Chebar.
Eze 10:21  Fiecare avea câte patru fe?e ?i fiecare avea câte patru aripi, iar sub aripi aveau un fel de mâini omene?ti.
Eze 10:22  Chipul fe?elor ?i înfa?i?area lor era ca acelea pe care le vazusem la râul Chebar, ba ?i ei în?i?i erau aceia?i. Fiecare mergea în partea înspre care era cu fa?a.
Eze 11:1  Atunci m-a ridicat Duhul ?i m-a dus la poarta de rasarit a templului Domnului, care este spre rasarit. ?i iata în poarta, la intrare, erau douazeci ?i cinci de oameni ?i în mijlocul lor am vazut pe Iaazania, fiul lui Azur, ?i pe Pelatia, fiul lui Benaia, capeteniile poporului.
Eze 11:2  ?i mi-a zis Domnul: "Fiul omului, iata oamenii care gândesc nelegiuirea ?i care dau sfaturi rele în cetatea aceasta,
Eze 11:3  Zicând: "Înca n-a venit vremea sa ne zidim case; cetatea este cazanul, iar noi carnea".
Eze 11:4  De aceea prooroce?te, fiul omului, prooroce?te împotriva lor!"
Eze 11:5  Atunci S-a pogorât peste mine Duhul Domnului ?i mi-a zis: "Spune: A?a graie?te Domnul: Casa lui Israel, cele ce zice?i voi ?i cele ce va vin în minte le ?tiu.
Eze 11:6  Mul?i a?i ucis voi în cetatea aceasta ?i a?i umplut uli?ele ei de trupuri.
Eze 11:7  De aceea a?a zice Domnul Dumnezeu: Uci?ii pe care i-a?i îngropat în ea sunt carnea, iar ea e caldarea. Pe voi însa va voi scoate din ea.
Eze 11:8  Voi va teme?i de sabie, ?i sabie voi aduce asupra voastra, zice Domnul Dumnezeu.
Eze 11:9  Va voi scoate din ea, va voi da pe mâna strainilor ?i va voi judeca.
Eze 11:10  De sabie ve?i cadea; la hotarele lui Israel va voi judeca, ?i ve?i cunoa?te ca Eu sunt Domnul.
Eze 11:11  Cetatea aceasta nu va fi pentru voi cazan, nici voi nu ve?i fi pentru ea carne; la hotarele lui Israel va voi judeca.
Eze 11:12  ?i ve?i cunoa?te ca Eu sunt Domnul, ca nu v-a?i purtat dupa poruncile Mele, nici legile Mele nu le-a?i împlinit, ci v-aii purtat dupa obiceiurile neamurilor care va înconjoara".
Eze 11:13  Pe când prooroceam eu, Pelatia, fiul lui Benaia, a murit. Atunci eu am cazut cu fa?a la pamânt ?i am strigat cu glas mare, zicând: "O, Doamne Dumnezeule, oare voie?ti sa pierzi ce a mai ramas lui Israel?"
Eze 11:14  Atunci a fost catre mine cuvântul Domnului ?i mi-a zis:
Eze 11:15  "Fiul omului, ace?tia sunt fra?ii tai, fra?ii tai de un sânge cu tine, ?i toata casa lui Israel, carora locuitorii Ierusalimului le zic: "Departa?i-va de Domnul! ?ara asta noua ni s-a dat de mo?tenire".
Eze 11:16  La aceasta sa zici: A?a zice Domnul Dumnezeu: De?i i-am departat printre popoare ?i de?i i-am risipit prin ?ari, totu?i voi fi pentru ei un loca? sfânt în acele ?ari unde îi voi risipi.
Eze 11:17  Dupa aceea zi: A?a graie?te Domnul Dumnezeu: Va voi aduna de prin popoare ?i va voi întoarce de prin ?arile unde sunte?i risipi?i ?i va voi da pamântul lui Israel.
Eze 11:18  Atunci vor veni acolo ?i vor departa din el toate urâciunile ?i to?i idolii.
Eze 11:19  ?i le voi da aceea?i inima ?i duh nou voi pune în ei; voi scoate din trupul lor inima cea de piatra ?i le voi da inima de carne,
Eze 11:20  Ca sa urmeze poruncile Mele ?i legile sa le pazeasca ?i sa le împlineasca; vor fi poporul Meu, iar Eu le voi fi Dumnezeu.
Eze 11:21  Celor a caror inima este legata de idolii lor ?i de urâciunile lor le voi cere socoteala de purtarea lor", zice Domnul Dumnezeu.
Eze 11:22  Atunci heruvimii ?i-au întins aripile ?i ro?ile erau lânga ei, iar slava Dumnezeului lui Israel, sus, deasupra lor.
Eze 11:23  Apoi slava Domnului s-a ridicat din mijlocul ceta?ii ?i a stat deasupra muntelui, care se afla spre rasarit de cetate.
Eze 11:24  Atunci iar m-a luat Duhul ?i m-a dus în Caldeea, la cei robi?i, cuprins de vedeniile ce le aveam prin Duhul lui Dumnezeu. ?i vedenia pe care o vazusem s-a departat de la mine.
Eze 11:25  ?i am spus celor ce au fost du?i în robie toate cuvintele Domnului pe care mi le descoperise El.
Eze 12:1  Fost-a catre mine cuvântul Domnului ?i mi-a zis:
Eze 12:2  "Fiul omului, traie?ti în mijlocul unui neam razvratit. Ace?tia au ochi ca sa vada, dar nu vad; au urechi ca sa auda, dar nu aud, pentru ca sunt un neam de razvrati?i.
Eze 12:3  Tu însa, fiul omului, fa-?i toate cele de trebuin?a pentru pribegie ?i pleaca ziua, sub ochii lor, ?i pribege?te sub ochii lor din loc în loc, poate vor în?elege ca sunt un neam de razvrati?i.
Eze 12:4  Lucrurile tale sa le sco?i cum se scot lucrurile la vreme de pribegie, ziua, în ochii lor, iar tu pleaca seara, în ochii lor, cum se pleaca în surghiun.
Eze 12:5  În ochii lor fa-?i spartura în zid pe unde vei ie?i.
Eze 12:6  Pune-?i lucrurile pe umar în ochii lor ?i ie?i pe întuneric, ?i fa?a sa ?i-o acoperi, ca sa nu vezi pamântul, caci te-am pus semn casei lui Israel".
Eze 12:7  ?i am facut cum mi se poruncise: lucrurile mele, ca pe ni?te lucruri trebuincioase la vreme de pribegie, le-am scos ziua, iar seara mi-am sapat cu mâna o spartura în zid ?i pe întuneric mi-am scos povara ?i am luat-o pe umar înaintea ochilor lor.
Eze 12:8  Apoi a fost catre mine cuvântul Domnului diminea?a ?i mi-a zis:
Eze 12:9  "Fiul omului, casa lui Israel, neam de razvrati?i, nu te-a întrebat ce faci?
Eze 12:10  Spune-le: A?a graie?te Domnul Dumnezeu: Aceasta este o prevestire pentru cârmuitorul Ierusalimului ?i pentru toata casa lui Israel care se afla acolo.
Eze 12:11  ?i sa le mai spui: Eu sunt un semn pentru voi; ceea ce fac eu, aceea li se va face ?i lor: în pribegie ?i în robie vor merge.
Eze 12:12  ?i capetenia, care se afla între ei, va lua povara pe umeri ?i pe întuneric va ie?i prin zidul pe care îl vor strapunge ca sa-l scoata afara. El î?i va acoperi fa?a, ca sa nu vada cu ochii sai pamântul acesta.
Eze 12:13  Dar voi întinde mreaja Mea în calea lui ?i-l voi prinde în la?ul Meu, ?i-l voi duce la Babilon, în ?ara Caldeilor, dar el nu o va vedea ?i va muri acolo,
Eze 12:14  Iar pe to?i cei dimprejurul lui, aparatorii lui ?i toata o?tirea lui, o voi împra?tia în toate vânturile ?i în urma lor voi trage sabia Mea.
Eze 12:15  ?i vor cunoa?te ca Eu sunt Domnul, când îi voi risipi printre popoare ?i-i voi împra?tia pe fa?a pamântului.
Eze 12:16  Dar un mic numar din ei voi cru?a de sabie, de ciuma ?i de foamete ca sa povesteasca popoarelor, la care vor merge, despre toate ticalo?iile lor, ?i sa ?tie ?i ele ca Eu sunt Domnul".
Eze 12:17  Fost-a catre mine cuvântul Domnului ?i mi-a zis:
Eze 12:18  "Fiul omului, tremurând sa-?i manânci pâinea ta ?i apa ta s-o bei amarât ?i necajit!
Eze 12:19  Spune poporului ?arii: A?a graie?te Domnul Dumnezeu despre locuitorii Ierusalimului ?i despre ?ara lui Israel: Cu întristare î?i vor mânca pâinea lor ?i apa lor ?i-o vor bea cu groaza, pentru ca ?ara lor va fi lipsita de orice bel?ug pentru nedrepta?ile tuturor celor ce o locuiesc.
Eze 12:20  Ora?ele cele locuite ale lor vor fi darâmate ?i ?ara pustiita ?i vor ?ti ca Eu sunt Domnul".
Eze 12:21  Fost-a cuvântul Domnului catre mine ?i mi-a zis:
Eze 12:22  "Fiul omului, ce înseamna zicatoarea care este la voi, în ?ara lui Israel: "Zilele se prelungesc ?i orice vedenie prooroceasca a pierit?"
Eze 12:23  De aceea spune-le: A?a graie?te Domnul Dumnezeu: Nimici-voi aceasta zicatoare ?i o asemenea zicatoare nu se va mai auzi în ?ara lui Israel. Spune-le însa ca aproape este vremea ?i toata vedenia prooroceasca se va împlini.
Eze 12:24  Ca nu va mai ramâne zadarnica nici o vedenie prooroceasca ?i nici o proorocie nu va mai fi mincinoasa în casa lui Israel.
Eze 12:25  Caci Eu, Eu Domnul graiesc, ?i cuvântul care-l spun Eu se va împlini ?i nu se va schimba. Neam de razvrati?i, în zilele voastre am rostit cuvântul ?i-l voi împlini", zice Domnul Dumnezeu.
Eze 12:26  ?i iara?i a fost cuvântul Domnului catre mine ?i mi-a zis:
Eze 12:27  "Fiul omului, iata casa lui Israel zice: "Vedenia prooroceasca, pe care a vazut-o acesta, se va împlini dupa multa vreme, pentru ca el prooroce?te pentru ni?te timpuri departate".
Eze 12:28  De aceea spune-le: "A?a graie?te Domnul Dumnezeu: Nici unul din cuvintele Mele nu va fi amânat, ci cuvântul pe care-l voi rosti se va împlini", zice Domnul Dumnezeu.
Eze 13:1  Fost-a cuvântul Domnului catre mine ?i mi-a zis:
Eze 13:2  "Fiul omului, roste?te profe?ie împotriva proorocilor lui Israel, care prevestesc, ?i zi celor ce proorocesc dupa îndemnul inimii lor: Asculta?i cuvântul Domnului!
Eze 13:3  A?a graie?te Domnul Dumnezeu: Vai de proorocii cei mincino?i, care urmeaza duhul lor ?i nu vad nimic!
Eze 13:4  Proorocii tai, Israele, sunt ca vulpile din ruine.
Eze 13:5  La sparturile zidurilor nu se suie, nici nu apara cu zid casa lui Israel, ca sa stea tare la lupta în ziua Domnului.
Eze 13:6  Vedeniile lor sunt de?arte ?i prevestirile lor mincinoase; ei zic: "Domnul a spus", dar Domnul nu i-a trimis ?i ei încredin?eaza ca se va împlini cuvântul lor.
Eze 13:7  Domnul întreaba: Vedeniile ce le-a?i vazut nu sunt ele, oare, de?arte ?i prevestirile ce le-a?i rostit nu sunt ele, oare, mincinoase? Voi zice?i: Domnul a spus, dar Eu n-am spus.
Eze 13:8  De aceea a?a zice Domnul Dumnezeu: Pentru ca spune?i lucruri de?arte ?i pentru ca vedeniile voastre sunt minciuni, iata Eu vin împotriva voastra, zice Domnul Dumnezeu.
Eze 13:9  Mâna Mea va fi împotriva acestor prooroci, care vad lucruri de?arte ?i prevestesc minciuni; în sfatul poporului Meu ei nu vor fi, nici nu se vor înscrie în cartea casei lui Israel ?i vor ?ti ca Eu sunt Domnul Dumnezeu.
Eze 13:10  Pentru ca duc poporul Meu în ratacire, spunând: "Pace", atunci când nu e pace; ?i pentru ca atunci când el face zid, ei îl tencuiesc cu ipsos.
Eze 13:11  Spune celor ce tencuiesc zidul ca acesta va cadea. Ploaie potopitoare se va varsa, pietre de grindina vor cadea, ?i puhoi vijelios va doborî zidul.
Eze 13:12  Iata el va cadea; atunci nu va vor întreba oare: "Unde este tencuiala cu care l-a?i acoperit?"
Eze 13:13  De aceea a?a zice Domnul Dumnezeu: Iata, Eu, în mânia Mea, voi slobozi furtuna mare, ploaie potopitoare voi varsa în urgia Mea ?i în mânia Mea va cadea grindina pustiitoare.
Eze 13:14  ?i voi darâma zidul pe care voi l-a?i acoperit cu ipsos ?i-l voi doborî la pamânt; temeliile lui se vor dezgoli ?i el va cadea; o data cu el, ve?i pieri ?i voi ?i ve?i ?ti ca Eu sunt Domnul.
Eze 13:15  A?a îmi voi potoli mânia Mea împotriva zidului ?i împotriva celor ce l-au acoperit cu ipsos ?i va voi zice: Nu mai este zidul, nici cei ce l-au tencuit;
Eze 13:16  Nu mai sunt proorocii lui Israel care au proorocit Ierusalimului ?i i-au prevestit vedenii de pace, atunci când nu era pace, zice Domnul Dumnezeu.
Eze 13:17  Iar tu, fiul omului, îndreapta-?i fa?a spre fiicele poporului tau, care proorocesc dupa inima lor; roste?te împotriva lor proorocie,
Eze 13:18  ?i le spune: A?a graie?te Domnul Dumnezeu: Vai de cele ce cos perni?e fermecate pentru subsuori ?i fac marame pentru capul celor de orice statura, pentru a vâna sufletele! Au doar vânând sufletele poporului Meu, va ve?i mântui sufletele voastre?
Eze 13:19  Voi Ma necinsti?i înaintea poporului Meu pentru o mâna de orz ?i pentru o bucata de pâine, ucigând sufletele care nu trebuie sa moara, cru?ând via?a sufletelor care nu trebuie sa traiasca ?i amagind astfel pe poporul Meu, care asculta minciuna.
Eze 13:20  De aceea a?a graie?te Domnul Dumnezeu: Iata Eu sunt împotriva perni?elor voastre fermecate, cu care voi prinde?i sufletele în la? ca pe pasari; vi le voi smulge de la subsuori ?i voi da drumul sufletelor, pe care voi le prinde?i în la?.
Eze 13:21  Voi sfâ?ia maramele voastre ?i voi izbavi poporul Meu din mâinile voastre, ca sa nu mai fie prada în mâinile voastre ?i ve?i ?ti ca Eu sunt Domnul.
Eze 13:22  Pentru ca întrista?i prin minciuni inima dreptului, pe care Eu n-am voit sa o întristez, ?i pentru ca sprijini?i bra?ul celui nelegiuit, ca acesta sa nu se întoarca de la calea lui cea rea spre a-?i pastra via?a.
Eze 13:23  De aceea nu ve?i mai avea vedenii de?arte ?i în viitor nu ve?i mai rosti prevestiri; voi izbavi poporul Meu din mâinile voastre ?i ve?i cunoa?te ca Eu sunt Domnul".
Eze 14:1  Atunci au venit la mine câ?iva din batrânii lui Israel ?i au ?ezut înaintea mea.
Eze 14:2  ?i a fost catre mine cuvântul Domnului ?i a zis:
Eze 14:3  "Fiul omului, ace?ti barba?i î?i poarta idolii în inima ?i î?i au ochii a?inti?i spre ceea ce i-a facut sa cada în nedrepta?i. Pot Eu oare sa le raspund?
Eze 14:4  De aceea graie?te cu ei ?i le spune: A?a graie?te Domnul Dumnezeu: Daca cineva din casa lui Israel, care poarta în inima sa idolii sai ?i î?i are privirile a?intite spre ceea ce l-a facut sa cada în nedrepta?i, va veni sa întrebe pe prooroc, oare a? putea Eu, Domnul, sa-i dau raspuns, din cauza mul?imii idolilor lui?
Eze 14:5  Sa în?eleaga dar casa lui Israel în inima sa, caci ei cu to?ii au ajuns straini de Mine, prin idolii lor.
Eze 14:6  De aceea spune casei lui Israel: A?a graie?te Domnul Dumnezeu: Abate?i-va ?i va întoarce?i de la idolii vo?tri ?i de la toate urâciunile voastre întoarce?i-va fa?a.
Eze 14:7  Ca daca cineva din casa lui Israel ?i din strainii care traiesc în Israel s-a departat de la Mine, îngaduind idolii în inima sa ?i a?intindu-?i ochii spre ceea ce l-a facut sa cada în nedrepta?i, va veni la prooroc ca sa Ma întrebe prin el, îi voi raspunde Eu oare?
Eze 14:8  Voi îndrepta fa?a Mea împotriva omului aceluia, îl voi face sa fie semn ?i pilda ?i-l voi pierde din poporul Meu ?i ve?i cunoa?te ca Eu sunt Domnul.
Eze 14:9  Iar daca proorocul va amagi ?i va spune cuvânt, ca ?i cum Eu, Domnul, i l-a? fi spus, atunci Eu îmi voi întinde mâna ?i-l voi stârpi din poporul Meu Israel.
Eze 14:10  ?i a?a î?i vor lua to?i pedeapsa pentru nelegiuirea lor; cum va fi pedeapsa celui ce întreaba, a?a va fi ?i pedeapsa celui ce prooroce?te,
Eze 14:11  Ca în viitor casa lui Israel sa nu se mai abata de la Mine ?i ca sa nu se mai întineze cu tot felul de nelegiuiri; ca sa fie poporul Meu, iar Eu sa fiu Dumnezeul lor", zice Domnul Dumnezeu.
Eze 14:12  Fost-a cuvântul Domnului catre mine ?i mi-a zis:
Eze 14:13  "Fiul omului, daca vreo ?ara ar pacatui înaintea Mea, abatându-se în chip nelegiuit de la Mine, ?i Eu a? întinde mâna Mea asupra ei, a? zdrobi în ea tot spicul de grâu ?i a? trimite asupra ei foametea ?i a? începe sa pierd în ea pe oameni ?i dobitoace;
Eze 14:14  Daca s-ar afla acolo cei trei barba?i: Noe, Daniel ?i Iov, apoi ace?tia, prin dreptatea lor, ?i-ar scapa numai via?a lor, zice Domnul Dumnezeu.
Eze 14:15  Sau daca a? trimite asupra acestui pamânt fiare rele, care l-ar vaduvi de popor, ?i daca ?el din pricina fiarelor ar ajunge pustiu ?i de nestrabatut,
Eze 14:16  Atunci ace?ti trei barba?i, aflându-se în el, precum este de adevarat ca Eu sunt viu, zice Domnul, tot a?a este de adevarat ca ei n-ar scapa nici pe fii, nici pe fiice, ci numai ei singuri ar scapa, iar ?ara ar ajunge pustie.
Eze 14:17  Sau daca a? aduce în ?ara aceasta sabie ?i a? zice: "Sabie, strabate ?ara" ?i a? începe a pierde acolo pe oameni ?i animale,
Eze 14:18  Atunci ace?ti trei barba?i, aflându-se în ea, precum este adevarat ca Eu sunt viu, tot a?a este de adevarat, zice Domnul, ca ei n-ar scapa nici pe fii, nici pe fiice, ci numai ei singuri ar scapa.
Eze 14:19  Sau daca Eu a? trimite ciuma în ?ara aceasta ?i a? revarsa asupra ei urgia Mea în varsare de sânge, ca sa pierd din ea ?i pe oameni ?i pe animale,
Eze 14:20  Apoi Noe, Daniel ?i Iov, aflându-se în ea, precum este adevarat ca Eu sunt viu, zice Domnul, tot a?a este de adevarat ca n-ar scapa nici fii, nici fiice; prin dreptatea lor ei ?i-ar scapa numai via?a lor".
Eze 14:21  Ca a?a zice Domnul Dumnezeu: "Chiar de a? trimite aceste patru pedepse grozave ale Mele: sabia, foametea, fiarele salbatice ?i ciuma împotriva Ierusalimului, ca sa stârpesc din el oamenii ?i dobitoacele,
Eze 14:22  ?i atunci va ramâne în ei o rama?i?a de fii ?i fiice, care vor fi sco?i de acolo. Iata ei vor veni la voi ?i voi ve?i vedea purtarea lor ?i faptele dor ?i va ve?i mângâia de nenorocirea pe care Eu am adus-o asupra Ierusalimului ?i de toate câte am adus asupra lui.
Eze 14:23  Ei va vor mângâia, când ve?i vedea purtarea lor ?i faptele lor, ?i ve?i cunoa?te ca nu în zadar am facut Eu toate câte am facut în el", zice Domnul Dumnezeu.
Eze 15:1  ?i a fost iara?i cuvântul Domnului catre mine:
Eze 15:2  "Fiul omului, ce întâietate are lemnul de vi?a de vie fa?a de oricare alt lemn, ?i coarda de vi?a de vie între arborii din padure?
Eze 15:3  Se ia oare din el vreo buca?ica pentru vreun lucru? Se ia oare din el macar pentru un cui, ca sa atârni în el un lucru oarecare?
Eze 15:4  Iata el se da focului spre ardere; amândoua capetele lui le mistuie focul, ?i mijlocul arde ?i el. Va mai fi el bun de ceva?
Eze 15:5  Nici când era întreg nu era bun la ceva, cu atât mai mult nu va fi acum, când l-a mistuit focul; acum când a ars se mai poate face ceva cu el?
Eze 15:6  De aceea a?a zice Domnul Dumnezeu: Precum lemnul de vi?a de vie dintre arborii padurii l-am dat focului ca sa-l arda, a?a voi da ?i pe locuitorii Ierusalimului.
Eze 15:7  Îmi voi întoarce fa?a Mea împotriva lor. Dintr-un foc au scapat, dar focul îi va mistui, ?i ve?i ?ti ca Eu sunt Domnul, când îmi voi întoarce fa?a împotriva lor.
Eze 15:8  Voi face din ?ara un pustiu, pentru ca ei Mi-au fost necredincio?i", zice Domnul Dumnezeu.
Eze 16:1  Fost-a cuvântul Domnului catre mine:
Eze 16:2  "Fiul omului, spune Ierusalimului urâciunile lui,
Eze 16:3  ?i-i spune: A?a graie?te Domnul Dumnezeu catre fiica Ierusalimului: Obâr?ia ta ?i patria ta e ?ara Canaan; tatal tau e amoreu ?i mama ta e hetita.
Eze 16:4  La na?terea ta, în ziua în care te-ai nascut, nu ti s-a taiat buricul ?i cu apa n-ai fost spalata pentru cura?ire ?i cu sare n-ai fost sarata, nici cu scutece înfa?ata.
Eze 16:5  Ochiul nimanui nu s-a înduio?at spre tine, ca din mila de tine sa-?i fi facut vreuna din acestea; ci ai fost aruncata în câmp din dispre? catre via?a ta, în ziua na?terii tale.
Eze 16:6  ?i am trecut Eu pe lânga tine ?i te-am vazut zbatându-te în sângele tau ?i ?i-am zis: "Traie?te în sângele tau!" A?a ?i-am zis: "Traie?te în sângele tau!"
Eze 16:7  ?i te-am înmul?it ca pe iarba câmpului; ai crescut, te-ai facut mare ?i ai ajuns la o frumuse?e desavâr?ita; ?i s-a ridicat pieptul ?i ?i-a crescut parul; dar erai goala, de tot goala.
Eze 16:8  Atunci am trecut Eu pe lânga tine ?i te-am vazut, ?i iata aceea era vremea ta, vremea iubirii. Atunci mi-am întins Eu poala mantiei Mele peste tine ?i am acoperit goliciunea ta, ?i-am facut un juramânt, am facut un legamânt cu tine, zice Domnul Dumnezeu ?i tu ai fost a Mea.
Eze 16:9  Apoi te-am spalat cu apa, am cura?it de pe tine sângele tau ?i te-am uns cu untdelemn.
Eze 16:10  ?i-am dat ve?minte brodate, încal?aminte de piele fina, o legatura de vison pentru cap ?i o mantie de matase.
Eze 16:11  Te-am gatit cu podoabe ?i ?i-am pus bra?ari la mâini ?i salbe la gât.
Eze 16:12  ?i-am dat inel în nas ?i cercei în urechi ?i pe cap ?i-am pus o coroana minunata.
Eze 16:13  A?a ai fost împodobita cu aur ?i cu argint ?i îmbracamintea ta era de vison, de matase ?i de ?esaturi brodate; te-ai hranit cu pâine din cea mai buna faina de grâu, cu miere ?i untdelemn, ?i erai foarte frumoasa ?i ai ajuns la vrednicia de regina.
Eze 16:14  Ai fost renumita printre neamuri pentru frumuse?ea ta, pentru ca ea era desavâr?ita datorita stralucirii Mele cu care te-am îmbracat", zice Domnul Dumnezeu.
Eze 16:15  Dar tu te-ai încrezut în frumuse?ea ta ?i, folosindu-te de renumele tau, ai început sa te desfrânezi; ?i-ai cheltuit desfrânarea ta cu tot trecatorul, dându-te pe tine lui.
Eze 16:16  Ai luat din hainele tale, ca sa-?i faci locuri înalte în culori felurite, ?i te-ai desfrânat acolo, cum niciodata nu s-a întâmplat ?i nici nu va mai fi.
Eze 16:17  Ai luat lucrurile tale de gateala, facute din aurul Meu ?i din argintul Meu, pe care ?i le-am dat Eu, ?i ?i-ai facut chipuri de barbat ?i te-ai desfrânat cu ele.
Eze 16:18  Ai luat hainele tale cele brodate ?i i-ai îmbracat pe ei cu ele; ai pus înaintea lor uleiul ?i tamâia Mea.
Eze 16:19  ?i pâinea Mea, pe care Eu ?i-o dadeam ?ie; faina de grâu, uleiul ?i mierea, pe care Eu ?i le dadeam ?ie, tu le puneai înaintea lor spre miros de buna mireasma; iata ce s-a întâmplat, zice Domnul Dumnezeu.
Eze 16:20  Ai luat pe fiii tai ?i pe fiicele tale pe care Mi i-ai nascut Mie ?i i-ai adus lor jertfa spre mâncare. Dar pu?in te-ai desfrânat tu oare?
Eze 16:21  Ba tu ?i pe fiii Mei i-ai junghiat ?i i-ai dat lor, trecându-i prin foc.
Eze 16:22  Pe lânga toate urâciunile ?i desfrânarile tale tu nu ?i-ai adus aminte de zilele tinere?ii tale, când erai goala, cu totul goala, zbatându-te ?i aruncata în sângele tau.
Eze 16:23  Dupa toate nelegiuirile tale, vai, vai de tine, zice Domnul Dumnezeu.
Eze 16:24  Ca ?i-ai facut case de desfrânare, ai a?ezat locuri înalte în fiecare pia?a;
Eze 16:25  La raspântia fiecarui drum ?i-ai facut locuri înalte, ?i-ai batjocorit frumuse?ea ta ?i ?i-ai aratat picioarele înaintea fiecarui trecator ?i ?i-ai înmul?it desfrânarile.
Eze 16:26  Te-ai desfrânat cu fiii Egiptului, vecinii tai, oameni înal?i la statura, ?i ?i-ai înmul?it desfrânarile, mâniindu-Ma pe Mine.
Eze 16:27  Iata, Mi-am întins asupra ta mâna Mea, am împu?inat cele menite pentru tine ?i te-am lasat prada fetelor Filistenilor, du?mancele tale, care s-au ru?inat de purtarea ta cea nelegiuita.
Eze 16:28  Tu te-ai desfrânat cu Asirienii ?i nu te-ai saturat; te-ai desfrânat cu ei, dar nu te-ai mul?umit cu atât,
Eze 16:29  Ci ai înmul?it desfrânarile tale din pamântul Canaan pâna în pamântul Caldeii, dar nici cu atât nu te-ai mul?umit.
Eze 16:30  Cât de obosita trebuie sa fie inima ta, zice Domnul Dumnezeu, dupa ce ai facut toate acestea, ca o desfrânata nestapânita!
Eze 16:31  Când ?i-ai facut case de desfrânare la fiecare raspântie de drum ?i ?i-ai facut locuri înalte în fiecare pia?a, nu erai ca o desfrânata, pentru ca respingeai darurile,
Eze 16:32  Ci ca o femeie adultera, care, în locul barbatului sau, prime?te pe al?ii.
Eze 16:33  Tuturor desfrânatelor se dau daruri; tu însa dadeai însa?i daruri aman?ilor tai ?i îi cumparai, ca sa vina ace?tia din toate par?ile la tine ?i sa se desfrâneze cu tine.
Eze 16:34  La tine desfrânarile tale se petreceau în alt fel decât se întâmpla cu femeile; nu umblau barba?ii dupa tine, ci tu dadeai daruri, iar ?ie nu ?i se dadeau daruri, ?i deci tu te-ai purtat cu totul altfel decât altele.
Eze 16:35  De aceea asculta, desfrânato, cuvântul Domnului:
Eze 16:36  A?a zice Domnul Dumnezeu: Pentru ca tu ?i-ai varsat astfel banii tai ?i pentru ca în desfrânarile tale ?i s-a descoperit goliciunea ta înaintea aman?ilor tai ?i înaintea tuturor oamenilor tai neru?ina?i ?i pentru sângele fiilor tai, pe care tu i-ai dat lor,
Eze 16:37  Pentru toate acestea iata Eu voi aduna pe to?i aman?ii tai cu care te-ai desfrânat tu ?i pe care i-ai iubit ?i pe to?i aceia pe care i-ai urât, ?i-i voi aduna pretutindeni împotriva ta ?i voi descoperi înaintea lor goliciunea ta ?i vor vedea toata ru?inea ta.
Eze 16:38  Te voi judeca cum se judeca femeile adultere ?i cele ce varsa sânge ?i te voi preda urgiei ?i pismei;
Eze 16:39  Te voi da în mâinile acelora ?i ei vor darâma casele tale de desfrânare, vor risipi locurile tale înalte, vor rupe de pe tine hainele tale, vor lua podoabele tale ?i te vor lasa goala, de tot goala.
Eze 16:40  Voi strânge împotriva ta adunare ?i te vor ucide cu pietre ?i cu sabie te vor taia.
Eze 16:41  Vor arde casele tale cu foc ?i te vor judeca înaintea ochilor multor femei; a?a voi pune capat desfrânarii tale ?i nu vei mai face daruri.
Eze 16:42  Îmi voi potoli cu tine urgia Mea ?i Ma voi lini?ti ?i nu Ma vai mai mânia.
Eze 16:43  Pentru ca tu nu ?i-ai adus aminte de zilele tinere?ii tale ?i de toate cu câte M-ai mâniat, iata ?i Eu voi întoarce purtarea ta asupra capului tau, zice Domnul Dumnezeu, ca tu sa nu te mai dedal desfrâului cu to?i idolii tai.
Eze 16:44  Iata, tot cel ce graie?te în pilde poate sa zica de tine: "Cum e mama, a?a e ?i fiica!"
Eze 16:45  Tu e?ti cu adevarat fiica mamei tale, care ?i-a lepadat barbatul ?i copiii; tu e?ti cu adevarat sora surorilor tale, care ?i-au lepadat barba?ii ?i copiii lor. Mama voastra este hetita ?i tatal vostru amoreu.
Eze 16:46  Iar sora ta cea mai mare este Samaria, care traie?te cu fiicele sale în stânga ta. Iar sora ta cea mai mica, aceea care traie?te la dreapta ta, este Sodoma cu fiicele ei.
Eze 16:47  Tu însa n-ai mers nici macar pe caile lor ?i nici macar la urâciunile lor nu te-ai marginit; aceasta ?i s-a parut pu?in; tu te-ai aratat mai stricata decât ele în toate caile tale.
Eze 16:48  Viu sunt Eu, zice Domnul Dumnezeu; Sodoma, sora ta, n-a facut nici ea, nici fiicele ei ce-ai facut tu ?i fiicele tale.
Eze 16:49  Iata care au fost faradelegile Sodomei, sora ta, ?i ale fiicelor ei: mândria, îmbuibarea ?i trândavia; iar mâinile saracului ?i ale celui nevoia? nu le-au sprijinit.
Eze 16:50  Ele s-au mândrit ?i au facut urâciune înaintea Mea; de aceea le-am ?i nimicit, cum ai vazut.
Eze 16:51  Iar Samaria n-a pacatuit nici pe jumatate din ce ai pacatuit tu. Tu le-ai întrecut în urâciuni ?i prin urâciunile tale, pe care le-ai facut tu, surorile tale s-au dovedit mai drepte decât tine.
Eze 16:52  Poarta-?i dar ru?inea ta ?i tu, ceea ce osândeai pe surorile tale; fa?a de pacatele tale, cu care tu te-ai batjocorit, acelea sunt mai drepte decât tine. Ro?e?te dar de ru?ine ?i tu ?i du-?i batjocura ta, pentru ca ai îndrepta?it astfel pe surorile tale.
Eze 16:53  Dar Eu voi aduce înapoi pe prin?ii lor de razboi, pe prin?ii de razboi ai Sodomei ?i ai fiicelor ei, pe prin?ii de razboi ai Samariei ?i ai fiilor ei ?i pe prin?ii tai de razboi în mijlocul lor,
Eze 16:54  Ca sa-?i por?i ru?inea ta ?i sa te ru?inezi de tot ceea ce ai facut, servind ca mângâiere pentru ele.
Eze 16:55  Surorile tale, Sodoma, ?i fiicele ei, se vor întoarce la starea lor de mai înainte; Samaria ?i fiicele ei se vor întoarce la starea lor de mai înainte; ?i tu ?i fiicele tale va ve?i întoarce la starea voastra de altadata.
Eze 16:56  De sora ta, Sodoma, nici pomenire nu a fost pe buzele tale în zilele trufiei tale,
Eze 16:57  Pâna nu se descoperise goliciunea ta, ca în zilele batjocurei ce ?i-a venit din partea fiicelor Siriei ?i a tuturor celor ce o înconjoara, din partea fiicelor Filistenilor, care din toate par?ile se uitau la tine cu dispre?.
Eze 16:58  Tu suferi din pricina desfrâului tau ?i din pricina ticalo?iei tale, zice Domnul Dumnezeu;
Eze 16:59  Ca a?a graie?te Domnul Dumnezeu: Ma voi purta cu tine, cum te-ai purtat tu, dispre?uind juramântul prin ruperea legamântului.
Eze 16:60  Dar Eu Îmi voi aduce aminte de legamântul Meu încheiat cu tine în zilele tinere?ii tale ?i voi înnoi cu tine un a?ezamânt ve?nic.
Eze 16:61  ?i tu î?i vei aduce aminte de caile tale ?i-?i va fi ru?ine când vei începe sa prime?ti la tine pe surorile tale, cele mai mari decât tine ?i pe cele mai mici decât tine ?i când Eu le voi da ?ie ca fiice, însa nu dupa a?ezamântul tau.
Eze 16:62  Voi înnoi a?ezamântul Meu cu tine ?i vei cunoa?te ca Eu sunt Domnul,
Eze 16:63  Ca sa-?i aduci aminte ?i sa te ru?inezi, ca pe viitor sa nu po?i nici gura sa-?i deschizi de ru?ine, când i?i voi ierta ceea ce ai facut", zice Domnul Dumnezeu.
Eze 17:1  Fost-a iara?i cuvântul Domnului catre mine ?i mi-a zis:
Eze 17:2  "Fiul omului, spune casei lui Israel o ghicitoare, spune-i o pilda
Eze 17:3  ?i zi: A?a graie?te Domnul Dumnezeu: Un vultur mare eu aripi mari, cu pene lungi, pufos ?i pestri? a zburat în Liban ?i a frânt vârful unui cedru.
Eze 17:4  A rupt pe cel mai de sus din lastarii lui cei tineri ?i l-a adus în ?ara Canaanului ?i l-a pus într-o cetate de negustori;
Eze 17:5  A luat apoi pe unul din lastarii cedrului ?i l-a pus într-un pamânt roditor, l-a sadit lânga o apa mare, ca pe o salcie.
Eze 17:6  ?i lastarul a crescut ?i s-a facut un butuc de vi?a întins, dar nu înalt, ale carui ramuri se întorceau spre vultur, iar radacinile îi erau sub el; el s-a facut vi?a, a dat lastari ?i a facut coarde.
Eze 17:7  ?i a mai fost un alt vultur cu aripi mari ?i pufos. ?i iata acest butuc de vi?a s-a întins spre el cu radacinile ?i cu coardele sale, ca sa-l ude acela mai mult decât era udat în locul unde fusese sadit.
Eze 17:8  El era sadit într-o ?arina buna, la ni?te ape mari, încât î?i putea întinde vi?ele ca sa aduca rod ?i sa ajunga un minunat butuc de vie.
Eze 17:9  Spune dar: A?a graie?te Domnul Dumnezeu: Spori-va el oare? Nu cumva i se vor smulge radacinile ?i i se vor rupe vi?ele, încât sa se usuce? To?i lastarii tineri, care au crescut din el se vor usca. ?i se va smulge din radacinile lui nu cu putere mare, nici cu oameni mul?i.
Eze 17:10  Iata, de?i a fost sadit, îi va merge oare, bine? Oare nu se va usca îndata ce se va atinge de el vântul de rasarit? Se va usca pe mu?uroiul pe care a fost sadit".
Eze 17:11  Fost-a cuvântul Domnului catre mine ?i mi-a zis:
Eze 17:12  "Spune neamului de razvrati?i: Oare nu ?ti?i ce înseamna acestea? Spune: Iata a venit regele Babilonului la Ierusalim ?i a luat pe rege ?i pe capeteniile lui ?i i-a dus cu sine în Babilon.
Eze 17:13  A luat pe unul din neamul regesc ?i a încheiat cu acela legamânt, l-a legat  cu juramânt, iar pe puternicii ?arii i-a luat cu sine,
Eze 17:14  Ca regatul sa ramâna ascultator ?i fara trufie, ca sa pazeasca legamântul ?i sa-s fie credincios.
Eze 17:15  Dar acela s-a departat de el, trimi?ându-?i soli în Egipt ca sa li se dea cai ?i oameni mul?i. Va izbuti el oare ?i va scapa el, cel ce a facut una ca aceasta? El a calcat învoiala; va scapa el oare?
Eze 17:16  Precum este adevarat ca Eu sunt viu, zice Domnul Dumnezeu, tot a?a este adevarat ca acolo unde ?ade regele care l-a facut pe el rege, adica la Babilon, va muri, pentru ca a nesocotit juramântul pe care l-a dat aceluia ?i a calcat legamântul încheiat cu el.
Eze 17:17  Faraon cu puterea cea mare ?i cu mul?imea poporului nu va face nimic pentru el în acest razboi, când se va ridica val ?i se vor zidi turnuri de împresurare spre pieirea a multe suflete.
Eze 17:18  El a nesocotit juramântul, ca sa calce legamântul, ?i iata ?i-a dat mâna ?i a facut toate acestea ?i nu va scapa.
Eze 17:19  De aceea, a?a zice Domnul Dumnezeu: Precum este adevarat ca Eu sunt viu, tot a?a este de adevarat ca juramântul Meu, pe care l-a nesocotit, ?i legamântul Meu, pe care l-a calcat, le voi întoarce asupra capului sau;
Eze 17:20  Voi arunca asupra lui mreaja Mea ?i va fi prins în la?urile Mele; îl voi aduce la Babilon ?i acolo Ma voi judeca cu el pentru necredincio?ia lui fa?a de Mine.
Eze 17:21  Iar to?i fugarii lui din toate taberele lui vor cadea de sabie, ?i cei rama?i vor fi împra?tia?i în toate vânturile ?i ve?i cunoa?te ca Eu, Domnul, am zis acestea".
Eze 17:22  A?a zice Domnul Dumnezeu: "Voi lua din vârful cedrului celui înalt un lastar ?i-l voi sadi; din cre?tetul lui voi rupe lastarul cel plapând ?i-l voi sadi pe un munte înalt ?i mare?.
Eze 17:23  Pe muntele cel înalt al lui Israel îl voi sadi ?i va slobozi ramuri, va aduce rod, ?i se va face un cedru falnic ?i vor locui în el tot felul de pasari; tot felul de înaripate vor locui în umbra ramurilor lui.
Eze 17:24  ?i vor cunoa?te to?i pomii câmpului ca Eu, Domnul, pomul înalt îl fac mic ?i pe cel mic îl fac înalt; pomul verde îl usuc ?i pomul uscat îl înverzesc. Eu, Domnul, am spus acestea".
Eze 18:1  ?i a mai fost cuvântul Domnului catre mine ?i mi-a zis:
Eze 18:2  "Pentru ce spune?i voi în ?ara lui Israel pilda aceasta ?i zice?i: Parin?ii au mâncat agurida ?i copiilor li s-au strepezit din?ii?
Eze 18:3  Precum este adevarat ca Eu sunt viu, zice Domnul Dumnezeu, tot a?a este de adevarat ca pe viitor nu se va mai grai pilda aceasta lui Israel.
Eze 18:4  Ca iata toate sufletele sunt ale Mele; cum este al Meu sufletul tatalui, tot a?a ?i sufletul fiului; sufletul care a gre?it va muri.
Eze 18:5  De este cineva drept ?i face judecata ?i dreptate;
Eze 18:6  De nu manânca jertfit în munte ?i spre idolii casei lui Israel nu-?i întoarce ochii sai; femeia aproapelui sau nu o necinste?te ?i de femeie nu se apropie în timpul perioadei ei de necura?ie;
Eze 18:7  Pe nimeni nu strâmtoreaza ?i datornicului îi întoarce zalogul, furt nu face, celui flamând îi da din pâinea sa ?i pe cel gol îl îmbraca cu haina;
Eze 18:8  Banii sai cu camata nu-i da ?i camata nu ia; de la nedreptate mâinile ?i le stapâne?te ?i judecata dintre un om ?i altul o face cu dreptate;
Eze 18:9  De se poarta dupa poruncile Mele ?i legile Mele cu credincio?ie le paze?te, acela este drept ?i fara îndoiala viu va fi, zice Domnul Dumnezeu.
Eze 18:10  Dar de i s-a nascut fiu ho?, care varsa sânge sau face ceva de felul acesta,
Eze 18:11  ?i care nu urmeaza calea tatalui sau, ci manânca cele jertfite în mun?i, necinste?te femeia aproapelui sau;
Eze 18:12  Pe sarac ?i pe lipsit îl apasa, rape?te avutul altuia ?i zalogul nu-l întoarce; î?i ridica ochii la idoli ?i face ticalo?ii;
Eze 18:13  Banii ?i-i da cu dobânda ?i ia camata; unul ca acesta va trai oare? Nu! De va face asemenea ticalo?ii nu va trai, ci sigur va muri ?i sângele lui va fi asupra lui.
Eze 18:14  Iar de i s-a nascut un fiu, care, vazând pacatele, vazând toate câte le-a facut tatal sau, el se paze?te ?i nu face nimic asemenea;
Eze 18:15  În mun?i nu manânca jertfe idole?ti, nu-?i ridica ochii spre idolii cei mincino?i ai casei lui Israel ?i femeia aproapelui sau nu o necinste?te;
Eze 18:16  Pe nimeni nu apasa, zalog nu ia ?i avutul altuia nu-l risipe?te; celui flamând îi da din pâinea sa, cu haina sa îmbraca pe cel gol;
Eze 18:17  Pe sarac nu asupre?te, nu ia nici camata, nici dobânda; poruncile Mele le paze?te ?i se poarta dupa legile Mele; acest om nu va muri pentru nedrepta?ile parintelui sau, ci în veci va trai.
Eze 18:18  Iar tatal sau, pentru ca a apasat pe al?ii, a rapit ceea ce era al fratelui sau ?i a facut în poporul sau ceea ce nu era îngaduit, iata va muri pentru nedreptatea sa.
Eze 18:19  Dar ve?i zice: Pentru ce fiul sa nu poarte nedreptatea tatalui sau? Pentru ca fiul a facut ceea ce era drept ?i legiuit ?i toate legile Mele le-a ?inut ?i le-a împlinit; de aceea va trai.
Eze 18:20  Sufletul care pacatuie?te va muri. Fiul nu va purta nedreptatea tatalui, ?i tatal nu va purta nedreptatea fiului. Celui drept i se va socoti dreptatea sa, iar celui rau, rautatea sa.
Eze 18:21  Dar daca cel rau se întoarce de la nelegiuirile sale pe care le-a facut ?i paze?te toate legile Mele ?i face ceea ce e bun ?i drept, el va trai ?i nu va muri.
Eze 18:22  Nu se vor pomeni deloc nelegiuirile pe care el le va fi facut, ci va trai pentru dreptatea pe care va fi facut-o.
Eze 18:23  Oare voiesc Eu moartea pacatosului, zice Domnul Dumnezeu - ?i nu mai degraba sa se întoarca de la caile sale ?i sa fie viu?
Eze 18:24  Dar ?i dreptul, daca se va abate de la dreptatea sa ?i se va purta cu nedreptate ?i va face toate acele ticalo?ii pe care le face nelegiuitul, va fi el oare viu? Toate faptele lui bune, pe care le va fi facut, nu se vor pomeni, ci pentru nelegiuirea sa, pe care va fi facut-o, ?i pentru pacatele sale, pe care le-a savâr?it, va muri.
Eze 18:25  Dar voi zice?i: "Calea Domnului nu este dreapta". Asculta?i, casa lui Israel: Oare calea Mea nu este dreapta, sau nu sunt drepte caile voastre?
Eze 18:26  Daca cel drept se abate de la dreptatea sa ?i face nelegiuire ?i din pricina aceasta moare, apoi el moare pentru nelegiuirea sa, pe care a facut-o.
Eze 18:27  ?i cel nelegiuit, daca se întoarce de la nelegiuirea sa, pe care a facut-o ?i face judecata ?i dreptate, î?i întoarce sufletul sau la via?a;
Eze 18:28  Caci el a vazut ?i s-a întors de la toate nelegiuirile sale, pe care le-a facut; de aceea va fi viu ?i nu va muri.
Eze 18:29  Însa casa lui Israel zice: "Calea Domnului nu este dreapta". Casa lui Israel, oare calea Mea nu este dreapta, sau nu sunt drepte caile voastre?
Eze 18:30  De aceea va voi judeca pe voi din casa lui Israel, pe fiecare dupa caile sale, zice Domnul Dumnezeu; pocai?i-va ?i va întoarce?i de la toate nelegiuirile voastre, ca necredin?a sa nu va fie piedica.
Eze 18:31  Lepada?i de la voi toate pacatele voastre cu care a?i gre?it ?i va face?i o inima noua ?i un duh nou. De ce sa muri?i voi, casa lui Israel?
Eze 18:32  Caci Eu nu voiesc moartea pacatosului, zice Domnul Dumnezeu; întoarce?i-va deci ?i trai?i!"
Eze 19:1  Iar tu, fiul omului, fa o tânguire pentru capeteniile lui Israel ?i spune:
Eze 19:2  "Ce este mama ta? O leoaica! Ea a dormit între lei ?i între puii de leu ?i-a alaptat puii sai.
Eze 19:3  A crescut pe unul din puii sai, care s-a facut leu, ?i a înva?at sa sfâ?ie prada, ?i a mâncat oameni.
Eze 19:4  ?i auzind popoarele de dânsul, a fost prins în groapa lor ?i l-au adus în lan?uri în ?ara Egiptului.
Eze 19:5  A?teptând leoaica pu?in timp ?i vazând ca a?teptarea sa este zadarnica, a luat pe altul din puii sai ?i l-a facut leu.
Eze 19:6  Acesta, facându-se leu, a început sa umble printre lei ?i a înva?at sa sfâ?ie prada ?i a mâncat oameni;
Eze 19:7  A atacat casele lor ?i le-a pustiit ceta?ile; ?i de larma ragetului lui, ?ara ?i locuitorii ei s-au îngrozit ?i s-a pustiit ?ara ?i toate satele ei.
Eze 19:8  Atunci s-au sculat împotriva lui neamurile din ?inuturile vecine, ?i-au întins cursele lor împotriva lui ?i el a fost prins în groapa lor;
Eze 19:9  Atunci l-au pus în lan?uri într-o cu?ca ?i l-au dus la regele Babilonului; apoi l-au vârât într-o cetate, ca sa nu se mai auda glasul lui prin mun?ii lui Israel.
Eze 19:10  Mama ta a fost ca o vi?a de vie, rasadita lânga apa; ramuroasa ?i roditoare a fost ea din pricina bel?ugului de apa.
Eze 19:11  ?i ea avea coarde tari, pentru sceptre regale, ?i ?i-a ridicat trunchiul sau sus printre ramuri dese ?i atragea privirile cu înal?imea ei ?i mul?imea ramurilor ei.
Eze 19:12  Dar a fost rupta cu mânie ?i aruncata la pamânt, vântul de rasarit a uscat rodul ei; ramurile ei cele puternice au fost rupte, s-au uscat ?i le-a mistuit focul.
Eze 19:13  Iar acum ea a fost sadita în pustiu, într-un pamânt sec ?i însetat.
Eze 19:14  Din trunchiul ce poarta ramurile ei a ie?it foc, a mistuit rodul ei ?i nu au mai ramas în ea coarde puternice pentru sceptrele regale. Aceasta este cântare de jale ?i de jale va ramâne".
Eze 20:1  În ziua a zecea a lunii a cincea din anul al ?aptelea al robiei lui Ioiachim, au venit ni?te barba?i dintre batrânii lui Israel sa întrebe pe Domnul ?i au ?ezut înaintea fe?ei mele.
Eze 20:2  Atunci a fost cuvântul Domnului catre mine ?i mi-a zis:
Eze 20:3  "Fiul omului, vorbe?te cu batrânii lui Israel ?i spune-le: A?a zice Domnul Dumnezeu: A?i venit sa Ma întreba?i? Precum este adevarat ca Eu sunt viu, tot a?a este de adevarat ca nu voi da raspuns, zice Domnul Dumnezeu.
Eze 20:4  Vrei sa-i judeci, fiul omului? Daca vrei sa-i judeci, spune-le ticalo?iile parin?ilor lor.
Eze 20:5  ?i spune-le: A?a zice Domnul Dumnezeu: În ziua în care am ales pe Israel ?i, ridicându-Mi mâna, M-am jurat semin?iei casei lui Iacov ?i M-am descoperit lor în ?ara Egiptului ?i, ridicându-Mi mâna, le-am zis: Eu sunt Domnul Dumnezeul vostru;
Eze 20:6  În ziua aceea, ridicându-Mi mina, M-am jurat lor sa-i scot din ?ara Egiptului ?i sa-i duc în ?ara pe care o alesesem pentru ei, în care curge lapte ?i miere, cea mai frumoasa dintre toate ?arile.
Eze 20:7  ?i le-am zis: Lepada?i fiecare ticalo?iile de la ochii vo?tri ?i nu va mai întina?i cu idolii Egiptului, ca Eu sunt Domnul Dumnezeul vostru!
Eze 20:8  Dar ei s-au razvratit împotriva Mea ?i n-au vrut sa Ma asculte; nimeni n-a lepadat ticalo?iile de la ochii sai ?i idolii Egiptului nu i-a parasit. Atunci am zis: Voi varsa asupra lor mânia Mea ?i urgia Mea o voi de?erta peste ei în mijlocul ?arii Egiptului.
Eze 20:9  Însa, daca i-am scos din ?ara Egiptului, am facut aceasta pentru ca numele Meu sa nu fie pângarit în ochii neamurilor printre care se aflau ei ?i înaintea carora Ma descoperisem, ca sa-i scot din ?ara Egiptului.
Eze 20:10  Sco?ându-i din ?ara Egiptului, i-am adus în pustiu.
Eze 20:11  ?i le-am dat legile Mele, le-am aratat rânduielile Mele, prin care omul, care le va ?ine, va trai.
Eze 20:12  De asemenea le-am dat ?i zilele Mele de odihna, ca sa fie semn între Mine ?i ei, ca sa cunoasca ei ca Eu sunt Domnul, Sfin?itorul lor.
Eze 20:13  Însa casa lui Israel s-a razvratit împotriva Mea în pustiu; dupa legile Mele n-a umblat ?i a lepadat rânduielile Mele, pe care omul trebuie sa le împlineasca, ca sa traiasca prin ele, ?i zilele Mele de odihna le-au calcat. Atunci Eu am gândit sa revars asupra lor mânia Mea în pustiu, ca sa-i pierd.
Eze 20:14  Dar Eu am facut ca numele Meu sa nu fie pângarit în ochii neamurilor, la vederea carora îi scosesem.
Eze 20:15  Ba ridicându-Mi chiar mâna asupra lor în pustiu, M-am jurat ca nu-i voi duce în ?ara ce rânduisem, în care curge lapte ?i miere ?i este cea mai frumoasa din toate ?arile,
Eze 20:16  Pentru ca ei lepadasera a?ezamintele Mele ?i dupa poruncile Mele nu se purtau, ci calcau zilele Mele de odihna, ca inima lor era îndreptata spre idolii lor.
Eze 20:17  Dar ochiul Meu nu s-a îndurat sa-i piarda ?i nu i-am stârpit în pustiu.
Eze 20:18  Am zis catre fiii lor în pustiu: Nu va purta?i dupa rânduielile parin?ilor vo?tri ?i obiceiurile lor sa nu le pazi?i, nici sa nu va întina?i cu idolii lor.
Eze 20:19  Eu sunt Domnul Dumnezeul vostru; purta?i-va dupa poruncile Mele ?i hotarârile Mele pazi?i-le ?i le împlini?i.
Eze 20:20  Cinsti?i zilele Mele de odihna, ca sa fie semn între Mine ?i voi, ca sa ?ti?i ca Eu sunt Domnul Dumnezeul vostru.
Eze 20:21  Dar ?i fiii s-au razvratit împotriva Mea; nici ei nu s-au purtat dupa rânduielile Mele ?i legile Mele nu le-au pazit; n-au împlinit ceea ce omul trebuie sa împlineasca, ca sa fie viu; au calcat zilele Mele de odihna ?i atunci am gândit sa vars peste ei urgia Mea ?i sa sting mânia Mea împotriva lor în pustiu;
Eze 20:22  Dar iara?i Mi-am tras mâna înapoi ?i am facut ca numele Meu sa nu fie pângarit în ochii neamurilor, la vederea carora îi scosesem.
Eze 20:23  De asemenea Mi-am ridicat mâna în pustiu ?i M-am jurat sa-i risipesc printre popoare ?i sa-i împra?tii prin ?ari,
Eze 20:24  Pentru ca nu împlinisera rânduielile Mele, legile Mele le lepadasera ?i calcasera zilele Mele de odihna, ca ochii lor erau îndrepta?i spre idolii parin?ilor lor.
Eze 20:25  Ba înca le-am dat ?i legi care nu erau bune ?i rânduieli prin care ei nu puteau trai;
Eze 20:26  ?i i-am întinat prin ofrandele lor, facându-i sa jertfeasca pe to?i întâi-nascu?ii lor, pentru a-i pedepsi ca sa ?tie ca Eu sunt Domnul.
Eze 20:27  De aceea, fiul omului, vorbe?te casei lui Israel ?i le spune: A?a zice Domnul Dumnezeu: Iata cu ce M-au mai hulit înca parin?ii vo?tri, purtându-se cu necredincio?ie fa?a de Mine:
Eze 20:28  Eu i-am adus în ?ara pe care cu juramânt fagaduisem sa le-o dau, iar ei, punându-?i ochii pe toata colina înalta ?i pe tot arborele umbros, au început sa junghie acolo jertfele lor ?i au pus acolo prinoasele lor ?i miresmele lor de tamâiere cele jignitoare pentru Mine ?i au savâr?it acolo jertfele lor cu turnare.
Eze 20:29  Atunci le-am zis: Ce este locul înalt unde va duce?i voi? ?i ei l-au numit cu numele Bama (înal?ime) pâna în ziua de astazi.
Eze 20:30  Pentru aceea, zi casei lui Israel: Acestea zice Domnul Dumnezeu: Nu va întina?i oare dupa pilda parin?ilor vo?tri ?i oare nu va desfrâna?i dupa ticalo?iile lor?
Eze 20:31  Adunându-va darurile ?i trecând copiii vo?tri prin foc, va pângari?i cu to?i idolii vo?tri pâna în ziua de astazi. ?i mai voi?i înca sa Ma întreba?i, casa lui Israel? Precum este adevarat ca Eu sunt viu, zice Domnul Dumnezeu, tot a?a este de adevarat ca nu va voi da raspuns.
Eze 20:32  Ceea ce va framânta mintea nu se va împlini nicidecum. Voi zice?i: Vom sluji lemnului ?i pietrei, ca neamurile, ca triburile din ?arile straine.
Eze 20:33  Cum este adevarat ca Eu sunt viu, tot a?a este de adevarat ca, zice Domnul Dumnezeu, cu mâna tare, cu bra? ridicat ?i cu varsarea urgiei Mele voi domni peste voi.
Eze 20:34  Va voi scoate din mijlocul popoarelor ?i va voi aduna din ?arile straine unde a?i fost împra?tia?i ?i va voi aduna cu mâna tare, cu bra? ridicat ?i cu varsarea mâniei,
Eze 20:35  ?i va voi aduce în pustiul popoarelor ?i acolo Ma voi judeca cu voi, fa?a catre fa?a.
Eze 20:36  Cum M-am judecat cu parin?ii vo?tri în pustiul din ?ara Egiptului, a?a Ma voi judeca ?i cu voi, zice Domnul Dumnezeu.
Eze 20:37  Va voi trece sub toiag ?i va voi vârî în legaturile a?ezamântului.
Eze 20:38  Voi alege din voi pe razvrati?i, pe cei nesupu?i Mie; îi voi scoate din ?ara unde sala?luiesc ei, dar în ?ara lui Israel nu vor intra ?i ve?i ?ti ca Eu sunt Domnul.
Eze 20:39  Iar voi, casa lui Israel - a?a zice Domnul Dumnezeu - duce?i-va fiecare la idolii sai ?i le sluji?i, daca nu Ma asculta?i pe Mine, dar nu mai pângari?i numele Meu cel sfânt cu darurile voastre ?i cu idolii vo?tri.
Eze 20:40  Pentru ca pe muntele Meu cel sfânt, pe muntele cel înalt al lui Israel, - zice Domnul Dumnezeu - acolo îmi va sluji toata casa lui Israel, toata, oricâta ar fi ea pe pamânt; acolo îi voi primi cu bunavoin?a ?i acolo voi cere prinoasele voastre ?i pârgile voastre cu toate cele sfinte ale voastre.
Eze 20:41  Va voi primi ca pe ni?te tamâie mirositoare, când va voi scoate din mijlocul popoarelor ?i va voi aduna din ?arile straine, unde a?i fost împra?tia?i ?i Ma voi sfin?i întru voi înaintea ochilor neamurilor.
Eze 20:42  ?i ve?i ?ti atunci ca Eu sunt Domnul, când va voi duce în ?ara lui Israel, în ?ara pe care am jurat sa o dau parin?ilor vo?tri, ridicându-Mi mâna.
Eze 20:43  Va ve?i aminti acolo de caile voastre ?i de toate faptele voastre, cu care v-a?i întinat ?i va ve?i dezgusta singuri de toate nelegiuirile voastre, pe care le-a?i facut.
Eze 20:44  Ve?i ?ti ca Eu sunt Domnul, când voi face cu voi dupa numele Meu, nu dupa caile voastre cele rele, nici dupa faptele voastre cele stricate, casa lui Israel", zice Domnul Dumnezeu.
Eze 20:45  ?i a fost iar cuvântul Domnului catre mine:
Eze 20:46  "Fiul omului, întoarce-?i fa?a spre Teman ?i roste?te cuvântul tau spre miazazi ?i prooroce?te împotriva padurii din Negheb.
Eze 20:47  ?i zi padurii celei de la miazazi: Asculta cuvântul Domnului: Asa zice Domnul Dumnezeu: Iata Eu voi aprinde foc în tine ?i va arde în tine tot copacul verde ?i tot lemnul uscat; ?i flacara vâlvâitoare nu se va stinge ?i toata fa?a va fi arsa de ea de la Negheb pâna la miazanoapte.
Eze 20:48  ?i va vedea tot trupul ca Eu, Domnul, am aprins focul ?i nu se va stinge".
Eze 20:49  ?i am zis: "0, Doamne, Dumnezeule, ei zic despre mine: "Nu cumva ne spune el o pilda?"
Eze 21:1  Fost-a iara?i cuvântul Domnului catre mine ?i mi-a zis:
Eze 21:2  "Fiul omului, întoarce-?i fa?a spre Ierusalim ?i vorbe?te împotriva loca?ului lor cel sfânt ?i prooroce?te împotriva ?arii lui Israel.
Eze 21:3  ?i zi pamântului lui Israel: A?a graie?te Domnul Dumnezeu: Iata, Eu sunt asupra ta ?i-Mi voi trage sabia din teaca ei ?i voi stârpi din tine pe cel drept ?i pe cel necredincios.
Eze 21:4  ?i ca sa pierd din tine pe cel drept ?i pe cel necredincios, sabia Mea va ie?i din teaca sa spre a lovi tot trupul de la miazazi pâna la miazanoapte.
Eze 21:5  Atunci va cunoa?te tot trupul ca Eu, Domnul, Mi-am scos sabia din teaca ei ?i nu se va mai întoarce.
Eze 21:6  Tu, fiul omului, suspina atât încât sa se frânga ?alele tale, suspina cu amaraciune înaintea ochilor lor.
Eze 21:7  Iar când î?i vor zice: "Pentru ce suspini?" Spune-le: Pentru vestea ce vine... ?i toata inima se va tulbura, toate mâinile vor slabi ?i to?i genunchii vor tremura ca frunza. Iat-o vine ?i este aproape, zice Domnul Dumnezeu.
Eze 21:8  ?i a mai fost cuvântul Domnului catre mine ?i mi-a zis:
Eze 21:9  "Fiul omului, prooroce?te ?i zi: A?a graie?te Domnul Dumnezeu: "Sabia este ascu?ita ?i o?elita.
Eze 21:10  E ascu?ita ca sa junghie mai mult; e o?elita ca sa scânteieze ca fulgerul. Ne vom bucura, oare, ca sabia fiului meu dispre?uie?te tot lemnul?
Eze 21:11  Eu am dat-o la o?elit, ca sa se ia în mâna, ?i acum aceasta sabie e ascu?ita ?i o?elita, gata sa fie data în mâna ucigatorului.
Eze 21:12  Fiul omului, suspina ?i te vaita, ca ea e trasa asupra poporului Meu, asupra tuturor capeteniilor lui Israel; ace?tia vor fi da?i sub sabie cu poporul Meu, de aceea love?te-te cu mâinile peste coapse,
Eze 21:13  Caci ea e ?i încercata acum; ?i ce este de mirare daca ea va nesocoti ?i sceptrul? Nici acesta nu va ramâne, zice Domnul Dumnezeu.
Eze 21:14  Dar tu, fiul omului, prooroce?te ?i love?te palmele una de alta ?i loviturile sabiei se vor îndoi ?i se vor întrei; aceasta este sabia macelului, sabia cumplitului macel, sabia care trebuie sa-i urmareasca,
Eze 21:15  Pentru a arunca groaza în inimi ?i a înmul?i jertfele tot mai mult. Vai! La toate por?ile lor voi pune sabie grozava ?i ascu?ita spre junghiere, care scânteiaza ca fulgerul.
Eze 21:16  Sabie, pregate?te-te! Ia-o la dreapta ?i apuca la stânga, încotro vrei sa-?i întorci fa?a ta!
Eze 21:17  Iar Eu voi bate din palme ?i-Mi voi potoli mânia. Eu, Domnul, graiesc acestea".
Eze 21:18  Fost-a cuvântul Domnului catre mine ?i mi-a zis:
Eze 21:19  "Tu, fiul omului, închipuie?te-?i doua drumuri pe care trebuie sa treaca sabia regelui Babilonului; acestea amândoua trebuie sa plece din aceea?i ?ara; apoi închipuie?te-?i o mâna, închipuie?te-o în cetatea Babilonului, de unde pleaca drumurile.
Eze 21:20  Închipuie?te drumul pe care sabia trebuie sa vina împotriva ceta?ii Raba, a fiilor lui Amon ?i împotriva lui Iuda, împotriva Ierusalimului celui întarit,
Eze 21:21  Pentru ca regele Babilonului s-a oprit la o raspântie, unde încep doua drumuri, ?i sta sa ghiceasca: scutura sage?ile, întreaba terafimii ?i cerceteaza ficatul.
Eze 21:22  Sor?ul din dreapta are scris pe el: "spre Ierusalim", încât el trebuie sa îndrepte berbecii, sa îndemne la ucidere ?i sa scoata strigate de razboi; sa a?eze berbecii împotriva por?ilor, sa ridice valuri ?i sa faca turnuri de împresurare.
Eze 21:23  Acest sor? s-a parut neadevarat în ochii acelora care au facut juraminte mincinoase; dar regele Babilonului, aducându-?i aminte de aceasta necredin?a a lor, a hotarât sa ia Ierusalimul.
Eze 21:24  De aceea a?a zice Domnul Dumnezeu: De vreme ce voi va aduce?i aminte de nelegiuirea voastra, facând ca faradelegile voastre sa fie vadite ?i sco?ând la iveala pacatele voastre în toate faptele voastre; de vreme ce voi singuri va aduce?i aminte de acestea, ve?i fi prin?i cu mâna.
Eze 21:25  ?i ?ie, capetenie nelegiuita ?i rea a lui Israel, careia ?i-a venit ziua acum, când nelegiuirea ta a ajuns la culme,
Eze 21:26  A?a zice Domnul Dumnezeu: Diadema se va scoate, cununa va fi ridicata, lucrurile se vor schimba; cele smerite se vor înal?a ?i cele înalte se vor smeri;
Eze 21:27  O voi lepada, o voi lepada, o voi lepada ?i nu va mai fi pâna va veni acela caruia se cuvine ?i o voi da lui.
Eze 21:28  Iar tu, fiul omului, prooroce?te ?i spune: A?a graie?te Domnul Dumnezeu despre fiii lui Amon ?i despre ocara lor. ?i le spune: Sabia, sabia este trasa pentru junghiere, este o?elita pentru macel, ca sa scânteieze ca fulgerul.
Eze 21:29  În mijlocul acelor vedenii de?arte ?i prezicerilor tale mincinoase, ea te va face sa cazi printre trupurile necredincio?ilor, carora le-a venit ziua atunci când nedreptatea ?i-a ajuns culmea.
Eze 21:30  O voi întoarce oare în teaca ei? La locul unde ai fost facut, în ?ara unde te-ai nascut, te voi judeca;
Eze 21:31  Voi varsa asupra ta mânia Mea; voi sufla asupra ta focul urgiei Mele ?i te voi da în mâna oamenilor barbari, fauritori ai distrugerii.
Eze 21:32  Mâncare focului vei fi; sângele tau va curge în mijlocul ?arii ?i nici nu se va mai pomeni de tine, ca Eu, Domnul, am spus acestea".
Eze 22:1  Fost-a cuvântul Domnului catre mine:
Eze 22:2  "Tu, fiul omului, voie?ti oare sa judeci, sa judeci cetatea sângelui? Spune-i toate urâciunile ei,
Eze 22:3  ?i-i spune: A?a zice Domnul Dumnezeu: O, cetate, care ver?i sânge în mijlocul tau, ca sa-?i vina vremea, ?i care-?i faci idoli, ca sa te întinezi!
Eze 22:4  Cu sângele, pe care l-ai varsat, te-ai facut vinovata ?i cu idolii, pe care i-ai facut, te-ai spurcat ?i ?i-ai apropiat zilele tale ?i ai ajuns la sfâr?itul anilor tai. De aceea te voi da popoarelor spre batjocura ?i tuturor ?arilor spre bataie de joc.
Eze 22:5  Cei de aproape ?i cei de departe ai tai î?i vor bate joc de tine, care ?i-ai întinat numele ?i e?ti plina de neorânduiala.
Eze 22:6  Iata cei ce pova?uiesc în Israel, fiecare dupa masura puterilor sale, au fost întru tine ca sa verse sânge.
Eze 22:7  La tine, tatal ?i mama sunt dispre?ui?i, la tine strainul este chinuit, la tine orfanul ?i vaduva sunt asupri?i.
Eze 22:8  Cele sfinte ale tale, tu nu le cinste?ti ?i zilele Mele de odihna le calci.
Eze 22:9  Clevetitori sunt destui în tine, ca sa verse sânge; în mun?ii tai se manânca jertfe idole?ti ?i în mijlocul tau se fac urâciuni.
Eze 22:10  La tine se descopera goliciunea tatalui, la tine se siluie?te femeia în vremea perioadei ei de necura?ie.
Eze 22:11  Unul face ticalo?ie cu femeia vecinului sau, altul se spurca cu nora sa; unul siluie?te pe sora sa, fiica tatalui sau.
Eze 22:12  În tine se ia mita, ca sa se verse sânge; tu iei dobânda ?i camata ?i cu silnicie apuci câ?tig de la fratele tau, iar pe Mine M-ai uitat, zice Domnul Dumnezeu.
Eze 22:13  ?i iata, Eu Mi-am lovit palmele Mele una de alta, vazând lacomia ta, care se vede la tine, ?i varsarea de sânge, care se savâr?e?te în mijlocul tau.
Eze 22:14  Va suferi oare inima ta ?i vor fi oare tari mâinile tale în acele zile, când voi lucra împotriva ta? Eu, Domnul, am zis ?i voi face.
Eze 22:15  Te voi împra?tia printre popoare ?i te voi vântura prin ?ari, voi pune capat urâciunilor tale cele din tine.
Eze 22:16  Singura te vei face dispre?uita înaintea ochilor popoarelor ?i vei cunoa?te ca Eu sunt Domnul".
Eze 22:17  ?i a fost cuvântul Domnului catre mine:
Eze 22:18  "Fiul omului, casa lui Israel Mi s-a facut zgura; to?i sunt plumb ?i fier ?i cositor în cuptor, au ajuns ca ni?te zgura de argint.
Eze 22:19  De aceea, a?a zice Domnul Dumnezeu: De vreme ce to?i v-a?i facut zgura, de aceea iata Eu va voi aduna în Ierusalim.
Eze 22:20  Dupa cum se pune în cuptor la un loc argint ?i arama ?i fier ?i cositor ?i plumb, ca sa aprinda asupra lor foc ?i sa le topeasca,
Eze 22:21  Va voi aduna ?i voi aprinde asupra voastra focul mâniei Mele ?i va voi topi în mijlocul ceta?ii.
Eze 22:22  Cum se tope?te argintul în cuptor, a?a va ve?i topi ?i voi în mijlocul ceta?ii, ?i ve?i ?ti ca Eu, Domnul, am varsat asupra voastra urgia Mea".
Eze 22:23  Fost-a cuvântul Domnului catre mine ?i mi-a zis:
Eze 22:24  "Fiul omului, spune Ierusalimului: E?ti o ?ara necura?ita ?i neudata de ploaie în ziua mâniei.
Eze 22:25  Mai-marii lui urzesc în el intrigi; ca ni?te lei ce mugesc ?i sfâ?ie prada, a?a manânca ei sufletele; strâng avu?ii ?i lucruri pre?ioase ?i sporesc numarul vaduvelor.
Eze 22:26  Preo?ii lui calca legea Mea ?i pângaresc lucrurile sfinte ale Mele; nu osebesc ce este sfânt de ce nu este sfânt ?i nu fac deosebire între curat ?i necurat; de la zilele Mele de odihna ?i-au întors ochii, ?i Eu sunt înjosit de catre ace?tia.
Eze 22:27  Capeteniile lui în sânul lui sunt ca ni?te lupi care sfâ?ie prada; ei varsa sânge ?i ucid sufletele, ca sa-?i sature lacomia.
Eze 22:28  Proorocii lui tencuiesc toate cu ipsos, cu vedenii de?arte ?i cu prevestiri mincinoase, zicând: "A?a graie?te Domnul Dumnezeu", când Domnul nu le graie?te nimic.
Eze 22:29  Iar poporul savâr?e?te silnicii ?i jaf; asupre?te pe sarac ?i pe cel în necaz, chinuie?te pe strain fara nici un drept.
Eze 22:30  Am cautat printre ei sa gasesc un om ca sa se poarte cu dreptate ?i sa stea înaintea fe?ei Mele, pentru ?ara aceasta, ca sa nu o pierd, ?i nu am gasit.
Eze 22:31  Deci voi varsa asupra lor mânia Mea; cu focul urgiei Mele îi voi pierde ?i purtarea lor o voi întoarce asupra capului lor", zice Domnul Dumnezeu.
Eze 23:1  ?i a fost iara?i cuvântul Domnului catre mine ?i mi-a zis:
Eze 23:2  "Fiul omului, au fost odata doua femei, fiicele unei mame;
Eze 23:3  ?i acestea s-au desfrânat în tinere?ea lor, s-au desfrânat în Egipt. Acolo sânul lor a fost atins ?i trupul lor feciorelnic acolo a fost pângarit.
Eze 23:4  Numele lor erau: al celei mai mari Ohola ?i al surorii ei Oholiba. Ele erau ale Mele ?i au nascut fii ?i fiice. Ohola este Samaria ?i Oholiba este Ierusalimul.
Eze 23:5  Ohola a început sa-Mi fie necredincioasa ?i s-a aprins dupa aman?ii ei, dupa Asirieni, vecinii sai,
Eze 23:6  Dregatori ?i capetenii de ceta?i, to?i tineri ?i frumo?i, îmbraca?i în purpura ?i calare?i iscusi?i.
Eze 23:7  Ea ?i-a facut placerile cu aceia, care erau frunta?ii Asirienilor, ?i cu to?i aceia dupa care înnebunea, spurcându-se cu to?i idolii lor.
Eze 23:8  Dar ea n-a încetat a se desfrâna ?i cu Egiptenii, caci ace?tia dormisera cu ea în tinere?ile ei, atinsesera sânul ei fecioresc ?i î?i varsasera desfrânarea lor peste dânsa.
Eze 23:9  De aceea am dat-o ?i în mâna aman?ilor ei, în mâinile Asirienilor, dupa care înnebunea.
Eze 23:10  Ace?tia au dezvelit ru?inea ei ?i au luat pe fiii ei ?i pe fiicele ei, iar pe ea au ucis-o cu sabia. ?i a ajuns ea de ocara între femei, dupa ce a fost osândita.
Eze 23:11  Sora ei, Oholiba, a vazut aceasta ?i a fost ?i mai stricata în poftele sale ?i desfrânarile ei au întrecut pe ale surorii sale.
Eze 23:12  Ea s-a aprins dupa fiii lui Asur, vecinii ei, dupa dregatori ?i capetenii de ceta?i, to?i tineri ale?i, îmbraca?i frumos ?i calare?i iscusi?i.
Eze 23:13  Am vazut deci ca s-a pângarit ?i aceasta; ca ?i sora sa, urmând amândoua aceea?i cale.
Eze 23:14  Dar aceasta a mers ?i mai departe cu desfrânarea, pentru ca vazând zugravite pe pere?i chipuri de barba?i, chipuri de caldei, zugravi?i cu vopsele,
Eze 23:15  Încin?i peste mijloc cu centuri ?i pe cap cu turbane largi, având înfa?i?area de mari viteji, dupa felul Babilonenilor, a caror patrie este ?ara Caldeilor,
Eze 23:16  Ea s-a aprins dupa ei la cea dintâi privire a lor ?i a trimis soli la ei, în Caldeea.
Eze 23:17  ?i au venit fiii Babilonului la aceasta, în palatul ei de desfrânare, ?i au pângarit-o cu desfrânarile lor ?i ea s-a pângarit cu ei ?i apoi sufletul sau s-a departat de ei.
Eze 23:18  Iar când ea s-a dedat pe fa?a la desfrânarile sale ?i ?i-a descoperit goliciunea sa, atunci s-a departat inima Mea ?i de ea, cum se departase inima Mea ?i de sora ei;
Eze 23:19  Caci ea a sporit desfrânarile sale, aducându-?i aminte de zilele tinere?ii sale, când se desfrânase în ?ara Egiptului;
Eze 23:20  ?i s-a aprins dupa aman?ii ei, cei cu trup ca de magar ?i cu înfierbântarea ca de armasar.
Eze 23:21  ?i a?a ?i-ai adus tu aminte de desfrânarea din tinere?ile tale, când Egiptenii î?i strângeau sânii feciorelnici ai pieptului tau.
Eze 23:22  De aceea, Oholiba, a?a zice Domnul Dumnezeu: Iata, Eu voi a?â?a împotriva ta pe aman?ii tai de. care s-a dezgustat sufletul tau ?i-i voi aduce împotriva ta din toate par?ile.
Eze 23:23  Voi aduce pe fiii Babilonului ?i pe to?i Caldeii: din Pecod, din ?oa ?i din Coa ?i, împreuna cu ei, pe to?i Asirienii, tineri frumo?i, capetenii de provincii, capetenii de ceta?i, slujitori ?i oameni vesti?i; to?i calare?i iscusi?i.
Eze 23:24  ?i vor veni împotriva ta cu arme, cu cai ?i cu caru?e ?i cu mul?ime de popor ?i te vor înconjura din toate par?ile, cu suli?e, cu sabii ?i cu scuturi; ?i te voi da lor spre judecata, ?i cu judecata lor te vor judeca.
Eze 23:25  ?i voi întoarce zelul Meu împotriva ta, ?i ei se vor purta cu tine cu urgie; î?i vor taia nasul ?i urechile, iar celelalte ale tale vor cadea de sabie. Lua-vor pe fiii ?i pe fiicele tale, iar celelalte ale tale de foc vor fi mâncate.
Eze 23:26  Vor lua de pe tine hainele tale, vor smulge gatelile tale,
Eze 23:27  ?i voi pune capat desfrânarilor tale ?i destrabalarilor tale aduse din ?ara Egiptului, iar tu nu-?i vei mai întoarce ochii spre ei ?i de Egipt nu-?i vei mai aminti,
Eze 23:28  Ca a?a zice Domnul Dumnezeu: Iata, te dau în mâinile acelora pe care i-ai urât, în mâinile acelora de la care s-a întors sufletul tau.
Eze 23:29  Aceia se vor purta cu tine crud ?i-?i vor lua tot ce ai dobândit cu osteneala; te vor lasa goala, cu totul goala, ?i descoperita va fi goliciunea ta cea ru?inoasa ?i desfrânarea ta ?i pângarirea.
Eze 23:30  Acestea ?i se vor face pentru desfrânarea ta cu popoarele, cu idolii cu care te-ai pângarit.
Eze 23:31  Ai mers pe calea surorii tale ?i de aceea î?i voi da în mâna ?i paharul ei.
Eze 23:32  A?a zice Domnul Dumnezeu: Vei bea paharul surorii tale, cel adânc ?i larg, ?i vei ajunge de râs ?i de batjocura, caci este mare paharul acesta.
Eze 23:33  Cuprinsa vei fi de be?ie ?i de amaraciune, ca acesta este paharul groazei ?i al pustiirii, paharul surorii tale, Samaria.
Eze 23:34  ?i tu îl vei bea, îl vei goli ?i cioburile lui le vei roade ?i-?i vei sfâ?ia pieptul; ca Eu am spus aceasta, zice Domnul Dumnezeu.
Eze 23:35  De aceea, a?a zice Domnul Dumnezeu: Pentru ca M-ai uitat ?i te-ai întors de la Mine, de aceea sufera ?i tu pentru nelegiuirea ta ?i pentru desfrânarea ta".
Eze 23:36  Zis-a Domnul catre mine: "Fiul omului, vrei tu sa judeci pe Ohola ?i pe Oholiba? Spune-le ticalo?iile lor!
Eze 23:37  Caci ele s-au desfrânat ?i pe mâinile lor au sânge; s-au desfrânat cu idolii lor, ?i pe fiii lor, care Mi i-au nascut, i-au trecut prin foc, ca sa le fie de mâncare.
Eze 23:38  Iata ce Mi-au mai facut ele: Au pângarit loca?ul Meu cel sfânt în aceea?i zi ?i zilele Mele de odihna le-au calcat;
Eze 23:39  Pentru ca atunci când ele au junghiat copiii lor pentru idolii lor, în aceea?i zi au venit în loca?ul Meu cel sfânt ca sa-l pângareasca; iata cum s-au purtat ele în loca?ul Meu!
Eze 23:40  Afara de aceasta, ele umblau dupa oameni veni?i de departe; trimiteau la ei soli ?i ei veneau. Pentru ei te-ai îmbaiat, ?i-ai facut ochii cu dresuri ?i te-ai gatit cu podoabe.
Eze 23:41  Te-ai a?ezat pe pat luxos, în fa?a caruia era gatita o masa pe care tu ai pus tamâia Mea ?i untdelemnul Meu.
Eze 23:42  ?i rasunau strigatele de veselie ale mul?imii nepasatoare, din pricina gloatei de oameni adu?i din pustiu; ?i ei au pus bra?ari la mâinile femeilor ?i coroane mare?e pe capetele lor.
Eze 23:43  Atunci am zis despre cea îmbatrânita în desfrânari: Acum vor sfâr?i desfrânarile ei o data cu ea.
Eze 23:44  ?i au venit la ea ca la o desfrânata; a?a veneau la aceste femei desfrânate, adica la Ohola ?i la Oholiba.
Eze 23:45  Dar barba?i în?elep?i le vor judeca ?i le or osândi cu osânda desfrânatelor ?i cu osânda celor ce varsa sânge, pentru ca ele sunt ni?te desfrânate ?i mâinile lor sunt însângerate.
Eze 23:46  Caci a?a zice Domnul Dumnezeu: Sa se adune împotriva lor mul?imea ?i sa le dea urgiei ?i jafului.
Eze 23:47  Mul?imea le va ucide cu pietre ?i le va taia cu sabiile; vor omorî pe fiii lor ?i pe fiicele lor ?i casele lor le vor arde cu foc.
Eze 23:48  A?a voi pune capat desfrânarii în ?ara aceasta ?i toate femeile vor lua înva?atura ?i nu vor mai face lucruri ru?inoase ca ele.
Eze 23:49  Nelegiuirea voastra va cadea asupra voastra, ?i ve?i purta greutatea pacatelor voastre de închinare la idoli, ?i ve?i ?ti ca Eu sunt Domnul Dumnezeu".
Eze 24:1  Fost-a cuvântul Domnului catre mine, în anul al noualea, în luna a zecea, în ziua a zecea a lunii, ?i mi-a zis:
Eze 24:2  "Fiul omului, scrie-?i numele acestei zile, anume al acestei zile, ca chiar în ziua aceasta regele Babilonului va pa?i spre Ierusalim.
Eze 24:3  Roste?te dar pentru neamul de razvrati?i o pilda ?i le spune: A?a zice Domnul Dumnezeu: Pune un cazan ?i toarna în el apa;
Eze 24:4  Pune în el buca?i de carne, tot buca?i din cele mai bune, ?olduri ?i spete, ?i-l umple cu cele mai bune oase;
Eze 24:5  Ia ceea ce este mai bun din turma, pune lemne dedesubt, fierbe-l în clocote, a?a ca sa fiarba ?i oasele din el.
Eze 24:6  De aceea, a?a zice Domnul Dumnezeu: Vai de cetatea sângelui! Vai de caldarea în care este rugina ?i de pe care nu se mai ia rugina! Arunca?i bucata cu bucata din el ?i nu alege?i prin sor?.
Eze 24:7  Ca sângele pe care ea l-a varsat este în mijlocul ei. Ea l-a lasat pe stânca goala; nu l-a varsat pe pamânt, unde s-ar fi putut acoperi cu ?arâna.
Eze 24:8  Pentru a-Mi a?â?a mânia, pentru a Ma razbuna, am lasat sângele ei pe stânca goala, ca sa nu fie acoperit".
Eze 24:9  De aceea a?a zice Domnul Dumnezeu: "Vai de cetatea sângelui! Ca voi aprinde un foc mare.
Eze 24:10  Adu lemne, aprinde focul, fierbe carnea; lasa sa se îngroa?e tot ?i oasele sa se arda.
Eze 24:11  Pune caldarea goala pe carbuni, ca sa se încalzeasca ?i ca arama ei sa se înfierbânte ?i sa se topeasca murdaria din ea ?i toata rugina de pe ea sa se duca.
Eze 24:12  Munca va fi grea, dar rugina cea multa de pe ea nu se va duce; ?i în foc rugina va ramâne pe ea.
Eze 24:13  Murdaria ta e atât de grozava, ca oricât te voi cura?i, tu tot necurata vei fi; de acum înainte nu te vei mai curari de murdaria ta, pâna ce nu-Mi voi potoli urgia Mea asupra ta.
Eze 24:14  Eu sunt Domnul, ?i Eu spun ca vor veni acestea ?i le voi împlini, nu le voi schimba, nici nu voi cru?a ceva; dupa purtarile ?i dupa faptele tale te voi judeca", zice Domnul Dumnezeu.
Eze 24:15  Fost-a cuvântul Domnului catre mine ?i mi-a zis:
Eze 24:16  "Fiul omului, iata Eu printr-o lovitura î?i voi lua mângâierea ochilor tai; dar tu nu te tângui ?i nu plânge, ?i nici lacrimile sa nu-?i curga;
Eze 24:17  Suspina în ascuns ?i plângere pentru mor?i sa nu faci; pune-?i turbanul pe cap ?i încal?aminte în picioare; barba sa nu ?i-o acoperi ?i pâine de la straini sa nu manânci".
Eze 24:18  Dupa ce diminea?a am grait eu poporului cuvântul Domnului, seara mi-a murit femeia ?i a doua zi am facut precum mi se poruncise.
Eze 24:19  Poporul însa mi-a zis: "Pentru ce nu ne spui ce însemnatate au pentru noi cele ce le faci tu?"
Eze 24:20  ?i le-am raspuns: "A fost cuvântul Domnului catre mine ?i mi-a zis:
Eze 24:21  Spune casei lui Israel: A?a graie?te Domnul Dumnezeu: Iata loca?ul Meu cel sfânt, mândria puterilor voastre, bucuria ochilor vo?tri ?i dragostea sufletelor voastre, îl voi da spre pângarire, iax fiii ?i fiicele voastre, pe care i-a?i parasit, vor cadea de sabie;
Eze 24:22  ?i ve?i face ?i voi ceea ce fac eu: barbile nu le ve?i acoperi ?i pâine de la straini nu ve?i mânca;
Eze 24:23  Turbanele voastre le ve?i avea pe cap ?i încal?amintele voastre în picioare; nu va ve?i tângui, nici nu va ve?i plânge, ci va ve?i stinge pentru pacatele voastre ?i ve?i suspina unul catre altul.
Eze 24:24  Iezechiel va fi semn pentru voi: ceea ce el a facut, ve?i face ?i voi întocmai; ?i când se vor împlini acestea, ve?i ?ti ca Eu sunt Domnul Dumnezeu.
Eze 24:25  ?i tu, fiul omului, în ziua când le voi fi luat puterea lor, mândria lor falnica, bucuria ochilor lor, desfatarile sufletului lor, fiii ?i fiicele lor,
Eze 24:26  În ziua aceea va veni la tine cel scapat de acolo, care î?i va aduce ?tirea;
Eze 24:27  În ziua aceea se va deschide gura ta catre cel scapat ?i vei grai ?i nu vei mai fi mut ?i vei fi semn pentru ei, ?i ei vor afla ca Eu sunt Domnul".
Eze 25:1  Fost-a cuvântul Domnului catre mine ?i mi-a zis:
Eze 25:2  "Fiul omului, întoarce-?i fa?a spre fiii lui Amon ?i graie?te împotriva lor.
Eze 25:3  Spune fiilor lui Amon: Asculta?i cuvântul Domnului Dumnezeu: A?a zice Domnul Dumnezeu: Deoarece tu faci: Aha! Aha! împotriva loca?ului Meu cel sfânt când el e pângarit ?i când ?ara lui Israel e pustiita ?i casa lui Iuda dusa în robie,
Eze 25:4  De aceea, iata Eu te voi da de mo?tenire fiilor Rasaritului, care î?i vor face sala?ele la tine ?i vor a?eza la tine corturile ?i vor mânca fructele ?i laptele tau.
Eze 25:5  Voi face din Raba un staul de camile ?i din ceta?ile lui Amon o stâna de oi, ?i ve?i ?ti ca Eu sunt Domnul.
Eze 25:6  Ca a?a zice Domnul Dumnezeu: Pentru ca ai batut din palme ?i ai batut din picioare ?i din suflet te-ai bucurat cu tot dispre?ul tau pentru ?ara lui Israel,
Eze 25:7  De aceea iata Eu Îmi voi întinde mâna împotriva ta ?i te voi da popoarelor spre jefuire ?i te voi stârpi din numarul popoarelor ?i din numarul ?arilor te voi ?terge; te voi zdrobi ?i vei ?ti ca Eu sunt Domnul".
Eze 25:8  A?a zice Domnul Dumnezeu: "Pentru ca Moab ?i Seir zic: Iata ?i casa lui Iuda este ca toate neamurile;
Eze 25:9  De aceea iata Eu, începând de la ceta?i, de la toate ceta?ile lui de pe hotar, de la Bet-Ie?imot, Baal-Meon ?i Chiriataim, podoabele ?arii, voi descoperi colinele Moabului ?i voi nimici ceta?ile lui în toata întinderea lui.
Eze 25:10  ?i-l voi deschide ?i-l voi da în stapânire fiilor Rasaritului împreuna cu ?ara fiilor lui Amon, ca sa nu se mai aminteasca despre fiii lui Amon printre neamuri.
Eze 25:11  Voi savâr?i astfel judeca?i împotriva lui Moab, ?i ei vor ?ti ca Eu sunt Domnul".
Eze 25:12  A?a zice Domnul Dumnezeu: "Pentru ca Edom s-a razbunat cumplit împotriva casei lui Iuda ?i a pacatuit greu, savâr?ind razbunare asupra ei,
Eze 25:13  De aceea, a?a zice Domnul Dumnezeu: Îmi voi întinde mâna împotriva Edomului ?i voi pierde din el pe oameni ?i pe animale ?i îl voi face pustietate; de la Teman pâna la Dedan to?i vor cadea de sabie.
Eze 25:14  Razbunarea Mea împotriva Edomului o voi savâr?i prin mâna poporului Meu Israel; el va lucra în Edom dupa mânia Mea ?i dupa urgia Mea ?i vor ?ti Edomi?ii ce este razbunarea Mea", zice Domnul Dumnezeu.
Eze 25:15  A?a graie?te Domnul Dumnezeu: "Pentru ca Filistenii s-au purtat razbunatori ?i s-au razbunat cu ura în suflet ?i cu o ve?nica du?manie de moarte,
Eze 25:16  De aceea a?a zice Domnul Dumnezeu: "Iata Eu Îmi voi întinde mâna împotriva Filistenilor ?i voi pierde pe Cheretieni ?i restul locuitorilor de pe ?armul marii îl voi stârpi.
Eze 25:17  Voi savâr?i împotriva lor cumplite razbunari, pedepse grele, ?i vor ?ti ca Eu sunt Domnul, când voi savâr?i asupra lor razbunarea Mea".
Eze 26:1  În anul al unsprezecelea, în ziua întâi a lunii întâi, a fost cuvântul Domnului catre mine:
Eze 26:2  "Fiul omului, pentru ca Tirul face împotriva Ierusalimului: Aha! Aha! ?i zice: Iata el - poarta popoarelor - este darâmat; acum alearga la mine; eu ma umplu, iar el se pustie?te,
Eze 26:3  De aceea, a?a zice Domnul Dumnezeu: Iata sunt împotriva ta, Tirule, ?i voi ridica împotriva ta popoare multe, cum î?i ridica marea valurile sale.
Eze 26:4  Voi sfarâma zidurile Tirului ?i turnurile lui le voi darâma; voi matura praful din el ?i-l voi face stânca goala.
Eze 26:5  Loc de uscat mrejele va fi el la mare, pentru ca a zis acestea, graie?te Domnul Dumnezeu, ?i va fi prada neamurilor.
Eze 26:6  Iar fiicele lui, care sunt pe pamânt, vor fi ucise cu sabia ?i vor ?ti ca Eu sunt Domnul".
Eze 26:7  Ca a?a zice Domnul Dumnezeu: "Iata Eu voi aduce împotriva Tirului de la miazanoapte pe Nabucodonosor, regele Babilonului, regele regilor, cu cai, cu care ?i cu calare?i, cu o?tire ?i cu mul?ime de neamuri.
Eze 26:8  Pe fiicele tale cele din câmpie el le va ucide cu sabia ?i va ridica împotriva ta turnuri de împresurare, va face val împrejurul tau ?i va pune împotriva ta scuturile.
Eze 26:9  Spre zidurile tale va împinge berbecii de spart ziduri ?i turnurile tale le va darâma cu topoarele.
Eze 26:10  De mul?imea cailor lui vei fi acoperit de praf ?i de zgomotul calare?ilor, al carelor ?i al ro?ilor se vor cutremura zidurile tale, când va intra el pe por?ile tale, cum se intra într-o cetate sfarâmata.
Eze 26:11  Cu copitele cailor sai va calca el toate uli?ele tale, pe poporul tau îl va ucide cu sabie, iar puternicele tale columne le va rasturna la pamânt.
Eze 26:12  Vor jefui boga?ia ta ?i marfurile tale le vor fura, vor darâma zidurile tale ?i frumoasele tale case le vor strica, ?i pietrele tale ?i arborii tai ?i pamântul tau le vor arunca în apa.
Eze 26:13  Voi curma zgomotul cântecelor tale ?i sunet de chitara nu se va mai auzi la tine.
Eze 26:14  Te voi face stânca goala ?i loc de uscat mrejele vei fi; nu vei mai fi zidit din nou, caci Domnul a spus acestea", zice Domnul Dumnezeu.
Eze 26:15  A?a zice Domnul Dumnezeu Tirului: "De zgomotul caderii tale ?i de geamatul rani?ilor tai, când se va face macelul în tine, nu se vor cutremura oare insulele?
Eze 26:16  To?i stapânitorii marii se vor cobori de pe tronurile lor, î?i vor scoate purpurile ?i î?i vor dezbraca hainele lor cele brodate; cu groaza se vor îmbraca, vor ?edea la pamânt ?i vor tremura fara încetare ?i vor fi umili?i din pricina ta.
Eze 26:17  Vor ridica plângere împotriva ta ?i-?i vor zice: "Cum ai pierit tu, cel locuit de stapânitorii marilor, cetate vestita, care erai tare pe mare ?i tu ?i locuitorii tai, care aduceai groaza asupra tuturor celor care locuiau pe uscat!
Eze 26:18  Acum, în ziua caderii tale, s-au cutremurat insulele; insulele de pe mare sunt îngrozite de sfâr?itul tau".
Eze 26:19  Ca a?a zice Domnul Dumnezeu: "Când te voi face cetate pustie, asemenea ceta?ilor nelocuite, când voi ridica împotriva ta adâncul ?i te vor acoperi apele cele mari,
Eze 26:20  Atunci te voi coborî cu cei ce se coboara în mormânt, la poporul de odinioara ?i te voi a?eza în adâncurile pamântului, în pustieta?i ve?nice, cu cei ce s-au dus în mormânt, ca sa nu mai fii locuit ?i sa nu mai dainuie?ti în ?ara celor vii.
Eze 26:21  Groaza te voi face ?i nu vei mai fi; te vor cauta ?i nu te vor mai gasi în veci", zice Domnul Dumnezeu.
Eze 27:1  Fost-a cuvântul Domnului catre mine ?i mi-a zis:
Eze 27:2  "?i tu, fiul omului, ridica plângere împotriva Tirului,
Eze 27:3  ?i zi catre el: O, tu, cel ce e?ti a?ezat la marginea marii ?i faci nego? cu popoarele a nenumarate insule, a?a graie?te Domnul Dumnezeu: Tirule, tu zici: "Eu sunt o corabie de desavâr?ita frumuse?e!"
Eze 27:4  ?inutul tau este în largul marii; cei ce te-au zidit te-au facut minunat de frumos.
Eze 27:5  Toate acoperi?urile corabiilor tale le-ai facut din chiparos de Senir ?i cedru de Liban s-a adus, ca sa-?i faca ?ie catarge.
Eze 27:6  Vâslele tale s-au facut de stejar din Vasan; bancile ?i le-au facut din lemn de cim?ir, împodobite cu filde? din insulele Chitim;
Eze 27:7  Pânzele tale, din vison de Egipt brodat, î?i slujeau ca steag. Porfira violeta ?i stacojie din insulele Eli?a alcatuiau acoperamântul tau.
Eze 27:8  Locuitorii Sidonului ?i ai Arvadului erau vâsla?ii tai; ?i cei mai iscusi?i ai tai, Tirule, erau cârmaci.
Eze 27:9  Batrânii din Ghebal ?i me?terii lui erau la tine, ca sa-?i repare stricaciunile. Toate corabiile marii ?i corabiile lor erau la tine, ca sa faca nego?ul tau.
Eze 27:10  Per?i, Lidieni ?i Libieni se aflau în o?tirea ta ?i erau oamenii tai de razboi; atârnau în tine scuturile ?i coifurile lor.
Eze 27:11  Fiii Arvadului împreuna cu o?tirea ta stateau împrejur pe zidurile tale ?i în turnurile tale se aflau oameni viteji; ace?tia î?i atârnau tolbele sus pe zidurile tale ?i desavâr?eau frumuse?ea ta.
Eze 27:12  Cei din Tarsis faceau nego? cu tine pentru tot felul de boga?ii ?i veneau la târgul tau cu argint, cu fier, cu cositor ?i plumb.
Eze 27:13  Iavan, Tubal ?i Me?ec faceau nego? cu tine, dând în schimb, pe marfurile tale, suflete omene?ti ?i vase de arama, în pie?ele tale.
Eze 27:14  Cei din casa Togarma aduceau la târgul tau cai ?i caru?e.
Eze 27:15  Fiii lui Dedan faceau nego? cu tine; insule multe luau marfurile tale ?i-?i plateau cu filde? ?i abanos.
Eze 27:16  Pentru mul?imea marfurilor tale facea nego? cu tine Siria ?i venea la târgul tau cu smaralde, cu purpura, cu ?esaturi alese, cu în sub?ire, cu margean ?i cu rubine.
Eze 27:17  Iuda ?i ?inuturile lui Israel faceau nego? cu tine ?i pe marfurile tale dadeau grâu de Minit ?i turte, miere, ulei ?i balsam.
Eze 27:18  Damascul facea nego? cu tine ?i, pentru mul?imea multa de lucruri ?i pentru toate bunata?ile ce aveai tu din bel?ug, î?i aducea vin de Helbon ?i lâna alba.
Eze 27:19  Vedan ?i Iavan din Uzal î?i plateau pe marfurile tale fier lucrat; casie ?i trestie mirositoare ?i se aduceau în schimb.
Eze 27:20  Dedan facea nego? cu tine cu paturi pentru pus pe cai.
Eze 27:21  Arabia ?i toate capeteniile din Chedar faceau nego? cu tine; miei, berbeci ?i ?api î?i dadeau în schimb pentru marfurile tale.
Eze 27:22  Negu?atorii din ?eba ?i Rama faceau nego? cu tine, dând în schimb tot felul de aromate alese, felurite pietre scumpe ?i aur, pentru marfurile tale.
Eze 27:23  Haran, Cane ?i Eden, negu?atorii din ?eba, Asiria ?i Chilmad faceau nego? cu tine.
Eze 27:24  Ace?tia faceau nego? cu tine cu haine scumpe, cu mantii de purpura violeta ?i brodata, stofe ?esute cu felurite culori, funii împletite ?i tari, puse în lazi de cedru, aduse pe pie?ele tale.
Eze 27:25  Corabiile Tarsisului erau caravanele tale pentru nego?ul tau ?i prin acestea ai ajuns tu bogat ?i foarte slavit pe mare.
Eze 27:26  Vâsla?ii tai te-au facut sa calatore?ti pe apele cele mari, dar un vânt de la rasarit te va sfarâma în mijlocul marilor.
Eze 27:27  Boga?ia ta ?i marfurile tale, corabierii ?i cârmacii tai, cei ce dreg crapaturile corabiilor, cei ce fac schimb de marfuri cu tine, to?i osta?ii care se afla în tine se vor prabu?i în inima marilor în ziua caderii tale.
Eze 27:28  De strigatul cârmacilor tai se vor cutremura împrejurimile.
Eze 27:29  Se vor coborî din corabiile lor to?i vâsla?ii, corabierii ?i to?i cârmacii marii vor sta pe uscat;
Eze 27:30  Vor plânge pentru tine cu mare glas, presarându-?i capetele lor cu cenu?a ?i tavalindu-se în pulbere;
Eze 27:31  Î?i vor tunde pentru tine parul pâna la piele, cu sac se vor îmbraca ?i vor plânge dupa tine cu plângere mare de durerea inimii;
Eze 27:32  ?i în durerea lor vor cânta cântare de jale pentru tine ?i te vor baci a?a: "Cine a fost ca Tirul, ca aceasta cetate darâmata în mare!"
Eze 27:33  Când veneau marfurile tale de pe mari, tu saturai popoare multe; prin mul?imea boga?iei tale ?i prin nego?ul tau îmboga?eai pe regii pamântului;
Eze 27:34  Iar când ai fost sfarâmat de mari în adâncul apelor, marfurile tale ?i tot ce se gramadea în tine au cazut împreuna cu tine.
Eze 27:35  To?i locuitorii insulelor s-au îngrozit de tine ?i regii lor s-au cutremurat ?i s-au schimbat la fa?a.
Eze 27:36  Negustorii popoarelor fluiera asupra ta. Tu ai ajuns o groaza, e?ti nimicit pentru totdeauna!
Eze 28:1  Fost-a cuvântul Domnului catre mine ?i mi-a zis:
Eze 28:2  "Fiul omului, spune celui ce domne?te în Tir: A?a zice Domnul Dumnezeu: Inima ta s-a înal?at ?i a zis: "Sunt un dumnezeu ?i stau pe scaunul lui Dumnezeu în inima marilor, dar tu, de?i nu e?ti Dumnezeu, ci om, î?i închipui în inima ta ca e?ti la fel cu Dumnezeu;
Eze 28:3  Iata, tu î?i închipui ca e?ti mai în?elept decât Daniel ?i nu sunt taine ascunse pentru tine;
Eze 28:4  Prin în?elepciunea ta ?i cu mintea ta ?i-ai agonisit boga?ie ?i ai adunat în vistieriile tale argint ?i aur;
Eze 28:5  Prin în?elepciunea ta cea mare, prin ajutorul nego?ului tau, ?i-ai sporit boga?ia ?i mintea ta s-a îngâmfat cu boga?ia ta;
Eze 28:6  De aceea, a?a zice Domnul Dumnezeu: Pentru ca tu te-ai asemanat cu Dumnezeu,
Eze 28:7  Iata, Eu voi aduce împotriva ta pe strainii cei mai rai din toate popoarele, ?i aceia î?i vor scoate sabia împotriva frumoasei tale în?elepciuni ?i vor întina stralucirea ta;
Eze 28:8  În mormânt te voi coborî ?i vei muri în inima marilor de moartea celor uci?i.
Eze 28:9  Spune-vei oare înaintea uciga?ului tau: "Eu sunt un dumnezeu", când tu e?ti un om în mâna celui care te ucide, iar nu Dumnezeu?
Eze 28:10  Vei muri de mâna strainilor, de moartea celor netaia?i împrejur, caci Eu am spus aceasta", zice Domnul Dumnezeu.
Eze 28:11  ?i a fost cuvântul Domnului catre mine ?i mi-a zis:
Eze 28:12  "Fiul omului, plânge pe regele Tirului ?i-i spune: A?a zice Domnul Dumnezeu: Tu erai pecetea desavâr?iri, deplinatatea în?elepciunii ?i cununa frumuse?ii.
Eze 28:13  Tu te aflai în Eden, în gradina lui Dumnezeu; hainele tale erau împodobite cu tot felul de pietre scumpe: cu rubine, topaze ?i diamante, cu crisolit, onix ?i iaspis, cu safir, smarald, carbuncul ?i aur; toate erau pregatite ?i a?ezate cu iscusin?a în cuibule?e ?i puse pe tine în ziua în care ai fost facut.
Eze 28:14  Tu erai heruvimul pus ca sa ocrote?ti; te a?ezasem pe muntele cel sfânt al lui Dumnezeu, ?i umblai prin mijlocul pietrelor celor de foc.
Eze 28:15  Fost-ai fara prihana în caile tale din ziua facerii tale ?i pâna s-a încuibat în tine nelegiuirea.
Eze 28:16  Din pricina întinderii nego?ului tau, launtrul tau s-a umplut de nedreptate ?i ai pacatuit, ?i Eu te-am izgonit pe tine, heruvim ocrotitor, din pietrele cele scânteietoare ?i te-am aruncat din muntele lui Dumnezeu, ca pe un necurat.
Eze 28:17  Din pricina frumuse?ii tale s-a îngâmfat inima ta, ?i pentru trufia ta ?i-ai pierdut în?elepciunea. De aceea te-am aruncat la pamânt ?i te voi da înaintea regilor spre batjocura.
Eze 28:18  Prin mul?imea nelegiuirilor tale, savâr?ite în nego?ul tau nedrept, ?i-ai pângarit altarele tale; ?i Eu voi scoate din mijlocul tau foc, care te va ?i mistui; ?i te voi preface în cenu?a pe pamânt înaintea ochilor tuturor celor ce te vad.
Eze 28:19  To?i cei ce te cunosc între popoare se vor mira de tine, vei ajunge o groaza ?i în veci nu vei mai fi".
Eze 28:20  Fost-a catre mine cuvântul Domnului ?i mi-a zis:
Eze 28:21  "Fiul omului, întoarce-?i fa?a spre Sidon, prooroce?te împotriva lui ?i spune:
Eze 28:22  A?a graie?te Domnul Dumnezeu: Iata Eu sunt împotriva ta, Sidoane; Ma voi preaslavi în mijlocul tau ?i se va ?ti ca Eu sunt Domnul, când te voi judeca ?i-Mi voi arata sfin?enia Mea în mijlocul tau.
Eze 28:23  Voi trimite împotriva ta ciuma ?i varsare de sânge pe uli?ele tale ?i vor cadea uci?i în mijlocul tau de sabia care te va lovi din toate par?ile, ?i vor ?ti to?i ca Eu sunt Domnul.
Eze 28:24  ?i nu vei mai fi pentru casa lui Israel spin care rane?te ?i ciulin care sfâ?ie printre cei ce o înconjoara ?i o urasc, ?i vor ?ti to?i ca Eu sunt Domnul Dumnezeu".
Eze 28:25  A?a zice Domnul Dumnezeu: "Când voi aduna casa lui Israel din mijlocul popoarelor unde este împra?tiata ?i voi arata prin aceasta sfin?enia Mea în ochii neamurilor ?i când va locui ea în pamântul sau, pe care l-am dat robului Meu Iacov,
Eze 28:26  Ei vor locui acolo în siguran?a, î?i vor face case, vor sadi vii. Când voi face judeca?i asupra tuturor celor dimprejur care îi dispre?uiesc, vor ?ti ca Eu sunt Domnul Dumnezeul lor".
Eze 29:1  În ziua a douasprezecea a lunii a zecea din anul al zecelea dupa robirea lui Ioiachim, a fost cuvântul Domnului catre mine ?i mi-a zis:
Eze 29:2  "Fiul omului, întoarce-?i fa?a spre Faraon, regele Egiptului, ?i prooroce?te împotriva lui ?i a tot Egiptul,
Eze 29:3  ?i graie?te ?i zi: A?a zice Domnul Dumnezeu: Iata Eu sunt împotriva ta, Faraoane, rege al Egiptului, crocodilul cel mare, care stai lungit între râurile tale ?i zici: "Al meu este râul ?i eu l-am facut pentru mine".
Eze 29:4  Eu însa voi înfige cârligul în falcile tale ?i de solzii tai voi lipi pe?tii râurilor tale ?i te voi târî afara din râurile tale, cu tot pe?tele râurilor tale, care s-a lipit de solzii tai;
Eze 29:5  ?i te voi arunca în pustiu pe tine ?i tot pe?tele din râurile tale ?i vei cadea în câmpia goala ?i nu te vor lua, nici te vor ridica; te voi da mâncare fiarelor pamântului ?i pasarilor cerului.
Eze 29:6  ?i vor ?ti to?i locuitorii Egiptului ca Eu sunt Domnul; pentru ca ei au fost pentru casa lui Israel toiag de trestie.
Eze 29:7  Când ei te prindeau, tu te sfarâmai în mâinile lor ?i tu le sfâ?iai toata mâna; iar când ei se sprijineau de tine, tu te rupeai ?i le zdruncinai coapsele.
Eze 29:8  De aceea, a?a graie?te Domnul Dumnezeu: Iata, Eu voi aduce împotriva ta sabie ?i voi nimici oamenii ?i vitele din tine;
Eze 29:9  ?i va ajunge pamântul Egiptului pustiu ?i de?ert ?i vor afla ca Eu sunt Domnul. Pentru ca el zice: "Al meu este râul ?i eu l-am facut".
Eze 29:10  De aceea, iata Eu vin împotriva fluviilor tale ?i voi face pamântul Egiptului pustiu între pustiurile ce se întind de la Migdol pâna la Siena ?i pâna în hotarele Etiopiei.
Eze 29:11  Picior de om nu va trece prin el, nici picior de vita nu va trece prin el ?i nu va locui nimic în el patruzeci de ani.
Eze 29:12  Voi face pamântul Egiptului o pustietate între ?arile pustiite; ?i ceta?ile lui între ceta?ile pustiite vor fi pustii patruzeci de ani ?i pe Egipteni îi voi împra?tia printre neamuri ?i-i voi risipi prin ?ari".
Eze 29:13  A?a zice Domnul Dumnezeu: "Dupa trecerea celor patruzeci de ani, voi aduna pe Egipteni dintre neamurile printre care au fost împra?tia?i,
Eze 29:14  ?i voi aduce înapoi pe prin?ii de razboi ai Egiptului ?i-i voi a?eza iara?i în ?ara Patros, în pamântul na?terii lor, ?i vor fi acolo un regat slab.
Eze 29:15  Va fi mai slab decât celelalte regate ?i nu se vor mai înal?a peste popoare; îl voi mai mic?ora, ca sa nu mai domneasca peste popoare;
Eze 29:16  ?i nu va mai fi de acum înainte pentru casa lui Israel pricina de încredere, ci îi va aduce aminte de nelegiuirea ei, ca s-a dat de partea Egiptului, ?i vor cunoa?te ca Eu sunt Domnul".
Eze 29:17  În ziua întâi a lunii întâi din anul al douazeci ?i ?aptelea de la robirea lui Ioiachim, a fost cuvântul Domnului catre mine ?i mi-a zis:
Eze 29:18  "Fiul omului, Nabucodonosor, regele Babilonului, ?i-a obosit o?tirile sale printr-o munca grea împotriva Tirului; toate capetele s-au ple?uvit ?i to?i umerii sunt rani?i; dar nici pentru el, nici pentru o?tirile lui nu este nici o rasplata de la Tir pentru lucrarea pe care a facut-o împotriva lui.
Eze 29:19  De aceea, a?a zice Domnul Dumnezeu: Iata, Eu dau lui Nabucodonosor, regele Babilonului, ?ara Egiptului, ca sa prade boga?ia lui ?i sa faca jaf în el; aceasta va fi rasplata o?tirilor lui.
Eze 29:20  Ca rasplata pentru lucrarea pe care a facut-o Tirului, Eu îi dau ?ara Egiptului, pentru ca acest lucru l-a facut el pentru Mine, zice Domnul Dumnezeu.
Eze 29:21  În ziua aceea voi face sa creasca cornul casei lui Israel ?i ?ie-?i voi deschide gura în mijlocul lor ?i vor ?ti ca Eu sunt Domnul".
Eze 30:1  Fost-a cuvântul Domnului catre mine ?i mi-a zis:
Eze 30:2  "Fiul omului, prooroce?te ?i zi: A?a graie?te Domnul Dumnezeu: Plânge?i! O, ce zi!
Eze 30:3  Ca se apropie ziua, se apropie ziua Domnului, ziua cea întunecata! Vine vremea neamurilor.
Eze 30:4  Atunci se va duce sabia în Egipt ?i groaza se va la?i în Etiopia; vor cadea în Egipt cei lovi?i ?i boga?iile lui se vor lua ?i vor fi darâmate temeliile lui.
Eze 30:5  Etiopia, Libia, Lidia, straini de toate neamurile, Cub ?i fiii ?arii a?ezamântului vor cadea împreuna cu ei de sabie.
Eze 30:6  A?a zice Domnul: "Vor cadea sprijinitorii Egiptului ?i îngâmfarea puterii lui se va prabu?i; de la Migdol ?i pâna la Siena vor cadea în el de sabie; a?a zice Domnul Dumnezeu.
Eze 30:7  ?i va fi el pustiu între pustiuri ?i ceta?ile lor vor face parte dintre ceta?ile pustiite.
Eze 30:8  ?i vor ?ti ca Eu sunt Domnul, când voi trimite foc asupra Egiptului ?i to?i sprijinitorii lui vor fi zdrobi?i.
Eze 30:9  În ziua aceea vor merge vestitori de la Mine pe corabii, ca sa îngrozeasca pe Etiopienii cei fara de grija, ?i se va întinde groaza la ei, ca în ziua Egiptului, caci iat-o ca vine".
Eze 30:10  A?a zice Domnul Dumnezeu: "Voi pune capat mul?imii Egiptului prin mâna lui Nabucodonosor, regele Babilonului.
Eze 30:11  El ?i, împreuna cu el, poporul lui, cel mai stra?nic dintre popoare, vor fi adu?i pentru pieirea acestei ?ari; ?i î?i vor scoate sabiile lor împotriva Egiptului ?i vor umple ?ara de uci?i.
Eze 30:12  Fluviile lor le voi usca ?i voi da pamântul în mâinile celor rai; prin mâna strainilor voi pustii ?ara ?i tot ce este în ea. Eu, Domnul, am spus acestea".
Eze 30:13  A?a zice Domnul Dumnezeu: "Pierde-voi idolii ?i pe dumnezeii cei mincino?i din Nof (Memfis). Nu va mai fi prin? în ?ara Egiptului ?i voi raspândi groaza în ?ara Egiptului.
Eze 30:14  Voi pustii Patrosul ?i foc voi trimite asupra ?oanului ?i voi rosti judecata asupra lui No (Teba).
Eze 30:15  Varsa-voi urgia Mea peste Sin, cetatea Egiptului ?i mul?imea oamenilor din No o voi pierde.
Eze 30:16  Voi trimite foc asupra Egiptului; cutremura-se-va Sin ?i No se va prabu?i, iar asupra Nofului vor navali vrajma?ii în ziua cea mare.
Eze 30:17  Tinerii din On ?i din Bubastis vor cadea de sabie, iar ceilal?i se vor duce în robie.
Eze 30:18  ?i în Tahpanhes se va întuneca ziua, când voi zdrobi acolo jugul Egiptului ?i se va curma puterea cea mândra a lui. Un nor îl va acoperi ?i fiicele lui vor fi duse în robie.
Eze 30:19  A?a voi face Eu judecata împotriva Egiptului ?i vor ?ti ca Eu sunt Domnul".
Eze 30:20  În anul al unsprezecelea, în luna întâi, în ziua a ?aptea a lunii, a fost cuvântul Domnului catre mine:
Eze 30:21  "Fiul omului, Eu am ?i zdrobit un bra? al lui Faraon, regele Egiptului, ?i iata, nimeni nu l-a legat ca sa se vindece ?i nu l-a înfa?urat cu legaturi ca sa capete putere pentru a mânui sabia".
Eze 30:22  De aceea, a?a zice Domnul Dumnezeu: Iata, Eu sunt împotriva lui Faraon, regele Egiptului, ?i voi zdrobi bra?ele lui, pe cel sanatos ?i pe cel zdrobit, încât sabia va cadea din mâinile lui.
Eze 30:23  Voi împra?tia pe Egipteni printre popoare ?i-i voi vântura prin ?ari.
Eze 30:24  Iar bra?ele regelui Babilonului le voi întari ?i-i voi da sabia Mea în mâna, iar bra?ele lui Faraon le voi zdrobi ?i el, ranit cumplit, va geme înaintea lui.
Eze 30:25  Întari-voi bra?ele regelui Babilonului, iar bra?ele lui Faraon vor cadea fara putere; ?i vor ?ti ca Eu sunt Domnul când voi da sabia Mea în mâinile regelui Babilonului, ?i acesta o va întinde asupra Egiptului.
Eze 30:26  Voi împra?tia pe Egipteni printre neamuri; îi voi risipi prin ?ari ?i vor ?ti ca Eu sunt Domnul".
Eze 31:1  În anul al unsprezecelea, în ziua întâi a lunii a treia, a fost cuvântul Domnului catre mine:
Eze 31:2  "Fiul omului, spune lui Faraon, regele Egiptului, ?i poporului lui: Cu cine te asemeni tu în marirea ta?
Eze 31:3  Iata Asiria era un cedru în Liban, cu ramuri frumoase, cu frunzi? umbros ?i cu trunchi înalt; vârful lui se ridicase pâna la nori.
Eze 31:4  Apele îl facusera sa creasca, adâncul îl ridicase ?i râurile acestuia înconjurau locul unde fusese sadit ?i trimiteau apele lor la to?i arborii câmpului.
Eze 31:5  De aceea înal?imea lui întrecuse pe to?i arborii câmpului ?i avea pe dânsul mul?i lastari ?i ramurile lui se înmul?isera; lastarii lui se facusera înal?i, pentru ca avusesera apa multa la cre?terea lor.
Eze 31:6  în lastarii lui î?i împletisera cuiburi tot felul de pasari de-ale cerului; sub ramurile lui î?i scoteau puii tot felul de fiare de ale pamântului, iar la umbra lui traiau numeroase ?i felurite popoare.
Eze 31:7  El era frumos prin înal?imea trunchiului sau, ?i prin lungimea ramurilor sale, caci radacinile  sale se aflau lânga ni?te ape mari.
Eze 31:8  Cedrii din gradina lui Dumnezeu nu-l umbreau, chiparo?ii nu se puteau asemana cu crengile lui ?i castanii nu erau la marime ca ramurile lui; nici un copac din gradina lui Dumnezeu nu se asemana cu el în frumuse?e.
Eze 31:9  Eu îl împodobisem cu mul?imea ramurilor lui, încât to?i arborii Edenului din gradina lui Dumnezeu îl pizmuiau.
Eze 31:10  De aceea, a?a a zis Domnul Dumnezeu: Pentru ca s-a facut înalt la statura ?i cre?tetul lui ?i l-a înal?at pâna la nori ?i inima lui s-a îngâmfat cu înal?imea lui,
Eze 31:11  De aceea l-am dat în mâna regelui neamurilor, ?i acesta s-a purtat cu el dupa rautatea lui; pentru nelegiuirea lui l-am lepadat.
Eze 31:12  Taiatu-l-au strainii cei mai rai dintre popoare ?i l-au pravalit peste mun?i; ramurile lui au cazut prin toate vaile, iar lastarii lui s-au frânt prin toate vagaunile pamântului ?i de sub umbra lui au fugit toate popoarele pamântului ?i l-au parasit.
Eze 31:13  Pe darâmaturile lui s-au a?ezat toate pasarile cerului ?i în lastarii lui erau toate fiarele câmpului.
Eze 31:14  Aceasta s-a facut pentru ca nici unul dintre arborii de pe lânga ape sa nu se îngâmfe cu statura sa înalta ?i sa nu-?i înal?e vârful pâna la nori; ca to?i stejarii ce se adapa din ape sa nu mai vada înal?imea lor, caci to?i vor fi da?i mor?ii, în latura cea de dedesubt a pamântului, împreuna cu fiii oamenilor, care s-au coborât în mormânt.
Eze 31:15  A?a zice Domnul Dumnezeu: "În ziua aceea, când s-a coborât el în locuin?a mor?ilor, în semn de jale, am închis peste el adâncul; am oprit râurile lui ?i apele cele mari au secat; am întunecat Libanul pentru el ?i to?i arborii câmpului s-au uscat din cauza lui.
Eze 31:16  La vuietul caderii lui am facut sa se cutremure neamurile, când l-am prabu?it în locuin?a mor?ilor, la cei ce se coborâsera în mormânt, ?i s-au bucurat în latura cea de dedesubt to?i arborii Edenului, cei mai ale?i ?i mai buni ai Libanului, to?i cei adapa?i cu apa;
Eze 31:17  ?i ace?tia s-au coborât cu el în locuin?a mor?ilor la cei uci?i de sabie, care erau bra?ul lui ?i traiau în umbra lui, printre neamuri".
Eze 31:18  Deci, cu care din arborii Edenului te-ai asemanat tu în stralucire ?i mare?ie? Acum însa la rând cu arborii Edenului vei fi prabu?it în adânc, vei. zacea în mijlocul celor netaia?i împrejur, cu cei uci?i de sabie. Iata pe Faraon ?i toata mul?imea supu?ilor sai", zice Domnul.
Eze 32:1  În anul al doisprezecelea de la robirea lui Ioiachim, în luna a douasprezecea, în ziua întâi a acestei luni, a fost catre mine cuvântul Domnului ?i mi-a zis:
Eze 32:2  "Fiul omului, ridica plângere asupra lui Faraon, regele Egiptului, ?i spune-i: Leu al neamurilor, iata-te nimicit. Tu erai ca un crocodil în ape, tu suflai din narile tale, tu tulburai apele cu picioarele tale ?i întarâtai valurile lor.
Eze 32:3  A?a zice Domnul Dumnezeu: Cu putere voi arunca asupra ta mreaja Mea în adunarea a multor popoare ?i ele te vor târî afara cu mreaja Mea.
Eze 32:4  Te voi arunca pe uscat, în câmp deschis te voi arunca ?i se vor a?eza pe tine toate pasarile cerului ?i se vor satura cu tine toate fiarele pamântului.
Eze 32:5  Voi împar?i carnurile tale pe mun?i ?i vaile le voi umple cu stârvul tau.
Eze 32:6  ?ara în care îno?i tu o voi umple cu sângele tau pâna în mun?i ?i vagaunile lor vor fi umplute cu tine.
Eze 32:7  Când te vei stinge, voi acoperi cerurile ?i stelele lor le voi întuneca; soarele îl voi acoperi cu nor ?i luna nu va mai lumina cu lumina sa.
Eze 32:8  Toate stelele care lumineaza pe cer le voi întuneca deasupra ta ?i asupra ?arii tale voi aduce negura, zice Domnul Dumnezeu.
Eze 32:9  Voi umple de tulburare inima multor popoare, când voi vesti caderea ta la popoarele de prin ?arile pe care tu nu le-ai cunoscut.
Eze 32:10  Voi umple prin tine de groaza multe popoare ?i regii lor se vor cutremura de frica prin tine, când voi flutura sabia Mea înaintea lor, ?i în orice clipa va tremura fiecare pentru sufletul sau în ziua caderii tale.
Eze 32:11  Caci a?a zice Domnul Dumnezeu: Sabia regelui Babilonului va veni asupra ta;
Eze 32:12  Cu sabia razboinicilor voi face sa cada mul?imea supu?ilor tai; ace?tia sunt cei mai cruzi dintre popoare, vor zdrobi mândria Egiptului, ?i toata mul?imea poporului lui va pieri.
Eze 32:13  Voi nimici toate vie?uitoarele din apele tale cele mari ?i mai mult nu le va mai tulbura picior de om, nici copita de dobitoc nu le va tulbura.
Eze 32:14  Atunci voi potoli apele; voi face sa curga ca untdelemnul fluviile lui, zice Domnul Dumnezeu.
Eze 32:15  Când voi face din ?ara Egiptului un pustiu, când ?ara va fi jefuita de tot ce are ?i când voi lovi pe to?i cei ce traiesc în ea, atunci vor ?ti ca Eu sunt Domnul.
Eze 32:16  Iata cântarea de jale pe care o vor striga fiicele neamurilor. Ele o vor striga asupra Egiptului ?i asupra întregului sau popor. Ele vor striga aceasta cântare de jale", zice Domnul Dumnezeu.
Eze 32:17  În anul al doisprezecelea, în ziua a cincisprezecea a lunii întâi, a fost cuvântul Domnului catre mine:
Eze 32:18  "Fiul omului, plângi pentru mul?imea poporului Egiptului ?i fa-o sa coboare, ea ?i fiicele neamurilor stralucite, în adâncurile pamântului, cu cei ce s-au coborât în mormânt.
Eze 32:19  Pe cine întreci tu! Coboara-te ?i zaci cu cei netaia?i împrejur!
Eze 32:20  Aceia au cazut printre cei uci?i de sabie; ?i el este dat sabiei; târâ?i-l pe el ?i toate mul?imile lui!
Eze 32:21  În mijlocul locuin?ei mor?ilor se va vorbi de el ?i de cei ce-l sprijineau, cei mai vesti?i dintre eroi; aceia au cazut ?i zac printre cei netaia?i împrejur, rapu?i de sabie.
Eze 32:22  Acolo este Asiria ?i oamenii ei de razboi, împrejurul mormântului ei, to?i uci?i, cazu?i sub sabie.
Eze 32:23  Mormintele lor sunt a?ezate chiar în fundul adâncului ?i armata ei împrejurul mormântului ei; to?i uci?i, cazu?i sub sabie, ei care împra?tiau groaza în pamântul celor vii.
Eze 32:24  Acolo este Elam cu toata armata lui împrejurul mormântului sau; to?i uci?i, cazu?i sub sabie ?i netaia?i împrejur s-au coborât în adâncuri, ei care împra?tiasera groaza pe pamântul celor vii, iar acum î?i poarta ru?inea cu cei ce s-au coborât în mormânt.
Eze 32:25  În mijlocul uci?ilor l-am culcat împreuna cu toata mul?imea oamenilor lui ?i mormintele lor sunt împrejurul lui; to?i ace?ti netaia?i împrejur sunt uci?i de sabie; ?i dupa cum au împra?tiat groaza pe pamântul celor vii, a?a î?i poarta ru?inea lor alaturi de cei ce s-au coborât în mormânt ?i sunt a?eza?i printre cei uci?i.
Eze 32:26  Acolo sunt Me?ec ?i Tubal cu toata mul?imea lor de popor ?i mormintele lor sunt împrejurul lor; to?i ace?ti netaia?i împrejur au murit de sabie, pentru ca au împra?tiat groaza pe pamântul celor vii.
Eze 32:27  Nu trebuia oare sa zaca ?i ei printre vitejii cazu?i dintre cei netaia?i împrejur, care cu armele lor de razboi s-au coborât în locuin?a mor?ilor ?i sabiile lor ?i le-au pus sub cap ?i faradelegea lor a ramas pe oasele lor, pentru ca ei, ca ni?te puternici, au fost o spaima pe pamântul celor vii?
Eze 32:28  ?i tu, Faraoane, vei fi zdrobit în mijlocul celor netaia?i împrejur ?i vei zacea împreuna cu cei uci?i de sabie.
Eze 32:29  Acolo sunt Edom ?i regii lui ?i toate capeteniile lui, care cu toata vitejia lor au fost a?eza?i printre cei cazu?i de sabie ?i zac cu cei netaia?i împrejur, care s-au coborât în mormânt.
Eze 32:30  Acolo sunt stapânitorii de la miazanoapte ?i to?i Sidonienii, care s-au coborât acolo cu cei uci?i ?i, fiind ru?ina?i în puternicia lor cu care au împra?tiat groaza, zac cu cei netaia?i împrejur, care au fost uci?i cu sabia ?i î?i poarta ru?inea cu cei coborâ?i în mormânt.
Eze 32:31  Faraon îi va vedea ?i se va mângâia la vederea acestei întregi mul?imi, ucisa de sabie Faraon ?i toata armata lui, zice Domnul.
Eze 32:32  Caci voi împra?tia frica Mea peste pamântul celor vii ?i Faraon cu toata mul?imea lui va fi pus printre cei netaia?i împrejur cu cei cazu?i de sabie, zice Domnul Dumnezeu.
Eze 33:1  Fost-a cuvântul Domnului catre mine ?i mi-a zis:
Eze 33:2  "Fiul omului, roste?te cuvânt catre fiii poporului tau ?i le spune: De voi aduce sabie asupra unei ?ari ?i poporul ?arii aceleia va lua din mijlocul sau un om ?i îl va pune strajer,
Eze 33:3  ?i el, vazând sabia venind împotriva ?arii, va trâmbi?a din trâmbi?a ?i va vesti poporul;
Eze 33:4  De va auzi cineva sunetul trâmbi?ei, dar nu se va pazi, când va veni sabia ?i-l va prinde, sângele aceluia va fi asupra capului sau.
Eze 33:5  Pentru ca a auzit glasul trâmbi?ei ?i nu s-a pazit, sângele lui va fi asupra lui; iar cel ce se va pazi î?i va scapa via?a sa.
Eze 33:6  Daca însa strajerul a vazut sabia venind ?i nu a sunat din trâmbi?a ?i poporul n-a fost vestit ?i va veni sabia ?i va ridica via?a cuiva, acela a fost rapit pentru pacatele lui, dar sângele lui îl voi cere din mâna strajerului.
Eze 33:7  ?i pe tine, fiul omului, te-am pus Eu strajer casei lui Israel ?i tu vei auzi cuvânt din gura Mea ?i îl vei vesti din partea Mea.
Eze 33:8  Când Eu voi zice pacatosului: "Pacatosule, vei muri", ?i tu nu-i vei grai nimic, ca sa veste?ti pe pacatos sa se abata de la calea lui, atunci pacatosul acela va muri pentru pacatele sale, iar sângele lui îl voi cere din mâna ta.
Eze 33:9  Iar daca tu ai vestit pe pacatos sa se abata de la calea lui ?i sa se întoarca de la ea, ?i el nu s-a abatut de la calea lui, atunci el va muri pentru pacatele lui, iar tu ?i-ai scapat via?a.
Eze 33:10  ?i tu, fiul omului, spune casei lui Israel: Voi zice?i a?a: "Nelegiuirile noastre ?i pacatele noastre sunt asupra noastra ?i ne stingem în ele; cum vom putea dar sa traim?"
Eze 33:11  Spune-le: Precum este adevarat ca Eu sunt viu, tot a?a este de adevarat ca Eu nu voiesc moartea. pacatosului, ci ca pacatosul sa se întoarca de la calea sa ?i sa fie viu. Întoarce?i-va, întoarce?i-va de la caile voastre cele rele! Pentru ce sa muri?i voi, casa lui Israel?
Eze 33:12  ?i tu, fiul omului, spune fiilor poporului tau: Dreptatea dreptului nu-l va scapa în ziua pacatuirii lui ?i nelegiuitul nu va cadea pentru nelegiuirea sa în ziua întoarcerii sale de la nelegiuirea sa, precum nici dreptul în ziua pacatuirii sale nu va putea ramâne cu via?a pentru dreptatea sa.
Eze 33:13  Când voi zice dreptului ca va fi viu, iar el se va încrede în dreptatea sa ?i va face nedreptate, atunci nu se va mai pomeni toata dreptatea lui, ci el va muri pentru tot raul pe care l-a facut.
Eze 33:14  ?i când voi zice pacatosului: "Vei muri", dar el se va întoarce de la pacatele sale ?i va face judecata ?i dreptate,
Eze 33:15  Daca acest pacatos va înapoia zalogul, pentru cele rapite va despagubi, va umbla dupa legile vie?ii, nefacând nimic rau, atunci el va fi viu ?i nu va muri.
Eze 33:16  Nici unul din pacatele sale, pe care le-a facut, nu i se vor pomeni ?i, pentru ca a început a face dreptate ?i judecata, va fi viu.
Eze 33:17  Fiii poporului tau zic: "Calea Domnului nu este dreapta!" Dar nedreapta este calea lor.
Eze 33:18  Daca dreptul se va abate de la dreptatea sa ?i va începe sa faca nelegiuire, va muri pentru aceasta.
Eze 33:19  De asemenea, daca nelegiuitul s-a întors de la nelegiuirea sa ?i a început sa faca judecata ?i dreptate, pentru aceasta el va trai.
Eze 33:20  Voi însa zice?i: "Calea Domnului este nedreapta". Eu va voi judeca pe voi, casa lui Israel, ?i voi judeca pe fiecare dupa purtarile lui".
Eze 33:21  În anul al doisprezecelea dupa robirea noastra, în ziua a cincea a lunii a zecea, a venit la mine unul din cei scapa?i din Ierusalim ?i mi-a spus: "Cetatea este darâmata!"
Eze 33:22  Mâna Domnului a fost peste mine, seara, înca înainte de a veni acest fugar; iar diminea?a, când a venit acesta la mine, Domnul îmi deschisese gura ?i nu mai eram mut, ei mi se deschisese gura.
Eze 33:23  ?i a fost cuvântul Domnului catre mine ?i mi-a zis:
Eze 33:24  Fiul omului, cei ce traiesc în locurile pustiite din ?ara lui Israel zic: "Avraam a fost unul ?i a primit în stapânire ?ara aceasta, iar noi suntem mul?i; deci noua ne este data în stapânire ?ara aceasta.
Eze 33:25  De aceea spune-le: "A?a graie?te Domnul Dumnezeu: Voi mânca?i mâncare cu sânge; va ridica?i ochii spre idolii vo?tri ?i varsa?i sânge ?i apoi voi?i sa stapâni?i ?ara?
Eze 33:26  Voi va rezema?i pe sabia voastra, face?i ticalo?ii, va pângari?i femeile unii altora ?i apoi voi?i sa stapâni?i ?ara?
Eze 33:27  Iata ce sa le spui: A?a graie?te Domnul Dumnezeu: Precum este adevarat ca Eu sunt viu, tot a?a de adevarat este ca cei ce locuiesc în locurile pustiite vor cadea de sabie; iar cel ce se afla în câmp, pe acela îl voi da fiarelor spre mâncare, iar cei din ceta?i ?i din pe?teri vor muri de ciuma.
Eze 33:28  ?i voi face din ?ara un pustiu ?i o singuratate, trufia puterii ei va înceta, ?i mun?ii lui Israel se vor pustii, încât nici un trecator nu va mai trece prin ei.
Eze 33:29  ?i vor cunoa?te ca Eu sunt Domnul, când voi face ?ara pustietatea pustieta?ilor pentru toate ticalo?iile pe care le-au facut ei.
Eze 33:30  Iar despre tine, fiul omului, fiii poporului tau graiesc pe la ziduri ?i pe la u?ile caselor ?i zice unul catre altul ?i frate catre frate: "Merge?i de vede?i ce cuvânt a ie?it de la Domnul!"
Eze 33:31  ?i ei vin la tine, ca la o adunare de petrecere; poporul Meu se a?aza înaintea ta ?i asculta cuvintele tale. Dar nu le împline?te; caci ei cu gura lor fac din acestea o petrecere, iar inima lor e târâta dupa poftele lor.
Eze 33:32  Iata ca tu e?ti pentru ei un cântare? placut, cu glas frumos ?i care cânta bine din instrumentul sau; ei asculta cuvintele tale, dar nimeni nu le împline?te,
Eze 33:33  Iar când aceste lucruri vor veni, ?i iata ca ele vin, atunci vor ?ti ca în mijlocul lor era un prooroc".
Eze 34:1  Fost-a cuvântul Domnului catre mine ?i mi-a zis:
Eze 34:2  "Fiul omului, prooroce?te împotriva pastorilor lui Israel, prooroce?te ?i le spune: A?a graie?te Domnul Dumnezeu: Vai de pastorii lui Israel, care s-au pastorit pe ei în?i?i! Pastorii nu trebuia ei oare sa pastoreasca turma?
Eze 34:3  Dar voi a?i mâncat grasimea ?i cu lâna v-a?i îmbracat; oile cele grase le-a?i junghiat, iar turma n-a?i pascut-o.
Eze 34:4  Pe cele slabe nu le-a?i întarit; oaia bolnava n-a?i lecuit-o ?i pe cea ranita n-a?i legat-o; pe cea ratacita n-a?i întors-o ?i pe cea pierduta n-a?i cautat-o, ci le-a?i stapânit cu asprime ?i cruzime.
Eze 34:5  ?i ele, neavând pastor, s-au risipit ?i, risipindu-se, au ajuns mâncarea tuturor fiarelor câmpului.
Eze 34:6  De aceea ratacesc oile Mele prin to?i mun?ii ?i pe tot dealul înalt; împra?tiatu-s-au oile Mele peste toata fa?a pamântului ?i nimeni nu îngrije?te de ele ?i nimeni nu le cauta.
Eze 34:7  De aceea, asculta?i pastori, cuvântul Domnului:
Eze 34:8  Precum este adevarat ca Eu sunt viu, zice Domnul Dumnezeu, tot a?a este de adevarat ca voi face dreptate; pentru ca oile Mele au fost lasate prada ?i fara pastor, oile Mele au ajuns mâncarea tuturor fiarelor câmpului, iar pastorii Mei n-au purtat grija de oile Mele, ci pastorii s-au pascut pe ei în?i?i ?i oile Mele nu le-au pascut.
Eze 34:9  De aceea asculta?i, pastori, cuvântul Domnului.
Eze 34:10  Asa zice Domnul Dumnezeu: Iata Eu vin la pastori; le voi cere înapoi oile Mele din mâna lor ?i îi voi împiedica sa nu mai pasca oile Mele ?i nu se vor mai pa?te pastorii pe ei în?i?i ?i voi smulge oile Mele din gura lor ?i ele nu vor mai fi pentru ei o prada de sfâ?iat.
Eze 34:11  Caci a?a zice Domnul Dumnezeu: Iata Eu Însumi voi purta grija de oile Mele ?i le voi cerceta.
Eze 34:12  Cum cerceteaza pastorul turma sa în ziua când se afla în mijlocul turmei sale risipite, a?a voi cerceta ?i Eu oile Mele ?i le voi aduna din toate locurile, unde au fost ele risipite în ziua cea ce?oasa ?i întunecata.
Eze 34:13  Le voi face sa iasa din mijlocul popoarelor, le voi aduna din diferite ?ari ?i le voi aduce în ?ara lor ?i le voi pa?te prin mun?ii lui Israel, pe lânga cursurile de apa ?i prin toate locurile de locuit ale ?arii acesteia.
Eze 34:14  Le voi pa?te în pa?une buna ?i staulul va fi pe mun?ii cei înal?i ai lui Israel; acolo se vor odihni ele, în staul bun ?i vor pa?te în pa?une grasa în mun?ii lui Israel.
Eze 34:15  Eu voi pa?te oile Mele ?i Eu le voi odihni, zice Domnul Dumnezeu.
Eze 34:16  Oaia pierduta ?i ratacita o voi întoarce la staul, pe cea ranita o voi lega ?i pe cea bolnava o voi întari, iar pe cea grasa ?i tare o voi pazi ?i voi pastori cu dreptate.
Eze 34:17  Iar despre voi, oile Mele, a?a zice Domnul Dumnezeu: Iata voi face judecata între oaie ?i oaie, între berbec ?i ?ap.
Eze 34:18  Oare nu va ajunge ca pa?te?i în pa?une buna, iar ce ramâne calca?i cu picioarele voastre ?i ca be?i apa curata, iar pe cea care ramâne o tulbura?i cu picioarele voastre,
Eze 34:19  Asa ca oile Mele sunt nevoite sa se hraneasca cu ceea ce este calcat de picioarele voastre ?i sa bea ceea ce este tulburat de picioarele voastre?"
Eze 34:20  De aceea a?a le zice Domnul Dumnezeu: "Iata Eu Însumi voi face judecata între oaia grasa ?i oaia slaba.
Eze 34:21  Deoarece voi izbi?i cu umarul, cu ?oldul ?i cu coarnele voastre, împunge?i pe toate oile bolnavicioase, pâna când le scoate?i afara,
Eze 34:22  Eu voi veni sa scap oile Mele, ca sa nu mai fie prada ?i voi judeca între oaie ?i oaie.
Eze 34:23  Voi  pune peste ele un singur pastor, care le va pa?te; voi pune pe robul Meu David; el le va pa?te ?i el va fi pastorul lor.
Eze 34:24  Iar Eu, Domnul, le voi fi Dumnezeu, iar robul Meu David va fi prin? între ei. Eu, Domnul, am grait acestea.
Eze 34:25  Voi încheia cu acela legamântul pacii ?i voi departa din ?ara fiarele salbatice, încât oile Mele sa traiasca în siguran?a în pustiu ?i sa doarma în padure.
Eze 34:26  Voi darui lor ?i împrejurimilor muntelui Meu binecuvântare ?i ploaie le voi trimite la vreme; ploi de binecuvântare vor fi acestea.
Eze 34:27  Pomul din câmp î?i va da rodul sau ?i pamântul î?i va da roadele sale ?i oile Mele vor fi în siguran?a pe pamântul lor ?i vor ?ti ca Eu sunt Domnul, când voi sfarâma catu?ele jugului lor ?i le voi scapa din mâinile celor ce le-au robit.
Eze 34:28  Nu vor mai fi ele prada popoarelor ?i fiarele câmpului nu le vor mai sfâ?ia; ele vor trai în siguran?a ?i nimeni nu le va mai tulbura.
Eze 34:29  Voi face acolo sadire vestita ?i nu vor mai pieri de foame pe pamânt, nici nu vor mai suferi ocara de la popoare.
Eze 34:30  ?i vor ?ti ca Eu, Domnul Dumnezeul lor, sunt cu ele, iar ele, casa lui Israel, sunt poporul Meu, zice Domnul Dumnezeu.
Eze 34:31  ?i voi, oile Mele, sunte?i turma pe care o pasc, iar Eu sunt Dumnezeul vostru, zice Domnul Dumnezeu.
Eze 35:1  Fost-a cuvântul Domnului catre mine ?i mi-a zis:
Eze 35:2  "Fiul omului, întoarce-?i fa?a spre muntele Seir ?i prooroce?te împotriva lui,
Eze 35:3  ?i spune-i: A?a graie?te Domnul Dumnezeu: Iata Eu sunt împotriva ta, munte Seir, ?i-Mi voi întinde mâna împotriva ta ?i te voi face pustiu ?i nelocuit.
Eze 35:4  Ceta?ile tale le voi preface în ruine ?i tu însu?i vei fi pustiit ?i vei ?ti ca Eu sunt Domnul;
Eze 35:5  Fiindca ai du?manie ve?nica ?i ai dat pe fiii lui Israel în mâna sabiei în timpul necredin?ei lor, în vremea pieirii desavâr?ite,
Eze 35:6  De aceea, precum este adevarat ca Eu sunt viu, zice Domnul Dumnezeu, tot a?a este de adevarat ca te voi umple de sânge ?i sângele te va urmari; ?i pentru ca tu n-ai urât varsarea de sânge, de aceea sângele te va ?i urmari.
Eze 35:7  Voi face muntele Seir o singuratate ?i un pustiu; voi nimici pe oricine strabate ?ara.
Eze 35:8  ?i voi umple înal?imile lui de uci?ii lui. Pe dealurile tale, în vaile tale ?i în toate vagaunile tale vor cadea rapu?i de sabie.
Eze 35:9  Te voi face pustiu ve?nic ?i în ceta?ile tale nu vor mai trai oameni, ?i ve?i ?ti ca Eu sunt Domnul.
Eze 35:10  De vreme ce tu ai zis: "Aceste doua popoare ?i aceste doua ?ari vor fi ale mele ?i le voi stapâni", cu toate ca Domnul era acolo,
Eze 35:11  De aceea precum este adevarat ca Eu sunt viu, zice Domnul Dumnezeu, tot a?a este de adevarat ca Ma voi purta cu tine dupa masura urii tale ?i a pizmei tale, pe care le-ai aratat catre ele, ?i Ma voi face cunoscut lor, când te voi judeca.
Eze 35:12  Atunci vei ?ti ca Eu, Domnul, am auzit toate hulele tale pe care le-ai rostit împotriva mun?ilor lui Israel, zicând: "S-au pustiit ?i ne sunt da?i noua spre mâncare!"
Eze 35:13  Auzit-am ca v-a?i laudat înaintea Mea cu limba voastra ?i a?i înmul?it vorbele voastre împotriva Mea.
Eze 35:14  A?a zice Domnul Dumnezeu: Când tot pamântul se va bucura, pe tine te voi face pustiu.
Eze 35:15  Cum te-ai bucurat tu, ca partea casei lui Israel s-a pustiit, a?a voi face ?i cu tine: pustiit vei fi, munte Seir, ?i împreuna cu tine ?i tot Edomul ?i vor ?ti ca Eu sunt Domnul".
Eze 36:1  "?i tu, fiul omului, prooroce?te asupra mun?ilor lui Israel ?i spune: "Mun?ii lui Israel, asculta?i cuvântul Domnului.
Eze 36:2  A?a zice Domnul Dumnezeu: Deoarece vrajma?ul graie?te despre voi ?i zice: "Aha, ?i înal?imile cele ve?nice au ajuns mo?tenirea noastra",
Eze 36:3  De aceea prooroce?te ?i spune: A?a graie?te Domnul Dumnezeu: Pentru ca va pustiesc, ?i anume pentru ca va pustiesc ?i va înghit din toate par?ile, ca sa ajunge?i mo?tenirea celorlalte popoare ?i a?i ajuns clevetirea ?i ocara oamenilor,
Eze 36:4  De aceea, mun?i ai lui Israel, asculta?i cuvântul Domnului Dumnezeu: A?a graie?te Domnul Dumnezeu catre mun?i ?i dealuri., catre vai ?i vâlcele, catre ruinele pustii ?i catre ceta?ile parasite, care au ajuns prada ?i ocara celorlalte popoare de primprejur;
Eze 36:5  De aceea, a?a zice Domnul Dumnezeu: În focul zelului Meu am rostit cuvânt împotriva celorlalte popoare ?i împotriva întregului Edom, care au socotit ?ara Mea ca mo?tenire a lor, ?i s-au bucurat din toata inima lor ?i cu tot dispre?ul sufletului lor, ca sa-i jefuiasca roadele.
Eze 36:6  De aceea roste?te proorocie asupra ?arii lui Israel ?i spune mun?ilor ?i dealurilor, vailor ?i vâlcelelor: A?a zice Domnul Dumnezeu: Iata Eu am rostit aceasta în zelul urgiei Mele, pentru ca voi purta?i asupra voastra ocara neamurilor.
Eze 36:7  De aceea, a?a zice Domnul Dumnezeu: Ridicatu-Mi-am mâna cu juramânt, ca popoarele care sunt împrejurul vostru vor purta ele singure ru?inea lor.
Eze 36:8  Iar voi, mun?ii lui Israel, ve?i întinde ramurile voastre ?i ve?i aduce roadele voastre poporului Meu Israel; ca se apropie venirea lui.
Eze 36:9  Caci iata Eu Ma întorc spre voi ?i ve?i fi lucra?i ?i semana?i.
Eze 36:10  ?i voi a?eza pe voi mul?ime de oameni, toata casa lui Israel. Ceta?ile vor fi locuite ?i ruinele zidite din nou.
Eze 36:11  Voi înmul?i la voi oamenii ?i dobitoacele; se vor prasi acestea ?i se vor înmul?i ?i va voi face sa fi?i locui?i, ca ?i mai înainte ?i va voi face bine mai mult decât alta data ?i ve?i ?ti ca Eu sunt Domnul.
Eze 36:12  Voi aduce pe voi oameni, pe poporul Meu Israel ?i ei te vor stapâni pe tine, ?ara, ?i tu vei fi mo?tenirea lor ?i nu-i vei mai lipsi de copiii lor".
Eze 36:13  A?a zice Domnul Dumnezeu: "Pentru ca se zice despre voi: "Tu e?ti o rara care manânci oameni ?i lipse?ti neamul tau de copiii sai",
Eze 36:14  De aceea nu vei mai mânca pe oameni ?i pe poporul tau nu-l vei mai lipsi de copiii sai, zice Domnul Dumnezeu.
Eze 36:15  ?i nu vei mai auzi batjocuri de la popoare ?i hula de la neamuri nu vei mai purta pe obrazul tau; pe poporul tau de acum înainte nu-l vei mai lipsi de capii", zice Domnul Dumnezeu.
Eze 36:16  Fost-a cuvântul Domnului catre mine ?i mi-a zis:
Eze 36:17  "Fiul omului, când casa lui Israel traia în ?ara sa, au pângarit-o cu purtarea lor ?i cu faptele lor; caile lor erau înaintea fe?ei Mele ca necura?enia femeii în timpul regulei ei;
Eze 36:18  Eu am varsat asupra lor mânia Mea pentru sângele pe care l-au varsat în ?ara ?i pentru ca au întinat-o cu idolii lor.
Eze 36:19  I-am risipit printre neamuri ?i au fost împra?tia?i prin ?arile straine; dupa purtarile lor ?i dupa faptele lor i-am judecat.
Eze 36:20  ?i au mers la neamurile la care s-au dus ?i au necinstit numele Meu cel sfânt, încât se zicea despre ei: "Acesta este poporul Domnului, care a ie?it din ?ara sa".
Eze 36:21  Am luat aminte la numele Meu cel sfânt, pe care l-a necinstit casa lui Israel printre popoarele la care s-a dus.
Eze 36:22  ?i de aceea spune casei lui Israel: A?a graie?te Domnul Dumnezeu: Aceasta o fac nu pentru voi, casa lui Israel, ci pentru numele Meu cel sfânt pe care l-a?i necinstit voi printre neamurile la care a?i mers.
Eze 36:23  Voi sfin?i numele Meu cel mare care a fost necinstit la neamurile printre care l-a?i necinstit voi, ?i vor ?ti neamurile ca Eu sunt Domnul, zice Domnul Dumnezeu, când Ma voi sfin?i în voi, înaintea ochilor lor.
Eze 36:24  De aceea va voi scoate dintre neamuri ?i din toate ?arile va voi aduna ?i va voi aduce în pamântul vostru.
Eze 36:25  ?i va voi stropi cu apa curata ?i va ve?i cura?i de toate întinaciunile voastre ?i de to?i idolii vo?tri va voi cura?i.
Eze 36:26  Va voi da inima noua ?i duh nou va voi da; voi lua din trupul vostru inima cea de piatra ?i va voi da inima de carne.
Eze 36:27  Pune-voi înauntrul vostru Duhul Meu ?i voi face ca sa umbla?i dupa legile Mele ?i sa pazi?i ?i sa urma?i rânduielile Mele.
Eze 36:28  Ve?i locui în ?ara pe care am dat-o parin?ilor vo?tri ?i ve?i fi poporul Meu ?i Eu voi fi Dumnezeul vostru.
Eze 36:29  Va voi scapa de toate necura?iile voastre ?i voi chema pâinea ?i o voi înmul?i ?i nu va voi lasa sa suferi?i de foame.
Eze 36:30  Voi înmul?i fructele în pom ?i roadele în câmp, ca sa nu mai suferi?i de acum înainte ocara neamurilor din pricina foametei.
Eze 36:31  Atunci va ve?i aduce aminte de purtarile voastre cele rele ?i de faptele voastre care n-au fost bune ?i va ve?i scârbi de voi în?iva pentru nelegiuirile voastre ?i pentru ticalo?iile voastre.
Eze 36:32  ?tiut sa va fie, ca nu pentru voi, zice Domnul Dumnezeu, voi face aceasta. Ro?i?i ?i va ru?ina?i de caile voastre, casa lui Israel!"
Eze 36:33  A?a zice Domnul Dumnezeu: "În ziua aceea, când va voi cura?i de toate faradelegile voastre ?i voi face sa fie ceta?ile locuite, când a?ezarile darâmate vor fi iara?i zidite,
Eze 36:34  ?i pamântul cel pustiit, care în ochii oricarui trecator era o pustietate, va fi lucrat,
Eze 36:35  Atunci se va zice: "Acest pamânt, alta data pustiit, s-a facut ca gradina Edenului ?i ceta?ile acestea pustiite ?i darâmate sunt iara?i întarituri locuite".
Eze 36:36  ?i neamurile care vor ramâne împrejurul vostru, vor ?ti ca Eu, Domnul, zidesc din nou cele ruinate ?i sadesc cele pustiite. Eu, Domnul, am zis ?i fac".
Eze 36:37  A?a graie?te Domnul Dumnezeu: "Iata înca ?i pentru aceasta voi lasa casa lui Israel sa Ma caute; îi voi înmul?i pe oamenii sai ca pe o turma.
Eze 36:38  Cum sunt de multe oile de jertfa în Ierusalim, în timpul sarbatorilor, a?a vor fi pline de oameni ceta?ile pustiite, ?i vor ?ti ca Eu sunt Domnul".
Eze 37:1  Fost-a mâna Domnului peste mine ?i m-a dus Domnul cu Duhul ?i m-a a?ezat în mijlocul unui câmp plin de oase omene?ti,
Eze 37:2  ?i m-a purtat împrejurul lor; dar iata oasele acestea erau foarte multe pe fa?a pamântului ?i uscate de tot.
Eze 37:3  ?i mi-a zis Domnul: "Fiul omului, vor învia, oasele acestea?" Iar eu am zis: "Dumnezeule, numai Tu ?tii aceasta".
Eze 37:4  Domnul însa mi-a zis: "Prooroce?te asupra oaselor acestora ?i le spune: Oase uscate, asculta?i cuvântul Domnului!
Eze 37:5  A?a graie?te Domnul Dumnezeu oaselor acestora: Iata Eu voi face sa intre în voi duh ?i ve?i învia.
Eze 37:6  Voi pune pe voi vine ?i carne va cre?te pe voi; va voi acoperi cu piele, voi face sa intre în voi duh ?i ve?i învia ?i ve?i ?ti ca Eu sunt Domnul".
Eze 37:7  Proorocit-am deci cum mi se poruncise. ?i când am proorocit, iata s-a facut un vuiet ?i o mi?care ?i oasele au început sa se apropie, fiecare os la încheietura sa.
Eze 37:8  ?i am privit ?i eu ?i iata erau pe ele vine ?i crescuse carne ?i pielea le acoperea pe deasupra, iar duh nu era în ele.
Eze 37:9  Atunci mi-a zis Domnul: "Fiul omului, prooroce?te duhului, prooroce?te ?i spune duhului: A?a graie?te Domnul Dumnezeu: Duhule, vino din cele patru vânturi ?i sufla peste mor?ii ace?tia ?i vor învia!"
Eze 37:10  Deci am proorocit eu, cum mi se poruncise, ?i a intrat în ei duhul ?i au, înviat ?i mul?ime multa foarte de oameni s-au ridicat pe picioarele lor.
Eze 37:11  ?i mi-a zis iara?i Domnul: "Fiul omului, oasele acestea sunt toata casa lui Israel. Iata ei zic: "S-au uscat oasele noastre ?i nadejdea noastra a pierit; suntem smul?i din radacina".
Eze 37:12  De aceea prooroce?te ?i le spune: A?a graie?te Domnul Dumnezeu: Iata, Eu voi deschide mormintele voastre ?i va voi scoate pe voi, poporul Meu, din mormintele voastre ?i va voi duce în ?ara lui Israel.
Eze 37:13  Astfel ve?i ?ti ca Eu sunt Domnul, când voi deschide mormintele voastre ?i va voi scoate pe voi, poporul Meu, din mormintele voastre.
Eze 37:14  ?i voi pune în voi Duhul Meu ?i ve?i învia ?i va voi a?eza în ?ara voastra ?i ve?i ?ti ca Eu, Domnul, am zis aceasta ?i am facut", zice Domnul.
Eze 37:15  Fost-a iara?i catre mine cuvântul Domnului ?i mi-a zis:
Eze 37:16  "Iar tu, fiul omului, ia-?i un toiag ?i scrie pe el: "Lui Iuda ?i fiilor lui Israel, care sunt uni?i cu el". ?i sa mai iei un toiag ?i sa scrii pe el: "Lui Iosif". Acesta este toiagul lui Efraim ?i a toata casa lui Israel, care este unita cu el.
Eze 37:17  Apoi sa le apropii unul de altul încât ele sa fie în mâna ta ca un singur toiag.
Eze 37:18  Iar când te vor întreba fiii poporului tau: "Nu ne vei talmaci oare ?i noua ce înseamna ceea ce ai în mâna?
Eze 37:19  Tu sa le spui: A?a graie?te Domnul Dumnezeu: Iata Eu voi lua toiagul lui Iosif, care este în mâna lui Efraim ?i a semin?iilor lui Israel unite cu el ?i le voi împreuna cu toiagul lui Iuda ?i voi face din ele un singur toiag ?i vor fi în mâna Mea una.
Eze 37:20  Când însa amândoua toiegele pe care vei scrie vor fi în mâna ta înaintea ochilor lor,
Eze 37:21  Atunci sa le spui: A?a graie?te Domnul Dumnezeu: Iata, Eu voi lua pe fiii lui Israel din mijlocul neamurilor, printre care se afla, îi voi aduna din toate par?ile ?i-i voi aduce în ?ara lor;
Eze 37:22  Iar în ?ara aceasta, pe mun?ii lui Israel, îi voi face un singur neam ?i un singur rege va fi peste to?i; nu vor mai fi doua neamuri ?i în viitor nu se vor mai împar?i în doua regate;
Eze 37:23  Nu se vor mai pângari cu idolii lor, cu urâciunile lor ?i cu toate pacatele lor. ?i voi izbavi de toate faradelegile pe care le-au savâr?it, îi voi cura?i ?i vor fi poporul Meu, iar Eu voi fi Dumnezeul lor.
Eze 37:24  Iar robul Meu David va fi rege peste ei ?i pastorul lor al tuturor, ?i ei se vor purta dupa cum cer poruncile Mele ?i legile Mele le vor pazi ?i le vor împlini.
Eze 37:25  Vor locui ?ara pe care am dat-o Eu robului Meu Iacov, unde au trait parin?ii lor; acolo vor locui ei ?i copiii lor în veci; iar robul Meu David va fi peste ei rege în veac.
Eze 37:26  Voi încheia cu ei un legamânt al pacii, legamânt ve?nic voi avea cu ei. Voi pune rânduiala la ei, îi voi înmul?i ?i voi a?eza în mijlocul lor loca?ul Meu pe veci.
Eze 37:27  Fi-va loca?ul Meu la ei ?i voi fi Dumnezeul lor, iar ei vor fi poporul Meu.
Eze 37:28  Atunci vor ?ti popoarele ca Eu sunt Domnul Care sfin?e?te pe Israel, când loca?ul Meu va fi ve?nic în mijlocul lor".
Eze 38:1  Fost-a cuvântul Domnului catre mine ?i mi-a zis:
Eze 38:2  "Fiul omului, întoarce-îi fa?a spre Gog din ?ara lui Magog, regele lui Ro?, al lui Me?ec ?i al lui Tubal; prooroce?te împotriva lor,
Eze 38:3  ?i spune: A?a graie?te Domnul Dumnezeu: Iata, Eu sunt împotriva ta, Gog, rege al lui Ro? ?i al lui Me?ec ?i al lui Tubal!
Eze 38:4  Te voi prinde, voi pune zabale în falcile tale ?i te voi scoate pe tine ?i toata o?tirea ta, caii ?i to?i calare?ii stralucit îmbraca?i, ceata mare cu plato?e ?i cu scuturi, to?i înarma?i cu sabii;
Eze 38:5  ?i cu ei voi scoate pe Per?i, pe Etiopieni ?i pe Libieni, to?i cu scuturi ?i coifuri;
Eze 38:6  ?i pe Gomer cu toate o?tirile lui; casa lui Togarma din hotarele de la miazanoapte, cu toate o?tirile lui ?i voi mai scoate ?i alte multe popoare cu tine.
Eze 38:7  Gate?te-te ?i fii gata, tu ai toata mul?imea ta strânsa împrejurul tau, ?i fii capetenia lor.
Eze 38:8  Dupa zile multe tu vei primi porunci. În anii de pe urma vei veni în ?ara izbavita de sabie, ai carei locuitori au fost aduna?i dintr-o mul?ime de popoare, în mun?ii lui Israel, care au fost mult timp pustii?i. De când au fost despar?i?i de celelalte popoare, ei locuiesc to?i în siguran?a.
Eze 38:9  ?i tu te vei ridica, cum se ridica furtuna ?i te vei duce ca norul, ca sa acoperi ?ara, tu ?i toata oastea ta ?i multe popoare împreuna cu tine.
Eze 38:10  A?a zice Domnul Dumnezeu: n ziua aceea î?i vor veni gânduri în mintea ta ?i vei face planuri rele,
Eze 38:11  ?i vei zice: "Ma voi ridica împotriva unei ?ari fara aparare, voi merge împotriva oamenilor pa?nici care traiesc în siguran?a, caci aceia to?i traiesc în ceta?i fara ziduri ?i n-au nici por?i, nici zavoare,
Eze 38:12  Ca sa fac jaf ?i sa iau prada, punând mâna pe ruinele locuite din nou ?i pe poporul cel adunat din mijlocul neamurilor, care cre?te turme ?i strânge averi ?i care locuie?te în mijlocul pamântului".
Eze 38:13  ?eba, Dedan ?i negustorii Tarsisului cu to?i puii de lei ai lor vor zice: "Ai venit tu oare ca sa faci jaf, ai adunat taberile tale, ca sa faci prada, sa iei argint ?i aur, sa ridici dobitoace ?i avere ?i sa apuci prada mare?"
Eze 38:14  De aceea, roste?te proorocie, fiul omului, ?i spune lui Gog: A?a graie?te Domnul Dumnezeu: Nu este a?a oare ca în ziua când poporul Meu Israel va trai în siguran?a, tu vei porni la drum?
Eze 38:15  ?i vei pleca de la locul tau, din hotarele de la miazanoapte, tu ?i multe popoare împreuna cu tine, to?i calari pe cai, tabara mare ?i o?tire nenumarata?
Eze 38:16  ?i te vei ridica împotriva poporului Meu, împotriva lui Israel, ca un nor care acopera pamântul; aceasta va fi în zilele cele de pe urma când te voi aduce împotriva ?arii Mele, ca popoarele sa Ma cunoasca pe Mine, când Eu voi fi aratat sfin?enia Mea înaintea ochilor lor, asupra ta, o, Gog!
Eze 38:17  A?a zice Domnul Dumnezeu: "Nu e?ti tu, oare, acela?i despre care am grait Eu în zilele cele de demult prin robii Mei, proorocii lui Israel, care au proorocit în vremurile acelea ca te voi aduce împotriva lor?
Eze 38:18  ?i în ziua aceea, când Gog va veni împotriva ?arii lui Israel, zice Domnul Dumnezeu, mânia Mea se va aprinde pe fa?a Mea.
Eze 38:19  ?i în zelul Meu, în vapaia urgiei Mele am zis: Cu adevarat în ziua aceea va fi un mare cutremur în ?ara lui Israel.
Eze 38:20  Atunci vor tremura înaintea Mea pe?tii marii ?i pasarile cerului, fiarele câmpului ?i toate târâtoarele care se târasc pe pamânt ?i to?i oamenii care sunt pe fa?a pamântului; ?i se vor prabu?i mun?ii, stâncile se vor darâma ?i toate zidurile vor cadea la pamânt.
Eze 38:21  Prin to?i mun?ii Mei voi chema sabia împotriva lui, zice Domnul Dumnezeu; sabia fiecarui om va fi împotriva fratelui sau.
Eze 38:22  ?i îl voi pedepsi cu ciuma ?i varsare de sânge; voi varsa asupra lui ?i a taberilor lui ?i asupra multor popoare care sunt cu el, ploaie potopitoare ?i grindina de piatra, foc ?i pucioasa;
Eze 38:23  Voi arata slava Mea ?i sfin?enia Mea ?i Ma voi arata înaintea ochilor multor popoare ?i vor ?ti ca Eu sunt Domnul".
Eze 39:1  "Iar tu, fiul omului, roste?te proorocie împotriva lui Gog ?i spune: A?a graie?te Domnul Dumnezeu: Iata, Eu sunt împotriva ta, Gog, prin?ul lui Ro?, al lui Me?ec ?i al lui `Tubal!
Eze 39:2  Te voi ademeni ?i te voi trage, te voi scoate din hotarele de la miazanoapte ?i te voi aduce în mun?ii lui Israel.
Eze 39:3  Acolo voi scoate aurul tau din mâna stânga a ta ?i voi arunca sage?ile tale din mâna dreapta a ta.
Eze 39:4  Cadea-vei în mun?ii lui Israel, tu ?i toate o?tile tale ?i popoarele cele ce sunt cu tine; ?i te voi da spre mâncare la tot felul de pasari de prada ?i fiarelor câmpului.
Eze 39:5  Cadea-vei în câmp deschis, caci Eu am spus acestea", zice Domnul Dumnezeu.
Eze 39:6  "?i voi trimite foc în pamântul lui Magog ?i asupra locuitorilor insulelor, care traiesc fara grija ?i vor ?ti ca Eu sunt Domnul.
Eze 39:7  Voi arata numele Meu cel sfânt poporului Meu Israel ?i nu voi mai lasa de acum sa se necinsteasca sfânt numele Meu ?i vor ?ti neamurile ca Eu sunt Domnul cel sfânt în Israel.
Eze 39:8  Iata, aceasta va veni ?i se va împlini, zice Domnul Dumnezeu; aceasta este ziua aceea de care am grait Eu.
Eze 39:9  Atunci locuitorii ceta?ilor lui Israel vor ie?i ?i vor aprinde foc, vor arde armele, scuturile, pavezele, arcurile, sage?ile; lancile ?i suli?ele; ?apte ani le vor arde.
Eze 39:10  Nu vor aduce lemne din câmp, nici nu var taia din padure, ci vor arde numai arme; vor jefui pe jefuitorii lor ?i vor prada pe pradatorii for, zice Domnul Dumnezeu.
Eze 39:11  În ziua aceea voi da lui Gog loc de mormânt, în Israel, valea trecatorilor, la rasarit de Marea Moarta ?i mormântul acela va împiedica pe trecatori; acolo vor îngropa pe Gog ?i toata tabara lui ?i vor numi valea aceea Valea taberei lui Gog.
Eze 39:12  ?apte luni îi va îngropa casa lui Israel, ca sa cure?e ?ara.
Eze 39:13  Tot poporul ?arii îi va îngropa ?i va fi vestita la ei ziua în care Ma voi preaslavi, zice Domnul Dumnezeu;
Eze 39:14  Apoi se vor ridica oameni care sa cutreiere necontenit ?ara ?i cu ajutorul trecatorilor sa îngroape pe cei ce au ramas pe fa?a pamântului, ca sa cure?e ?ara; iar dupa trecerea a ?apte luni, vor începe sa faca cercetari.
Eze 39:15  ?i când cineva din cei ce cutreiera ?ara va vedea os de om, va pune semn lânga el pâna ce groparii îl vor îngropa în Valea taberei lui Gog.
Eze 39:16  Numele ceta?ii va fi Hamona (cimitir). ?i a?a vor cura?i ei ?ara".
Eze 39:17  "A?a zice Domnul Dumnezeu: Iar tu, fiul omului, spune la tot felul de pasari ?i tuturor fiarelor câmpului: "Aduna?i-va ?i merge?i din toate par?ile, aduna?i-va la jertfa Mea, pe care o voi junghia Eu pentru voi, la jertfa cea mare din mun?ii lui Israel ?i ve?i mânca acolo carne ?i ve?i bea sânge.
Eze 39:18  Carnea razboinicilor o ve?i mânca ?i ve?i bea sângele capeteniilor pamântului, al berbecilor, al mieilor, al ?apilor, al vi?eilor ?i al tuturor celor îngra?a?i din Vasan;
Eze 39:19  Ve?i mânca grasime pâna va ve?i satura ?i ve?i bea sânge pâna va ve?i îmbata din jertfa Mea, pe care o voi junghia pentru voi.
Eze 39:20  ?i va ve?i satura la masa Mea de cai ?i de calare?i, de razboinici ?i de tot felul de osta?i, zice Domnul Dumnezeu.
Eze 39:21  Voi arata slava Mea între neamuri, ?i toate neamurile vor vedea judecata Mea, pe care o voi savâr?i Eu, ?i mâna Mea, pe care o voi pune asupra lor.
Eze 39:22  Atunci va ?ti casa lui Israel ca Eu sunt Domnul Dumnezeul lor, din ziua de astazi înainte.
Eze 39:23  Popoarele de asemenea vor cunoa?te ca neamul lui Israel a fost dus în robie pentru nedreptatea lui; pentru ca ei s-au purtat cu necredincio?ie înaintea Mea, am ascuns Eu fa?a ?i i-am dat pe mâna vrajma?ilor lor ?i au cazut ei cu to?ii de sabie;
Eze 39:24  Pentru necura?iile lor ?i pentru faradelegile lor am facut Eu aceasta cu ei ?i Mi-am ascuns Eu fa?a de la dân?ii.
Eze 39:25  De aceea a?a zice Domnul Dumnezeu: Acum voi întoarce prizonierii lui Iacov, Ma voi îndura de toata casa lui Israel ?i voi fi zelos pentru numele Meu cel sfânt.
Eze 39:26  Ei vor uita ocara lor ?i toate nelegiuirile lor pe care le-au facut înaintea Mea, când vor trai în ?ara lor în siguran?a ?i nimeni nu-i va tulbura.
Eze 39:27  Când îi voi întoarce dintre popoare ?i-i voi aduna dintre ?arile vrajma?ilor lor ?i-Mi voi arata în ei sfin?enia Mea înaintea ochilor a multor neamuri.
Eze 39:28  Atunci vor ?ti ca Eu sunt Domnul Dumnezeul lor, când, dupa ce i-am risipit printre popoare, iara?i îi voi aduna în ?ara lor ?i nu voi mai lasa acolo nici unul din ei.
Eze 39:29  ?i nu voi mai ascunde de ei fa?a Mea pentru ca voi revarsa duhul Meu asupra casei lui Israel, zice Domnul Dumnezeu.
Eze 40:1  În anul al douazeci ?i cincilea dupa robirea noastra, la începutul anului, în ziua a zecea a lunii, la paisprezece ani dupa darâmarea ceta?ii Ierusalimului, tocmai în ziua aceea a fost mâna Domnului peste mine ?i m-a dus în ?ara lui Israel.
Eze 40:2  Dar am fost dus acolo în ni?te vedenii dumnezeie?ti ?i am fost a?ezat pe un munte foarte înalt. Pe acest munte, pe partea lui de miazazi, era un fel de ziduri de cetate.
Eze 40:3  Am fost dus acolo ?i iata era un om, a carui înfa?i?are era ca înfa?i?area aramei stralucitoare, el avea în mâna o sfoara de in ?i o prajina de masurat ?i statea la poarta.
Eze 40:4  Omul acela mi-a zis: "Fiul omului, prive?te cu ochii tai, asculta cu urechile tale ? ia aminte la toate câte am sa-?i arat, caci de aceea ai fost tu adus aici, ca sa-?i arat acestea. Sa veste?ti casei lui Israel tot ce vei vedea".
Eze 40:5  Iata, un zid înconjura templul pe dinafara de jur împrejur ?i în mâna omului aceluia era o prajina de masurat, lunga de ?ase co?i, socotind cotul cât lungimea mâinii de la cot în jos, cu palma cu tot. Omul acela a masurat zidul ?i era gros de o prajina ?i înalt tot de o prajina.
Eze 40:6  Apoi a mers la poarta cea cu fa?a spre rasarit, se urca pe cele ?apte trepte ale ei ?i gasi terasa ei lata de o prajina ?i terasa cea dinauntru lata tot de o prajina.
Eze 40:7  Fiecare din odaile laterale avea lungimea de o prajina ?i la?imea tot de o prajina, iar tinda dintre odai era de cinci co?i.
Eze 40:8  Apoi a masurat pridvorul por?ii dinauntru, ?i era de o prajina.
Eze 40:9  Iar pridvorul celalalt a avut la masuratoare opt co?i ?i stâlpii câte doi co?i. Acest pridvor era la poarta, înauntru, spre templu.
Eze 40:10  Odai de paza la por?ile dinspre rasarit erau trei de o parte ?i trei de cealalta parte; tustrele aveau aceea?i masura ?i aceea?i masura aveau ?i stâlpii de o parte ?i de cealalta.
Eze 40:11  A masurat apoi deschizatura por?ii ?i a gasit zece co?i la?ime ?i treisprezece co?i lungime.
Eze 40:12  Dinaintea odailor de paza era o prispa de un cot ?i la cele de dincolo o prispa tot de un cot. Odaile de dincoace aveau ?ase co?i ?i tot ?ase co?i aveau ?i odaile de dincolo.
Eze 40:13  Apoi a masurat el fa?a por?ii, de la acoperi?ul unei camere pâna la acoperi?ul celeilalte, douazeci ?i cinci de co?i în la?ime. U?ile camerelor erau fa?a în fa?a.
Eze 40:14  La masurarea pridvorului a masurat douazeci de co?i; dinaintea pridvorului era o curte, în fa?a por?ii.
Eze 40:15  De la fa?a de dinafara a por?ii pâna la fa?a ei dinauntru erau cincizeci de co?i.
Eze 40:16  Odaile de paza aveau ferestre cu gratii; asemenea ferestre erau ?i printre stâlpi, spre poarta de jur împrejur. Iar pe stâlpi erau sapate ramuri de finic.
Eze 40:17  Apoi m-a dus omul acela în curtea cea din afara ?l iata acolo erau camere ?i împrejurul cur?ii era facut caldarâm de piatra. Pe acel caldarâm erau treizeci de camere.
Eze 40:18  Caldarâmul acesta era pe laturile por?ii, e raspunzând lungimii lor. Acest caldarâm era mai jos.
Eze 40:19  A masurat apoi la?imea, de la poarta de jos pâna la marginea de afara a cur?ii launtrice, o suta de co?i.
Eze 40:20  Apoi m-a dus spre miazanoapte ?i iata, era ?i acolo o poarta la curtea cea de la margine, care dadea spre miazanoapte ?i a masurat cladirea por?ii cât e de lunga ?i de lata;
Eze 40:21  Camerele de pe laturile ei ?i prispele ei, trei de o parte ?i trei de alta, stâlpii ei erau de masura celor de la poarta cea dinspre rasarit; lungimea cladirii por?ii, cincizeci de co?i ?i la?imea, douazeci ?i cinci de co?i.
Eze 40:22  Ferestrele ei, prispele ei ?i palmierii ei erau ca ?i la poarta care dadea spre rasarit; la ea duc ?apte trepte ?i înaintea ei are pridvor.
Eze 40:23  Dinaintea ei, în curtea cea dinauntru, este o poarta care da spre miazanoapte, ca ?i cea care da spre rasarit. ?i a masurat de la poarta cur?ii de la margine pâna la poarta cur?ii dinauntru ?i a gasit o suta de co?i.
Eze 40:24  Dupa aceea m-a dus spre miazazi, unde era poarta de miazazi; ?i a masurat-o pe ea, stâlpii ?i pridvorul; ?i aveau aceea?i masura.
Eze 40:25  Ferestrele camerelor ?i ale pridvorului erau la fel cu ferestrele cladirilor celorlalte doua por?i; lungimea era de cincizeci de co?i ?i la?imea de douazeci ?i cinci de co?i.
Eze 40:26  Scara dinaintea ei era cu ?apte trepte ?i avea înaintea ei pridvor; ?i palmierii de podoaba erau unul pe un stâlp ?i altul pe alt stâlp de la intrare.
Eze 40:27  ?i în fala ei se afla poarta cur?ii celei dinauntru. A masurat de la poarta de miazazi pâna la poarta cur?ii celei dinauntru o suta de co?i.
Eze 40:28  Apoi m-a dus pe poarta de miazazi în curtea cea dinauntru; ?i a masurat el poarta cea dinspre miazazi ?i a gasit aceea?i masura.
Eze 40:29  Camerele ei de pe laturi, stâlpii ei ?i pridvorul ei aveau aceea?i masura. Împrejur, la camere ?i la pridvor avea ferestre; lungimea era de cincizeci de co?i ?i la?imea de douazeci ?i cinci de coji.
Eze 40:30  Împrejur avea coridor lung de douazeci ?i cinci de co?i ?i lat de cinci.
Eze 40:31  Spre curtea cea de la margine, avea ?i pridvor; pe stâlpii ei erau palmieri sapa?i, iar scara dinaintea ei era cu opt trepte.
Eze 40:32  Apoi m-a dus la poarta cea dinspre rasarit a cur?ii celei dinauntru ?i a masurat-o ?i a gasit aceea?i masura.
Eze 40:33  Camerele ei cele de pe laturi, stâlpii ei ?i pridvorul ei erau de aceea?i masura. Împrejur avea ferestre la camere ?i la pridvor. Lungimea ei era de cincizeci de co?i ?i la?imea de douazeci ?i cinci de co?i.
Eze 40:34  Pridvorul ei era spre curtea cea de la margine ?i avea palmieri sapa?i pe stâlpii ei de o parte ?i de alta a intrarii; iar scara ei avea opt trepte.
Eze 40:35  Dupa aceea m-a dus la poarta cea dinspre miazanoapte ?i a masurat-o ?i a gasit aceea?i masura.
Eze 40:36  Ea avea camere pe de laturi, stâlpi, pridvor ?i ferestre pe din afara; în lungime avea cincizeci de coli, iar în la?ime douazeci ?i cinci de coli.
Eze 40:37  Pridvorul ei era spre curtea cea de la margine ?i palmieri avea ?i pe unul ?i pe altul din stâlpii de la intrare; iar scara ei avea opt trepte.
Eze 40:38  Mai era o camera care se deschidea spre pridvorul por?ii; acolo se spala jertfa arderii de tot.
Eze 40:39  Iar în pridvorul por?ii erau doua mese de o parte a intrarii ?i doua mese de cealalta parte, pe care se taiau jertfele arderii de tot, jertfele pentru pacat ?i jertfele pentru vina.
Eze 40:40  Pe latura din afara a pridvorului, spre rasarit, aproape de intrarea por?ii celei dinspre miazanoapte, se aflau doua mese ?i pe cealalta latura a pridvorului, spre apus, se aflau iar doua mese.
Eze 40:41  A?adar erau patru mese de o parte ?i patru mese de cealalta parte; în total opt mese, pe care se taiau jertfele.
Eze 40:42  Patru mese, pentru pregatirea arderilor de tot, erau de piatra cioplita, lungi de un cot ?i jumatate ?i late de un cot ?i jumatate ?i înalte de un cot. Pe ele se puneau uneltele de junghiat, arderile de tot ?i jertfele celelalte.
Eze 40:43  Pe margine, de jur împrejur mesele aveau un pervaz din ele, înalt de un lat de mâna; iar deasupra meselor era acoperamântul, ca sa le apere de ploaie ?i de caldura.
Eze 40:44  În curtea cea dinauntru, în partea din afara a cladirilor por?ilor, erau doua camere pentru cântare?i: una pe latura cladirii por?ii dinspre miazanoapte, cu fa?a spre miazazi, iar cealalta pe latura cladirii por?ii celei de miazazi, cu fa?a spre miazanoapte.
Eze 40:45  ?i mi-a zis barbatul acela: Camera aceasta, cu fa?a spre miazazi, este pentru preo?i care vegheaza la paza templului;
Eze 40:46  Iar camera cea cu fa?a spre miazanoapte este pentru preo?ii care fac slujba la altar; ace?tia sunt fiii lui ?adoc, singurii dintre fiii lui Levi care se apropie de Domnul ca sa-I slujeasca.
Eze 40:47  Apoi a masurat curtea ?i a gasit o suta de co?i în lungime ?i o suta de co?i în la?ime; ea era în patru col?uri, iar în fa?a templului se ridica altarul.
Eze 40:48  Apoi m-a dus în pridvorul templului, a masurat stâlpi pridvorului ?i a gasit cinci co?i de o parte ?i cinci co?i de cealalta parte, de la ?â?ânile u?ilor pâna în pere?i trei co?i de o parte ?i trei co?i de alta parte.
Eze 40:49  Lungimea pridvorului era de unsprezece co?i ?i la?imea de douazeci de co?i. La el suia o scara cu zece trepte. ?i lânga stâlpi erau coloane: una de o parte ?i alta de alta parte a intrarii.
Eze 41:1  Dupa aceea m-a dus în templu, a masurat stâlpii ?i a gasit ?ase co?i în la?ime de o parte ?i ?ase co?i de cealalta parte; aceasta era largimea cortului adunarii.
Eze 41:2  Largimea u?ii era de zece co?i ?i de amândoua par?ile u?ii câte cinci co?i. A masurat apoi lungimea templului ?i a gasit-o de patruzeci de co?i, iar la?imea de douazeci de coti.
Eze 41:3  A mers înauntrul templului ?i a masurat stâlpii de la u?a ?i i-a gasit de doi co?i, iar u?a de ?ase co?i; de la ?â?ânile u?ii pâna în perete a gasit ?apte co?i de o parte ?i ?apte co?i de cealalta parte.
Eze 41:4  A masurat loca?ul ?i a gasit douazeci de co?i în lungime, douazeci de co?i în la?ime ?i mi-a zis: "Aceasta este Sfânta Sfintelor".
Eze 41:5  Apoi a masurat peretele templului ?i l-a gasit gros de ?ase co?i; la?imea camerelor de pe laturile templului de jur împrejur a gasit-o de patru co?i.
Eze 41:6  În jurul templului sunt trei rânduri de câte treizeci de camere, camera lânga camera. Ele intra într-un zid, care este facut împrejurul templului anume pentru aceste camere, ca ele sa fie întarite, dar de peretele templului nu se ating.
Eze 41:7  Camerele cu cât sunt mai sus, cu atât sunt mai încapatoare, sub?iindu-se peretele. Din camerele de jos te urci la cele din mijloc ?i de la cele din mijloc la cele de sus; sui?ul este învârtit, caci te sui pe o scara în chipul melcului.
Eze 41:8  ?i am vazut un caldarâm înalt împrejurul templului, care slujea de temelie pentru camerele laterale, care avea la?imea de o prajina întreaga, adica de sase coti mari.
Eze 41:9  Grosimea zidului camerelor laterale, care ie?eau în afara, era de cinci co?i, iar lânga camerele laterale era un loc gol.
Eze 41:10  Locul gol dintre camerele laterale ale templului ?i dintre camerele dimprejurul cur?ii templului are o la?ime de douazeci de co?i de jur împrejur.
Eze 41:11  U?ile camerelor laterale dadeau într-un loc deschis, o u?a în partea de miazanoapte, iar alta u?a în partea de miazazi; iar la?imea locului deschis era de cinci co?i.
Eze 41:12  Cladirea de dinaintea locului deschis din partea de apus avea o la?ime de ?aptezeci do co?i; zidul acestei cladiri era de cinci co?i de jur împrejur, iar lungimea ei era de nouazeci de co?i.
Eze 41:13  A masurat templul; el avea o suta de co?i în lungime; locul liber, cladirea de la apus ?i zidurile lui de asemenea aveau o lungime de o suta de coti.
Eze 41:14  La?imea fe?ei templului ?i curtea din partea dinspre rasarit era de o suta de co?i.
Eze 41:15  Apoi a masurat lungimea cladirii din fa?a locului liber din spatele templului cu camerele laterale de o parte ?i de alta a lui ?i avea o suta de co?i.
Eze 41:16  U?orii u?ilor ?i ai ferestrelor cu gratii, camerele laterale din cele trei caturi, pe jos ?i de jos pâna Ia ferestre, de jur împrejur erau captu?ite cu lemn. Ferestrele erau închise.
Eze 41:17  Pâna la înal?imea u?ilor, tot peretele, atât la despar?itura cea mai dinauntru, cât ?i la cea mai din afara, de jur împrejur, înauntru ?i pe afara, era împodobit cu chipuri sapate cu anumita masura;
Eze 41:18  ?i anume, erau sapa?i heruvimi ?i palmieri, astfel: între fiecare doi heruvimi un palmier ?i fiecare heruvim avea doua fe?e.
Eze 41:19  Heruvimul într-o parte avea o fa?a de om, întoarsa spre un palmier, ?i în cealalta parte avea o fa?a de leu, întoarsa spre alt palmier. A?a erau facute chipuri în tot templul ?i împrejur.
Eze 41:20  De jos pâna la înal?imea u?ilor erau sculpta?i heruvimi ?i palmieri ca ?i pe pere?ii templului.
Eze 41:21  În templu, u?orii u?ilor erau în patru muchii. Iar în fa?a Sfintei Sfintelor se vedea un fel de altar de lemn.
Eze 41:22  Altarul era de lemn, înalt de trei co?i ?i lung de doi co?i; ?i coarnele lui ?i postamentul lui ?i pere?ii lui erau de lemn. ?i mi-a zis barbatul acela: Aceasta este masa cea de dinaintea Domnului.
Eze 41:23  Sfânta Sfintelor avea doua u?i ?i Sfânta avea doua u?i.
Eze 41:24  Fiecare din acele doua u?i avea câte doua canaturi, care se deschideau într-o parte ?i în alta, caci doua canaturi erau la o u?a ?i doua la cealalta u?a.
Eze 41:25  ?i pe ele, pe u?ile templului, erau sculpta?i heruvimi ?i palmieri, ca ?i pe pere?i. Iar în fa?a pridvorului, afara, era o pardoseala de lemn.
Eze 41:26  Pe o latura ?i pe alta a pridvorului erau ferestre cu gratii ?i chipuri de palmieri; asemenea erau ?i în camerele laterale ?i pe captu?eala de lemn.
Eze 42:1  Dupa aceea m-a scos spre curtea cea din afara spre miazanoapte ?i m-a dus la camerele din fa?a cur?ii, din fa?a cladirii, spre miazanoapte,
Eze 42:2  La acel loc care este spre poarta de miazanoapte a cur?ii dinauntru ?i care are în lungime o suta de co?i, iar în la?ime cincizeci de co?i;
Eze 42:3  Adica în dreptul locului de douazeci de co?i al cur?ii dinauntrul ?i în dreptul caldarâmului cur?ii din afara, unde era o galerie cu trei rânduri în fa?a altei galerii cu trei rânduri.
Eze 42:4  Pe dinaintea camerelor era un loc de trecere de zece co?i lat ?i de o suta de co?i lung. U?ile erau spre miazanoapte.
Eze 42:5  Camerele cele de sus erau mai strâmte decât cele de jos ?i cele de la mijloc ale cladirii, pentru ca galeriile le rapeau o parte din întinderea lor.
Eze 42:6  Ele aveau trei caturi, dar nu aveau stâlpi ca în curte. De aceea, plecând de la pamânt, camerele de sus erau mai strâmte decât cele de jos ?i decât cele de la mijloc.
Eze 42:7  Zidul din afara, paralel cu camerele, dinspre curtea cea din afara, din fa?a camerelor, avea în lungime cincizeci de co?i,
Eze 42:8  Pentru ca ?i camerele dinspre curtea cea din afara aveau o lungime tot numai de cincizeci de co?i. ?i aceste doua cladiri care erau în dreptul templului aveau o suta de co?i.
Eze 42:9  Iar de jos, intrarea la aceste camere era dinspre rasarit, cum veneai din curtea cea din afara.
Eze 42:10  Se aflau de asemenea camere ?i în lungul zidului cur?ii dinauntru, din partea dinspre miazazi, în fa?a cur?ii ?i a cladirii templului;
Eze 42:11  Dinaintea lor era un loc de trecere întocmai ca ?i la camerele cele dinspre miazanoapte ?i avea aceea?i lungime ca ?i acelea ?i aceea?i la?ime; toate ie?irile lor, întocmirea lor ?i u?ile lor erau la fel;
Eze 42:12  Tot a?a era ?i cu u?ile camerelor de la miazazi. Apoi era o u?a de la capatul locului de trecere, ce mergea de-a lungul zidului drept spre rasarit.
Eze 42:13  ?i mi-a zis barbatul acela: "Camerele dinspre miazanoapte ?i camerele dinspre miazazi, care sunt în fa?a cur?ii, sunt camere sfinte, în care preo?ii care se apropie de Domnul manânca cele mai sfinte jertfe; tot acolo pun ei cele mai sfinte jertfe ?i prinosul de pâine, jertfa pentru pacat ?i jertfa pentru vina, caci acesta este loc sfânt.
Eze 42:14  Când preo?ii intra acolo, nu se cuvine sa iasa din acest loc sfânt în curtea cea din afara, pâna nu lasa acolo hainele lor cu care au fost îmbraca?i la slujba, ca acestea sunt sfin?ite; ei trebuie sa se îmbrace cu alte haine ?i numai dupa aceea sa iasa la popor.
Eze 42:15  Dupa ce a ispravit el de masurat templul ?i cur?ile cele dinauntrul zidurilor, m-a scos pe poarta dinspre rasarit ?i a început sa masoare împrejur.
Eze 42:16  ?i a masurat latura cea dinspre rasarit cu prajina de masurat ?i a gasit în total cinci sute de co?i.
Eze 42:17  Pe latura de miazanoapte cu aceea?i prajina a masurat cinci sute de co?i.
Eze 42:18  Pe latura de miazazi a masurat cu prajina de masurat tot cinci sute de co?i.
Eze 42:19  Apoi, apucând pe latura de apus, a masurat cu prajina de masurat cinci sute de co?i.
Eze 42:20  A masurat în cele patru laturi zidul de jur împrejurul loca?ului sfânt; lungimea era de cinci sute de co?i ?i la?imea de cinci sute de co?i; zidul acesta despar?ea locul sfânt de cel ce nu este sfânt.
Eze 43:1  Apoi m-a dus la poarta, la poarta dinspre rasarit.
Eze 43:2  ?i iata slava Dumnezeului lui Israel venea dinspre rasarit; glasul Lui era ca glasul de ape multe ?i pamântul stralucea de slava Lui.
Eze 43:3  Vedenia aceasta era ca aceea pe care o vazusem mai înainte, tocmai ca aceea pe care o vazusem când am venit sa vestesc pieirea ceta?ii; vedenia aceasta era asemenea vedeniei pe care o vazusem la râul Chebar. Atunci am cazut cu fa?a la pamânt.
Eze 43:4  ?i slava Domnului a intrat în templu pe poarta care este cu fa?a spre rasarit.
Eze 43:5  ?i m-a ridicat Duhul ?i m-a dus în curtea cea dinauntru ?i iata slava Domnului umplea tot templul.
Eze 43:6  ?i am auzit pe Cineva Care-mi graia din templu, iar barbatul acela de mai înainte statea lânga mine.
Eze 43:7  ?i mi-a zis glasul: "Fiul omului, acesta este locul tronului Meu ?i locul pe care-Mi pun talpile picioarelor Mele, unde voi locui ve?nic în mijlocul fiilor lui Israel; casa lui Israel nu va mai întina numele Meu cel sfânt, nici ea, nici regii ei, prin desfrânarile lor, prin cadavrele regilor lor, cu locurile lor înalte.
Eze 43:8  El î?i puneau pragurile lor lânga pragurile Mele ?i ?â?ânile u?ilor lor lânga ?â?ânile u?ilor Mele, încât un singur perete era între Mine ?i ei, ?i întinau numele Meu cel sfânt cu urâciunile lor pe care le faceau, ?i de aceea i-am pierdut întru mânia Mea.
Eze 43:9  Iar acum ei vor departa de la Mine desfrânarile lor ?i trupurile moarte ale regilor lor ?i Eu voi locui în mijlocul lor în veci.
Eze 43:10  Iar tu, fiul omului, descrie casei lui Israel acest templu, ca sa se ru?ineze ei de faradelegile lor ?i sa-i masoare planul.
Eze 43:11  Daca ei se vor ru?ina de toate acelea câte au facut, atunci sa le ara?i chipul templului ?i planul lui, ie?irile lui ?i intrarile lui ?i forma lui ?i toate întocmirile lui, toate formele lui ?i toate legile lui; pune toate acestea în scris înaintea ochilor lor, ca ei sa vada forma lui ?i toate întocmirile lui ?i sa le urmeze întocmai.
Eze 43:12  Iata acum legea templului: pe vârful muntelui tot locul care îl înconjoara este locul cel mai sfânt. Aceasta este legea templului.
Eze 43:13  Iata masurile jertfelnicului în co?i, socotind drept cot bra?ul de la cot în jos, împreuna cu palma: soclul de jos al lui era înalt de un cot ?i lat tot de un cot; iar brâul, care-l încingea pe margine, avea o palma în la?ime. Acesta era soclul jertfelnicului.
Eze 43:14  Pe soclu, care era la pamânt, se înal?a un fundament mic, care mergea ca un fel de prispa de jur împrejurul jertfelnicului, înalta de doi co?i ?i lata de un cot. Pe fundamentul mic se înal?a fundamentul mare, care iara?i încingea jertfelnicul ca o prispa înalta de patru co?i ?i lata de un cot.
Eze 43:15  Pe fundamentul mare se înal?a însu?i jertfelnicul, înalt de patru co?i; ?i din jertfelnic se ridicau patru coarne.
Eze 43:16  Jertfelnicul avea doisprezece co?i înal?ime ?i doisprezece co?i în lungime. El era în patru col?uri, având toate cele patru laturi ale sale deopotriva de lungi.
Eze 43:17  Iar fa?a soclului jertfelnicului era de paisprezece co?i în lungime ?i de paisprezece co?i înal?ime ?i împrejurul ei avea un brâu de o jumatate de cot. Soclul de pus împrejur era lat de un cot, iar scara de suit la jertfelnic era în partea de rasarit.
Eze 43:18  Apoi mi-a zis barbatul acela: "Fiul omului, a?a graie?te Domnul Dumnezeu: Iata rânduielile jertfelnicului pentru ziua când va fi el facut, ca sa se aduca pe el arderi de tot ?i ca sa fie stropit cu sânge.
Eze 43:19  Preo?ilor din tribul lui Levi, care sunt din neamul lui Sadoc ?i care se apropie de Mine ca sa-Mi slujeasca, da-le, zice Domnul Dumnezeu, un vi?el din cireada de boi ca jertfa pentru pacat;
Eze 43:20  ?i sa iei sângele lui ?i sa strope?ti cele patru coarne ale jertfelnicului ?i pe cele patru col?uri ale soclului lui ?i brâul cel dimprejur ?i astfel sa-l cure?i ?i sa-l sfin?e?ti.
Eze 43:21  Apoi ia vi?elul cel de jertfa pentru pacat ?i arde-l la locul rânduit al templului, dar afara din locul cel sfânt.
Eze 43:22  Iar a doua zi, ca jertfa pentru pacat, sa aduci din turma de capre un ?ap fara meteahna ?i sa cure?i jertfelnicul tot a?a, cum l-a?i cura?it cu vi?elul.
Eze 43:23  Iar dupa ce vei ispravi cura?irea, ia din cireada de boi un vi?el fara meteahna ?i din turma de oi un berbec fara meteahna,
Eze 43:24  ?i-i adu înaintea Domnului; preo?ii sa arunce asupra lor sare ?i sa-i înal?e ardere de tot Domnului.
Eze 43:25  ?apte zile se aduce jertfa pentru pacat câte un ?ap pe zi; de asemenea sa se aduca jertfa câte un vi?el din cireada de boi ?i câte un berbec din turma de oi, fara meteahna.
Eze 43:26  ?apte zile sa faca ispa?ire pentru jertfelnic, sa-l cure?e ?i sa-l sfin?easca.
Eze 43:27  Iar dupa sfâr?itul acestor zile, în ziua a opta ?i mai departe, preo?ii vor înal?a, pe jertfelnic, arderile de tot ale voastre ?i jertfele de împacare ?i Eu Ma voi milostivi spre voi", zice Domnul Dumnezeu.
Eze 44:1  Apoi m-a dus barbatul acela înapoi la poarta cea din afara a templului, spre rasarit, ?i aceasta era închisa.
Eze 44:2  ?i mi-a zis Domnul: "Poarta aceasta va fi închisa, nu se va deschide ?i nici un om nu va intra pe ea, caci Domnul Dumnezeul lui Israel a intrat pe ea. De aceea va fi închisa.
Eze 44:3  Cit prive?te pe rege, el se va a?eza acolo, ca sa manânce pâine înaintea Domnului; pe calea por?ii va intra ?i pe aceea?i cale va ie?i".
Eze 44:4  Dupa aceea m-a dus pe calea por?ii de la miazanoapte, în fa?a templului, ?i am privit, ?i iata slava Domnului umplea templul Domnului, ?i am cazut cu fa?a la pamânt.
Eze 44:5  ?i mi-a zis Domnul: "Fiul omului, pleaca-?i inima ta, prive?te cu ochii tai ?i asculta cu urechile tale toate câte-?i voi grai despre toate a?ezamintele templului Domnului ?i despre toate legile ei. Uita-te cu bagare de seama la intrarea templului ?i la toate ie?irile din loca?ul cel sfânt.
Eze 44:6  ?i spune casei celei razvratite a lui Israel: A?a graie?te Domnul Dumnezeu: Destul voua, casa lui Israel, cu toate urâciunile voastre;
Eze 44:7  Ca a?i bagat înauntru fii straini, netaia?i împrejur la inima ?i netaia?i împrejur la trup, ca sa stea în loca?ul Meu cel sfânt ?i sa spurce templul Meu; a?i adus pâinea Mea, grasimea ?i sângele ?i a?i stricat legamântul Meu cu tot felul de urâciuni de ale voastre.
Eze 44:8  Voi n-a?i facut slujba Mea în templu, ci i-a?i pus pe ei sa îndeplineasca slujba voastra în templul Meu în locul vostru".
Eze 44:9  A?a zice Domnul Dumnezeu: "Nici un fiu strain, netaiat împrejur ia inima ?i netaiat împrejur la trup, nu trebuie sa intre în loca?ul Meu cel sfânt, nici chiar acel fiu care locuie?te în mijlocul fiilor lui Israel.
Eze 44:10  Chiar ?i levi?ii, care s-au departat de Mine în timpul ratacirii lui Israel pentru a-?i urma idolii lor, î?i vor purta greutatea pacatului lor.
Eze 44:11  Ei vor sluji în templul Meu ca strajeri la por?ile templului ?i facând slujba templului; ei vor junghia pentru popor arderi de tot ?i alte jertfe ?i vor sta înaintea lui ca sa-i slujeasca.
Eze 44:12  Pentru ca ei au slujit înaintea idolilor lui ?i au fost pentru casa lui Israel sminteala ?i au dus-o la necredin?a, Mi-am ridicat mâna împotriva lor, zice Domnul Dumnezeu, ?i î?i vor lua pedeapsa pentru vinova?ia lor.
Eze 44:13  Ei nu se vor apropia de Mine ca sa slujeasca înaintea Mea; nu se vor apropia de lucrurile Mele cele sfinte, nici de Sfânta Sfintelor, ci vor purta asupra lor necinstea ?i urâciunile lor, pe care le-au facut.
Eze 44:14  Îi voi face strajeri la templu, sa faca slujba lui, ?i tot ce trebuie facut la el.
Eze 44:15  Iar preo?ii din semin?ia lui Levi, fiii lui Sadoc, care în vremea abaterii de la Mine a fiilor lui Israel au îndeplinit slujirea Mea în loca?ul Meu cel sfânt, aceia se vor apropia de Mine, ca sa-Mi slujeasca ?i vor sta înaintea fe?ei Mele, ca sa-Mi aduca grasime ?i sânge, zice Domnul Dumnezeu.
Eze 44:16  Ei vor intra în loca?ul Meu cel sfânt ?i se vor apropia de masa Mea, ca sa-Mi slujeasca; ei vor îndeplini slujirea Mea.
Eze 44:17  Când vor veni la poarta cur?ii dinauntru, atunci se vor îmbraca în haine de in, iar haine de lâna nu trebuie sa aiba pe ei în timpul slujbei lor în por?ile cur?ii dinauntru ?i în templu.
Eze 44:18  Turbanele de pe capetele lor trebuie sa fie tot de in; hainele cele de pe dedesubt de pe coapsele lor sa fie de asemenea de in. Ei nu trebuie sa se încinga, ca sa nu transpire.
Eze 44:19  Iar când va trebui sa iasa în curtea cea de la margine, la popor, atunci trebuie sa dezbrace hainele lor cu care au slujit ?i sa le lase în camerele cele sfin?ite ?i sa se îmbrace "u alte haine, ca sa nu se atinga de popor cu hainele lor cele sfin?ite.
Eze 44:20  Capetele lor nu trebuie sa ?i le rada, dar nici parul sa nu-?i lase sa creasca, ci sa-?i tunda neaparat capul.
Eze 44:21  Vin nu trebuie sa bea nici un preot când are sa intre în curtea cea dinauntru;
Eze 44:22  Nici vaduva, nici despar?ita de barbat nu trebuie sa ia ei de femeie, ci pot sa ia numai fata din neamul casei lui Israel ?i vaduva care a ramas în vaduvie dupa moartea unui preot.
Eze 44:23  Ei trebuie sa înve?e pe poporul Meu a deosebi ce este sfânt de ce nu este sfânt ?i sa le lamureasca ce este curat ?i ce este necurat.
Eze 44:24  În pricinile nehotarâte, ei trebuie sa ia parte la judecata ?i vor judeca dupa a?ezamintele Mele ?i legile Mele vor pazi ?i toate rânduielile Mele cele pentru sarbatorile Mele ?i pentru zilele Mele de odihna le vor pazi cu sfin?enie.
Eze 44:25  De omul mort nimeni din ei nu trebuie sa se apropie, ca sa nu se faca necurat; numai pentru tata ?i pentru mama, pentru fiu ?i pentru fiica, pentru frate ?i sora nemaritata pot sa se faca necura?i.
Eze 44:26  Dupa cura?irea acestuia, trebuie sa i se mai socoteasca înca ?apte zile.
Eze 44:27  ?i în ziua aceea, când trebuie sa se apropie de cele sfinte în curtea cea dinauntru, ca sa slujeasca în templu, trebuie sa aduca jertfa pentru pacat, zice Domnul Dumnezeu.
Eze 44:28  Iar cât prive?te partea lor de mo?ie, apoi Eu sunt partea lor; ?i mo?ie nu li se va da întru Israel, caci Eu sunt mo?ia lor.
Eze 44:29  Ei vor mânca din prinosul de pâine, din jertfa pentru pacat ?i din jertfa pentru vina. ?i tot ce este afierosit în Israel al lor este.
Eze 44:30  Pârga din toate roadele voastre ?i din tot felul de prinoase, ori din ce s-ar alcatui prinoasele voastre, este a preo?ilor. Pârga din cele treierate ale voastre sa o da?i preotului, ca sa odihneasca binecuvântarea asupra casei tale.
Eze 44:31  Nici un fel de mortaciuni ?i nimic sfâ?iat de fiara, nici de pasari, nici de dobitoace nu trebuie sa manânce preo?ii.
Eze 45:1  Când ve?i împar?i pamântul în par?i prin sor?i, atunci sa osebi?i o parte sfânta a Domnului de douazeci ?i cinci de mii de co?i în lungime ?i douazeci de mii în la?ime, ca sa fie sfânt acest loc în toate hotarele lui de jur împrejur.
Eze 45:2  Din el va merge la loca?ul sfânt o bucata în patru col?uri de cinci sute de co?i pe cinci sute de co?i ?i împrejurul lui o fâ?ie de loc lata de cincizeci de coti.
Eze 45:3  De la locul pe care va fi loca?ul cel sfânt, Sfânta Sfintelor, vei masura cele douazeci ?i cinci de mii de co?i în lungime ?i zece mii de co?i în la?ime;
Eze 45:4  Aceasta parte sfânta de pamânt va fi a preo?ilor care slujesc loca?ului sfânt ?i care se apropie sa slujeasca Domnului; acesta va fi pentru ei loc de case ?i pentru loca?ul sfânt.
Eze 45:5  Douazeci ?i cinci de mii de co?i în lungime ?i zece mii de co?i în la?ime va fi bucata de pamânt a Levi?ilor, slujitorii templului, ca mo?ie a lor cu ceta?i de locuit.
Eze 45:6  În stapânirea ceta?ii sa da?i cinci mii de co?i în la?ime ?i douazeci ?i cinci de mii în lungime, în fa?a locului sfânt, care este osebit pentru Domnul. Acesta trebuie sa fie al întregii oase a lui Israel.
Eze 45:7  ?i regelui sa-i da?i parte de pamânt, de o parte ?i de alta a locului sfânt care este osebit Domnului ?i a locului ceta?ii, adica o parte la rasarit, în partea de rasarit a celor doua por?iuni ?i o parte la asfin?it, în partea de asfin?it a celor doua por?iuni. În lungime se vor întinde ca una din acele par?i de la hotarul de apus pâna la hotarul de rasarit al ?arii.
Eze 45:8  Acesta este pamântul lui, mo?ia lui în Israel, ca regii Mei de acum sa nu mai strâmtoreze poporul Meu, ?i ca sa lase pamântul casei lui Israel dupa triburile ei.
Eze 45:9  A?a zice Domnul Dumnezeu: "Destul, regi ai lui Israel! Lasa?i nedrepta?ile ?i împilarile ?i face?i judecata ?i dreptate! Înceta?i de a mai asupri pe poporul Meu, zice Domnul Dumnezeu.
Eze 45:10  Sa ave?i cântar drept ?i efa dreapta ?i bat drept.
Eze 45:11  Efa ?i batul trebuie sa fie masuri deopotriva de mari, încât într-un bat sa încapa a zecea parte dintr-un homer ?i într-o efa sa încapa a zecea parte dintr-un homer. Marimea lor trebuie masurata cu homerul.
Eze 45:12  Siclul sa aiba douazeci de ghere. Mina va fi de douazeci de sicli, de douazeci ?i cinci de sicli ?i de cincisprezece sicli.
Eze 45:13  lata ofranda ce trebuie sa da?i regelui: a ?asea parte de efa din homerul de grâu ?i a ?asea parte de efa din homerul de orz.
Eze 45:14  Hotarârea pentru untdelemn: dintr-o cora de untdelemn ve?i da a zecea parte dintr-un bat; zece baturi fac un homer, caci homerul are zece baturi;
Eze 45:15  Ve?i da o oaie dintr-o turma de doua sute de oi, din pa?unile cele manoase ale lui Israel. Toate acestea le ve?i da pentru prinosul de pâine ?i ardere de tot ?i jertfa de împacare spre cura?irea voastra, zice Domnul Dumnezeu.
Eze 45:16  Tot poporul ?arii este îndatorat sa dea aceste prinoase regelui lui Israel,
Eze 45:17  Iar în sarcina regelui vor fi arderile de tot, prinosul de pâine ?i turnarile la sarbatori, la luna noua, la ziua de odihna ?i la toate praznuirile casei lui Israel. El va trebui sa aduca jertfa pentru pacat, prinos de pâine, ardere de tot ?i jertfa de împacare pentru ispa?irea casei lui Israel".
Eze 45:18  Asa zice Domnul Dumnezeu: "În ziua întâi a lunii întâi ia din cireada de boi un junc fara meteahna ?i cura?a loca?ul sfânt.
Eze 45:19  Preotul sa ia din sângele acestei jertfe pentru pacat ?i sa stropeasca cu el u?orii u?ii templului, cele patru laturi ala jertfelnicului ?i u?orii por?ilor cur?ii celei dinauntru.
Eze 45:20  Acela?i lucru sa-l faci ?i în ziua a ?aptea a lunii, pentru cei ce au gre?it cu ?tiin?a ?i din ne?tiin?a ?i a?a sa cure?i templul.
Eze 45:21  În ziua a paisprezecea a lunii întâi, trebuie sa praznui?i Pa?tile, sarbatoare de ?apte zile, când trebuie sa se manânce azime.
Eze 45:22  În aceasta zi regele va aduce pentru sâne ?i pentru tot poporul ?arii un vi?el ca jertfa pentru pacat.
Eze 45:23  ?i în cele ?apte zile ale sarbatorii el trebuie sa aduca ardere de tot Domnului în fiecare zi câte ?apte vi?ei ?i câte ?apte berbeci fara meteahna, iar ca jertfa pentru pacat sa aduca în fiecare zi câte un ?ap din turma de capre.
Eze 45:24  Prinos de pâine trebuie sa aduca el câte o efa de fiecare vi?el ?i câte o efa de fiecare berbec ?i câte un hin de untdelemn la efa.
Eze 45:25  În ziua a cincisprezecea a lunii a ?aptea, la sarbatoarea corturilor, timp de ?apte zile, el trebuie sa aduca la fel: aceea?i jertfa pentru pacat, aceea?i ardere de tot ?i tot atâta prinos de pâine ?i tot atâta untdelemn".
Eze 46:1  A?a zice Domnul Dumnezeu: "Poarta cur?ii celei dinauntru, care da spre rasarit, trebuie sa fie încuiata în timpul celor ?apte zile de lucru, iar în ziua odihnei ea trebuie sa fie deschisa, ?i în ziua de luna noua trebuie sa fie deschisa.
Eze 46:2  Regele va trece prin pridvorul din afara al por?ii acesteia ?i va sta la u?orul acestei por?i; iar preo?ii vor savâr?i arderea de tot a lui, ?i jertfa lui cea de împacare; ?i el din pragul por?ii se va închina Domnului ?i va ie?i, iar poarta va ramâne neîncuiata pâna seara.
Eze 46:3  Poporul ?arii se va închina înaintea Domnului, dinaintea por?ii, în ziua odihnei ?i la luna noua.
Eze 46:4  Arderea de tot pe care regele trebuie sa o aduca Domnului în ziua odihnei, trebuie sa fie de ?ase miei fara meteahna ?i un berbec fara meteahna,
Eze 46:5  Prinosul de pâine de o efa de faina, cu berbecul ?i cu mieii, cât îi va da mâna, iar untdelemnul, un hin la efa.
Eze 46:6  În ziua de luna noua se va aduce de el un junc fara meteahna din cireada de boi ?i de asemenea ?ase miei ?i un berbec fara meteahna.
Eze 46:7  Prinos de pâine va aduce o efa cu vi?elul ?i o efa cu berbecul, iar cu mieii, cât îi va da mâna, untdelemn, câte un hin la efa.
Eze 46:8  Când vine regele, trebuie sa intre prin pridvorul por?ii celei dinauntru ?i tot pe acolo sa iasa.
Eze 46:9  Iar când va veni poporul ?arii înaintea fe?ei Domnului, la sarbatori, atunci, intrând pe poarta de miazanoapte pentru închinare, trebuie sa iasa pe poarta de miazazi, iar când intra pe poarta de miazazi, trebuie sa iasa pe poarta de miazanoapte; el nu trebuie sa iasa tot pe acea poarta pe care a intrat, ci trebuie sa iasa pe cea din fa?a aceleia.
Eze 46:10  Regele trebuie sa fie în mijlocul lor; când intra ei, intra ?i el, ?i când ies ei, iese ?i el.
Eze 46:11  În zilele de sarbatoare ?i în zilele de bucurie, prinosul de pâine din partea lui trebuie sa fie de câte o efa, cu vi?elul ?i berbecul, iar cu mieii, cât îi va da mina, iar untdelemn, câte un hin la efa.
Eze 46:12  Iar daca regele, din evlavie, vrea sa aduca ardere de tot Domnului, trebuie sa i se deschida por?ile cete de la rasarit ?i el va savâr?i arderea sa de tot ?i jertfa sa de mul?umire tot a?a cum a savâr?it-o în ziua odihnei, ?i dupa aceea va ie?i, iar dupa ie?irea lui poarta se va închide.
Eze 46:13  În fiecare zi sa aduci Domnului ardere de tot un miel de un an, fara meteahna; în fiecare diminea?a sa-l aduci.
Eze 46:14  Iar ca prinos de pâine sa adaugi la el în fiecare diminea?a a ?asea parte de efa ?i untdelemn a treia parte din hin, ca sa amesteci faina. Aceasta este o lege ve?nica despre prinosul de pâine ce trebuie sa se aduca Domnului totdeauna.
Eze 46:15  Sa se aduca ardere de tot un miel ?i prinos de pâine ?i untdelemn necontenit în fiecare diminea?a.
Eze 46:16  A?a zice Domnul Dumnezeu: "Daca regele va da vreunuia din fiii sai dar, acest dar trebuie sa treaca mo?tenire ?i la fiii aceluia. Ei îl vor stapâni ca pe o mo?tenire.
Eze 46:17  Iar daca el da din mo?tenirea sa cuiva din robii sai dar, acesta va fi al aceluia numai pâna la anul jubileu ?i atunci se va întoarce la rege. Mo?tenirea lui poate trece numai la fiii lui.
Eze 46:18  Dar regele nu poate lua din partea de mo?tenire a poporului, strâmtorându-l în mo?tenirea lui. El numai din mo?tenirea sa poate împar?i copiilor sai, ca nimeni din poporul Meu sa nu fie izgonit din mo?tenirea iuim.
Eze 46:19  Apoi m-a dus barbatul acela pe calea por?ii celei de miazazi la camerele sfin?ite ale preo?ilor, dinspre miazanoapte, ?i iata era un loc în fund, spre apus.
Eze 46:20  ?i mi-a zis: "Acesta este locul unde preo?ii trebuie sa fiarba jertfa cea pentru vina ?i jertfa cea pentru pacat; unde trebuie sa coaca pâinile din prinoase, fara sa le scoata în curtea cea din afara, pentru sfin?irea poporului".
Eze 46:21  Dupa aceea m-a scos în curtea cea din afara ?i m-a dus în cele patru col?uri ale cur?ii ?i iata în fiecare col? al cur?ii era înca o curte.
Eze 46:22  În toate patru col?urile cur?ii erau cur?i acoperite, de patruzeci de co?i în lungime ?i de treizeci în la?ime; cur?ile din toate patru col?urile aveau o singura masura.
Eze 46:23  ?i împrejurul celor patru cur?i erau ziduri, iar pe lânga pere?i de jur împrejur erau vetre pentru gatit mâncare.
Eze 46:24  ?i mi-a zis barbatul acela: "Iata bucatariile în care slujitorii templului fierb jertfele poporului".
Eze 47:1  Apoi m-a dus înapoi la u?a templului ?i iata de sub pragul templului curgea o apa spre rasarit; pentru ca templul era cu fa?a spre rasarit ?i apa curgea de sub partea dreapta a templului, pe partea de miazazi a jertfelnicului.
Eze 47:2  M-a scos spoi pe partea cea de la miazanoapte ?i m-a dus pe din afara, împrejur, la poarta care da spre rasarit ?i iata apa curgea pe partea dreapta.
Eze 47:3  Când barbatul acela mergea spre rasarit, ?inea în mâna sfoara ?i a masurat o mie de co?i; ?i m-a dus prin apa ?i apa era pâna la glezne.
Eze 47:4  A mai masurat apoi o mie de co?i ?i m-a dus prin apa, ?i apa era pâna la genunchi. ?i a mai masurat înca o mie de con ?i apa era pâna la brâu.
Eze 47:5  ?i a mai masurat înca o mie de coji ?i era un râu pe care nu-l puteam trece, caci apele crescusera; erau ape de trecut înotând, un fluviu care nu se putea trece.
Eze 47:6  Atunci mi-a zis barbatul acela: "Ai vazut, fiul omului?" ?i el m-a dus înapoi la malul râului.
Eze 47:7  ?i cârd am venit înapoi, iata pe malurile râului erau mul?i arbori pe o parte ?i pe alta.
Eze 47:8  ?i mi-a zis acela: "Aceasta apa curge în partea de rasarit a ?arii, se va coborî în ?es ?i va intra în mare, ?i apele ei se vor face sanatoase.
Eze 47:9  Toata vietatea care mi?una acolo pe unde va trece râul, va trai. Pe?te va fi foarte mult, pentru ca va intra acolo apa aceasta ?i apele din mare se vor face sanatoase; unde va intra râul acesta, toate vor trai acolo.
Eze 47:10  Lânga el vor sta pescarii de la En-Gaddi pâna la En-Eglaim ?i î?i vor arunca mrejele. Vor fi pe?ti de tot soiul, ca cei din Marea cea Mare (Mediterana) ?i foarte numero?i.
Eze 47:11  Mla?tinile lui ?i lacurile lui, care nu se vor face sanatoase, vor ramâne pentru sare.
Eze 47:12  La râu, pe amândoua laturile lui, vor cre?te tot felul de arbori care dau hrana. Frunzele lor nu se vor ve?teji ?i fructele din ei nu se vor mai ispravi. n fiecare luna se vor coace fructe noi, pentru ca apa pentru ele vine din locul cel sfânt; fructele lor se vor întrebuin?a ca hrana, iar frunzele la leacuri".
Eze 47:13  A?a zice Domnul Dumnezeu: "Iata hotarele pamântului pe care-l ve?i împar?i ca mo?tenire celor douasprezece semin?ii ale lui Israel: Iosif va avea doua par?i.
Eze 47:14  ?i ve?i stapâni din el to?i deopotriva, pentru ca, ridicându-Mi mâna, M-am jurat sa-l dau parin?ilor vo?tri, de aceea va fi pamântul acesta mo?tenirea voastra.
Eze 47:15  Iata care sunt hotarele ?arii acesteia: În partea de miazanoapte de la Marea cea Mare, drumul Hetlonului pâna la intrarea Hamatului: ?edad,
Eze 47:16  Berot, Sibraim care e între hotarul Damascului ?i al Hamatului, Ha?er-Ha?icon, spre hotarul Hauranului.
Eze 47:17  ?i va fi hotarul de la mare pâna la Ha?ar-Enon în hotarul Damascului, având la miazanoapte ?inutul Hamat. Aceasta este latura de miazanoapte.
Eze 47:18  Hotarul de rasarit sa-l trage?i printre Hauran ?i Damasc, printre Galaad ?i ?ara lui Israel, pe Iordan, de la hotarul de miazanoapte pâna la marea de rasarit spre Tamar. Aceasta este latura de rasarit.
Eze 47:19  Iar latura de miazazi începe spre miazazi de la Tamar ?i merge pâna la apele Meriba la Cade?, apoi urmeaza cursul râului pâna la Marea cea Mare. Aceasta este latura de miazazi.
Eze 47:20  Iar hotarul de la apus este Marea cea Mare, de la hotarul de miazazi pâna în dreptul Hamatului, Aceasta este latura de la apus.
Eze 47:21  ?i sa va împar?i?i între voi ?ara aceasta, dupa semin?iile lui Israel.
Eze 47:22  ?i s-o împar?i?i prin sor?i ca mo?tenire a voastra ?i a strainilor care vor trai la voi, care au nascut copii între voi; ?i ei trebuie sa se socoteasca între fiii lui Israel deopotriva cu locuitorii ba?tina?i. Cu voi sa primeasca ?i ei la sor?i mo?tenirea lor intre semin?iile lui Israel.
Eze 47:23  În care semin?ie va trai strainul, în aceea i se va da ?i mo?tenire", zice Domnul Dumnezeu.
Eze 48:1  Iata numele semin?iilor: la capatul de miazanoapte al ?arii, pe calea care duce de la Hetlon la Hamat, Ha?ar-Enon la hotarul Damascului, pe hotarul de miazanoapte, care este spre Hamat, pe toata întinderea aceasta de la rasarit pâna la mare este o singura parte, a lui Dan.
Eze 48:2  Lânga hotarul lui Dan, de la hotarul de rasarit pâna la cel de apus, este partea lui A?er.
Eze 48:3  Lânga hotarul lui A?er, de la hotarul de rasarit pâna la cel de apus, este partea lui Neftali.
Eze 48:4  Lânga hotarul lui Neftali, de la hotarul de rasarit pâna la cel de apus, este partea lui Manase.
Eze 48:5  Lânga hotarul lui Manase, de la hotarul de rasarit pâna la cel de la apus, este partea lui Efraim.
Eze 48:6  Lânga hotarul lui Efraim, de la hotarul de rasarit pâna la cel de apus, este partea lui Ruben.
Eze 48:7  Lânga hotarul lui Ruben, de la hotarul de rasarit pâna la cel de apus, este partea lui Iuda.
Eze 48:8  Iar lânga hotarul lui Iuda, de la hotarul de rasarit pâna la cel de apus, este partea sfânta, în la?ime de douazeci ?i cinci de mii de co?i, iar în lungime deopotriva cu celelalte par?i, de la marginea de rasarit ?a ?arii pâna la cea de apus. În mijlocul acestei par?i va fi templul.
Eze 48:9  Partea pe care o ve?i osebi Domnului va fi lunga de douazeci ?i cinci de mii de co?i, iar la?imea de zece mii de co?i.
Eze 48:10  ?i aceasta parte sfânta trebuie sa fie a preo?ilor: la miazanoapte douazeci ?i cinci de mii de co?i, la apus în la?ime zece mii de co?i, spre rasarit în la?ime zece mii ?i spre miazazi în lungime douazeci ?i cinci de mii, iar la mijloc va fi loca?ul sfânt al Domnului.
Eze 48:11  Aceasta sa o afierosi?i preo?ilor din fiii lui Sadoc, care au stat în slujba Mea ?i care, în vremea ratacirii fiilor lui Israel, nu s-au abatut de la Mine, cum s-au abatut ceilal?i levi?i.
Eze 48:12  Aceasta parte sfânta de pamânt, va fi a lor, sfin?enie mare, la hotarul levi?ilor.
Eze 48:13  Levi?ii vor primi lânga partea preo?ilor o parte de douazeci ?i cinci de mii de co?i în lungime ?i zece mii de co?i în la?ime: toata partea douazeci ?i cinci de mii în lungime ?i zece mii în la?ime.
Eze 48:14  ?i din aceasta parte ei nu pot nici sa vânda, nici sa schimbe; pârga pamântului nu poate fi trecuta altora, pentru ca ea este lucru sfânt, al Domnului.
Eze 48:15  Iar urmatoarele cinci mii în la?ime ?i douazeci ?i cinci de mii în lungime se dau pentru cetate, pentru întrebuin?area ob?teasca, pentru locuin?e ?i pentru pie?e; cetatea va fi în mijloc.
Eze 48:16  Iata masurile ei: latura de miazanoapte patru mii cinci sute de co?i, latura de miazazi patru mii cinci sute de co?i, latura de rasarit patru mii cinci sute de co?i ?i latura de la apus de patru mii cinci sute de co?i.
Eze 48:17  Iar pa?unea împrejurul ceta?ii va fi doua sute cincizeci de co?i la miaza-noapte, la rasarit doua sute cincizeci, la miazazi doua sute cincizeci ?i la apus doua sute cincizeci.
Eze 48:18  Iar ce ramâne din lungime în linie cu por?iunea sfânta - zece mii spre rasarit ?i zece mii spre apus - va forma un venit pentru hrana muncitorilor din cetate.
Eze 48:19  În cetate pot sa lucreze lucratori din toate semin?iile lui Israel.
Eze 48:20  Toata partea aceasta, un patrat de douazeci ?i cinci de mii de co?i în lungime ?i douazeci ?i cinci de mii de co?i în la?ime, va fi partea sfânta ?i va cuprinde ?i pamântul ceta?ii.
Eze 48:21  Ce va ramâne va fi al regelui de o parte ?i de alta a por?iunii sfinte ?i a por?iunii ceta?ii, ce va trece peste cele douazeci ?i cinci de mii de coli spre rasarit pâna la hotarul ?arii ?i ce va trece peste cele douazeci ?i cinci de mii de co?i spre apus pâna la hotarul ?arii. Aceasta va fi partea regelui, a?a încât partea sfânta ?i loca?ul sfânt vor fi la mijloc.
Eze 48:22  ?i ceea ce ramâne de la stapânirea levi?ilor ?i de la stapânirea ora?ului, la mijloc, între hotarul lui Iuda ?i hotarul lui Veniamin, va fi al regelui.
Eze 48:23  Iata acum celelalte semin?ii: alaturi de partea levi?ilor, de la hotarul de rasarit al ?arii pâna la cel de apus, este partea lui Veniamin.
Eze 48:24  Lânga hotarul de miazazi al lui Veniamin, de la hotarul de rasarit al ?arii pâna la cel de apus, este partea lui Simeon.
Eze 48:25  Lânga hotarul de miazazi al lui Simeon, de la hotarul de rasarit al ?arii pâna la cel de apus, este partea lui Isahar.
Eze 48:26  Lânga hotarul de miazazi al lui Isahar, de la hotarul de rasarit al ?arii pâna la cel de apus, este partea lui Zabulon.
Eze 48:27  Lânga hotarul de miazazi al lui Zabulon, de la hotarul de rasarit al ?arii pâna la cel de apus, este partea lui Gad.
Eze 48:28  Iar pe hotarul de miazazi al lui Gad merge ?i hotarul de miazazi al jarii de la Tamar spre apele Meriba de la Cade?, de-a lungul râului acestuia pâna la Marea cea Mare.
Eze 48:29  Aceasta este ?ara pe care o ve?i împar?i prin sor?i semin?iilor lui Israel, ?i acestea sunt par?ile semin?iilor, zice Domnul Dumnezeu.
Eze 48:30  Iata acum care sunt por?ile ceta?ii. Por?ile ceta?ii vor purta numele semin?iilor lui Israel.
Eze 48:31  Latura de miazanoapte va avea patru mii cinci sute de coli ?i va avea trei por?i: una, poarta lui Ruben, una, poarta lui Iuda ?i una, poarta lui Levi.
Eze 48:32  Latura de rasarit va avea patru mii cinci sute de co?i ?i trei por?i: una, poarta lui Iosif, una, poarta lui Veniamin ?i una, poarta lui Dan.
Eze 48:33  Latura de miazazi va avea patru mii cinci sute de co?i ?i trei por?i: una, poarta lui Simeon, una, poarta lui Isahar ?i una, poarta lui Zabulon.
Eze 48:34  ?i latura dinspre mare va avea patru mii cinci sute de coli ?i trei por?i: una, poarta lui Gad, una, poarta lui A?er ?i una, poarta lui Neftali.
Eze 48:35  De jur împrejurul ceta?ii sunt optsprezece mii de co?i. Iar din ziua aceea înainte numele ceta?ii va fi: Domnul este acolo.


\end{document}