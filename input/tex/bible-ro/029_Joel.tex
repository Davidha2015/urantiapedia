\begin{document}

\title{Ioel}


\chapter{1}

\par 1 Cuvântul Domnului care a fost catre Ioil, fiul lui Petuel.
\par 2 Asculta?i acestea, voi batrânilor, lua?i aminte, voi to?i locuitorii ?arii! Oare s-a mai întâmplat a?a în vremea voastra sau în zilele parin?ilor vo?tri?
\par 3 Povesti?i-le feciorilor vo?tri, iar ei feciorilor lor, iar aceia neamului care va veni dupa ei!
\par 4 Ceea ce a ramas de la lacustele mici au mâncat lacustele mari, ?i ceea ce a ramas de la cele mari au mâncat cele zburatoare, ?i ceea ce a ramas de la cele zburatoare au prapadit stolurile de lacuste.
\par 5 De?tepta?i-va, be?ivilor, ?i plânge?i! ?i voi, bautorilor de vin, tângui?i-va pentru vinul cel nou, caci vi s-a luat de la gura.
\par 6 Caci un popor a navalit în ?ara mea, puternic ?i fara de numar; din?ii lui ca din?ii de leu, iar falcile, ca falcile de leoaica.
\par 7 Pustiit-a via mea, iar smochinul meu mi l-a facut buca?i; l-a jupuit de coaja ?i l-a trântit la pamânt. Mladi?ele de vita au ramas albe.
\par 8 Tânguie?te-te ca o fecioara încinsa cu sac dupa barbatul din tinere?ea ei!
\par 9 Prinosul ?i jertfa cu turnare nu mai sunt în templul Domnului! Preo?ii, slujitorii Domnului, sunt în mare jale.
\par 10 Câmpul a ramas pustiu, ?arina este plina de jale; grâul nu mai este, mustul ajuns-a de ocara, iar untdelemn n-a ramas chiar nici un strop!
\par 11 Plugarii sunt zapaci?i, stapânii de vii se tânguiesc pentru grâu ?i pentru orz, caci seceri?ul din ?arini este pierdut.
\par 12 Vi?a de vie este fara vlaga, smochinul tânje?te, rodiile de asemenea ?i finicii, merii ?i to?i copacii de pe câmp s-au uscat; ba mai mult, bucuria ajuns-a ocara pentru fiii oamenilor.
\par 13 Încinge?i-va ?i va tângui?i, voi preo?ilor, izbucni?i în bocete, voi slujitori ai altarului! Intra?i în templu ?i petrece?i noaptea în sac de jale, voi slujitori ai lui Dumnezeu, caci prinosul ?i jertfa cu turnare au fost îndepartate din templul Dumnezeului vostru!
\par 14 Posti?i post sfânt, strânge?i ob?te de praznuire, aduna?i pe batrâni, pe to?i locuitorii ?arii în templul Dumnezeului vostru ?i striga?i catre Domnul rugându-L:
\par 15 "O, ce zi! Caci aproape este ziua Domnului ?i vine ca o pustiire de la Cel Atotputernic.
\par 16 Oare nu ni s-a luat hrana de sub ochii no?tri, bucuria ?i veselia din templul Dumnezeului nostru?
\par 17 Semin?ele s-au uscat sub bulgarii din brazda, grânarele sunt goale, hambarele sunt fara nimic în ele, caci nu mai este grâu.
\par 18 Cum gem vitele! Cirezile de boi mugesc, caci nu afla nicaieri pa?une; chiar ?i turmele de oi sunt în mare lipsa.
\par 19 Catre Tine, Doamne, strig! Caci focul a mistuit toate pa?unile pustiului ?i vapaia lui a dogorit to?i copacii de pe câmp.
\par 20 ?i fiarele câmpului zbiara catre tine, caci pâraiele de apa au secat ?i focul a mistuit pa?unile stepei".

\chapter{2}

\par 1 Suna?i din trâmbi?a în Sion, striga?i din rasputeri în muntele cel sfânt al Meu, ca sa se cutremure to?i locuitorii ?arii! Caci vine ziua Domnului; iata ea este aproape;
\par 2 O zi de întuneric ?i de bezna, zi cu nori ?i cu negura deasa. Precum zorile se revarsa pe deasupra mun?ilor, tot a?a da navala un popor numeros ?i puternic, cum n-a mai fost niciodata ?i cum nu va mai fi dupa el pâna în anii vremurilor celor mai îndepartate.
\par 3 Înaintea lui merge mistuind focul, iar dupa el arde vapaia. Pamântul este înaintea lui ca gradina raiului, iar dupa trecerea lui, pustiu înfrico?ator, caci nimic nu scapa din fa?a lui.
\par 4 Cum sunt caii, a?a le este chipul lor; ?i ca sprinteni calare?i, a?a alearga.
\par 5 Se aud ca un duruit de care de razboi, care se avânta în goana pe cre?tetul mun?ilor, ca pârâitul flacarilor care mistuie o miri?te, ca o o?tire puternica a?ezata în linie de bataie.
\par 6 Înaintea lui popoarele tremura de spaima, toate fe?ele ard ca vapaia.
\par 7 Alearga ca ni?te viteji; ca razboinicii încerca?i se avânta peste ziduri; om dupa om î?i urmeaza calea, fara ca vreunul sa se rataceasca.
\par 8 Nimeni nu se îmbrânce?te cu cel de alaturi, fiecare î?i vede de drumul lui; printre sage?i î?i croiesc cale, fara ca nici unul sa rupa rândul.
\par 9 Dau navala în cetate, alearga pe deasupra zidurilor, patrund în case ?i intra pe ferestre ca furii.
\par 10 Înaintea lor tremura pamântul, cerul se cutremura, soarele ?i luna se întuneca, iar stelele î?i pierd stralucirea lor.
\par 11 Dar Domnul Î?i face auzit glasul în fruntea o?tirilor Sale, caci întinsa foarte este tabara Lui ?i puternic este cel ce împline?te poruncile Lui. Ziua Domnului este mare ?i înfrico?atoare foarte ?i cine va putea sta împotriva ei?
\par 12 "?i acum, zice Domnul, întoarce?i-va la Mine din toata inima voastra, cu postiri, cu plâns ?i cu tânguire".
\par 13 Sfâ?ia?i inimile ?i nu hainele voastre, ?i întoarce?i-va catre Domnul Dumnezeul vostru, caci El este milostiv ?i îndurat, încet la mânie ?i mult Milostiv ?i-I pare rau de raul pe care l-a trimis asupra voastra.
\par 14 O, de v-a?i întoarce ?i v-a?i pocai, ar ramâne de pe urma voastra o binecuvântare, un prinos ?i o jertfa cu turnare pentru Domnul Dumnezeul vostru!
\par 15 Suna?i din trâmbi?a în Sion, gati?i postiri sfinte, praznui?i sarbatoarea cea pentru to?i!
\par 16 Aduna?i poporul, vesti?i o adunare sfânta, strânge?i laolalta pe batrâni, aduce?i copiii ?i pruncii care sug la sân; sa iasa mirele din camara lui ?i mireasa din iatacul ei!
\par 17 Între tinda ?i altar sa plânga preo?ii, slujitorii Domnului, ?i sa zica: "Milostive?te-Te, Doamne, catre poporul Tau ?i nu face de ocara mo?tenirea Ta ca sa-?i bata joc de ea neamurile!" Pentru ce sa se spuna printre neamuri: "Unde este Dumnezeul lor?"
\par 18 Dar Domnul este plin de zel pentru ?ara Sa ?i Se îndura de poporul Sau.
\par 19 Pentru aceasta a raspuns Domnul catre poporul Sau, graind: "Iata, Eu va voi trimite grâu, must ?i untdelemn ?i va voi satura ?i nu va voi mai face de ocara printre neamuri!
\par 20 Prapadul du?manului de la miazanoapte îl voi departa de la voi ?i-l voi izgoni înspre un ?inut uscat ?i pustiu; capatul lui spre marea cea de la rasarit, iar sfâr?itul la marea cea dinspre apus; ?i se va ridica din el duhoare ?i miros de stârv va porni din el, caci a savâr?it lucruri mari".
\par 21 Nu te teme, tu ?ara, bucura-te ?i te vesele?te, caci Domnul a facut lucruri mari!
\par 22 Nu va teme?i nici voi dobitoacele câmpului, caci au înverzit pa?unile pustiului; caci pomii dau roadele lor; smochinul ?i vi?a de vie vor fi plini de roade.
\par 23 ?i voi locuitori ai Sionului, bucura?i-va ?i va veseli?i în Domnul Dumnezeul vostru, caci El v-a dat pe Înva?atorul drepta?ii; ?i v-a mai trimis ?i ploaie, ploaie timpurie ?i târzie, ca odinioara.
\par 24 ?i ariile se vor umple de grâu; iar teascurile vor da peste margini de must ?i de untdelemn.
\par 25 ?i va voi da ani de bel?ug în locul anilor în care au mâncat lacustele mici, cele mari, cele zburatoare ?i stolurile de lacuste, marea Mea o?tire pe care am trimis-o împotriva voastra.
\par 26 ?i ve?i mânca din destul ?i va ve?i satura ?i ve?i preaslavi numele Domnului Dumnezeului vostru, Care a facut cu voi lucruri minunate. ?i poporul Meu nu se va ru?ina în veci de veci!
\par 27 Atunci va ve?i da seama ca Eu sunt în mijlocul lui Israel ?i ca Eu sunt Domnul Dumnezeul vostru ?i nu este altul, iar poporul Meu nu va mai fi niciodata de ocara!
\par 28 Dar dupa aceea, varsa-voi Duhul Meu peste tot trupul, ?i fiii ?i fiicele voastre vor profe?i, batrânii vo?tri visuri vor visa iar tinerii vo?tri vedenii vor vedea.
\par 29 Chiar ?i peste robi ?i peste roabe voi varsa Duhul Meu.
\par 30 ?i va voi arata semne minunate în cer ?i pe pamânt: sânge, foc ?i stâlpi de fum;
\par 31 Soarele se va întuneca ?i luna va fi ro?ie ca sângele, înainte de venirea zilei celei mari ?i înfrico?atoare a Domnului.
\par 32 ?i oricine va chema numele Domnului se va izbavi, caci în muntele Sionului ?i în Ierusalim va fi mântuirea, precum a zis Domnul; ?i între cei mântui?i, numai cei ce cheama pe Domnul.

\chapter{3}

\par 1 Caci iata în zilele acelea ?i în vremea aceea, când voi întoarce pe Iuda ?i pe Ierusalim din robie,
\par 2 Aduna-voi toate popoarele ?i le voi coborî în valea lui Iosafat ?i Ma voi judeca acolo cu ele pentru poporul Meu ?i pentru mo?tenirea Mea Israel, pe care au împra?tiat-o între neamuri ?i ?ara Mea au împar?it-o în buca?i.
\par 3 Caci ei au aruncat sor?i asupra poporului Meu, au dat copilul pentru o desfrânata ?i fata tânara au vândut-o pentru vin ?i au baut pre?ul ei.
\par 4 Ce-Mi cauta?i cearta voi, Tirule ?i Sidonule, împreuna cu toate ?inuturile Filisteii? Oare vre?i sa va razbuna?i împotriva Mea? Daca vre?i sa va razbuna?i asupra Mea, îndata voi face sa cada razbunarea asupra capetelor voastre.
\par 5 Voi a?i luat aurul ?i argintul Meu, precum ?i odoarele Mele, ?i le-a?i dus în palatele voastre.
\par 6 Pe fiii lui Iuda ?i pe cei ai Ierusalimului i-a?i vândut Grecilor, ca sa-i departa?i din ?ara lor.
\par 7 Iata Eu îi voi stârpi din ?inutul unde i-a?i vândut ?i voi întoarce fapta voastra asupra capului vostru.
\par 8 ?i voi vinde pe fiii ?i pe fiicele voastre feciorilor lui Iuda, iar ei îi vor vinde Sabeenilor, popor tare departe, caci a?a a grait Domnul.
\par 9 Da?i neamurilor de ?tire lucrul acesta: Pregati?i-va de razboi! Înflacara?i vitejii! To?i barba?ii buni de lupta sa dea fuga ?i sa se apropie.
\par 10 Face?i din brazdarele voastre sabii ?i din secerile voastre lanci! Cel slab sa zica: "Eu sunt viteaz!"
\par 11 Alerga?i în graba ?i veni?i, voi, toate popoarele din jur, ?i aduna?i-va aici! - Acolo, coboara, Doamne, pe vitejii Tai!
\par 12 Sa se trezeasca toate neamurile ?i sa vina în valea lui Iosafat, caci acolo voi a?eza scaun de judecata pentru toate popoarele din jur.
\par 13 Aduce?i seceri, caci holda este coapta, veni?i, coborâ?i-va, caci teascul este plin, albiile dau peste margini; faradelegile lor n-au seaman.
\par 14 Mul?imi ?i iar mul?imi în Valea Judeca?ii. Caci aproape este ziua Domnului în Valea judeca?ii!
\par 15 Soarele ?i luna se întuneca ?i stelele î?i pierd lumina lor.
\par 16 Din Sion Domnul va striga puternic ?i din Ierusalim va slobozi tunetul Sau: pamântul ?i cerurile se vor cutremura atunci.
\par 17 Atunci ve?i cunoa?te ca Eu sunt Domnul Dumnezeul vostru, Care sala?luie?te în Sion, în muntele cel sfânt al Meu; Ierusalimul va fi altarul Meu, iar strainii nu vor mai trece pe acolo.
\par 18 ?i în vremea aceea, din mun?i va curge must, vaile vor fi pline de lapte, toate pâraiele din Iuda vor ?erpui umplute de apa, iar din templul Domnului va ie?i un izvor oare va uda valea ?itim.
\par 19 Egiptul va fi pustiit ?i Edomul se va preface în ?inut nelocuit, din pricina silniciilor împotriva fiilor lui Iuda, fiindca au varsat sânge nevinovat în pamântul lor.
\par 20 Dar Iuda va fi locuit în veci ?i de-a pururi ?i Ierusalimul din neam în neam.
\par 21 Voi razbuna sângele pe care nu l-am razbunat; ?i Domnul va locui Sionul în veac.


\end{document}