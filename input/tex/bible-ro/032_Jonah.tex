\begin{document}

\title{Jonah}

Jon 1:1  ?i a fost cuvântul Domnului catre Iona, fiul lui Amitai, zicând:
Jon 1:2  "Scoala-te ?i du-te în cetatea cea mare a Ninivei ?i propovaduie?te acolo, caci faradelegile lor au ajuns pâna în fa?a Mea!"
Jon 1:3  ?i s-a sculat Iona sa fuga la Tarsis, departe de Domnul. ?i s-a coborât la Iope, unde a gasit o corabie, care mergea la Tarsis, ?i, platind pre?ul calatoriei, s-a coborât în ea ca sa mearga la Tarsis împreuna cu to?i cei de acolo, el fugind din fa?a Domnului.
Jon 1:4  Dar Domnul a ridicat un vânt napraznic pe mare ?i o furtuna puternica s-a stârnit, încât corabia era gata sa se sfarâme.
Jon 1:5  Corabierii s-au înfrico?at ?i au strigat fiecare catre dumnezeul sau ?i au aruncat în mare încarcatura corabiei ca sa se u?ureze. Dar Iona se coborâse în fundul corabiei, se culcase ?i adormise.
Jon 1:6  Atunci s-a apropiat de el cârmaciul corabiei, ?i i-a zis: "Pentru ce dormi? Scoala-te ?i striga catre Dumnezeul tau, poate El Î?i va aduce aminte de noi, ca sa nu pierim!"
Jon 1:7  ?i au zis unul catre altul: "Haidem sa aruncam sor?i, ca sa ?tim din pricina cui a venit peste noi nenorocirea aceasta!" ?i au aruncat sor?i, ?i sor?ul a cazut pe Iona.
Jon 1:8  ?i l-au întrebat pe el: "Spune-ne noua din pricina cui s-a abatut nenorocirea aceasta asupra noastra? Care este me?te?ugul tau, de unde ?i din ce ?ara vii ?i din ce popor e?ti?"
Jon 1:9  Atunci el le-a raspuns: "Sunt evreu ?i Domnului Dumnezeului cerului ma închin - Cel care a facut marea ?i uscatul".
Jon 1:10  ?i to?i oamenii s-au temut cu frica mare ?i i-au zis lui: "Pentru ce ai savâr?it una ca aceasta?" Caci ei ?tiau ca el fuge din fa?a lui Dumnezeu, fiindca el le spusese.
Jon 1:11  ?i i-au zis lui: "Ce sa-?i facem ca sa se potoleasca marea?" Caci marea se ridica din ce în ce mai mult.
Jon 1:12  Atunci el a raspuns: "Lua?i-ma ?i ma arunca?i în mare ?i ea se va potoli, caci ?tiu bine ca din pricina mea s-a pornit peste voi aceasta vijelie".
Jon 1:13  ?i marinarii vâsleau ca sa ajunga la ?arm, dar în zadar, caci marea se ridica din ce în ce mai mult împotriva lor.
Jon 1:14  Atunci au strigat catre Domnul ?i au zis: "O, Doamne, de-am putea sa nu pierim din pricina vie?ii acestui om ?i sa nu ne împovarezi pe noi cu un sânge nevinovat! Ca Tu, Doamne, precum ai voit ai facut!"
Jon 1:15  ?i îl ridicara pe Iona ?i îl aruncara în mare ?i s-a potolit urgia ei.
Jon 1:16  ?i oamenii s-au temut cu teama mare de Domnul ?i au adus jertfa lui Dumnezeu ?i I-au facut Lui fagaduin?e.
Jon 1:17  ?i Dumnezeu a dat porunca unui pe?te mare sa înghita pe Iona. ?i a stat Iona în pântecele pe?telui trei zile ?i trei nop?i.
Jon 2:1  Atunci s-a rugat Iona din pântecele pe?telui catre Domnul Dumnezeul lui, zicând:
Jon 2:2  "Strigat-am catre Domnul în strâmtorarea mea, ?i El m-a auzit; din pântecele locuin?ei mor?ilor catre El am strigat, ?i El a luat aminte la glasul meu!
Jon 2:3  Tu m-ai aruncat în adânc, în sânul marii ?i undele m-au înconjurat; toate talazurile ?i valurile Tale au trecut peste mine.
Jon 2:4  ?i gândeam: Aruncat sunt dinaintea ochilor Tai! Dar voi vedea din nou templul cel sfânt al Tau!
Jon 2:5  Apele m-au învaluit pe de-a întregul, adâncul m-a împresurat, iarba marii s-a încolacit în jurul capului meu;
Jon 2:6  Ma coborâsem pâna la temeliile mun?ilor, zavoarele pamântului erau trase asupra mea pentru totdeauna, dar Tu ai scos din stricaciune via?a mea, Doamne Dumnezeul meu!
Jon 2:7  Când se sfâr?ea în mine duhul meu, de Domnul mi-am adus aminte, ?i la Tine a ajuns rugaciunea mea, în templul Tau cel sfânt!
Jon 2:8  Cei ce slujesc idolilor de?er?i dispre?uiesc harul Tau;
Jon 2:9  Dar eu Î?i voi aduce ?ie jertfe cu glas de lauda ?i toate fagaduin?ele mele le voi împlini, caci mântuirea vine de la Domnul!"
Jon 2:10  ?i Domnul a dat porunca pe?telui ?i pe?tele a aruncat pe Iona la ?arm!
Jon 3:1  ?i a fost cuvântul Domnului catre Iona, pentru a doua oara, zicând:
Jon 3:2  "Scoala ?i porne?te catre cetatea cea mare a Ninivei ?i veste?te-le ceea ce î?i voi spune!"
Jon 3:3  ?i s-a sculat Iona ?i a mers în Ninive, dupa cuvântul Domnului. ?i Ninive era cetate mare înaintea lui Dumnezeu; î?i trebuia trei zile ca s-o straba?i.
Jon 3:4  ?i a patruns Iona în cetate, zicând: "Patruzeci de zile mai sunt, ?i Ninive va fi distrusa!"
Jon 3:5  Atunci Ninivitenii au crezut în Dumnezeu, au ?inut post ?i s-au îmbracat cu sac, de la cei mai mari ?i pâna la cei mai mici.
Jon 3:6  ?i a ajuns vestea pâna la regele Ninivei. Acesta s-a sculat de pe tronul sau, ?i-a lepadat ve?mântul lui cel scump, s-a acoperit cu sac ?i s-a culcat în cenu?a.
Jon 3:7  Apoi, din porunca regelui ?i a dregatorilor sai, s-au strigat ?i s-au zis acestea: Oamenii ?i animalele, vitele mari ?i mici sa nu manânce nimic, sa nu pasca ?i nici sa bea apa;
Jon 3:8  Iar oamenii sa se îmbrace cu sac ?i catre Dumnezeu sa strige din toata puterea ?i fiecare sa se întoarca de pe calea lui cea rea ?i de la nedreptatea pe care o savâr?esc mâinile lui;
Jon 3:9  Poate ca Dumnezeu Se va întoarce ?i Se va milostivi ?i va ?ine în loc iu?imea mâniei Lui ca sa nu pierim!"
Jon 3:10  Atunci Dumnezeu a vazut faptele lor cele de pocain?a, ca s-au întors din caile lor cele rele. ?i i-a parut rau Domnului de prezicerile de rau pe care li le facuse ?i nu le-a împlinit.
Jon 4:1  ?i Iona a fost cuprins de mare suparare ?i s-a aprins de mânie.
Jon 4:2  ?i a rugat pe Domnul, zicând: "O, Doamne, iata tocmai ceea ce cugetam eu când eram în ?ara mea! Pentru aceasta eu am încercat sa fug în Tarsis, ca ?tiam ca Tu e?ti Dumnezeu îndurat ?i milostiv, îndelung-rabdator ?i mult-milosârd ?i Î?i pare rau de faradelegi.
Jon 4:3  ?i acum, Doamne, ia-mi sufletul meu, caci este mai bine sa mor decât sa fiu viu!"
Jon 4:4  ?i a zis Domnul: "Faci tu oare bine ca ?i-ai aprins mânia?"
Jon 4:5  Atunci Iona a ie?it din cetate ?i s-a a?ezat la rasaritul ei, ?i-a facut o coliba ?i a stat sub ea la umbra, ca sa vada ce se va întâmpla cu cetatea.
Jon 4:6  ?i Domnul Dumnezeu a facut sa creasca un vrej care s-a ridicat deasupra capului lui Iona, ca sa-i ?ina umbra ?i sa-i mai potoleasca mânia. ?i s-a bucurat Iona cu bucurie mare pentru vrej.
Jon 4:7  Dar Dumnezeu, a doua zi, la revarsatul zorilor, a poruncit unui vierme sa reteze vrejul. Iar el s-a uscat.
Jon 4:8  ?i la rasaritul soarelui a pornit Dumnezeu un vânt arzator de la rasarit ?i soarele a dogorit capul lui Iona, încât el se prapadea de caldura. ?i ?i-a rugat moartea zicând: "Mai bine este sa mor decât sa traiesc!"
Jon 4:9  ?i a grait Domnul catre Iona: "Ai tu dreptate sa te mânii pentru vrej?" ?i el a raspuns: "Da, am dreptate sa fiu suparat de moarte!"
Jon 4:10  ?i a zis Domnul: "Tu ?i-ai facut necaz pentru acest vrej pentru care nu te-ai trudit ?i nici nu l-ai crescut, care ?i-a luat fiin?a într-o noapte ?i într-alta a pierit!
Jon 4:11  Dar Mie cum sa nu-Mi fie mila de cetatea cea mare a Ninivei cu mai mult de o suta douazeci de mii de oameni, care nu ?tiu sa deosebeasca dreapta de stânga lor, ?i cu un mare numar de dobitoace?"


\end{document}