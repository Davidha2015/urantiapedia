\begin{document}

\title{Micah}

Mic 1:1  Cuvântul Domnului care a fast catre Miheia din More?et în vremea domniei lui Iotam, Ahaz ?i Iezechia, regii lui Iuda, când a avut vedenie pentru Samaria ?i Ierusalim.
Mic 1:2  Asculta?i, voi neamuri toate, ia aminte pamântule ?i tot ce se afla pe tine! Caci Domnul Dumnezeu va fi voua marturie împotriva voastra, Domnul din templul cel sfânt al Sau!
Mic 1:3  Caci iata Domnul va ie?i din loca?ul Lui, Se va pogorî ?i va pa?i peste înal?imile pamântului;
Mic 1:4  Mun?ii se vor topi sub El ca ceara în fa?a focului ?i vaile vor curge, cum se pornesc la vale apele pe povârni?.
Mic 1:5  Toate acestea vor fi din pricina faradelegii lui Iacov ?i a pacatului casei lui Israel. Care este faradelegea lui Iacov? Oare nu Samaria? ?i care sunt înal?imile lui Iuda? Oare nu Ierusalimul?
Mic 1:6  Pentru aceasta voi preface Samaria într-o gramada de pietre adunate de pe câmp, când se sade?te via; voi pravali în vale pietrele ei ?i temeliile ei le voi dezveli.
Mic 1:7  Toate chipurile ei cele cioplite vor fi doborâte, toata boga?ia adunata din pre?ul desfrânarii va fi arsa în foc ?i toate chipurile ei de idoli le voi face una cu pamântul; caci din plata de desfrânata i-a adunat ?i în plata de desfrânata se vor preface!
Mic 1:8  Din aceasta pricina Ma voi tângui ?i voi urla, voi merge descul? ?i gol; voi scoate urlete ca ?acalii ?i ?ipete ca stru?ii;
Mic 1:9  Caci rana ei este fara de leac, ajuns-a pâna în Iuda, atins-a por?ile poporului Meu, pâna la Ierusalim.
Mic 1:10  Nu da?i de veste în Gat, ?i în Aco, nu plânge?i! În Bet-Leafra tavali?i-va în pulbere!
Mic 1:11  Cornul suna pentru voi, locuitori din ?afir. ?i din cetatea lor nu ies cei din Taanan. Plângere se aude în Bet-Hae?el, caci ve?i fi lipsi?i de sprijin.
Mic 1:12  Cum sa nadajduiasca locuitorii cei din Marot ca le va fi bine, când nenorocirea de la Domnul s-a coborât la por?ile Ierusalimului?
Mic 1:13  Înhama?i caii la caru?a, locuitori ai Lachi?ului! Iata începutul ispa?irii pacatului fiicei Sionului, caci în tine s-au aflat faradelegile lui Israel.
Mic 1:14  De aceea tu vei renun?a sa stapâne?ti More?et-Gat; casele Aczibului vor fi o dezamagire pentru regii Israelului!
Mic 1:15  Î?i voi aduce un nou cuceritor ?ie care locuie?ti în Mare?a; stralucirea Israelului va merge pâna la Odolam!
Mic 1:16  Fii ple?uva ?i te rade pentru copiii tai cei dragi, mare?te-?i ple?uvia ca a vulturului, caci ei au fost du?i în robie, departe de tine!
Mic 2:1  Vai de cei ce cugeta gânduri silnice stând în a?ternuturile lor ?i de cei ce savâr?esc faradelegea la lumina zilei de îndata ce afla prilej.
Mic 2:2  Le plac ?arinile, pentru aceea le ?i rapesc; casele, de aceea le ?i iau. Calca silnic drepturile stapânului ?i ale casei lui, ale omului ?i ale mo?tenirii lui.
Mic 2:3  Pentru aceasta a?a zice Domnul  Iata ca Eu pun la cale pentru neamul acesta o nenorocire, de care nu ve?i putea sa va feri?i grumazul vostru ?i nu ve?i mai calca mândri; caci vremurile sunt rele.
Mic 2:4  În vremea aceea ve?i ajunge de batjocura ?i vi se va cânta un cântec de jale: "Domnul a spus, ?i iata ca au fost nimici?i! El da în alte mâini partea de mo?tenire a poporului Sau! Cum ia El fara sa mai dea înapoi ?i ogoarele noastre le împarte vrajma?ului!
Mic 2:5  Pentru aceasta voi nu ve?i avea pe nimeni care sa arunce funia pentru sor?ii vo?tri în ob?tea Domnului!"
Mic 2:6  Nu zice?i nimic împotriva! Nu va ridica?i împotriva, caci aceasta nu poate sa departeze ru?inea voastra!
Mic 2:7  Casa lui Iacov zice: "Oare Dumnezeu a pierdut rabdarea? Acestea sunt lucrurile Sale?" Cuvintele Mele nu sunt ele, oare, binevoitoare pentru cel ce umbla pe calea cea dreapta?
Mic 2:8  Ci voi, împotriva poporului Meu, sta?i într-ajutor vrajma?ului lui. Mai înainte ca sa navaleasca Salmanasar voi jefui?i ca la razboi pe cei care merg cu încredere pe calea lor;
Mic 2:9  Voi da?i afara pe femeile poporului Meu din casele lor placute, iar feciorilor lor le lua?i pentru totdeauna slava Mea!
Mic 2:10  Scula?i-va ?i pleca?i, caci aici nu este loc de odihna; din pricina întinaciunii lui, va avea parte de chinuri grozave.
Mic 2:11  O! Daca ar putea cineva sa fie purtat de duhul nascocirilor ?i sa spuna cuvinte mincinoase, zicând: "Am sa-?i proorocesc despre vin ?i despre bautura cea îmbatatoare", acesta ar fi proorocul poporului acestuia.
Mic 2:12  Te voi aduna pe tine, Iacove tot, voi strânge laolalta ceea ce a mai ramas din Israel! Îl voi aduna ca pe oile ce sunt în mare primejdie, ca pe o turma în mijlocul unei nenorociri. Spaimânta?i, ei vor fugi departe de locul prapadului!
Mic 2:13  Cel ce croie?te cale este în fruntea lor; ei î?i croiesc cale ?i trec ?i ies pe o poarta, iar regele merge înaintea lor Domnul este în fruntea lor.
Mic 3:1  ?i am zis: "Asculta?i acum, capetenii ale lui Iacov ?i judecatori ai casei lui Israel: Oare nu este datoria voastra sa cunoa?te?i dreptatea?
Mic 3:2  Ei urasc binele ?i iubesc raul, jupoaie pielea de pe oameni ?i carnea de pe oasele lor.
Mic 3:3  ?i, dupa ce vor fi mâncat carnea poporului meu, vor fi jupuit pielea ?i oasele lor le vor fi sfarâmat ?i prefacut în buca?i ca ?i carnea în oala, ca trupul lor în caldare,
Mic 3:4  Atunci vor striga catre Domnul, dar El nu le va raspunde ?i în vremea aceea Î?i va ascunde fa?a Sa de la ei, din pricina faptelor lor celor rele.
Mic 3:5  A?a zice Domnul împotriva proorocilor care ratacesc pe poporul meu, care atât cât au în ce sa-?i înfiga din?ii propovaduiesc pacea, dar împotriva celor care nu le arunca nimic în gura se pornesc cu razboi.
Mic 3:6  Pentru aceasta va fi pentru voi noapte în loc de vedenie ?i întuneric în loc de proorocie! Soarele va apune peste prooroci ?i ziua se va preface în întuneric!
Mic 3:7  Vazatorii se vor face de ru?ine ?i ghicitorii vor fi de ocara ?i to?i î?i vor acoperi barba, caci nu vor avea nici un raspuns de la Domnul.
Mic 3:8  Iar eu mul?umita Duhului lui Dumnezeu sunt plin de putere, de dreptate ?i de tarie, ca sa vadesc faradelegea lui Iacov ?i pacatul lui Israel.
Mic 3:9  Asculta?i acestea voi, capetenii ale casei lui Iacov, judecatori ai casei lui Israel, voi cei ce sim?i?i scârba pentru dreptate ?i strâmba?i ceea ce este drept,
Mic 3:10  Care zidi?i Sionul cu faradelegi ?i Ierusalimul cu strâmbata?i!
Mic 3:11  Capeteniile lor fac judecata pentru plocoane, preo?ii dau înva?atura legii pentru plata ?i profe?ii profe?esc pentru bani ?i se sprijina pe Domnul zicând: "Domnul este în mijlocul nostru ?i prapadul nu va veni peste noi!"
Mic 3:12  Deci, din pricina voastra, Sionul va fi arat cu plugul ca o ?arina ?i Ierusalimul va fi prefacut într-un morman de ruine ?i muntele templului va ajunge o înal?ime acoperita cu padure!"
Mic 4:1  ?i în zilele cele de apoi, muntele templului Domnului se va înal?a peste vârfurile mun?ilor ?i mai sus decât dealurile ?i catre el vor curge popoarele;
Mic 4:2  Popoare multe se vor îndrepta spre el zicând: "Veni?i sa ne suim în muntele Domnului, în templul Dumnezeului lui Iacov, ?i El ne va înva?a caile Sale, ?i sa mergem pe cararile Sale, ca din Sion va ie?i legea ?i cuvântul lui Dumnezeu din Ierusalim!"
Mic 4:3  El va fi judecator al multor popoare ?i dreptate va împar?i la neamuri puternice pâna departe. Acelea vor preface sabiile lor în fiare de plug ?i lancile lor în cosoare. ?i nici un neam nu va mai ridica sabia împotriva altuia ?i nu se vor mai înva?a cum sa se lupte;
Mic 4:4  Ci fiecare va sta lini?tit sub vita ?i smochinul lui ?i nimeni nu-i va înfrico?a, caci gura Domnului Savaot a grait!
Mic 4:5  Atunci toate popoarele vor umbla fiecare în numele dumnezeului sau, iar noi vom merge în numele Domnului Dumnezeului nostru de acum ?i pâna în veac!
Mic 4:6  În ziua aceea, zice Domnul, voi aduna laolalta pe cei care o luasera la fuga ?i pe cei chinui?i;
Mic 4:7  ?i din cei ?chiopi voi face un rest ?i din cei ce au fost împovara?i, un neam puternic, ?i Domnul va împara?i peste ei în muntele Sionului, de acum ?i pâna m veac!
Mic 4:8  Iar la tine, turnul de paza al turmei, colina fiicei Sionului, la tine se va întoarce stapânirea de odinioara, împara?ia fiicei Ierusalimului!
Mic 4:9  ?i acum pentru ce strigi a?a de tare? Oare î?i lipse?te rege ?i sfatuitorul tau a pierit ?i te-au cuprins durerile ca pe una care na?te?
Mic 4:10  Sufere dureri de na?tere ?i zvârcole?te-te ca o femeie în chinuri, fiica a Sionului, caci vei ie?i din cetate ?i te vei sala?lui în câmp ?i te vei duce în Babilon. Acolo tu vei fi izbavita ?i acolo te va rascumpara Domnul din mâna vrajma?ilor tai.
Mic 4:11  Acum s-au adunat la tine neamuri fara de numar, care zic: "Pângarita sa fie, ?i ochii no?tri sa priveasca nesa?io?i Sionul!"
Mic 4:12  Dar ele nu cunosc cugetele Domnului ?i nu pricep sfatul Lui, ca El le-a adunat ca snopii pe arie.
Mic 4:13  Scoala-te ?i calca în picioare pe fiica Sionului; ca voi face cornul tau de fier ?i copitele tale de arama! Tu vei zdrobi popoare multe ?i prada luata de la ele, Domnului o vei închina ?i boga?iile lor Stapânului a tot pamântul.
Mic 5:1  ?i acum închide-te cu zid, Bet-Gader! Suntem cuprin?i de toate par?ile! Ei lovesc cu toiagul peste obraz pe toate semin?iile lui Israel!
Mic 5:2  ?i tu, Betleeme Efrata, de?i e?ti mic între miile lui Iuda, din tine va ie?i Stapânitor peste Israel, iar obâr?ia Lui este dintru început, din zilele ve?niciei.
Mic 5:3  Pentru aceasta îi va lasa pâna în vremea când aceea ce trebuie sa nasca va na?te. Atunci rama?i?a fra?ilor sai se va întoarce la fiii lui Israel.
Mic 5:4  El va fi voinic ?i va pa?te poporul prin puterea Domnului, întru slava numelui Domnului Dumnezeului Sau ?i to?i vor fi fara de grija, iar El va fi mare, pâna la marginile pamântului.
Mic 5:5  ?i El Însu?i va fi pacea noastra! Când Asiria va navali în ?ara noastra ?i va patrunde calcând în palatele noastre, noi vom ridica împotriva lui ?apte pastori ?i opt capetenii,
Mic 5:6  Care vor pustii ?ara Asiriei cu sabia ?i ?ara lui Nimrod cu sabia scoasa din teaca. ?i ne va izbavi de Asiria, când aceasta va navali în ?ara noastra ?i când va trece peste hotarele noastre.
Mic 5:7  Iar restul lui Iacov va fi în mijlocul multor popoare ca roua de la Domnul ?i ca bura de ploaie pe iarba, care nu se bizuie pe nimeni ?i nu a?teapta ajutor de la fiii oamenilor.
Mic 5:8  ?i va mai fi restul lui Iacov, între popoare multe, ca un  leu între dobitoacele din padure ?i ca un pui de leu în turma de oi, care trece, calca în picioare ?i prada, fara ca nimeni sa-i poata smulge prada.
Mic 5:9  Mâna ta se va ridica împotriva vrajma?ilor tai ?i to?i asupritorii tai vor fi nimici?i!
Mic 5:10  ?i în ziua aceea, zice Domnul, voi nimici caii tai din mijlocul tau ?i toate carele tale le voi distruge.
Mic 5:11  ?i voi darâma ora?ele din ?ara ta ?i toate ceta?ile tale le voi face una cu pamântul.
Mic 5:12  ?i din mâna ta voi nimici pe vrajitori ?i tu nu vei mai avea ghicitori;
Mic 5:13  Voi darâma idolii tai ?i stâlpii idole?ti din mijlocul tau ?i nu te vei mai închina lucrurilor mâinilor tale;
Mic 5:14  Voi nimici dumbravile din mijlocul tau ?i voi taia copacii tai, afierosi?i idolilor.
Mic 5:15  Iar în mânia ?i în iu?imea Mea Ma voi razbuna asupra popoarelor care nu M-au ascultat.
Mic 6:1  Asculta?i deci ce zice Domnul: "Scoala-te! Fa judecata cu mun?ii, iar colinele sa auda glasul tau".
Mic 6:2  Asculta?i, voi mun?ilor, certarea Domnului ?i voi, neclintite temelii ale pamântului! Ca Domnul este în judecata cu poporul Sau ?i va grai împotriva lui Israel.
Mic 6:3  Poporul Meu! Ce ?i-am facut ?i cu ce te-am împovarat? Raspunde-Mi!
Mic 6:4  Eu sunt Cel care te-am scos din ?ara Egiptului ?i din casa robiei te-am rascumparat ?i ?i-am trimis înainte pe Moise, pe Aaron ?i pe Mariam!
Mic 6:5  Poporul Meu! Adu-?i aminte de sfatul lui Balac, regele Moabului, ?i ce i-a raspuns lui Balaam, feciorul lui Peor - când tu ai mers de la ?itim pâna la Ghilgal - ca sa cuno?ti dreptatea lui Dumnezeu".
Mic 6:6  "Cu ce ma voi înfa?i?a înaintea Domnului ?i ma voi pleca înaintea Dumnezeului celui Preaînalt? Înfa?i?a-ma-voi cu arderi de tot, cu vi?ei de un an?
Mic 6:7  Dar, oare, Domnului Îi vor placea miile de berbeci, zecile de mii de râuri de untdelemn? Oare Îi voi da pe cel dintâi nascut al meu ca pre? pentru faradelegea mea ?i rodul pântecelui meu pentru pacatul sufletului meu?"
Mic 6:8  ?i s-a aratat, omule, ceea ce este bun ?i ceea ce Dumnezeu cere de la tine: dreptate, iubire ?i milostivire ?i cu smerenie sa mergi înaintea Domnului Dumnezeului tau!
Mic 6:9  Glasul Domnului striga catre cetate - ?i este în?elept cine se teme de numele Tau -: "Asculta?i, voi semin?ii ?i tu ob?te a ceta?ii!
Mic 6:10  Oare voi scapa din vedere casa celui fara de lege, comorile celui pacatos ?i efa cea mica blestemata?
Mic 6:11  Oare voi ierta pe cel cu cântare nedrepte ?i cu greuta?i în?elatoare în sac?
Mic 6:12  Ca boga?ii din cetate sunt plini de silnicie ?i locuitorii graiesc cuvinte mincinoase ?i limba lor este numai viclenie în gura lor.
Mic 6:13  ?i Eu am început sa te bat ?i sa te pustiesc din pricina pacatelor tale.
Mic 6:14  Vei mânca, dar nu te vei satura ?i foamea va roade launtrul tau; vei pune la o parte, dar nu vei putea scapa nimic, ?i ceea ce vei scapa, voi trece prin ascu?i?ul sabiei.
Mic 6:15  Tu vei semana, dar nu vei secera; vei calca în teasc smochine, dar nu te vei unge cu untdelemn; vei face must, dar nu vei bea vin.
Mic 6:16  Caci voi a?i luat aminte la rânduielile lui Omri ?i la toate faptele casei lui Ahab ?i v-a?i purtat dupa sfaturile lor, ca sa va dau pustiirii ?i pe locuitorii tai, batjocurii. Pentru aceasta ve?i avea în sarcina voastra ocara poporului Meu!"
Mic 7:1  Vai mie! Caci am ajuns ca dupa culesul fructelor de vara, ca dupa culesul viilor! Nu se mai afla nici un strugure de mâncare, nici o smochina timpurie pe care o dore?te sufletul meu!
Mic 7:2  Om cucernic nu mai este în ?ara ?i nici un om drept pe pamânt; to?i pândesc sa verse sânge, unii întind cursa altora.
Mic 7:3  Mâinile lor sunt gata sa savâr?easca raul: capetenia cere daruri, judecatorul cere plata ?i cel mare graie?te dupa pofta sufletului sau.
Mic 7:4  Cel mai bun dintre ei este ca un spin, cel mai cinstit dintre ei este mai rau decât un gard de maracini. Ziua vestita de strajile Tale, ziua pedepsirii Tale a sosit; acum ei vor fi în mare tulburare!
Mic 7:5  Nu va încrede?i în prieteni ?i în cel de aproape nu va pune?i nadejdea ?i de aceea care se sprijina pe pieptul tau paze?te cuvintele gurii tale!
Mic 7:6  Caci feciorul defaima pe tatal sau ?i fiica se scoala împotriva mamei sale; iar du?manii omului sunt cei din casa lui.
Mic 7:7  Ci eu numai spre Domnul voi a?inti privirea, în Domnul Dumnezeul meu îmi voi pune nadejdea ?i Dumnezeul meu ma va asculta.
Mic 7:8  Nu te bucura de mine, vrajma?a mea, caci daca eu cad, ma scol, iar când stau în întuneric, Domnul este lumina mea.
Mic 7:9  Îndura-voi mânia Domnului, caci am pacatuit împotriva Lui, pâna când El va judeca pricina mea ?i îmi va face dreptate; El ma va scoate la lumina ?i voi privi dreptatea Lui.
Mic 7:10  Sa vada vrajma?a mea ?i sa se ascunda de ru?ine, ea care-mi spunea: "Unde este Domnul Dumnezeul tau?" Ochii mei se vor uita la ea, ?i ea va fi calcata în picioare ca noroiul de pe uli?e.
Mic 7:11  Vine ziua când zidurile tale vor fi zidite iara?i, când nu va mai fi nici o lege din acestea.
Mic 7:12  În ziua aceea vor veni la tine din Asiria ?i din ceta?ile Egiptului, din Egipt ?i pâna la Eufrat, de la o mare ?i pâna la cealalta ?i de la un munte pâna la altul.
Mic 7:13  ?ara se va preface în pustiu din pricina locuitorilor ei, ca pre? al purtarii lor.
Mic 7:14  Pa?te poporul cu toiagul tau, turma mo?tenirii tale, cea care sala?luie?te singura în padure, în mijlocul Carmelului! Sa pasca în Vasan ?i în Galaad ca în zilele cele de altadata!
Mic 7:15  Ca în ziua când ai ie?it din ?ara Egiptului î?i voi arata lucruri minunate!
Mic 7:16  Vedea-vor neamurile ?i se vor ru?ina, cu toata puterea lor; vor pune mâna la gura, iar urechile lor vor fi cuprinse de surzenie.
Mic 7:17  Vor linge pulbere ca ?erpii, ca ?i târâtoarele pamântului vor fi cuprinse de spaima, ie?ind din ascunzi?urile lor; vor veni tremurând catre Domnul Dumnezeul nostru ?i se vor înfrico?a de Tine.
Mic 7:18  Cine este Dumnezeu ca Tine, Care ier?i faradelegea ?i treci cu vederea pacatele restului mo?tenirii Tale? Mânia Lui nu ?ine totdeauna, caci El iube?te îndurarea.
Mic 7:19  ?i milostive?te-Te iara?i spre noi ?i faradelegile noastre calca-le în picioare! Arunca în adâncul marii toate pacatele noastre.
Mic 7:20  Pastreaza credincio?ia fagaduita lui Iacob ?i îndurarea pe care ai aratat-o lui Avraam, precum ai jurat catre parin?ii no?tri în zilele de odinioara.


\end{document}