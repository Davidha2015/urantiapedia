\begin{document}

\title{Naum}


\chapter{1}

\par 1 Proorocie despre Ninive. Cartea vedeniei lui Naum cel din Elco?.
\par 2 Domnul este un Dumnezeu zelos, Domnul se razbuna, El cunoa?te mânia. Domnul se razbuna pe potrivnicii Sai ?i împotriva du?manilor Sai sta neînduplecat.
\par 3 Domnul este îndelung-rabdator ?i mult-milostiv, dar nepedepsit nimic nu lasa. În vifor ?i în furtuna Î?i face loc, norii sunt pulberea de sub picioarele Lui.
\par 4 El cearta marea ?i o de?arta ?i toate râurile cele mari le seaca; Vasanul ?i Carmelul se ofilesc ?i floarea Libanului se ve?teje?te;
\par 5 Mun?ii se cutremura înaintea Lui ?i calinele se fac una cu pamântul; pamântul se zbuciuma în fa?a Lui, lumea ?i to?i cei ce locuiesc în ea.
\par 6 Împotriva furiei Lui cine sta ?i cine poate sa stavileasca iu?imea urgiei Lui? Vapaia mâniei Lui se varsa ca focul ?i stâncile se pravalesc în fa?a Lui!
\par 7 Bun este Domnul; loc de scapare în zi de strâmtorare, ?i pe cei ce se încred în El îi ?tie!
\par 8 Dar, printr-o revarsare de ape, va nimici locul Ninivei ?i pe vrajma?ii Lui pâna în întuneric îi va urmari.
\par 9 Ce pune?i la cale împotriva Domnului? El îi va stârpi cu totul, fiindca urgia nu vine de doua ori.
\par 10 Caci se încâlcesc unii într-al?ii ca maracinii, când se îmbata la ospe?ele lor; pentru aceasta mistui?i vor fi ca paiele uscate.
\par 11 Din tine a ie?it urzitorul de rele împotriva Domnului ?i cel ce sfatuie?te fapte ticaloase.
\par 12 A?a graie?te Domnul: "Vor fi pierdu?i ?i secera?i, oricât ar fi de viteji ?i de mul?i!"
\par 13 ?i te-am smerit, dar nu te voi mai smeri. Voi sfarâma jugul de pe grumazul tau ?i voi rupe lan?urile tale!
\par 14 ?i iata ce a poruncit Domnul pentru tine: "Nu vor mai fi urma?i cu numele tau; din casa dumnezeilor tai voi sfarâma chipurile cioplite ?i turnate ?i mormântul tau îl voi pregati, caci de pu?in pre? ai fost!"
\par 15 Iata ca pe mun?i sunt picioarele celui ce bineveste?te, ale celui ce veste?te pacea! Praznuie?te, Iuda, sarbatorile tale, împline?te fagaduin?ele tale, ca nu va mai trece pe la tine du?manul cel înver?unat! El este cu totul pierdut!

\chapter{2}

\par 1 Un vrajma? distrugator a pornit împotriva ta, Ninive; paze?te întariturile, strajuie?te calea, încinge-?i coapsele, aduna-?i toate puterile!
\par 2 Caci Domnul a sadit din nou via lui Iacov - slava lui Israel - pe care furii au pradat-o ?i i-au stricat vi?ele.
\par 3 Scuturile razboinicilor sai sunt ro?ii; îmbraca?i în ro?u aprins sunt o?tenii; în focul armelor de o?el carele se ivesc. În ziua de navala, calare?ii peste câmpii se avânta.
\par 4 Carele de lupta huruie pe uli?e, în pie?e dau unul peste altul. Înfa?i?area lor e ca a flacarilor de foc, care scapara fulgere.
\par 5 Capeteniile ei îi aduc aminte... ?i se poticnesc în mersul lor... Dau zor spre ziduri ?i pregatesc aparatori.
\par 6 Por?ile fluviilor se deschid ?i palatul de spaima este cuprins...
\par 7 S-a sfâr?it! Ea este prinsa ?i dusa în robie. Slujnicele ei scot suspine ca porumbeii ?i se bat în piept.
\par 8 Ninive este ca un iaz de apa pe vremuri! Locuitorii ei fug. "Sta?i, opri?i-va!" Nici unul nu se întoarce.
\par 9 "Prada?i aurul! Jefui?i argintul", caci fara sfâr?it sunt comorile ei: tot felul de boga?ii de lucruri scumpe.
\par 10 Ea este pustiita, jefuita ?i pradata!... Inimi frânte, genunchi care se încovoaie, coapse care tremura de groaza, ?i fe?ele tuturor sunt îngalbenite de spaima.
\par 11 Unde este culcu?ul leilor, pe?tera puilor de leu în care leul, leoaica ?i puiul de leu se învârteau ?i nimeni nu îndraznea sa-i tulbure?
\par 12 Leul, care sfâ?ia ?i sugruma prada pentru pui ?i leoaice, î?i umplea de prada pe?tera ?i culcu?urile lui de vânat?
\par 13 "Iata, sunt împotriva ta!", zice Domnul Savaot. Voi arde cu foc carele tale; pe puii tai de leu sabia îi va strapunge; voi stârpi din ?ara prada ta ?i nu se va mai auzi glasul solilor tai!"

\chapter{3}

\par 1 Vai de cetatea cea varsatoare de sânge, plina de minciuna ?i de silnicie, din care nu se mai curma jaful!
\par 2 Pocnet de bici, duruit cutremurator de ro?i, cai în galop, care de razboi care zdruncina pamântul;
\par 3 Calare?i care se avânta, sclipiri de sabie, scaparari de lance, mul?ime de rani?i, gramezi de cadavre, le?uri fara de sfâr?it, hoituri, care stau în cale!
\par 4 Numai din pricina multelor desfrânari ale celei desfrânate, frumoasa la chip ?i me?tera în farmece, care duce în robie neamurile prin desfrânarile ei ?i popoarele prin fermecatoriile ei!
\par 5 "Sunt împotriva ta!", zice Domnul. "Voi smulge ve?mântul tau, sa arat neamurilor goliciunea ?i regatelor ocara ta;
\par 6 Voi arunca asupra ta cu spurcaciuni, te voi batjocori, priveli?te te voi face
\par 7 ?i oricine te va vedea va întoarce capul de la tine, zicând: "Pierit-a Ninive! Cine o va jeli ?i de unde îi voi cauta mângâietori?"
\par 8 Oare tu e?ti mai bine întarita decât No-Amon, cea a?ezata pe malurile Nilului, înconjurata de apa, a carei întaritura era marea ?i ziduri erau apele?
\par 9 Etiopia împreuna cu Egiptul erau nesfâr?ita ei putere. Put ?i Libienii, ajutoarele ei de lupta!
\par 10 Dar ?i ea a fost surghiunita ?i dusa în robie! ?i tot a?a pruncii ei au fost zdrobi?i la raspântiile tuturor uli?elor; asupra celor de vi?a buna au aruncat sor?i ?i to?i frunta?ii ei au fost fereca?i în lan?uri.
\par 11 Tu te vei îmbata de sichera mâniei Mele ?i fara vlaga vei ajunge; tot a?a vei cere ?i tu ajutor de la du?manii tai.
\par 12 Toate întariturile tale sunt smochini cu pârga în ei; daca le scutura cineva pârga, cade în gura celui ce vrea sa o manânce.
\par 13 Iata poporul tau este ca femeile înauntrul tau! Por?ile ?arii tale se vor deschide în fa?a vrajma?ilor tai ?i focul va mistui zavoarele tale!
\par 14 Adu-?i apa pentru vremuri de împresurare, repara-?i întariturile, framânta lutul, calca pamântul cel clisos ?i fa caramida tare!
\par 15 Dar ?i atunci te va mistui focul ?i sabia te va stârpi ?i te va prapadi ca lacusta, chiar daca ai fi fara de numar ca lacustele ?i ca stolurile lor.
\par 16 Înmul?itu-?i-ai negu?atorii tai mai mult decât stelele de pe cer. Ei sunt ca lacusta care întinde aripile ?i zboara!
\par 17 Capeteniile tale sunt ca lacustele fara numar, iar dregatorii tai, ca stolurile de lacuste care se a?aza pe ziduri în vreme friguroasa ?i, dupa ce rasare soarele, zboara ?i nu se mai cunoa?te locul unde au fost.
\par 18 Pastorii tai dorm, rege al Asiriei! Vitejii dormiteaza; poporul s-a împra?tiat în mun?i ?i nu este cine sa-l adune!
\par 19 Prabu?irea ta este fara leac, ?i napraznic prapadul tau! To?i cei care vor auzi aceasta veste despre tine vor bate din palme, ca peste cine nu s-a abatut necontenit rautatea ta?


\end{document}