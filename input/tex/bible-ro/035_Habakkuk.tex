\begin{document}

\title{Habacuc}


\chapter{1}

\par 1 Vedenia pe care a văzut-o Avacum proorocul.
\par 2 Până când, Doamne, voi striga fără ca să mă asculți și voi ridica glasul meu către Tine din pricina silniciei, fără ca Tu să mă izbăvești?
\par 3 Pentru ce Tu mă lași să văd nedreptatea și privești apăsarea? Prăpădul și silnicia sînt în fața mea, certuri și gâlcevi se iscă!
\par 4 Pentru aceasta, legea nu are nici o tărie și dreptatea nu se arată niciodată; cel nelegiuit biruie pe cel drept, iar judecata iese strâmbă.
\par 5 Aruncați privirea printre popoare, fiți cu băgare de seamă și înspăimântați-vă, căci se săvârșește în vremea voastră un lucru pe care voi nu l-ați crede, dacă l-ar povesti cineva!
\par 6 Iată Eu stârnesc pe Caldei, neam amarnic și iute, care cutreieră ținuturi necuprinse, ca să pună stăpânire pe locuințe care nu sînt ale lui.
\par 7 El este năpraznic și înfricoșător. Puterea lui face dreptul lui, strălucirea lui.
\par 8 Mai iuți decât leoparzii sînt caii lui și mai mușcători decât lupii de seară. Călăreții lui se avântă, de departe, vin și zboară, ca vulturul grăbit să sfâșie prada.
\par 9 Tot neamul acesta vine ca să săvârșească silnicie; spaima merge înaintea lui și robii îi adună laolaltă ca nisipul.
\par 10 El își bate joc de regi și râde de căpetenii. El își bate joc de toate întăriturile, căci ridică în jurul lor valuri de pământ, și le cuprinde.
\par 11 Pe urmă furtuna s-a întors și a trecut; el se face vinovat. Se încrede în tăria lui; iată dumnezeul lui!
\par 12 Nu ești Tu, oare, din străvechile vremuri, Domnul Dumnezeul meu, Sfântul meu? Tu, Care nu poți muri! Tu, Doamne, ai rânduit acest popor spre dreptate și pe stâncă Tu l-ai întărit, ca să săvârșească drepte rânduieli.
\par 13 Ochii Tăi sînt prea curați ca să vadă răul, Tu nu poți privi apăsarea. Pentru ce ai privi Tu, oare, pe cei vicleni și ai tăcea, când cel nelegiuit sfâșie pe unul mai drept decât el?
\par 14 Tu socotești pe oameni ca pe peștii mării și ca pe târâtoarele care n-au stăpân?
\par 15 El îi scoate pe toți cu undița, îi trage în mreaja sa și îi adună pe toți în năvodul său. Pentru aceasta se bucură și se veselește.
\par 16 Drept aceea el aduce jertfă mrejei sale și tămâieri năvodului său, căci cu ajutorul lor partea lui este grasă și mîncărurile lui mai sățioase.
\par 17 Oare își va deșerta el într-una mreaja și va junghia fără milă popoarele?

\chapter{2}

\par 1 Voi sta de strajă și mă voi așeza în turnul cel de veghe ca să priveghez și să văd ce-mi va grăi mie și ce-mi va răspunde la tânguirea mea.
\par 2 Și Domnul mi-a răspuns și mi-a zis: "Scrie vedenia și o sapă cu slove pe table, ca să se poată citi ușor;
\par 3 Căci este o vedenie pentru un timp hotărât; ea se va împlini la vreme și nu va fi vedenie mincinoasă. Dacă întârzie, așteapt-o, căci ea va veni sigur, fără greș.
\par 4 Iată că va pieri acela al cărui suflet nu este pe calea cea dreaptă, iar dreptul din credință va fi viu!
\par 5 Cât de mult va fi copleșit vrăjmașul, omul cel trufaș, și nu va rămâne cu viață el, care ține gura lui căscată ca locuința morților și nu se satură ca moartea; el, care adună toate neamurile și cuprinde în el toate popoarele!
\par 6 Oare toate aceste popoare nu vor rosti împotriva lui pilde, fabule și cuvinte cu tâlc? și vor zice: "Vai de cel ce-și sporește averea cu ceea ce nu este al lui - până când? - și se încarcă cu povara zălogurilor luate!
\par 7 Oare nu se vor scula fără de veste cei ce te-au împrumutat și nu se vor trezi oare călăii tăi? Și tu vei ajunge prada lor!
\par 8 Și fiindcă tu ai prădat popoare fără număr, și celelalte neamuri te vor prăda pe tine, din pricina vărsărilor de sânge și a silniciilor săvârșite împotriva țării, împotriva cetății și împotriva tuturor locuitorilor lor.
\par 9 Vai de cei ce strâng câștiguri nelegiuite pentru casa lor și își așează sus de tot cuibul, ca să scape din mâna nenorocirii!
\par 10 Prin sfaturile tale ai hărăzit rușine casei tale, ai nimicit multe popoare și de aceea vei ispăși cu sufletul tău!
\par 11 Căci piatra cea din zid strigă și grinda din căpriorii casei îi răspunde.
\par 12 Vai de cel ce zidește cetatea cu vărsări de sânge și o întemeiază pe fărădelegi!
\par 13 Oare nu este aceasta o rânduială de la Domnul Savaot că popoarele se trudesc pentru foc și neamurile se muncesc pentru nimica toată?
\par 14 Căci pământul se va umple de cunoștința slavei Domnului, întocmai ca apele care acoperă sânul mării.
\par 15 Vai de cel ce adapă pe prietenul său din cupa lui otrăvită, până îl îmbată, ca să vadă goliciunea lui!
\par 16 Tu te-ai săturat de ocară în loc de mărire; bea și tu și te îmbată! Atunci cupa dreptei Domnului se va întoarce împotriva ta și rușinea va acoperi mărirea ta.
\par 17 Căci silnicia împotriva Libanului te va acoperi și distrugerea dobitoacelor te va înfricoșa din pricina sângelui vărsat și a silniciilor făcute țărilor și cetăților și tuturor locuitorilor din ele.
\par 18 La ce slujește un chip cioplit, ca să-l facă meșterul său? Un chip turnat și o proorocie mincinoasă, ca să-și pună nădejdea în ele cel care le face, făurind idoli fără glas?
\par 19 Vai de cel care zice lemnului: "Deșteaptă-te!" și pietrei mute: "Trezește-te!" Ne poate ea învăța? Poleită cu aur și cu argint, ea nu are în ea suflare de viață.
\par 20 Dar Domnul este în templul Său cel sfânt; pământule întreg, taci înaintea Lui!

\chapter{3}

\par 1 Rugăciunea proorocului Avacum cântată din harpe.
\par 2 Doamne, auzit-am de faima Ta și m-am temut de punerile Tale la cale, Dumnezeule! Fă să trăiască, în cursul anilor, lucrarea Ta și, în trecerea vremii, fă-o să fie cunoscută! Dar, întru mânia Ta, adu-ți aminte că ești și milostiv!
\par 3 Dumnezeu vine din Teman, și Cel Sfânt din muntele Paran! Sela (oprire) - Slava Lui acoperă cerurile și tot pământul este plin de slava Lui!
\par 4 Izbucnire de lumină ca la răsărit de soare, raze vii din mâna Lui pornesc!... Acolo stă tainic ascunsă puterea Lui!
\par 5 Înaintea Lui merge molima, iar prăpădul vine pe urma Lui.
\par 6 Se oprește!... Zguduie pământul! Și cu privirea Lui aruncă spaima printre neamuri! Munții cei din veac se desprind din locul lor, colinele străvechi se smeresc și pier sub pașii veșniciei Sale.
\par 7 Am văzut corturile lui Cușan (Etiopia) lovite de groază, iar colibele țării Madianului sînt cuprinse de cutremure.
\par 8 Oare împotriva fluviilor Și-a aprins Domnul văpaia Sa? Sau asupra marilor râuri mânia Sa? Ori împotriva mării urgia Ta, când Tu încaleci caii Tăi și Te sui în carele Tale de biruință?
\par 9 Arcul Tău se încordează! Săgețile Tale sînt jurămintele pe care le-ai rostit. Sela (Oprire). Cu șuvoaiele Tale spinteci pământul!
\par 10 Munții Te-au văzut și s-au cutremurat; puhoaie de apă au trecut. Adâncul și-a slobozit glasul său și mâinile sale în sus le ridică.
\par 11 Soarele și luna s-au oprit în locuința lor; ca să facă lumină, săgețile Tale pornesc și fulgerele lăncilor Tale fără încetare scapără.
\par 12 Cu mânie Tu pășești pe pământ și întru urgie Tu calci în picioare popoarele!
\par 13 Ieșit-ai ca să zdrobești poporul Tău, ca să izbăvești pe unsul Tău; ai doborât acoperișul casei celui fără de lege și temeliile ei le-ai dezvelit până jos la piatră. Sela (Oprire).
\par 14 Străpuns-ai cu săgețile tale capul lui Faraon și al celor care se năpusteau asupra mea ca să mă sfărâme, în strigăte de veselie, ca și cum porneau să sfâșie pe cel nenorocit în adăpostul lor.
\par 15 Cu caii Tăi Tu cutreieri marea, puhoiul întinselor ape.
\par 16 Auzit-am de aceasta și lăuntrul meu s-a zbuciumat la glasul Tău, tremurat-au buzele mele; putreziciunea a cuprins oasele mele și picioarele mele au șovăit. Liniștit voi aștepta vremea marii îngrijorări care va veni peste poporul care ne asuprește!
\par 17 Smochinul să nu mai înmugurească și via rod să nu mai dea; înșelătoare să fie rodirea măslinului, și ogoarele nimic să nu rodească! Turme să nu mai fie în țarcuri și vite în staule niciodată!
\par 18 Ci eu voi tresălta de veselie în Domnul, bucura-mă-voi de Dumnezeu, Mântuitorul meu!
\par 19 Domnul, Stăpânul meu, este tăria mea; El face picioarele mele ca ale căprioarelor, pe culmi poartă pașii mei! - Mai-marelui cântăreților, cu cântare din harpe.


\end{document}