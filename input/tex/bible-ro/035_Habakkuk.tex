\begin{document}

\title{Habakkuk}

Hab 1:1  Vedenia pe care a vazut-o Avacum proorocul.
Hab 1:2  Pâna când, Doamne, voi striga fara ca sa ma ascul?i ?i voi ridica glasul meu catre Tine din pricina silniciei, fara ca Tu sa ma izbave?ti?
Hab 1:3  Pentru ce Tu ma la?i sa vad nedreptatea ?i prive?ti apasarea? Prapadul ?i silnicia sînt în fa?a mea, certuri ?i gâlcevi se isca!
Hab 1:4  Pentru aceasta, legea nu are nici o tarie ?i dreptatea nu se arata niciodata; cel nelegiuit biruie pe cel drept, iar judecata iese strâmba.
Hab 1:5  Arunca?i privirea printre popoare, fi?i cu bagare de seama ?i înspaimânta?i-va, caci se savâr?e?te în vremea voastra un lucru pe care voi nu l-a?i crede, daca l-ar povesti cineva!
Hab 1:6  Iata Eu stârnesc pe Caldei, neam amarnic ?i iute, care cutreiera ?inuturi necuprinse, ca sa puna stapânire pe locuin?e care nu sînt ale lui.
Hab 1:7  El este napraznic ?i înfrico?ator. Puterea lui face dreptul lui, stralucirea lui.
Hab 1:8  Mai iu?i decât leoparzii sînt caii lui ?i mai mu?catori decât lupii de seara. Calare?ii lui se avânta, de departe, vin ?i zboara, ca vulturul grabit sa sfâ?ie prada.
Hab 1:9  Tot neamul acesta vine ca sa savâr?easca silnicie; spaima merge înaintea lui ?i robii îi aduna laolalta ca nisipul.
Hab 1:10  El î?i bate joc de regi ?i râde de capetenii. El î?i bate joc de toate întariturile, caci ridica în jurul lor valuri de pamânt, ?i le cuprinde.
Hab 1:11  Pe urma furtuna s-a întors ?i a trecut; el se face vinovat. Se încrede în taria lui; iata dumnezeul lui!
Hab 1:12  Nu e?ti Tu, oare, din stravechile vremuri, Domnul Dumnezeul meu, Sfântul meu? Tu, Care nu po?i muri! Tu, Doamne, ai rânduit acest popor spre dreptate ?i pe stânca Tu l-ai întarit, ca sa savâr?easca drepte rânduieli.
Hab 1:13  Ochii Tai sînt prea cura?i ca sa vada raul, Tu nu po?i privi apasarea. Pentru ce ai privi Tu, oare, pe cei vicleni ?i ai tacea, când cel nelegiuit sfâ?ie pe unul mai drept decât el?
Hab 1:14  Tu socote?ti pe oameni ca pe pe?tii marii ?i ca pe târâtoarele care n-au stapân?
Hab 1:15  El îi scoate pe to?i cu undi?a, îi trage în mreaja sa ?i îi aduna pe to?i în navodul sau. Pentru aceasta se bucura ?i se vesele?te.
Hab 1:16  Drept aceea el aduce jertfa mrejei sale ?i tamâieri navodului sau, caci cu ajutorul lor partea lui este grasa ?i mîncarurile lui mai sa?ioase.
Hab 1:17  Oare î?i va de?erta el într-una mreaja ?i va junghia fara mila popoarele?
Hab 2:1  Voi sta de straja ?i ma voi a?eza în turnul cel de veghe ca sa priveghez ?i sa vad ce-mi va grai mie ?i ce-mi va raspunde la tânguirea mea.
Hab 2:2  ?i Domnul mi-a raspuns ?i mi-a zis: "Scrie vedenia ?i o sapa cu slove pe table, ca sa se poata citi u?or;
Hab 2:3  Caci este o vedenie pentru un timp hotarât; ea se va împlini la vreme ?i nu va fi vedenie mincinoasa. Daca întârzie, a?teapt-o, caci ea va veni sigur, fara gre?.
Hab 2:4  Iata ca va pieri acela al carui suflet nu este pe calea cea dreapta, iar dreptul din credin?a va fi viu!
Hab 2:5  Cât de mult va fi cople?it vrajma?ul, omul cel trufa?, ?i nu va ramâne cu via?a el, care ?ine gura lui cascata ca locuin?a mor?ilor ?i nu se satura ca moartea; el, care aduna toate neamurile ?i cuprinde în el toate popoarele!
Hab 2:6  Oare toate aceste popoare nu vor rosti împotriva lui pilde, fabule ?i cuvinte cu tâlc? ?i vor zice: "Vai de cel ce-?i spore?te averea cu ceea ce nu este al lui - pâna când? - ?i se încarca cu povara zalogurilor luate!
Hab 2:7  Oare nu se vor scula fara de veste cei ce te-au împrumutat ?i nu se vor trezi oare calaii tai? ?i tu vei ajunge prada lor!
Hab 2:8  ?i fiindca tu ai pradat popoare fara numar, ?i celelalte neamuri te vor prada pe tine, din pricina varsarilor de sânge ?i a silniciilor savâr?ite împotriva ?arii, împotriva ceta?ii ?i împotriva tuturor locuitorilor lor.
Hab 2:9  Vai de cei ce strâng câ?tiguri nelegiuite pentru casa lor ?i î?i a?eaza sus de tot cuibul, ca sa scape din mâna nenorocirii!
Hab 2:10  Prin sfaturile tale ai harazit ru?ine casei tale, ai nimicit multe popoare ?i de aceea vei ispa?i cu sufletul tau!
Hab 2:11  Caci piatra cea din zid striga ?i grinda din capriorii casei îi raspunde.
Hab 2:12  Vai de cel ce zide?te cetatea cu varsari de sânge ?i o întemeiaza pe faradelegi!
Hab 2:13  Oare nu este aceasta o rânduiala de la Domnul Savaot ca popoarele se trudesc pentru foc ?i neamurile se muncesc pentru nimica toata?
Hab 2:14  Caci pamântul se va umple de cuno?tin?a slavei Domnului, întocmai ca apele care acopera sânul marii.
Hab 2:15  Vai de cel ce adapa pe prietenul sau din cupa lui otravita, pâna îl îmbata, ca sa vada goliciunea lui!
Hab 2:16  Tu te-ai saturat de ocara în loc de marire; bea ?i tu ?i te îmbata! Atunci cupa dreptei Domnului se va întoarce împotriva ta ?i ru?inea va acoperi marirea ta.
Hab 2:17  Caci silnicia împotriva Libanului te va acoperi ?i distrugerea dobitoacelor te va înfrico?a din pricina sângelui varsat ?i a silniciilor facute ?arilor ?i ceta?ilor ?i tuturor locuitorilor din ele.
Hab 2:18  La ce sluje?te un chip cioplit, ca sa-l faca me?terul sau? Un chip turnat ?i o proorocie mincinoasa, ca sa-?i puna nadejdea în ele cel care le face, faurind idoli fara glas?
Hab 2:19  Vai de cel care zice lemnului: "De?teapta-te!" ?i pietrei mute: "Treze?te-te!" Ne poate ea înva?a? Poleita cu aur ?i cu argint, ea nu are în ea suflare de via?a.
Hab 2:20  Dar Domnul este în templul Sau cel sfânt; pamântule întreg, taci înaintea Lui!
Hab 3:1  Rugaciunea proorocului Avacum cântata din harpe.
Hab 3:2  Doamne, auzit-am de faima Ta ?i m-am temut de punerile Tale la cale, Dumnezeule! Fa sa traiasca, în cursul anilor, lucrarea Ta ?i, în trecerea vremii, fa-o sa fie cunoscuta! Dar, întru mânia Ta, adu-?i aminte ca e?ti ?i milostiv!
Hab 3:3  Dumnezeu vine din Teman, ?i Cel Sfânt din muntele Paran! Sela (oprire) - Slava Lui acopera cerurile ?i tot pamântul este plin de slava Lui!
Hab 3:4  Izbucnire de lumina ca la rasarit de soare, raze vii din mâna Lui pornesc!... Acolo sta tainic ascunsa puterea Lui!
Hab 3:5  Înaintea Lui merge molima, iar prapadul vine pe urma Lui.
Hab 3:6  Se opre?te!... Zguduie pamântul! ?i cu privirea Lui arunca spaima printre neamuri! Mun?ii cei din veac se desprind din locul lor, colinele stravechi se smeresc ?i pier sub pa?ii ve?niciei Sale.
Hab 3:7  Am vazut corturile lui Cu?an (Etiopia) lovite de groaza, iar colibele ?arii Madianului sînt cuprinse de cutremure.
Hab 3:8  Oare împotriva fluviilor ?i-a aprins Domnul vapaia Sa? Sau asupra marilor râuri mânia Sa? Ori împotriva marii urgia Ta, când Tu încaleci caii Tai ?i Te sui în carele Tale de biruin?a?
Hab 3:9  Arcul Tau se încordeaza! Sage?ile Tale sînt juramintele pe care le-ai rostit. Sela (Oprire). Cu ?uvoaiele Tale spinteci pamântul!
Hab 3:10  Mun?ii Te-au vazut ?i s-au cutremurat; puhoaie de apa au trecut. Adâncul ?i-a slobozit glasul sau ?i mâinile sale în sus le ridica.
Hab 3:11  Soarele ?i luna s-au oprit în locuin?a lor; ca sa faca lumina, sage?ile Tale pornesc ?i fulgerele lancilor Tale fara încetare scapara.
Hab 3:12  Cu mânie Tu pa?e?ti pe pamânt ?i întru urgie Tu calci în picioare popoarele!
Hab 3:13  Ie?it-ai ca sa zdrobe?ti poporul Tau, ca sa izbave?ti pe unsul Tau; ai doborât acoperi?ul casei celui fara de lege ?i temeliile ei le-ai dezvelit pâna jos la piatra. Sela (Oprire).
Hab 3:14  Strapuns-ai cu sage?ile tale capul lui Faraon ?i al celor care se napusteau asupra mea ca sa ma sfarâme, în strigate de veselie, ca ?i cum porneau sa sfâ?ie pe cel nenorocit în adapostul lor.
Hab 3:15  Cu caii Tai Tu cutreieri marea, puhoiul întinselor ape.
Hab 3:16  Auzit-am de aceasta ?i launtrul meu s-a zbuciumat la glasul Tau, tremurat-au buzele mele; putreziciunea a cuprins oasele mele ?i picioarele mele au ?ovait. Lini?tit voi a?tepta vremea marii îngrijorari care va veni peste poporul care ne asupre?te!
Hab 3:17  Smochinul sa nu mai înmugureasca ?i via rod sa nu mai dea; în?elatoare sa fie rodirea maslinului, ?i ogoarele nimic sa nu rodeasca! Turme sa nu mai fie în ?arcuri ?i vite în staule niciodata!
Hab 3:18  Ci eu voi tresalta de veselie în Domnul, bucura-ma-voi de Dumnezeu, Mântuitorul meu!
Hab 3:19  Domnul, Stapânul meu, este taria mea; El face picioarele mele ca ale caprioarelor, pe culmi poarta pa?ii mei! - Mai-marelui cântare?ilor, cu cântare din harpe.


\end{document}