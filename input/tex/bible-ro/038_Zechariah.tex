\begin{document}

\title{Zechariah}

Zec 1:1  În luna a opta, în anul al doilea al lui Darius, fost-a cuvântul Domnului catre proorocul Zaharia, fiul lui Berechia, fiul lui Ido, zicând:
Zec 1:2  "Domnul S-a mâniat mult împotriva parin?ilor vo?tri!"
Zec 1:3  ?i spune-le celor rama?i din popor: "A?a zice Domnul Savaot: Întoarce?i-va catre Mine", zice Domnul Savaot, "?i atunci Ma voi întoarce ?i Eu catre voi", zice Domnul Savaot.
Zec 1:4  "Nu fi?i ca parin?ii vo?tri, carora le-au propovaduit proorocii de mai înainte, strigând: A?a zice Domnul Savaot: Întoarce?i-va din caile voastre cele rele ?i de la faptele voastre cele rele! Dar ei n-au ascultat ?i nu au voit sa ia aminte la Mine", zice Domnul.
Zec 1:5  "Unde sunt parin?ii vo?tri? ?i oare profe?ii traiesc ei înca?
Zec 1:6  Dar cuvintele Mele ?i poruncile Mele pe care le-am dat slujitorilor Mei prooroci ca sa le vesteasca n-au ajuns ele la parin?ii vo?tri? Domnul S-a mâniat mult împotriva parin?ilor vo?tri, încât ei s-au pocait ?i au marturisit: Domnul Savaot ne-a rasplatit noua dupa caile ?i dupa faptele noastre, a?a cum hotarâse sa faca".
Zec 1:7  În ziua a douazeci ?i patra din luna a doua, care se cheama ?ebat, în anul al doilea al lui Darius, a fost cuvântul Domnului catre proorocul Zaharia, fiul lui Berechia, fiul lui Ido, zicând:
Zec 1:8  "Am avut o vedenie în timpul nop?ii, ?i iata un om calare pe un cal ro?u; ?i statea între mir?i, într-un loc umbros, ?i în urma lui cai roibi, murgi ?i albi.
Zec 1:9  ?i am zis: "Cine sunt ace?tia, domnul meu?" ?i mira raspuns atunci îngerul oare graia cu mine: "Î?i voi arata acum cine sunt ace?tia".
Zec 1:10  ?i omul care statea între mir?i a raspuns: "Ace?tia sunt solii pe care i-a trimis Domnul ca sa cutreiere pamântul!"
Zec 1:11  ?i ei au raspuns catre îngerul Domnului care statea între mir?i ?i au grait: "Am cutreierat pamântul ?i iata tot pamântul este locuit ?i lini?tit!"
Zec 1:12  ?i a raspuns îngerul Domnului ?i a zis: "Doamne Savaot, pâna când vei întârzia sa ara?i mila Ierusalimului ?i ceta?ilor lui Iuda, pe care le faci sa simta mânia Ta de ?aptezeci de ani?"
Zec 1:13  ?i îngerului care vorbea cu mine, Domnul i-a raspuns cu cuvinte de mângâiere.
Zec 1:14  ?i a grait catre mine îngerul care vorbea cu mine: "Veste?te aceasta: "A?a zice Domnul Savaot: Sunt plin de zel fa?a de Ierusalim ?i fa?a de Sion;
Zec 1:15  Dar este puternica mânia Mea împotriva neamurilor trufa?e! Caci Eu Ma întarâtasem doar pu?in, dar ele s-au înver?unat în rele".
Zec 1:16  Pentru aceasta, a?a zice Domnul: "Ma întorc iara?i catre Ierusalim cu milostivire; templul Meu va fi zidit în el", zice Domnul Savaot, "?i funia de masurat se va întinde peste Ierusalim!"
Zec 1:17  ?i veste?te înca ?i aceasta: "A?a zice Domnul Savaot: Înca o data ceta?ile Mele vor avea bel?ug de bunuri, iar Domnul Î?i va revarsa din nou îndurarea Sa peste Sion ?i va alege iara?i Ierusalimul".
Zec 1:18  ?i am ridicat ochii mei ?i am privit, ?i iata patru coarne.
Zec 1:19  Atunci am zis îngerului care graia cu mine: "Ce sunt acestea?" ?i el mi-a raspuns: "Acestea sunt coarnele care au împra?tiat pe Iuda, pe Israel ?i Ierusalimul".
Zec 1:20  Apoi Domnul m-a facut sa vad patru faurari.
Zec 1:21  ?i am întrebat: "Ce vor sa faca ace?tia?" ?i El a raspuns: "Acestea sunt coarnele care au risipit pe Iuda, încât nimeni n-a îndraznit sa ridice capul. Iar ei au venit ca sa doboare la pamânt coarnele acelor neamuri care ?i-au ridicat cornul împotriva ?arii lui Iuda, pentru a o risipi".
Zec 2:1  ?i am ridicat ochii, m-am uitat ?i iata un om cu o funie de masurat în mâna. ?i i-am zis: "Încotro ai pornit?"
Zec 2:2  ?i el mi-a raspuns: "Sa masor Ierusalimul, ca sa vad care este lungimea ?i la?imea lui!"
Zec 2:3  ?i iata ca s-a aratat îngerul care graia cu mine ?i alt înger a ie?it în întâmpinarea lui.
Zec 2:4  ?i i?a grait, zicând: "Alearga ?i spune tânarului acestuia: Ierusalimul va fi locuit ca un ora? deschis, atât de mare va fi mul?imea de oameni ?i de dobitoace înauntrul lui.
Zec 2:5  ?i Eu voi fi pentru el un zid de foc de jur împrejurul lui", zice Domnul, "?i voi fi slava lui în mijlocul lui".
Zec 2:6  "Scula?i, scula?i ?i fugi?i din ?ara de la miazanoapte", zice Domnul, "ca v-am împra?tiat în cele patru vânturi ale cerului;
Zec 2:7  Fugi, Sioane, ?i te izbave?te, tu care locuie?ti la fiica Babilonului.
Zec 2:8  Caci a?a zice Domnul Savaot: "Pentru slava Sa, El m-a trimis la neamurile care v-au jefuit pe voi; caci cel care se atinge de voi se atinge de lumina ochiului Lui!
Zec 2:9  Caci iata ca Eu rotesc mâna Mea peste ei ?i ei vor fi prada pentru cei care au fost robii lor, ca sa va da?i seama ca Domnul Savaot m-a trimis.
Zec 2:10  Bucura-te ?i te vesele?te, fiica Sionului, caci iata Eu vin sa locuiesc în mijlocul tau", zice Domnul.
Zec 2:11  "?i multe neamuri se vor alipi de Domnul în ziua aceea ?i Îmi vor fi Mie popor ?i voi locui în mijlocul tau, ca sa ?tii ca Domnul Savaot m-a trimis la tine.
Zec 2:12  ?i va lua Domnul ca mo?tenire a Sa pe Iuda, în ?ara cea sfânta, ?i va alege înca o data Ierusalimul.
Zec 2:13  Sa taca tot trupul înaintea Domnului, caci El S-a ridicat din loca?ul Sau cel sfânt".
Zec 3:1  ?i mi-a aratat pe Iosua, marele preot, stând înaintea îngerului Domnului, ?i pe Satana, stând la dreapta lui ca sa-l învinuiasca.
Zec 3:2  ?i a zis Domnul catre Satana: "Cearta-te pe tine Domnul, diavole, cearta-te pe tine Domnul, Cel care a ales Ierusalimul! Acesta nu este el, oare, un taciune scos din foc?"
Zec 3:3  ?i era Iosua îmbracat în ve?minte murdare ?i statea înaintea îngerului.
Zec 3:4  ?i a raspuns ?i a zis celor care stateau înaintea lui: "Dezbraca?i-l de ve?mintele cele murdare!" ?i i-a zis lui: "Iata ?i-am iertat faradelegile ?i te-am îmbracat cu ve?mânt de praznuire!"
Zec 3:5  ?i a mai zis: "Pune?i mitra curata pe capul lui!" ?i ei i-au pus mitra curata pe cap ?i l-au înve?mântat, iar îngerul Domnului sta de fa?a.
Zec 3:6  ?i îngerul Domnului i-a hotarât lui Iosua astfel:
Zec 3:7  "A?a zice Domnul Savaot: Daca vei umbla în caile Mele ?i de vei pazi poruncile Mele, atunci tu vei cârmui casa Mea ?i vei pazi cur?ile Mele, ?i Eu î?i voi da ?ie dregatorie printre slujitorii Mei de aici.
Zec 3:8  Asculta deci, Iosua, mare preot, tu ?i cei împreuna cu tine care stau înaintea ta; caci ei sunt oameni de prezicere. Iata Eu aduc pe Servul Meu Odrasla.
Zec 3:9  Iata piatra pe care am pus-o înaintea lui Iosua; pe aceasta piatra sunt ?apte ochi: iata ca Eu voi sapa sculptura ei", zice Domnul Savaot, "?i într-o singura zi voi îndeparta nedreptatea din ?ara aceasta.
Zec 3:10  În ziua aceea", zice Domnul Savaot, "fiecare din voi va pofti pe aproapele sau sub vi?a ?i sub smochinul sau".
Zec 4:1  ?i s-a întors îngerul care graia cu mine ?i m-a de?teptat, ca pe un om pe care îl treze?ti din somn.
Zec 4:2  ?i mi-a zis: "Ce vezi?" ?i am zis: "Iata vad un candelabru cu totul de aur, cu ?apte candele, iar deasupra candelabrului este un vas cu untdelemn din care pornesc ?apte ?evi catre cele ?apte candele;
Zec 4:3  Iar alaturi, doi maslini, unul de-a dreapta vasului cu untdelemn ?i altul de-a stânga".
Zec 4:4  ?i am întrebat ?i am zis catre îngerul care vorbea cu mine: "Ce sunt astea, domnul meu?"
Zec 4:5  ?i mi-a raspuns îngerul care graia cu mine ?i mi-a zis: "Oare nu ?tii ce sunt toate acestea?" ?i am zis: "Nu, domnul meu!"
Zec 4:6  ?i mi-a vorbit iar ?i a zis: "Acesta este cuvântul Domnului catre Zorobabel: Nu prin putere, nici prin tarie, ci prin Duhul Meu" - zice Domnul Savaot.
Zec 4:7  "Ce e?ti tu, munte înalt? Înaintea lui Zorobabel devii o câmpie. ?i el va smulge piatra din vârf în strigatele mul?imii: "Har, har, peste ea!"
Zec 4:8  ?i a fost cuvântul Domnului catre mine ?i mi-a zis:
Zec 4:9  "Mâinile lui Zorobabel au pus temelia acestui templu ?i tot mâinile lui îl vor termina, ?i tu vei ?ti ca Domnul Savaot m-a trimis la voi.
Zec 4:10  Caci cine a dispre?uit vremea acestor începuturi mici? Ei se vor bucura vazând cumpana zidarului în mâna lui Zorobabel. Iar aceste ?apte (candele) sunt ochii Domnului care cutreiera tot pamântul".
Zec 4:11  ?i mi-am luat îndemnul ?i am zis catre el: "Ce înseamna ace?ti maslini unul de-a dreapta ?i altul de-a stânga candelabrului?"
Zec 4:12  ?i l-am întrebat ?i a doua oara: "Ce înseamna cele doua crengi de maslin, care sunt lânga cele doua ?evi de aur prin care se lasa în jos untdelemnul?"
Zec 4:13  ?i mi-a grait, zicând: "Oare nu ?tii ce înseamna acestea?" ?i am raspuns: "Nu, domnul meu".
Zec 4:14  ?i m-a lamurit: "Ace?tia sunt cei doi fii un?i, care stau înaintea Domnului a tot pamântul".
Zec 5:1  ?i iara?i am ridicat ochii ?i am privit ?i iata un sul de carte care se înal?a în zbor.
Zec 5:2  ?i a grait catre mine, zicând: "Ce vezi tu?" ?i am zis: "Vad un sul de carte care zboara, lung de douazeci de co?i ?i lat de zece".
Zec 5:3  ?i el mi-a tâlcuit: "Acesta este blestemul care se raspânde?te peste fa?a a tot pamântul, caci orice fur va fi nimicit de aici ?i orice am care jura strâmb va fi pierdut.
Zec 5:4  I-am dat drumul", zice Domnul Savaot, "?i se va duce în casa furului ?i în casa celui care jura strâmb pe numele Meu ?i va ramâne în ea ?i va nimici lemnele ?i pietrele ei".
Zec 5:5  ?i îngerul care graia cu mine a ie?it la iveala ?i mi-a zis: "Ridica ochii tai ?i vezi: Ce este aratarea aceasta?"
Zec 5:6  Atunci am grait: "Ce este aceasta?" ?i el mi-a raspuns: "Este efa care iese la iveala". ?i a spus mai departe: "În ea se afla faradelegea a tot pamântul!"
Zec 5:7  ?i iata ca s-a ridicat un disc de plumb, iar o femeie statea în mijlocul efei.
Zec 5:8  ?i el a tâlcuit: "Aceasta este faradelegea!" ?i el a aruncat-o în efa ?i a rasturnat lespedea de plumb deasupra ei.
Zec 5:9  ?i am ridicat ochii mei ?i am privit ?i iata ca au ie?it doua femei. ?i vântul batea în aripile lor, iar aripile lor erau ca de barza. ?i ele au ridicat efa intre pamânt ?i cer.
Zec 5:10  ?i am zis catre îngerul oare graia cu mine: "Încotro duc ele efa?"
Zec 5:11  Atunci el mi-a raspuns: "Ele merg sa-i zideasca o casa în pamântul ?inear ?i Acad ?i sa o a?eze acolo la locul ei".
Zec 6:1  ?i am ridicat iara?i ochii mei ?i m-am uitat. ?i iata ca ie?eau patru care dintre doi mun?i ?i mun?ii erau de arama.
Zec 6:2  La carul cel dintâi erau înhama?i cai ro?ii, iar la carul cel de-al doilea erau înhama?i cai negri.
Zec 6:3  La cel de-al treilea car erau înhama?i cai albi, iar la cel de-al patrulea, cai bal?a?i, puternici.
Zec 6:4  ?i mi-am luat îndemnul ?i am zis catre îngerul care graia cu mine: "Ce sunt acestea, domnul meu?"
Zec 6:5  Atunci mi-a raspuns îngerul ?i mi-a zis: "Acestea sunt cele patru vânturi ale cerului, care ies dupa ce s-au înfa?i?at înaintea Stapânului a tot pamântul.
Zec 6:6  Caii ro?ii înainteaza spre ?ara de la rasarit; cei negri înainteaza spre ?ara de la miazanoapte; cei albi înainteaza spre ?ara de la apus, ?i cei bal?a?i înainteaza spre ?ara de la miazazi.
Zec 6:7  Puternici, ei înaintau nerabdatori sa strabata pamântul. El le-a zis: "Pleca?i ?i cutreiera?i pamântul!" ?i ei au cutreierat pamântul.
Zec 6:8  ?i a strigat catre mine ?i mi-a grait, zicând: "Iata, cei ce se îndreapta catre ?inutul cel de miazanoapte au potolit Duhul Meu în ?ara de la miazanoapte".
Zec 6:9  ?i a fost cuvântul Domnului catre mine ?i mi-a zis:
Zec 6:10  "Tu vei primi darurile celor întor?i din robie, de la Heldai, Tobia ?i Iedaia, ?i în aceea?i zi mergi în casa lui Io?ia, fiul lui Sofonie, care a sosit din Babilon.
Zec 6:11  ?i vei lua argint ?i aur ?i vei face o cununa ?i o vei pune pe capul lui Iosua marele preot, feciorul lui Io?adoc.
Zec 6:12  ?i-i vei zice lui în acest chip: "A?a graie?te Domnul Savaot: Iata un om care va fi chemat Odrasla; acesta va odrasli ?i va zidi templul Domnului.
Zec 6:13  ?i acesta va rezidi templul Domnului. El va purta semnele regale ?i va stapâni ?i va domni pe tronul lui ?i un preot va fi la dreapta lui. Între ei doi va fi o pace desavâr?ita.
Zec 6:14  ?i cununa va fi pentru Heldai, Tobia ?i Iedaia ?i pentru Io?ia, fiul lui Sofonie, ca aducere aminte în templul Domnului.
Zec 6:15  ?i oameni de la mari departari vor veni ?i vor zidi la templul Domnului, ?i atunci voi ve?i ?ti ca Domnul Savaot m-a trimis catre voi. ?i aceasta se va întâmpla, daca voi ve?i asculta cu credincio?ie de glasul Domnului Dumnezeului vostru!"
Zec 7:1  ?i a fost, în anul al patrulea al regelui Darius, cuvântul Domnului catre Zaharia în ziua a patra a lunii a noua, care se cheama Chislev.
Zec 7:2  Casa lui Israel a trimis pe ?are?er, dregatorul cel mare al o?tirii regelui, cu oameni, ca sa câ?tige milostivirea Domnului
Zec 7:3  Sa graiasca preo?ilor din templul Domnului Savaot ?i proorocilor într-acest chip: "Sa mai plâng eu oare în luna a cincea ?i sa ma înfrânez precum am facut atâ?ia ani?"
Zec 7:4  ?i a fost cuvântul Domnului Savaot catre mine ?i mi-a zis:
Zec 7:5  "Veste?te la tot poporul ?arii ?i preo?ilor: "Daca a?i ?inut post ?i v-a?i tânguit în luna a cincea ?i într-a ?aptea, vreme de ?aptezeci de ani, pentru Mine, oare, a?i postit voi?
Zec 7:6  ?i daca mânca?i ?i be?i, oare nu sunte?i voi aceia care mânca?i ?i care be?i?
Zec 7:7  Nu cunoa?te?i voi cuvintele pe care le-a vestit Domnul prin graiul profe?ilor de mai înainte, când Ierusalimul era locuit ?i pa?nic, cu ceta?ile sale dimprejur, ?i când Neghebul ?i ?efela erau locuite?"
Zec 7:8  ?i a fost cuvântul Domnului catre Zaharia într-acest chip:
Zec 7:9  "A?a graie?te Domnul Savaot: "Face?i dreptate adevarata ?i purta?i-va fiecare cu bunatate ?i îndurare fa?a de fratele sau;
Zec 7:10  Nu apasa?i pe vaduva, pe orfan, pe strain ?i pe cel sarman ?i nimeni sa nu puna la cale faradelegi în inima lui împotriva fratelui sau!"
Zec 7:11  Dar ei n-au voit sa ia aminte, ci au întors spatele cu îndaratnicie ?i ?i-au astupat urechile ca sa nu auda;
Zec 7:12  ?i ?i-au învârto?at inima ca diamantul, ca sa nu asculte legea ?i cuvintele pe care le-a trimis Domnul Savaot prin Duhul Lui, prin graiul proorocilor celor de altadata. ?i a fost o mânie mare de la Domnul Savaot.
Zec 7:13  ?i s-a întâmplat ca dupa cum El a strigat ?i ei n-au auzit, tot a?a "ei vor striga ?i Eu nu-i voi auzi", zice Domnul Savaot;
Zec 7:14  "?i îi voi împra?tia printre toate neamurile pe care ei nu le cunosc!" ?i ?ara a fost pustiita dupa aceea ?i n-a mai ramas nimeni în ea, fiind prefacuta ?ara cea placuta în pustiu.
Zec 8:1  ?i a fost cuvântul Domnului Savaot, zicând:
Zec 8:2  "A?a graie?te Domnul Savaot: Iubesc Sionul cu zel mare ?i cu patima aprinsa îl doresc".
Zec 8:3  A?a graie?te Domnul: "M-am întors cu milostivire catre Sion ?i voi locui în mijlocul Ierusalimului; ?i Ierusalimul se va chema cetate credincioasa ?i muntele Domnului Savant, munte sfânt".
Zec 8:4  A?a zice Domnul Savaot: "Batrâni ?i batrâne vor ?edea iara?i în pie?ele Ierusalimului, fiecare cu toiagul în mâna, din pricina vârstei lor înaintate.
Zec 8:5  ?i se vor umple pie?ele de baie?i ?i de fete, care se vor juca în pie?ele lui".
Zec 8:6  A?a zice Domnul Savaot: "Daca acest lucru se va parea greu de facut înaintea ochilor restului acestui popor în zilele acelea, care va fi ?i înaintea ochilor Mei cu neputin?a?" zice Domnul Savaot.
Zec 8:7  A?a graie?te Domnul Savaot: "Iata ca Eu voi izbavi pe poporul Meu din ?arile de la rasarit ?i din ?arile de la apus.
Zec 8:8  ?i îi voi aduce înapoi ?i ei vor locui în mijlocul Ierusalimului ?i ei Îmi vor fi Mie popor, iar Eu le voi fi Dumnezeu, în credincio?ie ?i în dreptate".
Zec 8:9  A?a graie?te Domnul Savaot: "Întari?i-va mâinile voastre, voi care auzi?i în zilele acestea cuvintele din gura proorocilor din vremea în care s-a pus temelia templului Domnului Savaot, ca sa se zideasca templul!
Zec 8:10  Caci înainte vreme nu se rasplatea nici omul, nici dobitocul, ?i cel care intra ?i cel ce ie?ea n-aveau tihna din pricina du?manilor, ?i Eu pornisem pe to?i oamenii, unii împotriva altora;
Zec 8:11  Dar acum nu mai sunt ca înainte vreme fa?a de restul poporului Meu", zice Domnul Savaot.
Zec 8:12  "Ci acum se fac semanaturi în buna pace! Via î?i va da rodul ei ?i pamântul va da roadele lui, cerul va lasa sa pice roua, iar Eu voi da în stapânire, restului acestui popor, toate bunata?ile acestea.
Zec 8:13  ?i se va întâmpla ca, precum a?i fost un blestem printre neamuri, tot a?a, o, voi, casa lui Iuda ?i ra?a lui Israel, va voi izbavi pe voi, ?i veri fi o binecuvântare. Nu va teme?i, ci întari?i-va mâinile!"
Zec 8:14  Caci a?a zice Domnul Savaot: "Precum M-am hotarât sa va pedepsesc, când aii întarâtat mânia Mea", zice Domnul Savaot, "?i nu Mi-a fost mila de voi,
Zec 8:15  Tot astfel M-am gândit ?i am sa fac bine Ierusalimului în aceste zile ?i casei lui Iuda. Nu va teme?i!"
Zec 8:16  Iata rânduielile pe care trebuie sa le pazi?i: "Sa spuna omul adevarat aproapelui sau. Judeca?i ?i da?i hotarâri drepte la por?ile voastre;
Zec 8:17  Sa nu cugeta?i faradelege unul împotriva altuia ?i juramântul strâmb sa nu-l iubi?i, caci toate acestea le urasc", zice Domnul.
Zec 8:18  ?i a fost cuvântul Domnului Savaot catre mine zicând:
Zec 8:19  "A?a zice Domnul Savaot: Postul din luna a patra, a cincea, a ?aptea ?i a zecea vor fi pentru casa lui Iuda spre veselie ?i bucurie ?i zile bune de sarbatoare! Dar iubi?i adevarul ?i pacea!
Zec 8:20  A?a zice Domnul Savaot: "?i vor veni înca popoare ?i locuitori din ceta?i numeroase;
Zec 8:21  ?i locuitorii dintr-o cetate vor merge în cealalta, zicând: "Sa mergem sa dobândim îndurarea lui Iuda ?i sa cautam pe Domnul Savaot!"Merg ?i eu!"
Zec 8:22  ?i vor veni popoare multe ?i neamuri puternice ca sa caute pe Domnul Savaot în Ierusalim ?i sa se roage Domnului".
Zec 8:23  A?a graie?te Domnul Savaot: "?i în zilele acelea zece oameni dintre limbile neamurilor vor apuca pe un iudeu de poala hainei ?i vor zice: Mergem ?i noi cu tine, caci am aflat ca Dumnezeu este cu voi!"
Zec 9:1  Proorocie. Cuvântul Domnului împotriva ?arii Hadrac ?i catre Damasc, cetatea sa, caci Domnul are ochii Sai asupra oamenilor ?i asupra tuturor semin?iilor lui Israel;
Zec 9:2  De asemenea ?i catre Hamat, care se învecineaza cu Damascul, catre Tir ?i catre Sidon, cu toata în?elepciunea lor.
Zec 9:3  Tirul ?i-a zidit întarituri ?i a strâns argint ca pulberea ?i aur ca tina de pe uli?e.
Zec 9:4  Dar iata ca Domnul îl va cuceri ?i va pravali în mare puterea lui ?i el va fi mistuit prin foc.
Zec 9:5  Ascalonul va vedea ?i se va înspaimânta, Gaza va fi cuprinsa de dureri naprasnice ?i Ecronul la fel, caci nadejdea lui s-a prabu?it. ?i nu va mai fi rege în Gaza ?i Ascalonul pustiu va ramâne nelocuit.
Zec 9:6  Cel de alt neam va locui în A?dod! ?i trufia filisteanului o voi nimici,
Zec 9:7  ?i îi voi scoate sângele din gura ?i urâciunile dintre din?ii lui. Va fi ?i el un rest pentru Dumnezeul nostru ?i va fi ca o familie în Iuda, iar Ecronul va fi ca Iebusitul.
Zec 9:8  ?i Ma voi a?eza ca straja împrejurul Casei Mele, ca nimeni dintre cei ce vin ?i pleaca ?i ca nici un asupritor sa nu mai vina asupra ei, caci Eu vad acum aceasta cu ochii Mei.
Zec 9:9  Bucura-te foarte, fiica Sionului, vesele?te-te, fiica Ierusalimului, caci iata Împaratul tau vine la tine drept ?i biruitor; smerit ?i calare pe asin, pe mânzul asinei.
Zec 9:10  El va nimici carele din Efraim, caii din Ierusalim ?i arcul de razboi va fi frânt. El va vesti pacea popoarelor ?i împara?ia Lui se va întinde de la o mare pâna la cealalta mare ?i de la Eufrat pâna la marginile pamântului.
Zec 9:11  Iar pentru tine, pentru sângele legamântului tau, voi da drumul robilor tai din fântâna fara apa.
Zec 9:12  La tine, fiica Sionului, se vor întoarce robii care a?teapta. Pentru zilele surghiunului tau, chiar astazi î?i vestesc: Î?i voi rasplati îndoit.
Zec 9:13  Caci am întins pe Iuda ca pe un arc ?i pe Efraim îl pun sageata pe coarda. ?i voi a?â?a pe fiii tai, Sioane, împotriva fiilor tai, Iavane (Grecia), ?i te voi preface în sabie de viteaz.
Zec 9:14  ?i Domnul Se va arata deasupra lor ?i sageata Lui va ?â?ni ca fulgerul ?i Domnul Dumnezeu va suna din trâmbi?a ?i va înainta în vijelia de la miazazi.
Zec 9:15  Domnul Savaot îi va ocroti ?i ei vor sfâ?ia ?i vor calca în picioare pietrele de pra?tie ?i vor bea sângele ca pe vin. ?i vor fi plini ca o cupa de jertfa, ca ?i coarnele jertfelnicului.
Zec 9:16  ?i în ziua aceea îi va izbavi Domnul Dumnezeul lor; ca pe o turma va pa?te El pe poporul Sau, ?i ei vor fi ca ni?te pietre de diadema, stralucind în ?ara Sa.
Zec 9:17  Ce fericire ?i ce bel?ug va fi atunci! Grâul va veseli pe flacaii lui ?i vinul pe fecioarele lui!
Zec 10:1  Cere?i de la Domnul ploaie la vreme, ploaie timpurie ?i târzie! Domnul scoate fulgerele ?i trimite ploaie; El da omului pâinea ?i dobitoacelor iarba.
Zec 10:2  Caci terafimii rostesc cuvinte de?arte, vrajitorii au vedenii mincinoase. ?i spun visuri amagitoare ?i mângâieri de?arte. Pentru aceasta ei au plecat ca o turma ?i au fost supu?i, caci n-aveau pastor.
Zec 10:3  Ci împotriva pastorilor arde mânia Mea ?i pe ?api îi voi pedepsi. Caci Domnul Savaot va cerceta turma Sa, casa lui Iuda, ?i o va face calul Sau de cinste în vreme de razboi.
Zec 10:4  Din ea va ie?i Piatra din capul unghiului, din ea ?aru?ii de corturi, din ea arcul de razboi, din ea vor ie?i toate capeteniile laolalta.
Zec 10:5  Ei vor fi ca vitejii care calca în picioare tina drumurilor în vreme de razboi ?i se vor razboi, caci Domnul va fi cu ei, iar calare?ii vor fi de ru?ine.
Zec 10:6  Eu voi întari casa lui Iuda ?i voi izbavi casa lui Iosif ?i îi voi aduce înapoi, caci Îmi este mila de ei, ?i vor fi ca ?i când nu i-am aruncat, caci Eu sunt Domnul Dumnezeul lor ?i îi voi auzi.
Zec 10:7  ?i Efraim va fi ca un viteaz. Inima lor se va bucura ca de vin ?i copiii lor vor vedea ?i se vor bucura ?i inima lor va tresalta de bucurie în Domnul.
Zec 10:8  ?i voi ?uiera dupa ei ?i îi voi strânge la un loc, caci i-am rascumparat, ?i ei se vor înmul?i neîncetat.
Zec 10:9  ?i îi voi împra?tia printre popoare, ?i din mari departari î?i vor aduce aminte de Mine ?i vor trai acolo cu fiii lor ?i vor veni înapoi.
Zec 10:10  Îi voi scoate din ?ara Egiptului ?i din Asiria îi voi aduna ?i îi voi aduce în pamântul Galaadului ?i Libanului ?i nu va fi atâta loc pentru ei.
Zec 10:11  ?i ei vor trece prin marea Egiptului, ?i el va lovi valurile marii, ?i toate adâncurile Nilului vor fi secate. Trufia Asiriei va fi doborâta ?i sceptrul Egiptului va fi scos.
Zec 10:12  ?i în Domnul va fi puterea lor ?i în numele Lui se vor lauda", zice Domnul.
Zec 11:1  Deschide, Libane, por?ile tale, ca focul sa mistuie cedrii tai!
Zec 11:2  Vaita-te, chiparosule, caci a cazut cedrul ?i cei mai falnici, s-au prabu?it la pamânt. Vaita?i-va, stejari ai Vasanului, caci a cazut padurea cea deasa.
Zec 11:3  Se aude vaietul pastorilor, caci mândrele lor pa?uni au fost pustiite; se aude ragetul puilor de leu, caci boga?ia desi?urilor Iordanului a fost pârjolita.
Zec 11:4  A?a zice Domnul Dumnezeul meu: "Pa?te oile de junghiat".
Zec 11:5  Caci cei care le cumpara le junghie ?i nu au nici o vina, iar vânzatorul lor zice: "Binecuvântat sa fie Domnul, caci, iata, m-am îmboga?it", ?i pastorii lor nu le cru?a.
Zec 11:6  "?i Eu nu Ma voi mai îndura de locuitorii ?arii", zice Domnul. "Iata ca Eu voi da pe oameni, pe fiecare în mâinile vecinului sau ?i ale regelui sau, ?i vor pustii ?ara ?i nu-i voi scapa din mâinile lor".
Zec 11:7  ?i m-am facut cioban peste oile de junghiat din pricina negu?atorilor de oi. ?i am luat doua toiege. Pe unul l-am numit "Îndurare", iar pe celalalt "Legamânt" ?i am început sa pasc turma.
Zec 11:8  ?i într-o luna am stârpit pe cei trei pastori, caci sufletul Meu prinsese scârba de ei ?i sufletul lor nu Ma mai putea suferi pe Mine.
Zec 11:9  ?i am grait: "Nu va mai pasc! Cea care este de murit sa moara, cea de pierit sa piara, iar oile care vor mai ramâne sa se sfâ?ie între ele!"
Zec 11:10  Atunci am luat toiagul Meu "Îndurare" ?i l-am frânt, ca sa stric legamântul pe care l-am încheiat cu toate popoarele.
Zec 11:11  El a fost stricat în acea zi, iar negu?atorii de oi, care luau seama la Mine au în?eles ca acesta a fost cuvântul Domnului.
Zec 11:12  ?i le-am zis: "Daca socoti?i cu cale, da?i-Mi simbria, iar daca nu, sa nu Mi-o plati?i. ?i Mi-au cântarit simbria Mea treizeci de argin?i.
Zec 11:13  Atunci a grait Domnul catre Mine: "Arunca-l olarului pre?ul acela scump cu care Eu am fost pre?uit de ei". ?i am luat cei treizeci de argin?i ?i i-am aruncat în vistieria templului Domnului, pentru olar.
Zec 11:14  Apoi am rupt ?i toiagul cel de-al doilea, "Legamânt", ca sa stric fra?ia dintre Iuda ?i Israel.
Zec 11:15  ?i a zis Domnul catre mine: "Ia lucrurile unui pastor nebun;
Zec 11:16  Caci, iata, Eu voi ridica un pastor nebun în aceasta ?ara care nu va umbla dupa oaia cea pierduta ?i care nu va cauta pe cea ratacita ?i pe cea ranita nu o va vindeca ?i nu va hrani pe cea sanatoasa, ci va mânca pe cea grasa ?i îi va smulge unghiile.
Zec 11:17  Vai de ciobanul rau, care lasa turma în parasire! Sabia sa loveasca bra?ul lui ?i ochiul lui cel drept! Bra?ul lui sa se usuce cu totul, iar ochiul sa ramâna orb de tot!"
Zec 12:1  Proorocie. Cuvântul Domnului asupra lui Israel. A?a graie?te Domnul, Care întinde cerurile ca un cort, Care pune temelie pamântului ?i Care zide?te duhul omului înauntrul sau:
Zec 12:2  "Iata, voi face din Ierusalim cupa de ame?ire pentru toate popoarele din jur; tot a?a va fi ?i pentru Iuda, când Ierusalimul va fi împresurat.
Zec 12:3  ?i în ziua aceea voi preface Ierusalimul în piatra de povara pentru toate popoarele. To?i care o vor ridica se vor rani grav ?i se vor aduna împotriva lui toate neamurile pamântului.
Zec 12:4  În ziua aceea", zice Domnul, "voi lovi to?i caii cu spaima ?i pe calare?i cu nebunie; ?i voi deschide ochii Mei asupra lui Iuda ?i pe to?i capii popoarelor îi voi lovi cu orbire.
Zec 12:5  ?i vor grai în inima lor capeteniile lui Iuda: "Puterea pentru locuitorii Ierusalimului este în Domnul Savaot, Dumnezeul lor!"
Zec 12:6  În ziua aceea voi face pe conducatorii lui Iuda ca un vas cu jeratic în mijlocul lemnelor ?i o tor?a aprinsa într-un stog de snopi; ?i ei vor mistui, la dreapta ?i la stânga, toate popoarele dimprejur, ?i Ierusalimul va fi locuit ?i mai departe pe locul lui, în Ierusalim.
Zec 12:7  ?i Domnul va izbavi mai întâi corturile lui Iuda, ca semin?ia casei lui David ?i trufia locuitorilor Ierusalimului sa nu se ridice deasupra lui Iuda.
Zec 12:8  În ziua aceea Domnul va întinde ocrotirea Sa asupra celor ce locuiesc în Ierusalim, încât cel mai slab între ei sa fie ca David, ?i casa lui David sa fie ca Însu?i Dumnezeu, ca îngerul Domnului care merge în fruntea lor.
Zec 12:9  ?i în ziua aceea Ma voi sârgui sa pierd toate neamurile care vor veni împotriva Ierusalimului.
Zec 12:10  Atunci voi varsa peste casa lui David ?i peste locuitorii Ierusalimului duh de milostivire ?i de rugaciune, ?i î?i vor a?inti privirile înspre Mine, pe Care ei L-au strapuns ?i vor face plângere asupra Lui, cum se face pentru un fiu unul nascut ?i-L vor jeli ca pe cel întâi nascut.
Zec 12:11  În ziua aceea, va fi plângere mare în Ierusalim, ca plângerea de la Hadad-Rimon, în câmpia Meghidonului.
Zec 12:12  ?ara se va tângui, fiecare familie deosebit: familia casei lui David deosebit ?i femeile ei deosebit; familia casei lui Natan deosebit ?i femeile ei deosebit;
Zec 12:13  Familia casei lui Levi deosebit ?i femeile ei deosebit; familia lui ?imei deosebit ?i femeile ei deosebit;
Zec 12:14  Toate familiile care ramân, fiecare pentru sine ?i femeile lor pentru sine.
Zec 13:1  În vremea aceea va fi un izvor cu apa curgatoare pentru casa lui David ?i pentru locuitorii Ierusalimului, pentru cura?irea de pacat ?i de orice alta întinare.
Zec 13:2  ?i în ziua aceea", zice Domnul Savaot, "voi stârpi din ?ara numele idolilor, ca nimeni sa nu-i mai pomeneasca; de asemenea voi da afara din ?ara pe proorocii lor ?i duhul cel spurcat.
Zec 13:3  ?i daca va mai profe?i cineva, vor zice catre el tatal sau ?i mama sa, care l-au nascut: "Tu nu vei trai, caci ai grait minciuna în numele Domnului!" ?i atunci tatal sau ?i mama sa îl vor strapunge în clipa când va prooroci.
Zec 13:4  ?i în ziua aceea se vor ru?ina proorocii, fiecare de vedenia lui, când va grai ca un prooroc, ?i nu se vor mai îmbraca cu mantie de par ca sa spuna minciuni.
Zec 13:5  ?i fiecare va zice: "Nu sunt prooroc, ci plugar, caci cu plugaria m-am îndeletnicit din tinere?ile mele".
Zec 13:6  ?i daca va fi întrebat: "De unde ai ranile acestea la mâini?" El va raspunde: "Am fost lovit în casa prietenilor mei!"
Zec 13:7  Sabie, de?teapta-te împotriva pastorului Meu, împotriva tovara?ului Meu, zice Domnul Savaot. Voi bate pastorul ?i se vor risipi oile, ?i Îmi voi întoarce mina Mea împotriva celor mici.
Zec 13:8  ?i în toata ?ara", zice Domnul, "doua treimi vor pieri, ?i vor muri, iar cealalta treime va fi lasata acolo.
Zec 13:9  Iar pe aceasta a treia o voi pune în foc; îi voi cura?i ca pe argint ?i îi voi încerca cum se încearca aurul. Ei vor chema numele Meu ?i Eu îi voi asculta; ?i Eu voi zice: "Acesta este poporul Meu", ?i el va raspunde: "Domnul este Dumnezeul meu!"
Zec 14:1  Iata ca vine ziua Domnului, când se vor împar?i prazile tale în mijlocul tau.
Zec 14:2  ?i voi aduna toate neamurile pentru razboi împotriva Ierusalimului, ?i cetatea va fi luata, casele vor fi jefuite ?i femeile necinstite. Atunci jumatate din cetate va fi dusa în robie, iar restul poporului Meu nu va fi stârpit din cetate.
Zec 14:3  Atunci Domnul va ie?i la lupta ?i se va razboi împotriva acestor popoare, ca în vreme de lupta, ca în vreme de razboi.
Zec 14:4  ?i în vremea aceea se vor sprijini picioarele Lui pe Muntele Maslinilor, care este în fa?a Ierusalimului, la rasarit; iar Muntele Maslinilor se va crapa în doua de la rasarit la apus ?i se va face o vale foarte mare ?i jumatate din munte se va da înapoi catre miazanoapte ?i cealalta jumatate catre miazazi.
Zec 14:5  ?i voi ve?i alerga prin valea mun?ilor Mei, caci valea mun?ilor se va întinde pâna la locul unde Eu voi da izbavire. ?i ve?i alerga cum a?i alergat de frica cutremurului, în vremea domniei lui Ozia, regele lui Iuda. Atunci va veni Domnul Dumnezeul meu ?i to?i sfin?ii împreuna cu El.
Zec 14:6  În vremea aceea nu va mai fi lumina, ci frig ?i ger.
Zec 14:7  Va fi o zi deosebita pe care Domnul singur o ?tie; nu va fi nici zi ?i nici noapte, ci în vremea serii va fi lumina.
Zec 14:8  Iar în ziua aceea vor izvorî din Ierusalim ape vii: jumatate se vor îndrepta catre marea cea de la rasarit, iar cealalta jumatate catre marea cea de la apus. ?i a?a va fi ?i vara ?i iarna.
Zec 14:9  ?i va fi Domnul Împarat peste tot pamântul; în vremea aceea va fi Domnul unul singur ?i tot a?a ?i numele Sau unul singur.
Zec 14:10  ?ara întreaga va fi prefacuta în câmpie de la Gheba ?i pâna la Rimon, spre miazazi de Ierusalim. ?i Ierusalimul va fi înal?at ?i va fi locuit pe locul unde se afla el, de la poarta lui Veniamin ?i pâna la locul por?ii dintâi, adica pâna la poarta din col?, ?i de la turnul lui Hananeel pâna la teascurile regelui.
Zec 14:11  ?i vor locui în el, ?i nu va mai fi acolo blestem, ci Ierusalimul va fi locuit în siguran?a.
Zec 14:12  Dar iata care va fi prapadul cu care Domnul va lovi toate popoarele care s-au razboit cu Ierusalimul: trupul du?manului va putrezi stând în picioare, ochii în orbitele lor ?i limba în gura.
Zec 14:13  În ziua aceea va fi de la Domnul mare spaima printre ei ?i fiecare va apuca de mâna pe aproapele sau ?i î?i vor ridica mâna unii împotriva altora.
Zec 14:14  ?i Iuda va lupta împotriva Ierusalimului, ?i boga?iile tuturor neamurilor din jur, aur, argint ?i ve?minte fara numar vor fi strânse la un loc.
Zec 14:15  ?i la fel cu acest prapad va fi prapadul care va lovi calul, catârul, camila, asinul ?i toate dobitoacele care vor fi în acele tabere.
Zec 14:16  ?i to?i cei care vor fi ramas cu via?a dintre neamurile acelea care venisera sa lupte împotriva Ierusalimului se vor sui în fiecare an sa se închine Împaratului, Domnul Savaot, ?i sa praznuiasca sarbatoarea corturilor.
Zec 14:17  Iar cele din neamurile pamântului care nu se vor sui sa se închine în Ierusalim, Împaratului, Domnul Savaot, nu vor avea parte de ploaie.
Zec 14:18  ?i daca neamul Egiptului nu se va sui ?i nu va veni, nu numai ca ei nu vor avea parte de ploaie, ci va veni peste ei prapadul cu care Domnul va lovi neamurile care nu se vor sui sa praznuiasca sarbatoarea corturilor.
Zec 14:19  Aceasta va fi pedeapsa Egiptului ?i pedeapsa tuturor neamurilor care nu se vor sui în Ierusalim sa praznuiasca sarbatoarea corturilor.
Zec 14:20  În ziua aceea va fi (scris) pe clopo?eii cailor: "Sfânt lui Dumnezeu!" ?i vor fi caldarile în templul Domnului ca nastrapele înaintea jertfelnicului.
Zec 14:21  ?i orice caldare din Ierusalim ?i din Iuda va fi sfânta Domnului Savaot. ?i vor veni to?i cei ce vor sa jertfeasca ?i le vor lua ?i vor fierbe carne ?i nu va mai fi nici un negu?ator, în ziua aceea, în templul Domnului Savaot.


\end{document}