\begin{document}

\title{Zaharia}


\chapter{1}

\par 1 În luna a opta, în anul al doilea al lui Darius, fost-a cuvântul Domnului către proorocul Zaharia, fiul lui Berechia, fiul lui Ido, zicând:
\par 2 "Domnul S-a mâniat mult împotriva părinților voștri!"
\par 3 Și spune-le celor rămași din popor: "Așa zice Domnul Savaot: Întoarceți-vă către Mine", zice Domnul Savaot, "și atunci Mă voi întoarce și Eu către voi", zice Domnul Savaot.
\par 4 "Nu fiți ca părinții voștri, cărora le-au propovăduit proorocii de mai înainte, strigând: Așa zice Domnul Savaot: Întoarceți-vă din căile voastre cele rele și de la faptele voastre cele rele! Dar ei n-au ascultat și nu au voit să ia aminte la Mine", zice Domnul.
\par 5 "Unde sunt părinții voștri? Și oare profeții trăiesc ei încă?
\par 6 Dar cuvintele Mele și poruncile Mele pe care le-am dat slujitorilor Mei prooroci ca să le vestească n-au ajuns ele la părinții voștri? Domnul S-a mâniat mult împotriva părinților voștri, încât ei s-au pocăit și au mărturisit: Domnul Savaot ne-a răsplătit nouă după căile și după faptele noastre, așa cum hotărâse să facă".
\par 7 În ziua a douăzeci și patra din luna a doua, care se cheamă Șebat, în anul al doilea al lui Darius, a fost cuvântul Domnului către proorocul Zaharia, fiul lui Berechia, fiul lui Ido, zicând:
\par 8 "Am avut o vedenie în timpul nopții, și iată un om călare pe un cal roșu; și stătea între mirți, într-un loc umbros, și în urma lui cai roibi, murgi și albi.
\par 9 Și am zis: "Cine sunt aceștia, domnul meu?" Și mira răspuns atunci îngerul oare grăia cu mine: "Îți voi arăta acum cine sunt aceștia".
\par 10 Și omul care stătea între mirți a răspuns: "Aceștia sunt solii pe care i-a trimis Domnul ca să cutreiere pământul!"
\par 11 Și ei au răspuns către îngerul Domnului care stătea între mirți și au grăit: "Am cutreierat pământul și iată tot pământul este locuit și liniștit!"
\par 12 Și a răspuns îngerul Domnului și a zis: "Doamne Savaot, până când vei întârzia să arăți milă Ierusalimului și cetăților lui Iuda, pe care le faci să simtă mânia Ta de șaptezeci de ani?"
\par 13 Și îngerului care vorbea cu mine, Domnul i-a răspuns cu cuvinte de mângâiere.
\par 14 Și a grăit către mine îngerul care vorbea cu mine: "Vestește aceasta: "Așa zice Domnul Savaot: Sunt plin de zel față de Ierusalim și față de Sion;
\par 15 Dar este puternică mânia Mea împotriva neamurilor trufașe! Căci Eu Mă întărâtasem doar puțin, dar ele s-au înverșunat în rele".
\par 16 Pentru aceasta, așa zice Domnul: "Mă întorc iarăși către Ierusalim cu milostivire; templul Meu va fi zidit în el", zice Domnul Savaot, "și funia de măsurat se va întinde peste Ierusalim!"
\par 17 Și vestește încă și aceasta: "Așa zice Domnul Savaot: Încă o dată cetățile Mele vor avea belșug de bunuri, iar Domnul Își va revărsa din nou îndurarea Sa peste Sion și va alege iarăși Ierusalimul".
\par 18 Și am ridicat ochii mei și am privit, și iată patru coarne.
\par 19 Atunci am zis îngerului care grăia cu mine: "Ce sunt acestea?" Și el mi-a răspuns: "Acestea sunt coarnele care au împrăștiat pe Iuda, pe Israel și Ierusalimul".
\par 20 Apoi Domnul m-a făcut să văd patru făurari.
\par 21 Și am întrebat: "Ce vor să facă aceștia?" Și El a răspuns: "Acestea sunt coarnele care au risipit pe Iuda, încât nimeni n-a îndrăznit să ridice capul. Iar ei au venit ca să doboare la pământ coarnele acelor neamuri care și-au ridicat cornul împotriva țării lui Iuda, pentru a o risipi".

\chapter{2}

\par 1 Și am ridicat ochii, m-am uitat și iată un om cu o funie de măsurat în mână. Și i-am zis: "Încotro ai pornit?"
\par 2 Și el mi-a răspuns: "Să măsor Ierusalimul, ca să văd care este lungimea și lățimea lui!"
\par 3 Și iată că s-a arătat îngerul care grăia cu mine și alt înger a ieșit în întâmpinarea lui.
\par 4 Și ița grăit, zicând: "Aleargă și spune tânărului acestuia: Ierusalimul va fi locuit ca un oraș deschis, atât de mare va fi mulțimea de oameni și de dobitoace înăuntrul lui.
\par 5 Și Eu voi fi pentru el un zid de foc de jur împrejurul lui", zice Domnul, "și voi fi slava lui în mijlocul lui".
\par 6 "Sculați, sculați și fugiți din țara de la miazănoapte", zice Domnul, "că v-am împrăștiat în cele patru vânturi ale cerului;
\par 7 Fugi, Sioane, și te izbăvește, tu care locuiești la fiica Babilonului.
\par 8 Căci așa zice Domnul Savaot: "Pentru slava Sa, El m-a trimis la neamurile care v-au jefuit pe voi; căci cel care se atinge de voi se atinge de lumina ochiului Lui!
\par 9 Căci iată că Eu rotesc mâna Mea peste ei și ei vor fi pradă pentru cei care au fost robii lor, ca să vă dați seama că Domnul Savaot m-a trimis.
\par 10 Bucură-te și te veselește, fiica Sionului, căci iată Eu vin să locuiesc în mijlocul tău", zice Domnul.
\par 11 "Și multe neamuri se vor alipi de Domnul în ziua aceea și Îmi vor fi Mie popor și voi locui în mijlocul tău, ca să știi că Domnul Savaot m-a trimis la tine.
\par 12 Și va lua Domnul ca moștenire a Sa pe Iuda, în țara cea sfântă, și va alege încă o dată Ierusalimul.
\par 13 Să tacă tot trupul înaintea Domnului, căci El S-a ridicat din locașul Său cel sfânt".

\chapter{3}

\par 1 Și mi-a arătat pe Iosua, marele preot, stând înaintea îngerului Domnului, și pe Satana, stând la dreapta lui ca să-l învinuiască.
\par 2 Și a zis Domnul către Satana: "Ceartă-te pe tine Domnul, diavole, ceartă-te pe tine Domnul, Cel care a ales Ierusalimul! Acesta nu este el, oare, un tăciune scos din foc?"
\par 3 Și era Iosua îmbrăcat în veșminte murdare și stătea înaintea îngerului.
\par 4 Și a răspuns și a zis celor care stăteau înaintea lui: "Dezbrăcați-l de veșmintele cele murdare!" Și i-a zis lui: "Iată ți-am iertat fărădelegile și te-am îmbrăcat cu veșmânt de prăznuire!"
\par 5 Și a mai zis: "Puneți mitră curată pe capul lui!" Și ei i-au pus mitră curată pe cap și l-au înveșmântat, iar îngerul Domnului sta de față.
\par 6 Și îngerul Domnului i-a hotărât lui Iosua astfel:
\par 7 "Așa zice Domnul Savaot: Dacă vei umbla în căile Mele și de vei păzi poruncile Mele, atunci tu vei cârmui casa Mea și vei păzi curțile Mele, și Eu îți voi da ție dregătorie printre slujitorii Mei de aici.
\par 8 Ascultă deci, Iosua, mare preot, tu și cei împreună cu tine care stau înaintea ta; căci ei sunt oameni de prezicere. Iată Eu aduc pe Servul Meu Odraslă.
\par 9 Iată piatra pe care am pus-o înaintea lui Iosua; pe această piatră sunt șapte ochi: iată că Eu voi săpa sculptura ei", zice Domnul Savaot, "și într-o singură zi voi îndepărta nedreptatea din țara aceasta.
\par 10 În ziua aceea", zice Domnul Savaot, "fiecare din voi va pofti pe aproapele său sub vița și sub smochinul său".

\chapter{4}

\par 1 Și s-a întors îngerul care grăia cu mine și m-a deșteptat, ca pe un om pe care îl trezești din somn.
\par 2 și mi-a zis: "Ce vezi?" și am zis: "Iată văd un candelabru cu totul de aur, cu șapte candele, iar deasupra candelabrului este un vas cu untdelemn din care pornesc șapte țevi către cele șapte candele;
\par 3 Iar alături, doi măslini, unul de-a dreapta vasului cu untdelemn și altul de-a stânga".
\par 4 Și am întrebat și am zis către îngerul care vorbea cu mine: "Ce sunt astea, domnul meu?"
\par 5 Și mi-a răspuns îngerul care grăia cu mine și mi-a zis: "Oare nu știi ce sunt toate acestea?" Și am zis: "Nu, domnul meu!"
\par 6 Și mi-a vorbit iar și a zis: "Acesta este cuvântul Domnului către Zorobabel: Nu prin putere, nici prin tărie, ci prin Duhul Meu" - zice Domnul Savaot.
\par 7 "Ce ești tu, munte înalt? Înaintea lui Zorobabel devii o câmpie. Și el va smulge piatra din vârf în strigătele mulțimii: "Har, har, peste ea!"
\par 8 Și a fost cuvântul Domnului către mine și mi-a zis:
\par 9 "Mâinile lui Zorobabel au pus temelia acestui templu și tot mâinile lui îl vor termina, și tu vei ști că Domnul Savaot m-a trimis la voi.
\par 10 Căci cine a disprețuit vremea acestor începuturi mici? Ei se vor bucura văzând cumpăna zidarului în mâna lui Zorobabel. Iar aceste șapte (candele) sunt ochii Domnului care cutreieră tot pământul".
\par 11 Și mi-am luat îndemnul și am zis către el: "Ce înseamnă acești măslini unul de-a dreapta și altul de-a stânga candelabrului?"
\par 12 Și l-am întrebat și a doua oară: "Ce înseamnă cele două crengi de măslin, care sunt lângă cele două țevi de aur prin care se lasă în jos untdelemnul?"
\par 13 Și mi-a grăit, zicând: "Oare nu știi ce înseamnă acestea?" Și am răspuns: "Nu, domnul meu".
\par 14 Și m-a lămurit: "Aceștia sunt cei doi fii unși, care stau înaintea Domnului a tot pământul".

\chapter{5}

\par 1 Și iarăși am ridicat ochii și am privit și iată un sul de carte care se înălța în zbor.
\par 2 și a grăit către mine, zicând: "Ce vezi tu?" Și am zis: "Văd un sul de carte care zboară, lung de douăzeci de coți și lat de zece".
\par 3 Și el mi-a tâlcuit: "Acesta este blestemul care se răspândește peste fața a tot pământul, căci orice fur va fi nimicit de aici și orice am care jură strâmb va fi pierdut.
\par 4 I-am dat drumul", zice Domnul Savaot, "și se va duce în casa furului și în casa celui care jură strâmb pe numele Meu și va rămâne în ea și va nimici lemnele și pietrele ei".
\par 5 Și îngerul care grăia cu mine a ieșit la iveală și mi-a zis: "Ridică ochii tăi și vezi: Ce este arătarea aceasta?"
\par 6 Atunci am grăit: "Ce este aceasta?" Și el mi-a răspuns: "Este efa care iese la iveală". Și a spus mai departe: "În ea se află fărădelegea a tot pământul!"
\par 7 Și iată că s-a ridicat un disc de plumb, iar o femeie stătea în mijlocul efei.
\par 8 Și el a tâlcuit: "Aceasta este fărădelegea!" Și el a aruncat-o în efă și a răsturnat lespedea de plumb deasupra ei.
\par 9 Și am ridicat ochii mei și am privit și iată că au ieșit două femei. Și vântul bătea în aripile lor, iar aripile lor erau ca de barză. Și ele au ridicat efa intre pământ și cer.
\par 10 Și am zis către îngerul oare grăia cu mine: "Încotro duc ele efa?"
\par 11 Atunci el mi-a răspuns: "Ele merg să-i zidească o casă în pământul Șinear și Acad și să o așeze acolo la locul ei".

\chapter{6}

\par 1 Și am ridicat iarăși ochii mei și m-am uitat. Și iată că ieșeau patru care dintre doi munți și munții erau de aramă.
\par 2 La carul cel dintâi erau înhămați cai roșii, iar la carul cel de-al doilea erau înhămați cai negri.
\par 3 La cel de-al treilea car erau înhămați cai albi, iar la cel de-al patrulea, cai bălțați, puternici.
\par 4 Și mi-am luat îndemnul și am zis către îngerul care grăia cu mine: "Ce sunt acestea, domnul meu?"
\par 5 Atunci mi-a răspuns îngerul și mi-a zis: "Acestea sunt cele patru vânturi ale cerului, care ies după ce s-au înfățișat înaintea Stăpânului a tot pământul.
\par 6 Caii roșii înaintează spre țara de la răsărit; cei negri înaintează spre țara de la miazănoapte; cei albi înaintează spre țara de la apus, și cei bălțați înaintează spre țara de la miazăzi.
\par 7 Puternici, ei înaintau nerăbdători să străbată pământul. El le-a zis: "Plecați și cutreierați pământul!" Și ei au cutreierat pământul.
\par 8 Și a strigat către mine și mi-a grăit, zicând: "Iată, cei ce se îndreaptă către ținutul cel de miazănoapte au potolit Duhul Meu în țara de la miazănoapte".
\par 9 Și a fost cuvântul Domnului către mine și mi-a zis:
\par 10 "Tu vei primi darurile celor întorși din robie, de la Heldai, Tobia și Iedaia, și în aceeași zi mergi în casa lui Ioșia, fiul lui Sofonie, care a sosit din Babilon.
\par 11 Și vei lua argint și aur și vei face o cunună și o vei pune pe capul lui Iosua marele preot, feciorul lui Ioțadoc.
\par 12 Și-i vei zice lui în acest chip: "Așa grăiește Domnul Savaot: Iată un om care va fi chemat Odraslă; acesta va odrăsli și va zidi templul Domnului.
\par 13 Și acesta va rezidi templul Domnului. El va purta semnele regale și va stăpâni și va domni pe tronul lui și un preot va fi la dreapta lui. Între ei doi va fi o pace desăvârșită.
\par 14 Și cununa va fi pentru Heldai, Tobia și Iedaia și pentru Ioșia, fiul lui Sofonie, ca aducere aminte în templul Domnului.
\par 15 Și oameni de la mari depărtări vor veni și vor zidi la templul Domnului, și atunci voi veți ști că Domnul Savaot m-a trimis către voi. Și aceasta se va întâmpla, dacă voi veți asculta cu credincioșie de glasul Domnului Dumnezeului vostru!"

\chapter{7}

\par 1 Și a fost, în anul al patrulea al regelui Darius, cuvântul Domnului către Zaharia în ziua a patra a lunii a noua, care se cheamă Chislev.
\par 2 Casa lui Israel a trimis pe Șarețer, dregătorul cel mare al oștirii regelui, cu oameni, ca să câștige milostivirea Domnului
\par 3 Să grăiască preoților din templul Domnului Savaot și proorocilor într-acest chip: "Să mai plâng eu oare în luna a cincea și să mă înfrânez precum am făcut atâția ani?"
\par 4 Și a fost cuvântul Domnului Savaot către mine și mi-a zis:
\par 5 "Vestește la tot poporul țării și preoților: "Dacă ați ținut post și v-ați tânguit în luna a cincea și într-a șaptea, vreme de șaptezeci de ani, pentru Mine, oare, ați postit voi?
\par 6 Și dacă mâncați și beți, oare nu sunteți voi aceia care mâncați și care beți?
\par 7 Nu cunoașteți voi cuvintele pe care le-a vestit Domnul prin graiul profeților de mai înainte, când Ierusalimul era locuit și pașnic, cu cetățile sale dimprejur, și când Neghebul și Șefela erau locuite?"
\par 8 Și a fost cuvântul Domnului către Zaharia într-acest chip:
\par 9 "Așa grăiește Domnul Savaot: "Faceți dreptate adevărată și purtați-vă fiecare cu bunătate și îndurare față de fratele său;
\par 10 Nu apăsați pe văduvă, pe orfan, pe străin și pe cel sărman și nimeni să nu pună la cale fărădelegi în inima lui împotriva fratelui său!"
\par 11 Dar ei n-au voit să ia aminte, ci au întors spatele cu îndărătnicie și și-au astupat urechile ca să nu audă;
\par 12 Și și-au învârtoșat inima ca diamantul, ca să nu asculte legea și cuvintele pe care le-a trimis Domnul Savaot prin Duhul Lui, prin graiul proorocilor celor de altădată. Și a fost o mânie mare de la Domnul Savaot.
\par 13 Și s-a întâmplat că după cum El a strigat și ei n-au auzit, tot așa "ei vor striga și Eu nu-i voi auzi", zice Domnul Savaot;
\par 14 "Și îi voi împrăștia printre toate neamurile pe care ei nu le cunosc!" Și țara a fost pustiită după aceea și n-a mai rămas nimeni în ea, fiind prefăcută țara cea plăcută în pustiu.

\chapter{8}

\par 1 Și a fost cuvântul Domnului Savaot, zicând:
\par 2 "Așa grăiește Domnul Savaot: Iubesc Sionul cu zel mare și cu patimă aprinsă îl doresc".
\par 3 Așa grăiește Domnul: "M-am întors cu milostivire către Sion și voi locui în mijlocul Ierusalimului; și Ierusalimul se va chema cetate credincioasă și muntele Domnului Savant, munte sfânt".
\par 4 Așa zice Domnul Savaot: "Bătrâni și bătrâne vor ședea iarăși în piețele Ierusalimului, fiecare cu toiagul în mână, din pricina vârstei lor înaintate.
\par 5 Și se vor umple piețele de băieți și de fete, care se vor juca în piețele lui".
\par 6 Așa zice Domnul Savaot: "Dacă acest lucru se va părea greu de făcut înaintea ochilor restului acestui popor în zilele acelea, care va fi și înaintea ochilor Mei cu neputință?" zice Domnul Savaot.
\par 7 Așa grăiește Domnul Savaot: "Iată că Eu voi izbăvi pe poporul Meu din țările de la răsărit și din țările de la apus.
\par 8 Și îi voi aduce înapoi și ei vor locui în mijlocul Ierusalimului și ei Îmi vor fi Mie popor, iar Eu le voi fi Dumnezeu, în credincioșie și în dreptate".
\par 9 Așa grăiește Domnul Savaot: "Întăriți-vă mâinile voastre, voi care auziți în zilele acestea cuvintele din gura proorocilor din vremea în care s-a pus temelia templului Domnului Savaot, ca să se zidească templul!
\par 10 Căci înainte vreme nu se răsplătea nici omul, nici dobitocul, și cel care intra și cel ce ieșea n-aveau tihnă din pricina dușmanilor, și Eu pornisem pe toți oamenii, unii împotriva altora;
\par 11 Dar acum nu mai sunt ca înainte vreme față de restul poporului Meu", zice Domnul Savaot.
\par 12 "Ci acum se fac semănături în bună pace! Via își va da rodul ei și pământul va da roadele lui, cerul va lăsa să pice rouă, iar Eu voi da în stăpânire, restului acestui popor, toate bunătățile acestea.
\par 13 Și se va întâmpla că, precum ați fost un blestem printre neamuri, tot așa, o, voi, casa lui Iuda și rața lui Israel, vă voi izbăvi pe voi, și veri fi o binecuvântare. Nu vă temeți, ci întăriți-vă mâinile!"
\par 14 Căci așa zice Domnul Savaot: "Precum M-am hotărât să vă pedepsesc, când aii întărâtat mânia Mea", zice Domnul Savaot, "și nu Mi-a fost milă de voi,
\par 15 Tot astfel M-am gândit și am să fac bine Ierusalimului în aceste zile și casei lui Iuda. Nu vă temeți!"
\par 16 Iată rânduielile pe care trebuie să le păziți: "Să spună omul adevărat aproapelui său. Judecați și dați hotărâri drepte la porțile voastre;
\par 17 Să nu cugetați fărădelege unul împotriva altuia și jurământul strâmb să nu-l iubiți, căci toate acestea le urăsc", zice Domnul.
\par 18 Și a fost cuvântul Domnului Savaot către mine zicând:
\par 19 "Așa zice Domnul Savaot: Postul din luna a patra, a cincea, a șaptea și a zecea vor fi pentru casa lui Iuda spre veselie și bucurie și zile bune de sărbătoare! Dar iubiți adevărul și pacea!
\par 20 Așa zice Domnul Savaot: "Și vor veni încă popoare și locuitori din cetăți numeroase;
\par 21 Și locuitorii dintr-o cetate vor merge în cealaltă, zicând: "Să mergem să dobândim îndurarea lui Iuda și să căutăm pe Domnul Savaot!"Merg și eu!"
\par 22 Și vor veni popoare multe și neamuri puternice ca să caute pe Domnul Savaot în Ierusalim și să se roage Domnului".
\par 23 Așa grăiește Domnul Savaot: "Și în zilele acelea zece oameni dintre limbile neamurilor vor apuca pe un iudeu de poala hainei și vor zice: Mergem și noi cu tine, căci am aflat că Dumnezeu este cu voi!"

\chapter{9}

\par 1 Proorocie. Cuvântul Domnului împotriva țării Hadrac și către Damasc, cetatea sa, căci Domnul are ochii Săi asupra oamenilor și asupra tuturor semințiilor lui Israel;
\par 2 De asemenea și către Hamat, care se învecinează cu Damascul, către Tir și către Sidon, cu toată înțelepciunea lor.
\par 3 Tirul și-a zidit întărituri și a strâns argint ca pulberea și aur ca tina de pe ulițe.
\par 4 Dar iată că Domnul îl va cuceri și va prăvăli în mare puterea lui și el va fi mistuit prin foc.
\par 5 Ascalonul va vedea și se va înspăimânta, Gaza va fi cuprinsă de dureri năprasnice și Ecronul la fel, căci nădejdea lui s-a prăbușit. Și nu va mai fi rege în Gaza și Ascalonul pustiu va rămâne nelocuit.
\par 6 Cel de alt neam va locui în Așdod! Și trufia filisteanului o voi nimici,
\par 7 Și îi voi scoate sângele din gură și urâciunile dintre dinții lui. Va fi și el un rest pentru Dumnezeul nostru și va fi ca o familie în Iuda, iar Ecronul va fi ca Iebusitul.
\par 8 Și Mă voi așeza ca strajă împrejurul Casei Mele, ca nimeni dintre cei ce vin și pleacă și ca nici un asupritor să nu mai vină asupra ei, căci Eu văd acum aceasta cu ochii Mei.
\par 9 Bucură-te foarte, fiica Sionului, veselește-te, fiica Ierusalimului, căci iată Împăratul tău vine la tine drept și biruitor; smerit și călare pe asin, pe mânzul asinei.
\par 10 El va nimici carele din Efraim, caii din Ierusalim și arcul de război va fi frânt. El va vesti pacea popoarelor și împărăția Lui se va întinde de la o mare până la cealaltă mare și de la Eufrat până la marginile pământului.
\par 11 Iar pentru tine, pentru sângele legământului tău, voi da drumul robilor tăi din fântâna fără apă.
\par 12 La tine, fiica Sionului, se vor întoarce robii care așteaptă. Pentru zilele surghiunului tău, chiar astăzi îți vestesc: Îți voi răsplăti îndoit.
\par 13 Căci am întins pe Iuda ca pe un arc și pe Efraim îl pun săgeată pe coardă. Și voi ațâța pe fiii tăi, Sioane, împotriva fiilor tăi, Iavane (Grecia), și te voi preface în sabie de viteaz.
\par 14 Și Domnul Se va arăta deasupra lor și săgeata Lui va țâșni ca fulgerul și Domnul Dumnezeu va suna din trâmbiță și va înainta în vijelia de la miazăzi.
\par 15 Domnul Savaot îi va ocroti și ei vor sfâșia și vor călca în picioare pietrele de praștie și vor bea sângele ca pe vin. Și vor fi plini ca o cupă de jertfă, ca și coarnele jertfelnicului.
\par 16 Și în ziua aceea îi va izbăvi Domnul Dumnezeul lor; ca pe o turmă va paște El pe poporul Său, și ei vor fi ca niște pietre de diademă, strălucind în țara Sa.
\par 17 Ce fericire și ce belșug va fi atunci! Grâul va veseli pe flăcăii lui și vinul pe fecioarele lui!

\chapter{10}

\par 1 Cereți de la Domnul ploaie la vreme, ploaie timpurie și târzie! Domnul scoate fulgerele și trimite ploaie; El dă omului pâinea și dobitoacelor iarba.
\par 2 Căci terafimii rostesc cuvinte deșarte, vrăjitorii au vedenii mincinoase. și spun visuri amăgitoare și mângâieri deșarte. Pentru aceasta ei au plecat ca o turmă și au fost supuși, căci n-aveau păstor.
\par 3 Ci împotriva păstorilor arde mânia Mea și pe țapi îi voi pedepsi. Căci Domnul Savaot va cerceta turma Sa, casa lui Iuda, și o va face calul Său de cinste în vreme de război.
\par 4 Din ea va ieși Piatra din capul unghiului, din ea țărușii de corturi, din ea arcul de război, din ea vor ieși toate căpeteniile laolaltă.
\par 5 Ei vor fi ca vitejii care calcă în picioare tina drumurilor în vreme de război și se vor război, căci Domnul va fi cu ei, iar călăreții vor fi de rușine.
\par 6 Eu voi întări casa lui Iuda și voi izbăvi casa lui Iosif și îi voi aduce înapoi, căci Îmi este milă de ei, și vor fi ca și când nu i-am aruncat, căci Eu sunt Domnul Dumnezeul lor și îi voi auzi.
\par 7 Și Efraim va fi ca un viteaz. Inima lor se va bucura ca de vin și copiii lor vor vedea și se vor bucura și inima lor va tresălta de bucurie în Domnul.
\par 8 Și voi șuiera după ei și îi voi strânge la un loc, căci i-am răscumpărat, și ei se vor înmulți neîncetat.
\par 9 Și îi voi împrăștia printre popoare, și din mari depărtări își vor aduce aminte de Mine și vor trăi acolo cu fiii lor și vor veni înapoi.
\par 10 Îi voi scoate din țara Egiptului și din Asiria îi voi aduna și îi voi aduce în pământul Galaadului și Libanului și nu va fi atâta loc pentru ei.
\par 11 Și ei vor trece prin marea Egiptului, și el va lovi valurile mării, și toate adâncurile Nilului vor fi secate. Trufia Asiriei va fi doborâtă și sceptrul Egiptului va fi scos.
\par 12 Și în Domnul va fi puterea lor și în numele Lui se vor lăuda", zice Domnul.

\chapter{11}

\par 1 Deschide, Libane, porțile tale, ca focul să mistuie cedrii tăi!
\par 2 Vaită-te, chiparosule, căci a căzut cedrul și cei mai falnici, s-au prăbușit la pământ. Văitați-vă, stejari ai Vasanului, căci a căzut pădurea cea deasă.
\par 3 Se aude vaietul păstorilor, căci mândrele lor pășuni au fost pustiite; se aude răgetul puilor de leu, căci bogăția desișurilor Iordanului a fost pârjolită.
\par 4 Așa zice Domnul Dumnezeul meu: "Paște oile de junghiat".
\par 5 Căci cei care le cumpără le junghie și nu au nici o vină, iar vânzătorul lor zice: "Binecuvântat să fie Domnul, căci, iată, m-am îmbogățit", și păstorii lor nu le cruță.
\par 6 "Și Eu nu Mă voi mai îndura de locuitorii țării", zice Domnul. "Iată că Eu voi da pe oameni, pe fiecare în mâinile vecinului său și ale regelui său, și vor pustii țara și nu-i voi scăpa din mâinile lor".
\par 7 Și m-am făcut cioban peste oile de junghiat din pricina neguțătorilor de oi. Și am luat două toiege. Pe unul l-am numit "Îndurare", iar pe celălalt "Legământ" și am început să pasc turma.
\par 8 Și într-o lună am stârpit pe cei trei păstori, căci sufletul Meu prinsese scârbă de ei și sufletul lor nu Mă mai putea suferi pe Mine.
\par 9 Și am grăit: "Nu vă mai pasc! Cea care este de murit să moară, cea de pierit să piară, iar oile care vor mai rămâne să se sfâșie între ele!"
\par 10 Atunci am luat toiagul Meu "Îndurare" și l-am frânt, ca să stric legământul pe care l-am încheiat cu toate popoarele.
\par 11 El a fost stricat în acea zi, iar neguțătorii de oi, care luau seama la Mine au înțeles că acesta a fost cuvântul Domnului.
\par 12 Și le-am zis: "Dacă socotiți cu cale, dați-Mi simbria, iar dacă nu, să nu Mi-o plătiți. Și Mi-au cântărit simbria Mea treizeci de arginți.
\par 13 Atunci a grăit Domnul către Mine: "Aruncă-l olarului prețul acela scump cu care Eu am fost prețuit de ei". Și am luat cei treizeci de arginți și i-am aruncat în vistieria templului Domnului, pentru olar.
\par 14 Apoi am rupt și toiagul cel de-al doilea, "Legământ", ca să stric frăția dintre Iuda și Israel.
\par 15 Și a zis Domnul către mine: "Ia lucrurile unui păstor nebun;
\par 16 Căci, iată, Eu voi ridica un păstor nebun în această țară care nu va umbla după oaia cea pierdută și care nu va căuta pe cea rătăcită și pe cea rănită nu o va vindeca și nu va hrăni pe cea sănătoasă, ci va mânca pe cea grasă și îi va smulge unghiile.
\par 17 Vai de ciobanul rău, care lasă turma în părăsire! Sabia să lovească brațul lui și ochiul lui cel drept! Brațul lui să se usuce cu totul, iar ochiul să rămână orb de tot!"

\chapter{12}

\par 1 Proorocie. Cuvântul Domnului asupra lui Israel. Așa grăiește Domnul, Care întinde cerurile ca un cort, Care pune temelie pământului și Care zidește duhul omului înăuntrul său:
\par 2 "Iată, voi face din Ierusalim cupă de amețire pentru toate popoarele din jur; tot așa va fi și pentru Iuda, când Ierusalimul va fi împresurat.
\par 3 Și în ziua aceea voi preface Ierusalimul în piatră de povară pentru toate popoarele. Toți care o vor ridica se vor răni grav și se vor aduna împotriva lui toate neamurile pământului.
\par 4 În ziua aceea", zice Domnul, "voi lovi toți caii cu spaimă și pe călăreți cu nebunie; și voi deschide ochii Mei asupra lui Iuda și pe toți capii popoarelor îi voi lovi cu orbire.
\par 5 Și vor grăi în inima lor căpeteniile lui Iuda: "Puterea pentru locuitorii Ierusalimului este în Domnul Savaot, Dumnezeul lor!"
\par 6 În ziua aceea voi face pe conducătorii lui Iuda ca un vas cu jeratic în mijlocul lemnelor și o torță aprinsă într-un stog de snopi; și ei vor mistui, la dreapta și la stânga, toate popoarele dimprejur, și Ierusalimul va fi locuit și mai departe pe locul lui, în Ierusalim.
\par 7 Și Domnul va izbăvi mai întâi corturile lui Iuda, ca seminția casei lui David și trufia locuitorilor Ierusalimului să nu se ridice deasupra lui Iuda.
\par 8 În ziua aceea Domnul va întinde ocrotirea Sa asupra celor ce locuiesc în Ierusalim, încât cel mai slab între ei să fie ca David, și casa lui David să fie ca Însuși Dumnezeu, ca îngerul Domnului care merge în fruntea lor.
\par 9 Și în ziua aceea Mă voi sârgui să pierd toate neamurile care vor veni împotriva Ierusalimului.
\par 10 Atunci voi vărsa peste casa lui David și peste locuitorii Ierusalimului duh de milostivire și de rugăciune, și își vor aținti privirile înspre Mine, pe Care ei L-au străpuns și vor face plângere asupra Lui, cum se face pentru un fiu unul născut și-L vor jeli ca pe cel întâi născut.
\par 11 În ziua aceea, va fi plângere mare în Ierusalim, ca plângerea de la Hadad-Rimon, în câmpia Meghidonului.
\par 12 Țara se va tângui, fiecare familie deosebit: familia casei lui David deosebit și femeile ei deosebit; familia casei lui Natan deosebit și femeile ei deosebit;
\par 13 Familia casei lui Levi deosebit și femeile ei deosebit; familia lui Șimei deosebit și femeile ei deosebit;
\par 14 Toate familiile care rămân, fiecare pentru sine și femeile lor pentru sine.

\chapter{13}

\par 1 În vremea aceea va fi un izvor cu apă curgătoare pentru casa lui David și pentru locuitorii Ierusalimului, pentru curățirea de păcat și de orice altă întinare.
\par 2 Și în ziua aceea", zice Domnul Savaot, "voi stârpi din țară numele idolilor, ca nimeni să nu-i mai pomenească; de asemenea voi da afară din țară pe proorocii lor și duhul cel spurcat.
\par 3 Și dacă va mai profeți cineva, vor zice către el tatăl său și mama sa, care l-au născut: "Tu nu vei trăi, căci ai grăit minciună în numele Domnului!" și atunci tatăl său și mama sa îl vor străpunge în clipa când va prooroci.
\par 4 Și în ziua aceea se vor rușina proorocii, fiecare de vedenia lui, când va grăi ca un prooroc, și nu se vor mai îmbrăca cu mantie de păr ca să spună minciuni.
\par 5 Și fiecare va zice: "Nu sunt prooroc, ci plugar, căci cu plugăria m-am îndeletnicit din tinerețile mele".
\par 6 Și dacă va fi întrebat: "De unde ai rănile acestea la mâini?" El va răspunde: "Am fost lovit în casa prietenilor mei!"
\par 7 Sabie, deșteaptă-te împotriva păstorului Meu, împotriva tovarășului Meu, zice Domnul Savaot. Voi bate păstorul și se vor risipi oile, și Îmi voi întoarce mina Mea împotriva celor mici.
\par 8 Și în toată țara", zice Domnul, "două treimi vor pieri, și vor muri, iar cealaltă treime va fi lăsată acolo.
\par 9 Iar pe aceasta a treia o voi pune în foc; îi voi curăți ca pe argint și îi voi încerca cum se încearcă aurul. Ei vor chema numele Meu și Eu îi voi asculta; și Eu voi zice: "Acesta este poporul Meu", și el va răspunde: "Domnul este Dumnezeul meu!"

\chapter{14}

\par 1 Iată că vine ziua Domnului, când se vor împărți prăzile tale în mijlocul tău.
\par 2 Și voi aduna toate neamurile pentru război împotriva Ierusalimului, și cetatea va fi luată, casele vor fi jefuite și femeile necinstite. Atunci jumătate din cetate va fi dusă în robie, iar restul poporului Meu nu va fi stârpit din cetate.
\par 3 Atunci Domnul va ieși la luptă și se va război împotriva acestor popoare, ca în vreme de luptă, ca în vreme de război.
\par 4 Și în vremea aceea se vor sprijini picioarele Lui pe Muntele Măslinilor, care este în fața Ierusalimului, la răsărit; iar Muntele Măslinilor se va crăpa în două de la răsărit la apus și se va face o vale foarte mare și jumătate din munte se va da înapoi către miazănoapte și cealaltă jumătate către miazăzi.
\par 5 Și voi veți alerga prin valea munților Mei, căci valea munților se va întinde până la locul unde Eu voi da izbăvire. Și veți alerga cum ați alergat de frica cutremurului, în vremea domniei lui Ozia, regele lui Iuda. Atunci va veni Domnul Dumnezeul meu și toți sfinții împreună cu El.
\par 6 În vremea aceea nu va mai fi lumină, ci frig și ger.
\par 7 Va fi o zi deosebită pe care Domnul singur o știe; nu va fi nici zi și nici noapte, ci în vremea serii va fi lumină.
\par 8 Iar în ziua aceea vor izvorî din Ierusalim ape vii: jumătate se vor îndrepta către marea cea de la răsărit, iar cealaltă jumătate către marea cea de la apus. Și așa va fi și vara și iarna.
\par 9 Și va fi Domnul Împărat peste tot pământul; în vremea aceea va fi Domnul unul singur și tot așa și numele Său unul singur.
\par 10 Țara întreagă va fi prefăcută în câmpie de la Gheba și până la Rimon, spre miazăzi de Ierusalim. Și Ierusalimul va fi înălțat și va fi locuit pe locul unde se află el, de la poarta lui Veniamin și până la locul porții dintâi, adică până la poarta din colț, și de la turnul lui Hananeel până la teascurile regelui.
\par 11 Și vor locui în el, și nu va mai fi acolo blestem, ci Ierusalimul va fi locuit în siguranță.
\par 12 Dar iată care va fi prăpădul cu care Domnul va lovi toate popoarele care s-au războit cu Ierusalimul: trupul dușmanului va putrezi stând în picioare, ochii în orbitele lor și limba în gură.
\par 13 În ziua aceea va fi de la Domnul mare spaimă printre ei și fiecare va apuca de mână pe aproapele său și își vor ridica mâna unii împotriva altora.
\par 14 Și Iuda va lupta împotriva Ierusalimului, și bogățiile tuturor neamurilor din jur, aur, argint și veșminte fără număr vor fi strânse la un loc.
\par 15 Și la fel cu acest prăpăd va fi prăpădul care va lovi calul, catârul, cămila, asinul și toate dobitoacele care vor fi în acele tabere.
\par 16 Și toți cei care vor fi rămas cu viață dintre neamurile acelea care veniseră să lupte împotriva Ierusalimului se vor sui în fiecare an să se închine Împăratului, Domnul Savaot, și să prăznuiască sărbătoarea corturilor.
\par 17 Iar cele din neamurile pământului care nu se vor sui să se închine în Ierusalim, Împăratului, Domnul Savaot, nu vor avea parte de ploaie.
\par 18 Și dacă neamul Egiptului nu se va sui și nu va veni, nu numai că ei nu vor avea parte de ploaie, ci va veni peste ei prăpădul cu care Domnul va lovi neamurile care nu se vor sui să prăznuiască sărbătoarea corturilor.
\par 19 Aceasta va fi pedeapsa Egiptului și pedeapsa tuturor neamurilor care nu se vor sui în Ierusalim să prăznuiască sărbătoarea corturilor.
\par 20 În ziua aceea va fi (scris) pe clopoțeii cailor: "Sfânt lui Dumnezeu!" Și vor fi căldările în templul Domnului ca năstrapele înaintea jertfelnicului.
\par 21 Și orice căldare din Ierusalim și din Iuda va fi sfântă Domnului Savaot. Și vor veni toți cei ce vor să jertfească și le vor lua și vor fierbe carne și nu va mai fi nici un neguțător, în ziua aceea, în templul Domnului Savaot.


\end{document}