\begin{document}

\title{Matthew}

Mat 1:1  Cartea neamului lui Iisus Hristos, fiul lui David, fiul lui Avraam.
Mat 1:2  Avraam a nascut pe Isaac; Isaac a nascut pe Iacov; Iacov a nascut pe Iuda ?i fra?ii lui;
Mat 1:3  Iuda a nascut pe Fares ?i pe Zara, din Tamar; Fares a nascut pe Esrom; Esrom a nascut pe Aram;
Mat 1:4  Aram a nascut pe Aminadav; Aminadav a nascut pe Naason; Naason a nascut pe Salmon;
Mat 1:5  Salmon a nascut pe Booz, din Rahav; Booz a nascut pe Iobed, din Rut; Iobed a nascut pe Iesei;
Mat 1:6  Iesei a nascut pe David regele; David a nascut pe Solomon din femeia lui Urie;
Mat 1:7  Solomon a nascut pe Roboam; Roboam a nascut pe Abia; Abia a nascut pe Asa;
Mat 1:8  Asa a nascut pe Iosafat; Iosafat a nascut pe Ioram; Ioram a nascut pe Ozia;
Mat 1:9  Ozia a nascut pe Ioatam; Ioatam a nascut pe Ahaz; Ahaz a nascut pe Iezechia;
Mat 1:10  Iezechia a nascut pe Manase; Manase a nascut pe Amon; Amon a nascut pe Iosia;
Mat 1:11  Iosia a nascut pe Iehonia ?i pe fra?ii lui, la stramutarea în Babilon;
Mat 1:12  Dupa stramutarea în Babilon, Iehonia a nascut pe Salatiel; Salatiel a nascut pe Zorobabel;
Mat 1:13  Zorobabel a nascut pe Abiud; Abiud a nascut pe Eliachim; Eliachim a nascut pe Azor;
Mat 1:14  Azor a nascut pe Sadoc; Sadoc a nascut pe Achim; Achim a nascut pe Eliud;
Mat 1:15  Eliud a nascut pe Eleazar; Eleazar a nascut pe Matan; Matan a nascut pe Iacov;
Mat 1:16  Iacov a nascut pe Iosif, logodnicul Mariei, din care S-a nascut Iisus, Care se cheama Hristos.
Mat 1:17  A?adar, toate neamurile de la Avraam pâna la David sunt paisprezece; ?i de la David pâna la stramutarea în Babilon sunt paisprezece; ?i de la stramutarea în Babilon pâna la Hristos sunt paisprezece neamuri.
Mat 1:18  Iar na?terea lui Iisus Hristos a?a a fost: Maria, mama Lui, fiind logodita cu Iosif, fara sa fi fost ei înainte împreuna, s-a aflat având în pântece de la Duhul Sfânt.
Mat 1:19  Iosif, logodnicul ei, drept fiind ?i nevrând s-o vadeasca, a voit s-o lase în ascuns.
Mat 1:20  ?i cugetând el acestea, iata îngerul Domnului i s-a aratat în vis, graind: Iosife, fiul lui David, nu te teme a lua pe Maria, logodnica ta, ca ce s-a zamislit într-însa este de la Duhul Sfânt.
Mat 1:21  Ea va na?te Fiu ?i vei chema numele Lui: Iisus, caci El va mântui poporul Sau de pacatele lor.
Mat 1:22  Acestea toate s-au facut ca sa se împlineasca ceea ce s-a zis de Domnul prin proorocul care zice:
Mat 1:23  "Iata, Fecioara va avea în pântece ?i va na?te Fiu ?i vor chema numele Lui Emanuel, care se tâlcuie?te: Cu noi este Dumnezeu".
Mat 1:24  ?i de?teptându-se din somn, Iosif a facut a?a precum i-a poruncit îngerul Domnului ?i a luat la el pe logodnica sa.
Mat 1:25  ?i fara sa fi cunoscut-o pe ea Iosif, Maria a nascut pe Fiul sau Cel Unul-Nascut, Caruia I-a pus numele Iisus.
Mat 2:1  Iar daca S-a nascut Iisus în Betleemul Iudeii, în zilele lui Irod regele, iata magii de la Rasarit au venit în Ierusalim, întrebând:
Mat 2:2  Unde este regele Iudeilor, Cel ce S-a nascut? Caci am vazut la Rasarit steaua Lui ?i am venit sa ne închinam Lui.
Mat 2:3  ?i auzind, regele Irod s-a tulburat ?i tot Ierusalimul împreuna cu el.
Mat 2:4  ?i adunând pe to?i arhiereii ?i carturarii poporului, cauta sa afle de la ei: Unde este sa Se nasca Hristos?
Mat 2:5  Iar ei i-au zis: În Betleemul Iudeii, ca a?a este scris de proorocul:
Mat 2:6  "?i tu, Betleeme, pamântul lui Iuda, nu e?ti nicidecum cel mai mic între capeteniile lui Iuda, caci din tine va ie?i Conducatorul care va pa?te pe poporul Meu Israel".
Mat 2:7  Atunci Irod chemând în ascuns pe magi, a aflat de la ei lamurit în ce vreme s-a aratat steaua.
Mat 2:8  ?i trimi?ându-i la Betleem, le-a zis: Merge?i ?i cerceta?i cu de-amanuntul despre Prunc ?i, daca Îl ve?i afla, vesti?i-mi ?i mie, ca, venind ?i eu, sa ma închin Lui.
Mat 2:9  Iar ei, ascultând pe rege, au plecat ?i iata, steaua pe care o vazusera în Rasarit mergea înaintea lor, pâna ce a venit ?i a stat deasupra, unde era Pruncul.
Mat 2:10  ?i vazând ei steaua, s-au bucurat cu bucurie mare foarte.
Mat 2:11  ?i intrând în casa, au vazut pe Prunc împreuna cu Maria, mama Lui, ?i cazând la pamânt, s-au închinat Lui; ?i deschizând vistieriile lor, I-au adus Lui daruri: aur, tamâie ?i smirna.
Mat 2:12  Iar luând în?tiin?are în vis sa nu se mai întoarca la Irod, pe alta cale s-au dus în ?ara lor.
Mat 2:13  Dupa plecarea magilor, iata îngerul Domnului se arata în vis lui Iosif, zicând: Scoala-te, ia Pruncul ?i pe mama Lui ?i fugi în Egipt ?i stai acolo pâna ce-?i voi spune, fiindca Irod are sa caute Pruncul ca sa-L ucida.
Mat 2:14  ?i sculându-se, a luat, noaptea, Pruncul ?i pe mama Lui ?i a plecat în Egipt.
Mat 2:15  ?i au stat acolo pâna la moartea lui Irod, ca sa se împlineasca cuvântul spus de Domnul, prin proorocul: "Din Egipt am chemat pe Fiul Meu".
Mat 2:16  Iar când Irod a vazut ca a fost amagit de magi, s-a mâniat foarte ?i, trimi?ând a ucis pe to?i pruncii care erau în Betleem ?i în toate hotarele lui, de doi ani ?i mai jos, dupa timpul pe care îl aflase de la magi.
Mat 2:17  Atunci s-a împlinit ceea ce se spusese prin Ieremia proorocul:
Mat 2:18  "Glas în Rama s-a auzit, plângere ?i tânguire multa; Rahela î?i plânge copiii ?i nu voie?te sa fie mângâiata pentru ca nu sunt".
Mat 2:19  Dupa moartea lui Irod, iata ca îngerul Domnului s-a aratat în vis lui Iosif în Egipt,
Mat 2:20  ?i i-a zis: Scoala-te, ia Pruncul ?i pe mama Lui ?i mergi în pamântul lui Israel, caci au murit cei ce cautau sa ia sufletul Pruncului.
Mat 2:21  Iosif, sculându-se, a luat Pruncul ?i pe mama Lui ?i a venit în pamântul lui Israel.
Mat 2:22  ?i auzind ca domne?te Arhelau în Iudeea, în locul lui Irod, tatal sau, s-a temut sa mearga acolo ?i, luând porunca, în vis, s-a dus în par?ile Galileii.
Mat 2:23  ?i venind a locuit în ora?ul numit Nazaret, ca sa se împlineasca ceea ce s-a spus prin prooroci, ca Nazarinean Se va chema.
Mat 3:1  În zilele acelea, a venit Ioan Botezatorul ?i propovaduia în pustia Iudeii,
Mat 3:2  Spunând: Pocai?i-va ca s-a apropiat împara?ia cerurilor.
Mat 3:3  El este acela despre care a zis proorocul Isaia: "Glasul celui ce striga în pustie: Pregati?i calea Domnului, drepte face?i cararile Lui".
Mat 3:4  Iar Ioan avea îmbracamintea lui din par de camila, ?i cingatoare de piele împrejurul mijlocului, iar hrana era lacuste ?i miere salbatica.
Mat 3:5  Atunci a ie?it la el Ierusalimul ?i toata Iudeea ?i toata împrejurimea Iordanului.
Mat 3:6  ?i erau boteza?i de catre el în râul Iordan, marturisindu-?i pacatele.
Mat 3:7  Dar vazând Ioan pe mul?i din farisei ?i saduchei venind la botez, le-a zis: Pui de vipere, cine v-a aratat sa fugi?i de mânia ce va sa fie?
Mat 3:8  Face?i deci roada, vrednica de pocain?a,
Mat 3:9  ?i sa nu crede?i ca pute?i zice în voi în?iva: Parinte avem pe Avraam, caci va spun ca Dumnezeu poate ?i din pietrele acestea sa ridice fii lui Avraam.
Mat 3:10  Iata securea sta la radacina pomilor ?i tot pomul care nu face roada buna se taie ?i se arunca în foc.
Mat 3:11  Eu unul va botez cu apa spre pocain?a, dar Cel ce vine dupa mine este mai puternic decât mine; Lui nu sunt vrednic sa-I duc încal?amintea; Acesta va va boteza cu Duh Sfânt ?i cu foc.
Mat 3:12  El are lopata în mâna ?i va cura?a aria Sa ?i va aduna grâul în jitni?a, iar pleava o va arde cu foc nestins.
Mat 3:13  În acest timp a venit Iisus din Galileea, la Iordan, catre Ioan, ca sa se boteze de catre el.
Mat 3:14  Ioan însa Îl oprea, zicând: Eu am trebuin?a sa fiu botezat de Tine, ?i Tu vii la mine?
Mat 3:15  ?i raspunzând, Iisus a zis catre el: Lasa acum, ca a?a se cuvine noua sa împlinim toata dreptatea. Atunci L-a lasat.
Mat 3:16  Iar botezându-se Iisus, când ie?ea din apa, îndata cerurile s-au deschis ?i Duhul lui Dumnezeu s-a vazut pogorându-se ca un porumbel ?i venind peste El.
Mat 3:17  ?i iata glas din ceruri zicând: "Acesta este Fiul Meu cel iubit întru Care am binevoit".
Mat 4:1  Atunci Iisus a fost dus de Duhul în pustiu, ca sa fie ispitit de catre diavolul.
Mat 4:2  ?i dupa ce a postit patruzeci de zile ?i patruzeci de nop?i, la urma a flamânzit.
Mat 4:3  ?i apropiindu-se, ispititorul a zis catre El: De e?ti Tu Fiul lui Dumnezeu, zi ca pietrele acestea sa se faca pâini.
Mat 4:4  Iar El, raspunzând, a zis: Scris este: "Nu numai cu pâine va trai omul, ci cu tot cuvântul care iese din gura lui Dumnezeu".
Mat 4:5  Atunci diavolul L-a dus pe aripa în sfânta cetate, L-a pus pe aripa templului,
Mat 4:6  ?i I-a zis: Daca Tu e?ti Fiul lui Dumnezeu, arunca-Te jos, ca scris este: "Îngerilor Sai va porunci pentru Tine ?i Te vor ridica pe mâini, ca nu cumva sa izbe?ti de piatra piciorul Tau".
Mat 4:7  Iisus i-a raspuns: Iara?i este scris: "Sa nu ispite?ti pe Domnul Dumnezeul tau".
Mat 4:8  Din nou diavolul L-a dus pe un munte foarte înalt ?i I-a aratat toate împara?iile lumii ?i slava lor.
Mat 4:9  ?i I-a zis Lui: Acestea toate ?i le voi da ?ie, daca vei cadea înaintea mea ?i Te vei închina mie.
Mat 4:10  Atunci Iisus i-a zis: Piei, satano, caci scris este: "Domnului Dumnezeului tau sa te închini ?i Lui singur sa-I sluje?ti".
Mat 4:11  Atunci L-a lasat diavolul ?i iata îngerii, venind la El, Îi slujeau.
Mat 4:12  ?i Iisus, auzind ca Ioan a fost întemni?at, a plecat în Galileea.
Mat 4:13  ?i parasind Nazaretul, a venit de a locuit în Capernaum, lânga mare, în hotarele lui Zabulon ?i Neftali,
Mat 4:14  Ca sa se împlineasca ce s-a zis prin Isaia proorocul care zice:
Mat 4:15  "Pamântul lui Zabulon ?i pamântul lui Neftali spre mare, dincolo de Iordan, Galileea neamurilor;
Mat 4:16  Poporul care statea în întuneric a vazut lumina mare ?i celor ce ?edeau în latura ?i în umbra mor?ii lumina le-a rasarit".
Mat 4:17  De atunci a început Iisus sa propovaduiasca ?i sa spuna: Pocai?i-va, caci s-a apropiat împara?ia cerurilor.
Mat 4:18  Pe când umbla pe lânga Marea Galileii, a vazut pe doi fra?i, pe Simon ce se nume?te Petru ?i pe Andrei, fratele lui, care aruncau mreaja în mare, caci erau pescari.
Mat 4:19  ?i le-a zis: Veni?i dupa Mine ?i va voi face pescari de oameni.
Mat 4:20  Iar ei, îndata lasând mrejele, au mers dupa El.
Mat 4:21  ?i de acolo, mergând mai departe, a vazut al?i doi fra?i, pe Iacov al lui Zevedeu ?i pe Ioan fratele lui, în corabie cu Zevedeu, tatal lor, dregându-?i mrejele ?i i-a chemat.
Mat 4:22  Iar ei îndata, lasând corabia ?i pe tatal lor, au mers dupa El.
Mat 4:23  ?i a strabatut Iisus toata Galileea, înva?ând în sinagogile lor ?i propovaduind Evanghelia împara?iei ?i tamaduind toata boala ?i toata neputin?a în popor.
Mat 4:24  ?i s-a dus vestea despre El în toata Siria, ?i aduceau la El pe to?i cei ce se aflau în suferin?e, fiind cuprin?i de multe feluri de boli ?i de chinuri, pe demoniza?i, pe lunatici, pe slabanogi, ?i El îi vindeca.
Mat 4:25  ?i mul?imi multe mergeau dupa El, din Galileea, din Decapole, din Ierusalim, din Iudeea ?i de dincolo de Iordan.
Mat 5:1  Vazând mul?imile, Iisus S-a suit în munte, ?i a?ezându-se, ucenicii Lui au venit la El.
Mat 5:2  ?i deschizându-?i gura, îi înva?a zicând:
Mat 5:3  Ferici?i cei saraci cu duhul, ca a lor este împara?ia cerurilor.
Mat 5:4  Ferici?i cei ce plâng, ca aceia se vor mângâia.
Mat 5:5  Ferici?i cei blânzi, ca aceia vor mo?teni pamântul.
Mat 5:6  Ferici?i cei ce flamânzesc ?i înseteaza de dreptate, ca aceia se vor satura.
Mat 5:7  Ferici?i cei milostivi, ca aceia se vor milui.
Mat 5:8  Ferici?i cei cura?i cu inima, ca aceia vor vedea pe Dumnezeu.
Mat 5:9  Ferici?i facatorii de pace, ca aceia fiii lui Dumnezeu se vor chema.
Mat 5:10  Ferici?i cei prigoni?i pentru dreptate, ca a lor este împara?ia cerurilor.
Mat 5:11  Ferici?i ve?i fi voi când va vor ocarî ?i va vor prigoni ?i vor zice tot cuvântul rau împotriva voastra, min?ind din pricina Mea.
Mat 5:12  Bucura?i-va ?i va veseli?i, ca plata voastra multa este în ceruri, ca a?a au prigonit pe proorocii cei dinainte de voi.
Mat 5:13  Voi sunte?i sarea pamântului; daca sarea se va strica, cu ce se va sara? De nimic nu mai e buna decât sa fie aruncata afara ?i calcata în picioare de oameni.
Mat 5:14  Voi sunte?i lumina lumii; nu poate o cetate aflata pe vârf de munte sa se ascunda.
Mat 5:15  Nici nu aprind faclie ?i o pun sub obroc, ci în sfe?nic, ?i lumineaza tuturor celor din casa.
Mat 5:16  A?a sa lumineze lumina voastra înaintea oamenilor, a?a încât sa vada faptele voastre cele bune ?i sa slaveasca pe Tatal vostru Cel din ceruri.
Mat 5:17  Sa nu socoti?i ca am venit sa stric Legea sau proorocii; n-am venit sa stric, ci sa împlinesc.
Mat 5:18  Caci adevarat zic voua: Înainte de a trece cerul ?i pamântul, o iota sau o cirta din Lege nu va trece, pâna ce se vor face toate.
Mat 5:19  Deci, cel ce va strica una din aceste porunci, foarte mici, ?i va înva?a a?a pe oameni, foarte mic se va chema în împara?ia cerurilor; iar cel ce va face ?i va înva?a, acesta mare se va chema în împara?ia cerurilor.
Mat 5:20  Caci zic voua: Ca de nu va prisosi dreptatea voastra mai mult decât a carturarilor ?i a fariseilor, nu ve?i intra în împara?ia cerurilor.
Mat 5:21  A?i auzit ca s-a zis celor de demult: "Sa nu ucizi"; iar cine va ucide, vrednic va fi de osânda.
Mat 5:22  Eu însa va spun voua: Ca oricine se mânie pe fratele sau vrednic va fi de osânda; ?i cine va zice fratelui sau: netrebnicule, vrednic va fi de judecata sinedriului; iar cine va zice: nebunule, vrednic va fi de gheena focului.
Mat 5:23  Deci, daca î?i vei aduce darul tau la altar ?i acolo î?i vei aduce aminte ca fratele tau are ceva împotriva ta,
Mat 5:24  Lasa darul tau acolo, înaintea altarului, ?i mergi întâi ?i împaca-te cu fratele tau ?i apoi, venind, adu darul tau.
Mat 5:25  Împaca-te cu pârâ?ul tau degraba, pâna e?ti cu el pe cale, ca nu cumva pârâ?ul sa te dea judecatorului, ?i judecatorul slujitorului ?i sa fii aruncat în temni?a.
Mat 5:26  Adevarat graiesc ?ie: Nu vei ie?i de acolo, pâna ce nu vei fi dat cel din urma ban.
Mat 5:27  A?i auzit ca s-a zis celor de demult: "Sa nu savâr?e?ti adulter".
Mat 5:28  Eu însa va spun voua: Ca oricine se uita la femeie, poftind-o, a ?i savâr?it adulter cu ea în inima lui.
Mat 5:29  Iar daca ochiul tau cel drept te sminte?te pe tine, scoate-l ?i arunca-l de la tine, caci mai de folos î?i este sa piara unul din madularele tale, decât tot trupul sa fie aruncat în gheena.
Mat 5:30  ?i daca mâna ta cea dreapta te sminte?te pe tine, taie-o ?i o arunca de la tine, caci mai de folos î?i este sa piara unul din madularele tale, decât tot trupul tau sa fie aruncat în gheena.
Mat 5:31  S-a zis iara?i: "Cine va lasa pe femeia sa, sa-i dea carte de despar?ire".
Mat 5:32  Eu însa va spun voua: Ca oricine va lasa pe femeia sa, în afara de pricina de desfrânare, o face sa savâr?easca adulter, ?i cine va lua pe cea lasata savâr?e?te adulter.
Mat 5:33  A?i auzit ce s-a zis celor de demult: "Sa nu juri strâmb, ci sa ?ii înaintea Domnului juramintele tale".
Mat 5:34  Eu însa va spun voua: Sa nu va jura?i nicidecum nici pe cer, fiindca este tronul lui Dumnezeu,
Mat 5:35  Nici pe pamânt, fiindca este a?ternut al picioarelor Lui, nici pe Ierusalim, fiindca este cetate a marelui Împarat,
Mat 5:36  Nici pe capul tau sa nu te juri, fiindca nu po?i sa faci un fir de par alb sau negru,
Mat 5:37  Ci cuvântul vostru sa fie: Ceea ce este da, da; ?i ceea ce este nu, nu; iar ce e mai mult decât acestea, de la cel rau este.
Mat 5:38  A?i auzit ca s-a zis: "Ochi pentru ochi ?i dinte pentru dinte".
Mat 5:39  Eu însa va spun voua: Nu va împotrivi?i celui rau; iar cui te love?te peste obrazul drept, întoarce-i ?i pe celalalt.
Mat 5:40  Celui ce voie?te sa se judece cu tine ?i sa-?i ia haina, lasa-i ?i cama?a.
Mat 5:41  Iar de te va sili cineva sa mergi o mila, mergi cu el doua.
Mat 5:42  Celui care cere de la tine, da-i; ?i de la cel ce voie?te sa se împrumute de la tine, nu întoarce fa?a ta.
Mat 5:43  A?i auzit ca s-a zis: "Sa iube?ti pe aproapele tau ?i sa ura?ti pe vrajma?ul tau".
Mat 5:44  Iar Eu zic voua: Iubi?i pe vrajma?ii vo?tri, binecuvânta?i pe cei ce va blestema, face?i bine celor ce va urasc ?i ruga?i-va pentru cei ce va vatama ?i va prigonesc,
Mat 5:45  Ca sa fi?i fiii Tatalui vostru Celui din ceruri, ca El face sa rasara soarele ?i peste cei rai ?i peste cei buni ?i trimite ploaie peste cei drep?i ?i peste cei nedrep?i.
Mat 5:46  Caci daca iubi?i pe cei ce va iubesc, ce rasplata ve?i avea? Au nu fac ?i vame?ii acela?i lucru?
Mat 5:47  ?i daca îmbra?i?a?i numai pe fra?ii vo?tri, ce face?i mai mult? Au nu fac ?i neamurile acela?i lucru?
Mat 5:48  Fi?i, dar, voi desavâr?i?i, precum Tatal vostru Cel ceresc desavâr?it este.
Mat 6:1  Lua?i aminte ca faptele drepta?ii voastre sa nu le face?i înaintea oamenilor ca sa fi?i vazu?i de ei; altfel nu ve?i avea plata de la Tatal vostru Cel din ceruri.
Mat 6:2  Deci, când faci milostenie, nu trâmbi?a înaintea ta, cum fac fa?arnicii în sinagogi ?i pe uli?e, ca sa fie slavi?i de oameni; adevarat graiesc voua: ?i-au luat plata lor.
Mat 6:3  Tu însa, când faci milostenie, sa nu ?tie stânga ta ce face dreapta ta,
Mat 6:4  Ca milostenia ta sa fie într-ascuns ?i Tatal tau, Care vede în ascuns, î?i va rasplati ?ie.
Mat 6:5  Iar când va ruga?i, nu fi?i ca fa?arnicii carora le place, prin sinagogi ?i prin col?urile uli?elor, stând în picioare, sa se roage, ca sa se arate oamenilor; adevarat graiesc voua: ?i-au luat plata lor.
Mat 6:6  Tu însa, când te rogi, intra în camara ta ?i, închizând u?a, roaga-te Tatalui tau, Care este în ascuns, ?i Tatal tau, Care este în ascuns, î?i va rasplati ?ie.
Mat 6:7  Când va ruga?i, nu spune?i multe ca neamurile, ca ele cred ca în multa lor vorbarie vor fi ascultate.
Mat 6:8  Deci nu va asemana?i lor, ca ?tie Tatal vostru de cele ce ave?i trebuin?a mai înainte ca sa cere?i voi de la El.
Mat 6:9  Deci voi a?a sa va ruga?i: Tatal nostru, Care e?ti în ceruri, sfin?easca-se numele Tau;
Mat 6:10  Vie împara?ia Ta; faca-se voia Ta, precum în cer ?i pe pamânt.
Mat 6:11  Pâinea noastra cea spre fiin?a da-ne-o noua astazi;
Mat 6:12  ?i ne iarta noua gre?ealele noastre, precum ?i noi iertam gre?i?ilor no?tri;
Mat 6:13  ?i nu ne duce pe noi în ispita, ci ne izbave?te de cel rau. Ca a Ta este împara?ia ?i puterea ?i slava în veci. Amin!
Mat 6:14  Ca de ve?i ierta oamenilor gre?ealele lor, ierta-va ?i voua Tatal vostru Cel ceresc;
Mat 6:15  Iar de nu ve?i ierta oamenilor gre?ealele lor, nici Tatal vostru nu va va ierta gre?ealele voastre.
Mat 6:16  Când posti?i, nu fi?i tri?ti ca fa?arnicii; ca ei î?i smolesc fe?ele, ca sa se arate oamenilor ca postesc. Adevarat graiesc voua, ?i-au luat plata lor.
Mat 6:17  Tu însa, când poste?ti, unge capul tau ?i fa?a ta o spala,
Mat 6:18  Ca sa nu te ara?i oamenilor ca poste?ti, ci Tatalui tau care este în ascuns, ?i Tatal tau, Care vede în ascuns, î?i va rasplati ?ie.
Mat 6:19  Nu va aduna?i comori pe pamânt, unde molia ?i rugina le strica ?i unde furii le sapa ?i le fura.
Mat 6:20  Ci aduna?i-va comori în cer, unde nici molia, nici rugina nu le strica, unde furii nu le sapa ?i nu le fura.
Mat 6:21  Caci unde este comoara ta, acolo va fi ?i inima ta.
Mat 6:22  Luminatorul trupului este ochiul; de va fi ochiul tau curat, tot trupul tau va fi luminat.
Mat 6:23  Iar de va fi ochiul tau rau, tot trupul tau va fi întunecat. Deci, daca lumina care e în tine este întuneric, dar întunericul cu cât mai mult!
Mat 6:24  Nimeni nu poate sa slujeasca la doi domni, caci sau pe unul îl va urî ?i pe celalalt îl va iubi, sau de unul se va lipi ?i pe celalalt îl va dispre?ui; nu pute?i sa sluji?i lui Dumnezeu ?i lui mamona.
Mat 6:25  De aceea zic voua: Nu va îngriji?i pentru sufletul vostru ce ve?i mânca, nici pentru trupul vostru cu ce va ve?i îmbraca; au nu este sufletul mai mult decât hrana ?i trupul decât îmbracamintea?
Mat 6:26  Privi?i la pasarile cerului, ca nu seamana, nici nu secera, nici nu aduna în jitni?e, ?i Tatal vostru Cel ceresc le hrane?te. Oare nu sunte?i voi cu mult mai presus decât ele?
Mat 6:27  ?i cine dintre voi, îngrijindu-se poate sa adauge staturii sale un cot?
Mat 6:28  Iar de îmbracaminte de ce va îngriji?i? Lua?i seama la crinii câmpului cum cresc: nu se ostenesc, nici nu torc.
Mat 6:29  ?i va spun voua ca nici Solomon, în toata marirea lui, nu s-a îmbracat ca unul dintre ace?tia.
Mat 6:30  Iar daca iarba câmpului, care astazi este ?i mâine se arunca în cuptor, Dumnezeu astfel o îmbraca, oare nu cu mult mai mult pe voi, pu?in credincio?ilor?
Mat 6:31  Deci, nu duce?i grija, spunând: Ce vom mânca, ori ce vom bea, ori cu ce ne vom îmbraca?
Mat 6:32  Ca dupa toate acestea se straduiesc neamurile; ?tie doar Tatal vostru Cel ceresc ca ave?i nevoie de ele.
Mat 6:33  Cauta?i mai întâi împara?ia lui Dumnezeu ?i dreptatea Lui ?i toate acestea se vor adauga voua.
Mat 6:34  Nu va îngriji?i de ziua de mâine, caci ziua de mâine se va îngriji de ale sale. Ajunge zilei rautatea ei.
Mat 7:1  Nu judeca?i, ca sa nu fi?i judeca?i.
Mat 7:2  Caci cu judecata cu care judeca?i, ve?i fi judeca?i, ?i cu masura cu care masura?i, vi se va masura.
Mat 7:3  De ce vezi paiul din ochiul fratelui tau, ?i bârna din ochiul tau nu o iei în seama?
Mat 7:4  Sau cum vei zice fratelui tau: Lasa sa scot paiul din ochiul tau ?i iata bârna este în ochiul tau?
Mat 7:5  Fa?arnice, scoate întâi bârna din ochiul tau ?i atunci vei vedea sa sco?i paiul din ochiul fratelui tau.
Mat 7:6  Nu da?i cele sfinte câinilor, nici nu arunca?i margaritarele voastre înaintea porcilor, ca nu cumva sa le calce în picioare ?i, întorcându-se, sa va sfâ?ie pe voi.
Mat 7:7  Cere?i ?i vi se va da; cauta?i ?i ve?i afla; bate?i ?i vi se va deschide.
Mat 7:8  Ca oricine cere ia, cel care cauta afla, ?i celui ce bate i se va deschide.
Mat 7:9  Sau cine este omul acela între voi care, de va cere fiul sau pâine, oare el îi va da piatra?
Mat 7:10  Sau de-i va cere pe?te, oare el îi va da ?arpe?
Mat 7:11  Deci, daca voi, rai fiind, ?ti?i sa da?i daruri bune fiilor vo?tri, cu cât mai mult Tatal vostru Cel din ceruri va da cele bune celor care cer de la El?
Mat 7:12  Ci toate câte voi?i sa va faca voua oamenii, asemenea ?i voi face?i lor, ca aceasta este Legea ?i proorocii.
Mat 7:13  Intra?i prin poarta cea strâmta, ca larga este poarta ?i lata este calea care duce la pieire ?i mul?i sunt cei care o afla.
Mat 7:14  ?i strâmta este poarta ?i îngusta este calea care duce la via?a ?i pu?ini sunt care o afla.
Mat 7:15  Feri?i-va de proorocii mincino?i, care vin la voi în haine de oi, iar pe dinauntru sunt lupi rapitori.
Mat 7:16  Dupa roadele lor îi ve?i cunoa?te. Au doara culeg oamenii struguri din spini sau smochine din maracini?
Mat 7:17  A?a ca orice pom bun face roade bune, iar pomul rau face roade rele.
Mat 7:18  Nu poate pom bun sa faca roade rele, nici pom rau sa faca roade bune.
Mat 7:19  Iar orice pom care nu face roada buna se taie ?i se arunca în foc.
Mat 7:20  De aceea, dupa roadele lor îi ve?i cunoa?te.
Mat 7:21  Nu oricine Îmi zice: Doamne, Doamne, va intra în împara?ia cerurilor, ci cel ce face voia Tatalui Meu Celui din ceruri.
Mat 7:22  Mul?i Îmi vor zice în ziua aceea: Doamne, Doamne, au nu în numele Tau am proorocit ?i nu în numele Tau am scos demoni ?i nu în numele Tau minuni multe am facut?
Mat 7:23  ?i atunci voi marturisi lor: Niciodata nu v-am cunoscut pe voi. Departa?i-va de la Mine cei ce lucra?i faradelegea.
Mat 7:24  De aceea, oricine aude aceste cuvinte ale Mele ?i la îndepline?te asemana-se-va barbatului în?elept care a cladit casa lui pe stânca.
Mat 7:25  A cazut ploaia, au venit râurile mari, au suflat vânturile ?i au batut în casa aceea, dar ea n-a cazut, fiindca era întemeiata pe stânca.
Mat 7:26  Iar oricine aude aceste cuvinte ale Mele ?i nu le îndepline?te, asemana-se-va barbatului nechibzuit care ?i-a cladit casa pe nisip.
Mat 7:27  ?i a cazut ploaia ?i au venit râurile mari ?i au suflat vânturile ?i au izbit casa aceea, ?i a cazut. ?i caderea ei a fost mare.
Mat 7:28  Iar când Iisus a sfâr?it cuvintele acestea, mul?imile erau uimite de înva?atura Lui.
Mat 7:29  Ca îi înva?a pe ei ca unul care are putere, iar nu cum îi înva?au carturarii lor.
Mat 8:1  ?i coborându-Se El din munte, mul?imi multe au mers dupa El.
Mat 8:2  ?i iata un lepros, apropiindu-se, I se închina, zicând: Doamne, daca voie?ti, po?i sa ma cura?e?ti.
Mat 8:3  ?i Iisus, întinzând mâna, S-a atins de el, zicând: Voiesc, cura?e?te-te. ?i îndata s-a cura?it lepra lui.
Mat 8:4  ?i i-a zis Iisus: Vezi, nu spune nimanui, ci mergi, arata-te preotului ?i adu darul pe care l-a rânduit Moise, spre marturie lor.
Mat 8:5  Pe când intra în Capernaum, s-a apropiat de El un suta?, rugându-L,
Mat 8:6  ?i zicând: Doamne, sluga mea zace în casa, slabanog, chinuindu-se cumplit.
Mat 8:7  ?i i-a zis Iisus: Venind, îl voi vindeca.
Mat 8:8  Dar suta?ul, raspunzând, I-a zis: Doamne, nu sunt vrednic sa intri sub acoperi?ul meu, ci numai zi cu cuvântul ?i se va vindeca sluga mea.
Mat 8:9  Ca ?i eu sunt om sub stapânirea altora ?i am sub mine osta?i ?i-i spun acestuia: Du-te, ?i se duce; ?i celuilalt: Vino, ?i vine; ?i slugii mele: Fa aceasta, ?i face.
Mat 8:10  Auzind, Iisus S-a minunat ?i a zis celor ce veneau dupa El: Adevarat graiesc voua: la nimeni, în Israel, n-am gasit atâta credin?a.
Mat 8:11  ?i zic voua ca mul?i de la rasarit ?i de la apus vor veni ?i vor sta la masa cu Avraam, cu Isaac ?i cu Iacov în împara?ia cerurilor.
Mat 8:12  Iar fiii împara?iei vor fi arunca?i în întunericul cel mai din afara; acolo va fi plângerea ?i scrâ?nirea din?ilor.
Mat 8:13  ?i a zis Iisus suta?ului: Du-te, fie ?ie dupa cum ai crezut. ?i s-a însanato?it sluga lui în ceasul acela.
Mat 8:14  ?i venind Iisus în casa lui Petru, a vazut pe soacra acestuia zacând, prinsa de friguri.
Mat 8:15  ?i S-a atins de mâna ei, ?i au lasat-o frigurile ?i s-a sculat ?i Îi slujea Lui.
Mat 8:16  ?i facându-se seara, au adus la El mul?i demoniza?i ?i a scos duhurile cu cuvântul ?i pe to?i cei bolnavi i-a vindecat,
Mat 8:17  Ca sa se împlineasca ceea ce s-a spus prin Isaia proorocul, care zice: "Acesta neputin?ele noastre a luat ?i bolile noastre le-a purtat".
Mat 8:18  ?i vazând Iisus mul?ime împrejurul Lui, a poruncit ucenicilor sa treaca de cealalta parte a marii.
Mat 8:19  ?i apropiindu-se un carturar, i-a zis: Înva?atorule, Te voi urma oriunde vei merge.
Mat 8:20  Dar Iisus i-a raspuns: Vulpile au vizuini ?i pasarile cerului cuiburi; Fiul Omului însa nu are unde sa-?i plece capul.
Mat 8:21  Un altul dintre ucenici I-a zis: Doamne, da-mi voie întâi sa ma duc ?i sa îngrop pe tatal meu.
Mat 8:22  Iar Iisus i-a zis: Vino dupa Mine ?i lasa mor?ii sa-?i îngroape mor?ii lor.
Mat 8:23  Intrând El în corabie, ucenicii Lui L-au urmat.
Mat 8:24  ?i, iata, furtuna mare s-a ridicat pe mare, încât corabia se acoperea de valuri; iar El dormea.
Mat 8:25  ?i venind ucenicii la El, L-au de?teptat zicând: Doamne, mântuie?te-ne, ca pierim.
Mat 8:26  Iisus le-a zis: De ce va este frica, pu?in credincio?ilor? S-a sculat atunci, a certat vânturile ?i marea ?i s-a facut lini?te deplina.
Mat 8:27  Iar oamenii s-au mirat, zicând: Cine este Acesta ca ?i vânturile ?i marea asculta de El?
Mat 8:28  ?i trecând El dincolo, în ?inutul Gadarenilor, L-au întâmpinat doi demoniza?i, care ie?eau din morminte, foarte cumpli?i, încât nimeni nu putea sa treaca pe calea aceea.
Mat 8:29  ?i iata, au început sa strige ?i sa zica: Ce ai Tu cu noi, Iisuse, Fiul lui Dumnezeu? Ai venit aici mai înainte de vreme ca sa ne chinuie?ti?
Mat 8:30  Departe de ei era o turma mare de porci, pascând.
Mat 8:31  Iar demonii Îl rugau, zicând: Daca ne sco?i afara, trimite-ne în turma de porci.
Mat 8:32  ?i El le-a zis: Duce?i-va. Iar ei, ie?ind, s-au dus în turma de porci. ?i iata, toata turma s-a aruncat de pe ?arm în mare ?i a pierit în apa.
Mat 8:33  Iar pazitorii au fugit ?i, ducându-se în cetate, au spus toate cele întâmplate cu demoniza?ii.
Mat 8:34  ?i iata toata cetatea a ie?it în întâmpinarea lui Iisus ?i, vazându-L, L-au rugat sa treaca din hotarele lor.
Mat 9:1  Intrând în corabie, Iisus a trecut ?i a venit în cetatea Sa.
Mat 9:2  ?i iata, I-au adus un slabanog zacând pe pat. ?i Iisus, vazând credin?a lor, a zis slabanogului: Îndrazne?te, fiule! Iertate sunt pacatele tale!
Mat 9:3  Dar unii dintre carturari ziceau în sine: Acesta hule?te.
Mat 9:4  ?i Iisus, ?tiind gândurile lor, le-a zis: Pentru ce cugeta?i rele în inimile voastre?
Mat 9:5  Caci ce este mai lesne a zice: Iertate sunt pacatele tale, sau a zice: Scoala-te ?i umbla?
Mat 9:6  Dar ca sa ?ti?i ca putere are Fiul Omului pe pamânt a ierta pacatele, a zis slabanogului: Scoala-te, ia-?i patul ?i mergi la casa ta.
Mat 9:7  ?i, sculându-se, s-a dus la casa sa.
Mat 9:8  Iar mul?imile vazând acestea, s-au înspaimântat ?i au slavit pe Dumnezeu, Cel care da oamenilor asemenea putere.
Mat 9:9  ?i plecând Iisus de acolo, a vazut un om care ?edea la vama, cu numele Matei, ?i i-a zis acestuia: Vino dupa Mine. ?i sculându-se, a mers dupa El.
Mat 9:10  ?i pe când ?edea El la masa, în casa, iata mul?i vame?i ?i pacato?i au venit ?i au ?ezut la masa împreuna cu Iisus ?i cu ucenicii Lui.
Mat 9:11  ?i vazând fariseii, au zis ucenicilor: Pentru ce manânca Înva?atorul vostru cu vame?ii ?i cu pacato?ii?
Mat 9:12  ?i auzind El, a zis: Nu cei sanato?i au nevoie de doctor, ci cei bolnavi.
Mat 9:13  Dar mergând, înva?a?i ce înseamna: Mila voiesc, iar nu jertfa; ca n-am venit sa chem pe drep?i, ci pe pacato?i la pocain?a.
Mat 9:14  Atunci au venit la El ucenicii lui Ioan, zicând: Pentru ce noi ?i fariseii postim mult, iar ucenicii Tai nu postesc?
Mat 9:15  ?i Iisus le-a zis: Pot oare, fiii nun?ii sa fie tri?ti câta vreme mirele este cu ei? Dar vor veni zile când mirele va fi luat de la ei ?i atunci vor posti.
Mat 9:16  Nimeni nu pune un petic de postav nou la o haina veche, caci peticul acesta, ca umplutura, trage din haina ?i se face o ruptura ?i mai rea.
Mat 9:17  Nici nu pun oamenii vin nou în burdufuri vechi; alminterea burdufurile crapa: vinul se varsa ?i burdufurile se strica; ci pun vin nou în burdufuri noi ?i amândoua se pastreaza împreuna.
Mat 9:18  Pe când le spunea acestea, iata un dregator, venind, I s-a închinat, zicând: Fiica mea a murit de curând dar, venind, pune mâna Ta peste ea ?i va fi vie.
Mat 9:19  Atunci Iisus, sculându-Se, a mers dupa el împreuna cu ucenicii.
Mat 9:20  ?i iata o femeie cu scurgere de sânge de doisprezece ani, apropiindu-se de El pe la spate, s-a atins de poala hainei Lui.
Mat 9:21  Caci zicea în gândul ei: Numai sa ma ating de haina Lui ?i ma voi face sanatoasa;
Mat 9:22  Iar Iisus, întorcându-Se ?i vazând-o, i-a zis: Îndrazne?te, fiica, credin?a ta te-a mântuit. ?i s-a tamaduit femeia din ceasul acela.
Mat 9:23  Iisus, venind la casa dregatorului ?i vazând pe cântare?ii din flaut ?i mul?imea tulburata, a zis:
Mat 9:24  Departa?i-va, caci copila n-a murit, ci doarme. Dar ei râdeau de El.
Mat 9:25  Iar dupa ce mul?imea a fost scoasa afara, intrând, a luat-o de mâna, ?i copila s-a sculat.
Mat 9:26  ?i a ie?it vestea aceasta în tot ?inutul acela.
Mat 9:27  Plecând Iisus de acolo, doi orbi se ?ineau dupa El strigând ?i zicând: Miluie?te-ne pe noi, Fiule al lui David.
Mat 9:28  Dupa ce a intrat în casa, au venit la El orbii ?i Iisus i-a întrebat: Crede?i ca pot sa fac Eu aceasta? Zis-au Lui: Da, Doamne!
Mat 9:29  Atunci S-a atins de ochii lor, zicând: Dupa credin?a voastra, fie voua!
Mat 9:30  ?i s-au deschis ochii lor. Iar Iisus le-a poruncit cu asprime, zicând: Vede?i, nimeni sa nu ?tie.
Mat 9:31  Iar ei, ie?ind, L-au vestit în tot ?inutul acela.
Mat 9:32  ?i plecând ei, iata au adus la El un om mut, având demon.
Mat 9:33  ?i fiind scos demonul, mutul a grait. Iar mul?imile se minunau zicând: Niciodata nu s-a aratat a?a în Israel.
Mat 9:34  Dar fariseii ziceau: Cu domnul demonilor scoate pe demoni.
Mat 9:35  ?i Iisus strabatea toate ceta?ile ?i satele, înva?ând în sinagogile lor, propovaduind Evanghelia împara?iei ?i vindecând toata boala ?i toata neputin?a în popor.
Mat 9:36  ?i vazând mul?imile, I s-a facut mila de ele ca erau necajite ?i ratacite ca ni?te oi care n-au pastor.
Mat 9:37  Atunci a zis ucenicilor Lui: Seceri?ul e mult, dar lucratorii sunt pu?ini.
Mat 9:38  Ruga?i, deci, pe Domnul seceri?ului, ca sa scoata lurcatori la seceri?ul Sau.
Mat 10:1  Chemând la Sine pe cei doisprezece ucenici ai Sai, le-a dat lor putere asupra duhurilor celor necurate, ca sa le scoata ?i sa tamaduiasca orice boala ?i orice neputin?a.
Mat 10:2  Numele celor doisprezece apostoli sunt acestea: Întâi Simon, cel numit Petru, ?i Andrei, fratele lui; Iacov al lui Zevedeu ?i Ioan fratele lui;
Mat 10:3  Filip ?i Vartolomeu, Toma ?i Matei vame?ul, Iacov al lui Alfeu ?i Levi ce se zice Tadeu;
Mat 10:4  Simon Cananeul ?i Iuda Iscarioteanul, cel care L-a vândut.
Mat 10:5  Pe ace?ti doisprezece i-a trimis Iisus, poruncindu-le lor ?i zicând: În calea pagânilor sa nu merge?i, ?i în vreo cetate de samarineni sa nu intra?i;
Mat 10:6  Ci mai degraba merge?i catre oile cele pierdute ale casei lui Israel.
Mat 10:7  ?i mergând, propovadui?i, zicând: S-a apropiat împara?ia cerurilor.
Mat 10:8  Tamadui?i pe cei neputincio?i, învia?i pe cei mor?i, cura?i?i pe cei lepro?i, pe demoni scoate?i-i; în dar a?i luat, în dar sa da?i.
Mat 10:9  Sa nu ave?i nici aur, nici argin?i, nici bani în cingatorile voastre;
Mat 10:10  Nici traista pe drum, nici doua haine, nici încal?aminte, nici toiag; ca vrednic este lucratorul de hrana sa.
Mat 10:11  În orice cetate sau sat ve?i intra, cerceta?i cine este în el vrednic ?i acolo ramâne?i pâna ce ve?i ie?i.
Mat 10:12  ?i intrând în casa, ura?i-i, zicând: "Pace casei acesteia".
Mat 10:13  ?i daca este casa aceea vrednica, vina pacea voastra peste ea. Iar de nu este vrednica, pacea voastra întoarca-se la voi.
Mat 10:14  Cine nu va va primi pe voi, nici nu va asculta cuvintele voastre, ie?ind din casa sau din cetatea aceea, scutura?i praful de pe picioarele voastre.
Mat 10:15  Adevarat graiesc voua, mai u?or va fi pamântului Sodomei ?i Gomorei, în ziua judeca?ii, decât ceta?ii aceleia.
Mat 10:16  Iata Eu va trimit pe voi ca pe ni?te oi în mijlocul lupilor; fi?i dar în?elep?i ca ?erpii ?i nevinova?i ca porumbeii.
Mat 10:17  Feri?i-va de oameni, caci va vor da pe mâna sinedri?tilor ?i în sinagogile lor va vor bate cu biciul.
Mat 10:18  La dregatori ?i la regi ve?i fi du?i pentru Mine, spre marturie lor ?i pagânilor.
Mat 10:19  Iar când va vor da pe voi în mâna lor, nu va îngriji?i cum sau ce ve?i vorbi, caci se va da voua în ceasul acela ce sa vorbi?i;
Mat 10:20  Fiindca nu voi sunte?i care vorbi?i, ci Duhul Tatalui vostru este care graie?te întru voi.
Mat 10:21  Va da frate pe frate la moarte ?i tata pe fiu ?i se vor scula copiii împotriva parin?ilor ?i-i vor ucide.
Mat 10:22  ?i ve?i fi urâ?i de to?i pentru numele Meu; iar cel ce va rabda pâna în sfâr?it, acela se va mântui.
Mat 10:23  Când va urmaresc pe voi în cetatea aceasta, fugi?i în cealalta; adevarat graiesc voua: nu ve?i sfâr?i ceta?ile lui Israel, pâna ce va veni Fiul Omului.
Mat 10:24  Nu este ucenic mai presus de înva?atorul sau, nici sluga mai presus de stapânul sau.
Mat 10:25  Destul este ucenicului sa fie ca înva?atorul ?i slugii ca stapânul. Daca pe stapânul casei l-au numit Beelzebul, cu cât mai mult pe casnicii lui?
Mat 10:26  Deci nu va teme?i de ei, caci nimic nu este acoperit care sa nu iasa la iveala ?i nimic ascuns care sa nu ajunga cunoscut.
Mat 10:27  Ceea ce va graiesc la întuneric, spune?i la lumina ?i ceea ce auzi?i la ureche, propovadui?i de pe case.
Mat 10:28  Nu va teme?i de cei ce ucid trupul, iar sufletul nu pot sa-l ucida; teme?i-va mai curând de acela care poate ?i sufletul ?i trupul sa le piarda în gheena.
Mat 10:29  Au nu se vând doua vrabii pe un ban? ?i nici una din ele nu va cadea pe pamânt fara ?tirea Tatalui vostru.
Mat 10:30  La voi însa ?i perii capului, to?i sunt numara?i.
Mat 10:31  A?adar nu va teme?i; voi sunte?i cu mult mai de pre? decât pasarile.
Mat 10:32  Oricine va marturisi pentru Mine înaintea oamenilor, marturisi-voi ?i Eu pentru el înaintea Tatalui Meu, Care este în ceruri.
Mat 10:33  Iar de cel ce se va lepada de Mine înaintea oamenilor ?i Eu Ma voi lepada de el înaintea Tatalui Meu, Care este în ceruri.
Mat 10:34  Nu socoti?i ca am venit sa aduc pace pe pamânt; n-am venit sa aduc pace, ci sabie.
Mat 10:35  Caci am venit sa despart pe fiu de tatal sau, pe fiica de mama sa, pe nora de soacra sa.
Mat 10:36  ?i du?manii omului (vor fi) casnicii lui.
Mat 10:37  Cel ce iube?te pe tata ori pe mama mai mult decât pe Mine nu este vrednic de Mine; cel ce iube?te pe fiu ori pe fiica mai mult decât pe Mine nu este vrednic de Mine.
Mat 10:38  ?i cel ce nu-?i ia crucea ?i nu-Mi urmeaza Mie nu este vrednic de Mine.
Mat 10:39  Cine ?ine la sufletul lui îl va pierde, iar cine-?i pierde sufletul lui pentru Mine îl va gasi.
Mat 10:40  Cine va prime?te pe voi pe Mine Ma prime?te, ?i cine Ma prime?te pe Mine prime?te pe Cel ce M-a trimis pe Mine.
Mat 10:41  Cine prime?te prooroc în nume de prooroc plata de prooroc va lua, ?i cine prime?te pe un drept în nume de drept rasplata dreptului va lua.
Mat 10:42  ?i cel ce va da de baut unuia dintre ace?tia mici numai un pahar cu apa rece, în nume de ucenic, adevarat graiesc voua: nu va pierde plata sa.
Mat 11:1  Sfâr?ind Iisus de dat aceste înva?aturi celor doisprezece ucenici ai Sai, a trecut de acolo ca sa înve?e ?i sa propovaduiasca mai departe prin ceta?ile lor.
Mat 11:2  ?i auzind Ioan, în închisoare, despre faptele lui Hristos, ?i trimi?ând pe doi dintre ucenicii sai, au zis Lui:
Mat 11:3  Tu e?ti Cel ce vine, sau sa a?teptam pe altul?
Mat 11:4  ?i Iisus, raspunzând, le-a zis: Merge?i ?i spune?i lui Ioan cele ce auzi?i ?i vede?i:
Mat 11:5  Orbii î?i capata vederea ?i ?chiopii umbla, lepro?ii se cura?esc ?i surzii aud, mor?ii înviaza ?i saracilor li se bineveste?te.
Mat 11:6  ?i fericit este acela care nu se va sminti întru Mine.
Mat 11:7  Dupa plecarea acestora, Iisus a început sa vorbeasca mul?imilor despre Ioan: Ce-a?i ie?it sa vede?i în pustie? Au trestie clatinata de vânt?
Mat 11:8  Dar de ce a?i ie?it? Sa vede?i un om îmbracat în haine moi? Iata, cei ce poarta haine moi sunt în casele regilor.
Mat 11:9  Atunci de ce-a?i ie?it? Sa vede?i un prooroc? Da, zic voua, ?i mai mult decât un prooroc.
Mat 11:10  Ca el este acela despre care s-a scris: "Iata Eu trimit, înaintea fe?ei Tale, pe îngerul Meu, care va pregati calea Ta, înaintea Ta".
Mat 11:11  Adevarat zic voua: Nu s-a ridicat între cei nascu?i din femei unul mai mare decât Ioan Botezatorul; totu?i cel mai mic în împara?ia cerurilor este mai mare decât el.
Mat 11:12  Din zilele lui Ioan Botezatorul pâna acum împara?ia cerurilor se ia prin straduin?a ?i cei ce se silesc pun mâna pe ea.
Mat 11:13  To?i proorocii ?i Legea au proorocit pâna la Ioan.
Mat 11:14  ?i daca voi?i sa în?elege?i, el este Ilie, cel ce va sa vina.
Mat 11:15  Cine are urechi de auzit sa auda.
Mat 11:16  Dar cu cine voi asemana neamul acesta? Este asemenea copiilor care ?ed în pie?e ?i striga catre al?ii,
Mat 11:17  Zicând: V-am cântat din fluier ?i n-a?i jucat; v-am cântat de jale ?i nu v-a?i tânguit.
Mat 11:18  Caci a venit Ioan, nici mâncând, nici bând, ?i spun: Are demon.
Mat 11:19  A venit Fiul Omului, mâncând ?i bând ?i spun: Iata om mâncacios ?i bautor de vin, prieten al vame?ilor ?i al pacato?ilor. Dar în?elepciunea s-a dovedit dreapta din faptele ei.
Mat 11:20  Atunci a început Iisus sa mustre ceta?ile în care se facusera cele mai multe minuni ale Sale, caci nu s-au pocait.
Mat 11:21  Vai ?ie, Horazine, vai ?ie, Betsaida, ca daca în Tir ?i în Sidon s-ar fi facut minunile ce s-au facut în voi, de mult, în sac ?i în cenu?a, s-ar fi pocait.
Mat 11:22  Dar zic voua: Tirului ?i Sidonului le va fi mai u?or în ziua judeca?ii, decât voua.
Mat 11:23  ?i tu, Capernaume: N-ai fost înal?at pâna la cer? Pâna la iad te vei coborî. Caci de s-ar fi facut în Sodoma minunile ce s-au facut în tine, ar fi ramas pâna astazi.
Mat 11:24  Dar zic voua ca pamântului Sodomei îi va fi mai u?or în ziua judeca?ii decât ?ie.
Mat 11:25  În vremea aceea, raspunzând, Iisus a zis: Te slavesc pe Tine, Parinte, Doamne al cerului ?i al pamântului, caci ai ascuns acestea de cei în?elep?i ?i pricepu?i ?i le-ai descoperit pruncilor.
Mat 11:26  Da, Parinte, caci a?a a fost bunavoirea înaintea Ta.
Mat 11:27  Toate Mi-au fost date de catre Tatal Meu ?i nimeni nu cunoa?te pe Fiul, decât numai Tatal, nici pe Tatal nu-L cunoa?te nimeni, decât numai Fiul ?i cel caruia va voi Fiul sa-i descopere.
Mat 11:28  Veni?i la Mine to?i cei osteni?i ?i împovara?i ?i Eu va voi odihni pe voi.
Mat 11:29  Lua?i jugul Meu asupra voastra ?i înva?a?i-va de la Mine, ca sunt blând ?i smerit cu inima ?i ve?i gasi odihna sufletelor voastre.
Mat 11:30  Caci jugul Meu e bun ?i povara Mea este u?oara.
Mat 12:1  În vremea aceea, mergea Iisus, într-o zi de sâmbata, printre semanaturi, iar ucenicii Lui au flamânzit ?i au început sa smulga spice ?i sa manânce.
Mat 12:2  Vazând aceasta, fariseii au zis Lui: Iata, ucenicii Tai fac ceea ce nu se cuvine sa faca sâmbata.
Mat 12:3  Iar El le-a zis: Au n-a?i citit ce-a facut David când a flamânzit, el ?i cei ce erau cu el?
Mat 12:4  Cum a intrat în casa Domnului ?i a mâncat pâinile punerii înainte, care nu se cuveneau lui sa le manânce, nici celor ce erau cu el, ci numai preo?ilor?
Mat 12:5  Sau n-a?i citit în Lege ca preo?ii, sâmbata, în templu, calca sâmbata ?i sunt fara de vina?
Mat 12:6  Ci graiesc voua ca mai mare decât templul este aici.
Mat 12:7  Daca ?tia?i ce înseamna: Mila voiesc iar nu jertfa, n-a?i fi osândit pe cei nevinova?i.
Mat 12:8  Ca Domn este ?i al sâmbetei Fiul Omului.
Mat 12:9  ?i trecând de acolo, a venit în sinagoga lor.
Mat 12:10  ?i iata un om având mâna uscata. ?i L-au întrebat, zicând: Cade-se, oare, a vindeca sâmbata? Ca sa-L învinuiasca.
Mat 12:11  El le-a zis: Cine va fi între voi omul care va avea o oaie ?i, de va cadea ea sâmbata în groapa, nu o va apuca ?i o va scoate?
Mat 12:12  Cu cât se deosebe?te omul de oaie! De aceea se cade a face bine sâmbata.
Mat 12:13  Atunci i-a zis omului: Întinde mâna ta. El a întins-o ?i s-a facut sanatoasa ca ?i cealalta.
Mat 12:14  ?i ie?ind, fariseii s-au sfatuit împotriva Lui cum sa-L piarda.
Mat 12:15  Iisus însa, cunoscându-i, S-a dus de acolo. ?i mul?i au venit dupa El ?i i-a vindecat pe to?i.
Mat 12:16  Dar le-a poruncit ca sa nu-L dea în vileag,
Mat 12:17  Ca sa se împlineasca ceea ce s-a spus prin Isaia proorocul, care zice:
Mat 12:18  "Iata Fiul Meu pe Care L-am ales, iubitul Meu întru Care a binevoit sufletul Meu; pune-voi Duhul Meu peste El ?i judecata neamurilor va vesti.
Mat 12:19  Nu se va certa, nici nu va striga, nu va auzi nimeni, pe uli?e, glasul Lui.
Mat 12:20  Trestie strivita nu va frânge ?i fe?tila fumegânda nu va stinge, pâna ce nu va scoate, spre biruin?a, judecata.
Mat 12:21  ?i în numele Lui vor nadajdui neamurile."
Mat 12:22  Atunci au adus la El pe un demonizat, orb ?i mut, ?i l-a vindecat, încât cel orb ?i mut vorbea ?i vedea.
Mat 12:23  Mul?imile toate se mirau zicând: Nu este, oare, Acesta, Fiul lui David?
Mat 12:24  Fariseii însa, auzind, ziceau: Acesta nu scoate pe demoni decât cu Beelzebul, capetenia demonilor.
Mat 12:25  Cunoscând gândurile lor, Iisus le-a zis: Orice împara?ie care se dezbina în sine se pustie?te, orice cetate sau casa care se dezbina în sine nu va dainui.
Mat 12:26  Daca satana scoate pe satana, s-a dezbinat în sine; dar atunci cum va dainui împara?ia lui?
Mat 12:27  ?i daca Eu scot pe demoni cu Beelzebul, feciorii vo?tri cu cine îi scot? De aceea ei va vor fi judecatori.
Mat 12:28  Iar daca Eu cu Duhul lui Dumnezeu scot pe demoni, iata a ajuns la voi împara?ia lui Dumnezeu.
Mat 12:29  Cum poate cineva sa intre în casa celui tare ?i sa-i jefuiasca lucrurile, daca nu va lega întâi pe cel tare ?i pe urma sa-i prade casa?
Mat 12:30  Cine nu este cu Mine este împotriva Mea ?i cine nu aduna cu Mine risipe?te.
Mat 12:31  De aceea va zic: Orice pacat ?i orice hula se va ierta oamenilor, dar hula împotriva Duhului nu se va ierta.
Mat 12:32  Celui care va zice cuvânt împotriva Fiului Omului, se va ierta lui; dar celui care va zice împotriva Duhului Sfânt, nu i se va ierta lui, nici în veacul acesta, nici în cel ce va sa fie.
Mat 12:33  Ori spune?i ca pomul este bun ?i rodul lui e bun, ori spune?i ca pomul e rau ?i rodul lui e rau, caci dupa roada se cunoa?te pomul.
Mat 12:34  Pui de vipere, cum pute?i sa grai?i cele bune, odata ce sunte?i rai? Caci din prisosul inimii graie?te gura.
Mat 12:35  Omul cel bun din comoara lui cea buna scoate afara cele bune, pe când omul cel rau, din comoara lui cea rea scoate afara cele rele.
Mat 12:36  Va spun ca pentru orice cuvânt de?ert, pe care-l vor rosti, oamenii vor da socoteala în ziua judeca?ii.
Mat 12:37  Caci din cuvintele tale vei fi gasit drept, ?i din cuvintele tale vei fi osândit.
Mat 12:38  Atunci I-au raspuns unii dintre carturari ?i farisei, zicând: Înva?atorule, voim sa vedem de la Tine un semn.
Mat 12:39  Iar El, raspunzând, le-a zis: Neam viclean ?i desfrânat cere semn, dar semn nu i se va da, decât semnul lui Iona proorocul.
Mat 12:40  Ca precum a fost Iona în pântecele chitului trei zile ?i trei nop?i, a?a va fi ?i Fiul Omului în inima pamântului trei zile ?i trei nop?i.
Mat 12:41  Barba?ii din Ninive se vor scula la judecata cu neamul acesta ?i-l vor osândi, ca s-au pocait la propovaduirea lui Iona; iata aici este mai mult decât Iona.
Mat 12:42  Regina de la miazazi se va scula la judecata cu neamul acesta ?i-l va osândi, caci a venit de la marginile pamântului ca sa asculte în?elepciunea lui Solomon, ?i iata aici este mai mult decât Solomon.
Mat 12:43  ?i când duhul necurat a ie?it din om, umbla prin locuri fara apa, cautând odihna ?i nu gase?te.
Mat 12:44  Atunci zice: Ma voi întoarce la casa mea de unde am ie?it; ?i venind, o afla golita, maturata ?i împodobita.
Mat 12:45  Atunci se duce ?i ia cu sine alte ?apte duhuri mai rele decât el ?i, intrând, sala?luiesc aici ?i se fac cele de pe urma ale omului aceluia mai rele decât cele dintâi. A?a va fi ?i cu acest neam viclean.
Mat 12:46  ?i înca vorbind El mul?imilor, iata mama ?i fra?ii Lui stateau afara, cautând sa vorbeasca cu El.
Mat 12:47  Cineva I-a zis: Iata mama Ta ?i fra?ii Tai stau afara, cautând sa-?i vorbeasca.
Mat 12:48  Iar El i-a zis: Cine este mama Mea ?i cine sunt fra?ii Mei?
Mat 12:49  ?i, întinzând mâna catre ucenicii Sai, a zis: Iata mama Mea ?i fra?ii Mei.
Mat 12:50  Ca oricine va face voia Tatalui Meu Celui din ceruri, acela îmi este frate ?i sora ?i mama.
Mat 13:1  În ziua aceea, ie?ind Iisus din casa, ?edea lânga mare.
Mat 13:2  ?i s-au adunat la El mul?imi multe, încât intrând în corabie ?edea în ea ?i toata mul?imea sta pe ?arm.
Mat 13:3  ?i le-a grait lor multe, în pilde, zicând: Iata a ie?it semanatorul sa semene.
Mat 13:4  ?i pe când semana, unele semin?e au cazut lânga drum ?i au venit pasarile ?i le-au mâncat.
Mat 13:5  Altele au cazut pe loc pietros, unde n-aveau pamânt mult ?i îndata au rasarit, ca n-aveau pamânt adânc;
Mat 13:6  Iar când s-a ivit soarele, s-au palit de ar?i?a ?i, neavând radacina, s-au uscat.
Mat 13:7  Altele au cazut între spini, dar spinii au crescut ?i le-au înabu?it.
Mat 13:8  Altele au cazut pe pamânt bun ?i au dat rod: una o suta, alta ?aizeci, alta treizeci.
Mat 13:9  Cine are urechi de auzit sa auda.
Mat 13:10  ?i ucenicii, apropiindu-se de El, I-au zis: De ce le vorbe?ti lor în pilde?
Mat 13:11  Iar El, raspunzând, le-a zis: Pentru ca voua vi s-a dat sa cunoa?te?i tainele împara?iei cerurilor, pe când acestora nu li s-a dat.
Mat 13:12  Caci celui ce are i se va da ?i-i va prisosi, iar de la cel ce nu are, ?i ce are i se va lua.
Mat 13:13  De aceea le vorbesc în pilde, ca, vazând, nu vad ?i, auzind, nu aud, nici nu în?eleg.
Mat 13:14  ?i se împline?te cu ei proorocia lui Isaia, care zice: "Cu urechile ve?i auzi, dar nu ve?i în?elege, ?i cu ochii va ve?i uita, dar nu ve?i vedea".
Mat 13:15  Caci inima acestui popor s-a învârto?at ?i cu urechile aude greu ?i ochii lui s-au închis, ca nu cumva sa vada cu ochii ?i sa auda cu urechile ?i cu inima sa în?eleaga ?i sa se întoarca, ?i Eu sa-i tamaduiesc pe ei.
Mat 13:16  Dar ferici?i sunt ochii vo?tri ca vad ?i urechile voastre ca aud.
Mat 13:17  Caci adevarat graiesc voua ca mul?i prooroci ?i drep?i au dorit sa vada cele ce privi?i voi, ?i n-au vazut, ?i sa auda cele ce auzi?i voi, ?i n-au auzit.
Mat 13:18  Voi, deci, asculta?i pilda semanatorului:
Mat 13:19  De la oricine aude cuvântul împara?iei ?i nu-l în?elege, vine cel viclean ?i rape?te ce s-a semanat în inima lui; aceasta este samân?a semanata lânga drum.
Mat 13:20  Cea semanata pe loc pietros este cel care aude cuvântul ?i îndata îl prime?te cu bucurie,
Mat 13:21  Dar nu are radacina în sine, ci ?ine pâna la o vreme ?i, întâmplându-se strâmtorare sau prigoana pentru cuvânt, îndata se sminte?te.
Mat 13:22  Cea semanata în spini este cel care aude cuvântul, dar grija acestei lumi ?i în?elaciunea avu?iei înabu?a cuvântul ?i îl face neroditor.
Mat 13:23  Iar samân?a semanata în pamânt bun este cel care aude cuvântul ?i-l în?elege, deci care aduce rod ?i face: unul o suta, altul ?aizeci, altul treizeci.
Mat 13:24  Alta pilda le-a pus lor înainte, zicând: Asemenea este împara?ia cerurilor omului care a semanat samân?a buna în ?arina sa.
Mat 13:25  Dar pe când oamenii dormeau, a venit vrajma?ul lui, a semanat neghina printre grâu ?i s-a dus.
Mat 13:26  Iar daca a crescut paiul ?i a facut rod, atunci s-a aratat ?i neghina.
Mat 13:27  Venind slugile stapânului casei, i-au zis: Doamne, n-ai semanat tu, oare, samân?a buna în ?arina ta? De unde dar are neghina?
Mat 13:28  Iar el le-a raspuns: Un om vrajma? a facut aceasta. Slugile i-au zis: Voie?ti deci sa ne ducem ?i s-o plivim?
Mat 13:29  El însa a zis: Nu, ca nu cumva, plivind neghina, sa smulge?i odata cu ea ?i grâul.
Mat 13:30  Lasa?i sa creasca împreuna ?i grâul ?i neghina, pâna la seceri?, ?i la vremea seceri?ului voi zice seceratorilor: Plivi?i întâi neghina ?i lega?i-o în snopi ca s-o ardem, iar grâul aduna?i-l în jitni?a mea.
Mat 13:31  O alta pilda le-a pus înainte, zicând: Împara?ia cerurilor este asemenea grauntelui de mu?tar, pe care, luându-l, omul l-a semanat în ?arina sa,
Mat 13:32  ?i care este mai mic decât toate semin?ele, dar când a crescut este mai mare decât toate legumele ?i se face pom, încât vin pasarile cerului ?i se sala?luiesc în ramurile lui.
Mat 13:33  Alta pilda le-a spus lor: Asemenea este împara?ia cerurilor aluatului pe care, luându-l, o femeie l-a ascuns în trei masuri de faina, pâna ce s-a dospit toata.
Mat 13:34  Toate acestea le-a vorbit Iisus mul?imilor în pilde, ?i fara pilda nu le graia nimic,
Mat 13:35  Ca sa se împlineasca ce s-a spus prin proorocul care zice: "Deschide-voi în pilde gura Mea, spune-voi cele ascunse de la întemeierea lumii".
Mat 13:36  Dupa aceea, lasând mul?imile, a venit în casa, iar ucenicii Lui s-au apropiat de El, zicând: Lamure?te-ne noua pilda cu neghina din ?arina.
Mat 13:37  El, raspunzând, le-a zis: Cel ce seamana samân?a cea buna este Fiul Omului.
Mat 13:38  ?arina este lumea; samân?a cea buna sunt fiii împara?iei; iar neghina sunt fiii celui rau.
Mat 13:39  Du?manul care a semanat-o este diavolul; seceri?ul este sfâr?itul lumii, iar seceratorii sunt îngerii.
Mat 13:40  ?i, dupa cum se alege neghina ?i se arde în foc, a?a va fi la sfâr?itul veacului.
Mat 13:41  Trimite-va Fiul Omului pe îngerii Sai, vor culege din împara?ia Lui toate smintelile ?i pe cei ce fac faradelegea,
Mat 13:42  ?i-i vor arunca pe ei în cuptorul cu foc; acolo va fi plângerea ?i scrâ?nirea din?ilor.
Mat 13:43  Atunci cei drep?i vor straluci ca soarele în împara?ia Tatalui lor. Cel ce are urechi de auzit sa auda.
Mat 13:44  Asemenea este împara?ia cerurilor cu o comoara ascunsa în ?arina, pe care, gasind-o un om, a ascuns-o, ?i de bucuria ei se duce ?i vinde tot ce are ?i cumpara ?arina aceea.
Mat 13:45  Iara?i asemenea este împara?ia cerurilor cu un negu?ator care cauta margaritare bune.
Mat 13:46  ?i aflând un margaritar de mult pre?, s-a dus, a vândut toate câte avea ?i l-a cumparat.
Mat 13:47  Asemenea este iara?i împara?ia cerurilor cu un navod aruncat în mare ?i care aduna tot felul de pe?ti.
Mat 13:48  Iar când s-a umplut, l-au tras pescarii la mal ?i, ?ezând, au ales în vase pe cei buni, iar pe cei rai i-au aruncat afara.
Mat 13:49  A?a va fi la sfâr?itul veacului: vor ie?i îngerii ?i vor despar?i pe cei rai din mijlocul celor drep?i.
Mat 13:50  ?i îi vor arunca în cuptorul cel de foc; acolo va fi plângerea ?i scrâ?nirea din?ilor.
Mat 13:51  În?eles-a?i toate acestea? Zis-au Lui: Da, Doamne.
Mat 13:52  Iar El le-a zis: De aceea, orice carturar cu înva?atura despre împara?ia cerurilor este asemenea unui om gospodar, care scoate din vistieria sa noi ?i vechi.
Mat 13:53  Iar dupa ce Iisus a sfâr?it aceste pilde, a trecut de acolo.
Mat 13:54  ?i venind în patria Sa, îi înva?a pe ei în sinagoga lor, încât ei erau uimi?i ?i ziceau: De unde are El în?elepciunea aceasta ?i puterile?
Mat 13:55  Au nu este Acesta fiul teslarului? Au nu se nume?te mama Lui Maria ?i fra?ii (verii) Lui: Iacov ?i Iosif ?i Simon ?i Iuda?
Mat 13:56  ?i surorile (veri?oarele) Lui au nu sunt toate la noi? Deci, de unde are El toate acestea?
Mat 13:57  ?i se sminteau întru El. Iar Iisus le-a zis: Nu este prooroc dispre?uit decât în patria lui ?i în casa lui.
Mat 13:58  ?i n-a facut acolo multe minuni, din pricina necredin?ei lor.
Mat 14:1  În vremea aceea, a auzit tetrarhul Irod de vestea ce se dusese despre Iisus.
Mat 14:2  ?i a zis slujitorilor sai: Acesta este Ioan Botezatorul; el s-a sculat din mor?i ?i de aceea se fac minuni prin el.
Mat 14:3  Caci Irod, prinzând pe Ioan, l-a legat ?i l-a pus în temni?a, pentru Irodiada, femeia lui Filip, fratele sau.
Mat 14:4  Caci Ioan îi zicea lui: Nu ?i se cuvine s-o ai de so?ie.
Mat 14:5  ?i voind sa-l ucida, s-a temut de mul?ime, ca-l socotea pe el ca prooroc.
Mat 14:6  Iar praznuind Irod ziua lui de na?tere, fiica Irodiadei a jucat în mijloc ?i i-a placut lui Irod.
Mat 14:7  De aceea, cu juramânt i-a fagaduit sa-i dea orice va cere.
Mat 14:8  Iar ea, îndemnata fiind de mama sa, a zis: Da-mi, aici pe tipsie, capul lui Ioan Botezatorul.
Mat 14:9  ?i regele s-a întristat, dar, pentru juramânt ?i pentru cei care ?edeau cu el la masa, a poruncit sa i se dea.
Mat 14:10  ?i a trimis ?i a taiat capul lui Ioan, în temni?a.
Mat 14:11  ?i capul lui a fost adus pe tipsie ?i a fost dat fetei, iar ea l-a dus mamei sale.
Mat 14:12  ?i, venind ucenicii lui, au luat trupul lui ?i l-au înmormântat ?i s-au dus sa dea de ?tire lui Iisus.
Mat 14:13  Iar Iisus, auzind, S-a dus de acolo singur, cu corabia, în loc pustiu dar, aflând, mul?imile au venit dupa El, pe jos, din ceta?i.
Mat 14:14  ?i ie?ind, a vazut mul?ime mare ?i I S-a facut mila de ei ?i a vindecat pe bolnavii lor.
Mat 14:15  Iar când s-a facut seara, ucenicii au venit la El ?i I-au zis: locul este pustiu ?i vremea iata a trecut; deci, da drumul mul?imilor ca sa se duca în sate, sa-?i cumpere mâncare.
Mat 14:16  Iisus însa le-a raspuns: N-au trebuin?a sa se duca; da?i-le voi sa manânce.
Mat 14:17  Iar ei I-au zis: Nu avem aici decât cinci pâini ?i doi pe?ti.
Mat 14:18  ?i El a zis: Aduce?i-Mi-le aici.
Mat 14:19  ?i poruncind sa se a?eze mul?imile pe iarba ?i luând cele cinci pâini ?i cei doi pe?ti ?i privind la cer, a binecuvântat ?i, frângând, a dat ucenicilor pâinile, iar ucenicii mul?imilor.
Mat 14:20  ?i au mâncat to?i ?i s-au saturat ?i au strâns rama?i?ele de farâmituri, douasprezece co?uri pline.
Mat 14:21  Iar cei ce mâncasera erau ca la cinci mii de barba?i, afara de femei ?i de copii.
Mat 14:22  ?i îndata Iisus a silit pe ucenici sa intre în corabie ?i sa treaca înaintea Lui, pe ?armul celalalt, pâna ce El va da drumul mul?imilor.
Mat 14:23  Iar dând drumul mul?imilor, S-a suit în munte, ca sa Se roage singur. ?i, facându-se seara, era singur acolo.
Mat 14:24  Iar corabia era acum la multe stadii departe de pamânt, fiind învaluita de valuri, caci vântul era împotriva.
Mat 14:25  Iar la a patra straja din noapte, a venit la ei Iisus, umblând pe mare.
Mat 14:26  Vazându-L umblând pe mare, ucenicii s-au înspaimântat, zicând ca e naluca ?i de frica au strigat.
Mat 14:27  Dar El le-a vorbit îndata, zicându-le: Îndrazni?i, Eu sunt; nu va teme?i!
Mat 14:28  Iar Petru, raspunzând, a zis: Doamne, daca e?ti Tu, porunce?te sa vin la Tine pe apa.
Mat 14:29  El i-a zis: Vino. Iar Petru, coborându-se din corabie, a mers pe apa ?i a venit catre Iisus.
Mat 14:30  Dar vazând vântul, s-a temut ?i, începând sa se scufunde, a strigat, zicând: Doamne, scapa-ma!
Mat 14:31  Iar Iisus, întinzând îndata mâna, l-a apucat ?i a zis: Pu?in credinciosule, pentru ce te-ai îndoit?
Mat 14:32  ?i suindu-se ei în corabie, s-a potolit vântul.
Mat 14:33  Iar cei din corabie I s-au închinat, zicând: Cu adevarat Tu e?ti Fiul lui Dumnezeu.
Mat 14:34  ?i, trecând dincolo, au venit în pamântul Ghenizaretului.
Mat 14:35  ?i, cunoscându-L, oamenii locului aceluia au trimis în tot acel ?inut ?i au adus la El pe to?i bolnavii.
Mat 14:36  ?i-L rugau ca numai sa se atinga de poala hainei Lui; ?i câ?i se atingeau se vindecau.
Mat 15:1  Atunci au venit din Ierusalim, la Iisus, fariseii ?i carturarii, zicând:
Mat 15:2  Pentru ce ucenicii Tai calca datina batrânilor? Caci nu-?i spala mâinile când manânca pâine.
Mat 15:3  Iar El, raspunzând, le-a zis: De ce ?i voi calca?i porunca lui Dumnezeu pentru datina voastra?
Mat 15:4  Caci Dumnezeu a zis: Cinste?te pe tatal tau ?i pe mama ta, iar cine va blestema pe tata sau pe mama, cu moarte sa se sfâr?easca.
Mat 15:5  Voi însa spune?i: Cel care va zice tatalui sau sau mamei sale: Cu ce te-a? fi putut ajuta este daruit lui Dumnezeu,
Mat 15:6  Acela nu va cinsti pe tatal sau sau pe mama sa; ?i a?i desfiin?at cuvântul lui Dumnezeu pentru datina voastra.
Mat 15:7  Fa?arnicilor, bine a proorocit despre voi Isaia, când a zis:
Mat 15:8  "Poporul acesta Ma cinste?te cu buzele, dar inima lor este departe de Mine.
Mat 15:9  ?i zadarnic Ma cinstesc ei, înva?ând înva?aturi ce sunt porunci ale oamenilor".
Mat 15:10  ?i chemând la Sine mul?imile, le-a zis: Asculta?i ?i în?elege?i:
Mat 15:11  Nu ceea ce intra pe gura spurca pe om, ci ceea ce iese din gura, aceea spurca pe om.
Mat 15:12  Atunci, apropiindu-se, ucenicii I-au zis: ?tii ca fariseii, auzind cuvântul, s-au scandalizat?
Mat 15:13  Iar El, raspunzând, a zis: Orice rasad pe care nu l-a sadit Tatal Meu cel ceresc, va fi smuls din radacina.
Mat 15:14  Lasa?ii pe ei; sunt calauze oarbe, orbilor; ?i daca orb pe orb va calauzi, amândoi vor cadea în groapa.
Mat 15:15  ?i Petru, raspunzând, I-a zis: Lamure?te-ne noua pilda aceasta.
Mat 15:16  El a zis: Acum ?i voi sunte?i nepricepu?i?
Mat 15:17  Nu în?elege?i ca tot ce intra în gura se duce în pântece ?i se arunca afara?
Mat 15:18  Iar cele ce ies din gura pornesc din inima ?i acelea spurca pe om.
Mat 15:19  Caci din inima ies: gânduri rele, ucideri, adultere, desfrânari, furti?aguri, marturii mincinoase, hule.
Mat 15:20  Acestea sunt care spurca pe om, dar a mânca cu mâini nespalate nu spurca pe om.
Mat 15:21  ?i ie?ind de acolo, a plecat Iisus în par?ile Tirului ?i ale Sidonului.
Mat 15:22  ?i iata o femeie cananeianca, din acele ?inuturi, ie?ind striga, zicând: Miluie?te-ma, Doamne, Fiul lui David! Fiica mea este rau chinuita de demon.
Mat 15:23  El însa nu i-a raspuns nici un cuvânt; ?i apropiindu-se, ucenicii Lui Îl rugau, zicând: Sloboze?te-o, ca striga în urma noastra.
Mat 15:24  Iar El, raspunzând, a zis: Nu sunt trimis decât catre oile cele pierdute ale casei lui Israel.
Mat 15:25  Iar ea, venind, s-a închinat Lui, zicând: Doamne, ajuta-ma.
Mat 15:26  El însa, raspunzând, i-a zis: Nu este bine sa iei pâinea copiilor ?i s-o arunci câinilor.
Mat 15:27  Dar ea a zis: Da, Doamne, dar ?i câinii manânca din farâmiturile care cad de la masa stapânilor lor.
Mat 15:28  Atunci, raspunzând, Iisus i-a zis: O, femeie, mare este credin?a ta; fie ?ie dupa cum voie?ti. ?i s-a tamaduit fiica ei în ceasul acela.
Mat 15:29  ?i trecând Iisus de acolo, a venit lânga Marea Galileii ?i, suindu-Se în munte, a ?ezut acolo.
Mat 15:30  ?i mul?imi multe au venit la El, având cu ei ?chiopi, orbi, mu?i, ciungi, ?i mul?i al?ii ?i i-au pus la picioarele Lui, iar El i-a vindecat.
Mat 15:31  Încât mul?imea se minuna vazând pe mu?i vorbind, pe ciungi sanato?i, pe ?chiopi umblând ?i pe orbi vazând, ?i slaveau pe Dumnezeul lui Israel.
Mat 15:32  Iar Iisus, chemând la Sine pe ucenicii Sai, a zis: Mila îmi este de mul?ime, ca iata sunt trei zile de când a?teapta lânga Mine ?i n-au ce sa manânce; ?i sa-i slobozesc flamânzi nu voiesc, ca sa nu se istoveasca pe drum.
Mat 15:33  ?i ucenicii I-au zis: De unde sa avem noi, în pustie, atâtea pâini, cât sa saturam atâta mul?ime?
Mat 15:34  ?i Iisus i-a întrebat: Câte pâini ave?i? Ei au raspuns: ?apte ?i pu?ini pe?ti?ori.
Mat 15:35  ?i poruncind mul?imii sa ?ada jos pe pamânt,
Mat 15:36  A luat cele ?apte pâini ?i pe?ti ?i, mul?umind, a frânt ?i a dat ucenicilor, iar ucenicii mul?imilor.
Mat 15:37  ?i au mâncat to?i ?i s-au saturat ?i au luat ?apte co?uri pline, cu rama?i?e de farâmituri.
Mat 15:38  Iar cei ce au mâncat erau ca la patru mii de barba?i, afara de copii ?i de femei.
Mat 15:39  Dupa aceea a dat drumul mul?imilor, S-a suit în corabie ?i S-a dus în ?inutul Magdala.
Mat 16:1  ?i apropiindu-se fariseii ?i saducheii ?i ispitindu-L, I-au cerut sa le arate semn din cer.
Mat 16:2  Iar El, raspunzând, le-a zis: Când se face seara, zice?i: Mâine va fi timp frumos, pentru ca e cerul ro?u.
Mat 16:3  Iar diminea?a zice?i: Astazi va fi furtuna, pentru ca cerul este ro?u-posomorât. Fa?arnicilor, fa?a cerului ?ti?i s-o judeca?i, dar semnele vremilor nu pute?i!
Mat 16:4  Neam viclean ?i adulter cere semn ?i semn nu se va da lui, decât numai semnul lui Iona. ?i lasându-i, a plecat.
Mat 16:5  ?i venind ucenicii pe celalalt ?arm, au uitat sa ia pâini.
Mat 16:6  Iar Iisus le-a zis: Lua?i aminte ?i feri?i-va de aluatul fariseilor ?i al saducheilor.
Mat 16:7  Iar ei cugetau în sinea lor, zicând: Aceasta, pentru ca n-am luat pâine.
Mat 16:8  Dar Iisus, cunoscându-le gândul, a zis: Ce cugeta?i în voi în?iva, pu?in credincio?ilor, ca n-a?i luat pâine?
Mat 16:9  Tot nu în?elege?i, nici nu va aduce?i aminte de cele cinci pâini, la cei cinci mii de oameni, ?i câte co?uri a?i luat?
Mat 16:10  Nici de cele ?apte pâini, la cei patru mii de oameni, ?i câte co?uri a?i luat?
Mat 16:11  Cum nu în?elege?i ca nu despre pâini v-am zis? Ci feri?i-va de aluatul fariseilor ?i al saducheilor.
Mat 16:12  Atunci au în?eles ca nu le-a spus sa se fereasca de aluatul pâinii, ci de înva?atura fariseilor ?i a saducheilor.
Mat 16:13  ?i venind Iisus în par?ile Cezareii lui Filip, îi întreba pe ucenicii Sai, zicând: Cine zic oamenii ca sunt Eu, Fiul Omului?
Mat 16:14  Iar ei au raspuns: Unii, Ioan Botezatorul, al?ii Ilie, al?ii Ieremia sau unul dintre prooroci.
Mat 16:15  ?i le-a zis: Dar voi cine zice?i ca sunt?
Mat 16:16  Raspunzând Simon Petru a zis: Tu e?ti Hristosul, Fiul lui Dumnezeu Celui viu.
Mat 16:17  Iar Iisus, raspunzând, i-a zis: Fericit e?ti Simone, fiul lui Iona, ca nu trup ?i sânge ?i-au descoperit ?ie aceasta, ci Tatal Meu, Cel din ceruri.
Mat 16:18  ?i Eu î?i zic ?ie, ca tu e?ti Petru ?i pe aceasta piatra voi zidi Biserica Mea ?i por?ile iadului nu o vor birui.
Mat 16:19  ?i î?i voi da cheile împara?iei cerurilor ?i orice vei lega pe pamânt va fi legat ?i în ceruri, ?i orice vei dezlega pe pamânt va fi dezlegat ?i în ceruri.
Mat 16:20  Atunci a poruncit ucenicilor Lui sa nu spuna nimanui ca El este Hristosul.
Mat 16:21  De atunci a început Iisus sa le arate ucenicilor Lui ca El trebuie sa mearga la Ierusalim ?i sa patimeasca multe de la batrâni ?i de la arhierei ?i de la carturari ?i sa fie ucis, ?i a treia zi sa învieze.
Mat 16:22  ?i Petru, luându-L la o parte, a început sa-L dojeneasca, zicându-I: Fie-?i mila de Tine sa nu ?i se întâmple ?ie aceasta.
Mat 16:23  Iar El, întorcându-se, a zis lui Petru: Mergi înapoia Mea, satano! Sminteala Îmi e?ti; ca nu cuge?i cele ale lui Dumnezeu, ci cele ale oamenilor.
Mat 16:24  Atunci Iisus a zis ucenicilor Sai: Daca vrea cineva sa vina dupa Mine, sa se lepede de sine, sa-?i ia crucea ?i sa-Mi urmeze Mie.
Mat 16:25  Ca cine va voi sa-?i scape sufletul îl va pierde; iar cine î?i va pierde sufletul pentru Mine îl va afla.
Mat 16:26  Pentru ca ce-i va folosi omului, daca va câ?tiga lumea întreaga, iar sufletul sau îl va pierde? Sau ce va da omul în schimb pentru sufletul sau?
Mat 16:27  Caci Fiul Omului va sa vina întru slava Tatalui Sau, cu îngerii Sai; ?i atunci va rasplati fiecaruia dupa faptele sale.
Mat 16:28  Adevarat graiesc voua: Sunt unii din cei ce stau aici care nu vor gusta moartea pâna ce nu vor vedea pe Fiul Omului, venind în împara?ia Sa.
Mat 17:1  ?i dupa ?ase zile, Iisus a luat cu Sine pe Petru ?i pe Iacov ?i pe Ioan, fratele lui, ?i i-a dus într-un munte înalt, de o parte.
Mat 17:2  ?i S-a schimbat la fa?a, înaintea lor, ?i a stralucit fa?a Lui ca soarele, iar ve?mintele Lui s-au facut albe ca lumina.
Mat 17:3  ?i iata, Moise ?i Ilie s-au aratat lor, vorbind cu El.
Mat 17:4  ?i, raspunzând, Petru a zis lui Iisus: Doamne, bine este sa fim noi aici; daca voie?ti, voi face aici trei colibe: ?ie una, ?i lui Moise una, ?i lui Ilie una.
Mat 17:5  Vorbind el înca, iata un nor luminos i-a umbrit pe ei, ?i iata glas din nor zicând: "Acesta este Fiul Meu Cel iubit, în Care am binevoit; pe Acesta asculta?i-L".
Mat 17:6  ?i, auzind, ucenicii au cazut cu fa?a la pamânt ?i s-au spaimântat foarte.
Mat 17:7  ?i Iisus S-a apropiat de ei, ?i, atingându-i, le-a zis: Scula?i-va ?i nu va teme?i.
Mat 17:8  ?i, ridicându-?i ochii, nu au vazut pe nimeni, decât numai pe Iisus singur.
Mat 17:9  ?i pe când se coborau din munte, Iisus le-a poruncit, zicând: Nimanui sa nu spune?i ceea ce a?i vazut, pâna când Fiul Omului Se va scula din mor?i.
Mat 17:10  ?i ucenicii L-au întrebat, zicând: Pentru ce dar zic carturarii ca trebuie sa vina mai întâi Ilie?
Mat 17:11  Iar El, raspunzând, a zis: Ilie într-adevar va veni ?i va a?eza la loc toate.
Mat 17:12  Eu însa va spun voua ca Ilie a ?i venit, dar ei nu l-au cunoscut, ci au facut cu el câte au voit; a?a ?i Fiul Omului va patimi de la ei.
Mat 17:13  Atunci au în?eles ucenicii ca Iisus le-a vorbit despre Ioan Botezatorul.
Mat 17:14  ?i mergând ei spre mul?ime, s-a apropiat de El un om, cazându-I în genunchi,
Mat 17:15  ?i zicând: Doamne, miluie?te pe fiul meu ca este lunatic ?i patime?te rau, caci adesea cade în foc ?i adesea în apa.
Mat 17:16  ?i l-am dus la ucenicii Tai ?i n-au putut sa-l vindece.
Mat 17:17  Iar Iisus, raspunzând, a zis: O, neam necredincios ?i îndaratnic, pâna când voi fi cu voi? Pâna când va voi suferi pe voi? Aduce?i-l aici la Mine.
Mat 17:18  ?i Iisus l-a certat ?i demonul a ie?it din el ?i copilul s-a vindecat din ceasul acela.
Mat 17:19  Atunci, apropiindu-se ucenicii de Iisus, I-au zis de o parte: De ce noi n-am putut sa-l scoatem?
Mat 17:20  Iar Iisus le-a raspuns: Pentru pu?ina voastra credin?a. Caci adevarat graiesc voua: Daca ve?i avea credin?a în voi cât un graunte de mu?tar, ve?i zice muntelui acestuia: Muta-te de aici dincolo, ?i se va muta; ?i nimic nu va fi voua cu neputin?a.
Mat 17:21  Dar acest neam de demoni nu iese decât numai cu rugaciune ?i cu post.
Mat 17:22  Pe când strabateau Galileea, Iisus le-a spus: Fiul Omului va sa fie dat în mâinile oamenilor.
Mat 17:23  ?i-L vor omorî, dar a treia zi va învia. ?i ei s-au întristat foarte!
Mat 17:24  Venind ei în Capernaum, s-au apropiat de Petru cei ce strâng darea (pentru Templu) ?i i-au zis: Înva?atorul vostru nu plate?te darea?
Mat 17:25  Ba, da! - a zis el. Dar intrând în casa, Iisus i-a luat înainte, zicând: Ce ?i se pare, Simone? Regii pamântului de la cine iau dari sau bir? De la fiii lor sau de la straini?
Mat 17:26  El I-a zis: De la straini. Iisus i-a zis: A?adar, fiii sunt scuti?i.
Mat 17:27  Ci ca sa nu-i smintim pe ei, mergând la mare, arunca undi?a ?i pe?tele care va ie?i întâi, ia-l, ?i, deschizându-i gura, vei gasi un statir (un ban de argint). Ia-l ?i da-l lor pentru Mine ?i pentru tine.
Mat 18:1  În ceasul acela, s-au apropiat ucenicii de Iisus ?i I-au zis: Cine, oare, este mai mare în împara?ia cerurilor?
Mat 18:2  ?i chemând la Sine un prunc, l-a pus în mijlocul lor,
Mat 18:3  ?i a zis: Adevarat zic voua: De nu va ve?i întoarce ?i nu ve?i fi precum pruncii, nu ve?i intra în împara?ia cerurilor.
Mat 18:4  Deci cine se va smeri pe sine ca pruncul acesta, acela este cel mai mare în împara?ia cerurilor.
Mat 18:5  ?i cine va primi un prunc ca acesta în numele Meu, pe Mine Ma prime?te.
Mat 18:6  Iar cine va sminti pe unul dintr-ace?tia mici care cred în Mine, mai bine i-ar fi lui sa i se atârne de gât o piatra de moara ?i sa fie afundat în adâncul marii.
Mat 18:7  Vai lumii, din pricina smintelilor! Ca smintelile trebuie sa vina, dar vai omului aceluia prin care vine sminteala.
Mat 18:8  Iar daca mâna ta sau piciorul tau te sminte?te, taie-l ?i arunca-l de la tine, ca este bine pentru tine sa intri în via?a ciung sau ?chiop, decât, având amândoua mâinile sau amândoua picioarele, sa fii aruncat în focul cel ve?nic.
Mat 18:9  ?i daca ochiul tau te sminte?te, scoate-l ?i arunca-l de la tine, ca mai bine  este pentru tine sa intri în via?a cu un singur ochi, decât, având amândoi ochii, sa fii aruncat în gheena focului.
Mat 18:10  Vede?i sa nu dispre?ui?i pe vreunul din ace?tia mici, ca zic voua: Ca îngerii lor, în ceruri, pururea vad fa?a Tatalui Meu, Care este în ceruri.
Mat 18:11  Caci Fiul Omului a venit sa caute ?i sa mântuiasca pe cel pierdut.
Mat 18:12  Ce vi se pare? Daca un om ar avea o suta de oi ?i una din ele s-ar rataci, nu va lasa, oare, în mun?i pe cele nouazeci ?i noua ?i ducându-se va cauta pe cea ratacita?
Mat 18:13  ?i daca s-ar întâmpla s-o gaseasca, adevar graiesc voua ca se bucura de ea mai mult decât de cele nouazeci ?i noua, care nu s-au ratacit.
Mat 18:14  Astfel nu este vrere înaintea Tatalui vostru, Cel din ceruri, ca sa piara vreunul dintr-ace?tia mici.
Mat 18:15  De-?i va gre?i ?ie fratele tau, mergi, mustra-l pe el între tine ?i el singur. ?i de te va asculta, ai câ?tigat pe fratele tau.
Mat 18:16  Iar de nu te va asculta, ia cu tine înca unul sau doi, ca din gura a doi sau trei martori sa se statorniceasca tot cuvântul.
Mat 18:17  ?i de nu-i va asculta pe ei, spune-l Bisericii; iar de nu va asculta nici de Biserica, sa-?i fie ?ie ca un pagân ?i vame?.
Mat 18:18  Adevarat graiesc voua: Oricâte ve?i lega pe pamânt, vor fi legate ?i în cer, ?i oricâte ve?i dezlega pe pamânt, vor fi dezlegate ?i în cer.
Mat 18:19  Iara?i graiesc voua ca, daca doi dintre voi se vor învoi pe pamânt în privin?a unui lucru pe care îl vor cere, se va da lor de catre Tatal Meu, Care este în ceruri.
Mat 18:20  Ca unde sunt doi sau trei, aduna?i în numele Meu, acolo sunt ?i Eu în mijlocul lor.
Mat 18:21  Atunci Petru, apropiindu-se de El, I-a zis: Doamne, de câte ori va gre?i fa?a de mine fratele meu ?i-i voi ierta lui? Oare pâna de ?apte ori?
Mat 18:22  Zis-a lui  Iisus: Nu zic ?ie pâna de ?apte ori, ci pâna de ?aptezeci de ori câte ?apte.
Mat 18:23  De aceea, asemanatu-s-a împara?ia cerurilor omului împarat care a voit sa se socoteasca cu slugile sale.
Mat 18:24  ?i, începând sa se socoteasca cu ele, i s-a adus un datornic cu zece mii de talan?i.
Mat 18:25  Dar neavând el cu ce sa plateasca, stapânul sau a poruncit sa fie vândut el ?i femeia ?i copii ?i pe toate câte le are, ca sa se plateasca.
Mat 18:26  Deci, cazându-i în genunchi, sluga aceea i se închina, zicând: Doamne, îngaduie?te-ma ?i-?i voi plati ?ie tot.
Mat 18:27  Iar stapânul slugii aceleia, milostivindu-se de el, i-a dat drumul ?i i-a iertat ?i datoria.
Mat 18:28  Dar, ie?ind, sluga aceea a gasit pe unul dintre cei ce slujeau cu el ?i care-i datora o suta de dinari. ?i punând mâna pe el, îl sugruma zicând: Plate?te-mi ce e?ti dator.
Mat 18:29  Deci, cazând cel ce era sluga ca ?i el, îl ruga zicând: Îngaduie?te-ma ?i î?i voi plati.
Mat 18:30  Iar el nu voia, ci, mergând, l-a aruncat în închisoare, pâna ce va plati datoria.
Mat 18:31  Iar celelalte slugi, vazând deci cele petrecute, s-au întristat foarte ?i, venind, au spus stapânului toate cele întâmplate.
Mat 18:32  Atunci, chemându-l stapânul sau îi zise: Sluga vicleana, toata datoria aceea ?i-am iertat-o, fiindca m-ai rugat.
Mat 18:33  Nu se cadea, oare, ca ?i tu sa ai mila de cel împreuna sluga cu tine, precum ?i eu am avut mila de tine?
Mat 18:34  ?i mâniindu-se stapânul lui, l-a dat pe mâna chinuitorilor, pâna ce-i va plati toata datoria.
Mat 18:35  Tot a?a ?i Tatal Meu cel ceresc va va face voua, daca nu ve?i ierta - fiecare fratelui sau - din inimile voastre.
Mat 19:1  Iar dupa ce Iisus a sfâr?it cuvintele acestea, a plecat din Galileea ?i a venit în hotarele Iudeii, dincolo de Iordan.
Mat 19:2  ?i au mers dupa El mul?imi multe ?i i-a vindecat pe ei acolo.
Mat 19:3  ?i s-au apropiat de El fariseii, ispitindu-L ?i zicând: Se cuvine, oare, omului sa-?i lase femeia sa, pentru orice pricina?
Mat 19:4  Raspunzând, El a zis: N-a?i citit ca Cel ce i-a facut de la început i-a facut barbat ?i femeie?
Mat 19:5  ?i a zis: Pentru aceea va lasa omul pe tatal sau ?i pe mama sa ?i se va lipi de femeia sa ?i vor fi amândoi un trup.
Mat 19:6  A?a încât nu mai sunt doi, ci un trup. Deci, ce a împreunat Dumnezeu omul sa nu desparta.
Mat 19:7  Ei I-au zis Lui: Pentru ce, dar, Moise a poruncit sa-i dea carte de despar?ire ?i sa o lase?
Mat 19:8  El le-a zis: Pentru învârto?area inimii voastre, v-a dat voie Moise sa lasa?i pe femeile voastre, dar din început nu a fost a?a.
Mat 19:9  Iar Eu zic voua ca oricine va lasa pe femeia sa, în afara de pricina de desfrânare, ?i se va însura cu alta, savâr?e?te adulter; ?i cine s-a însurat cu cea lasata savâr?e?te adulter.
Mat 19:10  Ucenicii I-au zis: Daca astfel este pricina omului cu femeia, nu este de folos sa se însoare.
Mat 19:11  Iar El le-a zis: Nu to?i pricep cuvântul acesta, ci aceia carora le este dat.
Mat 19:12  Ca sunt fameni care s-au nascut a?a din pântecele mamei lor; sunt fameni pe care oamenii i-au facut fameni, ?i sunt fameni care s-au facut fameni pe ei în?i?i, pentru împara?ia cerurilor. Cine poate în?elege sa în?eleaga.
Mat 19:13  Atunci I s-au adus copii, ca sa-?i puna mâinile peste ei ?i sa Se roage; dar ucenicii îi certau.
Mat 19:14  Iar Iisus a zis: Lasa?i copiii ?i nu-i opri?i sa vina la Mine, ca a unora ca ace?tia este împara?ia cerurilor.
Mat 19:15  ?i punându-?i mâinile peste ei, S-a dus de acolo.
Mat 19:16  ?i, iata, venind un tânar la El, I-a zis: Bunule Înva?ator, ce bine sa fac, ca sa am via?a ve?nica?
Mat 19:17  Iar El a zis: De ce-Mi zici bun? Nimeni nu este bun decât numai Unul Dumnezeu. Iar de vrei sa intri în via?a, paze?te poruncile.
Mat 19:18  El I-a zis: Care? Iar Iisus a zis: Sa nu ucizi, sa nu savâr?e?ti adulter, sa nu furi, sa nu marturise?ti strâmb;
Mat 19:19  Cinste?te pe tatal tau ?i pe mama ta ?i sa iube?ti pe aproapele tau ca pe tine însu?i.
Mat 19:20  Zis-a lui tânarul: Toate acestea le-am pazit din copilaria mea. Ce-mi mai lipse?te?
Mat 19:21  Iisus i-a zis: Daca voie?ti sa fii desavâr?it, du-te, vinde averea ta, da-o saracilor ?i vei avea comoara în cer; dupa aceea, vino ?i urmeaza-Mi.
Mat 19:22  Ci, auzind cuvântul acesta, tânarul a plecat întristat, caci avea multe avu?ii.
Mat 19:23  Iar Iisus a zis ucenicilor Sai: Adevarat zic voua ca un bogat cu greu va intra în împara?ia cerurilor.
Mat 19:24  ?i iara?i zic voua ca mai lesne este sa treaca camila prin urechile acului, decât sa intre un bogat în împara?ia lui Dumnezeu.
Mat 19:25  Auzind, ucenicii s-au uimit foarte, zicând: Dar cine poate sa se mântuiasca?
Mat 19:26  Dar Iisus, privind la ei, le-a zis: La oameni aceasta e cu neputin?a, la Dumnezeu însa toate sunt cu putin?a.
Mat 19:27  Atunci Petru, raspunzând, I-a zis: Iata noi am lasat toate ?i ?i-am urmat ?ie. Cu noi oare ce va fi?
Mat 19:28  Iar Iisus le-a zis: Adevarat zic voua ca voi cei ce Mi-a?i urmat Mie, la înnoirea lumii, când Fiul Omului va ?edea pe tronul slavei Sale, ve?i ?edea ?i voi pe douasprezece tronuri, judecând cele douasprezece semin?ii ale lui Israel.
Mat 19:29  ?i oricine a lasat case sau fra?i, sau surori, sau tata, sau mama, sau femeie, sau copii, sau ?arine, pentru numele Meu, înmul?it va lua înapoi ?i va mo?teni via?a ve?nica.
Mat 19:30  ?i mul?i dintâi vor fi pe urma, ?i cei de pe urma vor fi întâi.
Mat 20:1  Caci împara?ia cerurilor este asemenea unui om stapân de casa, care a ie?it dis-de-diminea?a sa tocmeasca lucratori pentru via sa.
Mat 20:2  ?i învoindu-se cu lucratorii cu un dinar pe zi, i-a trimis în via sa.
Mat 20:3  ?i ie?ind pe la ceasul al treilea, a vazut pe al?ii stând în pia?a fara lucru.
Mat 20:4  ?i le-a zis acelora: Merge?i ?i voi în vie, ?i ce va fi cu dreptul, va voi da.
Mat 20:5  Iar ei s-au dus. Ie?ind iara?i pe la ceasul al ?aselea ?i al noualea, a facut tot a?a.
Mat 20:6  Ie?ind pe la ceasul al unsprezecelea, a gasit pe al?ii, stând fara lucru, ?i le-a zis: De ce a?i stat aici toata ziua fara lucru?
Mat 20:7  Zis-au lui: Fiindca nimeni nu ne-a tocmit. Zis-a lor: Duce?i-va ?i voi în vie ?i ce va fi cu dreptul ve?i lua.
Mat 20:8  Facându-se seara, stapânul viei a zis catre îngrijitorul sau: Cheama pe lucratori ?i da-le plata, începând de cei din urma pâna la cei dintâi.
Mat 20:9  Venind cei din ceasul al unsprezecelea, au luat câte un dinar.
Mat 20:10  ?i venind cei dintâi, au socotit ca vor lua mai mult, dar au luat ?i ei tot câte un dinar.
Mat 20:11  ?i dupa ce au luat, cârteau împotriva stapânului casei,
Mat 20:12  Zicând: Ace?tia de pe urma au facut un ceas ?i i-ai pus deopotriva cu noi, care am dus greutatea zilei ?i ar?i?a.
Mat 20:13  Iar el, raspunzând, a zis unuia dintre ei: Prietene, nu-?i fac nedreptate. Oare nu te-ai învoit cu mine un dinar?
Mat 20:14  Ia ce este al tau ?i pleaca. Voiesc sa dau acestuia de pe urma ca ?i ?ie.
Mat 20:15  Au nu mi se cuvine mie sa fac ce voiesc cu ale mele? Sau ochiul tau este rau, pentru ca eu sunt bun?
Mat 20:16  Astfel vor fi cei de pe urma întâi ?i cei dintâi pe urma, ca mul?i sunt chema?i, dar pu?ini ale?i.
Mat 20:17  ?i suindu-Se la Ierusalim, Iisus a luat de o parte pe cei doisprezece ucenici ?i le-a spus lor, pe cale:
Mat 20:18  Iata ne suim la Ierusalim ?i Fiul Omului va fi dat pe mâna arhiereilor ?i a carturarilor, ?i-L vor osândi la moarte;
Mat 20:19  ?i Îl vor da pe mâna pagânilor, ca sa-L batjocoreasca ?i sa-L rastigneasca, dar a treia zi va învia.
Mat 20:20  Atunci a venit la El mama fiilor lui Zevedeu, împreuna cu fiii ei, închinându-se ?i cerând ceva de la El.
Mat 20:21  Iar El a zis ei: Ce voie?ti? Ea a zis Lui: Zi ca sa ?ada ace?ti doi fii ai mei, unul de-a dreapta 1i altul de-a stânga Ta, întru împara?ia Ta.
Mat 20:22  Dar Iisus, raspunzând, a zis: Nu ?ti?i ce cere?i. Pute?i, oare, sa be?i paharul pe care-l voi bea Eu ?i cu botezul cu care Eu Ma botez sa va boteza?i? Ei I-au zis: Putem.
Mat 20:23  ?i El a zis lor: Paharul Meu ve?i bea ?i cu botezul cu care Eu Ma botez va ve?i boteza, dar a ?edea de-a dreapta ?i de-a stânga Mea nu este al Meu a da, ci se va da celor pentru care s-a pregatit de catre Tatal Meu.
Mat 20:24  ?i auzind cei zece s-au mâniat pe cei doi fra?i.
Mat 20:25  Dar Iisus, chemându-i la Sine, a zis: ?ti?i ca ocârmuitorii neamurilor domnesc peste ele ?i cei mari le stapânesc.
Mat 20:26  Nu tot a?a va fi între voi, ci care între voi va vrea sa fie mare sa fie slujitorul vostru.
Mat 20:27  ?i care între voi va vrea sa fie întâiul sa va fie voua sluga,
Mat 20:28  Dupa cum ?i Fiul Omului n-a venit sa I se slujeasca, ci ca sa slujeasca El ?i sa-?i dea sufletul rascumparare pentru mul?i.
Mat 20:29  ?i plecând ei din Ierihon, mul?ime mare venea în urma Lui.
Mat 20:30  ?i iata doi orbi, care ?edeau lânga drum, auzind ca trece Iisus, au strigat, zicând: Miluie?te-ne pe noi, Doamne, Fiul lui David!
Mat 20:31  Dar mul?imea îi certa ca sa taca; ei însa ?i mai tare strigau, zicând: Miluie?te-ne pe noi, Doamne, Fiul lui David.
Mat 20:32  ?i Iisus, stând, i-a chemat ?i le-a zis: Ce voi?i sa va fac?
Mat 20:33  Zis-au Lui: Doamne, sa se deschida ochii no?tri.
Mat 20:34  ?i facându-I-se mila, Iisus S-a atins de ochii lor, ?i îndata au vazut ?i I-au urmat Lui.
Mat 21:1  Iar când s-au apropiat de Ierusalim ?i au venit la Betfaghe la Muntele Maslinilor, atunci Iisus a trimis pe doi ucenici,
Mat 21:2  Zicându-le: Merge?i în satul care este înaintea voastra ?i îndata ve?i gasi o asina legata ?i un mânz cu ea; dezlega?i-o ?i aduce?i-o la Mine.
Mat 21:3  ?i daca va va zice cineva ceva, ve?i spune ca-I trebuie Domnului; ?i le va trimite îndata.
Mat 21:4  Iar acestea toate s-au facut, ca sa se împlineasca ceea ce s-a spus prin proorocul, care zice:
Mat 21:5  "Spune?i fiicei Sionului: Iata Împaratul tau vine la tine blând ?i ?ezând pe asina, pe mânz, fiul celei de sub jug".
Mat 21:6  Mergând deci ucenicii ?i facând dupa cum le-a poruncit Iisus,
Mat 21:7  Au adus asina ?i mânzul ?i deasupra lor ?i-au pus ve?mintele, iar El a ?ezut peste ele.
Mat 21:8  ?i cei mai mul?i din mul?ime î?i a?terneau hainele pe cale, iar al?ii taiau ramuri din copaci ?i le a?terneau pe cale,
Mat 21:9  Iar mul?imile care mergeau înaintea Lui ?i care veneau dupa El strigau zicând: Osana Fiului lui David; binecuvântat este Cel ce vine întru numele Domnului! Osana întru cei de sus!
Mat 21:10  ?i intrând El în Ierusalim, toata cetatea s-a cutremurat, zicând: Cine este Acesta?
Mat 21:11  Iar mul?imile raspundeau: Acesta este Iisus, proorocul din Nazaretul Galileii.
Mat 21:12  ?i a intrat Iisus în templu ?i a alungat pe to?i cei ce vindeau ?i cumparau în templu ?i a rasturnat mesele schimbatorilor de bani ?i scaunele celor care vindeau porumbei.
Mat 21:13  ?i a zis lor: Scris este: "Casa Mea, casa de rugaciune se va chema, iar voi o face?i pe?tera de tâlhari!"
Mat 21:14  ?i au venit la El, în templu, orbi ?i ?chiopi ?i i-a facut sanato?i.
Mat 21:15  ?i vazând arhiereii ?i carturarii minunile pe care le facuse ?i pe copiii care strigau în templu ?i ziceau: Osana Fiului lui David, s-au mâniat,
Mat 21:16  ?i I-au zis: Auzi ce zic ace?tia? Iar Iisus le-a zis: Da. Au niciodata n-a?i citit ca din gura copiilor ?i a celor ce sug ?i-ai pregatit lauda?
Mat 21:17  ?i lasându-i, a ie?it afara din cetate la Betania, ?i noaptea a ramas acolo.
Mat 21:18  Diminea?a, a doua zi, pe când se întorcea în cetate, a flamânzit;
Mat 21:19  ?i vazând un smochin lânga cale, S-a dus la el, dar n-a gasit nimic în el decât numai frunze, ?i a zis lui: De acum înainte sa nu mai fie rod din tine în veac! ?i smochinul s-a uscat îndata.
Mat 21:20  Vazând aceasta, ucenicii s-au minunat, zicând: Cum s-a uscat smochinul îndata?
Mat 21:21  Iar Iisus, raspunzând, le-a zis: Adevarat graiesc voua: Daca ve?i avea credin?a ?i nu va ve?i îndoi, ve?i face nu numai ce s-a facut cu smochinul, ci ?i muntelui acestuia de ve?i zice: Ridica-te ?i arunca-te în mare, va fi a?a.
Mat 21:22  ?i toate câte ve?i cere, rugându-va cu credin?a, ve?i primi.
Mat 21:23  Iar dupa ce a intrat în templu, s-au apropiat de El, pe când înva?a, arhiereii ?i batrânii poporului ?i au zis: Cu ce putere faci acestea? ?i cine ?i-a dat puterea aceasta?
Mat 21:24  Raspunzând, Iisus le-a zis: Va voi întreba ?i Eu pe voi un cuvânt, pe care, de Mi-l ve?i spune, ?i Eu va voi spune voua cu ce putere fac acestea:
Mat 21:25  Botezul lui Ioan de unde a fost? Din cer sau de la oameni? Iar ei cugetau întru sine, zicând: De vom zice: Din cer, ne va spune: De ce, dar, n-a?i crezut lui?
Mat 21:26  Iar de vom zice: De la oameni, ne temem de popor, fiindca to?i îl socotesc pe Ioan de prooroc.
Mat 21:27  ?i raspunzând ei lui Iisus, au zis: Nu ?tim. Zis-a lor ?i El: Nici Eu nu va spun cu ce putere fac acestea.
Mat 21:28  Dar ce vi se pare? Un om avea doi fii. ?i, ducându-se la cel dintâi, i-a zis: Fiule, du-te astazi ?i lucreaza în via mea.
Mat 21:29  Iar el, raspunzând, a zis: Ma duc, Doamne, ?i nu s-a dus.
Mat 21:30  Mergând la al doilea, i-a zis tot a?a; acesta, raspunzând, a zis: Nu vreau, apoi caindu-se, s-a dus.
Mat 21:31  Care dintr-ace?tia doi a facut voia Tatalui? Zis-au Lui: Cel de-al doilea. Zis-a lor Iisus: Adevarat graiesc voua ca vame?ii ?i desfrânatele merg înaintea voastra în împara?ia lui Dumnezeu.
Mat 21:32  Caci a venit Ioan la voi în calea drepta?ii ?i n-a?i crezut în el, ci vame?ii ?i desfrânatele au crezut, iar voi a?i vazut ?i nu v-a?i cait nici dupa aceea, ca sa crede?i în el.
Mat 21:33  Asculta?i alta pilda: Era un om oarecare stapân al casei sale, care a sadit vie. A împrejmuit-o cu gard, a sapat în ea teasc, a cladit un turn ?i a dat-o lucratorilor, iar el s-a dus departe.
Mat 21:34  Când a sosit timpul roadelor, a trimis pe slugile sale la lucratori, ca sa-i ia roadele.
Mat 21:35  Dar lucratorii, punând mâna pe slugi, pe una au batut-o, pe alta au omorât-o, iar pe alta au ucis-o cu pietre.
Mat 21:36  Din nou a trimis alte slugi, mai multe decât cele dintâi, ?i au facut cu ele tot a?a.
Mat 21:37  La urma, a trimis la ei pe fiul sau zicând: Se vor ru?ina de fiul meu.
Mat 21:38  Iar lucratorii viei, vazând pe fiul, au zis între ei: Acesta este mo?tenitorul; veni?i sa-l omorâm ?i sa avem noi mo?tenirea lui.
Mat 21:39  ?i, punând mâna pe el, l-au scos afara din vie ?i l-au ucis.
Mat 21:40  Deci, când va veni stapânul viei, ce va face acelor lucratori?
Mat 21:41  I-au raspuns: Pe ace?ti rai, cu rau îi va pierde, iar via o va da altor lucratori, care vor da roadele la timpul lor.
Mat 21:42  Zis-a lor Iisus: Au n-a?i citit niciodata în Scripturi: "Piatra pe care au nesocotit-o ziditorii, aceasta a ajuns sa fie în capul unghiului. De la Domnul a fost aceasta ?i este lucru minunat în ochii no?tri"?
Mat 21:43  De aceea va spun ca împara?ia lui Dumnezeu se va lua de la voi ?i se va da neamului care va face roadele ei.
Mat 21:44  Cine va cadea pe piatra aceasta se va sfarâma, iar pe cine va cadea îl va strivi.
Mat 21:45  Iar arhiereii ?i fariseii, ascultând pildele Lui, au în?eles ca despre ei vorbe?te.
Mat 21:46  ?i cautând sa-L prinda, s-au temut de popor pentru ca Îl socotea prooroc.
Mat 22:1  ?i, raspunzând, Iisus a vorbit iara?i în pilde, zicându-le:
Mat 22:2  Împara?ia cerurilor asemanatu-s-a omului împarat care a facut nunta fiului sau.
Mat 22:3  ?i a trimis pe slugile sale ca sa cheme pe cei pofti?i la nunta, dar ei n-au voit sa vina.
Mat 22:4  Iara?i a trimis alte slugi, zicând: Spune?i celor chema?i: Iata, am pregatit ospa?ul meu; juncii mei ?i cele îngra?ate s-au junghiat ?i toate sunt gata. Veni?i la nunta.
Mat 22:5  Dar ei, fara sa ?ina seama, s-au dus: unul la ?arina sa, altul la negu?atoria lui;
Mat 22:6  Iar ceilal?i, punând mâna pe slugile lui, le-au batjocorit ?i le-au ucis.
Mat 22:7  ?i auzind împaratul de acestea, s-a umplut de mânie, ?i trimi?ând o?tile sale, a nimicit pe uciga?ii aceia ?i ceta?ii lor i-au dat foc.
Mat 22:8  Apoi a zis catre slugile sale: Nunta este gata, dar cei pofti?i n-au fost vrednici.
Mat 22:9  Merge?i deci la raspântiile drumurilor ?i pe câ?i ve?i gasi, chema?i-i la nunta.
Mat 22:10  ?i ie?ind slugile acelea la drumuri, au adunat pe to?i câ?i i-au gasit, ?i rai ?i buni, ?i s-a umplut casa nun?ii cu oaspe?i.
Mat 22:11  Iar intrând împaratul ca sa priveasca pe oaspe?i, a vazut acolo un om care nu era îmbracat în haina de nunta,
Mat 22:12  ?i i-a zis: Prietene, cum ai intrat aici fara haina de nunta? El însa a tacut.
Mat 22:13  Atunci împaratul a zis slugilor: Lega?i-l de picioare ?i de mâini ?i arunca?i-l în întunericul cel mai din afara. Acolo va fi plângerea ?i scrâ?nirea din?ilor.
Mat 22:14  Caci mul?i sunt chema?i, dar pu?ini ale?i.
Mat 22:15  Atunci s-au dus fariseii ?i au ?inut sfat ca sa-L prinda pe El în cuvânt.
Mat 22:16  ?i au trimis la El pe ucenicii lor, împreuna cu irodianii, zicând: Înva?atorule, ?tim ca e?ti omul adevarului ?i întru adevar înve?i calea lui Dumnezeu ?i nu-?i pasa de nimeni, pentru ca nu cau?i la fa?a oamenilor.
Mat 22:17  Spune-ne deci noua: Ce ?i se pare? Se cuvine sa dam dajdie Cezarului sau nu?
Mat 22:18  Iar Iisus, cunoscând viclenia lor, le-a raspuns: Ce Ma ispiti?i, fa?arnicilor?
Mat 22:19  Arata?i-Mi banul de dajdie. Iar ei I-au adus un dinar.
Mat 22:20  Iisus le-a zis: Al cui e chipul acesta ?i inscrip?ia de pe el?
Mat 22:21  Raspuns-au ei: Ale Cezarului. Atunci a zis lor: Da?i deci Cezarului cele ce sunt ale Cezarului ?i lui Dumnezeu cele ce sunt ale lui Dumnezeu.
Mat 22:22  Auzind aceasta, s-au minunat ?i, lasându-L, s-au dus.
Mat 22:23  În ziua aceea, s-au apropiat de El saducheii, cei ce zic ca nu este înviere, ?i L-au întrebat,
Mat 22:24  Zicând: Înva?atorule, Moise a zis: Daca cineva moare neavând copii, fratele lui sa ia de so?ie pe cea vaduva ?i sa ridice urma?i fratelui sau.
Mat 22:25  Deci erau, la noi, ?apte fra?i; ?i cel dintâi s-a însurat ?i a murit ?i, neavând urma?, a lasat pe femeia sa fratelui sau.
Mat 22:26  Asemenea ?i al doilea ?i al treilea, pâna la al ?aptelea.
Mat 22:27  În urma tuturor a murit ?i femeia.
Mat 22:28  La înviere, deci, a carui dintre cei ?apte va fi femeia? Caci to?i au avut-o de so?ie.
Mat 22:29  Raspunzând, Iisus le-a zis: Va rataci?i ne?tiind Scripturile, nici puterea lui Dumnezeu.
Mat 22:30  Caci la înviere, nici nu se însoara, nici nu se marita, ci sunt ca îngerii lui Dumnezeu în cer.
Mat 22:31  Iar despre învierea mor?ilor, au n-a?i citit ce vi s-a spus voua de Dumnezeu, zicând:
Mat 22:32  "Eu sunt Dumnezeul lui Avraam ?i Dumnezeul lui Isaac ?i Dumnezeul lui Iacov"? Nu este Dumnezeul mor?ilor, ci al viilor.
Mat 22:33  Iar mul?imile, ascultându-L, erau uimite de înva?atura Lui.
Mat 22:34  ?i auzind fariseii ca a închis gura saducheilor, s-au adunat laolalta.
Mat 22:35  Unul dintre ei, înva?ator de Lege, ispitindu-L pe Iisus, L-a întrebat:
Mat 22:36  Înva?atorule, care porunca este mai mare în Lege?
Mat 22:37  El i-a raspuns: Sa iube?ti pe Domnul Dumnezeul tau, cu toata inima ta, cu tot sufletul tau ?i cu tot cugetul tau.
Mat 22:38  Aceasta este marea ?i întâia porunca.
Mat 22:39  Iar a doua, la fel ca aceasta: Sa iube?ti pe aproapele tau ca pe tine însu?i.
Mat 22:40  În aceste doua porunci se cuprind toata Legea ?i proorocii.
Mat 22:41  ?i fiind aduna?i fariseii, i-a întrebat Iisus,
Mat 22:42  Zicând: Ce vi se pare despre Hristos? Al cui Fiu este? Zis-au Lui: Al lui David.
Mat 22:43  Zis-a lor: Cum deci David, în duh, Îl nume?te pe El Domn? - zicând:
Mat 22:44  "Zis-a Domnul Domnului meu: ?ezi de-a dreapta Mea, pâna ce voi pune pe vrajma?ii Tai a?ternut picioarelor Tale".
Mat 22:45  Deci daca David Îl nume?te pe El domn, cum este fiu al lui?
Mat 22:46  ?i nimeni nu putea sa-I raspunda cuvânt ?i nici n-a mai îndraznit cineva, din ziua aceea, sa-L mai întrebe.
Mat 23:1  Atunci a vorbit Iisus mul?imilor ?i ucenicilor Sai,
Mat 23:2  Zicând: Carturarii ?i fariseii au ?ezut în scaunul lui Moise;
Mat 23:3  Deci toate câte va vor zice voua, face?i-le ?i pazi?i-le; dar dupa faptele lor nu face?i, ca ei zic, dar nu fac.
Mat 23:4  Ca leaga sarcini grele ?i cu anevoie de purtat ?i le pun pe umerii oamenilor, iar ei nici cu degetul nu voiesc sa le mi?te.
Mat 23:5  Toate faptele lor le fac ca sa fie privi?i de oameni; caci î?i la?esc filacteriile ?i î?i maresc ciucurii de pe poale.
Mat 23:6  ?i le place sa stea în capul mesei la ospe?e ?i în bancile dintâi, în sinagogi,
Mat 23:7  ?i sa li se plece lumea în pie?e ?i sa fie numi?i de oameni: Rabi.
Mat 23:8  Voi însa sa nu va numi?i rabi, ca unul este Înva?atorul vostru: Hristos, iar voi to?i sunte?i fra?i.
Mat 23:9  ?i tata al vostru sa nu numi?i pe pamânt, ca Tatal vostru unul este, Cel din ceruri.
Mat 23:10  Nici înva?atori sa nu va numi?i, ca Înva?atorul vostru este unul: Hristos.
Mat 23:11  ?i care este mai mare între voi sa fie slujitorul vostru.
Mat 23:12  Cine se va înal?a pe sine se va smeri, ?i cine se va smeri pe sine se va înal?a.
Mat 23:13  Vai voua, carturarilor ?i fariseilor fa?arnici! Ca închide?i împara?ia cerurilor înaintea oamenilor; ca voi nu intra?i, ?i nici pe cei ce vor sa intre nu-i lasa?i.
Mat 23:14  Vai voua, carturarilor ?i fariseilor fa?arnici! Ca mânca?i casele vaduvelor ?i cu fa?arnicie va ruga?i îndelung; pentru aceasta mai multa osânda ve?i lua.
Mat 23:15  Vai voua, carturarilor ?i fariseilor fa?arnici! Ca înconjura?i marea ?i uscatul ca sa face?i un ucenic, ?i daca l-a?i facut, îl face?i fiu al gheenei ?i îndoit decât voi.
Mat 23:16  Vai voua, calauze oarbe, care zice?i: Cel ce se va jura pe templu nu este cu nimic legat, dar cel ce se va jura pe aurul templului este legat.
Mat 23:17  Nebuni ?i orbi! Ce este mai mare, aurul sau templul care sfin?e?te aurul?
Mat 23:18  Zice?i iar: Cel ce se va jura pe altar cu nimic nu este legat, dar cel ce se va jura pe darul ce este deasupra altarului este legat.
Mat 23:19  Nebuni ?i orbi! Ce este mai mare, darul sau altarul care sfin?e?te darul?
Mat 23:20  Deci, cel ce se jura pe altar se jura pe el ?i pe toate câte sunt deasupra lui.
Mat 23:21  Deci cel ce se jura pe templu se jura pe el ?i pe Cel care locuie?te în el.
Mat 23:22  Cel ce se jura pe cer se jura pe tronul lui Dumnezeu ?i pe Cel ce ?ade pe el.
Mat 23:23  Vai voua, carturarilor ?i fariseilor fa?arnici! Ca da?i zeciuiala din izma, din marar ?i din chimen, dar a?i lasat par?ile mai grele ale Legii: judecata, mila ?i credin?a; pe acestea trebuia sa le face?i ?i pe acelea sa nu le lasa?i
Mat 23:24  Calauze oarbe care strecura?i ?ân?arul ?i înghi?i?i camila!
Mat 23:25  Vai voua, carturarilor ?i fariseilor fa?arnici! Ca voi cura?i?i partea din afara a paharului ?i a blidului, iar înauntru sunt pline de rapire ?i de lacomie.
Mat 23:26  Fariseule orb! Cura?a întâi partea dinauntru a paharului ?i a blidului, ca sa fie curata ?i cea din afara.
Mat 23:27  Vai voua, carturarilor ?i fariseilor fa?arnici! Ca semana?i cu mormintele cele varuite, care pe din afara se arata frumoase, înauntru însa sunt pline de oase de mor?i ?i de toata necura?ia.
Mat 23:28  A?a ?i voi, pe din afara va arata?i drep?i oamenilor, înauntru însa sunte?i plini de fa?arnicie ?i de faradelege.
Mat 23:29  Vai voua, carturarilor ?i fariseilor fa?arnici! Ca zidi?i mormintele proorocilor ?i împodobi?i pe ale drep?ilor,
Mat 23:30  ?i zice?i: De am fi fost noi în zilele parin?ilor no?tri, n-am fi fost parta?i cu ei la varsarea sângelui proorocilor.
Mat 23:31  Astfel, dar, marturisi?i voi în?iva ca sunte?i fii ai celor ce au ucis pe prooroci.
Mat 23:32  Dar voi întrece?i masura parin?ilor vo?tri!
Mat 23:33  ?erpi, pui de vipere, cum ve?i scapa de osânda gheenei?
Mat 23:34  De aceea, iata Eu trimit la voi prooroci ?i în?elep?i ?i carturari; dintre ei ve?i ucide ?i ve?i rastigni; dintre ei ve?i biciui în sinagogi ?i-i ve?i urmari din cetate în cetate,
Mat 23:35  Ca sa cada asupra voastra tot sângele drep?ilor raspândit pe pamânt, de la sângele dreptului Abel, pâna la sângele lui Zaharia, fiul lui Varahia, pe care l-a?i ucis între templu ?i altar.
Mat 23:36  Adevarat graiesc voua, vor veni acestea toate asupra acestui neam.
Mat 23:37  Ierusalime, Ierusalime, care omori pe prooroci ?i cu pietre ucizi pe cei trimi?i la tine; de câte ori am voit sa adun pe fiii tai, dupa cum aduna pasarea puii sai sub aripi, dar nu a?i voit.
Mat 23:38  Iata, casa voastra vi se lasa pustie;
Mat 23:39  Caci va zic voua: De acum nu Ma ve?i mai vedea, pâna când nu ve?i zice: Binecuvântat este Cel ce vine întru numele Domnului.
Mat 24:1  ?i ie?ind Iisus din templu, S-a dus ?i s-au apropiat de el ucenicii Lui, ca sa-I arate cladirile templului.
Mat 24:2  Iar El, raspunzând, le-a zis: Vede?i toate acestea? Adevarat graiesc voua: Nu va ramâne aici piatra pe piatra, care sa nu se risipeasca.
Mat 24:3  ?i ?ezând El pe Muntele Maslinilor, au venit la El ucenicii, de o parte, zicând: Spune noua când vor fi acestea ?i care este semnul venirii Tale ?i al sfâr?itului veacului?
Mat 24:4  Raspunzând, Iisus le-a zis: Vede?i sa nu va amageasca cineva.
Mat 24:5  Caci mul?i vor veni în numele Meu, zicând: Eu sunt Hristos, ?i pe mul?i îi vor amagi.
Mat 24:6  ?i ve?i auzi de razboaie ?i de zvonuri de razboaie; lua?i seama sa nu va speria?i, caci trebuie sa fie toate, dar înca nu este sfâr?itul.
Mat 24:7  Caci se va ridica neam peste neam ?i împara?ie peste împara?ie ?i va fi foamete ?i ciuma ?i cutremure pe alocuri.
Mat 24:8  Dar toate acestea sunt începutul durerilor.
Mat 24:9  Atunci va vor da pe voi spre asuprire ?i va vor ucide ?i ve?i fi urâ?i de toate neamurile pentru numele Meu.
Mat 24:10  Atunci mul?i se vor sminti ?i se vor vinde unii pe al?ii; ?i se vor urî unii pe al?ii.
Mat 24:11  ?i mul?i prooroci mincino?i se vor scula ?i vor amagi pe mul?i.
Mat 24:12  Iar din pricina înmul?irii faradelegii, iubirea multora se va raci.
Mat 24:13  Dar cel ce va rabda pâna sfâr?it, acela se va mântui.
Mat 24:14  ?i se va propovadui aceasta Evanghelie a împara?iei în toata lumea spre marturie la toate neamurile; ?i atunci va veni sfâr?itul.
Mat 24:15  Deci, când ve?i vedea urâciunea pustiirii ce s-a zis prin Daniel proorocul, stând în locul cel sfânt - cine cite?te sa în?eleaga -
Mat 24:16  Atunci cei din Iudeea sa fuga în mun?i.
Mat 24:17  Cel ce va fi pe casa sa nu se coboare, ca sa-?i ia lucrurile din casa.
Mat 24:18  Iar cel ce va fi în ?arina sa nu se întoarca înapoi, ca sa-?i ia haina.
Mat 24:19  Vai de cele însarcinate ?i de cele ce vor alapta în zilele acelea!
Mat 24:20  Ruga?i-va ca sa nu fie fuga voastra iarna, nici sâmbata.
Mat 24:21  Caci va fi atunci strâmtorare mare, cum n-a fost de la începutul lumii pâna acum ?i nici nu va mai fi.
Mat 24:22  ?i de nu s-ar fi scurtat acele zile, n-ar mai scapa nici un trup, dar pentru cei ale?i se vor scurta acele zile.
Mat 24:23  Atunci, de va va zice cineva: Iata, Mesia este aici sau dincolo, sa nu-l crede?i.
Mat 24:24  Caci se vor ridica hristo?i mincino?i ?i prooroci mincino?i ?i vor da semne mari ?i chiar minuni, ca sa amageasca, de va fi cu putin?a, ?i pe cei ale?i.
Mat 24:25  Iata, v-am spus de mai înainte.
Mat 24:26  Deci, de va vor zice voua: Iata este în pustie, sa nu ie?i?i; iata este în camari, sa nu crede?i.
Mat 24:27  Caci precum fulgerul iese de la rasarit ?i se arata pâna la apus, a?a va fi ?i venirea Fiului Omului.
Mat 24:28  Caci unde va fi stârvul, acolo se vor aduna vulturii.
Mat 24:29  Iar îndata dupa strâmtorarea acelor zile, soarele se va întuneca ?i luna nu va mai da lumina ei, iar stelele vor cadea din cer ?i puterile cerurilor se vor zgudui.
Mat 24:30  Atunci se va arata pe cer semnul Fiului Omului ?i vor plânge toate neamurile pamântului ?i vor vedea pe Fiul Omului venind pe norii cerului, cu putere ?i cu slava multa.
Mat 24:31  ?i va trimite pe îngerii Sai, cu sunet mare de trâmbi?a, ?i vor aduna pe cei ale?i ai Lui din cele patru vânturi, de la marginile cerurilor pâna la celelalte margini.
Mat 24:32  Înva?a?i de la smochin pilda: Când mladi?a lui se face frageda ?i odrasle?te frunze, cunoa?te?i ca vara e aproape.
Mat 24:33  Asemenea ?i voi, când ve?i vedea toate acestea, sa ?ti?i ca este aproape, la u?i.
Mat 24:34  Adevarat graiesc voua ca nu va trece neamul acesta, pâna ce nu vor fi toate acestea.
Mat 24:35  Cerul ?i pamântul vor trece, dar cuvintele Mele nu vor trece.
Mat 24:36  Iar de ziua ?i de ceasul acela nimeni nu ?tie, nici îngerii din ceruri, nici Fiul, ci numai Tatal.
Mat 24:37  ?i precum a fost în zilele lui Noe, a?a va fi venirea Fiului Omului.
Mat 24:38  Caci precum în zilele acelea dinainte de potop, oamenii mâncau ?i beau, se însurau ?i se maritau, pâna în ziua când a intrat Noe în corabie,
Mat 24:39  ?i n-au ?tiut pâna ce a venit potopul ?i i-a luat pe to?i, la fel va fi ?i venirea Fiului Omului.
Mat 24:40  Atunci, din doi care vor fi în ?arina, unul se va lua ?i altul se va lasa.
Mat 24:41  Din doua care vor macina la moara, una se va lua ?i alta se va lasa.
Mat 24:42  Priveghea?i deci, ca nu ?ti?i în care zi vine Domnul vostru.
Mat 24:43  Aceea cunoa?te?i, ca de-ar ?ti stapânul casei la ce straja din noapte vine furul, ar priveghea ?i n-ar lasa sa i se sparga casa.
Mat 24:44  De aceea ?i voi fi?i gata, ca în ceasul în care nu gândi?i Fiul Omului va veni.
Mat 24:45  Cine, oare, este sluga credincioasa ?i în?eleapta pe care a pus-o stapânul peste slugile sale, ca sa le dea hrana la timp?
Mat 24:46  Fericita este sluga aceea, pe care venind stapânul sau, o va afla facând a?a.
Mat 24:47  Adevarat zic voua ca peste toate avu?iile sale o va pune.
Mat 24:48  Iar daca acea sluga, rea fiind, va zice în inima sa: Stapânul meu întârzie,
Mat 24:49  ?i va începe sa bata pe cei ce slujesc împreuna cu el, sa manânce ?i sa bea cu be?ivii,
Mat 24:50  Veni-va stapânul slugii aceleia în ziua când nu se a?teapta ?i în ceasul pe care nu-l cunoa?te,
Mat 24:51  ?i o va taia din dregatorie ?i partea ei o va pune cu fa?arnicii. Acolo va fi plângerea ?i scrâ?nirea din?ilor.
Mat 25:1  Împara?ia cerurilor se va asemana cu zece fecioare, care luând candelele lor, au ie?it în întâmpinarea mirelui.
Mat 25:2  Cinci însa dintre ele erau fara minte, iar cinci în?elepte.
Mat 25:3  Caci cele fara de minte, luând candelele, n-au luat cu sine untdelemn.
Mat 25:4  Iar cele în?elepte au luat untdelemn în vase, odata cu candelele lor.
Mat 25:5  Dar mirele întârziind, au a?ipit toate ?i au adormit.
Mat 25:6  Iar la miezul nop?ii s-a facut strigare: Iata, mirele vine! Ie?i?i întru întâmpinarea lui!
Mat 25:7  Atunci s-au de?teptat toate acele fecioare ?i au împodobit candelele lor.
Mat 25:8  ?i cele fara de minte au zis catre cele în?elepte: Da?i-ne din untdelemnul vostru, ca se sting candelele noastre.
Mat 25:9  Dar cele în?elepte le-au raspuns, zicând: Nu, ca nu cumva sa nu ne ajunga nici noua ?i nici voua. Mai bine merge?i la cei ce vând ?i cumpara?i pentru voi.
Mat 25:10  Deci plecând ele ca sa cumpere, a venit mirele ?i cele ce erau gata au intrat cu el la nunta ?i u?a s-a închis.
Mat 25:11  Iar mai pe urma, au sosit ?i celelalte fecioare, zicând: Doamne, Doamne, deschide-ne noua.
Mat 25:12  Iar el, raspunzând, a zis: Adevarat zic voua: Nu va cunosc pe voi.
Mat 25:13  Drept aceea, priveghea?i, ca nu ?ti?i ziua, nici ceasul când vine Fiul Omului.
Mat 25:14  ?i mai este ca un om care, plecând departe, ?i-a chemat slugile ?i le-a dat pe mâna avu?ia sa.
Mat 25:15  Unuia i-a dat cinci talan?i, altuia doi, altuia unul, fiecaruia dupa puterea lui ?i a plecat.
Mat 25:16  Îndata, mergând, cel ce luase cinci talan?i a lucrat cu ei ?i a câ?tigat al?i cinci talan?i.
Mat 25:17  De asemenea ?i cel cu doi a câ?tigat al?i doi.
Mat 25:18  Iar cel ce luase un talant s-a dus, a sapat o groapa în pamânt ?i a ascuns argintul stapânului sau.
Mat 25:19  Dupa multa vreme a venit ?i stapânul acelor slugi ?i a facut socoteala cu ele.
Mat 25:20  ?i apropiindu-se cel care luase cinci talan?i, a adus al?i cinci talan?i, zicând: Doamne, cinci talan?i mi-ai dat, iata al?i cinci talan?i am câ?tigat cu ei.
Mat 25:21  Zis-a lui stapânul: Bine, sluga buna ?i credincioasa, peste pu?ine ai fost credincioasa, peste multe te voi pune; intra întru bucuria domnului tau.
Mat 25:22  Apropiindu-se ?i cel cu doi talan?i, a zis: Doamne, doi talan?i mi-ai dat, iata al?i doi talan?i am câ?tigat cu ei.
Mat 25:23  Zis-a lui stapânul: Bine, sluga buna ?i credincioasa, peste pu?ine ai fost credincioasa, peste multe te voi pune; intra întru bucuria domnului tau.
Mat 25:24  Apropiindu-se apoi ?i cel care primise un talant, a zis: Doamne, te-am ?tiut ca e?ti om aspru, care seceri unde n-ai semanat ?i aduni de unde n-ai împra?tiat.
Mat 25:25  ?i temându-ma, m-am dus de am ascuns talantul tau în pamânt; iata ai ce este al tau.
Mat 25:26  ?i raspunzând stapânul sau i-a zis: Sluga vicleana ?i lene?a, ?tiai ca secer de unde n-am semanat ?i adun de unde n-am împra?tiat?
Mat 25:27  Se cuvenea deci ca tu sa pui banii mei la zarafi, ?i eu, venind, a? fi luat ce este al meu cu dobânda.
Mat 25:28  Lua?i deci de la el talantul ?i da?i-l celui ce are zece talan?i.
Mat 25:29  Caci tot celui ce are i se va da ?i-i va prisosi, iar de la cel ce n-are ?i ce are i se va lua.
Mat 25:30  Iar pe sluga netrebnica arunca?i-o întru întunericul cel mai din afara. Acolo va fi plângerea ?i scrâ?nirea din?ilor.
Mat 25:31  Când va veni Fiul Omului întru slava Sa, ?i to?i sfin?ii îngeri cu El, atunci va ?edea pe tronul slavei Sale.
Mat 25:32  ?i se vor aduna înaintea Lui toate neamurile ?i-i va despar?i pe unii de al?ii, precum desparte pastorul oile de capre.
Mat 25:33  ?i va pune oile de-a dreapta Sa, iar caprele de-a stânga.
Mat 25:34  Atunci va zice Împaratul celor de-a dreapta Lui: Veni?i, binecuvânta?ii Tatalui Meu, mo?teni?i împara?ia cea pregatita voua de la întemeierea lumii.
Mat 25:35  Caci flamând am fost ?i Mi-a?i dat sa manânc; însetat am fost ?i Mi-a?i dat sa beau; strain am fost ?i M-a?i primit;
Mat 25:36  Gol am fost ?i M-a?i îmbracat; bolnav am fost ?i M-a?i cercetat; în temni?a am fost ?i a?i venit la Mine.
Mat 25:37  Atunci drep?ii Îi vor raspunde, zicând: Doamne, când Te-am vazut flamând ?i Te-am hranit? Sau însetat ?i ?i-am dat sa bei?
Mat 25:38  Sau când Te-am vazut strain ?i Te-am primit, sau gol ?i Te-am îmbracat?
Mat 25:39  Sau când Te-am vazut bolnav sau în temni?a ?i am venit la Tine?
Mat 25:40  Iar Împaratul, raspunzând, va zice catre ei: Adevarat zic voua, întrucât a?i facut unuia dintr-ace?ti fra?i ai Mei, prea mici, Mie Mi-a?i facut.
Mat 25:41  Atunci va zice ?i celor de-a stânga: Duce?i-va de la Mine, blestema?ilor, în focul cel ve?nic, care este gatit diavolului ?i îngerilor lui.
Mat 25:42  Caci flamând am fost ?i nu Mi-a?i dat sa manânc; însetat am fost ?i nu Mi-a?i dat sa beau;
Mat 25:43  Strain am fost ?i nu M-a?i primit; gol, ?i nu M-a?i îmbracat; bolnav ?i în temni?a, ?i nu M-a?i cercetat.
Mat 25:44  Atunci vor raspunde ?i ei, zicând: Doamne, când Te-am vazut flamând, sau însetat, sau strain, sau gol, sau bolnav, sau în temni?a ?i nu ?i-am slujit?
Mat 25:45  El însa le va raspunde, zicând: Adevarat zic voua: Întrucât nu a?i facut unuia dintre ace?ti prea mici, nici Mie nu Mi-a?i facut.
Mat 25:46  ?i vor merge ace?tia la osânda ve?nica, iar drep?ii la via?a ve?nica.
Mat 26:1  Iar dupa ce a sfâr?it toate aceste cuvinte, a zis Iisus catre ucenicii Sai:
Mat 26:2  ?ti?i ca peste doua zile va fi Pa?tile ?i Fiul Omului va fi dat sa fie rastignit.
Mat 26:3  Atunci arhiereii ?i batrânii poporului s-au adunat în curtea arhiereului, care se numea Caiafa.
Mat 26:4  ?i împreuna s-au sfatuit ca sa prinda pe Iisus, cu vicle?ug, ?i sa-L ucida.
Mat 26:5  Dar ziceau: Nu în ziua praznicului, ca sa nu se faca tulburare în popor.
Mat 26:6  Fiind Iisus în Betania, în casa lui Simon Leprosul,
Mat 26:7  S-a apropiat de El o femeie, având un alabastru cu mir de mare pre?, ?i l-a turnat pe capul Lui, pe când ?edea la masa.
Mat 26:8  ?i vazând ucenicii, s-au mâniat ?i au zis: De ce risipa aceasta?
Mat 26:9  Caci mirul acesta se putea vinde scump, iar banii sa se dea saracilor.
Mat 26:10  Dar Iisus, cunoscând gândul lor, le-a zis: Pentru ce face?i suparare femeii? Caci lucru bun a facut ea fa?a de Mine.
Mat 26:11  Caci pe saraci totdeauna îi ave?i cu voi, dar pe Mine nu Ma ave?i totdeauna;
Mat 26:12  Ca ea, turnând mirul acesta pe trupul Meu, a facut-o spre îngroparea Mea.
Mat 26:13  Adevarat zic voua: Oriunde se va propovadui Evanghelia aceasta, în toata lumea, se va spune ?i ce-a facut ea, spre pomenirea ei.
Mat 26:14  Atunci unul din cei doisprezece, numit Iuda Iscarioteanul, ducându-se la arhierei,
Mat 26:15  A zis: Ce voi?i sa-mi da?i ?i eu Îl voi da în mâinile voastre? Iar ei i-au dat treizeci de argin?i.
Mat 26:16  ?i de atunci cauta un prilej potrivit ca sa-L dea în mâinile lor.
Mat 26:17  În cea dintâi zi a Azimelor, au venit ucenicii la Iisus ?i L-au întrebat: Unde voie?ti sa-?i pregatim sa manânci Pa?tile?
Mat 26:18  Iar El a zis: Merge?i în cetate, la cutare ?i spune?i-i: Înva?atorul zice: Timpul Meu este aproape; la tine vreau sa fac Pa?tile cu ucenicii Mei.
Mat 26:19  ?i ucenicii au facut precum le-a poruncit Iisus ?i au pregatit Pa?tile.
Mat 26:20  Iar când s-a facut seara, a ?ezut la masa cu cei doisprezece ucenici.
Mat 26:21  ?i pe când mâncau, Iisus a zis: Adevarat graiesc voua, ca unul dintre voi Ma va vinde.
Mat 26:22  ?i ei, întristându-se foarte, au început sa-I zica fiecare: Nu cumva eu sunt, Doamne?
Mat 26:23  Iar El, raspunzând, a zis: Cel ce a întins cu Mine mâna în blid, acela Ma va vinde.
Mat 26:24  Fiul Omului merge precum este scris despre El. Vai, însa, acelui om prin care Fiul Omului se vinde! Bine era de omul acela daca nu se na?tea.
Mat 26:25  ?i Iuda, cel ce L-a vândut, raspunzând a zis: Nu cumva sunt eu, Înva?atorule? Raspuns-a lui: Tu ai zis.
Mat 26:26  Iar pe când mâncau ei, Iisus, luând pâine ?i binecuvântând, a frânt ?i, dând ucenicilor, a zis: Lua?i, mânca?i, acesta este trupul Meu.
Mat 26:27  ?i luând paharul ?i mul?umind, le-a dat, zicând: Be?i dintru acesta to?i,
Mat 26:28  Ca acesta este Sângele Meu, al Legii celei noi, care pentru mul?i se varsa spre iertarea pacatelor.
Mat 26:29  ?i va spun voua ca nu voi mai bea de acum din acest rod al vi?ei pâna în ziua aceea când îl voi bea cu voi, nou, întru împara?ia Tatalui Meu.
Mat 26:30  ?i dupa ce au cântat laude, au ie?it la Muntele Maslinilor.
Mat 26:31  Atunci Iisus le-a zis: Voi to?i va ve?i sminti întru Mine în noaptea aceasta caci scris este: "Bate-voi pastorul ?i se vor risipi oile turmei".
Mat 26:32  Dar dupa învierea Mea voi merge mai înainte de voi în Galileea.
Mat 26:33  Iar Petru, raspunzând, I-a zis: Daca to?i se vor sminti întru Tine, eu niciodata nu ma voi sminti.
Mat 26:34  Zis-a Iisus lui: Adevarat zic ?ie ca în noaptea aceasta, mai înainte de a cânta coco?ul, de trei ori te vei lepada de Mine.
Mat 26:35  Petru i-a zis: ?i de ar fi sa mor împreuna cu Tine, nu ma voi lepada de Tine. ?i to?i ucenicii au zis la fel.
Mat 26:36  Atunci Iisus a mers împreuna cu ei la un loc ce se cheama Ghetsimani ?i a zis ucenicilor: ?ede?i aici, pâna ce Ma voi duce acolo ?i Ma voi ruga.
Mat 26:37  ?i luând cu Sine pe Petru ?i pe cei doi fii ai lui Zevedeu, a început a Se întrista ?i a Se mâhni.
Mat 26:38  Atunci le-a zis: Întristat este sufletul Meu pâna la moarte. Ramâne?i aici ?i priveghea?i împreuna cu Mine.
Mat 26:39  ?i mergând pu?in mai înainte, a cazut cu fa?a la pamânt, rugându-Se ?i zicând: Parintele Meu, de este cu putin?a, treaca de la Mine paharul acesta! Însa nu precum voiesc Eu, ci precum Tu voie?ti.
Mat 26:40  ?i a venit la ucenici ?i i-a gasit dormind ?i i-a zis lui Petru: A?a, n-a?i putut un ceas sa priveghea?i cu Mine!
Mat 26:41  Priveghea?i ?i va ruga?i, ca sa nu intra?i în ispita. Caci duhul este osârduitor, dar trupul este neputincios.
Mat 26:42  Iara?i ducându-se, a doua oara, s-a rugat, zicând: Parintele Meu, daca nu este cu putin?a sa treaca acest pahar, ca sa nu-l beau, faca-se voia Ta.
Mat 26:43  ?i venind iara?i, i-a aflat dormind, caci ochii lor erau îngreuia?i.
Mat 26:44  ?i lasându-i, S-a dus iara?i ?i a treia oara S-a rugat, acela?i cuvânt zicând.
Mat 26:45  Atunci a venit la ucenici ?i le-a zis: Dormi?i de acum ?i va odihni?i! Iata s-a apropiat ceasul ?i Fiul Omului va fi dat în mâinile pacato?ilor.
Mat 26:46  Scula?i-va sa mergem, iata s-a apropiat cel ce M-a vândut.
Mat 26:47  ?i pe când vorbea înca, iata a sosit Iuda, unul dintre cei doisprezece, ?i împreuna cu el mul?ime multa, cu sabii ?i cu ciomege, de la arhierei ?i de la batrânii poporului.
Mat 26:48  Iar vânzatorul le-a dat semn, zicând: Pe care-L voi saruta, Acela este: pune?i mâna pe El.
Mat 26:49  ?i îndata, apropiindu-se de Iisus, a zis: Bucura-Te, Înva?atorule! ?i L-a sarutat.
Mat 26:50  Iar Iisus i-a zis: Prietene, pentru ce ai venit? Atunci ei, apropiindu-se, au pus mâinile pe Iisus ?i L-au prins.
Mat 26:51  ?i iata, unul dintre cei ce erau cu Iisus, întinzând mâna, a tras sabia ?i, lovind pe sluga arhiereului, i-a taiat urechea.
Mat 26:52  Atunci Iisus i-a zis: Întoarce sabia ta la locul ei, ca to?i cei ce scot sabia, de sabie vor pieri.
Mat 26:53  Sau ?i se pare ca nu pot sa rog pe Tatal Meu ?i sa-Mi trimita acum mai mult de douasprezece legiuni de îngeri?
Mat 26:54  Dar cum se vor împlini Scripturile, ca a?a trebuie sa fie?
Mat 26:55  În ceasul acela, a zis Iisus mul?imilor: Ca la un tâlhar a?i ie?it cu sabii ?i cu ciomege, ca sa Ma prinde?i. În fiecare zi ?edeam în templu ?i înva?am ?i n-a?i pus mâna pe Mine.
Mat 26:56  Dar toate acestea s-au facut ca sa se împlineasca Scripturile proorocilor. Atunci to?i ucenicii, lasându-L, au fugit.
Mat 26:57  Iar cei care au prins pe Iisus L-au dus la Caiafa arhiereul, unde erau aduna?i carturarii ?i batrânii.
Mat 26:58  Iar Petru Îl urma de departe pâna a ajuns la curtea arhiereului ?i, intrând înauntru, ?edea cu slugile, ca sa vada sfâr?itul.
Mat 26:59  Iar arhiereii, batrânii ?i tot sinedriul cautau marturie mincinoasa împotriva lui Iisus, ca sa-L omoare.
Mat 26:60  ?i n-au gasit, de?i venisera mul?i martori mincino?i. Mai pe urma însa au venit doi ?i au spus:
Mat 26:61  Acesta a zis: Pot sa darâm templul lui Dumnezeu ?i în trei zile sa-l cladesc.
Mat 26:62  ?i, sculându-se, arhiereul I-a zis: Nu raspunzi nimic la ceea ce marturisesc ace?tia împotriva Ta?
Mat 26:63  Dar Iisus tacea. ?i arhiereul I-a zis: Te jur pe Dumnezeul cel viu, sa ne spui noua de e?ti Tu Hristosul, Fiul lui Dumnezeu.
Mat 26:64  Iisus i-a raspuns: Tu ai zis. ?i va spun înca: De acum ve?i vedea pe Fiul Omului ?ezând de-a dreapta puterii ?i venind pe norii cerului.
Mat 26:65  Atunci arhiereul ?i-a sfâ?iat hainele, zicând: A hulit! Ce ne mai trebuie martori? Iata acum a?i auzit hula Lui.
Mat 26:66  Ce vi se pare? Iar ei, raspunzând, au zis: Este vinovat de moarte.
Mat 26:67  ?i au scuipat în obrazul Lui, batându-L cu pumnii, iar unii Îi dadeau palme,
Mat 26:68  Zicând: Prooroce?te-ne, Hristoase, cine este cel ce Te-a lovit.
Mat 26:69  Iar Petru ?edea afara, în curte. ?i o slujnica s-a apropiat de el, zicând: ?i tu erai cu Iisus Galileianul.
Mat 26:70  Dar el s-a lepadat înaintea tuturor, zicând: Nu ?tiu ce zici.
Mat 26:71  ?i ie?ind el la poarta, l-a vazut alta ?i a zis celor de acolo: ?i acesta era cu Iisus Nazarineanul.
Mat 26:72  ?i iara?i s-a lepadat cu juramânt: Nu cunosc pe omul acesta.
Mat 26:73  Iar dupa pu?in, apropiindu-se cei ce stateau acolo au zis lui Petru: Cu adevarat ?i tu e?ti dintre ei, caci ?i graiul te vade?te.
Mat 26:74  Atunci el a început a se blestema ?i a se jura: Nu cunosc pe omul acesta. ?i îndata a cântat coco?ul.
Mat 26:75  ?i Petru ?i-a adus aminte de cuvântul lui Iisus, care zisese: Mai înainte de a cânta coco?ul, de trei ori te vei lepada de Mine. ?i ie?ind afara, a plâns cu amar.
Mat 27:1  Iar facându-se diminea?a, to?i arhiereii ?i batrânii poporului au ?inut sfat împotriva lui Iisus, ca sa-L omoare.
Mat 27:2  ?i, legându-L, L-au dus ?i L-au predat dregatorului Pon?iu Pilat.
Mat 27:3  Atunci Iuda, cel ce L-a vândut, vazând ca a fost osândit, s-a cait ?i a adus înapoi arhiereilor ?i batrânilor cei treizeci de argin?i,
Mat 27:4  Zicând: Am gre?it vânzând sânge nevinovat. Ei i-au zis: Ce ne prive?te pe noi? Tu vei vedea.
Mat 27:5  ?i el, aruncând argin?ii în templu, a plecat ?i, ducându-se, s-a spânzurat.
Mat 27:6  Iar arhiereii, luând banii, au zis: Nu se cuvine sa-i punem în vistieria templului, deoarece sunt pre? de sânge.
Mat 27:7  ?i ?inând ei sfat, au cumparat cu ei ?arina Olarului, pentru îngroparea strainilor.
Mat 27:8  Pentru aceea s-a numit ?arina aceea ?arina Sângelui, pâna în ziua de astazi.
Mat 27:9  Atunci s-a împlinit cuvântul spus de Ieremia proorocul, care zice: "?i au luat cei treizeci de argin?i, pre?ul celui pre?uit, pe care l-au pre?uit fiii lui Israel,
Mat 27:10  ?i i-au dat pe ?arina Olarului dupa cum mi-a spus mie Domnul".
Mat 27:11  Iar Iisus statea înaintea dregatorului. ?i L-a întrebat dregatorul, zicând: Tu e?ti regele iudeilor? Iar Iisus i-a raspuns: Tu zici.
Mat 27:12  ?i la învinuirile aduse Lui de catre arhierei ?i batrâni, nu raspundea nimic.
Mat 27:13  Atunci I-a zis Pilat: Nu auzi câte marturisesc ei împotriva Ta?
Mat 27:14  ?i nu i-a raspuns lui nici un cuvânt, încât dregatorul se mira foarte.
Mat 27:15  La sarbatoarea Pa?tilor, dregatorul avea obiceiul sa elibereze mul?imii un întemni?at pe care-l voiau.
Mat 27:16  ?i aveau atunci un vinovat vestit, care se numea Baraba.
Mat 27:17  Deci aduna?i fiind ei, Pilat le-a zis: Pe cine voi?i sa vi-l eliberez, pe Baraba sau pe Iisus, care se zice Hristos?
Mat 27:18  Ca ?tia ca din invidie L-au dat în mâna lui.
Mat 27:19  ?i pe când statea Pilat în scaunul de judecata, femeia lui i-a trimis acest cuvânt: Nimic sa nu-I faci Dreptului acestuia, ca mult am suferit azi, în vis, pentru El.
Mat 27:20  Însa arhiereii ?i batrânii au a?â?at mul?imile ca sa ceara pe Baraba, iar pe Iisus sa-L piarda.
Mat 27:21  Iar dregatorul, raspunzând, le-a zis: Pe cine din cei doi voi?i sa va eliberez? Iar ei au raspuns: Pe Baraba.
Mat 27:22  ?i Pilat le-a zis: Dar ce voi face cu Iisus, ce se cheama Hristos? To?i au raspuns: Sa fie rastignit!
Mat 27:23  A zis iara?i Pilat: Dar ce rau a facut? Ei însa mai tare strigau ?i ziceau: Sa fie rastignit!
Mat 27:24  ?i vazând Pilat ca nimic nu folose?te, ci mai mare tulburare se face, luând apa ?i-a spalat mâinile înaintea mul?imii, zicând: Nevinovat sunt de sângele Dreptului acestuia. Voi ve?i vedea.
Mat 27:25  Iar tot poporul a raspuns ?i a zis: Sângele Lui asupra noastra ?i asupra copiilor no?tri!
Mat 27:26  Atunci le-a eliberat pe Baraba, iar pe Iisus L-a biciuit ?i L-a dat sa fie rastignit.
Mat 27:27  Atunci osta?ii dregatorului, ducând ei pe Iisus în pretoriu, au adunat în jurul Lui toata cohorta,
Mat 27:28  ?i dezbracându-L de toate hainele Lui, I-au pus o hlamida ro?ie.
Mat 27:29  ?i împletind o cununa de spini, I-au pus-o pe cap ?i în mâna Lui cea dreapta trestie; ?i, îngenunchind înaintea lui î?i bateau joc de El, zicând: Bucura-Te, regele iudeilor!
Mat 27:30  ?i scuipând asupra Lui, au luat trestia ?i-L bateau peste cap.
Mat 27:31  Iar dupa ce L-au batjocorit, L-au dezbracat de hlamida, L-au îmbracat cu hainele Lui ?i L-au dus sa-L rastigneasca.
Mat 27:32  ?i ie?ind, au gasit pe un om din Cirene, cu numele Simon; pe acesta l-au silit sa duca crucea Lui.
Mat 27:33  ?i venind la locul numit Golgota, care înseamna: Locul Capa?ânii,
Mat 27:34  I-au dat sa bea vin amestecat cu fiere; ?i, gustând, nu a voit sa bea.
Mat 27:35  Iar dupa ce L-au rastignit, au împar?it hainele Lui, aruncând sor?i, ca sa se împlineasca ceea ce s-a zis de proorocul: "Împar?it-au hainele Mele între ei, iar pentru cama?a Mea au aruncat sor?i".
Mat 27:36  ?i osta?ii, ?ezând, Îl pazeau acolo.
Mat 27:37  ?i deasupra capului au pus vina Lui scrisa: Acesta este Iisus, regele iudeilor.
Mat 27:38  Atunci au fost rastigni?i împreuna cu El doi tâlhari, unul de-a dreapta ?i altul de-a stânga.
Mat 27:39  Iar trecatorii Îl huleau, clatinându-?i capetele,
Mat 27:40  ?i zicând: Tu, Cel ce darâmi templul ?i în trei zile îl zide?ti, mântuie?te-Te pe Tine Însu?i! Daca e?ti Fiul lui Dumnezeu, coboara-Te de pe cruce!
Mat 27:41  Asemenea ?i arhiereii, batându-?i joc de El, cu carturarii ?i cu batrânii, ziceau:
Mat 27:42  Pe al?ii i-a mântuit, iar pe Sine nu poate sa Se mântuiasca! Daca este regele lui Israel, sa Se coboare acum de pe cruce, ?i vom crede în El.
Mat 27:43  S-a încrezut în Dumnezeu: Sa-L scape acum, daca-L vrea pe El! Caci a zis: Sunt Fiul lui Dumnezeu.
Mat 27:44  În acela?i chip Îl ocarau ?i tâlharii cei împreuna-rastigni?i cu El.
Mat 27:45  Iar de la ceasul al ?aselea, s-a facut întuneric peste tot pamântul, pâna la ceasul al noualea.
Mat 27:46  Iar în ceasul al noualea a strigat Iisus cu glas mare, zicând: Eli, Eli, lama sabahtani? adica: Dumnezeul Meu, Dumnezeul Meu, pentru ce M-ai parasit?
Mat 27:47  Iar unii dintre cei ce stateau acolo, auzind ziceau: Pe Ilie îl striga Acesta.
Mat 27:48  ?i unul dintre ei, alergând îndata ?i luând un burete, ?i umplându-l de o?et ?i punându-l într-o trestie, Îi da sa bea.
Mat 27:49  Iar ceilal?i ziceau: Lasa, sa vedem daca vine Ilie sa-L mântuiasca.
Mat 27:50  Iar Iisus, strigând iara?i cu glas mare, ?i-a dat duhul.
Mat 27:51  ?i iata, catapeteasma templului s-a sfâ?iat în doua de sus pâna jos, ?i pamântul s-a cutremurat ?i pietrele s-au despicat;
Mat 27:52  Mormintele s-au deschis ?i multe trupuri ale sfin?ilor adormi?i s-au sculat.
Mat 27:53  ?i ie?ind din morminte, dupa învierea Lui, au intrat în cetatea sfânta ?i s-au aratat multora.
Mat 27:54  Iar suta?ul ?i cei ce împreuna cu el pazeau pe Iisus, vazând cutremurul ?i cele întâmplate, s-au înfrico?at foarte, zicând: Cu adevarat, Fiul lui Dumnezeu era Acesta!
Mat 27:55  ?i erau acolo multe femei, privind de departe, care urmasera din Galileea pe Iisus, slujindu-I,
Mat 27:56  Între care era Maria Magdalena ?i Maria, mama lui Iacov ?i a lui Iosi, ?i mama fiilor lui Zevedeu.
Mat 27:57  Iar facându-se seara, a venit un om bogat din Arimateea, cu numele Iosif, care ?i el era un ucenic al lui Iisus.
Mat 27:58  Acesta, ducându-se la Pilat, a cerut trupul lui Iisus. Atunci Pilat a poruncit sa i se dea.
Mat 27:59  ?i Iosif, luând trupul, l-a înfa?urat în giulgiu curat de in,
Mat 27:60  ?i l-a pus în mormântul nou al sau, pe care-l sapase în stânca, ?i, pravalind o piatra mare la u?a mormântului, s-a dus.
Mat 27:61  Iar acolo era Maria Magdalena ?i cealalta Marie, ?ezând în fa?a mormântului.
Mat 27:62  Iar a doua zi, care este dupa vineri, s-au adunat arhiereii ?i fariseii la Pilat,
Mat 27:63  Zicând: Doamne, ne-am adus aminte ca amagitorul Acela a spus, fiind înca în via?a: Dupa trei zile Ma voi scula.
Mat 27:64  Deci, porunce?te ca mormântul sa fie pazit pâna a treia zi, ca nu cumva ucenicii Lui sa vina ?i sa-L fure ?i sa spuna poporului: S-a sculat din mor?i. ?i va fi ratacirea de pe urma mai rea decât cea dintâi.
Mat 27:65  Pilat le-a zis: Ave?i straja; merge?i ?i întari?i cum ?ti?i.
Mat 27:66  Iar ei, ducându-se, au întarit mormântul cu straja, pecetluind piatra.
Mat 28:1  Dupa ce a trecut sâmbata, când se lumina de ziua întâi a saptamânii (Duminica), au venit Maria Magdalena ?i cealalta Marie, ca sa vada mormântul.
Mat 28:2  ?i iata s-a facut cutremur mare, ca îngerul Domnului, coborând din cer ?i venind, a pravalit piatra ?i ?edea deasupra ei.
Mat 28:3  ?i înfa?i?area lui era ca fulgerul ?i îmbracamintea lui alba ca zapada.
Mat 28:4  ?i de frica lui s-au cutremurat cei ce pazeau ?i s-au facut ca mor?i.
Mat 28:5  Iar îngerul, raspunzând, a zis femeilor: Nu va teme?i, ca ?tiu ca pe Iisus cel rastignit Îl cauta?i.
Mat 28:6  Nu este aici; caci S-a sculat precum a zis; veni?i de vede?i locul unde a zacut.
Mat 28:7  ?i degraba mergând, spune?i ucenicilor Lui ca S-a sculat din mor?i ?i iata va merge înaintea voastra în Galileea; acolo Îl ve?i vedea. Iata v-am spus voua.
Mat 28:8  Iar plecând ele în graba de la mormânt, cu frica ?i cu bucurie mare au alergat sa vesteasca ucenicilor Lui.
Mat 28:9  Dar când mergeau ele sa vesteasca ucenicilor, iata Iisus le-a întâmpinat, zicând: Bucura?i-va! Iar ele, apropiindu-se, au cuprins picioarele Lui ?i I s-au închinat.
Mat 28:10  Atunci Iisus le-a zis: Nu va teme?i. Duce?i-va ?i vesti?i fra?ilor Mei, ca sa mearga în Galileea, ?i acolo Ma vor vedea.
Mat 28:11  ?i plecând ele, iata unii din straja, venind în cetate, au vestit arhiereilor toate cele întâmplate.
Mat 28:12  ?i, adunându-se ei împreuna cu batrânii ?i ?inând sfat, au dat bani mul?i osta?ilor,
Mat 28:13  Zicând: Spune?i ca ucenicii Lui, venind noaptea, L-au furat, pe când noi dormeam;
Mat 28:14  ?i de se va auzi aceasta la dregatorul, noi îl vom îndupleca ?i pe voi fara grija va vom face.
Mat 28:15  Iar ei, luând argin?ii, au facut precum au fost înva?a?i. ?i s-a raspândit cuvântul acesta între Iudei, pâna în ziua de azi.
Mat 28:16  Iar cei unsprezece ucenici au mers în Galileea, la muntele unde le poruncise lor Iisus.
Mat 28:17  ?i vazându-L, I s-au închinat, ei care se îndoisera.
Mat 28:18  ?i apropiindu-Se Iisus, le-a vorbit lor, zicând: Datu-Mi-s-a toata puterea, în cer ?i pe pamânt.
Mat 28:19  Drept aceea, mergând, înva?a?i toate neamurile, botezându-le în numele  Tatalui ?i al Fiului ?i al Sfântului Duh,
Mat 28:20  Înva?ându-le sa pazeasca toate câte v-am poruncit voua, ?i iata Eu cu voi sunt în toate zilele, pâna la sfâr?itul veacului. Amin.


\end{document}