\begin{document}

\title{Mark}

Mar 1:1  Începutul Evangheliei lui Iisus Hristos, Fiul lui Dumnezeu,
Mar 1:2  Precum este scris în proorocie (la Maleahi) ?i Isaia: "Iata Eu trimit îngerul Meu înaintea fe?ei Tale, care va pregati calea Ta.
Mar 1:3  Glasul celui ce striga în pustie: Gati?i calea Domnului, drepte face?i cararile Lui".
Mar 1:4  Ioan boteza în pustie, propovaduind botezul pocain?ei întru iertarea pacatelor.
Mar 1:5  ?i ie?eau la el tot ?inutul Iudeii ?i to?i cei din Ierusalim ?i se botezau de catre el, în râul Iordan, marturisindu-?i pacatele.
Mar 1:6  ?i Ioan era îmbracat în haina de par de camila, avea cingatoare de piele împrejurul mijlocului ?i mânca lacuste ?i miere salbatica.
Mar 1:7  ?i propovaduia, zicând: Vine în urma mea Cel ce este mai tare decât mine, Caruia nu sunt vrednic, plecându-ma, sa-I dezleg cureaua încal?amintelor.
Mar 1:8  Eu v-am botezat pe voi cu apa, El însa va va boteza cu Duh Sfânt.
Mar 1:9  ?i în zilele acelea, Iisus a venit din Nazaretul Galileii ?i s-a botezat în Iordan, de catre Ioan.
Mar 1:10  ?i îndata, ie?ind din apa, a vazut cerurile deschise ?i Duhul ca un porumbel coborându-Se peste El.
Mar 1:11  ?i glas s-a facut din ceruri: Tu e?ti Fiul Meu cel iubit, întru Tine am binevoit.
Mar 1:12  ?i îndata Duhul L-a mânat în pustie.
Mar 1:13  ?i a fost în pustie patruzeci de zile, fiind ispitit de satana. ?i era împreuna cu fiarele ?i îngerii Îi slujeau.
Mar 1:14  Dupa ce Ioan a fost prins, Iisus a venit în Galileea, propovaduind Evanghelia împara?iei lui Dumnezeu.
Mar 1:15  ?i zicând: S-a împlinit vremea ?i s-a apropiat împara?ia lui Dumnezeu. Pocai?i-va ?i crede?i în Evanghelie.
Mar 1:16  ?i umblând pe lânga Marea Galileii, a vazut pe Simon ?i pe Andrei, fratele lui Simon, aruncând mrejele în mare, caci ei erau pescari.
Mar 1:17  ?i le-a zis Iisus: Veni?i dupa Mine ?i va voi face sa fi?i pescari de oameni.
Mar 1:18  ?i îndata, lasând mrejele, au mers dupa El.
Mar 1:19  ?i mergând pu?in mai înainte, a vazut pe Iacov al lui Zevedeu ?i pe Ioan, fratele lui. ?i ei erau în corabie, dregându-?i mrejele.
Mar 1:20  ?i i-a chemat pe ei îndata. Iar ei, lasând pe tatal lor Zevedeu în corabie, cu lucratorii, s-au dus dupa El.
Mar 1:21  ?i venind în Capernaum ?i îndata intrând sâmbata în sinagoga, îi înva?a.
Mar 1:22  ?i erau uimi?i de înva?atura Lui, caci El îi înva?a pe ei ca Cel ce are putere, iar nu în felul carturarilor.
Mar 1:23  ?i era în sinagoga lor un om cu duh necurat, care striga tare,
Mar 1:24  Zicând: Ce ai cu noi, Iisuse Nazarinene? Ai venit ca sa ne pierzi? Te ?tim cine e?ti: Sfântul lui Dumnezeu.
Mar 1:25  ?i Iisus l-a certat, zicând: Taci ?i ie?i din el.
Mar 1:26  ?i scuturându-l duhul cel necurat ?i strigând cu glas mare, a ie?it din el.
Mar 1:27  ?i s-au spaimântat to?i, încât se întrebau între ei, zicând: Ce este aceasta? O înva?atura noua ?i cu putere; ca ?i duhurilor necurate le porunce?te, ?i I se supun.
Mar 1:28  ?i a ie?it vestea despre El îndata pretutindeni în toata împrejurimea Galileii.
Mar 1:29  ?i îndata ie?ind ei din sinagoga, au venit în casa lui Simon ?i a lui Andrei, cu Iacov ?i cu Ioan.
Mar 1:30  Iar soacra lui Simon zacea, prinsa de friguri, ?i îndata I-au vorbit despre ea.
Mar 1:31  ?i apropiindu-Se a ridicat-o, apucând-o de mâna. ?i au lasat-o frigurile ?i ea le slujea.
Mar 1:32  Iar când s-a facut seara ?i soarele apusese, au adus la El pe to?i bolnavii ?i demoniza?ii.
Mar 1:33  ?i toata cetatea era adunata la u?a.
Mar 1:34  ?i a tamaduit pe mul?i care patimeau de felurite boli ?i demoni mul?i a alungat. Iar pe demoni nu-i lasa sa vorbeasca, pentru ca-L ?tiau ca El e Hristos.
Mar 1:35  ?i a doua zi, foarte de diminea?a, sculându-Se, a ie?it ?i S-a dus într-un loc pustiu ?i Se ruga acolo.
Mar 1:36  ?i a mers dupa El Simon ?i cei ce erau cu el.
Mar 1:37  ?i aflându-L, I-au zis: To?i Te cauta pe Tine.
Mar 1:38  ?i El a zis lor: Sa mergem în alta parte, prin ceta?ile ?i satele învecinate, ca sa propovaduiesc ?i acolo, caci pentru aceasta am venit.
Mar 1:39  ?i venind propovaduia în sinagogile lor, în toata Galileea, alungând pe demoni.
Mar 1:40  ?i un lepros a venit la El, rugându-L ?i îngenunchind ?i zicând: De voie?ti, po?i sa ma cura?e?ti.
Mar 1:41  ?i facându-I-se mila, a întins mâna ?i S-a atins de el ?i i-a zis: Voiesc, cura?e?te-te.
Mar 1:42  ?i îndata s-a îndepartat lepra de la el ?i s-a cura?it.
Mar 1:43  ?i poruncindu-i cu asprime, îndata l-a alungat,
Mar 1:44  ?i i-a zis: Vezi, nimanui sa nu spui nimic, ci mergi de te arata preotului ?i adu, pentru cura?irea ta, cele ce a rânduit Moise, spre marturie lor.
Mar 1:45  Iar el, ie?ind, a început sa propovaduiasca multe ?i sa raspândeasca cuvântul, încât Iisus nu mai putea sa intre pe fa?a în cetate, ci statea afara, în locuri pustii, ?i veneau la El de pretutindeni.
Mar 2:1  ?i intrând iara?i în Capernaum, dupa câteva zile s-a auzit ca este în casa.
Mar 2:2  ?i îndata s-au adunat mul?i, încât nu mai era loc, nici înaintea u?ii, ?i le graia lor cuvântul.
Mar 2:3  ?i au venit la El, aducând un slabanog, pe care-l purtau patru in?i.
Mar 2:4  ?i neputând ei, din pricina mul?imii, sa se apropie de El, au desfacut acoperi?ul casei unde era Iisus ?i, prin spartura, au lasat în jos patul în care zacea slabanogul.
Mar 2:5  ?i vazând Iisus credin?a lor, i-a zis slabanogului: Fiule, iertate î?i sunt pacatele tale!
Mar 2:6  ?i erau acolo unii dintre carturari, care ?edeau ?i cugetau în inimile lor:
Mar 2:7  Pentru ce vorbe?te Acesta astfel? El hule?te. Cine poate sa ierte pacatele, fara numai unul Dumnezeu?
Mar 2:8  ?i îndata cunoscând Iisus, cu duhul Lui, ca a?a cugetau ei în sine, le-a zis lor: De ce cugeta?i acestea în inimile voastre?
Mar 2:9  Ce este mai u?or a zice slabanogului: Iertate î?i sunt pacatele, sau a zice: Scoala-te, ia-?i patul tau ?i umbla?
Mar 2:10  Dar, ca sa ?ti?i ca putere are Fiul Omului a ierta pacatele pe pamânt, a zis slabanogului:
Mar 2:11  Zic ?ie: Scoala-te, ia-?i patul tau ?i mergi la casa ta.
Mar 2:12  ?i s-a sculat îndata ?i, luându-?i patul, a ie?it înaintea tuturor, încât erau to?i uimi?i ?i slaveau pe Dumnezeu, zicând: Asemenea lucruri n-am vazut niciodata.
Mar 2:13  ?i iara?i a ie?it la mare ?i toata mul?imea venea la El ?i îi înva?a.
Mar 2:14  ?i trecând, a vazut pe Levi al lui Alfeu, ?ezând la vama, ?i i-a zis: Urmeaza-Mi! Iar el, sculându-se, I-a urmat.
Mar 2:15  ?i când ?edea El în casa lui Levi, mul?i vame?i ?i pacato?i ?edeau la masa cu Iisus ?i cu ucenicii Lui. Ca erau mul?i ?i-I urmau.
Mar 2:16  Iar carturarii ?i fariseii, vazându-L ca manânca împreuna cu vame?ii ?i pacato?ii, ziceau catre ucenicii Lui: De ce manânca ?i bea Înva?atorul vostru cu vame?ii ?i pacato?ii?
Mar 2:17  Dar, auzind, Iisus le-a zis: Nu cei sanato?i au nevoie de doctor, ci cei bolnavi. N-am venit sa chem pe cei drep?i ci pe pacato?i la pocain?a.
Mar 2:18  Ucenicii lui Ioan ?i ai fariseilor posteau ?i au venit ?i I-au zis Lui: De ce ucenicii lui Ioan ?i ucenicii fariseilor postesc, iar ucenicii Tai nu postesc?
Mar 2:19  ?i Iisus le-a zis: Pot, oare, prietenii mirelui sa posteasca cât timp este mirele cu ei? Câta vreme au pe mire cu ei, nu pot sa posteasca.
Mar 2:20  Dar vor veni zile, când se va lua mirele de la ei ?i atunci vor posti în acele zile.
Mar 2:21  Nimeni nu coase la haina veche petic dintr-o bucata de stofa noua, iar de nu, peticul nou va trage din haina veche ?i se va face o ruptura ?i mai rea.
Mar 2:22  Nimeni, iara?i, nu pune vin nou în burdufuri vechi, iar de nu, vinul nou sparge burdufurile ?i vinul se vara ?i burdufurile se strica; încât vinul nou trebuie sa fie în burdufuri noi.
Mar 2:23  ?i pe când mergea El într-o sâmbata prin semanaturi, ucenicii Lui, în drumul lor, au început sa smulga spice.
Mar 2:24  ?i fariseii Îi ziceau: Vezi, de ce fac sâmbata ce nu se cuvine?
Mar 2:25  ?i Iisus le-a raspuns: Au niciodata n-a?i citit ce a facut David, când a avut nevoie ?i a flamânzit, el ?i cei ce erau cu el?
Mar 2:26  Cum a intrat în casa lui Dumnezeu, în zilele lui Abiatar arhiereul, ?i a mâncat pâinile punerii înainte, pe care nu se cuvenea sa le manânce decât numai preo?ii, ?i a dat ?i celor ce erau cu el?
Mar 2:27  ?i le zicea lor: Sâmbata a fost facut pentru om, iar nu omul pentru sâmbata.
Mar 2:28  Astfel ca Fiul Omului este domn ?i al sâmbetei.
Mar 3:1  ?i iara?i a intrat în sinagoga. ?i era acolo un om având mâna uscata.
Mar 3:2  ?i Îl pândeau pe Iisus sa vada daca îl va vindeca sâmbata, ca sa-L învinuiasca.
Mar 3:3  ?i a zis omului care avea mâna uscata: Ridica-te în mijloc!
Mar 3:4  ?i a zis lor: Se cuvine, sâmbata, a face bine sau a face rau, a mântui un suflet sau a-l pierde? Dar ei taceau;
Mar 3:5  ?i privindu-i pe ei cu mânie ?i întristându-Se de învârto?area inimii lor, a zis omului: Întinde mâna ta! ?i a întins-o, ?i mâna lui s-a facut sanatoasa.
Mar 3:6  ?i ie?ind, fariseii au facut îndata sfat cu irodianii împotriva Lui, ca sa-L piarda.
Mar 3:7  Iisus, împreuna cu ucenicii Lui, a plecat înspre mare ?i mul?ime multa din Galileea ?i din Iudeea L-a urmat.
Mar 3:8  Din Ierusalim, din Idumeea, de dincolo de Iordan, dimprejurul Tirului ?i Sidonului, mul?ime mare, care, auzind câte facea, a venit la El.
Mar 3:9  ?i a zis ucenicilor Sai sa-I fie pusa la îndemâna o corabioara, ca sa nu-L îmbulzeasca mul?imea;
Mar 3:10  Fiindca vindecase pe mul?i, de aceea navaleau asupra Lui, ca sa se atinga de El to?i câ?i erau bolnavi.
Mar 3:11  Iar duhurile cele necurate, când Îl vedeau, cadeau înaintea Lui ?i strigau, zicând: Tu e?ti Fiul lui Dumnezeu.
Mar 3:12  ?i El le certa mult ca sa nu-L dea pe fa?a.
Mar 3:13  ?i S-a suit pe munte ?i a chemat la Sine pe câ?i a voit, ?i au venit la El.
Mar 3:14  ?i a rânduit pe cei doisprezece, pe care i-a numit apostoli, ca sa fie cu El ?i sa-i trimita  sa propovaduiasca,
Mar 3:15  ?i sa aiba putere sa vindece bolile ?i sa alunge demonii.
Mar 3:16  Deci a rânduit pe cei doisprezece: pe Simon, caruia i-a pus numele Petru;
Mar 3:17  Pe Iacov al lui Zevedeu ?i pe Ioan, fratele lui Iacov, ?i le-a pus lor numele Boanerghes, adica fii tunetului.
Mar 3:18  ?i pe Andrei, ?i pe Filip, ?i pe Bartolomeu, pe Matei, ?i pe Toma, ?i pe Iacov al lui Alfeu, ?i pe Tadeu, ?i pe Simon Cananeul,
Mar 3:19  ?i pe Iuda Iscarioteanul, cel care L-a ?i vândut.
Mar 3:20  ?i a venit în casa, ?i iara?i mul?imea s-a adunat, încât ei nu puteau nici sa manânce.
Mar 3:21  ?i auzind ai Sai, au ie?it ca sa-L prinda, ca ziceau: ?i-a ie?it din fire.
Mar 3:22  Iar carturarii, care veneau din Ierusalim, ziceau ca are pe Beelzebul ?i ca, cu domnul demonilor, alunga demonii.
Mar 3:23  ?i chemându-i la Sine, le-a vorbit în pilde: Cum poate satana sa alunge pe satana?
Mar 3:24  Daca o împara?ie se va dezbina în sine, acea împara?ie nu mai poate dainui.
Mar 3:25  ?i daca o casa se va dezbina în sine, casa aceea nu va mai putea sa se ?ina.
Mar 3:26  ?i daca satana s-a sculat împotriva sa însu?i ?i s-a dezbinat, nu poate sa dainuiasca, ci are sfâr?it.
Mar 3:27  Dar nimeni nu poate, intrând în casa celui tare, sa-i rapeasca lucrurile, de nu va lega întâi pe cel tare, ?i atunci va jefui casa lui.
Mar 3:28  Adevarat graiesc voua ca toate vor fi iertate fiilor oamenilor, pacatele ?i hulele câte vor fi hulit;
Mar 3:29  Dar cine va huli împotriva Duhului Sfânt nu are iertare în veac, ci este vinovat de osânda ve?nica.
Mar 3:30  Pentru ca ziceau: Are duh necurat.
Mar 3:31  ?i au venit mama Lui ?i fra?ii Lui ?i, stând afara, au trimis la El ca sn-L cheme.
Mar 3:32  Iar mul?imea ?edea împrejurul Lui. ?i I-au zis unii: Iata mama Ta ?i fra?ii Tai ?i surorile Tale sunt afara. Te cauta.
Mar 3:33  ?i, raspunzând lor, le-a zis: Cine este mama Mea ?i fra?ii Mei?
Mar 3:34  ?i privind pe cei ce ?edeau în jurul Lui, a zis: Iata mama Mea ?i fra?ii Mei.
Mar 3:35  Ca oricine va face voia lui Dumnezeu, acesta este fratele Meu ?i sora Mea ?i mama Mea.
Mar 4:1  ?i iara?i a început Iisus sa înve?e, lânga mare, ?i s-a adunat la El mul?ime foarte multa, încât El a intrat în corabie ?i ?edea pe mare, iar toata mul?imea era lânga mare, pe uscat.
Mar 4:2  ?i-i înva?a multe în pilde, ?i în înva?atura Sa le zicea:
Mar 4:3  Asculta?i: Iata, ie?it-a semanatorul sa semene.
Mar 4:4  ?i pe când semana el, o samân?a a cazut lânga cale ?i pasarile cerului au venit ?i au mâncat-o.
Mar 4:5  ?i alta a cazut pe loc pietros, unde nu avea pamânt mult, ?i îndata a rasarit, pentru ca nu avea pamânt mult.
Mar 4:6  ?i când s-a ridicat soarele, s-a ve?tejit ?i, neavând radacina, s-a uscat.
Mar 4:7  Alta samân?a a cazut în spini, a crescut, dar spinii au înabu?it-o ?i rod n-a dat.
Mar 4:8  ?i altele au cazut pe pamântul cel bun ?i, înal?ându-se ?i crescând, au dat roade ?i au adus: una treizeci, alta ?aizeci, alta o suta.
Mar 4:9  ?i zicea: Cine are urechi de auzit sa auda.
Mar 4:10  Iar când a fost singur, cei ce erau lânga El, împreuna cu cei doisprezece, Îl întrebau despre pilde.
Mar 4:11  ?i le-a raspuns: Voua va e dat sa cunoa?te?i taina împara?iei lui Dumnezeu, dar pentru cei de afara totul se face în pilde,
Mar 4:12  Ca uitându-se, sa priveasca ?i sa nu vada, ?i, auzind, sa nu în?eleaga, ca nu cumva sa se întoarca ?i sa fie ierta?i.
Mar 4:13  ?i le-a zis: Nu pricepe?i pilda aceasta? Dar cum ve?i în?elege toate pildele?
Mar 4:14  Semanatorul seamana cuvântul.
Mar 4:15  Cele de lânga cale sunt aceia în care se seamana cuvântul, ?i, când îl aud, îndata vine satana ?i ia cuvântul cel semanat în inimile lor.
Mar 4:16  Cele semanate pe loc pietros sunt aceia care, când aud cuvântul, îl primesc îndata cu bucurie,
Mar 4:17  Dar n-au radacina în ei, ci ?in pâna la un timp; apoi când se întâmpla strâmtorare sau prigoana pentru cuvânt, îndata se smintesc.
Mar 4:18  ?i cele semanate între spini sunt cei ce asculta cuvântul,
Mar 4:19  Dar grijile veacului ?i în?elaciunea boga?iei ?i poftele dupa celelalte, patrunzând în ei, înabu?a cuvântul ?i îl fac neroditor.
Mar 4:20  Iar cele semanate pe pamântul cel bun sunt cei ce aud cuvântul ?i-l primesc ?i aduc roade: unul treizeci, altul ?aizeci ?i altul o suta.
Mar 4:21  ?i le zicea: Se aduce oare faclia ca sa fie pusa sub obroc sau sub pat? Oare nu ca sa fie pusa în sfe?nic?
Mar 4:22  Caci nu e nimic ascuns ca sa nu se dea pe fa?a; nici n-a fost ceva tainuit, decât ca sa vina la aratare.
Mar 4:23  Cine are urechi de auzit sa auda.
Mar 4:24  ?i le zicea: Lua?i seama la ce auzi?i: Cu ce masura masura?i, vi se va masura; iar voua celor ce asculta?i, vi se va da ?i va va prisosi.
Mar 4:25  Caci celui ce are i se va da; dar de la cel ce nu are, ?i ce are i se va lua.
Mar 4:26  ?i zicea: A?a este împara?ia lui Dumnezeu, ca un om care arunca samân?a în pamânt,
Mar 4:27  ?i doarme ?i se scoala, noaptea ?i ziua, ?i samân?a rasare ?i cre?te, cum nu ?tie el.
Mar 4:28  Pamântul rode?te de la sine: mai întâi pai, apoi spic, dupa aceea grâu deplin în spic.
Mar 4:29  Iar când rodul se coace, îndata trimite secera, ca a sosit seceri?ul.
Mar 4:30  ?i zicea: Cum vom asemana împara?ia lui Dumnezeu, sau în ce pilda o vom închipui?
Mar 4:31  Cu grauntele de mu?tar care, când se seamana în pamânt, este mai mic decât toate semin?ele de pe pamânt;
Mar 4:32  Dar, dupa ce s-a semanat, cre?te ?i se face mai mare decât toate legumele ?i face ramuri mari, încât sub umbra lui pot sa sala?luiasca pasarile cerului.
Mar 4:33  ?i cu multe pilde ca acestea le graia cuvântul dupa cum puteau sa în?eleaga.
Mar 4:34  Iar fara pilda nu le graia; ?i ucenicilor Sai le lamurea toate, deosebi.
Mar 4:35  ?i în ziua aceea, când s-a înserat, a zis catre ei: Sa trecem pe ?armul celalalt.
Mar 4:36  ?i lasând ei mul?imea, L-au luat cu ei în corabie, a?a cum era, caci erau cu El ?i alte corabii.
Mar 4:37  ?i s-a pornit o furtuna mare de vânt ?i valurile se pravaleau peste corabie, încât corabia era aproape sa se umple.
Mar 4:38  Iar Iisus era la partea dindarat a corabiei, dormind pe capatâi. L-au de?teptat ?i I-au zis: Înva?atorule, nu-?i este grija ca pierim?
Mar 4:39  ?i El, sculându-Se, a certat vântul ?i a poruncit marii: Taci! Înceteaza! ?i vântul s-a potolit ?i s-a facut lini?te mare.
Mar 4:40  ?i le-a zis lor: Pentru ce sunte?i a?a de frico?i? Cum de nu ave?i credin?a?
Mar 4:41  ?i s-au înfrico?at cu frica mare ?i ziceau unul catre altul: Cine este oare, Acesta, ca ?i vântul ?i marea I se supun?
Mar 5:1  ?i a venit de cealalta parte a marii în ?inutul Gadarenilor.
Mar 5:2  Iar dupa ce a ie?it din corabie, îndata L-a întâmpinat, din morminte, un om cu duh necurat,
Mar 5:3  Care î?i avea locuin?a în morminte, ?i nimeni nu putea sa-l lege nici macar în lan?uri,
Mar 5:4  Pentru ca de multe ori fiind legat în obezi ?i lan?uri, el rupea lan?urile, ?i obezile le sfarâma, ?i nimeni nu putea sa-l potoleasca;
Mar 5:5  ?i neîncetat noaptea ?i ziua era prin morminte ?i prin mun?i, strigând ?i taindu-se cu pietre.
Mar 5:6  Iar vazându-L de departe pe Iisus, a alergat ?i s-a închinat Lui.
Mar 5:7  ?i strigând cu glas puternic, a zis: Ce ai cu mine, Iisuse, Fiule al lui Dumnezeu Celui Preaînalt? Te jur pe Dumnezeu sa nu ma chinuie?ti.
Mar 5:8  Caci îi zicea: Ie?i duh necurat din omul acesta.
Mar 5:9  ?i l-a întrebat: Care î?i este numele? ?i I-a raspuns: Legiune este numele meu, caci suntem mul?i.
Mar 5:10  ?i Îl rugau mult sa nu-i trimita afara din acel ?inut.
Mar 5:11  Iar acolo, lânga munte, era o turma mare de porci, care pa?tea.
Mar 5:12  ?i L-au rugat, zicând: Trimite-ne pe noi în porci, ca sa intram în ei.
Mar 5:13  ?i El le-a dat voie. Atunci, ie?ind, duhurile necurate au intrat în porci ?i turma s-a aruncat de pe ?armul înalt, în mare. ?i erau ca la doua mii ?i s-au înecat în mare.
Mar 5:14  Iar cei care-i pa?teau au fugit ?i au vestit în cetate ?i prin sate. ?i au venit oamenii sa vada ce s-a întâmplat.
Mar 5:15  ?i s-au dus la Iisus ?i au vazut pe cel demonizat ?ezând jos, îmbracat ?i întreg la minte, el care avusese legiune de demoni, ?i s-au înfrico?at.
Mar 5:16  Iar cei ce au vazut le-au povestit cum a fost cu demonizatul ?i despre porci.
Mar 5:17  ?i ei au început sa-L roage sa se duca din hotarele lor.
Mar 5:18  Iar intrând El în corabie, cel ce fusese demonizat Îl ruga ca sa-l ia cu El.
Mar 5:19  Iisus însa nu l-a lasat, ci i-a zis: Mergi în casa ta, la ai tai, ?i spune-le câte ?i-a facut ?ie Domnul ?i cum te-a miluit.
Mar 5:20  Iar el s-a dus ?i a început sa vesteasca în Decapole câte i-a facut Iisus lui; ?i to?i se minunau.
Mar 5:21  ?i trecând Iisus cu corabia iara?i de partea cealalta, s-a adunat la El mul?ime multa ?i era lânga mare.
Mar 5:22  ?i a venit unul din mai-marii sinagogilor, anume Iair, ?i vazându-L pe Iisus, a cazut la picioarele Lui,
Mar 5:23  ?i L-a rugat mult, zicând: Fiica mea este pe moarte, ci, venind, pune mâinile tale peste ea, ca sa scape ?i sa traiasca.
Mar 5:24  ?i a mers cu el. ?i mul?ime multa îl urma pe Iisus ?i Îl îmbulzea.
Mar 5:25  ?i era o femeie care avea, de doisprezece ani, curgere de sânge.
Mar 5:26  ?i multe îndurase de la mul?i doctori, cheltuindu-?i toate ale sale, dar nefolosind nimic, ci mai mult mergând înspre mai rau.
Mar 5:27  Auzind ea cele despre Iisus, a venit în mul?ime ?i pe la spate s-a atins de haina Lui.
Mar 5:28  Caci î?i zicea: De ma voi atinge macar de haina Lui, ma voi vindeca!
Mar 5:29  ?i îndata izvorul sângelui ei a încetat ?i ea a sim?it în trup ca s-a vindecat de boala.
Mar 5:30  ?i îndata, cunoscând Iisus în Sine puterea ie?ita din El, întorcându-Se catre mul?ime, a întrebat: Cine s-a atins de Mine?
Mar 5:31  ?i I-au zis ucenicii Lui: Vezi mul?imea îmbulzindu-Te ?i zici: Cine s-a atins de Mine?
Mar 5:32  ?i Se uita împrejur sa vada pe aceea care facuse aceasta.
Mar 5:33  Iar femeia, înfrico?ându-se ?i tremurând, ?tiind ce i se facuse, a venit ?i a cazut înaintea Lui ?i I-a marturisit tot adevarul;
Mar 5:34  Iar El i-a zis: Fiica, credin?a ta te-a mântuit, mergi în pace ?i fii sanatoasa de boala ta!
Mar 5:35  Înca vorbind El, au venit unii de la mai-marele sinagogii, zicând: Fiica ta a murit. De ce mai superi pe Înva?atorul?
Mar 5:36  Dar Iisus, auzind cuvântul ce s-a grait, a zis mai-marelui sinagogii: Nu te teme. Crede numai.
Mar 5:37  ?i n-a lasat pe nimeni sa mearga cu El, decât numai pe Petru ?i pe Iacov ?i pe Ioan, fratele lui Iacov.
Mar 5:38  ?i au venit la casa mai-marelui sinagogii ?i a vazut tulburare ?i pe cei ce plângeau ?i se tânguiau mult.
Mar 5:39  ?i intrând, le-a zis: De va tulbura?i ?i plânge?i? Copila n-a murit, ci doarme.
Mar 5:40  ?i-L luau în râs. Iar El, sco?ându-i pe to?i afara, a luat cu Sine pe tatal copilei, pe mama ei ?i pe cei ce îl înso?eau ?i a intrat unde era copila.
Mar 5:41  ?i apucând pe copila de mâna, i-a grait: Talita kumi, care se tâlcuie?te: Fiica, ?ie zic, scoala-te!
Mar 5:42  ?i îndata s-a sculat copila ?i umbla, caci era de doisprezece ani. ?i s-au mirat îndata cu uimire mare.
Mar 5:43  Dar El le-a poruncit, cu staruin?a, ca nimeni sa nu afle de aceasta. ?i le-a zis sa-i dea copilei sa manânce.
Mar 6:1  ?i a ie?it de acolo ?i a venit în patria Sa, iar ucenicii Lui au mers dupa El.
Mar 6:2  ?i, fiind sâmbata, a început sa înve?e în sinagoga. ?i mul?i, auzindu-L, erau uimi?i ?i ziceau: De unde are El acestea? ?i ce este în?elepciunea care I s-a dat Lui? ?i cum se fac minuni ca acestea prin mâinile Lui?
Mar 6:3  Au nu este Acesta teslarul, fiul Mariei ?i fratele lui Iacov ?i al lui Iosi ?i al lui Iuda ?i al lui Simon? ?i nu sunt, oare, surorile Lui aici la noi? ?i se sminteau întru El.
Mar 6:4  ?i le zicea Iisus: Nu este prooroc dispre?uit, decât în patria sa ?i între rudele sale ?i în casa sa.
Mar 6:5  ?i n-a putut acolo sa faca nici o minune, decât ca, punându-?i mâinile peste pu?ini bolnavi, i-a vindecat.
Mar 6:6  ?i se mira de necredin?a lor. ?i strabatea satele dimprejur înva?ând.
Mar 6:7  ?i a chemat la Sine pe cei doisprezece ?i a început sa-i trimita doi câte doi ?i le-a dat putere asupra duhurilor necurate.
Mar 6:8  ?i le-a poruncit sa nu ia nimic cu ei, pe cale, ci numai toiag. Nici pâine, nici traista, nici bani la cingatoare;
Mar 6:9  Ci sa fie încal?a?i cu sandale ?i sa nu se îmbrace cu doua haine.
Mar 6:10  ?i le zicea: În orice casa ve?i intra, acolo sa ramâne?i pâna ce ve?i ie?i de acolo.
Mar 6:11  ?i daca într-un loc nu va vor primi pe voi, nici nu va vor asculta, ie?ind de acolo, scutura?i praful de sub picioarele voastre, spre marturie lor. Adevarat graiesc voua: Mai u?or va fi Sodomei ?i Gomorei, în ziua judeca?ii, decât ceta?ii aceleia.
Mar 6:12  ?i ie?ind, ei propovaduiau sa se pocaiasca.
Mar 6:13  ?i scoteau mul?i demoni ?i ungeau cu untdelemn pe mul?i bolnavi ?i-i vindecau.
Mar 6:14  ?i a auzit regele Irod, caci numele lui Iisus se facuse cunoscut, ?i zicea ca Ioan Botezatorul s-a sculat din mor?i ?i de aceea se fac minuni prin el.
Mar 6:15  Al?ii însa ziceau ca este Ilie ?i al?ii ca este prooroc, ca unul din prooroci.
Mar 6:16  Iar Irod, auzind zicea: Este Ioan caruia eu am pus sa-i taie capul; el s-a sculat din mor?i.
Mar 6:17  Caci Irod, trimi?ând, l-a prins pe Ioan ?i l-a legat, în temni?a, din pricina Irodiadei, femeia lui Filip, fratele sau, pe care o luase de so?ie.
Mar 6:18  Caci Ioan îi zicea lui Irod: Nu-?i este îngaduit sa ?ii pe femeia fratelui tau.
Mar 6:19  Iar Irodiada îl ura ?i voia sa-l omoare, dar nu putea,
Mar 6:20  Caci Irod se temea de Ioan, ?tiindu-l barbat drept ?i sfânt, ?i-l ocrotea. ?i ascultându-l, multe facea ?i cu drag îl asculta.
Mar 6:21  ?i fiind o zi cu bun prilej, când Irod, de ziua sa de na?tere, a facut ospa? dregatorilor lui ?i capeteniilor o?tirii ?i frunta?ilor din Galileea,
Mar 6:22  ?i fiica Irodiadei, intrând ?i jucând, a placut lui Irod ?i celor ce ?edeau cu el la masa. Iar regele a zis fetei: Cere de la mine orice vei voi ?i î?i voi da.
Mar 6:23  ?i s-a jurat ei: Orice vei cere de la mine î?i voi da, pâna la jumatate din regatul meu.
Mar 6:24  ?i ea, ie?ind, a zis mamei sale: Ce sa cer? Iar Irodiada i-a zis: Capul lui Ioan Botezatorul.
Mar 6:25  ?i intrând îndata, cu graba, la rege, i-a cerut, zicând: Vreau sa-mi dai îndata, pe tipsie, capul lui Ioan Botezatorul.
Mar 6:26  ?i regele s-a mâhnit adânc, dar pentru juramânt ?i pentru cei ce ?edeau cu el la masa, n-a voit s-o întristeze.
Mar 6:27  ?i îndata trimi?ând regele un paznic, a poruncit a-i aduce capul.
Mar 6:28  ?i acela, mergând, i-a taiat capul în temni?a, l-a adus pe tipsie ?i l-a dat fetei, iar fata l-a dat mamei sale.
Mar 6:29  ?i auzind, ucenicii lui au venit, au luat trupul lui Ioan ?i l-au pus în mormânt.
Mar 6:30  ?i s-au adunat apostolii la Iisus ?i I-au spus Lui toate câte au facut ?i au înva?at.
Mar 6:31  ?i El le-a zis: Veni?i voi în?iva de o parte, în loc pustiu, ?i odihni?i-va pu?in. Caci mul?i erau care veneau ?i mul?i erau care se duceau ?i nu mai aveau timp nici sa manânce.
Mar 6:32  ?i au plecat cu corabia spre un loc pustiu, de o parte.
Mar 6:33  ?i i-au vazut plecând ?i mul?i au în?eles ?i au alergat acolo pe jos de prin toate ceta?ile ?i au sosit înaintea lor.
Mar 6:34  ?i ie?ind din corabie, Iisus a vazut mul?ime mare ?i I s-a facut mila de ei, caci erau ca ni?te oi fara pastor, ?i a început sa-i înve?e multe.
Mar 6:35  Dar facându-se târziu, ucenicii Lui, apropiindu-se, I-au zis: Locul e pustiu ?i ceasul e târziu;
Mar 6:36  Sloboze?te-i, ca mergând prin ceta?ile ?i prin satele dimprejur, sa-?i cumpere sa manânce.
Mar 6:37  Raspunzând, El le-a zis: Da?i-le voi sa manânce. ?i ei I-au zis: Sa mergem noi sa cumparam pâini de doua sute de dinari ?i sa le dam sa manânce?
Mar 6:38  Iar El le-a zis: Câte pâini ave?i? Duce?i-va ?i vede?i. ?i dupa ce au vazut, I-au spus: Cinci pâini ?i doi pe?ti.
Mar 6:39  ?i El le-a poruncit sa-i a?eze pe to?i cete, cete, pe iarba verde.
Mar 6:40  ?i au ?ezut cete, cete, câte o suta ?i câte cincizeci.
Mar 6:41  ?i luând cele cinci pâini ?i cei doi pe?ti, privind la cer, a binecuvântat ?i a frânt pâinile ?i le-a dat ucenicilor, ca sa le puna înainte, asemenea ?i cei doi pe?ti i-a împar?it tuturor.
Mar 6:42  ?i au mâncat to?i ?i s-au saturat.
Mar 6:43  ?i au luat douasprezece co?uri pline cu farâmituri ?i cu ce-a ramas din pe?ti.
Mar 6:44  Iar cei ce au mâncat pâinile erau cinci mii de barba?i.
Mar 6:45  ?i îndata a silit pe ucenicii Lui sa intre în corabie ?i sa mearga înaintea Lui, de cealalta parte, spre Betsaida, pâna ce El va slobozi mul?imea.
Mar 6:46  Iar dupa ce i-a slobozit, S-a dus în munte ca sa Se roage.
Mar 6:47  ?i facându-se seara, era corabia în mijlocul marii, iar El singur pe ?arm.
Mar 6:48  ?i i-a vazut cum se chinuiau vâslind, caci vântul le era împotriva. ?i catre a patra straja a nop?ii a venit la ei umblând pe mare ?i voia sa treaca pe lânga ei.
Mar 6:49  Iar lor, vazându-L umblând pe mare, li s-a parut ca este naluca ?i au strigat.
Mar 6:50  Caci to?i L-au vazut ?i s-au tulburat. Dar îndata El a vorbit cu ei ?i le-a zis: Îndrazni?i! Eu sunt; nu va teme?i!
Mar 6:51  ?i s-a suit la ei în corabie ?i s-a potolit vântul. ?i erau peste masura de uimi?i în sinea lor;
Mar 6:52  Caci nu pricepusera nimic de la minunea pâinilor, deoarece inima lor era învârto?ata.
Mar 6:53  ?i trecând marea, au venit în ?inutul Ghenizaretului ?i au tras la ?arm.
Mar 6:54  ?i ie?ind ei din corabie, îndata L-au cunoscut.
Mar 6:55  ?i strabateau tot ?inutul acela ?i au început sa-I aduca pe bolnavi pe paturi, acolo unde auzeau ca este El.
Mar 6:56  ?i oriunde intra în sate sau în ceta?i sau în satule?e, puneau la raspântii pe cei bolnavi, ?i-L rugau sa le îngaduie sa se atinga macar de poala hainei Sale. ?i câ?i se atingeau de El se vindecau.
Mar 7:1  ?i s-au adunat la El fariseii ?i unii dintre carturari, care venisera din Ierusalim.
Mar 7:2  ?i vazând pe unii din ucenicii Lui ca manânca cu mâinile necurate, adica nespalate, cârteau;
Mar 7:3  Caci fariseii ?i to?i iudeii, daca nu-?i spala mâinile pâna la cot, nu manânca, ?inând datina batrânilor.
Mar 7:4  ?i când vin din pia?a, daca nu se spala, nu manânca; ?i alte multe sunt pe care au primit sa le ?ina: spalarea paharelor ?i a urcioarelor ?i a vaselor de arama ?i a paturilor.
Mar 7:5  ?i L-au întrebat pe El fariseii ?i carturarii: Pentru ce nu umbla ucenicii Tai dupa datina batrânilor, ci manânca cu mâinile nespalate?
Mar 7:6  Iar El le-a zis: Bine a proorocit Isaia despre voi, fa?arnicilor, precum este scris: "Acest popor Ma cinste?te cu buzele, dar inima lui este departe de Mine".
Mar 7:7  Dar în zadar Ma cinstesc, înva?ând înva?aturi care sunt porunci omene?ti.
Mar 7:8  Caci lasând porunca lui Dumnezeu, ?ine?i datina oamenilor: spalarea urcioarelor ?i a paharelor ?i altele ca acestea multe, pe care le face?i.
Mar 7:9  ?i le zicea lor: Bine, a?i lepadat porunca lui Dumnezeu, ca sa ?ine?i datina voastra!
Mar 7:10  Caci Moise a zis: "Cinste?te pe tatal tau ?i pe mama ta", ?i "cel ce va grai de rau pe tatal sau, sau pe mama sa, cu moarte sa se sfâr?easca".
Mar 7:11  Voi însa zice?i: Daca un om va spune tatalui sau mamei: Corban! adica: Cu ce te-a? fi putut ajuta este daruit lui Dumnezeu,
Mar 7:12  Nu-l mai lasa?i sa faca nimic pentru tatal sau sau pentru mama sa.
Mar 7:13  ?i astfel desfiin?a?i cuvântul lui Dumnezeu cu datina voastra pe care singuri a?i dat-o. ?i face?i multe asemanatoare cu acestea.
Mar 7:14  ?i chemând iara?i mul?imea la El, le zicea: Asculta?i-Ma to?i ?i în?elege?i:
Mar 7:15  Nu este nimic din afara de om care, intrând în el, sa poata sa-l spurce. Dar cele ce ies din om, acelea sunt care îl spurca.
Mar 7:16  De are cineva urechi de auzit sa auda.
Mar 7:17  ?i când a intrat în casa de la mul?ime, L-au întrebat ucenicii despre aceasta pilda.
Mar 7:18  ?i El le-a zis: A?adar ?i voi sunte?i nepricepu?i? Nu în?elege?i, oare, ca tot ce intra în om, din afara, nu poate sa-l spurce?
Mar 7:19  Ca nu intra în inima lui, ci în pântece, ?i iese afara, pe calea sa, bucatele fiind toate curate.
Mar 7:20  Dar zicea ca ceea ce iese din om, aceea spurca pe om.
Mar 7:21  Caci dinauntru, din inima omului, ies cugetele cele rele, desfrânarile, ho?iile, uciderile,
Mar 7:22  Adulterul, lacomiile, vicleniile, în?elaciunea, neru?inarea, ochiul pizma?, hula, trufia, u?uratatea.
Mar 7:23  Toate aceste rele ies dinauntru ?i spurca pe om.
Mar 7:24  ?i ridicându-Se de acolo, S-a dus în hotarele Tirului ?i ale Sidonului ?i, intrând într-o casa, voia ca nimeni sa nu ?tie, dar n-a putut sa ramâna tainuit.
Mar 7:25  Caci îndata auzind despre El o femeie, a carei fiica avea duh necurat, a venit ?i a cazut la picioarele Lui.
Mar 7:26  ?i femeia era pagâna, de neam din Fenicia Siriei. ?i Îl ruga sa alunge demonii din fiica ei.
Mar 7:27  Dar Iisus i-a vorbit: Lasa întâi sa se sature copiii. Caci nu este bine sa iei pâinea copiilor ?i s-o arunci câinilor.
Mar 7:28  Ea însa a raspuns ?i I-a zis: Da, Doamne, dar ?i câinii, sub masa, manânca din farâmiturile copiilor.
Mar 7:29  ?i Iisus i-a zis: Pentru acest cuvânt, mergi. A ie?it demonul din fiica ta.
Mar 7:30  Iar ea, ducându-se acasa, a gasit pe copila culcata în pat, iar demonul ie?ise.
Mar 7:31  ?i, ie?ind din par?ile Tirului, a venit, prin Sidon, la Marea Galileii, prin mijlocul hotarelor Decapolei.
Mar 7:32  ?i I-au adus un surd, care era ?i gângav, ?i L-au rugat ca sa-?i puna mâna peste el.
Mar 7:33  ?i luându-l din mul?ime, la o parte, ?i-a pus degetele în urechile lui, ?i scuipând, S-a atins de limba lui.
Mar 7:34  ?i privind la cer, a suspinat ?i a zis lui: Effatta! ceea ce înseamna: Deschide-te!
Mar 7:35  ?i urechile lui s-au deschis, iar legatura limbii lui îndata s-a dezlegat, ?i vorbea bine.
Mar 7:36  ?i le poruncea sa nu spuna nimanui. Dar, cu cât le poruncea, cu atât mai mult ei Îl vesteau.
Mar 7:37  ?i erau uimi?i peste masura, zicând: Toate le-a facut bine: pe surzi îi face sa auda ?i pe mu?i sa vorbeasca.
Mar 8:1  În zilele acelea, fiind iara?i mul?ime multa ?i neavând ce sa manânce, Iisus, chemând la Sine pe ucenici, le-a zis:
Mar 8:2  Mila Îmi este de mul?ime, ca sunt trei zile de când a?teapta lânga Mine ?i n-au ce sa manânce.
Mar 8:3  ?i de-i voi slobozi flamânzi la casa lor, se vor istovi pe drum, ca unii dintre ei au venit de departe.
Mar 8:4  ?i ucenicii Lui I-au raspuns: De unde va putea cineva sa-i sature pe ace?tia cu pâine, aici în pustie.
Mar 8:5  El însa i-a întrebat: Câte pâini ave?i? Raspuns-au Lui: ?apte.
Mar 8:6  ?i a poruncit mul?imii sa ?eada jos pe pamânt. ?i, luând cele ?apte pâini, a mul?umit, a frânt ?i a dat ucenicilor Sai, ca sa le puna înainte. ?i ei le-au pus mul?imii înainte.
Mar 8:7  ?i aveau ?i pu?ini pe?ti?ori. ?i binecuvântându-i, a zis sa-i puna ?i pe ace?tia înaintea lor.
Mar 8:8  ?i au mâncat ?i s-au saturat ?i au luat ?apte co?uri cu rama?i?e de farâmituri.
Mar 8:9  ?i ei erau ca la patru mii. ?i i-a slobozit.
Mar 8:10  ?i îndata intrând în corabie cu ucenicii Sai, a venit în par?ile Dalmanutei.
Mar 8:11  ?i au ie?it fariseii ?i se sfadeau cu El, cerând de la El semn din cer, ispitindu-L.
Mar 8:12  ?i Iisus, suspinând cu duhul Sau, a zis: Pentru ce neamul acesta cere semn? Adevarat graiesc voua ca nu se va da semn acestui neam.
Mar 8:13  ?i lasându-i, a intrat iara?i în corabie ?i a trecut de cealalta parte.
Mar 8:14  Dar ucenicii au uitat sa ia pâine ?i numai o pâine aveau cu ei în corabie.
Mar 8:15  ?i El le-a poruncit, zicând: Vede?i, pazi?i-va de aluatul fariseilor ?i de aluatul lui Irod.
Mar 8:16  ?i vorbeau între ei, zicând: Aceasta o zice, fiindca n-avem pâine.
Mar 8:17  ?i Iisus, în?elegând, le-a zis: De ce gândi?i ca n-ave?i pâine? Tot nu în?elege?i, nici nu pricepe?i? Atât de învârto?ata este inima voastra?
Mar 8:18  Ochi ave?i ?i nu vede?i, urechi ave?i ?i nu auzi?i ?i nu va aduce?i aminte.
Mar 8:19  Când am frânt cele cinci pâini, la cei cinci mii de oameni, atunci câte co?uri pline de farâmituri a?i luat? Zis-au Lui: Douasprezece.
Mar 8:20  ?i când cu cele ?apte pâini, la cei patru mii de oameni, câte co?uri pline de farâmituri a?i luat? Iar ei au zis: ?apte.
Mar 8:21  ?i le zicea: Tot nu pricepe?i?
Mar 8:22  ?i au venit la Betsaida. ?i au adus la El un orb ?i L-au rugat sa se atinga de el.
Mar 8:23  ?i luând pe orb de mâna, l-a scos afara din sat ?i, scuipând în ochii lui ?i punându-?i mâinile peste el, l-a întrebat daca vede ceva.
Mar 8:24  ?i el, ridicându-?i ochii, a zis: zaresc oamenii; îi vad ca pe ni?te copaci umblând.
Mar 8:25  Dupa aceea a pus iara?i mâinile pe ochii lui, ?i el a vazut bine ?i s-a îndreptat, caci vedea toate, lamurit.
Mar 8:26  ?i l-a trimis la casa sa, zicându-i: Sa nu intri în sat, nici sa spui cuiva din sat.
Mar 8:27  ?i a ie?it Iisus ?i ucenicii Lui prin satele din preajma Cezareii lui Filip. ?i pe drum întreba pe ucenicii Sai, zicându-le: Cine zic oamenii ca sunt?
Mar 8:28  Ei au raspuns Lui, zicând: Unii spun ca e?ti Ioan Botezatorul, al?ii ca e?ti Ilie, iar al?ii ca e?ti unul din prooroci.
Mar 8:29  ?i El i-a întrebat: Dar voi cine zice?i ca sunt Eu? Raspunzând, Petru a zis Lui: Tu e?ti Hristosul.
Mar 8:30  ?i El le-a dat porunca sa nu spuna nimanui despre El.
Mar 8:31  ?i a început sa-i înve?e ca Fiul Omului trebuie sa patimeasca multe ?i sa fie defaimat de batrâni, de arhierei ?i de carturari ?i sa fie omorât, iar dupa trei zile sa învieze.
Mar 8:32  ?i spunea acest cuvânt pe fa?a. ?i luându-L Petru de o parte, a început sa-L dojeneasca.
Mar 8:33  Dar El, întorcându-Se ?i uitându-Se la ucenicii Sai, a certat pe Petru ?i i-a zis: Mergi, înapoia mea, satano! Caci tu nu cuge?i cele ale lui Dumnezeu, ci cele ale oamenilor.
Mar 8:34  ?i chemând la Sine mul?imea, împreuna cu ucenicii Sai, le-a zis: Oricine voie?te sa vina dupa Mine sa se lepede de sine, sa-?i ia crucea ?i sa-Mi urmeze Mie.
Mar 8:35  Caci cine va voi sa-?i scape sufletul îl va pierde, iar cine va pierde sufletul Sau pentru Mine ?i pentru Evanghelie, acela îl va scapa.
Mar 8:36  Caci ce-i folose?te omului sa câ?tige lumea întreaga, daca-?i pierde sufletul?
Mar 8:37  Sau ce ar putea sa dea omul, în schimb, pentru sufletul sau?
Mar 8:38  Caci de cel ce se va ru?ina de Mine ?i de cuvintele Mele, în neamul acesta desfrânat ?i pacatos, ?i Fiul Omului Se va ru?ina de el, când va veni întru slava Tatalui sau cu sfin?ii îngeri.
Mar 9:1  ?i le zicea lor: Adevarat graiesc voua ca sunt unii, din cei ce stau aici, care nu vor gusta moartea, pâna ce nu vor vedea împara?ia lui Dumnezeu, venind întru putere.
Mar 9:2  ?i dupa ?ase zile a luat Iisus cu Sine pe Petru ?i pe Iacov ?i pe Ioan ?i i-a dus într-un munte înalt, de o parte, pe ei singuri, ?i S-a schimbat la fa?a înaintea lor.
Mar 9:3  ?i ve?mintele Lui s-au facut stralucitoare, albe foarte, ca zapada, cum nu poate înalbi a?a pe pamânt înalbitorul.
Mar 9:4  ?i li s-a aratat Ilie împreuna cu Moise ?i vorbeau cu Iisus.
Mar 9:5  ?i raspunzând Petru, a zis lui Iisus: Înva?atorule, bine este ca noi sa fim aici; ?i sa facem trei colibe: ?ie una ?i lui Moise una ?i lui Ilie una.
Mar 9:6  Caci nu ?tia ce sa spuna, fiindca erau înspaimânta?i.
Mar 9:7  ?i s-a facut un nor care îi umbrea, iar un glas din nor a venit zicând: Acesta este Fiul Meu cel iubit, pe Acesta sa-L asculta?i.
Mar 9:8  Dar, deodata, privind ei împrejur, n-au mai vazut pe nimeni decât pe Iisus, singur cu ei.
Mar 9:9  ?i coborându-se ei din munte, le-a poruncit ca nimanui sa nu spuna cele ce vazusera, decât numai când Fiul Omului va învia din mor?i.
Mar 9:10  Iar ei au ?inut cuvântul, întrebându-se între ei: Ce înseamna a învia din mor?i?
Mar 9:11  ?i L-au întrebat pe El, zicând: Pentru ce zic fariseii ?i carturarii ca trebuie sa vina mai întâi Ilie?
Mar 9:12  Iar El le-a raspuns: Ilie, venind întâi, va a?eza iara?i toate. ?i cum este scris despre Fiul Omului ca va sa patimeasca multe ?i sa fie defaimat?
Mar 9:13  Dar va zic voua ca Ilie a ?i venit ?i i-au facut toate câte au voit, precum s-a scris despre el.
Mar 9:14  ?i venind la ucenici, a vazut mul?ime mare împrejurul lor ?i pe carturari sfadindu-se între ei.
Mar 9:15  ?i îndata toata mul?imea, vazându-L, s-a spaimântat ?i, alergând, I se închina.
Mar 9:16  ?i Iisus a întrebat pe carturari: Ce va sfadi?i între voi?
Mar 9:17  ?i I-a raspuns Lui unul din mul?ime: Înva?atorule, am adus la Tine pe fiul meu, care are duh mut.
Mar 9:18  ?i oriunde-l apuca, îl arunca la pamânt ?i face spume la gura ?i scrâ?ne?te din din?i ?i în?epene?te. ?i am zis ucenicilor Tai sa-l alunge, dar ei n-au putut.
Mar 9:19  Iar El, raspunzând lor, a zis: O, neam necredincios, pâna când voi fi cu voi? Pâna când va voi rabda pe voi? Aduce?i-l la Mine.
Mar 9:20  ?i l-au adus la El. ?i vazându-L pe Iisus, duhul îndata a zguduit pe copil, ?i, cazând la pamânt, se zvârcolea spumegând.
Mar 9:21  ?i l-a întrebat pe tatal lui: Câta vreme este de când i-a venit aceasta? Iar el a raspuns: din pruncie.
Mar 9:22  ?i de multe ori l-a aruncat ?i în foc ?i în apa ca sa-l piarda. Dar de po?i ceva, ajuta-ne, fiindu-?i mila de noi.
Mar 9:23  Iar Iisus i-a zis: De po?i crede, toate sunt cu putin?a celui ce crede.
Mar 9:24  ?i îndata strigând tatal copilului, a zis cu lacrimi: Cred, Doamne! Ajuta necredin?ei mele.
Mar 9:25  Iar Iisus, vazând ca mul?imea da navala, a certat duhul cel necurat, zicându-i: Duh mut ?i surd, Eu î?i poruncesc: Ie?i din el ?i sa nu mai intri în el!
Mar 9:26  ?i racnind ?i zguduindu-l cu putere, duhul a ie?it; iar copilul a ramas ca mort, încât mul?i ziceau ca a murit.
Mar 9:27  Dar Iisus, apucându-l de mâna, l-a ridicat, ?i el s-a sculat în picioare.
Mar 9:28  Iar dupa ce a intrat în casa, ucenicii Lui L-au întrebat, de o parte: Pentru ce noi n-am putut sa-l izgonim?
Mar 9:29  El le-a zis: Acest neam de demoni cu nimic nu poate ie?i, decât numai cu rugaciune ?i cu post.
Mar 9:30  ?i, ie?ind ei de acolo, strabateau Galileea, dar El nu voia sa ?tie cineva.
Mar 9:31  Caci înva?a pe ucenicii Sai ?i le spunea ca Fiul Omului se va da în mâinile oamenilor ?i-L vor ucide, iar dupa ce-L vor ucide, a treia zi va învia.
Mar 9:32  Ei însa nu în?elegeau cuvântul ?i se temeau sa-L întrebe.
Mar 9:33  ?i au venit în Capernaum. ?i fiind în casa, i-a întrebat: Ce vorbea?i între voi pe drum?
Mar 9:34  Iar ei taceau, fiindca pe cale se întrebasera unii pe al?ii cine dintre ei este mai mare.
Mar 9:35  ?i ?ezând jos, a chemat pe cei doisprezece ?i le-a zis: Daca cineva vrea sa fie întâiul, sa fie cel din urma dintre to?i ?i slujitor al tuturor.
Mar 9:36  ?i luând un copil, l-a pus în mijlocul lor ?i, luându-l în bra?e, le-a zis:
Mar 9:37  Oricine va primi, în numele Meu, pe unul din ace?ti copii pe Mine Ma prime?te; ?i oricine Ma prime?te, nu pe Mine Ma prime?te, ci pe Cel ce M-a trimis pe Mine.
Mar 9:38  ?i I-a zis Ioan: Înva?atorule, am vazut pe cineva sco?ând demoni în numele Tau, care nu merge dupa noi, ?i l-am oprit, pentru ca nu merge dupa noi.
Mar 9:39  Iar Iisus a zis: Nu-l opri?i, caci nu e nimeni care, facând vreo minune în numele Meu, sa poata, degraba, sa Ma vorbeasca de rau.
Mar 9:40  Caci cine nu este împotriva noastra este pentru noi.
Mar 9:41  Iar oricine va va da sa be?i un pahar de apa, în numele Meu, fiindca sunte?i ai lui Hristos, adevarat zic voua ca nu-?i va pierde plata sa.
Mar 9:42  ?i cine va sminti pe unul din ace?tia mici, care cred în Mine, mai bine i-ar fi lui daca ?i-ar lega de gât o piatra de moara ?i sa fie aruncat în mare.
Mar 9:43  ?i de te sminte?te mâna ta, tai-o ca mai bine î?i este sa intri ciung în via?a, decât, amândoua mâinile având, sa te duci în gheena, în focul cel nestins.
Mar 9:44  Unde viermele lor nu moare ?i focul nu se stinge.
Mar 9:45  ?i de te sminte?te piciorul tau, taie-l, ca mai bine î?i este ?ie sa intri fara un picior în via?a, decât având amândoua picioarele sa fii azvârlit în gheena, în focul cel nestins,
Mar 9:46  Unde viermele lor nu moare ?i focul nu se stinge.
Mar 9:47  ?i de te sminte?te ochiul tau, scoate-l, ca mai bine î?i este ?ie cu un singur ochi în împara?ia lui Dumnezeu, decât, având amândoi ochii, sa fii aruncat în gheena focului.
Mar 9:48  Unde viermele lor nu moare ?i focul nu se stinge.
Mar 9:49  Caci fiecare (om) va fi sarat cu foc, dupa cum orice jertfa va fi sarata cu sare.
Mar 9:50  Buna este sarea; daca însa sarea î?i pierde puterea, cu ce o ve?i drege? Ave?i sare întru voi ?i trai?i în pace unii cu al?ii.
Mar 10:1  ?i sculându-Se de acolo, a venit în hotarele Iudeii, de cealalta parte a Iordanului, ?i mul?imile s-au adunat iara?i la El ?i iara?i le înva?a, dupa cum obi?nuia.
Mar 10:2  ?i apropiindu-se fariseii, Îl întrebau, ispitindu-L, daca este îngaduit unui barbat sa-?i lase femeia.
Mar 10:3  Iar El, raspunzând, le-a zis: Ce v-a poruncit voua Moise?
Mar 10:4  Iar ei au zis: Moise a dat voie sa-i scrie carte de despar?ire ?i sa o lase.
Mar 10:5  ?i raspunzând, Iisus le-a zis: Pentru învârto?area inimii voastre, v-a scris porunca aceasta;
Mar 10:6  Dar de la începutul fapturii, barbat ?i femeie i-a facut Dumnezeu.
Mar 10:7  De aceea va lasa omul pe tatal sau ?i pe mama sa ?i se va lipi de femeia sa.
Mar 10:8  ?i vor fi amândoi un trup; a?a ca nu mai sunt doi, ci un trup.
Mar 10:9  Deci ceea ce a împreunat Dumnezeu, omul sa nu mai desparta.
Mar 10:10  Dar în casa ucenicii L-au întrebat iara?i despre aceasta.
Mar 10:11  ?i El le-a zis: Oricine va lasa pe femeia sa ?i va lua alta, savâr?e?te adulter cu ea.
Mar 10:12  Iar femeia, de-?i va lasa barbatul ei ?i se va marita cu altul, savâr?e?te adulter.
Mar 10:13  ?i aduceau la El copii, ca sa-?i puna mâinile peste ei, dar ucenicii certau pe cei ce-i aduceau.
Mar 10:14  Iar Iisus, vazând, S-a mâhnit ?i le-a zis: Lasa?i copiii sa vina la Mine ?i nu-i opri?i, caci a unora ca ace?tia este împara?ia lui Dumnezeu.
Mar 10:15  Adevarat zic voua: Cine nu va primi împara?ia lui Dumnezeu ca un copil nu va intra în ea.
Mar 10:16  ?i, luându-i în bra?e, i-a binecuvântat, punându-?i mâinile peste ei.
Mar 10:17  ?i când ie?ea El în drum, alergând la El unul ?i îngenunchind înaintea Lui, Îl întreba: Înva?atorule bun, ce sa fac ca sa mo?tenesc via?a ve?nica?
Mar 10:18  Iar Iisus i-a raspuns: De ce-Mi zici bun? Nimeni nu este bun decât unul Dumnezeu.
Mar 10:19  ?tii poruncile: Sa nu ucizi, sa nu savâr?e?ti adulter, sa nu furi, sa nu marturise?ti strâmb, sa nu în?eli pe nimeni, cinste?te pe tatal tau ?i pe mama ta.
Mar 10:20  Iar el I-a zis: Înva?atorule, acestea toate le-am pazit din tinere?ile mele.
Mar 10:21  Iar Iisus, privind la el cu dragoste, i-a zis: Un lucru î?i mai lipse?te: Mergi, vinde tot ce ai, da saracilor ?i vei avea comoara în cer; ?i apoi, luând crucea, vino ?i urmeaza Mie.
Mar 10:22  Dar el, întristându-se de cuvântul acesta, a plecat mâhnit, caci avea multe boga?ii.
Mar 10:23  ?i Iisus, uitându-Se în jur, a zis catre ucenicii Sai: Cât de greu vor intra boga?ii în împara?ia lui Dumnezeu!
Mar 10:24  Iar ucenicii erau uimi?i de cuvintele Lui. Dar Iisus, raspunzând iara?i, le-a zis: Fiilor, cât  de greu este celor ce se încred în boga?ii sa intre în împara?ia lui Dumnezeu!
Mar 10:25  Mai lesne este camilei sa treaca prin urechile acului, decât bogatului sa intre în împara?ia lui Dumnezeu.
Mar 10:26  Iar ei, mai mult uimindu-se, ziceau unii catre al?ii: ?i cine poate sa se mântuiasca?
Mar 10:27  Iisus, privind la ei, le-a zis: La oameni lucrul e cu neputin?a, dar nu la Dumnezeu. Caci la Dumnezeu toate sunt cu putin?a.
Mar 10:28  ?i a început Petru a-I zice: Iata, noi am lasat toate ?i ?i-am urmat.
Mar 10:29  Iisus i-a raspuns: Adevarat graiesc voua: Nu este nimeni care ?i-a lasat casa, sau fra?i, sau surori, sau mama, sau tata, sau copii, sau ?arine pentru Mine ?i pentru Evanghelie,
Mar 10:30  ?i sa nu ia însutit - acum, în vremea aceasta, de prigoniri - case ?i fra?i ?i surori ?i mame ?i copii ?i ?arine, iar în veacul ce va sa vina: via?a ve?nica.
Mar 10:31  ?i mul?i dintre cei dintâi vor fi pe urma, ?i din cei de pe urma întâi.
Mar 10:32  ?i erau pe drum, suindu-se la Ierusalim, iar Iisus mergea înaintea lor. ?i ei erau uimi?i ?i cei ce mergeau dupa El se temeau. ?i luând la Sine, iara?i, pe cei doisprezece, a început sa le spuna ce aveau sa I se întâmple:
Mar 10:33  Ca, iata, ne suim la Ierusalim ?i Fiul Omului va fi predat arhiereilor ?i carturarilor; ?i-L vor osândi la moarte ?i-L vor da în mâna pagânilor.
Mar 10:34  ?i-L vor batjocori ?i-L vor scuipa ?i-L vor biciui ?i-L vor omorî, dar dupa trei zile va învia.
Mar 10:35  ?i au venit la El Iacov ?i Ioan, fiii lui Zevedeu, zicându-I: Înva?atorule, voim sa ne faci ceea ce vom cere de la Tine.
Mar 10:36  Iar El le-a zis: Ce voi?i sa va fac?
Mar 10:37  Iar ei I-au zis: Da-ne noua sa ?edem unul de-a dreapta Ta, ?i altul de-a stânga Ta, întru slava Ta.
Mar 10:38  Dar Iisus le-a raspuns: Nu ?ti?i ce cere?i! Pute?i sa be?i paharul pe care îl beau Eu sau sa va boteza?i cu botezul cu care Ma botez Eu?
Mar 10:39  Iar ei I-au zis: Putem. ?i Iisus le-a zis: Paharul pe care Eu îl beau îl ve?i bea, ?i cu botezul cu care Eu ma botez va ve?i boteza.
Mar 10:40  Dar a ?edea de-a dreapta Mea, sau de-a stânga Mea, nu este al Meu a da, ci celor pentru care s-a pregatit.
Mar 10:41  ?i auzind cei zece, au început a se mânia pe Iacov ?i pe Ioan.
Mar 10:42  ?i Iisus, chemându-i la Sine, le-a zis: ?ti?i ca cei ce se socotesc cârmuitori ai neamurilor domnesc peste ele ?i cei mai mari ai lor le stapânesc.
Mar 10:43  Dar între voi nu trebuie sa fie a?a, ci care va vrea sa fie mare între voi, sa fie slujitor al vostru.
Mar 10:44  ?i care va vrea sa fie întâi între voi, sa fie tuturor sluga.
Mar 10:45  Ca ?i Fiul Omului n-a venit ca sa I se slujeasca, ci ca El sa slujeasca ?i sa-?i dea sufletul rascumparare pentru mul?i.
Mar 10:46  ?i au venit în Ierihon. ?i ie?ind din Ierihon El, ucenicii Lui ?i mul?ime mare, Bartimeu orbul, fiul lui Timeu, ?edea jos, pe marginea drumului.
Mar 10:47  ?i, auzind ca este Iisus Nazarineanul, a început sa strige ?i sa zica: Iisuse, Fiul lui David, miluie?te-ma!
Mar 10:48  ?i mul?i îl certau ca sa taca, el însa cu mult mai tare striga: Fiule al lui David, miluie?te-ma!
Mar 10:49  ?i Iisus, oprindu-Se, a zis: Chema?i-l! ?i l-au chemat pe orb, zicându-i: Îndrazne?te, scoala-te! Te cheama.
Mar 10:50  Iar orbul, lepadând haina de pe el, a sarit în picioare ?i a venit la Iisus.
Mar 10:51  ?i l-a întrebat Iisus, zicându-i: Ce voie?ti sa-?i fac? Iar orbul I-a raspuns: Înva?atorule, sa vad iara?i.
Mar 10:52  Iar Iisus i-a zis: Mergi, credin?a ta te-a mântuit. ?i îndata a vazut ?i urma lui Iisus pe cale.
Mar 11:1  ?i când s-au apropiat de Ierusalim, la Betfaghe ?i la Betania, lânga Muntele Maslinilor, a trimis pe doi dintre ucenicii Sai,
Mar 11:2  ?i le-a zis: Merge?i în satul care este înaintea voastra ?i, intrând în el, îndata ve?i afla un mânz legat, pe care n-a ?ezut pâna acum nici un om. Dezlega?i-l ?i aduce?i-l.
Mar 11:3  Iar de va va zice cineva: De ce face?i aceasta? Spune?i ca Domnul are trebuin?a de el ?i îndata îl va trimite aici.
Mar 11:4  Deci au mers ?i au gasit mânzul legat la o poarta, afara la raspântie, ?i l-au dezlegat.
Mar 11:5  ?i unii din cei ce stateau acolo, le-au zis: De ce dezlega?i mânzul?
Mar 11:6  Iar ei le-au spus precum le zisese Iisus, ?i i-au lasat.
Mar 11:7  ?i au adus mânzul la Iisus ?i ?i-au pus hainele pe el ?i Iisus a ?ezut pe el.
Mar 11:8  ?i mul?i î?i a?terneau hainele pe cale, iar al?ii a?terneau ramuri, pe care le taiau de prin gradini.
Mar 11:9  Iar cei ce mergeau înainte ?i cei ce veneau pe urma strigau, zicând: Osana! Bine este cuvântat Cel ce vine întru numele Domnului!
Mar 11:10  Binecuvântata este împara?ia ce vine a parintelui nostru David! Osana întru cei de sus!
Mar 11:11  ?i a intrat Iisus în Ierusalim ?i în templu ?i, privind toate în jur ?i vremea fiind spre seara, a ie?it spre Betania cu cei doisprezece.
Mar 11:12  ?i a doua zi, ie?ind ei din Betania, El a flamânzit.
Mar 11:13  ?i vazând de departe un smochin care avea frunze, a mers acolo, doar va gasi ceva în el; ?i, ajungând la smochin, n-a gasit nimic decât frunze. Caci nu era timpul smochinelor.
Mar 11:14  ?i, vorbind, i-a zis: De acum înainte, rod din tine nimeni în veac sa nu manânce. ?i ucenicii Lui ascultau.
Mar 11:15  ?i au venit în Ierusalim. ?i, intrând în templu, a început sa dea afara pe cei ce vindeau ?i pe cei ce cumparau în templu, iar mesele schimbatorilor de bani ?i scaunele vânzatorilor de porumbei le-a rasturnat.
Mar 11:16  ?i nu îngaduia sa mai treaca nimeni cu vreun vas prin templu.
Mar 11:17  ?i-i înva?a ?i le spunea: Nu este, oare, scris: "Casa Mea casa de rugaciune se va chema, pentru toate neamurile"? Voi însa a?i facut din ea pe?tera de tâlhari.
Mar 11:18  ?i au auzit arhiereii ?i carturarii. ?i cautau cum sa-L piarda. Caci se temeau de El, pentru ca toata mul?imea era uimita de înva?atura Lui.
Mar 11:19  Iar când s-a facut seara, au ie?it afara din cetate.
Mar 11:20  Diminea?a, trecând pe acolo, au vazut smochinul uscat din radacini.
Mar 11:21  ?i Petru, aducându-?i aminte, I-a zis: Înva?atorule, iata smochinul pe care l-ai blestemat s-a uscat.
Mar 11:22  ?i raspunzând, Iisus le-a zis: Ave?i credin?a în Dumnezeu.
Mar 11:23  Adevarat zic voua ca oricine va zice acestui munte: Ridica-te ?i te arunca în mare, ?i nu se va îndoi în inima lui, ci va crede ca ceea ce spune se va face, fi-va lui orice va zice.
Mar 11:24  De aceea va zic voua: Toate câte cere?i, rugându-va, sa crede?i ca le-a?i primit ?i le ve?i avea.
Mar 11:25  Iar când sta?i de va ruga?i, ierta?i orice ave?i împotriva cuiva, ca ?i Tatal vostru Cel din ceruri sa va ierte voua gre?ealele voastre.
Mar 11:26  Ca de nu ierta?i voi, nici Tatal vostru Cel din ceruri nu va va ierta voua gre?ealele voastre.
Mar 11:27  ?i au intrat iara?i în Ierusalim. ?i pe când se plimba Iisus prin templu, au venit la El arhiereii, carturarii ?i batrânii.
Mar 11:28  ?i I-au zis: Cu ce putere faci acestea? Sau cine ?i-a dat ?ie puterea aceasta, ca sa le faci?
Mar 11:29  Iar Iisus le-a zis: Va voi întreba ?i Eu un cuvânt: raspunde?i-Mi ?i va voi spune ?i Eu cu ce putere fac acestea:
Mar 11:30  Botezul lui Ioan din cer a fost, sau de la oameni? Raspunde?i-Mi!
Mar 11:31  ?i ei vorbeau între ei, zicând: De vom zice: Din cer, va zice: Pentru ce, dar, n-a?i crezut în el?
Mar 11:32  Iar de vom zice: De la oameni - se temeau de mul?ime, caci to?i îl socoteau ca Ioan era într-adevar prooroc.
Mar 11:33  ?i raspunzând, au zis lui Iisus: Nu ?tim. ?i Iisus le-a zis: Nici Eu nu va spun voua cu ce putere fac acestea.
Mar 12:1  ?i a început sa le vorbeasca în pilde: Un om a sadit o vie, a împrejmuit-o cu gard, a sapat în ea teasc, a cladit turn ?i a dat-o lucratorilor, iar el s-a dus departe.
Mar 12:2  ?i la vreme, a trimis la lucratori o sluga, ca sa ia de la ei din roadele viei.
Mar 12:3  Dar ei, punând mâna pe ea, au batut-o ?i i-au dat drumul fara nimic.
Mar 12:4  ?i a trimis la ei, iara?i, alta sluga, dar ?i pe aceea, lovind-o cu pietre, i-au spart capul ?i au ocarât-o.
Mar 12:5  ?i a trimis alta. Dar ?i pe aceea au ucis-o; ?i pe multe altele: pe unele batându-le, iar pe altele ucigându-le.
Mar 12:6  Mai avea ?i un fiu iubit al sau ?i în cele din urma l-a trimis la lucratori, zicând: Se vor ru?ina de fiul meu.
Mar 12:7  Dar acei lucratori au zis între ei: Acesta este mo?tenitorul; veni?i sa-l omorâm ?i mo?tenirea va fi a noastra.
Mar 12:8  ?i prinzându-l l-au omorât ?i l-au aruncat afara din vie.
Mar 12:9  Ce va face acum stapânul viei? Va veni ?i va pierde pe lucratori, iar via o va da altora.
Mar 12:10  Oare nici Scriptura aceasta n-a?i citit-o: "Piatra pe care au nesocotit-o ziditorii, aceasta a ajuns sa fie în capul unghiului?
Mar 12:11  De la Domnul s-a facut aceasta ?i este lucru minunat în ochii no?tri".
Mar 12:12  ?i cautau sa-L prinda, dar se temeau de popor. Caci în?elesesera ca împotriva lor zisese pilda aceasta. ?i lasându-L, s-au dus.
Mar 12:13  ?i au trimis la El pe unii din farisei ?i din irodiani, ca sa-L prinda în cuvânt.
Mar 12:14  Iar ei, venind, I-au zis: Înva?atorule, ?tim ca spui adevarul ?i nu-?i pasa de nimeni, fiindca nu cau?i la fa?a oamenilor, ci cu adevarat înve?i calea lui Dumnezeu. Se cuvine a da dajdie Cezarului sau nu? Sa dam sau sa nu dam?
Mar 12:15  El însa, cunoscând fa?arnicia lor, le-a zis: Pentru ce Ma ispiti?i? Aduce?i-Mi un dinar ca sa-l vad.
Mar 12:16  ?i I-au adus. ?i i-a întrebat Iisus: Al cui e chipul acesta în inscrip?ia de pe el? Iar ei I-au zis: Ale Cezarului.
Mar 12:17  Iar Iisus a zis: Da?i Cezarului cele ale Cezarului, iar lui Dumnezeu cele ale lui Dumnezeu. ?i se mirau de El.
Mar 12:18  ?i au venit la El saducheii care zic ca nu este înviere ?i-L întrebau zicând:
Mar 12:19  Înva?atorule, Moise ne-a lasat scris, ca de va muri fratele cuiva ?i va lasa femeia fara copil, sa ia fratele sau pe femeia lui ?i sa ridice urma? fratelui.
Mar 12:20  ?i erau ?apte fra?i. ?i cel dintâi ?i-a luat femeie, dar, murind, n-a lasat urma?.
Mar 12:21  ?i a luat-o pe ea al doilea, ?i a murit, nelasând urma?. Tot a?a ?i al treilea.
Mar 12:22  ?i au luat-o to?i ?apte ?i n-au lasat urma?. În urma tuturor a murit ?i femeia.
Mar 12:23  La înviere, când vor învia, a caruia dintre ei va fi femeia? Caci to?i ?apte au avut-o de so?ie.
Mar 12:24  ?i le-a zis Iisus: Oare nu pentru aceasta rataci?i, ne?tiind Scripturile, nici puterea lui Dumnezeu?
Mar 12:25  Caci, când vor învia din mor?i, nici nu se mai însoara, nici nu se mai marita, ci sunt ca îngerii din ceruri.
Mar 12:26  Iar despre mor?i ca vor învia, n-a?i citit, oare, în cartea lui Moise, când i-a vorbit Dumnezeu din rug, zicând: "Eu sunt Dumnezeul lui Avraam ?i Dumnezeul lui Isaac ?i Dumnezeul lui Iacov"?
Mar 12:27  Dumnezeu nu este Dumnezeul celor mor?i, ci a celor vii. Mult rataci?i.
Mar 12:28  ?i apropiindu-se unul din carturari, care îi auzise vorbind între ei ?i, vazând ca bine le-a raspuns, L-a întrebat: Care porunca este întâia dintre toate?
Mar 12:29  Iisus i-a raspuns ca întâia este: "Asculta Israele, Domnul Dumnezeul nostru este singurul Domn".
Mar 12:30  ?i: "Sa iube?ti pe Domnul Dumnezeul tau din toata inima ta, din tot sufletul tau, din tot cugetul tau ?i din toata puterea ta". Aceasta este cea dintâi porunca.
Mar 12:31  Iar a doua e aceasta: "Sa iube?ti pe aproapele tau ca pe tine însu?i". Mai mare decât acestea nu este alta porunca.
Mar 12:32  ?i I-a zis carturarul: Bine, Înva?atorule. Adevarat ai zis ca unul este Dumnezeu ?i nu este altul afara de El.
Mar 12:33  ?i a-L iubi pe El din toata inima, din tot sufletul, din tot cugetul ?i din toata puterea ?i a iubi pe aproapele tau ca pe tine însu?i este mai mult decât toate arderile de tot ?i decât toate jertfele.
Mar 12:34  Iar Iisus, vazându-l ca a raspuns cu în?elepciune, i-a zis: Nu e?ti departe de împara?ia lui Dumnezeu. ?i nimeni nu mai îndraznea sa-L mai întrebe.
Mar 12:35  ?i înva?ând Iisus în templu, graia zicând: Cum zic carturarii ca Hristos este Fiul lui David?
Mar 12:36  Însu?i David a zis întru Duhul Sfânt: "Zis-a Domnul Domnului meu: ?ezi de-a dreapta Mea pâna ce voi pune pe vrajma?ii tai a?ternut picioarelor Tale".
Mar 12:37  Deci însu?i David Îl nume?te pe El Domn; de unde dar este fiul lui? ?i mul?imea cea multa Îl asculta cu bucurie.
Mar 12:38  ?i le zicea în înva?atura Sa: Lua?i seama la carturari carora le place sa se plimbe în haine lungi ?i sa li se plece lumea în pie?e,
Mar 12:39  ?i sa stea în bancile dintâi în sinagogi ?i sa stea în capul mesei la ospe?e,
Mar 12:40  Ei, care secatuiesc casele vaduvelor ?i de ochii lumii se roaga îndelung, î?i vor lua mai multa osânda.
Mar 12:41  ?i ?ezând în preajma cutiei darurilor, Iisus privea cum mul?imea arunca bani în cutie. ?i mul?i boga?i aruncau mult.
Mar 12:42  ?i venind o vaduva saraca, a aruncat doi bani, adica un codrant.
Mar 12:43  ?i chemând la Sine pe ucenicii Sai le-a zis: Adevarat graiesc voua ca aceasta vaduva saraca a aruncat în cutia darurilor mai mult decât to?i ceilal?i.
Mar 12:44  Pentru ca to?i au aruncat din prisosul lor, pe când ea, din saracia ei, a aruncat tot ce avea, toata avu?ia sa.
Mar 13:1  ?i ie?ind din templu, unul dintre ucenicii Sai I-a zis: Înva?atorule, prive?te ce fel de pietre ?i ce cladiri!
Mar 13:2  Dar Iisus a zis: Vezi aceste mari cladiri? Nu va ramâne piatra peste piatra sa nu se risipeasca.
Mar 13:3  ?i ?ezând pe Muntele Maslinilor, în fa?a templului, Îl întrebau, de o parte, Petru, Iacov, Ioan ?i cu Andrei:
Mar 13:4  Spune-ne noua când vor fi acestea? ?i care va fi semnul când va fi sa se împlineasca toate acestea?
Mar 13:5  Iar Iisus a început sa le spuna: Vede?i sa nu va în?ele cineva.
Mar 13:6  Caci mul?i vor veni în numele Meu, zicând ca sunt Eu, ?i vor amagi pe mul?i.
Mar 13:7  Iar când ve?i auzi de razboaie, ?i de zvonuri de razboaie, sa nu va tulbura?i, caci trebuie sa fie, dar înca nu va fi sfâr?itul.
Mar 13:8  ?i se va ridica neam peste neam ?i împara?ie peste împara?ie, vor fi cutremure pe alocuri ?i foamete ?i tulburari vor fi. Iar acestea sunt începutul durerilor.
Mar 13:9  Lua?i seama la voi în?iva. Ca va vor da în adunari ?i ve?i fi batu?i în sinagogi ?i ve?i sta înaintea conducatorilor ?i a regilor, pentru Mine, spre marturie lor.
Mar 13:10  Ci mai întâi Evanghelia trebuie sa se propovaduiasca la toate neamurile.
Mar 13:11  Iar când va vor duce ca sa va predea, nu va îngriji?i dinainte ce ve?i vorbi, ci sa vorbi?i ceea ce se va da voua în ceasul acela. Caci nu voi sunte?i cei care ve?i vorbi, ci Duhul Sfânt.
Mar 13:12  ?i va da frate pe frate la moarte ?i tata pe copil ?i copiii se vor razvrati împotriva parin?ilor ?i îi vor ucide.
Mar 13:13  ?i ve?i fi urâ?i de to?i pentru numele Meu; iar cel ce va rabda pâna la urma, acela se va mântui.
Mar 13:14  Iar când ve?i vedea urâciunea pustiirii, stând unde nu se cuvine - cine cite?te sa în?eleaga - atunci cei ce vor fi în Iudeea sa fuga în mun?i,
Mar 13:15  ?i cel de pe acoperi? sa nu se coboare în casa, nici sa intre ca sa-?i ia ceva din casa sa,
Mar 13:16  ?i cel ce va fi în ?arina sa nu se întoarca îndarat, ca sa-?i ia haina.
Mar 13:17  Dar vai celor ce vor avea în pântece ?i celor ce vor alapta în zilele acelea!
Mar 13:18  Ruga?i-va, dar, ca sa nu fie fuga voastra iarna.
Mar 13:19  Caci în zilele acelea va fi necaz cum nu a mai fost pâna acum, de la începutul fapturii, pe care a zidit-o Dumnezeu, ?i nici nu va mai fi.
Mar 13:20  ?i de nu ar fi scurtat Domnul zilele acelea, n-ar scapa nici un trup, dar pentru cei ale?i, pe care i-a ales, a scurtat acele zile.
Mar 13:21  ?i atunci daca va va zice cineva: Iata, aci este Hristos, sau iata acolo, sa nu crede?i.
Mar 13:22  Se vor scula hristo?i mincino?i ?i prooroci mincino?i ?i vor face semne ?i minuni, ca sa duca în ratacire, de se poate, pe cei ale?i.
Mar 13:23  Dar voi lua?i seama. Iata dinainte v-am spus voua toate.
Mar 13:24  Ci în acele zile, dupa necazul acela, soarele se va întuneca ?i luna nu-?i va mai da lumina ei.
Mar 13:25  ?i stelele vor cadea din cer ?i puterile care sunt în ceruri se vor clatina.
Mar 13:26  Atunci vor vedea pe Fiul Omului venind pe nori, cu putere multa ?i cu slava.
Mar 13:27  ?i atunci El va trimite pe îngeri ?i va aduna pe ale?ii Sai din cele patru vânturi, de la marginea pamântului pâna la marginea cerului.
Mar 13:28  Înva?a?i de la smochin pilda: Când mladi?a lui se face frageda ?i înfrunze?te, cunoa?te?i ca vara este aproape.
Mar 13:29  Tot a?a ?i voi, când ve?i vedea împlinindu-se aceste lucruri, sa ?ti?i ca El este aproape, lânga u?i.
Mar 13:30  Adevarat graiesc voua ca nu va trece neamul acesta pâna ce nu vor fi toate acestea.
Mar 13:31  Cerul ?i pamântul vor trece, dar cuvintele Mele nu vor trece.
Mar 13:32  Iar despre ziua aceea ?i despre ceasul acela nimeni nu ?tie, nici îngerii din cer, nici Fiul, ci numai Tatal.
Mar 13:33  Lua?i aminte, priveghea?i ?i va ruga?i, ca nu ?ti?i când va fi acea vreme.
Mar 13:34  Este ca un om care a plecat în alta ?ara ?i, lasându-?i casa, a dat puterea în mâna slugilor, dând fiecaruia lucrul lui, iar portarului i-a poruncit sa vegheze.
Mar 13:35  Veghea?i, dar, ca nu ?ti?i când va veni stapânul casei: sau seara, sau la miezul nop?ii, sau la cântatul coco?ilor, sau diminea?a.
Mar 13:36  Ca nu cumva venind fara veste, sa va afle pe voi dormind.
Mar 13:37  Iar ceea ce zic voua, zic tuturor: Priveghea?i!
Mar 14:1  ?i dupa doua zile erau Pa?tile ?i Azimile. ?i arhiereii ?i carturarii cautau cum sa-l prinda cu vicle?ug, ca sa-L omoare.
Mar 14:2  Dar ziceau: Nu la sarbatoare, ca sa nu fie tulburare în popor.
Mar 14:3  ?i fiind El în Betania, în casa lui Simon Leprosul, ?i ?ezând la masa, a venit o femeie având un alabastru, cu mir de nard curat, de mare pre?, ?i, spargând vasul, a varsat mirul pe capul lui Iisus.
Mar 14:4  Dar erau unii mâhni?i între ei, zicând: Pentru ce s-a facut aceasta risipa de mir?
Mar 14:5  Caci putea sa se vânda acest mir cu peste trei sute de dinari, ?i sa se dea saracilor. ?i cârteau împotriva ei.
Mar 14:6  Dar Iisus a zis: Lasa?i-o. De ce îi face?i suparare? Lucru bun a facut ea cu Mine.
Mar 14:7  Ca pe saraci totdeauna îi ave?i cu voi ?i, oricând voi?i, pute?i sa le face?i bine, dar pe mine nu Ma ave?i totdeauna.
Mar 14:8  Ea a facut ceea ce avea de facut: mai dinainte a uns trupul Meu, spre înmormântare.
Mar 14:9  Adevarat zic voua: Oriunde se va propovadui Evanghelia, în toata lumea, se va spune ?i ce-a facut aceasta, spre pomenirea ei.
Mar 14:10  Iar Iuda Iscarioteanul, unul din cei doisprezece, s-a dus la arhierei ca sa li-L dea pe Iisus.
Mar 14:11  ?i, auzind ei, s-au bucurat ?i au fagaduit sa-i dea bani. ?i el cauta cum sa-L dea lor, la timp potrivit.
Mar 14:12  Iar în ziua cea dintâi a Azimilor, când jertfeau Pa?tile, ucenicii Lui L-au întrebat: Unde voie?ti sa gatim, ca sa manânci Pa?tile?
Mar 14:13  ?i a trimis doi din ucenicii Lui, zicându-le: Merge?i în cetate ?i va va întâmpina un om, ducând un urcior cu apa; merge?i dupa el.
Mar 14:14  ?i unde va intra, spune?i stapânului casei ca Înva?atorul zice: Unde este odaia în care sa manânc Pa?tile împreuna cu ucenicii Mei?
Mar 14:15  Iar el va va arata un foi?or mare a?ternut gata. Acolo sa pregati?i pentru noi.
Mar 14:16  ?i au ie?it ucenicii ?i au venit în cetate ?i au gasit a?a precum le-a spus ?i au pregatit Pa?tile.
Mar 14:17  Iar facându-se seara, a venit cu cei doisprezece.
Mar 14:18  Pe când ?edeau la masa ?i mâncau, Iisus a zis: Adevarat graiesc voua ca unul dintre voi, care manânca împreuna cu Mine, Ma va vinde.
Mar 14:19  Ei au început sa se întristeze ?i sa-I zica, unul câte unul: Nu cumva sunt eu?
Mar 14:20  Iar El le-a zis: Unul dintre cei doisprezece, care întinge cu Mine în blid.
Mar 14:21  Ca Fiul Omului merge precum este scris despre El; dar vai de omul acela prin care este vândut Fiul Omului. Bine era de omul acela daca nu s-ar fi nascut.
Mar 14:22  ?i, mâncând ei, a luat Iisus pâine ?i binecuvântând, a frânt ?i le-a dat lor ?i a zis: Lua?i, mânca?i, acesta este Trupul Meu.
Mar 14:23  ?i luând paharul, mul?umind, le-a dat ?i au baut din el to?i.
Mar 14:24  ?i a zis lor: Acesta este Sângele Meu, al Legii celei noi, care pentru mul?i se varsa.
Mar 14:25  Adevarat graiesc voua ca de acum nu voi mai bea din rodul vi?ei pâna în ziua aceea când îl voi bea nou în împara?ia lui Dumnezeu.
Mar 14:26  ?i dupa ce au cântat cântari de lauda, au ie?it la Muntele Maslinilor.
Mar 14:27  ?i le-a zis Iisus: To?i va ve?i sminti, ca scris este: "Bate-voi pastorul ?i se vor risipi oile".
Mar 14:28  Dar dupa învierea Mea, voi merge mai înainte de voi în Galileea.
Mar 14:29  Iar Petru I-a zis: Chiar daca to?i se vor sminti întru Tine, totu?i eu nu.
Mar 14:30  ?i i-a zis Iisus: Adevarat graiesc ?ie: Ca tu astazi, în noaptea aceasta, mai înainte de a cânta de doua ori coco?ul, de trei ori te vei lepada de Mine.
Mar 14:31  El însa spunea mai staruitor: ?i de-ar fi sa mor cu Tine, nu Te voi tagadui. ?i tot a?a ziceau to?i.
Mar 14:32  ?i au venit la un loc al carui nume este Ghetsimani, ?i acolo a zis catre ucenicii Sai: ?ede?i aici pâna ce Ma voi ruga.
Mar 14:33  ?i a luat cu El pe Petru ?i pe Iacov ?i pe Ioan ?i a început a Se tulbura ?i a Se mâhni.
Mar 14:34  ?i le-a zis lor: Întristat este sufletul Meu pâna la moarte. Ramâne?i aici ?i privegehea?i.
Mar 14:35  ?i mergând pu?in mai înainte, a cazut cu fa?a la pamânt ?i Se ruga, ca, de este cu putin?a, sa treaca de la El ceasul (acesta).
Mar 14:36  ?i zicea: Avva Parinte, toate sunt ?ie cu putin?a. Departeaza paharul acesta de la Mine. Dar nu ce voiesc Eu, ci ceea ce voie?ti Tu.
Mar 14:37  ?i a venit ?i i-a gasit dormind ?i a zis lui Petru: Simone, dormi? N-ai avut tarie ca sa veghezi un ceas?
Mar 14:38  Priveghea?i ?i va ruga?i, ca sa nu intra?i în ispita. Caci duhul este osârduitor, dar trupul neputincios.
Mar 14:39  ?i iara?i mergând, s-a rugat, acela?i cuvânt zicând.
Mar 14:40  ?i iara?i venind, i-a gasit dormind, caci ochii lor erau îngreuia?i ?i nu ?tiau ce sa-I raspunda.
Mar 14:41  ?i a venit a treia oara ?i le-a zis: Dormi?i de acum ?i va odihni?i! E gata! A sosit ceasul. Iata Fiul Omului este dat în mâinile pacato?ilor.
Mar 14:42  Scula?i-va sa mergem. Iata, cel ce M-a vândut s-a apropiat.
Mar 14:43  ?i îndata, înca vorbind El, a venit Iuda Iscarioteanul, unul din cei doisprezece, ?i cu el mul?ime cu sabii ?i cu ciomege, de la arhierei, de la carturari ?i de la batrâni.
Mar 14:44  Iar vânzatorul le daduse semn, zicând: Pe care-L voi saruta, Acela este. Prinde?i-L ?i duce?i-L cu paza.
Mar 14:45  ?i venind îndata ?i apropiindu-se de El, a zis Lui: Înva?atorule! ?i L-a sarutat.
Mar 14:46  Iar ei au pus mâna pe El ?i L-au prins.
Mar 14:47  Unul din cei ce stateau pe lânga El, sco?ând sabia, a lovit pe sluga arhiereului ?i i-a taiat urechea.
Mar 14:48  ?i raspunzând, Iisus le-a zis: Ca la un tâlhar a?i ie?it cu sabii ?i cu toiege, ca sa Ma prinde?i.
Mar 14:49  În fiecare zi eram la voi în templu, înva?ând, ?i nu M-a?i prins. Dar acestea sunt ca sa se împlineasca Scripturile.
Mar 14:50  ?i, lasându-L, au fugit to?i.
Mar 14:51  Iar un tânar mergea dupa El, înfa?urat într-o pânzatura, pe trupul gol, ?i au pus mâna pe el.
Mar 14:52  El însa, smulgându-se din pânzatura, a fugit gol.
Mar 14:53  ?i au dus pe Iisus la arhiereu ?i s-au adunat acolo to?i arhiereii ?i batrânii ?i carturarii.
Mar 14:54  Iar Petru, de departe, a mers dupa El, pâna a intrat înauntru în curtea arhiereului ?i ?edea împreuna cu slugile, încalzindu-se la foc.
Mar 14:55  Arhiereii ?i tot sinedriul cautau împotriva lui Iisus marturie ca sa-L dea la moarte, dar nu gaseau.
Mar 14:56  Ca mul?i marturiseau mincinos împotriva Lui, dar marturiile nu se potriveau.
Mar 14:57  ?i ridicându-se unii, au dat marturie mincinoasa împotriva Lui, zicând:
Mar 14:58  Noi L-am auzit zicând: Voi darâma acest templu facut de mâna, ?i în trei zile altul, nefacut de mâna, voi cladi.
Mar 14:59  Dar nici a?a marturia lor nu era la fel.
Mar 14:60  ?i, sculându-se în mijlocul lor, arhiereul L-a întrebat pe Iisus, zicând: Nu raspunzi nimic la tot ce marturisesc împotriva Ta ace?tia?
Mar 14:61  Iar El tacea ?i nu raspundea nimic. Iara?i L-a întrebat arhiereul ?i I-a zis: E?ti tu Hristosul, Fiul Celui binecuvântat?
Mar 14:62  Iar Iisus a zis: Eu sunt ?i ve?i vedea pe Fiul Omului ?ezând de-a dreapta Celui Atotputernic ?i venind pe norii cerului.
Mar 14:63  Iar arhiereul, sfâ?iindu-?i hainele, a zis: Ce trebuin?a mai avem de martori?
Mar 14:64  A?i auzit hula. Ce vi se pare voua? Iar ei to?i au judecat ca El este vinovat de moarte.
Mar 14:65  ?i unii au început sa-L scuipe ?i sa-I acopere fa?a ?i sa-L bata cu pumnii ?i sa-I zica: Prooroce?te! ?i slugile Îl bateau cu palmele.
Mar 14:66  ?i Petru fiind jos în curte, a venit una din slujnicele arhiereului,
Mar 14:67  ?i vazându-l pe Petru, încalzindu-se, s-a uitat la el ?i a zis: ?i tu erai cu Iisus Nazarineanul.
Mar 14:68  El însa a tagaduit, zicând: Nici nu ?tiu, nici nu în?eleg ce zici. ?i a ie?it afara înaintea cur?ii; ?i a cântat coco?ul.
Mar 14:69  Iar slujnica, vazându-l, a început iara?i sa spuna celor de fa?a ca acesta este dintre ei.
Mar 14:70  Iar el a tagaduit iara?i. ?i dupa pu?in timp, cei de fa?a ziceau iara?i lui Petru: Cu adevarat e?ti dintre ei, caci e?ti ?i galileian ?i vorbirea ta se aseamana.
Mar 14:71  Iar el a început sa se blesteme ?i sa se jure: Nu ?tiu pe omul acesta despre care zice?i.
Mar 14:72  ?i îndata coco?ul a cântat a doua oara. ?i Petru ?i-a adus aminte de cuvântul pe care i-l spusese Iisus: Înainte de a cânta de doua ori coco?ul, de trei ori te vei lepada de Mine. ?i a început sa plânga.
Mar 15:1  ?i îndata diminea?a, arhiereii, ?inând sfat cu batrânii, cu carturarii ?i cu tot sinedriul ?i legând pe Iisus, L-au dus ?i L-au predat lui Pilat.
Mar 15:2  ?i L-a întrebat Pilat: Tu e?ti regele iudeilor? Iar El, raspunzând, i-a zis: Tu zici.
Mar 15:3  Iar arhiereii Îl învinuiau de multe.
Mar 15:4  Iar Pilat L-a întrebat: Nu raspunzi nimic? Iata câte spun împotriva Ta.
Mar 15:5  Dar Iisus nimic n-a mai raspuns, încât Pilat se mira.
Mar 15:6  Iar la sarbatoarea Pa?tilor, le elibera un întemni?at pe care-l cereau ei.
Mar 15:7  ?i era unul cu numele Baraba închis împreuna cu ni?te razvrati?i, care în rascoala savâr?isera ucidere.
Mar 15:8  ?i mul?imea, venind sus, a început sa ceara lui Pilat sa le faca precum obi?nuia pentru ei.
Mar 15:9  Iar Pilat le-a raspuns, zicând: Voi?i sa va eliberez pe regele iudeilor?
Mar 15:10  Fiindca ?tia ca arhiereii Îl dadusera în mâna lui din invidie.
Mar 15:11  Dar arhiereii au a?â?at mul?imea ca sa le elibereze mai degraba pe Baraba.
Mar 15:12  Iar Pilat, raspunzând iara?i, le-a zis: Ce voi face deci cu cel despre care zice?i ca este regele iudeilor?
Mar 15:13  Ei iara?i au strigat: Rastigne?te-L!
Mar 15:14  Iar Pilat le-a zis: Dar ce rau a facut? Iar ei mai mult strigau: Rastigne?te-L!
Mar 15:15  ?i Pilat, vrând sa faca pe voia mul?imii, le-a eliberat pe Baraba, iar pe Iisus, biciuindu-L, L-a dat ca sa fie rastignit.
Mar 15:16  Iar osta?ii L-au dus înauntrul cur?ii, adica în pretoriu, ?i au adunat toata cohorta.
Mar 15:17  ?i L-au îmbracat în purpura ?i, împletindu-I o cununa de spini, I-au pus-o pe cap.
Mar 15:18  ?i au început sa se plece în fa?a Lui, zicând: Bucura-Te regele iudeilor!
Mar 15:19  ?i-L bateau peste cap cu o trestie ?i-L scuipau ?i, cazând în genunchi, I se închinau.
Mar 15:20  ?i dupa ce L-au batjocorit, L-au dezbracat de purpura ?i L-au îmbracat cu hainele Lui. ?i L-au dus afara ca sa-L rastigneasca.
Mar 15:21  ?i au silit pe un trecator, care venea din ?arina, pe Simon Cirineul, tatal lui Alexandru ?i al lui Ruf, ca sa duca crucea Lui.
Mar 15:22  ?i L-au dus la locul zis Golgota, care se talmace?te "locul Capa?ânii".
Mar 15:23  ?i I-au dat sa bea vin amestecat cu smirna, dar El n-a luat.
Mar 15:24  ?i L-au rastignit ?i au împar?it între ei hainele Lui, aruncând sor?i pentru ele, care ce sa ia.
Mar 15:25  Iar când L-au rastignit, era ceasul al treilea.
Mar 15:26  ?i vina Lui era scrisa deasupra: Regele iudeilor.
Mar 15:27  ?i împreuna cu El au rastignit doi tâlhari: unul de-a dreapta ?i altul de-a stânga Lui.
Mar 15:28  ?i s-a împlinit Scriptura care zice: Cu cei fara de lege a fost socotit.
Mar 15:29  Iar cei ce treceau pe acolo Îl huleau, clatinându-?i capetele ?i zicând: Huu! Cel care darâmi templul ?i în trei zile îl zide?ti.
Mar 15:30  Mântuie?te-Te pe Tine Însu?i, coborându-Te de pe cruce!
Mar 15:31  De asemenea ?i arhiereii, batjocorindu-L între ei, împreuna cu carturarii, ziceau: Pe al?ii a mântuit, dar pe Sine nu poate sa Se mântuiasca!
Mar 15:32  Hristos, regele lui Israel, sa Se coboare de pe cruce, ca sa vedem ?i sa credem. ?i cei împreuna rastigni?i cu El Îl ocarau.
Mar 15:33  Iar când a fost ceasul al ?aselea, întuneric s-a facut peste tot pamântul pâna la ceasul al noualea.
Mar 15:34  ?i la al noualea ceas, a strigat Iisus cu glas mare: Eloi, Eloi, lama sabahtani?, care se talmace?te: Dumnezeul Meu, Dumnezeul Meu, de ce M-ai parasit?
Mar 15:35  Iar unii din cei ce stateau acolo, auzind, ziceau: Iata, îl striga pe Ilie.
Mar 15:36  ?i, alergând, unul a înmuiat un burete în o?et, l-a pus într-o trestie ?i I-a dat sa bea, zicând: Lasa?i sa vedem daca vine Ilie ca sa-L coboare.
Mar 15:37  Iar Iisus, sco?ând un strigat mare, ?i-a dat duhul.
Mar 15:38  ?i catapeteasma templului s-a rupt în doua, de sus pâna jos.
Mar 15:39  Iar suta?ul care statea în fa?a Lui, vazând ca astfel ?i-a dat duhul, a zis: Cu adevarat omul acesta era Fiul lui Dumnezeu!
Mar 15:40  ?i erau ?i femei care priveau de departe; între ele: Maria Magdalena, Maria, mama lui Iacov cel Mic ?i a lui Iosi, ?i Salomeea,
Mar 15:41  Care, pe când era El în Galileea, mergeau dupa El ?i Îi slujeau, ?i multe altele care se suisera cu El la Ierusalim.
Mar 15:42  ?i facându-se seara, fiindca era vineri, care este înaintea sâmbetei,
Mar 15:43  ?i venind Iosif cel din Arimateea, sfetnic ales, care a?tepta ?i el împara?ia lui Dumnezeu, ?i, îndraznind, a intrat la Pilat ?i a cerut trupul lui Iisus.
Mar 15:44  Iar Pilat s-a mirat ca a ?i murit ?i, chemând pe suta?, l-a întrebat daca a murit de mult.
Mar 15:45  ?i aflând de la suta?, a daruit lui Iosif trupul.
Mar 15:46  ?i Iosif, cumparând giulgiu ?i coborându-L de pe cruce, L-a înfa?urat în giulgiu ?i L-a pus într-un mormânt care era sapat în stânca, ?i a pravalit o piatra la u?a mormântului.
Mar 15:47  Iar Maria Magdalena ?i Maria, mama lui Iosi, priveau unde L-au pus.
Mar 16:1  ?i dupa ce a trecut ziua sâmbetei, Maria Magdalena, Maria, mama lui Iacov, ?i Salomeea au cumparat miresme, ca sa vina sa-L unga.
Mar 16:2  ?i dis-de-diminea?a, în prima zi a saptamânii (Duminica), pe când rasarea soarele, au venit la mormânt.
Mar 16:3  ?i ziceau între ele: Cine ne va pravali noua piatra de la u?a mormântului?
Mar 16:4  Dar, ridicându-?i ochii, au vazut ca piatra fusese rasturnata; caci era foarte mare.
Mar 16:5  ?i, intrând în mormânt, au vazut un tânar ?ezând în partea dreapta, îmbracat în ve?mânt alb, ?i s-au spaimântat.
Mar 16:6  Iar el le-a zis: Nu va înspaimânta?i! Cauta?i pe Iisus Nazarineanul, Cel rastignit? A înviat! Nu este aici. Iata locul unde L-au pus.
Mar 16:7  Dar merge?i ?i spune?i ucenicilor Lui ?i lui Petru ca va merge în Galileea, mai înainte de voi; acolo îl ve?i vedea, dupa cum v-a spus.
Mar 16:8  ?i ie?ind, au fugit de la mormânt, ca erau cuprinse de frica ?i de uimire, ?i nimanui nimic n-au spus, caci se temeau.
Mar 16:9  ?i înviind diminea?a, în ziua cea dintâi a saptamânii (Duminica) El s-a aratat întâi Mariei Magdalena, din care scosese ?apte demoni.
Mar 16:10  Aceea, mergând, a vestit pe cei ce fusesera cu El ?i care se tânguiau ?i plângeau.
Mar 16:11  ?i ei, auzind ca este viu ?i ca a fost vazut de ea, n-au crezut.
Mar 16:12  Dupa aceea, S-a aratat în alt chip, la doi dintre ei, care mergeau la o ?arina.
Mar 16:13  ?i aceia, mergând, au vestit celorlal?i, dar nici pe ei nu i-au crezut.
Mar 16:14  La urma, pe când cei unsprezece ?edeau la masa, li S-a aratat ?i I-a mustrat pentru necredin?a ?i împietrirea inimii lor, caci n-au crezut pe cei ce-L vazusera înviat.
Mar 16:15  ?i le-a zis: Merge?i în toata lumea ?i propovadui?i Evanghelia la toata faptura.
Mar 16:16  Cel ce va crede ?i se va boteza se va mântui; iar cel ce nu va crede se va osândi.
Mar 16:17  Iar celor ce vor crede, le vor urma aceste semne: în numele Meu, demoni vor izgoni, în limbi noi vor grai,
Mar 16:18  ?erpi vor lua în mâna ?i chiar ceva datator de moarte de vor bea nu-i va vatama, peste cei bolnavi î?i vor pune mâinile ?i se vor face sanato?i.
Mar 16:19  Deci Domnul Iisus, dupa ce a vorbit cu ei, S-a înal?at la cer ?i a ?ezut de-a dreapta lui Dumnezeu.
Mar 16:20  Iar ei, plecând, au propovaduit pretutindeni ?i Domnul lucra cu ei ?i întarea cuvântul, prin semnele care urmau. Amin.


\end{document}