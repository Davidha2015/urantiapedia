\begin{document}

\title{Luke}

Luk 1:1  Deoarece mul?i s-au încercat sa alcatuiasca o istorisire despre faptele deplin adeverite între noi,
Luk 1:2  A?a cum ni le-au lasat cei ce le-au vazut de la început ?i au fost slujitori ai Cuvântului,
Luk 1:3  Am gasit ?i eu cu cale, preaputernice Teofile, dupa ce am urmarit toate cu de-amanuntul de la început, sa ?i le scriu pe rând,
Luk 1:4  Ca sa te încredin?ezi despre temenicia înva?aturii pe care ai primit-o.
Luk 1:5  Era în zilele lui Irod, regele Iudeii, un preot cu numele Zaharia din ceata preo?easca a lui Abia, iar femeia lui era din fiicele lui Aaron ?i se numea Elisabeta.
Luk 1:6  ?i erau amândoi drep?i înaintea lui Dumnezeu, umblând fara prihana în toate poruncile ?i rânduielile Domnului.
Luk 1:7  Dar nu aveau nici un copil, deoarece Elisabeta era stearpa ?i amândoi erau înainta?i în zilele lor.
Luk 1:8  ?i pe când Zaharia slujea înaintea lui Dumnezeu, în rândul saptamânii sale,
Luk 1:9  A ie?it la sor?i, dupa obiceiul preo?iei, sa tamâieze intrând în templul Domnului.
Luk 1:10  Iar toata mul?imea poporului, în ceasul tamâierii, era afara ?i se ruga.
Luk 1:11  ?i i s-a aratat îngerul Domnului, stând de-a dreapta altarului tamâierii.
Luk 1:12  ?i vazându-l, Zaharia s-a tulburat ?i frica a cazut peste el.
Luk 1:13  Iar îngerul a zis catre el: Nu te teme, Zaharia, pentru ca rugaciunea ta a fost ascultata ?i Elisabeta, femeia ta, î?i va na?te un fiu ?i-l vei numi Ioan.
Luk 1:14  ?i bucurie ?i veselie vei avea ?i, de na?terea lui, mul?i se vor bucura.
Luk 1:15  Caci va fi mare înaintea Domnului; nu va bea vin, nici alta bautura ame?itoare ?i înca din pântecele mamei sale se va umple de Duhul Sfânt.
Luk 1:16  ?i pe mul?i din fiii lui Israel îi va întoarce la Domnul Dumnezeul lor.
Luk 1:17  ?i va merge înaintea Lui cu duhul ?i puterea lui Ilie, ca sa întoarca inimile parin?ilor spre copii ?i pe cei neascultatori la în?elepciunea drep?ilor, ca sa gateasca Domnului un popor pregatit.
Luk 1:18  ?i a zis Zaharia catre înger: Dupa ce voi cunoa?te aceasta? Caci eu sunt batrân ?i femeia mea înaintata în zilele ei.
Luk 1:19  ?i îngerul, raspunzând, i-a zis: Eu sunt Gavriil, cel ce sta înaintea lui Dumnezeu. ?i am fost trimis sa graiesc catre tine ?i sa-?i binevestesc acestea.
Luk 1:20  ?i iata vei fi mut ?i nu vei putea sa vorbe?ti pâna în ziua când vor fi acestea, pentru ca n-ai crezut în cuvintele mele, care se vor împlini la timpul lor.
Luk 1:21  ?i poporul a?tepta pe Zaharia ?i se mira ca întârzie în templu.
Luk 1:22  ?i ie?ind, nu putea sa vorbeasca. ?i ei au în?eles ca a vazut vedenie în templu; ?i el le facea semne ?i a ramas mut.
Luk 1:23  ?i când s-au împlinit zilele slujirii lui la templu, s-a dus la casa sa.
Luk 1:24  Iar dupa aceste zile, Elisabeta, femeia lui, a zamislit ?i cinci luni s-a tainuit pe sine, zicând:
Luk 1:25  Ca a?a mi-a facut mie Domnul în zilele în care a socotit sa ridice dintre oameni ocara mea.
Luk 1:26  Iar în a ?asea luna a fost trimis îngerul Gavriil de la Dumnezeu, într-o cetate din Galileea, al carei nume era Nazaret,
Luk 1:27  Catre o fecioara logodita cu un barbat care se chema Iosif, din casa lui David; iar numele fecioarei era Maria.
Luk 1:28  ?i intrând îngerul la ea, a zis: Bucura-te, ceea ce e?ti plina de har, Domnul este cu tine. Binecuvântata e?ti tu între femei.
Luk 1:29  Iar ea, vazându-l, s-a tulburat de cuvântul lui ?i cugeta în sine: Ce fel de închinaciune poate sa fie aceasta?
Luk 1:30  ?i îngerul i-a zis: Nu te teme, Marie, caci ai aflat har la Dumnezeu.
Luk 1:31  ?i iata vei lua în pântece ?i vei na?te fiu ?i vei chema numele lui Iisus.
Luk 1:32  Acesta va fi mare ?i Fiul Celui Preaînalt se va chema ?i Domnul Dumnezeu Îi va da Lui tronul lui David, parintele Sau.
Luk 1:33  ?i va împara?i peste casa lui Iacov în veci ?i împara?ia Lui nu va avea sfâr?it.
Luk 1:34  ?i a zis Maria catre înger: Cum va fi aceasta, de vreme ce eu nu ?tiu de barbat?
Luk 1:35  ?i raspunzând, îngerul i-a zis: Duhul Sfânt Se va pogorî peste tine ?i puterea Celui Preaînalt te va umbri; pentru aceea ?i Sfântul care Se va na?te din tine, Fiul lui Dumnezeu se va chema.
Luk 1:36  ?i iata Elisabeta, rudenia ta, a zamislit ?i ea fiu la batrâne?ea ei ?i aceasta este a ?asea luna pentru ea, cea numita stearpa.
Luk 1:37  Ca la Dumnezeu nimic nu este cu neputin?a.
Luk 1:38  ?i a zis Maria: Iata roaba Domnului. Fie mie dupa cuvântul tau! ?i îngerul a plecat de la ea.
Luk 1:39  ?i în acele zile, sculându-se Maria, s-a dus în graba în ?inutul muntos, într-o cetate a semin?iei lui Iuda.
Luk 1:40  ?i a intrat în casa lui Zaharia ?i a salutat pe Elisabeta.
Luk 1:41  Iar când a auzit Elisabeta salutarea Mariei, pruncul a saltat în pântecele ei ?i Elisabeta s-a umplut de Duh Sfânt,
Luk 1:42  ?i cu glas mare a strigat ?i a zis: Binecuvântata e?ti tu între femei ?i binecuvântat este rodul pântecelui tau.
Luk 1:43  ?i de unde mie aceasta, ca sa vina la mine Maica Domnului meu?
Luk 1:44  Ca iata, cum veni la urechile mele glasul salutarii tale, pruncul a saltat de bucurie în pântecele meu.
Luk 1:45  ?i fericita este aceea care a crezut ca se vor împlini cele spuse ei de la Domnul.
Luk 1:46  ?i a zis Maria: Mare?te sufletul meu pe Domnul.
Luk 1:47  ?i s-a bucurat duhul meu de Dumnezeu, Mântuitorul meu,
Luk 1:48  Ca a cautat spre smerenia roabei Sale. Ca, iata, de acum ma vor ferici toate neamurile.
Luk 1:49  Ca mi-a facut mie marire Cel Puternic ?i sfânt este numele Lui.
Luk 1:50  ?i mila Lui în neam ?i în neam spre cei ce se tem de El.
Luk 1:51  Facut-a tarie cu bra?ul Sau, risipit-a pe cei mândri în cugetul inimii lor.
Luk 1:52  Coborât-a pe cei puternici de pe tronuri ?i a înal?at pe cei smeri?i,
Luk 1:53  Pe cei flamânzi i-a umplut de bunata?i ?i pe cei boga?i i-a scos afara de?er?i.
Luk 1:54  A sprijinit pe Israel, slujitorul Sau, ca sa-?i aduca aminte de mila Sa,
Luk 1:55  Precum a grait catre parin?ii no?tri, lui Avraam ?i semin?iei lui, în veac.
Luk 1:56  ?i a ramas Maria împreuna cu ea ca la trei luni; ?i s-a înapoiat la casa sa.
Luk 1:57  ?i dupa ce s-a împlinit vremea sa nasca, Elisabeta a nascut un fiu.
Luk 1:58  ?i au auzit vecinii ?i rudele ei ca Domnul a marit mila Sa fa?a de ea ?i se bucurau împreuna cu ea.
Luk 1:59  Iar când a fost în ziua a opta, au venit sa taie împrejur pruncul ?i-l numeau Zaharia, dupa numele tatalui sau.
Luk 1:60  ?i raspunzând, mama lui a zis: Nu! Ci se va chema Ioan.
Luk 1:61  ?i au zis catre ea: Nimeni din rudenia ta nu se cheama cu numele acesta.
Luk 1:62  ?i au facut semn tatalui sau cum ar vrea el sa fie numit.
Luk 1:63  ?i cerând o tabli?a, el a scris, zicând: Ioan este numele lui. ?i to?i s-au mirat.
Luk 1:64  ?i îndata i s-a deschis gura ?i limba ?i vorbea, binecuvântând pe Dumnezeu.
Luk 1:65  ?i frica i-a cuprins pe to?i care locuiau împrejurul lor; ?i în tot ?inutul muntos al Iudeii s-au vestit toate aceste cuvinte.
Luk 1:66  ?i to?i care le auzeau le puneau la inima, zicând: Ce va fi, oare, acest copil? Caci mâna Domnului era cu el.
Luk 1:67  ?i Zaharia, tatal lui, s-a umplut de Duh Sfânt ?i a proorocit, zicând:
Luk 1:68  Binecuvântat este Domnul Dumnezeul lui Israel, ca a cercetat ?i a facut rascumparare poporului Sau;
Luk 1:69  ?i ne-a ridicat putere de mântuire în casa lui David, slujitorul Sau,
Luk 1:70  Precum a grait prin gura sfin?ilor Sai prooroci din veac;
Luk 1:71  Mântuire de vrajma?ii no?tri ?i din mâna tuturor celor ce ne urasc pe noi.
Luk 1:72  ?i sa faca mila cu parin?ii no?tri, ca ei sa-?i aduca aminte de legamântul Sau cel sfânt;
Luk 1:73  De juramântul cu care S-a jurat catre Avraam, parintele nostru,
Luk 1:74  Ca, fiind izbavi?i din mâna vrajma?ilor, sa ne dea noua fara frica,
Luk 1:75  Sa-I slujim în sfin?enie ?i în dreptate, înaintea fe?ei Sale, în toate zilele vie?ii noastre.
Luk 1:76  Iar tu, pruncule, prooroc al Celui Preaînalt te vei chema, ca vei merge înaintea fe?ei Domnului, ca sa gate?ti caile Lui,
Luk 1:77  Sa dai poporului Sau cuno?tin?a mântuirii întru iertarea pacatelor lor,
Luk 1:78  Prin milostivirea milei Dumnezeului nostru, cu care ne-a cercetat pe noi  Rasaritul cel de Sus,
Luk 1:79  Ca sa lumineze pe cei care ?ed în întuneric ?i în umbra mor?ii ?i sa îndrepte picioarele noastre pe calea pacii.
Luk 1:80  Iar copilul cre?tea ?i se întarea cu duhul. ?i a fost în pustie pâna în ziua aratarii lui catre Israel.
Luk 2:1  În zilele acelea a ie?it porunca de la Cezarul August sa se înscrie toata lumea.
Luk 2:2  Aceasta înscriere s-a facut întâi pe când Quirinius ocârmuia Siria.
Luk 2:3  ?i se duceau to?i sa se înscrie, fiecare în cetatea sa.
Luk 2:4  ?i s-a suit ?i Iosif din Galileea, din cetatea Nazaret, în Iudeea, în cetatea lui David care se nume?te Betleem, pentru ca el era din casa ?i din neamul lui David.
Luk 2:5  Ca sa se înscrie împreuna cu Maria, cea logodita cu el, care era însarcinata.
Luk 2:6  Dar pe când erau ei acolo, s-au împlinit zilele ca ea sa nasca,
Luk 2:7  ?i a nascut pe Fiul sau, Cel Unul-Nascut ?i L-a înfa?at ?i L-a culcat în iesle, caci nu mai era loc de gazduire pentru ei.
Luk 2:8  ?i în ?inutul acela erau pastori, stând pe câmp ?i facând de straja noaptea împrejurul turmei lor.
Luk 2:9  ?i iata îngerul Domnului a statut lânga ei ?i slava Domnului a stralucit împrejurul lor, ?i ei s-au înfrico?at cu frica mare.
Luk 2:10  Dar îngerul le-a zis: Nu va teme?i. Caci, iata, va binevestesc voua bucurie mare, care va fi pentru tot poporul.
Luk 2:11  Ca vi s-a nascut azi Mântuitor, Care este Hristos Domnul, în cetatea lui David.
Luk 2:12  ?i acesta va fi semnul: Ve?i gasi un prunc înfa?at, culcat în iesle.
Luk 2:13  ?i deodata s-a vazut, împreuna cu îngerul, mul?ime de oaste cereasca, laudând pe Dumnezeu ?i zicând:
Luk 2:14  Slava întru cei de sus lui Dumnezeu ?i pe pamânt pace, între oameni bunavoire!
Luk 2:15  Iar dupa ce îngerii au plecat de la ei, la cer, pastorii vorbeau unii catre al?ii: Sa mergem dar pâna la Betleem, sa vedem cuvântul acesta ce s-a facut ?i pe care Domnul ni l-a facut cunoscut.
Luk 2:16  ?i, grabindu-se, au venit ?i au aflat pe Maria ?i pe Iosif ?i pe Prunc, culcat în iesle.
Luk 2:17  ?i vazându-L, au vestit cuvântul grait lor despre acest Copil.
Luk 2:18  ?i to?i câ?i auzeau se mirau de cele spuse lor de catre pastori.
Luk 2:19  Iar Maria pastra toate aceste cuvinte, punându-le în inima sa.
Luk 2:20  ?i s-au întors pastorii, slavind ?i laudând pe Dumnezeu, pentru toate câte auzisera ?i vazusera precum li se spusese.
Luk 2:21  ?i când s-au împlinit opt zile, ca sa-L taie împrejur, I-au pus numele Iisus, cum a fost numit de înger, mai înainte de a se zamisli în pântece.
Luk 2:22  ?i când s-au împlinit zilele cura?irii lor, dupa legea lui Moise, L-au adus pe Prunc la Ierusalim, ca sa-L puna înaintea Domnului.
Luk 2:23  Precum este scris în Legea Domnului, ca orice întâi-nascut de parte barbateasca sa fie închinat Domnului.
Luk 2:24  ?i sa dea jertfa, precum s-a zis în Legea Domnului, o pereche de turturele sau doi pui de porumbel.
Luk 2:25  ?i iata era un om în Ierusalim, cu numele Simeon; ?i omul acesta era drept ?i temator de Dumnezeu, a?teptând mângâierea lui Israel, ?i Duhul Sfânt era asupra lui.
Luk 2:26  ?i lui i se vestise de catre Duhul Sfânt ca nu va vedea moartea pâna ce nu va vedea pe Hristosul Domnului.
Luk 2:27  ?i din îndemnul Duhului a venit la templu; ?i când parin?ii au adus înauntru pe Pruncul Iisus, ca sa faca pentru El dupa obiceiul Legii,
Luk 2:28  El L-a primit în bra?ele sale ?i a binecuvântat pe Dumnezeu ?i a zis:
Luk 2:29  Acum sloboze?te pe robul Tau, dupa cuvântul Tau, în pace,
Luk 2:30  Ca ochii mei vazura mântuirea Ta,
Luk 2:31  Pe care ai gatit-o înaintea fe?ei tuturor popoarelor,
Luk 2:32  Lumina spre descoperirea neamurilor ?i slava poporului Tau Israel.
Luk 2:33  Iar Iosif ?i mama Lui se mirau de ceea ce se vorbea despre Prunc.
Luk 2:34  ?i i-a binecuvântat Simeon ?i a zis catre Maria, mama Lui: Iata, Acesta este pus spre caderea ?i spre ridicarea multora din Israel ?i ca un semn care va stârni împotriviri.
Luk 2:35  ?i prin sufletul tau va trece sabie, ca sa se descopere gândurile din multe inimi.
Luk 2:36  ?i era ?i Ana prooroci?a, fiica lui Fanuel, din semin?ia lui A?er, ajunsa la adânci batrâne?e ?i care traise cu barbatul ei ?apte ani de la fecioria sa.
Luk 2:37  ?i ea era vaduva, în vârsta de optzeci ?i patru de ani, ?i nu se departa de templu, slujind noaptea ?i ziua în post ?i în rugaciuni.
Luk 2:38  ?i venind ea în acel ceas, lauda pe Dumnezeu ?i vorbea despre Prunc tuturor celor ce a?teptau mântuire în Ierusalim.
Luk 2:39  Dupa ce au savâr?it toate, s-au întors în Galileea, în cetatea lor Nazaret.
Luk 2:40  Iar Copilul cre?tea ?i Se întarea cu duhul, umplându-Se de în?elepciune ?i harul lui Dumnezeu era asupra Lui.
Luk 2:41  ?i parin?ii Lui, în fiecare an, se duceau de sarbatoarea Pa?tilor, la Ierusalim.
Luk 2:42  Iar când a fost El de doisprezece ani, s-au suit la Ierusalim, dupa obiceiul sarbatorii.
Luk 2:43  ?i sfâr?indu-se zilele, pe când se întorceau ei, Copilul Iisus a ramas în Ierusalim ?i parin?ii Lui nu ?tiau.
Luk 2:44  ?i socotind ca este în ceata calatorilor de drum, au venit cale de o zi, cautându-L printre rude ?i printre cunoscu?i.
Luk 2:45  ?i, negasindu-L, s-au întors la Ierusalim, cautându-L.
Luk 2:46  Iar dupa trei zile L-au aflat în templu, ?ezând în mijlocul înva?atorilor, ascultându-i ?i întrebându-i.
Luk 2:47  ?i to?i care Îl auzeau se minunau de priceperea ?i de raspunsurile Lui.
Luk 2:48  ?i vazându-L, ramasera uimi?i, iar mama Lui a zis catre El: Fiule, de ce ne-ai facut noua a?a? Iata, tatal Tau ?i eu Te-am cautat îngrijora?i.
Luk 2:49  ?i El a zis catre ei: De ce era sa Ma cauta?i? Oare, nu ?tia?i ca în cele ale Tatalui Meu trebuie sa fiu?
Luk 2:50  Dar ei n-au în?eles cuvântul pe care l-a spus lor.
Luk 2:51  ?i a coborât cu ei ?i a venit în Nazaret ?i le era supus. Iar mama Lui pastra în inima ei toate aceste cuvinte.
Luk 2:52  ?i Iisus sporea cu în?elepciunea ?i cu vârsta ?i cu harul la Dumnezeu ?i la oameni.
Luk 3:1  În al cincisprezecelea an al domniei Cezarului Tiberiu, pe când Pon?iu Pilat era procuratorul Iudeii, Irod, tetrarh al Galileii, Filip, fratele sau, tetrarh al Itureii ?i al ?inutului Trahonitidei, iar Lisanias, tetrarh al Abilenei,
Luk 3:2  În zilele arhiereilor Anna ?i Caiafa, a fost cuvântul lui Dumnezeu catre Ioan, fiul lui Zaharia, în pustie.
Luk 3:3  ?i a venit el în toata împrejurimea Iordanului, propovaduind botezul pocain?ei, spre iertarea pacatelor.
Luk 3:4  Precum este scris în cartea cuvintelor lui Isaia proorocul: "Este glasul celui ce striga în pustie: Gati?i calea Domnului, drepte face?i cararile Lui.
Luk 3:5  Orice vale se va umple ?i orice munte ?i orice deal se va pleca; caile cele strâmbe se vor face drepte ?i cele col?uroase, drumuri netede.
Luk 3:6  ?i toata faptura va vedea mântuirea lui Dumnezeu".
Luk 3:7  Deci zicea Ioan mul?imilor care veneau sa se boteze de el: Pui de vipere, cine v-a aratat sa fugi?i de mânia ce va sa fie?
Luk 3:8  Face?i, dar, roade vrednice de pocain?a ?i nu începe?i a zice în voi în?iva: Avem tata pe Avraam, caci va spun ca Dumnezeu poate ?i din pietrele acestea sa ridice fii lui Avraam.
Luk 3:9  Acum securea sta la radacina pomilor; deci orice pom care nu face roada buna se taie ?i se arunca în foc.
Luk 3:10  ?i mul?imile îl întrebau, zicând: Ce sa facem deci?
Luk 3:11  Raspunzând, Ioan le zicea: Cel ce are doua haine sa dea celui ce nu are ?i cel ce are bucate sa faca asemenea.
Luk 3:12  ?i au venit ?i vame?ii sa se boteze ?i i-au spus: Înva?atorule, noi ce sa facem?
Luk 3:13  El le-a raspuns: Nu face?i nimic mai mult peste ce va este rânduit.
Luk 3:14  ?i îl întrebau ?i osta?ii, zicând: Dar noi ce sa facem? ?i le-a zis: Sa nu asupri?i pe nimeni, nici sa învinui?i pe nedrept, ?i sa fi?i mul?umi?i cu solda voastra.
Luk 3:15  Iar poporul fiind în a?teptare ?i întrebându-se to?i despre Ioan în cugetele lor: Nu cumva el este Hristosul?
Luk 3:16  A raspuns Ioan tuturor, zicând: Eu va botez cu apa, dar vine Cel ce este mai tare decât mine, Caruia nu sunt vrednic sa-I dezleg cureaua încal?amintelor. El va va boteza cu Duh Sfânt ?i cu foc,
Luk 3:17  A Carui lopata este în mâna Lui, ca sa cure?e aria ?i sa adune grâul în jitni?a Sa, iar pleava o va arde cu foc nestins.
Luk 3:18  Înca ?i alte multe îndemnând, propovaduia poporului vestea cea buna.
Luk 3:19  Iar Irod tetrarhul, mustrat fiind de el pentru Irodiada, femeia lui Filip, fratele sau, ?i pentru toate relele pe care le-a facut Irod,
Luk 3:20  A adaugat la toate ?i aceasta, încât a închis pe Ioan în temni?a.
Luk 3:21  ?i dupa ce s-a botezat tot poporul, botezându-Se ?i Iisus ?i rugându-Se, s-a deschis cerul,
Luk 3:22  ?i S-a coborât Duhul Sfânt peste El, în chip trupesc, ca un porumbel, ?i s-a facut glas din cer: Tu e?ti Fiul Meu cel iubit, întru Tine am binevoit.
Luk 3:23  ?i Iisus Însu?i era ca de treizeci de ani când a început (sa propovaduiasca), fiind, precum se socotea, fiu al lui Iosif, care era fiul lui Eli,
Luk 3:24  Fiul lui Matat, fiul lui Levi, fiul lui Melhi, fiul lui Ianai, fiul lui Iosif,
Luk 3:25  Fiul lui Matatia, fiul lui Amos, fiul lui Naum, fiul lui Esli, fiul lui Nagai,
Luk 3:26  Fiul lui Iosua, fiul lui Matatia, fiul lui Semein, fiul lui Ioseh, fiul lui Ioda,
Luk 3:27  Fiul lui Ioanan, fiul lui Resa, fiul lui Zorobabel, fiul lui Salatiel, fiul lui Neri,
Luk 3:28  Fiul lui Melhi, fiul lui Adi, fiul lui Cosam, fiul lui Elmadam, fiul lui Er,
Luk 3:29  Fiul lui Iosua, fiul lui Eliezer, fiul lui Lorim, fiul lui Matat, fiul lui Levi,
Luk 3:30  Fiul lui Simeon, fiul lui Iuda, fiul lui Iosif, fiul lui Ionam, fiul lui Eliachim,
Luk 3:31  Fiul lui Melea, fiul lui Mena, fiul lui Matata, fiul lui Natan, fiul lui David,
Luk 3:32  Fiul lui Iesei, fiul lui Iobed, fiul lui Booz, fiul lui Sala, fiul lui Naason,
Luk 3:33  Fiul Aminadav, fiul lui Admin, fiul lui Arni, fiul lui Esrom, fiul lui Fares, fiul lui Iuda.
Luk 3:34  Fiul lui Iacov, fiul lui Isaac, fiul lui Avraam, fiul lui Tara, fiul lui Nahor,
Luk 3:35  Fiul lui Serug, fiul lui Ragav, fiul lui Falec, fiul lui Eber, fiul lui Sala,
Luk 3:36  Fiul lui Cainam, fiul lui Arfaxad, fiul lui Sim, fiul lui Noe, fiul lui Lameh,
Luk 3:37  Fiul lui Matusala, fiul lui Enoh, fiul Iaret, fiul lui Maleleil, fiul lui Cainam,
Luk 3:38  Fiul lui Enos, fiul lui Set, fiul lui Adam, fiul lui Dumnezeu.
Luk 4:1  Iar Iisus, plin de Duhul Sfânt, S-a întors de la Iordan ?i a fost dus de Duhul în pustie,
Luk 4:2  Timp de patruzeci de zile, fiind ispitit de diavolul. ?i în aceste zile nu a mâncat nimic; ?i, sfâr?indu-se ele, a flamânzit.
Luk 4:3  ?i I-a spus diavolul: Daca e?ti Fiul lui Dumnezeu, zi acestei pietre sa se faca pâine.
Luk 4:4  ?i a raspuns Iisus catre el: Scris este ca nu numai cu pâine va trai omul, ci cu orice cuvânt al lui Dumnezeu.
Luk 4:5  ?i suindu-L diavolul pe un munte înalt, I-a aratat într-o clipa toate împara?iile lumii.
Luk 4:6  ?i I-a zis diavolul: ?ie î?i voi da toata stapânirea aceasta ?i stralucirea lor, caci mi-a fost data mie ?i eu o dau cui voiesc;
Luk 4:7  Deci daca Tu Te vei închina înaintea mea, toata va fi a Ta.
Luk 4:8  ?i raspunzând, Iisus i-a zis: Mergi înapoia Mea, satano, caci scris este: "Domnului Dumnezeului tau sa te închini ?i numai Lui Unuia sa-I sluje?ti".
Luk 4:9  ?i L-a dus în Ierusalim ?i L-a a?ezat pe aripa templului ?i I-a zis: Daca e?ti Fiul lui Dumnezeu, arunca-Te de aici jos;
Luk 4:10  Caci scris este: "Ca îngerilor Sai va porunci pentru Tine, ca sa Te pazeasca";
Luk 4:11  ?i te vor ridica pe mâini, ca nu cumva sa love?ti de piatra piciorul Tau.
Luk 4:12  ?i raspunzând, Iisus i-a zis: S-a spus: "Sa nu ispite?ti pe Domnul Dumnezeul tau".
Luk 4:13  ?i diavolul, sfâr?ind toata ispita, s-a îndepartat de la El, pâna la o vreme.
Luk 4:14  ?i S-a întors Iisus în puterea Duhului în Galileea ?i a ie?it vestea despre El în toata împrejurimea.
Luk 4:15  ?i înva?a în sinagogile lor, slavit fiind de to?i.
Luk 4:16  ?i a venit în Nazaret, unde fusese crescut, ?i, dupa obiceiul Sau, a intrat în ziua sâmbetei în sinagoga ?i S-a sculat sa citeasca.
Luk 4:17  ?i I s-a dat cartea proorocului Isaia. ?i, deschizând El cartea, a gasit locul unde era scris:
Luk 4:18  "Duhul Domnului este peste Mine, pentru care M-a uns sa binevestesc saracilor; M-a trimis sa vindec pe cei zdrobi?i cu inima; sa propovaduiesc robilor dezrobirea ?i celor orbi vederea; sa slobozesc pe cei apasa?i,
Luk 4:19  ?i sa vestesc anul placut Domnului".
Luk 4:20  ?i închizând cartea ?i dând-o slujitorului, a ?ezut, iar ochii tuturor erau a?inti?i asupra Lui.
Luk 4:21  ?i El a început a zice catre ei: Astazi s-a împlinit Scriptura aceasta în urechile voastre.
Luk 4:22  ?i to?i Îl încuviin?au ?i se mirau de cuvintele harului care ie?eau din gura Lui ?i ziceau: Nu este, oare, Acesta fiul lui Iosif?
Luk 4:23  ?i El le-a zis: Cu adevarat Îmi ve?i spune aceasta pilda: Doctore, vindeca-te pe tine însu?i! Câte am auzit ca s-au facut în Capernaum, fa ?i aici în patria Ta.
Luk 4:24  ?i le-a zis: Adevarat zic voua ca nici un prooroc nu este bine primit în patria sa.
Luk 4:25  ?i adevarat va spun ca multe vaduve erau în zilele lui Ilie, în Israel, când  s-a închis cerul trei ani ?i ?ase luni, încât a fost foamete mare peste tot pamântul.
Luk 4:26  ?i la nici una dintre ele n-a fost trimis Ilie, decât la Sarepta Sidonului, la o femeie vaduva.
Luk 4:27  ?i mul?i lepro?i erau în Israel în zilele proorocului Elisei, dar nici unul dintre ei nu s-a cura?at, decât Neeman Sirianul.
Luk 4:28  ?i to?i, în sinagoga, auzind acestea, s-au umplut de mânie.
Luk 4:29  ?i sculându-se, L-au scos afara din cetate ?i L-au dus pe sprânceana muntelui, pe care era zidita cetatea lor, ca sa-L arunce în prapastie;
Luk 4:30  Iar El, trecând prin mijlocul lor, S-a dus.
Luk 4:31  ?i S-a coborât la Capernaum, cetate a Galileii, ?i îi înva?a sâmbata.
Luk 4:32  ?i erau uimi?i de înva?atura Lui, caci cuvântul Lui era cu putere.
Luk 4:33  Iar în sinagoga era un om, având duh de demon necurat, ?i a strigat cu glas tare:
Luk 4:34  Lasa! Ce ai cu noi, Iisuse Nazarinene? Ai venit ca sa ne pierzi? Te ?tim cine e?ti: Sfântul lui Dumnezeu.
Luk 4:35  ?i l-a certat Iisus, zicând: Taci ?i ie?i din el. Iar demonul, aruncându-l în mijlocul sinagogii, a ie?it din el, cu nimic vatamându-l.
Luk 4:36  ?i frica li s-a facut tuturor ?i spuneau unii catre al?ii, zicând: Ce este acest cuvânt? Ca porunce?te duhurilor necurate, cu stapânire ?i cu putere, ?i ele ies.
Luk 4:37  ?i a ie?it vestea despre El în tot locul din împrejurimi.
Luk 4:38  ?i sculându-Se din sinagoga, a intrat în casa lui Simon. Iar soacra lui Simon era prinsa de friguri rele ?i L-au rugat pentru ea.
Luk 4:39  ?i El, plecându-Se asupra ei, a certat frigurile ?i frigurile au lasat-o. Iar ea, îndata sculându-se, le slujea;
Luk 4:40  Dar apunând soarele, to?i câ?i aveau bolnavi de felurite boli îi aduceau la El; iar El, punându-?i mâinile pe fiecare dintre ei, îi facea sanato?i.
Luk 4:41  Din mul?i ie?eau ?i demoni, care strigau ?i ziceau: Tu e?ti Fiul lui Dumnezeu. Dar El, certându-i, nu-i lasa sa vorbeasca acestea, ca ?tiau ca El este Hristosul.
Luk 4:42  Iar facându-se ziua, a ie?it ?i S-a dus într-un loc pustiu; ?i mul?imile Îl cautau ?i au venit pâna la El, ?i-L ?ineau ca sa nu plece de la ei.
Luk 4:43  ?i El a zis catre ei: Trebuie sa binevestesc împara?ia lui Dumnezeu ?i altor ceta?i, fiindca pentru aceasta am fost trimis.
Luk 4:44  ?i propovaduia în sinagogile Galileii.
Luk 5:1  Pe când mul?imea Îl îmbulzea, ca sa asculte cuvântul lui Dumnezeu, ?i El ?edea lânga lacul Ghenizaret,
Luk 5:2  A vazut doua corabii oprite lânga ?arm, iar pescarii, coborând din ele, spalau mrejele.
Luk 5:3  ?i urcându-Se într-una din corabii care era a lui Simon, l-a rugat s-o departeze pu?in de la uscat. ?i ?ezând în corabie, înva?a, din ea, mul?imile.
Luk 5:4  Iar când a încetat de a vorbi, i-a zis lui Simon: Mâna la adânc, ?i lasa?i în jos mrejele voastre, ca sa pescui?i.
Luk 5:5  ?i, raspunzând, Simon a zis: Înva?atorule, toata noaptea ne-am trudit ?i nimic nu am prins, dar, dupa cuvântul Tau, voi arunca mrejele.
Luk 5:6  ?i facând ei aceasta, au prins mul?ime mare de pe?te, ca li se rupeau mrejele.
Luk 5:7  ?i au facut semn celor care erau în cealalta corabie, sa vina sa le ajute. ?i au venit ?i au umplut amândoua corabiile, încât erau gata sa se afunde,
Luk 5:8  Iar Simon Petru, vazând aceasta, a cazut la genunchii lui Iisus, zicând: Ie?i de la mine, Doamne, ca sunt om pacatos.
Luk 5:9  Caci spaima îl cuprinsese pe el ?i pe to?i cei ce erau cu el, pentru pescuitul pe?tilor pe care îi prinsesera.
Luk 5:10  Tot a?a ?i pe Iacov ?i pe Ioan, fiii lui Zevedeu, care erau împreuna cu Simon. ?i a zis Iisus catre Simon: Nu te teme; de acum înainte vei fi pescar de oameni.
Luk 5:11  ?i tragând corabiile la ?arm, au lasat totul ?i au mers dupa El.
Luk 5:12  ?i pe când erau într-una din ceta?i, iata un om plin de lepra; vazând pe Iisus, a cazut cu fa?a la pamânt ?i I s-a rugat zicând: Doamne, daca voie?ti, po?i sa ma cura?e?ti.
Luk 5:13  ?i întinzând El mâna, S-a atins de lepros, zicând: Voiesc, fii cura?at! ?i îndata s-a dus lepra de pe el.
Luk 5:14  Iar Iisus i-a poruncit sa nu spuna nimanui, ci, mergând, arata-te preotului ?i, pentru cura?irea ta, du jertfa, precum a orânduit Moise, spre marturie lor.
Luk 5:15  Dar ?i mai mult strabatea vorba despre El ?i mul?imi multe se adunau, ca sa asculte ?i sa se vindece de bolile lor.
Luk 5:16  Iar El Se retragea în locuri pustii ?i Se ruga.
Luk 5:17  ?i într-una din zile Iisus înva?a ?i de fa?a ?edeau farisei ?i înva?atori ai Legii, veni?i din toate satele Galileii, din Iudeea ?i din Ierusalim. ?i puterea Domnului se arata în tamaduiri.
Luk 5:18  ?i iata ni?te barba?i aduceau pe pat un om care era slabanog ?i cautau sa-l duca înauntru ?i sa-l puna înaintea Lui;
Luk 5:19  Dar negasind pe unde sa-l duca, din pricina mul?imii, s-au suit pe acoperi? ?i, printre caramizi, l-au lasat cu patul în mijloc, înaintea lui Iisus.
Luk 5:20  ?i vazând credin?a lor, El le-a zis: Omule, iertate î?i sunt pacatele tale.
Luk 5:21  Iar fariseii ?i carturarii au început sa cârteasca, zicând: Cine este Acesta care graie?te hule? Cine poate sa ierte pacatele decât unul Dumnezeu?
Luk 5:22  Iar Iisus, cunoscând gândurile lor, raspunzând a zis catre ei: Ce cugeta?i în inimile voastre?
Luk 5:23  Ce este mai u?or? A zice: Iertate sunt pacatele tale, sau a zice: Scoala ?i umbla?
Luk 5:24  Iar ca sa ?ti?i ca Fiul Omului are pe pamânt putere sa ierte pacatele, a zis slabanogului: ?ie î?i zic: Scoala-te, ia patul tau ?i mergi la casa ta.
Luk 5:25  ?i îndata, ridicându-se înaintea lor, luând patul pe care zacuse, s-a dus la casa sa, slavind pe Dumnezeu.
Luk 5:26  ?i uimire i-a cuprins pe to?i ?i slaveau pe Dumnezeu ?i, plini de frica, ziceau: Am vazut astazi lucruri minunate.
Luk 5:27  ?i dupa aceasta a ie?it ?i a vazut un vame?, cu numele Levi, care ?edea la vama, ?i i-a zis: Vino dupa Mine.
Luk 5:28  ?i, lasând toate, el s-a sculat ?i a mers dupa El.
Luk 5:29  ?i I-a facut Levi un ospa? mare în casa sa. ?i era mul?ime multa de vame?i ?i de al?ii care ?edeau cu ei la masa.
Luk 5:30  Dar fariseii ?i carturarii lor murmurau catre ucenicii Lui, zicând: De ce mânca?i ?i be?i împreuna cu vame?ii ?i cu pacato?ii?
Luk 5:31  ?i Iisus, raspunzând, a zis catre ei: N-au trebuin?a de doctor cei sanato?i, ci cei bolnavi.
Luk 5:32  N-am venit sa chem pe drep?i, ci pe pacato?i la pocain?a.
Luk 5:33  Iar ei au zis catre El: Ucenicii lui Ioan postesc adesea ?i fac rugaciuni, de asemenea ?i ai fariseilor, iar ai Tai manânca ?i beau.
Luk 5:34  Iar Iisus a zis catre ei: Pute?i, oare, sa face?i pe fiii nun?ii sa posteasca, cât timp Mirele este cu ei?
Luk 5:35  Dar vor veni zile când Mirele se va lua de la ei; atunci vor posti în acele zile.
Luk 5:36  Le-a spus lor ?i o pilda: Nimeni, rupând petic de la haina noua, nu-l pune la haina veche, altfel rupe haina cea noua, iar peticul luat din ea nu se potrive?te la cea veche.
Luk 5:37  ?i nimeni nu pune vin nou în burdufuri vechi, altfel vinul nou va sparge burdufurile; ?i se varsa ?i vinul ?i se strica ?i burdufurile.
Luk 5:38  Ci vinul nou trebuie pus în burdufuri noi ?i împreuna se vor pastra.
Luk 5:39  ?i nimeni, bând vin vechi, nu voie?te de cel nou, caci zice: E mai bun cel vechi.
Luk 6:1  Într-o sâmbata, a doua dupa Pa?ti, Iisus mergea prin semanaturi ?i ucenicii Lui smulgeau spice, le frecau cu mâinile ?i mâncau.
Luk 6:2  Dar unii dintre farisei au zis: De ce face?i ce nu se cade a face sâmbata?
Luk 6:3  ?i Iisus, raspunzând, a zis catre ei: Oare n-a?i citit ce a facut David, când a flamânzit el ?i cei ce erau cu el?
Luk 6:4  Cum a intrat în casa lui Dumnezeu ?i a luat pâinile punerii înainte ?i a mâncat ?i a dat ?i înso?itorilor sai, din ele, pe care nu se cuvine sa le manânce decât numai preo?ii?
Luk 6:5  ?i le zicea: Fiul Omului este Domn ?i al sâmbetei.
Luk 6:6  Iar în alta sâmbata, a intrat El în sinagoga ?i înva?a. ?i era acolo un om a carui mâna dreapta era uscata.
Luk 6:7  Dar carturarii ?i fariseii Îl pândeau de-l va vindeca sâmbata, ca sa-I gaseasca vina.
Luk 6:8  Însa El ?tia gândurile lor ?i a zis omului care avea mâna uscata: Scoala-te ?i stai la mijloc. El s-a sculat ?i a stat.
Luk 6:9  Atunci Iisus a zis catre ei: Va întreb pe voi, ce se cade sâmbata: a face bine sau a face rau? A scapa un suflet sau a-l pierde?
Luk 6:10  ?i privind împrejur pe to?i ace?tia, i-a zis: Întinde mâna ta. Iar el a facut a?a ?i mâna lui s-a facut sanatoasa, ca ?i cealalta.
Luk 6:11  Ei însa s-au umplut de mânie ?i vorbeau unii cu al?ii ce sa faca cu Iisus.
Luk 6:12  ?i în zilele acelea, Iisus a ie?it la munte ca sa Se roage ?i a petrecut noaptea în rugaciune catre Dumnezeu.
Luk 6:13  ?i când s-a facut ziua, a chemat la Sine pe ucenicii Sai ?i a ales dintre ei doisprezece, pe care i-a numit Apostoli.
Luk 6:14  Pe Simon, caruia i-a zis Petru, ?i pe Andrei, fratele lui, ?i pe Iacov, ?i pe Ioan, ?i pe Filip, ?i pe Vartolomeu,
Luk 6:15  ?i pe Matei, ?i pe Toma, ?i pe Iacov al lui Alfeu ?i pe Simon numit Zilotul,
Luk 6:16  ?i pe Iuda al lui Iacov ?i pe Iuda Iscarioteanul, care s-a facut tradator.
Luk 6:17  ?i coborând împreuna cu ei, a stat în loc ?es, El ?i mul?ime multa de ucenici ai Sai ?i mul?ime mare de popor din toata Iudeea, din Ierusalim ?i de pe ?armul Tirului ?i al Sidonului, care venisera ca sa-L asculte ?i sa se vindece de bolile lor.
Luk 6:18  ?i cei chinui?i de duhuri necurate se vindecau.
Luk 6:19  ?i toata mul?imea cauta sa se atinga de El ca putere ie?ea din El ?i-i vindeca pe to?i.
Luk 6:20  ?i El, ridicându-?i ochii spre ucenicii Sai, zicea: Ferici?i voi cei saraci, ca a voastra este împara?ia lui Dumnezeu.
Luk 6:21  Ferici?i voi care flamânzi?i acum, ca va ve?i satura. Ferici?i cei ce plânge?i acum, ca ve?i râde.
Luk 6:22  Ferici?i ve?i fi când oamenii va vor urî pe voi ?i va vor izgoni dintre ei, ?i va vor batjocori ?i vor lepada numele voastre ca rau din pricina Fiului Omului.
Luk 6:23  Bucura?i-va în ziua aceea ?i va veseli?i, ca, iata, plata voastra multa este în cer; pentru ca tot a?a faceau proorocilor parin?ii lor.
Luk 6:24  Dar vai voua boga?ilor, ca va lua?i pe pamânt mângâierea voastra.
Luk 6:25  Vai voua celor ce sunte?i satui acum, ca ve?i flamânzi. Vai voua celor ce astazi râde?i, ca ve?i plânge ?i va ve?i tângui.
Luk 6:26  Vai voua când to?i oamenii va vor vorbi de bine. Caci tot a?a faceau proorocilor mincino?i parin?ii lor.
Luk 6:27  Iar voua celor ce asculta?i va spun: Iubi?i pe vrajma?ii vo?tri, face?i bine celor ce va urasc pe voi;
Luk 6:28  Binecuvânta?i pe cei ce va blestema, ruga?i-va pentru cei ce va fac necazuri.
Luk 6:29  Celui ce te love?te peste obraz, întoarce-i ?i pe celalalt; pe cel ce-?i ia haina, nu-l împiedica sa-?i ia ?i cama?a;
Luk 6:30  Oricui î?i cere, da-i; ?i de la cel care ia lucrurile tale, nu cere înapoi.
Luk 6:31  ?i precum voi?i sa va faca voua oamenii, face?i-le ?i voi asemenea;
Luk 6:32  ?i daca iubi?i pe cei ce va iubesc, ce rasplata pute?i avea? Caci ?i pacato?ii iubesc pe cei ce îi iubesc pe ei.
Luk 6:33  ?i daca face?i bine celor ce va fac voua bine, ce mul?umire pute?i avea? Ca ?i pacato?ii acela?i lucru fac.
Luk 6:34  ?i daca da?i împrumut celor de la care nadajdui?i sa lua?i înapoi, ce mul?umire pute?i avea? Ca ?i pacato?ii dau cu împrumut pacato?ilor, ca sa primeasca înapoi întocmai.
Luk 6:35  Ci iubi?i pe vrajma?ii vo?tri ?i face?i bine ?i da?i cu împrumut, fara sa nadajdui?i nimic în schimb, ?i rasplata voastra va fi multa ?i ve?i fi fiii Celui Preaînalt, ca El este bun cu cei nemul?umitori ?i rai.
Luk 6:36  Fi?i milostivi, precum ?i Tatal vostru este milostiv.
Luk 6:37  Nu judeca?i ?i nu ve?i fi judeca?i; nu osândi?i ?i nu ve?i fi osândi?i; ierta?i ?i ve?i fi ierta?i.
Luk 6:38  Da?i ?i se va da. Turna-vor în sânul vostru o masura buna, îndesata, clatinata ?i cu vârf, caci cu ce masura ve?i masura, cu aceea?i vi se va masura.
Luk 6:39  ?i le-a spus ?i pilda: Poate orb pe orb sa calauzeasca? Nu vor cadea amândoi în groapa?
Luk 6:40  Nu este ucenic mai presus decât înva?atorul sau; dar orice ucenic desavâr?it va fi ca înva?atorul sau.
Luk 6:41  De ce vezi paiul din ochiul fratelui tau, iar bârna din ochiul tau nu o iei în seama?
Luk 6:42  Sau cum po?i sa zici fratelui tau: Frate, lasa sa scot paiul din ochiul tau, nevazând bârna care este în ochiul tau? Fa?arnice, scoate mai întâi bârna din ochiul tau ?i atunci vei vedea sa sco?i paiul din ochiul fratelui tau.
Luk 6:43  Caci nu este pom bun care sa faca roade rele ?i, iara?i, nici pom rau care sa faca roade bune.
Luk 6:44  Caci fiecare pom se cunoa?te dupa roadele lui. Ca nu se aduna smochine din maracini ?i nici nu se culeg struguri din spini.
Luk 6:45  Omul bun, din vistieria cea buna a inimii sale, scoate cele bune, pe când omul rau, din vistieria cea rea a inimii lui, scoate cele rele. Caci din prisosul inimii graie?te gura lui.
Luk 6:46  ?i pentru ce Ma chema?i: Doamne, Doamne, ?i nu face?i ce va spun?
Luk 6:47  Oricine vine la Mine ?i aude cuvintele Mele ?i le face, va voi arata cu cine se aseamana:
Luk 6:48  Asemenea este unui om care, zidindu-?i casa, a sapat, a adâncit ?i i-a pus temelia pe piatra, ?i venind apele mari ?i puhoiul izbind în casa aceea, n-a putut s-o clinteasca, fiindca era bine cladita pe piatra.
Luk 6:49  Iar cel ce aude, dar nu face, este asemenea omului care ?i-a zidit casa pe pamânt fara temelie, ?i izbind în ea puhoiul de ape, îndata a cazut ?i prabu?irea acelei case a fost mare.
Luk 7:1  ?i dupa ce a sfâr?it toate aceste cuvinte ale Sale în auzul poporului, a intrat în Capernaum.
Luk 7:2  Iar sluga unui suta?, care era la el în cinste, fiind bolnava era sa moara.
Luk 7:3  ?i auzind despre Iisus, a trimis la El batrâni ai iudeilor, rugându-L sa vina ?i sa vindece pe sluga lui.
Luk 7:4  Iar ei, venind la Iisus, L-au rugat staruitor, zicând: Vrednic este sa-i faci lui aceasta,
Luk 7:5  Caci iube?te neamul nostru ?i el ne-a zidit sinagoga.
Luk 7:6  Iar Iisus mergea cu ei. ?i nefiind El acum departe de casa, a trimis la El prieteni, zicându-I: Doamne, nu Te osteni, ca nu sunt vrednic ca sa intri sub acoperamântul meu.
Luk 7:7  De aceea nici pe mine nu m-am socotit vrednic sa vin la Tine. Ci spune cu cuvântul ?i se va vindeca sluga mea.
Luk 7:8  Caci ?i eu sunt om pus sub stapânire, având sub mine osta?i, ?i zic acestuia: Du-te, ?i se duce, ?i altuia: Vino, ?i vine, ?i slugii mele: Fa aceasta, ?i face.
Luk 7:9  Iar Iisus, auzind acestea, S-a minunat de el ?i, întorcându-Se, a zis mul?imii care venea dupa El: Zic voua ca nici în Israel n-am aflat atâta credin?a;
Luk 7:10  ?i întorcându-se cei trimi?i acasa, au gasit sluga sanatoasa.
Luk 7:11  ?i dupa aceea, S-a dus într-o cetate numita Nain ?i cu El împreuna mergeau ucenicii Lui ?i multa mul?ime.
Luk 7:12  Iar când S-a apropiat de poarta ceta?ii, iata scoteau un mort, singurul copil al mamei sale, ?i ea era vaduva, ?i mul?ime mare din cetate era cu ea.
Luk 7:13  ?i, vazând-o Domnul, I s-a facut mila de ea ?i i-a zis: Nu plânge!
Luk 7:14  ?i apropiindu-Se, S-a atins de sicriu, iar cei ce-l duceau s-au oprit. ?i a zis: Tinere, ?ie î?i zic, scoala-te.
Luk 7:15  ?i s-a ridicat mortul ?i a început sa vorbeasca, ?i l-a dat mamei lui.
Luk 7:16  ?i frica i-a cuprins pe to?i ?i slaveau pe Dumnezeu, zicând: Prooroc mare s-a ridicat între noi ?i Dumnezeu a cercetat pe poporul Sau.
Luk 7:17  ?i a ie?it cuvântul acesta despre El în toata Iudeea ?i în toata împrejurimea.
Luk 7:18  ?i au vestit lui Ioan ucenicii lui de toate acestea.
Luk 7:19  ?i chemând la sine pe doi dintre ucenicii sai, Ioan i-a trimis catre Domnul, zicând: Tu e?ti Cel ce va sa vina sau sa a?teptam pe altul?
Luk 7:20  ?i ajungând la El, barba?ii au zis: Ioan Botezatorul ne-a trimis la Tine, zicând: Tu e?ti Cel ce va sa vina sau sa a?teptam pe altul?
Luk 7:21  ?i în acel ceas El a vindecat pe mul?i de boli ?i de rani ?i de duhuri rele ?i multor orbi le-a daruit vederea.
Luk 7:22  ?i raspunzând, le-a zis: Merge?i ?i spune?i lui Ioan cele ce a?i vazut ?i cele ce a?i auzit: Orbii vad, ?chiopii umbla, lepro?ii se cura?esc, surzii aud, mor?ii înviaza ?i saracilor li se bineveste?te.
Luk 7:23  ?i fericit este acela care nu se va sminti întru Mine.
Luk 7:24  Iar, dupa ce trimi?ii lui Ioan au plecat, El a început sa vorbeasca mul?imilor despre Ioan: Ce a?i ie?it sa privi?i, în pustie? Oare trestie clatinata de vânt?
Luk 7:25  Dar ce a?i ie?it sa vede?i? Oare om îmbracat în haine moi? Iata, cei ce poarta haine scumpe ?i petrec în desfatare sunt în casele regilor.
Luk 7:26  Dar ce-a?i ie?it sa vede?i? Oare prooroc? Da! Zic voua: ?i mai mult decât un prooroc.
Luk 7:27  Acesta este cel despre care s-a scris: "Iata trimit înaintea fe?ei Tale pe îngerul Meu care va gati calea Ta, înaintea Ta".
Luk 7:28  Zic voua: Între cei nascu?i din femei, nimeni nu este mai mare decât Ioan; dar cel mai mic în împara?ia lui Dumnezeu este mai mare decât el.
Luk 7:29  ?i tot poporul auzind, ?i vame?ii s-au încredin?at de dreptatea lui Dumnezeu, botezându-se cu botezul lui Ioan.
Luk 7:30  Iar fariseii ?i înva?atorii de lege au calcat voia lui Dumnezeu în ei în?i?i, nebotezându-se de el.
Luk 7:31  Cu cine voi asemana pe oamenii acestui neam? ?i cu cine sunt ei asemenea?
Luk 7:32  Sunt asemenea copiilor care ?ed în pia?a ?i striga unii catre al?ii, zicând: V-am cântat din fluier ?i n-a?i jucat; v-am cântat de jale ?i n-a?i plâns.
Luk 7:33  Caci a venit Ioan Botezatorul, nemâncând pâine ?i negustând vin, ?i zice?i: Are demon!
Luk 7:34  A venit ?i Fiul Omului, mâncând ?i bând, ?i zice?i: Iata un om mâncacios ?i bautor de vin, prieten al vame?ilor ?i al pacato?ilor!
Luk 7:35  Dar În?elepciunea a fost gasita dreapta de catre to?i fiii ei.
Luk 7:36  Unul din farisei L-a rugat pe Iisus sa manânce cu el. ?i intrând în casa fariseului, a ?ezut la masa.
Luk 7:37  ?i iata era în cetate o femeie pacatoasa ?i, aflând ca ?ade la masa, în casa fariseului, a adus un alabastru cu mir.
Luk 7:38  ?i, stând la spate, lânga picioarele Lui, plângând, a început sa ude cu lacrimi picioarele Lui, ?i cu parul capului ei le ?tergea. ?i saruta picioarele Lui ?i le ungea cu mir.
Luk 7:39  ?i vazând, fariseul, care-L chemase, a zis în sine: Acesta, de-ar fi prooroc, ar ?ti cine e ?i ce fel e femeia care se atinge de El, ca este pacatoasa.
Luk 7:40  ?i raspunzând, Iisus a zis catre el: Simone, am sa-?i spun ceva. Înva?atorule, spune, zise el.
Luk 7:41  Un camatar avea doi datornici. Unul era dator cu cinci sute de dinari, iar celalalt cu cincizeci.
Luk 7:42  Dar, neavând ei cu ce sa plateasca, i-a iertat pe amândoi. Deci, care dintre ei îl va iubi mai mult?
Luk 7:43  Simon, raspunzând, a zis: Socotesc ca acela caruia i-a iertat mai mult. Iar El i-a zis: Drept ai judecat.
Luk 7:44  ?i întorcându-se catre femeie, a zis lui Simon: Vezi pe femeia aceasta? Am intrat în casa ta ?i apa pe picioare nu Mi-ai dat; ea însa, cu lacrimi, Mi-a udat picioarele ?i le-a ?ters cu parul ei.
Luk 7:45  Sarutare nu Mi-ai dat; ea însa de când am intrat, n-a încetat sa-Mi sarute picioarele.
Luk 7:46  Cu untdelemn capul Meu nu l-ai uns; ea însa cu mir Mi-a uns picioarele.
Luk 7:47  De aceea î?i zic: Iertate sunt pacatele ei cele multe, caci mult a iubit. Iar cui se iarta pu?in, pu?in iube?te.
Luk 7:48  ?i a zis ei: Iertate î?i sunt pacatele.
Luk 7:49  ?i au început cei ce ?edeau împreuna la masa sa zica în sine: Cine este Acesta care iarta ?i pacatele?
Luk 7:50  Iar catre femeie a zis: Credin?a ta te-a mântuit; mergi în pace.
Luk 8:1  ?i dupa aceea Iisus umbla prin ceta?i ?i prin sate, propovaduind ?i binevestind împara?ia lui Dumnezeu, ?i cei doisprezece erau cu El;
Luk 8:2  ?i unele femei care fusesera vindecate de duhuri rele ?i de boli: Maria, numita Magdalena, din care ie?isera ?apte demoni,
Luk 8:3  ?i Ioana, femeia lui Huza, un iconom al lui Irod, ?i Suzana ?i multe altele care le slujeau din avutul lor.
Luk 8:4  ?i adunându-se mul?ime multa ?i venind de prin ceta?i la El, a zis în pilda:
Luk 8:5  Ie?it-a semanatorul sa semene samân?a sa. ?i semanând el, una a cazut lânga drum ?i a fost calcata cu picioarele ?i pasarile cerului au mâncat-o.
Luk 8:6  ?i alta a cazut pe piatra, ?i, rasarind, s-a uscat, pentru ca nu avea umezeala.
Luk 8:7  ?i alta a cazut între spini ?i spinii, crescând cu ea, au înabu?it-o.
Luk 8:8  ?i alta a cazut pe pamântul cel bun ?i, crescând, a facut rod însutit. Acestea zicând, striga: Cine are urechi de auzit sa auda.
Luk 8:9  ?i ucenicii Lui Îl întrebau: Ce înseamna pilda aceasta?
Luk 8:10  El a zis: Voua va este dat sa cunoa?te?i tainele împara?iei lui Dumnezeu, iar celorlal?i în pilde, ca, vazând, sa nu vada ?i, auzind, sa nu în?eleaga.
Luk 8:11  Iar pilda aceasta înseamna: Samân?a este cuvântul lui Dumnezeu.
Luk 8:12  Iar cea de lânga drum sunt cei care aud, apoi vine diavolul ?i ia cuvântul din inima lor, ca nu cumva, crezând, sa se mântuiasca.
Luk 8:13  Iar cea de pe piatra sunt aceia care, auzind cuvântul îl primesc cu bucurie, dar ace?tia nu au radacina; ei cred pâna la o vreme, iar la vreme de încercare se leapada.
Luk 8:14  Cea cazuta între spini sunt cei ce aud cuvântul, dar umblând cu grijile ?i cu boga?ia ?i cu placerile vie?ii, se înabu?a ?i nu rodesc.
Luk 8:15  Iar cea de pe pamânt bun sunt cei ce, cu inima curata ?i buna, aud cuvântul, îl pastreaza ?i rodesc întru rabdare.
Luk 8:16  Nimeni, aprinzând faclia, n-o ascunde sub un vas, sau n-o pune sub pat, ci o a?eaza în sfe?nic, pentru ca cei ce intra sa vada lumina.
Luk 8:17  Caci nu este nimic ascuns, care sa nu se dea pe fa?a ?i nimic tainic, care sa nu se cunoasca ?i sa nu vina la aratare.
Luk 8:18  Lua?i seama deci cum auzi?i: Celui ce are i se va da; iar de la cel ce nu are, ?i ce i se pare ca are se va lua de la el.
Luk 8:19  ?i au venit la El mama Lui ?i fra?ii; dar nu puteau sa se apropie de El din pricina mul?imii.
Luk 8:20  ?i I s-a vestit: Mama Ta ?i fra?ii Tai stau afara ?i voiesc sa Te vada.
Luk 8:21  Iar El, raspunzând, a zis catre ei: Mama mea ?i fra?ii Mei sunt ace?tia care asculta cuvântul lui Dumnezeu ?i-l îndeplinesc.
Luk 8:22  ?i într-una din zile a intrat în corabie cu ucenicii Sai ?i a zis catre ei: Sa trecem de cealalta parte a lacului. ?i au plecat.
Luk 8:23  Dar, pe când ei vâsleau, El a adormit. ?i s-a lasat pe lac o furtuna de vânt, ?i corabia se umplea de apa ?i erau în primejdie.
Luk 8:24  ?i, apropiindu-se, L-au de?teptat, zicând: Înva?atorule, Înva?atorule, pierim. Iar El, sculându-Se, a certat vântul ?i valul apei ?i ele au încetat ?i s-a facut lini?te.
Luk 8:25  ?i le-a zis: Unde este credin?a voastra? Iar ei, temându-se, s-au mirat, zicând unii catre al?ii: Oare cine este Acesta, ca porunce?te ?i vânturilor ?i apei, ?i-L asculta?
Luk 8:26  ?i au ajuns cu corabia în ?inutul Gerghesenilor, care este în fa?a Galileii.
Luk 8:27  ?i ie?ind pe uscat, L-a întâmpinat un barbat din cetate, care avea demon ?i care de multa vreme nu mai punea haina pe el ?i în casa nu mai locuia, ci prin morminte.
Luk 8:28  ?i vazând pe Iisus, strigând, a cazut înaintea Lui ?i cu glas mare a zis: Ce ai cu mine, Iisuse, Fiul lui Dumnezeu Celui Preaînalt? Rogu-Te, nu ma chinui.
Luk 8:29  Caci poruncea duhului necurat sa iasa din om, pentru ca de mul?i ani îl stapânea, ?i era legat în lan?uri ?i în obezi, pazindu-l, dar el, sfarâmând legaturile, era mânat de demon, în pustie.
Luk 8:30  ?i l-a întrebat Iisus, zicând: Care-?i este numele? Iar el a zis: Legiune. Caci demoni mul?i intrasera în el.
Luk 8:31  ?i-L rugau pe El sa nu le porunceasca sa mearga în adânc.
Luk 8:32  ?i era acolo o turma mare de porci, care pa?teau pe munte. ?i L-au rugat sa le îngaduie sa intre în ei; ?i le-a îngaduit.
Luk 8:33  ?i, ie?ind demonii din om, au intrat în porci, iar turma s-a aruncat de pe ?arm în lac ?i s-a înecat.
Luk 8:34  Iar pazitorii vazând ce s-a întâmplat, au fugit ?i au vestit în cetate ?i prin sate.
Luk 8:35  ?i au ie?it sa vada ce s-a întâmplat ?i au venit la Iisus ?i au gasit pe omul din care ie?isera demonii, îmbracat ?i întreg la minte, ?ezând jos, la picioarele lui Iisus ?i s-au înfrico?at.
Luk 8:36  ?i cei ce vazusera le-au spus cum a fost izbavit demonizatul.
Luk 8:37  ?i L-a rugat pe El toata mul?imea din ?inutul Gerghesenilor sa plece de la ei, caci erau cuprin?i de frica mare. Iar El, intrând în corabie, S-a înapoiat.
Luk 8:38  Iar barbatul din care ie?isera demonii Îl ruga sa ramâna cu El. Iisus însa i-a dat drumul zicând:
Luk 8:39  Întoarce-te în casa ta ?i spune cât bine ?i-a facut ?ie Dumnezeu. ?i a plecat, vestind în toata cetatea câte îi facuse Iisus.
Luk 8:40  ?i când s-a întors Iisus, L-a primit mul?imea, caci to?i Îl a?teptau.
Luk 8:41  ?i iata a venit un barbat, al carui nume era Iair ?i care era mai-marele sinagogii. ?i cazând la picioarele lui Iisus, Îl ruga sa intre în casa Lui,
Luk 8:42  Caci avea numai o fiica, ca de doisprezece ani, ?i ea era pe moarte. ?i, pe când se ducea El, mul?imile Îl împresurau.
Luk 8:43  ?i o femeie, care de doisprezece ani avea scurgere de sânge ?i cheltuise cu doctorii toata averea ei, ?i de nici unul nu putuse sa fie vindecata,
Luk 8:44  Apropiindu-se pe la spate, s-a atins de poala hainei Lui ?i îndata s-a oprit curgerea sângelui ei.
Luk 8:45  ?i a zis Iisus: Cine este cel ce s-a atins de Mine? Dar to?i tagaduind, Petru ?i ceilal?i care erau cu El, au zis: Înva?atorule, mul?imile Te îmbulzesc ?i Te strâmtoreaza ?i Tu zici: Cine este cel ce s-a atins de mine?
Luk 8:46  Iar Iisus a zis: S-a atins de Mine cineva. Caci am sim?it o putere care a ie?it din Mine.
Luk 8:47  ?i, femeia, vazându-se vadita, a venit tremurând ?i, cazând înaintea Lui, a spus de fa?a cu tot poporul din ce cauza s-a atins de El ?i cum s-a tamaduit îndata.
Luk 8:48  Iar El i-a zis: Îndrazne?te, fiica, credin?a ta te-a mântuit. Mergi în pace.
Luk 8:49  ?i înca vorbind El, a venit cineva de la mai-marele sinagogii, zicând: A murit fiica ta. Nu mai supara pe Înva?atorul.
Luk 8:50  Dar Iisus, auzind, i-a raspuns: Nu te teme; crede numai ?i se va izbavi.
Luk 8:51  ?i venind în casa n-a lasat pe nimeni sa intre cu El, decât numai pe Petru ?i pe Ioan ?i pe Iacov ?i pe tatal copilei ?i pe mama.
Luk 8:52  ?i to?i plângeau ?i se tânguiau pentru ea. Iar El a zis: Nu plânge?i; n-a murit, ci doarme.
Luk 8:53  ?i râdeau de El, ?tiind ca a murit.
Luk 8:54  Iar El, sco?ând pe to?i afara ?i apucând-o de mâna, a strigat, zicând: Copila, scoala-te!
Luk 8:55  ?i duhul ei s-a întors ?i a înviat îndata; ?i a poruncit El sa i se dea sa manânce.
Luk 8:56  ?i au ramas uimi?i parin?ii ei. Iar El le-a poruncit sa nu spuna nimanui ce s-a întâmplat.
Luk 9:1  ?i chemând pe cei doisprezece ucenici ai Sai, le-a dat putere ?i stapânire peste to?i demonii ?i sa vindece bolile.
Luk 9:2  ?i i-a trimis sa propovaduiasca împara?ia lui Dumnezeu ?i sa vindece pe cei bolnavi.
Luk 9:3  ?i a zis catre ei: Sa nu lua?i nimic pe drum, nici toiag, nici traista, nici pâine, nici bani ?i nici sa nu ave?i câte doua haine.
Luk 9:4  ?i în orice casa ve?i intra, acolo sa ramâne?i ?i de acolo sa pleca?i.
Luk 9:5  ?i câ?i nu va vor primi, ie?ind din acea cetate scutura?i praful de pe picioarele voastre, spre marturie împotriva lor.
Luk 9:6  Iar ei, plecând, mergeau prin sate binevestind ?i vindecând pretutindeni.
Luk 9:7  ?i a auzit Irod tetrarhul toate cele facute ?i era nedumerit, ca se zicea de catre unii ca Ioan s-a sculat din mor?i;
Luk 9:8  Iar de unii ca Ilie s-a aratat, iar de al?ii, ca un prooroc dintre cei vechi a înviat.
Luk 9:9  Iar Irod a zis: Lui Ioan eu i-am taiat capul. Cine este dar Acesta despre care aud asemenea lucruri? ?i cauta sa-l vada.
Luk 9:10  ?i, întorcându-se apostolii, I-au spus toate câte au facut. ?i, luându-i cu Sine, S-a dus de o parte într-un loc pustiu, aproape de cetatea numita Betsaida.
Luk 9:11  Iar mul?imile, aflând, au mers dupa El ?i El, primindu-le, le vorbea despre împara?ia lui Dumnezeu, iar pe cei care aveau trebuin?a de vindecare îi facea sanato?i.
Luk 9:12  Dar ziua a început sa se plece spre seara. ?i, venind la El, cei doisprezece I-au spus: Da drumul mul?imii sa se duca prin satele ?i prin satule?ele dimprejur, ca sa poposeasca ?i sa-?i gaseasca mâncare, ca aici suntem în loc pustiu.
Luk 9:13  Iar El a zis catre ei: Da?i-le voi sa manânce. Iar ei au zis: Nu avem mai mult de cinci pâini ?i doi pe?ti, afara numai daca, ducându-ne noi, vom cumpara merinde pentru tot poporul acesta.
Luk 9:14  Caci erau ca la cinci mii de barba?i. Dar El a zis catre ucenicii Sai: A?eza?i-i jos, în cete de câte cincizeci de in?i.
Luk 9:15  ?i au facut a?a ?i i-au a?ezat pe to?i.
Luk 9:16  Iar Iisus, luând cele cinci pâini ?i cei doi pe?ti ?i privind la cer, le-a binecuvântat, a frânt ?i a dat ucenicilor, ca sa puna mul?imii înainte.
Luk 9:17  ?i au mâncat ?i s-au saturat to?i ?i au luat ceea ce le-a ramas, douasprezece co?uri de farâmituri.
Luk 9:18  ?i când Se ruga El singur, erau cu El ucenicii, ?i i-a întrebat, zicând: Cine zic mul?imile ca sunt Eu?
Luk 9:19  Iar ei, raspunzând, au zis: Ioan Botezatorul, iar al?ii Ilie, iar al?ii ca a înviat un prooroc din cei vechi.
Luk 9:20  ?i El le-a zis: Dar voi cine zice?i ca sunt Eu? Iar Petru, raspunzând, a zis: Hristosul lui Dumnezeu.
Luk 9:21  Iar El, certându-i, le-a poruncit sa nu spuna nimanui aceasta,
Luk 9:22  Zicând ca Fiul Omului trebuie sa patimeasca multe ?i sa fie defaimat de catre batrâni ?i de catre arhierei ?i de catre carturari ?i sa fie omorât, iar a treia zi sa învieze.
Luk 9:23  ?i zicea catre to?i: Daca voie?te cineva sa vina dupa Mine, sa se lepede de sine, sa-?i ia crucea în fiecare zi ?i sa-Mi urmeze Mie;
Luk 9:24  Caci cine va voi sa-?i scape sufletul îl va pierde; iar cine-?i va pierde sufletul pentru Mine, acela îl va mântui.
Luk 9:25  Ca ce folose?te omului daca va câ?tiga lumea toata, iar pe sine se va pierde sau se va pagubi?
Luk 9:26  Caci de cel ce se va ru?ina de Mine ?i de cuvintele Mele, de acesta ?i Fiul Omului se va ru?ina, când va veni întru slava Sa ?i a Tatalui ?i a sfin?ilor îngeri.
Luk 9:27  Cu adevarat însa spun voua: Sunt unii, dintre cei ce stau aici, care nu vor gusta moartea, pâna ce nu vor vedea împara?ia lui Dumnezeu.
Luk 9:28  Iar dupa cuvintele acestea, ca la opt zile, luând cu Sine pe Petru ?i pe Ioan ?i pe Iacov, S-a suit pe munte ca sa Se roage.
Luk 9:29  ?i pe când se ruga El, chipul fe?ei Sale s-a facut altul ?i îmbracamintea Lui alba stralucind.
Luk 9:30  ?i iata doi barba?i vorbeau cu El, care erau Moise ?i Ilie,
Luk 9:31  ?i care, aratându-se întru slava, vorbeau despre sfâr?itul Lui, pe care avea sa-l împlineasca în Ierusalim.
Luk 9:32  Iar Petru ?i cei ce erau cu el erau îngreuia?i de somn; ?i de?teptându-se, au vazut slava Lui ?i pe cei doi barba?i stând cu El.
Luk 9:33  ?i când s-au despar?it ei de El, Petru a zis catre Iisus: Înva?atorule, bine este ca noi sa fim aici ?i sa facem trei colibe: una ?ie, una lui Moise ?i una lui Ilie, ne?tiind ce spune.
Luk 9:34  ?i, pe când vorbea el acestea, s-a facut un nor ?i i-a umbrit; ?i ei s-au spaimântat când au intrat în nor.
Luk 9:35  ?i glas s-a facut din nor, zicând: Acesta este Fiul Meu cel ales, de acesta sa asculta?i!
Luk 9:36  ?i când a trecut glasul, S-a aflat Iisus singur. ?i ei au tacut ?i nimanui n-au spus nimic, în zilele acelea, din cele ce vazusera.
Luk 9:37  În ziua urmatoare, când s-au coborât din munte, L-a întâmpinat mul?ime multa.
Luk 9:38  ?i iata un barbat din mul?ime a strigat, zicând: Înva?atorule, rogu-ma ?ie, cauta spre fiul meu, ca îl am numai pe el;
Luk 9:39  ?i iata un duh îl apuca ?i îndata striga ?i-l zguduie cu spume ?i abia pleaca de la el, dupa ce l-a zdrobit.
Luk 9:40  ?i m-am rugat de ucenicii Tai ca sa-l alunge, ?i n-au putut.
Luk 9:41  Iar Iisus, raspunzând, a zis: O, neam necredincios ?i îndaratnic! Pâna când voi fi cu voi ?i va voi suferi? Adu aici pe fiul tau.
Luk 9:42  ?i, apropiindu-se el, demonul l-a aruncat la pamânt ?i l-a zguduit. Iar Iisus a certat pe duhul cel necurat ?i a vindecat pe copil ?i l-a dat tatalui lui.
Luk 9:43  Iar to?i au ramas uimi?i de marirea lui Dumnezeu. ?i mirându-se to?i de toate câte facea, a zis catre ucenicii Sai:
Luk 9:44  Pune?i în urechile voastre cuvintele acestea: Caci Fiul Omului va fi dat în mâinile oamenilor.
Luk 9:45  Iar ei nu în?elegeau cuvântul acesta, caci era ascuns pentru ei ca sa nu-l priceapa ?i se temeau sa-L întrebe despre acest cuvânt.
Luk 9:46  ?i a intrat gând în inima lor: Cine dintre ei ar fi mai mare?
Luk 9:47  Iar Iisus, cunoscând cugetul inimii lor, a luat un copil, l-a pus lânga Sine,
Luk 9:48  ?i le-a zis: Oricine va primi pruncul acesta, în numele Meu, pe Mine Ma prime?te; iar oricine Ma va primi pe Mine, prime?te pe Cel ce M-a trimis pe Mine. Caci cel ce este mai mic între voi to?i, acesta este mare.
Luk 9:49  Iar Ioan, raspunzând, a zis: Înva?atorule, am vazut pe unul care, în numele Tau, scoate demoni ?i l-am oprit, pentru ca nu-?i urmeaza împreuna cu noi.
Luk 9:50  Iar Iisus a zis catre el: Nu-l opri?i; caci cine nu este împotriva voastra este pentru voi.
Luk 9:51  ?i când s-au împlinit zilele înal?arii Sale, El S-a hotarât sa mearga la Ierusalim.
Luk 9:52  ?i a trimis vestitori înaintea Lui. ?i ei, mergând, au intrat într-un sat de samarineni, ca sa faca pregatiri pentru El.
Luk 9:53  Dar ei nu L-au primit, pentru ca El se îndrepta spre Ierusalim.
Luk 9:54  ?i vazând aceasta, ucenicii Iacov ?i Ioan I-au zis: Doamne, vrei sa zicem sa se coboare foc din cer ?i sa-i mistuie, cum a facut ?i Ilie?
Luk 9:55  Iar El, întorcându-Se, i-a certat ?i le-a zis: Nu ?ti?i, oare, fiii carui duh sunte?i? Caci Fiul Omului n-a venit ca sa piarda sufletele oamenilor, ci ca sa le mântuiasca.
Luk 9:56  ?i s-au dus în alt sat.
Luk 9:57  ?i pe când mergeau ei pe cale, zis-a unul catre El: Te voi înso?i, oriunde Te vei duce.
Luk 9:58  ?i i-a zis Iisus: Vulpile au vizuini ?i pasarile cerului cuiburi; dar Fiul Omului n-are unde sa-?i plece capul.
Luk 9:59  ?i a zis catre altul: urmeaza-Mi. Iar el a zis: Doamne, da-mi voie întâi sa merg sa îngrop pe tatal meu.
Luk 9:60  Iar El i-a zis: Lasa mor?ii sa-?i îngroape mor?ii lor, iar tu mergi de veste?te împara?ia lui Dumnezeu.
Luk 9:61  Dar altul a zis: Î?i voi urma, Doamne, dar întâi îngaduie-mi ca sa rânduiesc cele din casa mea.
Luk 9:62  Iar Iisus a zis catre el: Nimeni care pune mâna pe plug ?i se uita îndarat nu este potrivit pentru împara?ia lui Dumnezeu.
Luk 10:1  Iar dupa acestea, Domnul a ales al?i ?aptezeci (?i doi) ?i i-a trimis câte doi înaintea fe?ei Sale, în fiecare cetate ?i loc, unde Însu?i avea sa vina.
Luk 10:2  ?i zicea catre ei: Seceri?ul este mult, dar lucratorii sunt pu?ini; ruga?i deci pe Domnul seceri?ului, ca sa scoata lucratori la seceri?ul Sau.
Luk 10:3  Merge?i; iata, Eu va trimit ca pe ni?te miei în mijlocul lupilor.
Luk 10:4  Nu purta?i punga, nici traista, nici încal?aminte; ?i pe nimeni sa nu saluta?i pe cale.
Luk 10:5  Iar în orice casa ve?i intra, întâi zice?i: Pace casei acesteia.
Luk 10:6  ?i de va fi acolo un fiu al pacii, pacea voastra se va odihni peste el, iar de nu, se va întoarce la voi.
Luk 10:7  ?i în aceasta casa ramâne?i, mâncând ?i bând cele ce va vor da, caci vrednic este lucratorul de plata sa. Nu va muta?i din casa în casa.
Luk 10:8  ?i în orice cetate ve?i intra ?i va vor primi, mânca?i cele ce va vor pune înainte.
Luk 10:9  ?i vindeca?i pe bolnavii din ea ?i zice?i-le: S-a apropiat de voi împara?ia lui Dumnezeu.
Luk 10:10  ?i în orice cetate ve?i intra ?i nu va vor primi, ie?ind în pie?ele ei, zice?i:
Luk 10:11  ?i praful care s-a lipit de picioarele noastre din cetatea noastra vi-l scuturam voua. Dar aceasta sa ?ti?i, ca s-a apropiat împara?ia lui Dumnezeu.
Luk 10:12  Zic voua: Ca mai u?or va fi Sodomei în ziua aceea, decât ceta?ii aceleia.
Luk 10:13  Vai ?ie, Horazine! Vai ?ie, Betsaido! Caci daca în Tir ?i în Sidon s-ar fi facut minunile care s-au facut la voi, de mult s-ar fi pocait, stând în sac ?i în cenu?a.
Luk 10:14  Dar Tirului ?i Sidonului mai u?or le va fi la judecata, decât voua.
Luk 10:15  ?i tu, Capernaume, nu ai fost înal?at, oare, pâna la cer? Pâna la iad vei fi coborât!
Luk 10:16  Cel ce va asculta pe voi pe Mine Ma asculta, ?i cel ce se leapada de voi se leapada de Mine; iar cine se leapada de Mine se leapada de Cel ce M-a trimis pe Mine.
Luk 10:17  ?i s-au întors cei ?aptezeci (?i doi) cu bucurie, zicând: Doamne, ?i demonii ni se supun în numele Tau.
Luk 10:18  ?i le-a zis: Am vazut pe satana ca un fulger cazând din cer.
Luk 10:19  Iata, v-am dat putere sa calca?i peste ?erpi ?i peste scorpii, ?i peste toata puterea vrajma?ului, ?i nimic nu va va vatama.
Luk 10:20  Dar nu va bucura?i de aceasta, ca duhurile vi se pleaca, ci va bucura?i ca numele voastre sunt scrise în ceruri.
Luk 10:21  În acesta ceas, El S-a bucurat în Duhul Sfânt ?i a zis: Te slavesc pe Tine, Parinte, Doamne al cerului ?i al pamântului, ca ai ascuns acestea de cei în?elep?i ?i de cei pricepu?i ?i le-ai descoperit pruncilor. A?a, Parinte, caci a?a a fost înaintea Ta, bunavoin?a Ta.
Luk 10:22  Toate Mi-au fost date de catre Tatal Meu ?i nimeni nu cunoa?te cine este Fiul, decât numai Tatal, ?i cine este Tatal, decât numai Fiul ?i caruia voie?te Fiul sa-i descopere.
Luk 10:23  ?i întorcându-Se catre ucenici, de o parte a zis: Ferici?i sunt ochii care vad cele ce vede?i voi!
Luk 10:24  Caci zic voua: Mul?i prooroci ?i regi au voit sa vada ceea ce vede?i voi, dar n-au vazut, ?i sa auda ceea ce auzi?i, dar n-au auzit.
Luk 10:25  ?i iata, un înva?ator de lege s-a ridicat, ispitindu-L ?i zicând: Înva?atorule, ce sa fac ca sa mo?tenesc via?a de veci?
Luk 10:26  Iar Iisus a zis catre el: Ce este scris în Lege? Cum cite?ti?
Luk 10:27  Iar el, raspunzând, a zis: Sa iube?ti pe Domnul Dumnezeul tau din toata inima ta ?i din tot sufletul tau ?i din toata puterea ta ?i din tot cugetul tau, iar pe aproapele tau ca pe tine însu?i.
Luk 10:28  Iar El i-a zis: Drept ai raspuns, fa aceasta ?i vei trai.
Luk 10:29  Dar el, voind sa se îndrepteze pe sine, a zis catre Iisus: ?i cine este aproapele meu?
Luk 10:30  Iar Iisus, raspunzând, a zis: Un om cobora de la Ierusalim la Ierihon, ?i a cazut între tâlhari, care, dupa ce l-au dezbracat ?i l-au ranit, au plecat, lasându-l aproape mort.
Luk 10:31  Din întâmplare un preot cobora pe calea aceea ?i, vazându-l, a trecut pe alaturi.
Luk 10:32  De asemenea ?i un levit, ajungând în acel loc ?i vazând, a trecut pe alaturi.
Luk 10:33  Iar un samarinean, mergând pe cale, a venit la el ?i, vazându-l, i s-a facut mila,
Luk 10:34  ?i, apropiindu-se, i-a legat ranile, turnând pe ele untdelemn ?i vin, ?i, punându-l pe dobitocul sau, l-a dus la o casa de oaspe?i ?i a purtat grija de el.
Luk 10:35  Iar a doua zi, sco?ând doi dinari i-a dat gazdei ?i i-a zis: Ai grija de el ?i, ce vei mai cheltui, eu, când ma voi întoarce, î?i voi da.
Luk 10:36  Care din ace?ti trei ?i se pare ca a fost aproapele celui cazut între tâlhari?
Luk 10:37  Iar el a zis: Cel care a facut mila cu el. ?i Iisus i-a zis: Mergi ?i fa ?i tu asemenea.
Luk 10:38  ?i pe când mergeau ei, El a intrat într-un sat, iar o femeie, cu numele Marta, L-a primit în casa ei.
Luk 10:39  ?i ea avea o sora ce se numea Maria, care, a?ezându-se la picioarele Domnului, asculta cuvântul Lui.
Luk 10:40  Iar Marta se silea cu multa slujire ?i, apropiindu-se, a zis: Doamne, au nu socote?ti ca sora mea m-a lasat singura sa slujesc? Spune-i deci sa-mi ajute.
Luk 10:41  ?i, raspunzând, Domnul i-a zis: Marto, Marto, te îngrije?ti ?i pentru multe te sile?ti;
Luk 10:42  Dar un lucru trebuie: caci Maria partea buna ?i-a ales, care nu se va lua de la ea.
Luk 11:1  ?i pe când Se ruga Iisus într-un loc, când a încetat, unul dintre ucenicii Lui I-a zis: Doamne, înva?a-ne sa ne rugam, cum a înva?at ?i Ioan pe ucenicii lui.
Luk 11:2  ?i le-a zis: Când va ruga?i, zice?i: Tatal nostru, Care e?ti în ceruri, sfin?easca-se numele Tau. Vie împara?ia Ta. Faca-se voia Ta, precum în cer a?a ?i pe pamânt.
Luk 11:3  Pâinea noastra cea spre fiin?a, da-ne-o noua în fiecare zi.
Luk 11:4  ?i ne iarta noua pacatele noastre, caci ?i noi în?ine iertam tuturor celor ce ne gre?esc noua. ?i nu ne duce pe noi în ispita, ci ne izbave?te de cel rau.
Luk 11:5  ?i a zis catre ei: Cine dintre voi, având un prieten ?i se va duce la el în miez de noapte ?i-i va zice: Prietene, împrumuta-mi trei pâini,
Luk 11:6  Ca a venit, din cale, un prieten la mine ?i n-am ce sa-i pun înainte,
Luk 11:7  Iar acela, raspunzând dinauntru, sa-i zica: Nu ma da de osteneala. Acum u?a e încuiata ?i copiii mei sunt în pat cu mine. Nu pot sa ma scol sa-?i dau.
Luk 11:8  Zic voua: Chiar daca, sculându-se, nu i-ar da pentru ca-i este prieten, dar, pentru îndrazneala lui, sculându-se, îi va da cât îi trebuie.
Luk 11:9  ?i Eu zic voua: Cere?i ?i vi se va da; cauta?i ?i ve?i afla; bate?i ?i vi se va deschide.
Luk 11:10  Ca oricine cere ia; ?i cel ce cauta gase?te, ?i celui ce bate i se va deschide.
Luk 11:11  ?i care tata dintre voi, daca îi va cere fiul pâine, oare, îi va da piatra? Sau daca îi va cere pe?te, oare îi va da, în loc de pe?te, ?arpe?
Luk 11:12  Sau daca-i va cere un ou, îi va da scorpie?
Luk 11:13  Deci daca voi, rai fiind, ?ti?i sa da?i fiilor vo?tri daruri bune, cu cât mai mult Tatal vostru Cel din ceruri va da Duh Sfânt celor care îl cer de la El!
Luk 11:14  ?i a scos un demon, ?i acela era mut. ?i când a ie?it demonul, mutul a vorbit, iar mul?imile s-au minunat.
Luk 11:15  Iar unii dintre ei au zis: Cu Beelzebul, capetenia demonilor, scoate pe demoni.
Luk 11:16  Iar al?ii, ispitindu-L, cereau de la El semn din cer.
Luk 11:17  Dar El, cunoscând gândurile lor, le-a zis: Orice împara?ie, dezbinându-se în sine, se pustie?te ?i casa peste casa cade.
Luk 11:18  ?i daca satana s-a dezbinat în sine, cum va mai sta împara?ia lui? Fiindca zice?i ca Eu scot pe demoni cu Beelzebul.
Luk 11:19  Iar daca Eu scot demonii cu Beelzebul, fiii vo?tri cu cine îi scot? De aceea ei va vor fi judecatori.
Luk 11:20  Iar daca Eu, cu degetul lui Dumnezeu, scot pe demoni iata a ajuns la voi împara?ia lui Dumnezeu.
Luk 11:21  Când cel tare ?i înarmat fiind î?i paze?te curtea, avu?iile lui sunt în pace.
Luk 11:22  Dar când unul mai tare decât el vine asupra lui ?i-l înfrânge, îi ia toate armele pe care se bizuia, iar prazile de la el le împarte.
Luk 11:23  Cel ce nu este cu Mine este împotriva Mea; ?i cel ce nu aduna cu Mine risipe?te.
Luk 11:24  Când duhul cel necurat iese din om, umbla prin locuri fara apa, cautând odihna, ?i, negasind, zice: Ma voi întoarce la casa mea, de unde am ie?it.
Luk 11:25  ?i, venind, o afla maturata ?i împodobita.
Luk 11:26  Atunci merge ?i ia cu el alte ?apte duhuri mai rele decât el ?i, intrând, locuie?te acolo; ?i se fac cele de pe urma ale omului aceluia mai rele decât cele dintâi.
Luk 11:27  ?i când zicea El acestea, o femeie din mul?ime, ridicând glasul, I-a zis: Fericit este pântecele care Te-a purtat ?i ferici?i sunt sânii pe care i-ai supt!
Luk 11:28  Iar El a zis: A?a este, dar ferici?i sunt cei ce asculta cuvântul lui Dumnezeu ?i-l pazesc.
Luk 11:29  Iar îngramadindu-se mul?imile, El a început a zice: Neamul acesta este un neam viclean; cere semn dar semn nu i se va da decât semnul proorocului Iona.
Luk 11:30  Caci precum a fost Iona un semn pentru Niniviteni a?a va fi ?i Fiul Omului semn pentru acest neam.
Luk 11:31  Regina de la miazazi se va ridica la judecata cu barba?ii neamului acestuia ?i-i va osândi, pentru ca a venit de la marginile pamântului, ca sa asculte în?elepciunea lui Solomon; ?i, iata, mai mult decât Solomon este aici.
Luk 11:32  Barba?ii din Ninive se vor scula la judecata cu neamul acesta ?i-l vor osândi, pentru ca s-au pocait la propovaduirea lui Iona; ?i, iata, mai mult decât Iona este aici.
Luk 11:33  Nimeni, aprinzând faclie, nu o pune în loc ascuns, nici sub obroc, ci în sfe?nic, ca aceia care intra sa vada lumina.
Luk 11:34  Luminatorul trupului este ochiul tau. Când ochiul tau este curat, atunci tot trupul tau e luminat; dar când ochiul tau e rau, atunci ?i trupul tau e întunecat.
Luk 11:35  Ia seama deci ca lumina din tine sa nu fie întuneric.
Luk 11:36  A?adar, daca tot trupul tau e luminat, neavând nici o parte întunecata, luminat va fi în întregime, ca ?i când te lumineaza faclia cu stralucirea ei.
Luk 11:37  ?i pe când Iisus vorbea, un fariseu Îl ruga sa prânzeasca la el; ?i, intrând, a ?ezut la masa.
Luk 11:38  Iar fariseul s-a mirat vazând ca El nu S-a spalat înainte de masa.
Luk 11:39  ?i Domnul a zis catre el: Acum, voi fariseilor, cura?i?i partea din afara a paharului ?i a blidului, dar launtrul vostru este plin de rapire ?i de viclenie.
Luk 11:40  Nebunilor! Oare, cel ce a facut partea din afara n-a facut ?i partea dinauntru?
Luk 11:41  Da?i mai întâi milostenie cele ce sunt înlauntrul vostru ?i, iata, toate va vor fi curate.
Luk 11:42  Dar vai voua, fariseilor! Ca da?i zeciuiala din izma ?i din untari?a ?i din toate legumele ?i lasa?i la o parte dreptatea ?i iubirea de Dumnezeu; pe acestea se cuvenea sa le face?i ?i pe acelea sa nu le lasa?i.
Luk 11:43  Vai voua, fariseilor! Ca iubi?i scaunele din fa?a în sinagogi ?i în închinaciunile din pie?e.
Luk 11:44  Vai voua, carturarilor ?i fariseilor fa?arnici! Ca sunte?i ca mormintele ce nu se vad, ?i oamenii, care umbla peste ele, nu le ?tiu.
Luk 11:45  ?i raspunzând, unul dintre înva?atorii de Lege I-a zis: Înva?atorule, acestea zicând, ne mustri ?i pe noi!
Luk 11:46  Iar El a zis: Vai ?i voua, înva?atorilor de Lege! Ca împovara?i pe oameni cu sarcini anevoie de purtat, iar voi nu atinge?i sarcinile nici cel pu?in cu un deget.
Luk 11:47  Vai voua! Ca zidi?i mormintele proorocilor pe care parin?ii vo?tri i-au ucis.
Luk 11:48  A?adar, marturisi?i ?i încuviin?a?i faptele parin?ilor vo?tri, pentru ca ei i-au ucis, iar voi le cladi?i mormintele.
Luk 11:49  De aceea ?i în?elepciunea lui Dumnezeu a zis: "Voi trimite la ei prooroci ?i apostoli ?i dintre ei vor ucide ?i vor prigoni";
Luk 11:50  Ca sa se ceara de la neamul acesta sângele tuturor proorocilor, care s-a varsat de la facerea lumii,
Luk 11:51  De la sângele lui Abel pâna la sângele lui Zaharia, care a pierit între altar ?i templu. Adevarat va spun: Se va cere de la neamul acesta.
Luk 11:52  Vai voua, înva?atorilor de Lege! Ca a?i luat cheia cuno?tin?ei; voi în?iva  n-a?i intrat, iar pe cei ce voiau sa intre i-a?i împiedecat.
Luk 11:53  Iar ie?ind El de acolo, carturarii ?i fariseii au început sa-L urasca groaznic ?i sa-L sileasca sa vorbeasca despre multe,
Luk 11:54  Pândindu-L ?i cautând sa prinda ceva din gura Lui, ca sa-I gaseasca vina.
Luk 12:1  ?i în acela?i timp, adunându-se mul?ime mii ?i mii de oameni, încât se calcau unii pe al?ii, Iisus a început sa vorbeasca întâi catre ucenicii Sai: Feri?i-va de aluatul fariseilor, care este fa?arnicia.
Luk 12:2  Ca nimic nu este acoperit care sa nu se descopere ?i nimic ascuns care sa nu se cunoasca.
Luk 12:3  De aceea, câte a?i spus la întuneric se vor auzi la lumina; ?i ceea ce a?i vorbit la ureche, în odai, se va vesti de pe acoperi?uri.
Luk 12:4  Dar va spun voua, prietenii Mei: Nu va teme?i de cei care ucid trupul ?i dupa aceasta n-au ce sa mai faca.
Luk 12:5  Va voi arata însa de cine sa va teme?i: Teme?i-va de acela care, dupa ce a ucis, are putere sa arunce în gheena; da, va zic voua, de acela sa va teme?i.
Luk 12:6  Nu se vând oare cinci vrabii cu doi bani? ?i nici una dintre ele nu este uitata înaintea lui Dumnezeu.
Luk 12:7  Ci ?i perii capului vostru, to?i sunt numara?i. Nu va teme?i; voi sunte?i mai de pre? decât multe vrabii.
Luk 12:8  ?i zic voua: Oricine va marturisi pentru Mine înaintea oamenilor, ?i Fiul Omului va marturisi pentru el înaintea îngerilor lui Dumnezeu.
Luk 12:9  Iar cel ce se va lepada de Mine înaintea oamenilor, lepadat va fi înaintea îngerilor lui Dumnezeu.
Luk 12:10  Oricui va spune vreun cuvânt împotriva Fiului Omului, i se va ierta; dar celui ce va huli împotriva Duhului Sfânt, nu i se va ierta.
Luk 12:11  Iar când va vor duce în sinagogi ?i la dregatori ?i la stapâniri nu va îngriji?i cum sau ce ve?i raspunde, sau ce ve?i zice,
Luk 12:12  Ca Duhul Sfânt va va înva?a chiar în ceasul acela, ce trebuie sa spune?i.
Luk 12:13  Zis-a Lui cineva din mul?ime: Înva?atorule, zi fratelui meu sa împarta cu mine mo?tenirea.
Luk 12:14  Iar El i-a zis: Omule, cine M-a pus pe Mine judecator sau împar?itor peste voi?
Luk 12:15  ?i a zis catre ei: Vede?i ?i pazi?i-va de toata lacomia, caci via?a cuiva nu sta în prisosul avu?iilor sale.
Luk 12:16  ?i le-a spus lor aceasta pilda, zicând: Unui om bogat i-a rodit din bel?ug ?arina.
Luk 12:17  ?i el cugeta în sine, zicând: Ce voi face, ca n-am unde sa adun roadele mele?
Luk 12:18  ?i a zis: Aceasta voi face: Voi strica jitni?ele mele ?i mai mari le voi zidi ?i voi strânge acolo tot grâul ?i bunata?ile mele;
Luk 12:19  ?i voi zice sufletului meu: Suflete, ai multe bunata?i strânse pentru mul?i ani; odihne?te-te, manânca, bea, vesele?te-te.
Luk 12:20  Iar Dumnezeu i-a zis: Nebune! În aceasta noapte vor cere de la tine sufletul tau. ?i cele ce ai pregatit ale cui vor fi?
Luk 12:21  A?a se întâmpla cu cel ce-?i aduna comori sie?i ?i nu se îmboga?e?te în Dumnezeu.
Luk 12:22  ?i a zis catre ucenicii Sai: De aceea zic voua: Nu va îngriji?i pentru via?a voastra ce ve?i mânca, nici pentru trupul vostru cu ce va ve?i îmbraca.
Luk 12:23  Via?a este mai mult decât hrana ?i trupul mai mult decât îmbracamintea.
Luk 12:24  Privi?i la corbi, ca nici nu seamana, nici nu secera; ei n-au camara, nici jitni?a, ?i Dumnezeu îi hrane?te. Cu cât mai de pre? sunte?i voi decât pasarile!
Luk 12:25  ?i cine dintre voi, îngrijindu-se, poate sa adauge staturii sale un cot?
Luk 12:26  Deci daca nu pute?i sa face?i nici cel mai mic lucru, de ce va îngriji?i de celelalte?
Luk 12:27  Privi?i la crini cum cresc: Nu torc, nici nu ?es. ?i zic voua ca nici Solomon, în toata marirea lui, nu s-a îmbracat ca unul dintre ace?tia.
Luk 12:28  Iar daca iarba care este azi pe câmp, iar mâine se arunca în cuptor, Dumnezeu a?a o îmbraca, cu cât mai mult pe voi, pu?in credincio?ilor!
Luk 12:29  ?i voi sa nu cauta?i ce ve?i mânca sau ce ve?i bea ?i nu fi?i îngrijora?i.
Luk 12:30  Caci toate acestea pagânii lumii le cauta; dar Tatal vostru ?tie ca ave?i nevoie de acestea;
Luk 12:31  Cauta?i mai întâi împara?ia Lui. ?i toate acestea se vor adauga voua.
Luk 12:32  Nu te teme, turma mica, pentru ca Tatal vostru a binevoit sa va dea voua împara?ia.
Luk 12:33  Vinde?i averile voastre ?i da?i milostenie; face?i-va pungi care nu se învechesc, comoara neîmpu?inata în ceruri, unde fur nu se apropie, nici molie nu o strica.
Luk 12:34  Caci unde este comoara voastra, acolo este inima voastra.
Luk 12:35  Sa fie mijloacele voastre încinse ?i facliile voastre aprinse.
Luk 12:36  ?i voi fi?i asemenea oamenilor care a?teapta pe stapânul lor când se întoarce de la nunta, ca, venind, ?i batând, îndata sa-i deschida.
Luk 12:37  Fericite sunt slugile acelea pe care, venind, stapânul le va afla priveghind. Adevarat zic voua ca se va încinge ?i le va pune la masa ?i, apropiindu-se le va sluji.
Luk 12:38  Fie ca va veni la straja a doua, fie ca va veni la straja a treia, ?i le va gasi a?a, fericite sunt acelea.
Luk 12:39  Iar aceasta sa ?ti?i ca, de ar ?ti stapânul casei în care ceas vine furul, ar veghea ?i n-ar lasa sa i se sparga casa.
Luk 12:40  Deci ?i voi fi?i gata, ca în ceasul în care nu gândi?i Fiul Omului va veni.
Luk 12:41  ?i a zis Petru: Doamne, catre noi spui pilda aceasta sau ?i catre to?i?
Luk 12:42  ?i a zis Domnul: Cine este iconomul credincios ?i în?elept pe care stapânul îl va pune peste slugile sale, ca sa le dea, la vreme, partea lor de grâu?
Luk 12:43  Fericita este sluga aceea pe care, venind stapânul, o va gasi facând a?a.
Luk 12:44  Adevarat va spun ca o va pune peste toate avu?iile sale.
Luk 12:45  Iar de va zice sluga aceea în inima sa: Stapânul meu zabove?te sa vina, ?i va începe sa bata pe slugi ?i pe slujnice, ?i sa manânce, ?i sa bea ?i sa se îmbete,
Luk 12:46  Veni-va stapânul slugii aceleia în ziua în care ea nu se a?teapta ?i în ceasul în care ea nu ?tie ?i o va taia în doua, iar partea ei va pune-o cu cei necredincio?i.
Luk 12:47  Iar sluga aceea care a ?tiut voia stapânului ?i nu s-a pregatit, nici n-a facut dupa voia lui, va fi batuta mult.
Luk 12:48  ?i cea care n-a ?tiut, dar a facut lucruri vrednice de bataie, va fi batuta pu?in. ?i oricui i s-a dat mult, mult i se va cere, ?i cui i s-a încredin?at mult, mai mult i se va cere.
Luk 12:49  Foc am venit sa arunc pe pamânt ?i cât a? vrea sa fie acum aprins!
Luk 12:50  ?i cu botez am a Ma boteza, ?i câta nerabdare am pâna ce se va îndeplini!
Luk 12:51  Vi se pare ca am venit sa dau pace pe pamânt? Va spun ca nu, ci dezbinare.
Luk 12:52  Caci de acum înainte cinci dintr-o casa vor fi dezbina?i: trei împotriva a doi ?i doi împotriva a trei.
Luk 12:53  Dezbina?i vor fi: tatal împotriva fiului ?i fiul împotriva tatalui, mama împotriva fiicei ?i fiica împotriva mamei, soacra împotriva nurorii sale ?i nora împotriva soacrei.
Luk 12:54  ?i zicea mul?imilor: Când vede?i un nor ridicându-se dinspre apus, îndata zice?i ca vine ploaie mare; ?i a?a este.
Luk 12:55  Iar când sufla vântul de la miazazi, zice?i ca va fi ar?i?a, ?i a?a este.
Luk 12:56  Fa?arnicilor! Fa?a pamântului ?i a cerului ?ti?i sa o deosebi?i, dar vremea aceasta cum de nu o deosebi?i?
Luk 12:57  De ce, dar, de la voi în?iva nu judeca?i ce este drept?
Luk 12:58  ?i când mergi cu pârâ?ul tau la dregator, da-?i silin?a sa te scapi de el pe cale, ca nu cumva sa te târasca la judecator, ?i judecatorul sa te dea în mâna temnicerului, iar temnicerul sa te arunce în temni?a.
Luk 12:59  Zic ?ie: Nu vei ie?i de acolo, pâna ce nu vei plati ?i cel din urma ban.
Luk 13:1  ?i erau de fa?a în acel timp unii care-I vesteau despre galileienii al caror sânge Pilat l-a amestecat cu jertfele lor.
Luk 13:2  ?i El, raspunzând, le-a zis: Crede?i, oare, ca ace?ti galileieni au fost ei mai  pacato?i decât to?i galileienii, fiindca au suferit aceasta?
Luk 13:3  Nu! zic voua; dar daca nu va ve?i pocai, to?i ve?i pieri la fel.
Luk 13:4  Sau acei optsprezece in?i, peste care s-a surpat turnul în Siloam ?i i-a ucis, gândi?i, oare, ca ei au fost mai pacato?i decât to?i oamenii care locuiau în Ierusalim?
Luk 13:5  Nu! zic voua; dar de nu va ve?i pocai, to?i ve?i pieri la fel.
Luk 13:6  ?i le-a spus pilda aceasta: Cineva avea un smochin, sadit în via sa ?i a venit sa caute rod în el, dar n-a gasit.
Luk 13:7  ?i a zis catre vier: Iata trei ani sunt de când vin ?i caut rod în smochinul acesta ?i nu gasesc. Taie-l; de ce sa ocupe locul în zadar?
Luk 13:8  Iar el, raspunzând, a zis: Doamne, lasa-l ?i anul acesta, pâna ce îl voi sapa împrejur ?i voi pune gunoi.
Luk 13:9  Poate va face rod în viitor; iar de nu, îl vei taia.
Luk 13:10  ?i înva?a Iisus într-una din sinagogi sâmbata.
Luk 13:11  ?i iata o femeie care avea de optsprezece ani un duh de neputin?a ?i care era gârbova, de nu putea sa se ridice în sus nicidecum;
Luk 13:12  Iar Iisus, vazând-o, a chemat-o ?i i-a zis: Femeie, e?ti dezlegata de neputin?a ta.
Luk 13:13  ?i ?i-a pus mâinile asupra ei, ?i ea îndata s-a îndreptat ?i slavea pe Dumnezeu.
Luk 13:14  Iar mai-marele sinagogii, mâniindu-se ca Iisus a vindecat-o sâmbata, raspunzând, zicea mul?imii: ?ase zile sunt în care trebuie sa se lucreze; venind deci într-acestea, vindeca?i-va, dar nu în ziua sâmbetei!
Luk 13:15  Iar Domnul i-a raspuns ?i a zis: Fa?arnicilor! Fiecare dintre voi nu dezleaga, oare, sâmbata boul sau, sau asinul de la iesle, ?i nu-l duce sa-l adape?
Luk 13:16  Dar aceasta, fiica a lui Avraam fiind, pe care a legat-o satana, iata de optsprezece ani, nu se cuvenea, oare, sa fie dezlegata de legatura aceasta, în ziua sâmbetei?
Luk 13:17  ?i zicând El acestea, s-au ru?inat to?i care erau împotriva Lui, ?i toata mul?imea se bucura de faptele stralucite savâr?ite de El.
Luk 13:18  Deci zicea: Cu ce este asemenea împara?ia lui Dumnezeu ?i cu ce o voi asemana?
Luk 13:19  Asemenea este grauntelui de mu?tar pe care, luându-l, un om l-a aruncat în gradina sa, ?i a crescut ?i s-a facut copac, iar pasarile cerului s-au sala?luit în ramurile lui.
Luk 13:20  ?i iara?i a zis: Cu ce voi asemana împara?ia lui Dumnezeu?
Luk 13:21  Asemenea este aluatului pe care, luându-l, femeia l-a ascuns în trei masuri de faina, pâna ce s-a dospit totul.
Luk 13:22  ?i mergea El prin ceta?i ?i prin sate, înva?ând ?i calatorind spre Ierusalim.
Luk 13:23  ?i I-a zis cineva: Doamne, pu?ini sunt, oare, cei ce se mântuiesc? Iar El le-a zis:
Luk 13:24  Sili?i-va sa intra?i prin poarta cea strâmta, ca mul?i, zic voua, vor cauta sa intre ?i nu vor putea.
Luk 13:25  Dupa ce se va scula stapânul casei ?i va încuia u?a ?i ve?i începe sa sta?i afara ?i sa bate?i la u?a, zicând: Doamne, deschide-ne! - ?i el, raspunzând, va va zice: Nu va ?tiu de unde sunte?i,
Luk 13:26  Atunci voi ve?i începe sa zice?i: Am mâncat înaintea ta ?i am baut ?i în pie?ele noastre ai înva?at.
Luk 13:27  ?i el va va zice: Va spun: Nu ?tiu de unde sunte?i. Departa?i-va de la mine to?i lucratorii nedrepta?ii.
Luk 13:28  Acolo va fi plângerea ?i scrâ?nirea din?ilor, când ve?i vedea pe Avraam ?i pe Isaac ?i pe Iacov ?i pe to?i proorocii în Împara?ia lui Dumnezeu, iar pe voi arunca?i afara.
Luk 13:29  ?i vor veni al?ii de la rasarit ?i de la apus, de la miazanoapte ?i de la miazazi ?i vor ?edea la masa în împara?ia lui Dumnezeu.
Luk 13:30  ?i iata, sunt unii de pe urma care vor fi întâi, ?i sunt al?ii întâi care vor fi pe urma.
Luk 13:31  În ceasul acela au venit la El unii din farisei, zicându-I: Ie?i ?i du-Te de aici, ca Irod vrea sa Te ucida.
Luk 13:32  ?i El le-a zis: Mergând, spune?i vulpii acesteia: Iata, alung demoni ?i fac vindecari, astazi ?i mâine, iar a treia zi voi sfâr?i.
Luk 13:33  Însa ?i astazi ?i mâine ?i în ziua urmatoare merg, fiindca nu este cu putin?a sa piara prooroc afara din Ierusalim.
Luk 13:34  Ierusalime, Ierusalime, care omori pe prooroci ?i ucizi cu pietre pe cei trimi?i la tine, de câte ori am voit sa adun pe fiii tai, cum aduna pasarea puii sai sub aripi, dar n-a?i voit.
Luk 13:35  Iata vi se lasa casa voastra pustie, ca adevarat graiesc voua. Nu Ma ve?i mai vedea pâna ce va veni vremea când ve?i zice: Binecuvântat este Cel ce vine întru numele Domnului!
Luk 14:1  ?i când a intrat El în casa unuia dintre capeteniile fariseilor sâmbata, ca sa manânce, ?i ei Îl pândeau,
Luk 14:2  Iata un om bolnav de idropica era înaintea Lui.
Luk 14:3  ?i, raspunzând, Iisus a zis catre înva?atorii de lege ?i catre farisei, spunând: Cuvine-se a vindeca sâmbata ori nu?
Luk 14:4  Ei însa au tacut. ?i luându-l, l-a vindecat ?i i-a dat drumul.
Luk 14:5  ?i catre ei a zis: Care dintre voi, de-i cadea fiul sau boul în fântâna nu-l va scoate îndata în ziua sâmbetei?
Luk 14:6  ?i n-au putut sa-i raspunda la acestea.
Luk 14:7  ?i luând seama cum î?i alegeau la masa cele dintâi locuri, a spus celor chema?i o pilda, zicând între ei:
Luk 14:8  Când vei fi chemat de cineva la nunta, nu te a?eza în locul cel dintâi, ca nu cumva sa fie chemat de el altul mai de cinste decât tine.
Luk 14:9  ?i venind cel care te-a chemat pe tine ?i pe el, î?i va zice: Da acestuia locul. ?i atunci, cu ru?ine, te vei duce sa te a?ezi pe locul cel mai de pe urma.
Luk 14:10  Ci, când vei fi chemat, mergând a?eaza-te în cel din urma loc, ca atunci când va veni cel ce te-a chemat, el sa-?i zica: Prietene, muta-te mai sus. Atunci vei avea cinstea în fa?a tuturor celor care vor ?edea împreuna cu tine.
Luk 14:11  Caci, oricine se înal?a pe sine se va smeri, iar cel ce se smere?te pe sine se va înal?a.
Luk 14:12  Zis-a ?i celui ce-L chemase: Când faci prânz sau cina, nu chema pe prietenii tai, nici pe fra?ii tai, nici pe rudele tale, nici vecinii boga?i, ca nu cumva sa te cheme ?i ei, la rândul lor, pe tine, ?i sa-?i fie ca rasplata.
Luk 14:13  Ci, când faci un ospa?, cheama pe saraci, pe neputincio?i, pe ?chiopi, pe orbi,
Luk 14:14  ?i fericit vei fi ca nu pot sa-?i rasplateasca. Caci ?i se va rasplati la învierea drep?ilor.
Luk 14:15  ?i auzind acestea, unul dintre cei ce ?edeau cu El la masa I-a zis: Fericit este cel ce va prânzi în împara?ia lui Dumnezeu!
Luk 14:16  Iar El i-a zis: Un om oarecare a facut cina mare ?i a chemat pe mul?i;
Luk 14:17  ?i a trimis la ceasul cinei pe sluga sa ca sa spuna celor chema?i: Veni?i, ca iata toate sunt gata.
Luk 14:18  ?i au început unul câte unul, sa-?i ceara iertare. Cel dintâi i-a zis: ?arina am cumparat ?i am nevoie sa ies ca s-o vad; te rog iarta-ma.
Luk 14:19  ?i altul a zis: Cinci perechi de boi am cumparat ?i ma duc sa-i încerc; te rog iarta-ma.
Luk 14:20  Al treilea a zis: Femeie mi-am luat ?i de aceea nu pot veni.
Luk 14:21  ?i întorcându-se, sluga a spus stapânului sau acestea. Atunci, mâniindu-se, stapânul casei a zis: Ie?i îndata în pie?ele ?i uli?ele ceta?ii, ?i pe saraci, ?i pe neputincio?i, ?i pe orbi, ?i pe ?chiopi adu-i aici.
Luk 14:22  ?i a zis sluga: Doamne, s-a facut precum ai poruncit ?i tot mai este loc.
Luk 14:23  ?i a zis stapânul catre sluga: Ie?i la drumuri ?i la garduri ?i sile?te sa intre, ca sa mi se umple casa,
Luk 14:24  Caci zic voua: Nici unul din barba?ii aceia care au fost chema?i nu va gusta din cina mea.
Luk 14:25  ?i mergeau cu El mul?imi multe; ?i întorcându-Se, a zis catre ele:
Luk 14:26  Daca vine cineva la Mine ?i nu ura?te pe tatal sau ?i pe mama ?i pe femeie ?i pe copii ?i pe fra?i ?i pe surori, chiar ?i sufletul sau însu?i, nu poate sa fie ucenicul Meu.
Luk 14:27  ?i cel ce nu-?i poarta crucea sa ?i nu vine dupa Mine nu poate sa fie ucenicul Meu.
Luk 14:28  Ca cine dintre voi vrând sa zideasca un turn nu sta mai întâi ?i-?i face socoteala cheltuielii, daca are cu ce sa-l ispraveasca?
Luk 14:29  Ca nu cumva, punându-i temelia ?i neputând sa-l termine, to?i cei care vor vedea sa înceapa a-l lua în râs,
Luk 14:30  Zicând: Acest om a început sa zideasca, dar n-a putut ispravi.
Luk 14:31  Sau care rege, plecând sa se bata în razboi cu alt rege, nu va sta întâi sa se sfatuiasca, daca va putea sa întâmpine cu zece mii pe cel care vine împotriva lui cu douazeci de mii?
Luk 14:32  Iar de nu, înca fiind el departe, îi trimite solie ?i se roaga de pace.
Luk 14:33  A?adar oricine dintre voi care nu se leapada de tot ce are nu poate sa fie ucenicul Meu.
Luk 14:34  Buna este sarea, dar daca ?i sarea se va strica, cu ce va fi dreasa?
Luk 14:35  Nici în pamânt, nici în gunoi, nu este de folos, ci o arunca afara. Cine are urechi de auzit sa auda.
Luk 15:1  ?i se apropiau de El to?i vame?ii ?i pacato?ii, ca sa-L asculte.
Luk 15:2  ?i fariseii ?i carturarii cârteau, zicând: Acesta prime?te la Sine pe pacato?i ?i manânca cu ei.
Luk 15:3  ?i a zis catre ei pilda aceasta, spunând:
Luk 15:4  Care om dintre voi, având o suta de oi ?i pierzând din ele una, nu lasa pe cele nouazeci ?i noua în pustie ?i se duce dupa cea pierduta, pâna ce o gase?te?
Luk 15:5  ?i gasind-o, o pune pe umerii sai, bucurându-se;
Luk 15:6  ?i sosind acasa, cheama prietenii ?i vecinii, zicându-le: Bucura?i-va cu mine, ca am gasit oaia cea pierduta.
Luk 15:7  Zic voua: Ca a?a ?i în cer va fi mai multa bucurie pentru un pacatos care se pocaie?te, decât pentru nouazeci ?i noua de drep?i, care n-au nevoie de pocain?a.
Luk 15:8  Sau care femeie, având zece drahme, daca pierde o drahma, nu aprinde lumina ?i nu matura casa ?i nu cauta cu grija pâna ce o gase?te?
Luk 15:9  ?i gasind-o, cheama prietenele ?i vecinele sale, spunându-le: Bucura?i-va cu mine, caci am gasit drahma pe care o pierdusem.
Luk 15:10  Zic voua, a?a se face bucurie îngerilor lui Dumnezeu pentru un pacatos care se pocaie?te.
Luk 15:11  ?i a zis: Un om avea doi fii.
Luk 15:12  ?i a zis cel mai tânar dintre ei tatalui sau: Tata, da-mi partea ce mi se cuvine din avere. ?i el le-a împar?it averea.
Luk 15:13  ?i nu dupa multe zile, adunând toate, fiul cel mai tânar s-a dus într-o ?ara departata ?i acolo ?i-a risipit averea, traind în desfrânari.
Luk 15:14  ?i dupa ce a cheltuit totul, s-a facut foamete mare în ?ara aceea, ?i el a început sa duca lipsa.
Luk 15:15  ?i ducându-se, s-a alipit el de unul din locuitorii acelei ?ari, ?i acesta l-a trimis la ?arinile sale sa pazeasca porcii.
Luk 15:16  ?i dorea sa-?i sature pântecele din ro?covele pe care le mâncau porcii, însa nimeni nu-i dadea.
Luk 15:17  Dar, venindu-?i în sine, a zis: Câ?i arga?i ai tatalui meu sunt îndestula?i de pâine, iar eu pier aici de foame!
Luk 15:18  Sculându-ma, ma voi duce la tatal meu ?i-i voi spune: Tata, am gre?it la cer ?i înaintea ta;
Luk 15:19  Nu mai sunt vrednic sa ma numesc fiul tau. Fa-ma ca pe unul din arga?ii tai.
Luk 15:20  ?i, sculându-se, a venit la tatal sau. ?i înca departe fiind el, l-a vazut tatal sau ?i i s-a facut mila ?i, alergând, a cazut pe grumazul lui ?i l-a sarutat.
Luk 15:21  ?i i-a zis fiul: Tata, am gre?it la cer ?i înaintea ta ?i nu mai sunt vrednic sa ma numesc fiul tau.
Luk 15:22  ?i a zis tatal catre slugile sale: Aduce?i degraba haina lui cea dintâi ?i-l îmbraca?i ?i da?i inel în mâna lui ?i încal?aminte în picioarele lui;
Luk 15:23  ?i aduce?i vi?elul cel îngra?at ?i-l înjunghia?i ?i, mâncând, sa ne veselim;
Luk 15:24  Caci acest fiu al meu mort era ?i a înviat, pierdut era ?i s-a aflat. ?i au început sa se veseleasca.
Luk 15:25  Iar fiul cel mare era la ?arina. ?i când a venit ?i s-a apropiat de casa, a auzit cântece ?i jocuri.
Luk 15:26  ?i, chemând la sine pe una dintre slugi, a întrebat ce înseamna acestea.
Luk 15:27  Iar ea i-a raspuns: Fratele tau a venit, ?i tatal tau a înjunghiat vi?elul cel îngra?at, pentru ca l-a primit sanatos.
Luk 15:28  ?i el s-a mâniat ?i nu voia sa intre; dar tatal lui, ie?ind, îl ruga.
Luk 15:29  Însa el, raspunzând, a zis tatalui sau: Iata, atâ?ia ani î?i slujesc ?i niciodata n-am calcat porunca ta. ?i mie niciodata nu mi-ai dat un ied, ca sa ma veselesc cu prietenii mei.
Luk 15:30  Dar când a venit acest fiu al tau, care ?i-a mâncat averea cu desfrânatele, ai înjunghiat pentru el vi?elul cel îngra?at.
Luk 15:31  Tatal însa i-a zis: Fiule, tu totdeauna e?ti cu mine ?i toate ale mele ale tale sunt.
Luk 15:32  Trebuia însa sa ne veselim ?i sa ne bucuram, caci fratele tau acesta mort era ?i a înviat, pierdut era ?i s-a aflat.
Luk 16:1  ?i zicea ?i catre ucenicii Sai: Era un om bogat care avea un iconom ?i acesta a fost pârât lui ca-i risipe?te avu?iile.
Luk 16:2  ?i chemându-l, i-a zis: Ce este aceasta ce aud despre tine? Da-mi socoteala de iconomia ta, caci nu mai po?i sa fii iconom.
Luk 16:3  Iar iconomul a zis în sine: Ce voi face ca stapânul meu ia iconomia de la mine? Sa sap, nu pot; sa cer?esc, mi-e ru?ine.
Luk 16:4  ?tiu ce voi face, ca sa ma primeasca în casele lor, când voi fi scos din iconomie.
Luk 16:5  ?i chemând la sine, unul câte unul, pe datornicii stapânului sau, a zis celui dintâi: Cât e?ti dator stapânului meu?
Luk 16:6  Iar el a zis: O suta de masuri de untdelemn. Iconomul i-a zis: Ia-?i zapisul ?i, ?ezând, scrie degraba cincizeci.
Luk 16:7  Dupa aceea a zis altuia: Dar tu, cât e?ti dator? El i-a spus: O suta de masuri de grâu. Zis-a iconomul: Ia-?i zapisul ?i scrie optzeci.
Luk 16:8  ?i a laudat stapânul pe iconomul cel nedrept, caci a lucrat în?elep?e?te. Caci fiii veacului acestuia sunt mai în?elep?i în neamul lor decât fiii luminii.
Luk 16:9  ?i Eu zic voua: Face?i-va prieteni cu boga?ia nedreapta, ca atunci, când ve?i parasi via?a, sa va primeasca ei în corturile cele ve?nice.
Luk 16:10  Cel ce este credincios în foarte pu?in ?i în mult este credincios; ?i cel ce e nedrept în foarte pu?in ?i în mult este nedrept.
Luk 16:11  Deci daca n-a?i fost credincio?i în boga?ia nedreapta, cine va va încredin?a pe cea adevarata?
Luk 16:12  ?i daca în ceea ce este strain nu a?i fost credincio?i, cine va va da ce este al vostru?
Luk 16:13  Nici o sluga nu poate sa slujeasca la doi stapâni. Fiindca sau pe unul îl va urî ?i pe celalalt îl va iubi, sau de unul se va ?ine ?i pe celalalt îl va dispre?ui. Nu pute?i sa sluji?i lui Dumnezeu ?i lui mamona.
Luk 16:14  Toate acestea le auzeau ?i fariseii, care erau iubitori de argint ?i-L luau în bataie de joc.
Luk 16:15  ?i El le-a zis: Voi sunte?i cei ce va face?i pe voi drep?i înaintea oamenilor, dar Dumnezeu cunoa?te inimile voastre; caci ceea ce la oameni este înalt, urâciune este înaintea lui Dumnezeu.
Luk 16:16  Legea ?i proorocii au fost pâna la Ioan; de atunci împara?ia lui Dumnezeu se bineveste?te ?i fiecare se sile?te spre ea.
Luk 16:17  Dar mai lesne e sa treaca cerul ?i pamântul, decât sa cada din Lege un corn de litera.
Luk 16:18  Oricine-?i lasa femeia sa ?i ia pe alta savâr?e?te adulter; ?i cel ce ia pe cea lasata de barbat savâr?e?te adulter.
Luk 16:19  Era un om bogat care se îmbraca în porfira ?i în vison, veselindu-se în toate zilele în chip stralucit.
Luk 16:20  Iar un sarac, anume Lazar, zacea înaintea por?ii lui, plin de bube,
Luk 16:21  Poftind sa se sature din cele ce cadeau de la masa bogatului; dar ?i câinii venind, lingeau bubele lui.
Luk 16:22  ?i a murit saracul ?i a fost dus de catre îngeri în sânul lui Avraam. A murit ?i bogatul ?i a fost înmormântat.
Luk 16:23  ?i în iad, ridicându-?i ochii, fiind în chinuri, el a vazut de departe pe Avraam ?i pe Lazar în sânul lui.
Luk 16:24  ?i el, strigând, a zis: Parinte Avraame, fie-?i mila de mine ?i trimite pe Lazar sa-?i ude vârful degetului în apa ?i sa-mi racoreasca limba, caci ma chinuiesc în aceasta vapaie.
Luk 16:25  Dar Avraam a zis: Fiule, adu-?i aminte ca ai primit cele bune ale tale în via?a ta, ?i Lazar, asemenea, pe cele rele; iar acum aici el se mângâie, iar tu te chinuie?ti.
Luk 16:26  ?i peste toate acestea, între noi ?i voi s-a întarit prapastie mare, ca cei care voiesc sa treaca de aici la voi sa nu poata, nici cei de acolo sa treaca la noi.
Luk 16:27  Iar el a zis: Rogu-te, dar, parinte, sa-l trimi?i în casa tatalui meu,
Luk 16:28  Caci am cinci fra?i, sa le spuna lor acestea, ca sa nu vina ?i ei în acest loc de chin.
Luk 16:29  ?i i-a zis Avraam: Au pe Moise ?i pe prooroci; sa asculte de ei.
Luk 16:30  Iar el a zis: Nu, parinte Avraam, ci, daca cineva dintre mor?i se va duce la ei, se vor pocai.
Luk 16:31  ?i i-a zis Avraam: Daca nu asculta de Moise ?i de prooroci, nu vor crede nici daca ar învia cineva dintre mor?i.
Luk 17:1  ?i a zis catre ucenicii Sai: Cu neputin?a este sa nu vina smintelile, dar vai aceluia prin care ele vin!
Luk 17:2  Mai de folos i-ar fi daca i s-ar lega de gât o piatra de moara ?i ar fi aruncat în mare, decât sa sminteasca pe unul din ace?tia mici.
Luk 17:3  Lua?i aminte la voi în?iva. De-?i va gre?i fratele tau, dojene?te-l ?i daca se va pocai, iarta-l.
Luk 17:4  ?i chiar daca î?i va gre?i de ?apte ori într-o zi ?i de ?apte ori se va întoarce catre tine, zicând: Ma caiesc, iarta-l.
Luk 17:5  ?i au zis apostolii catre Domnul: Spore?te-ne credin?a.
Luk 17:6  Iar Domnul a zis: De a?i avea credin?a cât un graunte de mu?tar, a?i zice acestui sicomor: Dezradacineaza-te ?i te sade?te în mare, ?i va va asculta.
Luk 17:7  Cine dintre voi, având o sluga la arat sau la pascut turme, îi va zice când se întoarce din ?arina: Vino îndata ?i ?ezi la masa?
Luk 17:8  Oare, nu-i va zice: Pregate?te-mi ca sa cinez ?i, încingându-te, sluje?te-mi, pâna ce voi mânca ?i voi bea ?i dupa aceea vei mânca ?i vei bea ?i tu?
Luk 17:9  Va mul?umi, oare, slugii ca a facut cele poruncite? Cred ca nu.
Luk 17:10  A?a ?i voi, când ve?i face toate cele poruncite voua, sa zice?i: Suntem slugi netrebnice, pentru ca am facut ceea ce eram datori sa facem.
Luk 17:11  Iar pe când Iisus mergea spre Ierusalim ?i trecea prin mijlocul Samariei ?i al Galileii,
Luk 17:12  Intrând într-un sat, L-au întâmpinat zece lepro?i care stateau departe,
Luk 17:13  ?i care au ridicat glasul ?i au zis: Iisuse, Înva?atorule, fie-?i mila de noi!
Luk 17:14  ?i vazându-i, El le-a zis: Duce?i-va ?i va arata?i preo?ilor. Dar, pe când ei se duceau, s-au cura?it.
Luk 17:15  Iar unul dintre ei, vazând ca s-a vindecat, s-a întors cu glas mare slavind pe Dumnezeu.
Luk 17:16  ?i a cazut cu fa?a la pamânt la picioarele lui Iisus, mul?umindu-I. ?i acela era samarinean.
Luk 17:17  ?i raspunzând, Iisus a zis: Au nu zece s-au cura?it? Dar cei noua unde sunt?
Luk 17:18  Nu s-a gasit sa se întoarca sa dea slava lui Dumnezeu decât numai acesta, care este de alt neam?
Luk 17:19  ?i i-a zis: Scoala-te ?i du-te; credin?a ta te-a mântuit.
Luk 17:20  ?i fiind întrebat de farisei când va veni împara?ia lui Dumnezeu, le-a raspuns ?i a zis: Împara?ia lui Dumnezeu nu va veni în chip vazut.
Luk 17:21  ?i nici nu vor zice: Iat-o aici sau acolo. Caci, iata, împara?ia lui Dumnezeu este înauntrul vostru.
Luk 17:22  Zis-a catre ucenici: Veni-vor zile când ve?i dori sa vede?i una din zilele Fiului Omului, ?i nu ve?i vedea.
Luk 17:23  ?i vor zice voua: Iata este acolo; iata, aici; nu va duce?i ?i nu va lua?i dupa ei.
Luk 17:24  Caci dupa cum fulgerul, fulgerând dintr-o parte de sub cer, lumineaza pâna la cealalta parte de sub cer, a?a va fi ?i Fiul Omului în ziua Sa.
Luk 17:25  Dar mai întâi El trebuie sa sufere multe ?i sa fie lepadat de neamul acesta.
Luk 17:26  ?i precum a fost în zilele lui Noe, tot a?a va fi ?i în zilele Fiului Omului:
Luk 17:27  Mâncau, beau, se însurau, se maritau pâna în ziua când a intrat Noe în corabie ?i a venit potopul ?i i-a nimicit pe to?i.
Luk 17:28  Tot a?a precum a fost în zilele lui Lot: mâncau, beau, cumparau, vindeau, sadeau, ?i zideau,
Luk 17:29  Iar în ziua în care a ie?it Lot din Sodoma a plouat din cer foc ?i pucioasa ?i i-a nimicit pe to?i,
Luk 17:30  La fel va fi în ziua în care se va arata Fiul Omului.
Luk 17:31  În ziua aceea, cel care va fi pe acoperi?ul casei, ?i lucrurile lui în casa, sa nu se coboare ca sa le ia; de asemenea, cel ce va fi în ?arina sa nu se întoarca înapoi.
Luk 17:32  Aduce?i-va aminte de femeia lui Lot.
Luk 17:33  Cine va cauta sa-?i scape sufletul, îl va pierde; iar cine îl va pierde, acela îl va dobândi.
Luk 17:34  Zic voua: În noaptea aceea vor fi doi într-un pat; unul va fi luat, iar celalalt va fi lasat.
Luk 17:35  Doua vor macina împreuna; una va fi luata ?i alta va fi lasata.
Luk 17:36  Doi vor fi în ogor; unul se va lua altul se va lasa.
Luk 17:37  ?i raspunzând, ucenicii I-au zis: Unde, Doamne? Iar El le-a zis: Unde va fi stârvul, acolo se vor aduna vulturii.
Luk 18:1  ?i le spunea o pilda cum trebuie sa se roage totdeauna ?i sa nu-?i piarda nadejdea,
Luk 18:2  Zicând: Într-o cetate era un judecator care de Dumnezeu nu se temea ?i de om nu se ru?ina.
Luk 18:3  ?i era, în cetatea aceea, o vaduva, care venea la el, zicând: Fa-mi dreptate fa?a de potrivnicul meu.
Luk 18:4  ?i un timp n-a voit, dar dupa acestea a zis întru sine: De?i de Dumnezeu nu ma tem ?i de om nu ma ru?inez,
Luk 18:5  Totu?i, fiindca vaduva aceasta îmi face suparare, îi voi face dreptate, ca sa nu vina mereu sa ma supere.
Luk 18:6  ?i a zis Domnul: Auzi?i ce spune judecatorul cel nedrept?
Luk 18:7  Dar Dumnezeu, oare, nu va face dreptate ale?ilor Sai care striga catre El ziua ?i noaptea ?i pentru care El rabda îndelung?
Luk 18:8  Zic voua ca le va face dreptate în curând. Dar Fiul Omului, când va veni, va gasi, oare, credin?a pe pamânt?
Luk 18:9  Catre unii care se credeau ca sunt drep?i ?i priveau cu dispre? pe ceilal?i, a zis pilda aceasta:
Luk 18:10  Doi oameni s-au suit la templu, ca sa se roage: unul fariseu ?i celalalt vame?.
Luk 18:11  Fariseul, stând, a?a se ruga în sine: Dumnezeule, Î?i mul?umesc ca nu sunt ca ceilal?i oameni, rapitori, nedrep?i, adulteri, sau ca ?i acest vame?.
Luk 18:12  Postesc de doua ori pe saptamâna, dau zeciuiala din toate câte câ?tig.
Luk 18:13  Iar vame?ul, departe stând, nu voia nici ochii sa-?i ridice catre cer, ci-?i batea pieptul, zicând: Dumnezeule, fii milostiv mie, pacatosului.
Luk 18:14  Zic voua ca acesta s-a coborât mai îndreptat la casa sa, decât acela. Fiindca oricine se înal?a pe sine se va smeri, iar cel ce se smere?te pe sine se va înal?a.
Luk 18:15  ?i aduceau la El ?i pruncii, ca sa Se atinga de ei. Iar ucenicii, vazând, îi certau.
Luk 18:16  Iar Iisus i-a chemat la Sine, zicând: Lasa?i copii sa vina la Mine ?i nu-i opri?i, caci împara?ia lui Dumnezeu este a unora ca ace?tia.
Luk 18:17  Adevarat graiesc voua: Cine nu va primi împara?ia lui Dumnezeu ca un prunc nu va intra în ea.
Luk 18:18  ?i L-a întrebat un dregator, zicând: Bunule Înva?ator, ce sa fac ca sa mo?tenesc via?a de veci?
Luk 18:19  Iar Iisus i-a zis: Pentru ce Ma nume?ti bun? Nimeni nu este bun, decât unul Dumnezeu.
Luk 18:20  ?tii poruncile: Sa nu savâr?e?ti adulter, sa nu ucizi, sa nu furi, sa nu marturise?ti strâmb, cinste?te pe tatal tau ?i pe mama ta.
Luk 18:21  Iar el a zis: Toate acestea le-am pazit din tinere?ile mele.
Luk 18:22  Auzind Iisus i-a zis: Înca una î?i lipse?te: Vinde toate câte ai ?i le împarte saracilor ?i vei avea comoara în ceruri; ?i vino de urmeaza Mie.
Luk 18:23  Iar el, auzind acestea, s-a întristat, caci era foarte bogat.
Luk 18:24  ?i vazându-l întristat, Iisus a zis: Cât de greu vor intra cei ce au averi în împara?ia lui Dumnezeu!
Luk 18:25  Ca mai lesne este a trece camila prin urechile acului decât sa intre bogatul în împara?ia lui Dumnezeu.
Luk 18:26  Zis-au cei ce ascultau: ?i cine poate sa se mântuiasca?
Luk 18:27  Iar El a zis: Cele ce sunt cu neputin?a la oameni sunt cu putin?a la Dumnezeu.
Luk 18:28  Iar Petru a zis: Iata, noi, lasând toate ale noastre, am urmat ?ie.
Luk 18:29  ?i El le-a zis: Adevarat graiesc voua: Nu este nici unul care a lasat casa, sau femeie, sau fra?i, sau parin?i, sau copii, pentru împara?ia lui Dumnezeu,
Luk 18:30  ?i sa nu ia cu mult mai mult în vremea aceasta, iar în veacul ce va sa vina, via?a ve?nica.
Luk 18:31  ?i luând la Sine pe cei doisprezece, a zis catre ei: Iata ne suim la Ierusalim ?i se vor împlini toate cele scrise prin prooroci despre Fiul Omului.
Luk 18:32  Caci va fi dat pagânilor ?i va fi batjocorit ?i va fi ocarât ?i scuipat.
Luk 18:33  ?i, dupa ce Îl vor biciui, Îl vor ucide; iar a treia zi va învia.
Luk 18:34  ?i ei n-au în?eles nimic din acestea, caci cuvântul acesta era ascuns pentru ei ?i nu în?elegeau cele spuse.
Luk 18:35  ?i când S-a apropiat Iisus de Ierihon, un orb ?edea lânga drum, cer?ind.
Luk 18:36  ?i, auzind el mul?imea care trecea, întreba ce e aceasta.
Luk 18:37  ?i i-au spus ca trece Iisus Nazarineanul.
Luk 18:38  ?i el a strigat, zicând: Iisuse, Fiul lui David, fie-?i mila de mine!
Luk 18:39  ?i cei care mergeau înainte îl certau ca sa taca, iar el cu mult mai mult striga: Fiule al lui David, fie-?i mila de mine!
Luk 18:40  ?i oprindu-Se, Iisus a poruncit sa-l aduca la El; ?i apropiindu-se, l-a întrebat:
Luk 18:41  Ce voie?ti sa-?i fac? Iar el a zis: Doamne, sa vad!
Luk 18:42  ?i Iisus i-a zis: Vezi! Credin?a ta te-a mântuit.
Luk 18:43  ?i îndata a vazut ?i mergea dupa El, slavind pe Dumnezeu. ?i tot poporul, care vazuse, a dat lauda lui Dumnezeu.
Luk 19:1  ?i intrând, trecea prin Ierihon.
Luk 19:2  ?i iata un barbat, cu numele Zaheu, ?i acesta era mai-marele vame?ilor ?i era bogat.
Luk 19:3  ?i cauta sa vada cine este Iisus, dar nu putea de mul?ime, pentru ca era mic de statura.
Luk 19:4  ?i alergând el înainte, s-a suit într-un sicomor, ca sa-L vada, caci pe acolo avea sa treaca.
Luk 19:5  ?i când a sosit la locul acela, Iisus, privind în sus, a zis catre el: Zahee, coboara-te degraba, caci astazi în casa ta trebuie sa ramân.
Luk 19:6  ?i a coborât degraba ?i L-a primit, bucurându-se.
Luk 19:7  ?i vazând, to?i murmurau, zicând ca a intrat sa gazduiasca la un om pacatos.
Luk 19:8  Iar Zaheu, stând, a zis catre Domnul: Iata, jumatate din averea mea, Doamne, o dau saracilor ?i, daca am napastuit pe cineva cu ceva, întorc împatrit.
Luk 19:9  ?i a zis catre el Iisus: Astazi s-a facut mântuire casei acesteia, caci ?i acesta este fiu al lui Avraam.
Luk 19:10  Caci Fiul Omului a venit sa caute ?i sa mântuiasca pe cel pierdut.
Luk 19:11  ?i ascultând ei acestea, Iisus, adaugând, le-a spus o pilda, fiindca El era aproape de Ierusalim, iar ei credeau ca împara?ia lui Dumnezeu se va arata îndata.
Luk 19:12  Deci a zis: Un om de neam mare s-a dus într-o ?ara îndepartata, ca sa-?i ia domnie ?i sa se întoarca.
Luk 19:13  ?i chemând zece slugi ale sale, le-a dat zece mine ?i a zis catre ele: Negu?atori?i cu ele pâna ce voi veni!
Luk 19:14  Dar ceta?enii lui îl urau ?i au trimis solie în urma lui, zicând: Nu voim ca acesta sa domneasca peste noi.
Luk 19:15  ?i când s-a întors el, dupa ce luase domnia, a zis sa fie chemate slugile acelea, carora le daduse banii, ca sa ?tie cine ce a negu?atorit.
Luk 19:16  ?i a venit cea dintâi, zicând: Doamne, mina ta a adus câ?tig zece mine.
Luk 19:17  ?i i-a zis stapânul: Bine sluga buna, fiindca întru pu?in ai fost credincioasa, sa ai stapânire peste zece ceta?i.
Luk 19:18  ?i a venit a doua, zicând: Mina ta, stapâne, a mai adus cinci mine.
Luk 19:19  Iar el a zis ?i acesteia: Sa ai ?i tu stapânire peste cinci ceta?i.
Luk 19:20  A venit ?i cealalta, zicând: Doamne, iata mina ta, pe care am pastrat-o într-un ?tergar,
Luk 19:21  Ca ma temeam de tine, pentru ca e?ti om aspru: iei ce nu ai pus ?i seceri ce n-ai semanat.
Luk 19:22  Zis-a lui stapânul: Din cuvintele tale te voi judeca, sluga vicleana. Ai ?tiut ca sunt om aspru: iau ce nu am pus ?i secer ce nu am semanat;
Luk 19:23  Pentru ce deci n-ai dat banul meu schimbatorilor de bani? ?i eu, venind, l-a? fi luat cu dobânda.
Luk 19:24  ?i a zis celor ce stateau de fa?a: Lua?i de la el mina ?i da?i-o celui ce are zece mine.
Luk 19:25  ?i ei au zis lui: Doamne, acela are zece mine.
Luk 19:26  Zic voua: Ca oricui are i se va da, iar de la cel ce nu are ?i ceea ce are i se va lua.
Luk 19:27  Iar pe acei vrajma?i ai mei, care n-au voit sa domnesc peste ei, aduce?i-i aici ?i taia?i-i în fa?a mea.
Luk 19:28  ?i zicând acestea, mergea înainte, suindu-Se la Ierusalim.
Luk 19:29  Iar când S-a apropiat de Betfaghe ?i de Betania, catre muntele care se zice Muntele Maslinilor, a trimis pe doi dintre ucenici,
Luk 19:30  Zicând: Merge?i în satul dinaintea voastra ?i, intrând în el, ve?i gasi un mânz legat pe care nimeni dintre oameni n-a ?ezut vreodata. ?i, dezlegându-l, aduce?i-l.
Luk 19:31  ?i daca va va întreba cineva: Pentru ce-l dezlega?i?, ve?i zice a?a: Pentru ca Domnul are trebuin?a de el.
Luk 19:32  ?i, plecând, cei trimi?i au gasit precum le-a spus.
Luk 19:33  Pe când ace?tia dezlegau mânzul, au zis stapânii lui catre ei: De ce dezlega?i mânzul?
Luk 19:34  Iar ei au raspuns: Pentru ca are trebuin?a de el Domnul.
Luk 19:35  ?i i-au adus la Iisus ?i, aruncându-?i hainele lor pe mânz, l-au ajutat pe Iisus sa urce pe el.
Luk 19:36  Iar pe când mergea El, a?terneau hainele lor pe cale.
Luk 19:37  ?i apropiindu-se de poalele Muntelui Maslinilor, toata mul?imea ucenicilor, bucurându-se, a început sa laude pe Dumnezeu, cu glas tare, pentru toate minunile pe care le vazuse,
Luk 19:38  Zicând: Binecuvântat este Împaratul care vine întru numele Domnului! Pace în cer ?i slava întru cei de sus.
Luk 19:39  Dar unii farisei din mul?ime au zis catre El: Înva?atorule, cearta-?i ucenicii.
Luk 19:40  ?i El, raspunzând, a zis: Zic voua: Daca vor tacea ace?tia, pietrele vor striga.
Luk 19:41  ?i când S-a apropiat, vazând cetatea, a plâns pentru ea, zicând:
Luk 19:42  Daca ai fi cunoscut ?i tu, în ziua aceasta, cele ce sunt spre pacea ta! Dar acum ascunse sunt de ochii tai.
Luk 19:43  Caci vor veni zile peste tine, când du?manii tai vor sapa ?an? în jurul tau ?i te vor împresura ?i te vor strâmtora din toate par?ile.
Luk 19:44  ?i te vor face una cu pamântul, ?i pe fiii tai care sunt în tine, ?i nu vor lasa în tine piatra pe piatra pentru ca nu ai cunoscut vremea cercetarii tale.
Luk 19:45  ?i intrând în templu, a început sa scoata pe cei ce vindeau ?i cumparau în el.
Luk 19:46  Zicându-le: Scris este: "?i va fi casa Mea casa de rugaciune"; dar voi a?i facut din ea pe?tera de tâlhari.
Luk 19:47  ?i era în fiecare zi în templu ?i înva?a. Dar arhiereii ?i carturarii ?i frunta?ii poporului cautau sa-L piarda.
Luk 19:48  ?i nu gaseau ce sa-I faca, caci tot poporul se ?inea dupa El, ascultându-L.
Luk 20:1  ?i într-una din zile, pe când Iisus înva?a poporul în templu ?i binevestea, au venit arhiereii ?i carturarii, împreuna cu batrânii,
Luk 20:2  ?i, vorbind, au zis catre El: Spune noua, cu ce putere faci acestea, sau cine este Cel ce ?i-a dat aceasta putere?
Luk 20:3  Iar El, raspunzând, a zis catre ei: Va voi întreba ?i Eu pe voi un cuvânt, ?i spune?i-Mi:
Luk 20:4  Botezul lui Ioan era din cer sau de la oameni?
Luk 20:5  ?i ei cugetau în sinea lor, zicând: Daca vom spune: Din cer, va zice: Pentru ce n-a?i crezut în el?
Luk 20:6  Iar daca vom zice: De la oameni, tot poporul ne va ucide cu pietre, caci este încredin?at ca Ioan a fost prooroc.
Luk 20:7  ?i au raspuns ca nu ?tiu de unde.
Luk 20:8  ?i Iisus le-a zis: Nici Eu nu va spun voua cu ce putere fac acestea.
Luk 20:9  ?i a început sa spuna catre popor pilda aceasta: Un om a sadit vie ?i a dat-o lucratorilor ?i a plecat departe pentru multa vreme.
Luk 20:10  ?i la timpul potrivit, a trimis la lucratori o sluga ca sa-i dea din rodul viei. Lucratorii însa, batând-o, au trimis-o fara nimic.
Luk 20:11  ?i a trimis apoi alta sluga, dar ei, batând-o ?i pe aceea ?i batjocorind-o, au trimis-o fara nimic.
Luk 20:12  ?i a trimis apoi pe a treia; iar ei, ranind-o ?i pe aceea, au alungat-o.
Luk 20:13  ?i stapânul viei a zis: Ce voi face? Voi trimite pe fiul meu cel iubit; poate se vor ru?ina de el.
Luk 20:14  Iar lucratorii, vazându-l, s-au vorbit între ei, zicând: Acesta este mo?tenitorul; sa-l omorâm ca mo?tenirea sa fie a noastra.
Luk 20:15  ?i sco?ându-l afara din vie, l-au ucis. Ce va face, deci, acestora, stapânul viei?
Luk 20:16  Va veni ?i va pierde pe lucratorii aceia, iar via o va da altora. Iar ei auzind, au zis: Sa nu se întâmple!
Luk 20:17  El însa, privind la ei, a zis: Ce înseamna, deci, scriptura aceasta: "Piatra pe care n-au luat-o în seama ziditorii, aceasta a ajuns în capul unghiului"?
Luk 20:18  Oricine va cadea pe aceasta piatra va fi sfarâmat, iar pe cine va cadea ea îl va zdrobi.
Luk 20:19  Iar carturarii ?i arhiereii cautau sa puna mâna pe El, în ceasul acela, dar s-au temut de popor. Caci ei au în?eles ca Iisus spusese pilda aceasta pentru ei.
Luk 20:20  ?i pândindu-L, I-au trimis iscoade, care se prefaceau ca sunt drep?i, ca sa-L prinda în cuvânt ?i sa-L dea stapânirii ?i puterii dregatorului.
Luk 20:21  ?i L-au întrebat, zicând: Înva?atorule, ?tim ca vorbe?ti ?i înve?i drept ?i nu cau?i la fa?a omului, ci cu adevarat înve?i calea lui Dumnezeu:
Luk 20:22  Se cuvine ca noi sa dam dajdie Cezarului sau nu?
Luk 20:23  Dar Iisus, cunoscând vicle?ugul lor, a zis catre ei: De ce Ma ispiti?i?
Luk 20:24  Arata?i-mi un dinar. Al cui chip ?i scriere are pe el? Iar ei au zis: Ale Cezarului.
Luk 20:25  ?i El a zis catre ei: A?adar, da?i cele ce sunt ale Cezarului, Cezarului ?i cele ce sunt ale lui Dumnezeu, lui Dumnezeu.
Luk 20:26  ?i nu L-au putut prinde în cuvânt înaintea poporului ?i, mirându-se de cuvântul Lui, au tacut.
Luk 20:27  ?i apropiindu-se unii dintre saducheii care zic ca nu este înviere, L-au întrebat:
Luk 20:28  Zicând: Înva?atorule, Moise a scris pentru noi: Daca moare fratele cuiva, având femeie, ?i el n-a avut copii, sa ia fratele lui pe femeie ?i sa ridice urma? fratelui sau.
Luk 20:29  Erau deci ?apte fra?i. ?i cel dintâi, luându-?i femeie, a murit fara de copii.
Luk 20:30  ?i a luat-o al doilea, ?i a murit ?i el fara copii.
Luk 20:31  A luat-o ?i al treilea; ?i tot a?a to?i ?apte n-au lasat copii ?i au murit.
Luk 20:32  La urma a murit ?i femeia.
Luk 20:33  Deci femeia, la înviere, a caruia dintre ei va fi so?ie, caci to?i ?apte au avut-o de so?ie?
Luk 20:34  ?i le-a zis lor Iisus: Fiii veacului acestuia se însoara ?i se marita;
Luk 20:35  Iar cei ce se vor învrednici sa dobândeasca veacul acela ?i învierea cea din mor?i, nici nu se însoara, nici nu se marita.
Luk 20:36  Caci nici sa moara nu mai pot, caci sunt la fel cu îngerii ?i sunt fii ai lui Dumnezeu, fiind fii ai învierii.
Luk 20:37  Iar ca mor?ii înviaza a aratat chiar Moise la rug, când nume?te Domn pe Dumnezeul lui Avraam, ?i Dumnezeul lui Isaac, ?i Dumnezeul lui Iacov.
Luk 20:38  Dumnezeu deci nu este Dumnezeu al mor?ilor, ci al viilor, caci to?i traiesc în El.
Luk 20:39  Iar unii dintre carturari, raspunzând, au zis: Înva?atorule, bine ai zis.
Luk 20:40  ?i nu mai cutezau sa-L întrebe nimic.
Luk 20:41  Iar El i-a întrebat: Cum se zice, dar, ca Hristos este Fiul lui David?
Luk 20:42  Caci însu?i David spune în Cartea Psalmilor: "Zis-a Domnul Domnului meu: ?ezi de-a dreapta Mea,
Luk 20:43  Pâna ce voi pune pe vrajma?ii Tai a?ternut picioarelor Tale".
Luk 20:44  Deci David Îl nume?te Domn; ?i cum este fiu al lui?
Luk 20:45  ?i ascultând tot poporul, a zis ucenicilor:
Luk 20:46  Pazi?i-va de carturari, carora le place sa se plimbe în haine lungi, care iubesc plecaciunile în pie?e ?i scaunele cele dintâi în sinagogi ?i locurile cele dintâi la ospe?e,
Luk 20:47  Mâncând casele vaduvelor ?i de ochii lumii rugându-se îndelung; ace?tia vor lua mai mare osânda.
Luk 21:1  ?i privind, a vazut pe cei boga?i, aruncând darurile lor în vistieria templului.
Luk 21:2  ?i a vazut ?i pe o vaduva saraca, aruncând acolo doi bani.
Luk 21:3  ?i a zis: Adevarat va spun ca aceasta vaduva saraca a aruncat mai mult decât to?i.
Luk 21:4  Caci to?i ace?tia din prisosul lor au aruncat la daruri, aceasta însa din saracia ei a aruncat tot ce avea pentru via?a.
Luk 21:5  Iar unii vorbind despre templu ca este împodobit cu pietre frumoase ?i cu podoabe, El a zis:
Luk 21:6  Vor veni zile când, din cele ce vede?i, nu va ramâne piatra peste piatra care sa nu se risipeasca.
Luk 21:7  ?i ei L-au întrebat, zicând: Înva?atorule, când oare, vor fi acestea? ?i care este semnul când au sa fie acestea?
Luk 21:8  Iar El a zis: Vede?i sa nu fi?i amagi?i, caci mul?i vor veni în numele Meu, zicând: Eu sunt, ?i vremea s-a apropiat. Nu merge?i dupa ei.
Luk 21:9  Iar când ve?i auzi de razboaie ?i de razmeri?e, sa nu va înspaimânta?i; caci acestea trebuie sa fie întâi, dar sfâr?itul nu va fi curând.
Luk 21:10  Atunci le-a zis: Se va ridica neam peste neam ?i împara?ie peste împara?ie.
Luk 21:11  ?i vor fi cutremure mari ?i, pe alocurea, foamete ?i ciuma ?i spaime ?i semne mari din cer vor fi.
Luk 21:12  Dar, mai înainte de toate acestea, î?i vor pune mâinile pe voi ?i va vor prigoni, dându-va în sinagogi ?i în temni?e, ducându-va la împara?i ?i la dregatori, pentru numele Meu.
Luk 21:13  ?i va fi voua spre marturie.
Luk 21:14  Pune?i deci în inimile voastre sa nu gândi?i de mai înainte ce ve?i raspunde;
Luk 21:15  Caci Eu va voi da gura ?i în?elepciune, careia nu-i vor putea sta împotriva, nici sa-i raspunda to?i potrivnicii vo?tri.
Luk 21:16  ?i ve?i fi da?i ?i de parin?i ?i de fra?i ?i de neamuri ?i de prieteni, ?i vor ucide dintre voi.
Luk 21:17  ?i ve?i fi urâ?i de to?i pentru numele Meu.
Luk 21:18  ?i par din capul vostru nu va pieri.
Luk 21:19  Prin rabdarea voastra ve?i dobândi sufletele voastre.
Luk 21:20  Iar când ve?i vedea Ierusalimul înconjurat de o?ti, atunci sa ?ti?i ca s-a apropiat pustiirea lui.
Luk 21:21  Atunci cei din Iudeea sa fuga la mun?i ?i cei din mijlocul lui sa iasa din el ?i cei de prin ?arina sa nu intre în el.
Luk 21:22  Caci acestea sunt zilele razbunarii, ca sa se împlineasca toate cele scrise.
Luk 21:23  Dar vai celor care vor avea în pântece ?i celor care vor alapta în acele zile. Caci va fi în ?ara mare strâmtorare ?i mânie împotriva acestui popor.
Luk 21:24  ?i vor cadea de ascu?i?ul sabiei ?i vor fi du?i robi la toate neamurile, ?i Ierusalimul va fi calcat în picioare de neamuri, pâna ce se vor împlini vremurile neamurilor.
Luk 21:25  ?i vor fi semne în soare, în luna ?i în stele, iar pe pamânt spaima întru neamuri ?i nedumerire din pricina vuietului marii ?i al valurilor.
Luk 21:26  Iar oamenii vor muri de frica ?i de a?teptarea celor ce au sa vina peste lume, caci puterile cerurilor se vor clatina.
Luk 21:27  ?i atunci vor vedea pe Fiul Omului venind pe nori cu putere ?i cu slava multa.
Luk 21:28  Iar când vor începe sa fie acestea, prinde?i curaj ?i ridica?i capetele voastre, pentru ca rascumpararea voastra se apropie.
Luk 21:29  ?i le-a spus o pilda: Vede?i smochinul ?i to?i copacii:
Luk 21:30  Când înfrunzesc ace?tia, vazându-i, de la voi în?iva ?ti?i ca vara este aproape.
Luk 21:31  A?a ?i voi, când ve?i vedea facându-se acestea, sa ?ti?i ca aproape este împara?ia lui Dumnezeu.
Luk 21:32  Adevarat graiesc voua ca nu va trece neamul acesta pâna ce nu vor fi toate acestea.
Luk 21:33  Cerul ?i pamântul vor trece, dar cuvintele Mele nu vor trece.
Luk 21:34  Lua?i seama la voi în?iva, sa nu se îngreuieze inimile voastre de mâncare ?i de bautura ?i de grijile vie?ii, ?i ziua aceea sa vine peste voi fara de veste,
Luk 21:35  Ca o cursa; caci va veni peste to?i cei ce locuiesc pe fa?a întregului pamânt.
Luk 21:36  Priveghea?i dar în toata vremea rugându-va, ca sa va întari?i sa scapa?i de toate acestea care au sa vina ?i sa sta?i înaintea Fiului Omului.
Luk 21:37  ?i ziua era în templu ?i înva?a, iar noaptea, ie?ind, o petrecea pe muntele ce se cheama al Maslinilor.
Luk 21:38  ?i tot poporul venea dis-de-diminea?a la El în templu, ca sa-L asculte.
Luk 22:1  ?i se apropia sarbatoarea Azimelor, care se chema Pa?ti.
Luk 22:2  ?i arhiereii ?i carturarii cautau cum sa-L omoare; caci se temeau de popor.
Luk 22:3  ?i a intrat satana în Iuda, cel numit Iscarioteanul, care era din numarul celor doisprezece.
Luk 22:4  ?i, ducându-se, el a vorbit cu arhiereii ?i cu capeteniile oastei, cum sa-L dea în mâinile lor.
Luk 22:5  ?i ei s-au bucurat ?i s-au învoit sa-i dea bani.
Luk 22:6  ?i el a primit ?i cauta prilej sa-L dea lor, fara ?tirea mul?imii.
Luk 22:7  ?i a sosit ziua Azimelor, în care trebuia sa se jertfeasca Pa?tile.
Luk 22:8  ?i a trimis pe Petru ?i pe Ioan, zicând: Merge?i ?i ne pregati?i Pa?tile, ca sa mâncam.
Luk 22:9  Iar ei I-au zis: Unde voie?ti sa pregatim?
Luk 22:10  Iar El le-a zis: Iata, când ve?i intra în cetate, va va întâmpina un om ducând un urcior cu apa; merge?i dupa el în casa în care va intra.
Luk 22:11  ?i spune?i stapânului casei: Înva?atorul î?i zice: Unde este încaperea în care sa manânc Pa?tile cu ucenicii mei?
Luk 22:12  ?i acela va va arata un foi?or mare, a?ternut; acolo sa pregati?i.
Luk 22:13  Iar, ei, ducându-se, au aflat precum le spusese ?i au pregatit Pa?tile.
Luk 22:14  ?i când a fost ceasul, S-a a?ezat la masa, ?i apostolii împreuna cu El.
Luk 22:15  ?i a zis catre ei: Cu dor am dorit sa manânc cu voi acest Pa?ti, mai înainte de patima Mea,
Luk 22:16  Caci zic voua ca de acum nu-l voi mai mânca, pâna când nu va fi desavâr?it în împara?ia lui Dumnezeu.
Luk 22:17  ?i luând paharul, mul?umind, a zis: Lua?i acesta ?i împar?i?i-l între voi;
Luk 22:18  Ca zic voua: Nu voi mai bea de acum din rodul vi?ei, pâna ce nu va veni împara?ia lui Dumnezeu.
Luk 22:19  ?i luând pâinea, mul?umind, a frânt ?i le-a dat lor, zicând: Acesta este Trupul Meu care se da pentru voi; aceasta sa face?i spre pomenirea Mea.
Luk 22:20  Asemenea ?i paharul, dupa ce au cinat, zicând: Acest pahar este Legea cea noua, întru Sângele Meu, care se varsa pentru voi.
Luk 22:21  Dar iata, mâna celui ce Ma vinde este cu Mine la masa.
Luk 22:22  ?i Fiul Omului merge precum a fost orânduit, dar vai omului aceluia prin care este vândut!
Luk 22:23  Iar ei au început sa se întrebe, unul pe altul, cine dintre ei ar fi acela, care avea sa faca aceasta?
Luk 22:24  ?i s-a iscat între ei ?i neîn?elegere: cine dintre ei se pare ca e mai mare?
Luk 22:25  Iar El le-a zis: Regii neamurilor domnesc peste ele ?i se numesc binefacatori.
Luk 22:26  Dar între voi sa nu fie astfel, ci cel mai mare dintre voi sa fie ca cel mai tânar, ?i capetenia ca acela care sluje?te.
Luk 22:27  Caci cine este mai mare: cel care sta la masa, sau cel care sluje?te? Oare, nu cel ce sta la masa? Iar Eu, în mijlocul vostru, sunt ca unul ce sluje?te.
Luk 22:28  ?i voi sunte?i aceia care a?i ramas cu Mine în încercarile Mele.
Luk 22:29  ?i Eu va rânduiesc voua împara?ie, precum Mi-a rânduit Mie Tatal Meu,
Luk 22:30  Ca sa mânca?i ?i sa be?i la masa Mea, în împara?ia Mea ?i sa ?ede?i pe tronuri, judecând cele douasprezece semin?ii ale lui Israel.
Luk 22:31  ?i a zis Domnul: Simone, Simone, iata satana v-a cerut sa va cearna ca pe grâu;
Luk 22:32  Iar Eu M-am rugat pentru tine sa nu piara credin?a ta. ?i tu, oarecând, întorcându-te, întare?te pe fra?ii tai.
Luk 22:33  Iar el I-a zis: Doamne, cu Tine sunt gata sa merg ?i în temni?a ?i la moarte.
Luk 22:34  Iar Iisus i-a zis: Zic ?ie, Petre, nu va cânta astazi coco?ul, pâna ce de trei ori te vei lepada de Mine, ca nu Ma cuno?ti.
Luk 22:35  ?i le-a zis: Când v-am trimis pe voi fara punga, fara traista ?i fara încal?aminte, a?i avut lipsa de ceva? Iar ei au zis: De nimic.
Luk 22:36  ?i El le-a zis: Acum însa cel ce are punga sa o ia, tot a?a ?i traista, ?i cel ce nu are sabie sa-?i vânda haina ?i sa-?i cumpere.
Luk 22:37  Caci va spun ca trebuie sa se împlineasca întru Mine Scriptura aceasta: "?i cu cei fara de lege s-a socotit", caci cele despre Mine au ajuns la sfâr?it.
Luk 22:38  Iar ei au zis: Doamne, iata aici doua sabii. Zis-a lor: Sunt de ajuns.
Luk 22:39  ?i, ie?ind, s-a dus dupa obicei în Muntele Maslinilor, ?i ucenicii l-au urmat.
Luk 22:40  ?i când a sosit în acest loc, le-a zis: Ruga?i-va, ca sa nu intra?i în ispita.
Luk 22:41  ?i El S-a departat de ei ca la o aruncatura de piatra, ?i îngenunchind, Se ruga.
Luk 22:42  Zicând: Parinte, de voie?ti, treaca de la Mine acest pahar. Dar nu voia Mea, ci voia Ta sa se faca.
Luk 22:43  Iar un înger din cer s-a aratat Lui ?i-L întarea.
Luk 22:44  Iar El, fiind în chin de moarte, mai staruitor Se ruga. ?i sudoarea Lui s-a facut ca picaturi de sânge care picurau pe pamânt.
Luk 22:45  ?i, ridicându-Se din rugaciune, a venit la ucenicii Lui ?i i-a aflat adormi?i de întristare.
Luk 22:46  ?i le-a zis: De ce dormi?i? Scula?i-va ?i va ruga?i, ca sa nu intra?i în ispita.
Luk 22:47  ?i vorbind El, iata o mul?ime ?i cel ce se numea Iuda, unul dintre cei doisprezece, venea în fruntea lor. ?i s-a apropiat de Iisus, ca sa-L sarute.
Luk 22:48  Iar Iisus i-a zis: Iuda, cu sarutare vinzi pe Fiul Omului?
Luk 22:49  Iar cei din preajma Lui, vazând ce avea sa se întâmple, au zis: Doamne, daca vom lovi cu sabia?
Luk 22:50  ?i unul dintre ei a lovit pe sluga arhiereului ?i i-a taiat urechea dreapta.
Luk 22:51  Dar Iisus, raspunzând, a zis: Lasa?i, pâna aici. ?i atingându-Se de urechea lui l-a vindecat
Luk 22:52  ?i catre arhiereii, catre capeteniile templului ?i catre batrânii care venisera asupra Lui, Iisus a zis: Ca la un tâlhar a?i ie?it, cu sabii ?i cu toiege.
Luk 22:53  În toate zilele fiind cu voi în templu, n-a?i întins mâinile asupra Mea. Dar acesta este ceasul vostru ?i stapânirea întunericului.
Luk 22:54  ?i, prinzându-L, L-au dus ?i L-au bagat în casa arhiereului. Iar Petru Îl urma de departe.
Luk 22:55  ?i, aprinzând ei foc în mijlocul cur?ii ?i ?ezând împreuna, a ?ezut ?i Petru în mijlocul lor.
Luk 22:56  ?i o slujnica, vazându-l ?ezând la foc, ?i uitându-se bine la el, a zis: ?i acesta era cu El.
Luk 22:57  Iar el s-a lepadat, zicând: Femeie, nu-L cunosc.
Luk 22:58  ?i dupa pu?in timp, vazându-l un altul, i-a zis: ?i tu e?ti dintre ei. Petru însa a zis: Omule, nu sunt.
Luk 22:59  Iar când a trecut ca un ceas, un altul sus?inea zicând: Cu adevarat ?i acesta era cu El, caci este galileian.
Luk 22:60  ?i Petru a zis: Omule, nu ?tiu ce spui. ?i îndata, înca vorbind el, a cântat coco?ul.
Luk 22:61  ?i întorcându-Se, Domnul a privit spre Petru; ?i Petru ?i-a adus aminte de cuvântul Domnului, cum îi zisese ca, mai înainte de a cânta coco?ul astazi, tu te vei lepada de Mine de trei ori.
Luk 22:62  ?i ie?ind afara, Petru a plâns cu amar.
Luk 22:63  Iar barba?ii care Îl pazeau pe Iisus, Îl batjocoreau, batându-L.
Luk 22:64  ?i acoperindu-I fa?a, Îl întrebau, zicând: Prooroce?te cine este cel ce Te-a lovit?
Luk 22:65  ?i hulindu-L, multe altele spuneau împotriva Lui.
Luk 22:66  ?i când s-a facut ziua, s-au adunat batrânii poporului, arhiereii ?i carturarii ?i L-au dus pe El în sinedriul lor.
Luk 22:67  Zicând: Spune noua daca e?ti Tu Hristosul. ?i El le-a zis: Daca va voi spune, nu ve?i crede;
Luk 22:68  Iar daca va voi întreba, nu-Mi ve?i raspunde.
Luk 22:69  De acum însa Fiul Omului va ?edea de-a dreapta puterii lui Dumnezeu.
Luk 22:70  Iar ei au zis to?i: A?adar, Tu e?ti Fiul lui Dumnezeu? ?i El a zis catre ei: Voi zice?i ca Eu sunt.
Luk 22:71  ?i ei au zis: Ce ne mai trebuie marturii, caci noi în?ine am auzit din gura Lui?
Luk 23:1  ?i sculându-se toata mul?imea acestora, L-au dus înaintea lui Pilat.
Luk 23:2  ?i au început sa-L pârasca, zicând: Pe Acesta L-am gasit razvratind neamul nostru ?i împiedicând sa dam dajdie Cezarului ?i zicând ca El este Hristos rege.
Luk 23:3  Iar Pilat L-a întrebat, zicând: Tu e?ti regele iudeilor? Iar El, raspunzând, a zis: Tu zici.
Luk 23:4  ?i Pilat a zis catre arhierei ?i catre mul?imi: Nu gasesc nici o vina în Omul acesta.
Luk 23:5  Dar ei staruiau, zicând ca întarâta poporul, înva?ând prin toata Iudeea, începând din Galileea pâna aici.
Luk 23:6  ?i Pilat auzind, a întrebat daca omul este galileian.
Luk 23:7  ?i aflând ca este sub stapânirea lui Irod, l-a trimis la Irod, care era ?i el în Ierusalim în acele zile.
Luk 23:8  Iar Irod, vazând pe Iisus, s-a bucurat foarte, ca de multa vreme dorea sa-L cunoasca pentru ca auzise despre El, ?i nadajduia sa vada vreo minune savâr?ita de El.
Luk 23:9  ?i L-a întrebat Irod multe lucruri, dar El nu i-a raspuns nimic.
Luk 23:10  ?i arhiereii ?i carturarii erau de fa?a, învinuindu-L foarte tare.
Luk 23:11  Iar Irod, împreuna cu osta?ii sai, batjocorindu-L ?i luându-L în râs, L-a îmbracat cu o haina stralucitoare ?i L-a trimis iara?i la Pilat.
Luk 23:12  ?i în ziua aceea, Irod ?i Pilat s-au facut prieteni unul cu altul, caci mai înainte erau în du?manie între ei.
Luk 23:13  Iar Pilat, chemând arhiereii ?i capeteniile ?i poporul,
Luk 23:14  A zis catre ei: A?i adus la mine pe Omul acesta, ca pe un razvratitor al poporului; dar iata eu, cercetându-L în fa?a voastra, nici o vina n-am gasit în acest Om, din cele ce aduce?i împotriva Lui.
Luk 23:15  ?i nici Irod n-a gasit, caci L-a trimis iara?i la noi. ?i iata, El n-a savâr?it nimic vrednic de moarte.
Luk 23:16  Deci, pedepsindu-L, Îl voi elibera.
Luk 23:17  ?i trebuia, la praznic, sa le elibereze un vinovat.
Luk 23:18  Dar ei, cu to?ii, au strigat, zicând: Ia-L pe Acesta ?i elibereaza-ne pe Baraba,
Luk 23:19  Care era aruncat în temni?a pentru o rascoala facuta în cetate ?i pentru omor.
Luk 23:20  ?i iara?i le-a vorbit Pilat, voind sa le elibereze pe Iisus.
Luk 23:21  Dar ei strigau, zicând: Rastigne?te-L! Rastigne?te-L!
Luk 23:22  Iar el a zis a treia oara catre ei: Ce rau a savâr?it Acesta? Nici o vina de moarte nu am aflat întru El. Deci, pedepsindu-L, Îl voi elibera.
Luk 23:23  Dar ei staruiau, cerând cu strigate mari ca El sa fie rastignit, ?i strigatele lor au biruit.
Luk 23:24  Deci Pilat a hotarât sa se împlineasca cererea lor.
Luk 23:25  ?i le-a eliberat pe cel aruncat în temni?a pentru rascoala ?i ucidere, pe care îl cereau ei, iar pe Iisus L-a dat în voia lor.
Luk 23:26  ?i pe când Îl duceau, oprind pe un oarecare Simon Cirineul, care venea din ?arina, i-au pus crucea, ca s-o duca în urma lui Iisus.
Luk 23:27  Iar dupa El venea mul?ime multa de popor ?i de femei, care se bateau în piept ?i Îl plângeau.
Luk 23:28  ?i întorcându-Se catre ele, Iisus le-a zis: Fiice ale Ierusalimului, nu Ma plânge?i pe Mine, ci pe voi plânge?i-va ?i pe copiii vo?tri.
Luk 23:29  Caci iata, vin zile în care vor zice: Fericite sunt cele sterpe ?i pântecele care n-au nascut ?i sânii care n-au alaptat!
Luk 23:30  Atunci vor începe sa spuna mun?ilor: Cade?i peste noi; ?i dealurilor: Acoperi?i-ne.
Luk 23:31  Caci daca fac acestea cu lemnul verde, cu cel uscat ce va fi?
Luk 23:32  ?i erau du?i ?i al?ii, doi facatori de rele, ca sa-i omoare împreuna cu El.
Luk 23:33  ?i când au ajuns la locul ce se cheama al Capa?ânii, L-au rastignit acolo pe El ?i pe facatorii de rele, unul de-a dreapta ?i unul de-a stânga.
Luk 23:34  Iar Iisus zicea: Parinte, iarta-le lor, ca nu ?tiu ce fac. ?i împar?ind hainele Lui, au aruncat sor?i.
Luk 23:35  ?i sta poporul privind, iar capeteniile î?i bateau joc de El, zicând: Pe al?ii i-a mântuit; sa Se mântuiasca ?i pe Sine Însu?i, daca El este Hristosul, alesul lui Dumnezeu.
Luk 23:36  ?i Îl luau în râs ?i osta?ii care se apropiau, aducându-I o?et.
Luk 23:37  ?i zicând: Daca Tu e?ti regele iudeilor, mântuie?te-Te pe Tine Însu?i!
Luk 23:38  ?i deasupra Lui era scris cu litere grece?ti, latine?ti ?i evreie?ti: Acesta este regele iudeilor.
Luk 23:39  Iar unul dintre facatorii de rele rastigni?i, Îl hulea zicând: Nu e?ti Tu Hristosul? Mântuie?te-Te pe Tine Însu?i ?i pe noi.
Luk 23:40  ?i celalalt, raspunzând, îl certa, zicând: Nu te temi tu de Dumnezeu, ca e?ti în aceea?i osânda?
Luk 23:41  ?i noi pe drept, caci noi primim cele cuvenite dupa faptele noastre; Acesta însa n-a facut nici un rau.
Luk 23:42  ?i zicea lui Iisus: Pomene?te-ma, Doamne, când vei veni în împara?ia Ta.
Luk 23:43  ?i Iisus i-a zis: Adevarat graiesc ?ie, astazi vei fi cu Mine în rai.
Luk 23:44  ?i era acum ca la ceasul al ?aselea ?i întuneric s-a facut peste tot pamântul pâna la ceasul al noualea.
Luk 23:45  Când soarele s-a întunecat; iar catapeteasma templului s-a sfâ?iat pe la mijloc.
Luk 23:46  ?i Iisus, strigând cu glas tare, a zis: Parinte, în mâinile Tale încredin?ez duhul Meu. ?i acestea zicând, ?i-a dat duhul.
Luk 23:47  Iar suta?ul, vazând cele ce s-au facut, a slavit pe Dumnezeu, zicând: Cu adevarat, Omul Acesta drept a fost.
Luk 23:48  ?i toate mul?imile care venisera la aceasta priveli?te, vazând cele întâmplate, se întorceau batându-?i pieptul.
Luk 23:49  ?i to?i cunoscu?ii Lui, ?i femeile care Îl înso?isera din Galileea, stateau departe, privind acestea.
Luk 23:50  ?i iata un barbat cu numele Iosif, sfetnic fiind, barbat bun ?i drept,
Luk 23:51  - Acesta nu se învoise cu sfatul ?i cu fapta lor. El era din Arimateea, cetate a iudeilor, a?teptând împara?ia lui Dumnezeu.
Luk 23:52  Acesta, venind la Pilat, a cerut trupul lui Iisus.
Luk 23:53  ?i coborându-L, L-a înfa?urat în giulgiu de in ?i L-a pus într-un mormânt sapat în piatra, în care nimeni, niciodata, nu mai fusese pus.
Luk 23:54  ?i ziua aceea era vineri, ?i se lumina spre sâmbata.
Luk 23:55  ?i urmându-I femeile, care venisera cu El din Galileea, au privit mormântul ?i cum a fost pus trupul Lui.
Luk 23:56  ?i, întorcându-se, au pregatit miresme ?i miruri; iar sâmbata s-au odihnit, dupa Lege.
Luk 24:1  Iar în prima zi dupa sâmbata, foarte de diminea?a, au venit ele la mormânt, aducând miresmele pe care le pregatisera.
Luk 24:2  ?i au gasit piatra rasturnata de pe mormânt.
Luk 24:3  ?i intrând, nu au gasit trupul Domnului Iisus.
Luk 24:4  ?i fiind ele înca nedumerite de aceasta, iata doi barba?i au stat înaintea lor, în ve?minte stralucitoare.
Luk 24:5  ?i, înfrico?ându-se ele ?i plecându-?i fe?ele la pamânt, au zis aceia catre ele: De ce cauta?i pe Cel viu între cei mor?i?
Luk 24:6  Nu este aici, ci S-a sculat. Aduce?i-va aminte cum v-a vorbit, fiind înca în Galileea,
Luk 24:7  Zicând ca Fiul Omului trebuie sa fie dat în mâinile oamenilor pacato?i ?i sa fie rastignit, iar a treia zi sa învieze.
Luk 24:8  ?i ele ?i-au adus aminte de cuvântul Lui.
Luk 24:9  ?i întorcându-se de la mormânt, au vestit toate acestea celor unsprezece ?i tuturor celorlal?i.
Luk 24:10  Iar ele erau: Maria Magdalena, ?i Ioana ?i Maria lui Iacov ?i celelalte împreuna cu ele, care ziceau catre apostoli acestea.
Luk 24:11  ?i cuvintele acestea au parut înaintea lor ca o aiurare ?i nu le-au crezut.
Luk 24:12  ?i Petru, sculându-se, a alergat la mormânt ?i, plecându-se, a vazut giulgiurile singure zacând. ?i a plecat, mirându-se în sine de ceea ce se întâmplase.
Luk 24:13  ?i iata, doi dintre ei mergeau în aceea?i zi la un sat care era departe de Ierusalim, ca la ?aizeci de stadii, al carui nume era Emaus.
Luk 24:14  ?i aceia vorbeau între ei despre toate întâmplarile acestea.
Luk 24:15  ?i pe când vorbeau ?i se întrebau între ei. ?i Iisus Însu?i, apropiindu-Se, mergea împreuna cu ei.
Luk 24:16  Dar ochii lor erau ?inu?i ca sa nu-L cunoasca.
Luk 24:17  ?i El a zis catre ei: Ce sunt cuvintele acestea pe care le schimba?i unul cu altul în drumul vostru? Iar ei s-au oprit, cuprin?i de întristare.
Luk 24:18  Raspunzând, unul cu numele Cleopa a zis catre El: Tu singur e?ti strain în Ierusalim ?i nu ?tii cele ce s-au întâmplat în el în zilele acestea?
Luk 24:19  El le-a zis: Care? Iar ei I-au raspuns: Cele despre Iisus Nazarineanul, Care era prooroc puternic în fapta ?i în cuvânt înaintea lui Dumnezeu ?i a întregului popor.
Luk 24:20  Cum L-au osândit la moarte ?i L-au rastignit arhiereii ?i mai-marii no?tri;
Luk 24:21  Iar noi nadajduiam ca El este Cel ce avea sa izbaveasca pe Israel; ?i, cu toate acestea, astazi este a treia zi de când s-au petrecut acestea.
Luk 24:22  Dar ?i ni?te femei de ale noastre ne-au spaimântat ducându-se dis-de-diminea?a la mormânt,
Luk 24:23  ?i, negasind trupul Lui, au venit zicând ca au vazut aratare de îngeri, care le-au spus ca El este viu.
Luk 24:24  Iar unii dintre noi s-au dus la mormânt ?i au gasit a?a precum spusesera femeile, dar pe El nu L-au vazut.
Luk 24:25  ?i El a zis catre ei: O, nepricepu?ilor ?i zabavnici cu inima ca sa crede?i toate câte au spus proorocii!
Luk 24:26  Nu trebuia oare, ca Hristos sa patimeasca acestea ?i sa intre în slava Sa?
Luk 24:27  ?i începând de la Moise ?i de la to?i proorocii, le-a tâlcuit lor, din toate Scripturile cele despre El.
Luk 24:28  ?i s-au apropiat de satul unde se duceau, iar El se facea ca merge mai departe.
Luk 24:29  Dar ei Îl rugau staruitor, zicând: Ramâi cu noi ca este spre seara ?i s-a plecat ziua. ?i a intrat sa ramâna cu ei.
Luk 24:30  ?i, când a stat împreuna cu ei la masa, luând El pâinea, a binecuvântat ?i, frângând, le-a dat lor.
Luk 24:31  ?i s-au deschis ochii lor ?i L-au cunoscut; ?i El s-a facut nevazut de ei.
Luk 24:32  ?i au zis unul catre altul: Oare, nu ardea în noi inima noastra, când ne vorbea pe cale ?i când ne tâlcuia Scripturile?
Luk 24:33  ?i, în ceasul acela sculându-se, s-au întors la Ierusalim ?i au gasit aduna?i pe cei unsprezece ?i pe cei ce erau împreuna cu ei,
Luk 24:34  Care ziceau ca a înviat cu adevarat Domnul ?i S-a aratat lui Simon.
Luk 24:35  ?i ei au istorisit cele petrecute pe cale ?i cum a fost cunoscut de ei la frângerea pâinii.
Luk 24:36  ?i pe când vorbeau ei acestea, El a stat în mijlocul lor ?i le-a zis: Pace voua.
Luk 24:37  Iar ei, înspaimântându-se ?i înfrico?ându-se, credeau ca vad duh.
Luk 24:38  ?i Iisus le-a zis: De ce sunte?i tulbura?i ?i pentru ce se ridica astfel de gânduri în inima voastra?
Luk 24:39  Vede?i mâinile Mele ?i picioarele Mele, ca Eu Însumi sunt; pipai?i-Ma ?i vede?i, ca duhul nu are carne ?i oase, precum Ma vede?i pe Mine ca am.
Luk 24:40  ?i zicând acestea, le-a aratat mâinile ?i picioarele Sale.
Luk 24:41  Iar ei înca necrezând de bucurie ?i minunându-se, El le-a zis: Ave?i aici ceva de mâncare?
Luk 24:42  Iar ei i-au dat o bucata de pe?te fript ?i dintr-un fagure de miere.
Luk 24:43  ?i luând, a mâncat înaintea lor.
Luk 24:44  ?i le-a zis: Acestea sunt cuvintele pe care le-am grait catre voi fiind înca împreuna cu voi, ca trebuie sa se împlineasca toate cele scrise despre Mine în Legea lui Moise, în prooroci ?i în psalmi.
Luk 24:45  Atunci le-a deschis mintea ca sa priceapa Scripturile.
Luk 24:46  ?i le-a spus ca a?a este scris ?i a?a trebuie sa patimeasca Hristos ?i a?a sa învieze din mor?i a treia zi.
Luk 24:47  ?i sa se propovaduiasca în numele Sau pocain?a spre iertarea pacatelor la toate neamurile, începând de la Ierusalim.
Luk 24:48  Voi sunte?i martorii acestora.
Luk 24:49  ?i iata, Eu trimit peste voi fagaduin?a Tatalui Meu; voi însa ?ede?i în cetate, pâna ce va ve?i îmbraca cu putere de sus.
Luk 24:50  ?i i-a dus afara pâna spre Betania ?i, ridicându-?i mâinile, i-a binecuvântat.
Luk 24:51  ?i pe când îi binecuvânta, S-a despar?it de ei ?i S-a înal?at la cer.
Luk 24:52  Iar ei, închinându-se Lui, s-au întors în Ierusalim cu bucurie mare.
Luk 24:53  ?i erau în toata vremea în templu, laudând ?i binecuvântând pe Dumnezeu. Amin.


\end{document}