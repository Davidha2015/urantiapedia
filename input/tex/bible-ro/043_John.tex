\begin{document}

\title{John}

Joh 1:1  La început era Cuvântul ?i Cuvântul era la Dumnezeu ?i Dumnezeu era Cuvântul.
Joh 1:2  Acesta era întru început la Dumnezeu.
Joh 1:3  Toate prin El s-au facut; ?i fara El nimic nu s-a facut din ce s-a facut.
Joh 1:4  Întru El era via?a ?i via?a era lumina oamenilor.
Joh 1:5  ?i lumina lumineaza în întuneric ?i întunericul nu a cuprins-o.
Joh 1:6  Fost-a om trimis de la Dumnezeu, numele lui era Ioan.
Joh 1:7  Acesta a venit spre marturie, ca sa marturiseasca despre Lumina, ca to?i sa creada prin el.
Joh 1:8  Nu era el Lumina ci ca sa marturiseasca despre Lumina.
Joh 1:9  Cuvântul era Lumina cea adevarata care lumineaza pe tot omul, care vine în lume.
Joh 1:10  În lume era ?i lumea prin El s-a facut, dar lumea nu L-a cunoscut.
Joh 1:11  Întru ale Sale a venit, dar ai Sai nu L-au primit.
Joh 1:12  ?i celor câ?i L-au primit, care cred în numele Lui, le-a dat putere ca sa se faca fii ai lui Dumnezeu,
Joh 1:13  Care nu din sânge, nici din pofta trupeasca, nici din pofta barbateasca, ci de la Dumnezeu s-au nascut.
Joh 1:14  ?i Cuvântul S-a facut trup ?i S-a sala?luit între noi ?i am vazut slava Lui, slava ca a Unuia-Nascut din Tatal, plin de har ?i de adevar.
Joh 1:15  Ioan marturisea despre El ?i striga, zicând: Acesta era despre Care am zis: Cel care vine dupa mine a fost înaintea mea, pentru ca mai înainte de mine era.
Joh 1:16  ?i din plinatatea Lui noi to?i am luat, ?i har peste har.
Joh 1:17  Pentru ca Legea prin Moise s-a dat, iar harul ?i adevarul au venit prin Iisus Hristos.
Joh 1:18  Pe Dumnezeu nimeni nu L-a vazut vreodata; Fiul cel Unul-Nascut, Care este în sânul Tatalui, Acela L-a facut cunoscut.
Joh 1:19  ?i aceasta este marturia lui Ioan, când au trimis la El iudeii din Ierusalim, preo?i ?i levi?i, ca sa-l întrebe: Cine e?ti tu?
Joh 1:20  ?i el a marturisit ?i n-a tagaduit; ?i a marturisit: Nu sunt eu Hristosul.
Joh 1:21  ?i ei l-au întrebat: Dar cine e?ti? E?ti Ilie? Zis-a el: Nu sunt. E?ti tu Proorocul? ?i a raspuns: Nu.
Joh 1:22  Deci i-au zis: Cine e?ti? Ca sa dam un raspuns celor ce ne-au trimis. Ce spui tu despre tine însu?i?
Joh 1:23  El a zis: Eu sunt glasul celui ce striga în pustie: "Îndrepta?i calea Domnului", precum a zis Isaia proorocul.
Joh 1:24  ?i trimi?ii erau dintre farisei.
Joh 1:25  ?i l-au întrebat ?i i-au zis: De ce botezi deci, daca tu nu e?ti Hristosul, nici Ilie, nici Proorocul?
Joh 1:26  Ioan le-a raspuns, zicând: Eu botez cu apa; dar în mijlocul vostru Se afla Acela pe Care voi nu-L ?ti?i,
Joh 1:27  Cel care vine dupa mine, Care înainte de mine a fost ?i Caruia eu nu sunt vrednic sa-I dezleg cureaua încal?amintei.
Joh 1:28  Acestea se petreceau în Betabara, dincolo de Iordan, unde boteza Ioan.
Joh 1:29  A doua zi a vazut Ioan pe Iisus venind catre el ?i a zis: Iata Mielul lui Dumnezeu, Cel ce ridica pacatul lumii.
Joh 1:30  Acesta este despre Care eu am zis: Dupa mine vine un barbat, Care a fost înainte de mine, fiindca mai înainte de mine era,
Joh 1:31  ?i eu nu-L ?tiam; dar ca sa fie aratat lui Israel, de aceea am venit eu, botezând cu apa.
Joh 1:32  ?i a marturisit Ioan zicând: Am vazut Duhul coborându-Se, din cer, ca un porumbel ?i a ramas peste El.
Joh 1:33  ?i eu nu-L cuno?team pe El, dar Cel ce m-a trimis sa botez cu apa, Acela mi-a zis: Peste Care vei vedea Duhul coborându-Se ?i ramânând peste El, Acela este Cel ce boteaza cu Duh Sfânt.
Joh 1:34  ?i eu am vazut ?i am marturisit ca Acesta este Fiul lui Dumnezeu.
Joh 1:35  A doua zi iara?i statea Ioan ?i doi dintre ucenicii lui.
Joh 1:36  ?i privind pe Iisus, Care trecea, a zis: Iata Mielul lui Dumnezeu!
Joh 1:37  ?i cei doi ucenici l-au auzit când a spus aceasta ?i au mers dupa Iisus.
Joh 1:38  Iar Iisus, întorcându-Se ?i vazându-i ca merg dupa El, le-a zis: Ce cauta?i? Iar ei I-au zis: Rabi (care se tâlcuie?te: Înva?atorule), unde locuie?ti?
Joh 1:39  El le-a zis: Veni?i ?i ve?i vedea. Au mers deci ?i au vazut unde locuia; ?i au ramas la El în ziua aceea. Era ca la ceasul al zecelea.
Joh 1:40  Unul dintre cei doi care auzisera de la Ioan ?i venisera dupa Iisus era Andrei, fratele lui Simon Petru.
Joh 1:41  Acesta a gasit întâi pe Simon, fratele sau, ?i i-a zis: am gasit pe Mesia (care se tâlcuie?te: Hristos).
Joh 1:42  ?i l-a adus la Iisus. Iisus, privind la el, i-a zis: Tu e?ti Simon, fiul lui Iona; tu te vei numi Chifa (ce se tâlcuie?te: Petru).
Joh 1:43  A doua zi voia sa plece în Galileea ?i a gasit pe Filip. ?i i-a zis Iisus: Urmeaza-Mi.
Joh 1:44  Iar Filip era din Betsaida, din cetatea lui Andrei ?i a lui Petru.
Joh 1:45  Filip a gasit pe Natanael ?i i-a zis: Am aflat pe Acela despre Care au scris Moise în Lege ?i proorocii, pe Iisus, fiul lui Iosif din Nazaret.
Joh 1:46  ?i i-a zis Natanael: Din Nazaret poate fi ceva bun? Filip i-a zis: Vino ?i vezi.
Joh 1:47  Iisus a vazut pe Natanael venind catre El ?i a zis despre el: Iata, cu adevarat, israelit în care nu este vicle?ug.
Joh 1:48  Natanael I-a zis: De unde ma cuno?ti? A raspuns Iisus ?i i-a zis: Mai înainte de a te chema Filip, te-am vazut când erai sub smochin.
Joh 1:49  Raspunsu-I-a Natanael: Rabi, Tu e?ti Fiul lui Dumnezeu, Tu e?ti regele lui Israel.
Joh 1:50  Raspuns-a Iisus ?i i-a zis: Pentru ca ?i-am spus ca te-am vazut sub smochin, crezi? Mai mari decât acestea vei vedea.
Joh 1:51  ?i i-a zis: Adevarat, adevarat zic voua, de acum ve?i vedea cerul deschizându-se ?i pe îngerii lui Dumnezeu suindu-se ?i coborându-se peste Fiul Omului.
Joh 2:1  ?i a treia zi s-a facut nunta în Cana Galileii ?i era ?i mama lui Iisus acolo.
Joh 2:2  ?i a fost chemat ?i Iisus ?i ucenicii Sai la nunta.
Joh 2:3  ?i sfâr?indu-se vinul, a zis mama lui Iisus catre El: Nu mai au vin.
Joh 2:4  A zis ei Iisus: Ce ne prive?te pe mine ?i pe tine, femeie? Înca n-a venit ceasul Meu.
Joh 2:5  Mama Lui a zis celor ce slujeau: Face?i orice va va spune.
Joh 2:6  ?i erau acolo ?ase vase de piatra, puse pentru cura?irea iudeilor, care luau câte doua sau trei vedre.
Joh 2:7  Zis-a lor Iisus: Umple?i vasele cu apa. ?i le-au umplut pâna sus.
Joh 2:8  ?i le-a zis: Scoate?i acum ?i aduce?i nunului. Iar ei i-au dus.
Joh 2:9  ?i când nunul a gustat apa care se facuse vin ?i nu ?tia de unde este, ci numai slujitorii care scosesera apa ?tiau, a chemat nunul pe mire,
Joh 2:10  ?i i-a zis: Orice om pune întâi vinul cel bun ?i, când se ame?esc, pune pe cel mai slab. Dar tu ai ?inut vinul cel bun pâna acum.
Joh 2:11  Acest început al minunilor l-a facut Iisus în Cana Galileii ?i ?i-a aratat slava Sa; ?i ucenicii Sai au crezut în El.
Joh 2:12  Dupa aceasta S-a coborât în Capernaum, El ?i mama Sa ?i fra?ii ?i ucenicii Sai, ?i acolo n-a ramas decât pu?ine zile.
Joh 2:13  ?i erau aproape Pa?tile iudeilor, ?i Iisus S-a urcat la Ierusalim.
Joh 2:14  ?i a gasit ?ezând în templu pe cei ce vindeau boi ?i oi ?i porumbei ?i pe schimbatorii de bani.
Joh 2:15  ?i, facându-?i un bici din ?treanguri, i-a scos pe to?i afara din templu, ?i oile ?i boii, ?i schimbatorilor le-a varsat banii ?i le-a rasturnat mesele.
Joh 2:16  ?i celor ce vindeau porumbei le-a zis: Lua?i acestea de aici. Nu face?i casa Tatalui Meu casa de negustorie.
Joh 2:17  ?i ?i-au adus aminte ucenicii Lui ca este scris: "Râvna casei Tale ma mistuie".
Joh 2:18  Au raspuns deci iudeii ?i I-au zis: Ce semn ne ara?i ca faci acestea?
Joh 2:19  Iisus a raspuns ?i le-a zis: Darâma?i templul acesta ?i în trei zile îl voi ridica.
Joh 2:20  ?i au zis deci iudeii: În patruzeci ?i ?ase de ani s-a zidit templul acesta! ?i Tu îl vei ridica în trei zile?
Joh 2:21  Iar El vorbea despre templul trupului Sau.
Joh 2:22  Deci, când S-a sculat din mor?i, ucenicii Lui ?i-au adus aminte ca aceasta o spusese ?i au crezut Scripturii ?i cuvântului pe care Îl spusese Iisus.
Joh 2:23  ?i când era în Ierusalim, la sarbatoarea Pa?tilor, mul?i au crezut în numele Lui, vazând minunile pe care le facea.
Joh 2:24  Iar Iisus însu?i nu Se încredea în ei, pentru ca îi cuno?tea pe to?i.
Joh 2:25  ?i pentru ca nu avea nevoie sa-I marturiseasca cineva despre om, caci El însu?i cuno?tea ce era în om.
Joh 3:1  ?i era un om dintre farisei, care se numea Nicodim ?i care era frunta? al iudeilor.
Joh 3:2  Acesta a venit noaptea la Iisus ?i I-a zis: Rabi, ?tim ca de la Dumnezeu ai venit înva?ator; ca nimeni nu poate face aceste minuni, pe care le faci Tu, daca nu este Dumnezeu cu el.
Joh 3:3  Raspuns-a Iisus ?i i-a zis: Adevarat, adevarat zic ?ie: De nu se va na?te cineva de sus, nu va putea sa vada împara?ia lui Dumnezeu.
Joh 3:4  Iar Nicodim a zis catre El: Cum poate omul sa se nasca, fiind batrân? Oare, poate sa intre a doua oara în pântecele mamei sale ?i sa se nasca?
Joh 3:5  Iisus a raspuns: Adevarat, adevarat zic ?ie: De nu se va na?te cineva din apa ?i din Duh, nu va putea sa intre în împara?ia lui Dumnezeu.
Joh 3:6  Ce este nascut din trup, trup este; ?i ce este nascut din Duh, duh este.
Joh 3:7  Nu te mira ca ?i-am zis: Trebuie sa va na?te?i de sus.
Joh 3:8  Vântul sufla unde voie?te ?i tu auzi glasul lui, dar nu ?tii de unde vine, nici încotro se duce. Astfel este cu oricine e nascut din Duhul.
Joh 3:9  A raspuns Nicodim ?i i-a zis: Cum pot sa fie acestea?
Joh 3:10  Iisus a raspuns ?i i-a zis: Tu e?ti înva?atorul lui Israel ?i nu cuno?ti acestea?
Joh 3:11  Adevarat, adevarat zic ?ie, ca noi ceea ce ?tim vorbim ?i ce am vazut marturisim, dar marturia noastra nu o primi?i.
Joh 3:12  Daca v-am spus cele pamânte?ti ?i nu crede?i, cum ve?i crede cele cere?ti?
Joh 3:13  ?i nimeni nu s-a suit în cer, decât Cel ce S-a coborât din cer, Fiul Omului, Care este în cer.
Joh 3:14  ?i dupa cum Moise a înal?at ?arpele în pustie, a?a trebuie sa se înal?e Fiul Omului,
Joh 3:15  Ca tot cel ce crede în El sa nu piara, ci sa aiba via?a ve?nica.
Joh 3:16  Caci Dumnezeu a?a a iubit lumea, încât pe Fiul Sau Cel Unul-Nascut L-a dat ca oricine crede în El sa nu piara, ci sa aiba via?a ve?nica.
Joh 3:17  Caci n-a trimis Dumnezeu pe Fiul Sau în lume ca sa judece lumea, ci ca sa se mântuiasca, prin El, lumea.
Joh 3:18  Cel ce crede în El nu este judecat, iar cel ce nu crede a ?i fost judecat, fiindca nu a crezut în numele Celui Unuia-Nascut, Fiul lui Dumnezeu.
Joh 3:19  Iar aceasta este judecata, ca Lumina a venit în lume ?i oamenii au iubit întunericul mai mult decât Lumina. Caci faptele lor erau rele.
Joh 3:20  Ca oricine face rele ura?te Lumina ?i nu vine la Lumina, pentru ca faptele lui sa nu se vadeasca.
Joh 3:21  Dar cel care lucreaza adevarul vine la Lumina, ca sa se arate faptele lui, ca în Dumnezeu sunt savâr?ite.
Joh 3:22  Dupa acestea a venit Iisus ?i ucenicii Lui în pamântul Iudeii ?i statea acolo ?i boteza.
Joh 3:23  ?i boteza ?i Ioan în Enom, aproape de Salim, ca erau acolo ape multe ?i veneau ?i se botezau.
Joh 3:24  Caci Ioan nu fusese înca aruncat în închisoare.
Joh 3:25  ?i s-a iscat o neîn?elegere între ucenicii lui Ioan ?i un iudeu, asupra cura?irii.
Joh 3:26  ?i au venit la Ioan ?i i-au zis: Rabi, Acela care era cu tine, dincolo de Iordan, ?i despre Care tu ai marturisit, iata El boteaza ?i to?i se duc la El.
Joh 3:27  Ioan a raspuns ?i a zis: Nu poate un om sa ia nimic, daca nu i s-a dat lui din cer.
Joh 3:28  Voi în?iva îmi sunte?i martori ca am zis: Nu sunt eu Hristosul, ci sunt trimis înaintea Lui.
Joh 3:29  Cel ce are mireasa este mire, iar prietenul mirelui, care sta ?i asculta pe mire, se bucura cu bucurie de glasul lui. Deci aceasta bucurie a mea s-a împlinit.
Joh 3:30  Acela trebuie sa creasca, iar eu sa ma mic?orez.
Joh 3:31  Cel ce vine de sus este deasupra tuturor; cel ce este de pe pamânt pamântesc este ?i de pe pamânt graie?te. Cel ce vine din cer este deasupra tuturor.
Joh 3:32  ?i ce a vazut ?i a auzit, aceea marturise?te, dar marturia Lui nu o prime?te nimeni.
Joh 3:33  Cel ce a primit marturia Lui a pecetluit ca Dumnezeu este adevarat.
Joh 3:34  Caci cel pe care l-a trimis Dumnezeu vorbe?te cuvintele lui Dumnezeu, pentru ca Dumnezeu nu da Duhul cu masura.
Joh 3:35  Tatal iube?te pe Fiul ?i toate le-a dat în mâna Lui.
Joh 3:36  Cel ce crede în Fiul are via?a ve?nica, iar cel ce nu asculta de Fiul nu va vedea via?a, ci mânia lui Dumnezeu ramâne peste el.
Joh 4:1  Deci când a cunoscut Iisus ca fariseii au auzit ca El face ?i boteaza mai mul?i ucenici ca Ioan,
Joh 4:2  De?i Iisus nu boteza El, ci ucenicii Lui,
Joh 4:3  A lasat Iudeea ?i S-a dus iara?i în Galileea.
Joh 4:4  ?i trebuia sa treaca prin Samaria.
Joh 4:5  Deci a venit la o cetate a Samariei, numita Sihar, aproape de locul pe care Iacov l-a dat lui Iosif, fiul sau;
Joh 4:6  ?i era acolo fântâna lui Iacov. Iar Iisus, fiind ostenit de calatorie, S-a a?ezat lânga fântâna ?i era ca la al ?aselea ceas.
Joh 4:7  Atunci a venit o femeie din Samaria sa scoata apa. Iisus i-a zis: Da-Mi sa beau.
Joh 4:8  Caci ucenicii Lui se dusesera în cetate, ca sa cumpere merinde.
Joh 4:9  Femeia samarineanca I-a zis: Cum Tu, care e?ti iudeu, ceri sa bei de la mine, care sunt femeie samarineanca? Pentru ca iudeii nu au amestec cu samarinenii.
Joh 4:10  Iisus a raspuns ?i i-a zis: Daca ai fi ?tiut darul lui Dumnezeu ?i Cine este Cel ce-?i zice: Da-Mi sa beau, tu ai fi cerut de la El, ?i ?i-ar fi dat apa vie.
Joh 4:11  Femeia I-a zis: Doamne, nici galeata nu ai, ?i fântâna e adânca; de unde, dar, ai apa cea vie?
Joh 4:12  Nu cumva e?ti Tu mai mare decât parintele nostru Iacov, care ne-a dat aceasta fântâna ?i el însu?i a baut din ea ?i fiii lui ?i turmele lui?
Joh 4:13  Iisus a raspuns ?i i-a zis: Oricine bea din apa aceasta va înseta iara?i;
Joh 4:14  Dar cel ce va bea din apa pe care i-o voi da Eu nu va mai înseta în veac, caci apa pe care i-o voi da Eu se va face în el izvor de apa curgatoare spre via?a ve?nica.
Joh 4:15  Femeia a zis catre El: Doamne, da-mi aceasta apa ca sa nu mai însetez, nici sa mai vin aici sa scot.
Joh 4:16  Iisus i-a zis: Mergi ?i cheama pe barbatul tau ?i vino aici.
Joh 4:17  Femeia a raspuns ?i a zis: N-am barbat. Iisus i-a zis: Bine ai zis ca nu ai barbat.
Joh 4:18  Caci cinci barba?i ai avut ?i cel pe care îl ai acum nu-?i este barbat. Aceasta adevarat ai spus.
Joh 4:19  Femeia I-a zis: Doamne, vad ca Tu e?ti prooroc.
Joh 4:20  Parin?ii no?tri s-au închinat pe acest munte, iar voi zice?i ca în Ierusalim este locul unde trebuie sa ne închinam.
Joh 4:21  ?i Iisus i-a zis: Femeie, crede-Ma ca vine ceasul când nici pe muntele acesta, nici în Ierusalim nu va ve?i închina Tatalui.
Joh 4:22  Voi va închina?i caruia nu ?ti?i; noi ne închinam Caruia ?tim, pentru ca mântuirea din iudei este.
Joh 4:23  Dar vine ceasul ?i acum este, când adevara?ii închinatori se vor închina Tatalui în duh ?i în adevar, ca ?i Tatal astfel de închinatori î?i dore?te.
Joh 4:24  Duh este Dumnezeu ?i cei ce I se închina trebuie sa i se închine în duh ?i în adevar.
Joh 4:25  I-a zis femeia: ?tim ca va veni Mesia care se cheama Hristos; când va veni, Acela ne va vesti noua toate.
Joh 4:26  Iisus i-a zis: Eu sunt, Cel ce vorbesc cu tine.
Joh 4:27  Dar atunci au sosit ucenicii Lui. ?i se mirau ca vorbea cu o femeie. Însa nimeni n-a zis: Ce o întrebi, sau: Ce vorbe?ti cu ea?
Joh 4:28  Iar femeia ?i-a lasat galeata ?i s-a dus în cetate ?i a zis oamenilor:
Joh 4:29  Veni?i de vede?i un om care mi-a spus toate câte am facut. Nu cumva aceasta este Hristosul?
Joh 4:30  ?i au ie?it din cetate ?i veneau catre El.
Joh 4:31  Între timp, ucenicii Lui Îl rugau, zicând: Înva?atorule, manânca.
Joh 4:32  Iar El le-a zis: Eu am de mâncat o mâncare pe care voi nu o ?ti?i.
Joh 4:33  Ziceau deci ucenicii între ei: Nu cumva I-a adus cineva sa manânce?
Joh 4:34  Iisus le-a zis: Mâncarea Mea este sa fac voia Celui ce M-a trimis pe Mine ?i sa savâr?esc lucrul Lui.
Joh 4:35  Nu zice?i voi ca mai sunt patru luni ?i vine seceri?ul? Iata zic voua: Ridica?i ochii vo?tri ?i privi?i holdele ca sunt albe pentru seceri?.
Joh 4:36  Iar cel ce secera prime?te plata ?i aduna roade spre via?a ve?nica, ca sa se bucure împreuna ?i cel ce seamana ?i cel ce secera.
Joh 4:37  Caci în aceasta se adevere?te cuvântul: Ca unul este semanatorul ?i altul seceratorul.
Joh 4:38  Eu v-am trimis sa secera?i ceea ce voi n-a?i muncit; al?ii au muncit ?i voi a?i intrat în munca lor.
Joh 4:39  ?i mul?i samarineni din cetatea aceea au crezut în El, pentru cuvântul femeii care marturisea: Mi-a spus toate câte am facut.
Joh 4:40  Deci, dupa ce au venit la El, samarinenii Îl rugau sa ramâna la ei. ?i a ramas acolo doua zile.
Joh 4:41  ?i cu mult mai mul?i au crezut pentru cuvântul Lui,
Joh 4:42  Iar femeii i-au zis: Credem nu numai pentru cuvântul tau, caci noi în?ine am auzit ?i ?tim ca Acesta este cu adevarat Hristosul, Mântuitorul lumii.
Joh 4:43  ?i dupa cele doua zile, a plecat de acolo în Galileea.
Joh 4:44  Caci Iisus însu?i a marturisit ca un prooroc nu e cinstit în ?ara lui.
Joh 4:45  Deci, când a venit în Galileea, L-au primit galileenii, cei ce vazusera toate câte facuse El în Ierusalim, la sarbatoare, caci ?i ei venisera la sarbatoare.
Joh 4:46  Deci iara?i a mers în Cana Galileii, unde prefacuse apa în vin. ?i era un slujitor regesc, al carui fiu era bolnav în Capernaum.
Joh 4:47  Acesta, auzind ca Iisus a venit din Iudeea în Galileea, s-a dus la El ?i Îl ruga sa Se coboare ?i sa vindece pe fiul lui, ca era gata sa moara.
Joh 4:48  Deci Iisus i-a zis: Daca nu ve?i vedea semne ?i minuni, nu ve?i crede.
Joh 4:49  Slujitorul regesc a zis catre El: Doamne, coboara-Te înainte de a muri copilul meu.
Joh 4:50  Iisus i-a zis: Mergi, copilul tau traie?te. ?i omul a crezut cuvântului pe care i l-a spus Iisus ?i a plecat.
Joh 4:51  Iar pe când cobora, slugile lui, l-au întâmpinat spunându-i ca fiul lui traie?te.
Joh 4:52  ?i cerea, deci, sa afle de la ele ceasul în care i-a fost mai bine. Deci i-au spus ca ieri, în ceasul al ?aptelea, l-au lasat frigurile.
Joh 4:53  A?adar tatal a cunoscut ca în ceasul acela a fost în care Iisus i-a zis: Fiul tau traie?te. ?i a crezut el ?i toata casa lui.
Joh 4:54  Aceasta este a doua minune pe care a facut-o iara?i Iisus, venind din Iudeea în Galileea.
Joh 5:1  Dupa acestea era o sarbatoare a iudeilor ?i Iisus S-a suit la Ierusalim.
Joh 5:2  Iar în Ierusalim, lânga Poarta Oilor, era o scaldatoare, care pe evreie?te se nume?te Vitezda, având cinci pridvoare.
Joh 5:3  În acestea zaceau mul?ime de bolnavi, orbi, ?chiopi, usca?i, a?teptând mi?carea apei.
Joh 5:4  Caci un înger al Domnului se cobora la vreme în scaldatoare ?i tulbura apa ?i cine intra întâi, dupa tulburarea apei, se facea sanatos, de orice boala era ?inut.
Joh 5:5  ?i era acolo un om, care era bolnav de treizeci ?i opt de ani.
Joh 5:6  Iisus, vazându-l pe acesta zacând ?i ?tiind ca este a?a înca de multa vreme, i-a zis: Voie?ti sa te faci sanatos?
Joh 5:7  Bolnavul I-a raspuns: Doamne, nu am om, care sa ma arunce în scaldatoare, când se tulbura apa; ca, pâna când vin eu, altul se coboara înaintea mea.
Joh 5:8  Iisus i-a zis: Scoala-te, ia-?i patul tau ?i umbla.
Joh 5:9  ?i îndata omul s-a facut sanatos, ?i-a luat patul ?i umbla. Dar în ziua aceea era sâmbata.
Joh 5:10  Deci ziceau iudeii catre cel vindecat: Este zi de sâmbata ?i nu-?i este îngaduit sa-?i iei patul.
Joh 5:11  El le-a raspuns: Cel ce m-a facut sanatos, Acela mi-a zis: Ia-?i patul ?i umbla.
Joh 5:12  Ei l-au întrebat: Cine este omul care ?i-a zis: Ia-?i patul tau ?i umbla?
Joh 5:13  Iar cel vindecat nu ?tia cine este, caci Iisus se daduse la o parte din mul?imea care era în acel loc.
Joh 5:14  Dupa aceasta Iisus l-a aflat în templu ?i i-a zis: Iata ca te-ai facut sanatos. De acum sa nu mai pacatuie?ti, ca sa nu-?i fie ceva mai rau.
Joh 5:15  Atunci omul a plecat ?i a spus iudeilor ca Iisus este Cel ce l-a facut sanatos.
Joh 5:16  Pentru aceasta iudeii prigoneau pe Iisus ?i cautau sa-L omoare, ca facea aceasta sâmbata.
Joh 5:17  Dar Iisus le-a raspuns: Tatal Meu pâna acum lucreaza; ?i Eu lucrez.
Joh 5:18  Deci pentru aceasta cautau mai mult iudeii sa-L omoare, nu numai pentru ca dezlega sâmbata, ci ?i pentru ca zicea ca Dumnezeu este Tatal Sau, facându-Se pe Sine deopotriva cu Dumnezeu.
Joh 5:19  A raspuns Iisus ?i le-a zis: Adevarat, adevarat zic voua: Fiul nu poate sa faca nimic de la Sine, daca nu va vedea pe Tatal facând; caci cele ce face Acela, acestea le face ?i Fiul întocmai.
Joh 5:20  Ca Tatal iube?te pe Fiul ?i-I arata toate câte face El ?i lucruri mai mari decât acestea va arata Lui, ca voi sa va mira?i.
Joh 5:21  Caci, dupa cum Tatal scoala pe cei mor?i ?i le da via?a, tot a?a ?i Fiul da via?a celor ce voie?te.
Joh 5:22  Tatal nu judeca pe nimeni, ci toata judecata a dat-o Fiului.
Joh 5:23  Ca to?i sa cinsteasca pe Fiul cum cinstesc pe Tatal. Cine nu cinste?te pe Fiul nu cinste?te pe Tatal care L-a trimis.
Joh 5:24  Adevarat, adevarat zic voua: Cel ce asculta cuvântul Meu ?i crede în Cel ce M-a trimis are via?a ve?nica ?i la judecata nu va veni, ci s-a mutat de la moarte la via?a.
Joh 5:25  Adevarat, adevarat zic voua, ca vine ceasul ?i acum este, când mor?ii vor auzi glasul Fiului lui Dumnezeu ?i cei ce vor auzi vor învia.
Joh 5:26  Caci precum Tatal are via?a în Sine, a?a I-a dat ?i Fiului sa aiba via?a în Sine;
Joh 5:27  ?i I-a dat putere sa faca judecata, pentru ca este Fiul Omului.
Joh 5:28  Nu va mira?i de aceasta; caci vine ceasul când to?i cei din morminte vor auzi glasul Lui,
Joh 5:29  ?i vor ie?i, cei ce au facut cele bune spre învierea vie?ii ?i cei ce au facut cele rele spre învierea osândirii.
Joh 5:30  Eu nu pot sa fac de la Mine nimic; precum aud, judec; dar judecata Mea este dreapta, pentru ca nu caut la voia Mea, ci voia Celui care M-a trimis.
Joh 5:31  Daca marturisesc Eu despre mine însumi, marturia Mea nu este adevarata.
Joh 5:32  Altul marturise?te despre Mine; ?i ?tiu ca adevarata este marturia pe care o marturise?te despre Mine.
Joh 5:33  Voi a?i trimis la Ioan, ?i el a marturisit adevarul.
Joh 5:34  Dar Eu nu de la om iau marturia, ci spun aceasta ca sa va mântui?i.
Joh 5:35  Acela (Ioan) era faclia care arde ?i lumineaza, ?i voi a?i voit sa va veseli?i o clipa în lumina lui.
Joh 5:36  Iar Eu am marturie mai mare decât a lui Ioan; caci lucrurile pe care Mi le-a dat Tatal ca sa le savâr?esc, lucrurile acestea pe care le fac Eu, marturisesc despre Mine ca Tatal M-a trimis.
Joh 5:37  ?i Tatal care M-a trimis, Acela a marturisit despre Mine. Nici glasul Lui nu l-a?i vazut vreodata, nici fa?a Lui nu a?i vazut-o;
Joh 5:38  ?i cuvântul Lui nu sala?luie?te în voi, pentru ca voi nu crede?i în Cel pe care l-a trimis Acela.
Joh 5:39  Cerceta?i Scripturile, ca socoti?i ca în ele ave?i via?a ve?nica. ?i acelea sunt care marturisesc despre Mine.
Joh 5:40  ?i nu voi?i sa veni?i la Mine, ca sa ave?i via?a!
Joh 5:41  Slava de la oameni nu primesc;
Joh 5:42  Dar v-am cunoscut ca n-ave?i în voi dragostea lui Dumnezeu.
Joh 5:43  Eu am venit în numele Tatalui Meu ?i voi nu Ma primi?i; daca va veni altul în numele sau, pe acela îl ve?i primi.
Joh 5:44  Cum pute?i voi sa crede?i, când primi?i slava unii de la al?ii ?i slava care vine de la unicul Dumnezeu nu o cauta?i?
Joh 5:45  Sa nu socoti?i ca Eu va voi învinui la Tatal; cel ce va învinuie?te este Moise, în care voi a?i nadajduit.
Joh 5:46  Ca daca a?i fi crezut lui Moise, a?i fi crezut ?i Mie, caci despre Mine a scris acela.
Joh 5:47  Iar daca celor scrise de el nu crede?i, cum ve?i crede în cuvintele Mele?
Joh 6:1  Dupa acestea, Iisus S-a dus dincolo de marea Galileii, în par?ile Tiberiadei.
Joh 6:2  ?i a mers dupa El mul?ime multa, pentru ca vedeau minunile pe care le facea cu cei bolnavi.
Joh 6:3  ?i S-a suit Iisus în munte ?i a ?ezut acolo cu ucenicii Sai.
Joh 6:4  ?i era aproape Pa?tile, sarbatoarea iudeilor.
Joh 6:5  Deci ridicându-?i Iisus ochii ?i vazând ca mul?ime multa vine catre El, a zis catre Filip: De unde vom cumpara pâine, ca sa manânce ace?tia?
Joh 6:6  Iar aceasta o zicea ca sa-l încerce, ca El ?tia ce avea sa faca.
Joh 6:7  ?i Filip i-a raspuns: Pâini de doua sute de dinari nu le vor ajunge, ca sa ia fiecare câte pu?in.
Joh 6:8  ?i a zis Lui unul din ucenici, Andrei, fratele lui Simon Petru:
Joh 6:9  Este aici un baiat care are cinci pâini de orz ?i doi pe?ti. Dar ce sunt acestea la atâ?ia?
Joh 6:10  ?i a zis Iisus: Face?i pe oameni sa se a?eze. ?i era iarba multa în acel loc. Deci au ?ezut barba?ii în numar ca la cinci mii.
Joh 6:11  ?i Iisus a luat pâinile ?i, mul?umind, a dat ucenicilor, iar ucenicii celor ce ?edeau; asemenea ?i din pe?ti, cât au voit.
Joh 6:12  Iar dupa ce s-au saturat, a zis ucenicilor Sai: aduna?i farâmiturile ce au ramas, ca sa nu se piarda ceva.
Joh 6:13  Deci au adunat ?i au umplut douasprezece co?uri de farâmituri, care au ramas de la cei ce au mâncat din cele cinci pâini de orz.
Joh 6:14  Iar oamenii vazând minunea pe care a facut-o, ziceau: Acesta este într-adevar Proorocul, Care va sa vina în lume.
Joh 6:15  Cunoscând deci Iisus ca au sa vina ?i sa-L ia cu sila, ca sa-L faca rege, S-a dus iara?i în munte, El singur.
Joh 6:16  ?i când s-a facut sera, ucenicii Lui s-au coborât la mare.
Joh 6:17  ?i intrând în corabie, mergeau spre Capernaum, dincolo de mare. ?i s-a facut întuneric ?i Iisus înca nu venise la ei,
Joh 6:18  ?i suflând vânt mare, marea se întarâta.
Joh 6:19  Dupa ce au vâslit deci ca la douazeci ?i cinci sau treizeci de stadii, au vazut pe Iisus umblând pe apa ?i apropiindu-Se de corabie, ei s-au înfrico?at.
Joh 6:20  Iar El le-a zis: Eu sunt; nu va teme?i!
Joh 6:21  Deci voiau sa-L ia în corabie, ?i îndata corabia a sosit la ?armul la care mergeau.
Joh 6:22  A doua zi, mul?imea, care sta de cealalta parte a marii, a vazut ca nu era acolo decât numai o corabie mai mica ?i ca Iisus nu intrase în corabie împreuna cu ucenicii Sai, ci plecasera numai ucenicii Lui.
Joh 6:23  ?i alte corabii mai mic au venit din Tiberiada în apropiere de locul unde ei mâncasera pâinea, dupa ce Domnul mul?umise.
Joh 6:24  Deci, când a vazut mul?imea ca Iisus nu este acolo, nici ucenicii Lui, au intrat ?i ei în corabiile cele mici ?i au venit în Capernaum, cautându-L pe Iisus.
Joh 6:25  ?i gasindu-L dincolo de mare, I-au zis: Înva?atorule, când ai venit aici?
Joh 6:26  Iisus le-a raspuns ?i a zis: Adevarat, adevarat zic voua: Ma cauta?i nu pentru ca a?i vazut minuni, ci pentru ca a?i mâncat din pâini ?i v-a?i saturat.
Joh 6:27  Lucra?i nu pentru mâncarea cea pieritoare, ci pentru mâncarea ce ramâne spre via?a ve?nica ?i pe care o va da voua Fiul Omului, caci pe El L-a pecetluit Dumnezeu-Tatal.
Joh 6:28  Deci au zis catre El: Ce sa facem, ca sa savâr?im lucrarile lui Dumnezeu?
Joh 6:29  Iisus a raspuns ?i le-a zis: Aceasta este lucrarea lui Dumnezeu, ca sa crede?i în Acela pe Care El L-a trimis.
Joh 6:30  Deci I-au zis: Dar ce minune faci Tu, ca sa vedem ?i sa credem în Tine? Ce lucrezi?
Joh 6:31  Parin?ii no?tri au mâncat mana în pustie, precum este scris: "Pâine din cer le-a dat lor sa manânce".
Joh 6:32  Deci Iisus le-a zis: Adevarat, adevarat zic voua: Nu Moise v-a dat pâinea cea din cer; ci Tatal Meu va da din cer pâinea cea adevarata.
Joh 6:33  Caci pâinea lui Dumnezeu este cea care se coboara din cer ?i care da via?a lumii.
Joh 6:34  Deci au zis catre El: Doamne, da-ne totdeauna pâinea aceasta.
Joh 6:35  ?i Iisus le-a zis: Eu sunt pâinea vie?ii; cel ce vine la Mine nu va flamânzi ?i cel ce va crede în Mine nu va înseta niciodata.
Joh 6:36  Dar am spus voua ca M-a?i ?i vazut ?i nu crede?i.
Joh 6:37  Tot ce-Mi da Tatal, va veni la Mine; ?i pe cel ce vine la Mine nu-l voi scoate afara;
Joh 6:38  Pentru ca M-am coborât din cer, nu ca sa fac voia mea, ci voia Celui ce M-a trimis pe Mine.
Joh 6:39  ?i aceasta este voia Celui ce M-a trimis, ca din to?i pe care Mi i-a dat Mie sa nu pierd nici unul, ci sa-i înviez pe ei în ziua cea de apoi.
Joh 6:40  Ca aceasta este voia Tatalui Meu, ca oricine vede pe Fiul ?i crede în El sa aiba via?a ve?nica ?i Eu îl voi învia în ziua cea de apoi.
Joh 6:41  Deci iudeii murmurau împotriva Lui, fiindca zisese: Eu sunt pâinea ce s-a coborât din cer.
Joh 6:42  ?i ziceau: Au nu este Acesta Iisus, fiul lui Iosif, ?i nu ?tim noi pe tatal Sau ?i pe mama Sa? Cum spune El acum: M-am coborât din cer?
Joh 6:43  Iisus a raspuns ?i le-a zis: Nu murmura?i între voi.
Joh 6:44  Nimeni nu poate sa vina la Mine, daca nu-l va trage Tatal, Care M-a trimis, ?i Eu îl voi învia în ziua de apoi.
Joh 6:45  Scris este în prooroci: "?i vor fi to?i înva?a?i de Dumnezeu". Deci oricine a auzit ?i a înva?at de la Tatal la Mine vine.
Joh 6:46  Nu doar ca pe Tatal l-a vazut cineva, decât numai Cel ce este la Dumnezeu; Acesta L-a vazut pe Tatal.
Joh 6:47  Adevarat, adevarat zic voua: Cel ce crede în Mine are via?a ve?nica.
Joh 6:48  Eu sunt pâinea vie?ii.
Joh 6:49  Parin?ii vo?tri au mâncat mana în pustie ?i au murit.
Joh 6:50  Pâinea care se coboara din cer este aceea din care, daca manânca cineva, nu moare.
Joh 6:51  Eu sunt pâinea cea vie, care s-a pogorât din cer. Cine manânca din pâinea aceasta viu va fi în veci. Iar pâinea pe care Eu o voi da pentru via?a lumii este trupul Meu.
Joh 6:52  Deci iudeii se certau între ei, zicând: Cum poate Acesta sa ne dea trupul Lui sa-l mâncam?
Joh 6:53  ?i le-a zis Iisus: Adevarat, adevarat zic voua, daca nu ve?i mânca trupul Fiului Omului ?i nu ve?i bea sângele Lui, nu ve?i avea via?a în voi.
Joh 6:54  Cel ce manânca trupul Meu ?i bea sângele Meu are via?a ve?nica, ?i Eu îl voi învia în ziua cea de apoi.
Joh 6:55  Trupul este adevarata mâncare ?i sângele Meu, adevarata bautura.
Joh 6:56  Cel ce manânca trupul Meu ?i bea sângele Meu ramâne întru Mine ?i Eu întru el.
Joh 6:57  Precum M-a trimis pe Mine Tatal cel viu ?i Eu viez pentru Tatal, ?i cel ce Ma manânca pe Mine va trai prin Mine.
Joh 6:58  Aceasta este pâinea care s-a pogorât din cer, nu precum au mâncat parin?ii vo?tri mana ?i au murit. Cel ce manânca aceasta pâine va trai în veac.
Joh 6:59  Acestea le-a zis pe când înva?a în sinagoga din Capernaum.
Joh 6:60  Deci mul?i din ucenicii Lui, auzind, au zis: Greu este cuvântul acesta! Cine poate sa-l asculte?
Joh 6:61  Iar Iisus, ?tiind în Sine ca ucenicii Lui murmura împotriva Lui, le-a zis: Va sminte?te aceasta?
Joh 6:62  Daca ve?i vedea pe Fiul Omului, suindu-Se acolo unde era mai înainte?
Joh 6:63  Duhul este cel ce da via?a; trupul nu folose?te la nimic. Cuvintele pe care vi le-am spus sunt duh ?i sunt via?a.
Joh 6:64  Dar sunt unii dintre voi care nu cred. Caci Iisus ?tia de la început cine sunt cei ce nu cred ?i cine este cel care Îl va vinde.
Joh 6:65  ?i zicea: De aceea am spus voua ca nimeni nu poate sa vina la Mine, daca nu-i este dat de la Tatal.
Joh 6:66  ?i de atunci mul?i dintre ucenicii Sai s-au dus înapoi ?i nu mai umblau cu El.
Joh 6:67  Deci a zis Iisus celor doisprezece: Nu vre?i ?i voi sa va duce?i?
Joh 6:68  Simon Petru I-a raspuns: Doamne, la cine ne vom duce? Tu ai cuvintele vie?ii celei ve?nice.
Joh 6:69  ?i noi am crezut ?i am cunoscut ca Tu e?ti Hristosul, Fiul Dumnezeului Celui viu.
Joh 6:70  Le-a raspuns Iisus: Oare, nu v-am ales Eu pe voi, cei doisprezece? ?i unul dintre voi este diavol!
Joh 6:71  Iar El zicea de Iuda al lui Simon Iscarioteanul, caci acesta, unul din cei doisprezece fiind, avea sa-L vânda.
Joh 7:1  ?i dupa aceea mergea Iisus prin Galileea, caci nu voia sa mearga prin Iudeea, deoarece iudeii cautau sa-L ucida.
Joh 7:2  ?i era aproape sarbatoarea iudaica a corturilor.
Joh 7:3  Au zis deci catre El fra?ii Lui: Pleaca de aici ?i du-Te în Iudeea, pentru ca ?i ucenicii Tai sa vada lucrurile pe care Tu le faci.
Joh 7:4  Caci nimeni nu lucreaza ceva în ascuns, ci cauta sa se faca cunoscut. Daca faci acestea, arata-Te pe Tine lumii.
Joh 7:5  Pentru ca nici fra?ii Lui nu credeau în El.
Joh 7:6  Deci le-a zis Iisus: Vremea Mea înca n-a sosit; dar vremea voastra totdeauna este gata.
Joh 7:7  Pe voi lumea nu poate sa va urasca, dar pe Mine Ma ura?te, pentru ca Eu marturisesc despre ea ca lucrurile ei sunt rele.
Joh 7:8  Voi duce?i-va la sarbatoare; Eu nu merg la sarbatoarea aceasta, caci vremea Mea nu s-a împlinit înca.
Joh 7:9  Acestea spunându-le, a ramas în Galileea.
Joh 7:10  Dar dupa ce fra?ii Sai s-au dus la sarbatoare, atunci S-a suit ?i El, dar nu pe fa?a, ci pe ascuns.
Joh 7:11  În timpul sarbatorii iudeii Îl cautau ?i ziceau: Unde este Acela?
Joh 7:12  ?i cârtire multa era despre El în mul?ime; unii ziceau: Este bun; iar al?ii ziceau: Nu, ci amage?te mul?imea.
Joh 7:13  Totu?i, de frica iudeilor, nimeni nu vorbea despre El pe fa?a.
Joh 7:14  Iar la jumatatea praznicului Iisus S-a suit în templu ?i înva?a.
Joh 7:15  ?i iudeii se mirau zicând: Cum ?tie Acesta carte fara sa fi înva?at?
Joh 7:16  Deci le-a raspuns Iisus ?i a zis: Înva?atura Mea nu este a Mea, ci a Celui ce M-a trimis.
Joh 7:17  De vrea cineva sa faca voia Lui, va cunoa?te despre înva?atura aceasta daca este de la Dumnezeu sau daca Eu vorbesc de la Mine Însumi.
Joh 7:18  Cel care vorbe?te de la sine î?i cauta slava sa; iar cel care cauta slava celui ce l-a trimis pe el, acela este adevarat ?i nedreptate nu este în el.
Joh 7:19  Oare nu Moise v-a dat Legea? ?i nimeni dintre voi nu ?ine Legea. De ce cauta?i sa Ma ucide?i?
Joh 7:20  ?i mul?imea a raspuns: Ai demon. Cine cauta sa te ucida?
Joh 7:21  Iisus a raspuns ?i le-a zis: Un lucru am facut ?i to?i va mira?i.
Joh 7:22  De aceea Moise v-a dat taierea împrejur, nu ca este de la Moise, ci de la parin?i, ?i sâmbata taia?i împrejur pe om.
Joh 7:23  Daca omul prime?te taierea împrejur sâmbata, ca sa nu se strice Legea lui Moise, va mânia?i pe Mine ca am facut sâmbata un om întreg sanatos?
Joh 7:24  Nu judeca?i dupa înfa?i?are, ci judeca?i judecata dreapta.
Joh 7:25  Deci ziceau unii dintre ierusalimiteni: Nu este, oare, Acesta pe care-L cautau sa-L ucida?
Joh 7:26  ?i iata ca vorbe?te pe fa?a ?i ei nu-I zic nimic. Nu cumva capeteniile au cunoscut cu adevarat ca Acesta e Hristos?
Joh 7:27  Dar pe Acesta Îl ?tim de unde este. Însa Hristosul, când va veni, nimeni nu ?tie de unde este.
Joh 7:28  Deci a strigat Iisus în templu, înva?ând ?i zicând: ?i pe Mine Ma ?ti?i ?i ?ti?i de unde sunt; ?i Eu n-am venit de la Mine, dar adevarat este Cel ce M-a trimis pe Mine ?i pe Care voi nu-L ?ti?i.
Joh 7:29  Eu Îl ?tiu pe El, caci de la El sunt ?i El M-a trimis pe Mine.
Joh 7:30  Deci cautau sa-L prinda, dar nimeni n-a pus mâna pe El, pentru ca nu venise înca ceasul Lui.
Joh 7:31  Dar mul?i din mul?ime au crezut în El ?i ziceau: Hristosul când va veni va face El minuni mai multe decât a facut Acesta?
Joh 7:32  Au auzit fariseii mul?imea murmurând acestea despre El ?i au trimis arhiereii ?i fariseii slujitori ca sa-L prinda.
Joh 7:33  Dar Iisus le-a zis: Pu?in timp mai sunt cu voi ?i Ma duc la Cel ce M-a trimis.
Joh 7:34  Ma ve?i cauta ?i nu Ma ve?i gasi; ?i unde sunt Eu, voi nu pute?i sa veni?i.
Joh 7:35  Deci au zis iudeii, între ei: Unde are sa Se duca Acesta, ca noi sa nu-L gasim? Nu cumva va merge la cei împra?tia?i printre elini ?i va înva?a pe elini?
Joh 7:36  Ce înseamna acest cuvânt pe care l-a spus: Ma ve?i cauta ?i nu Ma ve?i gasi ?i unde sunt Eu, voi nu pute?i sa veni?i?
Joh 7:37  Iar în ziua cea din urma - ziua cea mare a sarbatorii - Iisus a stat între ei ?i a strigat, zicând: Daca înseteaza cineva, sa vina la Mine ?i sa bea.
Joh 7:38  Cel ce crede în Mine, precum a zis Scriptura: râuri de apa vie vor curge din pântecele lui.
Joh 7:39  Iar aceasta a zis-o despre Duhul pe Care aveau sa-L primeasca acei ce cred în El. Caci înca nu era (dat) Duhul, pentru ca Iisus înca nu fusese preaslavit.
Joh 7:40  Deci din mul?ime, auzind cuvintele acestea, ziceau: Cu adevarat, Acesta este Proorocul.
Joh 7:41  Iar al?ii ziceau: Acesta este Hristosul. Iar al?ii ziceau: Nu cumva din Galileea va sa vina Hristos?
Joh 7:42  N-a zis, oare, Scriptura ca Hristos va sa vina din samân?a lui David ?i din Betleem, cetatea lui David?
Joh 7:43  ?i s-a facut dezbinare în mul?ime pentru El.
Joh 7:44  ?i unii dintre ei voiau sa-L prinda, dar nimeni n-a pus mâinile pe El.
Joh 7:45  Deci slugile au venit la arhierei ?i farisei, ?i le-au zis aceia: De ce nu L-a?i adus?
Joh 7:46  Slugile au raspuns: Niciodata n-a vorbit un om a?a cum vorbe?te Acest Om.
Joh 7:47  ?i le-au raspuns deci fariseii: Nu cumva a?i fost ?i voi amagi?i?
Joh 7:48  Nu cumva a crezut în El cineva dintre capetenii sau dintre farisei?
Joh 7:49  Dar mul?imea aceasta, care nu cunoa?te Legea, este blestemata!
Joh 7:50  A zis catre ei Nicodim, cel ce venise mai înainte la El, noaptea, fiind unul dintre ei:
Joh 7:51  Nu cumva Legea noastra judeca pe om, daca nu-l asculta mai întâi ?i nu ?tie ce a facut?
Joh 7:52  Ei au raspuns ?i i-au zis: Nu cumva ?i tu e?ti din Galileea? Cerceteaza ?i vezi ca din Galileea nu s-a ridicat prooroc.
Joh 7:53  ?i s-a dus fiecare la casa sa.
Joh 8:1  Iar Iisus S-a dus la Muntele Maslinilor.
Joh 8:2  Dar diminea?a iara?i a venit în templu, ?i tot poporul venea la El; ?i El, ?ezând, îi înva?a.
Joh 8:3  ?i au adus la El fariseii ?i carturarii pe o femeie, prinsa în adulter ?i, a?ezând-o în mijloc,
Joh 8:4  Au zis Lui: Înva?atorule, aceasta femeie a fost prinsa asupra faptului de adulter;
Joh 8:5  Iar Moise ne-a poruncit în Lege ca pe unele ca acestea sa le ucidem cu pietre. Dar Tu ce zici?
Joh 8:6  ?i aceasta ziceau, ispitindu-L, ca sa aiba de ce sa-L învinuiasca. Iar Iisus, plecându-Se în jos, scria cu degetul pe pamânt.
Joh 8:7  ?i staruind sa-L întrebe, El S-a ridicat ?i le-a zis: Cel fara de pacat dintre voi sa arunce cel dintâi piatra asupra ei.
Joh 8:8  Iara?i plecându-Se, scria pe pamânt.
Joh 8:9  Iar ei auzind aceasta ?i mustra?i fiind de cuget, ie?eau unul câte unul, începând de la cei mai batrâni ?i pâna la cel din urma, ?i a ramas Iisus singur ?i femeia, stând în mijloc.
Joh 8:10  ?i ridicându-Se Iisus ?i nevazând pe nimeni decât pe femeie, i-a zis: Femeie, unde sunt pârâ?ii tai? Nu te-a osândit nici unul?
Joh 8:11  Iar ea a zis: Nici unul, Doamne. ?i Iisus i-a zis: Nu te osândesc nici Eu. Mergi; de acum sa nu mai pacatuie?ti.
Joh 8:12  Deci iara?i le-a vorbit Iisus zicând: Eu sunt Lumina lumii; cel ce Îmi urmeaza Mie nu va umbla în întuneric, ci va avea lumina vie?ii.
Joh 8:13  De aceea fariseii I-au zis: Tu marturise?ti despre Tine Însu?i; marturia Ta nu este adevarata.
Joh 8:14  A raspuns Iisus ?i le-a zis: Chiar daca Eu marturisesc despre Mine Însumi, marturia Mea este adevarata, fiindca ?tiu de unde am venit ?i unde Ma duc. Voi nu ?ti?i de unde vin, nici unde Ma duc.
Joh 8:15  Voi judeca?i dupa trup; Eu nu judec pe nimeni.
Joh 8:16  ?i chiar daca Eu judec, judecata Mea este adevarata, pentru ca nu sunt singur, ci Eu ?i Cel ce M-a trimis pe Mine.
Joh 8:17  ?i în Legea voastra este scris ca marturia a doi oameni este adevarata.
Joh 8:18  Eu sunt Cel ce marturisesc despre Mine Însumi ?i marturise?te despre Mine Tatal, Cel ce M-a trimis.
Joh 8:19  Îi ziceau deci: Unde este Tatal Tau? Raspuns-a Iisus: Nu ma ?ti?i nici pe Mine nici pe Tatal Meu; daca M-a?i ?ti pe Mine, a?i ?ti ?i pe Tatal Meu.
Joh 8:20  Cuvintele acestea le-a grait Iisus în vistierie, pe când înva?a în templu; ?i nimeni nu L-a prins, ca înca nu venise ceasul Lui.
Joh 8:21  ?i iara?i le-a zis: Eu Ma duc ?i Ma ve?i cauta ?i ve?i muri în pacatul vostru. Unde Ma duc Eu, voi nu pute?i veni.
Joh 8:22  Deci ziceau iudeii: Nu cumva Î?i va ridica singur via?a? Ca zice: Unde Ma duc Eu, voi nu pute?i veni.
Joh 8:23  ?i El le zicea: Voi sunte?i din cele de jos; Eu sunt din cele de sus. Voi sunte?i din lumea aceasta; Eu nu sunt din lumea aceasta.
Joh 8:24  V-am spus deci voua ca ve?i muri în pacatele voastre. Caci daca nu crede?i ca Eu sunt, ve?i muri în pacatele voastre.
Joh 8:25  Deci Îi ziceau ei: Cine e?ti Tu? ?i a zis lor Iisus: Ceea ce v-am spus de la început.
Joh 8:26  Multe am de spus despre voi ?i de judecat. Dar Cel ce M-a trimis pe Mine adevarat este, ?i cele ce am auzit de la El, Eu acestea le graiesc în lume.
Joh 8:27  ?i ei n-au în?eles ca le vorbea despre Tatal.
Joh 8:28  Deci le-a zis Iisus: Când ve?i înal?a pe Fiul Omului, atunci ve?i cunoa?te ca Eu sunt ?i ca de la Mine însumi nu fac nimic, ci precum M-a înva?at Tatal, a?a vorbesc.
Joh 8:29  ?i Cel ce M-a trimis este cu Mine; nu M-a lasat singur, fiindca Eu fac pururea cele placute Lui.
Joh 8:30  Spunând El acestea, mul?i au crezut în El.
Joh 8:31  Deci zicea Iisus catre iudeii care crezusera în El: Daca ve?i ramâne în cuvântul Meu, sunte?i cu adevarat ucenici ai Mei;
Joh 8:32  ?i ve?i cunoa?te adevarul, iar adevarul va va face liberi.
Joh 8:33  Ei însa I-au raspuns: Noi suntem samân?a lui Avraam ?i nimanui niciodata n-am fost robi. Cum zici Tu ca: Ve?i fi liberi?
Joh 8:34  Iisus le-a raspuns: Adevarat, adevarat va spun: Oricine savâr?e?te pacatul este rob al pacatului.
Joh 8:35  Iar robul nu ramâne în casa în veac; Fiul însa ramâne în veac.
Joh 8:36  Deci, daca Fiul va va face liberi, liberi ve?i fi într-adevar.
Joh 8:37  ?tiu va sunte?i samân?a lui Avraam, dar cauta?i sa Ma omorâ?i, pentru ca cuvâtul Meu nu încape în voi.
Joh 8:38  Eu vorbesc ceea ce am vazut la Tatal Meu, iar voi face?i ceea ce a?i auzit de la tatal vostru.
Joh 8:39  Ei au raspuns ?i I-au zis: Tatal nostru este Avraam. Iisus le-a zis: Daca a?i fi fiii lui Avraam, a?i face faptele lui Avraam.
Joh 8:40  Dar voi acum cauta?i sa Ma ucide?i pe Mine, Omul care v-am spus adevarul pe care l-am auzit de la Dumnezeu. Avraam n-a facut aceasta.
Joh 8:41  Voi face?i faptele tatalui vostru. Zis-au Lui: Noi nu ne-am nascut din desfrânare. Un tata avem: pe Dumnezeu.
Joh 8:42  Le-a zis Iisus: Daca Dumnezeu are fi Tatal vostru, M-a?i iubi pe Mine, caci de la Dumnezeu am ie?it ?i am venit. Pentru ca n-am venit de la Mine însumi, ci El M-a trimis.
Joh 8:43  De ce nu în?elege?i vorbirea Mea? Fiindca nu pute?i sa da?i ascultare cuvântului Meu.
Joh 8:44  Voi sunte?i din tatal vostru diavolul ?i vre?i sa face?i poftele tatalui vostru. El, de la început, a fost ucigator de oameni ?i nu a stat întru adevar, pentru ca nu este adevar întru el. Când graie?te minciuna, graie?te dintru ale sale, caci este mincinos ?i tatal minciunii.
Joh 8:45  Dar pe Mine, fiindca spun adevarul, nu Ma crede?i.
Joh 8:46  Cine dintre voi Ma vade?te de pacat? Daca spun adevarul, de ce voi nu Ma crede?i?
Joh 8:47  Cel care este de la Dumnezeu asculta cuvintele lui Dumnezeu; de aceea voi nu asculta?i pentru ca nu sunte?i de la Dumnezeu.
Joh 8:48  Au raspuns iudeii ?i I-au zis: Oare, nu zicem noi bine ca Tu e?ti samarinean ?i ai demon?
Joh 8:49  A raspuns Iisus: Eu nu am demon, ci cinstesc pe Tatal Meu, ?i voi nu Ma cinsti?i pe Mine.
Joh 8:50  Dar Eu nu caut slava Mea. Este cine sa o caute ?i sa judece.
Joh 8:51  Adevarat, adevarat zic voua: Daca cineva va pazi cuvântul Meu, nu va vedea moartea în veac.
Joh 8:52  Iudeii I-au zis: Acum am cunoscut ca ai demon. Avraam a murit, de asemenea ?i proorocii; ?i Tu zici: Daca cineva va pazi cuvântul Meu, nu va gusta moartea în veac.
Joh 8:53  Nu cumva e?ti Tu mai mare decât tatal nostru Avraam, care a murit? ?i au murit ?i proorocii.  Cine te faci Tu a fi?
Joh 8:54  Iisus a raspuns: Daca Ma slavesc Eu pe Mine Însumi, slava Mea nimic nu este. Tatal Meu este Cel care Ma slave?te, despre Care zice?i voi ca este Dumnezeul vostru.
Joh 8:55  ?i nu L-a?i cunoscut, dar Eu Îl ?tiu; ?i daca a? zice ca nu-L ?tiu, a? fi mincinos asemenea voua. Ci Îl ?tiu ?i pazesc cuvântul Lui.
Joh 8:56  Avraam, parintele vostru, a fost bucuros sa vada ziua Mea ?i a vazut-o ?i s-a bucurat.
Joh 8:57  Deci au zis iudeii catre El: Înca nu ai cincizeci de ani ?i l-ai vazut pe Avraam?
Joh 8:58  Iisus le-a zis: Adevarat, adevarat zic voua: Eu sunt mai înainte de a fi fost Avraam.
Joh 8:59  Deci au luat pietre ca sa arunce asupra Lui. Dar Iisus S-a ferit ?i a ie?it din templu ?i, trecând prin mijlocul lor, S-a dus.
Joh 9:1  ?i trecând Iisus, a vazut un om orb din na?tere.
Joh 9:2  ?i ucenicii Lui L-au întrebat, zicând: Înva?atorule, cine a pacatuit; acesta sau parin?ii lui, de s-a nascut orb?
Joh 9:3  Iisus a raspuns: Nici el n-a pacatuit, nici parin?ii lui, ci ca sa se arate în el lucrarile lui Dumnezeu.
Joh 9:4  Trebuie sa fac, pâna este ziua, lucrarile Celui ce M-a trimis pe Mine; ca vine noaptea, când nimeni nu poate sa lucreze.
Joh 9:5  Atât cât sunt în lume, Lumina a lumii sunt.
Joh 9:6  Acestea zicând, a scuipat jos ?i a facut tina din scuipat, ?i a uns cu tina ochii orbului.
Joh 9:7  ?i i-a zis: Mergi de te spala în scaldatoarea Siloamului (care se tâlcuie?te: trimis). Deci s-a dus ?i s-a spalat ?i a venit vazând.
Joh 9:8  Iar vecinii ?i cei ce-l vazusera mai înainte ca era orb ziceau: Nu este acesta cel ce ?edea ?i cer?ea?
Joh 9:9  Unii ziceau: El este. Al?ii ziceau: Nu este el, ci seamana cu el. Dar acela zicea: Eu sunt.
Joh 9:10  Deci îi ziceau: Cum ?i s-au deschis ochii?
Joh 9:11  Acela a raspuns: Omul care se nume?te Iisus a facut tina ?i a uns ochii mei; ?i mi-a zis: Mergi la scaldatoarea Siloamului ?i te spala. Deci, ducându-ma ?i spalându-ma, am vazut.
Joh 9:12  Zis-au lui: Unde este Acela? ?i el a zis: Nu ?tiu.
Joh 9:13  L-au dus la farisei pe cel ce fusese oarecând orb.
Joh 9:14  ?i era sâmbata în ziua în care Iisus a facut tina ?i i-a deschis ochii.
Joh 9:15  Deci iara?i îl întrebau ?i fariseii cum a vazut. Iar el le-a zis: Tina a pus pe ochii mei, ?i m-am spalat ?i vad.
Joh 9:16  Deci ziceau unii dintre farisei: Acest om nu este de la Dumnezeu, fiindca nu ?ine sâmbata. Iar al?ii ziceau: Cum poate un om pacatos sa faca asemenea minuni? ?i era dezbinare între ei.
Joh 9:17  Au zis deci orbului iara?i: Dar tu ce zici despre El, ca ?i-a deschis ochii? Iar el a zis ca prooroc este.
Joh 9:18  Dar iudeii n-au crezut despre el ca era orb ?i a vazut, pâna ce n-au chemat pe parin?ii celui ce vedea.
Joh 9:19  ?i i-au întrebat, zicând: Acesta este fiul vostru, despre care zice?i ca s-a nascut orb? Deci cum vede el acum?
Joh 9:20  Au raspuns deci parin?ii lui ?i au zis: ?tim ca acesta este fiul nostru ?i ca s-a nascut orb.
Joh 9:21  Dar cum vede el acum, noi nu ?tim; sau cine i-a deschis ochii lui, noi nu ?tim. Întreba?i-l pe el; este în vârsta; va vorbi singur despre sine.
Joh 9:22  Acestea le-au spus parin?ii lui, pentru ca se temeau de iudei. Caci iudeii pusesera acum la cale ca, daca cineva va marturisi ca El este Hristos, sa fie dat afara din sinagoga.
Joh 9:23  De aceea au zis parin?ii lui: Este în vârsta; întreba?i-l pe el.
Joh 9:24  Deci au chemat a doua oara pe omul care fusese orb ?i i-au zis: Da slava lui Dumnezeu. Noi ?tim ca Omul Acesta e pacatos.
Joh 9:25  A raspuns deci acela: Daca este pacatos, nu ?tiu. Un lucru ?tiu: ca fiind orb, acum vad.
Joh 9:26  Deci i-au zis: Ce ?i-a facut? Cum ?i-a deschis ochii?
Joh 9:27  Le-a raspuns: V-am spus acum ?i n-a?i auzit? De ce voi?i sa auzi?i iara?i? Nu cumva voi?i ?i voi sa va face?i ucenici ai Lui?
Joh 9:28  ?i l-au ocarât ?i i-au zis: Tu e?ti ucenic al Aceluia, iar noi suntem ucenici ai lui Moise.
Joh 9:29  Noi ?tim ca Dumnezeu a vorbit lui Moise, iar pe Acesta nu-L ?tim de unde este.
Joh 9:30  A raspuns omul ?i le-a zis: Tocmai în aceasta sta minunea: ca voi nu ?ti?i de unde este ?i El mi-a deschis ochii.
Joh 9:31  ?i noi ?tim ca Dumnezeu nu-i asculta pe pacato?i; dar de este cineva cinstitor de Dumnezeu ?i face voia Lui, pe acesta îl asculta.
Joh 9:32  Din veac nu s-a auzit sa fi deschis cineva ochii unui orb din na?tere.
Joh 9:33  De n-ar fi Acesta de la Dumnezeu n-ar putea sa faca nimic.
Joh 9:34  Au raspuns ?i i-au zis: În pacate te-ai nascut tot, ?i tu ne înve?i pe noi? ?i l-au dat afara.
Joh 9:35  ?i a auzit Iisus ca l-au dat afara. ?i, gasindu-l, i-a zis: Crezi tu în Fiul lui Dumnezeu?
Joh 9:36  El a raspuns ?i a zis: Dar cine este, Doamne, ca sa cred în El?
Joh 9:37  ?i a zis Iisus: L-ai ?i vazut! ?i Cel ce vorbe?te cu tine Acela este.
Joh 9:38  Iar el a zis: Cred, Doamne. ?i s-a închinat Lui.
Joh 9:39  ?i a zis: Spre judecata am venit în lumea aceasta, ca cei care nu vad sa vada, iar cei care vad sa fie orbi.
Joh 9:40  ?i au auzit acestea unii dintre fariseii, care erau cu El, ?i I-au zis: Oare ?i noi suntem orbi?
Joh 9:41  Iisus le-a zis: Daca a?i fi orbi n-a?i avea pacat. Dar acum zice?i: Noi vedem. De aceea pacatul ramâne asupra voastra.
Joh 10:1  Adevarat, adevarat zic voua: Cel ce nu intra pe u?a, în staulul oilor, ci sare pe aiurea, acela este fur ?i tâlhar.
Joh 10:2  Iar cel ce intra prin u?a este pastorul oilor.
Joh 10:3  Acestuia portarul îi deschide ?i oile asculta de glasul lui, ?i oile sale le cheama pe nume ?i le mâna afara.
Joh 10:4  ?i când le scoate afara pe toate ale sale, merge înaintea lor, ?i oile merg dupa el, caci cunosc glasul lui.
Joh 10:5  Iar dupa un strain, ele nu vor merge, ci vor fugi de el, pentru ca nu cunosc glasul lui.
Joh 10:6  Aceasta pilda le-a spus-o Iisus, dar ei n-au în?eles ce înseamna cuvintele Lui.
Joh 10:7  A zis deci iara?i Iisus: Adevarat, adevarat zic voua: Eu sunt u?a oilor.
Joh 10:8  To?i câ?i au venit mai înainte de Mine sunt furi ?i tâlhari, dar oile nu i-au ascultat.
Joh 10:9  Eu sunt u?a: de va intra cineva prin Mine, se va mântui; ?i va intra ?i va ie?i ?i pa?une va afla.
Joh 10:10  Furul nu vine decât ca sa fure ?i sa junghie ?i sa piarda. Eu am venit ca via?a sa aiba ?i din bel?ug sa aiba.
Joh 10:11  Eu sunt pastorul cel bun. Pastorul cel bun î?i pune sufletul pentru oile sale.
Joh 10:12  Iar cel platit ?i cel care nu este pastor, ?i ale carui oi nu sunt ale lui, vede lupul venind ?i lasa oile ?i fuge; ?i lupul le rape?te ?i le risipe?te.
Joh 10:13  Dar cel platit fuge, pentru ca este platit ?i nu are grija de oi.
Joh 10:14  Eu sunt pastorul cel bun ?i cunosc pe ale Mele ?i ale Mele Ma cunosc pe Mine.
Joh 10:15  Precum Ma cunoa?te Tatal ?i Eu cunosc pe Tatal. ?i sufletul Îmi pun pentru oi.
Joh 10:16  Am ?i alte oi, care sunt din staulul acesta. ?i pe acelea trebuie sa le aduc, ?i vor auzi glasul Meu ?i va fi o turma ?i un pastor.
Joh 10:17  Pentru aceasta Ma iube?te Tatal, fiindca Eu Îmi pun sufletul, ca iara?i sa-l iau.
Joh 10:18  Nimeni nu-l ia de la Mine, ci Eu de la Mine Însumi îl pun. Putere am Eu ca sa-l pun ?i putere am iara?i ca sa-l iau. Aceasta porunca am primit-o de la Tatal Meu.
Joh 10:19  Iara?i s-a facut dezbinare între iudei, pentru cuvintele acestea.
Joh 10:20  ?i mul?i dintre ei ziceau: Are demon ?i este nebun. De ce sa-L asculta?i?
Joh 10:21  Al?ii ziceau: Cuvintele acestea nu sunt ale unui demonizat. Cum poate un demon sa deschida ochii orbilor?
Joh 10:22  ?i era atunci la Ierusalim sarbatoarea înnoirii templului ?i era iarna.
Joh 10:23  Iar Iisus umbla prin templu, în pridvorul lui Solomon.
Joh 10:24  Deci L-au împresurat iudeii ?i Îi ziceau: Pâna când ne sco?i sufletul? Daca Tu e?ti Hristosul, spune-o noua pe fa?a.
Joh 10:25  Iisus le-a raspuns: V-am spus ?i nu crede?i. Lucrarile pe care le fac în numele Tatalui Meu, acestea marturisesc despre Mine.
Joh 10:26  Dar voi nu crede?i, pentru ca nu sunte?i dintre oile Mele.
Joh 10:27  Oile Mele asculta de glasul Meu ?i Eu le cunosc pe ele, ?i ele vin dupa Mine.
Joh 10:28  ?i Eu le dau via?a ve?nica ?i nu vor pieri în veac, ?i din mâna Mea nimeni nu le va rapi.
Joh 10:29  Tatal Meu, Care Mi le-a dat, este mai mare decât to?i, ?i nimeni nu poate sa le rapeasca din mâna Tatalui Meu.
Joh 10:30  Iar Eu ?i Tatal Meu una suntem.
Joh 10:31  Iara?i au luat pietre iudeii ca sa arunce asupra Lui.
Joh 10:32  Iisus le-a raspuns: Multe lucruri bune v-am aratat voua de la Tatal Meu. Pentru care din ele, arunca?i cu pietre asupra Mea?
Joh 10:33  I-au raspuns iudeii: Nu pentru lucru bun aruncam cu pietre asupra Ta, ci pentru hula ?i pentru ca Tu, om fiind, Te faci pe Tine Dumnezeu.
Joh 10:34  Iisus le-a raspuns: Nu e scris în Legea voastra ca "Eu am zis: dumnezei sunte?i?"
Joh 10:35  Daca i-a numit dumnezei pe aceia catre care a fost cuvântul lui Dumnezeu - ?i Scriptura nu poate sa fie desfiin?ata -
Joh 10:36  Despre Cel pe care Tatal L-a sfin?it ?i L-a trimis în lume, voi zice?i: Tu hule?ti, caci am spus: Fiul lui Dumnezeu sunt?
Joh 10:37  Daca nu fac lucrarile Tatalui Meu, sa nu crede?i în Mine.
Joh 10:38  Iar daca le fac, chiar daca nu crede?i în Mine, crede?i în aceste lucrari, ca sa ?ti?i ?i sa cunoa?te?i ca Tatal este în Mine ?i Eu în Tatal.
Joh 10:39  Cautau deci iara?i sa-L prinda ?i Iisus a scapat din mâna lor.
Joh 10:40  ?i a plecat iara?i dincolo de Iordan, în locul unde Ioan boteza la început, ?i a ramas acolo.
Joh 10:41  ?i mul?i au venit la El ?i ziceau: Ioan n-a facut nici o minune, dar toate câte Ioan a zis despre Acesta erau adevarate.
Joh 10:42  ?i mul?i au crezut în El acolo.
Joh 11:1  ?i era bolnav un oarecare Lazar din Betania, satul Mariei ?i al Martei, sora ei.
Joh 11:2  Iar Maria era aceea care a uns cu mir pe Domnul ?i I-a ?ters picioarele cu parul capului ei, al carei frate Lazar era bolnav.
Joh 11:3  Deci au trimis surorile la El, zicând: Doamne, iata, cel pe care îl iube?ti este bolnav.
Joh 11:4  Iar Iisus, auzind, a zis: Aceasta boala nu este spre moarte, ci pentru slava lui Dumnezeu, ca, prin ea, Fiul lui Dumnezeu sa Se slaveasca.
Joh 11:5  ?i iubea Iisus pe Marta ?i pe sora ei ?i pe Lazar.
Joh 11:6  Când a auzit, deci, ca este bolnav, atunci a ramas doua zile în locul în care era.
Joh 11:7  Apoi, dupa aceea, a zis ucenicilor: Sa mergem iara?i în Iudeea.
Joh 11:8  Ucenicii I-au zis: Înva?atorule, acum cautau iudeii sa Te ucida cu pietre, ?i iara?i Te duci acolo?
Joh 11:9  A raspuns Iisus: Nu sunt oare douasprezece ceasuri într-o zi? Daca umbla cineva ziua, nu se împiedica, pentru ca el vede lumina acestei lumi;
Joh 11:10  Iar daca umbla cineva noaptea se împiedica, pentru ca lumina nu este în el.
Joh 11:11  A zis acestea, ?i dupa aceea le-a spus: Lazar, prietenul nostru, a adormit; Ma duc sa-l trezesc.
Joh 11:12  Deci I-au zis ucenicii: Doamne, daca a adormit, se va face bine.
Joh 11:13  Iar Iisus vorbise despre moartea lui, iar ei credeau ca vorbe?te despre somn ca odihna.
Joh 11:14  Deci atunci Iisus le-a spus lor pe fa?a: Lazar a murit.
Joh 11:15  ?i Ma bucur pentru voi, ca sa crede?i ca n-am fost acolo. Dar sa mergem la el.
Joh 11:16  Deci a zis Toma, care se nume?te Geamanul, celorlal?i ucenici: Sa mergem ?i noi ?i sa murim cu El.
Joh 11:17  Deci, venind, Iisus l-a gasit pus de patru zile în mormânt.
Joh 11:18  Iar Betania era aproape de Ierusalim, ca la cincisprezece stadii.
Joh 11:19  ?i mul?i dintre iudei venisera la Marta ?i Maria ca sa le mângâie pentru fratele lor.
Joh 11:20  Deci Marta, când a auzit ca vine Iisus, a ie?it în întâmpinarea Lui, iar Maria ?edea în casa.
Joh 11:21  ?i a zis catre Iisus: Doamne, daca ai fi fost aici, fratele meu n-ar fi murit.
Joh 11:22  Dar ?i acum ?tiu ca oricâte vei cere de la Dumnezeu, Dumnezeu î?i va da.
Joh 11:23  Iisus i-a zis: Fratele tau va învia.
Joh 11:24  Marta i-a zis: ?tiu ca va învia la înviere, în ziua cea de apoi.
Joh 11:25  ?i Iisus i-a zis: Eu sunt învierea ?i via?a; cel ce crede în Mine, chiar daca va muri, va trai.
Joh 11:26  ?i oricine traie?te ?i crede în Mine nu va muri în veac. Crezi tu aceasta?
Joh 11:27  Zis-a Lui: Da, Doamne. Eu am crezut ca Tu e?ti Hristosul, Fiul lui Dumnezeu, Care a venit în lume.
Joh 11:28  ?i zicând aceasta, s-a dus ?i a chemat pe Maria, sora ei, zicându-i în taina: Înva?atorul este aici ?i te cheama.
Joh 11:29  Când a auzit aceea, s-a sculat degraba ?i a venit la El.
Joh 11:30  ?i Iisus nu venise înca în sat, ci era în locul unde Îl întâmpinase Marta.
Joh 11:31  Iar iudeii care erau cu ea în casa ?i o mângâiau, vazând pe Maria ca s-a sculat degraba ?i a ie?it afara, au mers dupa ea socotind ca a plecat la mormânt, ca sa plânga acolo.
Joh 11:32  Deci Maria, când a venit unde era Iisus, vazându-L, a cazut la picioarele Lui, zicându-I: Doamne, daca ai fi fost aici, fratele meu n-ar fi murit.
Joh 11:33  Deci Iisus, când a vazut-o plângând ?i pe iudeii care venisera cu ea plângând ?i ei, a suspinat cu duhul ?i S-a tulburat întru Sine.
Joh 11:34  ?i a zis: Unde l-a?i pus? Zis-au Lui: Doamne, vino ?i vezi.
Joh 11:35  ?i a lacrimat Iisus.
Joh 11:36  Deci ziceau iudeii: Iata cât de mult îl iubea.
Joh 11:37  Iar unii dintre ei ziceau: Nu putea, oare, Acesta care a deschis ochii orbului sa faca a?a ca ?i acesta sa nu moara?
Joh 11:38  Deci suspinând iara?i Iisus întru Sine, a mers la mormânt. ?i era o pe?tera ?i o piatra era a?ezata pe ea.
Joh 11:39  Iisus a zis: Ridica?i piatra. Marta, sora celui raposat, I-a zis: Doamne, deja miroase, ca este a patra zi.
Joh 11:40  Iisus i-a zis: Nu ?i-am spus ca daca vei crede, vei vedea slava lui Dumnezeu?
Joh 11:41  Au ridicat deci piatra, iar Iisus ?i-a ridicat ochii în sus ?i a zis: Parinte, Î?i mul?umesc ca M-ai ascultat.
Joh 11:42  Eu ?tiam ca întotdeauna Ma ascul?i, dar pentru mul?imea care sta împrejur am zis, ca sa creada ca Tu M-ai trimis.
Joh 11:43  ?i zicând acestea, a strigat cu glas mare: Lazare, vino afara!
Joh 11:44  ?i a ie?it mortul, fiind legat la picioare ?i la mâini cu fâ?ii de pânza ?i fa?a lui era înfa?urata cu mahrama. Iisus le-a zis: Dezlega?i-l ?i lasa?i-l sa mearga.
Joh 11:45  Deci mul?i dintre iudeii care venisera la Maria ?i vazusera ce a facut Iisus au crezut în El.
Joh 11:46  Iar unii dintre ei s-au dus la farisei ?i le-au spus cele ce facuse Iisus.
Joh 11:47  Deci arhiereii ?i fariseii au adunat sinedriul ?i ziceau: Ce facem, pentru ca Omul Acesta face multe minuni?
Joh 11:48  Daca-L lasam a?a to?i vor crede în El, ?i vor veni romanii ?i ne vor lua ?ara ?i neamul.
Joh 11:49  Iar Caiafa, unul dintre ei, care în anul acela era arhiereu le-a zis: Voi nu ?ti?i nimic;
Joh 11:50  Nici nu gândi?i ca ne este mai de folos sa moara un om pentru popor, decât sa piara tot neamul.
Joh 11:51  Dar aceasta n-a zis-o de la sine, ci, fiind arhiereu al anului aceluia, a proorocit ca Iisus avea sa moara pentru neam,
Joh 11:52  ?i nu numai pentru neam, ci ?i ca sa adune laolalta pe fiii lui Dumnezeu cei împra?tia?i.
Joh 11:53  Deci, din ziua aceea, s-au hotarât ca sa-L ucida.
Joh 11:54  De aceea Iisus nu mai umbla pe fa?a printre iudei, ci a plecat de acolo într-un ?inut aproape de pustie, într-o cetate numita Efraim, ?i acolo a ramas cu ucenicii Sai.
Joh 11:55  ?i era aproape Pa?tile iudeilor ?i mul?i din ?ara s-au suit la Ierusalim, mai înainte de Pa?ti, ca sa se cura?easca.
Joh 11:56  Deci cautau pe Iisus ?i, pe când stateau în templu, ziceau între ei: Ce vi se pare? Oare nu va veni la sarbatoare?
Joh 11:57  Iar arhiereii ?i fariseii dadusera porunci, ca daca va ?ti cineva unde este, sa dea de veste, ca sa-L prinda.
Joh 12:1  Deci, cu ?ase zile înainte de Pa?ti, Iisus a venit în Betania, unde era Lazar, pe care îl înviase din mor?i.
Joh 12:2  ?i I-au facut acolo cina ?i Marta slujea. Iar Lazar era unul dintre cei ce ?edeau cu El la masa.
Joh 12:3  Deci Maria, luând o litra cu mir de nard curat, de mare pre?, a uns picioarele lui Iisus ?i le-a ?ters cu parul capului ei, iar casa s-a umplut de mirosul mirului.
Joh 12:4  Iar Iuda Iscarioteanul, unul dintre ucenicii Lui, care avea sa-L vânda, a zis:
Joh 12:5  Pentru ce nu s-a vândut mirul acesta cu trei sute de dinari ?i sa-i fi dat saracilor?
Joh 12:6  Dar el a zis aceasta, nu pentru ca îi era grija de saraci, ci pentru ca era fur ?i, având punga, lua din ce se punea în ea.
Joh 12:7  A zis deci Iisus: Las-o, ca pentru ziua îngroparii Mele l-a pastrat.
Joh 12:8  Ca pe saraci totdeauna îi ave?i cu voi, dar pe Mine nu Ma ave?i totdeauna.
Joh 12:9  Deci mul?ime mare de iudei au aflat ca este acolo ?i au venit nu numai pentru Iisus, ci sa vada ?i pe Lazar pe care-l înviase din mor?i.
Joh 12:10  ?i s-au sfatuit arhiereii ca ?i pe Lazar sa-l omoare.
Joh 12:11  Caci, din cauza lui mul?i dintre iudei mergeau ?i credeau în Iisus.
Joh 12:12  A doua zi, mul?ime multa, care venise la sarbatoare, auzind ca Iisus vine în Ierusalim,
Joh 12:13  Au luat ramuri de finic ?i au ie?it întru întâmpinarea Lui ?i strigau: Osana! Binecuvântat este Cel ce vine întru numele Domnului, Împaratul lui Israel!
Joh 12:14  ?i Iisus, gasind un asin tânar, a ?ezut pe el, precum este scris:
Joh 12:15  "Nu te teme, fiica Sionului! Iata Împaratul tau vine, ?ezând pe mânzul asinei".
Joh 12:16  Acestea nu le-au în?eles ucenicii Lui la început, dar când S-a preaslavit Iisus, atunci ?i-au adus aminte ca acestea I le-au facut Lui.
Joh 12:17  Deci da marturie mul?imea care era cu El, când l-a strigat pe Lazar din mormânt ?i l-a înviat din mor?i.
Joh 12:18  De aceea L-a ?i întâmpinat mul?imea, pentru ca auzise ca El a facut minunea aceasta.
Joh 12:19  Deci fariseii ziceau între ei: Vede?i ca nimic nu folosi?i! Iata, lumea s-a dus dupa El.
Joh 12:20  ?i erau ni?te elini din cei ce se suisera sa se închine la sarbatoare.
Joh 12:21  Deci ace?tia au venit la Filip, cel ce era din Betsaida Galileii, ?i l-au rugat zicând: Doamne, voim sa vedem pe Iisus.
Joh 12:22  Filip a venit ?i i-a spus la Andrei, ?i Andrei ?i Filip au venit ?i I-au spus lui Iisus.
Joh 12:23  Iar Iisus le-a raspuns, zicând: A venit ceasul ca sa fie preaslavit Fiul Omului.
Joh 12:24  Adevarat, adevarat zic voua ca daca grauntele de grâu, când cade în pamânt, nu va muri, ramâne singur; iar daca va muri, aduce multa roada.
Joh 12:25  Cel ce î?i iube?te sufletul îl va pierde; iar cel ce î?i ura?te sufletul în lumea aceasta îl va pastra pentru via?a ve?nica.
Joh 12:26  Daca-Mi sluje?te cineva, sa-Mi urmeze, ?i unde sunt Eu, acolo va fi ?i slujitorul Meu. Daca-Mi sluje?te cineva, Tatal Meu îl va cinsti.
Joh 12:27  Acum sufletul Meu e tulburat, ?i ce voi zice? Parinte, izbave?te-Ma, de ceasul acesta. Dar pentru aceasta am venit în ceasul acesta.
Joh 12:28  Parinte, preaslave?te-?i numele! Atunci a venit glas din cer: ?i L-am preaslavit ?i iara?i Îl voi preaslavi.
Joh 12:29  Iar mul?imea care sta ?i auzea zicea: A fost tunet! Al?ii ziceau: Înger I-a vorbit!
Joh 12:30  Iisus a raspuns ?i a zis: Nu pentru Mine s-a facut glasul acesta, ci pentru voi.
Joh 12:31  Acum este judecata acestei lumi; acum stapânitorul lumii acesteia va fi aruncat afara.
Joh 12:32  Iar Eu, când Ma voi înal?a de pe pamânt, îi voi trage pe to?i la Mine.
Joh 12:33  Iar aceasta zicea, aratând cu ce moarte avea sa moara.
Joh 12:34  I-a raspuns deci mul?imea: Noi am auzit din Lege ca Hristosul ramâne în veac; ?i cum zici Tu ca Fiul Omului trebuie sa fie înal?at? Cine este acesta, Fiul Omului?
Joh 12:35  Deci le-a zis Iisus: Înca pu?ina vreme Lumina este cu voi. Umbla?i cât ave?i Lumina ca sa nu va prinda întunericul. Caci cel ce umbla în întuneric nu ?tie unde merge.
Joh 12:36  Cât ave?i Lumina, crede?i în Lumina, ca sa fi?i fii ai Luminii. Acestea le-a vorbit iisus ?i, plecând, S-a ascuns de ei.
Joh 12:37  ?i, de?i a facut atâtea minuni înaintea lor, ei tot nu credeau în El,
Joh 12:38  Ca sa se împlineasca cuvântul proorocului Isaia, pe care l-a zis: "Doamne, cine a crezut în ceea ce a auzit de la noi? ?i bra?ul Domnului cui s-a descoperit?"
Joh 12:39  De aceea nu puteau sa creada, ca iara?i a zis Isaia:
Joh 12:40  "Au orbit ochii lor ?i a împietrit inima lor, ca sa nu vada cu ochii ?i sa nu în?eleaga cu inima ?i ca nu cumva sa se întoarca ?i Eu sa-i vindec"
Joh 12:41  Acestea a zis Isaia, când a vazut slava Lui ?i a grait despre El.
Joh 12:42  Totu?i ?i dintre capetenii mul?i au crezut în El, dar nu marturiseau din pricina fariseilor, ca sa nu fie izgoni?i din sinagoga;
Joh 12:43  Caci au iubit slava oamenilor mai mult decât slava lui Dumnezeu.
Joh 12:44  Iar Iisus a strigat ?i a zis: Cel ce crede în Mine nu crede în Mine, ci în Cel ce M-a trimis pe Mine.
Joh 12:45  ?i cel ce Ma vede pe Mine vede pe Cel ce M-a trimis pe Mine.
Joh 12:46  Eu, Lumina am venit în lume, ca tot cel ce crede în Mine sa nu ramâna întuneric.
Joh 12:47  ?i daca aude cineva cuvintele Mele ?i nu le paze?te, nu Eu îl judec; caci n-am venit ca sa judec lumea ci ca sa mântuiesc lumea.
Joh 12:48  Cine Ma nesocote?te pe Mine ?i nu prime?te cuvintele Mele are judecator ca sa-l judece: cuvântul pe care l-am spus acela îl va judeca în ziua cea de apoi.
Joh 12:49  Pentru ca Eu n-am vorbit de la Mine, ci Tatal care M-a trimis, Acesta Mi-a dat porunca ce sa spun ?i ce sa vorbesc.
Joh 12:50  ?i ?tiu ca porunca Lui este via?a ve?nica. Deci cele ce vorbesc Eu, precum Mi-a spus Mie Tatal, a?a vorbesc.
Joh 13:1  Iar înainte de sarbatoarea Pa?tilor, ?tiind Iisus ca a sosit ceasul Lui, ca sa treaca din lumea aceasta la Tatal, iubind pe ai Sai cei din lume, pâna la sfâr?it i-a iubit.
Joh 13:2  ?i facându-se Cina, ?i diavolul punând în inima lui Iuda fiul lui Simon Iscarioteanul, ca sa-l vânda,
Joh 13:3  Iisus, ?tiind ca Tatal I-a dat Lui toate în mâini ?i ca de la Dumnezeu a ie?it ?i la Dumnezeu merge,
Joh 13:4  S-a sculat de la Cina, S-a dezbracat de haine ?i, luând un ?tergar, S-a încins cu el.
Joh 13:5  Dupa aceea a turnat apa în vasul de spalat ?i a început sa spele picioarele ucenicilor ?i sa le ?tearga cu ?tergarul cu care era încins.
Joh 13:6  A venit deci la Simon Petru. Acesta I-a zis: Doamne, oare Tu sa-mi speli mie picioarele?
Joh 13:7  A raspuns Iisus ?i i-a zis: Ceea ce fac Eu, tu nu ?tii acum, dar vei în?elege dupa aceasta.
Joh 13:8  Petru I-a zis: Nu-mi vei spala picioarele în veac. Iisus i-a raspuns: Daca nu te voi spala, nu ai parte de Mine.
Joh 13:9  Zis-a Simon Petru Lui: Doamne, spala-mi nu numai picioarele mele, ci ?i mâinile ?i capul.
Joh 13:10  Iisus i-a zis: Cel ce a facut baie n-are nevoie sa-i fie spalate decât picioarele, caci este curat tot. ?i voi sunte?i cura?i, însa nu to?i.
Joh 13:11  Ca ?tia pe cel ce avea sa-L vânda; de aceea a zis: Nu to?i sunte?i cura?i.
Joh 13:12  Dupa ce le-a spalat picioarele ?i ?i-a luat hainele, S-a a?ezat iar la masa ?i le-a zis: În?elege?i ce v-am facut Eu?
Joh 13:13  Voi Ma numi?i pe Mine: Înva?atorul ?i Domnul, ?i bine zice?i, caci sunt.
Joh 13:14  Deci daca Eu, Domnul ?i Înva?atorul, v-am spalat voua picioarele, ?i voi sunte?i datori sa ca sa spala?i picioarele unii altora;
Joh 13:15  Ca v-am dat voua pilda, ca, precum v-am facut Eu voua, sa face?i ?i voi.
Joh 13:16  Adevarat, zic voua: Nu este sluga mai mare decât stapânul sau, nici solul mai mare decât cel ce l-a trimis pe el.
Joh 13:17  Când ?ti?i acestea, ferici?i sunte?i daca le ve?i face.
Joh 13:18  Nu zic despre voi to?i; caci Eu ?tiu pe cei pe care i-am ales. Ci ca sa se împlineasca Scriptura: "Cel ce manânca pâinea cu Mine a ridicat calcâiul împotriva Mea".
Joh 13:19  De acum va spun voua, înainte de a fi aceasta, ca sa crede?i, când se va îndeplini, ca Eu sunt.
Joh 13:20  Adevarat, adevarat zic voua: Cel care prime?te pe cel pe care-l voi trimite Eu, pe Mine Ma prime?te; iar cine Ma prime?te pe Mine prime?te pe Cel ce M-a trimis pe Mine.
Joh 13:21  Iisus, zicând acestea, S-a tulburat cu duhul ?i a marturisit ?i a zis: Adevarat, adevarat zic voua ca unul dintre voi Ma va vinde.
Joh 13:22  Deci ucenicii se uitau unii la al?ii, nedumerindu-se despre cine vorbe?te.
Joh 13:23  Iar la masa era rezemat la pieptul lui Iisus unul dintre ucenicii Lui, pe care-l iubea Iisus.
Joh 13:24  Deci Simon Petru i-a facut semn acestuia ?i i-a zis: Întreaba cine este despre care vorbe?te.
Joh 13:25  ?i cazând acela astfel la pieptul lui Iisus, I-a zis: Doamne, cine este?
Joh 13:26  Iisus i-a raspuns: Acela este, caruia Eu, întingând buca?ica de pâine, i-o voi da. ?i întingând buca?ica, a luat-o ?i a dat-o lui Iuda, fiul lui Simon Iscarioteanul.
Joh 13:27  ?i dupa îmbucatura a intrat satana în el. Iar Iisus i-a zis: Ceea ce faci, fa mai curând.
Joh 13:28  Dar nimeni din cei care ?edeau la masa n-a în?eles pentru ce i-a zis aceasta.
Joh 13:29  Caci unii socoteau, deoarece Iuda avea punga, ca lui îi zice Iisus: Cumpara cele de care avem de trebuin?a la sarbatoare, sau sa dea ceva saracilor.
Joh 13:30  Deci dupa ce a luat acela buca?ica de pâine, a ie?it numaidecât. ?i era noapte.
Joh 13:31  ?i când a ie?it el, Iisus a zis: Acum a fost preaslavit Fiul Omului ?i Dumnezeu a fost preaslavit întru El.
Joh 13:32  Iar daca Dumnezeu a fost preaslavit întru El, ?i Dumnezeu Îl va preaslavi întru El ?i îndata Îl va preaslavi.
Joh 13:33  Fiilor, înca pu?in timp sunt cu voi. Voi Ma ve?i cauta, dar, dupa cum am spus iudeilor - ca unde Ma duc Eu, voi nu pute?i veni - va spun voua acum.
Joh 13:34  Porunca noua dau voua: Sa va iubi?i unul pe altul. Precum Eu v-am iubit pe voi, a?a ?i voi sa va iubi?i unul pe altul.
Joh 13:35  Întru aceasta vor cunoa?te to?i ca sunte?i ucenicii Mei, daca ve?i avea dragoste unii fa?a de al?ii.
Joh 13:36  Doamne, L-a întrebat Simon-Petru, unde Te duci? Raspuns-a Iisus: Unde Ma duc Eu, tu nu po?i sa urmezi Mie acum, dar mai târziu Îmi vei urma.
Joh 13:37  Zis-a Petru Lui: Doamne, de ce nu pot sa urmez ?ie acum? Sufletul meu îl voi da pentru Tine.
Joh 13:38  Iisus i-a raspuns: Vei pune sufletul tau pentru Mine? Adevarat, adevarat zic ?ie ca nu va cânta coco?ul, pâna ce nu te vei lepada de Mine de trei ori!
Joh 14:1  Sa nu se tulbure inima voastra; crede?i în Dumnezeu, crede?i ?i în Mine.
Joh 14:2  În casa Tatalui Meu multe loca?uri sunt. Iar de nu, v-a? fi spus. Ma duc sa va gatesc loc.
Joh 14:3  ?i daca Ma voi duce ?i va voi gati loc, iara?i voi veni ?i va voi lua la Mine, ca sa fi?i ?i voi unde sunt Eu.
Joh 14:4  ?i unde Ma duc Eu, voi ?ti?i ?i ?ti?i ?i calea.
Joh 14:5  Toma i-a zis: Doamne, nu ?tim unde Te duci; ?i cum putem ?ti calea?
Joh 14:6  Iisus i-a zis: Eu sunt Calea, Adevarul ?i Via?a. Nimeni nu vine la Tatal Meu decât prin Mine.
Joh 14:7  Daca M-a?i fi cunoscut pe Mine, ?i pe Tatal Meu L-a?i fi cunoscut; dar de acum Îl cunoa?te?i pe El ?i L-a?i ?i vazut.
Joh 14:8  Filip I-a zis: Doamne, arata-ne noua pe Tatal ?i ne este de ajuns.
Joh 14:9  Iisus i-a zis: De atâta vreme sunt cu voi ?i nu M-ai cunoscut, Filipe? Cel ce M-a vazut pe Mine a vazut pe Tatal. Cum zici tu: Arata-ne pe Tatal?
Joh 14:10  Nu crezi tu ca Eu sunt întru Tatal ?i Tatal este întru Mine? Cuvintele pe care vi le spun nu le vorbesc de la Mine, ci Tatal - Care ramâne întru Mine - face lucrarile Lui.
Joh 14:11  Crede?i Mie ca Eu sunt întru Tatal ?i Tatal întru Mine, iar de nu, crede?i-Ma pentru lucrarile acestea.
Joh 14:12  Adevarat, adevarat zic voua: cel ce crede în Mine va face ?i el lucrarile pe care le fac Eu ?i mai mari decât acestea va face, pentru ca Eu Ma duc la Tatal.
Joh 14:13  ?i orice ve?i cere întru numele Meu, aceea voi face, ca sa fie slavit Tatal întru Fiul.
Joh 14:14  Daca ve?i cere ceva în numele Meu, Eu voi face.
Joh 14:15  De Ma iubi?i, pazi?i poruncile Mele.
Joh 14:16  ?i Eu voi ruga pe Tatal ?i alt Mângâietor va va da voua ca sa fie cu voi în veac,
Joh 14:17  Duhul Adevarului, pe Care lumea nu poate sa-L primeasca, pentru ca nu-L vede, nici nu-L cunoa?te; voi Îl cunoa?te?i, ca ramâne la voi ?i în voi va fi!
Joh 14:18  Nu va voi lasa orfani: voi veni la voi.
Joh 14:19  Înca pu?in timp ?i lumea nu Ma va mai vedea; voi însa Ma ve?i vedea, pentru ca Eu sunt viu ?i voi ve?i fi vii.
Joh 14:20  În ziua aceea ve?i cunoa?te ca Eu sunt întru Tatal Meu ?i voi în Mine ?i Eu în voi.
Joh 14:21  Cel ce are poruncile Mele ?i le paze?te, acela este care Ma iube?te; iar cel ce Ma iube?te pe Mine va fi iubit de Tatal Meu ?i-l voi iubi ?i Eu ?i Ma voi arata lui.
Joh 14:22  I-a zis Iuda, nu Iscarioteanul: Doamne, ce este ca ai sa Te ara?i noua, ?i nu lumii?
Joh 14:23  Iisus a raspuns ?i i-a zis: Daca Ma iube?te cineva, va pazi cuvântul Meu, ?i Tatal Meu îl va iubi, ?i vom veni la el ?i vom face loca? la el.
Joh 14:24  Cel ce nu Ma iube?te nu paze?te cuvintele Mele. Dar cuvântul pe care îl auzi?i nu este al Meu, ci al Tatalui care M-a trimis.
Joh 14:25  Acestea vi le-am spus, fiind cu voi;
Joh 14:26  Dar Mângâietorul, Duhul Sfânt, pe Care-L va trimite Tatal, în numele Meu, Acela va va înva?a toate ?i va va aduce aminte despre toate cele ce v-am spus Eu.
Joh 14:27  Pace va las voua, pacea Mea o dau voua, nu precum da lumea va dau Eu. Sa nu se tulbure inima voastra, nici sa se înfrico?eze.
Joh 14:28  A?i auzit ca v-am spus: Ma duc ?i voi veni la voi. De M-a?i iubi v-a?i bucura ca Ma duc la Tatal, pentru ca Tatal este mai mare decât Mine.
Joh 14:29  ?i acum v-am spus acestea înainte de a se întâmpla, ca sa crede?i când se vor întâmpla.
Joh 14:30  Nu voi mai vorbi multe cu voi, caci vine stapânitorul acestei lumi ?i el nu are nimic în Mine;
Joh 14:31  Dar ca sa cunoasca lumea ca Eu iubesc pe Tatal ?i precum Tatal Mi-a poruncit a?a fac. Scula?i-va, sa mergem de aici.
Joh 15:1  Eu sunt vi?a cea adevarata ?i Tatal Meu este lucratorul.
Joh 15:2  Orice mladi?a care nu aduce roada întru Mine, El o taie; ?i orice mladi?a care aduce roada, El o cura?e?te, ca mai multa roada sa aduca.
Joh 15:3  Acum voi sunte?i cura?i, pentru cuvântul pe care vi l-am spus.
Joh 15:4  Ramâne?i în Mine ?i Eu în voi. Precum mladi?a nu poate sa aduca roada de la sine, daca nu ramâne în vi?a, tot a?a nici voi, daca nu ramâne?i în Mine.
Joh 15:5  Eu sunt vi?a, voi sunte?i mladi?ele. Cel ce ramâne întru Mine ?i Eu în el, acela aduce roada multa, caci fara Mine nu pute?i face nimic.
Joh 15:6  Daca cineva nu ramâne în Mine se arunca afara ca mladi?a ?i se usuca; ?i le aduna ?i le arunca în foc ?i ard.
Joh 15:7  Daca ramâne?i întru Mine ?i cuvintele Mele ramân în voi, cere?i ceea ce voi?i ?i se va da voua.
Joh 15:8  Întru aceasta a fost slavit Tatal Meu, ca sa aduce?i roada multa ?i sa va face?i ucenici ai Mei.
Joh 15:9  Precum M-a iubit pe Mine Tatal, a?a v-am iubit ?i Eu pe voi; ramâne?i întru iubirea Mea.
Joh 15:10  Daca pazi?i poruncile Mele, ve?i ramâne întru iubirea Mea dupa cum ?i Eu am pazit poruncile Tatalui Meu ?i ramân întru iubirea Lui.
Joh 15:11  Acestea vi le-am spus, ca bucuria Mea sa fie în voi ?i ca bucuria voastra sa fie deplina.
Joh 15:12  Aceasta este porunca Mea: sa va iubi?i unul pe altul, precum v-am iubit Eu.
Joh 15:13  Mai mare dragoste decât aceasta nimeni nu are, ca sufletul lui sa ?i-l puna pentru prietenii sai.
Joh 15:14  Voi sunte?i prietenii Mei, daca face?i ceea ce va poruncesc.
Joh 15:15  De acum nu va mai zic slugi, ca sluga nu ?tie ce face stapânul sau, ci v-am numit pe voi prieteni, pentru ca toate câte am auzit de la Tatal Meu vi le-am facut cunoscute.
Joh 15:16  Nu voi M-a?i ales pe Mine, ci Eu v-am ales pe voi ?i v-am rânduit sa merge?i ?i roada sa aduce?i, ?i roada voastra sa ramâna, ca Tatal sa va dea orice-I ve?i cere în numele Meu.
Joh 15:17  Aceasta va poruncesc: sa va iubi?i unul pe altul.
Joh 15:18  Daca va ura?te pe voi lumea, sa ?ti?i ca pe Mine mai înainte decât pe voi M-a urât.
Joh 15:19  Daca a?i fi din lume, lumea ar iubi ce este al sau; dar pentru ca nu sunte?i din lume, ci Eu v-am ales pe voi din lume, de aceea lumea va ura?te.
Joh 15:20  Aduce?i-va aminte de cuvântul pe care vi l-am spus: Nu este sluga mai mare decât stapânul sau. Daca M-au prigonit pe Mine, ?i pe voi va vor prigoni; daca au pazit cuvântul Meu, ?i pe al vostru îl vor pazi.
Joh 15:21  Iar toate acestea le vor face voua din cauza numelui Meu, fiindca ei nu cunosc pe Cel ce M-a trimis.
Joh 15:22  De n-a? fi venit ?i nu le-a? fi vorbit, pacat nu ar avea; dar acum n-au cuvânt de dezvinova?ire pentru pacatul lor.
Joh 15:23  Cel ce Ma ura?te pe Mine, ura?te ?i pe Tatal Meu.
Joh 15:24  De nu a? fi facut între ei lucruri pe care nimeni altul nu le-a facut pacat nu ar avea; dar acum M-au ?i vazut ?i M-au urât ?i pe Mine ?i pe Tatal Meu.
Joh 15:25  Dar (aceasta), ca sa se împlineasca cuvântul cel scris în Legea lor: "M-au urât pe nedrept".
Joh 15:26  Iar când va veni Mângâietorul, pe Care Eu Îl voi trimite voua de la Tatal, Duhul Adevarului, Care de la Tatal purcede, Acela va marturisi despre Mine.
Joh 15:27  ?i voi marturisi?i, pentru ca de la început sunte?i cu Mine.
Joh 16:1  Acestea vi le-am spus, ca sa nu va sminti?i.
Joh 16:2  Va vor scoate pe voi din sinagogi; dar vine ceasul când tot cel ce va va ucide sa creada ca aduce închinare lui Dumnezeu.
Joh 16:3  ?i acestea le vor face, pentru ca n-au cunoscut nici pe Tatal, nici pe Mine.
Joh 16:4  Iar acestea vi le-am spus, ca sa va aduce?i aminte de ele, când va veni ceasul lor, ca Eu vi le-am spus. ?i acestea nu vi le-am spus de la început, fiindca eram cu voi.
Joh 16:5  Dar acum Ma duc la Cel ce M-a trimis ?i nimeni dintre voi nu întreaba: Unde Te duci?
Joh 16:6  Ci, fiindca v-am spus acestea, întristarea a umplut inima voastra.
Joh 16:7  Dar Eu va spun adevarul: Va este de folos ca sa ma duc Eu. Caci daca nu Ma voi duce, Mângâietorul nu va veni la voi, iar daca Ma voi duce, Îl voi trimite la voi.
Joh 16:8  ?i El, venind, va vadi lumea de pacat ?i de dreptate ?i de judecata.
Joh 16:9  De pacat, pentru ca ei nu cred în Mine;
Joh 16:10  De dreptate, pentru ca Ma duc la Tatal Meu ?i nu Ma ve?i mai vedea;
Joh 16:11  ?i de judecata, pentru ca stapânitorul acestei lumi a fost judecat.
Joh 16:12  Înca multe am a va spune, dar acum nu pute?i sa le purta?i.
Joh 16:13  Iar când va veni Acela, Duhul Adevarului, va va calauzi la tot adevarul; caci nu va vorbi de la Sine, ci toate câte va auzi va vorbi ?i cele viitoare va va vesti.
Joh 16:14  Acela Ma va slavi, pentru ca din al Meu va lua ?i va va vesti.
Joh 16:15  Toate câte are Tatal ale Mele sunt; de aceea am zis ca din al Meu ia ?i va veste?te voua.
Joh 16:16  Pu?in ?i nu Ma ve?i mai vedea, ?i iara?i pu?in ?i Ma ve?i vedea, pentru ca Eu Ma duc la Tatal.
Joh 16:17  Deci unii dintre ucenicii Lui ziceau între ei: Ce este aceasta ce ne spune: Pu?in ?i nu Ma ve?i mai vedea, ?i iara?i pu?in ?i Ma ve?i vedea, ?i ca Ma duc la Tatal?
Joh 16:18  Deci ziceau: Ce este aceasta ce zice: Pu?in? Nu ?tim ce zice.
Joh 16:19  ?i a cunoscut Iisus ca voiau sa-L întrebe ?i le-a zis: Despre aceasta va întreba?i între voi, ca am zis: Pu?in ?i nu Ma ve?i mai vedea ?i iara?i pu?in ?i Ma ve?i vedea?
Joh 16:20  Adevarat, adevarat zic voua ca voi ve?i plânge ?i va ve?i tângui, iar lumea se va bucura. Voi va ve?i întrista, dar întristarea voastra se va preface în bucurie.
Joh 16:21  Femeia, când e sa nasca, se întristeaza, fiindca a sosit ceasul ei; dar dupa ce a nascut copilul, nu-?i mai aduce aminte de durere, pentru bucuria ca s-a nascut om în lume.
Joh 16:22  Deci ?i voi acum sunte?i tri?ti, dar iara?i va voi vedea ?i se va bucura inima voastra ?i bucuria voastra nimeni nu o va lua de la voi.
Joh 16:23  ?i în ziua aceea nu Ma ve?i întreba nimic. Adevarat, adevarat zic voua: Orice ve?i cere de la Tatal în numele Meu El va va da.
Joh 16:24  Pâna acum n-a?i cerut nimic în numele Meu; cere?i ?i ve?i primi, ca bucuria voastra sa fie deplina.
Joh 16:25  Acestea vi le-am spus în pilde, dar vine ceasul când nu va voi mai vorbi în pilde, ci pe fa?a va voi vesti despre Tatal.
Joh 16:26  În ziua aceea ve?i cere în numele Meu; ?i nu va zic ca voi ruga pe Tatal pentru voi,
Joh 16:27  Caci Însu?i Tatal va iube?te pe voi, fiindca voi M-a?i iubit pe Mine ?i a?i crezut ca de la Dumnezeu am ie?it.
Joh 16:28  Ie?it-am de la Tatal ?i am venit în lume; iara?i las lumea ?i Ma duc la Tatal.
Joh 16:29  Au zis ucenicii Sai: Iata acum vorbe?ti pe fa?a ?i nu spui nici o pilda.
Joh 16:30  Acum ?tim ca Tu ?tii toate ?i nu ai nevoie ca sa Te întrebe cineva. De aceea credem ca ai ie?it de la Dumnezeu.
Joh 16:31  Iisus le-a raspuns: Acum crede?i?
Joh 16:32  Iata vine ceasul, ?i a ?i venit, ca sa va risipi?i fiecare la ale sale ?i pe Mine sa Ma lasa?i singur. Dar nu sunt singur, pentru ca Tatal este cu Mine.
Joh 16:33  Acestea vi le-am grait, ca întru Mine pace sa ave?i. În lume necazuri ve?i avea; dar îndrazni?i. Eu am biruit lumea.
Joh 17:1  Acestea a vorbit Iisus ?i, ridicând ochii Sai la cer, a zis: Parinte, a venit ceasul! Preaslave?te pe Fiul Tau, ca ?i Fiul sa Te preaslaveasca.
Joh 17:2  Precum I-ai dat stapânire peste tot trupul, ca sa dea via?a ve?nica tuturor acelora pe care  Tu i-ai dat Lui.
Joh 17:3  ?i aceasta este via?a ve?nica: Sa Te cunoasca pe Tine, singurul Dumnezeu adevarat, ?i pe Iisus Hristos pe Care L-ai trimis.
Joh 17:4  Eu Te-am preaslavit pe Tine pe pamânt; lucrul pe care Mi l-ai dat sa-l fac, l-am savâr?it.
Joh 17:5  ?i acum, preaslave?te-Ma Tu, Parinte, la Tine Însu?i, cu slava pe care am avut-o la Tine, mai înainte de a fi lumea.
Joh 17:6  Aratat-am numele Tau oamenilor pe care Mi i-ai dat Mie din lume. Ai Tai erau ?i Mie Mi i-ai dat ?i cuvântul Tau l-au pazit.
Joh 17:7  Acum au cunoscut ca toate câte Mi-ai dat sunt de la Tine;
Joh 17:8  Pentru ca cuvintele pe care Mi le-ai dat le-am dat lor, iar ei le-au primit ?i au cunoscut cu adevarat ca de la Tine am ie?it, ?i au crezut ca Tu M-ai trimis.
Joh 17:9  Eu pentru ace?tia Ma rog; nu pentru lume Ma rog, ci pentru cei pe care Mi i-ai dat, ca ai Tai sunt.
Joh 17:10  ?i toate ale Mele sunt ale Tale, ?i ale Tale sunt ale Mele ?i M-am preaslavit întru ei.
Joh 17:11  ?i Eu nu mai sunt în lume, iar ei în lume sunt ?i Eu vin la Tine. Parinte Sfinte, paze?te-i în numele Tau, în care Mi i-ai dat, ca sa fie una precum suntem ?i Noi.
Joh 17:12  Când eram cu ei în lume, Eu îi pazeam în numele Tau, pe cei ce Mi i-ai dat; ?i i-am pazit ?i n-a pierit nici unul dintre ei, decât fiul pierzarii, ca sa se împlineasca Scriptura.
Joh 17:13  Iar acum, vin la Tine ?i acestea le graiesc în lume, ca sa fie deplina bucuria Mea în ei.
Joh 17:14  Eu le-am dat cuvântul Tau, ?i lumea i-a urât, pentru ca nu sunt din lume, precum Eu nu sunt din lume.
Joh 17:15  Nu Ma rog ca sa-i iei din lume, ci ca sa-i paze?ti pe ei de cel viclean.
Joh 17:16  Ei nu sunt din lume, precum nici Eu nu sunt din lume.
Joh 17:17  Sfin?e?te-i pe ei întru adevarul Tau; cuvântul Tau este adevarul.
Joh 17:18  Precum M-ai trimis pe Mine în lume, ?i Eu i-am trimis pe ei în lume.
Joh 17:19  Pentru ei Eu Ma sfin?esc pe Mine Însumi, ca ?i ei sa fie sfin?i?i întru adevar.
Joh 17:20  Dar nu numai pentru ace?tia Ma rog, ci ?i pentru cei ce vor crede în Mine, prin cuvântul lor,
Joh 17:21  Ca to?i sa fie una, dupa cum Tu, Parinte, întru Mine ?i Eu întru Tine, a?a ?i ace?tia în Noi sa fie una, ca lumea sa creada ca Tu M-ai trimis.
Joh 17:22  ?i slava pe care Tu Mi-ai dat-o, le-am dat-o lor, ca sa fie una, precum Noi una suntem:
Joh 17:23  Eu întru ei ?i Tu întru Mine, ca ei sa fie desavâr?i?i întru unime, ?i sa cunoasca lumea ca Tu M-ai trimis ?i ca i-ai iubit pe ei, precum M-ai iubit pe Mine.
Joh 17:24  Parinte, voiesc ca, unde sunt Eu, sa fie împreuna cu Mine ?i aceia pr care Mi i-ai dat, ca sa vada slava mea pe care Mi-ai dat-o, pentru ca Tu M-ai iubit pe Mine mai înainte de întemeierea lumii.
Joh 17:25  Parinte drepte, lumea pe Tine nu te-a cunoscut, dar Eu Te-am cunoscut, ?i ace?tia au cunoscut ca Tu M-ai trimis.
Joh 17:26  ?i le-am facut cunoscut numele Tau ?i-l voi face cunoscut, ca iubirea cu care M-ai iubit Tu sa fie în ei ?i Eu în ei.
Joh 18:1  Zicând acestea, Iisus a ie?it cu ucenicii Lui dincolo de pârâul Cedrilor, unde era o gradina, în care a intrat El ?i ucenicii Sai.
Joh 18:2  Iar Iuda vânzatorul cuno?tea acest loc, pentru ca adesea Iisus ?i ucenicii Sai se adunau acolo.
Joh 18:3  Deci Iuda, luând oaste ?i slujitori, de la arhierei ?i farisei, a venit acolo cu felinare ?i cu faclii ?i cu arme.
Joh 18:4  Iar Iisus, ?tiind toate cele ce erau sa vina asupra Lui, a ie?it ?i le-a zis: Pe cine cauta?i?
Joh 18:5  Raspuns-au Lui: Pe Iisus Nazarineanul. El le-a zis: Eu sunt. Iar Iuda vânzatorul era ?i el cu ei.
Joh 18:6  Atunci când le-a spus: Eu sunt, ei s-au dat înapoi ?i au cazut la pamânt.
Joh 18:7  ?i iara?i i-a întrebat: Pe cine cauta?i? Iar ei au zis: Pe Iisus Nazarineanul.
Joh 18:8  Raspuns-a Iisus: V-am spus ca Eu sunt. Deci, daca Ma cauta?i pe Mine, lasa?i pe ace?tia sa se duca;
Joh 18:9  Ca sa se împlineasca cuvântul pe care l-a spus: Dintre cei pe care Mi i-ai dat, n-am pierdut pe nici unul.
Joh 18:10  Dar Simon-Petru, având sabie, a scos-o ?i a lovit pe sluga arhiereului ?i i-a taiat urechea dreapta; iar numele slugii era Malhus.
Joh 18:11  Deci a zis Iisus lui Petru: Pune sabia în teaca. Nu voi bea, oare, paharul pe care Mi l-a dat Tatal?
Joh 18:12  Deci osta?ii ?i comandantul ?i slujitorii iudeilor au prins pe Iisus ?i L-au legat.
Joh 18:13  ?i L-au dus întâi la Anna, caci era socrul lui Caiafa, care era arhiereu al anului aceluia.
Joh 18:14  ?i Caiafa era cel ce sfatuise pe iudei ca este de folos sa moara un om pentru popor.
Joh 18:15  ?i Simon-Petru ?i un alt ucenic mergeau dupa Iisus. Iar ucenicul acela era cunoscut arhiereului ?i a intrat împreuna cu Iisus în curtea arhiereului;
Joh 18:16  Iar Petru a stat la poarta, afara. Deci a ie?it celalalt ucenic, care era cunoscut arhiereului, ?i a vorbit cu portareasa ?i a bagat pe Petru înauntru.
Joh 18:17  Deci slujnica portareasa i-a zis lui Petru: Nu cumva e?ti ?i tu dintre ucenicii Omului acestuia? Acela a zis: Nu sunt.
Joh 18:18  Iar slugile ?i slujitorii facusera foc, ?i stateau ?i se încalzeau, ca era frig, ?i era cu ei ?i Petru, stând ?i încalzindu-se.
Joh 18:19  Deci arhiereul L-a întrebat pe Iisus despre ucenicii Lui ?i despre înva?atura Lui.
Joh 18:20  Iisus i-a raspuns: Eu am vorbit pe fa?a lumii; Eu am înva?at întotdeauna în sinagoga ?i în templu, unde se aduna to?i iudeii ?i nimic nu am vorbit în ascuns.
Joh 18:21  De ce Ma întrebi pe Mine? Întreaba pe cei ce au auzit ce le-am vorbit. Iata ace?tia ?tiu ce am spus Eu.
Joh 18:22  ?i zicând El acestea, unul din slujitorii, care era de fa?a, I-a dat lui Iisus o palma, zicând: A?a raspunzi Tu arhiereului?
Joh 18:23  Iisus i-a raspuns: Daca am vorbit rau, dovede?te ce este rau, iar daca am vorbit bine, de ce Ma ba?i?
Joh 18:24  Deci Anna L-a trimis legat la Caiafa arhiereul.
Joh 18:25  Iar Simon-Petru statea ?i se încalzea. Deci i-au zis: Nu cumva e?ti ?i tu dintre ucenicii Lui? El s-a lepadat ?i a zis: Nu sunt.
Joh 18:26  Una din slugile arhiereului, care era ruda cu cel caruia Petru îi taiase urechea, a zis: Nu te-am vazut eu pe tine, în gradina, cu El?
Joh 18:27  ?i iara?i s-a lepadat Petru ?i îndata a cântat coco?ul.
Joh 18:28  Deci L-au adus pe Iisus de la Caiafa la pretoriu; ?i era diminea?a. ?i ei n-au intrat în pretoriu, ca sa nu se spurce, ci sa manânce Pa?tile.
Joh 18:29  Deci Pilat a ie?it la ei, afara, ?i le-a zis: Ce învinuire aduce?i Omului Acestuia?
Joh 18:30  Ei au raspuns ?i i-au zis: Daca Acesta n-ar fi raufacator, nu ?i L-am fi dat ?ie.
Joh 18:31  Deci le-a zis Pilat: Lua?i-L voi ?i judeca?i-L dupa legea voastra. Iudeii însa i-au raspuns: Noua nu ne este îngaduit sa omorâm pe nimeni;
Joh 18:32  Ca sa se împlineasca cuvântul lui Iisus, pe care îl spusese, însemnând cu ce moarte avea sa moara.
Joh 18:33  Deci Pilat a intrat iara?i în pretoriu ?i a chemat pe Iisus ?i I-a zis: Tu e?ti regele iudeilor?
Joh 18:34  Raspuns-a Iisus: De la tine însu?i zici aceasta, sau al?ii ?i-au spus-o despre Mine?
Joh 18:35  Pilat a raspuns: Nu cumva sunt iudeu eu? Poporul Tau ?i arhiereii Te-au predat mie. Ce ai facut?
Joh 18:36  Iisus a raspuns: Împara?ia Mea nu este din lumea aceasta. Daca împara?ia Mea ar fi din lumea aceasta, slujitorii Mei s-ar fi luptat ca sa nu fiu predat iudeilor. Dar acum împara?ia Mea nu este de aici.
Joh 18:37  Deci i-a zis Pilat: A?adar e?ti Tu împarat? Raspuns-a Iisus: Tu zici ca Eu sunt împarat. Eu spre aceasta M-am nascut ?i pentru aceasta am venit în lume, ca sa dau marturie pentru adevar; oricine este din adevar asculta glasul Meu.
Joh 18:38  Pilat I-a zis: Ce este adevarul? ?i zicând aceasta, a ie?it iara?i la iudei ?i le-a zis: Eu nu gasesc în El nici o vina;
Joh 18:39  Dar este la voi obiceiul ca la Pa?ti sa va eliberez pe unul. Voi?i deci sa va eliberez pe regele iudeilor?
Joh 18:40  Deci au strigat iara?i, zicând: Nu pe Acesta, ci pe Baraba. Iar Baraba era tâlhar.
Joh 19:1  Deci atunci Pilat a luat pe Iisus ?i L-a biciuit.
Joh 19:2  ?i osta?ii, împletind cununa din spini, I-au pus-o pe cap ?i L-au îmbracat cu o mantie purpurie.
Joh 19:3  ?i veneau catre El ?i ziceau: Bucura-te, regele iudeilor! ?i-I dadeau palme.
Joh 19:4  ?i Pilat a ie?it iara?i afara ?i le-a zis: Iata vi-L aduc pe El afara, ca sa ?ti?i ca nu gasesc în El nici o vina.
Joh 19:5  Deci a ie?it Iisus afara, purtând cununa de spini ?i mantia purpurie. ?i le-a zis Pilat: Iata Omul!
Joh 19:6  Când L-au vazut deci arhiereii ?i slujitorii au strigat, zicând: Rastigne?te-L! Rastigne?te-L!  Zis-a lor Pilat: Lua?i-L voi ?i rastigni?i-L, caci eu nu-I gasesc nici o vina.
Joh 19:7  Iudeii i-au raspuns: Noi avem lege ?i dupa legea noastra El trebuie sa moara, ca S-a facut pe Sine Fiu al lui Dumnezeu.
Joh 19:8  Deci, când a auzit Pilat acest cuvânt, mai mult s-a temut.
Joh 19:9  ?i a intrat iara?i în pretoriu ?i I-a zis lui Iisus: De unde e?ti Tu? Iar Iisus nu i-a dat nici un raspuns.
Joh 19:10  Deci Pilat i-a zis: Mie nu-mi vorbe?ti? Nu ?tii ca am putere sa Te eliberez ?i putere am sa Te rastignesc?
Joh 19:11  Iisus a raspuns: N-ai avea nici o putere asupra Mea, daca nu ?i-ar fi fost dat ?ie de sus. De aceea cel ce M-a predat ?ie mai mare pacat are.
Joh 19:12  Pentru aceasta, Pilat cauta sa-L elibereze; iar iudeii strigau zicând: Daca Îl eliberezi pe Acesta, nu e?ti prieten al Cezarului. Oricine se face pe sine împarat este împotriva Cezarului.
Joh 19:13  Deci Pilat, auzind cuvintele acestea, L-a dus afara pe Iisus ?i a ?ezut pe scaunul de judecata, în locul numit pardosit cu pietre, iar evreie?te Gabbata.
Joh 19:14  ?i era Vinerea Pa?tilor, cam la al ?aselea ceas, ?i a zis Pilat iudeilor: Iata Împaratul vostru.
Joh 19:15  Deci au strigat aceia: Ia-L! Ia-L! Rastigne?te-L! Pilat le-a zis: Sa rastignesc pe Împaratul vostru? Arhiereii au raspuns: Nu avem împarat decât pe Cezarul.
Joh 19:16  Atunci L-a predat lor ca sa fie rastignit. ?i ei au luat pe Iisus ?i L-au dus ca sa fie rastignit.
Joh 19:17  ?i ducându-?i crucea, a ie?it la locul ce se cheama al Capa?ânii, care evreie?te se zice Golgota,
Joh 19:18  Unde L-au rastignit, ?i împreuna cu El pe al?i doi, de o parte ?i de alta, iar în mijloc pe Iisus.
Joh 19:19  Iar Pilat a scris ?i titlu ?i l-a pus deasupra Crucii. ?i era scris: Iisus Nazarineanul, Împaratul iudeilor!
Joh 19:20  Deci mul?i dintre iudei au citit acest titlu, caci locul unde a fost rastignit Iisus era aproape de cetate. ?i era scris: evreie?te, latine?te ?i grece?te.
Joh 19:21  Deci arhiereii iudeilor au zis lui Pilat: Nu scrie: Împaratul iudeilor, ci ca Acela a zis: Eu sunt Împaratul iudeilor.
Joh 19:22  Pilat a raspuns: Ce am scris, am scris.
Joh 19:23  Dupa ce au rastignit pe Iisus, osta?ii au luat hainele Lui ?i le-au facut patru par?i, fiecarui osta? câte o parte, ?i cama?a. Dar cama?a era fara cusatura, de sus ?esuta în întregime.
Joh 19:24  Deci au zis unii catre al?ii: Sa n-o sfâ?iem, ci sa aruncam sor?ii pentru ea, a cui sa fie; ca sa se împlineasca Scriptura care zice: "Împar?it-au hainele Mele loru?i, ?i pentru cama?a Mea au aruncat sor?ii". A?adar osta?ii acestea au facut.
Joh 19:25  ?i stateau, lânga crucea lui Iisus, mama Lui ?i sora mamei Lui, Maria lui Cleopa, ?i Maria Magdalena.
Joh 19:26  Deci Iisus, vazând pe mama Sa ?i pe ucenicul pe care Îl iubea stând alaturi, a zis mamei Sale: Femeie, iata fiul tau!
Joh 19:27  Apoi a zis ucenicului: Iata mama ta! ?i din ceasul acela ucenicul a luat-o la sine.
Joh 19:28  Dupa aceea, ?tiind Iisus ca toate s-au savâr?it acum, ca sa se împlineasca Scriptura, a zis: Mi-e sete.
Joh 19:29  ?i era acolo un vas plin cu o?et; iar cei care Îl lovisera, punând în vârful unei trestii de isop un burete înmuiat în o?et, l-au dus la gura Lui.
Joh 19:30  Deci dupa ce a luat o?etul, Iisus a zis: Savâr?itu-s-a. ?i plecându-?i capul, ?i-a dat duhul.
Joh 19:31  Deci iudeii, fiindca era vineri, ca sa nu ramâna trupurile sâmbata pe cruce, caci era mare ziua sâmbetei aceleia, au rugat pe Pilat sa le zdrobeasca fluierele picioarelor ?i sa-i ridice.
Joh 19:32  Deci au venit osta?ii ?i au zdrobit fluierele celui dintâi ?i ale celuilalt, care era rastignit împreuna cu el.
Joh 19:33  Dar venind la Iisus, daca au vazut ca deja murise, nu I-au zdrobit fluierele.
Joh 19:34  Ci unul din osta?i cu suli?a a împuns coasta Lui ?i îndata a ie?it sânge ?i apa.
Joh 19:35  ?i cel ce a vazut a marturisit ?i marturia lui e adevarata; ?i acela ?tie ca spune adevarul, ca ?i voi sa crede?i.
Joh 19:36  Caci s-au facut acestea, ca sa se împlineasca Scriptura: "Nu I se va zdrobi nici un os".
Joh 19:37  ?i iara?i alta Scriptura zice: "Vor privi la Acela pe care L-au împuns".
Joh 19:38  Dupa acestea Iosif din Arimateea, fiind ucenic al lui Iisus, dar într-ascuns, de frica iudeilor, a rugat pe Pilat ca sa ridice trupul lui Iisus. ?i Pilat i-a dat voie. Deci a venit ?i a ridicat trupul Lui.
Joh 19:39  ?i a venit ?i Nicodim, cel care venise la El mai înainte noaptea, aducând ca la o suta de litre de amestec de smirna ?i aloe.
Joh 19:40  Au luat deci trupul lui Iisus ?i l-au înfa?urat în giulgiu cu miresme, precum este obiceiul de înmormântare la iudei.
Joh 19:41  Iar în locul unde a fost rastignit era o gradina, ?i în gradina un mormânt nou, în care nu mai fusese nimeni îngropat.
Joh 19:42  Deci, din pricina vinerii iudeilor, acolo L-au pus pe Iisus, pentru ca mormântul era aproape.
Joh 20:1  Iar în ziua întâia a saptamânii (duminica), Maria Magdalena a venit la mormânt dis-de-diminea?a, fiind înca întuneric, ?i a vazut piatra ridicata de pe mormânt.
Joh 20:2  Deci a alergat ?i a venit la Simon-Petru ?i la celalalt ucenic pe care-l iubea Iisus, ?i le-a zis: Au luat pe Domnul din mormânt ?i noi nu ?tim unde L-au pus.
Joh 20:3  Deci a ie?it Petru ?i celalalt ucenic ?i veneau la mormânt.
Joh 20:4  ?i cei doi alergau împreuna, dar celalalt ucenic, alergând înainte, mai repede decât Petru, a sosit cel dintâi la mormânt.
Joh 20:5  ?i, aplecându-se, a vazut giulgiurile puse jos, dar n-a intrat.
Joh 20:6  A sosit ?i Simon-Petru, urmând dupa el, ?i a intrat în mormânt ?i a vazut giulgiurile puse jos,
Joh 20:7  Iar mahrama, care fusese pe capul Lui, nu era pusa împreuna cu giulgiurile, ci înfa?urata, la o parte, într-un loc.
Joh 20:8  Atunci a intrat ?i celalalt ucenic care sosise întâi la mormânt, ?i a vazut ?i a crezut.
Joh 20:9  Caci înca nu ?tiau Scriptura, ca Iisus trebuia sa învieze din mor?i.
Joh 20:10  ?i s-au dus ucenicii iara?i la ai lor.
Joh 20:11  Iar Maria statea afara lânga mormânt plângând. ?i pe când plângea, s-a aplecat spre mormânt.
Joh 20:12  ?i a vazut doi îngeri în ve?minte albe ?ezând, unul catre cap ?i altul catre picioare, unde zacuse trupul lui Iisus.
Joh 20:13  ?i aceia i-au zis: Femeie, de ce plângi? Pe cine cau?i? Ea le-a zis: Ca au luat pe Domnul meu ?i nu ?tiu unde L-au pus.
Joh 20:14  Zicând acestea, ea s-a întors cu fa?a ?i a vazut pe Iisus stând, dar nu ?tia ca este Iisus.
Joh 20:15  Zis-a ei Iisus: Femeie, de ce plângi? Pe cine cau?i? Ea, crezând ca este gradinarul, I-a zis: Doamne, daca Tu L-ai luat, spune-mi unde L-ai pus ?i eu Îl voi ridica.
Joh 20:16  Iisus i-a zis: Maria! Întorcându-se, aceea I-a zis evreie?te: Rabuni! (adica, Înva?atorule)
Joh 20:17  Iisus i-a zis: Nu te atinge de Mine, caci înca nu M-am suit la Tatal Meu. Mergi la fra?ii Mei ?i le spune: Ma sui la Tatal Meu ?i Tatal vostru ?i la Dumnezeul Meu ?i Dumnezeul vostru.
Joh 20:18  ?i a venit Maria Magdalena vestind ucenicilor ca a vazut pe Domnul ?i acestea i-a zis ei.
Joh 20:19  ?i fiind seara, în ziua aceea, întâia a saptamânii (duminica), ?i u?ile fiind încuiate, unde erau aduna?i ucenicii de frica iudeilor, a venit Iisus ?i a stat în mijloc ?i le-a zis: Pace voua!
Joh 20:20  ?i zicând acestea, le-a aratat mâinile ?i coasta Sa. Deci s-au bucurat ucenicii, vazând pe Domnul.
Joh 20:21  ?i Iisus le-a zis iara?i: Pace voua! Precum M-a trimis pe Mine Tatal, va trimit ?i Eu pe voi.
Joh 20:22  ?i zicând acestea, a suflat asupra lor ?i le-a zis: Lua?i Duh Sfânt;
Joh 20:23  Carora ve?i ierta pacatele, le vor fi iertate ?i carora le ve?i ?ine, vor fi ?inute.
Joh 20:24  Iar Toma, unul din cei doisprezece, cel numit Geamanul, nu era cu ei când a venit Iisus.
Joh 20:25  Deci au zis lui ceilal?i ucenici: Am vazut pe Domnul! Dar el le-a zis: Daca nu voi vedea, în mâinile Lui, semnul cuielor, ?i daca nu voi pune degetul meu în semnul cuielor, ?i daca nu voi pune mâna mea în coasta Lui, nu voi crede.
Joh 20:26  ?i dupa opt zile, ucenicii Lui erau iara?i înauntru, ?i Toma, împreuna cu ei. ?i a venit Iisus, u?ile fiind încuiate, ?i a stat în mijloc ?i a zis: Pace voua!
Joh 20:27  Apoi a zis lui Toma: Adu degetul tau încoace ?i vezi mâinile Mele ?i adu mâna ta ?i o pune în coasta Mea ?i nu fi necredincios ci credincios.
Joh 20:28  A raspuns Toma ?i I-a zis: Domnul meu ?i Dumnezeul meu!
Joh 20:29  Iisus I-a zis: Pentru ca M-ai vazut ai crezut. Ferici?i cei ce n-au vazut ?i au crezut!
Joh 20:30  Deci ?i alte multe minuni a facut Iisus înaintea ucenicilor Sai, care nu sunt scrise în cartea aceasta.
Joh 20:31  Iar acestea s-au scris, ca sa crede?i ca Iisus este Hristosul, Fiul lui Dumnezeu, ?i, crezând, sa ave?i via?a în numele Lui.
Joh 21:1  Dupa acestea, Iisus S-a aratat iara?i ucenicilor la Marea Tiberiadei, ?i S-a aratat a?a:
Joh 21:2  Erau împreuna Simon-Petru ?i Toma, cel numit Geamanul, ?i Natanael, cel din Cana Galileii, ?i fiii lui Zevedeu ?i al?i doi din ucenicii Lui.
Joh 21:3  Simon-Petru le-a zis: Ma duc sa pescuiesc. ?i i-au zis ei: Mergem ?i noi cu tine. ?i au ie?it ?i s-au suit în corabie, ?i în noaptea aceea n-au prins nimic.
Joh 21:4  Iar facându-se diminea?a, Iisus a stat la ?arm; dar ucenicii n-au ?tiut ca este Iisus.
Joh 21:5  Deci le-a zis Iisus: Fiilor, nu cumva ave?i ceva de mâncare? Ei I-au raspuns: Nu.
Joh 21:6  Iar El le-a zis: Arunca?i mreaja în partea dreapta a corabiei ?i ve?i afla. Deci au aruncat-o ?i nu mai puteau s-o traga de mul?imea pe?tilor.
Joh 21:7  ?i a zis lui Petru ucenicul acela pe care-l iubea Iisus: Domnul este! Deci Simon-Petru, auzind ca este Domnul, ?i-a încins haina, caci era dezbracat, ?i s-a aruncat în apa.
Joh 21:8  ?i ceilal?i ucenici au venit cu corabia, caci nu erau departe de ?arm, ci la doua sute de co?i, tragând mreaja cu pe?ti.
Joh 21:9  Deci, când au ie?it la ?arm, au vazut jar pus jos ?i pe?te pus deasupra, ?i pâine.
Joh 21:10  Iisus le-a zis: Aduce?i din pe?tele pe care l-a?i prins acum.
Joh 21:11  Simon-Petru s-a suit în corabie ?i a tras mreaja la ?arm, plina de pe?ti mari: o suta cincizeci ?i trei, ?i, de?i erau atâ?ia, nu s-a rupt mreaja.
Joh 21:12  Iisus le-a zis: Veni?i de prânzi?i. ?i nici unul din ucenici nu îndraznea sa-L întrebe: Cine e?ti Tu?, ?tiind ca este Domnul.
Joh 21:13  Deci a venit Iisus ?i a luat pâinea ?i le-a dat lor, ?i de asemenea ?i pe?tele.
Joh 21:14  Aceasta este, acum, a treia oara când Iisus S-a aratat ucenicilor, dupa ce S-a sculat din mor?i.
Joh 21:15  Deci dupa ce au prânzit, a zis Iisus lui Simon-Petru: Simone, fiul lui Iona, Ma iube?ti tu mai mult decât ace?tia? El I-a raspuns: Da, Doamne, Tu ?tii ca Te iubesc. Zis-a lui: Pa?te mielu?eii Mei.
Joh 21:16  Iisus i-a zis iara?i, a doua oara: Simone, fiul lui Iona, Ma iube?ti? El I-a zis: Da, Doamne, Tu ?tii ca Te iubesc. Zis-a Iisus lui: Pastore?te oile Mele.
Joh 21:17  Iisus i-a zis a treia oara: Simone, fiul lui Iona, Ma iube?ti? Petru s-a întristat, ca i-a zis a treia oara: Ma iube?ti? ?i I-a zis: Doamne, Tu ?tii toate. Tu ?tii ca Te iubesc. Iisus i-a zis: Pa?te oile Mele.
Joh 21:18  Adevarat, adevarat zic ?ie: Daca erai mai tânar, te încingeai singur ?i umblai unde voiai; dar când vei îmbatrâni, vei întinde mâinile tale ?i altul te va încinge ?i te va duce unde nu voie?ti.
Joh 21:19  Iar aceasta a zis-o, însemnând cu ce fel de moarte va preaslavi pe Dumnezeu. ?i spunând aceasta, i-a zis: Urmeaza Mie.
Joh 21:20  Dar întorcându-se, Petru a vazut venind dupa el pe ucenicul pe care-l iubea Iisus, acela care la Cina s-a rezemat de pieptul Lui ?i I-a zis: Doamne, cine este cel ce Te va vinde?
Joh 21:21  Pe acesta deci, vazându-l, Petru a zis lui Iisus: Doamne, dar cu acesta ce se va întâmpla?
Joh 21:22  Zis-a Iisus lui: Daca voiesc ca acesta sa ramâna pâna voi veni, ce ai tu? Tu urmeaza Mie.
Joh 21:23  De aceea a ie?it cuvântul acesta între fra?i, ca ucenicul acela nu va muri; dar Iisus nu i-a spus ca nu va muri ci: daca voiesc ca acesta sa ramâna pâna voi veni, ce ai tu?
Joh 21:24  Acesta este ucenicul care marturise?te despre acestea ?i care a scris acestea, ?i ?tim ca marturia lui e adevarata.
Joh 21:25  Dar sunt ?i alte multe lucruri pe care le-a facut Iisus ?i care, daca s-ar fi scris cu de-amanuntul, cred ca lumea aceasta n-ar cuprinde car?ile ce s-ar fi scris. Amin.


\end{document}