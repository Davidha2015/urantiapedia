\begin{document}

\title{Faptele Apostolilor}


\chapter{1}

\par 1 Cuvântul cel dintâi l-am facut o, Teofile, despre toate cele ce a început Iisus a face ?i a înva?a,
\par 2 Pâna în ziua în care S-a înal?at la cer, poruncind prin Duhul Sfânt apostolilor pe care i-a ales,
\par 3 Carora S-a ?i înfa?i?at pe Sine viu dupa patima Sa prin multe semne doveditoare, aratându-li-Se timp de patruzeci de zile ?i vorbind cele despre împara?ia lui Dumnezeu.
\par 4 ?i cu ei petrecând, le-a poruncit sa nu se departeze de Ierusalim, ci sa a?tepte fagaduin?a Tatalui, pe care (a zis El) a?i auzit-o de la Mine:
\par 5 Ca Ioan a botezat cu apa, iar voi ve?i fi boteza?i cu Duhul Sfânt, nu mult dupa aceste zile.
\par 6 Iar ei, adunându-se, Îl întrebau, zicând: Doamne, oare, în acest timp vei a?eza Tu, la loc, împara?ia lui Israel?
\par 7 El a zis catre ei: Nu este al vostru a ?ti anii sau vremile pe care Tatal le-a pus în stapânirea Sa,
\par 8 Ci ve?i lua putere, venind Duhul Sfânt peste voi, ?i Îmi ve?i fi Mie martori în Ierusalim ?i în toata Iudeea ?i în Samaria ?i pâna la marginea pamântului.
\par 9 ?i acestea zicând, pe când ei priveau, S-a înal?at ?i un nor L-a luat de la ochii lor.
\par 10 ?i privind ei, pe când El mergea la cer, iata doi barba?i au stat lânga ei, îmbraca?i în haine albe,
\par 11 Care au ?i zis: Barba?i galileieni, de ce sta?i privind la cer? Acest Iisus care S-a înal?at de la voi la cer, astfel va ?i veni, precum L-a?i vazut mergând la cer.
\par 12 Atunci ei s-au întors la Ierusalim de la muntele ce se cheama al Maslinilor, care este aproape de Ierusalim, cale de o sâmbata.
\par 13 ?i când au intrat, s-au suit în încaperea de sus, unde se adunau de obicei: Petru ?i Ioan ?i Iacov ?i Andrei, Filip ?i Toma, Bartolomeu ?i Matei, Iacov al lui Alfeu ?i Simon Zelotul ?i Iuda al lui Iacov.
\par 14 To?i ace?tia, într-un cuget, staruiau în rugaciune împreuna cu femeile ?i cu Maria, mama lui Iisus ?i cu fra?ii Lui.
\par 15 ?i în zilele acelea, sculându-se Petru în mijlocul fra?ilor (iar numarul lor era ca la o suta douazeci), a zis:
\par 16 Barba?i fra?i, trebuia sa se împlineasca Scriptura aceasta pe care Duhul Sfânt, prin gura lui David, a spus-o dinainte despre Iuda, care s-a facut calauza celor ce L-au prins pe Iisus.
\par 17 Caci era numarat cu noi ?i luase sor?ul acestei slujiri.
\par 18 Deci acesta a dobândit o ?arina din plata nedrepta?ii ?i, cazând cu capul înainte, a crapat pe la mijloc ?i i s-au varsat toate maruntaiele.
\par 19 ?i s-a facut cunoscuta aceasta tuturor celor ce locuiesc în Ierusalim, încât ?arina aceasta s-a numit în limba lor Hacheldamah, adica ?arina Sângelui.
\par 20 Caci este scris în Cartea Psalmilor: "Faca-se casa lui pustie ?i sa nu aiba cine sa locuiasca în ea! ?i slujirea lui s-o ia altul".
\par 21 Deci trebuie ca unul din ace?ti barba?i, care s-au adunat cu noi în timpul cât a petrecut între noi Domnul Iisus,
\par 22 Începând de la botezul lui Ioan, pâna în ziua în care S-a înal?at de la noi, sa fie împreuna cu noi martor al învierii Lui.
\par 23 ?i au pus înainte pe doi: pe Iosif, numit Barsaba, zis ?i Iustus, ?i pe Matia.
\par 24 ?i, rugându-se, au zis: Tu, Doamne, Care cuno?ti inimile tuturor, arata pe care din ace?tia doi l-ai ales,
\par 25 Ca sa ia locul acestei slujiri ?i al apostoliei din care Iuda a cazut, ca sa mearga în locul lui.
\par 26 ?i au tras la sor?i, ?i sor?ul a cazut pe Matia, ?i s-a socotit împreuna cu cei unsprezece apostoli.

\chapter{2}

\par 1 ?i când a sosit ziua Cincizecimii, erau to?i împreuna în acela?i loc.
\par 2 ?i din cer, fara de veste, s-a facut un vuiet, ca de suflare de vânt ce vine repede, ?i a umplut toata casa unde ?edeau ei.
\par 3 ?i li s-au aratat, împar?ite, limbi ca de foc ?i au ?ezut pe fiecare dintre ei.
\par 4 ?i s-au umplut to?i de Duhul Sfânt ?i au început sa vorbeasca în alte limbi, precum le dadea lor Duhul a grai.
\par 5 ?i erau în Ierusalim locuitori iudei, barba?i cucernici, din toate neamurile care sunt sub cer.
\par 6 ?i iscându-se vuietul acela, s-a adunat mul?imea ?i s-a tulburat, caci fiecare îi auzea pe ei vorbind în limba sa.
\par 7 ?i erau uimi?i to?i ?i se minunau zicând: Iata, nu sunt ace?tia care vorbesc to?i galileieni?
\par 8 ?i cum auzim noi fiecare limba noastra, în care ne-am nascut?
\par 9 Par?i ?i mezi ?i elami?i ?i cei ce locuiesc în Mesopotamia, în Iudeea ?i în Capadocia, în Pont ?i în Asia,
\par 10 În Frigia ?i în Pamfilia, în Egipt ?i în par?ile Libiei cea de lânga Cirene, ?i romani în treacat, iudei ?i prozeli?i,
\par 11 Cretani ?i arabi, îi auzim pe ei vorbind în limbile noastre despre faptele minunate ale lui Dumnezeu!
\par 12 ?i to?i erau uimi?i ?i nu se dumireau, zicând unul catre altul: Ce va sa fie aceasta?
\par 13 Iar al?ii batjocorindu-i, ziceau ca sunt plini de must.
\par 14 ?i stând Petru cu cei unsprezece, a ridicat glasul ?i le-a vorbit: Barba?i iudei, ?i to?i care locui?i în Ierusalim, aceasta sa va fie cunoscuta ?i lua?i în urechi cuvintele mele;
\par 15 Ca ace?tia nu sunt be?i, cum vi se pare voua, caci este al treilea ceas din zi;
\par 16 Ci aceasta este ce s-a spus prin proorocul Ioil:
\par 17 "Iar în zilele din urma, zice Domnul, voi turna din Duhul Meu peste tot trupul ?i fiii vo?tri ?i fiicele voastre vor prooroci ?i cei mai tineri ai vo?tri vor vedea vedenii ?i batrânii vo?tri vise vor visa.
\par 18 Înca ?i peste slugile Mele ?i peste slujnicele Mele voi turna în acele zile, din Duhul Meu ?i vor prooroci.
\par 19 ?i minuni voi face sus în cer ?i jos pe pamânt semne: sânge, foc ?i fumegare de fum.
\par 20 Soarele se va schimba în întuneric ?i luna în sânge, înainte de a veni ziua Domnului, cea mare ?i stralucita.
\par 21 ?i tot cel ce va chema numele Domnului se va mântui".
\par 22 Barba?i israeli?i, asculta?i cuvintele acestea: Pe Iisus Nazarineanul, barbat adeverit între voi de Dumnezeu, prin puteri, prin minuni ?i prin semne pe care le-a facut prin El Dumnezeu în mijlocul vostru, precum ?i voi ?ti?i,
\par 23 Pe Acesta, fiind dat, dupa sfatul cel rânduit ?i dupa ?tiin?a cea dinainte a lui Dumnezeu, voi L-a?i luat ?i, pironindu-L, prin mâinile celor fara de lege, L-a?i omorât,
\par 24 Pe Care Dumnezeu L-a înviat, dezlegând durerile mor?ii, întrucât nu era cu putin?a ca El sa fie ?inut de ea.
\par 25 Caci David zice despre El: "Totdeauna am vazut pe Domnul înaintea mea, caci El este de-a dreapta mea, ca sa nu ma clatin.
\par 26 De aceea s-a bucurat inima mea ?i s-a veselit limba mea; chiar ?i trupul meu se va odihni întru nadejde.
\par 27 Caci nu vei lasa sufletul meu în iad, nici nu vei da pe cel sfânt al Tau sa vada stricaciune.
\par 28 Facutu-mi-ai cunoscute caile vie?ii; cu înfa?i?area Ta ma vei umple de bucurie".
\par 29 Barba?i fra?i, cuvine-se a vorbi cu îndraznire catre voi despre stramo?ul David, ca a murit ?i s-a îngropat, iar mormântul lui este la noi, pâna în ziua aceasta.
\par 30 Deci el, fiind prooroc ?i ?tiind ca Dumnezeu i S-a jurat cu juramânt sa a?eze pe tronu-i din rodul coapselor lui,
\par 31 Mai înainte vazând, a vorbit despre învierea lui Hristos: ca n-a fost lasat în iad sufletul Lui ?i nici trupul Lui n-a vazut putreziciunea.
\par 32 Dumnezeu a înviat pe Acest Iisus, Caruia noi to?i suntem martori.
\par 33 Deci, înal?ându-Se prin dreapta lui Dumnezeu ?i primind de la Tatal fagaduin?a Duhului Sfânt, L-a revarsat pe Acesta, cum vede?i ?i auzi?i voi.
\par 34 Caci David nu s-a suit la ceruri, dar el a zis: "Zis-a Domnul Domnului meu: ?ezi de-a dreapta Mea,
\par 35 Pâna ce voi pune pe vrajma?ii Tai a?ternut picioarelor Tale".
\par 36 Cu siguran?a sa ?tie deci toata casa lui Israel ca Dumnezeu, pe Acest Iisus pe Care voi L-a?i rastignit, L-a facut Domn ?i Hristos.
\par 37 Ei auzind acestea, au fost patrun?i la inima ?i au zis catre Petru ?i ceilal?i apostoli: Barba?i fra?i, ce sa facem?
\par 38 Iar Petru a zis catre ei: Pocai?i-va ?i sa se boteze fiecare dintre voi în numele lui Iisus Hristos, spre iertarea pacatelor voastre, ?i ve?i primi darul Duhului Sfânt.
\par 39 Caci voua este data fagaduin?a ?i copiilor vo?tri ?i tuturor celor de departe, pe oricâ?i îi va chema Domnul Dumnezeul nostru.
\par 40 ?i cu alte mai multe vorbe marturisea ?i-i îndemna, zicând: Mântui?i-va de acest neam viclean.
\par 41 Deci cei ce au primit cuvântul lui s-au botezat ?i în ziua aceea s-au adaugat ca la trei mii de suflete.
\par 42 ?i staruiau în înva?atura apostolilor ?i în împarta?ire, în frângerea pâinii ?i în rugaciuni.
\par 43 ?i tot sufletul era cuprins de teama, caci multe minuni ?i semne se faceau în Ierusalim prin apostoli, ?i mare frica îi stapânea pe to?i.
\par 44 Iar to?i cei ce credeau erau laolalta ?i aveau toate de ob?te.
\par 45 ?i î?i vindeau bunurile ?i averile ?i le împar?eau tuturor, dupa cum avea nevoie fiecare.
\par 46 ?i în fiecare zi, staruiau într-un cuget în templu ?i, frângând pâinea în casa, luau împreuna hrana întru bucurie ?i întru cura?ia inimii.
\par 47 Laudând pe Dumnezeu ?i având har la tot poporul. Iar Domnul adauga zilnic Bisericii pe cei ce se mântuiau.

\chapter{3}

\par 1 Iar Petru ?i Ioan se suiau la templu pentru rugaciunea din ceasul al noualea.
\par 2 ?i era un barbat olog din pântecele mamei sale, pe care-l aduceau ?i-l puneau în fiecare zi la poarta templului, zisa Poarta Frumoasa, ca sa ceara milostenie de la cei ce intrau în templu,
\par 3 Care, vazând ca Petru ?i Ioan vor sa intre în templu, le-a cerut milostenie.
\par 4 Iar Petru, cautând spre el, împreuna cu Ioan, a zis: Prive?te noi;
\par 5 Iar el se uita la ei cu luare-aminte, a?teptând sa primeasca ceva de la ei.
\par 6 Iar Petru a zis: Argint ?i aur nu am; dar ce am, aceea î?i dau. În numele lui Iisus Hristos Nazarineanul, scoala-te ?i umbla!
\par 7 ?i apucându-l de mâna dreapta, l-a ridicat ?i îndata gleznele ?i talpile picioarelor lui s-au întarit.
\par 8 ?i sarind, a stat în picioare ?i umbla, ?i a intrat cu ei în templu, umblând ?i sarind ?i laudând pe Dumnezeu.
\par 9 ?i tot poporul l-a vazut umblând ?i laudând pe Dumnezeu.
\par 10 ?i îl cuno?teau ca el era cel care ?edea pentru milostenie, la Poarta Frumoasa a templului, ?i s-au umplut de uimire ?i de mirare pentru ceea ce i s-a întâmplat.
\par 11 ?i ?inându-se el de Petru ?i de Ioan, tot poporul, uimit, alerga la ei, în pridvorul numit al lui Solomon.
\par 12 Iar Petru, vazând aceasta, a raspuns catre popor: Barba?i israeli?i, de ce va mira?i de acest lucru, sau de ce sta?i cu ochii a?inti?i la noi, ca ?i cum cu a noastra putere sau cucernicie l-am fi facut pe acesta sa umble?
\par 13 Dumnezeul lui Avraam ?i al lui Isaac ?i al lui Iacov, Dumnezeul parin?ilor no?tri a slavit pe Fiul Sau Iisus, pe Care voi L-a?i predat ?i L-a?i tagaduit în fa?a lui Pilat, care gasise cu cale sa-L elibereze.
\par 14 Dar voi v-a?i lepadat de Cel sfânt ?i drept ?i a?i cerut sa va daruiasca un barbat uciga?.
\par 15 Iar pe Începatorul vie?ii L-a?i omorât, pe Care însa Dumnezeu L-a înviat din mor?i ?i ai Carui martori suntem noi.
\par 16 ?i prin credin?a în numele Lui, pe acesta pe care îl vede?i ?i îl cunoa?te?i, l-a întarit numele lui Iisus ?i credin?a cea întru El i-a dat lui întregirea aceasta a trupului, înaintea voastra, a tuturor.
\par 17 ?i acum, fra?ilor, ?tiu ca din ne?tiin?a a?i facut rau ca ?i mai-marii vo?tri.
\par 18 Dar Dumnezeu a împlinit astfel cele ce vestise dinainte prin gura tuturor proorocilor, ca Hristosul Sau va patimi.
\par 19 Deci pocai?i-va ?i va întoarce?i, ca sa se ?tearga pacatele voastre,
\par 20 Ca sa vina de la fa?a Domnului vremuri de u?urare ?i ca sa va trimita pe Cel mai dinainte vestit voua, pe Iisus Hristos,
\par 21 Pe Care trebuie sa-L primeasca Cerul pâna la vremile stabilirii din nou a tuturor celor despre care a vorbit Dumnezeu prin gura sfin?ilor Sai prooroci din veac.
\par 22 Moise a zis catre parin?i: "Domnul Dumnezeu va ridica voua dintre fra?ii vo?tri Prooroc ca mine. Pe El sa-L asculta?i în toate câte va va spune.
\par 23 ?i tot sufletul care nu va asculta de Proorocul Acela, va fi nimicit din popor".
\par 24 Iar to?i proorocii de la Samuel ?i cei câ?i le-au urmat au vorbit ?i au vestit zilele acestea.
\par 25 Voi sunte?i fiii proorocilor ?i ai legamântului pe care l-a încheiat Dumnezeu cu parin?ii no?tri, graind catre Avraam: "?i întru semin?ia ta se vor binecuvânta toate neamurile pamântului".
\par 26 Dumnezeu, înviind pe Fiul Sau, L-a trimis întâi la voi, sa va binecuvânteze, ca fiecare sa se întoarca de la rauta?ile sale.

\chapter{4}

\par 1 Dar pe când vorbeau ei catre popor, au venit peste ei preo?ii, capetenia garzii templului ?i saducheii,
\par 2 Mâniindu-se ca ei înva?a poporul ?i vestesc întru Iisus învierea din mor?i.
\par 3 ?i punând mâna pe ei, i-au pus sub paza, pâna a doua zi, caci acum era seara.
\par 4 Totu?i mul?i din cei ce auzisera cuvântul au crezut ?i numarul barba?ilor credincio?i s-a facut ca la cinci mii.
\par 5 ?i a doua zi s-au adunat capeteniile lor ?i batrânii ?i carturarii din Ierusalim,
\par 6 ?i Anna arhiereul ?i Caiafa ?i Ioan ?i Alexandru ?i câ?i erau din neamul arhieresc,
\par 7 ?i punându-i în mijloc, îi întrebau: Cu ce putere sau în al cui nume a?i facut voi aceasta?
\par 8 Atunci Petru, plin fiind de Duhul Sfânt, le-a vorbit: Capetenii ale poporului ?i batrâni ai lui Israel,
\par 9 Fiindca noi suntem astazi cerceta?i pentru facere de bine unui om bolnav, prin cine a fost el vindecat,
\par 10 Cunoscut sa va fie voua tuturor, ?i la tot poporul Israel, ca în numele lui Iisus Hristos Nazarineanul, pe Care voi L-a?i rastignit, dar pe Care Dumnezeu L-a înviat din mor?i, întru Acela sta acesta sanatos înaintea voastra!
\par 11 Acesta este piatra cea neluata în seama de catre voi, zidarii, care a ajuns în capul unghiului;
\par 12 ?i întru nimeni altul nu este mântuirea, caci nu este sub cer nici un alt nume, dat între oameni, în care trebuie sa ne mântuim noi.
\par 13 ?i vazând ei îndrazneala lui Petru ?i a lui Ioan ?i ?tiind ca sunt oameni fara carte ?i simpli, se mirau, ?i îi cuno?teau ca fusesera împreuna cu Iisus;
\par 14 ?i vazând pe omul cel tamaduit, stând cu ei, n-aveau nimic de zis împotriva,
\par 15 Dar poruncindu-le sa iasa afara din sinedriu, vorbeau între ei,
\par 16 Zicând: ce vom face acestor oameni? Caci este învederat tuturor celor ce locuiesc în Ierusalim ca prin ei s-a facut o minune cunoscuta ?i nu putem sa tagaduim.
\par 17 Dar ca aceasta sa nu se raspândeasca mai mult în popor, sa le poruncim cu amenin?are sa nu mai vorbeasca, în numele acesta, nici unui om.
\par 18 ?i chemându-i, le-au poruncit ca nicidecum sa nu mai graiasca, nici sa mai înve?e în numele lui Iisus.
\par 19 Iar Petru ?i Ioan, raspunzând, au zis catre ei: Judeca?i daca este drept înaintea lui Dumnezeu sa ascultam de voi mai mult decât de Dumnezeu.
\par 20 Caci noi nu putem sa nu vorbim cele ce am vazut ?i am auzit.
\par 21 Dar ei, amenin?ându-i din nou, le-au dat drumul, negasind nici un chip cum sa-i pedepseasca, din cauza poporului, fiindca to?i slaveau pe Dumnezeu, pentru ceea ce se facuse.
\par 22 Caci omul cu care se facuse aceasta minune a vindecarii avea mai mult ca patruzeci de ani.
\par 23 Fiind slobozi?i, au venit la ai lor ?i le-au spus câte le-au vorbit lor arhiereii ?i batrânii.
\par 24 Iar ei, auzind, într-un cuget au ridicat glasul catre Dumnezeu ?i au zis: Stapâne, Dumnezeule, Tu, Care ai facut cerul ?i pamântul ?i marea ?i toate cele ce sunt în ele,
\par 25 Care prin Duhul Sfânt ?i prin gura parintelui nostru David, slujitorul Tau, ai zis: Pentru ce s-au întarâtat neamurile ?i popoarele au cugetat cele de?arte?
\par 26 Ridicatu-s-au regii pamântului ?i capeteniile s-au adunat laolalta împotriva Domnului ?i împotriva Unsului Lui,
\par 27 Caci asupra Sfântului Tau Fiu Iisus, pe care Tu L-ai uns, s-au adunat laolalta, cu adevarat, în cetatea aceasta, ?i Irod ?i Pontius Pilat cu pagânii ?i cu popoarele lui Israel,
\par 28 Ca sa faca toate câte mâna Ta ?i sfatul Tau mai dinainte au rânduit sa fie.
\par 29 ?i acum, Doamne, cauta spre amenin?arile lor ?i da robilor Tai sa graiasca cuvântul Tau cu toata îndrazneala,
\par 30 Întinzând dreapta Ta spre vindecare ?i savâr?ind semne ?i minuni, prin numele Sfântului Tau Fiu Iisus.
\par 31 ?i pe când se rugau astfel, s-a cutremurat locul în care erau aduna?i, ?i s-au umplut to?i de Duhul Sfânt ?i graiau cu îndrazneala cuvântul lui Dumnezeu.
\par 32 Iar inima ?i sufletul mul?imii celor ce au crezut erau una ?i nici unul nu zicea ca este al sau ceva din averea sa, ci toate le erau de ob?te.
\par 33 ?i cu mare putere apostolii marturiseau despre învierea Domnului Iisus Hristos ?i mare har era peste ei to?i.
\par 34 ?i nimeni nu era între ei lipsit, fiindca to?i câ?i aveau ?arini sau case le vindeau ?i aduceau pre?ul celor vândute,
\par 35 ?i-l puneau la picioarele apostolilor. ?i se împar?ea fiecaruia dupa cum avea cineva trebuin?a.
\par 36 Iar Iosif, cel numit de apostoli Barnaba (care se tâlcuie?te fiul mângâierii), un levit, nascut în Cipru,
\par 37 Având ?arina ?i vânzând-o, a adus banii ?i i-a pus la picioarele apostolilor.

\chapter{5}

\par 1 Iar un om, anume Anania, cu Safira, femeia lui, ?i-a vândut ?arina.
\par 2 ?i a dosit din pre?, ?tiind ?i femeia lui, ?i aducând o parte, a pus-o la picioarele apostolilor.
\par 3 Iar Petru a zis: Anania, de ce a umplut satana inima ta, ca sa min?i tu Duhului Sfânt ?i sa dose?ti din pre?ul ?arinei?
\par 4 Oare, pastrând-o, nu-?i ramânea ?ie, ?i vânduta nu era în stapânirea ta? Pentru ce ai pus în inima ta lucrul acesta? N-ai min?it oamenilor, ci lui Dumnezeu.
\par 5 Iar Anania, auzind aceste cuvinte, a cazut ?i a murit. ?i frica mare i-a cuprins pe to?i care au auzit.
\par 6 ?i sculându-se cei mai tineri, l-au înfa?urat ?i, sco?ându-l afara, l-au îngropat.
\par 7 Dupa un rastimp, ca de trei ceasuri, a intrat ?i femeia lui, ne?tiind ce se întâmplase.
\par 8 Iar Petru a zis catre ea: Spune-mi daca a?i vândut ?arina cu atât? Iar ea a zis: Da, cu atât.
\par 9 Iar Petru a zis catre ea: De ce v-a?i învoit voi sa ispiti?i Duhul Domnului? Iata picioarele celor ce au îngropat pe barbatul tau sunt la u?a ?i te vor scoate afara ?i pe tine.
\par 10 ?i ea a cazut îndata la picioarele lui Petru ?i a murit. ?i intrând tinerii, au gasit-o moarta ?i, sco?ând-o afara, au îngropat-o lânga barbatul ei.
\par 11 ?i frica mare a cuprins toata Biserica ?i pe to?i care au auzit acestea.
\par 12 Iar prin mâinile apostolilor se faceau semne ?i minuni multe în popor, ?i erau to?i, într-un cuget, în pridvorul lui Solomon.
\par 13 ?i nimeni dintre ceilal?i nu cuteza sa se alipeasca de ei, dar poporul îi lauda.
\par 14 ?i din ce în ce mai mult se adaugau cei ce credeau în Domnul, mul?ime de barba?i ?i de femei,
\par 15 Încât scoteau pe cei bolnavi în uli?e ?i-i puneau pe paturi ?i pe targi, ca venind Petru, macar umbra lui sa umbreasca pe vreunul dintre ei.
\par 16 ?i se aduna ?i mul?imea din ceta?ile dimprejurul Ierusalimului, aducând bolnavi ?i bântui?i de duhuri necurate, ?i to?i se vindecau.
\par 17 ?i sculându-se arhiereul ?i to?i cei împreuna cu el - cei din eresul saducheilor - s-au umplut de pizma.
\par 18 ?i au pus mâna pe apostoli ?i i-au bagat în temni?a ob?teasca.
\par 19 Iar un înger al Domnului, în timpul nop?ii, a deschis u?ile temni?ei ?i, sco?ându-i, le-a zis:
\par 20 Merge?i ?i, stând, grai?i poporului în templu toate cuvintele vie?ii acesteia.
\par 21 ?i, auzind, au intrat de diminea?a în templu ?i înva?au. Dar venind arhiereul ?i cei împreuna cu el, au adunat sinedriul ?i tot sfatul batrânilor fiilor lui Israel ?i au trimis la temni?a sa-i aduca pe apostoli.
\par 22 Dar, ducându-se, slugile nu i-au gasit în temni?a ?i, întorcându-se, au vestit,
\par 23 Zicând: Temni?a am gasit-o încuiata în toata siguran?a ?i pe paznici stând la u?i, dar când am descuiat, înauntru n-am gasit pe nimeni.
\par 24 Când au auzit aceste cuvinte, capetenia pazei templului ?i arhiereii erau nedumeri?i cu privire la ei, ce-ar putea sa fie aceasta.
\par 25 Dar venind cineva, le-a dat de veste: Iata, barba?ii pe care i-a?i pus în temni?a sunt în templu, stând ?i înva?ând poporul.
\par 26 Atunci, ducându-se, capetenia pazei templului împreuna cu slujitorii i-au adus dar nu cu sila, ca se temeau de popor sa nu-i omoare cu pietre.
\par 27 ?i, aducându-i, i-au pus în fa?a sinedriului. Iar arhiereul i-a întrebat,
\par 28 Zicând: Au nu v-am poruncit voua cu porunca sa nu mai înva?a?i în numele acesta? ?i iata a?i umplut Ierusalimul cu înva?atura voastra ?i voi?i sa aduce?i asupra noastra sângele Acestui Om!
\par 29 Iar Petru ?i apostolii, raspunzând, au zis: Trebuie sa ascultam pe Dumnezeu mai mult decât pe oameni.
\par 30 Dumnezeul parin?ilor no?tri a înviat pe Iisus, pe Care voi L-a?i omorât, spânzurându-L pe lemn.
\par 31 Pe Acesta, Dumnezeu, prin dreapta Sa, L-a înal?at Stapânitor ?i Mântuitor, ca sa dea lui Israel pocain?a ?i iertarea pacatelor.
\par 32 ?i suntem martori ai acestor cuvinte noi ?i Duhul Sfânt, pe Care Dumnezeu L-a dat celor ce Îl asculta.
\par 33 Iar ei, auzind, se mâniau foarte ?i se sfatuiau sa-i omoare.
\par 34 ?i ridicându-se în sinedriu un fariseu, anume Gamaliel, înva?ator de Lege, cinstit de tot poporul, a poruncit sa-i scoata pe oameni afara pu?in,
\par 35 ?i a zis catre ei: Barba?i israeli?i, lua?i aminte la voi, ce ave?i sa face?i cu ace?ti oameni.
\par 36 Ca înainte de zilele acestea s-a ridicat Teudas, zicând ca el este cineva, caruia i s-au alaturat un numar de barba?i ca la patru sute, care a fost ucis ?i to?i câ?i l-au ascultat au fost risipi?i ?i nimici?i.
\par 37 Dupa aceasta s-a ridicat Iuda Galileianul, în vremea numaratorii, ?i a atras popor mult dupa el; ?i acela a pierit ?i to?i câ?i au ascultat de el au fost împra?tia?i.
\par 38 ?i acum zic voua: Feri?i-va de oamenii ace?tia ?i lasa?i-i, caci daca aceasta hotarâre sau lucrul acesta este de la oameni, se va nimici;
\par 39 Iar daca este de la Dumnezeu, nu ve?i putea sa-i nimici?i, ca nu cumva sa va afla?i ?i luptatori împotriva lui Dumnezeu.
\par 40 ?i l-au ascultat pe el; ?i chemând pe apostoli ?i batându-i, le-au poruncit sa nu mai vorbeasca în numele lui Iisus, ?i le-au dat drumul.
\par 41 Iar ei au plecat din fa?a sinedriului, bucurându-se ca s-au învrednicit, pentru numele Lui, sa sufere ocara.
\par 42 ?i toata ziua, în templu ?i prin case, nu încetau sa înve?e ?i sa binevesteasca pe Hristos Iisus.

\chapter{6}

\par 1 În zilele acelea, înmul?indu-se ucenicii, eleni?tii (iudei) murmurau împotriva evreilor, pentru ca vaduvele lor erau trecute cu vederea la slujirea cea de fiecare zi.
\par 2 ?i chemând cei doisprezece mul?imea ucenicilor, au zis: Nu este drept ca noi, lasând de-o parte cuvântul lui Dumnezeu, sa slujim la mese.
\par 3 Drept aceea, fra?ilor, cauta?i ?apte barba?i dintre voi, cu nume bun, plini de Duh Sfânt ?i de în?elepciune, pe care noi sa-i rânduim la aceasta slujba.
\par 4 Iar noi vom starui în rugaciune ?i în slujirea cuvântului.
\par 5 ?i a placut cuvântul înaintea întregii mul?imi, ?i au ales pe ?tefan, barbat plin de credin?a ?i de Duh Sfânt, ?i pe Filip, ?i pe Prohor, ?i pe Nicanor, ?i pe Timon, ?i pe Parmena, ?i pe Nicolae, prozelit din Antiohia,
\par 6 Pe care i-au pus înaintea apostolilor, ?i ei, rugându-se ?i-au pus mâinile peste ei.
\par 7 ?i cuvântul lui Dumnezeu cre?tea, ?i se înmul?ea foarte numarul ucenicilor în Ierusalim, înca ?i mul?ime de preo?i se supuneau credin?ei.
\par 8 Iar ?tefan, plin de har ?i de putere, facea minuni ?i semne mari în popor.
\par 9 ?i s-au ridicat unii din sinagoga ce se zicea a libertinilor ?i a cirenenilor ?i a alexandrinilor ?i a celor din Cilicia ?i din Asia, sfadindu-se cu ?tefan.
\par 10 ?i nu puteau sa stea împotriva în?elepciunii ?i a Duhului cu care el vorbea.
\par 11 Atunci au pus pe ni?te barba?i sa zica: L-am auzit spunând cuvinte de hula împotriva lui Moise ?i a lui Dumnezeu.
\par 12 ?i au întarâtat poporul ?i pe batrâni ?i pe carturari ?i, navalind asupra-i, l-au rapit ?i l-au dus în sinedriu.
\par 13 ?i au pus martori mincino?i, care ziceau: Acest om nu înceteaza a vorbi cuvinte de hula împotriva acestui loc sfânt ?i a Legii.
\par 14 Ca l-au auzit zicând ca Acest Iisus Nazarineanul va strica locul acesta ?i va schimba datinile pe care ni le-a lasat noua Moise.
\par 15 ?i a?intindu-?i ochii asupra lui, to?i cei ce ?edeau în sinedriu au vazut fa?a lui ca o fa?a de înger.

\chapter{7}

\par 1 ?i a zis arhiereul: Adevarate sunt acestea?
\par 2 Iar el a zis: Barba?i fra?i ?i parin?i, asculta?i! Dumnezeul slavei S-a aratat parintelui nostru Avraam, când era în Mesopotamia, mai înainte de a locui în Haran,
\par 3 ?i a zis catre el: Ie?i din pamântul tau ?i din rudenia ta ?i vino în pamântul pe care ?i-l voi arata.
\par 4 Atunci, ie?ind din pamântul caldeilor, a locuit în Haran. Iar de acolo, dupa moartea tatalui sau, l-a stramutat în aceasta ?ara, în care locui?i voi acum,
\par 5 ?i nu i-a dat mo?tenire în ea nici o palma de pamânt, ci i-a fagaduit ca i-o va da lui spre stapânire ?i urma?ilor lui dupa el, neavând el copil.
\par 6 ?i Dumnezeu a vorbit astfel: "Urma?ii lui vor fi straini în pamânt strain, ?i acolo îi vor robi ?i-i vor asupri patru sute de ani.
\par 7 ?i pe poporul la care vor fi robi, Eu îl voi judeca - a zis Dumnezeu - iar dupa acestea vor ie?i ?i-Mi vor sluji Mie în locul acesta".
\par 8 ?i i-a dat legamântul taierii împrejur; ?i a?a a nascut pe Isaac ?i l-a taiat împrejur a opta zi; ?i Isaac a nascut pe Iacov ?i Iacov pe cei doisprezece patriarhi.
\par 9 ?i patriarhii, pizmuind pe Iosif, l-au vândut în Egipt; dar Dumnezeu era cu el,
\par 10 ?i l-a scos din toate necazurile lui ?i i-a dat har ?i în?elepciune înaintea lui Faraon, regele Egiptului, iar acesta l-a pus mai mare peste Egipt ?i peste toata casa lui.
\par 11 ?i a venit foamete peste tot Egiptul ?i peste Canaan, ?i strâmtorare mare, ?i parin?ii no?tri nu mai gaseau hrana.
\par 12 ?i Iacov, auzind ca este grâu în Egipt, a trimis pe parin?ii no?tri întâia oara.
\par 13 Iar a doua oara Iosif s-a facut cunoscut fra?ilor sai ?i Faraon a aflat neamul lui Iosif.
\par 14 ?i Iosif, trimi?ând, a chemat pe Iacov, tatal sau, ?i toata rudenia sa, cu ?aptezeci ?i cinci de suflete.
\par 15 ?i Iacov s-a coborât în Egipt; ?i a murit ?i el ?i parin?ii no?tri;
\par 16 ?i au fost stramuta?i la Sichem ?i au fost pu?i în mormântul pe care Avraam l-a cumparat cu pre? de argint, de la fiii lui Emor, în Sichem;
\par 17 Dar cum se apropia vremea fagaduin?ei pentru care s-a jurat Dumnezeu lui Avraam, a crescut poporul ?i s-a înmul?it în Egipt,
\par 18 Pâna când s-a ridicat peste Egipt alt rege, care nu ?tia de Iosif.
\par 19 Acesta, purtându-se ca un viclean cu neamul nostru, a asuprit pe parin?ii no?tri sa-?i lepede pruncii lor, ca sa nu mai traiasca.
\par 20 În vremea aceea s-a nascut Moise ?i era placut lui Dumnezeu. ?i trei luni a fost hranit în casa tatalui sau.
\par 21 ?i fiind lepadat, l-a luat fiica lui Faraon ?i l-a crescut ca pe un fiu al ei.
\par 22 ?i a fost înva?at Moise în toata în?elepciunea egiptenilor ?i era puternic în cuvintele ?i în faptele lui.
\par 23 Iar când a împlinit patruzeci de ani, ?i-a pus în gând sa cerceteze pe fra?ii sai, fiii lui Israel.
\par 24 ?i vazând pe unul dintre ei ca sufera strâmbatate, l-a aparat ?i, omorând pe egiptean, a razbunat pe cel asuprit.
\par 25 ?i el credea ca fra?ii sai vor pricepe ca Dumnezeu, prin mâna lui, le daruie?te izbavire, dar ei n-au în?eles.
\par 26 ?i a doua zi s-a aratat unora care se bateau ?i i-a îndemnat la pace, zicând: Barba?ilor, sunte?i fra?i. De ce va face?i rau unul altuia?
\par 27 Dar cel ce asuprea pe aproapele l-a îmbrâncit, zicând: Cine te-a pus pe tine domn ?i judecator peste noi?
\par 28 Nu cumva vrei sa ma omori, cum ai omorât ieri pe egiptean?
\par 29 La acest cuvânt, Moise a fugit ?i a trait ca strain în ?ara Madian, unde a nascut doi fii.
\par 30 ?i dupa ce s-au împlinit patruzeci de ani, îngerul Domnului i s-a aratat în pustiul Muntelui Sinai, în flacara focului unui rug.
\par 31 Iar Moise, vazând, s-a minunat de vedenie, dar când s-a apropiat ca sa ia seama mai bine, a fost glasul Domnului catre el:
\par 32 "Eu sunt Dumnezeul parin?ilor tai, Dumnezeul lui Avraam ?i Dumnezeul lui Isaac ?i Dumnezeul lui Iacov". ?i Moise, tremurând, nu îndraznea sa priveasca;
\par 33 Iar Domnul i-a zis: "Dezleaga încal?amintea picioarelor tale, caci locul pe care stai este pamânt sfânt.
\par 34 Privind, am vazut asuprirea poporului Meu în Egipt ?i suspinul lor l-am auzit ?i M-am pogorât ca sa-i scot. ?i acum vino, sa te trimit în Egipt".
\par 35 Pe Moise acesta de care s-au lepadat, zicând: Cine te-a pus pe tine domn ?i judecator?, pe acesta l-a trimis Dumnezeu domn ?i rascumparator, prin mâna îngerului care i se aratase lui în rug.
\par 36 Acesta i-a scos afara, facând minuni ?i semne în ?ara Egiptului ?i în Marea Ro?ie ?i în pustie, timp de patruzeci de ani.
\par 37 Acesta este Moise cel ce a zis fiilor lui Israel: "Prooroc ca mine va va ridica Dumnezeu din fra?ii vo?tri; pe EL sa-L asculta?i".
\par 38 Acesta este cel ce a fost la adunarea în pustie, cu îngerul care i-a vorbit pe Muntele Sinai ?i cu parin?ii no?tri, primind cuvinte de via?a ca sa ni le dea noua.
\par 39 De acesta n-au voit sa asculte parin?ii no?tri, ci l-au lepadat ?i inimile lor s-au întors catre Egipt,
\par 40 Zicând lui Aaron: "Fa-ne dumnezei care sa mearga înaintea noastra; caci acestui Moise, care ne-a scos din ?ara Egiptului, nu ?tim ce i s-a întâmplat".
\par 41 ?i au facut, în zilele acelea, un vi?el ?i au adus idolului jertfa ?i se veseleau de lucrurile mâinilor lor.
\par 42 ?i S-a întors Dumnezeu ?i i-a dat pe ei sa slujeasca o?tirii cerului, precum este scris în cartea proorocilor: "Adus-a?i voi Mie, casa a lui Israel, timp de patruzeci de ani, în pustie, junghieri ?i jertfe?
\par 43 ?i a?i purtat cortul lui Moloh ?i steaua dumnezeului vostru Remfan, chipurile pe care le-a?i facut, ca sa va închina?i la ele! De aceea va voi stramuta dincolo de Babilon".
\par 44 Parin?ii no?tri aveau în pustie cortul marturiei, precum orânduise Cel ce a vorbit cu Moise, ca sa-l faca dupa chipul pe care îl vazuse;
\par 45 ?i pe acesta primindu-l, parin?ii no?tri l-au adus cu Iosua în ?ara stapânita de neamuri, pe care Dumnezeu le-a izgonit din fa?a parin?ilor no?tri, pâna în zilele lui David,
\par 46 Care a aflat har înaintea lui Dumnezeu ?i a cerut sa gaseasca un loca? pentru Dumnezeul lui Iacov.
\par 47 Iar Solomon I-a zidit Lui casa,
\par 48 Dar Cel Preaînalt nu locuie?te în temple facute de mâini, precum zice proorocul:
\par 49 "Cerul este tronul Meu ?i pamântul a?ternut picioarelor Mele. Ce casa Îmi ve?i zidi Mie? - zice Domnul - sau care este locul odihnei Mele?
\par 50 Nu mâna Mea a facut toate acestea?"
\par 51 Voi cei tari în cerbice ?i netaia?i împrejur la inima ?i la urechi, voi pururea sta?i împotriva Duhului Sfânt, precum ?i parin?ii vo?tri, a?a ?i voi!
\par 52 Pe care dintre prooroci nu l-au prigonit parin?ii vo?tri? ?i au ucis pe cei ce au vestit mai dinainte sosirea Celui Drept, ai Carui vânzatori ?i uciga?i v-a?i facut voi acum,
\par 53 Voi, care a?i primit Legea întru rânduieli de îngeri ?i n-a?i pazit-o!
\par 54 Iar ei, auzind acestea, frematau de furie în inimile lor ?i scrâ?neau din din?i împotriva lui.
\par 55 Iar ?tefan, fiind plin de Duh Sfânt ?i privind la cer, a vazut slava lui Dumnezeu ?i pe Iisus stând de-a dreapta lui Dumnezeu.
\par 56 ?i a zis: Iata, vad cerurile deschise ?i pe Fiul Omului stând de-a dreapta lui Dumnezeu!
\par 57 Iar ei, strigând cu glas mare, ?i-au astupat urechile ?i au navalit asupra lui.
\par 58 ?i sco?ându-l afara din cetate, îl bateau cu pietre. Iar martorii ?i-au pus hainele la picioarele unui tânar, numit Saul.
\par 59 ?i îl bateau cu pietre pe ?tefan, care se ruga ?i zicea: Doamne, Iisuse, prime?te duhul meu!
\par 60 ?i, îngenunchind, a strigat cu glas mare: Doamne, nu le socoti lor pacatul acesta! ?i zicând acestea, a murit.

\chapter{8}

\par 1 ?i Saul s-a facut parta? la uciderea lui. ?i s-a facut în ziua aceea prigoana mare împotriva Bisericii din Ierusalim. ?i to?i, afara de apostoli, s-au împra?tiat prin ?inuturile Iudeii ?i ale Samariei.
\par 2 Iar barba?i cucernici au îngropat pe ?tefan ?i au facut plângere mare pentru el.
\par 3 ?i Saul pustiia Biserica, intrând prin case ?i, târând pe barba?i ?i pe femei, îi preda la temni?a.
\par 4 Iar cei ce se împra?tiasera strabateau ?ara, binevestind cuvântul.
\par 5 Iar Filip, coborându-se într-o cetate a Samariei, le propovaduia pe Hristos.
\par 6 ?i mul?imile luau aminte într-un cuget la cele spuse de catre Filip, ascultându-l ?i vazând semnele pe care le savâr?ea.
\par 7 Caci din mul?i care aveau duhuri necurate, strigând cu glas mare, ele ie?eau ?i mul?i slabanogi ?i ?chiopi s-au vindecat.
\par 8 ?i s-a facut mare bucurie în cetatea aceea.
\par 9 Dar era mai dinainte în cetate un barbat, anume Simon, vrajind ?i uimind neamul Samariei, zicând ca el este cineva mare,
\par 10 La care luau aminte to?i, de la mic pâna la mare, zicând: Acesta este puterea lui Dumnezeu, numita cea mare.
\par 11 ?i luau aminte la el, fiindca de multa vreme, cu vrajile lui, îi uimise.
\par 12 Iar când au crezut lui Filip, care le propovaduia despre împara?ia lui Dumnezeu ?i despre numele lui Iisus Hristos, barba?i ?i femei se botezau.
\par 13 Iar Simon a crezut ?i el ?i, botezându-se, era mereu cu Filip. ?i vazând semnele ?i minunile mari ce se faceau, era uimit.
\par 14 Iar apostolii din Ierusalim, auzind ca Samaria a primit cuvântul lui Dumnezeu, au trimis la ei pe Petru ?i pe Ioan,
\par 15 Care, coborând, s-au rugat pentru ei, ca sa primeasca Duhul Sfânt.
\par 16 Caci nu Se pogorâse înca peste nici unul dintre ei, ci erau numai boteza?i în numele Domnului Iisus.
\par 17 Atunci î?i puneau mâinile peste ei, ?i ei luau Duhul Sfânt.
\par 18 ?i Simon vazând ca prin punerea mâinilor apostolilor se da Duhul Sfânt, le-a adus bani,
\par 19 Zicând: Da?i-mi ?i mie puterea aceasta, ca acela pe care voi pune mâinile sa primeasca Duhul Sfânt.
\par 20 Iar Petru a zis catre el: Banii tai sa fie cu tine spre pierzare! Caci ai socotit ca darul lui Dumnezeu se agonise?te cu bani.
\par 21 Tu n-ai parte, nici mo?tenire, la chemarea aceasta, pentru ca inima ta nu este dreapta înaintea lui Dumnezeu.
\par 22 Pocaie?te-te deci de aceasta rautate a ta ?i te roaga lui Dumnezeu, doara ?i se va ierta cugetul inimii tale,
\par 23 Caci întru amaraciunea fierii ?i întru legatura nedrepta?ii te vad ca e?ti.
\par 24 ?i raspunzând, Simon a zis: Ruga?i-va voi la Domnul, pentru mine, ca sa nu vina asupra mea nimic din cele ce a?i zis.
\par 25 Iar ei, marturisind ?i graind cuvântul Domnului, s-au întors la Ierusalim ?i în multe sate ale samarinenilor binevesteau.
\par 26 ?i un înger al Domnului a grait catre Filip, zicând: Ridica-te ?i mergi spre miazazi, pe calea care coboara de la Ierusalim la Gaza; aceasta este pustie.
\par 27 Ridicându-se, a mers. ?i iata un barbat din Etiopia, famen, mare dregator al Candachiei, regina Etiopiei, care era peste toata vistieria ei ?i care venise la Ierusalim ca sa se închine,
\par 28 Se întorcea acasa; ?i ?ezând în carul sau, citea pe proorocul Isaia.
\par 29 Iar Duhul i-a zis lui Filip: Apropie-te ?i te alipe?te de carul acesta.
\par 30 ?i alergând Filip l-a auzit citind pe proorocul Isaia, ?i i-a zis: În?elegi, oare, ce cite?ti?
\par 31 Iar el a zis: Cum a? putea sa în?eleg, daca nu ma va calauzi cineva? ?i a rugat pe Filip sa se urce ?i sa ?ada cu el.
\par 32 Iar locul din Scriptura pe care-l citea era acesta: "Ca un miel care se aduce spre junghiere ?i ca o oaie fara de glas înaintea celui ce-o tunde, a?a nu ?i-a deschis gura sa.
\par 33 Întru smerenia Lui, judecata Lui s-a ridicat ?i neamul Lui cine-l va spune? Ca se ridica de pe pamânt via?a Lui".
\par 34 Iar famenul, raspunzând, a zis lui Filip: Rogu-te, despre cine zice proorocul acesta, despre sine ori despre altcineva?
\par 35 Iar Filip, deschizând gura sa ?i începând de la scriptura aceasta, i-a binevestit pe Iisus.
\par 36 ?i, pe când mergeau pe cale, au ajuns la o apa; iar famenul a zis: Iata apa. Ce ma împiedica sa fiu botezat?
\par 37 Filip a zis: Daca crezi din toata inima, este cu putin?a. ?i el, raspunzând, a zis: Cred ca Iisus Hristos este Fiul lui Dumnezeu.
\par 38 ?i a poruncit sa stea carul; ?i s-au coborât amândoi în apa, ?i Filip ?i famenul, ?i l-a botezat.
\par 39 Iar când au ie?it din apa, Duhul Domnului a rapit pe Filip, ?i famenul nu l-a mai vazut. ?i el s-a dus în calea sa, bucurându-se.
\par 40 Iar Filip s-a aflat în Azot ?i, mergând, binevestea prin toate ceta?ile, pâna ce a sosit în Cezareea.

\chapter{9}

\par 1 Iar Saul, suflând înca amenin?are ?i ucidere împotriva ucenicilor Domnului, a mers la arhiereu,
\par 2 ?i a cerut de la el scrisori catre sinagogile din Damasc ca, daca va afla acolo pe vreunii, atât barba?i, cât ?i femei, ca merg pe calea aceasta, sa-i aduca lega?i la Ierusalim.
\par 3 Dar pe când calatorea el ?i se apropia de Damasc, o lumina din cer, ca de fulger, l-a învaluit deodata.
\par 4 ?i, cazând la pamânt, a auzit un glas, zicându-i: Saule, Saule, de ce Ma prigone?ti?
\par 5 Iar el a zis: Cine e?ti, Doamne? ?i Domnul a zis: Eu sunt Iisus, pe Care tu Îl prigone?ti. Greu î?i este sa izbe?ti cu piciorul în ?epu?a.
\par 6 ?i el, tremurând ?i înspaimântat fiind, a zis: Doamne, ce voie?ti sa fac? Iar Domnul i-a zis: Ridica-te, intra în cetate ?i ?i se va spune ce trebuie sa faci.
\par 7 Iar barba?ii, care erau cu el pe cale, stateau înmarmuri?i, auzind glasul, dar nevazând pe nimeni.
\par 8 ?i s-a ridicat Saul de la pamânt, dar, de?i avea ochii deschi?i, nu vedea nimic. ?i luându-l de mâna, l-au dus în Damasc.
\par 9 ?i trei zile a fost fara vedere; ?i n-a mâncat, nici n-a baut.
\par 10 ?i era în Damasc un ucenic, anume Anania, ?i Domnul i-a zis în vedenie: Anania! Iar el a zis: Iata-ma, Doamne;
\par 11 ?i Domnul a zis catre el: Sculându-te, mergi pe uli?a care se cheama Uli?a Dreapta ?i cauta în casa lui Iuda, pe un om din Tars, cu numele Saul; ca, iata, se roaga.
\par 12 ?i a vazut în vedenie pe un barbat, anume Anania, intrând la el ?i punându-?i mâinile peste el, ca sa vada iara?i.
\par 13 ?i a raspuns Anania: Doamne, despre barbatul acesta am auzit de la mul?i câte rele a facut sfin?ilor Tai în Ierusalim.
\par 14 ?i aici are putere de la arhierei sa lege pe to?i care cheama numele Tau.
\par 15 ?i a zis Domnul catre el: Mergi, fiindca acesta Îmi este vas ales, ca sa poarte numele Meu înaintea neamurilor ?i a regilor ?i a fiilor lui Israel;
\par 16 Caci Eu îi voi arata câte trebuie sa patimeasca el pentru numele Meu.
\par 17 ?i a mers Anania ?i a intrat în casa ?i, punându-?i mâinile pe el, a zis: Frate Saul, Domnul Iisus, Cel ce ?i S-a aratat pe calea pe care tu veneai, m-a trimis ca sa vezi iara?i ?i sa te umpli de Duh Sfânt.
\par 18 ?i îndata au cazut de pe ochii lui ca ni?te solzi; ?i a vazut iara?i ?i, sculându-se, a fost botezat.
\par 19 ?i luând mâncare, s-a întarit. ?i a stat cu ucenicii din Damasc câteva zile.
\par 20 Apoi propovaduia în sinagogi pe Iisus, ca Acesta este Fiul lui Dumnezeu.
\par 21 ?i se mirau to?i care îl auzeau ?i ziceau: Nu este, oare, acesta cel care prigonea în Ierusalim pe cei ce cheama acest nume ?i a venit aici pentru aceea ca sa-i duca pe ei lega?i la arhierei?
\par 22 ?i Saul se întarea mai mult ?i tulbura pe iudeii care locuiau în Damasc, dovedind ca Acesta este Hristos.
\par 23 ?i dupa ce au trecut destule zile, iudeii s-au sfatuit sa-l omoare.
\par 24 ?i s-a facut cunoscut lui Saul vicle?ugul lor. ?i ei pazeau por?ile ?i ziua ?i noaptea, ca sa-l ucida.
\par 25 ?i luându-l ucenicii lui noaptea, l-au coborât peste zid, lasându-l jos într-un co?.
\par 26 ?i venind la Ierusalim, Saul încerca sa se alipeasca de ucenici; ?i to?i se temeau de el, necrezând ca este ucenic.
\par 27 Iar Barnaba, luându-l pe el, l-a dus la apostoli ?i le-a istorisit cum a vazut pe cale pe Domnul ?i ca El i-a vorbit lui ?i cum a propovaduit la Damasc, cu îndrazneala în numele lui Iisus.
\par 28 ?i era cu ei intrând ?i ie?ind în Ierusalim ?i propovaduia cu îndrazneala în numele Domnului.
\par 29 ?i vorbea ?i se sfadea cu eleni?tii, iar ei cautau sa-l ucida.
\par 30 Dar fra?ii, aflând aceasta, l-au dus pe Saul la Cezareea ?i de acolo l-au trimis la Tars.
\par 31 Deci Biserica, în toata Iudeea ?i Galileea ?i Samaria, avea pace, zidindu-se ?i umblând în frica de Domnul, ?i sporea prin mângâierea Duhului Sfânt.
\par 32 ?i trecând Petru pe la to?i, a coborât ?i la sfin?ii care locuiau în Lida.
\par 33 ?i acolo a gasit pe un om, anume Enea, care de opt ani zacea în pat, fiindca era paralitic.
\par 34 ?i Petru i-a zis: Enea, te vindeca Iisus Hristos. Ridica-te ?i strânge-?i patul. ?i îndata s-a ridicat.
\par 35 ?i l-au vazut to?i cei ce locuiau în Lida ?i în Saron, care s-au ?i întors la Domnul.
\par 36 Iar în Iope era o uceni?a, cu numele Tavita, care, tâlcuindu-se, se zice Caprioara. Aceasta era plina de fapte bune ?i de milosteniile pe care le facea.
\par 37 ?i în zilele acelea ea s-a îmbolnavit ?i a murit. ?i, scaldând-o, au pus-o în camera de sus.
\par 38 ?i fiind aproape Lida de Iope, ucenicii, auzind ca Petru este în Lida, au trimis pe doi barba?i la el, rugându-l: Nu pregeta sa vii pâna la noi.
\par 39 ?i Petru, sculându-se, a venit cu ei. Când a sosit, l-au dus în camera de sus ?i l-au înconjurat toate vaduvele, plângând ?i aratând cama?ile ?i hainele câte le facea Caprioara, pe când era cu ele.
\par 40 ?i Petru, sco?ând afara pe to?i, a îngenunchiat ?i s-a rugat ?i, întorcându-se catre trup, a zis: Tavita scoala-te! Iar ea ?i-a deschis ochii ?i, vazând pe Petru, a ?ezut.
\par 41 ?i dându-i mâna, Petru a ridicat-o ?i, chemând pe sfin?i ?i pe vaduve, le-a dat-o vie.
\par 42 ?i s-a facut cunoscuta aceasta în întreaga Iope ?i mul?i au crezut în Domnul.
\par 43 ?i el a ramas în Iope multe zile, la un oarecare Simon, tabacar.

\chapter{10}

\par 1 Iar în Cezareea era un barbat cu numele Corneliu, suta?, din cohorta ce se chema Italica,
\par 2 Cucernic ?i temator de Dumnezeu, cu toata casa lui ?i care facea multe milostenii poporului ?i se ruga lui Dumnezeu totdeauna.
\par 3 ?i a vazut în vedenie, lamurit, cam pe la ceasul al noualea din zi, un înger al lui Dumnezeu, intrând la el ?i zicându-i: Corneliu!
\par 4 Iar Corneliu, cautând spre el ?i înfrico?ându-se, a zis: Ce este, Doamne? ?i îngerul i-a zis: Rugaciunile tale ?i milosteniile tale s-au suit, spre pomenire, înaintea lui Dumnezeu.
\par 5 ?i acum, trimite barba?i la Iope ?i cheama sa vina un oarecare Simon, care se nume?te ?i Petru.
\par 6 El este gazduit la un om oarecare Simon, tabacar, a carui casa este lânga mare. Acesta î?i va spune ce sa faci.
\par 7 ?i dupa ce s-a dus îngerul care vorbea cu el, Corneliu a chemat doua din slugile sale de casa ?i pe un osta? cucernic din cei ce îi erau mai apropia?i.
\par 8 ?i dupa ce le-a istorisit toate, i-a trimis la Iope.
\par 9 Iar a doua zi, pe când ei mergeau pe drum ?i se apropiau de cetate, Petru s-a suit pe acoperi?, sa se roage pe la ceasul al ?aselea.
\par 10 ?i i s-a facut foame ?i voia sa manânce, dar, pe când ei îi pregateau sa manânce, a cazut în extaz.
\par 11 ?i a vazut cerul deschis ?i coborându-se ceva ca o fa?a mare de pânza, legata în patru col?uri, lasându-se pe pamânt.
\par 12 În ea erau toate dobitoacele cu patru picioare ?i târâtoarele pamântului ?i pasarile cerului.
\par 13 ?i glas a fost catre el: Sculându-te, Petre, junghie ?i manânca.
\par 14 Iar Petru a zis: Nicidecum, Doamne, caci niciodata n-am mâncat nimic spurcat ?i necurat.
\par 15 ?i iara?i, a doua oara, a fost glas catre el: Cele ce Dumnezeu a cura?it, tu sa nu le nume?ti spurcate.
\par 16 ?i aceasta s-a facut de trei ori ?i îndata acel ceva s-a ridicat la cer.
\par 17 Pe când Petru nu se dumirea întru sine ce ar putea sa fie vedenia pe care o vazuse, iata barba?ii cei trimi?i de Corneliu, întrebând de casa lui Simon, s-au oprit la poarta,
\par 18 ?i, strigând, întrebau daca Simon, numit Petru, este gazduit acolo.
\par 19 ?i tot gândindu-se Petru la vedenie, Duhul i-a zis: Iata, trei barba?i te cauta;
\par 20 Ci sculându-te, coboara-te ?i mergi împreuna cu ei, de nimic îndoindu-te, fiindca Eu i-am trimis.
\par 21 ?i Petru, coborându-se la barba?ii trimi?i la el de Corneliu, le-a zis: Iata, eu sunt acela pe care îl cauta?i. Care este pricina pentru care a?i venit?
\par 22 Iar ei au zis: Corneliu suta?ul, om drept ?i temator de Dumnezeu ?i marturisit de tot neamul iudeilor, a fost în?tiin?at de catre un sfânt înger sa trimita sa te cheme acasa la el, ca sa auda cuvinte de la tine.
\par 23 Deci, chemându-i înauntru, i-a gazduit. Iar a doua zi, sculându-se, a plecat împreuna cu ei; iar câ?iva din fra?ii cei din Iope l-au înso?it.
\par 24 ?i în ziua urmatoare au intrat în Cezareea. Iar Corneliu îi a?tepta ?i chemase acasa la el rudeniile sale ?i prietenii cei mai de aproape.
\par 25 ?i când a fost sa intre Petru, Corneliu, întâmpinându-l, i s-a închinat, cazând la picioarele lui.
\par 26 Iar Petru l-a ridicat, zicându-i: Scoala-te. ?i eu sunt om.
\par 27 ?i, vorbind cu el, a intrat ?i a gasit pe mul?i aduna?i.
\par 28 ?i a zis catre ei: Voi ?ti?i ca nu se cuvine unui barbat iudeu sa se uneasca sau sa se apropie de cel de alt neam, dar mie Dumnezeu mi-a aratat sa nu numesc pe nici un om spurcat sau necurat.
\par 29 De aceea, chemat fiind sa vin, am venit fara împotrivire. Deci va întreb: Pentru care cuvânt a?i trimis dupa mine?
\par 30 Corneliu a zis: Acum patru zile eram postind pâna la ceasul acesta ?i ma rugam în casa mea, în ceasul al noualea, ?i iata un barbat în haina stralucitoare a stat în fa?a mea.
\par 31 ?i el a zis: Corneliu, a fost ascultata rugaciunea ta ?i milosteniile tale au fost pomenite înaintea lui Dumnezeu.
\par 32 Trimite, deci, la Iope ?i cheama pe Simon, cel ce se nume?te Petru; el este gazduit în casa lui Simon, tabacarul, lânga mare.
\par 33 Deci îndata am trimis la tine; ?i tu ai facut bine ca ai venit. ?i acum noi to?i suntem de fa?a înaintea lui Dumnezeu, ca sa ascultam toate cele poruncite ?ie de Domnul.
\par 34 Iar Petru, deschizându-?i gura, a zis: Cu adevarat în?eleg ca Dumnezeu nu este partinitor.
\par 35 Ci, în orice neam, cel ce se teme de El ?i face dreptate este primit de El.
\par 36 ?i El a trimis fiilor lui Israel cuvântul, binevestind pacea prin Iisus Hristos: Acesta este Domn peste toate.
\par 37 Voi ?ti?i cuvântul care a fost în toata Iudeea, începând din Galileea, dupa botezul pe care l-a propovaduit Ioan.
\par 38 (Adica despre) Iisus din Nazaret, cum a uns Dumnezeu cu Duhul Sfânt ?i cu putere pe Acesta care a umblat facând bine ?i vindecând pe to?i cei asupri?i de diavolul, pentru ca Dumnezeu era cu El.
\par 39 ?i noi suntem martori pentru toate cele ce a facut El în ?ara iudeilor ?i în Ierusalim; pe Acesta L-au omorât, spânzurându-L pe lemn.
\par 40 Dar Dumnezeu L-a înviat a treia zi ?i I-a dat sa Se arate,
\par 41 Nu la tot poporul, ci noua martorilor, dinainte rândui?i de Dumnezeu, care am mâncat ?i am baut cu El, dupa învierea Lui din mor?i.
\par 42 ?i ne-a poruncit sa propovaduim poporului ?i sa marturisim ca El este Cel rânduit de Dumnezeu sa fie judecator al celor vii ?i al celor mor?i.
\par 43 Despre Acesta marturisesc to?i proorocii, ca tot cel ce crede în El va primi iertarea pacatelor, prin numele Lui.
\par 44 Înca pe când Petru vorbea aceste cuvinte, Duhul Sfânt a cazut peste to?i cei care ascultau cuvântul.
\par 45 Iar credincio?ii taia?i împrejur, care venisera cu Petru, au ramas uimi?i pentru ca darul Duhului Sfânt s-a revarsat ?i peste neamuri.
\par 46 Caci îi auzeau pe ei vorbind în limbi ?i slavind pe Dumnezeu. Atunci a raspuns Petru:
\par 47 Poate, oare, cineva sa opreasca apa, ca sa nu fie boteza?i ace?tia care au primit Duhul Sfânt ca ?i noi?
\par 48 ?i a poruncit ca ace?tia sa fie boteza?i în numele lui Iisus Hristos. Atunci l-au rugat pe Petru sa ramâna la ei câteva zile.

\chapter{11}

\par 1 Apostolii ?i fra?ii, care erau prin Iudeea, au auzit ca ?i pagânii au primit cuvântul lui Dumnezeu.
\par 2 ?i când Petru s-a suit în Ierusalim, credincio?ii taia?i împrejur se împotriveau,
\par 3 Zicându-i: Ai intrat la oameni netaia?i împrejur ?i ai mâncat cu ei.
\par 4 ?i începând, Petru le-a înfa?i?at pe rând, zicând:
\par 5 Eu eram în cetatea Iope ?i ma rugam; ?i am vazut, în extaz, o vedenie: ceva coborându-se ca o fa?a mare de pânza, legata în patru col?uri, lasându-se în jos din cer, ?i a venit pâna la mine.
\par 6 Privind spre aceasta, cu luare aminte, am vazut dobitoacele cele cu patru picioare ale pamântului ?i fiarele ?i târâtoarele ?i pasarile cerului.
\par 7 ?i am auzit un glas, care-mi zicea: Sculându-te, Petre, junghie ?i manânca.
\par 8 ?i am zis: Nicidecum, Doamne, caci nimic spurcat sau necurat n-a intrat vreodata în gura mea.
\par 9 ?i glasul mi-a grait a doua oara din cer: Cele ce Dumnezeu a cura?it, tu sa nu le nume?ti spurcate.
\par 10 ?i aceasta s-a facut de trei ori ?i au fost luate iara?i toate în cer.
\par 11 ?i iata, îndata, trei barba?i, trimi?i de la Cezareea catre mine, s-au oprit la casa în care eram.
\par 12 Iar Duhul mi-a zis sa merg cu ei, de nimic îndoindu-ma. ?i au mers cu mine ?i ace?ti ?ase fra?i ?i am intrat în casa barbatului;
\par 13 ?i el ne-a povestit cum a vazut îngerul stând în casa lui ?i zicând: Trimite la Iope ?i cheama pe Simon, cel numit ?i Petru.
\par 14 Care va grai catre tine cuvinte, prin care te vei mântui tu ?i toata casa ta.
\par 15 ?i când am început eu sa vorbesc, Duhul Sfânt a cazut peste ei, ca ?i peste noi la început.
\par 16 ?i mi-am adus aminte de cuvântul Domnului, când zicea: Ioan a botezat cu apa; voi însa va ve?i boteza cu Duh Sfânt.
\par 17 Deci, daca Dumnezeu a dat lor acela?i dar ca ?i noua, acelora care au crezut în Domnul Iisus Hristos, cine eram eu ca sa-L pot opri pe Dumnezeu?
\par 18 Auzind acestea, au tacut ?i au slavit pe Dumnezeu, zicând: A?adar ?i pagânilor le-a dat Dumnezeu pocain?a spre via?a;
\par 19 Deci cei ce se risipisera din cauza tulburarii facute pentru ?tefan, au trecut pâna în Fenicia ?i în Cipru, ?i în Antiohia, nimanui graind cuvântul, decât numai iudeilor.
\par 20 ?i erau unii dintre ei, barba?i ciprieni ?i cireneni care, venind în Antiohia, vorbeau ?i catre elini, binevestind pe Domnul Iisus.
\par 21 ?i mâna Domnului era cu ei ?i era mare numarul celor care au crezut ?i s-au întors la Domnul.
\par 22 ?i vorba despre ei a ajuns la urechile Bisericii din Ierusalim, ?i au trimis pe Barnaba pâna la Antiohia.
\par 23 Acesta, sosind ?i vazând harul lui Dumnezeu, s-a bucurat ?i îndemna pe to?i sa ramâna în Domnul, cu inima statornica.
\par 24 Caci era barbat bun ?i plin de Duh Sfânt ?i de credin?a. ?i s-a adaugat Domnului mul?ime multa.
\par 25 ?i a plecat Barnaba la Tars, ca sa caute pe Saul
\par 26 ?i aflându-l, l-a adus la Antiohia. ?i au stat acolo un an întreg, adunându-se în biserica ?i înva?ând mult popor. ?i în Antiohia, întâia oara, ucenicii s-au numit cre?tini.
\par 27 În acele zile s-au coborât, de la Ierusalim în Antiohia, prooroci.
\par 28 ?i sculându-se unul dintre ei, cu numele Agav, a aratat prin Duhul, ca va fi în toata lumea foamete mare, care a ?i fost în zilele lui Claudiu.
\par 29 Iar ucenicii au hotarât ca fiecare dintre ei, dupa putere, sa trimita spre ajutorare fra?ilor care locuiau în Iudeea;
\par 30 Ceea ce au ?i facut, trimi?ând preo?ilor prin mâna lui Barnaba ?i a lui Saul.

\chapter{12}

\par 1 ?i în vremea aceea, regele Irod (Agripa) a pus mâna pe unii din Biserica, ca sa-i piarda.
\par 2 ?i a ucis cu sabia pe Iacov, fratele lui Ioan.
\par 3 ?i vazând ca este pe placul iudeilor, a mai luat ?i pe Petru, (?i erau zilele Azimelor)
\par 4 Pe care ?i prinzându-l l-a bagat în temni?a, dându-l la patru straji de câte patru osta?i, ca sa-l pazeasca, vrând sa-l scoata la popor dupa Pa?ti.
\par 5 Deci Petru era pazit în temni?a ?i se facea necontenit rugaciune catre Dumnezeu pentru el, de catre Biserica.
\par 6 Dar când Irod era sa-l scoata afara, în noaptea aceea, Petru dormea între doi osta?i, legat cu doua lan?uri, iar înaintea u?ii paznicii pazeau temni?a.
\par 7 ?i iata un înger al Domnului a venit deodata, iar în camera a stralucit lumina. ?i lovind pe Petru în coasta, îngerul l-a de?teptat, zicând: Scoala-te degraba! ?i lan?urile i-au cazut de la mâini.
\par 8 ?i a zis îngerul catre el: Încinge-te ?i încal?a-te cu sandalele. ?i el a facut a?a. ?i i-a zis lui: Pune haina pe tine ?i vino dupa mine.
\par 9 ?i, ie?ind, mergea dupa înger, dar nu ?tia ca ceea ce s-a facut prin înger este adevarat, ci i se parea ca vede vedenie.
\par 10 ?i trecând de straja întâi ?i de a doua, au ajuns la poarta cea de fier care duce în cetate, ?i poarta s-a deschis singura. ?i ie?ind, au trecut o uli?a ?i îndata îngerul s-a departat de la el.
\par 11 ?i Petru, venindu-?i în sine, a zis: Acum ?tiu cu adevarat ca Domnul a trimis pe îngerul Sau ?i m-a scos din mâna lui Irod ?i din toate câte a?tepta poporul iudeilor.
\par 12 ?i chibzuind, a venit la casa Mariei, mama lui Ioan, cel numit Marcu, unde erau aduna?i mul?i ?i se rugau.
\par 13 ?i batând Petru la u?a de la poarta, o slujnica cu numele Rodi, s-a dus sa asculte.
\par 14 ?i recunoscând glasul lui Petru, de bucurie nu a deschis u?a, ci, alergând înauntru, a spus ca Petru sta înaintea por?ii.
\par 15 Iar ei au zis catre ea: Ai înnebunit. Dar ea staruia ca este a?a. Iar ei ziceau: Este îngerul lui.
\par 16 Dar Petru batea mereu în poarta. ?i deschizându-i, l-au vazut ?i au ramas uimi?i.
\par 17 ?i facându-le semn cu mâna sa taca, le-a istorisit cum l-a scos Domnul pe el din temni?a. ?i a zis: Vesti?i acestea lui Iacov ?i fra?ilor. ?i ie?ind, s-a dus în alt loc.
\par 18 ?i facându-se ziua, mare a fost tulburarea între osta?i: Ce s-a facut, oare, cu Petru?
\par 19 Iar Irod cerându-l ?i negasindu-l, dupa ce au fost cerceta?i paznicii, a poruncit sa fie uci?i. ?i el, coborând din Iudeea la Cezareea, a ramas acolo.
\par 20 ?i Irod era mânios pe locuitorii din Tir ?i din Sidon. Dar ei, în?elegându-se între ei, au venit la el ?i câ?tigând pe Vlast, care era camara?ul regelui, cereau pace, pentru ca ?ara lor se hranea din cea a regelui.
\par 21 ?i într-o zi rânduita, Irod, îmbracându-se în ve?minte rege?ti ?i ?ezând la tribuna, vorbea catre ei;
\par 22 Iar poporul striga: Acesta e glas dumnezeiesc, nu omenesc!
\par 23 ?i îndata îngerul Domnului l-a lovit, pentru ca nu a dat slava lui Dumnezeu. ?i mâncându-l viermii, a murit.
\par 24 Iar cuvântul lui Dumnezeu cre?tea ?i se înmul?ea.
\par 25 Iar Barnaba ?i Saul, dupa ce au îndeplinit slujba lor, s-au întors de la Ierusalim la Antiohia, luând cu ei pe Ioan, cel numit Marcu.

\chapter{13}

\par 1 ?i erau în Biserica din Antiohia prooroci ?i înva?atori: Barnaba ?i Simeon, ce se numea Niger, Luciu Cirineul, Manain, cel ce fusese crescut împreuna cu Irod tetrarhul, ?i Saul.
\par 2 ?i pe când slujeau Domnului ?i posteau, Duhul Sfânt a zis: Osebi?i-mi pe Barnaba ?i pe Saul, pentru lucrul la care i-am chemat.
\par 3 Atunci, postind ?i rugându-se, ?i-au pus mâinile peste ei ?i i-au lasat sa plece.
\par 4 Deci, ei, mâna?i de Duhul Sfânt, au coborât la Seleucia ?i de acolo au plecat cu corabia la Cipru.
\par 5 ?i ajungând în Salamina, au vestit cuvântul lui Dumnezeu în sinagogile iudeilor. ?i aveau ?i pe Ioan slujitor.
\par 6 ?i strabatând toata insula pâna la Pafos, au gasit pe un oarecare barbat iudeu, vrajitor, prooroc mincinos, al carui nume era Bariisus,
\par 7 Care era în preajma proconsulului Sergius Paulus, barbat în?elept. Acesta chemând la sine pe Barnaba ?i pe Saul, dorea sa auda cuvântul lui Dumnezeu,
\par 8 Dar le statea împotriva Elimas vrajitorul - caci a?a se tâlcuie?te numele lui - cautând sa întoarca pe proconsul de la credin?a.
\par 9 Iar Saul - care se nume?te ?i Pavel - plin fiind de Duh Sfânt, a privit ?inta la el,
\par 10 ?i a zis: O, tu cel plin de toata viclenia ?i de toata în?elaciunea, fiule al diavolului, vrajma?ule a toata dreptatea, nu vei înceta de a strâmba caile Domnului cele drepte?
\par 11 ?i acum, iata mâna Domnului este asupra ta ?i vei fi orb, nevazând soarele pâna la o vreme. ?i îndata a cazut peste el pâcla ?i întuneric ?i, dibuind împrejur, cauta cine sa-l duca de mâna.
\par 12 Atunci proconsulul, vazând ce s-a facut, a crezut, mirându-se foarte de înva?atura Domnului.
\par 13 ?i plecând cu corabia de la Pafos, Pavel ?i cei împreuna cu el au venit la Perga Pamfiliei. Iar Ioan, despar?indu-se de ei, s-a întors la Ierusalim.
\par 14 Iar ei, trecând de la Perga, au ajuns la Antiohia Pisidiei ?i, intrând în sinagoga, într-o zi de sâmbata, au ?ezut.
\par 15 ?i dupa citirea Legii ?i a Proorocilor, mai-marii sinagogii au trimis la ei, zicându-le: Barba?i fra?i, daca ave?i vreun cuvânt de mângâiere catre popor, vorbi?i.
\par 16 ?i, ridicându-se Pavel ?i facându-le semn cu mâna, a zis: Barba?i israeli?i ?i cei tematori de Dumnezeu, asculta?i:
\par 17 Dumnezeul acestui popor al lui Israel a ales pe parin?ii no?tri ?i pe popor l-a înal?at, când era strain în pamântul Egiptului, ?i cu bra? înalt i-a scos de acolo,
\par 18 ?i vreme de patruzeci de ani i-a hranit în pustie.
\par 19 ?i nimicind ?apte neamuri în ?ara Canaanului, pamântul acela l-a dat lor spre mo?tenire.
\par 20 ?i dupa acestea, ca la patru sute cincizeci de ani, le-a dat judecatori, pâna la Samuel proorocul.
\par 21 ?i de acolo au cerut rege ?i Dumnezeu le-a dat, timp de patruzeci de ani, pe Saul, fiul lui Chi?, barbat din semin?ia lui Veniamin.
\par 22 ?i înlaturându-l, le-a ridicat rege pe David, pentru care a zis, marturisind: "Aflat-am pe David al lui Iesei, barbat dupa inima Mea, care va face toate voile Mele".
\par 23 Din urma?ii acestuia, Dumnezeu, dupa fagaduin?a, i-a adus lui Israel un Mântuitor, pe Iisus,
\par 24 Dupa ce Ioan a propovaduit, înaintea venirii Lui, botezul pocain?ei, la tot poporul lui Israel.
\par 25 Iar daca ?i-a împlinit Ioan calea sa, zicea: Nu sunt eu ce socoti?i voi ca sunt. Dar, iata, vine dupa mine Cel caruia nu sunt vrednic sa-I dezleg încal?amintea picioarelor.
\par 26 Barba?i fra?i, fii din neamul lui Avraam ?i cei dintre voi tematori de Dumnezeu, voua vi s-a trimis cuvântul acestei mântuiri.
\par 27 Caci cei ce locuiesc în Ierusalim ?i capeteniile lor, necunoscându-L ?i osândindu-L, au împlinit glasurile proorocilor care se citesc în fiecare sâmbata.
\par 28 ?i, neaflând în El nici o vina de moarte, au cerut de la Pilat ca sa-L omoare.
\par 29 Iar când au savâr?it toate cele scrise despre El, coborându-L de pe cruce, L-au pus în mormânt.
\par 30 Dar Dumnezeu L-a înviat din mor?i.
\par 31 El S-a aratat mai multe zile celor ce împreuna cu El s-au suit din Galileea la Ierusalim ?i care sunt acum martorii Lui catre popor.
\par 32 ?i noi va binevestim fagaduin?a facuta parin?ilor.
\par 33 Ca pe aceasta Dumnezeu a împlinit-o cu noi, copiii lor, înviindu-L pe Iisus, precum este scris ?i în Psalmul al doilea: "Fiul Meu e?ti Tu; Eu astazi Te-am nascut".
\par 34 ?i cum ca L-a înviat din mor?i, ca sa nu se mai întoarca la stricaciune, a spus-o altfel: "Va voi da voua cele sfinte ?i vrednice de credin?a ale lui David".
\par 35 De aceea ?i în alt loc zice: "Nu vei da pe Sfântul Tau sa vada stricaciune".
\par 36 Pentru ca David, slujind în timpul sau voii lui Dumnezeu, a adormit ?i s-a adaugat la parin?ii lui ?i a vazut stricaciune.
\par 37 Dar Acela pe Care Dumnezeu L-a înviat n-a vazut stricaciune.
\par 38 Cunoscut deci sa va fie voua, barba?i fra?i, ca prin Acesta vi se veste?te iertarea pacatelor ?i ca, de toate câte n-a?i putut sa va îndrepta?i în Legea lui Moise,
\par 39 Întru Acesta tot cel ce crede se îndrepteaza.
\par 40 Deci lua?i aminte sa nu vina peste voi ceea ce s-a zis în prooroci:
\par 41 "Vede?i, îngâmfa?ilor, mira?i-va ?i pieri?i, ca Eu lucrez un lucru, în zilele voastre, un lucru pe care nu-l ve?i crede, daca va va spune cineva".
\par 42 ?i ie?ind ei din sinagoga iudeilor, îi rugau neamurile ca sâmbata viitoare sa li se graiasca cuvintele acestea.
\par 43 Dupa ce s-a terminat adunarea, mul?i dintre iudei ?i dintre prozeli?ii cucernici au mers dupa Pavel ?i dupa Barnaba, care, vorbind catre ei, îi îndemnau sa staruie în harul lui Dumnezeu.
\par 44 Iar în sâmbata urmatoare, mai toata cetatea s-a adunat ca sa auda cuvântul lui Dumnezeu.
\par 45 Dar iudeii, vazând mul?imile, s-au umplut de pizma ?i vorbeau împotriva celor spuse de Pavel, hulind.
\par 46 Iar Pavel ?i Barnaba, îndraznind, au zis: Voua se cadea sa vi se graiasca, mai întâi, cuvântul lui Dumnezeu; dar de vreme ce îl lepada?i ?i va judeca?i pe voi nevrednici de via?a ve?nica, iata ne întoarcem catre neamuri.
\par 47 Caci a?a ne-a poruncit noua Domnul: "Te-am pus spre lumina neamurilor, ca sa fii Tu spre mântuire pâna la marginea pamântului".
\par 48 Iar neamurile pamântului, auzind, se bucurau ?i slaveau cuvântul lui Dumnezeu ?i câ?i erau rândui?i spre via?a ve?nica au crezut;
\par 49 Iar cuvântul Domnului se raspândea prin tot ?inutul.
\par 50 Dar iudeii au întarâtat pe femeile cucernice ?i de cinste ?i pe cei de frunte ai ceta?ii, ?i au ridicat prigoane împotriva lui Pavel ?i a lui Barnaba, ?i i-au scos din hotarele lor.
\par 51 Iar ei, scuturând asupra lor praful de pe picioare, au venit la Iconiu;
\par 52 ?i ucenicii se umpleau de bucurie ?i de Duh Sfânt.

\chapter{14}

\par 1 ?i în Iconiu au intrat ei, ca de obicei, în sinagoga iudeilor ?i astfel au vorbit, încât o mare mul?ime de iudei ?i de elini au crezut.
\par 2 Dar iudeii care n-au crezut au rasculat ?i au înrait sufletele pagânilor împotriva fra?ilor.
\par 3 Deci multa vreme au stat acolo, graind cu îndrazneala întru Domnul, Care da marturie pentru cuvântul harului Sau, facând semne ?i minuni prin mâinile lor.
\par 4 ?i mul?imea din cetate s-a dezbinat ?i unii ?ineau cu iudeii, iar al?ii ?ineau cu apostolii.
\par 5 ?i când pagânii ?i iudeii, împreuna cu capeteniile lor, au dat navala ca sa-i ocarasca ?i sa-i ucida cu pietre,
\par 6 În?elegând, au fugit în ceta?ile Licaoniei: la Listra ?i Derbe ?i în ?inutul dimprejur.
\par 7 ?i acolo propovaduiau Evanghelia.
\par 8 ?i ?edea jos în Listra un om neputincios de picioare, fiind olog, din pântecele maicii sale ?i care nu umblase niciodata.
\par 9 Acesta asculta la Pavel când vorbea. Iar Pavel, cautând spre el ?i vazând ca are credin?a ca sa se mântuiasca,
\par 10 A zis cu glas puternic: Scoala-te drept, pe picioarele tale. ?i el a sarit ?i umbla.
\par 11 Iar mul?imile, vazând ceea ce facuse Pavel, au ridicat glasul lor în limba licaona, zicând: Zeii, asemanându-se oamenilor, s-au coborât la noi.
\par 12 ?i numeau pe Barnaba Zeus, iar pe Pavel Hermes, fiindca el era purtatorul cuvântului.
\par 13 Iar preotul lui Zeus, care era înaintea ceta?ii, aducând la por?i tauri ?i cununi, voia sa le aduca jertfa împreuna cu mul?imile.
\par 14 ?i auzind Apostolii Pavel ?i Barnaba, ?i-au rupt ve?mintele, au sarit în mul?ime, strigând,
\par 15 ?i zicând: Barba?ilor, de ce face?i acestea? Doar ?i noi suntem oameni, asemenea patimitori ca voi, binevestind sa va întoarce?i de la aceste de?ertaciuni catre Dumnezeu cel viu, Care a facut cerul ?i pamântul, marea ?i toate cele ce sunt în ele,
\par 16 ?i Care, în veacurile trecute, a lasat ca toate neamurile sa mearga în caile lor,
\par 17 De?i El nu S-a lasat pe Sine nemarturisit, facându-va bine, dându-va din cer ploi ?i timpuri roditoare, umplând de hrana ?i de bucurie inimile voastre.
\par 18 ?i acestea zicând, abia au potolit mul?imile, ca sa nu le aduca jertfa.
\par 19 Iar de la Antiohia ?i de la Iconiu au venit iudei, care au atras mul?imile de partea lor, ?i, batând pe Pavel cu pietre, l-au târât afara din cetate, gândind ca a murit.
\par 20 Dar înconjurându-l ucenicii, el s-a sculat ?i a intrat în cetate. ?i a doua zi a ie?it cu Barnaba catre Derbe.
\par 21 ?i binevestind ceta?ii aceleia ?i facând ucenici mul?i, s-au înapoiat la Listra, la Iconiu ?i la Antiohia,
\par 22 Întarind sufletele ucenicilor, îndemnându-i sa staruie în credin?a ?i (aratându-le) ca prin multe suferin?e trebuie sa intram în împara?ia lui Dumnezeu.
\par 23 ?i hirotonindu-le preo?i în fiecare biserica, rugându-se cu postiri, i-au încredin?at pe ei Domnului în Care crezusera.
\par 24 ?i strabatând Pisidia, au venit în Pamfilia.
\par 25 ?i dupa ce au grait cuvântul Domnului în Perga, au coborât la Atalia.
\par 26 ?i de acolo au mers cu corabia spre Antiohia, de unde fusesera încredin?a?i harului lui Dumnezeu, spre lucrul pe care l-au împlinit.
\par 27 ?i venind ?i adunând Biserica, au vestit câte a facut Dumnezeu cu ei ?i ca a deschis pagânilor u?a credin?ei.
\par 28 ?i au petrecut acolo cu ucenicii nu pu?ina vreme.

\chapter{15}

\par 1 ?i unii, coborându-se din Iudeea, înva?au pe fra?i ca: Daca nu va taia?i împrejur, dupa rânduiala lui Moise, nu pute?i sa va mântui?i.
\par 2 ?i facându-se pentru ei împotrivire ?i discu?ie nu pu?ina cu Pavel ?i Barnaba, au rânduit ca Pavel ?i Barnaba ?i al?i câ?iva dintre ei sa se suie la apostolii ?i la preo?ii din Ierusalim pentru aceasta întrebare.
\par 3 Deci ei, trimi?i fiind de Biserica, au trecut prin Fenicia ?i prin Samaria, istorisind despre convertirea neamurilor ?i faceau tuturor fra?ilor mare bucurie.
\par 4 ?i sosind ei la Ierusalim, au fost primi?i de Biserica ?i de apostoli ?i de preo?i ?i au vestit câte a facut Dumnezeu cu ei.
\par 5 Dar unii din eresul fariseilor, care trecusera la credin?a, s-au ridicat zicând ca trebuie sa-i taie împrejur ?i sa le porunceasca a pazi Legea lui Moise.
\par 6 ?i apostolii ?i preo?ii s-au adunat ca sa cerceteze despre acest cuvânt.
\par 7 ?i facându-se multa vorbire, s-a sculat Petru ?i le-a zis: Barba?i fra?i, voi ?ti?i ca, din primele zile, Dumnezeu m-a ales între voi, ca prin gura mea neamurile sa auda cuvântul Evangheliei ?i sa creada.
\par 8 ?i Dumnezeu, Cel ce cunoa?te inimile, le-a marturisit, dându-le Duhul Sfânt, ca ?i noua.
\par 9 ?i nimic n-a deosebit între noi ?i ei, cura?ind inimile lor prin credin?a.
\par 10 Acum deci, de ce ispiti?i pe Dumnezeu ?i vre?i sa pune?i pe grumazul ucenicilor un jug pe care nici parin?ii no?tri, nici noi n-am putut sa-l purtam?
\par 11 Ci prin harul Domnului nostru Iisus Hristos, credem ca ne vom mântui în acela?i chip ca ?i aceia.
\par 12 ?i a tacut toata mul?imea ?i asculta pe Barnaba ?i pe Pavel, care istoriseau câte semne ?i minuni a facut Dumnezeu prin ei între neamuri.
\par 13 ?i dupa ce au tacut ei, a raspuns Iacob, zicând: Barba?i fra?i, asculta?i-ma!
\par 14 Simon a istorisit cum de la început a avut grija Dumnezeu sa ia dintre neamuri un popor pentru numele Sau.
\par 15 ?i cu aceasta se potrivesc cuvintele proorocilor, precum este scris:
\par 16 "Dupa acestea Ma voi întoarce ?i voi ridica iara?i cortul cel cazut al lui David ?i cele distruse ale lui iara?i le voi zidi ?i-l voi îndrepta,
\par 17 Ca sa-L caute pe Domnul ceilal?i oameni ?i toate neamurile peste care s-a chemat numele Meu asupra lor, zice Domnul, Cel ce a facut acestea".
\par 18 Lui Dumnezeu Îi sunt cunoscute din veac lucrurile Lui.
\par 19 De aceea eu socotesc sa nu tulburam pe cei ce, dintre neamuri, se întorc la Dumnezeu,
\par 20 Ci sa le scriem sa se fereasca de întinarile idolilor ?i de desfrâu ?i de (animale) sugrumate ?i de sânge.
\par 21 Caci Moise are din timpuri vechi prin toate ceta?ile propovaduitorii sai, fiind citit în sinagogi în fiecare sâmbata.
\par 22 Atunci apostolii ?i preo?ii, cu toata Biserica, au hotarât sa aleaga barba?i dintre ei ?i sa-i trimita la Antiohia, cu Pavel ?i cu Barnaba: pe Iuda cel numit Barsaba, ?i pe Sila, barba?i cu vaza între fra?i.
\par 23 Scriind prin mâinile lor acestea: Apostolii ?i preo?ii ?i fra?ii, fra?ilor dintre neamuri, care sunt în Antiohia ?i în Siria ?i în Cilicia, salutare!
\par 24 Deoarece am auzit ca unii dintre noi, fara sa fi avut porunca noastra, venind, v-au tulburat cu vorbele lor ?i au rava?it sufletele voastre, zicând ca trebuie sa va taia?i împrejur ?i sa pazi?i legea,
\par 25 Noi am hotarât, aduna?i într-un gând, ca sa trimitem la voi barba?i ale?i, împreuna cu iubi?ii no?tri Barnaba ?i Pavel,
\par 26 Oameni care ?i-au pus sufletele lor pentru numele Domnului nostru Iisus Hristos.
\par 27 Drept aceea, am trimis pe Iuda ?i pe Sila, care va vor vesti ?i ei, cu cuvântul, acelea?i lucruri.
\par 28 Pentru ca, parutu-s-a Duhului Sfânt ?i noua, sa nu vi se puna nici o greutate în plus în afara de cele ce sunt necesare:
\par 29 Sa va feri?i de cele jertfite idolilor ?i de sânge ?i de (animale) sugrumate ?i de desfrâu, de care pazindu-va, bine ve?i face. Fi?i sanato?i!
\par 30 Deci cei trimi?i au coborât la Antiohia ?i, adunând mul?imea, au predat scrisoarea.
\par 31 ?i citind-o s-au bucurat pentru mângâiere.
\par 32 Iar Iuda ?i cu Sila, fiind ?i ei prooroci, au mângâiat prin multe cuvântari pe fra?i ?i i-au întarit.
\par 33 ?i petrecând un timp, au fost trimi?i cu pace de catre fra?i la apostoli.
\par 34 Iar Sila s-a hotarât sa ramâna acolo, ?i Iuda a plecat singur la Ierusalim.
\par 35 Iar Pavel ?i Barnaba petreceau în Antiohia, înva?ând ?i binevestind, împreuna cu mul?i al?ii, cuvântul Domnului.
\par 36 ?i dupa câteva zile, Pavel a zis catre Barnaba: Întorcându-ne, sa cercetam cum se afla fra?ii no?tri în toate ceta?ile în care am vestit cuvântul Domnului.
\par 37 Barnaba voia sa ia împreuna cu ei ?i pe Ioan cel numit Marcu;
\par 38 Dar Pavel cerea sa nu-l ia pe acesta cu ei, fiindca se despar?ise de ei din Pamfilia ?i nu venise alaturi de ei la lucrul la care fusesera trimi?i.
\par 39 Deci s-a iscat neîn?elegere între ei, încât s-au despar?it unul de altul, ?i Barnaba, luând pe Marcu, a plecat cu corabia în Cipru;
\par 40 Iar Pavel, alegând pe Sila, a plecat, fiind încredin?at de catre fra?i harului Domnului.
\par 41 ?i strabatea Siria ?i Cilicia, întarind Bisericile.

\chapter{16}

\par 1 ?i a sosit la Derbe ?i la Listra. ?i iata era acolo un ucenic cu numele Timotei, fiul unei femei iudee credincioase, ?i al unui tata elin,
\par 2 Care avea bune marturii de la fra?ii din Listra ?i din Iconiu.
\par 3 Pavel a voit ca acesta sa vina împreuna cu el ?i, luându-l, l-a taiat împrejur, din pricina iudeilor care erau în acele locuri; caci to?i ?tiau ca tatal lui era elin.
\par 4 ?i când treceau prin ceta?i, înva?au sa pazeasca înva?aturile rânduite de apostolii ?i de preo?ii din Ierusalim.
\par 5 Deci Bisericile se întareau în credin?a ?i sporeau cu numarul în fiecare zi.
\par 6 ?i ei au strabatut Frigia ?i ?inutul Galatiei, opri?i fiind de Duhul Sfânt ca sa propovaduiasca cuvântul în Asia.
\par 7 Venind la hotarele Misiei, încercau sa mearga în Bitinia, dar Duhul lui Iisus nu i-a lasat.
\par 8 ?i trecând dincolo de Misia, au coborât la Troa.
\par 9 ?i noaptea i s-a aratat lui Pavel o vedenie: Un barbat macedonean sta rugându-l ?i zicând: Treci în Macedonia ?i ne ajuta.
\par 10 Când a vazut el aceasta vedenie, am cautat sa plecam îndata în Macedonia, în?elegând ca Dumnezeu ne cheama sa le vestim Evanghelia.
\par 11 Pornind cu corabia de la Troa, am mers drept la Samotracia, iar a doua zi la Neapoli,
\par 12 ?i de acolo la Filipi, care este cea dintâi cetate a acestei par?i a Macedoniei ?i colonie romana. Iar în aceasta cetate am ramas câteva zile.
\par 13 ?i în ziua sâmbetei am ie?it în afara por?ii, lânga râu, unde credeam ca este loc de rugaciune ?i, ?ezând, vorbeam femeilor care se adunasera.
\par 14 ?i o femeie, cu numele Lidia, vânzatoare de porfira, din cetatea Tiatirelor, tematoare de Dumnezeu, asculta. Acesteia Dumnezeu i-a deschis inima ca sa ia aminte la cele graite de Pavel.
\par 15 Iar dupa ce s-a botezat ?i ea ?i casa ei, ne-a rugat, zicând: De m-a?i socotit ca sunt credincioasa Domnului, intrând în casa mea, ramâne?i. ?i ne-a facut sa ramânem.
\par 16 Dar odata, pe când ne duceam la rugaciune, ne-a întâmpinat o slujnica, care avea duh pitonicesc ?i care aducea mult câ?tig stapânilor ei, ghicind.
\par 17 Aceasta, ?inându-se dupa Pavel ?i dupa noi, striga, zicând: Ace?ti oameni sunt robi ai Dumnezeului celui Preaînalt, care va vestesc voua calea mântuirii.
\par 18 ?i aceasta o facea timp de multe zile. Iar Pavel mâniindu-se ?i întorcându-se, a zis duhului: În numele lui Iisus Hristos î?i poruncesc sa ie?i din ea. ?i în acel ceas a ie?it.
\par 19 ?i stapânii ei, vazând ca s-a dus nadejdea câ?tigului lor, au pus mâna pe Pavel ?i pe Sila ?i i-au în pia?a înaintea dregatorilor.
\par 20 ?i ducându-i la judecatori, au zis: Ace?ti oameni, care sunt iudei, tulbura cetatea noastra.
\par 21 ?i vestesc obiceiuri care noua nu ne este îngaduit sa le primim, nici sa le facem, fiindca suntem romani.
\par 22 ?i s-a sculat ?i mul?imea împotriva lor. ?i judecatorii, rupându-le hainele, au poruncit sa-i bata cu vergi.
\par 23 ?i, dupa ce le-au dat multe lovituri, i-au aruncat în temni?a, poruncind temnicerului sa-i pazeasca cu grija.
\par 24 Acesta, primind o asemenea porunca, i-a bagat în fundul temni?ei ?i le-a strâns picioarele în butuci;
\par 25 Iar la miezul nop?ii, Pavel ?i Sila, rugându-se, laudau pe Dumnezeu în cântari, iar cei ce erau în temni?a îi ascultau.
\par 26 ?i deodata s-a facut cutremur mare, încât s-au zguduit temeliile temni?ei ?i îndata s-au deschis toate u?ile ?i legaturile tuturor s-au dezlegat.
\par 27 ?i de?teptându-se temnicerul ?i vazând deschise u?ile temni?ei, sco?ând sabia, voia sa se omoare, socotind ca cei închi?i au fugit.
\par 28 Iar Pavel a strigat cu glas mare, zicând: Sa nu-?i faci nici un rau, ca to?i suntem aici.
\par 29 Iar el, cerând lumina, s-a repezit înauntru ?i, tremurând de spaima, a cazut înaintea lui Pavel ?i a lui Sila;
\par 30 ?i sco?ându-i afara (dupa ce pe ceilal?i i-a zavorât la loc), le-a zis: Domnilor, ce trebuie sa fac ca sa ma mântuiesc?
\par 31 Iar ei au zis: Crede în Domnul Iisus ?i te vei mântui tu ?i casa ta.
\par 32 ?i i-au grait lui cuvântul lui Dumnezeu ?i tuturor celor din casa lui.
\par 33 ?i el, luându-i la sine, în acel ceas al nop?ii, a spalat ranile lor ?i s-a botezat el ?i to?i ai lui îndata.
\par 34 ?i ducându-i în casa, a pus masa ?i s-a veselit cu toata casa, crezând în Dumnezeu.
\par 35 ?i facându-se ziua, judecatorii au trimis pe purtatorii de vergi, zicând: Da drumul oamenilor acelora.
\par 36 Iar temnicerul a spus cuvintele acestea catre Pavel: Ca au trimis judecatorii sa fi?i lasa?i liberi. Acum deci ie?i?i ?i merge?i în pace.
\par 37 Dar Pavel a zis catre ei: Dupa ce, fara judecata, ne-au batut în fa?a lumii, pe noi care suntem ceta?eni romani ?i ne-au bagat în temni?a, acum ne scot afara pe ascuns? Nu a?a! Ci sa vina ei în?i?i sa ne scoata afara.
\par 38 ?i purtatorii de vergi au spus judecatorilor aceste cuvinte. ?i auzind ca sunt ceta?eni romani, judecatorii s-au temut.
\par 39 ?i venind, se rugau de ei ?i, sco?ându-i afara, îi rugau sa plece din cetate.
\par 40 Iar ei, ie?ind din închisoare, s-au dus în casa Lidiei; ?i vazând pe fra?i, i-au mângâiat ?i au plecat.

\chapter{17}

\par 1 ?i dupa ce au trecut prin Amfipoli ?i prin Apolonia, au venit la Tesalonic, unde era o sinagoga a iudeilor.
\par 2 ?i dupa obiceiul sau, Pavel a intrat la ei ?i în trei sâmbete le-a grait din Scripturi,
\par 3 Deschizându-le ?i aratându-le ca Hristos trebuia sa patimeasca ?i sa învieze din mor?i, ?i ca Acesta, pe Care vi-L vestesc eu, este Hristosul, Iisus.
\par 4 ?i unii dintre ei au crezut ?i au trecut de partea lui Pavel ?i a lui Sila, ?i mare mul?ime de elini închinatori la Dumnezeu ?i dintre femeile de frunte nu pu?ine.
\par 5 Iar iudeii, umplându-se de invidie ?i luând cu ei pe câ?iva oameni de rând, rai, adunând gloata întarâtau cetatea ?i, ducându-se la casa lui Iason, cautau sa-i scoata afara, înaintea poporului.
\par 6 Dar, negasindu-i, târau pe Iason ?i pe câ?iva fra?i la mai-marii ceta?ii, strigând ca cei ce au tulburat toata lumea au venit ?i aici;
\par 7 Pe ace?tia i-a gazduit Iason; ?i to?i ace?tia lucreaza împotriva poruncilor Cezarului, zicând ca este un alt împarat: Iisus.
\par 8 ?i au tulburat mul?imea ?i pe mai-marii ceta?ii, care auzeau acestea.
\par 9 ?i luând cheza?ie de la Iason ?i de la ceilal?i, le-au dat drumul.
\par 10 Iar fra?ii au trimis îndata, noaptea, la Bereea, pe Pavel ?i pe Sila care, ajungând acolo, au intrat în sinagoga iudeilor.
\par 11 ?i ace?tia erau mai buni la suflet decât cei din Tesalonic; ei au primit cuvântul cu toata osârdia, în toate zilele, cercetând Scripturile, daca ele sunt a?a.
\par 12 Au crezut mul?i dintre ei ?i dintre femeile de cinste ale elinilor, ?i dintre barba?i nu pu?ini.
\par 13 ?i când au aflat iudeii din Tesalonic ca ?i în Bereea s-a vestit de catre Pavel cuvântul lui Dumnezeu, au venit ?i acolo, întarâtând ?i tulburând mul?imile.
\par 14 ?i atunci îndata fra?ii au trimis pe Pavel, ca sa mearga spre mare; iar Sila ?i cu Timotei au ramas acolo în Bereea.
\par 15 Iar cei ce înso?eau pe Pavel l-au dus pâna la Atena; ?i luând ei porunci catre Sila ?i Timotei, ca sa vina la el cât mai curând, au plecat.
\par 16 Iar în Atena, pe când Pavel îi a?tepta, duhul lui se îndârjea în el, vazând ca cetatea este plina de idoli.
\par 17 Deci discuta în sinagoga cu iudeii ?i cu cei credincio?i, ?i în pia?a, în fiecare zi, cu cei ce erau de fa?a.
\par 18 Iar unii dintre filozofii epicurei ?i stoici discutau cu el, ?i unii ziceau: Ce voie?te, oare, sa ne spuna acest semanator de cuvinte? Iar al?ii ziceau: Se pare ca este vestitor de dumnezei straini, fiindca bineveste?te pe Iisus ?i Învierea.
\par 19 ?i luându-l cu ei, l-au dus în Areopag, zicând: Putem sa cunoa?tem ?i noi ce este aceasta înva?atura noua, graita de tine?
\par 20 Caci tu aduci la auzul nostru lucruri straine. Voim deci sa ?tim ce vor sa fie acestea.
\par 21 To?i atenienii ?i strainii, care locuiau acolo, nu-?i petreceau timpul decât spunând sau auzind ceva nou.
\par 22 ?i Pavel, stând în mijlocul Areopagului, a zis: Barba?i atenieni, în toate va vad ca sunte?i foarte evlavio?i.
\par 23 Caci strabatând cetatea voastra ?i privind locurile voastre de închinare, am aflat ?i un altar pe care era scris: "Dumnezeului necunoscut". Deci pe Cel pe Care voi, necunoscându-L, Îl cinsti?i, pe Acesta Îl vestesc eu voua.
\par 24 Dumnezeu, Care a facut lumea ?i toate cele ce sunt în ea, Acesta fiind Domnul cerului ?i al pamântului, nu locuie?te în temple facute de mâini,
\par 25 Nici nu este slujit de mâini omene?ti, ca ?i cum ar avea nevoie de ceva, El dând tuturor via?a ?i suflare ?i toate.
\par 26 ?i a facut dintr-un sânge tot neamul omenesc, ca sa locuiasca peste toata fa?a pamântului, a?ezând vremile cele de mai înainte rânduite ?i hotarele locuirii lor,
\par 27 Ca ei sa caute pe Dumnezeu, doar L-ar pipai ?i L-ar gasi, de?i nu e departe de fiecare dintre noi.
\par 28 Caci în El traim ?i ne mi?cam ?i suntem, precum au zis ?i unii dintre poe?ii vo?tri: caci ai Lui neam ?i suntem.
\par 29 Fiind deci neamul lui Dumnezeu, nu trebuie sa socotim ca dumnezeirea este asemenea aurului sau argintului sau pietrei cioplite de me?te?ugul ?i de iscusin?a omului.
\par 30 Dar Dumnezeu, trecând cu vederea veacurile ne?tiin?ei, veste?te acum oamenilor ca to?i de pretutindeni sa se pocaiasca,
\par 31 Pentru ca a hotarât o zi în care va sa judece lumea întru dreptate, prin Barbatul pe care L-a rânduit, daruind tuturor încredin?are, prin Învierea Lui din mor?i.
\par 32 ?i auzind despre învierea mor?ilor, unii l-au luat în râs, iar al?ii i-au zis: Te vom asculta despre aceasta ?i altadata.
\par 33 Astfel Pavel a ie?it din mijlocul lor.
\par 34 Iar unii barba?i, alipindu-se de el, au crezut, între care ?i Dionisie Areopagitul ?i o femeie cu numele Damaris, ?i al?ii împreuna cu ei.

\chapter{18}

\par 1 Dupa acestea Pavel, plecând din Atena, a venit la Corint.
\par 2 ?i gasind pe un iudeu, cu numele Acvila, de neam din Pont, venit de curând din Italia, ?i pe Priscila, femeia lui, pentru ca poruncise Claudiu ca to?i iudeii sa plece din Roma, a venit la ei.
\par 3 ?i pentru ca erau de aceea?i meserie, a ramas la ei ?i lucrau, caci erau facatori de corturi.
\par 4 ?i vorbea în sinagoga în fiecare sâmbata ?i aducea la credin?a iudei ?i elini.
\par 5 Iar când Sila ?i Timotei au venit din Macedonia, Pavel era prins cu totul de cuvânt, marturisind iudeilor ca Iisus este Hristosul.
\par 6 ?i stând ei împotriva ?i hulind, el, scuturându-?i hainele, a zis catre ei: Sângele vostru asupra capului vostru! Eu sunt curat. De acum înainte ma voi duce la neamuri.
\par 7 ?i mutându-se de acolo, a venit în casa unuia, cu numele Titus Iustus, cinstitor al lui Dumnezeu, a carui casa era alaturi de sinagoga.
\par 8 Dar Crispus, mai-marele sinagogii, a crezut în Domnul, împreuna cu toata casa sa; ?i mul?i dintre corinteni, auzind, credeau ?i se botezau.
\par 9 ?i Domnul a zis lui Pavel, noaptea în vedenie: Nu te teme, ci vorbe?te ?i nu tacea,
\par 10 Pentru ca Eu sunt cu tine ?i nimeni nu va pune mâna pe tine, ca sa-?i faca rau. Caci am mult popor în cetatea aceasta.
\par 11 ?i a stat în Corint un an ?i ?ase luni, înva?ând între ei cuvântul lui Dumnezeu.
\par 12 Dar pe când Galion era proconsulul Ahaiei, iudeii s-au ridicat to?i într-un cuget împotriva lui Pavel ?i l-au adus la tribunal,
\par 13 Zicând ca acesta cauta sa convinga pe oameni sa se închine lui Dumnezeu, împotriva legii.
\par 14 ?i când Pavel era gata sa deschida gura, Galion a zis catre iudei: Daca ar fi vreo nedreptate sau vreo fapta vicleana, o, iudeilor, v-a? asculta precum se cuvine;
\par 15 Dar daca sunt la voi nedumeriri despre înva?atura ?i despre nume ?i despre legea voastra, vede?i-va voi în?iva de ele. Judecator pentru acestea eu nu voiesc sa fiu.
\par 16 ?i i-a izgonit de la tribunal.
\par 17 ?i punând mâna to?i pe Sostene, mai-marele sinagogii, îl bateau înaintea tribunalului. Dar Galion nu lua în seama nimic din acestea;
\par 18 Iar Pavel, dupa ce a stat înca multe zile în Corint, ?i-a luat ramas bun de la fra?i ?i a plecat cu corabia în Siria, împreuna cu Priscila ?i cu Acvila, care ?i-a tuns capul la Chenhrea, caci facuse o fagaduin?a.
\par 19 ?i au sosit la Efes ?i pe aceia i-a lasat acolo, iar el, intrând în sinagoga, discuta cu iudeii.
\par 20 ?i rugându-l sa ramâna la ei mai multa vreme, n-a voit,
\par 21 Ci, despar?indu-se de ei, a zis: Trebuie, negre?it, ca sarbatoarea care vine s-o fac la Ierusalim, dar, cu voia Domnului, ma voi întoarce iara?i la voi. ?i a plecat de la Efes, cu corabia.
\par 22 ?i coborându-se la Cezareea, s-a suit (la Ierusalim) ?i, îmbra?i?ând Biserica, s-a coborât la Antiohia.
\par 23 ?i stând acolo câtva timp, a plecat, strabatând pe rând ?inutul Galatiei ?i Frigia, întarind pe to?i ucenicii.
\par 24 Iar un iudeu, cu numele Apollo, alexandrin de neam, barbat iscusit la cuvânt, puternic fiind în Scripturi, a sosit la Efes.
\par 25 Acesta era înva?at în calea Domnului ?i, arzând cu duhul, graia ?i înva?a drept cele despre Iisus, cunoscând numai botezul lui Ioan.
\par 26 ?i el a început sa vorbeasca, fara sfiala, în sinagoga. Auzindu-l, Priscila ?i Acvila l-au luat cu ei ?i i-au aratat mai cu de-amanuntul calea lui Dumnezeu.
\par 27 ?i voind el sa treaca în Ahaia, l-au îndemnat fra?ii ?i au scris ucenicilor sa-l primeasca. ?i sosind (în Corint), a ajutat mult cu harul celor ce crezusera;
\par 28 Caci cu tarie ?i în fa?a tuturor, el înfrunta pe iudei, dovedind din Scripturi ca Iisus este Hristos.

\chapter{19}

\par 1 ?i pe când Apollo era în Corint, Pavel, dupa ce a strabatut par?ile de sus, a venit în Efes. ?i gasind câ?iva ucenici,
\par 2 A zis catre ei: Primit-a?i voi Duhul Sfânt când a?i crezut? Iar ei au zis catre el: Dar nici n-am auzit daca este Duh Sfânt.
\par 3 ?i el a zis: Deci în ce v-a?i botezat? Ei au zis: În botezul lui Ioan.
\par 4 Iar Pavel a zis: Ioan a botezat cu botezul pocain?ei, spunând poporului sa creada în Cel ce avea sa vina dupa el, adica în Iisus Hristos.
\par 5 ?i auzind ei, s-au botezat în numele Domnului Iisus.
\par 6 ?i punându-?i Pavel mâinile peste ei, Duhul Sfânt a venit asupra lor ?i vorbeau în limbi ?i prooroceau.
\par 7 ?i erau to?i ca la doisprezece barba?i.
\par 8 ?i el, intrând în sinagoga, a vorbit cu îndrazneala timp de trei luni, vorbind cu ei ?i cautând sa-i încredin?eze de împara?ia lui Dumnezeu.
\par 9 Dar fiindca unii erau învârto?a?i ?i nu credeau, bârfind calea Domnului înaintea mul?imii, Pavel, plecând de la ei, a osebit pe ucenici, înva?ând în fiecare zi în ?coala unuia Tiranus.
\par 10 ?i acesta a ?inut vreme de doi ani, încât to?i, cei ce locuiau în Asia, ?i iudei ?i elini, au auzit cuvântul Domnului.
\par 11 ?i Dumnezeu facea, prin mâinile lui Pavel, minuni nemaiîntâlnite.
\par 12 Încât ?i peste cei ce erau bolnavi se puneau ?tergare sau ?or?uri purtate de Pavel, ?i bolile se departau de ei, iar duhurile cele rele ie?eau din ei.
\par 13 ?i au încercat unii dintre iudeii care cutreierau lumea, sco?ând demoni, sa cheme peste cei ce aveau duhuri rele, numele Domnului Iisus, zicând: Va jur pe Iisus, pe Care-l propovaduie?te Pavel!
\par 14 Iar cei care faceau aceasta erau cei ?apte fii ai unuia Scheva, arhiereu iudeu.
\par 15 ?i raspunzând, duhul cel rau le-a zis: Pe Iisus Îl cunosc ?i îl ?tiu ?i pe Pavel, dar voi cine sunte?i?
\par 16 ?i sarind asupra lor omul în care era duhul cel rau ?i biruindu-i, s-a întarâtat asupra lor, încât ei au fugit goi ?i rani?i din casa aceea.
\par 17 ?i acest lucru s-a facut cunoscut tuturor iudeilor ?i elinilor care locuiau în Efes, ?i frica a cazut peste to?i ace?tia ?i se slavea numele Domnului Iisus.
\par 18 ?i mul?i dintre cei ce crezusera veneau ca sa se marturiseasca ?i sa spuna faptele lor.
\par 19 Iar mul?i dintre cei ce facusera vrajitorie, aducând car?ile, le ardeau în fa?a tuturor. ?i au socotit pre?ul lor ?i au gasit cincizeci de mii de argin?i.
\par 20 Astfel cre?tea cu putere cuvântul Domnului ?i se întarea.
\par 21 ?i dupa ce s-au savâr?it acestea, Pavel ?i-a pus în gând sa treaca prin Macedonia ?i prin Ahaia ?i sa se duca la Ierusalim, zicând ca: Dupa ce voi fi acolo, trebuie sa vad ?i Roma.
\par 22 ?i trimi?ând în Macedonia pe doi dintre cei care îl slujeau, pe Timotei ?i pe Erast, el a ramas o vreme în Asia.
\par 23 ?i în vremea aceea s-a facut mare tulburare pentru calea Domnului.
\par 24 Caci un argintar, cu numele Dimitrie, care facea temple de argint Artemisei ?i da me?terilor sai foarte mare câ?tig,
\par 25 I-a adunat pe ace?tia ?i pe cei care lucrau unele ca acestea, ?i le-a zis: Barba?ilor, ?ti?i ca din aceasta îndeletnicire este câ?tigul vostru.
\par 26 ?i voi vede?i ?i auzi?i ca nu numai în Efes, ci aproape în toata Asia, Pavel acesta, convingând, a întors multa mul?ime, zicând ca nu sunt dumnezei cei facu?i de mâini.
\par 27 Din aceasta nu numai ca meseria noastra e în primejdie sa ajunga fara trecere, dar ?i templul marii zei?e Artemisa e în primejdie sa nu mai aiba nici un pre?, iar cu vremea, marirea ei - careia i se închina toata Asia ?i toata lumea - sa fie doborâta.
\par 28 ?i auzind ei ?i umplându-se de mânie, strigau zicând: Mare este Artemisa efesenilor!
\par 29 ?i s-a umplut toata cetatea de tulburare ?i au pornit într-un cuget la teatru, rapind împreuna pe macedonenii Gaius ?i Aristarh, înso?itorii lui Pavel.
\par 30 Iar Pavel, voind sa intre în mijlocul poporului, ucenicii nu l-au lasat.
\par 31 Înca ?i unii dintre dregatorii Asiei, care îi erau prieteni, trimi?ând la el, îl rugau sa nu se duca la teatru.
\par 32 Deci unii strigau una, al?ii strigau alta, caci adunarea era învalma?ita, iar cei mai mul?i nu ?tiau pentru ce s-au adunat acolo.
\par 33 Iar unii din mul?ime l-au smuls pe Alexandru, pe care l-au împins înainte iudeii. Iar el, facând semn cu mâna, voia sa se apere înaintea poporului.
\par 34 ?i cunoscând ei ca este iudeu, to?i într-un glas au strigat aproape doua ceasuri: Mare este Artemisa efesenilor!
\par 35 Iar secretarul, potolind mul?imea, a zis: Barba?i efeseni, cine este, între oameni, care sa nu ?tie ca cetatea efesenilor este pazitoarea templului Artemisei celei mari ?i a statuii ei, cazuta din cer?
\par 36 Deci, acestea fiind mai presus de orice îndoiala, trebuie sa va lini?ti?i ?i sa nu face?i nimic cu u?urin?a.
\par 37 Caci a?i adus pe barba?ii ace?tia, care nu sunt nici furi de cele sfinte, nici nu hulesc pe zei?a voastra.
\par 38 Deci daca Dimitrie ?i me?terii cei împreuna cu el au vreo plângere împotriva cuiva, au judecatori ?i proconsuli care sa judece, ?i sa se cheme în judecata unii pe al?ii.
\par 39 Iar daca urmari?i altceva, se va hotarî în adunarea cea legiuita,
\par 40 Caci noi suntem în primejdie sa fim învinui?i de rascoala pentru ziua de azi, fiindca nu avem nici o pricina pentru care am putea da seama de tulburarea aceasta.
\par 41 Zicând acestea, a slobozit adunarea.

\chapter{20}

\par 1 Iar dupa ce a încetat tulburarea, Pavel, chemând pe ucenici ?i dându-le îndemnuri, dupa ce ?i-a luat ramas bun, a ie?it sa mearga în Macedonia.
\par 2 ?i strabatând acele par?i ?i dând ucenicilor multe sfaturi ?i îndemnuri, a sosit în Grecia.
\par 3 ?i a stat acolo trei luni. Dar când era sa plece pe apa în Siria, iudeii au uneltit împotriva vie?ii lui, iar el s-a hotarât sa se întoarca prin Macedonia.
\par 4 ?i mergeau împreuna cu el, pâna în Asia, Sosipatru al lui Piru din Bereea, Aristarh ?i Secundus din Tesalonic ?i Gaius din Derbe ?i Timotei, iar din Asia: Tihic ?i Trofim.
\par 5 Ace?tia, plecând înainte, ne-au a?teptat în Troa.
\par 6 Iar noi, dupa zilele Azimelor, am pornit cu corabia de la Filipi ?i în cinci zile am sosit la ei în Troa, unde am ramas ?apte zile.
\par 7 În ziua întâi a saptamânii (Duminica) adunându-ne noi sa frângem pâinea, Pavel, care avea de gând sa plece a doua zi, a început sa le vorbeasca ?i a prelungit cuvântul lui pâna la miezul nop?ii.
\par 8 Iar în camera de sus, unde erau aduna?i, erau multe lumini aprinse.
\par 9 Dar un tânar cu numele Eutihie, ?ezând pe fereastra, pe când Pavel ?inea lungul sau cuvânt, a adormit adânc ?i, doborât de somn, a cazut jos de la catul al treilea, ?i l-au ridicat mort.
\par 10 Iar Pavel, coborându-se, s-a plecat peste el ?i, luându-l în bra?e, a zis: Nu va tulbura?i, caci sufletul lui este în el.
\par 11 ?i suindu-se ?i frângând pâinea ?i mâncând, a vorbit cu ei mult pâna în zori, ?i atunci a plecat.
\par 12 Iar pe tânar l-au adus viu ?i foarte mult s-au mângâiat.
\par 13 Iar noi, venind la corabie, am plutit spre Asson, ca sa luam de acolo pe Pavel, caci astfel rânduise el, voind sa mearga pe jos.
\par 14 Dupa ce s-a întâlnit cu noi la Asson, luându-l cu noi, am venit la Mitilene.
\par 15 ?i de acolo, mergând cu corabia, am sosit a doua zi în fa?a insulei Hios. Iar în ziua urmatoare, am ajuns în Samos ?i, dupa ce am poposit la Troghilion, a doua zi am venit la Milet.
\par 16 Caci Pavel hotarâse sa treaca pe apa pe lânga Efes, ca sa nu i se întârzie în Asia, pentru ca se grabea sa fie, daca i-ar fi cu putin?a, la Ierusalim, de ziua Cincizecimii.
\par 17 ?i trimi?ând din Milet la Efes, a chemat la sine pe preo?ii Bisericii.
\par 18 ?i când ei au venit la el, le-a zis: Voi ?ti?i cum m-am purtat cu voi, în toata vremea, din ziua cea dintâi, când am venit în Asia,
\par 19 Slujind Domnului cu toata smerenia ?i cu multe lacrimi ?i încercari care mi s-au întâmplat prin uneltirile iudeilor.
\par 20 ?i cum n-am ascuns nimic din cele folositoare, ca sa nu vi le vestesc ?i sa nu va înva?, fie înaintea poporului, fie prin case,
\par 21 Marturisind ?i iudeilor ?i elinilor întoarcerea la Dumnezeu prin pocain?a ?i credin?a în Domnul nostru Iisus Hristos.
\par 22 Iar acum iata ca fiind eu mânat de Duhul, merg la Ierusalim, ne?tiind cele ce mi se vor întâmpla acolo,
\par 23 Decât numai ca Duhul Sfânt marturise?te prin ceta?i, spunându-mi ca ma a?teapta lan?uri ?i necazuri.
\par 24 Dar nimic nu iau în seama ?i nu pun nici un pre? pe sufletul meu, numai sa împlinesc calea mea ?i slujba mea pe care am luat-o de la Domnul Iisus, de a marturisi Evanghelia harului lui Dumnezeu.
\par 25 ?i acum, iata, eu ?tiu ca voi to?i, printre care am petrecut propovaduind împara?ia lui Dumnezeu, nu ve?i mai vedea fa?a mea.
\par 26 Pentru aceea va marturisesc în ziua de astazi ca sunt curat de sângele tuturor.
\par 27 Caci nu m-am ferit sa va vestesc toata voia lui Dumnezeu.
\par 28 Drept aceea, lua?i aminte de voi în?iva ?i de toata turma, întru care Duhul Sfânt v-a pus pe voi episcopi, ca sa pastra?i Biserica lui Dumnezeu, pe care a câ?tigat-o cu însu?i sângele Sau.
\par 29 Caci eu ?tiu aceasta, ca dupa plecarea mea vor intra, între voi, lupi îngrozitori, care nu vor cru?a turma.
\par 30 ?i dintre voi în?iva se vor ridica barba?i, graind înva?aturi rastalmacite, ca sa traga pe ucenici dupa ei.
\par 31 Drept aceea, priveghea?i, aducându-va aminte ca, timp de trei ani, n-am încetat noaptea ?i ziua sa va îndemn, cu lacrimi, pe fiecare dintre voi.
\par 32 ?i acum va încredin?ez lui Dumnezeu ?i cuvântului harului Sau, cel ce poate sa va zideasca ?i sa va dea mo?tenire între to?i cei sfin?i?i.
\par 33 Argint, sau aur, sau haina, n-am poftit de la nimeni;
\par 34 Voi în?iva ?ti?i ca mâinile acestea au lucrat pentru trebuin?ele mele ?i ale celor ce erau cu mine.
\par 35 Toate vi le-am aratat, caci ostenindu-va astfel, trebuie sa ajuta?i pe cei slabi ?i sa va aduce?i aminte de cuvintele Domnului Iisus, caci El a zis: Mai fericit este a da decât a lua.
\par 36 ?i dupa ce a spus acestea, plecându-?i genunchii, s-a rugat împreuna cu to?i ace?tia.
\par 37 ?i mare jale i-a cuprins pe to?i ?i, cazând pe grumazul lui Pavel, îl sarutau,
\par 38 Cuprin?i de jale mai ales pentru cuvântul pe care îl spusese, ca n-au sa mai vada fa?a lui. ?i îl petrecura la corabie.

\chapter{21}

\par 1 ?i dupa ce ne-am despar?it de ei, am plecat pe apa ?i, mergând drept, am venit la Cos ?i a doua zi la Rodos, iar de acolo la Patara.
\par 2 ?i gasind o corabie, care mergea în Fenicia, ne-am urcat în ea ?i am plecat.
\par 3 ?i zarind Ciprul ?i lasându-l la stânga, am plutit spre Siria ?i ne-am coborât în Tir, caci acolo corabia avea sa descarce povara.
\par 4 ?i gasind pe ucenici, am ramas acolo ?apte zile. Ace?tia spuneau lui Pavel, prin duhul, sa nu se suie la Ierusalim.
\par 5 ?i când am împlinit zilele, ie?ind, am plecat, petrecându-ne to?i, împreuna cu femei ?i cu copii, pâna afara din cetate ?i, plecând genunchii pe ?arm, ne-am rugat.
\par 6 ?i ne-am îmbra?i?at unii pe al?ii ?i ne-am urcat în corabie, iar aceia s-au întors la ale lor.
\par 7 Iar noi, sfâr?ind calatoria noastra pe apa, de la Tir am venit la Ptolemaida ?i, îmbra?i?ând pe fra?i, am ramas la ei o zi.
\par 8 Iar a doua zi, ie?ind, am venit la Cezareea. ?i intrând în casa lui Filip binevestitorul, care era dintre cei ?apte (diaconi), am ramas la el.
\par 9 ?i acesta avea patru fiice, fecioare, care prooroceau.
\par 10 ?i ramânând noi acolo mai multe zile, a coborât din Iudeea un prooroc cu numele Agav;
\par 11 ?i, venind el la noi, a luat brâul lui Pavel ?i legându-?i picioarele ?i mâinile a zis: Acestea zice Duhul Sfânt: Pe barbatul al caruia este acest brâu, a?a îl vor lega iudeii la Ierusalim ?i-l vor da în mâinile neamurilor.
\par 12 ?i când am auzit acestea, îl rugam ?i noi ?i localnicii sa nu se suie la Ierusalim.
\par 13 Atunci a raspuns Pavel: Ce face?i de plânge?i ?i-mi sfâ?ia?i inima? Caci eu sunt gata nu numai sa fiu legat, ci sa ?i mor în Ierusalim pentru numele Domnului Iisus.
\par 14 ?i neînduplecându-se el, ne-am lini?tit, zicând: Faca-se voia Domnului.
\par 15 Iar dupa zilele acestea, pregatindu-ne, ne-am suit la Ierusalim.
\par 16 ?i au venit împreuna cu noi ?i dintre ucenicii din Cezareea, ducându-ne la un oarecare Mnason din Cipru, vechi ucenic, la care am fost gazdui?i.
\par 17 ?i sosind la Ierusalim, fra?ii ne-au primit cu bucurie.
\par 18 Iar a doua zi Pavel a mers cu noi la Iacov ?i au venit acolo to?i preo?ii.
\par 19 ?i îmbra?i?ându-i le povestea cu de-amanuntul cele ce a facut Dumnezeu între neamuri, prin slujirea lui.
\par 20 Iar ei, auzind, slaveau pe Dumnezeu, ?i i-au zis: Vezi frate, câte mii de iudei au crezut ?i to?i sunt plini de râvna pentru lege.
\par 21 ?i ei au auzit despre tine ca înve?i pe to?i iudeii, care traiesc printre neamuri, sa se lepede de Moise, spunându-le sa nu-?i taie împrejur copiii, nici sa umble dupa datini.
\par 22 Ce este deci? Fara îndoiala, trebuie sa se adune mul?ime, caci vor auzi ca ai venit.
\par 23 Fa, deci, ceea ce î?i spunem. Noi avem patru barba?i, care au asupra lor o fagaduin?a;
\par 24 Pe ace?tia luându-i, cura?e?te-te împreuna cu ei ?i cheltuie?te pentru ei ca sa-?i rada capul, ?i vor cunoa?te to?i ca nimic nu este (adevarat) din cele ce au auzit despre tine, dar ca tu însu?i umbli dupa Lege ?i o paze?ti.
\par 25 Cât despre pagânii care au crezut, noi le-am trimis o scrisoare, hotarându-le sa se fereasca de ceea ce este jertfit idolilor ?i de sânge ?i de (animal) sugrumat ?i de desfrâu.
\par 26 Atunci Pavel, luând cu el pe acei barba?i, cura?indu-se împreuna cu ei a doua zi, a intrat în templu, vestind împlinirea zilelor cura?irii, pâna când a fost adusa ofranda pentru fiecare din ei.
\par 27 ?i când era sa se împlineasca cele ?apte zile, iudeii din Asia, vazându-l în templu, au întarâtat toata mul?imea ?i au pus mâna pe el,
\par 28 Strigând: Barba?i israeli?i, ajuta?i! Acesta este omul care înva?a pe to?i pretutindeni, împotriva poporului ?i a Legii ?i a locului acestuia; înca ?i elini a adus în templu ?i a spurcat acest loc sfânt.
\par 29 Caci ei vazusera mai înainte cu el împreuna în cetate pe Trofim din Efes, pe care socoteau ca Pavel l-a adus în templu.
\par 30 ?i s-a mi?cat toata cetatea ?i poporul a alergat din toate par?ile ?i, punând mâna pe Pavel, îl trageau afara din templu ?i îndata au închis por?ile.
\par 31 Dar când cautau ei ca sa-l omoare, a ajuns veste la comandantul cohortei, ca tot Ierusalimul s-a tulburat.
\par 32 Acela, luând îndata osta?i ?i suta?i, a alergat la ei; iar ei, vazând pe comandant ?i pe osta?i, au încetat de a mai bate pe Pavel.
\par 33 Apropiindu-se atunci comandantul, a pus mâna pe el ?i a poruncit sa fie legat cu doua lan?uri ?i întrebat cine este ?i ce a facut.
\par 34 Iar unii strigau în mul?ime una, al?ii altceva ?i neputând sa în?eleaga adevarul, din cauza tulburarii, a poruncit sa fie dus în fortarea?a.
\par 35 Când a ajuns la trepte, a trebuit, de furia mul?imii, sa fie purtat de osta?i.
\par 36 Caci mergea dupa el mul?ime de popor, strigând: Omoara-l!
\par 37 ?i vrând sa-l duca în fortarea?a, Pavel a zis comandantului: Îmi este îngaduit sa vorbesc ceva cu tine? Iar el a zis: ?tii grece?te?
\par 38 Nu e?ti tu, oare, egipteanul care, înainte de zilele acestea, te-ai rasculat ?i ai scos în pustie pe cei patru mii de barba?i razvrati?i?
\par 39 ?i a zis Pavel: Eu sunt iudeu din Tarsul Ciliciei, ceta?ean al unei ceta?i care nu este neînsemnata. Te rog da-mi voie sa vorbesc catre popor.
\par 40 ?i dându-i-se voie, Pavel, stând în picioare pe trepte, a facut poporului semn cu mâna. ?i facându-se mare tacere, a vorbit în limba evreiasca, zicând:

\chapter{22}

\par 1 Barba?i fra?i ?i parin?i, asculta?i acum, apararea mea fa?a de voi!
\par 2 ?i auzind ca le vorbea în limba evreiasca, au facut mai multa lini?te, ?i el le-a zis:
\par 3 Eu sunt barbat iudeu, nascut în Tarsul Ciliciei ?i crescut în cetatea aceasta, înva?ând la picioarele lui Gamaliel în chip amanun?it Legea parinteasca, plin fiind de râvna pentru Dumnezeu, precum ?i voi to?i sunte?i astazi.
\par 4 Eu am prigonit pâna la moarte aceasta cale, legând ?i dând la închisoare ?i barba?i ?i femei,
\par 5 Precum marturise?te pentru mine ?i arhiereul ?i tot sfatul batrânilor, de la care primind ?i scrisori catre fra?i, mergeam la Damasc, ca sa-i aduc lega?i la Ierusalim ?i pe cei ce erau acolo, spre a fi pedepsi?i.
\par 6 Dar pe când mergeam eu ?i ma apropiam de Damasc, pe la amiaza, deodata o lumina puternica din cer m-a învaluit ca un fulger.
\par 7 ?i am cazut la pamânt ?i am auzit un glas, zicându-mi: Saule, Saule, de ce Ma prigone?ti?
\par 8 Iar eu am raspuns: Cine e?ti, Doamne? Zis-a catre mine: Eu sunt Iisus Nazarineanul, pe Care tu Îl prigone?ti.
\par 9 Iar cei ce erau cu mine au vazut lumina ?i s-au înfrico?at, dar glasul Celui care îmi vorbea ei nu l-au auzit.
\par 10 ?i am zis: Ce sa fac, Doamne? Iar Domnul a zis catre mine: Ridica-te ?i mergi în Damasc ?i acolo ?i se va spune despre toate cele ce ?i s-au rânduit sa faci.
\par 11 ?i pentru ca nu mai vedeam, din cauza stralucirii acelei lumini, am venit în Damasc, fiind dus de mâna de catre cei ce erau împreuna cu mine.
\par 12 Iar un oarecare Anania, barbat evlavios, dupa Lege, marturisit de to?i iudeii care locuiau în Damasc,
\par 13 Venind la mine ?i stând alaturi, mi-a zis: Frate Saule, vezi iara?i!  ?i eu în ceasul acela l-am vazut.
\par 14 Iar el a zis: Dumnezeul parin?ilor no?tri te-a ales de mai înainte pe tine ca sa cuno?ti voia Lui ?i sa vezi pe Cel Drept ?i sa auzi glas din gura Lui;
\par 15 Ca martor vei fi Lui, în fa?a tuturor oamenilor, despre cele ce ai vazut ?i auzit.
\par 16 ?i acum de ce zabove?ti? Sculându-te, boteaza-te ?i spala-?i pacatele, chemând numele Lui.
\par 17 ?i s-a întâmplat, când m-am întors la Ierusalim ?i ma rugam în templu, sa fiu în extaz,
\par 18 ?i sa-L vad zicându-mi: Grabe?te-te, ?i ie?i degraba din Ierusalim, pentru ca nu vor primi marturia ta despre Mine.
\par 19 ?i eu am zis: Doamne, ei ?tiu ca eu duceam la închisoare ?i bateam, prin sinagogi, pe cei care credeau în Tine;
\par 20 ?i când se varsa sângele lui ?tefan, mucenicul Tau, eram ?i eu de fa?a ?i încuviin?am uciderea lui ?i pazeam hainele celor care îl ucideau.
\par 21 ?i a zis catre mine: Mergi, ca Eu te voi trimite departe, la neamuri.
\par 22 ?i l-au ascultat pâna la acest cuvânt, ?i au ridicat glasul lor, zicând: Ia-l de pe pamânt pe unul ca acesta! Caci nu se cuvine ca el sa mai traiasca.
\par 23 ?i strigând ei ?i aruncând hainele ?i azvârlind pulbere în aer,
\par 24 Comandantul a poruncit sa-l duca în fortarea?a, spunând sa-l ia la cercetare, cu biciul, ca sa cunoasca pentru care pricina strigau a?a împotriva lui.
\par 25 ?i când l-au întins ca sa-l biciuiasca, Pavel a zis catre suta?ul care era de fa?a: Oare va este îngaduit sa biciui?i un ceta?ean roman ?i nejudecat?
\par 26 ?i auzind suta?ul s-a dus la comandant sa-i vesteasca, zicând: Ce ai de gând sa faci? Ca omul acesta este (ceta?ean) roman.
\par 27 ?i venind la el, comandantul i-a zis: Spune-mi, e?ti tu (ceta?ean) roman? Iar el a zis: Da!
\par 28 ?i a raspuns comandantul: Eu am dobândit aceasta ceta?enie cu multa cheltuiala. Iar Pavel a zis: Eu însa m-am ?i nascut.
\par 29 Deci cei ce erau gata sa-l ia la cercetare s-au departat îndata de la el, iar comandantul s-a temut, aflând ca el este (ceta?ean) roman ?i ca a fost legat.
\par 30 ?i a doua zi, voind sa cunoasca adevarul, pentru care era pârât de iudei, l-a dezlegat ?i a poruncit sa se adune arhiereii ?i tot sinedriul ?i, aducând pe Pavel, l-a pus înaintea lor.

\chapter{23}

\par 1 ?i Pavel, fixând sinedriul cu privirea, a zis: Barba?i fra?i, eu cu bun cuget am vie?uit înaintea lui Dumnezeu pâna în ziua aceasta.
\par 2 Arhiereul Anania a poruncit celor ce ?edeau lânga el sa-l bata peste gura.
\par 3 Atunci Pavel a zis catre el: Te va bate Dumnezeu, perete varuit! ?i tu ?ezi sa ma judeci pe mine dupa Lege, ?i, calcând Legea, porunce?ti sa ma bata?
\par 4 Iar cei ce stateau lânga el au zis: Pe arhiereul lui Dumnezeu îl faci tu de ocara?
\par 5 Iar Pavel a zis: Fra?ilor, nu ?tiam ca este arhiereu; caci este scris: "Pe mai-marele poporului tau sa nu-l vorbe?ti de rau".
\par 6 Dar Pavel, ?tiind ca o parte erau saduchei ?i cealalta farisei, a strigat în sinedriu: Barba?i fra?i! Eu sunt fariseu, fiu de farisei. Pentru nadejdea ?i învierea mor?ilor sunt eu judecat!
\par 7 ?i graind el aceasta, între farisei ?i saduchei s-a iscat neîn?elegere ?i mul?imea s-a dezbinat;
\par 8 Caci saducheii zic ca nu este înviere, nici înger, nici duh, iar fariseii marturisesc ?i una ?i alta.
\par 9 ?i s-a facut mare strigare, ?i, ridicându-se unii carturari din partea fariseilor, se certau zicând: Nici un rau nu gasim în acest om; iar daca i-a vorbit lui un duh sau înger, sa nu ne împotrivim lui Dumnezeu.
\par 10 Deci facându-se mare neîn?elegere ?i temându-se comandantul ca Pavel sa nu fie sfâ?iat de ei, a poruncit osta?ilor sa se coboare ?i sa-l smulga din mijlocul lor ?i sa-l duca în fortarea?a.
\par 11 Iar în noaptea urmatoare, aratându-i-Se, Domnul i-a zis: Îndrazne?te, Pavele! Caci precum ai marturisit cele despre Mine la Ierusalim, a?a trebuie sa marturise?ti ?i la Roma.
\par 12 Iar când s-a facut ziua, iudeii, facând sfat împotriva-i, s-au legat cu blestem zicând ca nu vor mânca, nici nu vor bea pâna ce nu vor ucide pe Pavel.
\par 13 ?i cei ce facusera între ei acest juramânt erau mai mul?i de patruzeci,
\par 14 Care, ducându-se la arhierei ?i la batrâni, au zis: Ne-am legat pe noi în?ine cu blestem sa nu gustam nimic pâna ce nu vom ucide pe Pavel.
\par 15 Acum deci voi, împreuna cu sinedriul, face?i cunoscut comandantului sa-l coboare mâine la voi, ca având sa cerceteze mai cu de-amanuntul cele despre el; iar noi, înainte de a se apropia el, suntem gata sa-l ucidem.
\par 16 Dar fiul surorii lui Pavel, auzind despre aceasta uneltire, ducându-se ?i intrând în fortarea?a, i-a vestit lui Pavel.
\par 17 ?i chemând Pavel pe unul din suta?i, i-a zis: Du pe tânarul acesta la comandant, caci are sa-i vesteasca ceva.
\par 18 Iar el, luându-l, l-a dus la comandant ?i a zis: Pavel cel legat, chemându-ma, m-a rugat sa aduc pe acest tânar la tine, având sa-?i spuna ceva.
\par 19 Comandantul, luându-l de mâna, s-a retras cu el la o parte ?i îl întreba: Ce ai sa-mi veste?ti?
\par 20 Iar el a zis ca iudeii s-au în?eles sa te roage, ca mâine sa-l cobori pe Pavel la sinedriu, ca având sa cerceteze mai cu de-amanuntul despre el;
\par 21 Dar tu sa nu te încrezi în ei, caci dintre ei îl pândesc mai mul?i de patruzeci de barba?i, care s-au legat cu blestem sa nu manânce, nici sa bea pâna ce nu-l vor ucide; ?i acum ei sunt gata, a?teptând aprobarea ta.
\par 22 Deci comandantul a dat drumul tânarului, poruncindu-i: Nimanui sa nu spui ca mi-ai facut cunoscut acestea.
\par 23 ?i chemând la sine pe doi dintre suta?i, le-a zis: Pregati?i de la ceasul al treilea din noapte doua sute de osta?i, ?aptezeci de calare?i ?i doua sute de suli?a?i, ca sa mearga pâna la Cezareea.
\par 24 ?i sa fie animale (de calarie), ca punând pe Pavel sa-l duca teafar la Felix procuratorul.
\par 25 Scriind o scrisoare, având acest cuprins:
\par 26 Claudius Lysias, prea puternicului procurator, Felix, salutare!
\par 27 Pe acest barbat, prins de iudei ?i având sa fie ucis de ei, mergând eu cu oaste l-am scos, aflând ca este (ceta?ean) roman.
\par 28 ?i vrând sa ?tiu pricina pentru care îl pârau, l-am coborât la sinedriul lor.
\par 29 ?i am aflat ca este pârât pentru întrebari din legea lor, dar fara sa aiba vreo vina vrednica de moarte sau de lan?uri.
\par 30 ?i vestindu-mi-se ca va sa fie o cursa împotriva acestui barbat din partea iudeilor, îndata l-am trimis la tine, poruncind ?i pârâ?ilor sa spuna înaintea ta cele ce au asupra lui. Fii sanatos!
\par 31 Deci osta?ii, luând pe Pavel, precum li se poruncise, l-au adus noaptea la Antipatrida.
\par 32 Iar a doua zi, lasând pe calare?i sa mearga cu el, s-au întors la fortarea?a.
\par 33 ?i ei, intrând în Cezareea ?i dând procuratorului scrisoarea, i-au înfa?i?at ?i pe Pavel.
\par 34 ?i citind procuratorul ?i întrebând din ce provincie este el ?i aflând ca este din Cilicia,
\par 35 A zis: Te voi asculta când vor veni ?i pârâ?ii tai. ?i a poruncit sa fie pazit în pretoriul lui Irod.

\chapter{24}

\par 1 Iar dupa cinci zile s-a coborât arhiereul Anania cu câ?iva batrâni ?i cu un oarecare retor Tertul, care s-au înfa?i?at procuratorului împotriva lui Pavel.
\par 2 Iar dupa ce l-au chemat pe Pavel, Tertul a început sa-l învinuiasca, zicând: Prin tine dobândim multa pace ?i îndreptarile facute acestui neam, prin purtarea ta de grija,
\par 3 Totdeauna ?i pretutindeni le primim, prea puternice Felix, cu toata mul?umirea.
\par 4 Dar, ca sa nu te ostenesc mai mult, te rog sa ne ascul?i, pe scurt, cu bunavoin?a ta.
\par 5 Caci am aflat pe omul acesta ca o ciuma ?i urzitor de razvratiri printre to?i iudeii din lume, fiind capetenia eresului nazarinenilor,
\par 6 Care a încercat sa pângareasca ?i templul ?i pe care l-am prins ?i am voit sa-l judecam dupa legea noastra.
\par 7 Dar venind Lysias comandantul l-a scos cu de-a sila din mâinile noastre,
\par 8 Poruncind pârâ?ilor lui sa vina la tine. De la el vei putea, cercetând tu însu?i, sa cuno?ti toate învinuirile aduse de noi.
\par 9 Iar iudeii împreuna sus?ineau, zicând ca acestea a?a sunt.
\par 10 ?i, procuratorul facându-i semn sa vorbeasca, Pavel a raspuns: Fiindca ?tiu ca de mul?i ani e?ti judecator acestui neam, bucuros vorbesc pentru apararea mea.
\par 11 Tu po?i sa afli ca nu sunt mai mult decât douasprezece zile de când m-am suit la Ierusalim ca sa ma închin.
\par 12 ?i nici în templu nu m-au gasit discutând cu cineva sau facând tulburare în mul?ime, nici în sinagogi, nici în cetate,
\par 13 Nici nu pot sa-?i dovedeasca cele ce spun acum împotriva mea.
\par 14 ?i-?i marturisesc aceasta, ca a?a ma închin Dumnezeului parin?ilor mei, dupa înva?atura pe care ei o numesc eres, ?i cred toate cele scrise în Lege ?i în Prooroci,
\par 15 Având nadejde în Dumnezeu, pe care ?i ace?tia în?i?i o a?teapta, ca va sa fie învierea mor?ilor: ?i a drep?ilor ?i a nedrep?ilor.
\par 16 ?i întru aceasta ma straduiesc ?i eu ca sa am totdeauna înaintea lui Dumnezeu ?i a oamenilor un cuget neîntinat.
\par 17 Dupa mul?i ani, am venit ca sa aduc neamului meu milostenii ?i prinoase,
\par 18 Când ni?te iudei din Asia m-au gasit, cura?it, în templu, dar nu cu mul?ime, nici cu gâlceava.
\par 19 Aceia trebuia sa fie de fa?a înaintea ta ?i sa ma învinuiasca, daca aveau ceva împotriva mea;
\par 20 Sau chiar ace?tia sa spuna ce nedreptate mi-au gasit când am stat înaintea sinedriului,
\par 21 Decât numai pentru acest singur cuvânt pe care l-am strigat stând între ei, ca pentru învierea mor?ilor sunt eu astazi judecat între voi.
\par 22 ?i Felix, auzind acestea, i-a amânat, cunoscând destul de bine cele privitoare la înva?atura (cre?tina), zicând: Când se va coborî comandantul Lysias, voi hotarî asupra acelora ale voastre.
\par 23 ?i a poruncit suta?ului sa ?ina pe Pavel sub paza, dar sa-i lase tihna ?i sa nu opreasca pe nimeni dintre ai lui, ca sa vina sa-i slujeasca.
\par 24 Iar dupa câteva zile, Felix, venind cu Drusila, femeia lui, care era din neamul iudeilor, a trimis sa cheme pe Pavel ?i l-a ascultat despre credin?a în Hristos Iisus.
\par 25 ?i vorbind el despre dreptate ?i despre înfrânare ?i despre judecata viitoare, Felix s-a înfrico?at ?i a raspuns: Acum mergi, ?i când voi gasi timp potrivit te voi mai chema.
\par 26 În acela?i timp el nadajduia ca i se vor da bani de catre Pavel; de aceea, ?i mai des trimi?ând sa-l cheme, vorbea cu el.
\par 27 Dar când s-au împlinit doi ani, în locul lui Felix a urmat Porcius Festus. ?i voind sa le fie iudeilor pe plac, Felix a lasat pe Pavel legat.

\chapter{25}

\par 1 Deci Festus, trecând în ?inutul sau, dupa trei zile s-a suit de la Cezareea la Ierusalim.
\par 2 ?i arhiereii ?i frunta?ii iudeilor i s-au înfa?i?at cu învinuiri împotriva lui Pavel ?i îl rugau,
\par 3 Cerându-i ca o favoare asupra lui, sa fie trimis la Ierusalim, pregatind cursa ca sa-l ucida pe drum.
\par 4 Dar Festus a raspuns ca Pavel e pazit Cezareea ?i ca el însu?i avea sa plece în curând.
\par 5 Deci a zis el: Cei dintre voi care pot, sa se coboare cu mine, ?i daca este ceva rau în acest barbat, sa-l învinova?easca.
\par 6 ?i ramânând la ei nu mai mult de opt sau zece zile, s-a coborât în Cezareea, iar a doua zi, ?ezând la judecata, a poruncit sa fie adus Pavel.
\par 7 ?i venind el, iudeii coborâ?i din Ierusalim l-au înconjurat, aducând împotriva lui multe ?i grele învinuiri, pe care nu puteau sa le dovedeasca.
\par 8 Iar Pavel se apara: N-am gre?it cu nimic nici fa?a de legea iudeilor, nici fa?a de templu, nici fa?a de Cezarul.
\par 9 Iar Festus, voind sa faca placere iudeilor, raspunzând lui Pavel, a zis: Vrei sa mergi la Ierusalim ?i acolo sa fi judecat înaintea mea pentru acestea?
\par 10 Dar Pavel a zis: Stau la judecata Cezarului, unde trebuie sa fiu judecat. Iudeilor nu le-am facut nici un rau, precum mai bine ?tii ?i tu.
\par 11 Dar daca fac nedreptate ?i am savâr?it ceva vrednic de moarte, nu ma feresc de moarte; daca însa nu este nimic din cele de care ei ma învinuiesc - nimeni nu poate sa ma daruiasca lor. Cer sa fiu judecat de Cezarul.
\par 12 Atunci Festus, vorbind cu sfatul sau, a raspuns: Ai cerut sa fii judecat de Cezarul, la Cezarul te vei duce.
\par 13 ?i dupa ce au trecut câteva zile, regele Agripa ?i Berenice au sosit la Cezareea, ca sa salute pe Festus.
\par 14 ?i ramânând acolo mai multe zile, Festus a vorbit regelui despre Pavel, zicând: Este aici un barbat, lasat legat de Felix,
\par 15 În privin?a caruia, când am fost în Ierusalim, mi s-au înfa?i?at arhiereii ?i batrânii iudeilor, cerând osândirea lui.
\par 16 Eu le-am raspuns ca romanii n-au obiceiul sa dea pe vreun om la pierzare, înainte ca cel învinuit sa aiba de fa?a pe pârâ?ii lui ?i sa aiba putin?a sa se apere pentru vina sa.
\par 17 Adunându-se deci ei aici ?i nefacând eu nici o amânare, a doua zi am stat la judecata ?i am poruncit sa fie adus barbatul.
\par 18 Dar pârâ?ii care s-au ridicat împotriva lui nu i-au adus nici o învinuire dintre cele rele, pe care le banuiam eu,
\par 19 Ci aveau cu el ni?te neîn?elegeri cu privire la religia lor ?i la un oarecare Iisus mort, de Care Pavel zice ca traie?te.
\par 20 ?i nedumerindu-ma cu privire la cercetarea acestor lucruri, l-am întrebat daca voie?te sa mearga la Ierusalim ?i sa fie judecat acolo pentru acestea.
\par 21 Dar Pavel, cerând sa fie re?inut pentru judecata Cezarului, am poruncit sa fie ?inut pâna ce îl voi trimite la Cezarul.
\par 22 Iar Agripa a zis catre Festus: A? vrea sa aud ?i eu pe acest om. Iar el a zis: Mâine îl vei auzi.
\par 23 Deci a doua zi, Agripa ?i Berenice venind cu mare alai ?i intrând în sala de judecata împreuna cu tribunii ?i cu barba?ii cei mai de frunte ai ceta?ii, Festus a dat porunca sa fie adus Pavel.
\par 24 ?i a zis Festus: Rege Agripa, ?i voi to?i barba?ii care sunte?i cu noi de fa?a, vede?i pe acela pentru care toata mul?imea iudeilor a venit la mine, ?i în Ierusalim ?i aici, strigând ca el nu trebuie sa mai traiasca.
\par 25 Iar eu am în?eles ca n-a facut nimic vrednic de moarte; iar el însu?i cerând sa fie judecat de Cezarul, am hotarât sa-l trimit.
\par 26 Dar ceva sigur sa scriu stapânului despre el, nu am. De aceea l-am adus înaintea voastra ?i mai ales înaintea ta, rege Agripa, ca, dupa ce va fi cercetat, sa am ce sa scriu,
\par 27 Caci mi se pare nepotrivit sa-l trimit legat, fara sa arat învinuirile ce i se aduc.

\chapter{26}

\par 1 Agripa a zis catre Pavel: Î?i este îngaduit sa vorbe?ti pentru tine. Atunci Pavel, întinzând mâna, se apara:
\par 2 Ma socotesc fericit, o, rege Agripa, ca astazi, înaintea ta, pot sa ma apar de toate câte ma învinuiesc iudeii;
\par 3 Mai ales, pentru ca tu cuno?ti toate obiceiurile ?i neîn?elegerile iudeilor. De aceea te rog sa ma ascul?i cu îngaduin?a.
\par 4 Vie?uirea mea din tinere?e, cum a fost ea de la început în poporul meu ?i în Ierusalim, o ?tiu to?i iudeii.
\par 5 Daca vor sa dea marturie, ei ?tiu despre mine, de mult, ca am trait ca fariseu, în tagma cea mai riguroasa a religiei noastre.
\par 6 ?i acum stau la judecata pentru nadejdea fagaduin?ei facute de Dumnezeu catre parin?ii no?tri,
\par 7 ?i la care cele douasprezece semin?ii ale noastre, slujind lui Dumnezeu fara încetare, zi ?i noapte, nadajduiesc sa ajunga. Pentru nadejdea aceasta, o, rege Agripa, sunt pârât de iudei.
\par 8 De ce se socote?te la voi lucru de necrezut ca Dumnezeu înviaza pe cei mor?i?
\par 9 Eu unul am socotit, în sinea mea, ca fa?a de numele lui Iisus Nazarineanul trebuia sa fac multe împotriva;
\par 10 Ceea ce am ?i facut în Ierusalim, ?i pe mul?i dintre sfin?i i-am închis în temni?e cu puterea pe care o luasem de la arhierei. Iar când erau da?i la moarte, mi-am dat ?i eu încuviin?area.
\par 11 ?i îi pedepseam adesea prin toate sinagogile ?i-i sileam sa huleasca ?i, mult înfuriindu-ma  împotriva lor, îi urmaream pâna ?i prin ceta?ile de din afara;
\par 12 ?i în felul acesta, mergând ?i la Damasc, cu putere ?i cu însarcinare de la arhierei,
\par 13 Am vazut, o, rege, la amiaza, în calea mea, o lumina din cer, mai puternica decât stralucirea soarelui, stralucind împrejurul meu ?i a celor ce mergeau împreuna cu mine.
\par 14 ?i noi to?i cazând la pamânt, eu am auzit un glas care-mi zicea în limba evreiasca: Saule, Saule, de ce Ma prigone?ti? Greu î?i este sa love?ti în ?epu?a cu piciorul.
\par 15 Iar eu am zis: Cine e?ti Doamne? Iar Domnul a zis: Eu sunt Iisus, pe Care tu Îl prigone?ti.
\par 16 Dar, scoala-te ?i stai pe picioarele tale. Caci spre aceasta M-am aratat ?ie: ca sa te rânduiesc slujitor ?i martor, ?i al celor ce ai vazut, ?i al celor întru care Ma voi arata ?ie.
\par 17 Alegându-te pe tine din popor ?i din neamurile la care te trimit,
\par 18 Sa le deschizi ochii, ca sa se întoarca de la întuneric la lumina ?i de la stapânirea lui satana la Dumnezeu, ca sa ia iertarea pacatelor ?i parte cu cei ce s-au sfin?it, prin credin?a în Mine.
\par 19 Drept aceea, rege Agripa, n-am fost neascultator cere?tii aratari;
\par 20 Ci mai întâi celor din Damasc ?i din Ierusalim, ?i din toata ?ara Iudeii, ?i neamurilor le-am vestit sa se pocaiasca ?i sa se întoarca la Dumnezeu, facând lucruri vrednice de pocain?a.
\par 21 Pentru acestea, iudeii, prinzându-ma în templu, încercau sa ma ucida.
\par 22 Dobândind deci ajutorul de la Dumnezeu, am stat pâna în ziua aceasta, marturisind la mic ?i la mare, fara sa spun nimic decât ceea ce ?i proorocii ?i Moise au spus ca va sa fie:
\par 23 Ca Hristos avea sa patimeasca ?i sa fie cel dintâi înviat din mor?i ?i sa vesteasca lumina ?i poporului ?i neamurilor.
\par 24 ?i acestea graind el, întru apararea sa, i-a zis Festus cu glas mare: Pavele, e?ti nebun!  Înva?atura ta cea multa te duce la nebunie.
\par 25 Iar Pavel a zis: Nu sunt nebun, prea puternice Festus, ci graiesc cuvintele adevarului ?i ale în?elepciunii.
\par 26 Regele ?tie despre acestea, ?i în fa?a lui vorbesc fara sfiala, fiind încredin?at ca nimic nu i-a ramas ascuns, pentru ca aceasta nu s-a întâmplat, într-un ungher.
\par 27 Crezi tu, rege Agripa, în prooroci? ?tiu ca crezi.
\par 28 Iar Agripa a zis catre Pavel: Cu pu?in de nu ma îndupleci sa ma fac ?i eu cre?tin!
\par 29 Iar Pavel a zis: Ori cu pu?in, ori cu mult, eu m-a? ruga lui Dumnezeu ca nu numai tu, ci ?i to?i care ma asculta astazi sa fie a?a cum sunt ?i eu, afara de aceste lan?uri.
\par 30 ?i s-a ridicat ?i regele ?i guvernatorul ?i Berenice ?i cei care ?edeau împreuna cu ei,
\par 31 ?i plecând, vorbeau unii cu al?ii zicând: Omul acesta n-a facut nimic vrednic de moarte sau de lan?uri.
\par 32 Iar Agripa a zis lui Festus: Acest om putea sa fie lasat liber, daca n-ar fi cerut sa fie judecat de Cezarul.

\chapter{27}

\par 1 Iar dupa ce s-a hotarât sa plecam pe apa în Italia, au dat în primire pe Pavel ?i pe al?i câ?iva lega?i unui suta? cu numele Iuliu, din cohorta Augusta.
\par 2 ?i întorcându-se pe o corabie de la Adramit, care avea sa treaca prin locurile de pe coasta Asiei, am plecat; ?i era cu noi Aristarh, macedonean din Tesalonic.
\par 3 ?i a doua zi am ajuns la Sidon. Iuliu, purtându-se fa?a de Pavel cu omenie, i-a dat voie sa se duca la prieteni ca sa primeasca purtarea lor de grija.
\par 4 ?i plecând de acolo, am plutit pe lânga Cipru, pentru ca vânturile erau împotriva.
\par 5 ?i strabatând marea Ciliciei ?i a Pamfiliei, am sosit la Mira Liciei.
\par 6 ?i gasind suta?ul acolo o corabie din Alexandria plutind spre Italia, ne-a suit în ea.
\par 7 ?i multe zile plutind cu încetineala, abia am ajuns în dreptul Cnidului ?i, fiindca vântul nu ne slabea am plutit pe sub Creta, pe lânga Salmone.
\par 8 ?i abia trecând noi pe lânga ea, am ajuns într-un loc numit Limanuri Bune, de care era aproape ora?ul Lasea.
\par 9 ?i trecând multa vreme ?i plutirea fiind periculoasa, fiindca trecuse ?i postul (sarbatorii Ispa?irii, care se ?inea la evrei toamna), Pavel îi îndemna,
\par 10 Zicându-le: Barba?ilor, vad ca plutirea va sa fie cu necaz ?i cu multa paguba, nu numai pentru încarcatura ?i pentru corabie, ci ?i pentru sufletele noastre.
\par 11 Iar suta?ul se încredea mai mult în cârmaci ?i în stapânul corabiei decât în cele spuse de Pavel.
\par 12 ?i limanul nefiind bun de iernat, cei mai mul?i dintre ei au dat sfatul sa plecam de acolo ?i, daca s-ar putea, sa ajungem ?i sa iernam la Fenix, un port al Cretei, deschis spre vântul de miazazi-apus ?i spre vântul de miazanoapte-apus.
\par 13 ?i suflând u?or un vânt de miazazi ?i crezând ca sunt în stare sa-?i împlineasca gândul, ridicând ancora, pluteau cât mai aproape de Creta.
\par 14 ?i nu dupa multa vreme s-a pornit asupra ei un vânt puternic, numit Euroclidon (dinspre miazanoapte-rasarit).
\par 15 ?i smulgând corabia, iar ea neputând sa mearga împotriva vântului, ne-am lasat du?i în voia lui.
\par 16 ?i trecând pe lânga o insula mica, numita Clauda, cu greu am putut sa fim stapâni pe corabie.
\par 17 ?i dupa ce au ridicat-o, au folosit unelte ajutatoare, încingând corabia pe dedesubt. ?i temându-se sa nu cada în Sirta, au lasat pânzele jos ?i erau du?i a?a.
\par 18 ?i fiind tare lovi?i de furtuna, în ziua urmatoare au aruncat încarcatura.
\par 19 ?i a treia zi, cu mâinile lor, au aruncat uneltele corabiei.
\par 20 ?i nearatându-se nici soarele, nici stelele, timp de mai multe zile, ?i amenin?ând furtuna mare, ni se luase orice nadejde de scapare.
\par 21 ?i fiindca nu mâncasera de mult, Pavel, stând în mijlocul lor, le-a zis: Trebuia, o, barba?ilor, ca ascultându-ma pe mine, sa nu fi plecat din Creta; ?i n-a?i fi îndurat nici primejdia aceasta, nici paguba aceasta.
\par 22 Dar acum va îndemn sa ave?i voie buna, caci nici un suflet dintre voi nu va pieri, ci numai corabia.
\par 23 Caci mi-a aparut în noaptea aceasta un înger al Dumnezeului, al Caruia eu sunt ?i Caruia ma închin,
\par 24 Zicând: Nu te teme, Pavele. Tu trebuie sa stai înaintea Cezarului; ?i iata, Dumnezeu ?i-a daruit pe to?i cei ce sunt în corabie cu tine.
\par 25 De aceea, barba?ilor, ave?i curaj, caci am încredere în Dumnezeu, ca a?a va fi dupa cum mi s-a spus.
\par 26 ?i trebuie sa ajungem pe o insula.
\par 27 ?i când a fost a paisprezecea noapte de când eram purta?i încoace ?i încolo pe Adriatica, pe la miezul nop?ii corabierii au presim?it ca se apropie de un ?arm.
\par 28 ?i aruncând masuratoarea în jos au gasit douazeci de stânjeni ?i, trecând pu?in mai departe ?i masurând iara?i, au gasit cincisprezece stânjeni.
\par 29 ?i temându-se ca nu cumva sa nimerim pe locuri stâncoase, au aruncat patru ancore de la partea din urma a corabiei, ?i doreau sa se faca ziua.
\par 30 Dar corabierii cautau sa fuga din corabie ?i au coborât luntrea în mare, sub motiv ca vor sa întinda ?i ancorele de la partea dinainte.
\par 31 Pavel a spus suta?ului ?i osta?ilor: Daca ace?tia nu ramân în corabie, voi nu pute?i sa scapa?i.
\par 32 Atunci osta?ii au taiat funiile luntrei ?i au lasat-o sa cada.
\par 33 Iar, pâna sa se faca ziua, Pavel îi ruga pe to?i sa manânce, zicându-le: Paisprezece zile sunt azi de când n-a?i mâncat, a?teptând ?i nimic gustând.
\par 34 De aceea, va rog sa mânca?i, caci aceasta este spre scaparea voastra. Ca nici unuia din voi un fir de par din cap nu-i va pieri.
\par 35 ?i zicând acestea ?i luând pâine, a mul?umit lui Dumnezeu înaintea tuturor ?i, frângând, a început sa manânce.
\par 36 ?i devenind to?i voio?i, au luat ?i ei ?i au mâncat.
\par 37 ?i eram în corabie, de to?i, doua sute ?aptezeci ?i ?ase de suflete.
\par 38 ?i saturându-se de bucate, au u?urat corabia, aruncând grâul în mare.
\par 39 ?i când s-a facut ziua, ei n-au cunoscut pamântul, dar au zarit un sân de mare, având ?arm nisipos, în care voiau, daca ar putea, sa scoata corabia.
\par 40 ?i desfacând ancorele, le-au lasat în mare, slabind totodata funiile cârmelor ?i, ridicând pânza din frunte în bataia vântului, se îndreptau spre ?arm.
\par 41 ?i cazând pe un dâmb de nisip au în?epenit corabia ?i partea dinainte, înfigându-se, statea neclintita, iar partea dinapoi se sfarâma de puterea valurilor.
\par 42 Iar osta?ii au facut sfat sa omoare pe cei lega?i, ca sa nu scape vreunul, înotând.
\par 43 Dar suta?ul, voind sa fereasca pe Pavel, i-a împiedicat de la gândul lor ?i a poruncit ca aceia care pot sa înoate, aruncându-se cei dintâi, sa iasa la uscat;
\par 44 Iar ceilal?i, care pe scânduri, care pe câte ceva de la corabie. ?i a?a au ajuns cu to?ii sa scape, la uscat.

\chapter{28}

\par 1 ?i dupa ce am scapat, am aflat ca insula se nume?te Malta.
\par 2 Iar locuitorii ei ne aratau o deosebita omenie, caci, aprinzând foc, ne-au luat pe to?i la ei din pricina ploii care era ?i a frigului.
\par 3 ?i strângând Pavel gramada de gateje ?i punându-le în foc, o vipera a ie?it de caldura ?i s-a prins de mâna lui.
\par 4 ?i când locuitorii au vazut vipera atârnând de mâna lui, ziceau unii catre al?ii: Desigur ca uciga? este omul acesta, pe care dreptatea nu l-a lasat sa traiasca, de?i a scapat din mare.
\par 5 Deci el, scuturând vipera în foc, n-a patimit nici un rau.
\par 6 Iar ei a?teptau ca el sa se umfle, sau sa cada deodata mort. Dar a?teptând ei mult ?i vazând ca nu i se întâmpla nimic rau, ?i-au schimbat gândul ?i ziceau ca el este un zeu.
\par 7 ?i împrejurul acelui loc erau ?arinile capeteniei insulei, Publius, care, primindu-ne, ne-a gazduit prietenos trei zile.
\par 8 ?i s-a întâmplat ca tatal lui Publius zacea în pat, cuprins de friguri ?i de urdinare cu sânge, la care intrând Pavel ?i rugându-se, ?i-a pus mâinile peste el ?i l-a vindecat.
\par 9 ?i întâmplându-se aceasta, veneau la el ?i ceilal?i din insula care aveau boli ?i se vindecau;
\par 10 ?i ace?tia ne-au cinstit mult ?i, când am plecat, ne-au pus la îndemâna toate cele de trebuin?a.
\par 11 Dupa trei luni am pornit cu o corabie din Alexandria, care iernase în insula ?i care avea pe ea însemnul Dioscurilor.
\par 12 ?i ajungând la Siracuza, am ramas acolo trei zile.
\par 13 De unde, înconjurând, am sosit la Regium. ?i dupa o zi, suflând vânt de miazazi, am ajuns la Puteoli în cealalta zi.
\par 14 Gasind acolo fra?i, am fost ruga?i sa ramânem la ei ?apte zile. ?i a?a am venit la Roma.
\par 15 ?i de acolo, auzind fra?ii cele despre noi, au venit întru întâmpinarea noastra pâna la Forul lui Apius ?i la Trei Taverne, pe care, vazându-i, Pavel a mul?umit lui Dumnezeu ?i s-a îmbarbatat.
\par 16 Iar când am intrat în Roma, suta?ul a predat pe cei lega?i comandantului taberei, iar lui Pavel i s-a îngaduit sa locuiasca aparte cu osta?ul care îl pazea.
\par 17 ?i dupa trei zile Pavel a chemat la el pe cei care erau frunta?ii iudeilor. ?i, adunându-se, zicea catre ei: Barba?i fra?i, de?i eu n-am facut nimic rau împotriva poporului (nostru) sau a datinilor parinte?ti, am fost predat de la Ierusalim, în mâinile romanilor.
\par 18 Ace?tia, dupa ce m-au cercetat, voiau sa-mi dea drumul, fiindca nu era în mine nici o vina vrednica de moarte.
\par 19 Dar iudeii, împotrivindu-mi-se, am fost nevoit sa cer sa fiu judecat de Cezarul, dar nu ca a? avea de adus vreo pâra neamului meu.
\par 20 Deci pentru aceasta cauza v-am chemat sa va vad ?i sa vorbesc cu voi. Caci pentru nadejdea lui Israel ma aflu eu în acest lan?.
\par 21 Iar ei au zis catre el: Noi n-am primit din Iudeea nici scrisori despre tine, nici nu a venit cineva dintre fra?i, ca sa ne vesteasca sau sa ne vorbeasca ceva rau despre tine.
\par 22 Dar dorim sa auzim de la tine cele ce gânde?ti; caci despre eresul acesta ne este cunoscut; ca pretutindeni i se sta împotriva.
\par 23 Deci, rânduindu-i o zi, au venit la el, la gazda, mai mul?i. ?i de diminea?a pâna seara, el le vorbea, dând marturie despre împara?ia lui Dumnezeu, cautând sa-i încredin?eze despre Iisus din Legea lui Moise ?i din prooroci.
\par 24 ?i unii credeau celor spuse, iar al?ii nu credeau.
\par 25 ?i neîn?elegându-se unii cu al?ii, au plecat, zicând Pavel un cuvânt ca: Bine a vorbit Duhul Sfânt prin Isaia proorocul, catre parin?ii no?tri,
\par 26 Când a zis: "Mergi la poporul acesta ?i zi: Cu auzul ve?i auzi ?i nu ve?i în?elege ?i uitându-va ve?i privi, dar nu ve?i vedea.
\par 27 Caci inima acestui popor s-a învârto?at ?i cu urechile greu au auzit ?i ochii lor i-au închis. Ca nu cumva sa vada cu ochii ?i sa auda cu urechile ?i cu inima sa în?eleaga ?i sa se întoarca ?i Eu sa-l vindec".
\par 28 Deci cunoscut sa va fie voua ca aceasta mântuire a lui Dumnezeu s-a trimis pagânilor, ?i ei vor asculta.
\par 29 ?i dupa ce a zis el acestea, iudeii au plecat având între ei mare neîn?elegere.
\par 30 Iar Pavel a ramas doi ani întregi în casa luata de el cu chirie, ?i primea pe to?i care veneau la el,
\par 31 Propovaduind împara?ia lui Dumnezeu ?i înva?ând cele despre Domnul Iisus Hristos, cu toata îndrazneala ?i fara nici o piedica.


\end{document}