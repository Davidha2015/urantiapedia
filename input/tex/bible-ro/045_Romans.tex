\begin{document}

\title{Romans}

Rom 1:1  Pavel, rob al lui Iisus Hristos, chemat de El apostol, rânduit pentru vestirea Evangheliei lui Dumnezeu,
Rom 1:2  Pe care a fagaduit-o mai înainte, prin proorocii Sai, în Sfintele Scripturi,
Rom 1:3  Despre Fiul Sau, Cel nascut din samân?a lui David, dupa trup,
Rom 1:4  Care a fost rânduit Fiu al lui Dumnezeu întru putere, dupa Duhul sfin?eniei, prin învierea Lui din mor?i, Iisus Hristos, Domnul nostru,
Rom 1:5  Prin Care am primit har ?i apostolie, ca sa aduc, în numele Sau, la ascultarea credin?ei, toate neamurile,
Rom 1:6  Întru care sunte?i ?i voi chema?i ai lui Iisus Hristos:
Rom 1:7  Tuturor celor ce sunte?i în Roma, iubi?i de Dumnezeu, chema?i ?i sfin?i: har voua ?i pace de la Dumnezeu, Tatal nostru, ?i de la Domnul Iisus Hristos!
Rom 1:8  Mul?umesc, întâi Dumnezeului meu, prin Iisus Hristos, pentru voi to?i, fiindca credin?a voastra se veste?te în toata lumea.
Rom 1:9  Caci martor îmi este Dumnezeu, Caruia Îi slujesc cu duhul meu, întru Evanghelia Fiului Sau, ca neîncetat fac pomenire despre voi,
Rom 1:10  Cerând totdeauna în rugaciunile mele ca sa am cumva, prin voin?a Lui, vreodata, bun prilej ca sa vin la voi.
Rom 1:11  Pentru ca doresc mult sa va vad ca sa va împarta?esc vreun dar duhovnicesc, spre întarirea voastra.
Rom 1:12  ?i aceasta ca sa ma mângâi împreuna cu voi prin credin?a noastra laolalta, a voastra ?i a mea.
Rom 1:13  Fra?ilor, nu vreau ca voi sa nu ?ti?i ca, de multe ori, mi-am pus în gând sa vin la voi, dar am fost pâna acum împiedicat, ca sa am ?i între voi vreo roada, ca ?i la celelalte neamuri.
Rom 1:14  Dator sunt ?i elinilor ?i barbarilor ?i înva?a?ilor ?i neînva?a?ilor;
Rom 1:15  Astfel, cât despre mine, sunt bucuros sa va vestesc Evanghelia ?i voua, celor din Roma.
Rom 1:16  Caci nu ma ru?inez de Evanghelia lui Hristos, pentru ca este putere a lui Dumnezeu spre mântuirea a tot celui care crede, iudeului întâi, ?i elinului.
Rom 1:17  Caci dreptatea lui Dumnezeu se descopera în ea din credin?a spre credin?a, precum este scris: "Iar dreptul din credin?a va fi viu".
Rom 1:18  Caci mânia lui Dumnezeu se descopera din cer peste toata faradelegea ?i peste toata nedreptatea oamenilor care ?in nedreptatea drept adevar.
Rom 1:19  Pentru ca ceea ce se poate cunoa?te despre Dumnezeu este cunoscut de catre ei; fiindca Dumnezeu le-a aratat lor.
Rom 1:20  Cele nevazute ale Lui se vad de la facerea lumii, în?elegându-se din fapturi, adica ve?nica Lui putere ?i dumnezeire, a?a ca ei sa fie fara cuvânt de aparare,
Rom 1:21  Pentru ca, cunoscând pe Dumnezeu, nu L-au slavit ca pe Dumnezeu, nici nu I-au mul?umit, ci s-au ratacit în gândurile lor ?i inima lor cea nesocotita s-a întunecat.
Rom 1:22  Zicând ca sunt în?elep?i, au ajuns nebuni.
Rom 1:23  ?i au schimbat slava lui Dumnezeu Celui nestricacios cu asemanarea chipului omului celui stricacios ?i al pasarilor ?i al celor cu patru picioare ?i al târâtoarelor.
Rom 1:24  De aceea Dumnezeu i-a dat necura?iei, dupa poftele inimilor lor, ca sa-?i pângareasca trupurile lor între ei,
Rom 1:25  Ca unii care au schimbat adevarul lui Dumnezeu în minciuna ?i s-au închinat ?i au slujit fapturii, în locul Facatorului, Care este binecuvântat în veci, amin!
Rom 1:26  Pentru aceea, Dumnezeu i-a dat unor patimi de ocara, caci ?i femeile lor au schimbat fireasca rânduiala cu cea împotriva firii;
Rom 1:27  Asemenea ?i barba?ii lasând rânduiala cea dupa fire a par?ii femeie?ti, s-au aprins în pofta lor unii pentru al?ii, barba?i cu barba?i, savâr?ind ru?inea ?i luând cu ei rasplata cuvenita ratacirii lor.
Rom 1:28  ?i precum n-au încercat sa aiba pe Dumnezeu în cuno?tin?a, a?a ?i Dumnezeu i-a lasat la mintea lor fara judecata, sa faca cele ce nu se cuvine.
Rom 1:29  Plini fiind de toata nedreptatea, de desfrânare, de viclenie, de lacomie, de rautate; plini de pizma, de ucidere, de cearta, de în?elaciune, de purtari rele, bârfitori,
Rom 1:30  Graitori de rau, urâtori de Dumnezeu, ocarâtori, seme?i, trufa?i, laudaro?i, nascocitori de rele, nesupu?i parin?ilor,
Rom 1:31  Neîn?elep?i, calcatori de cuvânt, fara dragoste, fara mila;
Rom 1:32  Ace?tia, de?i au cunoscut dreapta orânduire a lui Dumnezeu, ca cei ce fac unele ca acestea sunt vrednici de moarte, nu numai ca fac ei acestea, ci le ?i încuviin?eaza celor care le fac.
Rom 2:1  Pentru aceea, oricine ai fi, o, omule, care judeci, e?ti fara cuvânt de raspuns, caci, în ceea ce judeci pe altul, pe tine însu?i te osânde?ti, caci acela?i lucruri faci ?i tu care judeci.
Rom 2:2  ?i noi ?tim ca judecata lui Dumnezeu este dupa adevar, fa?a de cei ce fac unele ca acestea.
Rom 2:3  ?i socote?ti tu, oare, omule, care judeci pe cei ce fac unele ca acestea, dar le faci ?i tu, ca tu vei scapa de judecata lui Dumnezeu?
Rom 2:4  Sau dispre?uie?ti tu boga?ia bunata?ii Lui ?i a îngaduin?ei ?i a îndelungii Lui rabdari, ne?tiind ca bunatatea lui Dumnezeu te îndeamna la pocain?a?
Rom 2:5  Dar dupa învârto?area ta ?i dupa inima ta nepocaita, î?i aduni mânie în ziua mâniei ?i a aratarii dreptei judeca?i a lui Dumnezeu,
Rom 2:6  Care va rasplati fiecaruia dupa faptele lui:
Rom 2:7  Via?a ve?nica celor ce, prin staruin?a în fapta buna, cauta marire, cinste ?i nestricaciune,
Rom 2:8  Iar iubitorilor de cearta, care nu se supun adevarului, ci se supun nedrepta?ii: mânie ?i furie.
Rom 2:9  Necaz ?i strâmtorare peste sufletul oricarui om care savâr?e?te raul, al iudeului mai întâi, ?i al elinului;
Rom 2:10  Dar marire, cinste ?i pace oricui face binele: iudeului mai întâi, ?i elinului.
Rom 2:11  Caci nu este partinire la Dumnezeu!
Rom 2:12  Câ?i, deci, fara lege, au pacatuit, fara lege vor ?i pieri; iar câ?i au pacatuit în lege, prin lege vor fi judeca?i.
Rom 2:13  Fiindca nu cei ce aud legea sunt drep?i la Dumnezeu, ci cei ce împlinesc legea vor fi îndrepta?i.
Rom 2:14  Caci, când pagânii care nu au lege, din fire fac ale legii, ace?tia, neavând lege, î?i sunt loru?i lege,
Rom 2:15  Ceea ce arata fapta legii scrisa în inimile lor, prin marturia con?tiin?ei lor ?i prin judeca?ile lor, care îi învinova?esc sau îi ?i apara,
Rom 2:16  În ziua în care Dumnezeu va judeca, prin Iisus Hristos, dupa Evanghelia mea, cele ascunse ale oamenilor.
Rom 2:17  Dar daca tu te nume?ti iudeu ?i te reazimi pe lege ?i te lauzi cu Dumnezeu,
Rom 2:18  ?i cuno?ti voia Lui ?i ?tii sa încuviin?ezi cele bune, fiind înva?at din lege,
Rom 2:19  ?i e?ti încredin?at ca tu e?ti calauza orbilor, lumina celor ce sunt în întuneric,
Rom 2:20  Pova?uitor celor fara de minte, înva?ator celor nevârstnici, având în lege dreptarul cuno?tiin?ei ?i al adevarului,
Rom 2:21  Deci tu, cel care înve?i pe altul, pe tine însu?i nu te înve?i? Tu cel care propovaduie?ti: Sa nu furi - ?i tu furi?
Rom 2:22  Tu, cel care zici: Sa nu savâr?e?ti adulter, savâr?e?ti adulter? Tu cel care ura?ti idolii, furi cele sfinte?
Rom 2:23  Tu, care te lauzi cu legea, Îl necinste?ti pe Dumnezeu, prin calcarea legii?
Rom 2:24  "Caci numele lui Dumnezeu, din pricina voastra, este hulit între neamuri", precum este scris.
Rom 2:25  Caci taierea împrejur folose?te, daca paze?ti legea; daca însa e?ti calcator de lege, taierea ta împrejur s-a facut netaiere împrejur.
Rom 2:26  Deci daca cel netaiat împrejur paze?te hotarârile legii, netaierea lui împrejur nu va fi, oare, socotita ca taiere împrejur?
Rom 2:27  Iar el - din fire netaiat împrejur, dar împlinitor al legii - nu te va judeca, oare, pe tine, care, prin litera legii ?i prin taierea împrejur, e?ti calcator de lege?
Rom 2:28  Pentru ca nu cel ce se arata pe din afara e iudeu, nici cea aratata pe dinafara în trup, este taiere împrejur;
Rom 2:29  Ci este iudeu cel întru ascuns, iar taierea împrejur este aceea a inimii, în duh, nu în litera; a carui lauda nu vine de la oameni, ci de la Dumnezeu.
Rom 3:1  Care este deci întâietatea iudeului ?i folosul taierii împrejur?
Rom 3:2  Este mare în toate privin?ele. Întâi, pentru ca lor li s-au încredin?at cuvintele lui Dumnezeu.
Rom 3:3  Caci ce este daca unii n-au crezut? Oare necredin?a lor va nimici credincio?ia lui Dumnezeu?
Rom 3:4  Nicidecum! Ci Dumnezeu se vade?te în adevarul Sau, pe când tot omul întru minciuna, precum este scris: "Drept e?ti Tu întru cuvintele Tale ?i biruitor când vei judeca Tu".
Rom 3:5  Iar daca nedreptatea noastra învedereaza dreptatea lui Dumnezeu, ce vom zice? Nu cumva este nedrept Dumnezeu care aduce mânia? - Ca om vorbesc.
Rom 3:6  Nicidecum! Caci atunci cum va judeca Dumnezeu lumea?
Rom 3:7  Caci daca adevarul lui Dumnezeu, prin minciuna mea, a prisosit spre slava Lui, pentru ce dar mai sunt ?i eu judecat ca pacatos?
Rom 3:8  ?i de ce n-am face cele rele, ca sa vina cele bune, precum suntem huli?i ?i precum spun unii ca zicem noi? Osânda aceasta este dreapta.
Rom 3:9  Dar ce? Avem noi vreo precadere? Nicidecum. Caci am învinuit mai înainte ?i pe iudei, ?i pe elini, ca to?i sunt sub pacat,
Rom 3:10  Dupa cum este scris: "Nu este drept nici unul;
Rom 3:11  Nu este cel ce în?elege, nu este cel ce cauta pe Dumnezeu.
Rom 3:12  To?i s-au abatut, împreuna, netrebnici s-au facut. Nu este cine sa faca binele, nici macar unul nu este.
Rom 3:13  Mormânt deschis este gâtlejul lor; viclenii vorbit-au cu limbile lor; venin de vipera este sub buzele lor;
Rom 3:14  Gura lor e plina de blestem ?i amaraciune;
Rom 3:15  Iu?i sunt picioarele lor sa verse sânge;
Rom 3:16  Pustiire ?i nenorocire sunt în drumurile lor;
Rom 3:17  ?i calea pacii ei nu au cunoscut-o;
Rom 3:18  Nu este frica de Dumnezeu înaintea ochilor lor".
Rom 3:19  Dar ?tim ca cele câte zice Legea le spune celor care sunt sub Lege, ca orice gura sa fie închisa ?i ca toata lumea sa fie vinovata înaintea lui Dumnezeu.
Rom 3:20  Pentru ca din faptele Legii nici un om nu se va îndrepta înaintea Lui, caci prin Lege vine cuno?tin?a pacatului.
Rom 3:21  Dar acum, în afara de Lege, s-a aratat dreptatea lui Dumnezeu, fiind marturisita de Lege ?i de prooroci.
Rom 3:22  Dar dreptatea lui Dumnezeu vine prin credin?a în Iisus Hristos, pentru to?i ?i peste to?i cei ce cred, caci nu este deosebire.
Rom 3:23  Fiindca to?i au pacatuit ?i sunt lipsi?i de slava lui Dumnezeu;
Rom 3:24  Îndreptându-se în dar cu harul Lui, prin rascumpararea cea în Hristos Iisus.
Rom 3:25  Pe Care Dumnezeu L-a rânduit (jertfa de) ispa?ire, prin credin?a în sângele Lui, ca sa-?i arate dreptatea Sa, pentru iertarea pacatelor celor mai înainte facute,
Rom 3:26  Întru îngaduin?a lui Dumnezeu - ca sa-?i arate dreptatea Sa, în vremea de acum, spre a fi El Însu?i drept, ?i îndreptând pe cel ce traie?te din credin?a în Iisus.
Rom 3:27  Deci, unde este pricina de lauda? A fost înlaturata. Prin care Lege? Prin Legea faptelor? Nu, ci prin Legea credin?ei.
Rom 3:28  Caci socotim ca prin credin?a se va îndrepta omul, fara faptele Legii.
Rom 3:29  Oare Dumnezeu este numai al iudeilor? Nu este El ?i Dumnezeul pagânilor? Da, ?i al pagânilor.
Rom 3:30  Fiindca este un singur Dumnezeu, Care va îndrepta din credin?a pe cei taia?i împrejur ?i, prin credin?a, pe cei netaia?i împrejur.
Rom 3:31  Desfiin?am deci noi Legea prin credin?a? Nicidecum! Dimpotriva, întarim Legea.
Rom 4:1  Deci, ce vom zice ca a dobândit dupa trup stramo?ul nostru Avraam?
Rom 4:2  Caci daca Avraam s-a îndreptat din fapte, are de ce sa se laude, dar nu înaintea lui Dumnezeu.
Rom 4:3  Caci, ce spune Scriptura? ?i "Avraam a crezut lui Dumnezeu ?i i s-a socotit lui ca dreptate".
Rom 4:4  Celui care face fapte, nu i se socote?te plata dupa har, ci dupa datorie;
Rom 4:5  Iar celui care nu face fapte, ci crede în Cel ce îndrepteaza pe cel pacatos, credin?a lui i se socote?te ca dreptate.
Rom 4:6  Precum ?i David vorbe?te despre fericirea omului caruia Dumnezeu îi socote?te dreptatea fara fapte:
Rom 4:7  "Ferici?i aceia, carora li s-au iertat faradelegile ?i ale caror pacate li s-au acoperit!
Rom 4:8  Fericit barbatul caruia Domnul nu-i va socoti pacatul".
Rom 4:9  Deci fericirea aceasta este ea numai pentru cei taia?i împrejur sau ?i pentru cei netaia?i împrejur? Caci zicem: "I s-a socotit lui Avraam credin?a ca dreptate".
Rom 4:10  Dar cum i s-a socotit? Când era taiat împrejur sau când era netaiat împrejur? Nu când era taiat împrejur, ci când era netaiat împrejur.
Rom 4:11  Iar semnul taierii împrejur l-a primit ca pecete a drepta?ii pentru credin?a lui din vremea netaierii împrejur, ca sa fie el parinte al tuturor celor ce cred, netaia?i împrejur, pentru a li se socoti ?i lor (credin?a) ca dreptate,
Rom 4:12  ?i parinte al celor taia?i împrejur. Dar nu numai al celor care sunt taia?i împrejur, ci ?i care umbla pe urmele credin?ei pe care o avea parintele nostru Avraam, pe când era netaiat împrejur.
Rom 4:13  Pentru ca Avraam ?i semin?ia lui nu prin lege au primit fagaduin?a ca vor mo?teni lumea, ci prin dreptatea cea din credin?a.
Rom 4:14  Caci daca mo?tenitorii sunt cei ce au legea, atunci credin?a a ajuns zadarnica, iar fagaduin?a s-a desfiin?at,
Rom 4:15  Caci legea pricinuie?te mâine; dar unde nu este lege, nu este nici calcare de lege.
Rom 4:16  De aceea (mo?tenirea fagaduita) este din credin?a, ca sa fie din har ?i ca fagaduin?a sa ramâna sigura pentru to?i urma?ii, nu numai pentru to?i cei ce se ?in de lege, ci ?i pentru cei ce se ?in de credin?a lui Avraam, care este parinte al nostru al tuturor,
Rom 4:17  Precum este scris: "Te-am pus parinte al multor neamuri", în fa?a Celui în Care a crezut, a lui Dumnezeu, Care înviaza mor?ii ?i cheama la fiin?a cele ce înca nu sunt;
Rom 4:18  Împotriva oricarei nadejdi, Avraam a crezut cu nadejde ca el va fi parinte al multor neamuri, dupa cum i s-a spus: "A?a va fi semin?ia ta";
Rom 4:19  ?i neslabind în credin?a, nu s-a uitat la trupul sau amor?it - caci era aproape de o suta de ani - ?i nici la amor?irea pântecelui Sarrei;
Rom 4:20  ?i nu s-a îndoit, prin necredin?a, de fagaduin?a lui Dumnezeu, ci s-a întarit în credin?a, dând slava lui Dumnezeu,
Rom 4:21  ?i fiind încredin?at ca ceea ce i-a fagaduit are putere sa ?i faca.
Rom 4:22  De acea, credin?a lui i s-a socotit ca dreptate.
Rom 4:23  ?i nu s-a scris numai pentru el ca i s-a socotit ca dreptate,
Rom 4:24  Ci se va socoti ?i pentru noi, cei care credem în Cel ce a înviat din mor?i pe Iisus, Domnul nostru,
Rom 4:25  Care S-a dat pentru pacatele noastre ?i a înviat pentru îndreptarea noastra.
Rom 5:1  Deci fiind îndrepta?i din credin?a, avem pace cu Dumnezeu, prin Domnul nostru Iisus Hristos,
Rom 5:2  Prin Care am avut ?i apropiere, prin credin?a, la harul acesta, în care stam, ?i ne laudam întru nadejdea slavei lui Dumnezeu.
Rom 5:3  ?i nu numai atât, ci ne laudam ?i în suferin?e, bine ?tiind ca suferin?a aduce rabdare,
Rom 5:4  ?i rabdarea încercare, ?i încercarea nadejde
Rom 5:5  Iar nadejdea nu ru?ineaza pentru ca iubirea lui Dumnezeu s-a varsat în inimile noastre, prin Duhul Sfânt, Cel daruit noua.
Rom 5:6  Caci Hristos, înca fiind noi neputincio?i, la timpul hotarât a murit pentru cei necredincio?i.
Rom 5:7  Caci cu greu va muri cineva pentru un drept; dar pentru cel bun poate se hotara?te cineva sa moara.
Rom 5:8  Dar Dumnezeu Î?i arata dragostea Lui fa?a de noi prin aceea ca, pentru noi, Hristos a murit când noi eram înca pacato?i.
Rom 5:9  Cu atât mai vârtos, deci, acum, fiind îndrepta?i prin sângele Lui, ne vom izbavi prin El de mânie.
Rom 5:10  Caci daca, pe când eram vrajma?i, ne-am împacat cu Dumnezeu, prin moartea Fiului Sau, cu atât mai mult, împaca?i fiind, ne vom mântui prin via?a Lui.
Rom 5:11  ?i nu numai atât, ci ?i ne laudam în Dumnezeu prin Domnul nostru Iisus Hristos, prin Care am primit acum împacarea.
Rom 5:12  De aceea, precum printr-un om a intrat pacatul în lume ?i prin pacat moartea, a?a ?i moartea a trecut la to?i oamenii, pentru ca to?i au pacatuit în el.
Rom 5:13  Caci, pâna la lege, pacatul era în lume, dar pacatul nu se socote?te când nu este lege.
Rom 5:14  Ci a împara?it moartea de la Adam pâna la Moise ?i peste cei ce nu pacatuisera, dupa asemanarea gre?elii lui Adam, care este chip al Celui ce avea sa vina.
Rom 5:15  Dar nu este cu gre?eala cum este cu harul, caci daca prin gre?eala unuia cei mul?i au murit, cu mult mai mult harul lui Dumnezeu ?i darul Lui au prisosit asupra celor mul?i, prin harul unui singur om, Iisus Hristos.
Rom 5:16  ?i ce aduce darul nu seamana cu ce a adus acel unul care a pacatuit; caci judecata dintr-unul duce la osândire, iar harul din multe gre?eli duce la îndreptare.
Rom 5:17  Caci, daca prin gre?eala unuia moartea a împara?it printr-unul, cu mult mai mult cei ce primesc prisosin?a harului ?i a darului drepta?ii vor împara?i în via?a prin Unul Iisus Hristos.
Rom 5:18  A?adar, precum prin gre?eala unuia a venit osânda pentru to?i oamenii, a?a ?i prin îndreptarea adusa de Unul a venit, pentru to?i oamenii, îndreptarea care da via?a;
Rom 5:19  Caci precum prin neascultarea unui om s-au facut pacato?i cei mul?i, tot a?a prin ascultarea unuia se vor face drep?i cei mul?i.
Rom 5:20  Iar Legea a intrat ?i ea ca se înmul?easca gre?eala; iar unde s-a înmul?it pacatul, a prisosit harul;
Rom 5:21  Pentru ca precum a împara?it pacatul prin moarte, a?a ?i harul sa împara?easca prin dreptate, spre via?a ve?nica, prin Iisus Hristos, Domnul nostru.
Rom 6:1  Ce vom zice deci? Ramâne-vom, oare, în pacat, ca sa se înmul?easca harul?
Rom 6:2  Nicidecum! Noi care am murit pacatului, cum vom mai trai în pacat?
Rom 6:3  Au nu ?ti?i ca to?i câ?i în Hristos Iisus ne-am botezat, întru moartea Lui ne-am botezat?
Rom 6:4  Deci ne-am îngropat cu El, în moarte, prin botez, pentru ca, precum Hristos a înviat din mor?i, prin slava Tatalui, a?a sa umblam ?i noi întru înnoirea vie?ii;
Rom 6:5  Caci daca am fost altoi?i pe El prin asemanarea mor?ii Lui, atunci vom fi parta?i ?i ai învierii Lui,
Rom 6:6  Cunoscând aceasta, ca omul nostru cel vechi a fost rastignit împreuna cu El, ca sa se nimiceasca trupul pacatului, pentru a nu mai fi robi ai pacatului.
Rom 6:7  Caci Cel care a murit a fost cura?it de pacat.
Rom 6:8  Iar daca am murit împreuna cu Hristos, credem ca vom ?i vie?ui împreuna cu El,
Rom 6:9  ?tiind ca Hristos, înviat din mor?i, nu mai moare. Moarta nu mai are stapânire asupra Lui.
Rom 6:10  Caci ce a murit, a murit pacatului o data pentru totdeauna, iar ce traie?te, traie?te lui Dumnezeu.
Rom 6:11  A?a ?i voi, socoti?i-va ca sunte?i mor?i pacatului, dar vii pentru Dumnezeu, în Hristos Iisus, Domnul nostru.
Rom 6:12  Deci sa nu împara?easca pacatul în trupul vostru cel muritor, ca sa va supune?i poftelor lui;
Rom 6:13  Nici sa nu pune?i madularele voastre ca arme ale nedrepta?ii în slujba pacatului, ci, înfa?i?a?i-va pe voi lui Dumnezeu, ca vii, scula?i din mor?i, ?i madularele voastre ca arme ale drepta?ii lui Dumnezeu.
Rom 6:14  Caci pacatul nu va avea stapânire asupra voastra, fiindca nu sunte?i sub lege, ci sub har.
Rom 6:15  Oare, atunci sa pacatuim fiindca nu suntem sub lege, ci sub har? Nicidecum!
Rom 6:16  Au nu ?ti?i ca celui ce va da?i spre ascultare robi, sunte?i robi aceluia caruia va supune?i: fie ai pacatului spre moarte, fie ai ascultarii spre dreptate?
Rom 6:17  Mul?umim însa lui Dumnezeu, ca (de?i) era?i robi ai pacatului, v-a?i supus din toata inima dreptarului înva?aturii careia a?i fost încredin?a?i,
Rom 6:18  ?i izbavindu-va de pacat, v-a?i facut robi ai drepta?ii.
Rom 6:19  Omene?te vorbesc, pentru slabiciunea trupului vostru. - Caci precum a?i facut madularele voastre roabe necura?iei ?i faradelegii, spre faradelege, tot a?a face?i acum madularele voastre roabe drepta?ii, spre sfin?ire.
Rom 6:20  Caci atunci, când era?i robi ai pacatului, era?i liberi fa?a de dreptate.
Rom 6:21  Deci ce roada avea?i atunci? Roade de care acum va e ru?ine; pentru ca sfâr?itul acelora este moartea.
Rom 6:22  Dar acum, izbavi?i fiind de pacat ?i robi facându-va lui Dumnezeu, ave?i roada voastra spre sfin?ire, iar sfâr?itul, via?a ve?nica.
Rom 6:23  Pentru ca plata pacatului este moartea, iar harul lui Dumnezeu, via?a ve?nica, în Hristos Iisus, Domnul nostru.
Rom 7:1  Oare nu ?ti?i, fra?ilor - caci celor ce cunosc Legea vorbesc - ca Legea are putere asupra omului, atâta timp cât el traie?te?
Rom 7:2  Caci femeia maritata e legata, prin lege, de barbatul sau atâta timp cât el traie?te; iar daca i-a murit barbatul, este dezlegata de legea barbatului.
Rom 7:3  Deci, traindu-i barbatul, se va numi adultera daca va fi cu alt barbat; iar daca i-a murit barbatul este libera fa?a de lege, ca sa nu fie adultera, luând un alt barbat.
Rom 7:4  A?a ca, fra?ii mei, ?i voi a?i murit Legii, prin trupul lui Hristos, spre a fi ai altuia, ai Celui ce a înviat din mor?i, ca sa aducem roade lui Dumnezeu.
Rom 7:5  Caci pe când eram în trup, patimile pacatelor, care erau prin Lege, lucrau în madularele noastre, ca sa aducem roade mor?ii;
Rom 7:6  Dar acum ne-am desfacut de Lege, murind aceluia în care eram ?inu?i robi, ca noi sa slujim întru înnoirea Duhului, iar nu dupa slova cea veche.
Rom 7:7  Ce vom zice deci? Au doara Legea este pacat? Nicidecum. Dar eu n-am cunoscut pacatul,  decât prin Lege. Caci n-a? fi ?tiut pofta, daca Legea n-ar fi zis: Sa nu pofte?ti!
Rom 7:8  Dar pacatul, luând pricina prin porunca, a lucrat în mine tot felul de pofte. Caci fara lege, pacatul era mort.
Rom 7:9  Iar eu cândva traiam fara lege, dar dupa ce a venit porunca, pacatul a prins via?a;
Rom 7:10  Iar eu am murit! ?i porunca, data spre via?a, mi s-a aflat a fi spre moarte.
Rom 7:11  Pentru ca pacatul, luând îndemn prin porunca, m-a în?elat ?i m-a ucis prin ea.
Rom 7:12  Deci, Legea e sfânta ?i porunca e sfânta ?i dreapta ?i buna.
Rom 7:13  Atunci, ce era bun s-a facut pentru mine pricina mor?ii? Nicidecum! Ci pacatul, ca sa se arate pacat, mi-a adus moartea, prin ceea ce a fost bun, pentru ca pacatul, prin porunca, sa fie peste masura de pacatos.
Rom 7:14  Caci ?tim ca Legea e duhovniceasca; dar eu sunt trupesc, vândut sub pacat.
Rom 7:15  Pentru ca ceea ce fac nu ?tiu; caci nu savâr?esc ceea ce voiesc, ci fac ceea ce urasc.
Rom 7:16  Iar daca fac ceea ce nu voiesc, recunosc ca Legea este buna.
Rom 7:17  Dar acum nu eu fac acestea, ci pacatul care locuie?te în mine.
Rom 7:18  Fiindca ?tiu ca nu locuie?te în mine, adica în trupul meu, ce este bun. Caci a voi se afla în mine, dar a face binele nu aflu;
Rom 7:19  Caci nu fac binele pe care îl voiesc, ci raul pe care nu-l voiesc, pe acela îl savâr?esc.
Rom 7:20  Iar daca fac ceea ce nu voiesc eu, nu eu fac aceasta, ci pacatul care locuie?te în mine.
Rom 7:21  Gasesc deci în mine, care voiesc sa fac bine, legea ca raul este legat de mine.
Rom 7:22  Ca, dupa omul cel launtric, ma bucur de legea lui Dumnezeu;
Rom 7:23  Dar vad în madularele mele o alta lege, luptându-se împotriva legii min?ii mele ?i facându-ma rob legii pacatului, care este în madularele mele.
Rom 7:24  Om nenorocit ce sunt! Cine ma va izbavi de trupul mor?ii acesteia?
Rom 7:25  Mul?umesc lui Dumnezeu, prin Iisus Hristos, Domnul nostru! Deci, dar, eu însumi, cu mintea mea, slujesc legii lui Dumnezeu, iar cu trupul, legii pacatului.
Rom 8:1  Drept aceea nici o osânda nu este acum asupra celor ce sunt în Hristos Iisus.
Rom 8:2  Caci legea duhului vie?ii în Hristos Iisus m-a eliberat de legea pacatului ?i a mor?ii,
Rom 8:3  Pentru ca ceea ce era cu neputin?a Legii - fiind slaba prin trup - a savâr?it Dumnezeu, trimi?ând pe Fiul Sau întru asemanarea trupului pacatului ?i pentru pacat a osândit pacatul în trup,
Rom 8:4  Pentru ca îndreptarea din Lege sa se împlineasca în noi, care nu umblam dupa trup, ci dupa duh.
Rom 8:5  Caci cei ce sunt dupa trup cugeta cele ale trupului, iar cei ce sunt dupa Duh, cele ale Duhului.
Rom 8:6  Caci dorin?a carnii este moarte dar dorin?a Duhului este via?a ?i pace;
Rom 8:7  Fiindca dorin?a carnii este vrajma?ie împotriva lui Dumnezeu, caci nu se supune legii lui Dumnezeu, ca nici nu poate.
Rom 8:8  Iar cei ce sunt în carne nu pot sa placa lui Dumnezeu.
Rom 8:9  Dar voi nu sunte?i în carne, ci în Duh, daca Duhul lui Dumnezeu locuie?te în voi. Iar daca cineva nu are Duhul lui Hristos, acela nu este al Lui.
Rom 8:10  Iar daca Hristos este în voi, trupul este mort pentru pacat; iar Duhul, via?a pentru dreptate,
Rom 8:11  Iar daca Duhul Celui ce a înviat pe Iisus din mor?i locuie?te în voi, Cel ce a înviat pe Hristos Iisus din mor?i va face vii ?i trupurile voastre cele muritoare, prin Duhul Sau care locuie?te în voi.
Rom 8:12  Drept aceea, fra?ilor, nu suntem datori trupului, ca sa vie?uim dupa trup.
Rom 8:13  Caci daca vie?ui?i dupa trup, ve?i muri, iar daca ucide?i, cu Duhul, faptele trupului, ve?i fi vii.
Rom 8:14  Caci câ?i sunt mâna?i de Duhul lui Dumnezeu sunt fii ai lui Dumnezeu.
Rom 8:15  Pentru ca n-a?i primit iara?i un duh al robiei, spre temere, ci a?i primit Duhul înfierii, prin care strigam: Avva! Parinte!
Rom 8:16  Duhul însu?i marturise?te împreuna cu duhul nostru ca suntem fii ai lui Dumnezeu.
Rom 8:17  ?i daca suntem fii, suntem ?i mo?tenitori - mo?tenitori ai lui Dumnezeu ?i împreuna-mo?tenitori cu Hristos, daca patimim împreuna cu El, ca împreuna cu El sa ne ?i preamarim.
Rom 8:18  Caci socotesc ca patimirile vremii de acum nu sunt vrednice de marirea care ni se va descoperi.
Rom 8:19  Pentru ca faptura a?teapta cu nerabdare descoperirea fiilor lui Dumnezeu.
Rom 8:20  Caci faptura a fost supusa de?ertaciunii - nu din voia ei, ci din cauza aceluia care a supus-o - cu nadejde,
Rom 8:21  Pentru ca ?i faptura însa?i se va izbavi din robia stricaciunii, ca sa fie parta?a la libertatea maririi fiilor lui Dumnezeu.
Rom 8:22  Caci ?tim ca toata faptura împreuna suspina ?i împreuna are dureri pâna acum.
Rom 8:23  ?i nu numai atât, ci ?i noi, care avem pârga Duhului, ?i noi în?ine suspinam în noi, a?teptând înfierea, rascumpararea trupului nostru.
Rom 8:24  Caci prin nadejde ne-am mântuit; dar nadejdea care se vede nu mai e nadejde. Cum ar nadajdui cineva ceea ce vede?
Rom 8:25  Iar daca nadajduim ceea ce nu vedem, a?teptam prin rabdare.
Rom 8:26  De asemenea ?i Duhul vine în ajutor slabiciunii noastre, caci noi nu ?tim sa ne rugam cum trebuie, ci Însu?i Duhul Se roaga pentru noi cu suspine negraite.
Rom 8:27  Iar Cel ce cerceteaza inimile ?tie care este dorin?a Duhului, caci dupa Dumnezeu El Se roaga pentru sfin?i.
Rom 8:28  ?i ?tim ca Dumnezeu toate le lucreaza spre binele celor ce iubesc pe Dumnezeu, al celor care sunt chema?i dupa voia Lui;
Rom 8:29  Caci pe cei pe care i-a cunoscut mai înainte, mai înainte i-a ?i hotarât sa fie asemenea chipului Fiului Sau, ca El sa fie întâi nascut între mul?i fra?i.
Rom 8:30  Iar pe care i-a hotarât mai înainte, pe ace?tia i-a ?i chemat; ?i pe care i-a chemat, pe ace?tia i-a ?i îndreptat; iar pe care i-a îndreptat, pe ace?tia i-a ?i marit.
Rom 8:31  Ce vom zice deci la acestea? Daca Dumnezeu e pentru noi, cine este împotriva noastra?
Rom 8:32  El, Care pe Însu?i Fiul Sau nu L-a cru?at, ci L-a dat mor?ii, pentru noi to?i, cum nu ne va da, oare, toate împreuna cu El?
Rom 8:33  Cine va ridica pâra împotriva ale?ilor lui Dumnezeu? Dumnezeu este Cel ce îndrepteaza;
Rom 8:34  Cine este Cel ce osânde?te? Hristos, Cel ce a murit, ?i mai ales Cel ce a înviat, Care ?i este de-a dreapta lui Dumnezeu, Care mijloce?te pentru noi!
Rom 8:35  Cine ne va despar?i pe noi de iubirea lui Hristos? Necazul, sau strâmtorarea, sau prigoana, sau foametea, sau lipsa de îmbracaminte, sau primejdia, sau sabia?
Rom 8:36  Precum este scris: "Pentru Tine suntem omorâ?i toata ziua, socoti?i am fost ca ni?te oi de junghiere".
Rom 8:37  Dar în toate acestea suntem mai mult decât biruitori, prin Acela Care ne-a iubit.
Rom 8:38  Caci sunt încredin?at ca nici moartea, nici via?a, nici îngerii, nici stapânirile, nici cele de acum, nici cele ce vor fi, nici puterile,
Rom 8:39  Nici înal?imea, nici adâncul ?i nici o alta faptura nu va putea sa ne desparta pe noi de dragostea lui Dumnezeu, cea întru Hristos Iisus, Domnul nostru.
Rom 9:1  Spun adevarul în Hristos, nu mint, martor fiindu-mi con?tiin?a mea în Duhul Sfânt,
Rom 9:2  Ca mare îmi este întristarea ?i necurmata durerea inimii.
Rom 9:3  Caci a? fi dorit sa fiu eu însumi anatema de la Hristos pentru fra?ii mei, cei de un neam cu mine, dupa trup,
Rom 9:4  Care sunt israeli?i, ale carora sunt înfierea ?i slava ?i legamintele ?i Legea ?i închinarea ?i fagaduin?ele,
Rom 9:5  Ai carora sunt parin?ii ?i din care dupa trup este Hristos, Cel ce este peste toate Dumnezeu, binecuvântat în veci. Amin!
Rom 9:6  Dar nu a?a ca ar fi cazut cuvântul lui Dumnezeu: caci nu to?i cei din Israel sunt ?i israeli?i;
Rom 9:7  Nici pentru ca sunt urma?ii lui Avraam, sunt to?i fii, ci "întru Isaac, a zis, se vor chema ?ie urma?i",
Rom 9:8  Adica: Nu copiii trupului sunt copii ai lui Dumnezeu, ci fiii fagaduin?ei se socotesc urma?i.
Rom 9:9  Caci al fagaduin?ei este cuvântul acesta: "(La anul) pe vremea aceasta voi veni ?i Sara va avea un fiu".
Rom 9:10  Dar nu numai ea, ci ?i Rebeca, având copii gemeni dintr-unul, Isaac, parintele nostru;
Rom 9:11  ?i nefiind ei înca nascu?i ?i nefacând ei ceva bun sau rau, ca sa ramâna voia lui Dumnezeu cea dupa alegere, nu din fapte, ci de la Cel care cheama,
Rom 9:12  I s-a zis ei ca "cel mai mare va sluji celui mai mic",
Rom 9:13  Precum este scris: "Pe Iacov l-am iubit, iar pe Isav l-am urât".
Rom 9:14  Ce vom zice dar? Nu cumva la Dumnezeu este nedreptate? Nicidecum!
Rom 9:15  Caci graie?te catre Moise: "Voi milui pe cine vreau sa-l miluiesc ?i Ma voi îndura de cine vreau sa Ma îndur".
Rom 9:16  Deci, dar, nu este nici de la cel care voie?te, nici de la cel ce alearga, ci de la Dumnezeu care miluie?te.
Rom 9:17  Caci Scriptura zice lui Faraon: "Pentru aceasta chiar te-am ridicat, ca sa arat în tine puterea Mea ?i ca numele Meu sa se vesteasca în tot pamântul".
Rom 9:18  Deci, dar, Dumnezeu pe cine voie?te îl miluie?te, iar pe cine voie?te îl împietre?te.
Rom 9:19  Îmi vei zice deci: De ce mai dojene?te? Caci voin?ei Lui cine i-a stat împotriva?
Rom 9:20  Dar, omule, tu cine e?ti care raspunzi împotriva lui Dumnezeu? Oare faptura va zice Celui ce a facut-o: De ce m-ai facut a?a?
Rom 9:21  Sau nu are olarul putere peste lutul lui, ca din aceea?i framântatura sa faca un vas de cinste, iar altul de necinste?
Rom 9:22  ?i ce este daca Dumnezeu, voind sa-?i arate mânia ?i sa faca cunoscuta puterea Sa, a suferit cu multa rabdare vasele mâniei Sale, gatite spre pierire,
Rom 9:23  ?i ca sa faca cunoscuta boga?ia slavei Sale catre vasele milei, pe care mai dinainte le-a gatit spre slava?
Rom 9:24  Adica pe noi, pe care ne-a ?i chemat, nu numai dintre iudei, ci ?i dintre pagâni,
Rom 9:25  Precum zice El ?i la Osea: "Chema-voi poporul Meu pe cel ce nu este poporul Meu, ?i iubita pe cea care nu era iubita;
Rom 9:26  ?i va fi în locul unde li s-a zis lor: Nu voi sunte?i poporul Meu - acolo se vor chema fii ai Dumnezeului Celui viu".
Rom 9:27  Iar Isaia striga pentru Israel : "Daca numarul fiilor lui Israel ar fi ca nisipul marii, rama?i?a se va mântui.
Rom 9:28  Pentru ca împlinind ?i scurtând, Domnul va îndeplini, pe pamânt, cuvântul Sau".
Rom 9:29  ?i precum a proorocit Isaia: "Daca Domnul Savaot nu ne-ar fi lasat noua urma?i, am fi ajuns ca Sodoma ?i ne-am fi asemanat cu Gomora".
Rom 9:30  Ce vom zice, deci? Ca neamurile care nu cautau dreptatea au dobândit dreptatea, însa dreptatea din credin?a;
Rom 9:31  Iar Israel, urmarind legea drepta?ii, n-a ajuns la legea drepta?ii.
Rom 9:32  Pentru ce? Pentru ca nu o cautau din credin?a, ci ca din faptele Legii. S-au poticnit de piatra poticnirii,
Rom 9:33  Precum este scris: "Iata pun în Sion piatra de poticnire ?i piatra de sminteala; ?i tot cel ce crede în El nu se va ru?ina".
Rom 10:1  Fra?ilor, bunavoin?a inimii mele ?i rugaciunea mea catre Dumnezeu, pentru Israel, este spre mântuire.
Rom 10:2  Caci le marturisesc ca au râvna pentru Dumnezeu, dar sunt fara cuno?tin?a.
Rom 10:3  Deoarece, necunoscând dreptatea lui Dumnezeu ?i cautând sa statorniceasca dreptatea lor, drepta?ii lui Dumnezeu ei nu s-au supus.
Rom 10:4  Caci sfâr?itul Legii este Hristos, spre dreptate tot celui ce crede.
Rom 10:5  Caci Moise scrie despre dreptatea care vine din lege, ca: "Omul care o va îndeplini va trai prin ea".
Rom 10:6  Iar dreptatea din credin?a graie?te a?a: "Sa nu zici în inima ta: Cine se va sui la cer?", ca adica sa coboare pe Hristos!
Rom 10:7  Sau: "Cine se va coborî întru adânc?", ca sa ridice pe Hristos din mor?i!
Rom 10:8  Dar ce zice Scriptura? "Aproape este de tine cuvântul, în gura ta ?i în inima ta", - adica cuvântul credin?ei pe care-l propovaduim.
Rom 10:9  Ca de vei marturisi cu gura ta ca Iisus este Domnul ?i vei crede în inima ta ca Dumnezeu L-a înviat pe El din mor?i, te vei mântui.
Rom 10:10  Caci cu inima se crede spre dreptate, iar cu gura se marturise?te spre mântuire.
Rom 10:11  Caci zice Scriptura: "Tot cel ce crede în El nu se va ru?ina".
Rom 10:12  Caci nu este deosebire între iudeu ?i elin, pentru ca Acela?i este Domnul tuturor, Care îmboga?e?te pe to?i cei ce-L cheama pe El.
Rom 10:13  Caci: "Oricine va chema numele Domnului se va mântui".
Rom 10:14  Dar cum vor chema numele Aceluia în Care înca n-au crezut? ?i cum vor crede în Acela de Care n-au auzit? ?i cum vor auzi, fara propovaduitor?
Rom 10:15  ?i cum vor propovadui, de nu vor fi trimi?i? Precum este scris: "Cât de frumoase sunt picioarele celor ce vestesc pacea, ale celor ce vestesc cele bune!"
Rom 10:16  Dar nu to?i s-au supus Evangheliei, caci Isaia zice: "Doamne, cine a crezut celor auzite de la noi?"
Rom 10:17  Prin urmare, credin?a este din auzire, iar auzirea prin cuvântul lui Hristos.
Rom 10:18  Dar întreb: Oare n-au auzit? Dimpotriva: "În tot pamântul a ie?it vestirea lor ?i la marginile lumii cuvintele lor".
Rom 10:19  Dar zic: Nu cumva Israel n-a în?eles? Moise spune cel dintâi: "Voi întarâta râvna voastra prin cel ce nu este poporul (Meu) ?i voi a?â?a mânia voastra cu un popor nepriceput".
Rom 10:20  Isaia îndrazne?te ?i zice: "Am fost aflat de cei ce nu Ma cautau ?i M-am facut aratat celor ce nu întrebau de Mine".
Rom 10:21  Dar catre Israel zice: "Toata ziua întins-am mâinile Mele catre un popor neascultator ?i împotriva graitor".
Rom 11:1  Întreb deci: Oare lepadat-a Dumnezeu pe poporul Sau? Nicidecum! Caci ?i eu sunt israelit, din urma?ii lui Avraam, din semin?ia lui Veniamin.
Rom 11:2  Nu a lepadat Dumnezeu pe poporul Sau, pe care mai înainte l-a cunoscut. Nu ?ti?i, oare, ce zice Scriptura despre Ilie? Cum se roaga el împotriva lui Israel, zicând:
Rom 11:3  "Doamne, pe proorocii Tai i-au omorât, jertfelnicele Tale le-au surpat ?i eu am ramas singur ?i ei cauta sa-mi ia sufletul!".
Rom 11:4  Dar ce-i spune dumnezeiescul raspuns? "Mi-am pus deoparte ?apte mii de barba?i, care nu ?i-au plecat genunchiul înaintea lui Baal".
Rom 11:5  Deci tot a?a ?i în vremea de acum este o rama?i?a aleasa prin har.
Rom 11:6  Iar daca este prin har, nu mai este din fapte; altfel harul nu mai este har. Iar daca este din fapte, nu mai este har, altfel fapta nu mai este fapta.
Rom 11:7  Ce este deci? Nu tot Israelul a dobândit ceea ce cauta; ci cei ale?i au dobândit, iar ceilal?i s-au împietrit,
Rom 11:8  Precum este scris: "Dumnezeu le-a dat duh de amor?ire, ochi ca sa nu vada ?i urechi ca sa nu auda pâna în ziua de azi".
Rom 11:9  Iar David zice: "Faca-se masa lor cursa ?i la? ?i sminteala ?i rasplatire lor!
Rom 11:10  Întunce-se ochii lor ca sa nu vada ?i spinarea lor încovoaie-o pentru totdeauna!"
Rom 11:11  Deci, întreb: S-a poticnit, oare, ca sa cada? Nicidecum! ?i prin caderea lor, neamurilor le-a venit mântuirea, ca Israel sa-?i întarâte râvna fa?a de ele.
Rom 11:12  Dar daca gre?eala lor a fost boga?ie lumii ?i mic?orarea lor boga?ie neamurilor, cu cât mai mult întreg numarul lor!
Rom 11:13  Caci v-o spun voua, neamurilor: Întru cât sunt eu, deci, apostol al neamurilor, slavesc slujirea mea,
Rom 11:14  Doar voi izbuti sa a?â? râvna celor din neamul meu ?i sa mântuiesc pe unii dintre ei.
Rom 11:15  Caci daca înlaturarea lor a adus împacarea lumii, ce va fi primirea lor la loc, daca nu o înviere din mor?i?
Rom 11:16  Iar daca este pârga (de faina) sfânta, ?i framântatura este sfânta; ?i daca radacina este sfânta, ?i ramurile sunt.
Rom 11:17  Iar daca unele din ramuri au fost taiate, ?i tu, care erai maslin salbatic, ai fost altoit printre cele ramase, ?i parta? te-ai facut radacinii ?i grasimii maslinului,
Rom 11:18  Nu te mândri fa?a de ramuri; iar daca te mândre?ti, nu tu por?i radacina, ci radacina pe tine.
Rom 11:19  Dar vei zice: Au fost taiate ramurile, ca sa fiu altoit eu.
Rom 11:20  Bine! Din cauza necredin?ei au fost taiate, iar tu stai prin credin?a. Nu te îngâmfa, ci teme-te;
Rom 11:21  Caci daca Dumnezeu n-a cru?at ramurile fire?ti, nici pe tine nu te va cru?a.
Rom 11:22  Vezi deci bunatatea ?i asprimea lui Dumnezeu: Asprimea Lui catre cei ce au cazut în bunatatea Lui catre tine, daca vei starui în aceasta bunatate; altfel ?i tu vei fi taiat.
Rom 11:23  Dar ?i aceia, de nu vor starui în necredin?a, vor fi altoi?i; caci puternic este Dumnezeu sa-i altoiasca iara?i.
Rom 11:24  Caci daca tu ai fost taiat din maslinul cel din fire salbatic ?i împotriva firii ai fost altoit în maslin bun, cu atât mai vârtos ace?tia, care sunt dupa fire, vor fi altoi?i în însu?i maslinul lor.
Rom 11:25  Pentru ca nu voiesc, fra?ilor, ca voi sa nu ?ti?i taina aceasta, ca sa nu va socoti?i pe voi în?iva în?elep?i; ca împietrirea s-a facut lui Israel în parte, pâna ce va intra tot numarul neamurilor.
Rom 11:26  ?i astfel întregul Israel se va mântui, precum este scris: "Din Sion va veni Izbavitorul ?i va îndeparta nelegiuirile de la Iacov;
Rom 11:27  ?i acesta este legamântul Meu cu ei, când voi ridica pacatele lor".
Rom 11:28  Cât prive?te Evanghelia, ei sunt vrajma?i din pricina voastra, dar cu privire la alegere ei sunt iubi?i, din cauza parin?ilor.
Rom 11:29  Caci darurile ?i chemarea lui Dumnezeu nu se pot lua înapoi.
Rom 11:30  Dupa cum voi, cândva, n-a?i ascultat de Dumnezeu, dar acum a?i fost milui?i prin neascultarea acestora,
Rom 11:31  Tot a?a ?i ace?tia n-au ascultat acum, ca, prin mila catre voi, sa fie milui?i ?i ei acum.
Rom 11:32  Caci Dumnezeu i-a închis pe to?i în neascultare, pentru ca pe to?i sa-i miluiasca.
Rom 11:33  O, adâncul boga?iei ?i al în?elepciunii ?i al ?tiin?ei lui Dumnezeu! Cât sunt de necercetate judeca?ile Lui ?i cât sunt de nepatrunse caile Lui!
Rom 11:34  Caci cine a cunoscut gândul Domnului sau cine a fost sfetnicul Lui?
Rom 11:35  Sau cine mai înainte I-a dat Lui ?i va lua înapoi de la El?
Rom 11:36  Pentru ca de la El ?i prin El ?i întru El sunt toate. A Lui sa fie marirea în veci. Amin!
Rom 12:1  Va îndemn, deci, fra?ilor, pentru îndurarile lui Dumnezeu, sa înfa?i?a?i trupurile voastre ca pe o jertfa vie, sfânta, bine placuta lui Dumnezeu, ca închinarea voastra cea duhovniceasca,
Rom 12:2  ?i sa nu va potrivi?i cu acest veac, ci sa va schimba?i prin înnoirea min?ii, ca sa deosebi?i care este voia lui Dumnezeu, ce este bun ?i placut ?i desavâr?it.
Rom 12:3  Caci, prin harul ce mi s-a dat, spun fiecaruia din voi sa nu cugete despre sine mai mult decât trebuie sa cugete, ci sa cugete fiecare spre a fi în?elept, precum Dumnezeu i-a împar?it masura credin?ei.
Rom 12:4  Ci precum într-un singur trup avem multe madulare ?i madularele nu au toate aceea?i lucrare,
Rom 12:5  A?a ?i noi, cei mul?i, un trup suntem în Hristos ?i fiecare suntem madulare unii altora;
Rom 12:6  Dar avem felurite daruri, dupa harul ce ni s-a dat. Daca avem proorocie, sa proorocim dupa  masura credin?ei;
Rom 12:7  Daca avem slujba, sa staruim în slujba; daca unul înva?a, sa se sârguiasca în înva?atura;
Rom 12:8  Daca îndeamna, sa fie la îndemnare; daca împarte altora, sa împarta cu fireasca nevinova?ie; daca sta în frunte, sa fie cu tragere de inima; daca miluie?te, sa miluiasca cu voie buna!
Rom 12:9  Dragostea sa fie nefa?arnica. Urâ?i raul, alipi?i-va de bine.
Rom 12:10  În iubire fra?easca, unii pe al?ii iubi?i-va; în cinste, unii altora da?i-va întâietate.
Rom 12:11  La sârguin?a, nu pregeta?i; cu duhul fi?i fierbin?i; Domnului sluji?i.
Rom 12:12  Bucura?i-va în nadejde; în suferin?a fi?i rabdatori; la rugaciune starui?i.
Rom 12:13  Face?i-va parta?i la trebuin?ele sfin?ilor, iubirea de straini urmând.
Rom 12:14  Binecuvânta?i pe cei ce va prigonesc, binecuvânta?i-i ?i nu-i blestema?i.
Rom 12:15  Bucura?i-va cu cei ce se bucura; plânge?i cu cei ce plâng.
Rom 12:16  Cugeta?i acela?i lucru unii pentru al?ii; nu cugeta?i la cele înalte, ci lasa?i-va du?i de spre cele smerite. Nu va socoti?i voi în?iva în?elep?i.
Rom 12:17  Nu rasplati?i nimanui raul cu rau. Purta?i grija de cele bune înaintea tuturor oamenilor.
Rom 12:18  Daca se poate, pe cât sta în puterea voastra, trai?i în buna pace cu to?i oamenii.
Rom 12:19  Nu va razbuna?i singuri, iubi?ilor, ci lasa?i loc mâniei (lui Dumnezeu), caci scris este: "A Mea este razbunarea; Eu voi rasplati, zice Domnul".
Rom 12:20  Deci, daca vrajma?ul tau este flamând, da-i de mâncare; daca îi este sete, da-i sa bea, caci, facând acestea, vei gramadi carbuni de foc pe capul lui.
Rom 12:21  Nu te lasa biruit de rau, ci biruie?te raul cu binele.
Rom 13:1  Tot sufletul sa se supuna înaltelor stapâniri, caci nu este stapânire decât de la Dumnezeu; iar cele ce sunt, de Dumnezeu sunt rânduite.
Rom 13:2  Pentru aceea, cel ce se împotrive?te stapânirii se împotrive?te rânduielii lui Dumnezeu. Iar cel ce se împotrivesc î?i vor lua osânda.
Rom 13:3  Caci dregatorii nu sunt frica pentru fapta buna, ci pentru cea rea. Voie?ti, deci, sa nu-?i fie frica de stapânire? Fa binele ?i vei avea lauda de la ea.
Rom 13:4  Caci ea este slujitoare a lui Dumnezeu spre binele tau. Iar daca faci rau, teme-te; caci nu în zadar poarta sabia; pentru ca ea este slujitoare a lui Dumnezeu ?i razbunatoare a mâniei Lui, asupra celui ce savâr?e?te raul.
Rom 13:5  De aceea este nevoie sa va supune?i, nu numai pentru mânie, ci ?i pentru con?tiin?a.
Rom 13:6  Ca pentru aceasta plati?i ?i dari. Caci (dregatorii) sunt slujitorii lui Dumnezeu, staruind în aceasta slujire neîncetat.
Rom 13:7  Da?i deci tuturor cele ce sunte?i datori: celui cu darea, darea; celui cu vama, vama; celui cu teama, teama; celui cu cinstea, cinste.
Rom 13:8  Nimanui cu nimic nu fi?i datori, decât cu iubirea unuia fa?a de altul; ca cel care iube?te pe aproapele a împlinit legea.
Rom 13:9  Pentru ca (poruncile): Sa nu savâr?e?ti adulter; sa nu ucizi; sa nu furi; sa nu marturise?ti strâmb; sa nu pofte?ti... ?i orice alta porunca ar mai fi se cuprind în acest cuvânt: Sa iube?ti pe aproapele tau ca pe tine însu?i.
Rom 13:10  Iubirea nu face rau aproapelui; iubirea este deci împlinirea legii.
Rom 13:11  ?i aceasta, fiindca ?ti?i în ce timp ne gasim, caci este chiar ceasul sa va trezi?i din somn; caci acum mântuirea este mai aproape de noi, decât atunci când am crezut.
Rom 13:12  Noaptea e pe sfâr?ite; ziua este aproape. Sa lepadam dar lucrurile întunericului ?i sa ne îmbracam cu armele luminii.
Rom 13:13  Sa umblam cuviincios, ca ziua: nu în ospe?e ?i în be?ii, nu în desfrânari ?i în fapte de ru?ine, nu în cearta ?i în pizma;
Rom 13:14  Ci îmbraca?i-va în Domnul Iisus Hristos ?i grija de trup sa nu o face?i spre pofte.
Rom 14:1  Primi?i-l pe cel slab în credin?a fara sa-i judeca?i gândurile.
Rom 14:2  Unul crede sa manânce de toate; cel slab însa manânca legume.
Rom 14:3  Cel ce manânca sa nu dispre?uiasca pe cel ce nu manânca; iar cel ce nu manânca sa nu osândeasca pe cel ce manânca, fiindca Dumnezeu l-a primit.
Rom 14:4  Cine e?ti tu, ca sa judeci pe sluga altuia? Pentru stapânul sau sta sau cade. Dar va sta, caci Domnul are putere ca sa-l faca sa stea.
Rom 14:5  Unul deosebe?te o zi de alta, iar altul judeca toate zilele la fel. Fiecare sa fie deplin încredin?at în mintea lui.
Rom 14:6  Cel ce ?ine ziua, o ?ine pentru Domnul; ?i cel ce nu ?ine ziua, nu o ?ine pentru Domnul. ?i cel ce manânca pentru Domnul manânca, caci mul?ume?te lui Dumnezeu; ?i cel ce nu manânca pentru Domnul nu manânca, ?i mul?ume?te lui Dumnezeu.
Rom 14:7  Caci nimeni dintre noi nu traie?te pentru sine ?i nimeni nu moare pentru sine.
Rom 14:8  Ca daca traim, pentru Domnul traim, ?i daca murim, pentru Domnul murim. Deci ?i daca traim, ?i daca murim, ai Domnului suntem.
Rom 14:9  Caci pentru aceasta a murit ?i a înviat Hristos, ca sa stapâneasca ?i peste mor?i ?i peste vii.
Rom 14:10  Dar tu, de ce judeci pe fratele tau? Sau ?i tu, de ce dispre?uie?ti pe fratele tau? Caci to?i ne vom înfa?i?a înaintea judeca?ii lui Dumnezeu.
Rom 14:11  Caci scris este: "Viu sunt Eu! - zice Domnul - Tot genunchiul sa Mi se plece ?i toata limba  sa dea slava lui Dumnezeu".
Rom 14:12  Deci, dar, fiecare din voi va da seama despre sine lui Dumnezeu.
Rom 14:13  Deci sa nu ne mai judecam unii pe al?ii, ci mai degraba judeca?i aceasta: Sa nu da?i fratelui prilej de poticnire sau de sminteala.
Rom 14:14  ?tiu ?i sunt încredin?at în Domnul Iisus ca nimic nu este întinat prin sine, decât numai pentru cel care gânde?te ca e ceva întinat; pentru acela întinat este.
Rom 14:15  Dar daca, pentru mâncare, fratele tau se mâhne?te, nu mai umbli potrivit iubirii. Nu pierde, cu mâncarea ta, pe acela pentru care a murit Hristos.
Rom 14:16  Nu lasa?i ca bunul vostru sa fie defaimat.
Rom 14:17  Caci împara?ia lui Dumnezeu nu este mâncare ?i bautura, ci dreptate ?i pace ?i bucurie în Duhul Sfânt.
Rom 14:18  Iar cel ce sluje?te lui Hristos, în aceasta este placut lui Dumnezeu ?i cinstit de oameni.
Rom 14:19  Drept aceea sa urmarim cele ale pacii ?i cele ale zidirii unuia de catre altul.
Rom 14:20  Nu strica, pentru mâncare, lucrul lui Dumnezeu. Toate sunt curate, dar rau este pentru omul care manânca spre poticnire.
Rom 14:21  Bine este sa nu manânci carne, nici sa bei vin, nici sa faci ceva de care fratele tau se poticne?te, se sminte?te sau slabe?te (în credin?a).
Rom 14:22  Credin?a pe care o ai, s-o ai pentru tine însu?i, înaintea lui Dumnezeu. Fericit este cel ce nu se judeca pe sine în ceea ce aproba!
Rom 14:23  Iar cel ce se îndoie?te, daca va mânca, se osânde?te, fiindca n-a fost din credin?a. ?i tot ce nu este din credin?a este pacat.
Rom 15:1  Datori suntem noi cei tari sa purtam slabiciunile celor neputincio?i ?i sa nu cautam placerea noastra.
Rom 15:2  Ci fiecare dintre noi sa caute sa placa aproapelui sau, la ce este bine, spre zidire.
Rom 15:3  Ca ?i Hristos n-a cautat placerea Sa, ci, precum este scris: "Ocarile celor ce Te ocarasc pe Tine, au cazut asupra Mea".
Rom 15:4  Caci toate câte s-au scris mai înainte, s-au scris spre înva?atura noastra, ca prin rabdarea ?i mângâierea, care vin din Scripturi, sa avem nadejde.
Rom 15:5  Iar Dumnezeul rabdarii ?i al mângâierii sa va dea voua a gândi la fel unii pentru al?ii, dupa Iisus Hristos,
Rom 15:6  Pentru ca to?i laolalta ?i cu o singura gura sa slavi?i pe Dumnezeu ?i Tatal Domnului nostru Iisus Hristos.
Rom 15:7  De aceea, primi?i-va unii pe al?ii, precum ?i Hristos v-a primit pe voi, spre slava lui Dumnezeu.
Rom 15:8  Caci spun: Ca Hristos S-a facut slujitor al taierii împrejur pentru adevarul lui Dumnezeu, ca sa întareasca fagaduin?ele date parin?ilor,
Rom 15:9  Iar neamurile sa slaveasca pe Dumnezeu pentru mila Lui, precum este scris: "Pentru aceasta Te voi lauda între neamuri ?i voi cânta numele Tau".
Rom 15:10  ?i iara?i zice Scriptura: "Veseli?i-va, neamuri, cu poporul Lui".
Rom 15:11  ?i iara?i: "Lauda?i pe Domnul toate neamurile; lauda?i-L pe El toate popoarele".
Rom 15:12  ?i iara?i Isaia zice: "?i Se va arata radacina lui Iesei, Cel care Se ridica sa domneasca peste neamuri; întru Acela neamurile vor nadajdui".
Rom 15:13  Iar Dumnezeul nadejdii sa va umple pe voi de toata bucuria ?i pacea în credin?a, ca sa prisoseasca nadejdea voastra, prin puterea Duhului Sfânt.
Rom 15:14  ?i, fra?ii mei, sunt încredin?at eu însumi despre voi, ca ?i voi sunte?i plini de bunatate, plini de toata cuno?tin?a, putând sa va pova?ui?i unii pe al?ii.
Rom 15:15  ?i v-am scris, fra?ilor, mai cu îndrazneala, în parte, ca sa va amintesc despre harul ce mi-a fost dat de Dumnezeu,
Rom 15:16  Ca sa fiu slujitor al lui Iisus Hristos la neamuri, slujind Evanghelia lui Dumnezeu, pentru ca prinosul neamurilor, fiind sfin?it în Duhul Sfânt, sa fie bine primit.
Rom 15:17  A?adar, în Hristos Iisus am lauda, în cele catre Dumnezeu.
Rom 15:18  Caci nu voi cuteza sa spun ceva din cele ce n-a savâr?it Hristos prin mine, spre ascultarea neamurilor, prin cuvânt ?i prin fapta,
Rom 15:19  Prin puterea semnelor ?i a minunilor, prin puterea Duhului Sfânt, a?a încât de la Ierusalim ?i din ?inuturile de primprejur pâna la Iliria, am împlinit propovaduirea Evangheliei lui Hristos,
Rom 15:20  Râvnind astfel sa binevestesc acolo unde Hristos nu fusese numit, ca sa nu zidesc pe temelie straina,
Rom 15:21  Ci precum este scris: "Carora nu li s-a vestit despre El, aceia Îl vor vedea; ?i cei ce n-au auzit Îl vor în?elege".
Rom 15:22  De aceea am ?i fost împiedicat, de multe ori, ca sa vin la voi.
Rom 15:23  Dar acum, nemaiavând loc în aceste ?inuturi ?i având dorin?a de mul?i ani sa vin la voi,
Rom 15:24  Când ma voi duce în Spania, voi veni la voi. Caci nadajduiesc sa va vad în trecere ?i, de catre voi, sa fiu înso?it pâna acolo, dupa ce ma voi bucura întâi, în parte, de voi.
Rom 15:25  Acum însa ma duc la Ierusalim, ca sa slujesc sfin?ilor.
Rom 15:26  Caci Macedonia ?i Ahaia au binevoit sa faca o strângere de ajutoare pentru saracii dintre sfin?ii de la Ierusalim.
Rom 15:27  Caci ei au binevoit ?i sunt datori fa?a de ei. Caci daca neamurile s-au împarta?it de cele duhovnice?ti ale lor, datori sunt ?i ei sa le slujeasca în cele trupe?ti.
Rom 15:28  Savâr?ind deci aceasta ?i încredin?ându-le roada aceasta, voi trece pe la voi, în Spania.
Rom 15:29  ?i ?tiu ca, venind la voi, voi veni cu deplinatatea binecuvântarii lui Hristos.
Rom 15:30  Dar va îndemn, fra?ilor, pentru Domnul nostru Iisus Hristos ?i pentru iubirea Duhului Sfânt, ca împreuna cu mine sa lupta?i în rugaciuni catre Dumnezeu pentru mine,
Rom 15:31  Ca sa scap de necredincio?ii din Iudeea ?i ca ajutorul meu la Ierusalim sa fie bine primit de catre sfin?i,
Rom 15:32  Ca sa vin la voi cu bucurie prin voia lui Dumnezeu ?i sa-mi gasesc lini?tea împreuna cu voi.
Rom 15:33  Iar Dumnezeul pacii sa fie cu voi cu to?i. Amin!
Rom 16:1  ?i va încredin?ez pe Febe, sora noastra, care este diaconi?a a Bisericii din Chenhrea,
Rom 16:2  Ca s-o primi?i în Domnul, cu vrednicia cuvenita sfin?ilor ?i sa-i fi?i de ajutor la orice ar avea nevoie de ajutorul vostru. Caci ?i ea a ajutat pe mul?i ?i pe mine însumi.
Rom 16:3  Îmbra?i?a?i pe Priscila ?i Acvila, împreuna-lucratori cu mine în Hristos Iisus,
Rom 16:4  Care ?i-au pus grumazul lor pentru via?a mea ?i carora nu numai eu le mul?umesc, ci ?i toate Bisericile dintre neamuri,
Rom 16:5  ?i Biserica din casa lor.  Îmbra?i?a?i pe Epenet, iubitul meu, care este pârga Asiei, în Hristos.
Rom 16:6  Îmbra?i?a?i pe Maria care s-a ostenit mult pentru voi.
Rom 16:7  Îmbra?i?a?i pe Andronic ?i pe Iunias, cei de un neam cu mine ?i împreuna închi?i cu mine, care sunt vesti?i între apostoli ?i care înaintea mea au fost în Hristos.
Rom 16:8  Îmbra?i?a?i pe Ampliat, iubitul meu în Domnul.
Rom 16:9  Îmbra?i?a?i pe Urban, împreuna-lucrator cu mine în Hristos, ?i pe Stahis, iubitul meu.
Rom 16:10  Îmbra?i?a?i pe Apelles, cel încercat în Hristos. Îmbra?i?a?i pe cei ce sunt din casa lui Aristobul.
Rom 16:11  Îmbra?i?a?i pe Irodion, cel de un neam cu mine. Îmbra?i?a?i pe cei din casa lui Narcis, care sunt în Domnul.
Rom 16:12  Îmbra?i?a?i pe Trifena ?i pe Trifosa, care s-au ostenit în Domnul. Îmbra?i?a?i pe iubita Persida, care mult s-a ostenit în Domnul.
Rom 16:13  Îmbra?i?a?i pe Ruf, cel ales întru Domnul, ?i pe mama lui, care este ?i a mea.
Rom 16:14  Îmbra?i?a?i pe Asincrit, pe Flegon, pe Hermes, pe Patrova, pe Hermas ?i pe fra?ii care sunt împreuna cu ei.
Rom 16:15  Îmbra?i?a?i pe Filolog ?i pe Iulia, pe Nereu ?i pe sora lui, pe Olimpian ?i pe to?i sfin?ii care sunt împreuna cu ei.
Rom 16:16  Îmbra?i?a?i-va unii pe al?ii cu sarutare sfânta. Va îmbra?i?eaza pe voi toate Bisericile lui Hristos.
Rom 16:17  ?i va îndemn, fra?ilor, sa va pazi?i de cei ce fac dezbinari ?i sminteli împotriva înva?aturii pe care a?i primit-o. Departa?i-va de ei.
Rom 16:18  Caci unii ca ace?tia nu slujesc Domnului nostru Iisus Hristos, ci pântecelui lor, ?i prin vorbele lor frumoase ?i magulitoare, în?eala inimile celor fara de rautate.
Rom 16:19  Caci ascultarea voastra este cunoscuta de to?i. Ma bucur deci de voi ?i voiesc sa fi?i în?elep?i spre bine ?i nevinova?i la rau.
Rom 16:20  Iar Dumnezeul pacii va zdrobi repede sub picioarele voastre pe satana. Harul Domnului nostru Iisus Hristos cu voi!
Rom 16:21  Va îmbra?i?eaza Timotei, cel împreuna-lucrator cu mine, ?i Luciu ?i Iason ?i Sosipatru, cei de un neam cu mine,
Rom 16:22  Va îmbra?i?ez în Domnul eu, Tertius, care am scris epistola.
Rom 16:23  Va îmbra?i?eaza Gaius, gazda mea ?i a toata Biserica. Va îmbra?i?eaza Erast, vistiernicul ceta?ii, ?i fratele Cvartus.
Rom 16:24  Harul Domnului nostru Iisus Hristos sa fie cu voi cu to?i. Amin!
Rom 16:25  Iar celui ce poate sa va întareasca dupa Evanghelia mea ?i dupa propovaduirea lui Iisus Hristos, potrivit cu descoperirea tainei celei ascunse din timpuri ve?nice,
Rom 16:26  Iar acum aratata prin Scripturile proorocilor, dupa porunca ve?nicului Dumnezeu ?i cunoscuta la toate neamurile spre ascultarea credin?ei,
Rom 16:27  Unuia în?eleptului Dumnezeu, prin Iisus Hristos, fie slava în vecii vecilor. Amin!


\end{document}