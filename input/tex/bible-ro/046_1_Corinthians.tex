\begin{document}

\title{1 Corinthians}

1Co 1:1  Pavel, chemat apostol al lui Hristos, prin voia lui Dumnezeu, ?i fratele Sostene,
1Co 1:2  Bisericii lui Dumnezeu care este în Corint, celor sfin?i?i în Iisus Hristos, celor numi?i sfin?i, împreuna cu to?i cei ce cheama numele Domnului nostru Iisus Hristos în tot locul, ?i al lor ?i al nostru:
1Co 1:3  Har voua ?i pace de la Dumnezeu, Tatal nostru, ?i de la Domnul nostru Iisus Hristos.
1Co 1:4  Mul?umesc totdeauna Dumnezeului meu pentru voi, pentru harul lui Dumnezeu, dat voua în Hristos Iisus.
1Co 1:5  Caci întru El v-a?i îmboga?it deplin întru toate, în tot cuvântul ?i în toata cuno?tin?a;
1Co 1:6  Astfel marturia lui Hristos s-a întarit în voi,
1Co 1:7  Încât voi nu sunte?i lipsi?i de nici un dar, a?teptând aratarea Domnului nostru Iisus Hristos,
1Co 1:8  Care va va ?i întari pâna la sfâr?it, ca sa fi?i nevinova?i în ziua Domnului nostru Iisus Hristos.
1Co 1:9  Credincios este Dumnezeu, prin Care a?i fost chema?i la împarta?irea cu Fiul Sau Iisus Hristos, Domnul nostru.
1Co 1:10  Va îndemn, fra?ilor, pentru numele Domnului nostru Iisus Hristos, ca to?i sa vorbi?i la fel ?i sa nu fie dezbinari între voi; ci sa fi?i cu totul uni?i în acela?i cuget ?i în aceea?i în?elegere.
1Co 1:11  Caci, fra?ii mei, despre voi, prin cei din casa lui Hloe mi-a venit ?tire ca la voi sunt certuri;
1Co 1:12  ?i spun aceasta, ca fiecare dintre voi zice: Eu sunt al lui Pavel, iar eu sunt al lui Apollo, iar eu sunt al lui Chefa, iar eu sunt al lui Hristos!
1Co 1:13  Oare s-a împar?it Hristos? Nu cumva s-a rastignit Pavel pentru voi? Sau fost-a?i boteza?i în numele lui Pavel?
1Co 1:14  Mul?umesc lui Dumnezeu ca pe nici unul din voi n-am botezat, decât pe Crispus ?i pe Gaius,
1Co 1:15  Ca sa nu zica cineva ca a?i fost boteza?i în numele meu.
1Co 1:16  Am botezat ?i casa lui ?tefana; afara de ace?tia nu ?tiu sa mai fi botezat pe altcineva.
1Co 1:17  Caci Hristos nu m-a trimis ca sa botez, ci sa binevestesc, dar nu cu în?elepciunea cuvântului, ca sa nu ramâna zadarnica crucea lui Hristos.
1Co 1:18  Caci cuvântul Crucii, pentru cei ce pier, este nebunie; iar pentru noi, cei ce ne mântuim, este puterea lui Dumnezeu.
1Co 1:19  Caci scris este: "Pierde-voi în?elepciunea în?elep?ilor ?i ?tiin?a celor înva?a?i voi nimici-o".
1Co 1:20  Unde este în?eleptul? Unde e carturarul? Unde e cercetatorul acestui veac? Au n-a dovedit Dumnezeu nebuna în?elepciunea lumii acesteia?
1Co 1:21  Caci de vreme ce întru în?elepciunea lui Dumnezeu lumea n-a cunoscut prin în?elepciune pe Dumnezeu, a binevoit Dumnezeu sa mântuiasca pe cei ce cred prin nebunia propovaduirii.
1Co 1:22  Fiindca ?i iudeii cer semne, iar elinii cauta în?elepciune,
1Co 1:23  Însa noi propovaduim pe Hristos cel rastignit: pentru iudei, sminteala; pentru neamuri, nebunie.
1Co 1:24  Dar pentru cei chema?i, ?i iudei ?i elini: pe Hristos, puterea lui Dumnezeu ?i în?elepciunea lui Dumnezeu.
1Co 1:25  Pentru ca fapta lui Dumnezeu, socotita de catre oameni nebunie, este mai în?eleapta decât în?elepciunea lor ?i ceea ce se pare ca slabiciune a lui Dumnezeu, mai puternica decât taria oamenilor.
1Co 1:26  Caci, privi?i chemarea voastra, fra?ilor, ca nu mul?i sunt în?elep?i dupa trup, nu mul?i sunt puternici, nu mul?i sunt de bun neam;
1Co 1:27  Ci Dumnezeu ?i-a ales pe cele nebune ale lumii, ca sa ru?ineze pe cei în?elep?i; Dumnezeu ?i-a ales pe cele slabe ale lumii, ca sa le ru?ineze pe cele tari;
1Co 1:28  Dumnezeu ?i-a ales pe cele de neam jos ale lumii, pe cele nebagate în seama, pe cele ce nu sunt, ca sa nimiceasca pe cele ce sunt,
1Co 1:29  Ca nici un trup sa nu se laude înaintea lui Dumnezeu.
1Co 1:30  Din El, dar, sunte?i voi în Hristos Iisus, Care pentru noi S-a facut în?elepciune de la Dumnezeu ?i dreptate ?i sfin?ire ?i rascumparare,
1Co 1:31  Pentru ca, dupa cum este scris: "Cel ce se lauda în Domnul sa se laude".
1Co 2:1  ?i eu, fra?ilor, când am venit la voi ?i v-am vestit taina lui Dumnezeu, n-am venit ca iscusit cuvântator sau ca în?elept.
1Co 2:2  Caci am judecat sa nu ?tiu între voi altceva, decât pe Iisus Hristos, ?i pe Acesta rastignit.
1Co 2:3  ?i eu întru slabiciune ?i cu frica ?i cu cutremur mare am fost la voi.
1Co 2:4  Iar cuvântul meu ?i propovaduirea mea nu stateau în cuvinte de înduplecare ale în?elepciunii omene?ti, ci în adeverirea Duhului ?i a puterii,
1Co 2:5  Pentru ca credin?a voastra sa nu fie în în?elepciunea oamenilor, ci în puterea lui Dumnezeu.
1Co 2:6  ?i în?elepciunea o propovaduim la cei desavâr?i?i, dar nu în?elepciunea acestui veac, nici a stapânitorilor acestui veac, care sunt pieritori,
1Co 2:7  Ci propovaduim în?elepciunea de taina a lui Dumnezeu, ascunsa, pe care Dumnezeu a rânduit-o mai înainte de veci, spre slava noastra,
1Co 2:8  Pe care nici unul dintre stapânitorii acestui veac n-a cunoscut-o, caci, daca ar fi cunoscut-o, n-ar fi rastignit pe Domnul slavei;
1Co 2:9  Ci precum este scris: "Cele ce ochiul n-a vazut ?i urechea n-a auzit, ?i la inima omului nu s-au suit, pe acestea le-a gatit Dumnezeu celor ce-L iubesc pe El".
1Co 2:10  Iar noua ni le-a descoperit Dumnezeu prin Duhul Sau, fiindca Duhul toate le cerceteaza, chiar ?i adâncurile lui Dumnezeu.
1Co 2:11  Caci cine dintre oameni ?tie ale omului, decât duhul omului, care este în el? A?a ?i cele ale lui Dumnezeu, nimeni nu le-a cunoscut, decât Duhul lui Dumnezeu.
1Co 2:12  Iar noi n-am primit duhul lumii, ci Duhul cel de la Dumnezeu, ca sa cunoa?tem cele daruite noua de Dumnezeu;
1Co 2:13  Pe care le ?i graim, dar nu în cuvinte înva?ate din în?elepciunea omeneasca, ci în cuvinte înva?ate de la Duhul Sfânt, lamurind lucruri duhovnice?ti oamenilor duhovnice?ti.
1Co 2:14  Omul firesc nu prime?te cele ale Duhului lui Dumnezeu, caci pentru el sunt nebunie ?i nu poate sa le în?eleaga, fiindca ele se judeca duhovnice?te.
1Co 2:15  Dar omul duhovnicesc toate le judeca, pe el însa nu-l judeca nimeni;
1Co 2:16  Caci "Cine a cunoscut gândul Domnului, ca sa-L înve?e pe El?" Noi însa avem gândul lui Hristos.
1Co 3:1  ?i eu, fra?ilor, n-am putut sa va vorbesc ca unor oameni duhovnice?ti, ci ca unora trupe?ti, ca unor prunci în Hristos.
1Co 3:2  Cu lapte v-am hranit, nu cu bucate, caci înca nu putea?i mânca ?i înca nici acum nu pute?i,
1Co 3:3  Fiindca sunte?i tot trupe?ti. Câta vreme este între voi pizma ?i cearta ?i dezbinari, nu sunte?i, oare, trupe?ti ?i nu dupa firea omeneasca umbla?i?
1Co 3:4  Caci, când zice unul: Eu sunt al lui Pavel, iar altul: Eu sunt al lui Apollo, au nu sunte?i oameni trupe?ti?
1Co 3:5  Dar ce este Apollo? ?i ce este Pavel? Slujitori prin care a?i crezut voi ?i dupa cum i-a dat Domnul fiecaruia.
1Co 3:6  Eu am sadit, Apollo a udat, dar Dumnezeu a facut sa creasca.
1Co 3:7  Astfel nici cel ce sade?te nu e ceva, nici cel ce uda, ci numai Dumnezeu care face sa creasca.
1Co 3:8  Cel care sade?te ?i cel care uda sunt una ?i fiecare î?i va lua plata dupa osteneala sa.
1Co 3:9  Caci noi împreuna-lucratori cu Dumnezeu suntem; voi sunte?i ogorul lui Dumnezeu, zidirea lui Dumnezeu.
1Co 3:10  Dupa harul lui Dumnezeu, cel dat mie, eu, ca un în?elept me?ter, am pus temelia; iar altul zide?te. Dar fiecare sa ia seama cum zide?te;
1Co 3:11  Caci nimeni nu poate pune alta temelie, decât cea pusa, care este Iisus Hristos.
1Co 3:12  Iar de zide?te cineva pe aceasta temelie: aur, argint, sau pietre scumpe, lemne, fân, trestie.
1Co 3:13  Lucrul fiecaruia se va face cunoscut; îl va vadi ziua (Domnului). Pentru ca în foc se descopera, ?i focul însu?i va lamuri ce fel este lucrul fiecaruia.
1Co 3:14  Daca lucrul cuiva, pe care l-a zidit, va ramâne, va lua plata.
1Co 3:15  Daca lucrul cuiva se va arde, el va fi pagubit; el însa se va mântui, dar a?a ca prin foc.
1Co 3:16  Nu ?ti?i, oare, ca voi sunte?i templu al lui Dumnezeu ?i ca Duhul lui Dumnezeu locuie?te în voi?
1Co 3:17  De va strica cineva templul lui Dumnezeu, îl va strica Dumnezeu pe el, pentru ca sfânt este templul lui Dumnezeu, care sunte?i voi.
1Co 3:18  Nimeni sa nu se amageasca. Daca i se pare cuiva, între voi, ca este în?elept în veacul acesta, sa se faca nebun, ca sa fie în?elept.
1Co 3:19  Caci în?elepciunea lumii acesteia este nebunie înaintea lui Dumnezeu, pentru ca scris este: "El prinde pe cei în?elep?i în viclenia lor".
1Co 3:20  ?i iara?i: "Domnul cunoa?te gândurile în?elep?ilor, ca sunt de?arte".
1Co 3:21  A?a ca nimeni sa nu se laude cu oameni. Caci toate sunt ale voastre:
1Co 3:22  Fie Pavel, fie Apollo, fie Chefa, fie lumea, fie via?a, fie moartea, fie cele de fa?a, fie cele viitoare, toate sunt ale voastre.
1Co 3:23  Iar voi sunte?i ai lui Hristos, iar Hristos al lui Dumnezeu.
1Co 4:1  A?a sa ne socoteasca pe noi fiecare om: ca slujitori ai lui Hristos ?i ca iconomi ai tainelor lui Dumnezeu.
1Co 4:2  Iar, la iconomi, mai ales, se cere ca fiecare sa fie aflat credincios.
1Co 4:3  Dar mie prea pu?in îmi este ca sunt judecat de voi sau de vreo omeneasca judecata de toata ziua; fiindca nici eu nu ma judec pe mine însumi.
1Co 4:4  Caci nu ma ?tiu vinovat cu nimic, dar nu întru aceasta m-am îndreptat. Cel care ma judeca pe mine este Domnul.
1Co 4:5  De aceea, nu judeca?i ceva înainte de vreme, pâna ce nu va veni Domnul, Care va lumina cele ascunse ale întunericului ?i va vadi sfaturile inimilor. ?i atunci fiecare va avea de la Dumnezeu lauda.
1Co 4:6  ?i acestea, fra?ilor, le-am zis ca despre mine ?i despre Apollo, dar ele sunt pentru voi, ca sa înva?a?i din pilda noastra, sa nu trece?i peste ce e scris, ca sa nu va fali?i unul cu altul împotriva celuilalt.
1Co 4:7  Caci cine te deosebe?te pe tine? ?i ce ai, pe care sa nu-l fi primit? Iar daca l-ai primit, de ce te fale?ti, ca ?i cum nu l-ai fi primit?
1Co 4:8  Iata, sunte?i satui; iata, v-a?i îmboga?it; fara de noi a?i domnit, ?i, macar nu a?i domnit, ca ?i noi sa domnim împreuna cu voi.
1Co 4:9  Caci mi se pare ca Dumnezeu, pe noi, apostolii, ne-a aratat ca pe cei din urma oameni, ca pe ni?te osândi?i la moarte, fiindca ne-am facut priveli?te lumii ?i îngerilor ?i oamenilor.
1Co 4:10  Noi suntem nebuni pentru Hristos; voi însa în?elep?i întru Hristos. Noi suntem slabi; voi însa sunte?i tari. Voi sunte?i întru slava, iar noi suntem întru necinste!
1Co 4:11  Pâna în ceasul de acum flamânzim ?i însetam; suntem goi ?i suntem palmui?i ?i pribegim,
1Co 4:12  ?i ne ostenim, lucrând cu mâinile noastre. Ocarâ?i fiind, binecuvântam. Prigoni?i fiind, rabdam.
1Co 4:13  Huli?i fiind, ne rugam. Am ajuns ca gunoiul lumii, ca maturatura tuturor, pâna astazi.
1Co 4:14  Nu ca sa va ru?inez va scriu acestea, ci ca sa va dojenesc, ca pe ni?te copii ai mei iubi?i.
1Co 4:15  Caci de a?i avea zeci de mii de înva?atori în Hristos, totu?i nu ave?i mul?i parin?i. Caci eu v-am nascut prin Evanghelie în Iisus Hristos.
1Co 4:16  Deci, va rog, sa-mi fi?i mie urmatori, precum ?i eu lui Hristos.
1Co 4:17  Pentru aceasta am trimis la voi pe Timotei, care este fiul meu iubit ?i credincios în Domnul. El va va aduce aminte caile mele cele în Hristos Iisus, cum înva? eu pretutindeni în toata Biserica.
1Co 4:18  ?i unii, crezând ca n-am sa mai vin la voi, s-au seme?it.
1Co 4:19  Dar eu voi veni la voi degraba - daca Domnul va voi - ?i voi cunoa?te nu cuvântul celor ce s-au seme?it, ci puterea lor.
1Co 4:20  Caci împara?ia lui Dumnezeu nu sta în cuvânt, ci în putere.
1Co 4:21  Ce voi?i? Sa vin la voi cu toiagul sau sa vin cu dragoste ?i cu duhul blânde?ii?
1Co 5:1  Îndeob?te se aude ca la voi e desfrânare, ?i o astfel de desfrânare cum nici între neamuri nu se pomene?te, ca unul sa traiasca cu femeia tatalui sau.
1Co 5:2  Iar voi v-a?i seme?it, în loc mai degraba sa va fi întristat, ca sa fie scos din mijlocul vostru cel ce a savâr?it aceasta fapta.
1Co 5:3  Ci eu, de?i departe cu trupul, însa de fa?a cu duhul, am ?i judecat, ca ?i cum a? fi de fa?a, pe cel ce a facut una ca aceasta:
1Co 5:4  În numele Domnului nostru Iisus Hristos, adunându-va voi ?i duhul meu, cu puterea Domnului nostru Iisus Hristos,
1Co 5:5  Sa da?i pe unul ca acesta satanei, spre pieirea trupului, ca duhul sa se mântuiasca în ziua Domnului Iisus.
1Co 5:6  Seme?ia voastra nu e buna. Oare nu ?ti?i ca pu?in aluat dospe?te toata framântatura?
1Co 5:7  Cura?i?i aluatul cel vechi, ca sa fi?i framântatura noua, precum ?i sunte?i fara aluat; caci Pa?tile nostru Hristos S-a jertfit pentru noi.
1Co 5:8  De aceea sa praznuim nu cu aluatul cel vechi, nici cu aluatul rauta?ii ?i al vicle?ugului, ci cu azimele cura?iei ?i ale adevarului.
1Co 5:9  V-am scris în epistola sa nu va amesteca?i cu desfrâna?ii;
1Co 5:10  Dar nu am spus, desigur, despre desfrâna?ii acestei lumi, sau despre lacomi, sau despre rapitori, sau despre închinatorii la idoli, caci altfel ar trebui sa ie?i?i afara din lume.
1Co 5:11  Dar eu v-am scris acum sa nu va amesteca?i cu vreunul, care, numindu-se frate, va fi desfrânat, sau lacom, sau închinator la idoli, sau ocarâtor, sau be?iv, sau rapitor. Cu unul ca acesta nici sa nu ?ede?i la masa.
1Co 5:12  Caci ce am eu ca sa judec ?i pe cei din afara? Însa pe cei dinauntru, oare, nu-i judeca?i voi?
1Co 5:13  Iar pe cei din afara îi va judeca Dumnezeu. Scoate?i afara dintre voi pe cel rau.
1Co 6:1  Îndrazne?te, oare, cineva dintre voi, având vreo pâra împotriva altuia, sa se judece înaintea celor nedrep?i ?i nu înaintea celor sfin?i?
1Co 6:2  Au nu ?ti?i ca sfin?ii vor judeca lumea? ?i daca lumea este judecata de voi, oare sunte?i voi nevrednici sa judeca?i lucruri atât de mici?
1Co 6:3  Nu ?ti?i, oare, ca noi vom judeca pe îngeri? Cu cât mai mult cele lume?ti?
1Co 6:4  Deci daca ave?i judeca?i lume?ti, pune?i pe cei nebaga?i în seama din Biserica, ca sa va judece.
1Co 6:5  O spun spre ru?inea voastra. Nu este, oare, între voi nici un om în?elept, care sa poata judeca între frate ?i frate?
1Co 6:6  Ci frate cu frate se judeca, ?i aceasta înaintea necredincio?ilor?
1Co 6:7  Negre?it, ?i aceasta este o scadere pentru voi, ca ave?i judeca?i unii cu al?ii. Pentru ce nu suferi?i mai bine strâmbatatea? Pentru ce nu rabda?i mai bine paguba?
1Co 6:8  Ci voi în?iva face?i strâmbatate ?i aduce?i paguba, ?i aceasta, fra?ilor!
1Co 6:9  Nu ?ti?i, oare, ca nedrep?ii nu vor mo?teni împara?ia lui Dumnezeu? Nu va amagi?i: Nici desfrâna?ii, nici închinatorii la idoli, nici adulterii, nici malahienii, nici sodomi?ii,
1Co 6:10  Nici furii, nici lacomii, nici be?ivii, nici batjocoritorii, nici rapitorii nu vor mo?teni împara?ia lui Dumnezeu.
1Co 6:11  ?i a?a era?i unii dintre voi. Dar v-a?i spalat, dar v-a?i sfin?it, dar v-a?i îndreptat în numele Domnului Iisus Hristos ?i în Duhul Dumnezeului nostru.
1Co 6:12  Toate îmi sunt îngaduite, dar nu toate îmi sunt de folos. Toate îmi sunt îngaduite, dar nu ma voi lasa biruit de ceva.
1Co 6:13  Bucatele sunt pentru pântece ?i pântecele pentru bucate ?i Dumnezeu va nimici ?i pe unul ?i pe celelalte. Trupul însa nu e pentru desfrânare, ci pentru Domnul, ?i Domnul este pentru trup.
1Co 6:14  Iar Dumnezeu, Care a înviat pe Domnul, ne va învia ?i pe noi prin puterea Sa.
1Co 6:15  Au nu ?ti?i ca trupurile voastre sunt madularele lui Hristos? Luând deci madularele lui Hristos le voi face madularele unei desfrânate? Nicidecum!
1Co 6:16  Sau nu ?ti?i ca cel ce se alipe?te de desfrânate este un singur trup cu ea? "Caci vor fi - zice Scriptura - cei doi un singur trup".
1Co 6:17  Iar cel ce se alipe?te de Domnul este un duh cu El.
1Co 6:18  Fugi?i de desfrânare! Orice pacat pe care-l va savâr?i omul este în afara de trup. Cine se deda însa desfrânarii pacatuie?te în însu?i trupul sau.
1Co 6:19  Sau nu ?ti?i ca trupul vostru este templu al Duhului Sfânt care este în voi, pe care-L ave?i de la Dumnezeu ?i ca voi nu sunte?i ai vo?tri?
1Co 6:20  Caci a?i fost cumpara?i cu pre?! Slavi?i, dar, pe Dumnezeu în trupul vostru ?i în duhul vostru, care sunt ale lui Dumnezeu.
1Co 7:1  Cât despre cele ce mi-a?i scris, bine este pentru om sa nu se atinga de femeie.
1Co 7:2  Dar din cauza desfrânarii, fiecare sa-?i aiba femeia sa ?i fiecare femeie sa-?i aiba barbatul sau.
1Co 7:3  Barbatul sa-i dea femeii iubirea datorata, asemenea ?i femeia barbatului.
1Co 7:4  Femeia nu este stapâna pe trupul sau, ci barbatul; asemenea nici barbatul nu este stapân pe trupul sau, ci femeia.
1Co 7:5  Sa nu va lipsi?i unul de altul, decât cu buna învoiala pentru un timp, ca sa va îndeletnici?i cu postul ?i cu rugaciunea, ?i iara?i sa fi?i împreuna, ca sa nu va ispiteasca satana, din pricina neînfrânarii voastre.
1Co 7:6  ?i aceasta o spun ca un sfat, nu ca o porunca.
1Co 7:7  Eu voiesc ca to?i oamenii sa fie cum sunt eu însumi. Dar fiecare are de la Dumnezeu darul lui: unul a?a, altul într-alt fel.
1Co 7:8  Celor ce sunt necasatori?i ?i vaduvelor le spun: Bine este pentru ei sa ramâna ca ?i mine.
1Co 7:9  Daca însa nu pot sa se înfrâneze, sa se casatoreasca. Fiindca mai bine este sa se casatoreasca, decât sa arda.
1Co 7:10  Iar celor ce sunt casatori?i, le poruncesc, nu eu, ci Domnul: Femeia sa nu se desparta de barbat!
1Co 7:11  Iar daca s-a despar?it, sa ramâna nemaritata, sau sa se împace cu barbatul sau; tot a?a barbatul sa nu-?i lase femeia.
1Co 7:12  Celorlal?i le graiesc eu, nu Domnul: Daca un frate are o femeie necredincioasa, ?i ea voie?te sa vie?uiasca cu el, sa nu o lase.
1Co 7:13  ?i o femeie, daca are barbat necredincios, ?i el binevoie?te sa locuiasca cu ea, sa nu-?i lase barbatul.
1Co 7:14  Caci barbatul necredincios se sfin?e?te prin femeia credincioasa ?i femeia necredincioasa se sfin?e?te prin barbatul credincios. Altminterea, copiii vo?tri ar fi necura?i, dar acum ei sunt sfin?i.
1Co 7:15  Daca însa cel necredincios se desparte, sa se desparta. În astfel de împrejurare, fratele sau sora nu sunt lega?i; caci Dumnezeu ne-a chemat spre pace.
1Co 7:16  Caci, ce ?tii tu, femeie, daca î?i vei mântui barbatul? Sau ce ?tii tu, barbate, daca î?i vei mântui femeia?
1Co 7:17  Numai ca, a?a cum a dat Domnul fiecaruia, a?a cum l-a chemat Dumnezeu pe fiecare, astfel sa umble. ?i a?a rânduiesc în toate Bisericile.
1Co 7:18  A fost cineva chemat, fiind taiat împrejur? Sa nu se ascunda. A fost cineva chemat în netaiere împrejur? Sa nu se taie împrejur.
1Co 7:19  Taierea împrejur nu este nimic; ?i netaierea împrejur nu este nimic, ci paza poruncilor lui Dumnezeu.
1Co 7:20  Fiecare, în chemarea în care a fost chemat, în aceasta sa ramâna.
1Co 7:21  Ai fost chemat fiind rob? Fii fara grija. Iar de po?i sa fii liber, mai mult folose?te-te!
1Co 7:22  Caci robul, care a fost chemat în Domnul, este un liberat al Domnului. Tot a?a cel chemat liber este rob al lui Hristos.
1Co 7:23  Cu pre? a?i fost cumpara?i. Nu va face?i robi oamenilor.
1Co 7:24  Fiecare, fra?ilor, în starea în care a fost chemat, în aceea sa ramâna înaintea lui Dumnezeu.
1Co 7:25  Cât despre feciorie, n-am porunca de la Domnul. Va dau însa sfatul meu, ca unul care am fost miluit de Domnul sa fiu vrednic de crezare.
1Co 7:26  Socotesc deci ca aceasta este bine pentru nevoia ceasului de fa?a: Bine este pentru om sa fie a?a.
1Co 7:27  Te-ai legat de femeie? Nu cauta dezlegare. Te-ai dezlegat de femeie? Nu cauta femeie.
1Co 7:28  Daca însa te vei însura, n-ai gre?it. Ci fecioara, de se va marita, n-a gre?it. Numai ca unii ca ace?tia vor avea suferin?a în trupul lor. Eu însa va cru? pe voi.
1Co 7:29  ?i aceasta v-o spun, fra?ilor: Ca vremea s-a scurtat de acum, a?a încât ?i cei ce au femei sa fie ca ?i cum n-ar avea.
1Co 7:30  ?i cei ce plâng sa fie ca ?i cum n-ar plânge; ?i cei ce se bucura, ca ?i cum nu s-ar bucura; ?i cei ce cumpara, ca ?i cum n-ar stapâni;
1Co 7:31  ?i cei ce se folosesc de lumea aceasta, ca ?i cum nu s-ar folosi deplin de ea. Caci chipul acestei lumi trece.
1Co 7:32  Dar eu vreau ca voi sa fi?i fara de  grija. Cel necasatorit se îngrije?te de cele ale Domnului, cum sa placa Domnului.
1Co 7:33  Cel ce s-a casatorit se îngrije?te de cele ale lumii, cum sa placa femeii.
1Co 7:34  ?i este împar?ire: ?i femeia nemaritata ?i fecioara poarta de grija de cele ale Domnului, ca sa fie sfânta ?i cu trupul ?i cu duhul. Iar cea care s-a maritat poarta de grija de cele ale lumii, cum sa placa barbatului.
1Co 7:35  ?i aceasta o spun chiar în folosul vostru, nu ca sa va întind o cursa, ci spre bunul chip ?i alipirea de Domnul, fara clintire.
1Co 7:36  Iar de socote?te cineva ca i se va face vreo necinste pentru fecioara sa, daca trece de floarea vârstei, ?i ca trebuie sa faca a?a, faca ce voie?te. Nu pacatuie?te; casatoreasca-se.
1Co 7:37  Dar cel ce sta neclintit în inima sa ?i nu este silit, ci are stapânire peste voin?a sa ?i a hotarât aceasta în inima sa, ca sa-?i ?ina fecioara, bine va face.
1Co 7:38  A?a ca, cel ce î?i marita fecioara bine face; dar cel ce n-o marita ?i mai bine face.
1Co 7:39  Femeia este legata prin lege atâta vreme cât traie?te barbatul ei. Iar daca barbatul ei va muri, este libera sa se marite cu cine vrea, numai întru Domnul.
1Co 7:40  Dar mai fericita este daca ramâne a?a, dupa parerea mea. ?i socot ca ?i eu am Duhul lui Dumnezeu.
1Co 8:1  Cât despre cele jertfite idolilor, ?tim ca to?i avem cuno?tin?a. Cuno?tin?a însa seme?e?te, iar iubirea zide?te.
1Co 8:2  Iar daca i se pare cuiva ca cunoa?te ceva, înca n-a cunoscut cum trebuie sa cunoasca.
1Co 8:3  Dar daca iube?te cineva pe Dumnezeu, acela este cunoscut de El.
1Co 8:4  Iar despre mâncarea celor jertfite idolilor, ?tim ca idolul nu este nimic în lume ?i ca nu este alt Dumnezeu decât Unul singur.
1Co 8:5  Caci de?i sunt a?a-zi?i dumnezei, fie în cer, fie pe pamânt, - precum ?i sunt dumnezei mul?i ?i domni mul?i, S
1Co 8:6  Totu?i, pentru noi, este un singur Dumnezeu, Tatal, din Care sunt toate ?i noi întru El; ?i un singur Domn, Iisus Hristos, prin Care sunt toate ?i noi prin El.
1Co 8:7  Dar nu to?i au cuno?tin?a. Caci unii, din obi?nuin?a de pâna acum cu idolul, manânca din carnuri jertfite idolilor, ?i con?tiin?a lor fiind slaba, se întineaza.
1Co 8:8  Dar nu mâncarea ne va pune înaintea lui Dumnezeu. Ca nici daca vom mânca, nu ne prisose?te, nici daca nu vom mânca, nu ne lipse?te.
1Co 8:9  Dar vede?i ca nu cumva aceasta libertate a voastra sa ajunga poticnire pentru cei slabi.
1Co 8:10  Caci daca cineva te-ar vedea pe tine, cel ce ai cuno?tin?a, ?ezând la masa în templul idolilor, oare con?tiin?a lui, slab fiind el, nu se va întari sa manânce din cele jertfite idolilor?
1Co 8:11  ?i va pieri prin cuno?tin?a ta cel slab, fratele tau, pentru care a murit Hristos!
1Co 8:12  ?i a?a, pacatuind împotriva fra?ilor ?i lovind con?tiin?a lor slaba, pacatui?i fa?a de Hristos.
1Co 8:13  De aceea, daca o mâncare sminte?te pe fratele meu, nu voi mânca în veac carne, ca sa nu aduc sminteala fratelui meu.
1Co 9:1  Oare nu sunt eu liber? Nu sunt eu apostol? N-am vazut eu pe Iisus Domnul nostru? Nu sunte?i voi lucrul meu întru Domnul?
1Co 9:2  Daca altora nu le sunt apostol, voua, negre?it, va sunt. Caci voi sunte?i pecetea apostoliei mele în Domnul.
1Co 9:3  Apararea mea catre cei ce ma judeca aceasta este.
1Co 9:4  N-avem, oare, dreptul, sa mâncam ?i sa bem?
1Co 9:5  N-avem, oare, dreptul sa purtam cu noi o femeie sora, ca ?i ceilal?i apostoli, ca ?i fra?ii Domnului, ca ?i Chefa?
1Co 9:6  Sau numai eu ?i Barnaba nu avem dreptul de a nu lucra?
1Co 9:7  Cine sluje?te vreodata, în oaste, cu solda lui? Cine sade?te vie ?i nu manânca din roada ei? Sau cine pa?te o turma ?i nu manânca din laptele turmei?
1Co 9:8  Nu în felul oamenilor spun eu acestea. Nu spune, oare, ?i legea acestea?
1Co 9:9  Caci în Legea lui Moise este scris: "Sa nu legi gura boului care treiera". Oare de boi se îngrije?te Dumnezeu?
1Co 9:10  Sau în adevar pentru noi zice? Caci pentru noi s-a scris: "Cel ce ara trebuie sa are cu nadejde, ?i cel ce treiera, cu nadejdea ca va avea parte de roade".
1Co 9:11  Daca noi am semanat la voi cele duhovnice?ti, este, oare, mare lucru daca noi vom secera cele pamânte?ti ale voastre?
1Co 9:12  Daca al?ii se bucura de acest drept asupra voastra, oare, nu cu atât mai mult noi? Dar nu ne-am folosit de dreptul acesta, ci toate le rabdam, ca sa nu punem piedica Evangheliei lui Hristos.
1Co 9:13  Au nu ?ti?i ca cei ce savâr?esc cele sfinte manânca de la templu ?i cei ce slujesc altarului au parte de la altar?
1Co 9:14  Tot a?a a poruncit ?i Domnul celor ce propovaduiesc Evanghelia, ca sa traiasca din Evanghelie.
1Co 9:15  Dar eu nu m-am folosit de nimic din acestea ?i nu am scris acestea, ca sa se faca cu mine a?a. Caci mai bine este pentru mine sa mor, decât sa-mi zadarniceasca cineva lauda.
1Co 9:16  Caci daca vestesc Evanghelia, nu-mi este lauda, pentru ca sta asupra mea datoria. Caci, vai mie daca nu voi binevesti!
1Co 9:17  Caci daca fac aceasta de buna voie, am plata; dar daca o fac fara voie, am numai o slujire încredin?ata.
1Co 9:18  Care este, deci, plata mea? Ca, binevestind, pun fara plata Evanghelia lui Hristos înaintea oamenilor, fara sa ma folosesc de dreptul meu din Evanghelie.
1Co 9:19  Caci, de?i sunt liber fa?a de to?i, m-am facut rob tuturor, ca sa dobândesc pe cei mai mul?i;
1Co 9:20  Cu iudeii am fost ca un iudeu, ca sa dobândesc pe iudei; cu cei de sub lege, ca unul de sub lege, de?i eu nu sunt sub lege, ca sa dobândesc pe cei de sub lege;
1Co 9:21  Cu cei ce n-au Legea, m-am facut ca unul fara lege, de?i nu sunt fara Legea lui Dumnezeu, ci având Legea lui Hristos, ca sa dobândesc pe cei ce n-au Legea;
1Co 9:22  Cu cei slabi m-am facut slab, ca pe cei slabi sa-i dobândesc; tuturor toate m-am facut, ca, în orice chip, sa mântuiesc pe unii.
1Co 9:23  Dar toate le fac pentru Evanghelie, ca sa fiu parta? la ea.
1Co 9:24  Nu ?ti?i voi ca acei care alearga în stadion, to?i alearga, dar numai unul ia premiul? Alerga?i a?a ca sa-l lua?i.
1Co 9:25  ?i oricine se lupta se înfrâneaza de la toate. ?i aceia, ca sa ia o cununa stricacioasa, iar noi, nestricacioasa.
1Co 9:26  Eu, deci, a?a alerg, nu ca la întâmplare. A?a ma lupt, nu ca lovind în aer,
1Co 9:27  Ci îmi chinuiesc trupul meu ?i îl supun robiei; ca nu cumva, altora propovaduind, eu însumi sa ma fac netrebnic.
1Co 10:1  Caci nu voiesc, fra?ilor, ca voi sa nu ?ti?i ca parin?ii no?tri au fost to?i sub nor ?i ca to?i au trecut prin mare.
1Co 10:2  ?i to?i, întru Moise, au fost boteza?i în nor ?i în mare.
1Co 10:3  ?i to?i au mâncat aceea?i mâncare duhovniceasca;
1Co 10:4  ?i to?i, aceea?i bautura duhovniceasca au baut, pentru ca beau din piatra duhovniceasca ce îi urma. Iar piatra era Hristos.
1Co 10:5  Dar cei mai mul?i dintre ei nu au placut lui Dumnezeu, caci au cazut în pustie.
1Co 10:6  ?i acestea s-au facut pilde pentru noi, ca sa nu poftim la cele rele, cum au poftit aceia;
1Co 10:7  Nici închinatori la idoli sa nu va face?i, ca unii dintre ei, precum este scris: "A ?ezut poporul sa manânce ?i sa bea ?i s-au sculat la joc";
1Co 10:8  Nici sa ne desfrânam cum s-au desfrânat unii dintre ei, ?i au cazut, într-o zi, douazeci ?i trei de mii;
1Co 10:9  Nici sa ispitim pe Domnul, precum L-au ispitit unii dintre ei ?i au pierit de ?erpi;
1Co 10:10  Nici sa cârti?i, precum au cârtit unii dintre ei ?i au fost nimici?i de catre pierzatorul.
1Co 10:11  ?i toate acestea li s-au întâmplat acelora, ca preînchipuiri ale viitorului, ?i au fost scrise spre pova?uirea noastra, la care au ajuns sfâr?iturile veacurilor.
1Co 10:12  De aceea, cel caruia i se pare ca sta neclintit sa ia seama sa nu cada.
1Co 10:13  Nu v-a cuprins ispita care sa fi fost peste puterea omeneasca. Dar credincios este Dumnezeu; El nu va îngadui ca sa fi?i ispiti?i mai mult decât pute?i, ci odata cu ispita va aduce ?i scaparea din ea, ca sa pute?i rabda.
1Co 10:14  De aceea, iubi?ii mei, fugi?i de închinarea la idoli.
1Co 10:15  Ca unor în?elep?i va vorbesc. Judeca?i voi ce va spun.
1Co 10:16  Paharul binecuvântarii, pe care-l binecuvântam, nu este, oare, împarta?irea cu sângele lui Hristos? Pâinea pe care o frângem nu este, oare, împarta?irea cu trupul lui Hristos?
1Co 10:17  Ca o pâine, un trup, suntem cei mul?i; caci to?i ne împarta?im dintr-o pâine.
1Co 10:18  Privi?i pe Israel dupa trup: Cei care manânca jertfele nu sunt ei, oare, parta?i altarului?
1Co 10:19  Deci ce spun eu? Ca ce s-a jertfit pentru idol e ceva? Sau idolul este ceva?
1Co 10:20  Ci (zic) ca cele ce jertfesc neamurile, jertfesc demonilor ?i nu lui Dumnezeu. ?i nu voiesc ca voi sa fi?i parta?i ai demonilor.
1Co 10:21  Nu pute?i sa be?i paharul Domnului ?i paharul demonilor; nu pute?i sa va împarta?i?i din masa Domnului ?i din masa demonilor.
1Co 10:22  Oare vrem sa mâniem pe Domnul? Nu cumva suntem mai tari decât El?
1Co 10:23  Toate îmi sunt îngaduite, dar nu toate îmi folosesc. Toate îmi sunt îngaduite, dar nu toate zidesc.
1Co 10:24  Nimeni sa nu caute pe ale sale, ci fiecare pe ale aproapelui.
1Co 10:25  Mânca?i tot ce se vinde în macelarie, fara sa întreba?i nimic pentru cugetul vostru.
1Co 10:26  Caci "al Domnului este pamântul ?i plinirea lui".
1Co 10:27  Daca cineva dintre necredincio?i va cheama pe voi la masa ?i voi?i sa va duce?i, mânca?i orice va este pus înainte, fara sa întreba?i nimic pentru con?tiin?a.
1Co 10:28  Dar de va va spune cineva: Aceasta este din jertfa idolilor, sa nu mânca?i pentru cel care v-a spus ?i pentru con?tiin?a.
1Co 10:29  Iar con?tiin?a, zic, nu a ta însu?i, ci a altuia. Caci de ce libertatea mea sa fie judecata de o alta con?tiin?a?
1Co 10:30  Daca eu sunt parta? harului, de ce sa fiu hulit pentru ceea ce aduc mul?umire?
1Co 10:31  De aceea, ori de mânca?i, ori de be?i, ori altceva de face?i, toate spre slava lui Dumnezeu sa le face?i.
1Co 10:32  Nu fi?i piatra de poticnire nici iudeilor, nici elinilor, nici Bisericii lui Dumnezeu,
1Co 10:33  Precum ?i eu plac tuturor în toate, necautând folosul meu, ci pe al celor mul?i, ca sa se mântuiasca.
1Co 11:1  Fi?i urmatori ai mei, precum ?i eu sunt al lui Hristos.
1Co 11:2  Fra?ilor, va laud ca în toate va aduce?i aminte de mine ?i ?ine?i predaniile cum vi le-am dat.
1Co 11:3  Dar voiesc ca voi sa ?ti?i ca Hristos este capul oricarui barbat, iar capul femeii este barbatul, iar capul lui Hristos: Dumnezeu.
1Co 11:4  Orice barbat care se roaga sau prooroce?te, având capul acoperit, necinste?te capul sau.
1Co 11:5  Iar orice femeie care se roaga sau prooroce?te, cu capul neacoperit, î?i necinste?te capul; caci tot una este ca ?i cum ar fi rasa.
1Co 11:6  Caci daca o femeie nu-?i pune val pe cap, atunci sa se ?i tunda. Iar daca este lucru de ru?ine pentru femeie ca sa se tunda ori sa se rada, sa-?i puna val.
1Co 11:7  Caci barbatul nu trebuie sa-?i acopere capul, fiind chip ?i slava a lui Dumnezeu, iar femeia este slava barbatului.
1Co 11:8  Pentru ca nu barbatul este din femeie, ci femeia din barbat.
1Co 11:9  ?i pentru ca n-a fost zidit barbatul pentru femeie, ci femeia pentru barbat.
1Co 11:10  De aceea ?i femeia este datoare sa aiba (semn de) supunere asupra capului ei, pentru îngeri.
1Co 11:11  Totu?i, nici femeia fara barbat, nici barbatul fara femeie, în Domnul.
1Co 11:12  Caci precum femeia este din barbat, a?a ?i barbatul este prin femeie ?i toate sunt de la Dumnezeu.
1Co 11:13  Judeca?i în voi în?iva: Este, oare, cuviincios ca o femeie sa se roage lui Dumnezeu cu capul descoperit?
1Co 11:14  Nu va înva?a oare însa?i firea ca necinste este pentru un barbat sa-?i lase parul lung?
1Co 11:15  ?i ca pentru o femeie, daca î?i lasa parul lung, este cinste? Caci parul i-a fost dat ca acoperamânt.
1Co 11:16  Iar daca se pare cuiva ca aici poate sa ne gaseasca pricina, un astfel de obicei (ca femeile sa se roage cu capul descoperit) noi nu avem, nici Bisericile lui Dumnezeu.
1Co 11:17  ?i aceasta poruncindu-va, nu va laud, fiindca voi va aduna?i nu spre mai bine, ci spre mai rau.
1Co 11:18  Caci mai întâi aud ca atunci când va aduna?i în biserica, între voi sunt dezbinari, ?i în parte cred.
1Co 11:19  Caci trebuie sa fie între voi ?i eresuri, ca sa se învedereze între voi cei încerca?i.
1Co 11:20  Când va aduna?i deci laolalta, nu se poate mânca Cina Domnului;
1Co 11:21  Caci, ?ezând la masa, fiecare se grabe?te sa ia mâncarea sa, încât unuia îi este foame, iar altul se îmbata.
1Co 11:22  N-ave?i, oare, case ca sa mânca?i ?i sa be?i? Sau dispre?ui?i Biserica lui Dumnezeu ?i ru?ina?i pe cei ce nu au? Ce sa va zic? Sa va laud? În aceasta nu va laud.
1Co 11:23  Caci eu de la Domnul am primit ceea ce v-am dat ?i voua: Ca Domnul Iisus, în noaptea în care a fost vândut, a luat pâine,
1Co 11:24  ?i, mul?umind, a frânt ?i a zis: Lua?i, mânca?i; acesta este trupul Meu care se frânge pentru voi. Aceasta sa face?i spre pomenirea Mea.
1Co 11:25  Asemenea ?i paharul dupa Cina, zicând: Acest pahar este Legea cea noua întru sângele Meu. Aceasta sa face?i ori de câte ori ve?i bea, spre pomenirea Mea.
1Co 11:26  Caci de câte ori ve?i mânca aceasta pâine ?i ve?i bea acest pahar, moartea Domnului vesti?i pâna când va veni.
1Co 11:27  Astfel, oricine va mânca pâinea aceasta sau va bea paharul Domnului cu nevrednicie, va fi vinovat fa?a de trupul ?i sângele Domnului.
1Co 11:28  Sa se cerceteze însa omul pe sine ?i a?a sa manânce din pâine ?i sa bea din pahar.
1Co 11:29  Caci cel ce manânca ?i bea cu nevrednicie, osânda î?i manânca ?i bea, nesocotind trupul Domnului.
1Co 11:30  De aceea, mul?i dintre voi sunt neputincio?i ?i bolnavi ?i mul?i au murit.
1Co 11:31  Caci de ne-am fi judecat noi în?ine, nu am mai fi judeca?i.
1Co 11:32  Dar, fiind judeca?i de Domnul, suntem pedepsi?i, ca sa nu fim osândi?i împreuna cu lumea.
1Co 11:33  De aceea, fra?ii mei, când va aduna?i ca sa mânca?i, a?tepta?i-va unii pe al?ii.
1Co 11:34  Iar daca îi este cuiva foame, sa manânce acasa, ca sa nu va aduna?i spre osânda. Celelalte însa le voi rândui când voi veni.
1Co 12:1  Iar cât prive?te darurile duhovnice?ti nu vreau, fra?ilor, sa fi?i în necuno?tin?a.
1Co 12:2  ?ti?i ca, pe când era?i pagâni, va ducea?i la idolii cei mu?i, ca ?i cum era?i mâna?i.
1Co 12:3  De aceea, va fac cunoscut ca precum nimeni, graind în Duhul lui Dumnezeu, nu zice: Anatema fie Iisus! - tot a?a nimeni nu poate sa zica: Domn este Iisus, - decât în Duhul Sfânt.
1Co 12:4  Darurile sunt felurite, dar acela?i Duh.
1Co 12:5  ?i felurite slujiri sunt, dar acela?i Domn.
1Co 12:6  ?i lucrarile sunt felurite, dar este acela?i Dumnezeu, care lucreaza toate în to?i.
1Co 12:7  ?i fiecaruia se da aratarea Duhului spre folos.
1Co 12:8  Ca unuia i se da prin Duhul Sfânt cuvânt de în?elepciune, iar altuia, dupa acela?i Duh, cuvântul cuno?tin?ei.
1Co 12:9  ?i unuia i se da întru acela?i Duh credin?a, iar altuia, darurile vindecarilor, întru acela?i Duh;
1Co 12:10  Unuia faceri de minuni, iar altuia proorocie; unuia deosebirea duhurilor, iar altuia feluri de limbi ?i altuia talmacirea limbilor.
1Co 12:11  ?i toate acestea le lucreaza unul ?i acela?i Duh, împar?ind fiecaruia deosebi, dupa cum voie?te.
1Co 12:12  Caci precum trupul unul este, ?i are madulare multe, iar toate madularele trupului, multe fiind, sunt un trup, a?a ?i Hristos.
1Co 12:13  Pentru ca într-un Duh ne-am botezat noi to?i, ca sa fim un singur trup, fie iudei, fie elini, fie robi, fie liberi, ?i to?i la un Duh ne-am adapat.
1Co 12:14  Caci ?i trupul nu este un madular, ci multe.
1Co 12:15  Daca piciorul ar zice: Fiindca nu sunt mâna nu sunt din trup, pentru aceasta nu este el din trup?
1Co 12:16  ?i urechea daca ar zice: Fiindca nu sunt ochi, nu fac parte din trup, - pentru aceasta nu este ea din trup?
1Co 12:17  Daca tot trupul ar fi ochi, unde ar fi auzul? ?i daca ar fi tot auz, unde ar fi mirosul?
1Co 12:18  Dar acum Dumnezeu a pus madularele, pe fiecare din ele, în trup, cum a voit.
1Co 12:19  Daca toate ar fi un singur madular, unde ar fi trupul?
1Co 12:20  Dar acum sunt multe madulare, însa un singur trup.
1Co 12:21  ?i nu poate ochiul sa zica mâinii: N-am trebuin?a de tine; sau, iara?i capul sa zica picioarelor: N-am trebuin?a de voi.
1Co 12:22  Ci cu mult mai mult madularele trupului, care par a fi mai slabe, sunt mai trebuincioase.
1Co 12:23  ?i pe cele ale trupului care ni se par ca sunt mai de necinste, pe acelea cu mai multa evlavie le îmbracam; ?i cele necuviincioase ale noastre au mai multa cuviin?a.
1Co 12:24  Iar cele cuviincioase ale noastre n-au nevoie de acoperamânt. Dar Dumnezeu a întocmit astfel trupul, dând mai multa cinste celui caruia îi lipse?te,
1Co 12:25  Ca sa nu fie dezbinare în trup, ci madularele sa se îngrijeasca deopotriva unele de altele.
1Co 12:26  ?i daca un madular sufera, toate madularele sufera împreuna; ?i daca un madular este cinstit, toate madularele se bucura împreuna.
1Co 12:27  Iar voi sunte?i trupul lui Hristos ?i madulare (fiecare) în parte.
1Co 12:28  ?i pe unii i-a pus Dumnezeu, în Biserica: întâi apostoli, al doilea prooroci, al treilea înva?atori; apoi pe cei ce au darul de a face minuni; apoi darurile vindecarilor, ajutorarile, cârmuirile, felurile limbilor.
1Co 12:29  Oare to?i sunt apostoli? Oare to?i sunt prooroci? Oare to?i înva?atori? Oare to?i au putere sa savâr?easca minuni?
1Co 12:30  Oare to?i au darurile vindecarilor? Oare to?i vorbesc în limbi? Oare to?i pot sa talmaceasca?
1Co 12:31  Râvni?i însa la darurile cele mai bune. ?i va arat înca o cale care le întrece pe toate:
1Co 13:1  De a? grai în limbile oamenilor ?i ale îngerilor, iar dragoste nu am, facutu-m-am arama sunatoare ?i chimval rasunator.
1Co 13:2  ?i de a? avea darul proorociei ?i tainele toate le-a? cunoa?te ?i orice ?tiin?a, ?i de a? avea atâta credin?a încât sa mut ?i mun?ii, iar dragoste nu am, nimic nu sunt.
1Co 13:3  ?i de a? împar?i toata avu?ia mea ?i de a? da trupul meu ca sa fie ars, iar dragoste nu am, nimic nu-mi folose?te.
1Co 13:4  Dragostea îndelung rabda; dragostea este binevoitoare, dragostea nu pizmuie?te, nu se lauda, nu se trufe?te.
1Co 13:5  Dragostea nu se poarta cu necuviin?a, nu cauta ale sale, nu se aprinde de mânie, nu gânde?te raul.
1Co 13:6  Nu se bucura de nedreptate, ci se bucura de adevar.
1Co 13:7  Toate le sufera, toate le crede, toate le nadajduie?te, toate le rabda.
1Co 13:8  Dragostea nu cade niciodata. Cât despre proorocii - se vor desfiin?a; darul limbilor va înceta; ?tiin?a se va sfâr?i;
1Co 13:9  Pentru ca în parte cunoa?tem ?i în parte proorocim.
1Co 13:10  Dar când va veni ceea ce e desavâr?it, atunci ceea ce este în parte se va desfiin?a.
1Co 13:11  Când eram copil, vorbeam ca un copil, sim?eam ca un copil; judecam ca un copil; dar când m-am facut barbat, am lepadat cele ale copilului.
1Co 13:12  Caci vedem acum ca prin oglinda, în ghicitura, iar atunci, fa?a catre fa?a; acum cunosc în parte, dar atunci voi cunoa?te pe deplin, precum am fost cunoscut ?i eu.
1Co 13:13  ?i acum ramân acestea trei: credin?a, nadejdea ?i dragostea. Iar mai mare dintre acestea este dragostea.
1Co 14:1  Cauta?i dragostea. Râvni?i însa cele duhovnice?ti, dar mai ales ca sa prooroci?i.
1Co 14:2  Pentru ca cel ce vorbe?te într-o limba straina nu vorbe?te oamenilor, ci lui Dumnezeu; ?i nimeni nu-l în?elege, fiindca el, în duh, graie?te taine.
1Co 14:3  Cel ce prooroce?te vorbe?te oamenilor, spre zidire, îndemn ?i mângâiere.
1Co 14:4  Cel ce graie?te într-o limba straina pe sine singur se zide?te, iar cel ce prooroce?te zide?te Biserica.
1Co 14:5  Voiesc ca voi to?i sa grai?i în limbi; dar mai cu seama sa prooroci?i. Cel ce prooroce?te e mai mare decât cel ce graie?te în limbi, afara numai daca talmace?te, ca Biserica sa ia întarire.
1Co 14:6  Iar acum, fra?ilor, daca a? veni la voi, graind în limbi, de ce folos v-a? fi, daca nu v-a? vorbi - sau în descoperire, sau în cuno?tin?a, sau în proorocie, sau în înva?atura?
1Co 14:7  Ca precum cele neînsufle?ite, care dau sunet, fie fluier, fie chitara, de nu vor da sunete deosebite, cum se va cunoa?te ce este din fluier, sau ce este din chitara?
1Co 14:8  ?i daca trâmbi?a va da sunet nelamurit, cine se va pregati de razboi?
1Co 14:9  A?a ?i voi: Daca prin limba nu ve?i da cuvânt lesne de în?eles, cum se va cunoa?te ce a?i grait? Ve?i fi ni?te oameni care vorbesc în vânt.
1Co 14:10  Sunt a?a de multe feluri de limbi în lume, dar nici una din ele nu este fara în?elesul ei.
1Co 14:11  Deci daca nu voi ?ti în?elesul cuvintelor, voi fi barbar pentru cel care vorbe?te, ?i cel care vorbe?te barbar pentru mine.
1Co 14:12  A?a ?i voi, de vreme ce sunte?i râvnitori dupa cele duhovnice?ti, cauta?i sa prisosi?i în ele, spre zidirea Bisericii.
1Co 14:13  De aceea, cel ce graie?te într-o limba straina sa se roage ca sa ?i talmaceasca.
1Co 14:14  Caci, daca ma rog într-o limba straina, duhul meu se roaga, dar mintea mea este neroditoare.
1Co 14:15  Atunci ce voi face? Ma voi ruga cu duhul, dar ma voi ruga ?i cu mintea; voi cânta cu duhul, dar voi cânta ?i cu mintea.
1Co 14:16  Fiindca daca vei binecuvânta cu duhul, cum va raspunde omul simplu "Amin" la mul?umirea ta, de vreme ce el nu ?tie ce zici?
1Co 14:17  Caci tu, într-adevar, mul?ume?ti bine, dar celalalt nu se zide?te.
1Co 14:18  Mul?umesc Dumnezeului meu, ca vorbesc în limbi mai mult decât voi to?i;
1Co 14:19  Dar în Biserica vreau sa graiesc cinci cuvinte cu mintea mea, ca sa înva? ?i pe al?ii, decât zeci de mii de cuvinte într-o limba straina.
1Co 14:20  Fra?ilor, nu fi?i copii la minte. Fi?i copii când e vorba de rautate. La minte însa, fi?i desavâr?i?i.
1Co 14:21  În Lege este scris: "Voi grai acestui popor în alte limbi ?i prin buzele altora, ?i nici a?a nu vor asculta de Mine, zice Domnul".
1Co 14:22  A?a ca vorbirea în limbi este semn nu pentru cei credincio?i ci pentru cei necredincio?i; iar proorocia nu pentru cei necredincio?i, ci pentru cei ce cred.
1Co 14:23  Deci, daca s-ar aduna Biserica toata laolalta ?i to?i ar vorbi în limbi ?i ar intra ne?tiutori sau necredincio?i, nu vor zice, oare, ca sunte?i nebuni?
1Co 14:24  Iar daca to?i ar prooroci ?i ar intra vreun necredincios sau vreun ne?tiutor, el este dovedit de to?i, el este judecat de to?i;
1Co 14:25  Cele ascunse ale inimii lui se învedereaza, ?i astfel, cazând cu fa?a la pamânt, se va închina lui Dumnezeu, marturisind ca Dumnezeu este într-adevar printre voi.
1Co 14:26  Ce este deci, fra?ilor? Când va aduna?i împreuna, fiecare din voi are psalm, are înva?atura, are descoperire, are limba, are talmacire: toate spre zidire sa se faca.
1Co 14:27  Daca graie?te cineva într-o limba straina, sa fie câte doi, sau cel mult trei ?i pe rând sa graiasca ?i unul sa talmaceasca.
1Co 14:28  Iar daca nu e talmacitor, sa taca în biserica ?i sa-?i graiasca numai lui ?i lui Dumnezeu.
1Co 14:29  Iar proorocii sa vorbeasca doi sau trei, iar ceilal?i sa judece.
1Co 14:30  Iar daca se va descoperi ceva altuia care ?ade, sa taca cei dintâi.
1Co 14:31  Caci pute?i sa prooroci?i to?i câte unul, ca to?i sa înve?e ?i to?i sa se mângâie.
1Co 14:32  ?i duhurile proorocilor se supun proorocilor.
1Co 14:33  Pentru ca Dumnezeu nu este al neorânduielii, ci al pacii.
1Co 14:34  Ca în toate Bisericile sfin?ilor, femeile voastre sa taca în biserica, caci lor nu le este îngaduit sa vorbeasca, ci sa se supuna, precum zice ?i Legea.
1Co 14:35  Iar daca voiesc sa înve?e ceva, sa întrebe acasa pe barba?ii lor, caci este ru?inos ca femeile sa vorbeasca în biserica.
1Co 14:36  Oare de la voi a ie?it cuvântul lui Dumnezeu sau a ajuns numai la voi?
1Co 14:37  Daca i se pare cuiva ca este prooroc sau om duhovnicesc, sa cunoasca ca cele ce va scriu sunt porunci ale Domnului.
1Co 14:38  Iar daca cineva nu vrea sa ?tie, sa nu ?tie.
1Co 14:39  A?a ca, fra?ii mei, râvni?i a prooroci ?i nu opri?i sa se graiasca în limbi.
1Co 14:40  Dar toate sa se faca cu cuviin?a ?i dupa rânduiala.
1Co 15:1  Va aduc aminte, fra?ilor, Evanghelia pe care v-am binevestit-o, pe care a?i ?i primit-o, întru care ?i sta?i,
1Co 15:2  Prin care ?i sunte?i mântui?i; cu ce cuvânt v-am binevestit-o - daca o ?ine?i cu tarie, afara numai daca n-a?i crezut în zadar S
1Co 15:3  Caci v-am dat, întâi de toate, ceea ce ?i eu am primit, ca Hristos a murit pentru pacatele noastre dupa Scripturi;
1Co 15:4  ?i ca a fost îngropat ?i ca a înviat a treia zi, dupa Scripturi;
1Co 15:5  ?i ca S-a aratat lui Chefa, apoi celor doisprezece;
1Co 15:6  În urma S-a aratat deodata la peste cinci sute de fra?i, dintre care cei mai mul?i traiesc pâna astazi, iar unii au ?i adormit;
1Co 15:7  Dupa aceea S-a aratat lui Iacov, apoi tuturor apostolilor;
1Co 15:8  Iar la urma tuturor, ca unui nascut înainte de vreme, mi S-a aratat ?i mie.
1Co 15:9  Caci eu sunt cel mai mic dintre apostoli, care nu sunt vrednic sa ma numesc apostol, pentru ca am prigonit Biserica lui Dumnezeu.
1Co 15:10  Dar prin harul lui Dumnezeu sunt ceea ce sunt; ?i harul Lui care este în mine n-a fost în zadar, ci m-am ostenit mai mult decât ei to?i. Dar nu eu, ci harul lui Dumnezeu care este cu mine.
1Co 15:11  Deci ori eu, ori aceia, a?a propovaduim ?i voi a?a a?i crezut.
1Co 15:12  Iar daca se propovaduie?te ca Hristos a înviat din mor?i, cum zic unii dintre voi ca nu este înviere a mor?ilor?
1Co 15:13  Daca nu este înviere a mor?ilor, nici Hristos n-a înviat.
1Co 15:14  ?i daca Hristos n-a înviat, zadarnica este atunci propovaduirea noastra, zadarnica este ?i credin?a voastra.
1Co 15:15  Ne aflam înca ?i martori mincino?i ai lui Dumnezeu, pentru ca am marturisit împotriva lui Dumnezeu ca a înviat pe Hristos, pe Care nu L-a înviat, daca deci mor?ii nu înviaza.
1Co 15:16  Caci daca mor?ii nu înviaza, nici Hristos n-a înviat.
1Co 15:17  Iar daca Hristos n-a înviat, zadarnica este credin?a voastra, sunte?i înca în pacatele voastre;
1Co 15:18  ?i atunci ?i cei ce au adormit în Hristos au pierit.
1Co 15:19  Iar daca nadajduim în Hristos numai în via?a aceasta, suntem mai de plâns decât to?i oamenii.
1Co 15:20  Dar acum Hristos a înviat din mor?i, fiind începatura (a învierii) celor adormi?i.
1Co 15:21  Ca de vreme ce printr-un om a venit moartea, tot printr-un om ?i învierea mor?ilor.
1Co 15:22  Caci, precum în Adam to?i mor, a?a ?i în Hristos to?i vor învia.
1Co 15:23  Dar fiecare în rândul cetei sale: Hristos începatura, apoi cei ai lui Hristos, la venirea Lui,
1Co 15:24  Dupa aceea, sfâr?itul, când Domnul va preda împara?ia lui Dumnezeu ?i Tatalui, când va desfiin?a orice domnie ?i orice stapânire ?i orice putere.
1Co 15:25  Caci El trebuie sa împara?easca pâna ce va pune pe to?i vrajma?ii Sai sub picioarele Sale.
1Co 15:26  Vrajma?ul cel din urma, care va fi nimicit, este moartea.
1Co 15:27  "Caci toate le-a supus sub picioarele Lui". Dar când zice: "Ca toate I-au fost supuse Lui" - învederat este ca afara de Cel care I-a supus Lui toate.
1Co 15:28  Iar când toate vor fi supuse Lui, atunci ?i Fiul însu?i Se va supune Celui ce I-a supus Lui toate, ca Dumnezeu sa fie toate în to?i.
1Co 15:29  Fiindca ce vor face cei care se boteaza pentru mor?i? Daca mor?ii nu înviaza nicidecum, pentru ce se mai boteaza pentru ei?
1Co 15:30  De ce mai suntem ?i noi în primejdie în tot ceasul?
1Co 15:31  Mor în fiecare zi! V-o spun, fra?ilor, pe lauda pe care o am pentru voi, în Hristos Iisus, Domnul nostru.
1Co 15:32  Daca m-am luptat, ca om, cu fiarele în Efes, care îmi este folosul? Daca mor?ii nu înviaza, sa bem ?i sa mâncam, caci mâine vom muri!
1Co 15:33  Nu va lasa?i în?ela?i. Tovara?iile rele strica obiceiurile bune.
1Co 15:34  Trezi?i-va cum se cuvine ?i nu pacatui?i. Caci unii nu au cuno?tin?a de Dumnezeu; o spun spre ru?inea voastra.
1Co 15:35  Dar va zice cineva: Cum înviaza mor?ii? ?i cu ce trup au sa vina?
1Co 15:36  Nebun ce e?ti! Tu ce semeni nu da via?a, daca nu va fi murit.
1Co 15:37  ?i ceea ce semeni nu este trupul ce va sa fie, ci graunte gol, poate de grâu, sau de altceva din celelalte;
1Co 15:38  Iar Dumnezeu îi da un trup, precum a voit, ?i fiecarei semin?e un trup al sau.
1Co 15:39  Nu toate trupurile sunt acela?i trup, ci unul este trupul oamenilor ?i altul este trupul dobitoacelor ?i altul este trupul pasarilor ?i altul este trupul pe?tilor.
1Co 15:40  Sunt ?i trupuri cere?ti ?i trupuri pamânte?ti; dar alta este slava celor cere?ti ?i alta a celor pamânte?ti.
1Co 15:41  Alta este stralucirea soarelui ?i alta stralucirea lunii ?i alta stralucirea stelelor. Caci stea de stea se deosebe?te în stralucire.
1Co 15:42  A?a este ?i învierea mor?ilor: Se seamana (trupul) întru stricaciune, înviaza întru nestricaciune;
1Co 15:43  Se seamana întru necinste, înviaza întru slava, se seamana întru slabiciune, înviaza întru putere;
1Co 15:44  Se seamana trup firesc, înviaza trup duhovnicesc. Daca este trup firesc, este ?i trup duhovnicesc.
1Co 15:45  Precum ?i este scris: "Facutu-s-a omul cel dintâi, Adam, cu suflet viu; iar Adam cel de pe urma cu duh datator de via?a";
1Co 15:46  Dar nu este întâi cel duhovnicesc, ci cel firesc, apoi cel duhovnicesc.
1Co 15:47  Omul cel dintâi este din pamânt, pamântesc; omul cel de-al doilea este din cer.
1Co 15:48  Cum este cel pamântesc, a?a sunt ?i cei pamânte?ti; ?i cum este cel ceresc, a?a sunt ?i cei cere?ti.
1Co 15:49  ?i dupa cum am purtat chipul celui pamântesc, sa purtam ?i chipul celui ceresc.
1Co 15:50  Aceasta însa zic, fra?ilor: Carnea ?i sângele nu pot sa mo?teneasca împara?ia lui Dumnezeu, nici stricaciunea nu mo?tene?te nestricaciunea.
1Co 15:51  Iata, taina va spun voua: Nu to?i vom muri, dar to?i ne vom schimba,
1Co 15:52  Deodata, într-o clipeala de ochi la trâmbi?a cea de apoi. Caci trâmbi?a va suna ?i mor?ii vor învia nestricacio?i, iar noi ne vom schimba.
1Co 15:53  Caci trebuie ca acest trup stricacios sa se îmbrace în nestricaciune ?i acest (trup) muritor sa se îmbrace în nemurire.
1Co 15:54  Iar când acest (trup) stricacios se va îmbraca în nestricaciune ?i acest (trup) muritor se va îmbraca în nemurire, atunci va fi cuvântul care este scris: "Moartea a fost înghi?ita de biruin?a.
1Co 15:55  Unde î?i este, moarte, biruin?a ta? Unde î?i este, moarte, boldul tau?".
1Co 15:56  ?i boldul mor?ii este pacatul, iar puterea pacatului este legea.
1Co 15:57  Dar sa dam mul?umire lui Dumnezeu, Care ne-a dat biruin?a prin Domnul nostru Iisus Hristos!
1Co 15:58  Drept aceea, fra?ii mei iubi?i, fi?i tari, neclinti?i, sporind totdeauna în lucrul Domnului, ?tiind ca osteneala voastra nu este zadarnica în Domnul.
1Co 16:1  Cât despre strângerea de ajutoare pentru sfin?i, precum am rânduit pentru Bisericile Galatiei, a?a sa face?i ?i voi.
1Co 16:2  În ziua întâi a saptamânii (Duminica), fiecare dintre voi sa-?i puna deoparte, strângând cât poate, ca sa nu se faca strângerea abia atunci când voi veni.
1Co 16:3  Iar când voi veni, pe cei pe care îi ve?i socoti, pe aceia îi voi trimite cu scrisori sa duca darul vostru la Ierusalim.
1Co 16:4  ?i de se va cuveni sa merg ?i eu, vor merge împreuna cu mine.
1Co 16:5  Ci voi veni la voi, când voi trece prin Macedonia, caci prin Macedonia trec.
1Co 16:6  La voi ma voi opri, poate, sau voi ?i ierna, ca sa ma petrece?i în calatoria ce voi face.
1Co 16:7  Caci nu vreau sa va vad acum numai în treacat, ci nadajduiesc sa ramân la voi câtava vreme, daca va îngadui Domnul.
1Co 16:8  Voi ramâne însa în Efes, pâna la praznicul Cincizecimii.
1Co 16:9  Caci mi s-a deschis u?a mare spre lucru mult, dar sunt mul?i potrivnici.
1Co 16:10  Iar de va veni Timotei, vede?i sa fie fara teama la voi, caci lucreaza ca ?i mine lucrul Domnului.
1Co 16:11  Nimeni deci sa nu-l dispre?uiasca; ci sa-l petrece?i cu pace, ca sa vina la mine; ca îl a?tept cu fra?ii.
1Co 16:12  Cât despre fratele Apollo, l-am rugat mult sa vina la voi cu fra?ii; totu?i nu i-a fost voia sa vina acum. Ci va veni când va gasi prilej.
1Co 16:13  Priveghea?i, sta?i tari în credin?a, îmbarbata?i-va, întari?i-va.
1Co 16:14  Toate ale voastre cu dragoste sa se faca.
1Co 16:15  Va îndemn însa, fra?ilor, - ?ti?i casa lui ?tefanas, ca este pârga Ahaei ?i ca spre slujirea sfin?ilor s-au rânduit pe ei în?i?i S
1Co 16:16  Ca ?i voi sa va supune?i unora ca ace?tia ?i oricui lucreaza ?i se ostene?te împreuna cu ei.
1Co 16:17  Ma bucur de venirea lui ?tefanas, a lui Fortunat ?i a lui Ahaic, pentru ca ace?tia au împlinit lipsa voastra.
1Co 16:18  ?i au lini?tit duhul meu ?i al vostru. Cunoa?te?i bine deci pe unii ca ace?tia.
1Co 16:19  Va îmbra?i?eaza Bisericile Asiei. Va îmbra?i?eaza mult, în Domnul, Acvila ?i Priscila, împreuna cu Biserica din casa lor.
1Co 16:20  Va îmbra?i?eaza fra?ii to?i. Îmbra?i?a?i-va unii pe al?ii cu sarutare sfânta.
1Co 16:21  Salutarea cu mâna mea, Pavel.
1Co 16:22  Cel ce nu iube?te pe Domnul sa fie anatema! Maran atha! (Domnul vine).
1Co 16:23  Harul Domnului Iisus Hristos cu voi.
1Co 16:24  Dragostea mea cu voi to?i, în Hristos Iisus! Amin.


\end{document}