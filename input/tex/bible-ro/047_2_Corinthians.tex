\begin{document}

\title{2 Corinthians}

2Co 1:1  Pavel, apostol al lui Hristos Iisus, prin voin?a lui Dumnezeu, ?i Timotei, fratele: Bisericii lui Dumnezeu celei din Corint, împreuna cu to?i sfin?ii care sunt în toata Ahaia:
2Co 1:2  Har voua ?i pace de la Dumnezeu Tatal nostru ?i de la Domnul Iisus Hristos.
2Co 1:3  Binecuvântat este Dumnezeu ?i Tatal Domnului nostru Iisus Hristos, Parintele îndurarilor ?i Dumnezeul a toata mângâierea,
2Co 1:4  Cel ce ne mângâie pe noi în tot necazul nostru, ca sa putem sa mângâiem ?i noi pe cei care se afla în tot necazul, prin mângâierea cu care noi în?ine suntem mângâia?i de Dumnezeu.
2Co 1:5  Ca precum prisosesc patimirile lui Hristos întru noi, a?a prisose?te prin Hristos ?i mângâierea noastra.
2Co 1:6  Deci fie ca suntem strâmtora?i, este pentru a voastra mângâiere ?i mântuire, fie ca suntem mângâia?i, este pentru a voastra mângâiere, care va da putere sa îndura?i cu rabdare acelea?i suferin?e pe care le suferim ?i noi.
2Co 1:7  ?i nadejdea noastra este tare pentru voi, ?tiind ca precum sunte?i parta?i suferin?elor, a?a ?i mângâierii.
2Co 1:8  Caci nu voim, fra?ilor, ca voi sa nu ?ti?i de necazul nostru, care ni s-a facut în Asia, ca peste masura, peste puteri, am fost îngreuia?i, încât nu mai nadajduiam sa mai scapam cu via?a.
2Co 1:9  Ci noi, în noi în?ine, ne-am socotit ca osândi?i la moarte, ca sa nu ne punem încrederea în noi, ci în Dumnezeu, Cel ce înviaza pe cei mor?i,
2Co 1:10  Care ne-a izbavit pe noi dintr-o moarte ca aceasta ?i ne izbave?te ?i în Care nadajduim ca înca ne va mai izbavi,
2Co 1:11  Ajutându-ne ?i voi cu rugaciunea pentru noi, a?a încât darul acesta facut noua, prin rugaciunea multora, sa ne fie prilej de mul?umire adusa de catre mul?i, pentru noi.
2Co 1:12  Caci lauda noastra aceasta este: marturia con?tiin?ei noastre ca am umblat în lume, ?i mai ales la voi, în sfin?enie ?i în cura?ie dumnezeiasca, nu în în?elepciune trupeasca, ci în harul lui Dumnezeu.
2Co 1:13  Caci nu va scriem voua altele decât cele ce citi?i ?i în?elege?i. ?i am nadejde ca pâna în sfâr?it ve?i în?elege;
2Co 1:14  Dupa cum ne-a?i ?i în?eles în parte, - ca noi suntem lauda voastra, precum ?i voi lauda noastra, în ziua Domnului nostru Iisus.
2Co 1:15  Cu aceasta încredin?are voiam sa vin întâi la voi, ca sa ave?i bucurie a doua oara,
2Co 1:16  ?i sa trec pe la voi în Macedonia ?i din Macedonia iara?i sa vin la voi ?i sa fiu petrecut de voi în Iudeea.
2Co 1:17  Deci, aceasta voind, m-am purtat, oare, cu u?urin?a? Sau cele ce hotarasc, le hotarasc trupe?te ca la mine da, da sa fie ?i nu, nu?
2Co 1:18  Credincios este Dumnezeu, ca n-a fost cuvântul nostru catre voi da ?i nu.
2Co 1:19  Fiul lui Dumnezeu, Iisus Hristos, Cel propovaduit voua - prin noi, prin mine, prin Silvan ?i prin Timotei - nu a fost da ?i nu, ci da a fost în El.
2Co 1:20  Caci toate fagaduin?ele lui Dumnezeu, în El, sunt da; ?i prin El, amin, spre slava lui Dumnezeu prin noi.
2Co 1:21  Iar Cel ce ne întare?te pe noi împreuna cu voi, în Hristos, ?i ne-a uns pe noi este Dumnezeu,
2Co 1:22  Care ne-a ?i pecetluit pe noi ?i a dat arvuna Duhului, în inimile noastre.
2Co 1:23  ?i eu chem pe Dumnezeu marturie asupra sufletului meu, ca din cru?are pentru voi n-am venit înca la Corint.
2Co 1:24  Nu ca doar avem stapânire peste credin?a voastra, dar suntem împreuna-lucratori ai bucuriei voastre; caci sta?i tari în credin?a.
2Co 2:1  ?i am judecat în mine aceasta, sa nu vin iara?i la voi cu întristare.
2Co 2:2  Caci daca eu va întristez, cine este cel care sa ma înveseleasca, daca nu cel întristat de mine?
2Co 2:3  ?i v-am scris voua aceasta, ca nu cumva la venirea mea sa am întristare de la aceia care trebuie sa ma bucure, fiind încredin?at despre voi to?i ca bucuria mea este ?i a voastra a tuturor.
2Co 2:4  Caci din multa suparare ?i cu inima strânsa de durere, v-am scris cu multe lacrimi, nu ca sa va întrista?i, ci ca sa cunoa?te?i dragostea pe care o am cu prisosin?a catre voi.
2Co 2:5  ?i daca m-a întristat cineva, nu pe mine m-a întristat, ci în parte - ca sa nu spun mai mult - pe voi to?i.
2Co 2:6  Destul este pentru un astfel de om pedeapsa aceasta data de catre cei mai mul?i.
2Co 2:7  A?a încât voi, dimpotriva, mai bine sa-l ierta?i ?i sa-l mângâia?i, ca sa nu fie cople?it de prea multa întristare unul ca acesta.
2Co 2:8  De aceea va îndemn sa întari?i în el dragostea.
2Co 2:9  Caci pentru aceasta v-am ?i scris, ca sa cunosc încercarea voastra, daca sunte?i ascultatori în toate.
2Co 2:10  Iar cui îi ierta?i ceva, îi iert ?i eu; pentru ca ?i eu, daca am iertat ceva, am iertat pentru voi, în fa?a lui Hristos,
2Co 2:11  Ca sa nu ne lasam covâr?i?i de satana, caci gândurile lui nu ne sunt necunoscute.
2Co 2:12  ?i venind eu la Troa, pentru Evanghelia lui Hristos, ?i u?a fiindu-mi deschisa în Domnul,
2Co 2:13  N-am avut odihna în duhul meu, fiindca n-am gasit pe Tit, fratele meu, ci despar?indu-ma de ei, am plecat în Macedonia.
2Co 2:14  Mul?umire fie adusa deci lui Dumnezeu, Celui ce ne face pururea biruitori în Hristos ?i descopera prin noi, în tot locul, mireasma cuno?tin?ei Sale!
2Co 2:15  Pentru ca suntem lui Dumnezeu buna mireasma a lui Hristos între cei ce se mântuiesc ?i între cei ce pier;
2Co 2:16  Unora, adica, mireasma a mor?ii spre moarte, iar altora mireasma a vie?ii spre via?a. ?i pentru acestea, cine e destoinic?
2Co 2:17  Caci nu suntem ca cei mul?i, care strica cuvântul lui Dumnezeu, ci graim ca din cura?ia inimii, ca de la Dumnezeu înaintea lui Dumnezeu, în Hristos.
2Co 3:1  Au doara începem iara?i sa spunem cine suntem? Sau nu cumva avem nevoie - cum au unii - de scrisori de lauda catre voi sau de la voi?
2Co 3:2  Scrisoarea noastra sunte?i voi, scrisa în inimile noastre, cunoscuta ?i citita de to?i oamenii,
2Co 3:3  Aratându-va ca sunte?i scrisoare a lui Hristos, slujita de noi, scrisa nu cu cerneala, ci cu Duhul Dumnezeului celui viu, nu pe table de piatra, ci pe tablele de carne ale inimii.
2Co 3:4  ?i o astfel de încredere avem în Hristos fa?a de Dumnezeu;
2Co 3:5  Nu ca de la noi în?ine suntem destoinici sa cugetam ceva ca de la noi în?ine, ci destoinicia noastra este de la Dumnezeu,
2Co 3:6  Cel ce ne-a învrednicit sa fim slujitori ai Noului Testament, nu ai literei, ci ai duhului; pentru ca litera ucide, iar duhul face viu.
2Co 3:7  Iar daca slujirea cea spre moarte, sapata în litere, pe piatra, s-a facut întru slava, încât fiii lui Israel nu puteau sa-?i a?inteasca ochii la fa?a lui Moise, din pricina slavei celei trecatoare a fe?ei lui,
2Co 3:8  Cum sa nu fie mai mult întru slava slujirea Duhului?
2Co 3:9  Caci de a avut parte de slava slujirea care aduce osânda, cu mult mai mult prisose?te în slava slujirea drepta?ii.
2Co 3:10  ?i nici macar nu este slavit ceea ce era slavit în aceasta privin?a, fa?a de slava cea covâr?itoare.
2Co 3:11  Caci daca ce este trecator s-a savâr?it prin slava, cu atât mai mult ce e netrecator va fi în  slava.
2Co 3:12  Având deci o astfel de nadejde, noi lucram cu multa îndrazneala,
2Co 3:13  ?i nu ca Moise, care î?i punea un val pe fa?a sa, ca fiii lui Israel sa nu priveasca sfâr?itul a ceea ce era trecator.
2Co 3:14  Dar min?ile lor s-au învârto?at, caci pâna în ziua de azi, la citirea Vechiului Testament, ramâne acela?i val, neridicându-se, caci el se desfiin?eaza prin Hristos;
2Co 3:15  Ci pâna astazi, când se cite?te Moise, sta un val pe inima lor;
2Co 3:16  Iar când se vor întoarce catre Domnul, valul se va ridica.
2Co 3:17  Domnul este Duh, ?i unde este Duhul Domnului, acolo este libertate.
2Co 3:18  Iar noi to?i, privind ca în oglinda, cu fa?a descoperita, slava Domnului, ne prefacem în acela?i chip din slava în slava, ca de la Duhul Domnului.
2Co 4:1  De aceea, având aceasta slujire, dupa cum am fost milui?i, nu ne pierdem nadejdea,
2Co 4:2  Ci ne-am lepadat de cele ascunse ale ru?inii, neumblând în vicle?ug, nici stricând cuvântul lui Dumnezeu, ci facându-ne cunoscu?i prin aratarea adevarului fa?a de orice con?tiin?a omeneasca înaintea lui Dumnezeu.
2Co 4:3  Iar daca Evanghelia noastra este înca acoperita, este pentru cei pierdu?i,
2Co 4:4  În care Dumnezeul veacului acestuia a orbit min?ile necredincio?ilor, ca sa nu le lumineze lumina Evangheliei slavei lui Hristos, Care este chipul lui Dumnezeu.
2Co 4:5  Caci nu ne propovaduim pe noi în?ine, ci pe Hristos Iisus, Domnul, iar noi în?ine suntem slugile voastre, pentru Iisus.
2Co 4:6  Fiindca Dumnezeu, Care a zis: "Straluceasca, din întuneric, lumina" - El a stralucit în inimile noastre, ca sa straluceasca cuno?tin?a slavei lui Dumnezeu, pe fa?a lui Hristos.
2Co 4:7  ?i avem comoara aceasta în vase de lut, ca sa se învedereze ca puterea covâr?itoare este a lui Dumnezeu ?i nu de la noi,
2Co 4:8  În toate patimind necaz, dar nefiind strivi?i; lipsi?i fiind, dar nu deznadajdui?i;
2Co 4:9  Prigoni?i fiind, dar nu parasi?i; doborâ?i, dar nu nimici?i;
2Co 4:10  Purtând totdeauna în trup omorârea lui Iisus, pentru ca ?i via?a lui Iisus sa se arate în trupul nostru.
2Co 4:11  Caci pururea noi cei vii suntem da?i spre moarte pentru Iisus, ca ?i via?a lui Iisus sa se arate în trupul nostru cel muritor.
2Co 4:12  Astfel ca în noi lucreaza moartea, iar în voi via?a.
2Co 4:13  Dar având acela?i duh al credin?ei, - dupa cum este scris: "Crezut-am, pentru aceea am ?i grait", - ?i noi credem: pentru aceea ?i graim,
2Co 4:14  ?tiind ca Cel ce a înviat pe Domnul Iisus ne va învia ?i pe noi cu Iisus ?i ne va înfa?i?a împreuna cu voi.
2Co 4:15  Caci toate sunt pentru voi, pentru ca, înmul?indu-se harul sa prisoseasca prin mai mul?i mul?umirea, spre slava lui Dumnezeu.
2Co 4:16  De aceea nu ne pierdem curajul ?i, chiar daca omul nostru cel din afara se trece, cel dinauntru însa se înnoie?te din zi în zi.
2Co 4:17  Caci necazul nostru de acum, u?or ?i trecator, ne aduce noua, mai presus de orice masura, slava ve?nica covâr?itoare,
2Co 4:18  Neprivind noi la cele ce se vad, ci la cele ce nu se vad, fiindca cele ce se vad sunt trecatoare, iar cele ce nu se vad sunt ve?nice.
2Co 5:1  Caci ?tim ca, daca acest cort, locuin?a noastra pamânteasca, se va strica, avem zidire de la Dumnezeu, casa nefacuta de mâna, ve?nica, în ceruri.
2Co 5:2  Caci de aceea ?i suspinam, în acest trup, dorind sa ne îmbracam cu locuin?a noastra cea din cer,
2Co 5:3  Daca totu?i vom fi gasi?i îmbraca?i, iar nu goi.
2Co 5:4  Ca noi, cei ce suntem în cortul acesta, suspinam îngreuia?i, de vreme ce dorim sa nu ne scoatem haina noastra, ci sa ne îmbracam cu cealalta pe deasupra, ca ceea ce este muritor sa fie înghi?it de via?a.
2Co 5:5  Iar Cel ce ne-a facut spre aceasta este Dumnezeu, Care ne-a dat noua arvuna Duhului.
2Co 5:6  Îndraznind deci totdeauna ?i ?tiind ca, petrecând în trup, suntem departe de Domnul,
2Co 5:7  Caci umblam prin credin?a, nu prin vedere,
2Co 5:8  Avem încredere ?i voim mai bine sa plecam din trup ?i sa petrecem la Domnul.
2Co 5:9  De aceea ne ?i straduim ca, fie ca petrecem în trup, fie ca plecam din el, sa fim bineplacu?i Lui.
2Co 5:10  Pentru ca noi to?i trebuie sa ne înfa?i?am înaintea scaunului de judecata al lui Hristos, ca sa ia fiecare dupa cele ce a facut prin trup, ori bine, ori rau.
2Co 5:11  Cunoscând deci frica de Domnul, cautam sa înduplecam pe oameni, dar lui Dumnezeu Îi suntem binecunoscu?i ?i nadajduiesc ca suntem binecunoscu?i ?i în cugetele voastre.
2Co 5:12  Caci nu va spunem iara?i cine suntem, ci va dam prilej de lauda pentru noi, ca sa ave?i ce sa spune?i acelora care se lauda cu fa?a ?i nu cu inima.
2Co 5:13  Caci, daca ne-am ie?it din fire, este pentru Dumnezeu, iar daca suntem cu mintea întreaga, este pentru voi.
2Co 5:14  Caci dragostea lui Hristos ne stapâne?te pe noi care socotim aceasta, ca daca unul a murit pentru to?i, au murit deci to?i.
2Co 5:15  ?i a murit pentru to?i, ca cei ce viaza sa nu mai loru?i, ci Aceluia care, pentru ei, a murit ?i a înviat.
2Co 5:16  De aceea, noi nu mai ?tim de acum pe nimeni dupa trup; chiar daca am cunoscut pe Hristos dupa trup, acum nu-L mai cunoa?tem.
2Co 5:17  Deci, daca este cineva în Hristos, este faptura noua; cele vechi au trecut, iata toate s-au facut noi.
2Co 5:18  ?i toate sunt de la Dumnezeu, Care ne-a împacat cu Sine prin Hristos ?i Care ne-a dat noua slujirea împacarii.
2Co 5:19  Pentru ca Dumnezeu era în Hristos, împacând lumea cu Sine însu?i, nesocotindu-le gre?elile lor ?i punând în noi cuvântul împacarii.
2Co 5:20  În numele lui Hristos, a?adar, ne înfa?i?am ca mijlocitori, ca ?i cum Însu?i Dumnezeu v-ar îndemna prin noi. Va rugam, în numele lui Hristos, împaca?i-va cu Dumnezeu!
2Co 5:21  Caci pe El, Care n-a cunoscut pacatul, L-a facut pentru noi pacat, ca sa dobândim, întru El, dreptatea lui Dumnezeu.
2Co 6:1  Fiind, dar, împreuna-lucratori cu Hristos, va îndemnam sa nu primi?i în zadar harul lui Dumnezeu.
2Co 6:2  Caci zice: "La vreme potrivita te-am ascultat ?i în ziua mântuirii te-am ajutat"; iata acum vreme potrivita, iata acum ziua mântuirii,
2Co 6:3  Nedând nici o sminteala întru nimic, ca sa nu fie slujirea noastra defaimata,
2Co 6:4  Ci în toate înfa?i?ându-ne pe noi în?ine ca slujitori ai lui Dumnezeu, în multa rabdare, în necazuri, în nevoi, în strâmtorari,
2Co 6:5  În batai, în temni?a, în tulburari, în osteneli, în privegheri, în posturi;
2Co 6:6  În cura?ie, în cuno?tin?a, în îndelunga-rabdare, în bunatate, în Duhul Sfânt, în dragoste nefa?arnica;
2Co 6:7  În cuvântul adevarului, în puterea lui Dumnezeu, prin armele drepta?ii, cele de-a dreapta ?i cele de-a stânga,
2Co 6:8  Prin slava ?i necinste, prin defaimare ?i lauda; ca ni?te amagitori, de?i iubitori de adevar,
2Co 6:9  Ca ni?te necunoscu?i, de?i bine cunoscu?i, ca fiind pe pragul mor?ii, de?i iata ca traim, ca ni?te pedepsi?i, dar nu uci?i;
2Co 6:10  Ca ni?te întrista?i, dar pururea bucurându-ne; ca ni?te saraci, dar pe mul?i îmboga?ind; ca unii care n-au nimic, dar toate le stapânesc.
2Co 6:11  O, corintenilor, gura noastra s-a deschis catre voi, inima noastra s-a largit.
2Co 6:12  În inima noastra nu sunte?i la strâmtorare; dar strâmtorare este pentru noi, în inimile voastre.
2Co 6:13  Plati?i-mi ?i voi aceea?i plata, va vorbesc ca unor copii ai mei - largi?i ?i voi inimile voastre!
2Co 6:14  Nu va înjuga?i la jug strain cu cei necredincio?i, caci ce înso?ire are dreptatea cu faradelegea? Sau ce împarta?ire are lumina cu întunericul?
2Co 6:15  ?i ce învoire este între Hristos ?i Veliar sau ce parte are un credincios cu un necredincios?
2Co 6:16  Sau ce în?elegere este între templul lui Dumnezeu ?i idoli? Caci noi suntem templu al Dumnezeului celui viu, precum Dumnezeu a zis ca: "Voi locui în ei ?i voi umbla ?i voi fi Dumnezeul lor ?i ei vor fi poporul Meu".
2Co 6:17  De aceea: "Ie?i?i din mijlocul lor ?i va osebi?i, zice Domnul, ?i de ce este necurat sa nu va atinge?i ?i Eu va voi primi pe voi.
2Co 6:18  ?i voi fi voua tata, ?i ve?i fi Mie fii ?i fiice", zice Domnul Atot?iitorul.
2Co 7:1  Având deci aceste fagaduin?e, iubi?ilor, sa ne cura?im pe noi de toata întinarea trupului ?i a duhului, desavâr?ind sfin?enia în frica lui Dumnezeu.
2Co 7:2  Face?i-ne loc în inimile voastre! N-am nedrepta?it pe nimeni; n-am vatamat pe nimeni, n-am în?elat pe nimeni.
2Co 7:3  Nu o spun spre osândirea voastra, caci v-am spus înainte ca sunte?i în inimile noastre, ca împreuna sa murim ?i împreuna sa traim.
2Co 7:4  Multa îmi este încrederea în voi! Multa îmi este lauda pentru voi! Umplutu-m-am de mângâiere! Cu tot necazul nostru, sunt covâr?it de bucurie!
2Co 7:5  Caci, dupa ce am sosit în Macedonia, trupul nostru n-a avut nici o odihna, necaji?i fiind în tot felul: din afara lupte, dinauntru temeri.
2Co 7:6  Dar Dumnezeu, Cel ce mângâie pe cei smeri?i, ne-a mângâiat pe noi cu venirea lui Tit.
2Co 7:7  ?i nu numai cu venirea lui, ci ?i cu mângâierea cu care el a fost mângâiat la voi, vestindu-ne noua dorin?a voastra, plânsul vostru, râvna voastra pentru mine, ca eu mai mult sa ma bucur.
2Co 7:8  Ca, chiar daca v-am întristat prin scrisoare, nu-mi pare rau, de?i îmi parea rau; caci vad ca scrisoarea aceea, fie ?i numai pentru un timp, v-a întristat.
2Co 7:9  Acum ma bucur, nu pentru ca v-a?i întristat, ci pentru ca v-a?i întristat spre pocain?a. Caci v-a?i întristat dupa Dumnezeu, ca sa nu fi?i întru nimic pagubi?i de catre noi.
2Co 7:10  Caci întristarea cea dupa Dumnezeu aduce pocain?a spre mântuire, fara parere de rau; iar întristarea lumii aduce moarte.
2Co 7:11  Ca iata, însa?i aceasta, ca v-a?i întristat dupa Dumnezeu, câta sârguin?a v-a adus, ba înca ?i dezvinova?ire ?i mâhnire ?i teama ?i dorin?a ?i râvna ?i ispa?ire! Întru totul a?i dovedit ca voi în?iva sunte?i cura?i în acest lucru.
2Co 7:12  Deci, de?i v-am scris, aceasta n-a fost din cauza celui ce a nedrepta?it, nici din cauza celui ce a fost nedrepta?it, ci ca sa ne învedereze la voi sârguin?a voastra pentru noi, înaintea lui Dumnezeu.
2Co 7:13  De aceea, ne-am mângâiat; dar pe lânga mângâierea noastra, ne-am bucurat peste masura mai ales de bucuria lui Tit, caci duhul lui s-a lini?tit din partea voastra a tuturor.
2Co 7:14  Caci daca m-am laudat înaintea lui cu ceva pentru voi, n-am fost dat de ru?ine; ci precum toate vi le-am grait întru adevar, a?a ?i lauda noastra pentru Tit s-a facut adevar.
2Co 7:15  ?i inima lui este ?i mai mult la voi, aducându-?i aminte de ascultarea voastra a tuturor, cum l-a?i primit cu frica ?i cu cutremur.
2Co 7:16  Ma bucur ca în toate pot sa ma încred în voi.
2Co 8:1  ?i va fac cunoscut, fra?ilor, harul lui Dumnezeu cel daruit în Bisericile Macedoniei;
2Co 8:2  Ca în multa lor încercare de necaz, prisosul bucuriei lor ?i saracia lor cea adânca au sporit în boga?ia darniciei lor,
2Co 8:3  Caci marturisesc ca de voia lor au dat, dupa putere ?i peste putere,
2Co 8:4  Cu multa rugaminte cerând har de a lua ?i ei parte la ajutorarea sfin?ilor.
2Co 8:5  ?i au facut nu dupa cum au nadajduit, ci s-au dat pe ei în?i?i întâi Domnului ?i apoi noua, prin voia lui Dumnezeu,
2Co 8:6  Încât am rugat pe Tit ca, precum a început dinainte, a?a sa ?i desavâr?easca, la voi, ?i darul acesta.
2Co 8:7  Ci precum întru toate prisosi?i: în credin?a, în cuvânt, în cuno?tin?a, în orice sârguin?a, în iubirea voastra catre noi, a?a ?i în acest dar sa prisosi?i.
2Co 8:8  Nu cu porunca o spun, ci încercând ?i cura?ia dragostei voastre, prin sârguin?a altora.
2Co 8:9  Caci cunoa?te?i harul Domnului nostru Iisus Hristos, ca El, bogat fiind, pentru voi a saracit, ca voi cu saracia Lui sa va îmboga?i?i.
2Co 8:10  ?i sfat va dau în aceasta: ca aceasta va este de folos voua, care înca de anul trecut a?i început nu numai sa face?i, ci sa ?i voi?i.
2Co 8:11  Duce?i dar acum pâna la capat fapta, ca precum a?i fost gata sa voi?i, tot a?a sa ?i îndeplini?i din ce ave?i.
2Co 8:12  Caci daca este bunavoin?a, bine primit este darul, dupa cât are cineva, nu dupa cât nu are.
2Co 8:13  Nu doar ca sa fie altora u?urare, iar voua necaz, ci ca sa fie potrivire:
2Co 8:14  Prisosin?a voastra sa împlineasca lipsa acelora, pentru ca ?i prisosin?a lor sa împlineasca lipsa voastra, spre a fi potrivire,
2Co 8:15  Precum este scris: "Celui cu mult nu i-a prisosit, ?i celui cu pu?in nu i-a lipsit".
2Co 8:16  Mul?umire fie adusa lui Dumnezeu, Care a dat aceea?i râvna pentru voi în inima lui Tit.
2Co 8:17  Caci, pe de o parte, a primit îndemnul nostru, iar, pe de alta parte, facându-se ?i mai sârguitor, de buna voie a plecat catre voi.
2Co 8:18  ?i am trimis împreuna cu el ?i pe fratele a carui lauda, întru Evanghelie, este în toate Bisericile;
2Co 8:19  Dar nu numai atât, ci este ?i ales de catre Biserici ca împreuna-calator cu noi la darul acesta, slujit de noi, spre slava Domnului însu?i ?i spre osârdia noastra.
2Co 8:20  Prin aceasta ne ferim ca sa nu ne defaimeze cineva, în aceasta îmbel?ugata strângere de daruri, de care ne îngrijim noi.
2Co 8:21  Pentru ca ne îngrijim de cele bune nu numai înaintea Domnului, ci ?i înaintea oamenilor.
2Co 8:22  ?i l-am trimis împreuna cu ei ?i pe fratele nostru, pe care l-am încercat în multe, de multe ori, ca fiind sârguitor, iar acum este ?i mai sârguitor, prin multa încredere în voi.
2Co 8:23  Astfel, daca e vorba de Tit, el este înso?itorul meu ?i împreuna-lucrator la voi; daca e vorba despre fra?ii no?tri, ei sunt apostoli ai Bisericilor, slava a lui Hristos.
2Co 8:24  Arata?i deci catre ei, în fa?a Bisericilor, dovada dragostei voastre, ca ?i a laudei noastre pentru voi.
2Co 9:1  Despre strângerea de ajutoare pentru sfin?i îmi este de prisos sa va scriu.
2Co 9:2  Ca ?tiu bunavoin?a voastra, cu care, pentru voi, ma laud catre macedoneni; ca Ahaia s-a pregatit din anul trecut, ?i râvna voastra a însufle?it pe cei mai mul?i.
2Co 9:3  Am trimis dar pe fra?i, ca lauda noastra pentru voi, în privin?a aceasta, sa nu fie zadarnica, ci sa fi?i gata, precum ziceam,
2Co 9:4  Ca nu cumva, când macedonenii vor veni împreuna cu mine ?i va vor gasi nepregati?i, sa fim ru?ina?i noi, ca sa nu zicem voi, în aceasta lauda încrezatoare.
2Co 9:5  Am socotit deci ca este nevoie sa îndemn pe fra?i sa vina întâi la voi ?i sa pregateasca darul vostru cel dinainte fagaduit, ca el sa fie gata, a?a ca un dar, nu ca o fapta de zgârcenie.
2Co 9:6  Aceasta însa zic: Cel ce seamana cu zgârcenie, cu zgârcenie va ?i secera, iar cel ce seamana cu darnicie, cu darnicie va ?i secera.
2Co 9:7  Fiecare sa dea cum socote?te cu inima sa, nu cu parere de rau, sau de sila, caci Dumnezeu iube?te pe cel care da cu voie buna.
2Co 9:8  ?i Dumnezeu poate sa înmul?easca tot harul la voi, ca, având totdeauna toata îndestularea în toate, sa prisosi?i spre tot lucrul bun,
2Co 9:9  Precum este scris: "Împar?it-a, dat-a saracilor; dreptatea Lui ramâne în veac".
2Co 9:10  Iar Cel ce da samân?a semanatorului ?i pâine spre mâncare, va va da ?i va înmul?i samân?a voastra ?i va face sa creasca roadele drepta?ii voastre,
2Co 9:11  Ca întru toate sa va îmboga?i?i, spre toata darnicia care aduce prin noi mul?umire lui Dumnezeu.
2Co 9:12  Pentru ca slujirea acestui dar nu numai ca împline?te lipsurile sfin?ilor, ci prisose?te prin multe mul?umiri în fa?a lui Dumnezeu;
2Co 9:13  Slavind ei pe Dumnezeu, prin adeverirea acestei ajutorari, pentru supunerea marturisirii voastre Evangheliei lui Hristos ?i pentru darnicia împarta?irii catre ei ?i catre to?i,
2Co 9:14  Se roaga pentru voi, ?i va iubesc pentru harul lui Dumnezeu cel ce prisose?te la voi.
2Co 9:15  Iar lui Dumnezeu mul?umire pentru darul Sau cel negrait.
2Co 10:1  Însumi eu, Pavel, va îndemn prin blânde?ea ?i îngaduin?a lui Hristos - eu care de fa?a sunt smerit între voi, dar, în lipsa, îndraznesc fa?a de voi S
2Co 10:2  Va rog, dar, sa nu ma sili?i, când voi fi de fa?a, sa îndraznesc cu încrederea cu care gândesc ca voi îndrazni împotriva unora care ne socotesc ca umblam dupa trup.
2Co 10:3  Pentru ca, de?i umblam în trup, nu ne luptam trupe?te.
2Co 10:4  Caci armele luptei noastre nu sunt trupe?ti, ci puternice înaintea lui Dumnezeu, spre darâmarea întariturilor. Noi surpam iscodirile min?ii,
2Co 10:5  ?i toata trufia care se ridica împotriva cunoa?terii lui Dumnezeu ?i tot gândul îl robim, spre ascultarea lui Hristos,
2Co 10:6  ?i gata suntem sa pedepsim toata neascultarea, atunci când supunerea voastra va fi deplina.
2Co 10:7  Judeca?i lucrurile a?a cum se arata: daca cineva are încredere în sine ca este al lui Hristos, sa gândeasca iara?i de la sine aceasta, ca precum este el al lui Hristos, tot a?a suntem ?i noi.
2Co 10:8  ?i chiar de ma voi lauda, ceva mai mult, cu puterea noastra, pe care ne-a dat-o Domnul spre zidirea, iar nu spre darâmarea voastra, nu ma voi ru?ina,
2Co 10:9  Ca sa nu par ca v-a? înfrico?a prin scrisori.
2Co 10:10  Ca scrisorile lui, zic ei, sunt grele ?i tari, dar înfa?i?area trupului este slaba ?i cuvântul lui este dispre?uit.
2Co 10:11  Cel ce vorbe?te astfel sa-?i dea seama ca a?a cum suntem cu cuvântul prin scrisori, când nu suntem de fa?a, tot a?a suntem ?i cu fapta, când suntem de fa?a.
2Co 10:12  Caci nu îndraznim sa ne numaram sau sa ne asemanam cu unii care se lauda singuri; dar aceia, masurându-se ?i asemanându-se pe ei cu ei în?i?i, nu au pricepere.
2Co 10:13  Iar noi nu ne vom lauda fara masura, ci dupa masura dreptarului cu care ne-a masurat noua Dumnezeu, ca sa ajungem ?i pâna la voi.
2Co 10:14  Caci nu ne întindem peste masura, ca ?i cum n-am fi ajuns la voi, caci am ?i ajuns cu Evanghelia lui Hristos pâna la voi.
2Co 10:15  Nu ne laudam peste masura cu ostenelile altora, ci avem nadejde ca, tot crescând credin?a voastra, ne vom mari în voi cu prisosin?a, dupa masura noastra,
2Co 10:16  Ca sa propovaduim Evanghelia ?i în ?inuturile de dincolo de voi, dar fara sa ne laudam cu masura straina, în cele de-a gata.
2Co 10:17  Iar cel ce se lauda, în Domnul sa se laude.
2Co 10:18  Pentru ca nu cel ce se lauda singur este dovedit bun, ci acela pe care Domnul îl lauda.
2Co 11:1  O, de mi-a?i îngadui pu?ina neîn?elep?ie! Dar îmi ?i îngadui?i,
2Co 11:2  Caci va râvnesc pe voi cu râvna lui Dumnezeu, pentru ca v-am logodit unui singur barbat, ca sa va înfa?i?ez lui Hristos fecioara neprihanita.
2Co 11:3  Dar ma tem ca nu cumva, precum ?arpele a amagit pe Eva în viclenia lui, tot a?a sa se abata ?i gândurile voastre de la cura?ia ?i nevinova?ia cea în Hristos.
2Co 11:4  Caci daca cel ce vine va propovaduie?te un alt Iisus, pe care nu l-am propovaduit noi, sau lua?i un alt duh, pe care nu l-a?i luat, sau alta evanghelie pe care nu a?i primit-o, - voi l-a?i îngadui foarte bine.
2Co 11:5  Dar eu socotesc ca nu sunt cu nimic mai prejos decât cei mai de frunte dintre apostoli.
2Co 11:6  ?i chiar daca sunt neiscusit în cuvânt, nu însa în cuno?tin?a, ci v-am dovedit-o în totul fa?a de voi to?i.
2Co 11:7  Sau am facut pacat ca v-am propovaduit în dar Evanghelia lui Dumnezeu, smerindu-ma pe mine însumi, pentru ca voi sa va înal?a?i?
2Co 11:8  Alte Biserici am pradat, luând plata ca sa va slujesc pe voi.
2Co 11:9  ?i de fa?a fiind la voi ?i în lipsuri aflându-ma, n-am facut suparare nimanui. Caci în cele ce mi-au lipsit, m-au îndestulat fra?ii veni?i din Macedonia. ?i în toate m-am pazit ?i ma voi pazi, sa nu va fiu povara.
2Co 11:10  Este în mine adevarul lui Hristos, ca lauda aceasta nu-mi va fi îngradita în ?inuturile Ahaei.
2Co 11:11  Pentru ce? Pentru ca nu va iubesc? Dumnezeu ?tie!
2Co 11:12  Dar ceea ce fac, voi face ?i în viitor, ca sa tai pricina celor ce poftesc pricina, pentru a se afla ca ?i noi în ceea ce se lauda.
2Co 11:13  Pentru ca unii ca ace?tia sunt apostoli mincino?i, lucratori vicleni, care iau chip de apostoli ai lui Hristos.
2Co 11:14  Nu este de mirare, deoarece însu?i satana se preface în înger al luminii.
2Co 11:15  Nu este deci lucru mare daca ?i slujitorii lui iau chip de slujitori ai drepta?ii, al caror sfâr?it va fi dupa faptele lor.
2Co 11:16  Iara?i zic: Sa nu ma socoteasca cineva ca sunt fara minte, iar de nu primi?i-ma macar ca pe un fara-de-minte, ca sa ma laud ?i eu pu?in.
2Co 11:17  Ceea ce graiesc, nu dupa Domnul graiesc, ci ca în neîn?elep?ie, în aceasta stare de lauda.
2Co 11:18  Deoarece mul?i se lauda dupa trup, ma voi lauda ?i eu.
2Co 11:19  Pentru ca voi, în?elep?i fiind, îngadui?i bucuros pe cei neîn?elep?i.
2Co 11:20  Caci de va robe?te cineva, de va manânca cineva, de va ia ce e al vostru, de va prive?te cineva cu mândrie, de va love?te cineva peste obraz, - rabda?i.
2Co 11:21  Spre necinste o spun, ca noi ne-am aratat slabi. Dar în orice ar cuteza cineva - întru neîn?elep?ie zic, - cutez ?i eu!
2Co 11:22  Sunt ei evrei? Sunt ?i eu. Sunt ei israeli?i? Israelit sunt ?i eu. Sunt ei samân?a lui Avraam? Sunt ?i eu.
2Co 11:23  Sunt ei slujitori ai lui Hristos? Nebune?te spun: eu nu mai mult ca ei! În osteneli mai mult, în închisori mai mult, în batai peste masura, la moarte adeseori.
2Co 11:24  De la iudei, de cinci ori am luat patruzeci de lovituri de bici fara una.
2Co 11:25  De trei ori am fost batut cu vergi; o data am fost batut cu pietre; de trei ori s-a sfarâmat corabia cu mine; o noapte ?i o zi am petrecut în largul marii.
2Co 11:26  În calatorii adeseori, în primejdii de râuri, în primejdii de la tâlhari, în primejdii de la neamul meu, în primejdii de la pagâni; în primejdii în ceta?i, în primejdii în pustie, în primejdii pe mare, în primejdii între fra?ii cei mincino?i;
2Co 11:27  În osteneala ?i în truda, în privegheri adeseori, în foame ?i în sete, în posturi de multe ori, în frig ?i în lipsa de haine.
2Co 11:28  Pe lânga cele din afara, ceea ce ma împresoara în toate zilele este grija de toate Bisericile.
2Co 11:29  Cine este slab ?i eu sa nu fiu slab? Cine se sminte?te ?i eu sa nu ard?
2Co 11:30  Daca trebuie sa ma laud, ma voi lauda cu cele ale slabiciunii mele!
2Co 11:31  Dumnezeu ?i Tatal Domnului nostru Iisus, Cel ce este binecuvântat în veci, ?tie ca nu mint!
2Co 11:32  În Damasc, dregatorul regelui Areta pazea cetatea Damascului, ca sa ma prinda,
2Co 11:33  ?i printr-o fereastra am fost lasat în jos, peste zid, într-un co?, ?i am scapat din mâinile lui.
2Co 12:1  Daca trebuie sa ma laud, nu-mi este de folos, dar voi veni totu?i la vedenii ?i la descoperiri de la Domnul.
2Co 12:2  Cunosc un om în Hristos, care acum paisprezece ani - fie în trup, nu ?tiu; fie în afara de trup, nu ?tiu, Dumnezeu ?tie - a fost rapit unul ca acesta pâna la al treilea cer.
2Co 12:3  ?i-l ?tiu pe un astfel de om - fie în trup, fie în afara de trup, nu ?tiu, Dumnezeu ?tie S
2Co 12:4  Ca a fost rapit în rai ?i a auzit cuvinte de nespus, pe care nu se cuvine omului sa le graiasca.
2Co 12:5  Pentru unul ca acesta ma voi lauda; iar pentru mine însumi nu ma voi lauda decât numai în slabiciunile mele.
2Co 12:6  Caci chiar daca a? vrea sa ma laud, nu voi fi fara minte, caci voi spune adevarul; dar ma feresc de aceasta, ca sa nu ma socoteasca nimeni mai presus decât ceea ce vede sau aude de la mine.
2Co 12:7  ?i pentru ca sa nu ma trufesc cu mare?ia descoperirilor, datu-mi-s-a mie un ghimpe în trup, un înger al satanei, sa ma bata peste obraz, ca sa nu ma trufesc.
2Co 12:8  Pentru aceasta de trei ori am rugat pe Domnul ca sa-l îndeparteze de la mine;
2Co 12:9  ?i mi-a zis: Î?i este de ajuns harul Meu, caci puterea Mea se desavâr?e?te în slabiciune. Deci, foarte bucuros, ma voi lauda mai ales întru slabiciunile mele, ca sa locuiasca în mine puterea lui Hristos.
2Co 12:10  De aceea ma bucur în slabiciuni, în defaimari, în nevoi, în prigoniri, în strâmtorari pentru Hristos, caci, când sunt slab, atunci sunt tare.
2Co 12:11  M-am facut ca unul fara minte, laudându-ma. Voi m-a?i silit! Caci se cuvenea sa vorbi?i voi de bine despre mine, pentru ca nu sunt cu nimic mai prejos decât cei mai de frunte dintre apostoli, de?i nu sunt nimic.
2Co 12:12  Dovezile mele de apostol s-au aratat la voi în toata rabdarea, prin semne, prin minuni ?i prin puteri.
2Co 12:13  Caci cu ce sunte?i voi mai prejos decât celelalte Biserici, decât numai ca eu nu v-am fost povara? Darui?i-mi mie aceasta nedreptate.
2Co 12:14  Iata, a treia oara sunt gata sa vin la voi ?i nu va voi fi povara, caci nu caut ale voastre, ci pe voi. Pentru ca nu copiii sunt datori sa agoniseasca pentru parin?i, ci parin?ii pentru copii.
2Co 12:15  Deci eu foarte bucuros voi cheltui ?i ma voi cheltui pentru sufletele voastre, de?i, iubindu-va mai mult, eu sunt iubit mai pu?in.
2Co 12:16  Dar fie! Eu nu v-am împovarat. Ci, fiind iste?, v-am prins cu în?elaciune.
2Co 12:17  Am tras eu folos de la voi, prin vreunul din aceia pe care i-am trimis?
2Co 12:18  L-am rugat pe Tit ?i am trimis, împreuna cu el, pe fratele. V-a asuprit Tit cu ceva? N-am umblat noi în acela?i duh? N-am calcat noi pe acelea?i urme?
2Co 12:19  De mult vi se pare ca ne aparam fa?a de voi. Dar noi graim în Hristos, înaintea lui Dumnezeu. ?i toate acestea, iubi?ii mei, pentru zidirea voastra.
2Co 12:20  Caci ma tem ca nu cumva venind, sa nu va gasesc pe voi a?a precum voiesc, iar eu sa fiu gasit de voi a?a precum nu voi?i; ma tem adica de certuri, de pizma, de mânii, de întarâtari, de clevetiri, de murmure, de îngâmfari, de tulburari;
2Co 12:21  Ma tem ca nu cumva, venind iara?i, sa ma umileasca Dumnezeul meu la voi ?i sa plâng pe mul?i care au pacatuit înainte ?i nu s-au pocait de necura?ia ?i de desfrânarea ?i de necumpatarea pe care le-au facut.
2Co 13:1  A treia oara vin la voi. În gura a doi sau trei martori va sta tot cuvântul.
2Co 13:2  Am spus dinainte ?i spun iara?i dinainte, ca atunci când am fost de fa?a a doua oara, ?i acum, nefiind de fa?a, scriu celor ce au pacatuit înainte ?i tuturor celorlal?i ca, de voi veni iara?i, nu voi cru?a,
2Co 13:3  De vreme ce voi cauta?i dovada ca Hristos graie?te întru mine, Care nu este slab fa?a de voi, ci puternic în voi.
2Co 13:4  Caci, de?i a fost rastignit din slabiciune, din puterea lui Dumnezeu este însa viu. ?i noi suntem slabi întru El, dar vom fi împreuna cu El, din puterea lui Dumnezeu fa?a de voi.
2Co 13:5  Cerceta?i-va pe voi în?iva daca sunte?i în credin?a; încerca?i-va pe voi în?iva. Sau nu va cunoa?te?i voi singuri bine ca Hristos Iisus este întru voi? Afara numai daca nu sunte?i netrebnici.
2Co 13:6  Nadajduiesc însa ca ve?i cunoa?te ca noi nu suntem netrebnici.
2Co 13:7  ?i ne rugam lui Dumnezeu ca sa nu savâr?i?i voi nici un rau, nu ca sa ne aratam noi încerca?i, ci pentru ca voi sa face?i binele, iar noi sa fim ca ni?te netrebnici.
2Co 13:8  Caci împotriva adevarului n-avem nici o putere; avem pentru adevar.
2Co 13:9  Caci ne bucuram când noi suntem slabi, iar voi sunte?i tari. Aceasta ?i cerem în rugaciunea noastra: desavâr?irea voastra.
2Co 13:10  Pentru aceea va scriu acestea, nefiind de fa?a, ca atunci, când voi fi de fa?a, sa nu cutez cu asprime, dupa puterea pe care mi-a dat-o Domnul spre zidire, iar nu spre darâmare.
2Co 13:11  Deci, fra?ilor, bucura?i-va! Desavâr?i?i-va, mângâia?i-va, fi?i uni?i în cuget, trai?i în pace ?i Dumnezeul dragostei ?i al pacii va fi cu voi.
2Co 13:12  Îmbra?i?a?i-va unii pe al?ii cu sarutare sfânta.
2Co 13:13  Sfin?ii to?i va îmbra?i?eaza.
2Co 13:14  Harul Domnului nostru Iisus Hristos ?i dragostea lui Dumnezeu ?i împarta?irea Sfântului Duh sa fie cu voi cu to?i!


\end{document}