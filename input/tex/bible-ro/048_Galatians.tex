\begin{document}

\title{Galatians}

Gal 1:1  Pavel, apostol nu de la oameni, nici prin vreun om, ci prin Iisus Hristos ?i prin Dumnezeu-Tatal, Care L-a înviat pe El din mor?i.
Gal 1:2  ?i to?i fra?ii care sunt împreuna cu mine - Bisericilor Galatiei:
Gal 1:3  Har voua ?i pace de la Dumnezeu-Tatal ?i de la Domnul nostru Iisus Hristos,
Gal 1:4  Cel ce S-a dat pe Sine pentru pacatele noastre, ca sa ne scoata pe noi din acest veac rau de acum, dupa voia lui Dumnezeu ?i a Tatalui nostru,
Gal 1:5  Caruia fie slava în vecii vecilor. Amin!
Gal 1:6  Ma mir ca a?a degraba trece?i de la cel ce v-a chemat pe voi, prin harul lui Hristos, la alta Evanghelie,
Gal 1:7  Care nu este alta, decât ca sunt unii care va tulbura ?i voiesc sa schimbe Evanghelia lui Hristos.
Gal 1:8  Dar chiar daca noi sau un înger din cer v-ar vesti alta Evanghelie decât aceea pe care v-am vestit-o - sa fie anatema!
Gal 1:9  Precum v-am spus mai înainte, ?i acum va spun iara?i: Daca va propovaduie?te cineva altceva decât a?i primit - sa fie anatema!
Gal 1:10  Caci acum caut bunavoin?a oamenilor sau pe a lui Dumnezeu? Sau caut sa plac oamenilor? Daca a? placea însa oamenilor, n-a? fi rob al lui Hristos.
Gal 1:11  Dar va fac cunoscut, fra?ilor, ca Evanghelia cea binevestita de mine nu este dupa om;
Gal 1:12  Pentru ca nici eu n-am primit-o de la om, nici n-am înva?at-o, ci prin descoperirea lui Iisus Hristos.
Gal 1:13  Caci a?i auzit despre purtarea mea de altadata întru iudaism, ca prigoneam peste masura Biserica lui Dumnezeu ?i o pustiiam.
Gal 1:14  ?i spoream în iudaism mai mult decât mul?i dintre cei care erau de vârsta mea în neamul meu, fiind mult râvnitor al datinilor mele parinte?ti.
Gal 1:15  Dar când a binevoit Dumnezeu Care m-a ales din pântecele mamei mele ?i m-a chemat prin harul Sau,
Gal 1:16  Sa descopere pe Fiul Sau întru mine, pentru ca sa-L binevestesc la neamuri, îndata nu am primit sfat de la trup ?i de la sânge,
Gal 1:17  Nici nu m-am suit la Ierusalim, la Apostolii cei dinainte de mine, ci m-am dus în Arabia ?i m-am întors iara?i la Damasc.
Gal 1:18  Apoi, dupa trei ani, m-am suit la Ierusalim, ca sa-l cunosc pe Chefa ?i am ramas la el cincisprezece zile.
Gal 1:19  Iar pe altul din apostoli n-am vazut decât numai pe Iacov, fratele Domnului.
Gal 1:20  Dar cele ce va scriu, iata (spun) înaintea lui Dumnezeu, ca nu va mint.
Gal 1:21  Dupa aceea am venit în ?inuturile Siriei ?i ale Ciliciei.
Gal 1:22  ?i dupa fa?a eram necunoscut Bisericilor lui Hristos celor din Iudeea.
Gal 1:23  Ci numai auzisera ca cel ce ne prigonea pe noi, odinioara, acum bineveste?te credin?a pe care altadata o nimicea;
Gal 1:24  ?i slaveau pe Dumnezeu în mine.
Gal 2:1  Apoi, dupa paisprezece ani, m-am suit iara?i la Ierusalim cu Barnaba, luând cu mine ?i pe Tit.
Gal 2:2  M-am suit, potrivit unei descoperiri, ?i le-am aratat Evanghelia pe care o propovaduiesc la neamuri, îndeosebi celor mai de seama, ca nu cumva sa alerg sau sa fi alergat în zadar.
Gal 2:3  Dar nici Tit, care era cu mine ?i care era elin, n-a fost silit sa se taie împrejur,
Gal 2:4  Din cauza fra?ilor mincino?i, care venisera, furi?ându-se, sa iscodeasca libertatea noastra, pe care o avem în Hristos Iisus, ca sa ne robeasca,
Gal 2:5  Carora nici macar un ceas nu ne-am plecat cu supunere, pentru ca adevarul Evangheliei sa ramâna neclintit la voi.
Gal 2:6  Iar de cei ce sunt mai de seama - oricine ar fi fost ei cândva, nu ma prive?te; Dumnezeu nu cauta la fa?a omului, - cei mai de seama n-au adaugat nimic la Evanghelia mea,
Gal 2:7  Ci dimpotriva, vazând ca mie mi-a fost încredin?ata Evanghelia pentru cei netaia?i împrejur, dupa cum lui Petru, Evanghelia pentru cei taia?i împrejur,
Gal 2:8  Caci Cel ce a lucrat prin Petru în apostolia taierii împrejur a lucrat ?i prin mine la neamuri;
Gal 2:9  ?i cunoscând harul ce mi-a fost dat mie, Iacov ?i Chefa ?i Ioan, cei socoti?i a fi stâlpi, mi-au dat mie ?i lui Barnaba dreapta spre unire cu ei, pentru ca noi sa binevestim la neamuri, iar ei la cei taia?i împrejur,
Gal 2:10  Numai sa ne aducem aminte de saraci, ceea ce tocmai m-am ?i silit sa fac.
Gal 2:11  Iar când Chefa a venit în Antiohia, pe fa?a i-am stat împotriva, caci era vrednic de înfruntare.
Gal 2:12  Caci înainte de a veni unii de la Iacov, el mânca cu cei dintre neamuri; dar când au venit ei, se ferea ?i se osebea, temându-se de cei din taierea împrejur.
Gal 2:13  ?i, împreuna cu el, s-au fa?arnicit ?i ceilal?i iudei, încât ?i Barnaba a fost atras în fa?arnicia lor.
Gal 2:14  Dar când am vazut ca ei nu calca drept, dupa adevarul Evangheliei, am zis lui Chefa, înaintea tuturor: Daca tu, care e?ti iudeu, traie?ti ca pagânii ?i nu ca iudeii, de ce sile?ti pe pagâni sa traiasca ca iudeii?
Gal 2:15  Noi suntem din fire iudei, iar nu pacato?i dintre neamuri.
Gal 2:16  ?tiind însa ca omul nu se îndrepteaza din faptele Legii, ci prin credin?a în Hristos Iisus, am crezut ?i noi în Hristos Iisus, ca sa ne îndrepta din credin?a în Hristos, iar nu din faptele Legii, caci din faptele Legii, nimeni nu se va îndrepta.
Gal 2:17  Daca însa, cautând sa ne îndreptam în Hristos, ne-am aflat ?i noi în?ine pacato?i, este, oare, Hristos slujitor al pacatului? Nicidecum!
Gal 2:18  Caci daca zidesc iara?i ceea ce am darâmat, ma arat pe mine însumi calcator (de porunca).
Gal 2:19  Caci, eu, prin Lege, am murit fa?a de Lege, ca sa traiesc lui Dumnezeu.
Gal 2:20  M-am rastignit împreuna cu Hristos; ?i nu eu mai traiesc, ci Hristos traie?te în mine. ?i via?a de acum, în trup, o traiesc în credin?a în Fiul lui Dumnezeu, Care m-a iubit ?i S-a dat pe Sine însu?i pentru mine.
Gal 2:21  Nu lepad harul lui Dumnezeu; caci daca dreptatea vine prin Lege, atunci Hristos a murit în zadar.
Gal 3:1  O, galateni fara de minte, cine v-a ademenit pe voi, sa nu va încrede?i adevarului, - pe voi, în ochii carora a fost zugravit Iisus Hristos rastignit?
Gal 3:2  Numai aceasta voiesc sa aflu de la voi: Din faptele Legii primit-a?i voi Duhul, sau din ascultarea credin?ei?
Gal 3:3  Atât de fara de minte sunte?i? Dupa ce a?i început în Duh, sfâr?i?i acum în trup?
Gal 3:4  A?i patimit atâtea în zadar? - daca a fost în zadar, cu adevarat.
Gal 3:5  Deci Cel care va da voua Duhul ?i savâr?e?te minuni la voi, le face, oare, din faptele Legii, sau din ascultarea credin?ei?
Gal 3:6  Precum ?i Avraam a crezut în Dumnezeu ?i i s-a socotit lui ca dreptate.
Gal 3:7  Sa ?ti?i, deci, ca cei ce sunt din credin?a, ace?tia sunt fii ai lui Avraam.
Gal 3:8  Iar Scriptura, vazând dinainte ca Dumnezeu îndrepteaza neamurile din credin?a, dinainte a binevestit lui Avraam: "Ca se vor binecuvânta în tine toate neamurile".
Gal 3:9  Deci cei ce sunt din credin?a se binecuvinteaza împreuna cu credinciosul Avraam.
Gal 3:10  Caci to?i câ?i sunt din faptele Legii sub blestem sunt, ca scris este: "Blestemat este oricine nu staruie întru toate cele scrise în cartea Legii, ca sa le faca".
Gal 3:11  Iar acum ca, prin Lege, nu se îndrepteaza nimeni înaintea lui Dumnezeu este lucru lamurit, deoarece "dreptul din credin?a va fi viu".
Gal 3:12  Legea însa nu este din credin?a, dar cel care va face acestea, va fi viu prin ele.
Gal 3:13  Hristos ne-a rascumparat din blestemul Legii, facându-Se pentru noi blestem; pentru ca scris este: "Blestemat este tot cel spânzurat pe lemn".
Gal 3:14  Ca, prin Hristos Iisus, sa vina la neamuri binecuvântarea lui Avraam, ca sa primim, prin credin?a, fagaduin?a Duhului.
Gal 3:15  Fra?ilor, ca un om graiesc; ca ?i testamentul întarit al unui om nimeni nu-l strica, sau îi mai adauga ceva.
Gal 3:16  Fagaduin?ele au fost rostite lui Avraam ?i urma?ului sau. Nu zice: "?i urma?ilor", - ca de mai mul?i, - ci ca de unul singur: "?i Urma?ului tau", Care este Hristos.
Gal 3:17  Aceasta zic dar: Un testament întarit dinainte de Dumnezeu în Hristos nu desfiin?eaza Legea, care a venit dupa patru sute treizeci de ani, ca sa desfiin?eze fagaduin?a.
Gal 3:18  Caci daca mo?tenirea este din Lege, nu mai este din fagaduin?a, dar Dumnezeu i-a daruit lui Avraam mo?tenirea prin fagaduin?a.
Gal 3:19  Deci ce este Legea? Ea a fost adaugata pentru calcarile de lege, pâna când era sa vina Urma?ul, Caruia I s-a dat fagaduin?a, ?i a fost rânduita prin îngeri, în mâna unui Mijlocitor.
Gal 3:20  Mijlocitorul însa nu este al unuia singur, iar Dumnezeu este unul.
Gal 3:21  Este deci Legea împotriva fagaduin?elor lui Dumnezeu? Nicidecum! Caci daca s-ar fi dat Lege, care sa poata da via?a, cu adevarat dreptatea ar veni din Lege.
Gal 3:22  Dar Scriptura a închis toate sub pacat, pentru ca fagaduin?a sa se dea din credin?a în Iisus Hristos celor ce cred.
Gal 3:23  Iar înainte de venirea credin?ei, noi eram pazi?i sub Lege, fiind închi?i pentru credin?a care avea sa se descopere.
Gal 3:24  Astfel ca Legea ne-a fost calauza spre Hristos, pentru ca sa ne îndreptam din credin?a.
Gal 3:25  Iar daca a venit credin?a, nu mai suntem sub calauza.
Gal 3:26  Caci to?i sunte?i fii ai lui Dumnezeu prin credin?a în Hristos Iisus.
Gal 3:27  Caci, câ?i în Hristos v-a?i botezat, în Hristos v-a?i îmbracat.
Gal 3:28  Nu mai este iudeu, nici elin; nu mai este nici rob, nici liber; nu mai este parte barbateasca ?i parte femeiasca, pentru ca voi to?i una sunte?i în Hristos Iisus.
Gal 3:29  Iar daca voi sunte?i ai lui Hristos, sunte?i deci urma?ii lui Avraam, mo?tenitori dupa fagaduin?a.
Gal 4:1  Zic însa: Câta vreme mo?tenitorul este copil, nu se deosebe?te cu nimic de rob, de?i este stapân peste toate;
Gal 4:2  Ci este sub epitropi ?i iconomi, pâna la vremea rânduita de tatal sau.
Gal 4:3  Tot a?a ?i noi, când eram copii, eram robi în?elesurilor celor slabe ale lumii;
Gal 4:4  Iar când a venit plinirea vremii, Dumnezeu, a trimis pe Fiul Sau, nascut din femeie, nascut sub Lege,
Gal 4:5  Ca pe cei de sub Lege sa-i rascumpere, ca sa dobândim înfierea.
Gal 4:6  ?i pentru ca sunte?i fii, a trimis Dumnezeu pe Duhul Fiului Sau în inimile noastre, care striga: Avva, Parinte!
Gal 4:7  Astfel dar, nu mai e?ti rob, ci fiu; iar de e?ti fiu, e?ti ?i mo?tenitor al lui Dumnezeu, prin Iisus Hristos.
Gal 4:8  Dar atunci necunoscând pe Dumnezeu, slujea?i celor ce din fire nu sunt dumnezei;
Gal 4:9  Acum însa, dupa ce a?i cunoscut pe Dumnezeu, sau mai degraba dupa ce a?i fost cunoscu?i de Dumnezeu, cum va întoarce?i iara?i la în?elesurile cele slabe ?i sarace, carora iara?i voi?i sa le sluji?i ca înainte?
Gal 4:10  ?ine?i zile ?i luni ?i timpuri ?i ani?
Gal 4:11  Ma tem de voi, sa nu ma fi ostenit la voi, în zadar.
Gal 4:12  Fi?i, va rog, fra?ilor, precum sunt eu, ca ?i eu am fost precum sunte?i voi. Nu mi-a?i facut nici un rau;
Gal 4:13  Dar ?ti?i ca din cauza unei slabiciuni a trupului, am binevestit voua mai întâi,
Gal 4:14  ?i voi nu a?i dispre?uit încercarea mea, ce era în trupul meu, nici nu v-a?i scârbit, ci m-a?i primit ca pe un înger al lui Dumnezeu, ca pe Hristos Iisus.
Gal 4:15  Unde este deci fericirea voastra? Caci va marturisesc ca, de ar fi fost cu putin?a, v-a?i fi scos ochii vo?tri ?i mi i-a?i fi dat mie.
Gal 4:16  Am ajuns deci vrajma?ul vostru spunându-va adevarul?
Gal 4:17  Aceia va râvnesc, dar nu cu gând bun; ci vor sa va desparta (de mine), ca sa-i iubi?i pe ei.
Gal 4:18  Dar e bine sa râvni?i totdeauna binele, ?i nu numai atunci când eu sunt de fa?a la voi.
Gal 4:19  O, copiii mei, pentru care sufar iara?i durerile na?terii, pâna ce Hristos va lua chip în voi!
Gal 4:20  A? vrea acum sa ma gasesc la voi ?i glasul sa mi-l schimb, caci nu ?tiu ce sa cred despre voi!
Gal 4:21  Spune?i-mi voi, care vre?i sa fi?i sub Lege, nu auzi?i Legea?
Gal 4:22  Caci scris este ca Avraam a avut doi fii: unul din femeia roaba ?i altul din femeia libera.
Gal 4:23  Dar cel din roaba s-a nascut dupa trup, iar cel din cea libera s-a nascut prin fagaduin?a.
Gal 4:24  Unele ca acestea au alta însemnare, caci acestea (femei) sunt doua testamente: Unul de la Muntele Sinai, care na?te spre robie ?i care este Agar;
Gal 4:25  Caci Agar este Muntele Sinai, în Arabia, ?i raspunde Ierusalimului de acum, care zace în robie cu copiii lui;
Gal 4:26  Iar cea libera este Ierusalimul cel de sus, care este mama noastra.
Gal 4:27  Caci scris este: "Vesele?te-te, tu, cea stearpa, care nu na?ti! Izbucne?te de bucurie ?i striga, tu care nu ai durerile na?terii, caci mul?i sunt copiii celei parasite, mai mul?i decât ai celei care are barbat".
Gal 4:28  Iar noi, fra?ilor, suntem dupa Isaac, fii ai fagaduin?ei.
Gal 4:29  Ci precum atunci cel ce se nascuse dupa trup prigonea pe cel ce se nascuse dupa Duh, tot a?a ?i acum.
Gal 4:30  Dar ce zice Scriptura? "Izgone?te pe roaba ?i fiul ei, caci nu va mo?teni fiul roabei, împreuna cu fiul celei libere".
Gal 4:31  Deci, fra?ilor, nu suntem copii ai roabei, ci copii ai celei libere.
Gal 5:1  Sta?i deci tari în libertatea cu care Hristos ne-a facut liberi ?i nu va prinde?i iara?i în jugul robiei.
Gal 5:2  Iata eu, Pavel, va spun voua: Ca de va ve?i taia împrejur, Hristos nu va va folosi la nimic.
Gal 5:3  ?i marturisesc, iara?i, oricarui om ce se taie împrejur, ca el este dator sa împlineasca toata Legea.
Gal 5:4  Cei ce voi?i sa va îndrepta?i prin Lege v-a?i îndepartat de Hristos, a?i cazut din har;
Gal 5:5  Caci noi a?teptam în Duh nadejdea drepta?ii din credin?a.
Gal 5:6  Caci în Hristos Iisus, nici taierea împrejur nu poate ceva, nici netaierea împrejur, ci credin?a care este lucratoare prin iubire.
Gal 5:7  Voi alerga?i bine. Cine v-a oprit ca sa nu va supune?i adevarului?
Gal 5:8  Înduplecarea aceasta nu este de la cel care va cheama.
Gal 5:9  Pu?in aluat dospe?te toata framântatura.
Gal 5:10  Eu am încredere în voi, întru Domnul, ca nimic altceva nu ve?i cugeta; dar cel ce va tulbura pe voi î?i va purta osânda, oricine ar fi el.
Gal 5:11  Dar eu, fra?ilor, daca propovaduiesc înca taierea împrejur, pentru ce mai sunt prigonit? Deci, sminteala crucii a încetat!
Gal 5:12  O, de s-ar taia de tot cei ce va razvratesc pe voi!
Gal 5:13  Caci voi, fra?ilor, a?i fost chema?i la libertate; numai sa nu folosi?i libertatea ca prilej de a sluji trupului, ci sluji?i unul altuia prin iubire.
Gal 5:14  Caci toata Legea se cuprinde într-un singur cuvânt, în acesta: Iube?te pe aproapele tau ca pe tine însu?i.
Gal 5:15  Iar daca va mu?ca?i unul pe altul ?i va mânca?i, vede?i sa nu va nimici?i voi între voi.
Gal 5:16  Zic dar: În Duhul sa umbla?i ?i sa nu împlini?i pofta trupului.
Gal 5:17  Caci trupul pofte?te împotriva duhului, iar duhul împotriva trupului; caci acestea se împotrivesc unul altuia, ca sa nu face?i cele ce a?i voi.
Gal 5:18  Iar de va purta?i în Duhul nu sunte?i sub Lege.
Gal 5:19  Iar faptele trupului sunt cunoscute, ?i ele sunt: adulter, desfrânare, necura?ie, destrabalare,
Gal 5:20  Închinare la idoli, fermecatorie, vrajbe, certuri, zavistii, mânii, gâlcevi, dezbinari, eresuri,
Gal 5:21  Pizmuiri, ucideri, be?ii, chefuri ?i cele asemenea acestora, pe care vi le spun dinainte, precum dinainte v-am ?i spus, ca cei ce fac unele ca acestea nu vor mo?teni împara?ia lui Dumnezeu.
Gal 5:22  Iar roada Duhului este dragostea, bucuria, pacea, îndelunga-rabdarea, bunatatea, facerea de bine, credin?a,
Gal 5:23  Blânde?ea, înfrânarea, cura?ia; împotriva unora ca acestea nu este lege.
Gal 5:24  Iar cei ce sunt ai lui Hristos Iisus ?i-au rastignit trupul împreuna cu patimile ?i cu poftele.
Gal 5:25  Daca traim în Duhul, în Duhul sa ?i umblam.
Gal 5:26  Sa nu fim iubitori de marire de?arta, suparându-ne unii pe al?ii ?i pizmuindu-ne unii pe al?ii.
Gal 6:1  Fra?ilor, chiar de va cadea un om în vreo gre?eala, voi cei duhovnice?ti îndrepta?i-l, pe unul ca acesta cu duhul blânde?ii, luând seama la tine însu?i, ca sa nu cazi ?i tu în ispita.
Gal 6:2  Purta?i-va sarcinile unii altora ?i a?a ve?i împlini legea lui Hristos.
Gal 6:3  Caci de se socote?te cineva ca este ceva, de?i nu este nimic, se în?eala pe sine însu?i.
Gal 6:4  Iar fapta lui însu?i sa ?i-o cerceteze fiecare ?i atunci va avea lauda, dar numai fa?a de sine însu?i ?i nu fa?a de altul.
Gal 6:5  Caci fiecare î?i va purta sarcina sa.
Gal 6:6  Cel care prime?te cuvântul înva?aturii sa faca parte înva?atorului sau din toate bunurile.
Gal 6:7  Nu va amagi?i: Dumnezeu nu Se lasa batjocorit; caci ce va semana omul, aceea va ?i secera.
Gal 6:8  Cel ce seamana în trupul sau însu?i, din trup va secera stricaciune; iar cel ce seamana în Duhul, din Duh va secera via?a ve?nica.
Gal 6:9  Sa nu încetam de a face binele, caci vom secera la timpul sau, daca nu ne vom lenevi.
Gal 6:10  Deci, dar, pâna când avem vreme, sa facem binele catre to?i, dar mai ales catre cei de o credin?a cu noi.
Gal 6:11  Vede?i cu ce fel de litere v-am scris eu, cu mâna mea.
Gal 6:12  Câ?i vor sa placa în trup, aceia va silesc sa va taia?i împrejur, numai ca sa nu fie prigoni?i pentru crucea lui Hristos.
Gal 6:13  Caci nici ei singuri, cei ce se taie împrejur, nu pazesc Legea, ci voiesc sa va taia?i voi împrejur, ca sa se laude ei în trupul vostru.
Gal 6:14  Iar mie, sa nu-mi fie a ma lauda, decât numai în crucea Domnului nostru Iisus Hristos, prin care lumea este rastignita pentru mine, ?i eu pentru lume!
Gal 6:15  Ca în Hristos Iisus nici taierea împrejur nu este ceva, nici netaierea împrejur, ci faptura cea noua.
Gal 6:16  ?i câ?i vor umbla dupa dreptarul acesta, - pace ?i mila asupra lor ?i asupra Israelului lui Dumnezeu!
Gal 6:17  De acum înainte, nimeni sa nu-mi mai faca suparare, caci eu port în trupul meu, semnele Domnului Iisus.
Gal 6:18  Harul Domnului nostru Iisus Hristos sa fie cu duhul vostru, fra?ilor! Amin.


\end{document}