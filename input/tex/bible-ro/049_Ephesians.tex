\begin{document}

\title{Efeseni}


\chapter{1}

\par 1 Pavel, apostol al lui Iisus Hristos prin voin?a lui Dumnezeu, sfin?ilor care sunt în Efes ?i credincio?ilor întru Hristos Iisus:
\par 2 Har voua ?i pace de la Dumnezeu, Tatal nostru, ?i de la Domnul Iisus Hristos!
\par 3 Binecuvântat fie Dumnezeu ?i Tatal Domnului nostru Iisus Hristos, Cel ce, întru Hristos, ne-a binecuvântat pe noi, în ceruri, cu toata binecuvântarea duhovniceasca;
\par 4 Precum întru El ne-a ?i ales, înainte de întemeierea lumii, ca sa fim sfin?i ?i fara de prihana înaintea Lui,
\par 5 Mai înainte rânduindu-ne, în a Sa iubire, spre înfierea întru El, prin Iisus Hristos, dupa buna socotin?a a voii Sale,
\par 6 Spre lauda slavei harului Sau, cu care ne-a daruit pe noi prin Fiul Sau cel iubit;
\par 7 Întru El avem rascumpararea prin sângele Lui ?i iertarea pacatelor, dupa boga?ia harului Lui,
\par 8 Pe care l-a facut sa prisoseasca în noi, în toata în?elepciunea ?i priceperea;
\par 9 Facându-ne cunoscuta taina voii Sale, dupa buna Lui socotin?a, astfel cum hotarâse în Sine mai înainte,
\par 10 Spre iconomia plinirii vremilor, ca toate sa fie iara?i unite în Hristos, cele din ceruri ?i cele de pe pamânt - toate întru El,
\par 11 Întru Care ?i mo?tenire am primit, rândui?i fiind mai înainte - dupa rânduiala Celui ce toate le lucreaza, potrivit sfatului voii Sale, -
\par 12 Ca sa fim spre lauda slavei Sale, noi cei ce mai înainte am nadajduit întru Hristos.
\par 13 Întru Care ?i voi, auzind cuvântul adevarului, Evanghelia mântuirii voastre, crezând în El, a?i fost pecetlui?i cu Sfântul Duh al fagaduin?ei,
\par 14 Care este arvuna mo?tenirii noastre, spre rascumpararea celor dobândi?i de El ?i spre lauda slavei Sale.
\par 15 Drept aceea, ?i eu auzind de credin?a voastra în Domnul Iisus ?i de dragostea cea catre to?i sfin?ii,
\par 16 Nu încetez a mul?umi pentru voi, pomenindu-va în rugaciunile mele,
\par 17 Ca Dumnezeul Domnului nostru Iisus Hristos, Tatal slavei, sa va dea voua duhul în?elepciunii ?i al descoperirii, spre deplina Lui cunoa?tere,
\par 18 ?i sa va lumineze ochii inimii, ca sa pricepe?i care este nadejdea la care v-a chemat, care este boga?ia slavei mo?tenirii Lui, în cei sfin?i,
\par 19 ?i cât de covâr?itoare este marimea puterii Lui fa?a de noi, dupa lucrarea puterii tariei Lui, pentru noi cei ce credem.
\par 20 Pe aceasta, Dumnezeu a lucrat-o în Hristos, sculându-L din mor?i ?i a?ezându-L de-a dreapta Sa, în ceruri,
\par 21 Mai presus decât toata începatoria ?i stapânia ?i puterea ?i domnia ?i decât tot numele ce se nume?te, nu numai în veacul acesta, ci ?i în cel viitor.
\par 22 ?i toate le-a supus sub picioarele Lui ?i, mai presus de toate, L-a dat pe El cap Bisericii,
\par 23 Care este trupul Lui, plinirea Celui ce pline?te toate întru to?i.

\chapter{2}

\par 1 Iar pe voi v-a facut vii, cei ce era?i mor?i prin gre?ealele ?i prin pacatele voastre,
\par 2 În care a?i umblat mai înainte, potrivit veacului lumii acesteia, potrivit stapânitorului puterii vazduhului, a duhului care lucreaza acum în fiii neascultarii,
\par 3 Întru care ?i noi to?i am petrecut odinioara, în poftele trupului nostru, facând voile trupului ?i ale sim?urilor ?i, din fire, eram fiii mâniei ca ?i ceilal?i.
\par 4 Dar Dumnezeu, bogat fiind în mila, pentru multa Sa iubire cu care ne-a iubit,
\par 5 Pe noi cei ce eram mor?i prin gre?ealele noastre, ne-a facut vii împreuna cu Hristos - prin har sunte?i mântui?i! -
\par 6 ?i împreuna cu El ne-a sculat ?i împreuna ne-a a?ezat întru ceruri, în Hristos Iisus,
\par 7 Ca sa arate în veacurile viitoare covâr?itoarea boga?ie a harului Sau, prin bunatatea ce a avut catre noi întru Hristos Iisus.
\par 8 Caci în har sunte?i mântui?i, prin credin?a, ?i aceasta nu e de la voi: este darul lui Dumnezeu;
\par 9 Nu din fapte, ca sa nu se laude nimeni.
\par 10 Pentru ca a Lui faptura suntem, zidi?i în Hristos Iisus spre fapte bune, pe care Dumnezeu le-a gatit mai înainte, ca sa umblam întru ele.
\par 11 De aceea, aduce?i-va aminte ca, odinioara, voi, pagânii cu trupul, numi?i netaiere-împrejur de catre cei numi?i taiere-împrejur, facuta de mâna în trup,
\par 12 Era?i, în vremea aceea, în afara de Hristos, înstraina?i de ceta?enia lui Israel, lipsi?i de nadejde ?i fara de Dumnezeu, în lume.
\par 13 Acum însa, fiind în Hristos Iisus, voi care altadata era?i departe, v-a?i apropiat prin sângele lui Hristos,
\par 14 Caci El este pacea noastra, El care a facut din cele doua - una, surpând peretele din mijloc al despar?iturii,
\par 15 Desfiin?ând vrajma?ia în trupul Sau, legea poruncilor ?i înva?aturile ei, ca, întru Sine, pe cei doi sa-i zideasca într-un singur om nou ?i sa întemeieze pacea,
\par 16 ?i sa-i împace cu Dumnezeu pe amândoi, uni?i într-un trup, prin cruce, omorând prin ea vrajma?ia.
\par 17 ?i, venind, a binevestit pace, voua celor de departe ?i pace celor de aproape;
\par 18 Ca prin El avem ?i unii ?i al?ii apropierea catre Tatal, într-un Duh.
\par 19 Deci, dar, nu mai sunte?i straini ?i locuitori vremelnici, ci sunte?i împreuna ceta?eni cu sfin?ii ?i casnici ai lui Dumnezeu,
\par 20 Zidi?i fiind pe temelia apostolilor ?i a proorocilor, piatra cea din capul unghiului fiind însu?i Iisus Hristos.
\par 21 Întru El, orice zidire bine alcatuita cre?te ca sa ajunga un loca? sfânt în Domnul,
\par 22 În Care voi împreuna sunte?i zidi?i, spre a fi loca? al lui Dumnezeu în Duh.

\chapter{3}

\par 1 Pentru aceasta, eu Pavel, întemni?atul lui Iisus Hristos pentru voi, neamurile,
\par 2 Daca în adevar a?i auzit de iconomia harului lui Dumnezeu care mi-a fost dat mie pentru voi,
\par 3 Ca prin descoperire mi s-a dat în cuno?tin?a aceasta taina, precum v-am scris înainte pe scurt.
\par 4 De unde, citind, pute?i sa cunoa?te?i în?elegerea mea în taina lui Hristos,
\par 5 Care, în alte veacuri, nu s-a facut cunoscuta fiilor oamenilor, cum s-a descoperit acum sfin?ilor Sai apostoli ?i prooroci, prin Duhul:
\par 6 Anume ca neamurile sunt împreuna mo?tenitoare (cu iudeii) ?i madulare ale aceluia?i trup ?i împreuna-parta?i ai fagaduin?ei, în Hristos Iisus, prin Evanghelie,
\par 7 Al carei slujitor m-am facut dupa darul harului lui Dumnezeu, ce mi-a fost dat mie, prin lucrarea puterii Sale;
\par 8 Mie, celui mai mic decât to?i sfin?ii, mi-a fost dat harul acesta, ca sa binevestesc neamurilor boga?ia lui Hristos, de nepatruns,
\par 9 ?i sa descopar tuturor care este iconomia tainei celei din veci ascunse în Dumnezeu, Ziditorul a toate, prin Iisus Hristos,
\par 10 Pentru ca în?elepciunea lui Dumnezeu cea de multe feluri sa se faca cunoscuta acum, prin Biserica, începatoriilor ?i stapâniilor, în ceruri,
\par 11 Dupa sfatul cel din veci, pe care El l-a împlinit în Hristos Iisus, Domnul nostru,
\par 12 Întru Care avem, prin credin?a în El, îndrazneala ?i apropiere de Dumnezeu, cu deplina încredere.
\par 13 De aceea, va rog sa nu va pierde?i cumpatul, din pricina necazurilor mele pentru voi; ele sunt slava voastra.
\par 14 Pentru aceasta, îmi plec genunchii înaintea Tatalui Domnului nostru Iisus Hristos,
\par 15 Din Care î?i trage numele orice neam în cer ?i pe pamânt,
\par 16 Sa va daruiasca, dupa boga?ia slavei Sale, ca sa fi?i puternic întari?i, prin Duhul Sau, în omul dinauntru,
\par 17 ?i Hristos sa Se sala?luiasca, prin credin?a, în inimile voastre, înradacina?i ?i întemeia?i fiind în iubire,
\par 18 Ca sa pute?i în?elege împreuna cu to?i sfin?ii care este largimea ?i lungimea ?i înal?imea ?i adâncimea,
\par 19 ?i sa cunoa?te?i iubirea lui Hristos, cea mai presus de cuno?tin?a, ca sa va umple?i de toata plinatatea lui Dumnezeu.
\par 20 Iar Celui ce poate sa faca, prin puterea cea lucratoare în noi, cu mult mai presus decât toate câte cerem sau pricepem noi,
\par 21 Lui fie slava în Biserica ?i întru Hristos Iisus în toate neamurile veacului veacurilor. Amin!

\chapter{4}

\par 1 De aceea, va îndemn, eu cel întemni?at pentru Domnul, sa umbla?i cu vrednicie, dupa chemarea cu care a?i fost chema?i,
\par 2 Cu toata smerenia ?i blânde?ea, cu îndelunga-rabdare, îngaduindu-va unii pe al?ii în iubire,
\par 3 Silindu-va sa pazi?i unitatea Duhului, întru legatura pacii.
\par 4 Este un trup ?i un Duh, precum ?i chema?i a?i fost la o singura nadejde a chemarii voastre;
\par 5 Este un Domn, o credin?a, un botez,
\par 6 Un Dumnezeu ?i Tatal tuturor, Care este peste toate ?i prin toate ?i întru to?i.
\par 7 Iar fiecaruia dintre noi, i s-a dat harul dupa masura darului lui Hristos.
\par 8 Pentru aceea zice: "Suindu-Se la înal?ime, a robit robime ?i a dat daruri oamenilor".
\par 9 Iar aceea ca "S-a suit" - ce înseamna decât ca S-a pogorât în par?ile cele mai de jos ale pamântului?
\par 10 Cel ce S-a pogorât, Acela este Care S-a suit mai presus de toate cerurile, ca pe toate sa le umple.
\par 11 ?i el a dat pe unii apostoli, pe al?ii prooroci, pe al?ii evangheli?ti, pe al?ii pastori ?i înva?atori,
\par 12 Spre desavâr?irea sfin?ilor, la lucrul slujirii, la zidirea trupului lui Hristos,
\par 13 Pâna vom ajunge to?i la unitatea credin?ei ?i a cunoa?terii Fiului lui Dumnezeu, la starea barbatului desavâr?it, la masura vârstei deplinata?ii lui Hristos.
\par 14 Ca sa nu mai fim copii du?i de valuri, purta?i încoace ?i încolo de orice vânt al înva?aturii, prin în?elaciunea oamenilor, prin vicle?ugul lor, spre uneltirea ratacirii,
\par 15 Ci ?inând adevarul, în iubire, sa cre?tem întru toate pentru El, Care este capul - Hristos.
\par 16 Din El, tot trupul bine alcatuit ?i bine încheiat, prin toate legaturile care îi dau tarie, î?i savâr?e?te cre?terea, potrivit lucrarii masurate fiecaruia din madulare, ?i se zide?te întru dragoste.
\par 17 A?adar, aceasta zic ?i marturisesc în Domnul, ca voi sa nu mai umbla?i de acum cum umbla neamurile, în de?ertaciunea min?ii lor,
\par 18 Întuneca?i fiind la cuget, înstraina?i fiind de via?a lui Dumnezeu, din pricina necuno?tin?ei care este în ei, din pricina împietririi inimii lor;
\par 19 Ace?tia petrec în nesim?ire ?i s-au dat pe sine desfrânarii, savâr?ind cu nesa? toate faptele necura?iei.
\par 20 Voi însa n-a?i înva?at a?a pe Hristos,
\par 21 Daca, într-adevar, L-a?i ascultat ?i a?i fost înva?a?i întru El, a?a cum este adevarul întru Iisus;
\par 22 Sa va dezbraca?i de vie?uirea voastra de mai înainte, de omul cel vechi, care se strica prin poftele amagitoare,
\par 23 ?i sa va înnoi?i în duhul min?ii voastre,
\par 24 ?i sa va îmbraca?i în omul cel nou, cel dupa Dumnezeu, zidit întru dreptate ?i în sfin?enia adevarului.
\par 25 Pentru aceea, lepadând minciuna, grai?i adevarul fiecare cu aproapele sau, caci unul altuia suntem madulare.
\par 26 Mânia?i-va ?i nu gre?i?i; soarele sa nu apuna peste mânia voastra.
\par 27 Nici nu da?i loc diavolului.
\par 28 Cel ce fura sa nu mai fure, ci mai vârtos sa se osteneasca lucrând cu mâinile sale, lucrul cel bun, ca sa aiba sa dea ?i celui ce are nevoie.
\par 29 Din gura voastra sa nu iasa nici un cuvânt rau, ci numai ce este bun, spre zidirea cea de trebuin?a, ca sa dea har celor ce asculta.
\par 30 Sa nu întrista?i Duhul cel Sfânt al lui Dumnezeu, întru Care a?i fost pecetlui?i pentru ziua rascumpararii.
\par 31 Orice amaraciune ?i suparare ?i mânie ?i izbucnire ?i defaimare sa piara de la voi, împreuna cu orice rautate.
\par 32 Ci fi?i buni între voi ?i milostivi, iertând unul altuia, precum ?i Dumnezeu v-a iertat voua, în Hristos.

\chapter{5}

\par 1 Fi?i dar urmatori ai lui Dumnezeu, ca ni?te fii iubi?i,
\par 2 ?i umbla?i întru iubire, precum ?i Hristos ne-a iubit pe noi ?i S-a dat pe Sine pentru noi, prinos ?i jertfa lui Dumnezeu, întru miros cu buna mireasma.
\par 3 Iar desfrâu ?i orice necura?ie ?i lacomie de avere nici sa se pomeneasca între voi, cum se cuvine sfin?ilor;
\par 4 Nici vorbe de ru?ine, nici vorbe nebune?ti, nici glume care nu se cuvin, ci mai degraba mul?umire.
\par 5 Caci aceasta s-o ?ti?i bine, ca nici un desfrânat, sau necurat, sau lacom de avere, care este un închinator la idoli, nu are mo?tenire în împara?ia lui Hristos ?i a lui Dumnezeu.
\par 6 Nimeni sa nu va amageasca cu cuvinte de?arte, caci pentru acestea vine mânia lui Dumnezeu peste fiii neascultarii.
\par 7 Deci sa nu va face?i parta?i cu ei.
\par 8 Altadata era?i întuneric, iar acum sunte?i lumina întru Domnul; umbla?i ca fii ai luminii!
\par 9 Pentru ca roada luminii e în orice bunatate, dreptate ?i adevar.
\par 10 Încercând ce este bineplacut Domnului.
\par 11 ?i nu fi?i parta?i la faptele cele fara roada ale întunericului, ci mai degraba, osândi?i-le pe fa?a.
\par 12 Caci cele ce se fac întru ascuns de ei, ru?ine este a le ?i grai.
\par 13 Iar tot ce este pe fa?a, se descopera prin lumina,
\par 14 Caci tot ceea ce este descoperit, lumina este. Pentru aceea zice: "De?teapta-te cel ce dormi ?i te scoala din mor?i ?i te va lumina Hristos".
\par 15 Deci lua?i seama cu grija, cum umbla?i, nu ca ni?te neîn?elep?i, ci ca cei în?elep?i,
\par 16 Rascumparând vremea, caci zilele rele sunt.
\par 17 Drept aceea, nu fi?i fara de minte, ci în?elege?i care este voia Domnului.
\par 18 ?i nu va îmbata?i de vin, în care este pierzare, ci va umple?i de Duhul.
\par 19 Vorbi?i între voi în psalmi ?i în laude ?i în cântari duhovnice?ti, laudând ?i cântând Domnului, în inimile voastre,
\par 20 Mul?umind totdeauna pentru toate întru numele Domnului nostru Iisus Hristos, lui Dumnezeu (?i) Tatal.
\par 21 Supune?i-va unul altuia, întru frica lui Hristos.
\par 22 Femeile sa se supuna barba?ilor lor ca Domnului,
\par 23 Pentru ca barbatul este cap femeii, precum ?i Hristos este cap Bisericii, trupul Sau, al carui mântuitor ?i este.
\par 24 Ci precum Biserica se supune lui Hristos, a?a ?i femeile barba?ilor lor, întru totul.
\par 25 Barba?ilor, iubi?i pe femeile voastre, dupa cum ?i Hristos a iubit Biserica, ?i S-a dat pe Sine pentru ea,
\par 26 Ca s-o sfin?easca, cura?ind-o cu baia apei prin cuvânt,
\par 27 ?i ca s-o înfa?i?eze Sie?i, Biserica slavita, neavând pata sau zbârcitura, ori altceva de acest fel, ci ca sa fie sfânta ?i fara de prihana.
\par 28 A?adar, barba?ii sunt datori sa-?i iubeasca femeile ca pe înse?i trupurile lor. Cel ce-?i iube?te femeia pe sine se iube?te.
\par 29 Caci nimeni vreodata nu ?i-a urât trupul sau, ci fiecare îl hrane?te ?i îl încalze?te, precum ?i Hristos Biserica,
\par 30 Pentru ca suntem madulare ale trupului Lui, din carnea Lui ?i din oasele Lui.
\par 31 De aceea, va lasa omul pe tatal sau ?i pe mama sa ?i se va alipi de femeia sa ?i vor fi amândoi un trup.
\par 32 Taina aceasta mare este; iar eu zic în Hristos ?i în Biserica.
\par 33 Astfel ?i voi, fiecare a?a sa-?i iubeasca femeia ca pe sine însu?i; iar femeia sa se teama de barbat.

\chapter{6}

\par 1 Copii, asculta?i pe parin?ii vo?tri în Domnul ca aceasta este cu dreptate.
\par 2 "Cinste?te pe tatal tau ?i pe mama ta, care este porunca cea dintâi cu fagaduin?a:
\par 3 Ca sa-?i fie ?ie bine ?i sa traie?ti ani mul?i pe pamânt".
\par 4 ?i voi, parin?ilor, nu întarâta?i la mânie pe copiii vo?tri, ci cre?te?i-i întru înva?atura ?i certarea Domnului.
\par 5 Slugilor, asculta?i de stapânii vo?tri cei dupa trup, cu frica ?i cu cutremur, întru cura?ia inimii voastre, ca ?i de Hristos,
\par 6 Nu slujind numai când sunt cu ochii pe voi, ca cei ce cauta sa placa oamenilor, ci ca slugile lui Hristos, facând din suflet voia lui Dumnezeu,
\par 7 Slujind cu bunavoin?a, ca ?i Domnului ?i nu ca oamenilor,
\par 8 ?tiind fiecare, fie rob, fie de sine stapân, ca faptele bune pe care le va face, pe acelea le va lua ca plata de la Domnul.
\par 9 Iar voi, stapânilor, face?i tot a?a fa?a de ei, lasând la o parte amenin?area, ?tiind ca Domnul lor ?i al vostru este în ceruri ?i ca la El nu încape partinire.
\par 10 În sfâr?it, fra?ilor, întari?i-va în Domnul ?i întru puterea tariei Lui.
\par 11 Îmbraca?i-va cu toate armele lui Dumnezeu, ca sa pute?i sta împotriva uneltirilor diavolului.
\par 12 Caci lupta noastra nu este împotriva trupului ?i a sângelui, ci împotriva începatoriilor, împotriva stapâniilor, împotriva stapânitorilor întunericului acestui veac, împotriva duhurilor rauta?ii, care sunt în vazduh.
\par 13 Pentru aceea, lua?i toate armele lui Dumnezeu, ca sa pute?i împotriva în ziua cea rea, ?i, toate biruindu-le, sa ramâne?i în picioare.
\par 14 Sta?i deci tari, având mijlocul vostru încins cu adevarul ?i îmbracându-va cu plato?a drepta?ii,
\par 15 ?i încal?a?i picioarele voastre, gata fiind pentru Evanghelia pacii.
\par 16 În toate lua?i pavaza credin?ei, cu care ve?i putea sa stinge?i toate sage?ile cele arzatoare ale vicleanului.
\par 17 Lua?i ?i coiful mântuirii ?i sabia Duhului, care este cuvântul lui Dumnezeu.
\par 18 Face?i în toata vremea, în Duhul, tot felul de rugaciuni ?i de cereri, ?i întru aceasta priveghind cu toata staruin?a ?i rugaciunea pentru to?i sfin?ii.
\par 19 Ruga?i-va ?i pentru mine, ca sa mi se dea mie cuvânt, când voi deschide gura mea, sa fac cunoscuta cu îndrazneala taina Evangheliei,
\par 20 Pe care o binevestesc, în lan?uri, ca sa vorbesc despre Evanghelie, fara sfiala, precum ?i trebuie sa vorbesc.
\par 21 Iar ca sa ?ti?i ?i voi cele despre mine ?i ce fac, Tihic, iubitul frate ?i credinciosul slujitor întru Domnul, vi le va aduce la cuno?tin?a pe toate;
\par 22 L-am trimis pe el la voi, pentru aceasta, ca sa afla?i cele despre noi ?i sa mângâie inimile voastre.
\par 23 Pace fra?ilor ?i dragoste, cu credin?a de la Dumnezeu-Tatal ?i de la Domnul Iisus Hristos.
\par 24 Harul fie cu to?i care iubesc pe Domnul nostru Iisus Hristos întru cura?ie.


\end{document}