\begin{document}

\title{Philippians}

Php 1:1  Pavel ?i Timotei, robi ai lui Hristos Iisus, tuturor sfin?ilor întru Hristos Iisus, celor ce sunt în Filipi, împreuna cu episcopii ?i diaconii:
Php 1:2  Har voua ?i pace, de la Dumnezeu, Tatal nostru, ?i de la Domnul Iisus Hristos.
Php 1:3  Mul?umesc Dumnezeului meu, ori de câte ori îmi amintesc de voi,
Php 1:4  Caci totdeauna, în toate rugaciunile mele, ma rog pentru voi to?i, cu bucurie,
Php 1:5  Pentru împarta?irea voastra întru Evanghelie, din ziua dintâi pâna acum.
Php 1:6  Sunt încredin?at de aceasta, ca cel ce a început în voi lucrul cel bun îl va duce la capat, pâna în ziua lui Hristos Iisus,
Php 1:7  Precum este cu dreptate sa gândesc astfel despre voi to?i; caci va port în inima mea, ?i în lan?urile mele, ?i în apararea ?i în întarirea Evangheliei, fiindca voi to?i sunte?i parta?i la acela?i har cu mine.
Php 1:8  Caci martor îmi este Dumnezeu, ca va doresc pe voi pe to?i, cu dragostea lui Hristos Iisus.
Php 1:9  ?i aceasta ma rog, ca iubirea voastra sa prisoseasca tot mai mult ?i mai mult, întru cuno?tin?a ?i întru orice pricepere,
Php 1:10  Ca sa cerca?i voi cele ce sunt mai de folos ?i ca sa fi?i cura?i ?i fara poticnire în ziua lui Hristos,
Php 1:11  Plini de roada drepta?ii, care este prin Iisus Hristos, spre slava ?i lauda lui Dumnezeu.
Php 1:12  Voiesc ca voi sa ?ti?i, fra?ilor, ca cele petrecute cu mine s-au întors mai degraba spre sporirea Evangheliei,
Php 1:13  În a?a fel ca lan?urile mele, pentru Hristos, au ajuns cunoscute în tot pretoriul ?i tuturor celorlal?i;
Php 1:14  ?i cei mai mul?i dintre fra?ii întru Domnul, îmbarbata?i prin lan?urile mele, au mai multa îndrazneala sa propovaduiasca fara teama cuvântul lui Dumnezeu.
Php 1:15  Unii, e drept, vestesc pe Hristos din pizma ?i din duh de cearta, al?ii însa din bunavoin?a.
Php 1:16  Unii o fac din iubire, ?tiind ca stau aici pentru apararea Evangheliei;
Php 1:17  Ceilal?i, însa, - din zavistie - vestesc pe Hristos, nu cu gânduri curate, ci socotind sa-mi sporeasca necazul în lan?urile mele.
Php 1:18  Dar ce este? Nimic altceva decât ca, în tot chipul, fie din fa?arie, fie în adevar, Hristos se propovaduie?te ?i întru aceasta ma bucur. ?i mereu ma voi bucura.
Php 1:19  Caci ?tiu ca aceasta îmi va fi mie spre mântuire, prin rugaciunile voastre ?i cu ajutorul Duhului lui Iisus Hristos,
Php 1:20  Dupa a?teptarea ?i nadejdea mea ca întru nimic nu voi fi ru?inat, ci, întru toata îndrazneala, precum totdeauna, a?a ?i acum, Hristos va fi preaslavit în trupul meu, fie prin via?a, fie prin moarte;
Php 1:21  Caci pentru mine via?a este Hristos ?i moartea un câ?tig.
Php 1:22  Daca însa a vie?ui în trup înseamna a da roada lucrului meu, nu ?tiu ce voi alege.
Php 1:23  Sunt strâns din doua par?i: doresc sa ma despart de trup ?i sa fiu împreuna cu Hristos, ?i aceasta e cu mult mai bine;
Php 1:24  Dar este mai de folos pentru voi sa zabovesc în trup.
Php 1:25  ?i având aceasta încredin?are, ?tiu ca voi ramâne ?i împreuna voi petrece cu voi cu to?i, spre sporirea voastra ?i spre bucuria credin?ei,
Php 1:26  Pentru ca lauda voastra sa prisoseasca în Hristos Iisus prin mine, atunci când voi veni iara?i între voi.
Php 1:27  Sa va purta?i numai în chip vrednic de Evanghelia lui Hristos, pentru ca, fie venind eu ?i vazându-va, fie nefiind de fa?a, sa aud despre voi ca sta?i într-un duh, nevoindu-va împreuna întru-un suflet, pentru credin?a Evangheliei,
Php 1:28  Fara sa va înfrico?a?i întru nimic de cei potrivnici, ceea ce pentru ei este un semn de pierzare, iar pentru voi de mântuire, ?i aceasta este de la Dumnezeu.
Php 1:29  Caci voua vi s-a daruit, pentru Hristos, nu numai sa crede?i în El, ci sa ?i patimi?i pentru El,
Php 1:30  Ducând aceea?i lupta, pe care a?i vazut-o la mine ?i o auzi?i acum despre mine.
Php 2:1  Deci, daca este vreun îndemn în Hristos, daca este vreo mângâiere a dragostei, daca este vreo împarta?ire a Duhului, daca este vreo milostivire ?i îndurare,
Php 2:2  Face?i-mi bucuria deplina, ca sa gândi?i la fel, având aceea?i iubire, acelea?i sim?iri, aceea?i cugetare.
Php 2:3  Nu face?i nimic din duh de cearta, nici din slava de?arta, ci cu smerenie unul pe altul socoteasca-l mai de cinste decât el însu?i.
Php 2:4  Sa nu caute nimeni numai ale sale, ci fiecare ?i ale altuia.
Php 2:5  Gândul acesta sa fie în voi care era ?i în Hristos Iisus,
Php 2:6  Care, Dumnezeu fiind în chip, n-a socotit o ?tirbire a fi El întocmai cu Dumnezeu,
Php 2:7  Ci S-a de?ertat pe Sine, chip de rob luând, facându-Se asemenea oamenilor, ?i la înfa?i?are aflându-Se ca un om,
Php 2:8  S-a smerit pe Sine, ascultator facându-Se pâna la moarte, ?i înca moarte pe cruce.
Php 2:9  Pentru aceea, ?i Dumnezeu L-a preaînal?at ?i I-a daruit Lui nume, care este mai presus de orice nume;
Php 2:10  Ca întru numele lui Iisus tot genunchiul sa se plece, al celor cere?ti ?i al celor pamânte?ti ?i al celor de dedesubt.
Php 2:11  ?i sa marturiseasca toata limba ca Domn este Iisus Hristos, întru slava lui Dumnezeu-Tatal.
Php 2:12  Drept aceea, iubi?ii mei, precum totdeauna m-a?i ascultat, nu numai când eram de fa?a, ci cu atât mai mult acum când sunt departe, cu frica ?i cu cutremur lucra?i mântuirea voastra;
Php 2:13  Caci Dumnezeu este Cel ce lucreaza în voi ?i ca sa voi?i ?i ca sa savâr?i?i, dupa a Lui bunavoin?a.
Php 2:14  Toate sa le face?i fara de cârtire ?i fara de îndoiala,
Php 2:15  Ca sa fi?i fara de prihana ?i cura?i, fii ai lui Dumnezeu neîntina?i în mijlocul unui neam rau ?i stricat ?i întru care straluci?i ca ni?te luminatori în lume,
Php 2:16  ?inând cu putere cuvântul vie?ii, spre lauda mea în ziua lui Hristos, ca nu în zadar am alergat, nici în zadar m-am ostenit.
Php 2:17  ?i chiar daca mi-a? varsa sângele pentru jertfa ?i slujirea credin?ei voastre, ma bucur ?i va fericesc pe voi pe to?i.
Php 2:18  Asemenea ?i voi bucura?i-va ?i ferici?i-ma.
Php 2:19  Ci nadajduiesc întru Domnul Iisus, ca voi trimite pe Timotei la voi, fara de zabava, ca ?i eu sa fiu cu inima buna, aflând ve?ti despre voi.
Php 2:20  Caci nu am pe nimeni altul, la un gând cu mine ?i care sa va poarte grija cu adevarat,
Php 2:21  Fiindca to?i cauta ale lor, nu ale lui Iisus Hristos.
Php 2:22  Dar încercarea lui o cunoa?te?i, caci împreuna cu mine a slujit Evanghelia, întocmai ca un copil lânga tatal sau.
Php 2:23  Pe el deci nadajduiesc sa-l trimit, îndata ce voi vedea ce va fi cu mine.
Php 2:24  Sunt, însa, încredin?at în Domnul ca eu însumi voi veni în curând.
Php 2:25  Am socotit de grabnica nevoie sa va trimit pe Epafrodit, fratele ?i împreuna cu mine lucrator ?i luptator, cum ?i trimisul vostru ?i slujitorul nevoilor mele,
Php 2:26  Fiindca avea mare dor de voi to?i ?i era mâhnit fiindca a?i auzit ca a fost bolnav.
Php 2:27  Într-adevar, bolnav a fost aproape de moarte, dar Dumnezeu a avut mila de el ?i nu numai de el, ci ?i de mine, ca sa nu am întristare peste întristare.
Php 2:28  Deci l-am trimis mai degraba, ca, vazându-l, voi iara?i sa va bucura?i, iar eu sa fiu mai pu?in mâhnit.
Php 2:29  Primi?i-l dar întru Domnul, cu toata bucuria ?i pe unii ca ace?tia întru cinste sa-i ave?i,
Php 2:30  Fiindca pentru lucrul lui Hristos a mers pâna aproape de moarte, punându-?i via?a în primejdie, ca sa împlineasca lipsa voastra în slujirea mea.
Php 3:1  Mai departe, fra?ii mei, bucura?i-va întru Domnul. Ca sa va scriu acelea?i lucruri, mie nu-mi este anevoie, iar voua va este de folos.
Php 3:2  Pazi?i-va de câini! Pazi?i-va de lucratorii cei rai! Pazi?i-va de taierea împrejur.
Php 3:3  Pentru ca noi suntem taierea împrejur, noi cei ce slujim în Duhul lui Dumnezeu, ?i ne laudam întru Hristos Iisus ?i nu ne bizuim pe trup,
Php 3:4  De?i eu a? putea sa ma bizui ?i pe trup. Daca vreun altul socote?te ca se poate bizui pe trup, eu cu atât mai mult!
Php 3:5  La opt zile, am fost taiat împrejur; sunt din neamul lui Israel, din semin?ia lui Veniamin, evreu din evrei, dupa lege fariseu;
Php 3:6  În ce prive?te râvna, prigonitor al Bisericii; în ce prive?te dreptatea cea din Lege, fara de prihana.
Php 3:7  Dar cele ce îmi erau mie câ?tig, acestea le-am socotit pentru Hristos paguba.
Php 3:8  Ba mai mult: eu pe toate le socotesc ca sunt paguba, fa?a de înal?imea cunoa?terii lui Hristos Iisus, Domnul meu, pentru Care m-am lipsit de toate ?i le privesc drept gunoaie, ca pe Hristos sa dobândesc,
Php 3:9  ?i sa ma aflu întru El, nu având dreptatea mea cea din Lege, ci pe aceea care este prin credin?a în Hristos, dreptatea cea de la Dumnezeu, pe temeiul credin?ei,
Php 3:10  Ca sa-L cunosc pe El ?i puterea învierii Lui ?i sa fiu primit parta? la patimile Lui, facându-ma asemenea cu El în moartea Lui,
Php 3:11  Ca, doar, sa pot ajunge la învierea cea din mor?i.
Php 3:12  Nu (zic) ca am ?i dobândit îndreptarea, ori ca sunt desavâr?it; dar o urmaresc ca doar o voi prinde, întrucât ?i eu am fost prins de Hristos Iisus.
Php 3:13  Fra?ilor, eu înca nu socotesc sa o fi cucerit,
Php 3:14  Dar una fac: uitând cele ce sunt în urma mea, ?i tinzând catre cele dinainte, alerg la ?inta, la rasplata chemarii de sus, a lui Dumnezeu, întru Hristos Iisus.
Php 3:15  A?adar, câ?i suntem desavâr?i?i aceasta sa gândim; ?i daca gândi?i ceva în alt fel, Dumnezeu va va descoperi ?i aceea.
Php 3:16  Dar de acolo unde am ajuns, sa urmam acela?i dreptar, sa gândim la fel.
Php 3:17  Fra?ilor, face?i-va urmaritorii mei ?i uita?i-va la aceia care umbla astfel precum ne ave?i pilda pe noi.
Php 3:18  Caci mul?i, despre care v-am vorbit adeseori, iar acum va spun ?i plângând, se poarta ca du?mani ai crucii lui Hristos.
Php 3:19  Sfâr?itul acestora este pieirea. Pântecele este dumnezeul lor, iar marirea lor este întru ru?inea lor, ca unii care au în gând cele pamânte?ti.
Php 3:20  Cât despre noi, cetatea noastra este în ceruri, de unde ?i a?teptam Mântuitor, pe Domnul Iisus Hristos,
Php 3:21  Care va schimba la înfa?i?are trupul smereniei noastre ca sa fie asemenea trupului slavei Sale, lucrând cu puterea ce are de a-?i supune Sie?i toate.
Php 4:1  Deci, fra?ii mei iubi?i ?i mult dori?i, bucuria ?i cununa mea, a?a sa sta?i întru Domnul, iubi?ii mei.
Php 4:2  Rog pe Evodia ?i rog pe Sintihi sa aiba acelea?i gânduri în Domnul.
Php 4:3  Înca te rog ?i pe tine, credinciosule Sizig, ajuta-le lor, ca pe unele care au luptat pentru Evanghelie, împreuna cu mine ?i cu Clement ?i cu ceilal?i împreuna-lucratori cu mine, ale caror nume sunt scrise în cartea vie?ii.
Php 4:4  Bucura?i-va pururea întru Domnul. ?i iara?i zic: Bucura?i-va.
Php 4:5  Îngaduin?a voastra sa se faca ?tiuta tuturor oamenilor. Domnul este aproape.
Php 4:6  Nu va împovara?i cu nici o grija. Ci întru toate, prin închinaciune ?i prin ruga cu mul?umire, cererile voastre sa fie aratate lui Dumnezeu.
Php 4:7  ?i pacea lui Dumnezeu, care covâr?e?te orice minte, sa pazeasca inimile voastre ?i cugetele voastre, întru Hristos Iisus.
Php 4:8  Mai departe, fra?ilor, câte sunt adevarate, câte sunt de cinste, câte sunt drepte, câte sunt curate, câte sunt vrednice de iubit, câte sunt cu nume bun, orice virtute ?i orice lauda, la acestea sa va fie gândul.
Php 4:9  Cele ce a?i înva?at ?i a?i primit ?i a?i auzit ?i a?i vazut la mine, acestea sa le face?i, ?i Dumnezeul pacii va fi cu voi.
Php 4:10  M-am bucurat mult în Domnul, ca a înflorit iara?i purtarea voastra de grija pentru mine, precum o ?i avea?i, dar v-a lipsit prilejul.
Php 4:11  N-o spun ca ?i cum a? duce lipsa, fiindca eu m-am deprins sa fiu îndestulat cu ceea ce am.
Php 4:12  ?tiu sa fiu ?i smerit, ?tiu sa am ?i de prisos; în orice ?i în toate m-am înva?at sa fiu ?i satul ?i flamând, ?i în bel?ug ?i în lipsa.
Php 4:13  Toate le pot întru Hristos, Cel care ma întare?te.
Php 4:14  Însa bine a?i facut ca a?i împarta?it cu mine necazul.
Php 4:15  Doar ?i voi ?ti?i, filipenilor, ca la începutul Evangheliei, când am plecat din Macedonia, nici o Biserica nu s-a unit cu mine, când era vorba de dat ?i de primit, decât voi singuri.
Php 4:16  Pentru ca ?i în Tesalonic, o data ?i a doua oara, mi-a?i trimis ca sa am cele trebuincioase.
Php 4:17  Nu ca doar caut darul vostru, dar caut rodul care prisose?te, în folosul vostru.
Php 4:18  Am de toate ?i am ?i de prisos; m-am îndestulat primind de la Epafrodit cele ce mi-a?i trimis, miros cu buna mireasma, jertfa primita, bineplacuta lui Dumnezeu.
Php 4:19  Iar Dumnezeul meu sa împlineasca toata lipsa voastra dupa boga?ia Sa, cu slava, întru Hristos Iisus.
Php 4:20  Iar lui Dumnezeu ?i Tatalui nostru, slava în vecii vecilor! Amin.
Php 4:21  Îmbra?i?a?i în Hristos Iisus pe to?i sfin?ii. Va îmbra?i?eaza pe voi fra?ii care sunt împreuna cu mine.
Php 4:22  Va îmbra?i?eaza pe voi to?i sfin?ii, mai ales cei din casa Cesarului.
Php 4:23  Harul Domnului Iisus Hristos sa fie cu duhul vostru!


\end{document}