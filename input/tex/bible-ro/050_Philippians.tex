\begin{document}

\title{Filipeni}


\chapter{1}

\par 1 Pavel și Timotei, robi ai lui Hristos Iisus, tuturor sfinților întru Hristos Iisus, celor ce sunt în Filipi, împreună cu episcopii și diaconii:
\par 2 Har vouă și pace, de la Dumnezeu, Tatăl nostru, și de la Domnul Iisus Hristos.
\par 3 Mulțumesc Dumnezeului meu, ori de câte ori îmi amintesc de voi,
\par 4 Căci totdeauna, în toate rugăciunile mele, mă rog pentru voi toți, cu bucurie,
\par 5 Pentru împărtășirea voastră întru Evanghelie, din ziua dintâi până acum.
\par 6 Sunt încredințat de aceasta, că cel ce a început în voi lucrul cel bun îl va duce la capăt, până în ziua lui Hristos Iisus,
\par 7 Precum este cu dreptate să gândesc astfel despre voi toți; căci vă port în inima mea, și în lanțurile mele, și în apărarea și în întărirea Evangheliei, fiindcă voi toți sunteți părtași la același har cu mine.
\par 8 Căci martor îmi este Dumnezeu, că vă doresc pe voi pe toți, cu dragostea lui Hristos Iisus.
\par 9 Și aceasta mă rog, ca iubirea voastră să prisosească tot mai mult și mai mult, întru cunoștință și întru orice pricepere,
\par 10 Ca să cercați voi cele ce sunt mai de folos și ca să fiți curați și fără poticnire în ziua lui Hristos,
\par 11 Plini de roada dreptății, care este prin Iisus Hristos, spre slava și lauda lui Dumnezeu.
\par 12 Voiesc ca voi să știți, fraților, că cele petrecute cu mine s-au întors mai degrabă spre sporirea Evangheliei,
\par 13 În așa fel că lanțurile mele, pentru Hristos, au ajuns cunoscute în tot pretoriul și tuturor celorlalți;
\par 14 Și cei mai mulți dintre frații întru Domnul, îmbărbătați prin lanțurile mele, au mai multă îndrăzneală să propovăduiască fără teamă cuvântul lui Dumnezeu.
\par 15 Unii, e drept, vestesc pe Hristos din pizmă și din duh de ceartă, alții însă din bunăvoință.
\par 16 Unii o fac din iubire, știind că stau aici pentru apărarea Evangheliei;
\par 17 Ceilalți, însă, - din zavistie - vestesc pe Hristos, nu cu gânduri curate, ci socotind să-mi sporească necazul în lanțurile mele.
\par 18 Dar ce este? Nimic altceva decât că, în tot chipul, fie din fățărie, fie în adevăr, Hristos se propovăduiește și întru aceasta mă bucur. Și mereu mă voi bucura.
\par 19 Căci știu că aceasta îmi va fi mie spre mântuire, prin rugăciunile voastre și cu ajutorul Duhului lui Iisus Hristos,
\par 20 După așteptarea și nădejdea mea că întru nimic nu voi fi rușinat, ci, întru toată îndrăzneala, precum totdeauna, așa și acum, Hristos va fi preaslăvit în trupul meu, fie prin viață, fie prin moarte;
\par 21 Căci pentru mine viață este Hristos și moartea un câștig.
\par 22 Dacă însă a viețui în trup înseamnă a da roadă lucrului meu, nu știu ce voi alege.
\par 23 Sunt strâns din două părți: doresc să mă despart de trup și să fiu împreună cu Hristos, și aceasta e cu mult mai bine;
\par 24 Dar este mai de folos pentru voi să zăbovesc în trup.
\par 25 Și având această încredințare, știu că voi rămâne și împreună voi petrece cu voi cu toți, spre sporirea voastră și spre bucuria credinței,
\par 26 Pentru ca lauda voastră să prisosească în Hristos Iisus prin mine, atunci când voi veni iarăși între voi.
\par 27 Să vă purtați numai în chip vrednic de Evanghelia lui Hristos, pentru ca, fie venind eu și văzându-vă, fie nefiind de față, să aud despre voi că stați într-un duh, nevoindu-vă împreună întru-un suflet, pentru credința Evangheliei,
\par 28 Fără să vă înfricoșați întru nimic de cei potrivnici, ceea ce pentru ei este un semn de pierzare, iar pentru voi de mântuire, și aceasta este de la Dumnezeu.
\par 29 Căci vouă vi s-a dăruit, pentru Hristos, nu numai să credeți în El, ci să și pătimiți pentru El,
\par 30 Ducând aceeași luptă, pe care ați văzut-o la mine și o auziți acum despre mine.

\chapter{2}

\par 1 Deci, dacă este vreun îndemn în Hristos, dacă este vreo mângâiere a dragostei, dacă este vreo împărtășire a Duhului, dacă este vreo milostivire și îndurare,
\par 2 Faceți-mi bucuria deplină, ca să gândiți la fel, având aceeași iubire, aceleași simțiri, aceeași cugetare.
\par 3 Nu faceți nimic din duh de ceartă, nici din slavă deșartă, ci cu smerenie unul pe altul socotească-l mai de cinste decât el însuși.
\par 4 Să nu caute nimeni numai ale sale, ci fiecare și ale altuia.
\par 5 Gândul acesta să fie în voi care era și în Hristos Iisus,
\par 6 Care, Dumnezeu fiind în chip, n-a socotit o știrbire a fi El întocmai cu Dumnezeu,
\par 7 Ci S-a deșertat pe Sine, chip de rob luând, făcându-Se asemenea oamenilor, și la înfățișare aflându-Se ca un om,
\par 8 S-a smerit pe Sine, ascultător făcându-Se până la moarte, și încă moarte pe cruce.
\par 9 Pentru aceea, și Dumnezeu L-a preaînălțat și I-a dăruit Lui nume, care este mai presus de orice nume;
\par 10 Ca întru numele lui Iisus tot genunchiul să se plece, al celor cerești și al celor pământești și al celor de dedesubt.
\par 11 Și să mărturisească toată limba că Domn este Iisus Hristos, întru slava lui Dumnezeu-Tatăl.
\par 12 Drept aceea, iubiții mei, precum totdeauna m-ați ascultat, nu numai când eram de față, ci cu atât mai mult acum când sunt departe, cu frică și cu cutremur lucrați mântuirea voastră;
\par 13 Căci Dumnezeu este Cel ce lucrează în voi și ca să voiți și ca să săvârșiți, după a Lui bunăvoință.
\par 14 Toate să le faceți fără de cârtire și fără de îndoială,
\par 15 Ca să fiți fără de prihană și curați, fii ai lui Dumnezeu neîntinați în mijlocul unui neam rău și stricat și întru care străluciți ca niște luminători în lume,
\par 16 Ținând cu putere cuvântul vieții, spre lauda mea în ziua lui Hristos, că nu în zadar am alergat, nici în zadar m-am ostenit.
\par 17 Și chiar dacă mi-aș vărsa sângele pentru jertfa și slujirea credinței voastre, mă bucur și vă fericesc pe voi pe toți.
\par 18 Asemenea și voi bucurați-vă și fericiți-mă.
\par 19 Ci nădăjduiesc întru Domnul Iisus, că voi trimite pe Timotei la voi, fără de zăbavă, ca și eu să fiu cu inima bună, aflând vești despre voi.
\par 20 Căci nu am pe nimeni altul, la un gând cu mine și care să vă poarte grija cu adevărat,
\par 21 Fiindcă toți caută ale lor, nu ale lui Iisus Hristos.
\par 22 Dar încercarea lui o cunoașteți, căci împreună cu mine a slujit Evanghelia, întocmai ca un copil lângă tatăl său.
\par 23 Pe el deci nădăjduiesc să-l trimit, îndată ce voi vedea ce va fi cu mine.
\par 24 Sunt, însă, încredințat în Domnul că eu însumi voi veni în curând.
\par 25 Am socotit de grabnică nevoie să vă trimit pe Epafrodit, fratele și împreună cu mine lucrător și luptător, cum și trimisul vostru și slujitorul nevoilor mele,
\par 26 Fiindcă avea mare dor de voi toți și era mâhnit fiindcă ați auzit că a fost bolnav.
\par 27 Într-adevăr, bolnav a fost aproape de moarte, dar Dumnezeu a avut milă de el și nu numai de el, ci și de mine, ca să nu am întristare peste întristare.
\par 28 Deci l-am trimis mai degrabă, ca, văzându-l, voi iarăși să vă bucurați, iar eu să fiu mai puțin mâhnit.
\par 29 Primiți-l dar întru Domnul, cu toată bucuria și pe unii ca aceștia întru cinste să-i aveți,
\par 30 Fiindcă pentru lucrul lui Hristos a mers până aproape de moarte, punându-și viața în primejdie, ca să împlinească lipsa voastră în slujirea mea.

\chapter{3}

\par 1 Mai departe, frații mei, bucurați-vă întru Domnul. Ca să vă scriu aceleași lucruri, mie nu-mi este anevoie, iar vouă vă este de folos.
\par 2 Păziți-vă de câini! Păziți-vă de lucrătorii cei răi! Păziți-vă de tăierea împrejur.
\par 3 Pentru că noi suntem tăierea împrejur, noi cei ce slujim în Duhul lui Dumnezeu, și ne lăudăm întru Hristos Iisus și nu ne bizuim pe trup,
\par 4 Deși eu aș putea să mă bizui și pe trup. Dacă vreun altul socotește că se poate bizui pe trup, eu cu atât mai mult!
\par 5 La opt zile, am fost tăiat împrejur; sunt din neamul lui Israel, din seminția lui Veniamin, evreu din evrei, după lege fariseu;
\par 6 În ce privește râvna, prigonitor al Bisericii; în ce privește dreptatea cea din Lege, fără de prihană.
\par 7 Dar cele ce îmi erau mie câștig, acestea le-am socotit pentru Hristos pagubă.
\par 8 Ba mai mult: eu pe toate le socotesc că sunt pagubă, față de înălțimea cunoașterii lui Hristos Iisus, Domnul meu, pentru Care m-am lipsit de toate și le privesc drept gunoaie, ca pe Hristos să dobândesc,
\par 9 Și să mă aflu întru El, nu având dreptatea mea cea din Lege, ci pe aceea care este prin credința în Hristos, dreptatea cea de la Dumnezeu, pe temeiul credinței,
\par 10 Ca să-L cunosc pe El și puterea învierii Lui și să fiu primit părtaș la patimile Lui, făcându-mă asemenea cu El în moartea Lui,
\par 11 Ca, doar, să pot ajunge la învierea cea din morți.
\par 12 Nu (zic) că am și dobândit îndreptarea, ori că sunt desăvârșit; dar o urmăresc ca doar o voi prinde, întrucât și eu am fost prins de Hristos Iisus.
\par 13 Fraților, eu încă nu socotesc să o fi cucerit,
\par 14 Dar una fac: uitând cele ce sunt în urma mea, și tinzând către cele dinainte, alerg la țintă, la răsplata chemării de sus, a lui Dumnezeu, întru Hristos Iisus.
\par 15 Așadar, câți suntem desăvârșiți aceasta să gândim; și dacă gândiți ceva în alt fel, Dumnezeu vă va descoperi și aceea.
\par 16 Dar de acolo unde am ajuns, să urmăm același dreptar, să gândim la fel.
\par 17 Fraților, faceți-vă urmăritorii mei și uitați-vă la aceia care umblă astfel precum ne aveți pildă pe noi.
\par 18 Căci mulți, despre care v-am vorbit adeseori, iar acum vă spun și plângând, se poartă ca dușmani ai crucii lui Hristos.
\par 19 Sfârșitul acestora este pieirea. Pântecele este dumnezeul lor, iar mărirea lor este întru rușinea lor, ca unii care au în gând cele pământești.
\par 20 Cât despre noi, cetatea noastră este în ceruri, de unde și așteptăm Mântuitor, pe Domnul Iisus Hristos,
\par 21 Care va schimba la înfățișare trupul smereniei noastre ca să fie asemenea trupului slavei Sale, lucrând cu puterea ce are de a-Și supune Sieși toate.

\chapter{4}

\par 1 Deci, frații mei iubiți și mult doriți, bucuria și cununa mea, așa să stați întru Domnul, iubiții mei.
\par 2 Rog pe Evodia și rog pe Sintihi să aibă aceleași gânduri în Domnul.
\par 3 Încă te rog și pe tine, credinciosule Sizig, ajută-le lor, ca pe unele care au luptat pentru Evanghelie, împreună cu mine și cu Clement și cu ceilalți împreună-lucrători cu mine, ale căror nume sunt scrise în cartea vieții.
\par 4 Bucurați-vă pururea întru Domnul. Și iarăși zic: Bucurați-vă.
\par 5 Îngăduința voastră să se facă știută tuturor oamenilor. Domnul este aproape.
\par 6 Nu vă împovărați cu nici o grijă. Ci întru toate, prin închinăciune și prin rugă cu mulțumire, cererile voastre să fie arătate lui Dumnezeu.
\par 7 Și pacea lui Dumnezeu, care covârșește orice minte, să păzească inimile voastre și cugetele voastre, întru Hristos Iisus.
\par 8 Mai departe, fraților, câte sunt adevărate, câte sunt de cinste, câte sunt drepte, câte sunt curate, câte sunt vrednice de iubit, câte sunt cu nume bun, orice virtute și orice laudă, la acestea să vă fie gândul.
\par 9 Cele ce ați învățat și ați primit și ați auzit și ați văzut la mine, acestea să le faceți, și Dumnezeul păcii va fi cu voi.
\par 10 M-am bucurat mult în Domnul, că a înflorit iarăși purtarea voastră de grijă pentru mine, precum o și aveați, dar v-a lipsit prilejul.
\par 11 N-o spun ca și cum aș duce lipsă, fiindcă eu m-am deprins să fiu îndestulat cu ceea ce am.
\par 12 Știu să fiu și smerit, știu să am și de prisos; în orice și în toate m-am învățat să fiu și sătul și flămând, și în belșug și în lipsă.
\par 13 Toate le pot întru Hristos, Cel care mă întărește.
\par 14 Însă bine ați făcut că ați împărtășit cu mine necazul.
\par 15 Doar și voi știți, filipenilor, că la începutul Evangheliei, când am plecat din Macedonia, nici o Biserică nu s-a unit cu mine, când era vorba de dat și de primit, decât voi singuri.
\par 16 Pentru că și în Tesalonic, o dată și a doua oară, mi-ați trimis ca să am cele trebuincioase.
\par 17 Nu că doar caut darul vostru, dar caut rodul care prisosește, în folosul vostru.
\par 18 Am de toate și am și de prisos; m-am îndestulat primind de la Epafrodit cele ce mi-ați trimis, miros cu bună mireasmă, jertfă primită, bineplăcută lui Dumnezeu.
\par 19 Iar Dumnezeul meu să împlinească toată lipsa voastră după bogăția Sa, cu slavă, întru Hristos Iisus.
\par 20 Iar lui Dumnezeu și Tatălui nostru, slavă în vecii vecilor! Amin.
\par 21 Îmbrățișați în Hristos Iisus pe toți sfinții. Vă îmbrățișează pe voi frații care sunt împreună cu mine.
\par 22 Vă îmbrățișează pe voi toți sfinții, mai ales cei din casa Cesarului.
\par 23 Harul Domnului Iisus Hristos să fie cu duhul vostru!


\end{document}