\begin{document}

\title{Coloseni}


\chapter{1}

\par 1 Pavel, apostol al lui Hristos Iisus prin voin?a lui Dumnezeu, ?i Timotei, fratele nostru,
\par 2 Sfin?ilor ?i credincio?ilor fra?i întru Hristos, care sunt în Colose: Har voua ?i pace de la Dumnezeu, Tatal nostru, ?i de la Domnul nostru Iisus Hristos.
\par 3 Mul?umim lui Dumnezeu ?i Tatal Domnului nostru Iisus Hristos, rugându-ne totdeauna pentru voi,
\par 4 Caci am auzit despre credin?a voastra în Hristos Iisus ?i despre dragostea ce ave?i catre to?i sfin?ii,
\par 5 Pentru nadejdea cea gatita voua în ceruri, de care a?i auzit mai înainte în cuvântul adevarului Evangheliei,
\par 6 Care, ajungând la voi, precum ?i în toata lumea, aduce roada ?i spore?te întocmai ca la voi, din ziua în care a?i auzit ?i a?i cunoscut, întru adevar, harul lui Dumnezeu.
\par 7 Precum a?i primit înva?atura de la Epafras, iubitul, împreuna-slujitor cu noi, care este pentru voi credincios slujitor al lui Hristos,
\par 8 ?i care ne-a aratat iubirea voastra cea întru Duhul.
\par 9 De aceea ?i noi, din ziua în care am auzit aceasta, nu încetam sa ne rugam pentru voi ?i sa cerem sa va umple?i de cunoa?terea voii Lui, întru toata în?elepciunea ?i priceperea duhovniceasca,
\par 10 Ca voi sa umbla?i cu vrednicie întru Domnul, placu?i Lui în toate, aducând roada în orice lucru bun ?i sporind în cunoa?terea lui Dumnezeu.
\par 11 ?i întari?i fiind cu toata puterea, dupa puterea slavei Lui, spre toata staruin?a ?i îndelunga-rabdare,
\par 12 Mul?umind cu bucurie Tatalui celui ce ne-a învrednicit pe noi sa luam parte la mo?tenirea sfin?ilor, întru lumina.
\par 13 El ne-a scos de sub puterea întunericului ?i ne-a stramutat în împara?ia Fiului iubirii Sale,
\par 14 Întru Care avem rascumpararea prin sângele Lui, adica iertarea pacatelor;
\par 15 Acesta este chipul lui Dumnezeu celui nevazut, mai întâi nascut decât toata faptura.
\par 16 Pentru ca întru El au fost facute toate, cele din ceruri ?i cele de pe pamânt, cele vazute, ?i cele nevazute, fie tronuri, fie domnii, fie începatorii, fie stapânii. Toate s-au facut prin El ?i pentru El.
\par 17 El este mai înainte decât toate ?i toate prin El sunt a?ezate.
\par 18 ?i El este capul trupului, al Bisericii; El este începutul, întâiul nascut din mor?i, ca sa fie El cel dintâi întru toate.
\par 19 Caci în El a binevoit (Dumnezeu) sa sala?luiasca toata plinirea.
\par 20 ?i printr-Însul toate cu Sine sa le împace, fie cele de pe pamânt, fie cele din ceruri, facând pace prin El, prin sângele crucii Sale.
\par 21 Dar pe voi, care oarecând era?i înstraina?i ?i vrajma?i cu mintea voastra catre lucrurile rele, de acum v-a împacat,
\par 22 Prin moartea (Fiului Sau) în trupul carnii Lui, ca sa va puna înaintea Sa sfin?i, fara de prihana ?i nevinova?i,
\par 23 Daca, într-adevar, ramâne?i întemeia?i în credin?a, întari?i ?i neclinti?i de la nadejdea Evangheliei pe care a?i auzit-o, care a fost propovaduita la toata faptura de sub cer ?i al carei slujitor m-am facut eu, Pavel.
\par 24 Acum ma bucur de suferin?ele mele pentru voi ?i împlinesc, în trupul meu, lipsurile necazurilor lui Hristos, pentru trupul Lui, adica Biserica,
\par 25 Al carei slujitor m-am facut, potrivit iconomiei lui Dumnezeu, ce mi-a fost data mie pentru voi, ca sa aduc la îndeplinire cuvântul lui Dumnezeu,
\par 26 Taina cea din veci ascunsa neamurilor, iar acum descoperita sfin?ilor Sai,
\par 27 Carora a voit Dumnezeu sa le arate care este boga?ia slavei acestei taine între neamuri, adica Hristos cel dintru voi, nadejdea slavei.
\par 28 Pe El noi Îl vestim, sfatuind pe orice om ?i înva?ând pe orice om, întru toata în?elepciunea, ca sa înfa?i?am pe tot omul, desavâr?it, în Hristos Iisus.
\par 29 Spre aceasta ma ?i ostenesc ?i ma lupt, potrivit lucrarii Lui, care se savâr?e?te în mine cu putere.

\chapter{2}

\par 1 Caci voiesc ca voi sa ?ti?i cât de mare lupta am pentru voi ?i pentru cei din Laodiceea ?i pentru to?i câ?i n-au vazut fa?a mea în trup,
\par 2 Ca sa se mângâie inimile lor, ?i ca ei, strâns uni?i în iubire, sa aiba bel?ugul deplinei în?elegeri pentru cunoa?terea tainei lui Dumnezeu-Tatal ?i a lui Hristos,
\par 3 Întru care sunt ascunse toate vistieriile în?elepciunii ?i ale cuno?tin?ei.
\par 4 Va spun aceasta, ca nimeni sa nu va în?ele prin cuvinte amagitoare.
\par 5 Caci de?i cu trupul sunt departe, cu duhul însa sunt împreuna cu voi, bucurându-ma ?i vazând buna voastra rânduiala ?i taria credin?ei voastre în Hristos.
\par 6 Deci, precum a?i primit pe Hristos Iisus, Domnul, a?a sa umbla?i întru El.
\par 7 Înradacina?i ?i zidi?i fiind într-Însul, întari?i în credin?a, dupa cum a?i fost înva?a?i, ?i prisosind în ea cu mul?umire.
\par 8 Lua?i aminte sa nu va fure min?ile cineva cu filozofia ?i cu de?arta în?elaciune din predania omeneasca, dupa în?elesurile cele slabe ale lumii ?i nu dupa Hristos.
\par 9 Caci întru El locuie?te, trupe?te, toata plinatatea Dumnezeirii,
\par 10 ?i sunte?i deplini întru El, Care este cap a toata domnia ?i stapânirea.
\par 11 În El a?i ?i fost taia?i împrejur, cu taiere împrejur nefacuta de mâna, prin dezbracarea de trupul carnii, întru taierea împrejur a lui Hristos.
\par 12 Îngropa?i fiind împreuna cu El prin botez, cu El a?i ?i înviat prin credin?a în lucrarea lui Dumnezeu, Cel ce L-a înviat pe El din mor?i.
\par 13 Iar pe voi care era?i mor?i, în faradelegile ?i în netaierea împrejur a trupului vostru, v-a facut vii, împreuna cu Sine, iertându-ne toate gre?ealele;
\par 14 ?tergând zapisul ce era asupra noastra, care ne era potrivnic cu rânduielile lui, ?i l-a luat din mijloc, pironindu-l pe cruce.
\par 15 Dezbracând (de putere) începatoriile ?i stapâniile, le-a dat de ocara în vazul tuturor, biruind asupra lor prin cruce.
\par 16 Nimeni deci sa nu va judece pentru mâncare sau bautura, sau cu privire la vreo sarbatoare, sau luna noua, sau la sâmbete,
\par 17 Care sunt umbra celor viitoare iar trupul (este) al lui Hristos.
\par 18 Nimeni sa nu va smulga biruin?a printr-o prefacuta smerenie ?i printr-o fa?arnica închinare la îngeri, încercând sa patrunda în cele ce n-a vazut, ?i îngâmfându-se zadarnic cu închipuirea lui trupeasca,
\par 19 În loc sa se ?ina strâns de capul de la care trupul tot, - prin încheieturi ?i legaturi, îndestulându-se ?i întocmindu-se S, spore?te în cre?terea lui Dumnezeu.
\par 20 Daca deci a?i murit împreuna cu Hristos pentru în?elesurile cele slabe ale lumii, pentru ce atunci, ca ?i cum a?i vie?ui în lume, rabda?i porunci ca acestea:
\par 21 Nu lua, nu gusta, nu te atinge!
\par 22 - Toate lucruri menite sa piara prin întrebuin?are - potrivit unor rânduieli ?i înva?aturi omene?ti?
\par 23 Unele ca acestea au oarecare înfa?i?are de în?elepciune, în paruta lor cucernicie, în smerenie ?i în necru?area trupului, dar n-au nici un pre? ?i sunt numai pentru sa?iul trupului.

\chapter{3}

\par 1 A?adar, daca a?i înviat împreuna cu Hristos, cauta?i cele de sus, unde se afla Hristos, ?ezând de-a dreapta lui Dumnezeu;
\par 2 Cugeta?i cele de sus, nu cele de pe pamânt;
\par 3 Caci voi a?i murit ?i via?a voastra este ascunsa cu Hristos întru Dumnezeu.
\par 4 Iar când Hristos, Care este via?a voastra, Se va arata, atunci ?i voi, împreuna cu El, va ve?i arata întru slava.
\par 5 Drept aceea, omorâ?i madularele voastre, cele pamânte?ti: desfrânarea, necura?ia, patima, pofta rea ?i lacomia, care este închinare la idoli,
\par 6 Pentru care vine mânia lui Dumnezeu peste fiii neascultarii,
\par 7 În care pacate a?i umblat ?i voi odinioara, pe când traia?i în ele.
\par 8 Acum deci va lepada?i ?i voi de toate acestea: mânia, iu?imea, rautatea, defaimarea, cuvântul de ru?ine din gura voastra.
\par 9 Nu va min?i?i unul pe altul, fiindca v-a?i dezbracat de omul cel vechi, dimpreuna cu faptele lui,
\par 10 ?i v-a?i îmbracat cu cel nou, care se înnoie?te, spre deplina cuno?tin?a, dupa chipul Celui ce l-a zidit,
\par 11 Unde nu mai este elin ?i iudeu, taiere împrejur ?i netaiere împrejur, barbar, scit, rob ori liber, ci toate ?i întru to?i Hristos.
\par 12 Îmbraca?i-va, dar, ca ale?i ai lui Dumnezeu, sfin?i ?i prea iubi?i, cu milostivirile îndurarii, cu bunatate, cu smerenie, cu blânde?e, cu îndelunga-rabdare,
\par 13 Îngaduindu-va unii pe al?ii ?i iertând unii altora, daca are cineva vreo plângere împotriva cuiva; dupa cum ?i Hristos v-a iertat voua, a?a sa ierta?i ?i voi.
\par 14 Iar peste toate acestea, îmbraca?i-va întru dragoste, care este legatura desavâr?irii.
\par 15 ?i pacea lui Hristos, întru care a?i fost chema?i, ca sa fi?i un singur trup, sa stapâneasca în inimile voastre; ?i fi?i mul?umitori.
\par 16 Cuvântul lui Hristos sa locuiasca întru voi cu boga?ie. Înva?a?i-va ?i pova?ui?i-va între voi, cu toata în?elepciunea. Cânta?i în inimile voastre lui Dumnezeu, mul?umindu-I, în psalmi, în laude ?i în cântari duhovnice?ti.
\par 17 Orice a?i face, cu cuvântul sau cu lucrul, toate sa le face?i în numele Domnului Iisus ?i prin El sa mul?umi?i lui Dumnezeu-Tatal.
\par 18 Femeilor, supune?i-va barba?ilor vo?tri, precum se cuvine, în Domnul.
\par 19 Barba?ilor, iubi?i pe femeile voastre ?i nu fi?i aspri cu ele.
\par 20 Copiilor, asculta?i pe parin?ii vo?tri întru toate, caci aceasta este bine-placut Domnului.
\par 21 Parin?ilor, nu a?â?a?i la mânie pe copiii vo?tri, ca sa nu se deznadajduiasca.
\par 22 Slugilor, asculta?i întru toate pe stapânii vo?tri cei trupe?ti, nu slujind numai când sunt cu ochii pe voi, ca cei ce cauta sa placa oamenilor, ci în cura?ia inimii, temându-va de Domnul.
\par 23 Orice a?i face, lucra?i din toata inima, ca pentru Domnul ?i nu ca pentru oameni,
\par 24 Bine ?tiind ca de la Domnul ve?i primi rasplata mo?tenirii; caci Domnului Hristos sluji?i.
\par 25 Iar cel ce face nedreptate î?i va lua plata nedrepta?ii, întrucât la Dumnezeu nu este partinire.

\chapter{4}

\par 1 Stapânilor, da?i slugilor voastre ce este drept ?i potrivit, ?tiind ca ?i voi ave?i Stapân în ceruri.
\par 2 Starui?i în rugaciune, priveghind în ea cu mul?umire,
\par 3 Rugându-va totodata ?i pentru noi, ca Dumnezeu sa ne deschida u?a cuvântului, spre a vesti taina lui Hristos, pentru care ma ?i gasesc în lan?uri,
\par 4 Ca sa o arat a?a cum se cuvine sa graiesc.
\par 5 Umbla?i cu în?elepciune fa?a de cei ce sunt afara (de Biserica), pre?uind vremea.
\par 6 Vorba voastra sa fie totdeauna placuta, dreasa cu sare, ca sa ?ti?i cum trebuie sa raspunde?i fiecaruia.
\par 7 Toate câte ma privesc pe mine le va face cunoscute Tihic, iubitul frate, credincios slujitor ?i împreuna-rob cu mine în Domnul.
\par 8 L-am trimis pe el la voi tocmai pentru aceasta, ca sa ?ti?i cum ne aflam ?i ca sa mângâie inimile voastre,
\par 9 Împreuna cu Onisim, credinciosul ?i iubitul frate, care este dintre voi; ei va vor aduce la cuno?tin?a toate cele de aici.
\par 10 Va îmbra?i?eaza Aristarh, cel întemni?at împreuna cu mine, ?i Marcu, varul lui Barnaba - în privin?a caruia a?i primit porunci; de va veni la voi primi?i-l -,
\par 11 Asemenea ?i Iisus, cel ce se nume?te Iustus, care sunt din taierea împrejur; numai ace?tia au lucrat împreuna cu mine pentru împara?ia lui Dumnezeu. Ei au fost cei ce mi-au adus mângâiere.
\par 12 Va îmbra?i?eaza Epafras, care este dintre voi, rob al lui Iisus Hristos, pururea luptând pentru voi în rugaciunile sale, ca sa sta?i desavâr?i?i ?i plini de tot ce este voin?a lui Dumnezeu.
\par 13 Caci martor îi sunt ca are multa râvna pentru voi ?i pentru cei din Laodiceea ?i din Ierapole.
\par 14 Va îmbra?i?eaza Luca, doctorul cel iubit, ?i Dima.
\par 15 Îmbra?i?a?i pe fra?ii din Laodiceea ?i pe Nimfas ?i pe Biserica din casa lui.
\par 16 ?i dupa ce scrisoarea aceasta se va citi de catre voi, face?i sa se citeasca ?i în Biserica laodiceenilor, iar pe cea din Laodiceea sa o citi?i ?i voi.
\par 17 ?i spune?i lui Arhip: Vezi de slujba pe care ai primit-o întru Domnul, ca sa o îndepline?ti.
\par 18 Salutarea cu mâna mea, a lui Pavel. Aduce?i-va aminte de lan?urile mele. Harul fie cu voi! Amin.


\end{document}