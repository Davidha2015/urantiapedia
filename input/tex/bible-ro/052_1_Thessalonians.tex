\begin{document}

\title{1 Tesaloniceni}


\chapter{1}

\par 1 Pavel ?i Silvan ?i Timotei, Bisericii tesalonicenilor în Dumnezeu-Tatal ?i în Domnul Iisus Hristos: Har voua ?i pace de la Dumnezeu, Tatal nostru, ?i de la Domnul Iisus Hristos.
\par 2 Mul?umim lui Dumnezeu totdeauna pentru voi to?i ?i va pomenim în rugaciunile noastre,
\par 3 Aducându-ne aminte neîncetat, înaintea lui Dumnezeu, Tatal nostru, de lucrul credin?ei voastre ?i de osteneala iubirii ?i de staruin?a nadejdii voastre în Domnul nostru Iisus Hristos.
\par 4 Fra?ilor de Dumnezeu iubi?i, noi ?tim cum a?i fost ale?i;
\par 5 Ca Evanghelia noastra n-a fost la voi numai în cuvânt, ci ?i întru putere ?i în Duhul Sfânt ?i în deplina încredin?are, precum bine ?ti?i ce fel am fost între voi, pentru voi.
\par 6 ?i voi v-a?i facut urmatori ai no?tri ?i ai Domnului, primind cuvântul cu bucuria Duhului Sfânt, de?i a?i avut multe necazuri.
\par 7 A?a încât v-a?i facut pilda tuturor celor ce cred din Macedonia ?i din Ahaia,
\par 8 Caci, de la voi, cuvântul Domnului a rasunat nu numai în Macedonia ?i în Ahaia, ci credin?a voastra în Dumnezeu s-a raspândit în tot locul, astfel ca nu e nevoie sa mai spunem noi ceva.
\par 9 Caci ei în?i?i istorisesc despre noi cum am fost primi?i la voi ?i cum v-a?i întors la Dumnezeu, de la idoli, ca sa sluji?i Dumnezeului celui viu ?i adevarat,
\par 10 ?i sa a?tepta?i pe Fiul Sau din ceruri, pe Care L-a înviat din mor?i, pe Iisus, Cel ce ne izbave?te de mânia cea viitoare.

\chapter{2}

\par 1 Caci ?ti?i voi în?iva, fra?ilor, ca venirea noastra la voi n-a fost zadarnica.
\par 2 Ci, dupa ce am suferit ?i am fost, precum ?ti?i, ocarâ?i în Filipi, am îndraznit întru Dumnezeul nostru sa graim în fa?a voastra, cu multa lupta, Evanghelia lui Dumnezeu.
\par 3 Pentru ca îndemnul nostru nu venea din ratacire, nici din gânduri necurate, nici din în?elaciune,
\par 4 Ci, dupa cum am fost socoti?i vrednici de Dumnezeu ca sa ni se încredin?eze Evanghelia, a?a vorbim, nu cautând sa placem oamenilor, ci lui Dumnezeu care ne încearca inimile.
\par 5 Caci niciodata nu ne-am aratat cu cuvinte de lingu?ire, dupa cum ?ti?i, nici cu ascunse porniri de lacomie. Dumnezeu îmi este martor.
\par 6 Nici n-am cautat slava de la oameni, nici de la voi, nici de la al?ii, de?i puteam sa fim cu greutate, ca apostoli ai lui Hristos.
\par 7 Noi însa am fost blânzi în mijlocul vostru, a?a precum o doica îngrije?te pe fiii sai;
\par 8 Astfel, iubindu-va, eram bucuro?i sa va dam nu numai Evanghelia lui Dumnezeu, ci chiar ?i sufletele noastre pentru ca ne-a?i devenit iubi?i.
\par 9 Va aduce?i aminte, fra?ilor, de osteneala ?i de truda noastra; lucrând zi ?i noapte, ca sa nu fim povara nici unuia din voi, a?a v-am propovaduit Evanghelia lui Dumnezeu.
\par 10 Voi sunte?i martori, ?i Dumnezeu de asemenea, cât de sfânt ?i cât de drept ?i fara de prihana ne-am purtat între voi credincio?ii;
\par 11 Ca un parinte pe copiii sai, precum ?ti?i, a?a v-am rugat ?i v-am mângâiat.
\par 12 ?i v-am rugat cu staruin?a sa umbla?i cum se cuvine înaintea lui Dumnezeu, Celui ce va cheama la împara?ia ?i la slava Sa.
\par 13 De aceea ?i noi mul?umim lui Dumnezeu neîncetat, ca luând voi cuvântul ascultarii de Dumnezeu de la noi, nu l-a?i primit ca pe un cuvânt al oamenilor, ci, a?a precum este într-adevar, ca pe un cuvânt al lui Dumnezeu, care ?i lucreaza întru voi cei ce crede?i.
\par 14 Caci voi, fra?ilor, v-a?i facut urmatori ai Bisericilor lui Dumnezeu, care sunt în Iudeea, întru Hristos Iisus, pentru ca a?i suferit ?i voi acelea?i de la cei de un neam cu voi, dupa cum ?i ele de la iudei,
\par 15 Care ?i pe Domnul Iisus L-au omorât ca ?i pe proorocii lor; ?i pe noi ne-au prigonit ?i sunt neplacu?i lui Dumnezeu ?i tuturor oamenilor sunt potrivnici,
\par 16 Fiindca ne opresc sa vorbim neamurilor, ca sa se mântuiasca, spre a se împlini pururea masura pacatelor lor. Dar la urma, i-a ajuns mânia lui Dumnezeu.
\par 17 Iar noi, fra?ilor, fiind despar?i?i de voi, o bucata de vreme, cu ochii nu cu inima, ne-am sârguit cu atât mai mult, cu mare dor, sa vedem fa?a voastra.
\par 18 Pentru aceea, am voit sa venim la voi, îndeosebi eu Pavel - o data ?i înca alta data, - dar ne-a împiedicat satana.
\par 19 Caci care este nadejdea noastra, sau bucuria, sau cununa laudei noastre, daca nu chiar voi, înaintea Domnului nostru Iisus, întru a Lui venire?
\par 20 Caci voi sunte?i slava ?i bucuria noastra.

\chapter{3}

\par 1 De aceea, nemaiputând rabda, noi am hotarât sa ramânem singuri la Atena.
\par 2 ?i am trimis pe Timotei, fratele nostru ?i slujitorul lui Dumnezeu ?i împreuna-lucrator cu noi la Evanghelia lui Hristos, ca sa va întareasca ?i sa va îndemne în credin?a voastra,
\par 3 Ca nimeni sa nu se clatine în aceste necazuri, caci singuri ?ti?i ca spre aceasta suntem pu?i.
\par 4 Caci ?i când eram la voi, v-am spus de mai înainte ca vom avea de suferit necazuri, precum s-a ?i întâmplat ?i ?ti?i prea bine.
\par 5 Pentru aceea ?i eu, fiind nerabdator, am trimis ca sa cunosc credin?a voastra, ca nu cumva sa va fi ispitit ispititorul ?i în zadar sa ne fie osteneala.
\par 6 Acum însa, venind Timotei de la voi la noi ?i dându-ne vestea buna despre credin?a ?i dragostea voastra ?i ca ave?i buna amintire de noi totdeauna, dorind sa ne vede?i; la fel ?i noi pe voi,
\par 7 De aceea, fra?ilor, ne-am sim?it mângâia?i întru voi, prin credin?a voastra, în toata nevoia ?i strâmtorarea noastra.
\par 8 Caci acum noi suntem vii, daca voi sta?i neclinti?i întru Domnul.
\par 9 ?i ce mul?umire, pentru voi, putem sa dam în schimb lui Dumnezeu, pentru toata bucuria cu care ne bucuram pentru voi, înaintea Dumnezeului nostru?
\par 10 Noaptea ?i ziua ne rugam cu prisosin?a, ca sa vedem fa?a voastra ?i sa împlinim lipsurile credin?ei voastre.
\par 11 Dar însu?i Dumnezeu ?i Tatal nostru, ?i Domnul nostru Iisus Hristos sa îndrepteze calea noastra catre voi!
\par 12 Iar pe voi, Domnul sa va înmul?easca ?i sa prisosi?i în dragoste unul catre altul ?i catre to?i, precum ?i noi fa?a de voi,
\par 13 Spre întarirea inimilor voastre, ca sa fi?i fara de prihana întru sfin?enie, înaintea lui Dumnezeu, Tatal nostru, la venirea Domnului nostru Iisus Hristos, cu to?i sfin?ii Sai.

\chapter{4}

\par 1 În sfâr?it, fra?ilor, va rugam ?i va îndemnam în Domnul Iisus, ca a?a cum a?i primit de la noi dreptar cum se cuvine sa umbla?i ?i sa place?i lui Dumnezeu - în care chip ?i umbla?i - a?a sa spori?i tot mai mult.
\par 2 Fiindca ?ti?i ce porunci v-am dat, prin Domnul Iisus.
\par 3 Caci voia lui Dumnezeu aceasta este: sfin?irea voastra; sa va feri?i de desfrânare,
\par 4 Ca sa ?tie fiecare dintre voi sa-?i stapâneasca vasul sau în sfin?enie ?i cinste,
\par 5 Nu în patima poftei cum fac neamurile, care nu cunosc pe Dumnezeu.
\par 6 ?i nimeni sa nu întreaca masura ?i sa nu nedrepta?easca pe fratele sau, în aceasta privin?a, caci Domnul este razbunator pentru toate acestea, dupa cum v-am ?i spus mai înainte ?i v-am dat marturie.
\par 7 Caci Dumnezeu nu ne-a chemat la necura?ie, ci la sfin?ire.
\par 8 De aceea, cel ce dispre?uie?te (acestea), nu dispre?uie?te un om, ci pe Dumnezeu, Care v-a dat pe Duhul Sau cel Sfânt.
\par 9 Despre iubirea fra?easca nu ave?i trebuin?a sa va scriu, pentru ca voi în?iva sunte?i înva?a?i de Dumnezeu ca sa va iubi?i unul pe altul.
\par 10 Aceasta o ?i face?i, fa?a de to?i fra?ii, din întreaga Macedonie. Dar va îndemnam, fra?ilor, sa prisosi?i mai mult!
\par 11 ?i sa râvni?i ca sa trai?i în lini?te, sa face?i fiecare cele ale sale ?i sa lucra?i cu mâinile voastre precum v-am dat porunca,
\par 12 Ca sa umbla?i cuviincios fa?a de cei din afara (de Biserica) ?i sa nu ave?i trebuin?a de nimeni.
\par 13 Fra?ilor, despre cei ce au adormit, nu voim sa fi?i în ne?tiin?a, ca sa nu va întrista?i, ca ceilal?i, care nu au nadejde,
\par 14 Pentru ca de credem ca Iisus a murit ?i a înviat, tot a?a (credem) ca Dumnezeu, pe cei adormi?i întru Iisus, îi va aduce împreuna cu El.
\par 15 Caci aceasta va spunem, dupa cuvântul Domnului, ca noi cei vii, care vom fi ramas pâna la venirea Domnului, nu vom lua înainte celor adormi?i,
\par 16 Pentru ca Însu?i Domnul, întru porunca, la glasul arhanghelului ?i întru trâmbi?a lui Dumnezeu, Se va pogorî din cer, ?i cei mor?i întru Hristos vor învia întâi,
\par 17 Dupa aceea, noi cei vii, care vom fi ramas, vom fi rapi?i, împreuna cu ei, în nori, ca sa întâmpinam pe Domnul în vazduh, ?i a?a pururea vom fi cu Domnul.
\par 18 De aceea, mângâia?i-va unii pe al?ii cu aceste cuvinte.

\chapter{5}

\par 1 Iar despre ani ?i despre vremuri, fra?ilor, nu ave?i nevoie sa va scriem,
\par 2 Caci voi în?iva ?ti?i bine ca ziua Domnului vine a?a, ca un fur noaptea.
\par 3 Atunci când vor zice: pace ?i lini?te, atunci, fara de veste, va veni peste ei pieirea, ca ?i durerile peste cea însarcinata, ?i scapare nu vor avea.
\par 4 Voi însa, fra?ilor, nu sunte?i în întuneric, ca sa va apuce ziua aceea ca un fur.
\par 5 Caci voi to?i sunte?i fii ai luminii ?i fii ai zilei; nu suntem ai nop?ii, nici ai întunericului.
\par 6 De aceea sa nu dormim ca ?i ceilal?i, ci sa priveghem ?i sa fim treji.
\par 7 Fiindca cei ce dorm, noaptea dorm; ?i cei ce se îmbata, noaptea se îmbata.
\par 8 Dar noi, fiind ai zilei, sa fim treji, îmbracându-ne în plato?a credin?ei ?i a dragostei ?i punând coiful nadejdii de mântuire;
\par 9 Ca Dumnezeu nu ne-a rânduit spre mânie, ci spre dobândirea mântuirii, prin Domnul nostru Iisus Hristos,
\par 10 Care a murit pentru noi, pentru ca noi, fie ca veghem, fie ca dormim, cu El împreuna sa vie?uim.
\par 11 De aceea, îndemna?i-va ?i zidi?i-va unul pe altul, a?a precum ?i face?i.
\par 12 Va mai rugam, fra?ilor, sa cinsti?i pe cei ce se ostenesc între voi, care sunt mai-marii vo?tri în Domnul ?i va pova?uiesc;
\par 13 ?i pentru lucrarea lor, sa-i socoti?i pe ei vrednici de dragoste prisositoare. Trai?i între voi în buna pace.
\par 14 Va rugam însa, fra?ilor, dojeni?i pe cei fara de rânduiala, îmbarbata?i pe cei slabi la suflet, sprijini?i pe cei neputincio?i, fi?i îndelung-rabdatori fa?a de to?i.
\par 15 Lua?i seama sa nu rasplateasca cineva cuiva raul cu rau, ci totdeauna sa urma?i cele bune unul fa?a de altul ?i fa?a de to?i.
\par 16 Bucura?i-va pururea.
\par 17 Ruga?i-va neîncetat.
\par 18 Da?i mul?umire pentru toate, caci aceasta este voia lui Dumnezeu, întru Hristos Iisus, pentru voi.
\par 19 Duhul sa nu-l stinge?i.
\par 20 Proorociile sa nu le dispre?ui?i.
\par 21 Toate sa le încerca?i; ?ine?i ce este bine;
\par 22 Feri?i-va de orice înfa?i?are a raului.
\par 23 Însu?i Dumnezeul pacii sa va sfin?easca pe voi desavâr?it, ?i întreg duhul vostru, ?i sufletul, ?i trupul sa se pazeasca, fara de prihana, întru venirea Domnului nostru Iisus Hristos.
\par 24 Credincios este Cel care va cheama. El va ?i îndeplini.
\par 25 Fra?ilor, ruga?i-va pentru noi.
\par 26 Îmbra?i?a?i pe to?i fra?ii cu sarutare sfânta.
\par 27 Va îndemn staruitor pe voi întru Domnul, ca sa citi?i scrisoarea aceasta tuturor sfin?ilor fra?i.
\par 28 Harul Domnului nostru Iisus Hristos sa fie cu voi. Amin!


\end{document}