\begin{document}

\title{2 Tesaloniceni}


\chapter{1}

\par 1 Pavel, Silvan ?i Timotei, Bisericii tesalonicenilor întru Dumnezeu, Tatal nostru, ?i întru Domnul Iisus Hristos:
\par 2 Har voua ?i pace, de la Dumnezeu, Tatal nostru, ?i de la Domnul Iisus Hristos.
\par 3 Datori suntem, fra?ilor, sa mul?umim pentru voi pururea lui Dumnezeu, precum se cuvine, fiindca mult cre?te credin?a voastra ?i dragostea fiecaruia dintre voi to?i prisose?te, a unuia fa?a de altul,
\par 4 Încât noi în?ine ne laudam cu voi, în Bisericile lui Dumnezeu, pentru statornicia ?i credin?a voastra, în toate prigonirile voastre ?i în strâmtorarile pe care le suferi?i.
\par 5 Ele sunt o dovada a dreptei judeca?i a lui Dumnezeu, ca sa va învrednici?i de împara?ia lui Dumnezeu, pentru care ?i patimi?i,
\par 6 De vreme ce drept este înaintea lui Dumnezeu sa rasplateasca cu necaz celor ce va necajesc pe voi,
\par 7 Iar voua celor necaji?i, sa va dea odihna, împreuna cu noi, la aratarea Domnului Iisus din cer, cu îngerii puterii Sale,
\par 8 În vapaie de foc, osândind pe cei ce nu cunosc pe Dumnezeu ?i pe cei ce nu se supun Evangheliei Domnului nostru Iisus.
\par 9 Ei vor lua ca pedeapsa pieirea ve?nica de la fa?a Domnului ?i de la slava puterii Lui,
\par 10 Când va veni sa se preamareasca întru sfin?ii Sai ?i sa fie privit cu uimire de catre to?i cei ce au crezut, pentru ca marturia noastra catre voi a gasit crezare în ziua aceea.
\par 11 Pentru aceasta, ne ?i rugam pururea pentru voi, ca Dumnezeul nostru sa va faca vrednici de chemarea Sa ?i sa împlineasca cu putere toata pornirea voastra spre bunatate ?i orice lucrare a credin?ei voastre,
\par 12 Ca sa se preaslaveasca în voi numele Domnului nostru Iisus ?i voi întru El, prin harul Dumnezeului nostru ?i al Domnului Iisus Hristos.

\chapter{2}

\par 1 În privin?a venirii Domnului nostru Iisus Hristos ?i a adunarii noastre împreuna cu El, va rugam, fra?ilor,
\par 2 Sa nu va clinti?i degraba cu mintea, nici sa va spaimânta?i - nici de vreun duh, nici de vreun cuvânt, nici de vreo scrisoare ca pornita de la noi, cum ca ziua Domnului a ?i sosit.
\par 3 Sa nu va amageasca nimeni, cu nici un chip; caci ziua Domnului nu va sosi pâna ce mai întâi nu va veni lepadarea de credin?a ?i nu se va da pe fa?a omul nelegiuirii, fiul pierzarii,
\par 4 Potrivnicul, care se înal?a mai presus de tot ce se nume?te Dumnezeu, sau se cinste?te cu închinare, a?a încât sa se a?eze el în templul lui Dumnezeu, dându-se pe sine drept dumnezeu.
\par 5 Nu va aduce?i aminte ca, pe când eram înca la voi, va spuneam aceste lucruri?
\par 6 ?i acum ?ti?i ce-l opre?te, ca sa nu se arate decât la vremea lui.
\par 7 Pentru ca taina faradelegii se ?i lucreaza, pâna când cel care o împiedica acum va fi dat la o parte.
\par 8 ?i atunci se va arata cel fara de lege, pe care Domnul Iisus îl va ucide cu suflarea gurii Sale ?i-l va nimici cu stralucirea venirii Sale.
\par 9 Iar venirea aceluia va fi prin lucrarea lui satan, înso?ita de tot felul de puteri ?i de semne ?i de minuni mincinoase,
\par 10 ?i de amagiri nelegiuite, pentru fiii pierzarii, fiindca ei n-au primit iubirea adevarului, ca ei sa se mântuiasca.
\par 11 ?i de aceea, Dumnezeu le trimite o lucrare de amagire, ca ei sa creada minciuni,
\par 12 Ca sa fie osândi?i to?i cei ce n-au crezut adevarul, ci le-a placut nedreptatea.
\par 13 Iar noi, fra?ilor, iubi?i de Domnul, datori suntem totdeauna sa mul?umim lui Dumnezeu pentru voi, ca v-a ales Dumnezeu dintru început, spre mântuire, întru sfin?irea duhului ?i întru credin?a adevarului,
\par 14 La care v-a chemat prin Evanghelia noastra, spre dobândirea slavei Domnului nostru Iisus Hristos.
\par 15 Deci, dar, fra?ilor, sta?i neclinti?i ?i ?ine?i predaniile pe care le-a?i înva?at, fie prin cuvânt, fie prin epistola noastra.
\par 16 Însu?i Domnul nostru Iisus Hristos ?i Dumnezeu Tatal nostru, Care ne-a iubit pe noi ?i ne-a dat, prin har, ve?nica mângâiere ?i buna nadejde,
\par 17 Sa mângâie inimile voastre ?i sa va întareasca, la tot lucrul ?i cuvântul bun.

\chapter{3}

\par 1 În sfâr?it, fra?ilor, ruga?i-va pentru noi, ca cuvântul Domnului sa se raspândeasca ?i sa se preamareasca, ca ?i la voi,
\par 2 ?i ca sa ne izbavim de oamenii cei nesocoti?i ?i vicleni. Caci credin?a nu este a tuturor;
\par 3 Dar credincios este Domnul, Care va va întari ?i va va pazi de cel viclean.
\par 4 Despre voi, încredin?a?i suntem în Domnul, ca cele ce va poruncim, voi le face?i ?i le ve?i face.
\par 5 Iar Domnul sa îndrepteze inimile voastre spre dragostea lui Dumnezeu ?i spre rabdarea lui Hristos!
\par 6 Fra?ilor, va poruncim în numele Domnului nostru Iisus Hristos, sa va feri?i de orice frate care umbla fara rânduiala ?i nu dupa predania pe care a?i primit-o de la noi.
\par 7 Caci voi în?iva ?ti?i cum trebuie sa va asemana?i noua, ca noi n-am umblat fara rânduiala între voi,
\par 8 Nici n-am mâncat de la cineva pâine în dar, ci, cu munca ?i cu truda, am lucrat noaptea ?i ziua, ca sa nu împovaram pe nimeni dintre voi.
\par 9 Nu doar ca n-avem putere, ci ca sa ne dam pe noi în?ine pilda voua, spre a ne urma.
\par 10 Caci ?i când ne aflam la voi, v-am dat porunca aceasta: daca cineva nu vrea sa lucreze, acela nici sa nu manânce.
\par 11 Pentru ca auzim ca unii de la voi umbla fara rânduiala, nelucrând nimic, ci iscodind.
\par 12 Dar unora ca ace?tia le poruncim ?i-i rugam, în Domnul Iisus Hristos, ca sa munceasca în lini?te ?i sa-?i manânce pâinea lor.
\par 13 Iar voi, fra?ilor, nu pregeta?i sa face?i ce e bine.
\par 14 ?i daca vreunul nu asculta de cuvântul nostru prin epistola, pe acela sa-l însemna?i ?i sa nu mai ave?i cu el nici un amestec, ca sa-i fie ru?ine.
\par 15 Dar sa nu-l socoti?i ca pe un vrajma?, ci pova?ui?i-l ca pe un frate.
\par 16 ?i Însu?i Domnul pacii sa va daruiasca voua pace totdeauna ?i în tot chipul! Domnul fie cu voi cu to?i!
\par 17 Salutarea cu mâna mea - a lui Pavel; acesta este semnul meu în orice scrisoare. A?a scriu.
\par 18 Harul Domnului nostru Iisus Hristos cu voi cu to?i! Amin.


\end{document}