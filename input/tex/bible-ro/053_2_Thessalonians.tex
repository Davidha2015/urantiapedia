\begin{document}

\title{2 Tesaloniceni}


\chapter{1}

\par 1 Pavel, Silvan și Timotei, Bisericii tesalonicenilor întru Dumnezeu, Tatăl nostru, și întru Domnul Iisus Hristos:
\par 2 Har vouă și pace, de la Dumnezeu, Tatăl nostru, și de la Domnul Iisus Hristos.
\par 3 Datori suntem, fraților, să mulțumim pentru voi pururea lui Dumnezeu, precum se cuvine, fiindcă mult crește credința voastră și dragostea fiecăruia dintre voi toți prisosește, a unuia față de altul,
\par 4 Încât noi înșine ne lăudăm cu voi, în Bisericile lui Dumnezeu, pentru statornicia și credința voastră, în toate prigonirile voastre și în strâmtorările pe care le suferiți.
\par 5 Ele sunt o dovadă a dreptei judecăți a lui Dumnezeu, ca să vă învredniciți de împărăția lui Dumnezeu, pentru care și pătimiți,
\par 6 De vreme ce drept este înaintea lui Dumnezeu să răsplătească cu necaz celor ce vă necăjesc pe voi,
\par 7 Iar vouă celor necăjiți, să vă dea odihnă, împreună cu noi, la arătarea Domnului Iisus din cer, cu îngerii puterii Sale,
\par 8 În văpaie de foc, osândind pe cei ce nu cunosc pe Dumnezeu și pe cei ce nu se supun Evangheliei Domnului nostru Iisus.
\par 9 Ei vor lua ca pedeapsă pieirea veșnică de la fața Domnului și de la slava puterii Lui,
\par 10 Când va veni să se preamărească întru sfinții Săi și să fie privit cu uimire de către toți cei ce au crezut, pentru că mărturia noastră către voi a găsit crezare în ziua aceea.
\par 11 Pentru aceasta, ne și rugăm pururea pentru voi, ca Dumnezeul nostru să vă facă vrednici de chemarea Sa și să împlinească cu putere toată pornirea voastră spre bunătate și orice lucrare a credinței voastre,
\par 12 Ca să se preaslăvească în voi numele Domnului nostru Iisus și voi întru El, prin harul Dumnezeului nostru și al Domnului Iisus Hristos.

\chapter{2}

\par 1 În privința venirii Domnului nostru Iisus Hristos și a adunării noastre împreună cu El, vă rugăm, fraților,
\par 2 Să nu vă clintiți degrabă cu mintea, nici să vă spăimântați - nici de vreun duh, nici de vreun cuvânt, nici de vreo scrisoare ca pornită de la noi, cum că ziua Domnului a și sosit.
\par 3 Să nu vă amăgească nimeni, cu nici un chip; căci ziua Domnului nu va sosi până ce mai întâi nu va veni lepădarea de credință și nu se va da pe față omul nelegiuirii, fiul pierzării,
\par 4 Potrivnicul, care se înalță mai presus de tot ce se numește Dumnezeu, sau se cinstește cu închinare, așa încât să se așeze el în templul lui Dumnezeu, dându-se pe sine drept dumnezeu.
\par 5 Nu vă aduceți aminte că, pe când eram încă la voi, vă spuneam aceste lucruri?
\par 6 Și acum știți ce-l oprește, ca să nu se arate decât la vremea lui.
\par 7 Pentru că taina fărădelegii se și lucrează, până când cel care o împiedică acum va fi dat la o parte.
\par 8 Și atunci se va arăta cel fără de lege, pe care Domnul Iisus îl va ucide cu suflarea gurii Sale și-l va nimici cu strălucirea venirii Sale.
\par 9 Iar venirea aceluia va fi prin lucrarea lui satan, însoțită de tot felul de puteri și de semne și de minuni mincinoase,
\par 10 Și de amăgiri nelegiuite, pentru fiii pierzării, fiindcă ei n-au primit iubirea adevărului, ca ei să se mântuiască.
\par 11 Și de aceea, Dumnezeu le trimite o lucrare de amăgire, ca ei să creadă minciuni,
\par 12 Ca să fie osândiți toți cei ce n-au crezut adevărul, ci le-a plăcut nedreptatea.
\par 13 Iar noi, fraților, iubiți de Domnul, datori suntem totdeauna să mulțumim lui Dumnezeu pentru voi, că v-a ales Dumnezeu dintru început, spre mântuire, întru sfințirea duhului și întru credința adevărului,
\par 14 La care v-a chemat prin Evanghelia noastră, spre dobândirea slavei Domnului nostru Iisus Hristos.
\par 15 Deci, dar, fraților, stați neclintiți și țineți predaniile pe care le-ați învățat, fie prin cuvânt, fie prin epistola noastră.
\par 16 Însuși Domnul nostru Iisus Hristos și Dumnezeu Tatăl nostru, Care ne-a iubit pe noi și ne-a dat, prin har, veșnică mângâiere și bună nădejde,
\par 17 Să mângâie inimile voastre și să vă întărească, la tot lucrul și cuvântul bun.

\chapter{3}

\par 1 În sfârșit, fraților, rugați-vă pentru noi, ca cuvântul Domnului să se răspândească și să se preamărească, ca și la voi,
\par 2 Și ca să ne izbăvim de oamenii cei nesocotiți și vicleni. Căci credința nu este a tuturor;
\par 3 Dar credincios este Domnul, Care vă va întări și vă va păzi de cel viclean.
\par 4 Despre voi, încredințați suntem în Domnul, că cele ce vă poruncim, voi le faceți și le veți face.
\par 5 Iar Domnul să îndrepteze inimile voastre spre dragostea lui Dumnezeu și spre răbdarea lui Hristos!
\par 6 Fraților, vă poruncim în numele Domnului nostru Iisus Hristos, să vă feriți de orice frate care umblă fără rânduială și nu după predania pe care ați primit-o de la noi.
\par 7 Căci voi înșivă știți cum trebuie să vă asemănați nouă, că noi n-am umblat fără rânduială între voi,
\par 8 Nici n-am mâncat de la cineva pâine în dar, ci, cu muncă și cu trudă, am lucrat noaptea și ziua, ca să nu împovărăm pe nimeni dintre voi.
\par 9 Nu doar că n-avem putere, ci ca să ne dăm pe noi înșine pildă vouă, spre a ne urma.
\par 10 Căci și când ne aflam la voi, v-am dat porunca aceasta: dacă cineva nu vrea să lucreze, acela nici să nu mănânce.
\par 11 Pentru că auzim că unii de la voi umblă fără rânduială, nelucrând nimic, ci iscodind.
\par 12 Dar unora ca aceștia le poruncim și-i rugăm, în Domnul Iisus Hristos, ca să muncească în liniște și să-și mănânce pâinea lor.
\par 13 Iar voi, fraților, nu pregetați să faceți ce e bine.
\par 14 Și dacă vreunul nu ascultă de cuvântul nostru prin epistolă, pe acela să-l însemnați și să nu mai aveți cu el nici un amestec, ca să-i fie rușine.
\par 15 Dar să nu-l socotiți ca pe un vrăjmaș, ci povățuiți-l ca pe un frate.
\par 16 Și Însuși Domnul păcii să vă dăruiască vouă pace totdeauna și în tot chipul! Domnul fie cu voi cu toți!
\par 17 Salutarea cu mâna mea - a lui Pavel; acesta este semnul meu în orice scrisoare. Așa scriu.
\par 18 Harul Domnului nostru Iisus Hristos cu voi cu toți! Amin.


\end{document}