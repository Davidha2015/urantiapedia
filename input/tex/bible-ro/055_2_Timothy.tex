\begin{document}

\title{2 Timothy}

2Ti 1:1  Pavel, apostolul lui Hristos Iisus, prin voia lui Dumnezeu, dupa fagaduin?a vie?ii care este în Hristos Iisus,
2Ti 1:2  Lui Timotei, iubitului fiu: Har, mila, pace de la Dumnezeu-Tatal ?i de la Hristos Iisus, Domnul nostru!
2Ti 1:3  Mul?umesc lui Dumnezeu, Caruia Îi slujesc din stramo?i, cu cuget curat, ca te pomenesc neîncetat, zi ?i noapte, în rugaciunile mele.
2Ti 1:4  ?i pentru ca îmi aduc aminte de lacrimile tale, doresc mult sa te vad, ca sa ma umplu de bucurie;
2Ti 1:5  Îmi aduc iara?i aminte de credin?a ta neprefacuta, care, precum s-a sala?luit întâi în bunica ta Loida ?i în mama Eunichi, tot a?a, sunt încredin?at, ca ?i întru tine.
2Ti 1:6  Din aceasta pricina, î?i amintesc sa aprinzi ?i mai mult din nou harul lui Dumnezeu, care este în tine, prin punerea mâinilor mele.
2Ti 1:7  Caci Dumnezeu nu ne-a dat duhul temerii, ci al puterii ?i al dragostei ?i al în?elepciunii.
2Ti 1:8  Deci, nu te ru?ina de a marturisi pe Domnul nostru, nici de mine, cel pus în lan?uri pentru El, ci patime?te împreuna cu mine pentru Evanghelie dupa puterea de la Dumnezeu.
2Ti 1:9  El ne-a mântuit ?i ne-a chemat cu chemare sfânta, nu dupa faptele noastre, ci dupa a Sa hotarâre ?i dupa harul ce ne-a fost dat în Hristos Iisus, mai înainte de începutul veacurilor,
2Ti 1:10  Iar acum s-a dat pe fa?a prin aratarea Mântuitorului nostru Iisus Hristos, Cel ce a nimicit moartea ?i a adus la lumina via?a ?i nemurirea, prin Evanghelie.
2Ti 1:11  Spre aceasta am fost pus eu propovaduitor ?i apostol ?i înva?ator al neamurilor.
2Ti 1:12  Din aceasta pricina ?i sufar toate acestea, dar nu ma ru?inez, ca ?tiu în cine am crezut ?i sunt încredin?at ca puternic este sa pazeasca comoara ce mi-a încredin?at, pâna în ziua aceea.
2Ti 1:13  ?ine dreptarul cuvintelor sanatoase pe care le-ai auzit de la mine, cu credin?a ?i cu iubirea ce este în Hristos Iisus.
2Ti 1:14  Comoara cea buna ce ?i s-a încredin?at, paze?te-o cu ajutorul Sfântului Duh, Care sala?luie?te întru noi.
2Ti 1:15  Tu ?tii ca to?i cei din Asia s-au lepadat de mine, între care Fighel ?i Ermoghen.
2Ti 1:16  Domnul sa aiba mila de casa lui Onisifor, caci de multe ori m-a însufle?it ?i de lan?urile mele nu s-a ru?inat,
2Ti 1:17  Ci venind în Roma, cu multa osârdie m-a cautat ?i m-a gasit.
2Ti 1:18  Sa-i dea Domnul ca, în ziua aceea, el sa afle mila de la Domnul. ?i cât de mult mi-a slujit el în Efes, tu ?tii prea bine.
2Ti 2:1  Tu, deci, fiul meu, întare?te-te în harul care e în Hristos Iisus,
2Ti 2:2  ?i cele ce ai auzit de la mine, cu mul?i martori de fa?a, acestea le încredin?eaza la oameni credincio?i, care vor fi destoinici sa înve?e ?i pe al?ii.
2Ti 2:3  Sufera împreuna cu mine, ca un bun osta? al lui Hristos Iisus.
2Ti 2:4  Nici un osta? nu se încurca cu treburile vie?ii, ca sa fie pe plac celui care strânge oaste.
2Ti 2:5  Iar când se lupta cineva, la jocuri, nu ia cununa, daca nu s-a luptat dupa regulile jocului.
2Ti 2:6  Cuvine-se ca plugarul ce se ostene?te sa manânce el mai întâi din roade.
2Ti 2:7  În?elege cele ce-?i graiesc, caci Domnul î?i va da pricepere în toate.
2Ti 2:8  Adu-?i aminte de Iisus Hristos, Care a înviat din mor?i, din neamul lui David, dupa Evanghelia mea,
2Ti 2:9  Pentru Care sufar pâna ?i lan?uri ca un facator de rele, dar cuvântul lui Dumnezeu nu se leaga.
2Ti 2:10  De aceea toate le rabd, pentru cei ale?i, ca ?i ei sa aiba parte de mântuirea care este întru Hristos Iisus ?i de slava ve?nica.
2Ti 2:11  Vrednic de crezare este cuvântul: caci daca am murit împreuna cu El, vom ?i învia împreuna cu El.
2Ti 2:12  Daca ramânem întru El, vom ?i împara?i împreuna cu El; de-L vom tagadui, ?i El ne va tagadui pe noi.
2Ti 2:13  Daca nu-I suntem credincio?i, El ramâne credincios, caci nu poate sa Se tagaduiasca pe Sine însu?i.
2Ti 2:14  Aminte?te-le acestea ?i îndeamna staruitor înaintea lui Dumnezeu sa nu se certe pe cuvinte, ceea ce la nimic nu folose?te, decât la pierzarea ascultatorilor.
2Ti 2:15  Sile?te-te sa te ara?i încercat, înaintea lui Dumnezeu lucrator cu fa?a curata, drept înva?ând cuvântul adevarului.
2Ti 2:16  Iar de de?artele vorbiri lume?ti fere?te-te, caci ele vor spori nelegiuirea tot mai mult.
2Ti 2:17  Cuvântul lor va roade ca o cangrena. Dintre ei sunt Imeneu ?i Filet,
2Ti 2:18  Care au ratacit de la adevar, zicând ca învierea s-a ?i petrecut, ?i rastoarna credin?a unora.
2Ti 2:19  Dar temelia cea tare a lui Dumnezeu sta neclintita, având pecetea aceasta: "Cunoscut-a Domnul pe cei ce sunt ai Sai"; ?i "sa se departeze de la nedreptate oricine cheama numele Domnului".
2Ti 2:20  Iar într-o casa mare nu sunt numai vase de aur ?i de argint, ci ?i de lemn ?i de lut; ?i unele sunt spre cinste, iar altele spre necinste.
2Ti 2:21  Deci, de se va cura?i cineva pe sine de acestea, va fi vas de cinste, sfin?it, de buna trebuin?a stapânului, potrivit pentru tot lucrul bun.
2Ti 2:22  Fugi de poftele tinere?ilor ?i urmeaza dreptatea, credin?a, dragostea, pacea cu cei ce cheama pe Domnul din inima curata.
2Ti 2:23  Fere?te-te de întrebarile nebune?ti, ?tiind ca dau prilej de cearta.
2Ti 2:24  Un slujitor al Domnului nu trebuie sa se certe, ci sa fie blând fa?a cu to?i, destoinic sa dea înva?atura, îngaduitor,
2Ti 2:25  Certând cu blânde?e pe cei ce stau împotriva, ca doar le va da Dumnezeu pocain?a spre cunoa?terea adevarului,
2Ti 2:26  ?i ei sa scape din cursa diavolului, de care sunt prin?i, pentru a-i face voia.
2Ti 3:1  ?i aceasta sa ?tii ca, în zilele din urma, vor veni vremuri grele;
2Ti 3:2  Ca vor fi oameni iubitori de sine, iubitori de argin?i, laudaro?i, trufa?i, hulitori, neascultatori de parin?i, nemul?umitori, fara cucernicie,
2Ti 3:3  Lipsi?i de dragoste, neîndupleca?i, clevetitori, neînfrâna?i, cruzi, neiubitori de bine,
2Ti 3:4  Tradatori, necuviincio?i, îngâmfa?i, iubitori de desfatari mai mult decât iubitori de Dumnezeu,
2Ti 3:5  Având înfa?i?area adevaratei credin?e, dar tagaduind puterea ei. Departeaza-te ?i de ace?tia.
2Ti 3:6  Caci dintre ace?tia sunt cei ce se vâra prin case ?i robesc femeiu?ti împovarate de pacate ?i purtate de multe feluri de pofte,
2Ti 3:7  Mereu înva?ând ?i neputând niciodata sa ajunga la cunoa?terea adevarului.
2Ti 3:8  Dupa cum Iannes ?i Iambres s-au împotrivit lui Moise, a?a ?i ace?tia stau împotriva adevarului, oameni strica?i la minte ?i netrebnici pentru credin?a.
2Ti 3:9  Dar nu vor merge mai departe, pentru ca nebunia lor va fi vadita tuturor, precum a fost ?i a acelora.
2Ti 3:10  Tu însa mi-ai urmat în înva?atura, în purtare, în nazuin?a, în credin?a, în îndelunga rabdare, în dragoste, în staruin?a,
2Ti 3:11  În prigonirile ?i suferin?ele care mi s-au facut în Antiohia, în Iconiu, în Listra; câte prigoniri am rabdat! ?i din toate m-a izbavit Domnul.
2Ti 3:12  ?i to?i care voiesc sa traiasca cucernic în Hristos Iisus vor fi prigoni?i.
2Ti 3:13  Iar oamenii rai ?i amagitori vor merge spre tot mai rau, ratacind pe al?ii ?i rataci?i fiind ei în?i?i.
2Ti 3:14  Tu însa ramâi în cele ce ai înva?at ?i de care e?ti încredin?at, deoarece ?tii de la cine le-ai înva?at,
2Ti 3:15  ?i fiindca de mic copil cuno?ti Sfintele Scripturi, care pot sa te în?elep?easca spre mântuire, prin credin?a cea întru Hristos Iisus.
2Ti 3:16  Toata Scriptura este insuflata de Dumnezeu ?i de folos spre înva?atura, spre mustrare, spre îndreptare, spre în?elep?irea cea întru dreptate,
2Ti 3:17  Astfel ca omul lui Dumnezeu sa fie desavâr?it, bine pregatit pentru orice lucru bun.
2Ti 4:1  Eu te îndemn deci staruitor în fa?a lui Dumnezeu ?i a lui Hristos Iisus, Care va sa judece viii ?i mor?ii, la aratarea Lui ?i în împara?ia Lui;
2Ti 4:2  Propovaduie?te cuvântul, staruie?te cu timp ?i fara de timp, mustra, cearta, îndeamna, cu toata îndelunga-rabdare ?i înva?atura.
2Ti 4:3  Caci va veni o vreme când nu vor mai suferi înva?atura sanatoasa, ci - dornici sa-?i desfateze auzul - î?i vor gramadi înva?atori dupa poftele lor,
2Ti 4:4  ?i î?i vor întoarce auzul de la adevar ?i se vor abate catre basme.
2Ti 4:5  Tu fii treaz în toate, sufera raul, fa lucru de evanghelist, slujba ta fa-o deplin!
2Ti 4:6  Ca eu de-acum ma jertfesc ?i vremea despar?irii mele s-a apropiat.
2Ti 4:7  Lupta cea buna m-am luptat, calatoria am savâr?it, credin?a am pazit.
2Ti 4:8  De acum mi s-a gatit cununa drepta?ii, pe care Domnul îmi va da-o în ziua aceea, El, Dreptul Judecator, ?i nu numai mie, ci ?i tuturor celor ce au iubit aratarea Lui.
2Ti 4:9  Sile?te-te sa vii curând la mine,
2Ti 4:10  Ca Dimas, iubind veacul de acum, m-a lasat ?i s-a dus la Tesalonic, Crescent în Galatia, Tit în Dalma?ia;
2Ti 4:11  Numai Luca este cu mine. Ia pe Marcu ?i adu-l cu tine, caci îmi este de folos în slujire.
2Ti 4:12  Pe Tihic l-am trimis la Efes.
2Ti 4:13  Când vei veni, adu-mi felonul pe care l-am lasat în Troada, la Carp, precum ?i car?ile, mai ales pergamentele.
2Ti 4:14  Alexandru aramarul mi-a facut multe rele; Domnul sa-i rasplateasca dupa faptele lui.
2Ti 4:15  Paze?te-te ?i tu de el, caci s-a împotrivit foarte mult cuvântarilor noastre.
2Ti 4:16  La întâia mea aparare, nimeni nu mi-a venit într-ajutor, ci to?i m-au parasit. Sa nu li se ?ina în socoteala!
2Ti 4:17  Dar Domnul mi-a stat într-ajutor ?i m-a întarit, pentru ca, prin mine, Evanghelia sa fie pe deplin vestita ?i s-o auda toate neamurile; iar eu am fost izbavit din gura leului.
2Ti 4:18  Domnul ma va izbavi de orice lucru rau ?i ma va mântui, în împara?ia Sa cereasca. Lui fie slava în vecii vecilor. Amin!
2Ti 4:19  Îmbra?i?eaza pe Priscila ?i pe Acvila ?i casa lui Onisifor.
2Ti 4:20  Erast a ramas în Corint; pe Trofim l-am lasat în Milet, fiind bolnav.
2Ti 4:21  Sile?te-te sa vii mai înainte de începutul iernii. Te îmbra?i?eaza Eubul ?i Puden?iu ?i Linos ?i Claudia ?i fra?ii to?i.
2Ti 4:22  Domnul Iisus Hristos sa fie cu duhul tau! Harul fie cu voi! Amin.


\end{document}