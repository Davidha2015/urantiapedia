\begin{document}

\title{Titus}

Tit 1:1  Pavel, robul lui Dumnezeu ?i apostol al lui Iisus Hristos, dupa credin?a ale?ilor lui Dumnezeu ?i dupa cuno?tin?a adevarului cel întocmai cu dreapta credin?a,
Tit 1:2  - Întru nadejdea vie?ii ve?nice, pe care a fagaduit-o mai înainte de anii veacurilor Dumnezeu, Care nu minte,
Tit 1:3  ?i Care, la timpul cuvenit, ?i-a facut cunoscut cuvântul Sau, prin propovaduirea încredin?ata mie, dupa porunca Mântuitorului nostru Dumnezeu -
Tit 1:4  Lui Tit, adevaratul fiu dupa credin?a cea de ob?te: Har, mila ?i pace, de la Dumnezeu-Tatal ?i de la Domnul Iisus Hristos, Mântuitorul nostru.
Tit 1:5  Pentru aceasta te-am lasat în Creta, ca sa îndreptezi cele ce mai lipsesc ?i sa a?ezi preo?i prin ceta?i, precum ?i-am rânduit:
Tit 1:6  De este cineva fara de prihana, barbat al unei femei, având fii credincio?i, nu sub învinuire de desfrânare sau neascultatori.
Tit 1:7  Caci se cuvine ca episcopul sa fie fara de prihana, ca un iconom al lui Dumnezeu, neîngâmfat, nu grabnic la mânie, nu dat la bautura, pa?nic, nepoftitor de câ?tig urât,
Tit 1:8  Ci iubitor de straini, iubitor de bine, în?elept, drept, cuvios, cumpatat,
Tit 1:9  ?inându-se de cuvântul cel credincios al înva?aturii, ca sa fie destoinic ?i sa îndemne la înva?atura cea sanatoasa ?i sa mustre pe cei potrivnici.
Tit 1:10  Pentru ca mul?i sunt razvrati?i, graitori în de?ert ?i în?elatori, mai ales cei din taierea împrejur,
Tit 1:11  Carora trebuie sa li se închida gura ca unora care razvratesc case întregi, înva?ând, pentru câ?tig urât, cele ce nu se cuvin.
Tit 1:12  Unul dintre ei, chiar un prooroc al lor, a rostit: Cretanii sunt pururea mincino?i, fiare rele, pântece lene?e.
Tit 1:13  Marturia aceasta este adevarata; pentru care pricina, mustra-i cu asprime, ca sa fie sanato?i în credin?a,
Tit 1:14  ?i sa nu dea ascultare basmelor iudaice?ti ?i poruncilor unor oameni, care se întorc de la adevar.
Tit 1:15  Toate sunt curate pentru cei cura?i; iar pentru cei întina?i ?i necredincio?i nimeni nu este curat, ci li s-au întinat lor ?i mintea ?i cugetul.
Tit 1:16  Ei marturisesc ca Îl cunosc pe Dumnezeu, dar cu faptele lor Îl tagaduiesc, urâcio?i fiind, nesupu?i, ?i la orice lucru bun, netrebnici.
Tit 2:1  Dar tu graie?te cele ce se cuvin înva?aturii sanatoase.
Tit 2:2  Batrânii sa fie treji, cinsti?i, întregi la minte, sanato?i în credin?a, în dragoste, în rabdare;
Tit 2:3  Batrânele de asemenea sa aiba, în înfa?i?are, sfin?ita cuviin?a, sa fie neclevetitoare, nerobite de vin mult, sa înve?e de bine,
Tit 2:4  Ca sa în?elep?easca pe cele tinere sa-?i iubeasca barba?ii, sa-?i iubeasca copiii,
Tit 2:5  ?i sa fie cumpatate, curate, gospodine, bune, plecate barba?ilor lor, ca sa nu fie defaimat cuvântul lui Dumnezeu.
Tit 2:6  Îndeamna, de asemenea, pe cei tineri sa fie cumpata?i.
Tit 2:7  Întru toate arata-te pe tine pilda de fapte bune, dovedind în înva?atura neschimbare, cuviin?a,
Tit 2:8  Cuvânt sanatos ?i fara prihana, pentru ca cel potrivnic sa se ru?ineze, neavând de zis nimic rau despre noi.
Tit 2:9  Slugile sa se supuna stapânilor lor, întru toate, ca sa fie bine-placute, neîntorcându-le vorba,
Tit 2:10  Sa nu doseasca ceva, ci sa le arate toata buna credin?a, ca sa faca de cinste întru toate înva?atura Mântuitorului nostru Dumnezeu.
Tit 2:11  Caci harul mântuitor al lui Dumnezeu s-a aratat tuturor oamenilor,
Tit 2:12  Înva?ându-ne pe noi sa lepadam faradelegea ?i poftele lume?ti ?i, în veacul de acum, sa traim cu în?elepciune, cu dreptate ?i cu cucernicie;
Tit 2:13  ?i sa a?teptam fericita nadejde ?i aratarea slavei marelui Dumnezeu ?i Mântuitorului nostru Hristos Iisus,
Tit 2:14  Care S-a dat pe Sine pentru noi, ca sa ne izbaveasca de toata faradelegea ?i sa-?i cura?easca Lui popor ales, râvnitor de fapte bune.
Tit 2:15  Acestea graie?te, îndeamna ?i mustra cu toata taria. Nimeni sa nu te dispre?uiasca.
Tit 3:1  Adu-le aminte sa se supuna stapânirilor ?i dregatorilor, sa asculte, sa fie gata la orice lucru bun,
Tit 3:2  Sa nu defaime pe nimeni, sa fie pa?nici, sa fie îngaduitori, aratând întreaga blânde?e fa?a de to?i oamenii.
Tit 3:3  Caci ?i noi eram altadata fara de minte, neascultatori, amagi?i, slujind poftelor ?i multor feluri de desfatari, petrecând via?a în rautate ?i pizmuire, urâ?i fiind ?i urându-ne unul pe altul;
Tit 3:4  Iar când bunatatea ?i iubirea de oameni a Mântuitorului nostru Dumnezeu s-au aratat,
Tit 3:5  El ne-a mântuit, nu din faptele cele întru dreptate, savâr?ite de noi, ci dupa a Lui îndurare, prin baia na?terii celei de a doua ?i prin înnoirea Duhului Sfânt,
Tit 3:6  Pe Care L-a varsat peste noi, din bel?ug, prin Iisus Hristos, Mântuitorul nostru,
Tit 3:7  Ca îndreptându-ne prin harul Lui, sa ne facem, dupa nadejde, mo?tenitorii vie?ii celei ve?nice.
Tit 3:8  Vrednic de crezare este cuvântul, ?i voiesc sa adevere?ti acestea cu tarie, pentru ca acei ce au crezut în Dumnezeu sa aiba grija sa fie în frunte la fapte bune. Ca acestea sunt cele bune ?i de folos oamenilor.
Tit 3:9  Iar de întrebarile nebune?ti ?i de în?irari de neamuri ?i de certuri ?i de sfadirile pentru lege, fere?te-te, caci sunt nefolositoare ?i de?arte.
Tit 3:10  De omul eretic, dupa întâia ?i a doua mustrare, departeaza-te,
Tit 3:11  ?tiind ca unul ca acesta s-a abatut ?i a cazut în pacat, fiind singur de sine osândit.
Tit 3:12  Când voi trimite pe Artemas la tine sau pe Tihic, sârguie?te-te sa vii la mine la Nicopole, caci acolo m-am hotarât sa iernez.
Tit 3:13  Pe Zenas, cunoscatorul de lege, ?i pe Apollo trimite-i mai înainte, cu buna grija, ca nimic sa nu le lipseasca.
Tit 3:14  Sa înve?e ?i ai no?tri sa poarte grija de lucrurile bune, spre treburile cele de neaparata nevoie, ca ei sa nu fie fara de roada.
Tit 3:15  Te îmbra?i?eaza to?i care sunt cu mine. Îmbra?i?eaza pe cei ce ne iubesc întru credin?a. Harul fie cu voi cu to?i! Amin.


\end{document}