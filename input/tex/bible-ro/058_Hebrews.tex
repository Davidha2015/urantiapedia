\begin{document}

\title{Hebrews}

Heb 1:1  Dupa ce Dumnezeu odinioara, în multe rânduri ?i în multe chipuri, a vorbit parin?ilor no?tri prin prooroci,
Heb 1:2  În zilele acestea mai de pe urma ne-a grait noua prin Fiul, pe Care L-a pus mo?tenitor a toate ?i prin Care a facut ?i veacurile;
Heb 1:3  Care, fiind stralucirea slavei ?i chipul fiin?ei Lui ?i Care ?ine toate cu cuvântul puterii Sale, dupa ce a savâr?it, prin El însu?i, cura?irea pacatelor noastre, a ?ezut de-a dreapta slavei, întru cele prea înalte,
Heb 1:4  Facându-Se cu atât mai presus de îngeri, cu cât a mo?tenit un nume mai deosebit decât ei.
Heb 1:5  Caci caruia dintre îngeri i-a zis Dumnezeu vreodata: "Fiul Meu e?ti Tu, Eu astazi Te-am nascut"; ?i iara?i: "Eu Îi voi fi Lui Tata ?i El Îmi va fi Mie Fiu"?
Heb 1:6  ?i iara?i, când aduce în lume pe Cel întâi nascut, El zice: "?i sa se închine Lui to?i îngerii lui Dumnezeu".
Heb 1:7  ?i de îngeri zice: "Cel ce face pe îngerii Sai duhuri ?i pe slujitorii Sai para de foc";
Heb 1:8  Iar catre Fiul: "Tronul Tau, Dumnezeule, în veacul veacului; ?i toiagul drepta?ii este toiagul împara?iei Tale.
Heb 1:9  Iubit-ai dreptatea ?i ai urât faradelegea; pentru aceea Te-a uns pe Tine, Dumnezeule, Dumnezeul Tau cu untdelemnul bucuriei, mai mult decât pe parta?ii Tai".
Heb 1:10  ?i: "Întru început Tu, Doamne, pamântul l-ai întemeiat ?i cerurile sunt lucrul mâinilor Tale;
Heb 1:11  Ele vor pieri, dar Tu ramâi, ?i toate ca o haina se vor învechi;
Heb 1:12  ?i ca un pe un ve?mânt le vei strânge ?i ca o haina vor fi schimbate. Dar Tu acela?i e?ti ?i anii Tai nu se vor sfâr?i".
Heb 1:13  ?i caruia dintre îngeri a zis Dumnezeu vreodata: "?ezi de-a dreapta Mea pâna când voi pune pe vrajma?ii tai a?ternut picioarelor Tale"?
Heb 1:14  Îngerii oare nu sunt to?i duhuri slujitoare, trimise ca sa slujeasca, pentru cei ce vor fi mo?tenitorii mântuirii?
Heb 2:1  Pentru aceea se cuvine ca noi sa luam aminte cu atât mai mult la cele auzite, ca nu cumva sa ne pierdem.
Heb 2:2  Caci, daca s-a adeverit cuvântul grait prin îngeri ?i orice calcare de porunca ?i orice neascultare ?i-a primit dreapta rasplatire,
Heb 2:3  Cum vom scapa noi, daca vom fi nepasatori la astfel de mântuire care, luând obâr?ie din propovaduirea Domnului, ne-a fost adeverita de cei ce au ascultat-o,
Heb 2:4  Împreuna marturisind ?i Dumnezeu cu semne ?i cu minuni ?i cu multe feluri de puteri ?i cu darurile Duhului Sfânt, împar?ite dupa a Sa voin?a?
Heb 2:5  Pentru ca nu îngerilor a supus Dumnezeu lumea viitoare, despre care vorbim.
Heb 2:6  Iar cineva a marturisit undeva, zicând: "Ce este omul, ca-l pomene?ti pe el, sau fiul omului, ca-l cercetezi pe el?
Heb 2:7  L-ai mic?orat pe el cu pu?in fa?a de îngeri; ?i cu marire ?i cu cinste l-ai încununat ?i l-ai pus peste lucrurile mâinilor Tale.
Heb 2:8  Toate le-ai supus sub picioarele lui". Dar prin faptul ca a supus lui toate (în?elegem) ca nimic nu i-a lasat nesupus. Acum însa, înca nu vedem cum toate i-au fost supuse.
Heb 2:9  Ci pe Cel mic?orat cu pu?in fa?a de îngeri, pe Iisus, Îl vedem încununat cu slava ?i cu cinste, din pricina mor?ii pe care a suferit-o, astfel ca, prin harul lui Dumnezeu, El a gustat moartea pentru fiecare om.
Heb 2:10  Caci ducând pe mul?i fii la marire, I se cadea Aceluia, pentru Care sunt toate ?i prin Care sunt toate, ca sa desavâr?easca prin patimire pe Începatorul mântuirii lor.
Heb 2:11  Pentru ca ?i Cel ce sfin?e?te ?i cei ce se sfin?esc, dintr-Unul sunt to?i; de aceea nu se ru?ineaza sa-i numeasca pe ei fra?i,
Heb 2:12  Zicând: "Spune-voi fra?ilor mei numele Tau. În mijlocul Bisericii Te voi lauda".
Heb 2:13  ?i iara?i: "Eu voi fi încrezator în El"; ?i iara?i: "Iata Eu ?i pruncii pe care Mi i-a dat  Dumnezeu".
Heb 2:14  Deci, de vreme ce pruncii s-au facut parta?i sângelui ?i trupului, în acela?i fel ?i El S-a împarta?it de acestea, ca sa surpe prin moartea Sa pe cel ce are stapânirea mor?ii, adica pe diavolul,
Heb 2:15  ?i sa izbaveasca pe acei pe care frica mor?ii îi ?inea în robie toata via?a.
Heb 2:16  Caci, într-adevar, nu a luat firea îngerilor, ci samân?a lui Avraam a luat.
Heb 2:17  Pentru aceea, dator era întru toate sa Se asemene fra?ilor, ca sa fie milostiv ?i credincios arhiereu în cele catre Dumnezeu, pentru cura?irea pacatelor poporului.
Heb 2:18  Caci prin ceea ce a patimit, fiind El însu?i ispitit, poate ?i celor ce se ispitesc sa le ajute.
Heb 3:1  Pentru aceea, fra?i sfin?i, parta?i chemarii cere?ti, lua?i aminte la Apostolul ?i Arhiereul marturisirii noastre, la Iisus Hristos,
Heb 3:2  Care credincios a fost Celui ce L-a rânduit, precum ?i Moise în toata casa Lui.
Heb 3:3  Pentru ca Acesta (Iisus) S-a învrednicit de mai multa slava decât Moise, dupa cum are mai multa cinste decât casa cel ce a zidit-o.
Heb 3:4  Caci orice casa e zidita de catre cineva, iar Ziditorul a toate este Dumnezeu.
Heb 3:5  Moise a fost credincios în toata casa Domnului, ca o sluga, spre marturia celor ce erau sa fie descoperite în viitor,
Heb 3:6  Iar Hristos a fost credincios ca Fiu peste casa Sa. ?i casa Lui suntem noi, numai daca ?inem pâna la sfâr?it cu neclintire, îndrazneala marturisirii ?i lauda nadejdii noastre.
Heb 3:7  De aceea, precum zice Duhul Sfânt: "Daca ve?i auzi astazi glasul Lui,
Heb 3:8  Nu va învârto?a?i inimile voastre, ca la razvratire în ziua ispitirii din pustie,
Heb 3:9  Unde M-au ispitit parin?ii vo?tri, M-au încercat, ?i au vazut faptele Mele, timp de patruzeci de ani.
Heb 3:10  De aceea M-am mâniat pe neamul acesta ?i am zis: Pururea ei ratacesc cu inima ?i caile Mele nu le-au cunoscut,
Heb 3:11  Ca M-am jurat în mânia Mea: "Nu vor intra întru odihna Mea!".
Heb 3:12  Lua?i seama, fra?ilor, sa nu fie cumva, în vreunul din voi, o inima vicleana a necredin?ei, ca sa va departeze de la Dumnezeul cel viu.
Heb 3:13  Ci îndemna?i-va unii pe al?ii, în fiecare zi, pâna ce putem sa zicem: astazi! ca nimeni dintre voi sa nu se învârto?eze cu în?elaciunea pacatului;
Heb 3:14  Caci ne-am facut parta?i ai lui Hristos, numai daca vom pastra temeinic, pâna la urma, începutul starii noastre întru El,
Heb 3:15  De vreme ce se zice: Daca ve?i auzi astazi glasul Lui, nu învârto?a?i inimile voastre, ca la razvratire.
Heb 3:16  Cine sunt cei care, auzind, s-au razvratit? Oare nu to?i care au ie?it din Egipt, prin Moise?
Heb 3:17  ?i împotriva cui a ?inut mâine timp de patruzeci de ani? Au nu împotriva celor ce au pacatuit, ale caror oase au cazut în pustie?
Heb 3:18  ?i cui S-a jurat ca nu vor intra întru odihna Sa, decât numai celor ce au fost neascultatori?
Heb 3:19  Vedem dar ca n-au putut sa intre, din pricina necredin?ei lor.
Heb 4:1  Sa ne temem, deci, ca nu cumva, câta vreme ni se lasa fagaduin?a sa intram în odihna Lui, sa para ca a ramas pe urma cineva dintre voi.
Heb 4:2  Pentru ca ?i noua ni s-a binevestit ca ?i acelora, dar cuvântul propovaduirii nu le-a fost lor de folos, nefiind unit cu credin?a la cei care l-au auzit.
Heb 4:3  Pe când noi, fiindca am crezut, intram în odihna, precum s-a zis: "M-am jurat întru mânia Mea: nu vor intra întru odihna Mea", macar ca lucrurile erau savâr?ite de la întemeierea lumii.
Heb 4:4  Caci undeva, despre ziua a ?aptea, a zis astfel: "?i S-a odihnit Dumnezeu în ziua a ?aptea de toate lucrurile Sale".
Heb 4:5  ?i în acela?i loc, zice iara?i: "Nu vor intra întru odihna Mea!".
Heb 4:6  Deci, de vreme ce ramâne ca unii sa intre în odihna, iar aceia carora mai dinainte li s-a binevestit, pentru nesupunerea lor, n-au intrat,
Heb 4:7  Dumnezeu hotara?te din nou o zi, astazi rostind prin gura lui David, dupa atâta vreme, precum s-a zis mai sus: "Daca ve?i auzi astazi glasul Lui, nu învârto?a?i inimile voastre".
Heb 4:8  Caci daca Iosua le-ar fi adus odihna, Dumnezeu n-ar mai fi vorbit, dupa acestea, de o alta zi de odihna.
Heb 4:9  Drept aceea, s-a lasat alta sarbatoare de odihna poporului lui Dumnezeu.
Heb 4:10  Pentru ca cine a intrat în odihna lui Dumnezeu s-a odihnit ?i el de lucrurile lui, precum Dumnezeu de ale Sale.
Heb 4:11  Sa ne silim, deci, ca sa intram în acea odihna, ca nimeni sa nu cada în aceea?i pilda a neascultarii,
Heb 4:12  Caci cuvântul lui Dumnezeu e viu ?i lucrator ?i mai ascu?it decât orice sabie cu doua tai?uri, ?i patrunde pâna la despar?itura sufletului ?i duhului, dintre încheieturi ?i maduva, ?i destoinic este sa judece sim?irile ?i cugetarile inimii,
Heb 4:13  ?i nu este nici o faptura ascunsa înaintea Lui, ci toate sunt goale ?i descoperite, pentru ochii Celui în fa?a Caruia noi vom da socoteala.
Heb 4:14  Drept aceea, având Arhiereu mare, Care a strabatut cerurile, pe Iisus, Fiul lui Dumnezeu, sa ?inem cu tarie marturisirea.
Heb 4:15  Ca nu avem Arhiereu care sa nu poata suferi cu noi în slabiciunile noastre, ci ispitit întru toate dupa asemanarea noastra, afara de pacat.
Heb 4:16  Sa ne apropiem, deci, cu încredere de tronul harului, ca sa luam mila ?i sa aflam har, spre ajutor, la timp potrivit.
Heb 5:1  Caci orice arhiereu, fiind luat dintre oameni, este pus pentru oameni, spre cele catre Dumnezeu, ca sa aduca daruri ?i jertfe pentru pacate;
Heb 5:2  El poate sa fie îngaduitor cu cei ne?tiutori ?i rataci?i, de vreme ce ?i el este cuprins de slabiciune.
Heb 5:3  Din aceasta pricina dator este, precum pentru popor, a?a ?i pentru sine sa jertfeasca pentru pacate.
Heb 5:4  ?i nimeni nu-?i ia singur cinstea aceasta, ci daca este chemat de Dumnezeu dupa cum ?i Aaron.
Heb 5:5  A?a ?i Hristos nu S-a preaslavit pe Sine însu?i, ca sa Se faca arhiereu, ci Cel ce a grait catre El: "Fiul Meu e?ti Tu, Eu astazi Te-am nascut".
Heb 5:6  În alt loc se zice: "Tu e?ti Preot în veac dupa rânduiala lui Melchisedec".
Heb 5:7  El, în zilele trupului Sau, a adus, cu strigat ?i cu lacrimi, cereri ?i rugaciuni catre Cel ce putea sa-L mântuiasca din moarte ?i auzit a fost pentru evlavia Sa,
Heb 5:8  ?i de?i era Fiu, a înva?at ascultarea din cele ce a patimit,
Heb 5:9  ?i desavâr?indu-Se, S-a facut tuturor celor ce-L asculta pricina de mântuire ve?nica.
Heb 5:10  Iar de Dumnezeu a fost numit: Arhiereu dupa rânduiala lui Melchisedec.
Heb 5:11  În privin?a aceasta avem mult de vorbit ?i lucruri grele de tâlcuit, de vreme ce v-a?i facut greoi la auzit.
Heb 5:12  Caci voi, care de multa vreme s-ar fi cuvenit sa fi?i înva?atori, ave?i iara?i trebuin?a ca cineva sa va înve?e cele dintâi începuturi ale cuvintelor lui Dumnezeu ?i a?i ajuns sa ave?i nevoie de lapte, nu de hrana tare.
Heb 5:13  Pentru ca oricine se hrane?te cu lapte este nepriceput în cuvântul drepta?ii, de vreme ce este prunc.
Heb 5:14  Iar hrana tare este pentru cei desavâr?i?i, care au prin obi?nuin?a sim?urile înva?ate sa deosebeasca binele ?i raul.
Heb 6:1  De aceea, lasând cuvântul de început despre Hristos, sa ne ridicam spre ceea ce este desavâr?it, fara sa mai punem din nou temelia înva?aturii despre pocain?a de faptele moarte ?i despre credin?a în Dumnezeu,
Heb 6:2  A înva?aturii despre botezuri, despre punerea mâinilor, despre învierea mor?ilor ?i despre judecata ve?nica.
Heb 6:3  ?i aceasta vom face-o cu voia lui Dumnezeu.
Heb 6:4  Caci este cu neputin?a pentru cei ce s-au luminat odata ?i au gustat darul cel ceresc ?i parta?i s-au facut Duhului Sfânt,
Heb 6:5  ?i au gustat cuvântul cel bun al lui Dumnezeu ?i puterile veacului viitor,
Heb 6:6  Cu neputin?a este pentru ei, daca au cazut, sa se înnoiasca iara?i spre pocain?a, fiindca ei rastignesc loru?i, a doua oara, pe Fiul lui Dumnezeu ?i-L fac de batjocura.
Heb 6:7  ?arina, când absoarbe ploaia ce se coboara adeseori asupra ei ?i rode?te iarba folositoare celor pentru care a fost muncita, prime?te binecuvântarea de la Dumnezeu;
Heb 6:8  Dar daca aduce spini ?i ciulini, se face netrebnica ?i blestemul îi sta aproape iar la urma focul o a?teapta.
Heb 6:9  Despre voi, iubi?ilor, de?i vorbim astfel, suntem încredin?a?i de lucruri mai bune ?i aducatoare de mântuire.
Heb 6:10  Caci Dumnezeu nu este nedrept, ca sa uite lucrul vostru ?i dragostea pe care a?i aratat-o pentru numele Lui, voi, care a?i slujit ?i sluji?i sfin?ilor.
Heb 6:11  Dorind dar, ca fiecare dintre voi sa arate aceea?i râvna spre adeverirea nadejdii, pâna la sfâr?it,
Heb 6:12  Ca sa nu fi?i greoi, ci urmatori ai celor ce, prin credin?a ?i îndelunga-rabdare, mo?tenesc fagaduin?ele.
Heb 6:13  Caci Dumnezeu, când a dat fagaduin?a lui Avraam, de vreme ce n-avea pe nimeni mai mare, pe care sa Se jure, S-a jurat pe Sine însu?i,
Heb 6:14  Zicând: "Cu adevarat, binecuvântând te voi binecuvânta, ?i înmul?ind te voi înmul?i".
Heb 6:15  ?i a?a, având Avraam îndelunga-rabdare, a dobândit fagaduin?a.
Heb 6:16  Pentru ca oamenii se jura pe cel ce e mai mare ?i juramântul e la ei o cheza?ie ?i sfâr?itul oricarei neîn?elegeri.
Heb 6:17  În aceasta, Dumnezeu voind sa arate ?i mai mult, mo?tenitorilor fagaduin?ei, nestramutarea hotarârii Sale, a pus la mijloc juramântul:
Heb 6:18  Ca prin doua fapte nestramutate - fagaduin?a ?i juramântul - în care e cu neputin?a ca Dumnezeu sa fi min?it, noi, cei ce cautam scapare, sa avem îndemn puternic ca sa ?inem nadejdea pusa înainte,
Heb 6:19  Pe care o avem ca o ancora a sufletului, neclintita ?i tare, intrând dincolo de catapeteasma,
Heb 6:20  Unde Iisus a intrat pentru noi ca înaintemergator, fiind facut Arhiereu în veac, dupa rânduiala lui Melchisedec.
Heb 7:1  Caci acest Melchisedec, rege al Salemului, preot al lui Dumnezeu cel Preaînalt, care a întâmpinat pe Avraam, pe când se întorcea de la nimicirea regilor ?i l-a binecuvântat,
Heb 7:2  Caruia Avraam i-a dat ?i zeciuiala din toate, se tâlcuie?te mai întâi: rege al drepta?ii, apoi ?i rege al Salemului, adica rege al pacii,
Heb 7:3  Fara tata, fara mama, fara spi?a de neam, neavând nici început al zilelor, nici sfâr?it al vie?ii, ci, asemanat fiind Fiului lui Dumnezeu, el ramâne preot pururea.
Heb 7:4  Vede?i, dar, cât de mare e acesta, caruia chiar patriarhul Avraam i-a dat zeciuiala din prada de razboi.
Heb 7:5  ?i cei dintre fiii lui Levi, care primesc preo?ia, au porunca dupa lege, ca sa ia zeciuiala de la popor, adica de la fra?ii lor, macar ca ?i ace?tia au ie?it din coapsele lui Avraam;
Heb 7:6  Iar Melchisedec, care nu-?i trage neamul din ei, a primit zeciuiala de la Avraam ?i pe Avraam, care avea fagaduin?ele, l-a binecuvântat.
Heb 7:7  Fara de nici o îndoiala, cel mai mic ia binecuvântare de la cel mai mare.
Heb 7:8  ?i aici iau zeciuiala ni?te oameni muritori, pe când dincolo, unul care e dovedit ca este viu.
Heb 7:9  ?i ca sa spun a?a, prin Avraam, a dat zeciuiala ?i Levi, cel ce lua zeciuiala,
Heb 7:10  Fiindca el era înca în coapsele lui Avraam, când l-a întâmpinat Melchisedec.
Heb 7:11  Daca deci desavâr?irea ar fi fost prin preo?ia Levi?ilor (caci legea s-a dat poporului pe temeiul preo?iei lor), ce nevoie mai era sa se ridice un alt preot dupa rânduiala lui Melchisedec, ?i sa nu se zica dupa rânduiala lui Aaron?
Heb 7:12  Iar daca preo?ia s-a schimbat urmeaza numaidecât ?i schimbarea Legii.
Heb 7:13  Caci Acela, despre Care se spun acestea, î?i ia obâr?ia dintr-o alta semin?ie, de unde nimeni n-a slujit altarului,
Heb 7:14  ?tiut fiind ca Domnul nostru a rasarit din Iuda, iar despre semin?ia acestora, cu privire la preo?i, Moise n-a vorbit nimic.
Heb 7:15  Apoi este lucru ?i mai lamurit ca, daca se ridica un alt preot dupa asemanarea lui Melchisedec,
Heb 7:16  El s-a facut nu dupa legea unei porunci trupe?ti, ci cu puterea unei vie?i nepieritoare,
Heb 7:17  Caci se marturise?te: "Tu e?ti Preot în veac, dupa rânduiala lui Melchisedec".
Heb 7:18  Astfel, porunca data întâi se desfiin?eaza, pentru neputin?a ?i nefolosul ei;
Heb 7:19  Caci Legea n-a desavâr?it nimic, iar în locul ei î?i face cale o nadejde mai buna, prin care ne apropiem de Dumnezeu.
Heb 7:20  Ci înca a fost la mijloc ?i un juramânt, caci pe când aceia s-au facut preo?i fara de juramânt,
Heb 7:21  El S-a facut cu juramântul Celui ce I-a grait: "Juratu-S-a Domnul ?i nu Se va cai: Tu e?ti Preot în veac, dupa rânduiala lui Melchisedec".
Heb 7:22  Cu aceasta, Iisus S-a facut cheza?ul unui mai bun testament.
Heb 7:23  Apoi acolo s-a ridicat un ?ir de preo?i, fiindca moartea îi împiedica sa dainuiasca.
Heb 7:24  Aici însa, Iisus, prin aceea ca ramâne în veac, are o preo?ie netrecatoare (ve?nica).
Heb 7:25  Pentru aceasta, ?i poate sa mântuiasca desavâr?it pe cei ce se apropie prin El de Dumnezeu, caci pururea e viu ca sa mijloceasca pentru ei.
Heb 7:26  Un astfel de Arhiereu se cuvenea sa avem: sfânt, fara de rautate, fara de pata, osebit de cei pacato?i, ?i fiind mai presus decât cerurile.
Heb 7:27  El nu are nevoie sa aduca zilnic jertfe, ca arhiereii: întâi pentru pacatele lor, apoi pentru ale poporului, caci El a facut aceasta o data pentru totdeauna, aducându-Se jertfa pe Sine însu?i.
Heb 7:28  Caci Legea pune ca arhierei oameni care au slabiciune, pe când cuvântul juramântului, venit în urma Legii, pune pe Fiul, desavâr?it în veacul veacului.
Heb 8:1  Lucru de capetenie din cele spuse este ca avem astfel de Arhiereu care a ?ezut de-a dreapta tronului slavei în ceruri,
Heb 8:2  Slujitor Altarului ?i Cortului celui adevarat, pe care l-a înfipt Dumnezeu ?i nu omul.
Heb 8:3  Apoi, orice arhiereu este pus ca sa aduca daruri ?i jertfe; de aceea trebuincios era ca ?i acest Arhiereu sa fi avut ceva ce sa aduca.
Heb 8:4  Daca ar fi pe pamânt, nici n-ar fi preot, fiindca aici sunt aceia care aduc darurile potrivit Legii,
Heb 8:5  Care slujesc închipuirii ?i umbrei celor cere?ti, precum a primit porunca Moise, când era sa faca cortul: "Ia seama, zice Domnul, sa faci toate dupa chipul ce ?i-a fost aratat în munte".
Heb 8:6  Acum însa, Arhiereul nostru a dobândit o slujire cu atât mai osebita, cu cât este ?i Mijlocitorul unui testament mai bun, ca unul care este întemeiat pe mai bune fagaduin?e.
Heb 8:7  Caci daca (testamentul) cel dintâi ar fi fost fara de prihana, nu s-ar mai fi cautat loc pentru al doilea;
Heb 8:8  Ci Dumnezeu îi mustra ?i le zice: "Iata vin zile, zice Domnul, când voi face, cu casa lui Israel ?i cu casa lui Iuda, testament nou,
Heb 8:9  Nu ca testamentul pe care l-am facut cu parin?ii lor, în ziua când i-am apucat de mâna ca sa-i scot din pamântul Egiptului; caci ei n-au ramas în testamentul Meu, de aceea ?i Eu i-am parasit - zice Domnul.
Heb 8:10  Ca acesta e testamentul pe care îl voi face cu casa lui Israel, dupa acele zile, zice Domnul: Pune-voi legile Mele în cugetul lor ?i în inima lor le voi scrie, ?i voi fi lor Dumnezeu ?i ei vor fi poporul Meu.
Heb 8:11  ?i nu va mai înva?a fiecare pe vecinul sau ?i fiecare pe fratele sau zicând: Cunoa?te pe Domnul! - caci to?i Ma vor cunoa?te, de la cel mai mic pâna la cel mai mare al lor;
Heb 8:12  Caci voi fi milostiv cu nedrepta?ile lor ?i de pacatele lor nu-Mi voi mai aduce aminte".
Heb 8:13  ?i zicând: "Nou", Domnul a învechit pe cel dintâi. Iar ce se înveche?te ?i îmbatrâne?te, aproape este de pieire.
Heb 9:1  Deci ?i cei dintâi (A?ezamânt) avea orânduieli pentru slujba dumnezeiasca ?i un altar pamântesc,
Heb 9:2  Caci s-a pregatit cortul marturiei. În el se aflau, mai întâi, sfe?nicul ?i masa ?i pâinile punerii înainte; partea aceasta se nume?te Sfânta.
Heb 9:3  Apoi, dupa catapeteasma a doua, era cortul numit Sfânta Sfintelor,
Heb 9:4  Având altarul tamâierii de aur ?i chivotul A?ezamântului ferecat peste tot cu aur, în care era nastrapa de aur, care avea mana, era toiagul lui Aaron ce odraslise ?i tablele Legii.
Heb 9:5  Deasupra chivotului erau heruvimii slavei, care umbreau altarul împacarii; despre acestea nu putem acum sa vorbim cu de-amanuntul.
Heb 9:6  Astfel fiind întocmite aceste încaperi, preo?ii intrau totdeauna în cortul cel dintâi, savâr?ind slujbele dumnezeie?ti;
Heb 9:7  În cel de-al doilea însa numai arhiereul, o data pe an, ?i nu fara de sânge, pe care îl aducea pentru sine însu?i ?i pentru gre?ealele poporului.
Heb 9:8  Prin aceasta, Duhul Sfânt ne lamure?te ca drumul catre Sfânta Sfintelor nu era sa fie aratat, câta vreme cortul întâi mai sta în picioare,
Heb 9:9  Care era o pilda pentru timpul de fa?a ?i însemna ca darurile ?i jertfele ce se aduceau n-aveau putere sa desavâr?easca cugetul închinatorului.
Heb 9:10  Acestea erau numai legiuiri pamânte?ti - despre mâncaruri, despre bauturi, despre felurite spalari - ?i erau porunci pâna la vremea îndreptarii.
Heb 9:11  Iar Hristos, venind Arhiereu al bunata?ilor celor viitoare, a trecut prin cortul cel mai mare ?i mai desavâr?it, nu facut de mâna, adica nu din zidirea aceasta;
Heb 9:12  El a intrat o data pentru totdeauna în Sfânta Sfintelor, nu cu sânge de ?api ?i de vi?ei, ci cu însu?i sângele Sau, ?i a dobândit o ve?nica rascumparare.
Heb 9:13  Caci daca sângele ?apilor ?i al taurilor ?i cenu?a junincii, stropind pe cei spurca?i, îi sfin?e?te spre cura?irea trupului,
Heb 9:14  Cu cât mai mult sângele lui Hristos, Care, prin Duhul cel ve?nic, S-a adus lui Dumnezeu pe Sine, jertfa fara de prihana, va cura?i cugetul vostru de faptele cele moarte, ca sa sluji?i Dumnezeului celui viu?
Heb 9:15  ?i pentru aceasta El este Mijlocitorul unui nou testament, ca prin moartea suferita spre rascumpararea gre?ealelor de sub întâiul testament, cei chema?i sa ia fagaduin?a mo?tenirii ve?nice.
Heb 9:16  Caci unde este testament, trebuie neaparat sa fie vorba despre moartea celui ce a facut testamentul.
Heb 9:17  Un testament ajunge temeinic dupa moarte, fiindca nu are nici o putere, câta vreme traie?te cel ce l-a facut.
Heb 9:18  De aceea, nici cel dintâi n-a fost sfin?it fara sânge.
Heb 9:19  Într-adevar Moise, dupa ce a rostit fa?a cu tot poporul toate poruncile din Lege, luând sângele cel de vi?ei ?i de ?api, cu apa ?i cu lâna ro?ie ?i cu isop, a stropit ?i cartea ?i pe tot poporul,
Heb 9:20  ?i a zis: "Acesta este sângele testamentului pe care l-a poruncit voua Dumnezeu".
Heb 9:21  ?i a stropit, de asemenea, cu sânge, cortul ?i toate vasele pentru slujba.
Heb 9:22  Dupa Lege, aproape toate se cura?esc cu sânge, ?i fara varsare de sânge nu se da iertare.
Heb 9:23  Trebuie dar ca chipurile celor din ceruri sa fie cura?ite prin acestea, iar cele cere?ti înse?i cu jertfe mai bune decât acestea.
Heb 9:24  Caci Hristos n-a intrat într-o Sfânta a Sfintelor facuta de mâini - închipuirea celei adevarate - ci chiar în cer, ca sa Se înfa?i?eze pentru noi înaintea lui Dumnezeu;
Heb 9:25  Iar nu ca sa Se aduca pe Sine însu?i jertfa de mai multe ori - ca arhiereul care intra în Sfânta Sfintelor cu sânge strain, în fiecare an.
Heb 9:26  Altfel, ar fi trebuit sa patimeasca de mai multe ori, de la întemeierea lumii; ci acum, la sfâr?itul veacurilor, S-a aratat o data, spre ?tergerea pacatului, prin jertfa Sa.
Heb 9:27  ?i precum este rânduit oamenilor o data sa moara, iar dupa aceea sa fie judecata,
Heb 9:28  Tot a?a ?i Hristos, dupa ce a fost adus o data jertfa, ca sa ridice pacatele multora, a doua oara fara de pacat Se va arata celor ce cu staruin?a Îl a?teapta spre mântuire.
Heb 10:1  În adevar, Legea având umbra bunurilor viitoare, iar nu însu?i chipul lucrurilor, nu poate niciodata - cu acelea?i jertfe, aduse neîncetat în fiecare an - sa faca desavâr?i?i pe cei ce se apropie.
Heb 10:2  Altfel, n-ar fi încetat oare jertfele aduse, daca cei ce savâr?esc slujba dumnezeiasca, fiind o data cura?i?i, n-ar mai avea nici o con?tiin?a a pacatelor?
Heb 10:3  Ci prin ele, an de an, se face amintirea pacatelor.
Heb 10:4  Pentru ca este cu neputin?a ca sângele de tauri ?i de ?api sa înlature pacatele.
Heb 10:5  Drept aceea, intrând în lume, zice: "Jertfa ?i prinos n-ai voit, dar mi-ai întocmit trup.
Heb 10:6  Arderi de tot ?i jertfe pentru pacat nu ?i-au placut;
Heb 10:7  Atunci am zis: Iata vin, în sulul car?ii este scris despre mine, sa fac voia Ta, Dumnezeule".
Heb 10:8  Zicând mai sus ca: "Jertfa ?i prinoase ?i arderile de tot ?i jertfele pentru pacat n-ai voit, nici nu ?i-au placut", care se aduc dupa Lege,
Heb 10:9  Atunci a zis: "Iata vin ca sa fac voia Ta, Dumnezeule". El desfiin?eaza deci pe cei dintâi ca sa statorniceasca pe al doilea.
Heb 10:10  Întru aceasta voin?a suntem sfin?i?i, prin jertfa trupului lui Iisus Hristos, o data pentru totdeauna.
Heb 10:11  ?i orice preot sta ?i sluje?te în fiecare zi ?i acelea?i jertfe aduce de multe ori, ca unele care niciodata nu pot sa înlature pacatele.
Heb 10:12  Acesta dimpotriva, aducând o singura jertfa pentru pacate, a ?ezut în vecii vecilor, de-a dreapta lui Dumnezeu,
Heb 10:13  ?i a?teapta pâna ce vrajma?ii Lui vor fi pu?i a?ternut picioarelor Lui.
Heb 10:14  Caci printr-o singura jertfa adusa, a adus la ve?nica desavâr?ire pe cei ce se sfin?esc;
Heb 10:15  Dar ?i Duhul cel Sfânt ne marturise?te aceasta, fiindca dupa ce a zis:
Heb 10:16  "Acesta este a?ezamântul pe care îl voi întocmi cu ei, dupa acele zile - zice Domnul: Da-voi legile Mele în inimile lor ?i le voi scrie în cugetele lor".
Heb 10:17  ?i adauga: "Iar de pacatele lor ?i de faradelegile lor nu-Mi voi mai aduce aminte".
Heb 10:18  Unde este dar iertarea acestora, nu mai este jertfa pentru pacate.
Heb 10:19  Drept aceea, fra?ilor, având îndrazneala, sa intram în Sfânta Sfintelor, prin sângele lui Iisus,
Heb 10:20  Pe calea cea noua ?i vie pe care pentru noi a înnoit-o, prin catapeteasma, adica prin trupul Sau,
Heb 10:21  ?i având mare preot peste casa lui Dumnezeu,
Heb 10:22  Sa ne apropiem cu inima curata, întru plinatatea credin?ei, cura?indu-ne prin stropire inimile de orice cuget rau, ?i spalându-ne trupul în apa curata,
Heb 10:23  Sa ?inem marturisirea nadejdii cu neclintire, pentru ca credincios este Cel ce a fagaduit,
Heb 10:24  ?i sa luam seama unul altuia, ca sa ne îndemnam la dragoste ?i la fapte bune,
Heb 10:25  Fara sa parasim Biserica noastra, precum le este obiceiul unora, ci îndemnatori facându-ne, cu atât mai mult, cu cât vede?i ca se apropie ziua aceea.
Heb 10:26  Caci daca pacatuim de voia noastra, dupa ce am luat cuno?tiin?a despre adevar, nu ne mai ramâne, pentru pacate, nici o jertfa,
Heb 10:27  Ci o înfrico?ata a?teptare a judeca?ii ?i iu?imea focului care va mistui pe cei potrivnici.
Heb 10:28  Calcând cineva Legea lui Moise, e ucis fara de mila, pe cuvântul a doi sau trei martori;
Heb 10:29  Gândi?i-va: cu cât mai aspra fi-va pedeapsa cuvenita celui ce a calcat în picioare pe Fiul lui Dumnezeu, ?i a nesocotit sângele testamentului cu care s-a sfin?it, ?i a batjocorit duhul harului.
Heb 10:30  Caci cunoa?tem pe Cel ce a zis: "A Mea este razbunarea; Eu voi rasplati". ?i iara?i: "Domnul va judeca pe poporul Sau".
Heb 10:31  Înfrico?ator lucru este sa cadem în mâinile Dumnezeului celui viu.
Heb 10:32  Aduce?i-va, dar, aminte mai întâi de zilele în care, dupa ce a?i fost lumina?i, a?i rabdat lupta grea de suferin?e,
Heb 10:33  Parte facându-va priveli?te cu ocarile ?i cu necazurile îndurate, parte suferind împreuna cu cei ce treceau prin unele ca acestea,
Heb 10:34  Caci a?i avut mila de cei închi?i, iar rapirea averilor voastre a?i primit-o cu bucurie, bine ?tiind ca voi ave?i o mai buna ?i statornica avere.
Heb 10:35  Nu lepada?i dar încrederea voastra, care are mare rasplatire.
Heb 10:36  Caci ave?i nevoie de rabdare ca, facând voia lui Dumnezeu, sa dobândi?i fagaduin?a.
Heb 10:37  "Caci mai este pu?in timp, prea pu?in, ?i Cel ce e sa vina, va veni ?i nu va întârzia;
Heb 10:38  Iar dreptul din credin?a va fi viu; ?i de se va îndoi cineva, nu va binevoi sufletul Meu întru el".
Heb 10:39  Noi nu suntem (fii) ai îndoielii spre pieire, ci ai credin?ei spre dobândirea sufletului.
Heb 11:1  Iar credin?a este încredin?area celor nadajduite, dovedirea lucrurilor celor nevazute.
Heb 11:2  Prin ea, cei din vechime au dat buna lor marturie.
Heb 11:3  Prin credin?a în?elegem ca s-au întemeiat veacurile prin cuvântul lui Dumnezeu, de s-au facut din nimic cele ce se vad.
Heb 11:4  Prin credin?a, Abel a adus lui Dumnezeu mai buna jertfa decât Cain, pentru care a luat marturie ca este drept, marturisind Dumnezeu despre darurile lui; ?i prin credin?a graie?te ?i azi, de?i a murit.
Heb 11:5  Prin credin?a, Enoh a fost luat de pe pamânt ca sa nu vada moartea, ?i nu s-a mai aflat, pentru ca Dumnezeu îl stramutase, caci mai înainte de a-l stramuta, el a avut marturie ca a bine-placut lui Dumnezeu.
Heb 11:6  Fara credin?a, dar, nu este cu putin?a sa fim placu?i lui Dumnezeu, caci cine se apropie de Dumnezeu trebuie sa creada ca El este ?i ca Se face rasplatitor celor care Îl cauta.
Heb 11:7  Prin credin?a, luând Noe în?tiin?are de la Dumnezeu despre cele ce nu se vedeau înca, a gatit, cu evlavie, o corabie spre mântuirea casei sale; prin credin?a el a osândit lumea ?i drepta?ii celei din credin?a s-a facut mo?tenitor.
Heb 11:8  Prin credin?a, Avraam, când a fost chemat, a ascultat ?i a ie?it la locul pe care era sa-l ia spre mo?tenire ?i a ie?it ne?tiind încotro merge.
Heb 11:9  Prin credin?a, a locuit vremelnic în pamântul fagaduin?ei, ca într-un pamânt strain, locuind în corturi cu Isaac ?i cu Iacov, cei dimpreuna mo?tenitori ai aceleia?i fagaduin?e;
Heb 11:10  Caci a?tepta cetatea cu temelii puternice, al carei me?ter ?i lucrator este Dumnezeu.
Heb 11:11  Prin credin?a, ?i Sara însa?i a primit putere sa zamisleasca fiu, de?i trecuse de vârsta cuvenita, pentru ca ea L-a socotit credincios pe Cel ce fagaduise.
Heb 11:12  Pentru aceea, dintr-un singur om, ?i acela ca ?i mort, s-au nascut atâ?ia urma?i - mul?i "ca stelele cerului ?i ca nisipul cel fara de numar de pe ?armul marii".
Heb 11:13  To?i ace?tia au murit întru credin?a, fara sa primeasca fagaduin?ele, ci vazându-le de departe ?i iubindu-le cu dor ?i marturisind ca pe pamânt ei sunt straini ?i calatori.
Heb 11:14  Iar cei ce graiesc unele ca acestea dovedesc ca ei î?i cauta lor patrie.
Heb 11:15  Într-adevar, daca ar fi avut în minte pe aceea din care ie?isera, aveau vreme sa se întoarca.
Heb 11:16  Dar acum ei doresc una mai buna, adica pe cea cereasca. Pentru aceea Dumnezeu nu Se ru?ineaza de ei ca sa Se numeasca Dumnezeul lor, caci le-a gatit lor cetate.
Heb 11:17  Prin credin?a, Avraam, când a fost încercat, a adus pe Isaac (jertfa). Cel ce primise fagaduin?ele aducea jertfa pe fiul sau unul nascut!
Heb 11:18  Catre el graise Dumnezeu: "Ca în Isaac ?i se va chema ?ie urma?".
Heb 11:19  Dar Avraam a socotit ca Dumnezeu este puternic sa-l învieze ?i din mor?i; drept aceea l-a dobândit înapoi ca un fel de pilda (a învierii) Lui.
Heb 11:20  Prin credin?a despre cele viitoare a binecuvântat Isaac pe Iacov ?i pe Esau.
Heb 11:21  Prin credin?a Iacov, când a fost sa moara, a binecuvântat pe fiecare din fiii lui Iosif ?i s-a închinat, rezemându-se pe vârful toiagului sau.
Heb 11:22  Prin credin?a Iosif, la sfâr?itul vie?ii, a pomenit despre ie?irea fiilor lui Israel ?i a dat porunci cu privire la oasele sale.
Heb 11:23  Prin credin?a, când s-a nascut Moise, a fost ascuns de parin?ii lui trei luni, caci l-au vazut prunc frumos ?i nu s-au temut de porunca regelui.
Heb 11:24  Prin credin?a, Moise, când s-a facut mare, n-a vrut sa fie numit fiul fiicei lui Faraon,
Heb 11:25  Ci a ales mai bine sa patimeasca cu poporul lui Dumnezeu, decât sa aiba dulcea?a cea trecatoare a pacatului,
Heb 11:26  Socotind ca batjocorirea pentru Hristos este mai mare boga?ie decât comorile Egiptului, fiindca se uita la rasplatire.
Heb 11:27  Prin credin?a, a parasit Egiptul, fara sa se teama de urgia regelui, caci a ramas neclintit, ca cel care vede pe Cel nevazut.
Heb 11:28  Prin credin?a, a rânduit Pa?tile ?i stropirea cu sânge, ca îngerul nimicitor sa nu se atinga de cei întâi-nascu?i ai lor.
Heb 11:29  Prin credin?a au trecut israeli?ii Marea Ro?ie, ca pe uscat, pe care egiptenii, încercând ?i ei s-o treaca, s-au înecat.
Heb 11:30  Prin credin?a, zidurile Ierihonului au cazut, dupa ce au fost înconjurate ?apte zile.
Heb 11:31  Prin credin?a Rahav, desfrânata, fiindca primise cu pace iscoadele, n-a pierit împreuna cu cei neascultatori.
Heb 11:32  ?i ce voi mai zice? Caci timpul nu-mi va ajunge, ca sa vorbesc de Ghedeon, de Barac, de Samson, de Ieftae, de David, de Samuel ?i de prooroci,
Heb 11:33  Care prin credin?a, au biruit împara?ii, au facut dreptate, au dobândit fagaduin?ele, au astupat gurile leilor,
Heb 11:34  Au stins puterea focului, au scapat de ascu?i?ul sabiei, s-au împuternicit, din slabi ce erau s-au facut tari în razboi, au întors taberele vrajma?ilor pe fuga;
Heb 11:35  Unele femei ?i-au luat pe mor?ii lor învia?i. Iar al?ii au fost chinui?i, neprimind izbavirea, ca sa dobândeasca mai buna înviere;
Heb 11:36  Al?ii au suferit batjocura ?i bici, ba chiar lan?uri ?i închisoare;
Heb 11:37  Au fost uci?i cu pietre, au fost pu?i la cazne, au fost taia?i cu fierastraul, au murit uci?i cu sabia, au pribegit în piei de oaie ?i în piei de capra, lipsi?i, strâmtora?i, rau primi?i.
Heb 11:38  Ei, de care lumea nu era vrednica, au ratacit în pustii, ?i în mun?i, ?i în pe?teri, ?i în crapaturile pamântului.
Heb 11:39  ?i to?i ace?tia, marturisi?i fiind prin credin?a, n-au primit fagaduin?a,
Heb 11:40  Pentru ca Dumnezeu rânduise pentru noi ceva mai bun, ca ei sa nu ia fara noi desavâr?irea.
Heb 12:1  De aceea ?i noi, având împrejurul nostru atâta nor de marturii, sa lepadam orice povara ?i pacatul ce grabnic ne împresoara ?i sa alergam cu staruin?a în lupta care ne sta înainte.
Heb 12:2  Cu ochii a?inti?i asupra lui Iisus, începatorul ?i plinitorul credin?ei, Care, pentru bucuria pusa înainte-I, a suferit crucea, n-a ?inut seama de ocara ei ?i a ?ezut de-a dreapta tronului lui Dumnezeu.
Heb 12:3  Lua?i aminte, dar, la Cel ce a rabdat de la pacato?i, asupra Sa, o atât de mare împotrivire, ca sa nu va lasa?i osteni?i, slabind în sufletele voastre.
Heb 12:4  În lupta voastra cu pacatul, nu v-a?i împotrivit înca pâna la sânge.
Heb 12:5  ?i a?i uitat îndemnul care va graie?te ca unor fii: "Fiul meu, nu dispre?ui certarea Domnului, nici nu te descuraja, când e?ti mustrat de El.
Heb 12:6  Caci pe cine îl iube?te Domnul îl cearta, ?i biciuie?te pe tot fiul pe care îl prime?te".
Heb 12:7  Rabda?i spre în?elep?ire, Dumnezeu se poarta cu voi ca fa?a de fii. Caci care este fiul pe care tatal sau nu-l pedepse?te?
Heb 12:8  Iar daca sunte?i fara de certare, de care to?i au parte, atunci sunte?i fii nelegitimi ?i nu fii adevara?i.
Heb 12:9  Apoi daca am avut pe parin?ii no?tri dupa trup, care sa ne certe, ?i ne sfiam de ei, oare nu ne vom supune cu atât mai vârtos Tatalui duhurilor, ca sa avem via?a?
Heb 12:10  Pentru ca ei, precum gaseau cu cale, ne pedepseau pentru pu?ine zile, iar Acesta, spre folosul nostru, ca sa ne împarta?im de sfin?enia Lui.
Heb 12:11  Orice mustrare, la început, nu pare ca e de bucurie, ci de întristare, dar mai pe urma da celor încerca?i cu ea roada pa?nica a drepta?ii.
Heb 12:12  Pentru aceea, "îndrepta?i mâinile cele ostenite ?i genunchii cei slabanogi?i.
Heb 12:13  Face?i carari drepte pentru picioarele voastre", a?a încât cine este ?chiop sa nu se abata, ci mai vârtos sa se vindece.
Heb 12:14  Cauta?i pacea cu to?i ?i sfin?enia, fara de care nimeni nu va vedea pe Domnul,
Heb 12:15  Veghind cu luare aminte ca nimeni sa nu ramâna lipsit de harul lui Dumnezeu ?i ca nu cumva, odraslind vreo pricina de amaraciune, sa va tulbure, ?i prin ea mul?i sa se molipseasca.
Heb 12:16  ?i sa nu fie vreunul desfrânat sau întinat ca Esau, care pentru o mâncare ?i-a vândut dreptul de întâi nascut.
Heb 12:17  ?ti?i ca mai pe urma, când a dorit sa mo?teneasca binecuvântarea, nu a fost luat în seama, caci, de?i cu lacrimi a cautat, n-a mai avut cum sa schimbe hotarârea.
Heb 12:18  Caci voi nu v-a?i apropiat nici de muntele ce putea fi pipait, nici de focul care ardea cu flacara, nici de nor, nici de bezna, nici de vijelie,
Heb 12:19  Nici de glasul trâmbi?ei, nici de rasunetul cuvintelor despre care cei ce îl auzeau s-au rugat sa nu li se mai graiasca,
Heb 12:20  Deoarece nu puteau sa sufere porunca: "Chiar daca ?i fiara de s-ar atinge de munte, sa fie ucisa cu pietre, sau sa fie strapunsa cu sageata",
Heb 12:21  ?i atât de înfrico?atoare era aratarea, încât Moise a zis: "Sunt înspaimântat ?i ma cutremur!".
Heb 12:22  Ci v-a?i apropiat de muntele Sion ?i de cetatea Dumnezeului celui viu, de Ierusalimul cel ceresc ?i de zeci de mii de îngeri, în adunare sarbatoreasca,
Heb 12:23  ?i de Biserica celor întâi nascu?i, care sunt scri?i în ceruri ?i de Dumnezeu, Judecatorul tuturor, ?i de duhurile drep?ilor celor desavâr?i?i,
Heb 12:24  ?i de Iisus, Mijlocitorul noului testament, ?i de sângele stropirii care graie?te mai bine decât al lui Abel.
Heb 12:25  Lua?i seama sa nu va lepada?i de Cel care vorbe?te. Caci daca aceia n-au scapat de pedeapsa, nevoind sa asculte pe cel ce le graia pe pamânt, cu atât mai mult noi - îndepartându-ne de Cel ce ne graie?te din ceruri -
Heb 12:26  Al Carui glas, odinioara, a zguduit pamântul, iar acum, vorbind, a fagaduit: "Înca o data voi clatina nu numai pamântul, ci ?i cerul".
Heb 12:27  Iar prin aceea ca zice: "Înca o data" arata schimbarea celor clatinate, ca a unor lucruri facute, ca sa ramâna cele neclintite.
Heb 12:28  De aceea, fiindca primim o împara?ie neclintita, sa fim mul?umitori, ?i a?a sa-I aducem lui Dumnezeu închinare placuta, cu evlavie ?i cu sfiala.
Heb 12:29  Caci "Dumnezeul nostru este ?i foc mistuitor".
Heb 13:1  Ramâne?i întru dragostea fra?easca.
Heb 13:2  Primirea de oaspe?i sa n-o uita?i caci prin aceasta unii, fara ca sa ?tie, au primit în gazda, îngeri.
Heb 13:3  Aduce?i-va aminte de cei închi?i, ca ?i cum a?i fi închi?i cu ei; aduce?i-va aminte de cei ce îndura rele, întrucât ?i voi sunte?i în trup.
Heb 13:4  Cinstita sa fie nunta întru toate ?i patul nespurcat. Iar pe desfrâna?i îi va judeca Dumnezeu.
Heb 13:5  Feri?i-va de iubirea de argint ?i îndestula?i-va cu cele ce ave?i, caci însu?i Dumnezeu a zis: "Nu te voi lasa, nici nu te voi parasi".
Heb 13:6  Pentru aceea, având buna îndrazneala, sa zicem: "Domnul este într-ajutorul meu; nu ma voi teme! Ce-mi va face mie omul?".
Heb 13:7  Aduce?i-va aminte de mai-marii vo?tri, care v-au grait voua cuvântul lui Dumnezeu; privi?i cu luare aminte cum ?i-au încheiat via?a ?i urma?i-le credin?a.
Heb 13:8  Iisus Hristos, ieri ?i azi ?i în veci, este acela?i.
Heb 13:9  Nu va lasa?i fura?i de înva?aturile straine cele de multe feluri; caci bine este sa va întari?i prin har inima voastra, nu cu mâncaruri, de la care n-au avut nici un folos cei ce au umblat cu ele.
Heb 13:10  Avem altar, de la care nu au dreptul sa manânce cei ce slujesc cortului.
Heb 13:11  Într-adevar, trupurile dobitoacelor - al caror sânge e adus de arhiereu, pentru împacare, în Sfânta Sfintelor - sunt arse afara din tabara.
Heb 13:12  Pentru aceea ?i Iisus, ca sa sfin?easca poporul cu sângele Sau, a patimit în afara por?ii.
Heb 13:13  Deci dar sa ie?im la El, afara din tabara, luând asupra noastra ocara Lui.
Heb 13:14  Caci nu avem aici cetate statatoare, ci o cautam pe aceea ce va sa fie.
Heb 13:15  A?adar, prin El sa aducem pururea lui Dumnezeu jertfa de lauda, adica rodul buzelor, care preaslavesc numele Lui.
Heb 13:16  Iar facerea de bine ?i întrajutorarea nu le da?i uitarii; caci astfel de jertfe sunt bine placute lui Dumnezeu.
Heb 13:17  Asculta?i pe mai-marii vo?tri ?i va supune?i lor, fiindca ei privegheaza pentru sufletele voastre, având sa dea de ele seama, ca sa faca aceasta cu bucurie ?i nu suspinând, caci aceasta nu v-ar fi de folos.
Heb 13:18  Ruga?i-va pentru noi; caci suntem încredin?a?i ca avem un cuget bun, dorind ca întru toate cu cinste sa traim.
Heb 13:19  ?i mai mult va rog sa face?i aceasta, ca sa va fiu dat înapoi mai curând.
Heb 13:20  Iar Dumnezeul pacii, Cel ce, prin sângele unui testament ve?nic, a sculat din mor?i pe Pastorul cel mare al oilor, pe Domnul nostru Iisus,
Heb 13:21  Sa va întareasca în orice lucru bun, ca sa face?i voia Lui, ?i sa lucreze în noi ceea ce este bine placut în fa?a Lui, prin Iisus Hristos, Caruia fie slava în vecii vecilor. Amin!
Heb 13:22  ?i va rog, fra?ilor, sa îngadui?i acest cuvânt de îndemn, caci vi l-am scris pe scurt.
Heb 13:23  Sa ?ti?i ca fratele Timotei este slobod. Daca vine mai degraba, va voi vedea împreuna cu el.
Heb 13:24  Îmbra?i?a?i pe to?i mai-marii vo?tri ?i pe to?i sfin?ii. Va îmbra?i?eaza cei din Italia.
Heb 13:25  Harul fie cu voi cu to?i! Amin.


\end{document}