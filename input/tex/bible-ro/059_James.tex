\begin{document}

\title{James}

Jas 1:1  Iacov, robul lui Dumnezeu ?i al Domnului Iisus Hristos, celor douasprezece semin?ii, care sunt în împra?tiere, salutare.
Jas 1:2  Mare bucurie sa socoti?i, fra?ii mei, când cade?i în felurite ispite,
Jas 1:3  ?tiind ca încercarea credin?ei voastre lucreaza rabdarea;
Jas 1:4  Iar rabdarea sa-?i aiba lucrul ei desavâr?it, ca sa fi?i desavâr?i?i ?i întregi, nelipsi?i fiind de nimic.
Jas 1:5  ?i de este cineva din voi lipsit de în?elepciune, sa o ceara de la Dumnezeu, Cel ce da tuturor fara deosebire ?i fara înfruntare; ?i i se va da.
Jas 1:6  Sa ceara însa cu credin?a, fara sa aiba nici o îndoiala, pentru ca cine se îndoie?te este asemenea valului marii, mi?cat de vânt ?i aruncat încoace ?i încolo.
Jas 1:7  Sa nu gândeasca omul acela ca va lua ceva de la Dumnezeu.
Jas 1:8  Barbatul îndoielnic este nestatornic în toate caile sale.
Jas 1:9  Iar fratele cel smerit sa se laude întru înal?imea sa,
Jas 1:10  ?i cel bogat întru smerenia sa, pentru ca va trece ca floarea ierbii.
Jas 1:11  Caci a rasarit soarele arzator ?i a uscat iarba ?i floarea ei a cazut ?i frumuse?ea fe?ei ei a pierit; tot a?a se va ve?teji ?i bogatul în alergaturile sale.
Jas 1:12  Fericit este barbatul care rabda ispita, caci lamurit facându-se va lua cununa vie?ii, pe care a fagaduit-o Dumnezeu celor ce Îl iubesc pe El.
Jas 1:13  Nimeni sa nu zica, atunci când este ispitit: De la Dumnezeu sunt ispitit, pentru ca Dumnezeu nu este ispitit de rele ?i El însu?i nu ispite?te pe nimeni.
Jas 1:14  Ci fiecare este ispitit când este tras ?i momit de însa?i pofta sa.
Jas 1:15  Apoi pofta, zamislind, na?te pacat, iar pacatul, odata savâr?it, aduce moarte.
Jas 1:16  Nu va în?ela?i, fra?ii mei prea iubi?i:
Jas 1:17  Toata darea cea buna ?i tot darul desavâr?it de sus este, pogorându-se de la Parintele luminilor, la Care nu este schimbare sau umbra de mutare.
Jas 1:18  Dupa voia Sa ne-a nascut prin cuvântul adevarului, ca sa fim începatura fapturilor Lui.
Jas 1:19  Sa ?ti?i, iubi?ii mei fra?i: orice om sa fie grabnic la ascultare, zabavnic la vorbire, zabavnic la mânie.
Jas 1:20  Caci mânia omului nu lucreaza dreptatea lui Dumnezeu.
Jas 1:21  Pentru aceea, lepadând toata spurcaciunea ?i prisosin?a rauta?ii, primi?i cu blânde?e cuvântul sadit în voi, care poate sa mântuiasca sufletele voastre.
Jas 1:22  Dar face?i-va împlinitori ai cuvântului, nu numai ascultatori ai lui, amagindu-va pe voi în?iva.
Jas 1:23  Caci daca cineva este ascultator al cuvântului, iar nu ?i împlinitor, el seamana cu omul care prive?te în oglinda fa?a firii sale;
Jas 1:24  S-a privit pe sine ?i s-a dus ?i îndata a uitat ce fel era.
Jas 1:25  Cine s-a uitat, însa, de aproape în legea cea desavâr?ita a liberta?ii ?i a staruit în ea, facându-se nu ascultator care uita, ci împlinitor al lucrului, acela fericit va fi în lucrarea sa.
Jas 1:26  Daca cineva socote?te ca e cucernic, dar nu î?i ?ine limba în frâu, ci î?i amage?te inima, cucernicia acestuia este zadarnica.
Jas 1:27  Cucernicia curata ?i neîntinata înaintea lui Dumnezeu ?i Tatal, aceasta este: sa cercetam pe orfani ?i pe vaduve în necazurile lor, ?i sa ne pazim pe noi fara de pata din partea lumii.
Jas 2:1  Fra?ii mei, nu cautând la fa?a omului sa ave?i credin?a în Domnul nostru Iisus Hristos, Domnul slavei.
Jas 2:2  Caci, daca va intra în adunarea voastra un om cu inele de aur în degete, în haina stralucita, ?i va intra ?i un sarac, în haina murdara,
Jas 2:3  Iar voi pune?i ochii pe cel care poarta haina stralucita ?i-i zice?i: Tu ?ezi bine aici, pe când saracului îi zice?i: Tu stai acolo, în picioare, sau: ?ezi jos, la picioarele mele,
Jas 2:4  N-a?i facut voi, oare, în gândul vostru, deosebire între unul ?i altul ?i nu v-a?i facut judecatori cu socoteli viclene?
Jas 2:5  Asculta?i, iubi?ii mei fra?i: Au nu Dumnezeu i-a ales pe cei ce sunt saraci în ochii lumii, dar boga?i în credin?a ?i mo?tenitori ai împara?iei pe care a fagaduit-o El celor ce Îl iubesc?
Jas 2:6  Iar voi a?i necinstit pe cel sarac! Oare nu boga?ii va asupresc pe voi ?i nu ei va târasc la judeca?i?
Jas 2:7  Nu sunt ei cei ce hulesc numele cel bun întru care a?i fost chema?i?
Jas 2:8  Daca, într-adevar, împlini?i legea împarateasca, potrivit Scripturii: "Sa iube?ti pe aproapele tau ca pe tine însu?i", bine face?i;
Jas 2:9  Iar de cauta?i la fa?a omului, face?i pacat ?i legea va osânde?te ca pe ni?te calcatori de lege.
Jas 2:10  Pentru ca cine va pazi toata legea, dar va gre?i într-o singura porunca, s-a facut vinovat fa?a de toate poruncile.
Jas 2:11  Caci Cel ce a zis: "Sa nu savâr?e?ti adulter", a zis ?i: "Sa nu ucizi". ?i daca nu savâr?e?ti adulter, dar ucizi, te-ai facut calcator de lege.
Jas 2:12  A?a sa grai?i ?i a?a sa lucra?i, ca unii care ve?i fi judeca?i prin legea liberta?ii.
Jas 2:13  Caci judecata este fara mila pentru cel care n-a facut mila. ?i mila biruie?te în fa?a judeca?ii.
Jas 2:14  Ce folos, fra?ii mei, daca zice cineva ca are credin?a, iar fapte nu are? Oare credin?a poate sa-l mântuiasca?
Jas 2:15  Daca un frate sau o sora sunt goi ?i lipsi?i de hrana cea de toate zilele,
Jas 2:16  ?i cineva dintre voi le-ar zice: Merge?i în pace! Încalzi?i-va ?i va satura?i, dar nu le da?i cele trebuincioase trupului, care ar fi folosul?
Jas 2:17  A?a ?i cu credin?a: daca nu are fapte, e moarta în ea însa?i.
Jas 2:18  Dar va zice cineva: Tu ai credin?a, iar eu am fapte; arata-mi credin?a ta fara fapte ?i eu î?i voi arata, din faptele mele, credin?a mea.
Jas 2:19  Tu crezi ca unul este Dumnezeu? Bine faci; dar ?i demonii cred ?i se cutremura.
Jas 2:20  Vrei însa sa în?elegi, omule nesocotit, ca credin?a fara de fapte moarta este?
Jas 2:21  Avraam, parintele nostru, au nu din fapte s-a îndreptat, când a pus pe Isaac, fiul sau, pe jertfelnic?
Jas 2:22  Vezi ca, credin?a lucra împreuna cu faptele lui ?i din fapte credin?a s-a desavâr?it?
Jas 2:23  ?i s-a împlinit Scriptura care zice: "?i a crezut Avraam lui Dumnezeu ?i i s-a socotit lui ca dreptate" ?i "a fost numit prieten al lui Dumnezeu".
Jas 2:24  Vede?i dar ca din fapte este îndreptat omul, iar nu numai din credin?a.
Jas 2:25  La fel ?i Rahav, desfrânata, au nu din fapte s-a îndreptat când a primit pe cei trimi?i ?i i-a scos afara, pe alta cale?
Jas 2:26  Caci precum trupul fara suflet mort este, astfel ?i credin?a fara de fapte, moarta este.
Jas 3:1  Nu va face?i voi mul?i înva?atori, fra?ii mei, ?tiind ca (noi, înva?atorii) mai mare osânda vom lua.
Jas 3:2  Pentru ca to?i gre?im în multe chipuri; daca nu gre?e?te cineva în cuvânt, acela este barbat desavâr?it, în stare sa înfrâneze ?i tot trupul.
Jas 3:3  Dar, daca noi punem în gura cailor frâul, ca sa ni-i supunem, ducem dupa noi ?i trupul lor întreg.
Jas 3:4  Iata ?i corabiile, de?i sunt atât de mari ?i împinse de vânturi aprige, sunt totu?i purtate de o cârma foarte mica încotro hotara?te vrerea cârmaciului.
Jas 3:5  A?a ?i limba: mic madular este, dar cu mari lucruri se fale?te! Iata pu?in foc ?i cât codru aprinde!
Jas 3:6  Foc este ?i limba, lume a faradelegii! Limba î?i are locul ei între madularele noastre, dar spurca tot trupul ?i arunca în foc drumul vie?ii, dupa ce aprinsa a fost ea de flacarile gheenei.
Jas 3:7  Pentru ca orice fel de fiare ?i de pasari, de târâtoare ?i de vieta?i din mare se domole?te ?i s-a domolit de firea omeneasca,
Jas 3:8  Dar limba, nimeni dintre oameni nu poate s-o domoleasca! Ea este un rau fara astâmpar; ea este plina de venin aducator de moarte.
Jas 3:9  Cu ea binecuvântam pe Dumnezeu ?i Tatal, ?i cu ea blestemam pe oameni, care sunt facu?i dupa asemanarea lui Dumnezeu.
Jas 3:10  Din aceea?i gura ies binecuvântarea ?i blestemul. Nu trebuie, fra?ii mei, sa fie acestea a?a.
Jas 3:11  Oare izvorul arunca din aceea?i vâna, ?i apa dulce ?i pe cea amara?
Jas 3:12  Nu cumva poate smochinul, fra?ilor, sa faca masline, sau vi?a de vie sa faca smochine? Tot a?a, izvorul sarat nu poate sa dea apa dulce.
Jas 3:13  Cine este, între voi, în?elept ?i priceput? Sa arate, din buna-i purtare, faptele lui, în blânde?ea în?elepciunii.
Jas 3:14  Iar daca ave?i râvnire amara ?i zavistie, în inimile voastre, nu va lauda?i, nici nu min?i?i împotriva adevarului.
Jas 3:15  În?elepciunea aceasta nu vine de sus, ci este pamânteasca, trupeasca, demonica.
Jas 3:16  Deci, unde este pizma ?i zavistie, acolo este neorânduiala ?i orice lucru rau.
Jas 3:17  Iar în?elepciunea cea de sus întâi este curata, apoi pa?nica, îngaduitoare, ascultatoare, plina de mila ?i de roade bune, neîndoielnica ?i nefa?arnica.
Jas 3:18  ?i roada drepta?ii se seamana întru pace de cei ce lucreaza pacea.
Jas 4:1  De unde vin razboaiele ?i de unde certurile dintre voi? Oare, nu de aici: din poftele voastre care se lupta în madularele voastre?
Jas 4:2  Pofti?i ?i nu ave?i; ucide?i ?i pizmui?i ?i nu pute?i dobândi ce dori?i; va sfatui?i ?i va razboi?i, ?i nu ave?i, pentru ca nu cere?i.
Jas 4:3  Cere?i ?i nu primi?i, pentru ca cere?i rau, ca voi sa risipi?i în placeri.
Jas 4:4  Preadesfrâna?ilor! Nu ?ti?i, oare, ca prietenia lumii este du?manie fa?a de Dumnezeu? Cine deci va voi sa fie prieten cu lumea se face vrajma? lui Dumnezeu.
Jas 4:5  Sau vi se pare ca Scriptura graie?te în de?ert? Duhul, care sala?luie?te în noi, ne pofte?te spre zavistie?
Jas 4:6  Nu, ci da mai mare har. Pentru aceea, zice: "Dumnezeu celor mândri le sta împotriva, iar celor smeri?i le da har".
Jas 4:7  Supune?i-va deci lui Dumnezeu. Sta?i împotriva diavolului ?i el va fugi de la voi.
Jas 4:8  Apropia?i-va de Dumnezeu ?i Se va apropia ?i El de voi. Cura?i?i-va mâinile, pacato?ilor, ?i sfin?i?i-va inimile, voi cei îndoielnici.
Jas 4:9  Patrunde?i-va de durere. Întrista?i-va ?i va jeli?i. Râsul întoarca-se în plâns ?i bucuria voastra în întristare.
Jas 4:10  Smeri?i-va înaintea Domnului ?i El va va înal?a.
Jas 4:11  Nu va grai?i de rau unul pe altul, fra?ilor. Cel ce graie?te de rau pe frate, ori judeca pe fratele sau, graie?te de rau legea ?i judeca legea; iar daca judeci legea nu e?ti împlinitor al legii, ci judecator.
Jas 4:12  Unul este Datatorul legii ?i Judecatorul: Cel ce poate sa mântuiasca ?i sa piarda. Iar tu cine e?ti, care judeci pe aproapele?
Jas 4:13  Veni?i acum cei care zice?i: Astazi sau mâine vom merge în cutare cetate, vom sta acolo un an ?i vom face nego? ?i vom câ?tiga,
Jas 4:14  Voi, care nu ?ti?i ce se va întâmpla mâine, ca ce este via?a voastra? Abur sunte?i, care se arata o clipa, apoi piere.
Jas 4:15  În loc ca voi sa zice?i: Daca Domnul voie?te, vom trai ?i vom face aceasta sau aceea.
Jas 4:16  ?i acum va lauda?i în trufia voastra. Orice lauda de acest fel este rea.
Jas 4:17  Drept aceea, cine ?tie sa faca ce e bine ?i nu face pacat are.
Jas 5:1  Veni?i acum, voi boga?ilor, plânge?i ?i va tângui?i de necazurile care vor sa vina asupra voastra.
Jas 5:2  Boga?ia voastra a putrezit ?i hainele voastre le-au mâncat moliile.
Jas 5:3  Aurul vostru ?i argintul au ruginit ?i rugina lor va fi marturie asupra voastra ?i ca focul va mistui trupurile voastre; a?i strâns comori în vremea din urma.
Jas 5:4  Dar, iata, plata lucratorilor care au secerat ?arinile voastre, pe care voi a?i oprit-o, striga; ?i strigatele seceratorilor au intrat în urechile Domnului Sabaot.
Jas 5:5  V-a?i desfatat pe pamânt ?i v-a?i dezmierdat; hranit-a?i inimile voastre în ziua înjunghierii.
Jas 5:6  Osândit-a?i, omorât-a?i pe cel drept; el nu vi se împotrive?te.
Jas 5:7  Drept aceea, fi?i îndelung-rabdatori, fra?ilor, pâna la venirea Domnului. Iata, plugarul a?teapta roada cea scumpa a pamântului, îndelung rabdând, pâna ce prime?te ploaia timpurie ?i târzie.
Jas 5:8  Fi?i, dar, ?i voi îndelung-rabdatori, întari?i inimile voastre, caci venirea Domnului s-a apropiat.
Jas 5:9  Nu va plânge?i, fra?ilor, unul împotriva celuilalt, ca sa nu fi?i judeca?i; iata judecatorul sta înaintea u?ilor.
Jas 5:10  Lua?i, fra?ilor, pilda de suferin?a ?i de îndelunga rabdare pe proorocii care au grait în numele Domnului.
Jas 5:11  Iata, noi fericim pe cei ce au rabdat: a?i auzit de rabdarea lui Iov ?i a?i vazut sfâr?itul harazit lui de Domnul; ca mult-milostiv este Domnul ?i îndurator.
Jas 5:12  Iar înainte de toate, fra?ii mei, sa nu va jura?i nici pe cer, nici pe pamânt, nici cu orice alt juramânt, ci sa va fie voua ce este da, da, ?i ce este nu, nu, ca sa nu cade?i sub judecata.
Jas 5:13  Este vreunul dintre voi în suferin?a? Sa se roage. Este cineva cu inima buna? Sa cânte psalmi.
Jas 5:14  Este cineva bolnav între voi? Sa cheme preo?ii Bisericii ?i sa se roage pentru el, ungându-l cu untdelemn, în numele Domnului.
Jas 5:15  ?i rugaciunea credin?ei va mântui pe cel bolnav ?i Domnul îl va ridica, ?i de va fi facut pacate se vor ierta lui.
Jas 5:16  Marturisi?i-va deci unul altuia pacatele ?i va ruga?i unul pentru altul, ca sa va vindeca?i, ca mult poate rugaciunea staruitoare a dreptului.
Jas 5:17  Ilie era om, cu slabiciuni asemenea noua, dar cu rugaciune s-a rugat ca sa nu ploua ?i nu a plouat trei ani ?i ?ase luni.
Jas 5:18  ?i iara?i s-a rugat ?i cerul a dat ploaie ?i pamântul a odraslit roada sa.
Jas 5:19  Fra?ii mei, daca vreunul va rataci de la adevar ?i-l va întoarce cineva,
Jas 5:20  Sa ?tie ca cel ce a întors pe pacatos de la ratacirea caii lui î?i va mântui sufletul din moarte ?i va acoperi mul?ime de pacate.


\end{document}