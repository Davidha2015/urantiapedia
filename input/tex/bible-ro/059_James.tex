\begin{document}

\title{Iacov}


\chapter{1}

\par 1 Iacov, robul lui Dumnezeu și al Domnului Iisus Hristos, celor douăsprezece seminții, care sunt în împrăștiere, salutare.
\par 2 Mare bucurie să socotiți, frații mei, când cădeți în felurite ispite,
\par 3 Știind că încercarea credinței voastre lucrează răbdarea;
\par 4 Iar răbdarea să-și aibă lucrul ei desăvârșit, ca să fiți desăvârșiți și întregi, nelipsiți fiind de nimic.
\par 5 Și de este cineva din voi lipsit de înțelepciune, să o ceară de la Dumnezeu, Cel ce dă tuturor fără deosebire și fără înfruntare; și i se va da.
\par 6 Să ceară însă cu credință, fără să aibă nici o îndoială, pentru că cine se îndoiește este asemenea valului mării, mișcat de vânt și aruncat încoace și încolo.
\par 7 Să nu gândească omul acela că va lua ceva de la Dumnezeu.
\par 8 Bărbatul îndoielnic este nestatornic în toate căile sale.
\par 9 Iar fratele cel smerit să se laude întru înălțimea sa,
\par 10 Și cel bogat întru smerenia sa, pentru că va trece ca floarea ierbii.
\par 11 Căci a răsărit soarele arzător și a uscat iarba și floarea ei a căzut și frumusețea feței ei a pierit; tot așa se va veșteji și bogatul în alergăturile sale.
\par 12 Fericit este bărbatul care rabdă ispita, căci lămurit făcându-se va lua cununa vieții, pe care a făgăduit-o Dumnezeu celor ce Îl iubesc pe El.
\par 13 Nimeni să nu zică, atunci când este ispitit: De la Dumnezeu sunt ispitit, pentru că Dumnezeu nu este ispitit de rele și El însuși nu ispitește pe nimeni.
\par 14 Ci fiecare este ispitit când este tras și momit de însăși pofta sa.
\par 15 Apoi pofta, zămislind, naște păcat, iar păcatul, odată săvârșit, aduce moarte.
\par 16 Nu vă înșelați, frații mei prea iubiți:
\par 17 Toată darea cea bună și tot darul desăvârșit de sus este, pogorându-se de la Părintele luminilor, la Care nu este schimbare sau umbră de mutare.
\par 18 După voia Sa ne-a născut prin cuvântul adevărului, ca să fim începătură făpturilor Lui.
\par 19 Să știți, iubiții mei frați: orice om să fie grabnic la ascultare, zăbavnic la vorbire, zăbavnic la mânie.
\par 20 Căci mânia omului nu lucrează dreptatea lui Dumnezeu.
\par 21 Pentru aceea, lepădând toată spurcăciunea și prisosința răutății, primiți cu blândețe cuvântul sădit în voi, care poate să mântuiască sufletele voastre.
\par 22 Dar faceți-vă împlinitori ai cuvântului, nu numai ascultători ai lui, amăgindu-vă pe voi înșivă.
\par 23 Căci dacă cineva este ascultător al cuvântului, iar nu și împlinitor, el seamănă cu omul care privește în oglindă fața firii sale;
\par 24 S-a privit pe sine și s-a dus și îndată a uitat ce fel era.
\par 25 Cine s-a uitat, însă, de aproape în legea cea desăvârșită a libertății și a stăruit în ea, făcându-se nu ascultător care uită, ci împlinitor al lucrului, acela fericit va fi în lucrarea sa.
\par 26 Dacă cineva socotește că e cucernic, dar nu își ține limba în frâu, ci își amăgește inima, cucernicia acestuia este zadarnică.
\par 27 Cucernicia curată și neîntinată înaintea lui Dumnezeu și Tatăl, aceasta este: să cercetăm pe orfani și pe văduve în necazurile lor, și să ne păzim pe noi fără de pată din partea lumii.

\chapter{2}

\par 1 Frații mei, nu căutând la fața omului să aveți credința în Domnul nostru Iisus Hristos, Domnul slavei.
\par 2 Căci, dacă va intra în adunarea voastră un om cu inele de aur în degete, în haină strălucită, și va intra și un sărac, în haină murdară,
\par 3 Iar voi puneți ochii pe cel care poartă haină strălucită și-i ziceți: Tu șezi bine aici, pe când săracului îi ziceți: Tu stai acolo, în picioare, sau: Șezi jos, la picioarele mele,
\par 4 N-ați făcut voi, oare, în gândul vostru, deosebire între unul și altul și nu v-ați făcut judecători cu socoteli viclene?
\par 5 Ascultați, iubiții mei frați: Au nu Dumnezeu i-a ales pe cei ce sunt săraci în ochii lumii, dar bogați în credință și moștenitori ai împărăției pe care a făgăduit-o El celor ce Îl iubesc?
\par 6 Iar voi ați necinstit pe cel sărac! Oare nu bogații vă asupresc pe voi și nu ei vă târăsc la judecăți?
\par 7 Nu sunt ei cei ce hulesc numele cel bun întru care ați fost chemați?
\par 8 Dacă, într-adevăr, împliniți legea împărătească, potrivit Scripturii: "Să iubești pe aproapele tău ca pe tine însuți", bine faceți;
\par 9 Iar de căutați la fața omului, faceți păcat și legea vă osândește ca pe niște călcători de lege.
\par 10 Pentru că cine va păzi toată legea, dar va greși într-o singură poruncă, s-a făcut vinovat față de toate poruncile.
\par 11 Căci Cel ce a zis: "Să nu săvârșești adulter", a zis și: "Să nu ucizi". Și dacă nu săvârșești adulter, dar ucizi, te-ai făcut călcător de lege.
\par 12 Așa să grăiți și așa să lucrați, ca unii care veți fi judecați prin legea libertății.
\par 13 Căci judecata este fără milă pentru cel care n-a făcut milă. Și mila biruiește în fața judecății.
\par 14 Ce folos, frații mei, dacă zice cineva că are credință, iar fapte nu are? Oare credința poate să-l mântuiască?
\par 15 Dacă un frate sau o soră sunt goi și lipsiți de hrana cea de toate zilele,
\par 16 Și cineva dintre voi le-ar zice: Mergeți în pace! Încălziți-vă și vă săturați, dar nu le dați cele trebuincioase trupului, care ar fi folosul?
\par 17 Așa și cu credința: dacă nu are fapte, e moartă în ea însăși.
\par 18 Dar va zice cineva: Tu ai credință, iar eu am fapte; arată-mi credința ta fără fapte și eu îți voi arăta, din faptele mele, credința mea.
\par 19 Tu crezi că unul este Dumnezeu? Bine faci; dar și demonii cred și se cutremură.
\par 20 Vrei însă să înțelegi, omule nesocotit, că credința fără de fapte moartă este?
\par 21 Avraam, părintele nostru, au nu din fapte s-a îndreptat, când a pus pe Isaac, fiul său, pe jertfelnic?
\par 22 Vezi că, credința lucra împreună cu faptele lui și din fapte credința s-a desăvârșit?
\par 23 Și s-a împlinit Scriptura care zice: "Și a crezut Avraam lui Dumnezeu și i s-a socotit lui ca dreptate" și "a fost numit prieten al lui Dumnezeu".
\par 24 Vedeți dar că din fapte este îndreptat omul, iar nu numai din credință.
\par 25 La fel și Rahav, desfrânata, au nu din fapte s-a îndreptat când a primit pe cei trimiși și i-a scos afară, pe altă cale?
\par 26 Căci precum trupul fără suflet mort este, astfel și credința fără de fapte, moartă este.

\chapter{3}

\par 1 Nu vă faceți voi mulți învățători, frații mei, știind că (noi, învățătorii) mai mare osândă vom lua.
\par 2 Pentru că toți greșim în multe chipuri; dacă nu greșește cineva în cuvânt, acela este bărbat desăvârșit, în stare să înfrâneze și tot trupul.
\par 3 Dar, dacă noi punem în gura cailor frâul, ca să ni-i supunem, ducem după noi și trupul lor întreg.
\par 4 Iată și corăbiile, deși sunt atât de mari și împinse de vânturi aprige, sunt totuși purtate de o cârmă foarte mică încotro hotărăște vrerea cârmaciului.
\par 5 Așa și limba: mic mădular este, dar cu mari lucruri se fălește! Iată puțin foc și cât codru aprinde!
\par 6 Foc este și limba, lume a fărădelegii! Limba își are locul ei între mădularele noastre, dar spurcă tot trupul și aruncă în foc drumul vieții, după ce aprinsă a fost ea de flăcările gheenei.
\par 7 Pentru că orice fel de fiare și de păsări, de târâtoare și de vietăți din mare se domolește și s-a domolit de firea omenească,
\par 8 Dar limba, nimeni dintre oameni nu poate s-o domolească! Ea este un rău fără astâmpăr; ea este plină de venin aducător de moarte.
\par 9 Cu ea binecuvântăm pe Dumnezeu și Tatăl, și cu ea blestemăm pe oameni, care sunt făcuți după asemănarea lui Dumnezeu.
\par 10 Din aceeași gură ies binecuvântarea și blestemul. Nu trebuie, frații mei, să fie acestea așa.
\par 11 Oare izvorul aruncă din aceeași vână, și apa dulce și pe cea amară?
\par 12 Nu cumva poate smochinul, fraților, să facă măsline, sau vița de vie să facă smochine? Tot așa, izvorul sărat nu poate să dea apă dulce.
\par 13 Cine este, între voi, înțelept și priceput? Să arate, din buna-i purtare, faptele lui, în blândețea înțelepciunii.
\par 14 Iar dacă aveți râvnire amară și zavistie, în inimile voastre, nu vă lăudați, nici nu mințiți împotriva adevărului.
\par 15 Înțelepciunea aceasta nu vine de sus, ci este pământească, trupească, demonică.
\par 16 Deci, unde este pizmă și zavistie, acolo este neorânduială și orice lucru rău.
\par 17 Iar înțelepciunea cea de sus întâi este curată, apoi pașnică, îngăduitoare, ascultătoare, plină de milă și de roade bune, neîndoielnică și nefățarnică.
\par 18 Și roada dreptății se seamănă întru pace de cei ce lucrează pacea.

\chapter{4}

\par 1 De unde vin războaiele și de unde certurile dintre voi? Oare, nu de aici: din poftele voastre care se luptă în mădularele voastre?
\par 2 Poftiți și nu aveți; ucideți și pizmuiți și nu puteți dobândi ce doriți; vă sfătuiți și vă războiți, și nu aveți, pentru că nu cereți.
\par 3 Cereți și nu primiți, pentru că cereți rău, ca voi să risipiți în plăceri.
\par 4 Preadesfrânaților! Nu știți, oare, că prietenia lumii este dușmănie față de Dumnezeu? Cine deci va voi să fie prieten cu lumea se face vrăjmaș lui Dumnezeu.
\par 5 Sau vi se pare că Scriptura grăiește în deșert? Duhul, care sălășluiește în noi, ne poftește spre zavistie?
\par 6 Nu, ci dă mai mare har. Pentru aceea, zice: "Dumnezeu celor mândri le stă împotrivă, iar celor smeriți le dă har".
\par 7 Supuneți-vă deci lui Dumnezeu. Stați împotriva diavolului și el va fugi de la voi.
\par 8 Apropiați-vă de Dumnezeu și Se va apropia și El de voi. Curățiți-vă mâinile, păcătoșilor, și sfințiți-vă inimile, voi cei îndoielnici.
\par 9 Pătrundeți-vă de durere. Întristați-vă și vă jeliți. Râsul întoarcă-se în plâns și bucuria voastră în întristare.
\par 10 Smeriți-vă înaintea Domnului și El vă va înălța.
\par 11 Nu vă grăiți de rău unul pe altul, fraților. Cel ce grăiește de rău pe frate, ori judecă pe fratele său, grăiește de rău legea și judecă legea; iar dacă judeci legea nu ești împlinitor al legii, ci judecător.
\par 12 Unul este Dătătorul legii și Judecătorul: Cel ce poate să mântuiască și să piardă. Iar tu cine ești, care judeci pe aproapele?
\par 13 Veniți acum cei care ziceți: Astăzi sau mâine vom merge în cutare cetate, vom sta acolo un an și vom face negoț și vom câștiga,
\par 14 Voi, care nu știți ce se va întâmpla mâine, că ce este viața voastră? Abur sunteți, care se arată o clipă, apoi piere.
\par 15 În loc ca voi să ziceți: Dacă Domnul voiește, vom trăi și vom face aceasta sau aceea.
\par 16 Și acum vă lăudați în trufia voastră. Orice laudă de acest fel este rea.
\par 17 Drept aceea, cine știe să facă ce e bine și nu face păcat are.

\chapter{5}

\par 1 Veniți acum, voi bogaților, plângeți și vă tânguiți de necazurile care vor să vină asupra voastră.
\par 2 Bogăția voastră a putrezit și hainele voastre le-au mâncat moliile.
\par 3 Aurul vostru și argintul au ruginit și rugina lor va fi mărturie asupra voastră și ca focul va mistui trupurile voastre; ați strâns comori în vremea din urmă.
\par 4 Dar, iată, plata lucrătorilor care au secerat țarinile voastre, pe care voi ați oprit-o, strigă; și strigătele secerătorilor au intrat în urechile Domnului Sabaot.
\par 5 V-ați desfătat pe pământ și v-ați dezmierdat; hrănit-ați inimile voastre în ziua înjunghierii.
\par 6 Osândit-ați, omorât-ați pe cel drept; el nu vi se împotrivește.
\par 7 Drept aceea, fiți îndelung-răbdători, fraților, până la venirea Domnului. Iată, plugarul așteaptă roada cea scumpă a pământului, îndelung răbdând, până ce primește ploaia timpurie și târzie.
\par 8 Fiți, dar, și voi îndelung-răbdători, întăriți inimile voastre, căci venirea Domnului s-a apropiat.
\par 9 Nu vă plângeți, fraților, unul împotriva celuilalt, ca să nu fiți judecați; iată judecătorul stă înaintea ușilor.
\par 10 Luați, fraților, pildă de suferință și de îndelungă răbdare pe proorocii care au grăit în numele Domnului.
\par 11 Iată, noi fericim pe cei ce au răbdat: ați auzit de răbdarea lui Iov și ați văzut sfârșitul hărăzit lui de Domnul; că mult-milostiv este Domnul și îndurător.
\par 12 Iar înainte de toate, frații mei, să nu vă jurați nici pe cer, nici pe pământ, nici cu orice alt jurământ, ci să vă fie vouă ce este da, da, și ce este nu, nu, ca să nu cădeți sub judecată.
\par 13 Este vreunul dintre voi în suferință? Să se roage. Este cineva cu inimă bună? Să cânte psalmi.
\par 14 Este cineva bolnav între voi? Să cheme preoții Bisericii și să se roage pentru el, ungându-l cu untdelemn, în numele Domnului.
\par 15 Și rugăciunea credinței va mântui pe cel bolnav și Domnul îl va ridica, și de va fi făcut păcate se vor ierta lui.
\par 16 Mărturisiți-vă deci unul altuia păcatele și vă rugați unul pentru altul, ca să vă vindecați, că mult poate rugăciunea stăruitoare a dreptului.
\par 17 Ilie era om, cu slăbiciuni asemenea nouă, dar cu rugăciune s-a rugat ca să nu plouă și nu a plouat trei ani și șase luni.
\par 18 Și iarăși s-a rugat și cerul a dat ploaie și pământul a odrăslit roada sa.
\par 19 Frații mei, dacă vreunul va rătăci de la adevăr și-l va întoarce cineva,
\par 20 Să știe că cel ce a întors pe păcătos de la rătăcirea căii lui își va mântui sufletul din moarte și va acoperi mulțime de păcate.


\end{document}