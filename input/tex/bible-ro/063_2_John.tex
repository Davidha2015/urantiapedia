\begin{document}

\title{2 Ioan}


\chapter{1}

\par 1 Preotul, catre aleasa Doamna ?i catre fiii ei, pe care ei îi iubesc întru adevar ?i nu numai eu, ci ?i to?i care au cunoscut adevarul,
\par 2 Pentru adevarul care ramâne în noi ?i va fi cu noi în veac:
\par 3 Har, mila, pace fie cu voi, de la Dumnezeu-Tatal ?i de la Iisus Hristos, Fiul Tatalui, în adevar ?i în dragoste.
\par 4 M-am bucurat foarte, ca am aflat pe unii din fiii tai umblând întru adevar, precum am primit porunca de la Tatal.
\par 5 ?i acum te rog, Doamna, nu ca ?i cum ?i-a? scrie porunca noua, ci pe aceea pe care noi o avem de la început, ca sa ne iubim unii pe al?ii.
\par 6 ?i aceasta este iubirea, ca sa umblam dupa poruncile Lui; aceasta este porunca, precum a?i auzit dintru început, ca sa umbla?i întru iubire.
\par 7 Pentru ca mul?i amagitori au ie?it în lume, care nu marturisesc ca Iisus Hristos a venit în trup; acesta este amagitorul ?i antihristul.
\par 8 Pazi?i-va pe voi în?iva, ca sa nu pierde?i ceea ce a?i lucrat, ci sa primi?i plata deplina.
\par 9 Oricine se abate ?i nu ramâne în înva?atura lui Hristos nu are pe Dumnezeu; cel ce ramâne în înva?atura Lui, acela are ?i pe Tatal ?i pe Fiul.
\par 10 Daca cineva vine la voi ?i nu aduce înva?atura aceasta, sa nu-l primi?i în casa ?i sa nu-i zice?i: Bun venit!
\par 11 Caci cel ce-i zice: Bun venit! se face parta? la faptele lui cele rele.
\par 12 Multe având a va scrie, n-am voit sa le scriu pe hârtie ?i cu cerneala, ci nadajduiesc sa vin la voi ?i sa vorbesc gura catre gura, ca bucuria noastra sa fie deplina.
\par 13 Te îmbra?i?eaza fiii surorii tale celei alese.


\end{document}