\begin{document}

\title{Iuda}


\chapter{1}

\par 1 Iuda, rob al lui Iisus Hristos ?i frate al lui Iacov, celor ce sunt chema?i, iubi?i în Dumnezeu-Tatal ?i pastra?i pentru Iisus Hristos:
\par 2 Mila voua ?i pace ?i iubirea sa se înmul?easca!
\par 3 Iubi?ilor, punând toata râvna sa va scriu despre mântuirea cea de ob?te, sim?it-am nevoie sa va scriu ?i sa va îndemn ca sa lupta?i pentru credin?a data sfin?ilor, odata pentru totdeauna.
\par 4 Caci s-au strecurat printre voi unii oameni nelegiui?i, care de mai înainte au fost rândui?i spre aceasta osânda, schimbând ei harul Dumnezeului nostru în desfrânare, ?i care tagaduiesc pe singurul nostru Stapân ?i Domn, pe Iisus Hristos.
\par 5 Voiesc dar sa va aduc aminte voua celor ce a?i ?tiut odata toate acestea ca Domnul, dupa ce a izbavit pe poporul Sau din pamântul Egiptului, a pierdut, dupa aceea, pe cei ce n-au crezut.
\par 6 Iar pe îngerii care nu ?i-au pazit vrednicia, ci au parasit loca?ul lor, i-a pus la pastrare sub întuneric, în lan?uri ve?nice, spre judecata zilei celei mari.
\par 7 Tot a?a, Sodoma ?i Gomora ?i ceta?ile dimprejurul lor care, în acela?i chip ca acestea, s-au dat la desfrânare ?i au umblat dupa trup strain, stau înainte ca pilda, suferind pedeapsa focului celui ve?nic.
\par 8 Asemenea deci ?i ace?tia, visând, pângaresc trupul, leapada stapânirea ?i hulesc maririle (cere?ti).
\par 9 Dar Mihail Arhanghelul, când se împotrivea diavolului, certându-se cu el pentru trupul lui Moise, n-a îndraznit sa aduca judecata de hula, ci a zis: "Sa te certe pe tine Domnul!"
\par 10 Ace?tia însa defaima cele ce nu cunosc, iar cele ce, - ca dobitoacele necuvântatoare, - ?tiu din fire, într-acestea î?i gasesc pieirea.
\par 11 Vai lor! Ca au umblat în calea lui Cain ?i, pentru plata, s-au dat cu totul în ratacirea lui Balaam ?i au pierit ca în razvratirea lui Core.
\par 12 Ace?tia sunt ca ni?te pete de necura?ie la mesele voastre ob?te?ti, ospatând fara sfiala împreuna cu voi, îmbuibându-se pe ei în?i?i, nori fara apa, purta?i de vânturi, pomi tomnatici fara roade, de doua ori usca?i ?i dezradacina?i,
\par 13 Valuri salbatice ale marii, care î?i spumega ru?inea lor, stele ratacitoare, carora întunericul întunericului li se pastreaza în ve?nicie.
\par 14 Dar ?i Enoh, al ?aptelea de la Adam, a proorocit despre ace?tia, zicând: Iata, a venit Domnul cu zecile de mii de sfin?i ai Lui,
\par 15 Ca sa faca judecata împotriva tuturor ?i sa mustre pe to?i nelegiui?ii de toate faptele nelegiuirii lor, în care au facut faradelege, ?i de toate cuvintele de ocara pe care ei, pacato?i, netematori de Dumnezeu, le-au rostit împotriva Lui.
\par 16 Ace?tia sunt cârtitori, nemul?umi?i cu starea lor, umblând dupa poftele lor ?i gura lor graie?te lucruri trufa?e, de?i, pentru folos, dau unor fe?e mare cinste.
\par 17 Voi, însa, iubi?ilor, aduce?i-va aminte de cuvintele zise mai dinainte de catre apostolii Domnului nostru Iisus Hristos,
\par 18 Ca ei va spuneau: În vremea de pe urma vor fi batjocoritori, umblând potrivit cu poftele lor nelegiuite.
\par 19 Ace?tia sunt cei ce fac dezbinari, (oameni) fire?ti, care nu au Duhul.
\par 20 Dar voi, iubi?ilor, zidi?i-va pe voi în?iva, întru a voastra prea sfânta credin?a, rugându-va în Duhul Sfânt.
\par 21 Pazi?i-va întru dragostea lui Dumnezeu ?i a?tepta?i mila Domnului nostru Iisus Hristos, spre via?a ve?nica.
\par 22 ?i pe unii, ?ovaitori, mustra?i-i,
\par 23 Pe al?ii, smulgându-i din foc, mântui?i-i; de al?ii, însa, fie-va mila cu frica, urând ?i cama?a spurcata de pe trupul lor.
\par 24 Iar Celui ce poate sa va pazeasca pe voi de orice cadere ?i sa va puna înaintea slavei Lui, neprihani?i cu bucurie mare,
\par 25 Singurului Dumnezeu, Mântuitorul nostru, prin Iisus Hristos, Domnul nostru, slava, preamarire, putere ?i stapânire, mai înainte de tot veacul ?i acum ?i întru to?i vecii. Amin!


\end{document}