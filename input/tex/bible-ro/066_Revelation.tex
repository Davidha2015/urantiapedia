\begin{document}

\title{Revelation}

Rev 1:1  Descoperirea lui Iisus Hristos, pe care I-a dat-o Dumnezeu, ca sa arate robilor Sai cele ce trebuie sa se petreaca în curând, iar El, prin trimiterea îngerului Sau, a destainuit-o robului Sau Ioan,
Rev 1:2  Care a marturisit cuvântul lui Dumnezeu ?i marturia lui Iisus Hristos, câte a vazut.
Rev 1:3  Fericit este cel ce cite?te ?i cei ce asculta cuvintele proorociei ?i pastreaza cele scrise în aceasta! Caci vremea este aproape.
Rev 1:4  Ioan, celor ?apte Biserici, care sunt în Asia: Har voua ?i pace de la Cel ce este ?i Cel ce era ?i Cel ce vine ?i de la cele ?apte duhuri, care sunt înaintea scaunului Lui,
Rev 1:5  ?i de la Iisus Hristos, Martorul cel credincios, Cel întâi nascut din mor?i, ?i Domnul împara?ilor pamântului. Lui, Care ne iube?te ?i ne-a dezlegat pe noi din pacatele noastre, prin sângele Sau,
Rev 1:6  ?i ne-a facut pe noi împara?ie, preo?i ai lui Dumnezeu ?i Tatal Sau, Lui fie slava ?i puterea, în vecii vecilor. Amin!
Rev 1:7  Iata, El vine cu norii ?i orice ochi Îl va vedea ?i-L vor vedea ?i cei ce L-au împuns ?i se vor jeli, din pricina Lui, toate semin?iile pamântului. A?a. Amin.
Rev 1:8  Eu sunt Alfa ?i Omega, zice Domnul Dumnezeu, Cel ce este, Cel ce era ?i Cel ce vine, Atot?iitorul.
Rev 1:9  Eu Ioan, fratele vostru ?i împreuna cu voi parta? la suferin?a ?i la împara?ia ?i la rabdarea în Iisus, fost-a în insula ce se cheama Patmos, pentru cuvântul lui Dumnezeu ?i pentru marturia lui Iisus.
Rev 1:10  Am fost în duh în zi de duminica ?i am auzit, în urma mea, glas mare de trâmbi?a,
Rev 1:11  Care zicea: Ceea ce vezi scrie în carte ?i trimite celor ?apte Biserici: la Efes, ?i la Smirna, ?i la Pergam, ?i la Tiatira, ?i la Sardes, ?i la Filadelfia, ?i la Laodiceea.
Rev 1:12  ?i m-am întors sa vad al cui este glasul care vorbea cu mine ?i, întorcându-ma, am vazut ?apte sfe?nice de aur.
Rev 1:13  ?i în mijlocul sfe?nicelor pe Cineva asemenea Fiului Omului, îmbracat în ve?mânt lung pâna la picioare ?i încins pe sub sân cu un brâu de aur.
Rev 1:14  Capul Lui ?i parul Lui erau albe ca lâna alba ?i ca zapada, ?i ochii Lui, ca para focului.
Rev 1:15  Picioarele Lui erau asemenea aramei arse în cuptor, iar glasul Lui era ca un vuiet de ape multe;
Rev 1:16  În mâna Lui cea dreapta avea ?apte stele; ?i din gura Lui ie?ea o sabie ascu?ita cu doua tai?uri, iar fa?a Lui era ca soarele, când straluce?te în puterea lui.
Rev 1:17  ?i când L-am vazut, am cazut la picioarele Lui ca un mort. ?i El a pus mâna dreapta peste mine, zicând: Nu te teme! Eu sunt Cel dintâi ?i Cel de pe urma,
Rev 1:18  ?i Cel ce sunt viu. Am fost mort, ?i, iata, sunt viu, în vecii vecilor, ?i am cheile mor?ii ?i ale iadului.
Rev 1:19  Scrie, deci, cele ce ai vazut ?i cele ce sunt ?i cele ce au sa fie dupa acestea.
Rev 1:20  Taina celor ?apte stele, pe care le-ai vazut în dreapta Mea, ?i a celor ?apte sfe?nice de aur este: Cele ?apte stele sunt îngerii celor ?apte Biserici, iar sfe?nicele cele ?apte sunt ?apte Biserici.
Rev 2:1  Scrie îngerului Bisericii din Efes: Acestea zice Cel ce ?ine cele ?apte stele în dreapta Sa, Cel care umbla în mijlocul celor ?apte sfe?nice de aur:
Rev 2:2  ?tiu faptele tale ?i osteneala ta ?i rabdarea ta ?i cum ca nu po?i suferi pe cei rai ?i ai cercat pe cei ce se zic pe sine apostoli ?i nu sunt ?i i-ai aflat mincino?i;
Rev 2:3  ?i starui în rabdare ?i ai suferit pentru numele Meu ?i nu ai obosit.
Rev 2:4  Dar am împotriva ta faptul ca ai parasit dragostea ta cea dintâi.
Rev 2:5  Drept aceea, adu-?i aminte de unde ai cazut ?i te pocaie?te ?i fa faptele de mai înainte; iar de nu, vin la tine curând ?i voi mi?ca sfe?nicul tau din locul lui, daca nu te vei pocai.
Rev 2:6  Ai însa partea buna ca ura?ti faptele nicolai?ilor, pe care le urasc ?i Eu.
Rev 2:7  Cine are urechi sa auda ceea ce Duhul zice Bisericilor: Celui ce va birui îi voi da sa manânce din pomul vie?ii, care este în raiul lui Dumnezeu.
Rev 2:8  Iar îngerului Bisericii din Smirna, scrie-i: Acestea zice Cel dintâi ?i Cel de pe urma, Cel care a murit ?i a înviat:
Rev 2:9  ?tiu necazul tau ?i saracia ta, dar e?ti bogat, ?i hula din partea celor ce zic despre ei în?i?i ca sunt iudei ?i nu sunt, ci sinagoga a satanei.
Rev 2:10  Nu te teme de cele ce ai sa patime?ti. Ca iata diavolul va sa arunce dintre voi în temni?a, ca sa fi?i ispiti?i, ?i ve?i avea necaz zece zile. Fii credincios pâna la moarte ?i î?i voi da cununa vie?ii.
Rev 2:11  Cine are urechi sa auda ceea ce Duhul zice Bisericilor: Cel ce biruie?te nu va fi vatamat de moartea cea de-a doua.
Rev 2:12  Iar îngerului Bisericii din Pergam scrie-i: Acestea zice Cel ce are sabia ascu?ita de amândoua par?ile:
Rev 2:13  ?tiu unde sala?luie?ti: unde este scaunul satanei; ?i ?ii numele Meu ?i n-ai tagaduit credin?a Mea, în zilele lui Antipa, martorul Meu cel credincios, care a fost ucis la voi, unde locuie?te satana.
Rev 2:14  Dar am împotriva ta câteva lucruri, ca ai acolo pe unii care ?in înva?atura lui Balaam, cel ce înva?a pe Balac sa puna piatra de poticneala înaintea fiilor lui Israel, ca sa manânce care jertfita idolilor ?i sa se dea desfrânarii.
Rev 2:15  Astfel ai ?i tu pe unii care, de asemenea, ?in înva?atura nicolai?ilor.
Rev 2:16  Pocaie?te-te deci, iar de nu, vin la tine curând ?i voi face cu ei razboi, cu sabia gurii Mele.
Rev 2:17  Cine are urechi sa auda ceea ce Duhul zice Bisericilor: Biruitorului îi voi da din mana cea ascunsa ?i-i voi da lui o pietricica alba ?i pe pietricica scris un nume nou, pe care nimeni nu-l ?tie, decât primitorul.
Rev 2:18  Iar îngerului Bisericii din Tiatira scrie-i: Acestea zice Fiul lui Dumnezeu, ai Carui ochi sunt ca para focului ?i picioarele asemenea aramei stralucitoare:
Rev 2:19  ?tiu faptele tale ?i dragostea ?i credin?a ?i slujirea ?i rabdarea ta ?i ?tiu ca faptele tale cele de pe urma sunt mai multe decât cele dintâi.
Rev 2:20  Dar am împotriva ta faptul ca la?i pe femeia Izabela, care se zice pe sine prooroci?a, de înva?a ?i amage?te pe robii Mei, ca sa faca desfrânari ?i sa manânce cele jertfite idolilor.
Rev 2:21  ?i i-am dat timp sa se pocaiasca ?i nu voie?te sa se pocaiasca de desfrânarea ei.
Rev 2:22  Iata, o arunc pe ea bolnava la pat ?i pe cei ce se desfrâneaza cu ea, în mare strâmtorare, daca nu se vor pocai de faptele lor.
Rev 2:23  ?i pe fiii ei cu moarte îi voi ucide ?i vor cunoa?te toate Bisericile ca Eu sunt Cel care cercetez ranunchii ?i inimile ?i voi da voua, fiecaruia, dupa faptele voastre.
Rev 2:24  Iar voua ?i celorlal?i din Tiatira câ?i nu au înva?atura aceasta, ca unii care n-au cunoscut adâncurile satanei, dupa cum spun ei, va zic: nu pun peste voi alta greutate.
Rev 2:25  Însa, ceea ce ave?i, ?ine?i pâna voi veni.
Rev 2:26  ?i celui ce biruie?te ?i celui ce paze?te pâna la capat faptele Mele, îi voi da lui stapânire peste neamuri.
Rev 2:27  ?i le va pastori pe ele cu toiag de fier ?i ca pe vasele olarului le va sfarâma, precum ?i eu am luat putere de la Tatal Meu.
Rev 2:28  ?i-i voi da lui steaua cea de diminea?a.
Rev 2:29  Cine are urechi sa auda ceea ce Duhul zice Bisericilor.
Rev 3:1  Iar îngerului Bisericii din Sardes scrie-i: Acestea zice Cel ce are cele ?apte duhuri ale lui Dumnezeu ?i cele ?apte stele: ?tiu faptele tale, ca ai nume, ca traie?ti, dar e?ti mort.
Rev 3:2  Privegheaza ?i întare?te ce a mai ramas ?i era sa moara. Caci n-am gasit faptele tale depline înaintea Dumnezeului Meu.
Rev 3:3  Drept aceea, adu-?i aminte cum ai primit ?i ai auzit ?i pastreaza ?i te pocaie?te. Iar de nu vei priveghea, voi veni ca un fur ?i nu vei ?ti în care ceas voi veni asupra ta.
Rev 3:4  Dar ai câ?iva oameni în Sardes, care nu ?i-au mânjit hainele lor, ci ei vor umbla cu Mine îmbraca?i în ve?minte albe, fiindca sunt vrednici.
Rev 3:5  Cel ce biruie?te va fi astfel îmbracat în ve?minte albe ?i nu voi ?terge deloc numele lui din cartea vie?ii ?i voi marturisi numele lui înaintea parintelui Meu ?i înaintea îngerilor Lui.
Rev 3:6  Cel ce are urechi sa auda ceea ce Duhul zice Bisericilor.
Rev 3:7  Iar îngerului Bisericii din Filadelfia scrie-i: Acestea zice Cel Sfânt, Cel Adevarat, Cel ce are cheia lui David, Cel ce deschide ?i nimeni nu va închide ?i închide ?i nimeni nu va deschide:
Rev 3:8  ?tiu faptele tale; iata, am lasat înaintea ta o u?a deschisa, pe care nimeni nu poate sa o închida, fiindca, de?i ai putere mica, tu ai pazit cuvântul Meu ?i nu ai tagaduit numele Meu.
Rev 3:9  Iata, î?i dau din sinagoga satanei, dintre cei care se zic pe sine ca sunt iudei ?i nu sunt, ci mint; iata, îi voi face sa vina ?i sa se închine înaintea picioarelor tale ?i vor cunoa?te ca te-am iubit.
Rev 3:10  Pentru ca ai pazit cuvântul rabdarii Mele, ?i Eu te voi pazi pe tine de ceasul ispitei ce va sa vina peste toata lumea, ca sa încerce pe cei ce locuiesc pe pamânt.
Rev 3:11  Vin curând; ?ine ce ai, ca nimeni sa nu ia cununa ta.
Rev 3:12  Pe cel ce biruie?te îl voi face stâlp în templul Dumnezeului Meu ?i afara nu va mai ie?i ?i voi scrie pe el numele Dumnezeului Meu ?i numele ceta?ii Dumnezeului Meu, - al noului Ierusalim, care se pogoara din cer, de la Dumnezeul Meu - ?i numele Meu cel nou.
Rev 3:13  Cine are urechi sa auda ceea ce Duhul zice Bisericilor.
Rev 3:14  Iar îngerului Bisericii din Laodiceea scrie-i: Acestea zice Cel ce este Amin, martorul cel credincios ?i adevarat, începutul zidirii lui Dumnezeu:
Rev 3:15  ?tiu faptele tale; ca nu e?ti nici rece, nici fierbinte. O, de ai fi rece sau fierbinte!
Rev 3:16  Astfel, fiindca e?ti caldicel - nici fierbinte, nici rece - am sa te vars din gura Mea.
Rev 3:17  Fiindca tu zici: Sunt bogat ?i m-am îmboga?it ?i de nimic nu am nevoie! ?i nu ?tii ca tu e?ti cel ticalos ?i vrednic de plâns, ?i sarac ?i orb ?i gol!
Rev 3:18  Te sfatuiesc sa cumperi de la Mine aur lamurit în foc, ca sa te îmboga?e?ti, ?i ve?minte albe ca sa te îmbraci ?i sa nu se dea pe fa?a ru?inea goliciunii tale, ?i alifie de ochi ca sa-?i ungi ochii ?i sa vezi.
Rev 3:19  Eu pe câ?i îi iubesc îi mustru ?i îi pedepsesc; sârguie?te dar ?i te pocaie?te.
Rev 3:20  Iata, stau la u?a ?i bat; de va auzi cineva glasul Meu ?i va deschide u?a, voi intra la el ?i voi cina cu el ?i el cu Mine.
Rev 3:21  Celui ce biruie?te îi voi da sa ?ada cu Mine pe scaunul Meu, precum ?i Eu am biruit ?i am ?ezut cu Tatal Meu pe scaunul Lui.
Rev 3:22  Cine are urechi sa auda ceea ce Duhul zice Bisericilor.
Rev 4:1  Dupa acestea, m-am uitat ?i iata o u?a era deschisa în cer ?i glasul cel dintâi - glasul ca de trâmbi?a, pe care l-am auzit vorbind cu mine - mi-a zis: Suie-te aici ?i î?i voi arata cele ce trebuie sa fie dupa acestea.
Rev 4:2  Îndata am fost în duh; ?i iata un tron era în cer ?i pe tron ?edea Cineva.
Rev 4:3  ?i Cel ce ?edea semana la vedere cu piatra de iasp ?i de sardiu, iar de jur împrejurul tronului era un curcubeu, cu înfa?i?area smaraldului.
Rev 4:4  ?i douazeci ?i patru de scaune înconjurau tronul ?i pe scaune douazeci ?i patru de batrâni, ?ezând, îmbraca?i în haine albe ?i purtând pe capetele lor cununi de aur.
Rev 4:5  ?i din tron ie?eau fulgere ?i glasuri ?i tunete; ?i ?apte faclii de foc ardeau înaintea tronului, care sunt cele ?apte duhuri ale lui Dumnezeu,
Rev 4:6  ?i înaintea tronului, ca o mare de sticla, asemenea cu cristalul. Iar în mijlocul tronului ?i împrejurul tronului patru fiin?e, pline de ochi, dinainte ?i dinapoi.
Rev 4:7  ?i fiin?a cea dintâi era asemenea leului, a doua fiin?a asemenea vi?elului, a treia fiin?a avea fa?a de om, iar a patra fiin?a era asemenea vulturului care zboara.
Rev 4:8  ?i cele patru fiin?e, având fiecare din ele câte ?ase aripi, sunt pline de ochi, de jur împrejur ?i pe dinauntru, ?i odihna nu au, ziua ?i noaptea, zicând: Sfânt, Sfânt, Sfânt, Domnul Dumnezeu, Atot?iitorul, Cel ce era ?i Cel ce este ?i Cel ce vine.
Rev 4:9  ?i când cele patru fiin?e dadeau slava, cinste ?i mul?umita Celui ce ?ade pe tron, Celui ce este viu în vecii vecilor,
Rev 4:10  Atunci cei douazeci ?i patru de batrâni, cazând înaintea Celui ce ?edea pe tron, se închinau Celui ce este viu în vecii vecilor ?i aruncau cununile lor înaintea tronului, zicând:
Rev 4:11  Vrednic e?ti, Doamne ?i Dumnezeul nostru, sa prime?ti slava ?i cinstea ?i puterea, caci Tu ai zidit toate lucrurile ?i prin voin?a Ta ele erau ?i s-au facut.
Rev 5:1  Am vazut apoi, în mâna dreapta a Celui ce ?edea pe tron, o carte scrisa înauntru ?i pe dos, pecetluita cu ?apte pece?i.
Rev 5:2  ?i am vazut un înger puternic, care striga cu glas mare: Cine este vrednic sa deschida cartea ?i sa desfaca toate pece?ile ei?
Rev 5:3  Dar nimeni în cer, nici pe pamânt, nici sub pamânt nu putea sa deschida cartea, nici sa se uite în ea.
Rev 5:4  ?i am plâns mult, fiindca nimeni n-a fost gasit vrednic sa deschida cartea, nici sa se uite în ea.
Rev 5:5  ?i unul dintre batrâni mi-a zis: Nu plânge. Iata, a biruit leul din semin?ia lui Iuda, radacina lui David, ca sa deschida cartea ?i cele ?apte pece?i ale ei.
Rev 5:6  ?i am vazut, la mijloc, între tron ?i cele patru fiin?e ?i în mijlocul batrânilor, stând un Miel, ca înjunghiat, ?i care avea ?apte coarne ?i ?apte ochi, care sunt cele ?apte duhuri ale lui Dumnezeu, trimise în tot pamântul.
Rev 5:7  ?i a venit ?i a luat cartea, din dreapta Celui ce ?edea pe tron.
Rev 5:8  ?i când a luat cartea, cele patru fiin?e ?i cei douazeci ?i patru de batrâni au cazut înaintea Mielului, având fiecare alauta ?i cupe de aur pline cu tamâie care sunt rugaciunile sfin?ilor.
Rev 5:9  ?i cântau o cântare noua, zicând: Vrednic e?ti sa iei cartea ?i sa deschizi pece?ile ei, fiindca ai fost înjunghiat ?i ai rascumparat lui Dumnezeu, cu sângele Tau, oameni din toata semin?ia ?i limba ?i poporul ?i neamul;
Rev 5:10  ?i I-ai facut Dumnezeului nostru împara?ie ?i preo?i, ?i vor împara?i pe pamânt.
Rev 5:11  ?i am vazut ?i am auzit glas de îngeri mul?i, de jur împrejurul tronului ?i al fiin?elor ?i al batrânilor, ?i era numarul lor zeci de mii de zeci de mii ?i mii de mii,
Rev 5:12  Zicând cu glas mare: Vrednic este Mielul cel înjunghiat ca sa ia puterea ?i boga?ia ?i în?elepciunea ?i taria ?i cinstea ?i slava ?i binecuvântarea.
Rev 5:13  ?i toata faptura care este în cer ?i pe pamânt ?i sub pamânt ?i în mare ?i toate câte sunt în acestea le-am auzit, zicând: Celui ce ?ade pe tron ?i Mielului fie binecuvântarea ?i cinstea ?i slava ?i puterea, în vecii vecilor!
Rev 5:14  ?i cele patru fiin?e ziceau: Amin! Iar batrânii cazura ?i se închinara.
Rev 6:1  ?i am vazut când Mielul a deschis pe cea dintâi din cele ?apte pece?i, ?i am auzit pe una din cele patru fiin?e zicând cu glas ca de tunet: Vino ?i vezi.
Rev 6:2  ?i m-am uitat ?i iata un cal alb ?i cel care ?edea pe el avea un arc; ?i i s-a dat lui cununa ?i a pornit ca un biruitor ca sa biruiasca.
Rev 6:3  ?i când a deschis pecetea a doua, am auzit, zicând, pe a doua fiin?a: Vino ?i vezi.
Rev 6:4  ?i a ie?it alt cal, ro?u ca focul; ?i celui ce ?edea pe el i s-a dat sa ia pacea de pe pamânt, ca oamenii sa se junghie între ei; ?i o sabie mare i s-a dat.
Rev 6:5  ?i când a deschis pecetea a treia, am auzit pe a treia fiin?a, zicând: Vino ?i vezi. ?i m-am uitat ?i iata un cal negru ?i cel care ?edea pe el avea un cântar în mâna lui.
Rev 6:6  ?i am auzit, în mijlocul celor patru fiin?e, ca un glas care zicea: Masura de grâu un dinar, ?i trei masuri de orz un dinar. Dar de untdelemn ?i de vin sa nu te atingi.
Rev 6:7  ?i când a deschis pecetea a patra, am auzit glasul fiin?ei a patra, zicând: Vino ?i vezi.
Rev 6:8  ?i m-am uitat ?i iata un cal galben-vânat ?i numele celui ce ?edea pe el era: Moartea; ?i iadul se ?inea dupa el; ?i li s-a dat lor putere peste a patra parte a pamântului, ca sa ucida cu sabie ?i cu foamete, ?i cu moarte ?i cu fiarele de pe pamânt.
Rev 6:9  ?i când a deschis pecetea a cincea, am vazut, sub jertfelnic, sufletele celor înjunghia?i pentru cuvântul lui Dumnezeu ?i pentru marturia pe care au dat-o.
Rev 6:10  ?i strigau cu glas mare ?i ziceau: Pâna când, Stapâne sfinte ?i adevarate, nu vei judeca ?i nu vei razbuna sângele nostru, fa?a de cei ce locuiesc pe pamânt?
Rev 6:11  ?i fiecaruia dintre ei i s-a dat câte un ve?mânt alb ?i li s-a spus ca sa stea în tihna, înca pu?ina vreme, pâna când vor împlini numarul ?i cei împreuna-slujitori cu ei ?i fra?ii lor, cei ce aveau sa fie omorâ?i ca ?i ei.
Rev 6:12  ?i m-am uitat când a deschis pecetea a ?asea ?i s-a facut cutremur mare, soarele s-a facut negru ca un sac de par ?i luna întreaga s-a facut ca sângele,
Rev 6:13  ?i stelele cerului au cazut pe pamânt, precum smochinul î?i leapada smochinele sale verzi, când este zguduit de vijelie.
Rev 6:14  Iar cerul s-a dat în laturi, ca o carte de piele pe care o faci sul ?i to?i mun?ii ?i toate insulele s-au mi?cat din locurile lor.
Rev 6:15  ?i împara?ii pamântului ?i domnii ?i capeteniile o?tilor ?i boga?ii ?i cei puternici ?i to?i robii ?i to?i slobozii s-au ascuns în pe?teri ?i în stâncile mun?ilor,
Rev 6:16  Strigând mun?ilor ?i stâncilor: Cade?i peste noi ?i ne ascunde?i pe noi de fa?a Celui ce ?ade pe tron ?i de mânia Mielului;
Rev 6:17  Ca a venit ziua cea mare a mâniei lor, ?i cine are putere ca sa stea pe loc?
Rev 7:1  Dupa aceasta am vazut patru îngeri, stând la cele patru unghiuri ale pamântului, ?inând cele patru vânturi ale pamântului, ca sa nu sufle vânt pe pamânt, nici peste mare, nici peste vreun copac.
Rev 7:2  ?i am vazut un alt înger care se ridica de la Rasaritul Soarelui ?i avea pecetea Viului Dumnezeu. Îngerul a strigat cu glas puternic catre cei patru îngeri, carora li s-a dat sa vatame pamântul ?i marea,
Rev 7:3  Zicând: Nu vatama?i pamântul, nici marea, nici copacii, pâna ce nu vom pecetlui, pe frun?ile lor, pe robii Dumnezeului nostru.
Rev 7:4  ?i am auzit numarul celor pecetlui?i: o suta patruzeci ?i patru de mii de pecetlui?i, din toate semin?iile fiilor lui Israel:
Rev 7:5  Din semin?ia lui Iuda, douasprezece mii de pecetlui?i; din semin?ia lui Ruben, douasprezece mii; din semin?ia lui Gad, douasprezece mii;
Rev 7:6  Din semin?ia lui A?er, douasprezece mii; din semin?ia lui Neftali, douasprezece mii; din semin?ia lui Manase, douasprezece mii;
Rev 7:7  Din semin?ia lui Simeon, douasprezece mii; din semin?ia lui Levi, douasprezece mii; din semin?ia lui Isahar, douasprezece mii;
Rev 7:8  Din semin?ia lui Zabulon, douasprezece mii; din semin?ia lui Iosif, douasprezece mii; din semin?ia Veniamin, douasprezece mii de pecetlui?i.
Rev 7:9  Dupa acestea, m-am uitat ?i iata mul?ime multa, pe care nimeni nu putea s-o numere, din tot neamul ?i semin?iile ?i popoarele ?i limbile, stând înaintea tronului ?i înaintea Mielului, îmbraca?i în ve?minte albe ?i având în mâna ramuri de finic.
Rev 7:10  ?i mul?imea striga cu glas mare, zicând: Mântuirea este de la Dumnezeul nostru, Care ?ade pe tron, ?i de la Mielul.
Rev 7:11  ?i to?i îngerii stateau împrejurul tronului batrânilor ?i al celor patru fiin?e, ?i au cazut înaintea tronului pe fe?ele lor ?i s-au închinat lui Dumnezeu,
Rev 7:12  Zicând: Amin! Binecuvântarea ?i slava ?i în?elepciunea ?i mul?umirea ?i cinstea ?i puterea ?i taria fie Dumnezeului nostru, în vecii vecilor. Amin!
Rev 7:13  Iar unul dintre batrâni a deschis gura ?i mi-a zis: Ace?tia care sunt îmbraca?i în ve?minte albe, cine sunt ?i de unde au venit?
Rev 7:14  ?i i-am zis: Doamne, Tu ?tii. El mi-a raspuns: Ace?tia sunt cei ce vin din strâmtorarea cea mare ?i ?i-au spalat ve?mintele lor ?i le-au facut albe în sângele Mielului.
Rev 7:15  Pentru aceea sunt înaintea tronului lui Dumnezeu, ?i Îi slujesc ziua ?i noaptea, în templul Lui, ?i Cel ce ?ade pe tron îi va adaposti în cortul Sau.
Rev 7:16  ?i nu vor mai flamânzi, nici nu vor mai înseta, nici nu va mai cadea soarele peste ei ?i nici o ar?i?a;
Rev 7:17  Caci Mielul, Cel ce sta în mijlocul tronului, îi va pa?te pe ei ?i-i va duce la izvoarele apelor vie?ii ?i Dumnezeu va ?terge orice lacrima din ochii lor.
Rev 8:1  ?i când Mielul a deschis pecetea a ?aptea, s-a facut tacere în cer, ca la o jumatate de ceas.
Rev 8:2  ?i am vazut pe cei ?apte îngeri, care stau înaintea lui Dumnezeu ?i li s-a dat lor ?apte trâmbi?e.
Rev 8:3  ?i a venit un alt înger ?i a stat la altar, având cadelni?a de aur, ?i i s-a dat lui tamâie multa, ca s-o aduca împreuna cu rugaciunile tuturor sfin?ilor, pe altarul de aur dinaintea tronului.
Rev 8:4  ?i fumul tamâiei s-a suit, din mâna îngerului, înaintea lui Dumnezeu, împreuna cu rugaciunile sfin?ilor.
Rev 8:5  ?i îngerul a luat cadelni?a ?i a umplut-o din focul altarului ?i a aruncat pe pamânt; ?i s-au pornit tunete ?i glasuri ?i fulgere ?i cutremur.
Rev 8:6  Iar cei ?apte îngeri, care aveau cele ?apte trâmbi?e, s-au gatit ca sa trâmbi?eze.
Rev 8:7  ?i a trâmbi?at întâiul înger, ?i s-a pornit grindina ?i foc amestecat cu sânge ?i au cazut pe pamânt; ?i a ars din pamânt a treia parte, ?i a ars din copaci a treia parte, iar iarba verde a ars de tot.
Rev 8:8  A trâmbi?at, apoi, al doilea înger, ?i ca un munte mare arzând în flacari s-a prabu?it în mare ?i a treia parte din mare s-a prefacut în sânge;
Rev 8:9  ?i a pierit a treia parte din fapturile cu via?a în ele, care sunt în mare, ?i a treia parte din corabii s-a sfarâmat.
Rev 8:10  ?i a trâmbi?at al treilea înger, ?i a cazut din cer o stea uria?a, arzând ca o faclie, ?i a cazut peste izvoarele apelor.
Rev 8:11  ?i numele stelei se cheama Absintos. ?i a treia parte din ape s-a facut ca pelinul ?i mul?i dintre oameni au murit din pricina apelor, pentru ca se facusera amare.
Rev 8:12  ?i a trâmbi?at al patrulea înger; ?i a fost lovita a treia parte din soare, ?i a treia parte din luna, ?i a treia parte din stele, ca sa fie întunecata a treia parte a lor ?i ziua sa-?i piarda din lumina a treia parte, ?i noaptea tot a?a.
Rev 8:13  ?i am vazut ?i am auzit un vultur, care zbura spre înaltul cerului ?i striga cu glas mare: Vai, vai, vai celor ce locuiesc pe pamânt, din pricina celorlalte glasuri ale trâmbi?ei celor trei îngeri, care sunt gata sa trâmbi?eze!
Rev 9:1  ?i a trâmbi?at al cincilea înger, ?i am vazut o stea cazuta din cer pe pamânt ?i i s-a dat cheia fântânii adâncului.
Rev 9:2  ?i a deschis fântâna adâncului ?i fum s-a ridicat din fântâna, ca fumul unui cuptor mare, ?i soarele ?i vazduhul s-au întunecat de fumul fântânii.
Rev 9:3  ?i din fum au ie?it lacuste pe pamânt ?i li s-a dat lor putere precum au putere scorpiile pamântului.
Rev 9:4  ?i li s-a poruncit sa nu vatame iarba pamântului ?i nici o verdea?a ?i nici un copac, fara numai pe oamenii care nu au pecetea lui Dumnezeu pe frun?ile lor.
Rev 9:5  ?i nu li s-a dat ca sa-i omoare, ci ca sa fie chinui?i cinci luni; ?i chinul lor este la fel cu chinul scorpiei, când a în?epat pe om.
Rev 9:6  ?i în zilele acelea vor cauta oamenii moartea ?i nu o vor afla ?i vor dori sa moara; moartea însa va fugi de ei.
Rev 9:7  Iar înfa?i?area lacustelor era asemenea unor cai pregati?i de razboi. Pe capete aveau cununi ca de aur, ?i fe?ele lor erau ca ni?te fe?e de oameni.
Rev 9:8  ?i aveau par ca parul de femei ?i din?ii lor erau ca din?ii leilor.
Rev 9:9  ?i aveau plato?e ca plato?ele de fier, iar vuietul aripilor era la fel cu vuietul unei mul?imi de care ?i de cai, care alearga la lupta.
Rev 9:10  ?i aveau cozi ?i bolduri asemenea scorpiilor; ?i puterea lor e în cozile lor, ca sa vatame pe oameni cinci luni.
Rev 9:11  ?i au ca împarat al lor pe îngerul adâncului, al carui nume, în evreie?te, este Abaddon, iar în eline?te are numele Apollion.
Rev 9:12  Întâiul "vai" a trecut; iata vine înca un "vai" ?i înca unul, dupa acestea.
Rev 9:13  ?i a trâmbi?at al ?aselea înger. ?i am auzit un glas, din cele patru cornuri ale altarului de aur, care este înaintea lui Dumnezeu,
Rev 9:14  Zicând catre îngerul al ?aselea, cel ce avea trâmbi?a: Dezleaga pe cei patru îngeri care sunt lega?i la râul cel mare, Eufratul.
Rev 9:15  ?i au fost dezlega?i cei patru îngeri, care erau gati?i spre ceasul ?i ziua ?i luna ?i anul acela, ca sa omoare a treia parte din oameni.
Rev 9:16  ?i numarul o?tilor era de douazeci de mii de ori câte zece mii de calare?i, caci am auzit numarul lor.
Rev 9:17  ?i a?a am vazut, în vedenie, caii ?i pe cei ce ?edeau pe ei, având plato?e ca de foc ?i de iachint ?i de pucioasa; iar capetele cailor semanau cu capetele leilor ?i din gurile lor ie?ea foc ?i fum ?i pucioasa.
Rev 9:18  De aceste trei plagi: de focul ?i de fumul ?i de pucioasa, care ie?ea din gurile lor, a fost ucisa a treia parte din oameni.
Rev 9:19  Pentru ca puterea cailor este în gura lor ?i în cozile lor; caci cozile lor sunt asemenea ?erpilor, având capete, ?i cu acestea vatama.
Rev 9:20  Dar ceilal?i oameni care nu au murit de plagile acestea, nu s-au pocait de faptele mâinilor lor, ca sa nu se mai închine idolilor de aur ?i de argint ?i de arama ?i de piatra ?i de lemn, care nu pot nici sa vada, nici sa auda, nici sa umble.
Rev 9:21  ?i nu s-au pocait de uciderile lor, nici de fermecatoriile lor, nici de desfrânarea lor, nici de furti?agurile lor.
Rev 10:1  ?i am vazut alt înger puternic, pogorându-se din cer, învaluit într-un nor ?i pe capul lui era curcubeul, iar fa?a lui stralucea ca soarele ?i picioarele lui erau ca ni?te stâlpi de foc,
Rev 10:2  ?i în mâna avea o carte mica, deschisa. ?i a pus piciorul lui cel drept pe mare, iar pe cel stâng pe pamânt,
Rev 10:3  ?i a strigat cu glas puternic, precum racne?te leul. Iar când a strigat, cele ?apte tunete au slobozit glasurile lor.
Rev 10:4  ?i când au vorbit cele ?apte tunete, voiam sa scriu, dar am auzit o voce care zicea din cer: Pecetluie?te cele ce au spus cele ?apte tunete ?i nu le scrie.
Rev 10:5  Iar îngerul pe care l-am vazut stând pe mare ?i pe pamânt, ?i-a ridicat mâna dreapta catre cer,
Rev 10:6  ?i s-a jurat pe Cel ce este viu în vecii vecilor, Care a facut cerul ?i cele ce sunt în cer ?i pamântul ?i cele ce sunt pe pamânt ?i marea ?i cele ce sunt în mare, ca timp nu va mai fi,
Rev 10:7  Ci, în zilele când va grai al ?aptelea înger - când va fi sa trâmbi?eze - atunci va fi savâr?ita taina lui Dumnezeu, precum bine a vestit robilor Sai, proorocilor.
Rev 10:8  Iar glasul din cer, pe care-l auzisem, iara?i a vorbit cu mine, zicând: Mergi de ia cartea cea deschisa din mâna îngerului, care sta pe mare ?i pe pamânt.
Rev 10:9  ?i m-am dus la înger ?i i-am zis sa-mi dea cartea. ?i mi-a raspuns: Ia-o ?i manânc-o ?i va amarî pântecele tau, dar în gura ta va fi dulce ca mierea.
Rev 10:10  Atunci am luat cartea din mâna îngerului ?i am mâncat-o; ?i era în gura mea dulce ca mierea, dar, dupa ce-am mâncat-o pântecele meu s-a amarât.
Rev 10:11  ?i apoi mi-a zis: Tu trebuie sa prooroce?ti, înca o data, la popoare ?i la neamuri ?i la limbi ?i la mul?i împara?i.
Rev 11:1  Apoi mi-au dat o trestie, asemenea unui toiag, zicând: Scoala-te ?i masoara templul lui Dumnezeu ?i altarul ?i pe cei ce se închina în el.
Rev 11:2  Iar curtea cea din afara a templului, scoate-o din socoteala ?i n-o masura, pentru ca a fost data neamurilor, care vor calca în picioare cetatea sfânta patruzeci ?i doua de luni.
Rev 11:3  ?i voi da putere celor doi martori ai mei ?i vor prooroci, îmbraca?i în sac, o mie doua sute ?i ?aizeci de zile.
Rev 11:4  Ace?tia sunt cei doi maslini ?i cele doua sfe?nice care stau înaintea Domnului pamântului.
Rev 11:5  ?i daca voie?te cineva sa-i vatame, foc iese din gura lor ?i mistuie?te pe vrajma?ii lor; ?i daca ar voi cineva sa-i vatame, acela trebuie ucis.
Rev 11:6  Ace?tia au putere sa închida cerul, ca ploaia sa nu ploua în zilele proorociei lor, ?i putere au peste ape sa le schimbe în sânge ?i sa bata pamântul cu orice fel de urgie, ori de câte ori vor voi.
Rev 11:7  Iar când vor ispravi cu marturia lor, fiara care se ridica din adânc va face razboi cu ei, ?i-i va birui ?i-i va omorî.
Rev 11:8  ?i trupurile lor vor zacea pe uli?ele ceta?ii celei mari, care se cheama, duhovnice?te, Sodoma ?i Egipt, unde a fost rastignit ?i Domnul lor.
Rev 11:9  ?i din popoare, din semin?ii, din limbi ?i din neamuri vor privi la trupurile lor trei zile ?i jumatate ?i nu vor îngadui ca ele sa fie puse în mormânt.
Rev 11:10  Iar locuitorii de pe pamânt se vor bucura de moartea lor ?i vor fi în veselie ?i î?i vor trimite daruri unul altuia, pentru ca ace?ti doi prooroci au chinuit pe locuitorii de pe pamânt.
Rev 11:11  ?i dupa cele trei zile ?i jumatate, duh de via?a de la Dumnezeu a intrat în ei ?i s-au ridicat pe picioarele lor ?i frica mare a cazut peste cei ce se uitau la ei.
Rev 11:12  ?i din cer au auzit glas puternic, zicându-le: Sui?i-va aici! ?i s-au suit la cer, în nori, ?i au privit la ei du?manii lor.
Rev 11:13  ?i în ceasul acela s-a facut cutremur mare ?i a zecea parte din cetate s-a prabu?it ?i au pierit în cutremur ?apte mii de oameni, iar ceilal?i s-au înfrico?at ?i au dat slava Dumnezeului cerului.
Rev 11:14  Al doilea "vai" a trecut; al treilea "vai", iata, vine degraba.
Rev 11:15  ?i a trâmbi?at al ?aptelea înger ?i s-au pornit, în cer, glasuri puternice care ziceau: Împara?ia lumii a ajuns a Domnului nostru ?i a Hristosului Sau ?i va împara?i în vecii vecilor.
Rev 11:16  ?i cei douazeci ?i patru de batrâni, care ?ed înaintea lui Dumnezeu pe scaunele lor, au cazut cu fe?ele la pamânt ?i s-au închinat lui Dumnezeu,
Rev 11:17  Zicând: Mul?umim ?ie, Doamne Dumnezeule, Atot?iitorule, Cel ce e?ti ?i Cel ce erai ?i Cel ce vii, ca ai luat puterea Ta cea mare ?i împara?e?ti.
Rev 11:18  ?i neamurile s-au mâniat, dar a venit mânia Ta ?i vremea celor mor?i, ca sa fie judeca?i, ?i sa rasplate?ti pe robii Tai, pe prooroci ?i pe sfin?i ?i pe cei ce se tem de numele Tau, pe cei mici ?i pe cei mari, ?i sa pierzi pe cei ce prapadesc pamântul.
Rev 11:19  ?i s-a deschis templul lui Dumnezeu, cel din cer, ?i s-a vazut în templul Lui chivotul legamântului Sau, ?i au fost fulgere ?i vuiete ?i tunete ?i cutremur ?i grindina mare.
Rev 12:1  ?i s-a aratat din cer un semn mare: o femeie înve?mântata cu soarele ?i luna era sub picioarele ei ?i pe cap purta cununa din douasprezece stele.
Rev 12:2  ?i era însarcinata ?i striga, chinuindu-se ?i muncindu-se ca sa nasca.
Rev 12:3  ?i alt semn s-a aratat în cer: iata un balaur mare, ro?u, având ?apte capete ?i zece coarne, ?i pe capetele lui, ?apte cununi împarate?ti.
Rev 12:4  Iar coada lui târa a treia parte din stelele cerului ?i le-a aruncat pe pamânt. ?i balaurul statu înaintea femeii, care era sa nasca, pentru ca sa înghita copilul, când se va na?te.
Rev 12:5  ?i a nascut un copil de parte barbateasca, care avea sa pastoreasca toate neamurile cu toiag de fier. ?i copilul ei fu rapit la Dumnezeu ?i la tronul Lui,
Rev 12:6  Iar femeia a fugit în pustie, unde are loc gatit de Dumnezeu, ca sa o hraneasca pe ea, acolo, o mie doua sute ?i ?aizeci de zile.
Rev 12:7  ?i s-a facut razboi în cer: Mihail ?i îngerii lui au pornit razboi cu balaurul. ?i se razboia ?i balaurul ?i îngerii lui.
Rev 12:8  ?i n-a izbutit el, nici nu s-a mai gasit pentru ei loc în cer.
Rev 12:9  ?i a fost aruncat balaurul cel mare, ?arpele de demult, care se cheama diavol ?i satana, cel ce în?eala pe toata lumea, aruncat a fost pe pamânt ?i îngerii lui au fost arunca?i cu el.
Rev 12:10  ?i am auzit glas mare, în cer, zicând: Acum s-a facut mântuirea ?i puterea ?i împara?ia Dumnezeului nostru ?i stapânirea Hristosului Sau, caci aruncat a fost pârâ?ul fra?ilor no?tri, cel ce îi pâra pe ei înaintea Dumnezeului nostru, ziua ?i noaptea.
Rev 12:11  ?i ei l-au biruit prin sângele Mielului ?i prin cuvântul marturiei lor ?i nu ?i-au iubit sufletul lor, pâna la moarte.
Rev 12:12  Pentru aceasta, bucura?i-va ceruri ?i cei ce locui?i în ele. Vai voua, pamântule ?i mare, fiindca diavolul a coborât la voi având mânie mare, caci ?tie ca timpul lui e scurt.
Rev 12:13  Iar când a vazut balaurul ca a fost aruncat pe pamânt, a prigonit pe femeia care nascuse pruncul.
Rev 12:14  ?i femeii i s-au dat cele doua aripi ale marelui vultur, ca sa zboare în pustie, la locul ei, unde e hranita acolo o vreme ?i vremuri ?i jumatate de vreme, departe de fa?a ?arpelui.
Rev 12:15  ?i ?arpele a aruncat din gura lui, dupa femeie, apa ca un râu ca s-o ia apa.
Rev 12:16  ?i pamântul i-a venit femeii într-ajutor, caci pamântul ?i-a deschis gura sa ?i a înghi?it râul pe care-l aruncase balaurul, din gura.
Rev 12:17  ?i balaurul s-a aprins de mânie asupra femeii ?i a pornit sa faca razboi cu ceilal?i din semin?ia ei, care pazesc poruncile lui Dumnezeu ?i ?in marturia lui Iisus.
Rev 13:1  ?i a stat pe nisipul marii. ?i am vazut ridicându-se din mare o fiara, care avea zece coarne ?i ?apte capete ?i pe coarnele ei zece cununi împarate?ti ?i pe capetele ei: nume de hula.
Rev 13:2  ?i fiara pe care am vazut-o era asemenea leopardului, picioarele ei erau ca ale ursului, iar gura ei ca o gura de leu. ?i balaurul i-a dat ei puterea lui ?i scaunul lui ?i stapânire mare.
Rev 13:3  ?i unul din capetele fiarei era ca înjunghiat de moarte, dar rana ei cea de moarte fu vindecata ?i tot pamântul s-a minunat mergând dupa fiara.
Rev 13:4  ?i s-au închinat balaurului, fiindca i-a dat fiarei stapânirea; ?i s-au închinat fiarei, zicând: Cine este asemenea fiarei ?i cine poate sa se lupte cu ea?
Rev 13:5  ?i i s-a dat ei gura sa graiasca seme?ii ?i hule ?i i s-a dat putere sa lucreze timp de patruzeci ?i doua de luni.
Rev 13:6  ?i ?i-a deschis gura sa spre hula lui Dumnezeu, ca sa huleasca numele Lui ?i cortul Lui ?i pe cei ce locuiesc în cer.
Rev 13:7  ?i i s-a dat sa faca razboi cu sfin?ii ?i sa-i biruiasca ?i i s-a dat ei stapânire peste toata semin?ia ?i poporul ?i limba ?i neamul.
Rev 13:8  ?i i se vor închina ei to?i cei ce locuiesc pe pamânt, ale caror nume nu sunt scrise, de la întemeierea lumii, în cartea vie?ii Mielului celui înjunghiat.
Rev 13:9  Daca are cineva urechi - sa auda!
Rev 13:10  Cine duce în robie de robie are parte; cine cu sabia va ucide trebuie sa fie ucis de sabie. Aici este rabdarea ?i credin?a sfin?ilor.
Rev 13:11  ?i am vazut o alta fiara, ridicându-se din pamânt, ?i avea doua coarne asemenea mielului, dar graia ca un balaur
Rev 13:12  ?i toata stapânirea celei dintâi fiare ea o pune în lucrare, în fa?a ei. ?i face pamântul ?i pe locuitorii de pe el sa se închine fiarei celei dintâi, a carei rana de moarte fusese vindecata.
Rev 13:13  ?i face semne mari, încât ?i foc face sa se pogoare din cer, pe pamânt, înaintea oamenilor,
Rev 13:14  ?i amage?te pe cei ce locuiesc pe pamânt prin semnele ce i s-au dat sa faca înaintea fiarei, zicând celor ce locuiesc pe pamânt sa faca un chip fiarei care a fost ranita cu sabia ?i a ramas în via?a.
Rev 13:15  ?i i s-a dat ei sa insufle duh chipului fiarei, ca chipul fiarei sa ?i graiasca ?i sa omoare pe to?i câ?i nu se vor închina chipului fiarei.
Rev 13:16  ?i ea îi sile?te pe to?i, pe cei mici ?i pe cei mari, ?i pe cei boga?i ?i pe cei saraci, ?i pe cei slobozi ?i pe cei robi, ca sa-?i puna semn pe mâna lor cea dreapta sau pe frunte.
Rev 13:17  Încât nimeni sa nu poata cumpara sau vinde, decât numai cel ce are semnul, adica numele fiarei, sau numarul numelui fiarei.
Rev 13:18  Aici este în?elepciunea. Cine are pricepere sa socoteasca numarul fiarei; caci este numar de om. ?i numarul ei este ?ase sute ?aizeci ?i ?ase.
Rev 14:1  ?i m-am uitat ?i iata Mielul statea pe muntele Sion ?i cu El o suta patruzeci ?i patru de mii, care aveau numele Lui ?i numele Tatalui Lui, scris pe frun?ile lor.
Rev 14:2  Atunci am auzit un glas din cer, ca un vuiet de ape multe ?i ca bubuitul unui tunet puternic, iar glasul pe care l-am auzit ca glasul celor ce cânta cu alautele lor.
Rev 14:3  ?i cântau o cântare noua, înaintea tronului ?i înaintea celor patru fiin?e ?i înaintea batrânilor; ?i nimeni nu putea sa înve?e cântarea decât numai cei o suta patruzeci ?i patru de mii, care fusesera rascumpara?i de pe pamânt.
Rev 14:4  Ace?tia sunt care nu s-au întinat cu femei, caci sunt feciorelnici. Ace?tia sunt care merg dupa Miel ori unde se va duce. Ace?tia au fost rascumpara?i dintre oameni, pârga lui Dumnezeu ?i Mielului.
Rev 14:5  Iar în gura lor nu s-a aflat minciuna, fiindca sunt fara prihana.
Rev 14:6  ?i am vazut apoi alt înger, care zbura prin mijlocul cerului, având sa binevesteasca Evanghelia ve?nica celor ce locuiesc pe pamânt ?i la tot neamul ?i semin?ia ?i limba ?i poporul,
Rev 14:7  Zicând cu glas puternic: Teme?i-va de Dumnezeu ?i da?i Lui slava, ca a venit ceasul judeca?ii Lui, ?i va închina?i Celui ce a facut cerul ?i pamântul ?i marea ?i izvoarele apelor.
Rev 14:8  ?i un al doilea înger a venit, zicând: A cazut, a cazut Babilonul, cetatea cea mare, care a adapat toate neamurile din vinul furiei desfrânarii sale.
Rev 14:9  ?i al treilea înger a venit dupa ei, strigând cu glas puternic: Cine se închina fiarei ?i chipului ei ?i prime?te semnul ei pe fruntea lui, sau pe mâna lui,
Rev 14:10  Va bea ?i el din vinul aprinderii lui Dumnezeu, turnat neamestecat, în potirul mâniei Sale, ?i se va chinui în foc ?i în pucioasa, înaintea sfin?ilor îngeri ?i înaintea Mielului.
Rev 14:11  ?i fumul chinului lor se siue în vecii vecilor. ?i nu au odihna nici ziua nici noaptea cei ce se închina fiarei ?i chipului ei ?i oricine prime?te semnul numelui ei.
Rev 14:12  Aici este rabdarea sfin?ilor, care pazesc poruncile lui Dumnezeu ?i credin?a lui Iisus.
Rev 14:13  ?i am auzit un glas din cer, zicând: Scrie: Ferici?i cei mor?i, cei ce acum mor întru Domnul! Da, graie?te Duhul, odihneasca-se de ostenelile lor, caci faptele lor vin cu ei,
Rev 14:14  ?i am privit ?i iata un nor alb ?i Cel ce ?edea pe nor era asemenea Fiului Omului, având pe cap cununa de aur ?i în mâna secera ascu?ita.
Rev 14:15  ?i iata un alt înger a ie?it din templu, strigând cu glas mare Celui ce ?edea pe nor: Trimite secera ?i secera, ca a venit ceasul de secerat, fiindca s-a copt seceri?ul pamântului.
Rev 14:16  ?i Cel ce ?edea pe nor a aruncat pe pamânt secera lui ?i pamântul a fost secerat.
Rev 14:17  ?i un alt înger a ie?it din templul cel ceresc, având ?i el un cu?ita? ascu?it.
Rev 14:18  ?i înca un înger a ie?it din altar, având putere asupra focului, ?i a strigat cu glas mare celui care avea cu?ita?ul ascu?it, zicând: Trimite cu?ita?ul tau cel ascu?it ?i culege ciorchinii viei pamântului, caci s-au copt.
Rev 14:19  ?i îngerul a aruncat, pe pamânt, cu?ita?ul lui ?i a cules via pamântului ?i strugurii i-a aruncat în teascul cel mare al mâniei lui Dumnezeu.
Rev 14:20  ?i teascul a fost calcat afara din cetate ?i a ie?it sânge din teasc, pâna la zabalele cailor, pe o întindere de o mie ?ase sute de stadii.
Rev 15:1  Am vazut, apoi, în cer, alt semn, mare ?i minunat: ?apte îngeri având ?apte pedepse - cele de pe urma - caci cu ele s-a sfâr?it mânia lui Dumnezeu.
Rev 15:2  ?i am vazut ca o mare de cristal, amestecata cu foc, ?i pe biruitorii fiarei ?i ai chipului ei ?i ai numarului numelui ei, stând în picioare pe marea de cristal ?i având alautele lui Dumnezeu.
Rev 15:3  ?i ei cântau cântarea lui Moise, robul lui Dumnezeu, ?i cântarea Mielului, zicând: Mari ?i minunate sunt lucrurile Tale, Doamne Dumnezeule, Atot?iitorule! Drepte ?i adevarate sunt caile Tale, Împarate al neamurilor!
Rev 15:4  Cine nu se va teme de Tine, Doamne, ?i nu va slavi numele Tau? Ca tu singur e?ti sfânt ?i toate neamurile vor veni ?i se vor închina înaintea Ta, pentru ca judeca?ile Tale s-au facut cunoscute.
Rev 15:5  ?i dupa aceasta, m-am uitat ?i s-a deschis templul cortului marturiei din cer.
Rev 15:6  ?i au ie?it din templu cei ?apte îngeri cu cele ?apte pedepse, îmbraca?i în ve?mânt de in curat, luminos, ?i încin?i, pe la piept, cu cingatori de aur.
Rev 15:7  ?i una din cele patru fapturi dadu celor ?apte îngeri cele ?apte cupe de aur pline de mânia lui Dumnezeu, Cel ce este viu în vecii vecilor.
Rev 15:8  Iar templul se umplu de fum, din slava lui Dumnezeu ?i din puterea Lui, ?i nimeni nu putea sa intre în templu, pâna ce se vor sfâr?i cele ?apte urgii ale celor ?apte îngeri.
Rev 16:1  ?i am auzit glas mare, din templu, zicând celor ?apte îngeri: Duce?i-va ?i varsa?i pe pamânt cele ?apte cupe ale mâniei lui Dumnezeu.
Rev 16:2  ?i s-a dus cel dintâi ?i a varsat cupa lui pe pamânt. ?i o buba rea ?i ucigatoare s-a ivit pe oamenii care aveau semnul fiarei ?i care se închinau chipului fiarei.
Rev 16:3  ?i al doilea înger a varsat cupa lui în mare, ?i marea s-a prefacut în sânge ca de mort, ?i orice suflare de via?a a murit, din cele ce sunt în mare.
Rev 16:4  Iar cel de al treilea a varsat cupa lui în râuri ?i în izvoarele apelor ?i s-au prefacut în sânge.
Rev 16:5  ?i am auzit pe îngerul apelor, zicând: Drept e?ti Tu, Cel ce e?ti ?i Cel ce erai, Cel Sfânt, ca ai judecat acestea:
Rev 16:6  Fiindca au varsat sângele sfin?ilor ?i al proorocilor, tot sânge le-ai dat sa bea. Vrednici sunt!
Rev 16:7  ?i am auzit din altar, graind: Da, Doamne Dumnezeule, Atot?iitorule, adevarate ?i drepte sunt judeca?ile Tale!
Rev 16:8  ?i al patrulea înger a varsat cupa lui în soare ?i i s-a dat sa dogoreasca pe oameni cu focul lui.
Rev 16:9  ?i oamenii au fost dogori?i cu mare ar?i?a ?i au hulit numele lui Dumnezeu, Care are putere peste urgiile acestea, ?i nu s-au pocait ca sa-I dea slava.
Rev 16:10  ?i al cincilea înger a varsat cupa lui pe scaunul fiarei ?i în împara?ia ei s-a facut întuneric ?i oamenii î?i mu?cau limbile de durere.
Rev 16:11  ?i au hulit pe Dumnezeul cerului din pricina durerilor ?i a bubelor lor, dar de faptele lor nu s-au pocait.
Rev 16:12  ?i al ?aselea înger a varsat cupa lui în râul cel mare Eufrat ?i apele lui au secat, ca sa fie gatita calea împara?ilor de la Rasaritul Soarelui.
Rev 16:13  ?i am vazut ie?ind din gura balaurului ?i din gura fiarei ?i din gura proorocului celui mincinos trei duhuri necurate ca ni?te broa?te.
Rev 16:14  Caci sunt duhuri diavole?ti, facatoare de semne ?i care se duc la împara?ii lumii întregi, sa-i adune la razboiul zilei celei mari a lui Dumnezeu, Atot?iitorul.
Rev 16:15  Iata, vin ca un fur. Fericit este cel ce privegheaza ?i pastreaza ve?mintele sale, ca sa nu umble gol ?i sa se vada ru?inea lui!
Rev 16:16  ?i i-au strâns la locul ce se cheama evreie?te Harmaghedon.
Rev 16:17  ?i al ?aptelea înger a varsat cupa lui în vazduh ?i glas mare a ie?it din templul cerului, de la tron, strigând: S-a facut!
Rev 16:18  ?i s-au pornit fulgere ?i vuiete ?i tunete ?i s-a facut cutremur mare, a?a cum nu a fost, de când este omul pe pamânt, un cutremur atât de puternic.
Rev 16:19  ?i cetatea cea mare s-a rupt în trei par?i ?i ceta?ile neamurilor s-au prabu?it, ?i Babilonul cel mare a fost pomenit înaintea lui Dumnezeu, ca sa-i dea paharul vinului aprinderii mâniei Lui.
Rev 16:20  ?i toate insulele pierira ?i mun?ii nu se mai aflara.
Rev 16:21  ?i grindina mare, cât talantul, se pravali din cer peste oameni. ?i oamenii hulira pe Dumnezeu, din pricina pedepsei cu grindina, caci urgia ei era foarte mare.
Rev 17:1  ?i a venit unul din cei ?apte îngeri, care aveau cele ?apte cupe, ?i a grait catre mine, zicând: Vino sa-?i arat judecata desfrânatei celei mari, care ?ade pe ape multe,
Rev 17:2  Cu care s-au desfrânat împara?ii pamântului ?i cei ce locuiesc pe pamânt s-au îmbatat de vinul desfrânarii ei.
Rev 17:3  ?i m-a dus, în duh, în pustie. ?i am vazut o femeie ?ezând pe o fiara ro?ie, plina de nume de hula, având ?apte capete ?i zece coarne.
Rev 17:4  ?i femeia era îmbracata în purpura ?i în stofa stacojie ?i împodobita cu aur ?i cu pietre scumpe ?i cu margaritare, având în mâna un pahar de aur, plin de urâciunile ?i de necura?iile desfrânarii ei.
Rev 17:5  Iar pe fruntea ei scris nume tainic: Babilonul cel mare, mama desfrânatelor ?i a urâciunilor pamântului.
Rev 17:6  ?i am vazut o femeie, beata de sângele sfin?ilor ?i de sângele mucenicilor lui Iisus, ?i vazând-o, m-am mirat cu mirare mare.
Rev 17:7  ?i îngerul mi-a zis: De ce te miri? Eu î?i voi spune taina femeii ?i a fiarei care o poarta ?i care are cele ?apte capete ?i cele zece coarne.
Rev 17:8  Fiara pe care ai vazut-o era ?i nu este ?i va sa se ridice din adânc ?i sa mearga spre pieire. ?i se vor mira cei ce locuiesc pe pamânt ale caror nume nu sunt scrise de la întemeierea lumii în cartea vie?ii, vazând pe fiara ca era ?i nu este, dar se va arata.
Rev 17:9  Aici trebuie minte care are în?elepciune. Cele ?apte capete sunt ?apte mun?i deasupra carora ?ade femeia.
Rev 17:10  Dar sunt ?i ?apte împara?i: cinci au cazut, unul mai este, celalalt înca nu a venit, iar când va veni are de stat pu?ina vreme.
Rev 17:11  ?i fiara care era ?i nu mai este - este al optulea împarat ?i este dintre cei ?apte ?i merge spre pieire.
Rev 17:12  ?i cele zece coarne pe care le-ai vazut sunt zece împara?i, care înca n-au luat împara?ia, dar care vor lua stapânire de împara?i, un ceas, împreuna cu fiara.
Rev 17:13  Ace?tia au un singur cuget ?i puterea ?i stapânirea lor o dau fiarei.
Rev 17:14  Ei vor porni razboi împotriva Mielului, dar Mielul îi va birui, pentru ca este Domnul domnilor ?i Împaratul împara?ilor ?i vor birui ?i cei împreuna cu El - chema?i ?i ale?i ?i credincio?i.
Rev 17:15  ?i mi-a zis: Apele pe care le-ai vazut ?i deasupra carora ?ade desfrânata, sunt popoare ?i gloate ?i neamuri ?i limbi.
Rev 17:16  ?i cele zece coarne pe care le-ai vazut ?i fiara vor urî pe desfrânata ?i o vor face pustie ?i goala ?i carnea ei o vor mânca ?i pe ea o vor arde în foc.
Rev 17:17  Caci Dumnezeu a pus în inimile lor sa faca voia Lui ?i sa se întâlneasca într-un gând ?i sa dea fiarei împara?ia lor, pâna se vor împlini cuvintele lui Dumnezeu.
Rev 17:18  Iar femeia pe care ai vazut-o este cetatea cea mare care are stapânire peste împara?ii pamântului.
Rev 18:1  Dupa acestea, am vazut un alt înger, pogorându-se din cer, având putere mare, ?i pamântul s-a luminat de slava lui,
Rev 18:2  ?i a strigat cu glas puternic ?i a zis: A cazut! A cazut Babilonul cel mare ?i a ajuns loca? demonilor, închisoare tuturor duhurilor necurate, ?i închisoare tuturor pasarilor spurcate ?i urâte.
Rev 18:3  Pentru ca din vinul aprinderii desfrânarii ei au baut toate neamurile ?i împara?ii pamântului s-au desfrânat cu ea ?i negu?atorii lumii din mul?imea desfatarilor ei s-au îmboga?it.
Rev 18:4  ?i am auzit un alt glas din cer, zicând: Ie?i?i din ea, poporul meu, ca sa nu va face?i parta?i la pacatele ei ?i sa nu fi?i lovi?i de pedepsele sortite ei;
Rev 18:5  Fiindca pacatele ei au ajuns pâna la cer ?i Dumnezeu ?i-a adus aminte de nedrepta?ile ei.
Rev 18:6  Da?i-i înapoi, precum v-a dat ?i ea ?i, dupa faptele ei, cu masura îndoita, îndoit masura?i-i; în paharul în care v-a turnat, turna?i-i de doua ori.
Rev 18:7  Pe cât s-a marit pe sine ?i a fost în desfatari, tot pe atâta da?i-i chin ?i plângere. Fiindca în inima ei zice: ?ed ca împarateasa ?i vaduva nu sunt ?i jale nu voi vedea nicidecum!
Rev 18:8  Pentru aceea într-o singura zi vor veni pedepsele peste ea: moarte ?i tânguire ?i foamete ?i focul va arde-o de tot, caci puternic este Domnul Dumnezeu, Cel ce o judeca.
Rev 18:9  Iar împara?ii pamântului, care s-au desfrânat cu ea ?i s-au dezmierdat cu ea, se vor jeli ?i se vor bate în piept pentru ea, când vor vedea fumul focului în care arde,
Rev 18:10  Stând departe de frica chipurilor ei, ?i zicând: Vai! Vai! Cetatea cea mare, Babilonul, cetatea cea tare, ca într-un ceas a venit judecata ta!
Rev 18:11  ?i negu?atorii lumii plâng ?i se tânguiesc asupra ei, caci nimeni nu mai cumpara marfa lor,
Rev 18:12  Marfa de aur ?i de argint, pietre pre?ioase ?i margaritare, vison ?i porfira, matase ?i stofa stacojie, tot felul de lemn bine mirositor ?i tot felul de lucruri de filde?, de lemn de mare pre? ?i marfa de arama ?i de fier ?i de marmura,
Rev 18:13  ?i scor?i?oara ?i balsam ?i mirodenii ?i mir ?i tamâie ?i vin ?i untdelemn ?i faina de grâu curat ?i grâu ?i vite ?i oi ?i cai ?i caru?e ?i trupuri ?i suflete de oameni.
Rev 18:14  ?i roadele cele dorite de sufletul tau s-au dus de la tine ?i toate cele grase ?i stralucite au pierit de la tine ?i niciodata nu le vor mai gasi.
Rev 18:15  Iar negu?atorii de aceste lucruri, care s-au îmboga?it de pe urma ei, vor sta departe, de frica chinurilor ei, plângând ?i tânguindu-se,
Rev 18:16  ?i zicând: Vai! Vai! Cetatea cea mare, cea înve?mântata în vison ?i în porfira ?i în stofa stacojie ?i împodobita cu aur ?i cu pietre scumpe ?i cu margaritare! Ca într-un ceas s-a pustiit atâta boga?ie!
Rev 18:17  ?i to?i cârmacii ?i to?i cei ce plutesc pe mare ?i corabierii ?i to?i câ?i lucreaza pe mare stateau departe,
Rev 18:18  ?i strigau, uitându-se la fumul focului în care ardea, zicând: Care cetate era asemenea cu cetatea cea mare!
Rev 18:19  ?i î?i puneau ?arâna pe capetele lor ?i strigau plângând ?i tânguindu-se ?i zicând: Vai! Vai! Cetatea cea mare, în care s-au îmboga?it din comorile ei to?i cei ce ?in corabii pe mare, ca într-un ceas s-a pustiit!
Rev 18:20  Vesele?te-te de ea, cerule ?i voi sfin?ilor, ?i voi apostolilor, ?i voi proorocilor, pentru ca Dumnezeu a pronun?at judecata voastra asupra ei.
Rev 18:21  ?i un înger puternic a ridicat o piatra, mare cât o piatra de moara, ?i a aruncat-o în mare, zicând: Cu astfel de repeziciune va fi aruncat Babilonul, cetatea cea mare, ?i nu se va mai afla.
Rev 18:22  ?i glasul celor ce cânta din chitara ?i din gura ?i din flaut ?i din trâmbi?a nu se va mai auzi de acum în tine ?i nici un me?te?ugar de orice fel de me?te?ug nu se va mai afla în tine ?i huruit de mori nu se va mai auzi în tine niciodata!
Rev 18:23  ?i niciodata lumina de lampa nu se va mai ivi în tine; ?i glasul de mire ?i mireasa nu se vor mai auzi în tine niciodata, pentru ca negu?atorii tai erau stapânitorii lumii ?i pentru ca toate neamurile s-au ratacit cu fermecatoria ta.
Rev 18:24  ?i s-a gasit în ea sânge de prooroci ?i de sfin?i ?i sângele tuturor celor înjunghia?i pe pamânt.
Rev 19:1  Dupa acestea, am auzit, în cer, ca un glas puternic de mul?ime multa zicând: Aliluia! Mântuirea ?i slava ?i puterea sunt ale Dumnezeului nostru!
Rev 19:2  Pentru ca adevarate ?i drepte sunt judeca?ile Lui! Pentru ca a judecat pe desfrânata cea mare, care a stricat pamântul cu desfrânarea ei, ?i a razbunat sângele robilor Sai, din mâna ei!
Rev 19:3  ?i a doua oara au zis: Aliluia! ?i fumul focului în care arde ea se ridica în vecii vecilor.
Rev 19:4  Iar cei douazeci ?i patru de batrâni ?i cele patru fiin?e au cazut ?i s-au închinat lui Dumnezeu, Cel ce ?ade pe tron, zicând: Amin! Aliluia!
Rev 19:5  ?i un glas a ie?it din tron, zicând: Lauda?i pe Dumnezeul nostru toate slugile Lui, cei ce va teme?i de El, mici ?i mari.
Rev 19:6  ?i am auzit ca un glas de mul?ime multa ?i ca un vuiet de ape multe ?i ca un bubuit de tunete puternice, zicând: Aliluia! pentru ca Domnul Dumnezeul nostru, Atot?iitorul, împara?e?te.
Rev 19:7  Sa ne bucuram ?i sa ne veselim ?i sa-I dam slava, caci a venit nunta Mielului ?i mireasa Lui s-a pregatit,
Rev 19:8  ?i i s-a dat ei sa se înve?mânteze cu vison curat, luminos, caci visonul sunt faptele cele drepte ale sfin?ilor.
Rev 19:9  ?i mi-a zis: Scrie: Ferici?i cei chema?i la cina nun?ii Mielului! ?i mi-a zis: Acestea sunt adevaratele cuvinte ale lui Dumnezeu.
Rev 19:10  ?i am cazut înaintea picioarelor lui, ca sa ma închin lui. ?i el mi-a zis: Vezi sa nu faci aceasta! Sunt împreuna-slujitor cu tine ?i cu fra?ii tai, care au marturia lui Iisus. Lui Dumnezeu închina-te, caci marturia lui Iisus este duhul proorociei.
Rev 19:11  ?i am vazut cerul deschis ?i iata un cal alb, ?i Cel ce ?edea pe el se nume?te Credincios ?i Adevarat ?i judeca ?i se razboie?te întru dreptate.
Rev 19:12  Iar ochii Lui sunt ca para focului ?i pe capul Lui sunt cununi multe ?i are nume scris pe care nimeni nu-l în?elege decât numai El.
Rev 19:13  ?i este îmbracat în ve?mânt stropit cu sânge ?i numele Lui se cheama: Cuvântul lui Dumnezeu.
Rev 19:14  ?i o?tile din cer veneau dupa El, calare pe cai albi, purtând ve?minte de vison alb, curat.
Rev 19:15  Iar din gura Lui ie?ea sabie ascu?ita, ca sa loveasca neamurile cu ea. ?i El îi va pastori cu toiag de fier ?i va calca teascul vinului aprinderii mâniei lui Dumnezeu, Atot?iitorul.
Rev 19:16  ?i pe haina Lui ?i pe coapsa Lui are nume scris: Împaratul împara?ilor ?i Domnul domnilor.
Rev 19:17  ?i am vazut un înger stând în soare; ?i a strigat cu glas puternic, graind tuturor pasarilor care zboara spre înaltul cerului: Veni?i ?i va aduna?i la ospa?ul cel mare al lui Dumnezeu,
Rev 19:18  Ca sa mânca?i trupuri de împara?i ?i trupuri de capetenii de o?ti ?i trupurile celor puternici, ?i trupurile cailor ?i ale calare?ilor lor, ?i trupurile tuturor celor slobozi ?i celor robi, ?i ale celor mici ?i celor mari.
Rev 19:19  ?i am vazut fiara ?i pe împara?ii pamântului, ?i o?tirile lor adunate, ca sa faca razboi ce Cel ce ?ade pe cal ?i cu o?tirea Lui.
Rev 19:20  ?i fiara a fost rapusa ?i, cu ea, proorocul cel mincinos, cel ce facea înaintea ei semnele cu care amagea pe cei ce au purtat semnul fiarei ?i pe cei ce s-au închinat chipului ei. Amândoi au fost arunca?i de vii în iezerul de foc unde arde pucioasa.
Rev 19:21  Iar ceilal?i au fost uci?i cu sabia care iese din gura Celui ce ?ade pe cal, ?i toate pasarile s-au saturat din trupurile lor.
Rev 20:1  ?i am vazut un înger, pogorându-se din cer, având cheia adâncului ?i un lan? mare în mâna lui.
Rev 20:2  ?i a prins pe balaur, ?arpele cel vechi, care este diavolul ?i satana, ?i l-a legat pe mii de ani,
Rev 20:3  ?i l-a aruncat în adânc ?i l-a închis ?i a pecetluit deasupra lui, ca sa nu mai amageasca neamurile, pâna ce se vor sfâr?i miile de ani. Dupa aceea, trebuie sa fie dezlegat câtava vreme.
Rev 20:4  ?i am vazut tronuri ?i celor ce ?edeau pe ele li s-a dat sa faca judecata. ?i am vazut sufletele celor taia?i pentru marturia lui Iisus ?i pentru cuvântul lui Dumnezeu, care nu s-au închinat fiarei, nici chipului ei, ?i nu au primit semnul ei pe fruntea ?i pe mâna lor. ?i ei au înviat ?i au împara?it cu Hristos mii de ani.
Rev 20:5  Iar ceilal?i mor?i nu înviaza pâna ce nu se vor sfâr?i miile de ani. Aceasta este învierea cea dintâi.
Rev 20:6  Fericit ?i sfânt este cel ce are parte de învierea cea dintâi. Peste ace?tia moartea cea de a doua nu are putere, ci vor fi preo?i ai lui Dumnezeu ?i ai lui Hristos ?i vor împara?i cu El mii de ani.
Rev 20:7  ?i catre sfâr?itul miilor de ani, satana va fi dezlegat din închisoarea lui,
Rev 20:8  ?i va ie?i sa amageasca neamurile, care sunt în cele patru unghiuri ale pamântului, pe Gog ?i pe Magog, ?i sa le adune la razboi; iar numarul lor este ca nisipul marii.
Rev 20:9  ?i s-au suit pe fa?a pamântului, ?i au înconjurat tabara sfin?ilor ?i cetatea cea iubita. Dar s-a pogorât foc din cer ?i i-a mistuit.
Rev 20:10  ?i diavolul, care-i amagise, a fost aruncat în iezerul de foc ?i de pucioasa, unde este ?i fiara ?i proorocul mincinos, ?i vor fi chinui?i acolo, zi ?i noapte, în vecii vecilor.
Rev 20:11  ?i am vazut, iar, un tron mare alb ?i pe Cel ce ?edea pe el, iar dinaintea fe?ei Lui pamântul ?i cerul au fugit ?i loc nu s-a mai gasit pentru ele.
Rev 20:12  ?i am vazut pe mor?i, pe cei mari ?i pe cei mici, stând înaintea tronului ?i car?ile au fost deschise; ?i o alta carte a fost deschisa, care este cartea vie?ii; ?i mor?ii au fost judeca?i din cele scrise în car?i, potrivit cu faptele lor.
Rev 20:13  ?i marea a dat pe mor?ii cei din ea ?i moartea ?i iadul au dat pe mor?ii lor, ?i judeca?i au fost, fiecare dupa faptele sale.
Rev 20:14  ?i moartea ?i iadul au fost aruncate în râul de foc. Aceasta e moartea cea de a doua: iezerul cel de foc.
Rev 20:15  Iar cine n-a fost aflat scris în cartea vie?ii, a fost aruncat în iezerul de foc.
Rev 21:1  ?i am vazut cer nou ?i pamânt nou. Caci cerul cel dintâi ?i pamântul cel dintâi au trecut; ?i marea nu mai este.
Rev 21:2  ?i am vazut cetatea sfânta, noul Ierusalim, pogorându-se din cer de la Dumnezeu, gatita ca o mireasa, împodobita pentru mirele ei.
Rev 21:3  ?i am auzit, din tron, un glas puternic care zicea: Iata, cortul lui Dumnezeu este cu oamenii ?i El va sala?lui cu ei ?i ei vor fi poporul Lui ?i însu?i Dumnezeu va fi cu ei.
Rev 21:4  ?i va ?terge orice lacrima din ochii lor ?i moarte nu va mai fi; nici plângere, nici strigat, nici durere nu vor mai fi, caci cele dintâi au trecut.
Rev 21:5  ?i Cel ce ?edea pe tron a grait: Iata, noi le facem pe toate. ?i a zis: Scrie, fiindca aceste cuvinte sunt vrednice de crezare ?i adevarate.
Rev 21:6  ?i iar mi-a zis: Facutu-s-a! Eu sunt Alfa ?i Omega, începutul ?i sfâr?itul. Celui ce înseteaza îi voi da sa bea, în dar, din izvorul apei vie?ii.
Rev 21:7  Cel ce va birui va mo?teni acestea ?i-i voi fi lui Dumnezeu ?i el Îmi va fi Mie fiu
Rev 21:8  Iar partea celor frico?i ?i necredincio?i ?i spurca?i ?i uciga?i ?i desfrâna?i ?i fermecatori ?i închinatori de idoli ?i a tuturor celor mincino?i este în iezerul care arde, cu foc ?i cu pucioasa, care este moartea a doua.
Rev 21:9  ?i a venit unul din cei ?apte îngeri, care aveau cele ?apte cupe pline cu cele din urma ?apte pedepse, ?i a grait catre mine zicând: Vino sa-?i arat pe mireasa, femeia Mielului.
Rev 21:10  ?i m-a dus pe mine, în duh, într-un munte mare ?i înalt ?i mi-a aratat cetatea cea sfânta, Ierusalimul, pogorându-se din cer, de la Dumnezeu,
Rev 21:11  Având slava lui Dumnezeu. Lumina ei era asemenea cu cea a pietrei de mare pre?, ca piatra de iaspis, limpede cum e cristalul.
Rev 21:12  ?i avea zid mare ?i înalt ?i avea douasprezece por?i, iar la por?i douasprezece îngeri ?i nume scrise deasupra, care sunt numele celor douasprezece semin?ii ale fiilor lui Israel.
Rev 21:13  Spre rasarit trei por?i ?i spre miazanoapte trei por?i ?i spre miazazi trei por?i ?i spre apus trei por?i.
Rev 21:14  Iar zidul ceta?ii avea douasprezece pietre de temelie ?i în ele douasprezece nume, ale celor douasprezece apostoli ai Mielului.
Rev 21:15  ?i cel ce vorbea cu mine avea pentru masurat o trestie de aur, ca sa masoare cetatea ?i por?ile ei ?i zidul ei.
Rev 21:16  ?i cetatea este în patru col?uri ?i lungimea ei este tot atâta cât ?i la?imea. ?i a masurat cetatea cu trestia: douasprezece mii de stadii. Lungimea ?i largimea ?i înal?imea ei sunt deopotriva.
Rev 21:17  ?i a masurat ?i zidul ei: o suta patruzeci ?i patru de co?i, dupa masura omeneasca, care este ?i a îngerului.
Rev 21:18  ?i zidaria zidului ei este de iaspis, iar cetatea este din aur curat, ca sticla cea curata.
Rev 21:19  Temeliile zidului ceta?ii sunt împodobite cu tot felul de pietre scumpe: întâia piatra de temelie este de iaspis, a doua din safir, a treia din halcedon, a patra de smarald,
Rev 21:20  A cincea de sardonix, a ?asea de sardiu, a ?aptea de hrisolit, a opta de beril, a noua de topaz, a zecea de hrisopras, a unsprezecea de iachint, a douasprezecea de ametist.
Rev 21:21  Iar cele douasprezece por?i sunt douasprezece margaritare; fiecare din por?i este dintr-un margaritar. ?i pia?a ceta?ii este de aur curat, ?i stravezie ca sticla.
Rev 21:22  ?i templu n-am vazut în ea, pentru ca Domnul Dumnezeu, Atot?iitorul, ?i Mielul este templul ei.
Rev 21:23  ?i cetatea nu are trebuin?a de soare, nici de luna, ca sa o lumineze, caci slava lui Dumnezeu a luminat-o ?i faclia ei este Mielul.
Rev 21:24  ?i neamurile vor umbla în lumina ei, iar împara?ii pamântului vor aduce la ea marirea lor.
Rev 21:25  ?i por?ile ceta?ii nu se vor mai închide ziua, caci noaptea nu va mai fi acolo.
Rev 21:26  ?i vor aduce în ea slava ?i cinstea neamurilor.
Rev 21:27  ?i în cetate nu va intra nimic pângarit ?i nimeni care e dedat cu spurcaciunea ?i cu minciuna, ci numai cei scri?i în Cartea vie?ii Mielului.
Rev 22:1  ?i mi-a aratat, apoi, râul ?i apa vie?ii, limpede cum e cristalul ?i care izvora?te din tronul lui Dumnezeu ?i al Mielului,
Rev 22:2  ?i în mijlocul pie?ei din cetate, de o parte ?i de alta a râului, cre?te pomul vie?ii, facând rod de douasprezece ori pe an, în fiecare luna dându-?i rodul; ?i frunzele pomului sunt spre tamaduirea neamurilor.
Rev 22:3  Nici un blestem nu va mai fi. ?i tronul lui Dumnezeu ?i al Mielului va fi în ea ?i slugile Lui Îi vor sluji Lui.
Rev 22:4  ?i vor vedea fa?a Lui ?i numele Lui va fi pe frun?ile lor.
Rev 22:5  ?i noapte nu va mai fi; ?i nu au trebuin?a de lumina lampii sau de lumina soarelui, pentru ca Domnul Dumnezeu le va fi lor lumina ?i vor împara?i în vecii vecilor.
Rev 22:6  ?i îngerul mi-a zis: Aceste cuvinte sunt vrednice de crezare ?i adevarate ?i Domnul, Dumnezeul duhurilor proorocilor, a trimis pe îngerul Sau sa arate robilor Sai cele ce trebuie sa se întâmple în curând.
Rev 22:7  ?i iata vin curând. Fericit cel ce paze?te cuvintele proorociei acestei car?i!
Rev 22:8  ?i eu, Ioan, sunt cel ce am vazut ?i am auzit acestea, iar când am auzit ?i am vazut, am cazut sa ma închin înaintea picioarelor îngerului care mi-a aratat acestea.
Rev 22:9  ?i el mi-a zis: Vezi sa nu faci aceasta! Caci sunt împreuna-slujitor cu tine ?i cu fra?ii tai, proorocii, ?i cu cei ce pastreaza cuvintele car?ii acesteia. Lui Dumnezeu închina-te!
Rev 22:10  Apoi mi-a zis: Sa nu pecetluie?ti cuvintele proorociei acestei car?i, caci vremea este aproape.
Rev 22:11  Cine e nedrept, sa nedrepta?easca înainte. Cine e spurcat, sa se spurce înca. Cine este drept, sa faca dreptate mai departe. Cine este sfânt, sa se sfin?easca înca.
Rev 22:12  Iata, vin curând ?i plata Mea este cu Mine, ca sa dau fiecaruia, dupa cum este fapta lui.
Rev 22:13  Eu sunt Alfa ?i Omega, cel dintâi ?i cel de pe urma, începutul ?i sfâr?itul.
Rev 22:14  Ferici?i cei ce spala ve?mintele lor ca sa aiba stapânire peste pomul vie?ii ?i prin por?i sa intre în cetate!
Rev 22:15  Afara câinii ?i vrajitorii ?i desfrâna?ii ?i uciga?ii ?i închinatorii de idoli ?i to?i cei ce lucreaza ?i iubesc minciuna!
Rev 22:16  Eu, Iisus, am trimis pe îngerul Meu ca sa marturiseasca voua acestea, cu privire la Biserici. Eu sunt radacina ?i odrasla lui David, steaua care straluce?te diminea?a.
Rev 22:17  ?i Duhul ?i mireasa zic: Vino. ?i cel ce aude sa zica: Vino. ?i cel însetat sa vina, cel ce dore?te sa ia în dar apa vie?ii.
Rev 22:18  ?i eu marturisesc oricui asculta cuvintele proorociei acestei car?i: De va mai adauga cineva ceva la ele, Dumnezeu va trimite asupra lui pedepsele ce sunt scrise în cartea aceasta;
Rev 22:19  Iar de va scoate cineva din cuvintele car?ii acestei proorocii, Dumnezeu va scoate partea lui din pomul vie?ii ?i din cetatea sfânta ?i de la cele scrise în cartea aceasta.
Rev 22:20  Cel ce marturise?te acestea zice: Da, vin curând. Amin! Vino, Doamne Iisuse!
Rev 22:21  Harul Domnului Iisus Hristos, cu voi cu to?i! Amin.


\end{document}