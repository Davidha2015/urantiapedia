\begin{document}

\title{Genesis}

Gen 1:1  В начале сотворил Бог небо и землю.
Gen 1:2  Земля же была безвидна и пуста, и тьма над бездною, и Дух Божий носился над водою.
Gen 1:3  И сказал Бог: да будет свет. И стал свет.
Gen 1:4  И увидел Бог свет, что он хорош, и отделил Бог свет от тьмы.
Gen 1:5  И назвал Бог свет днем, а тьму ночью. И был вечер, и было утро: день один.
Gen 1:6  И сказал Бог: да будет твердь посреди воды, и да отделяет она воду от воды.
Gen 1:7  И создал Бог твердь, и отделил воду, которая под твердью, от воды, которая над твердью. И стало так.
Gen 1:8  И назвал Бог твердь небом. И был вечер, и было утро: день второй.
Gen 1:9  И сказал Бог: да соберется вода, которая под небом, в одно место, и да явится суша. И стало так.
Gen 1:10  И назвал Бог сушу землею, а собрание вод назвал морями. И увидел Бог, что [это] хорошо.
Gen 1:11  И сказал Бог: да произрастит земля зелень, траву, сеющую семя дерево плодовитое, приносящее по роду своему плод, в котором семя его на земле. И стало так.
Gen 1:12  И произвела земля зелень, траву, сеющую семя по роду ее, и дерево, приносящее плод, в котором семя его по роду его. И увидел Бог, что [это] хорошо.
Gen 1:13  И был вечер, и было утро: день третий.
Gen 1:14  И сказал Бог: да будут светила на тверди небесной для отделения дня от ночи, и для знамений, и времен, и дней, и годов;
Gen 1:15  и да будут они светильниками на тверди небесной, чтобы светить на землю. И стало так.
Gen 1:16  И создал Бог два светила великие: светило большее, для управления днем, и светило меньшее, для управления ночью, и звезды;
Gen 1:17  и поставил их Бог на тверди небесной, чтобы светить на землю,
Gen 1:18  и управлять днем и ночью, и отделять свет от тьмы. И увидел Бог, что [это] хорошо.
Gen 1:19  И был вечер, и было утро: день четвертый.
Gen 1:20  И сказал Бог: да произведет вода пресмыкающихся, душу живую; и птицы да полетят над землею, по тверди небесной.
Gen 1:21  И сотворил Бог рыб больших и всякую душу животных пресмыкающихся, которых произвела вода, по роду их, и всякую птицу пернатую по роду ее. И увидел Бог, что [это] хорошо.
Gen 1:22  И благословил их Бог, говоря: плодитесь и размножайтесь, и наполняйте воды в морях, и птицы да размножаются на земле.
Gen 1:23  И был вечер, и было утро: день пятый.
Gen 1:24  И сказал Бог: да произведет земля душу живую по роду ее, скотов, и гадов, и зверей земных по роду их. И стало так.
Gen 1:25  И создал Бог зверей земных по роду их, и скот по роду его, и всех гадов земных по роду их. И увидел Бог, что [это] хорошо.
Gen 1:26  И сказал Бог: сотворим человека по образу Нашему по подобию Нашему, и да владычествуют они над рыбами морскими, и над птицами небесными, и над скотом, и над всею землею, и над всеми гадами, пресмыкающимися по земле.
Gen 1:27  И сотворил Бог человека по образу Своему, по образу Божию сотворил его; мужчину и женщину сотворил их.
Gen 1:28  И благословил их Бог, и сказал им Бог: плодитесь и размножайтесь, и наполняйте землю, и обладайте ею, и владычествуйте над рыбами морскими и над птицами небесными, и над всяким животным, пресмыкающимся по земле.
Gen 1:29  И сказал Бог: вот, Я дал вам всякую траву, сеющую семя, какая есть на всей земле, и всякое дерево, у которого плод древесный, сеющий семя; --вам [сие] будет в пищу;
Gen 1:30  а всем зверям земным, и всем птицам небесным, и всякому пресмыкающемуся по земле, в котором душа живая, [дал] Я всю зелень травную в пищу. И стало так.
Gen 1:31  И увидел Бог все, что Он создал, и вот, хорошо весьма. И был вечер, и было утро: день шестой.
Gen 2:1  Так совершены небо и земля и все воинство их.
Gen 2:2  И совершил Бог к седьмому дню дела Свои, которые Он делал, и почил в день седьмый от всех дел Своих, которые делал.
Gen 2:3  И благословил Бог седьмой день, и освятил его, ибо в оный почил от всех дел Своих, которые Бог творил и созидал.
Gen 2:4  Вот происхождение неба и земли, при сотворении их, в то время, когда Господь Бог создал землю и небо,
Gen 2:5  и всякий полевой кустарник, которого еще не было на земле, и всякую полевую траву, которая еще не росла, ибо Господь Бог не посылал дождя на землю, и не было человека для возделывания земли,
Gen 2:6  но пар поднимался с земли и орошал все лице земли.
Gen 2:7  И создал Господь Бог человека из праха земного, и вдунул в лице его дыхание жизни, и стал человек душею живою.
Gen 2:8  И насадил Господь Бог рай в Едеме на востоке, и поместил там человека, которого создал.
Gen 2:9  И произрастил Господь Бог из земли всякое дерево, приятное на вид и хорошее для пищи, и дерево жизни посреди рая, и дерево познания добра и зла.
Gen 2:10  Из Едема выходила река для орошения рая; и потом разделялась на четыре реки.
Gen 2:11  Имя одной Фисон: она обтекает всю землю Хавила, ту, где золото;
Gen 2:12  и золото той земли хорошее; там бдолах и камень оникс.
Gen 2:13  Имя второй реки Гихон: она обтекает всю землю Куш.
Gen 2:14  Имя третьей реки Хиддекель: она протекает пред Ассириею. Четвертая река Евфрат.
Gen 2:15  И взял Господь Бог человека, и поселил его в саду Едемском, чтобы возделывать его и хранить его.
Gen 2:16  И заповедал Господь Бог человеку, говоря: от всякого дерева в саду ты будешь есть,
Gen 2:17  а от дерева познания добра и зла не ешь от него, ибо в день, в который ты вкусишь от него, смертью умрешь.
Gen 2:18  И сказал Господь Бог: не хорошо быть человеку одному; сотворим ему помощника, соответственного ему.
Gen 2:19  Господь Бог образовал из земли всех животных полевых и всех птиц небесных, и привел к человеку, чтобы видеть, как он назовет их, и чтобы, как наречет человек всякую душу живую, так и было имя ей.
Gen 2:20  И нарек человек имена всем скотам и птицам небесным и всем зверям полевым; но для человека не нашлось помощника, подобного ему.
Gen 2:21  И навел Господь Бог на человека крепкий сон; и, когда он уснул, взял одно из ребр его, и закрыл то место плотию.
Gen 2:22  И создал Господь Бог из ребра, взятого у человека, жену, и привел ее к человеку.
Gen 2:23  И сказал человек: вот, это кость от костей моих и плоть от плоти моей; она будет называться женою, ибо взята от мужа.
Gen 2:24  Потому оставит человек отца своего и мать свою и прилепится к жене своей; и будут одна плоть.
Gen 2:25  И были оба наги, Адам и жена его, и не стыдились.
Gen 3:1  Змей был хитрее всех зверей полевых, которых создал Господь Бог. И сказал змей жене: подлинно ли сказал Бог: не ешьте ни от какого дерева в раю?
Gen 3:2  И сказала жена змею: плоды с дерев мы можем есть,
Gen 3:3  только плодов дерева, которое среди рая, сказал Бог, не ешьте их и не прикасайтесь к ним, чтобы вам не умереть.
Gen 3:4  И сказал змей жене: нет, не умрете,
Gen 3:5  но знает Бог, что в день, в который вы вкусите их, откроются глаза ваши, и вы будете, как боги, знающие добро и зло.
Gen 3:6  И увидела жена, что дерево хорошо для пищи, и что оно приятно для глаз и вожделенно, потому что дает знание; и взяла плодов его и ела; и дала также мужу своему, и он ел.
Gen 3:7  И открылись глаза у них обоих, и узнали они, что наги, и сшили смоковные листья, и сделали себе опоясания.
Gen 3:8  И услышали голос Господа Бога, ходящего в раю во время прохлады дня; и скрылся Адам и жена его от лица Господа Бога между деревьями рая.
Gen 3:9  И воззвал Господь Бог к Адаму и сказал ему: где ты?
Gen 3:10  Он сказал: голос Твой я услышал в раю, и убоялся, потому что я наг, и скрылся.
Gen 3:11  И сказал: кто сказал тебе, что ты наг? не ел ли ты от дерева, с которого Я запретил тебе есть?
Gen 3:12  Адам сказал: жена, которую Ты мне дал, она дала мне от дерева, и я ел.
Gen 3:13  И сказал Господь Бог жене: что ты это сделала? Жена сказала: змей обольстил меня, и я ела.
Gen 3:14  И сказал Господь Бог змею: за то, что ты сделал это, проклят ты пред всеми скотами и пред всеми зверями полевыми; ты будешь ходить на чреве твоем, и будешь есть прах во все дни жизни твоей;
Gen 3:15  и вражду положу между тобою и между женою, и между семенем твоим и между семенем ее; оно будет поражать тебя в голову, а ты будешь жалить его в пяту.
Gen 3:16  Жене сказал: умножая умножу скорбь твою в беременности твоей; в болезни будешь рождать детей; и к мужу твоему влечение твое, и он будет господствовать над тобою.
Gen 3:17  Адаму же сказал: за то, что ты послушал голоса жены твоей и ел от дерева, о котором Я заповедал тебе, сказав: не ешь от него, проклята земля за тебя; со скорбью будешь питаться от нее во все дни жизни твоей;
Gen 3:18  терния и волчцы произрастит она тебе; и будешь питаться полевою травою;
Gen 3:19  в поте лица твоего будешь есть хлеб, доколе не возвратишься в землю, из которой ты взят, ибо прах ты и в прах возвратишься.
Gen 3:20  И нарек Адам имя жене своей: Ева, ибо она стала матерью всех живущих.
Gen 3:21  И сделал Господь Бог Адаму и жене его одежды кожаные и одел их.
Gen 3:22  И сказал Господь Бог: вот, Адам стал как один из Нас, зная добро и зло; и теперь как бы не простер он руки своей, и не взял также от дерева жизни, и не вкусил, и не стал жить вечно.
Gen 3:23  И выслал его Господь Бог из сада Едемского, чтобы возделывать землю, из которой он взят.
Gen 3:24  И изгнал Адама, и поставил на востоке у сада Едемского Херувима и пламенный меч обращающийся, чтобы охранять путь к дереву жизни.
Gen 4:1  Адам познал Еву, жену свою; и она зачала, и родила Каина, и сказала: приобрела я человека от Господа.
Gen 4:2  И еще родила брата его, Авеля. И был Авель пастырь овец, а Каин был земледелец.
Gen 4:3  Спустя несколько времени, Каин принес от плодов земли дар Господу,
Gen 4:4  и Авель также принес от первородных стада своего и от тука их. И призрел Господь на Авеля и на дар его,
Gen 4:5  а на Каина и на дар его не призрел. Каин сильно огорчился, и поникло лице его.
Gen 4:6  И сказал Господь Каину: почему ты огорчился? и отчего поникло лице твое?
Gen 4:7  если делаешь доброе, то не поднимаешь ли лица? а если не делаешь доброго, то у дверей грех лежит; он влечет тебя к себе, но ты господствуй над ним.
Gen 4:8  И сказал Каин Авелю, брату своему. И когда они были в поле, восстал Каин на Авеля, брата своего, и убил его.
Gen 4:9  И сказал Господь Каину: где Авель, брат твой? Он сказал: не знаю; разве я сторож брату моему?
Gen 4:10  И сказал: что ты сделал? голос крови брата твоего вопиет ко Мне от земли;
Gen 4:11  и ныне проклят ты от земли, которая отверзла уста свои принять кровь брата твоего от руки твоей;
Gen 4:12  когда ты будешь возделывать землю, она не станет более давать силы своей для тебя; ты будешь изгнанником и скитальцем на земле.
Gen 4:13  И сказал Каин Господу: наказание мое больше, нежели снести можно;
Gen 4:14  вот, Ты теперь сгоняешь меня с лица земли, и от лица Твоего я скроюсь, и буду изгнанником и скитальцем на земле; и всякий, кто встретится со мною, убьет меня.
Gen 4:15  И сказал ему Господь: за то всякому, кто убьет Каина, отмстится всемеро. И сделал Господь Каину знамение, чтобы никто, встретившись с ним, не убил его.
Gen 4:16  И пошел Каин от лица Господня и поселился в земле Нод, на восток от Едема.
Gen 4:17  И познал Каин жену свою; и она зачала и родила Еноха. И построил он город; и назвал город по имени сына своего: Енох.
Gen 4:18  У Еноха родился Ирад; Ирад родил Мехиаеля; Мехиаель родил Мафусала; Мафусал родил Ламеха.
Gen 4:19  И взял себе Ламех две жены: имя одной: Ада, и имя второй: Цилла.
Gen 4:20  Ада родила Иавала: он был отец живущих в шатрах со стадами.
Gen 4:21  Имя брату его Иувал: он был отец всех играющих на гуслях и свирели.
Gen 4:22  Цилла также родила Тувалкаина, который был ковачом всех орудий из меди и железа. И сестра Тувалкаина Ноема.
Gen 4:23  И сказал Ламех женам своим: Ада и Цилла! послушайте голоса моего; жены Ламеховы! внимайте словам моим: я убил мужа в язву мне и отрока в рану мне;
Gen 4:24  если за Каина отмстится всемеро, то за Ламеха в семьдесят раз всемеро.
Gen 4:25  И познал Адам еще жену свою, и она родила сына, и нарекла ему имя: Сиф, потому что, [говорила она], Бог положил мне другое семя, вместо Авеля, которого убил Каин.
Gen 4:26  У Сифа также родился сын, и он нарек ему имя: Енос; тогда начали призывать имя Господа.
Gen 5:1  Вот родословие Адама: когда Бог сотворил человека, по подобию Божию создал его,
Gen 5:2  мужчину и женщину сотворил их, и благословил их, и нарек им имя: человек, в день сотворения их.
Gen 5:3  Адам жил сто тридцать лет и родил [сына] по подобию своему по образу своему, и нарек ему имя: Сиф.
Gen 5:4  Дней Адама по рождении им Сифа было восемьсот лет, и родил он сынов и дочерей.
Gen 5:5  Всех же дней жизни Адамовой было девятьсот тридцать лет; и он умер.
Gen 5:6  Сиф жил сто пять лет и родил Еноса.
Gen 5:7  По рождении Еноса Сиф жил восемьсот семь лет и родил сынов и дочерей.
Gen 5:8  Всех же дней Сифовых было девятьсот двенадцать лет; и он умер.
Gen 5:9  Енос жил девяносто лет и родил Каинана.
Gen 5:10  По рождении Каинана Енос жил восемьсот пятнадцать лет и родил сынов и дочерей.
Gen 5:11  Всех же дней Еноса было девятьсот пять лет; и он умер.
Gen 5:12  Каинан жил семьдесят лет и родил Малелеила.
Gen 5:13  По рождении Малелеила Каинан жил восемьсот сорок лет и родил сынов и дочерей.
Gen 5:14  Всех же дней Каинана было девятьсот десять лет; и он умер.
Gen 5:15  Малелеил жил шестьдесят пять лет и родил Иареда.
Gen 5:16  По рождении Иареда Малелеил жил восемьсот тридцать лет и родил сынов и дочерей.
Gen 5:17  Всех же дней Малелеила было восемьсот девяносто пять лет; и он умер.
Gen 5:18  Иаред жил сто шестьдесят два года и родил Еноха.
Gen 5:19  По рождении Еноха Иаред жил восемьсот лет и родил сынов и дочерей.
Gen 5:20  Всех же дней Иареда было девятьсот шестьдесят два года; и он умер.
Gen 5:21  Енох жил шестьдесят пять лет и родил Мафусала.
Gen 5:22  И ходил Енох пред Богом, по рождении Мафусала, триста лет и родил сынов и дочерей.
Gen 5:23  Всех же дней Еноха было триста шестьдесят пять лет.
Gen 5:24  И ходил Енох пред Богом; и не стало его, потому что Бог взял его.
Gen 5:25  Мафусал жил сто восемьдесят семь лет и родил Ламеха.
Gen 5:26  По рождении Ламеха Мафусал жил семьсот восемьдесят два года и родил сынов и дочерей.
Gen 5:27  Всех же дней Мафусала было девятьсот шестьдесят девять лет; и он умер.
Gen 5:28  Ламех жил сто восемьдесят два года и родил сына,
Gen 5:29  и нарек ему имя: Ной, сказав: он утешит нас в работе нашей и в трудах рук наших при [возделывании] земли, которую проклял Господь.
Gen 5:30  И жил Ламех по рождении Ноя пятьсот девяносто пять лет и родил сынов и дочерей.
Gen 5:31  Всех же дней Ламеха было семьсот семьдесят семь лет; и он умер.
Gen 5:32  Ною было пятьсот лет и родил Ной Сима, Хама и Иафета.
Gen 6:1  Когда люди начали умножаться на земле и родились у них дочери,
Gen 6:2  тогда сыны Божии увидели дочерей человеческих, что они красивы, и брали [их] себе в жены, какую кто избрал.
Gen 6:3  И сказал Господь: не вечно Духу Моему быть пренебрегаемым человеками; потому что они плоть; пусть будут дни их сто двадцать лет.
Gen 6:4  В то время были на земле исполины, особенно же с того времени, как сыны Божии стали входить к дочерям человеческим, и они стали рождать им: это сильные, издревле славные люди.
Gen 6:5  И увидел Господь, что велико развращение человеков на земле, и что все мысли и помышления сердца их были зло во всякое время;
Gen 6:6  и раскаялся Господь, что создал человека на земле, и восскорбел в сердце Своем.
Gen 6:7  И сказал Господь: истреблю с лица земли человеков, которых Я сотворил, от человека до скотов, и гадов и птиц небесных истреблю, ибо Я раскаялся, что создал их.
Gen 6:8  Ной же обрел благодать пред очами Господа.
Gen 6:9  Вот житие Ноя: Ной был человек праведный и непорочный в роде своем; Ной ходил пред Богом.
Gen 6:10  Ной родил трех сынов: Сима, Хама и Иафета.
Gen 6:11  Но земля растлилась пред лицем Божиим, и наполнилась земля злодеяниями.
Gen 6:12  И воззрел Бог на землю, и вот, она растленна, ибо всякая плоть извратила путь свой на земле.
Gen 6:13  И сказал Бог Ною: конец всякой плоти пришел пред лице Мое, ибо земля наполнилась от них злодеяниями; и вот, Я истреблю их с земли.
Gen 6:14  Сделай себе ковчег из дерева гофер; отделения сделай в ковчеге и осмоли его смолою внутри и снаружи.
Gen 6:15  И сделай его так: длина ковчега триста локтей; ширина его пятьдесят локтей, а высота его тридцать локтей.
Gen 6:16  И сделай отверстие в ковчеге, и в локоть сведи его вверху, и дверь в ковчег сделай с боку его; устрой в нем нижнее, второе и третье [жилье].
Gen 6:17  И вот, Я наведу на землю потоп водный, чтоб истребить всякую плоть, в которой есть дух жизни, под небесами; все, что есть на земле, лишится жизни.
Gen 6:18  Но с тобою Я поставлю завет Мой, и войдешь в ковчег ты, и сыновья твои, и жена твоя, и жены сынов твоих с тобою.
Gen 6:19  Введи также в ковчег из всех животных, и от всякой плоти по паре, чтоб они остались с тобою в живых; мужеского пола и женского пусть они будут.
Gen 6:20  Из птиц по роду их, и из скотов по роду их, и из всех пресмыкающихся по земле по роду их, из всех по паре войдут к тебе, чтобы остались в живых.
Gen 6:21  Ты же возьми себе всякой пищи, какою питаются, и собери к себе; и будет она для тебя и для них пищею.
Gen 6:22  И сделал Ной все: как повелел ему Бог, так он и сделал.
Gen 7:1  И сказал Господь Ною: войди ты и все семейство твое в ковчег, ибо тебя увидел Я праведным предо Мною в роде сем;
Gen 7:2  и всякого скота чистого возьми по семи, мужеского пола и женского, а из скота нечистого по два, мужеского пола и женского;
Gen 7:3  также и из птиц небесных по семи, мужеского пола и женского, чтобы сохранить племя для всей земли,
Gen 7:4  ибо чрез семь дней Я буду изливать дождь на землю сорок дней и сорок ночей; и истреблю все существующее, что Я создал, с лица земли.
Gen 7:5  Ной сделал все, что Господь повелел ему.
Gen 7:6  Ной же был шестисот лет, как потоп водный пришел на землю.
Gen 7:7  И вошел Ной и сыновья его, и жена его, и жены сынов его с ним в ковчег от вод потопа.
Gen 7:8  И из скотов чистых и из скотов нечистых, и из всех пресмыкающихся по земле
Gen 7:9  по паре, мужеского пола и женского, вошли к Ною в ковчег, как Бог повелел Ною.
Gen 7:10  Чрез семь дней воды потопа пришли на землю.
Gen 7:11  В шестисотый год жизни Ноевой, во второй месяц, в семнадцатый день месяца, в сей день разверзлись все источники великой бездны, и окна небесные отворились;
Gen 7:12  и лился на землю дождь сорок дней и сорок ночей.
Gen 7:13  В сей самый день вошел в ковчег Ной, и Сим, Хам и Иафет, сыновья Ноевы, и жена Ноева, и три жены сынов его с ними.
Gen 7:14  Они, и все звери по роду их, и всякий скот по роду его, и все гады, пресмыкающиеся по земле, по роду их, и все летающие по роду их, все птицы, все крылатые,
Gen 7:15  и вошли к Ною в ковчег по паре от всякой плоти, в которой есть дух жизни;
Gen 7:16  и вошедшие мужеский и женский пол всякой плоти вошли, как повелел ему Бог. И затворил Господь за ним.
Gen 7:17  И продолжалось на земле наводнение сорок дней, и умножилась вода, и подняла ковчег, и он возвысился над землею;
Gen 7:18  вода же усиливалась и весьма умножалась на земле, и ковчег плавал по поверхности вод.
Gen 7:19  И усилилась вода на земле чрезвычайно, так что покрылись все высокие горы, какие есть под всем небом;
Gen 7:20  на пятнадцать локтей поднялась над ними вода, и покрылись горы.
Gen 7:21  И лишилась жизни всякая плоть, движущаяся по земле, и птицы, и скоты, и звери, и все гады, ползающие по земле, и все люди;
Gen 7:22  все, что имело дыхание духа жизни в ноздрях своих на суше, умерло.
Gen 7:23  Истребилось всякое существо, которое было на поверхности земли; от человека до скота, и гадов, и птиц небесных, --все истребилось с земли, остался только Ной и что [было] с ним в ковчеге.
Gen 7:24  Вода же усиливалась на земле сто пятьдесят дней.
Gen 8:1  И вспомнил Бог о Ное, и о всех зверях, и о всех скотах, (и о всех птицах, и о всех гадах пресмыкающихся,) бывших с ним в ковчеге; и навел Бог ветер на землю, и воды остановились.
Gen 8:2  И закрылись источники бездны и окна небесные, и перестал дождь с неба.
Gen 8:3  Вода же постепенно возвращалась с земли, и стала убывать вода по окончании ста пятидесяти дней.
Gen 8:4  И остановился ковчег в седьмом месяце, в семнадцатый день месяца, на горах Араратских.
Gen 8:5  Вода постоянно убывала до десятого месяца; в первый день десятого месяца показались верхи гор.
Gen 8:6  По прошествии сорока дней Ной открыл сделанное им окно ковчега
Gen 8:7  и выпустил ворона, который, вылетев, отлетал и прилетал, пока осушилась земля от воды.
Gen 8:8  Потом выпустил от себя голубя, чтобы видеть, сошла ли вода с лица земли,
Gen 8:9  но голубь не нашел места покоя для ног своих и возвратился к нему в ковчег, ибо вода была еще на поверхности всей земли; и он простер руку свою, и взял его, и принял к себе в ковчег.
Gen 8:10  И помедлил еще семь дней других и опять выпустил голубя из ковчега.
Gen 8:11  Голубь возвратился к нему в вечернее время, и вот, свежий масличный лист во рту у него, и Ной узнал, что вода сошла с земли.
Gen 8:12  Он помедлил еще семь дней других и выпустил голубя; и он уже не возвратился к нему.
Gen 8:13  Шестьсот первого года к первому [дню] первого месяца иссякла вода на земле; и открыл Ной кровлю ковчега и посмотрел, и вот, обсохла поверхность земли.
Gen 8:14  И во втором месяце, к двадцать седьмому дню месяца, земля высохла.
Gen 8:15  И сказал Бог Ною:
Gen 8:16  выйди из ковчега ты и жена твоя, и сыновья твои, и жены сынов твоих с тобою;
Gen 8:17  выведи с собою всех животных, которые с тобою, от всякой плоти, из птиц, и скотов, и всех гадов, пресмыкающихся по земле: пусть разойдутся они по земле, и пусть плодятся и размножаются на земле.
Gen 8:18  И вышел Ной и сыновья его, и жена его, и жены сынов его с ним;
Gen 8:19  все звери, и все гады, и все птицы, все движущееся по земле, по родам своим, вышли из ковчега.
Gen 8:20  И устроил Ной жертвенник Господу; и взял из всякого скота чистого и из всех птиц чистых и принес во всесожжение на жертвеннике.
Gen 8:21  И обонял Господь приятное благоухание, и сказал Господь в сердце Своем: не буду больше проклинать землю за человека, потому что помышление сердца человеческого--зло от юности его; и не буду больше поражать всего живущего, как Я сделал:
Gen 8:22  впредь во все дни земли сеяние и жатва, холод и зной, лето и зима, день и ночь не прекратятся.
Gen 9:1  И благословил Бог Ноя и сынов его и сказал им: плодитесь и размножайтесь, и наполняйте землю.
Gen 9:2  да страшатся и да трепещут вас все звери земные, и все птицы небесные, все, что движется на земле, и все рыбы морские: в ваши руки отданы они;
Gen 9:3  все движущееся, что живет, будет вам в пищу; как зелень травную даю вам все;
Gen 9:4  только плоти с душею ее, с кровью ее, не ешьте;
Gen 9:5  Я взыщу и вашу кровь, [в которой] жизнь ваша, взыщу ее от всякого зверя, взыщу также душу человека от руки человека, от руки брата его;
Gen 9:6  кто прольет кровь человеческую, того кровь прольется рукою человека: ибо человек создан по образу Божию;
Gen 9:7  вы же плодитесь и размножайтесь, и распространяйтесь по земле, и умножайтесь на ней.
Gen 9:8  И сказал Бог Ною и сынам его с ним:
Gen 9:9  вот, Я поставляю завет Мой с вами и с потомством вашим после вас,
Gen 9:10  и со всякою душею живою, которая с вами, с птицами и со скотами, и со всеми зверями земными, которые у вас, со всеми вышедшими из ковчега, со всеми животными земными;
Gen 9:11  поставляю завет Мой с вами, что не будет более истреблена всякая плоть водами потопа, и не будет уже потопа на опустошение земли.
Gen 9:12  И сказал Бог: вот знамение завета, который Я поставляю между Мною и между вами и между всякою душею живою, которая с вами, в роды навсегда:
Gen 9:13  Я полагаю радугу Мою в облаке, чтоб она была знамением завета между Мною и между землею.
Gen 9:14  И будет, когда Я наведу облако на землю, то явится радуга в облаке;
Gen 9:15  и Я вспомню завет Мой, который между Мною и между вами и между всякою душею живою во всякой плоти; и не будет более вода потопом на истребление всякой плоти.
Gen 9:16  И будет радуга в облаке, и Я увижу ее, и вспомню завет вечный между Богом и между всякою душею живою во всякой плоти, которая на земле.
Gen 9:17  И сказал Бог Ною: вот знамение завета, который Я поставил между Мною и между всякою плотью, которая на земле.
Gen 9:18  Сыновья Ноя, вышедшие из ковчега, были: Сим, Хам и Иафет. Хам же был отец Ханаана.
Gen 9:19  Сии трое были сыновья Ноевы, и от них населилась вся земля.
Gen 9:20  Ной начал возделывать землю и насадил виноградник;
Gen 9:21  и выпил он вина, и опьянел, и [лежал] обнаженным в шатре своем.
Gen 9:22  И увидел Хам, отец Ханаана, наготу отца своего, и выйдя рассказал двум братьям своим.
Gen 9:23  Сим же и Иафет взяли одежду и, положив ее на плечи свои, пошли задом и покрыли наготу отца своего; лица их были обращены назад, и они не видали наготы отца своего.
Gen 9:24  Ной проспался от вина своего и узнал, что сделал над ним меньший сын его,
Gen 9:25  и сказал: проклят Ханаан; раб рабов будет он у братьев своих.
Gen 9:26  Потом сказал: благословен Господь Бог Симов; Ханаан же будет рабом ему;
Gen 9:27  да распространит Бог Иафета, и да вселится он в шатрах Симовых; Ханаан же будет рабом ему.
Gen 9:28  И жил Ной после потопа триста пятьдесят лет.
Gen 9:29  Всех же дней Ноевых было девятьсот пятьдесят лет, и он умер.
Gen 10:1  Вот родословие сынов Ноевых: Сима, Хама и Иафета. После потопа родились у них дети.
Gen 10:2  Сыны Иафета: Гомер, Магог, Мадай, Иаван, Фувал, Мешех и Фирас.
Gen 10:3  Сыны Гомера: Аскеназ, Рифат и Фогарма.
Gen 10:4  Сыны Иавана: Елиса, Фарсис, Киттим и Доданим.
Gen 10:5  От сих населились острова народов в землях их, каждый по языку своему, по племенам своим, в народах своих.
Gen 10:6  Сыны Хама: Хуш, Мицраим, Фут и Ханаан.
Gen 10:7  Сыны Хуша: Сева, Хавила, Савта, Раама и Савтеха. Сыны Раамы: Шева и дедан.
Gen 10:8  Хуш родил также Нимрода: сей начал быть силен на земле.
Gen 10:9  Он был сильный зверолов пред Господом; потому и говориться: сильный зверолов, как Нимрод, пред Господом.
Gen 10:10  Царство его вначале сщставляли: Вавилон, Эрех, аккад и Халне, в земле Сеннаар.
Gen 10:11  Из сей земли вышел Ассур, и построил Ниневию, Реховофир, Калах.
Gen 10:12  И ресен между Ниневию и между Калахом; это город великий.
Gen 10:13  От Мицраима произщшли Лудим, Анамим, Легавим, Нафтухим,
Gen 10:14  Патрусим, Каслухим, откуда вышли Филистимляне, и Кафторим.
Gen 10:15  От Ханаана родились: Сидон, первенец его, Хет,
Gen 10:16  Иевусей, Аморей, Гергесей,
Gen 10:17  Евей, Аркей, Синей,
Gen 10:18  Арвадей, Цемарей и Химарей. В последствии племена Ханаанские рассеялись.
Gen 10:19  И были пределы Хананеев от Сидона к Герару до Газы, Отсюда к Садому, Гаморре, Адме и Цевоиму до Лаши.
Gen 10:20  Это сыны Хамовы, по племенам их, по языкам их, в землях их, в народах их.
Gen 10:21  Были дети и у Сима, отца всех сынов Еверовых, старшего брата Иафетова.
Gen 10:22  Сыны Сима: Елам, Асур, Арфаксад, Луд, Арам.
Gen 10:23  Сыны Арама: Уц, Хул, Гефер и Маш.
Gen 10:24  Арфаксад родил Салу, Сала родил Евера.
Gen 10:25  У Евера родились два сына; имя одному: Фалек, потому что во дни его земля разделена; имя брата его: Иоктан.
Gen 10:26  Иоктан родил Алмодада, Шалефа, Хацармавефа, Иераха,
Gen 10:27  Гадорама, Узала, Диклу,
Gen 10:28  Овала, Авимаила, Шеву,
Gen 10:29  Офира, Хавилу и Иовава. Все эти сыновья Иоктана.
Gen 10:30  Поселения их были от Меши до Сефара, горы восточной.
Gen 10:31  Это сыновья Симовы по племенам их, по языкам их, в землях их, по народам их.
Gen 10:32  Вот племена сынов Ноевых, по Родословию их, в народах их. От них распространились народы по земле после потопа.
Gen 11:1  На всей земле был один язык и одно наречие.
Gen 11:2  Двинувшись с востока, они нашли в земле Сеннаар равнину и поселились там.
Gen 11:3  И сказали друг другу: наделаем кирпичей и обожжем огнем. И стали у них кирпичи вместо камней, а земляная смола вместо извести.
Gen 11:4  И сказали они: построим себе город и башню, высотою до небес, и сделаем себе имя, прежде нежели рассеемся по лицу всей земли.
Gen 11:5  И сошел Господь посмотреть город и башню, которые строили сыны человеческие.
Gen 11:6  И сказал Господь: вот, один народ, и один у всех язык; и вот что начали они делать, и не отстанут они от того, что задумали делать;
Gen 11:7  сойдем же и смешаем там язык их, так чтобы один не понимал речи другого.
Gen 11:8  И рассеял их Господь оттуда по всей земле; и они перестали строить город.
Gen 11:9  Посему дано ему имя: Вавилон, ибо там смешал Господь язык всей земли, и оттуда рассеял их Господь по всей земле.
Gen 11:10  Вот родословие Сима: Сим был ста лет и родил Арфаксада, чрез два года после потопа;
Gen 11:11  по рождении Арфаксада Сим жил пятьсот лет и родил сынов и дочерей.
Gen 11:12  Арфаксад жил тридцать пять лет и родил Салу.
Gen 11:13  По рождении Салы Арфаксад жил четыреста три года и родил сынов и дочерей.
Gen 11:14  Сала жил тридцать лет и родил Евера.
Gen 11:15  По рождении Евера Сала жил четыреста три года и родил сынов и дочерей.
Gen 11:16  Евер жил тридцать четыре года и родил Фалека.
Gen 11:17  По рождении Фалека Евер жил четыреста тридцать лет и родил сынов и дочерей.
Gen 11:18  Фалек жил тридцать лет и родил Рагава.
Gen 11:19  По рождении Рагава Фалек жил двести девять лет и родил сынов и дочерей.
Gen 11:20  Рагав жил тридцать два года и родил Серуха.
Gen 11:21  По рождении Серуха Рагав жил двести семь лет и родил сынов и дочерей.
Gen 11:22  Серух жил тридцать лет и родил Нахора.
Gen 11:23  По рождении Нахора Серух жил двести лет и родил сынов и дочерей.
Gen 11:24  Нахор жил двадцать девять лет и родил Фарру.
Gen 11:25  По рождении Фарры Нахор жил сто девятнадцать лет и родил сынов и дочерей.
Gen 11:26  Фарра жил семьдесят лет и родил Аврама, Нахора и Арана.
Gen 11:27  Вот родословие Фарры: Фарра родил Аврама, Нахора и Арана. Аран родил Лота.
Gen 11:28  И умер Аран при Фарре, отце своем, в земле рождения своего, в Уре Халдейском.
Gen 11:29  Аврам и Нахор взяли себе жен; имя жены Аврамовой: Сара; имя жены Нахоровой: Милка, дочь Арана, отца Милки и отца Иски.
Gen 11:30  И Сара была неплодна и бездетна.
Gen 11:31  И взял Фарра Аврама, сына своего, и Лота, сына Аранова, внука своего, и Сару, невестку свою, жену Аврама, сына своего, и вышел с ними из Ура Халдейского, чтобы идти в землю Ханаанскую; но, дойдя до Харрана, они остановились там.
Gen 11:32  И было дней [жизни] Фарры двести пять лет, и умер Фарра в Харране.
Gen 12:1  И сказал Господь Авраму: пойди из земли твоей, от родства твоего и из дома отца твоего, в землю, которую Я укажу тебе;
Gen 12:2  и Я произведу от тебя великий народ, и благословлю тебя, и возвеличу имя твое, и будешь ты в благословение;
Gen 12:3  Я благословлю благословляющих тебя, и злословящих тебя прокляну; и благословятся в тебе все племена земные.
Gen 12:4  И пошел Аврам, как сказал ему Господь; и с ним пошел Лот. Аврам был семидесяти пяти лет, когда вышел из Харрана.
Gen 12:5  И взял Аврам с собою Сару, жену свою, Лота, сына брата своего, и все имение, которое они приобрели, и всех людей, которых они имели в Харране; и вышли, чтобы идти в землю Ханаанскую; и пришли в землю Ханаанскую.
Gen 12:6  И прошел Аврам по земле сей до места Сихема, до дубравы Море. В этой земле тогда [жили] Хананеи.
Gen 12:7  И явился Господь Авраму и сказал: потомству твоему отдам Я землю сию. И создал [он] там жертвенник Господу, Который явился ему.
Gen 12:8  Оттуда двинулся он к горе, на восток от Вефиля; и поставил шатер свой [так, что от него] Вефиль [был] на запад, а Гай на восток; и создал там жертвенник Господу и призвал имя Господа.
Gen 12:9  И поднялся Аврам и продолжал идти к югу.
Gen 12:10  И был голод в той земле. И сошел Аврам в Египет, пожить там, потому что усилился голод в земле той.
Gen 12:11  Когда же он приближался к Египту, то сказал Саре, жене своей: вот, я знаю, что ты женщина, прекрасная видом;
Gen 12:12  и когда Египтяне увидят тебя, то скажут: это жена его; и убьют меня, а тебя оставят в живых;
Gen 12:13  скажи же, что ты мне сестра, дабы мне хорошо было ради тебя, и дабы жива была душа моя чрез тебя.
Gen 12:14  И было, когда пришел Аврам в Египет, Египтяне увидели, что она женщина весьма красивая;
Gen 12:15  увидели ее и вельможи фараоновы и похвалили ее фараону; и взята была она в дом фараонов.
Gen 12:16  И Авраму хорошо было ради ее; и был у него мелкий и крупный скот и ослы, и рабы и рабыни, и лошаки и верблюды.
Gen 12:17  Но Господь поразил тяжкими ударами фараона и дом его за Сару, жену Аврамову.
Gen 12:18  И призвал фараон Аврама и сказал: что ты это сделал со мною? для чего не сказал мне, что она жена твоя?
Gen 12:19  для чего ты сказал: она сестра моя? и я взял было ее себе в жену. И теперь вот жена твоя; возьми и пойди.
Gen 12:20  И дал о нем фараон повеление людям, и проводили его, и жену его, и все, что у него было.
Gen 13:1  И поднялся Аврам из Египта, сам и жена его, и все, что у него было, и Лот с ним, на юг.
Gen 13:2  И был Аврам очень богат скотом, и серебром, и золотом.
Gen 13:3  И продолжал он переходы свои от юга до Вефиля, до места, где прежде был шатер его между Вефилем и между Гаем,
Gen 13:4  до места жертвенника, который он сделал там вначале; и там призвал Аврам имя Господа.
Gen 13:5  И у Лота, который ходил с Аврамом, также был мелкий и крупный скот и шатры.
Gen 13:6  И непоместительна была земля для них, чтобы жить вместе, ибо имущество их было так велико, что они не могли жить вместе.
Gen 13:7  И был спор между пастухами скота Аврамова и между пастухами скота Лотова; и Хананеи и Ферезеи жили тогда в той земле.
Gen 13:8  И сказал Аврам Лоту: да не будет раздора между мною и тобою, и между пастухами моими и пастухами твоими, ибо мы родственники;
Gen 13:9  не вся ли земля пред тобою? отделись же от меня: если ты налево, то я направо; а если ты направо, то я налево.
Gen 13:10  Лот возвел очи свои и увидел всю окрестность Иорданскую, что она, прежде нежели истребил Господь Содом и Гоморру, вся до Сигора орошалась водою, как сад Господень, как земля Египетская;
Gen 13:11  и избрал себе Лот всю окрестность Иорданскую; и двинулся Лот к востоку. И отделились они друг от друга.
Gen 13:12  Аврам стал жить на земле Ханаанской; а Лот стал жить в городах окрестности и раскинул шатры до Содома.
Gen 13:13  Жители же Содомские были злы и весьма грешны пред Господом.
Gen 13:14  И сказал Господь Авраму, после того как Лот отделился от него: возведи очи твои и с места, на котором ты теперь, посмотри к северу и к югу, и к востоку и к западу;
Gen 13:15  ибо всю землю, которую ты видишь, тебе дам Я и потомству твоему навеки,
Gen 13:16  и сделаю потомство твое, как песок земной; если кто может сосчитать песок земной, то и потомство твое сочтено будет;
Gen 13:17  встань, пройди по земле сей в долготу и в широту ее, ибо Я тебе дам ее.
Gen 13:18  И двинул Аврам шатер, и пошел, и поселился у дубравы Мамре, что в Хевроне; и создал там жертвенник Господу.
Gen 14:1  И было во дни Амрафела, царя Сеннаарского, Ариоха, царя Елласарского, Кедорлаомера, царя Еламского, и Фидала, царя Гоимского,
Gen 14:2  пошли они войною против Беры, царя Содомского, против Бирши, царя Гоморрского, Шинава, царя Адмы, Шемевера, царя Севоимского, и против царя Белы, которая есть Сигор.
Gen 14:3  Все сии соединились в долине Сиддим, где [ныне] море Соленое.
Gen 14:4  Двенадцать лет были они в порабощении у Кедорлаомера, а в тринадцатом году возмутились.
Gen 14:5  В четырнадцатом году пришел Кедорлаомер и цари, которые с ним, и поразили Рефаимов в Аштероф-Карнаиме, Зузимов в Гаме, Эмимов в Шаве-Кириафаиме,
Gen 14:6  и Хорреев в горе их Сеире, до Эл-Фарана, что при пустыне.
Gen 14:7  И возвратившись оттуда, они пришли к источнику Мишпат, который есть Кадес, и поразили всю страну Амаликитян, и также Аморреев, живущих в Хацацон-Фамаре.
Gen 14:8  И вышли царь Содомский, царь Гоморрский, царь Адмы, царь Севоимский и царь Белы, которая есть Сигор; и вступили в сражение с ними в долине Сиддим,
Gen 14:9  с Кедорлаомером, царем Еламским, Фидалом, царем Гоимским, Амрафелом, царем Сеннаарским, Ариохом, царем Елласарским, --четыре царя против пяти.
Gen 14:10  В долине же Сиддим было много смоляных ям. И цари Содомский и Гоморрский, обратившись в бегство, упали в них, а остальные убежали в горы.
Gen 14:11  [Победители] взяли все имущество Содома и Гоморры и весь запас их и ушли.
Gen 14:12  И взяли Лота, племянника Аврамова, жившего в Содоме, и имущество его и ушли.
Gen 14:13  И пришел один из уцелевших и известил Аврама Еврея, жившего тогда у дубравы Мамре, Аморреянина, брата Эшколу и брата Анеру, которые были союзники Аврамовы.
Gen 14:14  Аврам, услышав, что сродник его взят в плен, вооружил рабов своих, рожденных в доме его, триста восемнадцать, и преследовал [неприятелей] до Дана;
Gen 14:15  и, разделившись, [напал] на них ночью, сам и рабы его, и поразил их, и преследовал их до Ховы, что по левую сторону Дамаска;
Gen 14:16  и возвратил все имущество и Лота, сродника своего, и имущество его возвратил, также и женщин и народ.
Gen 14:17  Когда он возвращался после поражения Кедорлаомера и царей, бывших с ним, царь Содомский вышел ему навстречу в долину Шаве, что [ныне] долина царская;
Gen 14:18  и Мелхиседек, царь Салимский, вынес хлеб и вино, --он был священник Бога Всевышнего, --
Gen 14:19  и благословил его, и сказал: благословен Аврам от Бога Всевышнего, Владыки неба и земли;
Gen 14:20  и благословен Бог Всевышний, Который предал врагов твоих в руки твои. [Аврам] дал ему десятую часть из всего.
Gen 14:21  И сказал царь Содомский Авраму: отдай мне людей, а имение возьми себе.
Gen 14:22  Но Аврам сказал царю Содомскому: поднимаю руку мою к Господу Богу Всевышнему, Владыке неба и земли,
Gen 14:23  что даже нитки и ремня от обуви не возьму из всего твоего, чтобы ты не сказал: я обогатил Аврама;
Gen 14:24  кроме того, что съели отроки, и кроме доли, принадлежащей людям, которые ходили со мною; Анер, Эшкол и Мамрий пусть возьмут свою долю.
Gen 15:1  После сих происшествий было слово Господа к Авраму в видении, и сказано: не бойся, Аврам; Я твой щит; награда твоя весьма велика.
Gen 15:2  Аврам сказал: Владыка Господи! что Ты дашь мне? я остаюсь бездетным; распорядитель в доме моем этот Елиезер из Дамаска.
Gen 15:3  И сказал Аврам: вот, Ты не дал мне потомства, и вот, домочадец мой наследник мой.
Gen 15:4  И было слово Господа к нему, и сказано: не будет он твоим наследником, но тот, кто произойдет из чресл твоих, будет твоим наследником.
Gen 15:5  И вывел его вон и сказал: посмотри на небо и сосчитай звезды, если ты можешь счесть их. И сказал ему: столько будет у тебя потомков.
Gen 15:6  Аврам поверил Господу, и Он вменил ему это в праведность.
Gen 15:7  И сказал ему: Я Господь, Который вывел тебя из Ура Халдейского, чтобы дать тебе землю сию во владение.
Gen 15:8  Он сказал: Владыка Господи! по чему мне узнать, что я буду владеть ею?
Gen 15:9  [Господь] сказал ему: возьми Мне трехлетнюю телицу, трехлетнюю козу, трехлетнего овна, горлицу и молодого голубя.
Gen 15:10  Он взял всех их, рассек их пополам и положил одну часть против другой; только птиц не рассек.
Gen 15:11  И налетели на трупы хищные птицы; но Аврам отгонял их.
Gen 15:12  При захождении солнца крепкий сон напал на Аврама, и вот, напал на него ужас и мрак великий.
Gen 15:13  И сказал [Господь] Авраму: знай, что потомки твои будут пришельцами в земле не своей, и поработят их, и будут угнетать их четыреста лет,
Gen 15:14  но Я произведу суд над народом, у которого они будут в порабощении; после сего они выйдут с большим имуществом,
Gen 15:15  а ты отойдешь к отцам твоим в мире [и] будешь погребен в старости доброй;
Gen 15:16  в четвертом роде возвратятся они сюда: ибо [мера] беззаконий Аморреев доселе еще не наполнилась.
Gen 15:17  Когда зашло солнце и наступила тьма, вот, дым [как бы из] печи и пламя огня прошли между рассеченными [животными].
Gen 15:18  В этот день заключил Господь завет с Аврамом, сказав: потомству твоему даю Я землю сию, от реки Египетской до великой реки, реки Евфрата:
Gen 15:19  Кенеев, Кенезеев, Кедмонеев,
Gen 15:20  Хеттеев, Ферезеев, Рефаимов,
Gen 15:21  Аморреев, Хананеев, Гергесеев и Иевусеев.
Gen 16:1  Но Сара, жена Аврамова, не рождала ему. У ней была служанка Египтянка, именем Агарь.
Gen 16:2  И сказала Сара Авраму: вот, Господь заключил чрево мое, чтобы мне не рождать; войди же к служанке моей: может быть, я буду иметь детей от нее. Аврам послушался слов Сары.
Gen 16:3  И взяла Сара, жена Аврамова, служанку свою, Египтянку Агарь, по истечении десяти лет пребывания Аврамова в земле Ханаанской, и дала ее Авраму, мужу своему, в жену.
Gen 16:4  Он вошел к Агари, и она зачала. Увидев же, что зачала, она стала презирать госпожу свою.
Gen 16:5  И сказала Сара Авраму: в обиде моей ты виновен; я отдала служанку мою в недро твое; а она, увидев, что зачала, стала презирать меня; Господь пусть будет судьею между мною и между тобою.
Gen 16:6  Аврам сказал Саре: вот, служанка твоя в твоих руках; делай с нею, что тебе угодно. И Сара стала притеснять ее, и она убежала от нее.
Gen 16:7  И нашел ее Ангел Господень у источника воды в пустыне, у источника на дороге к Суру.
Gen 16:8  И сказал ей: Агарь, служанка Сарина! откуда ты пришла и куда идешь? Она сказала: я бегу от лица Сары, госпожи моей.
Gen 16:9  Ангел Господень сказал ей: возвратись к госпоже своей и покорись ей.
Gen 16:10  И сказал ей Ангел Господень: умножая умножу потомство твое, так что нельзя будет и счесть его от множества.
Gen 16:11  И еще сказал ей Ангел Господень: вот, ты беременна, и родишь сына, и наречешь ему имя Измаил, ибо услышал Господь страдание твое;
Gen 16:12  он будет [между] людьми, [как] дикий осел; руки его на всех, и руки всех на него; жить будет он пред лицем всех братьев своих.
Gen 16:13  И нарекла [Агарь] Господа, Который говорил к ней, [сим] именем: Ты Бог видящий меня. Ибо сказала она: точно я видела здесь в след видящего меня.
Gen 16:14  Посему источник [тот] называется: Беэр-лахай-рои. Он находится между Кадесом и между Баредом.
Gen 16:15  Агарь родила Авраму сына; и нарек [Аврам] имя сыну своему, рожденному от Агари: Измаил.
Gen 16:16  Аврам был восьмидесяти шести лет, когда Агарь родила Авраму Измаила.
Gen 17:1  Аврам был девяноста девяти лет, и Господь явился Авраму и сказал ему: Я Бог Всемогущий; ходи предо Мною и будь непорочен;
Gen 17:2  и поставлю завет Мой между Мною и тобою, и весьма, весьма размножу тебя.
Gen 17:3  И пал Аврам на лице свое. Бог продолжал говорить с ним и сказал:
Gen 17:4  Я--вот завет Мой с тобою: ты будешь отцом множества народов,
Gen 17:5  и не будешь ты больше называться Аврамом, но будет тебе имя: Авраам, ибо Я сделаю тебя отцом множества народов;
Gen 17:6  и весьма, весьма распложу тебя, и произведу от тебя народы, и цари произойдут от тебя;
Gen 17:7  и поставлю завет Мой между Мною и тобою и между потомками твоими после тебя в роды их, завет вечный в том, что Я буду Богом твоим и потомков твоих после тебя;
Gen 17:8  и дам тебе и потомкам твоим после тебя землю, по которой ты странствуешь, всю землю Ханаанскую, во владение вечное; и буду им Богом.
Gen 17:9  И сказал Бог Аврааму: ты же соблюди завет Мой, ты и потомки твои после тебя в роды их.
Gen 17:10  Сей есть завет Мой, который вы [должны] соблюдать между Мною и между вами и между потомками твоими после тебя: да будет у вас обрезан весь мужеский пол;
Gen 17:11  обрезывайте крайнюю плоть вашу: и сие будет знамением завета между Мною и вами.
Gen 17:12  Восьми дней от рождения да будет обрезан у вас в роды ваши всякий [младенец] мужеского пола, рожденный в доме и купленный за серебро у какого-нибудь иноплеменника, который не от твоего семени.
Gen 17:13  Непременно да будет обрезан рожденный в доме твоем и купленный за серебро твое, и будет завет Мой на теле вашем заветом вечным.
Gen 17:14  Необрезанный же мужеского пола, который не обрежет крайней плоти своей, истребится душа та из народа своего, [ибо] он нарушил завет Мой.
Gen 17:15  И сказал Бог Аврааму: Сару, жену твою, не называй Сарою, но да будет имя ей: Сарра;
Gen 17:16  Я благословлю ее и дам тебе от нее сына; благословлю ее, и произойдут от нее народы, и цари народов произойдут от нее.
Gen 17:17  И пал Авраам на лице свое, и рассмеялся, и сказал сам в себе: неужели от столетнего будет сын? и Сарра, девяностолетняя, неужели родит?
Gen 17:18  И сказал Авраам Богу: о, хотя бы Измаил был жив пред лицем Твоим!
Gen 17:19  Бог же сказал: именно Сарра, жена твоя, родит тебе сына, и ты наречешь ему имя: Исаак; и поставлю завет Мой с ним заветом вечным [и] потомству его после него.
Gen 17:20  И о Измаиле Я услышал тебя: вот, Я благословлю его, и возращу его, и весьма, весьма размножу; двенадцать князей родятся от него; и Я произведу от него великий народ.
Gen 17:21  Но завет Мой поставлю с Исааком, которого родит тебе Сарра в сие самое время на другой год.
Gen 17:22  И Бог перестал говорить с Авраамом и восшел от него.
Gen 17:23  И взял Авраам Измаила, сына своего, и всех рожденных в доме своем и всех купленных за серебро свое, весь мужеский пол людей дома Авраамова; и обрезал крайнюю плоть их в тот самый день, как сказал ему Бог.
Gen 17:24  Авраам был девяноста девяти лет, когда была обрезана крайняя плоть его.
Gen 17:25  А Измаил, сын его, был тринадцати лет, когда была обрезана крайняя плоть его.
Gen 17:26  В тот же самый день обрезаны были Авраам и Измаил, сын его,
Gen 17:27  и с ним обрезан был весь мужеский пол дома его, рожденные в доме и купленные за серебро у иноплеменников.
Gen 18:1  И явился ему Господь у дубравы Мамре, когда он сидел при входе в шатер, во время зноя дневного.
Gen 18:2  Он возвел очи свои и взглянул, и вот, три мужа стоят против него. Увидев, он побежал навстречу им от входа в шатер и поклонился до земли,
Gen 18:3  и сказал: Владыка! если я обрел благоволение пред очами Твоими, не пройди мимо раба Твоего;
Gen 18:4  и принесут немного воды, и омоют ноги ваши; и отдохните под сим деревом,
Gen 18:5  а я принесу хлеба, и вы подкрепите сердца ваши; потом пойдите; так как вы идете мимо раба вашего. Они сказали: сделай так, как говоришь.
Gen 18:6  И поспешил Авраам в шатер к Сарре и сказал: поскорее замеси три саты лучшей муки и сделай пресные хлебы.
Gen 18:7  И побежал Авраам к стаду, и взял теленка нежного и хорошего, и дал отроку, и тот поспешил приготовить его.
Gen 18:8  И взял масла и молока и теленка приготовленного, и поставил перед ними, а сам стоял подле них под деревом. И они ели.
Gen 18:9  И сказали ему: где Сарра, жена твоя? Он отвечал: здесь, в шатре.
Gen 18:10  И сказал [один из них]: Я опять буду у тебя в это же время, и будет сын у Сарры, жены твоей. А Сарра слушала у входа в шатер, сзади его.
Gen 18:11  Авраам же и Сарра были стары и в летах преклонных, и обыкновенное у женщин у Сарры прекратилось.
Gen 18:12  Сарра внутренно рассмеялась, сказав: мне ли, когда я состарилась, иметь сие утешение? и господин мой стар.
Gen 18:13  И сказал Господь Аврааму: отчего это рассмеялась Сарра, сказав: `неужели я действительно могу родить, когда я состарилась'?
Gen 18:14  Есть ли что трудное для Господа? В назначенный срок буду Я у тебя в следующем году, и у Сарры [будет] сын.
Gen 18:15  Сарра же не призналась, а сказала: я не смеялась. Ибо она испугалась. Но Он сказал: нет, ты рассмеялась.
Gen 18:16  И встали те мужи и оттуда отправились к Содому; Авраам же пошел с ними, проводить их.
Gen 18:17  И сказал Господь: утаю ли Я от Авраама, что хочу делать!
Gen 18:18  От Авраама точно произойдет народ великий и сильный, и благословятся в нем все народы земли,
Gen 18:19  ибо Я избрал его для того, чтобы он заповедал сынам своим и дому своему после себя, ходить путем Господним, творя правду и суд; и исполнит Господь над Авраамом, что сказал о нем.
Gen 18:20  И сказал Господь: вопль Содомский и Гоморрский, велик он, и грех их, тяжел он весьма;
Gen 18:21  сойду и посмотрю, точно ли они поступают так, каков вопль на них, восходящий ко Мне, или нет; узнаю.
Gen 18:22  И обратились мужи оттуда и пошли в Содом; Авраам же еще стоял пред лицем Господа.
Gen 18:23  И подошел Авраам и сказал: неужели Ты погубишь праведного с нечестивым?
Gen 18:24  может быть, есть в этом городе пятьдесят праведников? неужели Ты погубишь, и не пощадишь места сего ради пятидесяти праведников, в нем?
Gen 18:25  не может быть, чтобы Ты поступил так, чтобы Ты погубил праведного с нечестивым, чтобы то же было с праведником, что с нечестивым; не может быть от Тебя! Судия всей земли поступит ли неправосудно?
Gen 18:26  Господь сказал: если Я найду в городе Содоме пятьдесят праведников, то Я ради них пощажу все место сие.
Gen 18:27  Авраам сказал в ответ: вот, я решился говорить Владыке, я, прах и пепел:
Gen 18:28  может быть, до пятидесяти праведников недостанет пяти, неужели за [недостатком] пяти Ты истребишь весь город? Он сказал: не истреблю, если найду там сорок пять.
Gen 18:29  [Авраам] продолжал говорить с Ним и сказал: может быть, найдется там сорок? Он сказал: не сделаю [того] и ради сорока.
Gen 18:30  И сказал [Авраам]: да не прогневается Владыка, что я буду говорить: может быть, найдется там тридцать? Он сказал: не сделаю, если найдется там тридцать.
Gen 18:31  [Авраам] сказал: вот, я решился говорить Владыке: может быть, найдется там двадцать? Он сказал: не истреблю ради двадцати.
Gen 18:32  [Авраам] сказал: да не прогневается Владыка, что я скажу еще однажды: может быть, найдется там десять? Он сказал: не истреблю ради десяти.
Gen 18:33  И пошел Господь, перестав говорить с Авраамом; Авраам же возвратился в свое место.
Gen 19:1  И пришли те два Ангела в Содом вечером, когда Лот сидел у ворот Содома. Лот увидел, и встал, чтобы встретить их, и поклонился лицем до земли
Gen 19:2  и сказал: государи мои! зайдите в дом раба вашего и ночуйте, и умойте ноги ваши, и встаньте поутру и пойдете в путь свой. Но они сказали: нет, мы ночуем на улице.
Gen 19:3  Он же сильно упрашивал их; и они пошли к нему и пришли в дом его. Он сделал им угощение и испек пресные хлебы, и они ели.
Gen 19:4  Еще не легли они спать, как городские жители, Содомляне, от молодого до старого, весь народ со [всех] концов [города], окружили дом
Gen 19:5  и вызвали Лота и говорили ему: где люди, пришедшие к тебе на ночь? выведи их к нам; мы познаем их.
Gen 19:6  Лот вышел к ним ко входу, и запер за собою дверь,
Gen 19:7  и сказал: братья мои, не делайте зла;
Gen 19:8  вот у меня две дочери, которые не познали мужа; лучше я выведу их к вам, делайте с ними, что вам угодно, только людям сим не делайте ничего, так как они пришли под кров дома моего.
Gen 19:9  Но они сказали: пойди сюда. И сказали: вот пришлец, и хочет судить? теперь мы хуже поступим с тобою, нежели с ними. И очень приступали к человеку сему, к Лоту, и подошли, чтобы выломать дверь.
Gen 19:10  Тогда мужи те простерли руки свои и ввели Лота к себе в дом, и дверь заперли;
Gen 19:11  а людей, бывших при входе в дом, поразили слепотою, от малого до большого, так что они измучились, искав входа.
Gen 19:12  Сказали мужи те Лоту: кто у тебя есть еще здесь? зять ли, сыновья ли твои, дочери ли твои, и кто бы ни был у тебя в городе, всех выведи из сего места,
Gen 19:13  ибо мы истребим сие место, потому что велик вопль на жителей его к Господу, и Господь послал нас истребить его.
Gen 19:14  И вышел Лот, и говорил с зятьями своими, которые брали за себя дочерей его, и сказал: встаньте, выйдите из сего места, ибо Господь истребит сей город. Но зятьям его показалось, что он шутит.
Gen 19:15  Когда взошла заря, Ангелы начали торопить Лота, говоря: встань, возьми жену твою и двух дочерей твоих, которые у тебя, чтобы не погибнуть тебе за беззакония города.
Gen 19:16  И как он медлил, то мужи те, по милости к нему Господней, взяли за руку его и жену его, и двух дочерей его, и вывели его и поставили его вне города.
Gen 19:17  Когда же вывели их вон, [то один из них] сказал: спасай душу свою; не оглядывайся назад и нигде не останавливайся в окрестности сей; спасайся на гору, чтобы тебе не погибнуть.
Gen 19:18  Но Лот сказал им: нет, Владыка!
Gen 19:19  вот, раб Твой обрел благоволение пред очами Твоими, и велика милость Твоя, которую Ты сделал со мною, что спас жизнь мою; но я не могу спасаться на гору, чтоб не застигла меня беда и мне не умереть;
Gen 19:20  вот, ближе бежать в сей город, он же мал; побегу я туда, --он же мал; и сохранится жизнь моя.
Gen 19:21  И сказал ему: вот, в угодность тебе Я сделаю и это: не ниспровергну города, о котором ты говоришь;
Gen 19:22  поспешай, спасайся туда, ибо Я не могу сделать дела, доколе ты не придешь туда. Потому и назван город сей: Сигор.
Gen 19:23  Солнце взошло над землею, и Лот пришел в Сигор.
Gen 19:24  И пролил Господь на Содом и Гоморру дождем серу и огонь от Господа с неба,
Gen 19:25  и ниспроверг города сии, и всю окрестность сию, и всех жителей городов сих, и произрастания земли.
Gen 19:26  Жена же [Лотова] оглянулась позади его, и стала соляным столпом.
Gen 19:27  И встал Авраам рано утром и [пошел] на место, где стоял пред лицем Господа,
Gen 19:28  и посмотрел к Содому и Гоморре и на все пространство окрестности и увидел: вот, дым поднимается с земли, как дым из печи.
Gen 19:29  И было, когда Бог истреблял города окрестности сей, вспомнил Бог об Аврааме и выслал Лота из среды истребления, когда ниспровергал города, в которых жил Лот.
Gen 19:30  И вышел Лот из Сигора и стал жить в горе, и с ним две дочери его, ибо он боялся жить в Сигоре. И жил в пещере, и с ним две дочери его.
Gen 19:31  И сказала старшая младшей: отец наш стар, и нет человека на земле, который вошел бы к нам по обычаю всей земли;
Gen 19:32  итак напоим отца нашего вином, и переспим с ним, и восставим от отца нашего племя.
Gen 19:33  И напоили отца своего вином в ту ночь; и вошла старшая и спала с отцом своим: а он не знал, когда она легла и когда встала.
Gen 19:34  На другой день старшая сказала младшей: вот, я спала вчера с отцом моим; напоим его вином и в эту ночь; и ты войди, спи с ним, и восставим от отца нашего племя.
Gen 19:35  И напоили отца своего вином и в эту ночь; и вошла младшая и спала с ним; и он не знал, когда она легла и когда встала.
Gen 19:36  И сделались обе дочери Лотовы беременными от отца своего,
Gen 19:37  и родила старшая сына, и нарекла ему имя: Моав. Он отец Моавитян доныне.
Gen 19:38  И младшая также родила сына, и нарекла ему имя: Бен-Амми. Он отец Аммонитян доныне.
Gen 20:1  Авраам поднялся оттуда к югу и поселился между Кадесом и между Суром; и был на время в Гераре.
Gen 20:2  И сказал Авраам о Сарре, жене своей: она сестра моя. И послал Авимелех, царь Герарский, и взял Сарру.
Gen 20:3  И пришел Бог к Авимелеху ночью во сне и сказал ему: вот, ты умрешь за женщину, которую ты взял, ибо она имеет мужа.
Gen 20:4  Авимелех же не прикасался к ней и сказал: Владыка! неужели ты погубишь и невинный народ?
Gen 20:5  Не сам ли он сказал мне: она сестра моя? И она сама сказала: он брат мой. Я сделал это в простоте сердца моего и в чистоте рук моих.
Gen 20:6  И сказал ему Бог во сне: и Я знаю, что ты сделал сие в простоте сердца твоего, и удержал тебя от греха предо Мною, потому и не допустил тебя прикоснуться к ней;
Gen 20:7  теперь же возврати жену мужу, ибо он пророк и помолится о тебе, и ты будешь жив; а если не возвратишь, то знай, что непременно умрешь ты и все твои.
Gen 20:8  И встал Авимелех утром рано, и призвал всех рабов своих, и пересказал все слова сии в уши их; и люди сии весьма испугались.
Gen 20:9  И призвал Авимелех Авраама и сказал ему: что ты с нами сделал? чем согрешил я против тебя, что ты навел было на меня и на царство мое великий грех? Ты сделал со мною дела, каких не делают.
Gen 20:10  И сказал Авимелех Аврааму: что ты имел в виду, когда делал это дело?
Gen 20:11  Авраам сказал: я подумал, что нет на месте сем страха Божия, и убьют меня за жену мою;
Gen 20:12  да она и подлинно сестра мне: она дочь отца моего, только не дочь матери моей; и сделалась моею женою;
Gen 20:13  когда Бог повел меня странствовать из дома отца моего, то я сказал ей: сделай со мною сию милость, в какое ни придем мы место, везде говори обо мне: это брат мой.
Gen 20:14  И взял Авимелех мелкого и крупного скота, и рабов и рабынь, и дал Аврааму; и возвратил ему Сарру, жену его.
Gen 20:15  И сказал Авимелех: вот, земля моя пред тобою; живи, где тебе угодно.
Gen 20:16  И Сарре сказал: вот, я дал брату твоему тысячу [сиклей] серебра; вот, это тебе покрывало для очей пред всеми, которые с тобою, и пред всеми ты оправдана.
Gen 20:17  И помолился Авраам Богу, и исцелил Бог Авимелеха, и жену его, и рабынь его, и они стали рождать;
Gen 20:18  ибо заключил Господь всякое чрево в доме Авимелеха за Сарру, жену Авраамову.
Gen 21:1  И призрел Господь на Сарру, как сказал; и сделал Господь Сарре, как говорил.
Gen 21:2  Сарра зачала и родила Аврааму сына в старости его во время, о котором говорил ему Бог;
Gen 21:3  и нарек Авраам имя сыну своему, родившемуся у него, которого родила ему Сарра, Исаак;
Gen 21:4  и обрезал Авраам Исаака, сына своего, в восьмой день, как заповедал ему Бог.
Gen 21:5  Авраам был ста лет, когда родился у него Исаак, сын его.
Gen 21:6  И сказала Сарра: смех сделал мне Бог; кто ни услышит обо мне, рассмеется.
Gen 21:7  И сказала: кто сказал бы Аврааму: Сарра будет кормить детей грудью? ибо в старости его я родила сына.
Gen 21:8  Дитя выросло и отнято от груди; и Авраам сделал большой пир в тот день, когда Исаак отнят был от груди.
Gen 21:9  И увидела Сарра, что сын Агари Египтянки, которого она родила Аврааму, насмехается,
Gen 21:10  и сказала Аврааму: выгони эту рабыню и сына ее, ибо не наследует сын рабыни сей с сыном моим Исааком.
Gen 21:11  И показалось это Аврааму весьма неприятным ради сына его.
Gen 21:12  Но Бог сказал Аврааму: не огорчайся ради отрока и рабыни твоей; во всем, что скажет тебе Сарра, слушайся голоса ее, ибо в Исааке наречется тебе семя;
Gen 21:13  и от сына рабыни Я произведу народ, потому что он семя твое.
Gen 21:14  Авраам встал рано утром, и взял хлеба и мех воды, и дал Агари, положив ей на плечи, и отрока, и отпустил ее. Она пошла, и заблудилась в пустыне Вирсавии;
Gen 21:15  и не стало воды в мехе, и она оставила отрока под одним кустом
Gen 21:16  и пошла, села вдали, в расстоянии на [один] выстрел из лука. Ибо она сказала: не [хочу] видеть смерти отрока. И она села против, и подняла вопль, и плакала;
Gen 21:17  и услышал Бог голос отрока; и Ангел Божий с неба воззвал к Агари и сказал ей: что с тобою, Агарь? не бойся; Бог услышал голос отрока оттуда, где он находится;
Gen 21:18  встань, подними отрока и возьми его за руку, ибо Я произведу от него великий народ.
Gen 21:19  И Бог открыл глаза ее, и она увидела колодезь с водою, и пошла, наполнила мех водою и напоила отрока.
Gen 21:20  И Бог был с отроком; и он вырос, и стал жить в пустыне, и сделался стрелком из лука.
Gen 21:21  Он жил в пустыне Фаран; и мать его взяла ему жену из земли Египетской.
Gen 21:22  И было в то время, Авимелех с Фихолом, военачальником своим, сказал Аврааму: с тобою Бог во всем, что ты ни делаешь;
Gen 21:23  и теперь поклянись мне здесь Богом, что ты не обидишь ни меня, ни сына моего, ни внука моего; и как я хорошо поступал с тобою, так и ты будешь поступать со мною и землею, в которой ты гостишь.
Gen 21:24  И сказал Авраам: я клянусь.
Gen 21:25  И Авраам упрекал Авимелеха за колодезь с водою, который отняли рабы Авимелеховы.
Gen 21:26  Авимелех же сказал: не знаю, кто это сделал, и ты не сказал мне; я даже и не слыхал [о том] доныне.
Gen 21:27  И взял Авраам мелкого и крупного скота и дал Авимелеху, и они оба заключили союз.
Gen 21:28  И поставил Авраам семь агниц из [стада] мелкого скота особо.
Gen 21:29  Авимелех же сказал Аврааму: на что здесь сии семь агниц, которых ты поставил особо?
Gen 21:30  [он] сказал: семь агниц сих возьми от руки моей, чтобы они были мне свидетельством, что я выкопал этот колодезь.
Gen 21:31  Потому и назвал он сие место: Вирсавия, ибо тут оба они клялись
Gen 21:32  и заключили союз в Вирсавии. И встал Авимелех, и Фихол, военачальник его, и возвратились в землю Филистимскую.
Gen 21:33  И насадил [Авраам] при Вирсавии рощу и призвал там имя Господа, Бога вечного.
Gen 21:34  И жил Авраам в земле Филистимской, как странник, дни многие.
Gen 22:1  И было, после сих происшествий Бог искушал Авраама и сказал ему: Авраам! Он сказал: вот я.
Gen 22:2  [Бог] сказал: возьми сына твоего, единственного твоего, которого ты любишь, Исаака; и пойди в землю Мориа и там принеси его во всесожжение на одной из гор, о которой Я скажу тебе.
Gen 22:3  Авраам встал рано утром, оседлал осла своего, взял с собою двоих из отроков своих и Исаака, сына своего; наколол дров для всесожжения, и встав пошел на место, о котором сказал ему Бог.
Gen 22:4  На третий день Авраам возвел очи свои, и увидел то место издалека.
Gen 22:5  И сказал Авраам отрокам своим: останьтесь вы здесь с ослом, а я и сын пойдем туда и поклонимся, и возвратимся к вам.
Gen 22:6  И взял Авраам дрова для всесожжения, и возложил на Исаака, сына своего; взял в руки огонь и нож, и пошли оба вместе.
Gen 22:7  И начал Исаак говорить Аврааму, отцу своему, и сказал: отец мой! Он отвечал: вот я, сын мой. Он сказал: вот огонь и дрова, где же агнец для всесожжения?
Gen 22:8  Авраам сказал: Бог усмотрит Себе агнца для всесожжения, сын мой. И шли [далее] оба вместе.
Gen 22:9  И пришли на место, о котором сказал ему Бог; и устроил там Авраам жертвенник, разложил дрова и, связав сына своего Исаака, положил его на жертвенник поверх дров.
Gen 22:10  И простер Авраам руку свою и взял нож, чтобы заколоть сына своего.
Gen 22:11  Но Ангел Господень воззвал к нему с неба и сказал: Авраам! Авраам! Он сказал: вот я.
Gen 22:12  [Ангел] сказал: не поднимай руки твоей на отрока и не делай над ним ничего, ибо теперь Я знаю, что боишься ты Бога и не пожалел сына твоего, единственного твоего, для Меня.
Gen 22:13  И возвел Авраам очи свои и увидел: и вот, позади овен, запутавшийся в чаще рогами своими. Авраам пошел, взял овна и принес его во всесожжение вместо сына своего.
Gen 22:14  И нарек Авраам имя месту тому: Иегова-ире. Посему [и] ныне говорится: на горе Иеговы усмотрится.
Gen 22:15  И вторично воззвал к Аврааму Ангел Господень с неба
Gen 22:16  и сказал: Мною клянусь, говорит Господь, что, так как ты сделал сие дело, и не пожалел сына твоего, единственного твоего,
Gen 22:17  то Я благословляя благословлю тебя и умножая умножу семя твое, как звезды небесные и как песок на берегу моря; и овладеет семя твое городами врагов своих;
Gen 22:18  и благословятся в семени твоем все народы земли за то, что ты послушался гласа Моего.
Gen 22:19  И возвратился Авраам к отрокам своим, и встали и пошли вместе в Вирсавию; и жил Авраам в Вирсавии.
Gen 22:20  После сих происшествий Аврааму возвестили, сказав: вот, и Милка родила Нахору, брату твоему, сынов:
Gen 22:21  Уца, первенца его, Вуза, брата сему, Кемуила, отца Арамова,
Gen 22:22  Кеседа, Хазо, Пилдаша, Идлафа и Вафуила;
Gen 22:23  от Вафуила родилась Ревекка. Восьмерых сих родила Милка Нахору, брату Авраамову;
Gen 22:24  и наложница его, именем Реума, также родила Теваха, Гахама, Тахаша и Мааху.
Gen 23:1  Жизни Сарриной было сто двадцать семь лет: [вот] лета жизни Сарриной;
Gen 23:2  и умерла Сарра в Кириаф-Арбе, что [ныне] Хеврон, в земле Ханаанской. И пришел Авраам рыдать по Сарре и оплакивать ее.
Gen 23:3  И отошел Авраам от умершей своей, и говорил сынам Хетовым, и сказал:
Gen 23:4  я у вас пришлец и поселенец; дайте мне в собственность [место] [для] гроба между вами, чтобы мне умершую мою схоронить от глаз моих.
Gen 23:5  Сыны Хета отвечали Аврааму и сказали ему:
Gen 23:6  послушай нас, господин наш; ты князь Божий посреди нас; в лучшем из наших погребальных мест похорони умершую твою; никто из нас не откажет тебе в погребальном месте, для погребения умершей твоей.
Gen 23:7  Авраам встал и поклонился народу земли той, сынам Хетовым;
Gen 23:8  и говорил им и сказал: если вы согласны, чтобы я похоронил умершую мою, то послушайте меня, попросите за меня Ефрона, сына Цохарова,
Gen 23:9  чтобы он отдал мне пещеру Махпелу, которая у него на конце поля его, чтобы за довольную цену отдал ее мне посреди вас, в собственность для погребения.
Gen 23:10  Ефрон же сидел посреди сынов Хетовых; и отвечал Ефрон Хеттеянин Аврааму вслух сынов Хета, всех входящих во врата города его, и сказал:
Gen 23:11  нет, господин мой, послушай меня: я даю тебе поле и пещеру, которая на нем, даю тебе, пред очами сынов народа моего дарю тебе ее, похорони умершую твою.
Gen 23:12  Авраам поклонился пред народом земли той
Gen 23:13  и говорил Ефрону вслух народа земли той и сказал: если послушаешь, я даю тебе за поле серебро; возьми у меня, и я похороню там умершую мою.
Gen 23:14  Ефрон отвечал Аврааму и сказал ему:
Gen 23:15  господин мой! послушай меня: земля [стоит] четыреста сиклей серебра; для меня и для тебя что это? похорони умершую твою.
Gen 23:16  Авраам выслушал Ефрона; и отвесил Авраам Ефрону серебра, сколько он объявил вслух сынов Хетовых, четыреста сиклей серебра, какое ходит у купцов.
Gen 23:17  И стало поле Ефроново, которое при Махпеле, против Мамре, поле и пещера, которая на нем, и все деревья, которые на поле, во всех пределах его вокруг,
Gen 23:18  владением Авраамовым пред очами сынов Хета, всех входящих во врата города его.
Gen 23:19  После сего Авраам похоронил Сарру, жену свою, в пещере поля в Махпеле, против Мамре, что [ныне] Хеврон, в земле Ханаанской.
Gen 23:20  Так достались Аврааму от сынов Хетовых поле и пещера, которая на нем, в собственность для погребения.
Gen 24:1  Авраам был уже стар и в летах преклонных. Господь благословил Авраама всем.
Gen 24:2  И сказал Авраам рабу своему, старшему в доме его, управлявшему всем, что у него было: положи руку твою под стегно мое
Gen 24:3  и клянись мне Господом, Богом неба и Богом земли, что ты не возьмешь сыну моему жены из дочерей Хананеев, среди которых я живу,
Gen 24:4  но пойдешь в землю мою, на родину мою, и возьмешь жену сыну моему Исааку.
Gen 24:5  Раб сказал ему: может быть, не захочет женщина идти со мною в эту землю, должен ли я возвратить сына твоего в землю, из которой ты вышел?
Gen 24:6  Авраам сказал ему: берегись, не возвращай сына моего туда;
Gen 24:7  Господь, Бог неба, Который взял меня из дома отца моего и из земли рождения моего, Который говорил мне и Который клялся мне, говоря: `потомству твоему дам сию землю', --Он пошлет Ангела Своего пред тобою, и ты возьмешь жену сыну моему оттуда;
Gen 24:8  если же не захочет женщина идти с тобою, ты будешь свободен от сей клятвы моей; только сына моего не возвращай туда.
Gen 24:9  И положил раб руку свою под стегно Авраама, господина своего, и клялся ему в сем.
Gen 24:10  И взял раб из верблюдов господина своего десять верблюдов и пошел. В руках у него были также всякие сокровища господина его. Он встал и пошел в Месопотамию, в город Нахора,
Gen 24:11  и остановил верблюдов вне города, у колодезя воды, под вечер, в то время, когда выходят женщины черпать,
Gen 24:12  и сказал: Господи, Боже господина моего Авраама! пошли [ее] сегодня навстречу мне и сотвори милость с господином моим Авраамом;
Gen 24:13  вот, я стою у источника воды, и дочери жителей города выходят черпать воду;
Gen 24:14  и девица, которой я скажу: `наклони кувшин твой, я напьюсь', и которая скажет: `пей, я и верблюдам твоим дам пить', --вот та, которую Ты назначил рабу Твоему Исааку; и по сему узнаю я, что Ты творишь милость с господином моим.
Gen 24:15  Еще не перестал он говорить, и вот, вышла Ревекка, которая родилась от Вафуила, сына Милки, жены Нахора, брата Авраамова, и кувшин ее на плече ее;
Gen 24:16  девица [была] прекрасна видом, дева, которой не познал муж. Она сошла к источнику, наполнила кувшин свой и пошла вверх.
Gen 24:17  И побежал раб навстречу ей и сказал: дай мне испить немного воды из кувшина твоего.
Gen 24:18  Она сказала: пей, господин мой. И тотчас спустила кувшин свой на руку свою и напоила его.
Gen 24:19  И, когда напоила его, сказала: я стану черпать и для верблюдов твоих, пока не напьются.
Gen 24:20  И тотчас вылила воду из кувшина своего в поило и побежала опять к колодезю почерпнуть, и начерпала для всех верблюдов его.
Gen 24:21  Человек тот смотрел на нее с изумлением в молчании, желая уразуметь, благословил ли Господь путь его, или нет.
Gen 24:22  Когда верблюды перестали пить, тогда человек тот взял золотую серьгу, весом полсикля, и два запястья на руки ей, весом в десять [сиклей] золота;
Gen 24:23  И сказал: чья ты дочь? скажи мне, есть ли в доме отца твоего место нам ночевать?
Gen 24:24  Она сказала ему: я дочь Вафуила, сына Милки, которого она родила Нахору.
Gen 24:25  И еще сказала ему: у нас много соломы и корму, и [есть] место для ночлега.
Gen 24:26  И преклонился человек тот и поклонился Господу,
Gen 24:27  и сказал: благословен Господь Бог господина моего Авраама, Который не оставил господина моего милостью Своею и истиною Своею! Господь прямым путем привел меня к дому брата господина моего.
Gen 24:28  Девица побежала и рассказала об этом в доме матери своей.
Gen 24:29  У Ревекки был брат, именем Лаван. Лаван выбежал к тому человеку, к источнику.
Gen 24:30  И когда он увидел серьгу и запястья на руках у сестры своей и услышал слова Ревекки, сестры своей, которая говорила: так говорил со мною этот человек, --то пришел к человеку, и вот, он стоит при верблюдах у источника;
Gen 24:31  и сказал: войди, благословенный Господом; зачем ты стоишь вне? я приготовил дом и место для верблюдов.
Gen 24:32  И вошел человек. [Лаван] расседлал верблюдов и дал соломы и корму верблюдам, и воды умыть ноги ему и людям, которые были с ним;
Gen 24:33  и предложена была ему пища; но он сказал: не стану есть, доколе не скажу дела своего. И сказали: говори.
Gen 24:34  Он сказал: я раб Авраамов;
Gen 24:35  Господь весьма благословил господина моего, и он сделался великим: Он дал ему овец и волов, серебро и золото, рабов и рабынь, верблюдов и ослов;
Gen 24:36  Сарра, жена господина моего, уже состарившись, родила господину моему сына, которому он отдал все, что у него;
Gen 24:37  и взял с меня клятву господин мой, сказав: не бери жены сыну моему из дочерей Хананеев, в земле которых я живу,
Gen 24:38  а пойди в дом отца моего и к родственникам моим, и возьмешь жену сыну моему.
Gen 24:39  Я сказал господину моему: может быть, не пойдет женщина со мною.
Gen 24:40  Он сказал мне: Господь, пред лицем Которого я хожу, пошлет с тобою Ангела Своего и благоустроит путь твой, и возьмешь жену сыну моему из родных моих и из дома отца моего;
Gen 24:41  тогда будешь ты свободен от клятвы моей, когда сходишь к родственникам моим; и если они не дадут тебе, то будешь свободен от клятвы моей.
Gen 24:42  И пришел я ныне к источнику, и сказал: Господи, Боже господина моего Авраама! Если Ты благоустроишь путь, который я совершаю,
Gen 24:43  то вот, я стою у источника воды, и девица, которая выйдет почерпать, и которой я скажу: дай мне испить немного из кувшина твоего,
Gen 24:44  и которая скажет мне: `и ты пей, и верблюдам твоим я начерпаю' --вот жена, которую Господь назначил сыну господина моего.
Gen 24:45  Еще не перестал я говорить в уме моем, и вот вышла Ревекка, и кувшин ее на плече ее, и сошла к источнику и почерпнула; и я сказал ей: напой меня.
Gen 24:46  Она тотчас спустила с себя кувшин свой и сказала: пей, и верблюдов твоих я напою. И я пил, и верблюдов она напоила.
Gen 24:47  Я спросил ее и сказал: чья ты дочь? Она сказала: дочь Вафуила, сына Нахорова, которого родила ему Милка. И дал я серьги ей и запястья на руки ее.
Gen 24:48  И преклонился я и поклонился Господу, и благословил Господа, Бога господина моего Авраама, Который прямым путем привел меня, чтобы взять дочь брата господина моего за сына его.
Gen 24:49  И ныне скажите мне: намерены ли вы оказать милость и правду господину моему или нет? скажите мне, и я обращусь направо, или налево.
Gen 24:50  И отвечали Лаван и Вафуил и сказали: от Господа пришло это дело; мы не можем сказать тебе вопреки ни худого, ни доброго;
Gen 24:51  вот Ревекка пред тобою; возьми и пойди; пусть будет она женою сыну господина твоего, как сказал Господь.
Gen 24:52  Когда раб Авраамов услышал слова их, то поклонился Господу до земли.
Gen 24:53  И вынул раб серебряные вещи и золотые вещи и одежды и дал Ревекке; также и брату ее и матери ее дал богатые подарки.
Gen 24:54  И ели и пили он и люди, бывшие с ним, и переночевали. Когда же встали поутру, то он сказал: отпустите меня к господину моему.
Gen 24:55  Но брат ее и мать ее сказали: пусть побудет с нами девица дней хотя десять, потом пойдешь.
Gen 24:56  Он сказал им: не удерживайте меня, ибо Господь благоустроил путь мой; отпустите меня, и я пойду к господину моему.
Gen 24:57  Они сказали: призовем девицу и спросим, что она скажет.
Gen 24:58  И призвали Ревекку и сказали ей: пойдешь ли с этим человеком? Она сказала: пойду.
Gen 24:59  И отпустили Ревекку, сестру свою, и кормилицу ее, и раба Авраамова, и людей его.
Gen 24:60  И благословили Ревекку и сказали ей: сестра наша! да родятся от тебя тысячи тысяч, и да владеет потомство твое жилищами врагов твоих!
Gen 24:61  И встала Ревекка и служанки ее, и сели на верблюдов, и поехали за тем человеком. И раб взял Ревекку и пошел.
Gen 24:62  А Исаак пришел из Беэр-лахай-рои, ибо жил он в земле полуденной.
Gen 24:63  При наступлении вечера Исаак вышел в поле поразмыслить, и возвел очи свои, и увидел: вот, идут верблюды.
Gen 24:64  Ревекка взглянула, и увидела Исаака, и спустилась с верблюда.
Gen 24:65  И сказала рабу: кто этот человек, который идет по полю навстречу нам? Раб сказал: это господин мой. И она взяла покрывало и покрылась.
Gen 24:66  Раб же сказал Исааку все, что сделал.
Gen 24:67  И ввел ее Исаак в шатер Сарры, матери своей, и взял Ревекку, и она сделалась ему женою, и он возлюбил ее; и утешился Исаак в [печали] по матери своей.
Gen 25:1  И взял Авраам еще жену, именем Хеттуру.
Gen 25:2  Она родила ему Зимрана, Иокшана, Медана, Мадиана, Ишбака и Шуаха.
Gen 25:3  Иокшан родил Шеву и Дедана. Сыны Дедана были: Ашурим, Летушим и Леюмим.
Gen 25:4  Сыны Мадиана: Ефа, Ефер, Ханох, Авида и Елдага. Все сии сыны Хеттуры.
Gen 25:5  И отдал Авраам все, что было у него, Исааку,
Gen 25:6  а сынам наложниц, которые были у Авраама, дал Авраам подарки и отослал их от Исаака, сына своего, еще при жизни своей, на восток, в землю восточную.
Gen 25:7  Дней жизни Авраамовой, которые он прожил, было сто семьдесят пять лет;
Gen 25:8  и скончался Авраам, и умер в старости доброй, престарелый и насыщенный [жизнью], и приложился к народу своему.
Gen 25:9  И погребли его Исаак и Измаил, сыновья его, в пещере Махпеле, на поле Ефрона, сына Цохара, Хеттеянина, которое против Мамре,
Gen 25:10  на поле, которые Авраам приобрел от сынов Хетовых. Там погребены Авраам и Сарра, жена его.
Gen 25:11  По смерти Авраама Бог благословил Исаака, сына его. Исаак жил при Беэр-лахай-рои.
Gen 25:12  Вот родословие Измаила, сына Авраамова, которого родила Аврааму Агарь Египтянка, служанка Саррина;
Gen 25:13  и вот имена сынов Измаиловых, имена их по родословию их: первенец Измаилов Наваиоф, [за ним] Кедар, Адбеел, Мивсам,
Gen 25:14  Мишма, Дума, Масса,
Gen 25:15  Хадад, Фема, Иетур, Нафиш и Кедма.
Gen 25:16  Сии суть сыны Измаиловы, и сии имена их, в селениях их, в кочевьях их. [Это] двенадцать князей племен их.
Gen 25:17  Лет же жизни Измаиловой было сто тридцать семь лет; и скончался он, и умер, и приложился к народу своему.
Gen 25:18  Они жили от Хавилы до Сура, что пред Египтом, как идешь к Ассирии. Они поселились пред лицем всех братьев своих.
Gen 25:19  Вот родословие Исаака, сына Авраамова. Авраам родил Исаака.
Gen 25:20  Исаак был сорока лет, когда он взял себе в жену Ревекку, дочь Вафуила Арамеянина из Месопотамии, сестру Лавана Арамеянина.
Gen 25:21  И молился Исаак Господу о жене своей, потому что она была неплодна; и Господь услышал его, и зачала Ревекка, жена его.
Gen 25:22  Сыновья в утробе ее стали биться, и она сказала: если так будет, то для чего мне это? И пошла вопросить Господа.
Gen 25:23  Господь сказал ей: два племени во чреве твоем, и два различных народа произойдут из утробы твоей; один народ сделается сильнее другого, и больший будет служить меньшему.
Gen 25:24  И настало время родить ей: и вот близнецы в утробе ее.
Gen 25:25  Первый вышел красный, весь, как кожа, косматый; и нарекли ему имя Исав.
Gen 25:26  Потом вышел брат его, держась рукою своею за пяту Исава; и наречено ему имя Иаков. Исаак же был шестидесяти лет, когда они родились.
Gen 25:27  Дети выросли, и стал Исав человеком искусным в звероловстве, человеком полей; а Иаков человеком кротким, живущим в шатрах.
Gen 25:28  Исаак любил Исава, потому что дичь его была по вкусу его, а Ревекка любила Иакова.
Gen 25:29  И сварил Иаков кушанье; а Исав пришел с поля усталый.
Gen 25:30  И сказал Исав Иакову: дай мне поесть красного, красного этого, ибо я устал. От сего дано ему прозвание: Едом.
Gen 25:31  Но Иаков сказал: продай мне теперь же свое первородство.
Gen 25:32  Исав сказал: вот, я умираю, что мне в этом первородстве?
Gen 25:33  Иаков сказал: поклянись мне теперь же. Он поклялся ему, и продал первородство свое Иакову.
Gen 25:34  И дал Иаков Исаву хлеба и кушанья из чечевицы; и он ел и пил, и встал и пошел; и пренебрег Исав первородство.
Gen 26:1  Был голод в земле, сверх прежнего голода, который был во дни Авраама; и пошел Исаак к Авимелеху, царю Филистимскому, в Герар.
Gen 26:2  Господь явился ему и сказал: не ходи в Египет; живи в земле, о которой Я скажу тебе,
Gen 26:3  странствуй по сей земле, и Я буду с тобою и благословлю тебя, ибо тебе и потомству твоему дам все земли сии и исполню клятву, которою Я клялся Аврааму, отцу твоему;
Gen 26:4  умножу потомство твое, как звезды небесные, и дам потомству твоему все земли сии; благословятся в семени твоем все народы земные,
Gen 26:5  за то, что Авраам послушался гласа Моего и соблюдал, что Мною [заповедано] было соблюдать: повеления Мои, уставы Мои и законы Мои.
Gen 26:6  Исаак поселился в Гераре.
Gen 26:7  Жители места того спросили о жене его, и он сказал: это сестра моя; потому что боялся сказать: жена моя, чтобы не убили меня, [думал он], жители места сего за Ревекку, потому что она прекрасна видом.
Gen 26:8  Но когда уже много времени он там прожил, Авимелех, царь Филистимский, посмотрев в окно, увидел, что Исаак играет с Ревеккою, женою своею.
Gen 26:9  И призвал Авимелех Исаака и сказал: вот, это жена твоя; как же ты сказал: она сестра моя? Исаак сказал ему: потому что я думал, не умереть бы мне ради ее.
Gen 26:10  Но Авимелех сказал: что это ты сделал с нами? едва один из народа не совокупился с женою твоею, и ты ввел бы нас в грех.
Gen 26:11  И дал Авимелех повеление всему народу, сказав: кто прикоснется к сему человеку и к жене его, тот предан будет смерти.
Gen 26:12  И сеял Исаак в земле той и получил в тот год ячменя во сто крат: так благословил его Господь.
Gen 26:13  И стал великим человек сей и возвеличивался больше и больше до того, что стал весьма великим.
Gen 26:14  У него были стада мелкого и стада крупного скота и множество пахотных полей, и Филистимляне стали завидовать ему.
Gen 26:15  И все колодези, которые выкопали рабы отца его при жизни отца его Авраама, Филистимляне завалили и засыпали землею.
Gen 26:16  И Авимелех сказал Исааку: удались от нас, ибо ты сделался гораздо сильнее нас.
Gen 26:17  И Исаак удалился оттуда, и расположился шатрами в долине Герарской, и поселился там.
Gen 26:18  И вновь выкопал Исаак колодези воды, которые выкопаны были во дни Авраама, отца его, и которые завалили Филистимляне по смерти Авраама; и назвал их теми же именами, которыми назвал их отец его.
Gen 26:19  И копали рабы Исааковы в долине и нашли там колодезь воды живой.
Gen 26:20  И спорили пастухи Герарские с пастухами Исаака, говоря: наша вода. И он нарек колодезю имя: Есек, потому что спорили с ним.
Gen 26:21  выкопали другой колодезь; спорили также и о нем; и он нарек ему имя: Ситна.
Gen 26:22  И он двинулся отсюда и выкопал иной колодезь, о котором уже не спорили, и нарек ему имя: Реховоф, ибо, сказал он, теперь Господь дал нам пространное место, и мы размножимся на земле.
Gen 26:23  Оттуда перешел он в Вирсавию.
Gen 26:24  И в ту ночь явился ему Господь и сказал: Я Бог Авраама, отца твоего; не бойся, ибо Я с тобою; и благословлю тебя и умножу потомство твое, ради Авраама, раба Моего.
Gen 26:25  И он устроил там жертвенник и призвал имя Господа. И раскинул там шатер свой, и выкопали там рабы Исааковы колодезь.
Gen 26:26  Пришел к нему из Герара Авимелех и Ахузаф, друг его, и Фихол, военачальник его.
Gen 26:27  Исаак сказал им: для чего вы пришли ко мне, когда вы возненавидели меня и выслали меня от себя?
Gen 26:28  Они сказали: мы ясно увидели, что Господь с тобою, и потому мы сказали: поставим между нами и тобою клятву и заключим с тобою союз,
Gen 26:29  чтобы ты не делал нам зла, как и мы не коснулись до тебя, а делали тебе одно доброе и отпустили тебя с миром; теперь ты благословен Господом.
Gen 26:30  Он сделал им пиршество, и они ели и пили.
Gen 26:31  И встав рано утром, поклялись друг другу; и отпустил их Исаак, и они пошли от него с миром.
Gen 26:32  В тот же день пришли рабы Исааковы и известили его о колодезе, который копали они, и сказали ему: мы нашли воду.
Gen 26:33  И он назвал его: Шива. Посему имя городу тому Беэршива до сего дня.
Gen 26:34  И был Исав сорока лет, и взял себе в жены Иегудифу, дочь Беэра Хеттеянина, и Васемафу, дочь Елона Хеттеянина;
Gen 26:35  и они были в тягость Исааку и Ревекке.
Gen 27:1  Когда Исаак состарился и притупилось зрение глаз его, он призвал старшего сына своего Исава и сказал ему: сын мой! Тот сказал ему: вот я.
Gen 27:2  Он сказал: вот, я состарился; не знаю дня смерти моей;
Gen 27:3  возьми теперь орудия твои, колчан твой и лук твой, пойди в поле, и налови мне дичи,
Gen 27:4  и приготовь мне кушанье, какое я люблю, и принеси мне есть, чтобы благословила тебя душа моя, прежде нежели я умру.
Gen 27:5  Ревекка слышала, когда Исаак говорил сыну своему Исаву. И пошел Исав в поле достать и принести дичи;
Gen 27:6  а Ревекка сказала сыну своему Иакову: вот, я слышала, как отец твой говорил брату твоему Исаву:
Gen 27:7  принеси мне дичи и приготовь мне кушанье; я поем и благословлю тебя пред лицем Господним, пред смертью моею.
Gen 27:8  Теперь, сын мой, послушайся слов моих в том, что я прикажу тебе:
Gen 27:9  пойди в [стадо] и возьми мне оттуда два козленка хороших, и я приготовлю из них отцу твоему кушанье, какое он любит,
Gen 27:10  а ты принесешь отцу твоему, и он поест, чтобы благословить тебя пред смертью своею.
Gen 27:11  Иаков сказал Ревекке, матери своей: Исав, брат мой, человек косматый, а я человек гладкий;
Gen 27:12  может статься, ощупает меня отец мой, и я буду в глазах его обманщиком и наведу на себя проклятие, а не благословение.
Gen 27:13  Мать его сказала ему: на мне пусть будет проклятие твое, сын мой, только послушайся слов моих и пойди, принеси мне.
Gen 27:14  Он пошел, и взял, и принес матери своей; и мать его сделала кушанье, какое любил отец его.
Gen 27:15  И взяла Ревекка богатую одежду старшего сына своего Исава, бывшую у ней в доме, и одела [в нее] младшего сына своего Иакова;
Gen 27:16  а руки его и гладкую шею его обложила кожею козлят;
Gen 27:17  и дала кушанье и хлеб, которые она приготовила, в руки Иакову, сыну своему.
Gen 27:18  Он вошел к отцу своему и сказал: отец мой! Тот сказал: вот я; кто ты, сын мой?
Gen 27:19  Иаков сказал отцу своему: я Исав, первенец твой; я сделал, как ты сказал мне; встань, сядь и поешь дичи моей, чтобы благословила меня душа твоя.
Gen 27:20  И сказал Исаак сыну своему: что так скоро нашел ты, сын мой? Он сказал: потому что Господь Бог твой послал мне навстречу.
Gen 27:21  И сказал Исаак Иакову: подойди, я ощупаю тебя, сын мой, ты ли сын мой Исав, или нет?
Gen 27:22  Иаков подошел к Исааку, отцу своему, и он ощупал его и сказал: голос, голос Иакова; а руки, руки Исавовы.
Gen 27:23  И не узнал его, потому что руки его были, как руки Исава, брата его, косматые; и благословил его
Gen 27:24  и сказал: ты ли сын мой Исав? Он отвечал: я.
Gen 27:25  [Исаак] сказал: подай мне, я поем дичи сына моего, чтобы благословила тебя душа моя. [Иаков] подал ему, и он ел; принес ему и вина, и он пил.
Gen 27:26  Исаак, отец его, сказал ему: подойди, поцелуй меня, сын мой.
Gen 27:27  Он подошел и поцеловал его. И ощутил [Исаак] запах от одежды его и благословил его и сказал: вот, запах от сына моего, как запах от поля, которое благословил Господь;
Gen 27:28  да даст тебе Бог от росы небесной и от тука земли, и множество хлеба и вина;
Gen 27:29  да послужат тебе народы, и да поклонятся тебе племена; будь господином над братьями твоими, и да поклонятся тебе сыны матери твоей; проклинающие тебя--прокляты; благословляющие тебя--благословенны!
Gen 27:30  Как скоро совершил Исаак благословение над Иаковом, и как только вышел Иаков от лица Исаака, отца своего, Исав, брат его, пришел с ловли своей.
Gen 27:31  Приготовил и он кушанье, и принес отцу своему, и сказал отцу своему: встань, отец мой, и поешь дичи сына твоего, чтобы благословила меня душа твоя.
Gen 27:32  Исаак же, отец его, сказал ему: кто ты? Он сказал: я сын твой, первенец твой, Исав.
Gen 27:33  И вострепетал Исаак весьма великим трепетом, и сказал: кто ж это, который достал дичи и принес мне, и я ел от всего, прежде нежели ты пришел, и я благословил его? он и будет благословен.
Gen 27:34  Исав, выслушав слова отца своего, поднял громкий и весьма горький вопль и сказал отцу своему: отец мой! благослови и меня.
Gen 27:35  Но он сказал: брат твой пришел с хитростью и взял благословение твое.
Gen 27:36  И сказал он: не потому ли дано ему имя: Иаков, что он запнул меня уже два раза? Он взял первородство мое, и вот, теперь взял благословение мое. И [еще] сказал: неужели ты не оставил мне благословения?
Gen 27:37  Исаак отвечал Исаву: вот, я поставил его господином над тобою и всех братьев его отдал ему в рабы; одарил его хлебом и вином; что же я сделаю для тебя, сын мой?
Gen 27:38  Но Исав сказал отцу своему: неужели, отец мой, одно у тебя благословение? благослови и меня, отец мой! И возвысил Исав голос свой и заплакал.
Gen 27:39  И отвечал Исаак, отец его, и сказал ему: вот, от тука земли будет обитание твое и от росы небесной свыше;
Gen 27:40  и ты будешь жить мечом твоим и будешь служить брату твоему; будет же [время], когда воспротивишься и свергнешь иго его с выи твоей.
Gen 27:41  И возненавидел Исав Иакова за благословение, которым благословил его отец его; и сказал Исав в сердце своем: приближаются дни плача по отце моем, и я убью Иакова, брата моего.
Gen 27:42  И пересказаны были Ревекке слова Исава, старшего сына ее; и она послала, и призвала младшего сына своего Иакова, и сказала ему: вот, Исав, брат твой, грозит убить тебя;
Gen 27:43  и теперь, сын мой, послушайся слов моих, встань, беги к Лавану, брату моему, в Харран,
Gen 27:44  и поживи у него несколько времени, пока утолится ярость брата твоего,
Gen 27:45  пока утолится гнев брата твоего на тебя, и он позабудет, что ты сделал ему: тогда я пошлю и возьму тебя оттуда; для чего мне в один день лишиться обоих вас?
Gen 27:46  И сказала Ревекка Исааку: я жизни не рада от дочерей Хеттейских; если Иаков возьмет жену из дочерей Хеттейских, каковы эти, из дочерей этой земли, то к чему мне и жизнь?
Gen 28:1  И призвал Исаак Иакова и благословил его, и заповедал ему и сказал: не бери себе жены из дочерей Ханаанских;
Gen 28:2  встань, пойди в Месопотамию, в дом Вафуила, отца матери твоей, и возьми себе жену оттуда, из дочерей Лавана, брата матери твоей;
Gen 28:3  Бог же Всемогущий да благословит тебя, да расплодит тебя и да размножит тебя, и да будет от тебя множество народов,
Gen 28:4  и да даст тебе благословение Авраама, тебе и потомству твоему с тобою, чтобы тебе наследовать землю странствования твоего, которую Бог дал Аврааму!
Gen 28:5  И отпустил Исаак Иакова, и он пошел в Месопотамию к Лавану, сыну Вафуила Арамеянина, к брату Ревекки, матери Иакова и Исава.
Gen 28:6  Исав увидел, что Исаак благословил Иакова и благословляя послал его в Месопотамию, взять себе жену оттуда, и заповедал ему, сказав: не бери жены из дочерей Ханаанских;
Gen 28:7  и что Иаков послушался отца своего и матери своей и пошел в Месопотамию.
Gen 28:8  И увидел Исав, что дочери Ханаанские не угодны Исааку, отцу его;
Gen 28:9  и пошел Исав к Измаилу и взял себе жену Махалафу, дочь Измаила, сына Авраамова, сестру Наваиофову, сверх [других] жен своих.
Gen 28:10  Иаков же вышел из Вирсавии и пошел в Харран,
Gen 28:11  и пришел на [одно] место, и [остался] там ночевать, потому что зашло солнце. И взял [один] из камней того места, и положил себе изголовьем, и лег на том месте.
Gen 28:12  И увидел во сне: вот, лестница стоит на земле, а верх ее касается неба; и вот, Ангелы Божии восходят и нисходят по ней.
Gen 28:13  И вот, Господь стоит на ней и говорит: Я Господь, Бог Авраама, отца твоего, и Бог Исаака. Землю, на которой ты лежишь, Я дам тебе и потомству твоему;
Gen 28:14  и будет потомство твое, как песок земной; и распространишься к морю и к востоку, и к северу и к полудню; и благословятся в тебе и в семени твоем все племена земные;
Gen 28:15  и вот Я с тобою, и сохраню тебя везде, куда ты ни пойдешь; и возвращу тебя в сию землю, ибо Я не оставлю тебя, доколе не исполню того, что Я сказал тебе.
Gen 28:16  Иаков пробудился от сна своего и сказал: истинно Господь присутствует на месте сем; а я не знал!
Gen 28:17  И убоялся и сказал: как страшно сие место! это не иное что, как дом Божий, это врата небесные.
Gen 28:18  И встал Иаков рано утром, и взял камень, который он положил себе изголовьем, и поставил его памятником, и возлил елей на верх его.
Gen 28:19  И нарек имя месту тому: Вефиль, а прежнее имя того города было: Луз.
Gen 28:20  И положил Иаков обет, сказав: если Бог будет со мною и сохранит меня в пути сем, в который я иду, и даст мне хлеб есть и одежду одеться,
Gen 28:21  и я в мире возвращусь в дом отца моего, и будет Господь моим Богом, --
Gen 28:22  то этот камень, который я поставил памятником, будет домом Божиим; и из всего, что Ты, [Боже], даруешь мне, я дам Тебе десятую часть.
Gen 29:1  И встал Иаков и пошел в землю сынов востока.
Gen 29:2  И увидел: вот, на поле колодезь, и там три стада мелкого скота, лежавшие около него, потому что из того колодезя поили стада. Над устьем колодезя был большой камень.
Gen 29:3  Когда собирались туда все стада, отваливали камень от устья колодезя и поили овец; потом опять клали камень на свое место, на устье колодезя.
Gen 29:4  Иаков сказал им: братья мои! откуда вы? Они сказали: мы из Харрана.
Gen 29:5  Он сказал им: знаете ли вы Лавана, сына Нахорова? Они сказали: знаем.
Gen 29:6  Он еще сказал им: здравствует ли он? Они сказали: здравствует; и вот, Рахиль, дочь его, идет с овцами.
Gen 29:7  И сказал: вот, дня еще много; не время собирать скот; напойте овец и пойдите, пасите.
Gen 29:8  Они сказали: не можем, пока не соберутся все стада, и не отвалят камня от устья колодезя; тогда будем мы поить овец.
Gen 29:9  Еще он говорил с ними, как пришла Рахиль с мелким скотом отца своего, потому что она пасла.
Gen 29:10  Когда Иаков увидел Рахиль, дочь Лавана, брата матери своей, и овец Лавана, брата матери своей, то подошел Иаков, отвалил камень от устья колодезя и напоил овец Лавана, брата матери своей.
Gen 29:11  И поцеловал Иаков Рахиль и возвысил голос свой и заплакал.
Gen 29:12  И сказал Иаков Рахили, что он родственник отцу ее и что он сын Ревеккин. А она побежала и сказала отцу своему.
Gen 29:13  Лаван, услышав о Иакове, сыне сестры своей, выбежал ему навстречу, обнял его и поцеловал его, и ввел его в дом свой; и он рассказал Лавану все сие.
Gen 29:14  Лаван же сказал ему: подлинно ты кость моя и плоть моя. И жил у него [Иаков] целый месяц.
Gen 29:15  И Лаван сказал Иакову: неужели ты даром будешь служить мне, потому что ты родственник? скажи мне, что заплатить тебе?
Gen 29:16  У Лавана же было две дочери; имя старшей: Лия; имя младшей: Рахиль.
Gen 29:17  Лия была слаба глазами, а Рахиль была красива станом и красива лицем.
Gen 29:18  Иаков полюбил Рахиль и сказал: я буду служить тебе семь лет за Рахиль, младшую дочь твою.
Gen 29:19  Лаван сказал: лучше отдать мне ее за тебя, нежели отдать ее за другого кого; живи у меня.
Gen 29:20  И служил Иаков за Рахиль семь лет; и они показались ему за несколько дней, потому что он любил ее.
Gen 29:21  И сказал Иаков Лавану: дай жену мою, потому что мне уже исполнилось время, чтобы войти к ней.
Gen 29:22  Лаван созвал всех людей того места и сделал пир.
Gen 29:23  Вечером же взял дочь свою Лию и ввел ее к нему; и вошел к ней [Иаков].
Gen 29:24  И дал Лаван служанку свою Зелфу в служанки дочери своей Лии.
Gen 29:25  Утром же оказалось, что это Лия. И сказал Лавану: что это сделал ты со мною? не за Рахиль ли я служил у тебя? зачем ты обманул меня?
Gen 29:26  Лаван сказал: в нашем месте так не делают, чтобы младшую выдать прежде старшей;
Gen 29:27  окончи неделю этой, потом дадим тебе и ту за службу, которую ты будешь служить у меня еще семь лет других.
Gen 29:28  Иаков так и сделал и окончил неделю этой. И [Лаван] дал Рахиль, дочь свою, ему в жену.
Gen 29:29  И дал Лаван служанку свою Валлу в служанки дочери своей Рахили.
Gen 29:30  [Иаков] вошел и к Рахили, и любил Рахиль больше, нежели Лию; и служил у него еще семь лет других.
Gen 29:31  Господь узрел, что Лия была нелюбима, и отверз утробу ее, а Рахиль была неплодна.
Gen 29:32  Лия зачала и родила сына, и нарекла ему имя: Рувим, потому что сказала она: Господь призрел на мое бедствие; ибо теперь будет любить меня муж мой.
Gen 29:33  И зачала опять и родила сына, и сказала: Господь услышал, что я нелюбима, и дал мне и сего. И нарекла ему имя: Симеон.
Gen 29:34  И зачала еще и родила сына, и сказала: теперь-то прилепится ко мне муж мой, ибо я родила ему трех сынов. От сего наречено ему имя: Левий.
Gen 29:35  И еще зачала и родила сына, и сказала: теперь-то я восхвалю Господа. Посему нарекла ему имя Иуда. И перестала рождать.
Gen 30:1  И увидела Рахиль, что она не рождает детей Иакову, и позавидовала Рахиль сестре своей, и сказала Иакову: дай мне детей, а если не так, я умираю.
Gen 30:2  Иаков разгневался на Рахиль и сказал: разве я Бог, Который не дал тебе плода чрева?
Gen 30:3  Она сказала: вот служанка моя Валла; войди к ней; пусть она родит на колени мои, чтобы и я имела детей от нее.
Gen 30:4  И дала она Валлу, служанку свою, в жену ему; и вошел к ней Иаков.
Gen 30:5  Валла зачала и родила Иакову сына.
Gen 30:6  И сказала Рахиль: судил мне Бог, и услышал голос мой, и дал мне сына. Посему нарекла ему имя: Дан.
Gen 30:7  И еще зачала и родила Валла, служанка Рахилина, другого сына Иакову.
Gen 30:8  И сказала Рахиль: борьбою сильною боролась я с сестрою моею и превозмогла. И нарекла ему имя: Неффалим.
Gen 30:9  Лия увидела, что перестала рождать, и взяла служанку свою Зелфу, и дала ее Иакову в жену.
Gen 30:10  И Зелфа, служанка Лиина, родила Иакову сына.
Gen 30:11  И сказала Лия: прибавилось. И нарекла ему имя: Гад.
Gen 30:12  И родила Зелфа, служанка Лии, другого сына Иакову.
Gen 30:13  И сказала Лия: к благу моему, ибо блаженною будут называть меня женщины. И нарекла ему имя: Асир.
Gen 30:14  Рувим пошел во время жатвы пшеницы, и нашел мандрагоровые яблоки в поле, и принес их Лии, матери своей. И Рахиль сказала Лии: дай мне мандрагоров сына твоего.
Gen 30:15  Но она сказала ей: неужели мало тебе завладеть мужем моим, что ты домогаешься и мандрагоров сына моего? Рахиль сказала: так пусть он ляжет с тобою эту ночь, за мандрагоры сына твоего.
Gen 30:16  Иаков пришел с поля вечером, и Лия вышла ему навстречу и сказала: войди ко мне; ибо я купила тебя за мандрагоры сына моего. И лег он с нею в ту ночь.
Gen 30:17  И услышал Бог Лию, и она зачала и родила Иакову пятого сына.
Gen 30:18  И сказала Лия: Бог дал возмездие мне за то, что я отдала служанку мою мужу моему. И нарекла ему имя: Иссахар.
Gen 30:19  И еще зачала Лия и родила Иакову шестого сына.
Gen 30:20  И сказала Лия: Бог дал мне прекрасный дар; теперь будет жить у меня муж мой, ибо я родила ему шесть сынов. И нарекла ему имя: Завулон.
Gen 30:21  Потом родила дочь и нарекла ей имя: Дина.
Gen 30:22  И вспомнил Бог о Рахили, и услышал ее Бог, и отверз утробу ее.
Gen 30:23  Она зачала и родила сына, и сказала: снял Бог позор мой.
Gen 30:24  И нарекла ему имя: Иосиф, сказав: Господь даст мне и другого сына.
Gen 30:25  После того, как Рахиль родила Иосифа, Иаков сказал Лавану: отпусти меня, и пойду я в свое место, и в свою землю;
Gen 30:26  отдай жен моих и детей моих, за которых я служил тебе, и я пойду, ибо ты знаешь службу мою, какую я служил тебе.
Gen 30:27  И сказал ему Лаван: о, если бы я нашел благоволение пред очами твоими! я примечаю, что за тебя Господь благословил меня.
Gen 30:28  И сказал: назначь себе награду от меня, и я дам.
Gen 30:29  И сказал ему [Иаков]: ты знаешь, как я служил тебе, и каков стал скот твой при мне;
Gen 30:30  ибо мало было у тебя до меня, а стало много; Господь благословил тебя с приходом моим; когда же я буду работать для своего дома?
Gen 30:31  И сказал [Лаван]: что дать тебе? Иаков сказал: не давай мне ничего. Если только сделаешь мне, что я скажу, то я опять буду пасти и стеречь овец твоих.
Gen 30:32  Я пройду сегодня по всему [стаду] овец твоих; отдели из него всякий скот с крапинами и с пятнами, всякую скотину черную из овец, также с пятнами и с крапинами из коз. [Такой скот] будет наградою мне.
Gen 30:33  И будет говорить за меня пред тобою справедливость моя в следующее время, когда придешь посмотреть награду мою. Всякая из коз не с крапинами и не с пятнами, и из овец не черная, краденое это у меня.
Gen 30:34  Лаван сказал: хорошо, пусть будет по твоему слову.
Gen 30:35  И отделил в тот день козлов пестрых и с пятнами, и всех коз с крапинами и с пятнами, всех, на которых было [несколько] белого, и всех черных овец, и отдал на руки сыновьям своим;
Gen 30:36  и назначил расстояние между собою и между Иаковом на три дня пути. Иаков же пас остальной мелкий скот Лаванов.
Gen 30:37  И взял Иаков свежих прутьев тополевых, миндальных и яворовых, и вырезал на них белые полосы, сняв кору до белизны, которая на прутьях,
Gen 30:38  и положил прутья с нарезкою перед скотом в водопойных корытах, куда скот приходил пить, и где, приходя пить, зачинал пред прутьями.
Gen 30:39  И зачинал скот пред прутьями, и рождался скот пестрый, и с крапинами, и с пятнами.
Gen 30:40  И отделял Иаков ягнят и ставил скот лицем к пестрому и всему черному скоту Лаванову; и держал свои стада особо и не ставил их вместе со скотом Лавана.
Gen 30:41  Каждый раз, когда зачинал скот крепкий, Иаков клал прутья в корытах пред глазами скота, чтобы он зачинал пред прутьями.
Gen 30:42  А когда зачинал скот слабый, тогда он не клал. И доставался слабый [скот] Лавану, а крепкий Иакову.
Gen 30:43  И сделался этот человек весьма, весьма богатым, и было у него множество мелкого скота, и рабынь, и рабов, и верблюдов, и ослов.
Gen 31:1  И услышал [Иаков] слова сынов Лавановых, которые говорили: Иаков завладел всем, что было у отца нашего, и из имения отца нашего составил все богатство сие.
Gen 31:2  И увидел Иаков лице Лавана, и вот, оно не таково к нему, как было вчера и третьего дня.
Gen 31:3  И сказал Господь Иакову: возвратись в землю отцов твоих и на родину твою; и Я буду с тобою.
Gen 31:4  И послал Иаков, и призвал Рахиль и Лию в поле, к [стаду] мелкого скота своего,
Gen 31:5  и сказал им: я вижу лице отца вашего, что оно ко мне не таково, как было вчера и третьего дня; но Бог отца моего был со мною;
Gen 31:6  вы сами знаете, что я всеми силами служил отцу вашему,
Gen 31:7  а отец ваш обманывал меня и раз десять переменял награду мою; но Бог не попустил ему сделать мне зло.
Gen 31:8  Когда сказал он, что [скот] с крапинами будет тебе в награду, то скот весь родил с крапинами. А когда он сказал: пестрые будут тебе в награду, то скот весь и родил пестрых.
Gen 31:9  И отнял Бог скот у отца вашего и дал мне.
Gen 31:10  Однажды в такое время, когда скот зачинает, я взглянул и увидел во сне, и вот козлы, поднявшиеся на скот, пестрые с крапинами и пятнами.
Gen 31:11  Ангел Божий сказал мне во сне: Иаков! Я сказал: вот я.
Gen 31:12  Он сказал: возведи очи твои и посмотри: все козлы, поднявшиеся на скот, пестрые, с крапинами и с пятнами, ибо Я вижу все, что Лаван делает с тобою;
Gen 31:13  Я Бог [явившийся тебе] в Вефиле, где ты возлил елей на памятник и где ты дал Мне обет; теперь встань, выйди из земли сей и возвратись в землю родины твоей.
Gen 31:14  Рахиль и Лия сказали ему в ответ: есть ли еще нам доля и наследство в доме отца нашего?
Gen 31:15  не за чужих ли он нас почитает? ибо он продал нас и съел даже серебро наше;
Gen 31:16  посему все богатство, которое Бог отнял у отца нашего, есть наше и детей наших; итак делай все, что Бог сказал тебе.
Gen 31:17  И встал Иаков, и посадил детей своих и жен своих на верблюдов,
Gen 31:18  и взял с собою весь скот свой и все богатство свое, которое приобрел, скот собственный его, который он приобрел в Месопотамии, чтобы идти к Исааку, отцу своему, в землю Ханаанскую.
Gen 31:19  И как Лаван пошел стричь скот свой, то Рахиль похитила идолов, которые были у отца ее.
Gen 31:20  Иаков же похитил сердце у Лавана Арамеянина, потому что не известил его, что удаляется.
Gen 31:21  И ушел со всем, что у него было; и, встав, перешел реку и направился к горе Галаад.
Gen 31:22  На третий день сказали Лавану, что Иаков ушел.
Gen 31:23  Тогда он взял с собою родственников своих, и гнался за ним семь дней, и догнал его на горе Галаад.
Gen 31:24  И пришел Бог к Лавану Арамеянину ночью во сне и сказал ему: берегись, не говори Иакову ни доброго, ни худого.
Gen 31:25  И догнал Лаван Иакова; Иаков же поставил шатер свой на горе, и Лаван со сродниками своими поставил на горе Галаад.
Gen 31:26  И сказал Лаван Иакову: что ты сделал? для чего ты обманул меня, и увел дочерей моих, как плененных оружием?
Gen 31:27  зачем ты убежал тайно, и укрылся от меня, и не сказал мне? я отпустил бы тебя с веселием и с песнями, с тимпаном и с гуслями;
Gen 31:28  ты не позволил мне даже поцеловать внуков моих и дочерей моих; безрассудно ты сделал.
Gen 31:29  Есть в руке моей сила сделать вам зло; но Бог отца вашего вчера говорил ко мне и сказал: берегись, не говори Иакову ни хорошего, ни худого.
Gen 31:30  Но пусть бы ты ушел, потому что ты нетерпеливо захотел быть в доме отца твоего, --зачем ты украл богов моих?
Gen 31:31  Иаков отвечал Лавану и сказал: [я] боялся, ибо я думал, не отнял бы ты у меня дочерей своих.
Gen 31:32  у кого найдешь богов твоих, тот не будет жив; при родственниках наших узнавай, что у меня, и возьми себе. Иаков не знал, что Рахиль украла их.
Gen 31:33  И ходил Лаван в шатер Иакова, и в шатер Лии, и в шатер двух рабынь, но не нашел. И, выйдя из шатра Лии, вошел в шатер Рахили.
Gen 31:34  Рахиль же взяла идолов, и положила их под верблюжье седло и села на них. И обыскал Лаван весь шатер; но не нашел.
Gen 31:35  Она же сказала отцу своему: да не прогневается господин мой, что я не могу встать пред тобою, ибо у меня обыкновенное женское. И он искал, но не нашел идолов.
Gen 31:36  Иаков рассердился и вступил в спор с Лаваном. И начал Иаков говорить и сказал Лавану: какая вина моя, какой грех мой, что ты преследуешь меня?
Gen 31:37  ты осмотрел у меня все вещи, что нашел ты из всех вещей твоего дома? покажи здесь пред родственниками моими и пред родственниками твоими; пусть они рассудят между нами обоими.
Gen 31:38  Вот, двадцать лет я [был] у тебя; овцы твои и козы твои не выкидывали; овнов стада твоего я не ел;
Gen 31:39  растерзанного зверем я не приносил к тебе, это был мой убыток; ты с меня взыскивал, днем ли что пропадало, ночью ли пропадало;
Gen 31:40  я томился днем от жара, а ночью от стужи, и сон мой убегал от глаз моих.
Gen 31:41  Таковы мои двадцать лет в доме твоем. Я служил тебе четырнадцать лет за двух дочерей твоих и шесть лет за скот твой, а ты десять раз переменял награду мою.
Gen 31:42  Если бы не был со мною Бог отца моего, Бог Авраама и страх Исаака, ты бы теперь отпустил меня ни с чем. Бог увидел бедствие мое и труд рук моих и вступился [за меня] вчера.
Gen 31:43  И отвечал Лаван и сказал Иакову: дочери--мои дочери; дети--мои дети; скот--мой скот, и все, что ты видишь, это мое: могу ли я что сделать теперь с дочерями моими и с детьми их, которые рождены ими?
Gen 31:44  Теперь заключим союз я и ты, и это будет свидетельством между мною и тобою.
Gen 31:45  И взял Иаков камень и поставил его памятником.
Gen 31:46  И сказал Иаков родственникам своим: наберите камней. Они взяли камни, и сделали холм, и ели там на холме.
Gen 31:47  И назвал его Лаван: Иегар-Сагадуфа; а Иаков назвал его Галаадом.
Gen 31:48  И сказал Лаван: сегодня этот холм между мною и тобою свидетель. Посему и наречено ему имя: Галаад,
Gen 31:49  [также]: Мицпа, оттого, что Лаван сказал: да надзирает Господь надо мною и над тобою, когда мы скроемся друг от друга;
Gen 31:50  если ты будешь худо поступать с дочерями моими, или если возьмешь жен сверх дочерей моих, то, хотя нет человека между нами, но смотри, Бог свидетель между мною и между тобою.
Gen 31:51  И сказал Лаван Иакову: вот холм сей и вот памятник, который я поставил между мною и тобою;
Gen 31:52  этот холм свидетель, и этот памятник свидетель, что ни я не перейду к тебе за этот холм, ни ты не перейдешь ко мне за этот холм и за этот памятник, для зла;
Gen 31:53  Бог Авраамов и Бог Нахоров да судит между нами, Бог отца их. Иаков поклялся страхом отца своего Исаака.
Gen 31:54  И заколол Иаков жертву на горе и позвал родственников своих есть хлеб; и они ели хлеб и ночевали на горе.
Gen 31:55  И встал Лаван рано утром и поцеловал внуков своих и дочерей своих, и благословил их. И пошел и возвратился Лаван в свое место.
Gen 32:1  А Иаков пошел путем своим. И встретили его Ангелы Божии.
Gen 32:2  Иаков, увидев их, сказал: это ополчение Божие. И нарек имя месту тому: Маханаим.
Gen 32:3  И послал Иаков пред собою вестников к брату своему Исаву в землю Сеир, в область Едом,
Gen 32:4  и приказал им, сказав: так скажите господину моему Исаву: вот что говорит раб твой Иаков: я жил у Лавана и прожил доныне;
Gen 32:5  и есть у меня волы и ослы и мелкий скот, и рабы и рабыни; и я послал известить [о себе] господина моего, дабы приобрести благоволение пред очами твоими.
Gen 32:6  И возвратились вестники к Иакову и сказали: мы ходили к брату твоему Исаву; он идет навстречу тебе, и с ним четыреста человек.
Gen 32:7  Иаков очень испугался и смутился; и разделил людей, бывших с ним, и скот мелкий и крупный и верблюдов на два стана.
Gen 32:8  И сказал: если Исав нападет на один стан и побьет его, то остальной стан может спастись.
Gen 32:9  И сказал Иаков: Боже отца моего Авраама и Боже отца моего Исаака, Господи, сказавший мне: возвратись в землю твою, на родину твою, и Я буду благотворить тебе!
Gen 32:10  Недостоин я всех милостей и всех благодеяний, которые Ты сотворил рабу Твоему, ибо я с посохом моим перешел этот Иордан, а теперь у меня два стана.
Gen 32:11  Избавь меня от руки брата моего, от руки Исава, ибо я боюсь его, чтобы он, придя, не убил меня [и] матери с детьми.
Gen 32:12  Ты сказал: Я буду благотворить тебе и сделаю потомство твое, как песок морской, которого не исчислить от множества.
Gen 32:13  И ночевал там [Иаков] в ту ночь. И взял из того, что у него было, в подарок Исаву, брату своему:
Gen 32:14  двести коз, двадцать козлов, двести овец, двадцать овнов,
Gen 32:15  тридцать верблюдиц дойных с жеребятами их, сорок коров, десять волов, двадцать ослиц, десять ослов.
Gen 32:16  И дал в руки рабам своим каждое стадо особо и сказал рабам своим: пойдите предо мною и оставляйте расстояние от стада до стада.
Gen 32:17  И приказал первому, сказав: когда брат мой Исав встретится тебе и спросит тебя, говоря: чей ты? и куда идешь? и чье это [стадо] пред тобою?
Gen 32:18  то скажи: раба твоего Иакова; это подарок, посланный господину моему Исаву; вот, и сам он за нами.
Gen 32:19  То же приказал он и второму, и третьему, и всем, которые шли за стадами, говоря: так скажите Исаву, когда встретите его;
Gen 32:20  и скажите: вот, и раб твой Иаков за нами. Ибо он сказал [сам в себе]: умилостивлю его дарами, которые идут предо мною, и потом увижу лице его; может быть, и примет меня.
Gen 32:21  И пошли дары пред ним, а он ту ночь ночевал в стане.
Gen 32:22  И встал в ту ночь, и, взяв двух жен своих и двух рабынь своих, и одиннадцать сынов своих, перешел через Иавок вброд;
Gen 32:23  и, взяв их, перевел через поток, и перевел все, что у него [было].
Gen 32:24  И остался Иаков один. И боролся Некто с ним до появления зари;
Gen 32:25  и, увидев, что не одолевает его, коснулся состава бедра его и повредил состав бедра у Иакова, когда он боролся с Ним.
Gen 32:26  И сказал: отпусти Меня, ибо взошла заря. Иаков сказал: не отпущу Тебя, пока не благословишь меня.
Gen 32:27  И сказал: как имя твое? Он сказал: Иаков.
Gen 32:28  И сказал: отныне имя тебе будет не Иаков, а Израиль, ибо ты боролся с Богом, и человеков одолевать будешь.
Gen 32:29  Спросил и Иаков, говоря: скажи имя Твое. И Он сказал: на что ты спрашиваешь о имени Моем? И благословил его там.
Gen 32:30  И нарек Иаков имя месту тому: Пенуэл; ибо, [говорил он], я видел Бога лицем к лицу, и сохранилась душа моя.
Gen 32:31  И взошло солнце, когда он проходил Пенуэл; и хромал он на бедро свое.
Gen 32:32  Поэтому и доныне сыны Израилевы не едят жилы, которая на составе бедра, потому что [Боровшийся] коснулся жилы на составе бедра Иакова.
Gen 33:1  Взглянул Иаков и увидел, и вот, идет Исав, и с ним четыреста человек. И разделил детей Лии, Рахили и двух служанок.
Gen 33:2  И поставил служанок и детей их впереди, Лию и детей ее за ними, а Рахиль и Иосифа позади.
Gen 33:3  А сам пошел пред ними и поклонился до земли семь раз, подходя к брату своему.
Gen 33:4  И побежал Исав к нему навстречу и обнял его, и пал на шею его и целовал его, и плакали.
Gen 33:5  И взглянул и увидел жен и детей и сказал: кто это у тебя? [Иаков] сказал: дети, которых Бог даровал рабу твоему.
Gen 33:6  И подошли служанки и дети их и поклонились;
Gen 33:7  подошла и Лия и дети ее и поклонились; наконец подошли Иосиф и Рахиль и поклонились.
Gen 33:8  И сказал Исав: для чего у тебя это множество, которое я встретил? И сказал Иаков: дабы приобрести благоволение в очах господина моего.
Gen 33:9  Исав сказал: у меня много, брат мой; пусть будет твое у тебя.
Gen 33:10  Иаков сказал: нет, если я приобрел благоволение в очах твоих, прими дар мой от руки моей, ибо я увидел лице твое, как бы кто увидел лице Божие, и ты был благосклонен ко мне;
Gen 33:11  прими благословение мое, которое я принес тебе, потому что Бог даровал мне, и есть у меня все. И упросил его, и тот взял
Gen 33:12  и сказал: поднимемся и пойдем; и я пойду пред тобою.
Gen 33:13  Иаков сказал ему: господин мой знает, что дети нежны, а мелкий и крупный скот у меня дойный: если погнать его один день, то помрет весь скот;
Gen 33:14  пусть господин мой пойдет впереди раба своего, а я пойду медленно, как пойдет скот, который предо мною, и как пойдут дети, и приду к господину моему в Сеир.
Gen 33:15  Исав сказал: оставлю я с тобою [несколько] из людей, которые при мне. Иаков сказал: к чему это? только бы мне приобрести благоволение в очах господина моего!
Gen 33:16  И возвратился Исав в тот же день путем своим в Сеир.
Gen 33:17  А Иаков двинулся в Сокхоф, и построил себе дом, и для скота своего сделал шалаши. От сего он нарек имя месту: Сокхоф.
Gen 33:18  Иаков, возвратившись из Месопотамии, благополучно пришел в город Сихем, который в земле Ханаанской, и расположился пред городом.
Gen 33:19  И купил часть поля, на котором раскинул шатер свой, у сынов Еммора, отца Сихемова, за сто монет.
Gen 33:20  И поставил там жертвенник, и призвал имя Господа Бога Израилева.
Gen 34:1  Дина, дочь Лии, которую она родила Иакову, вышла посмотреть на дочерей земли той.
Gen 34:2  И увидел ее Сихем, сын Еммора Евеянина, князя земли той, и взял ее, и спал с нею, и сделал ей насилие.
Gen 34:3  И прилепилась душа его в Дине, дочери Иакова, и он полюбил девицу и говорил по сердцу девицы.
Gen 34:4  И сказал Сихем Еммору, отцу своему, говоря: возьми мне эту девицу в жену.
Gen 34:5  Иаков слышал, что [сын Емморов] обесчестил Дину, дочь его, но как сыновья его были со скотом его в поле, то Иаков молчал, пока не пришли они.
Gen 34:6  И вышел Еммор, отец Сихемов, к Иакову, поговорить с ним.
Gen 34:7  Сыновья же Иакова пришли с поля, и когда услышали, то огорчились мужи те и воспылали гневом, потому что бесчестие сделал он Израилю, переспав с дочерью Иакова, а так не надлежало делать.
Gen 34:8  Еммор стал говорить им, и сказал: Сихем, сын мой, прилепился душею к дочери вашей; дайте же ее в жену ему;
Gen 34:9  породнитесь с нами; отдавайте за нас дочерей ваших, а наших дочерей берите себе.
Gen 34:10  и живите с нами; земля сия пред вами, живите и промышляйте на ней и приобретайте ее во владение.
Gen 34:11  Сихем же сказал отцу ее и братьям ее: только бы мне найти благоволение в очах ваших, я дам, что ни скажете мне;
Gen 34:12  назначьте самое большое вено и дары; я дам, что ни скажете мне, только отдайте мне девицу в жену.
Gen 34:13  И отвечали сыновья Иакова Сихему и Еммору, отцу его, с лукавством; а говорили так потому, что он обесчестил Дину, сестру их;
Gen 34:14  и сказали им: не можем этого сделать, выдать сестру нашу за человека, который необрезан, ибо это бесчестно для нас;
Gen 34:15  только на том условии мы согласимся с вами, если вы будете как мы, чтобы и у вас весь мужеский пол был обрезан;
Gen 34:16  и будем отдавать за вас дочерей наших и брать за себя ваших дочерей, и будем жить с вами, и составим один народ;
Gen 34:17  а если не послушаетесь нас в том, чтобы обрезаться, то мы возьмем дочь нашу и удалимся.
Gen 34:18  И понравились слова сии Еммору и Сихему, сыну Емморову.
Gen 34:19  Юноша не умедлил исполнить это, потому что любил дочь Иакова. А он более всех уважаем был из дома отца своего.
Gen 34:20  И пришел Еммор и Сихем, сын его, к воротам города своего, и стали говорить жителям города своего и сказали:
Gen 34:21  сии люди мирны с нами; пусть они селятся на земле и промышляют на ней; земля же вот пространна пред ними. Станем брать дочерей их себе в жены и наших дочерей выдавать за них.
Gen 34:22  Только на том условии сии люди соглашаются жить с нами и быть одним народом, чтобы и у нас обрезан был весь мужеский пол, как они обрезаны.
Gen 34:23  Не для нас ли стада их, и имение их, и весь скот их? Только согласимся с ними, и будут жить с нами.
Gen 34:24  И послушались Еммора и Сихема, сына его, все выходящие из ворот города его: и обрезан был весь мужеский пол, --все выходящие из ворот города его.
Gen 34:25  На третий день, когда они были в болезни, два сына Иакова, Симеон и Левий, братья Динины, взяли каждый свой меч, и смело напали на город, и умертвили весь мужеский пол;
Gen 34:26  и самого Еммора и Сихема, сына его, убили мечом; и взяли Дину из дома Сихемова и вышли.
Gen 34:27  Сыновья Иакова пришли к убитым и разграбили город за то, что обесчестили сестру их.
Gen 34:28  Они взяли мелкий и крупный скот их, и ослов их, и что ни было в городе, и что ни было в поле;
Gen 34:29  и все богатство их, и всех детей их, и жен их взяли в плен, и разграбили все, что было в домах.
Gen 34:30  И сказал Иаков Симеону и Левию: вы возмутили меня, сделав меня ненавистным для жителей сей земли, для Хананеев и Ферезеев. У меня людей мало; соберутся против меня, поразят меня, и истреблен буду я и дом мой.
Gen 34:31  Они же сказали: а разве можно поступать с сестрою нашею, как с блудницею!
Gen 35:1  Бог сказал Иакову: встань, пойди в Вефиль и живи там, и устрой там жертвенник Богу, явившемуся тебе, когда ты бежал от лица Исава, брата твоего.
Gen 35:2  И сказал Иаков дому своему и всем бывшим с ним: бросьте богов чужих, находящихся у вас, и очиститесь, и перемените одежды ваши;
Gen 35:3  встанем и пойдем в Вефиль; там устрою я жертвенник Богу, Который услышал меня в день бедствия моего и был со мною в пути, которым я ходил.
Gen 35:4  И отдали Иакову всех богов чужих, бывших в руках их, и серьги, бывшие в ушах у них, и закопал их Иаков под дубом, который близ Сихема.
Gen 35:5  И отправились они. И был ужас Божий на окрестных городах, и не преследовали сынов Иаковлевых.
Gen 35:6  И пришел Иаков в Луз, что в земле Ханаанской, то есть в Вефиль, сам и все люди, бывшие с ним,
Gen 35:7  и устроил там жертвенник, и назвал сие место: Эл-Вефиль, ибо тут явился ему Бог, когда он бежал от лица брата своего.
Gen 35:8  И умерла Девора, кормилица Ревеккина, и погребена ниже Вефиля под дубом, который и назвал [Иаков] дубом плача.
Gen 35:9  И явился Бог Иакову по возвращении его из Месопотамии, и благословил его,
Gen 35:10  и сказал ему Бог: имя твое Иаков; отныне ты не будешь называться Иаковом, но будет имя тебе: Израиль. И нарек ему имя: Израиль.
Gen 35:11  И сказал ему Бог: Я Бог Всемогущий; плодись и умножайся; народ и множество народов будет от тебя, и цари произойдут из чресл твоих;
Gen 35:12  землю, которую Я дал Аврааму и Исааку, Я дам тебе, и потомству твоему по тебе дам землю сию.
Gen 35:13  И восшел от него Бог с места, на котором говорил ему.
Gen 35:14  И поставил Иаков памятник на месте, на котором говорил ему [Бог], памятник каменный, и возлил на него возлияние, и возлил на него елей;
Gen 35:15  и нарек Иаков имя месту, на котором Бог говорил ему: Вефиль.
Gen 35:16  И отправились из Вефиля. И когда еще оставалось некоторое расстояние земли до Ефрафы, Рахиль родила, и роды ее были трудны.
Gen 35:17  Когда же она страдала в родах, повивальная бабка сказала ей: не бойся, ибо и это тебе сын.
Gen 35:18  И когда выходила из нее душа, ибо она умирала, то нарекла ему имя: Бенони. Но отец его назвал его Вениамином.
Gen 35:19  И умерла Рахиль, и погребена на дороге в Ефрафу, то есть Вифлеем.
Gen 35:20  Иаков поставил над гробом ее памятник. Это надгробный памятник Рахили до сего дня.
Gen 35:21  И отправился Израиль и раскинул шатер свой за башнею Гадер.
Gen 35:22  Во время пребывания Израиля в той стране, Рувим пошел и переспал с Валлою, наложницею отца своего. И услышал Израиль. Сынов же у Иакова было двенадцать.
Gen 35:23  Сыновья Лии: первенец Иакова Рувим, [по нем] Симеон, Левий, Иуда, Иссахар и Завулон.
Gen 35:24  Сыновья Рахили: Иосиф и Вениамин.
Gen 35:25  Сыновья Валлы, служанки Рахилиной: Дан и Неффалим.
Gen 35:26  Сыновья Зелфы, служанки Лииной: Гад и Асир. Сии сыновья Иакова, родившиеся ему в Месопотамии.
Gen 35:27  И пришел Иаков к Исааку, отцу своему, в Мамре, в Кириаф-Арбу, то есть Хеврон где странствовал Авраам и Исаак.
Gen 35:28  И было дней [жизни] Исааковой сто восемьдесят лет.
Gen 35:29  И испустил Исаак дух и умер, и приложился к народу своему, будучи стар и насыщен жизнью; и погребли его Исав и Иаков, сыновья его.
Gen 36:1  Вот родословие Исава, он же Едом.
Gen 36:2  Исав взял себе жен из дочерей Ханаанских: Аду, дочь Елона Хеттеянина, и Оливему, дочь Аны, сына Цивеона Евеянина,
Gen 36:3  и Васемафу, дочь Измаила, сестру Наваиофа.
Gen 36:4  Ада родила Исаву Елифаза, Васемафа родила Рагуила,
Gen 36:5  Оливема родила Иеуса, Иеглома и Корея. Это сыновья Исава, родившиеся ему в земле Ханаанской.
Gen 36:6  И взял Исав жен своих и сыновей своих, и дочерей своих, и всех людей дома своего, и стада свои, и весь скот свой, и все имение свое, которое он приобрел в земле Ханаанской, и пошел в [другую] землю от лица Иакова, брата своего,
Gen 36:7  ибо имение их было так велико, что они не могли жить вместе, и земля странствования их не вмещала их, по множеству стад их.
Gen 36:8  И поселился Исав на горе Сеир, Исав, он же Едом.
Gen 36:9  И вот родословие Исава, отца Идумеев, на горе Сеир.
Gen 36:10  Вот имена сынов Исава: Елифаз, сын Ады, жены Исавовой, и Рагуил, сын Васемафы, жены Исавовой.
Gen 36:11  У Елифаза были сыновья: Феман, Омар, Цефо, Гафам и Кеназ.
Gen 36:12  Фамна же была наложница Елифаза, сына Исавова, и родила Елифазу Амалика. Вот сыновья Ады, жены Исавовой.
Gen 36:13  И вот сыновья Рагуила: Нахаф и Зерах, Шамма и Миза. Это сыновья Васемафы, жены Исавовой.
Gen 36:14  И сии были сыновья Оливемы, дочери Аны, сына Цивеонова, жены Исавовой: она родила Исаву Иеуса, Иеглома и Корея.
Gen 36:15  Вот старейшины сынов Исавовых. Сыновья Елифаза, первенца Исавова: старейшина Феман, старейшина Омар, старейшина Цефо, старейшина Кеназ,
Gen 36:16  старейшина Корей, старейшина Гафам, старейшина Амалик. Сии старейшины Елифазовы в земле Едома; сии сыновья Ады.
Gen 36:17  Сии сыновья Рагуила, сына Исавова: старейшина Нахаф, старейшина Зерах, старейшина Шамма, старейшина Миза. Сии старейшины Рагуиловы в земле Едома; сии сыновья Васемафы, жены Исавовой.
Gen 36:18  Сии сыновья Оливемы, жены Исавовой: старейшина Иеус, старейшина Иеглом, старейшина Корей. Сии старейшины Оливемы, дочери Аны, жены Исавовой.
Gen 36:19  Вот сыновья Исава, и вот старейшины их. Это Едом.
Gen 36:20  Сии сыновья Сеира Хорреянина, жившие в земле той: Лотан, Шовал, Цивеон, Ана,
Gen 36:21  Дишон, Эцер и Дишан. Сии старейшины Хорреев, сынов Сеира, в земле Едома.
Gen 36:22  Сыновья Лотана были: Хори и Геман; а сестра у Лотана: Фамна.
Gen 36:23  Сии сыновья Шовала: Алван, Манахаф, Эвал, Шефо и Онам.
Gen 36:24  Сии сыновья Цивеона: Аиа и Ана. Это тот Ана, который нашел теплые воды в пустыне, когда пас ослов Цивеона, отца своего.
Gen 36:25  Сии дети Аны: Дишон и Оливема, дочь Аны.
Gen 36:26  Сии сыновья Дишона: Хемдан, Эшбан, Ифран и Херан.
Gen 36:27  Сии сыновья Эцера: Билган, Зааван, и Акан.
Gen 36:28  Сии сыновья Дишана: Уц и Аран.
Gen 36:29  Сии старейшины Хорреев: старейшина Лотан, старейшина Шовал, старейшина Цивеон, старейшина Ана,
Gen 36:30  старейшина Дишон, старейшина Эцер, старейшина Дишан. Вот старейшины Хорреев, по старшинствам их в земле Сеир.
Gen 36:31  Вот цари, царствовавшие в земле Едома, прежде царствования царей у сынов Израилевых:
Gen 36:32  царствовал в Едоме Бела, сын Веоров, а имя городу его Дингава.
Gen 36:33  И умер Бела, и воцарился по нем Иовав, сын Зераха, из Восоры.
Gen 36:34  Умер Иовав, и воцарился по нем Хушам, из земли Феманитян.
Gen 36:35  И умер Хушам, и воцарился по нем Гадад, сын Бедадов, который поразил Мадианитян на поле Моава; имя городу его Авиф.
Gen 36:36  И умер Гадад, и воцарился по нем Самла из Масреки.
Gen 36:37  И умер Самла, и воцарился по нем Саул из Реховофа, что при реке.
Gen 36:38  И умер Саул, и воцарился по нем Баал-Ханан, сын Ахбора.
Gen 36:39  И умер Баал-Ханан, сын Ахбора, и воцарился по нем Гадар: имя городу его Пау; имя жене его Мегетавеель, дочь Матреды, сына Мезагава.
Gen 36:40  Сии имена старейшин Исавовых, по племенам их, по местам их, по именам их: старейшина Фимна, старейшина Алва, старейшина Иетеф,
Gen 36:41  старейшина Оливема, старейшина Эла, старейшина Пинон,
Gen 36:42  старейшина Кеназ, старейшина Феман, старейшина Мивцар,
Gen 36:43  старейшина Магдиил, старейшина Ирам. Вот старейшины Идумейские, по их селениям, в земле обладания их. Вот Исав, отец Идумеев.
Gen 37:1  Иаков жил в земле странствования отца своего, в земле Ханаанской.
Gen 37:2  Вот житие Иакова. Иосиф, семнадцати лет, пас скот вместе с братьями своими, будучи отроком, с сыновьями Валлы и с сыновьями Зелфы, жен отца своего. И доводил Иосиф худые о них слухи до отца их.
Gen 37:3  Израиль любил Иосифа более всех сыновей своих, потому что он был сын старости его, --и сделал ему разноцветную одежду.
Gen 37:4  И увидели братья его, что отец их любит его более всех братьев его; и возненавидели его и не могли говорить с ним дружелюбно.
Gen 37:5  И видел Иосиф сон, и рассказал братьям своим: и они возненавидели его еще более.
Gen 37:6  Он сказал им: выслушайте сон, который я видел:
Gen 37:7  вот, мы вяжем снопы посреди поля; и вот, мой сноп встал и стал прямо; и вот, ваши снопы стали кругом и поклонились моему снопу.
Gen 37:8  И сказали ему братья его: неужели ты будешь царствовать над нами? неужели будешь владеть нами? И возненавидели его еще более за сны его и за слова его.
Gen 37:9  И видел он еще другой сон и рассказал его братьям своим, говоря: вот, я видел еще сон: вот, солнце и луна и одиннадцать звезд поклоняются мне.
Gen 37:10  И он рассказал отцу своему и братьям своим; и побранил его отец его и сказал ему: что это за сон, который ты видел? неужели я и твоя мать, и твои братья придем поклониться тебе до земли?
Gen 37:11  Братья его досадовали на него, а отец его заметил это слово.
Gen 37:12  Братья его пошли пасти скот отца своего в Сихем.
Gen 37:13  И сказал Израиль Иосифу: братья твои не пасут ли в Сихеме? пойди, я пошлю тебя к ним. Он отвечал ему: вот я.
Gen 37:14  И сказал ему: пойди, посмотри, здоровы ли братья твои и цел ли скот, и принеси мне ответ. И послал его из долины Хевронской; и он пришел в Сихем.
Gen 37:15  И нашел его некто блуждающим в поле, и спросил его тот человек, говоря: чего ты ищешь?
Gen 37:16  Он сказал: я ищу братьев моих; скажи мне, где они пасут?
Gen 37:17  И сказал тот человек: они ушли отсюда, ибо я слышал, как они говорили: пойдем в Дофан. И пошел Иосиф за братьями своими и нашел их в Дофане.
Gen 37:18  И увидели они его издали, и прежде нежели он приблизился к ним, стали умышлять против него, чтобы убить его.
Gen 37:19  И сказали друг другу: вот, идет сновидец;
Gen 37:20  пойдем теперь, и убьем его, и бросим его в какой-нибудь ров, и скажем, что хищный зверь съел его; и увидим, что будет из его снов.
Gen 37:21  И услышал [сие] Рувим и избавил его от рук их, сказав: не убьем его.
Gen 37:22  И сказал им Рувим: не проливайте крови; бросьте его в ров, который в пустыне, а руки не налагайте на него. [Сие говорил он], чтобы избавить его от рук их и возвратить его к отцу его.
Gen 37:23  Когда Иосиф пришел к братьям своим, они сняли с Иосифа одежду его, одежду разноцветную, которая была на нем,
Gen 37:24  и взяли его и бросили его в ров; ров же тот был пуст; воды в нем не было.
Gen 37:25  И сели они есть хлеб, и, взглянув, увидели, вот, идет из Галаада караван Измаильтян, и верблюды их несут стираксу, бальзам и ладан: идут они отвезти это в Египет.
Gen 37:26  И сказал Иуда братьям своим: что пользы, если мы убьем брата нашего и скроем кровь его?
Gen 37:27  Пойдем, продадим его Измаильтянам, а руки наши да не будут на нем, ибо он брат наш, плоть наша. Братья его послушались
Gen 37:28  и, когда проходили купцы Мадиамские, вытащили Иосифа изо рва и продали Иосифа Измаильтянам за двадцать сребренников; а они отвели Иосифа в Египет.
Gen 37:29  Рувим же пришел опять ко рву; и вот, нет Иосифа во рве. И разодрал он одежды свои,
Gen 37:30  и возвратился к братьям своим, и сказал: отрока нет, а я, куда я денусь?
Gen 37:31  И взяли одежду Иосифа, и закололи козла, и вымарали одежду кровью;
Gen 37:32  и послали разноцветную одежду, и доставили к отцу своему, и сказали: мы это нашли; посмотри, сына ли твоего эта одежда, или нет.
Gen 37:33  Он узнал ее и сказал: [это] одежда сына моего; хищный зверь съел его; верно, растерзан Иосиф.
Gen 37:34  И разодрал Иаков одежды свои, и возложил вретище на чресла свои, и оплакивал сына своего многие дни.
Gen 37:35  И собрались все сыновья его и все дочери его, чтобы утешить его; но он не хотел утешиться и сказал: с печалью сойду к сыну моему в преисподнюю. Так оплакивал его отец его.
Gen 37:36  Мадианитяне же продали его в Египте Потифару, царедворцу фараонову, начальнику телохранителей.
Gen 38:1  В то время Иуда отошел от братьев своих и поселился близ одного Одолламитянина, которому имя: Хира.
Gen 38:2  И увидел там Иуда дочь одного Хананеянина, которому имя: Шуа; и взял ее и вошел к ней.
Gen 38:3  Она зачала и родила сына; и он нарек ему имя: Ир.
Gen 38:4  И зачала опять, и родила сына, и нарекла ему имя: Онан.
Gen 38:5  И еще родила сына и нарекла ему имя: Шела. Иуда был в Хезиве, когда она родила его.
Gen 38:6  И взял Иуда жену Иру, первенцу своему; имя ей Фамарь.
Gen 38:7  Ир, первенец Иудин, был неугоден пред очами Господа, и умертвил его Господь.
Gen 38:8  И сказал Иуда Онану: войди к жене брата твоего, женись на ней, как деверь, и восстанови семя брату твоему.
Gen 38:9  Онан знал, что семя будет не ему, и потому, когда входил к жене брата своего, изливал на землю, чтобы не дать семени брату своему.
Gen 38:10  Зло было пред очами Господа то, что он делал; и Он умертвил и его.
Gen 38:11  И сказал Иуда Фамари, невестке своей: живи вдовою в доме отца твоего, пока подрастет Шела, сын мой. Ибо он сказал: не умер бы и он подобно братьям его. Фамарь пошла и стала жить в доме отца своего.
Gen 38:12  Прошло много времени, и умерла дочь Шуи, жена Иудина. Иуда, утешившись, пошел в Фамну к стригущим скот его, сам и Хира, друг его, Одолламитянин.
Gen 38:13  И уведомили Фамарь, говоря: вот, свекор твой идет в Фамну стричь скот свой.
Gen 38:14  И сняла она с себя одежду вдовства своего, покрыла себя покрывалом и, закрывшись, села у ворот Енаима, что на дороге в Фамну. Ибо видела, что Шела вырос, и она не дана ему в жену.
Gen 38:15  И увидел ее Иуда и почел ее за блудницу, потому что она закрыла лице свое.
Gen 38:16  Он поворотил к ней и сказал: войду я к тебе. Ибо не знал, что это невестка его. Она сказала: что ты дашь мне, если войдешь ко мне?
Gen 38:17  Он сказал: я пришлю тебе козленка из стада. Она сказала: дашь ли ты мне залог, пока пришлешь?
Gen 38:18  Он сказал: какой дать тебе залог? Она сказала: печать твою, и перевязь твою, и трость твою, которая в руке твоей. И дал он ей и вошел к ней; и она зачала от него.
Gen 38:19  И, встав, пошла, сняла с себя покрывало свое и оделась в одежду вдовства своего.
Gen 38:20  Иуда же послал козленка чрез друга своего Одолламитянина, чтобы взять залог из руки женщины, но он не нашел ее.
Gen 38:21  И спросил жителей того места, говоря: где блудница, [которая] [была] в Енаиме при дороге? Но они сказали: здесь не было блудницы.
Gen 38:22  И возвратился он к Иуде и сказал: я не нашел ее; да и жители места того сказали: здесь не было блудницы.
Gen 38:23  Иуда сказал: пусть она возьмет себе, чтобы только не стали над нами смеяться; вот, я посылал этого козленка, но ты не нашел ее.
Gen 38:24  Прошло около трех месяцев, и сказали Иуде, говоря: Фамарь, невестка твоя, впала в блуд, и вот, она беременна от блуда. Иуда сказал: выведите ее, и пусть она будет сожжена.
Gen 38:25  Но когда повели ее, она послала сказать свекру своему: я беременна от того, чьи эти вещи. И сказала: узнавай, чья эта печать и перевязь и трость.
Gen 38:26  Иуда узнал и сказал: она правее меня, потому что я не дал ее Шеле, сыну моему. И не познавал ее более.
Gen 38:27  Во время родов ее оказалось, что близнецы в утробе ее.
Gen 38:28  И во время родов ее показалась рука; и взяла повивальная бабка и навязала ему на руку красную нить, сказав: этот вышел первый.
Gen 38:29  Но он возвратил руку свою; и вот, вышел брат его. И она сказала: как ты расторг себе преграду? И наречено ему имя: Фарес.
Gen 38:30  Потом вышел брат его с красной нитью на руке. И наречено ему имя: Зара.
Gen 39:1  Иосиф же отведен был в Египет, и купил его из рук Измаильтян, приведших его туда, Египтянин Потифар, царедворец фараонов, начальник телохранителей.
Gen 39:2  И был Господь с Иосифом: он был успешен в делах и жил в доме господина своего, Египтянина.
Gen 39:3  И увидел господин его, что Господь с ним и что всему, что он делает, Господь в руках его дает успех.
Gen 39:4  И снискал Иосиф благоволение в очах его и служил ему. И он поставил его над домом своим, и все, что имел, отдал на руки его.
Gen 39:5  И с того времени, как он поставил его над домом своим и над всем, что имел, Господь благословил дом Египтянина ради Иосифа, и было благословение Господне на всем, что имел он в доме и в поле.
Gen 39:6  И оставил он все, что имел, в руках Иосифа и не знал при нем ничего, кроме хлеба, который он ел. Иосиф же был красив станом и красив лицем.
Gen 39:7  И обратила взоры на Иосифа жена господина его и сказала: спи со мною.
Gen 39:8  Но он отказался и сказал жене господина своего: вот, господин мой не знает при мне ничего в доме, и все, что имеет, отдал в мои руки;
Gen 39:9  нет больше меня в доме сем; и он не запретил мне ничего, кроме тебя, потому что ты жена ему; как же сделаю я сие великое зло и согрешу пред Богом?
Gen 39:10  Когда так она ежедневно говорила Иосифу, а он не слушался ее, чтобы спать с нею и быть с нею,
Gen 39:11  случилось в один день, что он вошел в дом делать дело свое, а никого из домашних тут в доме не было;
Gen 39:12  она схватила его за одежду его и сказала: ложись со мной. Но он, оставив одежду свою в руках ее, побежал и выбежал вон.
Gen 39:13  Она же, увидев, что он оставил одежду свою в руках ее и побежал вон,
Gen 39:14  кликнула домашних своих и сказала им так: посмотрите, он привел к нам Еврея ругаться над нами. Он пришел ко мне, чтобы лечь со мною, но я закричала громким голосом,
Gen 39:15  и он, услышав, что я подняла вопль и закричала, оставил у меня одежду свою, и побежал, и выбежал вон.
Gen 39:16  И оставила одежду его у себя до прихода господина его в дом свой.
Gen 39:17  И пересказала ему те же слова, говоря: раб Еврей, которого ты привел к нам, приходил ко мне ругаться надо мною.
Gen 39:18  но, когда я подняла вопль и закричала, он оставил у меня одежду свою и убежал вон.
Gen 39:19  Когда господин его услышал слова жены своей, которые она сказала ему, говоря: так поступил со мною раб твой, то воспылал гневом;
Gen 39:20  и взял Иосифа господин его и отдал его в темницу, где заключены узники царя. И был он там в темнице.
Gen 39:21  И Господь был с Иосифом, и простер к нему милость, и даровал ему благоволение в очах начальника темницы.
Gen 39:22  И отдал начальник темницы в руки Иосифу всех узников, находившихся в темнице, и во всем, что они там ни делали, он был распорядителем.
Gen 39:23  Начальник темницы и не смотрел ни за чем, что было у него в руках, потому что Господь был с [Иосифом], и во всем, что он делал, Господь давал успех.
Gen 40:1  После сего виночерпий царя Египетского и хлебодар провинились пред господином своим, царем Египетским.
Gen 40:2  И прогневался фараон на двух царедворцев своих, на главного виночерпия и на главного хлебодара,
Gen 40:3  и отдал их под стражу в дом начальника телохранителей, в темницу, в место, где заключен был Иосиф.
Gen 40:4  Начальник телохранителей приставил к ним Иосифа, и он служил им. И пробыли они под стражею несколько времени.
Gen 40:5  Однажды виночерпию и хлебодару царя Египетского, заключенным в темнице, виделись сны, каждому свой сон, обоим в одну ночь, каждому сон особенного значения.
Gen 40:6  И пришел к ним Иосиф поутру, увидел их, и вот, они в смущении.
Gen 40:7  И спросил он царедворцев фараоновых, находившихся с ним в доме господина его под стражею, говоря: отчего у вас сегодня печальные лица?
Gen 40:8  Они сказали ему: нам виделись сны; а истолковать их некому. Иосиф сказал им: не от Бога ли истолкования? расскажите мне.
Gen 40:9  И рассказал главный виночерпий Иосифу сон свой и сказал ему: мне снилось, вот виноградная лоза предо мною;
Gen 40:10  на лозе три ветви; она развилась, показался на ней цвет, выросли и созрели на ней ягоды;
Gen 40:11  и чаша фараонова в руке у меня; я взял ягод, выжал их в чашу фараонову и подал чашу в руку фараону.
Gen 40:12  И сказал ему Иосиф: вот истолкование его: три ветви--это три дня;
Gen 40:13  через три дня фараон вознесет главу твою и возвратит тебя на место твое, и ты подашь чашу фараонову в руку его, по прежнему обыкновению, когда ты был у него виночерпием;
Gen 40:14  вспомни же меня, когда хорошо тебе будет, и сделай мне благодеяние, и упомяни обо мне фараону, и выведи меня из этого дома,
Gen 40:15  ибо я украден из земли Евреев; а также и здесь ничего не сделал, за что бы бросить меня в темницу.
Gen 40:16  Главный хлебодар увидел, что истолковал он хорошо, и сказал Иосифу: мне также снилось: вот на голове у меня три корзины решетчатых;
Gen 40:17  в верхней корзине всякая пища фараонова, изделие пекаря, и птицы клевали ее из корзины на голове моей.
Gen 40:18  И отвечал Иосиф и сказал: вот истолкование его: три корзины--это три дня;
Gen 40:19  через три дня фараон снимет с тебя голову твою и повесит тебя на дереве, и птицы будут клевать плоть твою с тебя.
Gen 40:20  На третий день, день рождения фараонова, сделал он пир для всех слуг своих и вспомнил о главном виночерпии и главном хлебодаре среди слуг своих;
Gen 40:21  и возвратил главного виночерпия на прежнее место, и он подал чашу в руку фараону,
Gen 40:22  а главного хлебодара повесил, как истолковал им Иосиф.
Gen 40:23  И не вспомнил главный виночерпий об Иосифе, но забыл его.
Gen 41:1  По прошествии двух лет фараону снилось: вот, он стоит у реки;
Gen 41:2  и вот, вышли из реки семь коров, хороших видом и тучных плотью, и паслись в тростнике;
Gen 41:3  но вот, после них вышли из реки семь коров других, худых видом и тощих плотью, и стали подле тех коров, на берегу реки;
Gen 41:4  и съели коровы худые видом и тощие плотью семь коров хороших видом и тучных. И проснулся фараон,
Gen 41:5  и заснул опять, и снилось ему в другой раз: вот, на одном стебле поднялось семь колосьев тучных и хороших;
Gen 41:6  но вот, после них выросло семь колосьев тощих и иссушенных восточным ветром;
Gen 41:7  и пожрали тощие колосья семь колосьев тучных и полных. И проснулся фараон и [понял, что] это сон.
Gen 41:8  Утром смутился дух его, и послал он, и призвал всех волхвов Египта и всех мудрецов его, и рассказал им фараон сон свой; но не было никого, кто бы истолковал его фараону.
Gen 41:9  И стал говорить главный виночерпий фараону и сказал: грехи мои вспоминаю я ныне;
Gen 41:10  фараон прогневался на рабов своих и отдал меня и главного хлебодара под стражу в дом начальника телохранителей;
Gen 41:11  и снился нам сон в одну ночь, мне и ему, каждому снился сон особенного значения;
Gen 41:12  там же был с нами молодой Еврей, раб начальника телохранителей; мы рассказали ему сны наши, и он истолковал нам каждому соответственно с его сновидением;
Gen 41:13  и как он истолковал нам, так и сбылось: я возвращен на место мое, а тот повешен.
Gen 41:14  И послал фараон и позвал Иосифа. И поспешно вывели его из темницы. Он остригся и переменил одежду свою и пришел к фараону.
Gen 41:15  Фараон сказал Иосифу: мне снился сон, и нет никого, кто бы истолковал его, а о тебе я слышал, что ты умеешь толковать сны.
Gen 41:16  И отвечал Иосиф фараону, говоря: это не мое; Бог даст ответ во благо фараону.
Gen 41:17  И сказал фараон Иосифу: мне снилось: вот, стою я на берегу реки;
Gen 41:18  и вот, вышли из реки семь коров тучных плотью и хороших видом и паслись в тростнике;
Gen 41:19  но вот, после них вышли семь коров других, худых, очень дурных видом и тощих плотью: я не видывал во всей земле Египетской таких худых, как они;
Gen 41:20  и съели тощие и худые коровы прежних семь коров тучных;
Gen 41:21  и вошли [тучные] в утробу их, но не приметно было, что они вошли в утробу их: они были так же худы видом, как и сначала. И я проснулся.
Gen 41:22  [Потом] снилось мне: вот, на одном стебле поднялись семь колосьев полных и хороших;
Gen 41:23  но вот, после них выросло семь колосьев тонких, тощих и иссушенных восточным ветром;
Gen 41:24  и пожрали тощие колосья семь колосьев хороших. Я рассказал это волхвам, но никто не изъяснил мне.
Gen 41:25  И сказал Иосиф фараону: сон фараонов один: что Бог сделает, то Он возвестил фараону.
Gen 41:26  Семь коров хороших, это семь лет; и семь колосьев хороших, это семь лет: сон один;
Gen 41:27  и семь коров тощих и худых, вышедших после тех, это семь лет, также и семь колосьев тощих и иссушенных восточным ветром, это семь лет голода.
Gen 41:28  Вот почему сказал я фараону: что Бог сделает, то Он показал фараону.
Gen 41:29  Вот, наступает семь лет великого изобилия во всей земле Египетской;
Gen 41:30  после них настанут семь лет голода, и забудется все то изобилие в земле Египетской, и истощит голод землю,
Gen 41:31  и неприметно будет прежнее изобилие на земле, по причине голода, который последует, ибо он будет очень тяжел.
Gen 41:32  А что сон повторился фараону дважды, [это значит], что сие истинно слово Божие, и что вскоре Бог исполнит сие.
Gen 41:33  И ныне да усмотрит фараон мужа разумного и мудрого и да поставит его над землею Египетскою.
Gen 41:34  Да повелит фараон поставить над землею надзирателей и собирать в семь лет изобилия пятую часть с земли Египетской;
Gen 41:35  пусть они берут всякий хлеб этих наступающих хороших годов и соберут в городах хлеб под ведение фараона в пищу, и пусть берегут;
Gen 41:36  и будет сия пища в запас для земли на семь лет голода, которые будут в земле Египетской, дабы земля не погибла от голода.
Gen 41:37  Сие понравилось фараону и всем слугам его.
Gen 41:38  И сказал фараон слугам своим: найдем ли мы такого, как он, человека, в котором был бы Дух Божий?
Gen 41:39  И сказал фараон Иосифу: так как Бог открыл тебе все сие, то нет столь разумного и мудрого, как ты;
Gen 41:40  ты будешь над домом моим, и твоего слова держаться будет весь народ мой; только престолом я буду больше тебя.
Gen 41:41  И сказал фараон Иосифу: вот, я поставляю тебя над всею землею Египетскою.
Gen 41:42  И снял фараон перстень свой с руки своей и надел его на руку Иосифа; одел его в виссонные одежды, возложил золотую цепь на шею ему;
Gen 41:43  велел везти его на второй из своих колесниц и провозглашать пред ним: преклоняйтесь! И поставил его над всею землею Египетскою.
Gen 41:44  И сказал фараон Иосифу: я фараон; без тебя никто не двинет ни руки своей, ни ноги своей во всей земле Египетской.
Gen 41:45  И нарек фараон Иосифу имя: Цафнаф-панеах, и дал ему в жену Асенефу, дочь Потифера, жреца Илиопольского. И пошел Иосиф по земле Египетской.
Gen 41:46  Иосифу было тридцать лет от рождения, когда он предстал пред лице фараона, царя Египетского. И вышел Иосиф от лица фараонова и прошел по всей земле Египетской.
Gen 41:47  Земля же в семь лет изобилия приносила [из зерна] по горсти.
Gen 41:48  И собрал он всякий хлеб семи лет, которые были [плодородны] в земле Египетской, и положил хлеб в городах; в [каждом] городе положил хлеб полей, окружающих его.
Gen 41:49  И скопил Иосиф хлеба весьма много, как песку морского, так что перестал и считать, ибо не стало счета.
Gen 41:50  До наступления годов голода, у Иосифа родились два сына, которых родила ему Асенефа, дочь Потифера, жреца Илиопольского.
Gen 41:51  И нарек Иосиф имя первенцу: Манассия, потому что [говорил он] Бог дал мне забыть все несчастья мои и весь дом отца моего.
Gen 41:52  А другому нарек имя: Ефрем, потому что [говорил он] Бог сделал меня плодовитым в земле страдания моего.
Gen 41:53  И прошли семь лет изобилия, которое было в земле Египетской,
Gen 41:54  и наступили семь лет голода, как сказал Иосиф. И был голод во всех землях, а во всей земле Египетской был хлеб.
Gen 41:55  Но когда и вся земля Египетская начала терпеть голод, то народ начал вопиять к фараону о хлебе. И сказал фараон всем Египтянам: пойдите к Иосифу и делайте, что он вам скажет.
Gen 41:56  И был голод по всей земле; и отворил Иосиф все житницы, и стал продавать хлеб Египтянам. Голод же усиливался в земле Египетской.
Gen 41:57  И из всех стран приходили в Египет покупать хлеб у Иосифа, ибо голод усилился по всей земле.
Gen 42:1  И узнал Иаков, что в Египте есть хлеб, и сказал Иаков сыновьям своим: что вы смотрите?
Gen 42:2  И сказал: вот, я слышал, что есть хлеб в Египте; пойдите туда и купите нам оттуда хлеба, чтобы нам жить и не умереть.
Gen 42:3  Десять братьев Иосифовых пошли купить хлеба в Египте,
Gen 42:4  а Вениамина, брата Иосифова, не послал Иаков с братьями его, ибо сказал: не случилось бы с ним беды.
Gen 42:5  И пришли сыны Израилевы покупать хлеб, вместе с другими пришедшими, ибо в земле Ханаанской был голод.
Gen 42:6  Иосиф же был начальником в земле той; он и продавал хлеб всему народу земли. Братья Иосифа пришли и поклонились ему лицем до земли.
Gen 42:7  И увидел Иосиф братьев своих и узнал их; но показал, будто не знает их, и говорил с ними сурово и сказал им: откуда вы пришли? Они сказали: из земли Ханаанской, купить пищи.
Gen 42:8  Иосиф узнал братьев своих, но они не узнали его.
Gen 42:9  И вспомнил Иосиф сны, которые снились ему о них; и сказал им: вы соглядатаи, вы пришли высмотреть наготу земли сей.
Gen 42:10  Они сказали ему: нет, господин наш; рабы твои пришли купить пищи;
Gen 42:11  мы все дети одного человека; мы люди честные; рабы твои не бывали соглядатаями.
Gen 42:12  Он сказал им: нет, вы пришли высмотреть наготу земли сей.
Gen 42:13  Они сказали: нас, рабов твоих, двенадцать братьев; мы сыновья одного человека в земле Ханаанской, и вот, меньший теперь с отцом нашим, а одного не стало.
Gen 42:14  И сказал им Иосиф: это самое я и говорил вам, сказав: вы соглядатаи;
Gen 42:15  вот как вы будете испытаны: [клянусь] жизнью фараона, вы не выйдете отсюда, если не придет сюда меньший брат ваш;
Gen 42:16  пошлите одного из вас, и пусть он приведет брата вашего, а вы будете задержаны; и откроется, правда ли у вас; и если нет, [то клянусь] жизнью фараона, что вы соглядатаи.
Gen 42:17  И отдал их под стражу на три дня.
Gen 42:18  И сказал им Иосиф в третий день: вот что сделайте, и останетесь живы, ибо я боюсь Бога:
Gen 42:19  если вы люди честные, то один брат из вас пусть содержится в доме, где вы заключены; а вы пойдите, отвезите хлеб, ради голода семейств ваших;
Gen 42:20  брата же вашего меньшого приведите ко мне, чтобы оправдались слова ваши и чтобы не умереть вам. Так они и сделали.
Gen 42:21  И говорили они друг другу: точно мы наказываемся за грех против брата нашего; мы видели страдание души его, когда он умолял нас, но не послушали; за то и постигло нас горе сие.
Gen 42:22  Рувим отвечал им и сказал: не говорил ли я вам: не грешите против отрока? но вы не послушались; вот, кровь его взыскивается.
Gen 42:23  А того не знали они, что Иосиф понимает; ибо между ними был переводчик.
Gen 42:24  И отошел от них, и заплакал. И возвратился к ним, и говорил с ними, и, взяв из них Симеона, связал его пред глазами их.
Gen 42:25  И приказал Иосиф наполнить мешки их хлебом, а серебро их возвратить каждому в мешок его, и дать им запасов на дорогу. Так и сделано с ними.
Gen 42:26  Они положили хлеб свой на ослов своих, и пошли оттуда.
Gen 42:27  И открыл один [из них] мешок свой, чтобы дать корму ослу своему на ночлеге, и увидел серебро свое в отверстии мешка его,
Gen 42:28  и сказал своим братьям: серебро мое возвращено; вот оно в мешке у меня. И смутилось сердце их, и они с трепетом друг другу говорили: что это Бог сделал с нами?
Gen 42:29  И пришли к Иакову, отцу своему, в землю Ханаанскую и рассказали ему все случившееся с ними, говоря:
Gen 42:30  начальствующий над тою землею говорил с нами сурово и принял нас за соглядатаев земли той.
Gen 42:31  И сказали мы ему: мы люди честные; мы не бывали соглядатаями;
Gen 42:32  нас двенадцать братьев, сыновей у отца нашего; одного не стало, а меньший теперь с отцом нашим в земле Ханаанской.
Gen 42:33  И сказал нам начальствующий над тою землею: вот как узнаю я, честные ли вы люди: оставьте у меня одного брата из вас, а вы возьмите хлеб ради голода семейств ваших и пойдите,
Gen 42:34  и приведите ко мне меньшого брата вашего; и узнаю я, что вы не соглядатаи, но люди честные; отдам вам брата вашего, и вы можете промышлять в этой земле.
Gen 42:35  Когда же они опорожняли мешки свои, вот, у каждого узел серебра его в мешке его. И увидели они узлы серебра своего, они и отец их, и испугались.
Gen 42:36  И сказал им Иаков, отец их: вы лишили меня детей: Иосифа нет, и Симеона нет, и Вениамина взять хотите, --все это на меня!
Gen 42:37  И сказал Рувим отцу своему, говоря: убей двух моих сыновей, если я не приведу его к тебе; отдай его на мои руки; я возвращу его тебе.
Gen 42:38  Он сказал: не пойдет сын мой с вами; потому что брат его умер, и он один остался; если случится с ним несчастье на пути, в который вы пойдете, то сведете вы седину мою с печалью во гроб.
Gen 43:1  Голод усилился на земле.
Gen 43:2  И когда они съели хлеб, который привезли из Египта, тогда отец их сказал им: пойдите опять, купите нам немного пищи.
Gen 43:3  И сказал ему Иуда, говоря: тот человек решительно объявил нам, сказав: не являйтесь ко мне на лице, если брата вашего не будет с вами.
Gen 43:4  Если пошлешь с нами брата нашего, то пойдем и купим тебе пищи,
Gen 43:5  а если не пошлешь, то не пойдем, ибо тот человек сказал нам: не являйтесь ко мне на лице, если брата вашего не будет с вами.
Gen 43:6  Израиль сказал: для чего вы сделали мне такое зло, сказав тому человеку, что у вас есть еще брат?
Gen 43:7  Они сказали: расспрашивал тот человек о нас и о родстве нашем, говоря: жив ли еще отец ваш? есть ли у вас брат? Мы и рассказали ему по этим расспросам. Могли ли мы знать, что он скажет: приведите брата вашего?
Gen 43:8  Иуда же сказал Израилю, отцу своему: отпусти отрока со мною, и мы встанем и пойдем, и живы будем и не умрем и мы, и ты, и дети наши;
Gen 43:9  я отвечаю за него, из моих рук потребуешь его; если я не приведу его к тебе и не поставлю его пред лицем твоим, то останусь я виновным пред тобою во все дни жизни;
Gen 43:10  если бы мы не медлили, то уже сходили бы два раза.
Gen 43:11  Израиль, отец их, сказал им: если так, то вот что сделайте: возьмите с собою плодов земли сей и отнесите в дар тому человеку несколько бальзама и несколько меду, стираксы и ладану, фисташков и миндальных орехов;
Gen 43:12  возьмите и другое серебро в руки ваши; а серебро, обратно положенное в отверстие мешков ваших, возвратите руками вашими: может быть, это недосмотр;
Gen 43:13  и брата вашего возьмите и, встав, пойдите опять к человеку тому;
Gen 43:14  Бог же Всемогущий да даст вам найти милость у человека того, чтобы он отпустил вам и другого брата вашего и Вениамина, а мне если уже быть бездетным, то пусть буду бездетным.
Gen 43:15  И взяли те люди дары эти, и серебра вдвое взяли в руки свои, и Вениамина, и встали, пошли в Египет и предстали пред лице Иосифа.
Gen 43:16  Иосиф, увидев между ними Вениамина, сказал начальнику дома своего: введи сих людей в дом и заколи что-нибудь из скота, и приготовь, потому что со мною будут есть эти люди в полдень.
Gen 43:17  И сделал человек тот, как сказал Иосиф, и ввел человек тот людей сих в дом Иосифов.
Gen 43:18  И испугались люди эти, что ввели их в дом Иосифов, и сказали: это за серебро, возвращенное прежде в мешки наши, ввели нас, чтобы придраться к нам и напасть на нас, и взять нас в рабство, и ослов наших.
Gen 43:19  И подошли они к начальнику дома Иосифова, и стали говорить ему у дверей дома,
Gen 43:20  и сказали: послушай, господин наш, мы приходили уже прежде покупать пищи,
Gen 43:21  и случилось, что, когда пришли мы на ночлег и открыли мешки наши, --вот серебро каждого в отверстии мешка его, серебро наше по весу его, и мы возвращаем его своими руками;
Gen 43:22  а для покупки пищи мы принесли другое серебро в руках наших, мы не знаем, кто положил серебро наше в мешки наши.
Gen 43:23  Он сказал: будьте спокойны, не бойтесь; Бог ваш и Бог отца вашего дал вам клад в мешках ваших; серебро ваше дошло до меня. И привел к ним Симеона.
Gen 43:24  И ввел тот человек людей сих в дом Иосифов и дал воды, и они омыли ноги свои; и дал корму ослам их.
Gen 43:25  И они приготовили дары к приходу Иосифа в полдень, ибо слышали, что там будут есть хлеб.
Gen 43:26  И пришел Иосиф домой; и они принесли ему в дом дары, которые были на руках их, и поклонились ему до земли.
Gen 43:27  Он спросил их о здоровье и сказал: здоров ли отец ваш старец, о котором вы говорили? жив ли еще он?
Gen 43:28  Они сказали: здоров раб твой, отец наш; еще жив. И преклонились они и поклонились.
Gen 43:29  И поднял глаза свои, и увидел Вениамина, брата своего, сына матери своей, и сказал: это брат ваш меньший, о котором вы сказывали мне? И сказал: да будет милость Божия с тобою, сын мой!
Gen 43:30  И поспешно удалился Иосиф, потому что воскипела любовь к брату его, и он готов был заплакать, и вошел он во внутреннюю комнату и плакал там.
Gen 43:31  И умыв лице свое, вышел, и скрепился и сказал: подавайте кушанье.
Gen 43:32  И подали ему особо, и им особо, и Египтянам, обедавшим с ним, особо, ибо Египтяне не могут есть с Евреями, потому что это мерзость для Египтян.
Gen 43:33  И сели они пред ним, первородный по первородству его, и младший по молодости его, и дивились эти люди друг пред другом.
Gen 43:34  И посылались им кушанья от него, и доля Вениамина была впятеро больше долей каждого из них. И пили, и довольно пили они с ним.
Gen 44:1  И приказал [Иосиф] начальнику дома своего, говоря: наполни мешки этих людей пищею, сколько они могут нести, и серебро каждого положи в отверстие мешка его,
Gen 44:2  а чашу мою, чашу серебряную, положи в отверстие мешка к младшему вместе с серебром за купленный им хлеб. И сделал тот по слову Иосифа, которое сказал он.
Gen 44:3  Утром, когда рассвело, эти люди были отпущены, они и ослы их.
Gen 44:4  Еще не далеко отошли они от города, как Иосиф сказал начальнику дома своего: ступай, догоняй этих людей и, когда догонишь, скажи им: для чего вы заплатили злом за добро?
Gen 44:5  Не та ли это, из которой пьет господин мой и он гадает на ней? Худо это вы сделали.
Gen 44:6  Он догнал их и сказал им эти слова.
Gen 44:7  Они сказали ему: для чего господин наш говорит такие слова? Нет, рабы твои не сделают такого дела.
Gen 44:8  Вот, серебро, найденное нами в отверстии мешков наших, мы обратно принесли тебе из земли Ханаанской: как же нам украсть из дома господина твоего серебро или золото?
Gen 44:9  У кого из рабов твоих найдется, тому смерть, и мы будем рабами господину нашему.
Gen 44:10  Он сказал: хорошо; как вы сказали, так пусть и будет: у кого найдется [чаша], тот будет мне рабом, а вы будете не виноваты.
Gen 44:11  Они поспешно спустили каждый свой мешок на землю и открыли каждый свой мешок.
Gen 44:12  Он обыскал, начал со старшего и окончил младшим; и нашлась чаша в мешке Вениаминовом.
Gen 44:13  И разодрали они одежды свои, и, возложив каждый на осла своего ношу, возвратились в город.
Gen 44:14  И пришли Иуда и братья его в дом Иосифа, который был еще дома, и пали пред ним на землю.
Gen 44:15  Иосиф сказал им: что это вы сделали? разве вы не знали, что такой человек, как я, конечно угадает?
Gen 44:16  Иуда сказал: что нам сказать господину нашему? что говорить? чем оправдываться? Бог нашел неправду рабов твоих; вот, мы рабы господину нашему, и мы, и тот, в чьих руках нашлась чаша.
Gen 44:17  Но [Иосиф] сказал: нет, я этого не сделаю; тот, в чьих руках нашлась чаша, будет мне рабом, а вы пойдите с миром к отцу вашему.
Gen 44:18  И подошел Иуда к нему и сказал: господин мой, позволь рабу твоему сказать слово в уши господина моего, и не прогневайся на раба твоего, ибо ты то же, что фараон.
Gen 44:19  Господин мой спрашивал рабов своих, говоря: есть ли у вас отец или брат?
Gen 44:20  Мы сказали господину нашему, что у нас есть отец престарелый, и младший сын, сын старости, которого брат умер, а он остался один [от] матери своей, и отец любит его.
Gen 44:21  Ты же сказал рабам твоим: приведите его ко мне, чтобы мне взглянуть на него.
Gen 44:22  Мы сказали господину нашему: отрок не может оставить отца своего, и если он оставит отца своего, то сей умрет.
Gen 44:23  Но ты сказал рабам твоим: если не придет с вами меньший брат ваш, то вы более не являйтесь ко мне на лице.
Gen 44:24  Когда мы пришли к рабу твоему, отцу нашему, то пересказали ему слова господина моего.
Gen 44:25  И сказал отец наш: пойдите опять, купите нам немного пищи.
Gen 44:26  Мы сказали: нельзя нам идти; а если будет с нами меньший брат наш, то пойдем; потому что нельзя нам видеть лица того человека, если не будет с нами меньшого брата нашего.
Gen 44:27  И сказал нам раб твой, отец наш: вы знаете, что жена моя родила мне двух [сынов];
Gen 44:28  один пошел от меня, и я сказал: верно он растерзан; и я не видал его доныне;
Gen 44:29  если и сего возьмете от глаз моих, и случится с ним несчастье, то сведете вы седину мою с горестью во гроб.
Gen 44:30  Теперь если я приду к рабу твоему, отцу нашему, и не будет с нами отрока, с душею которого связана душа его,
Gen 44:31  то он, увидев, что нет отрока, умрет; и сведут рабы твои седину раба твоего, отца нашего, с печалью во гроб.
Gen 44:32  Притом я, раб твой, взялся отвечать за отрока отцу моему, сказав: если не приведу его к тебе, то останусь я виновным пред отцом моим во все дни жизни.
Gen 44:33  Итак пусть я, раб твой, вместо отрока останусь рабом у господина моего, а отрок пусть идет с братьями своими:
Gen 44:34  ибо как пойду я к отцу моему, когда отрока не будет со мною? я увидел бы бедствие, которое постигло бы отца моего.
Gen 45:1  Иосиф не мог более удерживаться при всех стоявших около него и закричал: удалите от меня всех. И не оставалось при Иосифе никого, когда он открылся братьям своим.
Gen 45:2  И громко зарыдал он, и услышали Египтяне, и услышал дом фараонов.
Gen 45:3  И сказал Иосиф братьям своим: я--Иосиф, жив ли еще отец мой? Но братья его не могли отвечать ему, потому что они смутились пред ним.
Gen 45:4  И сказал Иосиф братьям своим: подойдите ко мне. Они подошли. Он сказал: я--Иосиф, брат ваш, которого вы продали в Египет;
Gen 45:5  но теперь не печальтесь и не жалейте о том, что вы продали меня сюда, потому что Бог послал меня перед вами для сохранения вашей жизни;
Gen 45:6  ибо теперь два года голода на земле: еще пять лет, в которые ни орать, ни жать не будут;
Gen 45:7  Бог послал меня перед вами, чтобы оставить вас на земле и сохранить вашу жизнь великим избавлением.
Gen 45:8  Итак не вы послали меня сюда, но Бог, Который и поставил меня отцом фараону и господином во всем доме его и владыкою во всей земле Египетской.
Gen 45:9  Идите скорее к отцу моему и скажите ему: так говорит сын твой Иосиф: Бог поставил меня господином над всем Египтом; приди ко мне, не медли;
Gen 45:10  ты будешь жить в земле Гесем; и будешь близ меня, ты, и сыны твои, и сыны сынов твоих, и мелкий и крупный скот твой, и все твое;
Gen 45:11  и прокормлю тебя там, ибо голод будет еще пять лет, чтобы не обнищал ты и дом твой и все твое.
Gen 45:12  И вот, очи ваши и очи брата моего Вениамина видят, что это мои уста говорят с вами;
Gen 45:13  скажите же отцу моему о всей славе моей в Египте и о всем, что вы видели, и приведите скорее отца моего сюда.
Gen 45:14  И пал он на шею Вениамину, брату своему, и плакал; и Вениамин плакал на шее его.
Gen 45:15  И целовал всех братьев своих и плакал, обнимая их. Потом говорили с ним братья его.
Gen 45:16  Дошел в дом фараона слух, что пришли братья Иосифа; и приятно было фараону и рабам его.
Gen 45:17  И сказал фараон Иосифу: скажи братьям твоим: вот что сделайте: навьючьте скот ваш, и ступайте в землю Ханаанскую;
Gen 45:18  и возьмите отца вашего и семейства ваши и придите ко мне; я дам вам лучшее в земле Египетской, и вы будете есть тук земли.
Gen 45:19  Тебе же повелеваю сказать им: сделайте сие: возьмите себе из земли Египетской колесниц для детей ваших и для жен ваших, и привезите отца вашего и придите;
Gen 45:20  и не жалейте вещей ваших, ибо лучшее из всей земли Египетской [дам] вам.
Gen 45:21  Так и сделали сыны Израилевы. И дал им Иосиф колесницы по приказанию фараона, и дал им путевой запас,
Gen 45:22  каждому из них он дал перемену одежд, а Вениамину дал триста сребренников и пять перемен одежд;
Gen 45:23  также и отцу своему послал десять ослов, навьюченных лучшими [произведениями] Египетскими, и десять ослиц, навьюченных зерном, хлебом и припасами отцу своему на путь.
Gen 45:24  И отпустил братьев своих, и они пошли. И сказал им: не ссорьтесь на дороге.
Gen 45:25  И пошли они из Египта, и пришли в землю Ханаанскую к Иакову, отцу своему,
Gen 45:26  и известили его, сказав: Иосиф жив, и теперь владычествует над всею землею Египетскою. Но сердце его смутилось, ибо он не верил им.
Gen 45:27  Когда же они пересказали ему все слова Иосифа, которые он говорил им, и когда увидел колесницы, которые прислал Иосиф, чтобы везти его, тогда ожил дух Иакова, отца их,
Gen 45:28  и сказал Израиль: довольно, еще жив сын мой Иосиф; пойду и увижу его, пока не умру.
Gen 46:1  И отправился Израиль со всем, что у него было, и пришел в Вирсавию, и принес жертвы Богу отца своего Исаака.
Gen 46:2  И сказал Бог Израилю в видении ночном: Иаков! Иаков! Он сказал: вот я.
Gen 46:3  [Бог] сказал: Я Бог, Бог отца твоего; не бойся идти в Египет, ибо там произведу от тебя народ великий;
Gen 46:4  Я пойду с тобою в Египет, Я и выведу тебя обратно. Иосиф своею рукою закроет глаза [твои].
Gen 46:5  Иаков отправился из Вирсавии; и повезли сыны Израилевы Иакова отца своего, и детей своих, и жен своих на колесницах, которые послал фараон, чтобы привезти его.
Gen 46:6  И взяли они скот свой и имущество свое, которое приобрели в земле Ханаанской, и пришли в Египет, --Иаков и весь род его с ним.
Gen 46:7  Сынов своих и внуков своих с собою, дочерей своих и внучек своих и весь род свой привел он с собою в Египет.
Gen 46:8  Вот имена сынов Израилевых, пришедших в Египет: Иаков и сыновья его. Первенец Иакова Рувим.
Gen 46:9  Сыны Рувима: Ханох и Фаллу, Хецрон и Харми.
Gen 46:10  Сыны Симеона: Иемуил и Иамин, и Огад, и Иахин, и Цохар, и Саул, сын Хананеянки.
Gen 46:11  Сыны Левия: Гирсон, Кааф и Мерари.
Gen 46:12  Сыны Иуды: Ир и Онан, и Шела, и Фарес, и Зара; но Ир и Онан умерли в земле Ханаанской. Сыны Фареса были: Есром и Хамул.
Gen 46:13  Сыны Иссахара: Фола и Фува, Иов и Шимрон.
Gen 46:14  Сыны Завулона: Серед и Елон, и Иахлеил.
Gen 46:15  Это сыны Лии, которых она родила Иакову в Месопотамии, и Дину, дочь его. Всех душ сынов его и дочерей его--тридцать три.
Gen 46:16  Сыны Гада: Цифион и Хагги, Шуни и Эцбон, Ери и Ароди и Арели.
Gen 46:17  Сыны Асира: Имна и Ишва, и Ишви, и Бриа, и Серах, сестра их. Сыны Брии: Хевер и Малхиил.
Gen 46:18  Это сыны Зелфы, которую Лаван дал Лии, дочери своей; она родила их Иакову шестнадцать душ.
Gen 46:19  Сыны Рахили, жены Иакова: Иосиф и Вениамин.
Gen 46:20  И родились у Иосифа в земле Египетской Манассия и Ефрем, которых родила ему Асенефа, дочь Потифера, жреца Илиопольского.
Gen 46:21  Сыны Вениамина: Бела и Бехер и Ашбел; Гера и Нааман, Эхи и Рош, Муппим и Хуппим и Ард.
Gen 46:22  Это сыны Рахили, которые родились у Иакова, всего четырнадцать душ.
Gen 46:23  Сын Дана: Хушим.
Gen 46:24  Сыны Неффалима: Иахцеил и Гуни, и Иецер, и Шиллем.
Gen 46:25  Это сыны Валлы, которую дал Лаван дочери своей Рахили; она родила их Иакову всего семь душ.
Gen 46:26  Всех душ, пришедших с Иаковом в Египет, которые произошли из чресл его, кроме жен сынов Иаковлевых, всего шестьдесят шесть душ.
Gen 46:27  Сынов Иосифа, которые родились у него в Египте, две души. Всех душ дома Иаковлева, перешедших в Египет, семьдесят.
Gen 46:28  Иуду послал он пред собою к Иосифу, чтобы он указал [путь] в Гесем. И пришли в землю Гесем.
Gen 46:29  Иосиф запряг колесницу свою и выехал навстречу Израилю, отцу своему, в Гесем, и, увидев его, пал на шею его, и долго плакал на шее его.
Gen 46:30  И сказал Израиль Иосифу: умру я теперь, увидев лице твое, ибо ты еще жив.
Gen 46:31  И сказал Иосиф братьям своим и дому отца своего: я пойду, извещу фараона и скажу ему: братья мои и дом отца моего, которые были в земле Ханаанской, пришли ко мне;
Gen 46:32  эти люди пастухи овец, ибо скотоводы они; и мелкий и крупный скот свой, и все, что у них, привели они.
Gen 46:33  Если фараон призовет вас и скажет: какое занятие ваше?
Gen 46:34  то вы скажите: [мы], рабы твои, скотоводами были от юности нашей доныне, и мы и отцы наши, чтобы вас поселили в земле Гесем. Ибо мерзость для Египтян всякий пастух овец.
Gen 47:1  И пришел Иосиф и известил фараона и сказал: отец мой и братья мои, с мелким и крупным скотом своим и со всем, что у них, пришли из земли Ханаанской; и вот, они в земле Гесем.
Gen 47:2  И из братьев своих он взял пять человек и представил их фараону.
Gen 47:3  И сказал фараон братьям его: какое ваше занятие? Они сказали фараону: пастухи овец рабы твои, и мы и отцы наши.
Gen 47:4  И сказали они фараону: мы пришли пожить в этой земле, потому что нет пажити для скота рабов твоих, ибо в земле Ханаанской сильный голод; итак позволь поселиться рабам твоим в земле Гесем.
Gen 47:5  И сказал фараон Иосифу: отец твой и братья твои пришли к тебе;
Gen 47:6  земля Египетская пред тобою; на лучшем месте земли посели отца твоего и братьев твоих; пусть живут они в земле Гесем; и если знаешь, что между ними есть способные люди, поставь их смотрителями над моим скотом.
Gen 47:7  И привел Иосиф Иакова, отца своего, и представил его фараону; и благословил Иаков фараона.
Gen 47:8  Фараон сказал Иакову: сколько лет жизни твоей?
Gen 47:9  Иаков сказал фараону: дней странствования моего сто тридцать лет; малы и несчастны дни жизни моей и не достигли до лет жизни отцов моих во днях странствования их.
Gen 47:10  И благословил фараона Иаков и вышел от фараона.
Gen 47:11  И поселил Иосиф отца своего и братьев своих, и дал им владение в земле Египетской, в лучшей части земли, в земле Раамсес, как повелел фараон.
Gen 47:12  И снабжал Иосиф отца своего и братьев своих и весь дом отца своего хлебом, по потребностям каждого семейства.
Gen 47:13  И не было хлеба по всей земле, потому что голод весьма усилился, и изнурены были от голода земля Египетская и земля Ханаанская.
Gen 47:14  Иосиф собрал все серебро, какое было в земле Египетской и в земле Ханаанской, за хлеб, который покупали, и внес Иосиф серебро в дом фараонов.
Gen 47:15  И серебро истощилось в земле Египетской и в земле Ханаанской. Все Египтяне пришли к Иосифу и говорили: дай нам хлеба; зачем нам умирать пред тобою, потому что серебро вышло у нас?
Gen 47:16  Иосиф сказал: пригоняйте скот ваш, и я буду давать вам за скот ваш, если серебро вышло у вас.
Gen 47:17  И пригоняли они к Иосифу скот свой; и давал им Иосиф хлеб за лошадей, и за стада мелкого скота, и за стада крупного скота, и за ослов; и снабжал их хлебом в тот год за весь скот их.
Gen 47:18  И прошел этот год; и пришли к нему на другой год и сказали ему: не скроем от господина нашего, что серебро истощилось и стада скота нашего у господина нашего; ничего не осталось у нас пред господином нашим, кроме тел наших и земель наших;
Gen 47:19  для чего нам погибать в глазах твоих, и нам и землям нашим? купи нас и земли наши за хлеб, и мы с землями нашими будем рабами фараону, а ты дай нам семян, чтобы нам быть живыми и не умереть, и чтобы не опустела земля.
Gen 47:20  И купил Иосиф всю землю Египетскую для фараона, потому что продали Египтяне каждый свое поле, ибо голод одолевал их. И досталась земля фараону.
Gen 47:21  И народ сделал он рабами от одного конца Египта до другого.
Gen 47:22  Только земли жрецов не купил, ибо жрецам от фараона положен был участок, и они питались своим участком, который дал им фараон; посему и не продали земли своей.
Gen 47:23  И сказал Иосиф народу: вот, я купил теперь для фараона вас и землю вашу; вот вам семена, и засевайте землю;
Gen 47:24  когда будет жатва, давайте пятую часть фараону, а четыре части останутся вам на засеяние полей, на пропитание вам и тем, кто в домах ваших, и на пропитание детям вашим.
Gen 47:25  Они сказали: ты спас нам жизнь; да обретем милость в очах господина нашего и да будем рабами фараону.
Gen 47:26  И поставил Иосиф в закон земле Египетской, даже до сего дня: пятую часть давать фараону, исключая только землю жрецов, которая не принадлежала фараону.
Gen 47:27  И жил Израиль в земле Египетской, в земле Гесем, и владели они ею, и плодились, и весьма умножились.
Gen 47:28  И жил Иаков в земле Египетской семнадцать лет; и было дней Иакова, годов жизни его, сто сорок семь лет.
Gen 47:29  И пришло время Израилю умереть, и призвал он сына своего Иосифа и сказал ему: если я нашел благоволение в очах твоих, положи руку твою под стегно мое и [клянись], что ты окажешь мне милость и правду, не похоронишь меня в Египте,
Gen 47:30  дабы мне лечь с отцами моими; вынесешь меня из Египта и похоронишь меня в их гробнице. [Иосиф] сказал: сделаю по слову твоему.
Gen 47:31  И сказал: клянись мне. И клялся ему. И поклонился Израиль на возглавие постели.
Gen 48:1  После того Иосифу сказали: вот, отец твой болен. И он взял с собою двух сынов своих, Манассию и Ефрема.
Gen 48:2  Иакова известили и сказали: вот, сын твой Иосиф идет к тебе. Израиль собрал силы свои и сел на постели.
Gen 48:3  И сказал Иаков Иосифу: Бог Всемогущий явился мне в Лузе, в земле Ханаанской, и благословил меня,
Gen 48:4  и сказал мне: вот, Я распложу тебя, и размножу тебя, и произведу от тебя множество народов, и дам землю сию потомству твоему после тебя, в вечное владение.
Gen 48:5  И ныне два сына твои, родившиеся тебе в земле Египетской, до моего прибытия к тебе в Египет, мои они; Ефрем и Манассия, как Рувим и Симеон, будут мои;
Gen 48:6  дети же твои, которые родятся от тебя после них, будут твои; они под именем братьев своих будут именоваться в их уделе.
Gen 48:7  Когда я шел из Месопотамии, умерла у меня Рахиль в земле Ханаанской, по дороге, не доходя несколько до Ефрафы, и я похоронил ее там на дороге к Ефрафе, что [ныне] Вифлеем.
Gen 48:8  И увидел Израиль сыновей Иосифа и сказал: кто это?
Gen 48:9  И сказал Иосиф отцу своему: это сыновья мои, которых Бог дал мне здесь. Иаков сказал: подведи их ко мне, и я благословлю их.
Gen 48:10  Глаза же Израилевы притупились от старости; не мог он видеть [ясно. Иосиф] подвел их к нему, и он поцеловал их и обнял их.
Gen 48:11  И сказал Израиль Иосифу: не надеялся я видеть твое лице; но вот, Бог показал мне и детей твоих.
Gen 48:12  И отвел их Иосиф от колен его и поклонился ему лицем своим до земли.
Gen 48:13  И взял Иосиф обоих, Ефрема в правую свою руку против левой Израиля, а Манассию в левую против правой Израиля, и подвел к нему.
Gen 48:14  Но Израиль простер правую руку свою и положил на голову Ефрему, хотя сей был меньший, а левую на голову Манассии. С намерением положил он так руки свои, хотя Манассия был первенец.
Gen 48:15  И благословил Иосифа и сказал: Бог, пред Которым ходили отцы мои Авраам и Исаак, Бог, пасущий меня с тех пор, как я существую, до сего дня,
Gen 48:16  Ангел, избавляющий меня от всякого зла, да благословит отроков сих; да будет на них наречено имя мое и имя отцов моих Авраама и Исаака, и да возрастут они во множество посреди земли.
Gen 48:17  И увидел Иосиф, что отец его положил правую руку свою на голову Ефрема; и прискорбно было ему это. И взял он руку отца своего, чтобы переложить ее с головы Ефрема на голову Манассии,
Gen 48:18  и сказал Иосиф отцу своему: не так, отец мой, ибо это--первенец; положи на его голову правую руку твою.
Gen 48:19  Но отец его не согласился и сказал: знаю, сын мой, знаю; и от него произойдет народ, и он будет велик; но меньший его брат будет больше его, и от семени его произойдет многочисленный народ.
Gen 48:20  И благословил их в тот день, говоря: тобою будет благословлять Израиль, говоря: Бог да сотворит тебе, как Ефрему и Манассии. И поставил Ефрема выше Манассии.
Gen 48:21  И сказал Израиль Иосифу: вот, я умираю; и Бог будет с вами и возвратит вас в землю отцов ваших;
Gen 48:22  я даю тебе, преимущественно пред братьями твоими, один участок, который я взял из рук Аморреев мечом моим и луком моим.
Gen 49:1  И призвал Иаков сыновей своих и сказал: соберитесь, и я возвещу вам, что будет с вами в грядущие дни;
Gen 49:2  сойдитесь и послушайте, сыны Иакова, послушайте Израиля, отца вашего.
Gen 49:3  Рувим, первенец мой! ты--крепость моя и начаток силы моей, верх достоинства и верх могущества;
Gen 49:4  но ты бушевал, как вода, --не будешь преимуществовать, ибо ты взошел на ложе отца твоего, ты осквернил постель мою, взошел.
Gen 49:5  Симеон и Левий братья, орудия жестокости мечи их;
Gen 49:6  в совет их да не внидет душа моя, и к собранию их да не приобщится слава моя, ибо они во гневе своем убили мужа и по прихоти своей перерезали жилы тельца;
Gen 49:7  проклят гнев их, ибо жесток, и ярость их, ибо свирепа; разделю их в Иакове и рассею их в Израиле.
Gen 49:8  Иуда! тебя восхвалят братья твои. Рука твоя на хребте врагов твоих; поклонятся тебе сыны отца твоего.
Gen 49:9  Молодой лев Иуда, с добычи, сын мой, поднимается. Преклонился он, лег, как лев и как львица: кто поднимет его?
Gen 49:10  Не отойдет скипетр от Иуды и законодатель от чресл его, доколе не приидет Примиритель, и Ему покорность народов.
Gen 49:11  Он привязывает к виноградной лозе осленка своего и к лозе лучшего винограда сына ослицы своей; моет в вине одежду свою и в крови гроздов одеяние свое;
Gen 49:12  блестящи очи [его] от вина, и белы зубы от молока.
Gen 49:13  Завулон при береге морском будет жить и у пристани корабельной, и предел его до Сидона.
Gen 49:14  Иссахар осел крепкий, лежащий между протоками вод;
Gen 49:15  и увидел он, что покой хорош, и что земля приятна: и преклонил плечи свои для ношения бремени и стал работать в уплату дани.
Gen 49:16  Дан будет судить народ свой, как одно из колен Израиля;
Gen 49:17  Дан будет змеем на дороге, аспидом на пути, уязвляющим ногу коня, так что всадник его упадет назад.
Gen 49:18  На помощь твою надеюсь, Господи!
Gen 49:19  Гад, --толпа будет теснить его, но он оттеснит ее по пятам.
Gen 49:20  Для Асира--слишком тучен хлеб его, и он будет доставлять царские яства.
Gen 49:21  Неффалим--теревинф рослый, распускающий прекрасные ветви.
Gen 49:22  Иосиф--отрасль плодоносного [дерева], отрасль плодоносного [дерева] над источником; ветви его простираются над стеною;
Gen 49:23  огорчали его, и стреляли и враждовали на него стрельцы,
Gen 49:24  но тверд остался лук его, и крепки мышцы рук его, от рук мощного [Бога] Иаковлева. Оттуда Пастырь и твердыня Израилева,
Gen 49:25  от Бога отца твоего, [Который] и да поможет тебе, и от Всемогущего, Который и да благословит тебя благословениями небесными свыше, благословениями бездны, лежащей долу, благословениями сосцов и утробы,
Gen 49:26  благословениями отца твоего, которые превышают благословения гор древних и приятности холмов вечных; да будут они на голове Иосифа и на темени избранного между братьями своими.
Gen 49:27  Вениамин, хищный волк, утром будет есть ловитву и вечером будет делить добычу.
Gen 49:28  Вот все двенадцать колен Израилевых; и вот что сказал им отец их; и благословил их, и дал им благословение, каждому свое.
Gen 49:29  И заповедал он им и сказал им: я прилагаюсь к народу моему; похороните меня с отцами моими в пещере, которая на поле Ефрона Хеттеянина,
Gen 49:30  в пещере, которая на поле Махпела, что пред Мамре, в земле Ханаанской, которую купил Авраам с полем у Ефрона Хеттеянина в собственность для погребения;
Gen 49:31  там похоронили Авраама и Сарру, жену его; там похоронили Исаака и Ревекку, жену его; и там похоронил я Лию;
Gen 49:32  это поле и пещера, которая на нем, куплена у сынов Хеттеевых.
Gen 49:33  И окончил Иаков завещание сыновьям своим, и положил ноги свои на постель, и скончался, и приложился к народу своему.
Gen 50:1  Иосиф пал на лице отца своего, и плакал над ним, и целовал его.
Gen 50:2  И повелел Иосиф слугам своим--врачам, бальзамировать отца его; и врачи набальзамировали Израиля.
Gen 50:3  И исполнилось ему сорок дней, ибо столько дней употребляется на бальзамирование, и оплакивали его Египтяне семьдесят дней.
Gen 50:4  Когда же прошли дни плача по нем, Иосиф сказал придворным фараона, говоря: если я обрел благоволение в очах ваших, то скажите фараону так:
Gen 50:5  отец мой заклял меня, сказав: вот, я умираю; во гробе моем, который я выкопал себе в земле Ханаанской, там похорони меня. И теперь хотел бы я пойти и похоронить отца моего и возвратиться.
Gen 50:6  И сказал фараон: пойди и похорони отца твоего, как он заклял тебя.
Gen 50:7  И пошел Иосиф хоронить отца своего. И пошли с ним все слуги фараона, старейшины дома его и все старейшины земли Египетской,
Gen 50:8  и весь дом Иосифа, и братья его, и дом отца его. Только детей своих и мелкий и крупный скот свой оставили в земле Гесем.
Gen 50:9  С ним отправились также колесницы и всадники, так что сонм был весьма велик.
Gen 50:10  И дошли они до Горен-гаатада при Иордане и плакали там плачем великим и весьма сильным; и сделал [Иосиф] плач по отце своем семь дней.
Gen 50:11  И видели жители земли той, Хананеи, плач в Горен-гаатаде, и сказали: велик плач этот у Египтян! Посему наречено имя [месту] тому: плач Египтян, что при Иордане.
Gen 50:12  И сделали сыновья [Иакова] с ним, как он заповедал им;
Gen 50:13  и отнесли его сыновья его в землю Ханаанскую и похоронили его в пещере на поле Махпела, которую купил Авраам с полем в собственность для погребения у Ефрона Хеттеянина, пред Мамре.
Gen 50:14  И возвратился Иосиф в Египет, сам и братья его и все ходившие с ним хоронить отца его, после погребения им отца своего.
Gen 50:15  И увидели братья Иосифовы, что умер отец их, и сказали: что, если Иосиф возненавидит нас и захочет отмстить нам за все зло, которое мы ему сделали?
Gen 50:16  И послали они сказать Иосифу: отец твой пред смертью своею завещал, говоря:
Gen 50:17  так скажите Иосифу: прости братьям твоим вину и грех их, так как они сделали тебе зло. И ныне прости вины рабов Бога отца твоего. Иосиф плакал, когда ему говорили это.
Gen 50:18  Пришли и сами братья его, и пали пред лицем его, и сказали: вот, мы рабы тебе.
Gen 50:19  И сказал Иосиф: не бойтесь, ибо я боюсь Бога;
Gen 50:20  вот, вы умышляли против меня зло; но Бог обратил это в добро, чтобы сделать то, что теперь есть: сохранить жизнь великому числу людей;
Gen 50:21  итак не бойтесь: я буду питать вас и детей ваших. И успокоил их и говорил по сердцу их.
Gen 50:22  И жил Иосиф в Египте сам и дом отца его; жил же Иосиф всего сто десять лет.
Gen 50:23  И видел Иосиф детей у Ефрема до третьего рода, также и сыновья Махира, сына Манассиина, родились на колени Иосифа.
Gen 50:24  И сказал Иосиф братьям своим: я умираю, но Бог посетит вас и выведет вас из земли сей в землю, о которой клялся Аврааму, Исааку и Иакову.
Gen 50:25  И заклял Иосиф сынов Израилевых, говоря: Бог посетит вас, и вынесите кости мои отсюда.
Gen 50:26  И умер Иосиф ста десяти лет. И набальзамировали его и положили в ковчег в Египте.


\end{document}