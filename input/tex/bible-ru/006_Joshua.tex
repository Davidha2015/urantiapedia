\begin{document}

\title{Joshua}

Jos 1:1  По смерти Моисея, раба Господня, Господь сказал Иисусу, сыну Навину, служителю Моисееву:
Jos 1:2  Моисей, раб Мой, умер; итак встань, перейди через Иордан сей, ты и весь народ сей, в землю, которую Я даю им, сынам Израилевым.
Jos 1:3  Всякое место, на которое ступят стопы ног ваших, Я даю вам, как Я сказал Моисею:
Jos 1:4  от пустыни и Ливана сего до реки великой, реки Евфрата, всю землю Хеттеев; и до великого моря к западу солнца будут пределы ваши.
Jos 1:5  Никто не устоит пред тобою во все дни жизни твоей; и как Я был с Моисеем, так буду и с тобою: не отступлю от тебя и не оставлю тебя.
Jos 1:6  Будь тверд и мужествен; ибо ты народу сему передашь во владение землю, которую Я клялся отцам их дать им;
Jos 1:7  только будь тверд и очень мужествен, и тщательно храни и исполняй весь закон, который завещал тебе Моисей, раб Мой; не уклоняйся от него ни направо ни налево, дабы поступать благоразумно во всех предприятиях твоих.
Jos 1:8  Да не отходит сия книга закона от уст твоих; но поучайся в ней день и ночь, дабы в точности исполнять все, что в ней написано: тогда ты будешь успешен в путях твоих и будешь поступать благоразумно.
Jos 1:9  Вот Я повелеваю тебе: будь тверд и мужествен, не страшись и не ужасайся; ибо с тобою Господь Бог твой везде, куда ни пойдешь.
Jos 1:10  И дал Иисус повеление надзирателям народа и сказал:
Jos 1:11  пройдите по стану и дайте повеление народу и скажите: заготовляйте себе пищу для пути, потому что, спустя три дня, вы пойдете за Иордан сей, дабы придти взять землю, которую Господь Бог [отцов] ваших дает вам в наследие.
Jos 1:12  А колену Рувимову, Гадову и половине колена Манассиина Иисус сказал:
Jos 1:13  вспомните, что заповедал вам Моисей, раб Господень, говоря: Господь Бог ваш успокоил вас и дал вам землю сию;
Jos 1:14  жены ваши, дети ваши и скот ваш пусть останутся в земле, которую дал вам Моисей за Иорданом; а вы все, могущие сражаться, вооружившись идите пред братьями вашими и помогайте им,
Jos 1:15  доколе Господь не успокоит братьев ваших, как и вас; доколе и они не получат в наследие землю, которую Господь Бог ваш дает им; тогда возвратитесь в наследие ваше и владейте землею, которую Моисей, раб Господень, дал вам за Иорданом к востоку солнца.
Jos 1:16  Они в ответ Иисусу сказали: все, что ни повелишь нам, сделаем, и куда ни пошлешь нас, пойдем;
Jos 1:17  как слушали мы Моисея, так будем слушать и тебя: только Господь, Бог твой, да будет с тобою, как Он был с Моисеем;
Jos 1:18  всякий, кто воспротивится повелению твоему и не послушает слов твоих во всем, что ты ни повелишь ему, будет предан смерти. Только будь тверд и мужествен!
Jos 2:1  И послал Иисус, сын Навин, из Ситтима двух соглядатаев тайно и сказал: пойдите, осмотрите землю и Иерихон. [Два юноши] пошли и пришли в дом блудницы, которой имя Раав, и остались ночевать там.
Jos 2:2  И сказано было царю Иерихонскому: вот, какие-то люди из сынов Израилевых пришли сюда в эту ночь, чтобы высмотреть землю.
Jos 2:3  Царь Иерихонский послал сказать Рааве: выдай людей, пришедших к тебе, которые вошли в твой дом, ибо они пришли высмотреть всю землю.
Jos 2:4  Но женщина взяла двух человек тех и скрыла их и сказала: точно приходили ко мне люди, но я не знала, откуда они;
Jos 2:5  когда же в сумерки надлежало затворять ворота, тогда они ушли; не знаю, куда они пошли; гонитесь скорее за ними, вы догоните их.
Jos 2:6  А сама отвела их на кровлю и скрыла их в снопах льна, разложенных у нее на кровле.
Jos 2:7  [Посланные] гнались за ними по дороге к Иордану до самой переправы; ворота же [тотчас] затворили, после того как вышли погнавшиеся за ними.
Jos 2:8  Прежде нежели они легли спать, она взошла к ним на кровлю
Jos 2:9  и сказала им: я знаю, что Господь отдал землю сию вам, ибо вы навели на нас ужас, и все жители земли сей пришли от вас в робость;
Jos 2:10  ибо мы слышали, как Господь иссушил пред вами воду Чермного моря, когда вы шли из Египта, и как поступили вы с двумя царями Аморрейскими за Иорданом, с Сигоном и Огом, которых вы истребили;
Jos 2:11  когда мы услышали об этом, ослабело сердце наше, и ни в ком [из нас] не стало духа против вас; ибо Господь Бог ваш есть Бог на небе вверху и на земле внизу;
Jos 2:12  итак поклянитесь мне Господом что, как я сделала вам милость, так и вы сделаете милость дому отца моего, и дайте мне верный знак,
Jos 2:13  что вы сохраните в живых отца моего и матерь мою, и братьев моих и сестер моих, и всех, кто есть у них, и избавите души наши от смерти.
Jos 2:14  Эти люди сказали ей: душа наша вместо вас [да будет предана] смерти, если вы не откроете сего дела нашего; когда же Господь предаст нам землю, мы окажем тебе милость и истину.
Jos 2:15  И спустила она их по веревке чрез окно, ибо дом ее был в городской стене, и она жила в стене;
Jos 2:16  и сказала им: идите на гору, чтобы не встретили вас преследующие, и скрывайтесь там три дня, доколе не возвратятся погнавшиеся [за вами]; а после пойдете в путь ваш.
Jos 2:17  И сказали ей те люди: мы свободны будем от твоей клятвы, которою ты нас закляла, [если не сделаешь так]:
Jos 2:18  вот, когда мы придем в эту землю, ты привяжи червленую веревку к окну, чрез которое ты нас спустила, а отца твоего и матерь твою и братьев твоих, все семейство отца твоего собери к себе в дом твой;
Jos 2:19  и если кто-нибудь выйдет из дверей твоего дома вон, того кровь на голове его, а мы свободны [будем от сей клятвы твоей]; а кто будет с тобою в [твоем] доме, того кровь на голове нашей, если чья рука коснется его;
Jos 2:20  если же ты откроешь сие наше дело, то мы также свободны будем от клятвы твоей, которою ты нас закляла.
Jos 2:21  Она сказала: да будет по словам вашим! И отпустила их, и они пошли, а она привязала к окну червленую веревку.
Jos 2:22  Они пошли и пришли на гору, и пробыли там три дня, доколе не возвратились гнавшиеся [за ними]. Гнавшиеся искали их по всей дороге и не нашли.
Jos 2:23  Таким образом два сии человека пошли назад, сошли с горы, перешли [Иордан] и пришли к Иисусу, сыну Навину, и пересказали ему все, что с ними случилось.
Jos 2:24  И сказали Иисусу: Господь предал всю землю сию в руки наши, и все жители земли в страхе от нас.
Jos 3:1  И встал Иисус рано поутру, и двинулись они от Ситтима и пришли к Иордану, он и все сыны Израилевы, и ночевали там, еще не переходя [его].
Jos 3:2  Чрез три дня пошли надзиратели по стану
Jos 3:3  и дали народу повеление, говоря: когда увидите ковчег завета Господа Бога вашего и священников [и] левитов, несущих его, то и вы двиньтесь с места своего и идите за ним;
Jos 3:4  впрочем расстояние между вами и им должно быть до двух тысяч локтей мерою; не подходите к нему близко, чтобы знать вам путь, по которому идти; ибо вы не ходили сим путем ни вчера, ни третьего дня.
Jos 3:5  И сказал Иисус народу: освятитесь, ибо завтра сотворит Господь среди вас чудеса.
Jos 3:6  Священникам же сказал Иисус: возьмите ковчег завета, и идите пред народом. [Священники] взяли ковчег завета, и пошли пред народом.
Jos 3:7  Тогда Господь сказал Иисусу: в сей день Я начну прославлять тебя пред очами всех [сынов] Израиля, дабы они узнали, что как Я был с Моисеем, так буду и с тобою;
Jos 3:8  а ты дай повеление священникам, несущим ковчег завета, и скажи: как только войдете в воды Иордана, остановитесь в Иордане.
Jos 3:9  Иисус сказал сынам Израилевым: подойдите сюда и выслушайте слова Господа, Бога вашего.
Jos 3:10  И сказал Иисус: из сего узнаете, что среди вас есть Бог живый, Который прогонит от вас Хананеев и Хеттеев, и Евеев, и Ферезеев, и Гергесеев, и Аморреев, и Иевусеев:
Jos 3:11  вот, ковчег завета Господа всей земли пойдет пред вами чрез Иордан;
Jos 3:12  и возьмите себе двенадцать человек из колен Израилевых, по одному человеку из колена;
Jos 3:13  и как только стопы ног священников, несущих ковчег Господа, Владыки всей земли, ступят в воду Иордана, вода Иорданская иссякнет, текущая же сверху вода остановится стеною.
Jos 3:14  Итак, когда народ двинулся от своих шатров, чтобы переходить Иордан, и священники понесли ковчег завета пред народом,
Jos 3:15  то, лишь только несущие ковчег вошли в Иордан, и ноги священников, несших ковчег, погрузились в воду Иордана--Иордан же выступает из всех берегов своих во все дни жатвы пшеницы, --
Jos 3:16  вода, текущая сверху, остановилась и стала стеною на весьма большое расстояние, до города Адама, который подле Цартана; а текущая в море равнины, в море Соленое, ушла и иссякла.
Jos 3:17  И народ переходил против Иерихона; священники же, несшие ковчег завета Господня, стояли на суше среди Иордана твердою ногою. Все [сыны] Израилевы переходили по суше, доколе весь народ не перешел чрез Иордан.
Jos 4:1  Когда весь народ перешел чрез Иордан, Господь сказал Иисусу:
Jos 4:2  возьмите себе из народа двенадцать человек, по одномучеловеку из колена,
Jos 4:3  и дайте им повеление и скажите: возьмите себе отсюда, изсредины Иордана, где стояли ноги священников неподвижно, двенадцать камней, и перенесите их с собою, и положите их наночлеге, где будете ночевать в эту ночь.
Jos 4:4  Иисус призвал двенадцать человек, которых назначил из Сыновизраилевых, по одному человеку из колена,
Jos 4:5  и сказал им Иисус: пойдите пред ковчегом Господа Бога вашегов средину Иордана и [возьмите оттуда] и положите на плечо своекаждый по одному камню, по числу колен сынов Израилевых,
Jos 4:6  чтобы они были у вас знамением; когдаспросят вас в последующее время сыны ваши и скажут: `к чему у васэти камни?',
Jos 4:7  вы скажете им: `[в память того], что вода Иорданаразделилась пред ковчегом завета Господа; когда онпереходил чрез Иордан, тогда вода Иордана разделилась'; такимобразом камни сии будут для сынов Израилевых памятникомна век.
Jos 4:8  И сделали сыны Израилевы так, как приказал Иисус: взялидвенадцать камней из Иордана, как говорил Господь Иисусу, почислу колен сынов Израилевых, и перенесли их с собою на ночлег, иположили их там.
Jos 4:9  И [другие] двенадцать камней поставил Иисус среди Иордана наместе, где стояли ноги священников, несших ковчег завета. Они там и до сего дня.
Jos 4:10  Священники, несшие ковчег, стояли Средииордана, доколе не окончено было все, что Господьповелел Иисусу сказать народу г так, как завещал Моисей Иисусу; анарод между тем поспешно переходил.
Jos 4:11  Когда весь народ перешел Иордан, тогда перешел и ковчег [завета] Господня, и священники пред народом;
Jos 4:12  и сыны Рувима и сыны Гада и половина колена Манассиинаперешли вооруженные впереди сынов Израилевых, как говорил Иммоисей.
Jos 4:13  Около сорока тысяч вооруженных на брань перешло Предгосподом на равнины Иерихонские, чтобы сразиться.
Jos 4:14  В тот день прославил Господь Иисуса пред очами Всегоизраиля и стали бояться его, как боялись Моисея, во все дни жизниего.
Jos 4:15  И сказал Господь Иисусу, говоря:
Jos 4:16  прикажи священникам, несущим ковчег откровения, выйти Изиордана.
Jos 4:17  Иисус приказал священникам и сказал: выйдите из Иордана.
Jos 4:18  И когда священники, несшие ковчег завета Господня, вышли Изиордана, то, лишь только стопы ног их ступили на сушу, Водаиордана устремилась по своему месту и пошла, как вчера и третьегодня, выше всех берегов своих.
Jos 4:19  И вышел народ из Иордана в десятый день первого месяца ипоставил стан в Галгале, на восточной стороне Иерихона.
Jos 4:20  И двенадцать камней, которые взяли они из Иордана, Иисуспоставил в Галгале
Jos 4:21  и сказал сынам Израилевым: когда спросят в последующеевремя сыны ваши отцов своих: `что значат эти камни?',
Jos 4:22  скажите сынам вашим: `Израиль перешел чрез Иордан сей посуше',
Jos 4:23  ибо Господь Бог ваш иссушил воды Иордана для вас, доколе вы не перешли его, так же, как Господь Бог ваш сделал с Чермным морем, которое иссушил пред нами, доколе мыне перешли его,
Jos 4:24  дабы все народы земли познали, что рука Господня сильна, идабы вы боялись Господа Бога вашего во все дни.
Jos 5:1  Когда все цари Аморрейские, которые жили по эту сторону Иордана к морю, и все цари Ханаанские, которые при море, услышали, что Господь иссушил воды Иордана пред сынами Израилевыми, доколе переходили они, тогда ослабело сердце их, и не стало уже в них духа против сынов Израилевых.
Jos 5:2  В то время сказал Господь Иисусу: сделай себе острые ножи и обрежь сынов Израилевых во второй раз.
Jos 5:3  И сделал себе Иисус острые ножи и обрезал сынов Израилевых на [месте, названном]: Холм обрезания.
Jos 5:4  Вот причина, почему обрезал Иисус [сынов Израилевых], весь народ, вышедший из Египта, мужеского пола, все способные к войне умерли в пустыне на пути, по исшествии из Египта;
Jos 5:5  весь же вышедший народ был обрезан, но весь народ, родившийся в пустыне на пути, после того как вышел из Египта, не был обрезан;
Jos 5:6  ибо сыны Израилевы сорок года ходили в пустыне, доколе не перемер весь народ, способный к войне, вышедший из Египта, которые не слушали гласа Господня, и которым Господь клялся, что они не увидят земли, которую Господь с клятвою обещал отцам их, дать нам землю, где течет молоко и мед,
Jos 5:7  а вместо их воздвиг сынов их. Сих обрезал Иисус, ибо они были необрезаны; потому что их, на пути, не обрезывали.
Jos 5:8  Когда весь народ был обрезан, оставался он на своем месте в стане, доколе не выздоровел.
Jos 5:9  И сказал Господь Иисусу: ныне Я снял с вас посрамление Египетское. Почему и называется то место `Галгал', даже до сего дня.
Jos 5:10  И стояли сыны Израилевы станом в Галгале и совершили Пасху в четырнадцатый день месяца вечером на равнинах Иерихонских;
Jos 5:11  и на другой день Пасхи стали есть из произведений земли сей, опресноки и сушеные зерна в самый тот день;
Jos 5:12  а манна перестала падать на другой день после того, как они стали есть произведения земли, и не было более манны у сынов Израилевых, но они ели в тот год произведения земли Ханаанской.
Jos 5:13  Иисус, находясь близ Иерихона, взглянул, и видит, и вот стоит пред ним человек, и в руке его обнаженный меч. Иисус подошел к нему и сказал ему: наш ли ты, или из неприятелей наших?
Jos 5:14  Он сказал: нет; я вождь воинства Господня, теперь пришел [сюда]. Иисус пал лицем своим на землю, и поклонился и сказал ему: что господин мой скажет рабу своему?
Jos 5:15  Вождь воинства Господня сказал Иисусу: сними обувь твою с ног твоих, ибо место, на котором ты стоишь, свято. Иисус так и сделал.
Jos 5:16  Иерихон заперся и был заперт от [страха] сынов Израилевых: никто не выходил [из него] и никто не входил.
Jos 6:1  Тогда сказал Господь Иисусу: вот, Я предаю в руки твои Иерихон и царя его, [и находящихся в нем] людей сильных;
Jos 6:2  пойдите вокруг города все способные к войне и обходите город однажды [в день]; и это делай шесть дней;
Jos 6:3  и семь священников пусть несут семь труб юбилейных пред ковчегом; а в седьмой день обойдите вокруг города семь раз, и священники пусть трубят трубами;
Jos 6:4  когда затрубит юбилейный рог, когда услышите звук трубы, тогда весь народ пусть воскликнет громким голосом, и стена города обрушится до своего основания, и [весь] народ пойдет [в город, устремившись] каждый с своей стороны.
Jos 6:5  И призвал Иисус, сын Навин, священников [Израилевых] и сказал им: несите ковчег завета; а семь священников пусть несут семь труб юбилейных пред ковчегом Господним.
Jos 6:6  И сказал народу: пойдите и обойдите вокруг города; вооруженные же пусть идут пред ковчегом Господним.
Jos 6:7  Как скоро Иисус сказал народу, семь священников, несших семь труб юбилейных пред Господом, пошли и затрубили трубами, и ковчег завета Господня шел за ними;
Jos 6:8  вооруженные же шли впереди священников, которые трубили трубами; а идущие позади следовали за ковчегом, во время шествия трубя трубами.
Jos 6:9  Народу же Иисус дал повеление и сказал: не восклицайте и не давайте слышать голоса вашего, и чтобы слово не выходило из уст ваших до [того] дня, доколе я не скажу вам: `воскликните!' и тогда воскликните.
Jos 6:10  Таким образом ковчег [завета] Господня пошел вокруг города и обошел однажды; и пришли в стан и ночевали в стане.
Jos 6:11  [На другой день] Иисус встал рано поутру, и священники понесли ковчег [завета] Господня;
Jos 6:12  и семь священников, несших семь труб юбилейных пред ковчегом Господним, шли и трубили трубами; вооруженные же шли впереди их, а идущие позади следовали за ковчегом [завета] Господня и идучи трубили трубами.
Jos 6:13  Таким образом и на другой день обошли вокруг города однажды и возвратились в стан. И делали это шесть дней.
Jos 6:14  В седьмой день встали рано, при появлении зари, и обошли таким же образом вокруг города семь раз; только в этот день обошли вокруг города семь раз.
Jos 6:15  Когда в седьмой раз священники трубили трубами, Иисус сказал народу: воскликните, ибо Господь предал вам город!
Jos 6:16  город будет под заклятием, и все, что в нем, Господу; только Раав блудница пусть останется в живых, она и всякий, кто у нее в доме; потому что она укрыла посланных, которых мы посылали;
Jos 6:17  но вы берегитесь заклятого, чтоб и самим не подвергнуться заклятию, если возьмете что-нибудь из заклятого, и чтобы на стан [сынов] Израилевых не навести заклятия и не сделать ему беды;
Jos 6:18  и все серебро и золото, и сосуды медные и железные да будут святынею Господу и войдут в сокровищницу Господню.
Jos 6:19  Народ воскликнул, и затрубили трубами. Как скоро услышал народ голос трубы, воскликнул народ громким голосом, и обрушилась стена [города] до своего основания, и народ пошел в город, каждый с своей стороны, и взяли город.
Jos 6:20  И предали заклятию все, что в городе, и мужей и жен, и молодых и старых, и волов, и овец, и ослов, [все истребили] мечом.
Jos 6:21  А двум юношам, высматривавшим землю, Иисус сказал: пойдите в дом оной блудницы и выведите оттуда ее и всех, которые у нее, так как вы поклялись ей.
Jos 6:22  И пошли юноши, высматривавшие [город, в дом женщины] и вывели Раав и отца ее и мать ее, и братьев ее, и всех, которые у нее [были], и всех родственников ее вывели, и поставили их вне стана Израильского.
Jos 6:23  А город и все, что в нем, сожгли огнем; только серебро и золото и сосуды медные и железные отдали, в сокровищницу дома Господня.
Jos 6:24  Раав же блудницу и дом отца ее и всех, которые у нее [были], Иисус оставил в живых, и она живет среди Израиля до сего дня, потому что она укрыла посланных, которых посылал Иисус для высмотрения Иерихона.
Jos 6:25  В то время Иисус поклялся и сказал: проклят пред Господом тот, кто восставит и построит город сей Иерихон; на первенце своем он положит основание его и на младшем своем поставит врата его.
Jos 6:26  И Господь был с Иисусом, и слава его носилась по всей земле.
Jos 7:1  Но сыны Израилевы сделали преступление [и взяли] из заклятого. Ахан, сын Хармия, сына Завдия, сына Зары, из колена Иудина, взял из заклятого, и гнев Господень возгорелся на сынов Израиля.
Jos 7:2  Иисус из Иерихона послал людей в Гай, что близ Беф-Авена, с восточной стороны Вефиля, и сказал им: пойдите, осмотрите землю. Они пошли и осмотрели Гай.
Jos 7:3  И возвратившись к Иисусу, сказали ему: не весь народ пусть идет, а пусть пойдет около двух тысяч или около трех тысяч человек, и поразят Гай; всего народа не утруждай туда, ибо их мало [там].
Jos 7:4  Итак пошло туда из народа около трех тысяч человек, но они обратились в бегство от жителей Гайских;
Jos 7:5  жители Гайские убили из них до тридцати шести человек, и преследовали их от ворот до Севарим и разбили их на спуске с горы; отчего сердце народа растаяло и стало, как вода.
Jos 7:6  Иисус разодрал одежды свои и пал лицем своим на землю пред ковчегом Господним [и лежал] до самого вечера, он и старейшины Израилевы, и посыпали прахом головы свои.
Jos 7:7  И сказал Иисус: о, Господи Владыка! для чего Ты перевел народ сей чрез Иордан, дабы предать нас в руки Аморреев и погубить нас? о, если бы мы остались и жили за Иорданом!
Jos 7:8  О, Господи! что сказать мне после того, как Израиль обратил тыл врагам своим?
Jos 7:9  Хананеи и все жители земли услышат и окружат нас и истребят имя наше с земли. И что сделаешь [тогда] имени Твоему великому?
Jos 7:10  Господь сказал Иисусу: встань, для чего ты пал на лице твое?
Jos 7:11  Израиль согрешил, и преступили они завет Мой, который Я завещал им; и взяли из заклятого, и украли, и утаили, и положили между своими вещами;
Jos 7:12  за то сыны Израилевы не могли устоять пред врагами своими и обратили тыл врагам своим, ибо они подпали заклятию; не буду более с вами, если не истребите из среды вашей заклятого.
Jos 7:13  Встань, освяти народ и скажи: освятитесь к утру, ибо так говорит Господь Бог Израилев: `заклятое среди тебя, Израиль; посему ты не можешь устоять пред врагами твоим, доколе не отдалишь от себя заклятого';
Jos 7:14  завтра подходите [все] по коленам вашим; колено же, которое укажет Господь, пусть подходит по племенам; племя, которое укажет Господь, пусть подходит по семействам; семейство, которое укажет Господь, пусть подходит по одному человеку;
Jos 7:15  и обличенного в [похищении] заклятого пусть сожгут огнем, его и все, что у него, за то, что он преступил завет Господень и сделал беззаконие среди Израиля.
Jos 7:16  Иисус, встав рано поутру, велел подходить Израилю по коленам его, и указано колено Иудино;
Jos 7:17  потом велел подходить племенам Иуды, и указано племя Зары; велел подходить племени Зарину по семействам, и указано [семейство] Завдиево;
Jos 7:18  велел подходить семейству его по одному человеку, и указан Ахан, сын Хармия, сына Завдия, сына Зары, из колена Иудина.
Jos 7:19  Тогда Иисус сказал Ахану: сын мой! воздай славу Господу, Богу Израилеву и сделай пред Ним исповедание и объяви мне, что ты сделал; не скрой от меня.
Jos 7:20  В ответ Иисусу Ахан сказал: точно, я согрешил пред Господом Богом Израилевым и сделал то и то:
Jos 7:21  между добычею увидел я одну прекрасную Сеннаарскую одежду и двести сиклей серебра и слиток золота весом в пятьдесят сиклей; это мне полюбилось и я взял это; и вот, оно спрятано в земле среди шатра моего, и серебро под ним.
Jos 7:22  Иисус послал людей, и они побежали в шатер; и вот, [все] это спрятано было в шатре его, и серебро под ним.
Jos 7:23  Они взяли это из шатра и принесли к Иисусу и ко всем сынам Израилевым и положили пред Господом.
Jos 7:24  Иисус и все Израильтяне с ним взяли Ахана, сына Зарина, и серебро, и одежду, и слиток золота, и сыновей его и дочерей его, и волов его и ослов его, и овец его и шатер его, и все, что у него [было], и вывели их на долину Ахор.
Jos 7:25  И сказал Иисус: за то, что ты навел на нас беду, Господь на тебя наводит беду в день сей. И побили его все Израильтяне камнями, и сожгли их огнем, и наметали на них камни.
Jos 7:26  И набросали на него большую груду камней, [которая уцелела] и до сего дня. После сего утихла ярость гнева Господня. Посему то место называется долиною Ахор даже до сего дня.
Jos 8:1  Господь сказал Иисусу: не бойся и не ужасайся; возьми с собою весь народ, способный к войне, и встав пойди к Гаю; вот, Я предаю в руки твои царя Гайского и народ его, город его и землю его;
Jos 8:2  сделай с Гаем и царем его то же, что сделал ты с Иерихоном и царем его, только добычу его и скот его разделите себе; сделай засаду позади города.
Jos 8:3  Иисус и весь народ, способный к войне, встал, чтобы идти к Гаю, и выбрал Иисус тридцать тысяч человек храбрых и послал их ночью,
Jos 8:4  и дал им приказание и сказал: смотрите, вы будете составлять засаду у города позади города; не отходите далеко от города и будьте все готовы;
Jos 8:5  а я и весь народ, который со мною, подойдем к городу; и когда [жители Гая] выступят против нас, как и прежде, то мы побежим от них;
Jos 8:6  они пойдут за нами, так что мы отвлечем их от города; ибо они скажут: `бегут от нас, как и прежде'; когда мы побежим от них,
Jos 8:7  тогда вы встаньте из засады и завладейте городом, и Господь Бог ваш предаст его в руки ваши;
Jos 8:8  когда возьмете город, зажгите город огнем, по слову Господню сделайте; смотрите, я повелеваю вам.
Jos 8:9  Таким образом послал их Иисус, и они пошли в засаду и засели между Вефилем и между Гаем, с западной стороны Гая; а Иисус в ту ночь ночевал среди народа.
Jos 8:10  Встав рано поутру, Иисус осмотрел народ, и пошел он и старейшины Израилевы впереди народа к Гаю;
Jos 8:11  и весь народ, способный к войне, который был с ним, пошел, приблизился и подошел к городу,
Jos 8:12  и поставил стан с северной стороны Гая, а между ним и Гаем была долина. Потом взял он около пяти тысяч человек и посадил их в засаде между Вефилем и Гаем, с западной стороны города.
Jos 8:13  И народ расположил весь стан, который был с северной стороны города, так, что задняя часть была с западной стороны города. И пришел Иисус в ту ночь на средину долины.
Jos 8:14  Когда увидел это царь Гайский, тотчас с жителями города, встав рано, выступил против Израиля на сражение, он и весь народ его, на назначенное место пред равниною; а он не знал, что для него есть засада позади города.
Jos 8:15  Иисус и весь Израиль, будто пораженные ими, побежали к пустыне;
Jos 8:16  а они кликнули весь народ, который был в городе, чтобы преследовать их, и, преследуя Иисуса, отдалились от города;
Jos 8:17  в Гае и Вефиле не осталось ни одного человека, который не погнался бы за Израилем; и город свой они оставили отворенным, преследуя Израиля.
Jos 8:18  Тогда Господь сказал Иисусу: простри копье, которое в руке твоей, к Гаю, ибо Я предам его в руки твои. Иисус простер копье, которое было в его руке, к городу.
Jos 8:19  Сидевшие в засаде тотчас встали с места своего и побежали, как скоро он простер руку свою, вошли в город и взяли его и тотчас зажгли город огнем.
Jos 8:20  Жители Гая, оглянувшись назад, увидели, что дым от города восходил к небу. И не было для них места, куда бы бежать--ни туда, ни сюда; ибо народ, бежавший к пустыне, обратился на преследователей.
Jos 8:21  Иисус и весь Израиль, увидев, что сидевшие в засаде взяли город, и дым от города восходил [к небу], возвратились и стали поражать жителей Гая;
Jos 8:22  а те из города вышли навстречу им, так что они находились в средине между Израильтянами, [из которых] одни были с той стороны, а другие с другой; так поражали их, что не оставили ни одного из них, уцелевшего или убежавшего;
Jos 8:23  а царя Гайского взяли живого и привели его к Иисусу.
Jos 8:24  Когда Израильтяне перебили всех жителей Гая на поле, в пустыне, куда они преследовали их, и когда все они до последнего пали от острия меча, тогда все Израильтяне обратились к Гаю и поразили его острием меча.
Jos 8:25  Падших в тот день мужей и жен, всех жителей Гая, было двенадцать тысяч.
Jos 8:26  Иисус не опускал руки своей, которую простер с копьем, доколе не предал заклятию всех жителей Гая;
Jos 8:27  только скот и добычу города сего [сыны] Израиля разделили между собою, по слову Господа, которое [Господь] сказал Иисусу.
Jos 8:28  И сожег Иисус Гай и обратил его в вечные развалины, в пустыню, до сего дня;
Jos 8:29  а царя Гайского повесил на дереве, до вечера; по захождении же солнца приказал Иисус, и сняли труп его с дерева, и бросили его у ворот городских, и набросали над ним большую груду камней, [которая уцелела] даже до сего дня.
Jos 8:30  Тогда Иисус устроил жертвенник Господу Богу Израилеву на горе Гевал,
Jos 8:31  как заповедал Моисей, раб Господень, сынам Израилевым, о чем написано в книге закона Моисеева, --жертвенник из камней цельных, на которые не поднимали железа; и принесли на нем всесожжение Господу и совершили жертвы мирные.
Jos 8:32  И написал [Иисус] там на камнях список с закона Моисеева, который он написал пред сынами Израилевыми.
Jos 8:33  Весь Израиль, старейшины его и надзиратели [его] и судьи его, стали с той и другой стороны ковчега против священников [и] левитов, носящих ковчег завета Господня, как пришельцы, так и природные жители, одна половина их у горы Гаризим, а другая половина у горы Гевал, как прежде повелел Моисей, раб Господень, благословлять народ Израилев.
Jos 8:34  И потом прочитал [Иисус] все слова закона, благословение и проклятие, как написано в книге закона;
Jos 8:35  из всего, что Моисей заповедал [Иисусу], не было [ни одного] слова, которого Иисус не прочитал бы пред всем собранием Израиля, и женами, и детьми, и пришельцами, находившимися среди них.
Jos 9:1  Услышав сие, все цари, которые за Иорданом, на горе и на равнине и по всему берегу великого моря, близ Ливана, Хеттеи, Аморреи, Хананеи, Ферезеи, Евеи и Иевусеи,
Jos 9:2  собрались вместе, дабы единодушно сразиться с Иисусом и Израилем.
Jos 9:3  Но жители Гаваона, услышав, что Иисус сделал с Иерихоном и Гаем,
Jos 9:4  употребили хитрость: пошли, запаслись хлебом на дорогу и положили ветхие мешки на ослов своих и ветхие, изорванные и заплатанные мехи вина;
Jos 9:5  и обувь на ногах их была ветхая с заплатами, и одежда на них ветхая; и весь дорожный хлеб их был сухой и заплесневелый.
Jos 9:6  Они пришли к Иисусу в стан [Израильский] в Галгал и сказали ему и всем Израильтянам: из весьма дальней земли пришли мы; итак заключите с нами союз.
Jos 9:7  Израильтяне же сказали Евеям: может быть, вы живете близ нас? как нам заключить с вами союз?
Jos 9:8  Они сказали Иисусу: мы рабы твои. Иисус же сказал им: кто вы и откуда пришли?
Jos 9:9  Они сказали ему: из весьма дальней земли пришли рабы твои во имя Господа Бога твоего; ибо мы слышали славу Его и все, что сделал Он в Египте,
Jos 9:10  и все, что Он сделал двум царям Аморрейским, которые по ту сторону Иордана, Сигону, царю Есевонскому, и Огу, царю Васанскому, который [жил] в Астарофе.
Jos 9:11  [Слыша сие], старейшины наши и все жители нашей земли сказали нам: возьмите в руки ваши хлеба на дорогу и пойдите навстречу им и скажите им: `мы рабы ваши; итак заключите с нами союз'.
Jos 9:12  Этот хлеб наш из домов наших мы взяли теплый в тот день, когда пошли к вам, а теперь вот, он сделался сухой и заплесневелый;
Jos 9:13  и эти мехи с вином, которые мы налили новые, вот, изорвались; и эта одежда наша и обувь наша обветшала от весьма дальней дороги.
Jos 9:14  Израильтяне взяли их хлеба, а Господа не вопросили.
Jos 9:15  И заключил Иисус с ними мир и постановил с ними условие в том, что он сохранит им жизнь; и поклялись им начальники общества.
Jos 9:16  А чрез три дня, как заключили они с ними союз, услышали, что они соседи их и живут близ них;
Jos 9:17  ибо сыны Израилевы, отправившись в путь, пришли в города их на третий день; города же их [были]: Гаваон, Кефира, Беероф и Кириаф-Иарим.
Jos 9:18  сыны Израилевы не побили их, потому что начальники общества клялись им Господом Богом Израилевым. За это все общество [Израилево] возроптало на начальников.
Jos 9:19  Все начальники сказали всему обществу: мы клялись им Господом Богом Израилевым и потому не можем коснуться их;
Jos 9:20  а вот что сделаем с ними: оставим их в живых, чтобы не постиг нас гнев за клятву, которою мы клялись им.
Jos 9:21  И сказали им начальники: пусть они живут, но будут рубить дрова и черпать воду для всего общества. [И сделало все общество] так, как сказали им начальники.
Jos 9:22  Иисус призвал их и сказал: для чего вы обманули нас, сказав: `мы весьма далеко от вас', тогда как вы живете близ нас?
Jos 9:23  за это прокляты вы! без конца вы будете рабами, будете рубить дрова и черпать воду для дома Бога моего!
Jos 9:24  Они в ответ Иисусу сказали: дошло до сведения рабов твоих, что Господь Бог твой повелел Моисею, рабу Своему, дать вам всю землю и погубить всех жителей сей земли пред лицем вашим; посему мы весьма боялись, чтобы вы не лишили нас жизни, и сделали это дело;
Jos 9:25  теперь вот мы в руке твоей: как лучше и справедливее тебе покажется поступить с нами, так и поступи.
Jos 9:26  И поступил с ними так: избавил их от руки сынов Израилевых, и они не умертвили их;
Jos 9:27  и определил в тот день Иисус, чтобы они рубили дрова и черпали воду для общества и для жертвенника Господня; --посему жители Гаваона сделались дровосеками и водоносами для жертвенника Божия, --даже до сего дня, на месте, какое ни избрал бы [Господь].
Jos 10:1  Когда Адониседек, царь Иерусалимский, услышал, что Иисус взял Гай и предал его заклятию, и что так же поступил с Гаем и царем его, как поступил с Иерихоном и царем его, и что жители Гаваона заключили мир с Израилем и остались среди их,
Jos 10:2  тогда он весьма испугался, потому что Гаваон [был] город большой, как один из царских городов, и больше Гая, и все жители его люди храбрые.
Jos 10:3  Посему Адониседек, царь Иерусалимский, послал к Гогаму, царю Хевронскому, и к Фираму, царю Иармуфскому, и к Яфию, царю Лахисскому, и к Девиру, царю Еглонскому, чтобы сказать:
Jos 10:4  придите ко мне и помогите мне поразить Гаваон за то, что он заключил мир с Иисусом и сынами Израилевыми.
Jos 10:5  Они собрались, и пошли пять царей Аморрейских: царь Иерусалимский, царь Хевронский, царь Иармуфский, царь Лахисский, царь Еглонский, они и все ополчение их, и расположились станом подле Гаваона, чтобы воевать против него.
Jos 10:6  Жители Гаваона послали к Иисусу в стан [Израильский], в Галгал, сказать: не отними руки твоей от рабов твоих; приди к нам скорее, спаси нас и подай нам помощь; ибо собрались против нас все цари Аморрейские, живущие на горах.
Jos 10:7  Иисус пошел из Галгала сам, и с ним весь народ, способный к войне, и все мужи храбрые.
Jos 10:8  И сказал Господь Иисусу: не бойся их, ибо Я предал их в руки твои: никто из них не устоит пред лицем твоим.
Jos 10:9  И пришел на них Иисус внезапно, [потому что] всю ночь шел он из Галгала.
Jos 10:10  Господь привел их в смятение при виде Израильтян, и они поразили их в Гаваоне сильным поражением, и преследовали их по дороге к возвышенности Вефорона, и поражали их до Азека и до Македа.
Jos 10:11  Когда же они бежали от Израильтян по скату горы Вефоронской, Господь бросал на них с небес большие камни до самого Азека, и они умирали; больше было тех, которые умерли от камней града, нежели тех, которых умертвили сыны Израилевы мечом.
Jos 10:12  Иисус воззвал к Господу в тот день, в который предал Господь Аморрея в руки Израилю, когда побил их в Гаваоне, и они побиты были пред лицем сынов Израилевых, и сказал пред Израильтянами: стой, солнце, над Гаваоном, и луна, над долиною Аиалонскою!
Jos 10:13  И остановилось солнце, и луна стояла, доколе народ мстил врагам своим. Не это ли написано в книге Праведного: `стояло солнце среди неба и не спешило к западу почти целый день'?
Jos 10:14  И не было такого дня ни прежде ни после того, в который Господь [так] слушал бы гласа человеческого. Ибо Господь сражался за Израиля.
Jos 10:15  Потом возвратился Иисус и весь Израиль с ним в стан, в Галгал.
Jos 10:16  А те пять царей убежали и скрылись в пещере в Македе.
Jos 10:17  Когда донесено было Иисусу и сказано: `нашлись пять царей, они скрываются в пещере в Македе',
Jos 10:18  Иисус сказал: `привалите большие камни к отверстию пещеры и приставьте к ней людей стеречь их;
Jos 10:19  а вы не останавливайтесь, но преследуйте врагов ваших и истребляйте заднюю часть войска их и не давайте им уйти в города их, ибо Господь Бог ваш предал их в руки ваши'.
Jos 10:20  После того, как Иисус и сыны Израилевы совершенно поразили их весьма великим поражением, и оставшиеся из них убежали в города укрепленные,
Jos 10:21  весь народ возвратился в стан к Иисусу в Макед с миром, и никто на сынов Израилевых не пошевелил языком своим.
Jos 10:22  Тогда Иисус сказал: откройте отверстие пещеры и выведите ко мне из пещеры пятерых царей тех.
Jos 10:23  Так и сделали: вывели к нему из пещеры пятерых царей тех: царя Иерусалимского, царя Хевронского, царя Иармуфского, царя Лахисского и царя Еглонского.
Jos 10:24  Когда вывели царей сих к Иисусу, Иисус призвал всех Израильтян и сказал вождям воинов, ходившим с ним: подойдите, наступите ногами вашими на выи царей сих. Они подошли и наступили ногами своими на выи их.
Jos 10:25  Иисус сказал им: не бойтесь и не ужасайтесь, будьте тверды и мужественны; ибо так поступит Господь со всеми врагами вашими, с которыми будете воевать.
Jos 10:26  Потом поразил их Иисус и убил их и повесил их на пяти деревах; и висели они на деревах до вечера.
Jos 10:27  При захождении солнца приказал Иисус, и сняли их с дерев, и бросили их в пещеру, в которой они скрывались, и привалили большие камни к отверстию пещеры, [которые там] даже до сего дня.
Jos 10:28  В тот же день взял Иисус Макед, и поразил [его] мечом и царя его, и предал заклятию их и все дышащее, что находилось в нем: никого не оставил, кто бы уцелел; и поступил с царем Македским так же, как поступил с царем Иерихонским.
Jos 10:29  И пошел Иисус и все Израильтяне с ним из Македа к Ливне и воевал против Ливны;
Jos 10:30  и предал Господь и ее в руки Израиля, и царя ее, и истребил ее Иисус мечом и все дышащее, что [находилось] в ней: никого не оставил в ней, кто бы уцелел, и поступил с царем ее так же, как поступил с царем Иерихонским.
Jos 10:31  Из Ливны пошел Иисус и все Израильтяне с ним к Лахису и расположился подле него станом и воевал против него;
Jos 10:32  и предал Господь Лахис в руки Израиля, и взял он его на другой день, и поразил его мечом и все дышащее, что было в нем, так, как поступил с Ливною.
Jos 10:33  Тогда пришел на помощь Лахису Горам, царь Газерский; но Иисус поразил его и народ его [мечом] так, что никого у него не оставил, кто бы уцелел.
Jos 10:34  И пошел Иисус и все Израильтяне с ним из Лахиса к Еглону и расположились подле него станом и воевали против него;
Jos 10:35  И взяли его в тот же день и поразили его мечом, и все дышащее, что находилось в нем в тот день, предал он заклятию, как поступил с Лахисом.
Jos 10:36  И пошел Иисус и все Израильтяне с ним из Еглона к Хеврону и воевали против него;
Jos 10:37  и взяли его и поразили его мечом, и царя его, и все города его, и все дышащее, что находилось в нем; никого не оставил, кто уцелел бы, как поступил он и с Еглоном: предал заклятию его и все дышащее, что находилось в нем.
Jos 10:38  Потом обратился Иисус и весь Израиль с ним к Давиру и воевал против него;
Jos 10:39  и взял его и царя его и все города его, и поразили их мечом, и предали заклятию все дышащее, что находилось в нем: никого не осталось, кто уцелел бы; как поступил с Хевроном и царем его, так поступил с Давиром и царем его, и как поступил с Ливною и царем ее.
Jos 10:40  И поразил Иисус всю землю нагорную и полуденную, и низменные места и землю, лежащую у гор, и всех царей их: никого не оставил, кто уцелел бы, и все дышащее предал заклятию, как повелел Господь Бог Израилев;
Jos 10:41  поразил их Иисус от Кадес-Варни до Газы, и всю землю Гошен даже до Гаваона;
Jos 10:42  и всех царей сих и земли их Иисус взял одним разом, ибо Господь Бог Израилев сражался за Израиля.
Jos 10:43  Потом Иисус и все Израильтяне с ним возвратились в стан, в Галгал.
Jos 11:1  Услышав [сие], Иавин, царь Асорский, послал к Иоваву, царю Мадонскому, и к царю Шимронскому, и к царю Ахсафскому,
Jos 11:2  и к царям, которые [жили] к северу на горе и на равнине с южной стороны Хиннарофа, и на низменных местах, и в Нафоф-Доре к западу,
Jos 11:3  к Хананеям, [которые жили] к востоку и к морю, к Аморреям и Хеттеям, к Ферезеям и к Иевусеям, [жившим] на горе, и к Евеям, [жившим] подле Ермона в земле Массифе.
Jos 11:4  И выступили они и все ополчение их с ними, многочисленный народ, который множеством равнялся песку на берегу морском; и коней и колесниц [было] весьма много.
Jos 11:5  И собрались все цари сии, и пришли и расположились станом вместе при водах Меромских, чтобы сразиться с Израилем.
Jos 11:6  Но Господь сказал Иисусу: не бойся их, ибо завтра, около сего времени, Я предам всех [их] на избиение [сынам] Израиля; коням же их перережь жилы и колесницы их сожги огнем.
Jos 11:7  Иисус и с ним весь народ, способный к войне, внезапно вышли на них к водам Меромским и напали на них.
Jos 11:8  И предал их Господь в руки Израильтян, и поразили они их, и преследовали их до Сидона великого и до Мисрефоф-Маима, и до долины Мицфы к востоку, и перебили их, так что никого из них не осталось, кто уцелел бы.
Jos 11:9  И поступил Иисус с ними так, как сказал ему Господь: коням их перерезал жилы и колесницы их сожег огнем.
Jos 11:10  В то же время возвратившись Иисус взял Асор и царя его убил мечом (Асор же прежде был главою всех царств сих);
Jos 11:11  и побили все дышащее, что было в нем, мечом, предав заклятию: не осталось ни одной души; а Асор сожег он огнем.
Jos 11:12  И все города царей сих и всех царей их взял Иисус и побил мечом, предав их заклятию, как повелел Моисей, раб Господень;
Jos 11:13  впрочем всех городов, лежавших на возвышенности, не жгли Израильтяне, кроме одного Асора, [который] сжег Иисус.
Jos 11:14  А всю добычу городов сих и скот разграбили сыны Израилевы себе; людей же всех перебили мечом, так что истребили [всех] их: не оставили ни одной души.
Jos 11:15  Как повелел Господь Моисею, рабу Своему, так Моисей заповедал Иисусу, а Иисус так и сделал: не отступил ни от одного слова во всем, что повелел Господь Моисею.
Jos 11:16  Таким образом Иисус взял всю эту нагорную землю, всю землю полуденную, всю землю Гошен и низменные места, и равнину и гору Израилеву, и низменные места,
Jos 11:17  от горы Халак, простирающейся к Сеиру, до Ваал-Гада в долине Ливанской, подле горы Ермона, и всех царей их взял, поразил их и убил.
Jos 11:18  Долгое время вел Иисус войну со всеми сими царями.
Jos 11:19  Не было города, который заключил бы мир с сынами Израилевыми, кроме Евеев, жителей Гаваона: все взяли они войною;
Jos 11:20  ибо от Господа было то, что они ожесточили сердце свое и войною встречали Израиля--для того, чтобы преданы были заклятию и чтобы не было им помилования, но чтобы истреблены были так, как повелел Господь Моисею.
Jos 11:21  В то же время пришел Иисус и поразил Енакимов на горе, в Хевроне, в Давире, в Анаве, на всей горе Иудиной и на всей горе Израилевой; с городами их предал их Иисус заклятию;
Jos 11:22  не осталось [ни одного] из Енакимов в земле сынов Израилевых, остались только в Газе, в Гефе и в Азоте.
Jos 11:23  Таким образом взял Иисус всю землю, как говорил Господь Моисею, и отдал ее Иисусу в удел Израильтянам, по разделению между коленами их. И успокоилась земля от войны.
Jos 12:1  Вот цари той земли, которых поразили сыны Израилевы и которых землю взяли в наследие по ту сторону Иордана к востоку солнца, от потока Арнона до горы Ермона, и всю равнину к востоку:
Jos 12:2  Сигон, царь Аморрейский, живший в Есевоне, владевший от Ароера, что при береге потока Арнона, и от средины потока, половиною Галаада, до потока Иавока, предела Аммонитян,
Jos 12:3  и равниною до самого моря Хиннерефского к востоку и до моря равнины, моря Соленого, к востоку по дороге к Беф-Иешимофу, а к югу местами, лежащими при подошве Фасги;
Jos 12:4  сопредельный [ему] Ог, царь Васанский, последний из Рефаимов, живший в Астарофе и в Едреи,
Jos 12:5  владевший горою Ермоном и Салхою и всем Васаном, до предела Гессурского и Маахского, и половиною Галаада, до предела Сигона, царя Есевонского.
Jos 12:6  Моисей, раб Господень, и сыны Израилевы убили их; и дал ее Моисей, раб Господень, в наследие [колену] Рувимову и Гадову и половине колена Манассиина.
Jos 12:7  И вот цари [Аморрейской] земли, которых поразил Иисус и сыны Израилевы по эту сторону Иордана к западу, от Ваал-Гада на долине Ливанской до Халака, горы, простирающейся к Сеиру, которую отдал Иисус коленам Израилевым в наследие, по разделению их,
Jos 12:8  на горе, на низменных местах, на равнине, на местах, лежащих при горах, и в пустыне и на юге, Хеттеев, Аморреев, Хананеев, Ферезеев, Евеев и Иевусеев:
Jos 12:9  один царь Иерихона, один царь Гая, что близ Вефиля,
Jos 12:10  один царь Иерусалима, один царь Хеврона,
Jos 12:11  один царь Иармуфа, один царь Лахиса,
Jos 12:12  один царь Еглона, один царь Газера,
Jos 12:13  один царь Давира, один царь Гадера,
Jos 12:14  один царь Хормы, один царь Арада,
Jos 12:15  один царь Ливны, один царь Одоллама,
Jos 12:16  один царь Македа, один царь Вефиля,
Jos 12:17  один царь Таппуаха, один царь Хефера.
Jos 12:18  Один царь Афека, один царь Шарона,
Jos 12:19  один царь Мадона, один царь Асора,
Jos 12:20  один царь Шимрон-Мерона, один царь Ахсафа,
Jos 12:21  один царь Фаанаха, один царь Мегиддона,
Jos 12:22  один царь Кедеса, один царь Иокнеама при Кармиле,
Jos 12:23  один царь Дора при Нафаф-Доре, один царь Гоима в Галгале,
Jos 12:24  один царь Фирцы. Всех царей тридцать один.
Jos 13:1  Когда Иисус состарился, вошел в лета [преклонные], тогда Господь сказал ему: ты состарился, вошел в лета [преклонные], а земли брать в наследие остается еще очень много.
Jos 13:2  Остается сия земля: все округи Филистимские и вся [земля] Гессурская.
Jos 13:3  От Сихора, что пред Египтом, до пределов Екрона к северу, считаются Ханаанскими пять владельцев Филистимских: Газский, Азотский, Аскалонский, Гефский, Екронский и Аввейский;
Jos 13:4  к югу же вся земля Ханаанская от Меары Сидонской до Афека, до пределов Аморрейских,
Jos 13:5  также земля Гевла и весь Ливан к востоку солнца от Ваал-Гада, [что] подле горы Ермона, до входа в Емаф.
Jos 13:6  Всех горных жителей от Ливана до Мисрефоф-Маима, всех Сидонян Я изгоню от лица сынов Израилевых. Раздели же ее в удел Израилю, как Я повелел тебе;
Jos 13:7  раздели землю сию в удел девяти коленам и половине колена Манассиина.
Jos 13:8  А [колено] Рувимово и Гадово с другою половиною колена Манассиина получили удел свой от Моисея за Иорданом к востоку, как дал им Моисей, раб Господень,
Jos 13:9  от Ароера, который на берегу потока Арнона, и город, который среди потока, и всю равнину Медеву до Дивона;
Jos 13:10  также все города Сигона, царя Аморрейского, который царствовал в Есевоне, до пределов Аммонитских,
Jos 13:11  также Галаад и область Гессурскую и Маахскую, и всю гору Ермон и весь Васан до Салхи,
Jos 13:12  все царство Ога Васанского, который царствовал в Астарофе и в Едреи. Он оставался один из Рефаимов, которых Моисей поразил и прогнал.
Jos 13:13  Но сыны Израилевы не выгнали жителей Гессура и Маахи, и живет Гессур и Мааха среди Израиля до сего дня.
Jos 13:14  Только колену Левиину не дал он удела: жертвы Господа Бога Израилева суть удел его, как сказал ему Господь.
Jos 13:15  колену сынов Рувимовых по племенам их дал [удел] Моисей:
Jos 13:16  пределом их был Ароер, который на берегу потока Арнона, и город, который среди потока, и вся равнина при Медеве,
Jos 13:17  Есевон и все города его, которые на равнине, и Дивон, Вамоф-Ваали Беф-Ваал-Меон,
Jos 13:18  Иааца, Кедемоф и Мефааф,
Jos 13:19  Кириафаим, Сивма и Цереф-Шахар на горе Емек,
Jos 13:20  Беф-Фегор и места при подошве Фасги и Беф-Иешимоф,
Jos 13:21  и все города на равнине, и все царство Сигона, царя Аморрейского, который царствовал в Есевоне, которого убил Моисей, равно как и вождей Мадиамских: Евия, и Рекема, и Цура, и Хура, и Реву, князей Сигоновых, живших в земле [той];
Jos 13:22  также Валаама, сына Веорова, прорицателя, убили сыны Израилевы мечом в числе убитых ими.
Jos 13:23  Пределом сынов Рувимовых был Иордан. Вот удел сынов Рувимовых по племенам их, города и села их.
Jos 13:24  Моисей дал также [удел] колену Гадову, сынам Гадовым, по племенам их:
Jos 13:25  пределом их был Иазер и все города Галаадские, и половина земли сынов Аммоновых до Ароера, что пред Раввою,
Jos 13:26  и [земли] от Есевона до Рамаф-Мицфы и Ветонима и от Маханаима до пределов Давира,
Jos 13:27  и на долине Беф-Гарам и Беф-Нимра и Сокхоф и Цафон, остаток царства Сигона, царя Есевонского; пределом его был Иордан до моря Хиннерефского за Иорданом к востоку.
Jos 13:28  Вот удел сынов Гадовых по племенам их, города и села их.
Jos 13:29  Моисей дал [удел] и половине колена Манассиина, который [принадлежал] половине колена сынов Манассииных, по племенам их;
Jos 13:30  предел их был: от Маханаима весь Васан, все царство Ога, царя Васанского, и все селения Иаировы, что в Васане, шестьдесят городов;
Jos 13:31  а половина Галаада и Астароф и Едрея, царственные города Ога Васанского, [даны] сынам Махира, сына Манассиина, половине сынов Махировых, по племенам их.
Jos 13:32  Вот что Моисей дал в удел на равнинах Моавитских за Иорданом против Иерихона к востоку.
Jos 13:33  Но колену Левиину Моисей не дал удела: Господь Бог Израилев Сам есть удел их, как Он говорил им.
Jos 14:1  Вот что получили в удел сыны Израилевы в земле Ханаанской, что разделили им в удел Елеазар священник и Иисус, сын Навин, и начальники поколений в коленах сынов Израилевых;
Jos 14:2  по жребию делили они, как повелел Господь чрез Моисея, девяти коленам и половине колена [Манассиина],
Jos 14:3  ибо двум коленам и половине колена [Манассиина] Моисей дал удел за Иорданом, левитам же не дал удела между ними;
Jos 14:4  ибо от сынов Иосифовых произошли два колена: Манассиино и Ефремово; посему они и не дали левитам части в земле, [а только] города для жительства с предместиями их для скота их и для [других] выгод их.
Jos 14:5  Как повелел Господь Моисею, так [и] сделали сыны Израилевы, когда делили на уделы землю.
Jos 14:6  Сыны Иудины пришли в Галгал к Иисусу. И сказал ему Халев, сын Иефоннии, Кенезеянин: ты знаешь, что говорил Господь Моисею, человеку Божию, о мне и о тебе в Кадес-Варне;
Jos 14:7  я был сорока лет, когда Моисей, раб Господень, посылал меня из Кадес-Варни осмотреть землю, и я принес ему в ответ, что было у меня на сердце:
Jos 14:8  братья мои, которые ходили со мною, привели в робость сердце народа, а я в точности следовал Господу Богу моему;
Jos 14:9  и клялся Моисей в тот день и сказал: `земля, по которой ходила нога твоя, будет уделом тебе и детям твоим на век, ибо ты в точности последовал Господу Богу моему';
Jos 14:10  итак, вот, Господь сохранил меня в живых, как Он говорил; уже сорок пять лет [прошло] от того времени, когда Господь сказал Моисею слово сие, и Израиль ходил по пустыне; теперь, вот, мне восемьдесят пять лет;
Jos 14:11  но и ныне я столько же крепок, как и тогда, когда посылал меня Моисей: сколько тогда было у меня силы, столько и теперь есть для того, чтобы воевать и выходить и входить;
Jos 14:12  итак дай мне сию гору, о которой говорил Господь в тот день; ибо ты слышал в тот день, что там [живут] сыны Енаковы, и города [у них] большие и укрепленные; может быть, Господь [будет] со мною, и я изгоню их, как говорил Господь.
Jos 14:13  Иисус благословил его, и дал в удел Халеву, сыну Иефонниину, Хеврон.
Jos 14:14  Таким образом Хеврон остался уделом Халева, сына Иефонниина, Кенезеянина, до сего дня, за то, что он в точности последовал [повелению] Господа Бога Израилева.
Jos 14:15  Имя Хеврону прежде было Кириаф-Арбы, как назывался между сынами Енака один человек великий. И земля успокоилась от войны.
Jos 15:1  Жребий колену сынов Иудиных, по племенам их, выпал такой: в смежности с Идумеею была пустыня Син, к югу, при конце Фемана;
Jos 15:2  южным пределом их был край моря Соленого от простирающегося к югу залива;
Jos 15:3  на юге идет он к возвышенности Акраввимской, проходит Цин и, восходя с южной стороны к Кадес-Варне, проходит Хецрон и, восходя до Аддара, поворачивает к Каркае,
Jos 15:4  потом проходит Ацмон, идет к потоку Египетскому, так что конец сего предела есть море. Сей будет южный ваш предел.
Jos 15:5  Пределом же к востоку море Соленое, до устья Иордана; а предел с северной стороны [начинается] от залива моря, от устья Иордана;
Jos 15:6  отсюда предел восходит к Беф-Хогле и проходит с северной стороны к Беф-Араве, и идет предел вверх до камня Богана, сына Рувимова;
Jos 15:7  потом восходит предел к Давиру от долины Ахор и на севере поворачивает к Галгалу, который против возвышенности Адуммима, лежащего с южной стороны потока; отсюда предел проходит к водам Ен-Шемеша и оканчивается у Ен-Рогела;
Jos 15:8  отсюда предел идет вверх к долине сына Енномова с южной стороны Иевуса, который [есть] Иерусалим, и восходит предел на вершину горы, которая к западу против долины Енномовой, которая на краю долины Рефаимов к северу;
Jos 15:9  от вершины горы предел поворачивает к источнику вод Нефтоах и идет к городам горы Ефрона, и поворачивает предел к Ваалу, который [есть] Кириаф-Иарим;
Jos 15:10  потом поворачивает предел от Ваала к морю к горе Сеиру, и идет северною стороною горы Иеарим, которая [есть] Кесалон, и, нисходя к Вефсамису, проходит чрез Фимну;
Jos 15:11  отсюда предел идет северною стороною Екрона, и поворачивает предел к Шикарону, проходит чрез гору Ваал и доходит до Иавнеила, и оканчивается предел у моря. Западный предел составляет великое море.
Jos 15:12  Вот предел сынов Иудиных с племенами их со всех сторон.
Jos 15:13  И Халеву, сыну Иефонниину, [Иисус] дал часть среди сынов Иудиных, как повелел Господь Иисусу; Кириаф-Арбы, отца Енакова, иначе Хеврон.
Jos 15:14  И выгнал оттуда Халев трех сынов Енаковых: Шешая, Ахимана и Фалмая, детей Енаковых.
Jos 15:15  Отсюда пошел против жителей Давира (имя Давиру прежде [было] Кириаф-Сефер).
Jos 15:16  И сказал Халев: кто поразит Кириаф-Сефер и возьмет его, тому отдам Ахсу, дочь мою, в жену.
Jos 15:17  И взял его Гофониил, сын Кеназа, брата Халевова, и отдал он в жену ему Ахсу, дочь свою.
Jos 15:18  Когда надлежало ей идти, ее научили просить у отца ее поле, и она сошла с осла. Халев сказал ей: что тебе?
Jos 15:19  Она сказала: дай мне благословение; ты дал мне землю полуденную, дай мне и источники вод. И дал он ей источники верхние и источники нижние.
Jos 15:20  Вот удел колена сынов Иудиных, по племенам их:
Jos 15:21  города с края колена сынов Иудиных в смежности с Идумеею на юге были: Кавцеил, Едер и Иагур,
Jos 15:22  Кина, Димона, Адада,
Jos 15:23  Кедес, Асор и Ифнан,
Jos 15:24  Зиф, Телем и Валоф,
Jos 15:25  Гацор-Хадафа, Кириаф, Хецрон, иначе Гацор,
Jos 15:26  Амам, Шема и Молада,
Jos 15:27  Хацар-Гадда, Хешмон и Веф-Палет,
Jos 15:28  Хацар-Шуал, Вирсавия и Визиофея,
Jos 15:29  Ваала, Иим и Ацем,
Jos 15:30  Елфолад, Кесил и Хорма,
Jos 15:31  Циклаг, Мадмана и Сансана,
Jos 15:32  Леваоф, Шелихим, Аин и Риммон: всех двадцать девять городов с их селами.
Jos 15:33  На низменных местах: Ештаол, Цора и Ашна,
Jos 15:34  Заноах, Ен-Ганним, Таппуах и Гаенам,
Jos 15:35  Иармуф, Одоллам, Сохо и Азека,
Jos 15:36  Шаараим, Адифаим, Гедера или Гедерофаим: четырнадцать городов с их селами.
Jos 15:37  Ценан, Хадаша, Мигдал-Гад,
Jos 15:38  Дилеан, Мицфе и Иокфеил,
Jos 15:39  Лахис, Воцкаф и Еглон,
Jos 15:40  Хаббон, Лахмас и Хифлис,
Jos 15:41  Гедероф, Беф-Дагон, Наема и Макед: шестнадцать городов с их селами.
Jos 15:42  Ливна, Ефер и Ашан,
Jos 15:43  Иффах, Ашна и Нецив,
Jos 15:44  Кеила, Ахзив и Мареша: девять городов с их селами.
Jos 15:45  Екрон с зависящими от него [городами] и селами его,
Jos 15:46  и от Екрона к морю все, что находится около Азота, с селами их,
Jos 15:47  Азот, зависящие от него города и села его, Газа, зависящие от нее города и села ее, до самого потока Египетского и великого моря, которое [есть] предел.
Jos 15:48  На горах: Шамир, Иаттир и Сохо,
Jos 15:49  Данна, Кириаф-Санна, иначе Давир,
Jos 15:50  Анаф, Ештемо и Аним,
Jos 15:51  Гошен, Холон и Гило: одиннадцать городов с их селами.
Jos 15:52  Арав, Дума и Ешан,
Jos 15:53  Ианум, Беф-Таппуах и Афека,
Jos 15:54  Хумта, Кириаф-Арбы, иначе Хеврон, и Цигор: девять городов с их селами.
Jos 15:55  Маон, Кармил, Зиф и Юта,
Jos 15:56  Изреель, Иокдам и Заноах,
Jos 15:57  Каин, Гива и Фимна: десять городов с их селами.
Jos 15:58  Халхул, Беф-Цур и Гедор,
Jos 15:59  Маараф, Беф-Аноф и Елтекон: шесть городов с их селами.
Jos 15:60  Кириаф-Ваал, иначе Кириаф-Иарим, и Аравва: два города с их селами.
Jos 15:61  В пустыне: Беф-Арава, Миддин и Секаха,
Jos 15:62  Нившан, Ир-Мелах и Ен-Геди: шесть городов с их селами.
Jos 15:63  Но Иевусеев, жителей Иерусалима, не могли изгнать сыны Иудины, и потому Иевусеи живут с сынами Иуды в Иерусалиме даже до сего дня.
Jos 16:1  Потом выпал жребий сынам Иосифа: от Иордана подле Иерихона, у вод Иерихонских на восток, пустыня, простирающаяся от Иерихона к горе Вефильской;
Jos 16:2  от Вефиля идет [предел] к Лузу и переходит к пределу Архи до Атарофа,
Jos 16:3  и спускается к морю, к пределу Иафлета, до предела нижнего Беф-Орона и до Газера, и оканчивается у моря.
Jos 16:4  Это получили в удел сыны Иосифа: Манассия и Ефрем.
Jos 16:5  Предел сынов Ефремовых по племенам их был сей: от востока пределом удела их был Атароф-Адар до Беф-Орона верхнего;
Jos 16:6  потом идет предел к морю северною стороною Михмефафа и поворачивает к восточной стороне Фаанаф-Силома и проходит его с восточной стороны Ианоха;
Jos 16:7  от Ианоха, нисходя к Атарофу и Наарафу, примыкает к Иерихону и доходит до Иордана;
Jos 16:8  от Таппуаха идет предел к морю, к потоку Кане, и оканчивается морем. Вот удел колена сынов Ефремовых, по племенам их.
Jos 16:9  И города отделены сынам Ефремовым в уделе сынов Манассииных, все города с селами их.
Jos 16:10  Но [Ефремляне] не изгнали Хананеев, живших в Газере; посему Хананеи жили среди Ефремлян до сего дня, платя им дань.
Jos 17:1  И выпал жребий колену Манассии, так как он был первенец Иосифа. Махиру, первенцу Манассии, отцу Галаада, который [был] храбр на войне, достался Галаад и Васан.
Jos 17:2  Достались [уделы] и прочим сынам Манассии, по племенам их, и сынам Авиезера, и сынам Хелека, и сынам Асриила, и сынам Шехема, и сынам Хефера, и сынам Шемиды. Вот дети Манассии, сына Иосифова, мужеского пола, по племенам их.
Jos 17:3  У Салпаада же, сына Хеферова, сына Галаадова, сына Махирова, сына Манассиина, не было сыновей, а [только] дочери. Вот имена дочерей его: Махла, Ноа, Хогла, Милка и Фирца.
Jos 17:4  Они пришли к священнику Елеазару и к Иисусу, сыну Навину, и к начальникам, и сказали: Господь повелел Моисею дать нам удел между братьями нашими. И дан им удел, по повелению Господню, между братьями отца их.
Jos 17:5  И выпало Манассии десять участков, кроме земли Галаадской и Васанской, которая за Иорданом;
Jos 17:6  ибо дочери Манассии получили удел среди сыновей его, а земля Галаадская досталась прочим сыновьям Манассии.
Jos 17:7  Предел Манассии идет от Асира к Михмефафу, который против Сихема; отсюда предел идет направо к жителям Ен-Таппуаха.
Jos 17:8  Земля Таппуах досталась Манассии, а [город] Таппуах у предела Манассиина--сынам Ефремовым.
Jos 17:9  Отсюда предел нисходит к потоку Кане, с южной стороны потока. Города сии [принадлежат] Ефрему, [хотя находятся] среди городов Манассии. Предел Манассии--на северной стороне потока и оканчивается морем.
Jos 17:10  Что к югу, то Ефремово, а что к северу, то Манассиино; море же было пределом их; к Асиру примыкали они с северной стороны и к Иссахару с восточной.
Jos 17:11  У Иссахара и Асира [принадлежат] Манассии Беф-Сан и зависящие от него места, Ивлеам и зависящие от него места, жители Дора и зависящие от него места, жители Ен-Дора и зависящие от него места, жители Фаанаха и зависящие от него места, жители Мегиддона и зависящие от него места, и третья часть Нафефа.
Jos 17:12  Сыны Манассиины не могли выгнать [жителей] городов сих, и Хананеи остались жить в земле сей.
Jos 17:13  Когда же сыны Израилевы пришли в силу, тогда Хананеев сделали они данниками, но изгнать не изгнали их.
Jos 17:14  Сыны Иосифа говорили Иисусу и сказали: почему ты дал мне в удел один жребий и один участок, тогда как я многолюден, потому что так благословил меня Господь?
Jos 17:15  Иисус сказал им: если ты многолюден, то пойди в леса и там, в земле Ферезеев и Рефаимов, расчисти себе [место], если гора Ефремова для тебя тесна.
Jos 17:16  Сыны Иосифа сказали: не останется за нами гора, потому что железные колесницы у всех Хананеев, живущих на долине, как у тех, которые в Беф-Сане и в зависящих от него местах, так и у тех, которые на долине Изреельской.
Jos 17:17  Но Иисус сказал дому Иосифову, Ефрему и Манассии: ты многолюден и сила у тебя велика; не один жребий будет у тебя:
Jos 17:18  и гора будет твоею, и лес сей; ты расчистишь его, и он будет твой до самого конца его; ибо ты изгонишь Хананеев, хотя у них колесницы железные, и хотя они сильны.
Jos 18:1  Все общество сынов Израилевых собралось в Силом, и поставили там скинию собрания, ибо земля была покорена ими.
Jos 18:2  Из сынов же Израилевых оставалось семь колен, которые еще не получили удела своего.
Jos 18:3  И сказал Иисус сынам Израилевым: долго ли вы будете нерадеть о том, чтобы пойти [и] взять в наследие землю, которую дал вам Господь Бог отцов ваших?
Jos 18:4  дайте от себя по три человека из колена; я пошлю их, и они встав пройдут по земле и опишут ее, как надобно разделить им на уделы, и придут ко мне;
Jos 18:5  пусть разделят ее на семь уделов; Иуда пусть остается в пределе своем на юге, а дом Иосифов пусть остается в пределе своем на севере;
Jos 18:6  а вы распишите землю на семь уделов и представьте мне сюда: я брошу вам жребий здесь пред лицем Господа Бога нашего;
Jos 18:7  а левитам нет части между вами, ибо священство Господне есть удел их; Гад же, Рувим и половина колена Манассиина получили удел свой за Иорданом к востоку, который дал им Моисей, раб Господень.
Jos 18:8  Эти люди встали и пошли. Иисус же пошедшим описывать землю дал такое приказание: пойдите, обойдите землю, опишите ее и возвратитесь ко мне; а я здесь брошу вам жребий пред лицем Господним, в Силоме.
Jos 18:9  Они пошли, прошли по земле, и описали ее, по городам ее, на семь уделов, в книге, и пришли к Иисусу в стан, в Силом.
Jos 18:10  Иисус бросил им жребий в Силоме пред Господом, и разделил там Иисус землю сынам Израилевым по участкам их.
Jos 18:11  [Первый] жребий вышел колену сынов Вениаминовых, по племенам их. Предел их по жребию шел между сынами Иуды и между сынами Иосифа;
Jos 18:12  предел их на северной стороне начинается у Иордана, и проходит предел сей подле Иерихона с севера, и восходит на гору к западу, и оканчивается в пустыне Бефавен;
Jos 18:13  оттуда предел идет к Лузу, к южной стороне Луза, иначе Вефиля, и нисходит предел к Атароф-Адару, к горе, которая на южной стороне Беф-Орона нижнего;
Jos 18:14  потом предел поворачивает и склоняется к морской стороне на юг от горы, которая на юге пред Беф-Ороном, и оканчивается у Кириаф-Ваала, иначе Кириаф-Иарима, города сынов Иудиных. Это западная сторона.
Jos 18:15  Южною же стороною от Кириаф-Иарима идет предел к морю и доходит до источника вод Нефтоаха;
Jos 18:16  потом предел нисходит к концу горы, которая пред долиною сына Енномова, на долине Рефаимов, к северу, и нисходит долиною Еннома к южной стороне Иевуса, и идет к Ен-Рогелу;
Jos 18:17  потом поворачивает от севера и идет к Ен-Шемешу, и идет к Гелилофу, который против возвышенности Адуммима, и нисходит к камню Богана, сына Рувимова;
Jos 18:18  потом проходит близ равнины к северу и нисходит на равнину;
Jos 18:19  отсюда проходит предел подле Беф-Хоглы к северу, и оканчивается предел у северного залива моря Соленого, у южного конца Иордана. Вот предел южный. С восточной же стороны пределом служит Иордан.
Jos 18:20  Вот удел сынов Вениаминовых, с пределами его со всех сторон, по племенам их.
Jos 18:21  Города колену сынов Вениаминовых, по племенам их, принадлежали сии: Иерихон, Беф-Хогла и Емек-Кециц,
Jos 18:22  Беф-Арава, Цемараим и Вефиль,
Jos 18:23  Аввим, Фара и Офра,
Jos 18:24  Кефар-Аммонай, Афни и Гева: двенадцать городов с их селами.
Jos 18:25  Гаваон, Рама и Бероф,
Jos 18:26  Мицфе, Кефира и Моца,
Jos 18:27  Рекем, Ирфеил и Фарала,
Jos 18:28  Цела, Елеф и Иевус, иначе Иерусалим, Гивеаф и Кириаф: четырнадцать городов с их селами. Вот удел сынов Вениаминовых, по племенам их.
Jos 19:1  Второй жребий вышел Симеону, колену сынов Симеоновых, по племенам их; и был удел их среди удела сынов Иудиных.
Jos 19:2  В уделе их были: Вирсавия или Шева, Молада,
Jos 19:3  Хацар-Шуал, Вала и Ацем,
Jos 19:4  Елтолад, Вефул и Хорма,
Jos 19:5  Циклаг, Беф-Маркавоф и Хацар-Суса,
Jos 19:6  Беф-Леваоф и Шарухен: тринадцать городов с их селами.
Jos 19:7  Аин, Риммон, Ефер и Ашан: четыре города с селами их,
Jos 19:8  и все села, которые находились вокруг городов сих даже до Ваалаф-Беера, или южной Рамы. Вот удел колена сынов Симеоновых, по племенам их.
Jos 19:9  От участка сынов Иудиных [выделен] удел [колену] сынов Симеоновых. Так как участок сынов Иудиных был слишком велик для них, то сыны Симеоновы и получили удел среди их удела.
Jos 19:10  Третий жребий выпал сынам Завулоновым по племенам их, и простирался предел удела их до Сарида;
Jos 19:11  предел их восходит к морю и Марале и примыкает к Дабешефу и примыкает к потоку, который пред Иокнеамом;
Jos 19:12  от Сарида идет назад к восточной стороне, к востоку солнца, до предела Кислоф-Фавора; отсюда идет к Даврафу и восходит к Иафие;
Jos 19:13  отсюда проходит к востоку в Геф-Хефер, в Итту-Кацин, и идет к Риммону, Мифоару и Нее;
Jos 19:14  и поворачивает предел от севера к Ханнафону и оканчивается долиною Ифтах-Ел;
Jos 19:15  далее: Каттаф, Нагалал, Шимрон, Идеала и Вифлеем: двенадцать городов с их селами.
Jos 19:16  Вот удел сынов Завулоновых, по их племенам; вот города и села их.
Jos 19:17  Четвертый жребий вышел Иссахару, сынам Иссахара, по племенам их;
Jos 19:18  пределом их был: Изреель, Кесуллоф и Сунем,
Jos 19:19  Хафараим, Шион и Анахараф,
Jos 19:20  Раввиф, Кишион и Авец,
Jos 19:21  Ремеф, Ен-Ганним, Ен-Хадда и Беф-Пацец;
Jos 19:22  и примыкает предел к Фавору и Шагациме и Вефсамису, и оканчивается предел их у Иордана: шестнадцать городов с селами их.
Jos 19:23  Вот удел колена сынов Иссахаровых по племенам их; вот города и села их.
Jos 19:24  Пятый жребий вышел колену сынов Асировых, по племенам их;
Jos 19:25  пределом их были: Хелкаф, Хали, Ветен и Ахсаф,
Jos 19:26  Аламелех, Амад и Мишал; и примыкает [предел] к Кармилу с западной стороны и к Шихор-Ливнафу;
Jos 19:27  потом идет назад к востоку солнца в Беф-Дагон, и примыкает к Завулону и к долине Ифтах-Ел с севера, в Беф-Емек и Неиел, и идет у Кавула, с левой стороны;
Jos 19:28  далее: Еврон, Рехов, Хаммон и Кана, до Сидона великого;
Jos 19:29  потом предел возвращается к Раме до укрепленного города Тира, и поворачивает предел к Хоссе, и оканчивается у моря, в местечке Ахзиве;
Jos 19:30  далее: Умма, Афек и Рехов: двадцать два города с селами их.
Jos 19:31  Вот удел колена сынов Асировых, по племенам их; вот города и села их.
Jos 19:32  Шестой жребий вышел сынам Неффалима, сынам Неффалима по племенам их;
Jos 19:33  предел их шел от Хелефа [и] от дубравы, [что] в Цананниме, к Адами-Некеву и Иавнеилу, до Лаккума, и оканчивался у Иордана;
Jos 19:34  отсюда возвращается предел на запад к Азноф-Фавору и идет оттуда к Хуккоку, и примыкает к Завулону с юга, и к Асиру примыкает с запада, и к Иуде у Иордана, от востока солнца.
Jos 19:35  Города укрепленные: Циддим, Цер, Хамаф, Раккаф и Хиннереф,
Jos 19:36  Адама, Рама и Асор,
Jos 19:37  Кедес, Едрея и Ен-Гацор,
Jos 19:38  Иреон, Мигдал-Ел, Хорем, Беф-Анаф и Вефсамис: девятнадцать городов с их селами.
Jos 19:39  Вот удел колена сынов Неффалимовых по племенам их; вот города и села их.
Jos 19:40  Колену сынов Дановых, по племенам их, вышел жребий седьмой;
Jos 19:41  пределом удела их были: Цора, Ештаол и Ир-Шемеш,
Jos 19:42  Шаалаввин, Аиалон и Ифла,
Jos 19:43  Елон, Фимнафа и Екрон,
Jos 19:44  Елтеке, Гиввефон и Ваалаф,
Jos 19:45  Игуд, Бене-Верак и Гаф-Риммон,
Jos 19:46  Ме-Иаркон и Ракон с пределом близ Иоппии. И вышел предел сынов Дановых мал для них.
Jos 19:47  И сыны Дановы пошли войною на Ласем и взяли его, и поразили его мечом, и получили его в наследие, и поселились в нем, и назвали Ласем Даном по имени Дана, отца своего.
Jos 19:48  Вот удел колена сынов Дановых, по племенам их; вот города и села их.
Jos 19:49  Когда окончили разделение земли, по пределам ее, тогда сыны Израилевы дали среди себя удел Иисусу, сыну Навину:
Jos 19:50  по повелению Господню дали ему город Фамнаф-Сараи, которого он просил, на горе Ефремовой; и построил он город и жил в нем.
Jos 19:51  Вот уделы, которые Елеазар священник, Иисус, сын Навин, и начальники поколений разделили коленам сынов Израилевых, по жребию, в Силоме, пред лицем Господним, у входа скинии собрания. И кончили разделение земли.
Jos 20:1  И сказал Господь Иисусу, говоря:
Jos 20:2  скажи сынам Израилевым: сделайте у себя города убежища, о которых Я говорил вам чрез Моисея,
Jos 20:3  чтобы мог убегать туда убийца, убивший человека по ошибке, без умысла; пусть [города сии] будут у вас убежищем от мстящего за кровь.
Jos 20:4  И кто убежит в один из городов сих, пусть станет у ворот города и расскажет вслух старейшин города сего дело свое; и они примут его к себе в город и дадут ему место, чтоб он жил у них;
Jos 20:5  и когда погонится за ним мстящий за кровь, то они не должны выдавать в руки его убийцу, потому что он без умысла убил ближнего своего, не имел к нему ненависти ни вчера, ни третьего дня;
Jos 20:6  пусть он живет в этом городе, доколе не предстанет пред общество на суд, доколе не умрет великий священник, который будет в те дни. А потом пусть возвратится убийца и пойдет в город свой и в дом свой, в город, из которого он убежал.
Jos 20:7  И отделили Кедес в Галилее на горе Неффалимовой, Сихем на горе Ефремовой, и Кириаф-Арбы, иначе Хеврон, на горе Иудиной;
Jos 20:8  за Иорданом, против Иерихона к востоку, отделили: Бецер в пустыне, на равнине, от колена Рувимова, и Рамоф в Галааде от колена Гадова, и Голан в Васане от колена Манассиина;
Jos 20:9  сии города назначены для всех сынов Израилевых и для пришельцев, живущих у них, дабы убегал туда всякий, убивший человека по ошибке, дабы не умер он от руки мстящего за кровь, доколе не предстанет пред общество [на суд].
Jos 21:1  Начальники поколений левитских пришли к Елеазару священнику и к Иисусу, сыну Навину, и к начальникам поколений сынов Израилевых,
Jos 21:2  и говорили им в Силоме, в земле Ханаанской, и сказали: Господь повелел чрез Моисея дать нам города для жительства и предместья их для скота нашего.
Jos 21:3  И дали сыны Израилевы левитам из уделов своих, по повелению Господню, сии города с предместьями их.
Jos 21:4  Вышел жребий племенам Каафовым; и досталось по жребию сынам Аарона священника, левитам, от колена Иудина, и от колена Симеонова, и от колена Вениаминова, тринадцать городов;
Jos 21:5  а прочим сынам Каафа от племен колен Ефремова, и от колена Данова, и от половины колена Манассиина, по жребию, [досталось] десять городов;
Jos 21:6  сынам Гирсоновым--от племен колена Иссахарова, и от колена Асирова, и от колена Неффалимова, и от половины колена Манассиина в Васане, по жребию, [досталось] тринадцать городов;
Jos 21:7  сынам Мерариным, по их племенам, от колена Рувимова, от колена Гадова и от колена Завулонова--двенадцать городов.
Jos 21:8  И отдали сыны Израилевы левитам сии города с предместьями их, как повелел Господь чрез Моисея, по жребию.
Jos 21:9  От колена сынов Иудиных, и от колена сынов Симеоновых, дали города, которые [здесь] названы по имени:
Jos 21:10  сынам Аарона, из племен Каафовых, из сынов Левия, так как жребий их был первый,
Jos 21:11  дали Кириаф-Арбы, отца Енакова, иначе Хеврон, на горе Иудиной, и предместья его вокруг его;
Jos 21:12  а поле сего города и села его отдали в собственность Халеву, сыну Иефонниину.
Jos 21:13  Итак сынам Аарона священника дали город убежища для убийцы--Хеврон и предместья его, Ливну и предместья ее,
Jos 21:14  Иаттир и предместья его, Ештемо и предместья его,
Jos 21:15  Холон и предместья его, Давир и предместья его,
Jos 21:16  Аин и предместья его, Ютту и предместья ее, Беф-Шемеш и предместья его: девять городов от двух колен сих;
Jos 21:17  а от колена Вениаминова: Гаваон и предместья его, Геву и предместья ее,
Jos 21:18  Анафоф и предместья его, Алмон и предместья его: четыре города.
Jos 21:19  Всех городов сынам Аароновым, священникам, [досталось] тринадцать городов с предместьями их.
Jos 21:20  И племенам сынов Каафовых, левитов, прочим из сынов Каафовых, по жребию их, достались города от колена Ефремова;
Jos 21:21  дали им город убежища для убийцы--Сихем и предместья его, на горе Ефремовой, Гезер и предместья его,
Jos 21:22  Кивцаим и предместья его, Беф-Орон и предместья его: четыре города;
Jos 21:23  от колена Данова: Елфеке и предместья его, Гиввефон и предместья его,
Jos 21:24  Аиалон и предместья его, Гаф-Риммон и предместья его: четыре города;
Jos 21:25  от половины колена Манассиина: Фаанах и предместья его, Гаф-Риммон и предместья его: два города.
Jos 21:26  Всех городов с предместьями их прочим племенам сынов Каафовых [досталось] десять.
Jos 21:27  А сынам Гирсоновым, из племен левитских [дали]: от половины колена Манассиина город убежища для убийцы--Голан в Васане и предместья его, и Беештеру и предместья ее: два города;
Jos 21:28  от колена Иссахарова: Кишион и предместья его, Давраф и предместья его,
Jos 21:29  Иармуф и предместья его, Ен-Ганним и предместья его: четыре города;
Jos 21:30  от колена Асирова: Мишал и предместья его, Авдон и предместья его,
Jos 21:31  Хелкаф и предместья его, Рехов и предместья его: четыре города;
Jos 21:32  от колена Неффалимова город убежища для убийцы--Кедес в Галилее и предместья его, Хамоф-Дор и предместья его, Карфан и предместья его: три города.
Jos 21:33  Всех городов сынам Гирсоновым, по племенам их, [досталось] тринадцать городов с предместьями их.
Jos 21:34  Племенам сынов Мерариных, остальным левитам, [дали]: от колена Завулонова Иокнеам и предместья его, Карфу и предместья ее,
Jos 21:35  Димну и предместья ее, Нагалал и предместья его: четыре города;
Jos 21:36  от колена Рувимова Бецер и предместья его, Иааца и предместья ее,
Jos 21:37  Кедемоф и предместья его, Мефааф и предместья его: четыре города;
Jos 21:38  от колена Гадова: города убежища для убийцы--Рамоф в Галааде и предместья его, Маханаим и предместья его,
Jos 21:39  Есевон и предместья его, Иазер и предместья его: всех городов четыре.
Jos 21:40  Всех городов сынам Мерариным по племенам их, остальным племенам левитским, по жребию досталось двенадцать городов.
Jos 21:41  Всех городов левитских среди владения сынов Израилевых [было] сорок восемь городов с предместьями их.
Jos 21:42  При городах сих были при каждом городе предместья вокруг него: так было при всех городах сих.
Jos 21:43  Таким образом отдал Господь Израилю всю землю, которую дать клялся отцам их, и они получили ее в наследие и поселились на ней.
Jos 21:44  И дал им Господь покой со всех сторон, как клялся отцам их, и никто из всех врагов их не устоял против них; всех врагов их предал Господь в руки их.
Jos 21:45  Не осталось неисполнившимся ни одно слово из всех добрых слов, которые Господь говорил дому Израилеву; все сбылось.
Jos 22:1  Тогда Иисус призвал [колено] Рувимово, Гадово и половину колена Манассиина и сказал им:
Jos 22:2  вы исполнили все, что повелел вам Моисей, раб Господень, и слушались слов моих во всем, что я приказывал вам;
Jos 22:3  вы не оставляли братьев своих в продолжение многих дней до сего дня и исполнили, что надлежало исполнить по повелению Господа, Бога вашего:
Jos 22:4  ныне Господь, Бог ваш, успокоил братьев ваших, как говорил им; итак возвратитесь и пойдите в шатры ваши, в землю вашего владения, которую дал вам Моисей, раб Господень, за Иорданом;
Jos 22:5  только старайтесь тщательно исполнять заповеди и закон, который завещал вам Моисей, раб Господень: любить Господа Бога вашего, ходить всеми путями Его, хранить заповеди Его, прилепляться к Нему и служить Ему всем сердцем вашим и всею душею вашею.
Jos 22:6  Потом Иисус благословил их и отпустил их, и они разошлись по шатрам своим.
Jos 22:7  Одной половине колена Манассиина дал Моисей удел в Васане, а другой половине его дал Иисус [удел] с братьями его по эту сторону Иордана к западу. И когда отпускал их Иисус в шатры их и благословил их,
Jos 22:8  то сказал им: с великим богатством возвращаетесь вы в шатры ваши, с великим множеством скота, с серебром, с золотом, с медью и с железом, и с великим множеством одежд; разделите же добычу, [взятую] у врагов ваших, с братьями своими.
Jos 22:9  И возвратились, и пошли сыны Рувимовы и сыны Гадовы и половина колена Манассиина от сынов Израилевых из Силома, который в земле Ханаанской, чтоб идти в землю Галаад, в землю своего владения, которую получили во владение по повелению Господню, [данному] чрез Моисея.
Jos 22:10  Придя в окрестности Иордана, что в земле Ханаанской, сыны Рувимовы и сыны Гадовы и половина колена Манассиина соорудили там подле Иордана жертвенник, жертвенник большой по виду.
Jos 22:11  И услышали сыны Израилевы, что говорят: вот, сыны Рувимовы и сыны Гадовы и половина колена Манассиина соорудили жертвенник на земле Ханаанской, в окрестностях Иордана, напротив сынов Израилевых.
Jos 22:12  Когда услышали [сие] сыны Израилевы, то собралось все общество сынов Израилевых в Силом, чтоб идти против них войною.
Jos 22:13  Впрочем сыны Израилевы [прежде] послали к сынам Рувимовым и к сынам Гадовым и к половине колена Манассиина в землю Галаадскую Финееса, сына Елеазара, священника,
Jos 22:14  и с ним десять начальников, по начальнику поколения от всех колен Израилевых; каждый из них был начальником поколения в тысячах Израилевых.
Jos 22:15  И пришли они к сынам Рувимовым и к сынам Гадовым и к половине колена Манассиина в землю Галаад и говорили им и сказали:
Jos 22:16  так говорит все общество Господне: что это за преступление сделали вы пред Богом Израилевым, отступив ныне от Господа, соорудив себе жертвенник и восстав ныне против Господа?
Jos 22:17  Разве мало для нас беззакония Фегорова, от которого мы не очистились до сего дня и [за которое] поражено было общество Господне?
Jos 22:18  А вы отступаете сегодня от Господа! Сегодня вы восстаете против Господа, а завтра прогневается [Господь] на все общество Израилево;
Jos 22:19  если же земля вашего владения кажется вам нечистою, то перейдите в землю владения Господня, в которой находится скиния Господня, возьмите удел среди нас, но не восставайте против Господа и против нас не восставайте, сооружая себе жертвенник, кроме жертвенника Господа, Бога нашего;
Jos 22:20  не [один] ли Ахан, сын Зары, сделал преступление, [взяв] из заклятого, а гнев был на все общество Израилево? не один он умер за свое беззаконие.
Jos 22:21  Сыны Рувимовы и сыны Гадовы и половина колена Манассиина в ответ [на сие] говорили начальникам тысяч Израилевых:
Jos 22:22  Бог богов Господь, Бог богов Господь, Он знает, и Израиль да знает! Если мы восстаем и отступаем от Господа, то да не пощадит нас [Господь] в сей день!
Jos 22:23  Если мы соорудили жертвенник для того, чтоб отступить от Господа, и для того, чтобы приносить на нем всесожжение и приношение хлебное и чтобы совершать на нем жертвы мирные, то да взыщет Сам Господь!
Jos 22:24  Но мы сделали сие по опасению того, чтобы в последующее время не сказали ваши сыны нашим сынам: `что вам до Господа Бога Израилева!
Jos 22:25  Господь поставил пределом между нами и вами, сыны Рувимовы и сыны Гадовы, Иордан: нет вам части в Господе'. Таким образом ваши сыны не допустили бы наших сынов чтить Господа.
Jos 22:26  Поэтому мы сказали: соорудим себе жертвенник не для всесожжения и не для жертв,
Jos 22:27  но чтобы он между нами и вами, между последующими родами нашими, был свидетелем, что мы можем служить Господу всесожжениями нашими и жертвами нашими и благодарениями нашими, и чтобы в последующее время не сказали ваши сыны сынам нашим: `нет вам части в Господе'.
Jos 22:28  Мы говорили: если скажут так нам и родам нашим в последующее время, то мы скажем: видите подобие жертвенника Господа, которое сделали отцы наши не для всесожжения и не для жертвы, но чтобы это было свидетелем между нами и вами.
Jos 22:29  Да не будет этого, чтобы восстать нам против Господа и отступить ныне от Господа, и соорудить жертвенник для всесожжения и для приношения хлебного и для жертв, кроме жертвенника Господа Бога нашего, который пред скиниею Его.
Jos 22:30  Финеес священник, начальники общества и головы тысяч Израилевых, которые были с ним, услышав слова, которые говорили сыны Рувимовы и сыны Гадовы и сыны Манассиины, одобрили их.
Jos 22:31  И сказал Финеес, сын Елеазара, священник, сынам Рувимовым и сынам Гадовым и сынам Манассииным: сегодня мы узнали, что Господь среди нас, что вы не сделали пред Господом преступления сего; теперь вы избавили сынов Израиля от руки Господней.
Jos 22:32  И возвратился Финеес, сын Елеазара, священник, и начальники от сынов Рувимовых и от сынов Гадовых в землю Ханаанскую к сынам Израилевым и принесли им ответ.
Jos 22:33  И сыны Израилевы одобрили это, и благословили сыны Израилевы Бога и отложили идти против них войною, чтобы разорить землю, на которой жили сыны Рувимовы и сыны Гадовы.
Jos 22:34  И назвали сыны Рувимовы и сыны Гадовы жертвенник: [Ед], потому что, [сказали они,] он свидетель между нами, что Господь есть Бог наш.
Jos 23:1  Спустя много времени после того, как Господь успокоил Израиля от всех врагов его со всех сторон, Иисус состарился, вошел в [преклонные] лета.
Jos 23:2  И призвал Иисус всех [сынов] Израилевых, старейшин их, начальников их, судей их и надзирателей их, и сказал им: я состарился, вошел в [преклонные] лета.
Jos 23:3  Вы видели все, что сделал Господь Бог ваш пред лицем вашим со всеми сими народами, ибо Господь Бог ваш Сам сражался за вас.
Jos 23:4  Вот, я разделил вам по жребию оставшиеся народы сии в удел коленам вашим, все народы, которые я истребил, от Иордана до великого моря, на запад солнца.
Jos 23:5  Господь Бог ваш Сам прогонит их от вас, и истребит их пред вами, дабы вы получили в наследие землю их, как говорил вам Господь Бог ваш.
Jos 23:6  Посему во всей точности старайтесь хранить и исполнять все написанное в книге закона Моисеева, не уклоняясь от него ни направо, ни налево.
Jos 23:7  Не сообщайтесь с сими народами, которые остались между вами, не воспоминайте имени богов их, не клянитесь [ими] и не служите им и не поклоняйтесь им,
Jos 23:8  но прилепитесь к Господу Богу вашему, как вы делали до сего дня.
Jos 23:9  Господь прогнал от вас народы великие и сильные, и пред вами никто не устоял до сего дня;
Jos 23:10  один из вас прогоняет тысячу, ибо Господь Бог ваш Сам сражается за вас, как говорил вам.
Jos 23:11  Посему всячески старайтесь любить Господа Бога вашего.
Jos 23:12  Если же вы отвратитесь и пристанете к оставшимся из народов сих, которые остались между вами, и вступите в родство с ними и будете ходить к ним и они к вам,
Jos 23:13  то знайте, что Господь Бог ваш не будет уже прогонять от вас народы сии, но они будут для вас петлею и сетью, бичом для ребр ваших и терном для глаз ваших, доколе не будете истреблены с сей доброй земли, которую дал вам Господь Бог ваш.
Jos 23:14  Вот, я ныне отхожу в путь всей земли. А вы знаете всем сердцем вашим и всею душею вашею, что не осталось тщетным ни одно слово из всех добрых слов, которые говорил о вас Господь Бог ваш; все сбылось для вас, ни одно слово не осталось неисполнившимся.
Jos 23:15  Но как сбылось над вами всякое доброе слово, которое говорил вам Господь Бог ваш, так Господь исполнит над вами всякое злое слово, доколе не истребит вас с этой доброй земли, которую дал вам Господь Бог ваш.
Jos 23:16  Если вы преступите завет Господа Бога вашего, который Он поставил с вами, и пойдете и будете служить другим богам и поклоняться им, то возгорится на вас гнев Господень, и скоро сгибнете с этой доброй земли, которую дал вам [Господь].
Jos 24:1  И собрал Иисус все колена Израилевы в Сихем и призвал старейшин Израиля и начальников его, и судей его и надзирателей его, и предстали пред [Господа] Бога.
Jos 24:2  И сказал Иисус всему народу: так говорит Господь Бог Израилев: `за рекою жили отцы ваши издревле, Фарра, отец Авраама и отец Нахора, и служили иным богам.
Jos 24:3  Но Я взял отца вашего Авраама из-за реки и водил его по всей земле Ханаанской, и размножил семя его и дал ему Исаака.
Jos 24:4  Исааку дал Иакова и Исава. Исаву дал Я гору Сеир в наследие; Иаков же и сыны его перешли в Египет
Jos 24:5  И послал Я Моисея и Аарона и поразил Египет язвами, которые делал Я среди его, и потом вывел вас.
Jos 24:6  Я вывел отцов ваших из Египта, и вы пришли к [Чермному] морю. Тогда Египтяне гнались за отцами вашими с колесницами и всадниками до Чермного моря;
Jos 24:7  но они возопили к Господу, и Он положил тьму между вами и Египтянами и навел на них море, которое их и покрыло. Глаза ваши видели, что Я сделал в Египте. [Потом] много времени пробыли вы в пустыне.
Jos 24:8  И привел Я вас к земле Аморреев, живших за Иорданом; они сразились с вами, но Я предал их в руки ваши, и вы получили в наследие землю их, и Я истребил их пред вами.
Jos 24:9  Восстал Валак, сын Сепфоров, царь Моавитский, и пошел войною на Израиля, и послал и призвал Валаама, сына Веорова, чтоб он проклял вас;
Jos 24:10  но Я не хотел послушать Валаама, --и он благословил вас, и Я избавил вас из рук его.
Jos 24:11  Вы перешли Иордан и пришли к Иерихону. И стали воевать с вами жители Иерихона, Аморреи, и Ферезеи, и Хананеи, и Хеттеи, и Гергесеи, и Евеи, и Иевусеи, но Я предал их в руки ваши.
Jos 24:12  Я послал пред вами шершней, которые прогнали их от вас, двух царей Аморрейских; не мечом твоим и не луком твоим [сделано это].
Jos 24:13  И дал Я вам землю, над которою ты не трудился, и города, которых вы не строили, и вы живете в них; из виноградных и масличных садов, которых вы не насаждали, вы едите [плоды]'.
Jos 24:14  Итак бойтесь Господа и служите Ему в чистоте и искренности; отвергните богов, которым служили отцы ваши за рекою и в Египте, а служите Господу.
Jos 24:15  Если же не угодно вам служить Господу, то изберите себе ныне, кому служить, богам ли, которым служили отцы ваши, бывшие за рекою, или богам Аморреев, в земле которых живете; а я и дом мой будем служить Господу.
Jos 24:16  И отвечал народ и сказал: нет, не будет того, чтобы мы оставили Господа и стали служить другим богам!
Jos 24:17  Ибо Господь--Бог наш, Он вывел нас и отцов наших из земли Египетской, из дома рабства, и делал пред глазами нашими великие знамения и хранил нас на всем пути, по которому мы шли, и среди всех народов, чрез которые мы проходили.
Jos 24:18  Господь прогнал от нас все народы и Аморреев, живших в сей земле. Посему и мы будем служить Господу, ибо Он--Бог наш.
Jos 24:19  Иисус сказал народу: не возможете служить Господу, ибо Он Бог святый, Бог ревнитель, не потерпит беззакония вашего и грехов ваших.
Jos 24:20  Если вы оставите Господа и будете служить чужим богам, то Он наведет на вас зло и истребит вас, после того как благотворил вам.
Jos 24:21  И сказал народ Иисусу: нет, мы Господу будем служить.
Jos 24:22  Иисус сказал народу: вы свидетели о себе, что вы избрали себе Господа--служить Ему? Они отвечали: свидетели.
Jos 24:23  Итак отвергните чужих богов, которые у вас, и обратите сердце свое к Господу Богу Израилеву.
Jos 24:24  Народ сказал Иисусу: Господу Богу нашему будем служить и гласа Его будем слушать.
Jos 24:25  И заключил Иисус с народом завет в тот день и дал ему постановления и закон в Сихеме.
Jos 24:26  И вписал Иисус слова сии в книгу закона Божия, и взял большой камень и положил его там под дубом, который подле святилища Господня.
Jos 24:27  И сказал Иисус всему народу: вот, камень сей будет нам свидетелем, ибо он слышал все слова Господа, которые Он говорил с нами; он да будет свидетелем против вас, чтобы вы не солгали пред Богом вашим.
Jos 24:28  И отпустил Иисус народ, каждого в свой удел.
Jos 24:29  После сего умер Иисус, сын Навин, раб Господень, будучи ста десяти лет.
Jos 24:30  И похоронили его в пределе его удела в Фамнаф-Сараи, что на горе Ефремовой, на север от горы Гааша.
Jos 24:31  И служил Израиль Господу во все дни Иисуса и во все дни старейшин, которых жизнь продлилась после Иисуса и которые видели все дела Господа, какие Он сделал Израилю.
Jos 24:32  И кости Иосифа, которые вынесли сыны Израилевы из Египта, схоронили в Сихеме, в участке поля, которое купил Иаков у сынов Еммора, отца Сихемова, за сто монет и которое досталось в удел сынам Иосифовым.
Jos 24:33  [После сего] умер и Елеазар, сын Аарона, и похоронили его на холме Финееса, сына его, который дан ему на горе Ефремовой.


\end{document}