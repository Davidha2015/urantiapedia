\begin{document}

\title{Судей Израилевых}


\chapter{1}

\par 1 По смерти Иисуса вопрошали сыны Израилевы Господа, говоря: кто из нас прежде пойдет на Хананеев--воевать с ними?
\par 2 И сказал Господь: Иуда пойдет; вот, Я предаю землю в руки его.
\par 3 Иуда же сказал Симеону, брату своему: войди со мною в жребий мой, и будем воевать с Хананеями; и я войду с тобою в твой жребий. И пошел с ним Симеон.
\par 4 И пошел Иуда, и предал Господь Хананеев и Ферезеев в руки их, и побили они из них в Везеке десять тысяч человек.
\par 5 В Везеке встретились они с Адони-Везеком, сразились с ним и разбили Хананеев и Ферезеев.
\par 6 Адони-Везек побежал, но они погнались за ним и поймали его и отсекли большие пальцы на руках его и на ногах его.
\par 7 Тогда сказал Адони-Везек: семьдесят царей с отсеченными на руках и на ногах их большими пальцами собирали [крохи] под столом моим; как делал я, так и мне воздал Бог. И привели его в Иерусалим, и он умер там.
\par 8 И воевали сыны Иудины против Иерусалима и взяли его, и поразили его мечом и город предали огню.
\par 9 Потом пошли сыны Иудины воевать с Хананеями, которые жили на горах и на полуденной земле и на низменных местах.
\par 10 И пошел Иуда на Хананеев, которые жили в Хевроне (имя же Хеврону [было] прежде Кириаф-Арбы), и поразили Шешая, Ахимана и Фалмая.
\par 11 Оттуда пошел он против жителей Давира; имя Давиру [было] прежде Кириаф-Сефер.
\par 12 И сказал Халев: кто поразит Кириаф-Сефер и возьмет его, тому отдам Ахсу, дочь мою, в жену.
\par 13 И взял его Гофониил, сын Кеназа, младшего брата Халевова, и [Халев] отдал в жену ему Ахсу, дочь свою.
\par 14 Когда надлежало ей идти, [Гофониил] научил ее просить у отца ее поле, и она сошла с осла. Халев сказал ей: что тебе?
\par 15 [Ахса] сказала ему: дай мне благословение; ты дал мне землю полуденную, дай мне и источники воды. И дал ей [Халев] источники верхние и источники нижние.
\par 16 И сыны [Иофора] Кенеянина, тестя Моисеева, пошли из города Пальм с сынами Иудиными в пустыню Иудину, которая на юг от Арада, и пришли и поселились среди народа.
\par 17 И пошел Иуда с Симеоном, братом своим, и поразили Хананеев, живших в Цефафе, и предали его заклятию, и [оттого] называется город сей Хорма.
\par 18 Иуда взял также Газу с пределами ее, Аскалон с пределами его, и Екрон с пределами его.
\par 19 Господь был с Иудою, и он овладел горою; но жителей долины не мог прогнать, потому что у них были железные колесницы.
\par 20 И отдали Халеву Хеврон, как говорил Моисей, и изгнал [он] оттуда трех сынов Енаковых.
\par 21 Но Иевусеев, которые жили в Иерусалиме, не изгнали сыны Вениаминовы, и живут Иевусеи с сынами Вениамина в Иерусалиме до сего дня.
\par 22 И сыны Иосифа пошли также на Вефиль, и Господь был с ними.
\par 23 И остановились и высматривали сыны Иосифовы Вефиль (имя же городу [было] прежде Луз).
\par 24 И увидели стражи человека, идущего из города, и сказали ему: покажи нам вход в город, и сделаем с тобою милость.
\par 25 Он показал им вход в город, и поразили они город мечом, а человека сего и все родство его отпустили.
\par 26 Человек сей пошел в землю Хеттеев, и построил город и нарек имя ему Луз. Это имя его до сего дня.
\par 27 И Манассия не выгнал [жителей] Бефсана и зависящих от него городов, Фаанаха и зависящих от него городов, жителей Дора и зависящих от него городов, жителей Ивлеама и зависящих от него городов, жителей Мегиддона и зависящих от него городов; и остались Хананеи жить в земле сей.
\par 28 Когда Израиль пришел в силу, тогда сделал он Хананеев данниками, но изгнать не изгнал их.
\par 29 И Ефрем не изгнал Хананеев, живущих в Газере; и жили Хананеи среди их в Газере.
\par 30 И Завулон не изгнал жителей Китрона и жителей Наглола, и жили Хананеи среди их и платили им дань.
\par 31 И Асир не изгнал жителей Акко и жителей Сидона и Ахлава, Ахзива, Хелвы, Афека и Рехова.
\par 32 И жил Асир среди Хананеев, жителей земли той, ибо не изгнал их.
\par 33 И Неффалим не изгнал жителей Вефсамиса и жителей Бефанафа и жил среди Хананеев, жителей земли той; жители же Вефсамиса и Бефанафа были его данниками.
\par 34 И стеснили Аморреи сынов Дановых в горах, ибо не давали им сходить на долину.
\par 35 И остались Аморреи жить на горе Херес, в Аиалоне и Шаалвиме; но рука сынов Иосифовых одолела [Аморреев], и сделались они данниками им.
\par 36 Пределы Аморреев от возвышенности Акравим и от Селы простирались и далее.

\chapter{2}

\par 1 И пришел Ангел Господень из Галгала в Бохим и сказал: Я вывел вас из Египта и ввел вас в землю, о которой клялся отцам вашим--[дать вам], и сказал Я: `не нарушу завета Моего с вами вовек;
\par 2 и вы не вступайте в союз с жителями земли сей; жертвенники их разрушьте'. Но вы не послушали гласа Моего. Что вы это сделали?
\par 3 И потому говорю Я: не изгоню их от вас, и будут они вам петлею, и боги их будут для вас сетью.
\par 4 Когда Ангел Господень сказал слова сии всем сынам Израилевым, то народ поднял громкий вопль и заплакал.
\par 5 От сего и называют то место Бохим. Там принесли они жертву Господу.
\par 6 Когда Иисус распустил народ, и пошли сыны Израилевы, каждый в свой удел, чтобы получить в наследие землю,
\par 7 тогда народ служил Господу во все дни Иисуса и во все дни старейшин, которых жизнь продлилась после Иисуса и которые видели все великие дела Господни, какие Он сделал Израилю.
\par 8 Но когда умер Иисус, сын Навин, раб Господень, будучи ста десяти лет,
\par 9 и похоронили его в пределе удела его в Фамнаф-Сараи, на горе Ефремовой, на север от горы Гааша;
\par 10 и когда весь народ оный отошел к отцам своим, и восстал после них другой род, который не знал Господа и дел Его, какие Он делал Израилю, --
\par 11 тогда сыны Израилевы стали делать злое пред очами Господа и стали служить Ваалам;
\par 12 оставили Господа Бога отцов своих, Который вывел их из земли Египетской, и обратились к другим богам, богам народов, окружавших их, и стали поклоняться им, и раздражили Господа;
\par 13 оставили Господа и стали служить Ваалу и Астартам.
\par 14 И воспылал гнев Господень на Израиля, и предал их в руки грабителей, и грабили их; и предал их в руки врагов, окружавших их, и не могли уже устоять пред врагами своими.
\par 15 Куда они ни пойдут, рука Господня везде была им во зло, как говорил им Господь и как клялся им Господь. И им было весьма тесно.
\par 16 И воздвигал [им] Господь судей, которые спасали их от рук грабителей их;
\par 17 но и судей они не слушали, а ходили блудно вслед других богов и поклонялись им, скоро уклонялись от пути, коим ходили отцы их, повинуясь заповедям Господним. Они так не делали.
\par 18 Когда Господь воздвигал им судей, то Сам Господь был с судьею и спасал их от врагов их во все дни судьи: ибо жалел [их] Господь, слыша стон их от угнетавших и притеснявших их.
\par 19 Но как скоро умирал судья, они опять делали хуже отцов своих, уклоняясь к другим богам, служа им и поклоняясь им. Не отставали от дел своих и от стропотного пути своего.
\par 20 И воспылал гнев Господень на Израиля, и сказал Он: за то, что народ сей преступает завет Мой, который Я поставил с отцами их, и не слушает гласа Моего,
\par 21 и Я не стану уже изгонять от них ни одного из тех народов, которых оставил Иисус, когда умирал, --
\par 22 чтобы искушать ими Израиля: станут ли они держаться пути Господня и ходить по нему, как держались отцы их, или нет?
\par 23 И оставил Господь народы сии и не изгнал их вскоре и не предал их в руки Иисуса.

\chapter{3}

\par 1 Вот те народы, которых оставил Господь, чтобы искушать ими Израильтян, всех, которые не знали о всех войнах Ханаанских, --
\par 2 для того только, чтобы знали и учились войне последующие роды сынов Израилевых, которые прежде не знали ее:
\par 3 пять владельцев Филистимских, все Хананеи, Сидоняне и Евеи, живущие на горе Ливане, от горы Ваал-Ермона до входа в Емаф.
\par 4 Они были [оставлены], чтобы искушать ими Израильтян и узнать, повинуются ли они заповедям Господним, которые Он заповедал отцам их чрез Моисея.
\par 5 И жили сыны Израилевы среди Хананеев, Хеттеев, Аморреев, Ферезеев, Евеев, и Иевусеев,
\par 6 и брали дочерей их себе в жены, и своих дочерей отдавали за сыновей их, и служили богам их.
\par 7 И сделали сыны Израилевы злое пред очами Господа, и забыли Господа Бога своего, и служили Ваалам и Астартам.
\par 8 И воспылал гнев Господень на Израиля, и предал их в руки Хусарсафема, царя Месопотамского, и служили сыны Израилевы Хусарсафему восемь лет.
\par 9 Тогда возопили сыны Израилевы к Господу, и воздвигнул Господь спасителя сынам Израилевым, который спас их, Гофониила, сына Кеназа, младшего брата Халевова.
\par 10 На нем был Дух Господень, и был он судьею Израиля. Он вышел на войну, и предал Господь в руки его Хусарсафема, царя Месопотамского, и преодолела рука его Хусарсафема.
\par 11 И покоилась земля сорок лет. И умер Гофониил, сын Кеназа.
\par 12 Сыны Израилевы опять стали делать злое пред очами Господа, и укрепил Господь Еглона, царя Моавитского, против Израильтян, за то, что они делали злое пред очами Господа.
\par 13 Он собрал к себе Аммонитян и Амаликитян, и пошел и поразил Израиля, и овладели они городом Пальм.
\par 14 И служили сыны Израилевы Еглону, царю Моавитскому, восемнадцать лет.
\par 15 Тогда возопили сыны Израилевы к Господу, и Господь воздвигнул им спасителя Аода, сына Геры, сына Иеминиева, который был левша. И послали сыны Израилевы с ним дары Еглону, царю Моавитскому.
\par 16 Аод сделал себе меч с двумя остриями, длиною в локоть, и припоясал его под плащом своим к правому бедру,
\par 17 и поднес дары Еглону, царю Моавитскому; Еглон же был человек очень тучный.
\par 18 Когда поднес [Аод] все дары и проводил людей, принесших дары,
\par 19 то сам возвратился от истуканов, которые в Галгале, и сказал: у меня есть тайное слово до тебя, царь. Он сказал: тише! И вышли от него все стоявшие при нем.
\par 20 Аод вошел к нему: он сидел в прохладной горнице, которая была у него отдельно. И сказал Аод: у меня есть до тебя, слово Божие. [Еглон] встал со стула.
\par 21 Аод простер левую руку свою и взял меч с правого бедра своего и вонзил его в чрево его,
\par 22 так что вошла за острием и рукоять, и тук закрыл острие, ибо Аод не вынул меча из чрева его, и он прошел в задние части.
\par 23 И вышел Аод в преддверие, и затворил за собою двери горницы, и замкнул.
\par 24 Когда он вышел, рабы [Еглона] пришли и видят, вот, двери горницы замкнуты, и говорят: верно он для нужды в прохладной комнате.
\par 25 Ждали довольно долго, но видя, что никто не отпирает дверей горницы, взяли ключ и отперли, и вот, господин их лежит на земле мертвый.
\par 26 Пока они недоумевали, Аод между тем ушел, прошел мимо истуканов и спасся в Сеираф.
\par 27 Придя же вострубил трубою на горе Ефремовой, и сошли с ним сыны Израилевы с горы, и он [шел] впереди их.
\par 28 И сказал им: идите за мною, ибо предал Господь врагов ваших Моавитян в руки ваши. И пошли за ним, и перехватили переправу через Иордан к Моаву, и не давали никому переходить.
\par 29 И побили в то время Моавитян около десяти тысяч человек, все здоровых и сильных, и никто не убежал.
\par 30 Так смирились в тот день Моавитяне пред Израилем, и покоилась земля восемьдесят лет.
\par 31 После него был Самегар, сын Анафов, который шестьсот человек Филистимлян побил воловьим рожном; и он также спас Израиля.

\chapter{4}

\par 1 Когда умер Аод, сыны Израилевы стали опять делать злое пред очами Господа.
\par 2 И предал их Господь в руки Иавина, царя Ханаанского, который царствовал в Асоре; военачальником у него был Сисара, который жил в Харошеф-Гоиме.
\par 3 И возопили сыны Израилевы к Господу, ибо у него было девятьсот железных колесниц, и он жестоко угнетал сынов Израилевых двадцать лет.
\par 4 В то время была судьею Израиля Девора пророчица, жена Лапидофова;
\par 5 она жила под Пальмою Девориною, между Рамою и Вефилем, на горе Ефремовой; и приходили к ней сыны Израилевы на суд.
\par 6 [Девора] послала и призвала Варака, сына Авиноамова, из Кедеса Неффалимова, и сказала ему: повелевает [тебе] Господь Бог Израилев: пойди, взойди на гору Фавор и возьми с собою десять тысяч человек из сынов Неффалимовых и сынов Завулоновых;
\par 7 а Я приведу к тебе, к потоку Киссону, Сисару, военачальника Иавинова, и колесницы его и многолюдное [войско] его, и предам его в руки твои.
\par 8 Варак сказал ей: если ты пойдешь со мною, пойду; а если не пойдешь со мною, не пойду.
\par 9 Она сказала [ему]: пойти пойду с тобою; только не тебе уже будет слава на сем пути, в который ты идешь; но в руки женщины предаст Господь Сисару. И встала Девора и пошла с Вараком в Кедес.
\par 10 Варак созвал Завулонян и Неффалимлян в Кедес, и пошли вслед за ним десять тысяч человек, и Девора пошла с ним.
\par 11 Хевер Кенеянин отделился [тогда] от Кенеян, сынов Ховава, родственника Моисеева, и раскинул шатер свой у дубравы в Цаанниме близ Кедеса.
\par 12 И донесли Сисаре, что Варак, сын Авиноамов, взошел на гору Фавор.
\par 13 Сисара созвал все колесницы свои, девятьсот железных колесниц, и весь народ, который у него, из Харошеф-Гоима к потоку Киссону.
\par 14 И сказала Девора Вараку: встань, ибо это тот день, в который Господь предаст Сисару в руки твои; Сам Господь пойдет пред тобою. И сошел Варак с горы Фавора, и за ним десять тысяч человек.
\par 15 Тогда Господь привел в замешательство Сисару и все колесницы его и все ополчение его от меча Варакова, и сошел Сисара с колесницы и побежал пеший.
\par 16 Варак преследовал колесницы [его] и ополчение до Харошеф--Гоима, и пало все ополчение Сисарино от меча, не осталось никого.
\par 17 Сисара же убежал пеший в шатер Иаили, жены Хевера Кенеянина; ибо между Иавином, царем Асорским, и домом Хевера Кенеянина был мир.
\par 18 И вышла Иаиль навстречу Сисаре и сказала ему: зайди, господин мой, зайди ко мне, не бойся. Он зашел к ней в шатер, и она покрыла его ковром.
\par 19 [Сисара] сказал ей: дай мне немного воды напиться, я пить хочу. Она развязала мех с молоком, и напоила его и [опять] покрыла его.
\par 20 [Сисара] сказал ей: стань у дверей шатра, и если кто придет и спросит у тебя и скажет: `нет ли здесь кого?', ты скажи: `нет'.
\par 21 Иаиль, жена Хеверова, взяла кол от шатра, и взяла молот в руку свою, и подошла к нему тихонько, и вонзила кол в висок его так, что приколола к земле; а он спал от усталости--и умер.
\par 22 И вот, Варак гонится за Сисарою. Иаиль вышла навстречу ему и сказала ему: войди, я покажу тебе человека, которого ты ищешь. Он вошел к ней, и вот, Сисара лежит мертвый, и кол в виске его.
\par 23 И смирил Бог в тот день Иавина, царя Ханаанского, пред сынами Израилевыми.
\par 24 Рука сынов Израилевых усиливалась более и более над Иавином, царем Ханаанским, доколе не истребили они Иавина, царя Ханаанского.

\chapter{5}

\par 1 В тот день воспела Девора и Варак, сын Авиноамов, сими словами:
\par 2 Израиль отмщен, народ показал рвение; прославьте Господа!
\par 3 Слушайте, цари, внимайте, вельможи: я Господу, я пою, бряцаю Господу Богу Израилеву.
\par 4 Когда выходил Ты, Господи, от Сеира, когда шел с поля Едомского, тогда земля тряслась, и небо капало, и облака проливали воду;
\par 5 горы таяли от лица Господа, даже этот Синай от лица Господа Бога Израилева.
\par 6 Во дни Самегара, сына Анафова, во дни Иаили, были пусты дороги, и ходившие прежде путями прямыми ходили тогда окольными дорогами.
\par 7 Не стало обитателей в селениях у Израиля, не стало, доколе не восстала я, Девора, доколе не восстала я, мать в Израиле.
\par 8 Избрали новых богов, оттого война у ворот. Виден ли был щит и копье у сорока тысяч Израиля?
\par 9 Сердце мое к вам, начальники Израилевы, к ревнителям в народе; прославьте Господа!
\par 10 Ездящие на ослицах белых, сидящие на коврах и ходящие по дороге, пойте песнь!
\par 11 Среди голосов собирающих стада при колодезях, там да воспоют хвалу Господу, хвалу вождям Израиля! Тогда выступил ко вратам народ Господень.
\par 12 Воспряни, воспряни, Девора! воспряни, воспряни! воспой песнь! Восстань, Варак! и веди пленников твоих, сын Авиноамов!
\par 13 Тогда немногим из сильных подчинил Он народ; Господь подчинил мне храбрых.
\par 14 От Ефрема пришли укоренившиеся в земле Амалика; за тобою Вениамин, среди народа твоего; от Махира шли начальники, и от Завулона владеющие тростью писца.
\par 15 И князья Иссахаровы с Деворою, и Иссахар так же, как Варак, бросился в долину пеший. В племенах Рувимовых большое разногласие.
\par 16 Что сидишь ты между овчарнями, слушая блеяние стад? В племенах Рувимовых большое разногласие.
\par 17 Галаад живет [спокойно] за Иорданом, и Дану чего бояться с кораблями? Асир сидит на берегу моря и у пристаней своих живет спокойно.
\par 18 Завулон--народ, обрекший душу свою на смерть, и Неффалим--на высотах поля.
\par 19 Пришли цари, сразились, тогда сразились цари Ханаанские в Фанаахе у вод Мегиддонских, но не получили нимало серебра.
\par 20 С неба сражались, звезды с путей своих сражались с Сисарою.
\par 21 Поток Киссон увлек их, поток Кедумим, поток Киссон. Попирай, душа моя, силу!
\par 22 Тогда ломались копыта конские от побега, от побега сильных его.
\par 23 Прокляните Мероз, говорит Ангел Господень, прокляните, прокляните жителей его за то, что не пришли на помощь Господу, на помощь Господу с храбрыми.
\par 24 Да будет благословенна между женами Иаиль, жена Хевера Кенеянина, между женами в шатрах да будет благословенна!
\par 25 Воды просил он: молока подала она, в чаше вельможеской принесла молока лучшего.
\par 26 [Левую] руку свою протянула к колу, а правую свою к молоту работников; ударила Сисару, поразила голову его, разбила и пронзила висок его.
\par 27 К ногам ее склонился, пал и лежал, к ногам ее склонился, пал; где склонился, там и пал сраженный.
\par 28 В окно выглядывает и вопит мать Сисарина сквозь решетку: что долго не идет конница его, что медлят колеса колесниц его?
\par 29 Умные из ее женщин отвечают ей, и сама она отвечает на слова свои:
\par 30 верно, они нашли, делят добычу, по девице, по две девицы на каждого воина, в добычу полученная разноцветная [одежда] Сисаре, полученная в добычу разноцветная одежда, вышитая с обеих сторон, снятая с плеч пленника.
\par 31 Так да погибнут все враги Твои, Господи! Любящие же Его [да] [будут] как солнце, восходящее во всей силе своей! --И покоилась земля сорок лет.

\chapter{6}

\par 1 Сыны Израилевы стали [опять] делать злое пред очами Господа, и предал их Господь в руки Мадианитян на семь лет.
\par 2 Тяжела была рука Мадианитян над Израилем, и сыны Израилевы сделали себе от Мадианитян ущелья в горах и пещеры и укрепления.
\par 3 Когда посеет Израиль, придут Мадианитяне и Амаликитяне и жители востока и ходят у них;
\par 4 и стоят у них шатрами, и истребляют произведения земли до самой Газы, и не оставляют для пропитания Израилю ни овцы, ни вола, ни осла.
\par 5 Ибо они приходили со скотом своим и с шатрами своими, приходили в таком множестве, как саранча; им и верблюдам их не было числа, и ходили по земле Израилевой, чтоб опустошать ее.
\par 6 И весьма обнищал Израиль от Мадианитян, и возопили сыны Израилевы к Господу.
\par 7 И когда возопили сыны Израилевы к Господу на Мадианитян,
\par 8 послал Господь пророка к сынам Израилевым, и сказал им: так говорит Господь Бог Израилев: Я вывел вас из Египта, вывел вас из дома рабства;
\par 9 избавил вас из руки Египтян и из руки всех, угнетавших вас, прогнал их от вас, и дал вам землю их,
\par 10 и сказал вам: `Я--Господь Бог ваш; не чтите богов Аморрейских, в земле которых вы живете'; но вы не послушали гласа Моего.
\par 11 И пришел Ангел Господень и сел в Офре под дубом, принадлежащим Иоасу, потомку Авиезерову; сын его Гедеон выколачивал тогда пшеницу в точиле, чтобы скрыться от Мадианитян.
\par 12 И явился ему Ангел Господень и сказал ему: Господь с тобою, муж сильный!
\par 13 Гедеон сказал ему: господин мой! если Господь с нами, то отчего постигло нас все это? и где все чудеса Его, о которых рассказывали нам отцы наши, говоря: `из Египта вывел нас Господь'? Ныне оставил нас Господь и предал нас в руки Мадианитян.
\par 14 Господь, воззрев на него, сказал: иди с этою силою твоею и спаси Израиля от руки Мадианитян; Я посылаю тебя.
\par 15 [Гедеон] сказал ему: Господи! как спасу я Израиля? вот, и племя мое в [колене] Манассиином самое бедное, и я в доме отца моего младший.
\par 16 И сказал ему Господь: Я буду с тобою, и ты поразишь Мадианитян, как одного человека.
\par 17 [Гедеон] сказал Ему: если я обрел благодать пред очами Твоими, то сделай мне знамение, что Ты говоришь со мною:
\par 18 не уходи отсюда, доколе я не приду к Тебе и не принесу дара моего и не предложу Тебе. Он сказал: Я останусь до возвращения твоего.
\par 19 Гедеон пошел и приготовил козленка и опресноков из ефы муки; мясо положил в корзину, а похлебку влил в горшок и принес к Нему под дуб и предложил.
\par 20 И сказал ему Ангел Божий: возьми мясо и опресноки, и положи на сей камень, и вылей похлебку. Он так и сделал.
\par 21 Ангел Господень простер конец жезла, который был в руке его, прикоснулся к мясу и опреснокам; и вышел огонь из камня и поел мясо и опресноки; и Ангел Господень скрылся от глаз его.
\par 22 И увидел Гедеон, что это Ангел Господень, и сказал Гедеон: [увы] [мне], Владыка Господи! потому что я видел Ангела Господня лицем к лицу.
\par 23 Господь сказал ему: мир тебе, не бойся, не умрешь.
\par 24 И устроил там Гедеон жертвенник Господу и назвал его: Иегова Шалом. Он еще до сего дня в Офре Авиезеровой.
\par 25 В ту ночь сказал ему Господь: возьми тельца из стада отца твоего и другого тельца семилетнего, и разрушь жертвенник Ваала, который у отца твоего, и сруби священное дерево, которое при нем,
\par 26 и поставь жертвенник Господу Богу твоему, на вершине скалы сей, в порядке, и возьми второго тельца и принеси во всесожжение на дровах дерева, которое срубишь.
\par 27 Гедеон взял десять человек из рабов своих и сделал, как говорил ему Господь; но как сделать это днем он боялся домашних отца своего и жителей города, то сделал ночью.
\par 28 Поутру встали жители города, и вот, жертвенник Ваалов разрушен, и дерево при нем срублено, и второй телец вознесен во всесожжение на новоустроенном жертвеннике.
\par 29 И говорили друг другу: кто это сделал? Искали, расспрашивали и сказали: Гедеон, сын Иоасов, сделал это.
\par 30 И сказали жители города Иоасу: выведи сына твоего; он должен умереть за то, что разрушил жертвенник Ваала и срубил дерево, которое было при нем.
\par 31 Иоас сказал всем приступившим к нему: вам ли вступаться за Ваала, вам ли защищать его? кто вступится за него, тот будет предан смерти в это же утро; если он Бог, то пусть сам вступится за себя, потому что он разрушил его жертвенник.
\par 32 И стал звать его с того дня Иероваалом, потому что сказал: пусть Ваал сам судится с ним за то, что он разрушил жертвенник его.
\par 33 Между тем все Мадианитяне и Амаликитяне и жители востока собрались вместе, перешли [реку] и стали станом на долине Изреельской.
\par 34 И Дух Господень объял Гедеона; он вострубил трубою, и созвано было племя Авиезерово идти за ним.
\par 35 И послал послов по всему колену Манассиину, и оно вызвалось идти за ним; также послал послов к Асиру, Завулону и Неффалиму, и сии пришли навстречу им.
\par 36 И сказал Гедеон Богу: если Ты спасешь Израиля рукою моею, как говорил Ты,
\par 37 то вот, я расстелю [здесь] на гумне стриженую шерсть: если роса будет только на шерсти, а на всей земле сухо, то буду знать, что спасешь рукою моею Израиля, как говорил Ты.
\par 38 Так и сделалось: на другой день, встав рано, он стал выжимать шерсть и выжал из шерсти росы целую чашу воды.
\par 39 И сказал Гедеон Богу: не прогневайся на меня, если еще раз скажу и еще только однажды сделаю испытание над шерстью: пусть будет сухо на одной только шерсти, а на всей земле пусть будет роса.
\par 40 Бог так и сделал в ту ночь: только на шерсти было сухо, а на всей земле была роса.

\chapter{7}

\par 1 Иероваал, он же и Гедеон, встал поутру и весь народ, бывший с ним, и расположились станом у источника Харода; Мадиамский же стан был от него к северу у холма Море в долине.
\par 2 И сказал Господь Гедеону: народа с тобою слишком много, не могу Я предать Мадианитян в руки их, чтобы не возгордился Израиль предо Мною и не сказал: `моя рука спасла меня';
\par 3 итак провозгласи вслух народа и скажи: `кто боязлив и робок, тот пусть возвратится и пойдет назад с горы Галаада'. И возвратилось народа двадцать две тысячи, а десять тысяч осталось.
\par 4 И сказал Господь Гедеону: все еще много народа; веди их к воде, там Я выберу их тебе; о ком Я скажу: `пусть идет с тобою', тот и пусть идет с тобою; а о ком скажу тебе: `не должен идти с тобою', тот пусть и не идет.
\par 5 Он привел народ к воде. И сказал Господь Гедеону: кто будет лакать воду языком своим, как лакает пес, того ставь особо, также и тех всех, которые будут наклоняться на колени свои и пить.
\par 6 И было число лакавших ртом своим с руки триста человек; весь же остальной народ наклонялся на колени свои пить воду.
\par 7 И сказал Господь Гедеону: тремя стами лакавших Я спасу вас и предам Мадианитян в руки ваши, а весь народ пусть идет, каждый в свое место.
\par 8 И взяли они съестной запас у народа себе и трубы их, и отпустил Гедеон всех Израильтян по шатрам и удержал у себя триста человек; стан же Мадиамский был у него внизу в долине.
\par 9 В ту ночь сказал ему Господь: встань, сойди в стан, Я предаю его в руки твои;
\par 10 если же ты боишься идти [один], то пойди в стан ты и Фура, слуга твой;
\par 11 и услышишь, что говорят, и тогда укрепятся руки твои, и пойдешь в стан. И сошел он и Фура, слуга его, к самому [полку] вооруженных, которые были в стане.
\par 12 Мадианитяне же и Амаликитяне и все жители востока расположились на долине в таком множестве, как саранча; верблюдам их не было числа, много было их, как песку на берегу моря.
\par 13 Гедеон пришел. И вот, один рассказывает другому сон и говорит: снилось мне, будто круглый ячменный хлеб катился по стану Мадиамскому и, прикатившись к шатру, ударил в него так, что он упал, опрокинул его, и шатер распался.
\par 14 Другой сказал в ответ ему: это не иное что, как меч Гедеона, сына Иоасова, Израильтянина; предал Бог в руки его Мадианитян и весь стан.
\par 15 Гедеон, услышав рассказ сна и толкование его, поклонился [Господу] и возвратился в стан Израильский и сказал: вставайте! предал Господь в руки ваши стан Мадиамский.
\par 16 И разделил триста человек на три отряда и дал в руки всем им трубы и пустые кувшины и в кувшины светильники.
\par 17 И сказал им: смотрите на меня и делайте то же; вот, я подойду к стану, и что буду делать, то и вы делайте;
\par 18 когда я и находящиеся со мною затрубим трубою, трубите и вы трубами вашими вокруг всего стана и кричите: [меч] Господа и Гедеона!
\par 19 И подошел Гедеон и сто человек с ним к стану, в начале средней стражи, и разбудили стражей, и затрубили трубами и разбили кувшины, которые были в руках их.
\par 20 И затрубили [все] три отряда трубами, и разбили кувшины, и держали в левой руке своей светильники, а в правой руке трубы, и трубили, и кричали: меч Господа и Гедеона!
\par 21 И стоял всякий на своем месте вокруг стана; и стали бегать во всем стане, и кричали, и обратились в бегство.
\par 22 Между тем как триста человек трубили трубами, обратил Господь меч одного на другого во всем стане, и бежало ополчение до Бефшитты к Царере, до предела Авелмехолы, близ Табафы.
\par 23 И созваны Израильтяне из колена Неффалимова, Асирова и всего колена Манассиина, и погнались за Мадианитянами.
\par 24 Гедеон же послал послов на всю гору Ефремову сказать: выйдите навстречу Мадианитянам и перехватите у них [переправу через] воду до Бефвары и Иордан. И созваны все Ефремляне и перехватили [переправы] [через] воду до Бефвары и Иордан;
\par 25 и поймали двух князей Мадиамских: Орива и Зива, и убили Орива в Цур-Ориве, а Зива в Иекев-Зиве и преследовали Мадианитян; головы же Орива и Зива принесли к Гедеону за Иордан.

\chapter{8}

\par 1 И сказали ему Ефремляне: зачем ты это сделал, что не позвал нас, когда шел воевать с Мадианитянами? И сильно ссорились с ним.
\par 2 [Гедеон] отвечал им: сделал ли я что такое, как вы ныне? Не счастливее ли Ефрем добирал виноград, нежели Авиезер обирал?
\par 3 В ваши руки предал Бог князей Мадиамских Орива и Зива, и что мог сделать я такое, как вы? Тогда успокоился дух их против него, когда сказал он им такие слова.
\par 4 И пришел Гедеон к Иордану, и перешел сам и триста человек, бывшие с ним. Они были утомлены, преследуя [врагов].
\par 5 И сказал он жителям Сокхофа: дайте хлеба народу, который идет за мною; они утомились, а я преследую Зевея и Салмана, царей Мадиамских.
\par 6 Князья Сокхофа сказали: разве рука Зевея и Салмана уже в твоей руке, чтобы нам войску твоему давать хлеб?
\par 7 И сказал Гедеон: за это, когда предаст Господь Зевея и Салмана в руки мои, я растерзаю тело ваше терновником пустынным и молотильными зубчатыми досками.
\par 8 Оттуда пошел он в Пенуэл и то же сказал жителям его, и жители Пенуэла отвечали ему то же, что отвечали жители Сокхофа.
\par 9 Он сказал и жителям Пенуэла: когда я возвращусь в мире, разрушу башню сию.
\par 10 Зевей же и Салман были в Каркоре и с ними их ополчение до пятнадцати тысяч, все, что осталось из всего ополчения жителей востока; пало же сто двадцать тысяч человек, обнажающих меч.
\par 11 Гедеон пошел к живущим в шатрах на восток от Новы и Иогбеги и поразил стан, когда стан стоял беспечно.
\par 12 Зевей и Салман побежали; он погнался за ними и схватил обоих царей Мадиамских, Зевея и Салмана, и весь стан привел в замешательство.
\par 13 И возвратился Гедеон, сын Иоаса, с войны от возвышенности Хереса.
\par 14 И захватил юношу из жителей Сокхофа и выспросил у него; и он написал ему князей и старейшин Сокхофских семьдесят семь человек.
\par 15 И пришел он к жителям Сокхофским, и сказал: вот Зевей и Салман, за которых вы посмеялись надо мною, говоря: разве рука Зевея и Салмана уже в твоей руке, чтобы нам давать хлеб утомившимся людям твоим?
\par 16 И взял старейшин города и терновник пустынный и зубчатые молотильные доски и наказал ими жителей Сокхофа;
\par 17 и башню Пенуэльскую разрушил, и перебил жителей города.
\par 18 И сказал Зевею и Салману: каковы были те, которых вы убили на Фаворе? Они сказали: они были такие, как ты, каждый имел вид сынов царских.
\par 19 [Гедеон] сказал: это были братья мои, сыны матери моей. Жив Господь! если бы вы оставили их в живых, я не убил бы вас.
\par 20 И сказал Иеферу, первенцу своему: встань, убей их. Но юноша не извлек меча своего, потому что боялся, так как был еще молод.
\par 21 И сказали Зевей и Салман: встань сам и порази нас, потому что по человеку и сила его. И встал Гедеон, и убил Зевея и Салмана, и взял пряжки, бывшие на шеях верблюдов их.
\par 22 И сказали Израильтяне Гедеону: владей нами ты и сын твой и сын сына твоего, ибо ты спас нас из руки Мадианитян.
\par 23 Гедеон сказал им: ни я не буду владеть вами, ни мой сын не будет владеть вами; Господь да владеет вами.
\par 24 И сказал им Гедеон: прошу у вас одного, дайте мне каждый по серьге из добычи своей. (Ибо у [неприятелей] много было золотых серег, потому что они были Измаильтяне.)
\par 25 Они сказали: дадим. И разостлали одежду и бросали туда каждый по серьге из добычи своей.
\par 26 Весу в золотых серьгах, которые он выпросил, было тысяча семьсот золотых [сиклей], кроме пряжек, пуговиц и пурпуровых одежд, которые были на царях Мадиамских, и кроме [золотых] цепочек, которые были на шее у верблюдов их.
\par 27 Из этого сделал Гедеон ефод и положил его в своем городе, в Офре, и стали все Израильтяне блудно ходить туда за ним, и был он сетью Гедеону и всему дому его.
\par 28 Так смирились Мадианитяне пред сынами Израиля и не стали уже поднимать головы своей, и покоилась земля сорок лет во дни Гедеона.
\par 29 И пошел Иероваал, сын Иоасов, и жил в доме своем.
\par 30 У Гедеона было семьдесят сыновей, происшедших от чресл его, потому что у него много было жен.
\par 31 Также и наложница, жившая в Сихеме, родила ему сына, и он дал ему имя Авимелех.
\par 32 И умер Гедеон, сын Иоасов, в глубокой старости, и погребен во гробе отца своего Иоаса, в Офре Авиезеровой.
\par 33 Когда умер Гедеон, сыны Израилевы опять стали блудно ходить вслед Ваалов и поставили себе богом Ваалверифа;
\par 34 и не вспомнили сыны Израилевы Господа Бога своего, Который избавлял их из руки всех врагов, окружавших их;
\par 35 и дому Иероваалову, [или] Гедеонову, не сделали милости за все благодеяния, какие он сделал Израилю.

\chapter{9}

\par 1 Авимелех, сын Иероваалов, пошел в Сихем к братьям матери своей и говорил им и всему племени отца матери своей, и сказал:
\par 2 внушите всем жителям Сихемским: что лучше для вас, чтобы владели вами все семьдесят сынов Иеровааловых, или чтобы владел один? и вспомните, что я кость ваша и плоть ваша.
\par 3 Братья матери его внушили о нем все сии слова жителям Сихемским; и склонилось сердце их к Авимелеху, ибо говорили они: он брат наш.
\par 4 И дали ему семьдесят [сиклей] серебра из дома Ваалверифа; Авимелех нанял на оные праздных и своевольных людей, которые и пошли за ним.
\par 5 И пришел он в дом отца своего в Офру и убил братьев своих, семьдесят сынов Иеровааловых, на одном камне. Остался только Иофам, младший сын Иероваалов, потому что скрылся.
\par 6 И собрались все жители Сихемские и весь дом Милло, и пошли и поставили царем Авимелеха у дуба, что близ Сихема.
\par 7 Когда рассказали об этом Иофаму, он пошел и стал на вершине горы Гаризима и, возвысив голос свой, кричал и говорил им: послушайте меня, жители Сихема, и послушает вас Бог!
\par 8 Пошли некогда дерева помазать над собою царя и сказали маслине: царствуй над нами.
\par 9 Маслина сказала им: оставлю ли я тук мой, которым чествуют богов и людей и пойду ли скитаться по деревам?
\par 10 И сказали дерева смоковнице: иди ты, царствуй над нами.
\par 11 Смоковница сказала им: оставлю ли я сладость мою и хороший плод мой и пойду ли скитаться по деревам?
\par 12 И сказали дерева виноградной лозе: иди ты, царствуй над нами.
\par 13 Виноградная лоза сказала им: оставлю ли я сок мой, который веселит богов и человеков, и пойду ли скитаться по деревам?
\par 14 Наконец сказали все дерева терновнику: иди ты, царствуй над нами.
\par 15 Терновник сказал деревам: если вы по истине поставляете меня царем над собою, то идите, покойтесь под тенью моею; если же нет, то выйдет огонь из терновника и пожжет кедры Ливанские.
\par 16 Итак смотрите, по истине ли и по правде ли вы поступили, поставив Авимелеха царем? И хорошо ли вы поступили с Иероваалом и домом его, и сообразно ли с его благодеяниями поступили вы?
\par 17 За вас отец мой сражался, не дорожил жизнью своею и избавил вас от руки Мадианитян;
\par 18 а вы теперь восстали против дома отца моего, и убили семьдесят сынов отца моего на одном камне, и поставили царем над жителями Сихемскими Авимелеха, сына рабыни его, потому что он брат ваш.
\par 19 Если вы ныне по истине и по правде поступили с Иероваалом и домом его, то радуйтесь об Авимелехе, и он пусть радуется о вас;
\par 20 если же нет, то да изыдет огонь от Авимелеха и да пожжет жителей Сихемских и весь дом Милло и да изыдет огонь от жителей Сихемских и от дома Милло, и да пожжет Авимелеха.
\par 21 И побежал Иофам, и убежал и пошел в Беэр, и жил там, [укрываясь] от брата своего Авимелеха.
\par 22 Авимелех же царствовал над Израилем три года.
\par 23 И послал Бог злого духа между Авимелехом и между жителями Сихема, и не стали покоряться жители Сихемские Авимелеху,
\par 24 дабы таким образом совершилось мщение за семьдесят сынов Иеровааловых, и кровь их обратилась на Авимелеха, брата их, который убил их, и на жителей Сихемских, которые подкрепили руки его, чтоб убить братьев своих.
\par 25 Жители Сихемские посадили против него в засаду людей на вершинах гор, которые грабили всякого проходящего мимо их по дороге. О сем донесено было Авимелеху.
\par 26 Пришел же и Гаал, сын Еведов, с братьями своими в Сихем, и ходили они по Сихему, и жители Сихемские положились на него.
\par 27 И вышли в поле, и собирали виноград свой, и давили в точилах, и делали праздники, ходили в дом бога своего, и ели и пили, и проклинали Авимелеха.
\par 28 Гаал, сын Еведов, говорил: кто Авимелех и что Сихем, чтобы нам служить ему? Не сын ли он Иероваалов, и не Зевул ли главный начальник его? Служите лучше потомкам Еммора, отца Сихемова, а ему для чего нам служить?
\par 29 Если бы кто дал народ сей в руки мои, я прогнал бы Авимелеха. И сказано было Авимелеху: умножь войско твое и выходи.
\par 30 Зевул, начальник города, услышал слова Гаала, сына Еведова, и воспылал гнев его.
\par 31 Он хитрым образом отправляет послов к Авимелеху, чтобы сказать: вот, Гаал, сын Еведов, и братья его пришли в Сихем, и вот, они возмущают против тебя город;
\par 32 итак, встань ночью, ты и народ, находящийся с тобою, и поставь засаду в поле;
\par 33 поутру же, при восхождении солнца, встань рано и приступи к городу; и когда он и народ, который у него, выйдут к тебе, тогда делай с ними, что может рука твоя.
\par 34 И встал ночью Авимелех и весь народ, находившийся с ним, и поставили в засаду у Сихема четыре отряда.
\par 35 Гаал, сын Еведов, вышел и стал у ворот городских; и встал Авимелех и народ, бывший с ним, из засады.
\par 36 Гаал, увидев народ, говорит Зевулу: вот, народ спускается с вершины гор. А Зевул сказал ему: тень гор тебе кажется людьми.
\par 37 Гаал опять говорил и сказал: вот, народ спускается с возвышенности, и один отряд идет от дуба Меонним.
\par 38 И сказал ему Зевул: где уста твои, которые говорили: `кто Авимелех, чтобы мы стали служить ему?' Это тот народ, который ты пренебрегал; выходи теперь и сразись с ним.
\par 39 И пошел Гаал впереди жителей Сихемских и сразился с Авимелехом.
\par 40 И погнался за ним Авимелех, и побежал он от него, и много пало убитых до самых ворот города.
\par 41 И остался Авимелех в Аруме, а Гаала и братьев его Зевул выгнал, чтоб они не жили в Сихеме.
\par 42 На другой день вышел народ в поле, и донесли о сем Авимелеху.
\par 43 Он взял свой народ и разделил его на три отряда и поставил в засаду в поле. И увидев, что народ вышел из города, восстал на них и побил их.
\par 44 Между тем как Авимелех и отряды, бывшие с ним, приступили и стали у ворот городских, другие два отряда напали на всех, бывших в поле, и убивали их.
\par 45 И сражался Авимелех с городом весь тот день, и взял город, и побил народ, бывший в нем, и разрушил город и засеял его солью.
\par 46 Услышав об этом, все бывшие в башне Сихемской ушли в башню капища [Ваал-Верифа].
\par 47 Авимелеху донесено, что собрались [туда] все бывшие в башне Сихемской.
\par 48 И пошел Авимелех на гору Селмон, сам и весь народ, бывший с ним, и взял Авимелех топоры с собою и нарубил сучьев древесных, и положил на плечи свои, и сказал народу, бывшему с ним: вы видели, что я делал; скорее делайте и вы то же, что я.
\par 49 И нарубил каждый из всего народа сучьев, и пошли за Авимелехом, и положили к башне, и сожгли посредством их башню огнем, и умерли все бывшие в башне Сихемской, около тысячи мужчин и женщин.
\par 50 Потом пошел Авимелех в Тевец и осадил Тевец и взял его.
\par 51 Среди города была крепкая башня, и убежали туда все мужчины и женщины и все жители города, и заперлись и взошли на кровлю башни.
\par 52 Авимелех пришел к башне и окружил ее и подошел к дверям башни, чтобы сжечь ее огнем.
\par 53 Тогда одна женщина бросила обломок жернова на голову Авимелеху и проломила ему череп.
\par 54 [Авимелех] тотчас призвал отрока, оруженосца своего, и сказал ему: обнажи меч твой и умертви меня, чтобы не сказали обо мне: женщина убила его. И пронзил его отрок его, и он умер.
\par 55 Израильтяне, видя, что умер Авимелех, пошли каждый в свое место.
\par 56 Так воздал Бог Авимелеху за злодеяние, которое он сделал отцу своему, убив семьдесят братьев своих.
\par 57 И все злодеяния жителей Сихемских обратил Бог на голову их; и постигло их проклятие Иофама, сына Иероваалова.

\chapter{10}

\par 1 После Авимелеха восстал для спасения Израиля Фола, сын Фуи, сына Додова, из колена Иссахарова. Он жил в Шамире на горе Ефремовой.
\par 2 Он был судьею Израиля двадцать три года, и умер, и погребен в Шамире.
\par 3 После него восстал Иаир из Галаада и был судьею Израиля двадцать два года.
\par 4 У него было тридцать сына, ездивших на тридцати молодых ослах, и тридцать города было у них; их до сего дня называют селениями Иаира, что в земле Галаадской.
\par 5 И умер Иаир и погребен в Камоне.
\par 6 Сыны Израилевы продолжали делать злое пред очами Господа и служили Ваалам и Астартам, и богам Арамейским, и богам Сидонским, и богам Моавитским, и богам Аммонитским, и богам Филистимским; а Господа оставили и не служили Ему.
\par 7 И воспылал гнев Господа на Израиля, и Он предал их в руки Филистимлян и в руки Аммонитян;
\par 8 они теснили и мучили сынов Израилевых с того года восемнадцать лет, всех сынов Израилевых по ту сторону Иордана в земле Аморрейской, которая в Галааде.
\par 9 Наконец Аммонитяне перешли Иордан, чтобы вести войну с Иудою и Вениамином и с домом Ефремовым. И весьма тесно было сынам Израиля.
\par 10 И возопили сыны Израилевы к Господу, и говорили: согрешили мы пред Тобою, потому что оставили Бога нашего и служили Ваалам.
\par 11 И сказал Господь сынам Израилевым: не угнетали ли вас Египтяне, и Аморреи, и Аммонитяне, и Филистимляне,
\par 12 и Сидоняне, и Амаликитяне, и Моавитяне, и когда вы взывали ко Мне, не спасал ли Я вас от рук их?
\par 13 А вы оставили Меня и стали служить другим богам; за то Я не буду уже спасать вас:
\par 14 пойдите, взывайте к богам, которых вы избрали, пусть они спасают вас в тесное для вас время.
\par 15 И сказали сыны Израилевы Господу: согрешили мы; делай с нами все, что Тебе угодно, только избавь нас ныне.
\par 16 И отвергли от себя чужих богов и стали служить Господу. И не потерпела душа Его страдания Израилева.
\par 17 Аммонитяне собрались и расположились станом в Галааде; собрались также сыны Израилевы и стали станом в Массифе.
\par 18 Народ [и] князья Галаадские сказали друг другу: кто начнет войну против Аммонитян, тот будет начальником всех жителей Галаадских.

\chapter{11}

\par 1 Иеффай Галаадитянин был человек храбрый. Он был сын блудницы; от Галаада родился Иеффай.
\par 2 И жена Галаадова родила ему сыновей. Когда возмужали сыновья жены, изгнали они Иеффая, сказав ему: ты не наследник в доме отца нашего, потому что ты сын другой женщины.
\par 3 И убежал Иеффай от братьев своих и жил в земле Тов; и собрались к Иеффаю праздные люди и выходили с ним.
\par 4 Чрез несколько времени Аммонитяне пошли войною на Израиля.
\par 5 Во время войны Аммонитян с Израильтянами пришли старейшины Галаадские взять Иеффая из земли Тов
\par 6 и сказали Иеффаю: приди, будь у нас вождем, и сразимся с Аммонитянами.
\par 7 Иеффай сказал старейшинам Галаадским: не вы ли возненавидели меня и выгнали из дома отца моего? зачем же пришли ко мне ныне, когда вы в беде?
\par 8 Старейшины Галаадские сказали Иеффаю: для того мы теперь пришли к тебе, чтобы ты пошел с нами и сразился с Аммонитянами и был у нас начальником всех жителей Галаадских.
\par 9 И сказал Иеффай старейшинам Галаадским: если вы возвратите меня, чтобы сразиться с Аммонитянами, и Господь предаст мне их, то останусь ли я у вас начальником?
\par 10 Старейшины Галаадские сказали Иеффаю: Господь да будет свидетелем между нами, что мы сделаем по слову твоему!
\par 11 И пошел Иеффай со старейшинами Галаадскими, и народ поставил его над собою начальником и вождем, и Иеффай произнес все слова свои пред лицем Господа в Массифе.
\par 12 И послал Иеффай послов к царю Аммонитскому сказать: что тебе до меня, что ты пришел ко мне воевать на земле моей?
\par 13 Царь Аммонитский сказал послам Иеффая: Израиль, когда шел из Египта, взял землю мою от Арнона до Иавока и Иордана; итак возврати мне ее с миром.
\par 14 Иеффай в другой раз послал послов к царю Аммонитскому,
\par 15 сказать ему: так говорит Иеффай: Израиль не взял земли Моавитской и земли Аммонитской;
\par 16 ибо когда шли из Египта, Израиль пошел в пустыню к Чермному морю и пришел в Кадес;
\par 17 оттуда послал Израиль послов к царю Едомскому сказать: `позволь мне пройти землею твоею'; но царь Едомский не послушал; и к царю Моавитскому он посылал, но и тот не согласился; посему Израиль оставался в Кадесе.
\par 18 И пошел пустынею, и миновал землю Едомскую и землю Моавитскую, и, придя к восточному пределу земли Моавитской, расположился станом за Арноном; но не входил в пределы Моавитские, ибо Арнон есть предел Моава.
\par 19 И послал Израиль послов к Сигону, царю Аморрейскому, царю Есевонскому, и сказал ему Израиль: позволь нам пройти землею твоею в свое место.
\par 20 Но Сигон не согласился пропустить Израиля чрез пределы свои, и собрал Сигон весь народ свой, и расположился станом в Иааце, и сразился с Израилем.
\par 21 И предал Господь Бог Израилев Сигона и весь народ его в руки Израилю, и он побил их; и получил Израиль в наследие всю землю Аморрея, жившего в земле той;
\par 22 и получили они в наследие все пределы Аморрея от Арнона до Иавока и от пустыни до Иордана.
\par 23 Итак Господь Бог Израилев изгнал Аморрея от лица народа Своего Израиля, а ты хочешь взять его наследие?
\par 24 Не владеешь ли ты тем, что дал тебе Хамос, бог твой? И мы владеем всем тем, что дал нам в наследие Господь Бог наш.
\par 25 Разве ты лучше Валака, сына Сепфорова, царя Моавитского? Ссорился ли он с Израилем, или воевал ли с ними?
\par 26 Израиль уже живет триста лет в Есевоне и в зависящих от него [городах], в Ароере и зависящих от него [городах], и во всех городах, которые близ Арнона; для чего вы в то время не отнимали [их]?
\par 27 А я не виновен пред тобою, и ты делаешь мне зло, выступив против меня войною. Господь Судия да будет ныне судьею между сынами Израиля и между Аммонитянами!
\par 28 Но царь Аммонитский не послушал слов Иеффая, с которыми он посылал к нему.
\par 29 И был на Иеффае Дух Господень, и прошел он Галаад и Манассию, и прошел Массифу Галаадскую, и из Массифы Галаадской пошел к Аммонитянам.
\par 30 И дал Иеффай обет Господу и сказал: если Ты предашь Аммонитян в руки мои,
\par 31 то по возвращении моем с миром от Аммонитян, что выйдет из ворот дома моего навстречу мне, будет Господу, и вознесу сие на всесожжение.
\par 32 И пришел Иеффай к Аммонитянам--сразиться с ними, и предал их Господь в руки его;
\par 33 и поразил их поражением весьма великим, от Ароера до Минифа двадцать городов, и до Авель-Керамима, и смирились Аммонитяне пред сынами Израилевыми.
\par 34 И пришел Иеффай в Массифу в дом свой, и вот, дочь его выходит навстречу ему с тимпанами и ликами: она была у него только одна, и не было у него еще ни сына, ни дочери.
\par 35 Когда он увидел ее, разодрал одежду свою и сказал: ах, дочь моя! ты сразила меня; и ты в числе нарушителей покоя моего! я отверз [о тебе] уста мои пред Господом и не могу отречься.
\par 36 Она сказала ему: отец мой! ты отверз уста твои пред Господом--и делай со мною то, что произнесли уста твои, когда Господь совершил чрез тебя отмщение врагам твоим Аммонитянам.
\par 37 И сказала отцу своему: сделай мне только вот что: отпусти меня на два месяца; я пойду, взойду на горы и оплачу девство мое с подругами моими.
\par 38 Он сказал: пойди. И отпустил ее на два месяца. Она пошла с подругами своими и оплакивала девство свое в горах.
\par 39 По прошествии двух месяцев она возвратилась к отцу своему, и он совершил над нею обет свой, который дал, и она не познала мужа. И вошло в обычай у Израиля,
\par 40 что ежегодно дочери Израилевы ходили оплакивать дочь Иеффая Галаадитянина, четыре дня в году.

\chapter{12}

\par 1 Ефремляне собрались и перешли в Севину и сказали Иеффаю: для чего ты ходил воевать с Аммонитянами, а нас не позвал с собою? мы сожжем дом твой огнем и с тобою вместе.
\par 2 Иеффай сказал им: я и народ мой имели с Аммонитянами сильную ссору; я звал вас, но вы не спасли меня от руки их;
\par 3 видя, что ты не спасаешь меня, я подверг опасности жизнь мою и пошел на Аммонитян, и предал их Господь в руки мои; зачем же вы пришли ныне воевать со мною?
\par 4 И собрал Иеффай всех жителей Галаадских и сразился с Ефремлянами, и побили жители Галаадские Ефремлян, говоря: вы беглецы Ефремовы, Галаад же среди Ефрема и среди Манассии.
\par 5 И перехватили Галаадитяне переправу чрез Иордан от Ефремлян, и когда кто из уцелевших Ефремлян говорил: `позвольте мне переправиться', то жители Галаадские говорили ему: не Ефремлянин ли ты? Он говорил: нет.
\par 6 Они говорили ему `скажи: шибболет', а он говорил: `сибболет', и не мог иначе выговорить. Тогда они, взяв его, заколали у переправы чрез Иордан. И пало в то время из Ефремлян сорок две тысячи.
\par 7 Иеффай был судьею Израиля шесть лет, и умер Иеффай Галаадитянин и погребен в одном из городов Галаадских.
\par 8 После него был судьею Израиля Есевон из Вифлеема.
\par 9 У него было тридцать сыновей, и тридцать дочерей отпустил он из дома [в замужество], а тридцать дочерей взял со стороны за сыновей своих, и был судьею Израиля семь лет.
\par 10 И умер Есевон и погребен в Вифлееме.
\par 11 После него был судьею Израиля Елон Завулонянин и судил Израиля десять лет.
\par 12 И умер Елон Завулонянин и погребен в Аиалоне, в земле Завулоновой.
\par 13 После него был судьею Израиля Авдон, сын Гиллела, Пирафонянин.
\par 14 У него было сорок сыновей и тридцать внуков, ездивших на семидесяти молодых ослах; он судил Израиля восемь лет.
\par 15 И умер Авдон, сын Гиллела, Пирафонянин, и погребен в Пирафоне в земле Ефремовой, на горе Амаликовой.

\chapter{13}

\par 1 Сыны Израилевы продолжали делать злое пред очами Господа, и предал их Господь в руки Филистимлян на сорок лет.
\par 2 В то время был человек из Цоры, от племени Данова, именем Маной; жена его была неплодна и не рождала.
\par 3 И явился Ангел Господень жене и сказал ей: вот, ты неплодна и не рождаешь; но зачнешь, и родишь сына;
\par 4 итак берегись, не пей вина и сикера, и не ешь ничего нечистого;
\par 5 ибо вот, ты зачнешь и родишь сына, и бритва не коснется головы его, потому что от самого чрева младенец сей будет назорей Божий, и он начнет спасать Израиля от руки Филистимлян.
\par 6 Жена пришла и сказала мужу своему: человек Божий приходил ко мне, которого вид, как вид Ангела Божия, весьма почтенный; я не спросила его, откуда он, и он не сказал мне имени своего;
\par 7 он сказал мне: `вот, ты зачнешь и родишь сына; итак не пей вина и сикера и не ешь ничего нечистого, ибо младенец от самого чрева до смерти своей будет назорей Божий'.
\par 8 Маной помолился Господу и сказал: Господи! пусть придет опять к нам человек Божий, которого посылал Ты, и научит нас, что нам делать с имеющим родиться младенцем.
\par 9 И услышал Бог голос Маноя, и Ангел Божий опять пришел к жене, когда она была в поле, и Маноя, мужа ее, не было с нею.
\par 10 Жена тотчас побежала и известила мужа своего и сказала ему: вот, явился мне человек, приходивший ко мне тогда.
\par 11 Маной встал и пошел с женою своею, и пришел к тому человеку и сказал ему: ты ли тот человек, который говорил с сею женщиною? [Ангел] сказал: я.
\par 12 И сказал Маной: итак, если исполнится слово твое, как нам поступать с младенцем сим и что делать с ним?
\par 13 Ангел Господень сказал Маною: пусть он остерегается всего, о чем я сказал жене;
\par 14 пусть не ест ничего, что производит виноградная лоза; пусть не пьет вина и сикера и не ест ничего нечистого и соблюдает все, что я приказал ей.
\par 15 И сказал Маной Ангелу Господню: позволь удержать тебя, пока мы изготовим для тебя козленка.
\par 16 Ангел Господень сказал Маною: хотя бы ты и удержал меня, но я не буду есть хлеба твоего; если же хочешь совершить всесожжение Господу, то вознеси его. Маной же не знал, что это Ангел Господень.
\par 17 И сказал Маной Ангелу Господню: как тебе имя? чтобы нам прославить тебя, когда исполнится слово твое.
\par 18 Ангел Господень сказал ему: что ты спрашиваешь об имени моем? оно чудно.
\par 19 И взял Маной козленка и хлебное приношение и вознес Господу на камне. И сделал Он чудо, которое видели Маной и жена его.
\par 20 Когда пламень стал подниматься от жертвенника к небу, Ангел Господень поднялся в пламени жертвенника. Видя это, Маной и жена его пали лицем на землю.
\par 21 И невидим стал Ангел Господень Маною и жене его. Тогда Маной узнал, что это Ангел Господень.
\par 22 И сказал Маной жене своей: верно мы умрем, ибо видели мы Бога.
\par 23 Жена его сказала ему: если бы Господь хотел умертвить нас, то не принял бы от рук наших всесожжения и хлебного приношения, и не показал бы нам всего того, и теперь не открыл бы нам сего.
\par 24 И родила жена сына, и нарекла имя ему: Самсон. И рос младенец, и благословлял его Господь.
\par 25 И начал Дух Господень действовать в нем в стане Дановом, между Цорою и Естаолом.

\chapter{14}

\par 1 И пошел Самсон в Фимнафу и увидел в Фимнафе женщину из дочерей Филистимских.
\par 2 Он пошел и объявил отцу своему и матери своей и сказал: я видел в Фимнафе женщину из дочерей Филистимских; возьмите ее мне в жену.
\par 3 Отец и мать его сказали ему: разве нет женщин между дочерями братьев твоих и во всем народе моем, что ты идешь взять жену у Филистимлян необрезанных? И сказал Самсон отцу своему: ее возьми мне, потому что она мне понравилась.
\par 4 Отец его и мать его не знали, что это от Господа, и что он ищет случая [отмстить] Филистимлянам. А в то время Филистимляне господствовали над Израилем.
\par 5 И пошел Самсон с отцом своим и с матерью своею в Фимнафу, и когда подходили к виноградникам Фимнафским, вот, молодой лев рыкая [идет] навстречу ему.
\par 6 И сошел на него Дух Господень, и он растерзал [льва] как козленка; а в руке у него ничего не было. И не сказал отцу своему и матери своей, что он сделал.
\par 7 И пришел и поговорил с женщиною, и она понравилась Самсону.
\par 8 Спустя несколько дней, опять пошел он, чтобы взять ее, и зашел посмотреть труп льва, и вот, рой пчел в трупе львином и мед.
\par 9 Он взял его в руки свои и пошел, и ел дорогою; и когда пришел к отцу своему и матери своей, дал и им, и они ели; но не сказал им, что из львиного трупа взял мед сей.
\par 10 И пришел отец его к женщине, и сделал там Самсон пир, как обыкновенно делают женихи.
\par 11 И как там увидели его, выбрали тридцать брачных друзей, которые были бы при нем.
\par 12 И сказал им Самсон: загадаю я вам загадку; если вы отгадаете мне ее в семь дней пира и отгадаете верно, то я дам вам тридцать синдонов и тридцать перемен одежд;
\par 13 если же не сможете отгадать мне, то вы дайте мне тридцать синдонов и тридцать перемен одежд. Они сказали ему: загадай загадку твою, послушаем.
\par 14 И сказал им: из ядущего вышло ядомое, и из сильного вышло сладкое. И не могли отгадать загадку в три дня.
\par 15 В седьмой день сказали они жене Самсоновой: уговори мужа твоего, чтоб он разгадал нам загадку; иначе сожжем огнем тебя и дом отца твоего; разве вы призвали нас, чтоб обобрать нас?
\par 16 И плакала жена Самсонова пред ним и говорила: ты ненавидишь меня и не любишь; ты загадал загадку сынам народа моего, а мне не разгадаешь ее. Он сказал ей: отцу моему и матери моей не разгадал ее; и тебе ли разгадаю?
\par 17 И плакала она пред ним семь дней, в которые продолжался у них пир. Наконец в седьмой день разгадал ей, ибо она усиленно просила его. А она разгадала загадку сынам народа своего.
\par 18 И в седьмой день до захождения солнечного сказали ему граждане: что слаще меда, и что сильнее льва! Он сказал им: если бы вы не орали на моей телице, то не отгадали бы моей загадки.
\par 19 И сошел на него Дух Господень, и пошел он в Аскалон, и, убив там тридцать человек, снял с них одежды, и отдал перемены [платья] их разгадавшим загадку. И воспылал гнев его, и ушел он в дом отца своего.
\par 20 А жена Самсонова вышла за брачного друга его, который был при нем другом.

\chapter{15}

\par 1 Чрез несколько дней, во время жатвы пшеницы, пришел Самсон повидаться с женою своею, принеся с собою козленка; и когда сказал: `войду к жене моей в спальню', отец ее не дал ему войти.
\par 2 И сказал отец ее: я подумал, что ты возненавидел ее, и я отдал ее другу твоему; вот, меньшая сестра красивее ее; пусть она будет тебе вместо ее.
\par 3 Но Самсон сказал им: теперь я буду прав пред Филистимлянами, если сделаю им зло.
\par 4 И пошел Самсон, и поймал триста лисиц, и взял факелы, и связал хвост с хвостом, и привязал по факелу между двумя хвостами;
\par 5 и зажег факелы, и пустил их на жатву Филистимскую, и выжег и копны и нежатый хлеб, и виноградные сады [и] масличные.
\par 6 И говорили Филистимляне: кто это сделал? И сказали: Самсон, зять Фимнафянина, ибо этот взял жену его и отдал другу его. И пошли Филистимляне и сожгли огнем ее и отца ее.
\par 7 Самсон сказал им: хотя вы сделали это, но я отмщу вам самим и тогда только успокоюсь.
\par 8 И перебил он им голени и бедра, и пошел и засел в ущелье скалы Етама.
\par 9 И пошли Филистимляне, и расположились станом в Иудее, и протянулись до Лехи.
\par 10 И сказали жители Иудеи: за что вы вышли против нас? Они сказали: мы пришли связать Самсона, чтобы поступить с ним, как он поступил с нами.
\par 11 И пошли три тысячи человек из Иудеи к ущелью скалы Етама и сказали Самсону: разве ты не знаешь, что Филистимляне господствуют над нами? что ты это сделал нам? Он сказал им: как они со мною поступили, так и я поступил с ними.
\par 12 И сказали ему: мы пришли связать тебя, чтобы отдать тебя в руки Филистимлянам. И сказал им Самсон: поклянитесь мне, что вы не убьете меня.
\par 13 И сказали ему: нет, мы только свяжем тебя и отдадим тебя в руки их, а умертвить не умертвим. И связали его двумя новыми веревками и повели его из ущелья.
\par 14 Когда он подошел к Лехе, Филистимляне с криком встретили его. И сошел на него Дух Господень, и веревки, бывшие на руках его, сделались, как перегоревший лен, и упали узы его с рук его.
\par 15 Нашел он свежую ослиную челюсть и, протянув руку свою, взял ее, и убил ею тысячу человек.
\par 16 И сказал Самсон: челюстью ослиною толпу, две толпы, челюстью ослиною убил я тысячу человек.
\par 17 Сказав это, бросил челюсть из руки своей и назвал то место: Рамаф-Лехи.
\par 18 И почувствовал сильную жажду и воззвал к Господу и сказал: Ты соделал рукою раба Твоего великое спасение сие; а теперь умру я от жажды, и попаду в руки необрезанных.
\par 19 И разверз Бог ямину в Лехе, и потекла из нее вода. Он напился, и возвратился дух его, и он ожил; оттого и наречено имя месту сему: `Источник взывающего', который в Лехе до сего дня.
\par 20 И был он судьею Израиля во дни Филистимлян двадцать лет.

\chapter{16}

\par 1 Пришел однажды Самсон в Газу и, увидев там блудницу, вошел к ней.
\par 2 Жителям Газы сказали: Самсон пришел сюда. И ходили они кругом, и подстерегали его всю ночь в воротах города, и таились всю ночь, говоря: до света утреннего [подождем, и] убьем его.
\par 3 А Самсон спал до полуночи; в полночь же встав, схватил двери городских ворот с обоими косяками, поднял их вместе с запором, положил на плечи свои и отнес их на вершину горы, которая на пути к Хеврону.
\par 4 После того полюбил он одну женщину, жившую на долине Сорек; имя ей Далида.
\par 5 К ней пришли владельцы Филистимские и говорят ей: уговори его, и выведай, в чем великая сила его и как нам одолеть его, чтобы связать его и усмирить его; а мы дадим тебе за то каждый тысячу сто [сиклей] серебра.
\par 6 И сказала Далида Самсону: скажи мне, в чем великая сила твоя и чем связать тебя, чтобы усмирить тебя?
\par 7 Самсон сказал ей: если свяжут меня семью сырыми тетивами, которые не засушены, то я сделаюсь бессилен и буду как и прочие люди.
\par 8 И принесли ей владельцы Филистимские семь сырых тетив, которые не засохли, и она связала его ими.
\par 9 (Между тем один скрытно сидел у нее в спальне.) И сказала ему: Самсон! Филистимляне [идут] на тебя. Он разорвал тетивы, как разрывают нитку из пакли, когда пережжет ее огонь. И не узнана сила его.
\par 10 И сказала Далида Самсону: вот, ты обманул меня и говорил мне ложь; скажи же теперь мне, чем связать тебя?
\par 11 Он сказал ей: если свяжут меня новыми веревками, которые не были в деле, то я сделаюсь бессилен и буду, как прочие люди.
\par 12 Далида взяла новые веревки и связала его и сказала ему: Самсон! Филистимляне [идут] на тебя. (Между тем один скрытно сидел в спальне.) И сорвал он их с рук своих, как нитки.
\par 13 И сказала Далида Самсону: все ты обманываешь меня и говоришь мне ложь; скажи мне, чем бы связать тебя? Он сказал ей: если ты воткешь семь кос головы моей в ткань [и прибьешь ее гвоздем к ткальной колоде].
\par 14 и прикрепила их к колоде, и сказала ему: Филистимляне [идут] на тебя, Самсон! Он пробудился от сна своего и выдернул ткальную колоду вместе с тканью.
\par 15 И сказала ему [Далида]: как же ты говоришь: `люблю тебя', а сердце твое не со мною? вот, ты трижды обманул меня, и не сказал мне, в чем великая сила твоя.
\par 16 И как она словами своими тяготила его всякий день и мучила его, то душе его тяжело стало до смерти.
\par 17 И он открыл ей все сердце свое, и сказал ей: бритва не касалась головы моей, ибо я назорей Божий от чрева матери моей; если же остричь меня, то отступит от меня сила моя; я сделаюсь слаб и буду, как прочие люди.
\par 18 Далида, видя, что он открыл ей все сердце свое, послала и звала владельцев Филистимских, сказав им: идите теперь; он открыл мне все сердце свое. И пришли к ней владельцы Филистимские и принесли серебро в руках своих.
\par 19 И усыпила его [Далида] на коленях своих, и призвала человека, и велела ему остричь семь кос головы его. И начал он ослабевать, и отступила от него сила его.
\par 20 Она сказала: Филистимляне [идут] на тебя, Самсон! Он пробудился от сна своего, и сказал: пойду, как и прежде, и освобожусь. А не знал, что Господь отступил от него.
\par 21 Филистимляне взяли его и выкололи ему глаза, привели его в Газу и оковали его двумя медными цепями, и он молол в доме узников.
\par 22 Между тем волосы на голове его начали расти, где они были острижены.
\par 23 Владельцы Филистимские собрались, чтобы принести великую жертву Дагону, богу своему, и повеселиться, и сказали: бог наш предал Самсона, врага нашего, в руки наши.
\par 24 Также и народ, видя его, прославлял бога своего, говоря: бог наш предал в руки наши врага нашего и опустошителя земли нашей, который побил многих из нас.
\par 25 И когда развеселилось сердце их, сказали: позовите Самсона, пусть он позабавит нас. И призвали Самсона из дома узников, и он забавлял их, и поставили его между столбами.
\par 26 И сказал Самсон отроку, который водил его за руку: подведи меня, чтобы ощупать мне столбы, на которых утвержден дом, и прислониться к ним.
\par 27 Дом же был полон мужчин и женщин; там были все владельцы Филистимские, и на кровле было до трех тысяч мужчин и женщин, смотревших на забавляющего [их] Самсона.
\par 28 И воззвал Самсон к Господу и сказал: Господи Боже! вспомни меня и укрепи меня только теперь, о Боже! чтобы мне в один раз отмстить Филистимлянам за два глаза мои.
\par 29 И сдвинул Самсон с места два средних столба, на которых утвержден был дом, упершись в них, в один правою рукою своею, а в другой левою.
\par 30 И сказал Самсон: умри, душа моя, с Филистимлянами! И уперся [всею] силою, и обрушился дом на владельцев и на весь народ, бывший в нем. И было умерших, которых умертвил [Самсон] при смерти своей, более, нежели сколько умертвил он в жизни своей.
\par 31 И пришли братья его и весь дом отца его, и взяли его, и пошли и похоронили его между Цорою и Естаолом, во гробе Маноя, отца его. Он был судьею Израиля двадцать лет.

\chapter{17}

\par 1 Был некто на горе Ефремовой, именем Миха.
\par 2 Он сказал матери своей: тысяча сто [сиклей] серебра, которые у тебя взяты и за которые ты при мне изрекла проклятие, это серебро у меня, я взял его. Мать его сказала: благословен сын мой у Господа!
\par 3 И возвратил он матери своей тысячу сто [сиклей] серебра. И сказала мать его: это серебро я от себя посвятила Господу для сына моего, чтобы сделать из него истукан и литый кумир; итак отдаю оное тебе.
\par 4 Но он возвратил серебро матери своей. Мать его взяла двести [сиклей] серебра и отдала их плавильщику. Он сделал из них истукан и литый кумир, который и находился в доме Михи.
\par 5 И был у Михи дом Божий. И сделал он ефод и терафим и посвятил одного из сыновей своих, чтоб он был у него священником.
\par 6 В те дни не было царя у Израиля; каждый делал то, что ему казалось справедливым.
\par 7 Один юноша из Вифлеема Иудейского, из колена Иудина, левит, тогда жил там;
\par 8 этот человек пошел из города Вифлеема Иудейского, чтобы пожить, где случится, и идя дорогою, пришел на гору Ефремову к дому Михи.
\par 9 И сказал ему Миха: откуда ты идешь? Он сказал ему: я левит из Вифлеема Иудейского и иду пожить, где случится.
\par 10 И сказал ему Миха: останься у меня и будь у меня отцом и священником; я буду давать тебе по десяти [сиклей] серебра на год, потребное одеяние и пропитание.
\par 11 Левит пошел к нему и согласился левит остаться у этого человека, и был юноша у него, как один из сыновей его.
\par 12 Миха посвятил левита, и этот юноша был у него священником и жил в доме у Михи.
\par 13 И сказал Миха: теперь я знаю, что Господь будет мне благотворить, потому что левит у меня священником.

\chapter{18}

\par 1 В те дни не было царя у Израиля; и в те дни колено Даново искало себе удела, где бы поселиться, потому что дотоле не выпало ему [полного] удела между коленами Израилевыми.
\par 2 И послали сыны Дановы от племени своего пять человек, мужей сильных, из Цоры и Естаола, чтоб осмотреть землю и узнать ее, и сказали им: пойдите, узнайте землю. Они пришли на гору Ефремову к дому Михи и ночевали там.
\par 3 Находясь у дома Михи, узнали они голос молодого левита и зашли туда и спрашивали его: кто тебя привел сюда? что ты здесь делаешь и зачем ты здесь?
\par 4 Он сказал им: то и то сделал для меня Миха, нанял меня, и я у него священником.
\par 5 Они сказали ему: вопроси Бога, чтобы знать нам, успешен ли будет путь наш, в который мы идем.
\par 6 Священник сказал им: идите с миром; пред Господом путь ваш, в который вы идете.
\par 7 И пошли те пять мужей, и пришли в Лаис, и увидели народ, который в нем, что он живет покойно, по обычаю Сидонян, тих и беспечен, и что не было в земле той, кто обижал бы в чем, или имел бы власть: от Сидонян они жили далеко, и ни с кем не было у них никакого дела.
\par 8 И возвратились к братьям своим в Цору и Естаол, и сказали им братья их: с чем вы?
\par 9 Они сказали: встанем и пойдем на них; мы видели землю, она весьма хороша; а вы задумались: не медлите пойти и взять в наследие ту землю;
\par 10 когда пойдете вы, придете к народу беспечному, и земля та обширна; Бог предает ее в руки ваши; это такое место, где нет ни в чем недостатка, что [получается] от земли.
\par 11 И отправились оттуда из колена Данова, из Цоры и Естаола, шестьсот мужей, препоясавшись воинским оружием.
\par 12 Они пошли и стали станом в Кириаф-Иариме, в Иудее. Посему и называют то место станом Дановым до сего дня. Он позади Кириаф--Иарима.
\par 13 Оттуда отправились они на гору Ефремову и пришли к дому Михи.
\par 14 И сказали те пять мужей, которые ходили осматривать землю Лаис, братьям своим: знаете ли, что в одном из домов сих есть ефод, терафим, истукан и литый кумир? итак подумайте, что сделать.
\par 15 И зашли туда, и вошли в дом молодого левита, в дом Михи, и приветствовали его.
\par 16 А шестьсот человек из сынов Дановых, перепоясанные воинским оружием, стояли у ворот.
\par 17 Пять же человек, ходивших осматривать землю, пошли, вошли туда, взяли истукан и ефод и терафим и литый кумир. Священник стоял у ворот с теми шестьюстами человек, препоясанных воинским оружием.
\par 18 Когда они вошли в дом Михи и взяли истукан, ефод, терафим и литый кумир, священник сказал им: что вы делаете?
\par 19 Они сказали ему: молчи, положи руку твою на уста твои и иди с нами и будь у нас отцом и священником; лучше ли тебе быть священником в доме одного человека, нежели быть священником в колене или в племени Израилевом?
\par 20 Священник обрадовался, и взял ефод, терафим и истукан, и пошел с народом.
\par 21 Они обратились и пошли, и отпустили детей, скот и тяжести вперед.
\par 22 Когда они удалились от дома Михи, жители домов соседних с домом Михи собрались и погнались за сынами Дана,
\par 23 и кричали сынам Дана. [Сыны Дановы] оборотились и сказали Михе: что тебе, что ты так кричишь?
\par 24 (Миха) сказал: вы взяли богов моих, которых я сделал, и священника, и ушли; чего еще более? как же вы говорите: что тебе?
\par 25 Сыны Дановы сказали ему: [молчи], чтобы мы не слышали голоса твоего; иначе некоторые из нас, рассердившись, нападут на вас, и ты погубишь себя и семейство твое.
\par 26 И пошли сыны Дановы путем своим; Миха же, видя, что они сильнее его, пошел назад и возвратился в дом свой.
\par 27 А [сыны Дановы] взяли то, что сделал Миха, и священника, который был у него, и пошли в Лаис, против народа спокойного и беспечного, и побили его мечом, а город сожгли огнем.
\par 28 Некому было помочь, потому что он был отдален от Сидона и ни с кем не имел дела. [Город сей] находился в долине, что близ Беф-Рехова. И построили [снова] город и поселились в нем,
\par 29 и нарекли имя городу: Дан, по имени отца своего Дана, сына Израилева; а прежде имя города тому было: Лаис.
\par 30 И поставили у себя сыны Дановы истукан; Ионафан же, сын Гирсона, сына Манассии, сам и сыновья его были священниками в колене Дановом до дня переселения [жителей той] земли;
\par 31 и имели у себя истукан, сделанный Михою, во все то время, когда дом Божий находился в Силоме.

\chapter{19}

\par 1 В те дни, когда не было царя у Израиля, жил один левит на склоне горы Ефремовой. Он взял себе наложницу из Вифлеема Иудейского.
\par 2 Наложница его поссорилась с ним и ушла от него в дом отца своего в Вифлеем Иудейский и была там четыре месяца.
\par 3 Муж ее встал и пошел за нею, чтобы поговорить к сердцу ее и возвратить ее к себе. С ним был слуга его и пара ослов. Она ввела его в дом отца своего.
\par 4 Отец этой молодой женщины, увидев его, с радостью встретил его, и удержал его тесть его, отец молодой женщины. И пробыл он у него три дня; они ели и пили и ночевали там.
\par 5 В четвертый день встали они рано, и он встал, чтоб идти. И сказал отец молодой женщины зятю своему: подкрепи сердце твое куском хлеба, и потом пойдете.
\par 6 Они остались, и оба вместе ели и пили. И сказал отец молодой женщины человеку тому: останься еще на ночь, и пусть повеселится сердце твое.
\par 7 Человек тот встал, было, чтоб идти, но тесть его упросил его, и он опять ночевал там.
\par 8 На пятый день встал он поутру, чтоб идти. И сказал отец молодой женщины той: подкрепи сердце твое [хлебом], и помедлите, доколе преклонится день. И ели оба они.
\par 9 И встал тот человек, чтоб идти, сам он, наложница его и слуга его. И сказал ему тесть его, отец молодой женщины: вот, день преклонился к вечеру, ночуйте, пожалуйте; вот, дню скоро конец, ночуй здесь, пусть повеселится сердце твое; завтра пораньше встанете в путь ваш, и пойдешь в дом твой.
\par 10 Но муж не согласился ночевать, встал и пошел; и пришел к Иевусу, что [ныне] Иерусалим; с ним пара навьюченных ослов и наложница его с ним.
\par 11 Когда они были близ Иевуса, день уже очень преклонился. И сказал слуга господину своему: зайдем в этот город Иевусеев и ночуем в нем.
\par 12 Господин его сказал ему: нет, не пойдем в город иноплеменников, которые не из сынов Израилевых, но дойдем до Гивы.
\par 13 И сказал слуге своему: дойдем до одного из сих мест и ночуем в Гиве, или в Раме.
\par 14 И пошли, и шли, и закатилось солнце подле Гивы Вениаминовой.
\par 15 И повернули они туда, чтобы пойти ночевать в Гиве. И пришел он и сел на улице в городе; но никто не приглашал их в дом для ночлега.
\par 16 И вот, идет один старик с работы своей с поля вечером; он родом был с горы Ефремовой и жил в Гиве. Жители же места сего были сыны Вениаминовы.
\par 17 Он, подняв глаза свои, увидел прохожего на улице городской. И сказал старик: куда идешь? и откуда ты пришел?
\par 18 Он сказал ему: мы идем из Вифлеема Иудейского к горе Ефремовой, откуда я; я ходил в Вифлеем Иудейский, а теперь иду к дому Господа; и никто не приглашает меня в дом;
\par 19 у нас есть и солома и корм для ослов наших; также хлеб и вино для меня и для рабы твоей и для сего слуги есть у рабов твоих; ни в чем нет недостатка.
\par 20 Старик сказал ему: будь спокоен: весь недостаток твой на мне, только не ночуй на улице.
\par 21 И ввел его в дом свой и дал корму ослам [его], а сами они омыли ноги свои и ели и пили.
\par 22 Тогда как они развеселили сердца свои, вот, жители города, люди развратные, окружили дом, стучались в двери и говорили старику, хозяину дома: выведи человека, вошедшего в дом твой, мы познаем его.
\par 23 Хозяин дома вышел к ним и сказал им: нет, братья мои, не делайте зла, когда человек сей вошел в дом мой, не делайте этого безумия;
\par 24 вот у меня дочь девица, и у него наложница, выведу я их, смирите их и делайте с ними, что вам угодно; а с человеком сим не делайте этого безумия.
\par 25 Но они не хотели слушать его. Тогда муж взял свою наложницу и вывел к ним на улицу. Они познали ее, и ругались над нею всю ночь до утра. И отпустили ее при появлении зари.
\par 26 И пришла женщина пред появлением зари, и упала у дверей дома того человека, у которого был господин ее, [и лежала] до света.
\par 27 Господин ее встал поутру, отворил двери дома и вышел, чтоб идти в путь свой: и вот, наложница его лежит у дверей дома, и руки ее на пороге.
\par 28 Он сказал ей: вставай, пойдем. Но ответа не было, [потому что она умерла]. Он положил ее на осла, встал и пошел в свое место.
\par 29 Придя в дом свой, взял нож и, взяв наложницу свою, разрезал ее по членам ее на двенадцать частей и послал во все пределы Израилевы.
\par 30 Всякий, видевший это, говорил: не бывало и не видано было подобного сему от дня исшествия сынов Израилевых из земли Египетской до сего дня. Обратите внимание на это, посоветуйтесь и скажите.

\chapter{20}

\par 1 И вышли все сыны Израилевы, и собралось [все] общество, как один человек, от Дана до Вирсавии, и земля Галаадская пред Господа в Массифу.
\par 2 И собрались начальники всего народа, все колена Израилевы, в собрание народа Божия, четыреста тысяч пеших, обнажающих меч.
\par 3 И сыны Вениаминовы услышали, что сыны Израилевы пришли в Массифу. И сказали сыны Израилевы: скажите, как происходило это зло?
\par 4 Левит, муж оной убитой женщины, отвечал и сказал: я с наложницею моею пришел ночевать в Гиву Вениаминову;
\par 5 и восстали на меня жители Гивы и окружили из-за меня дом ночью; меня намеревались убить, и наложницу мою замучили, так, что она умерла;
\par 6 я взял наложницу мою, разрезал ее и послал ее во все области владения Израилева, ибо они сделали беззаконное и срамное дело в Израиле;
\par 7 вот все вы, сыны Израилевы, рассмотрите это дело и решите здесь.
\par 8 И восстал весь народ, как один человек, и сказал: не пойдем никто в шатер свой и не возвратимся никто в дом свой;
\par 9 и вот что мы сделаем ныне с Гивою: [пойдем] на нее по жребию;
\par 10 и возьмем по десяти человек из ста от всех колен Израилевых, по сто от тысячи и по тысяче от тьмы, чтоб они принесли съестных припасов для народа, который пойдет против Гивы Вениаминовой, наказать ее за срамное дело, которое она сделала в Израиле.
\par 11 И собрались все Израильтяне против города единодушно, как один человек.
\par 12 И послали колена Израилевы во все колено Вениаминово сказать: какое это гнусное дело сделано у вас!
\par 13 Выдайте развращенных оных людей, которые в Гиве; мы умертвим их и искореним зло из Израиля. Но сыны Вениаминовы не хотели послушать голоса братьев своих, сынов Израилевых;
\par 14 а собрались сыны Вениаминовы из городов в Гиву, чтобы пойти войною против сынов Израилевых.
\par 15 И насчиталось в тот день сынов Вениаминовых, [собравшихся] из городов, двадцать шесть тысяч человек, обнажающих меч; кроме того, из жителей Гивы насчитано семьсот отборных;
\par 16 из всего народа сего было семьсот человек отборных, которые были левши, и все сии, бросая из пращей камни в волос, не бросали мимо.
\par 17 Израильтян же, кроме сынов Вениаминовых, насчиталось четыреста тысяч человек, обнажающих меч; все они были способны к войне.
\par 18 И встали и пошли в дом Божий, и вопрошали Бога и сказали сыны Израилевы: кто из нас прежде пойдет на войну с сынами Вениамина? И сказал Господь: Иуда [пойдет] впереди.
\par 19 И встали сыны Израилевы поутру и расположились станом подле Гивы;
\par 20 и выступили Израильтяне на войну против Вениамина, и стали сыны Израилевы в боевой порядок близ Гивы.
\par 21 И вышли сыны Вениаминовы из Гивы и положили в тот день двадцать две тысячи Израильтян на землю.
\par 22 Но народ Израильский ободрился, и опять стали в боевой порядок на том месте, где стояли в прежний день.
\par 23 И пошли сыны Израилевы, и плакали пред Господом до вечера, и вопрошали Господа: вступать ли мне еще в сражение с сынами Вениамина, брата моего? Господь сказал: идите против него.
\par 24 И подступили сыны Израилевы к сынам Вениамина во второй день.
\par 25 Вениамин вышел против них из Гивы во второй день, и еще положили на землю из сынов Израилевых восемнадцать тысяч человек, обнажающих меч.
\par 26 Тогда все сыны Израилевы и весь народ пошли и пришли в дом Божий и, сидя там, плакали пред Господом, и постились в тот день до вечера, и вознесли всесожжения и мирные жертвы пред Господом.
\par 27 И вопрошали сыны Израилевы Господа (в то время ковчег завета Божия находился там,
\par 28 и Финеес, сын Елеазара, сына Ааронова, предстоял пред ним): выходить ли мне еще на сражение с сынами Вениамина, брата моего, или нет? Господь сказал: идите; Я завтра предам его в руки ваши.
\par 29 И поставил Израиль засаду вокруг Гивы.
\par 30 И пошли сыны Израилевы на сынов Вениамина в третий день и стали в боевой порядок пред Гивою, как прежде.
\par 31 Сыны Вениаминовы выступили против народа и отдалились от города, и начали, как прежде, убивать из народа на дорогах, из которых одна идет к Вефилю, а другая к Гиве полем, и [убили] до тридцати человек из Израильтян.
\par 32 И сказали сыны Вениаминовы: они падают пред нами, как и прежде. А сыны Израилевы сказали: побежим от них и отвлечем их от города на дороги.
\par 33 И все Израильтяне встали с своего места и выстроились в Ваал-Фамаре. И засада Израилева устремилась из своего места, с западной стороны Гивы.
\par 34 И пришли пред Гиву десять тысяч человек отборных из всего Израиля, и началось жестокое сражение; но [сыны Вениамина] не знали, что предстоит им беда.
\par 35 И поразил Господь Вениамина пред Израильтянами, и положили в тот день Израильтяне из сынов Вениамина двадцать пять тысяч сто человек, обнажавших меч.
\par 36 Когда сыны Вениамина увидели, что они поражены, тогда Израильтяне уступили место сынам Вениамина, ибо надеялись на засаду, которую они поставили близ Гивы.
\par 37 Засада же поспешила и устремилась к Гиве, и вступила и поразила весь город мечом.
\par 38 Израильтяне поставили с засадою [условленным] знаком к нападению поднимающийся дым из города.
\par 39 Итак, когда Израильтяне отступили с места сражения, и Вениамин начал поражать и поверг Израильтян до тридцати человек и говорил: `опять падают они пред нами, как и в прежние сражения',
\par 40 тогда начал подниматься из города дым столбом. Вениамин оглянулся назад, и вот, [дым] от всего города восходит к небу.
\par 41 Израильтяне воротились, а Вениамин оробел, ибо увидел, что постигла его беда.
\par 42 И побежали они от Израильтян по дороге к пустыне; но сеча преследовала их, и выходившие из городов побивали их там;
\par 43 окружили Вениамина, и преследовали его до Менухи и поражали до самой восточной стороны Гивы.
\par 44 И пало из сынов Вениамина восемнадцать тысяч человек, людей сильных.
\par 45 [Оставшиеся] оборотились и побежали к пустыне, к скале Риммону, и побили еще [Израильтяне] на дорогах пять тысяч человек; и гнались за ними до Гидома и еще убили из них две тысячи человек.
\par 46 Всех же сынов Вениаминовых, павших в тот день, было двадцать пять тысяч человек, обнажавших меч, и все они были мужи сильные.
\par 47 И [обратились оставшиеся] и убежали в пустыню, к скале Риммону, шестьсот человек, и оставались там в каменной горе Риммоне четыре месяца.
\par 48 Израильтяне же опять пошли к сынам Вениаминовым и поразили их мечом, и людей в городе, и скот, и все, что ни встречалось, и все находившиеся [на пути] города сожгли огнем.

\chapter{21}

\par 1 И поклялись Израильтяне в Массифе, говоря: никто из нас не отдаст дочери своей сынам Вениамина в замужество.
\par 2 И пришел народ в дом Божий, и сидели там до вечера пред Богом, и подняли громкий вопль, и сильно плакали,
\par 3 и сказали: Господи, Боже Израилев! для чего случилось это в Израиле, что не стало теперь у Израиля одного колена?
\par 4 На другой день встал народ поутру, и устроили там жертвенник, и вознесли всесожжения и мирные жертвы.
\par 5 И сказали сыны Израилевы: кто не приходил в собрание пред Господа из всех колен Израилевых? Ибо великое проклятие [произнесено] было на тех, которые не пришли пред Господа в Массифу, и сказано было, что те преданы будут смерти.
\par 6 И сжалились сыны Израилевы над Вениамином, братом своим, и сказали: ныне отсечено одно колено от Израиля;
\par 7 как поступить нам с оставшимися из них [касательно] жен, когда мы поклялись Господом не давать им жен из дочерей наших?
\par 8 И сказали: нет ли кого из колен Израилевых, кто не приходил пред Господа в Массифу? И оказалось, что из Иависа Галаадского никто не приходил пред Господа в стан на собрание.
\par 9 И осмотрен народ, и вот, не было там ни одного из жителей Иависа Галаадского.
\par 10 И послало туда общество двенадцать тысяч человек, мужей сильных, и дали им приказание, говоря: идите и поразите жителей Иависа Галаадского мечом, и женщин и детей;
\par 11 и вот что сделайте: всякого мужчину и всякую женщину, познавшую ложе мужеское, предайте заклятию.
\par 12 И нашли они между жителями Иависа Галаадского четыреста девиц, не познавших ложа мужеского, и привели их в стан в Силом, что в земле Ханаанской.
\par 13 И послало все общество переговорить с сынами Вениамина, бывшими в скале Риммоне, и объявило им мир.
\par 14 Тогда возвратились сыны Вениамина, и дали им (Израильтяне) жен, которых оставили в живых из женщин Иависа Галаадского; но оказалось, что этого было недостаточно.
\par 15 Народ же сожалел о Вениамине, что Господь не сохранил целости колен Израилевых.
\par 16 И сказали старейшины общества: что нам делать с оставшимися [касательно] жен, ибо истреблены женщины у Вениамина?
\par 17 И сказали: наследственная земля пусть остается уцелевшим сынам Вениамина, чтобы не исчезло колено от Израиля;
\par 18 но мы не можем дать им жен из дочерей наших; ибо сыны Израилевы поклялись, говоря: проклят, кто даст жену Вениамину.
\par 19 И сказали: вот, каждый год бывает праздник Господень в Силоме, который на север от Вефиля и на восток от дороги, ведущей от Вефиля в Сихем, и на юг от Левоны.
\par 20 И приказали сынам Вениамина и сказали: подите и засядьте в виноградниках,
\par 21 и смотрите, когда выйдут девицы Силомские плясать в хороводах, тогда выйдите из виноградников и схватите себе каждый жену из девиц Силомских и идите в землю Вениаминову;
\par 22 и когда придут отцы их, или братья их с жалобою к нам, мы скажем им: простите нас за них, ибо мы не взяли для каждого из них жены на войне, и вы не дали им; теперь вы виновны.
\par 23 Сыны Вениамина так и сделали, и взяли жен по числу своему из бывших в хороводе, которых они похитили, и пошли и возвратились в удел свой, и построили города и стали жить в них.
\par 24 В то же время Израильтяне разошлись оттуда каждый в колено свое и в племя свое, и пошли оттуда каждый в удел свой.
\par 25 В те дни не было царя у Израиля; каждый делал то, что ему казалось справедливым.


\end{document}