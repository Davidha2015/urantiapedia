\begin{document}

\title{Judges}

Jdg 1:1  По смерти Иисуса вопрошали сыны Израилевы Господа, говоря: кто из нас прежде пойдет на Хананеев--воевать с ними?
Jdg 1:2  И сказал Господь: Иуда пойдет; вот, Я предаю землю в руки его.
Jdg 1:3  Иуда же сказал Симеону, брату своему: войди со мною в жребий мой, и будем воевать с Хананеями; и я войду с тобою в твой жребий. И пошел с ним Симеон.
Jdg 1:4  И пошел Иуда, и предал Господь Хананеев и Ферезеев в руки их, и побили они из них в Везеке десять тысяч человек.
Jdg 1:5  В Везеке встретились они с Адони-Везеком, сразились с ним и разбили Хананеев и Ферезеев.
Jdg 1:6  Адони-Везек побежал, но они погнались за ним и поймали его и отсекли большие пальцы на руках его и на ногах его.
Jdg 1:7  Тогда сказал Адони-Везек: семьдесят царей с отсеченными на руках и на ногах их большими пальцами собирали [крохи] под столом моим; как делал я, так и мне воздал Бог. И привели его в Иерусалим, и он умер там.
Jdg 1:8  И воевали сыны Иудины против Иерусалима и взяли его, и поразили его мечом и город предали огню.
Jdg 1:9  Потом пошли сыны Иудины воевать с Хананеями, которые жили на горах и на полуденной земле и на низменных местах.
Jdg 1:10  И пошел Иуда на Хананеев, которые жили в Хевроне (имя же Хеврону [было] прежде Кириаф-Арбы), и поразили Шешая, Ахимана и Фалмая.
Jdg 1:11  Оттуда пошел он против жителей Давира; имя Давиру [было] прежде Кириаф-Сефер.
Jdg 1:12  И сказал Халев: кто поразит Кириаф-Сефер и возьмет его, тому отдам Ахсу, дочь мою, в жену.
Jdg 1:13  И взял его Гофониил, сын Кеназа, младшего брата Халевова, и [Халев] отдал в жену ему Ахсу, дочь свою.
Jdg 1:14  Когда надлежало ей идти, [Гофониил] научил ее просить у отца ее поле, и она сошла с осла. Халев сказал ей: что тебе?
Jdg 1:15  [Ахса] сказала ему: дай мне благословение; ты дал мне землю полуденную, дай мне и источники воды. И дал ей [Халев] источники верхние и источники нижние.
Jdg 1:16  И сыны [Иофора] Кенеянина, тестя Моисеева, пошли из города Пальм с сынами Иудиными в пустыню Иудину, которая на юг от Арада, и пришли и поселились среди народа.
Jdg 1:17  И пошел Иуда с Симеоном, братом своим, и поразили Хананеев, живших в Цефафе, и предали его заклятию, и [оттого] называется город сей Хорма.
Jdg 1:18  Иуда взял также Газу с пределами ее, Аскалон с пределами его, и Екрон с пределами его.
Jdg 1:19  Господь был с Иудою, и он овладел горою; но жителей долины не мог прогнать, потому что у них были железные колесницы.
Jdg 1:20  И отдали Халеву Хеврон, как говорил Моисей, и изгнал [он] оттуда трех сынов Енаковых.
Jdg 1:21  Но Иевусеев, которые жили в Иерусалиме, не изгнали сыны Вениаминовы, и живут Иевусеи с сынами Вениамина в Иерусалиме до сего дня.
Jdg 1:22  И сыны Иосифа пошли также на Вефиль, и Господь был с ними.
Jdg 1:23  И остановились и высматривали сыны Иосифовы Вефиль (имя же городу [было] прежде Луз).
Jdg 1:24  И увидели стражи человека, идущего из города, и сказали ему: покажи нам вход в город, и сделаем с тобою милость.
Jdg 1:25  Он показал им вход в город, и поразили они город мечом, а человека сего и все родство его отпустили.
Jdg 1:26  Человек сей пошел в землю Хеттеев, и построил город и нарек имя ему Луз. Это имя его до сего дня.
Jdg 1:27  И Манассия не выгнал [жителей] Бефсана и зависящих от него городов, Фаанаха и зависящих от него городов, жителей Дора и зависящих от него городов, жителей Ивлеама и зависящих от него городов, жителей Мегиддона и зависящих от него городов; и остались Хананеи жить в земле сей.
Jdg 1:28  Когда Израиль пришел в силу, тогда сделал он Хананеев данниками, но изгнать не изгнал их.
Jdg 1:29  И Ефрем не изгнал Хананеев, живущих в Газере; и жили Хананеи среди их в Газере.
Jdg 1:30  И Завулон не изгнал жителей Китрона и жителей Наглола, и жили Хананеи среди их и платили им дань.
Jdg 1:31  И Асир не изгнал жителей Акко и жителей Сидона и Ахлава, Ахзива, Хелвы, Афека и Рехова.
Jdg 1:32  И жил Асир среди Хананеев, жителей земли той, ибо не изгнал их.
Jdg 1:33  И Неффалим не изгнал жителей Вефсамиса и жителей Бефанафа и жил среди Хананеев, жителей земли той; жители же Вефсамиса и Бефанафа были его данниками.
Jdg 1:34  И стеснили Аморреи сынов Дановых в горах, ибо не давали им сходить на долину.
Jdg 1:35  И остались Аморреи жить на горе Херес, в Аиалоне и Шаалвиме; но рука сынов Иосифовых одолела [Аморреев], и сделались они данниками им.
Jdg 1:36  Пределы Аморреев от возвышенности Акравим и от Селы простирались и далее.
Jdg 2:1  И пришел Ангел Господень из Галгала в Бохим и сказал: Я вывел вас из Египта и ввел вас в землю, о которой клялся отцам вашим--[дать вам], и сказал Я: `не нарушу завета Моего с вами вовек;
Jdg 2:2  и вы не вступайте в союз с жителями земли сей; жертвенники их разрушьте'. Но вы не послушали гласа Моего. Что вы это сделали?
Jdg 2:3  И потому говорю Я: не изгоню их от вас, и будут они вам петлею, и боги их будут для вас сетью.
Jdg 2:4  Когда Ангел Господень сказал слова сии всем сынам Израилевым, то народ поднял громкий вопль и заплакал.
Jdg 2:5  От сего и называют то место Бохим. Там принесли они жертву Господу.
Jdg 2:6  Когда Иисус распустил народ, и пошли сыны Израилевы, каждый в свой удел, чтобы получить в наследие землю,
Jdg 2:7  тогда народ служил Господу во все дни Иисуса и во все дни старейшин, которых жизнь продлилась после Иисуса и которые видели все великие дела Господни, какие Он сделал Израилю.
Jdg 2:8  Но когда умер Иисус, сын Навин, раб Господень, будучи ста десяти лет,
Jdg 2:9  и похоронили его в пределе удела его в Фамнаф-Сараи, на горе Ефремовой, на север от горы Гааша;
Jdg 2:10  и когда весь народ оный отошел к отцам своим, и восстал после них другой род, который не знал Господа и дел Его, какие Он делал Израилю, --
Jdg 2:11  тогда сыны Израилевы стали делать злое пред очами Господа и стали служить Ваалам;
Jdg 2:12  оставили Господа Бога отцов своих, Который вывел их из земли Египетской, и обратились к другим богам, богам народов, окружавших их, и стали поклоняться им, и раздражили Господа;
Jdg 2:13  оставили Господа и стали служить Ваалу и Астартам.
Jdg 2:14  И воспылал гнев Господень на Израиля, и предал их в руки грабителей, и грабили их; и предал их в руки врагов, окружавших их, и не могли уже устоять пред врагами своими.
Jdg 2:15  Куда они ни пойдут, рука Господня везде была им во зло, как говорил им Господь и как клялся им Господь. И им было весьма тесно.
Jdg 2:16  И воздвигал [им] Господь судей, которые спасали их от рук грабителей их;
Jdg 2:17  но и судей они не слушали, а ходили блудно вслед других богов и поклонялись им, скоро уклонялись от пути, коим ходили отцы их, повинуясь заповедям Господним. Они так не делали.
Jdg 2:18  Когда Господь воздвигал им судей, то Сам Господь был с судьею и спасал их от врагов их во все дни судьи: ибо жалел [их] Господь, слыша стон их от угнетавших и притеснявших их.
Jdg 2:19  Но как скоро умирал судья, они опять делали хуже отцов своих, уклоняясь к другим богам, служа им и поклоняясь им. Не отставали от дел своих и от стропотного пути своего.
Jdg 2:20  И воспылал гнев Господень на Израиля, и сказал Он: за то, что народ сей преступает завет Мой, который Я поставил с отцами их, и не слушает гласа Моего,
Jdg 2:21  и Я не стану уже изгонять от них ни одного из тех народов, которых оставил Иисус, когда умирал, --
Jdg 2:22  чтобы искушать ими Израиля: станут ли они держаться пути Господня и ходить по нему, как держались отцы их, или нет?
Jdg 2:23  И оставил Господь народы сии и не изгнал их вскоре и не предал их в руки Иисуса.
Jdg 3:1  Вот те народы, которых оставил Господь, чтобы искушать ими Израильтян, всех, которые не знали о всех войнах Ханаанских, --
Jdg 3:2  для того только, чтобы знали и учились войне последующие роды сынов Израилевых, которые прежде не знали ее:
Jdg 3:3  пять владельцев Филистимских, все Хананеи, Сидоняне и Евеи, живущие на горе Ливане, от горы Ваал-Ермона до входа в Емаф.
Jdg 3:4  Они были [оставлены], чтобы искушать ими Израильтян и узнать, повинуются ли они заповедям Господним, которые Он заповедал отцам их чрез Моисея.
Jdg 3:5  И жили сыны Израилевы среди Хананеев, Хеттеев, Аморреев, Ферезеев, Евеев, и Иевусеев,
Jdg 3:6  и брали дочерей их себе в жены, и своих дочерей отдавали за сыновей их, и служили богам их.
Jdg 3:7  И сделали сыны Израилевы злое пред очами Господа, и забыли Господа Бога своего, и служили Ваалам и Астартам.
Jdg 3:8  И воспылал гнев Господень на Израиля, и предал их в руки Хусарсафема, царя Месопотамского, и служили сыны Израилевы Хусарсафему восемь лет.
Jdg 3:9  Тогда возопили сыны Израилевы к Господу, и воздвигнул Господь спасителя сынам Израилевым, который спас их, Гофониила, сына Кеназа, младшего брата Халевова.
Jdg 3:10  На нем был Дух Господень, и был он судьею Израиля. Он вышел на войну, и предал Господь в руки его Хусарсафема, царя Месопотамского, и преодолела рука его Хусарсафема.
Jdg 3:11  И покоилась земля сорок лет. И умер Гофониил, сын Кеназа.
Jdg 3:12  Сыны Израилевы опять стали делать злое пред очами Господа, и укрепил Господь Еглона, царя Моавитского, против Израильтян, за то, что они делали злое пред очами Господа.
Jdg 3:13  Он собрал к себе Аммонитян и Амаликитян, и пошел и поразил Израиля, и овладели они городом Пальм.
Jdg 3:14  И служили сыны Израилевы Еглону, царю Моавитскому, восемнадцать лет.
Jdg 3:15  Тогда возопили сыны Израилевы к Господу, и Господь воздвигнул им спасителя Аода, сына Геры, сына Иеминиева, который был левша. И послали сыны Израилевы с ним дары Еглону, царю Моавитскому.
Jdg 3:16  Аод сделал себе меч с двумя остриями, длиною в локоть, и припоясал его под плащом своим к правому бедру,
Jdg 3:17  и поднес дары Еглону, царю Моавитскому; Еглон же был человек очень тучный.
Jdg 3:18  Когда поднес [Аод] все дары и проводил людей, принесших дары,
Jdg 3:19  то сам возвратился от истуканов, которые в Галгале, и сказал: у меня есть тайное слово до тебя, царь. Он сказал: тише! И вышли от него все стоявшие при нем.
Jdg 3:20  Аод вошел к нему: он сидел в прохладной горнице, которая была у него отдельно. И сказал Аод: у меня есть до тебя, слово Божие. [Еглон] встал со стула.
Jdg 3:21  Аод простер левую руку свою и взял меч с правого бедра своего и вонзил его в чрево его,
Jdg 3:22  так что вошла за острием и рукоять, и тук закрыл острие, ибо Аод не вынул меча из чрева его, и он прошел в задние части.
Jdg 3:23  И вышел Аод в преддверие, и затворил за собою двери горницы, и замкнул.
Jdg 3:24  Когда он вышел, рабы [Еглона] пришли и видят, вот, двери горницы замкнуты, и говорят: верно он для нужды в прохладной комнате.
Jdg 3:25  Ждали довольно долго, но видя, что никто не отпирает дверей горницы, взяли ключ и отперли, и вот, господин их лежит на земле мертвый.
Jdg 3:26  Пока они недоумевали, Аод между тем ушел, прошел мимо истуканов и спасся в Сеираф.
Jdg 3:27  Придя же вострубил трубою на горе Ефремовой, и сошли с ним сыны Израилевы с горы, и он [шел] впереди их.
Jdg 3:28  И сказал им: идите за мною, ибо предал Господь врагов ваших Моавитян в руки ваши. И пошли за ним, и перехватили переправу через Иордан к Моаву, и не давали никому переходить.
Jdg 3:29  И побили в то время Моавитян около десяти тысяч человек, все здоровых и сильных, и никто не убежал.
Jdg 3:30  Так смирились в тот день Моавитяне пред Израилем, и покоилась земля восемьдесят лет.
Jdg 3:31  После него был Самегар, сын Анафов, который шестьсот человек Филистимлян побил воловьим рожном; и он также спас Израиля.
Jdg 4:1  Когда умер Аод, сыны Израилевы стали опять делать злое пред очами Господа.
Jdg 4:2  И предал их Господь в руки Иавина, царя Ханаанского, который царствовал в Асоре; военачальником у него был Сисара, который жил в Харошеф-Гоиме.
Jdg 4:3  И возопили сыны Израилевы к Господу, ибо у него было девятьсот железных колесниц, и он жестоко угнетал сынов Израилевых двадцать лет.
Jdg 4:4  В то время была судьею Израиля Девора пророчица, жена Лапидофова;
Jdg 4:5  она жила под Пальмою Девориною, между Рамою и Вефилем, на горе Ефремовой; и приходили к ней сыны Израилевы на суд.
Jdg 4:6  [Девора] послала и призвала Варака, сына Авиноамова, из Кедеса Неффалимова, и сказала ему: повелевает [тебе] Господь Бог Израилев: пойди, взойди на гору Фавор и возьми с собою десять тысяч человек из сынов Неффалимовых и сынов Завулоновых;
Jdg 4:7  а Я приведу к тебе, к потоку Киссону, Сисару, военачальника Иавинова, и колесницы его и многолюдное [войско] его, и предам его в руки твои.
Jdg 4:8  Варак сказал ей: если ты пойдешь со мною, пойду; а если не пойдешь со мною, не пойду.
Jdg 4:9  Она сказала [ему]: пойти пойду с тобою; только не тебе уже будет слава на сем пути, в который ты идешь; но в руки женщины предаст Господь Сисару. И встала Девора и пошла с Вараком в Кедес.
Jdg 4:10  Варак созвал Завулонян и Неффалимлян в Кедес, и пошли вслед за ним десять тысяч человек, и Девора пошла с ним.
Jdg 4:11  Хевер Кенеянин отделился [тогда] от Кенеян, сынов Ховава, родственника Моисеева, и раскинул шатер свой у дубравы в Цаанниме близ Кедеса.
Jdg 4:12  И донесли Сисаре, что Варак, сын Авиноамов, взошел на гору Фавор.
Jdg 4:13  Сисара созвал все колесницы свои, девятьсот железных колесниц, и весь народ, который у него, из Харошеф-Гоима к потоку Киссону.
Jdg 4:14  И сказала Девора Вараку: встань, ибо это тот день, в который Господь предаст Сисару в руки твои; Сам Господь пойдет пред тобою. И сошел Варак с горы Фавора, и за ним десять тысяч человек.
Jdg 4:15  Тогда Господь привел в замешательство Сисару и все колесницы его и все ополчение его от меча Варакова, и сошел Сисара с колесницы и побежал пеший.
Jdg 4:16  Варак преследовал колесницы [его] и ополчение до Харошеф--Гоима, и пало все ополчение Сисарино от меча, не осталось никого.
Jdg 4:17  Сисара же убежал пеший в шатер Иаили, жены Хевера Кенеянина; ибо между Иавином, царем Асорским, и домом Хевера Кенеянина был мир.
Jdg 4:18  И вышла Иаиль навстречу Сисаре и сказала ему: зайди, господин мой, зайди ко мне, не бойся. Он зашел к ней в шатер, и она покрыла его ковром.
Jdg 4:19  [Сисара] сказал ей: дай мне немного воды напиться, я пить хочу. Она развязала мех с молоком, и напоила его и [опять] покрыла его.
Jdg 4:20  [Сисара] сказал ей: стань у дверей шатра, и если кто придет и спросит у тебя и скажет: `нет ли здесь кого?', ты скажи: `нет'.
Jdg 4:21  Иаиль, жена Хеверова, взяла кол от шатра, и взяла молот в руку свою, и подошла к нему тихонько, и вонзила кол в висок его так, что приколола к земле; а он спал от усталости--и умер.
Jdg 4:22  И вот, Варак гонится за Сисарою. Иаиль вышла навстречу ему и сказала ему: войди, я покажу тебе человека, которого ты ищешь. Он вошел к ней, и вот, Сисара лежит мертвый, и кол в виске его.
Jdg 4:23  И смирил Бог в тот день Иавина, царя Ханаанского, пред сынами Израилевыми.
Jdg 4:24  Рука сынов Израилевых усиливалась более и более над Иавином, царем Ханаанским, доколе не истребили они Иавина, царя Ханаанского.
Jdg 5:1  В тот день воспела Девора и Варак, сын Авиноамов, сими словами:
Jdg 5:2  Израиль отмщен, народ показал рвение; прославьте Господа!
Jdg 5:3  Слушайте, цари, внимайте, вельможи: я Господу, я пою, бряцаю Господу Богу Израилеву.
Jdg 5:4  Когда выходил Ты, Господи, от Сеира, когда шел с поля Едомского, тогда земля тряслась, и небо капало, и облака проливали воду;
Jdg 5:5  горы таяли от лица Господа, даже этот Синай от лица Господа Бога Израилева.
Jdg 5:6  Во дни Самегара, сына Анафова, во дни Иаили, были пусты дороги, и ходившие прежде путями прямыми ходили тогда окольными дорогами.
Jdg 5:7  Не стало обитателей в селениях у Израиля, не стало, доколе не восстала я, Девора, доколе не восстала я, мать в Израиле.
Jdg 5:8  Избрали новых богов, оттого война у ворот. Виден ли был щит и копье у сорока тысяч Израиля?
Jdg 5:9  Сердце мое к вам, начальники Израилевы, к ревнителям в народе; прославьте Господа!
Jdg 5:10  Ездящие на ослицах белых, сидящие на коврах и ходящие по дороге, пойте песнь!
Jdg 5:11  Среди голосов собирающих стада при колодезях, там да воспоют хвалу Господу, хвалу вождям Израиля! Тогда выступил ко вратам народ Господень.
Jdg 5:12  Воспряни, воспряни, Девора! воспряни, воспряни! воспой песнь! Восстань, Варак! и веди пленников твоих, сын Авиноамов!
Jdg 5:13  Тогда немногим из сильных подчинил Он народ; Господь подчинил мне храбрых.
Jdg 5:14  От Ефрема пришли укоренившиеся в земле Амалика; за тобою Вениамин, среди народа твоего; от Махира шли начальники, и от Завулона владеющие тростью писца.
Jdg 5:15  И князья Иссахаровы с Деворою, и Иссахар так же, как Варак, бросился в долину пеший. В племенах Рувимовых большое разногласие.
Jdg 5:16  Что сидишь ты между овчарнями, слушая блеяние стад? В племенах Рувимовых большое разногласие.
Jdg 5:17  Галаад живет [спокойно] за Иорданом, и Дану чего бояться с кораблями? Асир сидит на берегу моря и у пристаней своих живет спокойно.
Jdg 5:18  Завулон--народ, обрекший душу свою на смерть, и Неффалим--на высотах поля.
Jdg 5:19  Пришли цари, сразились, тогда сразились цари Ханаанские в Фанаахе у вод Мегиддонских, но не получили нимало серебра.
Jdg 5:20  С неба сражались, звезды с путей своих сражались с Сисарою.
Jdg 5:21  Поток Киссон увлек их, поток Кедумим, поток Киссон. Попирай, душа моя, силу!
Jdg 5:22  Тогда ломались копыта конские от побега, от побега сильных его.
Jdg 5:23  Прокляните Мероз, говорит Ангел Господень, прокляните, прокляните жителей его за то, что не пришли на помощь Господу, на помощь Господу с храбрыми.
Jdg 5:24  Да будет благословенна между женами Иаиль, жена Хевера Кенеянина, между женами в шатрах да будет благословенна!
Jdg 5:25  Воды просил он: молока подала она, в чаше вельможеской принесла молока лучшего.
Jdg 5:26  [Левую] руку свою протянула к колу, а правую свою к молоту работников; ударила Сисару, поразила голову его, разбила и пронзила висок его.
Jdg 5:27  К ногам ее склонился, пал и лежал, к ногам ее склонился, пал; где склонился, там и пал сраженный.
Jdg 5:28  В окно выглядывает и вопит мать Сисарина сквозь решетку: что долго не идет конница его, что медлят колеса колесниц его?
Jdg 5:29  Умные из ее женщин отвечают ей, и сама она отвечает на слова свои:
Jdg 5:30  верно, они нашли, делят добычу, по девице, по две девицы на каждого воина, в добычу полученная разноцветная [одежда] Сисаре, полученная в добычу разноцветная одежда, вышитая с обеих сторон, снятая с плеч пленника.
Jdg 5:31  Так да погибнут все враги Твои, Господи! Любящие же Его [да] [будут] как солнце, восходящее во всей силе своей! --И покоилась земля сорок лет.
Jdg 6:1  Сыны Израилевы стали [опять] делать злое пред очами Господа, и предал их Господь в руки Мадианитян на семь лет.
Jdg 6:2  Тяжела была рука Мадианитян над Израилем, и сыны Израилевы сделали себе от Мадианитян ущелья в горах и пещеры и укрепления.
Jdg 6:3  Когда посеет Израиль, придут Мадианитяне и Амаликитяне и жители востока и ходят у них;
Jdg 6:4  и стоят у них шатрами, и истребляют произведения земли до самой Газы, и не оставляют для пропитания Израилю ни овцы, ни вола, ни осла.
Jdg 6:5  Ибо они приходили со скотом своим и с шатрами своими, приходили в таком множестве, как саранча; им и верблюдам их не было числа, и ходили по земле Израилевой, чтоб опустошать ее.
Jdg 6:6  И весьма обнищал Израиль от Мадианитян, и возопили сыны Израилевы к Господу.
Jdg 6:7  И когда возопили сыны Израилевы к Господу на Мадианитян,
Jdg 6:8  послал Господь пророка к сынам Израилевым, и сказал им: так говорит Господь Бог Израилев: Я вывел вас из Египта, вывел вас из дома рабства;
Jdg 6:9  избавил вас из руки Египтян и из руки всех, угнетавших вас, прогнал их от вас, и дал вам землю их,
Jdg 6:10  и сказал вам: `Я--Господь Бог ваш; не чтите богов Аморрейских, в земле которых вы живете'; но вы не послушали гласа Моего.
Jdg 6:11  И пришел Ангел Господень и сел в Офре под дубом, принадлежащим Иоасу, потомку Авиезерову; сын его Гедеон выколачивал тогда пшеницу в точиле, чтобы скрыться от Мадианитян.
Jdg 6:12  И явился ему Ангел Господень и сказал ему: Господь с тобою, муж сильный!
Jdg 6:13  Гедеон сказал ему: господин мой! если Господь с нами, то отчего постигло нас все это? и где все чудеса Его, о которых рассказывали нам отцы наши, говоря: `из Египта вывел нас Господь'? Ныне оставил нас Господь и предал нас в руки Мадианитян.
Jdg 6:14  Господь, воззрев на него, сказал: иди с этою силою твоею и спаси Израиля от руки Мадианитян; Я посылаю тебя.
Jdg 6:15  [Гедеон] сказал ему: Господи! как спасу я Израиля? вот, и племя мое в [колене] Манассиином самое бедное, и я в доме отца моего младший.
Jdg 6:16  И сказал ему Господь: Я буду с тобою, и ты поразишь Мадианитян, как одного человека.
Jdg 6:17  [Гедеон] сказал Ему: если я обрел благодать пред очами Твоими, то сделай мне знамение, что Ты говоришь со мною:
Jdg 6:18  не уходи отсюда, доколе я не приду к Тебе и не принесу дара моего и не предложу Тебе. Он сказал: Я останусь до возвращения твоего.
Jdg 6:19  Гедеон пошел и приготовил козленка и опресноков из ефы муки; мясо положил в корзину, а похлебку влил в горшок и принес к Нему под дуб и предложил.
Jdg 6:20  И сказал ему Ангел Божий: возьми мясо и опресноки, и положи на сей камень, и вылей похлебку. Он так и сделал.
Jdg 6:21  Ангел Господень простер конец жезла, который был в руке его, прикоснулся к мясу и опреснокам; и вышел огонь из камня и поел мясо и опресноки; и Ангел Господень скрылся от глаз его.
Jdg 6:22  И увидел Гедеон, что это Ангел Господень, и сказал Гедеон: [увы] [мне], Владыка Господи! потому что я видел Ангела Господня лицем к лицу.
Jdg 6:23  Господь сказал ему: мир тебе, не бойся, не умрешь.
Jdg 6:24  И устроил там Гедеон жертвенник Господу и назвал его: Иегова Шалом. Он еще до сего дня в Офре Авиезеровой.
Jdg 6:25  В ту ночь сказал ему Господь: возьми тельца из стада отца твоего и другого тельца семилетнего, и разрушь жертвенник Ваала, который у отца твоего, и сруби священное дерево, которое при нем,
Jdg 6:26  и поставь жертвенник Господу Богу твоему, на вершине скалы сей, в порядке, и возьми второго тельца и принеси во всесожжение на дровах дерева, которое срубишь.
Jdg 6:27  Гедеон взял десять человек из рабов своих и сделал, как говорил ему Господь; но как сделать это днем он боялся домашних отца своего и жителей города, то сделал ночью.
Jdg 6:28  Поутру встали жители города, и вот, жертвенник Ваалов разрушен, и дерево при нем срублено, и второй телец вознесен во всесожжение на новоустроенном жертвеннике.
Jdg 6:29  И говорили друг другу: кто это сделал? Искали, расспрашивали и сказали: Гедеон, сын Иоасов, сделал это.
Jdg 6:30  И сказали жители города Иоасу: выведи сына твоего; он должен умереть за то, что разрушил жертвенник Ваала и срубил дерево, которое было при нем.
Jdg 6:31  Иоас сказал всем приступившим к нему: вам ли вступаться за Ваала, вам ли защищать его? кто вступится за него, тот будет предан смерти в это же утро; если он Бог, то пусть сам вступится за себя, потому что он разрушил его жертвенник.
Jdg 6:32  И стал звать его с того дня Иероваалом, потому что сказал: пусть Ваал сам судится с ним за то, что он разрушил жертвенник его.
Jdg 6:33  Между тем все Мадианитяне и Амаликитяне и жители востока собрались вместе, перешли [реку] и стали станом на долине Изреельской.
Jdg 6:34  И Дух Господень объял Гедеона; он вострубил трубою, и созвано было племя Авиезерово идти за ним.
Jdg 6:35  И послал послов по всему колену Манассиину, и оно вызвалось идти за ним; также послал послов к Асиру, Завулону и Неффалиму, и сии пришли навстречу им.
Jdg 6:36  И сказал Гедеон Богу: если Ты спасешь Израиля рукою моею, как говорил Ты,
Jdg 6:37  то вот, я расстелю [здесь] на гумне стриженую шерсть: если роса будет только на шерсти, а на всей земле сухо, то буду знать, что спасешь рукою моею Израиля, как говорил Ты.
Jdg 6:38  Так и сделалось: на другой день, встав рано, он стал выжимать шерсть и выжал из шерсти росы целую чашу воды.
Jdg 6:39  И сказал Гедеон Богу: не прогневайся на меня, если еще раз скажу и еще только однажды сделаю испытание над шерстью: пусть будет сухо на одной только шерсти, а на всей земле пусть будет роса.
Jdg 6:40  Бог так и сделал в ту ночь: только на шерсти было сухо, а на всей земле была роса.
Jdg 7:1  Иероваал, он же и Гедеон, встал поутру и весь народ, бывший с ним, и расположились станом у источника Харода; Мадиамский же стан был от него к северу у холма Море в долине.
Jdg 7:2  И сказал Господь Гедеону: народа с тобою слишком много, не могу Я предать Мадианитян в руки их, чтобы не возгордился Израиль предо Мною и не сказал: `моя рука спасла меня';
Jdg 7:3  итак провозгласи вслух народа и скажи: `кто боязлив и робок, тот пусть возвратится и пойдет назад с горы Галаада'. И возвратилось народа двадцать две тысячи, а десять тысяч осталось.
Jdg 7:4  И сказал Господь Гедеону: все еще много народа; веди их к воде, там Я выберу их тебе; о ком Я скажу: `пусть идет с тобою', тот и пусть идет с тобою; а о ком скажу тебе: `не должен идти с тобою', тот пусть и не идет.
Jdg 7:5  Он привел народ к воде. И сказал Господь Гедеону: кто будет лакать воду языком своим, как лакает пес, того ставь особо, также и тех всех, которые будут наклоняться на колени свои и пить.
Jdg 7:6  И было число лакавших ртом своим с руки триста человек; весь же остальной народ наклонялся на колени свои пить воду.
Jdg 7:7  И сказал Господь Гедеону: тремя стами лакавших Я спасу вас и предам Мадианитян в руки ваши, а весь народ пусть идет, каждый в свое место.
Jdg 7:8  И взяли они съестной запас у народа себе и трубы их, и отпустил Гедеон всех Израильтян по шатрам и удержал у себя триста человек; стан же Мадиамский был у него внизу в долине.
Jdg 7:9  В ту ночь сказал ему Господь: встань, сойди в стан, Я предаю его в руки твои;
Jdg 7:10  если же ты боишься идти [один], то пойди в стан ты и Фура, слуга твой;
Jdg 7:11  и услышишь, что говорят, и тогда укрепятся руки твои, и пойдешь в стан. И сошел он и Фура, слуга его, к самому [полку] вооруженных, которые были в стане.
Jdg 7:12  Мадианитяне же и Амаликитяне и все жители востока расположились на долине в таком множестве, как саранча; верблюдам их не было числа, много было их, как песку на берегу моря.
Jdg 7:13  Гедеон пришел. И вот, один рассказывает другому сон и говорит: снилось мне, будто круглый ячменный хлеб катился по стану Мадиамскому и, прикатившись к шатру, ударил в него так, что он упал, опрокинул его, и шатер распался.
Jdg 7:14  Другой сказал в ответ ему: это не иное что, как меч Гедеона, сына Иоасова, Израильтянина; предал Бог в руки его Мадианитян и весь стан.
Jdg 7:15  Гедеон, услышав рассказ сна и толкование его, поклонился [Господу] и возвратился в стан Израильский и сказал: вставайте! предал Господь в руки ваши стан Мадиамский.
Jdg 7:16  И разделил триста человек на три отряда и дал в руки всем им трубы и пустые кувшины и в кувшины светильники.
Jdg 7:17  И сказал им: смотрите на меня и делайте то же; вот, я подойду к стану, и что буду делать, то и вы делайте;
Jdg 7:18  когда я и находящиеся со мною затрубим трубою, трубите и вы трубами вашими вокруг всего стана и кричите: [меч] Господа и Гедеона!
Jdg 7:19  И подошел Гедеон и сто человек с ним к стану, в начале средней стражи, и разбудили стражей, и затрубили трубами и разбили кувшины, которые были в руках их.
Jdg 7:20  И затрубили [все] три отряда трубами, и разбили кувшины, и держали в левой руке своей светильники, а в правой руке трубы, и трубили, и кричали: меч Господа и Гедеона!
Jdg 7:21  И стоял всякий на своем месте вокруг стана; и стали бегать во всем стане, и кричали, и обратились в бегство.
Jdg 7:22  Между тем как триста человек трубили трубами, обратил Господь меч одного на другого во всем стане, и бежало ополчение до Бефшитты к Царере, до предела Авелмехолы, близ Табафы.
Jdg 7:23  И созваны Израильтяне из колена Неффалимова, Асирова и всего колена Манассиина, и погнались за Мадианитянами.
Jdg 7:24  Гедеон же послал послов на всю гору Ефремову сказать: выйдите навстречу Мадианитянам и перехватите у них [переправу через] воду до Бефвары и Иордан. И созваны все Ефремляне и перехватили [переправы] [через] воду до Бефвары и Иордан;
Jdg 7:25  и поймали двух князей Мадиамских: Орива и Зива, и убили Орива в Цур-Ориве, а Зива в Иекев-Зиве и преследовали Мадианитян; головы же Орива и Зива принесли к Гедеону за Иордан.
Jdg 8:1  И сказали ему Ефремляне: зачем ты это сделал, что не позвал нас, когда шел воевать с Мадианитянами? И сильно ссорились с ним.
Jdg 8:2  [Гедеон] отвечал им: сделал ли я что такое, как вы ныне? Не счастливее ли Ефрем добирал виноград, нежели Авиезер обирал?
Jdg 8:3  В ваши руки предал Бог князей Мадиамских Орива и Зива, и что мог сделать я такое, как вы? Тогда успокоился дух их против него, когда сказал он им такие слова.
Jdg 8:4  И пришел Гедеон к Иордану, и перешел сам и триста человек, бывшие с ним. Они были утомлены, преследуя [врагов].
Jdg 8:5  И сказал он жителям Сокхофа: дайте хлеба народу, который идет за мною; они утомились, а я преследую Зевея и Салмана, царей Мадиамских.
Jdg 8:6  Князья Сокхофа сказали: разве рука Зевея и Салмана уже в твоей руке, чтобы нам войску твоему давать хлеб?
Jdg 8:7  И сказал Гедеон: за это, когда предаст Господь Зевея и Салмана в руки мои, я растерзаю тело ваше терновником пустынным и молотильными зубчатыми досками.
Jdg 8:8  Оттуда пошел он в Пенуэл и то же сказал жителям его, и жители Пенуэла отвечали ему то же, что отвечали жители Сокхофа.
Jdg 8:9  Он сказал и жителям Пенуэла: когда я возвращусь в мире, разрушу башню сию.
Jdg 8:10  Зевей же и Салман были в Каркоре и с ними их ополчение до пятнадцати тысяч, все, что осталось из всего ополчения жителей востока; пало же сто двадцать тысяч человек, обнажающих меч.
Jdg 8:11  Гедеон пошел к живущим в шатрах на восток от Новы и Иогбеги и поразил стан, когда стан стоял беспечно.
Jdg 8:12  Зевей и Салман побежали; он погнался за ними и схватил обоих царей Мадиамских, Зевея и Салмана, и весь стан привел в замешательство.
Jdg 8:13  И возвратился Гедеон, сын Иоаса, с войны от возвышенности Хереса.
Jdg 8:14  И захватил юношу из жителей Сокхофа и выспросил у него; и он написал ему князей и старейшин Сокхофских семьдесят семь человек.
Jdg 8:15  И пришел он к жителям Сокхофским, и сказал: вот Зевей и Салман, за которых вы посмеялись надо мною, говоря: разве рука Зевея и Салмана уже в твоей руке, чтобы нам давать хлеб утомившимся людям твоим?
Jdg 8:16  И взял старейшин города и терновник пустынный и зубчатые молотильные доски и наказал ими жителей Сокхофа;
Jdg 8:17  и башню Пенуэльскую разрушил, и перебил жителей города.
Jdg 8:18  И сказал Зевею и Салману: каковы были те, которых вы убили на Фаворе? Они сказали: они были такие, как ты, каждый имел вид сынов царских.
Jdg 8:19  [Гедеон] сказал: это были братья мои, сыны матери моей. Жив Господь! если бы вы оставили их в живых, я не убил бы вас.
Jdg 8:20  И сказал Иеферу, первенцу своему: встань, убей их. Но юноша не извлек меча своего, потому что боялся, так как был еще молод.
Jdg 8:21  И сказали Зевей и Салман: встань сам и порази нас, потому что по человеку и сила его. И встал Гедеон, и убил Зевея и Салмана, и взял пряжки, бывшие на шеях верблюдов их.
Jdg 8:22  И сказали Израильтяне Гедеону: владей нами ты и сын твой и сын сына твоего, ибо ты спас нас из руки Мадианитян.
Jdg 8:23  Гедеон сказал им: ни я не буду владеть вами, ни мой сын не будет владеть вами; Господь да владеет вами.
Jdg 8:24  И сказал им Гедеон: прошу у вас одного, дайте мне каждый по серьге из добычи своей. (Ибо у [неприятелей] много было золотых серег, потому что они были Измаильтяне.)
Jdg 8:25  Они сказали: дадим. И разостлали одежду и бросали туда каждый по серьге из добычи своей.
Jdg 8:26  Весу в золотых серьгах, которые он выпросил, было тысяча семьсот золотых [сиклей], кроме пряжек, пуговиц и пурпуровых одежд, которые были на царях Мадиамских, и кроме [золотых] цепочек, которые были на шее у верблюдов их.
Jdg 8:27  Из этого сделал Гедеон ефод и положил его в своем городе, в Офре, и стали все Израильтяне блудно ходить туда за ним, и был он сетью Гедеону и всему дому его.
Jdg 8:28  Так смирились Мадианитяне пред сынами Израиля и не стали уже поднимать головы своей, и покоилась земля сорок лет во дни Гедеона.
Jdg 8:29  И пошел Иероваал, сын Иоасов, и жил в доме своем.
Jdg 8:30  У Гедеона было семьдесят сыновей, происшедших от чресл его, потому что у него много было жен.
Jdg 8:31  Также и наложница, жившая в Сихеме, родила ему сына, и он дал ему имя Авимелех.
Jdg 8:32  И умер Гедеон, сын Иоасов, в глубокой старости, и погребен во гробе отца своего Иоаса, в Офре Авиезеровой.
Jdg 8:33  Когда умер Гедеон, сыны Израилевы опять стали блудно ходить вслед Ваалов и поставили себе богом Ваалверифа;
Jdg 8:34  и не вспомнили сыны Израилевы Господа Бога своего, Который избавлял их из руки всех врагов, окружавших их;
Jdg 8:35  и дому Иероваалову, [или] Гедеонову, не сделали милости за все благодеяния, какие он сделал Израилю.
Jdg 9:1  Авимелех, сын Иероваалов, пошел в Сихем к братьям матери своей и говорил им и всему племени отца матери своей, и сказал:
Jdg 9:2  внушите всем жителям Сихемским: что лучше для вас, чтобы владели вами все семьдесят сынов Иеровааловых, или чтобы владел один? и вспомните, что я кость ваша и плоть ваша.
Jdg 9:3  Братья матери его внушили о нем все сии слова жителям Сихемским; и склонилось сердце их к Авимелеху, ибо говорили они: он брат наш.
Jdg 9:4  И дали ему семьдесят [сиклей] серебра из дома Ваалверифа; Авимелех нанял на оные праздных и своевольных людей, которые и пошли за ним.
Jdg 9:5  И пришел он в дом отца своего в Офру и убил братьев своих, семьдесят сынов Иеровааловых, на одном камне. Остался только Иофам, младший сын Иероваалов, потому что скрылся.
Jdg 9:6  И собрались все жители Сихемские и весь дом Милло, и пошли и поставили царем Авимелеха у дуба, что близ Сихема.
Jdg 9:7  Когда рассказали об этом Иофаму, он пошел и стал на вершине горы Гаризима и, возвысив голос свой, кричал и говорил им: послушайте меня, жители Сихема, и послушает вас Бог!
Jdg 9:8  Пошли некогда дерева помазать над собою царя и сказали маслине: царствуй над нами.
Jdg 9:9  Маслина сказала им: оставлю ли я тук мой, которым чествуют богов и людей и пойду ли скитаться по деревам?
Jdg 9:10  И сказали дерева смоковнице: иди ты, царствуй над нами.
Jdg 9:11  Смоковница сказала им: оставлю ли я сладость мою и хороший плод мой и пойду ли скитаться по деревам?
Jdg 9:12  И сказали дерева виноградной лозе: иди ты, царствуй над нами.
Jdg 9:13  Виноградная лоза сказала им: оставлю ли я сок мой, который веселит богов и человеков, и пойду ли скитаться по деревам?
Jdg 9:14  Наконец сказали все дерева терновнику: иди ты, царствуй над нами.
Jdg 9:15  Терновник сказал деревам: если вы по истине поставляете меня царем над собою, то идите, покойтесь под тенью моею; если же нет, то выйдет огонь из терновника и пожжет кедры Ливанские.
Jdg 9:16  Итак смотрите, по истине ли и по правде ли вы поступили, поставив Авимелеха царем? И хорошо ли вы поступили с Иероваалом и домом его, и сообразно ли с его благодеяниями поступили вы?
Jdg 9:17  За вас отец мой сражался, не дорожил жизнью своею и избавил вас от руки Мадианитян;
Jdg 9:18  а вы теперь восстали против дома отца моего, и убили семьдесят сынов отца моего на одном камне, и поставили царем над жителями Сихемскими Авимелеха, сына рабыни его, потому что он брат ваш.
Jdg 9:19  Если вы ныне по истине и по правде поступили с Иероваалом и домом его, то радуйтесь об Авимелехе, и он пусть радуется о вас;
Jdg 9:20  если же нет, то да изыдет огонь от Авимелеха и да пожжет жителей Сихемских и весь дом Милло и да изыдет огонь от жителей Сихемских и от дома Милло, и да пожжет Авимелеха.
Jdg 9:21  И побежал Иофам, и убежал и пошел в Беэр, и жил там, [укрываясь] от брата своего Авимелеха.
Jdg 9:22  Авимелех же царствовал над Израилем три года.
Jdg 9:23  И послал Бог злого духа между Авимелехом и между жителями Сихема, и не стали покоряться жители Сихемские Авимелеху,
Jdg 9:24  дабы таким образом совершилось мщение за семьдесят сынов Иеровааловых, и кровь их обратилась на Авимелеха, брата их, который убил их, и на жителей Сихемских, которые подкрепили руки его, чтоб убить братьев своих.
Jdg 9:25  Жители Сихемские посадили против него в засаду людей на вершинах гор, которые грабили всякого проходящего мимо их по дороге. О сем донесено было Авимелеху.
Jdg 9:26  Пришел же и Гаал, сын Еведов, с братьями своими в Сихем, и ходили они по Сихему, и жители Сихемские положились на него.
Jdg 9:27  И вышли в поле, и собирали виноград свой, и давили в точилах, и делали праздники, ходили в дом бога своего, и ели и пили, и проклинали Авимелеха.
Jdg 9:28  Гаал, сын Еведов, говорил: кто Авимелех и что Сихем, чтобы нам служить ему? Не сын ли он Иероваалов, и не Зевул ли главный начальник его? Служите лучше потомкам Еммора, отца Сихемова, а ему для чего нам служить?
Jdg 9:29  Если бы кто дал народ сей в руки мои, я прогнал бы Авимелеха. И сказано было Авимелеху: умножь войско твое и выходи.
Jdg 9:30  Зевул, начальник города, услышал слова Гаала, сына Еведова, и воспылал гнев его.
Jdg 9:31  Он хитрым образом отправляет послов к Авимелеху, чтобы сказать: вот, Гаал, сын Еведов, и братья его пришли в Сихем, и вот, они возмущают против тебя город;
Jdg 9:32  итак, встань ночью, ты и народ, находящийся с тобою, и поставь засаду в поле;
Jdg 9:33  поутру же, при восхождении солнца, встань рано и приступи к городу; и когда он и народ, который у него, выйдут к тебе, тогда делай с ними, что может рука твоя.
Jdg 9:34  И встал ночью Авимелех и весь народ, находившийся с ним, и поставили в засаду у Сихема четыре отряда.
Jdg 9:35  Гаал, сын Еведов, вышел и стал у ворот городских; и встал Авимелех и народ, бывший с ним, из засады.
Jdg 9:36  Гаал, увидев народ, говорит Зевулу: вот, народ спускается с вершины гор. А Зевул сказал ему: тень гор тебе кажется людьми.
Jdg 9:37  Гаал опять говорил и сказал: вот, народ спускается с возвышенности, и один отряд идет от дуба Меонним.
Jdg 9:38  И сказал ему Зевул: где уста твои, которые говорили: `кто Авимелех, чтобы мы стали служить ему?' Это тот народ, который ты пренебрегал; выходи теперь и сразись с ним.
Jdg 9:39  И пошел Гаал впереди жителей Сихемских и сразился с Авимелехом.
Jdg 9:40  И погнался за ним Авимелех, и побежал он от него, и много пало убитых до самых ворот города.
Jdg 9:41  И остался Авимелех в Аруме, а Гаала и братьев его Зевул выгнал, чтоб они не жили в Сихеме.
Jdg 9:42  На другой день вышел народ в поле, и донесли о сем Авимелеху.
Jdg 9:43  Он взял свой народ и разделил его на три отряда и поставил в засаду в поле. И увидев, что народ вышел из города, восстал на них и побил их.
Jdg 9:44  Между тем как Авимелех и отряды, бывшие с ним, приступили и стали у ворот городских, другие два отряда напали на всех, бывших в поле, и убивали их.
Jdg 9:45  И сражался Авимелех с городом весь тот день, и взял город, и побил народ, бывший в нем, и разрушил город и засеял его солью.
Jdg 9:46  Услышав об этом, все бывшие в башне Сихемской ушли в башню капища [Ваал-Верифа].
Jdg 9:47  Авимелеху донесено, что собрались [туда] все бывшие в башне Сихемской.
Jdg 9:48  И пошел Авимелех на гору Селмон, сам и весь народ, бывший с ним, и взял Авимелех топоры с собою и нарубил сучьев древесных, и положил на плечи свои, и сказал народу, бывшему с ним: вы видели, что я делал; скорее делайте и вы то же, что я.
Jdg 9:49  И нарубил каждый из всего народа сучьев, и пошли за Авимелехом, и положили к башне, и сожгли посредством их башню огнем, и умерли все бывшие в башне Сихемской, около тысячи мужчин и женщин.
Jdg 9:50  Потом пошел Авимелех в Тевец и осадил Тевец и взял его.
Jdg 9:51  Среди города была крепкая башня, и убежали туда все мужчины и женщины и все жители города, и заперлись и взошли на кровлю башни.
Jdg 9:52  Авимелех пришел к башне и окружил ее и подошел к дверям башни, чтобы сжечь ее огнем.
Jdg 9:53  Тогда одна женщина бросила обломок жернова на голову Авимелеху и проломила ему череп.
Jdg 9:54  [Авимелех] тотчас призвал отрока, оруженосца своего, и сказал ему: обнажи меч твой и умертви меня, чтобы не сказали обо мне: женщина убила его. И пронзил его отрок его, и он умер.
Jdg 9:55  Израильтяне, видя, что умер Авимелех, пошли каждый в свое место.
Jdg 9:56  Так воздал Бог Авимелеху за злодеяние, которое он сделал отцу своему, убив семьдесят братьев своих.
Jdg 9:57  И все злодеяния жителей Сихемских обратил Бог на голову их; и постигло их проклятие Иофама, сына Иероваалова.
Jdg 10:1  После Авимелеха восстал для спасения Израиля Фола, сын Фуи, сына Додова, из колена Иссахарова. Он жил в Шамире на горе Ефремовой.
Jdg 10:2  Он был судьею Израиля двадцать три года, и умер, и погребен в Шамире.
Jdg 10:3  После него восстал Иаир из Галаада и был судьею Израиля двадцать два года.
Jdg 10:4  У него было тридцать сына, ездивших на тридцати молодых ослах, и тридцать города было у них; их до сего дня называют селениями Иаира, что в земле Галаадской.
Jdg 10:5  И умер Иаир и погребен в Камоне.
Jdg 10:6  Сыны Израилевы продолжали делать злое пред очами Господа и служили Ваалам и Астартам, и богам Арамейским, и богам Сидонским, и богам Моавитским, и богам Аммонитским, и богам Филистимским; а Господа оставили и не служили Ему.
Jdg 10:7  И воспылал гнев Господа на Израиля, и Он предал их в руки Филистимлян и в руки Аммонитян;
Jdg 10:8  они теснили и мучили сынов Израилевых с того года восемнадцать лет, всех сынов Израилевых по ту сторону Иордана в земле Аморрейской, которая в Галааде.
Jdg 10:9  Наконец Аммонитяне перешли Иордан, чтобы вести войну с Иудою и Вениамином и с домом Ефремовым. И весьма тесно было сынам Израиля.
Jdg 10:10  И возопили сыны Израилевы к Господу, и говорили: согрешили мы пред Тобою, потому что оставили Бога нашего и служили Ваалам.
Jdg 10:11  И сказал Господь сынам Израилевым: не угнетали ли вас Египтяне, и Аморреи, и Аммонитяне, и Филистимляне,
Jdg 10:12  и Сидоняне, и Амаликитяне, и Моавитяне, и когда вы взывали ко Мне, не спасал ли Я вас от рук их?
Jdg 10:13  А вы оставили Меня и стали служить другим богам; за то Я не буду уже спасать вас:
Jdg 10:14  пойдите, взывайте к богам, которых вы избрали, пусть они спасают вас в тесное для вас время.
Jdg 10:15  И сказали сыны Израилевы Господу: согрешили мы; делай с нами все, что Тебе угодно, только избавь нас ныне.
Jdg 10:16  И отвергли от себя чужих богов и стали служить Господу. И не потерпела душа Его страдания Израилева.
Jdg 10:17  Аммонитяне собрались и расположились станом в Галааде; собрались также сыны Израилевы и стали станом в Массифе.
Jdg 10:18  Народ [и] князья Галаадские сказали друг другу: кто начнет войну против Аммонитян, тот будет начальником всех жителей Галаадских.
Jdg 11:1  Иеффай Галаадитянин был человек храбрый. Он был сын блудницы; от Галаада родился Иеффай.
Jdg 11:2  И жена Галаадова родила ему сыновей. Когда возмужали сыновья жены, изгнали они Иеффая, сказав ему: ты не наследник в доме отца нашего, потому что ты сын другой женщины.
Jdg 11:3  И убежал Иеффай от братьев своих и жил в земле Тов; и собрались к Иеффаю праздные люди и выходили с ним.
Jdg 11:4  Чрез несколько времени Аммонитяне пошли войною на Израиля.
Jdg 11:5  Во время войны Аммонитян с Израильтянами пришли старейшины Галаадские взять Иеффая из земли Тов
Jdg 11:6  и сказали Иеффаю: приди, будь у нас вождем, и сразимся с Аммонитянами.
Jdg 11:7  Иеффай сказал старейшинам Галаадским: не вы ли возненавидели меня и выгнали из дома отца моего? зачем же пришли ко мне ныне, когда вы в беде?
Jdg 11:8  Старейшины Галаадские сказали Иеффаю: для того мы теперь пришли к тебе, чтобы ты пошел с нами и сразился с Аммонитянами и был у нас начальником всех жителей Галаадских.
Jdg 11:9  И сказал Иеффай старейшинам Галаадским: если вы возвратите меня, чтобы сразиться с Аммонитянами, и Господь предаст мне их, то останусь ли я у вас начальником?
Jdg 11:10  Старейшины Галаадские сказали Иеффаю: Господь да будет свидетелем между нами, что мы сделаем по слову твоему!
Jdg 11:11  И пошел Иеффай со старейшинами Галаадскими, и народ поставил его над собою начальником и вождем, и Иеффай произнес все слова свои пред лицем Господа в Массифе.
Jdg 11:12  И послал Иеффай послов к царю Аммонитскому сказать: что тебе до меня, что ты пришел ко мне воевать на земле моей?
Jdg 11:13  Царь Аммонитский сказал послам Иеффая: Израиль, когда шел из Египта, взял землю мою от Арнона до Иавока и Иордана; итак возврати мне ее с миром.
Jdg 11:14  Иеффай в другой раз послал послов к царю Аммонитскому,
Jdg 11:15  сказать ему: так говорит Иеффай: Израиль не взял земли Моавитской и земли Аммонитской;
Jdg 11:16  ибо когда шли из Египта, Израиль пошел в пустыню к Чермному морю и пришел в Кадес;
Jdg 11:17  оттуда послал Израиль послов к царю Едомскому сказать: `позволь мне пройти землею твоею'; но царь Едомский не послушал; и к царю Моавитскому он посылал, но и тот не согласился; посему Израиль оставался в Кадесе.
Jdg 11:18  И пошел пустынею, и миновал землю Едомскую и землю Моавитскую, и, придя к восточному пределу земли Моавитской, расположился станом за Арноном; но не входил в пределы Моавитские, ибо Арнон есть предел Моава.
Jdg 11:19  И послал Израиль послов к Сигону, царю Аморрейскому, царю Есевонскому, и сказал ему Израиль: позволь нам пройти землею твоею в свое место.
Jdg 11:20  Но Сигон не согласился пропустить Израиля чрез пределы свои, и собрал Сигон весь народ свой, и расположился станом в Иааце, и сразился с Израилем.
Jdg 11:21  И предал Господь Бог Израилев Сигона и весь народ его в руки Израилю, и он побил их; и получил Израиль в наследие всю землю Аморрея, жившего в земле той;
Jdg 11:22  и получили они в наследие все пределы Аморрея от Арнона до Иавока и от пустыни до Иордана.
Jdg 11:23  Итак Господь Бог Израилев изгнал Аморрея от лица народа Своего Израиля, а ты хочешь взять его наследие?
Jdg 11:24  Не владеешь ли ты тем, что дал тебе Хамос, бог твой? И мы владеем всем тем, что дал нам в наследие Господь Бог наш.
Jdg 11:25  Разве ты лучше Валака, сына Сепфорова, царя Моавитского? Ссорился ли он с Израилем, или воевал ли с ними?
Jdg 11:26  Израиль уже живет триста лет в Есевоне и в зависящих от него [городах], в Ароере и зависящих от него [городах], и во всех городах, которые близ Арнона; для чего вы в то время не отнимали [их]?
Jdg 11:27  А я не виновен пред тобою, и ты делаешь мне зло, выступив против меня войною. Господь Судия да будет ныне судьею между сынами Израиля и между Аммонитянами!
Jdg 11:28  Но царь Аммонитский не послушал слов Иеффая, с которыми он посылал к нему.
Jdg 11:29  И был на Иеффае Дух Господень, и прошел он Галаад и Манассию, и прошел Массифу Галаадскую, и из Массифы Галаадской пошел к Аммонитянам.
Jdg 11:30  И дал Иеффай обет Господу и сказал: если Ты предашь Аммонитян в руки мои,
Jdg 11:31  то по возвращении моем с миром от Аммонитян, что выйдет из ворот дома моего навстречу мне, будет Господу, и вознесу сие на всесожжение.
Jdg 11:32  И пришел Иеффай к Аммонитянам--сразиться с ними, и предал их Господь в руки его;
Jdg 11:33  и поразил их поражением весьма великим, от Ароера до Минифа двадцать городов, и до Авель-Керамима, и смирились Аммонитяне пред сынами Израилевыми.
Jdg 11:34  И пришел Иеффай в Массифу в дом свой, и вот, дочь его выходит навстречу ему с тимпанами и ликами: она была у него только одна, и не было у него еще ни сына, ни дочери.
Jdg 11:35  Когда он увидел ее, разодрал одежду свою и сказал: ах, дочь моя! ты сразила меня; и ты в числе нарушителей покоя моего! я отверз [о тебе] уста мои пред Господом и не могу отречься.
Jdg 11:36  Она сказала ему: отец мой! ты отверз уста твои пред Господом--и делай со мною то, что произнесли уста твои, когда Господь совершил чрез тебя отмщение врагам твоим Аммонитянам.
Jdg 11:37  И сказала отцу своему: сделай мне только вот что: отпусти меня на два месяца; я пойду, взойду на горы и оплачу девство мое с подругами моими.
Jdg 11:38  Он сказал: пойди. И отпустил ее на два месяца. Она пошла с подругами своими и оплакивала девство свое в горах.
Jdg 11:39  По прошествии двух месяцев она возвратилась к отцу своему, и он совершил над нею обет свой, который дал, и она не познала мужа. И вошло в обычай у Израиля,
Jdg 11:40  что ежегодно дочери Израилевы ходили оплакивать дочь Иеффая Галаадитянина, четыре дня в году.
Jdg 12:1  Ефремляне собрались и перешли в Севину и сказали Иеффаю: для чего ты ходил воевать с Аммонитянами, а нас не позвал с собою? мы сожжем дом твой огнем и с тобою вместе.
Jdg 12:2  Иеффай сказал им: я и народ мой имели с Аммонитянами сильную ссору; я звал вас, но вы не спасли меня от руки их;
Jdg 12:3  видя, что ты не спасаешь меня, я подверг опасности жизнь мою и пошел на Аммонитян, и предал их Господь в руки мои; зачем же вы пришли ныне воевать со мною?
Jdg 12:4  И собрал Иеффай всех жителей Галаадских и сразился с Ефремлянами, и побили жители Галаадские Ефремлян, говоря: вы беглецы Ефремовы, Галаад же среди Ефрема и среди Манассии.
Jdg 12:5  И перехватили Галаадитяне переправу чрез Иордан от Ефремлян, и когда кто из уцелевших Ефремлян говорил: `позвольте мне переправиться', то жители Галаадские говорили ему: не Ефремлянин ли ты? Он говорил: нет.
Jdg 12:6  Они говорили ему `скажи: шибболет', а он говорил: `сибболет', и не мог иначе выговорить. Тогда они, взяв его, заколали у переправы чрез Иордан. И пало в то время из Ефремлян сорок две тысячи.
Jdg 12:7  Иеффай был судьею Израиля шесть лет, и умер Иеффай Галаадитянин и погребен в одном из городов Галаадских.
Jdg 12:8  После него был судьею Израиля Есевон из Вифлеема.
Jdg 12:9  У него было тридцать сыновей, и тридцать дочерей отпустил он из дома [в замужество], а тридцать дочерей взял со стороны за сыновей своих, и был судьею Израиля семь лет.
Jdg 12:10  И умер Есевон и погребен в Вифлееме.
Jdg 12:11  После него был судьею Израиля Елон Завулонянин и судил Израиля десять лет.
Jdg 12:12  И умер Елон Завулонянин и погребен в Аиалоне, в земле Завулоновой.
Jdg 12:13  После него был судьею Израиля Авдон, сын Гиллела, Пирафонянин.
Jdg 12:14  У него было сорок сыновей и тридцать внуков, ездивших на семидесяти молодых ослах; он судил Израиля восемь лет.
Jdg 12:15  И умер Авдон, сын Гиллела, Пирафонянин, и погребен в Пирафоне в земле Ефремовой, на горе Амаликовой.
Jdg 13:1  Сыны Израилевы продолжали делать злое пред очами Господа, и предал их Господь в руки Филистимлян на сорок лет.
Jdg 13:2  В то время был человек из Цоры, от племени Данова, именем Маной; жена его была неплодна и не рождала.
Jdg 13:3  И явился Ангел Господень жене и сказал ей: вот, ты неплодна и не рождаешь; но зачнешь, и родишь сына;
Jdg 13:4  итак берегись, не пей вина и сикера, и не ешь ничего нечистого;
Jdg 13:5  ибо вот, ты зачнешь и родишь сына, и бритва не коснется головы его, потому что от самого чрева младенец сей будет назорей Божий, и он начнет спасать Израиля от руки Филистимлян.
Jdg 13:6  Жена пришла и сказала мужу своему: человек Божий приходил ко мне, которого вид, как вид Ангела Божия, весьма почтенный; я не спросила его, откуда он, и он не сказал мне имени своего;
Jdg 13:7  он сказал мне: `вот, ты зачнешь и родишь сына; итак не пей вина и сикера и не ешь ничего нечистого, ибо младенец от самого чрева до смерти своей будет назорей Божий'.
Jdg 13:8  Маной помолился Господу и сказал: Господи! пусть придет опять к нам человек Божий, которого посылал Ты, и научит нас, что нам делать с имеющим родиться младенцем.
Jdg 13:9  И услышал Бог голос Маноя, и Ангел Божий опять пришел к жене, когда она была в поле, и Маноя, мужа ее, не было с нею.
Jdg 13:10  Жена тотчас побежала и известила мужа своего и сказала ему: вот, явился мне человек, приходивший ко мне тогда.
Jdg 13:11  Маной встал и пошел с женою своею, и пришел к тому человеку и сказал ему: ты ли тот человек, который говорил с сею женщиною? [Ангел] сказал: я.
Jdg 13:12  И сказал Маной: итак, если исполнится слово твое, как нам поступать с младенцем сим и что делать с ним?
Jdg 13:13  Ангел Господень сказал Маною: пусть он остерегается всего, о чем я сказал жене;
Jdg 13:14  пусть не ест ничего, что производит виноградная лоза; пусть не пьет вина и сикера и не ест ничего нечистого и соблюдает все, что я приказал ей.
Jdg 13:15  И сказал Маной Ангелу Господню: позволь удержать тебя, пока мы изготовим для тебя козленка.
Jdg 13:16  Ангел Господень сказал Маною: хотя бы ты и удержал меня, но я не буду есть хлеба твоего; если же хочешь совершить всесожжение Господу, то вознеси его. Маной же не знал, что это Ангел Господень.
Jdg 13:17  И сказал Маной Ангелу Господню: как тебе имя? чтобы нам прославить тебя, когда исполнится слово твое.
Jdg 13:18  Ангел Господень сказал ему: что ты спрашиваешь об имени моем? оно чудно.
Jdg 13:19  И взял Маной козленка и хлебное приношение и вознес Господу на камне. И сделал Он чудо, которое видели Маной и жена его.
Jdg 13:20  Когда пламень стал подниматься от жертвенника к небу, Ангел Господень поднялся в пламени жертвенника. Видя это, Маной и жена его пали лицем на землю.
Jdg 13:21  И невидим стал Ангел Господень Маною и жене его. Тогда Маной узнал, что это Ангел Господень.
Jdg 13:22  И сказал Маной жене своей: верно мы умрем, ибо видели мы Бога.
Jdg 13:23  Жена его сказала ему: если бы Господь хотел умертвить нас, то не принял бы от рук наших всесожжения и хлебного приношения, и не показал бы нам всего того, и теперь не открыл бы нам сего.
Jdg 13:24  И родила жена сына, и нарекла имя ему: Самсон. И рос младенец, и благословлял его Господь.
Jdg 13:25  И начал Дух Господень действовать в нем в стане Дановом, между Цорою и Естаолом.
Jdg 14:1  И пошел Самсон в Фимнафу и увидел в Фимнафе женщину из дочерей Филистимских.
Jdg 14:2  Он пошел и объявил отцу своему и матери своей и сказал: я видел в Фимнафе женщину из дочерей Филистимских; возьмите ее мне в жену.
Jdg 14:3  Отец и мать его сказали ему: разве нет женщин между дочерями братьев твоих и во всем народе моем, что ты идешь взять жену у Филистимлян необрезанных? И сказал Самсон отцу своему: ее возьми мне, потому что она мне понравилась.
Jdg 14:4  Отец его и мать его не знали, что это от Господа, и что он ищет случая [отмстить] Филистимлянам. А в то время Филистимляне господствовали над Израилем.
Jdg 14:5  И пошел Самсон с отцом своим и с матерью своею в Фимнафу, и когда подходили к виноградникам Фимнафским, вот, молодой лев рыкая [идет] навстречу ему.
Jdg 14:6  И сошел на него Дух Господень, и он растерзал [льва] как козленка; а в руке у него ничего не было. И не сказал отцу своему и матери своей, что он сделал.
Jdg 14:7  И пришел и поговорил с женщиною, и она понравилась Самсону.
Jdg 14:8  Спустя несколько дней, опять пошел он, чтобы взять ее, и зашел посмотреть труп льва, и вот, рой пчел в трупе львином и мед.
Jdg 14:9  Он взял его в руки свои и пошел, и ел дорогою; и когда пришел к отцу своему и матери своей, дал и им, и они ели; но не сказал им, что из львиного трупа взял мед сей.
Jdg 14:10  И пришел отец его к женщине, и сделал там Самсон пир, как обыкновенно делают женихи.
Jdg 14:11  И как там увидели его, выбрали тридцать брачных друзей, которые были бы при нем.
Jdg 14:12  И сказал им Самсон: загадаю я вам загадку; если вы отгадаете мне ее в семь дней пира и отгадаете верно, то я дам вам тридцать синдонов и тридцать перемен одежд;
Jdg 14:13  если же не сможете отгадать мне, то вы дайте мне тридцать синдонов и тридцать перемен одежд. Они сказали ему: загадай загадку твою, послушаем.
Jdg 14:14  И сказал им: из ядущего вышло ядомое, и из сильного вышло сладкое. И не могли отгадать загадку в три дня.
Jdg 14:15  В седьмой день сказали они жене Самсоновой: уговори мужа твоего, чтоб он разгадал нам загадку; иначе сожжем огнем тебя и дом отца твоего; разве вы призвали нас, чтоб обобрать нас?
Jdg 14:16  И плакала жена Самсонова пред ним и говорила: ты ненавидишь меня и не любишь; ты загадал загадку сынам народа моего, а мне не разгадаешь ее. Он сказал ей: отцу моему и матери моей не разгадал ее; и тебе ли разгадаю?
Jdg 14:17  И плакала она пред ним семь дней, в которые продолжался у них пир. Наконец в седьмой день разгадал ей, ибо она усиленно просила его. А она разгадала загадку сынам народа своего.
Jdg 14:18  И в седьмой день до захождения солнечного сказали ему граждане: что слаще меда, и что сильнее льва! Он сказал им: если бы вы не орали на моей телице, то не отгадали бы моей загадки.
Jdg 14:19  И сошел на него Дух Господень, и пошел он в Аскалон, и, убив там тридцать человек, снял с них одежды, и отдал перемены [платья] их разгадавшим загадку. И воспылал гнев его, и ушел он в дом отца своего.
Jdg 14:20  А жена Самсонова вышла за брачного друга его, который был при нем другом.
Jdg 15:1  Чрез несколько дней, во время жатвы пшеницы, пришел Самсон повидаться с женою своею, принеся с собою козленка; и когда сказал: `войду к жене моей в спальню', отец ее не дал ему войти.
Jdg 15:2  И сказал отец ее: я подумал, что ты возненавидел ее, и я отдал ее другу твоему; вот, меньшая сестра красивее ее; пусть она будет тебе вместо ее.
Jdg 15:3  Но Самсон сказал им: теперь я буду прав пред Филистимлянами, если сделаю им зло.
Jdg 15:4  И пошел Самсон, и поймал триста лисиц, и взял факелы, и связал хвост с хвостом, и привязал по факелу между двумя хвостами;
Jdg 15:5  и зажег факелы, и пустил их на жатву Филистимскую, и выжег и копны и нежатый хлеб, и виноградные сады [и] масличные.
Jdg 15:6  И говорили Филистимляне: кто это сделал? И сказали: Самсон, зять Фимнафянина, ибо этот взял жену его и отдал другу его. И пошли Филистимляне и сожгли огнем ее и отца ее.
Jdg 15:7  Самсон сказал им: хотя вы сделали это, но я отмщу вам самим и тогда только успокоюсь.
Jdg 15:8  И перебил он им голени и бедра, и пошел и засел в ущелье скалы Етама.
Jdg 15:9  И пошли Филистимляне, и расположились станом в Иудее, и протянулись до Лехи.
Jdg 15:10  И сказали жители Иудеи: за что вы вышли против нас? Они сказали: мы пришли связать Самсона, чтобы поступить с ним, как он поступил с нами.
Jdg 15:11  И пошли три тысячи человек из Иудеи к ущелью скалы Етама и сказали Самсону: разве ты не знаешь, что Филистимляне господствуют над нами? что ты это сделал нам? Он сказал им: как они со мною поступили, так и я поступил с ними.
Jdg 15:12  И сказали ему: мы пришли связать тебя, чтобы отдать тебя в руки Филистимлянам. И сказал им Самсон: поклянитесь мне, что вы не убьете меня.
Jdg 15:13  И сказали ему: нет, мы только свяжем тебя и отдадим тебя в руки их, а умертвить не умертвим. И связали его двумя новыми веревками и повели его из ущелья.
Jdg 15:14  Когда он подошел к Лехе, Филистимляне с криком встретили его. И сошел на него Дух Господень, и веревки, бывшие на руках его, сделались, как перегоревший лен, и упали узы его с рук его.
Jdg 15:15  Нашел он свежую ослиную челюсть и, протянув руку свою, взял ее, и убил ею тысячу человек.
Jdg 15:16  И сказал Самсон: челюстью ослиною толпу, две толпы, челюстью ослиною убил я тысячу человек.
Jdg 15:17  Сказав это, бросил челюсть из руки своей и назвал то место: Рамаф-Лехи.
Jdg 15:18  И почувствовал сильную жажду и воззвал к Господу и сказал: Ты соделал рукою раба Твоего великое спасение сие; а теперь умру я от жажды, и попаду в руки необрезанных.
Jdg 15:19  И разверз Бог ямину в Лехе, и потекла из нее вода. Он напился, и возвратился дух его, и он ожил; оттого и наречено имя месту сему: `Источник взывающего', который в Лехе до сего дня.
Jdg 15:20  И был он судьею Израиля во дни Филистимлян двадцать лет.
Jdg 16:1  Пришел однажды Самсон в Газу и, увидев там блудницу, вошел к ней.
Jdg 16:2  Жителям Газы сказали: Самсон пришел сюда. И ходили они кругом, и подстерегали его всю ночь в воротах города, и таились всю ночь, говоря: до света утреннего [подождем, и] убьем его.
Jdg 16:3  А Самсон спал до полуночи; в полночь же встав, схватил двери городских ворот с обоими косяками, поднял их вместе с запором, положил на плечи свои и отнес их на вершину горы, которая на пути к Хеврону.
Jdg 16:4  После того полюбил он одну женщину, жившую на долине Сорек; имя ей Далида.
Jdg 16:5  К ней пришли владельцы Филистимские и говорят ей: уговори его, и выведай, в чем великая сила его и как нам одолеть его, чтобы связать его и усмирить его; а мы дадим тебе за то каждый тысячу сто [сиклей] серебра.
Jdg 16:6  И сказала Далида Самсону: скажи мне, в чем великая сила твоя и чем связать тебя, чтобы усмирить тебя?
Jdg 16:7  Самсон сказал ей: если свяжут меня семью сырыми тетивами, которые не засушены, то я сделаюсь бессилен и буду как и прочие люди.
Jdg 16:8  И принесли ей владельцы Филистимские семь сырых тетив, которые не засохли, и она связала его ими.
Jdg 16:9  (Между тем один скрытно сидел у нее в спальне.) И сказала ему: Самсон! Филистимляне [идут] на тебя. Он разорвал тетивы, как разрывают нитку из пакли, когда пережжет ее огонь. И не узнана сила его.
Jdg 16:10  И сказала Далида Самсону: вот, ты обманул меня и говорил мне ложь; скажи же теперь мне, чем связать тебя?
Jdg 16:11  Он сказал ей: если свяжут меня новыми веревками, которые не были в деле, то я сделаюсь бессилен и буду, как прочие люди.
Jdg 16:12  Далида взяла новые веревки и связала его и сказала ему: Самсон! Филистимляне [идут] на тебя. (Между тем один скрытно сидел в спальне.) И сорвал он их с рук своих, как нитки.
Jdg 16:13  И сказала Далида Самсону: все ты обманываешь меня и говоришь мне ложь; скажи мне, чем бы связать тебя? Он сказал ей: если ты воткешь семь кос головы моей в ткань [и прибьешь ее гвоздем к ткальной колоде].
Jdg 16:14  и прикрепила их к колоде, и сказала ему: Филистимляне [идут] на тебя, Самсон! Он пробудился от сна своего и выдернул ткальную колоду вместе с тканью.
Jdg 16:15  И сказала ему [Далида]: как же ты говоришь: `люблю тебя', а сердце твое не со мною? вот, ты трижды обманул меня, и не сказал мне, в чем великая сила твоя.
Jdg 16:16  И как она словами своими тяготила его всякий день и мучила его, то душе его тяжело стало до смерти.
Jdg 16:17  И он открыл ей все сердце свое, и сказал ей: бритва не касалась головы моей, ибо я назорей Божий от чрева матери моей; если же остричь меня, то отступит от меня сила моя; я сделаюсь слаб и буду, как прочие люди.
Jdg 16:18  Далида, видя, что он открыл ей все сердце свое, послала и звала владельцев Филистимских, сказав им: идите теперь; он открыл мне все сердце свое. И пришли к ней владельцы Филистимские и принесли серебро в руках своих.
Jdg 16:19  И усыпила его [Далида] на коленях своих, и призвала человека, и велела ему остричь семь кос головы его. И начал он ослабевать, и отступила от него сила его.
Jdg 16:20  Она сказала: Филистимляне [идут] на тебя, Самсон! Он пробудился от сна своего, и сказал: пойду, как и прежде, и освобожусь. А не знал, что Господь отступил от него.
Jdg 16:21  Филистимляне взяли его и выкололи ему глаза, привели его в Газу и оковали его двумя медными цепями, и он молол в доме узников.
Jdg 16:22  Между тем волосы на голове его начали расти, где они были острижены.
Jdg 16:23  Владельцы Филистимские собрались, чтобы принести великую жертву Дагону, богу своему, и повеселиться, и сказали: бог наш предал Самсона, врага нашего, в руки наши.
Jdg 16:24  Также и народ, видя его, прославлял бога своего, говоря: бог наш предал в руки наши врага нашего и опустошителя земли нашей, который побил многих из нас.
Jdg 16:25  И когда развеселилось сердце их, сказали: позовите Самсона, пусть он позабавит нас. И призвали Самсона из дома узников, и он забавлял их, и поставили его между столбами.
Jdg 16:26  И сказал Самсон отроку, который водил его за руку: подведи меня, чтобы ощупать мне столбы, на которых утвержден дом, и прислониться к ним.
Jdg 16:27  Дом же был полон мужчин и женщин; там были все владельцы Филистимские, и на кровле было до трех тысяч мужчин и женщин, смотревших на забавляющего [их] Самсона.
Jdg 16:28  И воззвал Самсон к Господу и сказал: Господи Боже! вспомни меня и укрепи меня только теперь, о Боже! чтобы мне в один раз отмстить Филистимлянам за два глаза мои.
Jdg 16:29  И сдвинул Самсон с места два средних столба, на которых утвержден был дом, упершись в них, в один правою рукою своею, а в другой левою.
Jdg 16:30  И сказал Самсон: умри, душа моя, с Филистимлянами! И уперся [всею] силою, и обрушился дом на владельцев и на весь народ, бывший в нем. И было умерших, которых умертвил [Самсон] при смерти своей, более, нежели сколько умертвил он в жизни своей.
Jdg 16:31  И пришли братья его и весь дом отца его, и взяли его, и пошли и похоронили его между Цорою и Естаолом, во гробе Маноя, отца его. Он был судьею Израиля двадцать лет.
Jdg 17:1  Был некто на горе Ефремовой, именем Миха.
Jdg 17:2  Он сказал матери своей: тысяча сто [сиклей] серебра, которые у тебя взяты и за которые ты при мне изрекла проклятие, это серебро у меня, я взял его. Мать его сказала: благословен сын мой у Господа!
Jdg 17:3  И возвратил он матери своей тысячу сто [сиклей] серебра. И сказала мать его: это серебро я от себя посвятила Господу для сына моего, чтобы сделать из него истукан и литый кумир; итак отдаю оное тебе.
Jdg 17:4  Но он возвратил серебро матери своей. Мать его взяла двести [сиклей] серебра и отдала их плавильщику. Он сделал из них истукан и литый кумир, который и находился в доме Михи.
Jdg 17:5  И был у Михи дом Божий. И сделал он ефод и терафим и посвятил одного из сыновей своих, чтоб он был у него священником.
Jdg 17:6  В те дни не было царя у Израиля; каждый делал то, что ему казалось справедливым.
Jdg 17:7  Один юноша из Вифлеема Иудейского, из колена Иудина, левит, тогда жил там;
Jdg 17:8  этот человек пошел из города Вифлеема Иудейского, чтобы пожить, где случится, и идя дорогою, пришел на гору Ефремову к дому Михи.
Jdg 17:9  И сказал ему Миха: откуда ты идешь? Он сказал ему: я левит из Вифлеема Иудейского и иду пожить, где случится.
Jdg 17:10  И сказал ему Миха: останься у меня и будь у меня отцом и священником; я буду давать тебе по десяти [сиклей] серебра на год, потребное одеяние и пропитание.
Jdg 17:11  Левит пошел к нему и согласился левит остаться у этого человека, и был юноша у него, как один из сыновей его.
Jdg 17:12  Миха посвятил левита, и этот юноша был у него священником и жил в доме у Михи.
Jdg 17:13  И сказал Миха: теперь я знаю, что Господь будет мне благотворить, потому что левит у меня священником.
Jdg 18:1  В те дни не было царя у Израиля; и в те дни колено Даново искало себе удела, где бы поселиться, потому что дотоле не выпало ему [полного] удела между коленами Израилевыми.
Jdg 18:2  И послали сыны Дановы от племени своего пять человек, мужей сильных, из Цоры и Естаола, чтоб осмотреть землю и узнать ее, и сказали им: пойдите, узнайте землю. Они пришли на гору Ефремову к дому Михи и ночевали там.
Jdg 18:3  Находясь у дома Михи, узнали они голос молодого левита и зашли туда и спрашивали его: кто тебя привел сюда? что ты здесь делаешь и зачем ты здесь?
Jdg 18:4  Он сказал им: то и то сделал для меня Миха, нанял меня, и я у него священником.
Jdg 18:5  Они сказали ему: вопроси Бога, чтобы знать нам, успешен ли будет путь наш, в который мы идем.
Jdg 18:6  Священник сказал им: идите с миром; пред Господом путь ваш, в который вы идете.
Jdg 18:7  И пошли те пять мужей, и пришли в Лаис, и увидели народ, который в нем, что он живет покойно, по обычаю Сидонян, тих и беспечен, и что не было в земле той, кто обижал бы в чем, или имел бы власть: от Сидонян они жили далеко, и ни с кем не было у них никакого дела.
Jdg 18:8  И возвратились к братьям своим в Цору и Естаол, и сказали им братья их: с чем вы?
Jdg 18:9  Они сказали: встанем и пойдем на них; мы видели землю, она весьма хороша; а вы задумались: не медлите пойти и взять в наследие ту землю;
Jdg 18:10  когда пойдете вы, придете к народу беспечному, и земля та обширна; Бог предает ее в руки ваши; это такое место, где нет ни в чем недостатка, что [получается] от земли.
Jdg 18:11  И отправились оттуда из колена Данова, из Цоры и Естаола, шестьсот мужей, препоясавшись воинским оружием.
Jdg 18:12  Они пошли и стали станом в Кириаф-Иариме, в Иудее. Посему и называют то место станом Дановым до сего дня. Он позади Кириаф--Иарима.
Jdg 18:13  Оттуда отправились они на гору Ефремову и пришли к дому Михи.
Jdg 18:14  И сказали те пять мужей, которые ходили осматривать землю Лаис, братьям своим: знаете ли, что в одном из домов сих есть ефод, терафим, истукан и литый кумир? итак подумайте, что сделать.
Jdg 18:15  И зашли туда, и вошли в дом молодого левита, в дом Михи, и приветствовали его.
Jdg 18:16  А шестьсот человек из сынов Дановых, перепоясанные воинским оружием, стояли у ворот.
Jdg 18:17  Пять же человек, ходивших осматривать землю, пошли, вошли туда, взяли истукан и ефод и терафим и литый кумир. Священник стоял у ворот с теми шестьюстами человек, препоясанных воинским оружием.
Jdg 18:18  Когда они вошли в дом Михи и взяли истукан, ефод, терафим и литый кумир, священник сказал им: что вы делаете?
Jdg 18:19  Они сказали ему: молчи, положи руку твою на уста твои и иди с нами и будь у нас отцом и священником; лучше ли тебе быть священником в доме одного человека, нежели быть священником в колене или в племени Израилевом?
Jdg 18:20  Священник обрадовался, и взял ефод, терафим и истукан, и пошел с народом.
Jdg 18:21  Они обратились и пошли, и отпустили детей, скот и тяжести вперед.
Jdg 18:22  Когда они удалились от дома Михи, жители домов соседних с домом Михи собрались и погнались за сынами Дана,
Jdg 18:23  и кричали сынам Дана. [Сыны Дановы] оборотились и сказали Михе: что тебе, что ты так кричишь?
Jdg 18:24  (Миха) сказал: вы взяли богов моих, которых я сделал, и священника, и ушли; чего еще более? как же вы говорите: что тебе?
Jdg 18:25  Сыны Дановы сказали ему: [молчи], чтобы мы не слышали голоса твоего; иначе некоторые из нас, рассердившись, нападут на вас, и ты погубишь себя и семейство твое.
Jdg 18:26  И пошли сыны Дановы путем своим; Миха же, видя, что они сильнее его, пошел назад и возвратился в дом свой.
Jdg 18:27  А [сыны Дановы] взяли то, что сделал Миха, и священника, который был у него, и пошли в Лаис, против народа спокойного и беспечного, и побили его мечом, а город сожгли огнем.
Jdg 18:28  Некому было помочь, потому что он был отдален от Сидона и ни с кем не имел дела. [Город сей] находился в долине, что близ Беф-Рехова. И построили [снова] город и поселились в нем,
Jdg 18:29  и нарекли имя городу: Дан, по имени отца своего Дана, сына Израилева; а прежде имя города тому было: Лаис.
Jdg 18:30  И поставили у себя сыны Дановы истукан; Ионафан же, сын Гирсона, сына Манассии, сам и сыновья его были священниками в колене Дановом до дня переселения [жителей той] земли;
Jdg 18:31  и имели у себя истукан, сделанный Михою, во все то время, когда дом Божий находился в Силоме.
Jdg 19:1  В те дни, когда не было царя у Израиля, жил один левит на склоне горы Ефремовой. Он взял себе наложницу из Вифлеема Иудейского.
Jdg 19:2  Наложница его поссорилась с ним и ушла от него в дом отца своего в Вифлеем Иудейский и была там четыре месяца.
Jdg 19:3  Муж ее встал и пошел за нею, чтобы поговорить к сердцу ее и возвратить ее к себе. С ним был слуга его и пара ослов. Она ввела его в дом отца своего.
Jdg 19:4  Отец этой молодой женщины, увидев его, с радостью встретил его, и удержал его тесть его, отец молодой женщины. И пробыл он у него три дня; они ели и пили и ночевали там.
Jdg 19:5  В четвертый день встали они рано, и он встал, чтоб идти. И сказал отец молодой женщины зятю своему: подкрепи сердце твое куском хлеба, и потом пойдете.
Jdg 19:6  Они остались, и оба вместе ели и пили. И сказал отец молодой женщины человеку тому: останься еще на ночь, и пусть повеселится сердце твое.
Jdg 19:7  Человек тот встал, было, чтоб идти, но тесть его упросил его, и он опять ночевал там.
Jdg 19:8  На пятый день встал он поутру, чтоб идти. И сказал отец молодой женщины той: подкрепи сердце твое [хлебом], и помедлите, доколе преклонится день. И ели оба они.
Jdg 19:9  И встал тот человек, чтоб идти, сам он, наложница его и слуга его. И сказал ему тесть его, отец молодой женщины: вот, день преклонился к вечеру, ночуйте, пожалуйте; вот, дню скоро конец, ночуй здесь, пусть повеселится сердце твое; завтра пораньше встанете в путь ваш, и пойдешь в дом твой.
Jdg 19:10  Но муж не согласился ночевать, встал и пошел; и пришел к Иевусу, что [ныне] Иерусалим; с ним пара навьюченных ослов и наложница его с ним.
Jdg 19:11  Когда они были близ Иевуса, день уже очень преклонился. И сказал слуга господину своему: зайдем в этот город Иевусеев и ночуем в нем.
Jdg 19:12  Господин его сказал ему: нет, не пойдем в город иноплеменников, которые не из сынов Израилевых, но дойдем до Гивы.
Jdg 19:13  И сказал слуге своему: дойдем до одного из сих мест и ночуем в Гиве, или в Раме.
Jdg 19:14  И пошли, и шли, и закатилось солнце подле Гивы Вениаминовой.
Jdg 19:15  И повернули они туда, чтобы пойти ночевать в Гиве. И пришел он и сел на улице в городе; но никто не приглашал их в дом для ночлега.
Jdg 19:16  И вот, идет один старик с работы своей с поля вечером; он родом был с горы Ефремовой и жил в Гиве. Жители же места сего были сыны Вениаминовы.
Jdg 19:17  Он, подняв глаза свои, увидел прохожего на улице городской. И сказал старик: куда идешь? и откуда ты пришел?
Jdg 19:18  Он сказал ему: мы идем из Вифлеема Иудейского к горе Ефремовой, откуда я; я ходил в Вифлеем Иудейский, а теперь иду к дому Господа; и никто не приглашает меня в дом;
Jdg 19:19  у нас есть и солома и корм для ослов наших; также хлеб и вино для меня и для рабы твоей и для сего слуги есть у рабов твоих; ни в чем нет недостатка.
Jdg 19:20  Старик сказал ему: будь спокоен: весь недостаток твой на мне, только не ночуй на улице.
Jdg 19:21  И ввел его в дом свой и дал корму ослам [его], а сами они омыли ноги свои и ели и пили.
Jdg 19:22  Тогда как они развеселили сердца свои, вот, жители города, люди развратные, окружили дом, стучались в двери и говорили старику, хозяину дома: выведи человека, вошедшего в дом твой, мы познаем его.
Jdg 19:23  Хозяин дома вышел к ним и сказал им: нет, братья мои, не делайте зла, когда человек сей вошел в дом мой, не делайте этого безумия;
Jdg 19:24  вот у меня дочь девица, и у него наложница, выведу я их, смирите их и делайте с ними, что вам угодно; а с человеком сим не делайте этого безумия.
Jdg 19:25  Но они не хотели слушать его. Тогда муж взял свою наложницу и вывел к ним на улицу. Они познали ее, и ругались над нею всю ночь до утра. И отпустили ее при появлении зари.
Jdg 19:26  И пришла женщина пред появлением зари, и упала у дверей дома того человека, у которого был господин ее, [и лежала] до света.
Jdg 19:27  Господин ее встал поутру, отворил двери дома и вышел, чтоб идти в путь свой: и вот, наложница его лежит у дверей дома, и руки ее на пороге.
Jdg 19:28  Он сказал ей: вставай, пойдем. Но ответа не было, [потому что она умерла]. Он положил ее на осла, встал и пошел в свое место.
Jdg 19:29  Придя в дом свой, взял нож и, взяв наложницу свою, разрезал ее по членам ее на двенадцать частей и послал во все пределы Израилевы.
Jdg 19:30  Всякий, видевший это, говорил: не бывало и не видано было подобного сему от дня исшествия сынов Израилевых из земли Египетской до сего дня. Обратите внимание на это, посоветуйтесь и скажите.
Jdg 20:1  И вышли все сыны Израилевы, и собралось [все] общество, как один человек, от Дана до Вирсавии, и земля Галаадская пред Господа в Массифу.
Jdg 20:2  И собрались начальники всего народа, все колена Израилевы, в собрание народа Божия, четыреста тысяч пеших, обнажающих меч.
Jdg 20:3  И сыны Вениаминовы услышали, что сыны Израилевы пришли в Массифу. И сказали сыны Израилевы: скажите, как происходило это зло?
Jdg 20:4  Левит, муж оной убитой женщины, отвечал и сказал: я с наложницею моею пришел ночевать в Гиву Вениаминову;
Jdg 20:5  и восстали на меня жители Гивы и окружили из-за меня дом ночью; меня намеревались убить, и наложницу мою замучили, так, что она умерла;
Jdg 20:6  я взял наложницу мою, разрезал ее и послал ее во все области владения Израилева, ибо они сделали беззаконное и срамное дело в Израиле;
Jdg 20:7  вот все вы, сыны Израилевы, рассмотрите это дело и решите здесь.
Jdg 20:8  И восстал весь народ, как один человек, и сказал: не пойдем никто в шатер свой и не возвратимся никто в дом свой;
Jdg 20:9  и вот что мы сделаем ныне с Гивою: [пойдем] на нее по жребию;
Jdg 20:10  и возьмем по десяти человек из ста от всех колен Израилевых, по сто от тысячи и по тысяче от тьмы, чтоб они принесли съестных припасов для народа, который пойдет против Гивы Вениаминовой, наказать ее за срамное дело, которое она сделала в Израиле.
Jdg 20:11  И собрались все Израильтяне против города единодушно, как один человек.
Jdg 20:12  И послали колена Израилевы во все колено Вениаминово сказать: какое это гнусное дело сделано у вас!
Jdg 20:13  Выдайте развращенных оных людей, которые в Гиве; мы умертвим их и искореним зло из Израиля. Но сыны Вениаминовы не хотели послушать голоса братьев своих, сынов Израилевых;
Jdg 20:14  а собрались сыны Вениаминовы из городов в Гиву, чтобы пойти войною против сынов Израилевых.
Jdg 20:15  И насчиталось в тот день сынов Вениаминовых, [собравшихся] из городов, двадцать шесть тысяч человек, обнажающих меч; кроме того, из жителей Гивы насчитано семьсот отборных;
Jdg 20:16  из всего народа сего было семьсот человек отборных, которые были левши, и все сии, бросая из пращей камни в волос, не бросали мимо.
Jdg 20:17  Израильтян же, кроме сынов Вениаминовых, насчиталось четыреста тысяч человек, обнажающих меч; все они были способны к войне.
Jdg 20:18  И встали и пошли в дом Божий, и вопрошали Бога и сказали сыны Израилевы: кто из нас прежде пойдет на войну с сынами Вениамина? И сказал Господь: Иуда [пойдет] впереди.
Jdg 20:19  И встали сыны Израилевы поутру и расположились станом подле Гивы;
Jdg 20:20  и выступили Израильтяне на войну против Вениамина, и стали сыны Израилевы в боевой порядок близ Гивы.
Jdg 20:21  И вышли сыны Вениаминовы из Гивы и положили в тот день двадцать две тысячи Израильтян на землю.
Jdg 20:22  Но народ Израильский ободрился, и опять стали в боевой порядок на том месте, где стояли в прежний день.
Jdg 20:23  И пошли сыны Израилевы, и плакали пред Господом до вечера, и вопрошали Господа: вступать ли мне еще в сражение с сынами Вениамина, брата моего? Господь сказал: идите против него.
Jdg 20:24  И подступили сыны Израилевы к сынам Вениамина во второй день.
Jdg 20:25  Вениамин вышел против них из Гивы во второй день, и еще положили на землю из сынов Израилевых восемнадцать тысяч человек, обнажающих меч.
Jdg 20:26  Тогда все сыны Израилевы и весь народ пошли и пришли в дом Божий и, сидя там, плакали пред Господом, и постились в тот день до вечера, и вознесли всесожжения и мирные жертвы пред Господом.
Jdg 20:27  И вопрошали сыны Израилевы Господа (в то время ковчег завета Божия находился там,
Jdg 20:28  и Финеес, сын Елеазара, сына Ааронова, предстоял пред ним): выходить ли мне еще на сражение с сынами Вениамина, брата моего, или нет? Господь сказал: идите; Я завтра предам его в руки ваши.
Jdg 20:29  И поставил Израиль засаду вокруг Гивы.
Jdg 20:30  И пошли сыны Израилевы на сынов Вениамина в третий день и стали в боевой порядок пред Гивою, как прежде.
Jdg 20:31  Сыны Вениаминовы выступили против народа и отдалились от города, и начали, как прежде, убивать из народа на дорогах, из которых одна идет к Вефилю, а другая к Гиве полем, и [убили] до тридцати человек из Израильтян.
Jdg 20:32  И сказали сыны Вениаминовы: они падают пред нами, как и прежде. А сыны Израилевы сказали: побежим от них и отвлечем их от города на дороги.
Jdg 20:33  И все Израильтяне встали с своего места и выстроились в Ваал-Фамаре. И засада Израилева устремилась из своего места, с западной стороны Гивы.
Jdg 20:34  И пришли пред Гиву десять тысяч человек отборных из всего Израиля, и началось жестокое сражение; но [сыны Вениамина] не знали, что предстоит им беда.
Jdg 20:35  И поразил Господь Вениамина пред Израильтянами, и положили в тот день Израильтяне из сынов Вениамина двадцать пять тысяч сто человек, обнажавших меч.
Jdg 20:36  Когда сыны Вениамина увидели, что они поражены, тогда Израильтяне уступили место сынам Вениамина, ибо надеялись на засаду, которую они поставили близ Гивы.
Jdg 20:37  Засада же поспешила и устремилась к Гиве, и вступила и поразила весь город мечом.
Jdg 20:38  Израильтяне поставили с засадою [условленным] знаком к нападению поднимающийся дым из города.
Jdg 20:39  Итак, когда Израильтяне отступили с места сражения, и Вениамин начал поражать и поверг Израильтян до тридцати человек и говорил: `опять падают они пред нами, как и в прежние сражения',
Jdg 20:40  тогда начал подниматься из города дым столбом. Вениамин оглянулся назад, и вот, [дым] от всего города восходит к небу.
Jdg 20:41  Израильтяне воротились, а Вениамин оробел, ибо увидел, что постигла его беда.
Jdg 20:42  И побежали они от Израильтян по дороге к пустыне; но сеча преследовала их, и выходившие из городов побивали их там;
Jdg 20:43  окружили Вениамина, и преследовали его до Менухи и поражали до самой восточной стороны Гивы.
Jdg 20:44  И пало из сынов Вениамина восемнадцать тысяч человек, людей сильных.
Jdg 20:45  [Оставшиеся] оборотились и побежали к пустыне, к скале Риммону, и побили еще [Израильтяне] на дорогах пять тысяч человек; и гнались за ними до Гидома и еще убили из них две тысячи человек.
Jdg 20:46  Всех же сынов Вениаминовых, павших в тот день, было двадцать пять тысяч человек, обнажавших меч, и все они были мужи сильные.
Jdg 20:47  И [обратились оставшиеся] и убежали в пустыню, к скале Риммону, шестьсот человек, и оставались там в каменной горе Риммоне четыре месяца.
Jdg 20:48  Израильтяне же опять пошли к сынам Вениаминовым и поразили их мечом, и людей в городе, и скот, и все, что ни встречалось, и все находившиеся [на пути] города сожгли огнем.
Jdg 21:1  И поклялись Израильтяне в Массифе, говоря: никто из нас не отдаст дочери своей сынам Вениамина в замужество.
Jdg 21:2  И пришел народ в дом Божий, и сидели там до вечера пред Богом, и подняли громкий вопль, и сильно плакали,
Jdg 21:3  и сказали: Господи, Боже Израилев! для чего случилось это в Израиле, что не стало теперь у Израиля одного колена?
Jdg 21:4  На другой день встал народ поутру, и устроили там жертвенник, и вознесли всесожжения и мирные жертвы.
Jdg 21:5  И сказали сыны Израилевы: кто не приходил в собрание пред Господа из всех колен Израилевых? Ибо великое проклятие [произнесено] было на тех, которые не пришли пред Господа в Массифу, и сказано было, что те преданы будут смерти.
Jdg 21:6  И сжалились сыны Израилевы над Вениамином, братом своим, и сказали: ныне отсечено одно колено от Израиля;
Jdg 21:7  как поступить нам с оставшимися из них [касательно] жен, когда мы поклялись Господом не давать им жен из дочерей наших?
Jdg 21:8  И сказали: нет ли кого из колен Израилевых, кто не приходил пред Господа в Массифу? И оказалось, что из Иависа Галаадского никто не приходил пред Господа в стан на собрание.
Jdg 21:9  И осмотрен народ, и вот, не было там ни одного из жителей Иависа Галаадского.
Jdg 21:10  И послало туда общество двенадцать тысяч человек, мужей сильных, и дали им приказание, говоря: идите и поразите жителей Иависа Галаадского мечом, и женщин и детей;
Jdg 21:11  и вот что сделайте: всякого мужчину и всякую женщину, познавшую ложе мужеское, предайте заклятию.
Jdg 21:12  И нашли они между жителями Иависа Галаадского четыреста девиц, не познавших ложа мужеского, и привели их в стан в Силом, что в земле Ханаанской.
Jdg 21:13  И послало все общество переговорить с сынами Вениамина, бывшими в скале Риммоне, и объявило им мир.
Jdg 21:14  Тогда возвратились сыны Вениамина, и дали им (Израильтяне) жен, которых оставили в живых из женщин Иависа Галаадского; но оказалось, что этого было недостаточно.
Jdg 21:15  Народ же сожалел о Вениамине, что Господь не сохранил целости колен Израилевых.
Jdg 21:16  И сказали старейшины общества: что нам делать с оставшимися [касательно] жен, ибо истреблены женщины у Вениамина?
Jdg 21:17  И сказали: наследственная земля пусть остается уцелевшим сынам Вениамина, чтобы не исчезло колено от Израиля;
Jdg 21:18  но мы не можем дать им жен из дочерей наших; ибо сыны Израилевы поклялись, говоря: проклят, кто даст жену Вениамину.
Jdg 21:19  И сказали: вот, каждый год бывает праздник Господень в Силоме, который на север от Вефиля и на восток от дороги, ведущей от Вефиля в Сихем, и на юг от Левоны.
Jdg 21:20  И приказали сынам Вениамина и сказали: подите и засядьте в виноградниках,
Jdg 21:21  и смотрите, когда выйдут девицы Силомские плясать в хороводах, тогда выйдите из виноградников и схватите себе каждый жену из девиц Силомских и идите в землю Вениаминову;
Jdg 21:22  и когда придут отцы их, или братья их с жалобою к нам, мы скажем им: простите нас за них, ибо мы не взяли для каждого из них жены на войне, и вы не дали им; теперь вы виновны.
Jdg 21:23  Сыны Вениамина так и сделали, и взяли жен по числу своему из бывших в хороводе, которых они похитили, и пошли и возвратились в удел свой, и построили города и стали жить в них.
Jdg 21:24  В то же время Израильтяне разошлись оттуда каждый в колено свое и в племя свое, и пошли оттуда каждый в удел свой.
Jdg 21:25  В те дни не было царя у Израиля; каждый делал то, что ему казалось справедливым.


\end{document}