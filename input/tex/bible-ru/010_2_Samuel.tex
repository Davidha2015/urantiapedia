\begin{document}

\title{2-я Царств}


\chapter{1}

\par 1 По смерти Саула, когда Давид возвратился от поражения Амаликитян и пробыл в Секелаге два дня,
\par 2 вот, на третий день приходит человек из стана Саулова; одежда на нем разодрана и прах на голове его. Придя к Давиду, он пал на землю и поклонился [ему].
\par 3 И сказал ему Давид: откуда ты пришел? И сказал тот: я убежал из стана Израильского.
\par 4 И сказал ему Давид: что произошло? расскажи мне. И тот сказал: народ побежал со сражения, и множество из народа пало и умерло, и умерли и Саул и сын его Ионафан.
\par 5 И сказал Давид отроку, рассказывавшему ему: как ты знаешь, что Саул и сын его Ионафан умерли?
\par 6 И сказал отрок, рассказывавший ему: я случайно пришел на гору Гелвуйскую, и вот, Саул пал на свое копье, колесницы же и всадники настигали его.
\par 7 Тогда он оглянулся назад и, увидев меня, позвал меня.
\par 8 И я сказал: вот я. Он сказал мне: кто ты? И я сказал ему: я--Амаликитянин.
\par 9 Тогда он сказал мне: подойди ко мне и убей меня, ибо тоска смертная объяла меня, душа моя все еще во мне.
\par 10 И я подошел к нему и убил его, ибо знал, что он не будет жив после своего падения; и взял я венец, бывший на голове его, и запястье, бывшее на руке его, и принес их к господину моему сюда.
\par 11 Тогда схватил Давид одежды свои и разодрал их, также и все люди, бывшие с ним.
\par 12 и рыдали и плакали, и постились до вечера о Сауле и о сыне его Ионафане, и о народе Господнем и о доме Израилевом, что пали они от меча.
\par 13 И сказал Давид отроку, рассказывавшему ему: откуда ты? И сказал он: я--сын пришельца Амаликитянина.
\par 14 Тогда Давид сказал ему: как не побоялся ты поднять руку, чтобы убить помазанника Господня?
\par 15 И призвал Давид одного из отроков и сказал ему: подойди, убей его.
\par 16 И [тот] убил его, и он умер. И сказал к нему Давид: кровь твоя на голове твоей, ибо твои уста свидетельствовали на тебя, когда ты говорил: я убил помазанника Господня.
\par 17 И оплакал Давид Саула и сына его Ионафана сею плачевною песнью,
\par 18 и повелел научить сынов Иудиных луку, как написано в книге Праведного, и сказал:
\par 19 краса твоя, о Израиль, поражена на высотах твоих! как пали сильные!
\par 20 Не рассказывайте в Гефе, не возвещайте на улицах Аскалона, чтобы не радовались дочери Филистимлян, чтобы не торжествовали дочери необрезанных.
\par 21 Горы Гелвуйские! да [не сойдет] ни роса, ни дождь на вас, и да не будет [на вас] полей с плодами, ибо там повержен щит сильных, щит Саула, как бы не был он помазан елеем.
\par 22 Без крови раненых, без тука сильных лук Ионафана не возвращался назад, и меч Саула не возвращался даром.
\par 23 Саул и Ионафан, любезные и согласные в жизни своей, не разлучились и в смерти своей; быстрее орлов, сильнее львов [они были].
\par 24 Дочери Израильские! плачьте о Сауле, который одевал вас в багряницу с украшениями и доставлял на одежды ваши золотые уборы.
\par 25 Как пали сильные на брани! Сражен Ионафан на высотах твоих.
\par 26 Скорблю о тебе, брат мой Ионафан; ты был очень дорог для меня; любовь твоя была для меня превыше любви женской.
\par 27 Как пали сильные, погибло оружие бранное!

\chapter{2}

\par 1 После сего Давид вопросил Господа, говоря: идти ли мне в какой-либо из городов Иудиных? И сказал ему Господь: иди. И сказал Давид: куда идти? И сказал Он: в Хеврон.
\par 2 И пошел туда Давид и обе жены его, Ахиноама Изреелитянка и Авигея, [бывшая] жена Навала, Кармилитянка.
\par 3 И людей, бывших с ним, привел Давид, каждого с семейством его, и поселились в городе Хевроне.
\par 4 И пришли мужи Иудины и помазали там Давида на царство над домом Иудиным. И донесли Давиду, что жители Иависа Галаадского погребли Саула.
\par 5 И отправил Давид послов к жителям Иависа Галаадского, сказать им: благословенны вы у Господа за то, что оказали эту милость господину своему Саулу, и погребли его.
\par 6 и ныне да воздаст вам Господь милостью и истиною; и я сделаю вам благодеяние за то, что вы это сделали;
\par 7 ныне да укрепятся руки ваши, и будьте мужественны; ибо господин ваш Саул умер, а меня помазал дом Иудин царем над собою.
\par 8 Но Авенир, сын Ниров, начальник войска Саулова, взял Иевосфея, сына Саулова, и привел его в Маханаим,
\par 9 и воцарил его над Галаадом, и Ашуром, и Изреелем, и Ефремом, и Вениамином, и над всем Израилем.
\par 10 Сорок лет было Иевосфею, сыну Саулову, когда он воцарился над Израилем, и царствовал два года. Только дом Иудин остался с Давидом.
\par 11 Всего времени, в которое Давид царствовал в Хевроне над домом Иудиным, было семь лет и шесть месяцев.
\par 12 И вышел Авенир, сын Ниров, и слуги Иевосфея, сына Саулова, из Маханаима в Гаваон.
\par 13 Вышел и Иоав, сын Саруи, со слугами Давида, и встретились у Гаваонского пруда, и засели те на одной стороне пруда, а эти на другой стороне пруда.
\par 14 И сказал Авенир Иоаву: пусть встанут юноши и поиграют пред нами. И сказал Иоав: пусть встанут.
\par 15 И встали и пошли числом двенадцать Вениамитян со стороны Иевосфея, сына Саулова, и двенадцать из слуг Давидовых.
\par 16 Они схватили друг друга за голову, [вонзили] меч один другому в бок и пали вместе. И было названо это место Хелкаф-Хаццурим, что в Гаваоне.
\par 17 И произошло в тот день жесточайшее сражение, и Авенир с людьми Израильскими был поражен слугами Давида.
\par 18 И были там три сына Саруи: Иоав, и Авесса, и Асаил. Асаил же был легок на ноги, как серна в поле.
\par 19 И погнался Асаил за Авениром и преследовал его, не уклоняясь ни направо, ни налево от следов Авенира.
\par 20 И оглянулся Авенир назад и сказал: ты ли это, Асаил? Тот сказал: я.
\par 21 И сказал ему Авенир: уклонись направо или налево, и выбери себе одного из отроков и возьми себе его вооружение. Но Асаил не захотел отстать от него.
\par 22 И повторил Авенир еще, говоря Асаилу: отстань от меня, чтоб я не поверг тебя на землю; тогда с каким лицем явлюсь я к Иоаву, брату твоему?
\par 23 Но тот не захотел отстать. Тогда Авенир, поворотив копье, поразил его в живот; копье прошло насквозь его, и он упал там же и умер на месте. Все проходившие чрез то место, где пал и умер Асаил, останавливались.
\par 24 И преследовали Иоав и Авесса Авенира. Солнце уже зашло, когда они пришли к холму Амма, что против Гиаха, на дороге к пустыне Гаваонской.
\par 25 И собрались Вениамитяне вокруг Авенира и составили одно ополчение, и стали на вершине одного холма.
\par 26 И воззвал Авенир к Иоаву, и сказал: вечно ли будет пожирать меч? Или ты не знаешь, что последствия будут горестные? И доколе ты не скажешь людям, чтобы они перестали преследовать братьев своих?
\par 27 И сказал Иоав: жив Бог! если бы ты не говорил иначе, то еще утром перестали бы люди преследовать братьев своих.
\par 28 И затрубил Иоав трубою, и остановился весь народ, и не преследовали более Израильтян; сражение прекратилось.
\par 29 Авенир же и люди его шли равниною всю ту ночь и перешли Иордан, и прошли весь Битрон, и пришли в Маханаим.
\par 30 И возвратился Иоав от преследования Авенира и собрал весь народ, и недоставало из слуг Давидовых девятнадцати человек кроме Асаила.
\par 31 Слуги же Давидовы поразили Вениамитян и людей Авенировых; пало их триста шестьдесят человек.
\par 32 И взяли Асаила и похоронили его во гробе отца его, что в Вифлееме. Иоав же с людьми своими шел всю ночь и на рассвете прибыл в Хеврон.

\chapter{3}

\par 1 И была продолжительная распря между домом Сауловым и домом Давидовым. Давид все более и более усиливался, а дом Саулов более и более ослабевал.
\par 2 И родились у Давида сыновей в Хевроне. Первенец его был Амнон от Ахиноамы Изреелитянки,
\par 3 а второй [сын] его--Далуиа от Авигеи, [бывшей] жены Навала, Кармилитянки; третий--Авессалом, сын Маахи, дочери Фалмая, царя Гессурского;
\par 4 четвертый--Адония, сын Аггифы; пятый--Сафатия, сын Авиталы;
\par 5 шестой--Иефераам от Эглы, жены Давидовой. Они родились у Давида в Хевроне.
\par 6 Когда была распря между домом Саула и домом Давида, то Авенир поддерживал дом Саула.
\par 7 У Саула была наложница, по имени Рицпа, дочь Айя. И сказал [Иевосфей] Авениру: зачем ты вошел к наложнице отца моего?
\par 8 Авенир же сильно разгневался на слова Иевосфея и сказал: разве я--собачья голова? Я против Иуды оказал ныне милость дому Саула, отца твоего, братьям его и друзьям его, и не предал тебя в руки Давида, а ты взыскиваешь ныне на мне грех из-за женщины.
\par 9 То и то пусть сделает Бог Авениру и еще больше сделает ему! Как клялся Господь Давиду, так и сделаю ему:
\par 10 отниму царство от дома Саулова и поставлю престол Давида над Израилем и над Иудою, от Дана до Вирсавии.
\par 11 И не мог Иевосфей возразить Авениру, ибо боялся его.
\par 12 И послал Авенир от себя послов к Давиду, сказать: чья эта земля? И еще сказать: заключи союз со мною, и рука моя будет с тобою, чтобы обратить к тебе весь народ Израильский.
\par 13 И сказал [Давид]: хорошо, я заключу союз с тобою, только прошу тебя об одном, именно--ты не увидишь лица моего, если не приведешь с собою Мелхолы, дочери Саула, когда придешь увидеться со мною.
\par 14 И отправил Давид послов к Иевосфею, сыну Саулову, сказать: отдай жену мою Мелхолу, которую я получил за сто краеобрезаний Филистимских.
\par 15 И послал Иевосфей и взял ее от мужа, от Фалтия, сына Лаишева.
\par 16 Пошел с нею и муж ее и с плачем провожал ее до Бахурима; но Авенир сказал ему: ступай назад. И он возвратился.
\par 17 И обратился Авенир к старейшинам Израильским, говоря: и вчера и третьего дня вы желали, чтобы Давид был царем над вами,
\par 18 теперь сделайте [это], ибо Господь сказал Давиду: `рукою раба Моего Давида Я спасу народ Мой Израиля от руки Филистимлян и от руки всех врагов его'.
\par 19 То же говорил Авенир и Вениамитянам. И пошел Авенир в Хеврон, чтобы пересказать Давиду все, чего желали Израиль и весь дом Вениаминов.
\par 20 И пришел Авенир к Давиду в Хеврон и с ним двадцать человек, и сделал Давид пир для Авенира и людей, бывших с ним.
\par 21 И сказал Авенир Давиду: я встану и пойду и соберу к господину моему царю весь народ Израильский, и они вступят в завет с тобою, и будешь царствовать над всеми, как желает душа твоя. И отпустил Давид Авенира, и он ушел с миром.
\par 22 И вот, слуги Давидовы с Иоавом пришли из похода и принесли с собою много добычи; но Авенира уже не было с Давидом в Хевроне, ибо [Давид] отпустил его, и он ушел с миром.
\par 23 Когда Иоав и все войско, ходившее с ним, пришли, то Иоаву рассказали: приходил Авенир, сын Ниров, к царю, и тот отпустил его, и он ушел с миром.
\par 24 И пришел Иоав к царю и сказал: что ты сделал? Вот, приходил к тебе Авенир; зачем ты отпустил его, и он ушел?
\par 25 Ты знаешь Авенира, сына Нирова: он приходил обмануть тебя, узнать выход твой и вход твой и разведать все, что ты делаешь.
\par 26 И вышел Иоав от Давида и послал гонцов вслед за Авениром; и возвратили они его от колодезя Сира, без ведома Давида.
\par 27 Когда Авенир возвратился в Хеврон, то Иоав отвел его внутрь ворот, как будто для того, чтобы поговорить с ним тайно, и там поразил его в живот. И умер [Авенир] за кровь Асаила, брата Иоавова.
\par 28 И услышал после Давид [об этом] и сказал: невинен я и царство мое вовек пред Господом в крови Авенира, сына Нирова;
\par 29 пусть падет она на голову Иоава и на весь дом отца его; пусть никогда не остается дом Иоава без семеноточивого, или прокаженного, или опирающегося на посох, или падающего от меча, или нуждающегося в хлебе.
\par 30 Иоав же и брат его Авесса убили Авенира за то, что он умертвил брата их Асаила в сражении у Гаваона.
\par 31 И сказал Давид Иоаву и всем людям, бывшим с ним: раздерите одежды ваши и оденьтесь во вретища и плачьте над Авениром. И царь Давид шел за гробом [его].
\par 32 Когда погребали Авенира в Хевроне, то царь громко плакал над гробом Авенира; плакал и весь народ.
\par 33 И оплакал царь Авенира, говоря: смертью ли подлого умирать Авениру?
\par 34 Руки твои не были связаны, и ноги твои не в оковах, и ты пал, как падают от разбойников. И весь народ стал еще более плакать над ним.
\par 35 И пришел весь народ предложить Давиду хлеба, когда еще продолжался день; но Давид поклялся, говоря: то и то пусть сделает со мною Бог и еще больше сделает, если я до захождения солнца вкушу хлеба или чего-нибудь.
\par 36 И весь народ узнал это, и понравилось ему это, как и все, что делал царь, нравилось всему народу.
\par 37 И узнал весь народ и весь Израиль в тот день, что не от царя произошло умерщвление Авенира, сына Нирова.
\par 38 И сказал царь слугам своим: знаете ли, что вождь и великий муж пал в этот день в Израиле?
\par 39 Я теперь еще слаб, хотя и помазан на царство, а эти люди, сыновья Саруи, сильнее меня; пусть же воздаст Господь делающему злое по злобе его!

\chapter{4}

\par 1 И услышал [Иевосфей], сын Саулов, что умер Авенир в Хевроне, и опустились руки его, и весь Израиль смутился.
\par 2 У [Иевосфея], сына Саулова, два было предводителя войска; имя одного--Баана и имя другого--Рихав, сыновья Реммона Беерофянина, из потомков Вениаминовых, ибо и Беероф причислялся к Вениамину.
\par 3 И убежали Беерофяне в Гиффаим и остались там пришельцами до сего дня.
\par 4 У Ионафана, сына Саулова, был сын хромой. Пять лет было ему, когда пришло известие о Сауле и Ионафане из Изрееля, и нянька, взяв его, побежала. И когда она бежала поспешно, то он упал, и сделался хромым. Имя его Мемфивосфей.
\par 5 И пошли сыны Реммона Беерофянина, Рихав и Баана, и пришли в самый жар дня к дому Иевосфея; а он спал на постели в полдень.
\par 6 Рихав и Баана, брат его, вошли внутрь дома, [как бы] для того, чтобы взять пшеницы; и поразили его в живот и убежали.
\par 7 Когда они вошли в дом, [Иевосфей] лежал на постели своей, в спальной комнате своей; и они поразили его, и умертвили его, и отрубили голову его, и взяли голову его с собою, и шли пустынною дорогою всю ночь;
\par 8 и принесли голову Иевосфея к Давиду в Хеврон и сказали царю: вот голова Иевосфея, сына Саула, врага твоего, который искал души твоей; ныне Господь отмстил за господина моего царя Саулу и потомству его.
\par 9 И отвечал Давид Рихаву и Баане, брату его, сыновьям Реммона Беерофянина, и сказал им: жив Господь, избавивший душу мою от всякой скорби!
\par 10 если того, кто принес мне известие, сказав: `вот, умер Саул', и кто считал себя радостным вестником, я схватил и убил его в Секелаге, вместо того, чтобы дать ему награду,
\par 11 то теперь, когда негодные люди убили человека невинного в его доме на постели его, неужели я не взыщу крови его от руки вашей и не истреблю вас с земли?
\par 12 И приказал Давид слугам, и убили их, и отрубили им руки и ноги, и повесили их над прудом в Хевроне. А голову Иевосфея взяли и погребли во гробе Авенира, в Хевроне.

\chapter{5}

\par 1 И пришли все колена Израилевы к Давиду в Хеврон и сказали: вот, мы--кости твои и плоть твоя;
\par 2 еще вчера и третьего дня, когда Саул царствовал над нами, ты выводил и вводил Израиля; и сказал Господь тебе: `ты будешь пасти народ Мой Израиля и ты будешь вождем Израиля'.
\par 3 И пришли все старейшины Израиля к царю в Хеврон, и заключил с ними царь Давид завет в Хевроне пред Господом; и помазали Давида в царя над Израилем.
\par 4 Тридцать лет было Давиду, когда он воцарился; царствовал сорок лет.
\par 5 В Хевроне царствовал над Иудою семь лет и шесть месяцев, и в Иерусалиме царствовал тридцать три года над всем Израилем и Иудою.
\par 6 И пошел царь и люди его на Иерусалим против Иевусеев, жителей той страны; но они говорили Давиду: `ты не войдешь сюда; тебя отгонят слепые и хромые', --это значило: `не войдет сюда Давид'.
\par 7 Но Давид взял крепость Сион: это--город Давидов.
\par 8 И сказал Давид в тот день: всякий, убивая Иевусеев, пусть поражает копьем и хромых и слепых, ненавидящих душу Давида. Посему и говорится: слепой и хромой не войдет в дом [Господень].
\par 9 И поселился Давид в крепости, и назвал ее городом Давидовым, и обстроил кругом от Милло и внутри.
\par 10 И преуспевал Давид и возвышался, и Господь Бог Саваоф [был] с ним.
\par 11 И прислал Хирам, царь Тирский, послов к Давиду и кедровые деревья и плотников и каменщиков, и они построили дом Давиду.
\par 12 И уразумел Давид, что Господь утвердил его царем над Израилем и что возвысил царство его ради народа Своего Израиля.
\par 13 И взял Давид еще наложниц и жен из Иерусалима, после того, как пришел из Хеврона.
\par 14 И родились еще у Давида сыновья и дочери. И вот имена родившихся у него в Иерусалиме: Самус, и Совав, и Нафан, и Соломон,
\par 15 и Евеар, и Елисуа, и Нафек, и Иафиа,
\par 16 и Елисама, и Елидае, и Елифалеф.
\par 17 Когда Филистимляне услышали, что Давида помазали на царство над Израилем, то поднялись все Филистимляне искать Давида. И услышал Давид и пошел в крепость.
\par 18 А Филистимляне пришли и расположились в долине Рефаим.
\par 19 И вопросил Давид Господа, говоря: идти ли мне против Филистимлян? предашь ли их в руки мои? И сказал Господь Давиду: иди, ибо Я предам Филистимлян в руки твои.
\par 20 И пошел Давид в Ваал-Перацим и поразил их там, и сказал Давид: Господь разнес врагов моих предо мною, как разносит вода. Посему и месту тому дано имя Ваал-Перацим.
\par 21 И оставили там [Филистимляне] истуканов своих, а Давид с людьми своими взял их.
\par 22 И пришли опять Филистимляне и расположились в долине Рефаим.
\par 23 И вопросил Давид Господа, И Он отвечал ему: не выходи навстречу им, а зайди им с тылу и иди к ним со стороны тутовой рощи;
\par 24 и когда услышишь шум как бы идущего по вершинам тутовых дерев, то двинься, ибо тогда пошел Господь пред тобою, чтобы поразить войско Филистимское.
\par 25 И сделал Давид, как повелел ему Господь, и поразил Филистимлян от Гаваи до Газера.

\chapter{6}

\par 1 И собрал снова Давид всех отборных [людей] из Израиля, тридцать тысяч.
\par 2 И встал и пошел Давид и весь народ, бывший с ним из Ваала Иудина, чтобы перенести оттуда ковчег Божий, на котором нарицается имя Господа Саваофа, сидящего на херувимах.
\par 3 И поставили ковчег Божий на новую колесницу и вывезли его из дома Аминадава, что на холме. Сыновья же Аминадава, Оза и Ахио, вели новую колесницу.
\par 4 И повезли ее с ковчегом Божиим из дома Аминадава, что на холме; и Ахио шел пред ковчегом.
\par 5 А Давид и все сыны Израилевы играли пред Господом на всяких музыкальных орудиях из кипарисового дерева, и на цитрах, и на псалтирях, и на тимпанах, и на систрах, и на кимвалах.
\par 6 И когда дошли до гумна Нахонова, Оза простер руку свою к ковчегу Божию и взялся за него, ибо волы наклонили его.
\par 7 Но Господь прогневался на Озу, и поразил его Бог там же за дерзновение, и умер он там у ковчега Божия.
\par 8 И опечалился Давид, что Господь поразил Озу. Место сие и доныне называется: `поражение Озы'.
\par 9 И устрашился Давид в тот день Господа и сказал: как войти ко мне ковчегу Господню?
\par 10 И не захотел Давид везти ковчег Господень к себе, в город Давидов, а обратил его в дом Аведдара Гефянина.
\par 11 И оставался ковчег Господень в доме Аведдара Гефянина три месяца, и благословил Господь Аведдара и весь дом его.
\par 12 Когда донесли царю Давиду, говоря: `Господь благословил дом Аведдара и все, что было у него, ради ковчега Божия', то пошел Давид и с торжеством перенес ковчег Божий из дома Аведдара в город Давидов.
\par 13 И когда несшие ковчег Господень проходили по шести шагов, он приносил в жертву тельца и овна.
\par 14 Давид скакал из всей силы пред Господом; одет же был Давид в льняной ефод.
\par 15 Так Давид и весь дом Израилев несли ковчег Господень с восклицаниями и трубными звуками.
\par 16 Когда входил ковчег Господень в город Давидов, Мелхола, дочь Саула, смотрела в окно и, увидев царя Давида, скачущего и пляшущего пред Господом, уничижила его в сердце своем.
\par 17 И принесли ковчег Господень и поставили его на своем месте посреди скинии, которую устроил для него Давид; и принес Давид всесожжения пред Господом и жертвы мирные.
\par 18 Когда Давид окончил приношение всесожжений и жертв мирных, то благословил он народ именем Господа Саваофа;
\par 19 и роздал всему народу, всему множеству Израильтян, как мужчинам, так и женщинам, по одному хлебу и по куску жареного мяса и по одной лепешке каждому. И пошел весь народ, каждый в дом свой.
\par 20 Когда Давид возвратился, чтобы благословить дом свой, то Мелхола, дочь Саула, вышла к нему на встречу и сказала: как отличился сегодня царь Израилев, обнажившись сегодня пред глазами рабынь рабов своих, как обнажается какой--нибудь пустой человек!
\par 21 И сказал Давид Мелхоле: пред Господом, Который предпочел меня отцу твоему и всему дому его, утвердив меня вождем народа Господня, Израиля; пред Господом играть и плясать буду;
\par 22 и я еще больше уничижусь, и сделаюсь еще ничтожнее в глазах моих, и пред служанками, о которых ты говоришь, я буду славен.
\par 23 И у Мелхолы, дочери Сауловой, не было детей до дня смерти ее.

\chapter{7}

\par 1 Когда царь жил в доме своем, и Господь успокоил его от всех окрестных врагов его,
\par 2 тогда сказал царь пророку Нафану: вот, я живу в доме кедровом, а ковчег Божий находится под шатром.
\par 3 И сказал Нафан царю: все, что у тебя на сердце, иди, делай; ибо Господь с тобою.
\par 4 Но в ту же ночь было слово Господа к Нафану:
\par 5 пойди, скажи рабу Моему Давиду: так говорит Господь: ты ли построишь Мне дом для Моего обитания,
\par 6 когда Я не жил в доме с того времени, как вывел сынов Израилевых из Египта, и до сего дня, но переходил в шатре и в скинии?
\par 7 Где Я ни ходил со всеми сынами Израиля, говорил ли Я хотя слово какому-либо из колен, которому Я назначил пасти народ Мой Израиля: `почему не построите Мне кедрового дома'?
\par 8 И теперь так скажи рабу Моему Давиду: так говорит Господь Саваоф: Я взял тебя от стада овец, чтобы ты был вождем народа Моего, Израиля;
\par 9 и был с тобою везде, куда ни ходил ты, и истребил всех врагов твоих пред лицем твоим, и сделал имя твое великим, как имя великих на земле.
\par 10 И Я устрою место для народа Моего, для Израиля, и укореню его, и будет он спокойно жить на месте своем, и не будет тревожиться больше, и люди нечестивые не станут более теснить его, как прежде,
\par 11 с того времени, как Я поставил судей над народом Моим, Израилем; и Я успокою тебя от всех врагов твоих. И Господь возвещает тебе, что Он устроит тебе дом.
\par 12 Когда же исполнятся дни твои, и ты почиешь с отцами твоими, то Я восставлю после тебя семя твое, которое произойдет из чресл твоих, и упрочу царство его.
\par 13 Он построит дом имени Моему, и Я утвержу престол царства его на веки.
\par 14 Я буду ему отцом, и он будет Мне сыном; и если он согрешит, Я накажу его жезлом мужей и ударами сынов человеческих;
\par 15 но милости Моей не отниму от него, как Я отнял от Саула, которого Я отверг пред лицем твоим.
\par 16 И будет непоколебим дом твой и царство твое на веки пред лицем Моим, и престол твой устоит во веки.
\par 17 Все эти слова и все это видение Нафан пересказал Давиду.
\par 18 И пошел царь Давид, и предстал пред лицем Господа, и сказал: кто я, Господи, Господи, и что такое дом мой, что Ты меня так возвеличил!
\par 19 И этого еще мало показалось в очах Твоих, Господи мой, Господи; но Ты возвестил еще о доме раба Твоего вдаль. Это уже по-человечески. Господи мой, Господи!
\par 20 Что еще может сказать Тебе Давид? Ты знаешь раба Твоего, Господи мой, Господи!
\par 21 Ради слова Твоего и по сердцу Твоему Ты делаешь это, открывая все это великое рабу Твоему.
\par 22 По всему велик Ты, Господи мой, Господи! ибо нет подобного Тебе и нет Бога, кроме Тебя, по всему, что слышали мы своими ушами.
\par 23 И кто подобен народу Твоему, Израилю, единственному народу на земле, для которого приходил Бог, чтобы приобрести [его] Себе в народ и прославить Свое имя [и] совершить великое и страшное пред народом Твоим, который Ты приобрел Себе от Египтян, изгнав народы и богов их?
\par 24 И Ты укрепил за Собою народ Твой, Израиля, как собственный народ, на веки, и Ты, Господи, сделался его Богом.
\par 25 И ныне, Господи Боже, утверди на веки слово, которое изрек Ты о рабе Твоем и о доме его, и исполни то, что Ты изрек.
\par 26 И да возвеличится имя Твое во веки, чтобы говорили: `Господь Саваоф--Бог над Израилем'. И дом раба Твоего Давида да будет тверд пред лицем Твоим.
\par 27 Так как ты, Господи Саваоф, Боже Израилев, открыл рабу Твоему, говоря: `устрою тебе дом', то раб Твой уготовал сердце свое, чтобы молиться Тебе такою молитвою.
\par 28 Итак, Господи мой, Господи! Ты Бог, и слова Твои непреложны, и Ты возвестил рабу Твоему такое благо!
\par 29 И ныне начни и благослови дом раба Твоего, чтоб он был вечно пред лицем Твоим, ибо Ты, Господи мой, Господи, возвестил это, и благословением Твоим соделается дом раба Твоего благословенным во веки.

\chapter{8}

\par 1 После сего Давид поразил Филистимлян и смирил их, и взял Давид Мефег-Гаамма из рук Филистимлян.
\par 2 И поразил Моавитян и смерил их веревкою, положив их на землю; и отмерил две веревки на умерщвление, а одну веревку на оставление в живых. И сделались Моавитяне у Давида рабами, платящими дань.
\par 3 И поразил Давид Адраазара, сына Реховова, царя Сувского, когда тот шел, чтоб восстановить свое владычество при реке [Евфрате];
\par 4 и взял Давид у него тысячу семьсот всадников и двадцать тысяч человек пеших, и подрезал Давид жилы у всех коней колесничных, оставив [себе] из них для ста колесниц.
\par 5 И пришли Сирийцы Дамасские на помощь к Адраазару, царю Сувскому; но Давид поразил двадцать две тысячи человек Сирийцев.
\par 6 И поставил Давид охранные войска в Сирии Дамасской, и стали Сирийцы у Давида рабами, платящими дань. И хранил Господь Давида везде, куда он ни ходил.
\par 7 И взял Давид золотые щиты, которые были у рабов Адраазара, и принес их в Иерусалим.
\par 8 А в Бефе и Берофе, городах Адраазаровых, взял царь Давид весьма много меди.
\par 9 И услышал Фой, царь Имафа, что Давид поразил все войско Адраазарово,
\par 10 и послал Фой Иорама, сына своего, к царю Давиду, приветствовать его и благодарить его за то, что он воевал с Адраазаром и поразил его; ибо Адраазар вел войны с Фоем. В руках же [Иорама] были сосуды серебряные, золотые и медные.
\par 11 Их также посвятил царь Давид Господу, вместе с серебром и золотом, которое посвятил из [отнятого] у всех покоренных им народов:
\par 12 у Сирийцев, и Моавитян, и Аммонитян, и Филистимлян, и Амаликитян, и из отнятого у Адраазара, сына Реховова, царя Сувского.
\par 13 И сделал Давид себе имя, возвращаясь с поражения восемнадцати тысяч Сирийцев в долине Соленой.
\par 14 И поставил он охранные войска в Идумее; во всей Идумее поставил охранные войска, и все Идумеяне были рабами Давиду. И хранил Господь Давида везде, куда он ни ходил.
\par 15 И царствовал Давид над всем Израилем, и творил Давид суд и правду над всем народом своим.
\par 16 Иоав же, сын Саруи, [был начальником] войска; и Иосафат, сын Ахилуда, --дееписателем;
\par 17 Садок, сын Ахитува, и Ахимелех, сын Авиафара, --священниками, Сераия--писцом;
\par 18 и Ванея, сын Иодая--[начальником] над Хелефеями и Фелефеями, и сыновья Давида--первыми при дворе.

\chapter{9}

\par 1 И сказал Давид: не остался ли еще кто-нибудь из дома Саулова? я оказал бы ему милость ради Ионафана.
\par 2 В доме Саула был раб, по имени Сива; и позвали его к Давиду, и сказал ему царь: ты ли Сива? И тот сказал: я, раб твой.
\par 3 И сказал царь: нет ли еще кого-нибудь из дома Саулова? я оказал бы ему милость Божию. И сказал Сива царю: есть сын Ионафана, хромой ногами.
\par 4 И сказал ему царь: где он? И сказал Сива царю: вот, он в доме Махира, сына Аммиэлова, в Лодеваре.
\par 5 И послал царь Давид, и взяли его из дома Махира, сына Аммиэлова, из Лодевара.
\par 6 И пришел Мемфивосфей, сын Ионафана, сына Саулова, к Давиду, и пал на лице свое, и поклонился. И сказал Давид: Мемфивосфей! И сказал тот: вот раб твой.
\par 7 И сказал ему Давид: не бойся; я окажу тебе милость ради отца твоего Ионафана и возвращу тебе все поля Саула, отца твоего, и ты всегда будешь есть хлеб за моим столом.
\par 8 И поклонился [Мемфивосфей] и сказал: что такое раб твой, что ты призрел на такого мертвого пса, как я?
\par 9 И призвал царь Сиву, слугу Саула, и сказал ему: все, что принадлежало Саулу и всему дому его, я отдаю сыну господина твоего;
\par 10 итак обрабатывай для него землю ты и сыновья твои и рабы твои, и доставляй [плоды ее], чтобы у сына господина твоего был хлеб для пропитания; Мемфивосфей же, сын господина твоего, всегда будет есть за моим столом. У Сивы было пятнадцать сыновей и двадцать рабов.
\par 11 И сказал Сива царю: все, что приказывает господин мой царь рабу своему, исполнит раб твой. Мемфивосфей ел за столом [Давида], как один из сыновей царя.
\par 12 У Мемфивосфея был малолетний сын, по имени Миха. Все живущие в доме Сивы были рабами Мемфивосфея.
\par 13 И жил Мемфивосфей в Иерусалиме, ибо он ел всегда за царским столом. Он был хром на обе ноги.

\chapter{10}

\par 1 Спустя несколько времени умер царь Аммонитский, и воцарился вместо него сын его Аннон.
\par 2 И сказал Давид: окажу я милость Аннону, сыну Наасову, за благодеяние, которое оказал мне отец его. И послал Давид слуг своих утешить Аннона об отце его. И пришли слуги Давидовы в землю Аммонитскую.
\par 3 Но князья Аммонитские сказали Аннону, господину своему: неужели ты думаешь, что Давид из уважения к отцу твоему прислал к тебе утешителей? не для того ли, чтобы осмотреть город и высмотреть в нем и [после] разрушить его, прислал Давид слуг своих к тебе?
\par 4 И взял Аннон слуг Давидовых, и обрил каждому из них половину бороды, и обрезал одежды их наполовину, до чресл, и отпустил их.
\par 5 Когда донесли об этом Давиду, то он послал к ним навстречу, так как они были очень обесчещены. И велел царь сказать им: оставайтесь в Иерихоне, пока отрастут бороды ваши, и [тогда] возвратитесь.
\par 6 И увидели Аммонитяне, что они сделались ненавистными для Давида; и послали Аммонитяне нанять Сирийцев из Беф-Рехова и Сирийцев Сувы двадцать тысяч пеших, у царя Маахи тысячу человек и из Истова двенадцать тысяч человек.
\par 7 Когда услышал об этом Давид, то послал Иоава со всем войском храбрых.
\par 8 И вышли Аммонитяне и расположились к сражению у ворот, а Сирийцы Сувы и Рехова, и Истова, и Маахи, [стали] отдельно в поле.
\par 9 И увидел Иоав, что неприятельское войско было поставлено против него и спереди и сзади, и избрал [воинов] из всех отборных в Израиле, и выстроил их против Сирийцев;
\par 10 остальную же часть людей поручил Авессе, брату своему, чтоб он выстроил их против Аммонитян.
\par 11 И сказал [Иоав]: если Сирийцы будут одолевать меня, ты поможешь мне; а если Аммонитяне тебя будут одолевать, я приду к тебе на помощь;
\par 12 будь мужествен, и будем стоять твердо за народ наш и за города Бога нашего, а Господь сделает, что Ему угодно.
\par 13 И вступил Иоав в народ, который [был] у него, в сражение с Сирийцами, и они побежали от него.
\par 14 Аммонитяне же, увидев, что Сирийцы бегут, побежали от Авессы и ушли в город. И возвратился Иоав от Аммонитян и пришел в Иерусалим.
\par 15 Сирийцы, видя, что они поражены Израильтянами, собрались вместе.
\par 16 И послал Адраазар и призвал Сирийцев, которые за рекою, и пришли они к Еламу; а Совак, военачальник Адраазаров, предводительствовал ими.
\par 17 Когда донесли [об этом] Давиду, то он собрал всех Израильтян, и перешел Иордан и пришел к Еламу. Сирийцы выстроились против Давида и сразились с ним.
\par 18 И побежали Сирийцы от Израильтян. Давид истребил у Сирийцев семьсот колесниц и сорок тысяч всадников; поразил и военачальника Совака, который там и умер.
\par 19 Когда все цари покорные Адраазару увидели, что они поражены Израильтянами, то заключили мир с Израильтянами и покорились им. А Сирийцы боялись более помогать Аммонитянам.

\chapter{11}

\par 1 Через год, в то время, когда выходят цари [в походы], Давид послал Иоава и слуг своих с ним и всех Израильтян; и они поразили Аммонитян и осадили Равву; Давид же оставался в Иерусалиме.
\par 2 Однажды под вечер Давид, встав с постели, прогуливался на кровле царского дома и увидел с кровли купающуюся женщину; а та женщина была очень красива.
\par 3 И послал Давид разведать, кто эта женщина? И сказали ему: это Вирсавия, дочь Елиама, жена Урии Хеттеянина.
\par 4 Давид послал слуг взять ее; и она пришла к нему, и он спал с нею. Когда же она очистилась от нечистоты своей, возвратилась в дом свой.
\par 5 Женщина эта сделалась беременною и послала известить Давида, говоря: я беременна.
\par 6 И послал Давид [сказать] Иоаву: пришли ко мне Урию Хеттеянина. И послал Иоав Урию к Давиду.
\par 7 И пришел к нему Урия, и расспросил [его] Давид о положении Иоава и о положении народа, и о ходе войны.
\par 8 И сказал Давид Урии: иди домой и омой ноги свои. И вышел Урия из дома царского, а вслед за ним понесли и царское кушанье.
\par 9 Но Урия спал у ворот царского дома со всеми слугами своего господина, и не пошел в свой дом.
\par 10 И донесли Давиду, говоря: не пошел Урия в дом свой. И сказал Давид Урии: вот, ты пришел с дороги; отчего же не пошел ты в дом свой?
\par 11 И сказал Урия Давиду: ковчег и Израиль и Иуда находятся в шатрах, и господин мой Иоав и рабы господина моего пребывают в поле, а я вошел бы в дом свой и есть и пить и спать со своею женою! Клянусь твоею жизнью и жизнью души твоей, этого я не сделаю.
\par 12 И сказал Давид Урии: останься здесь и на этот день, а завтра я отпущу тебя. И остался Урия в Иерусалиме на этот день до завтра.
\par 13 И пригласил его Давид, и ел [Урия] пред ним и пил, и напоил его [Давид]. Но вечером [Урия] пошел спать на постель свою с рабами господина своего, а в свой дом не пошел.
\par 14 Поутру Давид написал письмо к Иоаву и послал [его] с Уриею.
\par 15 В письме он написал так: поставьте Урию там, где [будет] самое сильное сражение, и отступите от него, чтоб он был поражен и умер.
\par 16 Посему, когда Иоав осаждал город, то поставил он Урию на таком месте, о котором знал, что там храбрые люди.
\par 17 И вышли люди из города и сразились с Иоавом, и пало несколько из народа, из слуг Давидовых; был убит также и Урия Хеттеянин.
\par 18 И послал Иоав донести Давиду о всем ходе сражения.
\par 19 И приказал посланному, говоря: когда ты расскажешь царю о всем ходе сражения
\par 20 и увидишь, что царь разгневается, и скажет тебе: `зачем вы так близко подходили к городу сражаться? разве вы не знали, что со стены будут бросать на вас?
\par 21 кто убил Авимелеха, сына Иероваалова? не женщина ли бросила на него со стены обломок жернова, и он умер в Тевеце? Зачем же вы близко подходили к стене?' тогда ты скажи: и раб твой Урия Хеттеянин также умер.
\par 22 И пошел [посланный], и пришел, и рассказал Давиду обо всем, для чего послал его Иоав, обо всем ходе сражения.
\par 23 Тогда посланный сказал Давиду: одолевали нас те люди и вышли к нам в поле, и мы преследовали их до входа в ворота;
\par 24 тогда стреляли стрелки со стены на рабов твоих, и умерли [некоторые] из рабов царя; умер также и раб твой Урия Хеттеянин.
\par 25 Тогда сказал Давид посланному: так скажи Иоаву: `пусть не смущает тебя это дело, ибо меч поядает иногда того, иногда сего; усиль войну твою против города и разрушь его'. Так ободри его.
\par 26 И услышала жена Урии, что умер Урия, муж ее, и плакала по муже своем.
\par 27 Когда кончилось время плача, Давид послал, и взял ее в дом свой, и она сделалась его женою и родила ему сына. И было это дело, которое сделал Давид, зло в очах Господа.

\chapter{12}

\par 1 И послал Господь Нафана к Давиду, и тот пришел к нему и сказал ему: в одном городе были два человека, один богатый, а другой бедный;
\par 2 у богатого было очень много мелкого и крупного скота,
\par 3 а у бедного ничего, кроме одной овечки, которую он купил маленькую и выкормил, и она выросла у него вместе с детьми его; от хлеба его она ела, и из его чаши пила, и на груди у него спала, и была для него, как дочь;
\par 4 и пришел к богатому человеку странник, и тот пожалел взять из своих овец или волов, чтобы приготовить [обед] для странника, который пришел к нему, а взял овечку бедняка и приготовил ее для человека, который пришел к нему.
\par 5 Сильно разгневался Давид на этого человека и сказал Нафану: жив Господь! достоин смерти человек, сделавший это;
\par 6 и за овечку он должен заплатить вчетверо, за то, что он сделал это, и за то, что не имел сострадания.
\par 7 И сказал Нафан Давиду: ты--тот человек. Так говорит Господь Бог Израилев: Я помазал тебя в царя над Израилем и Я избавил тебя от руки Саула,
\par 8 и дал тебе дом господина твоего и жен господина твоего на лоно твое, и дал тебе дом Израилев и Иудин, и, если этого [для тебя] мало, прибавил бы тебе еще больше;
\par 9 зачем же ты пренебрег слово Господа, сделав злое пред очами Его? Урию Хеттеянина ты поразил мечом; жену его взял себе в жену, а его ты убил мечом Аммонитян;
\par 10 итак не отступит меч от дома твоего во веки, за то, что ты пренебрег Меня и взял жену Урии Хеттеянина, чтоб она была тебе женою.
\par 11 Так говорит Господь: вот, Я воздвигну на тебя зло из дома твоего, и возьму жен твоих пред глазами твоими, и отдам ближнему твоему, и будет он спать с женами твоими пред этим солнцем;
\par 12 ты сделал тайно, а Я сделаю это пред всем Израилем и пред солнцем.
\par 13 И сказал Давид Нафану: согрешил я пред Господом. И сказал Нафан Давиду: и Господь снял [с тебя] грех твой; ты не умрешь;
\par 14 но как ты этим делом подал повод врагам Господа хулить Его, то умрет родившийся у тебя сын.
\par 15 И пошел Нафан в дом свой. И поразил Господь дитя, которое родила жена Урии Давиду, и оно заболело.
\par 16 И молился Давид Богу о младенце, и постился Давид, и, уединившись провел ночь, лежа на земле.
\par 17 И вошли к нему старейшины дома его, чтобы поднять его с земли; но он не хотел, и не ел с ними хлеба.
\par 18 На седьмой день дитя умерло, и слуги Давидовы боялись донести ему, что умер младенец; ибо, говорили они, когда дитя было еще живо, и мы уговаривали его, и он не слушал голоса нашего, как же мы скажем ему: `умерло дитя'? Он сделает что--нибудь худое.
\par 19 И увидел Давид, что слуги его перешептываются между собою, и понял Давид, что дитя умерло, и спросил Давид слуг своих: умерло дитя? И сказали: умерло.
\par 20 Тогда Давид встал с земли и умылся, и помазался, и переменил одежды свои, и пошел в дом Господень, и молился. Возвратившись домой, потребовал, чтобы подали ему хлеба, и он ел.
\par 21 И сказали ему слуги его: что значит, что ты так поступаешь: когда дитя было еще живо, ты постился и плакал; а когда дитя умерло, ты встал и ел хлеб?
\par 22 И сказал Давид: доколе дитя было живо, я постился и плакал, ибо думал: кто знает, не помилует ли меня Господь, и дитя останется живо?
\par 23 А теперь оно умерло; зачем же мне поститься? Разве я могу возвратить его? Я пойду к нему, а оно не возвратится ко мне.
\par 24 И утешил Давид Вирсавию, жену свою, и вошел к ней и спал с нею; и она родила сына, и нарекла ему имя: Соломон. И Господь возлюбил его
\par 25 и послал пророка Нафана, и он нарек ему имя: Иедидиа по слову Господа.
\par 26 Иоав воевал против Раввы Аммонитской и взял [почти] царственный город.
\par 27 И послал Иоав к Давиду сказать ему: я нападал на Равву и овладел водою города;
\par 28 теперь собери остальной народ и подступи к городу и возьми его; ибо, если я возьму его, то мое имя будет наречено ему.
\par 29 И собрал Давид весь народ и пошел к Равве, и воевал против нее и взял ее.
\par 30 И взял Давид венец царя их с головы его, --а в нем было золота талант и драгоценный камень, --и возложил его Давид на свою голову, и добычи из города вынес очень много.
\par 31 А народ, бывший в нем, он вывел и положил их под пилы, под железные молотилки, под железные топоры, и бросил их в обжигательные печи. Так он поступил со всеми городами Аммонитскими. И возвратился после того Давид и весь народ в Иерусалим.

\chapter{13}

\par 1 И было после того: у Авессалома, сына Давидова, [была] сестра красивая, по имени Фамарь, и полюбил ее Амнон, сын Давида.
\par 2 И скорбел Амнон до того, что заболел из-за Фамари, сестры своей; ибо она была девица, и Амнону казалось трудным что-нибудь сделать с нею.
\par 3 Но у Амнона был друг, по имени Ионадав, сын Самая, брата Давидова; и Ионадав был человек очень хитрый.
\par 4 И он сказал ему: отчего ты так худеешь с каждым днем, сын царев, --не откроешь ли мне? И сказал ему Амнон: Фамарь, сестру Авессалома, брата моего, люблю я.
\par 5 И сказал ему Ионадав: ложись в постель твою, и притворись больным; и когда отец твой придет навестить тебя: скажи ему: пусть придет Фамарь, сестра моя, и подкрепит меня пищею, приготовив кушанье при моих глазах, чтоб я видел, и ел из рук ее.
\par 6 И лег Амнон и притворился больным, и пришел царь навестить его; и сказал Амнон царю: пусть придет Фамарь, сестра моя, и испечет при моих глазах лепешку, или две, и я поем из рук ее.
\par 7 И послал Давид к Фамари в дом сказать: пойди в дом Амнона, брата твоего, и приготовь ему кушанье.
\par 8 И пошла она в дом брата своего Амнона; а он лежит. И взяла она муки и замесила, и изготовила пред глазами его и испекла лепешки,
\par 9 и взяла сковороду и выложила пред ним; но он не хотел есть. И сказал Амнон: пусть все выйдут от меня. И вышли от него все люди,
\par 10 и сказал Амнон Фамари: отнеси кушанье во внутреннюю комнату, и я поем из рук твоих. И взяла Фамарь лепешки, которые приготовила, и отнесла Амнону, брату своему, во внутреннюю комнату.
\par 11 И когда она поставила пред ним, чтоб он ел, то он схватил ее, и сказал ей: иди, ложись со мною, сестра моя.
\par 12 Но она сказала: нет, брат мой, не бесчести меня, ибо не делается так в Израиле; не делай этого безумия.
\par 13 И я, куда пойду я с моим бесчестием? И ты, ты будешь одним из безумных в Израиле. Ты поговори с царем; он не откажет отдать меня тебе.
\par 14 Но он не хотел слушать слов ее, и преодолел ее, и изнасиловал ее, и лежал с нею.
\par 15 Потом возненавидел ее Амнон величайшею ненавистью, так что ненависть, какою он возненавидел ее, была сильнее любви, какую имел к ней; и сказал ей Амнон: встань, уйди.
\par 16 И [Фамарь] сказала ему: нет, прогнать меня--это зло больше первого, которое ты сделал со мною. Но он не хотел слушать ее.
\par 17 И позвал отрока своего, который служил ему, и сказал: прогони эту от меня вон и запри дверь за нею.
\par 18 На ней была разноцветная одежда, ибо такие верхние одежды носили царские дочери-девицы. И вывел ее слуга вон и запер за нею дверь.
\par 19 И посыпала Фамарь пеплом голову свою, и разодрала разноцветную одежду, которую имела на себе, и положила руки свои на голову свою, и так шла и вопила.
\par 20 И сказал ей Авессалом, брат ее: не Амнон ли, брат твой, был с тобою? --но теперь молчи, сестра моя; он--брат твой; не сокрушайся сердцем твоим об этом деле. И жила Фамарь в одиночестве в доме Авессалома, брата своего.
\par 21 И услышал царь Давид обо всем этом, и сильно разгневался.
\par 22 Авессалом же не говорил с Амноном ни худого, ни хорошего; ибо возненавидел Авессалом Амнона за то, что он обесчестил Фамарь, сестру его.
\par 23 Чрез два года было стрижение [овец] у Авессалома в Ваал-- Гацоре, что у Ефрема, и позвал Авессалом всех сыновей царских.
\par 24 И пришел Авессалом к царю и сказал: вот, ныне стрижение [овец] у раба твоего; пусть пойдет царь и слуги его с рабом твоим.
\par 25 Но царь сказал Авессалому: нет, сын мой, мы не пойдем все, чтобы не быть тебе в тягость. И сильно упрашивал его [Авессалом]; но он не захотел идти, и благословил его.
\par 26 И сказал ему Авессалом: по крайней мере пусть пойдет с нами Амнон, брат мой. И сказал ему царь: зачем ему идти с тобою?
\par 27 Но Авессалом упросил его, и он отпустил с ним Амнона и всех царских сыновей.
\par 28 Авессалом же приказал отрокам своим, сказав: смотрите, как только развеселится сердце Амнона от вина, и я скажу вам: `поразите Амнона', тогда убейте его, не бойтесь; это я приказываю вам, будьте смелы и мужественны.
\par 29 И поступили отроки Авессалома с Амноном, как приказал Авессалом. Тогда встали все царские сыновья, сели каждый на мула своего и убежали.
\par 30 Когда они были еще на пути, дошел слух до Давида, что Авессалом умертвил всех царских сыновей, и не осталось ни одного из них.
\par 31 И встал царь, и разодрал одежды свои, и повергся на землю, и все слуги его, предстоящие ему, разодрали одежды свои.
\par 32 Но Ионадав, сын Самая, брата Давидова, сказал: пусть не думает господин мой, что всех отроков, царских сыновей, умертвили; один только Амнон умер, ибо у Авессалома был этот замысел с того дня, как [Амнон] обесчестил сестру его;
\par 33 итак пусть господин мой, царь, не тревожится мыслью о том, будто умерли все царские сыновья: умер один только Амнон.
\par 34 И убежал Авессалом. И поднял отрок, стоявший на страже, глаза свои, и увидел: вот, много народа идет по дороге по скату горы.
\par 35 Тогда Ионадав сказал царю: это идут царские сыновья; как говорил раб твой, так и есть.
\par 36 И едва только сказал он это, вот пришли царские сыновья, и подняли вопль и плакали. И сам царь и все слуги его плакали очень великим плачем.
\par 37 Авессалом же убежал и пошел к Фалмаю, сыну Емиуда, царю Гессурскому. И плакал Давид о сыне своем во все дни.
\par 38 Авессалом убежал и пришел в Гессур и пробыл там три года.
\par 39 И не стал царь Давид преследовать Авессалома; ибо утешился о смерти Амнона.

\chapter{14}

\par 1 И заметил Иоав, сын Саруи, что сердце царя обратилось к Авессалому.
\par 2 И послал Иоав в Фекою, и взял оттуда умную женщину и сказал ей: притворись плачущею и надень печальную одежду, и не мажься елеем, и представься женщиною, много дней плакавшею по умершем;
\par 3 и пойди к царю и скажи ему так и так. И вложил Иоав в уста ее, что сказать.
\par 4 И вошла женщина Фекоитянка к царю и пала лицем своим на землю, и поклонилась и сказала: помоги, царь!
\par 5 И сказал ей царь: что тебе? И сказала она: я вдова, муж мой умер;
\par 6 и у рабы твоей [было] два сына; они поссорились в поле, и некому было разнять их, и поразил один другого и умертвил его.
\par 7 И вот, восстало все родство на рабу твою, и говорят: `отдай убийцу брата своего; мы убьем его за душу брата его, которую он погубил, и истребим даже наследника'. И так они погасят остальную искру мою, чтобы не оставить мужу моему имени и потомства на лице земли.
\par 8 И сказал царь женщине: иди спокойно домой, я дам приказание о тебе.
\par 9 Но женщина Фекоитянка сказала царю: на мне, господин мой царь, да будет вина и на доме отца моего, царь же и престол его неповинен.
\par 10 И сказал царь: того, кто будет против тебя, приведи ко мне, и он более не тронет тебя.
\par 11 Она сказала: помяни, царь, Господа Бога твоего, чтобы не умножились мстители за кровь и не погубили сына моего. И сказал [царь]: жив Господь! не падет и волос сына твоего на землю.
\par 12 И сказала женщина: позволь рабе твоей сказать [еще] слово господину моему царю.
\par 13 Он сказал: говори. И сказала женщина: почему ты так мыслишь против народа Божия? Царь, произнеся это слово, обвинил себя самого, потому что не возвращает изгнанника своего.
\par 14 Мы умрем и [будем] как вода, вылитая на землю, которую нельзя собрать; но Бог не желает погубить душу и помышляет, как бы не отвергнуть от Себя и отверженного.
\par 15 И теперь я пришла сказать царю, господину моему, эти слова, потому что народ пугает меня; и раба твоя сказала: поговорю я с царем, не сделает ли он по слову рабы своей;
\par 16 верно царь выслушает и избавит рабу свою от руки людей, [хотящих] истребить меня вместе с сыном моим из наследия Божия.
\par 17 И сказала раба твоя: да будет слово господина моего царя в утешение мне, ибо господин мой царь, как Ангел Божий, и может выслушать и доброе и худое. И Господь Бог твой будет с тобою.
\par 18 И отвечал царь и сказал женщине: не скрой от меня, о чем я спрошу тебя. И сказала женщина: говори, господин мой царь.
\par 19 И сказал царь: не рука ли Иоава во всем этом с тобою? И отвечала женщина и сказала: да живет душа твоя, господин мой царь; ни направо, ни налево нельзя уклониться от того, что сказал господин мой, царь; точно, раб твой Иоав приказал мне, и он вложил в уста рабы твоей все эти слова;
\par 20 чтобы притчею дать делу такой вид, раб твой Иоав научил меня; но господин мой мудр, как мудр Ангел Божий, чтобы знать все, что на земле.
\par 21 И сказал царь Иоаву: вот, я сделал [по слову твоему]; пойди же, возврати отрока Авессалома.
\par 22 Тогда Иоав пал лицем на землю и поклонился, и благословил царя и сказал: теперь знает раб твой, что обрел благоволение пред очами твоими, господин мой царь, так как царь сделал по слову раба своего.
\par 23 И встал Иоав, и пошел в Гессур, и привел Авессалома в Иерусалим.
\par 24 И сказал царь: пусть он возвратится в дом свой, а лица моего не видит. И пошел Авессалом в свой дом, а лица царского не видал.
\par 25 Не было во всем Израиле мужчины столь красивого, как Авессалом, и столько хвалимого, как он; от подошвы ног до верха головы его не было у него недостатка.
\par 26 Когда он стриг голову свою, --а он стриг ее каждый год, потому что она отягощала его, --то волоса с головы его весили двести сиклей по весу царскому.
\par 27 И родились у Авессалома три сына и одна дочь, по имени Фамарь; она была женщина красивая.
\par 28 И оставался Авессалом в Иерусалиме два года, а лица царского не видал.
\par 29 И послал Авессалом за Иоавом, чтобы послать его к царю, но тот не захотел придти к нему. Послал и в другой раз; но тот не захотел придти.
\par 30 И сказал [Авессалом] слугам своим: видите участок поля Иоава подле моего, и у него там ячмень; пойдите, выжгите его огнем. И выжгли слуги Авессалома тот участок поля огнем.
\par 31 И встал Иоав, и пришел к Авессалому в дом, и сказал ему: зачем слуги твои выжгли мой участок огнем?
\par 32 И сказал Авессалом Иоаву: вот, я посылал за тобою, говоря: приди сюда, и я пошлю тебя к царю сказать: зачем я пришел из Гессура? Лучше было бы мне оставаться там. Я хочу увидеть лице царя. Если же я виноват, то убей меня.
\par 33 И пошел Иоав к царю и пересказал ему [это]. И позвал [царь] Авессалома; он пришел к царю, и пал лицем своим на землю пред царем; и поцеловал царь Авессалома.

\chapter{15}

\par 1 После сего Авессалом завел у себя колесницы и лошадей и пятьдесят скороходов.
\par 2 И вставал Авессалом рано утром, и становился при дороге у ворот, и когда кто-нибудь, имея тяжбу, шел к царю на суд, то Авессалом подзывал его к себе и спрашивал: из какого города ты? И когда тот отвечал: из такого-то колена Израилева раб твой,
\par 3 тогда говорил ему Авессалом: вот, дело твое доброе и справедливое, но у царя некому выслушать тебя.
\par 4 И говорил Авессалом: о, если бы меня поставили судьею в этой земле! ко мне приходил бы всякий, кто имеет спор и тяжбу, и я судил бы его по правде.
\par 5 И когда подходил кто-нибудь поклониться ему, то он простирал руку свою и обнимал его и целовал его.
\par 6 Так поступал Авессалом со всяким Израильтянином, приходившим на суд к царю, и вкрадывался Авессалом в сердце Израильтян.
\par 7 По прошествии сорока лет [царствования Давида], Авессалом сказал царю: пойду я и исполню обет мой, который я дал Господу, в Хевроне;
\par 8 ибо я, раб твой, живя в Гессуре в Сирии, дал обет: если Господь возвратит меня в Иерусалим, то я принесу жертву Господу.
\par 9 И сказал ему царь: иди с миром. И встал он и пошел в Хеврон.
\par 10 И разослал Авессалом лазутчиков во все колена Израилевы, сказав: когда вы услышите звук трубы, то говорите: Авессалом воцарился в Хевроне.
\par 11 С Авессаломом пошли из Иерусалима двести человек, которые были приглашены им, и пошли по простоте своей, не зная, в чем дело.
\par 12 Во время жертвоприношения Авессалом послал и призвал Ахитофела Гилонянина, советника Давидова, из его города Гило. И составился сильный заговор, и народ стекался и умножался около Авессалома.
\par 13 И пришел вестник к Давиду и сказал: сердце Израильтян уклонилось на сторону Авессалома.
\par 14 И сказал Давид всем слугам своим, которые были при нем в Иерусалиме: встаньте, убежим, ибо не будет нам спасения от Авессалома; спешите, чтобы нам уйти, чтоб он не застиг и не захватил нас, и не навел на нас беды и не истребил города мечом.
\par 15 И сказали слуги царские царю: во всем, что угодно господину нашему царю, мы--рабы твои.
\par 16 И вышел царь и весь дом его за ним пешком. Оставил же царь десять жен, наложниц [своих], для хранения дома.
\par 17 И вышел царь и весь народ пешие, и остановились у Беф-Мерхата.
\par 18 И все слуги его шли по сторонам его, и все Хелефеи, и все Фелефеи, и все Гефяне до шестисот человек, пришедшие вместе с ним из Гефа, шли впереди царя.
\par 19 И сказал царь Еффею Гефянину: зачем и ты идешь с нами? Возвратись и оставайся с тем царем; ибо ты--чужеземец и пришел сюда из своего места;
\par 20 вчера ты пришел, а сегодня я заставлю тебя идти с нами? Я иду, куда случится; возвратись и возврати братьев своих с собою; милость и истину [с тобою]!
\par 21 И отвечал Еффей царю и сказал: жив Господь, и да живет господин мой царь: где бы ни был господин мой царь, в жизни ли, в смерти ли, там будет и раб твой.
\par 22 И сказал Давид Еффею: итак иди и ходи со мною. И пошел Еффей Гефянин и все люди его и все дети, бывшие с ним.
\par 23 И плакала вся земля громким голосом. И весь народ переходил, и царь перешел поток Кедрон; и пошел весь народ по дороге к пустыне.
\par 24 Вот и Садок, и все левиты с ним несли ковчег завета Божия из Вефары и поставили ковчег Божий; Авиафар же стоял на возвышении, доколе весь народ не вышел из города.
\par 25 И сказал царь Садоку: возврати ковчег Божий в город. Если я обрету милость пред очами Господа, то Он возвратит меня и даст мне видеть его и жилище его.
\par 26 А если Он скажет так: `нет Моего благоволения к тебе', то вот я; пусть творит со мною, что Ему благоугодно.
\par 27 И сказал царь Садоку священнику: видишь ли, --возвратись в город с миром, и Ахимаас, сын твой, и Ионафан, сын Авиафара, оба сына ваши с вами;
\par 28 видите ли, я помедлю на равнине в пустыне, доколе не придет известие от вас ко мне.
\par 29 И возвратили Садок и Авиафар ковчег Божий в Иерусалим, и остались там.
\par 30 А Давид пошел на гору Елеонскую, шел и плакал; голова у него была покрыта; он шел босой, и все люди, бывшие с ним, покрыли каждый голову свою, шли и плакали.
\par 31 Донесли Давиду и сказали: и Ахитофел в числе заговорщиков с Авессаломом. И сказал Давид: Господи! разрушь совет Ахитофела.
\par 32 Когда Давид взошел на вершину горы, где он поклонялся Богу, вот навстречу ему идет Хусий Архитянин, друг Давидов; одежда на нем была разодрана, и прах на голове его.
\par 33 И сказал ему Давид: если ты пойдешь со мною, то будешь мне в тягость;
\par 34 но если возвратишься в город и скажешь Авессалому: `царь, я раб твой; доселе я был рабом отца твоего, а теперь я--твой раб': то ты расстроишь для меня совет Ахитофела.
\par 35 Вот, там с тобою Садок и Авиафар священники, и всякое слово, какое услышишь из дома царя, пересказывай Садоку и Авиафару священникам.
\par 36 Там с ними и два сына их, Ахимаас, сын Садока, и Ионафан, сын Авиафара; чрез них посылайте ко мне всякое известие, какое услышите.
\par 37 И пришел Хусий, друг Давида, в город; Авессалом же вступал тогда в Иерусалим.

\chapter{16}

\par 1 Когда Давид немного сошел с вершины горы, вот встречается ему Сива, слуга Мемфивосфея, с парою навьюченных ослов, и на них двести хлебов, сто связок изюму, сто связок смокв и мех с вином.
\par 2 И сказал царь Сиве: для чего это у тебя? И отвечал Сива: ослы для дома царского, для езды, а хлеб и плоды для пищи отрокам, а вино для питья ослабевшим в пустыне.
\par 3 И сказал царь: где сын господина твоего? И отвечал Сива царю: вот, он остался в Иерусалиме и говорит: теперь-то дом Израилев возвратит мне царство отца моего.
\par 4 И сказал царь Сиве: вот тебе все, что у Мемфивосфея. И отвечал Сива, поклонившись: да обрету милость в глазах господина моего царя!
\par 5 Когда дошел царь Давид до Бахурима, вот вышел оттуда человек из рода дома Саулова, по имени Семей, сын Геры; он шел и злословил,
\par 6 и бросал камнями на Давида и на всех рабов царя Давида; все же люди и все храбрые были по правую и по левую сторону [царя].
\par 7 Так говорил Семей, злословя его: уходи, уходи, убийца и беззаконник!
\par 8 Господь обратил на тебя всю кровь дома Саулова, вместо которого ты воцарился, и предал Господь царство в руки Авессалома, сына твоего; и вот, ты в беде, ибо ты--кровопийца.
\par 9 И сказал Авесса, сын Саруин, царю: зачем злословит этот мертвый пес господина моего царя? пойду я и сниму с него голову.
\par 10 И сказал царь: что мне и вам, сыны Саруины? пусть он злословит, ибо Господь повелел ему злословить Давида. Кто же может сказать: зачем ты так делаешь?
\par 11 И сказал Давид Авессе и всем слугам своим: вот, если мой сын, который вышел из чресл моих, ищет души моей, тем больше сын Вениамитянина; оставьте его, пусть злословит, ибо Господь повелел ему;
\par 12 может быть, Господь призрит на уничижение мое, и воздаст мне Господь благостью за теперешнее его злословие.
\par 13 И шел Давид и люди его [своим] путем, а Семей шел по окраине горы, со стороны его, шел и злословил, и бросал камнями на сторону его и пылью.
\par 14 И пришел царь и весь народ, бывший с ним, утомленный, и отдыхал там.
\par 15 Авессалом же и весь народ Израильский пришли в Иерусалим, и Ахитофел с ним.
\par 16 Когда Хусий Архитянин, друг Давидов, пришел к Авессалому, то сказал Хусий Авессалому: да живет царь, да живет царь!
\par 17 И сказал Авессалом Хусию: таково-то усердие твое к твоему другу! отчего ты не пошел с другом твоим?
\par 18 И сказал Хусий Авессалому: нет, кого избрал Господь и этот народ и весь Израиль, с тем и я, и с ним останусь.
\par 19 И притом кому я буду служить? Не сыну ли его? Как служил я отцу твоему, так буду служить и тебе.
\par 20 И сказал Авессалом Ахитофелу: дайте совет, что нам делать.
\par 21 И сказал Ахитофел Авессалому: войди к наложницам отца твоего, которых он оставил охранять дом свой; и услышат все Израильтяне, что ты сделался ненавистным для отца твоего, и укрепятся руки всех, которые с тобою.
\par 22 И поставили для Авессалома палатку на кровле, и вошел Авессалом к наложницам отца своего пред глазами всего Израиля.
\par 23 Советы же Ахитофела, которые он давал, в то время [считались], как если бы кто спрашивал наставления у Бога. Таков был всякий совет Ахитофела как для Давида, так и для Авессалома.

\chapter{17}

\par 1 И сказал Ахитофел Авессалому: выберу я двенадцать тысяч человек и встану и пойду в погоню за Давидом в эту ночь;
\par 2 и нападу на него, когда он будет утомлен и с опущенными руками, и приведу его в страх; и все люди, которые с ним, разбегутся; и я убью одного царя
\par 3 и всех людей обращу к тебе; и когда не будет одного, душу которого ты ищешь, тогда весь народ будет в мире.
\par 4 И понравилось это слово Авессалому и всем старейшинам Израилевым.
\par 5 И сказал Авессалом: позовите Хусия Архитянина; послушаем, что он скажет.
\par 6 И пришел Хусий к Авессалому, и сказал ему Авессалом, говоря: вот что говорит Ахитофел; сделать ли по его словам? а если нет, то говори ты.
\par 7 И сказал Хусий Авессалому: нехорош на этот раз совет, который дал Ахитофел.
\par 8 И продолжал Хусий: ты знаешь твоего отца и людей его; они храбры и сильно раздражены, как медведица в поле, у которой отняли детей, и отец твой--человек воинственный; он не остановится ночевать с народом.
\par 9 Вот, теперь он скрывается в какой-нибудь пещере, или в другом месте, и если кто падет при первом нападении на них, и услышат и скажут: `было поражение людей, последовавших за Авессаломом',
\par 10 тогда и самый храбрый, у которого сердце, как сердце львиное, упадет духом; ибо всему Израилю известно, как храбр отец твой и мужественны те, которые с ним.
\par 11 Посему я советую: пусть соберется к тебе весь Израиль, от Дана до Вирсавии, во множестве, как песок при море, и ты сам пойдешь посреди его;
\par 12 и тогда мы пойдем против него, в каком бы месте он ни находился, и нападем на него, как падает роса на землю; и не останется у него ни одного человека из всех, которые с ним;
\par 13 а если он войдет в какой-либо город, то весь Израиль принесет к тому городу веревки, и мы стащим его в реку, так что не останется ни одного камешка.
\par 14 И сказал Авессалом и весь Израиль: совет Хусия Архитянина лучше совета Ахитофелова. Так Господь судил разрушить лучший совет Ахитофела, чтобы навести Господу бедствие на Авессалома.
\par 15 И сказал Хусий Садоку и Авиафару священникам: так и так советовал Ахитофел Авессалому и старейшинам Израилевым, а так и так посоветовал я.
\par 16 И теперь пошлите поскорее и скажите Давиду так: не оставайся в эту ночь на равнине в пустыне, но поскорее перейди, чтобы не погибнуть царю и всем людям, которые с ним.
\par 17 Ионафан и Ахимаас стояли у источника Рогель. И пошла служанка и рассказала им, а они пошли и известили царя Давида; ибо они не могли показаться в городе.
\par 18 И увидел их отрок и донес Авессалому; но они оба скоро ушли и пришли в Бахурим, в дом одного человека, у которого на дворе был колодезь, и спустились туда.
\par 19 А женщина взяла и растянула над устьем колодезя покрывало и насыпала на него крупы, так что не было ничего заметно.
\par 20 И пришли рабы Авессалома к женщине в дом, и сказали: где Ахимаас и Ионафан? И сказала им женщина: они перешли вброд реку. И искали они, и не нашли, и возвратились в Иерусалим.
\par 21 Когда они ушли, те вышли из колодезя, пошли и известили царя Давида и сказали Давиду: встаньте и поскорее перейдите воду; ибо так и так советовал о вас Ахитофел.
\par 22 И встал Давид и все люди, бывшие с ним, и перешли Иордан; к рассвету не осталось ни одного, который не перешел бы Иордана.
\par 23 И увидел Ахитофел, что не исполнен совет его, и оседлал осла, и собрался, и пошел в дом свой, в город свой, и сделал завещание дому своему, и удавился, и умер, и был погребен в гробе отца своего.
\par 24 И пришел Давид в Маханаим, а Авессалом перешел Иордан, сам и весь Израиль с ним.
\par 25 Авессалом поставил Амессая, вместо Иоава, над войском. Амессай был сын одного человека, по имени Иефера из Изрееля, который вошел к Авигее, дочери Нааса, сестре Саруи, матери Иоава.
\par 26 И Израиль с Авессаломом расположился станом в земле Галаадской.
\par 27 Когда Давид пришел в Маханаим, то Сови, сын Нааса, из Раввы Аммонитской, и Махир, сын Аммиила, из Лодавара, и Верзеллий Галаадитянин из Роглима,
\par 28 принесли постелей, блюд и глиняных сосудов, и пшеницы, и ячменя, и муки, и пшена, и бобов, и чечевицы, и жареных зерен,
\par 29 и меду, и масла, и овец, и сыра коровьего, принесли Давиду и людям, бывшим с ним, в пищу; ибо говорили они: народ голоден и утомлен и терпел жажду в пустыне.

\chapter{18}

\par 1 И осмотрел Давид людей, бывших с ним, и поставил над ними тысяченачальников и сотников.
\par 2 И отправил Давид людей--третью часть под предводительством Иоава, третью часть под предводительством Авессы, сына Саруина, брата Иоава, третью часть под предводительством Еффея Гефянина. И сказал царь людям: я сам пойду с вами.
\par 3 Но люди отвечали ему: не ходи; ибо, если мы и побежим, то не обратят внимания на это; если и умрет половина из нас, также не обратят внимания; а ты один то же, что нас десять тысяч; итак для нас лучше, чтобы ты помогал нам из города.
\par 4 И сказал им царь: что угодно в глазах ваших, то и сделаю. И стал царь у ворот, и весь народ выходил по сотням и по тысячам.
\par 5 И приказал царь Иоаву и Авессе и Еффею, говоря: сберегите мне отрока Авессалома. И все люди слышали, как приказывал царь всем начальникам об Авессаломе.
\par 6 И вышли люди в поле навстречу Израильтянам, и было сражение в лесу Ефремовом.
\par 7 И был поражен народ Израильский рабами Давида; было там поражение великое в тот день, --поражены двадцать тысяч [человек].
\par 8 Сражение распространилось по всей той стране, и лес погубил народа больше, чем сколько истребил меч, в тот день.
\par 9 И встретился Авессалом с рабами Давидовыми; он был на муле. Когда мул вбежал с ним под ветви большого дуба, то [Авессалом] запутался волосами своими в ветвях дуба и повис между небом и землею, а мул, бывший под ним, убежал.
\par 10 И увидел это некто и донес Иоаву, говоря: вот, я видел Авессалома висящим на дубе.
\par 11 И сказал Иоав человеку, донесшему об этом: вот, ты видел; зачем же ты не поверг его там на землю? я дал бы тебе десять сиклей серебра и один пояс.
\par 12 И отвечал тот Иоаву: если бы положили на руки мои и тысячу сиклей серебра, и тогда я не поднял бы руки на царского сына; ибо вслух нас царь приказывал тебе и Авессе и Еффею, говоря: `сберегите мне отрока Авессалома';
\par 13 и если бы я поступил иначе с опасностью жизни моей, то это не скрылось бы от царя, и ты же восстал бы против меня.
\par 14 Иоав сказал: нечего мне медлить с тобою. И взял в руки три стрелы и вонзил их в сердце Авессалома, который был еще жив на дубе.
\par 15 И окружили Авессалома десять отроков, оруженосцев Иоава, и поразили и умертвили его.
\par 16 И затрубил Иоав трубою, и возвратились люди из погони за Израилем, ибо Иоав щадил народ.
\par 17 И взяли Авессалома, и бросили его в лесу в глубокую яму, и наметали над ним огромную кучу камней. И все Израильтяне разбежались, каждый в шатер свой.
\par 18 Авессалом еще при жизни своей взял и поставил себе памятник в царской долине; ибо сказал он: нет у меня сына, чтобы сохранилась память имени моего. И назвал памятник своим именем. И называется он `памятник Авессалома' до сего дня.
\par 19 Ахимаас, сын Садоков, сказал Иоаву: побегу я, извещу царя, что Господь судом Своим избавил его от рук врагов его.
\par 20 Но Иоав сказал ему: не будешь ты сегодня добрым вестником; известишь в другой день, а не сегодня, ибо умер сын царя.
\par 21 И сказал Иоав Хусию: пойди, донеси царю, что видел ты. И поклонился Хусий Иоаву и побежал.
\par 22 Но Ахимаас, сын Садоков, настаивал и говорил Иоаву: что бы ни было, но и я побегу за Хусием. Иоав же отвечал: зачем бежать тебе, сын мой? не принесешь ты доброй вести.
\par 23 [И сказал Ахимаас]: пусть так, но я побегу. И сказал ему [Иоав]: беги. И побежал Ахимаас по прямой дороге и опередил Хусия.
\par 24 Давид тогда сидел между двумя воротами. И сторож взошел на кровлю ворот к стене и, подняв глаза, увидел: вот, бежит один человек.
\par 25 И закричал сторож и известил царя. И сказал царь: если один, то весть в устах его. А тот подходил все ближе и ближе.
\par 26 Сторож увидел и другого бегущего человека; и закричал сторож привратнику: вот, еще бежит один человек. Царь сказал: и это--вестник.
\par 27 Сторож сказал: я вижу походку первого, похожую на походку Ахимааса, сына Садокова. И сказал царь: это человек хороший и идет с хорошею вестью.
\par 28 И воскликнул Ахимаас и сказал царю: мир. И поклонился царю лицем своим до земли и сказал: благословен Господь Бог твой, предавший людей, которые подняли руки свои на господина моего царя!
\par 29 И сказал царь: благополучен ли отрок Авессалом? И сказал Ахимаас: я видел большое волнение, когда раб царев Иоав посылал раба твоего; но я не знаю, что [там] было.
\par 30 И сказал царь: отойди, стань здесь. Он отошел и стал.
\par 31 Вот, пришел и Хусий. И сказал Хусий: добрая весть господину моему царю! Господь явил тебе ныне правду в избавлении от руки всех восставших против тебя.
\par 32 И сказал царь Хусию: благополучен ли отрок Авессалом? И сказал Хусий: да будет с врагами господина моего царя и со всеми, злоумышляющими против тебя то же, что постигло отрока!
\par 33 И смутился царь, и пошел в горницу над воротами, и плакал, и когда шел, говорил так: сын мой Авессалом! сын мой, сын мой Авессалом! о, кто дал бы мне умереть вместо тебя, Авессалом, сын мой, сын мой!

\chapter{19}

\par 1 И сказали Иоаву: вот, царь плачет и рыдает об Авессаломе.
\par 2 И обратилась победа того дня в плач для всего народа; ибо народ услышал в тот день и говорил, что царь скорбит о своем сыне.
\par 3 И входил тогда народ в город украдкою, как крадутся люди стыдящиеся, которые во время сражения обратились в бегство.
\par 4 А царь закрыл лице свое и громко взывал: сын мой Авессалом! Авессалом, сын мой, сын мой!
\par 5 И пришел Иоав к царю в дом и сказал: ты в стыд привел сегодня всех слуг твоих, спасших ныне жизнь твою и жизнь сыновей и дочерей твоих, и жизнь жен и жизнь наложниц твоих;
\par 6 ты любишь ненавидящих тебя и ненавидишь любящих тебя, ибо ты показал сегодня, что ничто для тебя и вожди и слуги; сегодня я узнал, что если бы Авессалом остался жив, а мы все умерли, то тебе было бы приятнее;
\par 7 итак встань, выйди и поговори к сердцу рабов твоих, ибо клянусь Господом, что, если ты не выйдешь, в эту ночь не останется у тебя ни одного человека; и это будет для тебя хуже всех бедствий, какие находили на тебя от юности твоей доныне.
\par 8 И встал царь и сел у ворот, а всему народу возвестили, что царь сидит у ворот. И пришел весь народ пред лице царя; Израильтяне же разбежались по своим шатрам.
\par 9 И весь народ во всех коленах Израилевых спорил и говорил: царь избавил нас от рук врагов наших и освободил нас от рук Филистимлян, а теперь сам бежал из земли сей, от Авессалома.
\par 10 Но Авессалом, которого мы помазали [в царя] над нами, умер на войне; почему же теперь вы медлите возвратить царя?
\par 11 И царь Давид послал сказать священникам Садоку и Авиафару: скажите старейшинам Иудиным: зачем хотите вы быть последними, чтобы возвратить царя в дом его, тогда как слова всего Израиля дошли до царя в дом его?
\par 12 Вы братья мои, кости мои и плоть моя--вы; зачем хотите вы быть последними в возвращении царя в дом его?
\par 13 И Амессаю скажите: не кость ли моя и плоть моя--ты? Пусть то и то сделает со мною Бог и еще больше сделает, если ты не будешь военачальником при мне, вместо Иоава, навсегда!
\par 14 И склонил он сердце всех Иудеев, как одного человека; и послали они к царю [сказать]: возвратись ты и все слуги твои.
\par 15 И возвратился царь, и пришел к Иордану, а Иудеи пришли в Галгал, чтобы встретить царя и перевезти царя чрез Иордан.
\par 16 И поспешил Семей, сын Геры, Вениамитянин из Бахурима, и пошел с Иудеями навстречу царю Давиду,
\par 17 и тысяча человек из Вениамитян с ним, и Сива, слуга дома Саулова, с пятнадцатью сыновьями своими и двадцатью рабами своими; и перешли они Иордан пред лицем царя.
\par 18 Когда переправили судно, чтобы перевезти дом царя и послужить ему, тогда Семей, сын Геры, пал пред царем, как только он перешел Иордан,
\par 19 и сказал царю: не поставь мне, господин мой, в преступление, и не помяни того, чем согрешил раб твой в тот день, когда господин мой царь выходил из Иерусалима, и не держи [того], царь, на сердце своем;
\par 20 ибо знает раб твой, что согрешил, и вот, ныне я пришел первый из всего дома Иосифова, чтобы выйти навстречу господину моему царю.
\par 21 И отвечал Авесса, сын Саруин, и сказал: неужели Семей не умрет за то, что злословил помазанника Господня?
\par 22 И сказал Давид: что мне и вам, сыны Саруины, что вы делаетесь ныне мне наветниками? Ныне ли умерщвлять кого-либо в Израиле? Не вижу ли я, что ныне я--царь над Израилем?
\par 23 И сказал царь Семею: ты не умрешь. И поклялся ему царь.
\par 24 И Мемфивосфей, сын [Ионафана, сына] Саулова, вышел навстречу царю. Он не омывал ног своих, не заботился о бороде своей и не мыл одежд своих с того дня, как вышел царь, до дня, когда он возвратился с миром.
\par 25 Когда он вышел из Иерусалима навстречу царю, царь сказал ему: почему ты, Мемфивосфей, не пошел со мною?
\par 26 Тот отвечал: господин мой царь! слуга мой обманул меня; ибо я, раб твой, говорил: `оседлаю себе осла и сяду на нем и поеду с царем', так как раб твой хром.
\par 27 А он оклеветал раба твоего пред господином моим царем. Но господин мой царь, как Ангел Божий; делай, что тебе угодно;
\par 28 хотя весь дом отца моего был повинен смерти пред господином моим царем, но ты посадил раба твоего между ядущими за столом твоим; какое же имею я право жаловаться еще пред царем?
\par 29 И сказал ему царь: к чему ты говоришь все это? я сказал, чтобы ты и Сива разделили [между собою] поля.
\par 30 Но Мемфивосфей отвечал царю: пусть он возьмет даже все, после того как господин мой царь, с миром возвратился в дом свой.
\par 31 И Верзеллий Галаадитянин пришел из Роглима и перешел с царем Иордан, чтобы проводить его за Иордан.
\par 32 Верзеллий же был очень стар, лет восьмидесяти. Он продовольствовал царя в пребывание его в Маханаиме, потому что был человек богатый.
\par 33 И сказал царь Верзеллию: иди со мною, и я буду продовольствовать тебя в Иерусалиме.
\par 34 Но Верзеллий отвечал царю: долго ли мне осталось жить, чтоб идти с царем в Иерусалим?
\par 35 Мне теперь восемьдесят лет; различу ли хорошее от худого? Узнает ли раб твой вкус в том, что буду есть, и в том, что буду пить? И буду ли в состоянии слышать голос певцов и певиц? Зачем же рабу твоему быть в тягость господину моему царю?
\par 36 Еще немного пройдет раб твой с царем за Иордан; за что же царю награждать меня такою милостью?
\par 37 Позволь рабу твоему возвратиться, чтобы умереть в своем городе, около гроба отца моего и матери моей. Но вот, раб твой, [сын мой] Кимгам пусть пойдет с господином моим, царем, и поступи с ним, как тебе угодно.
\par 38 И сказал царь: пусть идет со мною Кимгам, и я сделаю для него, что тебе угодно; и все, чего бы ни пожелал ты от меня, я сделаю для тебя.
\par 39 И перешел весь народ Иордан, и царь [также]. И поцеловал царь Верзеллия и благословил его, и он возвратился в место свое.
\par 40 И отправился царь в Галгал, отправился с ним и Кимгам; и весь народ Иудейский провожал царя, и половина народа Израильского.
\par 41 И вот, все Израильтяне пришли к царю и сказали царю: зачем братья наши, мужи Иудины, похитили тебя и проводили царя в дом его и всех людей Давида с ним через Иордан?
\par 42 И отвечали все мужи Иудины Израильтянам: затем, что царь ближний нам; и из-за чего сердиться вам на это? Разве мы что-- нибудь съели у царя, или получили от него подарки?
\par 43 И отвечали Израильтяне мужам Иудиным и сказали: мы десять частей у царя, также и у Давида мы более, нежели вы; зачем же вы унизили нас? Не нам ли принадлежало первое слово о том, чтобы возвратить нашего царя? Но слово мужей Иудиных было сильнее, нежели слово Израильтян.

\chapter{20}

\par 1 Там случайно находился один негодный человек, по имени Савей, сын Бихри, Вениамитянин; он затрубил трубою и сказал: нет нам части в Давиде, и нет нам доли в сыне Иессеевом; все по шатрам своим, Израильтяне!
\par 2 И отделились все Израильтяне от Давида [и пошли] за Савеем, сыном Бихри; Иудеи же остались на стороне царя своего, от Иордана до Иерусалима.
\par 3 И пришел Давид в свой дом в Иерусалиме, и взял царь десять жен наложниц, которых он оставлял стеречь дом, и поместил их в особый дом под надзор, и содержал их, но не ходил к ним. И содержались они там до дня смерти своей, живя как вдовы.
\par 4 И сказал Давид Амессаю: созови ко мне Иудеев в течение трех дней и сам явись сюда.
\par 5 И пошел Амессай созвать Иудеев, но промедлил более назначенного ему времени.
\par 6 Тогда Давид сказал Авессе: теперь наделает нам зла Савей, сын Бихри, больше нежели Авессалом; возьми ты слуг господина твоего и преследуй его, чтобы он не нашел себе укрепленных городов и не скрылся от глаз наших.
\par 7 И вышли за ним люди Иоавовы, и Хелефеи и Фелефеи, и все храбрые пошли из Иерусалима преследовать Савея, сына Бихри.
\par 8 И когда они были близ большого камня, что у Гаваона, то встретился с ними Амессай. Иоав был одет в воинское одеяние свое и препоясан мечом, который висел при бедре в ножнах и который легко выходил из них и входил.
\par 9 И сказал Иоав Амессаю: здоров ли ты, брат мой? И взял Иоав правою рукою Амессая за бороду, чтобы поцеловать его.
\par 10 Амессай же не остерегся меча, бывшего в руке Иоава, и тот поразил его им в живот, так что выпали внутренности его на землю, и не повторил ему [удара], и он умер. Иоав и Авесса, брат его, погнались за Савеем, сыном Бихри.
\par 11 Один из отроков Иоавовых стоял над [Амессаем] и говорил: тот, кто предан Иоаву и кто за Давида, [пусть идет] за Иоавом!
\par 12 Амессай же лежал в крови среди дороги; и тот человек, увидев, что весь народ останавливается над ним, стащил Амессая с дороги в поле и набросил на него одежду, так как он видел, что всякий проходящий останавливался над ним.
\par 13 Но когда он был стащен с дороги, то весь народ Израильский пошел вслед за Иоавом преследовать Савея, сына Бихри.
\par 14 А он прошел чрез все колена Израильские до Авела-Беф-Мааха и чрез весь Берим; и [жители] собирались и шли за ним.
\par 15 И пришли и осадили его в Авеле-Беф-Маахе; и насыпали вал пред городом и подступили к стене, и все люди, бывшие с Иоавом, старались разрушить стену.
\par 16 [Тогда] одна умная женщина закричала со стены города: послушайте, послушайте, скажите Иоаву, чтоб он подошел сюда, и я поговорю с ним.
\par 17 И подошел к ней Иоав, и сказала женщина: ты ли Иоав? И сказал: я. Она сказала: послушай слов рабы твоей. И сказал он: слушаю.
\par 18 Она сказала: прежде говаривали: `кто хочет спросить, спроси в Авеле'; и так решали дело.
\par 19 Я из мирных, верных [городов] Израиля; а ты хочешь уничтожить город, и [притом] мать [городов] в Израиле; для чего тебе разрушать наследие Господне?
\par 20 И отвечал Иоав и сказал: да не будет этого от меня, чтобы я уничтожил или разрушил!
\par 21 Это не так; но человек с горы Ефремовой, по имени Савей, сын Бихри, поднял руку свою на царя Давида; выдайте мне его одного, и я отступлю от города. И сказала женщина Иоаву: вот, голова его [будет] тебе брошена со стены.
\par 22 И пошла женщина по всему народу со своим умным словом; и отсекли голову Савею, сыну Бихри, и бросили Иоаву. Тогда [Иоав] затрубил трубою, и разошлись от города все [люди] по своим шатрам; Иоав же возвратился в Иерусалим к царю.
\par 23 И был Иоав [поставлен] над всем войском Израильским, а Ванея, сын Иодаев, --над Хелефеями и над Фелефеями;
\par 24 Адорам--над сбором податей; Иосафат, сын Ахилуда--дееписателем;
\par 25 Суса--писцом; Садок и Авиафар--священниками;
\par 26 также и Ира Иаритянин был священником у Давида.

\chapter{21}

\par 1 Был голод на земле во дни Давида три года, год за годом. И вопросил Давид Господа. И сказал Господь: это ради Саула и кровожадного дома его, за то, что он умертвил Гаваонитян.
\par 2 Тогда царь призвал Гаваонитян и говорил с ними. Гаваонитяне были не из сынов Израилевых, но из остатков Аморреев; Израильтяне же дали им клятву, но Саул хотел истребить их по ревности своей о потомках Израиля и Иуды.
\par 3 И сказал Давид Гаваонитянам: что мне сделать для вас, и чем примирить вас, чтобы вы благословили наследие Господне?
\par 4 И сказали ему Гаваонитяне: не нужно нам ни серебра, ни золота от Саула, или от дома его, и не нужно нам, чтоб умертвили кого в Израиле. Он сказал: чего же вы хотите? я сделаю для вас.
\par 5 И сказали они царю: того человека, который губил нас и хотел истребить нас, чтобы не было нас ни в одном из пределов Израилевых, --
\par 6 из его потомков выдай нам семь человек, и мы повесим их пред Господом в Гиве Саула, избранного Господом. И сказал царь: я выдам.
\par 7 Но пощадил царь Мемфивосфея, сына Ионафана, сына Саулова, ради клятвы именем Господним, которая была между ними, между Давидом и Ионафаном, сыном Сауловым.
\par 8 И взял царь двух сыновей Рицпы, дочери Айя, которая родила Саулу Армона и Мемфивосфея, и пять сыновей Мелхолы, дочери Сауловой, которых она родила Адриэлу, сыну Верзеллия из Мехолы,
\par 9 и отдал их в руки Гаваонитян, и они повесили их на горе пред Господом. И погибли все семь вместе; они умерщвлены в первые дни жатвы, в начале жатвы ячменя.
\par 10 Тогда Рицпа, дочь Айя, взяла вретище и разостлала его себе на той горе [и сидела] от начала жатвы до того времени, пока не полились на них воды Божии с неба, и не допускала касаться их птицам небесным днем и зверям полевым ночью.
\par 11 И донесли Давиду, что сделала Рицпа, дочь Айя, наложница Саула.
\par 12 И пошел Давид и взял кости Саула и кости Ионафана, сына его, у жителей Иависа Галаадского, которые тайно взяли их с площади Беф-Сана, где они были повешены Филистимлянами, когда убили Филистимляне Саула на Гелвуе.
\par 13 И перенес он оттуда кости Саула и кости Ионафана, сына его; и собрали кости повешенных.
\par 14 И похоронили кости Саула и Ионафана, сына его, в земле Вениаминовой, в Цела, во гробе Киса, отца его. И сделали все, что повелел царь, и умилостивился Бог над страною после того.
\par 15 И открылась снова война между Филистимлянами и Израильтянами. И вышел Давид и слуги его с ним, и воевали с Филистимлянами; и Давид утомился.
\par 16 Тогда Иесвий, один из потомков Рефаимов, у которого копье было весом в триста сиклей меди и который опоясан был новым мечом, хотел поразить Давида.
\par 17 Но ему помог Авесса, сын Саруин, и поразил Филистимлянина и умертвил его. Тогда люди Давидовы поклялись, говоря: не выйдешь ты больше с нами на войну, чтобы не угас светильник Израиля.
\par 18 Потом была снова война с Филистимлянами в Гобе; тогда Совохай Хушатянин убил Сафута, одного из потомков Рефаимов.
\par 19 Было и другое сражение в Гобе; тогда убил Елханан, сын Ягаре-Оргима Вифлеемского, Голиафа Гефянина, у которого древко копья было, как навой у ткачей.
\par 20 Было еще сражение в Гефе; и был [там] один человек рослый, имевший по шести пальцев на руках и на ногах, всего двадцать четыре, также из потомков Рефаимов,
\par 21 и он поносил Израильтян; но его убил Ионафан, сын Сафая, брата Давидова.
\par 22 Эти четыре были из рода Рефаимов в Гефе, и они пали от руки Давида и слуг его.

\chapter{22}

\par 1 И воспел Давид песнь Господу в день, когда Господь избавил его от руки всех врагов его и от руки Саула, и сказал:
\par 2 Господь--твердыня моя и крепость моя и избавитель мой.
\par 3 Бог мой--скала моя; на Него я уповаю; щит мой, рог спасения моего, ограждение мое и убежище мое; Спаситель мой, от бед Ты избавил меня!
\par 4 Призову Господа достопоклоняемого и от врагов моих спасусь.
\par 5 Объяли меня волны смерти, и потоки беззакония устрашили меня;
\par 6 цепи ада облегли меня, и сети смерти опутали меня.
\par 7 Но в тесноте моей я призвал Господа и к Богу моему воззвал, и Он услышал из чертога Своего голос мой, и вопль мой [дошел] до слуха Его.
\par 8 Потряслась, всколебалась земля, дрогнули и подвиглись основания небес, ибо разгневался [на них Господь].
\par 9 Поднялся дым от гнева Его и из уст Его огонь поядающий; горящие угли сыпались от Него.
\par 10 Наклонил Он небеса и сошел; и мрак под ногами Его;
\par 11 и воссел на Херувимов, и полетел, и понесся на крыльях ветра;
\par 12 и мраком покрыл Себя, как сению, сгустив воды облаков небесных;
\par 13 от блистания пред Ним разгорались угли огненные.
\par 14 Возгремел с небес Господь, и Всевышний дал глас Свой;
\par 15 пустил стрелы и рассеял их; [блеснул] молниею и истребил их.
\par 16 И открылись источники моря, обнажились основания вселенной от грозного гласа Господа, от дуновения духа гнева Его.
\par 17 Простер Он [руку] с высоты и взял меня, и извлек меня из вод многих;
\par 18 избавил меня от врага моего сильного, от ненавидящих меня, которые были сильнее меня.
\par 19 Они восстали на меня в день бедствия моего; но Господь был опорою для меня
\par 20 и вывел меня на пространное место, избавил меня, ибо Он благоволит ко мне.
\par 21 Воздал мне Господь по правде моей, по чистоте рук моих вознаградил меня.
\par 22 Ибо я хранил пути Господа и не был нечестивым пред Богом моим,
\par 23 ибо все заповеди Его предо мною, и от уставов Его я не отступал,
\par 24 и был непорочен пред Ним, и остерегался, чтобы не согрешить мне.
\par 25 И воздал мне Господь по правде моей, по чистоте моей пред очами Его.
\par 26 С милостивым Ты поступаешь милостиво, с мужем искренним--искренно,
\par 27 с чистым--чисто, а с лукавым--по лукавству его.
\par 28 Людей угнетенных Ты спасаешь и взором Своим унижаешь надменных.
\par 29 Ты, Господи, светильник мой; Господь просвещает тьму мою.
\par 30 С Тобою я поражаю войско; с Богом моим восхожу на стену.
\par 31 Бог! --непорочен путь Его, чисто слово Господа, щит Он для всех, надеющихся на Него.
\par 32 Ибо кто Бог, кроме Господа, и кто защита, кроме Бога нашего?
\par 33 Бог препоясует меня силою, устрояет мне верный путь;
\par 34 делает ноги мои, как оленьи, и на высотах поставляет меня;
\par 35 научает руки мои брани и мышцы мои напрягает, как медный лук.
\par 36 Ты даешь мне щит спасения Твоего, и милость Твоя возвеличивает меня.
\par 37 Ты расширяешь шаг мой подо мною, и не колеблются ноги мои.
\par 38 Я гоняюсь за врагами моими и истребляю их, и не возвращаюсь, доколе не уничтожу их;
\par 39 и истребляю их и поражаю их, и не встают и падают под ноги мои.
\par 40 Ты препоясываешь меня силою для войны и низлагаешь предо мною восстающих на меня;
\par 41 Ты обращаешь ко мне тыл врагов моих, и я истребляю ненавидящих меня.
\par 42 Они взывают, но нет спасающего, --ко Господу, но Он не внемлет им.
\par 43 Я рассеваю их, как прах земной, как грязь уличную мну их и топчу их.
\par 44 Ты избавил меня от мятежа народа моего; Ты сохранил меня, чтоб быть мне главою над иноплеменниками; народ, которого я не знал, служит мне.
\par 45 Иноплеменники ласкательствуют предо мною; по слуху [обо мне] повинуются мне.
\par 46 Иноплеменники бледнеют и трепещут в укреплениях своих.
\par 47 Жив Господь и благословен защитник мой! Да будет превознесен Бог, убежище спасения моего,
\par 48 Бог, мстящий за меня и покоряющий мне народы
\par 49 и избавляющий меня от врагов моих! Над восстающими против меня Ты возвысил меня; от человека жестокого Ты избавил меня.
\par 50 За то я буду славить Тебя, Господи, между иноплеменниками и буду петь имени Твоему,
\par 51 величественно спасающий царя Своего и творящий милость помазаннику Своему Давиду и потомству его во веки!

\chapter{23}

\par 1 Вот последние слова Давида, изречение Давида, сына Иессеева, изречение мужа, поставленного высоко, помазанника Бога Иаковлева и сладкого певца Израилева:
\par 2 Дух Господень говорит во мне, и слово Его на языке у меня.
\par 3 Сказал Бог Израилев, говорил о мне скала Израилева: владычествующий над людьми будет праведен, владычествуя в страхе Божием.
\par 4 И как на рассвете утра, при восходе солнца на безоблачном небе, от сияния после дождя вырастает трава из земли,
\par 5 не так ли дом мой у Бога? Ибо завет вечный положил Он со мною, твердый и непреложный. Не так ли исходит от Него все спасение мое и все хотение мое?
\par 6 А нечестивые будут, как выброшенное терние, которого не берут рукою;
\par 7 но кто касается его, вооружается железом или деревом копья, и огнем сожигают его на месте.
\par 8 Вот имена храбрых у Давида: Исбосеф Ахаманитянин, главный из трех; он поднял копье свое на восемьсот человек и поразил их в один раз.
\par 9 По нем Елеазар, сын Додо, сына Ахохи, из трех храбрых, бывших с Давидом, когда они порицанием вызывали Филистимлян, собравшихся на войну;
\par 10 израильтяне вышли против них, и он стал и поражал Филистимлян до того, что рука его утомилась и прилипла к мечу. И даровал Господь в тот день великую победу, и народ последовал за ним для того только, чтоб обирать [убитых].
\par 11 За ним Шамма, сын Аге, Гараритянин. Когда Филистимляне собрались в Фирию, где было поле, засеянное чечевицею, и народ побежал от Филистимлян,
\par 12 то он стал среди поля и сберег его и поразил Филистимлян. И даровал тогда Господь великую победу.
\par 13 Трое сих главных из тридцати вождей пошли и вошли во время жатвы к Давиду в пещеру Одоллам, когда толпы Филистимлян стояли в долине Рефаимов.
\par 14 Давид был тогда в укрепленном месте, а отряд Филистимлян--в Вифлееме.
\par 15 И захотел Давид пить, и сказал: кто напоит меня водою из колодезя Вифлеемского, что у ворот?
\par 16 Тогда трое этих храбрых пробились сквозь стан Филистимский и почерпнули воды из колодезя Вифлеемского, что у ворот, и взяли и принесли Давиду. Но он не захотел пить ее и вылил ее во славу Господа,
\par 17 и сказал: сохрани меня Господь, чтоб я сделал это! не кровь ли это людей, ходивших с опасностью собственной жизни? И не захотел пить ее. Вот что сделали эти трое храбрых!
\par 18 И Авесса, брат Иоава, сын Саруин, был главным из трех; он убил копьем своим триста человек и был в славе у тех троих.
\par 19 Из трех он был знатнейшим и был начальником, но с теми тремя не равнялся.
\par 20 Ванея, сын Иодая, мужа храброго, великий по делам, из Кавцеила; он поразил двух сыновей Ариила Моавитского; он же сошел и убил льва во рве в снежное время;
\par 21 он же убил одного Египтянина человека видного; в руке Египтянина было копье, а он пошел к нему с палкою и отнял копье из руки Египтянина, и убил его собственным его копьем:
\par 22 вот что сделал Ванея, сын Иодаев, и он был в славе у трех храбрых;
\par 23 он был знатнее тридцати, но с теми тремя не равнялся. И поставил его Давид ближайшим исполнителем своих приказаний.
\par 24 Асаил, брат Иоава--в числе тридцати; Елханан, сын Додо, из Вифлеема,
\par 25 Шамма Хародитянин, Елика Хародитянин,
\par 26 Херец Палтитянин, Ира, сын Икеша, Фекоитянин,
\par 27 Евиезер Анафофянин, Мебуннай Хушатянин,
\par 28 Цалмон Ахохитянин, Магарай Нетофафянин,
\par 29 Хелев, сын Бааны, Нетофафянин, Иттай, сын Рибая, из Гивы сынов Вениаминовых,
\par 30 Ванея Пирафонянин, Иддай из Нахле-Гааша,
\par 31 Ави-Албон Арбатитянин, Азмавет Бархюмитянин,
\par 32 Елияхба Шаалбонянин; из сыновей Яшена--Ионафан,
\par 33 Шама Гараритянин, Ахиам, сын Шарара, Араритянин,
\par 34 Елифелет, сын Ахасбая, сына Магахати, Елиам, сын Ахитофела, Гилонянин,
\par 35 Хецрай Кармилитянин, Паарай Арбитянин,
\par 36 Игал, сын Нафана, из Цобы, Бани Гадитянин,
\par 37 Целек Аммонитянин, Нахарай Беротянин, оруженосец Иоава, сына Саруи,
\par 38 Ира Итритянин, Гареб Итритянин,
\par 39 Урия Хеттеянин. Всех тридцать семь.

\chapter{24}

\par 1 Гнев Господень опять возгорелся на Израильтян, и возбудил он в них Давида сказать: пойди, исчисли Израиля и Иуду.
\par 2 И сказал царь Иоаву военачальнику, который был при нем: пройди по всем коленам Израилевым от Дана до Вирсавии, и исчислите народ, чтобы мне знать число народа.
\par 3 И сказал Иоав царю: Господь Бог твой да умножит столько народа, сколько есть, и еще во сто раз столько, а очи господина моего царя да увидят [это]; но для чего господин мой царь желает этого дела?
\par 4 Но слово царя Иоаву и военачальникам превозмогло; и пошел Иоав с военачальниками от царя считать народ Израильский.
\par 5 И перешли они Иордан и остановились в Ароере, на правой стороне города, который среди долины Гадовой, к Иазеру;
\par 6 и пришли в Галаад и в землю Тахтим-Ходши; и пришли в Дан--Яан и обошли Сидон;
\par 7 и пришли к укреплению Тира и во все города Хивеян и Хананеян и вышли на юг Иудеи в Вирсавию;
\par 8 и обошли всю землю и пришли чрез девять месяцев и двадцать дней в Иерусалим.
\par 9 И подал Иоав список народной переписи царю; и оказалось, что Израильтян было восемьсот тысяч мужей сильных, способных к войне, а Иудеян пятьсот тысяч.
\par 10 И вздрогнуло сердце Давидово после того, как он сосчитал народ. И сказал Давид Господу: тяжко согрешил я, поступив так; и ныне молю Тебя, Господи, прости грех раба Твоего, ибо крайне неразумно поступил я.
\par 11 Когда Давид встал на другой день утром, то было слово Господа к Гаду пророку, прозорливцу Давида:
\par 12 пойди и скажи Давиду: так говорит Господь: три [наказания] предлагаю Я тебе; выбери себе одно из них, которое совершилось бы над тобою.
\par 13 И пришел Гад к Давиду, и возвестил ему, и сказал ему: избирай себе, быть ли голоду в стране твоей семь лет, или чтобы ты три месяца бегал от неприятелей твоих, и они преследовали тебя, или чтобы в продолжение трех дней была моровая язва в стране твоей? теперь рассуди и реши, что мне отвечать Пославшему меня.
\par 14 И сказал Давид Гаду: тяжело мне очень; но пусть впаду я в руки Господа, ибо велико милосердие Его; только бы в руки человеческие не впасть мне.
\par 15 И послал Господь язву на Израильтян от утра до назначенного времени; и умерло из народа, от Дана до Вирсавии, семьдесят тысяч человек.
\par 16 И простер Ангел руку свою на Иерусалим, чтобы опустошить его; но Господь пожалел о бедствии и сказал Ангелу, поражавшему народ: довольно, теперь опусти руку твою. Ангел же Господень был тогда у гумна Орны Иевусеянина.
\par 17 И сказал Давид Господу, когда увидел Ангела, поражавшего народ, говоря: вот, я согрешил, я поступил беззаконно; а эти овцы, что сделали они? пусть же рука Твоя обратится на меня и на дом отца моего.
\par 18 И пришел в тот день Гад к Давиду и сказал: иди, поставь жертвенник Господу на гумне Орны Иевусеянина.
\par 19 И пошел Давид по слову Гада, как повелел Господь.
\par 20 И взглянул Орна и увидел царя и слуг его, шедших к нему, и вышел Орна и поклонился царю лицем своим до земли.
\par 21 И сказал Орна: зачем пришел господин мой царь к рабу своему? И сказал Давид: купить у тебя гумно для устроения жертвенника Господу, чтобы прекратилось поражение народа.
\par 22 И сказал Орна Давиду: пусть возьмет и вознесет [в жертву] господин мой, царь, что ему угодно. Вот волы для всесожжения и повозки и упряжь воловья на дрова.
\par 23 Все это, царь, Орна отдает царю. Еще сказал Орна царю: Господь Бог твой да будет милостив к тебе!
\par 24 Но царь сказал Орне: нет, я заплачу тебе, что стоит, и не вознесу Господу Богу моему жертвы, [взятой] даром. И купил Давид гумно и волов за пятьдесят сиклей серебра.
\par 25 И соорудил там Давид жертвенник Господу и принес всесожжения и мирные жертвы. И умилостивился Господь над страною, и прекратилось поражение Израильтян.


\end{document}