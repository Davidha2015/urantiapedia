\begin{document}

\title{4-я Царств}


\chapter{1}

\par 1 И отложился Моав от Израиля по смерти Ахава.
\par 2 Охозия же упал чрез решетку с горницы своей, что в Самарии, и занемог. И послал послов, и сказал им: пойдите, спросите у Веельзевула, божества Аккаронского: выздоровею ли я от сей болезни?
\par 3 Тогда Ангел Господень сказал Илии Фесвитянину: встань, пойди навстречу посланным от царя Самарийского и скажи им: разве нет Бога в Израиле, что вы идете вопрошать Веельзевула, божество Аккаронское?
\par 4 За это так говорит Господь: с постели, на которую ты лег, не сойдешь с нее, но умрешь. И пошел Илия.
\par 5 И возвратились к [Охозии] посланные. И он сказал им: что это вы возвратились?
\par 6 И сказали ему: навстречу нам вышел человек и сказал нам: пойдите, возвратитесь к царю, который послал вас, и скажите ему: так говорит Господь: разве нет Бога в Израиле, что ты посылаешь вопрошать Веельзевула, божество Аккаронское? За то с постели, на которую ты лег, не сойдешь с нее, но умрешь.
\par 7 И сказал им: каков видом тот человек, который вышел навстречу вам и говорил вам слова сии?
\par 8 Они сказали ему: человек тот весь в волосах и кожаным поясом подпоясан по чреслам своим. И сказал он: это Илия Фесвитянин.
\par 9 И послал к нему пятидесятника с его пятидесятком. И он взошел к нему, когда Илия сидел на верху горы, и сказал ему: человек Божий! царь говорит: сойди.
\par 10 И отвечал Илия, и сказал пятидесятнику: если я человек Божий, то пусть сойдет огонь с неба и попалит тебя и твой пятидесяток. И сошел огонь с неба и попалил его и пятидесяток его.
\par 11 И послал к нему царь другого пятидесятника с его пятидесятком. И он стал говорить ему: человек Божий! так сказал царь: сойди скорее.
\par 12 И отвечал Илия и сказал ему: если я человек Божий, то пусть сойдет огонь с неба и попалит тебя и твой пятидесяток. И сошел огонь Божий с неба, и попалил его и пятидесяток его.
\par 13 И еще послал в третий раз пятидесятника с его пятидесятком. И поднялся, и пришел пятидесятник третий, и пал на колена свои пред Илиею, и умолял его, и говорил ему: человек Божий! да не будет презрена душа моя и душа рабов твоих--сих пятидесяти--пред очами твоими;
\par 14 вот, сошел огонь с неба, и попалил двух пятидесятников прежних с их пятидесятками; но теперь да не будет презрена душа моя пред очами твоими!
\par 15 И сказал Ангел Господень Илии: пойди с ним, не бойся его. И он встал, и пошел с ним к царю.
\par 16 И сказал ему: так говорит Господь: за то, что ты посылал послов вопрошать Веельзевула, божество Аккаронское, как будто в Израиле нет Бога, чтобы вопрошать о слове Его, --с постели, на которую ты лег, не сойдешь с нее, но умрешь.
\par 17 И умер он по слову Господню, которое изрек Илия. И воцарился Иорам вместо него, во второй год Иорама, сына Иосафатова, царя Иудейского, так как сына у того не было.
\par 18 Прочее об Охозии, что он сделал, написано в летописи царей Израильских.

\chapter{2}

\par 1 В то время, как Господь восхотел вознести Илию в вихре на небо, шел Илия с Елисеем из Галгала.
\par 2 И сказал Илия Елисею: останься здесь, ибо Господь посылает меня в Вефиль. Но Елисей сказал: жив Господь и жива душа твоя! не оставлю тебя. И пошли они в Вефиль.
\par 3 И вышли сыны пророков, которые в Вефиле, к Елисею и сказали ему: знаешь ли, что сегодня Господь вознесет господина твоего над главою твоею? Он сказал: я также знаю, молчите.
\par 4 И сказал ему Илия: Елисей, останься здесь, ибо Господь посылает меня в Иерихон. И сказал он: жив Господь и жива душа твоя! не оставлю тебя. И пришли в Иерихон.
\par 5 И подошли сыны пророков, которые в Иерихоне, к Елисею и сказали ему: знаешь ли, что сегодня Господь берет господина твоего и вознесет над главою твоею? Он сказал: я также знаю, молчите.
\par 6 И сказал ему Илия: останься здесь, ибо Господь посылает меня к Иордану. И сказал он: жив Господь и жива душа твоя! не оставлю тебя. И пошли оба.
\par 7 Пятьдесят человек из сынов пророческих пошли и стали вдали напротив их, а они оба стояли у Иордана.
\par 8 И взял Илия милоть свою, и свернул, и ударил ею по воде, и расступилась она туда и сюда, и перешли оба посуху.
\par 9 Когда они перешли, Илия сказал Елисею: проси, что сделать тебе, прежде нежели я буду взят от тебя. И сказал Елисей: дух, который в тебе, пусть будет на мне вдвойне.
\par 10 И сказал он: трудного ты просишь. Если увидишь, как я буду взят от тебя, то будет тебе так, а если не увидишь, не будет.
\par 11 Когда они шли и дорогою разговаривали, вдруг явилась колесница огненная и кони огненные, и разлучили их обоих, и понесся Илия в вихре на небо.
\par 12 Елисей же смотрел и воскликнул: отец мой, отец мой, колесница Израиля и конница его! И не видел его более. И схватил он одежды свои и разодрал их на две части.
\par 13 И поднял милоть Илии, упавшую с него, и пошел назад, и стал на берегу Иордана;
\par 14 и взял милоть Илии, упавшую с него, и ударил ею по воде, и сказал: где Господь, Бог Илии, --Он Самый? И ударил по воде, и она расступилась туда и сюда, и перешел Елисей.
\par 15 И увидели его сыны пророков, которые в Иерихоне, издали, и сказали: опочил дух Илии на Елисее. И пошли навстречу ему, и поклонились ему до земли,
\par 16 и сказали ему: вот, есть [у нас], рабов твоих, человек пятьдесят, люди сильные; пусть бы они пошли и поискали господина твоего; может быть, унес его Дух Господень и поверг его на одной из гор, или на одной из долин. Он же сказал: не посылайте.
\par 17 Но они приступали к нему долго, так что наскучили ему, и он сказал: пошлите. И послали пятьдесят человек, и искали три дня, и не нашли его,
\par 18 и возвратились к нему, между тем как он оставался в Иерихоне, и сказал им: не говорил ли я вам: не ходите?
\par 19 И сказали жители того города Елисею: вот, положение этого города хорошо, как видит господин мой; но вода нехороша и земля бесплодна.
\par 20 И сказал он: дайте мне новую чашу и положите туда соли. И дали ему.
\par 21 И вышел он к истоку воды, и бросил туда соли, и сказал: так говорит Господь: Я сделал воду сию здоровою, не будет от нее впредь ни смерти, ни бесплодия.
\par 22 И вода стала здоровою до сего дня, по слову Елисея, которое он сказал.
\par 23 И пошел он оттуда в Вефиль. Когда он шел дорогою, малые дети вышли из города и насмехались над ним и говорили ему: иди, плешивый! иди, плешивый!
\par 24 Он оглянулся и увидел их и проклял их именем Господним. И вышли две медведицы из леса и растерзали из них сорок два ребенка.
\par 25 Отсюда пошел он на гору Кармил, а оттуда возвратился в Самарию.

\chapter{3}

\par 1 Иорам, сын Ахава, воцарился над Израилем в Самарии в восемнадцатый год Иосафата, царя Иудейского, и царствовал двенадцать лет,
\par 2 и делал неугодное в очах Господних, хотя не так, как отец его и мать его: он снял статую Ваала, которую сделал отец его;
\par 3 однако же грехов Иеровоама, сына Наватова, который ввел в грех Израиля, он держался, не отставал от них.
\par 4 Меса, царь Моавитский, был богат скотом и присылал царю Израильскому по сто тысяч овец и по сто тысяч неостриженных баранов.
\par 5 Но когда умер Ахав, царь Моавитский отложился от царя Израильского.
\par 6 И выступил царь Иорам в то время из Самарии и сделал смотр всем Израильтянам;
\par 7 и пошел и послал к Иосафату, царю Иудейскому, сказать: царь Моавитский отложился от меня, пойдешь ли со мной на войну против Моава? Он сказал: пойду; как ты, так и я, как твой народ, так и мой народ; как твои кони, так и мои кони.
\par 8 И сказал: какою дорогою идти нам? Он сказал: дорогою пустыни Едомской.
\par 9 И пошел царь Израильский, и царь Иудейский, и царь Едомский, и шли они обходом семь дней, и не было воды для войска и для скота, который [шел] за ними.
\par 10 И сказал царь Израильский: ах! созвал Господь трех царей сих, чтобы предать их в руку Моава.
\par 11 И сказал Иосафат: нет ли здесь пророка Господня, чтобы нам вопросить Господа чрез него? И отвечал один из слуг царя Израильского и сказал: здесь Елисей, сын Сафатов, который подавал воду на руки Илии.
\par 12 И сказал Иосафат: есть у него слово Господне. И пошли к нему царь Израильский, и Иосафат, и царь Едомский.
\par 13 И сказал Елисей царю Израильскому: что мне и тебе? пойди к пророкам отца твоего и к пророкам матери твоей. И сказал ему царь Израильский: нет, потому что Господь созвал сюда трех царей сих, чтобы предать их в руку Моава.
\par 14 И сказал Елисей: жив Господь Саваоф, пред Которым я стою! Если бы я не почитал Иосафата, царя Иудейского, то не взглянул бы на тебя и не видел бы тебя;
\par 15 теперь позовите мне гуслиста. И когда гуслист играл на гуслях, тогда рука Господня коснулась Елисея,
\par 16 и он сказал: так говорит Господь: делайте на сей долине рвы за рвами,
\par 17 ибо так говорит Господь: не увидите ветра и не увидите дождя, а долина сия наполнится водою, которую будете пить вы и мелкий и крупный скот ваш;
\par 18 но этого мало пред очами Господа; Он и Моава предаст в руки ваши,
\par 19 и вы поразите все города укрепленные и все города главные, и все лучшие деревья срубите, и все источники водные запрудите, и все лучшие участки полевые испортите каменьями.
\par 20 Поутру, когда возносят хлебное приношение, вдруг полилась вода по пути от Едома, и наполнилась земля водою.
\par 21 Когда Моавитяне услышали, что идут цари воевать с ними, тогда собраны были все, начиная от носящего пояс и старше, и стали на границе.
\par 22 Поутру встали они рано, и когда солнце воссияло над водою, Моавитянам издали показалась эта вода красною, как кровь.
\par 23 И сказали они: это кровь; сразились цари между собою и истребили друг друга; теперь на добычу, Моав!
\par 24 И пришли они к стану Израильскому. И встали Израильтяне и стали бить Моавитян, и те побежали от них, а они продолжали идти на них и бить Моавитян.
\par 25 И города разрушили, и на всякий лучший участок в поле бросили каждый по камню и закидали его; и все протоки вод запрудили и все дерева лучшие срубили, так что оставались только каменья в Кир-Харешете. И обступили его пращники и разрушили его.
\par 26 И увидел царь Моавитский, что битва одолевает его, и взял с собою семьсот человек, владеющих мечом, чтобы пробиться к царю Едомскому; но не могли.
\par 27 И взял он сына своего первенца, которому следовало царствовать вместо него, и вознес его во всесожжение на стене. Это произвело большое негодование в Израильтянах, и они отступили от него и возвратились в свою землю.

\chapter{4}

\par 1 Одна из жен сынов пророческих с воплем говорила Елисею: раб твой, мой муж, умер; а ты знаешь, что раб твой боялся Господа; теперь пришел заимодавец взять обоих детей моих в рабы себе.
\par 2 И сказал ей Елисей: что мне сделать тебе? скажи мне, что есть у тебя в доме? Она сказала: нет у рабы твоей ничего в доме, кроме сосуда с елеем.
\par 3 И сказал он: пойди, попроси себе сосудов на стороне, у всех соседей твоих, сосудов порожних; набери немало,
\par 4 и пойди, запри дверь за собою и за сыновьями твоими, и наливай во все эти сосуды; полные отставляй.
\par 5 И пошла от него и заперла дверь за собой и за сыновьями своими. Они подавали ей, а она наливала.
\par 6 Когда наполнены были сосуды, она сказала сыну своему: подай мне еще сосуд. Он сказал ей: нет более сосудов. И остановилось масло.
\par 7 И пришла она, и пересказала человеку Божию. Он сказал: пойди, продай масло и заплати долги твои; а что останется, тем будешь жить с сыновьями твоими.
\par 8 В один день пришел Елисей в Сонам. Там одна богатая женщина упросила его [к себе] есть хлеба; и когда он ни проходил, всегда заходил туда есть хлеба.
\par 9 И сказала она мужу своему: вот, я знаю, что человек Божий, который проходит мимо нас постоянно, святой;
\par 10 сделаем небольшую горницу над стеною и поставим ему там постель, и стол, и седалище, и светильник; и когда он будет приходить к нам, пусть заходит туда.
\par 11 В один день он пришел туда, и зашел в горницу, и лег там,
\par 12 и сказал Гиезию, слуге своему: позови эту Сонамитянку. И позвал ее, и она стала пред ним.
\par 13 И сказал ему: скажи ей: `вот, ты так заботишься о нас; что сделать бы тебе? не нужно ли поговорить о тебе с царем, или с военачальником?' Она сказала: нет, среди своего народа я живу.
\par 14 И сказал он: что же сделать ей? И сказал Гиезий: да вот, сына нет у нее, а муж ее стар.
\par 15 И сказал он: позови ее. Он позвал ее, и стала она в дверях.
\par 16 И сказал он: через год, в это самое время ты будешь держать на руках сына. И сказала она: нет, господин мой, человек Божий, не обманывай рабы твоей.
\par 17 И женщина стала беременною и родила сына на другой год, в то самое время, как сказал ей Елисей.
\par 18 И подрос ребенок и в один день пошел к отцу своему, к жнецам.
\par 19 И сказал отцу своему: голова моя! голова моя болит! И сказал тот слуге своему: отнеси его к матери его.
\par 20 И понес его и принес его к матери его. И он сидел на коленях у нее до полудня, и умер.
\par 21 И пошла она, и положила его на постели человека Божия, и заперла его, и вышла,
\par 22 и позвала мужа своего и сказала: пришли мне одного из слуг и одну из ослиц, я поеду к человеку Божию и возвращусь.
\par 23 Он сказал: зачем тебе ехать к нему? сегодня не новомесячие и не суббота. Но она сказала: хорошо.
\par 24 И оседлала ослицу и сказала слуге своему: веди и иди; не останавливайся, доколе не скажу тебе.
\par 25 И отправилась и прибыла к человеку Божию, к горе Кармил. И когда увидел человек Божий ее издали, то сказал слуге своему Гиезию: это та Сонамитянка.
\par 26 Побеги к ней навстречу и скажи ей: `здорова ли ты? здоров ли муж твой? здоров ли ребенок?' --Она сказала: здоровы.
\par 27 Когда же пришла к человеку Божию на гору, ухватилась за ноги его. И подошел Гиезий, чтобы отвести ее; но человек Божий сказал: оставь ее, душа у нее огорчена, а Господь скрыл от меня и не объявил мне.
\par 28 И сказала она: просила ли я сына у господина моего? не говорила ли я: `не обманывай меня'?
\par 29 И сказал он Гиезию: опояшь чресла твои и возьми жезл мой в руку твою, и пойди; если встретишь кого, не приветствуй его, и если кто будет тебя приветствовать, не отвечай ему; и положи посох мой на лице ребенка.
\par 30 И сказала мать ребенка: жив Господь и жива душа твоя! не отстану от тебя. И он встал и пошел за нею.
\par 31 Гиезий пошел впереди их и положил жезл на лице ребенка. Но не было ни голоса, ни ответа. И вышел навстречу ему, и донес ему, и сказал: не пробуждается ребенок.
\par 32 И вошел Елисей в дом, и вот, ребенок умерший лежит на постели его.
\par 33 И вошел, и запер дверь за собою, и помолился Господу.
\par 34 И поднялся и лег над ребенком, и приложил свои уста к его устам, и свои глаза к его глазам, и свои ладони к его ладоням, и простерся на нем, и согрелось тело ребенка.
\par 35 И встал и прошел по горнице взад и вперед; потом опять поднялся и простерся на нем. И чихнул ребенок раз семь, и открыл ребенок глаза свои.
\par 36 И позвал он Гиезия и сказал: позови эту Сонамитянку. И тот позвал ее. Она пришла к нему, и он сказал: возьми сына твоего.
\par 37 И подошла, и упала ему в ноги, и поклонилась до земли; и взяла сына своего и пошла.
\par 38 Елисей же возвратился в Галгал. И был голод в земле той, и сыны пророков сидели пред ним. И сказал он слуге своему: поставь большой котел и свари похлебку для сынов пророческих.
\par 39 И вышел один из них в поле собирать овощи, и нашел дикое вьющееся растение, и набрал с него диких плодов полную одежду свою; и пришел и накрошил их в котел с похлебкою, так как они не знали [их].
\par 40 И налили им есть. Но как скоро они стали есть похлебку, то подняли крик и говорили: смерть в котле, человек Божий! И не могли есть.
\par 41 И сказал он: подайте муки. И всыпал ее в котел и сказал [Гиезию]: наливай людям, пусть едят. И не стало ничего вредного в котле.
\par 42 Пришел некто из Ваал-Шалиши, и принес человеку Божию хлебный начаток--двадцать ячменных хлебцев и сырые зерна в шелухе. И сказал Елисей: отдай людям, пусть едят.
\par 43 И сказал слуга его: что тут я дам ста человекам? И сказал он: отдай людям, пусть едят, ибо так говорит Господь: `насытятся, и останется'.
\par 44 Он подал им, и они насытились, и еще осталось, по слову Господню.

\chapter{5}

\par 1 Нееман, военачальник царя Сирийского, был великий человек у господина своего и уважаемый, потому что чрез него дал Господь победу Сириянам; и человек сей был отличный воин, но прокаженный.
\par 2 Сирияне [однажды] пошли отрядами и взяли в плен из земли Израильской маленькую девочку, и она служила жене Неемановой.
\par 3 И сказала она госпоже своей: о, если бы господин мой побывал у пророка, который в Самарии, то он снял бы с него проказу его!
\par 4 И пошел [Нееман] и передал это господину своему, говоря: так и так говорит девочка, которая из земли Израильской.
\par 5 И сказал царь Сирийский [Нееману]: пойди, сходи, а я пошлю письмо к царю Израильскому. Он пошел и взял с собою десять талантов серебра и шесть тысяч [сиклей] золота, и десять перемен одежд;
\par 6 и принес письмо царю Израильскому, в котором было сказано: вместе с письмом сим, вот, я посылаю к тебе Неемана, слугу моего, чтобы ты снял с него проказу его.
\par 7 Царь Израильский, прочитав письмо, разодрал одежды свои и сказал: разве я Бог, чтобы умерщвлять и оживлять, что он посылает ко мне, чтобы я снял с человека проказу его? вот, теперь знайте и смотрите, что он ищет предлога враждовать против меня.
\par 8 Когда услышал Елисей, человек Божий, что царь Израильский разодрал одежды свои, то послал сказать царю: для чего ты разодрал одежды свои? пусть он придет ко мне, и узнает, что есть пророк в Израиле.
\par 9 И прибыл Нееман на конях своих и на колеснице своей, и остановился у входа в дом Елисеев.
\par 10 И выслал к нему Елисей слугу сказать: пойди, омойся семь раз в Иордане, и обновится тело твое у тебя, и будешь чист.
\par 11 И разгневался Нееман, и пошел, и сказал: вот, я думал, что он выйдет, станет и призовет имя Господа Бога своего, и возложит руку свою на то место и снимет проказу;
\par 12 разве Авана и Фарфар, реки Дамасские, не лучше всех вод Израильских? разве я не мог бы омыться в них и очиститься? И оборотился и удалился в гневе.
\par 13 И подошли рабы его и говорили ему, и сказали: отец мой, [если] [бы] что-нибудь важное сказал тебе пророк, то не сделал ли бы ты? а тем более, когда он сказал тебе только: `омойся, и будешь чист'.
\par 14 И пошел он и окунулся в Иордане семь раз, по слову человека Божия, и обновилось тело его, как тело малого ребенка, и очистился.
\par 15 И возвратился к человеку Божию он и все сопровождавшие его, и пришел, и стал пред ним, и сказал: вот, я узнал, что на всей земле нет Бога, как только у Израиля; итак прими дар от раба твоего.
\par 16 И сказал он: жив Господь, пред лицем Которого стою! не приму. И тот принуждал его взять, но он не согласился.
\par 17 И сказал Нееман: если уже не так, то пусть рабу твоему дадут земли, сколько снесут два лошака, потому что не будет впредь раб твой приносить всесожжения и жертвы другим богам, кроме Господа;
\par 18 только вот в чем да простит Господь раба твоего: когда пойдет господин мой в дом Риммона для поклонения там и опрется на руку мою, и поклонюсь я в доме Риммона, то, за мое поклонение в доме Риммона, да простит Господь раба твоего в случае сем.
\par 19 И сказал ему: иди с миром. И он отъехал от него на небольшое пространство земли.
\par 20 И сказал Гиезий, слуга Елисея, человека Божия: вот, господин мой отказался взять из руки Неемана, этого Сириянина, то, что он приносил. Жив Господь! Побегу я за ним, и возьму у него что-- нибудь.
\par 21 И погнался Гиезий за Нееманом. И увидел Нееман бегущего за собою, и сошел с колесницы навстречу ему, и сказал: с миром ли?
\par 22 Он отвечал: с миром; господин мой послал меня сказать: `вот, теперь пришли ко мне с горы Ефремовой два молодых человека из сынов пророческих; дай им талант серебра и две перемены одежд'.
\par 23 И сказал Нееман: возьми, пожалуй, два таланта. И упрашивал его. И завязал он два таланта серебра в два мешка и две перемены одежд и отдал двум слугам своим, и понесли перед ним.
\par 24 Когда он пришел к холму, то взял из рук их и спрятал дома. И отпустил людей, и они ушли.
\par 25 Когда он пришел и явился к господину своему, Елисей сказал ему: откуда, Гиезий? И сказал он: никуда не ходил раб твой.
\par 26 И сказал он ему: разве сердце мое не сопутствовало тебе, когда обратился навстречу тебе человек тот с колесницы своей? время ли брать серебро и брать одежды, или масличные деревья и виноградники, и мелкий или крупный скот, и рабов или рабынь?
\par 27 Пусть же проказа Нееманова пристанет к тебе и к потомству твоему навек. И вышел он от него [белый] от проказы, как снег.

\chapter{6}

\par 1 И сказали сыны пророков Елисею: вот, место, где мы живем при тебе, тесно для нас;
\par 2 пойдем к Иордану и возьмем оттуда каждый по одному бревну и сделаем себе там место для жительства. Он сказал: пойдите.
\par 3 И сказал один: сделай милость, пойди и ты с рабами твоими. И сказал он: пойду.
\par 4 И пошел с ними, и пришли к Иордану и стали рубить деревья.
\par 5 И когда один валил бревно, топор его упал в воду. И закричал он и сказал: ах, господин мой! а он взят был на подержание!
\par 6 И сказал человек Божий: где он упал? Он указал ему место. И отрубил он [кусок] дерева и бросил туда, и всплыл топор.
\par 7 И сказал он: возьми себе. Он протянул руку свою и взял его.
\par 8 Царь Сирийский пошел войною на Израильтян, и советовался со слугами своими, говоря: в таком-то и в таком-то месте я расположу свой стан.
\par 9 И посылал человек Божий к царю Израильскому сказать: берегись проходить сим местом, ибо там Сирияне залегли.
\par 10 И посылал царь Израильский на то место, о котором говорил ему человек Божий и предостерегал его; и сберег себя там не раз и не два.
\par 11 И встревожилось сердце царя Сирийского по сему случаю, и призвал он рабов своих и сказал им: скажите мне, кто из наших [в сношении] с царем Израильским?
\par 12 И сказал один из слуг его: никто, господин мой царь; а Елисей пророк, который у Израиля, пересказывает царю Израильскому и те слова, которые ты говоришь в спальной комнате твоей.
\par 13 И сказал он: пойдите, узнайте, где он; я пошлю и возьму его. И донесли ему и сказали: вот, он в Дофаиме.
\par 14 И послал туда коней и колесницы и много войска. И пришли ночью и окружили город.
\par 15 Поутру служитель человека Божия встал и вышел; и вот, войско вокруг города, и кони и колесницы. И сказал ему слуга его: увы! господин мой, что нам делать?
\par 16 И сказал он: не бойся, потому что тех, которые с нами, больше, нежели тех, которые с ними.
\par 17 И молился Елисей, и говорил: Господи! открой ему глаза, чтоб он увидел. И открыл Господь глаза слуге, и он увидел, и вот, вся гора наполнена конями и колесницами огненными кругом Елисея.
\par 18 Когда пошли к нему Сирияне, Елисей помолился Господу и сказал: порази их слепотою. И Он поразил их слепотою по слову Елисея.
\par 19 И сказал им Елисей: это не та дорога и не тот город; идите за мною, и я провожу вас к тому человеку, которого вы ищете. И привел их в Самарию.
\par 20 Когда они пришли в Самарию, Елисей сказал: Господи! открой глаза им, чтобы они видели. И открыл Господь глаза их, и увидели, что они в средине Самарии.
\par 21 И сказал царь Израильский Елисею, увидев их: не избить ли их, отец мой?
\par 22 И сказал он: не убивай. Разве мечом твоим и луком твоим ты пленил их, чтобы убивать их? Предложи им хлеба и воды; пусть едят и пьют, и пойдут к государю своему.
\par 23 И приготовил им большой обед, и они ели и пили. И отпустил их, и пошли к государю своему. И не ходили более те полчища Сирийские в землю Израилеву.
\par 24 После того собрал Венадад, царь Сирийский, все войско свое и выступил, и осадил Самарию.
\par 25 И был большой голод в Самарии, когда они осадили ее, так что ослиная голова продавалась по восьмидесяти сиклей серебра, и четвертая часть каба голубиного помета--по пяти сиклей серебра.
\par 26 Однажды царь Израильский проходил по стене, и женщина с воплем говорила ему: помоги, господин мой царь.
\par 27 И сказал он: если не поможет тебе Господь, из чего я помогу тебе? с гумна ли, с точила ли?
\par 28 И сказал ей царь: что тебе? И сказала она: эта женщина говорила мне: `отдай своего сына, съедим его сегодня, а сына моего съедим завтра'.
\par 29 И сварили мы моего сына, и съели его. И я сказала ей на другой день: `отдай же твоего сына, и съедим его'. Но она спрятала своего сына.
\par 30 Царь, выслушав слова женщины, разодрал одежды свои; и проходил он по стене, и народ видел, что вретище на самом теле его.
\par 31 И сказал: пусть то и то сделает мне Бог, и еще более сделает, если останется голова Елисея, сына Сафатова, на нем сегодня.
\par 32 Елисей же сидел в своем доме, и старцы сидели у него. И послал [царь] человека от себя. Прежде нежели пришел посланный к нему, он сказал старцам: видите ли, что этот сын убийцы послал снять с меня голову? Смотрите, когда придет посланный, затворите дверь и прижмите его дверью. А вот и топот ног господина его за ним!
\par 33 Еще говорил он с ними, и вот посланный приходит к нему, и сказал: вот какое бедствие от Господа! чего мне впредь ждать от Господа?

\chapter{7}

\par 1 И сказал Елисей: выслушайте слово Господне: так говорит Господь: завтра в это время мера муки лучшей [будет] по сиклю и две меры ячменя по сиклю у ворот Самарии.
\par 2 И отвечал сановник, на руку которого царь опирался, человеку Божию, и сказал: если бы Господь и открыл окна на небе, и тогда может ли это быть? И сказал тот: вот увидишь глазами твоими, но есть этого не будешь.
\par 3 Четыре человека прокаженных находились при входе в ворота и говорили они друг другу: что нам сидеть здесь, ожидая смерти?
\par 4 Если решиться нам пойти в город, то в городе голод, и мы там умрем; если же сидеть здесь, то также умрем. Пойдем лучше в стан Сирийский. Если оставят нас в живых, будем жить, а если умертвят, умрем.
\par 5 И встали в сумерки, чтобы пойти в стан Сирийский. И пришли к краю стана Сирийского, и вот, нет там ни одного человека.
\par 6 Господь сделал то, что стану Сирийскому послышался стук колесниц и ржание коней, шум войска большого. И сказали они друг другу: верно нанял против нас царь Израильский царей Хеттейских и Египетских, чтобы пойти на нас.
\par 7 И встали и побежали в сумерки, и оставили шатры свои, и коней своих, и ослов своих, весь стан, как он был, и побежали, спасая себя.
\par 8 И пришли те прокаженные к краю стана, и вошли в один шатер, и ели и пили, и взяли оттуда серебро, и золото, и одежды, и пошли и спрятали. Пошли еще в другой шатер, и там взяли, и пошли и спрятали.
\par 9 И сказали друг другу: не так мы делаем. День сей--день радостной вести, если мы замедлим и будем дожидаться утреннего света, то падет на нас вина. Пойдем же и уведомим дом царский.
\par 10 И пришли, и позвали привратников городских, и рассказали им, говоря: мы ходили в стан Сирийский, и вот, нет там ни человека, ни голоса человеческого, а только кони привязанные, и ослы привязанные, и шатры, как быть им.
\par 11 И позвали привратников, и они передали весть в самый дворец царский.
\par 12 И встал царь ночью, и сказал слугам своим: скажу вам, что делают с нами Сирияне. Они знают, что мы терпим голод, и вышли из стана, чтобы спрятаться в поле, думая так: `когда они выйдут из города, мы захватим их живыми и вторгнемся в город'.
\par 13 И отвечал один из служащих при нем, и сказал: пусть возьмут пять из остальных коней, которые остались в городе, (из всего ополчения Израильтян только и осталось в нем, из всего ополчения Израильтян, которое погибло), и пошлем, и посмотрим.
\par 14 И взяли две пары коней, запряженных в колесницы. И послал царь вслед Сирийского войска, сказав: пойдите, посмотрите.
\par 15 И ехали за ним до Иордана, и вот вся дорога устлана одеждами и вещами, которые побросали Сирияне при торопливом побеге своем. И возвратились посланные, и донесли царю.
\par 16 И вышел народ, и разграбил стан Сирийский, и была мера муки лучшей по сиклю, и две меры ячменя по сиклю, по слову Господню.
\par 17 И царь поставил того сановника, на руку которого опирался, у ворот; и растоптал его народ в воротах, и он умер, как сказал человек Божий, который говорил, когда приходил к нему царь.
\par 18 Когда говорил человек Божий царю так: `две меры ячменя по сиклю, и мера муки лучшей по сиклю будут завтра в это время у ворот Самарии',
\par 19 тогда отвечал этот сановник человеку Божию и сказал: `если бы Господь и открыл окна на небе, и тогда может ли это быть?' А он сказал: `увидишь твоими глазами, но есть этого не будешь'.
\par 20 Так и сбылось с ним; и затоптал его народ в воротах, и он умер.

\chapter{8}

\par 1 И говорил Елисей женщине, сына которой воскресил он, и сказал: встань, и пойди, ты и дом твой, и поживи там, где можешь пожить, ибо призвал Господь голод, и он придет на сию землю на семь лет.
\par 2 И встала та женщина, и сделала по слову человека Божия; и пошла она и дом ее, и жила в земле Филистимской семь лет.
\par 3 По прошествии семи лет возвратилась эта женщина из земли Филистимской и пришла просить царя о доме своем и о поле своем.
\par 4 Царь тогда разговаривал с Гиезием, слугою человека Божия, и сказал: расскажи мне все замечательное, что сделал Елисей.
\par 5 И между тем как он рассказывал царю, что тот воскресил умершего, женщина, которой сына воскресил он, просила царя о доме своем и о поле своем. И сказал Гиезий: господин мой царь, это та самая женщина и тот самый сын ее, которого воскресил Елисей.
\par 6 И спросил царь у женщины, и она рассказала ему. И дал ей царь одного из придворных, сказав: возвратить ей все принадлежащее ей и все доходы с поля, с того дня, как она оставила землю, поныне.
\par 7 И пришел Елисей в Дамаск, когда Венадад, царь Сирийский, был болен. И донесли ему, говоря: пришел человек Божий сюда.
\par 8 И сказал царь Азаилу: возьми в руку твою дар и пойди навстречу человеку Божию, и вопроси Господа чрез него, говоря: выздоровею ли я от сей болезни?
\par 9 И пошел Азаил навстречу ему, и взял дар в руку свою и всего лучшего в Дамаске, сколько могут нести сорок верблюдов, и пришел и стал пред лице его, и сказал: сын твой Венадад, царь Сирийский, послал меня к тебе спросить: `выздоровею ли я от сей болезни?'
\par 10 И сказал ему Елисей: пойди, скажи ему: `выздоровеешь'; однакож открыл мне Господь, что он умрет.
\par 11 И устремил на него [Елисей] взор свой, и так оставался до того, что привел его в смущение; и заплакал человек Божий.
\par 12 И сказал Азаил: отчего господин мой плачет? И сказал он: оттого, что я знаю, какое наделаешь ты сынам Израилевым зло; крепости их предашь огню, и юношей их мечом умертвишь, и грудных детей их побьешь, и беременных [женщин] у них разрубишь.
\par 13 И сказал Азаил: что такое раб твой, пес, чтобы мог сделать такое большое дело? И сказал Елисей: указал мне Господь в тебе царя Сирии.
\par 14 И пошел он от Елисея, и пришел к государю своему. И сказал ему [этот]: что говорил тебе Елисей? И сказал: он говорил мне, что ты выздоровеешь.
\par 15 А на другой день он взял одеяло, намочил его водою, и положил на лице его, и он умер. И воцарился Азаил вместо него.
\par 16 В пятый год Иорама, сына Ахавова, царя Израильского, за Иосафатом, царем Иудейским, воцарился Иорам, сын Иосафатов, царь Иудейский.
\par 17 Тридцати двух лет был он, когда воцарился, и восемь лет царствовал в Иерусалиме,
\par 18 и ходил путем царей Израильских, как поступал дом Ахавов, потому что дочь Ахава была женою его, и делал неугодное в очах Господних.
\par 19 Однакож не хотел Господь погубить Иуду, ради Давида, раба Своего, так как Он обещал дать ему светильник в детях его на все времена.
\par 20 Во дни его выступил Едом из-под руки Иуды, и поставили они над собою царя.
\par 21 И пошел Иорам в Цаир, и все колесницы с ним; и встал он ночью, и поразил Идумеян, окружавших его, и начальников над колесницами, но народ убежал в шатры свои.
\par 22 И выступил Едом из-под руки Иуды до сего дня. В то же время выступила и Ливна.
\par 23 Прочее об Иораме и обо всем, что он сделал, написано в летописи царей Иудейских.
\par 24 И почил Иорам с отцами своими, и погребен с отцами своими в городе Давидовом. И воцарился Охозия, сын его, вместо него.
\par 25 В двенадцатый год Иорама, сына Ахавова, царя Израильского, воцарился Охозия, сын Иорама, царя Иудейского.
\par 26 Двадцати двух лет был Охозия, когда воцарился, и один год царствовал в Иерусалиме. Имя же матери его Гофолия, дочь Амврия, царя Израильского.
\par 27 И ходил путем дома Ахавова, и делал неугодное в очах Господних, подобно дому Ахавову, потому что он был в родстве с домом Ахавовым.
\par 28 И пошел он с Иорамом, сыном Ахавовым, на войну с Азаилом, царем Сирийским, в Рамоф Галаадский, и ранили Сирияне Иорама.
\par 29 И возвратился Иорам царь, чтобы лечиться в Изрееле от ран, которые причинили ему Сирияне в Рамофе, когда он воевал с Азаилом, царем Сирийским. И Охозия, сын Иорама, царь Иудейский, пришел посетить Иорама, сына Ахавова, в Изреель, так как он был болен.

\chapter{9}

\par 1 Елисей пророк призвал одного из сынов пророческих и сказал ему: опояшь чресла твои, и возьми сей сосуд с елеем в руку твою, и пойди в Рамоф Галаадский.
\par 2 Придя туда, отыщи там Ииуя, сына Иосафата, сына Намессиева, и подойди, и вели выступить ему из среды братьев своих, и введи его во внутреннюю комнату;
\par 3 и возьми сосуд с елеем, и вылей на голову его, и скажи: `так говорит Господь: помазую тебя в царя над Израилем'. Потом отвори дверь, и беги, и не жди.
\par 4 И пошел отрок, слуга пророка, в Рамоф Галаадский,
\par 5 и пришел, и вот сидят военачальники. И сказал: у меня слово до тебя, военачальник. И сказал Ииуй: до кого из всех нас? И сказал он: до тебя, военачальник.
\par 6 И встал он, и вошел в дом. И [отрок] вылил елей на голову его, и сказал ему: так говорит Господь Бог Израилев: `помазую тебя в царя над народом Господним, над Израилем,
\par 7 и ты истребишь дом Ахава, господина твоего, чтобы Мне отмстить за кровь рабов Моих пророков и за кровь всех рабов Господних, [павших] от руки Иезавели;
\par 8 и погибнет весь дом Ахава, и истреблю у Ахава мочащегося к стене, и заключенного и оставшегося в Израиле,
\par 9 и сделаю дом Ахава, как дом Иеровоама, сына Наватова, и как дом Ваасы, сына Ахиина;
\par 10 Иезавель же съедят псы на поле Изреельском, и никто не похоронит ее'. И отворил дверь, и убежал.
\par 11 И вышел Ииуй к слугам господина своего, и сказали ему: с миром ли? Зачем приходил этот неистовый к тебе? И сказал им: вы знаете этого человека и что он говорит.
\par 12 И сказали: неправда, скажи нам. И сказал он: то и то он сказал мне, говоря: `так говорит Господь: помазую тебя в царя над Израилем'.
\par 13 И поспешили они, и взяли каждый одежду свою, и подостлали ему на самых ступенях, и затрубили трубою, и сказали: воцарился Ииуй!
\par 14 И восстал Ииуй, сын Иосафата, сына Намессиева, против Иорама; Иорам же находился со всеми Израильтянами в Рамофе Галаадском на страже против Азаила, царя Сирийского.
\par 15 Впрочем сам царь Иорам возвратился, чтобы лечиться в Изрееле от ран, которые причинили ему Сирияне, когда он воевал с Азаилом, царем Сирийским. И сказал Ииуй: если вы согласны, то пусть никто не уходит из города, чтобы идти подать весть в Изрееле.
\par 16 И сел Ииуй на коня, и поехал в Изреель, где лежал Иорам, и куда Охозия, царь Иудейский, пришел посетить Иорама.
\par 17 На башне в Изрееле стоял сторож, и увидел он полчище Ииуево, когда оно шло, и сказал: полчище вижу я. И сказал Иорам: возьми всадника, и пошли навстречу им, и пусть скажет: с миром ли?
\par 18 И выехал всадник на коне навстречу ему, и сказал: так говорит царь: с миром ли? И сказал Ииуй: что тебе до мира? Поезжай за мною. И донес сторож, и сказал: доехал до них, но не возвращается.
\par 19 И послали другого всадника, и он приехал к ним, и сказал: так говорит царь: с миром ли? И сказал Ииуй: что тебе до мира? Поезжай за мною.
\par 20 И донес сторож, сказав: доехал до них, и не возвращается, а походка, как будто Ииуя, сына Намессиева, потому что он идет стремительно.
\par 21 И сказал Иорам: запрягай. И запрягли колесницу его. И выступил Иорам, царь Израильский, и Охозия, царь Иудейский, каждый на колеснице своей. И выступили навстречу Ииую, и встретились с ним на поле Навуфея Изреелитянина.
\par 22 И когда увидел Иорам Ииуя, то сказал: с миром ли Ииуй? И сказал он: какой мир при любодействе Иезавели, матери твоей, и при многих волхвованиях ее?
\par 23 И поворотил Иорам руки свои, и побежал, и сказал Охозии: измена, Охозия!
\par 24 А Ииуй натянул лук рукою своею, и поразил Иорама между плечами его, и прошла стрела чрез сердце его, и пал он на колеснице своей.
\par 25 И сказал Ииуй Бидекару, сановнику своему: возьми, брось его на участок поля Навуфея Изреелитянина, ибо вспомни, как мы с тобою ехали вдвоем сзади Ахава, отца его, и как Господь изрек на него такое пророчество:
\par 26 истинно, кровь Навуфея и кровь сыновей его видел Я вчера, говорит Господь, и отмщу тебе на сем поле. Итак возьми, брось его на поле, по слову Господню.
\par 27 Охозия, царь Иудейский, увидев сие, побежал по дороге к дому, что в саду. И погнался за ним Ииуй, и сказал: и его бейте на колеснице. [Это] [было] на возвышенности Гур, что при Ивлеаме. И побежал он в Мегиддон, и умер там.
\par 28 И отвезли его рабы его в Иерусалим, и похоронили его в гробнице его, с отцами его, в городе Давидовом.
\par 29 В одиннадцатый год Иорама, сына Ахавова, воцарился Охозия в Иудее.
\par 30 И прибыл Ииуй в Изреель. Иезавель же, получив весть, нарумянила лице свое и украсила голову свою, и глядела в окно.
\par 31 Когда Ииуй вошел в ворота, она сказала: мир ли Замврию, убийце государя своего?
\par 32 И поднял он лице свое к окну и сказал: кто со мною, кто? И выглянули к нему два, три евнуха.
\par 33 И сказал он: выбросьте ее. И выбросили ее. И брызнула кровь ее на стену и на коней, и растоптали ее.
\par 34 И пришел Ииуй, и ел, и пил, и сказал: отыщите эту проклятую и похороните ее, так как царская дочь она.
\par 35 И пошли хоронить ее, и не нашли от нее ничего, кроме черепа, и ног, и кистей рук.
\par 36 И возвратились, и донесли ему. И сказал он: таково было слово Господа, которое Он изрек чрез раба Своего Илию Фесвитянина, сказав: на поле Изреельском съедят псы тело Иезавели,
\par 37 и будет труп Иезавели на участке Изреельском, как навоз на поле, так что никто не скажет: это Иезавель.

\chapter{10}

\par 1 У Ахава было семьдесят сыновей в Самарии. И написал Ииуй письма, и послал в Самарию к начальникам Изреельским, старейшинам и воспитателям детей Ахавовых, такого содержания:
\par 2 когда придет это письмо к вам, то, так как у вас и сыновья господина вашего, у вас же и колесницы, и кони, и укрепленный город, и оружие, --
\par 3 выберите лучшего и достойнейшего из сыновей государя своего, и посадите на престол отца его, и воюйте за дом государя своего.
\par 4 Они испугались чрезвычайно и сказали: вот, два царя не устояли перед ним, как же нам устоять?
\par 5 И послал начальствующий над домом [царским], и градоначальник, и старейшины, и воспитатели к Ииую, сказать: мы рабы твои, и что скажешь нам, то и сделаем; мы никого не поставим царем, что угодно тебе, то и делай.
\par 6 И написал он к ним письмо во второй раз такое: если вы мои и слову моему повинуетесь, то возьмите головы сыновей государя своего, и придите ко мне завтра в это время в Изреель. (Царских же сыновей было семьдесят человек; воспитывали их знатнейшие в городе.)
\par 7 Когда пришло к ним письмо, они взяли царских сыновей, и закололи их--семьдесят человек, и положили головы их в корзины, и послали к нему в Изреель.
\par 8 И пришел посланный, и донес ему, и сказал: принесли головы сыновей царских. И сказал он: разложите их на две груды у входа в ворота, до утра.
\par 9 Поутру он вышел, и стал, и сказал всему народу: вы невиновны. Вот я восстал против государя моего и умертвил его, а их всех кто убил?
\par 10 Знайте же теперь, что не падет на землю ни одно слово Господа, которое Он изрек о доме Ахава; Господь сделал то, что изрек чрез раба Своего Илию.
\par 11 И умертвил Ииуй всех оставшихся из дома Ахава в Изрееле, и всех вельмож его, и близких его, и священников его, так что не осталось от него ни одного уцелевшего.
\par 12 И встал, и пошел, и пришел в Самарию. Находясь на пути при Беф-Екеде пастушеском,
\par 13 встретил Ииуй братьев Охозии, царя Иудейского, и сказал: кто вы? Они сказали: мы братья Охозии, идем узнать о здоровье сыновей царя и сыновей государыни.
\par 14 И сказал он: возьмите их живых. И взяли их живых, и закололи их--сорок два человека, при колодезе Беф-Екеда, и не осталось из них ни одного.
\par 15 И поехал оттуда, и встретился с Ионадавом, сыном Рихавовым, [шедшим] навстречу ему, и приветствовал его, и сказал ему: расположено ли твое сердце так, как мое сердце к твоему сердцу? И сказал Ионадав: да. Если так, то дай руку твою. И подал он руку свою, и приподнял он его к себе в колесницу,
\par 16 и сказал: поезжай со мною, и смотри на мою ревность о Господе. И посадили его в колесницу.
\par 17 Прибыв в Самарию, он убил всех, остававшихся у Ахава в Самарии, так что совсем истребил его, по слову Господа, которое Он изрек Илии.
\par 18 И собрал Ииуй весь народ и сказал им: Ахав мало служил Ваалу; Ииуй будет служить ему более.
\par 19 Итак созовите ко мне всех пророков Ваала, всех служителей его и всех священников его, чтобы никто не был в отсутствии, потому что у меня будет великая жертва Ваалу. А всякий, кто не явится, не останется жив. Ииуй делал [это] с хитрым намерением, чтобы истребить служителей Ваала.
\par 20 И сказал Ииуй: назначьте праздничное собрание ради Ваала. И провозгласили [собрание].
\par 21 И послал Ииуй по всему Израилю, и пришли все служители Ваала; не оставалось ни одного человека, кто бы не пришел; и вошли в дом Ваалов, и наполнился дом Ваалов от края до края.
\par 22 И сказал он хранителю одежд: принеси одежду для всех служителей Ваала. И он принес им одежду.
\par 23 И вошел Ииуй с Ионадавом, сыном Рихавовым, в дом Ваалов, и сказал служителям Ваала: разведайте и разглядите, не находится ли у вас кто-нибудь из служителей Господних, так как здесь должны находиться только одни служители Ваала.
\par 24 И приступили они к совершению жертв и всесожжений. А Ииуй поставил вне [дома] восемьдесят человек и сказал: душа того, у которого спасется кто-либо из людей, которых я отдаю вам в руки, будет вместо души [спасшегося].
\par 25 Когда кончено было всесожжение, сказал Ииуй скороходам и начальникам: пойдите, бейте их, чтобы ни один не ушел. И поразили их острием меча и бросили [их] скороходы и начальники, и пошли в город, где было капище Ваалово.
\par 26 И вынесли статуи из капища Ваалова и сожгли их.
\par 27 И разбили статую Ваала, и разрушили капище Ваалово; и сделали из него место нечистот, до сего дня.
\par 28 И истребил Ииуй Ваала с земли Израильской.
\par 29 Впрочем от грехов Иеровоама, сына Наватова, который ввел Израиля в грех, от них не отступал Ииуй, --от золотых тельцов, которые в Вефиле и которые в Дане.
\par 30 И сказал Господь Ииую: за то, что ты охотно сделал, что было праведно в очах Моих, выполнил над домом Ахавовым все, что было на сердце у Меня, сыновья твои до четвертого рода будут сидеть на престоле Израилевом.
\par 31 Но Ииуй не старался ходить в законе Господа Бога Израилева, от всего сердца. Он не отступал от грехов Иеровоама, который ввел Израиля в грех.
\par 32 В те дни начал Господь отрезать части от Израильтян, и поражал их Азаил во всем пределе Израилевом,
\par 33 на восток от Иордана, всю землю Галаад, [колено] Гадово, Рувимово, Манассиино, [начиная] от Ароера, который при потоке Арноне, и Галаад и Васан.
\par 34 Прочее об Ииуе и обо всем, что он сделал, и о мужественных подвигах его написано в летописи царей Израильских.
\par 35 И почил Ииуй с отцами своими, и похоронили его в Самарии. И воцарился Иоахаз, сын его, вместо него.
\par 36 Времени же царствования Ииуева над Израилем, в Самарии, было двадцать восемь лет.

\chapter{11}

\par 1 Гофолия, мать Охозии, видя, что сын ее умер, встала и истребила все царское племя.
\par 2 Но Иосавеф, дочь царя Иорама, сестра Охозии, взяла Иоаса, сына Охозии, и тайно увела его из среды умерщвляемых сыновей царских, его и кормилицу его, в постельную комнату; и скрыли его от Гофолии, и он не умерщвлен.
\par 3 И был он с нею скрываем в доме Господнем шесть лет, между тем как Гофолия царствовала над землею.
\par 4 В седьмой год послал Иодай, и взял сотников из телохранителей и скороходов, и привел их к себе в дом Господень, и сделал с ними договор, и взял с них клятву в доме Господнем, и показал им царского сына.
\par 5 И дал им приказание, сказав: вот что вы сделайте: третья часть из вас, из приходящих в субботу, будет содержать стражу при царском доме;
\par 6 третья часть у ворот Сур, и третья часть у ворот сзади телохранителей, и содержите стражу дома, чтобы не было повреждения;
\par 7 и две части из вас, из всех отходящих в субботу, будут содержать стражу при доме Господнем для царя;
\par 8 и окружите царя со всех сторон, каждый с оружием своим в руке своей; и кто вошел бы в ряды, тот да будет умерщвлен. И будьте при царе, когда он выходит и когда входит.
\par 9 И сделали сотники все, что приказал Иодай священник, и взяли каждый людей своих, приходящих в субботу и отходящих в субботу, и пришли к Иодаю священнику.
\par 10 И раздал священник сотникам копья и щиты царя Давида, которые были в доме Господнем.
\par 11 И стали скороходы, каждый с оружием в руке своей, от правой стороны дома до левой стороны дома, у жертвенника и у дома, вокруг царя.
\par 12 И вывел он царского сына, и возложил на него [царский] венец и украшения, и воцарили его, и помазали его, и рукоплескали и восклицали: да живет царь!
\par 13 И услышала Гофолия голос бегущего народа, и пошла к народу в дом Господень.
\par 14 И видит, и вот царь стоит на возвышении, по обычаю, и князья и трубы подле царя; и весь народ земли веселится, и трубят трубами. И разодрала Гофолия одежды свои, и закричала: заговор! заговор!
\par 15 И дал приказание Иодай священник сотникам, начальствующим над войском, и сказал им: `выведите ее за ряды, а кто пойдет за нею, умерщвляйте мечом', так как думал священник, чтобы не умертвили ее в доме Господнем.
\par 16 И дали ей место, и она прошла чрез вход конский к дому царскому, и умерщвлена там.
\par 17 И заключил Иодай завет между Господом и между царем и народом, чтоб он был народом Господним, и между царем и народом.
\par 18 И пошел весь народ земли в дом Ваала, и разрушили жертвенники его, и изображения его совершенно разбили, и Матфана, жреца Ваалова, убили пред жертвенниками. И учредил священник наблюдение над домом Господним.
\par 19 И взял сотников и телохранителей и скороходов и весь народ земли, и проводили царя из дома Господня, и пришли по дороге чрез ворота телохранителей в дом царский; и он воссел на престоле царей.
\par 20 И веселился весь народ земли, и город успокоился. А Гофолию умертвили мечом в царском доме.
\par 21 Семи лет был Иоас, когда воцарился.

\chapter{12}

\par 1 В седьмой год Ииуя воцарился Иоас и сорок лет царствовал в Иерусалиме. Имя матери его Цивья, из Вирсавии.
\par 2 И делал Иоас угодное в очах Господних во все дни свои, доколе наставлял его священник Иодай;
\par 3 только высоты не были отменены; народ еще приносил жертвы и курения на высотах.
\par 4 И сказал Иоас священникам: все серебро посвящаемое, которое приносят в дом Господень, серебро от приходящих, серебро, [вносимое] за каждую душу по оценке, все серебро, сколько кому приходит на сердце принести в дом Господень,
\par 5 пусть берут священники себе, каждый от своего знакомого, и пусть исправляют они поврежденное в храме, везде, где найдется повреждение.
\par 6 Но как до двадцать третьего года царя Иоаса священники не исправляли повреждений в храме,
\par 7 то царь Иоас позвал священника Иодая и священников и сказал им: почему вы не исправляете повреждений в храме? Не берите же отныне серебра у знакомых своих, а на [починку] повреждений в храме отдайте его.
\par 8 И согласились священники не брать серебра у народа на исправление повреждений в храме.
\par 9 И взял священник Иодай один ящик, и сделал отверстие сверху его, и поставил его подле жертвенника на правой стороне, где входили в дом Господень. И полагали туда священники, стоящие на страже у порога, все серебро, приносимое в дом Господень.
\par 10 И когда видели, что много серебра в ящике, приходили писец царский и первосвященник, и завязывали [в мешки], и пересчитывали серебро, найденное в доме Господнем;
\par 11 и отдавали сосчитанное серебро в руки производителям работ, приставленным к дому Господню, а сии издерживали его на плотников и строителей, работавших в доме Господнем,
\par 12 и на делателей стен и на каменотесов, также на покупку дерев и тесаных камней, для починки повреждений в доме Господнем, и на все, что расходовалось для поддержания храма.
\par 13 Но не сделано было для дома Господня серебряных блюд, ножей, чаш [для окропления], труб, всяких сосудов золотых и сосудов серебряных из серебра, приносимого в дом Господень,
\par 14 а производителям работ отдавали его, и починивали им дом Господень.
\par 15 И не требовали отчета от тех людей, которым поручали серебро для раздачи производителям работ, ибо они действовали честно.
\par 16 Серебро за жертву о преступлении и серебро за жертву о грехе не вносилось в дом Господень: священникам оно принадлежало.
\par 17 Тогда выступил в поход Азаил, царь Сирийский, и пошел войною на Геф, и взял его; и вознамерился Азаил идти на Иерусалим.
\par 18 Но Иоас, царь Иудейский, взял все пожертвованное, что пожертвовали [храму] Иосафат, и Иорам и Охозия, отцы его, цари Иудейские, и что он сам пожертвовал, и все золото, найденное в сокровищницах дома Господня и дома царского, и послал Азаилу, царю Сирийскому; и он отступил от Иерусалима.
\par 19 Прочее об Иоасе и обо всем, что он сделал, написано в летописи царей Иудейских.
\par 20 И восстали слуги его, и составили заговор, и убили Иоаса в доме Милло, на дороге к Силле.
\par 21 Его убили слуги его: Иозакар, сын Шимеаты, и Иегозавад, сын Шомеры; и он умер, и похоронили его с отцами его в городе Давидовом. И воцарился Амасия, сын его, вместо него.

\chapter{13}

\par 1 В двадцать третий год Иоаса, сына Охозиина, царя Иудейского, воцарился Иоахаз, сын Ииуя, над Израилем в Самарии, [и царствовал] семнадцать лет,
\par 2 и делал неугодное в очах Господних, и ходил в грехах Иеровоама, сына Наватова, который ввел Израиля в грех, и не отставал от них.
\par 3 И возгорелся гнев Господа на Израиля, и Он предавал их в руку Азаила, царя Сирийского, и в руку Венадада, сына Азаилова, во все дни.
\par 4 И помолился Иоахаз лицу Господню, и услышал его Господь, потому что видел стеснение Израильтян, как теснил их царь Сирийский.
\par 5 И дал Господь Израильтянам избавителя, и вышли они из-под руки Сириян, и жили сыны Израилевы в шатрах своих, как вчера и третьего дня.
\par 6 Однакож не отступали от грехов дома Иеровоама, который ввел Израиля в грех; ходили в них, и дубрава стояла в Самарии.
\par 7 У Иоахаза оставалось войска только пятьдесят всадников, десять колесниц и десять тысяч пеших, оттого, что истребил их царь Сирийский и обратил их в прах на попрание.
\par 8 Прочее об Иоахазе и обо всем, что он сделал, и о мужественных подвигах его, написано в летописи царей Израильских.
\par 9 И почил Иоахаз с отцами своими, и похоронили его в Самарии. И воцарился Иоас, сын его, вместо него.
\par 10 В тридцать седьмой год Иоаса, царя Иудейского, воцарился Иоас, сын Иоахазов, над Израилем в Самарии, [и царствовал] шестнадцать лет,
\par 11 и делал неугодное в очах Господних; не отставал от всех грехов Иеровоама, сына Наватова, который ввел Израиля в грех, но ходил в них.
\par 12 Прочее об Иоасе и обо всем, что он сделал, и о мужественных подвигах его, как он воевал с Амасиею, царем Иудейским, написано в летописи царей Израильских.
\par 13 И почил Иоас с отцами своими, а Иеровоам сел на престоле его. И погребен Иоас в Самарии с царями Израильскими.
\par 14 Елисей заболел болезнью, от которой [потом] и умер. И пришел к нему Иоас, царь Израильский, и плакал над ним, и говорил: отец мой! отец мой! колесница Израиля и конница его!
\par 15 И сказал ему Елисей: возьми лук и стрелы. И взял он лук и стрелы.
\par 16 И сказал царю Израильскому: положи руку твою на лук. И положил он руку свою. И наложил Елисей руки свои на руки царя,
\par 17 и сказал: отвори окно на восток. И он отворил. И сказал Елисей: выстрели. И он выстрелил. И сказал: эта стрела избавления от Господа и стрела избавления против Сирии, и ты поразишь Сириян в Афеке вконец.
\par 18 И сказал [Елисей]: возьми стрелы. И он взял. И сказал царю Израильскому: бей по земле. И ударил он три раза, и остановился.
\par 19 И разгневался на него человек Божий, и сказал: надобно было бы бить пять или шесть раз, тогда ты побил бы Сириян совершенно, а теперь [только] три раза поразишь Сириян.
\par 20 И умер Елисей, и похоронили его. И полчища Моавитян пришли в землю в следующем году.
\par 21 И было, что, когда погребали одного человека, то, увидев это полчище, [погребавшие] бросили того человека в гроб Елисеев; и он при падении своем коснулся костей Елисея, и ожил, и встал на ноги свои.
\par 22 Азаил, царь Сирийский, теснил Израильтян во все дни Иоахаза.
\par 23 Но Господь умилосердился над ними, и помиловал их, и обратился к ним ради завета Своего с Авраамом, Исааком и Иаковом, и не хотел истребить их, и не отверг их от лица Своего доныне.
\par 24 И умер Азаил, царь Сирийский, и воцарился Венадад, сын его, вместо него.
\par 25 И взял назад Иоас, сын Иоахаза, из руки Венадада, сына Азаила, города, которые он взял войною из руки отца его Иоахаза. Три раза разбил его Иоас и возвратил города Израилевы.

\chapter{14}

\par 1 Во второй год Иоаса, сына Иоахазова, царя Израильского, воцарился Амасия, сын Иоаса, царь Иудейский:
\par 2 двадцати пяти лет был он, когда воцарился, и двадцать девять лет царствовал в Иерусалиме. Имя матери его Иегоаддань, из Иерусалима.
\par 3 И делал он угодное в очах Господних, впрочем не так, как отец его Давид: он во всем поступал так, как отец его Иоас.
\par 4 Только высоты не были отменены: народ совершал еще жертвы и курения на высотах.
\par 5 Когда утвердилось царство в руках его, тогда он умертвил слуг своих, убивших царя, отца его.
\par 6 Но детей убийц не умертвил, так как написано в книге закона Моисеева, в которой заповедал Господь, говоря: `не должны быть наказываемы смертью отцы за детей, и дети не должны быть наказываемы смертью за отцов, но каждый за свое преступление должен быть наказываем смертью'.
\par 7 Он поразил десять тысяч Идумеян на долине Соляной, и взял Селу войною, и дал ей имя Иокфеил, которое [остается и] до сего дня.
\par 8 Тогда послал Амасия послов к Иоасу, царю Израильскому, сыну Иоахаза, сына Ииуева, сказать: выйди, повидаемся лично.
\par 9 И послал Иоас, царь Израильский, к Амасии, царю Иудейскому, сказать: терн, который на Ливане, послал к кедру, который на Ливане же, сказать: `отдай дочь свою в жену сыну моему'. Но прошли дикие двери, что на Ливане, и истоптали этот терн.
\par 10 Ты поразил Идумеян, и возгордилось сердце твое. Величайся и сиди у себя дома. К чему тебе затевать ссору на свою беду? Падешь ты и Иуда с тобою.
\par 11 Но не послушался Амасия. И выступил Иоас, царь Израильский, и увиделись лично он и Амасия, царь Иудейский, в Вефсамисе, что в Иудее.
\par 12 И разбиты были Иудеи Израильтянами, и разбежались по шатрам своим.
\par 13 И Амасию, царя Иудейского, сына Иоаса, сына Охозиина, захватил Иоас, царь Израильский, в Вефсамисе, и пошел в Иерусалим и разрушил стену Иерусалимскую от ворот Ефремовых до ворот угольных на четыреста локтей.
\par 14 И взял все золото и серебро, и все сосуды, какие нашлись в доме Господнем и в сокровищницах царского дома, и заложников, и возвратился в Самарию.
\par 15 Прочее об Иоасе, что он сделал, и о мужественных подвигах его, и как он воевал с Амасиею, царем Иудейским, написано в летописи царей Израильских.
\par 16 И почил Иоас с отцами своими, и погребен в Самарии с царями Израильскими. И воцарился Иеровоам, сын его, вместо него.
\par 17 И жил Амасия, сын Иоасов, царь Иудейский, по смерти Иоаса, сына Иоахазова, царя Израильского, пятнадцать лет.
\par 18 Прочие дела Амасии записаны в летописи царей Иудейских.
\par 19 И составили против него заговор в Иерусалиме, и убежал он в Лахис. И послали за ним в Лахис, и умертвили его там.
\par 20 И привезли его на конях, и погребен он был в Иерусалиме с отцами своими в городе Давидовом.
\par 21 И взял весь народ Иудейский Азарию, которому было шестнадцать лет, и воцарили его вместо отца его Амасии.
\par 22 Он обстроил Елаф, и возвратил его Иуде, после того как царь почил с отцами своими.
\par 23 В пятнадцатый год Амасии, сына Иоасова, царя Иудейского, воцарился Иеровоам, сын Иоасов, царь Израильский, в Самарии, и [царствовал] сорок один год,
\par 24 и делал он неугодное в очах Господних: не отступал от всех грехов Иеровоама, сына Наватова, который ввел Израиля в грех.
\par 25 Он восстановил пределы Израиля, от входа в Емаф до моря пустыни, по слову Господа Бога Израилева, которое Он изрек чрез раба Своего Иону, сына Амафиина, пророка из Гафхефера,
\par 26 ибо Господь видел бедствие Израиля, весьма горькое, так что не оставалось ни заключенного, ни оставшегося, и не было помощника у Израиля.
\par 27 И не восхотел Господь искоренить имя Израильтян из поднебесной, и спас их рукою Иеровоама, сына Иоасова.
\par 28 Прочее об Иеровоаме и обо всем, что он сделал, и о мужественных подвигах его, как он воевал и как возвратил Израилю Дамаск и Емаф, принадлежавших Иуде, написано в летописи царей Израильских.
\par 29 И почил Иеровоам с отцами своими, с царями Израильскими. И воцарился Захария, сын его, вместо него.

\chapter{15}

\par 1 В двадцать седьмой год Иеровоама, царя Израильского, воцарился Азария, сын Амасии, царь Иудейский:
\par 2 шестнадцати лет был он, когда воцарился, и пятьдесят два года царствовал в Иерусалиме. Имя матери его Иехолия, из Иерусалима.
\par 3 Он делал угодное в очах Господних во всем так, как поступал Амасия, отец его.
\par 4 Только высоты не были отменены: народ совершал еще жертвы и курения на высотах.
\par 5 И поразил Господь царя, и был он прокаженным до дня смерти своей и жил в отдельном доме. И Иофам, сын царя, [начальствовал] над дворцом и управлял народом земли.
\par 6 Прочее об Азарии и обо всем, что он сделал, написано в летописи царей Иудейских.
\par 7 И почил Азария с отцами своими, и похоронили его с отцами его в городе Давидовом. И воцарился Иофам, сын его, вместо него.
\par 8 В тридцать восьмой год Азарии, царя Иудейского, воцарился Захария, сын Иеровоама, над Израилем в Самарии [и царствовал] шесть месяцев.
\par 9 Он делал неугодное в очах Господних, как делали отцы его: не отставал от грехов Иеровоама, сына Наватова, который ввел Израиля в грех.
\par 10 И составил против него заговор Селлум, сын Иависа, и поразил его пред народом и убил его, и воцарился вместо него.
\par 11 Прочее о Захарии написано в летописи царей Израильских.
\par 12 Таково было слово Господа, которое он изрек Ииую, сказав: сыновья твои до четвертого рода будут сидеть на престоле Израилевом. И сбылось так.
\par 13 Селлум, сын Иависа, воцарился в тридцать девятый год Азарии, царя Иудейского, и царствовал один месяц в Самарии.
\par 14 И пошел Менаим, сын Гадия из Фирцы, и пришел в Самарию, и поразил Селлума, сына Иависова, в Самарии и умертвил его, и воцарился вместо него.
\par 15 Прочее о Селлуме и о заговоре его, который он составил, написано в летописи царей Израильских.
\par 16 И поразил Менаим Типсах и всех, которые были в нем и в пределах его, [начиная] от Фирцы, за то, что [город] не отворил [ворот], и разбил [его], и всех беременных женщин в нем разрубил.
\par 17 В тридцать девятом году Азарии, царя Иудейского, воцарился Менаим, сын Гадия, над Израилем [и царствовал] десять лет в Самарии;
\par 18 и делал он неугодное в очах Господних; не отставал от грехов Иеровоама, сына Наватова, который ввел Израиля в грех, во все дни свои.
\par 19 Тогда пришел Фул, царь Ассирийский, на землю [Израилеву]. И дал Менаим Фулу тысячу талантов серебра, чтобы руки его были за него и чтобы утвердить царство в руке своей.
\par 20 И разложил Менаим это серебро на Израильтян, на всех людей богатых, по пятидесяти сиклей серебра на каждого человека, чтобы отдать царю Ассирийскому. И пошел назад царь Ассирийский и не остался там в земле.
\par 21 Прочее о Менаиме и обо всем, что он сделал, написано в летописи царей Израильских.
\par 22 И почил Менаим с отцами своими. И воцарился Факия, сын его, вместо него.
\par 23 В пятидесятый год Азарии, царя Иудейского, воцарился Факия, сын Менаима, над Израилем в Самарии [и царствовал] два года;
\par 24 и делал он неугодное в очах Господних; не отставал от грехов Иеровоама, сына Наватова, который ввел Израиля в грех.
\par 25 И составил против него заговор Факей, сын Ремалии, сановник его, и поразил его в Самарии в палате царского дома, с Арговом и Арием, имея с собою пятьдесят человек Галаадитян, и умертвил его, и воцарился вместо него.
\par 26 Прочее о Факии и обо всем, что он сделал, написано в летописи царей Израильских.
\par 27 В пятьдесят второй год Азарии, царя Иудейского, воцарился Факей, сын Ремалии, над Израилем в Самарии [и царствовал] двадцать лет;
\par 28 и делал он неугодное в очах Господних: не отставал от грехов Иеровоама, сына Наватова, который ввел Израиля в грех.
\par 29 Во дни Факея, царя Израильского, пришел Феглаффелласар, царь Ассирийский, и взял Ион, Авел-Беф-Мааху, и Ианох, и Кедес, и Асор, и Галаад, и Галилею, всю землю Неффалимову, и переселил их в Ассирию.
\par 30 И составил заговор Осия, сын Илы, против Факея, сына Ремалиина, и поразил его, и умертвил его, и воцарился вместо него в двадцатый год Иоафама, сына Озиина.
\par 31 Прочее о Факее и обо всем, что он сделал, написано в летописи царей Израильских.
\par 32 Во второй год Факея, сына Ремалиина, царя Израильского, воцарился Иоафам, сын Озии, царя Иудейского.
\par 33 Двадцати пяти лет был он, когда воцарился, и шестнадцать лет царствовал в Иерусалиме. Имя матери его Иеруша, дочь Садока.
\par 34 Он делал угодное в очах Господних: во всем, как поступал Озия, отец его, так поступал и он.
\par 35 Только высоты не были отменены: народ совершал еще жертвы и курения на высотах. Он построил верхние ворота при доме Господнем.
\par 36 Прочее об Иоафаме и обо всем, что он сделал, написано в летописи царей Иудейских.
\par 37 В те дни начал Господь посылать на Иудею Рецина, царя Сирийского, и Факея, сына Ремалиина.
\par 38 И почил Иоафам с отцами своими, и погребен с отцами своими в городе Давида, отца его. И воцарился Ахаз, сын его, вместо него.

\chapter{16}

\par 1 В семнадцатый год Факея, сына Ремалиина, воцарился Ахаз, сын Иоафама, царя Иудейского.
\par 2 Двадцати лет был Ахаз, когда воцарился, и шестнадцать лет царствовал в Иерусалиме, и не делал угодного в очах Господа Бога своего, как Давид, отец его,
\par 3 но ходил путем царей Израильских, и даже сына своего провел чрез огонь, [подражая] мерзостям народов, которых прогнал Господь от лица сынов Израилевых,
\par 4 и совершал жертвы и курения на высотах и на холмах и под всяким тенистым деревом.
\par 5 Тогда пошел Рецин, царь Сирийский, и Факей, сын Ремалиин, царь Израильский, против Иерусалима, чтобы завоевать его, и держали Ахаза в осаде, но одолеть не могли.
\par 6 В то время Рецин, царь Сирийский, возвратил Сирии Елаф и изгнал Иудеев из Елафа; и Идумеяне вступили в Елаф, и живут там до сего дня.
\par 7 И послал Ахаз послов к Феглаффелласару, царю Ассирийскому, сказать: раб твой и сын твой я; приди и защити меня от руки царя Сирийского и от руки царя Израильского, восставших на меня.
\par 8 И взял Ахаз серебро и золото, какое нашлось в доме Господнем и в сокровищницах дома царского, и послал царю Ассирийскому в дар.
\par 9 И послушал его царь Ассирийский; и пошел царь Ассирийский в Дамаск, и взял его, и переселил жителей его в Кир, а Рецина умертвил.
\par 10 И пошел царь Ахаз навстречу Феглаффелласару, царю Ассирийскому, в Дамаск, и увидел жертвенник, который в Дамаске, и послал царь Ахаз к Урии священнику изображение жертвенника и чертеж всего устройства его.
\par 11 И построил священник Урия жертвенник по образцу, который прислал царь Ахаз из Дамаска; и сделал так священник Урия до прибытия царя Ахаза из Дамаска.
\par 12 И пришел царь из Дамаска, и увидел царь жертвенник, и подошел царь к жертвеннику, и принес на нем жертву;
\par 13 и сожег всесожжение свое и хлебное приношение, и совершил возлияние свое, и окропил кровью мирной жертвы свой жертвенник.
\par 14 А медный жертвенник, который пред лицем Господним, он передвинул от лицевой стороны храма, с [места] между жертвенником [новым] и домом Господним, и поставил его сбоку [сего] жертвенника на север.
\par 15 И дал приказание царь Ахаз священнику Урии, сказав: на большом жертвеннике сожигай утреннее всесожжение и вечернее хлебное приношение, и всесожжение от царя и хлебное приношение от него, и всесожжение от всех людей земли и хлебное приношение от них, и возлияние от них, и всякою кровью всесожжений и всякою кровью жертв окропляй его, а жертвенник медный останется до моего усмотрения.
\par 16 И сделал священник Урия все так, как приказал царь Ахаз.
\par 17 И обломал царь Ахаз ободки у подстав, и снял с них умывальницы, и море снял с медных волов, которые [были] под ним, и поставил его на каменный пол.
\par 18 И отменил крытый субботний ход, который построили при храме, и внешний царский вход к дому Господню, ради царя Ассирийского.
\par 19 Прочее об Ахазе, что он сделал, написано в летописи царей Иудейских.
\par 20 И почил Ахаз с отцами своими, и погребен с отцами своими в городе Давидовом. И воцарился Езекия, сын его, вместо него.

\chapter{17}

\par 1 В двенадцатый год Ахаза, царя Иудейского, воцарился Осия, сын Илы, в Самарии над Израилем [и царствовал] девять лет.
\par 2 И делал он неугодное в очах Господних, но не так, как цари Израильские, которые были прежде него.
\par 3 Против него выступил Салманассар, царь Ассирийский, и сделался Осия подвластным ему и давал ему дань.
\par 4 И заметил царь Ассирийский в Осии измену, так как он посылал послов к Сигору, царю Египетскому, и не доставлял дани царю Ассирийскому каждый год; и взял его царь Ассирийский под стражу, и заключил его в дом темничный.
\par 5 И пошел царь Ассирийский на всю землю, и приступил к Самарии, и держал ее в осаде три года.
\par 6 В девятый год Осии взял царь Ассирийский Самарию, и переселил Израильтян в Ассирию, и поселил их в Халахе и в Хаворе, при реке Гозан, и в городах Мидийских.
\par 7 Когда стали грешить сыны Израилевы пред Господом Богом своим, Который вывел их из земли Египетской, из-под руки фараона, царя Египетского, и стали чтить богов иных,
\par 8 и стали поступать по обычаям народов, которых прогнал Господь от лица сынов Израилевых, и [по обычаям] царей Израильских, как поступали они;
\par 9 и стали делать сыны Израилевы дела неугодные Господу Богу своему, и построили себе высоты во всех городах своих, [начиная] от сторожевой башни до укрепленного города,
\par 10 и поставили у себя статуи и изображения Астарт на всяком высоком холме и под всяким тенистым деревом,
\par 11 и стали там совершать курения на всех высотах, подобно народам, которых изгнал от них Господь, и делали худые дела, прогневляющие Господа,
\par 12 и служили идолам, о которых говорил им Господь: `не делайте сего';
\par 13 тогда Господь чрез всех пророков Своих, чрез всякого прозорливца предостерегал Израиля и Иуду, говоря: возвратитесь со злых путей ваших и соблюдайте заповеди Мои, уставы Мои, по всему учению, которое Я заповедал отцам вашим и которое Я преподал вам чрез рабов Моих, пророков.
\par 14 Но они не слушали и ожесточили выю свою, как была выя отцов их, которые не веровали в Господа, Бога своего;
\par 15 и презирали уставы Его, и завет Его, который Он заключил с отцами их, и откровения Его, какими Он предостерегал их, и пошли вслед суеты и осуетились, и вслед народов окрестных, о которых Господь заповедал им, чтобы не поступали так, как они,
\par 16 и оставили все заповеди Господа Бога своего, и сделали себе литые изображения двух тельцов, и устроили дубраву, и поклонялись всему воинству небесному, и служили Ваалу,
\par 17 и проводили сыновей своих и дочерей своих чрез огонь, и гадали, и волшебствовали, и предались тому, чтобы делать неугодное в очах Господа и прогневлять Его.
\par 18 И прогневался Господь сильно на Израильтян, и отверг их от лица Своего. Не осталось никого, кроме одного колена Иудина.
\par 19 И Иуда также не соблюдал заповедей Господа Бога своего, и поступал по обычаям Израильтян, как поступали они.
\par 20 И отвратился Господь от всех потомков Израиля, и смирил их, и отдавал их в руки грабителям, и наконец отверг их от лица Своего.
\par 21 Израильтяне отторглись от дома Давидова и воцарили Иеровоама, сына Наватова; и отклонил Иеровоам Израильтян от Господа, и вовлек их в великий грех.
\par 22 И поступали сыны Израилевы по всем грехам Иеровоама, какие он делал, не отставали от них,
\par 23 доколе Господь не отверг Израиля от лица Своего, как говорил чрез всех рабов Своих, пророков. И переселен Израиль из земли своей в Ассирию, где он и до сего дня.
\par 24 И перевел царь Ассирийский людей из Вавилона, и из Куты, и из Аввы, и из Емафа, и из Сепарваима, и поселил [их] в городах Самарийских вместо сынов Израилевых. И они овладели Самариею, и стали жить в городах ее.
\par 25 И как в начале жительства своего там они не чтили Господа, то Господь посылал на них львов, которые умерщвляли их.
\par 26 И донесли царю Ассирийскому, и сказали: народы, которых ты переселил и поселил в городах Самарийских, не знают закона Бога той земли, и за то Он посылает на них львов, и вот они умерщвляют их, потому что они не знают закона Бога той земли.
\par 27 И повелел царь Ассирийский, и сказал: отправьте туда одного из священников, которых вы выселили оттуда; пусть пойдет и живет там, и он научит их закону Бога той земли.
\par 28 И пришел один из священников, которых выселили из Самарии, и жил в Вефиле, и учил их, как чтить Господа.
\par 29 Притом сделал каждый народ и своих богов и поставил в капищах высот, какие устроили Самаряне, --каждый народ в своих городах, где живут они.
\par 30 Вавилоняне сделали Суккот-Беноф, Кутийцы сделали Нергала, Емафяне сделали Ашиму,
\par 31 Аввийцы сделали Нивхаза и Тартака, а Сепарваимцы сожигали сыновей своих в огне Адрамелеху и Анамелеху, богам Сепарваимским.
\par 32 Между тем чтили и Господа, и сделали у себя священников высот из среды своей, и они служили у них в капищах высот.
\par 33 Господа они чтили, и богам своим они служили по обычаю народов, из которых выселили их.
\par 34 До сего дня поступают они по прежним своим обычаям: не боятся Господа и не поступают по уставам и по обрядам, и по закону и по заповедям, которые заповедал Господь сынам Иакова, которому дал Он имя Израиля.
\par 35 Заключил Господь с ними завет и заповедал им, говоря: не чтите богов иных, и не поклоняйтесь им, и не служите им, и не приносите жертв им,
\par 36 но Господа, Который вывел вас из земли Египетской силою великою и мышцею простертою, --Его чтите и Ему поклоняйтесь, и Ему приносите жертвы,
\par 37 и уставы, и учреждения, и закон, и заповеди, которые Он написал вам, старайтесь исполнять во все дни, и не чтите богов иных;
\par 38 и завета, который Я заключил с вами, не забывайте, и не чтите богов иных,
\par 39 только Господа Бога вашего чтите, и Он избавит вас от руки всех врагов ваших.
\par 40 Но они не послушали, а поступали по прежним своим обычаям.
\par 41 Народы сии чтили Господа, но и истуканам своим служили. Да и дети их и дети детей их до сего дня поступают так же, как поступали отцы их.

\chapter{18}

\par 1 В третий год Осии, сына Илы, царя Израильского, воцарился Езекия, сын Ахаза, царя Иудейского.
\par 2 Двадцати пяти лет был он, когда воцарился, и двадцать девять лет царствовал в Иерусалиме; имя матери его Ави, дочь Захарии.
\par 3 И делал он угодное в очах Господних во всем так, как делал Давид, отец его;
\par 4 он отменил высоты, разбил статуи, срубил дубраву и истребил медного змея, которого сделал Моисей, потому что до самых тех дней сыны Израилевы кадили ему и называли его Нехуштан.
\par 5 На Господа Бога Израилева уповал он; и такого, как он, не бывало между всеми царями Иудейскими и после него и прежде него.
\par 6 И прилепился он к Господу и не отступал от Него, и соблюдал заповеди Его, какие заповедал Господь Моисею.
\par 7 И был Господь с ним: везде, куда он ни ходил, поступал он благоразумно. И отложился он от царя Ассирийского, и не стал служить ему.
\par 8 Он поразил Филистимлян до Газы и в пределах ее, от сторожевой башни до укрепленного города.
\par 9 В четвертый год царя Езекии, то есть в седьмой год Осии, сына Илы, царя Израильского, пошел Салманассар, царь Ассирийский, на Самарию, и осадил ее,
\par 10 и взял ее через три года; в шестой год Езекии, то есть в девятый год Осии, царя Израильского, взята Самария.
\par 11 И переселил царь Ассирийский Израильтян в Ассирию, и поселил их в Халахе и в Хаворе, при реке Гозан, и в городах Мидийских,
\par 12 за то, что они не слушали гласа Господа Бога своего и преступили завет Его, все, что заповедал Моисей раб Господень, они и не слушали и не исполняли.
\par 13 В четырнадцатый год царя Езекии, пошел Сеннахирим, царь Ассирийский, против всех укрепленных городов Иуды и взял их.
\par 14 И послал Езекия, царь Иудейский, к царю Ассирийскому в Лахис сказать: виновен я; отойди от меня; что наложишь на меня, я внесу. И наложил царь Ассирийский на Езекию, царя Иудейского, триста талантов серебра и тридцать талантов золота.
\par 15 И отдал Езекия все серебро, какое нашлось в доме Господнем и в сокровищницах дома царского.
\par 16 В то время снял Езекия [золото] с дверей дома Господня и с дверных столбов, которые позолотил Езекия, царь Иудейский, и отдал его царю Ассирийскому.
\par 17 И послал царь Ассирийский Тартана и Рабсариса и Рабсака из Лахиса к царю Езекии с большим войском в Иерусалим. И пошли, и пришли к Иерусалиму; и пошли, и пришли, и стали у водопровода верхнего пруда, который на дороге поля белильничьего.
\par 18 И звали они царя. И вышел к ним Елиаким, сын Хелкиин, начальник дворца, и Севна писец, и Иоах, сын Асафов, дееписатель.
\par 19 И сказал им Рабсак: скажите Езекии: так говорит царь великий, царь Ассирийский: что это за упование, на которое ты уповаешь?
\par 20 Ты говорил только пустые слова: для войны нужны совет и сила. Ныне же на кого ты уповаешь, что отложился от меня?
\par 21 Вот, ты думаешь опереться на Египет, на эту трость надломленную, которая, если кто опрется на нее, войдет ему в руку и проколет ее. Таков фараон, царь Египетский, для всех уповающих на него.
\par 22 А если вы скажете мне: `на Господа Бога нашего мы уповаем', то на того ли, которого высоты и жертвенники отменил Езекия, и сказал Иуде и Иерусалиму: `пред сим только жертвенником поклоняйтесь в Иерусалиме'?
\par 23 Итак вступи в союз с господином моим царем Ассирийским: я дам тебе две тысячи коней, можешь ли достать себе всадников на них?
\par 24 Как тебе одолеть и одного вождя из малейших слуг господина моего? И уповаешь на Египет ради колесниц и коней?
\par 25 Притом же разве я без воли Господней пошел на место сие, чтобы разорить его? Господь сказал мне: `пойди на землю сию и разори ее'.
\par 26 И сказал Елиаким, сын Хелкиин, и Севна и Иоах Рабсаку: говори рабам твоим по-арамейски, потому что понимаем мы, а не говори с нами по-иудейски вслух народа, который на стене.
\par 27 И сказал им Рабсак: разве [только] к господину твоему и к тебе послал меня господин мой сказать сии слова? Нет, также и к людям, которые сидят на стене, чтобы есть помет свой и пить мочу свою с вами.
\par 28 И встал Рабсак и возгласил громким голосом по-иудейски, и говорил, и сказал: слушайте слово царя великого, царя Ассирийского!
\par 29 Так говорит царь: пусть не обольщает вас Езекия, ибо он не может вас спасти от руки моей;
\par 30 и пусть не обнадеживает вас Езекия Господом, говоря: `спасет нас Господь и не будет город сей отдан в руки царя Ассирийского'.
\par 31 Не слушайте Езекии. Ибо так говорит царь Ассирийский: примиритесь со мною и выйдите ко мне, и пусть каждый ест [плоды] виноградной лозы своей и смоковницы своей, и пусть каждый пьет воду из своего колодезя,
\par 32 пока я не приду и не возьму вас в землю такую же, как и ваша земля, в землю хлеба и вина, в землю плодов и виноградников, в землю масличных дерев и меда, и будете жить, и не умрете. Не слушайте же Езекии, который обольщает вас, говоря: `Господь спасет нас'.
\par 33 Спасли ли боги народов, каждый свою землю, от руки царя Ассирийского?
\par 34 Где боги Емафа и Арпада? Где боги Сепарваима, Ены и Иввы? Спасли ли они Самарию от руки моей?
\par 35 Кто из всех богов земель сих спас землю свою от руки моей? Так неужели Господь спасет Иерусалим от руки моей?
\par 36 И молчал народ и не отвечали ему ни слова, потому что было приказание царя: `не отвечайте ему'.
\par 37 И пришел Елиаким, сын Хелкиин, начальник дворца, и Севна писец и Иоах, сын Асафов, дееписатель, к Езекии в разодранных одеждах, и пересказали ему слова Рабсаковы.

\chapter{19}

\par 1 Когда услышал [это] царь Езекия, то разодрал одежды свои и покрылся вретищем, и пошел в дом Господень.
\par 2 И послал Елиакима, начальника дворца, и Севну писца, и старших священников, покрытых вретищами, к Исаии пророку, сыну Амосову.
\par 3 И они сказали ему: так говорит Езекия: день скорби и наказания и посрамления--день сей; ибо дошли младенцы до отверстия утробы матерней, а силы нет родить.
\par 4 Может быть, услышит Господь Бог твой все слова Рабсака, которого послал царь Ассирийский, господин его, хулить Бога живаго и поносить словами, какие слышал Господь Бог твой. Принеси же молитву об оставшихся, которые находятся еще в живых.
\par 5 И пришли слуги царя Езекии к Исаии,
\par 6 и сказал им Исаия: так скажите господину вашему: так говорит Господь: не бойся слов, которые ты слышал, которыми поносили Меня слуги царя Ассирийского.
\par 7 Вот Я пошлю в него дух, и он услышит весть, и возвратится в землю свою, и Я поражу его мечом в земле его.
\par 8 И возвратился Рабсак, и нашел царя Ассирийского воюющим против Ливны, ибо он слышал, что тот отошел от Лахиса.
\par 9 И услышал он о Тиргаке, царе Ефиопском; ему сказали: вот, он вышел сразиться с тобою. И снова послал он послов к Езекии сказать:
\par 10 так скажите Езекии, царю Иудейскому: пусть не обманывает тебя Бог твой, на Которого ты уповаешь, думая: `не будет отдан Иерусалим в руки царя Ассирийского'.
\par 11 Ведь ты слышал, что сделали цари Ассирийские со всеми землями, положив на них заклятие, --и ты ли уцелеешь?
\par 12 Боги народов, которых разорили отцы мои, спасли ли их? [Спасли] [ли] Гозан, и Харан, и Рецеф, и сынов Едена, что в Фалассаре?
\par 13 Где царь Емафа, и царь Арпада, и царь города Сепарваима, Ены и Иввы?
\par 14 И взял Езекия письмо из руки послов, и прочитал его, и пошел в дом Господень, и развернул его Езекия пред лицем Господним,
\par 15 и молился Езекия пред лицем Господним и говорил: Господи Боже Израилев, седящий на Херувимах! Ты один Бог всех царств земли, Ты сотворил небо и землю.
\par 16 Приклони, Господи, ухо Твое и услышь; открой, Господи, очи Твои и воззри, и услышь слова Сеннахирима, который послал поносить Бога живаго!
\par 17 Правда, о, Господи, цари Ассирийские разорили народы и земли их,
\par 18 и побросали богов их в огонь; но это не боги, а изделие рук человеческих, дерево и камень; потому и истребили их.
\par 19 И ныне, Господи Боже наш, спаси нас от руки его, и узнают все царства земли, что Ты, Господи, Бог один.
\par 20 И послал Исаия, сын Амосов, к Езекии сказать: так говорит Господь Бог Израилев: то, о чем ты молился Мне против Сеннахирима, царя Ассирийского, Я услышал.
\par 21 Вот слово, которое изрек Господь о нем: презрит тебя, посмеется над тобою девствующая дочь Сиона; вслед тебя покачает головою дочь Иерусалима.
\par 22 Кого ты порицал и поносил? И на кого ты возвысил голос и поднял так высоко глаза свои? На Святаго Израилева!
\par 23 Чрез послов твоих ты порицал Господа и сказал: `со множеством колесниц моих я взошел на высоту гор, на ребра Ливана, и срубил рослые кедры его, отличные кипарисы его, и пришел на самое крайнее пристанище его, в рощу сада его;
\par 24 и откапывал я и пил воду чужую, и осушу ступнями ног моих все реки Египетские'.
\par 25 Разве ты не слышал, что Я издавна сделал это, в древние дни предначертал это, а ныне выполнил тем, что ты опустошаешь укрепленные города, [превращая] в груды развалин?
\par 26 И жители их сделались маломощны, трепещут и остаются в стыде. Они стали [как] трава на поле и нежная зелень, [как] порост на кровлях и опаленный хлеб, прежде нежели выколосился.
\par 27 Сядешь ли ты, выйдешь ли, войдешь ли, Я все знаю; [знаю] и дерзость твою против Меня.
\par 28 За твою дерзость против Меня и [за то, что] надмение твое дошло до ушей Моих, Я вложу кольцо Мое в ноздри твои и удила Мои в рот твой, и возвращу тебя назад тою же дорогою, которою пришел ты.
\par 29 И вот тебе, [Езекия], знамение: ешьте в этот год выросшее от упавшего зерна, и в другой год--самородное, а на третий год сейте и жните, и садите виноградные сады и ешьте плоды их.
\par 30 И уцелевшее в доме Иудином, оставшееся пустит опять корень внизу и принесет плод вверху,
\par 31 ибо из Иерусалима произойдет остаток, и спасенное от горы Сиона. Ревность Господа Саваофа сделает сие.
\par 32 Посему так говорит Господь о царе Ассирийском: `не войдет он в сей город, и не бросит туда стрелы, и не приступит к нему со щитом, и не насыплет против него вала.
\par 33 Тою же дорогою, которою пришел, возвратится, и в город сей не войдет, говорит Господь.
\par 34 Я буду охранять город сей, чтобы спасти его ради Себя и ради Давида, раба Моего'.
\par 35 И случилось в ту ночь: пошел Ангел Господень и поразил в стане Ассирийском сто восемьдесят пять тысяч. И встали поутру, и вот все тела мертвые.
\par 36 И отправился, и пошел, и возвратился Сеннахирим, царь Ассирийский, и жил в Ниневии.
\par 37 И когда он поклонялся в доме Нисроха, бога своего, то Адрамелех и Шарецер, сыновья его, убили его мечом, а сами убежали в землю Араратскую. И воцарился Асардан, сын его, вместо него.

\chapter{20}

\par 1 В те дни заболел Езекия смертельно, и пришел к нему Исаия, сын Амосов, пророк, и сказал ему: так говорит Господь: сделай завещание для дома твоего, ибо умрешь ты и не выздоровеешь.
\par 2 И отворотился [Езекия] лицем своим к стене и молился Господу, говоря:
\par 3 `О, Господи! вспомни, что я ходил пред лицем Твоим верно и с преданным [Тебе] сердцем, и делал угодное в очах Твоих'. И заплакал Езекия сильно.
\par 4 Исаия еще не вышел из города, как было к нему слово Господне:
\par 5 возвратись и скажи Езекии, владыке народа Моего: так говорит Господь Бог Давида, отца твоего: Я услышал молитву твою, увидел слезы твои. Вот, Я исцелю тебя; в третий день пойдешь в дом Господень;
\par 6 и прибавлю ко дням твоим пятнадцать лет, и от руки царя Ассирийского спасу тебя и город сей, и защищу город сей ради Себя и ради Давида, раба Моего.
\par 7 И сказал Исаия: возьмите пласт смокв. И взяли, и приложили к нарыву; и он выздоровел.
\par 8 И сказал Езекия Исаии: какое знамение, что Господь исцелит меня, и что пойду я на третий день в дом Господень?
\par 9 И сказал Исаия: вот тебе знамение от Господа, что исполнит Господь слово, которое Он изрек: вперед ли пройти тени на десять ступеней, или воротиться на десять ступеней?
\par 10 И сказал Езекия: легко тени подвинуться вперед на десять ступеней; нет, пусть воротится тень назад на десять ступеней.
\par 11 И воззвал Исаия пророк к Господу, и возвратил тень назад на ступенях, где она спускалась по ступеням Ахазовым, на десять ступеней.
\par 12 В то время послал Беродах Баладан, сын Баладана, царь Вавилонский, письма и подарок Езекии, ибо он слышал, что Езекия был болен.
\par 13 Езекия, выслушав посланных, показал им кладовые свои, серебро и золото, и ароматы, и масти дорогие, и весь оружейный дом свой и все, что находилось в сокровищницах его; не оставалось ни одной вещи, которой не показал бы им Езекия в доме своем и во всем владении своем.
\par 14 И пришел Исаия пророк к царю Езекии и сказал ему: что говорили эти люди, и откуда они приходили к тебе? И сказал Езекия: из земли далекой они приходили, из Вавилона.
\par 15 И сказал [Исаия]: что они видели в доме твоем? И сказал Езекия: все, что в доме моем, они видели, не осталось ни одной вещи, которой я не показал бы им в сокровищницах моих.
\par 16 И сказал Исаия Езекии: выслушай слово Господне:
\par 17 вот придут дни, и взято будет все, что в доме твоем, и что собрали отцы твои до сего дня, в Вавилон; ничего не останется, говорит Господь.
\par 18 Из сынов твоих, которые произойдут от тебя, которых ты родишь, возьмут, и будут они евнухами во дворце царя Вавилонского.
\par 19 И сказал Езекия Исаии: благо слово Господне, которое ты изрек. И продолжал: да будет мир и благосостояние во дни мои!
\par 20 Прочее об Езекии и о всех подвигах его, и о том, что он сделал пруд и водопровод и провел воду в город, написано в летописи царей Иудейских.
\par 21 И почил Езекия с отцами своими, и воцарился Манассия, сын его, вместо него.

\chapter{21}

\par 1 Двенадцати лет был Манассия, когда воцарился, и пятьдесят лет царствовал в Иерусалиме; имя матери его Хефциба.
\par 2 И делал он неугодное в очах Господних, [подражая] мерзостям народов, которых прогнал Господь от лица сынов Израилевых.
\par 3 И снова устроил высоты, которые уничтожил отец его Езекия, и поставил жертвенники Ваалу, и сделал дубраву, как сделал Ахав, царь Израильский; и поклонялся всему воинству небесному, и служил ему.
\par 4 И соорудил жертвенники в доме Господнем, о котором сказал Господь: `в Иерусалиме положу имя Мое'.
\par 5 И соорудил жертвенники всему воинству небесному на обоих дворах дома Господня,
\par 6 и провел сына своего чрез огонь, и гадал, и ворожил, и завел вызывателей мертвецов и волшебников; много сделал неугодного в очах Господа, чтобы прогневать Его.
\par 7 И поставил истукан Астарты, который сделал в доме, о котором говорил Господь Давиду и Соломону, сыну его: `в доме сем и в Иерусалиме, который Я избрал из всех колен Израилевых, Я полагаю имя Мое на век;
\par 8 и не дам впредь выступить ноге Израильтянина из земли, которую Я дал отцам их, если только они будут стараться поступать согласно со всем тем, что Я повелел им, и со всем законом, который заповедал им раб Мой Моисей'.
\par 9 Но они не послушались; и совратил их Манассия до того, что они поступали хуже тех народов, которых истребил Господь от лица сынов Израилевых.
\par 10 И говорил Господь чрез рабов Своих пророков и сказал:
\par 11 за то, что сделал Манассия, царь Иудейский, такие мерзости, хуже всего того, что делали Аморреи, которые были прежде его, и ввел Иуду в грех идолами своими,
\par 12 за то, так говорит Господь, Бог Израилев, вот, Я наведу такое зло на Иерусалим и на Иуду, о котором кто услышит, зазвенит в обоих ушах у того;
\par 13 и протяну на Иерусалим мерную вервь Самарии и отвес дома Ахавова, и вытру Иерусалим так, как вытирают чашу, --вытрут и опрокинут ее;
\par 14 и отвергну остаток удела Моего, и отдам их в руку врагов их, и будут на расхищение и разграбление всем неприятелям своим,
\par 15 за то, что они делали неугодное в очах Моих и прогневляли Меня с того дня, как вышли отцы их из Египта, и до сего дня.
\par 16 Еще же пролил Манассия и весьма много невинной крови, так что наполнил [ею] Иерусалим от края до края, сверх своего греха, что он завлек Иуду в грех--делать неугодное в очах Господних.
\par 17 Прочее о Манассии и обо всем, что он сделал, и о грехах его, в чем он согрешил, написано в летописи царей Иудейских.
\par 18 И почил Манассия с отцами своими, и погребен в саду при доме его, в саду Уззы. И воцарился Аммон, сын его, вместо него.
\par 19 Двадцати двух лет был Аммон, когда воцарился, и два года царствовал в Иерусалиме; имя матери его Мешуллемеф, дочь Харуца, из Ятбы.
\par 20 И делал он неугодное в очах Господних так, как делал Манассия, отец его;
\par 21 и ходил тою же точно дорогою, которою ходил отец его, и служил идолам, которым служил отец его, и поклонялся им,
\par 22 и оставил Господа Бога отцов своих, не ходил путем Господним.
\par 23 И составили заговор слуги Аммоновы против него, и умертвили царя в доме его.
\par 24 Но народ земли перебил всех, бывших в заговоре против царя Аммона; и воцарил народ земли Иосию, сына его, вместо него.
\par 25 Прочее об Аммоне, что он сделал, написано в летописи царей Иудейских.
\par 26 И похоронили его в гробнице его, в саду Уззы. И воцарился Иосия, сын его, вместо него.

\chapter{22}

\par 1 Восьми лет был Иосия, когда воцарился, и тридцать один год царствовал в Иерусалиме; имя матери его Иедида, дочь Адаии, из Боцкафы.
\par 2 И делал он угодное в очах Господних, и ходил во всем путем Давида, отца своего, и не уклонялся ни направо, ни налево.
\par 3 В восемнадцатый год царя Иосии, послал царь Шафана, сына Ацалии, сына Мешулламова, писца, в дом Господень, сказав:
\par 4 пойди к Хелкии первосвященнику, пусть он пересчитает серебро, принесенное в дом Господень, которое собрали от народа стоящие на страже у порога,
\par 5 и пусть отдадут его в руки производителям работ, приставленным к дому Господню, а сии пусть раздают его работающим в доме Господнем, на исправление повреждений дома,
\par 6 плотникам и каменщикам, и делателям стен, и на покупку дерев и тесаных камней для исправления дома;
\par 7 впрочем не требовать у них отчета в серебре, переданном в руки их, потому что они поступают честно.
\par 8 И сказал Хелкия первосвященник Шафану писцу: книгу закона я нашел в доме Господнем. И подал Хелкия книгу Шафану, и он читал ее.
\par 9 И пришел Шафан писец к царю, и принес царю ответ, и сказал: взяли рабы твои серебро, найденное в доме, и передали его в руки производителям работ, приставленным к дому Господню.
\par 10 И донес Шафан писец царю, говоря: книгу дал мне Хелкия священник. И читал ее Шафан пред царем.
\par 11 Когда услышал царь слова книги закона, то разодрал одежды свои.
\par 12 И повелел царь Хелкии священнику, и Ахикаму, сыну Шафанову, и Ахбору, сыну Михеину, и Шафану писцу, и Асаии, слуге царскому, говоря:
\par 13 пойдите, вопросите Господа за меня и за народ и за всю Иудею о словах сей найденной книги, потому что велик гнев Господень, который воспылал на нас за то, что не слушали отцы наши слов книги сей, чтобы поступать согласно с предписанным нам.
\par 14 И пошел Хелкия священник, и Ахикам, и Ахбор, и Шафан, и Асаия к Олдаме пророчице, жене Шаллума, сына Тиквы, сына Хархаса, хранителя одежд, --жила же она в Иерусалиме, во второй части, --и говорили с нею.
\par 15 И она сказала им: так говорит Господь, Бог Израилев: скажите человеку, который послал вас ко мне:
\par 16 так говорит Господь: наведу зло на место сие и на жителей его, --все слова книги, которую читал царь Иудейский.
\par 17 За то, что оставили Меня, и кадят другим богам, чтобы раздражать Меня всеми делами рук своих, воспылал гнев Мой на место сие, и не погаснет.
\par 18 А царю Иудейскому, пославшему вас вопросить Господа, скажите: так говорит Господь Бог Израилев, о словах, которые ты слышал:
\par 19 так как смягчилось сердце твое, и ты смирился пред Господом, услышав то, что Я изрек на место сие и на жителей его, что они будут предметом ужаса и проклятия, и ты разодрал одежды свои, и плакал предо Мною, то и Я услышал тебя, говорит Господь.
\par 20 За это, вот, Я приложу тебя к отцам твоим, и ты положен будешь в гробницу твою в мире, и не увидят глаза твои всего того бедствия, которое Я наведу на место сие. И принесли царю ответ.

\chapter{23}

\par 1 И послал царь, и собрали к нему всех старейшин Иуды и Иерусалима.
\par 2 И пошел царь в дом Господень, и все Иудеи, и все жители Иерусалима с ним, и священники, и пророки, и весь народ, от малого до большого, и прочел вслух их все слова книги завета, найденной в доме Господнем.
\par 3 Потом стал царь на возвышенное место и заключил пред лицем Господним завет--последовать Господу и соблюдать заповеди Его и откровения Его и уставы Его от всего сердца и от всей души, чтобы выполнить слова завета сего, написанные в книге сей. И весь народ вступил в завет.
\par 4 И повелел царь Хелкии первосвященнику и вторым священникам и стоящим на страже у порога вынести из храма Господня все вещи, сделанные для Ваала и для Астарты и для всего воинства небесного, и сжег их за Иерусалимом в долине Кедрон, и [велел] прах их отнести в Вефиль.
\par 5 И отставил жрецов, которых поставили цари Иудейские, чтобы совершать курения на высотах в городах Иудейских и окрестностях Иерусалима, --и которые кадили Ваалу, солнцу, и луне, и созвездиям, и всему воинству небесному;
\par 6 и вынес Астарту из дома Господня за Иерусалим к потоку Кедрону, и сжег ее у потока Кедрона, и истер ее в прах, и бросил прах ее на кладбище общенародное;
\par 7 и разрушил домы блудилищные, которые [были] при храме Господнем, где женщины ткали одежды для Астарты;
\par 8 и вывел всех жрецов из городов Иудейских, и осквернил высоты, на которых совершали курения жрецы, от Гевы до Вирсавии, и разрушил высоты [пред] воротами, --ту, которая у входа в ворота Иисуса градоначальника, и ту, которая на левой стороне у городских ворот.
\par 9 Впрочем жрецы высот не приносили жертв на жертвеннике Господнем в Иерусалиме, опресноки же ели вместе с братьями своими.
\par 10 И осквернил он Тофет, что на долине сыновей Еннома, чтобы никто не проводил сына своего и дочери своей чрез огонь Молоху;
\par 11 и отменил коней, которых ставили цари Иудейские солнцу пред входом в дом Господень близ комнат Нефан-Мелеха евнуха, что в Фаруриме, колесницы же солнца сжег огнем.
\par 12 И жертвенники на кровле горницы Ахазовой, которые сделали цари Иудейские, и жертвенники, которые сделал Манассия на обоих дворах дома Господня, разрушил царь, и низверг оттуда, и бросил прах их в поток Кедрон.
\par 13 И высоты, которые пред Иерусалимом, направо от Масличной горы, которые устроил Соломон, царь Израилев, Астарте, мерзости Сидонской, и Хамосу, мерзости Моавитской, и Милхому, мерзости Аммонитской, осквернил царь;
\par 14 и изломал статуи, и срубил дубравы, и наполнил место их костями человеческими.
\par 15 Также и жертвенник, который в Вефиле, высоту, устроенную Иеровоамом, сыном Наватовым, который ввел Израиля в грех, --также и жертвенник тот и высоту он разрушил, и сжег сию высоту, стер в прах, и сжег дубраву.
\par 16 И взглянул Иосия и увидел могилы, которые [были] там на горе, и послал и взял кости из могил, и сжег на жертвеннике, и осквернил его по слову Господню, которое провозгласил человек Божий, предрекший события сии.
\par 17 и сказал [Иосия]: что это за памятник, который я вижу? И сказали ему жители города: [это] могила человека Божия, который приходил из Иудеи и провозгласил о том, что ты делаешь над жертвенником Вефильским.
\par 18 И сказал он: оставьте его в покое, никто не трогай костей его. И сохранили кости его и кости пророка, который приходил из Самарии.
\par 19 Также и все капища высот в городах Самарийских, которые построили цари Израильские, прогневляя [Господа], разрушил Иосия, и сделал с ними то же, что сделал в Вефиле;
\par 20 и заколол всех жрецов высот, которые там были, на жертвенниках, и сожег кости человеческие на них, --и возвратился в Иерусалим.
\par 21 И повелел царь всему народу, сказав: `совершите пасху Господу Богу вашему, как написано в сей книге завета', --
\par 22 потому что не была совершена такая пасха от дней судей, которые судили Израиля, и во все дни царей Израильских и царей Иудейских;
\par 23 а в восемнадцатый год царя Иосии была совершена сия пасха Господу в Иерусалиме.
\par 24 И вызывателей мертвых, и волшебников, и терафимов, и идолов, и все мерзости, которые появлялись в земле Иудейской и в Иерусалиме, истребил Иосия, чтоб исполнить слова закона, написанные в книге, которую нашел Хелкия священник в доме Господнем.
\par 25 Подобного ему не было царя прежде его, который обратился бы к Господу всем сердцем своим, и всею душею своею, и всеми силами своими, по всему закону Моисееву; и после него не восстал подобный ему.
\par 26 Однакож Господь не отложил великой ярости гнева Своего, какою воспылал гнев Его на Иуду за все оскорбления, какими прогневал Его Манассия.
\par 27 И сказал Господь: и Иуду отрину от лица Моего, как отринул Я Израиля, и отвергну город сей Иерусалим, который Я избрал, и дом, о котором Я сказал: `будет имя Мое там'.
\par 28 Прочее об Иосии и обо всем, что он сделал, написано в летописи царей Иудейских.
\par 29 Во дни его пошел фараон Нехао, царь Египетский, против царя Ассирийского на реку Евфрат. И вышел царь Иосия навстречу ему, и тот умертвил его в Мегиддоне, когда увидел его.
\par 30 И рабы его повезли его мертвого из Мегиддона, и привезли его в Иерусалим, и похоронили его в гробнице его. И взял народ земли Иоахаза, сына Иосиина, и помазали его и воцарили его вместо отца его.
\par 31 Двадцати трех лет был Иоахаз, когда воцарился, и три месяца царствовал в Иерусалиме; имя матери его Хамуталь, дочь Иеремии, из Ливны.
\par 32 И делал он неугодное в очах Господних во всем так, как делали отцы его.
\par 33 И задержал его фараон Нехао в Ривле, в земле Емафской, чтобы он не царствовал в Иерусалиме, --и наложил пени на землю сто талантов серебра и талантов золота.
\par 34 И воцарил фараон Нехао Елиакима, сына Иосиина, вместо Иосии, отца его, и переменил имя его на Иоакима; Иоахаза же взял и отвел в Египет, где он и умер.
\par 35 И серебро и золото давал Иоаким фараону; он сделал оценку земле, чтобы давать серебро по приказанию фараона; от каждого из народа земли, по оценке своей, он взыскивал серебро и золото для того, чтобы отдавать фараону Нехао.
\par 36 Двадцати пяти лет был Иоаким, когда воцарился, и одиннадцать лет царствовал в Иерусалиме; имя матери его Зебудда, дочь Федаии, из Румы.
\par 37 И делал он неугодное в очах Господних во всем так, как делали отцы его.

\chapter{24}

\par 1 Во дни его выступил Навуходоносор, царь Вавилонский, и сделался Иоаким подвластным ему на три года, но потом отложился от него.
\par 2 И посылал на него Господь полчища Халдеев, и полчища Сириян, и полчища Моавитян, и полчища Аммонитян, --посылал их на Иуду, чтобы погубить его по слову Господа, которое Он изрек чрез рабов Своих пророков.
\par 3 По повелению Господа было [это] с Иудою, чтобы отвергнуть [его] от лица Его за грехи Манассии, за все, что он сделал;
\par 4 и за кровь невинную, которую он пролил, наполнив Иерусалим кровью невинною, Господь не захотел простить.
\par 5 Прочее об Иоакиме и обо всем, что он сделал, написано в летописи царей Иудейских.
\par 6 И почил Иоаким с отцами своими, и воцарился Иехония, сын его, вместо него.
\par 7 Царь Египетский не выходил более из земли своей, потому что взял царь Вавилонский все, от потока Египетского до реки Евфрата, что принадлежало царю Египетскому.
\par 8 Восемнадцати лет был Иехония, когда воцарился, и три месяца царствовал в Иерусалиме; имя матери его Нехушта, дочь Елнафана, из Иерусалима.
\par 9 И делал он неугодное в очах Господних во всем так, как делал отец его.
\par 10 В то время подступили рабы Навуходоносора, царя Вавилонского, к Иерусалиму, и подвергся город осаде.
\par 11 И пришел Навуходоносор, царь Вавилонский, к городу, когда рабы его осаждали его.
\par 12 И вышел Иехония, царь Иудейский, к царю Вавилонскому, он и мать его, и слуги его, и князья его, и евнухи его, --и взял его царь Вавилонский в восьмой год своего царствования.
\par 13 И вывез он оттуда все сокровища дома Господня и сокровища царского дома; и изломал, как изрек Господь, все золотые сосуды, которые Соломон, царь Израилев, сделал в храме Господнем;
\par 14 и выселил весь Иерусалим, и всех князей, и все храброе войско, --десять тысяч было переселенных, --и всех плотников и кузнецов; никого не осталось, кроме бедного народа земли.
\par 15 И переселил он Иехонию в Вавилон; и мать царя, и жен царя, и евнухов его, и сильных земли отвел на поселение из Иерусалима в Вавилон.
\par 16 И все войско [числом] семь тысяч, и художников и строителей тысячу, всех храбрых, ходящих на войну, отвел царь Вавилонский на поселение в Вавилон.
\par 17 И воцарил царь Вавилонский Матфанию, дядю [Иехонии], вместо него, и переменил имя его на Седекию.
\par 18 Двадцати одного года был Седекия, когда воцарился, и одиннадцать лет царствовал в Иерусалиме; имя матери его Хамуталь, дочь Иеремии, из Ливны.
\par 19 И делал он неугодное в очах Господних во всем так, как делал Иоаким.
\par 20 Гнев Господень был над Иерусалимом и над Иудою до того, что Он отверг их от лица Своего. И отложился Седекия от царя Вавилонского.

\chapter{25}

\par 1 В девятый год царствования своего, в десятый месяц, в десятый день месяца, пришел Навуходоносор, царь Вавилонский, со всем войском своим к Иерусалиму, и осадил его, и устроил вокруг него вал.
\par 2 И находился город в осаде до одиннадцатого года царя Седекии.
\par 3 В девятый день месяца усилился голод в городе, и не было хлеба у народа земли.
\par 4 И взят был город, и [побежали] все военные ночью по дороге к воротам, между двумя стенами, что подле царского сада; Халдеи же стояли вокруг города, и [царь] ушел дорогою к равнине.
\par 5 И погналось войско Халдейское за царем, и настигли его на равнинах Иерихонских, и все войско его разбежалось от него.
\par 6 И взяли царя, и отвели его к царю Вавилонскому в Ривлу, и произвели над ним суд:
\par 7 и сыновей Седекии закололи пред глазами его, а [самому] Седекии ослепили глаза и сковали его оковами, и отвели его в Вавилон.
\par 8 В пятый месяц, в седьмой день месяца, то есть в девятнадцатый год Навуходоносора, царя Вавилонского, пришел Навузардан, начальник телохранителей, слуга царя Вавилонского, в Иерусалим
\par 9 и сжег дом Господень и дом царя, и все домы в Иерусалиме, и все домы большие сожег огнем;
\par 10 и стены вокруг Иерусалима разрушило войско Халдейское, бывшее у начальника телохранителей.
\par 11 И прочий народ, остававшийся в городе, и переметчиков, которые передались царю Вавилонскому, и прочий простой народ выселил Навузардан, начальник телохранителей.
\par 12 Только несколько из бедного народа земли оставил начальник телохранителей работниками в виноградниках и землепашцами.
\par 13 И столбы медные, которые были у дома Господня, и подставы, и море медное, которое в доме Господнем, изломали Халдеи, и отнесли медь их в Вавилон;
\par 14 и тазы, и лопатки, и ножи, и ложки, и все сосуды медные, которые употреблялись при служении, взяли;
\par 15 и кадильницы, и чаши, что было золотое и что было серебряное, взял начальник телохранителей:
\par 16 столбы [числом] два, море одно, и подставы, которые сделал Соломон в дом Господень, --меди во всех сих вещах не было весу.
\par 17 Восемнадцать локтей вышины в одном столбе; венец на нем медный, а вышина венца три локтя, и сетка и гранатовые яблоки вокруг венца--все из меди. То же и на другом столбе с сеткою.
\par 18 И взял начальник телохранителей Сераию первосвященника и Цефанию, священника второго, и трех, стоявших на страже у порога.
\par 19 И из города взял одного евнуха, который был начальствующим над людьми военными, и пять человек, предстоявших лицу царя, которые находились в городе, и писца главного в войске, записывавшего в войско народ земли, и шестьдесят человек из народа земли, находившихся в городе.
\par 20 И взял их Навузардан, начальник телохранителей, и отвел их к царю Вавилонскому в Ривлу.
\par 21 И поразил их царь Вавилонский, и умертвил их в Ривле, в земле Емаф. И выселены Иудеи из земли своей.
\par 22 Над народом же, остававшимся в земле Иудейской, который оставил Навуходоносор, царь Вавилонский, --над ними поставил начальником Годолию, сына Ахикама, сына Шафанова.
\par 23 Когда услышали все военачальники, они и люди их, что царь Вавилонский поставил начальником Годолию, то пришли к Годолии в Массифу, и [именно]: Исмаил, сын Нефании, и Иоханан, сын Карея, и Сераия, сын Танхумефа из Нетофафа, и Иезания, сын Маахитянина, они и люди их.
\par 24 И поклялся Годолия им и людям их, и сказал им: не бойтесь быть подвластными Халдеям, селитесь на земле и служите царю Вавилонскому, и будет хорошо вам.
\par 25 Но в седьмой месяц пришел Исмаил, сын Нефании, сына Елишамы, из племени царского, с десятью человеками, и поразил Годолию, и он умер, и Иудеев и Халдеев, которые были с ним в Массифе.
\par 26 И встал весь народ, от малого до большого, и военачальники, и пошли в Египет, потому что боялись Халдеев.
\par 27 В тридцать седьмой год переселения Иехонии, царя Иудейского, в двенадцатый месяц, в двадцать седьмой день месяца, Евилмеродах, царь Вавилонский, в год своего воцарения, вывел Иехонию, царя Иудейского, из дома темничного
\par 28 и говорил с ним дружелюбно, и поставил престол его выше престола царей, которые были у него в Вавилоне;
\par 29 и переменил темничные одежды его, и он всегда имел пищу у него, во все дни жизни его.
\par 30 И содержание его, содержание постоянное, выдаваемо было ему от царя, изо дня в день, во все дни жизни его.


\end{document}