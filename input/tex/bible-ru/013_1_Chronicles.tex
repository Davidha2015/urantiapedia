\begin{document}

\title{1-я Паралипоменон}


\chapter{1}

\par 1 Адам, Сиф, Енос,
\par 2 Каинан, Малелеил, Иаред,
\par 3 Енох, Мафусал, Ламех,
\par 4 Ной, Сим, Хам и Иафет.
\par 5 Сыновья Иафета: Гомер, Магог, Мадай, Иаван, Фувал, Мешех и Фирас.
\par 6 Сыновья Гомера: Аскеназ, Рифат и Фогарма.
\par 7 Сыновья Иавана: Елиса, Фарсис, Киттим и Доданим.
\par 8 Сыновья Хама: Хуш, Мицраим, Фут и Ханаан.
\par 9 Сыновья Хуша: Сева, Хавила, Савта, Раама и Савтеха. Сыновья Раамы: Шева и Дедан.
\par 10 Хуш родил [также] Нимрода: сей начал быть сильным на земле.
\par 11 Мицраим родил: Лудима, Анамима, Легавима, Нафтухима,
\par 12 Патрусима, Каслухима, от которого произошли Филистимляне, и Кафторима.
\par 13 Ханаан родил Сидона, первенца своего, Хета,
\par 14 Иевусея, Аморрея, Гергесея,
\par 15 Евея, Аркея, Синея,
\par 16 Арвадея, Цемарея и Хамафея.
\par 17 Сыновья Сима: Елам, Ассур, Арфаксад, Луд, Арам, Уц, Хул, Гефер и Мешех.
\par 18 Арфаксад родил Салу, Сала же родил Евера.
\par 19 У Евера родились два сына: имя одному Фалек, потому что во дни его разделилась земля; имя брату его Иоктан.
\par 20 Иоктан родил Алмодада, Шалефа, Хацармавета, Иераха,
\par 21 Гадорама, Узала, Диклу,
\par 22 Евала, Авимаила, Шеву,
\par 23 Офира, Хавилу и Иовава. Все эти сыновья Иоктана.
\par 24 Сим, Арфаксад, Сала,
\par 25 Евер, Фалек, Рагав,
\par 26 Серух, Нахор, Фарра,
\par 27 Аврам, он же Авраам.
\par 28 Сыновья Авраама: Исаак и Измаил.
\par 29 Вот родословие их: первенец Измаилов Наваиоф, [за ним] Кедар, Адбеел, Мивсам,
\par 30 Мишма, Дума, Масса, Хадад, Фема,
\par 31 Иетур, Нафиш и Кедма. Это сыновья Измаиловы.
\par 32 Сыновья Хеттуры, наложницы Авраамовой: она родила Зимрана, Иокшана, Медана, Мадиана, Ишбака и Шуаха. Сыновья Иокшана: Шева и Дедан.
\par 33 Сыновья Мадиана: Ефа, Ефер, Ханох, Авида и Елдага. Все эти сыновья Хеттуры.
\par 34 И родил Авраам Исаака. Сыновья Исаака: Исав и Израиль.
\par 35 Сыновья Исава: Елифаз, Рагуил, Иеус, Иеглом и Корей.
\par 36 Сыновья Елифаза: Феман, Омар, Цефо, Гафам, Кеназ, Амалика.
\par 37 Сыновья Рагуила: Нахаф, Зерах, Шамма и Миза.
\par 38 Сыновья Сеира: Лотан, Шовал, Цивеон, Ана, Дишон, Ецер и Дишан.
\par 39 Сыновья Лотана: Хори и Гемам; а сестра у Лотана: Фимна.
\par 40 Сыновья Шовала: Алеан, Манахаф, Евал, Шефо и Онам. Сыновья Цивеона: Аиа и Ана.
\par 41 Дети Аны: Дишон. Сыновья Дишона: Хемдан, Ешбан, Ифран и Херан.
\par 42 Сыновья Ецера: Билган, Зааван и Акан. Сыновья Дишана: Уц и Аран.
\par 43 Сии суть цари, царствовавшие в земле Едома, прежде нежели воцарился царь над сынами Израилевыми: Бела, сын Веора, и имя городу его--Дингава;
\par 44 и умер Бела, и воцарился по нем Иовав, сын Зераха, из Восоры.
\par 45 И умер Иовав, и воцарился по нем Хушам, из земли Феманитян.
\par 46 И умер Хушам, и воцарился по нем Гадад, сын Бедадов, который поразил Мадианитян на поле Моава; имя городу его: Авив.
\par 47 И умер Гадад, и воцарился по нем Самла, из Масреки.
\par 48 И умер Самла, и воцарился по нем Саул из Реховофа, [что] при реке.
\par 49 И умер Саул, и воцарился по нем Баал-Ханан, сын Ахбора.
\par 50 И умер Баал-Ханан, и воцарился по нем Гадар; имя городу его Пау; имя жене его Мегетавеель, дочь Матреда, дочь Мезагава.
\par 51 И умер Гадар. И были старейшины у Едома: старейшина Фимна, старейшина Алва, старейшина Иетеф,
\par 52 старейшина Оливема, старейшина Эла, старейшина Пинон,
\par 53 старейшина Кеназ, старейшина Феман, старейшина Мивцар,
\par 54 старейшина Магдиил, старейшина Ирам. Вот старейшины Идумейские.

\chapter{2}

\par 1 Вот сыновья Израиля: Рувим, Симеон, Левий, Иуда, Иссахар, Завулон,
\par 2 Дан, Иосиф, Вениамин, Неффалим, Гад и Асир.
\par 3 Сыновья Иуды: Ир, Онан и Силом, --трое родились у него от дочери Шуевой, Хананеянки. И был Ир, первенец Иудин, не благоугоден в очах Господа, и Он умертвил его.
\par 4 И Фамарь, невестка его, родила ему Фареса и Зару. Всех сыновей у Иуды было пятеро.
\par 5 Сыновья Фареса: Есром и Хамул.
\par 6 Сыновья Зары: Зимри, Ефан, Еман, Халкол и Дара; всех их пятеро.
\par 7 Сыновья Харми: Ахар, наведший беду на Израиля, нарушив заклятие.
\par 8 Сын Ефана: Азария.
\par 9 Сыновья Есрома, которые родились у него: Иерахмеил, Арам и Хелувай.
\par 10 Арам же родил Аминадава; Аминадав родил Наассона, князя сынов Иудиных;
\par 11 Наассон родил Салмона, Салмон родил Вооза;
\par 12 Вооз родил Овида, Овид родил Иессея;
\par 13 Иессей родил первенца своего Елиава, второго--Аминадава, третьего--Самму,
\par 14 четвертого--Нафанаила, пятого--Раддая,
\par 15 шестого--Оцема, седьмого--Давида.
\par 16 Сестры их: Саруия и Авигея. Сыновья Саруии: Авесса, Иоав и Азаил, трое.
\par 17 Авигея родила Амессу; отец же Амессы--Иефер, Измаильтянин.
\par 18 Халев, сын Есрома, родил от Азувы, жены [своей], и от Иериофы, и вот сыновья его: Иешер, Шовав и Ардон.
\par 19 И умерла Азува; и взял себе Халев Ефрафу, и она родила ему Хура.
\par 20 Хур родил Урия, Урий родил Веселиила.
\par 21 После Есром вошел к дочери Махира, отца Галаадова, и взял ее, будучи шестидесяти лет, и она родила ему Сегува.
\par 22 Сегув родил Иаира, и было у него двадцать три города в земле Галаадской.
\par 23 Но Гессуряне и Сирияне взяли у них селения Иаира, Кенаф и зависящие от него города, --шестьдесят городов. Все эти города сыновей Махира, отца Галаадова.
\par 24 По смерти Есрома в Халев-Ефрафе, жена Есромова, Авия, родила ему Ашхура, отца Фекои.
\par 25 Сыновья Иерахмеила, первенца Есромова, были: первенец Рам, [за] [ним] Вуна, Орен, Оцем и Ахия.
\par 26 Была у Иерахмеила и другая жена, имя ее Афара; она мать Онама.
\par 27 Сыновья Рама, первенца Иерахмеилова, были: Маац, Иамин и Екер.
\par 28 Сыновья Онама были: Шаммай и Иада. Сыновья Шаммая: Надав и Авишур.
\par 29 Имя жене Авишуровой Авихаиль, и она родила ему Ахбана и Молида.
\par 30 Сыновья Надава: Селед и Афаим. И умер Селед бездетным.
\par 31 Сын Афаима: Иший. Сын Ишия: Шешан. Сын Шешана: Ахлай.
\par 32 Сыновья Иады, брата Шаммаева: Иефер и Ионафан. Иефер умер бездетным.
\par 33 Сыновья Ионафана: Пелеф и Заза. Это сыновья Иерахмеила.
\par 34 У Шешана не было сыновей, а только дочери. У Шешана [был] раб, Египтянин, имя его Иарха;
\par 35 Шешан отдал дочь свою Иархе, рабу своему, в жену: и она родила ему Аттая.
\par 36 Аттай родил Нафана, Нафан родил Завада;
\par 37 Завад родил Ефлала, Ефлал родил Овида;
\par 38 Овид родил Иеуя, Иеуй родил Азарию;
\par 39 Азария родил Хелеца, Хелец родил Елеасу;
\par 40 Елеаса родил Сисмая, Сисмай родил Саллума;
\par 41 Саллум родил Иекамию, Иекамия родил Елишаму.
\par 42 Сыновья Халева, брата Иерахмеилова: Меша, первенец его, --он отец Зифа; и сыновья Мареши, отца Хеврона.
\par 43 Сыновья Хеврона: Корей и Таппуах, и Рекем и Шема.
\par 44 Шема родил Рахама, отца Иоркеамова, а Рекем родил Шаммая.
\par 45 Сын Шаммая Маон, а Маон--отец Беф-Цура.
\par 46 И Ефа, наложница Халевова, родила Харана, Моцу и Газеза. И Харан родил Газеза.
\par 47 Сыновья Иегдая: Регем, Иофам, Гешан, Пелет, Ефа и Шааф.
\par 48 Наложница Халевова, Мааха, родила Шевера и Фирхану;
\par 49 она же родила Шаафа, отца Мадманны, Шеву, отца Махбены и отца Гивеи. Дочь же Халева--Ахса.
\par 50 Вот сыновья Халева: сын Хур, первенец Ефрафы; Шовал, отец Кириаф-Иарима;
\par 51 Салма, отец Вифлеема; Хареф, отец Бефгадера.
\par 52 У Шовала, отца Кириаф-Иарима, были сыновья: Гарое, Хаци, Галменюхот.
\par 53 Племена Кириаф-Иарима: Ифрияне, Футияне, Шумафане и Мидраитяне. От сих произошли Цоряне и Ештаоляне.
\par 54 Сыновья Салмы: Вифлеемляне и Нетофафяне, венец дома Иоавова и половина Менухотян--Цоряне,
\par 55 и племена Соферийцев, живших в Иабеце, Тирейцы, Шимейцы, Сухайцы: это Кинеяне, происшедшие от Хамафа, отца Бетрехава.

\chapter{3}

\par 1 Сыновья Давида, родившиеся у него в Хевроне, были: первенец Амнон, от Ахиноамы Изреелитянки; второй--Далуия, от Авигеи Кармилитянки;
\par 2 третий--Авессалом, сын Маахи, дочери Фалмая, царя Гессурского; четвертый--Адония, сын Аггифы;
\par 3 пятый--Сафатия, от Авиталы; шестой--Ифреам, от Аглаи, жены его, --
\par 4 шесть родившихся у него в Хевроне; царствовал же он там семь лет и шесть месяцев; а тридцать три года царствовал в Иерусалиме.
\par 5 А сии родились у него в Иерусалиме: Шима, Шовав, Нафан и Соломон, четверо от Вирсавии, дочери Аммииловой;
\par 6 Ивхар, Елишама, Елифелет,
\par 7 Ногаг, Нефег, Иафиа,
\par 8 Елишама, Елиада и Елифелет--девятеро.
\par 9 [Вот] все сыновья Давида, кроме сыновей от наложниц. Сестра их Фамарь.
\par 10 Сын Соломона Ровоам; его сын Авия, его сын Аса, его сын Иосафат,
\par 11 его сын Иорам, его сын Охозия, его сын Иоас,
\par 12 его сын Амасия, его сын Азария, его сын Иофам,
\par 13 его сын Ахаз, его сын Езекия, его сын Манассия,
\par 14 его сын Амон, его сын Иосия.
\par 15 Сыновья Иосии: первенец Иоахаз, второй Иоаким, третий Седекия, четвертый Селлум.
\par 16 Сыновья Иоакима: Иехония, сын его; Седекия, сын его.
\par 17 Сыновья Иехонии: Асир, Салафиил, сын его;
\par 18 Малкирам, Федаия, Шенацар, Иезекия, Гошама и Савадия.
\par 19 И сыновья Федаии: Зоровавель и Шимей. Сыновья же Зоровавеля: Мешуллам и Ханания, и Шеломиф, сестра их,
\par 20 и еще пять: Хашува, Огел, Берехия, Хасадия и Иушав-Хесед.
\par 21 И сыновья Ханании: Фелатия и Исаия; его сын Рефаия, его сын Арнан, его сын Овадия, его сын Шехания.
\par 22 Сын Шехании: Шемаия; сыновья Шемаии: Хаттуш, Игеал, Бариах, Неария и Шафат, шестеро.
\par 23 Сыновья Неарии: Елиоенай, Езекия и Азрикам, трое.
\par 24 Сыновья Елиоеная: Годавьягу, Елеашив, Фелаия, Аккув, Иоханан, Делаия и Анани, семеро.

\chapter{4}

\par 1 Сыновья Иуды: Фарес, Есром, Харми, Хур и Шовал.
\par 2 Реаия, сын Шовала, родил Иахафа; Иахаф родил Ахума и Лагада: от них племена Цорян.
\par 3 И сии сыновья Етама: Изреель, Ишма и Идбаш, и сестра их, по имени Гацлелпони,
\par 4 Пенуел, отец Гедора, и Езер, отец Хуша. Вот сыновья Хура, первенца Ефрафы, отца Вифлеема.
\par 5 У Ахшура, отца Фекои, были две жены: Хела и Наара.
\par 6 И родила ему Наара Ахузама, Хефера, Фимни и Ахашфари; это сыновья Наары.
\par 7 Сыновья Хелы: Цереф, Цохар и Ефнан.
\par 8 Коц родил: Анува и Цовева и племена Ахархела, сына Гарумова.
\par 9 Иавис был знаменитее своих братьев. Мать дала ему имя Иавис, сказав: я родила его с болезнью.
\par 10 И воззвал Иавис к Богу Израилеву [и] сказал: о, если бы Ты благословил меня Твоим благословением, распространил пределы мои, и рука Твоя была со мною, охраняя [меня] от зла, чтобы я не горевал!.. И Бог ниспослал [ему], чего он просил.
\par 11 Хелув же, брат Шухи, родил Махира; он есть отец Ештона.
\par 12 Ештон родил Беф-Рафу, Пасеаха и Техинну, отца города Нааса; это жители Рехи.
\par 13 Сыновья Кеназа: Гофониил и Сераия. Сын Гофониила: Хафаф.
\par 14 Меонофай родил Офру, а Сераия родил Иоава, родоначальника долины плотников, потому что они были плотники.
\par 15 Сыновья Халева, сына Иефонниина: Ир, Ила и Наам. Сын Илы: Кеназ.
\par 16 Сыновья Иегаллелела: Зиф, Зифа, Фирия и Асареел.
\par 17 Сыновья Езры: Иефер, Меред, Ефер и Иалон; Иефер же родил Мерома, Шаммая и Ишбаха, отца Ешфемои.
\par 18 И жена его Иудия родила Иереда, отца Гедора, и Хевера, отца Сохо, и Иекуфиила, отца Занаоха. Это сыновья Бифьи, дочери фараоновой, которую взял Меред.
\par 19 Сыновья жены его Годии, сестры Нахама, отца Кеилы: Гарми и Ешфемоа--Маахатянин.
\par 20 Сыновья Симеона: Амнон, Ринна, Бенханан и Филон. Сыновья Ишия: Зохеф и Бензохеф.
\par 21 Сыновья Силома, сына Иудина: Ир, отец Лехи, и Лаеда, отец Мареши, и семейства выделывавших виссон, из дома Ашбеи,
\par 22 и Иоким, и жители Хозевы, и Иоаш и Сараф, которые имели владение в Моаве, и Иашувилехем; но это события древние.
\par 23 Они [были] горшечники, и жили при садах и в огородах; у царя для работ его жили они там.
\par 24 Сыновья Симеона: Немуил, Иамин, Иарив, Зерах и Саул.
\par 25 Шаллум сын его; его сын Мивсам; его сын Мишма.
\par 26 Сыновья Мишмы: Хаммуил, сын его; его сын Закур; его сын Шимей.
\par 27 У Шимея [было] шестнадцать сыновей и шесть дочерей; у братьев же его сыновей [было] немного, и все племя их не так было многочисленно, как племя сынов Иуды.
\par 28 Они жили в Вирсавии, Моладе, Хацаршуале,
\par 29 в Билге, в Ецеме, в Фоладе,
\par 30 в Вефуиле, в Хорме, в Циклаге,
\par 31 в Беф-Маркавофе, в Хацарсусиме, в Беф-Биреи и в Шаариме. Вот города их до царствования Давидова,
\par 32 с селами их: Етам, Аин, Риммон, Фокен и Ашан, --пять городов.
\par 33 И все селения их, которые находились вокруг сих городов до Ваала; вот места жительства их и родословия их.
\par 34 Мешовав, Иамлех и Иосия, сын Амассии,
\par 35 Иоил и Иегу, сын Иошиви, сына Сераии, сына Асиилова,
\par 36 Елиоенай, Иакова, Ишохаия, Асаия, Адиил, Ишимиил и Ванея,
\par 37 и Зиза, сын Шифия, сын Аллона, сын Иедаии, сын Шимрия, сын Шемаии.
\par 38 Сии поименованные [были] князьями племен своих, и дом отцов их разделился на многие отрасли.
\par 39 Они доходили до Герары и до восточной стороны долины, чтобы найти пастбища для стад своих;
\par 40 и нашли пастбища тучные и хорошие и землю обширную, спокойную и безопасную, потому что до них жило там [только] немного Хамитян.
\par 41 И пришли сии, по именам записанные, во дни Езекии, царя Иудейского, и перебили кочующих и оседлых, которые там находились, и истребили их навсегда и поселились на месте их, ибо там были пастбища для стад их.
\par 42 Из них же, из сынов Симеоновых, пошли к горе Сеир пятьсот человек: Фелатия, Неария, Рефаия и Узиил, сыновья Ишия, [были] во главе их;
\par 43 и побили уцелевший там остаток Амаликитян, и живут там до сего дня.

\chapter{5}

\par 1 Сыновья Рувима, первенца Израилева, --он первенец; но, когда осквернил он постель отца своего, первенство его отдано сыновьям Иосифа, сына Израилева, с тем однакож, чтобы не писаться им первородными;
\par 2 потому что Иуда был сильнейшим из братьев своих, и вождь от него, но первенство [перенесено] на Иосифа.
\par 3 Сыновья Рувима, первенца Израилева: Ханох, Фаллу, Хецрон и Харми.
\par 4 Сыновья Иоиля: Шемая, сын его; его сын Гог, его сын Шимей,
\par 5 его сын Миха, его сын Реаия, его сын Ваал,
\par 6 его сын Беера, которого отвел в плен Феглафелласар, царь Ассирийский. Он [был] князем Рувимлян.
\par 7 И братья его, по племенам их, по родословному списку их, были: главный Иеиель, потом Захария,
\par 8 и Бела, сын Азаза, сына Шемы, сына Иоиля. Он обитал в Ароере до Нево и Ваал-Меона;
\par 9 а к востоку он обитал до входа в пустыню, идущую от реки Евфрата, потому что стада их были многочисленны в земле Галаадской.
\par 10 Во дни Саула они вели войну с Агарянами, которые пали от рук их, а они стали жить в шатрах и по всей восточной стороне Галаада.
\par 11 Сыновья Гада жили напротив их в земле Васанской до Салхи:
\par 12 в Васане Иоиль был главный, Шафан второй, потом Иаанай и Шафат.
\par 13 Братьев их с семействами их было семь: Михаил, Мешуллам, Шева, Иорай, Иаакан, Зия и Евер.
\par 14 Вот сыновья Авихаила, сына Хурия, сына Иароаха, сына Галаада, сына Михаила, сына Иешишая, сына Иахдо, сына Буза.
\par 15 Ахи, сын Авдиила, сына Гуниева, [был] главою своего рода.
\par 16 Они жили в Галааде, в Васане и в зависящих от него городах и во всех окрестностях Сарона, до исхода их.
\par 17 Все они перечислены во дни Иоафама, царя Иудейского, и во дни Иеровоама, царя Израильского.
\par 18 У потомков Рувима и Гада и полуплемени Манассиина было людей воинственных, мужей носящих щит и меч, стреляющих из лука и приученных к битве, сорок четыре тысячи семьсот шестьдесят, выходящих на войну.
\par 19 И воевали они с Агарянами, Иетуром, Нафишем и Надавом.
\par 20 И подана была им помощь против них, и преданы были в руки их Агаряне и все, что у них было, потому что они во время сражения воззвали к Богу, и Он услышал их, за то, что они уповали на Него.
\par 21 И взяли они стада их: верблюдов пятьдесят тысяч, из мелкого скота двести пятьдесят тысяч, ослов две тысячи, и сто тысяч душ людей,
\par 22 потому что много пало убитых, так как от Бога было сражение сие. И жили они на месте их до переселения.
\par 23 Потомки полуколена Манассиина жили в той земле, от Васана до Ваал-Ермона и Сенира и до горы Ермона; и их было много.
\par 24 И вот главы поколений их: Ефер, Ишьи, Елиил, Азриил, Иеремия, Годавия и Иагдиил, мужи мощные, мужи именитые, главы родов своих.
\par 25 Но когда они согрешили против Бога отцов своих и стали блудно ходить вслед богов народов той земли, которых изгнал Бог от лица их,
\par 26 тогда Бог Израилев возбудил дух Фула, царя Ассирийского, и дух Феглафелласара, царя Ассирийского, и он выселил Рувимлян и Гадитян и половину колена Манассиина, и отвел их в Халах, и Хавор, и Ару, и на реку Гозан, --[где они] до сего дня.

\chapter{6}

\par 1 Сыновья Левия: Гирсон, Кааф и Мерари.
\par 2 Сыновья Каафа: Амрам, Ицгар, Хеврон и Узиил.
\par 3 Дети Амрама: Аарон, Моисей и Мариам. Сыновья Аарона: Надав, Авиуд, Елеазар и Ифамар.
\par 4 Елеазар родил Финееса, Финеес родил Авишуя;
\par 5 Авишуй родил Буккия, Буккий родил Озию;
\par 6 Озия родил Зерахию, Зерахия родил Мераиофа;
\par 7 Мераиоф родил Амарию, Амария родил Ахитува;
\par 8 Ахитув родил Садока, Садок родил Ахимааса;
\par 9 Ахимаас родил Азарию, Азария родил Иоанана;
\par 10 Иоанан родил Азарию, --это тот, который был священником в храме, построенном Соломоном в Иерусалиме.
\par 11 И родил Азария Амарию, Амария родил Ахитува;
\par 12 Ахитув родил Садока, Садок родил Селлума;
\par 13 Селлум родил Хелкию, Хелкия родил Азарию;
\par 14 Азария родил Сераию, Сераия родил Иоседека.
\par 15 Иоседек пошел [в плен], когда Господь переселил Иудеев и Иерусалимлян рукою Навуходоносора.
\par 16 Итак сыновья Левия: Гирсон, Кааф и Мерари.
\par 17 Вот имена сыновей Гирсоновых: Ливни и Шимей.
\par 18 Сыновья Каафа: Амрам, Ицгар, Хеврон и Узиил.
\par 19 Сыновья Мерари: Махли и Муши. Вот потомки Левия по родам их.
\par 20 У Гирсона: Ливни, сын его; Иахав, сын его; Зимма, сын его;
\par 21 Иоах, сын его; Иддо, сын его; Зерах, сын его; Иеафрай, сын его.
\par 22 Сыновья Каафа: Аминадав, сын его; Корей, сын его; Асир, сын его;
\par 23 Елкана, сын его; Евиасаф, сын его; Асир, сын его;
\par 24 Тахаф, сын его; Уриил, сын его; Узия, сын его; Саул, сын его.
\par 25 Сыновья Елканы: Амасай и Ахимоф.
\par 26 Елкана, сын его; Цофай, сын его; Нахаф, сын его;
\par 27 Елиаф, сын его; Иерохам, сын его, Елкана, сын его.
\par 28 Сыновья Самуила: первенец Иоиль, второй Авия.
\par 29 Сыновья Мерари: Махли; Ливни, сын его; Шимей, сын его; Уза, сын его;
\par 30 Шима, сын его; Хаггия, сын его; Асаия, сын его.
\par 31 Вот те, которых Давид поставил начальниками над певцами в доме Господнем, со времени поставления в нем ковчега.
\par 32 Они служили певцами пред скиниею собрания, доколе Соломон не построил дома Господня в Иерусалиме. И они становились на службу свою по уставу своему.
\par 33 Вот те, которые становились с сыновьями своими: из сыновей Каафовых--Еман певец, сын Иоиля, сын Самуила,
\par 34 сын Елканы, сын Иерохама, сын Елиила, сын Тоаха,
\par 35 сын Цуфа, сын Елканы, сын Махафа, сын Амасая,
\par 36 сын Елканы, сын Иоиля, сын Азарии, сын Цефании,
\par 37 сын Тахафа, сын Асира, сын Авиасафа, сын Корея,
\par 38 сын Ицгара, сын Каафа, сын Левия, сын Израиля;
\par 39 и брат его Асаф, стоявший на правой стороне его, --Асаф, сын Берехии, сын Шимы,
\par 40 сын Михаила, сын Ваасеи, сын Малхии,
\par 41 сын Ефния, сын Зераха, сын Адаии,
\par 42 сын Ефана, сын Зиммы, сын Шимия,
\par 43 сын Иахафа, сын Гирсона, сын Левия.
\par 44 А из сыновей Мерари, братьев их, --на левой стороне: Ефан, сын Кишия, сын Авдия, сын Маллуха,
\par 45 сын Хашавии, сын Амасии, сын Хелкии,
\par 46 сын Амция, сын Вания, сын Шемера,
\par 47 сын Махлия, сын Мушия, сын Мерари, сын Левия.
\par 48 Братья их левиты определены на всякие службы при доме Божием;
\par 49 Аарон же и сыновья его сожигали на жертвеннике всесожжения и на жертвеннике кадильном, и совершали всякое священнодействие во Святом Святых и для очищения Израиля во всем, как заповедал раб Божий Моисей.
\par 50 Вот сыновья Аарона: Елеазар, сын его; Финеес, сын его; Авиуд, сын его;
\par 51 Буккий, сын его; Уззий, сын его; Зерахия, сын его;
\par 52 Мераиоф, сын его; Амария, сын его; Ахитув, сын его;
\par 53 Садок, сын его; Ахимаас, сын его.
\par 54 И вот жилища их по селениям их в пределах их: сыновьям Аарона из племени Каафова, так как жребий выпал им,
\par 55 дали Хеврон, в земле Иудиной, и предместья его вокруг его;
\par 56 поля же сего города и села его отдали Халеву, сыну Иефонниину.
\par 57 Сыновьям Аарона дали также города убежищ: Хеврон и Ливну с их предместьями, Иаттир и Ештемоа и предместья его,
\par 58 и Хилен и предместья его, Давир и предместья его,
\par 59 и Ашан и предместья его, Вефсамис и предместья его,
\par 60 а от колена Вениаминова--Геву и предместья ее, и Аллемеф и предместья его, и Анафоф и предместья его: всех городов их в племенах их тринадцать городов.
\par 61 Остальным сыновьям Каафа, из семейств этого колена, [дано] по жребию десять городов из удела половины колена Манассиина.
\par 62 Сыновьям Гирсона по племенам их, от колена Иссахарова, и от колена Асирова, и от колена Неффалимова, и от колена Манассиина в Васане, [дано] тринадцать городов.
\par 63 Сыновьям Мерари по племенам их, от колена Рувимова, и от колена Гадова, и от колена Завулонова, [дано] по жребию двенадцать городов.
\par 64 Так дали сыны Израилевы левитам города и предместья их.
\par 65 Дали они по жребию от колена сыновей Иудиных, и от колена сыновей Симеоновых, и от колена сыновей Вениаминовых те города, которые они назвали по именам.
\par 66 Некоторым же племенам сыновей Каафовых даны были города от колена Ефремова.
\par 67 И дали им города убежищ: Сихем и предместья его на горе Ефремовой, и Гезер и предместья его,
\par 68 и Иокмеам и предместья его, и Беф-Орон и предместья его,
\par 69 и Аиалон и предместья его, и Гаф-Риммон и предместья его;
\par 70 от половины колена Манассиина--Анер и предместья его, Билеам и предместья его. Это поколению остальных сыновей Каафовых.
\par 71 Сыновьям Гирсона от племени полуколена Манассиина [дали] Голан в Васане и предместья его, и Аштароф и предместья его.
\par 72 От колена Иссахарова--Кедес и предместья его, Давраф и предместья его,
\par 73 и Рамоф и предместья его, и Анем и предместья его;
\par 74 от колена Асирова--Машал и предместья его, и Авдон и предместья его,
\par 75 и Хукок и предместья его, и Рехов и предместья его;
\par 76 от колена Неффалимова--Кедес в Галилее и предместья его, и Хаммон и предместья его, и Кириафаим и предместья его.
\par 77 А прочим сыновьям Мерариным--от колена Завулонова Риммон и предместья его, Фавор и предместья его.
\par 78 По ту сторону Иордана, против Иерихона, на восток от Иордана, от колена Рувимова [дали] Восор в пустыне и предместья его, и Иаацу и предместья ее,
\par 79 и Кедемоф и предместья его, и Мефааф и предместья его;
\par 80 от колена Гадова--Рамоф в Галааде и предместья его, и Маханаим и предместья его,
\par 81 и Есевон и предместья его, и Иазер и предместья его.

\chapter{7}

\par 1 Сыновья Иссахара: Фола, Фуа, Иашув и Шимрон, четверо.
\par 2 Сыновья Фолы: Уззий, Рефаия, Иериил, Иахмай, Ивсам и Самуил, главные в поколениях Фолы, люди воинственные в своих поколениях; число их во дни Давида было двадцать две тысячи и шестьсот.
\par 3 Сын Уззия: Израхия; а сыновья Израхии: Михаил, Овадиа, Иоиль и Ишшия, пятеро. Все они главные.
\par 4 У них, по родам их, по поколениям их, было готово к сражению войска тридцать шесть тысяч; потому что у них было много жен и сыновей.
\par 5 Братьев же их, во всех поколениях Иссахаровых, людей воинственных, было восемьдесят семь тысяч, внесенных в родословные записи.
\par 6 У Вениамина: Бела, Бехер и Иедиаил, трое.
\par 7 Сыновья Белы: Ецбон, Уззий, Уззиил, Иеримоф и Ири, пятеро, главы поколений, люди воинственные. В родословных списках записано их двадцать две тысячи тридцать четыре.
\par 8 Сыновья Бехера: Земира, Иоаш, Елиезер, Елиоенай, Омри, Иремоф, Авия, Анафоф и Алемеф: все эти сыновья Бехера.
\par 9 В родословных списках записано их по родам их, по главам поколений, людей воинственных--двадцать тысяч и двести.
\par 10 Сын Иедиаила: Билган. Сыновья Билгана: Иеус, Вениамин, Егуд, Хенаана, Зефан, Фарсис и Ахишахар.
\par 11 Все эти сыновья Иедиаила были главами поколений, люди воинственные; семнадцать тысяч и двести было выходящих на войну.
\par 12 И Шупим и Хупим, сыновья Ира; Хушим, сын Ахера;
\par 13 сыновья Неффалима: Иахцеил, Гуни, Иецер и Шиллем, дети Валлы.
\par 14 Сыновья Манассии: Асриил, которого родила наложница его Арамеянка; она же родила Махира, отца Галаадова.
\par 15 Махир взял в жену сестру Хупима и Шупима, --имя сестры их Мааха; имя второму Салпаад. У Салпаада были [только] дочери.
\par 16 Мааха, жена Махирова, родила сына и нарекла ему имя Кереш, а имя брату его Шереш. Сыновья его: Улам и Рекем.
\par 17 Сын Улама: Бедан. Вот сыновья Галаада, сына Махира, сына Манассиина.
\par 18 Сестра его Молехеф родила Ишгода, Авиезера и Махлу.
\par 19 Сыновья Шемиды были: Ахиан, Шехем, Ликхи и Аниам.
\par 20 Сыновья Ефрема: Шутелах, и Беред, сын его, и Фахаф, сын его, и Елеада, сын его, и Фахаф, сын его,
\par 21 и Завад, сын его, и Шутелах, сын его, и Езер и Елеад. И убили их жители Гефа, уроженцы той земли, за то, что они пошли захватить стада их.
\par 22 И плакал о них Ефрем, отец их, много дней, и приходили братья его утешать его.
\par 23 Потом он вошел к жене своей, и она зачала и родила сына, и он нарек ему имя: Берия, потому что несчастье постигло дом его.
\par 24 И дочь у него [была] Шеера. Она построила Беф-Орон нижний и верхний и Уззен-Шееру.
\par 25 И Рефай, сын его, и Решеф, и Фелах, сын его, и Фахан, сын его,
\par 26 Лаедан, сын его, Аммиуд, сын его, Елишама, сын его,
\par 27 Нон, сын его, Иисус, сын его.
\par 28 Владения их и места жительства их [были]: Вефиль и зависящие от него города; к востоку Нааран, к западу Гезер и зависящие от него города; Сихем и зависящие от него города до Газы и зависящих от нее городов.
\par 29 А со стороны сыновей Манассииных: Беф-Сан и зависящие от него города, Фаанах и зависящие от него города, Мегиддо и зависящие от него города, Дор и зависящие от него города. В них жили сыновья Иосифа, сына Израилева.
\par 30 Сыновья Асира: Имна, Ишва, Ишви и Берия, и сестра их Серах.
\par 31 Сыновья Берии: Хевер и Малхиил. Он отец Бирзаифа.
\par 32 Хевер родил Иафлета, Шомера и Хофама, и Шую, сестру их.
\par 33 Сыновья Иафлета: Пасах, Бимгал и Ашваф. Вот сыновья Иафлета.
\par 34 Сыновья Шемера: Ахи, Рохга, Ихубба и Арам.
\par 35 Сыновья Гелема, брата его: Цофах, Имна, Шелеш и Амал.
\par 36 Сыновья Цофаха: Суах, Харнефер, Шуал, Бери, Имра,
\par 37 Бецер, Год, Шамма, Шилша, Ифран и Беера.
\par 38 Сыновья Иефера: Иефунни, Фиспа и Ара.
\par 39 Сыновья Уллы: Арах, Ханниил и Риция.
\par 40 Все эти сыновья Асира, главы поколений, люди отборные, воинственные, главные начальники. Записано у них в родословных списках в войске, для войны, по счету двадцать шесть тысяч человек.

\chapter{8}

\par 1 Вениамин родил Белу, первенца своего, второго Ашбела, третьего Ахрая,
\par 2 четвертого Ноху и пятого Рафу.
\par 3 Сыновья Белы были: Аддар, Гера, Авиуд,
\par 4 Авишуа, Нааман, Ахоах,
\par 5 Гера, Шефуфан и Хурам.
\par 6 И вот сыновья Егуда, которые были главами родов, живших в Геве и переселенных в Манахаф:
\par 7 Нааман, Ахия и Гера, который переселил их; он родил Уззу и Ахихуда.
\par 8 Шегараим родил детей в земле Моавитской после того, как отпустил от [себя] Хушиму и Баару, жен своих.
\par 9 И родил он от Ходеши, жены своей, Иовава, Цивию, Мешу, Малхама,
\par 10 Иеуца, Шахию и Мирму: вот сыновья его, главы поколений.
\par 11 От Хушимы родил он Авитува и Елпаала.
\par 12 Сыновья Елпаала: Евер, Мишам и Шемер, который построил Оно и Лод и зависящие от него города, --
\par 13 и Берия и Шема. Они были главами поколений жителей Аиалона. Они выгнали жителей Гефа.
\par 14 Ахио, Шашак, Иремоф,
\par 15 Зевадия, Арад, Едер,
\par 16 Михаил, Ишфа и Иоха--сыновья Берии.
\par 17 Зевадия, Мешуллам, Хизкий, Хевер,
\par 18 Ишмерай, Излия и Иовав--сыновья Елпаала.
\par 19 Иаким, Зихрий, Завдий,
\par 20 Елиенай, Цилфай, Елиил,
\par 21 Адаия, Бераия и Шимраф--сыновья Шимея.
\par 22 Ишпан, Евер, Елиил,
\par 23 Авдон, Зихрий, Ханан,
\par 24 Ханания, Елам, Антофия,
\par 25 Ифдия и Фенуил--сыновья Шашака.
\par 26 Шамшерай, Шехария, Афалия,
\par 27 Иаарешия, Елия и Зихрий, сыновья Иерохама.
\par 28 Это главы поколений, в родах своих главные. Они жили в Иерусалиме.
\par 29 В Гаваоне жили: отец Гаваонитян, --имя жены его Мааха, --
\par 30 и сын его, первенец Авдон, [за ним] Цур, Кис, Ваал, Надав,
\par 31 Гедор, Ахио, Зехер и Миклоф.
\par 32 Миклоф родил Шимея. И они подле братьев своих жили в Иерусалиме, вместе с братьями своими.
\par 33 Нер родил Киса; Кис родил Саула; Саул родил Иоанафана, Мелхисуя, Авинадава и Ешбаала.
\par 34 Сын Ионафана Мериббаал; Мериббаал родил Миху.
\par 35 Сыновья Михи: Пифон, Мелег, Фаарея и Ахаз.
\par 36 Ахаз родил Иоиадду; Иоиадда родил Алемефа, Азмавефа и Замврия; Замврий родил Моцу;
\par 37 Моца родил Бинею. Рефаия, сын его; Елеаса, сын его; Ацел, сын его.
\par 38 У Ацела шесть сыновей, и вот имена их: Азрикам, Бохру, Исмаил, Шеария, Овадия и Ханан; все они сыновья Ацела.
\par 39 Сыновья Ешека, брата его: Улам, первенец его, второй Иеуш, третий Елифелет.
\par 40 Сыновья Улама были люди воинственные, стрелявшие из лука, имевшие много сыновей и внуков: сто пятьдесят. Все они от сынов Вениамина.

\chapter{9}

\par 1 Так были перечислены по родам своим все Израильтяне, и вот они записаны в книге царей Израильских. Иудеи же за беззакония свои переселены в Вавилон.
\par 2 Первые жители, которые [жили] во владениях своих, по городам Израильским, были Израильтяне, священники, левиты и нефинеи.
\par 3 В Иерусалиме жили некоторые из сынов Иудиных и из сынов Вениаминовых, и из сынов Ефремовых и Манассииных:
\par 4 Уфай, сын Аммиуда, сын Омри, сын Имрия, сын Вания, --из сыновей Фареса, сына Иудина;
\par 5 из сыновей Шилона--Асаия первенец и сыновья его;
\par 6 из сыновей Зары--Иеуил и братья их, --шестьсот девяносто;
\par 7 из сыновей Вениаминовых Саллу, сын Мешуллама, сын Годавии, сын Гассенуи;
\par 8 и Ивния, сын Иерохама, и Эла, сын Уззия, сына Михриева, и Мешуллам, сын Шефатии, сына Регуила, сына Ивнии,
\par 9 и братья их, по родам их: девятьсот пятьдесят шесть, --все сии мужи были главы родов в поколениях своих.
\par 10 А из священников: Иедаия, Иоиарив, Иахин,
\par 11 и Азария, сын Хелкии, сын Мешуллама, сын Садока, сын Мераиофа, сын Ахитува, начальствующий в доме Божием;
\par 12 и Адаия, сын Иерохама, сын Пашхура, сын Малхии; и Маасай, сын Адиела, сын Иахзера, сын Мешуллама, сын Мешиллемифа, сын Иммера;
\par 13 и братья их, главы родов своих: тысяча семьсот шестьдесят, --люди отличные в деле служения в доме Божием.
\par 14 А из левитов: Шемаия, сын Хашува, сын Азрикама, сын Хашавии, --из сыновей Мерариных;
\par 15 и Вакбакар, Хереш, Галал, и Матфания, сын Михи, сын Зихрия, сын Асафа;
\par 16 и Овадия, сын Шемаии, сын Галала, сын Идифуна, и Берехия, сын Асы, сын Елканы, живший в селениях Нетофафских.
\par 17 А привратники: Шаллум, Аккуб, Талмон и Ахиман, и братья их; Шаллум [был] главным.
\par 18 И доныне сии привратники у ворот царских, к востоку, содержат стражу сынов Левииных.
\par 19 Шаллум, сын Коре, сын Евиасафа, сын Корея, и братья его из рода его, Кореяне, по делу служения своего, были стражами у порогов скинии, а отцы их охраняли вход в стан Господень.
\par 20 Финеес, сын Елеазаров, был прежде начальником над ними, и Господь был с ним.
\par 21 Захария, сын Мешелемии, [был] привратником у дверей скинии собрания.
\par 22 Всех их, выбранных в привратники к порогам, было двести двенадцать. Они внесены в список по селениям своим. Их поставил Давид и Самуил-прозорливец за верность их.
\par 23 И они и сыновья их были на страже у ворот дома Господня, при доме скинии.
\par 24 На четырех сторонах находились привратники: на восточной, западной, северной и южной.
\par 25 Братья же их жили в селениях своих, приходя к ним от времени до времени на семь дней.
\par 26 Сии четыре начальника привратников, левиты, были в доверенности; они же были приставлены к жилищам и к сокровищам дома Божия.
\par 27 Вокруг дома Божия они и ночь проводили, потому что на них [лежало] охранение, и они должны были каждое утро отпирать двери.
\par 28 [Одни] из них были приставлены к служебным сосудам, так что счетом принимали их и счетом выдавали.
\par 29 [Другим] из них поручена была прочая утварь и все священные потребности: мука лучшая, и вино, и елей, и ладан, и благовония.
\par 30 А из сыновей священнических [некоторые] составляли миро из веществ благовонных.
\par 31 Маттафии из левитов, --он первенец Селлума Кореянина, --вверено было приготовляемое на сковородах.
\par 32 [Некоторым] из братьев их, из сынов Каафовых, поручено было [заготовление] хлебов предложения, чтобы представлять [их] каждую субботу.
\par 33 Певцы же, главные в поколениях левитских, в комнатах храма свободны были от занятий, потому что день и ночь они обязаны были [заниматься] искусством [своим].
\par 34 Это главы поколений левитских, в родах своих главные. Они жили в Иерусалиме.
\par 35 В Гаваоне жили: отец Гаваонитян Иеил, --имя жены его Мааха,
\par 36 и сын его первенец Авдон, [за ним] Цур, Кис, Ваал, Нер, Надав,
\par 37 Гедор, Ахио, Захария и Миклоф.
\par 38 Миклоф родил Шимеама. И они подле братьев своих жили в Иерусалиме вместе с братьями своими.
\par 39 Нер родил Киса, Кис родил Саула, Саул родил Ионафана, Мелхисуя, Авинадава и Ешбаала.
\par 40 Сын Ионафана Мериббаал; Мериббаал родил Миху.
\par 41 Сыновья Михи: Пифон, Мелех, Фарей [и Ахаз].
\par 42 Ахаз родил Иаеру; Иаера родил Алемефа, Азмавефа и Замврия; Замврий родил Моцу;
\par 43 Моца родил Бинею: Рефаия, сын его; Елеаса, сын его; Ацел, сын его.
\par 44 У Ацела шесть сыновей, и вот имена их: Азрикам, Бохру, Исмаил, Шеария, Овадия и Ханан. Это сыновья Ацела.

\chapter{10}

\par 1 Филистимляне воевали с Израилем, и побежали Израильтяне от Филистимлян, и падали пораженные на горе Гелвуе.
\par 2 И погнались Филистимляне за Саулом и сыновьями его, и убили Филистимляне Ионафана и Авинадава и Мелхисуя, сыновей Сауловых.
\par 3 Сражение против Саула усилилось, и стрелки устремились на него, так что он изранен был стрелками.
\par 4 И сказал Саул оруженосцу своему: обнажи меч твой и заколи меня им, чтобы не пришли эти необрезанные и не надругались надо мною. Но оруженосец не решился, потому что очень испугался. Тогда Саул взял меч и пал на него.
\par 5 Оруженосец его, увидев, что Саул умер, и сам пал на меч и умер.
\par 6 И умер Саул, и три сына его, и весь дом его вместе с ним умер.
\par 7 Когда увидели Израильтяне, которые были в долине, что все бегут и что Саул и сыновья его умерли, то оставили города свои и разбежались; а Филистимляне пришли и поселились в них.
\par 8 На другой день пришли Филистимляне обирать убитых, и нашли Саула и сыновей его, павших на горе Гелвуйской,
\par 9 и раздели его, и сняли с него голову его и оружие его, и послали по земле Филистимской, чтобы возвестить [о сем] пред идолами их и пред народом.
\par 10 И положили оружие его в капище богов своих, и голову его воткнули в доме Дагона.
\par 11 И услышал весь Иавис Галаадский все, что сделали Филистимляне с Саулом.
\par 12 И поднялись все люди сильные, взяли тело Саулово и тела сыновей его, и принесли их в Иавис, и похоронили кости их под дубом в Иависе, и постились семь дней.
\par 13 Так умер Саул за свое беззаконие, которое он сделал пред Господом, за то, что не соблюл слова Господня и обратился к волшебнице с вопросом,
\par 14 а не взыскал Господа. [За то] Он и умертвил его, и передал царство Давиду, сыну Иессееву.

\chapter{11}

\par 1 И собрались все Израильтяне к Давиду в Хеврон и сказали: вот, мы кость твоя и плоть твоя;
\par 2 и вчера, и третьего дня, когда еще Саул был царем, ты выводил и вводил Израиля, и Господь Бог твой сказал тебе: `ты будешь пасти народ Мой, Израиля и ты будешь вождем народа Моего Израиля'.
\par 3 И пришли все старейшины Израилевы к царю в Хеврон, и заключил с ними Давид завет в Хевроне пред лицем Господним; и они помазали Давида в царя над Израилем, по слову Господню, чрез Самуила.
\par 4 И пошел Давид и весь Израиль к Иерусалиму, то есть к Иевусу. А там были Иевусеи, жители той земли.
\par 5 И сказали жители Иевуса Давиду: не войдешь сюда. Но Давид взял крепость Сион; это город Давидов.
\par 6 И сказал Давид: кто прежде всех поразит Иевусеев, тот будет главою и военачальником. И взошел прежде всех Иоав, сын Саруи, и сделался главою.
\par 7 Давид жил в той крепости, потому и называли ее городом Давидовым.
\par 8 И он обстроил город кругом, [начиная] от Милло, всю окружность, а Иоав возобновил остальные [части] города.
\par 9 И преуспевал Давид, и возвышался более и более, и Господь Саваоф [был] с ним.
\par 10 Вот главные из сильных у Давида, которые крепко подвизались с ним в царстве его, вместе со всем Израилем, чтобы воцарить его, по слову Господню, над Израилем,
\par 11 и вот число храбрых, которые были у Давида: Иесваал, сын Ахамани, главный из тридцати. Он поднял копье свое на триста человек и поразил их в один раз.
\par 12 По нем Елеазар, сын Додо Ахохиянина, из трех храбрых:
\par 13 он был с Давидом в Фасдамиме, куда Филистимляне собрались на войну. Там часть поля была засеяна ячменем, и народ побежал от Филистимлян;
\par 14 но они стали среди поля, сберегли его и поразили Филистимлян. И даровал Господь спасение великое!
\par 15 Трое сих главных из тридцати вождей взошли на скалу к Давиду, в пещеру Одоллам, когда стан Филистимлян был расположен в долине Рефаимов.
\par 16 Давид тогда был в укрепленном месте, а охранное войско Филистимлян было тогда в Вифлееме.
\par 17 И сильно захотелось [пить] Давиду, и он сказал: кто напоит меня водою из колодезя Вифлеемского, что у ворот?
\par 18 Тогда эти трое пробились сквозь стан Филистимский и почерпнули воды из колодезя Вифлеемского, что у ворот, и взяли, и принесли Давиду. Но Давид не захотел пить ее и вылил ее во славу Господа,
\par 19 и сказал: сохрани меня Господь, чтоб я сделал это! Стану ли я пить кровь мужей сих, полагавших души свои! Ибо с опасностью собственной жизни они принесли [воду]. И не захотел пить ее. Вот что сделали трое этих храбрых.
\par 20 И Авесса, брат Иоава, был главным из трех: он убил копьем своим триста человек, и был в славе у тех троих.
\par 21 Из трех он был знатнейшим и был начальником; но с теми тремя не равнялся.
\par 22 Ванея, сын Иодая, мужа храброго, великий по делам, из Кавцеила: он поразил двух Ариилов Моавитских; он же сошел и убил льва во рве, в снежное время;
\par 23 он же убил Египтянина, человека ростом в пять локтей: в руке Египтянина было копье, как навой у ткачей, а он подошел к нему с палкою и, вырвав копье из руки Египтянина, убил его его же копьем:
\par 24 вот что сделал Ванея, сын Иодая. И он был в славе у тех троих храбрых;
\par 25 он был знатнее тридцати, но с тремя не равнялся, и Давид поставил его ближайшим исполнителем своих приказаний.
\par 26 А главные из воинов: Асаил, брат Иоава; Елханан, сын Додо, из Вифлеема;
\par 27 Шамма Гародитянин; Херец Пелонитянин;
\par 28 Ира, сын Икеша, Фекоитянин; Евиезер Анафофянин;
\par 29 Сивхай Хушатянин; Илай Ахохиянин;
\par 30 Магарай Нетофафянин; Хелед, сын Вааны, Нетофафянин;
\par 31 Иттай, сын Рибая, из Гивы Вениаминовой; Ванея Пирафонянин;
\par 32 Хурай из Нагале-Гааша; Авиел из Аравы;
\par 33 Азмавеф Бахарумиянин; Елияхба Шаалбонянин.
\par 34 Сыновья Гашема Гизонитянина: Ионафан, сын Шаге, Гараритянин;
\par 35 Ахиам, сын Сахара, Гараритянин; Елифал, сын Уры;
\par 36 Хефер из Махеры; Ахиа Пелонитянин;
\par 37 Хецрой Кармилитянин; Наарай, сын Езбая;
\par 38 Иоиль, брат Нафана; Мивхар, сын Гагрия;
\par 39 Целек Аммонитянин; Нахарай Берофянин, оруженосец Иоава, сына Саруи;
\par 40 Ира Ифриянин; Гареб Ифриянин;
\par 41 Урия Хеттеянин; Завад, сын Ахлая;
\par 42 Адина, сын Шизы, Рувимлянин, глава Рувимлян, и у него [было] тридцать;
\par 43 Ханан, сын Маахи; Иосафат Мифниянин;
\par 44 Уззия Аштерофянин; Шама и Иеиел, сыновья Хофама Ароерянина;
\par 45 Иедиаел, сын Шимрия, и Иоха, брат его, Фициянин;
\par 46 Елиел из Махавима, и Иеривай и Иошавия, сыновья Елнаама, и Ифма Моавитянин;
\par 47 Елиел, Овед и Иасиел из Мецоваи.

\chapter{12}

\par 1 И сии также пришли к Давиду в Секелаг, когда он еще укрывался от Саула, сына Кисова, и были из храбрых, помогавших в сражении.
\par 2 Вооруженные луком, правою и левою рукою [бросавшие] каменья и [стрелявшие] стрелами из лука, --из братьев Саула, от Вениамина:
\par 3 главный Ахиезер, за ним Иоас, сыновья Шемаи, из Гивы; Иезиел и Фелет, сыновья Азмавефа; Бераха и Иегу из Анафофа;
\par 4 Ишмаия Гаваонитянин, храбрый из тридцати и [начальствовавший] над тридцатью; Иеремия, Иахазиил, Иоханан и Иозавад из Гедеры.
\par 5 Елузай, Иеримоф, Веалия, Шемария, Сафатия Харифиянин;
\par 6 Елкана, Ишшияху, Азариил, Иоезер и Иошавам, Кореяне;
\par 7 и Иоела и Зевадия, сыновья Иерохама, из Гедора.
\par 8 И из Гадитян перешли к Давиду в укрепление, в пустыню, люди мужественные, воинственные, вооруженные щитом и копьем; лица львиные--лица их, и они быстры как серны на горах.
\par 9 Главный Езер, второй Овадия, третий Елиав,
\par 10 четвертый Мишманна, пятый Иеремия,
\par 11 шестой Афай, седьмой Елиел,
\par 12 восьмой Иоханан, девятый Елзавад,
\par 13 десятый Иеремия, одиннадцатый Махбанай.
\par 14 Они из сыновей Гадовых [были] главами в войске: меньший над сотнею, и больший над тысячею.
\par 15 Они-то перешли Иордан в первый месяц, когда он выступает из берегов своих, и разогнали всех живших в долинах к востоку и западу.
\par 16 Пришли также и из сыновей Вениаминовых и Иудиных в укрепление к Давиду.
\par 17 Давид вышел навстречу им и сказал им: если с миром пришли вы ко мне, чтобы помогать мне, то да будет у меня с вами одно сердце; а если для того, чтобы коварно предать меня врагам моим, тогда как нет порока на руках моих, то да видит Бог отцов наших и рассудит.
\par 18 И объял дух Амасая, главу тридцати, [и сказал он]: мир тебе Давид, и с тобою, сын Иессеев; мир тебе, и мир помощникам твоим; ибо помогает тебе Бог твой. Тогда принял их Давид и поставил их во главе войска.
\par 19 И из колена Манассиина перешли [некоторые] к Давиду, когда он шел с Филистимлянами на войну против Саула, но не помогал им, потому что предводители Филистимские, посоветовавшись, отослали его, говоря: на нашу голову он перейдет к господину своему Саулу.
\par 20 Когда он возвращался в Секелаг, тогда перешли к нему из Манассиян: Аднах, Иозавад, Иедиаел, Михаил, Иозавад, Елигу и Цилльфай, тысяченачальники у Манассиян.
\par 21 И они помогали Давиду против полчищ, ибо все это были люди храбрые и были начальниками в войске.
\par 22 Так с каждым днем приходили к Давиду на помощь до того, что его ополчение стало велико, как ополчение Божие.
\par 23 Вот число главных в войске, которые пришли к Давиду в Хеврон, чтобы передать ему царство Саулово, по слову Господню:
\par 24 сыновей Иудиных, носящих щит и копье, было шесть тысяч восемьсот готовых к войне;
\par 25 из сыновей Симеоновых, людей храбрых, в войске было семь тысяч и сто;
\par 26 из сыновей Левииных четыре тысячи шестьсот;
\par 27 и Иоддай, князь от [племени] Аарона, и с ним три тысячи семьсот;
\par 28 и Садок, мужественный юноша, и род его, двадцать два начальника;
\par 29 из сыновей Вениаминовых, братьев Сауловых, три тысячи, --но еще многие из них держались дома Саулова;
\par 30 из сыновей Ефремовых двадцать тысяч восемьсот людей мужественных, людей именитых в родах своих;
\par 31 из полуколена Манассиина восемнадцать тысяч, которые вызваны были поименно, чтобы пойти воцарить Давида;
\par 32 из сынов Иссахаровых [пришли] люди разумные, которые знали, что когда надлежало делать Израилю, --их было двести главных, и все братья их следовали слову их;
\par 33 из [колена] Завулонова готовых к сражению, вооруженных всякими военными оружиями, пятьдесят тысяч, в строю, единодушных;
\par 34 из [колена] Неффалимова тысяча вождей и с ними тридцать семь тысяч с щитами и копьями;
\par 35 из [колена] Данова готовых к войне двадцать восемь тысяч шестьсот;
\par 36 от Асира воинов, готовых к сражению, сорок тысяч;
\par 37 из-за Иордана, от колена Рувимова, Гадова и полуколена Манассиина, сто двадцать тысяч, со всяким воинским оружием.
\par 38 Все эти воины, в строю, от полного сердца пришли в Хеврон воцарить Давида над всем Израилем. Да и все прочие Израильтяне были единодушны, чтобы воцарить Давида.
\par 39 И пробыли там у Давида три дня, ели и пили, потому что братья их [все] приготовили для них;
\par 40 да и близкие к ним, даже до [колена] Иссахарова, Завулонова и Неффалимова, привозили все съестное на ослах, и верблюдах, и мулах, и волах: муку, смоквы, и изюм, и вино, и елей, и крупного и мелкого скота множество, так как радость была для Израиля.

\chapter{13}

\par 1 И советовался Давид с тысяченачальниками, сотниками и со всеми вождями,
\par 2 и сказал [Давид] всему собранию Израильтян: если угодно вам, и если на то будет воля Господа Бога нашего, пошлем повсюду к прочим братьям нашим, по всей земле Израильской, и вместе с ними к священникам и левитам, в города и селения их, чтобы они собрались к нам;
\par 3 и перенесем к себе ковчег Бога нашего, потому что во дни Саула мы не обращались к нему.
\par 4 И сказало все собрание: `да будет так', потому что это дело всему народу казалось справедливым.
\par 5 Так собрал Давид всех Израильтян, от Шихора Египетского до входа в Емаф, чтобы перенести ковчег Божий из Кириаф-Иарима.
\par 6 И пошел Давид и весь Израиль в Кириаф-Иарим, что в Иудее, чтобы перенести оттуда ковчег Бога, Господа, седящего на Херувимах, на котором нарицается имя [Его].
\par 7 И повезли ковчег Божий на новой колеснице из дома Авинадава; и Оза и Ахия вели колесницу.
\par 8 Давид же и все Израильтяне играли пред Богом из всей силы, с пением, на цитрах и псалтирях, и тимпанах, и кимвалах и трубах.
\par 9 Когда дошли до гумна Хидона, Оза простер руку свою, чтобы придержать ковчег, ибо волы наклонили его.
\par 10 Но Господь разгневался на Озу, и поразил его за то, что он простер руку свою к ковчегу; и он умер тут же пред лицем Божиим.
\par 11 И опечалился Давид, что Господь поразил Озу. И назвал то место поражением Озы; так называется оно и до сего дня.
\par 12 И устрашился Давид Бога в день тот, и сказал: как я внесу к себе ковчег Божий?
\par 13 И не повез Давид ковчега к себе, в город Давидов, а обратил его к дому Аведдара Гефянина.
\par 14 И оставался ковчег Божий у Аведдара, в доме его, три месяца, и благословил Господь дом Аведдара и все, что у него.

\chapter{14}

\par 1 И послал Хирам, царь Тирский, к Давиду послов, и кедровые деревья, и каменщиков, и плотников, чтобы построить ему дом.
\par 2 Когда узнал Давид, что утвердил его Господь царем над Израилем, что вознесено высоко царство его, ради народа его Израиля,
\par 3 тогда взял Давид еще жен в Иерусалиме, и родил Давид еще сыновей и дочерей.
\par 4 И вот имена родившихся у него в Иерусалиме: Самус, Совав, Нафан, Соломон,
\par 5 Евеар, Елисуа, Елфалет,
\par 6 Ногах, Нафек, Иафиа,
\par 7 и Елисама, Веелиада и Елифалеф.
\par 8 И услышали Филистимляне, что помазан Давид в царя над всем Израилем, и поднялись все Филистимляне искать Давида. И услышал Давид [об] [этом] и пошел против них.
\par 9 И Филистимляне пришли и расположились в долине Рефаимов.
\par 10 И вопросил Давид Бога, говоря: идти ли мне против Филистимлян, и предашь ли их в руки мои? И сказал ему Господь: иди, и Я предам их в руки твои.
\par 11 И пошли они в Ваал-Перацим, и поразил их там Давид; и сказал Давид: сломил Бог врагов моих рукою моею, как прорыв воды. Посему и дали имя месту тому: Ваал-Перацим.
\par 12 И оставили там [Филистимляне] богов своих, и повелел Давид, и сожжены они огнем.
\par 13 И [пришли] опять Филистимляне и расположились по долине.
\par 14 И еще вопросил Давид Бога, и сказал ему Бог: не ходи [прямо] на них, уклонись от них и иди к ним со стороны тутовых дерев;
\par 15 и когда услышишь шум как бы шагов на вершинах тутовых дерев, тогда вступи в битву, ибо вышел Бог пред тобою, чтобы поразить стан Филистимлян.
\par 16 И сделал Давид, как повелел ему Бог; и поразили стан Филистимский, от Гаваона до Газера.
\par 17 И пронеслось имя Давидово по всем землям, и Господь сделал его страшным для всех народов.

\chapter{15}

\par 1 И построил он себе домы в городе Давидовом, и приготовил место для ковчега Божия, и устроил для него скинию.
\par 2 Тогда сказал Давид: [никто] не должен носить ковчега Божия, кроме левитов, потому что их избрал Господь на то, чтобы носить ковчег Божий и служить Ему во веки.
\par 3 И собрал Давид всех Израильтян в Иерусалим, чтобы внести ковчег Господень на место его, которое он для него приготовил.
\par 4 И созвал Давид сыновей Аароновых и левитов:
\par 5 из сыновей Каафовых, Уриила начальника и братьев его--сто двадцать [человек];
\par 6 из сыновей Мерариных, Асаию начальника и братьев его--двести двадцать [человек];
\par 7 из сыновей Гирсоновых, Иоиля начальника и братьев его--сто тридцать [человек];
\par 8 из сыновей Елисафановых, Шемаию начальника и братьев его--двести;
\par 9 из сыновей Хевроновых, Елиела начальника и братьев его--восемьдесят;
\par 10 из сыновей Уззииловых, Аминадава начальника и братьев его--сто двенадцать.
\par 11 И призвал Давид священников: Садока и Авиафара, и левитов: Уриила, Асаию, Иоиля, Шемаию, Елиела и Аминадава,
\par 12 и сказал им: вы, начальники родов левитских, освятитесь сами и братья ваши, и принесите ковчег Господа Бога Израилева на [место], [которое] я приготовил для него;
\par 13 ибо как прежде не вы это [делали], то Господь Бог наш поразил нас за то, что мы не взыскали Его, как должно.
\par 14 И освятились священники и левиты для того, чтобы нести ковчег Господа, Бога Израилева.
\par 15 И понесли сыновья левитов ковчег Божий, как заповедал Моисей по слову Господа, на плечах, на шестах.
\par 16 И приказал Давид начальникам левитов поставить братьев своих певцов с музыкальными орудиями, с псалтирями и цитрами и кимвалами, чтобы они громко возвещали глас радования.
\par 17 И поставили левиты Емана, сына Иоилева, и из братьев его, Асафа, сына Верехиина, а из сыновей Мерариных, братьев их, Ефана, сына Кушаии;
\par 18 и с ними братьев их второстепенных: Захарию, Бена, Иаазиила, Шемирамофа, Иехиила, Унния, Елиава, Ванею, Маасея, Маттафию, Елифлеуя, Микнея и Овед-Едома и Иеиела, привратников.
\par 19 Еман, Асаф и Ефан играли громко на медных кимвалах,
\par 20 а Захария, Азиил, Шемирамоф, Иехиил, Унний, Елиав, Маасей и Ванея--на псалтирях, тонким голосом.
\par 21 Маттафия же, Елифлеуй, Микней, Овед-Едом, Иеиел и Азазия--на цитрах, чтобы делать начало.
\par 22 А Хенания, начальник левитов, был учитель пения, потому что был искусен в нем.
\par 23 Верехия и Елкана были придверниками у ковчега.
\par 24 Шевания, Иосафат, Нафанаил, Амасай, Захария, Ванея и Елиезер, священники, трубили трубами пред ковчегом Божиим. Овед-- Едом и Иехия [были] придверниками у ковчега.
\par 25 Так Давид и старейшины Израилевы и тысяченачальники пошли перенести ковчег завета Господня из дома Овед-Едомова с веселием.
\par 26 И когда Бог помог левитам, несшим ковчег завета Господня, тогда закололи в жертву семь тельцов и семь овнов.
\par 27 Давид был одет в виссонную одежду, [а также] и все левиты, несшие ковчег, и певцы, и Хенания начальник музыкантов и певцов. На Давиде же был [еще] льняной ефод.
\par 28 Так весь Израиль вносил ковчег завета Господня с восклицанием, при звуке рога и труб и кимвалов, играя на псалтирях и цитрах.
\par 29 Когда ковчег завета Господня входил в город Давидов, Мелхола, дочь Саулова, смотрела в окно и, увидев царя Давида, скачущего и веселящегося, уничижила его в сердце своем.

\chapter{16}

\par 1 И принесли ковчег Божий, и поставили его среди скинии, которую устроил для него Давид, и вознесли Богу всесожжения и мирные жертвы.
\par 2 Когда Давид окончил всесожжения и приношение мирных жертв, то благословил народ именем Господа
\par 3 и роздал всем Израильтянам, и мужчинам и женщинам, по одному хлебу и по куску мяса и по кружке вина,
\par 4 и поставил на службу пред ковчегом Господним [некоторых] из левитов, чтобы они славословили, благодарили и превозносили Господа Бога Израилева:
\par 5 Асафа главным, вторым по нем Захарию, Иеиела, Шемирамофа, Иехиила, Маттафию, Елиава, и Ванею, Овед-Едома и Иеиела с псалтирями и цитрами, и Асафа для игры на кимвалах,
\par 6 а Ванею и Озиила, священников, [чтобы] постоянно [трубили] пред ковчегом завета Божия.
\par 7 В этот день Давид в первый раз дал псалом для славословия Господу чрез Асафа и братьев его:
\par 8 славьте Господа, провозглашайте имя Его; возвещайте в народах дела Его;
\par 9 пойте Ему, бряцайте Ему; поведайте о всех чудесах Его;
\par 10 хвалитесь именем Его святым; да веселится сердце ищущих Господа;
\par 11 взыщите Господа и силы Его, ищите непрестанно лица Его;
\par 12 поминайте чудеса, которые Он сотворил, знамения Его и суды уст Его,
\par 13 [вы], семя Израилево, рабы Его, сыны Иакова, избранные Его!
\par 14 Он Господь Бог наш; суды Его по всей земле.
\par 15 Помните вечно завет Его, слово, которое Он заповедал в тысячу родов,
\par 16 то, что завещал Аврааму, и в чем клялся Исааку,
\par 17 и что поставил Иакову в закон и Израилю в завет вечный,
\par 18 говоря: `тебе дам Я землю Ханаанскую, в наследственный удел вам'.
\par 19 Они были тогда малочисленны и ничтожны, и пришельцы в ней,
\par 20 и переходили от народа к народу и из одного царства к другому народу;
\par 21 но Он никому не позволил обижать их, и обличал за них царей:
\par 22 `Не прикасайтеся к помазанным Моим, и пророкам Моим не делайте зла'.
\par 23 Пойте Господу, вся земля, благовествуйте изо дня в день спасение Его.
\par 24 Возвещайте язычникам славу Его, всем народам чудеса Его,
\par 25 ибо велик Господь и достохвален, страшен паче всех богов.
\par 26 Ибо все боги народов ничто, а Господь небеса сотворил.
\par 27 Слава и величие пред лицем Его, могущество и радость на месте Его.
\par 28 Воздайте Господу, племена народов, воздайте Господу славу и честь,
\par 29 воздайте Господу славу имени Его. Возьмите дар, идите пред лице Его, поклонитесь Господу в благолепии святыни Его.
\par 30 Трепещи пред Ним, вся земля, ибо Он основал вселенную, она не поколеблется.
\par 31 Да веселятся небеса, да торжествует земля, и да скажут в народах: Господь царствует!
\par 32 Да плещет море и что наполняет его, да радуется поле и все, что на нем.
\par 33 Да ликуют вместе все дерева дубравные пред лицем Господа, ибо Он идет судить землю.
\par 34 Славьте Господа, ибо вовек милость Его,
\par 35 и скажите: спаси нас, Боже, Спаситель наш! Собери нас и избавь нас от народов, да славим святое имя Твое и да хвалимся славою Твоею!
\par 36 Благословен Господь Бог Израилев, от века и до века! И сказал весь народ: аминь! аллилуия!
\par 37 Давид оставил там, пред ковчегом завета Господня, Асафа и братьев его, чтоб они служили пред ковчегом постоянно, каждый день,
\par 38 и Овед-Едома и братьев его, шестьдесят восемь [человек]; Овед-Едома, сына Идифунова, и Хосу--привратниками,
\par 39 а Садока священника и братьев его священников пред жилищем Господним, что на высоте в Гаваоне,
\par 40 для возношения всесожжений Господу на жертвеннике всесожжения постоянно, утром и вечером, и для всего, что написано в законе Господа, который Он заповедал Израилю;
\par 41 и с ними Емана и Идифуна и прочих избранных, которые назначены поименно, чтобы славить Господа, ибо навек милость Его.
\par 42 При них Еман и Идифун прославляли Бога, играя на трубах, кимвалах и разных музыкальных орудиях; сыновей же Идифуна [поставил] при вратах.
\par 43 И пошел весь народ, каждый в свой дом; возвратился и Давид, чтобы благословить дом свой.

\chapter{17}

\par 1 Когда Давид жил в доме своем, то сказал Давид Нафану пророку: вот, я живу в доме кедровом, а ковчег завета Господня под шатром.
\par 2 И сказал Нафан Давиду: все, что у тебя на сердце, делай, ибо с тобою Бог.
\par 3 Но в ту же ночь было слово Божие к Нафану:
\par 4 пойди и скажи рабу Моему Давиду: так говорит Господь: не ты построишь Мне дом для обитания,
\par 5 ибо Я не жил в доме с того дня, как вывел сынов Израиля, и до сего дня, а [ходил] из скинии в скинию и из жилища [в жилище].
\par 6 Где ни ходил Я со всем Израилем, сказал ли Я хотя слово которому-либо из судей Израильских, которым Я повелел пасти народ Мой: зачем вы не построите Мне дома кедрового?
\par 7 И теперь так скажи рабу Моему Давиду: так говорит Господь Саваоф: Я взял тебя от стада овец, чтобы ты был вождем народа Моего Израиля;
\par 8 и был с тобою везде, куда ты ни ходил, и истребил всех врагов твоих пред лицем твоим, и сделал имя твое, как имя великих на земле;
\par 9 и Я устроил место для народа Моего Израиля, и укоренил его, и будет он спокойно жить на месте своем, и не будет более тревожим, и нечестивые не станут больше теснить его, как прежде,
\par 10 в те дни, когда Я поставил судей над народом Моим Израилем, и Я смирил всех врагов твоих, и возвещаю тебе, что Господь устроит тебе дом.
\par 11 Когда исполнятся дни твои, и ты отойдешь к отцам твоим, тогда Я восставлю семя твое после тебя, которое будет из сынов твоих, и утвержу царство его.
\par 12 Он построит Мне дом, и утвержу престол его на веки.
\par 13 Я буду ему отцом, и он будет Мне сыном, --и милости Моей не отниму от него, как Я отнял от того, который был прежде тебя.
\par 14 Я поставлю его в доме Моем и в царстве Моем на веки, и престол его будет тверд вечно.
\par 15 Все эти слова и все видение точно пересказал Нафан Давиду.
\par 16 И пришел царь Давид, и стал пред лицем Господним, и сказал: кто я, Господи Боже, и что такое дом мой, что Ты так возвысил меня?
\par 17 Но и этого еще мало показалось в очах Твоих, Боже; Ты возвещаешь о доме раба Твоего вдаль, и взираешь на меня, как на человека великого, Господи Боже!
\par 18 Что еще может прибавить пред Тобою Давид для возвеличения раба Твоего? Ты знаешь раба Твоего!
\par 19 Господи! для раба Твоего, по сердцу Твоему, Ты делаешь все это великое, чтобы явить всякое величие.
\par 20 Господи! Нет подобного Тебе, и нет Бога, кроме Тебя, по всему, что слышали мы своими ушами.
\par 21 И кто подобен народу Твоему Израилю, единственному народу на земле, к которому приходил Бог, [чтоб] искупить его Себе в народ, сделать Себе имя великим и страшным делом--прогнанием народов от лица народа Твоего, который Ты избавил из Египта.
\par 22 Ты соделал народ Твой Израиля Своим собственным народом навек, и Ты, Господи, стал Богом его.
\par 23 Итак теперь, о, Господи, слово, которое Ты сказал о рабе Твоем и о доме его, утверди навек, и сделай, как Ты сказал.
\par 24 И да пребудет и возвеличится имя Твое во веки, чтобы говорили: Господь Саваоф, Бог Израилев, есть Бог над Израилем, и дом раба Твоего Давида да будет тверд пред лицем Твоим.
\par 25 Ибо Ты, Боже мой, открыл рабу Твоему, что Ты устроишь ему дом, поэтому раб Твой и дерзнул молиться пред Тобою.
\par 26 И ныне, Господи, Ты Бог, и Ты сказал о рабе Твоем такое благо.
\par 27 Начни же благословлять дом раба Твоего, чтоб он был вечно пред лицем Твоим. Ибо если Ты, Господи, благословишь, то будет он благословен вовек.

\chapter{18}

\par 1 После сего Давид поразил Филистимлян и смирил их, и взял Геф и зависящие от него города из руки Филистимлян.
\par 2 Он поразил также Моавитян, --и сделались Моавитяне рабами Давида, принося ему дань.
\par 3 И поразил Давид Адраазара, царя Сувского, в Емафе, когда тот шел утвердить власть свою при реке Евфрате.
\par 4 И взял Давид у него тысячу колесниц, семь тысяч всадников и двадцать тысяч пеших, и разрушил Давид все колесницы, оставив из них [только] сто.
\par 5 Сирияне Дамасские пришли было на помощь к Адраазару, царю Сувскому, но Давид поразил двадцать две тысячи Сириян.
\par 6 И поставил Давид [охранное войско] в Сирии Дамасской, и сделались Сирияне рабами Давида, принося ему дань. И помогал Господь Давиду везде, куда он ни ходил.
\par 7 И взял Давид золотые щиты, которые были у рабов Адраазара, и принес их в Иерусалим.
\par 8 А из Тивхавы и Куна, городов Адраазаровых, взял Давид весьма много меди. Из нее Соломон сделал медное море и столбы и медные сосуды.
\par 9 И услышал Фой, царь Имафа, что Давид поразил все войско Адраазара, царя Сувского.
\par 10 И послал Иорама, сына своего, к царю Давиду, приветствовать его и благодарить за то, что он воевал с Адраазаром и поразил его, ибо Фой был в войне с Адраазаром, --и [с ним] всякие сосуды золотые, серебряные и медные.
\par 11 И посвятил их царь Давид Господу вместе с серебром и золотом, которое он взял от всех народов: от Идумеян, Моавитян, Аммонитян, Филистимлян и от Амаликитян.
\par 12 И Авесса, сын Саруи, поразил Идумеян на долине Соляной восемнадцать тысяч;
\par 13 и поставил в Идумее охранное войско, и сделались все Идумеяне рабами Давиду. Господь помогал Давиду везде, куда он ни ходил.
\par 14 И царствовал Давид над всем Израилем, и творил суд и правду всему народу своему.
\par 15 Иоав, сын Саруи, [был] начальником войска, Иосафат, сын Ахилуда, дееписателем,
\par 16 Садок, сын Ахитува, и Авимелех, сын Авиафара, священниками, а Суса писцом,
\par 17 Ванея, сын Иодая, над Хелефеями и Фелефеями, а сыновья Давидовы--первыми при царе.

\chapter{19}

\par 1 После сего умер Наас, царь Аммонитский, и воцарился сын его вместо него.
\par 2 И сказал Давид: окажу я милость Аннону, сыну Наасову, за благодеяние, которое отец его оказал мне. И послал Давид послов утешить его об отце его; и пришли слуги Давидовы в землю Аммонитскую, к Аннону, чтобы утешить его.
\par 3 Но князья Аммонитские сказали Аннону: неужели ты думаешь, что Давид из уважения к отцу твоему прислал к тебе утешителей? Не для того ли пришли слуги его к тебе, чтобы разведать и высмотреть землю и разорить ее?
\par 4 И взял Аннон слуг Давидовых и обрил их, и обрезал одежды их наполовину до чресл и отпустил их.
\par 5 И пошли они. И донесено было Давиду о людях сих, и он послал им навстречу, так как они были очень обесчещены; и сказал царь: останьтесь в Иерихоне, пока отрастут бороды ваши, и тогда возвратитесь.
\par 6 Когда Аммонитяне увидели, что они сделались ненавистными Давиду, тогда послал Аннон и Аммонитяне тысячу талантов серебра, чтобы нанять себе колесниц и всадников из Сирии Месопотамской и из Сирии Мааха и из Сувы.
\par 7 И наняли себе тридцать две тысячи колесниц и царя Мааха с народом его, которые пришли и расположились станом пред Медевою. И Аммонитяне собрались из городов своих и выступили на войну.
\par 8 Когда услышал об этом Давид, то послал Иоава со всем войском храбрых.
\par 9 И выступили Аммонитяне и выстроились к сражению у ворот города, а цари, которые пришли, отдельно в поле.
\par 10 Иоав, видя, что предстоит ему сражение спереди и сзади, избрал воинов из всех отборных в Израиле и выстроил [их] против Сириян.
\par 11 А остальную часть народа поручил Авессе, брату своему, чтоб они выстроились против Аммонитян.
\par 12 И сказал он: если Сирияне будут одолевать меня, то ты поможешь мне, а если Аммонитяне будут одолевать тебя, то я помогу тебе.
\par 13 Будь мужествен, и будем твердо стоять за народ наш и за города Бога нашего, --и Господь пусть сделает, что ему угодно.
\par 14 И вступил Иоав и люди, которые были у него, в сражение с Сириянами, и они побежали от него.
\par 15 Аммонитяне же, увидев, что Сирияне бегут, и сами побежали от Авессы, брата его, и ушли в город. И пришел Иоав в Иерусалим.
\par 16 Сирияне, видя, что они поражены Израильтянами, отправили послов и вывели Сириян, которые были по ту сторону реки, и Совак, военачальник Адраазаров, предводительствовал ими.
\par 17 Когда донесли об этом Давиду, он собрал всех Израильтян, перешел Иордан и, придя к ним, выстроился против них; и вступил Давид в сражение с Сириянами, и они сразились с ним.
\par 18 И Сирияне побежали от Израильтян, и истребил Давид у Сириян семь тысяч колесниц и сорок тысяч пеших, и Совака военачальника умертвил.
\par 19 Когда увидели слуги Адраазара, что они поражены Израильтянами, заключили с Давидом мир и подчинились ему. И не хотели Сирияне помогать более Аммонитянам.

\chapter{20}

\par 1 Через год, в то время когда цари выходят [на войну], вывел Иоав войско и стал разорять землю Аммонитян, и пришел и осадил Равву. Давид же оставался в Иерусалиме. Иоав, завоевав Равву, разрушил ее.
\par 2 И взял Давид венец царя их с головы его, и в нем оказалось весу талант золота, и драгоценные камни были на нем; и был он возложен на голову Давида. И добычи очень много вынес из города.
\par 3 А народ, который был в нем, вывел и умерщвлял их пилами, железными молотилами и секирами. Так поступил Давид со всеми городами Аммонитян, и возвратился Давид и весь народ в Иерусалим.
\par 4 После того началась война с Филистимлянами в Газере. Тогда Совохай Хушатянин поразил Сафа, одного из потомков Рефаимов. И они усмирились.
\par 5 И опять была война с Филистимлянами. Тогда Елханам, сын Иаира, поразил Лахмия, брата Голиафова, Гефянина, у которого древко копья было, как навой у ткачей.
\par 6 Было еще сражение в Гефе. Там был один рослый человек, у которого было по шести пальцев, [всего] двадцать четыре. И он также был из потомков Рефаимов.
\par 7 Он поносил Израиля, но Ионафан, сын Шимы, брата Давидова, поразил его.
\par 8 Это были родившиеся от Рефаимов в Гефе, и пали от руки Давида и от руки слуг его.

\chapter{21}

\par 1 И восстал сатана на Израиля, и возбудил Давида сделать счисление Израильтян.
\par 2 И сказал Давид Иоаву и начальствующим в народе: пойдите исчислите Израильтян, от Вирсавии до Дана, и представьте мне, чтоб я знал число их.
\par 3 И сказал Иоав: да умножит Господь народ Свой во сто раз против того, сколько есть его. Не все ли они, господин мой царь, рабы господина моего? Для чего же требует сего господин мой? Чтобы вменилось это в вину Израилю?
\par 4 Но царское слово превозмогло Иоава; и пошел Иоав, и обошел всего Израиля, и пришел в Иерусалим.
\par 5 И подал Иоав Давиду список народной переписи, и было всех Израильтян тысяча тысяч, и сто тысяч мужей, обнажающих меч, и Иудеев--четыреста семьдесят тысяч, обнажающих меч.
\par 6 А левитов и Вениаминян он не исчислял между ними, потому что царское слово противно было Иоаву.
\par 7 И не угодно было в очах Божиих дело сие, и Он поразил Израиля.
\par 8 И сказал Давид Богу: весьма согрешил я, что сделал это. И ныне прости вину раба Твоего, ибо я поступил очень безрассудно.
\par 9 И говорил Господь Гаду, прозорливцу Давидову, и сказал:
\par 10 пойди и скажи Давиду: так говорит Господь: три [наказания] Я предлагаю тебе, избери себе одно из них, --и Я пошлю его на тебя.
\par 11 И пришел Гад к Давиду и сказал ему: так говорит Господь: избирай себе:
\par 12 или три года--голод, или три месяца будешь ты преследуем неприятелями твоими и меч врагов твоих будет досягать [до тебя]; или три дня--меч Господень и язва на земле и Ангел Господень, истребляющий во всех пределах Израиля. Итак, рассмотри, что мне отвечать Пославшему меня с словом.
\par 13 И сказал Давид Гаду: тяжело мне очень, но пусть лучше впаду в руки Господа, ибо весьма велико милосердие Его, только бы не впасть мне в руки человеческие.
\par 14 И послал Господь язву на Израиля, и умерло Израильтян семьдесят тысяч человек.
\par 15 И послал Бог Ангела в Иерусалим, чтобы истреблять его. И когда он начал истреблять, увидел Господь и пожалел о сем бедствии, и сказал Ангелу-истребителю: довольно! теперь опусти руку твою. Ангел же Господень стоял [тогда] над гумном Орны Иевусеянина.
\par 16 И поднял Давид глаза свои, и увидел Ангела Господня, стоящего между землею и небом, с обнаженным в руке его мечом, простертым на Иерусалим; и пал Давид и старейшины, покрытые вретищем, на лица свои.
\par 17 И сказал Давид Богу: не я ли велел исчислить народ? я согрешил, я сделал зло, а эти овцы что сделали? Господи, Боже мой! да будет рука Твоя на мне и на доме отца моего, а не на народе Твоем, чтобы погубить [его].
\par 18 И Ангел Господень сказал Гаду, чтобы тот сказал Давиду: пусть Давид придет и поставит жертвенник Господу на гумне Орны Иевусеянина.
\par 19 И пошел Давид, по слову Гада, которое он говорил именем Господним.
\par 20 Орна обратился, увидел Ангела, и четыре сына его с ним скрылись. Орна молотил тогда пшеницу.
\par 21 И пришел Давид к Орне. Орна, взглянув и увидев Давида, вышел из гумна и поклонился Давиду лицем до земли.
\par 22 И сказал Давид Орне: отдай мне место под гумном, я построю на нем жертвенник Господу; за настоящую цену отдай мне его, чтобы прекратилось истребление народа.
\par 23 И сказал Орна Давиду: возьми себе; пусть делает господин мой царь что ему угодно; вот я отдаю и волов на всесожжение, и молотильные орудия на дрова, и пшеницу на приношение; все это отдаю даром.
\par 24 И сказал царь Давид Орне: нет, я хочу купить у тебя за настоящую цену, ибо не стану я приносить твоей собственности Господу, и не буду приносить во всесожжение [взятого] даром.
\par 25 И дал Давид Орне за это место шестьсот сиклей золота.
\par 26 И соорудил там Давид жертвенник Господу и вознес всесожжения и мирные жертвы; и призвал Господа, и Он услышал его, [послав] огонь с неба на жертвенник всесожжения.
\par 27 И сказал Господь Ангелу: возврати меч твой в ножны его.
\par 28 В это время Давид, видя, что Господь услышал его на гумне Орны Иевусеянина, принес там жертву.
\par 29 Скиния же Господня, которую сделал Моисей в пустыне, и жертвенник всесожжения [находились] в то время на высоте в Гаваоне.
\par 30 И не мог Давид пойти туда, чтобы взыскать Бога, потому что устрашен был мечом Ангела Господня.

\chapter{22}

\par 1 И сказал Давид: вот дом Господа Бога и вот жертвенник для всесожжений Израиля.
\par 2 И приказал Давид собрать пришельцев, находившихся в земле Израильской, и поставил каменотесов, чтобы обтесывать камни для построения дома Божия.
\par 3 И множество железа для гвоздей к дверям ворот и для связей заготовил Давид, и множество меди без весу,
\par 4 и кедровых дерев без счету, потому что Сидоняне и Тиряне доставили Давиду множество кедровых дерев.
\par 5 И сказал Давид: Соломон, сын мой, молод и малосилен, а дом, который следует выстроить для Господа, должен быть весьма величествен, на славу и украшение пред всеми землями: итак буду я заготовлять для него. И заготовил Давид до смерти своей много.
\par 6 И призвал Соломона, сына своего, и завещал ему построить дом Господу Богу Израилеву.
\par 7 И сказал Давид Соломону: сын мой! у меня было на сердце построить дом во имя Господа, Бога моего,
\par 8 но было ко мне слово Господне, и сказано: `ты пролил много крови и вел большие войны; ты не должен строить дома имени Моему, потому что пролил много крови на землю пред лицем Моим.
\par 9 Вот, у тебя родится сын: он будет человек мирный; Я дам ему покой от всех врагов его кругом: посему имя ему будет Соломон. И мир и покой дам Израилю во дни его.
\par 10 Он построит дом имени Моему, и он будет Мне сыном, а Я ему отцом, и утвержу престол царства его над Израилем навек'.
\par 11 И ныне, сын мой! да будет Господь с тобою, чтобы ты был благоуспешен и построил дом Господу Богу твоему, как Он говорил о тебе.
\par 12 Да даст тебе Господь смысл и разум, и поставит тебя над Израилем; и соблюди закон Господа Бога твоего.
\par 13 Тогда ты будешь благоуспешен, если будешь стараться исполнять уставы и законы, которые заповедал Господь Моисею для Израиля. Будь тверд и мужествен, не бойся и не унывай.
\par 14 И вот, я при скудости моей приготовил для дома Господня сто тысяч талантов золота и тысячу тысяч талантов серебра, а меди и железу нет веса, потому что их множество; и дерева и камни я также заготовил, а ты еще прибавь к этому.
\par 15 У тебя множество рабочих, и каменотесов, резчиков и плотников, и всяких способных на всякое дело;
\par 16 золоту, серебру и меди и железу нет счета: начни и делай; Господь будет с тобою.
\par 17 И завещал Давид всем князьям Израилевым помогать Соломону, сыну его:
\par 18 не с вами ли Господь Бог наш, давший вам покой со всех сторон? потому что Он предал в руки мои жителей земли, и покорилась земля пред Господом и пред народом Его.
\par 19 Итак расположите сердце ваше и душу вашу к тому, чтобы взыскать Господа Бога вашего. Встаньте и постройте святилище Господу Богу, чтобы перенести ковчег завета Господня и священные сосуды Божии в дом, созидаемый имени Господню.

\chapter{23}

\par 1 Давид, состарившись и насытившись [жизнью], воцарил над Израилем сына своего Соломона.
\par 2 И собрал всех князей Израилевых и священников и левитов,
\par 3 и исчислены были левиты, от тридцати лет и выше, и было число их, считая поголовно, тридцать восемь тысяч человек.
\par 4 Из них [назначены] для дела в доме Господнем двадцать четыре тысячи, писцов же и судей шесть тысяч,
\par 5 и четыре тысячи привратников, и четыре тысячи прославляющих Господа на [музыкальных] орудиях, которые он сделал для прославления.
\par 6 И разделил их Давид на череды по сынам Левия--Гирсону, Каафу и Мерари.
\par 7 Из Гирсонян--Лаедан и Шимей.
\par 8 Сыновья Лаедана: первый Иехиил, Зефам и Иоиль, трое.
\par 9 Сыновья Шимея: Шеломиф, Хазиил и Гаран, трое. Они главы поколений Лаедановых.
\par 10 Еще сыновья Шимея: Иахаф, Зиза, Иеуш и Берия. Это сыновья Шимея, четверо.
\par 11 Иахаф был главным, Зиза вторым; Иеуш и Берия имели детей немного, и потому они были в одном счете при доме отца.
\par 12 Сыновья Каафа: Амрам, Ицгар, Хеврон и Озиил, четверо.
\par 13 Сыновья Амрама: Аарон и Моисей. Аарон отделен был на посвящение ко Святому Святых, он и сыновья его, на веки, чтобы совершать курение пред лицем Господа, чтобы служить Ему и благословлять именем Его на веки.
\par 14 А Моисей, человек Божий, [и] сыновья его причтены к колену Левиину.
\par 15 Сыновья Моисея: Гирсон и Елиезер.
\par 16 Сыновья Гирсона: первый был Шевуил.
\par 17 Сыновья Елиезера были: первый Рехавия. И не было у Елиезера других сыновей; у Рехавии же было очень много сыновей.
\par 18 Сыновья Ицгара: первый Шеломиф.
\par 19 Сыновья Хеврона: первый Иерия и второй Амария, третий Иахазиил и четвертый Иекамам.
\par 20 Сыновья Озиила: первый Миха и второй Ишшия.
\par 21 Сыновья Мерарины: Махли и Муши. Сыновья Махлия: Елеазар и Кис.
\par 22 И умер Елеазар, и не было у него сыновей, а только дочери; и взяли их за себя сыновья Киса, братья их.
\par 23 Сыновья Мушия: Махли, Едер и Иремоф--трое.
\par 24 Вот сыновья Левиины, по домам отцов их, главы семейств, по именному счислению их поголовно, которые отправляли дела служения в доме Господнем, от двадцати лет и выше.
\par 25 Ибо Давид сказал: Господь, Бог Израилев, дал покой народу Своему и водворил его в Иерусалиме на веки,
\par 26 и левитам не нужно носить скинию и всякие вещи ее для служения в ней.
\par 27 Посему, по последним повелениям Давида, исчислены левиты от двадцати лет и выше,
\par 28 чтоб они были при сынах Аароновых, для служения дому Господню, во дворе и в пристройках, для соблюдения чистоты всего святилища и для исполнения всякой службы при доме Божием,
\par 29 для наблюдения за хлебами предложения и пшеничною мукою для хлебного приношения и пресными лепешками, за печеным, жареным и за всякою мерою и весом,
\par 30 и чтобы становились каждое утро благодарить и славословить Господа, также и вечером,
\par 31 и при всех всесожжениях, возносимых Господу в субботы, в новомесячия и в праздники по числу, как предписано о них, --постоянно пред лицем Господа,
\par 32 и чтобы охраняли скинию откровения и святилище и сынов Аароновых, братьев своих, при службах дому Господню.

\chapter{24}

\par 1 И вот распределения сыновей Аароновых: сыновья Аарона: Надав, Авиуд, Елеазар и Ифамар.
\par 2 Надав и Авиуд умерли прежде отца своего, сыновей же не было у них, и потому священствовали Елеазар и Ифамар.
\par 3 И распределил их Давид--Садока из сыновей Елеазара, и Ахимелеха из сыновей Ифамара, поочередно на службу их.
\par 4 И нашлось, что между сынами Елеазара глав поколений более, нежели между сынами Ифамара. И он распределил их [так]: из сынов Елеазара шестнадцать глав семейств, а из сынов Ифамара восемь.
\par 5 Распределял же их по жребиям, потому что главными во святилище и главными пред Богом были из сынов Елеазара и из сынов Ифамара,
\par 6 и записывал их Шемаия, сын Нафанаила, писец из левитов, пред лицем царя и князей и пред священником Садоком и Ахимелехом, сыном Авиафара, и пред главами семейств священнических и левитских: брали [при] [бросании жребия] одно семейство из [рода] Елеазарова, потом брали из [рода] Ифамарова.
\par 7 И вышел первый жребий Иегоиариву, второй Иедаии,
\par 8 третий Хариму, четвертый Сеориму,
\par 9 пятый Малхию, шестой Миямину,
\par 10 седьмой Гаккоцу, восьмой Авии,
\par 11 девятый Иешую, десятый Шехании,
\par 12 одиннадцатый Елиашиву, двенадцатый Иакиму,
\par 13 тринадцатый Хушаю, четырнадцатый Иешеваву,
\par 14 пятнадцатый Вилге, шестнадцатый Имеру,
\par 15 семнадцатый Хезиру, восемнадцатый Гапицецу,
\par 16 девятнадцатый Петахии, двадцатый Иезекиилю,
\par 17 двадцать первый Иахину, двадцать второй Гамулу,
\par 18 двадцать третий Делаии, двадцать четвертый Маазии.
\par 19 Вот порядок их при служении их, как [им] приходить в дом Господень, по уставу их чрез Аарона, отца их, как заповедал ему Господь Бог Израилев.
\par 20 У прочих сыновей Левия--[распределение]: из сынов Амрама: Шуваил; из сынов Шуваила: Иедия;
\par 21 от Рехавии: из сынов Рехавии Ишшия был первый;
\par 22 от Ицгара: Шеломоф; из сыновей Шеломофа: Иахав;
\par 23 из сыновей [Хеврона]: первый Иерия, второй Амария, третий Иахазиил, четвертый Иекамам.
\par 24 [Из] сыновей Озиила: Миха; из сыновей Михи: Шамир.
\par 25 Брат Михи Ишшия; из сыновей Ишшии: Захария.
\par 26 Сыновья Мерари: Махли и Муши; [из] сыновей Иаазии: Бено.
\par 27 [Из] сыновей Мерари у Иаазии: Бено и Шогам, и Заккур и Иври.
\par 28 У Махлия--Елеазар; у него сыновей не было.
\par 29 У Киса: [из] сыновей Киса: Иерахмиил;
\par 30 сыновья Мушия: Махли, Едер и Иеримоф. Вот сыновья левитов по поколениям их.
\par 31 Бросали и они жребий, наравне с братьями своими, сыновьями Аароновыми, пред лицем царя Давида и Садока и Ахимелеха, и глав семейств священнических и левитских: глава семейства наравне с меньшим братом своим.

\chapter{25}

\par 1 И отделил Давид и начальники войска на службу сыновей Асафа, Емана и Идифуна, чтобы они провещавали на цитрах, псалтирях и кимвалах; и были отчислены они на дело служения своего:
\par 2 из сыновей Асафа: Заккур, Иосиф, Нефания и Ашарела сыновья Асафа, под руководством Асафа, игравшего по наставлению царя.
\par 3 От Идифуна сыновья Идифуна: Гедалия, Цери, Исаия, Семей, Хашавия и Маттафия, шестеро, под руководством отца своего Идифуна, игравшего на цитре во славу и хвалу Господа.
\par 4 От Емана сыновья Емана: Буккия, Матфания, Озиил, Шевуил и Иеримоф, Ханания, Ханани, Елиафа, Гиддалти, Ромамти-Езер, Иошбекаша, Маллофи, Гофир и Махазиоф.
\par 5 Все эти сыновья Емана, прозорливца царского, по словам Божиим, чтобы возвышать славу его. И дал Бог Еману четырнадцать сыновей и трех дочерей.
\par 6 Все они под руководством отца своего пели в доме Господнем с кимвалами, псалтирями и цитрами в служении в доме Божием, по указанию царя, или Асафа, Идифуна и Емана.
\par 7 И было число их с братьями их, обученными петь пред Господом, всех знающих [сие дело], двести восемьдесят восемь.
\par 8 И бросили они жребий о череде служения, малый наравне с большим, учители [наравне] с учениками.
\par 9 И вышел первый жребий Асафу, для Иосифа; второй Гедалии с братьями его и сыновьями его; их было двенадцать;
\par 10 третий Заккуру с сыновьями его и братьями его; их--двенадцать;
\par 11 четвертый Ицрию с сыновьями его и братьями его; их--двенадцать;
\par 12 пятый Нефании с сыновьями его и братьями его; их--двенадцать;
\par 13 шестой Буккии с сыновьями его и братьями его; их--двенадцать;
\par 14 седьмой Иесареле с сыновьями его и братьями его; их--двенадцать;
\par 15 восьмой Исаии с сыновьями его и братьями его; их--двенадцать;
\par 16 девятый Матфании с сыновьями его и братьями его; их--двенадцать;
\par 17 десятый Шимею с сыновьями его и братьями его; их--двенадцать;
\par 18 одиннадцатый Азариилу с сыновьями его и братьями его; их--двенадцать;
\par 19 двенадцатый Хашавии с сыновьями его и братьями его; их--двенадцать;
\par 20 тринадцатый Шуваилу с сыновьями его и братьями его; их--двенадцать;
\par 21 четырнадцатый Маттафии с сыновьями его и братьями его; их--двенадцать;
\par 22 пятнадцатый Иеримофу с сыновьями его и братьями его; их--двенадцать;
\par 23 шестнадцатый Ханании с сыновьями его и братьями его; их--двенадцать;
\par 24 семнадцатый Иошбекаше с сыновьями его и братьями его; их--двенадцать;
\par 25 восемнадцатый Ханани с сыновьями его и братьями его; их--двенадцать;
\par 26 девятнадцатый Маллофию с сыновьями его и братьями его; их--двенадцать;
\par 27 двадцатый Елиафе с сыновьями его и братьями его; их--двенадцать;
\par 28 двадцать первый Гофиру с сыновьями его и братьями его; их--двенадцать;
\par 29 двадцать второй Гиддалтию с сыновьями его и братьями его; их--двенадцать;
\par 30 двадцать третий Махазиофу с сыновьями его и братьями его; их--двенадцать;
\par 31 двадцать четвертый Ромамти-Езеру с сыновьями его и братьями его; их--двенадцать.

\chapter{26}

\par 1 Вот распределение привратников: из Кореян: Мешелемия, сын Корея, из сыновей Асафовых.
\par 2 Сыновья Мешелемии: первенец Захария, второй Иедиаил, третий Зевадия, четвертый Иафниил,
\par 3 пятый Елам, шестой Иегоханан, седьмой Елиегоэнай.
\par 4 Сыновья Овед-Едома: первенец Шемаия, второй Иегозавад, третий Иоах, четвертый Сахар, пятый Нафанаил,
\par 5 шестой Аммиил, седьмой Иссахар, восьмой Пеульфай, потому что Бог благословил его.
\par 6 У сына его Шемаии родились также сыновья, начальствовавшие в своем роде, потому что они были люди сильные.
\par 7 Сыновья Шемаии: Офни, Рефаил, Овед и Елзавад, братья его, люди сильные, Елия, Семахия.
\par 8 Все они из сыновей Овед-Едома; они и сыновья их, и братья их были люди прилежные и к службе способные: их было у Овед-Едома шестьдесят два.
\par 9 У Мешелемии сыновей и братьев, людей способных, [было] восемнадцать.
\par 10 У Хосы, из сыновей Мерариных, сыновья: Шимри главный, --хотя он не был первенцем, но отец его поставил его главным;
\par 11 второй Хелкия, третий Тевалия, четвертый Захария; всех сыновей и братьев у Хосы было тринадцать.
\par 12 Вот распределение привратников по главам семейств, способных на службу вместе с братьями их, для служения в доме Господнем.
\par 13 И бросили они жребии, как малый, так и большой, по своим семействам, на каждые ворота.
\par 14 И выпал жребий на восток Шелемии; и Захарии, сыну его, умному советнику, бросили жребий, и вышел ему жребий на север;
\par 15 Овед-Едому на юг, а сыновьям его при кладовых.
\par 16 Шупиму и Хосе на запад, у ворот Шаллехет, где дорога поднимается и где стража против стражи.
\par 17 К востоку по шести левитов, к северу по четыре, к югу по четыре, а у кладовых по два.
\par 18 К западу у притвора на дороге по четыре, а у самого притвора по два.
\par 19 Вот распределение привратников из сыновей Кореевых и сыновей Мерариных.
\par 20 Левиты же, братья их, [смотрели] за сокровищами дома Божия и за сокровищницами посвященных вещей.
\par 21 Сыновья Лаедана, сына Герсонова--от Лаедана, главы семейств от Лаедана Герсонского: Иехиел.
\par 22 Сыновья Иехиела: Зефам и Иоиль, брат его, [смотрели] за сокровищами дома Господня,
\par 23 вместе с потомками Амрама, Ицгара, Хеврона, Озиила.
\par 24 Шевуил, сын Гирсона, сына Моисеева, [был] главным смотрителем за сокровищницами.
\par 25 У брата его Елиезера сын Рехавия, у него сын Исаия, у него сын Иорам, у него сын Зихрий, у него сын Шеломиф.
\par 26 Шеломиф и братья его [смотрели] за всеми сокровищницами посвященных вещей, которые посвятил царь Давид и главы семейств и тысяченачальники, стоначальники и предводители войска.
\par 27 Из завоеваний и из добыч они посвящали на поддержание дома Господня.
\par 28 И все, что посвятил Самуил пророк, и Саул, сын Киса, и Авенир, сын Нира, и Иоав, сын Саруи, все посвященное [было] на руках у Шеломифа и братьев его.
\par 29 Из племени Ицгарова: Хенания и сыновья его [определены] на внешнее служение у Израильтян, писцами и судьями.
\par 30 Из племени Хевронова: Хашавия и братья его, люди мужественные, тысяча семьсот, имели надзор над Израилем по эту сторону Иордана к западу, по всяким делам [служения] Господня и по службе царской.
\par 31 У племени Хевронова Иерия [был] главою Хевронян, в их родах, в поколениях. В сороковой год царствования Давида они исчислены, и найдены между ними люди мужественные в Иазере Галаадском.
\par 32 И братья его, люди способные, две тысячи семьсот, были главы семейств. Их поставил царь Давид над коленом Рувимовым и Гадовым и полуколеном Манассииным, по всем делам Божиим и делам царя.

\chapter{27}

\par 1 Вот сыны Израилевы по числу их, главы семейств, тысяченачальники и стоначальники и управители, которые по отделениям служили царю во всех делах, приходя и отходя каждый месяц, во все месяцы года. В каждом отделении было их по двадцать четыре тысячи.
\par 2 Над первым отделением, для первого месяца, [начальствовал] Иашовам, сын Завдиила; в его отделении было двадцать четыре тысячи;
\par 3 он [был] из сынов Фареса, главный над всеми военачальниками в первый месяц.
\par 4 Над отделением второго месяца был Додай Ахохиянин; в отделении его был и князь Миклоф, и в его отделении было двадцать четыре тысячи.
\par 5 Третий главный военачальник, для третьего месяца, Ванея, сын Иодая, священника, и в его отделении было двадцать четыре тысячи:
\par 6 этот Ванея--[один] из тридцати храбрых и [начальник] над ними, и в его отделе [находился] Аммизавад, сын его.
\par 7 Четвертый, для четвертого месяца, был Асаил, брат Иоава, и по нем Завадия, сын его, и в его отделении двадцать четыре тысячи.
\par 8 Пятый, для пятого месяца, князь Шамгуф Израхитянин, и в его отделении двадцать четыре тысячи.
\par 9 Шестой, для шестого месяца, Ира, сын Иккеша, Фекоянин, и в его отделении двадцать четыре тысячи.
\par 10 Седьмой, для седьмого месяца, Хелец Пелонитянин, из сынов Ефремовых, и в его отделении двадцать четыре тысячи.
\par 11 Восьмой, для восьмого месяца, Совохай Хушатянин, из племени Зары, и в его отделении двадцать четыре тысячи.
\par 12 Девятый, для девятого месяца, Авиезер Анафофянин, из сыновей Вениаминовых, и в его отделении двадцать четыре тысячи.
\par 13 Десятый, для десятого месяца, Магарай Нетофафянин, из племени Зары, и в его отделении двадцать четыре тысячи.
\par 14 Одиннадцатый, для одиннадцатого месяца, Ванея Пирафонянин, из сынов Ефремовых, и в его отделении двадцать четыре тысячи.
\par 15 Двенадцатый, для двенадцатого месяца, Хелдай Нетофафянин, из потомков Гофониила, и в его отделении двадцать четыре тысячи.
\par 16 А над коленами Израилевыми, --у Рувимлян главным начальником [был] Елиезер, сын Зихри; у Симеона--Сафатия, сын Маахи;
\par 17 у Левия--Хашавия, сын Кемуила; у Аарона--Садок;
\par 18 у Иуды--Елиав, из братьев Давида; у Иссахара--Омри, сын Михаила;
\par 19 у Завулона--Ишмаия, сын Овадии; у Неффалима--Иеримоф, сын Азриила;
\par 20 у сыновей Ефремовых--Осия, сын Азазии; у полуколена Манассиина--Иоиль, сын Федаии;
\par 21 у полуколена Манассии в Галааде--Иддо, сын Захарии; у Вениамина--Иаасиил, сын Авенира;
\par 22 у Дана--Азариил, сын Иерохама. Вот вожди колен Израилевых.
\par 23 Давид не делал счисления тех, которые были от двадцати лет и ниже, потому что Господь сказал, что Он умножит Израиля, как звезды небесные.
\par 24 Иоав, сын Саруи, начал делать счисление, но не кончил. И был за это гнев Божий на Израиля, и не вошло то счисление в летопись царя Давида.
\par 25 Над сокровищами царскими был Азмавеф, сын Адиилов, а над запасами в поле, в городах, и в селах и в башнях--Ионафан, сын Уззии;
\par 26 над занимающимися полевыми работами, земледелием--Езрий, сын Хелува;
\par 27 над виноградниками--Шимей из Рамы, а над запасами вина в виноградниках--Завдий из Шефама;
\par 28 над маслинами и смоковницами в долине--Баал-Ханан Гедеритянин, а над запасами деревянного масла--Иоас;
\par 29 над крупным скотом, пасущимся в Шароне--Шитрай Шаронянин, а над скотом в долинах--Шафат, сын Адлая;
\par 30 над верблюдами--Овил Исмаильтянин; над ослицами--Иехдия Меронифянин;
\par 31 над мелким скотом--Иазиз Агаритянин. Все эти были начальниками над имением, которое [было] у царя Давида.
\par 32 Ионафан, дядя Давидов, [был] советником, человек умный и писец; Иехиил, сын Хахмониев, [был] при сыновьях царя;
\par 33 Ахитофел [был] советником царя; Хусий Архитянин--другом царя;
\par 34 после же Ахитофела Иодай, сын Ванеи, и Авиафар, а Иоав был военачальником у царя.

\chapter{28}

\par 1 И собрал Давид в Иерусалим всех вождей Израильских, начальников колен и начальников отделов, служивших царю, и тысяченачальников, и стоначальников, и заведывавших всем имением и стадами царя и сыновей его с евнухами, военачальников и всех храбрых мужей.
\par 2 И стал Давид царь на ноги свои и сказал: послушайте меня, братья мои и народ мой! [было] у меня на сердце построить дом покоя для ковчега завета Господня и в подножие ногам Бога нашего, и [потребное] для строения я приготовил.
\par 3 Но Бог сказал мне: не строй дома имени Моему, потому что ты человек воинственный и проливал кровь.
\par 4 Однакоже избрал Господь Бог Израилев меня из всего дома отца моего, чтоб быть [мне] царем над Израилем вечно, потому что Иуду избрал Он князем, а в доме Иуды дом отца моего, а из сыновей отца моего меня благоволил поставить царем над всем Израилем,
\par 5 из всех же сыновей моих, --ибо много сыновей дал мне Господь, --Он избрал Соломона, сына моего, сидеть на престоле царства Господня над Израилем,
\par 6 и сказал мне: Соломон, сын твой, построит дом Мой и дворы Мои, потому что Я избрал его Себе в сына, и Я буду ему Отцом;
\par 7 и утвержу царство его на веки, если он будет тверд в исполнении заповедей Моих и уставов Моих, как до сего дня.
\par 8 И теперь пред очами всего Израиля, собрания Господня, и во уши Бога нашего [говорю]: соблюдайте и держитесь всех заповедей Господа Бога вашего, чтобы владеть вам сею доброю землею и оставить ее после себя в наследство детям своим на век;
\par 9 и ты, Соломон, сын мой, знай Бога отца твоего и служи Ему от всего сердца и от всей души, ибо Господь испытует все сердца и знает все движения мыслей. Если будешь искать Его, то найдешь Его, а если оставишь Его, Он оставит тебя навсегда.
\par 10 Смотри же, когда Господь избрал тебя построить дом для святилища, будь тверд и делай.
\par 11 И отдал Давид Соломону, сыну своему, чертеж притвора и домов его, и кладовых его, и горниц его, и внутренних покоев его, и дома для ковчега,
\par 12 и чертеж всего, что было у него на душе, дворов дома Господня и всех комнат кругом, сокровищниц дома Божия и сокровищниц вещей посвященных,
\par 13 и священнических и левитских отделений, и всякого служебного дела в доме Господнем, и всех служебных сосудов дома Господня,
\par 14 золотых вещей, с [означением] веса, для всякого из служебных сосудов, всех вещей серебряных, с [означением] веса, для всякого из сосудов служебных.
\par 15 И дал золота для светильников и золотых лампад их, с означением веса каждого из светильников и лампад его, также светильников серебряных, с означением веса каждого из светильников и лампад его, смотря по служебному назначению каждого светильника;
\par 16 и золота для столов предложения хлебов, для каждого [золотого] стола, и серебра для столов серебряных,
\par 17 и вилок, и чаш и кропильниц из чистого золота, и золотых блюд, с означением веса каждого блюда, и серебряных блюд, с означением веса каждого блюда,
\par 18 и для жертвенника курения из литого золота с означением веса, и устройства колесницы с золотыми херувимами, распростирающими [крылья] и покрывающими ковчег завета Господня.
\par 19 Все сие в письмени от Господа, [говорил Давид, как] Он вразумил меня на все дела постройки.
\par 20 И сказал Давид сыну своему Соломону: будь тверд и мужествен, и приступай к делу, не бойся и не ужасайся, ибо Господь Бог, Бог мой, с тобою; Он не отступит от тебя и не оставит тебя, доколе не совершишь всего дела, требуемого для дома Господня.
\par 21 И вот отделы священников и левитов, для всякой службы при Доме Божием. И у тебя есть для всякого дела усердные люди, искусные для всякой работы, и начальники и весь народ [готовы] на все твои приказания.

\chapter{29}

\par 1 И сказал царь Давид всему собранию: Соломон, сын мой, которого одного избрал Бог, молод и малосилен, а дело сие велико, потому что не для человека здание сие, а для Господа Бога.
\par 2 Всеми силами я заготовил для дома Бога моего золото для золотых вещей и серебро для серебряных, и медь для медных, железо для железных, и дерева для деревянных, камни оникса и [камни] вставные, камни красивые и разноцветные, и всякие дорогие камни, и множество мрамора;
\par 3 и еще по любви моей к дому Бога моего, есть у меня сокровище собственное из золота и серебра, [и его] я отдаю для дома Бога моего, сверх всего, что заготовил я для святого дома:
\par 4 три тысячи талантов золота, золота Офирского, и семь тысяч талантов серебра чистого, для обложения стен в домах,
\par 5 для каждой из золотых вещей, и для каждой из серебряных, и для всякого изделия рук художнических. Не поусердствует ли [еще] кто жертвовать сегодня для Господа?
\par 6 И стали жертвовать начальники семейств и начальники колен Израилевых, и начальники тысяч и сотен, и начальники над имениями царя.
\par 7 И дали на устроение дома Божия пять тысяч талантов и десять тысяч драхм золота, и серебра десять тысяч талантов, и меди восемнадцать тысяч талантов, и железа сто тысяч талантов.
\par 8 И у кого нашлись [дорогие] камни, те отдавали и их в сокровищницу дома Господня, на руки Иехиилу Герсонитянину.
\par 9 И радовался народ усердию их, потому что они от всего сердца жертвовали Господу, также и царь Давид весьма радовался.
\par 10 И благословил Давид Господа пред всем собранием, и сказал Давид: благословен Ты, Господи Боже Израиля, отца нашего, от века и до века!
\par 11 Твое, Господи, величие, и могущество, и слава, и победа и великолепие, и все, [что] на небе и на земле, [Твое]: Твое, Господи, царство, и Ты превыше всего, как Владычествующий.
\par 12 И богатство и слава от лица Твоего, и Ты владычествуешь над всем, и в руке Твоей сила и могущество, и во власти Твоей возвеличить и укрепить все.
\par 13 И ныне, Боже наш, мы славословим Тебя и хвалим величественное имя Твое.
\par 14 Ибо кто я и кто народ мой, что мы имели возможность так жертвовать? Но от Тебя все, и от руки Твоей [полученное] мы отдали Тебе,
\par 15 потому что странники мы пред Тобою и пришельцы, как и все отцы наши, как тень дни наши на земле, и нет ничего прочного.
\par 16 Господи Боже наш! все это множество, которое приготовили мы для построения дома Тебе, святому имени Твоему, от руки Твоей оно, и все Твое.
\par 17 Знаю, Боже мой, что Ты испытуешь сердце и любишь чистосердечие; я от чистого сердца моего пожертвовал все сие, и ныне вижу, что и народ Твой, здесь находящийся, с радостью жертвует Тебе.
\par 18 Господи, Боже Авраама, Исаака и Израиля, отцов наших! сохрани сие навек, [сие] расположение мыслей сердца народа Твоего, и направь сердце их к Тебе.
\par 19 Соломону же, сыну моему, дай сердце правое, чтобы соблюдать заповеди Твои, откровения Твои и уставы Твои, и исполнить все это и построить здание, для которого я сделал приготовление.
\par 20 И сказал Давид всему собранию: благословите Господа Бога нашего. --И благословило все собрание Господа Бога отцов своих, и пало, и поклонилось Господу и царю.
\par 21 И принесли Господу жертвы, и вознесли всесожжения Господу, на другой после сего день: тысячу тельцов, тысячу овнов, тысячу агнцев с их возлияниями, и множество жертв от всего Израиля.
\par 22 И ели и пили пред Господом в тот день, с великою радостью; и в другой раз воцарили Соломона, сына Давидова, и помазали пред Господом в правителя верховного, а Садока во священника.
\par 23 И сел Соломон на престоле Господнем, как царь, вместо Давида, отца своего, и был благоуспешен, и весь Израиль повиновался ему.
\par 24 И все начальники и сильные, также и все сыновья царя Давида подчинились Соломону царю.
\par 25 И возвеличил Господь Соломона пред очами всего Израиля, и даровал ему славу царства, какой не имел прежде его ни один царь у Израиля.
\par 26 И Давид, сын Иессеев, царствовал над всем Израилем.
\par 27 Времени царствования его над Израилем [было] сорок лет: в Хевроне царствовал он семь лет, и в Иерусалиме царствовал тридцать три [года].
\par 28 И умер в доброй старости, насыщенный жизнью, богатством и славою; и воцарился Соломон, сын его, вместо него.
\par 29 Дела царя Давида, первые и последние, описаны в записях Самуила провидца и в записях Нафана пророка и в записях Гада прозорливца,
\par 30 равно и все царствование его, и мужество его, и происшествия, случившиеся с ним и с Израилем и со всеми земными царствами.


\end{document}