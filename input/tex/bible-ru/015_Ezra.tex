\begin{document}

\title{Ezra}

Ezr 1:1  В первый год Кира, царя Персидского, во исполнение слова Господня из уст Иеремии, возбудил Господь дух Кира, царя Персидского, и он повелел объявить по всему царству своему, словесно и письменно:
Ezr 1:2  так говорит Кир, царь Персидский: все царства земли дал мне Господь Бог небесный, и Он повелел мне построить Ему дом в Иерусалиме, что в Иудее.
Ezr 1:3  Кто есть из вас, из всего народа Его, --да будет Бог его с ним, --и пусть он идет в Иерусалим, что в Иудее, и строит дом Господа Бога Израилева, Того Бога, Который в Иерусалиме.
Ezr 1:4  А все оставшиеся во всех местах, где бы тот ни жил, пусть помогут ему жители места того серебром и золотом и [иным] имуществом, и скотом, с доброхотным даянием для дома Божия, что в Иерусалиме.
Ezr 1:5  И поднялись главы поколений Иудиных и Вениаминовых, и священники и левиты, всякий, [в ком] возбудил Бог дух его, чтобы пойти строить дом Господень, который в Иерусалиме.
Ezr 1:6  И все соседи их вспомоществовали им серебряными сосудами, золотом, [иным] имуществом, и скотом, и дорогими вещами, сверх всякого доброхотного даяния [для храма].
Ezr 1:7  И царь Кир вынес сосуды дома Господня, которые Навуходоносор взял из Иерусалима и положил в доме бога своего, --
Ezr 1:8  и вынес их Кир, царь Персидский, рукою Мифредата сокровищехранителя, а он счетом сдал их Шешбацару князю Иудину.
Ezr 1:9  И вот число их: блюд золотых тридцать, блюд серебряных тысяча, ножей двадцать девять,
Ezr 1:10  чаш золотых тридцать, чаш серебряных двойных четыреста десять, других сосудов тысяча:
Ezr 1:11  всех сосудов, золотых и серебряных, пять тысяч четыреста. Все [это] взял [с собою] Шешбацар, при отправлении переселенцев из Вавилона в Иерусалим.
Ezr 2:1  Вот сыны страны из пленников переселения, которых Навуходоносор, царь Вавилонский, отвел в Вавилон, возвратившиеся в Иерусалим и Иудею, каждый в свой город, --
Ezr 2:2  пришедшие с Зоровавелем, Иисусом, Неемиею, Сараием, Реелаем, Мардохеем, Билшаном, Мисфаром, Бигваем, Рехумом, Вааном. Число людей народа Израилева:
Ezr 2:3  сыновей Пароша две тысячи сто семьдесят два;
Ezr 2:4  сыновей Сафатии триста семьдесят два;
Ezr 2:5  сыновей Араха семьсот семьдесят пять;
Ezr 2:6  сыновей Пахаф-Моава, из сыновей Иисуса [и] Иоава, две тысячи восемьсот двенадцать;
Ezr 2:7  сыновей Елама тысяча двести пятьдесят четыре;
Ezr 2:8  сыновей Заттуя девятьсот сорок пять;
Ezr 2:9  сыновей Закхая семьсот шестьдесят;
Ezr 2:10  сыновей Вания шестьсот сорок два;
Ezr 2:11  сыновей Бебая шестьсот двадцать три;
Ezr 2:12  сыновей Азгада тысяча двести двадцать два;
Ezr 2:13  сыновей Адоникама шестьсот шестьдесят шесть;
Ezr 2:14  сыновей Бигвая две тысячи пятьдесят шесть;
Ezr 2:15  сыновей Адина четыреста пятьдесят четыре;
Ezr 2:16  сыновей Атера, из [дома] Езекии, девяносто восемь;
Ezr 2:17  сыновей Бецая триста двадцать три;
Ezr 2:18  сыновей Иоры сто двенадцать;
Ezr 2:19  сыновей Хашума двести двадцать три;
Ezr 2:20  сыновей Гиббара девяносто пять;
Ezr 2:21  уроженцев Вифлеема сто двадцать три;
Ezr 2:22  жителей Нетофы пятьдесят шесть;
Ezr 2:23  жителей Анафофа сто двадцать восемь;
Ezr 2:24  уроженцев Азмавефа сорок два;
Ezr 2:25  уроженцев Кириаф-Иарима, Кефиры и Беерофа семьсот сорок три;
Ezr 2:26  уроженцев Рамы и Гевы шестьсот двадцать один;
Ezr 2:27  жителей Михмаса сто двадцать два;
Ezr 2:28  жителей Вефиля и Гая двести двадцать три;
Ezr 2:29  уроженцев Нево пятьдесят два;
Ezr 2:30  уроженцев Магбиша сто пятьдесят шесть;
Ezr 2:31  сыновей другого Елама тысяча двести пятьдесят четыре;
Ezr 2:32  сыновей Харима триста двадцать;
Ezr 2:33  уроженцев Лидды, Хадида и Оно семьсот двадцать пять;
Ezr 2:34  уроженцев Иерихона триста сорок пять;
Ezr 2:35  уроженцев Сенаи три тысячи шестьсот тридцать.
Ezr 2:36  Священников: сыновей Иедаии, из дома Иисусова, девятьсот семьдесят три;
Ezr 2:37  сыновей Иммера тысяча пятьдесят два;
Ezr 2:38  сыновей Пашхура тысяча двести сорок семь;
Ezr 2:39  сыновей Харима тысяча семнадцать.
Ezr 2:40  Левитов: сыновей Иисуса и Кадмиила, из сыновей Годавии, семьдесят четыре;
Ezr 2:41  певцов: сыновей Асафа сто двадцать восемь;
Ezr 2:42  сыновей привратников: сыновья Шаллума, сыновья Атера, сыновья Талмона, сыновья Аккува, сыновья Хатиты, сыновья Шовая, --всего сто тридцать девять.
Ezr 2:43  Нефинеев: сыновья Цихи, сыновья Хасуфы, сыновья Таббаофа,
Ezr 2:44  сыновья Кероса, сыновья Сиаги, сыновья Фадона,
Ezr 2:45  сыновья Лебаны, сыновья Хагабы, сыновья Аккува,
Ezr 2:46  сыновья Хагава, сыновья Шамлая, сыновья Ханана,
Ezr 2:47  сыновья Гиддела, сыновья Гахара, сыновья Реаии,
Ezr 2:48  сыновья Рецина, сыновья Некоды, сыновья Газзама,
Ezr 2:49  сыновья Уззы, сыновья Пасеаха, сыновья Бесая,
Ezr 2:50  сыновья Асны, сыновья Меунима, сыновья Нефисима,
Ezr 2:51  сыновья Бакбука, сыновья Хакуфы, сыновья Хархура,
Ezr 2:52  сыновья Бацлуфа, сыновья Мехиды, сыновья Харши,
Ezr 2:53  сыновья Баркоса, сыновья Сисры, сыновья Фамаха,
Ezr 2:54  сыновья Нециаха, сыновья Хатифы;
Ezr 2:55  сыновья рабов Соломоновых: сыновья Сотая, сыновья Гассоферефа, сыновья Феруды,
Ezr 2:56  сыновья Иаалы, сыновья Даркона, сыновья Гиддела,
Ezr 2:57  сыновья Сефатии, сыновья Хаттила, сыновья Похереф-Гаццебайима, сыновья Амия, --
Ezr 2:58  всего--нефинеев и сыновей рабов Соломоновых триста девяносто два.
Ezr 2:59  И вот вышедшие из Тел-Мелаха, Тел-Харши, Херуб-Аддан-Иммера, которые не могли показать о поколении своем и о племени своем--от Израиля ли они:
Ezr 2:60  сыновья Делайи, сыновья Товии, сыновья Некоды, шестьсот пятьдесят два.
Ezr 2:61  И из сыновей священнических: сыновья Хабайи, сыновья Гаккоца, сыновья Верзеллия, который взял жену из дочерей Верзеллия Галаадитянина и стал называться именем их.
Ezr 2:62  Они искали своей записи родословной, и не нашлось ее, а [потому] исключены из священства.
Ezr 2:63  И Тиршафа сказал им, чтоб они не ели великой святыни, доколе не восстанет священник с уримом и туммимом.
Ezr 2:64  Все общество вместе [состояло] из сорока двух тысяч трехсот шестидесяти [человек],
Ezr 2:65  кроме рабов их и рабынь их, которых [было] семь тысяч триста тридцать семь; и при них певцов и певиц двести.
Ezr 2:66  Коней у них семьсот тридцать шесть, лошаков у них двести сорок пять;
Ezr 2:67  верблюдов у них четыреста тридцать пять, ослов шесть тысяч семьсот двадцать.
Ezr 2:68  Из глав поколений [некоторые], придя к дому Господню, что в Иерусалиме, доброхотно жертвовали на дом Божий, чтобы восстановить его на основании его.
Ezr 2:69  По достатку своему, они дали в сокровищницу на [производство] работ шестьдесят одну тысячу драхм золота и пять тысяч мин серебра и сто священнических одежд.
Ezr 2:70  И стали жить священники и левиты, и народ и певцы, и привратники и нефинеи в городах своих, и весь Израиль в городах своих.
Ezr 3:1  Когда наступил седьмой месяц, и сыны Израилевы [уже были] в городах, тогда собрался народ, как один человек, в Иерусалиме.
Ezr 3:2  И встал Иисус, сын Иоседеков, и братья его священники, и Зоровавель, сын Салафиилов, и братья его, и соорудили они жертвенник Богу Израилеву, чтобы возносить на нем всесожжения, как написано в законе Моисея, человека Божия.
Ezr 3:3  И поставили жертвенник на основании его, так как они [были] в страхе от иноземных народов; и стали возносить на нем всесожжения Господу, всесожжения утренние и вечерние.
Ezr 3:4  И совершили праздник кущей, как предписано, с ежедневным всесожжением в определенном числе, по уставу [каждого] дня.
Ezr 3:5  И после того [совершали] всесожжение постоянное, и в новомесячия, и во все праздники, посвященные Господу, и добровольное приношение Господу от всякого усердствующего.
Ezr 3:6  С первого же дня седьмого месяца начали возносить всесожжения Господу. А храму Господню [еще] не было положено основание.
Ezr 3:7  И стали выдавать серебро каменотесам и плотникам, и пищу и питье и масло Сидонянам и Тирянам, чтоб они доставляли кедровый лес с Ливана по морю в Яфу, с дозволения им Кира, царя Персидского.
Ezr 3:8  Во второй год по приходе своем к дому Божию в Иерусалим, во второй месяц Зоровавель, сын Салафиилов, и Иисус, сын Иоседеков, и прочие братья их, священники и левиты, и все пришедшие из плена в Иерусалим положили начало и поставили левитов от двадцати лет и выше для надзора за работами дома Господня.
Ezr 3:9  И стали Иисус, сыновья его и братья его, Кадмиил и сыновья его, сыновья Иуды, как один [человек], для надзора за производителями работ в доме Божием, [а также и] сыновья Хенадада, сыновья их и братья их левиты.
Ezr 3:10  Когда строители положили основание храму Господню, тогда поставили священников в облачении их с трубами и левитов, сыновей Асафовых, с кимвалами, чтобы славить Господа по уставу Давида, царя Израилева.
Ezr 3:11  И начали они попеременно петь: `хвалите' и: `славьте Господа', `ибо--благ, ибо вовек милость Его к Израилю'. И весь народ восклицал громогласно, славя Господа за то, что положено основание дома Господня.
Ezr 3:12  Впрочем многие из священников и левитов и глав поколений, старики, которые видели прежний храм, при основании этого храма пред глазами их, плакали громко, но многие и восклицали от радости громогласно.
Ezr 3:13  И не мог народ распознать восклицаний радости от воплей плача народного, потому что народ восклицал громко, и голос слышен был далеко.
Ezr 4:1  И услышали враги Иуды и Вениамина, что возвратившиеся из плена строят храм Господу Богу Израилеву;
Ezr 4:2  и пришли они к Зоровавелю и к главам поколений, и сказали им: будем и мы строить с вами, потому что мы, как и вы, прибегаем к Богу вашему, и Ему приносим жертвы от дней Асардана, царя Сирийского, который перевел нас сюда.
Ezr 4:3  И сказал им Зоровавель и Иисус и прочие главы поколений Израильских: не строить вам вместе с нами дом нашему Богу; мы одни будем строить [дом] Господу, Богу Израилеву, как повелел нам царь Кир, царь Персидский.
Ezr 4:4  И стал народ земли той ослаблять руки народа Иудейского и препятствовать ему в строении;
Ezr 4:5  и подкупали против них советников, чтобы разрушить предприятие их, во все дни Кира, царя Персидского, и до царствования Дария, царя Персидского.
Ezr 4:6  А в царствование Ахашвероша, в начале царствования его, написали обвинение на жителей Иудеи и Иерусалима.
Ezr 4:7  И во дни Артаксеркса писали Бишлам, Мифредат, Табеел и прочие товарищи их к Артаксерксу, царю Персидскому. Письмо же написано [было] буквами Сирийскими и на Сирийском языке.
Ezr 4:8  Рехум советник и Шимшай писец писали одно письмо против Иерусалима к царю Артаксерксу такое:
Ezr 4:9  Тогда-то. Рехум советник и Шимшай писец и прочие товарищи их, --Динеи и Афарсафхеи, Тарпелеи, Апарсы, Арехьяне, Вавилоняне, Сусанцы, Даги, Еламитяне,
Ezr 4:10  и прочие народы, которых переселил Аснафар, великий и славный и поселил в городах Самарийских и в прочих [городах] за рекою, и прочее.
Ezr 4:11  И вот список с письма, которое послали к нему: Царю Артаксерксу--рабы твои, люди, [живущие] за рекою, и прочее.
Ezr 4:12  Да будет известно царю, что Иудеи, которые вышли от тебя, пришли к нам в Иерусалим, строят [этот] мятежный и негодный город, и стены делают, и основания [их уже] исправили.
Ezr 4:13  Да будет же известно царю, что если этот город будет построен и стены восстановлены, то [ни] подати, [ни] налога, ни пошлины не будут давать, и царской казне сделан будет ущерб.
Ezr 4:14  Так как мы едим соль от дворца царского, и ущерб для царя не можем видеть, поэтому мы посылаем донесение к царю:
Ezr 4:15  пусть поищут в памятной книге отцов твоих, --и найдешь в книге памятной, и узнаешь, что город сей--город мятежный и вредный для царей и областей, и [что] отпадения бывали в нем издавна, за что город сей и опустошен.
Ezr 4:16  Посему мы уведомляем царя, что если город сей будет достроен и стены его доделаны, то после этого не будет у тебя владения за рекою.
Ezr 4:17  Царь послал ответ Рехуму советнику и Шимшаю писцу и прочим товарищам их, которые живут в Самарии и [в] прочих [городах] заречных: Мир... и прочее.
Ezr 4:18  Письмо, которое вы прислали нам, внятно прочитано предо мною;
Ezr 4:19  и от меня дано повеление, --и разыскивали, и нашли, что город этот издавна восставал против царей, и производились в нем мятежи и волнения,
Ezr 4:20  и [что были] в Иерусалиме цари могущественные и владевшие всем заречьем, и им давали подать, налоги и пошлины.
Ezr 4:21  Итак дайте приказание, чтобы люди сии перестали работать, и [чтобы] город сей не строился, доколе от меня не будет дано повеление.
Ezr 4:22  И будьте осторожны, чтобы не сделать в этом недосмотра. К чему допускать размножение вредного в ущерб царям?
Ezr 4:23  Как скоро это письмо царя Артаксеркса было прочитано пред Рехумом и Шимшаем писцом и товарищами их, они немедленно пошли в Иерусалим к Иудеям, и сильною вооруженною рукою остановили работу их.
Ezr 4:24  Тогда остановилась работа при доме Божием, который в Иерусалиме, и остановка сия продолжалась до второго года царствования Дария, царя Персидского.
Ezr 5:1  Но пророк Аггей и пророк Захария, сын Адды, говорили Иудеям, которые в Иудее и Иерусалиме, пророческие речи во имя Бога Израилева.
Ezr 5:2  Тогда встали Зоровавель, сын Салафиилов, и Иисус, сын Иоседеков, и начали строить дом Божий в Иерусалиме, и с ними пророки Божии, подкреплявшие их.
Ezr 5:3  В то время пришел к ним Фафнай, заречный областеначальник, и Шефар-Бознай и товарищи их, и так сказали им: кто дал вам разрешение строить дом сей и доделывать стены сии?
Ezr 5:4  Тогда мы сказали им имена тех людей, которые строят это здание.
Ezr 5:5  Но око Бога их было над старейшинами Иудейскими, и те не возбраняли им, доколе дело не отправили к Дарию, и доколе не пришло решение по этому делу.
Ezr 5:6  Вот содержание письма, которое послал Фафнай, заречный областеначальник, и Шефар-Бознай с товарищами своими Афарсахеями, которые за рекою, к царю Дарию.
Ezr 5:7  В донесении, которое они послали к нему, вот что написано: Дарию царю--всякий мир!
Ezr 5:8  Да будет известно царю, что мы ходили в Иудейскую область, к дому Бога великого; и строится он из больших камней, и дерево вкладывается в стены; и работа сия производится быстро и с успехом идет в руках их.
Ezr 5:9  Тогда мы спросили у старейшин тех и так сказали им: кто дал вам разрешение строить дом сей и стены сии доделывать?
Ezr 5:10  И сверх того об именах их мы спросили их, чтобы дать знать тебе и написать имена тех людей, которые главными у них.
Ezr 5:11  И они ответили нам такими словами: мы рабы Бога неба и земли и строим дом, который был построен за много лет прежде сего, --и великий царь у Израиля строил его и довершил его.
Ezr 5:12  Когда же отцы наши прогневали Бога небесного, Он предал их в руку Навуходоносора, царя Вавилонского, Халдеянина; и дом сей он разрушил, и народ переселил в Вавилон.
Ezr 5:13  Но в первый год Кира, царя Вавилонского, царь Кир дал разрешение построить сей дом Божий;
Ezr 5:14  да и сосуды дома Божия, золотые и серебряные, которые Навуходоносор вынес из храма Иерусалимского и отнес в храм Вавилонский, --вынес Кир царь из храма Вавилонского; и отдали [их] по имени Шешбацару, которого он назначил областеначальником,
Ezr 5:15  и сказал ему: возьми сии сосуды, пойди и отнеси их в храм Иерусалимский, и пусть дом Божий строится на своем месте.
Ezr 5:16  Тогда Шешбацар тот пришел, положил основания дома Божия в Иерусалиме; и с тех пор доселе он строится, и еще не кончен.
Ezr 5:17  Итак, если царю благоугодно, пусть поищут в доме царских сокровищ, там в Вавилоне, точно ли царем Киром дано разрешение строить сей дом Божий в Иерусалиме, и царскую волю о сем пусть пришлют к нам.
Ezr 6:1  Тогда царь Дарий дал повеление, и разыскивали в Вавилоне в книгохранилище, куда полагали сокровища.
Ezr 6:2  И найден в Екбатане во дворце, который в области Мидии, один свиток, и в нем написано так: `Для памяти:
Ezr 6:3  в первый год царя Кира, царь Кир дал повеление о доме Божием в Иерусалиме: пусть строится дом на том месте, где приносят жертвы, и пусть будут положены прочные основания для него; вышина его в шестьдесят локтей, ширина его в шестьдесят локтей;
Ezr 6:4  рядов из камней больших три, и ряд из дерева один; издержки же пусть выдаются из царского дома.
Ezr 6:5  Да и сосуды дома Божия, золотые и серебряные, которые Навуходоносор вынес из храма Иерусалимского и отнес в Вавилон, пусть возвратятся и пойдут в храм Иерусалимский, [каждый] на место свое, и помещены будут в доме Божием.
Ezr 6:6  Итак, Фафнай, заречный областеначальник, и Шефар-Бознай, с товарищами вашими Афарсахеями, которые за рекою, --удалитесь оттуда.
Ezr 6:7  Не останавливайте работы при сем доме Божием; пусть Иудейский областеначальник и Иудейские старейшины строят сей дом Божий на месте его.
Ezr 6:8  И от меня дается повеление о том, чем вы должны содействовать старейшинам тем Иудейским в построении того дома Божия, и [именно]: из имущества царского--[из] заречной подати--немедленно берите и давайте тем людям, чтобы работа не останавливалась;
Ezr 6:9  и сколько нужно--тельцов ли, или овнов и агнцев, на всесожжения Богу небесному, также пшеницы, соли, вина и масла, как скажут священники Иерусалимские, пусть будет выдаваемо им изо дня в день без задержки,
Ezr 6:10  чтоб они приносили жертву приятную Богу небесному и молились о жизни царя и сыновей его.
Ezr 6:11  Мною же дается повеление, что [если] какой человек изменит это определение, то будет вынуто бревно из дома его, и будет поднят он и пригвожден к нему, а дом его за то будет обращен в развалины.
Ezr 6:12  И Бог, Которого имя там обитает, да низложит всякого царя и народ, который простер бы руку свою, чтобы изменить [сие] ко вреду этого дома Божия в Иерусалиме. Я, Дарий, дал это повеление; да будет оно в точности исполняемо'.
Ezr 6:13  Тогда Фафнай, заречный областеначальник, Шефар-Бознай и товарищи их, --как повелел царь Дарий, так в точности и делали.
Ezr 6:14  И старейшины Иудейские строили и преуспевали, по пророчеству Аггея пророка и Захарии, сына Адды. И построили и окончили, по воле Бога Израилева и по воле Кира и Дария и Артаксеркса, царей Персидских.
Ezr 6:15  И окончен дом сей к третьему дню месяца Адара, в шестой год царствования царя Дария.
Ezr 6:16  И совершили сыны Израилевы, священники и левиты и прочие, возвратившиеся из плена, освящение сего дома Божия с радостью.
Ezr 6:17  И принесли при освящении сего дома Божия: сто волов, двести овнов, четыреста агнцев и двенадцать козлов в жертву за грех за всего Израиля, по числу колен Израилевых.
Ezr 6:18  И поставили священников по отделениям их, и левитов по чередам их на службу Божию в Иерусалиме, как предписано в книге Моисея.
Ezr 6:19  И совершили возвратившиеся из плена пасху в четырнадцатый день первого месяца,
Ezr 6:20  потому что очистились священники и левиты, --все они, как один, [были] чисты; и закололи агнцев пасхальных для всех, возвратившихся из плена, для братьев своих священников и для себя.
Ezr 6:21  И ели сыны Израилевы, возвратившиеся из переселения, и все отделившиеся к ним от нечистоты народов земли, чтобы прибегать к Господу Богу Израилеву.
Ezr 6:22  И праздновали праздник опресноков семь дней в радости, потому что обрадовал их Господь и обратил к ним сердце царя Ассирийского, чтобы подкреплять руки их при строении дома Господа Бога Израилева.
Ezr 7:1  После сих происшествий, в царствование Артаксеркса, царя Персидского, Ездра, сын Сераии, сын Азарии, сын Хелкии,
Ezr 7:2  сын Шаллума, сын Садока, сын Ахитува,
Ezr 7:3  сын Амарии, сын Азарии, сын Марайофа,
Ezr 7:4  сын Захарии, сын Уззия, сын Буккия,
Ezr 7:5  сын Авишуя, сын Финееса, сын Елеазара, сын Аарона первосвященника, --
Ezr 7:6  сей Ездра вышел из Вавилона. Он был книжник, сведущий в законе Моисеевом, который дал Господь Бог Израилев. И дал ему царь все по желанию его, так как рука Господа Бога его [была] над ним.
Ezr 7:7  [С ним] пошли в Иерусалим и [некоторые] из сынов Израилевых, и из священников и левитов, и певцов и привратников и нефинеев в седьмой год царя Артаксеркса.
Ezr 7:8  И пришел он в Иерусалим в пятый месяц, --в седьмой же год царя.
Ezr 7:9  Ибо в первый день первого месяца [было] начало выхода из Вавилона, и в первый день пятого месяца он пришел в Иерусалим, так как благодеющая рука Бога его была над ним,
Ezr 7:10  потому что Ездра расположил сердце свое к тому, чтобы изучать закон Господень и исполнять [его], и учить в Израиле закону и правде.
Ezr 7:11  И вот содержание письма, которое дал царь Артаксеркс Ездре священнику, книжнику, учившему словам заповедей Господа и законов Его в Израиле:
Ezr 7:12  Артаксеркс, царь царей, Ездре священнику, учителю закона Бога небесного совершенному, и прочее.
Ezr 7:13  От меня дано повеление, чтобы в царстве моем всякий из народа Израилева и из священников его и левитов, желающий идти в Иерусалим, шел с тобою.
Ezr 7:14  Так как ты посылаешься от царя и семи советников его, чтобы обозреть Иудею и Иерусалим по закону Бога твоего, находящемуся в руке твоей,
Ezr 7:15  и чтобы доставить серебро и золото, которое царь и советники его пожертвовали Богу Израилеву, Которого жилище в Иерусалиме,
Ezr 7:16  и все серебро и золото, которое ты соберешь во всей области Вавилонской, вместе с доброхотными даяниями от народа и священников, которые пожертвуют они для дома Бога своего, что в Иерусалиме;
Ezr 7:17  поэтому немедленно купи на эти деньги волов, овнов, агнцев и хлебных приношений к ним и возлияний для них, и принеси их на жертвенник дома Бога вашего в Иерусалиме.
Ezr 7:18  И что тебе и братьям твоим заблагорассудится сделать из остального серебра и золота, то по воле Бога вашего делайте.
Ezr 7:19  И сосуды, которые даны тебе для служб [в] доме Бога твоего, поставь пред Богом Иерусалимским.
Ezr 7:20  И прочее потребное для дома Бога твоего, что ты признаешь нужным, давай из дома царских сокровищ.
Ezr 7:21  И от меня царя Артаксеркса дается повеление всем сокровищехранителям, которые за рекою: все, чего потребует у вас Ездра священник, учитель закона Бога небесного, немедленно давайте:
Ezr 7:22  серебра до ста талантов, и пшеницы до ста коров, и вина до ста батов, и до ста же батов масла, а соли без обозначения [количества].
Ezr 7:23  Все, что повелено Богом небесным, должно делаться со тщанием для дома Бога небесного; дабы не [было] гнева [Его] на царство, царя и сыновей его.
Ezr 7:24  И даем вам знать, чтобы [ни] на кого [из] священников или левитов, певцов, привратников, нефинеев и служащих при этом доме Божием, не налагать [ни] подати, [ни] налога, ни пошлины.
Ezr 7:25  Ты же, Ездра, по премудрости Бога твоего, которая в руке твоей, поставь правителей и судей, чтоб они судили весь народ за рекою, --всех знающих законы Бога твоего, а кто не знает, тех учите.
Ezr 7:26  Кто же не будет исполнять закон Бога твоего и закон царя, над тем немедленно пусть производят суд, на смерть ли, или на изгнание, или на денежную пеню, или на заключение в темницу.
Ezr 7:27  Благословен Господь, Бог отцов наших, вложивший в сердце царя--украсить дом Господень, который в Иерусалиме,
Ezr 7:28  и склонивший на меня милость царя и советников его, и всех могущественных князей царя! И я ободрился, ибо рука Господа Бога моего [была] надо мною, и собрал я глав Израиля, чтоб они пошли со мною.
Ezr 8:1  И вот главы поколений и родословие тех, которые вышли со мною из Вавилона, в царствование царя Артаксеркса:
Ezr 8:2  из сыновей Финееса Гирсон; из сыновей Ифамара Даниил; из сыновей Давида Хаттуш;
Ezr 8:3  из сыновей Шехании, из сыновей Пароша Захария, и с ним по списку родословному сто пятьдесят [человек] мужеского пола;
Ezr 8:4  из сыновей Пахаф-Моава Эльегоенай, сын Зерахии, и с ним двести [человек] мужеского пола;
Ezr 8:5  из сыновей Шехания, сын Яхазиила, и с ним триста [человек] мужеского пола;
Ezr 8:6  из сыновей Адина Евед, сын Ионафана, и с ним пятьдесят [человек] мужеского пола;
Ezr 8:7  из сыновей Елама Иешаия, сын Афалии, и с ним семьдесят [человек] мужеского пола;
Ezr 8:8  из сыновей Сафатии Зевадия, сын Михаилов, и с ним восемьдесят [человек] мужеского пола;
Ezr 8:9  из сыновей Иоава Овадия, сын Иехиелов, и с ним двести восемнадцать [человек] мужеского пола;
Ezr 8:10  из сыновей Шеломиф, сын Иосифии, и с ним сто шестьдесят [человек] мужеского пола;
Ezr 8:11  из сыновей Бевая Захария, сын Бевая, и с ним двадцать восемь [человек] мужеского пола;
Ezr 8:12  из сыновей Азгада Иоханан, сын Гаккатана, и с ним сто десять [человек] мужеского пола;
Ezr 8:13  из сыновей Адоникама последние, и вот имена их: Елифелет, Иеиел и Шемаия, и с ними шестьдесят [человек] мужеского пола;
Ezr 8:14  из сыновей Бигвая, Уфай и Заббуд, и с ними семьдесят [человек] мужеского пола.
Ezr 8:15  Я собрал их у реки, втекающей в Агаву, и мы простояли там три дня, и когда я осмотрел народ и священников, то из сынов Левия [никого] там не нашел.
Ezr 8:16  И послал я позвать Елиезера, Ариэла, Шемаию, и Элнафана, и Иарива, и Элнафана, и Нафана, и Захарию, и Мешуллама--главных, и Иоярива и Элнафана--ученых;
Ezr 8:17  и дал им поручение к Иддо, главному в местности Касифье, и вложил им в уста, что говорить к Иддо и братьям его, нефинеям в местности Касифье, чтобы они привели к нам служителей для дома Бога нашего.
Ezr 8:18  И привели они к нам, так как благодеющая рука Бога нашего была над нами, человека умного из сыновей Махлия, сына Левиина, сына Израилева, именно Шеревию, и сыновей его и братьев его, восемнадцать [человек];
Ezr 8:19  и Хашавию и с ним Иешаию из сыновей Мерариных, братьев его и сыновей их двадцать;
Ezr 8:20  и из нефинеев, которых дал Давид и князья [его] на прислугу левитам, двести двадцать нефинеев; все они названы поименно.
Ezr 8:21  И провозгласил я там пост у реки Агавы, чтобы смириться нам пред лицем Бога нашего, просить у Него благополучного пути для себя и для детей наших и для всего имущества нашего,
Ezr 8:22  так как мне стыдно было просить у царя войска и всадников для охранения нашего от врага на пути, ибо мы, говоря с царем, сказали: рука Бога нашего для всех прибегающих к Нему [есть] благодеющая, а на всех оставляющих Его--могущество Его и гнев Его!
Ezr 8:23  Итак мы постились и просили Бога нашего о сем, и Он услышал нас.
Ezr 8:24  И я отделил из начальствующих над священниками двенадцать [человек]: Шеревию, Хашавию и с ними десять из братьев их;
Ezr 8:25  и отдал им весом серебро, и золото, и сосуды, --все, пожертвованное [для] дома Бога нашего, что пожертвовали царь, и советники его, и князья его, и все Израильтяне, [там] находившиеся.
Ezr 8:26  И отдал на руки им весом: серебра--шестьсот пятьдесят талантов, и серебряных сосудов на сто талантов, золота--сто талантов;
Ezr 8:27  и чаш золотых--двадцать, в тысячу драхм, и два сосуда из лучшей блестящей меди, ценимой как золото.
Ezr 8:28  И сказал я им: вы--святыня Господу, и сосуды--святыня, и серебро и золото--доброхотное даяние Господу Богу отцов ваших.
Ezr 8:29  Будьте же бдительны и сберегите [это], доколе весом не сдадите начальствующим над священниками и левитами и главам поколений Израилевых в Иерусалиме, в хранилище при доме Господнем.
Ezr 8:30  И приняли священники и левиты взвешенное серебро, и золото, и сосуды, чтоб отнести в Иерусалим в дом Бога нашего.
Ezr 8:31  И отправились мы от реки Агавы в двенадцатый день первого месяца, чтобы идти в Иерусалим; и рука Бога нашего была над нами, и спасала нас от руки врага и от подстерегающих нас на пути.
Ezr 8:32  И пришли мы в Иерусалим, и пробыли там три дня.
Ezr 8:33  В четвертый день мы сдали весом серебро, и золото, и сосуды в дом Бога нашего, на руки Меремофу, сыну Урии, священнику, и с ним Елеазару, сыну Финеесову, и с ними Иозаваду, сыну Иисусову, и Ноадии, сыну Виннуя, левитам,
Ezr 8:34  все счетом и весом. И все взвешенное записано в то же время.
Ezr 8:35  Пришедшие из плена переселенцы принесли во всесожжение Богу Израилеву двенадцать тельцов из всего Израиля, девяносто шесть овнов, семьдесят семь агнцев и двенадцать козлов в жертву за грех: все это во всесожжение Господу.
Ezr 8:36  И отдали царские повеления царским сатрапам и заречным областеначальникам, и они почтили народ и дом Божий.
Ezr 9:1  По окончании сего, подошли ко мне начальствующие и сказали: народ Израилев и священники и левиты не отделились от народов иноплеменных с мерзостями их, от Хананеев, Хеттеев, Ферезеев, Иевусеев, Аммонитян, Моавитян, Египтян и Аморреев,
Ezr 9:2  потому что взяли дочерей их за себя и за сыновей своих, и смешалось семя святое с народами иноплеменными, и притом рука знатнейших и главнейших была в сем беззаконии первою.
Ezr 9:3  Услышав это слово, я разодрал нижнюю и верхнюю одежду мою и рвал волосы на голове моей и на бороде моей, и сидел печальный.
Ezr 9:4  Тогда собрались ко мне все, убоявшиеся слов Бога Израилева по причине преступления переселенцев, и я сидел в печали до вечерней жертвы.
Ezr 9:5  А во время вечерней жертвы я встал с места сетования моего, и в разодранной нижней и верхней одежде пал на колени мои и простер руки мои к Господу Богу моему
Ezr 9:6  и сказал: Боже мой! стыжусь и боюсь поднять лице мое к Тебе, Боже мой, потому что беззакония наши стали выше головы, и вина наша возросла до небес.
Ezr 9:7  Со дней отцов наших мы в великой вине до сего дня, и за беззакония наши преданы были мы, цари наши, священники наши, в руки царей иноземных, под меч, в плен и на разграбление и на посрамление, как это и ныне.
Ezr 9:8  И вот, по малом времени, даровано нам помилование от Господа Бога нашего, и Он оставил у нас [несколько] уцелевших и дал нам утвердиться на месте святыни Его, и просветил глаза наши Бог наш, и дал нам ожить немного в рабстве нашем.
Ezr 9:9  Мы--рабы, но и в рабстве нашем не оставил нас Бог наш. И склонил Он к нам милость царей Персидских, чтоб они дали нам ожить, воздвигнуть дом Бога нашего и восстановить [его] из развалин его, и дали нам ограждение в Иудее и в Иерусалиме.
Ezr 9:10  И ныне, что скажем мы, Боже наш, после этого? Ибо мы отступили от заповедей Твоих,
Ezr 9:11  которые заповедал Ты чрез рабов Твоих пророков, говоря: земля, в которую идете вы, чтоб овладеть ею, земля нечистая, она осквернена нечистотою иноплеменных народов, их мерзостями, которыми они наполнили ее от края до края в осквернениях своих.
Ezr 9:12  Итак дочерей ваших не выдавайте за сыновей их, и дочерей их не берите за сыновей ваших, и не ищите мира их и блага их во веки, чтобы укрепиться вам и питаться благами земли той и передать ее в наследие сыновьям вашим на веки.
Ezr 9:13  И после всего, постигшего нас за худые дела наши и за великую вину нашу, --ибо Ты, Боже наш, пощадил нас не по мере беззакония нашего и дал нам такое избавление, --
Ezr 9:14  неужели мы опять будем нарушать заповеди Твои и вступать в родство с этими отвратительными народами? Не прогневаешься ли Ты на нас даже до истребления [нас], так что не будет уцелевших и не будет спасения?
Ezr 9:15  Господи Боже Израилев! праведен Ты. Ибо мы остались уцелевшими до сего дня; и вот мы в беззакониях наших пред лицем Твоим, хотя после этого не надлежало бы нам стоять пред лицем Твоим.
Ezr 10:1  Когда [так] молился Ездра и исповедывался, плача и повергаясь пред домом Божиим, стеклось к нему весьма большое собрание Израильтян, мужчин и женщин и детей, потому что и народ много плакал.
Ezr 10:2  И отвечал Шехания, сын Иехиила из сыновей Еламовых, и сказал Ездре: мы сделали преступление пред Богом нашим, что взяли [себе] жен иноплеменных из народов земли, но есть еще надежда для Израиля в этом деле;
Ezr 10:3  заключим теперь завет с Богом нашим, что, по совету господина моего и благоговеющих пред заповедями Бога нашего, мы отпустим [от себя] всех жен и [детей], рожденных ими, --и да будет по закону!
Ezr 10:4  Встань, потому что это твое дело, и мы с тобою: ободрись и действуй!
Ezr 10:5  И встал Ездра, и велел начальствующим над священниками, левитами и всем Израилем дать клятву, что они сделают так. И они дали клятву.
Ezr 10:6  И встал Ездра и пошел от дома Божия в жилище Иоханана, сына Елияшивова, и пришел туда. Хлеба он не ел и воды не пил, потому что плакал о преступлении переселенцев.
Ezr 10:7  И объявили в Иудее и в Иерусалиме всем [бывшим] в плену, чтоб они собрались в Иерусалим;
Ezr 10:8  а кто не придет чрез три дня, на все имение того, по определению начальствующих и старейшин, будет положено заклятие, и сам он будет отлучен от общества переселенцев.
Ezr 10:9  И собрались все жители Иудеи и земли Вениаминовой в Иерусалим в три дня. Это [было] в девятом месяце, в двадцатый день месяца. И сидел весь народ на площади у дома Божия, дрожа как по этому делу, так и от дождей.
Ezr 10:10  И встал Ездра священник и сказал им: вы сделали преступление, взяв себе жен иноплеменных, и тем увеличили вину Израиля.
Ezr 10:11  Итак покайтесь [в сем] пред Господом Богом отцов ваших, и исполните волю Его, и отлучите себя от народов земли и от жен иноплеменных.
Ezr 10:12  И отвечало все собрание, и сказало громким голосом: как ты сказал, так и сделаем.
Ezr 10:13  Однако же народ многочислен и время [теперь] дождливое, и нет возможности стоять на улице. Да и это дело не одного дня и не двух, потому что мы много в этом деле погрешили.
Ezr 10:14  Пусть наши начальствующие заступят место всего общества, и все в городах наших, которые взяли жен иноплеменных, пусть приходят сюда в назначенные времена и с ними старейшины каждого города и судьи его, доколе не отвратится от нас пылающий гнев Бога нашего за это дело.
Ezr 10:15  Тогда Ионафан, сын Асаила, и Яхзеия, сын Фиквы, стали над этим делом, и Мешуллам и Шавфай левит были помощниками им.
Ezr 10:16  И сделали так вышедшие из плена. И отделены [на это] Ездра священник, главы поколений, от каждого поколения их, и все они [названы] поименно. И сделали они заседание в первый день десятого месяца, для исследования сего дела;
Ezr 10:17  и окончили [исследование] о всех, которые взяли жен иноплеменных, к первому дню первого месяца.
Ezr 10:18  И нашлись из сыновей священнических, которые взяли жен иноплеменных, --из сыновей Иисуса, сына Иоседекова, и братьев его: Маасея, Елиезер, Иарив и Гедалия;
Ezr 10:19  и они дали руки свои [во уверение], что отпустят жен своих, и [что они] повинны [принести] в жертву овна за свою вину;
Ezr 10:20  и из сыновей Иммера: Хананий и Зевадия;
Ezr 10:21  и из сыновей Харима: Маасея, Елия, Шемаия, Иехиил и Уззия;
Ezr 10:22  и из сыновей Пашхура: Елиоенай, Маасея, Исмаил, Нафанаил, Иозавад и Эласа;
Ezr 10:23  и из левитов: Иозавад, Шимей и Келаия, он же Клита, Пафахия, Иуда и Елиезер;
Ezr 10:24  и из певцов: Елияшив; и из привратников: Шаллум, Телем и Урий;
Ezr 10:25  а из Израильтян, --из сыновей Пароша: Рамаия, Иззия, Малхия, Миямин, Елеазар, Малхия и Венаия;
Ezr 10:26  и из сыновей Елама: Матфания, Захария, Иехиел, Авдий, Иремоф и Елия;
Ezr 10:27  и из сыновей Заффу: Елиоенай, Елияшив, Матфания, Иремоф, Завад и Азиса;
Ezr 10:28  и из сыновей Бевая: Иоханан, Ханания, Забвай и Афлай;
Ezr 10:29  и из сыновей Вания: Мешуллам, Маллух, Адая, Иашув, Шеал и Иерамоф;
Ezr 10:30  и из сыновей Пахаф-Моава: Адна, Хелал, Венаия, Маасея, Матфания, Веселеил, Биннуй и Манассия;
Ezr 10:31  и из сыновей Харима: Елиезер, Ишшия, Малхия, Шемаия, Симеон,
Ezr 10:32  Вениамин, Маллух, Шемария;
Ezr 10:33  и из сыновей Хашума: Мафнай, Мафафа, Завад, Елифелет, Иеремай, Манассия и Шимей;
Ezr 10:34  и из сыновей Вания: Маадай, Амрам и Уел,
Ezr 10:35  Бенаия, Бидья, Келуги,
Ezr 10:36  Ванея, Меремоф, Елиашив,
Ezr 10:37  Матфания, Мафнай, Иаасай,
Ezr 10:38  Ваний, Биннуй, Шимей,
Ezr 10:39  Шелемия, Нафан, Адаия,
Ezr 10:40  Махнадбай, Шашай, Шарай,
Ezr 10:41  Азариел, Шелемиягу, Шемария,
Ezr 10:42  Шаллум, Амария и Иосиф;
Ezr 10:43  и из сыновей Нево: Иеиел, Матфифия, Завад, Зевина, Иаддай, Иоель и Бенаия.
Ezr 10:44  Все сии взяли [за себя] жен иноплеменных, и некоторые из сих жен родили им детей.


\end{document}