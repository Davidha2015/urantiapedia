\begin{document}

\title{Неемии}


\chapter{1}

\par 1 Слова Неемии, сына Ахалиина. В месяце Кислеве, в двадцатом году, я находился в Сузах, престольном городе.
\par 2 И пришел Ханани, один из братьев моих, он и [несколько] человек из Иудеи. И спросил я их об уцелевших Иудеях, которые остались от плена, и об Иерусалиме.
\par 3 И сказали они мне: оставшиеся, которые остались от плена, [находятся] там, в стране [своей], в великом бедствии и в уничижении; и стена Иерусалима разрушена, и ворота его сожжены огнем.
\par 4 Услышав эти слова, я сел и заплакал, и печален был несколько дней, и постился и молился пред Богом небесным
\par 5 и говорил: Господи Боже небес, Боже великий и страшный, хранящий завет и милость к любящим Тебя и соблюдающим заповеди Твои!
\par 6 Да будут уши Твои внимательны и очи Твои отверсты, чтобы услышать молитву раба Твоего, которою я теперь день и ночь молюсь пред Тобою о сынах Израилевых, рабах Твоих, и исповедуюсь во грехах сынов Израилевых, которыми согрешили мы пред Тобою, согрешили--и я и дом отца моего.
\par 7 Мы стали преступны пред Тобою и не сохранили заповедей и уставов и определений, которые Ты заповедал Моисею, рабу Твоему.
\par 8 Но помяни слово, которое Ты заповедал Моисею, рабу Твоему, говоря: [если] вы сделаетесь преступниками, то Я рассею вас по народам;
\par 9 [когда] же обратитесь ко Мне и будете хранить заповеди Мои и исполнять их, то хотя бы вы изгнаны были на край неба, и оттуда соберу вас и приведу вас на место, которое избрал Я, чтобы водворить там имя Мое.
\par 10 Они же рабы Твои и народ Твой, который Ты искупил силою Твоею великою и рукою Твоею могущественною.
\par 11 Молю Тебя, Господи! Да будет ухо Твое внимательно к молитве раба Твоего и к молитве рабов Твоих, любящих благоговеть пред именем Твоим. И благопоспеши рабу Твоему теперь, и введи его в милость у человека сего. Я был виночерпием у царя.

\chapter{2}

\par 1 В месяце Нисане, в двадцатый год царя Артаксеркса, [было] перед ним вино. И я взял вино и подал царю, и, казалось, не был печален перед ним.
\par 2 Но царь сказал мне: отчего лице у тебя печально; ты не болен, этого нет, а верно печаль на сердце? Я сильно испугался
\par 3 и сказал царю: да живет царь во веки! Как не быть печальным лицу моему, когда город, дом гробов отцов моих, в запустении, и ворота его сожжены огнем!
\par 4 И сказал мне царь: чего же ты желаешь? Я помолился Богу небесному
\par 5 и сказал царю: если царю благоугодно, и если в благоволении раб твой пред лицем твоим, то пошли меня в Иудею, в город, [где] гробы отцов моих, чтоб я обстроил его.
\par 6 И сказал мне царь и царица, которая сидела подле него: сколько времени продлится путь твой, и когда возвратишься? И благоугодно было царю послать меня, после того как я назначил время.
\par 7 И сказал я царю: если царю благоугодно, то дал бы мне письма к заречным областеначальникам, чтоб они давали мне пропуск, доколе я не дойду до Иудеи,
\par 8 и письмо к Асафу, хранителю царских лесов, чтоб он дал мне дерев для ворот крепости, которая при доме [Божием], и для городской стены, и для дома, в котором бы мне жить. И дал мне царь, так как благодеющая рука Бога моего была надо мною.
\par 9 И пришел я к заречным областеначальникам и отдал им царские письма. Послал же со мною царь воинских начальников со всадниками.
\par 10 Когда услышал [сие] Санаваллат, Хоронит и Товия, Аммонитский раб, то им было весьма досадно, что пришел человек заботиться о благе сынов Израилевых.
\par 11 И пришел я в Иерусалим. И пробыв там три дня,
\par 12 встал я ночью с немногими людьми, [бывшими] при мне, и никому не сказал, что Бог мой положил мне на сердце сделать для Иерусалима; животного же не было со мною никакого, кроме того, на котором я ехал.
\par 13 И проехал я ночью через ворота Долины перед источником Драконовым к воротам Навозным, и осмотрел я стены Иерусалима разрушенные и его ворота, сожженные огнем.
\par 14 И подъехал я к воротам Источника и к царскому водоему, но [там] не было места пройти животному, которое было подо мною, --
\par 15 и я поднялся назад по лощине ночью и осматривал стену, и проехав [опять] воротами Долины, возвратился.
\par 16 И начальствующие не знали, куда я ходил и что я делаю: ни Иудеям, ни священникам, ни знатнейшим, ни начальствующим, ни прочим производителям работ я дотоле ничего не открывал.
\par 17 И сказал я им: вы видите бедствие, в каком мы находимся; Иерусалим пуст и ворота его сожжены огнем; пойдем, построим стену Иерусалима, и не будем впредь [в таком] уничижении.
\par 18 И я рассказал им о благодеявшей мне руке Бога моего, а также и слова царя, которые он говорил мне. И сказали они: будем строить, --и укрепили руки свои на благое [дело].
\par 19 Услышав это, Санаваллат, Хоронит и Товия, Аммонитский раб, и Гешем Аравитянин смеялись над нами и с презрением говорили: что это за дело, которое вы делаете? уже не думаете ли возмутиться против царя?
\par 20 Я дал им ответ и сказал им: Бог Небесный, Он благопоспешит нам, и мы, рабы Его, станем строить, а вам нет части и права и памяти в Иерусалиме.

\chapter{3}

\par 1 И встал Елияшив, великий священник, и братья его священники и построили Овечьи ворота: они освятили их и вставили двери их, и от башни Меа освятили их до башни Хананела.
\par 2 И подле него строили Иерихонцы, а подле них строил Закхур, сын Имрия.
\par 3 Ворота Рыбные строили уроженцы Сенаи: они покрыли их, и вставили двери их, замки их и засовы их.
\par 4 Подле них чинил [стену] Меремоф, сын Урии, сын Гаккоца; подле них чинил Мешуллам, сын Берехии, сын Мешизабела; подле них чинил Садок, сын Бааны;
\par 5 подле них чинили Фекойцы; впрочем знатнейшие из них не наклонили шеи своей поработать для Господа своего.
\par 6 Старые ворота чинили Иоиада, сын Пасеаха, и Мешуллам, сын Бесодии: они покрыли их и вставили двери их, и замки их и засовы их.
\par 7 Подле них чинил Мелатия Гаваонитянин, и Иадон из Меронофа, с жителями Гаваона и Мицфы, подвластными заречному областеначальнику.
\par 8 Подле него чинил Уззиил, сын Харгаии, серебряник, а подле него чинил Ханания, сын Гараккахима. И восстановили Иерусалим до стены широкой.
\par 9 Подле них чинил Рефаия, сын Хура, начальник полуокруга Иерусалимского.
\par 10 Подле них и против дома своего чинил Иедаия, сын Харумафа, а подле него чинил Хаттуш, сын Хашавнии.
\par 11 На втором участке чинил Малхия, сын Харима, и Хашшув, сын Пахаф-Моава; [они же чинили] и башню Печную.
\par 12 Подле них чинил Шаллум, сын Галлохеша, начальник полуокруга Иерусалимского, он и дочери его.
\par 13 Ворота Долины чинил Ханун и жители Заноаха: они построили их, и вставили двери их, замки их и засовы их, и [еще чинили] они тысячу локтей стены до ворот Навозных.
\par 14 А ворота Навозные чинил Малхия, сын Рехава, начальник Бефкаремского округа: он построил их и вставил двери их, замки их и засовы их.
\par 15 Ворота Источника чинил Шаллум, сын Колхозея, начальник округа Мицфы: он построил их, и покрыл их, и вставил двери их, замки их и засовы их, --[он же чинил] стену у водоема Селах против царского сада и до ступеней, спускающихся из города Давидова.
\par 16 За ним чинил Неемия, сын Азбука, начальник полуокруга Бефцурского, до гробниц Давидовых и до выкопанного пруда и до дома храбрых.
\par 17 За ним чинили левиты: Рехум, сын Вания; подле него чинил Хашавия, начальник полуокруга Кеильского, за свой округ.
\par 18 За ним чинили братья их: Баввай, сын Хенадада, начальник Кеильского полуокруга.
\par 19 А подле него чинил Езер, сын Иисуса, начальник Мицфы, на втором участке, напротив всхода к оружейне на углу.
\par 20 За ним ревностно чинил Варух, сын Забвая, на втором участке, от угла до дверей дома Елияшива, великого священника.
\par 21 За ним чинил Меремоф, сын Урии, сын Гаккоца, на втором участке, от дверей дома Елияшивова до конца дома Елияшивова.
\par 22 За ним чинили священники из окрестностей.
\par 23 За ними чинил Вениамин и Хашшув, против дома своего; за ними чинил Азария, сын Маасеи, сын Анании, возле дома своего.
\par 24 За ним чинил Биннуй, сын Хенадада, на втором участке, от дома Азарии до угла и поворота.
\par 25 [За ним] Фалал, сын Узая, напротив угла и башни, выступающей от верхнего дома царского, которая у двора темничного. За ним Федаия, сын Пароша.
\par 26 Нефинеи же, [которые] жили в Офеле, [починили] напротив Водяных ворот к востоку и до выступающей башни.
\par 27 За ними чинили Фекойцы, на втором участке, от [места] напротив большой выступающей башни до стены Офела.
\par 28 Далее ворот Конских чинили священники, каждый против своего дома.
\par 29 За ними чинил Садок, сын Иммера, против своего дома, а за ним чинил Шемаия, сын Шехании, сторож восточных ворот.
\par 30 За ним чинил Ханания, сын Шелемии, и Ханун, шестой сын Цалафа, на втором участке. За ним чинил Мешуллам, сын Берехии, против комнаты своей.
\par 31 За ним чинил Малхия, сын Гацорфия, до дома нефинеев и торговцев, против ворот Гаммифкад и до угольного жилья.
\par 32 А между угольным жильем до ворот Овечьих чинили серебряники и торговцы.

\chapter{4}

\par 1 Когда услышал Санаваллат, что мы строим стену, он рассердился и много досадовал и издевался над Иудеями;
\par 2 и говорил при братьях своих и при Самарийских военных людях, и сказал: что делают эти жалкие Иудеи? неужели им это дозволят? неужели будут они приносить жертвы? неужели они когда-либо кончат? неужели они оживят камни из груд праха, и притом пожженные?
\par 3 А Товия Аммонитянин, [бывший] подле него, сказал: пусть их строят; пойдет лисица, и разрушит их каменную стену.
\par 4 Услыши, Боже наш, в каком мы презрении, и обрати ругательство их на их голову, и предай их презрению в земле пленения;
\par 5 и не покрой беззаконий их, и грех их да не изгладится пред лицем Твоим, потому что они огорчили строящих!
\par 6 Мы однако же строили стену, и сложена была вся стена до половины ее. И у народа доставало усердия работать.
\par 7 Когда услышал Санаваллат и Товия, и Аравитяне, и Аммонитяне, и Азотяне, что стены Иерусалимские восстановляются, что повреждения начали заделываться, то им было весьма досадно.
\par 8 И сговорились все вместе пойти войною на Иерусалим и разрушить его.
\par 9 И мы молились Богу нашему, и ставили против них стражу днем и ночью, для спасения от них.
\par 10 Но Иудеи сказали: ослабела сила у носильщиков, а мусору много; мы не в состоянии строить стену.
\par 11 А неприятели наши говорили: не узнают и не увидят, как [вдруг] мы войдем в средину их и перебьем их, и остановим дело.
\par 12 Когда приходили Иудеи, жившие подле них, и говорили нам раз десять, со всех мест, что они нападут на нас:
\par 13 тогда в низменных местах у города, за стеною, на местах сухих поставил я народ по-племенно с мечами их, с копьями их и луками их.
\par 14 И осмотрел я, и стал, и сказал знатнейшим и начальствующим и прочему народу: не бойтесь их; помните Господа великого и страшного и сражайтесь за братьев своих, за сыновей своих и за дочерей своих, за жен своих и за домы свои.
\par 15 Когда услышали неприятели наши, что нам известно [намерение] [их], тогда разорил Бог замысел их, и все мы возвратились к стене, каждый на свою работу.
\par 16 С того дня половина молодых людей у меня занималась работою, а [другая] половина их держала копья, щиты и луки и латы; и начальствующие [находились] позади всего дома Иудина.
\par 17 Строившие стену и носившие тяжести, которые налагали [на них], одною рукою производили работу, а другою держали копье.
\par 18 Каждый из строивших препоясан был мечом по чреслам своим, и [так] они строили. Возле меня находился трубач.
\par 19 И сказал я знатнейшим и начальствующим и прочему народу: работа велика и обширна, и мы рассеяны по стене и отдалены друг от друга;
\par 20 поэтому, откуда услышите вы звук трубы, в то место собирайтесь к нам: Бог наш будет сражаться за нас.
\par 21 Так производили мы работу; и половина держала копья от восхода зари до появления звезд.
\par 22 Сверх сего, в то же время я сказал народу, чтобы в Иерусалиме ночевали все с рабами своими, --и будут они у нас ночью на страже, а днем на работе.
\par 23 И ни я, ни братья мои, ни слуги мои, ни стражи, сопровождавшие меня, не снимали с себя одеяния своего, у каждого были под рукою меч и вода.

\chapter{5}

\par 1 И сделался большой ропот в народе и у жен его на братьев своих Иудеев.
\par 2 Были такие, которые говорили: нас, сыновей наших и дочерей наших много; и мы желали бы доставать хлеб и кормиться и жить.
\par 3 Были и такие, которые говорили: поля свои, и виноградники свои, и домы свои мы закладываем, чтобы достать хлеба от голода.
\par 4 Были и такие, которые говорили: мы занимаем серебро на подать царю [под залог] полей наших и виноградников наших;
\par 5 у нас такие же тела, какие тела у братьев наших, и сыновья наши такие же, как их сыновья; а вот, мы должны отдавать сыновей наших и дочерей наших в рабы, и некоторые из дочерей наших уже находятся в порабощении. Нет никаких средств для выкупа в руках наших; и поля наши и виноградники наши у других.
\par 6 Когда я услышал ропот их и такие слова, я очень рассердился.
\par 7 Сердце мое возмутилось, и я строго выговорил знатнейшим и начальствующим и сказал им: вы берете лихву с братьев своих. И созвал я против них большое собрание
\par 8 и сказал им: мы выкупали братьев своих, Иудеев, проданных народам, сколько было сил у нас, а вы продаете братьев своих, и они продаются нам? Они молчали и не находили ответа.
\par 9 И сказал я: нехорошо вы делаете. Не в страхе ли Бога нашего должны ходить вы, дабы избегнуть поношения от народов, врагов наших?
\par 10 И я также, братья мои и [служащие] при мне давали им в заем и серебро и хлеб: оставим им долг сей.
\par 11 Возвратите им ныне же поля их, виноградные и масличные сады их, и домы их, и рост с серебра и хлеба, и вина и масла, за который вы ссудили их.
\par 12 И сказали они: возвратим и не будем с них требовать; сделаем так, как ты говоришь. И позвал я священников и велел им дать клятву, что они так сделают.
\par 13 И вытряхнул я [одежду] мою и сказал: так пусть вытряхнет Бог всякого человека, который не сдержит слова сего, из дома его и из имения его, и так да будет у него вытрясено и пусто! И сказало все собрание: аминь. И прославили Бога; и народ выполнил слово сие.
\par 14 Еще: с того дня, как определен я был областеначальником их в земле Иудейской, от двадцатого года до тридцать второго года царя Артаксеркса, в продолжение двенадцати лет я и братья мои не ели хлеба областеначальнического.
\par 15 А прежние областеначальники, которые [были] до меня, отягощали народ и брали с них хлеб и вино, кроме сорока сиклей серебра; даже и слуги их господствовали над народом. Я же не делал так по страху Божию.
\par 16 При этом работы на стене сей я поддерживал; и полей мы не закупали, и все слуги мои собирались туда на работу.
\par 17 Иудеев и начальствующих по сто пятидесяти человек [бывало] за столом у меня, кроме приходивших к нам из окрестных народов.
\par 18 И [вот] что было приготовляемо на один день: один бык, шесть отборных овец и птицы приготовлялись у меня; и в десять дней [издерживалось] множество всякого вина. И при [всем] том, хлеба областеначальнического я не требовал, так как тяжелая служба [лежала] на народе сем.
\par 19 Помяни, Боже мой, во благо мне все, что я сделал для народа сего!

\chapter{6}

\par 1 Когда дошло до слуха Санаваллата и Товии и Гешема Аравитянина и прочих неприятелей наших, что я отстроил стену, и не оставалось в ней повреждений--впрочем до того времени я еще не ставил дверей в ворота, --
\par 2 тогда прислал Санаваллат и Гешем ко мне сказать: приди, и сойдемся в одном из сел на равнине Оно. Они замышляли сделать мне зло.
\par 3 Но я послал к ним послов сказать: я занят большим делом, не могу сойти; дело остановилось бы, если бы я оставил его и сошел к вам.
\par 4 Четыре раза присылали они ко мне с таким же приглашением, и я отвечал им то же.
\par 5 Тогда прислал ко мне Санаваллат в пятый раз своего слугу, у которого в руке было открытое письмо.
\par 6 В нем было написано: слух носится у народов, и Гешем говорит, будто ты и Иудеи задумали отпасть, для чего и строишь стену и хочешь быть у них царем, по тем же слухам;
\par 7 и пророков поставил ты, чтоб они разглашали о тебе в Иерусалиме и говорили: царь Иудейский! И такие речи дойдут до царя. Итак приходи, и посоветуемся вместе.
\par 8 Но я послал к нему сказать: ничего такого не было, о чем ты говоришь; ты выдумал это своим умом.
\par 9 Ибо все они стращали нас, думая: опустятся руки их от дела сего, и оно не состоится; но я тем более укрепил руки мои.
\par 10 Пришел я в дом Шемаии, сына Делаии, сына Мегетавелова, и он заперся и сказал: пойдем в дом Божий, внутрь храма, и запрем за собою двери храма, потому что придут убить тебя, и придут убить тебя ночью.
\par 11 Но я сказал: может ли бежать такой человек, как я? Может ли такой, как я, войти в храм, чтобы остаться живым? Не пойду.
\par 12 Я знал, что не Бог послал его, хотя он пророчески говорил мне, но что Товия и Санаваллат подкупили его.
\par 13 Для того он был подкуплен, чтоб я устрашился и сделал так и согрешил, и чтобы имели о мне худое мнение и преследовали меня за это укоризнами.
\par 14 Помяни, Боже мой, Товию и Санаваллата по сим делам их, а также пророчицу Ноадию и прочих пророков, которые хотели устрашить меня!
\par 15 Стена была совершена в двадцать пятый день месяца Елула, в пятьдесят два дня.
\par 16 Когда услышали об этом все неприятели наши, и увидели это все народы, которые вокруг нас, тогда они очень упали в глазах своих и познали, что это дело сделано Богом нашим.
\par 17 Сверх того в те дни знатнейшие Иудеи много писали писем, которые посылались к Товии, а Товиины письма приходили к ним.
\par 18 Ибо многие в Иудее были в клятвенном союзе с ним, потому что он был зять Шехании, сын Арахова, а сын его Иоханан взял [за себя] дочь Мешуллама, сына Верехии.
\par 19 Даже о доброте его они говорили при мне, и мои слова переносились к нему. Товия присылал письма, чтоб устрашить меня.

\chapter{7}

\par 1 Когда стена была построена, и я вставил двери, и поставлены были на свое служение привратники и певцы и левиты,
\par 2 тогда приказал я брату моему Ханани и начальнику Иерусалимской крепости Хананию, ибо он более многих других был человек верный и богобоязненный,
\par 3 и сказал я им: пусть не отворяют ворот Иерусалимских, доколе не обогреет солнце, и доколе они стоят, пусть замыкают и запирают двери. И поставил я стражами жителей Иерусалима, каждого на свою стражу и каждого напротив дома его.
\par 4 Но город был пространен и велик, а народа в нем было немного, и домы не были построены.
\par 5 И положил мне Бог мой на сердце собрать знатнейших и начальствующих и народ, чтобы сделать перепись. И нашел я родословную перепись тех, которые сначала пришли, и в ней написано:
\par 6 вот жители страны, которые отправились из пленников, переселенных Навуходоносором, царем Вавилонским, и возвратились в Иерусалим и Иудею, каждый в свой город, --
\par 7 те, которые пошли с Зоровавелем, Иисусом, Неемиею, Азариею, Раамиею, Нахманием, Мардохеем, Билшаном, Мисферефом, Бигваем, Нехумом, Вааною. Число людей народа Израилева:
\par 8 сыновей Пароша две тысячи сто семьдесят два.
\par 9 Сыновей Сафатии триста семьдесят два.
\par 10 Сыновей Араха шестьсот пятьдесят два.
\par 11 Сыновей Пахаф-Моава, из сыновей Иисуса и Иоава, две тысячи восемьсот восемнадцать.
\par 12 Сыновей Елама тысяча двести пятьдесят четыре.
\par 13 Сыновей Заффу восемьсот сорок пять.
\par 14 Сыновей Закхая семьсот шестьдесят.
\par 15 Сыновей Биннуя шестьсот сорок восемь.
\par 16 Сыновей Бевая шестьсот двадцать восемь.
\par 17 Сыновей Азгада две тысячи триста двадцать два.
\par 18 Сыновей Адоникама шестьсот шестьдесят семь.
\par 19 Сыновей Бигвая две тысячи шестьсот семь.
\par 20 Сыновей Адина шестьсот пятьдесят пять.
\par 21 Сыновей Атера из [дома] Езекии девяносто восемь.
\par 22 Сыновей Хашума триста двадцать восемь.
\par 23 Сыновей Вецая триста двадцать четыре.
\par 24 Сыновей Харифа сто двенадцать.
\par 25 Уроженцев Гаваона девяносто пять.
\par 26 Жителей Вифлеема и Нетофы сто восемьдесят восемь.
\par 27 Жителей Анафофа сто двадцать восемь.
\par 28 Жителей Беф-Азмавефа сорок два.
\par 29 Жителей Кириаф-Иарима, Кефиры и Беерофа семьсот сорок три.
\par 30 Жителей Рамы и Гевы шестьсот двадцать один.
\par 31 Жителей Михмаса сто двадцать два.
\par 32 Жителей Вефиля и Гая сто двадцать три.
\par 33 Жителей Нево другого пятьдесят два.
\par 34 Сыновей Елама другого тысяча двести пятьдесят четыре.
\par 35 Сыновей Харима триста двадцать.
\par 36 Уроженцев Иерихона триста сорок пять.
\par 37 Уроженцев Лода, Хадида и Оно семьсот двадцать один.
\par 38 Уроженцев Сенаи три тысячи девятьсот тридцать.
\par 39 Священников, сыновей Иедаии, из дома Иисусова, девятьсот семьдесят три.
\par 40 Сыновей Иммера тысяча пятьдесят два.
\par 41 Сыновей Пашхура тысяча двести сорок семь.
\par 42 Сыновей Харима тысяча семнадцать.
\par 43 Левитов: сыновей Иисуса, из [дома] Кадмиилова, из дома сыновей Годевы, семьдесят четыре.
\par 44 Певцов: сыновей Асафа сто сорок восемь.
\par 45 Привратники: сыновья Шаллума, сыновья Атера, сыновья Талмона, сыновья Аккува, сыновья Хатиты, сыновья Шовая--сто тридцать восемь.
\par 46 Нефинеи: сыновья Цихи, сыновья Хасуфы, сыновья Таббаофа,
\par 47 сыновья Кироса, сыновья Сии, сыновья Фадона,
\par 48 сыновья Леваны, сыновья Хагавы, сыновья Салмая,
\par 49 сыновья Ханана, сыновья Гиддела, сыновья Гахара,
\par 50 сыновья Реаии, сыновья Рецина, сыновья Некоды,
\par 51 сыновья Газзама, сыновья Уззы, сыновья Пасеаха,
\par 52 сыновья Весая, сыновья Меунима, сыновья Нефишсима,
\par 53 сыновья Бакбука, сыновья Хакуфы, сыновья Хархура,
\par 54 сыновья Бацлифа, сыновья Мехиды, сыновья Харши,
\par 55 сыновья Баркоса, сыновья Сисары, сыновья Фамаха,
\par 56 сыновья Нециаха, сыновья Хатифы.
\par 57 Сыновья рабов Соломоновых: сыновья Сотая, сыновья Соферефа, сыновья Фериды,
\par 58 сыновья Иаалы, сыновья Даркона, сыновья Гиддела,
\par 59 сыновья Сафатии, сыновья Хаттила, сыновья Похереф--Гаццевайима, сыновья Амона.
\par 60 Всех нефинеев и сыновей рабов Соломоновых триста девяносто два.
\par 61 И вот вышедшие из Тел-Мелаха, Тел-Харши, Херув-Аддона и Иммера; но они не могли показать о поколении своем и о племени своем, от Израиля ли они.
\par 62 Сыновья Делаии, сыновья Товии, сыновья Некоды--шестьсот сорок два.
\par 63 И из священников: сыновья Ховаии, сыновья Гаккоца, сыновья Верзеллия, который взял жену из дочерей Верзеллия Галаадитянина и стал называться их именем.
\par 64 Они искали родословной своей записи, и не нашлось, и потому исключены из священства.
\par 65 И Тиршафа сказал им, чтоб они не ели великой святыни, доколе не восстанет священник с уримом и туммимом.
\par 66 Все общество вместе [состояло] из сорока двух тысяч трехсот шестидесяти [человек],
\par 67 кроме рабов их и рабынь их, которых было семь тысяч триста тридцать семь; и при них певцов и певиц двести сорок пять.
\par 68 Коней у них было семьсот тридцать шесть, лошаков у них двести сорок пять,
\par 69 верблюдов четыреста тридцать пять, ослов шесть тысяч семьсот двадцать.
\par 70 Некоторые главы поколений дали вклады на производство работ. Тиршафа дал в сокровищницу золотом тысячу драхм, пятьдесят чаш, пятьсот тридцать священнических одежд.
\par 71 И некоторые из глав поколений дали в сокровищницу на производство работ двадцать тысяч драхм золота и две тысячи двести мин серебра.
\par 72 Прочие из народа дали двадцать тысяч драхм золота и две тысячи мин серебра и шестьдесят семь священнических одежд.
\par 73 И стали жить священники и левиты, и привратники и певцы, и народ и нефинеи, и весь Израиль в городах своих.

\chapter{8}

\par 1 Когда наступил седьмой месяц, и сыны Израилевы [жили] по городам своим, тогда собрался весь народ, как один человек, на площадь, которая пред Водяными воротами, и сказали книжнику Ездре, чтобы он принес книгу закона Моисеева, который заповедал Господь Израилю.
\par 2 И принес священник Ездра закон пред собрание мужчин и женщин, и всех, которые могли понимать, в первый день седьмого месяца;
\par 3 и читал из него на площади, которая пред Водяными воротами, от рассвета до полудня, пред мужчинами и женщинами и всеми, которые могли понимать; и уши всего народа [были приклонены] к книге закона.
\par 4 Книжник Ездра стоял на деревянном возвышении, которое для сего сделали, а подле него, по правую руку его, стояли Маттифия и Шема, и Анаия и Урия, и Хелкия и Маасея, а по левую руку его Федаия и Мисаил, и Малхия и Хашум, и Хашбаддана, и Захария и Мешуллам.
\par 5 И открыл Ездра книгу пред глазами всего народа, потому что он стоял выше всего народа. И когда он открыл ее, весь народ встал.
\par 6 И благословил Ездра Господа Бога великого. И весь народ отвечал: аминь, аминь, поднимая вверх руки свои, --и поклонялись и повергались пред Господом лицем до земли.
\par 7 Иисус, Ванаия, Шеревия, Иамин, Аккув, Шавтай, Годия, Маасея, Клита, Азария, Иозавад, Ханан, Фелаия и левиты поясняли народу закон, между тем как народ стоял на своем месте.
\par 8 И читали из книги, из закона Божия, внятно, и присоединяли толкование, и [народ] понимал прочитанное.
\par 9 Тогда Неемия, он же Тиршафа, и книжник Ездра, священник, и левиты, учившие народ, сказали всему народу: день сей свят Господу Богу вашему; не печальтесь и не плачьте, потому что весь народ плакал, слушая слова закона.
\par 10 И сказал им: пойдите, ешьте тучное и пейте сладкое, и посылайте части тем, у кого ничего не приготовлено, потому что день сей свят Господу нашему. И не печальтесь, потому что радость пред Господом--подкрепление для вас.
\par 11 И левиты успокаивали весь народ, говоря: перестаньте, ибо день сей свят, не печальтесь.
\par 12 И пошел весь народ есть, и пить, и посылать части, и праздновать с великим веселием, ибо поняли слова, которые сказали им.
\par 13 На другой день собрались главы поколений от всего народа, священники и левиты к книжнику Ездре, чтобы он изъяснял им слова закона.
\par 14 И нашли написанное в законе, который Господь дал чрез Моисея, чтобы сыны Израилевы в седьмом месяце, в праздник, жили в кущах.
\par 15 И потому объявили и провозгласили по всем городам своим и в Иерусалиме, говоря: пойдите на гору и несите ветви маслины садовой и ветви маслины дикой, и ветви миртовые и ветви пальмовые, и ветви [других] широколиственных дерев, чтобы сделать кущи по написанному.
\par 16 И пошел народ, и принесли, и сделали себе кущи, каждый на своей кровле и на дворах своих, и на дворах дома Божия, и на площади у Водяных ворот, и на площади у Ефремовых ворот.
\par 17 Все общество возвратившихся из плена сделало кущи и жило в кущах. От дней Иисуса, сына Навина, до этого дня не делали так сыны Израилевы. Радость была весьма великая.
\par 18 И читали из книги закона Божия каждый день, от первого дня до последнего дня. И праздновали праздник семь дней, а в восьмой день попразднество по уставу.

\chapter{9}

\par 1 В двадцать четвертый день этого месяца собрались все сыны Израилевы, постящиеся и во вретищах и с пеплом на головах своих.
\par 2 И отделилось семя Израилево от всех инородных, и встали и исповедывались во грехах своих и в преступлениях отцов своих.
\par 3 И стояли на своем месте, и четверть дня читали из книги закона Господа Бога своего, и четверть исповедывались и поклонялись Господу Богу своему.
\par 4 И стали на возвышенное место левитов: Иисус, Вания, Кадмиил, Шевания, Вунний, Шеревия, Вания, Хенани, и громко взывали к Господу Богу своему.
\par 5 И сказали левиты--Иисус, Кадмиил, Вания, Хашавния, Шеревия, Годия, Шевания, Петахия: встаньте, славьте Господа Бога вашего, от века и до века. Да славословят достославное и превысшее всякого славословия и хвалы имя Твое!
\par 6 Ты, Господи, един, Ты создал небо, небеса небес и все воинство их, землю и все, что на ней, моря и все, что в них, и Ты живишь все сие, и небесные воинства Тебе поклоняются.
\par 7 Ты Сам, Господи Боже, избрал Аврама, и вывел его из Ура Халдейского, и дал ему имя Авраама,
\par 8 и нашел сердце его верным пред Тобою, и заключил с ним завет, чтобы дать семени его землю Хананеев, Хеттеев, Аморреев, Ферезеев, Иевусеев и Гергесеев. И Ты исполнил слово Свое, потому что Ты праведен.
\par 9 Ты увидел бедствие отцов наших в Египте и услышал вопль их у Чермного моря,
\par 10 и явил знамения и чудеса над фараоном и над всеми рабами его, и над всем народом земли его, так как Ты знал, что они надменно поступали с ними, и сделал Ты Себе имя до сего дня.
\par 11 Ты рассек пред ними море, и они среди моря прошли посуху, и гнавшихся за ними Ты поверг в глубины, как камень в сильные воды.
\par 12 В столпе облачном Ты вел их днем и в столпе огненном--ночью, чтоб освещать им путь, по которому идти им.
\par 13 И снисшел Ты на гору Синай и говорил с ними с неба, и дал им суды справедливые, законы верные, уставы и заповеди добрые.
\par 14 И указал им святую Твою субботу и заповеди, и уставы и закон преподал им чрез раба Твоего Моисея.
\par 15 И хлеб с неба Ты давал им в голоде их, и воду из камня источал им в жажде их, и сказал им, чтоб они пошли и овладели землею, которую Ты, подняв руку Твою, [клялся] дать им.
\par 16 Но они и отцы наши упрямствовали, и шею свою держали упруго, и не слушали заповедей Твоих;
\par 17 не захотели повиноваться и не вспомнили чудных дел Твоих, которые Ты делал с ними, и держали шею свою упруго, и, по упорству своему, поставили над собою вождя, чтобы возвратиться в рабство свое. Но Ты Бог, любящий прощать, благий и милосердый, долготерпеливый и многомилостивый, и Ты не оставил их.
\par 18 И хотя они сделали себе литаго тельца, и сказали: вот бог твой, который вывел тебя из Египта, и хотя делали великие оскорбления,
\par 19 но Ты, по великому милосердию Твоему, не оставлял их в пустыне; столп облачный не отходил от них днем, чтобы вести их по пути, и столп огненный--ночью, чтобы светить им на пути, по которому им идти.
\par 20 И Ты дал им Духа Твоего благого, чтобы наставлять их, и манну Твою не отнимал от уст их, и воду давал им для утоления жажды их.
\par 21 Сорок лет Ты питал их в пустыне; они ни в чем не терпели недостатка; одежды их не ветшали, и ноги их не пухли.
\par 22 И Ты дал им царства и народы и разделил им, и они овладели землею Сигона, и землею царя Есевонского, и землею Ога, царя Васанского.
\par 23 И сыновей их Ты размножил, как звезды небесные, и ввел их в землю, о которой Ты говорил отцам их, что они придут владеть [ею].
\par 24 И вошли сыновья их, и овладели землею. И Ты покорил им жителей земли, Хананеев, и отдал их в руки их, и царей их, и народы земли, чтобы они поступали с ними по своей воле.
\par 25 И заняли они укрепленные города и тучную землю, и взяли во владение домы, наполненные всяким добром, водоемы, высеченные [из камня], виноградные и масличные сады и множество дерев [с плодами] для пищи. Они ели, насыщались, тучнели и наслаждались по великой благости Твоей;
\par 26 и сделались упорны и возмутились против Тебя, и презрели закон Твой, убивали пророков Твоих, которые увещевали их обратиться к Тебе, и делали великие оскорбления.
\par 27 И Ты отдал их в руки врагов их, которые теснили их. Но когда, в тесное для них время, они взывали к Тебе, Ты выслушивал их с небес и, по великому милосердию Твоему, давал им спасителей, и они спасали их от рук врагов их.
\par 28 Когда же успокаивались, то снова начинали делать зло пред лицем Твоим, и Ты отдавал их в руки неприятелей их, и они господствовали над ними. Но когда они опять взывали к Тебе, Ты выслушивал их с небес и, по великому милосердию Твоему, избавлял их многократно.
\par 29 Ты напоминал им обратиться к закону Твоему, но они упорствовали и не слушали заповедей Твоих, и отклонялись от уставов Твоих, которыми жил бы человек, если бы исполнял их, и хребет [свой] сделали упорным, и шею свою держали упруго, и не слушали.
\par 30 Ожидая их [обращения], Ты медлил многие годы и напоминал им Духом Твоим чрез пророков Твоих, но они не слушали. И Ты предал их в руки иноземных народов.
\par 31 Но, по великому милосердию Твоему, Ты не истребил их до конца, и не оставлял их, потому что Ты Бог благий и милостивый.
\par 32 И ныне, Боже наш, Боже великий, сильный и страшный, хранящий завет и милость! да не будет малым пред лицем Твоим все страдание, которое постигло нас, царей наших, князей наших, и священников наших, и пророков наших, и отцов наших и весь народ Твой от дней царей Ассирийских до сего дня.
\par 33 Во всем постигшем нас Ты праведен, потому что Ты делал по правде, а мы виновны.
\par 34 Цари наши, князья наши, священники наши и отцы наши не исполняли закона Твоего, и не внимали заповедям Твоим и напоминаниям Твоим, которыми Ты напоминал им.
\par 35 И в царстве своем, при великом добре Твоем, которое Ты давал им, и на обширной и тучной земле, которую Ты отделил им, они не служили Тебе и не обращались от злых дел своих.
\par 36 И вот, мы ныне рабы; на той земле, которую Ты дал отцам нашим, чтобы питаться ее плодами и ее добром, вот, мы рабствуем.
\par 37 И произведения свои она во множестве приносит для царей, которым Ты покорил нас за грехи наши. И телами нашими и скотом нашим они владеют по своему произволу, и мы в великом стеснении.
\par 38 По всему этому мы даем твердое обязательство и подписываем, и на подписи печать князей наших, левитов наших и священников наших.

\chapter{10}

\par 1 Приложившие печати были: Неемия-Тиршафа, сын Гахалии, и Седекия,
\par 2 Сераия, Азария, Иеремия,
\par 3 Пашхур, Амария, Малхия,
\par 4 Хаттуш, Шевания, Маллух,
\par 5 Харим, Меремоф, Овадия,
\par 6 Даниил, Гиннефон, Варух,
\par 7 Мешуллам, Авия, Миямин,
\par 8 Маазия, Вилгай, Шемаия: это священники.
\par 9 Левиты: Иисус, сын Азании, Биннуй, из сыновей Хенадада, Кадмиил,
\par 10 и братья их: Шевания, Годия, Клита, Фелаия, Ханан,
\par 11 Миха, Рехов, Хашавия,
\par 12 Закхур, Шеревия, Шевания,
\par 13 Годия, Ваний, Венинуй.
\par 14 Главы народа: Парош, Пахаф-Моав, Елам, Заффу, Вания,
\par 15 Вунний, Азгар, Бевай,
\par 16 Адония, Бигвай, Адин,
\par 17 Атер, Езекия, Азур,
\par 18 Годия, Хашум, Бецай,
\par 19 Хариф, Анафоф, Невай,
\par 20 Магпиаш, Мешуллам, Хезир,
\par 21 Мешезавел, Садок, Иаддуй,
\par 22 Фелатия, Ханан, Анаия,
\par 23 Осия, Ханания, Хашшув,
\par 24 Лохеш, Пилха, Шовек,
\par 25 Рехум, Хашавна, Маасея,
\par 26 Ахия, Ханан, Анан,
\par 27 Маллух, Харим, Ваана.
\par 28 И прочий народ, священники, левиты, привратники, певцы, нефинеи и все, отделившиеся от народов иноземных к закону Божию, жены их, сыновья их и дочери их, все, которые могли понимать,
\par 29 пристали к братьям своим, к почетнейшим из них, и вступили в обязательство с клятвою и проклятием--поступать по закону Божию, который дан рукою Моисея, раба Божия, и соблюдать и исполнять все заповеди Господа Бога нашего, и уставы Его и предписания Его,
\par 30 и не отдавать дочерей своих иноземным народам, и их дочерей не брать за сыновей своих;
\par 31 и когда иноземные народы будут привозить товары и все продажное в субботу, не брать у них в субботу и в священный день, и в седьмой год оставлять долги всякого рода.
\par 32 И поставили мы себе в закон давать от себя по трети сикля в год на потребности для дома Бога нашего:
\par 33 на хлебы предложения, на всегдашнее хлебное приношение и на всегдашнее всесожжение, на субботы, на новомесячия, на праздники, на священные вещи и на жертвы за грех для очищения Израиля, и на все, совершаемое в доме Бога нашего.
\par 34 И бросили мы жребии о доставке дров, священники, левиты и народ, когда которому поколению нашему в назначенные времена, из года в год, привозить [их] к дому Бога нашего, чтоб они горели на жертвеннике Господа Бога нашего, по написанному в законе.
\par 35 [И обязались мы] каждый год приносить в дом Господень начатки с земли нашей и начатки всяких плодов со всякого дерева;
\par 36 также приводить в дом Бога нашего к священникам, служащим в доме Бога нашего, первенцев из сыновей наших и из скота нашего, как написано в законе, и первородное от крупного и мелкого скота нашего.
\par 37 И начатки из молотого хлеба нашего и приношений наших, и плодов со всякого дерева, вина и масла мы будем доставлять священникам в кладовые при доме Бога нашего и десятину с земли нашей левитам. Они, левиты, будут брать десятину во всех городах, где у нас земледелие.
\par 38 При левитах, когда они будут брать левитскую десятину, будет находиться священник, сын Аарона, чтобы левиты десятину из своих десятин отвозили в дом Бога нашего в комнаты, [отделенные] для кладовой,
\par 39 потому что в эти комнаты как сыны Израилевы, так и левиты должны доставлять приносимое в дар: хлеб, вино и масло. Там священные сосуды, и служащие священники, и привратники, и певцы. И мы не оставим дома Бога нашего.

\chapter{11}

\par 1 И жили начальники народа в Иерусалиме, а прочие из народа бросили жребии, чтоб одна из десяти частей их шла на жительство в святой город Иерусалим, а девять [оставались] в [прочих] городах.
\par 2 И благословил народ всех, которые добровольно согласились жить в Иерусалиме.
\par 3 Вот главы страны, которые жили в Иерусалиме, --а в городах Иудеи жили, всякий в своем владении, по городам своим: Израильтяне, священники, левиты и нефинеи и сыновья рабов Соломоновых; --
\par 4 в Иерусалиме жили из сыновей Иуды и из сыновей Вениамина. Из сыновей Иуды: Афаия, сын Уззии, сын Захарии, сын Амарии, сын Сафатии, сын Малелеила, из сыновей Фареса,
\par 5 и Маасея, сын Варуха, сын Колхозея, сын Хазаии, сын Адаии, сын Иоиарива, сын Захарии, сын Шилония.
\par 6 Всех сыновей Фареса, живших в Иерусалиме, четыреста шестьдесят восемь, люди отличные.
\par 7 И вот сыновья Вениамина: Саллу, сын Мешуллама, сын Иоеда, сын Федаии, сын Колаии, сын Маасеи, сын Ифиила, сын Исаии,
\par 8 и за ним Габбай, Саллай--девятьсот двадцать восемь.
\par 9 Иоиль, сын Зихри, был начальником над ними, а Иуда, сын Сенуи, был вторым над городом.
\par 10 Из священников: Иедаия, сын Иоиарива, Иахин,
\par 11 Сераия, сын Хелкии, сын Мешуллама, сын Садока, сын Мераиофа, сын Ахитува, начальствующий в доме Божием,
\par 12 и братья их, отправлявшие службу в доме [Божием] --восемьсот двадцать два; и Адаия, сын Иерохама, сын Фелалии, сын Амция, сын Захарии, сын Пашхура, сын Малхии,
\par 13 и братья его, главы поколений--двести сорок два; и Амашсай, сын Азариила, сын Ахзая, сын Мешиллемофа, сын Иммера,
\par 14 и братья его, люди отличные--сто двадцать восемь. Начальником над ними был Завдиил, сын Гагедолима.
\par 15 А из левитов: Шемаия, сын Хашшува, сын Азрикама, сын Хашавии, сын Вунния,
\par 16 и Шавфай, и Иозавад, из глав левитов по внешним делам дома Божия,
\par 17 и Матфания, сын Михи, сын Завдия, сын Асафа, главный начинатель славословия при молитве, и Бакбукия, второй [по нем] из братьев его, и Авда, сын Шаммуя, сын Галала, сын Идифуна.
\par 18 Всех левитов во святом городе двести восемьдесят четыре.
\par 19 А привратники: Аккув, Талмон и братья их, содержавшие стражу у ворот--сто семьдесят два.
\par 20 Прочие Израильтяне, священники, левиты [жили] по всем городам Иудеи, каждый в своем уделе.
\par 21 А нефинеи жили в Офеле; над нефинеями Циха и Гишфа.
\par 22 Начальником над левитами в Иерусалиме был Уззий, сын Вания, сын Хашавии, сын Матфании, сын Михи, из сыновей Асафовых, которые были певцами при служении в доме Божием,
\par 23 потому что от царя [было] о них [особое] повеление, и назначено было на каждый день для певцов определенное содержание.
\par 24 И Петахия, сын Мешезавела, из сыновей Зары, сына Иуды, был доверенным от царя по всяким делам, [касающимся] до народа.
\par 25 Из [живших] же в селах, на полях своих, сыновья Иуды жили в Кириаф-Арбе и зависящих от нее городах, в Дивоне и зависящих от него городах, в Иекавцеиле и селах его,
\par 26 в Иешуе, в Моладе и в Беф-Палете,
\par 27 в Хацар-Шуале, в Вирсавии и зависящих от нее городах,
\par 28 в Секелаге, в Мехоне и зависящих от нее городах,
\par 29 в Ен-Риммоне, в Цоре и в Иармуфе,
\par 30 в Заноахе, Одолламе и селах их, в Лахисе и на полях его, в Азеке и зависящих от нее городах. Они расположились от Вирсавии и до долины Енномовой.
\par 31 Сыновья Вениаминовы, [начиная] от Гевы, в Михмасе, Гае, в Вефиле и зависящих от него городах,
\par 32 в Анафофе, Нове, Анании,
\par 33 Гацоре, Раме, Гиффаиме,
\par 34 Хадиде, Цевоиме, Неваллате,
\par 35 Лоде, Оно, в долине Харашиме.
\par 36 И левиты имели жилища свои в участках Иуды и Вениамина.

\chapter{12}

\par 1 Вот священники и левиты, которые пришли с Зоровавелем, сыном Салафииловым, и с Иисусом: Сераия, Иеремия, Ездра,
\par 2 Амария, Маллух, Хаттуш,
\par 3 Шехания, Рехум, Меремоф,
\par 4 Иддо, Гиннефой, Авия,
\par 5 Миямин, Маадия, Вилга,
\par 6 Шемаия, Иоиарив, Иедаия,
\par 7 Саллу, Амок, Хелкия, Иедаия. Это главы священников и братья их во дни Иисуса.
\par 8 А левиты: Иисус, Биннуй, Кадмиил, Шеревия, Иуда, Матфания, [главный] при славословии, он и братья его,
\par 9 и Бакбукия и Унний, братья их, наряду с ними [державшие] стражу.
\par 10 Иисус родил Иоакима, Иоаким родил Елиашива, Елиашив родил Иоиаду,
\par 11 Иоиада родил Ионафана, Ионафан родил Иаддуя.
\par 12 Во дни Иоакима были священники, главы поколений: из [дома] Сераии Мераия, из [дома] Иеремии Ханания,
\par 13 из [дома] Ездры Мешуллам, из [дома] Амарии Иоханан,
\par 14 из [дома] Мелиху Ионафан, из [дома] Шевании Иосиф,
\par 15 из [дома] Харима Адна, из [дома] Мераиофа Хелкия,
\par 16 из [дома] Иддо Захария, из [дома] Гиннефона Мешуллам,
\par 17 из [дома] Авии Зихрий, из [дома] Миниамина, из [дома] Моадии Пилтай,
\par 18 из [дома] Вилги Шаммуй, из [дома] Шемаии Ионафан,
\par 19 из [дома] Иоиарива Мафнай, из [дома] Иедаии Уззий,
\par 20 из [дома] Саллая Каллай, из [дома] Амока Евер,
\par 21 из [дома] Хелкии Хашавия, из [дома] Иедаии Нафанаил.
\par 22 Левиты, главы поколений, внесены в запись во дни Елиашива, Иоиады, Иоханана и Иаддуя, и также священники в царствование Дария Персидского.
\par 23 Сыновья Левия, главы поколений, вписаны в летописи до дней Иоханана, сына Елиашивова.
\par 24 Главы левитов: Хашавия, Шеревия, и Иисус, сын Кадмиила, и братья их, при них [поставленные] для славословия при благодарениях, по установлению Давида, человека Божия--смена за сменою.
\par 25 Матфания, Бакбукия, Овадия, Мешуллам, Талмон, Аккув--стражи, привратники на страже у порогов ворот.
\par 26 Они были во дни Иоакима, сына Иисусова, сына Иоседекова, и во дни областеначальника Неемии и книжника Ездры, священника.
\par 27 При освящении стены Иерусалимской потребовали левитов из всех мест их, приказывая им придти в Иерусалим для совершения освящения и радостного празднества со славословиями и песнями при [звуке] кимвалов, псалтирей и гуслей.
\par 28 И собрались сыновья певцов из округа Иерусалимского и из сел Нетофафских,
\par 29 и из Беф-Гаггилгала, и с полей Гевы и Азмавета, потому что певцы выстроили себе села в окрестностях Иерусалима.
\par 30 И очистились священники и левиты, и очистили народ и ворота, и стену.
\par 31 Тогда я повел начальствующих в Иудее на стену и поставил два больших хора для шествия, и один из них шел по правой стороне стены к Навозным воротам.
\par 32 За ними шел Гошаия и половина начальствующих в Иудее,
\par 33 Азария, Ездра и Мешуллам,
\par 34 Иуда и Вениамин, и Шемаия и Иеремия,
\par 35 а из сыновей священнических с трубами: Захария, сын Ионафана, сын Шемаии, сын Матфании, сын Михея, сын Закхура, сын Асафа,
\par 36 и братья его: Шемаия, Азариил, Милалай, Гилалай, Маай, Нафанаил, Иуда и Хананий с музыкальными орудиями Давида, человека Божия, и книжник Ездра впереди них.
\par 37 Подле ворот Источника, против них, они взошли по ступеням города Давидова, по лестнице, ведущей на стену сверх дома Давидова до Водяных ворот к востоку.
\par 38 Другой хор шел напротив них, и за ним я и половина народа, по стене от Печной башни и до широкой стены,
\par 39 и от ворот Ефремовых, мимо старых ворот и ворот Рыбных, и башни Хананела, и башни Меа, к Овечьим воротам, и остановились у ворот Темничных.
\par 40 Потом оба хора стали у дома Божия, и я и половина начальствующих со мною,
\par 41 и священники: Елиаким, Маасея, Миниамин, Михей, Елиоенай, Захария, Ханания с трубами,
\par 42 и Маасея и Шемаия, и Елеазар и Уззий, и Иоханан и Малхия, и Елам и Езер. И пели певцы громко; главным [у них был] Израхия.
\par 43 И приносили в тот день большие жертвы и веселились, потому что Бог дал им великую радость. Веселились и жены и дети, и веселие Иерусалима далеко было слышно.
\par 44 В тот же день приставлены были люди к кладовым комнатам для приношений начатков и десятин, чтобы собирать с полей при городах части, положенные законом для священников и левитов, потому что Иудеям радостно было [смотреть] на стоящих священников и левитов,
\par 45 которые совершали службу Богу своему и дела очищения и были певцами и привратниками по установлению Давида и сына его Соломона.
\par 46 Ибо издавна во дни Давида и Асафа были установлены главы певцов и песни Богу, хвалебные и благодарственные.
\par 47 Все Израильтяне во дни Зоровавеля и во дни Неемии давали части певцам и привратникам на каждый день и отдавали святыни левитам, а левиты отдавали святыни сынам Аарона.

\chapter{13}

\par 1 В тот день читано было из книги Моисеевой вслух народа и найдено написанное в ней: Аммонитянин и Моавитянин не может войти в общество Божие во веки,
\par 2 потому что они не встретили сынов Израиля с хлебом и водою и наняли против него Валаама, чтобы проклясть его, но Бог наш обратил проклятие в благословение.
\par 3 Услышав этот закон, они отделили все иноплеменное от Израиля.
\par 4 А прежде того священник Елиашив, приставленный к комнатам при доме Бога нашего, близкий родственник Товии,
\par 5 отделал для него большую комнату, в которую прежде клали хлебное приношение, ладан и сосуды, и десятины хлеба, вина и масла, положенные законом для левитов, певцов и привратников, и приношения для священников.
\par 6 Когда все это [происходило], я не был в Иерусалиме, потому что в тридцать втором году Вавилонского царя Артаксеркса я ходил к царю, и по прошествии нескольких дней [опять] выпросился у царя.
\par 7 Когда я пришел в Иерусалим и узнал о худом деле, которое сделал Елиашив, отделав для Товии комнату на дворах дома Божия,
\par 8 тогда мне было весьма неприятно, и я выбросил все домашние вещи Товиины вон из комнаты
\par 9 и сказал, чтобы очистили комнаты, и велел опять внести туда сосуды дома Божия, хлебное приношение и ладан.
\par 10 Еще узнал я, что части левитам не отдаются, и что левиты и певцы, делавшие [свое] дело, разбежались, каждый на свое поле.
\par 11 Я сделал [за это] выговор начальствующим и сказал: зачем оставлен нами дом Божий? И я собрал их и поставил их на место их.
\par 12 И все Иудеи стали приносить десятины хлеба, вина и масла в кладовые.
\par 13 И приставил я к кладовым Шелемию священника и Садока книжника и Федаию из левитов, и при них Ханана, сына Закхура, сына Матфании, потому что они считались верными. И на них [возложено] раздавать части братьям своим.
\par 14 Помяни меня за это, Боже мой, и не изгладь усердных дел моих, которые я сделал для дома Бога моего и для служения при нем!
\par 15 В те дни я увидел в Иудее, что в субботу топчут точила, возят снопы и навьючивают ослов вином, виноградом, смоквами и всяким грузом, и отвозят в субботний день в Иерусалим. И я строго выговорил [им] в тот же день, когда они продавали съестное.
\par 16 И Тиряне жили в [Иудее] и привозили рыбу и всякий товар и продавали в субботу жителям Иудеи и в Иерусалиме.
\par 17 И я сделал выговор знатнейшим из Иудеев и сказал им: зачем вы делаете такое зло и оскверняете день субботний?
\par 18 Не так ли поступали отцы ваши, и за то Бог наш навел на нас и на город сей все это бедствие? А вы увеличиваете гнев [Его] на Израиля, оскверняя субботу.
\par 19 После сего, когда смеркалось у ворот Иерусалимских, перед субботою, я велел запирать двери и сказал, чтобы не отпирали их до [утра] после субботы. И слуг моих я ставил у ворот, чтобы никакая ноша не проходила в день субботний.
\par 20 И ночевали торговцы и продавцы всякого товара вне Иерусалима раз и два.
\par 21 Но я строго выговорил им и сказал им: зачем вы ночуете возле стены? Если сделаете это в другой раз, я наложу руку на вас. С того времени они не приходили в субботу.
\par 22 И сказал я левитам, чтобы они очистились и пришли содержать стражу у ворот, дабы святить день субботний. И за сие помяни меня, Боже мой, и пощади меня по великой милости Твоей!
\par 23 Еще в те дни я видел Иудеев, которые взяли себе жен из Азотянок, Аммонитянок и Моавитянок;
\par 24 и оттого сыновья их в половину говорят по-азотски, или языком других народов, и не умеют говорить по-иудейски.
\par 25 Я сделал за это выговор и проклинал их, и некоторых из мужей бил, рвал у них волоса и заклинал их Богом, чтобы они не отдавали дочерей своих за сыновей их и не брали дочерей их за сыновей своих и за себя.
\par 26 Не из-за них ли, [говорил я,] грешил Соломон, царь Израилев? У многих народов не было такого царя, как он. Он был любим Богом своим, и Бог поставил его царем над всеми Израильтянами; и однако же чужеземные жены ввели в грех и его.
\par 27 И можно ли нам слышать о вас, что вы делаете все сие великое зло, грешите пред Богом нашим, принимая в сожительство чужеземных жен?
\par 28 И из сыновей Иоиады, сына великого священника Елиашива, один был зятем Санаваллата, Хоронита. Я прогнал его от себя.
\par 29 Воспомяни им, Боже мой, что они опорочили священство и завет священнический и левитский!
\par 30 Так очистил я их от всего чужеземного и восстановил службы священников и левитов, каждого в деле его,
\par 31 и доставку дров в назначенные времена и начатки. Помяни меня, Боже мой, во благо [мне]!


\end{document}