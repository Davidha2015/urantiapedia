\begin{document}

\title{Иова}


\chapter{1}

\par 1 Был человек в земле Уц, имя его Иов; и был человек этот непорочен, справедлив и богобоязнен и удалялся от зла.
\par 2 И родились у него семь сыновей и три дочери.
\par 3 Имения у него было: семь тысяч мелкого скота, три тысячи верблюдов, пятьсот пар волов и пятьсот ослиц и весьма много прислуги; и был человек этот знаменитее всех сынов Востока.
\par 4 Сыновья его сходились, делая пиры каждый в своем доме в свой день, и посылали и приглашали трех сестер своих есть и пить с ними.
\par 5 Когда круг пиршественных дней совершался, Иов посылал [за ними] и освящал их и, вставая рано утром, возносил всесожжения по числу всех их. Ибо говорил Иов: может быть, сыновья мои согрешили и похулили Бога в сердце своем. Так делал Иов во все [такие] дни.
\par 6 И был день, когда пришли сыны Божии предстать пред Господа; между ними пришел и сатана.
\par 7 И сказал Господь сатане: откуда ты пришел? И отвечал сатана Господу и сказал: я ходил по земле и обошел ее.
\par 8 И сказал Господь сатане: обратил ли ты внимание твое на раба Моего Иова? ибо нет такого, как он, на земле: человек непорочный, справедливый, богобоязненный и удаляющийся от зла.
\par 9 И отвечал сатана Господу и сказал: разве даром богобоязнен Иов?
\par 10 Не Ты ли кругом оградил его и дом его и все, что у него? Дело рук его Ты благословил, и стада его распространяются по земле;
\par 11 но простри руку Твою и коснись всего, что у него, --благословит ли он Тебя?
\par 12 И сказал Господь сатане: вот, все, что у него, в руке твоей; только на него не простирай руки твоей. И отошел сатана от лица Господня.
\par 13 И был день, когда сыновья его и дочери его ели и вино пили в доме первородного брата своего.
\par 14 И [вот], приходит вестник к Иову и говорит:
\par 15 волы орали, и ослицы паслись подле них, как напали Савеяне и взяли их, а отроков поразили острием меча; и спасся только я один, чтобы возвестить тебе.
\par 16 Еще он говорил, как приходит другой и сказывает: огонь Божий упал с неба и опалил овец и отроков и пожрал их; и спасся только я один, чтобы возвестить тебе.
\par 17 Еще он говорил, как приходит другой и сказывает: Халдеи расположились тремя отрядами и бросились на верблюдов и взяли их, а отроков поразили острием меча; и спасся только я один, чтобы возвестить тебе.
\par 18 Еще этот говорил, приходит другой и сказывает: сыновья твои и дочери твои ели и вино пили в доме первородного брата своего;
\par 19 и вот, большой ветер пришел от пустыни и охватил четыре угла дома, и дом упал на отроков, и они умерли; и спасся только я один, чтобы возвестить тебе.
\par 20 Тогда Иов встал и разодрал верхнюю одежду свою, остриг голову свою и пал на землю и поклонился
\par 21 и сказал: наг я вышел из чрева матери моей, наг и возвращусь. Господь дал, Господь и взял; да будет имя Господне благословенно!
\par 22 Во всем этом не согрешил Иов и не произнес ничего неразумного о Боге.

\chapter{2}

\par 1 Был день, когда пришли сыны Божии предстать пред Господа; между ними пришел и сатана предстать пред Господа.
\par 2 И сказал Господь сатане: откуда ты пришел? И отвечал сатана Господу и сказал: я ходил по земле и обошел ее.
\par 3 И сказал Господь сатане: обратил ли ты внимание твое на раба Моего Иова? ибо нет такого, как он, на земле: человек непорочный, справедливый, богобоязненный и удаляющийся от зла, и доселе тверд в своей непорочности; а ты возбуждал Меня против него, чтобы погубить его безвинно.
\par 4 И отвечал сатана Господу и сказал: кожу за кожу, а за жизнь свою отдаст человек все, что есть у него;
\par 5 но простри руку Твою и коснись кости его и плоти его, --благословит ли он Тебя?
\par 6 И сказал Господь сатане: вот, он в руке твоей, только душу его сбереги.
\par 7 И отошел сатана от лица Господня и поразил Иова проказою лютою от подошвы ноги его по самое темя его.
\par 8 И взял он себе черепицу, чтобы скоблить себя ею, и сел в пепел.
\par 9 И сказала ему жена его: ты все еще тверд в непорочности твоей! похули Бога и умри.
\par 10 Но он сказал ей: ты говоришь как одна из безумных: неужели доброе мы будем принимать от Бога, а злого не будем принимать? Во всем этом не согрешил Иов устами своими.
\par 11 И услышали трое друзей Иова о всех этих несчастьях, постигших его, и пошли каждый из своего места: Елифаз Феманитянин, Вилдад Савхеянин и Софар Наамитянин, и сошлись, чтобы идти вместе сетовать с ним и утешать его.
\par 12 И подняв глаза свои издали, они не узнали его; и возвысили голос свой и зарыдали; и разодрал каждый верхнюю одежду свою, и бросали пыль над головами своими к небу.
\par 13 И сидели с ним на земле семь дней и семь ночей; и никто не говорил ему ни слова, ибо видели, что страдание его весьма велико.

\chapter{3}

\par 1 После того открыл Иов уста свои и проклял день свой.
\par 2 И начал Иов и сказал:
\par 3 погибни день, в который я родился, и ночь, в которую сказано: зачался человек!
\par 4 День тот да будет тьмою; да не взыщет его Бог свыше, и да не воссияет над ним свет!
\par 5 Да омрачит его тьма и тень смертная, да обложит его туча, да страшатся его, как палящего зноя!
\par 6 Ночь та, --да обладает ею мрак, да не сочтется она в днях года, да не войдет в число месяцев!
\par 7 О! ночь та--да будет она безлюдна; да не войдет в нее веселье!
\par 8 Да проклянут ее проклинающие день, способные разбудить левиафана!
\par 9 Да померкнут звезды рассвета ее: пусть ждет она света, и он не приходит, и да не увидит она ресниц денницы
\par 10 за то, что не затворила дверей чрева [матери] моей и не сокрыла горести от очей моих!
\par 11 Для чего не умер я, выходя из утробы, и не скончался, когда вышел из чрева?
\par 12 Зачем приняли меня колени? зачем было мне сосать сосцы?
\par 13 Теперь бы лежал я и почивал; спал бы, и мне было бы покойно
\par 14 с царями и советниками земли, которые застраивали для себя пустыни,
\par 15 или с князьями, у которых было золото, и которые наполняли домы свои серебром;
\par 16 или, как выкидыш сокрытый, я не существовал бы, как младенцы, не увидевшие света.
\par 17 Там беззаконные перестают наводить страх, и там отдыхают истощившиеся в силах.
\par 18 Там узники вместе наслаждаются покоем и не слышат криков приставника.
\par 19 Малый и великий там равны, и раб свободен от господина своего.
\par 20 На что дан страдальцу свет, и жизнь огорченным душею,
\par 21 которые ждут смерти, и нет ее, которые вырыли бы ее охотнее, нежели клад,
\par 22 обрадовались бы до восторга, восхитились бы, что нашли гроб?
\par 23 [На что дан свет] человеку, которого путь закрыт, и которого Бог окружил мраком?
\par 24 Вздохи мои предупреждают хлеб мой, и стоны мои льются, как вода,
\par 25 ибо ужасное, чего я ужасался, то и постигло меня; и чего я боялся, то и пришло ко мне.
\par 26 Нет мне мира, нет покоя, нет отрады: постигло несчастье.

\chapter{4}

\par 1 И отвечал Елифаз Феманитянин и сказал:
\par 2 [если] попытаемся мы [сказать] к тебе слово, --не тяжело ли будет тебе? Впрочем кто может возбранить слову!
\par 3 Вот, ты наставлял многих и опустившиеся руки поддерживал,
\par 4 падающего восставляли слова твои, и гнущиеся колени ты укреплял.
\par 5 А теперь дошло до тебя, и ты изнемог; коснулось тебя, и ты упал духом.
\par 6 Богобоязненность твоя не должна ли быть твоею надеждою, и непорочность путей твоих--упованием твоим?
\par 7 Вспомни же, погибал ли кто невинный, и где праведные бывали искореняемы?
\par 8 Как я видал, то оравшие нечестие и сеявшие зло пожинают его;
\par 9 от дуновения Божия погибают и от духа гнева Его исчезают.
\par 10 Рев льва и голос рыкающего [умолкает], и зубы скимнов сокрушаются;
\par 11 могучий лев погибает без добычи, и дети львицы рассеиваются.
\par 12 И вот, ко мне тайно принеслось слово, и ухо мое приняло нечто от него.
\par 13 Среди размышлений о ночных видениях, когда сон находит на людей,
\par 14 объял меня ужас и трепет и потряс все кости мои.
\par 15 И дух прошел надо мною; дыбом стали волосы на мне.
\par 16 Он стал, --но я не распознал вида его, --только облик был пред глазами моими; тихое веяние, --и я слышу голос:
\par 17 человек праведнее ли Бога? и муж чище ли Творца своего?
\par 18 Вот, Он и слугам Своим не доверяет и в Ангелах Своих усматривает недостатки:
\par 19 тем более--в обитающих в храминах из брения, которых основание прах, которые истребляются скорее моли.
\par 20 Между утром и вечером они распадаются; не увидишь, как они вовсе исчезнут.
\par 21 Не погибают ли с ними и достоинства их? Они умирают, не достигнув мудрости.

\chapter{5}

\par 1 Взывай, если есть отвечающий тебе. И к кому из святых обратишься ты?
\par 2 Так, глупца убивает гневливость, и несмысленного губит раздражительность.
\par 3 Видел я, как глупец укореняется, и тотчас проклял дом его.
\par 4 Дети его далеки от счастья, их будут бить у ворот, и не будет заступника.
\par 5 Жатву его съест голодный и из-за терна возьмет ее, и жаждущие поглотят имущество его.
\par 6 Так, не из праха выходит горе, и не из земли вырастает беда;
\par 7 но человек рождается на страдание, [как] искры, чтобы устремляться вверх.
\par 8 Но я к Богу обратился бы, предал бы дело мое Богу,
\par 9 Который творит дела великие и неисследимые, чудные без числа,
\par 10 дает дождь на лице земли и посылает воды на лице полей;
\par 11 униженных поставляет на высоту, и сетующие возносятся во спасение.
\par 12 Он разрушает замыслы коварных, и руки их не довершают предприятия.
\par 13 Он уловляет мудрецов их же лукавством, и совет хитрых становится тщетным:
\par 14 днем они встречают тьму и в полдень ходят ощупью, как ночью.
\par 15 Он спасает бедного от меча, от уст их и от руки сильного.
\par 16 И есть несчастному надежда, и неправда затворяет уста свои.
\par 17 Блажен человек, которого вразумляет Бог, и потому наказания Вседержителева не отвергай,
\par 18 ибо Он причиняет раны и Сам обвязывает их; Он поражает, и Его же руки врачуют.
\par 19 В шести бедах спасет тебя, и в седьмой не коснется тебя зло.
\par 20 Во время голода избавит тебя от смерти, и на войне--от руки меча.
\par 21 От бича языка укроешь себя и не убоишься опустошения, когда оно придет.
\par 22 Опустошению и голоду посмеешься и зверей земли не убоишься,
\par 23 ибо с камнями полевыми у тебя союз, и звери полевые в мире с тобою.
\par 24 И узнаешь, что шатер твой в безопасности, и будешь смотреть за домом твоим, и не согрешишь.
\par 25 И увидишь, что семя твое многочисленно, и отрасли твои, как трава на земле.
\par 26 Войдешь во гроб в зрелости, как укладываются снопы пшеницы в свое время.
\par 27 Вот, что мы дознали; так оно и есть: выслушай это и заметь для себя.

\chapter{6}

\par 1 И отвечал Иов и сказал:
\par 2 о, если бы верно взвешены были вопли мои, и вместе с ними положили на весы страдание мое!
\par 3 Оно верно перетянуло бы песок морей! Оттого слова мои неистовы.
\par 4 Ибо стрелы Вседержителя во мне; яд их пьет дух мой; ужасы Божии ополчились против меня.
\par 5 Ревет ли дикий осел на траве? мычит ли бык у месива своего?
\par 6 Едят ли безвкусное без соли, и есть ли вкус в яичном белке?
\par 7 До чего не хотела коснуться душа моя, то составляет отвратительную пищу мою.
\par 8 О, когда бы сбылось желание мое и чаяние мое исполнил Бог!
\par 9 О, если бы благоволил Бог сокрушить меня, простер руку Свою и сразил меня!
\par 10 Это было бы еще отрадою мне, и я крепился бы в моей беспощадной болезни, ибо я не отвергся изречений Святаго.
\par 11 Что за сила у меня, чтобы надеяться мне? и какой конец, чтобы длить мне жизнь мою?
\par 12 Твердость ли камней твердость моя? и медь ли плоть моя?
\par 13 Есть ли во мне помощь для меня, и есть ли для меня какая опора?
\par 14 К страждущему должно быть сожаление от друга его, если только он не оставил страха к Вседержителю.
\par 15 Но братья мои неверны, как поток, как быстро текущие ручьи,
\par 16 которые черны от льда и в которых скрывается снег.
\par 17 Когда становится тепло, они умаляются, а во время жары исчезают с мест своих.
\par 18 Уклоняют они направление путей своих, заходят в пустыню и теряются;
\par 19 смотрят на них дороги Фемайские, надеются на них пути Савейские,
\par 20 но остаются пристыженными в своей надежде; приходят туда и от стыда краснеют.
\par 21 Так и вы теперь ничто: увидели страшное и испугались.
\par 22 Говорил ли я: дайте мне, или от достатка вашего заплатите за меня;
\par 23 и избавьте меня от руки врага, и от руки мучителей выкупите меня?
\par 24 Научите меня, и я замолчу; укажите, в чем я погрешил.
\par 25 Как сильны слова правды! Но что доказывают обличения ваши?
\par 26 Вы придумываете речи для обличения? На ветер пускаете слова ваши.
\par 27 Вы нападаете на сироту и роете яму другу вашему.
\par 28 Но прошу вас, взгляните на меня; буду ли я говорить ложь пред лицем вашим?
\par 29 Пересмотрите, есть ли неправда? пересмотрите, --правда моя.
\par 30 Есть ли на языке моем неправда? Неужели гортань моя не может различить горечи?

\chapter{7}

\par 1 Не определено ли человеку время на земле, и дни его не то же ли, что дни наемника?
\par 2 Как раб жаждет тени, и как наемник ждет окончания работы своей,
\par 3 так я получил в удел месяцы суетные, и ночи горестные отчислены мне.
\par 4 Когда ложусь, то говорю: `когда-то встану?', а вечер длится, и я ворочаюсь досыта до самого рассвета.
\par 5 Тело мое одето червями и пыльными струпами; кожа моя лопается и гноится.
\par 6 Дни мои бегут скорее челнока и кончаются без надежды.
\par 7 Вспомни, что жизнь моя дуновение, что око мое не возвратится видеть доброе.
\par 8 Не увидит меня око видевшего меня; очи Твои на меня, --и нет меня.
\par 9 Редеет облако и уходит; так нисшедший в преисподнюю не выйдет,
\par 10 не возвратится более в дом свой, и место его не будет уже знать его.
\par 11 Не буду же я удерживать уст моих; буду говорить в стеснении духа моего; буду жаловаться в горести души моей.
\par 12 Разве я море или морское чудовище, что Ты поставил надо мною стражу?
\par 13 Когда подумаю: утешит меня постель моя, унесет горесть мою ложе мое,
\par 14 ты страшишь меня снами и видениями пугаешь меня;
\par 15 и душа моя желает лучше прекращения дыхания, лучше смерти, нежели [сбережения] костей моих.
\par 16 Опротивела мне жизнь. Не вечно жить мне. Отступи от меня, ибо дни мои суета.
\par 17 Что такое человек, что Ты столько ценишь его и обращаешь на него внимание Твое,
\par 18 посещаешь его каждое утро, каждое мгновение испытываешь его?
\par 19 Доколе же Ты не оставишь, доколе не отойдешь от меня, доколе не дашь мне проглотить слюну мою?
\par 20 Если я согрешил, то что я сделаю Тебе, страж человеков! Зачем Ты поставил меня противником Себе, так что я стал самому себе в тягость?
\par 21 И зачем бы не простить мне греха и не снять с меня беззакония моего? ибо, вот, я лягу в прахе; завтра поищешь меня, и меня нет.

\chapter{8}

\par 1 И отвечал Вилдад Савхеянин и сказал:
\par 2 долго ли ты будешь говорить так? --слова уст твоих бурный ветер!
\par 3 Неужели Бог извращает суд, и Вседержитель превращает правду?
\par 4 Если сыновья твои согрешили пред Ним, то Он и предал их в руку беззакония их.
\par 5 Если же ты взыщешь Бога и помолишься Вседержителю,
\par 6 и если ты чист и прав, то Он ныне же встанет над тобою и умиротворит жилище правды твоей.
\par 7 И если вначале у тебя было мало, то впоследствии будет весьма много.
\par 8 Ибо спроси у прежних родов и вникни в наблюдения отцов их;
\par 9 а мы--вчерашние и ничего не знаем, потому что наши дни на земле тень.
\par 10 Вот они научат тебя, скажут тебе и от сердца своего произнесут слова:
\par 11 поднимается ли тростник без влаги? растет ли камыш без воды?
\par 12 Еще он в свежести своей и не срезан, а прежде всякой травы засыхает.
\par 13 Таковы пути всех забывающих Бога, и надежда лицемера погибнет;
\par 14 упование его подсечено, и уверенность его--дом паука.
\par 15 Обопрется о дом свой и не устоит; ухватится за него и не удержится.
\par 16 Зеленеет он пред солнцем, за сад простираются ветви его;
\par 17 в кучу [камней] вплетаются корни его, между камнями врезываются.
\par 18 Но когда вырвут его с места его, оно откажется от него: `я не видало тебя!'
\par 19 Вот радость пути его! а из земли вырастают другие.
\par 20 Видишь, Бог не отвергает непорочного и не поддерживает руки злодеев.
\par 21 Он еще наполнит смехом уста твои и губы твои радостным восклицанием.
\par 22 Ненавидящие тебя облекутся в стыд, и шатра нечестивых не станет.

\chapter{9}

\par 1 И отвечал Иов и сказал:
\par 2 правда! знаю, что так; но как оправдается человек пред Богом?
\par 3 Если захочет вступить в прение с Ним, то не ответит Ему ни на одно из тысячи.
\par 4 Премудр сердцем и могущ силою; кто восставал против Него и оставался в покое?
\par 5 Он передвигает горы, и не узнают их: Он превращает их в гневе Своем;
\par 6 сдвигает землю с места ее, и столбы ее дрожат;
\par 7 скажет солнцу, --и не взойдет, и на звезды налагает печать.
\par 8 Он один распростирает небеса и ходит по высотам моря;
\par 9 сотворил Ас, Кесиль и Хима и тайники юга;
\par 10 делает великое, неисследимое и чудное без числа!
\par 11 Вот, Он пройдет предо мною, и не увижу Его; пронесется и не замечу Его.
\par 12 Возьмет, и кто возбранит Ему? кто скажет Ему: что Ты делаешь?
\par 13 Бог не отвратит гнева Своего; пред Ним падут поборники гордыни.
\par 14 Тем более могу ли я отвечать Ему и приискивать себе слова пред Ним?
\par 15 Хотя бы я и прав был, но не буду отвечать, а буду умолять Судию моего.
\par 16 Если бы я воззвал, и Он ответил мне, --я не поверил бы, что голос мой услышал Тот,
\par 17 Кто в вихре разит меня и умножает безвинно мои раны,
\par 18 не дает мне перевести духа, но пресыщает меня горестями.
\par 19 Если [действовать] силою, то Он могуществен; если судом, кто сведет меня с Ним?
\par 20 Если я буду оправдываться, то мои же уста обвинят меня; [если] я невинен, то Он признает меня виновным.
\par 21 Невинен я; не хочу знать души моей, презираю жизнь мою.
\par 22 Все одно; поэтому я сказал, что Он губит и непорочного и виновного.
\par 23 Если этого поражает Он бичом вдруг, то пытке невинных посмевается.
\par 24 Земля отдана в руки нечестивых; лица судей ее Он закрывает. Если не Он, то кто же?
\par 25 Дни мои быстрее гонца, --бегут, не видят добра,
\par 26 несутся, как легкие ладьи, как орел стремится на добычу.
\par 27 Если сказать мне: забуду я жалобы мои, отложу мрачный вид свой и ободрюсь;
\par 28 то трепещу всех страданий моих, зная, что Ты не объявишь меня невинным.
\par 29 Если же я виновен, то для чего напрасно томлюсь?
\par 30 Хотя бы я омылся и снежною водою и совершенно очистил руки мои,
\par 31 то и тогда Ты погрузишь меня в грязь, и возгнушаются мною одежды мои.
\par 32 Ибо Он не человек, как я, чтоб я мог отвечать Ему и идти вместе с Ним на суд!
\par 33 Нет между нами посредника, который положил бы руку свою на обоих нас.
\par 34 Да отстранит Он от меня жезл Свой, и страх Его да не ужасает меня, --
\par 35 и тогда я буду говорить и не убоюсь Его, ибо я не таков сам в себе.

\chapter{10}

\par 1 Опротивела душе моей жизнь моя; предамся печали моей; буду говорить в горести души моей.
\par 2 Скажу Богу: не обвиняй меня; объяви мне, за что Ты со мною борешься?
\par 3 Хорошо ли для Тебя, что Ты угнетаешь, что презираешь дело рук Твоих, а на совет нечестивых посылаешь свет?
\par 4 Разве у Тебя плотские очи, и Ты смотришь, как смотрит человек?
\par 5 Разве дни Твои, как дни человека, или лета Твои, как дни мужа,
\par 6 что Ты ищешь порока во мне и допытываешься греха во мне,
\par 7 хотя знаешь, что я не беззаконник, и что некому избавить меня от руки Твоей?
\par 8 Твои руки трудились надо мною и образовали всего меня кругом, --и Ты губишь меня?
\par 9 Вспомни, что Ты, как глину, обделал меня, и в прах обращаешь меня?
\par 10 Не Ты ли вылил меня, как молоко, и, как творог, сгустил меня,
\par 11 кожею и плотью одел меня, костями и жилами скрепил меня,
\par 12 жизнь и милость даровал мне, и попечение Твое хранило дух мой?
\par 13 Но и то скрывал Ты в сердце Своем, --знаю, что это было у Тебя, --
\par 14 что если я согрешу, Ты заметишь и не оставишь греха моего без наказания.
\par 15 Если я виновен, горе мне! если и прав, то не осмелюсь поднять головы моей. Я пресыщен унижением; взгляни на бедствие мое:
\par 16 оно увеличивается. Ты гонишься за мною, как лев, и снова нападаешь на меня и чудным являешься во мне.
\par 17 Выводишь новых свидетелей Твоих против меня; усиливаешь гнев Твой на меня; и беды, одни за другими, ополчаются против меня.
\par 18 И зачем Ты вывел меня из чрева? пусть бы я умер, когда еще ничей глаз не видел меня;
\par 19 пусть бы я, как небывший, из чрева перенесен был во гроб!
\par 20 Не малы ли дни мои? Оставь, отступи от меня, чтобы я немного ободрился,
\par 21 прежде нежели отойду, --и уже не возвращусь, --в страну тьмы и сени смертной,
\par 22 в страну мрака, каков есть мрак тени смертной, где нет устройства, [где] темно, как самая тьма.

\chapter{11}

\par 1 И отвечал Софар Наамитянин и сказал:
\par 2 разве на множество слов нельзя дать ответа, и разве человек многоречивый прав?
\par 3 Пустословие твое заставит ли молчать мужей, чтобы ты глумился, и некому было постыдить тебя?
\par 4 Ты сказал: суждение мое верно, и чист я в очах Твоих.
\par 5 Но если бы Бог возглаголал и отверз уста Свои к тебе
\par 6 и открыл тебе тайны премудрости, что тебе вдвое больше следовало бы понести! Итак знай, что Бог для тебя некоторые из беззаконий твоих предал забвению.
\par 7 Можешь ли ты исследованием найти Бога? Можешь ли совершенно постигнуть Вседержителя?
\par 8 Он превыше небес, --что можешь сделать? глубже преисподней, --что можешь узнать?
\par 9 Длиннее земли мера Его и шире моря.
\par 10 Если Он пройдет и заключит кого в оковы и представит на суд, то кто отклонит Его?
\par 11 Ибо Он знает людей лживых и видит беззаконие, и оставит ли его без внимания?
\par 12 Но пустой человек мудрствует, хотя человек рождается подобно дикому осленку.
\par 13 Если ты управишь сердце твое и прострешь к Нему руки твои,
\par 14 и если есть порок в руке твоей, а ты удалишь его и не дашь беззаконию обитать в шатрах твоих,
\par 15 то поднимешь незапятнанное лице твое и будешь тверд и не будешь бояться.
\par 16 Тогда забудешь горе: как о воде протекшей, будешь вспоминать о нем.
\par 17 И яснее полдня пойдет жизнь твоя; просветлеешь, как утро.
\par 18 И будешь спокоен, ибо есть надежда; ты огражден, и можешь спать безопасно.
\par 19 Будешь лежать, и не будет устрашающего, и многие будут заискивать у тебя.
\par 20 глаза беззаконных истают, и убежище пропадет у них, и надежда их исчезнет.

\chapter{12}

\par 1 И отвечал Иов и сказал:
\par 2 подлинно, [только] вы люди, и с вами умрет мудрость!
\par 3 И у меня [есть] сердце, как у вас; не ниже я вас; и кто не знает того же?
\par 4 Посмешищем стал я для друга своего, я, который взывал к Богу, и которому Он отвечал, посмешищем--[человек] праведный, непорочный.
\par 5 Так презрен по мыслям сидящего в покое факел, приготовленный для спотыкающихся ногами.
\par 6 Покойны шатры у грабителей и безопасны у раздражающих Бога, которые как бы Бога носят в руках своих.
\par 7 И подлинно: спроси у скота, и научит тебя, у птицы небесной, и возвестит тебе;
\par 8 или побеседуй с землею, и наставит тебя, и скажут тебе рыбы морские.
\par 9 Кто во всем этом не узнает, что рука Господа сотворила сие?
\par 10 В Его руке душа всего живущего и дух всякой человеческой плоти.
\par 11 Не ухо ли разбирает слова, и не язык ли распознает вкус пищи?
\par 12 В старцах--мудрость, и в долголетних--разум.
\par 13 У Него премудрость и сила; Его совет и разум.
\par 14 Что Он разрушит, то не построится; кого Он заключит, тот не высвободится.
\par 15 Остановит воды, и все высохнет; пустит их, и превратят землю.
\par 16 У Него могущество и премудрость, пред Ним заблуждающийся и вводящий в заблуждение.
\par 17 Он приводит советников в необдуманность и судей делает глупыми.
\par 18 Он лишает перевязей царей и поясом обвязывает чресла их;
\par 19 князей лишает достоинства и низвергает храбрых;
\par 20 отнимает язык у велеречивых и старцев лишает смысла;
\par 21 покрывает стыдом знаменитых и силу могучих ослабляет;
\par 22 открывает глубокое из среды тьмы и выводит на свет тень смертную;
\par 23 умножает народы и истребляет их; рассевает народы и собирает их;
\par 24 отнимает ум у глав народа земли и оставляет их блуждать в пустыне, где нет пути:
\par 25 ощупью ходят они во тьме без света и шатаются, как пьяные.

\chapter{13}

\par 1 Вот, все [это] видело око мое, слышало ухо мое и заметило для себя.
\par 2 Сколько знаете вы, знаю и я: не ниже я вас.
\par 3 Но я к Вседержителю хотел бы говорить и желал бы состязаться с Богом.
\par 4 А вы сплетчики лжи; все вы бесполезные врачи.
\par 5 О, если бы вы только молчали! это было бы [вменено] вам в мудрость.
\par 6 Выслушайте же рассуждения мои и вникните в возражение уст моих.
\par 7 Надлежало ли вам ради Бога говорить неправду и для Него говорить ложь?
\par 8 Надлежало ли вам быть лицеприятными к Нему и за Бога так препираться?
\par 9 Хорошо ли будет, когда Он испытает вас? Обманете ли Его, как обманывают человека?
\par 10 Строго накажет Он вас, хотя вы и скрытно лицемерите.
\par 11 Неужели величие Его не устрашает вас, и страх Его не нападает на вас?
\par 12 Напоминания ваши подобны пеплу; оплоты ваши--оплоты глиняные.
\par 13 Замолчите предо мною, и я буду говорить, что бы ни постигло меня.
\par 14 Для чего мне терзать тело мое зубами моими и душу мою полагать в руку мою?
\par 15 Вот, Он убивает меня, но я буду надеяться; я желал бы только отстоять пути мои пред лицем Его!
\par 16 И это уже в оправдание мне, потому что лицемер не пойдет пред лице Его!
\par 17 Выслушайте внимательно слово мое и объяснение мое ушами вашими.
\par 18 Вот, я завел судебное дело: знаю, что буду прав.
\par 19 Кто в состоянии оспорить меня? Ибо я скоро умолкну и испущу дух.
\par 20 Двух только [вещей] не делай со мною, и тогда я не буду укрываться от лица Твоего:
\par 21 удали от меня руку Твою, и ужас Твой да не потрясает меня.
\par 22 Тогда зови, и я буду отвечать, или буду говорить я, а Ты отвечай мне.
\par 23 Сколько у меня пороков и грехов? покажи мне беззаконие мое и грех мой.
\par 24 Для чего скрываешь лице Твое и считаешь меня врагом Тебе?
\par 25 Не сорванный ли листок Ты сокрушаешь и не сухую ли соломинку преследуешь?
\par 26 Ибо Ты пишешь на меня горькое и вменяешь мне грехи юности моей,
\par 27 и ставишь в колоду ноги мои и подстерегаешь все стези мои, --гонишься по следам ног моих.
\par 28 А он, как гниль, распадается, как одежда, изъеденная молью.

\chapter{14}

\par 1 Человек, рожденный женою, краткодневен и пресыщен печалями:
\par 2 как цветок, он выходит и опадает; убегает, как тень, и не останавливается.
\par 3 И на него-то Ты отверзаешь очи Твои, и меня ведешь на суд с Тобою?
\par 4 Кто родится чистым от нечистого? Ни один.
\par 5 Если дни ему определены, и число месяцев его у Тебя, если Ты положил ему предел, которого он не перейдет,
\par 6 то уклонись от него: пусть он отдохнет, доколе не окончит, как наемник, дня своего.
\par 7 Для дерева есть надежда, что оно, если и будет срублено, снова оживет, и отрасли от него [выходить] не перестанут:
\par 8 если и устарел в земле корень его, и пень его замер в пыли,
\par 9 но, лишь почуяло воду, оно дает отпрыски и пускает ветви, как бы вновь посаженное.
\par 10 А человек умирает и распадается; отошел, и где он?
\par 11 Уходят воды из озера, и река иссякает и высыхает:
\par 12 так человек ляжет и не станет; до скончания неба он не пробудится и не воспрянет от сна своего.
\par 13 О, если бы Ты в преисподней сокрыл меня и укрывал меня, пока пройдет гнев Твой, положил мне срок и потом вспомнил обо мне!
\par 14 Когда умрет человек, то будет ли он опять жить? Во все дни определенного мне времени я ожидал бы, пока придет мне смена.
\par 15 Воззвал бы Ты, и я дал бы Тебе ответ, и Ты явил бы благоволение творению рук Твоих;
\par 16 ибо тогда Ты исчислял бы шаги мои и не подстерегал бы греха моего;
\par 17 в свитке было бы запечатано беззаконие мое, и Ты закрыл бы вину мою.
\par 18 Но гора падая разрушается, и скала сходит с места своего;
\par 19 вода стирает камни; разлив ее смывает земную пыль: так и надежду человека Ты уничтожаешь.
\par 20 Теснишь его до конца, и он уходит; изменяешь ему лице и отсылаешь его.
\par 21 В чести ли дети его--он не знает, унижены ли--он не замечает;
\par 22 но плоть его на нем болит, и душа его в нем страдает.

\chapter{15}

\par 1 И отвечал Елифаз Феманитянин и сказал:
\par 2 станет ли мудрый отвечать знанием пустым и наполнять чрево свое ветром палящим,
\par 3 оправдываться словами бесполезными и речью, не имеющею никакой силы?
\par 4 Да ты отложил и страх и за малость считаешь речь к Богу.
\par 5 Нечестие твое настроило так уста твои, и ты избрал язык лукавых.
\par 6 Тебя обвиняют уста твои, а не я, и твой язык говорит против тебя.
\par 7 Разве ты первым человеком родился и прежде холмов создан?
\par 8 Разве совет Божий ты слышал и привлек к себе премудрость?
\par 9 Что знаешь ты, чего бы не знали мы? что разумеешь ты, чего не было бы и у нас?
\par 10 И седовласый и старец есть между нами, днями превышающий отца твоего.
\par 11 Разве малость для тебя утешения Божии? И это неизвестно тебе?
\par 12 К чему порывает тебя сердце твое, и к чему так гордо смотришь?
\par 13 Что устремляешь против Бога дух твой и устами твоими произносишь такие речи?
\par 14 Что такое человек, чтоб быть ему чистым, и чтобы рожденному женщиною быть праведным?
\par 15 Вот, Он и святым Своим не доверяет, и небеса нечисты в очах Его:
\par 16 тем больше нечист и растлен человек, пьющий беззаконие, как воду.
\par 17 Я буду говорить тебе, слушай меня; я расскажу тебе, что видел,
\par 18 что слышали мудрые и не скрыли слышанного от отцов своих,
\par 19 которым одним отдана была земля, и среди которых чужой не ходил.
\par 20 Нечестивый мучит себя во все дни свои, и число лет закрыто от притеснителя;
\par 21 звук ужасов в ушах его; среди мира идет на него губитель.
\par 22 Он не надеется спастись от тьмы; видит пред собою меч.
\par 23 Он скитается за куском хлеба повсюду; знает, что уже готов, в руках у него день тьмы.
\par 24 Устрашает его нужда и теснота; одолевает его, как царь, приготовившийся к битве,
\par 25 за то, что он простирал против Бога руку свою и противился Вседержителю,
\par 26 устремлялся против Него с [гордою] выею, под толстыми щитами своими;
\par 27 потому что он покрыл лице свое жиром своим и обложил туком лядвеи свои.
\par 28 И он селится в городах разоренных, в домах, в которых не живут, которые обречены на развалины.
\par 29 Не пребудет он богатым, и не уцелеет имущество его, и не распрострется по земле приобретение его.
\par 30 Не уйдет от тьмы; отрасли его иссушит пламя и дуновением уст своих увлечет его.
\par 31 Пусть не доверяет суете заблудший, ибо суета будет и воздаянием ему.
\par 32 Не в свой день он скончается, и ветви его не будут зеленеть.
\par 33 Сбросит он, как виноградная лоза, недозрелую ягоду свою и, как маслина, стряхнет цвет свой.
\par 34 Так опустеет дом нечестивого, и огонь пожрет шатры мздоимства.
\par 35 Он зачал зло и родил ложь, и утроба его приготовляет обман.

\chapter{16}

\par 1 И отвечал Иов и сказал:
\par 2 слышал я много такого; жалкие утешители все вы!
\par 3 Будет ли конец ветреным словам? и что побудило тебя так отвечать?
\par 4 И я мог бы так же говорить, как вы, если бы душа ваша была на месте души моей; ополчался бы на вас словами и кивал бы на вас головою моею;
\par 5 подкреплял бы вас языком моим и движением губ утешал бы.
\par 6 Говорю ли я, не утоляется скорбь моя; перестаю ли, что отходит от меня?
\par 7 Но ныне Он изнурил меня. Ты разрушил всю семью мою.
\par 8 Ты покрыл меня морщинами во свидетельство против меня; восстает на меня изможденность моя, в лицо укоряет меня.
\par 9 Гнев Его терзает и враждует против меня, скрежещет на меня зубами своими; неприятель мой острит на меня глаза свои.
\par 10 Разинули на меня пасть свою; ругаясь бьют меня по щекам; все сговорились против меня.
\par 11 Предал меня Бог беззаконнику и в руки нечестивым бросил меня.
\par 12 Я был спокоен, но Он потряс меня; взял меня за шею и избил меня и поставил меня целью для Себя.
\par 13 Окружили меня стрельцы Его; Он рассекает внутренности мои и не щадит, пролил на землю желчь мою,
\par 14 пробивает во мне пролом за проломом, бежит на меня, как ратоборец.
\par 15 Вретище сшил я на кожу мою и в прах положил голову мою.
\par 16 Лицо мое побагровело от плача, и на веждах моих тень смерти,
\par 17 при всем том, что нет хищения в руках моих, и молитва моя чиста.
\par 18 Земля! не закрой моей крови, и да не будет места воплю моему.
\par 19 И ныне вот на небесах Свидетель мой, и Заступник мой в вышних!
\par 20 Многоречивые друзья мои! К Богу слезит око мое.
\par 21 О, если бы человек мог иметь состязание с Богом, как сын человеческий с ближним своим!
\par 22 Ибо летам моим приходит конец, и я отхожу в путь невозвратный.

\chapter{17}

\par 1 Дыхание мое ослабело; дни мои угасают; гробы предо мною.
\par 2 Если бы не насмешки их, то и среди споров их око мое пребывало бы спокойно.
\par 3 Заступись, поручись [Сам] за меня пред Собою! иначе кто поручится за меня?
\par 4 Ибо Ты закрыл сердце их от разумения, и потому не дашь восторжествовать [им].
\par 5 Кто обрекает друзей своих в добычу, у детей того глаза истают.
\par 6 Он поставил меня притчею для народа и посмешищем для него.
\par 7 Помутилось от горести око мое, и все члены мои, как тень.
\par 8 Изумятся о сем праведные, и невинный вознегодует на лицемера.
\par 9 Но праведник будет крепко держаться пути своего, и чистый руками будет больше и больше утверждаться.
\par 10 Выслушайте, все вы, и подойдите; не найду я мудрого между вами.
\par 11 Дни мои прошли; думы мои--достояние сердца моего--разбиты.
\par 12 А они ночь [хотят] превратить в день, свет приблизить к лицу тьмы.
\par 13 Если бы я и ожидать стал, то преисподняя--дом мой; во тьме постелю я постель мою;
\par 14 гробу скажу: ты отец мой, червю: ты мать моя и сестра моя.
\par 15 Где же после этого надежда моя? и ожидаемое мною кто увидит?
\par 16 В преисподнюю сойдет она и будет покоиться со мною в прахе.

\chapter{18}

\par 1 И отвечал Вилдад Савхеянин и сказал:
\par 2 когда же положите вы конец таким речам? обдумайте, и потом будем говорить.
\par 3 Зачем считаться нам за животных и быть униженными в собственных глазах ваших?
\par 4 [О ты], раздирающий душу твою в гневе твоем! Неужели для тебя опустеть земле, и скале сдвинуться с места своего?
\par 5 Да, свет у беззаконного потухнет, и не останется искры от огня его.
\par 6 Померкнет свет в шатре его, и светильник его угаснет над ним.
\par 7 Сократятся шаги могущества его, и низложит его собственный замысл его,
\par 8 ибо он попадет в сеть своими ногами и по тенетам ходить будет.
\par 9 Петля зацепит за ногу его, и грабитель уловит его.
\par 10 Скрытно разложены по земле силки для него и западни на дороге.
\par 11 Со всех сторон будут страшить его ужасы и заставят его бросаться туда и сюда.
\par 12 Истощится от голода сила его, и гибель готова, сбоку у него.
\par 13 Съест члены тела его, съест члены его первенец смерти.
\par 14 Изгнана будет из шатра его надежда его, и это низведет его к царю ужасов.
\par 15 Поселятся в шатре его, потому что он уже не его; жилище его посыпано будет серою.
\par 16 Снизу подсохнут корни его, и сверху увянут ветви его.
\par 17 Память о нем исчезнет с земли, и имени его не будет на площади.
\par 18 Изгонят его из света во тьму и сотрут его с лица земли.
\par 19 Ни сына его, ни внука не будет в народе его, и никого не останется в жилищах его.
\par 20 О дне его ужаснутся потомки, и современники будут объяты трепетом.
\par 21 Таковы жилища беззаконного, и таково место того, кто не знает Бога.

\chapter{19}

\par 1 И отвечал Иов и сказал:
\par 2 доколе будете мучить душу мою и терзать меня речами?
\par 3 Вот, уже раз десять вы срамили меня и не стыдитесь теснить меня.
\par 4 Если я и действительно погрешил, то погрешность моя при мне остается.
\par 5 Если же вы хотите повеличаться надо мною и упрекнуть меня позором моим,
\par 6 то знайте, что Бог ниспроверг меня и обложил меня Своею сетью.
\par 7 Вот, я кричу: обида! и никто не слушает; вопию, и нет суда.
\par 8 Он преградил мне дорогу, и не могу пройти, и на стези мои положил тьму.
\par 9 Совлек с меня славу мою и снял венец с головы моей.
\par 10 Кругом разорил меня, и я отхожу; и, как дерево, Он исторг надежду мою.
\par 11 Воспылал на меня гневом Своим и считает меня между врагами Своими.
\par 12 Полки Его пришли вместе и направили путь свой ко мне и расположились вокруг шатра моего.
\par 13 Братьев моих Он удалил от меня, и знающие меня чуждаются меня.
\par 14 Покинули меня близкие мои, и знакомые мои забыли меня.
\par 15 Пришлые в доме моем и служанки мои чужим считают меня; посторонним стал я в глазах их.
\par 16 Зову слугу моего, и он не откликается; устами моими я должен умолять его.
\par 17 Дыхание мое опротивело жене моей, и я должен умолять ее ради детей чрева моего.
\par 18 Даже малые дети презирают меня: поднимаюсь, и они издеваются надо мною.
\par 19 Гнушаются мною все наперсники мои, и те, которых я любил, обратились против меня.
\par 20 Кости мои прилипли к коже моей и плоти моей, и я остался только с кожею около зубов моих.
\par 21 Помилуйте меня, помилуйте меня вы, друзья мои, ибо рука Божия коснулась меня.
\par 22 Зачем и вы преследуете меня, как Бог, и плотью моею не можете насытиться?
\par 23 О, если бы записаны были слова мои! Если бы начертаны были они в книге
\par 24 резцом железным с оловом, --на вечное время на камне вырезаны были!
\par 25 А я знаю, Искупитель мой жив, и Он в последний день восставит из праха распадающуюся кожу мою сию,
\par 26 и я во плоти моей узрю Бога.
\par 27 Я узрю Его сам; мои глаза, не глаза другого, увидят Его. Истаевает сердце мое в груди моей!
\par 28 Вам надлежало бы сказать: зачем мы преследуем его? Как будто корень зла найден во мне.
\par 29 Убойтесь меча, ибо меч есть отмститель неправды, и знайте, что есть суд.

\chapter{20}

\par 1 И отвечал Софар Наамитянин и сказал:
\par 2 размышления мои побуждают меня отвечать, и я поспешаю выразить их.
\par 3 Упрек, позорный для меня, выслушал я, и дух разумения моего ответит за меня.
\par 4 Разве не знаешь ты, что от века, --с того времени, как поставлен человек на земле, --
\par 5 веселье беззаконных кратковременно, и радость лицемера мгновенна?
\par 6 Хотя бы возросло до небес величие его, и голова его касалась облаков, --
\par 7 как помет его, на веки пропадает он; видевшие его скажут: где он?
\par 8 Как сон, улетит, и не найдут его; и, как ночное видение, исчезнет.
\par 9 Глаз, видевший его, больше не увидит его, и уже не усмотрит его место его.
\par 10 Сыновья его будут заискивать у нищих, и руки его возвратят похищенное им.
\par 11 Кости его наполнены грехами юности его, и с ним лягут они в прах.
\par 12 Если сладко во рту его зло, и он таит его под языком своим,
\par 13 бережет и не бросает его, а держит его в устах своих,
\par 14 то эта пища его в утробе его превратится в желчь аспидов внутри его.
\par 15 Имение, которое он глотал, изблюет: Бог исторгнет его из чрева его.
\par 16 Змеиный яд он сосет; умертвит его язык ехидны.
\par 17 Не видать ему ручьев, рек, текущих медом и молоком!
\par 18 Нажитое трудом возвратит, не проглотит; по мере имения его будет и расплата его, а он не порадуется.
\par 19 Ибо он угнетал, отсылал бедных; захватывал домы, которых не строил;
\par 20 не знал сытости во чреве своем и в жадности своей не щадил ничего.
\par 21 Ничего не спаслось от обжорства его, зато не устоит счастье его.
\par 22 В полноте изобилия будет тесно ему; всякая рука обиженного поднимется на него.
\par 23 Когда будет чем наполнить утробу его, Он пошлет на него ярость гнева Своего и одождит на него болезни в плоти его.
\par 24 Убежит ли он от оружия железного, --пронзит его лук медный;
\par 25 станет вынимать [стрелу], --и она выйдет из тела, выйдет, сверкая сквозь желчь его; ужасы смерти найдут на него!
\par 26 Все мрачное сокрыто внутри его; будет пожирать его огонь, никем не раздуваемый; зло постигнет и оставшееся в шатре его.
\par 27 Небо откроет беззаконие его, и земля восстанет против него.
\par 28 Исчезнет стяжание дома его; все расплывется в день гнева Его.
\par 29 Вот удел человеку беззаконному от Бога и наследие, определенное ему Вседержителем!

\chapter{21}

\par 1 И отвечал Иов и сказал:
\par 2 выслушайте внимательно речь мою, и это будет мне утешением от вас.
\par 3 Потерпите меня, и я буду говорить; а после того, как поговорю, насмехайся.
\par 4 Разве к человеку речь моя? как же мне и не малодушествовать?
\par 5 Посмотрите на меня и ужаснитесь, и положите перст на уста.
\par 6 Лишь только я вспомню, --содрогаюсь, и трепет объемлет тело мое.
\par 7 Почему беззаконные живут, достигают старости, да и силами крепки?
\par 8 Дети их с ними перед лицем их, и внуки их перед глазами их.
\par 9 Домы их безопасны от страха, и нет жезла Божия на них.
\par 10 Вол их оплодотворяет и не извергает, корова их зачинает и не выкидывает.
\par 11 Как стадо, выпускают они малюток своих, и дети их прыгают.
\par 12 Восклицают под [голос] тимпана и цитры и веселятся при [звуках] свирели;
\par 13 проводят дни свои в счастьи и мгновенно нисходят в преисподнюю.
\par 14 А между тем они говорят Богу: отойди от нас, не хотим мы знать путей Твоих!
\par 15 Что Вседержитель, чтобы нам служить Ему? и что пользы прибегать к Нему?
\par 16 Видишь, счастье их не от их рук. --Совет нечестивых будь далек от меня!
\par 17 Часто ли угасает светильник у беззаконных, и находит на них беда, и Он дает им в удел страдания во гневе Своем?
\par 18 Они должны быть, как соломинка пред ветром и как плева, уносимая вихрем.
\par 19 [Скажешь]: Бог бережет для детей его несчастье его. --Пусть воздаст Он ему самому, чтобы он это знал.
\par 20 Пусть его глаза увидят несчастье его, и пусть он сам пьет от гнева Вседержителева.
\par 21 Ибо какая ему забота до дома своего после него, когда число месяцев его кончится?
\par 22 Но Бога ли учить мудрости, когда Он судит и горних?
\par 23 Один умирает в самой полноте сил своих, совершенно спокойный и мирный;
\par 24 внутренности его полны жира, и кости его напоены мозгом.
\par 25 А другой умирает с душею огорченною, не вкусив добра.
\par 26 И они вместе будут лежать во прахе, и червь покроет их.
\par 27 Знаю я ваши мысли и ухищрения, какие вы против меня сплетаете.
\par 28 Вы скажете: где дом князя, и где шатер, в котором жили беззаконные?
\par 29 Разве вы не спрашивали у путешественников и незнакомы с их наблюдениями,
\par 30 что в день погибели пощажен бывает злодей, в день гнева отводится в сторону?
\par 31 Кто представит ему пред лице путь его, и кто воздаст ему за то, что он делал?
\par 32 Его провожают ко гробам и на его могиле ставят стражу.
\par 33 Сладки для него глыбы долины, и за ним идет толпа людей, а идущим перед ним нет числа.
\par 34 Как же вы хотите утешать меня пустым? В ваших ответах остается [одна] ложь.

\chapter{22}

\par 1 И отвечал Елифаз Феманитянин и сказал:
\par 2 разве может человек доставлять пользу Богу? Разумный доставляет пользу себе самому.
\par 3 Что за удовольствие Вседержителю, что ты праведен? И будет ли Ему выгода от того, что ты содержишь пути твои в непорочности?
\par 4 Неужели Он, боясь тебя, вступит с тобою в состязание, пойдет судиться с тобою?
\par 5 Верно, злоба твоя велика, и беззакониям твоим нет конца.
\par 6 Верно, ты брал залоги от братьев твоих ни за что и с полунагих снимал одежду.
\par 7 Утомленному жаждою не подавал воды напиться и голодному отказывал в хлебе;
\par 8 а человеку сильному ты [давал] землю, и сановитый селился на ней.
\par 9 Вдов ты отсылал ни с чем и сирот оставлял с пустыми руками.
\par 10 За то вокруг тебя петли, и возмутил тебя неожиданный ужас,
\par 11 или тьма, в которой ты ничего не видишь, и множество вод покрыло тебя.
\par 12 Не превыше ли небес Бог? посмотри вверх на звезды, как они высоко!
\par 13 И ты говоришь: что знает Бог? может ли Он судить сквозь мрак?
\par 14 Облака--завеса Его, так что Он не видит, а ходит [только] по небесному кругу.
\par 15 Неужели ты держишься пути древних, по которому шли люди беззаконные,
\par 16 которые преждевременно были истреблены, когда вода разлилась под основание их?
\par 17 Они говорили Богу: отойди от нас! и что сделает им Вседержитель?
\par 18 А Он наполнял домы их добром. Но совет нечестивых будь далек от меня!
\par 19 Видели праведники и радовались, и непорочный смеялся им:
\par 20 враг наш истреблен, а оставшееся после них пожрал огонь.
\par 21 Сблизься же с Ним--и будешь спокоен; чрез это придет к тебе добро.
\par 22 Прими из уст Его закон и положи слова Его в сердце твое.
\par 23 Если ты обратишься к Вседержителю, то вновь устроишься, удалишь беззаконие от шатра твоего
\par 24 и будешь вменять в прах блестящий металл, и в камни потоков--[золото] Офирское.
\par 25 И будет Вседержитель твоим золотом и блестящим серебром у тебя,
\par 26 ибо тогда будешь радоваться о Вседержителе и поднимешь к Богу лице твое.
\par 27 Помолишься Ему, и Он услышит тебя, и ты исполнишь обеты твои.
\par 28 Положишь намерение, и оно состоится у тебя, и над путями твоими будет сиять свет.
\par 29 Когда кто уничижен будет, ты скажешь: возвышение! и Он спасет поникшего лицем,
\par 30 избавит и небезвинного, и он спасется чистотою рук твоих.

\chapter{23}

\par 1 И отвечал Иов и сказал:
\par 2 еще и ныне горька речь моя: страдания мои тяжелее стонов моих.
\par 3 О, если бы я знал, где найти Его, и мог подойти к престолу Его!
\par 4 Я изложил бы пред Ним дело мое и уста мои наполнил бы оправданиями;
\par 5 узнал бы слова, какими Он ответит мне, и понял бы, что Он скажет мне.
\par 6 Неужели Он в полном могуществе стал бы состязаться со мною? О, нет! Пусть Он только обратил бы внимание на меня.
\par 7 Тогда праведник мог бы состязаться с Ним, --и я навсегда получил бы свободу от Судии моего.
\par 8 Но вот, я иду вперед--и нет Его, назад--и не нахожу Его;
\par 9 делает ли Он что на левой стороне, я не вижу; скрывается ли на правой, не усматриваю.
\par 10 Но Он знает путь мой; пусть испытает меня, --выйду, как золото.
\par 11 Нога моя твердо держится стези Его; пути Его я хранил и не уклонялся.
\par 12 От заповеди уст Его не отступал; глаголы уст Его хранил больше, нежели мои правила.
\par 13 Но Он тверд; и кто отклонит Его? Он делает, чего хочет душа Его.
\par 14 Так, Он выполнит положенное мне, и подобного этому много у Него.
\par 15 Поэтому я трепещу пред лицем Его; размышляю--и страшусь Его.
\par 16 Бог расслабил сердце мое, и Вседержитель устрашил меня.
\par 17 Зачем я не уничтожен прежде этой тьмы, и Он не сокрыл мрака от лица моего!

\chapter{24}

\par 1 Почему не сокрыты от Вседержителя времена, и знающие Его не видят дней Его?
\par 2 Межи передвигают, угоняют стада и пасут [у себя].
\par 3 У сирот уводят осла, у вдовы берут в залог вола;
\par 4 бедных сталкивают с дороги, все уничиженные земли принуждены скрываться.
\par 5 Вот они, [как] дикие ослы в пустыне, выходят на дело свое, вставая рано на добычу; степь [дает] хлеб для них и для детей их;
\par 6 жнут они на поле не своем и собирают виноград у нечестивца;
\par 7 нагие ночуют без покрова и без одеяния на стуже;
\par 8 мокнут от горных дождей и, не имея убежища, жмутся к скале;
\par 9 отторгают от сосцов сироту и с нищего берут залог;
\par 10 заставляют ходить нагими, без одеяния, и голодных кормят колосьями;
\par 11 между стенами выжимают масло оливковое, топчут в точилах и жаждут.
\par 12 В городе люди стонут, и душа убиваемых вопит, и Бог не воспрещает того.
\par 13 Есть из них враги света, не знают путей его и не ходят по стезям его.
\par 14 С рассветом встает убийца, умерщвляет бедного и нищего, а ночью бывает вором.
\par 15 И око прелюбодея ждет сумерков, говоря: ничей глаз не увидит меня, --и закрывает лице.
\par 16 В темноте подкапываются под домы, которые днем они заметили для себя; не знают света.
\par 17 Ибо для них утро--смертная тень, так как они знакомы с ужасами смертной тени.
\par 18 Легок такой на поверхности воды, проклята часть его на земле, и не смотрит он на дорогу садов виноградных.
\par 19 Засуха и жара поглощают снежную воду: так преисподняя--грешников.
\par 20 Пусть забудет его утроба [матери]; пусть лакомится им червь; пусть не остается о нем память; как дерево, пусть сломится беззаконник,
\par 21 который угнетает бездетную, не рождавшую, и вдове не делает добра.
\par 22 Он и сильных увлекает своею силою; он встает и никто не уверен за жизнь свою.
\par 23 А Он дает ему [все] для безопасности, и он [на то] опирается, и очи Его видят пути их.
\par 24 Поднялись высоко, --и вот, нет их; падают и умирают, как и все, и, как верхушки колосьев, срезываются.
\par 25 Если это не так, --кто обличит меня во лжи и в ничто обратит речь мою?

\chapter{25}

\par 1 И отвечал Вилдад Савхеянин и сказал:
\par 2 держава и страх у Него; Он творит мир на высотах Своих!
\par 3 Есть ли счет воинствам Его? и над кем не восходит свет Его?
\par 4 И как человеку быть правым пред Богом, и как быть чистым рожденному женщиною?
\par 5 Вот даже луна, и та несветла, и звезды нечисты пред очами Его.
\par 6 Тем менее человек, [который] есть червь, и сын человеческий, [который] есть моль.

\chapter{26}

\par 1 И отвечал Иов и сказал:
\par 2 как ты помог бессильному, поддержал мышцу немощного!
\par 3 Какой совет подал ты немудрому и как во всей полноте объяснил дело!
\par 4 Кому ты говорил эти слова, и чей дух исходил из тебя?
\par 5 Рефаимы трепещут под водами, и живущие в них.
\par 6 Преисподняя обнажена пред Ним, и нет покрывала Аваддону.
\par 7 Он распростер север над пустотою, повесил землю ни на чем.
\par 8 Он заключает воды в облаках Своих, и облако не расседается под ними.
\par 9 Он поставил престол Свой, распростер над ним облако Свое.
\par 10 Черту провел над поверхностью воды, до границ света со тьмою.
\par 11 Столпы небес дрожат и ужасаются от грозы Его.
\par 12 Силою Своею волнует море и разумом Своим сражает его дерзость.
\par 13 От духа Его--великолепие неба; рука Его образовала быстрого скорпиона.
\par 14 Вот, это части путей Его; и как мало мы слышали о Нем! А гром могущества Его кто может уразуметь?

\chapter{27}

\par 1 И продолжал Иов возвышенную речь свою и сказал:
\par 2 жив Бог, лишивший [меня] суда, и Вседержитель, огорчивший душу мою,
\par 3 что, доколе еще дыхание мое во мне и дух Божий в ноздрях моих,
\par 4 не скажут уста мои неправды, и язык мой не произнесет лжи!
\par 5 Далек я от того, чтобы признать вас справедливыми; доколе не умру, не уступлю непорочности моей.
\par 6 Крепко держал я правду мою и не опущу ее; не укорит меня сердце мое во все дни мои.
\par 7 Враг мой будет, как нечестивец, и восстающий на меня, как беззаконник.
\par 8 Ибо какая надежда лицемеру, когда возьмет, когда исторгнет Бог душу его?
\par 9 Услышит ли Бог вопль его, когда придет на него беда?
\par 10 Будет ли он утешаться Вседержителем и призывать Бога во всякое время?
\par 11 Возвещу вам, что в руке Божией; что у Вседержителя, не скрою.
\par 12 Вот, все вы и сами видели; и для чего вы столько пустословите?
\par 13 Вот доля человеку беззаконному от Бога, и наследие, какое получают от Вседержителя притеснители.
\par 14 Если умножаются сыновья его, то под меч; и потомки его не насытятся хлебом.
\par 15 Оставшихся по нем смерть низведет во гроб, и вдовы их не будут плакать.
\par 16 Если он наберет кучи серебра, как праха, и наготовит одежд, как брение,
\par 17 то он наготовит, а одеваться будет праведник, и серебро получит себе на долю беспорочный.
\par 18 Он строит, как моль, дом свой и, как сторож, делает себе шалаш;
\par 19 ложится спать богачом и таким не встанет; открывает глаза свои, и он уже не тот.
\par 20 Как воды, постигнут его ужасы; в ночи похитит его буря.
\par 21 Поднимет его восточный ветер и понесет, и он быстро побежит от него.
\par 22 Устремится на него и не пощадит, как бы он ни силился убежать от руки его.
\par 23 Всплеснут о нем руками и посвищут над ним с места его!

\chapter{28}

\par 1 Так! у серебра есть источная жила, и у золота место, [где его] плавят.
\par 2 Железо получается из земли; из камня выплавляется медь.
\par 3 [Человек] полагает предел тьме и тщательно разыскивает камень во мраке и тени смертной.
\par 4 Вырывают рудокопный колодезь в местах, забытых ногою, спускаются вглубь, висят [и] зыблются вдали от людей.
\par 5 Земля, на которой вырастает хлеб, внутри изрыта как бы огнем.
\par 6 Камни ее--место сапфира, и в ней песчинки золота.
\par 7 Стези [туда] не знает хищная птица, и не видал ее глаз коршуна;
\par 8 не попирали ее скимны, и не ходил по ней шакал.
\par 9 На гранит налагает он руку свою, с корнем опрокидывает горы;
\par 10 в скалах просекает каналы, и все драгоценное видит глаз его;
\par 11 останавливает течение потоков и сокровенное выносит на свет.
\par 12 Но где премудрость обретается? и где место разума?
\par 13 Не знает человек цены ее, и она не обретается на земле живых.
\par 14 Бездна говорит: не во мне она; и море говорит: не у меня.
\par 15 Не дается она за золото и не приобретается она за вес серебра;
\par 16 не оценивается она золотом Офирским, ни драгоценным ониксом, ни сапфиром;
\par 17 не равняется с нею золото и кристалл, и не выменяешь ее на сосуды из чистого золота.
\par 18 А о кораллах и жемчуге и упоминать нечего, и приобретение премудрости выше рубинов.
\par 19 Не равняется с нею топаз Ефиопский; чистым золотом не оценивается она.
\par 20 Откуда же исходит премудрость? и где место разума?
\par 21 Сокрыта она от очей всего живущего и от птиц небесных утаена.
\par 22 Аваддон и смерть говорят: ушами нашими слышали мы слух о ней.
\par 23 Бог знает путь ее, и Он ведает место ее.
\par 24 Ибо Он прозирает до концов земли и видит под всем небом.
\par 25 Когда Он ветру полагал вес и располагал воду по мере,
\par 26 когда назначал устав дождю и путь для молнии громоносной,
\par 27 тогда Он видел ее и явил ее, приготовил ее и еще испытал ее
\par 28 и сказал человеку: вот, страх Господень есть истинная премудрость, и удаление от зла--разум.

\chapter{29}

\par 1 И продолжал Иов возвышенную речь свою и сказал:
\par 2 о, если бы я был, как в прежние месяцы, как в те дни, когда Бог хранил меня,
\par 3 когда светильник Его светил над головою моею, и я при свете Его ходил среди тьмы;
\par 4 как был я во дни молодости моей, когда милость Божия [была] над шатром моим,
\par 5 когда еще Вседержитель [был] со мною, и дети мои вокруг меня,
\par 6 когда пути мои обливались молоком, и скала источала для меня ручьи елея!
\par 7 когда я выходил к воротам города и на площади ставил седалище свое, --
\par 8 юноши, увидев меня, прятались, а старцы вставали и стояли;
\par 9 князья удерживались от речи и персты полагали на уста свои;
\par 10 голос знатных умолкал, и язык их прилипал к гортани их.
\par 11 Ухо, слышавшее меня, ублажало меня; око видевшее восхваляло меня,
\par 12 потому что я спасал страдальца вопиющего и сироту беспомощного.
\par 13 Благословение погибавшего приходило на меня, и сердцу вдовы доставлял я радость.
\par 14 Я облекался в правду, и суд мой одевал меня, как мантия и увясло.
\par 15 Я был глазами слепому и ногами хромому;
\par 16 отцом был я для нищих и тяжбу, которой я не знал, разбирал внимательно.
\par 17 Сокрушал я беззаконному челюсти и из зубов его исторгал похищенное.
\par 18 И говорил я: в гнезде моем скончаюсь, и дни [мои] будут многи, как песок;
\par 19 корень мой открыт для воды, и роса ночует на ветвях моих;
\par 20 слава моя не стареет, лук мой крепок в руке моей.
\par 21 Внимали мне и ожидали, и безмолвствовали при совете моем.
\par 22 После слов моих уже не рассуждали; речь моя капала на них.
\par 23 Ждали меня, как дождя, и, [как] дождю позднему, открывали уста свои.
\par 24 Бывало, улыбнусь им--они не верят; и света лица моего они не помрачали.
\par 25 Я назначал пути им и сидел во главе и жил как царь в кругу воинов, как утешитель плачущих.

\chapter{30}

\par 1 А ныне смеются надо мною младшие меня летами, те, которых отцов я не согласился бы поместить с псами стад моих.
\par 2 И сила рук их к чему мне? Над ними уже прошло время.
\par 3 Бедностью и голодом истощенные, они убегают в степь безводную, мрачную и опустевшую;
\par 4 щиплют зелень подле кустов, и ягоды можжевельника--хлеб их.
\par 5 Из общества изгоняют их, кричат на них, как на воров,
\par 6 чтобы жили они в рытвинах потоков, в ущельях земли и утесов.
\par 7 Ревут между кустами, жмутся под терном.
\par 8 Люди отверженные, люди без имени, отребье земли!
\par 9 Их-то сделался я ныне песнью и пищею разговора их.
\par 10 Они гнушаются мною, удаляются от меня и не удерживаются плевать пред лицем моим.
\par 11 Так как Он развязал повод мой и поразил меня, то они сбросили с себя узду пред лицем моим.
\par 12 С правого боку встает это исчадие, сбивает меня с ног, направляет гибельные свои пути ко мне.
\par 13 А мою стезю испортили: все успели сделать к моей погибели, не имея помощника.
\par 14 Они пришли ко мне, как сквозь широкий пролом; с шумом бросились на меня.
\par 15 Ужасы устремились на меня; как ветер, развеялось величие мое, и счастье мое унеслось, как облако.
\par 16 И ныне изливается душа моя во мне: дни скорби объяли меня.
\par 17 Ночью ноют во мне кости мои, и жилы мои не имеют покоя.
\par 18 С великим трудом снимается с меня одежда моя; края хитона моего жмут меня.
\par 19 Он бросил меня в грязь, и я стал, как прах и пепел.
\par 20 Я взываю к Тебе, и Ты не внимаешь мне, --стою, а Ты [только] смотришь на меня.
\par 21 Ты сделался жестоким ко мне, крепкою рукою враждуешь против меня.
\par 22 Ты поднял меня и заставил меня носиться по ветру и сокрушаешь меня.
\par 23 Так, я знаю, что Ты приведешь меня к смерти и в дом собрания всех живущих.
\par 24 Верно, Он не прострет руки Своей на дом костей: будут ли они кричать при своем разрушении?
\par 25 Не плакал ли я о том, кто был в горе? не скорбела ли душа моя о бедных?
\par 26 Когда я чаял добра, пришло зло; когда ожидал света, пришла тьма.
\par 27 Мои внутренности кипят и не перестают; встретили меня дни печали.
\par 28 Я хожу почернелый, но не от солнца; встаю в собрании и кричу.
\par 29 Я стал братом шакалам и другом страусам.
\par 30 Моя кожа почернела на мне, и кости мои обгорели от жара.
\par 31 И цитра моя сделалась унылою, и свирель моя--голосом плачевным.

\chapter{31}

\par 1 Завет положил я с глазами моими, чтобы не помышлять мне о девице.
\par 2 Какая же участь [мне] от Бога свыше? И какое наследие от Вседержителя с небес?
\par 3 Не для нечестивого ли гибель, и не для делающего ли зло напасть?
\par 4 Не видел ли Он путей моих, и не считал ли всех моих шагов?
\par 5 Если я ходил в суете, и если нога моя спешила на лукавство, --
\par 6 пусть взвесят меня на весах правды, и Бог узнает мою непорочность.
\par 7 Если стопы мои уклонялись от пути и сердце мое следовало за глазами моими, и если что-либо [нечистое] пристало к рукам моим,
\par 8 то пусть я сею, а другой ест, и пусть отрасли мои искоренены будут.
\par 9 Если сердце мое прельщалось женщиною и я строил ковы у дверей моего ближнего, --
\par 10 пусть моя жена мелет на другого, и пусть другие издеваются над нею,
\par 11 потому что это--преступление, это--беззаконие, подлежащее суду;
\par 12 это--огонь, поядающий до истребления, который искоренил бы все добро мое.
\par 13 Если я пренебрегал правами слуги и служанки моей, когда они имели спор со мною,
\par 14 то что стал бы я делать, когда бы Бог восстал? И когда бы Он взглянул на меня, что мог бы я отвечать Ему?
\par 15 Не Он ли, Который создал меня во чреве, создал и его и равно образовал нас в утробе?
\par 16 Отказывал ли я нуждающимся в их просьбе и томил ли глаза вдовы?
\par 17 Один ли я съедал кусок мой, и не ел ли от него и сирота?
\par 18 Ибо с детства он рос со мною, как с отцом, и от чрева матери моей я руководил [вдову].
\par 19 Если я видел кого погибавшим без одежды и бедного без покрова, --
\par 20 не благословляли ли меня чресла его, и не был ли он согрет шерстью овец моих?
\par 21 Если я поднимал руку мою на сироту, когда видел помощь себе у ворот,
\par 22 то пусть плечо мое отпадет от спины, и рука моя пусть отломится от локтя,
\par 23 ибо страшно для меня наказание от Бога: пред величием Его не устоял бы я.
\par 24 Полагал ли я в золоте опору мою и говорил ли сокровищу: ты--надежда моя?
\par 25 Радовался ли я, что богатство мое было велико, и что рука моя приобрела много?
\par 26 Смотря на солнце, как оно сияет, и на луну, как она величественно шествует,
\par 27 прельстился ли я в тайне сердца моего, и целовали ли уста мои руку мою?
\par 28 Это также было бы преступление, подлежащее суду, потому что я отрекся бы [тогда] от Бога Всевышнего.
\par 29 Радовался ли я погибели врага моего и торжествовал ли, когда несчастье постигало его?
\par 30 Не позволял я устам моим грешить проклятием души его.
\par 31 Не говорили ли люди шатра моего: о, если бы мы от мяс его не насытились?
\par 32 Странник не ночевал на улице; двери мои я отворял прохожему.
\par 33 Если бы я скрывал проступки мои, как человек, утаивая в груди моей пороки мои,
\par 34 то я боялся бы большого общества, и презрение одноплеменников страшило бы меня, и я молчал бы и не выходил бы за двери.
\par 35 О, если бы кто выслушал меня! Вот мое желание, чтобы Вседержитель отвечал мне, и чтобы защитник мой составил запись.
\par 36 Я носил бы ее на плечах моих и возлагал бы ее, как венец;
\par 37 объявил бы ему число шагов моих, сблизился бы с ним, как с князем.
\par 38 Если вопияла на меня земля моя и жаловались на меня борозды ее;
\par 39 если я ел плоды ее без платы и отягощал жизнь земледельцев,
\par 40 то пусть вместо пшеницы вырастает волчец и вместо ячменя куколь. Слова Иова кончились.

\chapter{32}

\par 1 Когда те три мужа перестали отвечать Иову, потому что он был прав в глазах своих,
\par 2 тогда воспылал гнев Елиуя, сына Варахиилова, Вузитянина из племени Рамова: воспылал гнев его на Иова за то, что он оправдывал себя больше, нежели Бога,
\par 3 а на трех друзей его воспылал гнев его за то, что они не нашли, что отвечать, а между тем обвиняли Иова.
\par 4 Елиуй ждал, пока Иов говорил, потому что они летами были старше его.
\par 5 Когда же Елиуй увидел, что нет ответа в устах тех трех мужей, тогда воспылал гнев его.
\par 6 И отвечал Елиуй, сын Варахиилов, Вузитянин, и сказал: я молод летами, а вы--старцы; поэтому я робел и боялся объявлять вам мое мнение.
\par 7 Я говорил сам себе: пусть говорят дни, и многолетие поучает мудрости.
\par 8 Но дух в человеке и дыхание Вседержителя дает ему разумение.
\par 9 Не многолетние [только] мудры, и не старики разумеют правду.
\par 10 Поэтому я говорю: выслушайте меня, объявлю вам мое мнение и я.
\par 11 Вот, я ожидал слов ваших, --вслушивался в суждения ваши, доколе вы придумывали, что сказать.
\par 12 Я пристально смотрел на вас, и вот никто из вас не обличает Иова и не отвечает на слова его.
\par 13 Не скажите: мы нашли мудрость: Бог опровергнет его, а не человек.
\par 14 Если бы он обращал слова свои ко мне, то я не вашими речами отвечал бы ему.
\par 15 Испугались, не отвечают более; перестали говорить.
\par 16 И как я ждал, а они не говорят, остановились и не отвечают более,
\par 17 то и я отвечу с моей стороны, объявлю мое мнение и я,
\par 18 ибо я полон речами, и дух во мне теснит меня.
\par 19 Вот, утроба моя, как вино неоткрытое: она готова прорваться, подобно новым мехам.
\par 20 Поговорю, и будет легче мне; открою уста мои и отвечу.
\par 21 На лице человека смотреть не буду и никакому человеку льстить не стану,
\par 22 потому что я не умею льстить: сейчас убей меня, Творец мой.

\chapter{33}

\par 1 Итак слушай, Иов, речи мои и внимай всем словам моим.
\par 2 Вот, я открываю уста мои, язык мой говорит в гортани моей.
\par 3 Слова мои от искренности моего сердца, и уста мои произнесут знание чистое.
\par 4 Дух Божий создал меня, и дыхание Вседержителя дало мне жизнь.
\par 5 Если можешь, отвечай мне и стань передо мною.
\par 6 Вот я, по желанию твоему, вместо Бога. Я образован также из брения;
\par 7 поэтому страх передо мною не может смутить тебя, и рука моя не будет тяжела для тебя.
\par 8 Ты говорил в уши мои, и я слышал звук слов:
\par 9 чист я, без порока, невинен я, и нет во мне неправды;
\par 10 а Он нашел обвинение против меня и считает меня Своим противником;
\par 11 поставил ноги мои в колоду, наблюдает за всеми путями моими.
\par 12 Вот в этом ты неправ, отвечаю тебе, потому что Бог выше человека.
\par 13 Для чего тебе состязаться с Ним? Он не дает отчета ни в каких делах Своих.
\par 14 Бог говорит однажды и, если того не заметят, в другой раз:
\par 15 во сне, в ночном видении, когда сон находит на людей, во время дремоты на ложе.
\par 16 Тогда Он открывает у человека ухо и запечатлевает Свое наставление,
\par 17 чтобы отвести человека от какого-либо предприятия и удалить от него гордость,
\par 18 чтобы отвести душу его от пропасти и жизнь его от поражения мечом.
\par 19 Или он вразумляется болезнью на ложе своем и жестокою болью во всех костях своих, --
\par 20 и жизнь его отвращается от хлеба и душа его от любимой пищи.
\par 21 Плоть на нем пропадает, так что ее не видно, и показываются кости его, которых не было видно.
\par 22 И душа его приближается к могиле и жизнь его--к смерти.
\par 23 Если есть у него Ангел-наставник, один из тысячи, чтобы показать человеку прямой [путь] его, --
\par 24 [Бог] умилосердится над ним и скажет: освободи его от могилы; Я нашел умилостивление.
\par 25 Тогда тело его сделается свежее, нежели в молодости; он возвратится к дням юности своей.
\par 26 Будет молиться Богу, и Он--милостив к нему; с радостью взирает на лице его и возвращает человеку праведность его.
\par 27 Он будет смотреть на людей и говорить: грешил я и превращал правду, и не воздано мне;
\par 28 Он освободил душу мою от могилы, и жизнь моя видит свет.
\par 29 Вот, все это делает Бог два-три раза с человеком,
\par 30 чтобы отвести душу его от могилы и просветить его светом живых.
\par 31 Внимай, Иов, слушай меня, молчи, и я буду говорить.
\par 32 Если имеешь, что сказать, отвечай; говори, потому что я желал бы твоего оправдания;
\par 33 если же нет, то слушай меня: молчи, и я научу тебя мудрости.

\chapter{34}

\par 1 И продолжал Елиуй и сказал:
\par 2 выслушайте, мудрые, речь мою, и приклоните ко мне ухо, рассудительные!
\par 3 Ибо ухо разбирает слова, как гортань различает вкус в пище.
\par 4 Установим между собою рассуждение и распознаем, что хорошо.
\par 5 Вот, Иов сказал: я прав, но Бог лишил меня суда.
\par 6 Должен ли я лгать на правду мою? Моя рана неисцелима без вины.
\par 7 Есть ли такой человек, как Иов, который пьет глумление, как воду,
\par 8 вступает в сообщество с делающими беззаконие и ходит с людьми нечестивыми?
\par 9 Потому что он сказал: нет пользы для человека в благоугождении Богу.
\par 10 Итак послушайте меня, мужи мудрые! Не может быть у Бога неправда или у Вседержителя неправосудие,
\par 11 ибо Он по делам человека поступает с ним и по путям мужа воздает ему.
\par 12 Истинно, Бог не делает неправды и Вседержитель не извращает суда.
\par 13 Кто кроме Его промышляет о земле? И кто управляет всею вселенною?
\par 14 Если бы Он обратил сердце Свое к Себе и взял к Себе дух ее и дыхание ее, --
\par 15 вдруг погибла бы всякая плоть, и человек возвратился бы в прах.
\par 16 Итак, если ты имеешь разум, то слушай это и внимай словам моим.
\par 17 Ненавидящий правду может ли владычествовать? И можешь ли ты обвинить Всеправедного?
\par 18 Можно ли сказать царю: ты--нечестивец, и князьям: вы--беззаконники?
\par 19 Но Он не смотрит и на лица князей и не предпочитает богатого бедному, потому что все они дело рук Его.
\par 20 Внезапно они умирают; среди ночи народ возмутится, и они исчезают; и сильных изгоняют не силою.
\par 21 Ибо очи Его над путями человека, и Он видит все шаги его.
\par 22 Нет тьмы, ни тени смертной, где могли бы укрыться делающие беззаконие.
\par 23 Потому Он уже не требует от человека, чтобы шел на суд с Богом.
\par 24 Он сокрушает сильных без исследования и поставляет других на их места;
\par 25 потому что Он делает известными дела их и низлагает их ночью, и они истребляются.
\par 26 Он поражает их, как беззаконных людей, пред глазами других,
\par 27 за то, что они отвратились от Него и не уразумели всех путей Его,
\par 28 так что дошел до Него вопль бедных, и Он услышал стенание угнетенных.
\par 29 Дарует ли Он тишину, кто может возмутить? скрывает ли Он лице Свое, кто может увидеть Его? Будет ли это для народа, или для одного человека,
\par 30 чтобы не царствовал лицемер к соблазну народа.
\par 31 К Богу должно говорить: я потерпел, больше не буду грешить.
\par 32 А чего я не знаю, Ты научи меня; и если я сделал беззаконие, больше не буду.
\par 33 По твоему ли [рассуждению] Он должен воздавать? И как ты отвергаешь, то тебе следует избирать, а не мне; говори, что знаешь.
\par 34 Люди разумные скажут мне, и муж мудрый, слушающий меня:
\par 35 Иов не умно говорит, и слова его не со смыслом.
\par 36 Я желал бы, чтобы Иов вполне был испытан, по ответам его, свойственным людям нечестивым.
\par 37 Иначе он ко греху своему прибавит отступление, будет рукоплескать между нами и еще больше наговорит против Бога.

\chapter{35}

\par 1 И продолжал Елиуй и сказал:
\par 2 считаешь ли ты справедливым, что сказал: я правее Бога?
\par 3 Ты сказал: что пользы мне? и какую прибыль я имел бы пред тем, как если бы я и грешил?
\par 4 Я отвечу тебе и твоим друзьям с тобою:
\par 5 взгляни на небо и смотри; воззри на облака, они выше тебя.
\par 6 Если ты грешишь, что делаешь ты Ему? и если преступления твои умножаются, что причиняешь ты Ему?
\par 7 Если ты праведен, что даешь Ему? или что получает Он от руки твоей?
\par 8 Нечестие твое относится к человеку, как ты, и праведность твоя к сыну человеческому.
\par 9 От множества притеснителей стонут притесняемые, и от руки сильных вопиют.
\par 10 Но никто не говорит: где Бог, Творец мой, Который дает песни в ночи,
\par 11 Который научает нас более, нежели скотов земных, и вразумляет нас более, нежели птиц небесных?
\par 12 Там они вопиют, и Он не отвечает им, по причине гордости злых людей.
\par 13 Но неправда, что Бог не слышит и Вседержитель не взирает на это.
\par 14 Хотя ты сказал, что ты не видишь Его, но суд пред Ним, и--жди его.
\par 15 Но ныне, потому что гнев Его не посетил его и он не познал его во всей строгости,
\par 16 Иов и открыл легкомысленно уста свои и безрассудно расточает слова.

\chapter{36}

\par 1 И продолжал Елиуй и сказал:
\par 2 подожди меня немного, и я покажу тебе, что я имею еще что сказать за Бога.
\par 3 Начну мои рассуждения издалека и воздам Создателю моему справедливость,
\par 4 потому что слова мои точно не ложь: пред тобою--совершенный в познаниях.
\par 5 Вот, Бог могуществен и не презирает сильного крепостью сердца;
\par 6 Он не поддерживает нечестивых и воздает должное угнетенным;
\par 7 Он не отвращает очей Своих от праведников, но с царями навсегда посаждает их на престоле, и они возвышаются.
\par 8 Если же они окованы цепями и содержатся в узах бедствия,
\par 9 то Он указывает им на дела их и на беззакония их, потому что умножились,
\par 10 и открывает их ухо для вразумления и говорит им, чтоб они отстали от нечестия.
\par 11 Если послушают и будут служить Ему, то проведут дни свои в благополучии и лета свои в радости;
\par 12 если же не послушают, то погибнут от стрелы и умрут в неразумии.
\par 13 Но лицемеры питают в сердце гнев и не взывают к Нему, когда Он заключает их в узы;
\par 14 поэтому душа их умирает в молодости и жизнь их с блудниками.
\par 15 Он спасает бедного от беды его и в угнетении открывает ухо его.
\par 16 И тебя вывел бы Он из тесноты на простор, где нет стеснения, и поставляемое на стол твой было бы наполнено туком;
\par 17 но ты преисполнен суждениями нечестивых: суждение и осуждение--близки.
\par 18 Да не поразит тебя гнев [Божий] наказанием! Большой выкуп не спасет тебя.
\par 19 Даст ли Он какую цену твоему богатству? Нет, --ни золоту и никакому сокровищу.
\par 20 Не желай той ночи, когда народы истребляются на своем месте.
\par 21 Берегись, не склоняйся к нечестию, которое ты предпочел страданию.
\par 22 Бог высок могуществом Своим, и кто такой, как Он, наставник?
\par 23 Кто укажет Ему путь Его; кто может сказать: Ты поступаешь несправедливо?
\par 24 Помни о том, чтобы превозносить дела его, которые люди видят.
\par 25 Все люди могут видеть их; человек может усматривать их издали.
\par 26 Вот, Бог велик, и мы не можем познать Его; число лет Его неисследимо.
\par 27 Он собирает капли воды; они во множестве изливаются дождем:
\par 28 из облаков каплют и изливаются обильно на людей.
\par 29 Кто может также постигнуть протяжение облаков, треск шатра Его?
\par 30 Вот, Он распространяет над ним свет Свой и покрывает дно моря.
\par 31 Оттуда Он судит народы, дает пищу в изобилии.
\par 32 Он сокрывает в дланях Своих молнию и повелевает ей, кого разить.
\par 33 Треск ее дает знать о ней; скот также чувствует происходящее.

\chapter{37}

\par 1 И от сего трепещет сердце мое и подвиглось с места своего.
\par 2 Слушайте, слушайте голос Его и гром, исходящий из уст Его.
\par 3 Под всем небом раскат его, и блистание его--до краев земли.
\par 4 За ним гремит глас; гремит Он гласом величества Своего и не останавливает его, когда голос Его услышан.
\par 5 Дивно гремит Бог гласом Своим, делает дела великие, для нас непостижимые.
\par 6 Ибо снегу Он говорит: будь на земле; равно мелкий дождь и большой дождь в Его власти.
\par 7 Он полагает печать на руку каждого человека, чтобы все люди знали дело Его.
\par 8 Тогда зверь уходит в убежище и остается в своих логовищах.
\par 9 От юга приходит буря, от севера--стужа.
\par 10 От дуновения Божия происходит лед, и поверхность воды сжимается.
\par 11 Также влагою Он наполняет тучи, и облака сыплют свет Его,
\par 12 и они направляются по намерениям Его, чтоб исполнить то, что Он повелит им на лице обитаемой земли.
\par 13 Он повелевает им идти или для наказания, или в благоволение, или для помилования.
\par 14 Внимай сему, Иов; стой и разумевай чудные дела Божии.
\par 15 Знаешь ли, как Бог располагает ими и повелевает свету блистать из облака Своего?
\par 16 Разумеешь ли равновесие облаков, чудное дело Совершеннейшего в знании?
\par 17 Как нагревается твоя одежда, когда Он успокаивает землю от юга?
\par 18 Ты ли с Ним распростер небеса, твердые, как литое зеркало?
\par 19 Научи нас, что сказать Ему? Мы в этой тьме ничего не можем сообразить.
\par 20 Будет ли возвещено Ему, что я говорю? Сказал ли кто, что сказанное доносится Ему?
\par 21 Теперь не видно яркого света в облаках, но пронесется ветер и расчистит их.
\par 22 Светлая погода приходит от севера, и окрест Бога страшное великолепие.
\par 23 Вседержитель! мы не постигаем Его. Он велик силою, судом и полнотою правосудия. Он [никого] не угнетает.
\par 24 Посему да благоговеют пред Ним люди, и да трепещут пред Ним все мудрые сердцем!

\chapter{38}

\par 1 Господь отвечал Иову из бури и сказал:
\par 2 кто сей, омрачающий Провидение словами без смысла?
\par 3 Препояшь ныне чресла твои, как муж: Я буду спрашивать тебя, и ты объясняй Мне:
\par 4 где был ты, когда Я полагал основания земли? Скажи, если знаешь.
\par 5 Кто положил меру ей, если знаешь? или кто протягивал по ней вервь?
\par 6 На чем утверждены основания ее, или кто положил краеугольный камень ее,
\par 7 при общем ликовании утренних звезд, когда все сыны Божии восклицали от радости?
\par 8 Кто затворил море воротами, когда оно исторглось, вышло как бы из чрева,
\par 9 когда Я облака сделал одеждою его и мглу пеленами его,
\par 10 и утвердил ему Мое определение, и поставил запоры и ворота,
\par 11 и сказал: доселе дойдешь и не перейдешь, и здесь предел надменным волнам твоим?
\par 12 Давал ли ты когда в жизни своей приказания утру и указывал ли заре место ее,
\par 13 чтобы она охватила края земли и стряхнула с нее нечестивых,
\par 14 чтобы [земля] изменилась, как глина под печатью, и стала, как разноцветная одежда,
\par 15 и чтобы отнялся у нечестивых свет их и дерзкая рука их сокрушилась?
\par 16 Нисходил ли ты во глубину моря и входил ли в исследование бездны?
\par 17 Отворялись ли для тебя врата смерти, и видел ли ты врата тени смертной?
\par 18 Обозрел ли ты широту земли? Объясни, если знаешь все это.
\par 19 Где путь к жилищу света, и где место тьмы?
\par 20 Ты, конечно, доходил до границ ее и знаешь стези к дому ее.
\par 21 Ты знаешь это, потому что ты был уже тогда рожден, и число дней твоих очень велико.
\par 22 Входил ли ты в хранилища снега и видел ли сокровищницы града,
\par 23 которые берегу Я на время смутное, на день битвы и войны?
\par 24 По какому пути разливается свет и разносится восточный ветер по земле?
\par 25 Кто проводит протоки для излияния воды и путь для громоносной молнии,
\par 26 чтобы шел дождь на землю безлюдную, на пустыню, где нет человека,
\par 27 чтобы насыщать пустыню и степь и возбуждать травные зародыши к возрастанию?
\par 28 Есть ли у дождя отец? или кто рождает капли росы?
\par 29 Из чьего чрева выходит лед, и иней небесный, --кто рождает его?
\par 30 Воды, как камень, крепнут, и поверхность бездны замерзает.
\par 31 Можешь ли ты связать узел Хима и разрешить узы Кесиль?
\par 32 Можешь ли выводить созвездия в свое время и вести Ас с ее детьми?
\par 33 Знаешь ли ты уставы неба, можешь ли установить господство его на земле?
\par 34 Можешь ли возвысить голос твой к облакам, чтобы вода в обилии покрыла тебя?
\par 35 Можешь ли посылать молнии, и пойдут ли они и скажут ли тебе: вот мы?
\par 36 Кто вложил мудрость в сердце, или кто дал смысл разуму?
\par 37 Кто может расчислить облака своею мудростью и удержать сосуды неба,
\par 38 когда пыль обращается в грязь и глыбы слипаются?
\par 39 Ты ли ловишь добычу львице и насыщаешь молодых львов,
\par 40 когда они лежат в берлогах или покоятся под тенью в засаде?
\par 41 Кто приготовляет ворону корм его, когда птенцы его кричат к Богу, бродя без пищи?

\chapter{39}

\par 1 Знаешь ли ты время, когда рождаются дикие козы на скалах, и замечал ли роды ланей?
\par 2 можешь ли расчислить месяцы беременности их? и знаешь ли время родов их?
\par 3 Они изгибаются, рождая детей своих, выбрасывая свои ноши;
\par 4 дети их приходят в силу, растут на поле, уходят и не возвращаются к ним.
\par 5 Кто пустил дикого осла на свободу, и кто разрешил узы онагру,
\par 6 которому степь Я назначил домом и солончаки--жилищем?
\par 7 Он посмевается городскому многолюдству и не слышит криков погонщика,
\par 8 по горам ищет себе пищи и гоняется за всякою зеленью.
\par 9 Захочет ли единорог служить тебе и переночует ли у яслей твоих?
\par 10 Можешь ли веревкою привязать единорога к борозде, и станет ли он боронить за тобою поле?
\par 11 Понадеешься ли на него, потому что у него сила велика, и предоставишь ли ему работу твою?
\par 12 Поверишь ли ему, что он семена твои возвратит и сложит на гумно твое?
\par 13 Ты ли дал красивые крылья павлину и перья и пух страусу?
\par 14 Он оставляет яйца свои на земле, и на песке согревает их,
\par 15 и забывает, что нога может раздавить их и полевой зверь может растоптать их;
\par 16 он жесток к детям своим, как бы не своим, и не опасается, что труд его будет напрасен;
\par 17 потому что Бог не дал ему мудрости и не уделил ему смысла;
\par 18 а когда поднимется на высоту, посмевается коню и всаднику его.
\par 19 Ты ли дал коню силу и облек шею его гривою?
\par 20 Можешь ли ты испугать его, как саранчу? Храпение ноздрей его--ужас;
\par 21 роет ногою землю и восхищается силою; идет навстречу оружию;
\par 22 он смеется над опасностью и не робеет и не отворачивается от меча;
\par 23 колчан звучит над ним, сверкает копье и дротик;
\par 24 в порыве и ярости он глотает землю и не может стоять при звуке трубы;
\par 25 при трубном звуке он издает голос: гу! гу! и издалека чует битву, громкие голоса вождей и крик.
\par 26 Твоею ли мудростью летает ястреб и направляет крылья свои на полдень?
\par 27 По твоему ли слову возносится орел и устрояет на высоте гнездо свое?
\par 28 Он живет на скале и ночует на зубце утесов и на местах неприступных;
\par 29 оттуда высматривает себе пищу: глаза его смотрят далеко;
\par 30 птенцы его пьют кровь, и где труп, там и он.
\par 31 И продолжал Господь и сказал Иову:
\par 32 будет ли состязающийся со Вседержителем еще учить? Обличающий Бога пусть отвечает Ему.
\par 33 И отвечал Иов Господу и сказал:
\par 34 вот, я ничтожен; что буду я отвечать Тебе? Руку мою полагаю на уста мои.
\par 35 Однажды я говорил, --теперь отвечать не буду, даже дважды, но более не буду.

\chapter{40}

\par 1 И отвечал Господь Иову из бури и сказал:
\par 2 препояшь, как муж, чресла твои: Я буду спрашивать тебя, а ты объясняй Мне.
\par 3 Ты хочешь ниспровергнуть суд Мой, обвинить Меня, чтобы оправдать себя?
\par 4 Такая ли у тебя мышца, как у Бога? И можешь ли возгреметь голосом, как Он?
\par 5 Укрась же себя величием и славою, облекись в блеск и великолепие;
\par 6 излей ярость гнева твоего, посмотри на все гордое и смири его;
\par 7 взгляни на всех высокомерных и унизь их, и сокруши нечестивых на местах их;
\par 8 зарой всех их в землю и лица их покрой тьмою.
\par 9 Тогда и Я признаю, что десница твоя может спасать тебя.
\par 10 Вот бегемот, которого Я создал, как и тебя; он ест траву, как вол;
\par 11 вот, его сила в чреслах его и крепость его в мускулах чрева его;
\par 12 поворачивает хвостом своим, как кедром; жилы же на бедрах его переплетены;
\par 13 ноги у него, как медные трубы; кости у него, как железные прутья;
\par 14 это--верх путей Божиих; только Сотворивший его может приблизить к нему меч Свой;
\par 15 горы приносят ему пищу, и там все звери полевые играют;
\par 16 он ложится под тенистыми деревьями, под кровом тростника и в болотах;
\par 17 тенистые дерева покрывают его своею тенью; ивы при ручьях окружают его;
\par 18 вот, он пьет из реки и не торопится; остается спокоен, хотя бы Иордан устремился ко рту его.
\par 19 Возьмет ли кто его в глазах его и проколет ли ему нос багром?
\par 20 Можешь ли ты удою вытащить левиафана и веревкою схватить за язык его?
\par 21 вденешь ли кольцо в ноздри его? проколешь ли иглою челюсть его?
\par 22 будет ли он много умолять тебя и будет ли говорить с тобою кротко?
\par 23 сделает ли он договор с тобою, и возьмешь ли его навсегда себе в рабы?
\par 24 станешь ли забавляться им, как птичкою, и свяжешь ли его для девочек твоих?
\par 25 будут ли продавать его товарищи ловли, разделят ли его между Хананейскими купцами?
\par 26 можешь ли пронзить кожу его копьем и голову его рыбачьею острогою?
\par 27 Клади на него руку твою, и помни о борьбе: вперед не будешь.

\chapter{41}

\par 1 Надежда тщетна: не упадешь ли от одного взгляда его?
\par 2 Нет столь отважного, который осмелился бы потревожить его; кто же может устоять перед Моим лицем?
\par 3 Кто предварил Меня, чтобы Мне воздавать ему? под всем небом все Мое.
\par 4 Не умолчу о членах его, о силе и красивой соразмерности их.
\par 5 Кто может открыть верх одежды его, кто подойдет к двойным челюстям его?
\par 6 Кто может отворить двери лица его? круг зубов его--ужас;
\par 7 крепкие щиты его--великолепие; они скреплены как бы твердою печатью;
\par 8 один к другому прикасается близко, так что и воздух не проходит между ними;
\par 9 один с другим лежат плотно, сцепились и не раздвигаются.
\par 10 От его чихания показывается свет; глаза у него как ресницы зари;
\par 11 из пасти его выходят пламенники, выскакивают огненные искры;
\par 12 из ноздрей его выходит дым, как из кипящего горшка или котла.
\par 13 Дыхание его раскаляет угли, и из пасти его выходит пламя.
\par 14 На шее его обитает сила, и перед ним бежит ужас.
\par 15 Мясистые части тела его сплочены между собою твердо, не дрогнут.
\par 16 Сердце его твердо, как камень, и жестко, как нижний жернов.
\par 17 Когда он поднимается, силачи в страхе, совсем теряются от ужаса.
\par 18 Меч, коснувшийся его, не устоит, ни копье, ни дротик, ни латы.
\par 19 Железо он считает за солому, медь--за гнилое дерево.
\par 20 Дочь лука не обратит его в бегство; пращные камни обращаются для него в плеву.
\par 21 Булава считается у него за соломину; свисту дротика он смеется.
\par 22 Под ним острые камни, и он на острых камнях лежит в грязи.
\par 23 Он кипятит пучину, как котел, и море претворяет в кипящую мазь;
\par 24 оставляет за собою светящуюся стезю; бездна кажется сединою.
\par 25 Нет на земле подобного ему; он сотворен бесстрашным;
\par 26 на все высокое смотрит смело; он царь над всеми сынами гордости.

\chapter{42}

\par 1 И отвечал Иов Господу и сказал:
\par 2 знаю, что Ты все можешь, и что намерение Твое не может быть остановлено.
\par 3 Кто сей, омрачающий Провидение, ничего не разумея? --Так, я говорил о том, чего не разумел, о делах чудных для меня, которых я не знал.
\par 4 Выслушай, [взывал я,] и я буду говорить, и что буду спрашивать у Тебя, объясни мне.
\par 5 Я слышал о Тебе слухом уха; теперь же мои глаза видят Тебя;
\par 6 поэтому я отрекаюсь и раскаиваюсь в прахе и пепле.
\par 7 И было после того, как Господь сказал слова те Иову, сказал Господь Елифазу Феманитянину: горит гнев Мой на тебя и на двух друзей твоих за то, что вы говорили о Мне не так верно, как раб Мой Иов.
\par 8 Итак возьмите себе семь тельцов и семь овнов и пойдите к рабу Моему Иову и принесите за себя жертву; и раб Мой Иов помолится за вас, ибо только лице его Я приму, дабы не отвергнуть вас за то, что вы говорили о Мне не так верно, как раб Мой Иов.
\par 9 И пошли Елифаз Феманитянин и Вилдад Савхеянин и Софар Наамитянин, и сделали так, как Господь повелел им, --и Господь принял лице Иова.
\par 10 И возвратил Господь потерю Иова, когда он помолился за друзей своих; и дал Господь Иову вдвое больше того, что он имел прежде.
\par 11 Тогда пришли к нему все братья его и все сестры его и все прежние знакомые его, и ели с ним хлеб в доме его, и тужили с ним, и утешали его за все зло, которое Господь навел на него, и дали ему каждый по кесите и по золотому кольцу.
\par 12 И благословил Бог последние дни Иова более, нежели прежние: у него было четырнадцать тысяч мелкого скота, шесть тысяч верблюдов, тысяча пар волов и тысяча ослиц.
\par 13 И было у него семь сыновей и три дочери.
\par 14 И нарек он имя первой Емима, имя второй--Кассия, а имя третьей--Керенгаппух.
\par 15 И не было на всей земле таких прекрасных женщин, как дочери Иова, и дал им отец их наследство между братьями их.
\par 16 После того Иов жил сто сорок лет, и видел сыновей своих и сыновей сыновних до четвертого рода;
\par 17 и умер Иов в старости, насыщенный днями.


\end{document}