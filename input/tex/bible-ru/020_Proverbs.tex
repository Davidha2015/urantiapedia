\begin{document}

\title{Притчей Соломоновых}


\chapter{1}

\par 1 Притчи Соломона, сына Давидова, царя Израильского,
\par 2 чтобы познать мудрость и наставление, понять изречения разума;
\par 3 усвоить правила благоразумия, правосудия, суда и правоты;
\par 4 простым дать смышленость, юноше--знание и рассудительность;
\par 5 послушает мудрый--и умножит познания, и разумный найдет мудрые советы;
\par 6 чтобы разуметь притчу и замысловатую речь, слова мудрецов и загадки их.
\par 7 Начало мудрости--страх Господень; глупцы только презирают мудрость и наставление.
\par 8 Слушай, сын мой, наставление отца твоего и не отвергай завета матери твоей,
\par 9 потому что это--прекрасный венок для головы твоей и украшение для шеи твоей.
\par 10 Сын мой! если будут склонять тебя грешники, не соглашайся;
\par 11 если будут говорить: `иди с нами, сделаем засаду для убийства, подстережем непорочного без вины,
\par 12 живых проглотим их, как преисподняя, и--целых, как нисходящих в могилу;
\par 13 наберем всякого драгоценного имущества, наполним домы наши добычею;
\par 14 жребий твой ты будешь бросать вместе с нами, склад один будет у всех нас', --
\par 15 сын мой! не ходи в путь с ними, удержи ногу твою от стези их,
\par 16 потому что ноги их бегут ко злу и спешат на пролитие крови.
\par 17 В глазах всех птиц напрасно расставляется сеть,
\par 18 а делают засаду для их крови и подстерегают их души.
\par 19 Таковы пути всякого, кто алчет чужого добра: оно отнимает жизнь у завладевшего им.
\par 20 Премудрость возглашает на улице, на площадях возвышает голос свой,
\par 21 в главных местах собраний проповедует, при входах в городские ворота говорит речь свою:
\par 22 `доколе, невежды, будете любить невежество? [доколе] буйные будут услаждаться буйством? доколе глупцы будут ненавидеть знание?
\par 23 Обратитесь к моему обличению: вот, я изолью на вас дух мой, возвещу вам слова мои.
\par 24 Я звала, и вы не послушались; простирала руку мою, и не было внимающего;
\par 25 и вы отвергли все мои советы, и обличений моих не приняли.
\par 26 За то и я посмеюсь вашей погибели; порадуюсь, когда придет на вас ужас;
\par 27 когда придет на вас ужас, как буря, и беда, как вихрь, принесется на вас; когда постигнет вас скорбь и теснота.
\par 28 Тогда будут звать меня, и я не услышу; с утра будут искать меня, и не найдут меня.
\par 29 За то, что они возненавидели знание и не избрали [для себя] страха Господня,
\par 30 не приняли совета моего, презрели все обличения мои;
\par 31 за то и будут они вкушать от плодов путей своих и насыщаться от помыслов их.
\par 32 Потому что упорство невежд убьет их, и беспечность глупцов погубит их,
\par 33 а слушающий меня будет жить безопасно и спокойно, не страшась зла'.

\chapter{2}

\par 1 Сын мой! если ты примешь слова мои и сохранишь при себе заповеди мои,
\par 2 так что ухо твое сделаешь внимательным к мудрости и наклонишь сердце твое к размышлению;
\par 3 если будешь призывать знание и взывать к разуму;
\par 4 если будешь искать его, как серебра, и отыскивать его, как сокровище,
\par 5 то уразумеешь страх Господень и найдешь познание о Боге.
\par 6 Ибо Господь дает мудрость; из уст Его--знание и разум;
\par 7 Он сохраняет для праведных спасение; Он--щит для ходящих непорочно;
\par 8 Он охраняет пути правды и оберегает стезю святых Своих.
\par 9 Тогда ты уразумеешь правду и правосудие и прямоту, всякую добрую стезю.
\par 10 Когда мудрость войдет в сердце твое, и знание будет приятно душе твоей,
\par 11 тогда рассудительность будет оберегать тебя, разум будет охранять тебя,
\par 12 дабы спасти тебя от пути злого, от человека, говорящего ложь,
\par 13 от тех, которые оставляют стези прямые, чтобы ходить путями тьмы;
\par 14 от тех, которые радуются, делая зло, восхищаются злым развратом,
\par 15 которых пути кривы, и которые блуждают на стезях своих;
\par 16 дабы спасти тебя от жены другого, от чужой, которая умягчает речи свои,
\par 17 которая оставила руководителя юности своей и забыла завет Бога своего.
\par 18 Дом ее ведет к смерти, и стези ее--к мертвецам;
\par 19 никто из вошедших к ней не возвращается и не вступает на путь жизни.
\par 20 Посему ходи путем добрых и держись стезей праведников,
\par 21 потому что праведные будут жить на земле, и непорочные пребудут на ней;
\par 22 а беззаконные будут истреблены с земли, и вероломные искоренены из нее.

\chapter{3}

\par 1 Сын мой! наставления моего не забывай, и заповеди мои да хранит сердце твое;
\par 2 ибо долготы дней, лет жизни и мира они приложат тебе.
\par 3 Милость и истина да не оставляют тебя: обвяжи ими шею твою, напиши их на скрижали сердца твоего,
\par 4 и обретешь милость и благоволение в очах Бога и людей.
\par 5 Надейся на Господа всем сердцем твоим, и не полагайся на разум твой.
\par 6 Во всех путях твоих познавай Его, и Он направит стези твои.
\par 7 Не будь мудрецом в глазах твоих; бойся Господа и удаляйся от зла:
\par 8 это будет здравием для тела твоего и питанием для костей твоих.
\par 9 Чти Господа от имения твоего и от начатков всех прибытков твоих,
\par 10 и наполнятся житницы твои до избытка, и точила твои будут переливаться новым вином.
\par 11 Наказания Господня, сын мой, не отвергай, и не тяготись обличением Его;
\par 12 ибо кого любит Господь, того наказывает и благоволит к тому, как отец к сыну своему.
\par 13 Блажен человек, который снискал мудрость, и человек, который приобрел разум, --
\par 14 потому что приобретение ее лучше приобретения серебра, и прибыли от нее больше, нежели от золота:
\par 15 она дороже драгоценных камней; и ничто из желаемого тобою не сравнится с нею.
\par 16 Долгоденствие--в правой руке ее, а в левой у нее--богатство и слава;
\par 17 пути ее--пути приятные, и все стези ее--мирные.
\par 18 Она--древо жизни для тех, которые приобретают ее, --и блаженны, которые сохраняют ее!
\par 19 Господь премудростью основал землю, небеса утвердил разумом;
\par 20 Его премудростью разверзлись бездны, и облака кропят росою.
\par 21 Сын мой! не упускай их из глаз твоих; храни здравомыслие и рассудительность,
\par 22 и они будут жизнью для души твоей и украшением для шеи твоей.
\par 23 Тогда безопасно пойдешь по пути твоему, и нога твоя не споткнется.
\par 24 Когда ляжешь спать, --не будешь бояться; и когда уснешь, --сон твой приятен будет.
\par 25 Не убоишься внезапного страха и пагубы от нечестивых, когда она придет;
\par 26 потому что Господь будет упованием твоим и сохранит ногу твою от уловления.
\par 27 Не отказывай в благодеянии нуждающемуся, когда рука твоя в силе сделать его.
\par 28 Не говори другу твоему: `пойди и приди опять, и завтра я дам', когда ты имеешь при себе.
\par 29 Не замышляй против ближнего твоего зла, когда он без опасения живет с тобою.
\par 30 Не ссорься с человеком без причины, когда он не сделал зла тебе.
\par 31 Не соревнуй человеку, поступающему насильственно, и не избирай ни одного из путей его;
\par 32 потому что мерзость пред Господом развратный, а с праведными у Него общение.
\par 33 Проклятие Господне на доме нечестивого, а жилище благочестивых Он благословляет.
\par 34 Если над кощунниками Он посмевается, то смиренным дает благодать.
\par 35 Мудрые наследуют славу, а глупые--бесславие.

\chapter{4}

\par 1 Слушайте, дети, наставление отца, и внимайте, чтобы научиться разуму,
\par 2 потому что я преподал вам доброе учение. Не оставляйте заповеди моей.
\par 3 Ибо и я был сын у отца моего, нежно любимый и единственный у матери моей,
\par 4 и он учил меня и говорил мне: да удержит сердце твое слова мои; храни заповеди мои, и живи.
\par 5 Приобретай мудрость, приобретай разум: не забывай этого и не уклоняйся от слов уст моих.
\par 6 Не оставляй ее, и она будет охранять тебя; люби ее, и она будет оберегать тебя.
\par 7 Главное--мудрость: приобретай мудрость, и всем имением твоим приобретай разум.
\par 8 Высоко цени ее, и она возвысит тебя; она прославит тебя, если ты прилепишься к ней;
\par 9 возложит на голову твою прекрасный венок, доставит тебе великолепный венец.
\par 10 Слушай, сын мой, и прими слова мои, --и умножатся тебе лета жизни.
\par 11 Я указываю тебе путь мудрости, веду тебя по стезям прямым.
\par 12 Когда пойдешь, не будет стеснен ход твой, и когда побежишь, не споткнешься.
\par 13 Крепко держись наставления, не оставляй, храни его, потому что оно--жизнь твоя.
\par 14 Не вступай на стезю нечестивых и не ходи по пути злых;
\par 15 оставь его, не ходи по нему, уклонись от него и пройди мимо;
\par 16 потому что они не заснут, если не сделают зла; пропадает сон у них, если они не доведут кого до падения;
\par 17 ибо они едят хлеб беззакония и пьют вино хищения.
\par 18 Стезя праведных--как светило лучезарное, которое более и более светлеет до полного дня.
\par 19 Путь же беззаконных--как тьма; они не знают, обо что споткнутся.
\par 20 Сын мой! словам моим внимай, и к речам моим приклони ухо твое;
\par 21 да не отходят они от глаз твоих; храни их внутри сердца твоего:
\par 22 потому что они жизнь для того, кто нашел их, и здравие для всего тела его.
\par 23 Больше всего хранимого храни сердце твое, потому что из него источники жизни.
\par 24 Отвергни от себя лживость уст, и лукавство языка удали от себя.
\par 25 Глаза твои пусть прямо смотрят, и ресницы твои да направлены будут прямо пред тобою.
\par 26 Обдумай стезю для ноги твоей, и все пути твои да будут тверды.
\par 27 Не уклоняйся ни направо, ни налево; удали ногу твою от зла,

\chapter{5}

\par 1 Сын мой! внимай мудрости моей, и приклони ухо твое к разуму моему,
\par 2 чтобы соблюсти рассудительность, и чтобы уста твои сохранили знание.
\par 3 ибо мед источают уста чужой жены, и мягче елея речь ее;
\par 4 но последствия от нее горьки, как полынь, остры, как меч обоюдоострый;
\par 5 ноги ее нисходят к смерти, стопы ее достигают преисподней.
\par 6 Если бы ты захотел постигнуть стезю жизни ее, то пути ее непостоянны, и ты не узнаешь их.
\par 7 Итак, дети, слушайте меня и не отступайте от слов уст моих.
\par 8 Держи дальше от нее путь твой и не подходи близко к дверям дома ее,
\par 9 чтобы здоровья твоего не отдать другим и лет твоих мучителю;
\par 10 чтобы не насыщались силою твоею чужие, и труды твои не были для чужого дома.
\par 11 И ты будешь стонать после, когда плоть твоя и тело твое будут истощены, --
\par 12 и скажешь: `зачем я ненавидел наставление, и сердце мое пренебрегало обличением,
\par 13 и я не слушал голоса учителей моих, не приклонял уха моего к наставникам моим:
\par 14 едва не впал я во всякое зло среди собрания и общества!'
\par 15 Пей воду из твоего водоема и текущую из твоего колодезя.
\par 16 Пусть [не] разливаются источники твои по улице, потоки вод--по площадям;
\par 17 пусть они будут принадлежать тебе одному, а не чужим с тобою.
\par 18 Источник твой да будет благословен; и утешайся женою юности твоей,
\par 19 любезною ланью и прекрасною серною: груди ее да упоявают тебя во всякое время, любовью ее услаждайся постоянно.
\par 20 И для чего тебе, сын мой, увлекаться постороннею и обнимать груди чужой?
\par 21 Ибо пред очами Господа пути человека, и Он измеряет все стези его.
\par 22 Беззаконного уловляют собственные беззакония его, и в узах греха своего он содержится:
\par 23 он умирает без наставления, и от множества безумия своего теряется.

\chapter{6}

\par 1 Сын мой! если ты поручился за ближнего твоего и дал руку твою за другого, --
\par 2 ты опутал себя словами уст твоих, пойман словами уст твоих.
\par 3 Сделай же, сын мой, вот что, и избавь себя, так как ты попался в руки ближнего твоего: пойди, пади к ногам и умоляй ближнего твоего;
\par 4 не давай сна глазам твоим и дремания веждам твоим;
\par 5 спасайся, как серна из руки и как птица из руки птицелова.
\par 6 Пойди к муравью, ленивец, посмотри на действия его, и будь мудрым.
\par 7 Нет у него ни начальника, ни приставника, ни повелителя;
\par 8 но он заготовляет летом хлеб свой, собирает во время жатвы пищу свою.
\par 9 Доколе ты, ленивец, будешь спать? когда ты встанешь от сна твоего?
\par 10 Немного поспишь, немного подремлешь, немного, сложив руки, полежишь:
\par 11 и придет, как прохожий, бедность твоя, и нужда твоя, как разбойник.
\par 12 Человек лукавый, человек нечестивый ходит со лживыми устами,
\par 13 мигает глазами своими, говорит ногами своими, дает знаки пальцами своими;
\par 14 коварство в сердце его: он умышляет зло во всякое время, сеет раздоры.
\par 15 Зато внезапно придет погибель его, вдруг будет разбит--без исцеления.
\par 16 Вот шесть, что ненавидит Господь, даже семь, что мерзость душе Его:
\par 17 глаза гордые, язык лживый и руки, проливающие кровь невинную,
\par 18 сердце, кующее злые замыслы, ноги, быстро бегущие к злодейству,
\par 19 лжесвидетель, наговаривающий ложь и сеющий раздор между братьями.
\par 20 Сын мой! храни заповедь отца твоего и не отвергай наставления матери твоей;
\par 21 навяжи их навсегда на сердце твое, обвяжи ими шею твою.
\par 22 Когда ты пойдешь, они будут руководить тебя; когда ляжешь спать, будут охранять тебя; когда пробудишься, будут беседовать с тобою:
\par 23 ибо заповедь есть светильник, и наставление--свет, и назидательные поучения--путь к жизни,
\par 24 чтобы остерегать тебя от негодной женщины, от льстивого языка чужой.
\par 25 Не пожелай красоты ее в сердце твоем, и да не увлечет она тебя ресницами своими;
\par 26 потому что из-за жены блудной [обнищевают] до куска хлеба, а замужняя жена уловляет дорогую душу.
\par 27 Может ли кто взять себе огонь в пазуху, чтобы не прогорело платье его?
\par 28 Может ли кто ходить по горящим угольям, чтобы не обжечь ног своих?
\par 29 То же бывает и с тем, кто входит к жене ближнего своего: кто прикоснется к ней, не останется без вины.
\par 30 Не спускают вору, если он крадет, чтобы насытить душу свою, когда он голоден;
\par 31 но, будучи пойман, он заплатит всемеро, отдаст все имущество дома своего.
\par 32 Кто же прелюбодействует с женщиною, у того нет ума; тот губит душу свою, кто делает это:
\par 33 побои и позор найдет он, и бесчестие его не изгладится,
\par 34 потому что ревность--ярость мужа, и не пощадит он в день мщения,
\par 35 не примет никакого выкупа и не удовольствуется, сколько бы ты ни умножал даров.

\chapter{7}

\par 1 Сын мой! храни слова мои и заповеди мои сокрой у себя.
\par 2 Храни заповеди мои и живи, и учение мое, как зрачок глаз твоих.
\par 3 Навяжи их на персты твои, напиши их на скрижали сердца твоего.
\par 4 Скажи мудрости: `Ты сестра моя!' и разум назови родным твоим,
\par 5 чтобы они охраняли тебя от жены другого, от чужой, которая умягчает слова свои.
\par 6 Вот, однажды смотрел я в окно дома моего, сквозь решетку мою,
\par 7 и увидел среди неопытных, заметил между молодыми людьми неразумного юношу,
\par 8 переходившего площадь близ угла ее и шедшего по дороге к дому ее,
\par 9 в сумерки в вечер дня, в ночной темноте и во мраке.
\par 10 И вот--навстречу к нему женщина, в наряде блудницы, с коварным сердцем,
\par 11 шумливая и необузданная; ноги ее не живут в доме ее:
\par 12 то на улице, то на площадях, и у каждого угла строит она ковы.
\par 13 Она схватила его, целовала его, и с бесстыдным лицом говорила ему:
\par 14 `мирная жертва у меня: сегодня я совершила обеты мои;
\par 15 поэтому и вышла навстречу тебе, чтобы отыскать тебя, и--нашла тебя;
\par 16 коврами я убрала постель мою, разноцветными тканями Египетскими;
\par 17 спальню мою надушила смирною, алоем и корицею;
\par 18 зайди, будем упиваться нежностями до утра, насладимся любовью,
\par 19 потому что мужа нет дома: он отправился в дальнюю дорогу;
\par 20 кошелек серебра взял с собою; придет домой ко дню полнолуния'.
\par 21 Множеством ласковых слов она увлекла его, мягкостью уст своих овладела им.
\par 22 Тотчас он пошел за нею, как вол идет на убой, и как олень--на выстрел,
\par 23 доколе стрела не пронзит печени его; как птичка кидается в силки, и не знает, что они--на погибель ее.
\par 24 Итак, дети, слушайте меня и внимайте словам уст моих.
\par 25 Да не уклоняется сердце твое на пути ее, не блуждай по стезям ее,
\par 26 потому что многих повергла она ранеными, и много сильных убиты ею:
\par 27 дом ее--пути в преисподнюю, нисходящие во внутренние жилища смерти.

\chapter{8}

\par 1 Не премудрость ли взывает? и не разум ли возвышает голос свой?
\par 2 Она становится на возвышенных местах, при дороге, на распутиях;
\par 3 она взывает у ворот при входе в город, при входе в двери:
\par 4 `к вам, люди, взываю я, и к сынам человеческим голос мой!
\par 5 Научитесь, неразумные, благоразумию, и глупые--разуму.
\par 6 Слушайте, потому что я буду говорить важное, и изречение уст моих--правда;
\par 7 ибо истину произнесет язык мой, и нечестие--мерзость для уст моих;
\par 8 все слова уст моих справедливы; нет в них коварства и лукавства;
\par 9 все они ясны для разумного и справедливы для приобретших знание.
\par 10 Примите учение мое, а не серебро; лучше знание, нежели отборное золото;
\par 11 потому что мудрость лучше жемчуга, и ничто из желаемого не сравнится с нею.
\par 12 Я, премудрость, обитаю с разумом и ищу рассудительного знания.
\par 13 Страх Господень--ненавидеть зло; гордость и высокомерие и злой путь и коварные уста я ненавижу.
\par 14 У меня совет и правда; я разум, у меня сила.
\par 15 Мною цари царствуют и повелители узаконяют правду;
\par 16 мною начальствуют начальники и вельможи и все судьи земли.
\par 17 Любящих меня я люблю, и ищущие меня найдут меня;
\par 18 богатство и слава у меня, сокровище непогибающее и правда;
\par 19 плоды мои лучше золота, и золота самого чистого, и пользы от меня больше, нежели от отборного серебра.
\par 20 Я хожу по пути правды, по стезям правосудия,
\par 21 чтобы доставить любящим меня существенное благо, и сокровищницы их я наполняю.
\par 22 Господь имел меня началом пути Своего, прежде созданий Своих, искони;
\par 23 от века я помазана, от начала, прежде бытия земли.
\par 24 Я родилась, когда еще не существовали бездны, когда еще не было источников, обильных водою.
\par 25 Я родилась прежде, нежели водружены были горы, прежде холмов,
\par 26 когда еще Он не сотворил ни земли, ни полей, ни начальных пылинок вселенной.
\par 27 Когда Он уготовлял небеса, [я была] там. Когда Он проводил круговую черту по лицу бездны,
\par 28 когда утверждал вверху облака, когда укреплял источники бездны,
\par 29 когда давал морю устав, чтобы воды не переступали пределов его, когда полагал основания земли:
\par 30 тогда я была при Нем художницею, и была радостью всякий день, веселясь пред лицем Его во все время,
\par 31 веселясь на земном кругу Его, и радость моя [была] с сынами человеческими.
\par 32 Итак, дети, послушайте меня; и блаженны те, которые хранят пути мои!
\par 33 Послушайте наставления и будьте мудры, и не отступайте [от] [него].
\par 34 Блажен человек, который слушает меня, бодрствуя каждый день у ворот моих и стоя на страже у дверей моих!
\par 35 потому что, кто нашел меня, тот нашел жизнь, и получит благодать от Господа;
\par 36 а согрешающий против меня наносит вред душе своей: все ненавидящие меня любят смерть'.

\chapter{9}

\par 1 Премудрость построила себе дом, вытесала семь столбов его,
\par 2 заколола жертву, растворила вино свое и приготовила у себя трапезу;
\par 3 послала слуг своих провозгласить с возвышенностей городских:
\par 4 `кто неразумен, обратись сюда!' И скудоумному она сказала:
\par 5 `идите, ешьте хлеб мой и пейте вино, мною растворенное;
\par 6 оставьте неразумие, и живите, и ходите путем разума'.
\par 7 Поучающий кощунника наживет себе бесславие, и обличающий нечестивого--пятно себе.
\par 8 Не обличай кощунника, чтобы он не возненавидел тебя; обличай мудрого, и он возлюбит тебя;
\par 9 дай [наставление] мудрому, и он будет еще мудрее; научи правдивого, и он приумножит знание.
\par 10 Начало мудрости--страх Господень, и познание Святаго--разум;
\par 11 потому что чрез меня умножатся дни твои, и прибавится тебе лет жизни.
\par 12 если ты мудр, то мудр для себя; и если буен, то один потерпишь.
\par 13 Женщина безрассудная, шумливая, глупая и ничего не знающая
\par 14 садится у дверей дома своего на стуле, на возвышенных местах города,
\par 15 чтобы звать проходящих дорогою, идущих прямо своими путями:
\par 16 `кто глуп, обратись сюда!' и скудоумному сказала она:
\par 17 `воды краденые сладки, и утаенный хлеб приятен'.
\par 18 И он не знает, что мертвецы там, и что в глубине преисподней зазванные ею.

\chapter{10}

\par 1 Притчи Соломона. Сын мудрый радует отца, а сын глупый--огорчение для его матери.
\par 2 Не доставляют пользы сокровища неправедные, правда же избавляет от смерти.
\par 3 Не допустит Господь терпеть голод душе праведного, стяжание же нечестивых исторгнет.
\par 4 Ленивая рука делает бедным, а рука прилежных обогащает.
\par 5 Собирающий во время лета--сын разумный, спящий же во время жатвы--сын беспутный.
\par 6 Благословения--на голове праведника, уста же беззаконных заградит насилие.
\par 7 Память праведника пребудет благословенна, а имя нечестивых омерзеет.
\par 8 Мудрый сердцем принимает заповеди, а глупый устами преткнется.
\par 9 Кто ходит в непорочности, тот ходит безопасно; а кто превращает пути свои, тот будет наказан.
\par 10 Кто мигает глазами, тот причиняет досаду, а глупый устами преткнется.
\par 11 Уста праведника--источник жизни, уста же беззаконных заградит насилие.
\par 12 Ненависть возбуждает раздоры, но любовь покрывает все грехи.
\par 13 В устах разумного находится мудрость, но на теле глупого--розга.
\par 14 Мудрые сберегают знание, но уста глупого--близкая погибель.
\par 15 Имущество богатого--крепкий город его, беда для бедных--скудость их.
\par 16 Труды праведного--к жизни, успех нечестивого--ко греху.
\par 17 Кто хранит наставление, тот на пути к жизни; а отвергающий обличение--блуждает.
\par 18 Кто скрывает ненависть, у того уста лживые; и кто разглашает клевету, тот глуп.
\par 19 При многословии не миновать греха, а сдерживающий уста свои--разумен.
\par 20 Отборное серебро--язык праведного, сердце же нечестивых--ничтожество.
\par 21 Уста праведного пасут многих, а глупые умирают от недостатка разума.
\par 22 Благословение Господне--оно обогащает и печали с собою не приносит.
\par 23 Для глупого преступное деяние как бы забава, а человеку разумному свойственна мудрость.
\par 24 Чего страшится нечестивый, то и постигнет его, а желание праведников исполнится.
\par 25 Как проносится вихрь, [так] нет более нечестивого; а праведник--на вечном основании.
\par 26 Что уксус для зубов и дым для глаз, то ленивый для посылающих его.
\par 27 Страх Господень прибавляет дней, лета же нечестивых сократятся.
\par 28 Ожидание праведников--радость, а надежда нечестивых погибнет.
\par 29 Путь Господень--твердыня для непорочного и страх для делающих беззаконие.
\par 30 Праведник во веки не поколеблется, нечестивые же не поживут на земле.
\par 31 Уста праведника источают мудрость, а язык зловредный отсечется.
\par 32 Уста праведного знают благоприятное, а уста нечестивых--развращенное.

\chapter{11}

\par 1 Неверные весы--мерзость пред Господом, но правильный вес угоден Ему.
\par 2 Придет гордость, придет и посрамление; но со смиренными--мудрость.
\par 3 Непорочность прямодушных будет руководить их, а лукавство коварных погубит их.
\par 4 Не поможет богатство в день гнева, правда же спасет от смерти.
\par 5 Правда непорочного уравнивает путь его, а нечестивый падет от нечестия своего.
\par 6 Правда прямодушных спасет их, а беззаконники будут уловлены беззаконием своим.
\par 7 Со смертью человека нечестивого исчезает надежда, и ожидание беззаконных погибает.
\par 8 Праведник спасается от беды, а вместо него попадает [в нее] нечестивый.
\par 9 Устами лицемер губит ближнего своего, но праведники прозорливостью спасаются.
\par 10 При благоденствии праведников веселится город, и при погибели нечестивых [бывает] торжество.
\par 11 Благословением праведных возвышается город, а устами нечестивых разрушается.
\par 12 Скудоумный высказывает презрение к ближнему своему; но разумный человек молчит.
\par 13 Кто ходит переносчиком, тот открывает тайну; но верный человек таит дело.
\par 14 При недостатке попечения падает народ, а при многих советниках благоденствует.
\par 15 Зло причиняет себе, кто ручается за постороннего; а кто ненавидит ручательство, тот безопасен.
\par 16 Благонравная жена приобретает славу, а трудолюбивые приобретают богатство.
\par 17 Человек милосердый благотворит душе своей, а жестокосердый разрушает плоть свою.
\par 18 Нечестивый делает дело ненадежное, а сеющему правду--награда верная.
\par 19 Праведность [ведет] к жизни, а стремящийся к злу [стремится] к смерти своей.
\par 20 Мерзость пред Господом--коварные сердцем; но благоугодны Ему непорочные в пути.
\par 21 Можно поручиться, что порочный не останется ненаказанным; семя же праведных спасется.
\par 22 Что золотое кольцо в носу у свиньи, то женщина красивая и--безрассудная.
\par 23 Желание праведных [есть] одно добро, ожидание нечестивых--гнев.
\par 24 Иной сыплет щедро, и [ему] еще прибавляется; а другой сверх меры бережлив, и однако же беднеет.
\par 25 Благотворительная душа будет насыщена, и кто напояет [других], тот и сам напоен будет.
\par 26 Кто удерживает у себя хлеб, того клянет народ; а на голове продающего--благословение.
\par 27 Кто стремится к добру, тот ищет благоволения; а кто ищет зла, к тому оно и приходит.
\par 28 Надеющийся на богатство свое упадет; а праведники, как лист, будут зеленеть.
\par 29 Расстроивающий дом свой получит в удел ветер, и глупый будет рабом мудрого сердцем.
\par 30 Плод праведника--древо жизни, и мудрый привлекает души.
\par 31 Так праведнику воздается на земле, тем паче нечестивому и грешнику.

\chapter{12}

\par 1 Кто любит наставление, тот любит знание; а кто ненавидит обличение, тот невежда.
\par 2 Добрый приобретает благоволение от Господа; а человека коварного Он осудит.
\par 3 Не утвердит себя человек беззаконием; корень же праведников неподвижен.
\par 4 Добродетельная жена--венец для мужа своего; а позорная--как гниль в костях его.
\par 5 Промышления праведных--правда, а замыслы нечестивых--коварство.
\par 6 Речи нечестивых--засада для пролития крови, уста же праведных спасают их.
\par 7 Коснись нечестивых несчастие--и нет их, а дом праведных стоит.
\par 8 Хвалят человека по мере разума его, а развращенный сердцем будет в презрении.
\par 9 Лучше простой, но работающий на себя, нежели выдающий себя за знатного, но нуждающийся в хлебе.
\par 10 Праведный печется и о жизни скота своего, сердце же нечестивых жестоко.
\par 11 Кто возделывает землю свою, тот будет насыщаться хлебом; а кто идет по следам празднолюбцев, тот скудоумен.
\par 12 Нечестивый желает уловить в сеть зла; но корень праведных тверд.
\par 13 Нечестивый уловляется грехами уст своих; но праведник выйдет из беды.
\par 14 От плода уст [своих] человек насыщается добром, и воздаяние человеку--по делам рук его.
\par 15 Путь глупого прямой в его глазах; но кто слушает совета, тот мудр.
\par 16 У глупого тотчас же выкажется гнев его, а благоразумный скрывает оскорбление.
\par 17 Кто говорит то, что знает, тот говорит правду; а у свидетеля ложного--обман.
\par 18 Иной пустослов уязвляет как мечом, а язык мудрых--врачует.
\par 19 Уста правдивые вечно пребывают, а лживый язык--только на мгновение.
\par 20 Коварство--в сердце злоумышленников, радость--у миротворцев.
\par 21 Не приключится праведнику никакого зла, нечестивые же будут преисполнены зол.
\par 22 Мерзость пред Господом--уста лживые, а говорящие истину благоугодны Ему.
\par 23 Человек рассудительный скрывает знание, а сердце глупых высказывает глупость.
\par 24 Рука прилежных будет господствовать, а ленивая будет под данью.
\par 25 Тоска на сердце человека подавляет его, а доброе слово развеселяет его.
\par 26 Праведник указывает ближнему своему путь, а путь нечестивых вводит их в заблуждение.
\par 27 Ленивый не жарит своей дичи; а имущество человека прилежного многоценно.
\par 28 На пути правды--жизнь, и на стезе ее нет смерти.

\chapter{13}

\par 1 Мудрый сын [слушает] наставление отца, а буйный не слушает обличения.
\par 2 От плода уст [своих] человек вкусит добро, душа же законопреступников--зло.
\par 3 Кто хранит уста свои, тот бережет душу свою; а кто широко раскрывает свой рот, тому беда.
\par 4 Душа ленивого желает, но тщетно; а душа прилежных насытится.
\par 5 Праведник ненавидит ложное слово, а нечестивый срамит и бесчестит [себя].
\par 6 Правда хранит непорочного в пути, а нечестие губит грешника.
\par 7 Иной выдает себя за богатого, а у него ничего нет; другой выдает себя за бедного, а у него богатства много.
\par 8 Богатством своим человек выкупает жизнь [свою], а бедный и угрозы не слышит.
\par 9 Свет праведных весело горит, светильник же нечестивых угасает.
\par 10 От высокомерия происходит раздор, а у советующихся--мудрость.
\par 11 Богатство от суетности истощается, а собирающий трудами умножает его.
\par 12 Надежда, долго не сбывающаяся, томит сердце, а исполнившееся желание--[как] древо жизни.
\par 13 Кто пренебрегает словом, тот причиняет вред себе; а кто боится заповеди, тому воздается.
\par 14 Учение мудрого--источник жизни, удаляющий от сетей смерти.
\par 15 Добрый разум доставляет приятность, путь же беззаконных жесток.
\par 16 Всякий благоразумный действует с знанием, а глупый выставляет напоказ глупость.
\par 17 Худой посол попадает в беду, а верный посланник--спасение.
\par 18 Нищета и посрамление отвергающему учение; а кто соблюдает наставление, будет в чести.
\par 19 Желание исполнившееся--приятно для души; но несносно для глупых уклоняться от зла.
\par 20 Общающийся с мудрыми будет мудр, а кто дружит с глупыми, развратится.
\par 21 Грешников преследует зло, а праведникам воздается добром.
\par 22 Добрый оставляет наследство [и] внукам, а богатство грешника сберегается для праведного.
\par 23 Много хлеба [бывает] и на ниве бедных; но некоторые гибнут от беспорядка.
\par 24 Кто жалеет розги своей, тот ненавидит сына; а кто любит, тот с детства наказывает его.
\par 25 Праведник ест до сытости, а чрево беззаконных терпит лишение.

\chapter{14}

\par 1 Мудрая жена устроит дом свой, а глупая разрушит его своими руками.
\par 2 Идущий прямым путем боится Господа; но чьи пути кривы, тот небрежет о Нем.
\par 3 В устах глупого--бич гордости; уста же мудрых охраняют их.
\par 4 Где нет волов, [там] ясли пусты; а много прибыли от силы волов.
\par 5 Верный свидетель не лжет, а свидетель ложный наговорит много лжи.
\par 6 Распутный ищет мудрости, и не находит; а для разумного знание легко.
\par 7 Отойди от человека глупого, у которого ты не замечаешь разумных уст.
\par 8 Мудрость разумного--знание пути своего, глупость же безрассудных--заблуждение.
\par 9 Глупые смеются над грехом, а посреди праведных--благоволение.
\par 10 Сердце знает горе души своей, и в радость его не вмешается чужой.
\par 11 Дом беззаконных разорится, а жилище праведных процветет.
\par 12 Есть пути, которые кажутся человеку прямыми; но конец их--путь к смерти.
\par 13 И при смехе [иногда] болит сердце, и концом радости бывает печаль.
\par 14 Человек с развращенным сердцем насытится от путей своих, и добрый--от своих.
\par 15 Глупый верит всякому слову, благоразумный же внимателен к путям своим.
\par 16 Мудрый боится и удаляется от зла, а глупый раздражителен и самонадеян.
\par 17 Вспыльчивый может сделать глупость; но человек, умышленно делающий зло, ненавистен.
\par 18 Невежды получают в удел себе глупость, а благоразумные увенчаются знанием.
\par 19 Преклонятся злые пред добрыми и нечестивые--у ворот праведника.
\par 20 Бедный ненавидим бывает даже близким своим, а у богатого много друзей.
\par 21 Кто презирает ближнего своего, тот грешит; а кто милосерд к бедным, тот блажен.
\par 22 Не заблуждаются ли умышляющие зло? но милость и верность у благомыслящих.
\par 23 От всякого труда есть прибыль, а от пустословия только ущерб.
\par 24 Венец мудрых--богатство их, а глупость невежд глупость [и] [есть].
\par 25 Верный свидетель спасает души, а лживый наговорит много лжи.
\par 26 В страхе пред Господом--надежда твердая, и сынам Своим Он прибежище.
\par 27 Страх Господень--источник жизни, удаляющий от сетей смерти.
\par 28 Во множестве народа--величие царя, а при малолюдстве народа беда государю.
\par 29 У терпеливого человека много разума, а раздражительный выказывает глупость.
\par 30 Кроткое сердце--жизнь для тела, а зависть--гниль для костей.
\par 31 Кто теснит бедного, тот хулит Творца его; чтущий же Его благотворит нуждающемуся.
\par 32 За зло свое нечестивый будет отвергнут, а праведный и при смерти своей имеет надежду.
\par 33 Мудрость почиет в сердце разумного, и среди глупых дает знать о себе.
\par 34 Праведность возвышает народ, а беззаконие--бесчестие народов.
\par 35 Благоволение царя--к рабу разумному, а гнев его--против того, кто позорит его.

\chapter{15}

\par 1 Кроткий ответ отвращает гнев, а оскорбительное слово возбуждает ярость.
\par 2 Язык мудрых сообщает добрые знания, а уста глупых изрыгают глупость.
\par 3 На всяком месте очи Господни: они видят злых и добрых.
\par 4 Кроткий язык--древо жизни, но необузданный--сокрушение духа.
\par 5 Глупый пренебрегает наставлением отца своего; а кто внимает обличениям, тот благоразумен.
\par 6 В доме праведника--обилие сокровищ, а в прибытке нечестивого--расстройство.
\par 7 Уста мудрых распространяют знание, а сердце глупых не так.
\par 8 Жертва нечестивых--мерзость пред Господом, а молитва праведных благоугодна Ему.
\par 9 Мерзость пред Господом--путь нечестивого, а идущего путем правды Он любит.
\par 10 Злое наказание--уклоняющемуся от пути, и ненавидящий обличение погибнет.
\par 11 Преисподняя и Аваддон [открыты] пред Господом, тем более сердца сынов человеческих.
\par 12 Не любит распутный обличающих его, и к мудрым не пойдет.
\par 13 Веселое сердце делает лице веселым, а при сердечной скорби дух унывает.
\par 14 Сердце разумного ищет знания, уста же глупых питаются глупостью.
\par 15 Все дни несчастного печальны; а у кого сердце весело, у того всегда пир.
\par 16 Лучше немногое при страхе Господнем, нежели большое сокровище, и при нем тревога.
\par 17 Лучше блюдо зелени, и при нем любовь, нежели откормленный бык, и при нем ненависть.
\par 18 Вспыльчивый человек возбуждает раздор, а терпеливый утишает распрю.
\par 19 Путь ленивого--как терновый плетень, а путь праведных--гладкий.
\par 20 Мудрый сын радует отца, а глупый человек пренебрегает мать свою.
\par 21 Глупость--радость для малоумного, а человек разумный идет прямою дорогою.
\par 22 Без совета предприятия расстроятся, а при множестве советников они состоятся.
\par 23 Радость человеку в ответе уст его, и как хорошо слово вовремя!
\par 24 Путь жизни мудрого вверх, чтобы уклониться от преисподней внизу.
\par 25 Дом надменных разорит Господь, а межу вдовы укрепит.
\par 26 Мерзость пред Господом--помышления злых, слова же непорочных угодны Ему.
\par 27 Корыстолюбивый расстроит дом свой, а ненавидящий подарки будет жить.
\par 28 Сердце праведного обдумывает ответ, а уста нечестивых изрыгают зло.
\par 29 Далек Господь от нечестивых, а молитву праведников слышит.
\par 30 Светлый взгляд радует сердце, добрая весть утучняет кости.
\par 31 Ухо, внимательное к учению жизни, пребывает между мудрыми.
\par 32 Отвергающий наставление не радеет о своей душе; а кто внимает обличению, тот приобретает разум.
\par 33 Страх Господень научает мудрости, и славе предшествует смирение.

\chapter{16}

\par 1 Человеку [принадлежат] предположения сердца, но от Господа ответ языка.
\par 2 Все пути человека чисты в его глазах, но Господь взвешивает души.
\par 3 Предай Господу дела твои, и предприятия твои совершатся.
\par 4 Все сделал Господь ради Себя; и даже нечестивого [блюдет] на день бедствия.
\par 5 Мерзость пред Господом всякий надменный сердцем; можно поручиться, что он не останется ненаказанным.
\par 6 Милосердием и правдою очищается грех, и страх Господень отводит от зла.
\par 7 Когда Господу угодны пути человека, Он и врагов его примиряет с ним.
\par 8 Лучше немногое с правдою, нежели множество прибытков с неправдою.
\par 9 Сердце человека обдумывает свой путь, но Господь управляет шествием его.
\par 10 В устах царя--слово вдохновенное; уста его не должны погрешать на суде.
\par 11 Верные весы и весовые чаши--от Господа; от Него же все гири в суме.
\par 12 Мерзость для царей--дело беззаконное, потому что правдою утверждается престол.
\par 13 Приятны царю уста правдивые, и говорящего истину он любит.
\par 14 Царский гнев--вестник смерти; но мудрый человек умилостивит его.
\par 15 В светлом взоре царя--жизнь, и благоволение его--как облако с поздним дождем.
\par 16 Приобретение мудрости гораздо лучше золота, и приобретение разума предпочтительнее отборного серебра.
\par 17 Путь праведных--уклонение от зла: тот бережет душу свою, кто хранит путь свой.
\par 18 Погибели предшествует гордость, и падению--надменность.
\par 19 Лучше смиряться духом с кроткими, нежели разделять добычу с гордыми.
\par 20 Кто ведет дело разумно, тот найдет благо, и кто надеется на Господа, тот блажен.
\par 21 Мудрый сердцем прозовется благоразумным, и сладкая речь прибавит к учению.
\par 22 Разум для имеющих его--источник жизни, а ученость глупых--глупость.
\par 23 Сердце мудрого делает язык его мудрым и умножает знание в устах его.
\par 24 Приятная речь--сотовый мед, сладка для души и целебна для костей.
\par 25 Есть пути, которые кажутся человеку прямыми, но конец их путь к смерти.
\par 26 Трудящийся трудится для себя, потому что понуждает его [к] [тому] рот его.
\par 27 Человек лукавый замышляет зло, и на устах его как бы огонь палящий.
\par 28 Человек коварный сеет раздор, и наушник разлучает друзей.
\par 29 Человек неблагонамеренный развращает ближнего своего и ведет его на путь недобрый;
\par 30 прищуривает глаза свои, чтобы придумать коварство; закусывая себе губы, совершает злодейство.
\par 31 Венец славы--седина, которая находится на пути правды.
\par 32 Долготерпеливый лучше храброго, и владеющий собою [лучше] завоевателя города.
\par 33 В полу бросается жребий, но все решение его--от Господа.

\chapter{17}

\par 1 Лучше кусок сухого хлеба, и с ним мир, нежели дом, полный заколотого скота, с раздором.
\par 2 Разумный раб господствует над беспутным сыном и между братьями разделит наследство.
\par 3 Плавильня--для серебра, и горнило--для золота, а сердца испытывает Господь.
\par 4 Злодей внимает устам беззаконным, лжец слушается языка пагубного.
\par 5 Кто ругается над нищим, тот хулит Творца его; кто радуется несчастью, тот не останется ненаказанным.
\par 6 Венец стариков--сыновья сыновей, и слава детей--родители их.
\par 7 Неприлична глупому важная речь, тем паче знатному--уста лживые.
\par 8 Подарок--драгоценный камень в глазах владеющего им: куда ни обратится он, успеет.
\par 9 Прикрывающий проступок ищет любви; а кто снова напоминает о нем, тот удаляет друга.
\par 10 На разумного сильнее действует выговор, нежели на глупого сто ударов.
\par 11 Возмутитель ищет только зла; поэтому жестокий ангел будет послан против него.
\par 12 Лучше встретить человеку медведицу, лишенную детей, нежели глупца с его глупостью.
\par 13 Кто за добро воздает злом, от дома того не отойдет зло.
\par 14 Начало ссоры--как прорыв воды; оставь ссору прежде, нежели разгорелась она.
\par 15 Оправдывающий нечестивого и обвиняющий праведного--оба мерзость пред Господом.
\par 16 К чему сокровище в руках глупца? Для приобретения мудрости [у] [него] нет разума.
\par 17 Друг любит во всякое время и, как брат, явится во время несчастья.
\par 18 Человек малоумный дает руку и ручается за ближнего своего.
\par 19 Кто любит ссоры, любит грех, и кто высоко поднимает ворота свои, тот ищет падения.
\par 20 Коварное сердце не найдет добра, и лукавый язык попадет в беду.
\par 21 Родил кто глупого, --себе на горе, и отец глупого не порадуется.
\par 22 Веселое сердце благотворно, как врачевство, а унылый дух сушит кости.
\par 23 Нечестивый берет подарок из пазухи, чтобы извратить пути правосудия.
\par 24 Мудрость--пред лицем у разумного, а глаза глупца--на конце земли.
\par 25 Глупый сын--досада отцу своему и огорчение для матери своей.
\par 26 Нехорошо и обвинять правого, [и] бить вельмож за правду.
\par 27 Разумный воздержан в словах своих, и благоразумный хладнокровен.
\par 28 И глупец, когда молчит, может показаться мудрым, и затворяющий уста свои--благоразумным.

\chapter{18}

\par 1 Прихоти ищет своенравный, восстает против всего умного.
\par 2 Глупый не любит знания, а только бы выказать свой ум.
\par 3 С приходом нечестивого приходит и презрение, а с бесславием--поношение.
\par 4 Слова уст человеческих--глубокие воды; источник мудрости--струящийся поток.
\par 5 Нехорошо быть лицеприятным к нечестивому, чтобы ниспровергнуть праведного на суде.
\par 6 Уста глупого идут в ссору, и слова его вызывают побои.
\par 7 Язык глупого--гибель для него, и уста его--сеть для души его.
\par 8 Слова наушника--как лакомства, и они входят во внутренность чрева.
\par 9 Нерадивый в работе своей--брат расточителю.
\par 10 Имя Господа--крепкая башня: убегает в нее праведник--и безопасен.
\par 11 Имение богатого--крепкий город его, и как высокая ограда в его воображении.
\par 12 Перед падением возносится сердце человека, а смирение предшествует славе.
\par 13 Кто дает ответ не выслушав, тот глуп, и стыд ему.
\par 14 Дух человека переносит его немощи; а пораженный дух--кто может подкрепить его?
\par 15 Сердце разумного приобретает знание, и ухо мудрых ищет знания.
\par 16 Подарок у человека дает ему простор и до вельмож доведет его.
\par 17 Первый в тяжбе своей прав, но приходит соперник его и исследывает его.
\par 18 Жребий прекращает споры и решает между сильными.
\par 19 Озлобившийся брат [неприступнее] крепкого города, и ссоры подобны запорам замка.
\par 20 От плода уст человека наполняется чрево его; произведением уст своих он насыщается.
\par 21 Смерть и жизнь--во власти языка, и любящие его вкусят от плодов его.
\par 22 Кто нашел [добрую] жену, тот нашел благо и получил благодать от Господа.
\par 23 С мольбою говорит нищий, а богатый отвечает грубо.
\par 24 Кто хочет иметь друзей, тот и сам должен быть дружелюбным; и бывает друг, более привязанный, нежели брат.

\chapter{19}

\par 1 Лучше бедный, ходящий в своей непорочности, нежели [богатый] со лживыми устами, и притом глупый.
\par 2 Нехорошо душе без знания, и торопливый ногами оступится.
\par 3 Глупость человека извращает путь его, а сердце его негодует на Господа.
\par 4 Богатство прибавляет много друзей, а бедный оставляется и другом своим.
\par 5 Лжесвидетель не останется ненаказанным, и кто говорит ложь, не спасется.
\par 6 Многие заискивают у знатных, и всякий--друг человеку, делающему подарки.
\par 7 Бедного ненавидят все братья его, тем паче друзья его удаляются от него: гонится за ними, чтобы поговорить, но и этого нет.
\par 8 Кто приобретает разум, тот любит душу свою; кто наблюдает благоразумие, тот находит благо.
\par 9 Лжесвидетель не останется ненаказанным, и кто говорит ложь, погибнет.
\par 10 Неприлична глупцу пышность, тем паче рабу господство над князьями.
\par 11 Благоразумие делает человека медленным на гнев, и слава для него--быть снисходительным к проступкам.
\par 12 Гнев царя--как рев льва, а благоволение его--как роса на траву.
\par 13 Глупый сын--сокрушение для отца своего, и сварливая жена--сточная труба.
\par 14 Дом и имение--наследство от родителей, а разумная жена--от Господа.
\par 15 Леность погружает в сонливость, и нерадивая душа будет терпеть голод.
\par 16 Хранящий заповедь хранит душу свою, а нерадящий о путях своих погибнет.
\par 17 Благотворящий бедному дает взаймы Господу, и Он воздаст ему за благодеяние его.
\par 18 Наказывай сына своего, доколе есть надежда, и не возмущайся криком его.
\par 19 Гневливый пусть терпит наказание, потому что, если пощадишь [его], придется тебе еще больше наказывать его.
\par 20 Слушайся совета и принимай обличение, чтобы сделаться тебе впоследствии мудрым.
\par 21 Много замыслов в сердце человека, но состоится только определенное Господом.
\par 22 Радость человеку--благотворительность его, и бедный человек лучше, нежели лживый.
\par 23 Страх Господень [ведет] к жизни, и [кто имеет его], всегда будет доволен, и зло не постигнет его.
\par 24 Ленивый опускает руку свою в чашу, и не хочет донести ее до рта своего.
\par 25 Если ты накажешь кощунника, то и простой сделается благоразумным; и [если] обличишь разумного, то он поймет наставление.
\par 26 Разоряющий отца и выгоняющий мать--сын срамной и бесчестный.
\par 27 Перестань, сын мой, слушать внушения об уклонении от изречений разума.
\par 28 Лукавый свидетель издевается над судом, и уста беззаконных глотают неправду.
\par 29 Готовы для кощунствующих суды, и побои--на тело глупых.

\chapter{20}

\par 1 Вино--глумливо, сикера--буйна; и всякий, увлекающийся ими, неразумен.
\par 2 Гроза царя--как бы рев льва: кто раздражает его, тот грешит против самого себя.
\par 3 Честь для человека--отстать от ссоры; а всякий глупец задорен.
\par 4 Ленивец зимою не пашет: поищет летом--и нет ничего.
\par 5 Помыслы в сердце человека--глубокие воды, но человек разумный вычерпывает их.
\par 6 Многие хвалят человека за милосердие, но правдивого человека кто находит?
\par 7 Праведник ходит в своей непорочности: блаженны дети его после него!
\par 8 Царь, сидящий на престоле суда, разгоняет очами своими все злое.
\par 9 Кто может сказать: `я очистил мое сердце, я чист от греха моего?'
\par 10 Неодинаковые весы, неодинаковая мера, то и другое--мерзость пред Господом.
\par 11 Можно узнать даже отрока по занятиям его, чисто ли и правильно ли будет поведение его.
\par 12 Ухо слышащее и глаз видящий--и то и другое создал Господь.
\par 13 Не люби спать, чтобы тебе не обеднеть; держи открытыми глаза твои, и будешь досыта есть хлеб.
\par 14 `Дурно, дурно', говорит покупатель, а когда отойдет, хвалится.
\par 15 Есть золото и много жемчуга, но драгоценная утварь--уста разумные.
\par 16 Возьми платье его, так как он поручился за чужого; и за стороннего возьми от него залог.
\par 17 Сладок для человека хлеб, [приобретенный] неправдою; но после рот его наполнится дресвою.
\par 18 Предприятия получают твердость чрез совещание, и по совещании веди войну.
\par 19 Кто ходит переносчиком, тот открывает тайну; и кто широко раскрывает рот, с тем не сообщайся.
\par 20 Кто злословит отца своего и свою мать, того светильник погаснет среди глубокой тьмы.
\par 21 Наследство, поспешно захваченное вначале, не благословится впоследствии.
\par 22 Не говори: `я отплачу за зло'; предоставь Господу, и Он сохранит тебя.
\par 23 Мерзость пред Господом--неодинаковые гири, и неверные весы--не добро.
\par 24 От Господа направляются шаги человека; человеку же как узнать путь свой?
\par 25 Сеть для человека--поспешно давать обет, и после обета обдумывать.
\par 26 Мудрый царь вывеет нечестивых и обратит на них колесо.
\par 27 Светильник Господень--дух человека, испытывающий все глубины сердца.
\par 28 Милость и истина охраняют царя, и милостью он поддерживает престол свой.
\par 29 Слава юношей--сила их, а украшение стариков--седина.
\par 30 Раны от побоев--врачевство против зла, и удары, проникающие во внутренности чрева.

\chapter{21}

\par 1 Сердце царя--в руке Господа, как потоки вод: куда захочет, Он направляет его.
\par 2 Всякий путь человека прям в глазах его; но Господь взвешивает сердца.
\par 3 Соблюдение правды и правосудия более угодно Господу, нежели жертва.
\par 4 Гордость очей и надменность сердца, отличающие нечестивых, --грех.
\par 5 Помышления прилежного стремятся к изобилию, а всякий торопливый терпит лишение.
\par 6 Приобретение сокровища лживым языком--мимолетное дуновение ищущих смерти.
\par 7 Насилие нечестивых обрушится на них, потому что они отреклись соблюдать правду.
\par 8 Превратен путь человека развращенного; а кто чист, того действие прямо.
\par 9 Лучше жить в углу на кровле, нежели со сварливою женою в пространном доме.
\par 10 Душа нечестивого желает зла: не найдет милости в глазах его и друг его.
\par 11 Когда наказывается кощунник, простой делается мудрым; и когда вразумляется мудрый, то он приобретает знание.
\par 12 Праведник наблюдает за домом нечестивого: как повергаются нечестивые в несчастие.
\par 13 Кто затыкает ухо свое от вопля бедного, тот и сам будет вопить, --и не будет услышан.
\par 14 Подарок тайный тушит гнев, и дар в пазуху--сильную ярость.
\par 15 Соблюдение правосудия--радость для праведника и страх для делающих зло.
\par 16 Человек, сбившийся с пути разума, водворится в собрании мертвецов.
\par 17 Кто любит веселье, обеднеет; а кто любит вино и тук, не разбогатеет.
\par 18 Выкупом будет за праведного нечестивый и за прямодушного--лукавый.
\par 19 Лучше жить в земле пустынной, нежели с женою сварливою и сердитою.
\par 20 Вожделенное сокровище и тук--в доме мудрого; а глупый человек расточает их.
\par 21 Соблюдающий правду и милость найдет жизнь, правду и славу.
\par 22 Мудрый входит в город сильных и ниспровергает крепость, на которую они надеялись.
\par 23 Кто хранит уста свои и язык свой, тот хранит от бед душу свою.
\par 24 Надменный злодей--кощунник имя ему--действует в пылу гордости.
\par 25 Алчба ленивца убьет его, потому что руки его отказываются работать;
\par 26 всякий день он сильно алчет, а праведник дает и не жалеет.
\par 27 Жертва нечестивых--мерзость, особенно когда с лукавством приносят ее.
\par 28 Лжесвидетель погибнет; а человек, который говорит, что знает, будет говорить всегда.
\par 29 Человек нечестивый дерзок лицом своим, а праведный держит прямо путь свой.
\par 30 Нет мудрости, и нет разума, и нет совета вопреки Господу.
\par 31 Коня приготовляют на день битвы, но победа--от Господа.

\chapter{22}

\par 1 Доброе имя лучше большого богатства, и добрая слава лучше серебра и золота.
\par 2 Богатый и бедный встречаются друг с другом: того и другого создал Господь.
\par 3 Благоразумный видит беду, и укрывается; а неопытные идут вперед, и наказываются.
\par 4 За смирением следует страх Господень, богатство и слава и жизнь.
\par 5 Терны и сети на пути коварного; кто бережет душу свою, удались от них.
\par 6 Наставь юношу при начале пути его: он не уклонится от него, когда и состарится.
\par 7 Богатый господствует над бедным, и должник [делается] рабом заимодавца.
\par 8 Сеющий неправду пожнет беду, и трости гнева его не станет.
\par 9 Милосердый будет благословляем, потому что дает бедному от хлеба своего.
\par 10 Прогони кощунника, и удалится раздор, и прекратятся ссора и брань.
\par 11 Кто любит чистоту сердца, у того приятность на устах, тому царь--друг.
\par 12 Очи Господа охраняют знание, а слова законопреступника Он ниспровергает.
\par 13 Ленивец говорит: `лев на улице! посреди площади убьют меня!'
\par 14 Глубокая пропасть--уста блудниц: на кого прогневается Господь, тот упадет туда.
\par 15 Глупость привязалась к сердцу юноши, но исправительная розга удалит ее от него.
\par 16 Кто обижает бедного, чтобы умножить свое богатство, и кто дает богатому, тот обеднеет.
\par 17 Приклони ухо твое, и слушай слова мудрых, и сердце твое обрати к моему знанию;
\par 18 потому что утешительно будет, если ты будешь хранить их в сердце твоем, и они будут также в устах твоих.
\par 19 Чтобы упование твое было на Господа, я учу тебя и сегодня, и ты [помни].
\par 20 Не писал ли я тебе трижды в советах и наставлении,
\par 21 чтобы научить тебя точным словам истины, дабы ты мог передавать слова истины посылающим тебя?
\par 22 Не будь грабителем бедного, потому что он беден, и не притесняй несчастного у ворот,
\par 23 потому что Господь вступится в дело их и исхитит душу у грабителей их.
\par 24 Не дружись с гневливым и не сообщайся с человеком вспыльчивым,
\par 25 чтобы не научиться путям его и не навлечь петли на душу твою.
\par 26 Не будь из тех, которые дают руки и поручаются за долги:
\par 27 если тебе нечем заплатить, то для чего доводить себя, чтобы взяли постель твою из-под тебя?
\par 28 Не передвигай межи давней, которую провели отцы твои.
\par 29 Видел ли ты человека проворного в своем деле? Он будет стоять перед царями, он не будет стоять перед простыми.

\chapter{23}

\par 1 Когда сядешь вкушать пищу с властелином, то тщательно наблюдай, что перед тобою,
\par 2 и поставь преграду в гортани твоей, если ты алчен.
\par 3 Не прельщайся лакомыми яствами его; это--обманчивая пища.
\par 4 Не заботься о том, чтобы нажить богатство; оставь такие мысли твои.
\par 5 Устремишь глаза твои на него, и--его уже нет; потому что оно сделает себе крылья и, как орел, улетит к небу.
\par 6 Не вкушай пищи у человека завистливого и не прельщайся лакомыми яствами его;
\par 7 потому что, каковы мысли в душе его, таков и он; `ешь и пей', говорит он тебе, а сердце его не с тобою.
\par 8 Кусок, который ты съел, изблюешь, и добрые слова твои ты потратишь напрасно.
\par 9 В уши глупого не говори, потому что он презрит разумные слова твои.
\par 10 Не передвигай межи давней и на поля сирот не заходи,
\par 11 потому что Защитник их силен; Он вступится в дело их с тобою.
\par 12 Приложи сердце твое к учению и уши твои--к умным словам.
\par 13 Не оставляй юноши без наказания: если накажешь его розгою, он не умрет;
\par 14 ты накажешь его розгою и спасешь душу его от преисподней.
\par 15 Сын мой! если сердце твое будет мудро, то порадуется и мое сердце;
\par 16 и внутренности мои будут радоваться, когда уста твои будут говорить правое.
\par 17 Да не завидует сердце твое грешникам, но да пребудет оно во все дни в страхе Господнем;
\par 18 потому что есть будущность, и надежда твоя не потеряна.
\par 19 Слушай, сын мой, и будь мудр, и направляй сердце твое на прямой путь.
\par 20 Не будь между упивающимися вином, между пресыщающимися мясом:
\par 21 потому что пьяница и пресыщающийся обеднеют, и сонливость оденет в рубище.
\par 22 Слушайся отца твоего: он родил тебя; и не пренебрегай матери твоей, когда она и состарится.
\par 23 Купи истину и не продавай мудрости и учения и разума.
\par 24 Торжествует отец праведника, и родивший мудрого радуется о нем.
\par 25 Да веселится отец твой и да торжествует мать твоя, родившая тебя.
\par 26 Сын мой! отдай сердце твое мне, и глаза твои да наблюдают пути мои,
\par 27 потому что блудница--глубокая пропасть, и чужая жена--тесный колодезь;
\par 28 она, как разбойник, сидит в засаде и умножает между людьми законопреступников.
\par 29 У кого вой? у кого стон? у кого ссоры? у кого горе? у кого раны без причины? у кого багровые глаза?
\par 30 У тех, которые долго сидят за вином, которые приходят отыскивать [вина] приправленного.
\par 31 Не смотри на вино, как оно краснеет, как оно искрится в чаше, как оно ухаживается ровно:
\par 32 впоследствии, как змей, оно укусит, и ужалит, как аспид;
\par 33 глаза твои будут смотреть на чужих жен, и сердце твое заговорит развратное,
\par 34 и ты будешь, как спящий среди моря и как спящий на верху мачты.
\par 35 [И скажешь]: `били меня, мне не было больно; толкали меня, я не чувствовал. Когда проснусь, опять буду искать того же'.

\chapter{24}

\par 1 Не ревнуй злым людям и не желай быть с ними,
\par 2 потому что о насилии помышляет сердце их, и о злом говорят уста их.
\par 3 Мудростью устрояется дом и разумом утверждается,
\par 4 и с уменьем внутренности его наполняются всяким драгоценным и прекрасным имуществом.
\par 5 Человек мудрый силен, и человек разумный укрепляет силу свою.
\par 6 Поэтому с обдуманностью веди войну твою, и успех [будет] при множестве совещаний.
\par 7 Для глупого слишком высока мудрость; у ворот не откроет он уст своих.
\par 8 Кто замышляет сделать зло, того называют злоумышленником.
\par 9 Помысл глупости--грех, и кощунник--мерзость для людей.
\par 10 Если ты в день бедствия оказался слабым, то бедна сила твоя.
\par 11 Спасай взятых на смерть, и неужели откажешься от обреченных на убиение?
\par 12 Скажешь ли: `вот, мы не знали этого'? А Испытующий сердца разве не знает? Наблюдающий над душею твоею знает это, и воздаст человеку по делам его.
\par 13 Ешь, сын мой, мед, потому что он приятен, и сот, который сладок для гортани твоей:
\par 14 таково и познание мудрости для души твоей. Если ты нашел [ее], то есть будущность, и надежда твоя не потеряна.
\par 15 Не злоумышляй, нечестивый, против жилища праведника, не опустошай места покоя его,
\par 16 ибо семь раз упадет праведник, и встанет; а нечестивые впадут в погибель.
\par 17 Не радуйся, когда упадет враг твой, и да не веселится сердце твое, когда он споткнется.
\par 18 Иначе, увидит Господь, и неугодно будет это в очах Его, и Он отвратит от него гнев Свой.
\par 19 Не негодуй на злодеев и не завидуй нечестивым,
\par 20 потому что злой не имеет будущности, --светильник нечестивых угаснет.
\par 21 Бойся, сын мой, Господа и царя; с мятежниками не сообщайся,
\par 22 потому что внезапно придет погибель от них, и беду от них обоих кто предузнает?
\par 23 Сказано также мудрыми: иметь лицеприятие на суде--нехорошо.
\par 24 Кто говорит виновному: `ты прав', того будут проклинать народы, того будут ненавидеть племена;
\par 25 а обличающие будут любимы, и на них придет благословение.
\par 26 В уста целует, кто отвечает словами верными.
\par 27 Соверши дела твои вне дома, окончи их на поле твоем, и потом устрояй и дом твой.
\par 28 Не будь лжесвидетелем на ближнего твоего: к чему тебе обманывать устами твоими?
\par 29 Не говори: `как он поступил со мною, так и я поступлю с ним, воздам человеку по делам его'.
\par 30 Проходил я мимо поля человека ленивого и мимо виноградника человека скудоумного:
\par 31 и вот, все это заросло терном, поверхность его покрылась крапивою, и каменная ограда его обрушилась.
\par 32 И посмотрел я, и обратил сердце мое, и посмотрел и получил урок:
\par 33 `немного поспишь, немного подремлешь, немного, сложив руки, полежишь, --
\par 34 и придет, [как] прохожий, бедность твоя, и нужда твоя--как человек вооруженный'.

\chapter{25}

\par 1 И это притчи Соломона, которые собрали мужи Езекии, царя Иудейского.
\par 2 Слава Божия--облекать тайною дело, а слава царей--исследывать дело.
\par 3 Как небо в высоте и земля в глубине, так сердце царей--неисследимо.
\par 4 Отдели примесь от серебра, и выйдет у серебряника сосуд:
\par 5 удали неправедного от царя, и престол его утвердится правдою.
\par 6 Не величайся пред лицем царя, и на месте великих не становись;
\par 7 потому что лучше, когда скажут тебе: `пойди сюда повыше', нежели когда понизят тебя пред знатным, которого видели глаза твои.
\par 8 Не вступай поспешно в тяжбу: иначе что будешь делать при окончании, когда соперник твой осрамит тебя?
\par 9 Веди тяжбу с соперником твоим, но тайны другого не открывай,
\par 10 дабы не укорил тебя услышавший это, и тогда бесчестие твое не отойдет от тебя.
\par 11 Золотые яблоки в серебряных прозрачных сосудах--слово, сказанное прилично.
\par 12 Золотая серьга и украшение из чистого золота--мудрый обличитель для внимательного уха.
\par 13 Что прохлада от снега во время жатвы, то верный посол для посылающего его: он доставляет душе господина своего отраду.
\par 14 Что тучи и ветры без дождя, то человек, хвастающий ложными подарками.
\par 15 Кротостью склоняется к милости вельможа, и мягкий язык переламывает кость.
\par 16 Нашел ты мед, --ешь, сколько тебе потребно, чтобы не пресытиться им и не изблевать его.
\par 17 Не учащай входить в дом друга твоего, чтобы он не наскучил тобою и не возненавидел тебя.
\par 18 Что молот и меч и острая стрела, то человек, произносящий ложное свидетельство против ближнего своего.
\par 19 Что сломанный зуб и расслабленная нога, то надежда на ненадежного [человека] в день бедствия.
\par 20 Что снимающий с себя одежду в холодный день, что уксус на рану, то поющий песни печальному сердцу.
\par 21 Если голоден враг твой, накорми его хлебом; и если он жаждет, напой его водою:
\par 22 ибо, [делая сие], ты собираешь горящие угли на голову его, и Господь воздаст тебе.
\par 23 Северный ветер производит дождь, а тайный язык--недовольные лица.
\par 24 Лучше жить в углу на кровле, нежели со сварливою женою в пространном доме.
\par 25 Что холодная вода для истомленной жаждой души, то добрая весть из дальней страны.
\par 26 Что возмущенный источник и поврежденный родник, то праведник, падающий пред нечестивым.
\par 27 Как нехорошо есть много меду, так домогаться славы не есть слава.
\par 28 Что город разрушенный, без стен, то человек, не владеющий духом своим.

\chapter{26}

\par 1 Как снег летом и дождь во время жатвы, так честь неприлична глупому.
\par 2 Как воробей вспорхнет, как ласточка улетит, так незаслуженное проклятие не сбудется.
\par 3 Бич для коня, узда для осла, а палка для глупых.
\par 4 Не отвечай глупому по глупости его, чтобы и тебе не сделаться подобным ему;
\par 5 но отвечай глупому по глупости его, чтобы он не стал мудрецом в глазах своих.
\par 6 Подрезывает себе ноги, терпит неприятность тот, кто дает словесное поручение глупцу.
\par 7 Неровно поднимаются ноги у хромого, --и притча в устах глупцов.
\par 8 Что влагающий драгоценный камень в пращу, то воздающий глупому честь.
\par 9 Что [колючий] терн в руке пьяного, то притча в устах глупцов.
\par 10 Сильный делает все произвольно: и глупого награждает, и всякого прохожего награждает.
\par 11 Как пес возвращается на блевотину свою, так глупый повторяет глупость свою.
\par 12 Видал ли ты человека, мудрого в глазах его? На глупого больше надежды, нежели на него.
\par 13 Ленивец говорит: `лев на дороге! лев на площадях!'
\par 14 Дверь ворочается на крючьях своих, а ленивец на постели своей.
\par 15 Ленивец опускает руку свою в чашу, и ему тяжело донести ее до рта своего.
\par 16 Ленивец в глазах своих мудрее семерых, отвечающих обдуманно.
\par 17 Хватает пса за уши, кто, проходя мимо, вмешивается в чужую ссору.
\par 18 Как притворяющийся помешанным бросает огонь, стрелы и смерть,
\par 19 так--человек, который коварно вредит другу своему и потом говорит: `я только пошутил'.
\par 20 Где нет больше дров, огонь погасает, и где нет наушника, раздор утихает.
\par 21 Уголь--для жара и дрова--для огня, а человек сварливый--для разжжения ссоры.
\par 22 Слова наушника--как лакомства, и они входят во внутренность чрева.
\par 23 Что нечистым серебром обложенный глиняный сосуд, то пламенные уста и сердце злобное.
\par 24 Устами своими притворяется враг, а в сердце своем замышляет коварство.
\par 25 Если он говорит и нежным голосом, не верь ему, потому что семь мерзостей в сердце его.
\par 26 Если ненависть прикрывается наедине, то откроется злоба его в народном собрании.
\par 27 Кто роет яму, тот упадет в нее, и кто покатит вверх камень, к тому он воротится.
\par 28 Лживый язык ненавидит уязвляемых им, и льстивые уста готовят падение.

\chapter{27}

\par 1 Не хвались завтрашним днем, потому что не знаешь, что родит тот день.
\par 2 Пусть хвалит тебя другой, а не уста твои, --чужой, а не язык твой.
\par 3 Тяжел камень, весок и песок; но гнев глупца тяжелее их обоих.
\par 4 Жесток гнев, неукротима ярость; но кто устоит против ревности?
\par 5 Лучше открытое обличение, нежели скрытая любовь.
\par 6 Искренни укоризны от любящего, и лживы поцелуи ненавидящего.
\par 7 Сытая душа попирает и сот, а голодной душе все горькое сладко.
\par 8 Как птица, покинувшая гнездо свое, так человек, покинувший место свое.
\par 9 Масть и курение радуют сердце; так сладок [всякому] друг сердечным советом своим.
\par 10 Не покидай друга твоего и друга отца твоего, и в дом брата твоего не ходи в день несчастья твоего: лучше сосед вблизи, нежели брат вдали.
\par 11 Будь мудр, сын мой, и радуй сердце мое; и я буду иметь, что отвечать злословящему меня.
\par 12 Благоразумный видит беду и укрывается; а неопытные идут вперед [и] наказываются.
\par 13 Возьми у него платье его, потому что он поручился за чужого, и за стороннего возьми от него залог.
\par 14 Кто громко хвалит друга своего с раннего утра, того сочтут за злословящего.
\par 15 Непрестанная капель в дождливый день и сварливая жена--равны:
\par 16 кто хочет скрыть ее, тот хочет скрыть ветер и масть в правой руке своей, дающую знать о себе.
\par 17 Железо железо острит, и человек изощряет взгляд друга своего.
\par 18 Кто стережет смоковницу, тот будет есть плоды ее; и кто бережет господина своего, тот будет в чести.
\par 19 Как в воде лицо--к лицу, так сердце человека--к человеку.
\par 20 Преисподняя и Аваддон--ненасытимы; так ненасытимы и глаза человеческие.
\par 21 Что плавильня--для серебра, горнило--для золота, то для человека уста, которые хвалят его.
\par 22 Толки глупого в ступе пестом вместе с зерном, не отделится от него глупость его.
\par 23 Хорошо наблюдай за скотом твоим, имей попечение о стадах;
\par 24 потому что [богатство] не навек, да и власть разве из рода в род?
\par 25 Прозябает трава, и является зелень, и собирают горные травы.
\par 26 Овцы--на одежду тебе, и козлы--на покупку поля.
\par 27 И довольно козьего молока в пищу тебе, в пищу домашним твоим и на продовольствие служанкам твоим.

\chapter{28}

\par 1 Нечестивый бежит, когда никто не гонится [за ним]; а праведник смел, как лев.
\par 2 Когда страна отступит от закона, тогда много в ней начальников; а при разумном и знающем муже она долговечна.
\par 3 Человек бедный и притесняющий слабых [то же, что] проливной дождь, смывающий хлеб.
\par 4 Отступники от закона хвалят нечестивых, а соблюдающие закон негодуют на них.
\par 5 Злые люди не разумеют справедливости, а ищущие Господа разумеют все.
\par 6 Лучше бедный, ходящий в своей непорочности, нежели тот, кто извращает пути свои, хотя он и богат.
\par 7 Хранящий закон--сын разумный, а знающийся с расточителями срамит отца своего.
\par 8 Умножающий имение свое ростом и лихвою соберет его для благотворителя бедных.
\par 9 Кто отклоняет ухо свое от слушания закона, того и молитва--мерзость.
\par 10 Совращающий праведных на путь зла сам упадет в свою яму, а непорочные наследуют добро.
\par 11 Человек богатый--мудрец в глазах своих, но умный бедняк обличит его.
\par 12 Когда торжествуют праведники, великая слава, но когда возвышаются нечестивые, люди укрываются.
\par 13 Скрывающий свои преступления не будет иметь успеха; а кто сознается и оставляет их, тот будет помилован.
\par 14 Блажен человек, который всегда пребывает в благоговении; а кто ожесточает сердце свое, тот попадет в беду.
\par 15 Как рыкающий лев и голодный медведь, так нечестивый властелин над бедным народом.
\par 16 Неразумный правитель много делает притеснений, а ненавидящий корысть продолжит дни.
\par 17 Человек, виновный в пролитии человеческой крови, будет бегать до могилы, чтобы кто не схватил его.
\par 18 Кто ходит непорочно, то будет невредим; а ходящий кривыми путями упадет на одном из них.
\par 19 Кто возделывает землю свою, тот будет насыщаться хлебом, а кто подражает праздным, тот насытится нищетою.
\par 20 Верный человек богат благословениями, а кто спешит разбогатеть, тот не останется ненаказанным.
\par 21 Быть лицеприятным--нехорошо: такой человек и за кусок хлеба сделает неправду.
\par 22 Спешит к богатству завистливый человек, и не думает, что нищета постигнет его.
\par 23 Обличающий человека найдет после большую приязнь, нежели тот, кто льстит языком.
\par 24 Кто обкрадывает отца своего и мать свою и говорит: `это не грех', тот--сообщник грабителям.
\par 25 Надменный разжигает ссору, а надеющийся на Господа будет благоденствовать.
\par 26 Кто надеется на себя, тот глуп; а кто ходит в мудрости, тот будет цел.
\par 27 Дающий нищему не обеднеет; а кто закрывает глаза свои от него, на том много проклятий.
\par 28 Когда возвышаются нечестивые, люди укрываются, а когда они падают, умножаются праведники.

\chapter{29}

\par 1 Человек, который, будучи обличаем, ожесточает выю свою, внезапно сокрушится, и не будет [ему] исцеления.
\par 2 Когда умножаются праведники, веселится народ, а когда господствует нечестивый, народ стенает.
\par 3 Человек, любящий мудрость, радует отца своего; а кто знается с блудницами, тот расточает имение.
\par 4 Царь правосудием утверждает землю, а любящий подарки разоряет ее.
\par 5 Человек, льстящий другу своему, расстилает сеть ногам его.
\par 6 В грехе злого человека--сеть [для него], а праведник веселится и радуется.
\par 7 Праведник тщательно вникает в тяжбу бедных, а нечестивый не разбирает дела.
\par 8 Люди развратные возмущают город, а мудрые утишают мятеж.
\par 9 Умный человек, судясь с человеком глупым, сердится ли, смеется ли, --не имеет покоя.
\par 10 Кровожадные люди ненавидят непорочного, а праведные заботятся о его жизни.
\par 11 Глупый весь гнев свой изливает, а мудрый сдерживает его.
\par 12 Если правитель слушает ложные речи, то и все служащие у него нечестивы.
\par 13 Бедный и лихоимец встречаются друг с другом; но свет глазам того и другого дает Господь.
\par 14 Если царь судит бедных по правде, то престол его навсегда утвердится.
\par 15 Розга и обличение дают мудрость; но отрок, оставленный в небрежении, делает стыд своей матери.
\par 16 При умножении нечестивых умножается беззаконие; но праведники увидят падение их.
\par 17 Наказывай сына твоего, и он даст тебе покой, и доставит радость душе твоей.
\par 18 Без откровения свыше народ необуздан, а соблюдающий закон блажен.
\par 19 Словами не научится раб, потому что, хотя он понимает [их], но не слушается.
\par 20 Видал ли ты человека опрометчивого в словах своих? на глупого больше надежды, нежели на него.
\par 21 Если с детства воспитывать раба в неге, то впоследствии он захочет быть сыном.
\par 22 Человек гневливый заводит ссору, и вспыльчивый много грешит.
\par 23 Гордость человека унижает его, а смиренный духом приобретает честь.
\par 24 Кто делится с вором, тот ненавидит душу свою; слышит он проклятие, но не объявляет о том.
\par 25 Боязнь пред людьми ставит сеть; а надеющийся на Господа будет безопасен.
\par 26 Многие ищут [благосклонного] лица правителя, но судьба человека--от Господа.
\par 27 Мерзость для праведников--человек неправедный, и мерзость для нечестивого--идущий прямым путем.

\chapter{30}

\par 1 Слова Агура, сына Иакеева. Вдохновенные изречения, [которые] сказал этот человек Ифиилу, Ифиилу и Укалу:
\par 2 подлинно, я более невежда, нежели кто-либо из людей, и разума человеческого нет у меня,
\par 3 и не научился я мудрости, и познания святых не имею.
\par 4 Кто восходил на небо и нисходил? кто собрал ветер в пригоршни свои? кто завязал воду в одежду? кто поставил все пределы земли? какое имя ему? и какое имя сыну его? знаешь ли?
\par 5 Всякое слово Бога чисто; Он--щит уповающим на Него.
\par 6 Не прибавляй к словам Его, чтобы Он не обличил тебя, и ты не оказался лжецом.
\par 7 Двух вещей я прошу у Тебя, не откажи мне, прежде нежели я умру:
\par 8 суету и ложь удали от меня, нищеты и богатства не давай мне, питай меня насущным хлебом,
\par 9 дабы, пресытившись, я не отрекся [Тебя] и не сказал: `кто Господь?' и чтобы, обеднев, не стал красть и употреблять имя Бога моего всуе.
\par 10 Не злословь раба пред господином его, чтобы он не проклял тебя, и ты не остался виноватым.
\par 11 Есть род, который проклинает отца своего и не благословляет матери своей.
\par 12 Есть род, который чист в глазах своих, тогда как не омыт от нечистот своих.
\par 13 Есть род--о, как высокомерны глаза его, и как подняты ресницы его!
\par 14 Есть род, у которого зубы--мечи, и челюсти--ножи, чтобы пожирать бедных на земле и нищих между людьми.
\par 15 У ненасытимости две дочери: `давай, давай!' Вот три ненасытимых, и четыре, которые не скажут: `довольно!'
\par 16 Преисподняя и утроба бесплодная, земля, которая не насыщается водою, и огонь, который не говорит: `довольно!'
\par 17 Глаз, насмехающийся над отцом и пренебрегающий покорностью к матери, выклюют вороны дольные, и сожрут птенцы орлиные!
\par 18 Три вещи непостижимы для меня, и четырех я не понимаю:
\par 19 пути орла на небе, пути змея на скале, пути корабля среди моря и пути мужчины к девице.
\par 20 Таков путь и жены прелюбодейной; поела и обтерла рот свой, и говорит: `я ничего худого не сделала'.
\par 21 От трех трясется земля, четырех она не может носить:
\par 22 раба, когда он делается царем; глупого, когда он досыта ест хлеб;
\par 23 позорную женщину, когда она выходит замуж, и служанку, когда она занимает место госпожи своей.
\par 24 Вот четыре малых на земле, но они мудрее мудрых:
\par 25 муравьи--народ не сильный, но летом заготовляют пищу свою;
\par 26 горные мыши--народ слабый, но ставят домы свои на скале;
\par 27 у саранчи нет царя, но выступает вся она стройно;
\par 28 паук лапками цепляется, но бывает в царских чертогах.
\par 29 Вот трое имеют стройную походку, и четверо стройно выступают:
\par 30 лев, силач между зверями, не посторонится ни перед кем;
\par 31 конь и козел, и царь среди народа своего.
\par 32 Если ты в заносчивости своей сделал глупость и помыслил злое, то [положи] руку на уста;
\par 33 потому что, как сбивание молока производит масло, толчок в нос производит кровь, так и возбуждение гнева производит ссору.

\chapter{31}

\par 1 Слова Лемуила царя. Наставление, которое преподала ему мать его:
\par 2 что, сын мой? что, сын чрева моего? что, сын обетов моих?
\par 3 Не отдавай женщинам сил твоих, ни путей твоих губительницам царей.
\par 4 Не царям, Лемуил, не царям пить вино, и не князьям--сикеру,
\par 5 чтобы, напившись, они не забыли закона и не превратили суда всех угнетаемых.
\par 6 Дайте сикеру погибающему и вино огорченному душею;
\par 7 пусть он выпьет и забудет бедность свою и не вспомнит больше о своем страдании.
\par 8 Открывай уста твои за безгласного и для защиты всех сирот.
\par 9 Открывай уста твои для правосудия и для дела бедного и нищего.
\par 10 Кто найдет добродетельную жену? цена ее выше жемчугов;
\par 11 уверено в ней сердце мужа ее, и он не останется без прибытка;
\par 12 она воздает ему добром, а не злом, во все дни жизни своей.
\par 13 Добывает шерсть и лен, и с охотою работает своими руками.
\par 14 Она, как купеческие корабли, издалека добывает хлеб свой.
\par 15 Она встает еще ночью и раздает пищу в доме своем и урочное служанкам своим.
\par 16 Задумает она о поле, и приобретает его; от плодов рук своих насаждает виноградник.
\par 17 Препоясывает силою чресла свои и укрепляет мышцы свои.
\par 18 Она чувствует, что занятие ее хорошо, и--светильник ее не гаснет и ночью.
\par 19 Протягивает руки свои к прялке, и персты ее берутся за веретено.
\par 20 Длань свою она открывает бедному, и руку свою подает нуждающемуся.
\par 21 Не боится стужи для семьи своей, потому что вся семья ее одета в двойные одежды.
\par 22 Она делает себе ковры; виссон и пурпур--одежда ее.
\par 23 Муж ее известен у ворот, когда сидит со старейшинами земли.
\par 24 Она делает покрывала и продает, и поясы доставляет купцам Финикийским.
\par 25 Крепость и красота--одежда ее, и весело смотрит она на будущее.
\par 26 Уста свои открывает с мудростью, и кроткое наставление на языке ее.
\par 27 Она наблюдает за хозяйством в доме своем и не ест хлеба праздности.
\par 28 Встают дети и ублажают ее, --муж, и хвалит ее:
\par 29 `много было жен добродетельных, но ты превзошла всех их'.
\par 30 Миловидность обманчива и красота суетна; но жена, боящаяся Господа, достойна хвалы.
\par 31 Дайте ей от плода рук ее, и да прославят ее у ворот дела ее!


\end{document}