\begin{document}

\title{Proverbs}

Pro 1:1  Притчи Соломона, сына Давидова, царя Израильского,
Pro 1:2  чтобы познать мудрость и наставление, понять изречения разума;
Pro 1:3  усвоить правила благоразумия, правосудия, суда и правоты;
Pro 1:4  простым дать смышленость, юноше--знание и рассудительность;
Pro 1:5  послушает мудрый--и умножит познания, и разумный найдет мудрые советы;
Pro 1:6  чтобы разуметь притчу и замысловатую речь, слова мудрецов и загадки их.
Pro 1:7  Начало мудрости--страх Господень; глупцы только презирают мудрость и наставление.
Pro 1:8  Слушай, сын мой, наставление отца твоего и не отвергай завета матери твоей,
Pro 1:9  потому что это--прекрасный венок для головы твоей и украшение для шеи твоей.
Pro 1:10  Сын мой! если будут склонять тебя грешники, не соглашайся;
Pro 1:11  если будут говорить: `иди с нами, сделаем засаду для убийства, подстережем непорочного без вины,
Pro 1:12  живых проглотим их, как преисподняя, и--целых, как нисходящих в могилу;
Pro 1:13  наберем всякого драгоценного имущества, наполним домы наши добычею;
Pro 1:14  жребий твой ты будешь бросать вместе с нами, склад один будет у всех нас', --
Pro 1:15  сын мой! не ходи в путь с ними, удержи ногу твою от стези их,
Pro 1:16  потому что ноги их бегут ко злу и спешат на пролитие крови.
Pro 1:17  В глазах всех птиц напрасно расставляется сеть,
Pro 1:18  а делают засаду для их крови и подстерегают их души.
Pro 1:19  Таковы пути всякого, кто алчет чужого добра: оно отнимает жизнь у завладевшего им.
Pro 1:20  Премудрость возглашает на улице, на площадях возвышает голос свой,
Pro 1:21  в главных местах собраний проповедует, при входах в городские ворота говорит речь свою:
Pro 1:22  `доколе, невежды, будете любить невежество? [доколе] буйные будут услаждаться буйством? доколе глупцы будут ненавидеть знание?
Pro 1:23  Обратитесь к моему обличению: вот, я изолью на вас дух мой, возвещу вам слова мои.
Pro 1:24  Я звала, и вы не послушались; простирала руку мою, и не было внимающего;
Pro 1:25  и вы отвергли все мои советы, и обличений моих не приняли.
Pro 1:26  За то и я посмеюсь вашей погибели; порадуюсь, когда придет на вас ужас;
Pro 1:27  когда придет на вас ужас, как буря, и беда, как вихрь, принесется на вас; когда постигнет вас скорбь и теснота.
Pro 1:28  Тогда будут звать меня, и я не услышу; с утра будут искать меня, и не найдут меня.
Pro 1:29  За то, что они возненавидели знание и не избрали [для себя] страха Господня,
Pro 1:30  не приняли совета моего, презрели все обличения мои;
Pro 1:31  за то и будут они вкушать от плодов путей своих и насыщаться от помыслов их.
Pro 1:32  Потому что упорство невежд убьет их, и беспечность глупцов погубит их,
Pro 1:33  а слушающий меня будет жить безопасно и спокойно, не страшась зла'.
Pro 2:1  Сын мой! если ты примешь слова мои и сохранишь при себе заповеди мои,
Pro 2:2  так что ухо твое сделаешь внимательным к мудрости и наклонишь сердце твое к размышлению;
Pro 2:3  если будешь призывать знание и взывать к разуму;
Pro 2:4  если будешь искать его, как серебра, и отыскивать его, как сокровище,
Pro 2:5  то уразумеешь страх Господень и найдешь познание о Боге.
Pro 2:6  Ибо Господь дает мудрость; из уст Его--знание и разум;
Pro 2:7  Он сохраняет для праведных спасение; Он--щит для ходящих непорочно;
Pro 2:8  Он охраняет пути правды и оберегает стезю святых Своих.
Pro 2:9  Тогда ты уразумеешь правду и правосудие и прямоту, всякую добрую стезю.
Pro 2:10  Когда мудрость войдет в сердце твое, и знание будет приятно душе твоей,
Pro 2:11  тогда рассудительность будет оберегать тебя, разум будет охранять тебя,
Pro 2:12  дабы спасти тебя от пути злого, от человека, говорящего ложь,
Pro 2:13  от тех, которые оставляют стези прямые, чтобы ходить путями тьмы;
Pro 2:14  от тех, которые радуются, делая зло, восхищаются злым развратом,
Pro 2:15  которых пути кривы, и которые блуждают на стезях своих;
Pro 2:16  дабы спасти тебя от жены другого, от чужой, которая умягчает речи свои,
Pro 2:17  которая оставила руководителя юности своей и забыла завет Бога своего.
Pro 2:18  Дом ее ведет к смерти, и стези ее--к мертвецам;
Pro 2:19  никто из вошедших к ней не возвращается и не вступает на путь жизни.
Pro 2:20  Посему ходи путем добрых и держись стезей праведников,
Pro 2:21  потому что праведные будут жить на земле, и непорочные пребудут на ней;
Pro 2:22  а беззаконные будут истреблены с земли, и вероломные искоренены из нее.
Pro 3:1  Сын мой! наставления моего не забывай, и заповеди мои да хранит сердце твое;
Pro 3:2  ибо долготы дней, лет жизни и мира они приложат тебе.
Pro 3:3  Милость и истина да не оставляют тебя: обвяжи ими шею твою, напиши их на скрижали сердца твоего,
Pro 3:4  и обретешь милость и благоволение в очах Бога и людей.
Pro 3:5  Надейся на Господа всем сердцем твоим, и не полагайся на разум твой.
Pro 3:6  Во всех путях твоих познавай Его, и Он направит стези твои.
Pro 3:7  Не будь мудрецом в глазах твоих; бойся Господа и удаляйся от зла:
Pro 3:8  это будет здравием для тела твоего и питанием для костей твоих.
Pro 3:9  Чти Господа от имения твоего и от начатков всех прибытков твоих,
Pro 3:10  и наполнятся житницы твои до избытка, и точила твои будут переливаться новым вином.
Pro 3:11  Наказания Господня, сын мой, не отвергай, и не тяготись обличением Его;
Pro 3:12  ибо кого любит Господь, того наказывает и благоволит к тому, как отец к сыну своему.
Pro 3:13  Блажен человек, который снискал мудрость, и человек, который приобрел разум, --
Pro 3:14  потому что приобретение ее лучше приобретения серебра, и прибыли от нее больше, нежели от золота:
Pro 3:15  она дороже драгоценных камней; и ничто из желаемого тобою не сравнится с нею.
Pro 3:16  Долгоденствие--в правой руке ее, а в левой у нее--богатство и слава;
Pro 3:17  пути ее--пути приятные, и все стези ее--мирные.
Pro 3:18  Она--древо жизни для тех, которые приобретают ее, --и блаженны, которые сохраняют ее!
Pro 3:19  Господь премудростью основал землю, небеса утвердил разумом;
Pro 3:20  Его премудростью разверзлись бездны, и облака кропят росою.
Pro 3:21  Сын мой! не упускай их из глаз твоих; храни здравомыслие и рассудительность,
Pro 3:22  и они будут жизнью для души твоей и украшением для шеи твоей.
Pro 3:23  Тогда безопасно пойдешь по пути твоему, и нога твоя не споткнется.
Pro 3:24  Когда ляжешь спать, --не будешь бояться; и когда уснешь, --сон твой приятен будет.
Pro 3:25  Не убоишься внезапного страха и пагубы от нечестивых, когда она придет;
Pro 3:26  потому что Господь будет упованием твоим и сохранит ногу твою от уловления.
Pro 3:27  Не отказывай в благодеянии нуждающемуся, когда рука твоя в силе сделать его.
Pro 3:28  Не говори другу твоему: `пойди и приди опять, и завтра я дам', когда ты имеешь при себе.
Pro 3:29  Не замышляй против ближнего твоего зла, когда он без опасения живет с тобою.
Pro 3:30  Не ссорься с человеком без причины, когда он не сделал зла тебе.
Pro 3:31  Не соревнуй человеку, поступающему насильственно, и не избирай ни одного из путей его;
Pro 3:32  потому что мерзость пред Господом развратный, а с праведными у Него общение.
Pro 3:33  Проклятие Господне на доме нечестивого, а жилище благочестивых Он благословляет.
Pro 3:34  Если над кощунниками Он посмевается, то смиренным дает благодать.
Pro 3:35  Мудрые наследуют славу, а глупые--бесславие.
Pro 4:1  Слушайте, дети, наставление отца, и внимайте, чтобы научиться разуму,
Pro 4:2  потому что я преподал вам доброе учение. Не оставляйте заповеди моей.
Pro 4:3  Ибо и я был сын у отца моего, нежно любимый и единственный у матери моей,
Pro 4:4  и он учил меня и говорил мне: да удержит сердце твое слова мои; храни заповеди мои, и живи.
Pro 4:5  Приобретай мудрость, приобретай разум: не забывай этого и не уклоняйся от слов уст моих.
Pro 4:6  Не оставляй ее, и она будет охранять тебя; люби ее, и она будет оберегать тебя.
Pro 4:7  Главное--мудрость: приобретай мудрость, и всем имением твоим приобретай разум.
Pro 4:8  Высоко цени ее, и она возвысит тебя; она прославит тебя, если ты прилепишься к ней;
Pro 4:9  возложит на голову твою прекрасный венок, доставит тебе великолепный венец.
Pro 4:10  Слушай, сын мой, и прими слова мои, --и умножатся тебе лета жизни.
Pro 4:11  Я указываю тебе путь мудрости, веду тебя по стезям прямым.
Pro 4:12  Когда пойдешь, не будет стеснен ход твой, и когда побежишь, не споткнешься.
Pro 4:13  Крепко держись наставления, не оставляй, храни его, потому что оно--жизнь твоя.
Pro 4:14  Не вступай на стезю нечестивых и не ходи по пути злых;
Pro 4:15  оставь его, не ходи по нему, уклонись от него и пройди мимо;
Pro 4:16  потому что они не заснут, если не сделают зла; пропадает сон у них, если они не доведут кого до падения;
Pro 4:17  ибо они едят хлеб беззакония и пьют вино хищения.
Pro 4:18  Стезя праведных--как светило лучезарное, которое более и более светлеет до полного дня.
Pro 4:19  Путь же беззаконных--как тьма; они не знают, обо что споткнутся.
Pro 4:20  Сын мой! словам моим внимай, и к речам моим приклони ухо твое;
Pro 4:21  да не отходят они от глаз твоих; храни их внутри сердца твоего:
Pro 4:22  потому что они жизнь для того, кто нашел их, и здравие для всего тела его.
Pro 4:23  Больше всего хранимого храни сердце твое, потому что из него источники жизни.
Pro 4:24  Отвергни от себя лживость уст, и лукавство языка удали от себя.
Pro 4:25  Глаза твои пусть прямо смотрят, и ресницы твои да направлены будут прямо пред тобою.
Pro 4:26  Обдумай стезю для ноги твоей, и все пути твои да будут тверды.
Pro 4:27  Не уклоняйся ни направо, ни налево; удали ногу твою от зла,
Pro 5:1  Сын мой! внимай мудрости моей, и приклони ухо твое к разуму моему,
Pro 5:2  чтобы соблюсти рассудительность, и чтобы уста твои сохранили знание.
Pro 5:3  ибо мед источают уста чужой жены, и мягче елея речь ее;
Pro 5:4  но последствия от нее горьки, как полынь, остры, как меч обоюдоострый;
Pro 5:5  ноги ее нисходят к смерти, стопы ее достигают преисподней.
Pro 5:6  Если бы ты захотел постигнуть стезю жизни ее, то пути ее непостоянны, и ты не узнаешь их.
Pro 5:7  Итак, дети, слушайте меня и не отступайте от слов уст моих.
Pro 5:8  Держи дальше от нее путь твой и не подходи близко к дверям дома ее,
Pro 5:9  чтобы здоровья твоего не отдать другим и лет твоих мучителю;
Pro 5:10  чтобы не насыщались силою твоею чужие, и труды твои не были для чужого дома.
Pro 5:11  И ты будешь стонать после, когда плоть твоя и тело твое будут истощены, --
Pro 5:12  и скажешь: `зачем я ненавидел наставление, и сердце мое пренебрегало обличением,
Pro 5:13  и я не слушал голоса учителей моих, не приклонял уха моего к наставникам моим:
Pro 5:14  едва не впал я во всякое зло среди собрания и общества!'
Pro 5:15  Пей воду из твоего водоема и текущую из твоего колодезя.
Pro 5:16  Пусть [не] разливаются источники твои по улице, потоки вод--по площадям;
Pro 5:17  пусть они будут принадлежать тебе одному, а не чужим с тобою.
Pro 5:18  Источник твой да будет благословен; и утешайся женою юности твоей,
Pro 5:19  любезною ланью и прекрасною серною: груди ее да упоявают тебя во всякое время, любовью ее услаждайся постоянно.
Pro 5:20  И для чего тебе, сын мой, увлекаться постороннею и обнимать груди чужой?
Pro 5:21  Ибо пред очами Господа пути человека, и Он измеряет все стези его.
Pro 5:22  Беззаконного уловляют собственные беззакония его, и в узах греха своего он содержится:
Pro 5:23  он умирает без наставления, и от множества безумия своего теряется.
Pro 6:1  Сын мой! если ты поручился за ближнего твоего и дал руку твою за другого, --
Pro 6:2  ты опутал себя словами уст твоих, пойман словами уст твоих.
Pro 6:3  Сделай же, сын мой, вот что, и избавь себя, так как ты попался в руки ближнего твоего: пойди, пади к ногам и умоляй ближнего твоего;
Pro 6:4  не давай сна глазам твоим и дремания веждам твоим;
Pro 6:5  спасайся, как серна из руки и как птица из руки птицелова.
Pro 6:6  Пойди к муравью, ленивец, посмотри на действия его, и будь мудрым.
Pro 6:7  Нет у него ни начальника, ни приставника, ни повелителя;
Pro 6:8  но он заготовляет летом хлеб свой, собирает во время жатвы пищу свою.
Pro 6:9  Доколе ты, ленивец, будешь спать? когда ты встанешь от сна твоего?
Pro 6:10  Немного поспишь, немного подремлешь, немного, сложив руки, полежишь:
Pro 6:11  и придет, как прохожий, бедность твоя, и нужда твоя, как разбойник.
Pro 6:12  Человек лукавый, человек нечестивый ходит со лживыми устами,
Pro 6:13  мигает глазами своими, говорит ногами своими, дает знаки пальцами своими;
Pro 6:14  коварство в сердце его: он умышляет зло во всякое время, сеет раздоры.
Pro 6:15  Зато внезапно придет погибель его, вдруг будет разбит--без исцеления.
Pro 6:16  Вот шесть, что ненавидит Господь, даже семь, что мерзость душе Его:
Pro 6:17  глаза гордые, язык лживый и руки, проливающие кровь невинную,
Pro 6:18  сердце, кующее злые замыслы, ноги, быстро бегущие к злодейству,
Pro 6:19  лжесвидетель, наговаривающий ложь и сеющий раздор между братьями.
Pro 6:20  Сын мой! храни заповедь отца твоего и не отвергай наставления матери твоей;
Pro 6:21  навяжи их навсегда на сердце твое, обвяжи ими шею твою.
Pro 6:22  Когда ты пойдешь, они будут руководить тебя; когда ляжешь спать, будут охранять тебя; когда пробудишься, будут беседовать с тобою:
Pro 6:23  ибо заповедь есть светильник, и наставление--свет, и назидательные поучения--путь к жизни,
Pro 6:24  чтобы остерегать тебя от негодной женщины, от льстивого языка чужой.
Pro 6:25  Не пожелай красоты ее в сердце твоем, и да не увлечет она тебя ресницами своими;
Pro 6:26  потому что из-за жены блудной [обнищевают] до куска хлеба, а замужняя жена уловляет дорогую душу.
Pro 6:27  Может ли кто взять себе огонь в пазуху, чтобы не прогорело платье его?
Pro 6:28  Может ли кто ходить по горящим угольям, чтобы не обжечь ног своих?
Pro 6:29  То же бывает и с тем, кто входит к жене ближнего своего: кто прикоснется к ней, не останется без вины.
Pro 6:30  Не спускают вору, если он крадет, чтобы насытить душу свою, когда он голоден;
Pro 6:31  но, будучи пойман, он заплатит всемеро, отдаст все имущество дома своего.
Pro 6:32  Кто же прелюбодействует с женщиною, у того нет ума; тот губит душу свою, кто делает это:
Pro 6:33  побои и позор найдет он, и бесчестие его не изгладится,
Pro 6:34  потому что ревность--ярость мужа, и не пощадит он в день мщения,
Pro 6:35  не примет никакого выкупа и не удовольствуется, сколько бы ты ни умножал даров.
Pro 7:1  Сын мой! храни слова мои и заповеди мои сокрой у себя.
Pro 7:2  Храни заповеди мои и живи, и учение мое, как зрачок глаз твоих.
Pro 7:3  Навяжи их на персты твои, напиши их на скрижали сердца твоего.
Pro 7:4  Скажи мудрости: `Ты сестра моя!' и разум назови родным твоим,
Pro 7:5  чтобы они охраняли тебя от жены другого, от чужой, которая умягчает слова свои.
Pro 7:6  Вот, однажды смотрел я в окно дома моего, сквозь решетку мою,
Pro 7:7  и увидел среди неопытных, заметил между молодыми людьми неразумного юношу,
Pro 7:8  переходившего площадь близ угла ее и шедшего по дороге к дому ее,
Pro 7:9  в сумерки в вечер дня, в ночной темноте и во мраке.
Pro 7:10  И вот--навстречу к нему женщина, в наряде блудницы, с коварным сердцем,
Pro 7:11  шумливая и необузданная; ноги ее не живут в доме ее:
Pro 7:12  то на улице, то на площадях, и у каждого угла строит она ковы.
Pro 7:13  Она схватила его, целовала его, и с бесстыдным лицом говорила ему:
Pro 7:14  `мирная жертва у меня: сегодня я совершила обеты мои;
Pro 7:15  поэтому и вышла навстречу тебе, чтобы отыскать тебя, и--нашла тебя;
Pro 7:16  коврами я убрала постель мою, разноцветными тканями Египетскими;
Pro 7:17  спальню мою надушила смирною, алоем и корицею;
Pro 7:18  зайди, будем упиваться нежностями до утра, насладимся любовью,
Pro 7:19  потому что мужа нет дома: он отправился в дальнюю дорогу;
Pro 7:20  кошелек серебра взял с собою; придет домой ко дню полнолуния'.
Pro 7:21  Множеством ласковых слов она увлекла его, мягкостью уст своих овладела им.
Pro 7:22  Тотчас он пошел за нею, как вол идет на убой, и как олень--на выстрел,
Pro 7:23  доколе стрела не пронзит печени его; как птичка кидается в силки, и не знает, что они--на погибель ее.
Pro 7:24  Итак, дети, слушайте меня и внимайте словам уст моих.
Pro 7:25  Да не уклоняется сердце твое на пути ее, не блуждай по стезям ее,
Pro 7:26  потому что многих повергла она ранеными, и много сильных убиты ею:
Pro 7:27  дом ее--пути в преисподнюю, нисходящие во внутренние жилища смерти.
Pro 8:1  Не премудрость ли взывает? и не разум ли возвышает голос свой?
Pro 8:2  Она становится на возвышенных местах, при дороге, на распутиях;
Pro 8:3  она взывает у ворот при входе в город, при входе в двери:
Pro 8:4  `к вам, люди, взываю я, и к сынам человеческим голос мой!
Pro 8:5  Научитесь, неразумные, благоразумию, и глупые--разуму.
Pro 8:6  Слушайте, потому что я буду говорить важное, и изречение уст моих--правда;
Pro 8:7  ибо истину произнесет язык мой, и нечестие--мерзость для уст моих;
Pro 8:8  все слова уст моих справедливы; нет в них коварства и лукавства;
Pro 8:9  все они ясны для разумного и справедливы для приобретших знание.
Pro 8:10  Примите учение мое, а не серебро; лучше знание, нежели отборное золото;
Pro 8:11  потому что мудрость лучше жемчуга, и ничто из желаемого не сравнится с нею.
Pro 8:12  Я, премудрость, обитаю с разумом и ищу рассудительного знания.
Pro 8:13  Страх Господень--ненавидеть зло; гордость и высокомерие и злой путь и коварные уста я ненавижу.
Pro 8:14  У меня совет и правда; я разум, у меня сила.
Pro 8:15  Мною цари царствуют и повелители узаконяют правду;
Pro 8:16  мною начальствуют начальники и вельможи и все судьи земли.
Pro 8:17  Любящих меня я люблю, и ищущие меня найдут меня;
Pro 8:18  богатство и слава у меня, сокровище непогибающее и правда;
Pro 8:19  плоды мои лучше золота, и золота самого чистого, и пользы от меня больше, нежели от отборного серебра.
Pro 8:20  Я хожу по пути правды, по стезям правосудия,
Pro 8:21  чтобы доставить любящим меня существенное благо, и сокровищницы их я наполняю.
Pro 8:22  Господь имел меня началом пути Своего, прежде созданий Своих, искони;
Pro 8:23  от века я помазана, от начала, прежде бытия земли.
Pro 8:24  Я родилась, когда еще не существовали бездны, когда еще не было источников, обильных водою.
Pro 8:25  Я родилась прежде, нежели водружены были горы, прежде холмов,
Pro 8:26  когда еще Он не сотворил ни земли, ни полей, ни начальных пылинок вселенной.
Pro 8:27  Когда Он уготовлял небеса, [я была] там. Когда Он проводил круговую черту по лицу бездны,
Pro 8:28  когда утверждал вверху облака, когда укреплял источники бездны,
Pro 8:29  когда давал морю устав, чтобы воды не переступали пределов его, когда полагал основания земли:
Pro 8:30  тогда я была при Нем художницею, и была радостью всякий день, веселясь пред лицем Его во все время,
Pro 8:31  веселясь на земном кругу Его, и радость моя [была] с сынами человеческими.
Pro 8:32  Итак, дети, послушайте меня; и блаженны те, которые хранят пути мои!
Pro 8:33  Послушайте наставления и будьте мудры, и не отступайте [от] [него].
Pro 8:34  Блажен человек, который слушает меня, бодрствуя каждый день у ворот моих и стоя на страже у дверей моих!
Pro 8:35  потому что, кто нашел меня, тот нашел жизнь, и получит благодать от Господа;
Pro 8:36  а согрешающий против меня наносит вред душе своей: все ненавидящие меня любят смерть'.
Pro 9:1  Премудрость построила себе дом, вытесала семь столбов его,
Pro 9:2  заколола жертву, растворила вино свое и приготовила у себя трапезу;
Pro 9:3  послала слуг своих провозгласить с возвышенностей городских:
Pro 9:4  `кто неразумен, обратись сюда!' И скудоумному она сказала:
Pro 9:5  `идите, ешьте хлеб мой и пейте вино, мною растворенное;
Pro 9:6  оставьте неразумие, и живите, и ходите путем разума'.
Pro 9:7  Поучающий кощунника наживет себе бесславие, и обличающий нечестивого--пятно себе.
Pro 9:8  Не обличай кощунника, чтобы он не возненавидел тебя; обличай мудрого, и он возлюбит тебя;
Pro 9:9  дай [наставление] мудрому, и он будет еще мудрее; научи правдивого, и он приумножит знание.
Pro 9:10  Начало мудрости--страх Господень, и познание Святаго--разум;
Pro 9:11  потому что чрез меня умножатся дни твои, и прибавится тебе лет жизни.
Pro 9:12  если ты мудр, то мудр для себя; и если буен, то один потерпишь.
Pro 9:13  Женщина безрассудная, шумливая, глупая и ничего не знающая
Pro 9:14  садится у дверей дома своего на стуле, на возвышенных местах города,
Pro 9:15  чтобы звать проходящих дорогою, идущих прямо своими путями:
Pro 9:16  `кто глуп, обратись сюда!' и скудоумному сказала она:
Pro 9:17  `воды краденые сладки, и утаенный хлеб приятен'.
Pro 9:18  И он не знает, что мертвецы там, и что в глубине преисподней зазванные ею.
Pro 10:1  Притчи Соломона. Сын мудрый радует отца, а сын глупый--огорчение для его матери.
Pro 10:2  Не доставляют пользы сокровища неправедные, правда же избавляет от смерти.
Pro 10:3  Не допустит Господь терпеть голод душе праведного, стяжание же нечестивых исторгнет.
Pro 10:4  Ленивая рука делает бедным, а рука прилежных обогащает.
Pro 10:5  Собирающий во время лета--сын разумный, спящий же во время жатвы--сын беспутный.
Pro 10:6  Благословения--на голове праведника, уста же беззаконных заградит насилие.
Pro 10:7  Память праведника пребудет благословенна, а имя нечестивых омерзеет.
Pro 10:8  Мудрый сердцем принимает заповеди, а глупый устами преткнется.
Pro 10:9  Кто ходит в непорочности, тот ходит безопасно; а кто превращает пути свои, тот будет наказан.
Pro 10:10  Кто мигает глазами, тот причиняет досаду, а глупый устами преткнется.
Pro 10:11  Уста праведника--источник жизни, уста же беззаконных заградит насилие.
Pro 10:12  Ненависть возбуждает раздоры, но любовь покрывает все грехи.
Pro 10:13  В устах разумного находится мудрость, но на теле глупого--розга.
Pro 10:14  Мудрые сберегают знание, но уста глупого--близкая погибель.
Pro 10:15  Имущество богатого--крепкий город его, беда для бедных--скудость их.
Pro 10:16  Труды праведного--к жизни, успех нечестивого--ко греху.
Pro 10:17  Кто хранит наставление, тот на пути к жизни; а отвергающий обличение--блуждает.
Pro 10:18  Кто скрывает ненависть, у того уста лживые; и кто разглашает клевету, тот глуп.
Pro 10:19  При многословии не миновать греха, а сдерживающий уста свои--разумен.
Pro 10:20  Отборное серебро--язык праведного, сердце же нечестивых--ничтожество.
Pro 10:21  Уста праведного пасут многих, а глупые умирают от недостатка разума.
Pro 10:22  Благословение Господне--оно обогащает и печали с собою не приносит.
Pro 10:23  Для глупого преступное деяние как бы забава, а человеку разумному свойственна мудрость.
Pro 10:24  Чего страшится нечестивый, то и постигнет его, а желание праведников исполнится.
Pro 10:25  Как проносится вихрь, [так] нет более нечестивого; а праведник--на вечном основании.
Pro 10:26  Что уксус для зубов и дым для глаз, то ленивый для посылающих его.
Pro 10:27  Страх Господень прибавляет дней, лета же нечестивых сократятся.
Pro 10:28  Ожидание праведников--радость, а надежда нечестивых погибнет.
Pro 10:29  Путь Господень--твердыня для непорочного и страх для делающих беззаконие.
Pro 10:30  Праведник во веки не поколеблется, нечестивые же не поживут на земле.
Pro 10:31  Уста праведника источают мудрость, а язык зловредный отсечется.
Pro 10:32  Уста праведного знают благоприятное, а уста нечестивых--развращенное.
Pro 11:1  Неверные весы--мерзость пред Господом, но правильный вес угоден Ему.
Pro 11:2  Придет гордость, придет и посрамление; но со смиренными--мудрость.
Pro 11:3  Непорочность прямодушных будет руководить их, а лукавство коварных погубит их.
Pro 11:4  Не поможет богатство в день гнева, правда же спасет от смерти.
Pro 11:5  Правда непорочного уравнивает путь его, а нечестивый падет от нечестия своего.
Pro 11:6  Правда прямодушных спасет их, а беззаконники будут уловлены беззаконием своим.
Pro 11:7  Со смертью человека нечестивого исчезает надежда, и ожидание беззаконных погибает.
Pro 11:8  Праведник спасается от беды, а вместо него попадает [в нее] нечестивый.
Pro 11:9  Устами лицемер губит ближнего своего, но праведники прозорливостью спасаются.
Pro 11:10  При благоденствии праведников веселится город, и при погибели нечестивых [бывает] торжество.
Pro 11:11  Благословением праведных возвышается город, а устами нечестивых разрушается.
Pro 11:12  Скудоумный высказывает презрение к ближнему своему; но разумный человек молчит.
Pro 11:13  Кто ходит переносчиком, тот открывает тайну; но верный человек таит дело.
Pro 11:14  При недостатке попечения падает народ, а при многих советниках благоденствует.
Pro 11:15  Зло причиняет себе, кто ручается за постороннего; а кто ненавидит ручательство, тот безопасен.
Pro 11:16  Благонравная жена приобретает славу, а трудолюбивые приобретают богатство.
Pro 11:17  Человек милосердый благотворит душе своей, а жестокосердый разрушает плоть свою.
Pro 11:18  Нечестивый делает дело ненадежное, а сеющему правду--награда верная.
Pro 11:19  Праведность [ведет] к жизни, а стремящийся к злу [стремится] к смерти своей.
Pro 11:20  Мерзость пред Господом--коварные сердцем; но благоугодны Ему непорочные в пути.
Pro 11:21  Можно поручиться, что порочный не останется ненаказанным; семя же праведных спасется.
Pro 11:22  Что золотое кольцо в носу у свиньи, то женщина красивая и--безрассудная.
Pro 11:23  Желание праведных [есть] одно добро, ожидание нечестивых--гнев.
Pro 11:24  Иной сыплет щедро, и [ему] еще прибавляется; а другой сверх меры бережлив, и однако же беднеет.
Pro 11:25  Благотворительная душа будет насыщена, и кто напояет [других], тот и сам напоен будет.
Pro 11:26  Кто удерживает у себя хлеб, того клянет народ; а на голове продающего--благословение.
Pro 11:27  Кто стремится к добру, тот ищет благоволения; а кто ищет зла, к тому оно и приходит.
Pro 11:28  Надеющийся на богатство свое упадет; а праведники, как лист, будут зеленеть.
Pro 11:29  Расстроивающий дом свой получит в удел ветер, и глупый будет рабом мудрого сердцем.
Pro 11:30  Плод праведника--древо жизни, и мудрый привлекает души.
Pro 11:31  Так праведнику воздается на земле, тем паче нечестивому и грешнику.
Pro 12:1  Кто любит наставление, тот любит знание; а кто ненавидит обличение, тот невежда.
Pro 12:2  Добрый приобретает благоволение от Господа; а человека коварного Он осудит.
Pro 12:3  Не утвердит себя человек беззаконием; корень же праведников неподвижен.
Pro 12:4  Добродетельная жена--венец для мужа своего; а позорная--как гниль в костях его.
Pro 12:5  Промышления праведных--правда, а замыслы нечестивых--коварство.
Pro 12:6  Речи нечестивых--засада для пролития крови, уста же праведных спасают их.
Pro 12:7  Коснись нечестивых несчастие--и нет их, а дом праведных стоит.
Pro 12:8  Хвалят человека по мере разума его, а развращенный сердцем будет в презрении.
Pro 12:9  Лучше простой, но работающий на себя, нежели выдающий себя за знатного, но нуждающийся в хлебе.
Pro 12:10  Праведный печется и о жизни скота своего, сердце же нечестивых жестоко.
Pro 12:11  Кто возделывает землю свою, тот будет насыщаться хлебом; а кто идет по следам празднолюбцев, тот скудоумен.
Pro 12:12  Нечестивый желает уловить в сеть зла; но корень праведных тверд.
Pro 12:13  Нечестивый уловляется грехами уст своих; но праведник выйдет из беды.
Pro 12:14  От плода уст [своих] человек насыщается добром, и воздаяние человеку--по делам рук его.
Pro 12:15  Путь глупого прямой в его глазах; но кто слушает совета, тот мудр.
Pro 12:16  У глупого тотчас же выкажется гнев его, а благоразумный скрывает оскорбление.
Pro 12:17  Кто говорит то, что знает, тот говорит правду; а у свидетеля ложного--обман.
Pro 12:18  Иной пустослов уязвляет как мечом, а язык мудрых--врачует.
Pro 12:19  Уста правдивые вечно пребывают, а лживый язык--только на мгновение.
Pro 12:20  Коварство--в сердце злоумышленников, радость--у миротворцев.
Pro 12:21  Не приключится праведнику никакого зла, нечестивые же будут преисполнены зол.
Pro 12:22  Мерзость пред Господом--уста лживые, а говорящие истину благоугодны Ему.
Pro 12:23  Человек рассудительный скрывает знание, а сердце глупых высказывает глупость.
Pro 12:24  Рука прилежных будет господствовать, а ленивая будет под данью.
Pro 12:25  Тоска на сердце человека подавляет его, а доброе слово развеселяет его.
Pro 12:26  Праведник указывает ближнему своему путь, а путь нечестивых вводит их в заблуждение.
Pro 12:27  Ленивый не жарит своей дичи; а имущество человека прилежного многоценно.
Pro 12:28  На пути правды--жизнь, и на стезе ее нет смерти.
Pro 13:1  Мудрый сын [слушает] наставление отца, а буйный не слушает обличения.
Pro 13:2  От плода уст [своих] человек вкусит добро, душа же законопреступников--зло.
Pro 13:3  Кто хранит уста свои, тот бережет душу свою; а кто широко раскрывает свой рот, тому беда.
Pro 13:4  Душа ленивого желает, но тщетно; а душа прилежных насытится.
Pro 13:5  Праведник ненавидит ложное слово, а нечестивый срамит и бесчестит [себя].
Pro 13:6  Правда хранит непорочного в пути, а нечестие губит грешника.
Pro 13:7  Иной выдает себя за богатого, а у него ничего нет; другой выдает себя за бедного, а у него богатства много.
Pro 13:8  Богатством своим человек выкупает жизнь [свою], а бедный и угрозы не слышит.
Pro 13:9  Свет праведных весело горит, светильник же нечестивых угасает.
Pro 13:10  От высокомерия происходит раздор, а у советующихся--мудрость.
Pro 13:11  Богатство от суетности истощается, а собирающий трудами умножает его.
Pro 13:12  Надежда, долго не сбывающаяся, томит сердце, а исполнившееся желание--[как] древо жизни.
Pro 13:13  Кто пренебрегает словом, тот причиняет вред себе; а кто боится заповеди, тому воздается.
Pro 13:14  Учение мудрого--источник жизни, удаляющий от сетей смерти.
Pro 13:15  Добрый разум доставляет приятность, путь же беззаконных жесток.
Pro 13:16  Всякий благоразумный действует с знанием, а глупый выставляет напоказ глупость.
Pro 13:17  Худой посол попадает в беду, а верный посланник--спасение.
Pro 13:18  Нищета и посрамление отвергающему учение; а кто соблюдает наставление, будет в чести.
Pro 13:19  Желание исполнившееся--приятно для души; но несносно для глупых уклоняться от зла.
Pro 13:20  Общающийся с мудрыми будет мудр, а кто дружит с глупыми, развратится.
Pro 13:21  Грешников преследует зло, а праведникам воздается добром.
Pro 13:22  Добрый оставляет наследство [и] внукам, а богатство грешника сберегается для праведного.
Pro 13:23  Много хлеба [бывает] и на ниве бедных; но некоторые гибнут от беспорядка.
Pro 13:24  Кто жалеет розги своей, тот ненавидит сына; а кто любит, тот с детства наказывает его.
Pro 13:25  Праведник ест до сытости, а чрево беззаконных терпит лишение.
Pro 14:1  Мудрая жена устроит дом свой, а глупая разрушит его своими руками.
Pro 14:2  Идущий прямым путем боится Господа; но чьи пути кривы, тот небрежет о Нем.
Pro 14:3  В устах глупого--бич гордости; уста же мудрых охраняют их.
Pro 14:4  Где нет волов, [там] ясли пусты; а много прибыли от силы волов.
Pro 14:5  Верный свидетель не лжет, а свидетель ложный наговорит много лжи.
Pro 14:6  Распутный ищет мудрости, и не находит; а для разумного знание легко.
Pro 14:7  Отойди от человека глупого, у которого ты не замечаешь разумных уст.
Pro 14:8  Мудрость разумного--знание пути своего, глупость же безрассудных--заблуждение.
Pro 14:9  Глупые смеются над грехом, а посреди праведных--благоволение.
Pro 14:10  Сердце знает горе души своей, и в радость его не вмешается чужой.
Pro 14:11  Дом беззаконных разорится, а жилище праведных процветет.
Pro 14:12  Есть пути, которые кажутся человеку прямыми; но конец их--путь к смерти.
Pro 14:13  И при смехе [иногда] болит сердце, и концом радости бывает печаль.
Pro 14:14  Человек с развращенным сердцем насытится от путей своих, и добрый--от своих.
Pro 14:15  Глупый верит всякому слову, благоразумный же внимателен к путям своим.
Pro 14:16  Мудрый боится и удаляется от зла, а глупый раздражителен и самонадеян.
Pro 14:17  Вспыльчивый может сделать глупость; но человек, умышленно делающий зло, ненавистен.
Pro 14:18  Невежды получают в удел себе глупость, а благоразумные увенчаются знанием.
Pro 14:19  Преклонятся злые пред добрыми и нечестивые--у ворот праведника.
Pro 14:20  Бедный ненавидим бывает даже близким своим, а у богатого много друзей.
Pro 14:21  Кто презирает ближнего своего, тот грешит; а кто милосерд к бедным, тот блажен.
Pro 14:22  Не заблуждаются ли умышляющие зло? но милость и верность у благомыслящих.
Pro 14:23  От всякого труда есть прибыль, а от пустословия только ущерб.
Pro 14:24  Венец мудрых--богатство их, а глупость невежд глупость [и] [есть].
Pro 14:25  Верный свидетель спасает души, а лживый наговорит много лжи.
Pro 14:26  В страхе пред Господом--надежда твердая, и сынам Своим Он прибежище.
Pro 14:27  Страх Господень--источник жизни, удаляющий от сетей смерти.
Pro 14:28  Во множестве народа--величие царя, а при малолюдстве народа беда государю.
Pro 14:29  У терпеливого человека много разума, а раздражительный выказывает глупость.
Pro 14:30  Кроткое сердце--жизнь для тела, а зависть--гниль для костей.
Pro 14:31  Кто теснит бедного, тот хулит Творца его; чтущий же Его благотворит нуждающемуся.
Pro 14:32  За зло свое нечестивый будет отвергнут, а праведный и при смерти своей имеет надежду.
Pro 14:33  Мудрость почиет в сердце разумного, и среди глупых дает знать о себе.
Pro 14:34  Праведность возвышает народ, а беззаконие--бесчестие народов.
Pro 14:35  Благоволение царя--к рабу разумному, а гнев его--против того, кто позорит его.
Pro 15:1  Кроткий ответ отвращает гнев, а оскорбительное слово возбуждает ярость.
Pro 15:2  Язык мудрых сообщает добрые знания, а уста глупых изрыгают глупость.
Pro 15:3  На всяком месте очи Господни: они видят злых и добрых.
Pro 15:4  Кроткий язык--древо жизни, но необузданный--сокрушение духа.
Pro 15:5  Глупый пренебрегает наставлением отца своего; а кто внимает обличениям, тот благоразумен.
Pro 15:6  В доме праведника--обилие сокровищ, а в прибытке нечестивого--расстройство.
Pro 15:7  Уста мудрых распространяют знание, а сердце глупых не так.
Pro 15:8  Жертва нечестивых--мерзость пред Господом, а молитва праведных благоугодна Ему.
Pro 15:9  Мерзость пред Господом--путь нечестивого, а идущего путем правды Он любит.
Pro 15:10  Злое наказание--уклоняющемуся от пути, и ненавидящий обличение погибнет.
Pro 15:11  Преисподняя и Аваддон [открыты] пред Господом, тем более сердца сынов человеческих.
Pro 15:12  Не любит распутный обличающих его, и к мудрым не пойдет.
Pro 15:13  Веселое сердце делает лице веселым, а при сердечной скорби дух унывает.
Pro 15:14  Сердце разумного ищет знания, уста же глупых питаются глупостью.
Pro 15:15  Все дни несчастного печальны; а у кого сердце весело, у того всегда пир.
Pro 15:16  Лучше немногое при страхе Господнем, нежели большое сокровище, и при нем тревога.
Pro 15:17  Лучше блюдо зелени, и при нем любовь, нежели откормленный бык, и при нем ненависть.
Pro 15:18  Вспыльчивый человек возбуждает раздор, а терпеливый утишает распрю.
Pro 15:19  Путь ленивого--как терновый плетень, а путь праведных--гладкий.
Pro 15:20  Мудрый сын радует отца, а глупый человек пренебрегает мать свою.
Pro 15:21  Глупость--радость для малоумного, а человек разумный идет прямою дорогою.
Pro 15:22  Без совета предприятия расстроятся, а при множестве советников они состоятся.
Pro 15:23  Радость человеку в ответе уст его, и как хорошо слово вовремя!
Pro 15:24  Путь жизни мудрого вверх, чтобы уклониться от преисподней внизу.
Pro 15:25  Дом надменных разорит Господь, а межу вдовы укрепит.
Pro 15:26  Мерзость пред Господом--помышления злых, слова же непорочных угодны Ему.
Pro 15:27  Корыстолюбивый расстроит дом свой, а ненавидящий подарки будет жить.
Pro 15:28  Сердце праведного обдумывает ответ, а уста нечестивых изрыгают зло.
Pro 15:29  Далек Господь от нечестивых, а молитву праведников слышит.
Pro 15:30  Светлый взгляд радует сердце, добрая весть утучняет кости.
Pro 15:31  Ухо, внимательное к учению жизни, пребывает между мудрыми.
Pro 15:32  Отвергающий наставление не радеет о своей душе; а кто внимает обличению, тот приобретает разум.
Pro 15:33  Страх Господень научает мудрости, и славе предшествует смирение.
Pro 16:1  Человеку [принадлежат] предположения сердца, но от Господа ответ языка.
Pro 16:2  Все пути человека чисты в его глазах, но Господь взвешивает души.
Pro 16:3  Предай Господу дела твои, и предприятия твои совершатся.
Pro 16:4  Все сделал Господь ради Себя; и даже нечестивого [блюдет] на день бедствия.
Pro 16:5  Мерзость пред Господом всякий надменный сердцем; можно поручиться, что он не останется ненаказанным.
Pro 16:6  Милосердием и правдою очищается грех, и страх Господень отводит от зла.
Pro 16:7  Когда Господу угодны пути человека, Он и врагов его примиряет с ним.
Pro 16:8  Лучше немногое с правдою, нежели множество прибытков с неправдою.
Pro 16:9  Сердце человека обдумывает свой путь, но Господь управляет шествием его.
Pro 16:10  В устах царя--слово вдохновенное; уста его не должны погрешать на суде.
Pro 16:11  Верные весы и весовые чаши--от Господа; от Него же все гири в суме.
Pro 16:12  Мерзость для царей--дело беззаконное, потому что правдою утверждается престол.
Pro 16:13  Приятны царю уста правдивые, и говорящего истину он любит.
Pro 16:14  Царский гнев--вестник смерти; но мудрый человек умилостивит его.
Pro 16:15  В светлом взоре царя--жизнь, и благоволение его--как облако с поздним дождем.
Pro 16:16  Приобретение мудрости гораздо лучше золота, и приобретение разума предпочтительнее отборного серебра.
Pro 16:17  Путь праведных--уклонение от зла: тот бережет душу свою, кто хранит путь свой.
Pro 16:18  Погибели предшествует гордость, и падению--надменность.
Pro 16:19  Лучше смиряться духом с кроткими, нежели разделять добычу с гордыми.
Pro 16:20  Кто ведет дело разумно, тот найдет благо, и кто надеется на Господа, тот блажен.
Pro 16:21  Мудрый сердцем прозовется благоразумным, и сладкая речь прибавит к учению.
Pro 16:22  Разум для имеющих его--источник жизни, а ученость глупых--глупость.
Pro 16:23  Сердце мудрого делает язык его мудрым и умножает знание в устах его.
Pro 16:24  Приятная речь--сотовый мед, сладка для души и целебна для костей.
Pro 16:25  Есть пути, которые кажутся человеку прямыми, но конец их путь к смерти.
Pro 16:26  Трудящийся трудится для себя, потому что понуждает его [к] [тому] рот его.
Pro 16:27  Человек лукавый замышляет зло, и на устах его как бы огонь палящий.
Pro 16:28  Человек коварный сеет раздор, и наушник разлучает друзей.
Pro 16:29  Человек неблагонамеренный развращает ближнего своего и ведет его на путь недобрый;
Pro 16:30  прищуривает глаза свои, чтобы придумать коварство; закусывая себе губы, совершает злодейство.
Pro 16:31  Венец славы--седина, которая находится на пути правды.
Pro 16:32  Долготерпеливый лучше храброго, и владеющий собою [лучше] завоевателя города.
Pro 16:33  В полу бросается жребий, но все решение его--от Господа.
Pro 17:1  Лучше кусок сухого хлеба, и с ним мир, нежели дом, полный заколотого скота, с раздором.
Pro 17:2  Разумный раб господствует над беспутным сыном и между братьями разделит наследство.
Pro 17:3  Плавильня--для серебра, и горнило--для золота, а сердца испытывает Господь.
Pro 17:4  Злодей внимает устам беззаконным, лжец слушается языка пагубного.
Pro 17:5  Кто ругается над нищим, тот хулит Творца его; кто радуется несчастью, тот не останется ненаказанным.
Pro 17:6  Венец стариков--сыновья сыновей, и слава детей--родители их.
Pro 17:7  Неприлична глупому важная речь, тем паче знатному--уста лживые.
Pro 17:8  Подарок--драгоценный камень в глазах владеющего им: куда ни обратится он, успеет.
Pro 17:9  Прикрывающий проступок ищет любви; а кто снова напоминает о нем, тот удаляет друга.
Pro 17:10  На разумного сильнее действует выговор, нежели на глупого сто ударов.
Pro 17:11  Возмутитель ищет только зла; поэтому жестокий ангел будет послан против него.
Pro 17:12  Лучше встретить человеку медведицу, лишенную детей, нежели глупца с его глупостью.
Pro 17:13  Кто за добро воздает злом, от дома того не отойдет зло.
Pro 17:14  Начало ссоры--как прорыв воды; оставь ссору прежде, нежели разгорелась она.
Pro 17:15  Оправдывающий нечестивого и обвиняющий праведного--оба мерзость пред Господом.
Pro 17:16  К чему сокровище в руках глупца? Для приобретения мудрости [у] [него] нет разума.
Pro 17:17  Друг любит во всякое время и, как брат, явится во время несчастья.
Pro 17:18  Человек малоумный дает руку и ручается за ближнего своего.
Pro 17:19  Кто любит ссоры, любит грех, и кто высоко поднимает ворота свои, тот ищет падения.
Pro 17:20  Коварное сердце не найдет добра, и лукавый язык попадет в беду.
Pro 17:21  Родил кто глупого, --себе на горе, и отец глупого не порадуется.
Pro 17:22  Веселое сердце благотворно, как врачевство, а унылый дух сушит кости.
Pro 17:23  Нечестивый берет подарок из пазухи, чтобы извратить пути правосудия.
Pro 17:24  Мудрость--пред лицем у разумного, а глаза глупца--на конце земли.
Pro 17:25  Глупый сын--досада отцу своему и огорчение для матери своей.
Pro 17:26  Нехорошо и обвинять правого, [и] бить вельмож за правду.
Pro 17:27  Разумный воздержан в словах своих, и благоразумный хладнокровен.
Pro 17:28  И глупец, когда молчит, может показаться мудрым, и затворяющий уста свои--благоразумным.
Pro 18:1  Прихоти ищет своенравный, восстает против всего умного.
Pro 18:2  Глупый не любит знания, а только бы выказать свой ум.
Pro 18:3  С приходом нечестивого приходит и презрение, а с бесславием--поношение.
Pro 18:4  Слова уст человеческих--глубокие воды; источник мудрости--струящийся поток.
Pro 18:5  Нехорошо быть лицеприятным к нечестивому, чтобы ниспровергнуть праведного на суде.
Pro 18:6  Уста глупого идут в ссору, и слова его вызывают побои.
Pro 18:7  Язык глупого--гибель для него, и уста его--сеть для души его.
Pro 18:8  Слова наушника--как лакомства, и они входят во внутренность чрева.
Pro 18:9  Нерадивый в работе своей--брат расточителю.
Pro 18:10  Имя Господа--крепкая башня: убегает в нее праведник--и безопасен.
Pro 18:11  Имение богатого--крепкий город его, и как высокая ограда в его воображении.
Pro 18:12  Перед падением возносится сердце человека, а смирение предшествует славе.
Pro 18:13  Кто дает ответ не выслушав, тот глуп, и стыд ему.
Pro 18:14  Дух человека переносит его немощи; а пораженный дух--кто может подкрепить его?
Pro 18:15  Сердце разумного приобретает знание, и ухо мудрых ищет знания.
Pro 18:16  Подарок у человека дает ему простор и до вельмож доведет его.
Pro 18:17  Первый в тяжбе своей прав, но приходит соперник его и исследывает его.
Pro 18:18  Жребий прекращает споры и решает между сильными.
Pro 18:19  Озлобившийся брат [неприступнее] крепкого города, и ссоры подобны запорам замка.
Pro 18:20  От плода уст человека наполняется чрево его; произведением уст своих он насыщается.
Pro 18:21  Смерть и жизнь--во власти языка, и любящие его вкусят от плодов его.
Pro 18:22  Кто нашел [добрую] жену, тот нашел благо и получил благодать от Господа.
Pro 18:23  С мольбою говорит нищий, а богатый отвечает грубо.
Pro 18:24  Кто хочет иметь друзей, тот и сам должен быть дружелюбным; и бывает друг, более привязанный, нежели брат.
Pro 19:1  Лучше бедный, ходящий в своей непорочности, нежели [богатый] со лживыми устами, и притом глупый.
Pro 19:2  Нехорошо душе без знания, и торопливый ногами оступится.
Pro 19:3  Глупость человека извращает путь его, а сердце его негодует на Господа.
Pro 19:4  Богатство прибавляет много друзей, а бедный оставляется и другом своим.
Pro 19:5  Лжесвидетель не останется ненаказанным, и кто говорит ложь, не спасется.
Pro 19:6  Многие заискивают у знатных, и всякий--друг человеку, делающему подарки.
Pro 19:7  Бедного ненавидят все братья его, тем паче друзья его удаляются от него: гонится за ними, чтобы поговорить, но и этого нет.
Pro 19:8  Кто приобретает разум, тот любит душу свою; кто наблюдает благоразумие, тот находит благо.
Pro 19:9  Лжесвидетель не останется ненаказанным, и кто говорит ложь, погибнет.
Pro 19:10  Неприлична глупцу пышность, тем паче рабу господство над князьями.
Pro 19:11  Благоразумие делает человека медленным на гнев, и слава для него--быть снисходительным к проступкам.
Pro 19:12  Гнев царя--как рев льва, а благоволение его--как роса на траву.
Pro 19:13  Глупый сын--сокрушение для отца своего, и сварливая жена--сточная труба.
Pro 19:14  Дом и имение--наследство от родителей, а разумная жена--от Господа.
Pro 19:15  Леность погружает в сонливость, и нерадивая душа будет терпеть голод.
Pro 19:16  Хранящий заповедь хранит душу свою, а нерадящий о путях своих погибнет.
Pro 19:17  Благотворящий бедному дает взаймы Господу, и Он воздаст ему за благодеяние его.
Pro 19:18  Наказывай сына своего, доколе есть надежда, и не возмущайся криком его.
Pro 19:19  Гневливый пусть терпит наказание, потому что, если пощадишь [его], придется тебе еще больше наказывать его.
Pro 19:20  Слушайся совета и принимай обличение, чтобы сделаться тебе впоследствии мудрым.
Pro 19:21  Много замыслов в сердце человека, но состоится только определенное Господом.
Pro 19:22  Радость человеку--благотворительность его, и бедный человек лучше, нежели лживый.
Pro 19:23  Страх Господень [ведет] к жизни, и [кто имеет его], всегда будет доволен, и зло не постигнет его.
Pro 19:24  Ленивый опускает руку свою в чашу, и не хочет донести ее до рта своего.
Pro 19:25  Если ты накажешь кощунника, то и простой сделается благоразумным; и [если] обличишь разумного, то он поймет наставление.
Pro 19:26  Разоряющий отца и выгоняющий мать--сын срамной и бесчестный.
Pro 19:27  Перестань, сын мой, слушать внушения об уклонении от изречений разума.
Pro 19:28  Лукавый свидетель издевается над судом, и уста беззаконных глотают неправду.
Pro 19:29  Готовы для кощунствующих суды, и побои--на тело глупых.
Pro 20:1  Вино--глумливо, сикера--буйна; и всякий, увлекающийся ими, неразумен.
Pro 20:2  Гроза царя--как бы рев льва: кто раздражает его, тот грешит против самого себя.
Pro 20:3  Честь для человека--отстать от ссоры; а всякий глупец задорен.
Pro 20:4  Ленивец зимою не пашет: поищет летом--и нет ничего.
Pro 20:5  Помыслы в сердце человека--глубокие воды, но человек разумный вычерпывает их.
Pro 20:6  Многие хвалят человека за милосердие, но правдивого человека кто находит?
Pro 20:7  Праведник ходит в своей непорочности: блаженны дети его после него!
Pro 20:8  Царь, сидящий на престоле суда, разгоняет очами своими все злое.
Pro 20:9  Кто может сказать: `я очистил мое сердце, я чист от греха моего?'
Pro 20:10  Неодинаковые весы, неодинаковая мера, то и другое--мерзость пред Господом.
Pro 20:11  Можно узнать даже отрока по занятиям его, чисто ли и правильно ли будет поведение его.
Pro 20:12  Ухо слышащее и глаз видящий--и то и другое создал Господь.
Pro 20:13  Не люби спать, чтобы тебе не обеднеть; держи открытыми глаза твои, и будешь досыта есть хлеб.
Pro 20:14  `Дурно, дурно', говорит покупатель, а когда отойдет, хвалится.
Pro 20:15  Есть золото и много жемчуга, но драгоценная утварь--уста разумные.
Pro 20:16  Возьми платье его, так как он поручился за чужого; и за стороннего возьми от него залог.
Pro 20:17  Сладок для человека хлеб, [приобретенный] неправдою; но после рот его наполнится дресвою.
Pro 20:18  Предприятия получают твердость чрез совещание, и по совещании веди войну.
Pro 20:19  Кто ходит переносчиком, тот открывает тайну; и кто широко раскрывает рот, с тем не сообщайся.
Pro 20:20  Кто злословит отца своего и свою мать, того светильник погаснет среди глубокой тьмы.
Pro 20:21  Наследство, поспешно захваченное вначале, не благословится впоследствии.
Pro 20:22  Не говори: `я отплачу за зло'; предоставь Господу, и Он сохранит тебя.
Pro 20:23  Мерзость пред Господом--неодинаковые гири, и неверные весы--не добро.
Pro 20:24  От Господа направляются шаги человека; человеку же как узнать путь свой?
Pro 20:25  Сеть для человека--поспешно давать обет, и после обета обдумывать.
Pro 20:26  Мудрый царь вывеет нечестивых и обратит на них колесо.
Pro 20:27  Светильник Господень--дух человека, испытывающий все глубины сердца.
Pro 20:28  Милость и истина охраняют царя, и милостью он поддерживает престол свой.
Pro 20:29  Слава юношей--сила их, а украшение стариков--седина.
Pro 20:30  Раны от побоев--врачевство против зла, и удары, проникающие во внутренности чрева.
Pro 21:1  Сердце царя--в руке Господа, как потоки вод: куда захочет, Он направляет его.
Pro 21:2  Всякий путь человека прям в глазах его; но Господь взвешивает сердца.
Pro 21:3  Соблюдение правды и правосудия более угодно Господу, нежели жертва.
Pro 21:4  Гордость очей и надменность сердца, отличающие нечестивых, --грех.
Pro 21:5  Помышления прилежного стремятся к изобилию, а всякий торопливый терпит лишение.
Pro 21:6  Приобретение сокровища лживым языком--мимолетное дуновение ищущих смерти.
Pro 21:7  Насилие нечестивых обрушится на них, потому что они отреклись соблюдать правду.
Pro 21:8  Превратен путь человека развращенного; а кто чист, того действие прямо.
Pro 21:9  Лучше жить в углу на кровле, нежели со сварливою женою в пространном доме.
Pro 21:10  Душа нечестивого желает зла: не найдет милости в глазах его и друг его.
Pro 21:11  Когда наказывается кощунник, простой делается мудрым; и когда вразумляется мудрый, то он приобретает знание.
Pro 21:12  Праведник наблюдает за домом нечестивого: как повергаются нечестивые в несчастие.
Pro 21:13  Кто затыкает ухо свое от вопля бедного, тот и сам будет вопить, --и не будет услышан.
Pro 21:14  Подарок тайный тушит гнев, и дар в пазуху--сильную ярость.
Pro 21:15  Соблюдение правосудия--радость для праведника и страх для делающих зло.
Pro 21:16  Человек, сбившийся с пути разума, водворится в собрании мертвецов.
Pro 21:17  Кто любит веселье, обеднеет; а кто любит вино и тук, не разбогатеет.
Pro 21:18  Выкупом будет за праведного нечестивый и за прямодушного--лукавый.
Pro 21:19  Лучше жить в земле пустынной, нежели с женою сварливою и сердитою.
Pro 21:20  Вожделенное сокровище и тук--в доме мудрого; а глупый человек расточает их.
Pro 21:21  Соблюдающий правду и милость найдет жизнь, правду и славу.
Pro 21:22  Мудрый входит в город сильных и ниспровергает крепость, на которую они надеялись.
Pro 21:23  Кто хранит уста свои и язык свой, тот хранит от бед душу свою.
Pro 21:24  Надменный злодей--кощунник имя ему--действует в пылу гордости.
Pro 21:25  Алчба ленивца убьет его, потому что руки его отказываются работать;
Pro 21:26  всякий день он сильно алчет, а праведник дает и не жалеет.
Pro 21:27  Жертва нечестивых--мерзость, особенно когда с лукавством приносят ее.
Pro 21:28  Лжесвидетель погибнет; а человек, который говорит, что знает, будет говорить всегда.
Pro 21:29  Человек нечестивый дерзок лицом своим, а праведный держит прямо путь свой.
Pro 21:30  Нет мудрости, и нет разума, и нет совета вопреки Господу.
Pro 21:31  Коня приготовляют на день битвы, но победа--от Господа.
Pro 22:1  Доброе имя лучше большого богатства, и добрая слава лучше серебра и золота.
Pro 22:2  Богатый и бедный встречаются друг с другом: того и другого создал Господь.
Pro 22:3  Благоразумный видит беду, и укрывается; а неопытные идут вперед, и наказываются.
Pro 22:4  За смирением следует страх Господень, богатство и слава и жизнь.
Pro 22:5  Терны и сети на пути коварного; кто бережет душу свою, удались от них.
Pro 22:6  Наставь юношу при начале пути его: он не уклонится от него, когда и состарится.
Pro 22:7  Богатый господствует над бедным, и должник [делается] рабом заимодавца.
Pro 22:8  Сеющий неправду пожнет беду, и трости гнева его не станет.
Pro 22:9  Милосердый будет благословляем, потому что дает бедному от хлеба своего.
Pro 22:10  Прогони кощунника, и удалится раздор, и прекратятся ссора и брань.
Pro 22:11  Кто любит чистоту сердца, у того приятность на устах, тому царь--друг.
Pro 22:12  Очи Господа охраняют знание, а слова законопреступника Он ниспровергает.
Pro 22:13  Ленивец говорит: `лев на улице! посреди площади убьют меня!'
Pro 22:14  Глубокая пропасть--уста блудниц: на кого прогневается Господь, тот упадет туда.
Pro 22:15  Глупость привязалась к сердцу юноши, но исправительная розга удалит ее от него.
Pro 22:16  Кто обижает бедного, чтобы умножить свое богатство, и кто дает богатому, тот обеднеет.
Pro 22:17  Приклони ухо твое, и слушай слова мудрых, и сердце твое обрати к моему знанию;
Pro 22:18  потому что утешительно будет, если ты будешь хранить их в сердце твоем, и они будут также в устах твоих.
Pro 22:19  Чтобы упование твое было на Господа, я учу тебя и сегодня, и ты [помни].
Pro 22:20  Не писал ли я тебе трижды в советах и наставлении,
Pro 22:21  чтобы научить тебя точным словам истины, дабы ты мог передавать слова истины посылающим тебя?
Pro 22:22  Не будь грабителем бедного, потому что он беден, и не притесняй несчастного у ворот,
Pro 22:23  потому что Господь вступится в дело их и исхитит душу у грабителей их.
Pro 22:24  Не дружись с гневливым и не сообщайся с человеком вспыльчивым,
Pro 22:25  чтобы не научиться путям его и не навлечь петли на душу твою.
Pro 22:26  Не будь из тех, которые дают руки и поручаются за долги:
Pro 22:27  если тебе нечем заплатить, то для чего доводить себя, чтобы взяли постель твою из-под тебя?
Pro 22:28  Не передвигай межи давней, которую провели отцы твои.
Pro 22:29  Видел ли ты человека проворного в своем деле? Он будет стоять перед царями, он не будет стоять перед простыми.
Pro 23:1  Когда сядешь вкушать пищу с властелином, то тщательно наблюдай, что перед тобою,
Pro 23:2  и поставь преграду в гортани твоей, если ты алчен.
Pro 23:3  Не прельщайся лакомыми яствами его; это--обманчивая пища.
Pro 23:4  Не заботься о том, чтобы нажить богатство; оставь такие мысли твои.
Pro 23:5  Устремишь глаза твои на него, и--его уже нет; потому что оно сделает себе крылья и, как орел, улетит к небу.
Pro 23:6  Не вкушай пищи у человека завистливого и не прельщайся лакомыми яствами его;
Pro 23:7  потому что, каковы мысли в душе его, таков и он; `ешь и пей', говорит он тебе, а сердце его не с тобою.
Pro 23:8  Кусок, который ты съел, изблюешь, и добрые слова твои ты потратишь напрасно.
Pro 23:9  В уши глупого не говори, потому что он презрит разумные слова твои.
Pro 23:10  Не передвигай межи давней и на поля сирот не заходи,
Pro 23:11  потому что Защитник их силен; Он вступится в дело их с тобою.
Pro 23:12  Приложи сердце твое к учению и уши твои--к умным словам.
Pro 23:13  Не оставляй юноши без наказания: если накажешь его розгою, он не умрет;
Pro 23:14  ты накажешь его розгою и спасешь душу его от преисподней.
Pro 23:15  Сын мой! если сердце твое будет мудро, то порадуется и мое сердце;
Pro 23:16  и внутренности мои будут радоваться, когда уста твои будут говорить правое.
Pro 23:17  Да не завидует сердце твое грешникам, но да пребудет оно во все дни в страхе Господнем;
Pro 23:18  потому что есть будущность, и надежда твоя не потеряна.
Pro 23:19  Слушай, сын мой, и будь мудр, и направляй сердце твое на прямой путь.
Pro 23:20  Не будь между упивающимися вином, между пресыщающимися мясом:
Pro 23:21  потому что пьяница и пресыщающийся обеднеют, и сонливость оденет в рубище.
Pro 23:22  Слушайся отца твоего: он родил тебя; и не пренебрегай матери твоей, когда она и состарится.
Pro 23:23  Купи истину и не продавай мудрости и учения и разума.
Pro 23:24  Торжествует отец праведника, и родивший мудрого радуется о нем.
Pro 23:25  Да веселится отец твой и да торжествует мать твоя, родившая тебя.
Pro 23:26  Сын мой! отдай сердце твое мне, и глаза твои да наблюдают пути мои,
Pro 23:27  потому что блудница--глубокая пропасть, и чужая жена--тесный колодезь;
Pro 23:28  она, как разбойник, сидит в засаде и умножает между людьми законопреступников.
Pro 23:29  У кого вой? у кого стон? у кого ссоры? у кого горе? у кого раны без причины? у кого багровые глаза?
Pro 23:30  У тех, которые долго сидят за вином, которые приходят отыскивать [вина] приправленного.
Pro 23:31  Не смотри на вино, как оно краснеет, как оно искрится в чаше, как оно ухаживается ровно:
Pro 23:32  впоследствии, как змей, оно укусит, и ужалит, как аспид;
Pro 23:33  глаза твои будут смотреть на чужих жен, и сердце твое заговорит развратное,
Pro 23:34  и ты будешь, как спящий среди моря и как спящий на верху мачты.
Pro 23:35  [И скажешь]: `били меня, мне не было больно; толкали меня, я не чувствовал. Когда проснусь, опять буду искать того же'.
Pro 24:1  Не ревнуй злым людям и не желай быть с ними,
Pro 24:2  потому что о насилии помышляет сердце их, и о злом говорят уста их.
Pro 24:3  Мудростью устрояется дом и разумом утверждается,
Pro 24:4  и с уменьем внутренности его наполняются всяким драгоценным и прекрасным имуществом.
Pro 24:5  Человек мудрый силен, и человек разумный укрепляет силу свою.
Pro 24:6  Поэтому с обдуманностью веди войну твою, и успех [будет] при множестве совещаний.
Pro 24:7  Для глупого слишком высока мудрость; у ворот не откроет он уст своих.
Pro 24:8  Кто замышляет сделать зло, того называют злоумышленником.
Pro 24:9  Помысл глупости--грех, и кощунник--мерзость для людей.
Pro 24:10  Если ты в день бедствия оказался слабым, то бедна сила твоя.
Pro 24:11  Спасай взятых на смерть, и неужели откажешься от обреченных на убиение?
Pro 24:12  Скажешь ли: `вот, мы не знали этого'? А Испытующий сердца разве не знает? Наблюдающий над душею твоею знает это, и воздаст человеку по делам его.
Pro 24:13  Ешь, сын мой, мед, потому что он приятен, и сот, который сладок для гортани твоей:
Pro 24:14  таково и познание мудрости для души твоей. Если ты нашел [ее], то есть будущность, и надежда твоя не потеряна.
Pro 24:15  Не злоумышляй, нечестивый, против жилища праведника, не опустошай места покоя его,
Pro 24:16  ибо семь раз упадет праведник, и встанет; а нечестивые впадут в погибель.
Pro 24:17  Не радуйся, когда упадет враг твой, и да не веселится сердце твое, когда он споткнется.
Pro 24:18  Иначе, увидит Господь, и неугодно будет это в очах Его, и Он отвратит от него гнев Свой.
Pro 24:19  Не негодуй на злодеев и не завидуй нечестивым,
Pro 24:20  потому что злой не имеет будущности, --светильник нечестивых угаснет.
Pro 24:21  Бойся, сын мой, Господа и царя; с мятежниками не сообщайся,
Pro 24:22  потому что внезапно придет погибель от них, и беду от них обоих кто предузнает?
Pro 24:23  Сказано также мудрыми: иметь лицеприятие на суде--нехорошо.
Pro 24:24  Кто говорит виновному: `ты прав', того будут проклинать народы, того будут ненавидеть племена;
Pro 24:25  а обличающие будут любимы, и на них придет благословение.
Pro 24:26  В уста целует, кто отвечает словами верными.
Pro 24:27  Соверши дела твои вне дома, окончи их на поле твоем, и потом устрояй и дом твой.
Pro 24:28  Не будь лжесвидетелем на ближнего твоего: к чему тебе обманывать устами твоими?
Pro 24:29  Не говори: `как он поступил со мною, так и я поступлю с ним, воздам человеку по делам его'.
Pro 24:30  Проходил я мимо поля человека ленивого и мимо виноградника человека скудоумного:
Pro 24:31  и вот, все это заросло терном, поверхность его покрылась крапивою, и каменная ограда его обрушилась.
Pro 24:32  И посмотрел я, и обратил сердце мое, и посмотрел и получил урок:
Pro 24:33  `немного поспишь, немного подремлешь, немного, сложив руки, полежишь, --
Pro 24:34  и придет, [как] прохожий, бедность твоя, и нужда твоя--как человек вооруженный'.
Pro 25:1  И это притчи Соломона, которые собрали мужи Езекии, царя Иудейского.
Pro 25:2  Слава Божия--облекать тайною дело, а слава царей--исследывать дело.
Pro 25:3  Как небо в высоте и земля в глубине, так сердце царей--неисследимо.
Pro 25:4  Отдели примесь от серебра, и выйдет у серебряника сосуд:
Pro 25:5  удали неправедного от царя, и престол его утвердится правдою.
Pro 25:6  Не величайся пред лицем царя, и на месте великих не становись;
Pro 25:7  потому что лучше, когда скажут тебе: `пойди сюда повыше', нежели когда понизят тебя пред знатным, которого видели глаза твои.
Pro 25:8  Не вступай поспешно в тяжбу: иначе что будешь делать при окончании, когда соперник твой осрамит тебя?
Pro 25:9  Веди тяжбу с соперником твоим, но тайны другого не открывай,
Pro 25:10  дабы не укорил тебя услышавший это, и тогда бесчестие твое не отойдет от тебя.
Pro 25:11  Золотые яблоки в серебряных прозрачных сосудах--слово, сказанное прилично.
Pro 25:12  Золотая серьга и украшение из чистого золота--мудрый обличитель для внимательного уха.
Pro 25:13  Что прохлада от снега во время жатвы, то верный посол для посылающего его: он доставляет душе господина своего отраду.
Pro 25:14  Что тучи и ветры без дождя, то человек, хвастающий ложными подарками.
Pro 25:15  Кротостью склоняется к милости вельможа, и мягкий язык переламывает кость.
Pro 25:16  Нашел ты мед, --ешь, сколько тебе потребно, чтобы не пресытиться им и не изблевать его.
Pro 25:17  Не учащай входить в дом друга твоего, чтобы он не наскучил тобою и не возненавидел тебя.
Pro 25:18  Что молот и меч и острая стрела, то человек, произносящий ложное свидетельство против ближнего своего.
Pro 25:19  Что сломанный зуб и расслабленная нога, то надежда на ненадежного [человека] в день бедствия.
Pro 25:20  Что снимающий с себя одежду в холодный день, что уксус на рану, то поющий песни печальному сердцу.
Pro 25:21  Если голоден враг твой, накорми его хлебом; и если он жаждет, напой его водою:
Pro 25:22  ибо, [делая сие], ты собираешь горящие угли на голову его, и Господь воздаст тебе.
Pro 25:23  Северный ветер производит дождь, а тайный язык--недовольные лица.
Pro 25:24  Лучше жить в углу на кровле, нежели со сварливою женою в пространном доме.
Pro 25:25  Что холодная вода для истомленной жаждой души, то добрая весть из дальней страны.
Pro 25:26  Что возмущенный источник и поврежденный родник, то праведник, падающий пред нечестивым.
Pro 25:27  Как нехорошо есть много меду, так домогаться славы не есть слава.
Pro 25:28  Что город разрушенный, без стен, то человек, не владеющий духом своим.
Pro 26:1  Как снег летом и дождь во время жатвы, так честь неприлична глупому.
Pro 26:2  Как воробей вспорхнет, как ласточка улетит, так незаслуженное проклятие не сбудется.
Pro 26:3  Бич для коня, узда для осла, а палка для глупых.
Pro 26:4  Не отвечай глупому по глупости его, чтобы и тебе не сделаться подобным ему;
Pro 26:5  но отвечай глупому по глупости его, чтобы он не стал мудрецом в глазах своих.
Pro 26:6  Подрезывает себе ноги, терпит неприятность тот, кто дает словесное поручение глупцу.
Pro 26:7  Неровно поднимаются ноги у хромого, --и притча в устах глупцов.
Pro 26:8  Что влагающий драгоценный камень в пращу, то воздающий глупому честь.
Pro 26:9  Что [колючий] терн в руке пьяного, то притча в устах глупцов.
Pro 26:10  Сильный делает все произвольно: и глупого награждает, и всякого прохожего награждает.
Pro 26:11  Как пес возвращается на блевотину свою, так глупый повторяет глупость свою.
Pro 26:12  Видал ли ты человека, мудрого в глазах его? На глупого больше надежды, нежели на него.
Pro 26:13  Ленивец говорит: `лев на дороге! лев на площадях!'
Pro 26:14  Дверь ворочается на крючьях своих, а ленивец на постели своей.
Pro 26:15  Ленивец опускает руку свою в чашу, и ему тяжело донести ее до рта своего.
Pro 26:16  Ленивец в глазах своих мудрее семерых, отвечающих обдуманно.
Pro 26:17  Хватает пса за уши, кто, проходя мимо, вмешивается в чужую ссору.
Pro 26:18  Как притворяющийся помешанным бросает огонь, стрелы и смерть,
Pro 26:19  так--человек, который коварно вредит другу своему и потом говорит: `я только пошутил'.
Pro 26:20  Где нет больше дров, огонь погасает, и где нет наушника, раздор утихает.
Pro 26:21  Уголь--для жара и дрова--для огня, а человек сварливый--для разжжения ссоры.
Pro 26:22  Слова наушника--как лакомства, и они входят во внутренность чрева.
Pro 26:23  Что нечистым серебром обложенный глиняный сосуд, то пламенные уста и сердце злобное.
Pro 26:24  Устами своими притворяется враг, а в сердце своем замышляет коварство.
Pro 26:25  Если он говорит и нежным голосом, не верь ему, потому что семь мерзостей в сердце его.
Pro 26:26  Если ненависть прикрывается наедине, то откроется злоба его в народном собрании.
Pro 26:27  Кто роет яму, тот упадет в нее, и кто покатит вверх камень, к тому он воротится.
Pro 26:28  Лживый язык ненавидит уязвляемых им, и льстивые уста готовят падение.
Pro 27:1  Не хвались завтрашним днем, потому что не знаешь, что родит тот день.
Pro 27:2  Пусть хвалит тебя другой, а не уста твои, --чужой, а не язык твой.
Pro 27:3  Тяжел камень, весок и песок; но гнев глупца тяжелее их обоих.
Pro 27:4  Жесток гнев, неукротима ярость; но кто устоит против ревности?
Pro 27:5  Лучше открытое обличение, нежели скрытая любовь.
Pro 27:6  Искренни укоризны от любящего, и лживы поцелуи ненавидящего.
Pro 27:7  Сытая душа попирает и сот, а голодной душе все горькое сладко.
Pro 27:8  Как птица, покинувшая гнездо свое, так человек, покинувший место свое.
Pro 27:9  Масть и курение радуют сердце; так сладок [всякому] друг сердечным советом своим.
Pro 27:10  Не покидай друга твоего и друга отца твоего, и в дом брата твоего не ходи в день несчастья твоего: лучше сосед вблизи, нежели брат вдали.
Pro 27:11  Будь мудр, сын мой, и радуй сердце мое; и я буду иметь, что отвечать злословящему меня.
Pro 27:12  Благоразумный видит беду и укрывается; а неопытные идут вперед [и] наказываются.
Pro 27:13  Возьми у него платье его, потому что он поручился за чужого, и за стороннего возьми от него залог.
Pro 27:14  Кто громко хвалит друга своего с раннего утра, того сочтут за злословящего.
Pro 27:15  Непрестанная капель в дождливый день и сварливая жена--равны:
Pro 27:16  кто хочет скрыть ее, тот хочет скрыть ветер и масть в правой руке своей, дающую знать о себе.
Pro 27:17  Железо железо острит, и человек изощряет взгляд друга своего.
Pro 27:18  Кто стережет смоковницу, тот будет есть плоды ее; и кто бережет господина своего, тот будет в чести.
Pro 27:19  Как в воде лицо--к лицу, так сердце человека--к человеку.
Pro 27:20  Преисподняя и Аваддон--ненасытимы; так ненасытимы и глаза человеческие.
Pro 27:21  Что плавильня--для серебра, горнило--для золота, то для человека уста, которые хвалят его.
Pro 27:22  Толки глупого в ступе пестом вместе с зерном, не отделится от него глупость его.
Pro 27:23  Хорошо наблюдай за скотом твоим, имей попечение о стадах;
Pro 27:24  потому что [богатство] не навек, да и власть разве из рода в род?
Pro 27:25  Прозябает трава, и является зелень, и собирают горные травы.
Pro 27:26  Овцы--на одежду тебе, и козлы--на покупку поля.
Pro 27:27  И довольно козьего молока в пищу тебе, в пищу домашним твоим и на продовольствие служанкам твоим.
Pro 28:1  Нечестивый бежит, когда никто не гонится [за ним]; а праведник смел, как лев.
Pro 28:2  Когда страна отступит от закона, тогда много в ней начальников; а при разумном и знающем муже она долговечна.
Pro 28:3  Человек бедный и притесняющий слабых [то же, что] проливной дождь, смывающий хлеб.
Pro 28:4  Отступники от закона хвалят нечестивых, а соблюдающие закон негодуют на них.
Pro 28:5  Злые люди не разумеют справедливости, а ищущие Господа разумеют все.
Pro 28:6  Лучше бедный, ходящий в своей непорочности, нежели тот, кто извращает пути свои, хотя он и богат.
Pro 28:7  Хранящий закон--сын разумный, а знающийся с расточителями срамит отца своего.
Pro 28:8  Умножающий имение свое ростом и лихвою соберет его для благотворителя бедных.
Pro 28:9  Кто отклоняет ухо свое от слушания закона, того и молитва--мерзость.
Pro 28:10  Совращающий праведных на путь зла сам упадет в свою яму, а непорочные наследуют добро.
Pro 28:11  Человек богатый--мудрец в глазах своих, но умный бедняк обличит его.
Pro 28:12  Когда торжествуют праведники, великая слава, но когда возвышаются нечестивые, люди укрываются.
Pro 28:13  Скрывающий свои преступления не будет иметь успеха; а кто сознается и оставляет их, тот будет помилован.
Pro 28:14  Блажен человек, который всегда пребывает в благоговении; а кто ожесточает сердце свое, тот попадет в беду.
Pro 28:15  Как рыкающий лев и голодный медведь, так нечестивый властелин над бедным народом.
Pro 28:16  Неразумный правитель много делает притеснений, а ненавидящий корысть продолжит дни.
Pro 28:17  Человек, виновный в пролитии человеческой крови, будет бегать до могилы, чтобы кто не схватил его.
Pro 28:18  Кто ходит непорочно, то будет невредим; а ходящий кривыми путями упадет на одном из них.
Pro 28:19  Кто возделывает землю свою, тот будет насыщаться хлебом, а кто подражает праздным, тот насытится нищетою.
Pro 28:20  Верный человек богат благословениями, а кто спешит разбогатеть, тот не останется ненаказанным.
Pro 28:21  Быть лицеприятным--нехорошо: такой человек и за кусок хлеба сделает неправду.
Pro 28:22  Спешит к богатству завистливый человек, и не думает, что нищета постигнет его.
Pro 28:23  Обличающий человека найдет после большую приязнь, нежели тот, кто льстит языком.
Pro 28:24  Кто обкрадывает отца своего и мать свою и говорит: `это не грех', тот--сообщник грабителям.
Pro 28:25  Надменный разжигает ссору, а надеющийся на Господа будет благоденствовать.
Pro 28:26  Кто надеется на себя, тот глуп; а кто ходит в мудрости, тот будет цел.
Pro 28:27  Дающий нищему не обеднеет; а кто закрывает глаза свои от него, на том много проклятий.
Pro 28:28  Когда возвышаются нечестивые, люди укрываются, а когда они падают, умножаются праведники.
Pro 29:1  Человек, который, будучи обличаем, ожесточает выю свою, внезапно сокрушится, и не будет [ему] исцеления.
Pro 29:2  Когда умножаются праведники, веселится народ, а когда господствует нечестивый, народ стенает.
Pro 29:3  Человек, любящий мудрость, радует отца своего; а кто знается с блудницами, тот расточает имение.
Pro 29:4  Царь правосудием утверждает землю, а любящий подарки разоряет ее.
Pro 29:5  Человек, льстящий другу своему, расстилает сеть ногам его.
Pro 29:6  В грехе злого человека--сеть [для него], а праведник веселится и радуется.
Pro 29:7  Праведник тщательно вникает в тяжбу бедных, а нечестивый не разбирает дела.
Pro 29:8  Люди развратные возмущают город, а мудрые утишают мятеж.
Pro 29:9  Умный человек, судясь с человеком глупым, сердится ли, смеется ли, --не имеет покоя.
Pro 29:10  Кровожадные люди ненавидят непорочного, а праведные заботятся о его жизни.
Pro 29:11  Глупый весь гнев свой изливает, а мудрый сдерживает его.
Pro 29:12  Если правитель слушает ложные речи, то и все служащие у него нечестивы.
Pro 29:13  Бедный и лихоимец встречаются друг с другом; но свет глазам того и другого дает Господь.
Pro 29:14  Если царь судит бедных по правде, то престол его навсегда утвердится.
Pro 29:15  Розга и обличение дают мудрость; но отрок, оставленный в небрежении, делает стыд своей матери.
Pro 29:16  При умножении нечестивых умножается беззаконие; но праведники увидят падение их.
Pro 29:17  Наказывай сына твоего, и он даст тебе покой, и доставит радость душе твоей.
Pro 29:18  Без откровения свыше народ необуздан, а соблюдающий закон блажен.
Pro 29:19  Словами не научится раб, потому что, хотя он понимает [их], но не слушается.
Pro 29:20  Видал ли ты человека опрометчивого в словах своих? на глупого больше надежды, нежели на него.
Pro 29:21  Если с детства воспитывать раба в неге, то впоследствии он захочет быть сыном.
Pro 29:22  Человек гневливый заводит ссору, и вспыльчивый много грешит.
Pro 29:23  Гордость человека унижает его, а смиренный духом приобретает честь.
Pro 29:24  Кто делится с вором, тот ненавидит душу свою; слышит он проклятие, но не объявляет о том.
Pro 29:25  Боязнь пред людьми ставит сеть; а надеющийся на Господа будет безопасен.
Pro 29:26  Многие ищут [благосклонного] лица правителя, но судьба человека--от Господа.
Pro 29:27  Мерзость для праведников--человек неправедный, и мерзость для нечестивого--идущий прямым путем.
Pro 30:1  Слова Агура, сына Иакеева. Вдохновенные изречения, [которые] сказал этот человек Ифиилу, Ифиилу и Укалу:
Pro 30:2  подлинно, я более невежда, нежели кто-либо из людей, и разума человеческого нет у меня,
Pro 30:3  и не научился я мудрости, и познания святых не имею.
Pro 30:4  Кто восходил на небо и нисходил? кто собрал ветер в пригоршни свои? кто завязал воду в одежду? кто поставил все пределы земли? какое имя ему? и какое имя сыну его? знаешь ли?
Pro 30:5  Всякое слово Бога чисто; Он--щит уповающим на Него.
Pro 30:6  Не прибавляй к словам Его, чтобы Он не обличил тебя, и ты не оказался лжецом.
Pro 30:7  Двух вещей я прошу у Тебя, не откажи мне, прежде нежели я умру:
Pro 30:8  суету и ложь удали от меня, нищеты и богатства не давай мне, питай меня насущным хлебом,
Pro 30:9  дабы, пресытившись, я не отрекся [Тебя] и не сказал: `кто Господь?' и чтобы, обеднев, не стал красть и употреблять имя Бога моего всуе.
Pro 30:10  Не злословь раба пред господином его, чтобы он не проклял тебя, и ты не остался виноватым.
Pro 30:11  Есть род, который проклинает отца своего и не благословляет матери своей.
Pro 30:12  Есть род, который чист в глазах своих, тогда как не омыт от нечистот своих.
Pro 30:13  Есть род--о, как высокомерны глаза его, и как подняты ресницы его!
Pro 30:14  Есть род, у которого зубы--мечи, и челюсти--ножи, чтобы пожирать бедных на земле и нищих между людьми.
Pro 30:15  У ненасытимости две дочери: `давай, давай!' Вот три ненасытимых, и четыре, которые не скажут: `довольно!'
Pro 30:16  Преисподняя и утроба бесплодная, земля, которая не насыщается водою, и огонь, который не говорит: `довольно!'
Pro 30:17  Глаз, насмехающийся над отцом и пренебрегающий покорностью к матери, выклюют вороны дольные, и сожрут птенцы орлиные!
Pro 30:18  Три вещи непостижимы для меня, и четырех я не понимаю:
Pro 30:19  пути орла на небе, пути змея на скале, пути корабля среди моря и пути мужчины к девице.
Pro 30:20  Таков путь и жены прелюбодейной; поела и обтерла рот свой, и говорит: `я ничего худого не сделала'.
Pro 30:21  От трех трясется земля, четырех она не может носить:
Pro 30:22  раба, когда он делается царем; глупого, когда он досыта ест хлеб;
Pro 30:23  позорную женщину, когда она выходит замуж, и служанку, когда она занимает место госпожи своей.
Pro 30:24  Вот четыре малых на земле, но они мудрее мудрых:
Pro 30:25  муравьи--народ не сильный, но летом заготовляют пищу свою;
Pro 30:26  горные мыши--народ слабый, но ставят домы свои на скале;
Pro 30:27  у саранчи нет царя, но выступает вся она стройно;
Pro 30:28  паук лапками цепляется, но бывает в царских чертогах.
Pro 30:29  Вот трое имеют стройную походку, и четверо стройно выступают:
Pro 30:30  лев, силач между зверями, не посторонится ни перед кем;
Pro 30:31  конь и козел, и царь среди народа своего.
Pro 30:32  Если ты в заносчивости своей сделал глупость и помыслил злое, то [положи] руку на уста;
Pro 30:33  потому что, как сбивание молока производит масло, толчок в нос производит кровь, так и возбуждение гнева производит ссору.
Pro 31:1  Слова Лемуила царя. Наставление, которое преподала ему мать его:
Pro 31:2  что, сын мой? что, сын чрева моего? что, сын обетов моих?
Pro 31:3  Не отдавай женщинам сил твоих, ни путей твоих губительницам царей.
Pro 31:4  Не царям, Лемуил, не царям пить вино, и не князьям--сикеру,
Pro 31:5  чтобы, напившись, они не забыли закона и не превратили суда всех угнетаемых.
Pro 31:6  Дайте сикеру погибающему и вино огорченному душею;
Pro 31:7  пусть он выпьет и забудет бедность свою и не вспомнит больше о своем страдании.
Pro 31:8  Открывай уста твои за безгласного и для защиты всех сирот.
Pro 31:9  Открывай уста твои для правосудия и для дела бедного и нищего.
Pro 31:10  Кто найдет добродетельную жену? цена ее выше жемчугов;
Pro 31:11  уверено в ней сердце мужа ее, и он не останется без прибытка;
Pro 31:12  она воздает ему добром, а не злом, во все дни жизни своей.
Pro 31:13  Добывает шерсть и лен, и с охотою работает своими руками.
Pro 31:14  Она, как купеческие корабли, издалека добывает хлеб свой.
Pro 31:15  Она встает еще ночью и раздает пищу в доме своем и урочное служанкам своим.
Pro 31:16  Задумает она о поле, и приобретает его; от плодов рук своих насаждает виноградник.
Pro 31:17  Препоясывает силою чресла свои и укрепляет мышцы свои.
Pro 31:18  Она чувствует, что занятие ее хорошо, и--светильник ее не гаснет и ночью.
Pro 31:19  Протягивает руки свои к прялке, и персты ее берутся за веретено.
Pro 31:20  Длань свою она открывает бедному, и руку свою подает нуждающемуся.
Pro 31:21  Не боится стужи для семьи своей, потому что вся семья ее одета в двойные одежды.
Pro 31:22  Она делает себе ковры; виссон и пурпур--одежда ее.
Pro 31:23  Муж ее известен у ворот, когда сидит со старейшинами земли.
Pro 31:24  Она делает покрывала и продает, и поясы доставляет купцам Финикийским.
Pro 31:25  Крепость и красота--одежда ее, и весело смотрит она на будущее.
Pro 31:26  Уста свои открывает с мудростью, и кроткое наставление на языке ее.
Pro 31:27  Она наблюдает за хозяйством в доме своем и не ест хлеба праздности.
Pro 31:28  Встают дети и ублажают ее, --муж, и хвалит ее:
Pro 31:29  `много было жен добродетельных, но ты превзошла всех их'.
Pro 31:30  Миловидность обманчива и красота суетна; но жена, боящаяся Господа, достойна хвалы.
Pro 31:31  Дайте ей от плода рук ее, и да прославят ее у ворот дела ее!


\end{document}