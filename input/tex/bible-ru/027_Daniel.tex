\begin{document}

\title{Пр. Даниила}


\chapter{1}

\par 1 В третий год царствования Иоакима, царя Иудейского, пришел Навуходоносор, царь Вавилонский, к Иерусалиму и осадил его.
\par 2 И предал Господь в руку его Иоакима, царя Иудейского, и часть сосудов дома Божия, и он отправил их в землю Сеннаар, в дом бога своего, и внес эти сосуды в сокровищницу бога своего.
\par 3 И сказал царь Асфеназу, начальнику евнухов своих, чтобы он из сынов Израилевых, из рода царского и княжеского, привел
\par 4 отроков, у которых нет никакого телесного недостатка, красивых видом, и понятливых для всякой науки, и разумеющих науки, и смышленых и годных служить в чертогах царских, и чтобы научил их книгам и языку Халдейскому.
\par 5 И назначил им царь ежедневную пищу с царского стола и вино, которое сам пил, и велел воспитывать их три года, по истечении которых они должны были предстать пред царя.
\par 6 Между ними были из сынов Иудиных Даниил, Анания, Мисаил и Азария.
\par 7 И переименовал их начальник евнухов--Даниила Валтасаром, Ананию Седрахом, Мисаила Мисахом и Азарию Авденаго.
\par 8 Даниил положил в сердце своем не оскверняться яствами со стола царского и вином, какое пьет царь, и потому просил начальника евнухов о том, чтобы не оскверняться ему.
\par 9 Бог даровал Даниилу милость и благорасположение начальника евнухов;
\par 10 и начальник евнухов сказал Даниилу: боюсь я господина моего, царя, который сам назначил вам пищу и питье; если он увидит лица ваши худощавее, нежели у отроков, сверстников ваших, то вы сделаете голову мою виновною перед царем.
\par 11 Тогда сказал Даниил Амелсару, которого начальник евнухов приставил к Даниилу, Анании, Мисаилу и Азарии:
\par 12 сделай опыт над рабами твоими в течение десяти дней; пусть дают нам в пищу овощи и воду для питья;
\par 13 и потом пусть явятся перед тобою лица наши и лица тех отроков, которые питаются царскою пищею, и затем поступай с рабами твоими, как увидишь.
\par 14 Он послушался их в этом и испытывал их десять дней.
\par 15 По истечении же десяти дней лица их оказались красивее, и телом они были полнее всех тех отроков, которые питались царскими яствами.
\par 16 Тогда Амелсар брал их кушанье и вино для питья и давал им овощи.
\par 17 И даровал Бог четырем сим отрокам знание и разумение всякой книги и мудрости, а Даниилу еще даровал разуметь и всякие видения и сны.
\par 18 По окончании тех дней, когда царь приказал представить их, начальник евнухов представил их Навуходоносору.
\par 19 И царь говорил с ними, и из всех [отроков] не нашлось подобных Даниилу, Анании, Мисаилу и Азарии, и стали они служить пред царем.
\par 20 И во всяком деле мудрого уразумения, о чем ни спрашивал их царь, он находил их в десять раз выше всех тайноведцев и волхвов, какие были во всем царстве его.
\par 21 И был там Даниил до первого года царя Кира.

\chapter{2}

\par 1 Во второй год царствования Навуходоносора снились Навуходоносору сны, и возмутился дух его, и сон удалился от него.
\par 2 И велел царь созвать тайноведцев, и гадателей, и чародеев, и Халдеев, чтобы они рассказали царю сновидения его. Они пришли, и стали перед царем.
\par 3 И сказал им царь: сон снился мне, и тревожится дух мой; желаю знать этот сон.
\par 4 И сказали Халдеи царю по-арамейски: царь! вовеки живи! скажи сон рабам твоим, и мы объясним значение его.
\par 5 Отвечал царь и сказал Халдеям: слово отступило от меня; если вы не скажете мне сновидения и значения его, то в куски будете изрублены, и домы ваши обратятся в развалины.
\par 6 Если же расскажете сон и значение его, то получите от меня дары, награду и великую почесть; итак скажите мне сон и значение его.
\par 7 Они вторично отвечали и сказали: да скажет царь рабам своим сновидение, и мы объясним его значение.
\par 8 Отвечал царь и сказал: верно знаю, что вы хотите выиграть время, потому что видите, что слово отступило от меня.
\par 9 Так как вы не объявляете мне сновидения, то у вас один умысел: вы собираетесь сказать мне ложь и обман, пока минет время; итак расскажите мне сон, и тогда я узнаю, что вы можете объяснить мне и значение его.
\par 10 Халдеи отвечали царю и сказали: нет на земле человека, который мог бы открыть это дело царю, и потому ни один царь, великий и могущественный, не требовал подобного ни от какого тайноведца, гадателя и Халдея.
\par 11 Дело, которого царь требует, так трудно, что никто другой не может открыть его царю, кроме богов, которых обитание не с плотью.
\par 12 Рассвирепел царь и сильно разгневался на это, и приказал истребить всех мудрецов Вавилонских.
\par 13 Когда вышло это повеление, чтобы убивать мудрецов, искали Даниила и товарищей его, чтобы умертвить их.
\par 14 Тогда Даниил обратился с советом и мудростью к Ариоху, начальнику царских телохранителей, который вышел убивать мудрецов Вавилонских;
\par 15 и спросил Ариоха, сильного при царе: `почему такое грозное повеление от царя?' Тогда Ариох рассказал все дело Даниилу.
\par 16 И Даниил вошел, и упросил царя дать ему время, и он представит царю толкование [сна].
\par 17 Даниил пришел в дом свой, и рассказал дело Анании, Мисаилу и Азарии, товарищам своим,
\par 18 чтобы они просили милости у Бога небесного об этой тайне, дабы Даниил и товарищи его не погибли с прочими мудрецами Вавилонскими.
\par 19 И тогда открыта была тайна Даниилу в ночном видении, и Даниил благословил Бога небесного.
\par 20 И сказал Даниил: да будет благословенно имя Господа от века и до века! ибо у Него мудрость и сила;
\par 21 он изменяет времена и лета, низлагает царей и поставляет царей; дает мудрость мудрым и разумение разумным;
\par 22 он открывает глубокое и сокровенное, знает, что во мраке, и свет обитает с Ним.
\par 23 Славлю и величаю Тебя, Боже отцов моих, что Ты даровал мне мудрость и силу и открыл мне то, о чем мы молили Тебя; ибо Ты открыл нам дело царя.
\par 24 После сего Даниил вошел к Ариоху, которому царь повелел умертвить мудрецов Вавилонских, пришел и сказал ему: не убивай мудрецов Вавилонских; введи меня к царю, и я открою значение [сна].
\par 25 Тогда Ариох немедленно привел Даниила к царю и сказал ему: я нашел из пленных сынов Иудеи человека, который может открыть царю значение [сна].
\par 26 Царь сказал Даниилу, который назван был Валтасаром: можешь ли ты сказать мне сон, который я видел, и значение его?
\par 27 Даниил отвечал царю и сказал: тайны, о которой царь спрашивает, не могут открыть царю ни мудрецы, ни обаятели, ни тайноведцы, ни гадатели.
\par 28 Но есть на небесах Бог, открывающий тайны; и Он открыл царю Навуходоносору, что будет в последние дни. Сон твой и видения главы твоей на ложе твоем были такие:
\par 29 ты, царь, на ложе твоем думал о том, что будет после сего? и Открывающий тайны показал тебе то, что будет.
\par 30 А мне тайна сия открыта не потому, чтобы я был мудрее всех живущих, но для того, чтобы открыто было царю разумение и чтобы ты узнал помышления сердца твоего.
\par 31 Тебе, царь, было такое видение: вот, какой-то большой истукан; огромный был этот истукан, в чрезвычайном блеске стоял он пред тобою, и страшен был вид его.
\par 32 У этого истукана голова была из чистого золота, грудь его и руки его--из серебра, чрево его и бедра его медные,
\par 33 голени его железные, ноги его частью железные, частью глиняные.
\par 34 Ты видел его, доколе камень не оторвался от горы без содействия рук, ударил в истукана, в железные и глиняные ноги его, и разбил их.
\par 35 Тогда все вместе раздробилось: железо, глина, медь, серебро и золото сделались как прах на летних гумнах, и ветер унес их, и следа не осталось от них; а камень, разбивший истукана, сделался великою горою и наполнил всю землю.
\par 36 Вот сон! Скажем пред царем и значение его.
\par 37 Ты, царь, царь царей, которому Бог небесный даровал царство, власть, силу и славу,
\par 38 и всех сынов человеческих, где бы они ни жили, зверей земных и птиц небесных Он отдал в твои руки и поставил тебя владыкою над всеми ими. Ты--это золотая голова!
\par 39 После тебя восстанет другое царство, ниже твоего, и еще третье царство, медное, которое будет владычествовать над всею землею.
\par 40 А четвертое царство будет крепко, как железо; ибо как железо разбивает и раздробляет все, так и оно, подобно всесокрушающему железу, будет раздроблять и сокрушать.
\par 41 А что ты видел ноги и пальцы на ногах частью из глины горшечной, а частью из железа, то будет царство разделенное, и в нем останется несколько крепости железа, так как ты видел железо, смешанное с горшечною глиною.
\par 42 И как персты ног были частью из железа, а частью из глины, так и царство будет частью крепкое, частью хрупкое.
\par 43 А что ты видел железо, смешанное с глиною горшечною, это значит, что они смешаются через семя человеческое, но не сольются одно с другим, как железо не смешивается с глиною.
\par 44 И во дни тех царств Бог небесный воздвигнет царство, которое вовеки не разрушится, и царство это не будет передано другому народу; оно сокрушит и разрушит все царства, а само будет стоять вечно,
\par 45 так как ты видел, что камень отторгнут был от горы не руками и раздробил железо, медь, глину, серебро и золото. Великий Бог дал знать царю, что будет после сего. И верен этот сон, и точно истолкование его!
\par 46 Тогда царь Навуходоносор пал на лице свое и поклонился Даниилу, и велел принести ему дары и благовонные курения.
\par 47 И сказал царь Даниилу: истинно Бог ваш есть Бог богов и Владыка царей, открывающий тайны, когда ты мог открыть эту тайну!
\par 48 Тогда возвысил царь Даниила и дал ему много больших подарков, и поставил его над всею областью Вавилонскою и главным начальником над всеми мудрецами Вавилонскими.
\par 49 Но Даниил просил царя, и он поставил Седраха, Мисаха и Авденаго над делами страны Вавилонской, а Даниил остался при дворе царя.

\chapter{3}

\par 1 Царь Навуходоносор сделал золотой истукан, вышиною в шестьдесят локтей, шириною в шесть локтей, поставил его на поле Деире, в области Вавилонской.
\par 2 И послал царь Навуходоносор собрать сатрапов, наместников, воевод, верховных судей, казнохранителей, законоведцев, блюстителей суда и всех областных правителей, чтобы они пришли на торжественное открытие истукана, которого поставил царь Навуходоносор.
\par 3 И собрались сатрапы, наместники, военачальники, верховные судьи, казнохранители, законоведцы, блюстители суда и все областные правители на открытие истукана, которого Навуходоносор царь поставил, и стали перед истуканом, которого воздвиг Навуходоносор.
\par 4 Тогда глашатай громко воскликнул: объявляется вам, народы, племена и языки:
\par 5 в то время, как услышите звук трубы, свирели, цитры, цевницы, гуслей и симфонии и всяких музыкальных орудий, падите и поклонитесь золотому истукану, которого поставил царь Навуходоносор.
\par 6 А кто не падет и не поклонится, тотчас брошен будет в печь, раскаленную огнем.
\par 7 Посему, когда все народы услышали звук трубы, свирели, цитры, цевницы, гуслей и всякого рода музыкальных орудий, то пали все народы, племена и языки, и поклонились золотому истукану, которого поставил Навуходоносор царь.
\par 8 В это самое время приступили некоторые из Халдеев и донесли на Иудеев.
\par 9 Они сказали царю Навуходоносору: царь, вовеки живи!
\par 10 Ты, царь, дал повеление, чтобы каждый человек, который услышит звук трубы, свирели, цитры, цевницы, гуслей и симфонии и всякого рода музыкальных орудий, пал и поклонился золотому истукану;
\par 11 а кто не падет и не поклонится, тот должен быть брошен в печь, раскаленную огнем.
\par 12 Есть мужи Иудейские, которых ты поставил над делами страны Вавилонской: Седрах, Мисах и Авденаго; эти мужи не повинуются повелению твоему, царь, богам твоим не служат и золотому истукану, которого ты поставил, не поклоняются.
\par 13 Тогда Навуходоносор во гневе и ярости повелел привести Седраха, Мисаха и Авденаго; и приведены были эти мужи к царю.
\par 14 Навуходоносор сказал им: с умыслом ли вы, Седрах, Мисах и Авденаго, богам моим не служите, и золотому истукану, которого я поставил, не поклоняетесь?
\par 15 Отныне, если вы готовы, как скоро услышите звук трубы, свирели, цитры, цевницы, гуслей, симфонии и всякого рода музыкальных орудий, падите и поклонитесь истукану, которого я сделал; если же не поклонитесь, то в тот же час брошены будете в печь, раскаленную огнем, и тогда какой Бог избавит вас от руки моей?
\par 16 И отвечали Седрах, Мисах и Авденаго, и сказали царю Навуходоносору: нет нужды нам отвечать тебе на это.
\par 17 Бог наш, Которому мы служим, силен спасти нас от печи, раскаленной огнем, и от руки твоей, царь, избавит.
\par 18 Если же и не будет того, то да будет известно тебе, царь, что мы богам твоим служить не будем и золотому истукану, которого ты поставил, не поклонимся.
\par 19 Тогда Навуходоносор исполнился ярости, и вид лица его изменился на Седраха, Мисаха и Авденаго, и он повелел разжечь печь в семь раз сильнее, нежели как обыкновенно разжигали ее,
\par 20 и самым сильным мужам из войска своего приказал связать Седраха, Мисаха и Авденаго и бросить их в печь, раскаленную огнем.
\par 21 Тогда мужи сии связаны были в исподнем и верхнем платье своем, в головных повязках и в прочих одеждах своих, и брошены в печь, раскаленную огнем.
\par 22 И как повеление царя было строго, и печь раскалена была чрезвычайно, то пламя огня убило тех людей, которые бросали Седраха, Мисаха и Авденаго.
\par 23 А сии три мужа, Седрах, Мисах и Авденаго, упали в раскаленную огнем печь связанные.
\par 24 Навуходоносор царь изумился, и поспешно встал, и сказал вельможам своим: не троих ли мужей бросили мы в огонь связанными? Они в ответ сказали царю: истинно так, царь!
\par 25 На это он сказал: вот, я вижу четырех мужей несвязанных, ходящих среди огня, и нет им вреда; и вид четвертого подобен сыну Божию.
\par 26 Тогда подошел Навуходоносор к устью печи, раскаленной огнем, и сказал: Седрах, Мисах и Авденаго, рабы Бога Всевышнего! выйдите и подойдите! Тогда Седрах, Мисах и Авденаго вышли из среды огня.
\par 27 И, собравшись, сатрапы, наместники, военачальники и советники царя усмотрели, что над телами мужей сих огонь не имел силы, и волосы на голове не опалены, и одежды их не изменились, и даже запаха огня не было от них.
\par 28 Тогда Навуходоносор сказал: благословен Бог Седраха, Мисаха и Авденаго,  Который послал Ангела Своего  и избавил рабов Своих,  которые надеялись на Него и не послушались царского повеления, и предали тела свои [огню], чтобы не служить и не поклоняться иному богу, кроме Бога своего!
\par 29 И от меня дается повеление, чтобы из всякого народа, племени и языка кто произнесет хулу на Бога Седраха, Мисаха и Авденаго, был изрублен в куски, и дом его обращен в развалины, ибо нет иного бога, который мог бы так спасать.
\par 30 Тогда царь возвысил Седраха, Мисаха и Авденаго в стране Вавилонской.
\par 31 Навуходоносор царь всем народам, племенам и языкам, живущим
\par 32 Знамения и чудеса, какие совершил надо мною Всевышний Бог, угодно мне возвестить вам.
\par 33 Как велики знамения Его и как могущественны чудеса Его! Царство Его--царство вечное, и владычество Его--в роды и роды.

\chapter{4}

\par 1 Я, Навуходоносор, спокоен был в доме моем и благоденствовал в чертогах моих.
\par 2 Но я видел сон, который устрашил меня, и размышления на ложе моем и видения головы моей смутили меня.
\par 3 И дано было мною повеление привести ко мне всех мудрецов Вавилонских, чтобы они сказали мне значение сна.
\par 4 Тогда пришли тайноведцы, обаятели, Халдеи и гадатели; я рассказал им сон, но они не могли мне объяснить значения его.
\par 5 Наконец вошел ко мне Даниил, которому имя было Валтасар, по имени бога моего, и в котором дух святаго Бога; ему рассказал я сон.
\par 6 Валтасар, глава мудрецов! я знаю, что в тебе дух святаго Бога, и никакая тайна не затрудняет тебя; объясни мне видения сна моего, который я видел, и значение его.
\par 7 Видения же головы моей на ложе моем были такие: я видел, вот, среди земли дерево весьма высокое.
\par 8 Большое было это дерево и крепкое, и высота его достигала до неба, и оно видимо было до краев всей земли.
\par 9 Листья его прекрасные, и плодов на нем множество, и пища на нем для всех; под ним находили тень полевые звери, и в ветвях его гнездились птицы небесные, и от него питалась всякая плоть.
\par 10 И видел я в видениях головы моей на ложе моем, и вот, нисшел с небес Бодрствующий и Святый.
\par 11 Воскликнув громко, Он сказал: `срубите это дерево, обрубите ветви его, стрясите листья с него и разбросайте плоды его; пусть удалятся звери из-под него и птицы с ветвей его;
\par 12 но главный корень его оставьте в земле, и пусть он в узах железных и медных среди полевой травы орошается небесною росою, и с животными пусть будет часть его в траве земной.
\par 13 Сердце человеческое отнимется от него и дастся ему сердце звериное, и пройдут над ним семь времен.
\par 14 Повелением Бодрствующих это определено, и по приговору Святых назначено, дабы знали живущие, что Всевышний владычествует над царством человеческим, и дает его, кому хочет, и поставляет над ним уничиженного между людьми'.
\par 15 Такой сон видел я, царь Навуходоносор; а ты, Валтасар, скажи значение его, так как никто из мудрецов в моем царстве не мог объяснить его значения, а ты можешь, потому что дух святаго Бога в тебе.
\par 16 Тогда Даниил, которому имя Валтасар, около часа пробыл в изумлении, и мысли его смущали его. Царь начал говорить и сказал: Валтасар! да не смущает тебя этот сон и значение его. Валтасар отвечал и сказал: господин мой! твоим бы ненавистникам этот сон, и врагам твоим значение его!
\par 17 Дерево, которое ты видел, которое было большое и крепкое, высотою своею достигало до небес и видимо было по всей земле,
\par 18 на котором листья были прекрасные и множество плодов и пропитание для всех, под которым обитали звери полевые и в ветвях которого гнездились птицы небесные,
\par 19 это ты, царь, возвеличившийся и укрепившийся, и величие твое возросло и достигло до небес, и власть твоя--до краев земли.
\par 20 А что царь видел Бодрствующего и Святаго, сходящего с небес, Который сказал: `срубите дерево и истребите его, только главный корень его оставьте в земле, и пусть он в узах железных и медных, среди полевой травы, орошается росою небесною, и с полевыми зверями пусть будет часть его, доколе не пройдут над ним семь времен', --
\par 21 то вот значение этого, царь, и вот определение Всевышнего, которое постигнет господина моего, царя:
\par 22 тебя отлучат от людей, и обитание твое будет с полевыми зверями; травою будут кормить тебя, как вола, росою небесною ты будешь орошаем, и семь времен пройдут над тобою, доколе познаешь, что Всевышний владычествует над царством человеческим и дает его, кому хочет.
\par 23 А что повелено было оставить главный корень дерева, это значит, что царство твое останется при тебе, когда ты познаешь власть небесную.
\par 24 Посему, царь, да будет благоугоден тебе совет мой: искупи грехи твои правдою и беззакония твои милосердием к бедным; вот чем может продлиться мир твой.
\par 25 Все это сбылось над царем Навуходоносором.
\par 26 По прошествии двенадцати месяцев, расхаживая по царским чертогам в Вавилоне,
\par 27 царь сказал: это ли не величественный Вавилон, который построил я в дом царства силою моего могущества и в славу моего величия!
\par 28 Еще речь сия была в устах царя, как был с неба голос: `тебе говорят, царь Навуходоносор: царство отошло от тебя!
\par 29 И отлучат тебя от людей, и будет обитание твое с полевыми зверями; травою будут кормить тебя, как вола, и семь времен пройдут над тобою, доколе познаешь, что Всевышний владычествует над царством человеческим и дает его, кому хочет!'
\par 30 Тотчас и исполнилось это слово над Навуходоносором, и отлучен он был от людей, ел траву, как вол, и орошалось тело его росою небесною, так что волосы у него выросли как у льва, и ногти у него--как у птицы.
\par 31 По окончании же дней тех, я, Навуходоносор, возвел глаза мои к небу, и разум мой возвратился ко мне; и благословил я Всевышнего, восхвалил и прославил Присносущего, Которого владычество--владычество вечное, и Которого царство--в роды и роды.
\par 32 И все, живущие на земле, ничего не значат; по воле Своей Он действует как в небесном воинстве, так и у живущих на земле; и нет никого, кто мог бы противиться руке Его и сказать Ему: `что Ты сделал?'
\par 33 В то время возвратился ко мне разум мой, и к славе царства моего возвратились ко мне сановитость и прежний вид мой; тогда взыскали меня советники мои и вельможи мои, и я восстановлен на царство мое, и величие мое еще более возвысилось.
\par 34 Ныне я, Навуходоносор, славлю, превозношу и величаю Царя Небесного, Которого все дела истинны и пути праведны, и Который силен смирить ходящих гордо.

\chapter{5}

\par 1 Валтасар царь сделал большое пиршество для тысячи вельмож своих и перед глазами тысячи пил вино.
\par 2 Вкусив вина, Валтасар приказал принести золотые и серебряные сосуды, которые Навуходоносор, отец его, вынес из храма Иерусалимского, чтобы пить из них царю, вельможам его, женам его и наложницам его.
\par 3 Тогда принесли золотые сосуды, которые взяты были из святилища дома Божия в Иерусалиме; и пили из них царь и вельможи его, жены его и наложницы его.
\par 4 Пили вино, и славили богов золотых и серебряных, медных, железных, деревянных и каменных.
\par 5 В тот самый час вышли персты руки человеческой и писали против лампады на извести стены чертога царского, и царь видел кисть руки, которая писала.
\par 6 Тогда царь изменился в лице своем; мысли его смутили его, связи чресл его ослабели, и колени его стали биться одно о другое.
\par 7 Сильно закричал царь, чтобы привели обаятелей, Халдеев и гадателей. Царь начал говорить, и сказал мудрецам Вавилонским: кто прочитает это написанное и объяснит мне значение его, тот будет облечен в багряницу, и золотая цепь будет на шее у него, и третьим властелином будет в царстве.
\par 8 И вошли все мудрецы царя, но не могли прочитать написанного и объяснить царю значения его.
\par 9 Царь Валтасар чрезвычайно встревожился, и вид лица его изменился на нем, и вельможи его смутились.
\par 10 Царица же, по поводу слов царя и вельмож его, вошла в палату пиршества; начала говорить царица и сказала: царь, вовеки живи! да не смущают тебя мысли твои, и да не изменяется вид лица твоего!
\par 11 Есть в царстве твоем муж, в котором дух святаго Бога; во дни отца твоего найдены были в нем свет, разум и мудрость, подобная мудрости богов, и царь Навуходоносор, отец твой, поставил его главою тайноведцев, обаятелей, Халдеев и гадателей, --сам отец твой, царь,
\par 12 потому что в нем, в Данииле, которого царь переименовал Валтасаром, оказались высокий дух, ведение и разум, способный изъяснять сны, толковать загадочное и разрешать узлы. Итак пусть призовут Даниила и он объяснит значение.
\par 13 Тогда введен был Даниил пред царя, и царь начал речь и сказал Даниилу: ты ли Даниил, один из пленных сынов Иудейских, которых отец мой, царь, привел из Иудеи?
\par 14 Я слышал о тебе, что дух Божий в тебе и свет, и разум, и высокая мудрость найдена в тебе.
\par 15 Вот, приведены были ко мне мудрецы и обаятели, чтобы прочитать это написанное и объяснить мне значение его; но они не могли объяснить мне этого.
\par 16 А о тебе я слышал, что ты можешь объяснять значение и разрешать узлы; итак, если можешь прочитать это написанное и объяснить мне значение его, то облечен будешь в багряницу, и золотая цепь будет на шее твоей, и третьим властелином будешь в царстве.
\par 17 Тогда отвечал Даниил, и сказал царю: дары твои пусть останутся у тебя, и почести отдай другому; а написанное я прочитаю царю и значение объясню ему.
\par 18 Царь! Всевышний Бог даровал отцу твоему Навуходоносору царство, величие, честь и славу.
\par 19 Пред величием, которое Он дал ему, все народы, племена и языки трепетали и страшились его: кого хотел, он убивал, и кого хотел, оставлял в живых; кого хотел, возвышал, и кого хотел, унижал.
\par 20 Но когда сердце его надмилось и дух его ожесточился до дерзости, он был свержен с царского престола своего и лишен славы своей,
\par 21 и отлучен был от сынов человеческих, и сердце его уподобилось звериному, и жил он с дикими ослами; кормили его травою, как вола, и тело его орошаемо было небесною росою, доколе он познал, что над царством человеческим владычествует Всевышний Бог и поставляет над ним, кого хочет.
\par 22 И ты, сын его Валтасар, не смирил сердца твоего, хотя знал все это,
\par 23 но вознесся против Господа небес, и сосуды дома Его принесли к тебе, и ты и вельможи твои, жены твои и наложницы твои пили из них вино, и ты славил богов серебряных и золотых, медных, железных, деревянных и каменных, которые ни видят, ни слышат, ни разумеют; а Бога, в руке Которого дыхание твое и у Которого все пути твои, ты не прославил.
\par 24 За это и послана от Него кисть руки, и начертано это писание.
\par 25 И вот что начертано: мене, мене, текел, упарсин.
\par 26 Вот и значение слов: мене--исчислил Бог царство твое и положил конец ему;
\par 27 Текел--ты взвешен на весах и найден очень легким;
\par 28 Перес--разделено царство твое и дано Мидянам и Персам.
\par 29 Тогда по повелению Валтасара облекли Даниила в багряницу и возложили золотую цепь на шею его, и провозгласили его третьим властелином в царстве.
\par 30 В ту же самую ночь Валтасар, царь Халдейский, был убит,
\par 31 и Дарий Мидянин принял царство, будучи шестидесяти двух лет.

\chapter{6}

\par 1 Угодно было Дарию поставить над царством сто двадцать сатрапов, чтобы они были во всем царстве,
\par 2 а над ними трех князей, --из которых один был Даниил, --чтобы сатрапы давали им отчет и чтобы царю не было никакого обременения.
\par 3 Даниил превосходил прочих князей и сатрапов, потому что в нем был высокий дух, и царь помышлял уже поставить его над всем царством.
\par 4 Тогда князья и сатрапы начали искать предлога к обвинению Даниила по управлению царством; но никакого предлога и погрешностей не могли найти, потому что он был верен, и никакой погрешности или вины не оказывалось в нем.
\par 5 И эти люди сказали: не найти нам предлога против Даниила, если мы не найдем его против него в законе Бога его.
\par 6 Тогда эти князья и сатрапы приступили к царю и так сказали ему: царь Дарий! вовеки живи!
\par 7 Все князья царства, наместники, сатрапы, советники и военачальники согласились между собою, чтобы сделано было царское постановление и издано повеление, чтобы, кто в течение тридцати дней будет просить какого-либо бога или человека, кроме тебя, царь, того бросить в львиный ров.
\par 8 Итак утверди, царь, это определение и подпиши указ, чтобы он был неизменен, как закон Мидийский и Персидский, и чтобы он не был нарушен.
\par 9 Царь Дарий подписал указ и это повеление.
\par 10 Даниил же, узнав, что подписан такой указ, пошел в дом свой; окна же в горнице его были открыты против Иерусалима, и он три раза в день преклонял колени, и молился своему Богу, и славословил Его, как это делал он и прежде того.
\par 11 Тогда эти люди подсмотрели и нашли Даниила молящегося и просящего милости пред Богом своим,
\par 12 потом пришли и сказали царю о царском повелении: не ты ли подписал указ, чтобы всякого человека, который в течение тридцати дней будет просить какого-либо бога или человека, кроме тебя, царь, бросать в львиный ров? Царь отвечал и сказал: это слово твердо, как закон Мидян и Персов, не допускающий изменения.
\par 13 Тогда отвечали они и сказали царю, что Даниил, который из пленных сынов Иудеи, не обращает внимания ни на тебя, царь, ни на указ, тобою подписанный, но три раза в день молится своими молитвами.
\par 14 Царь, услышав это, сильно опечалился и положил в сердце своем спасти Даниила, и даже до захождения солнца усиленно старался избавить его.
\par 15 Но те люди приступили к царю и сказали ему: знай, царь, что по закону Мидян и Персов никакое определение или постановление, утвержденное царем, не может быть изменено.
\par 16 Тогда царь повелел, и привели Даниила, и бросили в ров львиный; при этом царь сказал Даниилу: Бог твой, Которому ты неизменно служишь, Он спасет тебя!
\par 17 И принесен был камень и положен на отверстие рва, и царь запечатал его перстнем своим, и перстнем вельмож своих, чтобы ничто не переменилось в распоряжении о Данииле.
\par 18 Затем царь пошел в свой дворец, лег спать без ужина, и даже не велел вносить к нему пищи, и сон бежал от него.
\par 19 Поутру же царь встал на рассвете и поспешно пошел ко рву львиному,
\par 20 и, подойдя ко рву, жалобным голосом кликнул Даниила, и сказал царь Даниилу: Даниил, раб Бога живаго! Бог твой, Которому ты неизменно служишь, мог ли спасти тебя от львов?
\par 21 Тогда Даниил сказал царю: царь! вовеки живи!
\par 22 Бог мой послал Ангела Своего и заградил пасть львам, и они не повредили мне, потому что я оказался пред Ним чист, да и перед тобою, царь, я не сделал преступления.
\par 23 Тогда царь чрезвычайно возрадовался о нем и повелел поднять Даниила изо рва; и поднят был Даниил изо рва, и никакого повреждения не оказалось на нем, потому что он веровал в Бога своего.
\par 24 И приказал царь, и приведены были те люди, которые обвиняли Даниила, и брошены в львиный ров, как они сами, так и дети их и жены их; и они не достигли до дна рва, как львы овладели ими и сокрушили все кости их.
\par 25 После того царь Дарий написал всем народам, племенам и языкам, живущим по всей земле: `Мир вам да умножится!
\par 26 Мною дается повеление, чтобы во всякой области царства моего трепетали и благоговели пред Богом Данииловым, потому что Он есть Бог живый и присносущий, и царство Его несокрушимо, и владычество Его бесконечно.
\par 27 Он избавляет и спасает, и совершает чудеса и знамения на небе и на земле; Он избавил Даниила от силы львов'.
\par 28 И Даниил благоуспевал и в царствование Дария, и в царствование Кира Персидского.

\chapter{7}

\par 1 В первый год Валтасара, царя Вавилонского, Даниил видел сон и пророческие видения головы своей на ложе своем. Тогда он записал этот сон, изложив сущность дела.
\par 2 Начав речь, Даниил сказал: видел я в ночном видении моем, и вот, четыре ветра небесных боролись на великом море,
\par 3 и четыре больших зверя вышли из моря, непохожие один на другого.
\par 4 Первый--как лев, но у него крылья орлиные; я смотрел, доколе не вырваны были у него крылья, и он поднят был от земли, и стал на ноги, как человек, и сердце человеческое дано ему.
\par 5 И вот еще зверь, второй, похожий на медведя, стоял с одной стороны, и три клыка во рту у него, между зубами его; ему сказано так: `встань, ешь мяса много!'
\par 6 Затем видел я, вот еще зверь, как барс; на спине у него четыре птичьих крыла, и четыре головы были у зверя сего, и власть дана была ему.
\par 7 После сего видел я в ночных видениях, и вот зверь четвертый, страшный и ужасный и весьма сильный; у него большие железные зубы; он пожирает и сокрушает, остатки же попирает ногами; он отличен был от всех прежних зверей, и десять рогов было у него.
\par 8 Я смотрел на эти рога, и вот, вышел между ними еще небольшой рог, и три из прежних рогов с корнем исторгнуты были перед ним, и вот, в этом роге были глаза, как глаза человеческие, и уста, говорящие высокомерно.
\par 9 Видел я, наконец, что поставлены были престолы, и воссел Ветхий днями; одеяние на Нем было бело, как снег, и волосы главы Его--как чистая волна; престол Его--как пламя огня, колеса Его--пылающий огонь.
\par 10 Огненная река выходила и проходила пред Ним; тысячи тысяч служили Ему и тьмы тем предстояли пред Ним; судьи сели, и раскрылись книги.
\par 11 Видел я тогда, что за изречение высокомерных слов, какие говорил рог, зверь был убит в глазах моих, и тело его сокрушено и предано на сожжение огню.
\par 12 И у прочих зверей отнята власть их, и продолжение жизни дано им только на время и на срок.
\par 13 Видел я в ночных видениях, вот, с облаками небесными шел как бы Сын человеческий, дошел до Ветхого днями и подведен был к Нему.
\par 14 И Ему дана власть, слава и царство, чтобы все народы, племена и языки служили Ему; владычество Его--владычество вечное, которое не прейдет, и царство Его не разрушится.
\par 15 Вострепетал дух мой во мне, Данииле, в теле моем, и видения головы моей смутили меня.
\par 16 Я подошел к одному из предстоящих и спросил у него об истинном значении всего этого, и он стал говорить со мною, и объяснил мне смысл сказанного:
\par 17 `эти большие звери, которых четыре, [означают], что четыре царя восстанут от земли.
\par 18 Потом примут царство святые Всевышнего и будут владеть царством вовек и вовеки веков'.
\par 19 Тогда пожелал я точного объяснения о четвертом звере, который был отличен от всех и очень страшен, с зубами железными и когтями медными, пожирал и сокрушал, а остатки попирал ногами,
\par 20 и о десяти рогах, которые были на голове у него, и о другом, вновь вышедшем, перед которым выпали три, о том самом роге, у которого были глаза и уста, говорящие высокомерно, и который по виду стал больше прочих.
\par 21 Я видел, как этот рог вел брань со святыми и превозмогал их,
\par 22 доколе не пришел Ветхий днями, и суд дан был святым Всевышнего, и наступило время, чтобы царством овладели святые.
\par 23 Об этом он сказал: зверь четвертый--четвертое царство будет на земле, отличное от всех царств, которое будет пожирать всю землю, попирать и сокрушать ее.
\par 24 А десять рогов значат, что из этого царства восстанут десять царей, и после них восстанет иной, отличный от прежних, и уничижит трех царей,
\par 25 и против Всевышнего будет произносить слова и угнетать святых Всевышнего; даже возмечтает отменить у них [праздничные] времена и закон, и они преданы будут в руку его до времени и времен и полувремени.
\par 26 Затем воссядут судьи и отнимут у него власть губить и истреблять до конца.
\par 27 Царство же и власть и величие царственное во всей поднебесной дано будет народу святых Всевышнего, Которого царство--царство вечное, и все властители будут служить и повиноваться Ему.
\par 28 Здесь конец слова. Меня, Даниила, сильно смущали размышления мои, и лице мое изменилось на мне; но слово я сохранил в сердце моем.

\chapter{8}

\par 1 В третий год царствования Валтасара царя явилось мне, Даниилу, видение после того, которое явилось мне прежде.
\par 2 И видел я в видении, и когда видел, я был в Сузах, престольном городе в области Еламской, и видел я в видении, --как бы я был у реки Улая.
\par 3 Поднял я глаза мои и увидел: вот, один овен стоит у реки; у него два рога, и рога высокие, но один выше другого, и высший поднялся после.
\par 4 Видел я, как этот овен бодал к западу и к северу и к югу, и никакой зверь не мог устоять против него, и никто не мог спасти от него; он делал, что хотел, и величался.
\par 5 Я внимательно смотрел на это, и вот, с запада шел козел по лицу всей земли, не касаясь земли; у этого козла был видный рог между его глазами.
\par 6 Он пошел на того овна, имеющего рога, которого я видел стоящим у реки, и бросился на него в сильной ярости своей.
\par 7 И я видел, как он, приблизившись к овну, рассвирепел на него и поразил овна, и сломил у него оба рога; и недостало силы у овна устоять против него, и он поверг его на землю и растоптал его, и не было никого, кто мог бы спасти овна от него.
\par 8 Тогда козел чрезвычайно возвеличился; но когда он усилился, то сломился большой рог, и на место его вышли четыре, обращенные на четыре ветра небесных.
\par 9 От одного из них вышел небольшой рог, который чрезвычайно разросся к югу и к востоку и к прекрасной стране,
\par 10 и вознесся до воинства небесного, и низринул на землю часть сего воинства и звезд, и попрал их,
\par 11 и даже вознесся на Вождя воинства сего, и отнята была у Него ежедневная жертва, и поругано было место святыни Его.
\par 12 И воинство предано вместе с ежедневною жертвою за нечестие, и он, повергая истину на землю, действовал и успевал.
\par 13 И услышал я одного святого говорящего, и сказал этот святой кому-то, вопрошавшему: `на сколько времени простирается это видение о ежедневной жертве и об опустошительном нечестии, когда святыня и воинство будут попираемы?'
\par 14 И сказал мне: `на две тысячи триста вечеров и утр; и тогда святилище очистится'.
\par 15 И было: когда я, Даниил, увидел это видение и искал значения его, вот, стал предо мною как облик мужа.
\par 16 И услышал я от средины Улая голос человеческий, который воззвал и сказал: `Гавриил! объясни ему это видение!'
\par 17 И он подошел к тому месту, где я стоял, и когда он пришел, я ужаснулся и пал на лице мое; и сказал он мне: `знай, сын человеческий, что видение относится к концу времени!'
\par 18 И когда он говорил со мною, я без чувств лежал лицем моим на земле; но он прикоснулся ко мне и поставил меня на место мое,
\par 19 и сказал: `вот, я открываю тебе, что будет в последние дни гнева; ибо это относится к концу определенного времени.
\par 20 Овен, которого ты видел с двумя рогами, это цари Мидийский и Персидский.
\par 21 А козел косматый--царь Греции, а большой рог, который между глазами его, это первый ее царь;
\par 22 он сломился, и вместо него вышли другие четыре: это--четыре царства восстанут из этого народа, но не с его силою.
\par 23 Под конец же царства их, когда отступники исполнят меру беззаконий своих, восстанет царь наглый и искусный в коварстве;
\par 24 и укрепится сила его, хотя и не его силою, и он будет производить удивительные опустошения и успевать и действовать и губить сильных и народ святых,
\par 25 и при уме его и коварство будет иметь успех в руке его, и сердцем своим он превознесется, и среди мира погубит многих, и против Владыки владык восстанет, но будет сокрушен--не рукою.
\par 26 Видение же о вечере и утре, о котором сказано, истинно; но ты сокрой это видение, ибо оно относится к отдаленным временам'.
\par 27 И я, Даниил, изнемог, и болел несколько дней; потом встал и начал заниматься царскими делами; я изумлен был видением сим и не понимал его.

\chapter{9}

\par 1 В первый год Дария, сына Ассуирова, из рода Мидийского, который поставлен был царем над царством Халдейским,
\par 2 в первый год царствования его я, Даниил, сообразил по книгам число лет, о котором было слово Господне к Иеремии пророку, что семьдесят лет исполнятся над опустошением Иерусалима.
\par 3 И обратил я лице мое к Господу Богу с молитвою и молением, в посте и вретище и пепле.
\par 4 И молился я Господу Богу моему, и исповедывался и сказал: `Молю Тебя, Господи Боже великий и дивный, хранящий завет и милость любящим Тебя и соблюдающим повеления Твои!
\par 5 Согрешили мы, поступали беззаконно, действовали нечестиво, упорствовали и отступили от заповедей Твоих и от постановлений Твоих;
\par 6 и не слушали рабов Твоих, пророков, которые Твоим именем говорили царям нашим, и вельможам нашим, и отцам нашим, и всему народу страны.
\par 7 У Тебя, Господи, правда, а у нас на лицах стыд, как день сей, у каждого Иудея, у жителей Иерусалима и у всего Израиля, у ближних и дальних, во всех странах, куда Ты изгнал их за отступление их, с каким они отступили от Тебя.
\par 8 Господи! у нас на лицах стыд, у царей наших, у князей наших и у отцов наших, потому что мы согрешили пред Тобою.
\par 9 А у Господа Бога нашего милосердие и прощение, ибо мы возмутились против Него
\par 10 и не слушали гласа Господа Бога нашего, чтобы поступать по законам Его, которые Он дал нам через рабов Своих, пророков.
\par 11 И весь Израиль преступил закон Твой и отвратился, чтобы не слушать гласа Твоего; и за то излились на нас проклятие и клятва, которые написаны в законе Моисея, раба Божия: ибо мы согрешили пред Ним.
\par 12 И Он исполнил слова Свои, которые изрек на нас и на судей наших, судивших нас, наведя на нас великое бедствие, какого не бывало под небесами и какое совершилось над Иерусалимом.
\par 13 Как написано в законе Моисея, так все это бедствие постигло нас; но мы не умоляли Господа Бога нашего, чтобы нам обратиться от беззаконий наших и уразуметь истину Твою.
\par 14 Наблюдал Господь это бедствие и навел его на нас: ибо праведен Господь Бог наш во всех делах Своих, которые совершает, но мы не слушали гласа Его.
\par 15 И ныне, Господи Боже наш, изведший народ Твой из земли Египетской рукою сильною и явивший славу Твою, как день сей! согрешили мы, поступали нечестиво.
\par 16 Господи! по всей правде Твоей да отвратится гнев Твой и негодование Твое от града Твоего, Иерусалима, от святой горы Твоей; ибо за грехи наши и беззакония отцов наших Иерусалим и народ Твой в поругании у всех, окружающих нас.
\par 17 И ныне услыши, Боже наш, молитву раба Твоего и моление его и воззри светлым лицем Твоим на опустошенное святилище Твое, ради Тебя, Господи.
\par 18 Приклони, Боже мой, ухо Твое и услыши, открой очи Твои и воззри на опустошения наши и на город, на котором наречено имя Твое; ибо мы повергаем моления наши пред Тобою, уповая не на праведность нашу, но на Твое великое милосердие.
\par 19 Господи! услыши; Господи! прости; Господи! внемли и соверши, не умедли ради Тебя Самого, Боже мой, ибо Твое имя наречено на городе Твоем и на народе Твоем'.
\par 20 И когда я еще говорил и молился, и исповедывал грехи мои и грехи народа моего, Израиля, и повергал мольбу мою пред Господом Богом моим о святой горе Бога моего;
\par 21 когда я еще продолжал молитву, муж Гавриил, которого я видел прежде в видении, быстро прилетев, коснулся меня около времени вечерней жертвы
\par 22 и вразумлял меня, говорил со мною и сказал: `Даниил! теперь я исшел, чтобы научить тебя разумению.
\par 23 В начале моления твоего вышло слово, и я пришел возвестить [его] [тебе], ибо ты муж желаний; итак вникни в слово и уразумей видение.
\par 24 Семьдесят седмин определены для народа твоего и святаго города твоего, чтобы покрыто было преступление, запечатаны были грехи и заглажены беззакония, и чтобы приведена была правда вечная, и запечатаны были видение и пророк, и помазан был Святый святых.
\par 25 Итак знай и разумей: с того времени, как выйдет повеление о восстановлении Иерусалима, до Христа Владыки семь седмин и шестьдесят две седмины; и возвратится [народ] и обстроятся улицы и стены, но в трудные времена.
\par 26 И по истечении шестидесяти двух седмин предан будет смерти Христос, и не будет; а город и святилище разрушены будут народом вождя, который придет, и конец его будет как от наводнения, и до конца войны будут опустошения.
\par 27 И утвердит завет для многих одна седмина, а в половине седмины прекратится жертва и приношение, и на крыле [святилища] будет мерзость запустения, и окончательная предопределенная гибель постигнет опустошителя'.

\chapter{10}

\par 1 В третий год Кира, царя Персидского, было откровение Даниилу, который назывался именем Валтасара; и истинно было это откровение и великой силы. Он понял это откровение и уразумел это видение.
\par 2 В эти дни я, Даниил, был в сетовании три седмицы дней.
\par 3 Вкусного хлеба я не ел; мясо и вино не входило в уста мои, и мастями я не умащал себя до исполнения трех седмиц дней.
\par 4 А в двадцать четвертый день первого месяца был я на берегу большой реки Тигра,
\par 5 и поднял глаза мои, и увидел: вот один муж, облеченный в льняную одежду, и чресла его опоясаны золотом из Уфаза.
\par 6 Тело его--как топаз, лице его--как вид молнии; очи его--как горящие светильники, руки его и ноги его по виду--как блестящая медь, и глас речей его--как голос множества людей.
\par 7 И только один я, Даниил, видел это видение, а бывшие со мною люди не видели этого видения; но сильный страх напал на них и они убежали, чтобы скрыться.
\par 8 И остался я один и смотрел на это великое видение, но во мне не осталось крепости, и вид лица моего чрезвычайно изменился, не стало во мне бодрости.
\par 9 И услышал я глас слов его; и как только услышал глас слов его, в оцепенении пал я на лице мое и лежал лицем к земле.
\par 10 Но вот, коснулась меня рука и поставила меня на колени мои и на длани рук моих.
\par 11 И сказал он мне: `Даниил, муж желаний! вникни в слова, которые я скажу тебе, и стань прямо на ноги твои; ибо к тебе я послан ныне'. Когда он сказал мне эти слова, я встал с трепетом.
\par 12 Но он сказал мне: `не бойся, Даниил; с первого дня, как ты расположил сердце твое, чтобы достигнуть разумения и смирить тебя пред Богом твоим, слова твои услышаны, и я пришел бы по словам твоим.
\par 13 Но князь царства Персидского стоял против меня двадцать один день; но вот, Михаил, один из первых князей, пришел помочь мне, и я остался там при царях Персидских.
\par 14 А теперь я пришел возвестить тебе, что будет с народом твоим в последние времена, так как видение относится к отдаленным дням'.
\par 15 Когда он говорил мне такие слова, я припал лицем моим к земле и онемел.
\par 16 Но вот, некто, по виду похожий на сынов человеческих, коснулся уст моих, и я открыл уста мои, стал говорить и сказал стоящему передо мною: `господин мой! от этого видения внутренности мои повернулись во мне, и не стало во мне силы.
\par 17 И как может говорить раб такого господина моего с таким господином моим? ибо во мне нет силы, и дыхание замерло во мне'.
\par 18 Тогда снова прикоснулся ко мне тот человеческий облик и укрепил меня
\par 19 и сказал: `не бойся, муж желаний! мир тебе; мужайся, мужайся!' И когда он говорил со мною, я укрепился и сказал: `говори, господин мой; ибо ты укрепил меня'.
\par 20 И он сказал: `знаешь ли, для чего я пришел к тебе? Теперь я возвращусь, чтобы бороться с князем Персидским; а когда я выйду, то вот, придет князь Греции.
\par 21 Впрочем я возвещу тебе, что начертано в истинном писании; и нет никого, кто поддерживал бы меня в том, кроме Михаила, князя вашего.

\chapter{11}

\par 1 Итак я с первого года Дария Мидянина стал ему подпорою и подкреплением.
\par 2 Теперь возвещу тебе истину: вот, еще три царя восстанут в Персии; потом четвертый превзойдет всех великим богатством, и когда усилится богатством своим, то поднимет всех против царства Греческого.
\par 3 И восстанет царь могущественный, который будет владычествовать с великою властью, и будет действовать по своей воле.
\par 4 Но когда он восстанет, царство его разрушится и разделится по четырем ветрам небесным, и не к его потомкам перейдет, и не с тою властью, с какою он владычествовал; ибо раздробится царство его и достанется другим, кроме этих.
\par 5 И усилится южный царь и один из князей его пересилит его и будет владычествовать, и велико будет владычество его.
\par 6 Но через несколько лет они сблизятся, и дочь южного царя придет к царю северному, чтобы установить правильные отношения между ними; но она не удержит силы в руках своих, не устоит и род ее, но преданы будут как она, так и сопровождавшие ее, и рожденный ею, и помогавшие ей в те времена.
\par 7 Но восстанет отрасль от корня ее, придет к войску и войдет в укрепления царя северного, и будет действовать в них, и усилится.
\par 8 Даже и богов их, истуканы их с драгоценными сосудами их, серебряными и золотыми, увезет в плен в Египет и на несколько лет будет стоять выше царя северного.
\par 9 Хотя этот и сделает нашествие на царство южного царя, но возвратится в свою землю.
\par 10 Потом вооружатся сыновья его и соберут многочисленное войско, и один из них быстро пойдет, наводнит и пройдет, и потом, возвращаясь, будет сражаться с ним до укреплений его.
\par 11 И раздражится южный царь, и выступит, сразится с ним, с царем северным, и выставит большое войско, и предано будет войско в руки его.
\par 12 И ободрится войско, и сердце [царя] вознесется; он низложит многие тысячи, но от этого не будет сильнее.
\par 13 Ибо царь северный возвратится и выставит войско больше прежнего, и через несколько лет быстро придет с огромным войском и большим богатством.
\par 14 В те времена многие восстанут против южного царя, и мятежные из сынов твоего народа поднимутся, чтобы исполнилось видение, и падут.
\par 15 И придет царь северный, устроит вал и овладеет укрепленным городом, и не устоят мышцы юга, ни отборное войско его; недостанет силы противостоять.
\par 16 И кто выйдет к нему, будет действовать по воле его, и никто не устоит перед ним; и на славной земле поставит стан свой, и она пострадает от руки его.
\par 17 И вознамерится войти со всеми силами царства своего, и праведные с ним, и совершит это; и дочь жен отдаст ему, на погибель ее, но этот замысел не состоится, и ему не будет пользы из того.
\par 18 Потом обратит лице свое к островам и овладеет многими; но некий вождь прекратит нанесенный им позор и даже свой позор обратит на него.
\par 19 Затем он обратит лице свое на крепости своей земли; но споткнется, падет и не станет его.
\par 20 На место его восстанет некий, который пошлет сборщика податей, пройти по царству славы; но и он после немногих дней погибнет, и не от возмущения и не в сражении.
\par 21 И восстанет на место его презренный, и не воздадут ему царских почестей, но он придет без шума и лестью овладеет царством.
\par 22 И всепотопляющие полчища будут потоплены и сокрушены им, даже и сам вождь завета.
\par 23 Ибо после того, как он вступит в союз с ним, он будет действовать обманом, и взойдет, и одержит верх с малым народом.
\par 24 Он войдет в мирные и плодоносные страны, и совершит то, чего не делали отцы его и отцы отцов его; добычу, награбленное имущество и богатство будет расточать своим и на крепости будет иметь замыслы свои, но только до времени.
\par 25 Потом возбудит силы свои и дух свой с многочисленным войском против царя южного, и южный царь выступит на войну с великим и еще более сильным войском, но не устоит, потому что будет против него коварство.
\par 26 Даже участники трапезы его погубят его, и войско его разольется, и падет много убитых.
\par 27 У обоих царей сих на сердце будет коварство, и за одним столом будут говорить ложь, но успеха не будет, потому что конец еще отложен до времени.
\par 28 И отправится он в землю свою с великим богатством и враждебным намерением против святаго завета, и он исполнит его, и возвратится в свою землю.
\par 29 В назначенное время опять пойдет он на юг; но последний [поход] не такой будет, как прежний,
\par 30 ибо в одно время с ним придут корабли Киттимские; и он упадет духом, и возвратится, и озлобится на святый завет, и исполнит свое намерение, и опять войдет в соглашение с отступниками от святаго завета.
\par 31 И поставлена будет им часть войска, которая осквернит святилище могущества, и прекратит ежедневную жертву, и поставит мерзость запустения.
\par 32 Поступающих нечестиво против завета он привлечет к себе лестью; но люди, чтущие своего Бога, усилятся и будут действовать.
\par 33 И разумные из народа вразумят многих, хотя будут несколько времени страдать от меча и огня, от плена и грабежа;
\par 34 и во время страдания своего будут иметь некоторую помощь, и многие присоединятся к ним, но притворно.
\par 35 Пострадают некоторые и из разумных для испытания их, очищения и для убеления к последнему времени; ибо есть еще время до срока.
\par 36 И будет поступать царь тот по своему произволу, и вознесется и возвеличится выше всякого божества, и о Боге богов станет говорить хульное и будет иметь успех, доколе не совершится гнев: ибо, что предопределено, то исполнится.
\par 37 И о богах отцов своих он не помыслит, и ни желания жен, ни даже божества никакого не уважит; ибо возвеличит себя выше всех.
\par 38 Но богу крепостей на месте его будет он воздавать честь, и этого бога, которого не знали отцы его, он будет чествовать золотом и серебром, и дорогими камнями, и разными драгоценностями,
\par 39 и устроит твердую крепость с чужим богом: которые признают его, тем увеличит почести и даст власть над многими, и землю раздаст в награду.
\par 40 Под конец же времени сразится с ним царь южный, и царь северный устремится как буря на него с колесницами, всадниками и многочисленными кораблями, и нападет на области, наводнит их, и пройдет через них.
\par 41 И войдет он в прекраснейшую из земель, и многие области пострадают и спасутся от руки его только Едом, Моав и большая часть сынов Аммоновых.
\par 42 И прострет руку свою на разные страны; не спасется и земля Египетская.
\par 43 И завладеет он сокровищами золота и серебра и разными драгоценностями Египта; Ливийцы и Ефиопляне последуют за ним.
\par 44 Но слухи с востока и севера встревожат его, и выйдет он в величайшей ярости, чтобы истреблять и губить многих,
\par 45 и раскинет он царские шатры свои между морем и горою преславного святилища; но придет к своему концу, и никто не поможет ему.

\chapter{12}

\par 1 И восстанет в то время Михаил, князь великий, стоящий за сынов народа твоего; и наступит время тяжкое, какого не бывало с тех пор, как существуют люди, до сего времени; но спасутся в это время из народа твоего все, которые найдены будут записанными в книге.
\par 2 И многие из спящих в прахе земли пробудятся, одни для жизни вечной, другие на вечное поругание и посрамление.
\par 3 И разумные будут сиять, как светила на тверди, и обратившие многих к правде--как звезды, вовеки, навсегда.
\par 4 А ты, Даниил, сокрой слова сии и запечатай книгу сию до последнего времени; многие прочитают ее, и умножится ведение'.
\par 5 Тогда я, Даниил, посмотрел, и вот, стоят двое других, один на этом берегу реки, другой на том берегу реки.
\par 6 И [один] сказал мужу в льняной одежде, который стоял над водами реки: `когда будет конец этих чудных происшествий?'
\par 7 И слышал я, как муж в льняной одежде, находившийся над водами реки, подняв правую и левую руку к небу, клялся Живущим вовеки, что к концу времени и времен и полувремени, и по совершенном низложении силы народа святого, все это совершится.
\par 8 Я слышал это, но не понял, и потому сказал: `господин мой! что же после этого будет?'
\par 9 И отвечал он: `иди, Даниил; ибо сокрыты и запечатаны слова сии до последнего времени.
\par 10 Многие очистятся, убелятся и переплавлены будут [в искушении]; нечестивые же будут поступать нечестиво, и не уразумеет сего никто из нечестивых, а мудрые уразумеют.
\par 11 Со времени прекращения ежедневной жертвы и поставления мерзости запустения пройдет тысяча двести девяносто дней.
\par 12 Блажен, кто ожидает и достигнет тысячи трехсот тридцати пяти дней.
\par 13 А ты иди к твоему концу и упокоишься, и восстанешь для получения твоего жребия в конце дней'.


\end{document}