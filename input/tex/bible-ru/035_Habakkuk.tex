\begin{document}

\title{Пр. Аввакума}


\chapter{1}

\par 1 Пророческое видение, которое видел пророк Аввакум.
\par 2 Доколе, Господи, я буду взывать, и Ты не слышишь, буду вопиять к Тебе о насилии, и Ты не спасаешь?
\par 3 Для чего даешь мне видеть злодейство и смотреть на бедствия? Грабительство и насилие предо мною, и восстает вражда и поднимается раздор.
\par 4 От этого закон потерял силу, и суда правильного нет: так как нечестивый одолевает праведного, то и суд происходит превратный.
\par 5 Посмотрите между народами и внимательно вглядитесь, и вы сильно изумитесь; ибо Я сделаю во дни ваши такое дело, которому вы не поверили бы, если бы вам рассказывали.
\par 6 Ибо вот, Я подниму Халдеев, народ жестокий и необузданный, который ходит по широтам земли, чтобы завладеть не принадлежащими ему селениями.
\par 7 Страшен и грозен он; от него самого происходит суд его и власть его.
\par 8 Быстрее барсов кони его и прытче вечерних волков; скачет в разные стороны конница его; издалека приходят всадники его, прилетают как орел, бросающийся на добычу.
\par 9 Весь он идет для грабежа; устремив лице свое вперед, он забирает пленников, как песок.
\par 10 И над царями он издевается, и князья служат ему посмешищем; над всякою крепостью он смеется: насыплет осадный вал и берет ее.
\par 11 Тогда надмевается дух его, и он ходит и буйствует; сила его--бог его.
\par 12 Но не Ты ли издревле Господь Бог мой, Святый мой? мы не умрем! Ты, Господи, только для суда попустил его. Скала моя! для наказания Ты назначил его.
\par 13 Чистым очам Твоим не свойственно глядеть на злодеяния, и смотреть на притеснение Ты не можешь; для чего же Ты смотришь на злодеев и безмолвствуешь, когда нечестивец поглощает того, кто праведнее его,
\par 14 и оставляешь людей как рыбу в море, как пресмыкающихся, у которых нет властителя?
\par 15 Всех их таскает удою, захватывает в сеть свою и забирает их в неводы свои, и оттого радуется и торжествует.
\par 16 За то приносит жертвы сети своей и кадит неводу своему, потому что от них тучна часть его и роскошна пища его.
\par 17 Неужели для этого он должен опорожнять свою сеть и непрестанно избивать народы без пощады?

\chapter{2}

\par 1 На стражу мою стал я и, стоя на башне, наблюдал, чтобы узнать, что скажет Он во мне, и что мне отвечать по жалобе моей?
\par 2 И отвечал мне Господь и сказал: запиши видение и начертай ясно на скрижалях, чтобы читающий легко мог прочитать,
\par 3 ибо видение относится еще к определенному времени и говорит о конце и не обманет; и хотя бы и замедлило, жди его, ибо непременно сбудется, не отменится.
\par 4 Вот, душа надменная не успокоится, а праведный своею верою жив будет.
\par 5 Надменный человек, как бродящее вино, не успокаивается, так что расширяет душу свою как ад, и как смерть он ненасытен, и собирает к себе все народы, и захватывает себе все племена.
\par 6 Но не все ли они будут произносить о нем притчу и насмешливую песнь: `горе тому, кто без меры обогащает себя не своим, --на долго ли? --и обременяет себя залогами!'
\par 7 Не восстанут ли внезапно те, которые будут терзать тебя, и не поднимутся ли против тебя грабители, и ты достанешься им на расхищение?
\par 8 Так как ты ограбил многие народы, то и тебя ограбят все остальные народы за пролитие крови человеческой, за разорение страны, города и всех живущих в нем.
\par 9 Горе тому, кто жаждет неправедных приобретений для дома своего, чтобы устроить гнездо свое на высоте и тем обезопасить себя от руки несчастья!
\par 10 Бесславие измыслил ты для твоего дома, истребляя многие народы, и согрешил против души твоей.
\par 11 Камни из стен возопиют и перекладины из дерева будут отвечать им:
\par 12 `горе строящему город на крови и созидающему крепости неправдою!'
\par 13 Вот, не от Господа ли Саваофа это, что народы трудятся для огня и племена мучат себя напрасно?
\par 14 Ибо земля наполнится познанием славы Господа, как воды наполняют море.
\par 15 Горе тебе, который подаешь ближнему твоему питье с примесью злобы твоей и делаешь его пьяным, чтобы видеть срамоту его!
\par 16 Ты пресытился стыдом вместо славы; пей же и ты и показывай срамоту, --обратится и к тебе чаша десницы Господней и посрамление на славу твою.
\par 17 Ибо злодейство твое на Ливане обрушится на тебя за истребление устрашенных животных, за пролитие крови человеческой, за опустошение страны, города и всех живущих в нем.
\par 18 Что за польза от истукана, сделанного художником, этого литаго лжеучителя, хотя ваятель, делая немые кумиры, полагается на свое произведение?
\par 19 Горе тому, кто говорит дереву: `встань!' и бессловесному камню: `пробудись!' Научит ли он чему-нибудь? Вот, он обложен золотом и серебром, но дыхания в нем нет.
\par 20 А Господь--во святом храме Своем: да молчит вся земля пред лицем Его!

\chapter{3}

\par 1 Молитва Аввакума пророка, для пения.
\par 2 Господи! услышал я слух Твой и убоялся. Господи! соверши дело Твое среди лет, среди лет яви его; во гневе вспомни о милости.
\par 3 Бог от Фемана грядет и Святый--от горы Фаран. Покрыло небеса величие Его, и славою Его наполнилась земля.
\par 4 Блеск ее--как солнечный свет; от руки Его лучи, и здесь тайник Его силы!
\par 5 Пред лицем Его идет язва, а по стопам Его--жгучий ветер.
\par 6 Он стал и поколебал землю; воззрел, и в трепет привел народы; вековые горы распались, первобытные холмы опали; пути Его вечные.
\par 7 Грустными видел я шатры Ефиопские; сотряслись палатки земли Мадиамской.
\par 8 Разве на реки воспылал, Господи, гнев Твой? разве на реки--негодование Твое, или на море--ярость Твоя, что Ты восшел на коней Твоих, на колесницы Твои спасительные?
\par 9 Ты обнажил лук Твой по клятвенному обетованию, данному коленам. Ты потоками рассек землю.
\par 10 Увидев Тебя, вострепетали горы, ринулись воды; бездна дала голос свой, высоко подняла руки свои;
\par 11 солнце и луна остановились на месте своем пред светом летающих стрел Твоих, пред сиянием сверкающих копьев Твоих.
\par 12 Во гневе шествуешь Ты по земле и в негодовании попираешь народы.
\par 13 Ты выступаешь для спасения народа Твоего, для спасения помазанного Твоего. Ты сокрушаешь главу нечестивого дома, обнажая его от основания до верха.
\par 14 Ты пронзаешь копьями его главу вождей его, когда они как вихрь ринулись разбить меня, в радости, как бы думая поглотить бедного скрытно.
\par 15 Ты с конями Твоими проложил путь по морю, через пучину великих вод.
\par 16 Я услышал, и вострепетала внутренность моя; при вести о сем задрожали губы мои, боль проникла в кости мои, и колеблется место подо мною; а я должен быть спокоен в день бедствия, когда придет на народ мой грабитель его.
\par 17 Хотя бы не расцвела смоковница и не было плода на виноградных лозах, и маслина изменила, и нива не дала пищи, хотя бы не стало овец в загоне и рогатого скота в стойлах, --
\par 18 но и тогда я буду радоваться о Господе и веселиться о Боге спасения моего.
\par 19 Господь Бог--сила моя: Он сделает ноги мои как у оленя и на высоты мои возведет меня! (Начальнику хора).


\end{document}