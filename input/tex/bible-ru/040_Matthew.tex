\begin{document}

\title{От Матфея}


\chapter{1}

\par 1 Родословие Иисуса Христа, Сына Давидова, Сына Авраамова.
\par 2 Авраам родил Исаака; Исаак родил Иакова; Иаков родил Иуду и братьев его;
\par 3 Иуда родил Фареса и Зару от Фамари; Фарес родил Есрома; Есром родил Арама;
\par 4 Арам родил Аминадава; Аминадав родил Наассона; Наассон родил Салмона;
\par 5 Салмон родил Вооза от Рахавы; Вооз родил Овида от Руфи; Овид родил Иессея;
\par 6 Иессей родил Давида царя; Давид царь родил Соломона от бывшей за Уриею;
\par 7 Соломон родил Ровоама; Ровоам родил Авию; Авия родил Асу;
\par 8 Аса родил Иосафата; Иосафат родил Иорама; Иорам родил Озию;
\par 9 Озия родил Иоафама; Иоафам родил Ахаза; Ахаз родил Езекию;
\par 10 Езекия родил Манассию; Манассия родил Амона; Амон родил Иосию;
\par 11 Иосия родил Иоакима; Иоаким родил Иехонию и братьев его, перед переселением в Вавилон.
\par 12 По переселении же в Вавилон, Иехония родил Салафииля; Салафииль родил Зоровавеля;
\par 13 Зоровавель родил Авиуда; Авиуд родил Елиакима; Елиаким родил Азора;
\par 14 Азор родил Садока; Садок родил Ахима; Ахим родил Елиуда;
\par 15 Елиуд родил Елеазара; Елеазар родил Матфана; Матфан родил Иакова;
\par 16 Иаков родил Иосифа, мужа Марии, от Которой родился Иисус, называемый Христос.
\par 17 Итак всех родов от Авраама до Давида четырнадцать родов; и от Давида до переселения в Вавилон четырнадцать родов; и от переселения в Вавилон до Христа четырнадцать родов.
\par 18 Рождество Иисуса Христа было так: по обручении Матери Его Марии с Иосифом, прежде нежели сочетались они, оказалось, что Она имеет во чреве от Духа Святаго.
\par 19 Иосиф же муж Ее, будучи праведен и не желая огласить Ее, хотел тайно отпустить Ее.
\par 20 Но когда он помыслил это, --се, Ангел Господень явился ему во сне и сказал: Иосиф, сын Давидов! не бойся принять Марию, жену твою, ибо родившееся в Ней есть от Духа Святаго;
\par 21 родит же Сына, и наречешь Ему имя Иисус, ибо Он спасет людей Своих от грехов их.
\par 22 А все сие произошло, да сбудется реченное Господом через пророка, который говорит:
\par 23 се, Дева во чреве приимет и родит Сына, и нарекут имя Ему Еммануил, что значит: с нами Бог.
\par 24 Встав от сна, Иосиф поступил, как повелел ему Ангел Господень, и принял жену свою,
\par 25 и не знал Ее. Как наконец Она родила Сына Своего первенца, и он нарек Ему имя: Иисус.

\chapter{2}

\par 1 Когда же Иисус родился в Вифлееме Иудейском во дни царя Ирода, пришли в Иерусалим волхвы с востока и говорят:
\par 2 где родившийся Царь Иудейский? ибо мы видели звезду Его на востоке и пришли поклониться Ему.
\par 3 Услышав это, Ирод царь встревожился, и весь Иерусалим с ним.
\par 4 И, собрав всех первосвященников и книжников народных, спрашивал у них: где должно родиться Христу?
\par 5 Они же сказали ему: в Вифлееме Иудейском, ибо так написано через пророка:
\par 6 и ты, Вифлеем, земля Иудина, ничем не меньше воеводств Иудиных, ибо из тебя произойдет Вождь, Который упасет народ Мой, Израиля.
\par 7 Тогда Ирод, тайно призвав волхвов, выведал от них время появления звезды
\par 8 и, послав их в Вифлеем, сказал: пойдите, тщательно разведайте о Младенце и, когда найдете, известите меня, чтобы и мне пойти поклониться Ему.
\par 9 Они, выслушав царя, пошли. И се, звезда, которую видели они на востоке, шла перед ними, [как] наконец пришла и остановилась над [местом], где был Младенец.
\par 10 Увидев же звезду, они возрадовались радостью весьма великою,
\par 11 и, войдя в дом, увидели Младенца с Мариею, Матерью Его, и, пав, поклонились Ему; и, открыв сокровища свои, принесли Ему дары: золото, ладан и смирну.
\par 12 И, получив во сне откровение не возвращаться к Ироду, иным путем отошли в страну свою.
\par 13 Когда же они отошли, --се, Ангел Господень является во сне Иосифу и говорит: встань, возьми Младенца и Матерь Его и беги в Египет, и будь там, доколе не скажу тебе, ибо Ирод хочет искать Младенца, чтобы погубить Его.
\par 14 Он встал, взял Младенца и Матерь Его ночью и пошел в Египет,
\par 15 и там был до смерти Ирода, да сбудется реченное Господом через пророка, который говорит: из Египта воззвал Я Сына Моего.
\par 16 Тогда Ирод, увидев себя осмеянным волхвами, весьма разгневался, и послал избить всех младенцев в Вифлееме и во всех пределах его, от двух лет и ниже, по времени, которое выведал от волхвов.
\par 17 Тогда сбылось реченное через пророка Иеремию, который говорит:
\par 18 глас в Раме слышен, плач и рыдание и вопль великий; Рахиль плачет о детях своих и не хочет утешиться, ибо их нет.
\par 19 По смерти же Ирода, --се, Ангел Господень во сне является Иосифу в Египте
\par 20 и говорит: встань, возьми Младенца и Матерь Его и иди в землю Израилеву, ибо умерли искавшие души Младенца.
\par 21 Он встал, взял Младенца и Матерь Его и пришел в землю Израилеву.
\par 22 Услышав же, что Архелай царствует в Иудее вместо Ирода, отца своего, убоялся туда идти; но, получив во сне откровение, пошел в пределы Галилейские
\par 23 и, придя, поселился в городе, называемом Назарет, да сбудется реченное через пророков, что Он Назореем наречется.

\chapter{3}

\par 1 В те дни приходит Иоанн Креститель и проповедует в пустыне Иудейской
\par 2 и говорит: покайтесь, ибо приблизилось Царство Небесное.
\par 3 Ибо он тот, о котором сказал пророк Исаия: глас вопиющего в пустыне: приготовьте путь Господу, прямыми сделайте стези Ему.
\par 4 Сам же Иоанн имел одежду из верблюжьего волоса и пояс кожаный на чреслах своих, а пищею его были акриды и дикий мед.
\par 5 Тогда Иерусалим и вся Иудея и вся окрестность Иорданская выходили к нему
\par 6 и крестились от него в Иордане, исповедуя грехи свои.
\par 7 Увидев же Иоанн многих фарисеев и саддукеев, идущих к нему креститься, сказал им: порождения ехиднины! кто внушил вам бежать от будущего гнева?
\par 8 сотворите же достойный плод покаяния
\par 9 и не думайте говорить в себе: `отец у нас Авраам', ибо говорю вам, что Бог может из камней сих воздвигнуть детей Аврааму.
\par 10 Уже и секира при корне дерев лежит: всякое дерево, не приносящее доброго плода, срубают и бросают в огонь.
\par 11 Я крещу вас в воде в покаяние, но Идущий за мною сильнее меня; я не достоин понести обувь Его; Он будет крестить вас Духом Святым и огнем;
\par 12 лопата Его в руке Его, и Он очистит гумно Свое и соберет пшеницу Свою в житницу, а солому сожжет огнем неугасимым.
\par 13 Тогда приходит Иисус из Галилеи на Иордан к Иоанну креститься от него.
\par 14 Иоанн же удерживал Его и говорил: мне надобно креститься от Тебя, и Ты ли приходишь ко мне?
\par 15 Но Иисус сказал ему в ответ: оставь теперь, ибо так надлежит нам исполнить всякую правду. Тогда [Иоанн] допускает Его.
\par 16 И, крестившись, Иисус тотчас вышел из воды, --и се, отверзлись Ему небеса, и увидел [Иоанн] Духа Божия, Который сходил, как голубь, и ниспускался на Него.
\par 17 И се, глас с небес глаголющий: Сей есть Сын Мой возлюбленный, в Котором Мое благоволение.

\chapter{4}

\par 1 Тогда Иисус возведен был Духом в пустыню, для искушения от диавола,
\par 2 и, постившись сорок дней и сорок ночей, напоследок взалкал.
\par 3 И приступил к Нему искуситель и сказал: если Ты Сын Божий, скажи, чтобы камни сии сделались хлебами.
\par 4 Он же сказал ему в ответ: написано: не хлебом одним будет жить человек, но всяким словом, исходящим из уст Божиих.
\par 5 Потом берет Его диавол в святой город и поставляет Его на крыле храма,
\par 6 и говорит Ему: если Ты Сын Божий, бросься вниз, ибо написано: Ангелам Своим заповедает о Тебе, и на руках понесут Тебя, да не преткнешься о камень ногою Твоею.
\par 7 Иисус сказал ему: написано также: не искушай Господа Бога твоего.
\par 8 Опять берет Его диавол на весьма высокую гору и показывает Ему все царства мира и славу их,
\par 9 и говорит Ему: все это дам Тебе, если, пав, поклонишься мне.
\par 10 Тогда Иисус говорит ему: отойди от Меня, сатана, ибо написано: Господу Богу твоему поклоняйся и Ему одному служи.
\par 11 Тогда оставляет Его диавол, и се, Ангелы приступили и служили Ему.
\par 12 Услышав же Иисус, что Иоанн отдан [под стражу], удалился в Галилею
\par 13 и, оставив Назарет, пришел и поселился в Капернауме приморском, в пределах Завулоновых и Неффалимовых,
\par 14 да сбудется реченное через пророка Исаию, который говорит:
\par 15 земля Завулонова и земля Неффалимова, на пути приморском, за Иорданом, Галилея языческая,
\par 16 народ, сидящий во тьме, увидел свет великий, и сидящим в стране и тени смертной воссиял свет.
\par 17 С того времени Иисус начал проповедывать и говорить: покайтесь, ибо приблизилось Царство Небесное.
\par 18 Проходя же близ моря Галилейского, Он увидел двух братьев: Симона, называемого Петром, и Андрея, брата его, закидывающих сети в море, ибо они были рыболовы,
\par 19 и говорит им: идите за Мною, и Я сделаю вас ловцами человеков.
\par 20 И они тотчас, оставив сети, последовали за Ним.
\par 21 Оттуда, идя далее, увидел Он других двух братьев, Иакова Зеведеева и Иоанна, брата его, в лодке с Зеведеем, отцом их, починивающих сети свои, и призвал их.
\par 22 И они тотчас, оставив лодку и отца своего, последовали за Ним.
\par 23 И ходил Иисус по всей Галилее, уча в синагогах их и проповедуя Евангелие Царствия, и исцеляя всякую болезнь и всякую немощь в людях.
\par 24 И прошел о Нем слух по всей Сирии; и приводили к Нему всех немощных, одержимых различными болезнями и припадками, и бесноватых, и лунатиков, и расслабленных, и Он исцелял их.
\par 25 И следовало за Ним множество народа из Галилеи и Десятиградия, и Иерусалима, и Иудеи, и из-за Иордана.

\chapter{5}

\par 1 Увидев народ, Он взошел на гору; и, когда сел, приступили к Нему ученики Его.
\par 2 И Он, отверзши уста Свои, учил их, говоря:
\par 3 Блаженны нищие духом, ибо их есть Царство Небесное.
\par 4 Блаженны плачущие, ибо они утешатся.
\par 5 Блаженны кроткие, ибо они наследуют землю.
\par 6 Блаженны алчущие и жаждущие правды, ибо они насытятся.
\par 7 Блаженны милостивые, ибо они помилованы будут.
\par 8 Блаженны чистые сердцем, ибо они Бога узрят.
\par 9 Блаженны миротворцы, ибо они будут наречены сынами Божиими.
\par 10 Блаженны изгнанные за правду, ибо их есть Царство Небесное.
\par 11 Блаженны вы, когда будут поносить вас и гнать и всячески неправедно злословить за Меня.
\par 12 Радуйтесь и веселитесь, ибо велика ваша награда на небесах: так гнали [и] пророков, бывших прежде вас.
\par 13 Вы--соль земли. Если же соль потеряет силу, то чем сделаешь ее соленою? Она уже ни к чему негодна, как разве выбросить ее вон на попрание людям.
\par 14 Вы--свет мира. Не может укрыться город, стоящий на верху горы.
\par 15 И, зажегши свечу, не ставят ее под сосудом, но на подсвечнике, и светит всем в доме.
\par 16 Так да светит свет ваш пред людьми, чтобы они видели ваши добрые дела и прославляли Отца вашего Небесного.
\par 17 Не думайте, что Я пришел нарушить закон или пророков: не нарушить пришел Я, но исполнить.
\par 18 Ибо истинно говорю вам: доколе не прейдет небо и земля, ни одна иота или ни одна черта не прейдет из закона, пока не исполнится все.
\par 19 Итак, кто нарушит одну из заповедей сих малейших и научит так людей, тот малейшим наречется в Царстве Небесном; а кто сотворит и научит, тот великим наречется в Царстве Небесном.
\par 20 Ибо, говорю вам, если праведность ваша не превзойдет праведности книжников и фарисеев, то вы не войдете в Царство Небесное.
\par 21 Вы слышали, что сказано древним: не убивай, кто же убьет, подлежит суду.
\par 22 А Я говорю вам, что всякий, гневающийся на брата своего напрасно, подлежит суду; кто же скажет брату своему: `рака', подлежит синедриону; а кто скажет: `безумный', подлежит геенне огненной.
\par 23 Итак, если ты принесешь дар твой к жертвеннику и там вспомнишь, что брат твой имеет что-нибудь против тебя,
\par 24 оставь там дар твой пред жертвенником, и пойди прежде примирись с братом твоим, и тогда приди и принеси дар твой.
\par 25 Мирись с соперником твоим скорее, пока ты еще на пути с ним, чтобы соперник не отдал тебя судье, а судья не отдал бы тебя слуге, и не ввергли бы тебя в темницу;
\par 26 истинно говорю тебе: ты не выйдешь оттуда, пока не отдашь до последнего кодранта.
\par 27 Вы слышали, что сказано древним: не прелюбодействуй.
\par 28 А Я говорю вам, что всякий, кто смотрит на женщину с вожделением, уже прелюбодействовал с нею в сердце своем.
\par 29 Если же правый глаз твой соблазняет тебя, вырви его и брось от себя, ибо лучше для тебя, чтобы погиб один из членов твоих, а не все тело твое было ввержено в геенну.
\par 30 И если правая твоя рука соблазняет тебя, отсеки ее и брось от себя, ибо лучше для тебя, чтобы погиб один из членов твоих, а не все тело твое было ввержено в геенну.
\par 31 Сказано также, что если кто разведется с женою своею, пусть даст ей разводную.
\par 32 А Я говорю вам: кто разводится с женою своею, кроме вины прелюбодеяния, тот подает ей повод прелюбодействовать; и кто женится на разведенной, тот прелюбодействует.
\par 33 Еще слышали вы, что сказано древним: не преступай клятвы, но исполняй пред Господом клятвы твои.
\par 34 А Я говорю вам: не клянись вовсе: ни небом, потому что оно престол Божий;
\par 35 ни землею, потому что она подножие ног Его; ни Иерусалимом, потому что он город великого Царя;
\par 36 ни головою твоею не клянись, потому что не можешь ни одного волоса сделать белым или черным.
\par 37 Но да будет слово ваше: да, да; нет, нет; а что сверх этого, то от лукавого.
\par 38 Вы слышали, что сказано: око за око и зуб за зуб.
\par 39 А Я говорю вам: не противься злому. Но кто ударит тебя в правую щеку твою, обрати к нему и другую;
\par 40 и кто захочет судиться с тобою и взять у тебя рубашку, отдай ему и верхнюю одежду;
\par 41 и кто принудит тебя идти с ним одно поприще, иди с ним два.
\par 42 Просящему у тебя дай, и от хотящего занять у тебя не отвращайся.
\par 43 Вы слышали, что сказано: люби ближнего твоего и ненавидь врага твоего.
\par 44 А Я говорю вам: любите врагов ваших, благословляйте проклинающих вас, благотворите ненавидящим вас и молитесь за обижающих вас и гонящих вас,
\par 45 да будете сынами Отца вашего Небесного, ибо Он повелевает солнцу Своему восходить над злыми и добрыми и посылает дождь на праведных и неправедных.
\par 46 Ибо если вы будете любить любящих вас, какая вам награда? Не то же ли делают и мытари?
\par 47 И если вы приветствуете только братьев ваших, что особенного делаете? Не так же ли поступают и язычники?
\par 48 Итак будьте совершенны, как совершен Отец ваш Небесный.

\chapter{6}

\par 1 Смотрите, не творите милостыни вашей пред людьми с тем, чтобы они видели вас: иначе не будет вам награды от Отца вашего Небесного.
\par 2 Итак, когда творишь милостыню, не труби перед собою, как делают лицемеры в синагогах и на улицах, чтобы прославляли их люди. Истинно говорю вам: они уже получают награду свою.
\par 3 У тебя же, когда творишь милостыню, пусть левая рука твоя не знает, что делает правая,
\par 4 чтобы милостыня твоя была втайне; и Отец твой, видящий тайное, воздаст тебе явно.
\par 5 И, когда молишься, не будь, как лицемеры, которые любят в синагогах и на углах улиц, останавливаясь, молиться, чтобы показаться перед людьми. Истинно говорю вам, что они уже получают награду свою.
\par 6 Ты же, когда молишься, войди в комнату твою и, затворив дверь твою, помолись Отцу твоему, Который втайне; и Отец твой, видящий тайное, воздаст тебе явно.
\par 7 А молясь, не говорите лишнего, как язычники, ибо они думают, что в многословии своем будут услышаны;
\par 8 не уподобляйтесь им, ибо знает Отец ваш, в чем вы имеете нужду, прежде вашего прошения у Него.
\par 9 Молитесь же так: Отче наш, сущий на небесах! да святится имя Твое;
\par 10 да приидет Царствие Твое; да будет воля Твоя и на земле, как на небе;
\par 11 хлеб наш насущный дай нам на сей день;
\par 12 и прости нам долги наши, как и мы прощаем должникам нашим;
\par 13 и не введи нас в искушение, но избавь нас от лукавого. Ибо Твое есть Царство и сила и слава во веки. Аминь.
\par 14 Ибо если вы будете прощать людям согрешения их, то простит и вам Отец ваш Небесный,
\par 15 а если не будете прощать людям согрешения их, то и Отец ваш не простит вам согрешений ваших.
\par 16 Также, когда поститесь, не будьте унылы, как лицемеры, ибо они принимают на себя мрачные лица, чтобы показаться людям постящимися. Истинно говорю вам, что они уже получают награду свою.
\par 17 А ты, когда постишься, помажь голову твою и умой лице твое,
\par 18 чтобы явиться постящимся не пред людьми, но пред Отцом твоим, Который втайне; и Отец твой, видящий тайное, воздаст тебе явно.
\par 19 Не собирайте себе сокровищ на земле, где моль и ржа истребляют и где воры подкапывают и крадут,
\par 20 но собирайте себе сокровища на небе, где ни моль, ни ржа не истребляют и где воры не подкапывают и не крадут,
\par 21 ибо где сокровище ваше, там будет и сердце ваше.
\par 22 Светильник для тела есть око. Итак, если око твое будет чисто, то все тело твое будет светло;
\par 23 если же око твое будет худо, то все тело твое будет темно. Итак, если свет, который в тебе, тьма, то какова же тьма?
\par 24 Никто не может служить двум господам: ибо или одного будет ненавидеть, а другого любить; или одному станет усердствовать, а о другом нерадеть. Не можете служить Богу и маммоне.
\par 25 Посему говорю вам: не заботьтесь для души вашей, что вам есть и что пить, ни для тела вашего, во что одеться. Душа не больше ли пищи, и тело одежды?
\par 26 Взгляните на птиц небесных: они ни сеют, ни жнут, ни собирают в житницы; и Отец ваш Небесный питает их. Вы не гораздо ли лучше их?
\par 27 Да и кто из вас, заботясь, может прибавить себе росту [хотя] на один локоть?
\par 28 И об одежде что заботитесь? Посмотрите на полевые лилии, как они растут: ни трудятся, ни прядут;
\par 29 но говорю вам, что и Соломон во всей славе своей не одевался так, как всякая из них;
\par 30 если же траву полевую, которая сегодня есть, а завтра будет брошена в печь, Бог так одевает, кольми паче вас, маловеры!
\par 31 Итак не заботьтесь и не говорите: что нам есть? или что пить? или во что одеться?
\par 32 потому что всего этого ищут язычники, и потому что Отец ваш Небесный знает, что вы имеете нужду во всем этом.
\par 33 Ищите же прежде Царства Божия и правды Его, и это все приложится вам.
\par 34 Итак не заботьтесь о завтрашнем дне, ибо завтрашний [сам] будет заботиться о своем: довольно для [каждого] дня своей заботы.

\chapter{7}

\par 1 Не судите, да не судимы будете,
\par 2 ибо каким судом судите, [таким] будете судимы; и какою мерою мерите, [такою] и вам будут мерить.
\par 3 И что ты смотришь на сучок в глазе брата твоего, а бревна в твоем глазе не чувствуешь?
\par 4 Или как скажешь брату твоему: `дай, я выну сучок из глаза твоего', а вот, в твоем глазе бревно?
\par 5 Лицемер! вынь прежде бревно из твоего глаза и тогда увидишь, [как] вынуть сучок из глаза брата твоего.
\par 6 Не давайте святыни псам и не бросайте жемчуга вашего перед свиньями, чтобы они не попрали его ногами своими и, обратившись, не растерзали вас.
\par 7 Просите, и дано будет вам; ищите, и найдете; стучите, и отворят вам;
\par 8 ибо всякий просящий получает, и ищущий находит, и стучащему отворят.
\par 9 Есть ли между вами такой человек, который, когда сын его попросит у него хлеба, подал бы ему камень?
\par 10 и когда попросит рыбы, подал бы ему змею?
\par 11 Итак если вы, будучи злы, умеете даяния благие давать детям вашим, тем более Отец ваш Небесный даст блага просящим у Него.
\par 12 Итак во всем, как хотите, чтобы с вами поступали люди, так поступайте и вы с ними, ибо в этом закон и пророки.
\par 13 Входите тесными вратами, потому что широки врата и пространен путь, ведущие в погибель, и многие идут ими;
\par 14 потому что тесны врата и узок путь, ведущие в жизнь, и немногие находят их.
\par 15 Берегитесь лжепророков, которые приходят к вам в овечьей одежде, а внутри суть волки хищные.
\par 16 По плодам их узнаете их. Собирают ли с терновника виноград, или с репейника смоквы?
\par 17 Так всякое дерево доброе приносит и плоды добрые, а худое дерево приносит и плоды худые.
\par 18 Не может дерево доброе приносить плоды худые, ни дерево худое приносить плоды добрые.
\par 19 Всякое дерево, не приносящее плода доброго, срубают и бросают в огонь.
\par 20 Итак по плодам их узнаете их.
\par 21 Не всякий, говорящий Мне: `Господи! Господи!', войдет в Царство Небесное, но исполняющий волю Отца Моего Небесного.
\par 22 Многие скажут Мне в тот день: Господи! Господи! не от Твоего ли имени мы пророчествовали? и не Твоим ли именем бесов изгоняли? и не Твоим ли именем многие чудеса творили?
\par 23 И тогда объявлю им: Я никогда не знал вас; отойдите от Меня, делающие беззаконие.
\par 24 Итак всякого, кто слушает слова Мои сии и исполняет их, уподоблю мужу благоразумному, который построил дом свой на камне;
\par 25 и пошел дождь, и разлились реки, и подули ветры, и устремились на дом тот, и он не упал, потому что основан был на камне.
\par 26 А всякий, кто слушает сии слова Мои и не исполняет их, уподобится человеку безрассудному, который построил дом свой на песке;
\par 27 и пошел дождь, и разлились реки, и подули ветры, и налегли на дом тот; и он упал, и было падение его великое.
\par 28 И когда Иисус окончил слова сии, народ дивился учению Его,
\par 29 ибо Он учил их, как власть имеющий, а не как книжники и фарисеи.

\chapter{8}

\par 1 Когда же сошел Он с горы, за Ним последовало множество народа.
\par 2 И вот подошел прокаженный и, кланяясь Ему, сказал: Господи! если хочешь, можешь меня очистить.
\par 3 Иисус, простерши руку, коснулся его и сказал: хочу, очистись. И он тотчас очистился от проказы.
\par 4 И говорит ему Иисус: смотри, никому не сказывай, но пойди, покажи себя священнику и принеси дар, какой повелел Моисей, во свидетельство им.
\par 5 Когда же вошел Иисус в Капернаум, к Нему подошел сотник и просил Его:
\par 6 Господи! слуга мой лежит дома в расслаблении и жестоко страдает.
\par 7 Иисус говорит ему: Я приду и исцелю его.
\par 8 Сотник же, отвечая, сказал: Господи! я недостоин, чтобы Ты вошел под кров мой, но скажи только слово, и выздоровеет слуга мой;
\par 9 ибо я и подвластный человек, но, имея у себя в подчинении воинов, говорю одному: пойди, и идет; и другому: приди, и приходит; и слуге моему: сделай то, и делает.
\par 10 Услышав сие, Иисус удивился и сказал идущим за Ним: истинно говорю вам, и в Израиле не нашел Я такой веры.
\par 11 Говорю же вам, что многие придут с востока и запада и возлягут с Авраамом, Исааком и Иаковом в Царстве Небесном;
\par 12 а сыны царства извержены будут во тьму внешнюю: там будет плач и скрежет зубов.
\par 13 И сказал Иисус сотнику: иди, и, как ты веровал, да будет тебе. И выздоровел слуга его в тот час.
\par 14 Придя в дом Петров, Иисус увидел тещу его, лежащую в горячке,
\par 15 и коснулся руки ее, и горячка оставила ее; и она встала и служила им.
\par 16 Когда же настал вечер, к Нему привели многих бесноватых, и Он изгнал духов словом и исцелил всех больных,
\par 17 да сбудется реченное через пророка Исаию, который говорит: Он взял на Себя наши немощи и понес болезни.
\par 18 Увидев же Иисус вокруг Себя множество народа, велел (ученикам) отплыть на другую сторону.
\par 19 Тогда один книжник, подойдя, сказал Ему: Учитель! я пойду за Тобою, куда бы Ты ни пошел.
\par 20 И говорит ему Иисус: лисицы имеют норы и птицы небесные--гнезда, а Сын Человеческий не имеет, где приклонить голову.
\par 21 Другой же из учеников Его сказал Ему: Господи! позволь мне прежде пойти и похоронить отца моего.
\par 22 Но Иисус сказал ему: иди за Мною, и предоставь мертвым погребать своих мертвецов.
\par 23 И когда вошел Он в лодку, за Ним последовали ученики Его.
\par 24 И вот, сделалось великое волнение на море, так что лодка покрывалась волнами; а Он спал.
\par 25 Тогда ученики Его, подойдя к Нему, разбудили Его и сказали: Господи! спаси нас, погибаем.
\par 26 И говорит им: что вы [так] боязливы, маловерные? Потом, встав, запретил ветрам и морю, и сделалась великая тишина.
\par 27 Люди же, удивляясь, говорили: кто это, что и ветры и море повинуются Ему?
\par 28 И когда Он прибыл на другой берег в страну Гергесинскую, Его встретили два бесноватые, вышедшие из гробов, весьма свирепые, так что никто не смел проходить тем путем.
\par 29 И вот, они закричали: что Тебе до нас, Иисус, Сын Божий? пришел Ты сюда прежде времени мучить нас.
\par 30 Вдали же от них паслось большое стадо свиней.
\par 31 И бесы просили Его: если выгонишь нас, то пошли нас в стадо свиней.
\par 32 И Он сказал им: идите. И они, выйдя, пошли в стадо свиное. И вот, все стадо свиней бросилось с крутизны в море и погибло в воде.
\par 33 Пастухи же побежали и, придя в город, рассказали обо всем, и о том, что было с бесноватыми.
\par 34 И вот, весь город вышел навстречу Иисусу; и, увидев Его, просили, чтобы Он отошел от пределов их.

\chapter{9}

\par 1 Тогда Он, войдя в лодку, переправился [обратно] и прибыл в Свой город.
\par 2 И вот, принесли к Нему расслабленного, положенного на постели. И, видя Иисус веру их, сказал расслабленному: дерзай, чадо! прощаются тебе грехи твои.
\par 3 При сем некоторые из книжников сказали сами в себе: Он богохульствует.
\par 4 Иисус же, видя помышления их, сказал: для чего вы мыслите худое в сердцах ваших?
\par 5 ибо что легче сказать: прощаются тебе грехи, или сказать: встань и ходи?
\par 6 Но чтобы вы знали, что Сын Человеческий имеет власть на земле прощать грехи, --тогда говорит расслабленному: встань, возьми постель твою, и иди в дом твой.
\par 7 И он встал, [взял постель свою] и пошел в дом свой.
\par 8 Народ же, видев это, удивился и прославил Бога, давшего такую власть человекам.
\par 9 Проходя оттуда, Иисус увидел человека, сидящего у сбора пошлин, по имени Матфея, и говорит ему: следуй за Мною. И он встал и последовал за Ним.
\par 10 И когда Иисус возлежал в доме, многие мытари и грешники пришли и возлегли с Ним и учениками Его.
\par 11 Увидев то, фарисеи сказали ученикам Его: для чего Учитель ваш ест и пьет с мытарями и грешниками?
\par 12 Иисус же, услышав это, сказал им: не здоровые имеют нужду во враче, но больные,
\par 13 пойдите, научитесь, что значит: милости хочу, а не жертвы? Ибо Я пришел призвать не праведников, но грешников к покаянию.
\par 14 Тогда приходят к Нему ученики Иоанновы и говорят: почему мы и фарисеи постимся много, а Твои ученики не постятся?
\par 15 И сказал им Иисус: могут ли печалиться сыны чертога брачного, пока с ними жених? Но придут дни, когда отнимется у них жених, и тогда будут поститься.
\par 16 И никто к ветхой одежде не приставляет заплаты из небеленой ткани, ибо вновь пришитое отдерет от старого, и дыра будет еще хуже.
\par 17 Не вливают также вина молодого в мехи ветхие; а иначе прорываются мехи, и вино вытекает, и мехи пропадают, но вино молодое вливают в новые мехи, и сберегается то и другое.
\par 18 Когда Он говорил им сие, подошел к Нему некоторый начальник и, кланяясь Ему, говорил: дочь моя теперь умирает; но приди, возложи на нее руку Твою, и она будет жива.
\par 19 И встав, Иисус пошел за ним, и ученики Его.
\par 20 И вот, женщина, двенадцать лет страдавшая кровотечением, подойдя сзади, прикоснулась к краю одежды Его,
\par 21 ибо она говорила сама в себе: если только прикоснусь к одежде Его, выздоровею.
\par 22 Иисус же, обратившись и увидев ее, сказал: дерзай, дщерь! вера твоя спасла тебя. Женщина с того часа стала здорова.
\par 23 И когда пришел Иисус в дом начальника и увидел свирельщиков и народ в смятении,
\par 24 сказал им: выйдите вон, ибо не умерла девица, но спит. И смеялись над Ним.
\par 25 Когда же народ был выслан, Он, войдя, взял ее за руку, и девица встала.
\par 26 И разнесся слух о сем по всей земле той.
\par 27 Когда Иисус шел оттуда, за Ним следовали двое слепых и кричали: помилуй нас, Иисус, сын Давидов!
\par 28 Когда же Он пришел в дом, слепые приступили к Нему. И говорит им Иисус: веруете ли, что Я могу это сделать? Они говорят Ему: ей, Господи!
\par 29 Тогда Он коснулся глаз их и сказал: по вере вашей да будет вам.
\par 30 И открылись глаза их; и Иисус строго сказал им: смотрите, чтобы никто не узнал.
\par 31 А они, выйдя, разгласили о Нем по всей земле той.
\par 32 Когда же те выходили, то привели к Нему человека немого бесноватого.
\par 33 И когда бес был изгнан, немой стал говорить. И народ, удивляясь, говорил: никогда не бывало такого явления в Израиле.
\par 34 А фарисеи говорили: Он изгоняет бесов силою князя бесовского.
\par 35 И ходил Иисус по всем городам и селениям, уча в синагогах их, проповедуя Евангелие Царствия и исцеляя всякую болезнь и всякую немощь в людях.
\par 36 Видя толпы народа, Он сжалился над ними, что они были изнурены и рассеяны, как овцы, не имеющие пастыря.
\par 37 Тогда говорит ученикам Своим: жатвы много, а делателей мало;
\par 38 итак молите Господина жатвы, чтобы выслал делателей на жатву Свою.

\chapter{10}

\par 1 И призвав двенадцать учеников Своих, Он дал им власть над нечистыми духами, чтобы изгонять их и врачевать всякую болезнь и всякую немощь.
\par 2 Двенадцати же Апостолов имена суть сии: первый Симон, называемый Петром, и Андрей, брат его, Иаков Зеведеев и Иоанн, брат его,
\par 3 Филипп и Варфоломей, Фома и Матфей мытарь, Иаков Алфеев и Леввей, прозванный Фаддеем,
\par 4 Симон Кананит и Иуда Искариот, который и предал Его.
\par 5 Сих двенадцать послал Иисус, и заповедал им, говоря: на путь к язычникам не ходите, и в город Самарянский не входите;
\par 6 а идите наипаче к погибшим овцам дома Израилева;
\par 7 ходя же, проповедуйте, что приблизилось Царство Небесное;
\par 8 больных исцеляйте, прокаженных очищайте, мертвых воскрешайте, бесов изгоняйте; даром получили, даром давайте.
\par 9 Не берите с собою ни золота, ни серебра, ни меди в поясы свои,
\par 10 ни сумы на дорогу, ни двух одежд, ни обуви, ни посоха, ибо трудящийся достоин пропитания.
\par 11 В какой бы город или селение ни вошли вы, наведывайтесь, кто в нем достоин, и там оставайтесь, пока не выйдете;
\par 12 а входя в дом, приветствуйте его, говоря: мир дому сему;
\par 13 и если дом будет достоин, то мир ваш придет на него; если же не будет достоин, то мир ваш к вам возвратится.
\par 14 А если кто не примет вас и не послушает слов ваших, то, выходя из дома или из города того, отрясите прах от ног ваших;
\par 15 истинно говорю вам: отраднее будет земле Содомской и Гоморрской в день суда, нежели городу тому.
\par 16 Вот, Я посылаю вас, как овец среди волков: итак будьте мудры, как змии, и просты, как голуби.
\par 17 Остерегайтесь же людей: ибо они будут отдавать вас в судилища и в синагогах своих будут бить вас,
\par 18 и поведут вас к правителям и царям за Меня, для свидетельства перед ними и язычниками.
\par 19 Когда же будут предавать вас, не заботьтесь, как или что сказать; ибо в тот час дано будет вам, что сказать,
\par 20 ибо не вы будете говорить, но Дух Отца вашего будет говорить в вас.
\par 21 Предаст же брат брата на смерть, и отец--сына; и восстанут дети на родителей, и умертвят их;
\par 22 и будете ненавидимы всеми за имя Мое; претерпевший же до конца спасется.
\par 23 Когда же будут гнать вас в одном городе, бегите в другой. Ибо истинно говорю вам: не успеете обойти городов Израилевых, как приидет Сын Человеческий.
\par 24 Ученик не выше учителя, и слуга не выше господина своего:
\par 25 довольно для ученика, чтобы он был, как учитель его, и для слуги, чтобы он был, как господин его. Если хозяина дома назвали веельзевулом, не тем ли более домашних его?
\par 26 Итак не бойтесь их, ибо нет ничего сокровенного, что не открылось бы, и тайного, что не было бы узнано.
\par 27 Что говорю вам в темноте, говорите при свете; и что на ухо слышите, проповедуйте на кровлях.
\par 28 И не бойтесь убивающих тело, души же не могущих убить; а бойтесь более Того, Кто может и душу и тело погубить в геенне.
\par 29 Не две ли малые птицы продаются за ассарий? И ни одна из них не упадет на землю без [воли] Отца вашего;
\par 30 у вас же и волосы на голове все сочтены;
\par 31 не бойтесь же: вы лучше многих малых птиц.
\par 32 Итак всякого, кто исповедает Меня пред людьми, того исповедаю и Я пред Отцем Моим Небесным;
\par 33 а кто отречется от Меня пред людьми, отрекусь от того и Я пред Отцем Моим Небесным.
\par 34 Не думайте, что Я пришел принести мир на землю; не мир пришел Я принести, но меч,
\par 35 ибо Я пришел разделить человека с отцом его, и дочь с матерью ее, и невестку со свекровью ее.
\par 36 И враги человеку--домашние его.
\par 37 Кто любит отца или мать более, нежели Меня, не достоин Меня; и кто любит сына или дочь более, нежели Меня, не достоин Меня;
\par 38 и кто не берет креста своего и следует за Мною, тот не достоин Меня.
\par 39 Сберегший душу свою потеряет ее; а потерявший душу свою ради Меня сбережет ее.
\par 40 Кто принимает вас, принимает Меня, а кто принимает Меня, принимает Пославшего Меня;
\par 41 кто принимает пророка, во имя пророка, получит награду пророка; и кто принимает праведника, во имя праведника, получит награду праведника.
\par 42 И кто напоит одного из малых сих только чашею холодной воды, во имя ученика, истинно говорю вам, не потеряет награды своей.

\chapter{11}

\par 1 И когда окончил Иисус наставления двенадцати ученикам Своим, перешел оттуда учить и проповедывать в городах их.
\par 2 Иоанн же, услышав в темнице о делах Христовых, послал двоих из учеников своих
\par 3 сказать Ему: Ты ли Тот, Который должен придти, или ожидать нам другого?
\par 4 И сказал им Иисус в ответ: пойдите, скажите Иоанну, что слышите и видите:
\par 5 слепые прозревают и хромые ходят, прокаженные очищаются и глухие слышат, мертвые воскресают и нищие благовествуют;
\par 6 и блажен, кто не соблазнится о Мне.
\par 7 Когда же они пошли, Иисус начал говорить народу об Иоанне: что смотреть ходили вы в пустыню? трость ли, ветром колеблемую?
\par 8 Что же смотреть ходили вы? человека ли, одетого в мягкие одежды? Носящие мягкие одежды находятся в чертогах царских.
\par 9 Что же смотреть ходили вы? пророка? Да, говорю вам, и больше пророка.
\par 10 Ибо он тот, о котором написано: се, Я посылаю Ангела Моего пред лицем Твоим, который приготовит путь Твой пред Тобою.
\par 11 Истинно говорю вам: из рожденных женами не восставал больший Иоанна Крестителя; но меньший в Царстве Небесном больше его.
\par 12 От дней же Иоанна Крестителя доныне Царство Небесное силою берется, и употребляющие усилие восхищают его,
\par 13 ибо все пророки и закон прорекли до Иоанна.
\par 14 И если хотите принять, он есть Илия, которому должно придти.
\par 15 Кто имеет уши слышать, да слышит!
\par 16 Но кому уподоблю род сей? Он подобен детям, которые сидят на улице и, обращаясь к своим товарищам,
\par 17 говорят: мы играли вам на свирели, и вы не плясали; мы пели вам печальные песни, и вы не рыдали.
\par 18 Ибо пришел Иоанн, ни ест, ни пьет; и говорят: в нем бес.
\par 19 Пришел Сын Человеческий, ест и пьет; и говорят: вот человек, который любит есть и пить вино, друг мытарям и грешникам. И оправдана премудрость чадами ее.
\par 20 Тогда начал Он укорять города, в которых наиболее явлено было сил Его, за то, что они не покаялись:
\par 21 горе тебе, Хоразин! горе тебе, Вифсаида! ибо если бы в Тире и Сидоне явлены были силы, явленные в вас, то давно бы они во вретище и пепле покаялись,
\par 22 но говорю вам: Тиру и Сидону отраднее будет в день суда, нежели вам.
\par 23 И ты, Капернаум, до неба вознесшийся, до ада низвергнешься, ибо если бы в Содоме явлены были силы, явленные в тебе, то он оставался бы до сего дня;
\par 24 но говорю вам, что земле Содомской отраднее будет в день суда, нежели тебе.
\par 25 В то время, продолжая речь, Иисус сказал: славлю Тебя, Отче, Господи неба и земли, что Ты утаил сие от мудрых и разумных и открыл то младенцам;
\par 26 ей, Отче! ибо таково было Твое благоволение.
\par 27 Все предано Мне Отцем Моим, и никто не знает Сына, кроме Отца; и Отца не знает никто, кроме Сына, и кому Сын хочет открыть.
\par 28 Придите ко Мне все труждающиеся и обремененные, и Я успокою вас;
\par 29 возьмите иго Мое на себя и научитесь от Меня, ибо Я кроток и смирен сердцем, и найдете покой душам вашим;
\par 30 ибо иго Мое благо, и бремя Мое легко.

\chapter{12}

\par 1 В то время проходил Иисус в субботу засеянными полями; ученики же Его взалкали и начали срывать колосья и есть.
\par 2 Фарисеи, увидев это, сказали Ему: вот, ученики Твои делают, чего не должно делать в субботу.
\par 3 Он же сказал им: разве вы не читали, что сделал Давид, когда взалкал сам и бывшие с ним?
\par 4 как он вошел в дом Божий и ел хлебы предложения, которых не должно было есть ни ему, ни бывшим с ним, а только одним священникам?
\par 5 Или не читали ли вы в законе, что в субботы священники в храме нарушают субботу, однако невиновны?
\par 6 Но говорю вам, что здесь Тот, Кто больше храма;
\par 7 если бы вы знали, что значит: милости хочу, а не жертвы, то не осудили бы невиновных,
\par 8 ибо Сын Человеческий есть господин и субботы.
\par 9 И, отойдя оттуда, вошел Он в синагогу их.
\par 10 И вот, там был человек, имеющий сухую руку. И спросили Иисуса, чтобы обвинить Его: можно ли исцелять в субботы?
\par 11 Он же сказал им: кто из вас, имея одну овцу, если она в субботу упадет в яму, не возьмет ее и не вытащит?
\par 12 Сколько же лучше человек овцы! Итак можно в субботы делать добро.
\par 13 Тогда говорит человеку тому: протяни руку твою. И он протянул, и стала она здорова, как другая.
\par 14 Фарисеи же, выйдя, имели совещание против Него, как бы погубить Его. Но Иисус, узнав, удалился оттуда.
\par 15 И последовало за Ним множество народа, и Он исцелил их всех
\par 16 и запретил им объявлять о Нем,
\par 17 да сбудется реченное через пророка Исаию, который говорит:
\par 18 Се, Отрок Мой, Которого Я избрал, Возлюбленный Мой, Которому благоволит душа Моя. Положу дух Мой на Него, и возвестит народам суд;
\par 19 не воспрекословит, не возопиет, и никто не услышит на улицах голоса Его;
\par 20 трости надломленной не переломит, и льна курящегося не угасит, доколе не доставит суду победы;
\par 21 и на имя Его будут уповать народы.
\par 22 Тогда привели к Нему бесноватого слепого и немого; и исцелил его, так что слепой и немой стал и говорить и видеть.
\par 23 И дивился весь народ и говорил: не это ли Христос, сын Давидов?
\par 24 Фарисеи же, услышав [сие], сказали: Он изгоняет бесов не иначе, как [силою] веельзевула, князя бесовского.
\par 25 Но Иисус, зная помышления их, сказал им: всякое царство, разделившееся само в себе, опустеет; и всякий город или дом, разделившийся сам в себе, не устоит.
\par 26 И если сатана сатану изгоняет, то он разделился сам с собою: как же устоит царство его?
\par 27 И если Я [силою] веельзевула изгоняю бесов, то сыновья ваши чьею [силою] изгоняют? Посему они будут вам судьями.
\par 28 Если же Я Духом Божиим изгоняю бесов, то конечно достигло до вас Царствие Божие.
\par 29 Или, как может кто войти в дом сильного и расхитить вещи его, если прежде не свяжет сильного? и тогда расхитит дом его.
\par 30 Кто не со Мною, тот против Меня; и кто не собирает со Мною, тот расточает.
\par 31 Посему говорю вам: всякий грех и хула простятся человекам, а хула на Духа не простится человекам;
\par 32 если кто скажет слово на Сына Человеческого, простится ему; если же кто скажет на Духа Святаго, не простится ему ни в сем веке, ни в будущем.
\par 33 Или признайте дерево хорошим и плод его хорошим; или признайте дерево худым и плод его худым, ибо дерево познается по плоду.
\par 34 Порождения ехиднины! как вы можете говорить доброе, будучи злы? Ибо от избытка сердца говорят уста.
\par 35 Добрый человек из доброго сокровища выносит доброе, а злой человек из злого сокровища выносит злое.
\par 36 Говорю же вам, что за всякое праздное слово, какое скажут люди, дадут они ответ в день суда:
\par 37 ибо от слов своих оправдаешься, и от слов своих осудишься.
\par 38 Тогда некоторые из книжников и фарисеев сказали: Учитель! хотелось бы нам видеть от Тебя знамение.
\par 39 Но Он сказал им в ответ: род лукавый и прелюбодейный ищет знамения; и знамение не дастся ему, кроме знамения Ионы пророка;
\par 40 ибо как Иона был во чреве кита три дня и три ночи, так и Сын Человеческий будет в сердце земли три дня и три ночи.
\par 41 Ниневитяне восстанут на суд с родом сим и осудят его, ибо они покаялись от проповеди Иониной; и вот, здесь больше Ионы.
\par 42 Царица южная восстанет на суд с родом сим и осудит его, ибо она приходила от пределов земли послушать мудрости Соломоновой; и вот, здесь больше Соломона.
\par 43 Когда нечистый дух выйдет из человека, то ходит по безводным местам, ища покоя, и не находит;
\par 44 тогда говорит: возвращусь в дом мой, откуда я вышел. И, придя, находит [его] незанятым, выметенным и убранным;
\par 45 тогда идет и берет с собою семь других духов, злейших себя, и, войдя, живут там; и бывает для человека того последнее хуже первого. Так будет и с этим злым родом.
\par 46 Когда же Он еще говорил к народу, Матерь и братья Его стояли вне [дома], желая говорить с Ним.
\par 47 И некто сказал Ему: вот Матерь Твоя и братья Твои стоят вне, желая говорить с Тобою.
\par 48 Он же сказал в ответ говорившему: кто Матерь Моя? и кто братья Мои?
\par 49 И, указав рукою Своею на учеников Своих, сказал: вот матерь Моя и братья Мои;
\par 50 ибо, кто будет исполнять волю Отца Моего Небесного, тот Мне брат, и сестра, и матерь.

\chapter{13}

\par 1 Выйдя же в день тот из дома, Иисус сел у моря.
\par 2 И собралось к Нему множество народа, так что Он вошел в лодку и сел; а весь народ стоял на берегу.
\par 3 И поучал их много притчами, говоря: вот, вышел сеятель сеять;
\par 4 и когда он сеял, иное упало при дороге, и налетели птицы и поклевали то;
\par 5 иное упало на места каменистые, где немного было земли, и скоро взошло, потому что земля была неглубока.
\par 6 Когда же взошло солнце, увяло, и, как не имело корня, засохло;
\par 7 иное упало в терние, и выросло терние и заглушило его;
\par 8 иное упало на добрую землю и принесло плод: одно во сто крат, а другое в шестьдесят, иное же в тридцать.
\par 9 Кто имеет уши слышать, да слышит!
\par 10 И, приступив, ученики сказали Ему: для чего притчами говоришь им?
\par 11 Он сказал им в ответ: для того, что вам дано знать тайны Царствия Небесного, а им не дано,
\par 12 ибо кто имеет, тому дано будет и приумножится, а кто не имеет, у того отнимется и то, что имеет;
\par 13 потому говорю им притчами, что они видя не видят, и слыша не слышат, и не разумеют;
\par 14 и сбывается над ними пророчество Исаии, которое говорит: слухом услышите--и не уразумеете, и глазами смотреть будете--и не увидите,
\par 15 ибо огрубело сердце людей сих и ушами с трудом слышат, и глаза свои сомкнули, да не увидят глазами и не услышат ушами, и не уразумеют сердцем, и да не обратятся, чтобы Я исцелил их.
\par 16 Ваши же блаженны очи, что видят, и уши ваши, что слышат,
\par 17 ибо истинно говорю вам, что многие пророки и праведники желали видеть, что вы видите, и не видели, и слышать, что вы слышите, и не слышали.
\par 18 Вы же выслушайте [значение] притчи о сеятеле:
\par 19 ко всякому, слушающему слово о Царствии и не разумеющему, приходит лукавый и похищает посеянное в сердце его--вот кого означает посеянное при дороге.
\par 20 А посеянное на каменистых местах означает того, кто слышит слово и тотчас с радостью принимает его;
\par 21 но не имеет в себе корня и непостоянен: когда настанет скорбь или гонение за слово, тотчас соблазняется.
\par 22 А посеянное в тернии означает того, кто слышит слово, но забота века сего и обольщение богатства заглушает слово, и оно бывает бесплодно.
\par 23 Посеянное же на доброй земле означает слышащего слово и разумеющего, который и бывает плодоносен, так что иной приносит плод во сто крат, иной в шестьдесят, а иной в тридцать.
\par 24 Другую притчу предложил Он им, говоря: Царство Небесное подобно человеку, посеявшему доброе семя на поле своем;
\par 25 когда же люди спали, пришел враг его и посеял между пшеницею плевелы и ушел;
\par 26 когда взошла зелень и показался плод, тогда явились и плевелы.
\par 27 Придя же, рабы домовладыки сказали ему: господин! не доброе ли семя сеял ты на поле твоем? откуда же на нем плевелы?
\par 28 Он же сказал им: враг человека сделал это. А рабы сказали ему: хочешь ли, мы пойдем, выберем их?
\par 29 Но он сказал: нет, --чтобы, выбирая плевелы, вы не выдергали вместе с ними пшеницы,
\par 30 оставьте расти вместе то и другое до жатвы; и во время жатвы я скажу жнецам: соберите прежде плевелы и свяжите их в снопы, чтобы сжечь их, а пшеницу уберите в житницу мою.
\par 31 Иную притчу предложил Он им, говоря: Царство Небесное подобно зерну горчичному, которое человек взял и посеял на поле своем,
\par 32 которое, хотя меньше всех семян, но, когда вырастет, бывает больше всех злаков и становится деревом, так что прилетают птицы небесные и укрываются в ветвях его.
\par 33 Иную притчу сказал Он им: Царство Небесное подобно закваске, которую женщина, взяв, положила в три меры муки, доколе не вскисло все.
\par 34 Все сие Иисус говорил народу притчами, и без притчи не говорил им,
\par 35 да сбудется реченное через пророка, который говорит: отверзу в притчах уста Мои; изреку сокровенное от создания мира.
\par 36 Тогда Иисус, отпустив народ, вошел в дом. И, приступив к Нему, ученики Его сказали: изъясни нам притчу о плевелах на поле.
\par 37 Он же сказал им в ответ: сеющий доброе семя есть Сын Человеческий;
\par 38 поле есть мир; доброе семя, это сыны Царствия, а плевелы--сыны лукавого;
\par 39 враг, посеявший их, есть диавол; жатва есть кончина века, а жнецы суть Ангелы.
\par 40 Посему как собирают плевелы и огнем сжигают, так будет при кончине века сего:
\par 41 пошлет Сын Человеческий Ангелов Своих, и соберут из Царства Его все соблазны и делающих беззаконие,
\par 42 и ввергнут их в печь огненную; там будет плач и скрежет зубов;
\par 43 тогда праведники воссияют, как солнце, в Царстве Отца их. Кто имеет уши слышать, да слышит!
\par 44 Еще подобно Царство Небесное сокровищу, скрытому на поле, которое, найдя, человек утаил, и от радости о нем идет и продает все, что имеет, и покупает поле то.
\par 45 Еще подобно Царство Небесное купцу, ищущему хороших жемчужин,
\par 46 который, найдя одну драгоценную жемчужину, пошел и продал все, что имел, и купил ее.
\par 47 Еще подобно Царство Небесное неводу, закинутому в море и захватившему рыб всякого рода,
\par 48 который, когда наполнился, вытащили на берег и, сев, хорошее собрали в сосуды, а худое выбросили вон.
\par 49 Так будет при кончине века: изыдут Ангелы, и отделят злых из среды праведных,
\par 50 и ввергнут их в печь огненную: там будет плач и скрежет зубов.
\par 51 И спросил их Иисус: поняли ли вы все это? Они говорят Ему: так, Господи!
\par 52 Он же сказал им: поэтому всякий книжник, наученный Царству Небесному, подобен хозяину, который выносит из сокровищницы своей новое и старое.
\par 53 И, когда окончил Иисус притчи сии, пошел оттуда.
\par 54 И, придя в отечество Свое, учил их в синагоге их, так что они изумлялись и говорили: откуда у Него такая премудрость и силы?
\par 55 не плотников ли Он сын? не Его ли Мать называется Мария, и братья Его Иаков и Иосий, и Симон, и Иуда?
\par 56 и сестры Его не все ли между нами? откуда же у Него все это?
\par 57 И соблазнялись о Нем. Иисус же сказал им: не бывает пророк без чести, разве только в отечестве своем и в доме своем.
\par 58 И не совершил там многих чудес по неверию их.

\chapter{14}

\par 1 В то время Ирод четвертовластник услышал молву об Иисусе
\par 2 и сказал служащим при нем: это Иоанн Креститель; он воскрес из мертвых, и потому чудеса делаются им.
\par 3 Ибо Ирод, взяв Иоанна, связал его и посадил в темницу за Иродиаду, жену Филиппа, брата своего,
\par 4 потому что Иоанн говорил ему: не должно тебе иметь ее.
\par 5 И хотел убить его, но боялся народа, потому что его почитали за пророка.
\par 6 Во время же [празднования] дня рождения Ирода дочь Иродиады плясала перед собранием и угодила Ироду,
\par 7 посему он с клятвою обещал ей дать, чего она ни попросит.
\par 8 Она же, по наущению матери своей, сказала: дай мне здесь на блюде голову Иоанна Крестителя.
\par 9 И опечалился царь, но, ради клятвы и возлежащих с ним, повелел дать ей,
\par 10 и послал отсечь Иоанну голову в темнице.
\par 11 И принесли голову его на блюде и дали девице, а она отнесла матери своей.
\par 12 Ученики же его, придя, взяли тело его и погребли его; и пошли, возвестили Иисусу.
\par 13 И, услышав, Иисус удалился оттуда на лодке в пустынное место один; а народ, услышав о том, пошел за Ним из городов пешком.
\par 14 И, выйдя, Иисус увидел множество людей и сжалился над ними, и исцелил больных их.
\par 15 Когда же настал вечер, приступили к Нему ученики Его и сказали: место здесь пустынное и время уже позднее; отпусти народ, чтобы они пошли в селения и купили себе пищи.
\par 16 Но Иисус сказал им: не нужно им идти, вы дайте им есть.
\par 17 Они же говорят Ему: у нас здесь только пять хлебов и две рыбы.
\par 18 Он сказал: принесите их Мне сюда.
\par 19 И велел народу возлечь на траву и, взяв пять хлебов и две рыбы, воззрел на небо, благословил и, преломив, дал хлебы ученикам, а ученики народу.
\par 20 И ели все и насытились; и набрали оставшихся кусков двенадцать коробов полных;
\par 21 а евших было около пяти тысяч человек, кроме женщин и детей.
\par 22 И тотчас понудил Иисус учеников Своих войти в лодку и отправиться прежде Его на другую сторону, пока Он отпустит народ.
\par 23 И, отпустив народ, Он взошел на гору помолиться наедине; и вечером оставался там один.
\par 24 А лодка была уже на средине моря, и ее било волнами, потому что ветер был противный.
\par 25 В четвертую же стражу ночи пошел к ним Иисус, идя по морю.
\par 26 И ученики, увидев Его идущего по морю, встревожились и говорили: это призрак; и от страха вскричали.
\par 27 Но Иисус тотчас заговорил с ними и сказал: ободритесь; это Я, не бойтесь.
\par 28 Петр сказал Ему в ответ: Господи! если это Ты, повели мне придти к Тебе по воде.
\par 29 Он же сказал: иди. И, выйдя из лодки, Петр пошел по воде, чтобы подойти к Иисусу,
\par 30 но, видя сильный ветер, испугался и, начав утопать, закричал: Господи! спаси меня.
\par 31 Иисус тотчас простер руку, поддержал его и говорит ему: маловерный! зачем ты усомнился?
\par 32 И, когда вошли они в лодку, ветер утих.
\par 33 Бывшие же в лодке подошли, поклонились Ему и сказали: истинно Ты Сын Божий.
\par 34 И, переправившись, прибыли в землю Геннисаретскую.
\par 35 Жители того места, узнав Его, послали во всю окрестность ту и принесли к Нему всех больных,
\par 36 и просили Его, чтобы только прикоснуться к краю одежды Его; и которые прикасались, исцелялись.

\chapter{15}

\par 1 Тогда приходят к Иисусу Иерусалимские книжники и фарисеи и говорят:
\par 2 зачем ученики Твои преступают предание старцев? ибо не умывают рук своих, когда едят хлеб.
\par 3 Он же сказал им в ответ: зачем и вы преступаете заповедь Божию ради предания вашего?
\par 4 Ибо Бог заповедал: почитай отца и мать; и: злословящий отца или мать смертью да умрет.
\par 5 А вы говорите: если кто скажет отцу или матери: дар [Богу] то, чем бы ты от меня пользовался,
\par 6 тот может и не почтить отца своего или мать свою; таким образом вы устранили заповедь Божию преданием вашим.
\par 7 Лицемеры! хорошо пророчествовал о вас Исаия, говоря:
\par 8 приближаются ко Мне люди сии устами своими, и чтут Меня языком, сердце же их далеко отстоит от Меня;
\par 9 но тщетно чтут Меня, уча учениям, заповедям человеческим.
\par 10 И, призвав народ, сказал им: слушайте и разумейте!
\par 11 не то, что входит в уста, оскверняет человека, но то, что выходит из уст, оскверняет человека.
\par 12 Тогда ученики Его, приступив, сказали Ему: знаешь ли, что фарисеи, услышав слово сие, соблазнились?
\par 13 Он же сказал в ответ: всякое растение, которое не Отец Мой Небесный насадил, искоренится;
\par 14 оставьте их: они--слепые вожди слепых; а если слепой ведет слепого, то оба упадут в яму.
\par 15 Петр же, отвечая, сказал Ему: изъясни нам притчу сию.
\par 16 Иисус сказал: неужели и вы еще не разумеете?
\par 17 еще ли не понимаете, что все, входящее в уста, проходит в чрево и извергается вон?
\par 18 а исходящее из уст--из сердца исходит--сие оскверняет человека,
\par 19 ибо из сердца исходят злые помыслы, убийства, прелюбодеяния, любодеяния, кражи, лжесвидетельства, хуления--
\par 20 это оскверняет человека; а есть неумытыми руками--не оскверняет человека.
\par 21 И, выйдя оттуда, Иисус удалился в страны Тирские и Сидонские.
\par 22 И вот, женщина Хананеянка, выйдя из тех мест, кричала Ему: помилуй меня, Господи, сын Давидов, дочь моя жестоко беснуется.
\par 23 Но Он не отвечал ей ни слова. И ученики Его, приступив, просили Его: отпусти ее, потому что кричит за нами.
\par 24 Он же сказал в ответ: Я послан только к погибшим овцам дома Израилева.
\par 25 А она, подойдя, кланялась Ему и говорила: Господи! помоги мне.
\par 26 Он же сказал в ответ: нехорошо взять хлеб у детей и бросить псам.
\par 27 Она сказала: так, Господи! но и псы едят крохи, которые падают со стола господ их.
\par 28 Тогда Иисус сказал ей в ответ: о, женщина! велика вера твоя; да будет тебе по желанию твоему. И исцелилась дочь ее в тот час.
\par 29 Перейдя оттуда, пришел Иисус к морю Галилейскому и, взойдя на гору, сел там.
\par 30 И приступило к Нему множество народа, имея с собою хромых, слепых, немых, увечных и иных многих, и повергли их к ногам Иисусовым; и Он исцелил их;
\par 31 так что народ дивился, видя немых говорящими, увечных здоровыми, хромых ходящими и слепых видящими; и прославлял Бога Израилева.
\par 32 Иисус же, призвав учеников Своих, сказал им: жаль Мне народа, что уже три дня находятся при Мне, и нечего им есть; отпустить же их неевшими не хочу, чтобы не ослабели в дороге.
\par 33 И говорят Ему ученики Его: откуда нам взять в пустыне столько хлебов, чтобы накормить столько народа?
\par 34 Говорит им Иисус: сколько у вас хлебов? Они же сказали: семь, и немного рыбок.
\par 35 Тогда велел народу возлечь на землю.
\par 36 И, взяв семь хлебов и рыбы, воздал благодарение, преломил и дал ученикам Своим, а ученики народу.
\par 37 И ели все и насытились; и набрали оставшихся кусков семь корзин полных,
\par 38 а евших было четыре тысячи человек, кроме женщин и детей.
\par 39 И, отпустив народ, Он вошел в лодку и прибыл в пределы Магдалинские.

\chapter{16}

\par 1 И приступили фарисеи и саддукеи и, искушая Его, просили показать им знамение с неба.
\par 2 Он же сказал им в ответ: вечером вы говорите: будет ведро, потому что небо красно;
\par 3 и поутру: сегодня ненастье, потому что небо багрово. Лицемеры! различать лице неба вы умеете, а знамений времен не можете.
\par 4 Род лукавый и прелюбодейный знамения ищет, и знамение не дастся ему, кроме знамения Ионы пророка. И, оставив их, отошел.
\par 5 Переправившись на другую сторону, ученики Его забыли взять хлебов.
\par 6 Иисус сказал им: смотрите, берегитесь закваски фарисейской и саддукейской.
\par 7 Они же помышляли в себе и говорили: [это значит], что хлебов мы не взяли.
\par 8 Уразумев то, Иисус сказал им: что помышляете в себе, маловерные, что хлебов не взяли?
\par 9 Еще ли не понимаете и не помните о пяти хлебах на пять тысяч [человек], и сколько коробов вы набрали?
\par 10 ни о семи хлебах на четыре тысячи, и сколько корзин вы набрали?
\par 11 как не разумеете, что не о хлебе сказал Я вам: берегитесь закваски фарисейской и саддукейской?
\par 12 Тогда они поняли, что Он говорил им беречься не закваски хлебной, но учения фарисейского и саддукейского.
\par 13 Придя же в страны Кесарии Филипповой, Иисус спрашивал учеников Своих: за кого люди почитают Меня, Сына Человеческого?
\par 14 Они сказали: одни за Иоанна Крестителя, другие за Илию, а иные за Иеремию, или за одного из пророков.
\par 15 Он говорит им: а вы за кого почитаете Меня?
\par 16 Симон же Петр, отвечая, сказал: Ты--Христос, Сын Бога Живаго.
\par 17 Тогда Иисус сказал ему в ответ: блажен ты, Симон, сын Ионин, потому что не плоть и кровь открыли тебе это, но Отец Мой, Сущий на небесах;
\par 18 и Я говорю тебе: ты--Петр, и на сем камне Я создам Церковь Мою, и врата ада не одолеют ее;
\par 19 и дам тебе ключи Царства Небесного: и что свяжешь на земле, то будет связано на небесах, и что разрешишь на земле, то будет разрешено на небесах.
\par 20 Тогда (Иисус) запретил ученикам Своим, чтобы никому не сказывали, что Он есть Иисус Христос.
\par 21 С того времени Иисус начал открывать ученикам Своим, что Ему должно идти в Иерусалим и много пострадать от старейшин и первосвященников и книжников, и быть убиту, и в третий день воскреснуть.
\par 22 И, отозвав Его, Петр начал прекословить Ему: будь милостив к Себе, Господи! да не будет этого с Тобою!
\par 23 Он же, обратившись, сказал Петру: отойди от Меня, сатана! ты Мне соблазн! потому что думаешь не о том, что Божие, но что человеческое.
\par 24 Тогда Иисус сказал ученикам Своим: если кто хочет идти за Мною, отвергнись себя, и возьми крест свой, и следуй за Мною,
\par 25 ибо кто хочет душу свою сберечь, тот потеряет ее, а кто потеряет душу свою ради Меня, тот обретет ее;
\par 26 какая польза человеку, если он приобретет весь мир, а душе своей повредит? или какой выкуп даст человек за душу свою?
\par 27 ибо приидет Сын Человеческий во славе Отца Своего с Ангелами Своими и тогда воздаст каждому по делам его.
\par 28 Истинно говорю вам: есть некоторые из стоящих здесь, которые не вкусят смерти, как уже увидят Сына Человеческого, грядущего в Царствии Своем.

\chapter{17}

\par 1 По прошествии дней шести, взял Иисус Петра, Иакова и Иоанна, брата его, и возвел их на гору высокую одних,
\par 2 и преобразился пред ними: и просияло лице Его, как солнце, одежды же Его сделались белыми, как свет.
\par 3 И вот, явились им Моисей и Илия, с Ним беседующие.
\par 4 При сем Петр сказал Иисусу: Господи! хорошо нам здесь быть; если хочешь, сделаем здесь три кущи: Тебе одну, и Моисею одну, и одну Илии.
\par 5 Когда он еще говорил, се, облако светлое осенило их; и се, глас из облака глаголющий: Сей есть Сын Мой Возлюбленный, в Котором Мое благоволение; Его слушайте.
\par 6 И, услышав, ученики пали на лица свои и очень испугались.
\par 7 Но Иисус, приступив, коснулся их и сказал: встаньте и не бойтесь.
\par 8 Возведя же очи свои, они никого не увидели, кроме одного Иисуса.
\par 9 И когда сходили они с горы, Иисус запретил им, говоря: никому не сказывайте о сем видении, доколе Сын Человеческий не воскреснет из мертвых.
\par 10 И спросили Его ученики Его: как же книжники говорят, что Илии надлежит придти прежде?
\par 11 Иисус сказал им в ответ: правда, Илия должен придти прежде и устроить все;
\par 12 но говорю вам, что Илия уже пришел, и не узнали его, а поступили с ним, как хотели; так и Сын Человеческий пострадает от них.
\par 13 Тогда ученики поняли, что Он говорил им об Иоанне Крестителе.
\par 14 Когда они пришли к народу, то подошел к Нему человек и, преклоняя пред Ним колени,
\par 15 сказал: Господи! помилуй сына моего; он в новолуния [беснуется] и тяжко страдает, ибо часто бросается в огонь и часто в воду,
\par 16 я приводил его к ученикам Твоим, и они не могли исцелить его.
\par 17 Иисус же, отвечая, сказал: о, род неверный и развращенный! доколе буду с вами? доколе буду терпеть вас? приведите его ко Мне сюда.
\par 18 И запретил ему Иисус, и бес вышел из него; и отрок исцелился в тот час.
\par 19 Тогда ученики, приступив к Иисусу наедине, сказали: почему мы не могли изгнать его?
\par 20 Иисус же сказал им: по неверию вашему; ибо истинно говорю вам: если вы будете иметь веру с горчичное зерно и скажете горе сей: `перейди отсюда туда', и она перейдет; и ничего не будет невозможного для вас;
\par 21 сей же род изгоняется только молитвою и постом.
\par 22 Во время пребывания их в Галилее, Иисус сказал им: Сын Человеческий предан будет в руки человеческие,
\par 23 и убьют Его, и в третий день воскреснет. И они весьма опечалились.
\par 24 Когда же пришли они в Капернаум, то подошли к Петру собиратели дидрахм и сказали: Учитель ваш не даст ли дидрахмы?
\par 25 Он говорит: да. И когда вошел он в дом, то Иисус, предупредив его, сказал: как тебе кажется, Симон? цари земные с кого берут пошлины или подати? с сынов ли своих, или с посторонних?
\par 26 Петр говорит Ему: с посторонних. Иисус сказал ему: итак сыны свободны;
\par 27 но, чтобы нам не соблазнить их, пойди на море, брось уду, и первую рыбу, которая попадется, возьми, и, открыв у ней рот, найдешь статир; возьми его и отдай им за Меня и за себя.

\chapter{18}

\par 1 В то время ученики приступили к Иисусу и сказали: кто больше в Царстве Небесном?
\par 2 Иисус, призвав дитя, поставил его посреди них
\par 3 и сказал: истинно говорю вам, если не обратитесь и не будете как дети, не войдете в Царство Небесное;
\par 4 итак, кто умалится, как это дитя, тот и больше в Царстве Небесном;
\par 5 и кто примет одно такое дитя во имя Мое, тот Меня принимает;
\par 6 а кто соблазнит одного из малых сих, верующих в Меня, тому лучше было бы, если бы повесили ему мельничный жернов на шею и потопили его во глубине морской.
\par 7 Горе миру от соблазнов, ибо надобно придти соблазнам; но горе тому человеку, через которого соблазн приходит.
\par 8 Если же рука твоя или нога твоя соблазняет тебя, отсеки их и брось от себя: лучше тебе войти в жизнь без руки или без ноги, нежели с двумя руками и с двумя ногами быть ввержену в огонь вечный;
\par 9 и если глаз твой соблазняет тебя, вырви его и брось от себя: лучше тебе с одним глазом войти в жизнь, нежели с двумя глазами быть ввержену в геенну огненную.
\par 10 Смотрите, не презирайте ни одного из малых сих; ибо говорю вам, что Ангелы их на небесах всегда видят лице Отца Моего Небесного.
\par 11 Ибо Сын Человеческий пришел взыскать и спасти погибшее.
\par 12 Как вам кажется? Если бы у кого было сто овец, и одна из них заблудилась, то не оставит ли он девяносто девять в горах и не пойдет ли искать заблудившуюся?
\par 13 и если случится найти ее, то, истинно говорю вам, он радуется о ней более, нежели о девяноста девяти незаблудившихся.
\par 14 Так, нет воли Отца вашего Небесного, чтобы погиб один из малых сих.
\par 15 Если же согрешит против тебя брат твой, пойди и обличи его между тобою и им одним; если послушает тебя, то приобрел ты брата твоего;
\par 16 если же не послушает, возьми с собою еще одного или двух, дабы устами двух или трех свидетелей подтвердилось всякое слово;
\par 17 если же не послушает их, скажи церкви; а если и церкви не послушает, то да будет он тебе, как язычник и мытарь.
\par 18 Истинно говорю вам: что вы свяжете на земле, то будет связано на небе; и что разрешите на земле, то будет разрешено на небе.
\par 19 Истинно также говорю вам, что если двое из вас согласятся на земле просить о всяком деле, то, чего бы ни попросили, будет им от Отца Моего Небесного,
\par 20 ибо, где двое или трое собраны во имя Мое, там Я посреди них.
\par 21 Тогда Петр приступил к Нему и сказал: Господи! сколько раз прощать брату моему, согрешающему против меня? до семи ли раз?
\par 22 Иисус говорит ему: не говорю тебе: до семи раз, но до седмижды семидесяти раз.
\par 23 Посему Царство Небесное подобно царю, который захотел сосчитаться с рабами своими;
\par 24 когда начал он считаться, приведен был к нему некто, который должен был ему десять тысяч талантов;
\par 25 а как он не имел, чем заплатить, то государь его приказал продать его, и жену его, и детей, и все, что он имел, и заплатить;
\par 26 тогда раб тот пал, и, кланяясь ему, говорил: государь! потерпи на мне, и все тебе заплачу.
\par 27 Государь, умилосердившись над рабом тем, отпустил его и долг простил ему.
\par 28 Раб же тот, выйдя, нашел одного из товарищей своих, который должен был ему сто динариев, и, схватив его, душил, говоря: отдай мне, что должен.
\par 29 Тогда товарищ его пал к ногам его, умолял его и говорил: потерпи на мне, и все отдам тебе.
\par 30 Но тот не захотел, а пошел и посадил его в темницу, пока не отдаст долга.
\par 31 Товарищи его, видев происшедшее, очень огорчились и, придя, рассказали государю своему все бывшее.
\par 32 Тогда государь его призывает его и говорит: злой раб! весь долг тот я простил тебе, потому что ты упросил меня;
\par 33 не надлежало ли и тебе помиловать товарища твоего, как и я помиловал тебя?
\par 34 И, разгневавшись, государь его отдал его истязателям, пока не отдаст ему всего долга.
\par 35 Так и Отец Мой Небесный поступит с вами, если не простит каждый из вас от сердца своего брату своему согрешений его.

\chapter{19}

\par 1 Когда Иисус окончил слова сии, то вышел из Галилеи и пришел в пределы Иудейские, Заиорданскою стороною.
\par 2 За Ним последовало много людей, и Он исцелил их там.
\par 3 И приступили к Нему фарисеи и, искушая Его, говорили Ему: по всякой ли причине позволительно человеку разводиться с женою своею?
\par 4 Он сказал им в ответ: не читали ли вы, что Сотворивший вначале мужчину и женщину сотворил их?
\par 5 И сказал: посему оставит человек отца и мать и прилепится к жене своей, и будут два одною плотью,
\par 6 так что они уже не двое, но одна плоть. Итак, что Бог сочетал, того человек да не разлучает.
\par 7 Они говорят Ему: как же Моисей заповедал давать разводное письмо и разводиться с нею?
\par 8 Он говорит им: Моисей по жестокосердию вашему позволил вам разводиться с женами вашими, а сначала не было так;
\par 9 но Я говорю вам: кто разведется с женою своею не за прелюбодеяние и женится на другой, [тот] прелюбодействует; и женившийся на разведенной прелюбодействует.
\par 10 Говорят Ему ученики Его: если такова обязанность человека к жене, то лучше не жениться.
\par 11 Он же сказал им: не все вмещают слово сие, но кому дано,
\par 12 ибо есть скопцы, которые из чрева матернего родились так; и есть скопцы, которые оскоплены от людей; и есть скопцы, которые сделали сами себя скопцами для Царства Небесного. Кто может вместить, да вместит.
\par 13 Тогда приведены были к Нему дети, чтобы Он возложил на них руки и помолился; ученики же возбраняли им.
\par 14 Но Иисус сказал: пустите детей и не препятствуйте им приходить ко Мне, ибо таковых есть Царство Небесное.
\par 15 И, возложив на них руки, пошел оттуда.
\par 16 И вот, некто, подойдя, сказал Ему: Учитель благий! что сделать мне доброго, чтобы иметь жизнь вечную?
\par 17 Он же сказал ему: что ты называешь Меня благим? Никто не благ, как только один Бог. Если же хочешь войти в жизнь [вечную], соблюди заповеди.
\par 18 Говорит Ему: какие? Иисус же сказал: не убивай; не прелюбодействуй; не кради; не лжесвидетельствуй;
\par 19 почитай отца и мать; и: люби ближнего твоего, как самого себя.
\par 20 Юноша говорит Ему: все это сохранил я от юности моей; чего еще недостает мне?
\par 21 Иисус сказал ему: если хочешь быть совершенным, пойди, продай имение твое и раздай нищим; и будешь иметь сокровище на небесах; и приходи и следуй за Мною.
\par 22 Услышав слово сие, юноша отошел с печалью, потому что у него было большое имение.
\par 23 Иисус же сказал ученикам Своим: истинно говорю вам, что трудно богатому войти в Царство Небесное;
\par 24 и еще говорю вам: удобнее верблюду пройти сквозь игольные уши, нежели богатому войти в Царство Божие.
\par 25 Услышав это, ученики Его весьма изумились и сказали: так кто же может спастись?
\par 26 А Иисус, воззрев, сказал им: человекам это невозможно, Богу же все возможно.
\par 27 Тогда Петр, отвечая, сказал Ему: вот, мы оставили все и последовали за Тобою; что же будет нам?
\par 28 Иисус же сказал им: истинно говорю вам, что вы, последовавшие за Мною, --в пакибытии, когда сядет Сын Человеческий на престоле славы Своей, сядете и вы на двенадцати престолах судить двенадцать колен Израилевых.
\par 29 И всякий, кто оставит домы, или братьев, или сестер, или отца, или мать, или жену, или детей, или земли, ради имени Моего, получит во сто крат и наследует жизнь вечную.
\par 30 Многие же будут первые последними, и последние первыми.

\chapter{20}

\par 1 Ибо Царство Небесное подобно хозяину дома, который вышел рано поутру нанять работников в виноградник свой
\par 2 и, договорившись с работниками по динарию на день, послал их в виноградник свой;
\par 3 выйдя около третьего часа, он увидел других, стоящих на торжище праздно,
\par 4 и им сказал: идите и вы в виноградник мой, и что следовать будет, дам вам. Они пошли.
\par 5 Опять выйдя около шестого и девятого часа, сделал то же.
\par 6 Наконец, выйдя около одиннадцатого часа, он нашел других, стоящих праздно, и говорит им: что вы стоите здесь целый день праздно?
\par 7 Они говорят ему: никто нас не нанял. Он говорит им: идите и вы в виноградник мой, и что следовать будет, получите.
\par 8 Когда же наступил вечер, говорит господин виноградника управителю своему: позови работников и отдай им плату, начав с последних до первых.
\par 9 И пришедшие около одиннадцатого часа получили по динарию.
\par 10 Пришедшие же первыми думали, что они получат больше, но получили и они по динарию;
\par 11 и, получив, стали роптать на хозяина дома
\par 12 и говорили: эти последние работали один час, и ты сравнял их с нами, перенесшими тягость дня и зной.
\par 13 Он же в ответ сказал одному из них: друг! я не обижаю тебя; не за динарий ли ты договорился со мною?
\par 14 возьми свое и пойди; я же хочу дать этому последнему [то же], что и тебе;
\par 15 разве я не властен в своем делать, что хочу? или глаз твой завистлив оттого, что я добр?
\par 16 Так будут последние первыми, и первые последними, ибо много званых, а мало избранных.
\par 17 И, восходя в Иерусалим, Иисус дорогою отозвал двенадцать учеников одних, и сказал им:
\par 18 вот, мы восходим в Иерусалим, и Сын Человеческий предан будет первосвященникам и книжникам, и осудят Его на смерть;
\par 19 и предадут Его язычникам на поругание и биение и распятие; и в третий день воскреснет.
\par 20 Тогда приступила к Нему мать сыновей Зеведеевых с сыновьями своими, кланяясь и чего-то прося у Него.
\par 21 Он сказал ей: чего ты хочешь? Она говорит Ему: скажи, чтобы сии два сына мои сели у Тебя один по правую сторону, а другой по левую в Царстве Твоем.
\par 22 Иисус сказал в ответ: не знаете, чего просите. Можете ли пить чашу, которую Я буду пить, или креститься крещением, которым Я крещусь? Они говорят Ему: можем.
\par 23 И говорит им: чашу Мою будете пить, и крещением, которым Я крещусь, будете креститься, но дать сесть у Меня по правую сторону и по левую--не от Меня [зависит], но кому уготовано Отцем Моим.
\par 24 Услышав [сие, прочие] десять [учеников] вознегодовали на двух братьев.
\par 25 Иисус же, подозвав их, сказал: вы знаете, что князья народов господствуют над ними, и вельможи властвуют ими;
\par 26 но между вами да не будет так: а кто хочет между вами быть большим, да будет вам слугою;
\par 27 и кто хочет между вами быть первым, да будет вам рабом;
\par 28 так как Сын Человеческий не [для того] пришел, чтобы Ему служили, но чтобы послужить и отдать душу Свою для искупления многих.
\par 29 И когда выходили они из Иерихона, за Ним следовало множество народа.
\par 30 И вот, двое слепых, сидевшие у дороги, услышав, что Иисус идет мимо, начали кричать: помилуй нас, Господи, Сын Давидов!
\par 31 Народ же заставлял их молчать; но они еще громче стали кричать: помилуй нас, Господи, Сын Давидов!
\par 32 Иисус, остановившись, подозвал их и сказал: чего вы хотите от Меня?
\par 33 Они говорят Ему: Господи! чтобы открылись глаза наши.
\par 34 Иисус же, умилосердившись, прикоснулся к глазам их; и тотчас прозрели глаза их, и они пошли за Ним.

\chapter{21}

\par 1 И когда приблизились к Иерусалиму и пришли в Виффагию к горе Елеонской, тогда Иисус послал двух учеников,
\par 2 сказав им: пойдите в селение, которое прямо перед вами; и тотчас найдете ослицу привязанную и молодого осла с нею; отвязав, приведите ко Мне;
\par 3 и если кто скажет вам что-нибудь, отвечайте, что они надобны Господу; и тотчас пошлет их.
\par 4 Все же сие было, да сбудется реченное через пророка, который говорит:
\par 5 Скажите дщери Сионовой: се, Царь твой грядет к тебе кроткий, сидя на ослице и молодом осле, сыне подъяремной.
\par 6 Ученики пошли и поступили так, как повелел им Иисус:
\par 7 привели ослицу и молодого осла и положили на них одежды свои, и Он сел поверх их.
\par 8 Множество же народа постилали свои одежды по дороге, а другие резали ветви с дерев и постилали по дороге;
\par 9 народ же, предшествовавший и сопровождавший, восклицал: осанна Сыну Давидову! благословен Грядущий во имя Господне! осанна в вышних!
\par 10 И когда вошел Он в Иерусалим, весь город пришел в движение и говорил: кто Сей?
\par 11 Народ же говорил: Сей есть Иисус, Пророк из Назарета Галилейского.
\par 12 И вошел Иисус в храм Божий и выгнал всех продающих и покупающих в храме, и опрокинул столы меновщиков и скамьи продающих голубей,
\par 13 и говорил им: написано, --дом Мой домом молитвы наречется; а вы сделали его вертепом разбойников.
\par 14 И приступили к Нему в храме слепые и хромые, и Он исцелил их.
\par 15 Видев же первосвященники и книжники чудеса, которые Он сотворил, и детей, восклицающих в храме и говорящих: осанна Сыну Давидову! --вознегодовали
\par 16 и сказали Ему: слышишь ли, что они говорят? Иисус же говорит им: да! разве вы никогда не читали: из уст младенцев и грудных детей Ты устроил хвалу?
\par 17 И, оставив их, вышел вон из города в Вифанию и провел там ночь.
\par 18 Поутру же, возвращаясь в город, взалкал;
\par 19 и увидев при дороге одну смоковницу, подошел к ней и, ничего не найдя на ней, кроме одних листьев, говорит ей: да не будет же впредь от тебя плода вовек. И смоковница тотчас засохла.
\par 20 Увидев это, ученики удивились и говорили: как это тотчас засохла смоковница?
\par 21 Иисус же сказал им в ответ: истинно говорю вам, если будете иметь веру и не усомнитесь, не только сделаете то, что [сделано] со смоковницею, но если и горе сей скажете: поднимись и ввергнись в море, --будет;
\par 22 и все, чего ни попросите в молитве с верою, получите.
\par 23 И когда пришел Он в храм и учил, приступили к Нему первосвященники и старейшины народа и сказали: какой властью Ты это делаешь? и кто Тебе дал такую власть?
\par 24 Иисус сказал им в ответ: спрошу и Я вас об одном; если о том скажете Мне, то и Я вам скажу, какою властью это делаю;
\par 25 крещение Иоанново откуда было: с небес, или от человеков? Они же рассуждали между собою: если скажем: с небес, то Он скажет нам: почему же вы не поверили ему?
\par 26 а если сказать: от человеков, --боимся народа, ибо все почитают Иоанна за пророка.
\par 27 И сказали в ответ Иисусу: не знаем. Сказал им и Он: и Я вам не скажу, какою властью это делаю.
\par 28 А как вам кажется? У одного человека было два сына; и он, подойдя к первому, сказал: сын! пойди сегодня работай в винограднике моем.
\par 29 Но он сказал в ответ: не хочу; а после, раскаявшись, пошел.
\par 30 И подойдя к другому, он сказал то же. Этот сказал в ответ: иду, государь, и не пошел.
\par 31 Который из двух исполнил волю отца? Говорят Ему: первый. Иисус говорит им: истинно говорю вам, что мытари и блудницы вперед вас идут в Царство Божие,
\par 32 ибо пришел к вам Иоанн путем праведности, и вы не поверили ему, а мытари и блудницы поверили ему; вы же, и видев это, не раскаялись после, чтобы поверить ему.
\par 33 Выслушайте другую притчу: был некоторый хозяин дома, который насадил виноградник, обнес его оградою, выкопал в нем точило, построил башню и, отдав его виноградарям, отлучился.
\par 34 Когда же приблизилось время плодов, он послал своих слуг к виноградарям взять свои плоды;
\par 35 виноградари, схватив слуг его, иного прибили, иного убили, а иного побили камнями.
\par 36 Опять послал он других слуг, больше прежнего; и с ними поступили так же.
\par 37 Наконец, послал он к ним своего сына, говоря: постыдятся сына моего.
\par 38 Но виноградари, увидев сына, сказали друг другу: это наследник; пойдем, убьем его и завладеем наследством его.
\par 39 И, схватив его, вывели вон из виноградника и убили.
\par 40 Итак, когда придет хозяин виноградника, что сделает он с этими виноградарями?
\par 41 Говорят Ему: злодеев сих предаст злой смерти, а виноградник отдаст другим виноградарям, которые будут отдавать ему плоды во времена свои.
\par 42 Иисус говорит им: неужели вы никогда не читали в Писании: камень, который отвергли строители, тот самый сделался главою угла? Это от Господа, и есть дивно в очах наших?
\par 43 Потому сказываю вам, что отнимется от вас Царство Божие и дано будет народу, приносящему плоды его;
\par 44 и тот, кто упадет на этот камень, разобьется, а на кого он упадет, того раздавит.
\par 45 И слышав притчи Его, первосвященники и фарисеи поняли, что Он о них говорит,
\par 46 и старались схватить Его, но побоялись народа, потому что Его почитали за Пророка.

\chapter{22}

\par 1 Иисус, продолжая говорить им притчами, сказал:
\par 2 Царство Небесное подобно человеку царю, который сделал брачный пир для сына своего
\par 3 и послал рабов своих звать званых на брачный пир; и не хотели придти.
\par 4 Опять послал других рабов, сказав: скажите званым: вот, я приготовил обед мой, тельцы мои и что откормлено, заколото, и все готово; приходите на брачный пир.
\par 5 Но они, пренебрегши то, пошли, кто на поле свое, а кто на торговлю свою;
\par 6 прочие же, схватив рабов его, оскорбили и убили [их].
\par 7 Услышав о сем, царь разгневался, и, послав войска свои, истребил убийц оных и сжег город их.
\par 8 Тогда говорит он рабам своим: брачный пир готов, а званые не были достойны;
\par 9 итак пойдите на распутия и всех, кого найдете, зовите на брачный пир.
\par 10 И рабы те, выйдя на дороги, собрали всех, кого только нашли, и злых и добрых; и брачный пир наполнился возлежащими.
\par 11 Царь, войдя посмотреть возлежащих, увидел там человека, одетого не в брачную одежду,
\par 12 и говорит ему: друг! как ты вошел сюда не в брачной одежде? Он же молчал.
\par 13 Тогда сказал царь слугам: связав ему руки и ноги, возьмите его и бросьте во тьму внешнюю; там будет плач и скрежет зубов;
\par 14 ибо много званых, а мало избранных.
\par 15 Тогда фарисеи пошли и совещались, как бы уловить Его в словах.
\par 16 И посылают к Нему учеников своих с иродианами, говоря: Учитель! мы знаем, что Ты справедлив, и истинно пути Божию учишь, и не заботишься об угождении кому-либо, ибо не смотришь ни на какое лице;
\par 17 итак скажи нам: как Тебе кажется? позволительно ли давать подать кесарю, или нет?
\par 18 Но Иисус, видя лукавство их, сказал: что искушаете Меня, лицемеры?
\par 19 покажите Мне монету, которою платится подать. Они принесли Ему динарий.
\par 20 И говорит им: чье это изображение и надпись?
\par 21 Говорят Ему: кесаревы. Тогда говорит им: итак отдавайте кесарево кесарю, а Божие Богу.
\par 22 Услышав это, они удивились и, оставив Его, ушли.
\par 23 В тот день приступили к Нему саддукеи, которые говорят, что нет воскресения, и спросили Его:
\par 24 Учитель! Моисей сказал: если кто умрет, не имея детей, то брат его пусть возьмет за себя жену его и восстановит семя брату своему;
\par 25 было у нас семь братьев; первый, женившись, умер и, не имея детей, оставил жену свою брату своему;
\par 26 подобно и второй, и третий, даже до седьмого;
\par 27 после же всех умерла и жена;
\par 28 итак, в воскресении, которого из семи будет она женою? ибо все имели ее.
\par 29 Иисус сказал им в ответ: заблуждаетесь, не зная Писаний, ни силы Божией,
\par 30 ибо в воскресении ни женятся, ни выходят замуж, но пребывают, как Ангелы Божии на небесах.
\par 31 А о воскресении мертвых не читали ли вы реченного вам Богом:
\par 32 Я Бог Авраама, и Бог Исаака, и Бог Иакова? Бог не есть Бог мертвых, но живых.
\par 33 И, слыша, народ дивился учению Его.
\par 34 А фарисеи, услышав, что Он привел саддукеев в молчание, собрались вместе.
\par 35 И один из них, законник, искушая Его, спросил, говоря:
\par 36 Учитель! какая наибольшая заповедь в законе?
\par 37 Иисус сказал ему: возлюби Господа Бога твоего всем сердцем твоим и всею душею твоею и всем разумением твоим:
\par 38 сия есть первая и наибольшая заповедь;
\par 39 вторая же подобная ей: возлюби ближнего твоего, как самого себя;
\par 40 на сих двух заповедях утверждается весь закон и пророки.
\par 41 Когда же собрались фарисеи, Иисус спросил их:
\par 42 что вы думаете о Христе? чей Он сын? Говорят Ему: Давидов.
\par 43 Говорит им: как же Давид, по вдохновению, называет Его Господом, когда говорит:
\par 44 сказал Господь Господу моему: седи одесную Меня, доколе положу врагов Твоих в подножие ног Твоих?
\par 45 Итак, если Давид называет Его Господом, как же Он сын ему?
\par 46 И никто не мог отвечать Ему ни слова; и с того дня никто уже не смел спрашивать Его.

\chapter{23}

\par 1 Тогда Иисус начал говорить народу и ученикам Своим
\par 2 и сказал: на Моисеевом седалище сели книжники и фарисеи;
\par 3 итак все, что они велят вам соблюдать, соблюдайте и делайте; по делам же их не поступайте, ибо они говорят, и не делают:
\par 4 связывают бремена тяжелые и неудобоносимые и возлагают на плечи людям, а сами не хотят и перстом двинуть их;
\par 5 все же дела свои делают с тем, чтобы видели их люди: расширяют хранилища свои и увеличивают воскрилия одежд своих;
\par 6 также любят предвозлежания на пиршествах и председания в синагогах
\par 7 и приветствия в народных собраниях, и чтобы люди звали их: учитель! учитель!
\par 8 А вы не называйтесь учителями, ибо один у вас Учитель--Христос, все же вы--братья;
\par 9 и отцом себе не называйте никого на земле, ибо один у вас Отец, Который на небесах;
\par 10 и не называйтесь наставниками, ибо один у вас Наставник--Христос.
\par 11 Больший из вас да будет вам слуга:
\par 12 ибо, кто возвышает себя, тот унижен будет, а кто унижает себя, тот возвысится.
\par 13 Горе вам, книжники и фарисеи, лицемеры, что затворяете Царство Небесное человекам, ибо сами не входите и хотящих войти не допускаете.
\par 14 Горе вам, книжники и фарисеи, лицемеры, что поедаете домы вдов и лицемерно долго молитесь: за то примете тем большее осуждение.
\par 15 Горе вам, книжники и фарисеи, лицемеры, что обходите море и сушу, дабы обратить хотя одного; и когда это случится, делаете его сыном геенны, вдвое худшим вас.
\par 16 Горе вам, вожди слепые, которые говорите: если кто поклянется храмом, то ничего, а если кто поклянется золотом храма, то повинен.
\par 17 Безумные и слепые! что больше: золото, или храм, освящающий золото?
\par 18 Также: если кто поклянется жертвенником, то ничего, если же кто поклянется даром, который на нем, то повинен.
\par 19 Безумные и слепые! что больше: дар, или жертвенник, освящающий дар?
\par 20 Итак клянущийся жертвенником клянется им и всем, что на нем;
\par 21 и клянущийся храмом клянется им и Живущим в нем;
\par 22 и клянущийся небом клянется Престолом Божиим и Сидящим на нем.
\par 23 Горе вам, книжники и фарисеи, лицемеры, что даете десятину с мяты, аниса и тмина, и оставили важнейшее в законе: суд, милость и веру; сие надлежало делать, и того не оставлять.
\par 24 Вожди слепые, оцеживающие комара, а верблюда поглощающие!
\par 25 Горе вам, книжники и фарисеи, лицемеры, что очищаете внешность чаши и блюда, между тем как внутри они полны хищения и неправды.
\par 26 Фарисей слепой! очисти прежде внутренность чаши и блюда, чтобы чиста была и внешность их.
\par 27 Горе вам, книжники и фарисеи, лицемеры, что уподобляетесь окрашенным гробам, которые снаружи кажутся красивыми, а внутри полны костей мертвых и всякой нечистоты;
\par 28 так и вы по наружности кажетесь людям праведными, а внутри исполнены лицемерия и беззакония.
\par 29 Горе вам, книжники и фарисеи, лицемеры, что строите гробницы пророкам и украшаете памятники праведников,
\par 30 и говорите: если бы мы были во дни отцов наших, то не были бы сообщниками их в [пролитии] крови пророков;
\par 31 таким образом вы сами против себя свидетельствуете, что вы сыновья тех, которые избили пророков;
\par 32 дополняйте же меру отцов ваших.
\par 33 Змии, порождения ехиднины! как убежите вы от осуждения в геенну?
\par 34 Посему, вот, Я посылаю к вам пророков, и мудрых, и книжников; и вы иных убьете и распнете, а иных будете бить в синагогах ваших и гнать из города в город;
\par 35 да придет на вас вся кровь праведная, пролитая на земле, от крови Авеля праведного до крови Захарии, сына Варахиина, которого вы убили между храмом и жертвенником.
\par 36 Истинно говорю вам, что все сие придет на род сей.
\par 37 Иерусалим, Иерусалим, избивающий пророков и камнями побивающий посланных к тебе! сколько раз хотел Я собрать детей твоих, как птица собирает птенцов своих под крылья, и вы не захотели!
\par 38 Се, оставляется вам дом ваш пуст.
\par 39 Ибо сказываю вам: не увидите Меня отныне, доколе не воскликнете: благословен Грядый во имя Господне!

\chapter{24}

\par 1 И выйдя, Иисус шел от храма; и приступили ученики Его, чтобы показать Ему здания храма.
\par 2 Иисус же сказал им: видите ли все это? Истинно говорю вам: не останется здесь камня на камне; все будет разрушено.
\par 3 Когда же сидел Он на горе Елеонской, то приступили к Нему ученики наедине и спросили: скажи нам, когда это будет? и какой признак Твоего пришествия и кончины века?
\par 4 Иисус сказал им в ответ: берегитесь, чтобы кто не прельстил вас,
\par 5 ибо многие придут под именем Моим, и будут говорить: `Я Христос', и многих прельстят.
\par 6 Также услышите о войнах и о военных слухах. Смотрите, не ужасайтесь, ибо надлежит всему тому быть, но это еще не конец:
\par 7 ибо восстанет народ на народ, и царство на царство; и будут глады, моры и землетрясения по местам;
\par 8 все же это--начало болезней.
\par 9 Тогда будут предавать вас на мучения и убивать вас; и вы будете ненавидимы всеми народами за имя Мое;
\par 10 и тогда соблазнятся многие, и друг друга будут предавать, и возненавидят друг друга;
\par 11 и многие лжепророки восстанут, и прельстят многих;
\par 12 и, по причине умножения беззакония, во многих охладеет любовь;
\par 13 претерпевший же до конца спасется.
\par 14 И проповедано будет сие Евангелие Царствия по всей вселенной, во свидетельство всем народам; и тогда придет конец.
\par 15 Итак, когда увидите мерзость запустения, реченную через пророка Даниила, стоящую на святом месте, --читающий да разумеет, --
\par 16 тогда находящиеся в Иудее да бегут в горы;
\par 17 и кто на кровле, тот да не сходит взять что-нибудь из дома своего;
\par 18 и кто на поле, тот да не обращается назад взять одежды свои.
\par 19 Горе же беременным и питающим сосцами в те дни!
\par 20 Молитесь, чтобы не случилось бегство ваше зимою или в субботу,
\par 21 ибо тогда будет великая скорбь, какой не было от начала мира доныне, и не будет.
\par 22 И если бы не сократились те дни, то не спаслась бы никакая плоть; но ради избранных сократятся те дни.
\par 23 Тогда, если кто скажет вам: вот, здесь Христос, или там, --не верьте.
\par 24 Ибо восстанут лжехристы и лжепророки, и дадут великие знамения и чудеса, чтобы прельстить, если возможно, и избранных.
\par 25 Вот, Я наперед сказал вам.
\par 26 Итак, если скажут вам: `вот, [Он] в пустыне', --не выходите; `вот, [Он] в потаенных комнатах', --не верьте;
\par 27 ибо, как молния исходит от востока и видна бывает даже до запада, так будет пришествие Сына Человеческого;
\par 28 ибо, где будет труп, там соберутся орлы.
\par 29 И вдруг, после скорби дней тех, солнце померкнет, и луна не даст света своего, и звезды спадут с неба, и силы небесные поколеблются;
\par 30 тогда явится знамение Сына Человеческого на небе; и тогда восплачутся все племена земные и увидят Сына Человеческого, грядущего на облаках небесных с силою и славою великою;
\par 31 и пошлет Ангелов Своих с трубою громогласною, и соберут избранных Его от четырех ветров, от края небес до края их.
\par 32 От смоковницы возьмите подобие: когда ветви ее становятся уже мягки и пускают листья, то знаете, что близко лето;
\par 33 так, когда вы увидите все сие, знайте, что близко, при дверях.
\par 34 Истинно говорю вам: не прейдет род сей, как все сие будет;
\par 35 небо и земля прейдут, но слова Мои не прейдут.
\par 36 О дне же том и часе никто не знает, ни Ангелы небесные, а только Отец Мой один;
\par 37 но, как было во дни Ноя, так будет и в пришествие Сына Человеческого:
\par 38 ибо, как во дни перед потопом ели, пили, женились и выходили замуж, до того дня, как вошел Ной в ковчег,
\par 39 и не думали, пока не пришел потоп и не истребил всех, --так будет и пришествие Сына Человеческого;
\par 40 тогда будут двое на поле: один берется, а другой оставляется;
\par 41 две мелющие в жерновах: одна берется, а другая оставляется.
\par 42 Итак бодрствуйте, потому что не знаете, в который час Господь ваш приидет.
\par 43 Но это вы знаете, что, если бы ведал хозяин дома, в какую стражу придет вор, то бодрствовал бы и не дал бы подкопать дома своего.
\par 44 Потому и вы будьте готовы, ибо в который час не думаете, приидет Сын Человеческий.
\par 45 Кто же верный и благоразумный раб, которого господин его поставил над слугами своими, чтобы давать им пищу во время?
\par 46 Блажен тот раб, которого господин его, придя, найдет поступающим так;
\par 47 истинно говорю вам, что над всем имением своим поставит его.
\par 48 Если же раб тот, будучи зол, скажет в сердце своем: не скоро придет господин мой,
\par 49 и начнет бить товарищей своих и есть и пить с пьяницами, --
\par 50 то придет господин раба того в день, в который он не ожидает, и в час, в который не думает,
\par 51 и рассечет его, и подвергнет его одной участи с лицемерами; там будет плач и скрежет зубов.

\chapter{25}

\par 1 Тогда подобно будет Царство Небесное десяти девам, которые, взяв светильники свои, вышли навстречу жениху.
\par 2 Из них пять было мудрых и пять неразумных.
\par 3 Неразумные, взяв светильники свои, не взяли с собою масла.
\par 4 Мудрые же, вместе со светильниками своими, взяли масла в сосудах своих.
\par 5 И как жених замедлил, то задремали все и уснули.
\par 6 Но в полночь раздался крик: вот, жених идет, выходите навстречу ему.
\par 7 Тогда встали все девы те и поправили светильники свои.
\par 8 Неразумные же сказали мудрым: дайте нам вашего масла, потому что светильники наши гаснут.
\par 9 А мудрые отвечали: чтобы не случилось недостатка и у нас и у вас, пойдите лучше к продающим и купите себе.
\par 10 Когда же пошли они покупать, пришел жених, и готовые вошли с ним на брачный пир, и двери затворились;
\par 11 после приходят и прочие девы, и говорят: Господи! Господи! отвори нам.
\par 12 Он же сказал им в ответ: истинно говорю вам: не знаю вас.
\par 13 Итак, бодрствуйте, потому что не знаете ни дня, ни часа, в который приидет Сын Человеческий.
\par 14 Ибо [Он поступит], как человек, который, отправляясь в чужую страну, призвал рабов своих и поручил им имение свое:
\par 15 и одному дал он пять талантов, другому два, иному один, каждому по его силе; и тотчас отправился.
\par 16 Получивший пять талантов пошел, употребил их в дело и приобрел другие пять талантов;
\par 17 точно так же и получивший два таланта приобрел другие два;
\par 18 получивший же один талант пошел и закопал [его] в землю и скрыл серебро господина своего.
\par 19 По долгом времени, приходит господин рабов тех и требует у них отчета.
\par 20 И, подойдя, получивший пять талантов принес другие пять талантов и говорит: господин! пять талантов ты дал мне; вот, другие пять талантов я приобрел на них.
\par 21 Господин его сказал ему: хорошо, добрый и верный раб! в малом ты был верен, над многим тебя поставлю; войди в радость господина твоего.
\par 22 Подошел также и получивший два таланта и сказал: господин! два таланта ты дал мне; вот, другие два таланта я приобрел на них.
\par 23 Господин его сказал ему: хорошо, добрый и верный раб! в малом ты был верен, над многим тебя поставлю; войди в радость господина твоего.
\par 24 Подошел и получивший один талант и сказал: господин! я знал тебя, что ты человек жестокий, жнешь, где не сеял, и собираешь, где не рассыпал,
\par 25 и, убоявшись, пошел и скрыл талант твой в земле; вот тебе твое.
\par 26 Господин же его сказал ему в ответ: лукавый раб и ленивый! ты знал, что я жну, где не сеял, и собираю, где не рассыпал;
\par 27 посему надлежало тебе отдать серебро мое торгующим, и я, придя, получил бы мое с прибылью;
\par 28 итак, возьмите у него талант и дайте имеющему десять талантов,
\par 29 ибо всякому имеющему дастся и приумножится, а у неимеющего отнимется и то, что имеет;
\par 30 а негодного раба выбросьте во тьму внешнюю: там будет плач и скрежет зубов. Сказав сие, возгласил: кто имеет уши слышать, да слышит!
\par 31 Когда же приидет Сын Человеческий во славе Своей и все святые Ангелы с Ним, тогда сядет на престоле славы Своей,
\par 32 и соберутся пред Ним все народы; и отделит одних от других, как пастырь отделяет овец от козлов;
\par 33 и поставит овец по правую Свою сторону, а козлов--по левую.
\par 34 Тогда скажет Царь тем, которые по правую сторону Его: приидите, благословенные Отца Моего, наследуйте Царство, уготованное вам от создания мира:
\par 35 ибо алкал Я, и вы дали Мне есть; жаждал, и вы напоили Меня; был странником, и вы приняли Меня;
\par 36 был наг, и вы одели Меня; был болен, и вы посетили Меня; в темнице был, и вы пришли ко Мне.
\par 37 Тогда праведники скажут Ему в ответ: Господи! когда мы видели Тебя алчущим, и накормили? или жаждущим, и напоили?
\par 38 когда мы видели Тебя странником, и приняли? или нагим, и одели?
\par 39 когда мы видели Тебя больным, или в темнице, и пришли к Тебе?
\par 40 И Царь скажет им в ответ: истинно говорю вам: так как вы сделали это одному из сих братьев Моих меньших, то сделали Мне.
\par 41 Тогда скажет и тем, которые по левую сторону: идите от Меня, проклятые, в огонь вечный, уготованный диаволу и ангелам его:
\par 42 ибо алкал Я, и вы не дали Мне есть; жаждал, и вы не напоили Меня;
\par 43 был странником, и не приняли Меня; был наг, и не одели Меня; болен и в темнице, и не посетили Меня.
\par 44 Тогда и они скажут Ему в ответ: Господи! когда мы видели Тебя алчущим, или жаждущим, или странником, или нагим, или больным, или в темнице, и не послужили Тебе?
\par 45 Тогда скажет им в ответ: истинно говорю вам: так как вы не сделали этого одному из сих меньших, то не сделали Мне.
\par 46 И пойдут сии в муку вечную, а праведники в жизнь вечную.

\chapter{26}

\par 1 Когда Иисус окончил все слова сии, то сказал ученикам Своим:
\par 2 вы знаете, что через два дня будет Пасха, и Сын Человеческий предан будет на распятие.
\par 3 Тогда собрались первосвященники и книжники и старейшины народа во двор первосвященника, по имени Каиафы,
\par 4 и положили в совете взять Иисуса хитростью и убить;
\par 5 но говорили: только не в праздник, чтобы не сделалось возмущения в народе.
\par 6 Когда же Иисус был в Вифании, в доме Симона прокаженного,
\par 7 приступила к Нему женщина с алавастровым сосудом мира драгоценного и возливала Ему возлежащему на голову.
\par 8 Увидев это, ученики Его вознегодовали и говорили: к чему такая трата?
\par 9 Ибо можно было бы продать это миро за большую цену и дать нищим.
\par 10 Но Иисус, уразумев сие, сказал им: что смущаете женщину? она доброе дело сделала для Меня:
\par 11 ибо нищих всегда имеете с собою, а Меня не всегда имеете;
\par 12 возлив миро сие на тело Мое, она приготовила Меня к погребению;
\par 13 истинно говорю вам: где ни будет проповедано Евангелие сие в целом мире, сказано будет в память ее и о том, что она сделала.
\par 14 Тогда один из двенадцати, называемый Иуда Искариот, пошел к первосвященникам
\par 15 и сказал: что вы дадите мне, и я вам предам Его? Они предложили ему тридцать сребренников;
\par 16 и с того времени он искал удобного случая предать Его.
\par 17 В первый же день опресночный приступили ученики к Иисусу и сказали Ему: где велишь нам приготовить Тебе пасху?
\par 18 Он сказал: пойдите в город к такому-то и скажите ему: Учитель говорит: время Мое близко; у тебя совершу пасху с учениками Моими.
\par 19 Ученики сделали, как повелел им Иисус, и приготовили пасху.
\par 20 Когда же настал вечер, Он возлег с двенадцатью учениками;
\par 21 и когда они ели, сказал: истинно говорю вам, что один из вас предаст Меня.
\par 22 Они весьма опечалились, и начали говорить Ему, каждый из них: не я ли, Господи?
\par 23 Он же сказал в ответ: опустивший со Мною руку в блюдо, этот предаст Меня;
\par 24 впрочем Сын Человеческий идет, как писано о Нем, но горе тому человеку, которым Сын Человеческий предается: лучше было бы этому человеку не родиться.
\par 25 При сем и Иуда, предающий Его, сказал: не я ли, Равви? [Иисус] говорит ему: ты сказал.
\par 26 И когда они ели, Иисус взял хлеб и, благословив, преломил и, раздавая ученикам, сказал: приимите, ядите: сие есть Тело Мое.
\par 27 И, взяв чашу и благодарив, подал им и сказал: пейте из нее все,
\par 28 ибо сие есть Кровь Моя Нового Завета, за многих изливаемая во оставление грехов.
\par 29 Сказываю же вам, что отныне не буду пить от плода сего виноградного до того дня, когда буду пить с вами новое [вино] в Царстве Отца Моего.
\par 30 И, воспев, пошли на гору Елеонскую.
\par 31 Тогда говорит им Иисус: все вы соблазнитесь о Мне в эту ночь, ибо написано: поражу пастыря, и рассеются овцы стада;
\par 32 по воскресении же Моем предварю вас в Галилее.
\par 33 Петр сказал Ему в ответ: если и все соблазнятся о Тебе, я никогда не соблазнюсь.
\par 34 Иисус сказал ему: истинно говорю тебе, что в эту ночь, прежде нежели пропоет петух, трижды отречешься от Меня.
\par 35 Говорит Ему Петр: хотя бы надлежало мне и умереть с Тобою, не отрекусь от Тебя. Подобное говорили и все ученики.
\par 36 Потом приходит с ними Иисус на место, называемое Гефсимания, и говорит ученикам: посидите тут, пока Я пойду, помолюсь там.
\par 37 И, взяв с Собою Петра и обоих сыновей Зеведеевых, начал скорбеть и тосковать.
\par 38 Тогда говорит им Иисус: душа Моя скорбит смертельно; побудьте здесь и бодрствуйте со Мною.
\par 39 И, отойдя немного, пал на лице Свое, молился и говорил: Отче Мой! если возможно, да минует Меня чаша сия; впрочем не как Я хочу, но как Ты.
\par 40 И приходит к ученикам и находит их спящими, и говорит Петру: так ли не могли вы один час бодрствовать со Мною?
\par 41 бодрствуйте и молитесь, чтобы не впасть в искушение: дух бодр, плоть же немощна.
\par 42 Еще, отойдя в другой раз, молился, говоря: Отче Мой! если не может чаша сия миновать Меня, чтобы Мне не пить ее, да будет воля Твоя.
\par 43 И, придя, находит их опять спящими, ибо у них глаза отяжелели.
\par 44 И, оставив их, отошел опять и помолился в третий раз, сказав то же слово.
\par 45 Тогда приходит к ученикам Своим и говорит им: вы все еще спите и почиваете? вот, приблизился час, и Сын Человеческий предается в руки грешников;
\par 46 встаньте, пойдем: вот, приблизился предающий Меня.
\par 47 И, когда еще говорил Он, вот Иуда, один из двенадцати, пришел, и с ним множество народа с мечами и кольями, от первосвященников и старейшин народных.
\par 48 Предающий же Его дал им знак, сказав: Кого я поцелую, Тот и есть, возьмите Его.
\par 49 И, тотчас подойдя к Иисусу, сказал: радуйся, Равви! И поцеловал Его.
\par 50 Иисус же сказал ему: друг, для чего ты пришел? Тогда подошли и возложили руки на Иисуса, и взяли Его.
\par 51 И вот, один из бывших с Иисусом, простерши руку, извлек меч свой и, ударив раба первосвященникова, отсек ему ухо.
\par 52 Тогда говорит ему Иисус: возврати меч твой в его место, ибо все, взявшие меч, мечом погибнут;
\par 53 или думаешь, что Я не могу теперь умолить Отца Моего, и Он представит Мне более, нежели двенадцать легионов Ангелов?
\par 54 как же сбудутся Писания, что так должно быть?
\par 55 В тот час сказал Иисус народу: как будто на разбойника вышли вы с мечами и кольями взять Меня; каждый день с вами сидел Я, уча в храме, и вы не брали Меня.
\par 56 Сие же все было, да сбудутся писания пророков. Тогда все ученики, оставив Его, бежали.
\par 57 А взявшие Иисуса отвели Его к Каиафе первосвященнику, куда собрались книжники и старейшины.
\par 58 Петр же следовал за Ним издали, до двора первосвященникова; и, войдя внутрь, сел со служителями, чтобы видеть конец.
\par 59 Первосвященники и старейшины и весь синедрион искали лжесвидетельства против Иисуса, чтобы предать Его смерти,
\par 60 и не находили; и, хотя много лжесвидетелей приходило, не нашли. Но наконец пришли два лжесвидетеля
\par 61 и сказали: Он говорил: могу разрушить храм Божий и в три дня создать его.
\par 62 И, встав, первосвященник сказал Ему: [что же] ничего не отвечаешь? что они против Тебя свидетельствуют?
\par 63 Иисус молчал. И первосвященник сказал Ему: заклинаю Тебя Богом живым, скажи нам, Ты ли Христос, Сын Божий?
\par 64 Иисус говорит ему: ты сказал; даже сказываю вам: отныне узрите Сына Человеческого, сидящего одесную силы и грядущего на облаках небесных.
\par 65 Тогда первосвященник разодрал одежды свои и сказал: Он богохульствует! на что еще нам свидетелей? вот, теперь вы слышали богохульство Его!
\par 66 как вам кажется? Они же сказали в ответ: повинен смерти.
\par 67 Тогда плевали Ему в лице и заушали Его; другие же ударяли Его по ланитам
\par 68 и говорили: прореки нам, Христос, кто ударил Тебя?
\par 69 Петр же сидел вне на дворе. И подошла к нему одна служанка и сказала: и ты был с Иисусом Галилеянином.
\par 70 Но он отрекся перед всеми, сказав: не знаю, что ты говоришь.
\par 71 Когда же он выходил за ворота, увидела его другая, и говорит бывшим там: и этот был с Иисусом Назореем.
\par 72 И он опять отрекся с клятвою, что не знает Сего Человека.
\par 73 Немного спустя подошли стоявшие там и сказали Петру: точно и ты из них, ибо и речь твоя обличает тебя.
\par 74 Тогда он начал клясться и божиться, что не знает Сего Человека. И вдруг запел петух.
\par 75 И вспомнил Петр слово, сказанное ему Иисусом: прежде нежели пропоет петух, трижды отречешься от Меня. И выйдя вон, плакал горько.

\chapter{27}

\par 1 Когда же настало утро, все первосвященники и старейшины народа имели совещание об Иисусе, чтобы предать Его смерти;
\par 2 и, связав Его, отвели и предали Его Понтию Пилату, правителю.
\par 3 Тогда Иуда, предавший Его, увидев, что Он осужден, и, раскаявшись, возвратил тридцать сребренников первосвященникам и старейшинам,
\par 4 говоря: согрешил я, предав кровь невинную. Они же сказали ему: что нам до того? смотри сам.
\par 5 И, бросив сребренники в храме, он вышел, пошел и удавился.
\par 6 Первосвященники, взяв сребренники, сказали: непозволительно положить их в сокровищницу церковную, потому что это цена крови.
\par 7 Сделав же совещание, купили на них землю горшечника, для погребения странников;
\par 8 посему и называется земля та `землею крови' до сего дня.
\par 9 Тогда сбылось реченное через пророка Иеремию, который говорит: и взяли тридцать сребренников, цену Оцененного, Которого оценили сыны Израиля,
\par 10 и дали их за землю горшечника, как сказал мне Господь.
\par 11 Иисус же стал пред правителем. И спросил Его правитель: Ты Царь Иудейский? Иисус сказал ему: ты говоришь.
\par 12 И когда обвиняли Его первосвященники и старейшины, Он ничего не отвечал.
\par 13 Тогда говорит Ему Пилат: не слышишь, сколько свидетельствуют против Тебя?
\par 14 И не отвечал ему ни на одно слово, так что правитель весьма дивился.
\par 15 На праздник же [Пасхи] правитель имел обычай отпускать народу одного узника, которого хотели.
\par 16 Был тогда у них известный узник, называемый Варавва;
\par 17 итак, когда собрались они, сказал им Пилат: кого хотите, чтобы я отпустил вам: Варавву, или Иисуса, называемого Христом?
\par 18 ибо знал, что предали Его из зависти.
\par 19 Между тем, как сидел он на судейском месте, жена его послала ему сказать: не делай ничего Праведнику Тому, потому что я ныне во сне много пострадала за Него.
\par 20 Но первосвященники и старейшины возбудили народ просить Варавву, а Иисуса погубить.
\par 21 Тогда правитель спросил их: кого из двух хотите, чтобы я отпустил вам? Они сказали: Варавву.
\par 22 Пилат говорит им: что же я сделаю Иисусу, называемому Христом? Говорят ему все: да будет распят.
\par 23 Правитель сказал: какое же зло сделал Он? Но они еще сильнее кричали: да будет распят.
\par 24 Пилат, видя, что ничто не помогает, но смятение увеличивается, взял воды и умыл руки перед народом, и сказал: невиновен я в крови Праведника Сего; смотрите вы.
\par 25 И, отвечая, весь народ сказал: кровь Его на нас и на детях наших.
\par 26 Тогда отпустил им Варавву, а Иисуса, бив, предал на распятие.
\par 27 Тогда воины правителя, взяв Иисуса в преторию, собрали на Него весь полк
\par 28 и, раздев Его, надели на Него багряницу;
\par 29 и, сплетши венец из терна, возложили Ему на голову и дали Ему в правую руку трость; и, становясь пред Ним на колени, насмехались над Ним, говоря: радуйся, Царь Иудейский!
\par 30 и плевали на Него и, взяв трость, били Его по голове.
\par 31 И когда насмеялись над Ним, сняли с Него багряницу, и одели Его в одежды Его, и повели Его на распятие.
\par 32 Выходя, они встретили одного Киринеянина, по имени Симона; сего заставили нести крест Его.
\par 33 И, придя на место, называемое Голгофа, что значит: Лобное место,
\par 34 дали Ему пить уксуса, смешанного с желчью; и, отведав, не хотел пить.
\par 35 Распявшие же Его делили одежды его, бросая жребий;
\par 36 и, сидя, стерегли Его там;
\par 37 и поставили над головою Его надпись, означающую вину Его: Сей есть Иисус, Царь Иудейский.
\par 38 Тогда распяты с Ним два разбойника: один по правую сторону, а другой по левую.
\par 39 Проходящие же злословили Его, кивая головами своими
\par 40 и говоря: Разрушающий храм и в три дня Созидающий! спаси Себя Самого; если Ты Сын Божий, сойди с креста.
\par 41 Подобно и первосвященники с книжниками и старейшинами и фарисеями, насмехаясь, говорили:
\par 42 других спасал, а Себя Самого не может спасти; если Он Царь Израилев, пусть теперь сойдет с креста, и уверуем в Него;
\par 43 уповал на Бога; пусть теперь избавит Его, если Он угоден Ему. Ибо Он сказал: Я Божий Сын.
\par 44 Также и разбойники, распятые с Ним, поносили Его.
\par 45 От шестого же часа тьма была по всей земле до часа девятого;
\par 46 а около девятого часа возопил Иисус громким голосом: Или, Или! лама савахфани? то есть: Боже Мой, Боже Мой! для чего Ты Меня оставил?
\par 47 Некоторые из стоявших там, слыша это, говорили: Илию зовет Он.
\par 48 И тотчас побежал один из них, взял губку, наполнил уксусом и, наложив на трость, давал Ему пить;
\par 49 а другие говорили: постой, посмотрим, придет ли Илия спасти Его.
\par 50 Иисус же, опять возопив громким голосом, испустил дух.
\par 51 И вот, завеса в храме раздралась надвое, сверху донизу; и земля потряслась; и камни расселись;
\par 52 и гробы отверзлись; и многие тела усопших святых воскресли
\par 53 и, выйдя из гробов по воскресении Его, вошли во святый град и явились многим.
\par 54 Сотник же и те, которые с ним стерегли Иисуса, видя землетрясение и все бывшее, устрашились весьма и говорили: воистину Он был Сын Божий.
\par 55 Там были также и смотрели издали многие женщины, которые следовали за Иисусом из Галилеи, служа Ему;
\par 56 между ними были Мария Магдалина и Мария, мать Иакова и Иосии, и мать сыновей Зеведеевых.
\par 57 Когда же настал вечер, пришел богатый человек из Аримафеи, именем Иосиф, который также учился у Иисуса;
\par 58 он, придя к Пилату, просил тела Иисусова. Тогда Пилат приказал отдать тело;
\par 59 и, взяв тело, Иосиф обвил его чистою плащаницею
\par 60 и положил его в новом своем гробе, который высек он в скале; и, привалив большой камень к двери гроба, удалился.
\par 61 Была же там Мария Магдалина и другая Мария, которые сидели против гроба.
\par 62 На другой день, который следует за пятницею, собрались первосвященники и фарисеи к Пилату
\par 63 и говорили: господин! Мы вспомнили, что обманщик тот, еще будучи в живых, сказал: после трех дней воскресну;
\par 64 итак прикажи охранять гроб до третьего дня, чтобы ученики Его, придя ночью, не украли Его и не сказали народу: воскрес из мертвых; и будет последний обман хуже первого.
\par 65 Пилат сказал им: имеете стражу; пойдите, охраняйте, как знаете.
\par 66 Они пошли и поставили у гроба стражу, и приложили к камню печать.

\chapter{28}

\par 1 По прошествии же субботы, на рассвете первого дня недели, пришла Мария Магдалина и другая Мария посмотреть гроб.
\par 2 И вот, сделалось великое землетрясение, ибо Ангел Господень, сошедший с небес, приступив, отвалил камень от двери гроба и сидел на нем;
\par 3 вид его был, как молния, и одежда его бела, как снег;
\par 4 устрашившись его, стерегущие пришли в трепет и стали, как мертвые;
\par 5 Ангел же, обратив речь к женщинам, сказал: не бойтесь, ибо знаю, что вы ищете Иисуса распятого;
\par 6 Его нет здесь--Он воскрес, как сказал. Подойдите, посмотрите место, где лежал Господь,
\par 7 и пойдите скорее, скажите ученикам Его, что Он воскрес из мертвых и предваряет вас в Галилее; там Его увидите. Вот, я сказал вам.
\par 8 И, выйдя поспешно из гроба, они со страхом и радостью великою побежали возвестить ученикам Его.
\par 9 Когда же шли они возвестить ученикам Его, и се Иисус встретил их и сказал: радуйтесь! И они, приступив, ухватились за ноги Его и поклонились Ему.
\par 10 Тогда говорит им Иисус: не бойтесь; пойдите, возвестите братьям Моим, чтобы шли в Галилею, и там они увидят Меня.
\par 11 Когда же они шли, то некоторые из стражи, войдя в город, объявили первосвященникам о всем бывшем.
\par 12 И сии, собравшись со старейшинами и сделав совещание, довольно денег дали воинам,
\par 13 и сказали: скажите, что ученики Его, придя ночью, украли Его, когда мы спали;
\par 14 и, если слух об этом дойдет до правителя, мы убедим его, и вас от неприятности избавим.
\par 15 Они, взяв деньги, поступили, как научены были; и пронеслось слово сие между иудеями до сего дня.
\par 16 Одиннадцать же учеников пошли в Галилею, на гору, куда повелел им Иисус,
\par 17 и, увидев Его, поклонились Ему, а иные усомнились.
\par 18 И приблизившись Иисус сказал им: дана Мне всякая власть на небе и на земле.
\par 19 Итак идите, научите все народы, крестя их во имя Отца и Сына и Святаго Духа,
\par 20 уча их соблюдать все, что Я повелел вам; и се, Я с вами во все дни до скончания века. Аминь.


\end{document}