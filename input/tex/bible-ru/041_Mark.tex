\begin{document}

\title{От Марка}


\chapter{1}

\par 1 Начало Евангелия Иисуса Христа, Сына Божия,
\par 2 как написано у пророков: вот, Я посылаю Ангела Моего пред лицем Твоим, который приготовит путь Твой пред Тобою.
\par 3 Глас вопиющего в пустыне: приготовьте путь Господу, прямыми сделайте стези Ему.
\par 4 Явился Иоанн, крестя в пустыне и проповедуя крещение покаяния для прощения грехов.
\par 5 И выходили к нему вся страна Иудейская и Иерусалимляне, и крестились от него все в реке Иордане, исповедуя грехи свои.
\par 6 Иоанн же носил одежду из верблюжьего волоса и пояс кожаный на чреслах своих, и ел акриды и дикий мед.
\par 7 И проповедывал, говоря: идет за мною Сильнейший меня, у Которого я недостоин, наклонившись, развязать ремень обуви Его;
\par 8 я крестил вас водою, а Он будет крестить вас Духом Святым.
\par 9 И было в те дни, пришел Иисус из Назарета Галилейского и крестился от Иоанна в Иордане.
\par 10 И когда выходил из воды, тотчас увидел [Иоанн] разверзающиеся небеса и Духа, как голубя, сходящего на Него.
\par 11 И глас был с небес: Ты Сын Мой возлюбленный, в Котором Мое благоволение.
\par 12 Немедленно после того Дух ведет Его в пустыню.
\par 13 И был Он там в пустыне сорок дней, искушаемый сатаною, и был со зверями; и Ангелы служили Ему.
\par 14 После же того, как предан был Иоанн, пришел Иисус в Галилею, проповедуя Евангелие Царствия Божия
\par 15 и говоря, что исполнилось время и приблизилось Царствие Божие: покайтесь и веруйте в Евангелие.
\par 16 Проходя же близ моря Галилейского, увидел Симона и Андрея, брата его, закидывающих сети в море, ибо они были рыболовы.
\par 17 И сказал им Иисус: идите за Мною, и Я сделаю, что вы будете ловцами человеков.
\par 18 И они тотчас, оставив свои сети, последовали за Ним.
\par 19 И, пройдя оттуда немного, Он увидел Иакова Зеведеева и Иоанна, брата его, также в лодке починивающих сети;
\par 20 и тотчас призвал их. И они, оставив отца своего Зеведея в лодке с работниками, последовали за Ним.
\par 21 И приходят в Капернаум; и вскоре в субботу вошел Он в синагогу и учил.
\par 22 И дивились Его учению, ибо Он учил их, как власть имеющий, а не как книжники.
\par 23 В синагоге их был человек, [одержимый] духом нечистым, и вскричал:
\par 24 оставь! что Тебе до нас, Иисус Назарянин? Ты пришел погубить нас! знаю Тебя, кто Ты, Святый Божий.
\par 25 Но Иисус запретил ему, говоря: замолчи и выйди из него.
\par 26 Тогда дух нечистый, сотрясши его и вскричав громким голосом, вышел из него.
\par 27 И все ужаснулись, так что друг друга спрашивали: что это? что это за новое учение, что Он и духам нечистым повелевает со властью, и они повинуются Ему?
\par 28 И скоро разошлась о Нем молва по всей окрестности в Галилее.
\par 29 Выйдя вскоре из синагоги, пришли в дом Симона и Андрея, с Иаковом и Иоанном.
\par 30 Теща же Симонова лежала в горячке; и тотчас говорят Ему о ней.
\par 31 Подойдя, Он поднял ее, взяв ее за руку; и горячка тотчас оставила ее, и она стала служить им.
\par 32 При наступлении же вечера, когда заходило солнце, приносили к Нему всех больных и бесноватых.
\par 33 И весь город собрался к дверям.
\par 34 И Он исцелил многих, страдавших различными болезнями; изгнал многих бесов, и не позволял бесам говорить, что они знают, что Он Христос.
\par 35 А утром, встав весьма рано, вышел и удалился в пустынное место, и там молился.
\par 36 Симон и бывшие с ним пошли за Ним
\par 37 и, найдя Его, говорят Ему: все ищут Тебя.
\par 38 Он говорит им: пойдем в ближние селения и города, чтобы Мне и там проповедывать, ибо Я для того пришел.
\par 39 И Он проповедывал в синагогах их по всей Галилее и изгонял бесов.
\par 40 Приходит к Нему прокаженный и, умоляя Его и падая пред Ним на колени, говорит Ему: если хочешь, можешь меня очистить.
\par 41 Иисус, умилосердившись над ним, простер руку, коснулся его и сказал ему: хочу, очистись.
\par 42 После сего слова проказа тотчас сошла с него, и он стал чист.
\par 43 И, посмотрев на него строго, тотчас отослал его
\par 44 и сказал ему: смотри, никому ничего не говори, но пойди, покажись священнику и принеси за очищение твое, что повелел Моисей, во свидетельство им.
\par 45 А он, выйдя, начал провозглашать и рассказывать о происшедшем, так что [Иисус] не мог уже явно войти в город, но находился вне, в местах пустынных. И приходили к Нему отовсюду.

\chapter{2}

\par 1 Через [несколько] дней опять пришел Он в Капернаум; и слышно стало, что Он в доме.
\par 2 Тотчас собрались многие, так что уже и у дверей не было места; и Он говорил им слово.
\par 3 И пришли к Нему с расслабленным, которого несли четверо;
\par 4 и, не имея возможности приблизиться к Нему за многолюдством, раскрыли [кровлю] дома, где Он находился, и, прокопав ее, спустили постель, на которой лежал расслабленный.
\par 5 Иисус, видя веру их, говорит расслабленному: чадо! прощаются тебе грехи твои.
\par 6 Тут сидели некоторые из книжников и помышляли в сердцах своих:
\par 7 что Он так богохульствует? кто может прощать грехи, кроме одного Бога?
\par 8 Иисус, тотчас узнав духом Своим, что они так помышляют в себе, сказал им: для чего так помышляете в сердцах ваших?
\par 9 Что легче? сказать ли расслабленному: прощаются тебе грехи? или сказать: встань, возьми свою постель и ходи?
\par 10 Но чтобы вы знали, что Сын Человеческий имеет власть на земле прощать грехи, --говорит расслабленному:
\par 11 тебе говорю: встань, возьми постель твою и иди в дом твой.
\par 12 Он тотчас встал и, взяв постель, вышел перед всеми, так что все изумлялись и прославляли Бога, говоря: никогда ничего такого мы не видали.
\par 13 И вышел [Иисус] опять к морю; и весь народ пошел к Нему, и Он учил их.
\par 14 Проходя, увидел Он Левия Алфеева, сидящего у сбора пошлин, и говорит ему: следуй за Мною. И [он], встав, последовал за Ним.
\par 15 И когда Иисус возлежал в доме его, возлежали с Ним и ученики Его и многие мытари и грешники: ибо много их было, и они следовали за Ним.
\par 16 Книжники и фарисеи, увидев, что Он ест с мытарями и грешниками, говорили ученикам Его: как это Он ест и пьет с мытарями и грешниками?
\par 17 Услышав [сие], Иисус говорит им: не здоровые имеют нужду во враче, но больные; Я пришел призвать не праведников, но грешников к покаянию.
\par 18 Ученики Иоанновы и фарисейские постились. Приходят к Нему и говорят: почему ученики Иоанновы и фарисейские постятся, а Твои ученики не постятся?
\par 19 И сказал им Иисус: могут ли поститься сыны чертога брачного, когда с ними жених? Доколе с ними жених, не могут поститься,
\par 20 но придут дни, когда отнимется у них жених, и тогда будут поститься в те дни.
\par 21 Никто к ветхой одежде не приставляет заплаты из небеленой ткани: иначе вновь пришитое отдерет от старого, и дыра будет еще хуже.
\par 22 Никто не вливает вина молодого в мехи ветхие: иначе молодое вино прорвет мехи, и вино вытечет, и мехи пропадут; но вино молодое надобно вливать в мехи новые.
\par 23 И случилось Ему в субботу проходить засеянными [полями], и ученики Его дорогою начали срывать колосья.
\par 24 И фарисеи сказали Ему: смотри, что они делают в субботу, чего не должно [делать]?
\par 25 Он сказал им: неужели вы не читали никогда, что сделал Давид, когда имел нужду и взалкал сам и бывшие с ним?
\par 26 как вошел он в дом Божий при первосвященнике Авиафаре и ел хлебы предложения, которых не должно было есть никому, кроме священников, и дал и бывшим с ним?
\par 27 И сказал им: суббота для человека, а не человек для субботы;
\par 28 посему Сын Человеческий есть господин и субботы.

\chapter{3}

\par 1 И пришел опять в синагогу; там был человек, имевший иссохшую руку.
\par 2 И наблюдали за Ним, не исцелит ли его в субботу, чтобы обвинить Его.
\par 3 Он же говорит человеку, имевшему иссохшую руку: стань на средину.
\par 4 А им говорит: должно ли в субботу добро делать, или зло делать? душу спасти, или погубить? Но они молчали.
\par 5 И, воззрев на них с гневом, скорбя об ожесточении сердец их, говорит тому человеку: протяни руку твою. Он протянул, и стала рука его здорова, как другая.
\par 6 Фарисеи, выйдя, немедленно составили с иродианами совещание против Него, как бы погубить Его.
\par 7 Но Иисус с учениками Своими удалился к морю; и за Ним последовало множество народа из Галилеи, Иудеи,
\par 8 Иерусалима, Идумеи и из-за Иордана. И [живущие] в окрестностях Тира и Сидона, услышав, что Он делал, шли к Нему в великом множестве.
\par 9 И сказал ученикам Своим, чтобы готова была для Него лодка по причине многолюдства, дабы не теснили Его.
\par 10 Ибо многих Он исцелил, так что имевшие язвы бросались к Нему, чтобы коснуться Его.
\par 11 И духи нечистые, когда видели Его, падали пред Ним и кричали: Ты Сын Божий.
\par 12 Но Он строго запрещал им, чтобы не делали Его известным.
\par 13 Потом взошел на гору и позвал к Себе, кого Сам хотел; и пришли к Нему.
\par 14 И поставил [из них] двенадцать, чтобы с Ним были и чтобы посылать их на проповедь,
\par 15 и чтобы они имели власть исцелять от болезней и изгонять бесов;
\par 16 [поставил] Симона, нарекши ему имя Петр,
\par 17 Иакова Зеведеева и Иоанна, брата Иакова, нарекши им имена Воанергес, то есть `сыны громовы',
\par 18 Андрея, Филиппа, Варфоломея, Матфея, Фому, Иакова Алфеева, Фаддея, Симона Кананита
\par 19 и Иуду Искариотского, который и предал Его.
\par 20 Приходят в дом; и опять сходится народ, так что им невозможно было и хлеба есть.
\par 21 И, услышав, ближние Его пошли взять Его, ибо говорили, что Он вышел из себя.
\par 22 А книжники, пришедшие из Иерусалима, говорили, что Он имеет [в] [Себе] веельзевула и что изгоняет бесов силою бесовского князя.
\par 23 И, призвав их, говорил им притчами: как может сатана изгонять сатану?
\par 24 Если царство разделится само в себе, не может устоять царство то;
\par 25 и если дом разделится сам в себе, не может устоять дом тот;
\par 26 и если сатана восстал на самого себя и разделился, не может устоять, но пришел конец его.
\par 27 Никто, войдя в дом сильного, не может расхитить вещей его, если прежде не свяжет сильного, и тогда расхитит дом его.
\par 28 Истинно говорю вам: будут прощены сынам человеческим все грехи и хуления, какими бы ни хулили;
\par 29 но кто будет хулить Духа Святаго, тому не будет прощения вовек, но подлежит он вечному осуждению.
\par 30 [Сие сказал Он], потому что говорили: в Нем нечистый дух.
\par 31 И пришли Матерь и братья Его и, стоя [вне] дома, послали к Нему звать Его.
\par 32 Около Него сидел народ. И сказали Ему: вот, Матерь Твоя и братья Твои и сестры Твои, [вне] дома, спрашивают Тебя.
\par 33 И отвечал им: кто матерь Моя и братья Мои?
\par 34 И обозрев сидящих вокруг Себя, говорит: вот матерь Моя и братья Мои;
\par 35 ибо кто будет исполнять волю Божию, тот Мне брат, и сестра, и матерь.

\chapter{4}

\par 1 И опять начал учить при море; и собралось к Нему множество народа, так что Он вошел в лодку и сидел на море, а весь народ был на земле, у моря.
\par 2 И учил их притчами много, и в учении Своем говорил им:
\par 3 слушайте: вот, вышел сеятель сеять;
\par 4 и, когда сеял, случилось, что иное упало при дороге, и налетели птицы и поклевали то.
\par 5 Иное упало на каменистое [место], где немного было земли, и скоро взошло, потому что земля была неглубока;
\par 6 когда же взошло солнце, увяло и, как не имело корня, засохло.
\par 7 Иное упало в терние, и терние выросло, и заглушило [семя], и оно не дало плода.
\par 8 И иное упало на добрую землю и дало плод, который взошел и вырос, и принесло иное тридцать, иное шестьдесят, и иное сто.
\par 9 И сказал им: кто имеет уши слышать, да слышит!
\par 10 Когда же остался без народа, окружающие Его, вместе с двенадцатью, спросили Его о притче.
\par 11 И сказал им: вам дано знать тайны Царствия Божия, а тем внешним все бывает в притчах;
\par 12 так что они своими глазами смотрят, и не видят; своими ушами слышат, и не разумеют, да не обратятся, и прощены будут им грехи.
\par 13 И говорит им: не понимаете этой притчи? Как же вам уразуметь все притчи?
\par 14 Сеятель слово сеет.
\par 15 [Посеянное] при дороге означает тех, в которых сеется слово, но [к которым], когда услышат, тотчас приходит сатана и похищает слово, посеянное в сердцах их.
\par 16 Подобным образом и посеянное на каменистом [месте] означает тех, которые, когда услышат слово, тотчас с радостью принимают его,
\par 17 но не имеют в себе корня и непостоянны; потом, когда настанет скорбь или гонение за слово, тотчас соблазняются.
\par 18 Посеянное в тернии означает слышащих слово,
\par 19 но в которых заботы века сего, обольщение богатством и другие пожелания, входя в них, заглушают слово, и оно бывает без плода.
\par 20 А посеянное на доброй земле означает тех, которые слушают слово и принимают, и приносят плод, один в тридцать, другой в шестьдесят, иной во сто крат.
\par 21 И сказал им: для того ли приносится свеча, чтобы поставить ее под сосуд или под кровать? не для того ли, чтобы поставить ее на подсвечнике?
\par 22 Нет ничего тайного, что не сделалось бы явным, и ничего не бывает потаенного, что не вышло бы наружу.
\par 23 Если кто имеет уши слышать, да слышит!
\par 24 И сказал им: замечайте, что слышите: какою мерою мерите, такою отмерено будет вам и прибавлено будет вам, слушающим.
\par 25 Ибо кто имеет, тому дано будет, а кто не имеет, у того отнимется и то, что имеет.
\par 26 И сказал: Царствие Божие подобно тому, как если человек бросит семя в землю,
\par 27 и спит, и встает ночью и днем; и как семя всходит и растет, не знает он,
\par 28 ибо земля сама собою производит сперва зелень, потом колос, потом полное зерно в колосе.
\par 29 Когда же созреет плод, немедленно посылает серп, потому что настала жатва.
\par 30 И сказал: чему уподобим Царствие Божие? или какою притчею изобразим его?
\par 31 Оно--как зерно горчичное, которое, когда сеется в землю, есть меньше всех семян на земле;
\par 32 а когда посеяно, всходит и становится больше всех злаков, и пускает большие ветви, так что под тенью его могут укрываться птицы небесные.
\par 33 И таковыми многими притчами проповедывал им слово, сколько они могли слышать.
\par 34 Без притчи же не говорил им, а ученикам наедине изъяснял все.
\par 35 Вечером того дня сказал им: переправимся на ту сторону.
\par 36 И они, отпустив народ, взяли Его с собою, как Он был в лодке; с Ним были и другие лодки.
\par 37 И поднялась великая буря; волны били в лодку, так что она уже наполнялась [водою].
\par 38 А Он спал на корме на возглавии. Его будят и говорят Ему: Учитель! неужели Тебе нужды нет, что мы погибаем?
\par 39 И, встав, Он запретил ветру и сказал морю: умолкни, перестань. И ветер утих, и сделалась великая тишина.
\par 40 И сказал им: что вы так боязливы? как у вас нет веры?
\par 41 И убоялись страхом великим и говорили между собою: кто же Сей, что и ветер и море повинуются Ему?

\chapter{5}

\par 1 И пришли на другой берег моря, в страну Гадаринскую.
\par 2 И когда вышел Он из лодки, тотчас встретил Его вышедший из гробов человек, [одержимый] нечистым духом,
\par 3 он имел жилище в гробах, и никто не мог его связать даже цепями,
\par 4 потому что многократно был он скован оковами и цепями, но разрывал цепи и разбивал оковы, и никто не в силах был укротить его;
\par 5 всегда, ночью и днем, в горах и гробах, кричал он и бился о камни;
\par 6 увидев же Иисуса издалека, прибежал и поклонился Ему,
\par 7 и, вскричав громким голосом, сказал: что Тебе до меня, Иисус, Сын Бога Всевышнего? заклинаю Тебя Богом, не мучь меня!
\par 8 Ибо [Иисус] сказал ему: выйди, дух нечистый, из сего человека.
\par 9 И спросил его: как тебе имя? И он сказал в ответ: легион имя мне, потому что нас много.
\par 10 И много просили Его, чтобы не высылал их вон из страны той.
\par 11 Паслось же там при горе большое стадо свиней.
\par 12 И просили Его все бесы, говоря: пошли нас в свиней, чтобы нам войти в них.
\par 13 Иисус тотчас позволил им. И нечистые духи, выйдя, вошли в свиней; и устремилось стадо с крутизны в море, а их было около двух тысяч; и потонули в море.
\par 14 Пасущие же свиней побежали и рассказали в городе и в деревнях. И [жители] вышли посмотреть, что случилось.
\par 15 Приходят к Иисусу и видят, что бесновавшийся, в котором был легион, сидит и одет, и в здравом уме; и устрашились.
\par 16 Видевшие рассказали им о том, как это произошло с бесноватым, и о свиньях.
\par 17 И начали просить Его, чтобы отошел от пределов их.
\par 18 И когда Он вошел в лодку, бесновавшийся просил Его, чтобы быть с Ним.
\par 19 Но Иисус не дозволил ему, а сказал: иди домой к своим и расскажи им, что сотворил с тобою Господь и [как] помиловал тебя.
\par 20 И пошел и начал проповедывать в Десятиградии, что сотворил с ним Иисус; и все дивились.
\par 21 Когда Иисус опять переправился в лодке на другой берег, собралось к Нему множество народа. Он был у моря.
\par 22 И вот, приходит один из начальников синагоги, по имени Иаир, и, увидев Его, падает к ногам Его
\par 23 и усильно просит Его, говоря: дочь моя при смерти; приди и возложи на нее руки, чтобы она выздоровела и осталась жива.
\par 24 [Иисус] пошел с ним. За Ним следовало множество народа, и теснили Его.
\par 25 Одна женщина, которая страдала кровотечением двенадцать лет,
\par 26 много потерпела от многих врачей, истощила все, что было у ней, и не получила никакой пользы, но пришла еще в худшее состояние, --
\par 27 услышав об Иисусе, подошла сзади в народе и прикоснулась к одежде Его,
\par 28 ибо говорила: если хотя к одежде Его прикоснусь, то выздоровею.
\par 29 И тотчас иссяк у ней источник крови, и она ощутила в теле, что исцелена от болезни.
\par 30 В то же время Иисус, почувствовав Сам в Себе, что вышла из Него сила, обратился в народе и сказал: кто прикоснулся к Моей одежде?
\par 31 Ученики сказали Ему: Ты видишь, что народ теснит Тебя, и говоришь: кто прикоснулся ко Мне?
\par 32 Но Он смотрел вокруг, чтобы видеть ту, которая сделала это.
\par 33 Женщина в страхе и трепете, зная, что с нею произошло, подошла, пала пред Ним и сказала Ему всю истину.
\par 34 Он же сказал ей: дщерь! вера твоя спасла тебя; иди в мире и будь здорова от болезни твоей.
\par 35 Когда Он еще говорил сие, приходят от начальника синагоги и говорят: дочь твоя умерла; что еще утруждаешь Учителя?
\par 36 Но Иисус, услышав сии слова, тотчас говорит начальнику синагоги: не бойся, только веруй.
\par 37 И не позволил никому следовать за Собою, кроме Петра, Иакова и Иоанна, брата Иакова.
\par 38 Приходит в дом начальника синагоги и видит смятение и плачущих и вопиющих громко.
\par 39 И, войдя, говорит им: что смущаетесь и плачете? девица не умерла, но спит.
\par 40 И смеялись над Ним. Но Он, выслав всех, берет с Собою отца и мать девицы и бывших с Ним и входит туда, где девица лежала.
\par 41 И, взяв девицу за руку, говорит ей: `талифа куми', что значит: девица, тебе говорю, встань.
\par 42 И девица тотчас встала и начала ходить, ибо была лет двенадцати. [Видевшие] пришли в великое изумление.
\par 43 И Он строго приказал им, чтобы никто об этом не знал, и сказал, чтобы дали ей есть.

\chapter{6}

\par 1 Оттуда вышел Он и пришел в Свое отечество; за Ним следовали ученики Его.
\par 2 Когда наступила суббота, Он начал учить в синагоге; и многие слышавшие с изумлением говорили: откуда у Него это? что за премудрость дана Ему, и как такие чудеса совершаются руками Его?
\par 3 Не плотник ли Он, сын Марии, брат Иакова, Иосии, Иуды и Симона? Не здесь ли, между нами, Его сестры? И соблазнялись о Нем.
\par 4 Иисус же сказал им: не бывает пророк без чести, разве только в отечестве своем и у сродников и в доме своем.
\par 5 И не мог совершить там никакого чуда, только на немногих больных возложив руки, исцелил [их].
\par 6 И дивился неверию их; потом ходил по окрестным селениям и учил.
\par 7 И, призвав двенадцать, начал посылать их по два, и дал им власть над нечистыми духами.
\par 8 И заповедал им ничего не брать в дорогу, кроме одного посоха: ни сумы, ни хлеба, ни меди в поясе,
\par 9 но обуваться в простую обувь и не носить двух одежд.
\par 10 И сказал им: если где войдете в дом, оставайтесь в нем, доколе не выйдете из того места.
\par 11 И если кто не примет вас и не будет слушать вас, то, выходя оттуда, отрясите прах от ног ваших, во свидетельство на них. Истинно говорю вам: отраднее будет Содому и Гоморре в день суда, нежели тому городу.
\par 12 Они пошли и проповедывали покаяние;
\par 13 изгоняли многих бесов и многих больных мазали маслом и исцеляли.
\par 14 Царь Ирод, услышав [об Иисусе], --ибо имя Его стало гласно, --говорил: это Иоанн Креститель воскрес из мертвых, и потому чудеса делаются им.
\par 15 Другие говорили: это Илия, а иные говорили: это пророк, или как один из пророков.
\par 16 Ирод же, услышав, сказал: это Иоанн, которого я обезглавил; он воскрес из мертвых.
\par 17 Ибо сей Ирод, послав, взял Иоанна и заключил его в темницу за Иродиаду, жену Филиппа, брата своего, потому что женился на ней.
\par 18 Ибо Иоанн говорил Ироду: не должно тебе иметь жену брата твоего.
\par 19 Иродиада же, злобясь на него, желала убить его; но не могла.
\par 20 Ибо Ирод боялся Иоанна, зная, что он муж праведный и святой, и берег его; многое делал, слушаясь его, и с удовольствием слушал его.
\par 21 Настал удобный день, когда Ирод, по случаю [дня] рождения своего, делал пир вельможам своим, тысяченачальникам и старейшинам Галилейским, --
\par 22 дочь Иродиады вошла, плясала и угодила Ироду и возлежавшим с ним; царь сказал девице: проси у меня, чего хочешь, и дам тебе;
\par 23 и клялся ей: чего ни попросишь у меня, дам тебе, даже до половины моего царства.
\par 24 Она вышла и спросила у матери своей: чего просить? Та отвечала: головы Иоанна Крестителя.
\par 25 И она тотчас пошла с поспешностью к царю и просила, говоря: хочу, чтобы ты дал мне теперь же на блюде голову Иоанна Крестителя.
\par 26 Царь опечалился, но ради клятвы и возлежавших с ним не захотел отказать ей.
\par 27 И тотчас, послав оруженосца, царь повелел принести голову его.
\par 28 Он пошел, отсек ему голову в темнице, и принес голову его на блюде, и отдал ее девице, а девица отдала ее матери своей.
\par 29 Ученики его, услышав, пришли и взяли тело его, и положили его во гробе.
\par 30 И собрались Апостолы к Иисусу и рассказали Ему все, и что сделали, и чему научили.
\par 31 Он сказал им: пойдите вы одни в пустынное место и отдохните немного, --ибо много было приходящих и отходящих, так что и есть им было некогда.
\par 32 И отправились в пустынное место в лодке одни.
\par 33 Народ увидел, [как] они отправлялись, и многие узнали их; и бежали туда пешие из всех городов, и предупредили их, и собрались к Нему.
\par 34 Иисус, выйдя, увидел множество народа и сжалился над ними, потому что они были, как овцы, не имеющие пастыря; и начал учить их много.
\par 35 И как времени прошло много, ученики Его, приступив к Нему, говорят: место [здесь] пустынное, а времени уже много, --
\par 36 отпусти их, чтобы они пошли в окрестные деревни и селения и купили себе хлеба, ибо им нечего есть.
\par 37 Он сказал им в ответ: вы дайте им есть. И сказали Ему: разве нам пойти купить хлеба динариев на двести и дать им есть?
\par 38 Но Он спросил их: сколько у вас хлебов? пойдите, посмотрите. Они, узнав, сказали: пять хлебов и две рыбы.
\par 39 Тогда повелел им рассадить всех отделениями на зеленой траве.
\par 40 И сели рядами, по сто и по пятидесяти.
\par 41 Он взял пять хлебов и две рыбы, воззрев на небо, благословил и преломил хлебы и дал ученикам Своим, чтобы они раздали им; и две рыбы разделил на всех.
\par 42 И ели все, и насытились.
\par 43 И набрали кусков хлеба и [остатков] от рыб двенадцать полных коробов.
\par 44 Было же евших хлебы около пяти тысяч мужей.
\par 45 И тотчас понудил учеников Своих войти в лодку и отправиться вперед на другую сторону к Вифсаиде, пока Он отпустит народ.
\par 46 И, отпустив их, пошел на гору помолиться.
\par 47 Вечером лодка была посреди моря, а Он один на земле.
\par 48 И увидел их бедствующих в плавании, потому что ветер им был противный; около же четвертой стражи ночи подошел к ним, идя по морю, и хотел миновать их.
\par 49 Они, увидев Его идущего по морю, подумали, что это призрак, и вскричали.
\par 50 Ибо все видели Его и испугались. И тотчас заговорил с ними и сказал им: ободритесь; это Я, не бойтесь.
\par 51 И вошел к ним в лодку, и ветер утих. И они чрезвычайно изумлялись в себе и дивились,
\par 52 ибо не вразумились [чудом] над хлебами, потому что сердце их было окаменено.
\par 53 И, переправившись, прибыли в землю Геннисаретскую и пристали [к] [берегу].
\par 54 Когда вышли они из лодки, тотчас [жители], узнав Его,
\par 55 обежали всю окрестность ту и начали на постелях приносить больных туда, где Он, как слышно было, находился.
\par 56 И куда ни приходил Он, в селения ли, в города ли, в деревни ли, клали больных на открытых местах и просили Его, чтобы им прикоснуться хотя к краю одежды Его; и которые прикасались к Нему, исцелялись.

\chapter{7}

\par 1 Собрались к Нему фарисеи и некоторые из книжников, пришедшие из Иерусалима,
\par 2 и, увидев некоторых из учеников Его, евших хлеб нечистыми, то есть неумытыми, руками, укоряли.
\par 3 Ибо фарисеи и все Иудеи, держась предания старцев, не едят, не умыв тщательно рук;
\par 4 и, [придя] с торга, не едят не омывшись. Есть и многое другое, чего они приняли держаться: наблюдать омовение чаш, кружек, котлов и скамей.
\par 5 Потом спрашивают Его фарисеи и книжники: зачем ученики Твои не поступают по преданию старцев, но неумытыми руками едят хлеб?
\par 6 Он сказал им в ответ: хорошо пророчествовал о вас, лицемерах, Исаия, как написано: люди сии чтут Меня устами, сердце же их далеко отстоит от Меня,
\par 7 но тщетно чтут Меня, уча учениям, заповедям человеческим.
\par 8 Ибо вы, оставив заповедь Божию, держитесь предания человеческого, омовения кружек и чаш, и делаете многое другое, сему подобное.
\par 9 И сказал им: хорошо ли, [что] вы отменяете заповедь Божию, чтобы соблюсти свое предание?
\par 10 Ибо Моисей сказал: почитай отца своего и мать свою; и: злословящий отца или мать смертью да умрет.
\par 11 А вы говорите: кто скажет отцу или матери: корван, то есть дар [Богу] то, чем бы ты от меня пользовался,
\par 12 тому вы уже попускаете ничего не делать для отца своего или матери своей,
\par 13 устраняя слово Божие преданием вашим, которое вы установили; и делаете многое сему подобное.
\par 14 И, призвав весь народ, говорил им: слушайте Меня все и разумейте:
\par 15 ничто, входящее в человека извне, не может осквернить его; но что исходит из него, то оскверняет человека.
\par 16 Если кто имеет уши слышать, да слышит!
\par 17 И когда Он от народа вошел в дом, ученики Его спросили Его о притче.
\par 18 Он сказал им: неужели и вы так непонятливы? Неужели не разумеете, что ничто, извне входящее в человека, не может осквернить его?
\par 19 Потому что не в сердце его входит, а в чрево, и выходит вон, [чем] очищается всякая пища.
\par 20 Далее сказал: исходящее из человека оскверняет человека.
\par 21 Ибо извнутрь, из сердца человеческого, исходят злые помыслы, прелюбодеяния, любодеяния, убийства,
\par 22 кражи, лихоимство, злоба, коварство, непотребство, завистливое око, богохульство, гордость, безумство, --
\par 23 все это зло извнутрь исходит и оскверняет человека.
\par 24 И, отправившись оттуда, пришел в пределы Тирские и Сидонские; и, войдя в дом, не хотел, чтобы кто узнал; но не мог утаиться.
\par 25 Ибо услышала о Нем женщина, у которой дочь одержима была нечистым духом, и, придя, припала к ногам Его;
\par 26 а женщина та была язычница, родом сирофиникиянка; и просила Его, чтобы изгнал беса из ее дочери.
\par 27 Но Иисус сказал ей: дай прежде насытиться детям, ибо нехорошо взять хлеб у детей и бросить псам.
\par 28 Она же сказала Ему в ответ: так, Господи; но и псы под столом едят крохи у детей.
\par 29 И сказал ей: за это слово, пойди; бес вышел из твоей дочери.
\par 30 И, придя в свой дом, она нашла, что бес вышел и дочь лежит на постели.
\par 31 Выйдя из пределов Тирских и Сидонских, [Иисус] опять пошел к морю Галилейскому через пределы Десятиградия.
\par 32 Привели к Нему глухого косноязычного и просили Его возложить на него руку.
\par 33 [Иисус], отведя его в сторону от народа, вложил персты Свои в уши ему и, плюнув, коснулся языка его;
\par 34 и, воззрев на небо, вздохнул и сказал ему: `еффафа', то есть: отверзись.
\par 35 И тотчас отверзся у него слух и разрешились узы его языка, и стал говорить чисто.
\par 36 И повелел им не сказывать никому. Но сколько Он ни запрещал им, они еще более разглашали.
\par 37 И чрезвычайно дивились, и говорили: все хорошо делает, --и глухих делает слышащими, и немых--говорящими.

\chapter{8}

\par 1 В те дни, когда собралось весьма много народа и нечего было им есть, Иисус, призвав учеников Своих, сказал им:
\par 2 жаль Мне народа, что уже три дня находятся при Мне, и нечего им есть.
\par 3 Если неевшими отпущу их в домы их, ослабеют в дороге, ибо некоторые из них пришли издалека.
\par 4 Ученики Его отвечали Ему: откуда мог бы кто [взять] здесь в пустыне хлебов, чтобы накормить их?
\par 5 И спросил их: сколько у вас хлебов? Они сказали: семь.
\par 6 Тогда велел народу возлечь на землю; и, взяв семь хлебов и воздав благодарение, преломил и дал ученикам Своим, чтобы они раздали; и они раздали народу.
\par 7 Было у них и немного рыбок: благословив, Он велел раздать и их.
\par 8 И ели, и насытились; и набрали оставшихся кусков семь корзин.
\par 9 Евших же было около четырех тысяч. И отпустил их.
\par 10 И тотчас войдя в лодку с учениками Своими, прибыл в пределы Далмануфские.
\par 11 Вышли фарисеи, начали с Ним спорить и требовали от Него знамения с неба, искушая Его.
\par 12 И Он, глубоко вздохнув, сказал: для чего род сей требует знамения? Истинно говорю вам, не дастся роду сему знамение.
\par 13 И, оставив их, опять вошел в лодку и отправился на ту сторону.
\par 14 При сем ученики Его забыли взять хлебов и кроме одного хлеба не имели с собою в лодке.
\par 15 А Он заповедал им, говоря: смотрите, берегитесь закваски фарисейской и закваски Иродовой.
\par 16 И, рассуждая между собою, говорили: [это значит], что хлебов нет у нас.
\par 17 Иисус, уразумев, говорит им: что рассуждаете о том, что нет у вас хлебов? Еще ли не понимаете и не разумеете? Еще ли окаменено у вас сердце?
\par 18 Имея очи, не видите? имея уши, не слышите? и не помните?
\par 19 Когда Я пять хлебов преломил для пяти тысяч [человек], сколько полных коробов набрали вы кусков? Говорят Ему: двенадцать.
\par 20 А когда семь для четырех тысяч, сколько корзин набрали вы оставшихся кусков. Сказали: семь.
\par 21 И сказал им: как же не разумеете?
\par 22 Приходит в Вифсаиду; и приводят к Нему слепого и просят, чтобы прикоснулся к нему.
\par 23 Он, взяв слепого за руку, вывел его вон из селения и, плюнув ему на глаза, возложил на него руки и спросил его: видит ли что?
\par 24 Он, взглянув, сказал: вижу проходящих людей, как деревья.
\par 25 Потом опять возложил руки на глаза ему и велел ему взглянуть. И он исцелел и стал видеть все ясно.
\par 26 И послал его домой, сказав: не заходи в селение и не рассказывай никому в селении.
\par 27 И пошел Иисус с учениками Своими в селения Кесарии Филипповой. Дорогою Он спрашивал учеников Своих: за кого почитают Меня люди?
\par 28 Они отвечали: за Иоанна Крестителя; другие же--за Илию; а иные--за одного из пророков.
\par 29 Он говорит им: а вы за кого почитаете Меня? Петр сказал Ему в ответ: Ты Христос.
\par 30 И запретил им, чтобы никому не говорили о Нем.
\par 31 И начал учить их, что Сыну Человеческому много должно пострадать, быть отвержену старейшинами, первосвященниками и книжниками, и быть убиту, и в третий день воскреснуть.
\par 32 И говорил о сем открыто. Но Петр, отозвав Его, начал прекословить Ему.
\par 33 Он же, обратившись и взглянув на учеников Своих, воспретил Петру, сказав: отойди от Меня, сатана, потому что ты думаешь не о том, что Божие, но что человеческое.
\par 34 И, подозвав народ с учениками Своими, сказал им: кто хочет идти за Мною, отвергнись себя, и возьми крест свой, и следуй за Мною.
\par 35 Ибо кто хочет душу свою сберечь, тот потеряет ее, а кто потеряет душу свою ради Меня и Евангелия, тот сбережет ее.
\par 36 Ибо какая польза человеку, если он приобретет весь мир, а душе своей повредит?
\par 37 Или какой выкуп даст человек за душу свою?
\par 38 Ибо кто постыдится Меня и Моих слов в роде сем прелюбодейном и грешном, того постыдится и Сын Человеческий, когда приидет в славе Отца Своего со святыми Ангелами.

\chapter{9}

\par 1 И сказал им: истинно говорю вам: есть некоторые из стоящих здесь, которые не вкусят смерти, как уже увидят Царствие Божие, пришедшее в силе.
\par 2 И, по прошествии дней шести, взял Иисус Петра, Иакова и Иоанна, и возвел на гору высокую особо их одних, и преобразился перед ними.
\par 3 Одежды Его сделались блистающими, весьма белыми, как снег, как на земле белильщик не может выбелить.
\par 4 И явился им Илия с Моисеем; и беседовали с Иисусом.
\par 5 При сем Петр сказал Иисусу: Равви! хорошо нам здесь быть; сделаем три кущи: Тебе одну, Моисею одну, и одну Илии.
\par 6 Ибо не знал, что сказать; потому что они были в страхе.
\par 7 И явилось облако, осеняющее их, и из облака исшел глас, глаголющий: Сей есть Сын Мой возлюбленный; Его слушайте.
\par 8 И, внезапно посмотрев вокруг, никого более с собою не видели, кроме одного Иисуса.
\par 9 Когда же сходили они с горы, Он не велел никому рассказывать о том, что видели, доколе Сын Человеческий не воскреснет из мертвых.
\par 10 И они удержали это слово, спрашивая друг друга, что значит: воскреснуть из мертвых.
\par 11 И спросили Его: как же книжники говорят, что Илии надлежит придти прежде?
\par 12 Он сказал им в ответ: правда, Илия должен придти прежде и устроить все; и Сыну Человеческому, как написано о Нем, [надлежит] много пострадать и быть уничижену.
\par 13 Но говорю вам, что и Илия пришел, и поступили с ним, как хотели, как написано о нем.
\par 14 Придя к ученикам, увидел много народа около них и книжников, спорящих с ними.
\par 15 Тотчас, увидев Его, весь народ изумился, и, подбегая, приветствовали Его.
\par 16 Он спросил книжников: о чем спорите с ними?
\par 17 Один из народа сказал в ответ: Учитель! я привел к Тебе сына моего, одержимого духом немым:
\par 18 где ни схватывает его, повергает его на землю, и он испускает пену, и скрежещет зубами своими, и цепенеет. Говорил я ученикам Твоим, чтобы изгнали его, и они не могли.
\par 19 Отвечая ему, Иисус сказал: о, род неверный! доколе буду с вами? доколе буду терпеть вас? Приведите его ко Мне.
\par 20 И привели его к Нему. Как скоро [бесноватый] увидел Его, дух сотряс его; он упал на землю и валялся, испуская пену.
\par 21 И спросил [Иисус] отца его: как давно это сделалось с ним? Он сказал: с детства;
\par 22 и многократно [дух] бросал его и в огонь и в воду, чтобы погубить его; но, если что можешь, сжалься над нами и помоги нам.
\par 23 Иисус сказал ему: если сколько-нибудь можешь веровать, все возможно верующему.
\par 24 И тотчас отец отрока воскликнул со слезами: верую, Господи! помоги моему неверию.
\par 25 Иисус, видя, что сбегается народ, запретил духу нечистому, сказав ему: дух немой и глухой! Я повелеваю тебе, выйди из него и впредь не входи в него.
\par 26 И, вскрикнув и сильно сотрясши его, вышел; и он сделался, как мертвый, так что многие говорили, что он умер.
\par 27 Но Иисус, взяв его за руку, поднял его; и он встал.
\par 28 И как вошел [Иисус] в дом, ученики Его спрашивали Его наедине: почему мы не могли изгнать его?
\par 29 И сказал им: сей род не может выйти иначе, как от молитвы и поста.
\par 30 Выйдя оттуда, проходили через Галилею; и Он не хотел, чтобы кто узнал.
\par 31 Ибо учил Своих учеников и говорил им, что Сын Человеческий предан будет в руки человеческие и убьют Его, и, по убиении, в третий день воскреснет.
\par 32 Но они не разумели сих слов, а спросить Его боялись.
\par 33 Пришел в Капернаум; и когда был в доме, спросил их: о чем дорогою вы рассуждали между собою?
\par 34 Они молчали; потому что дорогою рассуждали между собою, кто больше.
\par 35 И, сев, призвал двенадцать и сказал им: кто хочет быть первым, будь из всех последним и всем слугою.
\par 36 И, взяв дитя, поставил его посреди них и, обняв его, сказал им:
\par 37 кто примет одно из таких детей во имя Мое, тот принимает Меня; а кто Меня примет, тот не Меня принимает, но Пославшего Меня.
\par 38 При сем Иоанн сказал: Учитель! мы видели человека, который именем Твоим изгоняет бесов, а не ходит за нами; и запретили ему, потому что не ходит за нами.
\par 39 Иисус сказал: не запрещайте ему, ибо никто, сотворивший чудо именем Моим, не может вскоре злословить Меня.
\par 40 Ибо кто не против вас, тот за вас.
\par 41 И кто напоит вас чашею воды во имя Мое, потому что вы Христовы, истинно говорю вам, не потеряет награды своей.
\par 42 А кто соблазнит одного из малых сих, верующих в Меня, тому лучше было бы, если бы повесили ему жерновный камень на шею и бросили его в море.
\par 43 И если соблазняет тебя рука твоя, отсеки ее: лучше тебе увечному войти в жизнь, нежели с двумя руками идти в геенну, в огонь неугасимый,
\par 44 где червь их не умирает и огонь не угасает.
\par 45 И если нога твоя соблазняет тебя, отсеки ее: лучше тебе войти в жизнь хромому, нежели с двумя ногами быть ввержену в геенну, в огонь неугасимый,
\par 46 где червь их не умирает и огонь не угасает.
\par 47 И если глаз твой соблазняет тебя, вырви его: лучше тебе с одним глазом войти в Царствие Божие, нежели с двумя глазами быть ввержену в геенну огненную,
\par 48 где червь их не умирает и огонь не угасает.
\par 49 Ибо всякий огнем осолится, и всякая жертва солью осолится.
\par 50 Соль--добрая [вещь]; но ежели соль не солона будет, чем вы ее поправите? Имейте в себе соль, и мир имейте между собою.

\chapter{10}

\par 1 Отправившись оттуда, приходит в пределы Иудейские за Иорданскою стороною. Опять собирается к Нему народ, и, по обычаю Своему, Он опять учил их.
\par 2 Подошли фарисеи и спросили, искушая Его: позволительно ли разводиться мужу с женою?
\par 3 Он сказал им в ответ: что заповедал вам Моисей?
\par 4 Они сказали: Моисей позволил писать разводное письмо и разводиться.
\par 5 Иисус сказал им в ответ: по жестокосердию вашему он написал вам сию заповедь.
\par 6 В начале же создания, Бог мужчину и женщину сотворил их.
\par 7 Посему оставит человек отца своего и мать
\par 8 и прилепится к жене своей, и будут два одною плотью; так что они уже не двое, но одна плоть.
\par 9 Итак, что Бог сочетал, того человек да не разлучает.
\par 10 В доме ученики Его опять спросили Его о том же.
\par 11 Он сказал им: кто разведется с женою своею и женится на другой, тот прелюбодействует от нее;
\par 12 и если жена разведется с мужем своим и выйдет за другого, прелюбодействует.
\par 13 Приносили к Нему детей, чтобы Он прикоснулся к ним; ученики же не допускали приносящих.
\par 14 Увидев [то], Иисус вознегодовал и сказал им: пустите детей приходить ко Мне и не препятствуйте им, ибо таковых есть Царствие Божие.
\par 15 Истинно говорю вам: кто не примет Царствия Божия, как дитя, тот не войдет в него.
\par 16 И, обняв их, возложил руки на них и благословил их.
\par 17 Когда выходил Он в путь, подбежал некто, пал пред Ним на колени и спросил Его: Учитель благий! что мне делать, чтобы наследовать жизнь вечную?
\par 18 Иисус сказал ему: что ты называешь Меня благим? Никто не благ, как только один Бог.
\par 19 Знаешь заповеди: не прелюбодействуй, не убивай, не кради, не лжесвидетельствуй, не обижай, почитай отца твоего и мать.
\par 20 Он же сказал Ему в ответ: Учитель! все это сохранил я от юности моей.
\par 21 Иисус, взглянув на него, полюбил его и сказал ему: одного тебе недостает: пойди, все, что имеешь, продай и раздай нищим, и будешь иметь сокровище на небесах; и приходи, последуй за Мною, взяв крест.
\par 22 Он же, смутившись от сего слова, отошел с печалью, потому что у него было большое имение.
\par 23 И, посмотрев вокруг, Иисус говорит ученикам Своим: как трудно имеющим богатство войти в Царствие Божие!
\par 24 Ученики ужаснулись от слов Его. Но Иисус опять говорит им в ответ: дети! как трудно надеющимся на богатство войти в Царствие Божие!
\par 25 Удобнее верблюду пройти сквозь игольные уши, нежели богатому войти в Царствие Божие.
\par 26 Они же чрезвычайно изумлялись и говорили между собою: кто же может спастись?
\par 27 Иисус, воззрев на них, говорит: человекам это невозможно, но не Богу, ибо все возможно Богу.
\par 28 И начал Петр говорить Ему: вот, мы оставили все ипоследовали за Тобою.
\par 29 Иисус сказал в ответ: истинно говорю вам: нет никого, кто оставил бы дом, или братьев, или сестер, или отца, или мать, или жену, или детей, или земли, ради Меня и Евангелия,
\par 30 и не получил бы ныне, во время сие, среди гонений, во сто крат более домов, и братьев и сестер, и отцов, и матерей, и детей, и земель, а в веке грядущем жизни вечной.
\par 31 Многие же будут первые последними, и последние первыми.
\par 32 Когда были они на пути, восходя в Иерусалим, Иисус шел впереди их, а они ужасались и, следуя за Ним, были в страхе. Подозвав двенадцать, Он опять начал им говорить о том, что будет с Ним:
\par 33 вот, мы восходим в Иерусалим, и Сын Человеческий предан будет первосвященникам и книжникам, и осудят Его на смерть, и предадут Его язычникам,
\par 34 и поругаются над Ним, и будут бить Его, и оплюют Его, и убьют Его; и в третий день воскреснет.
\par 35 [Тогда] подошли к Нему сыновья Зеведеевы Иаков и Иоанн и сказали: Учитель! мы желаем, чтобы Ты сделал нам, о чем попросим.
\par 36 Он сказал им: что хотите, чтобы Я сделал вам?
\par 37 Они сказали Ему: дай нам сесть у Тебя, одному по правую сторону, а другому по левую в славе Твоей.
\par 38 Но Иисус сказал им: не знаете, чего просите. Можете ли пить чашу, которую Я пью, и креститься крещением, которым Я крещусь?
\par 39 Они отвечали: можем. Иисус же сказал им: чашу, которую Я пью, будете пить, и крещением, которым Я крещусь, будете креститься;
\par 40 а дать сесть у Меня по правую сторону и по левую--не от Меня [зависит], но кому уготовано.
\par 41 И, услышав, десять начали негодовать на Иакова и Иоанна.
\par 42 Иисус же, подозвав их, сказал им: вы знаете, что почитающиеся князьями народов господствуют над ними, и вельможи их властвуют ими.
\par 43 Но между вами да не будет так: а кто хочет быть большим между вами, да будем вам слугою;
\par 44 и кто хочет быть первым между вами, да будет всем рабом.
\par 45 Ибо и Сын Человеческий не для того пришел, чтобы Ему служили, но чтобы послужить и отдать душу Свою для искупления многих.
\par 46 Приходят в Иерихон. И когда выходил Он из Иерихона с учениками Своими и множеством народа, Вартимей, сын Тимеев, слепой сидел у дороги, прося [милостыни].
\par 47 Услышав, что это Иисус Назорей, он начал кричать и говорить: Иисус, Сын Давидов! помилуй меня.
\par 48 Многие заставляли его молчать; но он еще более стал кричать: Сын Давидов! помилуй меня.
\par 49 Иисус остановился и велел его позвать. Зовут слепого и говорят ему: не бойся, вставай, зовет тебя.
\par 50 Он сбросил с себя верхнюю одежду, встал и пришел к Иисусу.
\par 51 Отвечая ему, Иисус спросил: чего ты хочешь от Меня? Слепой сказал Ему: Учитель! чтобы мне прозреть.
\par 52 Иисус сказал ему: иди, вера твоя спасла тебя. И он тотчас прозрел и пошел за Иисусом по дороге.

\chapter{11}

\par 1 Когда приблизились к Иерусалиму, к Виффагии и Вифании, к горе Елеонской, [Иисус] посылает двух из учеников Своих
\par 2 и говорит им: пойдите в селение, которое прямо перед вами; входя в него, тотчас найдете привязанного молодого осла, на которого никто из людей не садился; отвязав его, приведите.
\par 3 И если кто скажет вам: что вы это делаете? --отвечайте, что он надобен Господу; и тотчас пошлет его сюда.
\par 4 Они пошли, и нашли молодого осла, привязанного у ворот на улице, и отвязали его.
\par 5 И некоторые из стоявших там говорили им: что делаете? [зачем] отвязываете осленка?
\par 6 Они отвечали им, как повелел Иисус; и те отпустили их.
\par 7 И привели осленка к Иисусу, и возложили на него одежды свои; [Иисус] сел на него.
\par 8 Многие же постилали одежды свои по дороге; а другие резали ветви с дерев и постилали по дороге.
\par 9 И предшествовавшие и сопровождавшие восклицали: осанна! благословен Грядущий во имя Господне!
\par 10 благословенно грядущее во имя Господа царство отца нашего Давида! осанна в вышних!
\par 11 И вошел Иисус в Иерусалим и в храм; и, осмотрев все, как время уже было позднее, вышел в Вифанию с двенадцатью.
\par 12 На другой день, когда они вышли из Вифании, Он взалкал;
\par 13 и, увидев издалека смоковницу, покрытую листьями, пошел, не найдет ли чего на ней; но, придя к ней, ничего не нашел, кроме листьев, ибо еще не время было [собирания] смокв.
\par 14 И сказал ей Иисус: отныне да не вкушает никто от тебя плода вовек! И слышали то ученики Его.
\par 15 Пришли в Иерусалим. Иисус, войдя в храм, начал выгонять продающих и покупающих в храме; и столы меновщиков и скамьи продающих голубей опрокинул;
\par 16 и не позволял, чтобы кто пронес через храм какую-либо вещь.
\par 17 И учил их, говоря: не написано ли: дом Мой домом молитвы наречется для всех народов? а вы сделали его вертепом разбойников.
\par 18 Услышали [это] книжники и первосвященники, и искали, как бы погубить Его, ибо боялись Его, потому что весь народ удивлялся учению Его.
\par 19 Когда же стало поздно, Он вышел вон из города.
\par 20 Поутру, проходя мимо, увидели, что смоковница засохла до корня.
\par 21 И, вспомнив, Петр говорит Ему: Равви! посмотри, смоковница, которую Ты проклял, засохла.
\par 22 Иисус, отвечая, говорит им:
\par 23 имейте веру Божию, ибо истинно говорю вам, если кто скажет горе сей: поднимись и ввергнись в море, и не усомнится в сердце своем, но поверит, что сбудется по словам его, --будет ему, что ни скажет.
\par 24 Потому говорю вам: все, чего ни будете просить в молитве, верьте, что получите, --и будет вам.
\par 25 И когда стоите на молитве, прощайте, если что имеете на кого, дабы и Отец ваш Небесный простил вам согрешения ваши.
\par 26 Если же не прощаете, то и Отец ваш Небесный не простит вам согрешений ваших.
\par 27 Пришли опять в Иерусалим. И когда Он ходил в храме, подошли к Нему первосвященники и книжники, и старейшины
\par 28 и говорили Ему: какою властью Ты это делаешь? и кто Тебе дал власть делать это?
\par 29 Иисус сказал им в ответ: спрошу и Я вас об одном, отвечайте Мне; [тогда] и Я скажу вам, какою властью это делаю.
\par 30 Крещение Иоанново с небес было, или от человеков? отвечайте Мне.
\par 31 Они рассуждали между собою: если скажем: с небес, --то Он скажет: почему же вы не поверили ему?
\par 32 а сказать: от человеков--боялись народа, потому что все полагали, что Иоанн точно был пророк.
\par 33 И сказали в ответ Иисусу: не знаем. Тогда Иисус сказал им в ответ: и Я не скажу вам, какою властью это делаю.

\chapter{12}

\par 1 И начал говорить им притчами: некоторый человек насадил виноградник и обнес оградою, и выкопал точило, и построил башню, и, отдав его виноградарям, отлучился.
\par 2 И послал в свое время к виноградарям слугу--принять от виноградарей плодов из виноградника.
\par 3 Они же, схватив его, били, и отослали ни с чем.
\par 4 Опять послал к ним другого слугу; и тому камнями разбили голову и отпустили его с бесчестьем.
\par 5 И опять иного послал: и того убили; и многих других то били, то убивали.
\par 6 Имея же еще одного сына, любезного ему, напоследок послал и его к ним, говоря: постыдятся сына моего.
\par 7 Но виноградари сказали друг другу: это наследник; пойдем, убьем его, и наследство будет наше.
\par 8 И, схватив его, убили и выбросили вон из виноградника.
\par 9 Что же сделает хозяин виноградника? --Придет и предаст смерти виноградарей, и отдаст виноградник другим.
\par 10 Неужели вы не читали сего в Писании: камень, который отвергли строители, тот самый сделался главою угла;
\par 11 это от Господа, и есть дивно в очах наших.
\par 12 И старались схватить Его, но побоялись народа, ибо поняли, что о них сказал притчу; и, оставив Его, отошли.
\par 13 И посылают к Нему некоторых из фарисеев и иродиан, чтобы уловить Его в слове.
\par 14 Они же, придя, говорят Ему: Учитель! мы знаем, что Ты справедлив и не заботишься об угождении кому-либо, ибо не смотришь ни на какое лице, но истинно пути Божию учишь. Позволительно ли давать подать кесарю или нет? давать ли нам или не давать?
\par 15 Но Он, зная их лицемерие, сказал им: что искушаете Меня? принесите Мне динарий, чтобы Мне видеть его.
\par 16 Они принесли. Тогда говорит им: чье это изображение и надпись? Они сказали Ему: кесаревы.
\par 17 Иисус сказал им в ответ: отдавайте кесарево кесарю, а Божие Богу. И дивились Ему.
\par 18 Потом пришли к Нему саддукеи, которые говорят, что нет воскресения, и спросили Его, говоря:
\par 19 Учитель! Моисей написал нам: если у кого умрет брат и оставит жену, а детей не оставит, то брат его пусть возьмет жену его и восстановит семя брату своему.
\par 20 Было семь братьев: первый взял жену и, умирая, не оставил детей.
\par 21 Взял ее второй и умер, и он не оставил детей; также и третий.
\par 22 Брали ее [за себя] семеро и не оставили детей. После всех умерла и жена.
\par 23 Итак, в воскресении, когда воскреснут, которого из них будет она женою? Ибо семеро имели ее женою?
\par 24 Иисус сказал им в ответ: этим ли приводитесь вы в заблуждение, не зная Писаний, ни силы Божией?
\par 25 Ибо, когда из мертвых воскреснут, [тогда] не будут ни жениться, ни замуж выходить, но будут, как Ангелы на небесах.
\par 26 А о мертвых, что они воскреснут, разве не читали вы в книге Моисея, как Бог при купине сказал ему: Я Бог Авраама, и Бог Исаака, и Бог Иакова?
\par 27 [Бог] не есть Бог мертвых, но Бог живых. Итак, вы весьма заблуждаетесь.
\par 28 Один из книжников, слыша их прения и видя, что [Иисус] хорошо им отвечал, подошел и спросил Его: какая первая из всех заповедей?
\par 29 Иисус отвечал ему: первая из всех заповедей: слушай, Израиль! Господь Бог наш есть Господь единый;
\par 30 и возлюби Господа Бога твоего всем сердцем твоим, и всею душею твоею, и всем разумением твоим, и всею крепостию твоею, --вот первая заповедь!
\par 31 Вторая подобная ей: возлюби ближнего твоего, как самого себя. Иной большей сих заповеди нет.
\par 32 Книжник сказал Ему: хорошо, Учитель! истину сказал Ты, что один есть Бог и нет иного, кроме Его;
\par 33 и любить Его всем сердцем и всем умом, и всею душею, и всею крепостью, и любить ближнего, как самого себя, есть больше всех всесожжений и жертв.
\par 34 Иисус, видя, что он разумно отвечал, сказал ему: недалеко ты от Царствия Божия. После того никто уже не смел спрашивать Его.
\par 35 Продолжая учить в храме, Иисус говорил: как говорят книжники, что Христос есть Сын Давидов?
\par 36 Ибо сам Давид сказал Духом Святым: сказал Господь Господу моему: седи одесную Меня, доколе положу врагов Твоих в подножие ног Твоих.
\par 37 Итак, сам Давид называет Его Господом: как же Он Сын ему? И множество народа слушало Его с услаждением.
\par 38 И говорил им в учении Своем: остерегайтесь книжников, любящих ходить в длинных одеждах и [принимать] приветствия в народных собраниях,
\par 39 сидеть впереди в синагогах и возлежать на первом [месте] на пиршествах, --
\par 40 сии, поядающие домы вдов и напоказ долго молящиеся, примут тягчайшее осуждение.
\par 41 И сел Иисус против сокровищницы и смотрел, как народ кладет деньги в сокровищницу. Многие богатые клали много.
\par 42 Придя же, одна бедная вдова положила две лепты, что составляет кодрант.
\par 43 Подозвав учеников Своих, [Иисус] сказал им: истинно говорю вам, что эта бедная вдова положила больше всех, клавших в сокровищницу,
\par 44 ибо все клали от избытка своего, а она от скудости своей положила все, что имела, все пропитание свое.

\chapter{13}

\par 1 И когда выходил Он из храма, говорит Ему один из учеников его: Учитель! посмотри, какие камни и какие здания!
\par 2 Иисус сказал ему в ответ: видишь сии великие здания? все это будет разрушено, так что не останется здесь камня на камне.
\par 3 И когда Он сидел на горе Елеонской против храма, спрашивали Его наедине Петр, и Иаков, и Иоанн, и Андрей:
\par 4 скажи нам, когда это будет, и какой признак, когда все сие должно совершиться?
\par 5 Отвечая им, Иисус начал говорить: берегитесь, чтобы кто не прельстил вас,
\par 6 ибо многие придут под именем Моим и будут говорить, что это Я; и многих прельстят.
\par 7 Когда же услышите о войнах и о военных слухах, не ужасайтесь: ибо надлежит [сему] быть, --но [это] еще не конец.
\par 8 Ибо восстанет народ на народ и царство на царство; и будут землетрясения по местам, и будут глады и смятения. Это--начало болезней.
\par 9 Но вы смотрите за собою, ибо вас будут предавать в судилища и бить в синагогах, и перед правителями и царями поставят вас за Меня, для свидетельства перед ними.
\par 10 И во всех народах прежде должно быть проповедано Евангелие.
\par 11 Когда же поведут предавать вас, не заботьтесь наперед, что вам говорить, и не обдумывайте; но что дано будет вам в тот час, то и говорите, ибо не вы будете говорить, но Дух Святый.
\par 12 Предаст же брат брата на смерть, и отец--детей; и восстанут дети на родителей и умертвят их.
\par 13 И будете ненавидимы всеми за имя Мое; претерпевший же до конца спасется.
\par 14 Когда же увидите мерзость запустения, реченную пророком Даниилом, стоящую, где не должно, --читающий да разумеет, --тогда находящиеся в Иудее да бегут в горы;
\par 15 а кто на кровле, тот не сходи в дом и не входи взять что-- нибудь из дома своего;
\par 16 и кто на поле, не обращайся назад взять одежду свою.
\par 17 Горе беременным и питающим сосцами в те дни.
\par 18 Молитесь, чтобы не случилось бегство ваше зимою.
\par 19 Ибо в те дни будет такая скорбь, какой не было от начала творения, которое сотворил Бог, даже доныне, и не будет.
\par 20 И если бы Господь не сократил тех дней, то не спаслась бы никакая плоть; но ради избранных, которых Он избрал, сократил те дни.
\par 21 Тогда, если кто вам скажет: вот, здесь Христос, или: вот, там, --не верьте.
\par 22 Ибо восстанут лжехристы и лжепророки и дадут знамения и чудеса, чтобы прельстить, если возможно, и избранных.
\par 23 Вы же берегитесь. Вот, Я наперед сказал вам все.
\par 24 Но в те дни, после скорби той, солнце померкнет, и луна не даст света своего,
\par 25 и звезды спадут с неба, и силы небесные поколеблются.
\par 26 Тогда увидят Сына Человеческого, грядущего на облаках с силою многою и славою.
\par 27 И тогда Он пошлет Ангелов Своих и соберет избранных Своих от четырех ветров, от края земли до края неба.
\par 28 От смоковницы возьмите подобие: когда ветви ее становятся уже мягки и пускают листья, то знаете, что близко лето.
\par 29 Так и когда вы увидите то сбывающимся, знайте, что близко, при дверях.
\par 30 Истинно говорю вам: не прейдет род сей, как все это будет.
\par 31 Небо и земля прейдут, но слова Мои не прейдут.
\par 32 О дне же том, или часе, никто не знает, ни Ангелы небесные, ни Сын, но только Отец.
\par 33 Смотрите, бодрствуйте, молитесь, ибо не знаете, когда наступит это время.
\par 34 Подобно как бы кто, отходя в путь и оставляя дом свой, дал слугам своим власть и каждому свое дело, и приказал привратнику бодрствовать.
\par 35 Итак бодрствуйте, ибо не знаете, когда придет хозяин дома: вечером, или в полночь, или в пение петухов, или поутру;
\par 36 чтобы, придя внезапно, не нашел вас спящими.
\par 37 А что вам говорю, говорю всем: бодрствуйте.

\chapter{14}

\par 1 Через два дня [надлежало] быть [празднику] Пасхи и опресноков. И искали первосвященники и книжники, как бы взять Его хитростью и убить;
\par 2 но говорили: [только] не в праздник, чтобы не произошло возмущения в народе.
\par 3 И когда был Он в Вифании, в доме Симона прокаженного, и возлежал, --пришла женщина с алавастровым сосудом мира из нарда чистого, драгоценного и, разбив сосуд, возлила Ему на голову.
\par 4 Некоторые же вознегодовали и говорили между собою: к чему сия трата мира?
\par 5 Ибо можно было бы продать его более нежели за триста динариев и раздать нищим. И роптали на нее.
\par 6 Но Иисус сказал: оставьте ее; что ее смущаете? Она доброе дело сделала для Меня.
\par 7 Ибо нищих всегда имеете с собою и, когда захотите, можете им благотворить; а Меня не всегда имеете.
\par 8 Она сделала, что могла: предварила помазать тело Мое к погребению.
\par 9 Истинно говорю вам: где ни будет проповедано Евангелие сие в целом мире, сказано будет, в память ее, и о том, что она сделала.
\par 10 И пошел Иуда Искариот, один из двенадцати, к первосвященникам, чтобы предать Его им.
\par 11 Они же, услышав, обрадовались, и обещали дать ему сребренники. И он искал, как бы в удобное время предать Его.
\par 12 В первый день опресноков, когда заколали пасхального [агнца], говорят Ему ученики Его: где хочешь есть пасху? мы пойдем и приготовим.
\par 13 И посылает двух из учеников Своих и говорит им: пойдите в город; и встретится вам человек, несущий кувшин воды; последуйте за ним
\par 14 и куда он войдет, скажите хозяину дома того: Учитель говорит: где комната, в которой бы Мне есть пасху с учениками Моими?
\par 15 И он покажет вам горницу большую, устланную, готовую: там приготовьте нам.
\par 16 И пошли ученики Его, и пришли в город, и нашли, как сказал им; и приготовили пасху.
\par 17 Когда настал вечер, Он приходит с двенадцатью.
\par 18 И, когда они возлежали и ели, Иисус сказал: истинно говорю вам, один из вас, ядущий со Мною, предаст Меня.
\par 19 Они опечалились и стали говорить Ему, один за другим: не я ли? и другой: не я ли?
\par 20 Он же сказал им в ответ: один из двенадцати, обмакивающий со Мною в блюдо.
\par 21 Впрочем Сын Человеческий идет, как писано о Нем; но горе тому человеку, которым Сын Человеческий предается: лучше было бы тому человеку не родиться.
\par 22 И когда они ели, Иисус, взяв хлеб, благословил, преломил, дал им и сказал: приимите, ядите; сие есть Тело Мое.
\par 23 И, взяв чашу, благодарив, подал им: и пили из нее все.
\par 24 И сказал им: сие есть Кровь Моя Нового Завета, за многих изливаемая.
\par 25 Истинно говорю вам: Я уже не буду пить от плода виноградного до того дня, когда буду пить новое вино в Царствии Божием.
\par 26 И, воспев, пошли на гору Елеонскую.
\par 27 И говорит им Иисус: все вы соблазнитесь о Мне в эту ночь; ибо написано: поражу пастыря, и рассеются овцы.
\par 28 По воскресении же Моем, Я предваряю вас в Галилее.
\par 29 Петр сказал Ему: если и все соблазнятся, но не я.
\par 30 И говорит ему Иисус: истинно говорю тебе, что ты ныне, в эту ночь, прежде нежели дважды пропоет петух, трижды отречешься от Меня.
\par 31 Но он еще с большим усилием говорил: хотя бы мне надлежало и умереть с Тобою, не отрекусь от Тебя. То же и все говорили.
\par 32 Пришли в селение, называемое Гефсимания; и Он сказал ученикам Своим: посидите здесь, пока Я помолюсь.
\par 33 И взял с Собою Петра, Иакова и Иоанна; и начал ужасаться и тосковать.
\par 34 И сказал им: душа Моя скорбит смертельно; побудьте здесь и бодрствуйте.
\par 35 И, отойдя немного, пал на землю и молился, чтобы, если возможно, миновал Его час сей;
\par 36 и говорил: Авва Отче! все возможно Тебе; пронеси чашу сию мимо Меня; но не чего Я хочу, а чего Ты.
\par 37 Возвращается и находит их спящими, и говорит Петру: Симон! ты спишь? не мог ты бодрствовать один час?
\par 38 Бодрствуйте и молитесь, чтобы не впасть в искушение: дух бодр, плоть же немощна.
\par 39 И, опять отойдя, молился, сказав то же слово.
\par 40 И, возвратившись, опять нашел их спящими, ибо глаза у них отяжелели, и они не знали, что Ему отвечать.
\par 41 И приходит в третий раз и говорит им: вы все еще спите и почиваете? Кончено, пришел час: вот, предается Сын Человеческий в руки грешников.
\par 42 Встаньте, пойдем; вот, приблизился предающий Меня.
\par 43 И тотчас, как Он еще говорил, приходит Иуда, один из двенадцати, и с ним множество народа с мечами и кольями, от первосвященников и книжников и старейшин.
\par 44 Предающий же Его дал им знак, сказав: Кого я поцелую, Тот и есть, возьмите Его и ведите осторожно.
\par 45 И, придя, тотчас подошел к Нему и говорит: Равви! Равви! и поцеловал Его.
\par 46 А они возложили на Него руки свои и взяли Его.
\par 47 Один же из стоявших тут извлек меч, ударил раба первосвященникова и отсек ему ухо.
\par 48 Тогда Иисус сказал им: как будто на разбойника вышли вы с мечами и кольями, чтобы взять Меня.
\par 49 Каждый день бывал Я с вами в храме и учил, и вы не брали Меня. Но да сбудутся Писания.
\par 50 Тогда, оставив Его, все бежали.
\par 51 Один юноша, завернувшись по нагому телу в покрывало, следовал за Ним; и воины схватили его.
\par 52 Но он, оставив покрывало, нагой убежал от них.
\par 53 И привели Иисуса к первосвященнику; и собрались к нему все первосвященники и старейшины и книжники.
\par 54 Петр издали следовал за Ним, даже внутрь двора первосвященникова; и сидел со служителями, и грелся у огня.
\par 55 Первосвященники же и весь синедрион искали свидетельства на Иисуса, чтобы предать Его смерти; и не находили.
\par 56 Ибо многие лжесвидетельствовали на Него, но свидетельства сии не были достаточны.
\par 57 И некоторые, встав, лжесвидетельствовали против Него и говорили:
\par 58 мы слышали, как Он говорил: Я разрушу храм сей рукотворенный, и через три дня воздвигну другой, нерукотворенный.
\par 59 Но и такое свидетельство их не было достаточно.
\par 60 Тогда первосвященник стал посреди и спросил Иисуса: что Ты ничего не отвечаешь? что они против Тебя свидетельствуют?
\par 61 Но Он молчал и не отвечал ничего. Опять первосвященник спросил Его и сказал Ему: Ты ли Христос, Сын Благословенного?
\par 62 Иисус сказал: Я; и вы узрите Сына Человеческого, сидящего одесную силы и грядущего на облаках небесных.
\par 63 Тогда первосвященник, разодрав одежды свои, сказал: на что еще нам свидетелей?
\par 64 Вы слышали богохульство; как вам кажется? Они же все признали Его повинным смерти.
\par 65 И некоторые начали плевать на Него и, закрывая Ему лице, ударять Его и говорить Ему: прореки. И слуги били Его по ланитам.
\par 66 Когда Петр был на дворе внизу, пришла одна из служанок первосвященника
\par 67 и, увидев Петра греющегося и всмотревшись в него, сказала: и ты был с Иисусом Назарянином.
\par 68 Но он отрекся, сказав: не знаю и не понимаю, что ты говоришь. И вышел вон на передний двор; и запел петух.
\par 69 Служанка, увидев его опять, начала говорить стоявшим тут: этот из них.
\par 70 Он опять отрекся. Спустя немного, стоявшие тут опять стали говорить Петру: точно ты из них; ибо ты Галилеянин, и наречие твое сходно.
\par 71 Он же начал клясться и божиться: не знаю Человека Сего, о Котором говорите.
\par 72 Тогда петух запел во второй раз. И вспомнил Петр слово, сказанное ему Иисусом: прежде нежели петух пропоет дважды, трижды отречешься от Меня; и начал плакать.

\chapter{15}

\par 1 Немедленно поутру первосвященники со старейшинами и книжниками и весь синедрион составили совещание и, связав Иисуса, отвели и предали Пилату.
\par 2 Пилат спросил Его: Ты Царь Иудейский? Он же сказал ему в ответ: ты говоришь.
\par 3 И первосвященники обвиняли Его во многом.
\par 4 Пилат же опять спросил Его: Ты ничего не отвечаешь? видишь, как много против Тебя обвинений.
\par 5 Но Иисус и на это ничего не отвечал, так что Пилат дивился.
\par 6 На всякий же праздник отпускал он им одного узника, о котором просили.
\par 7 Тогда был в узах [некто], по имени Варавва, со своими сообщниками, которые во время мятежа сделали убийство.
\par 8 И народ начал кричать и просить [Пилата] о том, что он всегда делал для них.
\par 9 Он сказал им в ответ: хотите ли, отпущу вам Царя Иудейского?
\par 10 Ибо знал, что первосвященники предали Его из зависти.
\par 11 Но первосвященники возбудили народ [просить], чтобы отпустил им лучше Варавву.
\par 12 Пилат, отвечая, опять сказал им: что же хотите, чтобы я сделал с Тем, Которого вы называете Царем Иудейским?
\par 13 Они опять закричали: распни Его.
\par 14 Пилат сказал им: какое же зло сделал Он? Но они еще сильнее закричали: распни Его.
\par 15 Тогда Пилат, желая сделать угодное народу, отпустил им Варавву, а Иисуса, бив, предал на распятие.
\par 16 А воины отвели Его внутрь двора, то есть в преторию, и собрали весь полк,
\par 17 и одели Его в багряницу, и, сплетши терновый венец, возложили на Него;
\par 18 и начали приветствовать Его: радуйся, Царь Иудейский!
\par 19 И били Его по голове тростью, и плевали на Него, и, становясь на колени, кланялись Ему.
\par 20 Когда же насмеялись над Ним, сняли с Него багряницу, одели Его в собственные одежды Его и повели Его, чтобы распять Его.
\par 21 И заставили проходящего некоего Киринеянина Симона, отца Александрова и Руфова, идущего с поля, нести крест Его.
\par 22 И привели Его на место Голгофу, что значит: Лобное место.
\par 23 И давали Ему пить вино со смирною; но Он не принял.
\par 24 Распявшие Его делили одежды Его, бросая жребий, кому что взять.
\par 25 Был час третий, и распяли Его.
\par 26 И была надпись вины Его: Царь Иудейский.
\par 27 С Ним распяли двух разбойников, одного по правую, а другого по левую [сторону] Его.
\par 28 И сбылось слово Писания: и к злодеям причтен.
\par 29 Проходящие злословили Его, кивая головами своими и говоря: э! разрушающий храм, и в три дня созидающий!
\par 30 спаси Себя Самого и сойди со креста.
\par 31 Подобно и первосвященники с книжниками, насмехаясь, говорили друг другу: других спасал, а Себя не может спасти.
\par 32 Христос, Царь Израилев, пусть сойдет теперь с креста, чтобы мы видели, и уверуем. И распятые с Ним поносили Его.
\par 33 В шестом же часу настала тьма по всей земле и [продолжалась] до часа девятого.
\par 34 В девятом часу возопил Иисус громким голосом: Элои! Элои! ламма савахфани? --что значит: Боже Мой! Боже Мой! для чего Ты Меня оставил?
\par 35 Некоторые из стоявших тут, услышав, говорили: вот, Илию зовет.
\par 36 А один побежал, наполнил губку уксусом и, наложив на трость, давал Ему пить, говоря: постойте, посмотрим, придет ли Илия снять Его.
\par 37 Иисус же, возгласив громко, испустил дух.
\par 38 И завеса в храме раздралась надвое, сверху донизу.
\par 39 Сотник, стоявший напротив Его, увидев, что Он, так возгласив, испустил дух, сказал: истинно Человек Сей был Сын Божий.
\par 40 Были [тут] и женщины, которые смотрели издали: между ними была и Мария Магдалина, и Мария, мать Иакова меньшего и Иосии, и Саломия,
\par 41 которые и тогда, как Он был в Галилее, следовали за Ним и служили Ему, и другие многие, вместе с Ним пришедшие в Иерусалим.
\par 42 И как уже настал вечер, --потому что была пятница, то есть [день] перед субботою, --
\par 43 пришел Иосиф из Аримафеи, знаменитый член совета, который и сам ожидал Царствия Божия, осмелился войти к Пилату, и просил тела Иисусова.
\par 44 Пилат удивился, что Он уже умер, и, призвав сотника, спросил его, давно ли умер?
\par 45 И, узнав от сотника, отдал тело Иосифу.
\par 46 Он, купив плащаницу и сняв Его, обвил плащаницею, и положил Его во гробе, который был высечен в скале, и привалил камень к двери гроба.
\par 47 Мария же Магдалина и Мария Иосиева смотрели, где Его полагали.

\chapter{16}

\par 1 По прошествии субботы Мария Магдалина и Мария Иаковлева и Саломия купили ароматы, чтобы идти помазать Его.
\par 2 И весьма рано, в первый [день] недели, приходят ко гробу, при восходе солнца,
\par 3 и говорят между собою: кто отвалит нам камень от двери гроба?
\par 4 И, взглянув, видят, что камень отвален; а он был весьма велик.
\par 5 И, войдя во гроб, увидели юношу, сидящего на правой стороне, облеченного в белую одежду; и ужаснулись.
\par 6 Он же говорит им: не ужасайтесь. Иисуса ищете Назарянина, распятого; Он воскрес, Его нет здесь. Вот место, где Он был положен.
\par 7 Но идите, скажите ученикам Его и Петру, что Он предваряет вас в Галилее; там Его увидите, как Он сказал вам.
\par 8 И, выйдя, побежали от гроба; их объял трепет и ужас, и никому ничего не сказали, потому что боялись.
\par 9 Воскреснув рано в первый [день] недели, [Иисус] явился сперва Марии Магдалине, из которой изгнал семь бесов.
\par 10 Она пошла и возвестила бывшим с Ним, плачущим и рыдающим;
\par 11 но они, услышав, что Он жив и она видела Его, --не поверили.
\par 12 После сего явился в ином образе двум из них на дороге, когда они шли в селение.
\par 13 И те, возвратившись, возвестили прочим; но и им не поверили.
\par 14 Наконец, явился самим одиннадцати, возлежавшим [на вечери], и упрекал их за неверие и жестокосердие, что видевшим Его воскресшего не поверили.
\par 15 И сказал им: идите по всему миру и проповедуйте Евангелие всей твари.
\par 16 Кто будет веровать и креститься, спасен будет; а кто не будет веровать, осужден будет.
\par 17 Уверовавших же будут сопровождать сии знамения: именем Моим будут изгонять бесов; будут говорить новыми языками;
\par 18 будут брать змей; и если что смертоносное выпьют, не повредит им; возложат руки на больных, и они будут здоровы.
\par 19 И так Господь, после беседования с ними, вознесся на небо и воссел одесную Бога.
\par 20 А они пошли и проповедывали везде, при Господнем содействии и подкреплении слова последующими знамениями. Аминь.


\end{document}