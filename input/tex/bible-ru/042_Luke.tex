\begin{document}

\title{От Луки}


\chapter{1}

\par 1 Как уже многие начали составлять повествования о совершенно известных между нами событиях,
\par 2 как передали нам то бывшие с самого начала очевидцами и служителями Слова,
\par 3 то рассудилось и мне, по тщательном исследовании всего сначала, по порядку описать тебе, достопочтенный Феофил,
\par 4 чтобы ты узнал твердое основание того учения, в котором был наставлен.
\par 5 Во дни Ирода, царя Иудейского, был священник из Авиевой чреды, именем Захария, и жена его из рода Ааронова, имя ей Елисавета.
\par 6 Оба они были праведны пред Богом, поступая по всем заповедям и уставам Господним беспорочно.
\par 7 У них не было детей, ибо Елисавета была неплодна, и оба были уже в летах преклонных.
\par 8 Однажды, когда он в порядке своей чреды служил пред Богом,
\par 9 по жребию, как обыкновенно было у священников, досталось ему войти в храм Господень для каждения,
\par 10 а все множество народа молилось вне во время каждения, --
\par 11 тогда явился ему Ангел Господень, стоя по правую сторону жертвенника кадильного.
\par 12 Захария, увидев его, смутился, и страх напал на него.
\par 13 Ангел же сказал ему: не бойся, Захария, ибо услышана молитва твоя, и жена твоя Елисавета родит тебе сына, и наречешь ему имя: Иоанн;
\par 14 и будет тебе радость и веселие, и многие о рождении его возрадуются,
\par 15 ибо он будет велик пред Господом; не будет пить вина и сикера, и Духа Святаго исполнится еще от чрева матери своей;
\par 16 и многих из сынов Израилевых обратит к Господу Богу их;
\par 17 и предъидет пред Ним в духе и силе Илии, чтобы возвратить сердца отцов детям, и непокоривым образ мыслей праведников, дабы представить Господу народ приготовленный.
\par 18 И сказал Захария Ангелу: по чему я узнаю это? ибо я стар, и жена моя в летах преклонных.
\par 19 Ангел сказал ему в ответ: я Гавриил, предстоящий пред Богом, и послан говорить с тобою и благовестить тебе сие;
\par 20 и вот, ты будешь молчать и не будешь иметь возможности говорить до того дня, как это сбудется, за то, что ты не поверил словам моим, которые сбудутся в свое время.
\par 21 Между тем народ ожидал Захарию и дивился, что он медлит в храме.
\par 22 Он же, выйдя, не мог говорить к ним; и они поняли, что он видел видение в храме; и он объяснялся с ними знаками, и оставался нем.
\par 23 А когда окончились дни службы его, возвратился в дом свой.
\par 24 После сих дней зачала Елисавета, жена его, и таилась пять месяцев и говорила:
\par 25 так сотворил мне Господь во дни сии, в которые призрел на меня, чтобы снять с меня поношение между людьми.
\par 26 В шестой же месяц послан был Ангел Гавриил от Бога в город Галилейский, называемый Назарет,
\par 27 к Деве, обрученной мужу, именем Иосифу, из дома Давидова; имя же Деве: Мария.
\par 28 Ангел, войдя к Ней, сказал: радуйся, Благодатная! Господь с Тобою; благословенна Ты между женами.
\par 29 Она же, увидев его, смутилась от слов его и размышляла, что бы это было за приветствие.
\par 30 И сказал Ей Ангел: не бойся, Мария, ибо Ты обрела благодать у Бога;
\par 31 и вот, зачнешь во чреве, и родишь Сына, и наречешь Ему имя: Иисус.
\par 32 Он будет велик и наречется Сыном Всевышнего, и даст Ему Господь Бог престол Давида, отца Его;
\par 33 и будет царствовать над домом Иакова во веки, и Царству Его не будет конца.
\par 34 Мария же сказала Ангелу: как будет это, когда Я мужа не знаю?
\par 35 Ангел сказал Ей в ответ: Дух Святый найдет на Тебя, и сила Всевышнего осенит Тебя; посему и рождаемое Святое наречется Сыном Божиим.
\par 36 Вот и Елисавета, родственница Твоя, называемая неплодною, и она зачала сына в старости своей, и ей уже шестой месяц,
\par 37 ибо у Бога не останется бессильным никакое слово.
\par 38 Тогда Мария сказала: се, Раба Господня; да будет Мне по слову твоему. И отошел от Нее Ангел.
\par 39 Встав же Мария во дни сии, с поспешностью пошла в нагорную страну, в город Иудин,
\par 40 и вошла в дом Захарии, и приветствовала Елисавету.
\par 41 Когда Елисавета услышала приветствие Марии, взыграл младенец во чреве ее; и Елисавета исполнилась Святаго Духа,
\par 42 и воскликнула громким голосом, и сказала: благословенна Ты между женами, и благословен плод чрева Твоего!
\par 43 И откуда это мне, что пришла Матерь Господа моего ко мне?
\par 44 Ибо когда голос приветствия Твоего дошел до слуха моего, взыграл младенец радостно во чреве моем.
\par 45 И блаженна Уверовавшая, потому что совершится сказанное Ей от Господа.
\par 46 И сказала Мария: величит душа Моя Господа,
\par 47 и возрадовался дух Мой о Боге, Спасителе Моем,
\par 48 что призрел Он на смирение Рабы Своей, ибо отныне будут ублажать Меня все роды;
\par 49 что сотворил Мне величие Сильный, и свято имя Его;
\par 50 и милость Его в роды родов к боящимся Его;
\par 51 явил силу мышцы Своей; рассеял надменных помышлениями сердца их;
\par 52 низложил сильных с престолов, и вознес смиренных;
\par 53 алчущих исполнил благ, и богатящихся отпустил ни с чем;
\par 54 воспринял Израиля, отрока Своего, воспомянув милость,
\par 55 как говорил отцам нашим, к Аврааму и семени его до века.
\par 56 Пребыла же Мария с нею около трех месяцев, и возвратилась в дом свой.
\par 57 Елисавете же настало время родить, и она родила сына.
\par 58 И услышали соседи и родственники ее, что возвеличил Господь милость Свою над нею, и радовались с нею.
\par 59 В восьмой день пришли обрезать младенца и хотели назвать его, по имени отца его, Захариею.
\par 60 На это мать его сказала: нет, а назвать его Иоанном.
\par 61 И сказали ей: никого нет в родстве твоем, кто назывался бы сим именем.
\par 62 И спрашивали знаками у отца его, как бы он хотел назвать его.
\par 63 Он потребовал дощечку и написал: Иоанн имя ему. И все удивились.
\par 64 И тотчас разрешились уста его и язык его, и он стал говорить, благословляя Бога.
\par 65 И был страх на всех живущих вокруг них; и рассказывали обо всем этом по всей нагорной стране Иудейской.
\par 66 Все слышавшие положили это на сердце своем и говорили: что будет младенец сей? И рука Господня была с ним.
\par 67 И Захария, отец его, исполнился Святаго Духа и пророчествовал, говоря:
\par 68 благословен Господь Бог Израилев, что посетил народ Свой и сотворил избавление ему,
\par 69 и воздвиг рог спасения нам в дому Давида, отрока Своего,
\par 70 как возвестил устами бывших от века святых пророков Своих,
\par 71 что спасет нас от врагов наших и от руки всех ненавидящих нас;
\par 72 сотворит милость с отцами нашими и помянет святой завет Свой,
\par 73 клятву, которою клялся Он Аврааму, отцу нашему, дать нам,
\par 74 небоязненно, по избавлении от руки врагов наших,
\par 75 служить Ему в святости и правде пред Ним, во все дни жизни нашей.
\par 76 И ты, младенец, наречешься пророком Всевышнего, ибо предъидешь пред лицем Господа приготовить пути Ему,
\par 77 дать уразуметь народу Его спасение в прощении грехов их,
\par 78 по благоутробному милосердию Бога нашего, которым посетил нас Восток свыше,
\par 79 просветить сидящих во тьме и тени смертной, направить ноги наши на путь мира.
\par 80 Младенец же возрастал и укреплялся духом, и был в пустынях до дня явления своего Израилю.

\chapter{2}

\par 1 В те дни вышло от кесаря Августа повеление сделать перепись по всей земле.
\par 2 Эта перепись была первая в правление Квириния Сириею.
\par 3 И пошли все записываться, каждый в свой город.
\par 4 Пошел также и Иосиф из Галилеи, из города Назарета, в Иудею, в город Давидов, называемый Вифлеем, потому что он был из дома и рода Давидова,
\par 5 записаться с Мариею, обрученною ему женою, которая была беременна.
\par 6 Когда же они были там, наступило время родить Ей;
\par 7 и родила Сына своего Первенца, и спеленала Его, и положила Его в ясли, потому что не было им места в гостинице.
\par 8 В той стране были на поле пастухи, которые содержали ночную стражу у стада своего.
\par 9 Вдруг предстал им Ангел Господень, и слава Господня осияла их; и убоялись страхом великим.
\par 10 И сказал им Ангел: не бойтесь; я возвещаю вам великую радость, которая будет всем людям:
\par 11 ибо ныне родился вам в городе Давидовом Спаситель, Который есть Христос Господь;
\par 12 и вот вам знак: вы найдете Младенца в пеленах, лежащего в яслях.
\par 13 И внезапно явилось с Ангелом многочисленное воинство небесное, славящее Бога и взывающее:
\par 14 слава в вышних Богу, и на земле мир, в человеках благоволение!
\par 15 Когда Ангелы отошли от них на небо, пастухи сказали друг другу: пойдем в Вифлеем и посмотрим, что там случилось, о чем возвестил нам Господь.
\par 16 И, поспешив, пришли и нашли Марию и Иосифа, и Младенца, лежащего в яслях.
\par 17 Увидев же, рассказали о том, что было возвещено им о Младенце Сем.
\par 18 И все слышавшие дивились тому, что рассказывали им пастухи.
\par 19 А Мария сохраняла все слова сии, слагая в сердце Своем.
\par 20 И возвратились пастухи, славя и хваля Бога за все то, что слышали и видели, как им сказано было.
\par 21 По прошествии восьми дней, когда надлежало обрезать [Младенца], дали Ему имя Иисус, нареченное Ангелом прежде зачатия Его во чреве.
\par 22 А когда исполнились дни очищения их по закону Моисееву, принесли Его в Иерусалим, чтобы представить пред Господа,
\par 23 как предписано в законе Господнем, чтобы всякий младенец мужеского пола, разверзающий ложесна, был посвящен Господу,
\par 24 и чтобы принести в жертву, по реченному в законе Господнем, две горлицы или двух птенцов голубиных.
\par 25 Тогда был в Иерусалиме человек, именем Симеон. Он был муж праведный и благочестивый, чающий утешения Израилева; и Дух Святый был на нем.
\par 26 Ему было предсказано Духом Святым, что он не увидит смерти, доколе не увидит Христа Господня.
\par 27 И пришел он по вдохновению в храм. И, когда родители принесли Младенца Иисуса, чтобы совершить над Ним законный обряд,
\par 28 он взял Его на руки, благословил Бога и сказал:
\par 29 Ныне отпускаешь раба Твоего, Владыко, по слову Твоему, с миром,
\par 30 ибо видели очи мои спасение Твое,
\par 31 которое Ты уготовал пред лицем всех народов,
\par 32 свет к просвещению язычников и славу народа Твоего Израиля.
\par 33 Иосиф же и Матерь Его дивились сказанному о Нем.
\par 34 И благословил их Симеон и сказал Марии, Матери Его: се, лежит Сей на падение и на восстание многих в Израиле и в предмет пререканий, --
\par 35 и Тебе Самой оружие пройдет душу, --да откроются помышления многих сердец.
\par 36 Тут была также Анна пророчица, дочь Фануилова, от колена Асирова, достигшая глубокой старости, прожив с мужем от девства своего семь лет,
\par 37 вдова лет восьмидесяти четырех, которая не отходила от храма, постом и молитвою служа Богу день и ночь.
\par 38 И она в то время, подойдя, славила Господа и говорила о Нем всем, ожидавшим избавления в Иерусалиме.
\par 39 И когда они совершили все по закону Господню, возвратились в Галилею, в город свой Назарет.
\par 40 Младенец же возрастал и укреплялся духом, исполняясь премудрости, и благодать Божия была на Нем.
\par 41 Каждый год родители Его ходили в Иерусалим на праздник Пасхи.
\par 42 И когда Он был двенадцати лет, пришли они также по обычаю в Иерусалим на праздник.
\par 43 Когда же, по окончании дней [праздника], возвращались, остался Отрок Иисус в Иерусалиме; и не заметили того Иосиф и Матерь Его,
\par 44 но думали, что Он идет с другими. Пройдя же дневной путь, стали искать Его между родственниками и знакомыми
\par 45 и, не найдя Его, возвратились в Иерусалим, ища Его.
\par 46 Через три дня нашли Его в храме, сидящего посреди учителей, слушающего их и спрашивающего их;
\par 47 все слушавшие Его дивились разуму и ответам Его.
\par 48 И, увидев Его, удивились; и Матерь Его сказала Ему: Чадо! что Ты сделал с нами? Вот, отец Твой и Я с великою скорбью искали Тебя.
\par 49 Он сказал им: зачем было вам искать Меня? или вы не знали, что Мне должно быть в том, что принадлежит Отцу Моему?
\par 50 Но они не поняли сказанных Им слов.
\par 51 И Он пошел с ними и пришел в Назарет; и был в повиновении у них. И Матерь Его сохраняла все слова сии в сердце Своем.
\par 52 Иисус же преуспевал в премудрости и возрасте и в любви у Бога и человеков.

\chapter{3}

\par 1 В пятнадцатый же год правления Тиверия кесаря, когда Понтий Пилат начальствовал в Иудее, Ирод был четвертовластником в Галилее, Филипп, брат его, четвертовластником в Итурее и Трахонитской области, а Лисаний четвертовластником в Авилинее,
\par 2 при первосвященниках Анне и Каиафе, был глагол Божий к Иоанну, сыну Захарии, в пустыне.
\par 3 И он проходил по всей окрестной стране Иорданской, проповедуя крещение покаяния для прощения грехов,
\par 4 как написано в книге слов пророка Исаии, который говорит: глас вопиющего в пустыне: приготовьте путь Господу, прямыми сделайте стези Ему;
\par 5 всякий дол да наполнится, и всякая гора и холм да понизятся, кривизны выпрямятся и неровные пути сделаются гладкими;
\par 6 и узрит всякая плоть спасение Божие.
\par 7 [Иоанн] приходившему креститься от него народу говорил: порождения ехиднины! кто внушил вам бежать от будущего гнева?
\par 8 Сотворите же достойные плоды покаяния и не думайте говорить в себе: отец у нас Авраам, ибо говорю вам, что Бог может из камней сих воздвигнуть детей Аврааму.
\par 9 Уже и секира при корне дерев лежит: всякое дерево, не приносящее доброго плода, срубают и бросают в огонь.
\par 10 И спрашивал его народ: что же нам делать?
\par 11 Он сказал им в ответ: у кого две одежды, тот дай неимущему, и у кого есть пища, делай то же.
\par 12 Пришли и мытари креститься, и сказали ему: учитель! что нам делать?
\par 13 Он отвечал им: ничего не требуйте более определенного вам.
\par 14 Спрашивали его также и воины: а нам что делать? И сказал им: никого не обижайте, не клевещите, и довольствуйтесь своим жалованьем.
\par 15 Когда же народ был в ожидании, и все помышляли в сердцах своих об Иоанне, не Христос ли он, --
\par 16 Иоанн всем отвечал: я крещу вас водою, но идет Сильнейший меня, у Которого я недостоин развязать ремень обуви; Он будет крестить вас Духом Святым и огнем.
\par 17 Лопата Его в руке Его, и Он очистит гумно Свое и соберет пшеницу в житницу Свою, а солому сожжет огнем неугасимым.
\par 18 Многое и другое благовествовал он народу, поучая его.
\par 19 Ирод же четвертовластник, обличаемый от него за Иродиаду, жену брата своего, и за все, что сделал Ирод худого,
\par 20 прибавил ко всему прочему и то, что заключил Иоанна в темницу.
\par 21 Когда же крестился весь народ, и Иисус, крестившись, молился: отверзлось небо,
\par 22 и Дух Святый нисшел на Него в телесном виде, как голубь, и был глас с небес, глаголющий: Ты Сын Мой Возлюбленный; в Тебе Мое благоволение!
\par 23 Иисус, начиная [Свое служение], был лет тридцати, и был, как думали, Сын Иосифов, Илиев,
\par 24 Матфатов, Левиин, Мелхиев, Ианнаев, Иосифов,
\par 25 Маттафиев, Амосов, Наумов, Еслимов, Наггеев,
\par 26 Маафов, Маттафиев, Семеиев, Иосифов, Иудин,
\par 27 Иоаннанов, Рисаев, Зоровавелев, Салафиилев, Нириев,
\par 28 Мелхиев, Аддиев, Косамов, Елмодамов, Иров,
\par 29 Иосиев, Елиезеров, Иоримов, Матфатов, Левиин,
\par 30 Симеонов, Иудин, Иосифов, Ионанов, Елиакимов,
\par 31 Мелеаев, Маинанов, Маттафаев, Нафанов, Давидов,
\par 32 Иессеев, Овидов, Воозов, Салмонов, Наассонов,
\par 33 Аминадавов, Арамов, Есромов, Фаресов, Иудин,
\par 34 Иаковлев, Исааков, Авраамов, Фаррин, Нахоров,
\par 35 Серухов, Рагавов, Фалеков, Еверов, Салин,
\par 36 Каинанов, Арфаксадов, Симов, Ноев, Ламехов,
\par 37 Мафусалов, Енохов, Иаредов, Малелеилов, Каинанов,
\par 38 Еносов, Сифов, Адамов, Божий.

\chapter{4}

\par 1 Иисус, исполненный Духа Святаго, возвратился от Иордана и поведен был Духом в пустыню.
\par 2 Там сорок дней Он был искушаем от диавола и ничего не ел в эти дни, а по прошествии их напоследок взалкал.
\par 3 И сказал Ему диавол: если Ты Сын Божий, то вели этому камню сделаться хлебом.
\par 4 Иисус сказал ему в ответ: написано, что не хлебом одним будет жить человек, но всяким словом Божиим.
\par 5 И, возведя Его на высокую гору, диавол показал Ему все царства вселенной во мгновение времени,
\par 6 и сказал Ему диавол: Тебе дам власть над всеми сими [царствами] и славу их, ибо она предана мне, и я, кому хочу, даю ее;
\par 7 итак, если Ты поклонишься мне, то все будет Твое.
\par 8 Иисус сказал ему в ответ: отойди от Меня, сатана; написано: Господу Богу твоему поклоняйся, и Ему одному служи.
\par 9 И повел Его в Иерусалим, и поставил Его на крыле храма, и сказал Ему: если Ты Сын Божий, бросься отсюда вниз,
\par 10 ибо написано: Ангелам Своим заповедает о Тебе сохранить Тебя;
\par 11 и на руках понесут Тебя, да не преткнешься о камень ногою Твоею.
\par 12 Иисус сказал ему в ответ: сказано: не искушай Господа Бога твоего.
\par 13 И, окончив все искушение, диавол отошел от Него до времени.
\par 14 И возвратился Иисус в силе духа в Галилею; и разнеслась молва о Нем по всей окрестной стране.
\par 15 Он учил в синагогах их, и от всех был прославляем.
\par 16 И пришел в Назарет, где был воспитан, и вошел, по обыкновению Своему, в день субботний в синагогу, и встал читать.
\par 17 Ему подали книгу пророка Исаии; и Он, раскрыв книгу, нашел место, где было написано:
\par 18 Дух Господень на Мне; ибо Он помазал Меня благовествовать нищим, и послал Меня исцелять сокрушенных сердцем, проповедывать пленным освобождение, слепым прозрение, отпустить измученных на свободу,
\par 19 проповедывать лето Господне благоприятное.
\par 20 И, закрыв книгу и отдав служителю, сел; и глаза всех в синагоге были устремлены на Него.
\par 21 И Он начал говорить им: ныне исполнилось писание сие, слышанное вами.
\par 22 И все засвидетельствовали Ему это, и дивились словам благодати, исходившим из уст Его, и говорили: не Иосифов ли это сын?
\par 23 Он сказал им: конечно, вы скажете Мне присловие: врач! исцели Самого Себя; сделай и здесь, в Твоем отечестве, то, что, мы слышали, было в Капернауме.
\par 24 И сказал: истинно говорю вам: никакой пророк не принимается в своем отечестве.
\par 25 Поистине говорю вам: много вдов было в Израиле во дни Илии, когда заключено было небо три года и шесть месяцев, так что сделался большой голод по всей земле,
\par 26 и ни к одной из них не был послан Илия, а только ко вдове в Сарепту Сидонскую;
\par 27 много также было прокаженных в Израиле при пророке Елисее, и ни один из них не очистился, кроме Неемана Сириянина.
\par 28 Услышав это, все в синагоге исполнились ярости
\par 29 и, встав, выгнали Его вон из города и повели на вершину горы, на которой город их был построен, чтобы свергнуть Его;
\par 30 но Он, пройдя посреди них, удалился.
\par 31 И пришел в Капернаум, город Галилейский, и учил их в дни субботние.
\par 32 И дивились учению Его, ибо слово Его было со властью.
\par 33 Был в синагоге человек, имевший нечистого духа бесовского, и он закричал громким голосом:
\par 34 оставь; что Тебе до нас, Иисус Назарянин? Ты пришел погубить нас; знаю Тебя, кто Ты, Святый Божий.
\par 35 Иисус запретил ему, сказав: замолчи и выйди из него. И бес, повергнув его посреди [синагоги], вышел из него, нимало не повредив ему.
\par 36 И напал на всех ужас, и рассуждали между собою: что это значит, что Он со властью и силою повелевает нечистым духам, и они выходят?
\par 37 И разнесся слух о Нем по всем окрестным местам.
\par 38 Выйдя из синагоги, Он вошел в дом Симона; теща же Симонова была одержима сильною горячкою; и просили Его о ней.
\par 39 Подойдя к ней, Он запретил горячке; и оставила ее. Она тотчас встала и служила им.
\par 40 При захождении же солнца все, имевшие больных различными болезнями, приводили их к Нему и Он, возлагая на каждого из них руки, исцелял их.
\par 41 Выходили также и бесы из многих с криком и говорили: Ты Христос, Сын Божий. А Он запрещал им сказывать, что они знают, что Он Христос.
\par 42 Когда же настал день, Он, выйдя [из дома], пошел в пустынное место, и народ искал Его и, придя к Нему, удерживал Его, чтобы не уходил от них.
\par 43 Но Он сказал им: и другим городам благовествовать Я должен Царствие Божие, ибо на то Я послан.
\par 44 И проповедывал в синагогах галилейских.

\chapter{5}

\par 1 Однажды, когда народ теснился к Нему, чтобы слышать слово Божие, а Он стоял у озера Геннисаретского,
\par 2 увидел Он две лодки, стоящие на озере; а рыболовы, выйдя из них, вымывали сети.
\par 3 Войдя в одну лодку, которая была Симонова, Он просил его отплыть несколько от берега и, сев, учил народ из лодки.
\par 4 Когда же перестал учить, сказал Симону: отплыви на глубину и закиньте сети свои для лова.
\par 5 Симон сказал Ему в ответ: Наставник! мы трудились всю ночь и ничего не поймали, но по слову Твоему закину сеть.
\par 6 Сделав это, они поймали великое множество рыбы, и даже сеть у них прорывалась.
\par 7 И дали знак товарищам, находившимся на другой лодке, чтобы пришли помочь им; и пришли, и наполнили обе лодки, так что они начинали тонуть.
\par 8 Увидев это, Симон Петр припал к коленям Иисуса и сказал: выйди от меня, Господи! потому что я человек грешный.
\par 9 Ибо ужас объял его и всех, бывших с ним, от этого лова рыб, ими пойманных;
\par 10 также и Иакова и Иоанна, сыновей Зеведеевых, бывших товарищами Симону. И сказал Симону Иисус: не бойся; отныне будешь ловить человеков.
\par 11 И, вытащив обе лодки на берег, оставили все и последовали за Ним.
\par 12 Когда Иисус был в одном городе, пришел человек весь в проказе и, увидев Иисуса, пал ниц, умоляя Его и говоря: Господи! если хочешь, можешь меня очистить.
\par 13 Он простер руку, прикоснулся к нему и сказал: хочу, очистись. И тотчас проказа сошла с него.
\par 14 И Он повелел ему никому не сказывать, а пойти показаться священнику и принести [жертву] за очищение свое, как повелел Моисей, во свидетельство им.
\par 15 Но тем более распространялась молва о Нем, и великое множество народа стекалось к Нему слушать и врачеваться у Него от болезней своих.
\par 16 Но Он уходил в пустынные места и молился.
\par 17 В один день, когда Он учил, и сидели тут фарисеи и законоучители, пришедшие из всех мест Галилеи и Иудеи и из Иерусалима, и сила Господня являлась в исцелении [больных], --
\par 18 вот, принесли некоторые на постели человека, который был расслаблен, и старались внести его [в дом] и положить перед Иисусом;
\par 19 и, не найдя, где пронести его за многолюдством, влезли на верх дома и сквозь кровлю спустили его с постелью на средину пред Иисуса.
\par 20 И Он, видя веру их, сказал человеку тому: прощаются тебе грехи твои.
\par 21 Книжники и фарисеи начали рассуждать, говоря: кто это, который богохульствует? кто может прощать грехи, кроме одного Бога?
\par 22 Иисус, уразумев помышления их, сказал им в ответ: что вы помышляете в сердцах ваших?
\par 23 Что легче сказать: прощаются тебе грехи твои, или сказать: встань и ходи?
\par 24 Но чтобы вы знали, что Сын Человеческий имеет власть на земле прощать грехи, --сказал Он расслабленному: тебе говорю: встань, возьми постель твою и иди в дом твой.
\par 25 И он тотчас встал перед ними, взял, на чем лежал, и пошел в дом свой, славя Бога.
\par 26 И ужас объял всех, и славили Бога и, быв исполнены страха, говорили: чудные дела видели мы ныне.
\par 27 После сего [Иисус] вышел и увидел мытаря, именем Левия, сидящего у сбора пошлин, и говорит ему: следуй за Мною.
\par 28 И он, оставив все, встал и последовал за Ним.
\par 29 И сделал для Него Левий в доме своем большое угощение; и там было множество мытарей и других, которые возлежали с ними.
\par 30 Книжники же и фарисеи роптали и говорили ученикам Его: зачем вы едите и пьете с мытарями и грешниками?
\par 31 Иисус же сказал им в ответ: не здоровые имеют нужду во враче, но больные;
\par 32 Я пришел призвать не праведников, а грешников к покаянию.
\par 33 Они же сказали Ему: почему ученики Иоанновы постятся часто и молитвы творят, также и фарисейские, а Твои едят и пьют?
\par 34 Он сказал им: можете ли заставить сынов чертога брачного поститься, когда с ними жених?
\par 35 Но придут дни, когда отнимется у них жених, и тогда будут поститься в те дни.
\par 36 При сем сказал им притчу: никто не приставляет заплаты к ветхой одежде, отодрав от новой одежды; а иначе и новую раздерет, и к старой не подойдет заплата от новой.
\par 37 И никто не вливает молодого вина в мехи ветхие; а иначе молодое вино прорвет мехи, и само вытечет, и мехи пропадут;
\par 38 но молодое вино должно вливать в мехи новые; тогда сбережется и то и другое.
\par 39 И никто, пив старое [вино], не захочет тотчас молодого, ибо говорит: старое лучше.

\chapter{6}

\par 1 В субботу, первую по втором дне Пасхи, случилось Ему проходить засеянными полями, и ученики Его срывали колосья и ели, растирая руками.
\par 2 Некоторые же из фарисеев сказали им: зачем вы делаете то, чего не должно делать в субботы?
\par 3 Иисус сказал им в ответ: разве вы не читали, что сделал Давид, когда взалкал сам и бывшие с ним?
\par 4 Как он вошел в дом Божий, взял хлебы предложения, которых не должно было есть никому, кроме одних священников, и ел, и дал бывшим с ним?
\par 5 И сказал им: Сын Человеческий есть господин и субботы.
\par 6 Случилось же и в другую субботу войти Ему в синагогу и учить. Там был человек, у которого правая рука была сухая.
\par 7 Книжники же и фарисеи наблюдали за Ним, не исцелит ли в субботу, чтобы найти обвинение против Него.
\par 8 Но Он, зная помышления их, сказал человеку, имеющему сухую руку: встань и выступи на средину. И он встал и выступил.
\par 9 Тогда сказал им Иисус: спрошу Я вас: что должно делать в субботу? добро, или зло? спасти душу, или погубить? Они молчали.
\par 10 И, посмотрев на всех их, сказал тому человеку: протяни руку твою. Он так и сделал; и стала рука его здорова, как другая.
\par 11 Они же пришли в бешенство и говорили между собою, что бы им сделать с Иисусом.
\par 12 В те дни взошел Он на гору помолиться и пробыл всю ночь в молитве к Богу.
\par 13 Когда же настал день, призвал учеников Своих и избрал из них двенадцать, которых и наименовал Апостолами:
\par 14 Симона, которого и назвал Петром, и Андрея, брата его, Иакова и Иоанна, Филиппа и Варфоломея,
\par 15 Матфея и Фому, Иакова Алфеева и Симона, прозываемого Зилотом,
\par 16 Иуду Иаковлева и Иуду Искариота, который потом сделался предателем.
\par 17 И, сойдя с ними, стал Он на ровном месте, и множество учеников Его, и много народа из всей Иудеи и Иерусалима и приморских мест Тирских и Сидонских,
\par 18 которые пришли послушать Его и исцелиться от болезней своих, также и страждущие от нечистых духов; и исцелялись.
\par 19 И весь народ искал прикасаться к Нему, потому что от Него исходила сила и исцеляла всех.
\par 20 И Он, возведя очи Свои на учеников Своих, говорил: Блаженны нищие духом, ибо ваше есть Царствие Божие.
\par 21 Блаженны алчущие ныне, ибо насытитесь. Блаженны плачущие ныне, ибо воссмеетесь.
\par 22 Блаженны вы, когда возненавидят вас люди и когда отлучат вас, и будут поносить, и пронесут имя ваше, как бесчестное, за Сына Человеческого.
\par 23 Возрадуйтесь в тот день и возвеселитесь, ибо велика вам награда на небесах. Так поступали с пророками отцы их.
\par 24 Напротив, горе вам, богатые! ибо вы уже получили свое утешение.
\par 25 Горе вам, пресыщенные ныне! ибо взалчете. Горе вам, смеющиеся ныне! ибо восплачете и возрыдаете.
\par 26 Горе вам, когда все люди будут говорить о вас хорошо! ибо так поступали с лжепророками отцы их.
\par 27 Но вам, слушающим, говорю: любите врагов ваших, благотворите ненавидящим вас,
\par 28 благословляйте проклинающих вас и молитесь за обижающих вас.
\par 29 Ударившему тебя по щеке подставь и другую, и отнимающему у тебя верхнюю одежду не препятствуй взять и рубашку.
\par 30 Всякому, просящему у тебя, давай, и от взявшего твое не требуй назад.
\par 31 И как хотите, чтобы с вами поступали люди, так и вы поступайте с ними.
\par 32 И если любите любящих вас, какая вам за то благодарность? ибо и грешники любящих их любят.
\par 33 И если делаете добро тем, которые вам делают добро, какая вам за то благодарность? ибо и грешники то же делают.
\par 34 И если взаймы даете тем, от которых надеетесь получить обратно, какая вам за то благодарность? ибо и грешники дают взаймы грешникам, чтобы получить обратно столько же.
\par 35 Но вы любите врагов ваших, и благотворите, и взаймы давайте, не ожидая ничего; и будет вам награда великая, и будете сынами Всевышнего; ибо Он благ и к неблагодарным и злым.
\par 36 Итак, будьте милосерды, как и Отец ваш милосерд.
\par 37 Не судите, и не будете судимы; не осуждайте, и не будете осуждены; прощайте, и прощены будете;
\par 38 давайте, и дастся вам: мерою доброю, утрясенною, нагнетенною и переполненною отсыплют вам в лоно ваше; ибо, какою мерою мерите, такою же отмерится и вам.
\par 39 Сказал также им притчу: может ли слепой водить слепого? не оба ли упадут в яму?
\par 40 Ученик не бывает выше своего учителя; но, и усовершенствовавшись, будет всякий, как учитель его.
\par 41 Что ты смотришь на сучок в глазе брата твоего, а бревна в твоем глазе не чувствуешь?
\par 42 Или, как можешь сказать брату твоему: брат! дай, я выну сучок из глаза твоего, когда сам не видишь бревна в твоем глазе? Лицемер! вынь прежде бревно из твоего глаза, и тогда увидишь, как вынуть сучок из глаза брата твоего.
\par 43 Нет доброго дерева, которое приносило бы худой плод; и нет худого дерева, которое приносило бы плод добрый,
\par 44 ибо всякое дерево познается по плоду своему, потому что не собирают смокв с терновника и не снимают винограда с кустарника.
\par 45 Добрый человек из доброго сокровища сердца своего выносит доброе, а злой человек из злого сокровища сердца своего выносит злое, ибо от избытка сердца говорят уста его.
\par 46 Что вы зовете Меня: Господи! Господи! --и не делаете того, что Я говорю?
\par 47 Всякий, приходящий ко Мне и слушающий слова Мои и исполняющий их, скажу вам, кому подобен.
\par 48 Он подобен человеку, строящему дом, который копал, углубился и положил основание на камне; почему, когда случилось наводнение и вода наперла на этот дом, то не могла поколебать его, потому что он основан был на камне.
\par 49 А слушающий и неисполняющий подобен человеку, построившему дом на земле без основания, который, когда наперла на него вода, тотчас обрушился; и разрушение дома сего было великое.

\chapter{7}

\par 1 Когда Он окончил все слова Свои к слушавшему народу, то вошел в Капернаум.
\par 2 У одного сотника слуга, которым он дорожил, был болен при смерти.
\par 3 Услышав об Иисусе, он послал к Нему Иудейских старейшин просить Его, чтобы пришел исцелить слугу его.
\par 4 И они, придя к Иисусу, просили Его убедительно, говоря: он достоин, чтобы Ты сделал для него это,
\par 5 ибо он любит народ наш и построил нам синагогу.
\par 6 Иисус пошел с ними. И когда Он недалеко уже был от дома, сотник прислал к Нему друзей сказать Ему: не трудись, Господи! ибо я недостоин, чтобы Ты вошел под кров мой;
\par 7 потому и себя самого не почел я достойным придти к Тебе; но скажи слово, и выздоровеет слуга мой.
\par 8 Ибо я и подвластный человек, но, имея у себя в подчинении воинов, говорю одному: пойди, и идет; и другому: приди, и приходит; и слуге моему: сделай то, и делает.
\par 9 Услышав сие, Иисус удивился ему и, обратившись, сказал идущему за Ним народу: сказываю вам, что и в Израиле не нашел Я такой веры.
\par 10 Посланные, возвратившись в дом, нашли больного слугу выздоровевшим.
\par 11 После сего Иисус пошел в город, называемый Наин; и с Ним шли многие из учеников Его и множество народа.
\par 12 Когда же Он приблизился к городским воротам, тут выносили умершего, единственного сына у матери, а она была вдова; и много народа шло с нею из города.
\par 13 Увидев ее, Господь сжалился над нею и сказал ей: не плачь.
\par 14 И, подойдя, прикоснулся к одру; несшие остановились, и Он сказал: юноша! тебе говорю, встань!
\par 15 Мертвый, поднявшись, сел и стал говорить; и отдал его [Иисус] матери его.
\par 16 И всех объял страх, и славили Бога, говоря: великий пророк восстал между нами, и Бог посетил народ Свой.
\par 17 Такое мнение о Нем распространилось по всей Иудее и по всей окрестности.
\par 18 И возвестили Иоанну ученики его о всем том.
\par 19 Иоанн, призвав двоих из учеников своих, послал к Иисусу спросить: Ты ли Тот, Который должен придти, или ожидать нам другого?
\par 20 Они, придя к [Иисусу], сказали: Иоанн Креститель послал нас к Тебе спросить: Ты ли Тот, Которому должно придти, или другого ожидать нам?
\par 21 А в это время Он многих исцелил от болезней и недугов и от злых духов, и многим слепым даровал зрение.
\par 22 И сказал им Иисус в ответ: пойдите, скажите Иоанну, что вы видели и слышали: слепые прозревают, хромые ходят, прокаженные очищаются, глухие слышат, мертвые воскресают, нищие благовествуют;
\par 23 и блажен, кто не соблазнится о Мне!
\par 24 По отшествии же посланных Иоанном, начал говорить к народу об Иоанне: что смотреть ходили вы в пустыню? трость ли, ветром колеблемую?
\par 25 Что же смотреть ходили вы? человека ли, одетого в мягкие одежды? Но одевающиеся пышно и роскошно живущие находятся при дворах царских.
\par 26 Что же смотреть ходили вы? пророка ли? Да, говорю вам, и больше пророка.
\par 27 Сей есть, о котором написано: вот, Я посылаю Ангела Моего пред лицем Твоим, который приготовит путь Твой пред Тобою.
\par 28 Ибо говорю вам: из рожденных женами нет ни одного пророка больше Иоанна Крестителя; но меньший в Царствии Божием больше его.
\par 29 И весь народ, слушавший [Его], и мытари воздали славу Богу, крестившись крещением Иоанновым;
\par 30 а фарисеи и законники отвергли волю Божию о себе, не крестившись от него.
\par 31 Тогда Господь сказал: с кем сравню людей рода сего? и кому они подобны?
\par 32 Они подобны детям, которые сидят на улице, кличут друг друга и говорят: мы играли вам на свирели, и вы не плясали; мы пели вам плачевные песни, и вы не плакали.
\par 33 Ибо пришел Иоанн Креститель: ни хлеба не ест, ни вина не пьет; и говорите: в нем бес.
\par 34 Пришел Сын Человеческий: ест и пьет; и говорите: вот человек, который любит есть и пить вино, друг мытарям и грешникам.
\par 35 И оправдана премудрость всеми чадами ее.
\par 36 Некто из фарисеев просил Его вкусить с ним пищи; и Он, войдя в дом фарисея, возлег.
\par 37 И вот, женщина того города, которая была грешница, узнав, что Он возлежит в доме фарисея, принесла алавастровый сосуд с миром
\par 38 и, став позади у ног Его и плача, начала обливать ноги Его слезами и отирать волосами головы своей, и целовала ноги Его, и мазала миром.
\par 39 Видя это, фарисей, пригласивший Его, сказал сам в себе: если бы Он был пророк, то знал бы, кто и какая женщина прикасается к Нему, ибо она грешница.
\par 40 Обратившись к нему, Иисус сказал: Симон! Я имею нечто сказать тебе. Он говорит: скажи, Учитель.
\par 41 Иисус сказал: у одного заимодавца было два должника: один должен был пятьсот динариев, а другой пятьдесят,
\par 42 но как они не имели чем заплатить, он простил обоим. Скажи же, который из них более возлюбит его?
\par 43 Симон отвечал: думаю, тот, которому более простил. Он сказал ему: правильно ты рассудил.
\par 44 И, обратившись к женщине, сказал Симону: видишь ли ты эту женщину? Я пришел в дом твой, и ты воды Мне на ноги не дал, а она слезами облила Мне ноги и волосами головы своей отерла;
\par 45 ты целования Мне не дал, а она, с тех пор как Я пришел, не перестает целовать у Меня ноги;
\par 46 ты головы Мне маслом не помазал, а она миром помазала Мне ноги.
\par 47 А потому сказываю тебе: прощаются грехи ее многие за то, что она возлюбила много, а кому мало прощается, тот мало любит.
\par 48 Ей же сказал: прощаются тебе грехи.
\par 49 И возлежавшие с Ним начали говорить про себя: кто это, что и грехи прощает?
\par 50 Он же сказал женщине: вера твоя спасла тебя, иди с миром.

\chapter{8}

\par 1 После сего Он проходил по городам и селениям, проповедуя и благовествуя Царствие Божие, и с Ним двенадцать,
\par 2 и некоторые женщины, которых Он исцелил от злых духов и болезней: Мария, называемая Магдалиною, из которой вышли семь бесов,
\par 3 и Иоанна, жена Хузы, домоправителя Иродова, и Сусанна, и многие другие, которые служили Ему имением своим.
\par 4 Когда же собралось множество народа, и из всех городов жители сходились к Нему, Он начал говорить притчею:
\par 5 вышел сеятель сеять семя свое, и когда он сеял, иное упало при дороге и было потоптано, и птицы небесные поклевали его;
\par 6 а иное упало на камень и, взойдя, засохло, потому что не имело влаги;
\par 7 а иное упало между тернием, и выросло терние и заглушило его;
\par 8 а иное упало на добрую землю и, взойдя, принесло плод сторичный. Сказав сие, возгласил: кто имеет уши слышать, да слышит!
\par 9 Ученики же Его спросили у Него: что бы значила притча сия?
\par 10 Он сказал: вам дано знать тайны Царствия Божия, а прочим в притчах, так что они видя не видят и слыша не разумеют.
\par 11 Вот что значит притча сия: семя есть слово Божие;
\par 12 а упавшее при пути, это суть слушающие, к которым потом приходит диавол и уносит слово из сердца их, чтобы они не уверовали и не спаслись;
\par 13 а упавшее на камень, это те, которые, когда услышат слово, с радостью принимают, но которые не имеют корня, и временем веруют, а во время искушения отпадают;
\par 14 а упавшее в терние, это те, которые слушают слово, но, отходя, заботами, богатством и наслаждениями житейскими подавляются и не приносят плода;
\par 15 а упавшее на добрую землю, это те, которые, услышав слово, хранят его в добром и чистом сердце и приносят плод в терпении. Сказав это, Он возгласил: кто имеет уши слышать, да слышит!
\par 16 Никто, зажегши свечу, не покрывает ее сосудом, или не ставит под кровать, а ставит на подсвечник, чтобы входящие видели свет.
\par 17 Ибо нет ничего тайного, что не сделалось бы явным, ни сокровенного, что не сделалось бы известным и не обнаружилось бы.
\par 18 Итак, наблюдайте, как вы слушаете: ибо, кто имеет, тому дано будет, а кто не имеет, у того отнимется и то, что он думает иметь.
\par 19 И пришли к Нему Матерь и братья Его, и не могли подойти к Нему по причине народа.
\par 20 И дали знать Ему: Матерь и братья Твои стоят вне, желая видеть Тебя.
\par 21 Он сказал им в ответ: матерь Моя и братья Мои суть слушающие слово Божие и исполняющие его.
\par 22 В один день Он вошел с учениками Своими в лодку и сказал им: переправимся на ту сторону озера. И отправились.
\par 23 Во время плавания их Он заснул. На озере поднялся бурный ветер, и заливало их [волнами], и они были в опасности.
\par 24 И, подойдя, разбудили Его и сказали: Наставник! Наставник! погибаем. Но Он, встав, запретил ветру и волнению воды; и перестали, и сделалась тишина.
\par 25 Тогда Он сказал им: где вера ваша? Они же в страхе и удивлении говорили друг другу: кто же это, что и ветрам повелевает и воде, и повинуются Ему?
\par 26 И приплыли в страну Гадаринскую, лежащую против Галилеи.
\par 27 Когда же вышел Он на берег, встретил Его один человек из города, одержимый бесами с давнего времени, и в одежду не одевавшийся, и живший не в доме, а в гробах.
\par 28 Он, увидев Иисуса, вскричал, пал пред Ним и громким голосом сказал: что Тебе до меня, Иисус, Сын Бога Всевышнего? умоляю Тебя, не мучь меня.
\par 29 Ибо [Иисус] повелел нечистому духу выйти из сего человека, потому что он долгое время мучил его, так что его связывали цепями и узами, сберегая его; но он разрывал узы и был гоним бесом в пустыни.
\par 30 Иисус спросил его: как тебе имя? Он сказал: легион, --потому что много бесов вошло в него.
\par 31 И они просили Иисуса, чтобы не повелел им идти в бездну.
\par 32 Тут же на горе паслось большое стадо свиней; и [бесы] просили Его, чтобы позволил им войти в них. Он позволил им.
\par 33 Бесы, выйдя из человека, вошли в свиней, и бросилось стадо с крутизны в озеро и потонуло.
\par 34 Пастухи, видя происшедшее, побежали и рассказали в городе и в селениях.
\par 35 И вышли видеть происшедшее; и, придя к Иисусу, нашли человека, из которого вышли бесы, сидящего у ног Иисуса, одетого и в здравом уме; и ужаснулись.
\par 36 Видевшие же рассказали им, как исцелился бесновавшийся.
\par 37 И просил Его весь народ Гадаринской окрестности удалиться от них, потому что они объяты были великим страхом. Он вошел в лодку и возвратился.
\par 38 Человек же, из которого вышли бесы, просил Его, чтобы быть с Ним. Но Иисус отпустил его, сказав:
\par 39 возвратись в дом твой и расскажи, что сотворил тебе Бог. Он пошел и проповедывал по всему городу, что сотворил ему Иисус.
\par 40 Когда же возвратился Иисус, народ принял Его, потому что все ожидали Его.
\par 41 И вот, пришел человек, именем Иаир, который был начальником синагоги; и, пав к ногам Иисуса, просил Его войти к нему в дом,
\par 42 потому что у него была одна дочь, лет двенадцати, и та была при смерти. Когда же Он шел, народ теснил Его.
\par 43 И женщина, страдавшая кровотечением двенадцать лет, которая, издержав на врачей все имение, ни одним не могла быть вылечена,
\par 44 подойдя сзади, коснулась края одежды Его; и тотчас течение крови у ней остановилось.
\par 45 И сказал Иисус: кто прикоснулся ко Мне? Когда же все отрицались, Петр сказал и бывшие с Ним: Наставник! народ окружает Тебя и теснит, --и Ты говоришь: кто прикоснулся ко Мне?
\par 46 Но Иисус сказал: прикоснулся ко Мне некто, ибо Я чувствовал силу, исшедшую из Меня.
\par 47 Женщина, видя, что она не утаилась, с трепетом подошла и, пав пред Ним, объявила Ему перед всем народом, по какой причине прикоснулась к Нему и как тотчас исцелилась.
\par 48 Он сказал ей: дерзай, дщерь! вера твоя спасла тебя; иди с миром.
\par 49 Когда Он еще говорил это, приходит некто из дома начальника синагоги и говорит ему: дочь твоя умерла; не утруждай Учителя.
\par 50 Но Иисус, услышав это, сказал ему: не бойся, только веруй, и спасена будет.
\par 51 Придя же в дом, не позволил войти никому, кроме Петра, Иоанна и Иакова, и отца девицы, и матери.
\par 52 Все плакали и рыдали о ней. Но Он сказал: не плачьте; она не умерла, но спит.
\par 53 И смеялись над Ним, зная, что она умерла.
\par 54 Он же, выслав всех вон и взяв ее за руку, возгласил: девица! встань.
\par 55 И возвратился дух ее; она тотчас встала, и Он велел дать ей есть.
\par 56 И удивились родители ее. Он же повелел им не сказывать никому о происшедшем.

\chapter{9}

\par 1 Созвав же двенадцать, дал силу и власть над всеми бесами и врачевать от болезней,
\par 2 и послал их проповедывать Царствие Божие и исцелять больных.
\par 3 И сказал им: ничего не берите на дорогу: ни посоха, ни сумы, ни хлеба, ни серебра, и не имейте по две одежды;
\par 4 и в какой дом войдете, там оставайтесь и оттуда отправляйтесь [в] [путь].
\par 5 А если где не примут вас, то, выходя из того города, отрясите и прах от ног ваших во свидетельство на них.
\par 6 Они пошли и проходили по селениям, благовествуя и исцеляя повсюду.
\par 7 Услышал Ирод четвертовластник о всем, что делал [Иисус], и недоумевал: ибо одни говорили, что это Иоанн восстал из мертвых;
\par 8 другие, что Илия явился, а иные, что один из древних пророков воскрес.
\par 9 И сказал Ирод: Иоанна я обезглавил; кто же Этот, о Котором я слышу такое? И искал увидеть Его.
\par 10 Апостолы, возвратившись, рассказали Ему, что они сделали; и Он, взяв их с Собою, удалился особо в пустое место, близ города, называемого Вифсаидою.
\par 11 Но народ, узнав, пошел за Ним; и Он, приняв их, беседовал с ними о Царствии Божием и требовавших исцеления исцелял.
\par 12 День же начал склоняться к вечеру. И, приступив к Нему, двенадцать говорили Ему: отпусти народ, чтобы они пошли в окрестные селения и деревни ночевать и достали пищи; потому что мы здесь в пустом месте.
\par 13 Но Он сказал им: вы дайте им есть. Они сказали: у нас нет более пяти хлебов и двух рыб; разве нам пойти купить пищи для всех сих людей?
\par 14 Ибо их было около пяти тысяч человек. Но Он сказал ученикам Своим: рассадите их рядами по пятидесяти.
\par 15 И сделали так, и рассадили всех.
\par 16 Он же, взяв пять хлебов и две рыбы и воззрев на небо, благословил их, преломил и дал ученикам, чтобы раздать народу.
\par 17 И ели, и насытились все; и оставшихся у них кусков набрано двенадцать коробов.
\par 18 В одно время, когда Он молился в уединенном месте, и ученики были с Ним, Он спросил их: за кого почитает Меня народ?
\par 19 Они сказали в ответ: за Иоанна Крестителя, а иные за Илию; другие же [говорят], что один из древних пророков воскрес.
\par 20 Он же спросил их: а вы за кого почитаете Меня? Отвечал Петр: за Христа Божия.
\par 21 Но Он строго приказал им никому не говорить о сем,
\par 22 сказав, что Сыну Человеческому должно много пострадать, и быть отвержену старейшинами, первосвященниками и книжниками, и быть убиту, и в третий день воскреснуть.
\par 23 Ко всем же сказал: если кто хочет идти за Мною, отвергнись себя, и возьми крест свой, и следуй за Мною.
\par 24 Ибо кто хочет душу свою сберечь, тот потеряет ее; а кто потеряет душу свою ради Меня, тот сбережет ее.
\par 25 Ибо что пользы человеку приобрести весь мир, а себя самого погубить или повредить себе?
\par 26 Ибо кто постыдится Меня и Моих слов, того Сын Человеческий постыдится, когда приидет во славе Своей и Отца и святых Ангелов.
\par 27 Говорю же вам истинно: есть некоторые из стоящих здесь, которые не вкусят смерти, как уже увидят Царствие Божие.
\par 28 После сих слов, дней через восемь, взяв Петра, Иоанна и Иакова, взошел Он на гору помолиться.
\par 29 И когда молился, вид лица Его изменился, и одежда Его сделалась белою, блистающею.
\par 30 И вот, два мужа беседовали с Ним, которые были Моисей и Илия;
\par 31 явившись во славе, они говорили об исходе Его, который Ему надлежало совершить в Иерусалиме.
\par 32 Петр же и бывшие с ним отягчены были сном; но, пробудившись, увидели славу Его и двух мужей, стоявших с Ним.
\par 33 И когда они отходили от Него, сказал Петр Иисусу: Наставник! хорошо нам здесь быть; сделаем три кущи: одну Тебе, одну Моисею и одну Илии, --не зная, что говорил.
\par 34 Когда же он говорил это, явилось облако и осенило их; и устрашились, когда вошли в облако.
\par 35 И был из облака глас, глаголющий: Сей есть Сын Мой Возлюбленный, Его слушайте.
\par 36 Когда был глас сей, остался Иисус один. И они умолчали, и никому не говорили в те дни о том, что видели.
\par 37 В следующий же день, когда они сошли с горы, встретило Его много народа.
\par 38 Вдруг некто из народа воскликнул: Учитель! умоляю Тебя взглянуть на сына моего, он один у меня:
\par 39 его схватывает дух, и он внезапно вскрикивает, и терзает его, так что он испускает пену; и насилу отступает от него, измучив его.
\par 40 Я просил учеников Твоих изгнать его, и они не могли.
\par 41 Иисус же, отвечая, сказал: о, род неверный и развращенный! доколе буду с вами и буду терпеть вас? приведи сюда сына твоего.
\par 42 Когда же тот еще шел, бес поверг его и стал бить; но Иисус запретил нечистому духу, и исцелил отрока, и отдал его отцу его.
\par 43 И все удивлялись величию Божию. Когда же все дивились всему, что творил Иисус, Он сказал ученикам Своим:
\par 44 вложите вы себе в уши слова сии: Сын Человеческий будет предан в руки человеческие.
\par 45 Но они не поняли слова сего, и оно было закрыто от них, так что они не постигли его, а спросить Его о сем слове боялись.
\par 46 Пришла же им мысль: кто бы из них был больше?
\par 47 Иисус же, видя помышление сердца их, взяв дитя, поставил его пред Собою
\par 48 и сказал им: кто примет сие дитя во имя Мое, тот Меня принимает; а кто примет Меня, тот принимает Пославшего Меня; ибо кто из вас меньше всех, тот будет велик.
\par 49 При сем Иоанн сказал: Наставник! мы видели человека, именем Твоим изгоняющего бесов, и запретили ему, потому что он не ходит с нами.
\par 50 Иисус сказал ему: не запрещайте, ибо кто не против вас, тот за вас.
\par 51 Когда же приближались дни взятия Его [от мира], Он восхотел идти в Иерусалим;
\par 52 и послал вестников пред лицем Своим; и они пошли и вошли в селение Самарянское; чтобы приготовить для Него;
\par 53 но [там] не приняли Его, потому что Он имел вид путешествующего в Иерусалим.
\par 54 Видя то, ученики Его, Иаков и Иоанн, сказали: Господи! хочешь ли, мы скажем, чтобы огонь сошел с неба и истребил их, как и Илия сделал?
\par 55 Но Он, обратившись к ним, запретил им и сказал: не знаете, какого вы духа;
\par 56 ибо Сын Человеческий пришел не губить души человеческие, а спасать. И пошли в другое селение.
\par 57 Случилось, что когда они были в пути, некто сказал Ему: Господи! я пойду за Тобою, куда бы Ты ни пошел.
\par 58 Иисус сказал ему: лисицы имеют норы, и птицы небесные--гнезда; а Сын Человеческий не имеет, где приклонить голову.
\par 59 А другому сказал: следуй за Мною. Тот сказал: Господи! позволь мне прежде пойти и похоронить отца моего.
\par 60 Но Иисус сказал ему: предоставь мертвым погребать своих мертвецов, а ты иди, благовествуй Царствие Божие.
\par 61 Еще другой сказал: я пойду за Тобою, Господи! но прежде позволь мне проститься с домашними моими.
\par 62 Но Иисус сказал ему: никто, возложивший руку свою на плуг и озирающийся назад, не благонадежен для Царствия Божия.

\chapter{10}

\par 1 После сего избрал Господь и других семьдесят [учеников], и послал их по два пред лицем Своим во всякий город и место, куда Сам хотел идти,
\par 2 и сказал им: жатвы много, а делателей мало; итак, молите Господина жатвы, чтобы выслал делателей на жатву Свою.
\par 3 Идите! Я посылаю вас, как агнцев среди волков.
\par 4 Не берите ни мешка, ни сумы, ни обуви, и никого на дороге не приветствуйте.
\par 5 В какой дом войдете, сперва говорите: мир дому сему;
\par 6 и если будет там сын мира, то почиет на нем мир ваш, а если нет, то к вам возвратится.
\par 7 В доме же том оставайтесь, ешьте и пейте, что у них есть, ибо трудящийся достоин награды за труды свои; не переходите из дома в дом.
\par 8 И если придете в какой город и примут вас, ешьте, что вам предложат,
\par 9 и исцеляйте находящихся в нем больных, и говорите им: приблизилось к вам Царствие Божие.
\par 10 Если же придете в какой город и не примут вас, то, выйдя на улицу, скажите:
\par 11 и прах, прилипший к нам от вашего города, отрясаем вам; однако же знайте, что приблизилось к вам Царствие Божие.
\par 12 Сказываю вам, что Содому в день оный будет отраднее, нежели городу тому.
\par 13 Горе тебе, Хоразин! горе тебе, Вифсаида! ибо если бы в Тире и Сидоне явлены были силы, явленные в вас, то давно бы они, сидя во вретище и пепле, покаялись;
\par 14 но и Тиру и Сидону отраднее будет на суде, нежели вам.
\par 15 И ты, Капернаум, до неба вознесшийся, до ада низвергнешься.
\par 16 Слушающий вас Меня слушает, и отвергающийся вас Меня отвергается; а отвергающийся Меня отвергается Пославшего Меня.
\par 17 Семьдесят [учеников] возвратились с радостью и говорили: Господи! и бесы повинуются нам о имени Твоем.
\par 18 Он же сказал им: Я видел сатану, спадшего с неба, как молнию;
\par 19 се, даю вам власть наступать на змей и скорпионов и на всю силу вражью, и ничто не повредит вам;
\par 20 однакож тому не радуйтесь, что духи вам повинуются, но радуйтесь тому, что имена ваши написаны на небесах.
\par 21 В тот час возрадовался духом Иисус и сказал: славлю Тебя, Отче, Господи неба и земли, что Ты утаил сие от мудрых и разумных и открыл младенцам. Ей, Отче! Ибо таково было Твое благоволение.
\par 22 И, обратившись к ученикам, сказал: все предано Мне Отцем Моим; и кто есть Сын, не знает никто, кроме Отца, и кто есть Отец, [не знает] [никто], кроме Сына, и кому Сын хочет открыть.
\par 23 И, обратившись к ученикам, сказал им особо: блаженны очи, видящие то, что вы видите!
\par 24 ибо сказываю вам, что многие пророки и цари желали видеть, что вы видите, и не видели, и слышать, что вы слышите, и не слышали.
\par 25 И вот, один законник встал и, искушая Его, сказал: Учитель! что мне делать, чтобы наследовать жизнь вечную?
\par 26 Он же сказал ему: в законе что написано? как читаешь?
\par 27 Он сказал в ответ: возлюби Господа Бога твоего всем сердцем твоим, и всею душею твоею, и всею крепостию твоею, и всем разумением твоим, и ближнего твоего, как самого себя.
\par 28 [Иисус] сказал ему: правильно ты отвечал; так поступай, и будешь жить.
\par 29 Но он, желая оправдать себя, сказал Иисусу: а кто мой ближний?
\par 30 На это сказал Иисус: некоторый человек шел из Иерусалима в Иерихон и попался разбойникам, которые сняли с него одежду, изранили его и ушли, оставив его едва живым.
\par 31 По случаю один священник шел тою дорогою и, увидев его, прошел мимо.
\par 32 Также и левит, быв на том месте, подошел, посмотрел и прошел мимо.
\par 33 Самарянин же некто, проезжая, нашел на него и, увидев его, сжалился
\par 34 и, подойдя, перевязал ему раны, возливая масло и вино; и, посадив его на своего осла, привез его в гостиницу и позаботился о нем;
\par 35 а на другой день, отъезжая, вынул два динария, дал содержателю гостиницы и сказал ему: позаботься о нем; и если издержишь что более, я, когда возвращусь, отдам тебе.
\par 36 Кто из этих троих, думаешь ты, был ближний попавшемуся разбойникам?
\par 37 Он сказал: оказавший ему милость. Тогда Иисус сказал ему: иди, и ты поступай так же.
\par 38 В продолжение пути их пришел Он в одно селение; здесь женщина, именем Марфа, приняла Его в дом свой;
\par 39 у нее была сестра, именем Мария, которая села у ног Иисуса и слушала слово Его.
\par 40 Марфа же заботилась о большом угощении и, подойдя, сказала: Господи! или Тебе нужды нет, что сестра моя одну меня оставила служить? скажи ей, чтобы помогла мне.
\par 41 Иисус же сказал ей в ответ: Марфа! Марфа! ты заботишься и суетишься о многом,
\par 42 а одно только нужно; Мария же избрала благую часть, которая не отнимется у нее.

\chapter{11}

\par 1 Случилось, что когда Он в одном месте молился, и перестал, один из учеников Его сказал Ему: Господи! научи нас молиться, как и Иоанн научил учеников своих.
\par 2 Он сказал им: когда молитесь, говорите: Отче наш, сущий на небесах! да святится имя Твое; да приидет Царствие Твое; да будет воля Твоя и на земле, как на небе;
\par 3 хлеб наш насущный подавай нам на каждый день;
\par 4 и прости нам грехи наши, ибо и мы прощаем всякому должнику нашему; и не введи нас в искушение, но избавь нас от лукавого.
\par 5 И сказал им: [положим, что] кто-нибудь из вас, имея друга, придет к нему в полночь и скажет ему: друг! дай мне взаймы три хлеба,
\par 6 ибо друг мой с дороги зашел ко мне, и мне нечего предложить ему;
\par 7 а тот изнутри скажет ему в ответ: не беспокой меня, двери уже заперты, и дети мои со мною на постели; не могу встать и дать тебе.
\par 8 Если, говорю вам, он не встанет и не даст ему по дружбе с ним, то по неотступности его, встав, даст ему, сколько просит.
\par 9 И Я скажу вам: просите, и дано будет вам; ищите, и найдете; стучите, и отворят вам,
\par 10 ибо всякий просящий получает, и ищущий находит, и стучащему отворят.
\par 11 Какой из вас отец, [когда] сын попросит у него хлеба, подаст ему камень? или, [когда попросит] рыбы, подаст ему змею вместо рыбы?
\par 12 Или, если попросит яйца, подаст ему скорпиона?
\par 13 Итак, если вы, будучи злы, умеете даяния благие давать детям вашим, тем более Отец Небесный даст Духа Святаго просящим у Него.
\par 14 Однажды изгнал Он беса, который был нем; и когда бес вышел, немой стал говорить; и народ удивился.
\par 15 Некоторые же из них говорили: Он изгоняет бесов силою веельзевула, князя бесовского.
\par 16 А другие, искушая, требовали от Него знамения с неба.
\par 17 Но Он, зная помышления их, сказал им: всякое царство, разделившееся само в себе, опустеет, и дом, [разделившийся] сам в себе, падет;
\par 18 если же и сатана разделится сам в себе, то как устоит царство его? а вы говорите, что Я силою веельзевула изгоняю бесов;
\par 19 и если Я силою веельзевула изгоняю бесов, то сыновья ваши чьею силою изгоняют их? Посему они будут вам судьями.
\par 20 Если же Я перстом Божиим изгоняю бесов, то, конечно, достигло до вас Царствие Божие.
\par 21 Когда сильный с оружием охраняет свой дом, тогда в безопасности его имение;
\par 22 когда же сильнейший его нападет на него и победит его, тогда возьмет все оружие его, на которое он надеялся, и разделит похищенное у него.
\par 23 Кто не со Мною, тот против Меня; и кто не собирает со Мною, тот расточает.
\par 24 Когда нечистый дух выйдет из человека, то ходит по безводным местам, ища покоя, и, не находя, говорит: возвращусь в дом мой, откуда вышел;
\par 25 и, придя, находит его выметенным и убранным;
\par 26 тогда идет и берет с собою семь других духов, злейших себя, и, войдя, живут там, --и бывает для человека того последнее хуже первого.
\par 27 Когда же Он говорил это, одна женщина, возвысив голос из народа, сказала Ему: блаженно чрево, носившее Тебя, и сосцы, Тебя питавшие!
\par 28 А Он сказал: блаженны слышащие слово Божие и соблюдающие его.
\par 29 Когда же народ стал сходиться во множестве, Он начал говорить: род сей лукав, он ищет знамения, и знамение не дастся ему, кроме знамения Ионы пророка;
\par 30 ибо как Иона был знамением для Ниневитян, так будет и Сын Человеческий для рода сего.
\par 31 Царица южная восстанет на суд с людьми рода сего и осудит их, ибо она приходила от пределов земли послушать мудрости Соломоновой; и вот, здесь больше Соломона.
\par 32 Ниневитяне восстанут на суд с родом сим и осудят его, ибо они покаялись от проповеди Иониной, и вот, здесь больше Ионы.
\par 33 Никто, зажегши свечу, не ставит ее в сокровенном месте, ни под сосудом, но на подсвечнике, чтобы входящие видели свет.
\par 34 Светильник тела есть око; итак, если око твое будет чисто, то и все тело твое будет светло; а если оно будет худо, то и тело твое будет темно.
\par 35 Итак, смотри: свет, который в тебе, не есть ли тьма?
\par 36 Если же тело твое все светло и не имеет ни одной темной части, то будет светло все так, как бы светильник освещал тебя сиянием.
\par 37 Когда Он говорил это, один фарисей просил Его к себе обедать. Он пришел и возлег.
\par 38 Фарисей же удивился, увидев, что Он не умыл [рук] перед обедом.
\par 39 Но Господь сказал ему: ныне вы, фарисеи, внешность чаши и блюда очищаете, а внутренность ваша исполнена хищения и лукавства.
\par 40 Неразумные! не Тот же ли, Кто сотворил внешнее, сотворил и внутреннее?
\par 41 Подавайте лучше милостыню из того, что у вас есть, тогда все будет у вас чисто.
\par 42 Но горе вам, фарисеям, что даете десятину с мяты, руты и всяких овощей, и нерадите о суде и любви Божией: сие надлежало делать, и того не оставлять.
\par 43 Горе вам, фарисеям, что любите председания в синагогах и приветствия в народных собраниях.
\par 44 Горе вам, книжники и фарисеи, лицемеры, что вы--как гробы скрытые, над которыми люди ходят и не знают того.
\par 45 На это некто из законников сказал Ему: Учитель! говоря это, Ты и нас обижаешь.
\par 46 Но Он сказал: и вам, законникам, горе, что налагаете на людей бремена неудобоносимые, а сами и одним перстом своим не дотрагиваетесь до них.
\par 47 Горе вам, что строите гробницы пророкам, которых избили отцы ваши:
\par 48 сим вы свидетельствуете о делах отцов ваших и соглашаетесь с ними, ибо они избили пророков, а вы строите им гробницы.
\par 49 Потому и премудрость Божия сказала: пошлю к ним пророков и Апостолов, и из них одних убьют, а других изгонят,
\par 50 да взыщется от рода сего кровь всех пророков, пролитая от создания мира,
\par 51 от крови Авеля до крови Захарии, убитого между жертвенником и храмом. Ей, говорю вам, взыщется от рода сего.
\par 52 Горе вам, законникам, что вы взяли ключ разумения: сами не вошли, и входящим воспрепятствовали.
\par 53 Когда Он говорил им это, книжники и фарисеи начали сильно приступать к Нему, вынуждая у Него ответы на многое,
\par 54 подыскиваясь под Него и стараясь уловить что-нибудь из уст Его, чтобы обвинить Его.

\chapter{12}

\par 1 Между тем, когда собрались тысячи народа, так что теснили друг друга, Он начал говорить сперва ученикам Своим: берегитесь закваски фарисейской, которая есть лицемерие.
\par 2 Нет ничего сокровенного, что не открылось бы, и тайного, чего не узнали бы.
\par 3 Посему, что вы сказали в темноте, то услышится во свете; и что говорили на ухо внутри дома, то будет провозглашено на кровлях.
\par 4 Говорю же вам, друзьям Моим: не бойтесь убивающих тело и потом не могущих ничего более сделать;
\par 5 но скажу вам, кого бояться: бойтесь того, кто, по убиении, может ввергнуть в геенну: ей, говорю вам, того бойтесь.
\par 6 Не пять ли малых птиц продаются за два ассария? и ни одна из них не забыта у Бога.
\par 7 А у вас и волосы на голове все сочтены. Итак не бойтесь: вы дороже многих малых птиц.
\par 8 Сказываю же вам: всякого, кто исповедает Меня пред человеками, и Сын Человеческий исповедает пред Ангелами Божиими;
\par 9 а кто отвергнется Меня пред человеками, тот отвержен будет пред Ангелами Божиими.
\par 10 И всякому, кто скажет слово на Сына Человеческого, прощено будет; а кто скажет хулу на Святаго Духа, тому не простится.
\par 11 Когда же приведут вас в синагоги, к начальствам и властям, не заботьтесь, как или что отвечать, или что говорить,
\par 12 ибо Святый Дух научит вас в тот час, что должно говорить.
\par 13 Некто из народа сказал Ему: Учитель! скажи брату моему, чтобы он разделил со мною наследство.
\par 14 Он же сказал человеку тому: кто поставил Меня судить или делить вас?
\par 15 При этом сказал им: смотрите, берегитесь любостяжания, ибо жизнь человека не зависит от изобилия его имения.
\par 16 И сказал им притчу: у одного богатого человека был хороший урожай в поле;
\par 17 и он рассуждал сам с собою: что мне делать? некуда мне собрать плодов моих?
\par 18 И сказал: вот что сделаю: сломаю житницы мои и построю большие, и соберу туда весь хлеб мой и все добро мое,
\par 19 и скажу душе моей: душа! много добра лежит у тебя на многие годы: покойся, ешь, пей, веселись.
\par 20 Но Бог сказал ему: безумный! в сию ночь душу твою возьмут у тебя; кому же достанется то, что ты заготовил?
\par 21 Так [бывает с тем], кто собирает сокровища для себя, а не в Бога богатеет.
\par 22 И сказал ученикам Своим: посему говорю вам, --не заботьтесь для души вашей, что вам есть, ни для тела, во что одеться:
\par 23 душа больше пищи, и тело--одежды.
\par 24 Посмотрите на воронов: они не сеют, не жнут; нет у них ни хранилищ, ни житниц, и Бог питает их; сколько же вы лучше птиц?
\par 25 Да и кто из вас, заботясь, может прибавить себе роста хотя на один локоть?
\par 26 Итак, если и малейшего сделать не можете, что заботитесь о прочем?
\par 27 Посмотрите на лилии, как они растут: не трудятся, не прядут; но говорю вам, что и Соломон во всей славе своей не одевался так, как всякая из них.
\par 28 Если же траву на поле, которая сегодня есть, а завтра будет брошена в печь, Бог так одевает, то кольми паче вас, маловеры!
\par 29 Итак, не ищите, что вам есть, или что пить, и не беспокойтесь,
\par 30 потому что всего этого ищут люди мира сего; ваш же Отец знает, что вы имеете нужду в том;
\par 31 наипаче ищите Царствия Божия, и это все приложится вам.
\par 32 Не бойся, малое стадо! ибо Отец ваш благоволил дать вам Царство.
\par 33 Продавайте имения ваши и давайте милостыню. Приготовляйте себе влагалища не ветшающие, сокровище неоскудевающее на небесах, куда вор не приближается и где моль не съедает,
\par 34 ибо где сокровище ваше, там и сердце ваше будет.
\par 35 Да будут чресла ваши препоясаны и светильники горящи.
\par 36 И вы будьте подобны людям, ожидающим возвращения господина своего с брака, дабы, когда придет и постучит, тотчас отворить ему.
\par 37 Блаженны рабы те, которых господин, придя, найдет бодрствующими; истинно говорю вам, он препояшется и посадит их, и, подходя, станет служить им.
\par 38 И если придет во вторую стражу, и в третью стражу придет, и найдет их так, то блаженны рабы те.
\par 39 Вы знаете, что если бы ведал хозяин дома, в который час придет вор, то бодрствовал бы и не допустил бы подкопать дом свой.
\par 40 Будьте же и вы готовы, ибо, в который час не думаете, приидет Сын Человеческий.
\par 41 Тогда сказал Ему Петр: Господи! к нам ли притчу сию говоришь, или и ко всем?
\par 42 Господь же сказал: кто верный и благоразумный домоправитель, которого господин поставил над слугами своими раздавать им в свое время меру хлеба?
\par 43 Блажен раб тот, которого господин его, придя, найдет поступающим так.
\par 44 Истинно говорю вам, что над всем имением своим поставит его.
\par 45 Если же раб тот скажет в сердце своем: не скоро придет господин мой, и начнет бить слуг и служанок, есть и пить и напиваться, --
\par 46 то придет господин раба того в день, в который он не ожидает, и в час, в который не думает, и рассечет его, и подвергнет его одной участи с неверными.
\par 47 Раб же тот, который знал волю господина своего, и не был готов, и не делал по воле его, бит будет много;
\par 48 а который не знал, и сделал достойное наказания, бит будет меньше. И от всякого, кому дано много, много и потребуется, и кому много вверено, с того больше взыщут.
\par 49 Огонь пришел Я низвести на землю, и как желал бы, чтобы он уже возгорелся!
\par 50 Крещением должен Я креститься; и как Я томлюсь, пока сие совершится!
\par 51 Думаете ли вы, что Я пришел дать мир земле? Нет, говорю вам, но разделение;
\par 52 ибо отныне пятеро в одном доме станут разделяться, трое против двух, и двое против трех:
\par 53 отец будет против сына, и сын против отца; мать против дочери, и дочь против матери; свекровь против невестки своей, и невестка против свекрови своей.
\par 54 Сказал же и народу: когда вы видите облако, поднимающееся с запада, тотчас говорите: дождь будет, и бывает так;
\par 55 и когда дует южный ветер, говорите: зной будет, и бывает.
\par 56 Лицемеры! лице земли и неба распознавать умеете, как же времени сего не узнаете?
\par 57 Зачем же вы и по самим себе не судите, чему быть должно?
\par 58 Когда ты идешь с соперником своим к начальству, то на дороге постарайся освободиться от него, чтобы он не привел тебя к судье, а судья не отдал тебя истязателю, а истязатель не вверг тебя в темницу.
\par 59 Сказываю тебе: не выйдешь оттуда, пока не отдашь и последней полушки.

\chapter{13}

\par 1 В это время пришли некоторые и рассказали Ему о Галилеянах, которых кровь Пилат смешал с жертвами их.
\par 2 Иисус сказал им на это: думаете ли вы, что эти Галилеяне были грешнее всех Галилеян, что так пострадали?
\par 3 Нет, говорю вам, но, если не покаетесь, все так же погибнете.
\par 4 Или думаете ли, что те восемнадцать человек, на которых упала башня Силоамская и побила их, виновнее были всех, живущих в Иерусалиме?
\par 5 Нет, говорю вам, но, если не покаетесь, все так же погибнете.
\par 6 И сказал сию притчу: некто имел в винограднике своем посаженную смоковницу, и пришел искать плода на ней, и не нашел;
\par 7 и сказал виноградарю: вот, я третий год прихожу искать плода на этой смоковнице и не нахожу; сруби ее: на что она и землю занимает?
\par 8 Но он сказал ему в ответ: господин! оставь ее и на этот год, пока я окопаю ее и обложу навозом, --
\par 9 не принесет ли плода; если же нет, то в следующий [год] срубишь ее.
\par 10 В одной из синагог учил Он в субботу.
\par 11 Там была женщина, восемнадцать лет имевшая духа немощи: она была скорчена и не могла выпрямиться.
\par 12 Иисус, увидев ее, подозвал и сказал ей: женщина! ты освобождаешься от недуга твоего.
\par 13 И возложил на нее руки, и она тотчас выпрямилась и стала славить Бога.
\par 14 При этом начальник синагоги, негодуя, что Иисус исцелил в субботу, сказал народу: есть шесть дней, в которые должно делать; в те и приходите исцеляться, а не в день субботний.
\par 15 Господь сказал ему в ответ: лицемер! не отвязывает ли каждый из вас вола своего или осла от яслей в субботу и не ведет ли поить?
\par 16 сию же дочь Авраамову, которую связал сатана вот уже восемнадцать лет, не надлежало ли освободить от уз сих в день субботний?
\par 17 И когда говорил Он это, все противившиеся Ему стыдились; и весь народ радовался о всех славных делах Его.
\par 18 Он же сказал: чему подобно Царствие Божие? и чему уподоблю его?
\par 19 Оно подобно зерну горчичному, которое, взяв, человек посадил в саду своем; и выросло, и стало большим деревом, и птицы небесные укрывались в ветвях его.
\par 20 Еще сказал: чему уподоблю Царствие Божие?
\par 21 Оно подобно закваске, которую женщина, взяв, положила в три меры муки, доколе не вскисло все.
\par 22 И проходил по городам и селениям, уча и направляя путь к Иерусалиму.
\par 23 Некто сказал Ему: Господи! неужели мало спасающихся? Он же сказал им:
\par 24 подвизайтесь войти сквозь тесные врата, ибо, сказываю вам, многие поищут войти, и не возмогут.
\par 25 Когда хозяин дома встанет и затворит двери, тогда вы, стоя вне, станете стучать в двери и говорить: Господи! Господи! отвори нам; но Он скажет вам в ответ: не знаю вас, откуда вы.
\par 26 Тогда станете говорить: мы ели и пили пред Тобою, и на улицах наших учил Ты.
\par 27 Но Он скажет: говорю вам: не знаю вас, откуда вы; отойдите от Меня все делатели неправды.
\par 28 Там будет плач и скрежет зубов, когда увидите Авраама, Исаака и Иакова и всех пророков в Царствии Божием, а себя изгоняемыми вон.
\par 29 И придут от востока и запада, и севера и юга, и возлягут в Царствии Божием.
\par 30 И вот, есть последние, которые будут первыми, и есть первые, которые будут последними.
\par 31 В тот день пришли некоторые из фарисеев и говорили Ему: выйди и удались отсюда, ибо Ирод хочет убить Тебя.
\par 32 И сказал им: пойдите, скажите этой лисице: се, изгоняю бесов и совершаю исцеления сегодня и завтра, и в третий [день] кончу;
\par 33 а впрочем, Мне должно ходить сегодня, завтра и в последующий день, потому что не бывает, чтобы пророк погиб вне Иерусалима.
\par 34 Иерусалим! Иерусалим! избивающий пророков и камнями побивающий посланных к тебе! сколько раз хотел Я собрать чад твоих, как птица птенцов своих под крылья, и вы не захотели!
\par 35 Се, оставляется вам дом ваш пуст. Сказываю же вам, что вы не увидите Меня, пока не придет время, когда скажете: благословен Грядый во имя Господне!

\chapter{14}

\par 1 Случилось Ему в субботу придти в дом одного из начальников фарисейских вкусить хлеба, и они наблюдали за Ним.
\par 2 И вот, предстал пред Него человек, страждущий водяною болезнью.
\par 3 По сему случаю Иисус спросил законников и фарисеев: позволительно ли врачевать в субботу?
\par 4 Они молчали. И, прикоснувшись, исцелил его и отпустил.
\par 5 При сем сказал им: если у кого из вас осел или вол упадет в колодезь, не тотчас ли вытащит его и в субботу?
\par 6 И не могли отвечать Ему на это.
\par 7 Замечая же, как званые выбирали первые места, сказал им притчу:
\par 8 когда ты будешь позван кем на брак, не садись на первое место, чтобы не случился кто из званых им почетнее тебя,
\par 9 и звавший тебя и его, подойдя, не сказал бы тебе: уступи ему место; и тогда со стыдом должен будешь занять последнее место.
\par 10 Но когда зван будешь, придя, садись на последнее место, чтобы звавший тебя, подойдя, сказал: друг! пересядь выше; тогда будет тебе честь пред сидящими с тобою,
\par 11 ибо всякий возвышающий сам себя унижен будет, а унижающий себя возвысится.
\par 12 Сказал же и позвавшему Его: когда делаешь обед или ужин, не зови друзей твоих, ни братьев твоих, ни родственников твоих, ни соседей богатых, чтобы и они тебя когда не позвали, и не получил ты воздаяния.
\par 13 Но, когда делаешь пир, зови нищих, увечных, хромых, слепых,
\par 14 и блажен будешь, что они не могут воздать тебе, ибо воздастся тебе в воскресение праведных.
\par 15 Услышав это, некто из возлежащих с Ним сказал Ему: блажен, кто вкусит хлеба в Царствии Божием!
\par 16 Он же сказал ему: один человек сделал большой ужин и звал многих,
\par 17 и когда наступило время ужина, послал раба своего сказать званым: идите, ибо уже все готово.
\par 18 И начали все, как бы сговорившись, извиняться. Первый сказал ему: я купил землю и мне нужно пойти посмотреть ее; прошу тебя, извини меня.
\par 19 Другой сказал: я купил пять пар волов и иду испытать их; прошу тебя, извини меня.
\par 20 Третий сказал: я женился и потому не могу придти.
\par 21 И, возвратившись, раб тот донес о сем господину своему. Тогда, разгневавшись, хозяин дома сказал рабу своему: пойди скорее по улицам и переулкам города и приведи сюда нищих, увечных, хромых и слепых.
\par 22 И сказал раб: господин! исполнено, как приказал ты, и еще есть место.
\par 23 Господин сказал рабу: пойди по дорогам и изгородям и убеди придти, чтобы наполнился дом мой.
\par 24 Ибо сказываю вам, что никто из тех званых не вкусит моего ужина, ибо много званых, но мало избранных.
\par 25 С Ним шло множество народа; и Он, обратившись, сказал им:
\par 26 если кто приходит ко Мне и не возненавидит отца своего и матери, и жены и детей, и братьев и сестер, а притом и самой жизни своей, тот не может быть Моим учеником;
\par 27 и кто не несет креста своего и идет за Мною, не может быть Моим учеником.
\par 28 Ибо кто из вас, желая построить башню, не сядет прежде и не вычислит издержек, имеет ли он, что нужно для совершения ее,
\par 29 дабы, когда положит основание и не возможет совершить, все видящие не стали смеяться над ним,
\par 30 говоря: этот человек начал строить и не мог окончить?
\par 31 Или какой царь, идя на войну против другого царя, не сядет и не посоветуется прежде, силен ли он с десятью тысячами противостать идущему на него с двадцатью тысячами?
\par 32 Иначе, пока тот еще далеко, он пошлет к нему посольство просить о мире.
\par 33 Так всякий из вас, кто не отрешится от всего, что имеет, не может быть Моим учеником.
\par 34 Соль--добрая вещь; но если соль потеряет силу, чем исправить ее?
\par 35 ни в землю, ни в навоз не годится; вон выбрасывают ее. Кто имеет уши слышать, да слышит!

\chapter{15}

\par 1 Приближались к Нему все мытари и грешники слушать Его.
\par 2 Фарисеи же и книжники роптали, говоря: Он принимает грешников и ест с ними.
\par 3 Но Он сказал им следующую притчу:
\par 4 кто из вас, имея сто овец и потеряв одну из них, не оставит девяноста девяти в пустыне и не пойдет за пропавшею, пока не найдет ее?
\par 5 А найдя, возьмет ее на плечи свои с радостью
\par 6 и, придя домой, созовет друзей и соседей и скажет им: порадуйтесь со мною: я нашел мою пропавшую овцу.
\par 7 Сказываю вам, что так на небесах более радости будет об одном грешнике кающемся, нежели о девяноста девяти праведниках, не имеющих нужды в покаянии.
\par 8 Или какая женщина, имея десять драхм, если потеряет одну драхму, не зажжет свечи и не станет мести комнату и искать тщательно, пока не найдет,
\par 9 а найдя, созовет подруг и соседок и скажет: порадуйтесь со мною: я нашла потерянную драхму.
\par 10 Так, говорю вам, бывает радость у Ангелов Божиих и об одном грешнике кающемся.
\par 11 Еще сказал: у некоторого человека было два сына;
\par 12 и сказал младший из них отцу: отче! дай мне следующую [мне] часть имения. И [отец] разделил им имение.
\par 13 По прошествии немногих дней младший сын, собрав все, пошел в дальнюю сторону и там расточил имение свое, живя распутно.
\par 14 Когда же он прожил все, настал великий голод в той стране, и он начал нуждаться;
\par 15 и пошел, пристал к одному из жителей страны той, а тот послал его на поля свои пасти свиней;
\par 16 и он рад был наполнить чрево свое рожками, которые ели свиньи, но никто не давал ему.
\par 17 Придя же в себя, сказал: сколько наемников у отца моего избыточествуют хлебом, а я умираю от голода;
\par 18 встану, пойду к отцу моему и скажу ему: отче! я согрешил против неба и пред тобою
\par 19 и уже недостоин называться сыном твоим; прими меня в число наемников твоих.
\par 20 Встал и пошел к отцу своему. И когда он был еще далеко, увидел его отец его и сжалился; и, побежав, пал ему на шею и целовал его.
\par 21 Сын же сказал ему: отче! я согрешил против неба и пред тобою и уже недостоин называться сыном твоим.
\par 22 А отец сказал рабам своим: принесите лучшую одежду и оденьте его, и дайте перстень на руку его и обувь на ноги;
\par 23 и приведите откормленного теленка, и заколите; станем есть и веселиться!
\par 24 ибо этот сын мой был мертв и ожил, пропадал и нашелся. И начали веселиться.
\par 25 Старший же сын его был на поле; и возвращаясь, когда приблизился к дому, услышал пение и ликование;
\par 26 и, призвав одного из слуг, спросил: что это такое?
\par 27 Он сказал ему: брат твой пришел, и отец твой заколол откормленного теленка, потому что принял его здоровым.
\par 28 Он осердился и не хотел войти. Отец же его, выйдя, звал его.
\par 29 Но он сказал в ответ отцу: вот, я столько лет служу тебе и никогда не преступал приказания твоего, но ты никогда не дал мне и козленка, чтобы мне повеселиться с друзьями моими;
\par 30 а когда этот сын твой, расточивший имение свое с блудницами, пришел, ты заколол для него откормленного теленка.
\par 31 Он же сказал ему: сын мой! ты всегда со мною, и все мое твое,
\par 32 а о том надобно было радоваться и веселиться, что брат твой сей был мертв и ожил, пропадал и нашелся.

\chapter{16}

\par 1 Сказал же и к ученикам Своим: один человек был богат и имел управителя, на которого донесено было ему, что расточает имение его;
\par 2 и, призвав его, сказал ему: что это я слышу о тебе? дай отчет в управлении твоем, ибо ты не можешь более управлять.
\par 3 Тогда управитель сказал сам в себе: что мне делать? господин мой отнимает у меня управление домом; копать не могу, просить стыжусь;
\par 4 знаю, что сделать, чтобы приняли меня в домы свои, когда отставлен буду от управления домом.
\par 5 И, призвав должников господина своего, каждого порознь, сказал первому: сколько ты должен господину моему?
\par 6 Он сказал: сто мер масла. И сказал ему: возьми твою расписку и садись скорее, напиши: пятьдесят.
\par 7 Потом другому сказал: а ты сколько должен? Он отвечал: сто мер пшеницы. И сказал ему: возьми твою расписку и напиши: восемьдесят.
\par 8 И похвалил господин управителя неверного, что догадливо поступил; ибо сыны века сего догадливее сынов света в своем роде.
\par 9 И Я говорю вам: приобретайте себе друзей богатством неправедным, чтобы они, когда обнищаете, приняли вас в вечные обители.
\par 10 Верный в малом и во многом верен, а неверный в малом неверен и во многом.
\par 11 Итак, если вы в неправедном богатстве не были верны, кто поверит вам истинное?
\par 12 И если в чужом не были верны, кто даст вам ваше?
\par 13 Никакой слуга не может служить двум господам, ибо или одного будет ненавидеть, а другого любить, или одному станет усердствовать, а о другом нерадеть. Не можете служить Богу и маммоне.
\par 14 Слышали все это и фарисеи, которые были сребролюбивы, и они смеялись над Ним.
\par 15 Он сказал им: вы выказываете себя праведниками пред людьми, но Бог знает сердца ваши, ибо что высоко у людей, то мерзость пред Богом.
\par 16 Закон и пророки до Иоанна; с сего времени Царствие Божие благовествуется, и всякий усилием входит в него.
\par 17 Но скорее небо и земля прейдут, нежели одна черта из закона пропадет.
\par 18 Всякий, разводящийся с женою своею и женящийся на другой, прелюбодействует, и всякий, женящийся на разведенной с мужем, прелюбодействует.
\par 19 Некоторый человек был богат, одевался в порфиру и виссон и каждый день пиршествовал блистательно.
\par 20 Был также некоторый нищий, именем Лазарь, который лежал у ворот его в струпьях
\par 21 и желал напитаться крошками, падающими со стола богача, и псы, приходя, лизали струпья его.
\par 22 Умер нищий и отнесен был Ангелами на лоно Авраамово. Умер и богач, и похоронили его.
\par 23 И в аде, будучи в муках, он поднял глаза свои, увидел вдали Авраама и Лазаря на лоне его
\par 24 и, возопив, сказал: отче Аврааме! умилосердись надо мною и пошли Лазаря, чтобы омочил конец перста своего в воде и прохладил язык мой, ибо я мучаюсь в пламени сем.
\par 25 Но Авраам сказал: чадо! вспомни, что ты получил уже доброе твое в жизни твоей, а Лазарь--злое; ныне же он здесь утешается, а ты страдаешь;
\par 26 и сверх всего того между нами и вами утверждена великая пропасть, так что хотящие перейти отсюда к вам не могут, также и оттуда к нам не переходят.
\par 27 Тогда сказал он: так прошу тебя, отче, пошли его в дом отца моего,
\par 28 ибо у меня пять братьев; пусть он засвидетельствует им, чтобы и они не пришли в это место мучения.
\par 29 Авраам сказал ему: у них есть Моисей и пророки; пусть слушают их.
\par 30 Он же сказал: нет, отче Аврааме, но если кто из мертвых придет к ним, покаются.
\par 31 Тогда [Авраам] сказал ему: если Моисея и пророков не слушают, то если бы кто и из мертвых воскрес, не поверят.

\chapter{17}

\par 1 Сказал также [Иисус] ученикам: невозможно не придти соблазнам, но горе тому, через кого они приходят;
\par 2 лучше было бы ему, если бы мельничный жернов повесили ему на шею и бросили его в море, нежели чтобы он соблазнил одного из малых сих.
\par 3 Наблюдайте за собою. Если же согрешит против тебя брат твой, выговори ему; и если покается, прости ему;
\par 4 и если семь раз в день согрешит против тебя и семь раз в день обратится, и скажет: каюсь, --прости ему.
\par 5 И сказали Апостолы Господу: умножь в нас веру.
\par 6 Господь сказал: если бы вы имели веру с зерно горчичное и сказали смоковнице сей: исторгнись и пересадись в море, то она послушалась бы вас.
\par 7 Кто из вас, имея раба пашущего или пасущего, по возвращении его с поля, скажет ему: пойди скорее, садись за стол?
\par 8 Напротив, не скажет ли ему: приготовь мне поужинать и, подпоясавшись, служи мне, пока буду есть и пить, и потом ешь и пей сам?
\par 9 Станет ли он благодарить раба сего за то, что он исполнил приказание? Не думаю.
\par 10 Так и вы, когда исполните все повеленное вам, говорите: мы рабы ничего не стоящие, потому что сделали, что должны были сделать.
\par 11 Идя в Иерусалим, Он проходил между Самариею и Галилеею.
\par 12 И когда входил Он в одно селение, встретили Его десять человек прокаженных, которые остановились вдали
\par 13 и громким голосом говорили: Иисус Наставник! помилуй нас.
\par 14 Увидев [их], Он сказал им: пойдите, покажитесь священникам. И когда они шли, очистились.
\par 15 Один же из них, видя, что исцелен, возвратился, громким голосом прославляя Бога,
\par 16 и пал ниц к ногам Его, благодаря Его; и это был Самарянин.
\par 17 Тогда Иисус сказал: не десять ли очистились? где же девять?
\par 18 как они не возвратились воздать славу Богу, кроме сего иноплеменника?
\par 19 И сказал ему: встань, иди; вера твоя спасла тебя.
\par 20 Быв же спрошен фарисеями, когда придет Царствие Божие, отвечал им: не придет Царствие Божие приметным образом,
\par 21 и не скажут: вот, оно здесь, или: вот, там. Ибо вот, Царствие Божие внутрь вас есть.
\par 22 Сказал также ученикам: придут дни, когда пожелаете видеть хотя один из дней Сына Человеческого, и не увидите;
\par 23 и скажут вам: вот, здесь, или: вот, там, --не ходите и не гоняйтесь,
\par 24 ибо, как молния, сверкнувшая от одного края неба, блистает до другого края неба, так будет Сын Человеческий в день Свой.
\par 25 Но прежде надлежит Ему много пострадать и быть отвержену родом сим.
\par 26 И как было во дни Ноя, так будет и во дни Сына Человеческого:
\par 27 ели, пили, женились, выходили замуж, до того дня, как вошел Ной в ковчег, и пришел потоп и погубил всех.
\par 28 Так же, как было и во дни Лота: ели, пили, покупали, продавали, садили, строили;
\par 29 но в день, в который Лот вышел из Содома, пролился с неба дождь огненный и серный и истребил всех;
\par 30 так будет и в тот день, когда Сын Человеческий явится.
\par 31 В тот день, кто будет на кровле, а вещи его в доме, тот не сходи взять их; и кто будет на поле, также не обращайся назад.
\par 32 Вспоминайте жену Лотову.
\par 33 Кто станет сберегать душу свою, тот погубит ее; а кто погубит ее, тот оживит ее.
\par 34 Сказываю вам: в ту ночь будут двое на одной постели: один возьмется, а другой оставится;
\par 35 две будут молоть вместе: одна возьмется, а другая оставится;
\par 36 двое будут на поле: один возьмется, а другой оставится.
\par 37 На это сказали Ему: где, Господи? Он же сказал им: где труп, там соберутся и орлы.

\chapter{18}

\par 1 Сказал также им притчу о том, что должно всегда молиться и не унывать,
\par 2 говоря: в одном городе был судья, который Бога не боялся и людей не стыдился.
\par 3 В том же городе была одна вдова, и она, приходя к нему, говорила: защити меня от соперника моего.
\par 4 Но он долгое время не хотел. А после сказал сам в себе: хотя я и Бога не боюсь и людей не стыжусь,
\par 5 но, как эта вдова не дает мне покоя, защищу ее, чтобы она не приходила больше докучать мне.
\par 6 И сказал Господь: слышите, что говорит судья неправедный?
\par 7 Бог ли не защитит избранных Своих, вопиющих к Нему день и ночь, хотя и медлит защищать их?
\par 8 сказываю вам, что подаст им защиту вскоре. Но Сын Человеческий, придя, найдет ли веру на земле?
\par 9 Сказал также к некоторым, которые уверены были о себе, что они праведны, и уничижали других, следующую притчу:
\par 10 два человека вошли в храм помолиться: один фарисей, а другой мытарь.
\par 11 Фарисей, став, молился сам в себе так: Боже! благодарю Тебя, что я не таков, как прочие люди, грабители, обидчики, прелюбодеи, или как этот мытарь:
\par 12 пощусь два раза в неделю, даю десятую часть из всего, что приобретаю.
\par 13 Мытарь же, стоя вдали, не смел даже поднять глаз на небо; но, ударяя себя в грудь, говорил: Боже! будь милостив ко мне грешнику!
\par 14 Сказываю вам, что сей пошел оправданным в дом свой более, нежели тот: ибо всякий, возвышающий сам себя, унижен будет, а унижающий себя возвысится.
\par 15 Приносили к Нему и младенцев, чтобы Он прикоснулся к ним; ученики же, видя то, возбраняли им.
\par 16 Но Иисус, подозвав их, сказал: пустите детей приходить ко Мне и не возбраняйте им, ибо таковых есть Царствие Божие.
\par 17 Истинно говорю вам: кто не примет Царствия Божия, как дитя, тот не войдет в него.
\par 18 И спросил Его некто из начальствующих: Учитель благий! что мне делать, чтобы наследовать жизнь вечную?
\par 19 Иисус сказал ему: что ты называешь Меня благим? никто не благ, как только один Бог;
\par 20 знаешь заповеди: не прелюбодействуй, не убивай, не кради, не лжесвидетельствуй, почитай отца твоего и матерь твою.
\par 21 Он же сказал: все это сохранил я от юности моей.
\par 22 Услышав это, Иисус сказал ему: еще одного недостает тебе: все, что имеешь, продай и раздай нищим, и будешь иметь сокровище на небесах, и приходи, следуй за Мною.
\par 23 Он же, услышав сие, опечалился, потому что был очень богат.
\par 24 Иисус, видя, что он опечалился, сказал: как трудно имеющим богатство войти в Царствие Божие!
\par 25 ибо удобнее верблюду пройти сквозь игольные уши, нежели богатому войти в Царствие Божие.
\par 26 Слышавшие сие сказали: кто же может спастись?
\par 27 Но Он сказал: невозможное человекам возможно Богу.
\par 28 Петр же сказал: вот, мы оставили все и последовали за Тобою.
\par 29 Он сказал им: истинно говорю вам: нет никого, кто оставил бы дом, или родителей, или братьев, или сестер, или жену, или детей для Царствия Божия,
\par 30 и не получил бы гораздо более в сие время, и в век будущий жизни вечной.
\par 31 Отозвав же двенадцать учеников Своих, сказал им: вот, мы восходим в Иерусалим, и совершится все, написанное через пророков о Сыне Человеческом,
\par 32 ибо предадут Его язычникам, и поругаются над Ним, и оскорбят Его, и оплюют Его,
\par 33 и будут бить, и убьют Его: и в третий день воскреснет.
\par 34 Но они ничего из этого не поняли; слова сии были для них сокровенны, и они не разумели сказанного.
\par 35 Когда же подходил Он к Иерихону, один слепой сидел у дороги, прося милостыни,
\par 36 и, услышав, что мимо него проходит народ, спросил: что это такое?
\par 37 Ему сказали, что Иисус Назорей идет.
\par 38 Тогда он закричал: Иисус, Сын Давидов! помилуй меня.
\par 39 Шедшие впереди заставляли его молчать; но он еще громче кричал: Сын Давидов! помилуй меня.
\par 40 Иисус, остановившись, велел привести его к Себе: и, когда тот подошел к Нему, спросил его:
\par 41 чего ты хочешь от Меня? Он сказал: Господи! чтобы мне прозреть.
\par 42 Иисус сказал ему: прозри! вера твоя спасла тебя.
\par 43 И он тотчас прозрел и пошел за Ним, славя Бога; и весь народ, видя это, воздал хвалу Богу.

\chapter{19}

\par 1 Потом [Иисус] вошел в Иерихон и проходил через него.
\par 2 И вот, некто, именем Закхей, начальник мытарей и человек богатый,
\par 3 искал видеть Иисуса, кто Он, но не мог за народом, потому что мал был ростом,
\par 4 и, забежав вперед, взлез на смоковницу, чтобы увидеть Его, потому что Ему надлежало проходить мимо нее.
\par 5 Иисус, когда пришел на это место, взглянув, увидел его и сказал ему: Закхей! сойди скорее, ибо сегодня надобно Мне быть у тебя в доме.
\par 6 И он поспешно сошел и принял Его с радостью.
\par 7 И все, видя то, начали роптать, и говорили, что Он зашел к грешному человеку;
\par 8 Закхей же, став, сказал Господу: Господи! половину имения моего я отдам нищим, и, если кого чем обидел, воздам вчетверо.
\par 9 Иисус сказал ему: ныне пришло спасение дому сему, потому что и он сын Авраама,
\par 10 ибо Сын Человеческий пришел взыскать и спасти погибшее.
\par 11 Когда же они слушали это, присовокупил притчу: ибо Он был близ Иерусалима, и они думали, что скоро должно открыться Царствие Божие.
\par 12 Итак сказал: некоторый человек высокого рода отправлялся в дальнюю страну, чтобы получить себе царство и возвратиться;
\par 13 призвав же десять рабов своих, дал им десять мин и сказал им: употребляйте их в оборот, пока я возвращусь.
\par 14 Но граждане ненавидели его и отправили вслед за ним посольство, сказав: не хотим, чтобы он царствовал над нами.
\par 15 И когда возвратился, получив царство, велел призвать к себе рабов тех, которым дал серебро, чтобы узнать, кто что приобрел.
\par 16 Пришел первый и сказал: господин! мина твоя принесла десять мин.
\par 17 И сказал ему: хорошо, добрый раб! за то, что ты в малом был верен, возьми в управление десять городов.
\par 18 Пришел второй и сказал: господин! мина твоя принесла пять мин.
\par 19 Сказал и этому: и ты будь над пятью городами.
\par 20 Пришел третий и сказал: господин! вот твоя мина, которую я хранил, завернув в платок,
\par 21 ибо я боялся тебя, потому что ты человек жестокий: берешь, чего не клал, и жнешь, чего не сеял.
\par 22 [Господин] сказал ему: твоими устами буду судить тебя, лукавый раб! ты знал, что я человек жестокий, беру, чего не клал, и жну, чего не сеял;
\par 23 для чего же ты не отдал серебра моего в оборот, чтобы я, придя, получил его с прибылью?
\par 24 И сказал предстоящим: возьмите у него мину и дайте имеющему десять мин.
\par 25 И сказали ему: господин! у него есть десять мин.
\par 26 Сказываю вам, что всякому имеющему дано будет, а у неимеющего отнимется и то, что имеет;
\par 27 врагов же моих тех, которые не хотели, чтобы я царствовал над ними, приведите сюда и избейте предо мною.
\par 28 Сказав это, Он пошел далее, восходя в Иерусалим.
\par 29 И когда приблизился к Виффагии и Вифании, к горе, называемой Елеонскою, послал двух учеников Своих,
\par 30 сказав: пойдите в противолежащее селение; войдя в него, найдете молодого осла привязанного, на которого никто из людей никогда не садился; отвязав его, приведите;
\par 31 и если кто спросит вас: зачем отвязываете? скажите ему так: он надобен Господу.
\par 32 Посланные пошли и нашли, как Он сказал им.
\par 33 Когда же они отвязывали молодого осла, хозяева его сказали им: зачем отвязываете осленка?
\par 34 Они отвечали: он надобен Господу.
\par 35 И привели его к Иисусу, и, накинув одежды свои на осленка, посадили на него Иисуса.
\par 36 И, когда Он ехал, постилали одежды свои по дороге.
\par 37 А когда Он приблизился к спуску с горы Елеонской, все множество учеников начало в радости велегласно славить Бога за все чудеса, какие видели они,
\par 38 говоря: благословен Царь, грядущий во имя Господне! мир на небесах и слава в вышних!
\par 39 И некоторые фарисеи из среды народа сказали Ему: Учитель! запрети ученикам Твоим.
\par 40 Но Он сказал им в ответ: сказываю вам, что если они умолкнут, то камни возопиют.
\par 41 И когда приблизился к городу, то, смотря на него, заплакал о нем
\par 42 и сказал: о, если бы и ты хотя в сей твой день узнал, что служит к миру твоему! Но это сокрыто ныне от глаз твоих,
\par 43 ибо придут на тебя дни, когда враги твои обложат тебя окопами и окружат тебя, и стеснят тебя отовсюду,
\par 44 и разорят тебя, и побьют детей твоих в тебе, и не оставят в тебе камня на камне за то, что ты не узнал времени посещения твоего.
\par 45 И, войдя в храм, начал выгонять продающих в нем и покупающих,
\par 46 говоря им: написано: дом Мой есть дом молитвы, а вы сделали его вертепом разбойников.
\par 47 И учил каждый день в храме. Первосвященники же и книжники и старейшины народа искали погубить Его,
\par 48 и не находили, что бы сделать с Ним; потому что весь народ неотступно слушал Его.

\chapter{20}

\par 1 В один из тех дней, когда Он учил народ в храме и благовествовал, приступили первосвященники и книжники со старейшинами,
\par 2 и сказали Ему: скажи нам, какою властью Ты это делаешь, или кто дал Тебе власть сию?
\par 3 Он сказал им в ответ: спрошу и Я вас об одном, и скажите Мне:
\par 4 крещение Иоанново с небес было, или от человеков?
\par 5 Они же, рассуждая между собою, говорили: если скажем: с небес, то скажет: почему же вы не поверили ему?
\par 6 а если скажем: от человеков, то весь народ побьет нас камнями, ибо он уверен, что Иоанн есть пророк.
\par 7 И отвечали: не знаем откуда.
\par 8 Иисус сказал им: и Я не скажу вам, какою властью это делаю.
\par 9 И начал Он говорить к народу притчу сию: один человек насадил виноградник и отдал его виноградарям, и отлучился на долгое время;
\par 10 и в свое время послал к виноградарям раба, чтобы они дали ему плодов из виноградника; но виноградари, прибив его, отослали ни с чем.
\par 11 Еще послал другого раба; но они и этого, прибив и обругав, отослали ни с чем.
\par 12 И еще послал третьего; но они и того, изранив, выгнали.
\par 13 Тогда сказал господин виноградника: что мне делать? Пошлю сына моего возлюбленного; может быть, увидев его, постыдятся.
\par 14 Но виноградари, увидев его, рассуждали между собою, говоря: это наследник; пойдем, убьем его, и наследство его будет наше.
\par 15 И, выведя его вон из виноградника, убили. Что же сделает с ними господин виноградника?
\par 16 Придет и погубит виноградарей тех, и отдаст виноградник другим. Слышавшие же это сказали: да не будет!
\par 17 Но Он, взглянув на них, сказал: что значит сие написанное: камень, который отвергли строители, тот самый сделался главою угла?
\par 18 Всякий, кто упадет на тот камень, разобьется, а на кого он упадет, того раздавит.
\par 19 И искали в это время первосвященники и книжники, чтобы наложить на Него руки, но побоялись народа, ибо поняли, что о них сказал Он эту притчу.
\par 20 И, наблюдая за Ним, подослали лукавых людей, которые, притворившись благочестивыми, уловили бы Его в каком-либо слове, чтобы предать Его начальству и власти правителя.
\par 21 И они спросили Его: Учитель! мы знаем, что Ты правдиво говоришь и учишь и не смотришь на лице, но истинно пути Божию учишь;
\par 22 позволительно ли нам давать подать кесарю, или нет?
\par 23 Он же, уразумев лукавство их, сказал им: что вы Меня искушаете?
\par 24 Покажите Мне динарий: чье на нем изображение и надпись? Они отвечали: кесаревы.
\par 25 Он сказал им: итак, отдавайте кесарево кесарю, а Божие Богу.
\par 26 И не могли уловить Его в слове перед народом, и, удивившись ответу Его, замолчали.
\par 27 Тогда пришли некоторые из саддукеев, отвергающих воскресение, и спросили Его:
\par 28 Учитель! Моисей написал нам, что если у кого умрет брат, имевший жену, и умрет бездетным, то брат его должен взять его жену и восставить семя брату своему.
\par 29 Было семь братьев, первый, взяв жену, умер бездетным;
\par 30 взял ту жену второй, и тот умер бездетным;
\par 31 взял ее третий; также и все семеро, и умерли, не оставив детей;
\par 32 после всех умерла и жена;
\par 33 итак, в воскресение которого из них будет она женою, ибо семеро имели ее женою?
\par 34 Иисус сказал им в ответ: чада века сего женятся и выходят замуж;
\par 35 а сподобившиеся достигнуть того века и воскресения из мертвых ни женятся, ни замуж не выходят,
\par 36 и умереть уже не могут, ибо они равны Ангелам и суть сыны Божии, будучи сынами воскресения.
\par 37 А что мертвые воскреснут, и Моисей показал при купине, когда назвал Господа Богом Авраама и Богом Исаака и Богом Иакова.
\par 38 Бог же не есть [Бог] мертвых, но живых, ибо у Него все живы.
\par 39 На это некоторые из книжников сказали: Учитель! Ты хорошо сказал.
\par 40 И уже не смели спрашивать Его ни о чем. Он же сказал им:
\par 41 как говорят, что Христос есть Сын Давидов,
\par 42 а сам Давид говорит в книге псалмов: сказал Господь Господу моему: седи одесную Меня,
\par 43 доколе положу врагов Твоих в подножие ног Твоих?
\par 44 Итак, Давид Господом называет Его; как же Он Сын ему?
\par 45 И когда слушал весь народ, Он сказал ученикам Своим:
\par 46 остерегайтесь книжников, которые любят ходить в длинных одеждах и любят приветствия в народных собраниях, председания в синагогах и предвозлежания на пиршествах,
\par 47 которые поедают домы вдов и лицемерно долго молятся; они примут тем большее осуждение.

\chapter{21}

\par 1 Взглянув же, Он увидел богатых, клавших дары свои в сокровищницу;
\par 2 увидел также и бедную вдову, положившую туда две лепты,
\par 3 и сказал: истинно говорю вам, что эта бедная вдова больше всех положила;
\par 4 ибо все те от избытка своего положили в дар Богу, а она от скудости своей положила все пропитание свое, какое имела.
\par 5 И когда некоторые говорили о храме, что он украшен дорогими камнями и вкладами, Он сказал:
\par 6 придут дни, в которые из того, что вы здесь видите, не останется камня на камне; все будет разрушено.
\par 7 И спросили Его: Учитель! когда же это будет? и какой признак, когда это должно произойти?
\par 8 Он сказал: берегитесь, чтобы вас не ввели в заблуждение, ибо многие придут под именем Моим, говоря, что это Я; и это время близко: не ходите вслед их.
\par 9 Когда же услышите о войнах и смятениях, не ужасайтесь, ибо этому надлежит быть прежде; но не тотчас конец.
\par 10 Тогда сказал им: восстанет народ на народ, и царство на царство;
\par 11 будут большие землетрясения по местам, и глады, и моры, и ужасные явления, и великие знамения с неба.
\par 12 Прежде же всего того возложат на вас руки и будут гнать [вас], предавая в синагоги и в темницы, и поведут пред царей и правителей за имя Мое;
\par 13 будет же это вам для свидетельства.
\par 14 Итак положите себе на сердце не обдумывать заранее, что отвечать,
\par 15 ибо Я дам вам уста и премудрость, которой не возмогут противоречить ни противостоять все, противящиеся вам.
\par 16 Преданы также будете и родителями, и братьями, и родственниками, и друзьями, и некоторых из вас умертвят;
\par 17 и будете ненавидимы всеми за имя Мое,
\par 18 но и волос с головы вашей не пропадет, --
\par 19 терпением вашим спасайте души ваши.
\par 20 Когда же увидите Иерусалим, окруженный войсками, тогда знайте, что приблизилось запустение его:
\par 21 тогда находящиеся в Иудее да бегут в горы; и кто в городе, выходи из него; и кто в окрестностях, не входи в него,
\par 22 потому что это дни отмщения, да исполнится все написанное.
\par 23 Горе же беременным и питающим сосцами в те дни; ибо великое будет бедствие на земле и гнев на народ сей:
\par 24 и падут от острия меча, и отведутся в плен во все народы; и Иерусалим будет попираем язычниками, доколе не окончатся времена язычников.
\par 25 И будут знамения в солнце и луне и звездах, а на земле уныние народов и недоумение; и море восшумит и возмутится;
\par 26 люди будут издыхать от страха и ожидания [бедствий], грядущих на вселенную, ибо силы небесные поколеблются,
\par 27 и тогда увидят Сына Человеческого, грядущего на облаке с силою и славою великою.
\par 28 Когда же начнет это сбываться, тогда восклонитесь и поднимите головы ваши, потому что приближается избавление ваше.
\par 29 И сказал им притчу: посмотрите на смоковницу и на все деревья:
\par 30 когда они уже распускаются, то, видя это, знаете сами, что уже близко лето.
\par 31 Так, и когда вы увидите то сбывающимся, знайте, что близко Царствие Божие.
\par 32 Истинно говорю вам: не прейдет род сей, как все это будет;
\par 33 небо и земля прейдут, но слова Мои не прейдут.
\par 34 Смотрите же за собою, чтобы сердца ваши не отягчались объядением и пьянством и заботами житейскими, и чтобы день тот не постиг вас внезапно,
\par 35 ибо он, как сеть, найдет на всех живущих по всему лицу земному;
\par 36 итак бодрствуйте на всякое время и молитесь, да сподобитесь избежать всех сих будущих [бедствий] и предстать пред Сына Человеческого.
\par 37 Днем Он учил в храме, а ночи, выходя, проводил на горе, называемой Елеонскою.
\par 38 И весь народ с утра приходил к Нему в храм слушать Его.

\chapter{22}

\par 1 Приближался праздник опресноков, называемый Пасхою,
\par 2 и искали первосвященники и книжники, как бы погубить Его, потому что боялись народа.
\par 3 Вошел же сатана в Иуду, прозванного Искариотом, одного из числа двенадцати,
\par 4 и он пошел, и говорил с первосвященниками и начальниками, как Его предать им.
\par 5 Они обрадовались и согласились дать ему денег;
\par 6 и он обещал, и искал удобного времени, чтобы предать Его им не при народе.
\par 7 Настал же день опресноков, в который надлежало заколать пасхального [агнца],
\par 8 и послал [Иисус] Петра и Иоанна, сказав: пойдите, приготовьте нам есть пасху.
\par 9 Они же сказали Ему: где велишь нам приготовить?
\par 10 Он сказал им: вот, при входе вашем в город, встретится с вами человек, несущий кувшин воды; последуйте за ним в дом, в который войдет он,
\par 11 и скажите хозяину дома: Учитель говорит тебе: где комната, в которой бы Мне есть пасху с учениками Моими?
\par 12 И он покажет вам горницу большую устланную; там приготовьте.
\par 13 Они пошли, и нашли, как сказал им, и приготовили пасху.
\par 14 И когда настал час, Он возлег, и двенадцать Апостолов с Ним,
\par 15 и сказал им: очень желал Я есть с вами сию пасху прежде Моего страдания,
\par 16 ибо сказываю вам, что уже не буду есть ее, пока она не совершится в Царствии Божием.
\par 17 И, взяв чашу и благодарив, сказал: приимите ее и разделите между собою,
\par 18 ибо сказываю вам, что не буду пить от плода виноградного, доколе не придет Царствие Божие.
\par 19 И, взяв хлеб и благодарив, преломил и подал им, говоря: сие есть тело Мое, которое за вас предается; сие творите в Мое воспоминание.
\par 20 Также и чашу после вечери, говоря: сия чаша [есть] Новый Завет в Моей крови, которая за вас проливается.
\par 21 И вот, рука предающего Меня со Мною за столом;
\par 22 впрочем, Сын Человеческий идет по предназначению, но горе тому человеку, которым Он предается.
\par 23 И они начали спрашивать друг друга, кто бы из них был, который это сделает.
\par 24 Был же и спор между ними, кто из них должен почитаться большим.
\par 25 Он же сказал им: цари господствуют над народами, и владеющие ими благодетелями называются,
\par 26 а вы не так: но кто из вас больше, будь как меньший, и начальствующий--как служащий.
\par 27 Ибо кто больше: возлежащий, или служащий? не возлежащий ли? А Я посреди вас, как служащий.
\par 28 Но вы пребыли со Мною в напастях Моих,
\par 29 и Я завещаваю вам, как завещал Мне Отец Мой, Царство,
\par 30 да ядите и пиете за трапезою Моею в Царстве Моем, и сядете на престолах судить двенадцать колен Израилевых.
\par 31 И сказал Господь: Симон! Симон! се, сатана просил, чтобы сеять вас как пшеницу,
\par 32 но Я молился о тебе, чтобы не оскудела вера твоя; и ты некогда, обратившись, утверди братьев твоих.
\par 33 Он отвечал Ему: Господи! с Тобою я готов и в темницу и на смерть идти.
\par 34 Но Он сказал: говорю тебе, Петр, не пропоет петух сегодня, как ты трижды отречешься, что не знаешь Меня.
\par 35 И сказал им: когда Я посылал вас без мешка и без сумы и без обуви, имели ли вы в чем недостаток? Они отвечали: ни в чем.
\par 36 Тогда Он сказал им: но теперь, кто имеет мешок, тот возьми его, также и суму; а у кого нет, продай одежду свою и купи меч;
\par 37 ибо сказываю вам, что должно исполниться на Мне и сему написанному: и к злодеям причтен. Ибо то, что о Мне, приходит к концу.
\par 38 Они сказали: Господи! вот, здесь два меча. Он сказал им: довольно.
\par 39 И, выйдя, пошел по обыкновению на гору Елеонскую, за Ним последовали и ученики Его.
\par 40 Придя же на место, сказал им: молитесь, чтобы не впасть в искушение.
\par 41 И Сам отошел от них на вержение камня, и, преклонив колени, молился,
\par 42 говоря: Отче! о, если бы Ты благоволил пронести чашу сию мимо Меня! впрочем не Моя воля, но Твоя да будет.
\par 43 Явился же Ему Ангел с небес и укреплял Его.
\par 44 И, находясь в борении, прилежнее молился, и был пот Его, как капли крови, падающие на землю.
\par 45 Встав от молитвы, Он пришел к ученикам, и нашел их спящими от печали
\par 46 и сказал им: что вы спите? встаньте и молитесь, чтобы не впасть в искушение.
\par 47 Когда Он еще говорил это, появился народ, а впереди его шел один из двенадцати, называемый Иуда, и он подошел к Иисусу, чтобы поцеловать Его. Ибо он такой им дал знак: Кого я поцелую, Тот и есть.
\par 48 Иисус же сказал ему: Иуда! целованием ли предаешь Сына Человеческого?
\par 49 Бывшие же с Ним, видя, к чему идет дело, сказали Ему: Господи! не ударить ли нам мечом?
\par 50 И один из них ударил раба первосвященникова, и отсек ему правое ухо.
\par 51 Тогда Иисус сказал: оставьте, довольно. И, коснувшись уха его, исцелил его.
\par 52 Первосвященникам же и начальникам храма и старейшинам, собравшимся против Него, сказал Иисус: как будто на разбойника вышли вы с мечами и кольями, чтобы взять Меня?
\par 53 Каждый день бывал Я с вами в храме, и вы не поднимали на Меня рук, но теперь ваше время и власть тьмы.
\par 54 Взяв Его, повели и привели в дом первосвященника. Петр же следовал издали.
\par 55 Когда они развели огонь среди двора и сели вместе, сел и Петр между ними.
\par 56 Одна служанка, увидев его сидящего у огня и всмотревшись в него, сказала: и этот был с Ним.
\par 57 Но он отрекся от Него, сказав женщине: я не знаю Его.
\par 58 Вскоре потом другой, увидев его, сказал: и ты из них. Но Петр сказал этому человеку: нет!
\par 59 Прошло с час времени, еще некто настоятельно говорил: точно и этот был с Ним, ибо он Галилеянин.
\par 60 Но Петр сказал тому человеку: не знаю, что ты говоришь. И тотчас, когда еще говорил он, запел петух.
\par 61 Тогда Господь, обратившись, взглянул на Петра, и Петр вспомнил слово Господа, как Он сказал ему: прежде нежели пропоет петух, отречешься от Меня трижды.
\par 62 И, выйдя вон, горько заплакал.
\par 63 Люди, державшие Иисуса, ругались над Ним и били Его;
\par 64 и, закрыв Его, ударяли Его по лицу и спрашивали Его: прореки, кто ударил Тебя?
\par 65 И много иных хулений произносили против Него.
\par 66 И как настал день, собрались старейшины народа, первосвященники и книжники, и ввели Его в свой синедрион
\par 67 и сказали: Ты ли Христос? скажи нам. Он сказал им: если скажу вам, вы не поверите;
\par 68 если же и спрошу вас, не будете отвечать Мне и не отпустите [Меня];
\par 69 отныне Сын Человеческий воссядет одесную силы Божией.
\par 70 И сказали все: итак, Ты Сын Божий? Он отвечал им: вы говорите, что Я.
\par 71 Они же сказали: какое еще нужно нам свидетельство? ибо мы сами слышали из уст Его.

\chapter{23}

\par 1 И поднялось все множество их, и повели Его к Пилату,
\par 2 и начали обвинять Его, говоря: мы нашли, что Он развращает народ наш и запрещает давать подать кесарю, называя Себя Христом Царем.
\par 3 Пилат спросил Его: Ты Царь Иудейский? Он сказал ему в ответ: ты говоришь.
\par 4 Пилат сказал первосвященникам и народу: я не нахожу никакой вины в этом человеке.
\par 5 Но они настаивали, говоря, что Он возмущает народ, уча по всей Иудее, начиная от Галилеи до сего места.
\par 6 Пилат, услышав о Галилее, спросил: разве Он Галилеянин?
\par 7 И, узнав, что Он из области Иродовой, послал Его к Ироду, который в эти дни был также в Иерусалиме.
\par 8 Ирод, увидев Иисуса, очень обрадовался, ибо давно желал видеть Его, потому что много слышал о Нем, и надеялся увидеть от Него какое-нибудь чудо,
\par 9 и предлагал Ему многие вопросы, но Он ничего не отвечал ему.
\par 10 Первосвященники же и книжники стояли и усильно обвиняли Его.
\par 11 Но Ирод со своими воинами, уничижив Его и насмеявшись над Ним, одел Его в светлую одежду и отослал обратно к Пилату.
\par 12 И сделались в тот день Пилат и Ирод друзьями между собою, ибо прежде были во вражде друг с другом.
\par 13 Пилат же, созвав первосвященников и начальников и народ,
\par 14 сказал им: вы привели ко мне человека сего, как развращающего народ; и вот, я при вас исследовал и не нашел человека сего виновным ни в чем том, в чем вы обвиняете Его;
\par 15 и Ирод также, ибо я посылал Его к нему; и ничего не найдено в Нем достойного смерти;
\par 16 итак, наказав Его, отпущу.
\par 17 А ему и нужно было для праздника отпустить им одного [узника].
\par 18 Но весь народ стал кричать: смерть Ему! а отпусти нам Варавву.
\par 19 Варавва был посажен в темницу за произведенное в городе возмущение и убийство.
\par 20 Пилат снова возвысил голос, желая отпустить Иисуса.
\par 21 Но они кричали: распни, распни Его!
\par 22 Он в третий раз сказал им: какое же зло сделал Он? я ничего достойного смерти не нашел в Нем; итак, наказав Его, отпущу.
\par 23 Но они продолжали с великим криком требовать, чтобы Он был распят; и превозмог крик их и первосвященников.
\par 24 И Пилат решил быть по прошению их,
\par 25 и отпустил им посаженного за возмущение и убийство в темницу, которого они просили; а Иисуса предал в их волю.
\par 26 И когда повели Его, то, захватив некоего Симона Киринеянина, шедшего с поля, возложили на него крест, чтобы нес за Иисусом.
\par 27 И шло за Ним великое множество народа и женщин, которые плакали и рыдали о Нем.
\par 28 Иисус же, обратившись к ним, сказал: дщери Иерусалимские! не плачьте обо Мне, но плачьте о себе и о детях ваших,
\par 29 ибо приходят дни, в которые скажут: блаженны неплодные, и утробы неродившие, и сосцы непитавшие!
\par 30 тогда начнут говорить горам: падите на нас! и холмам: покройте нас!
\par 31 Ибо если с зеленеющим деревом это делают, то с сухим что будет?
\par 32 Вели с Ним на смерть и двух злодеев.
\par 33 И когда пришли на место, называемое Лобное, там распяли Его и злодеев, одного по правую, а другого по левую сторону.
\par 34 Иисус же говорил: Отче! прости им, ибо не знают, что делают. И делили одежды Его, бросая жребий.
\par 35 И стоял народ и смотрел. Насмехались же вместе с ними и начальники, говоря: других спасал; пусть спасет Себя Самого, если Он Христос, избранный Божий.
\par 36 Также и воины ругались над Ним, подходя и поднося Ему уксус
\par 37 и говоря: если Ты Царь Иудейский, спаси Себя Самого.
\par 38 И была над Ним надпись, написанная словами греческими, римскими и еврейскими: Сей есть Царь Иудейский.
\par 39 Один из повешенных злодеев злословил Его и говорил: если Ты Христос, спаси Себя и нас.
\par 40 Другой же, напротив, унимал его и говорил: или ты не боишься Бога, когда и сам осужден на то же?
\par 41 и мы [осуждены] справедливо, потому что достойное по делам нашим приняли, а Он ничего худого не сделал.
\par 42 И сказал Иисусу: помяни меня, Господи, когда приидешь в Царствие Твое!
\par 43 И сказал ему Иисус: истинно говорю тебе, ныне же будешь со Мною в раю.
\par 44 Было же около шестого часа дня, и сделалась тьма по всей земле до часа девятого:
\par 45 и померкло солнце, и завеса в храме раздралась по средине.
\par 46 Иисус, возгласив громким голосом, сказал: Отче! в руки Твои предаю дух Мой. И, сие сказав, испустил дух.
\par 47 Сотник же, видев происходившее, прославил Бога и сказал: истинно человек этот был праведник.
\par 48 И весь народ, сшедшийся на сие зрелище, видя происходившее, возвращался, бия себя в грудь.
\par 49 Все же, знавшие Его, и женщины, следовавшие за Ним из Галилеи, стояли вдали и смотрели на это.
\par 50 Тогда некто, именем Иосиф, член совета, человек добрый и правдивый,
\par 51 не участвовавший в совете и в деле их; из Аримафеи, города Иудейского, ожидавший также Царствия Божия,
\par 52 пришел к Пилату и просил тела Иисусова;
\par 53 и, сняв его, обвил плащаницею и положил его в гробе, высеченном [в скале], где еще никто не был положен.
\par 54 День тот был пятница, и наступала суббота.
\par 55 Последовали также и женщины, пришедшие с Иисусом из Галилеи, и смотрели гроб, и как полагалось тело Его;
\par 56 возвратившись же, приготовили благовония и масти; и в субботу остались в покое по заповеди.

\chapter{24}

\par 1 В первый же день недели, очень рано, неся приготовленные ароматы, пришли они ко гробу, и вместе с ними некоторые другие;
\par 2 но нашли камень отваленным от гроба.
\par 3 И, войдя, не нашли тела Господа Иисуса.
\par 4 Когда же недоумевали они о сем, вдруг предстали перед ними два мужа в одеждах блистающих.
\par 5 И когда они были в страхе и наклонили лица [свои] к земле, сказали им: что вы ищете живого между мертвыми?
\par 6 Его нет здесь: Он воскрес; вспомните, как Он говорил вам, когда был еще в Галилее,
\par 7 сказывая, что Сыну Человеческому надлежит быть предану в руки человеков грешников, и быть распяту, и в третий день воскреснуть.
\par 8 И вспомнили они слова Его;
\par 9 и, возвратившись от гроба, возвестили все это одиннадцати и всем прочим.
\par 10 То были Магдалина Мария, и Иоанна, и Мария, [мать] Иакова, и другие с ними, которые сказали о сем Апостолам.
\par 11 И показались им слова их пустыми, и не поверили им.
\par 12 Но Петр, встав, побежал ко гробу и, наклонившись, увидел только пелены лежащие, и пошел назад, дивясь сам в себе происшедшему.
\par 13 В тот же день двое из них шли в селение, отстоящее стадий на шестьдесят от Иерусалима, называемое Эммаус;
\par 14 и разговаривали между собою о всех сих событиях.
\par 15 И когда они разговаривали и рассуждали между собою, и Сам Иисус, приблизившись, пошел с ними.
\par 16 Но глаза их были удержаны, так что они не узнали Его.
\par 17 Он же сказал им: о чем это вы, идя, рассуждаете между собою, и отчего вы печальны?
\par 18 Один из них, именем Клеопа, сказал Ему в ответ: неужели Ты один из пришедших в Иерусалим не знаешь о происшедшем в нем в эти дни?
\par 19 И сказал им: о чем? Они сказали Ему: что было с Иисусом Назарянином, Который был пророк, сильный в деле и слове пред Богом и всем народом;
\par 20 как предали Его первосвященники и начальники наши для осуждения на смерть и распяли Его.
\par 21 А мы надеялись было, что Он есть Тот, Который должен избавить Израиля; но со всем тем, уже третий день ныне, как это произошло.
\par 22 Но и некоторые женщины из наших изумили нас: они были рано у гроба
\par 23 и не нашли тела Его и, придя, сказывали, что они видели и явление Ангелов, которые говорят, что Он жив.
\par 24 И пошли некоторые из наших ко гробу и нашли так, как и женщины говорили, но Его не видели.
\par 25 Тогда Он сказал им: о, несмысленные и медлительные сердцем, чтобы веровать всему, что предсказывали пророки!
\par 26 Не так ли надлежало пострадать Христу и войти в славу Свою?
\par 27 И, начав от Моисея, из всех пророков изъяснял им сказанное о Нем во всем Писании.
\par 28 И приблизились они к тому селению, в которое шли; и Он показывал им вид, что хочет идти далее.
\par 29 Но они удерживали Его, говоря: останься с нами, потому что день уже склонился к вечеру. И Он вошел и остался с ними.
\par 30 И когда Он возлежал с ними, то, взяв хлеб, благословил, преломил и подал им.
\par 31 Тогда открылись у них глаза, и они узнали Его. Но Он стал невидим для них.
\par 32 И они сказали друг другу: не горело ли в нас сердце наше, когда Он говорил нам на дороге и когда изъяснял нам Писание?
\par 33 И, встав в тот же час, возвратились в Иерусалим и нашли вместе одиннадцать [Апостолов] и бывших с ними,
\par 34 которые говорили, что Господь истинно воскрес и явился Симону.
\par 35 И они рассказывали о происшедшем на пути, и как Он был узнан ими в преломлении хлеба.
\par 36 Когда они говорили о сем, Сам Иисус стал посреди них и сказал им: мир вам.
\par 37 Они, смутившись и испугавшись, подумали, что видят духа.
\par 38 Но Он сказал им: что смущаетесь, и для чего такие мысли входят в сердца ваши?
\par 39 Посмотрите на руки Мои и на ноги Мои; это Я Сам; осяжите Меня и рассмотрите; ибо дух плоти и костей не имеет, как видите у Меня.
\par 40 И, сказав это, показал им руки и ноги.
\par 41 Когда же они от радости еще не верили и дивились, Он сказал им: есть ли у вас здесь какая пища?
\par 42 Они подали Ему часть печеной рыбы и сотового меда.
\par 43 И, взяв, ел пред ними.
\par 44 И сказал им: вот то, о чем Я вам говорил, еще быв с вами, что надлежит исполниться всему, написанному о Мне в законе Моисеевом и в пророках и псалмах.
\par 45 Тогда отверз им ум к уразумению Писаний.
\par 46 И сказал им: так написано, и так надлежало пострадать Христу, и воскреснуть из мертвых в третий день,
\par 47 и проповедану быть во имя Его покаянию и прощению грехов во всех народах, начиная с Иерусалима.
\par 48 Вы же свидетели сему.
\par 49 И Я пошлю обетование Отца Моего на вас; вы же оставайтесь в городе Иерусалиме, доколе не облечетесь силою свыше.
\par 50 И вывел их вон [из города] до Вифании и, подняв руки Свои, благословил их.
\par 51 И, когда благословлял их, стал отдаляться от них и возноситься на небо.
\par 52 Они поклонились Ему и возвратились в Иерусалим с великою радостью.
\par 53 И пребывали всегда в храме, прославляя и благословляя Бога. Аминь.


\end{document}