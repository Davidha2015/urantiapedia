\begin{document}

\title{Galatians}

Gal 1:1  Павел Апостол, [избранный] не человеками и не через человека, но Иисусом Христом и Богом Отцем, воскресившим Его из мертвых,
Gal 1:2  и все находящиеся со мною братия--церквам Галатийским:
Gal 1:3  благодать вам и мир от Бога Отца и Господа нашего Иисуса Христа,
Gal 1:4  Который отдал Себя Самого за грехи наши, чтобы избавить нас от настоящего лукавого века, по воле Бога и Отца нашего;
Gal 1:5  Ему слава во веки веков. Аминь.
Gal 1:6  Удивляюсь, что вы от призвавшего вас благодатью Христовою так скоро переходите к иному благовествованию,
Gal 1:7  которое [впрочем] не иное, а только есть люди, смущающие вас и желающие превратить благовествование Христово.
Gal 1:8  Но если бы даже мы или Ангел с неба стал благовествовать вам не то, что мы благовествовали вам, да будет анафема.
Gal 1:9  Как прежде мы сказали, [так] и теперь еще говорю: кто благовествует вам не то, что вы приняли, да будет анафема.
Gal 1:10  У людей ли я ныне ищу благоволения, или у Бога? людям ли угождать стараюсь? Если бы я и поныне угождал людям, то не был бы рабом Христовым.
Gal 1:11  Возвещаю вам, братия, что Евангелие, которое я благовествовал, не есть человеческое,
Gal 1:12  ибо и я принял его и научился не от человека, но через откровение Иисуса Христа.
Gal 1:13  Вы слышали о моем прежнем образе жизни в Иудействе, что я жестоко гнал Церковь Божию, и опустошал ее,
Gal 1:14  и преуспевал в Иудействе более многих сверстников в роде моем, будучи неумеренным ревнителем отеческих моих преданий.
Gal 1:15  Когда же Бог, избравший меня от утробы матери моей и призвавший благодатью Своею, благоволил
Gal 1:16  открыть во мне Сына Своего, чтобы я благовествовал Его язычникам, --я не стал тогда же советоваться с плотью и кровью,
Gal 1:17  и не пошел в Иерусалим к предшествовавшим мне Апостолам, а пошел в Аравию, и опять возвратился в Дамаск.
Gal 1:18  Потом, спустя три года, ходил я в Иерусалим видеться с Петром и пробыл у него дней пятнадцать.
Gal 1:19  Другого же из Апостолов я не видел [никого], кроме Иакова, брата Господня.
Gal 1:20  А в том, что пишу вам, пред Богом, не лгу.
Gal 1:21  После сего отошел я в страны Сирии и Киликии.
Gal 1:22  Церквам Христовым в Иудее лично я не был известен,
Gal 1:23  а только слышали они, что гнавший их некогда ныне благовествует веру, которую прежде истреблял, --
Gal 1:24  и прославляли за меня Бога.
Gal 2:1  Потом, через четырнадцать лет, опять ходил я в Иерусалим с Варнавою, взяв с собою и Тита.
Gal 2:2  Ходил же по откровению, и предложил там, и особо знаменитейшим, благовествование, проповедуемое мною язычникам, не напрасно ли я подвизаюсь или подвизался.
Gal 2:3  Но они и Тита, бывшего со мною, хотя и Еллина, не принуждали обрезаться,
Gal 2:4  а вкравшимся лжебратиям, скрытно приходившим подсмотреть за нашею свободою, которую мы имеем во Христе Иисусе, чтобы поработить нас,
Gal 2:5  мы ни на час не уступили и не покорились, дабы истина благовествования сохранилась у вас.
Gal 2:6  И в знаменитых чем-либо, какими бы ни были они когда-либо, для меня нет ничего особенного: Бог не взирает на лице человека. И знаменитые не возложили на меня ничего более.
Gal 2:7  Напротив того, увидев, что мне вверено благовестие для необрезанных, как Петру для обрезанных--
Gal 2:8  ибо Содействовавший Петру в апостольстве у обрезанных содействовал и мне у язычников, --
Gal 2:9  и, узнав о благодати, данной мне, Иаков и Кифа и Иоанн, почитаемые столпами, подали мне и Варнаве руку общения, чтобы нам [идти] к язычникам, а им к обрезанным,
Gal 2:10  только чтобы мы помнили нищих, что и старался я исполнять в точности.
Gal 2:11  Когда же Петр пришел в Антиохию, то я лично противостал ему, потому что он подвергался нареканию.
Gal 2:12  Ибо, до прибытия некоторых от Иакова, ел вместе с язычниками; а когда те пришли, стал таиться и устраняться, опасаясь обрезанных.
Gal 2:13  Вместе с ним лицемерили и прочие Иудеи, так что даже Варнава был увлечен их лицемерием.
Gal 2:14  Но когда я увидел, что они не прямо поступают по истине Евангельской, то сказал Петру при всех: если ты, будучи Иудеем, живешь по-язычески, а не по-иудейски, то для чего язычников принуждаешь жить по-иудейски?
Gal 2:15  Мы по природе Иудеи, а не из язычников грешники;
Gal 2:16  однако же, узнав, что человек оправдывается не делами закона, а только верою в Иисуса Христа, и мы уверовали во Христа Иисуса, чтобы оправдаться верою во Христа, а не делами закона; ибо делами закона не оправдается никакая плоть.
Gal 2:17  Если же, ища оправдания во Христе, мы и сами оказались грешниками, то неужели Христос есть служитель греха? Никак.
Gal 2:18  Ибо если я снова созидаю, что разрушил, то сам себя делаю преступником.
Gal 2:19  Законом я умер для закона, чтобы жить для Бога. Я сораспялся Христу,
Gal 2:20  и уже не я живу, но живет во мне Христос. А что ныне живу во плоти, то живу верою в Сына Божия, возлюбившего меня и предавшего Себя за меня.
Gal 2:21  Не отвергаю благодати Божией; а если законом оправдание, то Христос напрасно умер.
Gal 3:1  О, несмысленные Галаты! кто прельстил вас не покоряться истине, [вас], у которых перед глазами предначертан был Иисус Христос, [как] [бы] у вас распятый?
Gal 3:2  Сие только хочу знать от вас: через дела ли закона вы получили Духа, или через наставление в вере?
Gal 3:3  Так ли вы несмысленны, что, начав духом, теперь оканчиваете плотью?
Gal 3:4  Столь многое потерпели вы неужели без пользы? О, если бы только без пользы!
Gal 3:5  Подающий вам Духа и совершающий между вами чудеса через дела ли закона [сие производит], или через наставление в вере?
Gal 3:6  Так Авраам поверил Богу, и это вменилось ему в праведность.
Gal 3:7  Познайте же, что верующие суть сыны Авраама.
Gal 3:8  И Писание, провидя, что Бог верою оправдает язычников, предвозвестило Аврааму: в тебе благословятся все народы.
Gal 3:9  Итак верующие благословляются с верным Авраамом,
Gal 3:10  а все, утверждающиеся на делах закона, находятся под клятвою. Ибо написано: проклят всяк, кто не исполняет постоянно всего, что написано в книге закона.
Gal 3:11  А что законом никто не оправдывается пред Богом, это ясно, потому что праведный верою жив будет.
Gal 3:12  А закон не по вере; но кто исполняет его, тот жив будет им.
Gal 3:13  Христос искупил нас от клятвы закона, сделавшись за нас клятвою--ибо написано: проклят всяк, висящий на древе, --
Gal 3:14  дабы благословение Авраамово через Христа Иисуса распространилось на язычников, чтобы нам получить обещанного Духа верою.
Gal 3:15  Братия! говорю по [рассуждению] человеческому: даже человеком утвержденного завещания никто не отменяет и не прибавляет [к нему].
Gal 3:16  Но Аврааму даны были обетования и семени его. Не сказано: и потомкам, как бы о многих, но как об одном: и семени твоему, которое есть Христос.
Gal 3:17  Я говорю то, что завета о Христе, прежде Богом утвержденного, закон, явившийся спустя четыреста тридцать лет, не отменяет так, чтобы обетование потеряло силу.
Gal 3:18  Ибо если по закону наследство, то уже не по обетованию; но Аврааму Бог даровал [оное] по обетованию.
Gal 3:19  Для чего же закон? Он дан после по причине преступлений, до времени пришествия семени, к которому [относится] обетование, и преподан через Ангелов, рукою посредника.
Gal 3:20  Но посредник при одном не бывает, а Бог один.
Gal 3:21  Итак закон противен обетованиям Божиим? Никак! Ибо если бы дан был закон, могущий животворить, то подлинно праведность была бы от закона;
Gal 3:22  но Писание всех заключило под грехом, дабы обетование верующим дано было по вере в Иисуса Христа.
Gal 3:23  А до пришествия веры мы заключены были под стражею закона, до того [времени], как надлежало открыться вере.
Gal 3:24  Итак закон был для нас детоводителем ко Христу, дабы нам оправдаться верою;
Gal 3:25  по пришествии же веры, мы уже не под [руководством] детоводителя.
Gal 3:26  Ибо все вы сыны Божии по вере во Христа Иисуса;
Gal 3:27  все вы, во Христа крестившиеся, во Христа облеклись.
Gal 3:28  Нет уже Иудея, ни язычника; нет раба, ни свободного; нет мужеского пола, ни женского: ибо все вы одно во Христе Иисусе.
Gal 3:29  Если же вы Христовы, то вы семя Авраамово и по обетованию наследники.
Gal 4:1  Еще скажу: наследник, доколе в детстве, ничем не отличается от раба, хотя и господин всего:
Gal 4:2  он подчинен попечителям и домоправителям до срока, отцом [назначенного].
Gal 4:3  Так и мы, доколе были в детстве, были порабощены вещественным началам мира;
Gal 4:4  но когда пришла полнота времени, Бог послал Сына Своего (Единородного), Который родился от жены, подчинился закону,
Gal 4:5  чтобы искупить подзаконных, дабы нам получить усыновление.
Gal 4:6  А как вы--сыны, то Бог послал в сердца ваши Духа Сына Своего, вопиющего: `Авва, Отче!'
Gal 4:7  Посему ты уже не раб, но сын; а если сын, то и наследник Божий через Иисуса Христа.
Gal 4:8  Но тогда, не знав Бога, вы служили [богам], которые в существе не боги.
Gal 4:9  Ныне же, познав Бога, или, лучше, получив познание от Бога, для чего возвращаетесь опять к немощным и бедным вещественным началам и хотите еще снова поработить себя им?
Gal 4:10  Наблюдаете дни, месяцы, времена и годы.
Gal 4:11  Боюсь за вас, не напрасно ли я трудился у вас.
Gal 4:12  Прошу вас, братия, будьте, как я, потому что и я, как вы. Вы ничем не обидели меня:
Gal 4:13  знаете, что, [хотя] я в немощи плоти благовествовал вам в первый раз,
Gal 4:14  но вы не презрели искушения моего во плоти моей и не возгнушались [им], а приняли меня, как Ангела Божия, как Христа Иисуса.
Gal 4:15  Как вы были блаженны! Свидетельствую о вас, что, если бы возможно было, вы исторгли бы очи свои и отдали мне.
Gal 4:16  Итак, неужели я сделался врагом вашим, говоря вам истину?
Gal 4:17  Ревнуют по вас нечисто, а хотят вас отлучить, чтобы вы ревновали по них.
Gal 4:18  Хорошо ревновать в добром всегда, а не в моем только присутствии у вас.
Gal 4:19  Дети мои, для которых я снова в муках рождения, доколе не изобразится в вас Христос!
Gal 4:20  Хотел бы я теперь быть у вас и изменить голос мой, потому что я в недоумении о вас.
Gal 4:21  Скажите мне вы, желающие быть под законом: разве вы не слушаете закона?
Gal 4:22  Ибо написано: Авраам имел двух сынов, одного от рабы, а другого от свободной.
Gal 4:23  Но который от рабы, тот рожден по плоти; а который от свободной, тот по обетованию.
Gal 4:24  В этом есть иносказание. Это два завета: один от горы Синайской, рождающий в рабство, который есть Агарь,
Gal 4:25  ибо Агарь означает гору Синай в Аравии и соответствует нынешнему Иерусалиму, потому что он с детьми своими в рабстве;
Gal 4:26  а вышний Иерусалим свободен: он--матерь всем нам.
Gal 4:27  Ибо написано: возвеселись, неплодная, нерождающая; воскликни и возгласи, не мучившаяся родами; потому что у оставленной гораздо более детей, нежели у имеющей мужа.
Gal 4:28  Мы, братия, дети обетования по Исааку.
Gal 4:29  Но, как тогда рожденный по плоти гнал [рожденного] по духу, так и ныне.
Gal 4:30  Что же говорит Писание? Изгони рабу и сына ее, ибо сын рабы не будет наследником вместе с сыном свободной.
Gal 4:31  Итак, братия, мы дети не рабы, но свободной.
Gal 5:1  Итак стойте в свободе, которую даровал нам Христос, и не подвергайтесь опять игу рабства.
Gal 5:2  Вот, я, Павел, говорю вам: если вы обрезываетесь, не будет вам никакой пользы от Христа.
Gal 5:3  Еще свидетельствую всякому человеку обрезывающемуся, что он должен исполнить весь закон.
Gal 5:4  Вы, оправдывающие себя законом, остались без Христа, отпали от благодати,
Gal 5:5  а мы духом ожидаем и надеемся праведности от веры.
Gal 5:6  Ибо во Христе Иисусе не имеет силы ни обрезание, ни необрезание, но вера, действующая любовью.
Gal 5:7  Вы шли хорошо: кто остановил вас, чтобы вы не покорялись истине?
Gal 5:8  Такое убеждение не от Призывающего вас.
Gal 5:9  Малая закваска заквашивает все тесто.
Gal 5:10  Я уверен о вас в Господе, что вы не будете мыслить иначе; а смущающий вас, кто бы он ни был, понесет на себе осуждение.
Gal 5:11  За что же гонят меня, братия, если я и теперь проповедую обрезание? Тогда соблазн креста прекратился бы.
Gal 5:12  О, если бы удалены были возмущающие вас!
Gal 5:13  К свободе призваны вы, братия, только бы свобода ваша не была поводом к [угождению] плоти, но любовью служите друг другу.
Gal 5:14  Ибо весь закон в одном слове заключается: люби ближнего твоего, как самого себя.
Gal 5:15  Если же друг друга угрызаете и съедаете, берегитесь, чтобы вы не были истреблены друг другом.
Gal 5:16  Я говорю: поступайте по духу, и вы не будете исполнять вожделений плоти,
Gal 5:17  ибо плоть желает противного духу, а дух--противного плоти: они друг другу противятся, так что вы не то делаете, что хотели бы.
Gal 5:18  Если же вы духом водитесь, то вы не под законом.
Gal 5:19  Дела плоти известны; они суть: прелюбодеяние, блуд, нечистота, непотребство,
Gal 5:20  идолослужение, волшебство, вражда, ссоры, зависть, гнев, распри, разногласия, (соблазны), ереси,
Gal 5:21  ненависть, убийства, пьянство, бесчинство и тому подобное. Предваряю вас, как и прежде предварял, что поступающие так Царствия Божия не наследуют.
Gal 5:22  Плод же духа: любовь, радость, мир, долготерпение, благость, милосердие, вера,
Gal 5:23  кротость, воздержание. На таковых нет закона.
Gal 5:24  Но те, которые Христовы, распяли плоть со страстями и похотями.
Gal 5:25  Если мы живем духом, то по духу и поступать должны.
Gal 5:26  Не будем тщеславиться, друг друга раздражать, друг другу завидовать.
Gal 6:1  Братия! если и впадет человек в какое согрешение, вы, духовные, исправляйте такового в духе кротости, наблюдая каждый за собою, чтобы не быть искушенным.
Gal 6:2  Носите бремена друг друга, и таким образом исполните закон Христов.
Gal 6:3  Ибо кто почитает себя чем-нибудь, будучи ничто, тот обольщает сам себя.
Gal 6:4  Каждый да испытывает свое дело, и тогда будет иметь похвалу только в себе, а не в другом,
Gal 6:5  ибо каждый понесет свое бремя.
Gal 6:6  Наставляемый словом, делись всяким добром с наставляющим.
Gal 6:7  Не обманывайтесь: Бог поругаем не бывает. Что посеет человек, то и пожнет:
Gal 6:8  сеющий в плоть свою от плоти пожнет тление, а сеющий в дух от духа пожнет жизнь вечную.
Gal 6:9  Делая добро, да не унываем, ибо в свое время пожнем, если не ослабеем.
Gal 6:10  Итак, доколе есть время, будем делать добро всем, а наипаче своим по вере.
Gal 6:11  Видите, как много написал я вам своею рукою.
Gal 6:12  Желающие хвалиться по плоти принуждают вас обрезываться только для того, чтобы не быть гонимыми за крест Христов,
Gal 6:13  ибо и сами обрезывающиеся не соблюдают закона, но хотят, чтобы вы обрезывались, дабы похвалиться в вашей плоти.
Gal 6:14  А я не желаю хвалиться, разве только крестом Господа нашего Иисуса Христа, которым для меня мир распят, и я для мира.
Gal 6:15  Ибо во Христе Иисусе ничего не значит ни обрезание, ни необрезание, а новая тварь.
Gal 6:16  Тем, которые поступают по сему правилу, мир им и милость, и Израилю Божию.
Gal 6:17  Впрочем никто не отягощай меня, ибо я ношу язвы Господа Иисуса на теле моем.
Gal 6:18  Благодать Господа нашего Иисуса Христа со духом вашим, братия. Аминь.


\end{document}