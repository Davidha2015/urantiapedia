\begin{document}

\title{Послание к Ефесянам}


\chapter{1}

\par 1 Павел, волею Божиею Апостол Иисуса Христа, находящимся в Ефесе святым и верным во Христе Иисусе:
\par 2 благодать вам и мир от Бога Отца нашего и Господа Иисуса Христа.
\par 3 Благословен Бог и Отец Господа нашего Иисуса Христа, благословивший нас во Христе всяким духовным благословением в небесах,
\par 4 так как Он избрал нас в Нем прежде создания мира, чтобы мы были святы и непорочны пред Ним в любви,
\par 5 предопределив усыновить нас Себе чрез Иисуса Христа, по благоволению воли Своей,
\par 6 в похвалу славы благодати Своей, которою Он облагодатствовал нас в Возлюбленном,
\par 7 в Котором мы имеем искупление Кровию Его, прощение грехов, по богатству благодати Его,
\par 8 каковую Он в преизбытке даровал нам во всякой премудрости и разумении,
\par 9 открыв нам тайну Своей воли по Своему благоволению, которое Он прежде положил в Нем,
\par 10 в устроении полноты времен, дабы все небесное и земное соединить под главою Христом.
\par 11 В Нем мы и сделались наследниками, быв предназначены [к тому] по определению Совершающего все по изволению воли Своей,
\par 12 дабы послужить к похвале славы Его нам, которые ранее уповали на Христа.
\par 13 В Нем и вы, услышав слово истины, благовествование вашего спасения, и уверовав в Него, запечатлены обетованным Святым Духом,
\par 14 Который есть залог наследия нашего, для искупления удела [Его], в похвалу славы Его.
\par 15 Посему и я, услышав о вашей вере во Христа Иисуса и о любви ко всем святым,
\par 16 непрестанно благодарю за вас [Бога], вспоминая о вас в молитвах моих,
\par 17 чтобы Бог Господа нашего Иисуса Христа, Отец славы, дал вам Духа премудрости и откровения к познанию Его,
\par 18 и просветил очи сердца вашего, дабы вы познали, в чем состоит надежда призвания Его, и какое богатство славного наследия Его для святых,
\par 19 и как безмерно величие могущества Его в нас, верующих по действию державной силы Его,
\par 20 которою Он воздействовал во Христе, воскресив Его из мертвых и посадив одесную Себя на небесах,
\par 21 превыше всякого Начальства, и Власти, и Силы, и Господства, и всякого имени, именуемого не только в сем веке, но и в будущем,
\par 22 и все покорил под ноги Его, и поставил Его выше всего, главою Церкви,
\par 23 которая есть Тело Его, полнота Наполняющего все во всем.

\chapter{2}

\par 1 И вас, мертвых по преступлениям и грехам вашим,
\par 2 в которых вы некогда жили, по обычаю мира сего, по [воле] князя, господствующего в воздухе, духа, действующего ныне в сынах противления,
\par 3 между которыми и мы все жили некогда по нашим плотским похотям, исполняя желания плоти и помыслов, и были по природе чадами гнева, как и прочие,
\par 4 Бог, богатый милостью, по Своей великой любви, которою возлюбил нас,
\par 5 и нас, мертвых по преступлениям, оживотворил со Христом, --благодатью вы спасены, --
\par 6 и воскресил с Ним, и посадил на небесах во Христе Иисусе,
\par 7 дабы явить в грядущих веках преизобильное богатство благодати Своей в благости к нам во Христе Иисусе.
\par 8 Ибо благодатью вы спасены через веру, и сие не от вас, Божий дар:
\par 9 не от дел, чтобы никто не хвалился.
\par 10 Ибо мы--Его творение, созданы во Христе Иисусе на добрые дела, которые Бог предназначил нам исполнять.
\par 11 Итак помните, что вы, некогда язычники по плоти, которых называли необрезанными так называемые обрезанные плотским [обрезанием], совершаемым руками,
\par 12 что вы были в то время без Христа, отчуждены от общества Израильского, чужды заветов обетования, не имели надежды и были безбожники в мире.
\par 13 А теперь во Христе Иисусе вы, бывшие некогда далеко, стали близки Кровию Христовою.
\par 14 Ибо Он есть мир наш, соделавший из обоих одно и разрушивший стоявшую посреди преграду,
\par 15 упразднив вражду Плотию Своею, а закон заповедей учением, дабы из двух создать в Себе Самом одного нового человека, устрояя мир,
\par 16 и в одном теле примирить обоих с Богом посредством креста, убив вражду на нем.
\par 17 И, придя, благовествовал мир вам, дальним и близким,
\par 18 потому что через Него и те и другие имеем доступ к Отцу, в одном Духе.
\par 19 Итак вы уже не чужие и не пришельцы, но сограждане святым и свои Богу,
\par 20 быв утверждены на основании Апостолов и пророков, имея Самого Иисуса Христа краеугольным [камнем],
\par 21 на котором все здание, слагаясь стройно, возрастает в святый храм в Господе,
\par 22 на котором и вы устрояетесь в жилище Божие Духом.

\chapter{3}

\par 1 Для сего-то я, Павел, [сделался] узником Иисуса Христа за вас язычников.
\par 2 Как вы слышали о домостроительстве благодати Божией, данной мне для вас,
\par 3 потому что мне через откровение возвещена тайна (о чем я и выше писал кратко),
\par 4 то вы, читая, можете усмотреть мое разумение тайны Христовой,
\par 5 которая не была возвещена прежним поколениям сынов человеческих, как ныне открыта святым Апостолам Его и пророкам Духом Святым,
\par 6 чтобы и язычникам быть сонаследниками, составляющими одно тело, и сопричастниками обетования Его во Христе Иисусе посредством благовествования,
\par 7 которого служителем сделался я по дару благодати Божией, данной мне действием силы Его.
\par 8 Мне, наименьшему из всех святых, дана благодать сия--благовествовать язычникам неисследимое богатство Христово
\par 9 и открыть всем, в чем состоит домостроительство тайны, сокрывавшейся от вечности в Боге, создавшем все Иисусом Христом,
\par 10 дабы ныне соделалась известною через Церковь начальствам и властям на небесах многоразличная премудрость Божия,
\par 11 по предвечному определению, которое Он исполнил во Христе Иисусе, Господе нашем,
\par 12 в Котором мы имеем дерзновение и надежный доступ через веру в Него.
\par 13 Посему прошу вас не унывать при моих ради вас скорбях, которые суть ваша слава.
\par 14 Для сего преклоняю колени мои пред Отцем Господа нашего Иисуса Христа,
\par 15 от Которого именуется всякое отечество на небесах и на земле,
\par 16 да даст вам, по богатству славы Своей, крепко утвердиться Духом Его во внутреннем человеке,
\par 17 верою вселиться Христу в сердца ваши,
\par 18 чтобы вы, укорененные и утвержденные в любви, могли постигнуть со всеми святыми, что широта и долгота, и глубина и высота,
\par 19 и уразуметь превосходящую разумение любовь Христову, дабы вам исполниться всею полнотою Божиею.
\par 20 А Тому, Кто действующею в нас силою может сделать несравненно больше всего, чего мы просим, или о чем помышляем,
\par 21 Тому слава в Церкви во Христе Иисусе во все роды, от века до века. Аминь.

\chapter{4}

\par 1 Итак я, узник в Господе, умоляю вас поступать достойно звания, в которое вы призваны,
\par 2 со всяким смиренномудрием и кротостью и долготерпением, снисходя друг ко другу любовью,
\par 3 стараясь сохранять единство духа в союзе мира.
\par 4 Одно тело и один дух, как вы и призваны к одной надежде вашего звания;
\par 5 один Господь, одна вера, одно крещение,
\par 6 один Бог и Отец всех, Который над всеми, и через всех, и во всех нас.
\par 7 Каждому же из нас дана благодать по мере дара Христова.
\par 8 Посему и сказано: восшед на высоту, пленил плен и дал дары человекам.
\par 9 А `восшел' что означает, как не то, что Он и нисходил прежде в преисподние места земли?
\par 10 Нисшедший, Он же есть и восшедший превыше всех небес, дабы наполнить все.
\par 11 И Он поставил одних Апостолами, других пророками, иных Евангелистами, иных пастырями и учителями,
\par 12 к совершению святых, на дело служения, для созидания Тела Христова,
\par 13 доколе все придем в единство веры и познания Сына Божия, в мужа совершенного, в меру полного возраста Христова;
\par 14 дабы мы не были более младенцами, колеблющимися и увлекающимися всяким ветром учения, по лукавству человеков, по хитрому искусству обольщения,
\par 15 но истинною любовью все возращали в Того, Который есть глава Христос,
\par 16 из Которого все тело, составляемое и совокупляемое посредством всяких взаимно скрепляющих связей, при действии в свою меру каждого члена, получает приращение для созидания самого себя в любви.
\par 17 Посему я говорю и заклинаю Господом, чтобы вы более не поступали, как поступают прочие народы, по суетности ума своего,
\par 18 будучи помрачены в разуме, отчуждены от жизни Божией, по причине их невежества и ожесточения сердца их.
\par 19 Они, дойдя до бесчувствия, предались распутству так, что делают всякую нечистоту с ненасытимостью.
\par 20 Но вы не так познали Христа;
\par 21 потому что вы слышали о Нем и в Нем научились, --так как истина во Иисусе, --
\par 22 отложить прежний образ жизни ветхого человека, истлевающего в обольстительных похотях,
\par 23 а обновиться духом ума вашего
\par 24 и облечься в нового человека, созданного по Богу, в праведности и святости истины.
\par 25 Посему, отвергнув ложь, говорите истину каждый ближнему своему, потому что мы члены друг другу.
\par 26 Гневаясь, не согрешайте: солнце да не зайдет во гневе вашем;
\par 27 и не давайте места диаволу.
\par 28 Кто крал, вперед не кради, а лучше трудись, делая своими руками полезное, чтобы было из чего уделять нуждающемуся.
\par 29 Никакое гнилое слово да не исходит из уст ваших, а только доброе для назидания в вере, дабы оно доставляло благодать слушающим.
\par 30 И не оскорбляйте Святаго Духа Божия, Которым вы запечатлены в день искупления.
\par 31 Всякое раздражение и ярость, и гнев, и крик, и злоречие со всякою злобою да будут удалены от вас;
\par 32 но будьте друг ко другу добры, сострадательны, прощайте друг друга, как и Бог во Христе простил вас.

\chapter{5}

\par 1 Итак, подражайте Богу, как чада возлюбленные,
\par 2 и живите в любви, как и Христос возлюбил нас и предал Себя за нас в приношение и жертву Богу, в благоухание приятное.
\par 3 А блуд и всякая нечистота и любостяжание не должны даже именоваться у вас, как прилично святым.
\par 4 Также сквернословие и пустословие и смехотворство не приличны [вам], а, напротив, благодарение;
\par 5 ибо знайте, что никакой блудник, или нечистый, или любостяжатель, который есть идолослужитель, не имеет наследия в Царстве Христа и Бога.
\par 6 Никто да не обольщает вас пустыми словами, ибо за это приходит гнев Божий на сынов противления;
\par 7 итак, не будьте сообщниками их.
\par 8 Вы были некогда тьма, а теперь--свет в Господе: поступайте, как чада света,
\par 9 потому что плод Духа состоит во всякой благости, праведности и истине.
\par 10 Испытывайте, что благоугодно Богу,
\par 11 и не участвуйте в бесплодных делах тьмы, но и обличайте.
\par 12 Ибо о том, что они делают тайно, стыдно и говорить.
\par 13 Все же обнаруживаемое делается явным от света, ибо все, делающееся явным, свет есть.
\par 14 Посему сказано: `встань, спящий, и воскресни из мертвых, и осветит тебя Христос'.
\par 15 Итак, смотрите, поступайте осторожно, не как неразумные, но как мудрые,
\par 16 дорожа временем, потому что дни лукавы.
\par 17 Итак, не будьте нерассудительны, но познавайте, что есть воля Божия.
\par 18 И не упивайтесь вином, от которого бывает распутство; но исполняйтесь Духом,
\par 19 назидая самих себя псалмами и славословиями и песнопениями духовными, поя и воспевая в сердцах ваших Господу,
\par 20 благодаря всегда за все Бога и Отца, во имя Господа нашего Иисуса Христа,
\par 21 повинуясь друг другу в страхе Божием.
\par 22 Жены, повинуйтесь своим мужьям, как Господу,
\par 23 потому что муж есть глава жены, как и Христос глава Церкви, и Он же Спаситель тела.
\par 24 Но как Церковь повинуется Христу, так и жены своим мужьям во всем.
\par 25 Мужья, любите своих жен, как и Христос возлюбил Церковь и предал Себя за нее,
\par 26 чтобы освятить ее, очистив банею водною посредством слова;
\par 27 чтобы представить ее Себе славною Церковью, не имеющею пятна, или порока, или чего-либо подобного, но дабы она была свята и непорочна.
\par 28 Так должны мужья любить своих жен, как свои тела: любящий свою жену любит самого себя.
\par 29 Ибо никто никогда не имел ненависти к своей плоти, но питает и греет ее, как и Господь Церковь,
\par 30 потому что мы члены тела Его, от плоти Его и от костей Его.
\par 31 Посему оставит человек отца своего и мать и прилепится к жене своей, и будут двое одна плоть.
\par 32 Тайна сия велика; я говорю по отношению ко Христу и к Церкви.
\par 33 Так каждый из вас да любит свою жену, как самого себя; а жена да боится своего мужа.

\chapter{6}

\par 1 Дети, повинуйтесь своим родителям в Господе, ибо сего [требует] справедливость.
\par 2 Почитай отца твоего и мать, это первая заповедь с обетованием:
\par 3 да будет тебе благо, и будешь долголетен на земле.
\par 4 И вы, отцы, не раздражайте детей ваших, но воспитывайте их в учении и наставлении Господнем.
\par 5 Рабы, повинуйтесь господам своим по плоти со страхом и трепетом, в простоте сердца вашего, как Христу,
\par 6 не с видимою только услужливостью, как человекоугодники, но как рабы Христовы, исполняя волю Божию от души,
\par 7 служа с усердием, как Господу, а не как человекам,
\par 8 зная, что каждый получит от Господа по мере добра, которое он сделал, раб ли, или свободный.
\par 9 И вы, господа, поступайте с ними так же, умеряя строгость, зная, что и над вами самими и над ними есть на небесах Господь, у Которого нет лицеприятия.
\par 10 Наконец, братия мои, укрепляйтесь Господом и могуществом силы Его.
\par 11 Облекитесь во всеоружие Божие, чтобы вам можно было стать против козней диавольских,
\par 12 потому что наша брань не против крови и плоти, но против начальств, против властей, против мироправителей тьмы века сего, против духов злобы поднебесной.
\par 13 Для сего приимите всеоружие Божие, дабы вы могли противостать в день злый и, все преодолев, устоять.
\par 14 Итак станьте, препоясав чресла ваши истиною и облекшись в броню праведности,
\par 15 и обув ноги в готовность благовествовать мир;
\par 16 а паче всего возьмите щит веры, которым возможете угасить все раскаленные стрелы лукавого;
\par 17 и шлем спасения возьмите, и меч духовный, который есть Слово Божие.
\par 18 Всякою молитвою и прошением молитесь во всякое время духом, и старайтесь о сем самом со всяким постоянством и молением о всех святых
\par 19 и о мне, дабы мне дано было слово--устами моими открыто с дерзновением возвещать тайну благовествования,
\par 20 для которого я исполняю посольство в узах, дабы я смело проповедывал, как мне должно.
\par 21 А дабы и вы знали о моих обстоятельствах и делах, обо всем известит вас Тихик, возлюбленный брат и верный в Господе служитель,
\par 22 которого я и послал к вам для того самого, чтобы вы узнали о нас и чтобы он утешил сердца ваши.
\par 23 Мир братиям и любовь с верою от Бога Отца и Господа Иисуса Христа.
\par 24 Благодать со всеми, неизменно любящими Господа нашего Иисуса Христа. Аминь.


\end{document}