\begin{document}

\title{Philippians}

Php 1:1  Павел и Тимофей, рабы Иисуса Христа, всем святым во Христе Иисусе, находящимся в Филиппах, с епископами и диаконами:
Php 1:2  благодать вам и мир от Бога Отца нашего и Господа Иисуса Христа.
Php 1:3  Благодарю Бога моего при всяком воспоминании о вас,
Php 1:4  всегда во всякой молитве моей за всех вас принося с радостью молитву мою,
Php 1:5  за ваше участие в благовествовании от первого дня даже доныне,
Php 1:6  будучи уверен в том, что начавший в вас доброе дело будет совершать его даже до дня Иисуса Христа,
Php 1:7  как и должно мне помышлять о всех вас, потому что я имею вас в сердце в узах моих, при защищении и утверждении благовествования, вас всех, как соучастников моих в благодати.
Php 1:8  Бог--свидетель, что я люблю всех вас любовью Иисуса Христа;
Php 1:9  и молюсь о том, чтобы любовь ваша еще более и более возрастала в познании и всяком чувстве,
Php 1:10  чтобы, познавая лучшее, вы были чисты и непреткновенны в день Христов,
Php 1:11  исполнены плодов праведности Иисусом Христом, в славу и похвалу Божию.
Php 1:12  Желаю, братия, чтобы вы знали, что обстоятельства мои послужили к большему успеху благовествования,
Php 1:13  так что узы мои о Христе сделались известными всей претории и всем прочим,
Php 1:14  и большая часть из братьев в Господе, ободрившись узами моими, начали с большею смелостью, безбоязненно проповедывать слово Божие.
Php 1:15  Некоторые, правда, по зависти и любопрению, а другие с добрым расположением проповедуют Христа.
Php 1:16  Одни по любопрению проповедуют Христа не чисто, думая увеличить тяжесть уз моих;
Php 1:17  а другие--из любви, зная, что я поставлен защищать благовествование.
Php 1:18  Но что до того? Как бы ни проповедали Христа, притворно или искренно, я и тому радуюсь и буду радоваться,
Php 1:19  ибо знаю, что это послужит мне во спасение по вашей молитве и содействием Духа Иисуса Христа,
Php 1:20  при уверенности и надежде моей, что я ни в чем посрамлен не буду, но при всяком дерзновении, и ныне, как и всегда, возвеличится Христос в теле моем, жизнью ли то, или смертью.
Php 1:21  Ибо для меня жизнь--Христос, и смерть--приобретение.
Php 1:22  Если же жизнь во плоти [доставляет] плод моему делу, то не знаю, что избрать.
Php 1:23  Влечет меня то и другое: имею желание разрешиться и быть со Христом, потому что это несравненно лучше;
Php 1:24  а оставаться во плоти нужнее для вас.
Php 1:25  И я верно знаю, что останусь и пребуду со всеми вами для вашего успеха и радости в вере,
Php 1:26  дабы похвала ваша во Христе Иисусе умножилась через меня, при моем вторичном к вам пришествии.
Php 1:27  Только живите достойно благовествования Христова, чтобы мне, приду ли я и увижу вас, или не приду, слышать о вас, что вы стоите в одном духе, подвизаясь единодушно за веру Евангельскую,
Php 1:28  и не страшитесь ни в чем противников: это для них есть предзнаменование погибели, а для вас--спасения. И сие от Бога,
Php 1:29  потому что вам дано ради Христа не только веровать в Него, но и страдать за Него
Php 1:30  таким же подвигом, какой вы видели во мне и ныне слышите о мне.
Php 2:1  Итак, если [есть] какое утешение во Христе, если [есть] какая отрада любви, если [есть] какое общение духа, если [есть] какое милосердие и сострадательность,
Php 2:2  то дополните мою радость: имейте одни мысли, имейте ту же любовь, будьте единодушны и единомысленны;
Php 2:3  ничего [не делайте] по любопрению или по тщеславию, но по смиренномудрию почитайте один другого высшим себя.
Php 2:4  Не о себе [только] каждый заботься, но каждый и о других.
Php 2:5  Ибо в вас должны быть те же чувствования, какие и во Христе Иисусе:
Php 2:6  Он, будучи образом Божиим, не почитал хищением быть равным Богу;
Php 2:7  но уничижил Себя Самого, приняв образ раба, сделавшись подобным человекам и по виду став как человек;
Php 2:8  смирил Себя, быв послушным даже до смерти, и смерти крестной.
Php 2:9  Посему и Бог превознес Его и дал Ему имя выше всякого имени,
Php 2:10  дабы пред именем Иисуса преклонилось всякое колено небесных, земных и преисподних,
Php 2:11  и всякий язык исповедал, что Господь Иисус Христос в славу Бога Отца.
Php 2:12  Итак, возлюбленные мои, как вы всегда были послушны, не только в присутствии моем, но гораздо более ныне во время отсутствия моего, со страхом и трепетом совершайте свое спасение,
Php 2:13  потому что Бог производит в вас и хотение и действие по [Своему] благоволению.
Php 2:14  Все делайте без ропота и сомнения,
Php 2:15  чтобы вам быть неукоризненными и чистыми, чадами Божиими непорочными среди строптивого и развращенного рода, в котором вы сияете, как светила в мире,
Php 2:16  содержа слово жизни, к похвале моей в день Христов, что я не тщетно подвизался и не тщетно трудился.
Php 2:17  Но если я и соделываюсь жертвою за жертву и служение веры вашей, то радуюсь и сорадуюсь всем вам.
Php 2:18  О сем самом и вы радуйтесь и сорадуйтесь мне.
Php 2:19  Надеюсь же в Господе Иисусе вскоре послать к вам Тимофея, дабы и я, узнав о ваших обстоятельствах, утешился духом.
Php 2:20  Ибо я не имею никого равно усердного, кто бы столь искренно заботился о вас,
Php 2:21  потому что все ищут своего, а не того, что [угодно] Иисусу Христу.
Php 2:22  А его верность вам известна, потому что он, как сын отцу, служил мне в благовествовании.
Php 2:23  Итак я надеюсь послать его тотчас же, как скоро узнаю, что будет со мною.
Php 2:24  Я уверен в Господе, что и сам скоро приду к вам.
Php 2:25  Впрочем я почел нужным послать к вам Епафродита, брата и сотрудника и сподвижника моего, а вашего посланника и служителя в нужде моей,
Php 2:26  потому что он сильно желал видеть всех вас и тяжко скорбел о том, что до вас дошел слух о его болезни.
Php 2:27  Ибо он был болен при смерти; но Бог помиловал его, и не его только, но и меня, чтобы не прибавилась мне печаль к печали.
Php 2:28  Посему я скорее послал его, чтобы вы, увидев его снова, возрадовались, и я был менее печален.
Php 2:29  Примите же его в Господе со всякою радостью, и таких имейте в уважении,
Php 2:30  ибо он за дело Христово был близок к смерти, подвергая опасности жизнь, дабы восполнить недостаток ваших услуг мне.
Php 3:1  Впрочем, братия мои, радуйтесь о Господе. Писать вам о том же для меня не тягостно, а для вас назидательно.
Php 3:2  Берегитесь псов, берегитесь злых делателей, берегитесь обрезания,
Php 3:3  потому что обрезание--мы, служащие Богу духом и хвалящиеся Христом Иисусом, и не на плоть надеющиеся,
Php 3:4  хотя я могу надеяться и на плоть. Если кто другой думает надеяться на плоть, то более я,
Php 3:5  обрезанный в восьмой день, из рода Израилева, колена Вениаминова, Еврей от Евреев, по учению фарисей,
Php 3:6  по ревности--гонитель Церкви Божией, по правде законной--непорочный.
Php 3:7  Но что для меня было преимуществом, то ради Христа я почел тщетою.
Php 3:8  Да и все почитаю тщетою ради превосходства познания Христа Иисуса, Господа моего: для Него я от всего отказался, и все почитаю за сор, чтобы приобрести Христа
Php 3:9  и найтись в Нем не со своею праведностью, которая от закона, но с тою, которая через веру во Христа, с праведностью от Бога по вере;
Php 3:10  чтобы познать Его, и силу воскресения Его, и участие в страданиях Его, сообразуясь смерти Его,
Php 3:11  чтобы достигнуть воскресения мертвых.
Php 3:12  [Говорю так] не потому, чтобы я уже достиг, или усовершился; но стремлюсь, не достигну ли я, как достиг меня Христос Иисус.
Php 3:13  Братия, я не почитаю себя достигшим; а только, забывая заднее и простираясь вперед,
Php 3:14  стремлюсь к цели, к почести вышнего звания Божия во Христе Иисусе.
Php 3:15  Итак, кто из нас совершен, так должен мыслить; если же вы о чем иначе мыслите, то и это Бог вам откроет.
Php 3:16  Впрочем, до чего мы достигли, так и должны мыслить и по тому правилу жить.
Php 3:17  Подражайте, братия, мне и смотрите на тех, которые поступают по образу, какой имеете в нас.
Php 3:18  Ибо многие, о которых я часто говорил вам, а теперь даже со слезами говорю, поступают как враги креста Христова.
Php 3:19  Их конец--погибель, их бог--чрево, и слава их--в сраме, они мыслят о земном.
Php 3:20  Наше же жительство--на небесах, откуда мы ожидаем и Спасителя, Господа нашего Иисуса Христа,
Php 3:21  Который уничиженное тело наше преобразит так, что оно будет сообразно славному телу Его, силою, [которою] Он действует и покоряет Себе все.
Php 4:1  Итак, братия мои возлюбленные и вожделенные, радость и венец мой, стойте так в Господе, возлюбленные.
Php 4:2  Умоляю Еводию, умоляю Синтихию мыслить то же о Господе.
Php 4:3  Ей, прошу и тебя, искренний сотрудник, помогай им, подвизавшимся в благовествовании вместе со мною и с Климентом и с прочими сотрудниками моими, которых имена--в книге жизни.
Php 4:4  Радуйтесь всегда в Господе; и еще говорю: радуйтесь.
Php 4:5  Кротость ваша да будет известна всем человекам. Господь близко.
Php 4:6  Не заботьтесь ни о чем, но всегда в молитве и прошении с благодарением открывайте свои желания пред Богом,
Php 4:7  и мир Божий, который превыше всякого ума, соблюдет сердца ваши и помышления ваши во Христе Иисусе.
Php 4:8  Наконец, братия мои, что только истинно, что честно, что справедливо, что чисто, что любезно, что достославно, что только добродетель и похвала, о том помышляйте.
Php 4:9  Чему вы научились, что приняли и слышали и видели во мне, то исполняйте, --и Бог мира будет с вами.
Php 4:10  Я весьма возрадовался в Господе, что вы уже вновь начали заботиться о мне; вы и прежде заботились, но вам не благоприятствовали обстоятельства.
Php 4:11  Говорю это не потому, что нуждаюсь, ибо я научился быть довольным тем, что у меня есть.
Php 4:12  Умею жить и в скудости, умею жить и в изобилии; научился всему и во всем, насыщаться и терпеть голод, быть и в обилии и в недостатке.
Php 4:13  Все могу в укрепляющем меня Иисусе Христе.
Php 4:14  Впрочем вы хорошо поступили, приняв участие в моей скорби.
Php 4:15  Вы знаете, Филиппийцы, что в начале благовествования, когда я вышел из Македонии, ни одна церковь не оказала мне участия подаянием и принятием, кроме вас одних;
Php 4:16  вы и в Фессалонику и раз и два присылали мне на нужду.
Php 4:17  [Говорю это] не потому, чтобы я искал даяния; но ищу плода, умножающегося в пользу вашу.
Php 4:18  Я получил все, и избыточествую; я доволен, получив от Епафродита посланное вами, [как] благовонное курение, жертву приятную, благоугодную Богу.
Php 4:19  Бог мой да восполнит всякую нужду вашу, по богатству Своему в славе, Христом Иисусом.
Php 4:20  Богу же и Отцу нашему слава во веки веков! Аминь.
Php 4:21  Приветствуйте всякого святого во Христе Иисусе. Приветствуют вас находящиеся со мною братия.
Php 4:22  Приветствуют вас все святые, а наипаче из кесарева дома.
Php 4:23  Благодать Господа нашего Иисуса Христа со всеми вами. Аминь.


\end{document}