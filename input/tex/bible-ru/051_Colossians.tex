\begin{document}

\title{Colossians}

Col 1:1  Павел, волею Божиею Апостол Иисуса Христа, и Тимофей брат,
Col 1:2  находящимся в Колоссах святым и верным братиям во Христе Иисусе:
Col 1:3  благодать вам и мир от Бога Отца нашего и Господа Иисуса Христа. Благодарим Бога и Отца Господа нашего Иисуса Христа, всегда молясь о вас,
Col 1:4  услышав о вере вашей во Христа Иисуса и о любви ко всем святым,
Col 1:5  в надежде на уготованное вам на небесах, о чем вы прежде слышали в истинном слове благовествования,
Col 1:6  которое пребывает у вас, как и во всем мире, и приносит плод, и возрастает, как и между вами, с того дня, как вы услышали и познали благодать Божию в истине,
Col 1:7  как и научились от Епафраса, возлюбленного сотрудника нашего, верного для вас служителя Христова,
Col 1:8  который и известил нас о вашей любви в духе.
Col 1:9  Посему и мы с того дня, как [о сем] услышали, не перестаем молиться о вас и просить, чтобы вы исполнялись познанием воли Его, во всякой премудрости и разумении духовном,
Col 1:10  чтобы поступали достойно Бога, во всем угождая [Ему], принося плод во всяком деле благом и возрастая в познании Бога,
Col 1:11  укрепляясь всякою силою по могуществу славы Его, во всяком терпении и великодушии с радостью,
Col 1:12  благодаря Бога и Отца, призвавшего нас к участию в наследии святых во свете,
Col 1:13  избавившего нас от власти тьмы и введшего в Царство возлюбленного Сына Своего,
Col 1:14  в Котором мы имеем искупление Кровию Его и прощение грехов,
Col 1:15  Который есть образ Бога невидимого, рожденный прежде всякой твари;
Col 1:16  ибо Им создано все, что на небесах и что на земле, видимое и невидимое: престолы ли, господства ли, начальства ли, власти ли, --все Им и для Него создано;
Col 1:17  и Он есть прежде всего, и все Им стоит.
Col 1:18  И Он есть глава тела Церкви; Он--начаток, первенец из мертвых, дабы иметь Ему во всем первенство,
Col 1:19  ибо благоугодно было [Отцу], чтобы в Нем обитала всякая полнота,
Col 1:20  и чтобы посредством Его примирить с Собою все, умиротворив через Него, Кровию креста Его, и земное и небесное.
Col 1:21  И вас, бывших некогда отчужденными и врагами, по расположению к злым делам,
Col 1:22  ныне примирил в теле Плоти Его, смертью Его, [чтобы] представить вас святыми и непорочными и неповинными пред Собою,
Col 1:23  если только пребываете тверды и непоколебимы в вере и не отпадаете от надежды благовествования, которое вы слышали, которое возвещено всей твари поднебесной, которого я, Павел, сделался служителем.
Col 1:24  Ныне радуюсь в страданиях моих за вас и восполняю недостаток в плоти моей скорбей Христовых за Тело Его, которое есть Церковь,
Col 1:25  которой сделался я служителем по домостроительству Божию, вверенному мне для вас, [чтобы] исполнить слово Божие,
Col 1:26  тайну, сокрытую от веков и родов, ныне же открытую святым Его,
Col 1:27  Которым благоволил Бог показать, какое богатство славы в тайне сей для язычников, которая есть Христос в вас, упование славы,
Col 1:28  Которого мы проповедуем, вразумляя всякого человека и научая всякой премудрости, чтобы представить всякого человека совершенным во Христе Иисусе;
Col 1:29  для чего я и тружусь и подвизаюсь силою Его, действующею во мне могущественно.
Col 2:1  Желаю, чтобы вы знали, какой подвиг имею я ради вас и ради тех, которые в Лаодикии и Иераполе, и ради всех, кто не видел лица моего в плоти,
Col 2:2  дабы утешились сердца их, соединенные в любви для всякого богатства совершенного разумения, для познания тайны Бога и Отца и Христа,
Col 2:3  в Котором сокрыты все сокровища премудрости и ведения.
Col 2:4  Это говорю я для того, чтобы кто-нибудь не прельстил вас вкрадчивыми словами;
Col 2:5  ибо хотя я и отсутствую телом, но духом нахожусь с вами, радуясь и видя ваше благоустройство и твердость веры вашей во Христа.
Col 2:6  Посему, как вы приняли Христа Иисуса Господа, [так] и ходите в Нем,
Col 2:7  будучи укоренены и утверждены в Нем и укреплены в вере, как вы научены, преуспевая в ней с благодарением.
Col 2:8  Смотрите, братия, чтобы кто не увлек вас философиею и пустым обольщением, по преданию человеческому, по стихиям мира, а не по Христу;
Col 2:9  ибо в Нем обитает вся полнота Божества телесно,
Col 2:10  и вы имеете полноту в Нем, Который есть глава всякого начальства и власти.
Col 2:11  В Нем вы и обрезаны обрезанием нерукотворенным, совлечением греховного тела плоти, обрезанием Христовым;
Col 2:12  быв погребены с Ним в крещении, в Нем вы и совоскресли верою в силу Бога, Который воскресил Его из мертвых,
Col 2:13  и вас, которые были мертвы во грехах и в необрезании плоти вашей, оживил вместе с Ним, простив нам все грехи,
Col 2:14  истребив учением бывшее о нас рукописание, которое было против нас, и Он взял его от среды и пригвоздил ко кресту;
Col 2:15  отняв силы у начальств и властей, властно подверг их позору, восторжествовав над ними Собою.
Col 2:16  Итак никто да не осуждает вас за пищу, или питие, или за какой-нибудь праздник, или новомесячие, или субботу:
Col 2:17  это есть тень будущего, а тело--во Христе.
Col 2:18  Никто да не обольщает вас самовольным смиренномудрием и служением Ангелов, вторгаясь в то, чего не видел, безрассудно надмеваясь плотским своим умом
Col 2:19  и не держась главы, от которой все тело, составами и связями будучи соединяемо и скрепляемо, растет возрастом Божиим.
Col 2:20  Итак, если вы со Христом умерли для стихий мира, то для чего вы, как живущие в мире, держитесь постановлений:
Col 2:21  `не прикасайся', `не вкушай', `не дотрагивайся' --
Col 2:22  что все истлевает от употребления, --по заповедям и учению человеческому?
Col 2:23  Это имеет только вид мудрости в самовольном служении, смиренномудрии и изнурении тела, в некотором небрежении о насыщении плоти.
Col 3:1  Итак, если вы воскресли со Христом, то ищите горнего, где Христос сидит одесную Бога;
Col 3:2  о горнем помышляйте, а не о земном.
Col 3:3  Ибо вы умерли, и жизнь ваша сокрыта со Христом в Боге.
Col 3:4  Когда же явится Христос, жизнь ваша, тогда и вы явитесь с Ним во славе.
Col 3:5  Итак, умертвите земные члены ваши: блуд, нечистоту, страсть, злую похоть и любостяжание, которое есть идолослужение,
Col 3:6  за которые гнев Божий грядет на сынов противления,
Col 3:7  в которых и вы некогда обращались, когда жили между ними.
Col 3:8  А теперь вы отложите все: гнев, ярость, злобу, злоречие, сквернословие уст ваших;
Col 3:9  не говорите лжи друг другу, совлекшись ветхого человека с делами его
Col 3:10  и облекшись в нового, который обновляется в познании по образу Создавшего его,
Col 3:11  где нет ни Еллина, ни Иудея, ни обрезания, ни необрезания, варвара, Скифа, раба, свободного, но все и во всем Христос.
Col 3:12  Итак облекитесь, как избранные Божии, святые и возлюбленные, в милосердие, благость, смиренномудрие, кротость, долготерпение,
Col 3:13  снисходя друг другу и прощая взаимно, если кто на кого имеет жалобу: как Христос простил вас, так и вы.
Col 3:14  Более же всего [облекитесь] в любовь, которая есть совокупность совершенства.
Col 3:15  И да владычествует в сердцах ваших мир Божий, к которому вы и призваны в одном теле, и будьте дружелюбны.
Col 3:16  Слово Христово да вселяется в вас обильно, со всякою премудростью; научайте и вразумляйте друг друга псалмами, славословием и духовными песнями, во благодати воспевая в сердцах ваших Господу.
Col 3:17  И все, что вы делаете, словом или делом, все [делайте] во имя Господа Иисуса Христа, благодаря через Него Бога и Отца.
Col 3:18  Жены, повинуйтесь мужьям своим, как прилично в Господе.
Col 3:19  Мужья, любите своих жен и не будьте к ним суровы.
Col 3:20  Дети, будьте послушны родителям вашим во всем, ибо это благоугодно Господу.
Col 3:21  Отцы, не раздражайте детей ваших, дабы они не унывали.
Col 3:22  Рабы, во всем повинуйтесь господам вашим по плоти, не в глазах только служа [им], как человекоугодники, но в простоте сердца, боясь Бога.
Col 3:23  И все, что делаете, делайте от души, как для Господа, а не для человеков,
Col 3:24  зная, что в воздаяние от Господа получите наследие, ибо вы служите Господу Христу.
Col 3:25  А кто неправо поступит, тот получит по своей неправде, [у Него] нет лицеприятия.
Col 4:1  Господа, оказывайте рабам должное и справедливое, зная, что и вы имеете Господа на небесах.
Col 4:2  Будьте постоянны в молитве, бодрствуя в ней с благодарением.
Col 4:3  Молитесь также и о нас, чтобы Бог отверз нам дверь для слова, возвещать тайну Христову, за которую я и в узах,
Col 4:4  дабы я открыл ее, как должно мне возвещать.
Col 4:5  Со внешними обходитесь благоразумно, пользуясь временем.
Col 4:6  Слово ваше [да будет] всегда с благодатию, приправлено солью, дабы вы знали, как отвечать каждому.
Col 4:7  О мне все скажет вам Тихик, возлюбленный брат и верный служитель и сотрудник в Господе,
Col 4:8  которого я для того послал к вам, чтобы он узнал о ваших [обстоятельствах] и утешил сердца ваши,
Col 4:9  с Онисимом, верным и возлюбленным братом нашим, который от вас. Они расскажут вам о всем здешнем.
Col 4:10  Приветствует вас Аристарх, заключенный вместе со мною, и Марк, племянник Варнавы--о котором вы получили приказания: если придет к вам, примите его, --
Col 4:11  также Иисус, прозываемый Иустом, оба из обрезанных. Они--единственные сотрудники для Царствия Божия, бывшие мне отрадою.
Col 4:12  Приветствует вас Епафрас ваш, раб Иисуса Христа, всегда подвизающийся за вас в молитвах, чтобы вы пребыли совершенны и исполнены всем, что угодно Богу.
Col 4:13  Свидетельствую о нем, что он имеет великую ревность и заботу о вас и о находящихся в Лаодикии и Иераполе.
Col 4:14  Приветствует вас Лука, врач возлюбленный, и Димас.
Col 4:15  Приветствуйте братьев в Лаодикии, и Нимфана, и домашнюю церковь его.
Col 4:16  Когда это послание прочитано будет у вас, то распорядитесь, чтобы оно было прочитано и в Лаодикийской церкви; а то, которое из Лаодикии, прочитайте и вы.
Col 4:17  Скажите Архиппу: смотри, чтобы тебе исполнить служение, которое ты принял в Господе.
Col 4:18  Приветствие моею рукою, Павловою. Помните мои узы. Благодать со всеми вами. Аминь.


\end{document}