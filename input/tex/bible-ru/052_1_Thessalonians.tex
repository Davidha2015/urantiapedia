\begin{document}

\title{1-е к Фессалоникийцам}


\chapter{1}

\par 1 Павел и Силуан и Тимофей--церкви Фессалоникской в Боге Отце и Господе Иисусе Христе: благодать вам и мир от Бога Отца нашего и Господа Иисуса Христа.
\par 2 Всегда благодарим Бога за всех вас, вспоминая о вас в молитвах наших,
\par 3 непрестанно памятуя ваше дело веры и труд любви и терпение упования на Господа нашего Иисуса Христа пред Богом и Отцем нашим,
\par 4 зная избрание ваше, возлюбленные Богом братия;
\par 5 потому что наше благовествование у вас было не в слове только, но и в силе и во Святом Духе, и со многим удостоверением, как вы [сами] знаете, каковы были мы для вас между вами.
\par 6 И вы сделались подражателями нам и Господу, приняв слово при многих скорбях с радостью Духа Святаго,
\par 7 так что вы стали образцом для всех верующих в Македонии и Ахаии.
\par 8 Ибо от вас пронеслось слово Господне не только в Македонии и Ахаии, но и во всяком месте прошла [слава] о вере вашей в Бога, так что нам ни о чем не нужно рассказывать.
\par 9 Ибо сами они сказывают о нас, какой вход имели мы к вам, и как вы обратились к Богу от идолов, [чтобы] служить Богу живому и истинному
\par 10 и ожидать с небес Сына Его, Которого Он воскресил из мертвых, Иисуса, избавляющего нас от грядущего гнева.

\chapter{2}

\par 1 Вы сами знаете, братия, о нашем входе к вам, что он был не бездейственный;
\par 2 но, прежде пострадав и быв поруганы в Филиппах, как вы знаете, мы дерзнули в Боге нашем проповедать вам благовестие Божие с великим подвигом.
\par 3 Ибо в учении нашем нет ни заблуждения, ни нечистых [побуждений], ни лукавства;
\par 4 но, как Бог удостоил нас того, чтобы вверить [нам] благовестие, так мы и говорим, угождая не человекам, но Богу, испытующему сердца наши.
\par 5 Ибо никогда не было у нас перед вами ни слов ласкательства, как вы знаете, ни видов корысти: Бог свидетель!
\par 6 Не ищем славы человеческой ни от вас, ни от других:
\par 7 мы могли явиться с важностью, как Апостолы Христовы, но были тихи среди вас, подобно как кормилица нежно обходится с детьми своими.
\par 8 Так мы, из усердия к вам, восхотели передать вам не только благовестие Божие, но и души наши, потому что вы стали нам любезны.
\par 9 Ибо вы помните, братия, труд наш и изнурение: ночью и днем работая, чтобы не отяготить кого из вас, мы проповедывали у вас благовестие Божие.
\par 10 Свидетели вы и Бог, как свято и праведно и безукоризненно поступали мы перед вами, верующими,
\par 11 потому что вы знаете, как каждого из вас, как отец детей своих,
\par 12 мы просили и убеждали и умоляли поступать достойно Бога, призвавшего вас в Свое Царство и славу.
\par 13 Посему и мы непрестанно благодарим Бога, что, приняв от нас слышанное слово Божие, вы приняли не [как] слово человеческое, но [как] слово Божие, --каково оно есть по истине, --которое и действует в вас, верующих.
\par 14 Ибо вы, братия, сделались подражателями церквам Божиим во Христе Иисусе, находящимся в Иудее, потому что и вы то же претерпели от своих единоплеменников, что и те от Иудеев,
\par 15 которые убили и Господа Иисуса и Его пророков, и нас изгнали, и Богу не угождают, и всем человекам противятся,
\par 16 которые препятствуют нам говорить язычникам, чтобы спаслись, и через это всегда наполняют меру грехов своих; но приближается на них гнев до конца.
\par 17 Мы же, братия, быв разлучены с вами на короткое время лицем, а не сердцем, тем с большим желанием старались увидеть лице ваше.
\par 18 И потому мы, я Павел, и раз и два хотели прийти к вам, но воспрепятствовал нам сатана.
\par 19 Ибо кто наша надежда, или радость, или венец похвалы? Не и вы ли пред Господом нашим Иисусом Христом в пришествие Его?
\par 20 Ибо вы--слава наша и радость.

\chapter{3}

\par 1 И потому, не терпя более, мы восхотели остаться в Афинах одни,
\par 2 и послали Тимофея, брата нашего и служителя Божия и сотрудника нашего в благовествовании Христовом, чтобы утвердить вас и утешить в вере вашей,
\par 3 чтобы никто не поколебался в скорбях сих: ибо вы сами знаете, что так нам суждено.
\par 4 Ибо мы и тогда, как были у вас, предсказывали вам, что будем страдать, как и случилось, и вы знаете.
\par 5 Посему и я, не терпя более, послал узнать о вере вашей, чтобы как не искусил вас искуситель и не сделался тщетным труд наш.
\par 6 Теперь же, когда пришел к нам от вас Тимофей и принес нам добрую весть о вере и любви вашей, и что вы всегда имеете добрую память о нас, желая нас видеть, как и мы вас,
\par 7 то мы, при всей скорби и нужде нашей, утешились вами, братия, ради вашей веры;
\par 8 ибо теперь мы живы, когда вы стоите в Господе.
\par 9 Какую благодарность можем мы воздать Богу за вас, за всю радость, которою радуемся о вас пред Богом нашим,
\par 10 ночь и день всеусердно молясь о том, чтобы видеть лице ваше и дополнить, чего недоставало вере вашей?
\par 11 Сам же Бог и Отец наш и Господь наш Иисус Христос да управит путь наш к вам.
\par 12 А вас Господь да исполнит и преисполнит любовью друг к другу и ко всем, какою мы исполнены к вам,
\par 13 чтобы утвердить сердца ваши непорочными во святыне пред Богом и Отцем нашим в пришествие Господа нашего Иисуса Христа со всеми святыми Его. Аминь.

\chapter{4}

\par 1 За сим, братия, просим и умоляем вас Христом Иисусом, чтобы вы, приняв от нас, как должно вам поступать и угождать Богу, более в том преуспевали,
\par 2 ибо вы знаете, какие мы дали вам заповеди от Господа Иисуса.
\par 3 Ибо воля Божия есть освящение ваше, чтобы вы воздерживались от блуда;
\par 4 чтобы каждый из вас умел соблюдать свой сосуд в святости и чести,
\par 5 а не в страсти похотения, как и язычники, не знающие Бога;
\par 6 чтобы вы ни в чем не поступали с братом своим противозаконно и корыстолюбиво: потому что Господь--мститель за все это, как и прежде мы говорили вам и свидетельствовали.
\par 7 Ибо призвал нас Бог не к нечистоте, но к святости.
\par 8 Итак непокорный непокорен не человеку, но Богу, Который и дал нам Духа Своего Святаго.
\par 9 О братолюбии же нет нужды писать к вам; ибо вы сами научены Богом любить друг друга,
\par 10 ибо вы так и поступаете со всеми братиями по всей Македонии. Умоляем же вас, братия, более преуспевать
\par 11 и усердно стараться о том, чтобы жить тихо, делать свое [дело] и работать своими собственными руками, как мы заповедывали вам;
\par 12 чтобы вы поступали благоприлично перед внешними и ни в чем не нуждались.
\par 13 Не хочу же оставить вас, братия, в неведении об умерших, дабы вы не скорбели, как прочие, не имеющие надежды.
\par 14 Ибо, если мы веруем, что Иисус умер и воскрес, то и умерших в Иисусе Бог приведет с Ним.
\par 15 Ибо сие говорим вам словом Господним, что мы живущие, оставшиеся до пришествия Господня, не предупредим умерших,
\par 16 потому что Сам Господь при возвещении, при гласе Архангела и трубе Божией, сойдет с неба, и мертвые во Христе воскреснут прежде;
\par 17 потом мы, оставшиеся в живых, вместе с ними восхищены будем на облаках в сретение Господу на воздухе, и так всегда с Господом будем.
\par 18 Итак утешайте друг друга сими словами.

\chapter{5}

\par 1 О временах же и сроках нет нужды писать к вам, братия,
\par 2 ибо сами вы достоверно знаете, что день Господень так придет, как тать ночью.
\par 3 Ибо, когда будут говорить: `мир и безопасность', тогда внезапно постигнет их пагуба, подобно как мука родами [постигает] имеющую во чреве, и не избегнут.
\par 4 Но вы, братия, не во тьме, чтобы день застал вас, как тать.
\par 5 Ибо все вы--сыны света и сыны дня: мы--не [сыны] ночи, ни тьмы.
\par 6 Итак, не будем спать, как и прочие, но будем бодрствовать и трезвиться.
\par 7 Ибо спящие спят ночью, и упивающиеся упиваются ночью.
\par 8 Мы же, будучи [сынами] дня, да трезвимся, облекшись в броню веры и любви и в шлем надежды спасения,
\par 9 потому что Бог определил нас не на гнев, но к получению спасения через Господа нашего Иисуса Христа,
\par 10 умершего за нас, чтобы мы, бодрствуем ли, или спим, жили вместе с Ним.
\par 11 Посему увещавайте друг друга и назидайте один другого, как вы и делаете.
\par 12 Просим же вас, братия, уважать трудящихся у вас, и предстоятелей ваших в Господе, и вразумляющих вас,
\par 13 и почитать их преимущественно с любовью за дело их; будьте в мире между собою.
\par 14 Умоляем также вас, братия, вразумляйте бесчинных, утешайте малодушных, поддерживайте слабых, будьте долготерпеливы ко всем.
\par 15 Смотрите, чтобы кто кому не воздавал злом за зло; но всегда ищите добра и друг другу и всем.
\par 16 Всегда радуйтесь.
\par 17 Непрестанно молитесь.
\par 18 За все благодарите: ибо такова о вас воля Божия во Христе Иисусе.
\par 19 Духа не угашайте.
\par 20 Пророчества не уничижайте.
\par 21 Все испытывайте, хорошего держитесь.
\par 22 Удерживайтесь от всякого рода зла.
\par 23 Сам же Бог мира да освятит вас во всей полноте, и ваш дух и душа и тело во всей целости да сохранится без порока в пришествие Господа нашего Иисуса Христа.
\par 24 Верен Призывающий вас, Который и сотворит [сие].
\par 25 Братия! молитесь о нас.
\par 26 Приветствуйте всех братьев лобзанием святым.
\par 27 Заклинаю вас Господом прочитать сие послание всем святым братиям.
\par 28 Благодать Господа нашего Иисуса Христа с вами. Аминь.


\end{document}