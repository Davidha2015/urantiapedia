\begin{document}

\title{1 Timothy}

1Ti 1:1  Павел, Апостол Иисуса Христа по повелению Бога, Спасителя нашего, и Господа Иисуса Христа, надежды нашей,
1Ti 1:2  Тимофею, истинному сыну в вере: благодать, милость, мир от Бога, Отца нашего, и Христа Иисуса, Господа нашего.
1Ti 1:3  Отходя в Македонию, я просил тебя пребыть в Ефесе и увещевать некоторых, чтобы они не учили иному
1Ti 1:4  и не занимались баснями и родословиями бесконечными, которые производят больше споры, нежели Божие назидание в вере.
1Ti 1:5  Цель же увещания есть любовь от чистого сердца и доброй совести и нелицемерной веры,
1Ti 1:6  от чего отступив, некоторые уклонились в пустословие,
1Ti 1:7  желая быть законоучителями, но не разумея ни того, о чем говорят, ни того, что утверждают.
1Ti 1:8  А мы знаем, что закон добр, если кто законно употребляет его,
1Ti 1:9  зная, что закон положен не для праведника, но для беззаконных и непокоривых, нечестивых и грешников, развратных и оскверненных, для оскорбителей отца и матери, для человекоубийц,
1Ti 1:10  для блудников, мужеложников, человекохищников, (клеветников, скотоложников,) лжецов, клятвопреступников, и для всего, что противно здравому учению,
1Ti 1:11  по славному благовестию блаженного Бога, которое мне вверено.
1Ti 1:12  Благодарю давшего мне силу, Христа Иисуса, Господа нашего, что Он признал меня верным, определив на служение,
1Ti 1:13  меня, который прежде был хулитель и гонитель и обидчик, но помилован потому, что [так] поступал по неведению, в неверии;
1Ti 1:14  благодать же Господа нашего (Иисуса Христа) открылась [во мне] обильно с верою и любовью во Христе Иисусе.
1Ti 1:15  Верно и всякого принятия достойно слово, что Христос Иисус пришел в мир спасти грешников, из которых я первый.
1Ti 1:16  Но для того я и помилован, чтобы Иисус Христос во мне первом показал все долготерпение, в пример тем, которые будут веровать в Него к жизни вечной.
1Ti 1:17  Царю же веков нетленному, невидимому, единому премудрому Богу честь и слава во веки веков. Аминь.
1Ti 1:18  Преподаю тебе, сын [мой] Тимофей, сообразно с бывшими о тебе пророчествами, такое завещание, чтобы ты воинствовал согласно с ними, как добрый воин,
1Ti 1:19  имея веру и добрую совесть, которую некоторые отвергнув, потерпели кораблекрушение в вере;
1Ti 1:20  таковы Именей и Александр, которых я предал сатане, чтобы они научились не богохульствовать.
1Ti 2:1  Итак прежде всего прошу совершать молитвы, прошения, моления, благодарения за всех человеков,
1Ti 2:2  за царей и за всех начальствующих, дабы проводить нам жизнь тихую и безмятежную во всяком благочестии и чистоте,
1Ti 2:3  ибо это хорошо и угодно Спасителю нашему Богу,
1Ti 2:4  Который хочет, чтобы все люди спаслись и достигли познания истины.
1Ti 2:5  Ибо един Бог, един и посредник между Богом и человеками, человек Христос Иисус,
1Ti 2:6  предавший Себя для искупления всех. [Таково было] в свое время свидетельство,
1Ti 2:7  для которого я поставлен проповедником и Апостолом, --истину говорю во Христе, не лгу, --учителем язычников в вере и истине.
1Ti 2:8  Итак желаю, чтобы на всяком месте произносили молитвы мужи, воздевая чистые руки без гнева и сомнения;
1Ti 2:9  чтобы также и жены, в приличном одеянии, со стыдливостью и целомудрием, украшали себя не плетением [волос], не золотом, не жемчугом, не многоценною одеждою,
1Ti 2:10  но добрыми делами, как прилично женам, посвящающим себя благочестию.
1Ti 2:11  Жена да учится в безмолвии, со всякою покорностью;
1Ti 2:12  а учить жене не позволяю, ни властвовать над мужем, но быть в безмолвии.
1Ti 2:13  Ибо прежде создан Адам, а потом Ева;
1Ti 2:14  и не Адам прельщен; но жена, прельстившись, впала в преступление;
1Ti 2:15  впрочем спасется через чадородие, если пребудет в вере и любви и в святости с целомудрием.
1Ti 3:1  Верно слово: если кто епископства желает, доброго дела желает.
1Ti 3:2  Но епископ должен быть непорочен, одной жены муж, трезв, целомудрен, благочинен, честен, страннолюбив, учителен,
1Ti 3:3  не пьяница, не бийца, не сварлив, не корыстолюбив, но тих, миролюбив, не сребролюбив,
1Ti 3:4  хорошо управляющий домом своим, детей содержащий в послушании со всякою честностью;
1Ti 3:5  ибо, кто не умеет управлять собственным домом, тот будет ли пещись о Церкви Божией?
1Ti 3:6  Не [должен быть] из новообращенных, чтобы не возгордился и не подпал осуждению с диаволом.
1Ti 3:7  Надлежит ему также иметь доброе свидетельство от внешних, чтобы не впасть в нарекание и сеть диавольскую.
1Ti 3:8  Диаконы также [должны быть] честны, не двоязычны, не пристрастны к вину, не корыстолюбивы,
1Ti 3:9  хранящие таинство веры в чистой совести.
1Ti 3:10  И таких надобно прежде испытывать, потом, если беспорочны, [допускать] до служения.
1Ti 3:11  Равно и жены [их должны быть] честны, не клеветницы, трезвы, верны во всем.
1Ti 3:12  Диакон должен быть муж одной жены, хорошо управляющий детьми и домом своим.
1Ti 3:13  Ибо хорошо служившие приготовляют себе высшую степень и великое дерзновение в вере во Христа Иисуса.
1Ti 3:14  Сие пишу тебе, надеясь вскоре придти к тебе,
1Ti 3:15  чтобы, если замедлю, ты знал, как должно поступать в доме Божием, который есть Церковь Бога живаго, столп и утверждение истины.
1Ti 3:16  И беспрекословно--великая благочестия тайна: Бог явился во плоти, оправдал Себя в Духе, показал Себя Ангелам, проповедан в народах, принят верою в мире, вознесся во славе.
1Ti 4:1  Дух же ясно говорит, что в последние времена отступят некоторые от веры, внимая духам обольстителям и учениям бесовским,
1Ti 4:2  через лицемерие лжесловесников, сожженных в совести своей,
1Ti 4:3  запрещающих вступать в брак [и] употреблять в пищу то, что Бог сотворил, дабы верные и познавшие истину вкушали с благодарением.
1Ti 4:4  Ибо всякое творение Божие хорошо, и ничто не предосудительно, если принимается с благодарением,
1Ti 4:5  потому что освящается словом Божиим и молитвою.
1Ti 4:6  Внушая сие братиям, будешь добрый служитель Иисуса Христа, питаемый словами веры и добрым учением, которому ты последовал.
1Ti 4:7  Негодных же и бабьих басен отвращайся, а упражняй себя в благочестии,
1Ti 4:8  ибо телесное упражнение мало полезно, а благочестие на все полезно, имея обетование жизни настоящей и будущей.
1Ti 4:9  Слово сие верно и всякого принятия достойно.
1Ti 4:10  Ибо мы для того и трудимся и поношения терпим, что уповаем на Бога живаго, Который есть Спаситель всех человеков, а наипаче верных.
1Ti 4:11  Проповедуй сие и учи.
1Ti 4:12  Никто да не пренебрегает юностью твоею; но будь образцом для верных в слове, в житии, в любви, в духе, в вере, в чистоте.
1Ti 4:13  Доколе не приду, занимайся чтением, наставлением, учением.
1Ti 4:14  Не неради о пребывающем в тебе даровании, которое дано тебе по пророчеству с возложением рук священства.
1Ti 4:15  О сем заботься, в сем пребывай, дабы успех твой для всех был очевиден.
1Ti 4:16  Вникай в себя и в учение; занимайся сим постоянно: ибо, так поступая, и себя спасешь и слушающих тебя.
1Ti 5:1  Старца не укоряй, но увещевай, как отца; младших, как братьев;
1Ti 5:2  стариц, как матерей; молодых, как сестер, со всякою чистотою.
1Ti 5:3  Вдовиц почитай, истинных вдовиц.
1Ti 5:4  Если же какая вдовица имеет детей или внучат, то они прежде пусть учатся почитать свою семью и воздавать должное родителям, ибо сие угодно Богу.
1Ti 5:5  Истинная вдовица и одинокая надеется на Бога и пребывает в молениях и молитвах день и ночь;
1Ti 5:6  а сластолюбивая заживо умерла.
1Ti 5:7  И сие внушай им, чтобы были беспорочны.
1Ti 5:8  Если же кто о своих и особенно о домашних не печется, тот отрекся от веры и хуже неверного.
1Ti 5:9  Вдовица должна быть избираема не менее, как шестидесятилетняя, бывшая женою одного мужа,
1Ti 5:10  известная по добрым делам, если она воспитала детей, принимала странников, умывала ноги святым, помогала бедствующим и была усердна ко всякому доброму делу.
1Ti 5:11  Молодых же вдовиц не принимай, ибо они, впадая в роскошь в противность Христу, желают вступать в брак.
1Ti 5:12  Они подлежат осуждению, потому что отвергли прежнюю веру;
1Ti 5:13  притом же они, будучи праздны, приучаются ходить по домам и [бывают] не только праздны, но и болтливы, любопытны, и говорят, чего не должно.
1Ti 5:14  Итак я желаю, чтобы молодые вдовы вступали в брак, рождали детей, управляли домом и не подавали противнику никакого повода к злоречию;
1Ti 5:15  ибо некоторые уже совратились вслед сатаны.
1Ti 5:16  Если какой верный или верная имеет вдов, то должны их довольствовать и не обременять Церкви, чтобы она могла довольствовать истинных вдовиц.
1Ti 5:17  Достойно начальствующим пресвитерам должно оказывать сугубую честь, особенно тем, которые трудятся в слове и учении.
1Ti 5:18  Ибо Писание говорит: не заграждай рта у вола молотящего; и: трудящийся достоин награды своей.
1Ti 5:19  Обвинение на пресвитера не иначе принимай, как при двух или трех свидетелях.
1Ti 5:20  Согрешающих обличай перед всеми, чтобы и прочие страх имели.
1Ti 5:21  Пред Богом и Господом Иисусом Христом и избранными Ангелами заклинаю тебя сохранить сие без предубеждения, ничего не делая по пристрастию.
1Ti 5:22  Рук ни на кого не возлагай поспешно, и не делайся участником в чужих грехах. Храни себя чистым.
1Ti 5:23  Впредь пей не [одну] воду, но употребляй немного вина, ради желудка твоего и частых твоих недугов.
1Ti 5:24  Грехи некоторых людей явны и прямо ведут к осуждению, а некоторых [открываются] впоследствии.
1Ti 5:25  Равным образом и добрые дела явны; а если и не таковы, скрыться не могут.
1Ti 6:1  Рабы, под игом находящиеся, должны почитать господ своих достойными всякой чести, дабы не было хулы на имя Божие и учение.
1Ti 6:2  Те, которые имеют господами верных, не должны обращаться с ними небрежно, потому что они братья; но тем более должны служить им, что они верные и возлюбленные и благодетельствуют [им]. Учи сему и увещевай.
1Ti 6:3  Кто учит иному и не следует здравым словам Господа нашего Иисуса Христа и учению о благочестии,
1Ti 6:4  тот горд, ничего не знает, но заражен [страстью] к состязаниям и словопрениям, от которых происходят зависть, распри, злоречия, лукавые подозрения.
1Ti 6:5  Пустые споры между людьми поврежденного ума, чуждыми истины, которые думают, будто благочестие служит для прибытка. Удаляйся от таких.
1Ti 6:6  Великое приобретение--быть благочестивым и довольным.
1Ti 6:7  Ибо мы ничего не принесли в мир; явно, что ничего не можем и вынести [из него].
1Ti 6:8  Имея пропитание и одежду, будем довольны тем.
1Ti 6:9  А желающие обогащаться впадают в искушение и в сеть и во многие безрассудные и вредные похоти, которые погружают людей в бедствие и пагубу;
1Ti 6:10  ибо корень всех зол есть сребролюбие, которому предавшись, некоторые уклонились от веры и сами себя подвергли многим скорбям.
1Ti 6:11  Ты же, человек Божий, убегай сего, а преуспевай в правде, благочестии, вере, любви, терпении, кротости.
1Ti 6:12  Подвизайся добрым подвигом веры, держись вечной жизни, к которой ты и призван, и исповедал доброе исповедание перед многими свидетелями.
1Ti 6:13  Пред Богом, все животворящим, и пред Христом Иисусом, Который засвидетельствовал пред Понтием Пилатом доброе исповедание, завещеваю тебе
1Ti 6:14  соблюсти заповедь чисто и неукоризненно, даже до явления Господа нашего Иисуса Христа,
1Ti 6:15  которое в свое время откроет блаженный и единый сильный Царь царствующих и Господь господствующих,
1Ti 6:16  единый имеющий бессмертие, Который обитает в неприступном свете, Которого никто из человеков не видел и видеть не может. Ему честь и держава вечная! Аминь.
1Ti 6:17  Богатых в настоящем веке увещевай, чтобы они не высоко думали [о] [себе] и уповали не на богатство неверное, но на Бога живаго, дающего нам все обильно для наслаждения;
1Ti 6:18  чтобы они благодетельствовали, богатели добрыми делами, были щедры и общительны,
1Ti 6:19  собирая себе сокровище, доброе основание для будущего, чтобы достигнуть вечной жизни.
1Ti 6:20  О, Тимофей! храни преданное тебе, отвращаясь негодного пустословия и прекословий лжеименного знания,
1Ti 6:21  которому предавшись, некоторые уклонились от веры. Благодать с тобою. Аминь.


\end{document}