\begin{document}

\title{2-е к Тимофею}


\chapter{1}

\par 1 Павел, волею Божиею Апостол Иисуса Христа, по обетованию жизни во Христе Иисусе,
\par 2 Тимофею, возлюбленному сыну: благодать, милость, мир от Бога Отца и Христа Иисуса, Господа нашего.
\par 3 Благодарю Бога, Которому служу от прародителей с чистою совестью, что непрестанно вспоминаю о тебе в молитвах моих днем и ночью,
\par 4 и желаю видеть тебя, вспоминая о слезах твоих, дабы мне исполниться радости,
\par 5 приводя на память нелицемерную веру твою, которая прежде обитала в бабке твоей Лоиде и матери твоей Евнике; уверен, что она и в тебе.
\par 6 По сей причине напоминаю тебе возгревать дар Божий, который в тебе через мое рукоположение;
\par 7 ибо дал нам Бог духа не боязни, но силы и любви и целомудрия.
\par 8 Итак, не стыдись свидетельства Господа нашего Иисуса Христа, ни меня, узника Его; но страдай с благовестием Христовым силою Бога,
\par 9 спасшего нас и призвавшего званием святым, не по делам нашим, но по Своему изволению и благодати, данной нам во Христе Иисусе прежде вековых времен,
\par 10 открывшейся же ныне явлением Спасителя нашего Иисуса Христа, разрушившего смерть и явившего жизнь и нетление через благовестие,
\par 11 для которого я поставлен проповедником и Апостолом и учителем язычников.
\par 12 По сей причине я и страдаю так; но не стыжусь. Ибо я знаю, в Кого уверовал, и уверен, что Он силен сохранить залог мой на оный день.
\par 13 Держись образца здравого учения, которое ты слышал от меня, с верою и любовью во Христе Иисусе.
\par 14 Храни добрый залог Духом Святым, живущим в нас.
\par 15 Ты знаешь, что все Асийские оставили меня; в числе их Фигелл и Ермоген.
\par 16 Да даст Господь милость дому Онисифора за то, что он многократно покоил меня и не стыдился уз моих,
\par 17 но, быв в Риме, с великим тщанием искал меня и нашел.
\par 18 Да даст ему Господь обрести милость у Господа в оный день; а сколько он служил мне в Ефесе, ты лучше знаешь.

\chapter{2}

\par 1 Итак укрепляйся, сын мой, в благодати Христом Иисусом,
\par 2 и что слышал от меня при многих свидетелях, то передай верным людям, которые были бы способны и других научить.
\par 3 Итак переноси страдания, как добрый воин Иисуса Христа.
\par 4 Никакой воин не связывает себя делами житейскими, чтобы угодить военачальнику.
\par 5 Если же кто и подвизается, не увенчивается, если незаконно будет подвизаться.
\par 6 Трудящемуся земледельцу первому должно вкусить от плодов.
\par 7 Разумей, что я говорю. Да даст тебе Господь разумение во всем.
\par 8 Помни Господа Иисуса Христа от семени Давидова, воскресшего из мертвых, по благовествованию моему,
\par 9 за которое я страдаю даже до уз, как злодей; но для слова Божия нет уз.
\par 10 Посему я все терплю ради избранных, дабы и они получили спасение во Христе Иисусе с вечною славою.
\par 11 Верно слово: если мы с Ним умерли, то с Ним и оживем;
\par 12 если терпим, то с Ним и царствовать будем; если отречемся, и Он отречется от нас;
\par 13 если мы неверны, Он пребывает верен, ибо Себя отречься не может.
\par 14 Сие напоминай, заклиная пред Господом не вступать в словопрения, что нимало не служит к пользе, а к расстройству слушающих.
\par 15 Старайся представить себя Богу достойным, делателем неукоризненным, верно преподающим слово истины.
\par 16 А непотребного пустословия удаляйся; ибо они еще более будут преуспевать в нечестии,
\par 17 и слово их, как рак, будет распространяться. Таковы Именей и Филит,
\par 18 которые отступили от истины, говоря, что воскресение уже было, и разрушают в некоторых веру.
\par 19 Но твердое основание Божие стоит, имея печать сию: `познал Господь Своих'; и: `да отступит от неправды всякий, исповедующий имя Господа'.
\par 20 А в большом доме есть сосуды не только золотые и серебряные, но и деревянные и глиняные; и одни в почетном, а другие в низком употреблении.
\par 21 Итак, кто будет чист от сего, тот будет сосудом в чести, освященным и благопотребным Владыке, годным на всякое доброе дело.
\par 22 Юношеских похотей убегай, а держись правды, веры, любви, мира со всеми призывающими Господа от чистого сердца.
\par 23 От глупых и невежественных состязаний уклоняйся, зная, что они рождают ссоры;
\par 24 рабу же Господа не должно ссориться, но быть приветливым ко всем, учительным, незлобивым,
\par 25 с кротостью наставлять противников, не даст ли им Бог покаяния к познанию истины,
\par 26 чтобы они освободились от сети диавола, который уловил их в свою волю.

\chapter{3}

\par 1 Знай же, что в последние дни наступят времена тяжкие.
\par 2 Ибо люди будут самолюбивы, сребролюбивы, горды, надменны, злоречивы, родителям непокорны, неблагодарны, нечестивы, недружелюбны,
\par 3 непримирительны, клеветники, невоздержны, жестоки, не любящие добра,
\par 4 предатели, наглы, напыщенны, более сластолюбивы, нежели боголюбивы,
\par 5 имеющие вид благочестия, силы же его отрекшиеся. Таковых удаляйся.
\par 6 К сим принадлежат те, которые вкрадываются в домы и обольщают женщин, утопающих во грехах, водимых различными похотями,
\par 7 всегда учащихся и никогда не могущих дойти до познания истины.
\par 8 Как Ианний и Иамврий противились Моисею, так и сии противятся истине, люди, развращенные умом, невежды в вере.
\par 9 Но они не много успеют; ибо их безумие обнаружится перед всеми, как и с теми случилось.
\par 10 А ты последовал мне в учении, житии, расположении, вере, великодушии, любви, терпении,
\par 11 в гонениях, страданиях, постигших меня в Антиохии, Иконии, Листрах; каковые гонения я перенес, и от всех избавил меня Господь.
\par 12 Да и все, желающие жить благочестиво во Христе Иисусе, будут гонимы.
\par 13 Злые же люди и обманщики будут преуспевать во зле, вводя в заблуждение и заблуждаясь.
\par 14 А ты пребывай в том, чему научен и что тебе вверено, зная, кем ты научен.
\par 15 Притом же ты из детства знаешь священные писания, которые могут умудрить тебя во спасение верою во Христа Иисуса.
\par 16 Все Писание богодухновенно и полезно для научения, для обличения, для исправления, для наставления в праведности,
\par 17 да будет совершен Божий человек, ко всякому доброму делу приготовлен.

\chapter{4}

\par 1 Итак заклинаю тебя пред Богом и Господом нашим Иисусом Христом, Который будет судить живых и мертвых в явление Его и Царствие Его:
\par 2 проповедуй слово, настой во время и не во время, обличай, запрещай, увещевай со всяким долготерпением и назиданием.
\par 3 Ибо будет время, когда здравого учения принимать не будут, но по своим прихотям будут избирать себе учителей, которые льстили бы слуху;
\par 4 и от истины отвратят слух и обратятся к басням.
\par 5 Но ты будь бдителен во всем, переноси скорби, совершай дело благовестника, исполняй служение твое.
\par 6 Ибо я уже становлюсь жертвою, и время моего отшествия настало.
\par 7 Подвигом добрым я подвизался, течение совершил, веру сохранил;
\par 8 а теперь готовится мне венец правды, который даст мне Господь, праведный Судия, в день оный; и не только мне, но и всем, возлюбившим явление Его.
\par 9 Постарайся придти ко мне скоро.
\par 10 Ибо Димас оставил меня, возлюбив нынешний век, и пошел в Фессалонику, Крискент в Галатию, Тит в Далматию; один Лука со мною.
\par 11 Марка возьми и приведи с собою, ибо он мне нужен для служения.
\par 12 Тихика я послал в Ефес.
\par 13 Когда пойдешь, принеси фелонь, который я оставил в Троаде у Карпа, и книги, особенно кожаные.
\par 14 Александр медник много сделал мне зла. Да воздаст ему Господь по делам его!
\par 15 Берегись его и ты, ибо он сильно противился нашим словам.
\par 16 При первом моем ответе никого не было со мною, но все меня оставили. Да не вменится им!
\par 17 Господь же предстал мне и укрепил меня, дабы через меня утвердилось благовестие и услышали все язычники; и я избавился из львиных челюстей.
\par 18 И избавит меня Господь от всякого злого дела и сохранит для Своего Небесного Царства, Ему слава во веки веков. Аминь.
\par 19 Приветствуй Прискиллу и Акилу и дом Онисифоров.
\par 20 Ераст остался в Коринфе; Трофима же я оставил больного в Милите.
\par 21 Постарайся придти до зимы. Приветствуют тебя Еввул, и Пуд, и Лин, и Клавдия, и все братия.
\par 22 Господь Иисус Христос со духом твоим. Благодать с вами. Аминь.


\end{document}