\begin{document}

\title{Послание к Евреям}


\chapter{1}

\par 1 Бог, многократно и многообразно говоривший издревле отцам в пророках,
\par 2 в последние дни сии говорил нам в Сыне, Которого поставил наследником всего, чрез Которого и веки сотворил.
\par 3 Сей, будучи сияние славы и образ ипостаси Его и держа все словом силы Своей, совершив Собою очищение грехов наших, воссел одесную (престола) величия на высоте,
\par 4 будучи столько превосходнее Ангелов, сколько славнейшее пред ними наследовал имя.
\par 5 Ибо кому когда из Ангелов сказал [Бог]: Ты Сын Мой, Я ныне родил Тебя? И еще: Я буду Ему Отцем, и Он будет Мне Сыном?
\par 6 Также, когда вводит Первородного во вселенную, говорит: и да поклонятся Ему все Ангелы Божии.
\par 7 Об Ангелах сказано: Ты творишь Ангелами Своими духов и служителями Своими пламенеющий огонь.
\par 8 А о Сыне: престол Твой, Боже, в век века; жезл царствия Твоего--жезл правоты.
\par 9 Ты возлюбил правду и возненавидел беззаконие, посему помазал Тебя, Боже, Бог Твой елеем радости более соучастников Твоих.
\par 10 И: в начале Ты, Господи, основал землю, и небеса--дело рук Твоих;
\par 11 они погибнут, а Ты пребываешь; и все обветшают, как риза,
\par 12 и как одежду свернешь их, и изменятся; но Ты тот же, и лета Твои не кончатся.
\par 13 Кому когда из Ангелов сказал [Бог]: седи одесную Меня, доколе положу врагов Твоих в подножие ног Твоих?
\par 14 Не все ли они суть служебные духи, посылаемые на служение для тех, которые имеют наследовать спасение?

\chapter{2}

\par 1 Посему мы должны быть особенно внимательны к слышанному, чтобы не отпасть.
\par 2 Ибо, если через Ангелов возвещенное слово было твердо, и всякое преступление и непослушание получало праведное воздаяние,
\par 3 то как мы избежим, вознерадев о толиком спасении, которое, быв сначала проповедано Господом, в нас утвердилось слышавшими [от Него],
\par 4 при засвидетельствовании от Бога знамениями и чудесами, и различными силами, и раздаянием Духа Святаго по Его воле?
\par 5 Ибо не Ангелам Бог покорил будущую вселенную, о которой говорим;
\par 6 напротив некто негде засвидетельствовал, говоря: что значит человек, что Ты помнишь его? или сын человеческий, что Ты посещаешь его?
\par 7 Не много Ты унизил его пред Ангелами; славою и честью увенчал его, и поставил его над делами рук Твоих,
\par 8 все покорил под ноги его. Когда же покорил ему все, то не оставил ничего непокоренным ему. Ныне же еще не видим, чтобы все было ему покорено;
\par 9 но видим, что за претерпение смерти увенчан славою и честью Иисус, Который не много был унижен пред Ангелами, дабы Ему, по благодати Божией, вкусить смерть за всех.
\par 10 Ибо надлежало, чтобы Тот, для Которого все и от Которого все, приводящего многих сынов в славу, вождя спасения их совершил через страдания.
\par 11 Ибо и освящающий и освящаемые, все--от Единого; поэтому Он не стыдится называть их братиями, говоря:
\par 12 возвещу имя Твое братиям Моим, посреди церкви воспою Тебя.
\par 13 И еще: Я буду уповать на Него. И еще: вот Я и дети, которых дал Мне Бог.
\par 14 А как дети причастны плоти и крови, то и Он также воспринял оные, дабы смертью лишить силы имеющего державу смерти, то есть диавола,
\par 15 и избавить тех, которые от страха смерти через всю жизнь были подвержены рабству.
\par 16 Ибо не Ангелов восприемлет Он, но восприемлет семя Авраамово.
\par 17 Посему Он должен был во всем уподобиться братиям, чтобы быть милостивым и верным первосвященником пред Богом, для умилостивления за грехи народа.
\par 18 Ибо, как Сам Он претерпел, быв искушен, то может и искушаемым помочь.

\chapter{3}

\par 1 Итак, братия святые, участники в небесном звании, уразумейте Посланника и Первосвященника исповедания нашего, Иисуса Христа,
\par 2 Который верен Поставившему Его, как и Моисей во всем доме Его.
\par 3 Ибо Он достоин тем большей славы пред Моисеем, чем большую честь имеет в сравнении с домом тот, кто устроил его,
\par 4 ибо всякий дом устрояется кем-либо; а устроивший все [есть] Бог.
\par 5 И Моисей верен во всем доме Его, как служитель, для засвидетельствования того, что надлежало возвестить;
\par 6 а Христос--как Сын в доме Его; дом же Его--мы, если только дерзновение и упование, которым хвалимся, твердо сохраним до конца.
\par 7 Почему, как говорит Дух Святый, ныне, когда услышите глас Его,
\par 8 не ожесточите сердец ваших, как во время ропота, в день искушения в пустыне,
\par 9 где искушали Меня отцы ваши, испытывали Меня, и видели дела Мои сорок лет.
\par 10 Посему Я вознегодовал на оный род и сказал: непрестанно заблуждаются сердцем, не познали они путей Моих;
\par 11 посему Я поклялся во гневе Моем, что они не войдут в покой Мой.
\par 12 Смотрите, братия, чтобы не было в ком из вас сердца лукавого и неверного, дабы вам не отступить от Бога живаго.
\par 13 Но наставляйте друг друга каждый день, доколе можно говорить: `ныне', чтобы кто из вас не ожесточился, обольстившись грехом.
\par 14 Ибо мы сделались причастниками Христу, если только начатую жизнь твердо сохраним до конца,
\par 15 доколе говорится: `ныне, когда услышите глас Его, не ожесточите сердец ваших, как во время ропота'.
\par 16 Ибо некоторые из слышавших возроптали; но не все вышедшие из Египта с Моисеем.
\par 17 На кого же негодовал Он сорок лет? Не на согрешивших ли, которых кости пали в пустыне?
\par 18 Против кого же клялся, что не войдут в покой Его, как не против непокорных?
\par 19 Итак видим, что они не могли войти за неверие.

\chapter{4}

\par 1 Посему будем опасаться, чтобы, когда еще остается обетование войти в покой Его, не оказался кто из вас опоздавшим.
\par 2 Ибо и нам оно возвещено, как и тем; но не принесло им пользы слово слышанное, не растворенное верою слышавших.
\par 3 А входим в покой мы уверовавшие, так как Он сказал: `Я поклялся в гневе Моем, что они не войдут в покой Мой', хотя дела [Его] были совершены еще в начале мира.
\par 4 Ибо негде сказано о седьмом [дне] так: и почил Бог в день седьмый от всех дел Своих.
\par 5 И еще здесь: `не войдут в покой Мой'.
\par 6 Итак, как некоторым остается войти в него, а те, которым прежде возвещено, не вошли в него за непокорность,
\par 7 [то] еще определяет некоторый день, `ныне', говоря через Давида, после столь долгого времени, как выше сказано: `ныне, когда услышите глас Его, не ожесточите сердец ваших'.
\par 8 Ибо если бы Иисус [Навин] доставил им покой, то не было бы сказано после того о другом дне.
\par 9 Посему для народа Божия еще остается субботство.
\par 10 Ибо, кто вошел в покой Его, тот и сам успокоился от дел своих, как и Бог от Своих.
\par 11 Итак постараемся войти в покой оный, чтобы кто по тому же примеру не впал в непокорность.
\par 12 Ибо слово Божие живо и действенно и острее всякого меча обоюдоострого: оно проникает до разделения души и духа, составов и мозгов, и судит помышления и намерения сердечные.
\par 13 И нет твари, сокровенной от Него, но все обнажено и открыто перед очами Его: Ему дадим отчет.
\par 14 Итак, имея Первосвященника великого, прошедшего небеса, Иисуса Сына Божия, будем твердо держаться исповедания [нашего].
\par 15 Ибо мы имеем не такого первосвященника, который не может сострадать нам в немощах наших, но Который, подобно [нам], искушен во всем, кроме греха.
\par 16 Посему да приступаем с дерзновением к престолу благодати, чтобы получить милость и обрести благодать для благовременной помощи.

\chapter{5}

\par 1 Ибо всякий первосвященник, из человеков избираемый, для человеков поставляется на служение Богу, чтобы приносить дары и жертвы за грехи,
\par 2 могущий снисходить невежествующим и заблуждающим, потому что и сам обложен немощью,
\par 3 и посему он должен как за народ, так и за себя приносить [жертвы] о грехах.
\par 4 И никто сам собою не приемлет этой чести, но призываемый Богом, как и Аарон.
\par 5 Так и Христос не Сам Себе присвоил славу быть первосвященником, но Тот, Кто сказал Ему: Ты Сын Мой, Я ныне родил Тебя;
\par 6 как и в другом [месте] говорит: Ты священник вовек по чину Мелхиседека.
\par 7 Он, во дни плоти Своей, с сильным воплем и со слезами принес молитвы и моления Могущему спасти Его от смерти; и услышан был за [Свое] благоговение;
\par 8 хотя Он и Сын, однако страданиями навык послушанию,
\par 9 и, совершившись, сделался для всех послушных Ему виновником спасения вечного,
\par 10 быв наречен от Бога Первосвященником по чину Мелхиседека.
\par 11 О сем надлежало бы нам говорить много; но трудно истолковать, потому что вы сделались неспособны слушать.
\par 12 Ибо, [судя] по времени, вам надлежало быть учителями; но вас снова нужно учить первым началам слова Божия, и для вас нужно молоко, а не твердая пища.
\par 13 Всякий, питаемый молоком, несведущ в слове правды, потому что он младенец;
\par 14 твердая же пища свойственна совершенным, у которых чувства навыком приучены к различению добра и зла.

\chapter{6}

\par 1 Посему, оставив начатки учения Христова, поспешим к совершенству; и не станем снова полагать основание обращению от мертвых дел и вере в Бога,
\par 2 учению о крещениях, о возложении рук, о воскресении мертвых и о суде вечном.
\par 3 И это сделаем, если Бог позволит.
\par 4 Ибо невозможно--однажды просвещенных, и вкусивших дара небесного, и соделавшихся причастниками Духа Святаго,
\par 5 и вкусивших благого глагола Божия и сил будущего века,
\par 6 и отпадших, опять обновлять покаянием, когда они снова распинают в себе Сына Божия и ругаются [Ему].
\par 7 Земля, пившая многократно сходящий на нее дождь и произращающая злак, полезный тем, для которых и возделывается, получает благословение от Бога;
\par 8 а производящая терния и волчцы негодна и близка к проклятию, которого конец--сожжение.
\par 9 Впрочем о вас, возлюбленные, мы надеемся, что вы в лучшем [состоянии] и держитесь спасения, хотя и говорим так.
\par 10 Ибо не неправеден Бог, чтобы забыл дело ваше и труд любви, которую вы оказали во имя Его, послужив и служа святым.
\par 11 Желаем же, чтобы каждый из вас, для совершенной уверенности в надежде, оказывал такую же ревность до конца,
\par 12 дабы вы не обленились, но подражали тем, которые верою и долготерпением наследуют обетования.
\par 13 Бог, давая обетование Аврааму, как не мог никем высшим клясться, клялся Самим Собою,
\par 14 говоря: истинно благословляя благословлю тебя и размножая размножу тебя.
\par 15 И так Авраам, долготерпев, получил обещанное.
\par 16 Люди клянутся высшим, и клятва во удостоверение оканчивает всякий спор их.
\par 17 Посему и Бог, желая преимущественнее показать наследникам обетования непреложность Своей воли, употребил в посредство клятву,
\par 18 дабы в двух непреложных вещах, в которых невозможно Богу солгать, твердое утешение имели мы, прибегшие взяться за предлежащую надежду,
\par 19 которая для души есть как бы якорь безопасный и крепкий, и входит во внутреннейшее за завесу,
\par 20 куда предтечею за нас вошел Иисус, сделавшись Первосвященником навек по чину Мелхиседека.

\chapter{7}

\par 1 Ибо Мелхиседек, царь Салима, священник Бога Всевышнего, тот, который встретил Авраама и благословил его, возвращающегося после поражения царей,
\par 2 которому и десятину отделил Авраам от всего, --во-первых, по знаменованию [имени] царь правды, а потом и царь Салима, то есть царь мира,
\par 3 без отца, без матери, без родословия, не имеющий ни начала дней, ни конца жизни, уподобляясь Сыну Божию, пребывает священником навсегда.
\par 4 Видите, как велик тот, которому и Авраам патриарх дал десятину из лучших добыч своих.
\par 5 Получающие священство из сынов Левииных имеют заповедь--брать по закону десятину с народа, то есть со своих братьев, хотя и сии произошли от чресл Авраамовых.
\par 6 Но сей, не происходящий от рода их, получил десятину от Авраама и благословил имевшего обетования.
\par 7 Без всякого же прекословия меньший благословляется большим.
\par 8 И здесь десятины берут человеки смертные, а там--имеющий о себе свидетельство, что он живет.
\par 9 И, так сказать, сам Левий, принимающий десятины, в [лице] Авраама дал десятину:
\par 10 ибо он был еще в чреслах отца, когда Мелхиседек встретил его.
\par 11 Итак, если бы совершенство достигалось посредством левитского священства, --ибо с ним сопряжен закон народа, --то какая бы еще нужда была восставать иному священнику по чину Мелхиседека, а не по чину Аарона именоваться?
\par 12 Потому что с переменою священства необходимо быть перемене и закона.
\par 13 Ибо Тот, о Котором говорится сие, принадлежал к иному колену, из которого никто не приступал к жертвеннику.
\par 14 Ибо известно, что Господь наш воссиял из колена Иудина, о котором Моисей ничего не сказал относительно священства.
\par 15 И это еще яснее видно [из того], что по подобию Мелхиседека восстает Священник иной,
\par 16 Который таков не по закону заповеди плотской, но по силе жизни непрестающей.
\par 17 Ибо засвидетельствовано: Ты священник вовек по чину Мелхиседека.
\par 18 Отменение же прежде бывшей заповеди бывает по причине ее немощи и бесполезности,
\par 19 ибо закон ничего не довел до совершенства; но вводится лучшая надежда, посредством которой мы приближаемся к Богу.
\par 20 И как [сие было] не без клятвы, --
\par 21 ибо те были священниками без клятвы, а Сей с клятвою, потому что о Нем сказано: клялся Господь, и не раскается: Ты священник вовек по чину Мелхиседека, --
\par 22 то лучшего завета поручителем соделался Иисус.
\par 23 Притом тех священников было много, потому что смерть не допускала пребывать одному;
\par 24 а Сей, как пребывающий вечно, имеет и священство непреходящее,
\par 25 посему и может всегда спасать приходящих чрез Него к Богу, будучи всегда жив, чтобы ходатайствовать за них.
\par 26 Таков и должен быть у нас Первосвященник: святой, непричастный злу, непорочный, отделенный от грешников и превознесенный выше небес,
\par 27 Который не имеет нужды ежедневно, как те первосвященники, приносить жертвы сперва за свои грехи, потом за грехи народа, ибо Он совершил это однажды, принеся [в жертву] Себя Самого.
\par 28 Ибо закон поставляет первосвященниками человеков, имеющих немощи; а слово клятвенное, после закона, [поставило] Сына, на веки совершенного.

\chapter{8}

\par 1 Главное же в том, о чем говорим, есть то: мы имеем такого Первосвященника, Который воссел одесную престола величия на небесах
\par 2 и [есть] священнодействователь святилища и скинии истинной, которую воздвиг Господь, а не человек.
\par 3 Всякий первосвященник поставляется для приношения даров и жертв; а потому нужно было, чтобы и Сей также имел, что принести.
\par 4 Если бы Он оставался на земле, то не был бы и священником, потому что [здесь] такие священники, которые по закону приносят дары,
\par 5 которые служат образу и тени небесного, как сказано было Моисею, когда он приступал к совершению скинии: смотри, сказано, сделай все по образу, показанному тебе на горе.
\par 6 Но Сей [Первосвященник] получил служение тем превосходнейшее, чем лучшего Он ходатай завета, который утвержден на лучших обетованиях.
\par 7 Ибо, если бы первый [завет] был без недостатка, то не было бы нужды искать места другому.
\par 8 Но [пророк], укоряя их, говорит: вот, наступают дни, говорит Господь, когда Я заключу с домом Израиля и с домом Иуды новый завет,
\par 9 не такой завет, какой Я заключил с отцами их в то время, когда взял их за руку, чтобы вывести их из земли Египетской, потому что они не пребыли в том завете Моем, и Я пренебрег их, говорит Господь.
\par 10 Вот завет, который завещаю дому Израилеву после тех дней, говорит Господь: вложу законы Мои в мысли их, и напишу их на сердцах их; и буду их Богом, а они будут Моим народом.
\par 11 И не будет учить каждый ближнего своего и каждый брата своего, говоря: познай Господа; потому что все, от малого до большого, будут знать Меня,
\par 12 потому что Я буду милостив к неправдам их, и грехов их и беззаконий их не воспомяну более.
\par 13 Говоря `новый', показал ветхость первого; а ветшающее и стареющее близко к уничтожению.

\chapter{9}

\par 1 И первый завет имел постановление о Богослужении и святилище земное:
\par 2 ибо устроена была скиния первая, в которой был светильник, и трапеза, и предложение хлебов, и которая называется `святое'.
\par 3 За второю же завесою была скиния, называемая `Святое-святых',
\par 4 имевшая золотую кадильницу и обложенный со всех сторон золотом ковчег завета, где были золотой сосуд с манною, жезл Ааронов расцветший и скрижали завета,
\par 5 а над ним херувимы славы, осеняющие очистилище; о чем не нужно теперь говорить подробно.
\par 6 При таком устройстве, в первую скинию всегда входят священники совершать Богослужение;
\par 7 а во вторую--однажды в год один только первосвященник, не без крови, которую приносит за себя и за грехи неведения народа.
\par 8 [Сим] Дух Святый показывает, что еще не открыт путь во святилище, доколе стоит прежняя скиния.
\par 9 Она есть образ настоящего времени, в которое приносятся дары и жертвы, не могущие сделать в совести совершенным приносящего,
\par 10 и которые с яствами и питиями, и различными омовениями и обрядами, [относящимися] до плоти, установлены были только до времени исправления.
\par 11 Но Христос, Первосвященник будущих благ, придя с большею и совершеннейшею скиниею, нерукотворенною, то есть не такового устроения,
\par 12 и не с кровью козлов и тельцов, но со Своею Кровию, однажды вошел во святилище и приобрел вечное искупление.
\par 13 Ибо если кровь тельцов и козлов и пепел телицы, через окропление, освящает оскверненных, дабы чисто было тело,
\par 14 то кольми паче Кровь Христа, Который Духом Святым принес Себя непорочного Богу, очистит совесть нашу от мертвых дел, для служения Богу живому и истинному!
\par 15 И потому Он есть ходатай нового завета, дабы вследствие смерти [Его], бывшей для искупления от преступлений, сделанных в первом завете, призванные к вечному наследию получили обетованное.
\par 16 Ибо, где завещание, там необходимо, чтобы последовала смерть завещателя,
\par 17 потому что завещание действительно после умерших: оно не имеет силы, когда завещатель жив.
\par 18 Почему и первый [завет] был утвержден не без крови.
\par 19 Ибо Моисей, произнеся все заповеди по закону перед всем народом, взял кровь тельцов и козлов с водою и шерстью червленою и иссопом, и окропил как самую книгу, так и весь народ,
\par 20 говоря: это кровь завета, который заповедал вам Бог.
\par 21 Также окропил кровью и скинию и все сосуды Богослужебные.
\par 22 Да и все почти по закону очищается кровью, и без пролития крови не бывает прощения.
\par 23 Итак образы небесного должны были очищаться сими, самое же небесное лучшими сих жертвами.
\par 24 Ибо Христос вошел не в рукотворенное святилище, по образу истинного [устроенное], но в самое небо, чтобы предстать ныне за нас пред лице Божие,
\par 25 и не для того, чтобы многократно приносить Себя, как первосвященник входит во святилище каждогодно с чужою кровью;
\par 26 иначе надлежало бы Ему многократно страдать от начала мира; Он же однажды, к концу веков, явился для уничтожения греха жертвою Своею.
\par 27 И как человекам положено однажды умереть, а потом суд,
\par 28 так и Христос, однажды принеся Себя в жертву, чтобы подъять грехи многих, во второй раз явится не [для очищения] греха, а для ожидающих Его во спасение.

\chapter{10}

\par 1 Закон, имея тень будущих благ, а не самый образ вещей, одними и теми же жертвами, каждый год постоянно приносимыми, никогда не может сделать совершенными приходящих [с ними].
\par 2 Иначе перестали бы приносить [их], потому что приносящие жертву, быв очищены однажды, не имели бы уже никакого сознания грехов.
\par 3 Но жертвами каждогодно напоминается о грехах,
\par 4 ибо невозможно, чтобы кровь тельцов и козлов уничтожала грехи.
\par 5 Посему [Христос], входя в мир, говорит: жертвы и приношения Ты не восхотел, но тело уготовал Мне.
\par 6 Всесожжения и [жертвы] за грех неугодны Тебе.
\par 7 Тогда Я сказал: вот, иду, [как] в начале книги написано о Мне, исполнить волю Твою, Боже.
\par 8 Сказав прежде, что `ни жертвы, ни приношения, ни всесожжений, ни [жертвы] за грех, --которые приносятся по закону, --Ты не восхотел и не благоизволил',
\par 9 потом прибавил: `вот, иду исполнить волю Твою, Боже'. Отменяет первое, чтобы постановить второе.
\par 10 По сей-то воле освящены мы единократным принесением тела Иисуса Христа.
\par 11 И всякий священник ежедневно стоит в служении, и многократно приносит одни и те же жертвы, которые никогда не могут истребить грехов.
\par 12 Он же, принеся одну жертву за грехи, навсегда воссел одесную Бога,
\par 13 ожидая затем, доколе враги Его будут положены в подножие ног Его.
\par 14 Ибо Он одним приношением навсегда сделал совершенными освящаемых.
\par 15 [О сем] свидетельствует нам и Дух Святый; ибо сказано:
\par 16 Вот завет, который завещаю им после тех дней, говорит Господь: вложу законы Мои в сердца их, и в мыслях их напишу их,
\par 17 и грехов их и беззаконий их не воспомяну более.
\par 18 А где прощение грехов, там не нужно приношение за них.
\par 19 Итак, братия, имея дерзновение входить во святилище посредством Крови Иисуса Христа, путем новым и живым,
\par 20 который Он вновь открыл нам через завесу, то есть плоть Свою,
\par 21 и [имея] великого Священника над домом Божиим,
\par 22 да приступаем с искренним сердцем, с полною верою, кроплением очистив сердца от порочной совести, и омыв тело водою чистою,
\par 23 будем держаться исповедания упования неуклонно, ибо верен Обещавший.
\par 24 Будем внимательны друг ко другу, поощряя к любви и добрым делам.
\par 25 Не будем оставлять собрания своего, как есть у некоторых обычай; но будем увещевать [друг друга], и тем более, чем более усматриваете приближение дня оного.
\par 26 Ибо если мы, получив познание истины, произвольно грешим, то не остается более жертвы за грехи,
\par 27 но некое страшное ожидание суда и ярость огня, готового пожрать противников.
\par 28 [Если] отвергшийся закона Моисеева, при двух или трех свидетелях, без милосердия [наказывается] смертью,
\par 29 то сколь тягчайшему, думаете, наказанию повинен будет тот, кто попирает Сына Божия и не почитает за святыню Кровь завета, которою освящен, и Духа благодати оскорбляет?
\par 30 Мы знаем Того, Кто сказал: у Меня отмщение, Я воздам, говорит Господь. И еще: Господь будет судить народ Свой.
\par 31 Страшно впасть в руки Бога живаго!
\par 32 Вспомните прежние дни ваши, когда вы, быв просвещены, выдержали великий подвиг страданий,
\par 33 то сами среди поношений и скорбей служа зрелищем [для других], то принимая участие в других, находившихся в таком же [состоянии];
\par 34 ибо вы и моим узам сострадали и расхищение имения вашего приняли с радостью, зная, что есть у вас на небесах имущество лучшее и непреходящее.
\par 35 Итак не оставляйте упования вашего, которому предстоит великое воздаяние.
\par 36 Терпение нужно вам, чтобы, исполнив волю Божию, получить обещанное;
\par 37 ибо еще немного, очень немного, и Грядущий придет и не умедлит.
\par 38 Праведный верою жив будет; а если [кто] поколеблется, не благоволит к тому душа Моя.
\par 39 Мы же не из колеблющихся на погибель, но [стоим] в вере к спасению души.

\chapter{11}

\par 1 Вера же есть осуществление ожидаемого и уверенность в невидимом.
\par 2 В ней свидетельствованы древние.
\par 3 Верою познаем, что веки устроены словом Божиим, так что из невидимого произошло видимое.
\par 4 Верою Авель принес Богу жертву лучшую, нежели Каин; ею получил свидетельство, что он праведен, как засвидетельствовал Бог о дарах его; ею он и по смерти говорит еще.
\par 5 Верою Енох переселен был так, что не видел смерти; и не стало его, потому что Бог переселил его. Ибо прежде переселения своего получил он свидетельство, что угодил Богу.
\par 6 А без веры угодить Богу невозможно; ибо надобно, чтобы приходящий к Богу веровал, что Он есть, и ищущим Его воздает.
\par 7 Верою Ной, получив откровение о том, что еще не было видимо, благоговея приготовил ковчег для спасения дома своего; ею осудил он (весь) мир, и сделался наследником праведности по вере.
\par 8 Верою Авраам повиновался призванию идти в страну, которую имел получить в наследие, и пошел, не зная, куда идет.
\par 9 Верою обитал он на земле обетованной, как на чужой, и жил в шатрах с Исааком и Иаковом, сонаследниками того же обетования;
\par 10 ибо он ожидал города, имеющего основание, которого художник и строитель Бог.
\par 11 Верою и сама Сарра (будучи неплодна) получила силу к принятию семени, и не по времени возраста родила, ибо знала, что верен Обещавший.
\par 12 И потому от одного, и притом омертвелого, родилось так много, как [много] звезд на небе и как бесчислен песок на берегу морском.
\par 13 Все сии умерли в вере, не получив обетований, а только издали видели оные, и радовались, и говорили о себе, что они странники и пришельцы на земле;
\par 14 ибо те, которые так говорят, показывают, что они ищут отечества.
\par 15 И если бы они в мыслях имели то [отечество], из которого вышли, то имели бы время возвратиться;
\par 16 но они стремились к лучшему, то есть к небесному; посему и Бог не стыдится их, называя Себя их Богом: ибо Он приготовил им город.
\par 17 Верою Авраам, будучи искушаем, принес в жертву Исаака и, имея обетование, принес единородного,
\par 18 о котором было сказано: в Исааке наречется тебе семя.
\par 19 Ибо он думал, что Бог силен и из мертвых воскресить, почему и получил его в предзнаменование.
\par 20 Верою в будущее Исаак благословил Иакова и Исава.
\par 21 Верою Иаков, умирая, благословил каждого сына Иосифова и поклонился на верх жезла своего.
\par 22 Верою Иосиф, при кончине, напоминал об исходе сынов Израилевых и завещал о костях своих.
\par 23 Верою Моисей по рождении три месяца скрываем был родителями своими, ибо видели они, что дитя прекрасно, и не устрашились царского повеления.
\par 24 Верою Моисей, придя в возраст, отказался называться сыном дочери фараоновой,
\par 25 и лучше захотел страдать с народом Божиим, нежели иметь временное греховное наслаждение,
\par 26 и поношение Христово почел большим для себя богатством, нежели Египетские сокровища; ибо он взирал на воздаяние.
\par 27 Верою оставил он Египет, не убоявшись гнева царского, ибо он, как бы видя Невидимого, был тверд.
\par 28 Верою совершил он Пасху и пролитие крови, дабы истребитель первенцев не коснулся их.
\par 29 Верою перешли они Чермное море, как по суше, --на что покусившись, Египтяне потонули.
\par 30 Верою пали стены Иерихонские, по семидневном обхождении.
\par 31 Верою Раав блудница, с миром приняв соглядатаев (и проводив их другим путем), не погибла с неверными.
\par 32 И что еще скажу? Недостанет мне времени, чтобы повествовать о Гедеоне, о Вараке, о Самсоне и Иеффае, о Давиде, Самуиле и (других) пророках,
\par 33 которые верою побеждали царства, творили правду, получали обетования, заграждали уста львов,
\par 34 угашали силу огня, избегали острия меча, укреплялись от немощи, были крепки на войне, прогоняли полки чужих;
\par 35 жены получали умерших своих воскресшими; иные же замучены были, не приняв освобождения, дабы получить лучшее воскресение;
\par 36 другие испытали поругания и побои, а также узы и темницу,
\par 37 были побиваемы камнями, перепиливаемы, подвергаемы пытке, умирали от меча, скитались в милотях и козьих кожах, терпя недостатки, скорби, озлобления;
\par 38 те, которых весь мир не был достоин, скитались по пустыням и горам, по пещерам и ущельям земли.
\par 39 И все сии, свидетельствованные в вере, не получили обещанного,
\par 40 потому что Бог предусмотрел о нас нечто лучшее, дабы они не без нас достигли совершенства.

\chapter{12}

\par 1 Посему и мы, имея вокруг себя такое облако свидетелей, свергнем с себя всякое бремя и запинающий нас грех и с терпением будем проходить предлежащее нам поприще,
\par 2 взирая на начальника и совершителя веры Иисуса, Который, вместо предлежавшей Ему радости, претерпел крест, пренебрегши посрамление, и воссел одесную престола Божия.
\par 3 Помыслите о Претерпевшем такое над Собою поругание от грешников, чтобы вам не изнемочь и не ослабеть душами вашими.
\par 4 Вы еще не до крови сражались, подвизаясь против греха,
\par 5 и забыли утешение, которое предлагается вам, как сынам: сын мой! не пренебрегай наказания Господня, и не унывай, когда Он обличает тебя.
\par 6 Ибо Господь, кого любит, того наказывает; бьет же всякого сына, которого принимает.
\par 7 Если вы терпите наказание, то Бог поступает с вами, как с сынами. Ибо есть ли какой сын, которого бы не наказывал отец?
\par 8 Если же остаетесь без наказания, которое всем обще, то вы незаконные дети, а не сыны.
\par 9 Притом, [если] мы, будучи наказываемы плотскими родителями нашими, боялись их, то не гораздо ли более должны покориться Отцу духов, чтобы жить?
\par 10 Те наказывали нас по своему произволу для немногих дней; а Сей--для пользы, чтобы нам иметь участие в святости Его.
\par 11 Всякое наказание в настоящее время кажется не радостью, а печалью; но после наученным через него доставляет мирный плод праведности.
\par 12 Итак укрепите опустившиеся руки и ослабевшие колени
\par 13 и ходите прямо ногами вашими, дабы хромлющее не совратилось, а лучше исправилось.
\par 14 Старайтесь иметь мир со всеми и святость, без которой никто не увидит Господа.
\par 15 Наблюдайте, чтобы кто не лишился благодати Божией; чтобы какой горький корень, возникнув, не причинил вреда, и чтобы им не осквернились многие;
\par 16 чтобы не было [между вами] какого блудника, или нечестивца, который бы, как Исав, за одну снедь отказался от своего первородства.
\par 17 Ибо вы знаете, что после того он, желая наследовать благословение, был отвержен; не мог переменить мыслей [отца], хотя и просил о том со слезами.
\par 18 Вы приступили не к горе, осязаемой и пылающей огнем, не ко тьме и мраку и буре,
\par 19 не к трубному звуку и гласу глаголов, который слышавшие просили, чтобы к ним более не было продолжаемо слово,
\par 20 ибо они не могли стерпеть того, что заповедуемо было: если и зверь прикоснется к горе, будет побит камнями (или поражен стрелою);
\par 21 и столь ужасно было это видение, [что и] Моисей сказал: `я в страхе и трепете'.
\par 22 Но вы приступили к горе Сиону и ко граду Бога живаго, к небесному Иерусалиму и тьмам Ангелов,
\par 23 к торжествующему собору и церкви первенцев, написанных на небесах, и к Судии всех Богу, и к духам праведников, достигших совершенства,
\par 24 и к Ходатаю нового завета Иисусу, и к Крови кропления, говорящей лучше, нежели Авелева.
\par 25 Смотрите, не отвратитесь и вы от говорящего. Если те, не послушав глаголавшего на земле, не избегли [наказания], то тем более [не] [избежим] мы, если отвратимся от [Глаголющего] с небес,
\par 26 Которого глас тогда поколебал землю, и Который ныне дал такое обещание: еще раз поколеблю не только землю, но и небо.
\par 27 Слова: `еще раз' означают изменение колеблемого, как сотворенного, чтобы пребыло непоколебимое.
\par 28 Итак мы, приемля царство непоколебимое, будем хранить благодать, которою будем служить благоугодно Богу, с благоговением и страхом,
\par 29 потому что Бог наш есть огнь поядающий.

\chapter{13}

\par 1 Братолюбие [между вами] да пребывает.
\par 2 Страннолюбия не забывайте, ибо через него некоторые, не зная, оказали гостеприимство Ангелам.
\par 3 Помните узников, как бы и вы с ними были в узах, и страждущих, как и сами находитесь в теле.
\par 4 Брак у всех [да будет] честен и ложе непорочно; блудников же и прелюбодеев судит Бог.
\par 5 Имейте нрав несребролюбивый, довольствуясь тем, что есть. Ибо Сам сказал: не оставлю тебя и не покину тебя,
\par 6 так что мы смело говорим: Господь мне помощник, и не убоюсь: что сделает мне человек?
\par 7 Поминайте наставников ваших, которые проповедывали вам слово Божие, и, взирая на кончину их жизни, подражайте вере их.
\par 8 Иисус Христос вчера и сегодня и во веки Тот же.
\par 9 Учениями различными и чуждыми не увлекайтесь; ибо хорошо благодатью укреплять сердца, а не яствами, от которых не получили пользы занимающиеся ими.
\par 10 Мы имеем жертвенник, от которого не имеют права питаться служащие скинии.
\par 11 Так как тела животных, которых кровь для [очищения] греха вносится первосвященником во святилище, сжигаются вне стана, --
\par 12 то и Иисус, дабы освятить людей Кровию Своею, пострадал вне врат.
\par 13 Итак выйдем к Нему за стан, нося Его поругание;
\par 14 ибо не имеем здесь постоянного града, но ищем будущего.
\par 15 Итак будем через Него непрестанно приносить Богу жертву хвалы, то есть плод уст, прославляющих имя Его.
\par 16 Не забывайте также благотворения и общительности, ибо таковые жертвы благоугодны Богу.
\par 17 Повинуйтесь наставникам вашим и будьте покорны, ибо они неусыпно пекутся о душах ваших, как обязанные дать отчет; чтобы они делали это с радостью, а не воздыхая, ибо это для вас неполезно.
\par 18 Молитесь о нас; ибо мы уверены, что имеем добрую совесть, потому что во всем желаем вести себя честно.
\par 19 Особенно же прошу делать это, дабы я скорее возвращен был вам.
\par 20 Бог же мира, воздвигший из мертвых Пастыря овец великого Кровию завета вечного, Господа нашего Иисуса (Христа),
\par 21 да усовершит вас во всяком добром деле, к исполнению воли Его, производя в вас благоугодное Ему через Иисуса Христа. Ему слава во веки веков! Аминь.
\par 22 Прошу вас, братия, примите сие слово увещания; я же не много и написал вам.
\par 23 Знайте, что брат наш Тимофей освобожден, и я вместе с ним, если он скоро придет, увижу вас.
\par 24 Приветствуйте всех наставников ваших и всех святых. Приветствуют вас Италийские.
\par 25 Благодать со всеми вами. Аминь.


\end{document}