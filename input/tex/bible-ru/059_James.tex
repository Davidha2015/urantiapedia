\begin{document}

\title{James}

Jas 1:1  Иаков, раб Бога и Господа Иисуса Христа, двенадцати коленам, находящимся в рассеянии, --радоваться.
Jas 1:2  С великою радостью принимайте, братия мои, когда впадаете в различные искушения,
Jas 1:3  зная, что испытание вашей веры производит терпение;
Jas 1:4  терпение же должно иметь совершенное действие, чтобы вы были совершенны во всей полноте, без всякого недостатка.
Jas 1:5  Если же у кого из вас недостает мудрости, да просит у Бога, дающего всем просто и без упреков, --и дастся ему.
Jas 1:6  Но да просит с верою, нимало не сомневаясь, потому что сомневающийся подобен морской волне, ветром поднимаемой и развеваемой.
Jas 1:7  Да не думает такой человек получить что-нибудь от Господа.
Jas 1:8  Человек с двоящимися мыслями не тверд во всех путях своих.
Jas 1:9  Да хвалится брат униженный высотою своею,
Jas 1:10  а богатый--унижением своим, потому что он прейдет, как цвет на траве.
Jas 1:11  Восходит солнце, [настает] зной, и зноем иссушает траву, цвет ее опадает, исчезает красота вида ее; так увядает и богатый в путях своих.
Jas 1:12  Блажен человек, который переносит искушение, потому что, быв испытан, он получит венец жизни, который обещал Господь любящим Его.
Jas 1:13  В искушении никто не говори: Бог меня искушает; потому что Бог не искушается злом и Сам не искушает никого,
Jas 1:14  но каждый искушается, увлекаясь и обольщаясь собственною похотью;
Jas 1:15  похоть же, зачав, рождает грех, а сделанный грех рождает смерть.
Jas 1:16  Не обманывайтесь, братия мои возлюбленные.
Jas 1:17  Всякое даяние доброе и всякий дар совершенный нисходит свыше, от Отца светов, у Которого нет изменения и ни тени перемены.
Jas 1:18  Восхотев, родил Он нас словом истины, чтобы нам быть некоторым начатком Его созданий.
Jas 1:19  Итак, братия мои возлюбленные, всякий человек да будет скор на слышание, медлен на слова, медлен на гнев,
Jas 1:20  ибо гнев человека не творит правды Божией.
Jas 1:21  Посему, отложив всякую нечистоту и остаток злобы, в кротости примите насаждаемое слово, могущее спасти ваши души.
Jas 1:22  Будьте же исполнители слова, а не слышатели только, обманывающие самих себя.
Jas 1:23  Ибо, кто слушает слово и не исполняет, тот подобен человеку, рассматривающему природные черты лица своего в зеркале:
Jas 1:24  он посмотрел на себя, отошел и тотчас забыл, каков он.
Jas 1:25  Но кто вникнет в закон совершенный, [закон] свободы, и пребудет в нем, тот, будучи не слушателем забывчивым, но исполнителем дела, блажен будет в своем действии.
Jas 1:26  Если кто из вас думает, что он благочестив, и не обуздывает своего языка, но обольщает свое сердце, у того пустое благочестие.
Jas 1:27  Чистое и непорочное благочестие пред Богом и Отцем есть то, чтобы призирать сирот и вдов в их скорбях и хранить себя неоскверненным от мира.
Jas 2:1  Братия мои! имейте веру в Иисуса Христа нашего Господа славы, не взирая на лица.
Jas 2:2  Ибо, если в собрание ваше войдет человек с золотым перстнем, в богатой одежде, войдет же и бедный в скудной одежде,
Jas 2:3  и вы, смотря на одетого в богатую одежду, скажете ему: тебе хорошо сесть здесь, а бедному скажете: ты стань там, или садись здесь, у ног моих, --
Jas 2:4  то не пересуживаете ли вы в себе и не становитесь ли судьями с худыми мыслями?
Jas 2:5  Послушайте, братия мои возлюбленные: не бедных ли мира избрал Бог быть богатыми верою и наследниками Царствия, которое Он обещал любящим Его?
Jas 2:6  А вы презрели бедного. Не богатые ли притесняют вас, и не они ли влекут вас в суды?
Jas 2:7  Не они ли бесславят доброе имя, которым вы называетесь?
Jas 2:8  Если вы исполняете закон царский, по Писанию: возлюби ближнего твоего, как себя самого, --хорошо делаете.
Jas 2:9  Но если поступаете с лицеприятием, то грех делаете, и перед законом оказываетесь преступниками.
Jas 2:10  Кто соблюдает весь закон и согрешит в одном чем-нибудь, тот становится виновным во всем.
Jas 2:11  Ибо Тот же, Кто сказал: не прелюбодействуй, сказал и: не убей; посему, если ты не прелюбодействуешь, но убьешь, то ты также преступник закона.
Jas 2:12  Так говорите и так поступайте, как имеющие быть судимы по закону свободы.
Jas 2:13  Ибо суд без милости не оказавшему милости; милость превозносится над судом.
Jas 2:14  Что пользы, братия мои, если кто говорит, что он имеет веру, а дел не имеет? может ли эта вера спасти его?
Jas 2:15  Если брат или сестра наги и не имеют дневного пропитания,
Jas 2:16  а кто-нибудь из вас скажет им: `идите с миром, грейтесь и питайтесь', но не даст им потребного для тела: что пользы?
Jas 2:17  Так и вера, если не имеет дел, мертва сама по себе.
Jas 2:18  Но скажет кто-нибудь: `ты имеешь веру, а я имею дела': покажи мне веру твою без дел твоих, а я покажу тебе веру мою из дел моих.
Jas 2:19  Ты веруешь, что Бог един: хорошо делаешь; и бесы веруют, и трепещут.
Jas 2:20  Но хочешь ли знать, неосновательный человек, что вера без дел мертва?
Jas 2:21  Не делами ли оправдался Авраам, отец наш, возложив на жертвенник Исаака, сына своего?
Jas 2:22  Видишь ли, что вера содействовала делам его, и делами вера достигла совершенства?
Jas 2:23  И исполнилось слово Писания: `веровал Авраам Богу, и это вменилось ему в праведность, и он наречен другом Божиим'.
Jas 2:24  Видите ли, что человек оправдывается делами, а не верою только?
Jas 2:25  Подобно и Раав блудница не делами ли оправдалась, приняв соглядатаев и отпустив их другим путем?
Jas 2:26  Ибо, как тело без духа мертво, так и вера без дел мертва.
Jas 3:1  Братия мои! не многие делайтесь учителями, зная, что мы подвергнемся большему осуждению,
Jas 3:2  ибо все мы много согрешаем. Кто не согрешает в слове, тот человек совершенный, могущий обуздать и все тело.
Jas 3:3  Вот, мы влагаем удила в рот коням, чтобы они повиновались нам, и управляем всем телом их.
Jas 3:4  Вот, и корабли, как ни велики они и как ни сильными ветрами носятся, небольшим рулем направляются, куда хочет кормчий;
Jas 3:5  так и язык--небольшой член, но много делает. Посмотри, небольшой огонь как много вещества зажигает!
Jas 3:6  И язык--огонь, прикраса неправды; язык в таком положении находится между членами нашими, что оскверняет все тело и воспаляет круг жизни, будучи сам воспаляем от геенны.
Jas 3:7  Ибо всякое естество зверей и птиц, пресмыкающихся и морских животных укрощается и укрощено естеством человеческим,
Jas 3:8  а язык укротить никто из людей не может: это--неудержимое зло; он исполнен смертоносного яда.
Jas 3:9  Им благословляем Бога и Отца, и им проклинаем человеков, сотворенных по подобию Божию.
Jas 3:10  Из тех же уст исходит благословение и проклятие: не должно, братия мои, сему так быть.
Jas 3:11  Течет ли из одного отверстия источника сладкая и горькая [вода]?
Jas 3:12  Не может, братия мои, смоковница приносить маслины или виноградная лоза смоквы. Также и один источник не [может] изливать соленую и сладкую воду.
Jas 3:13  Мудр ли и разумен кто из вас, докажи это на самом деле добрым поведением с мудрою кротостью.
Jas 3:14  Но если в вашем сердце вы имеете горькую зависть и сварливость, то не хвалитесь и не лгите на истину.
Jas 3:15  Это не есть мудрость, нисходящая свыше, но земная, душевная, бесовская,
Jas 3:16  ибо где зависть и сварливость, там неустройство и все худое.
Jas 3:17  Но мудрость, сходящая свыше, во-первых, чиста, потом мирна, скромна, послушлива, полна милосердия и добрых плодов, беспристрастна и нелицемерна.
Jas 3:18  Плод же правды в мире сеется у тех, которые хранят мир.
Jas 4:1  Откуда у вас вражды и распри? не отсюда ли, от вожделений ваших, воюющих в членах ваших?
Jas 4:2  Желаете--и не имеете; убиваете и завидуете--и не можете достигнуть; препираетесь и враждуете--и не имеете, потому что не просите.
Jas 4:3  Просите, и не получаете, потому что просите не на добро, а чтобы употребить для ваших вожделений.
Jas 4:4  Прелюбодеи и прелюбодейцы! не знаете ли, что дружба с миром есть вражда против Бога? Итак, кто хочет быть другом миру, тот становится врагом Богу.
Jas 4:5  Или вы думаете, что напрасно говорит Писание: `до ревности любит дух, живущий в нас'?
Jas 4:6  Но тем большую дает благодать; посему и сказано: Бог гордым противится, а смиренным дает благодать.
Jas 4:7  Итак покоритесь Богу; противостаньте диаволу, и убежит от вас.
Jas 4:8  Приблизьтесь к Богу, и приблизится к вам; очистите руки, грешники, исправьте сердца, двоедушные.
Jas 4:9  Сокрушайтесь, плачьте и рыдайте; смех ваш да обратится в плач, и радость--в печаль.
Jas 4:10  Смиритесь пред Господом, и вознесет вас.
Jas 4:11  Не злословьте друг друга, братия: кто злословит брата или судит брата своего, того злословит закон и судит закон; а если ты судишь закон, то ты не исполнитель закона, но судья.
Jas 4:12  Един Законодатель и Судия, могущий спасти и погубить; а ты кто, который судишь другого?
Jas 4:13  Теперь послушайте вы, говорящие: `сегодня или завтра отправимся в такой-то город, и проживем там один год, и будем торговать и получать прибыль';
Jas 4:14  вы, которые не знаете, что случится завтра: ибо что такое жизнь ваша? пар, являющийся на малое время, а потом исчезающий.
Jas 4:15  Вместо того, чтобы вам говорить: `если угодно будет Господу и живы будем, то сделаем то или другое', --
Jas 4:16  вы, по своей надменности, тщеславитесь: всякое такое тщеславие есть зло.
Jas 4:17  Итак, кто разумеет делать добро и не делает, тому грех.
Jas 5:1  Послушайте вы, богатые: плачьте и рыдайте о бедствиях ваших, находящих на вас.
Jas 5:2  Богатство ваше сгнило, и одежды ваши изъедены молью.
Jas 5:3  Золото ваше и серебро изоржавело, и ржавчина их будет свидетельством против вас и съест плоть вашу, как огонь: вы собрали себе сокровище на последние дни.
Jas 5:4  Вот, плата, удержанная вами у работников, пожавших поля ваши, вопиет, и вопли жнецов дошли до слуха Господа Саваофа.
Jas 5:5  Вы роскошествовали на земле и наслаждались; напитали сердца ваши, как бы на день заклания.
Jas 5:6  Вы осудили, убили Праведника; Он не противился вам.
Jas 5:7  Итак, братия, будьте долготерпеливы до пришествия Господня. Вот, земледелец ждет драгоценного плода от земли и для него терпит долго, пока получит дождь ранний и поздний.
Jas 5:8  Долготерпите и вы, укрепите сердца ваши, потому что пришествие Господне приближается.
Jas 5:9  Не сетуйте, братия, друг на друга, чтобы не быть осужденными: вот, Судия стоит у дверей.
Jas 5:10  В пример злострадания и долготерпения возьмите, братия мои, пророков, которые говорили именем Господним.
Jas 5:11  Вот, мы ублажаем тех, которые терпели. Вы слышали о терпении Иова и видели конец [оного] от Господа, ибо Господь весьма милосерд и сострадателен.
Jas 5:12  Прежде же всего, братия мои, не клянитесь ни небом, ни землею, и никакою другою клятвою, но да будет у вас: `да, да' и `нет, нет', дабы вам не подпасть осуждению.
Jas 5:13  Злостраждет ли кто из вас, пусть молится. Весел ли кто, пусть поет псалмы.
Jas 5:14  Болен ли кто из вас, пусть призовет пресвитеров Церкви, и пусть помолятся над ним, помазав его елеем во имя Господне.
Jas 5:15  И молитва веры исцелит болящего, и восставит его Господь; и если он соделал грехи, простятся ему.
Jas 5:16  Признавайтесь друг пред другом в проступках и молитесь друг за друга, чтобы исцелиться: много может усиленная молитва праведного.
Jas 5:17  Илия был человек, подобный нам, и молитвою помолился, чтобы не было дождя: и не было дождя на землю три года и шесть месяцев.
Jas 5:18  И опять помолился: и небо дало дождь, и земля произрастила плод свой.
Jas 5:19  Братия! если кто из вас уклонится от истины, и обратит кто его,
Jas 5:20  пусть тот знает, что обративший грешника от ложного пути его спасет душу от смерти и покроет множество грехов.


\end{document}