\begin{document}

\title{2 Moseboken}


\chapter{1}

\par 1 Och dessa äro namnen på Israels söner, som kommo till Egypten; med Jakob kommo de, var och en med sitt hus:
\par 2 Ruben, Simeon, Levi och Juda,
\par 3 Isaskar, Sebulon och Benjamin,
\par 4 Dan och Naftali, Gad och Aser.
\par 5 Tillsammans utgjorde de som hade utgått från Jakobs länd sjuttio personer; men Josef var redan förut i Egypten.
\par 6 Och Josef dog och alla hans bröder och hela det släktet.
\par 7 Men Israels barn voro fruktsamma och växte till och förökade sig och blevo övermåttan talrika, så att landet blev uppfyllt av dem.
\par 8 Då uppstod en ny konung över Egypten, en som icke visste av Josef.
\par 9 Och denne sade till sitt folk: "Se, Israels barns folk är oss för stort och mäktigt.
\par 10 Välan, låt oss då gå klokt till väga med dem; eljest kunde de ännu mer föröka sig; och om ett krig komme på, kunde de förena sig med våra fiender och begynna krig mot oss och sedan draga bort ur landet."
\par 11 Alltså satte man arbetsfogdar över dem och förtryckte dem med trälarbeten. Och de måste bygga åt Farao förrådsstäder, Pitom och Raamses.
\par 12 Men ju mer man förtryckte dem, dess mer förökade de sig, och dess mer utbredde de sig, så att man begynte gruva sig för Israels barn.
\par 13 Därför pålade egyptierna Israels barn ytterligare tvångsarbeten
\par 14 och förbittrade deras liv med hårt arbete på murbruk och tegel och med alla slags arbeten på marken korteligen, med tvångsarbeten av alla slag, som de läto dem utföra
\par 15 Och konungen i Egypten talade till de hebreiska kvinnor - den ena hette Sifra, den andra Pua - som hjälpte barnaföderskorna,
\par 16 han sade: "När I förlösen de hebreiska kvinnorna, så sen efter, då de föda: om det är ett gossebarn, så döden det; är det ett flickebarn, så må det leva."
\par 17 Men hjälpkvinnorna fruktade Gud och gjorde icke såsom konungen i Egypten hade sagt till dem, utan läto barnen leva.
\par 18 Då kallade konungen i Egypten hjälpkvinnorna till sig och sade till dem: "Varför gören I så och låten barnen leva?"
\par 19 Hjälpkvinnorna svarade Farao: "De hebreiska kvinnorna äro icke såsom de egyptiska. De äro kraftigare; förrän hjälpkvinnan kommer till dem, hava de fött."
\par 20 Och Gud lät det gå väl för hjälpkvinnorna; och folket förökade sig och blev mycket talrikt.
\par 21 Eftersom hjälpkvinnorna fruktade Gud, lät han deras hus förkovras.
\par 22 Då bjöd Farao allt sitt folk och sade: "Alla nyfödda gossebarn skolen I kasta i Nilfloden, men all flickebarn mån I låta leva."

\chapter{2}

\par 1 Och en man av Levis hus gick åstad och tog till hustru Levis dotter.
\par 2 Och hustrun blev havande och födde en son. Och hon såg att det var ett vackert barn och dolde honom i tre månader.
\par 3 Men när hon icke längre kunde dölja honom, tog hon en kista av rör, beströk den med jordbeck och tjära och lade barnet däri och satte den så i vassen vid stranden av Nilfloden.
\par 4 Och hans syster ställde sig ett stycke därifrån, för att se huru det skulle gå med honom.
\par 5 Och Faraos dotter kom ned till floden för att bada, och hennes tärnor gingo utmed floden. När hon nu fick se kistan i vassen, sände hon sin tjänarinna dit och lät hämta den till sig.
\par 6 Och när hon öppnade den, fick hon se barnet och såg att det var en gosse, och han grät. Då ömkade hon sig över honom och sade: "Detta är ett av de hebreiska barnen."
\par 7 Men hans syster frågade Faraos dotter: "Vill du att jag skall gå och kalla hit till dig en hebreisk amma som kan amma upp barnet åt dig?"
\par 8 Faraos dotter svarade henne: "Ja, gå." Då gick flickan och kallade dit barnets moder.
\par 9 Och Faraos dotter sade till henne: "Tag detta barn med dig, och amma upp det åt mig, så vill jag giva dig lön därför." Och kvinnan tog barnet och ammade upp det.
\par 10 När sedan gossen hade vuxit upp; förde hon honom till Faraos dotter, och denna upptog honom såsom sin son och gav honom namnet Mose, "ty", sade hon, "ur vattnet har jag dragit upp honom".
\par 11 På den tiden hände sig att Mose, sedan han hade blivit stor, gick ut till sina bröder och såg på deras trälarbete. Och han fick se att en egyptisk man slog en hebreisk man, en av hans bröder.
\par 12 då vände han sig åt alla sidor, och när han såg att ingen annan människa fanns där, slog han ihjäl egyptiern och gömde honom i sanden.
\par 13 Dagen därefter gick han åter ut och fick då se två hebreiska män träta med varandra. Då sade han till den som gjorde orätt: "Skall du slå din landsman?"
\par 14 Han svarade: "Vem har satt dig till hövding och domare över oss? Vill du dräpa mig, såsom du dräpte egyptiern?" Då blev Mose förskräckt och tänkte: "Så har då saken blivit känd."
\par 15 Också fick Farao höra om denna sak och ville dräpa Mose. Men Mose flydde bort undan Farao; och han stannade i Midjans land; där satte han sig vid en brunn.
\par 16 Och prästen i Midjan hade sju döttrar. Dessa kommo nu för att hämta upp vatten och skulle fylla hoarna för att vattna sin faders får.
\par 17 Då kommo herdarna och ville driva bort dem; men Mose stod upp och hjälpte dem och vattnade deras får.
\par 18 När de sedan kommo hem till sin fader Reguel, sade han: "Varför kommen I så snart hem i dag?"
\par 19 De svarade: "En egyptisk man hjälpte oss mot herdarna; därtill hämtade han upp vatten åt oss och vattnade fåren."
\par 20 Då sade han till sina döttrar: "Var är han då? Varför läten I mannen bliva kvar där? Inbjuden honom att komma och äta med oss".
\par 21 Och Mose beslöt sig för att stanna hos mannen, och denne gav åt Mose sin dotter Sippora till hustru.
\par 22 Hon födde en son, och han gav honom namnet Gersom, "ty", sade han, "jag är en främling i ett land som icke är mitt".
\par 23 Så förflöt en lång tid, och därunder dog konungen i Egypten. Men Israels barn suckade över sin träldom och klagade; och deras rop över träldomen steg upp till Gud.
\par 24 Och Gud hörde deras jämmer, och Gud tänkte på sitt förbund med Abraham, Isak och Jakob.
\par 25 Och Gud såg till Israels barn, och Gud lät sig vårda om dem.

\chapter{3}

\par 1 Och Mose vaktade fåren åt sin svärfader Jetro, prästen i Midjan. Och han drev en gång fåren bortom öknen och kom så till Guds berg Horeb.
\par 2 Där uppenbarade sig HERRENS ängel för honom i en eldslåga som slog upp ur en buske. Han såg att busken brann av elden, och att busken dock icke blev förtärd.
\par 3 Då tänkte Mose: "Jag vill gå ditbort och betrakta den underbara synen och se varför busken icke brinner upp."
\par 4 När då HERREN såg att han gick åstad för att se, ropade Gud till honom ur busken och sade: "Mose! Mose!" Han svarade: "Här är jag."
\par 5 Då sade han: "Träd icke hit; drag dina skor av dina fötter, ty platsen där du står är helig mark."
\par 6 Och han sade ytterligare: "Jag är din faders Gud, Abrahams Gud, Isaks Gud och Jakobs Gud." Då skylde Mose sitt ansikte, ty han fruktade för att se på Gud.
\par 7 Och HERREN sade: "Jag har nogsamt sett mitt folks betryck i Egypten, och jag har hört huru de ropa över sina plågare; jag vet vad de måste lida.
\par 8 Därför har jag stigit ned för att rädda dem ur egyptiernas våld och föra dem från det landet upp till ett gott och rymligt land, ett land som flyter av mjölk och honung, det land där kananéer, hetiter, amoréer, perisséer, hivéer och jebuséer bo.
\par 9 Fördenskull, eftersom Israels barns rop har kommit till mig, och jag därjämte har sett huru egyptierna förtrycka dem,
\par 10 därför må du nu gå åstad, jag vill sända dig till Farao; och du skall föra mitt folk, Israels barn, ut ur Egypten"
\par 11 Men Mose sade till Gud: "Vem är jag, att jag skulle gå till Farao, och att jag skulle föra Israels barn ut ur Egypten?"
\par 12 Han svarade: "Jag vill vara med dig. Och detta skall för dig vara tecknet på att det är jag som har sänt dig: när du har fört folket ut ur Egypten, skolen I hålla gudstjänst på detta berg."
\par 13 Då sade Mose till Gud: "När jag nu kommer till Israels barn och säger till dem: 'Edra fäders Gud har sänt mig till eder', och de fråga mig; 'Vad är hans namn?', vad skall jag då svara dem?"
\par 14 Gud sade till Mose: "Jag är den jag är." Och han sade vidare: "Så skall du säga till Israels barn: 'Jag är' har sänt mig till eder.
\par 15 Och Gud sade ytterligare till Mose: "Så skall du säga till Israels barn: HERREN, edra fäders Gud, Abrahams Gud, Isaks Gud och Jakobs Gud, har sänt mig till eder. Detta skall vara mitt namn evinnerligen, och så skall man nämna mig från släkte till släkte.
\par 16 Gå nu åstad och församla de äldste i Israel, och säg till dem: HERREN, edra fäders Gud, Abrahams, Isaks och Jakobs Gud, har uppenbarat sig för mig, och han har sagt: 'Jag har sett till eder och har sett det som vederfares eder Egypten;
\par 17 därför är nu mitt ord: jag vill föra eder bort ifrån betrycket i Egypten upp till kananéernas, hetiternas, amoréernas, perisséernas, hivéernas och jebuséernas land, ett land som flyter av mjölk och honung.'
\par 18 Och de skola lyssna till dina ord; och du skall tillika med de äldste i Israel gå till konungen i Egypten, och I skolen säga till honom: HERREN, hebréernas Gud, har visat sig för oss, så låt oss nu gå tre dagsresor in i öknen och offra åt HERREN, vår Gud.'
\par 19 Dock vet jag att konungen i Egypten icke skall tillstädja eder att gå, icke ens när han får känna min starka hand.
\par 20 Men jag skall räcka ut min hand och slå Egypten med alla slags under, som jag vill göra där; sedan skall han släppa eder.
\par 21 Och jag vill låta detta folk finna nåd för egyptiernas ögon, så att I, när I dragen bort, icke skolen draga bort med tomma händer;
\par 22 utan var kvinna skall av sin grannkvinna och av den främmande kvinna som bor i hennes hus begära klenoder av silver och guld, så ock kläder. Dessa skolen I sätta på edra söner och döttrar. Så skolen I taga byte från egyptierna."

\chapter{4}

\par 1 Mose svarade och sade: "Men om de nu icke tro mig eller lyssna till mina ord, utan säga: 'HERREN har icke uppenbarat sig för dig'?"
\par 2 Då sade HERREN till honom: "Vad är det du har i din hand?"
\par 3 Han svarade: "En stav." Han sade: "Kasta den på marken." När han då kastade den på marken, förvandlades den till en orm; och Mose flydde för honom.
\par 4 Men HERREN sade till Mose: "Räck ut din hand och tag honom i stjärten." Då räckte han ut sin hand och grep honom; och han förvandlades åter till en stav i hans hand.
\par 5 Och HERREN sade: "Så skola de. tro att HERREN, deras fäders Gud, Abrahams Gud, Isaks Gud och Jakobs Gud, har uppenbarat sig för dig.
\par 6 Och HERREN sade ytterligare till honom: "Stick din hand i barmen." Och han stack sin hand i barmen. När han sedan drog ut den, se, då var handen vit såsom snö av spetälska.
\par 7 Åter sade han. "Stick din hand tillbaka i barmen." Och han stack sin hand tillbaka i barmen. När han sedan drog ut den igen ur barmen, se, då var den åter lik hans övriga kropp.
\par 8 Och HERREN sade: "Om de icke vilja tro dig eller akta på det första tecknet, så måste de tro det andra tecknet.
\par 9 Men om de icke ens tro dessa två tecken eller lyssna till dina ord, så tag av Nilflodens vatten och gjut ut det på torra landet, så skall vattnet, som du har tagit ur floden, förvandlas till blod på torra landet."
\par 10 Då sade Mose till HERREN: "Ack I Herre, jag är ingen talför man; jag har icke varit det förut, och jag är det icke heller nu, sedan du har talat till din tjänare, ty jag har ett trögt målföre och en trög tunga.
\par 11 HERREN sade till honom: "Vem har givit människan munnen, eller vem gör henne stum eller döv, seende eller blind? Är det icke jag, HERREN?
\par 12 Så gå nu åstad, jag skall vara med din mun och lära dig vad du skall tala."
\par 13 Men han sade: "Ack Herre, sänd ditt budskap med vilken annan du vill."
\par 14 Då upptändes HERRENS vrede mot Mose, och han sade: "Har du icke din broder Aron, leviten? Jag vet att han är en man som kan tala. Och han går nu åstad för att möta dig, och när han får se dig, skall han glädjas i sitt hjärta.
\par 15 Och du skall tala till honom och lägga orden i hans mun; och jag skall vara med din mun och med hans mun, och jag skall lära eder vad I skolen göra.
\par 16 Och han skall tala för dig till folket; alltså skall han vara för dig såsom mun, och du skall vara för honom såsom en gud.
\par 17 Och du skall taga i din hand denna stav, med vilken du skall göra dina tecken."
\par 18 Därefter vände Mose tillbaka till sin svärfader Jeter och sade till honom: "Låt mig vända tillbaka till mina bröder i Egypten, för att se om de ännu leva." Jetro sade till Mose: "Gå i frid."
\par 19 Och HERREN sade till Mose i Midjan: "Vänd tillbaka till Egypten, ty alla de män äro döda, som stodo efter ditt liv."
\par 20 Då tog Mose sin hustru och sina söner och satte dem på sin åsna och for tillbaka till Egyptens land; och Mose tog Guds stav i sin hand.
\par 21 Och HERREN sade till Mose: "När du nu vänder tillbaka till Egypten så se till, att du inför Farao gör alla de under som jag har givit dig makt att göra. Men jag skall förstocka hans hjärta, så att han icke släpper folket.
\par 22 Och då skall du säga till Farao: Så säger HERREN: Israel är min förstfödde son,
\par 23 och jag har sagt till dig: 'Släpp min son, så att han kan hålla gudstjänst åt mig.' Men du har icke velat släppa honom. Därför skall jag nu dräpa din förstfödde son.
\par 24 Och under resan hände sig att HERREN på ett viloställe kom emot honom och ville döda honom.
\par 25 Då tog Sippora en skarp sten och skar bort förhuden på sin son och berörde honom därmed nedtill och sade: "Du är mig en blodsbrudgum."
\par 26 Så lät han honom vara. Då sade hon åter: "Ja, en blodsbrudgum till omskärelse."
\par 27 Och HERREN sade till Aron: "Gå åstad och möt Mose i öknen." Då gick han åstad och träffade honom på Guds berg; och han kysste honom.
\par 28 Och Mose berättade för Aron allt vad HERREN hade talat, när han sände honom, och om alla de tecken som han hade bjudit honom att göra.
\par 29 Sedan gingo Mose och Aron åstad och församlade alla Israels barns äldste.
\par 30 Och Aron omtalade allt vad HERREN hade talat till Mose; och han gjorde tecknen inför folkets ögon.
\par 31 Då trodde folket; och när de hörde att HERREN hade sett till Israels barn, och att han hade sett deras betryck, böjde de sig ned och tillbådo.

\chapter{5}

\par 1 Därefter kommo Mose och Aron och sade till Farao: "Så säger HERREN, Israels Gud: Släpp mitt folk, så att de kunna hålla högtid åt mig i öknen."
\par 2 Men Farao svarade: "Vem är HERREN, eftersom jag på hans befallning skulle släppa Israel? Jag vet icke av HERREN och vill ej heller släppa Israel."
\par 3 Då sade de: "Hebréernas Gud har visat sig för oss. Så låt oss nu gå tre dagsresor in i öknen och offra åt HERREN, vår Gud, för att han icke må komma över oss med pest eller med svärd."
\par 4 Men konungen i Egypten svarade dem: "Mose och Aron, varför dragen I folket ifrån dess arbete? Gån bort till edra dagsverken.
\par 5 Ytterligare sade Farao: "Folket är ju redan alltför talrikt i landet, och likväl viljen I skaffa dem frihet ifrån deras dagsverken!"
\par 6 Därefter bjöd Farao samma dag fogdarna och tillsyningsmännen över folket och sade:
\par 7 "I skolen icke vidare såsom förut giva folket halm till att göra tegel. Låten dem själva gå och skaffa sig halm.
\par 8 Men samma antal tegel som de förut hava gjort skolen I ändå ålägga dem, utan något avdrag; ty de äro lata, därför ropa de och säga: 'Låt oss gå och offra åt vår Gud.'
\par 9 Man måste lägga tungt arbete på dessa människor, så att de därigenom få något att göra och icke akta på lögnaktigt tal."
\par 10 Då gingo fogdarna och tillsyningsmännen över folket ut och sade till folket: "Så säger Farao: Jag vill icke längre giva eder halm.
\par 11 Gån själva och skaffen eder halm, var I kunnen finna sådan; men i edert arbete skall intet avdrag göras."
\par 12 Då spridde sig folket över hela Egyptens land och samlade strå för att bruka det såsom halm.
\par 13 Och fogdarna drevo på dem och sade: "Fullgören edert arbete, var dag det för den dagen bestämda, likasom när man gav eder halm."
\par 14 Och Israels barns tillsyningsmän, de som Faraos fogdar hade satt över dem, fingo uppbära hugg och slag, och man sade till dem: "Varför haven I icke såsom förut fullgjort edert förelagda dagsverke i tegel, varken i går eller i dag?"
\par 15 Då kommo Israels barns tillsyningsmän och ropade till Farao och sade: "Varför gör du så mot dina tjänare?
\par 16 Ingen halm giver man åt dina tjänare, och likväl säger man till oss: 'Skaffen fram tegel.' Och se, dina tjänare få nu uppbära hugg och slag, fastän skulden ligger hos ditt eget folk."
\par 17 Men han svarade: "I ären lata, ja lata ären I. Därför sägen I: 'Låt oss gå och offra åt HERREN.'
\par 18 Nej, gån i stället till edert arbete. Halm skall man icke giva eder, men det bestämda antalet tegel måsten I ändå lämna."
\par 19 Då märkte Israels barns tillsyningsmän att det var illa ställt för dem, eftersom de hade fått det svaret att de icke skulle få något avdrag i det antal tegel, som de skulle lämna för var dag.
\par 20 Och när de kommo ut ifrån Farao, träffade de Mose och Aron, som stodo där för att möta dem;
\par 21 och de sade till dem: "Må HERREN hemsöka eder och döma eder, eftersom I haven gjort oss förhatliga för Farao och hans tjänare och satt dem svärdet i hand till att dräpa oss.
\par 22 Då vände sig Mose åter till HERREN och sade: "Herre, varför har du gjort så illa mot detta folk? Varför har du sänt mig?
\par 23 Allt ifrån den tid då jag gick till Farao för att tala i ditt namn har han ju gjort illa mot detta folk, och du har ingalunda räddat ditt folk.

\chapter{6}

\par 1 Men HERREN sade till Mose: "Nu skall du få se vad jag skall göra med Farao; ty genom min starka hand skall han nödgas släppa dem, ja, han skall genom min starka hand nödgas driva dem ut ur sitt land."
\par 2 Och Gud talade till Mose och sade till honom: "Jag är HERREN.
\par 3 För Abraham, Isak och Jakob uppenbarade jag mig såsom 'Gud den Allsmäktige', men under mitt namn 'HERREN' var jag icke känd av dem.
\par 4 Och jag upprättade ett förbund med dem och lovade att giva dem Kanaans land, det land där de bodde såsom främlingar.
\par 5 Och nu har jag hört Israels barns jämmer över att egyptierna hålla dem i träldom, och jag har kommit ihåg mitt förbund.
\par 6 Säg därför till Israels barn: 'Jag är HERREN, och jag skall föra eder ut från trälarbetet hos egyptierna och rädda eder från träldomen under dem, och jag skall förlossa eder med uträckt arm och genom stora straffdomar.
\par 7 Och jag skall taga eder till mitt folk och vara eder Gud; och I skolen förnimma att jag är HERREN eder Gud, han som för eder ut från trälarbetet hos egyptierna.
\par 8 Och jag skall föra eder till det land som jag med upplyft hand har lovat giva åt Abraham, Isak och Jakob; det skall jag giva eder till besittning. Jag är HERREN.'"
\par 9 Detta allt sade Mose till Israels barn, men de hörde icke på Mose, av otålighet och för det hårda arbetets skull.
\par 10 Därefter talade HERREN till Mose och sade:
\par 11 "Gå och tala med Farao, konungen i Egypten, att han släpper Israels barn ut ur sitt land."
\par 12 Men Mose talade inför HERREN och sade: "Israels barn höra ju icke på mig; huru skulle då Farao vilja höra mig - mig som har oomskurna läppar?"
\par 13 Men HERREN talade till Mose och Aron och gav dem befallning till Israels barn och till Farao, konungen i Egypten, om att Israels barn skulle föras ut ur Egyptens land.
\par 14 Dessa voro huvudmännen för deras familjer. Rubens, Israels förstföddes, söner voro Hanok och Pallu, Hesron och Karmi. Dessa voro Rubens släkter.
\par 15 Simeons söner voro Jemuel, Jamin, Ohad, Jakin, Sohar och Saul, den kananeiska kvinnans son. Dessa voro Simeons släkter.
\par 16 Och dessa voro namnen på Levis söner, efter deras ättföljd: Gerson, Kehat och Merari. Och Levi blev ett hundra trettiosju år gammal.
\par 17 Gersons söner voro Libni och Simei, efter deras släkter.
\par 18 Kehats söner voro Amram, Jishar, Hebron och Ussiel. Och Kehat blev ett hundra trettiotre år gammal.
\par 19 Meraris söner voro Maheli och Musi. Dessa voro leviternas släkter, efter deras ättföljd.
\par 20 Men Amram tog sin faders syster Jokebed till hustru, och hon födde åt honom Aron och Mose. Och Amram blev ett hundra trettiosju år gammal.
\par 21 Jishars söner voro Kora, Nefeg och Sikri.
\par 22 Ussiels söner voro Misael, Elsafan och Sitri.
\par 23 Och Aron tog till hustru Eliseba, Amminadabs dotter, Nahesons syster, och hon födde åt honom Nadab och Abihu, Eleasar och Itamar.
\par 24 Koras söner voro Assir, Elkana och Abiasaf. Dessa voro koraiternas släkter.
\par 25 Och Eleasar, Arons son, tog en av Putiels döttrar till hustru, och hon födde åt honom Pinehas. Dessa voro huvudmännen för leviternas familjer, efter deras släkter.
\par 26 Så förhöll det sig med Aron och Mose, dem till vilka HERREN sade: "Fören Israels barn ut ur Egyptens land, efter deras härskaror.
\par 27 Det var dessa som talade med Farao, konungen i Egypten, om att de skulle föra Israels barn ut ur Egypten. Så förhöll det sig med Mose och Aron,
\par 28 Och när HERREN talade till Mose i Egyptens land,
\par 29 talade han så till Mose: "Jag är HERREN. Tala till Farao, konungen i Egypten, allt vad jag talar till dig."
\par 30 Men Mose sade inför HERREN: "Se, jag har oomskurna läppar; huru skulle då Farao vilja höra på mig?"

\chapter{7}

\par 1 Men HERREN sade till Mose: "Se, jag har satt dig att vara såsom en gud för Farao, och din broder Aron skall vara din profet.
\par 2 Du skall tala allt vad jag bjuder dig; sedan skall din broder Aron tala med Farao om att han måste släppa Israels barn ut ur sitt land.
\par 3 Men jag skall förhärda Faraos hjärta och skall göra många tecken och under i Egyptens land.
\par 4 Farao skall icke höra på eder; men jag skall lägga min hand på Egypten och skall föra mina härskaror, mitt folk, Israels barn, ut ur Egyptens land, genom stora straffdomar.
\par 5 Och egyptierna skola förnimma att jag är HERREN, när jag räcker ut min hand över Egypten och för Israels barn ut från dem."
\par 6 Och Mose och Aron gjorde så; de gjorde såsom HERREN hade bjudit dem.
\par 7 Men Mose var åttio år gammal och Aron åttiotre år gammal, när de talade med Farao.
\par 8 Och HERREN talade till Mose och Aron och sade:
\par 9 "När Farao talar till eder och säger: 'Låten oss se något under', då skall du säga till Aron: 'Tag din stav och kasta den inför Farao', så skall den bliva en stor orm."
\par 10 Då gingo Mose och Aron till Farao och gjorde såsom HERREN hade bjudit. Aron kastade sin stav inför Farao och hans tjänare, och den blev en stor orm.
\par 11 Då kallade också Farao till sig sina vise och trollkarlar; och dessa, de egyptiska spåmännen, gjorde ock detsamma genom sina hemliga konster:
\par 12 de kastade var och en sin stav, och dessa blevo stora ormar. Men Arons stav uppslukade deras stavar.
\par 13 Dock förblev Faraos hjärta förstockat, och han hörde icke på dem, såsom HERREN hade sagt.
\par 14 Därefter sade HERREN till Mose: "Faraos hjärta är tillslutet, han vill icke släppa folket.
\par 15 Gå till Farao i morgon bittida - han går nämligen då ut till vattnet - och ställ dig i hans väg, på stranden av Nilfloden. Och tag i din hand staven som förvandlades till en orm.
\par 16 Och säg till honom: HERREN, hebréernas Gud, sände mig till dig och lät säga dig: 'Släpp mitt folk, så att de kunna hålla gudstjänst åt mig i öknen.' Men se, du har hitintills icke velat höra.
\par 17 Därför säger nu HERREN så: 'Härav skall du förnimma att jag är HERREN: se, med staven som jag håller i min hand vill jag slå på vattnet i Nilfloden, och då skall det förvandlas till blod.
\par 18 Och fiskarna i floden skola dö, och floden skall bliva stinkande, så att egyptierna skola vämjas vid att dricka vatten ifrån floden.'"
\par 19 Och HERREN sade till Mose: "Säg till Aron: Tag din stav, och räck ut din hand över egyptiernas vatten, över deras strömmar, kanaler och sjöar och alla andra vattensamlingar, så skola de bliva blod; över hela Egyptens land skall vara blod, både i träkärl och i stenkärl."
\par 20 Och Mose och Aron gjorde såsom HERREN hade bjudit. Han lyfte upp staven och slog vattnet i Nilfloden inför Faraos och hans tjänares ögon; då förvandlades allt vatten floden till blod.
\par 21 Och fiskarna i floden dogo, och floden blev stinkande, så att egyptierna icke kunde dricka vatten ifrån floden; och blodet var över hela Egyptens land.
\par 22 Men de egyptiska spåmännen gjorde detsamma genom sina hemliga konster. Så förblev Faraos hjärta förstockat, och han hörde icke på dem, såsom HERREN hade sagt.
\par 23 Och Farao vände om och gick hem och aktade icke heller på detta.
\par 24 Men i hela Egypten grävde man runt omkring Nilfloden efter vatten till att dricka; ty vattnet i floden kunde man icke dricka.
\par 25 Och så förgingo sju dagar efter det att HERREN hade slagit Nilfloden.

\chapter{8}

\par 1 Därefter sade HERREN till Mose: "Gå till Farao och säg till honom: Så säger HERREN: Släpp mitt folk, så att de kunna hålla gudstjänst åt mig.
\par 2 Men om du icke vill släppa dem, se, då skall jag hemsöka hela ditt land med paddor.
\par 3 Nilfloden skall frambringa ett vimmel av paddor, och de skola stiga upp och komma in i ditt hus och i din sovkammare och upp i din säng, och in i dina tjänares hus och bland ditt folk, och i dina bakugnar och baktråg.
\par 4 Ja, på dig själv och ditt folk och alla dina tjänare skola paddorna stiga upp."
\par 5 Och HERREN sade till Mose: "Säg till Aron: Räck ut din hand med din stav över strömmarna, kanalerna och sjöarna, och låt så paddor stiga upp över Egyptens land."
\par 6 Då räckte Aron ut sin hand över Egyptens vatten, och paddor stego upp och övertäckte Egyptens land.
\par 7 Men spåmännen gjorde detsamma genom sina hemliga konster och läto paddor stiga upp över Egyptens land.
\par 8 Då kallade Farao Mose och Aron till sig och sade: "Bedjen till HERREN, att han tager bort paddorna från mig och mitt folk, så skall jag släppa folket, så att de kunna offra åt HERREN."
\par 9 Mose sade till Farao: "Dig vare tillstatt att förelägga mig en tid inom vilken jag, genom att bedja för dig och dina tjänare och ditt folk, skall skaffa bort paddorna från dig och dina hus, så att de finnas kvar allenast i Nilfloden.
\par 10 Han svarade: "Till i morgon." Då sade han: "Må det ske såsom du har sagt, så att du får förnimma att ingen är såsom HERREN, vår Gud.
\par 11 Paddorna skola vika bort ifrån dig och dina hus och ifrån dina tjänare och ditt folk och skola finnas kvar allenast i Nilfloden."
\par 12 Så gingo Mose och Aron ut ifrån Farao. Och Mose ropade till HERREN om hjälp mot paddorna som han hade låtit komma över Farao.
\par 13 Och HERREN gjorde såsom Mose hade begärt: paddorna dogo och försvunno ifrån husen, gårdarna och fälten.
\par 14 Och man kastade dem tillsammans i högar, här en och där en; och landet uppfylldes av stank.
\par 15 Men när Farao såg att han hade fått lättnad, tillslöt han sitt hjärta och hörde icke på dem, såsom HERREN hade sagt.
\par 16 Därefter sade HERREN till Mose: "Säg till Aron: Räck ut din stav och slå i stoftet på jorden, så skall därav bliva mygg i hela Egyptens land."
\par 17 Och de gjorde så: Aron räckte ut sin hand med sin stav och slog i stoftet på jorden; då kom mygg på människor och boskap. Av allt stoft på marken blev mygg i hela Egyptens land.
\par 18 Och spåmännen ville göra detsamma genom sina hemliga konster och försökte skaffa fram mygg, men de kunde icke. Och myggen kom på människor och boskap.
\par 19 Då sade spåmännen till Farao: "Detta är Guds finger." Men Faraos hjärta förblev förstockat, och han hörde icke på dem, såsom HERREN hade sagt.
\par 20 Därefter sade HERREN till Mose: "Träd i morgon bittida fram inför Farao - han går nämligen då ut till vattnet - och säg till honom: så säger HERREN: Släpp mitt folk, så att de kunna hålla guds tjänst åt mig.
\par 21 Ty om du icke släpper mitt folk, de, då skall jag sända svärmar av flugor över dig och dina tjänare och ditt folk och dina hus, så att egyptiernas hus skola bliva uppfyllda av flugsvärmar, ja, själva marken på vilken de stå.
\par 22 Men på den dagen skall jag göra ett undantag för landet Gosen, där mitt folk bor, så att inga flugsvärmar skola finnas där, på det att du må förnimma att jag är HERREN här i landet.
\par 23 Så skall jag förlossa mitt folk och göra en åtskillnad mellan mitt folk och ditt. I morgon skall detta tecken ske."
\par 24 Och HERREN gjorde så: stora flugsvärmar kommo in i Faraos och i hans tjänares hus; och överallt i Egypten blev landet fördärvat av flugsvärmarna.
\par 25 Då kallade Farao Mose och Aron till sig och sade: "Gån åstad och offren åt eder Gud här i landet."
\par 26 Men Mose svarade: "Det går icke an att vi göra så; ty vi offra åt HERREN, vår Gud, sådant som för egyptierna är en styggelse. Om vi nu inför egyptiernas ögon offra sådant som för dem är en styggelse, skola de säkert stena oss.
\par 27 Så låt oss nu gå tre dagsresor in i öknen och offra åt HERREN, vår Gud, såsom han befaller oss."
\par 28 Då sade Farao: "Jag vill släppa eder, så att I kunnen offra åt HERREN, eder Gud, i öknen; allenast mån I icke gå alltför långt bort. Bedjen för mig.
\par 29 Mose svarade: "Ja, när jag kommer ut från dig, skall jag bedja till HERREN, så att flugsvärmarna i morgon vika bort ifrån Farao, ifrån hans tjänare och hans folk. Allenast må Farao icke mer handla svikligt och vägra att släppa folket, så att de kunna offra åt HERREN.
\par 30 Och Mose gick ut ifrån Farao och bad till HERREN.
\par 31 Och HERREN gjorde såsom Mose sade begärt: han skaffade bort flugsvärmarna ifrån Farao, ifrån hans tjänare och hans folk, så att icke en enda fluga blev kvar.
\par 32 Men Farao tillslöt sitt hjärta också denna gång och släppte icke folket.

\chapter{9}

\par 1 Därefter sade HERREN till Mose: "Gå till Farao och tala till honom: Så säger HERREN, hebréernas Gud: Släpp mitt folk, så att de kunna hålla gudstjänst åt mig.
\par 2 Ty om du icke vill släppa dem, utan kvarhåller dem längre,
\par 3 se, då skall HERRENS hand med en mycket svår pest komma över din boskap på marken, över hästar, åsnor och kameler, över fäkreatur och får.
\par 4 Men HERREN skall därvid göra en åtskillnad mellan israeliternas boskap och egyptiernas, så att intet av de djur som tillhöra Israels barn skall dö."
\par 5 Och HERREN bestämde en tid och sade: "I morgon skall HERREN göra så i landet."
\par 6 Och dagen därefter gjorde HERREN så, och all egyptiernas boskap dog. Men av Israels barns boskap dog icke ett enda djur;
\par 7 när Farao sände och hörde efter, se, då hade icke så mycket som ett enda djur av Israels boskap dött. Men Faraos hjärta var tillslutet, och han släppte icke folket
\par 8 Därefter sade HERREN till Mose och Aron: "Tagen edra händer fulla med sot ur smältugnen, och må sedan Mose strö ut det, upp mot himmelen, inför Faraos ögon,
\par 9 så skall därav bliva ett damm över hela Egyptens land, och därav skola uppstå bulnader, som slå ut med blåsor, på människor och boskap i hela Egyptens land."
\par 10 Då togo de sot ur smältugnen och trädde inför Farao, och Mose strödde ut det, upp mot himmelen; och därav uppstodo bulnader, som slogo ut med blåsor, på människor och boskap.
\par 11 Och spåmännen kunde icke hålla stånd mot Mose för bulnadernas skull, ty bulnader uppstodo på spåmännen såväl som på alla andra egyptier.
\par 12 Men HERREN förstockade Faraos hjärta, så att han icke hörde på dem, såsom HERREN hade sagt till Mose.
\par 13 Därefter sade HERREN till Mose: "Träd i morgon bittida fram inför Farao och säg till honom: Så säger HERREN, hebréernas Gud: Släpp mitt folk, så att de kunna hålla gudstjänst åt mig.
\par 14 Annars skall jag nu sända alla mina hemsökelser över dig själv och över dina tjänare och ditt folk, på det att du må förnimma att ingen är såsom jag på hela jorden.
\par 15 Ty jag hade redan räckt ut min land för att slå dig och ditt folk med pest, så att du skulle bliva utrotad från jorden;
\par 16 men jag skonade dig, just därför att jag ville låta min kraft bliva uppenbarad för dig och mitt namn bliva förkunnat på hela jorden.
\par 17 Om du ytterligare lägger hinder i vägen för mitt folk och icke släpper dem,
\par 18 se, då skall jag i morgon vid denna tid låta ett mycket svårt hagel komma, sådant att dess like icke har varit i Egypten, allt ifrån den dag dess grund blev lagd ända till nu.
\par 19 Så sänd nu bort och låt bärga din boskap och allt vad du annars har ute på marken. Ty alla människor och all boskap som då finnas ute på marken och icke hava kommit under tak, de skola träffas av haglet och bliva dödade."
\par 20 Den som nu bland Faraos tjänare fruktade HERRENS ord, han lät sina tjänare och sin boskap söka skydd i husen;
\par 21 men den som icke aktade på HERRENS ord, han lät sina tjänare och sin boskap bliva kvar ute på marken.
\par 22 Och HERREN sade till Mose: "Räck din hand upp mot himmelen, så skall hagel falla över hela Egyptens land, över människor och boskap och över alla markens örter i Egyptens land."
\par 23 Då räckte Mose sin stav upp mot himmelen, och HERREN lät det dundra och hagla, och eld for ned mot jorden, så lät HERREN hagel komma över Egyptens land.
\par 24 Och det haglade, och bland hagelskurarna flammade eld; och haglet var så svårt, att dess like icke hade varit i hela Egyptens land från den tid det blev befolkat.
\par 25 Och i hela Egyptens land slog haglet ned allt som fanns på marken, både människor och djur; och haglet slog ned alla markens örter och slog sönder alla markens träd.
\par 26 Allenast i landet Gosen, där Israels barn voro, haglade det icke.
\par 27 Då sände Farao och lät kalla till sig Mose och Aron och sade till dem: "Jag har syndat denna gång. Det är HERREN som är rättfärdig; jag och mitt folk hava gjort orätt.
\par 28 Bedjen till HERREN, ty hans dunder och hagel har varat länge nog; så skall jag släppa eder, och I skolen icke behöva bliva kvar längre."
\par 29 Mose svarade honom: "När jag kommer ut ur staden, skall jag uträcka mina händer till HERREN; då skall dundret upphöra och intet hagel mer komma, på det att du må förnimma att landet är HERRENS.
\par 30 Dock vet jag väl att du och dina tjänare ännu icke frukten för HERREN Gud."
\par 31 Så slogos då linet och kornet ned, ty kornet hade gått i ax och linet stod i knopp;
\par 32 men vetet och spälten slogos icke ned, ty de äro sensäd.
\par 33 Och Mose gick ifrån Farao ut ur staden och uträckte sina händer till HERREN; och dundret och haglet upphörde, och regnet strömmade icke mer ned på jorden.
\par 34 Men när Farao såg att regnet och haglet och dundret hade upphört, framhärdade han i sin synd och tillslöt sitt hjärta, han själv såväl som hans tjänare.
\par 35 Så förblev Faraos hjärta förstockat, och han släppte icke Israels barn, såsom HERREN hade sagt genom Mose.

\chapter{10}

\par 1 Därefter sade HERREN till Mose: "Gå till Farao; ty jag har tillslutit hans och hans tjänares hjärtan, för att jag skulle göra dessa mina tecken mitt ibland dem,
\par 2 och för att du sedan skulle kunna förtälja för din son och din sonson vilka stora gärningar jag har utfört bland egyptierna, och vilka tecken jag har gjort bland dem, så att I förnimmen att jag är HERREN."
\par 3 Då gingo Mose och Aron till Farao och sade till honom: "Så säger HERREN, hebréernas Gud: Huru länge vill du vara motsträvig och icke ödmjuka dig inför mig? Släpp mitt folk, så att de kunna hålla gudstjänst åt mig.
\par 4 Ty om du icke vill släppa mitt folk, se, då skall jag i morgon låta gräshoppor komma över ditt land.
\par 5 Och de skola övertäcka marken så att man icke kan se marken; och de skola äta upp återstoden av den kvarleva som har blivit över åt eder efter haglet, och de skola aväta alla edra träd, som växa på marken.
\par 6 Och dina hus skola bliva uppfyllda av dem, så ock alla dina tjänares hus och alla egyptiers hus, så att dina fäder och dina faders fäder icke hava sett något sådant, från den dag de blevo till på jorden ända till denna dag." Och han vände sig om och gick ut ifrån Farao.
\par 7 Men Faraos tjänare sade till honom: "Huru länge skall denne vara oss till förfång? Släpp männen, så att de kunna hålla gudstjänst åt HERREN, sin Gud. Inser du icke ännu att Egypten bliver fördärvat?"
\par 8 Då hämtade man Mose och Aron tillbaka till Farao. Och han sade till dem: "I mån gå åstad och hålla gudstjänst åt HERREN, eder Gud. Men vilka äro nu de som skola gå?"
\par 9 Mose svarade: "Vi vilja gå både unga och gamla; vi vilja gå med söner och döttrar, med får och fäkreatur; ty en HERRENS högtid skola vi hålla."
\par 10 Då sade han till dem: "Må HERREN: vara med eder lika visst som jag släpper eder med edra kvinnor och barn! Där ser man att I haven ont i sinnet!
\par 11 Nej; I män mån gå åstad och hålla gudstjänst åt HERREN; det var ju detta som I begärden." Och man drev dem ut ifrån Farao.
\par 12 Och HERREN sade till Mose: "Räck ut din hand över Egyptens land, så att gräshoppor komma över Egyptens land och äta upp alla örter i landet, allt vad haglet har lämnat kvar."
\par 13 Då räckte Mose ut sin stav över Egyptens land, och HERREN lät en östanvind blåsa över landet hela den dagen och hela natten; och när det blev morgon, förde östanvinden gräshopporna fram med sig.
\par 14 Och gräshopporna kommo över hela Egyptens land och slogo i stor mängd ned över hela Egyptens område; en sådan myckenhet av gräshoppor hade aldrig tillförne kommit och skall icke heller hädanefter komma.
\par 15 De övertäckte hela marken, så att marken blev mörk; och de åto upp alla örter i landet och all frukt på träden, allt som haglet hade lämnat kvar; intet grönt blev kvar på träden eller på markens örter i hela Egyptens land.
\par 16 Då kallade Farao med hast Mose och Aron till sig och sade: "Jag har syndat mot HERREN, eder Gud, och mot eder.
\par 17 Men förlåt nu min synd denna enda gång; och bedjen till HERREN, eder Gud, att han avvänder allenast denna dödsplåga ifrån mig."
\par 18 Då gick han ut ifrån Farao och bad till HERREN.
\par 19 Och HERREN vände om vinden och lät en mycket stark västanvind komma; denna fattade i gräshopporna och kastade dem i Röda havet, så att icke en enda gräshoppa blev kvar inom Egyptens hela område.
\par 20 Men HERREN förstockade Faraos hjärta, så att han icke släppte Israels barn.
\par 21 Därefter sade HERREN till Mose: "Räck din hand upp mot himmelen, så skall över Egyptens land komma ett sådant mörker, att man kan taga på det."
\par 22 Då räckte Mose sin hand upp mot himmelen, och ett tjockt mörker kom över hela Egyptens land i tre dagar.
\par 23 Ingen kunde se den andre, och ingen kunde röra sig från sin plats i tre dagar. Men alla Israels barn hade ljust där de bodde.
\par 24 Då kallade Farao Mose till sig och sade: "Gån åstad och hållen gudstjänst åt HERREN; låten allenast edra får och fäkreatur bliva kvar. Också edra kvinnor och barn må gå med eder."
\par 25 Men Mose sade: "Du måste ock låta oss få slaktoffer och brännoffer att offra åt HERREN, vår Gud.
\par 26 Också vår boskap måste gå med oss, och icke en klöv får bliva kvar; ty därav måste vi taga det varmed vi skola hålla gudstjänst åt HERREN, vår Gud. Och förrän vi komma dit, veta vi själva icke vad vi böra offra åt HERREN."
\par 27 Men HERREN förstockade Faraos hjärta, så att han icke ville släppa dem.
\par 28 Och Farao sade till honom: "Gå bort ifrån mig, och tag dig till vara för att ännu en gång komma inför mitt ansikte; ty på den dag du kommer inför mitt ansikte skall du dö."
\par 29 Mose svarade: "Du har talat rätt; jag skall icke vidare komma inför ditt ansikte."

\chapter{11}

\par 1 Därefter sade HERREN till Mose: "Ännu en plåga skall jag låta komma över Farao och över Egypten; sedan skall han släppa eder härifrån; ja, han skall till och med driva eder ut härifrån, när han släpper eder.
\par 2 Så tala nu till folket, och säg att var och en av dem, man såväl som kvinna, skall av sin nästa begära klenoder av silver och guld."
\par 3 Och HERREN lät folket finna nåd för egyptiernas ögon. Ja, mannen Mose hade stort anseende i Egyptens land, både hos Faraos tjänare och hos folket.
\par 4 Och Mose sade: "Så säger HERREN: Vid midnattstid skall jag gå fram genom Egypten.
\par 5 Och då skall allt förstfött i Egyptens land dö, från den förstfödde hos Farao, som sitter på tronen, ända till den förstfödde hos tjänstekvinnan, som arbetar vid handkvarnen, så ock allt förstfött ibland boskapen.
\par 6 Och ett stort klagorop skall upphävas i hela Egyptens land, sådant att dess like aldrig har varit hört och aldrig mer skall höras.
\par 7 Men icke en hund skall gläfsa mot någon av Israels barn, varken mot människor eller mot boskap. Så skolen I förnimma att HERREN gör en åtskillnad mellan Egypten och Israel.
\par 8 Då skola alla dina tjänare här komma ned till mig och buga sig för mig och säga: 'Drag ut, du själv med allt folket som följer dig.' Och sedan skall jag draga ut." Därefter gick han bort ifrån Farao i vredesmod.
\par 9 Men HERREN sade till Mose: "Farao skall neka att höra eder, på det att jag må låta många under ske i Egyptens land."
\par 10 Och Mose och Aron gjorde alla dessa under inför Farao; men HERREN förstockade Faraos hjärta, så att han icke släppte Israels barn ut ur sitt land.

\chapter{12}

\par 1 Och HERREN talade till Mose och Aron i Egyptens land och sade:
\par 2 Denna månad skall hos eder vara den främsta månaden, den skall hos eder vara den första av årets månader.
\par 3 Talen till Israels hela menighet och sägen: På tionde dagen i denna månad skall var husfader taga sig ett lamm, så att vart hushåll får ett lamm.
\par 4 Men om hushållet är för litet till ett lamm, så skola husfadern och hans närmaste granne taga ett lamm tillsammans, efter personernas antal. För vart lamm skolen I beräkna ett visst antal, i mån av vad var och en äter.
\par 5 Ett felfritt årsgammalt lamm av hankön skolen I utvälja; av fåren eller av getterna skolen I taga det.
\par 6 Och I skolen förvara det intill fjortonde dagen i denna månad; då skall man - Israels hela församlade menighet - slakta det vid aftontiden.
\par 7 Och man skall taga av blodet och stryka på båda dörrposterna och på övre dörrträet i husen där man äter det.
\par 8 Och man skall äta köttet samma natt; det skall vara stekt på eld, och man skall äta det med osyrat bröd jämte bittra örter.
\par 9 I skolen icke äta något därav rått eller kokt i vatten, utan det skall vara stekt på eld, med huvud, fötter och innanmäte.
\par 10 Och I skolen icke lämna något därav kvar till morgonen; skulle något därav bliva kvar till morgonen, skolen I bränna upp det i eld.
\par 11 Och I skolen äta det så: I skolen vara omgjordade kring edra länder, hava edra skor på fötterna och edra stavar i händerna. Och I skolen äta det med hast. Detta är HERRENS Påsk.
\par 12 Ty jag skall på den natten gå fram genom Egyptens land och slå allt förstfött i Egyptens land, både människor och boskap; och över Egyptens alla gudar skall jag hålla dom; jag är HERREN.
\par 13 Och blodet skall vara ett tecken, eder till räddning, på de hus i vilka I ären; ty när jag ser blodet, skall jag gå förbi eder. Och ingen hemsökelse skall drabba eder med fördärv, när jag slår Egyptens land.
\par 14 Och I skolen hava denna dag till en åminnelsedag och fira den såsom en HERRENS högtid. Såsom en evärdlig stiftelse skolen I fira den, släkte efter släkte.
\par 15 I sju dagar skolen I äta osyrat bröd; redan på första dagen skolen I skaffa bort all surdeg ur edra hus. Ty var och en som äter något syrat, från den första dagen till den sjunde, han skall utrotas ur Israel.
\par 16 På den första dagen skolen I hålla en helig sammankomst; I skolen ock på den sjunde dagen hålla en helig sammankomst. På dem skall intet arbete göras; allenast det som var och en behöver till mat, det och intet annat må av eder tillredas.
\par 17 Och I skolen hålla det osyrade brödets högtid, eftersom jag på denna samma dag har fört edra härskaror ut ur Egyptens land. Därför skolen I, släkte efter släkte, hålla denna dag såsom en evärdlig stiftelse.
\par 18 I första månaden, på fjortonde dagen i månaden, om aftonen, skolen I äta osyrat bröd, och I skolen fortfara därmed ända till aftonen på tjuguförsta dagen i månaden.
\par 19 I sju dagar må ingen surdeg finnas i edra hus; ty var och en son äter något syrligt, han skall utrotas ur Israels menighet, evad han är främling eller inföding i landet.
\par 20 Intet syrligt skolen I äta; var I än ären bosatta skolen I äta osyrat bröd.
\par 21 Och Mose kallade till sig alla de äldste i Israel och sade till dem: "Begiven eder hem, och tagen eder ett lamm för vart hushåll och slakten påskalammet.
\par 22 Och tagen en knippa isop och doppen den i blodet som är i skålen, och bestryken det övre dörr träet och båda dörrposterna med blodet som är i skålen; och ingen av eder må gå ut genom sin hus dörr intill morgonen.
\par 23 Ty HERREN skall gå fram för att hemsöka Egypten; men när ha ser blodet på det övre dörrträet och på de två dörrposterna, skall HERREN gå förbi dörren och icke tillstädja Fördärvaren att komma i i edra hus och hemsöka eder.
\par 24 Detta skolen I hålla; det skall vara en stadga för dig och dina barn till evärdlig tid.
\par 25 Och när I kommen in i det land som HERREN skall giva åt eder, såsom han har lovat, skolen I hålla denna gudstjänst.
\par 26 När då edra barn fråga eder: 'Vad betyder denna eder gudstjänst?',
\par 27 skolen I svara: 'Det är ett påskoffer åt HERREN, därför att han gick förbi Israels barns hus i Egypten, när han hemsökte Egypten, men skonade våra hus.'" Då böjde folket sig ned och tillbad.
\par 28 Och Israels barn gingo åstad och gjorde så; de gjorde såsom HERREN hade bjudit Mose och Aron.
\par 29 Och vid midnattstiden slog HERREN allt förstfött i Egyptens land, från den förstfödde hos Farao, som satt på tronen, ända till den förstfödde hos fången, som satt i fängelset, så ock allt förstfött ibland boskapen.
\par 30 Då stod Farao upp om natten jämte alla sina tjänare och alla egyptier, och ett stort klagorop upphävdes i Egypten; ty intet hus fanns, där icke någon död låg.
\par 31 Och han kallade Mose och Aron till sig om natten och sade: "Stån upp och dragen ut från mitt folk, I själva och Israels barn; och gån åstad och hållen gudstjänst åt HERREN, såsom I haven begärt.
\par 32 Tagen ock edra får och edra fäkreatur, såsom I haven begärt, och gån åstad, och välsignen därvid mig."
\par 33 Och egyptierna trängde på folket för att med hast få dem ut ur landet, ty de tänkte: "Eljest dö vi allasammans."
\par 34 Och folket tog med sig sin deg, innan den ännu hade blivit syrad; de togo sina baktråg och lindade in dem i mantlarna och buro dem på sina axlar.
\par 35 Och Israels barn hade gjort såsom Mose sade: de hade av egyptierna begärt deras klenoder av silver och guld, så ock kläder.
\par 36 Och HERREN hade låtit folket finna nåd för egyptiernas ögon, så att de gåvo dem vad de begärde. Så togo de byte från egyptierna.
\par 37 Och Israels barn bröto upp och drogo från Rameses till Suckot, vid pass sex hundra tusen män till fots, förutom kvinnor och barn.
\par 38 En hop folk av allahanda slag drog ock åstad med dem, därtill får och fäkreatur, boskap i stor myckenhet.
\par 39 Och av degen som de hade fört med sig ur Egypten bakade de osyrade kakor, ty den hade icke blivit syrad; de hade ju drivits ut ur Egypten utan att få dröja; ej heller hade de kunnat tillreda någon reskost åt sig.
\par 40 Men den tid Israels barn hade bott i Egypten var fyra hundra trettio år.
\par 41 Just på den dag då de fyra hundra trettio åren voro förlidna drogo alla HERRENS härskaror ut ur Egyptens land.
\par 42 En HERRENS vakenatt var detta, när han skulle föra dem ut ur Egyptens land; denna samma natt är HERRENS, en högtidsvaka för alla Israels barn, släkte efter släkte.
\par 43 Och HERREN sade till Mose och Aron: "Detta är stadgan om påskalammet: Ingen utlänning skall äta därav;
\par 44 men en träl som är köpt för penningar må äta därav, sedan du ha omskurit honom.
\par 45 En inhysesman och en legodräng må icke äta därav.
\par 46 I ett och samma hus skall det ätas; du skall icke föra något av köttet ut ur huset, och intet ben skolen I sönderslå därpå.
\par 47 Israels hela menighet skall iakttaga detta.
\par 48 Och om någon främling bor hos dig och vill hålla HERRENS påskhögtid, så skall allt mankön hos honom omskäras, och sedan må han komma och hålla den; han skall då vara såsom en inföding i landet. Men ingen oomskuren må äta därav.
\par 49 En och samma lag skall gälla för infödingen och för främlingen som bor ibland eder."
\par 50 Och alla Israels barn gjorde så; de gjorde såsom HERREN hade bjudit Mose och Aron.
\par 51 Så förde då HERREN på denna samma dag Israels barn ut ur Egyptens land, efter deras härskaror.

\chapter{13}

\par 1 Och HERREN talade till Mose och sade:
\par 2 "Helga åt mig allt förstfött, allt hos Israels barn, som öppnar moderlivet, evad det är människor eller boskap; mig tillhör det.
\par 3 Och Mose sade till folket: "Kommen ihåg denna dag, på vilken I haven dragit ut ur Egypten, ur träldomshuset; ty med stark hand har HERREN fört eder ut därifrån. Fördenskull må intet syrat ätas.
\par 4 På denna dag i månaden Abib dragen I nu ut.
\par 5 Och när HERREN låter dig komma in i kananéernas, hetiternas, amoréernas, hivéernas och jebuséernas land, som han med ed har lovat dina fäder att giva dig, ett land som flyter av mjölk och honung, då skall du hålla denna gudstjänst i denna månad:
\par 6 I sju dagar skall du äta osyrat bröd, och på sjunde dagen skall hållas en HERRENS högtid.
\par 7 Under de sju dagarna skall man äta osyrat bröd; intet syrat skall man se hos dig, ej heller skall man se någon surdeg hos dig, i hela ditt land.
\par 8 Och du skall på den dagen berätta för din son och säga: 'Sådant gör jag av tacksamhet för vad HERREN gjorde med mig, när jag drog ut ur Egypten.'
\par 9 Och det skall vara för dig såsom ett tecken på din hand och såsom ett påminnelsemärke på din panna, för att HERRENS lag må vara i din mun; ty med stark hand har HERREN fört dig ut ur Egypten.
\par 10 Och denna stadga skall du hålla på bestämd tid, år efter år.
\par 11 Och när HERREN låter dig komma in i kananéernas land, såsom han med ed har lovat dig och dina fäder, och giver det åt dig,
\par 12 då skall du överlämna åt HERREN allt det som öppnar moderlivet. Allt som öppnar moderlivet av det som födes bland din boskap skall, om det är hankön, höra HERREN till.
\par 13 Men allt bland åsnor som öppnar moderlivet skall du lösa med ett får, och om du icke vill lösa det, skall du krossa nacken på det. Och allt förstfött av människa bland dina söner skall du lösa.
\par 14 Och när din son i framtiden frågar dig: 'Vad betyder detta?', skall du svara honom så: 'Med stark hand har HERREN fört oss ut ur Egypten, ur träldomshuset;
\par 15 ty då Farao i sin hårdnackenhet icke ville släppa oss, dräpte HERREN allt förstfött i Egyptens land, det förstfödda såväl ibland människor som ibland boskap. Därför offrar jag åt HERREN allt som öppnar moderlivet och är hankön, och allt förstfött bland mina söner löser jag.'
\par 16 Och det skall vara såsom ett tecken på din hand och såsom ett märke på din panna; ty med stark hand har HERREN fört oss ut ur Egypten."
\par 17 När Farao nu hade släppt folket, förde Gud dem icke på den väg som gick igenom filistéernas land, fastän denna var den genaste; ty Gud tänkte att folket, när det fick se krig hota, kunde ångra sig och vända tillbaka till Egypten;
\par 18 därför lät Gud folket taga en omväg genom öknen åt Röda havet till. Och Israels barn drogo väpnade upp ur Egyptens land.
\par 19 Och Mose tog med sig Josefs ben; ty denne hade tagit en ed av Israels barn och sagt: "När Gud ser till eder, fören då mina ben härifrån med eder."
\par 20 Så bröto de upp från Suckot och lägrade sig i Etam, där öknen begynte.
\par 21 Och HERREN gick framför dem, om dagen i en molnstod, för att leda dem på vägen, och om natten i en eldstod, för att lysa dem; så kunde de tåga både dag och natt.
\par 22 Molnstoden upphörde icke om dagen att gå framför folket, ej heller eldstoden om natten.

\chapter{14}

\par 1 Och HERREN talade till Mose och sade:
\par 2 "Säg till Israels barn att de skola vända om och lägra sig framför Pi-Hahirot, mellan Migdol och havet; mitt framför Baal-Sefon skolen I lägra eder vid havet.
\par 3 Men Farao skall tänka att Israels barn hava farit vilse i landet och blivit instängda i öknen.
\par 4 Och jag skall förstocka Faraos hjärta, så att han förföljer dem; och jag skall förhärliga mig på Farao och hela hans här, på det att egyptierna må förnimma att jag är HERREN." Och de gjorde så.
\par 5 Då man nu berättade för konungen i Egypten att folket hade flytt, förvandlades Faraos och hans tjänares hjärtan mot folket, och de sade: "Huru illa gjorde vi icke, när vi släppte Israel, så att de nu icke mer skola tjäna oss!"
\par 6 Och han lät spänna för sina vagnar och tog sitt folk med sig;
\par 7 han tog sex hundra utvalda vagnar, och alla vagnar som eljest funnos i Egypten, och kämpar på dem alla.
\par 8 Ty HERREN förstockade Faraos, den egyptiske konungens, hjärta, så att han förföljde Israels barn, när de nu drogo ut med upplyft hand.
\par 9 Och egyptierna, alla Faraos hästar, vagnar och ryttare och hela hans här, förföljde dem och hunno upp dem, där de voro lägrade vid havet, vid Pi-Hahirot, framför Baal-Sefon.
\par 10 När så Farao var helt nära, lyfte Israels barn upp sina ögon och fingo se att egyptierna kommo tågande efter dem. Då blevo Israels barn mycket förskräckta och ropade till HERREN.
\par 11 Och de sade till Mose: "Funnos då inga gravar i Egypten, eftersom du har fört oss hit till att dö i öknen? Huru illa gjorde du icke mot oss, när du förde oss ut ur Egypten!
\par 12 Var det icke det vi sade till dig i Egypten? Vi sade ju: 'Låt oss vara, så att vi få tjäna egyptierna.' Ty det vore oss bättre att tjäna egyptierna än att dö i öknen."
\par 13 Då svarade Mose folket: "Frukten icke; stån fasta, så skolen I se vilken frälsning HERREN i dag skall bereda eder; ty aldrig någonsin skolen I mer få se egyptierna så, som I sen dem i dag.
\par 14 HERREN skall strida för eder, och I skolen vara stilla därvid."
\par 15 Och HERREN sade till Mose: "Varför ropar du till mig? Säg till Israels barn att de draga vidare.
\par 16 Men lyft du upp din stav, och räck ut din hand över havet, och klyv det itu, så att Israels barn kunna gå mitt igenom havet på torr mark.
\par 17 Och se, jag skall förstocka egyptiernas hjärtan, så att de följa efter dem; och jag skall förhärliga mig på Farao och hela hans här, på hans vagnar och ryttare.
\par 18 Och egyptierna skola förnimma att jag är HERREN, när jag förhärligar mig på Farao, på hans vagnar och ryttare."
\par 19 Och Guds ängel, som hade gått framför Israels här, flyttade sig nu och gick bakom dem; molnstoden, som hade gått framför dem, flyttade sig och tog plats bakom dem.
\par 20 Den kom så emellan egyptiernas här och Israels här; och molnet var där med mörker, men tillika upplyste det natten. Så kunde den ena hären icke komma inpå den andra under hela natten.
\par 21 Och Mose räckte ut sin hand över havet; då drev HERREN undan havet genom en stark östanvind som blåste hela natten, och han gjorde så havet till torrt land; och vattnet klövs itu.
\par 22 Och Israels barn gingo mitt igenom havet på torr mark, under det att vattnet stod såsom en mur till höger och till vänster om dem.
\par 23 Och egyptierna, alla Faraos hästar, vagnar och ryttare, förföljde dem och kommo efter dem ut till mitten av havet.
\par 24 Men när morgonväkten var inne, blickade HERREN på egyptiernas här ur eldstoden och molnskyn och sände förvirring i egyptiernas här;
\par 25 och han lät hjulen falla ifrån deras vagnar, så att det blev dem svårt att komma framåt. Då sade egyptierna: "Låt oss fly för Israel, ty HERREN strider för dem mot egyptierna."
\par 26 Men HERREN sade till Mose: "Räck ut din hand över havet, så att vattnet vänder tillbaka och kommer över egyptierna, över deras vagnar och ryttare."
\par 27 Då räckte Mose ut sin hand över havet, och mot morgonen vände havet tillbaka till sin vanliga plats, och egyptierna som flydde möttes därav; och HERREN kringströdde egyptierna mitt i havet.
\par 28 Och vattnet som vände tillbaka övertäckte vagnarna och ryttarna, hela Faraos här, som hade kommit efter dem ut i havet; icke en enda av dem kom undan.
\par 29 Men Israels barn gingo på torr mark mitt igenom havet, och vattnet stod såsom en mur till höger och till vänster om dem.
\par 30 Så frälste HERREN på den dagen Israel från egyptiernas hand, och Israel såg egyptierna ligga döda på havsstranden.
\par 31 Och när Israel såg huru HERREN hade bevisat sin stora makt på egyptierna, fruktade folket HERREN; och de trodde på HERREN och på hans tjänare Mose.

\chapter{15}

\par 1 Då sjöngo Mose och Israels barn denna lovsång till HERRENS ära; de sade: "Jag vill sjunga till HERRENS ära, ty högt är han upphöjd. Häst och man störtade han i havet.
\par 2 HERREN är min starkhet och min lovsång, Och han blev mig till frälsning. Han är min Gud, jag vill ära honom, min faders Gud, jag vill upphöja honom.
\par 3 HERREN är en stridsman, 'HERREN' är hans namn.
\par 4 Faraos vagnar och härsmakt kastade han i havet, hans utvalda kämpar dränktes i Röda havet.
\par 5 De övertäcktes av vattenmassor, sjönko i djupet såsom stenar.
\par 6 Din högra hand, HERRE, du härlige och starke, din högra hand, HERRE, krossar fienden.
\par 7 Genom din stora höghet slår du ned dina motståndare; du släpper lös din förgrymmelse, den förtär dem såsom strå.
\par 8 För en fnysning av din näsa uppdämdes vattnen, böljorna reste sig och samlades hög, vattenmassorna stelnade i havets djup.
\par 9 Fienden sade: 'Jag vill förfölja dem, hinna upp dem, jag vill utskifta byte, släcka min hämnd på dem; jag vill draga ut mitt svärd, min hand skall förgöra dem.'
\par 10 Du andades på dem, då övertäckte dem havet; de sjönko såsom bly i de väldiga vattnen.
\par 11 Vilken bland gudar liknar dig, HERRE? Vem är dig lik, du härlige och helige, du fruktansvärde och högtlovade, du som gör under?
\par 12 Du räckte ut din högra hand, då uppslukades de av jorden.
\par 13 Men du ledde med din nåd det folk du hade förlossat, du förde dem med din makt till din heliga boning.
\par 14 Folken hörde det och måste då darra, av ångest grepos Filisteens inbyggare.
\par 15 Då förskräcktes Edoms furstar, Moabs hövdingar grepos av bävan, alla Kanaans inbyggare försmälte av ångest.
\par 16 Ja, över dem faller förskräckelse och fruktan; för din arms väldighet stå de såsom förstenade, medan ditt folk tågar fram, o HERRE medan det tågar fram, det folk du har förvärvat.
\par 17 Du för dem in och planterar dem på din arvedels berg, på den plats, o HERRE, som du har gjort till din boning, i den helgedom, Herre, som dina händer hava berett.
\par 18 HERREN är konung alltid och evinnerligen!"
\par 19 Ty när Faraos hästar med hans vagnar och ryttare hade kommit ned i havet, lät HERREN havets vatten vända tillbaka och komma över dem, sedan Israels barn på torr mark hade gått mitt igenom havet.
\par 20 Och profetissan Mirjam, Arons syster, tog en puka i sin hand, och alla kvinnorna följde efter henne med pukor och dans.
\par 21 Och Mirjam sjöng för dem: "Sjungen till HERRENS ära, ty högt är han upphöjd. Häst och man störtade han i havet."
\par 22 Därefter lät Mose israeliterna bryta upp från Röda havet, och de drogo ut i öknen Sur; och tre dagar vandrade de i öknen utan att finna vatten.
\par 23 Så kommo de till Mara; men de kunde icke dricka vattnet i Mara, ty det var bittert. Därav fick stället namnet Mara.
\par 24 Då knorrade folket emot Mose och sade: "Vad skola vi dricka?"
\par 25 Men han ropade till HERREN; och HERREN visade honom ett visst slags trä, som han kastade i vattnet, och så blev vattnet sött. Där förelade han folket lag och rätt, och där satte han det på prov.
\par 26 Han sade: "Om du hör HERRENS, din Guds, röst och gör vad rätt är i hans ögon och lyssnar till hans bud och håller alla hans stadgar, så skall jag icke lägga på dig någon av de sjukdomar som jag lade på egyptierna, ty jag är HERREN, din läkare."
\par 27 Sedan kommo de till Elim; där funnos tolv vattenkällor och sjuttio palmträd. Och de lägrade sig där vid vattnet.

\chapter{16}

\par 1 Därefter bröt Israels barns hela menighet upp från Elim och kom till öknen Sin, mellan Elim och Sinai, på femtonde dagen i andra månaden efter sitt uttåg ur Egyptens land.
\par 2 Och Israels barns hela menighet knorrade emot Mose och Aron i öknen;
\par 3 Israels barn sade till dem: "Ack att vi hade fått dö för HERRENS hand i Egyptens land, där vi sutto vid köttgrytorna och hade mat nog att äta! Men I haven fört oss hitut i öknen för att låta hela denna hop dö av hunger."
\par 4 Då sade HERREN till Mose: "Se, jag vill låta bröd från himmelen regna åt eder. Och folket skall gå ut och samla för var dag så mycket som behöves. Så skall jag sätta dem på prov, för att se om de vilja vandra efter min lag eller icke.
\par 5 Och när de på den sjätte dagen tillreda vad de hava fört hem, skall det vara dubbelt mot vad de eljest för var dag samla in."
\par 6 Och Mose och Aron sade till alla Israels barn: "I afton skolen I förnimma att det är HERREN som har fört eder ut ur Egyptens land,
\par 7 och i morgon skolen I se HERRENS härlighet, HERREN har nämligen hört huru I knorren mot honom. Ty vad äro vi, att I knorren mot oss?"
\par 8 Och Mose sade ytterligare: "Detta skall ske därigenom att HERREN i afton giver eder kött att äta, och i morgon bröd att mätta eder med då nu HERREN har hört huru I knorren mot honom. Ty vad äro vi? Det är icke mot oss I knorren, utan mot HERREN."
\par 9 Och Mose sade till Aron: "Säg till Israels barns hela menighet Träden fram inför HERREN, ty han har hört huru I knorren."
\par 10 När sedan Aron talade till Israels barns hela menighet, vände de sig mot öknen, och se, då visade sig HERRENS härlighet i molnskyn.
\par 11 Och HERREN talade till Mose och sade:
\par 12 "Jag har hört huru Israels barn knorra. Tala till dem och säg: Vid aftontiden skolen I få kött att äta och i morgon skolen I få bröd att mätta eder med; så skolen I förnimma att jag är HERREN, eder Gud."
\par 13 Och om aftonen kommo vaktlar och övertäckte lägret, och följande morgon låg dagg fallen runt omkring lägret.
\par 14 Och när daggen som hade fallit gick bort, se, då låg över öknen på jorden något fint, såsom fjäll, något fint, likt rimfrost.
\par 15 När Israels barn sågo detta, frågade de varandra: "Vad är det?" Ty de visste icke vad det var. Men Mose sade till dem: "Detta är brödet som HERREN har givit eder till föda.
\par 16 Och så har HERREN bjudit: Samlen därav, var och en så mycket han behöver till mat; en gomer på var person skolen I taga, efter antalet av edert husfolk, var och en åt så många som han har i sitt tält."
\par 17 Och Israels barn gjorde så, och den ene samlade mer, den andre mindre.
\par 18 Men när de mätte upp det med gomer-mått, hade den som hade samlat mycket intet till överlopps, och de som hade samlat litet, honom fattades intet; var och en hade så mycket samlat, som han behövde till mat.
\par 19 Och Mose sade till dem: "Ingen må behålla något kvar härav till i morgon.
\par 20 Men de lydde icke Mose, utan somliga behöllo något därav kvar till följande morgon. Då växte maskar däri, och det blev illaluktande. Och Mose blev förtörnad på dem.
\par 21 Så samlade de därav var morgon, var och en så mycket han behövde till mat. Men när solhettan kom smälte det bort.
\par 22 På den sjätte dagen hade de samlat dubbelt så mycket av brödet, två gomer för var och en. Och menighetens hövdingar kommo alla och omtalade detta för Mose.
\par 23 Då sade han till dem: "Detta är efter HERRENS ord; i morgon är sabbatsvila, en HERRENS heliga sabbat. Baken nu vad I viljen baka, och koken vad I viljen koka, men allt som är till överlopps skolen I ställa i förvar hos eder till i morgon."
\par 24 Och de ställde det i förvar till följande morgon, såsom Mose hade bjudit; och nu blev det icke illaluktande, ej heller kom mask däri.
\par 25 Och Mose sade: "Äten det i dag, ty i dag är HERRENS sabbat; i dag skolen I intet finna på marken.
\par 26 I sex dagar skolen I samla därav, men på sjunde dagen är sabbat; då skall intet vara att finna."
\par 27 Likväl gingo några av folket på den sjunde dagen ut för att samla, men de funno intet.
\par 28 Då sade HERREN till Mose: "Huru länge viljen I vara motsträviga och icke hålla mina bud och lagar?
\par 29 Se, HERREN har givit eder sabbaten; därför giver han eder på den sjätte dagen bröd för två dagar. Så stannen då hemma, var och en hos sig; ingen må gå hemifrån på den sjunde dagen."
\par 30 Alltså höll folket sabbat på den sjunde dagen.
\par 31 Och Israels barn kallade det manna. Och det liknade korianderfrö, det var vitt, och det smakade såsom semla med honung.
\par 32 Och Mose sade: "Så har HERREN bjudit: En gomer härav skall förvaras åt edra efterkommande, för att de må se det bröd som jag gav eder att äta i öknen, när jag förde eder ut ur Egyptens land.
\par 33 Och Mose sade till Aron: "Tag ett kärl och lägg däri en gomer manna, och ställ det inför HERREN till att förvaras åt edra efterkommande."
\par 34 Då gjorde man såsom HERREN hade bjudit Mose, och Aron ställde det framför vittnesbördet till att förvaras.
\par 35 Och Israels barn åto manna i fyrtio år, till dess de kommo till bebott land; de åto manna, till dess de kommo till gränsen av Kanaans land. -
\par 36 En gomer är tiondedelen av en efa.

\chapter{17}

\par 1 Därefter bröt Israels barns hela menighet upp från öknen Sin och tågade från lägerplats till lägerplats, efter HERRENS befallning. Och de lägrade sig i Refidim; där hade folket intet vatten att dricka.
\par 2 Då begynte folket tvista med Mose och sade: "Given oss vatten att dricka." Mose svarade dem: "Varför tvisten I med mig? Varför fresten I HERREN?"
\par 3 Men eftersom folket där törstade efter vatten, knorrade de ytterligare emot Mose och sade: "Varför har du fört oss upp ur Egypten, så att vi, våra barn och vår boskap nu måste dö av törst?"
\par 4 Då ropade Mose till HERREN och sade: "Vad skall jag göra med detta folk? Det fattas icke mycket i att de stena mig."
\par 5 HERREN svarade Mose: "Gå framför folket, och tag med dig några av de äldste i Israel. Och tag i din hand staven med vilken du slog Nilfloden, och begiv dig åstad.
\par 6 Se, jag vill stå där framför dig på Horebs klippa, och du skall slå på klippan, och vatten skall då komma ut ur den, så att folket får dricka." Och Mose gjorde så inför de äldste i Israel.
\par 7 Och han gav stället namnet Massa och Meriba, därför att Israels barn hade tvistat och frestat HERREN och sagt: "Är HERREN ibland oss eller icke?"
\par 8 Därefter kom Amalek och gav sig i strid med Israel i Refidim.
\par 9 Då sade Mose till Josua: "Välj ut manskap åt oss, och drag så åstad till strid mot Amalek. I morgon skall jag ställa mig överst på höjden, med Guds stav i min hand."
\par 10 Och Josua gjorde såsom Mose hade tillsagt honom, och gav sig i strid med Amalek. Men Mose, Aron och Hur stego upp överst på höjden.
\par 11 Och så länge Mose höll upp sin hand, rådde Israel, men när han lät sin hand sjunka, rådde Amalek.
\par 12 Och när Moses händer blevo tunga, togo de en sten och lade under honom, och på den satte han sig; sedan stödde Aron och Hur hans händer, en på vardera sidan. Så höllos hans händer stadiga, till dess solen gick ned.
\par 13 Och Josua slog Amalek och dess folk med svärdsegg.
\par 14 Och HERREN sade till Mose: "Teckna upp detta till en åminnelse i en bok, och inprägla det hos Josua, ty jag skall så i grund utplåna minnet av Amalek, att det icke mer skall finnas under himmelen."
\par 15 Och Mose byggde ett altare och gav det namnet HERREN mitt baner.
\par 16 Och han sade: "Ja, jag lyfter min hand upp mot HERRENS tron och betygar att HERREN skall strida mot Amalek från släkte till släkte."

\chapter{18}

\par 1 Och Jetro, prästen i Midjan, Moses svärfader, fick höra allt vad Gud hade gjort med Mose och med sitt folk Israel, huru HERREN hade fört Israel ut ur Egypten.
\par 2 Då tog Jetro, Moses svärfader, med sig Sippora, Moses hustru, som denne förut hade sänt hem,
\par 3 så ock hennes två söner - av dessa hade den ene fått namnet Gersom, "ty", sade Mose, "jag är en främling i ett land som icke är mitt",
\par 4 och den andre namnet Elieser, "ty", sade han, "min faders Gud blev mig till hjälp och räddade mig ifrån Faraos svärd". -
\par 5 Då så Jetro, Moses svärfader, kom med Moses söner och hans hustru till honom i öknen, där han hade slagit upp sitt läger vid Guds berg,
\par 6 lät han säga till Mose: "Jag, din svärfader Jetro, kommer till dig med din hustru och hennes båda söner."
\par 7 Då gick Mose sin svärfader till mötes och bugade sig för honom och kysste honom. Och när de hade hälsat varandra, gingo de in i tältet.
\par 8 Och Mose förtäljde för sin svärfader allt vad HERREN hade gjort med Farao och egyptierna, för Israels skull, och alla de vedermödor som de hade haft att utstå på vägen, och huru HERREN hade räddat dem.
\par 9 Och Jetro fröjdade sig över allt det goda som HERREN hade gjort mot Israel, i det han hade räddat dem ur egyptiernas hand.
\par 10 Och Jetro sade: "Lovad vare HERREN, som har räddat eder ur egyptiernas hand och ur Faraos hand, HERREN, som har räddat folket undan egyptiernas hand!
\par 11 Nu vet jag att HERREN är större än alla andra gudar, ty så bevisade han sig, när man handlade övermodigt mot detta folk."
\par 12 Och Jetro, Moses svärfader, frambar ett brännoffer och några slaktoffer åt Gud; och Aron och alla de äldste i Israel kommo och höllo måltid med Moses svärfader inför Gud.
\par 13 Dagen därefter satte Mose sig för att döma folket, och folket stod omkring Mose från morgonen ända till aftonen.
\par 14 Då nu Moses svärfader såg allt vad han hade att beställa med folket, sade han: "Vad är det allt du har att bestyra med folket? Varför sitter du här till doms ensam under det att allt folket måste stå omkring dig från morgonen ända till aftonen?"
\par 15 Mose svarade sin svärfader: "Folket kommer till mig för att fråga Gud.
\par 16 De komma till mig, när de hava någon rättssak, och jag dömer då mellan dem; och jag kungör då för dem Guds stadgar och lagar."
\par 17 Då sade Moses svärfader till honom: "Du går icke till väga på det rätta sättet.
\par 18 Både du själv och folket omkring dig måsten ju bliva uttröttade; ett sådant förfaringssätt är dig för svårt, du kan icke ensam bestyra detta.
\par 19 Så lyssna nu till mina ord; jag vill giva dig ett råd, och Gud skall vara med dig. Du må vara folkets målsman inför Gud och framlägga deras ärenden inför Gud.
\par 20 Och du må upplysa dem om stadgar och lagar och kungöra dem den väg de skola vandra och vad de skola göra.
\par 21 Men sök ut åt dig bland allt folket dugande män som frukta Gud, pålitliga män som hata orätt vinning, och sätt dessa till föreståndare för dem, somliga över tusen, andra över hundra, andra över femtio och andra över tio.
\par 22 Dessa må alltid döma folket. Kommer något viktigare ärende före, må de hänskjuta det till dig, men alla ringare ärenden må de själva avdöma. Så skall du göra din börda lättare, därigenom att de bära den med dig.
\par 23 Om du vill så göra och Gud så bjuder dig, skall du kunna hålla ut; och allt folket här skall då kunna gå hem i frid."
\par 24 Och Mose lyssnade till sin svärfaders ord och gjorde allt vad denne hade sagt.
\par 25 Mose utvalde dugande män ur hela Israel och gjorde dem till huvudmän för folket, till föreståndare somliga över tusen, andra över hundra, andra över femtio och andra över tio.
\par 26 Dessa skulle alltid döma folket. Alla svårare ärenden skulle de hänskjuta till Mose, men alla ringare ärenden skulle de själva avdöma.
\par 27 Därefter lät Mose sin svärfader fara hem, och denne begav sig till sitt land igen.

\chapter{19}

\par 1 På den dag då den tredje månaden ingick efter Israels barns uttåg ur Egyptens land kommo de in i Sinais öken.
\par 2 Ty de bröto upp från Refidim och kommo så till Sinais öken och lägrade sig i öknen; Israel lägrade sig där mitt emot berget.
\par 3 Och Mose steg upp till Gud; då ropade HERREN till honom uppifrån berget och sade: "Så skall du säga till Jakobs hus, så skall du förkunna för Israels barn:
\par 4 'I haven själva sett vad jag har gjort med egyptierna, och huru jag har burit eder på örnvingar och fört eder till mig.
\par 5 Om I nu hören min röst och hållen mitt förbund, så skolen I vara min egendom framför alla andra folk, ty hela jorden är min;
\par 6 Och I skolen vara mig ett rike av präster och ett heligt folk.' Detta är vad du skall tala till Israels barn.
\par 7 När Mose kom tillbaka, sammankallade han de äldste i folket och förelade dem allt detta som HERREN hade bjudit honom.
\par 8 Då svarade allt folket med en mun och sade: "Allt vad HERREN har talat vilja vi göra." Och Mose gick tillbaka till HERREN med folkets svar.
\par 9 Och HERREN sade till Mose: "Se, jag skall komma till dig i en tjock molnsky, för att folket skall höra, när jag talar med dig, och så tro på dig evärdligen." Och Mose framförde folkets svar till HERREN.
\par 10 Då sade HERREN till Mose: "Gå till folket, och helga dem i dag och i morgon, och I åt dem två sina kläder.
\par 11 Och må de hålla sig redo till i övermorgon; ty i övermorgon skall HERREN stiga ned på Sinai berg inför allt folkets ögon.
\par 12 Och du skall märka ut en gräns för folket runt omkring och säga: 'Tagen eder till vara för att stiga upp på berget eller komma vid dess fot. Var och en som kommer vid berget skall straffas med döden;
\par 13 men ingen hand må komma vid honom, utan han skall stenas eller skjutas ihjäl. Evad det är djur eller människa, skall en sådan mista livet.' När jubelhornet ljuder med utdragen ton, då må de stiga upp på berget."
\par 14 Och Mose steg ned från berget till folket och helgade folket, och de tvådde sina kläder.
\par 15 Och han sade till folket: "Hållen eder redo till i övermorgon; ingen komme vid en kvinna."
\par 16 På tredje dagen, när det hade blivit morgon, begynte det dundra och blixtra, och en tung molnsky kom över berget, och ett mycket starkt basunljud hördes; och allt folket i lägret bävade.
\par 17 Men Mose förde folket ut ur lägret, Gud till mötes; och de ställde sig nedanför berget.
\par 18 Och hela Sinai berg höljdes i rök, vid det att HERREN kom ned därpå i eld; och en rök steg upp därifrån, lik röken från en smältugn, och hela berget bävade storligen.
\par 19 Och basunljudet blev allt starkare och starkare. Mose talade, och Gud svarade honom med hög röst.
\par 20 Och HERREN steg ned på Sinai berg, på toppen av berget, och HERREN kallade Mose upp till bergets topp; då steg Mose ditupp.
\par 21 Och HERREN sade till Mose: "Stig ned och varna folket, så att de icke tränga sig fram för att se HERREN, ty då skola många av dem falla.
\par 22 Jämväl prästerna, som få nalkas HERREN, skola helga sig, för att HERREN icke må låta dem drabbas av fördärv."
\par 23 Men Mose svarade HERREN: "Folket kan icke stiga upp på Sinai berg, ty du har själv varnat oss och sagt att jag skulle märka ut en gräns omkring berget och helga det."
\par 24 Då sade HERREN till honom: "Gå ditned, och kom sedan åter upp och hav Aron med dig. Men prästerna och folket må icke tränga sig fram för att stiga upp till HERREN på det att han icke må låta dem drabbas av fördärv."
\par 25 Och Mose steg ned till folket och sade dem detta.

\chapter{20}

\par 1 Och Gud talade alla dessa ord och sade:
\par 2 Jag är HERREN, din Gud, som har fört dig ut ur Egyptens land, ur träldomshuset.
\par 3 Du skall inga andra gudar hava jämte mig.
\par 4 Du skall icke göra dig något beläte eller någon bild, vare sig av det som är uppe i himmelen, eller av det som är nere på jorden, eller av det som är i vattnet under jorden.
\par 5 Du skall icke tillbedja sådana, ej heller tjäna dem; ty jag, HERREN, din Gud, är en nitälskande Gud, som hemsöker fädernas missgärning på barn och efterkommande i tredje och fjärde led, när man hatar mig,
\par 6 men som gör nåd med tusenden, när man älskar mig och håller mina bud.
\par 7 Du skall icke missbruka HERRENS, din Guds, namn, ty HERREN skall icke låta den bliva ostraffad, som missbrukar hans namn.
\par 8 Tänk på sabbatsdagen, så att du helgar den.
\par 9 Sex dagar skall du arbeta och förrätta alla dina sysslor;
\par 10 men den sjunde dagen är HERRENS, din Guds, sabbat; då skall du ingen syssla förrätta, ej heller din son eller din dotter, ej heller din tjänare eller din tjänarinna eller din dragare, ej heller främlingen som är hos dig inom dina portar.
\par 11 Ty på sex dagar gjorde HERREN himmelen och jorden och havet och allt vad i dem är, men han vilade på sjunde dagen; därför har HERREN välsignat sabbatsdagen och helgat den.
\par 12 Hedra din fader och din moder, för att du må länge leva i det land som HERREN, din Gud, vill giva dig.
\par 13 Du skall icke dräpa.
\par 14 Du skall icke begå äktenskapsbrott.
\par 15 Du skall icke stjäla.
\par 16 Du skall icke bära falskt vittnesbörd mot din nästa.
\par 17 Du skall icke hava begärelse till din nästas hus. Du skall icke hava begärelse till din nästas hustru, ej heller till hans tjänare eller hans tjänarinna, ej heller till hans oxe eller hans åsna, ej heller till något annat som tillhör din nästa.
\par 18 Och allt folket förnam dundret och eldslågorna och basunljudet och röken från berget; och när folket förnam detta, bävade de och höllo sig på avstånd.
\par 19 Och de sade till Mose: "Tala du till oss, så vilja vi höra, men låt icke Gud tala till oss, på det att vi icke må dö."
\par 20 Men Mose sade till folket: "Frukten icke, ty Gud har kommit för att sätta eder på prov, och för att I skolen hava hans fruktan för ögonen, så att I icke synden."
\par 21 Alltså höll folket sig på avstånd, under det att Mose gick närmare till töcknet i vilket Gud var.
\par 22 Och HERREN sade till Mose: Så skall du säga till Israels barn: I haven själva förnummit att jag har talat till eder från himmelen.
\par 23 I skolen icke göra eder gudar jämte mig; gudar av silver eller guld skolen I icke göra åt eder.
\par 24 Ett altare av jord skall du göra åt mig och offra därpå dina brännoffer och tackoffer, din småboskap och dina fäkreatur. Överallt på den plats där jag stiftar en åminnelse åt mitt namn skall jag komma till dig och välsigna dig.
\par 25 Men om du vill göra åt mig ett altare av stenar, så må du icke bygga det av huggen sten; ty om du kommer vid stenen med din mejsel, så oskärar du den.
\par 26 Icke heller må du stiga upp till mitt altare på trappor, på det att icke din blygd må blottas därinvid.

\chapter{21}

\par 1 Dessa äro de rätter som du skall förelägga dem:
\par 2 Om du köper en hebreisk träl, skall han tjäna i sex år, men på det sjunde skall han givas fri, utan lösen.
\par 3 Har han kommit allena, så skall han givas fri allena; var han gift, så skall hans hustru givas fri med honom.
\par 4 Har hans herre givit honom en hustru, och har denna fött honom söner eller döttrar, så skola hustrun och hennes barn tillhöra hennes herre, och allenast mannen skall givas fri.
\par 5 Men om trälen säger: "Jag har min herre, min hustru och mina barn så kära, att jag icke vill givas fri",
\par 6 då skall hans herre föra honom fram för Gud och ställa honom vid dörren eller dörrposten, och hans herre skall genomborra hans öra med en syl; därefter vare han hans träl evärdligen.
\par 7 Om någon säljer sin dotter till trälinna, så skall hon icke givas fri såsom trälarna givas fria.
\par 8 Misshagar hon sin herre, sedan denne förut har ingått förbindelse med henne, så låte han henne köpas fri. Till främmande folk have han icke makt att sälja henne, när han så har handlat trolöst mot henne.
\par 9 Men om han låter sin son ingå förbindelse med henne, så förunne han henne döttrars rätt.
\par 10 Tager han sig ännu en hustru, så göre han icke någon minskning i den förras kost, beklädnad eller äktenskapsrätt.
\par 11 Om han icke låter henne njuta sin rätt i dessa tre stycken, så skall hon givas fri, utan lösen och betalning.
\par 12 Den som slår någon, så att han dör, han skall straffas med döden.
\par 13 Men om han icke traktade efter den andres liv, utan Gud lät denne oförvarandes träffas av hans hand, så skall jag anvisa dig en ort dit han kan fly.
\par 14 Men om någon begår det dådet att han dräper sin nästa med list, så skall du gripa honom, vore han ock invid mitt altare, och han måste dö.
\par 15 Den som slår sin fader eller sin moder, han skall straffas med döden.
\par 16 Den som stjäl en människa, vare sig han sedan säljer den stulne, eller denne finnes kvar hos honom, han skall straffas med döden.
\par 17 Den som uttalar förbannelser över sin fader eller sin moder, han skall straffas med döden.
\par 18 Om män tvista med varandra, och den ene slår den andre med en sten eller med knuten hand, så att denne väl icke dör, men bliver sängliggande,
\par 19 dock att han sedan kommer sig och kan gå ute, stödd vid sin stav, så skall den som slog honom vara fri ifrån straff; allenast ersätte han honom för den tid han har förlorat och besörje sjukvård åt honom.
\par 20 Om någon slår sin träl eller sin trälinna med en käpp, så att den slagne dör under hans hand, så skall han straffas därför.
\par 21 Men om den slagne lever en eller två dagar, skall han icke straffas, ty det var hans egna penningar.
\par 22 Om män träta med varandra, och någon av dem stöter till en havande kvinna, så att hon föder fram sitt foster, men eljest ingen olycka sker, så böte han vad kvinnans man ålägger honom och betale efter skiljedomares prövning.
\par 23 Men om olycka sker, skall liv givas för liv,
\par 24 öga för öga, tand för tand, hand för hand, fot för fot,
\par 25 brännskada för brännskada, sår för sår, blånad för blånad.
\par 26 Om någon slår sin träl eller sin trälinna i ögat och fördärvar det, så släppe han den skadade fri, till ersättning för ögat.
\par 27 Sammalunda, om någon slår ut en tand på sin träl eller sin trälinna, så släppe han den skadade fri, till ersättning för tanden.
\par 28 Om en oxe stångar någon till döds, man eller kvinna, så skall oxen stenas, och köttet må icke ätas; men oxens ägare vara fri ifrån straff.
\par 29 Men om oxen förut har haft vanan att stångas, och hans ägare har blivit varnad, men denne ändå icke tager vara på honom, och oxen så dödar någon, man eller kvinna, då skall oxen stenas, och hans ägare skall ock dödas.
\par 30 Men skulle lösepenning bliva denne ålagd, så give han till lösen för sitt liv så mycket som ålägges honom.
\par 31 Är det en gosse eller en flicka som har blivit stångad av oxen, så skall med denne förfaras efter samma lag.
\par 32 Men om oxen stångar en träl eller en trälinna, så skall ägaren giva åt den stångades herre trettio siklar silver, och oxen skall stenas.
\par 33 Om någon öppnar en brunn, eller om någon gräver en ny brunn och icke täcker över den, och sedan en oxe eller en åsna faller däri,
\par 34 så skall brunnens ägare giva ersättning i penningar åt djurets ägare, men den döda kroppen skall vara hans.
\par 35 Om någons oxe stångar en annans oxe, så att denne dör, så skola de sälja den levande oxen och dela betalningen för honom och därjämte dela den döda kroppen.
\par 36 Var det däremot känt att oxen förut hade vanan att stångas, men tog hans ägare ändå icke vara på honom, så skall han ersätta oxe med oxe, men den döda kroppen skall vara hans.

\chapter{22}

\par 1 Om någon stjäl en oxe eller ett får och slaktar eller säljer djuret, så skall han giva fem oxar i ersättning för oxen, och fyra får för fåret.
\par 2 Ertappas tjuven vid inbrottet och bliver slagen till döds, så vilar ingen blodskuld på dråparen.
\par 3 Men hade solen gått upp, när de skedde, då är det blodskuld. Tjuven skall giva full ersättning; äger han intet, så skall han säljas, till gäldande av vad han har stulit.
\par 4 Om det stulna djuret, det må vara oxe eller åsna eller får, påträffas levande i hans våld, skall han giva dubbel ersättning.
\par 5 Om någon låter avbeta en åker eller vingård, eller släpper sin boskap lös, så att denna betar på en annans åker, då skall han ersätta skadan med det bästa från sin åker och med det bästa från sin vingård.
\par 6 Om eld kommer lös och fattar i törnhäckar, och därvid sädesskylar bliva uppbrända eller oskuren säd eller annat på åkern, så skall den som har vållat branden giva full ersättning.
\par 7 Om någon giver åt en annan penningar eller gods att förvara, och detta bliver stulet ur hans hus, så skall tjuven, om han ertappas, giva dubbel ersättning.
\par 8 Ertappas icke tjuven, då skall man föra husets ägare fram för Gud, på det att det må utrönas om han icke har förgripit sig på sin nästas tillhörighet.
\par 9 Om fråga uppstår angående orättrådigt tillgrepp - det må gälla oxe eller åsna eller får eller kläder eller något annat som har förlorats - och någon påstår att en orättrådighet verkligen har ägt rum, så skall båda parternas sak komma inför Gud. Den som Gud dömer skyldig, han skall ersätta den andre dubbelt.
\par 10 Om någon giver åt en annan i förvar en åsna eller en oxe eller ett får, eller vilket annat husdjur det vara må, och detta dör eller bliver skadat eller bortrövat, utan att någon ser det,
\par 11 Så skall det dem emellan komma till en ed vid HERREN, för att det må utrönas om den ene icke har förgripit sig på den andres tillhörighet; denna ed skall ägaren antaga och den andre behöver icke giva någon ersättning.
\par 12 Men om det har blivit bortstulet från honom, då skall han ersätta ägaren därför.
\par 13 Har det blivit ihjälrivet, skall han föra fram det ihjälrivna djuret såsom bevis; han behöver då icke giva ersättning därför.
\par 14 Om någon lånar ett djur av en annan, och detta bliver skadat eller dör, och dess ägare därvid icke är tillstädes, så skall han giva full ersättning.
\par 15 Är dess ägare tillstädes, då behöver han icke giva ersättning. Var djuret lejt, då är legan ersättning.
\par 16 Om någon förför en jungfru som icke är trolovad och lägrar henne, så skall han giva brudgåva för henne och taga henne till hustru.
\par 17 Vägrar hennes fader att giva henne åt honom, då skall han gälda en så stor penningsumma som man plägar giva i brudgåva för en jungfru.
\par 18 En trollkvinna skall du icke låta leva.
\par 19 Var och en som beblandar sig med något djur skall straffas med döden.
\par 20 Den som offrar åt andra gudar än åt HERREN allena, han skall givas till spillo.
\par 21 En främling skall du icke förorätta eller förtrycka; I haven ju själva varit främlingar i Egyptens land.
\par 22 Änkor och faderlösa skolen I icke behandla illa.
\par 23 Behandlar du dem illa, så skall jag förvisso höra deras rop, när de ropa till mig;
\par 24 och min vrede skall upptändas, och jag skall dräpa eder med svärd; så att edra egna hustrur bliva änkor och edra barn faderlösa.
\par 25 Lånar du penningar åt någon fattig hos dig bland mitt folk, så skall du icke handla mot honom såsom en ockrare; I skolen icke pålägga honom någon ränta.
\par 26 Har du av din nästa tagit hans mantel i pant, så skall du giva den tillbaka åt honom, innan solen går ned;
\par 27 den är ju det enda täcke han har, och med den skyler han sin kropp. Vad skall han eljest hava på sig, när han ligger och sover? Om han måste ropa till mig, så skall jag höra, ty jag är barmhärtig.
\par 28 Gud skall du icke häda, och över en hövding i ditt folk skall du icke uttala förbannelser.
\par 29 Av det som fyller din lada och av det som flyter ifrån din press skall du utan dröjsmål frambära din gåva. Den förstfödde bland dina söner skall du giva åt mig.
\par 30 På samma sätt skall du göra med dina fäkreatur och din småboskap. I sju dagar skola de stanna hos sina mödrar; på åttonde dagen skall du giva dem åt mig.
\par 31 Och I skolen vara mig ett heligt: folk; kött av ett djur som har blivit ihjälrivet på marken skolen I icke äta, åt hundarna skolen I kasta det.

\chapter{23}

\par 1 Du skall icke utsprida falskt rykte; åt den som har en orätt sak skall du icke giva ditt bistånd genom att bliva ett orättfärdigt vittne.
\par 2 Du skall icke följa med hopen i vad ont är, eller vittna så i någon sak, att du böjer dig efter hopen och vränger rätten.
\par 3 Du skall icke vara partisk för den ringe i någon hans sak.
\par 4 Om du träffar på din fiendes oxe eller åsna som har kommit vilse, så skall du föra djuret tillbaka till honom.
\par 5 Om du ser din oväns åsna ligga dignad under sin börda, så skall du ingalunda lämna mannen ohulpen, utan hjälpa honom att lösa av bördan.
\par 6 Du skall icke i någon sak vränga rätten för den fattige som du har hos dig.
\par 7 Du skall hålla dig fjärran ifrån orätt sak; du skall icke dräpa den som är oskyldig och har rätt, ty jag skall icke giva rätt åt någon som är skyldig.
\par 8 Du skall icke taga mutor, ty mutor förblinda de seende och förvrida de rättfärdigas sak.
\par 9 En främling skall du icke förtrycka; I veten ju huru främlingen känner det, eftersom I själva haven varit främlingar i Egyptens land.
\par 10 I sex år skall du beså din jord och inbärga dess gröda;
\par 11 men under det sjunde året skall du låta den vila och ligga orörd, för att de fattiga bland ditt folk må äta därav; vad de lämna kvar, det må ätas av markens djur. Så skall du ock göra med din vingård och med din olivplantering.
\par 12 Sex dagar skall du göra ditt arbete, men på sjunde dagen skall du hålla vilodag, för att din oxe och din åsna må hava ro, och din tjänstekvinnas son och främlingen må njuta vila.
\par 13 I alla de stycken om vilka jag har talat till eder skolen I taga eder till vara. Och andra gudars namn skolen I icke nämna; de skola icke höras i din mun.
\par 14 Tre gånger om året skall du hålla högtid åt mig.
\par 15 Det osyrade brödets högtid skall du hålla: i sju dagar skall du äta osyrat bröd, såsom jag har bjudit dig, på den bestämda tiden i månaden Abib, eftersom du då drog ut ur Egypten; men med tomma händer skall ingen träda fram inför mitt ansikte.
\par 16 Du skall ock hålla skördehögtiden, när du skördar förstlingen av ditt arbete, av det du har sått på marken. Bärgningshögtiden skall du ock hålla, vid årets utgång, när du inbärgar frukten av ditt arbete från marken.
\par 17 Tre gånger om året skall allt ditt mankön träda fram inför HERRENS, din Herres, ansikte.
\par 18 Du skall icke offra blodet av mitt slaktoffer jämte något som är syrat. Och det feta av mitt högtidsoffer skall icke lämnas kvar över natten till morgonen.
\par 19 Det första av din marks förstlingsfrukter skall du föra till HERRENS, din Guds, hus. Du skall icke koka en killing i dess moders mjölk.
\par 20 Se, jag skall sända en ängel framför dig, som skall bevara dig på vägen och föra dig till den plats som jag har utsett.
\par 21 Tag dig till vara inför honom och hör hans röst, var icke gensträvig mot honom, han skall icke hava fördrag med edra överträdelser, ty mitt namn är i honom.
\par 22 Men om du hör hans röst och gör allt vad jag säger, så skall jag bliva en fiende till dina fiender och en ovän till dina ovänner.
\par 23 Ty min ängel skall gå framför dig och skall föra dig till amoréernas, hetiternas, perisséernas, kananéernas, hivéernas och jebuséernas land, och jag skall utrota dem.
\par 24 Du må icke tillbedja deras gudar eller tjäna dem eller göra såsom man där gör, utan du skall slå dem ned i grund och bryta sönder deras stoder.
\par 25 Men HERREN, eder Gud, skolen I tjäna, så skall han för dig välsigna både mat och dryck; sjukdom skall jag då ock avvända från dig.
\par 26 I ditt land skall då icke finnas någon kvinna som föder i otid eller är ofruktsam. Dina dagars mått skall jag göra fullt.
\par 27 Förskräckelse för mig skall jag sända framför dig och vålla förvirring bland alla de folk som du kommer till, och jag skall driva alla dina fiender på flykten för dig.
\par 28 Jag skall sända getingar framför dig, och de skola förjaga hivéerna, kananéerna och hetiterna undan för dig.
\par 29 Dock skall jag icke på ett och samma år förjaga dem för dig, på det att icke landet så må bliva en ödemark och vilddjuren föröka sig till din skada;
\par 30 utan småningom skall jag förjaga dem för dig, till dess du har förökat dig, så att du kan taga landet till din arvedel.
\par 31 Och jag skall låta ditt lands gränser gå från Röda havet till filistéernas hav, och från öknen till floden; ty jag skall giva landets inbyggare i eder hand, och du skall förjaga dem, så att de fly för dig.
\par 32 Du må icke sluta förbund med dem eller deras gudar.
\par 33 De skola icke få bo kvar i ditt land, på det att de icke må förleda dig till synd mot mig; ty du kunde ju komma att tjäna deras gudar, och detta skulle bliva dig till en snara.

\chapter{24}

\par 1 Och han sade till Mose: "Stig upp till HERREN, du själv jämte Aron, Nadab och Abihu och sjuttio av de äldste i Israel; och I skolen tillbedja på avstånd.
\par 2 Mose allena må träda fram till HERREN, de andra må icke träda fram; ej heller må folket stiga ditupp med honom."
\par 3 Och Mose kom till folket och förkunnade för det alla HERRENS ord och alla hans rätter. Då svarade allt folket med en mun och sade: "Efter alla de ord HERREN har talat vilja vi göra."
\par 4 Därefter upptecknade Mose alla HERRENS ord. Och följande morgon stod han bittida upp och byggde ett altare nedanför berget. Och han reste där tolv stoder, efter Israels tolv stammar.
\par 5 Och han sände israeliternas unga män åstad till att offra brännoffer, så ock slaktoffer av tjurar till tackoffer åt HERREN.
\par 6 Och Mose tog hälften av blodet och slog det i skålarna, och den andra hälften av blodet stänkte han på altaret.
\par 7 Och han tog förbundsboken och, föreläste den för folket. Och de sade: "Allt vad HERREN har sagt vilja vi göra och lyda.
\par 8 Då tog Mose blodet och stänkte därav på folket och sade: "Se, detta är förbundets blod, det förbunds som HERREN har slutit med eder, i enlighet med alla dessa ord."
\par 9 Och Mose och Aron, Nadab och Abihu och sjuttio av de äldste i Israel stego ditupp.
\par 10 Och de fingo se Israels Gud; och under hans fötter var likasom ett inlagt golv av safirer, likt själva himmelen i klarhet.
\par 11 Men han lät icke sin hand drabba Israels barns ypperste, utan sedan de hade skådat Gud, åto de och drucko.
\par 12 Och HERREN sade till Mose: "Stig upp till mig på berget och bliv kvar där, så skall jag giva dig stentavlorna och lagen och budorden som jag har skrivit, till undervisning för dessa."
\par 13 Då begav sig Mose åstad med sin tjänare Josua; och Mose steg upp på Guds berg.
\par 14 Men till de äldste sade han: "Vänten här på oss, till dess att vi komma tillbaka till eder. Se, Aron och Hur äro hos eder; den som har något att andraga, han må vända sig till dem."
\par 15 Så steg Mose upp på berget, och molnskyn övertäckte berget.
\par 16 Och HERRENS härlighet vilade på Sinai berg, och molnskyn övertäckte det i sex dagar; men den sjunde dagen kallade han på Mose ur skyn.
\par 17 Och HERRENS härlighet tedde sig inför Israels barns ögon såsom en förtärande eld, på toppen av berget.
\par 18 Och Mose gick mitt in i skyn och steg upp på berget. Sedan blev Mose kvar på berget i fyrtio dagar och fyrtio nätter.

\chapter{25}

\par 1 Och HERREN talade till Mose och sade:
\par 2 Säg till Israels barn att de upptaga en gärd åt mig; av var och en som har ett därtill villigt hjärta skolen I upptaga denna gärd åt mig.
\par 3 Och detta är vad I skolen upptaga av dem såsom gärd: guld, silver och koppar
\par 4 mörkblått, purpurrött, rosenrött och vitt garn och gethår,
\par 5 rödfärgade vädurskinn, tahasskinn, akacieträ,
\par 6 olja till ljusstaken, kryddor till smörjelseoljan och till den välluktande rökelsen,
\par 7 äntligen onyxstenar och infattningsstenar, till att användas för efoden och för bröstskölden.
\par 8 Och de skola göra åt mig en helgedom, för att jag må bo mitt ibland dem.
\par 9 Tabernaklet och alla dess tillbehör skolen I göra alldeles efter de mönsterbilder som jag visar dig.
\par 10 De skola göra en ark av akacieträ, två och en halv aln lång, en och en halv aln bred och en och en halv aln hög.
\par 11 Och du skall överdraga den med rent guld, innan och utan skall du överdraga den; och du skall på den göra en rand av guld runt omkring.
\par 12 Och du skall till den gjuta fyra ringar av guld och sätta dem över de fyra fötterna, två ringar på ena sidan och två ringar på andra sidan.
\par 13 Och du skall göra stänger av akacieträ och överdraga dem med guld.
\par 14 Och stängerna skall du skjuta in i ringarna, på sidorna av arken, så att man med dem kan bära arken.
\par 15 Stängerna skola sitta kvar i ringarna på arken; de få icke dragas ut ur dem.
\par 16 Och i arken skall du lägga vittnesbördet, som jag skall giva dig.
\par 17 Och du skall göra en nådastol av rent guld, två och en halv aln lång och en och en halv aln bred.
\par 18 Och du skall göra två keruber av guld; i drivet arbete skall du göra dem och sätta dem vid de båda ändarna av nådastolen.
\par 19 Du skall göra en kerub till att sätta vid ena ändan, och en kerub till att sätta vid andra ändan. I ett stycke med nådastolen skolen I göra keruberna vid dess båda ändar.
\par 20 Och keruberna skola breda ut sina vingar och hålla dem uppåt, så att de övertäcka nådastolen med sina vingar, under det att de hava sina ansikten vända mot varandra; ned mot nådastolen skola keruberna vända sina ansikten.
\par 21 Och du skall sätta nådastolen ovanpå arken, och i arken skall du lägga vittnesbördet, som jag skall giva dig.
\par 22 Och där skall jag uppenbara mig för dig; från nådastolen, från platsen mellan de två keruberna, som stå på vittnesbördets ark, skall jag tala med dig om alla bud som jag genom dig vill giva Israels barn.
\par 23 Du skall ock göra ett bord av akacieträ, två alnar långt, en aln brett och en och en halv aln högt.
\par 24 Och du skall överdraga det med rent guld; och du skall göra en rand av guld därpå runt omkring
\par 25 Och runt omkring det skall du göra en list av en hands bredd, och runt omkring listen skall du göra en rand av guld.
\par 26 Och du skall till bordet göra fyra ringar av guld och sätta ringarna i de fyra hörnen vid de fyra fötterna.
\par 27 Invid listen skola ringarna sitta, för att stänger må skjutas in i dem, så att man kan bära bordet.
\par 28 Och du skall göra stängerna av akacieträ och överdraga dem med guld, och med dem skall bordet bäras.
\par 29 Du skall ock göra därtill fat och skålar, kannor och bägare, med vilka man skall utgjuta drickoffer; av rent guld skall du göra dem.
\par 30 Och du skall beständigt hava skådebröd liggande på bordet inför mitt ansikte.
\par 31 Du skall ock göra en ljusstake av rent guld. I drivet arbete skall ljusstaken göras, med sin fotställning och sitt mittelrör; kalkarna därpå, kulor och blommor skola vara i ett stycke med den.
\par 32 Och sex armar skola utgå från ljusstakens sidor, tre armar från ena sidan och tre armar från andra sidan.
\par 33 På den ena armen skola vara tre kalkar, liknande mandelblommor, vardera bestående av en kula och en blomma, och på den andra armen sammalunda tre kalkar, liknande mandelblommor, vardera bestående av en kula och en blomma så skall det vara på de sex arma som utgå från ljusstaken.
\par 34 Men på själva ljusstaken skola vara fyra kalkar, liknande mandelblommor, med sina kulor och blommor.
\par 35 En kula skall sättas under del första armparet som utgår från ljusstaken, i ett stycke med den, och en kula under det andra armparet som utgår från ljusstaken, i ett stycke med den, och en kula under det tredje armparet som utgår från ljusstaken, i ett stycke med den: alltså under de sex armar som utgå från ljusstaken.
\par 36 Deras kulor och armar skola vara: i ett stycke med den, alltsammans ett enda stycke i drivet arbete av rent guld.
\par 37 Och du skall till den göra sju lampor, och lamporna skall man sätta upp så, att den kastar sitt sken över platsen därframför.
\par 38 Och lamptänger och brickor till den skall du göra av rent guld.
\par 39 Av en talent rent guld skall man göra den med alla dessa tillbehör.
\par 40 Och se till, att du gör detta efter de mönsterbilder som hava blivit dig visade på berget.

\chapter{26}

\par 1 Tabernaklet skall du göra av tio tygvåder; av tvinnat vitt garn och av mörkblått, purpurrött och rosenrött garn skall du göra dem, med keruber på, i konstvävnad.
\par 2 Var våd skall vara tjuguåtta alnar lång och fyra alnar bred; alla våderna skola hava samma mått.
\par 3 Fem av våderna skola fogas tillhopa med varandra; likaså skola de fem övriga våderna fogas tillhopa med varandra.
\par 4 Och du skall sätta öglor av mörkblått garn i kanten på den ena våden, ytterst på det hopfogade stycket; så skall du ock göra i kanten på den våd som sitter ytterst i det andra hopfogade stycket.
\par 5 Femtio öglor skall du sätta på den ena våden, och femtio öglor skall du sätta ytterst på motsvarande våd i det andra hopfogade stycket, så att öglorna svara emot varandra.
\par 6 Och du skall göra femtio häktor av guld och foga tillhopa våderna med varandra medelst häktorna, så att tabernaklet utgör ett helt.
\par 7 Du skall ock göra tygvåder av gethår till ett täckelse över tabernaklet; elva sådana våder skall du göra.
\par 8 Var vad skall vara trettio alnar lång och fyra alnar bred; de elva våderna skola hava samma mått.
\par 9 Fem av våderna skall du foga tillhopa till ett särskilt stycke, och likaledes de sex övriga våderna till ett särskilt stycke, och den sjätte våden skall du lägga dubbel på framsidan av tältet.
\par 10 Och du skall satta femtio öglor i kanten på den ena våden, den som sitter ytterst i det ena hopfogade stycket, och femtio öglor i kanten på motsvarande våd i det andra hopfogade stycket.
\par 11 Och du skall göra femtio häktor av koppar och haka in häktorna i öglorna och foga täckelset tillhopa, så att det utgör ett helt.
\par 12 Men vad överskottet av täckelsets våder angår, det som räcker över, så skall den halva våd som räcker över hänga ned på baksidan av tabernaklet.
\par 13 Och den aln på vardera sidan, som på längden av täckelsets våder räcker över, skall hänga ned på båda sidorna av tabernaklet för att övertäcka det.
\par 14 Vidare skall du göra ett överdrag av rödfärgade vädurskinn till täckelset, och ytterligare ett överdrag av tahasskinn att lägga ovanpå detta.
\par 15 Bräderna till tabernaklet skall du göra av akacieträ, och de skola ställas upprätt.
\par 16 Tio alnar långt och en och en halv aln brett skall vart bräde vara.
\par 17 Vart bräde skall hava två tappar, förbundna sinsemellan med en list; så skall du göra på alla bräderna till tabernaklet.
\par 18 Och av tabernaklets bräder skall du sätta tjugu på södra sidan, söderut.
\par 19 Och du skall göra fyrtio fotstycken av silver att sätta under de tjugu bräderna, två fotstycken under vart bräde för dess två tappar.
\par 20 Likaledes skall du på tabernaklets andra sida, den norra sidan, sätta tjugu bräder,
\par 21 med deras fyrtio fotstycken av silver, två fotstycken under vart bräde.
\par 22 Men på baksidan av tabernaklet, västerut, skall du sätta sex bräder.
\par 23 Och två bräder skall du sätta på tabernaklets hörn, på baksidan;
\par 24 och vartdera av dessa skall vara sammanfogat av två nedtill, och likaledes sammanhängande upptill, till den första ringen. Så skall det vara med dem båda. Dessa skola sättas i de båda hörnen.
\par 25 Således bliver det åtta bräder med tillhörande fotstycken av silver, sexton fotstycken, nämligen två fotstycken under vart bräde.
\par 26 Och du skall göra tvärstänger av akacieträ, fem till de bräder som äro på tabernaklets ena sida
\par 27 och fem tvärstänger till de bräder som äro på tabernaklets andra sida, och fem tvärstänger till de bräder som äro på tabernaklets baksida, västerut.
\par 28 Och den mellersta tvärstången, den som sitter mitt på bräderna, skall gå tvärs över, från den ena ändan till den andra.
\par 29 Och bräderna skall du överdraga med guld, och ringarna på dem, i vilka tvärstängerna skola skjutas in, skall du göra av guld, och tvärstängerna skall du överdraga med guld.
\par 30 Och du skall sätta upp tabernaklet, sådant det skall vara, såsom det har blivit dig visat på berget.
\par 31 Du skall ock göra en förlåt av mörkblått, purpurrött, rosenrött och tvinnat vitt garn; den skall göras i konstvävnad, med keruber på.
\par 32 Och du skall hänga upp den på fyra stolpar av akacieträ, som skola vara överdragna med guld och hava bakar av guld och stå på fyra fotstycken av silver.
\par 33 Och du skall hänga upp förlåten under häktorna, och föra dit vittnesbördets ark och ställa den innanför förlåten; och så skall förlåten för eder vara en skiljevägg mellan det heliga och det allraheligaste.
\par 34 Och du skall sätta nådastolen på vittnesbördets ark inne i det allraheligaste.
\par 35 Men bordet skall du ställa utanför förlåten, och ljusstaken mitt emot bordet, på tabernaklets södra sida; bordet skall du alltså ställa på norra sidan.
\par 36 Och du skall göra ett förhänge för ingången till tältet, i brokig vävnad av mörkblått, purpurrött, rosenrött och tvinnat vitt garn.
\par 37 Och du skall till förhänget göra fem stolpar av akacieträ och överdraga dem med guld, och hakarna på dem skola vara av guld, och du skall till dem gjuta fem fotstycken av koppar.

\chapter{27}

\par 1 Du skall ock göra ett altare av akacieträ, fem alnar långt och fem alnar brett - så att altaret bildar en liksidig fyrkant - och tre alnar högt.
\par 2 Och du skall göra hörn därtill, Som skola sitta i dess fyra hörn; i ett stycke därmed skola hörnen vara. Och du skall överdraga det med koppar.
\par 3 Och kärl till att föra bort askan skall du göra därtill, så ock skovlar, skålar, gafflar och fyrfat. Alla dess tillbehör skall du göra av koppar.
\par 4 Och du skall göra ett galler därtill, ett nätverk av koppar, och på nätet skall du sätta fyra ringar av koppar i dess fyra hörn.
\par 5 Och du skall sätta det under avsatsen på altaret, nedtill, så att nätet räcker upp till mitten av altaret.
\par 6 Och du skall göra stänger till altaret, stänger av akacieträ, och överdraga dem med koppar.
\par 7 Och stängerna skola skjutas in i ringarna, så att stängerna sitta på altarets båda sidor, när man bär det.
\par 8 Ihåligt skall du göra det, av plankor. Såsom det har blivit dig visat på berget, så skall det göras.
\par 9 Du skall ock göra en förgård till tabernaklet. För den södra sidan, söderut, skola omhängen till förgården göras av tvinnat vitt garn, hundra alnar långa - detta för den ena sidan;
\par 10 Och stolparna till dem skola vara tjugu och dessas fotstycken tjugu, av koppar, men stolparnas hakar och kransar skola vara av silver.
\par 11 Likaledes skola för norra långsidan omhängen göras, hundra alnar långa; och stolparna till dem skola vara tjugu och dessas fotstycken tjugu, av koppar, men stolparnas hakar och kransar skola vara av silver.
\par 12 Och förgårdens västra kortsida skall hava omhängen som äro femtio alnar långa; stolparna till dem skola vara tio och dessas fotstycken tio.
\par 13 Och förgårdens bredd på fram sidan, österut, skall vara femtio alnar
\par 14 Och omhängena skola vara femton alnar långa på ena sidan därav, med tre stolpar på tre fotstycken;
\par 15 likaledes skola omhängena på andra sidan vara femton alnar långa med tre stolpar på tre fotstycken.
\par 16 Och till förgårdens port skall göras ett förhänge, tjugu alnar långt i brokig vävnad av mörkblått, purpurrött, rosenrött och tvinnat vitt garn, med fyra stolpar på fyra fotstycken.
\par 17 Alla stolparna runt omkring för gården skola vara försedda med kransar av silver och hava hakar av silver; men deras fotstycken skola vara av koppar.
\par 18 Förgården skall vara hundra alnar lång och femtio alnar bred utefter hela längden; omhägnaden skall vara fem alnar hög, av tvinnat vitt garn; och fotstyckena skola vara av koppar.
\par 19 Alla tabernaklets tillbehör för allt arbete därvid, så ock alla dess pluggar och alla förgårdens pluggar skola vara av koppar.
\par 20 Och du skall bjuda Israels barn att bära till dig ren olja, av stötta oliver, till ljusstaken, så att lamporna dagligen kunna sättas upp.
\par 21 I uppenbarelsetältet, utanför den förlåt som hänger framför vittnesbördet, skola Aron och hans söner sköta den, från aftonen till morgonen, inför HERRENS ansikte. Detta skall vara en evärdlig stadga från släkte till släkte, en gärd av Israels barn.

\chapter{28}

\par 1 Och du skall låta din broder Aron och hans söner med honom, träd fram till dig ur Israels barns krets för att de må bliva plåster åt mig Aron själv och hans söner: Nadab och Abihu, Eleasar och Itamar.
\par 2 Och du skall göra åt din broder Aron heliga kläder, till ära och prydnad.
\par 3 Och du skall tillsäga alla edra konstförfarna män, som jag har uppfyllt med vishetens ande, att de skola göra kläder åt Aron, för att har må helgas till att bliva präst åt mig.
\par 4 Och dessa äro de kläder som de skola göra: bröstsköld, efod, kåpa, rutig livklädnad, huvudbindel och bälte. De skola göra heliga kläder åt din broder Aron och hans söner, för att han må bliva präst åt mig.
\par 5 Och härtill skola de taga av guldet och av det mörkblåa, det purpurröda, det rosenröda och det vita garnet.
\par 6 Efoden skola de göra av guld och av mörkblått purpurrött, rosenrött och tvinnat vitt garn, i konstvävnad.
\par 7 Den skall vid sina båda ändar hava två axelstycken, som skola fästas ihop, så att den hållen hopfäst.
\par 8 Och skärpet, som skall sitta på efoden och sammanhålla den, skall vara av samma slags vävnad och i ett stycke med den: av guld och av mörkblått, purpurrött, rosenrött och tvinnat vitt garn.
\par 9 Och du skall taga två onyxstenar och på dem inrista Israels söners namn,
\par 10 sex av namnen på den ena stenen och de sex övrigas namn på den andra stenen, efter ättföljd.
\par 11 Med stensnidarkonst, såsom man graverar signetringar, skall du inrista Israels söners namn på de två stenarna. Med nätverk av guld skall du omgiva dessa.
\par 12 Och du skall satta de båda stenarna på efodens axelstycken, för att stenarna må bringa Israels barn i åminnelse; Aron skall bära deras namn inför HERRENS ansikte på sina båda axlar, för att bringa dem i åminnelse.
\par 13 Och du skall göra flätverk av guld,
\par 14 så ock två kedjor av rent guld; i virat arbete skall du göra dessa, såsom man gör snodder. Och du skall fästa de snodda kedjorna vid flätverken.
\par 15 En domssköld skall du göra i konstvävnad; du skall göra den i samma slags vävnad som efoden: av guld och av mörkblått, purpurrött, rosenrött och tvinnat vitt garn skall du göra den.
\par 16 Den skall vara liksidigt fyrkantig och hava form av en väska, ett kvarter lång och ett kvarter bred.
\par 17 Och du skall besätta den med infattade stenar, ordnade på fyra rader: i första raden en karneol, en topas och en smaragd;
\par 18 i andra raden en karbunkel, en safir och en kalcedon;
\par 19 i tredje raden en hyacint, en agat och en ametist;
\par 20 i fjärde raden en krysolit, en onyx och en jaspis. Omgivna med flätverk av guld skola de sitta i sin infattning.
\par 21 Stenarna skola vara tolv, efter Israels söners namn, en för vart namn; var sten skall bära namnet på en av de tolv stammarna, inristat på samma sätt som man graverar signetringar.
\par 22 Och du skall till bröstskölden göra kedjor i virat arbete, såsom man gör snodder, av rent guld.
\par 23 Vidare skall du till bröstskölden göra två ringar av guld och sätta dessa båda ringar i två av bröstsköldens hörn.
\par 24 Och du skall fästa de båda guldsnodderna vid de båda ringarna, i bröstsköldens hörn.
\par 25 Och de två snoddernas båda andra ändar skall du fästa vid de två flätverken och så fästa dem vid efodens axelstycken på dess framsida.
\par 26 Och du skall göra två andra ringar av guld och sätta dem i bröstsköldens båda andra hörn, vid den kant därpå, som är vänd inåt mot efoden.
\par 27 Och ytterligare skall du göra två ringar av guld och fästa dem vid efodens båda axelstycken, nedtill på dess framsida, där den fästes ihop, ovanför efodens skärp.
\par 28 Och man skall knyta fast bröstskölden med ett mörkblått snöre, Som går från dess ringar in i efodens ringar, så att den sitter ovanför efodens skärp, på det att bröstskölden icke må lossna från efoden.
\par 29 Aron skall så bära Israels söners namn i domsskölden på sitt hjärta, när han går in helgedomen, för att bringa dem i åminnelse inför HERRENS ansikte beständigt.
\par 30 Och du skall lägga urim och tummim in i domsskölden, så att de ligga på Arons hjärta, när han ingår inför HERRENS ansikte; och Aron skall så bära Israels barns dom på sitt hjärta inför HERRENS ansikte beständigt.
\par 31 Efodkåpan skall du göra helt och hållet av mörkblått tyg;
\par 32 och mitt på den skall vara en öppning för huvudet, och denna öppning skall omgivas med en vävd kant, likasom öppningen på en pansarskjorta, för att den icke slitas sönder.
\par 33 Och på dess nedre fåll skall du sätta granatäpplen, gjorda av mörkblått, purpurrött och rosenrött garn, runt omkring fållen, och bjällror av guld mellan dessa runt omkring:
\par 34 en bjällra av guld och så ett granatäpple, sedan en bjällra av guld och så åter ett granatäpple, runt omkring fållen på kåpan.
\par 35 Och denna skall Aron hava på sig, när han gör tjänst, så att det höres, när han går in i helgedomen inför HERRENS ansikte, och när han går ut - detta på det att han icke må dö.
\par 36 Du skall ock göra en plåt av rent guld, och på den skall du rista, såsom man graverar signetringar: "Helgad åt HERREN."
\par 37 Och du skall fästa den vid ett mörkblått snöre, och den skall sitta på huvudbindeln; på framsidan av huvudbindeln skall den sitta.
\par 38 Den skall sitta på Arons panna, och Aron skall bära den missgärning som vidlåder de heliga gåvor Israels barn bära fram, när de giva några heliga gåvor; den skall sitta på hans panna beständigt, för att de må bliva välbehagliga inför HERRENS ansikte.
\par 39 Du skall ock väva en rutig livklädnad av vitt garn, och du skall göra en huvudbindel av vitt garn; och ett bälte skall du göra i brokig vävnad.
\par 40 Också åt Arons söner skall du göra livklädnader, och du skall göra bälten åt dem; och huvor skall du göra åt dem, till ära och prydnad. Och detta skall du kläda på din broder Aron och hans söner jämte honom;
\par 41 och du skall smörja dem och företaga handfyllning med dem och helga dem till att bliva präster åt mig.
\par 42 Och du skall göra åt dem benkläder av linne, som skyla deras blygd; dessa skola räcka från länderna ned på låren.
\par 43 Och Aron och hans söner skola hava dem på sig, när de gå in i uppenbarelsetältet eller träda fram till altaret för att göra tjänst i helgedomen - detta på det att de icke må komma att bära på missgärning och så träffas av döden. Detta skall vara en evärdlig stadga för honom och hans avkomlingar efter honom.

\chapter{29}

\par 1 Och detta är vad du skall göra med dem för att helga dem till att bliva präster åt mig: Tag en ungtjur och två vädurar, felfria djur,
\par 2 och osyrat bröd och osyrade kakor, begjutna med olja, och osyrade tunnkakor, smorda med olja; av fint vetemjöl skall du baka dem.
\par 3 Och du skall lägga dem i en och samma korg och bära fram dem i korgen såsom offergåva, när du för fram tjuren och de två vädurarna.
\par 4 Därefter skall du föra Aron och hans söner fram till uppenbarelsetältets ingång och två dem med vatten.
\par 5 Och du skall taga kläderna och sätta på Aron livklädnaden och efodkåpan och själva efoden och bröstskölden; och du skall fästa ihop alltsammans på honom med efodens skärp.
\par 6 Och du skall sätta huvudbindeln på hans huvud och fästa det heliga diademet på huvudbindeln.
\par 7 Och du skall taga smörjelseoljan och gjuta på hans huvud och smörja honom.
\par 8 Och du skall föra fram hans söner och sätta livklädnader på dem.
\par 9 Och du skall omgjorda dem, Aron och hans söner, med bälten och binda huvor på dem. Och de skola hava prästadömet såsom en evärdlig rätt. Så skall du företaga handfyllning med Aron och hans söner.
\par 10 Och du skall föra tjuren fram inför uppenbarelsetältet, och Aron och hans söner skola lägga sina händer på tjurens huvud.
\par 11 Sedan skall du slakta tjuren inför HERRENS ansikte, vid ingången till uppenbarelsetältet.
\par 12 Och du skall taga av tjurens blod och stryka med ditt finger på altarets hörn; men allt det övriga skall du gjuta ut vid foten av altaret.
\par 13 Och du skall taga allt det fett som omsluter inälvorna, så ock leverfettet och båda njurarna med det fett som sitter på dem, och förbränna det på altaret.
\par 14 Men köttet av tjuren och hans hud och hans orenlighet skall du bränna upp i eld utanför lägret. Det är ett syndoffer.
\par 15 Och du skall taga den ena väduren, och Aron och hans söner skola lägga sina händer på vädurens huvud.
\par 16 Sedan skall du slakta väduren och taga hans blod och stänka på altaret runt omkring;
\par 17 men själva väduren skall du dela i dess stycken, och du skall två inälvorna och fötterna och lägga dem på styckena och huvudet.
\par 18 Och du skall förbränna hela väduren på altaret; det är ett brännoffer åt HERREN. En välbehaglig lukt, ett eldsoffer åt HERREN är det.
\par 19 Därefter skall du taga den andra väduren, och Aron och hans söner skola lägga sina händer på vädurens huvud.
\par 20 Sedan skall du slakta väduren och taga av hans blod och bestryka Arons högra örsnibb och hans söners högra örsnibb och tummen på deras högra hand och stortån på deras högra fot; men det övriga blodet skall det stänka på altaret runt omkring.
\par 21 Och du skall taga av blodet på altaret och av smörjelseoljan och stänka på Aron och hans kläder, och likaledes på hans söner och hans söners kläder; så bliver han helig, han själv såväl som hans kläder, och likaledes hans söner såväl som hans söners kläder.
\par 22 Och du skall taga fettet av väduren, svansen och det fett som omsluter inälvorna, så ock leverfettet och båda njurarna med fettet på dem, därtill det högra lårstycket ty detta är handfyllningsväduren.
\par 23 Och du skall taga en rundkaka, en oljebrödskaka och en tunnkaka ur korgen med de osyrade bröden, som står inför HERRENS ansikte.
\par 24 Och du skall lägga alltsammans på Arons och hans söners händer och vifta det såsom ett viftoffer inför HERRENS ansikte.
\par 25 Sedan skall du taga det ur deras: händer och förbränna det på altaret ovanpå brännoffret, till en välbehaglig lukt inför HERREN; det är ett eldsoffer åt HERREN.
\par 26 Och du skall taga bringan av Arons handfyllningsvädur och vifta den såsom ett viftoffer inför HERRENS ansikte, och detta skall vara din del.
\par 27 Så skall du helga viftoffersbringan och offergärdslåret, det som viftas och det som gives såsom offergärd, de delar av handfyllningsväduren, som skola tillhöra Aron och hans söner.
\par 28 Och detta skall tillhöra Aron och; hans söner såsom en evärdlig rätt av Israels barn, ty det är en offergärd. Det skall vara en gärd av Israels barn, av deras tackoffer, en gärd av dem åt HERREN.
\par 29 Och Arons heliga kläder skola hans söner hava efter honom, för att de i dem må bliva smorda och mottaga handfyllning.
\par 30 I sju dagar skall den av hans söner, som bliver präst i hans ställe, ikläda sig dem, den som skall gå in i uppenbarelsetältet för att göra tjänst i helgedomen.
\par 31 Och du skall taga handfyllningsväduren och koka hans kött på en helig plats.
\par 32 Och vädurens kött jämte brödet: som är i korgen skola Aron och hans söner äta vid ingången till uppenbarelsetältet; de skola äta detta,
\par 33 det som har använts till att bringa försoning vid deras handfyllning och helgande, men ingen främmande får ta därav, ty det är heligt.
\par 34 Och om något av handfyllningsköttet eller av brödet bliver över till följande morgon, så skall du i eld bränna upp detta som har blivit över; det får icke ätas, ty det är heligt.
\par 35 Så skall du göra med Aron och hans söner, i alla stycken såsom jag har bjudit dig. Sju dagar skall deras handfyllning vara.
\par 36 Och var dag skall du offra en tjur såsom syndoffer till försoning och rena altaret, i det du bringar försoning för det; och du skall smörja det för att helga det.
\par 37 I sju dagar skall du bringa försoning för altaret och helga det. Så bliver altaret högheligt; var och en som kommer vid altaret bliver helig.
\par 38 Och detta är vad du skall offra på altaret: två årsgamla lamm för var dag beständigt.
\par 39 Det ena lammet skall du offra om morgonen, och det andra lammet skall du offra vid aftontiden,
\par 40 och till det första lammet en tiondedels efa fint mjöl, begjutet med en fjärdedels hin olja av stötta oliver, och såsom drickoffer en fjärdedels hin vin.
\par 41 Det andra lammet skall du offra vid aftontiden; med likadant spisoffer och drickoffer som om morgonen skall du offra det, till en välbehaglig lukt, ett eldsoffer åt HERREN.
\par 42 Detta skall vara edert dagliga brännoffer från släkte till släkte, vid ingången till uppenbarelsetältet, inför HERRENS ansikte, där jag skall uppenbara mig för eder, för att där tala med dig.
\par 43 Där skall jag uppenbara mig för Israels barn, och det rummet skall bliva helgat av min härlighet.
\par 44 Och jag skall helga uppenbarelsetältet och altaret, och Aron och hans söner skall jag helga till att bliva präster åt mig.
\par 45 Och jag skall bo mitt ibland Israels barn och vara deras Gud.
\par 46 Och de skola förnimma att jag är HERREN, deras Gud, som förde dem ut ur Egyptens land, för att jag skulle bo mitt ibland dem. Jag är HERREN, deras Gud.

\chapter{30}

\par 1 Och du skall göra ett altare för att antända rökelse därpå, av akacieträ skall du göra det.
\par 2 Det skall vara en aln långt och en aln brett - en liksidig fyrkant - och två alnar högt; dess horn skola vara i ett stycke därmed.
\par 3 Och du skall överdraga det med rent guld, dess skiva, dess väggar runt omkring och dess hörn; och du skall göra en rand av guld därpå runt omkring.
\par 4 Och du skall till det göra två ringar av guld och sätta dem nedanför randen, på dess båda sidor; på de båda sidostyckena skall du sätta dem. De skola vara där, för att stänger må skjutas in i dem, så att man med dem kan bära altaret.
\par 5 Och du skall göra stängerna av akacieträ och överdraga dem med guld.
\par 6 Och du skall ställa det framför den förlåt som hänger framför vittnesbördets ark, så att det står framför nådastolen, som är ovanpå vittnesbördet, där jag skall uppenbara mig för dig.
\par 7 Och Aron skall antända välluktande rökelse därpå; var morgon, när han tillreder lamporna, skall han antända rökelse;
\par 8 och likaledes skall Aron antända rökelse, när han vid aftontiden sätter upp lamporna. Detta skall vara det dagliga rökoffret inför HERRENS ansikte, från släkte till släkte.
\par 9 I skolen icke låta någon främmande rökelse komma därpå, ej heller brännoffer eller spisoffer; och intet drickoffer skolen utgjuta därpå.
\par 10 Och Aron skall en gång om året bringa försoning för dess horn; med blod av försoningssyndoffret skall han en gång om året bringa försoning för det, släkte efter släkte. Det är högheligt för HERREN.
\par 11 Och HERREN talade till Mose och sade:
\par 12 När du räknar antalet av Israels barn, nämligen av dem som inmönstras, skall vid mönstringen var och en giva åt HERREN en försoningsgåva för sig, på det att ingen hemsökelse må drabba dem vid mönstringen.
\par 13 Detta är vad var och en som upptages bland de inmönstrade skall giva: en halv sikel, efter helgedomssikelns vikt - sikeln räknad till tjugu gera - en halv sikel såsom offergärd åt HERREN,
\par 14 Var och en som upptages bland de inmönstrade, var och en som är tjugu år gammal eller därutöver, skall giva detta såsom offergärd åt HERREN.
\par 15 Den rike skall icke giva mer och den fattige icke mindre än en halv sikel, när I given offergärden åt HERREN, till att bringa försoning för eder.
\par 16 Och du skall taga försoningspenningarna av Israels barn och använda dem till arbetet vid uppenbarelsetältet. Så skall ske, för att Israels barn må vara i åminnelse inför HERRENS ansikte, och för att försoning må bringas för eder.
\par 17 Och HERREN talade till Mose och sade:
\par 18 Du skall ock göra ett bäcken av koppar med en fotställning av koppar, till tvagning, och ställa det mellan uppenbarelsetältet och altaret och gjuta vatten däri.
\par 19 Och Aron och hans söner skola två sina händer och fötter med vatten därur.
\par 20 När de gå in i uppenbarelsetältet, skola de två sig med vatten, på del att de icke må dö; så ock när de träda fram till altaret för att göra tjänst genom att antända eldsoffer åt HERREN.
\par 21 De skola två sina händer och fötter, på det att de icke må dö. Och detta skall vara en evärdlig stadga för dem: för honom själv och hans avkomlingar från släkte till släkte.
\par 22 Och HERREN talade till Mose och sade:
\par 23 Tag dig ock kryddor av yppersta slag: fem hundra siklar myrradropp, hälften så mycket kanel av finaste slag, alltså två hundra femtio siklar, likaledes två hundra femtio siklar kalmus av finaste slag,
\par 24 därtill fem hundra siklar kassia, efter helgedomssikelns vikt, och en hin olivolja.
\par 25 Och du skall av detta göra en helig smörjelseolja, en konstmässigt beredd salva; det skall vara en helig smörjelseolja.
\par 26 Och du skall därmed smörja uppenbarelsetältet, vittnesbördets ark,
\par 27 bordet med alla dess tillbehör, ljusstaken med dess tillbehör, rökelsealtaret,
\par 28 brännoffersaltaret med alla dess tillbehör, äntligen bäckenet med dess fotställning.
\par 29 Och du skall helga dem, så att de bliva högheliga; var och en som sedan kommer vid dem bliver helig.
\par 30 Och Aron och hans söner skall du smörja, och du skall helga dem till att bliva präster åt mig.
\par 31 Och till Israels barn skall du tala och säga: Detta skall vara min heliga smörjelseolja hos eder, från släkte till släkte.
\par 32 På ingen annan människas kropp må den komma, ej heller mån I göra någon annan så sammansatt som denna. Helig är den, helig skall den vara för eder.
\par 33 Den som bereder en sådan salva, och den som använder något därav på någon främmande, han skall utrotas ur sin släkt.
\par 34 Ytterligare sade HERREN till Mose: Tag dig välluktande kryddor, stakte och sjönagel och galban, och jämte dessa vällukter rent rökelseharts, lika mycket av vart slag,
\par 35 och gör därav rökelse, en konstmässigt beredd blandning, saltad, ren, helig.
\par 36 Och en del av den skall du stöta till pulver och lägga framför vittnesbördet i uppenbarelsetältet, där jag vill uppenbara mig för dig. Höghelig skall den vara för eder.
\par 37 Och ingen annan rökelse mån I göra åt eder så sammansatt som denna skall vara. Helig skall den vara dig för HERREN.
\par 38 Den som gör sådan för att njuta av dess lukt, han skall utrotas ur sin släkt.

\chapter{31}

\par 1 Och HERREN talade till Mose och sade:
\par 2 Se, jag har kallat och nämnt Besalel, son till Uri, son till Hur, av Juda stam;
\par 3 och jag har uppfyllt honom med Guds Ande, med vishet och förstånd och kunskap och med allt slags slöjdskicklighet,
\par 4 till att tänka ut konstarbeten, till att arbeta i guld, silver och koppar,
\par 5 till att snida stenar för infattning och till att snida i trä, korteligen, till att utföra alla slags arbeten.
\par 6 Och se, jag har givit honom till medhjälpare Oholiab, Ahisamaks son, av Dans stam, och åt alla edra konstförfarna män har jag givit vishet i hjärtat. Dessa skola kunna göra allt vad jag har bjudit dig:
\par 7 uppenbarelsetältet, vittnesbördets ark, nådastolen därpå, alla uppenbarelsetältets tillbehör,
\par 8 bordet med dess tillbehör, den gyllene ljusstaken med alla dess tillbehör, rökelsealtaret,
\par 9 brännoffersaltaret med alla dess tillbehör, bäckenet med dess fotställning,
\par 10 de stickade kläderna och prästen Arons andra heliga kläder, så och hans söners prästkläder,
\par 11 äntligen smörjelseoljan och den välluktande rökelsen till helgedomen. De skola utföra sitt arbete i alla stycken såsom jag har bjudit dig.
\par 12 Och HERREN talade till Mose och sade:
\par 13 Tala du till Israels barn och säg: Mina sabbater skolen I hålla, ty de äro ett tecken mellan mig och eder, från släkte till släkte, för att I skolen veta att jag är HERREN, som helgar eder.
\par 14 Så hållen nu sabbaten, ty den skall vara eder helig. Den som ohelgar den skall straffas med döden; ty var och en som på den dagen gör något arbete, han skall utrotas ur sin släkt.
\par 15 Sex dagar skall arbete göras, men på sjunde dagen är vilosabbat, en HERRENS helgdag. Var och en som gör något arbete på sabbatsdagen skall straffas med döden.
\par 16 Och Israels barn skola hålla sabbaten, så att de fira den släkte efter släkte, såsom ett evigt förbund.
\par 17 Den skall vara ett evärdligt tecken mellan mig och Israels barn; ty på sex dagar gjorde HERREN himmel och jord, men på sjunde dagen vilade han och tog sig ro.
\par 18 När han nu hade talat ut med; Mose på Sinai berg, gav han honom vittnesbördets två tavlor, tavlor av sten, på vilka Gud hade skrivit med sitt finger.

\chapter{32}

\par 1 Men när folket såg att Mose dröjde att komma ned från berget, församlade de sig omkring Aron och sade till honom: "Upp, gör oss en gud som kan gå framför oss; ty vi veta icke vad som har vederfarits denne Mose, honom som förde oss upp ur Egyptens land."
\par 2 Då sade Aron till dem: "Tagen guldringarna ut ur öronen på edra hustrur, edra söner och edra döttrar, och bären dem till mig.
\par 3 Då tog allt folket av sig guldringarna som de hade i öronen, och de buro dem till Aron;
\par 4 och han tog emot guldet av dem och formade det med en mejsel och gjorde därav en gjuten kalv. Och de sade: "Detta är din Gud, Israel, han som har fört dig upp ur Egyptens land."
\par 5 När Aron såg detta, byggde han ett altare åt honom. Och Aron lät utropa och säga: "I morgon bliver en HERRENS högtid."
\par 6 Dagen därefter stodo de bittida upp och offrade brännoffer och buro fram tackoffer, och folket satte sig ned till att äta och dricka, och därpå stodo de upp till att leka.
\par 7 Då sade HERREN till Mose: "Gå ditned, ty ditt folk, som du har fört upp ur Egyptens land, har tagit sig till, vad fördärvligt är.
\par 8 De hava redan vikit av ifrån den väg som jag bjöd dem gå; de hava gjort sig en gjuten kalv. Den hava de tillbett, åt den hava de offrat, och sagt: 'Detta är din Gud, Israel, han som har fört dig upp ur Egyptens land."
\par 9 Och HERREN sade ytterligare till Mose: "Jag ser att detta folk är ett hårdnackat folk.
\par 10 Så låt mig nu vara, på det att min vrede må brinna mot dem, och på det att jag må förgöra dem; dig vill jag sedan göra till ett stort folk."
\par 11 Men Mose bönföll inför HERREN, sin Gud, och sade: "HERRE, varför skulle din vrede brinna mot ditt folk, som du med stor kraft och stark hand har fört ut ur Egyptens land?
\par 12 Varför skulle egyptierna få säga: 'Till deras olycka har han fört dem ut, till att dräpa dem bland bergen och förgöra dem från jorden'? Vänd dig ifrån din vredes glöd, och ångra det onda du nu har i sinnet mot ditt folk.
\par 13 Tänk på Abraham, Isak och Israel, dina tjänare, åt vilka du med ed vid dig själv har givit det löftet: 'Jag skall göra eder säd talrik såsom stjärnorna på himmelen, och hela det land som jag har talat om skall jag giva åt eder säd, och de skola få det till evärdlig arvedel.'"
\par 14 Då ångrade HERREN det onda som han hade hotat att göra mot sitt folk.
\par 15 Och Mose vände sig om och steg ned från berget, och han hade med sig vittnesbördets två tavlor. Och på tavlornas båda sidor var skrivet både på den ena sidan och på den andra var skrivet.
\par 16 Och tavlorna voro gjorda av Gud, och skriften var Guds skrift, inristad på tavlorna.
\par 17 När Josua nu hörde huru folket skriade, sade han till Mose: "Krigsrop höres i lägret."
\par 18 Men han svarade: "Det är varken segerrop som höres, ej heller är det ett ropande såsom efter nederlag; det är sång jag hör."
\par 19 När sedan Mose kom närmare lägret och fick se kalven och dansen, upptändes hans vrede, och han kastade tavlorna ifrån sig och slog sönder dem nedanför berget.
\par 20 Sedan tog han kalven som de hade gjort, och brände den i eld och krossade den till stoft; detta strödde han i vattnet och gav det åt Israels barn att dricka.
\par 21 Och Mose sade till Aron: "Vad har folket gjort dig, eftersom du har kommit dem att begå en så stor synd?"
\par 22 Aron svarade: "Min herres vrede må icke upptändas; du vet själv att detta folk är ont.
\par 23 De sade till mig: 'Gör oss en gud som kan gå framför oss; ty vi veta icke vad som har vederfarits denne Mose, honom som förde oss upp ur Egyptens land.'
\par 24 Då sade jag till dem: 'Den som har guld tage det av sig'; och de gåvo det åt mig. Och jag kastade det i elden, och så kom kalven till."
\par 25 Då nu Mose såg att folket var lössläppt, eftersom Aron, till skadeglädje för deras fiender, hade släppt dem lösa,
\par 26 ställde han sig i porten till lägret och ropade: "Var och en som hör HERREN till komme hit till mig." Då församlade sig till honom alla Levi barn.
\par 27 Och han sade till dem: "Så säger HERREN, Israels Gud: Var och en binde sitt svärd vid sin länd. Gån så igenom lägret, fram och tillbaka, från den ena porten till den andra, Och dräpen vem I finnen, vore det också broder eller vän eller frände.
\par 28 Och Levi barn gjorde såsom Mose hade sagt; och på den dagen föllo av folket vid pass tre tusen män.
\par 29 Och Mose sade: "Eftersom I nu haven stått emot edra egna söner och bröder, mån I i dag taga handfyllning till HERRENS tjänst, för att välsignelse i dag må komma över eder."
\par 30 Dagen därefter sade Mose till folket: "I haven begått en stor synd. Jag vill nu stiga upp till HERREN och se till, om jag kan bringa försoning för eder synd."
\par 31 Och Mose gick tillbaka till HERREN och sade: "Ack, detta folk har begått en stor synd; de hava gjort sig en gud av guld.
\par 32 Men förlåt dem nu deras synd; varom icke, så utplåna mig ur boken som du skriver i."
\par 33 Men HERREN svarade Mose: "Den som har syndat mot mig, honom skall jag utplåna ur min bok.
\par 34 Gå nu och för folket dit jag har sagt dig; se, min ängel skall gå framför dig. Men när min hemsökelses dag kommer, skall jag på dem hemsöka deras synd."
\par 35 Så straffade HERREN folket, därför att de hade gjort kalven, den som Aron gjorde.

\chapter{33}

\par 1 Och HERREN sade till Mose: "Upp, drag åstad härifrån med folket som du har fört upp ur Egyptens land, och begiv dig till det land som jag med ed har lovat åt Abraham, Isak och Jakob, i det jag sade: 'Åt din säd skall jag giva det.'
\par 2 Jag skall sända en ängel framför dig och förjaga kananéerna, amoréerna, hetiterna, perisséerna, hivéerna och jebuséerna,
\par 3 för att du må komma till ett land som flyter av mjölk och honung. Ty eftersom du är ett hårdnackat folk, vill jag icke själv draga upp med dig; jag kunde då förgöra dig under vägen."
\par 4 När folket hörde detta stränga tal, blevo de sorgsna, och ingen tog sina smycken på sig.
\par 5 Och HERREN sade till Mose: "Säg till Israels barn: I ären ett hårdnackat folk. Om jag allenast ett ögonblick droge med dig, skulle jag förgöra dig. Men lägg nu av dig dina smycken, så vill jag se till, vad jag skall göra med dig."
\par 6 Så togo då Israels barn av sig sina smycken och voro utan dem allt ifrån vistelsen vid Horebs berg.
\par 7 Men Mose hade för sed att taga tältet och slå upp det ett stycke utanför lägret; och han kallade det "uppenbarelsetältet". Och var och en som ville rådfråga HERREN måste gå ut till uppenbarelsetältet utanför lägret.
\par 8 Och så ofta Mose gick ut till tältet, stod allt folket upp, och var och en ställde sig vid ingången till sitt tält och skådade efter Mose, till dess han hade kommit in i tältet.
\par 9 Och så ofta Mose kom in i tältet, steg molnstoden ned och blev stående vid ingången till tältet; och han talade med Mose.
\par 10 Och allt folket såg molnstoden stå vid ingången till tältet; då föll allt folket ned och tillbad, var och en vid ingången till sitt tält.
\par 11 Och HERREN talade med Mose ansikte mot ansikte, såsom när den ena människan talar med den andra. Sedan vände Mose tillbaka till lägret; men hans tjänare Josua, Nuns son, en ung man, lämnade icke tältet.
\par 12 Och Mose sade till HERREN: "Väl säger du till mig: 'För detta folk ditupp'; men du har icke låtit mig veta vem du vill sända med mig Du har dock sagt: 'Jag känner dig vid namn, och du har funnit nåd för mina ögon.'
\par 13 Om jag alltså har funnit nåd för dina ögon, så låt mig se dina vägar och lära känna dig; jag vill ju finna nåd för dina ögon. Och se därtill, att detta folk är ditt folk."
\par 14 Han sade: "Skall jag då själv gå med och föra dig till ro?"
\par 15 Han svarade honom: "Om du icke själv vill gå med, så låt oss alls icke draga upp härifrån.
\par 16 Ty varigenom skall man kunna veta att jag och ditt folk hava funnit nåd för dina ögon, om icke därigenom att du går med oss, så att vi, jag och ditt folk, utmärkas framför alla andra folk på jorden?"
\par 17 HERREN svarade Mose: "Vad du nu har begärt skall jag ock göra; ty du har funnit nåd för mina ögon, och jag känner dig vid namn."
\par 18 Då sade han: "Låt mig alltså se din härlighet."
\par 19 Han svarade: "Jag skall låta all min skönhet gå förbi dig där du står, och jag skall utropa namnet 'HERREN' inför dig; jag skall vara nådig mot den jag vill vara nådig emot, och skall förbarma mig över den jag vill förbarma mig över.
\par 20 Ytterligare sade han: "Mitt ansikte kan du dock icke få se, ty ingen människa kan se mig och leva."
\par 21 Därefter sade HERREN: "Se, här är en plats nära intill mig; ställ dig där på klippan.
\par 22 När nu min härlighet går förbi, skall jag låta dig stå där i en klyfta på berget, och jag skall övertäcka dig med min hand, till dess jag har gått förbi.
\par 23 Sedan skall jag taga bort min hand, och då skall du få se mig på ryggen; men mitt ansikte kan ingen se."

\chapter{34}

\par 1 Och HERREN sade till Mose: "Hugg ut åt dig två stentavlor, likadana som de förra voro, så vill jag skriva på tavlorna samma ord som stodo på de förra tavlorna, vilka du slog sönder.
\par 2 Och var redo till i morgon, du skall då på morgonen stiga upp på Sinai berg och ställa dig på toppen av berget, mig till mötes,
\par 3 men ingen må stiga upp med dig, och på hela berget för ingen annan visa sig; ej heller må får och fäkreatur gå i bet framemot detta berg."
\par 4 Och han högg ut två stentavlor likadana som de förra voro. Och bittida följande morgon begav sig Mose upp på Sinai berg, såsom HERREN hade bjudit honom, och tog de två stentavlorna med sig.
\par 5 Då steg HERREN ned i molnskyn. Och han ställde sig där nära intill honom och åkallade HERRENS namn.
\par 6 Och HERREN gick förbi honom, där han stod, och utropade: "HERREN! HERREN! - en Gud, barmhärtig och nådig, långmodig och stor i mildhet och trofasthet,
\par 7 som bevarar nåd mot tusenden, som förlåter missgärning och överträdelse och synd, men som ingalunda låter någon bliva ostraffad, utan hemsöker fädernas missgärning på barn och barnbarn och efterkommande i tredje och fjärde led."
\par 8 Då böjde Mose sig med hast ned mot jorden och tillbad
\par 9 och sade: "Om jag har funnit nåd för dina ögon, Herre, så må Herren gå med oss. Ty väl är det ett hårdnackat folk, men du vill ju förlåta oss vår missgärning och synd och taga oss till din arvedel."
\par 10 Han svarade: "Välan, jag vill sluta ett förbund. Inför hela ditt folk skall jag göra under, sådana som icke hava blivit gjorda i något land eller bland något folk. Och hela det folk som du tillhör skall se att HERRENS gärningar äro underbara, de som jag skall göra med dig.
\par 11 Håll de bud som jag i dag giver dig. Se, jag skall förjaga för dig amoréerna, kananéerna, hetiterna, perisséerna, hivéerna och jebuséerna.
\par 12 Tag dig till vara för att sluta förbund med inbyggarna i det land dit du kommer, och låt dem icke bliva till en snara bland eder.
\par 13 Fastmer skolen I bryta ned deras altaren och slå sönder deras stoder och hugga ned deras Aseror.
\par 14 Ja, du skall icke tillbedja någon annan gud, ty HERREN heter Nitälskare; en nitälskande Gud är han.
\par 15 Du må icke sluta något förbund med landets inbyggare. Ty i trolös avfällighet löpa de efter sina gudar och offra åt sina gudar; och när de då inbjuda dig, kommer du att äta av deras offer;
\par 16 du tager ock deras döttrar till hustrur åt dina söner, och när då deras döttrar i avfällighet löpa efter sina gudar, skola de förleda dina söner till att likaledes löpa efter deras gudar.
\par 17 Gjutna gudar skall du icke göra åt dig.
\par 18 Det osyrade brödets högtid skall, du hålla: i sju dagar skall du äta osyrat bröd, såsom jag har bjudit dig, på den bestämda tiden i månaden Abib; ty i månaden Abib drog du ut ur Egypten.
\par 19 Allt det som öppnar moderlivet skall höra mig till, också allt hankön bland din boskap, som öppnar moderlivet, såväl av fäkreaturen som av småboskapen.
\par 20 Men vad som bland åsnor öppnar moderlivet skall du lösa med ett får, och om du icke vill lösa det, skall du krossa nacken på det. Var förstfödd bland dina söner skall du läsa. Och ingen skall med tomma händer träda fram inför mitt ansikte.
\par 21 Sex dagar skall du arbeta, men på sjunde dagen skall du hålla vilodag; både under plöjningstiden och under skördetiden skall du hålla vilodag.
\par 22 Och veckohögtiden skall du hålla, för förstlingen av veteskörden, så ock bärgningshögtiden, när året har gått till ända.
\par 23 Tre gånger om året skall allt ditt mankön träda fram inför HERRENS, din herres, Israels Guds, ansikte.
\par 24 Ty jag skall fördriva folk för dig och utvidga ditt område; och ingen skall stå efter ditt land, när du drager upp, tre gånger om året, för att träda fram inför HERRENS, din Guds, ansikte.
\par 25 Du skall icke offra blodet av mitt slaktoffer jämte något som är syrat. Och påskhögtidens slaktoffer skall icke lämnas kvar över natten till morgonen.
\par 26 Det första av din marks förstlingsfrukter skall du föra till HERRENS, din Guds, hus. Du skall icke koka en killing i dess moders mjölk."
\par 27 Och HERREN sade till Mose: "Teckna upp åt dig dessa ord; ty i enlighet med dessa ord har jag slutit ett förbund med dig och med Israel."
\par 28 Och han blev kvar där hos HERREN i fyrtio dagar och fyrtio nätter, utan att äta och utan att dricka. Och han skrev på tavlorna förbundets ord, de tio orden.
\par 29 När sedan Mose steg ned från Sinai berg, och på vägen ned från berget hade vittnesbördets två tavlor med sig, visste han icke att hans ansiktes hy hade blivit strålande därav att han hade talat med honom.
\par 30 Och när Aron och alla Israels barn sågo huru Moses ansiktes hy strålade, fruktade de för att komma honom nära.
\par 31 Men Mose ropade till dem; då vände Aron och menighetens alla hövdingar tillbaka till honom, och Mose talade till dem.
\par 32 Därefter kommo alla Israels barn fram till honom, och han gav dem alla de bud som HERREN hade förkunnat för honom på Sinai berg.
\par 33 Och när Mose hade slutat sitt tal till dem, hängde han ett täckelse för sitt ansikte.
\par 34 Men så ofta Mose skulle träda inför HERRENS ansikte för att tala med honom, lade han av täckelset, till dess han åter gick ut. Och sedan han hade kommit ut, förkunnade han för Israels barn det som hade blivit honom bjudet.
\par 35 Då sågo Israels barn var gång huru Moses ansiktes by strålade, och Mose hängde då åter täckelset över sitt ansikte, till dess han ånyo skulle gå in för att tala med honom.

\chapter{35}

\par 1 Och Mose församlade Israels barns hela menighet och sade till dem: "Detta är vad HERREN har bjudit eder att göra:
\par 2 Sex dagar skall arbete göras, men på sjunde dagen skolen I hava helgdag, en HERRENS vilosabbat. Var och en som på den dagen gör något arbete skall dödas.
\par 3 I skolen icke tända upp eld på: sabbatsdagen, var I än ären bosatta.
\par 4 Och Mose sade till Israels barns hela menighet: "Detta är vad HERREN har bjudit och sagt:
\par 5 Låten bland eder upptaga en gärd åt HERREN, så att var och en som har ett därtill villigt hjärta bär fram denna gård åt HERREN: guld, silver och koppar,
\par 6 mörkblått, purpurrött, rosenrött och vitt garn och gethår,
\par 7 rödfärgade vädurskinn, tahasskinn, akacieträ,
\par 8 olja till ljusstaken, kryddor till smörjelseoljan och till den välluktande rökelsen,
\par 9 äntligen onyxstenar och infattningsstenar, till att användas för efoden och för bröstskölden.
\par 10 Och alla konstförfarna män bland eder må komma och förfärdiga allt vad HERREN har bjudit:
\par 11 tabernaklet, dess täckelse och överdraget till detta, dess häktor, bräder, tvärstänger, stolpar och fotstycken,
\par 12 arken med dess stänger, nådastolen och den förlåt som skall hänga framför den,
\par 13 bordet med dess stänger och alla dess tillbehör och skådebröden,
\par 14 ljusstaken med dess tillbehör och dess lampor, oljan till ljusstaken,
\par 15 rökelsealtaret med dess stänger, smörjelseoljan och den välluktande rökelsen, förhänget för ingången till tabernaklet,
\par 16 brännoffersaltaret med tillhörande koppargaller, dess stänger och alla dess tillbehör, bäckenet med dess fotställning,
\par 17 omhängena till förgården, dess stolpar och fotstycken, förhänget för porten till förgården,
\par 18 tabernaklets pluggar och förgårdens pluggar med deras streck,
\par 19 äntligen de stickade kläderna till tjänsten i helgedomen och prästen Arons andra heliga kläder, så ock hans söners prästkläder."
\par 20 Och Israels barns hela menighet gick sin väg bort ifrån Mose.
\par 21 Sedan kommo de tillbaka, var och en som av sitt hjärta manades därtill; och var och en som hade en därtill villig ande bar fram en gärd åt HERREN till förfärdigande av uppenbarelsetältet och till allt arbete därvid och till de heliga kläderna.
\par 22 De kommo, både män och kvinnor, och framburo, var och en efter sitt hjärtas villighet, spännen, örringar, fingerringar och halssmycken, alla slags klenoder av guld, var och en som kunde offra åt HERREN någon gåva av guld.
\par 23 Och var och en som hade i sin ägo mörkblått, purpurrött, rosenrött eller vitt garn eller gethår eller rödfärgade vädurskinn eller tahasskinn bar fram det.
\par 24 Och var och en som kunde giva såsom gärd något av silver eller koppar bar fram sin gärd åt HERREN. Och var och en som hade i sin ägo akacieträ till förfärdigande av något slags arbete bar fram det.
\par 25 Och alla konstförfarna kvinnor spunno med sina händer mörkblått, purpurrött, rosenrött och vitt garn och buro fram sin spånad;
\par 26 och alla kvinnor som av sitt hjärta manades därtill och hade lärt konsten spunno gethår.
\par 27 Och hövdingarna buro fram onyxstenar och infattningsstenar, till att användas för efoden och för bröstskölden,
\par 28 vidare kryddor och olja, till att användas för ljusstaken och smörjelseoljan och den välluktande rökelsen.
\par 29 Var och en av Israels barn, man eller kvinna, vilkens hjärta var villigt att bära fram något till förfärdigande av allt det som HERREN genom Mose hade bjudit att man skulle göra, bar fram sin frivilliga gåva åt HERREN.
\par 30 och Mose sade till Israels barn: "Sen, HERREN har kallat och nämnt Besalel, son till Uri, son till Hur, av Juda stam;
\par 31 och han har uppfyllt honom med Guds Ande, med vishet, med förstånd och kunskap och med allt slags slöjdskicklighet,
\par 32 både till att tänka ut konstarbeten och till att arbeta i guld, silver och koppar,
\par 33 till att smida stenar för infattning: och till att snida i trä, korteligen, till att utföra alla slags konstarbeten.
\par 34 Åt honom och åt Oholiab, Ahisamaks son, av Dans stam, har han ock givit förmåga att undervisa andra.
\par 35 Han har uppfyllt deras hjärtan med vishet till att utföra alla slags snideriarbeten och konstvävnader och brokiga vävnader av mörkblått, purpurrött, rosenrött och vitt garn, så ock andra vävnader, korteligen, alla slags arbeten och särskilt konstvävnadsarbeten.

\chapter{36}

\par 1 Och Besalel och Oholiab och alla andra konstförfarna män, åt vilka HERREN har givit vishet och förstånd till att veta huru de skola utföra allt arbete vid helgedomens förfärdigande, skola utföra det, i alla stycken såsom HERREN har bjudit."
\par 2 Därefter kallade Mose till sig Besalel och Oholiab och alla de andra konstförfarna männen, åt vilka HERREN hade givit vishet i hjärtat, alla som av sitt hjärta manades att träda fram för att utföra arbetet.
\par 3 Och de mottogo från Mose hela den gärd som Israels barn hade burit fram till utförande av arbetet vid helgedomens förfärdigande. Men man fortfor att bära fram till honom frivilliga gåvor, morgon efter morgon.
\par 4 Då kommo alla de konstförfarna män som utförde allt arbetet till helgedomen, var och en från det arbete som han utförde,
\par 5 och sade till Mose: "Folket bär fram mer än som behöves för att verkställa det arbete som HERREN har bjudit oss att utföra."
\par 6 Då bjöd Mose att man skulle låta utropa i lägret: "Ingen, vare sig man eller kvinna, må vidare arbeta för att göra något till helgedomen." Så avhölls folket ifrån att bära fram flera gåvor.
\par 7 Ty vad man hade skaffat samman var tillräckligt för allt det arbete som skulle göras, och man hade till och med över.
\par 8 Så gjorde nu alla de konstförfarna arbetarna tabernaklet av tio tygvåder; av tvinnat vitt garn och av mörkblått, purpurrött och rosenrött garn gjorde man dem, med keruber på, i konstvävnad.
\par 9 Var våd gjordes tjuguåtta alnar lång och fyra alnar bred; alla våderna fingo samma mått.
\par 10 Och man fogade tillhopa fem av våderna med varandra; likaså fogade man tillhopa de fem övriga våderna med varandra.
\par 11 Och man satte öglor av mörkblått garn i kanten på den ena våden, ytterst på det hopfogade stycket; så gjorde man ock i kanten på den våd som satt ytterst i det andra hopfogade stycket.
\par 12 Femtio öglor satte man på den ena våden, och femtio öglor satte man ytterst på motsvarande våd i det andra hopfogade stycket, så att öglorna svarade emot varandra.
\par 13 Och man gjorde femtio häktor av guld och fogade våderna tillhopa med varandra medelst häktorna, så att tabernaklet utgjorde ett helt.
\par 14 Man gjorde och tygvåder av gethår till ett täckelse över tabernaklet; elva sådana våder gjorde man.
\par 15 Var våd gjordes trettio alnar lång och fyra alnar bred, de elva våderna fingo samma mått.
\par 16 Fem av våderna fogade man tillhopa till ett särskilt stycke,
\par 17 och likaledes de sex övriga våderna till ett särskilt stycke. Och man satte femtio öglor i kanten på den våd som satt ytterst i det ena hopfogade stycket, och femtio öglor satte man i kanten på motsvarande våd i det andra hopfogade stycket.
\par 18 Och man gjorde femtio häktor av koppar för att foga tillhopa täckelset, så att det kom att utgöra ett helt.
\par 19 Vidare gjorde man ett överdrag av rödfärgade vädurskinn till täckelset, och ytterligare ett överdrag av tahasskinn att lägga ovanpå detta.
\par 20 Bräderna till tabernaklet gjorde man av akacieträ och ställde dem upprätt.
\par 21 Tio alnar långt och en och en halv aln brett gjordes vart bräde.
\par 22 På vart bräde sattes två tappar, förbundna sinsemellan med en list; så gjorde man på alla bräderna till tabernaklet.
\par 23 Och av tabernaklets bräder satte man tjugu på södra sidan, söderut.
\par 24 Och man gjorde fyrtio fotstycken av silver att sätta under de tjugu bräderna, två fotstycken under vart bräde för dess två tappar.
\par 25 Likaledes satte man på tabernaklets andra sida, den norra sidan, tjugu bräder,
\par 26 med deras fyrtio fotstycken av silver, två fotstycken under vart bräde.
\par 27 Men på baksidan av tabernaklet, västerut, satte man sex bräder.
\par 28 Och två bräder satte man på tabernaklets hörn, på baksidan;
\par 29 och vartdera av dessa var sammanfogat av två nedtill, och likaledes sammanhängande upptill, till den första ringen. Så gjorde man med dem båda, i de båda hörnen.
\par 30 Således blev det åtta bräder med tillhörande fotstycken av silver, sexton fotstycken, nämligen två fotstycken under vart bräde.
\par 31 Och man gjorde tvärstänger av akacieträ, fem till de bräder som voro på tabernaklets ena sida,
\par 32 och fem tvärstänger till de bräder som voro på tabernaklets andra sida, och fem tvärstänger till de bräder som voro på tabernaklets baksida, västerut.
\par 33 Och man satte den mellersta tvärstången så, att den gick tvärs över, mitt på bräderna, från den ena ändan till den andra.
\par 34 Och bräderna överdrog man med guld, och ringarna på dem, i vilka tvärstängerna skulle skjutas in, gjorde man av guld, och tvärstängerna överdrog man med guld.
\par 35 Man gjorde ock förlåten av mörkblått, purpurrött, rosenrött och tvinnat vitt garn; man gjorde den i konstvävnad, med keruber på.
\par 36 Och man gjorde till den fyra stolpar av akacieträ och överdrog dem med guld, och hakarna till dem gjordes av guld, och man göt till dem fyra fotstycken av silver.
\par 37 Och man gjorde ett förhänge för ingången till tältet, i brokig vävnad av mörkblått, purpurrött, rosenrött och tvinnat vitt garn,
\par 38 och till detta fem stolpar med deras hakar; och deras knoppar och deras kransar överdrog man med guld, och deras fem fotstycken gjordes av koppar.

\chapter{37}

\par 1 Och Besalel gjorde arken av akacieträ, två och en halv aln lång, en och en halv aln bred och en och en halv aln hög.
\par 2 Och han överdrog den med rent guld innan och utan; och han gjorde på den en rand av guld runt omkring.
\par 3 Och han göt till den fyra ringar: av guld och satte dem över de fyra fötterna, två ringar på ena sidan och två ringar på andra sidan.
\par 4 Och han gjorde stänger av akacieträ och överdrog dem med guld.
\par 5 Och stängerna sköt han in i ringarna, på sidorna av arken, så att man kunde bära arken.
\par 6 Och han gjorde en nådastol av rent guld, två och en halv aln lång och en och en halv aln bred.
\par 7 Och han gjorde två keruber av guld; i drivet arbete gjorde han dem och satte dem vid de båda ändarna av nådastolen,
\par 8 en kerub vid ena ändan och en kerub vid andra ändan. I ett stycke med nådastolen gjorde han keruberna vid dess båda ändar.
\par 9 Och keruberna bredde ut sina vingar och höllo dem uppåt, så att de övertäckte nådastolen med sina vingar, under det att de hade sina ansikten vända mot varandra; ned mot nådastolen vände keruberna sina ansikten.
\par 10 Han gjorde ock bordet av akacieträ, två alnar långt, en aln brett och en och en halv aln högt.
\par 11 Och han överdrog det med rent guld; och han gjorde en rand av guld därpå runt omkring.
\par 12 Och runt omkring det gjorde han en list av en hands bredd, och runt omkring listen gjorde han en rand av guld.
\par 13 Och han göt till bordet fyra ringar av guld och satte ringarna i de fyra hörnen vid de fyra fötterna.
\par 14 Invid listen sattes ringarna, för att stängerna skulle skjutas in i dem, så att man kunde bära bordet.
\par 15 Och han gjorde stängerna av akacieträ och överdrog dem med guld; så kunde man bära bordet.
\par 16 Och han gjorde kärlen till bordet av rent guld, faten och skålarna, bägarna och kannorna med vilka man skulle utgjuta drickoffer.
\par 17 Han gjorde ock ljusstaken av rent guld. I drivet arbete gjorde han ljusstaken med dess fotställning och dess mittelrör; kalkarna därpå, kulor och blommor, gjordes i ett stycke med den.
\par 18 Och sex armar utgingo från ljusstakens sidor, tre armar från ena sidan och tre armar från andra sidan.
\par 19 På den ena armen sattes tre kalkar, liknande mandelblommor, vardera bestående av en kula och en blomma, och på den andra armen sammalunda tre kalkar, liknande mandelblommor, vardera bestående av en kula och en blomma; så gjordes på de sex armar som utgingo från ljusstaken
\par 20 Men på själva ljusstaken sattes fyra kalkar, liknande mandelblommor, med sina kulor och blommor.
\par 21 En kula sattes under det första armparet som utgick från ljusstaken, i ett stycke med den, och en kula under det andra armparet som utgick från ljusstaken, i ett stycke med den, och en kula under det tredje armparet som utgick från ljusstaken, i ett stycke med den: alltså under de sex armar som utgingo från den.
\par 22 Deras kulor och armar gjordes i ett stycke med den, alltsammans ett enda stycke i drivet arbete av rent guld.
\par 23 Och han gjorde till den sju lampor, så ock lamptänger och brickor till den av rent guld.
\par 24 Av en talent rent guld gjorde han den med alla dess tillbehör.
\par 25 Och han gjorde rökelsealtaret av akacieträ, en aln långt och en aln brett - en liksidig fyrkant - och två alnar högt; dess horn gjordes i ett stycke därmed.
\par 26 Och han överdrog det med rent guld, dess skiva, dess väggar runt omkring och dess hörn; och han gjorde en rand av guld därpå runt omkring.
\par 27 Och han gjorde till det två ringar av guld och satte dem nedanför randen, på dess båda sidor, på de båda sidostyckena, för att stänger skulle skjutas in i dem, så att män med dem kunde bära altaret.
\par 28 Och han gjorde stängerna av akacieträ och överdrog dem med guld.
\par 29 Han gjorde ock den heliga smörjelseoljan och den rena, välluktande rökelsen, konstmässigt beredda.

\chapter{38}

\par 1 Han gjorde ock brännoffersaltaret av akacieträ, fem alnar långt och fem alnar brett - en liksidig fyrkant - och tre alnar högt.
\par 2 Och han gjorde hörn därtill och satte dem i dess fyra hörn; i ett stycke därmed gjordes hörnen. Och han överdrog det med koppar.
\par 3 Och han gjorde altarets alla tillbehör, askkärlen, skovlarna, skålarna, gafflarna och fyrfaten. Alla dess tillbehör gjorde han av koppar.
\par 4 Och han gjorde till altaret ett galler, ett nätverk av koppar, och satte det under dess avsats, nedtill, så att det räckte upp till mitten.
\par 5 Och han göt fyra ringar och satte dem i de fyra hörnen på koppargallret, för att stängerna skulle skjutas in i dem.
\par 6 Och han gjorde stängerna av akacieträ och överdrog dem med koppar.
\par 7 Och han sköt stängerna in i ringarna på altarets sidor, så att man kunde bära det med dem. Ihåligt gjorde han det, av plankor.
\par 8 Han gjorde ock bäckenet av koppar med dess fotställning av koppar och använde därtill speglar, som hade tillhört de kvinnor vilka hade tjänstgöring vid ingången till uppenbarelsetältet.
\par 9 Han gjorde ock förgården. För den södra sidan, söderut, gjordes omhängena till förgården av tvinnat sitt garn, hundra alnar långa;
\par 10 till dem gjordes tjugu stolpar, och till dessa tjugu fotstycken, av koppar, men stolparnas hakar och kransar gjordes av silver.
\par 11 Likaledes gjordes de för norra sidan hundra alnar långa; till dem gjordes tjugu stolpar, och till dessa tjugu fotstycken, av koppar, men stolparnas hakar och kransar gjordes av silver.
\par 12 Och för västra sidan gjordes omhängen som voro femtio alnar långa; till dem gjordes tio stolpar, och till dessa tio fotstycken, men stolparnas hakar och kransar gjordes av silver.
\par 13 Och för framsidan, österut, gjordes de femtio alnar långa.
\par 14 Omhängena gjordes femton alnar långa på ena sidan därav, med tre stolpar på tre fotstycken; likaledes gjordes omhängena på andra sidan femton alnar långa - alltså lika på båda sidor om porten till förgården - med tre stolpar på tre fotstycken.
\par 15 Alla omhängena runt omkring förgården gjordes av tvinnat vitt garn;
\par 16 och fotstyckena till stolparna gjordes av koppar, men stolparnas hakar och kransar gjordes av silver, och deras knoppar överdrogos med silver;
\par 17 alla förgårdens stolpar försågos med kransar av silver.
\par 18 Och förhänget för porten till förgården gjordes i brokig vävnad av mörkblått, purpurrött, rosenrött och tvinnat vitt garn, tjugu alnar långt och fem alnar högt, efter tygets bredd, i likhet med förgårdens omhängen;
\par 19 och till det gjordes fyra stolpar på fyra fotstycken, av koppar; men deras hakar gjordes av silver, och deras knoppar överdrogos med silver, och deras kransar gjordes av silver.
\par 20 Alla pluggarna till tabernaklet och till förgården runt omkring gjordes av koppar.
\par 21 Följande är vad som beräknas hava åtgått till tabernaklet, vittnesbördets tabernakel, vilken beräkning gjordes efter Moses befallning genom leviternas försorg, under ledning av Itamar, prästen Arons son;
\par 22 och Besalel, son till Uri, son till Hur, av Juda stam, förfärdigade allt vad HERREN hade bjudit Mose,
\par 23 och till medhjälpare hade han Oholiab, Ahisamaks son, av Dans stam, en man kunnig i snideri och konstvävnad och i konsten att väva brokigt med mörkblått, purpurrött, rosenrött och vitt garn.
\par 24 Det guld som användes till arbetet, vid förfärdigandet av hela helgedomen, det guld som hade blivit givet såsom offer, utgjorde sammanlagt tjugunio talenter och sju hundra trettio siklar, efter helgedomssikelns vikt.
\par 25 Och det silver som gavs av dem i menigheten, vilka inmönstrades, utgjorde ett hundra talenter och ett tusen sju hundra sjuttiofem siklar, efter helgedomssikelns vikt.
\par 26 En beka, det är en halv sikel, efter helgedomssikelns vikt, kom på var person, på var och en som upptogs bland de inmönstrade, var och en som var tjugu år gammal eller därutöver: sex hundra tre tusen fem hundra femtio personer.
\par 27 Och de hundra talenterna silver användes till gjutningen av fotstyckena för helgedomen och av fotstyckena för förlåten, ett hundra talenter till ett hundra fotstycken, en talent till vart fotstycke.
\par 28 Och de ett tusen sju hundra sjuttiofem siklarna användes till att göra hakar till stolparna och till att överdraga deras knoppar och göra kransar till dem.
\par 29 Och den koppar som hade blivit given såsom offer utgjorde sjuttio talenter och två tusen fyra hundra siklar.
\par 30 Därav gjorde man fotstyckena till uppenbarelsetältets ingång, så ock kopparaltaret med tillhörande koppargaller och altarets alla tillbehör,
\par 31 vidare fotstyckena till förgården, runt omkring, och fotstyckena till förgårdens port, äntligen alla tabernaklets pluggar och alla förgårdens pluggar, runt omkring.

\chapter{39}

\par 1 Och av det mörkblåa, det purpurröda och det rosenröda garnet gjorde man stickade kläder till tjänsten i helgedomen; och man gjorde de andra heliga kläderna som Aron skulle hava, såsom HERREN hade bjudit Mose.
\par 2 Efoden gjorde man av guld och av mörkblått, purpurrött, rosenrött och tvinnat vitt garn.
\par 3 Man hamrade ut guldet till tunna plåtar och skar dessa i trådar, så att man kunde väva in det i det mörkblåa, det purpurröda, det rosenröda och det vita garnet, med konstvävnad.
\par 4 Till den gjorde man axelstycken, som skulle fästas ihop; vid sina båda ändar fästes den ihop.
\par 5 Och skärpet, som skulle sitta på efoden och sammanhålla den, gjordes i ett stycke med den och av samma slags vävnad: av guld och av mörkblått, purpurrött, rosenrött och tvinnat vitt garn, allt såsom HERREN hade bjudit Mose.
\par 6 Och onyxstenarna omgav man med flätverk av guld; på dem voro Israels söners namn inristade, på samma sätt som man graverar signetringar.
\par 7 Och man satte dem på efodens axelstycken, för att stenarna skulle bringa Israels barn i åminnelse, allt såsom HERREN hade bjudit Mose.
\par 8 Bröstskölden gjorde man i konstvävnad, i samma slags vävnad som efoden: av guld och av mörkblått, purpurrött, rosenrött och tvinnat vitt garn.
\par 9 Bröstskölden gjordes liksidigt fyrkantig, i form av en väska gjorde man den: ett kvarter lång och ett kvarter bred, i form av en väska.
\par 10 Och man besatte den med fyra rader stenar: i första raden en karneol, en topas och en smaragd;
\par 11 i andra raden en karbunkel, en safir och en kalcedon;
\par 12 i tredje raden en hyacint, en agat och en ametist;
\par 13 i fjärde raden en krysolit, en onyx och en jaspis. Med flätverk av guld blevo de omgivna i sina infattningar.
\par 14 Stenarna voro tolv, efter Israels söners namn, en för vart namn; var sten bar namnet på en av de tolv stammarna, inristat på samma sätt som man graverar signetringar.
\par 15 Och man gjorde till bröstskölden kedjor i virat arbete, såsom man gör snodder, av rent guld.
\par 16 Man gjorde vidare två flätverk av guld och två ringar av guld och satte dessa båda ringar i två av bröstsköldens hörn.
\par 17 Och man fäste de båda guldsnodderna vid de båda ringarna, i bröstsköldens hörn.
\par 18 Och de två snoddernas båda andra ändar fäste man vid de två flätverken och fäste dem så vid efodens axelstycken på dess framsida.
\par 19 Och man gjorde två andra ringar av guld och satte dem i bröstsköldens båda andra hörn, vid den kant därpå, som var vänd inåt mot efoden.
\par 20 Och ytterligare gjorde man två ringar av guld och fäste dem vid efodens båda axelstycken, nedtill på dess framsida, där den fästes ihop, ovanför efodens skärp.
\par 21 Och man knöt fast bröstskölden med ett mörkblått snöre, som gick från dess ringar in i efodens ringar, så att den satt ovanför efodens skärp, på det att bröstskölden icke skulle lossna från efoden, allt såsom HERREN hade bjudit Mose.
\par 22 Efodkåpan gjorde man av vävt tyg, helt och hållet mörkblått.
\par 23 Och mitt på kåpan gjordes en öppning, lik öppningen på en pansarskjorta; öppningen omgavs nämligen med en kant, för att den icke skulle slitas sönder.
\par 24 Och på kåpans nedre fåll satte man granatäpplen, gjorda av mörkblått, purpurrött och rosenrött tvinnat garn.
\par 25 Och man gjorde bjällror av rent guld och satte dessa bjällror mellan granatäpplena runt omkring fållen på kåpan, mellan granatäpplena:
\par 26 en bjällra och så ett granatäpple, sedan en bjällra och så åter ett granatäpple, runt omkring fållen på kåpan, att bäras vid tjänstgöringen, såsom HERREN hade bjudit Mose.
\par 27 Och man gjorde åt Aron och hans söner livklädnaderna av vitt garn, i vävt arbete,
\par 28 huvudbindeln av vitt garn, högtidshuvorna av vitt garn och linnebenkläderna av tvinnat vitt garn,
\par 29 äntligen bältet av tvinnat vitt garn och av mörkblått, purpurrött och rosenrött garn, i brokig vävnad, allt såsom HERREN hade bjudit Mose.
\par 30 Och man gjorde plåten till det heliga diademet av rent guld, och på den skrev man, såsom man graverar signetringar: "Helgad åt HERREN."
\par 31 Och man fäste vid den ett mörkblått snöre och satte den ovanpå huvudbindeln, allt såsom HERREN hade bjudit Mose.
\par 32 Så blev då allt arbetet på uppenbarelsetältets tabernakel fullbordat. Israels barn utförde det; de gjorde i alla stycken såsom HERREN hade bjudit Mose.
\par 33 Och de förde fram till Mose tabernaklet, dess täckelse och alla dess tillbehör, dess häktor, bräder, tvärstänger, stol par och fotstycken,
\par 34 överdraget av rödfärgade vädurskinn och överdraget av tahasskinn och den förlåt som skulle hänga framför arken,
\par 35 vidare vittnesbördets ark med dess stänger, så ock nådastolen,
\par 36 bordet med alla dess tillbehör och skådebröden,
\par 37 den gyllene ljusstaken, lamporna som skulle sättas på den och alla dess andra tillbehör, oljan till ljusstaken,
\par 38 det gyllene altaret, smörjelseoljan och den välluktande rökelsen, förhänget för ingången till tältet,
\par 39 kopparaltaret med tillhörande koppargaller, dess stänger och alla dess tillbehör, bäckenet med dess fotställning.
\par 40 omhängena till förgården, dess stolpar och fotstycken, förhänget för porten till förgården, dess streck och pluggar, alla redskap till arbetet vid uppenbarelsetältets tabernakel
\par 41 äntligen de stickade kläderna till tjänsten i helgedomen och prästen Arons andra heliga kläder, så ock hans söners prästkläder.
\par 42 Såsom HERREN hade bjudit Mose så hade Israels barn i alla stycken gjort allt arbete.
\par 43 Och Mose besåg allt arbetet och fann att de hade utfört det, att de hade gjort såsom HERREN hade bjudit. Och Mose välsignade dem

\chapter{40}

\par 1 Och HERREN talade till Mose och sade:
\par 2 "När den första månaden ingår, skall du på första dagen i månaden uppsätta uppenbarelsetältets tabernakel.
\par 3 Och du skall däri sätta vittnesbördets ark och hänga förlåten framför arken.
\par 4 Och du skall föra bordet ditin och lägga upp på detta vad där skall vara upplagt; och du skall föra ditin ljusstaken och sätta upp lamporna på den.
\par 5 Och du skall ställa det gyllene rökelsealtaret framför vittnesbördets ark; och du skall sätta upp förhänget för ingången till tabernaklet.
\par 6 Och brännoffersaltaret skall du ställa framför ingången till uppenbarelsetältets tabernakel.
\par 7 Och du skall ställa bäckenet mellan uppenbarelsetältet och altaret och gjuta vatten däri.
\par 8 Och du skall sätta upp förgårdshägnaden runt omkring och hänga upp förhänget för porten till förgården.
\par 9 Och du skall taga smörjelseoljan och smörja tabernaklet och allt vad däri är och helga det jämte alla dess tillbehör, så att det bliver heligt.
\par 10 Du skall ock smörja brännoffersaltaret jämte alla dess tillbehör och helga altaret; så bliver altaret högheligt.
\par 11 Du skall ock smörja bäckenet jämte dess fotställning och helga det.
\par 12 Därefter skall du föra Aron och hans söner fram till uppenbarelsetältets ingång och två dem med vatten.
\par 13 Och du skall sätta på Aron del heliga kläderna, och smörja honom och helga honom till att bliva präst åt mig
\par 14 Och du skall föra fram hans söner och sätta livklädnader på dem.
\par 15 Och du skall smörja dem, såsom du smorde deras fader, till att bliva präster åt mig. Så skall denna deras smörjelse bliva för dem en invigning till ett evärdligt prästadöme, släkte efter släkte.
\par 16 Och Mose gjorde detta; han gjorde i alla stycken såsom HERREN hade bjudit honom.
\par 17 Så blev då tabernaklet uppsatt i första månaden av andra året, på första dagen i månaden.
\par 18 Då satte Mose upp tabernaklet. Han lade ut dess fotstycken, ställde fast dess bräder, sköt in dess tvärstänger och satte upp dess stolpar.
\par 19 Och han bredde täckelset över tabernaklet och lade ovanpå täckelset dess överdrag allt såsom HERREN hade bjudit Mose.
\par 20 Och han tog vittnesbördet och lade det in i arken och satte stängerna på arken; och han satte nådastolen ovanpå arken.
\par 21 Sedan förde han arken in i tabernaklet och satte upp förlåten som skulle hänga framför arken, och hängde så för vittnesbördets ark, allt såsom HERREN hade bjudit Mose.
\par 22 Och han satte bordet i uppenbarelsetältet, vid tabernaklets norra sida, utanför förlåten,
\par 23 och lade upp på detta de bröd som skulle vara upplagda inför HERRENS ansikte, allt såsom HERREN hade bjudit Mose.
\par 24 Och han ställde ljusstaken in i uppenbarelsetältet, mitt emot bordet, på tabernaklets södra sida,
\par 25 och satte upp lamporna inför HERRENS ansikte, såsom HERREN hade bjudit Mose.
\par 26 Och han ställde det gyllene altaret in i uppenbarelsetältet, framför förlåten,
\par 27 och antände välluktande rökelse därpå, såsom HERREN hade bjudit Mose.
\par 28 Och han satte upp förhänget för ingången till tabernaklet.
\par 29 Och brännoffersaltaret ställde han vid ingången till uppenbarelsetältets tabernakel och offrade brännoffer och spisoffer därpå, såsom HERREN hade bjudit Mose.
\par 30 Och han ställde bäckenet mellan uppenbarelsetältet och altaret och göt vatten till tvagning däri.
\par 31 Och Mose och Aron och hans söner tvådde sedermera sina händer och fötter med vatten därur;
\par 32 så ofta de skulle gå in i uppenbarelsetältet eller träda fram till altaret, tvådde de sig, såsom HERREN hade bjudit Mose.
\par 33 Och han satte upp förgårdshägnaden runt omkring tabernaklet och altaret, och hängde upp förhänget för porten till förgården. Så fullbordade Mose allt arbetet.
\par 34 Då övertäckte molnskyn uppenbarelsetältet, och HERRENS härlighet uppfyllde tabernaklet;
\par 35 och Mose kunde icke gå in i uppenbarelsetältet, eftersom molnskyn vilade däröver och HERRENS härlighet uppfyllde tabernaklet.
\par 36 Och så ofta molnskyn höjde sig från tabernaklet, bröto Israels barn upp; så gjorde de under hela sin vandring.
\par 37 Men så länge molnskyn icke höjde sig, bröto de icke upp, utan stannade ända till den dag då den åter höjde sig.
\par 38 Ty HERRENS molnsky vilade om dagen över tabernaklet, och om natten var eld i den; så var det inför alla Israels barns ögon under hela deras vandring.


\end{document}