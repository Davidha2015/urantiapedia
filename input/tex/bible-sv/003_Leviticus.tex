\begin{document}

\title{Leviticus}

Lev 1:1  Och HERREN kallade på Mose och talade till honom ur uppenbarelsetältet och sade:
Lev 1:2  Tala till Israels barn och säg till dem: När någon bland eder vill bära fram ett offer åt HERREN, skolen I taga edert offer av boskapen, antingen av fäkreaturen eller av småboskapen.
Lev 1:3  Om han vill bära fram ett brännoffer av fäkreaturen, så skall han därtill taga ett felfritt djur av hankön och föra det fram till uppenbarelsetältets ingång, för att han må bliva välbehaglig inför HERRENS ansikte.
Lev 1:4  Och han skall lägga sin hand på brännoffersdjurets huvud; så bliver det välbehagligt, och försoning bringas för honom.
Lev 1:5  Och han skall slakta ungtjuren inför HERRENS ansikte; och Arons söner, prästerna, skola bära fram blodet, och de skola stänka blodet runt omkring på det altare som står vid ingången till uppenbarelsetältet.
Lev 1:6  Och han skall draga av huden på brännoffersdjuret och dela det i dess stycken.
Lev 1:7  Och prästen Arons söner skola göra upp eld på altaret och lägga ved på elden.
Lev 1:8  Och Arons söner, prästerna, skola lägga styckena, huvudet och istret ovanpå veden som ligger på altarets eld.
Lev 1:9  Men inälvorna och fötterna skola tvås i vatten. Och prästen skall förbränna alltsammans på altaret: ett brännoffer, ett eldsoffer till en välbehaglig lukt för HERREN.
Lev 1:10  Men om han vill bära fram ett brännoffer av småboskapen, vare sig av fåren eller av getterna, så skall han därtill taga ett felfritt djur av hankön.
Lev 1:11  Och han skall slakta det vid sidan av altaret, norrut, inför HERRENS ansikte, och Arons söner, prästerna, skola stänka dess blod på altaret runt omkring.
Lev 1:12  Och han skall dela det i dess stycken och frånskilja dess huvud och ister; och prästen skall lägga detta ovanpå veden som ligger på altarets eld.
Lev 1:13  Men inälvorna och fötterna skola tvås i vatten. Och prästen skall offra alltsammans och förbränna det på altaret; det är ett brännoffer, ett eldsoffer till en välbehaglig lukt för HERREN.
Lev 1:14  Men om han vill bära fram åt HERREN ett brännoffer av fåglar, så skall han taga sitt offer av turturduvor eller av unga duvor.
Lev 1:15  Och prästen skall bära fram djuret till altaret och vrida huvudet av det och förbränna det på altaret. Och dess blod skall utkramas på altarets vägg.
Lev 1:16  Men dess kräva med orenligheten däri skall han taga ut, och han skall kasta den vid sidan av altaret, österut, på askhögen.
Lev 1:17  Och han skall fläka upp det invid vingarna, dock utan att frånskilja dessa; och prästen skall förbränna det på altaret, ovanpå veden som ligger på elden. Det är ett brännoffer, ett eldsoffer till en välbehaglig lukt för HERREN.
Lev 2:1  Och när någon vill bära fram ett spisoffer åt HERREN skall hans offer vara av fint mjöl, och han skall gjuta olja därpå och lägga rökelse därpå.
Lev 2:2  Och han skall bära det fram till Arons söner, prästerna; och prästen skall taga en handfull därav, nämligen av mjölet och oljan, därtill all rökelsen, och skall på altaret förbränna detta, som utgör själva altaroffret: ett eldsoffer till en välbehaglig lukt för HERREN.
Lev 2:3  Och det som är över av spisoffret skall tillhöra Aron och hans söner. Bland HERRENS eldsoffer är det högheligt.
Lev 2:4  Men när du vill bära fram ett spisoffer av det som bakas i ugn, skall det vara av fint mjöl, osyrade kakor, begjutna med olja, och osyrade tunnkakor, smorda med olja.
Lev 2:5  Och om ditt offer är ett spisoffer som tillredes på plåt, så skall det vara av fint mjöl, begjutet med olja, osyrat.
Lev 2:6  Du skall bryta sönder det i stycken och gjuta olja därpå. Det är ett spisoffer.
Lev 2:7  Och om ditt offer är ett spisoffer som tillredes i panna, så skall det tillredas av fint mjöl med olja.
Lev 2:8  Det spisoffer som är tillrett på något av dessa sätt skall du föra fram till HERREN; det skall bäras fram till prästen, och han skall hava det fram till altaret.
Lev 2:9  Och prästen skall av spisoffret taga den del som utgör själva altaroffret och förbränna den på altaret: ett eldsoffer till en välbehaglig lukt för HERREN.
Lev 2:10  Och det som är över av spisoffret skall tillhöra Aron och hans söner. Bland HERRENS eldsoffer är det högheligt.
Lev 2:11  Intet spisoffer som I viljen bära fram åt HERREN skall vara syrat, ty varken av surdeg eller av honung skolen I förbränna något såsom eldsoffer åt HERREN.
Lev 2:12  Såsom förstlingsoffer mån I bära fram sådant åt HERREN, men på altaret må det icke komma för att vara en välbehaglig lukt.
Lev 2:13  Och alla dina spisoffer skall du beströ med salt; du må icke låta din Guds förbunds salt fattas på ditt spisoffer. Till alla dina offer skall du offra salt.
Lev 2:14  Men om du vill bära fram åt HERREN ett spisoffer av förstlingsfrukter, skall du såsom ett sådant spisoffer av dina förstlingsfrukter böra fram ax, rostade vid eld, sönderstötta, av grönskuren säd.
Lev 2:15  Och du skall gjuta olja därpå och lägga rökelse därpå. Det är ett spisoffer.
Lev 2:16  Och prästen skall förbränna den del av de sönderstötta axen och av oljan, som utgör själva altaroffret, jämte all rökelsen därpå: ett eldsoffer åt Herren.
Lev 3:1  Och om någon vill bära fram ett tackoffer, och han vill taga sitt offer av fäkreaturen, så skall han ställa fram inför HERRENS ansikte ett felfritt djur, antingen av hankön eller av honkön.
Lev 3:2  Och han skall lägga sin hand på sitt offerdjurs huvud och sedan slakta det vid ingången till uppenbarelsetältet; och Arons söner, prästerna, skola stänka blodet på altaret runt omkring.
Lev 3:3  Och av tackoffret skall han såsom eldsoffer åt Herren bära fram det fett som omsluter inälvorna, och allt det fett som sitter på inälvorna,
Lev 3:4  och båda njurarna med det fett som sitter på dem invid länderna, så ock leverfettet, vilket han skall frånskilja invid njurarna.
Lev 3:5  Och Arons söner skola förbränna det på altaret, ovanpå brännoffret, på veden som ligger på elden: ett eldsoffer till en välbehaglig lukt för Herren.
Lev 3:6  Men om någon vill bära fram åt HERREN ett tackoffer av småboskapen, så skall han därtill taga ett felfritt djur, av hankön eller av honkön.
Lev 3:7  Om det är ett får som han vill offra, så skall han ställa fram det inför HERRENS ansikte.
Lev 3:8  Och han skall lägga sin hand på sitt offerdjurs huvud och sedan slakta det framför uppenbarelsetältet; och Arons söner skola stänka dess blod på altaret runt omkring.
Lev 3:9  Och av tackoffersdjuret skall han såsom eldsoffer åt HERREN offra dess fett, hela svansen, frånskild invid ryggraden, och det fett som omsluter inälvorna, och allt det fett som sitter på inälvorna,
Lev 3:10  och båda njurarna med det fett som sitter på dem invid länderna, så ock leverfettet, vilket han skall frånskilja invid njurarna.
Lev 3:11  Och prästen skall förbränna det på altaret: en eldsoffersspis åt HERREN.
Lev 3:12  Likaledes, om någon vill offra en get, så skall han ställa fram denna inför HERRENS ansikte.
Lev 3:13  Och han skall lägga sin hand på dess huvud och sedan slakta den framför uppenbarelsetältet; och Arons söner skola stänka dess blod på altaret runt omkring.
Lev 3:14  Och han skall därav såsom eldsoffer åt HERREN offra det fett som omsluter inälvorna, och allt det fett som sitter på inälvorna,
Lev 3:15  och båda njurarna med det fett som sitter på dem invid länderna, så ock leverfettet, vilket han skall frånskilja invid njurarna.
Lev 3:16  Och prästen skall förbränna detta på altaret: en eldsoffersspis, till en välbehaglig lukt. Allt fettet skall tillhöra HERREN.
Lev 3:17  Detta skall vara en evärdlig stadga för eder från släkte till släkte, var I än ären bosatta: intet fett och intet blod skolen I förtära.
Lev 4:1  Och HERREN talade till Mose och sade:
Lev 4:2  Tala till Israels barn och säg: Om någon ouppsåtligen syndar mot något HERRENS bud genom vilket något förbjudes, och han alltså gör något som är förbjudet, så gäller följande:
Lev 4:3  Om det är den smorde prästen som har syndat och därvid dragit skuld över folket, så skall han för den synd han har begått offra en felfri ungtjur åt HERREN till syndoffer.
Lev 4:4  Och han skall föra tjuren fram inför HERRENS ansikte, till uppenbarelsetältets ingång. Och han skall lägga sin hand på tjurens huvud och sedan slakta tjuren inför HERRENS ansikte.
Lev 4:5  Och den smorde prästen skall taga något av tjurens blod och bära det in i uppenbarelsetältet,
Lev 4:6  och prästen skall doppa sitt finger i blodet och stänka blodet sju gånger inför HERRENS ansikte, vid förlåten till helgedomen.
Lev 4:7  Därefter skall prästen med blodet bestryka hornen på den välluktande rökelsens altare, som står inför HERRENS ansikte i uppenbarelsetältet; men allt det övriga blodet av tjuren skall han gjuta ut vid foten av brännoffersaltaret, som står vid ingången till uppenbarelsetältet.
Lev 4:8  Och allt syndofferstjurens fett skall han taga ut ur honom - det fett som omsluter inälvorna, och allt det fett som sitter på inälvorna,
Lev 4:9  och båda njurarna med det fett som sitter på dem invid länderna, så ock leverfettet, vilket han skall frånskilja invid njurarna -
Lev 4:10  på samma sätt som detta tages ut ur tackofferstjuren; och prästen skall förbränna det på brännoffersaltaret.
Lev 4:11  Men tjurens hud och allt hans kött jämte hans huvud och hans fötter hans inälvor och hans orenlighet,
Lev 4:12  korteligen, allt det övriga av tjuren, skall han föra bort utanför lägret till en ren plats, där man slår ut askan, och bränna upp det på ved i eld; på den plats där man slår ut askan skall det brännas upp.
Lev 4:13  Och om Israels hela menighet begår synd ouppsåtligen, och utan att församlingen märker det, i det att de bryta mot något Herrens bud genom vilket något förbjudes och så ådraga sig skuld,
Lev 4:14  och den synd de hava begått sedan bliver känd, så skall församlingen offra en ungtjur till syndoffer. De skola föra honom fram inför uppenbarelsetältet;
Lev 4:15  och de äldste i menigheten skola lägga sina händer på tjurens huvud inför Herrens ansikte, och sedan skall man slakta tjuren inför HERRENS ansikte.
Lev 4:16  Och den smorde prästen skall bära något av tjurens blod in i uppenbarelsetältet,
Lev 4:17  och prästen skall doppa sitt finger i blodet och stänka sju gånger inför HERRENS ansikte, vid förlåten.
Lev 4:18  Därefter skall han med blodet bestryka hornen på det altare som står inför HERRENS ansikte i uppenbarelsetältet; men allt det övriga blodet skall han gjuta ut vid foten av brännoffersaltaret, som står vid ingången till uppenbarelsetältet.
Lev 4:19  Och allt tjurens fett skall han taga ut ur honom och förbränna det på altaret.
Lev 4:20  Så skall han göra med tjuren; såsom han skulle göra med den förra syndofferstjuren, så skall han göra med denna. När så prästen bringar försoning för dem, då bliver dem förlåtet.
Lev 4:21  Och han skall föra ut tjuren utanför lägret och bränna upp honom, såsom han skulle göra med den förra tjuren. Detta är syndoffret för församlingen.
Lev 4:22  Om en hövding syndar, i det att han ouppsåtligen bryter mot något HERRENS, sin Guds, bud genom vilket något förbjudes, och han själv märker att han har ådragit sig skuld,
Lev 4:23  eller av någon får veta vilken synd han har begått, så skall han såsom sitt offer föra fram en bock, ett felfritt djur av hankön.
Lev 4:24  Och han skall lägga sin hand på bockens huvud och sedan slakta honom på samma plats där man slaktar brännoffret, inför HERRENS ansikte. Det är ett syndoffer.
Lev 4:25  Och prästen skall taga något av syndoffrets blod på sitt finger och stryka på brännoffersaltarets horn; men det övriga blodet skall han gjuta ut vid foten av brännoffersaltaret.
Lev 4:26  Och allt fettet skall han förbränna på altaret, såsom det sker med tackoffersdjurets fett. När så prästen bringar försoning för honom, till rening från hans synd, då bliver honom förlåtet.
Lev 4:27  Och om någon av det meniga folket syndar ouppsåtligen, därigenom att han bryter mot något HERRENS bud genom vilket något förbjudes, och han själv märker att han har ådragit sig skuld,
Lev 4:28  eller av någon får veta vilken synd han har begått, så skall han, såsom sitt offer för den synd han har begått, föra fram en felfri get, ett djur av honkön.
Lev 4:29  Och han skall lägga sin hand på syndoffersdjurets huvud och sedan slakta syndoffersdjuret på den plats där brännoffersdjuren slaktas.
Lev 4:30  Och prästen skall taga något av blodet på sitt finger och stryka det på brännoffersaltarets horn; men allt det övriga blodet skall han gjuta ut vid foten av altaret.
Lev 4:31  Och allt fettet skall han taga ut, på samma sätt som fettet tages ut ur tackoffersdjuret, och prästen skall förbränna det på altaret, till en välbehaglig lukt för HERREN. När så prästen bringar försoning för honom, då bliver honom förlåtet.
Lev 4:32  Men om någon vill offra ett lamm till syndoffer, så skall han föra fram ett felfritt djur av honkön.
Lev 4:33  Och han skall lägga sin hand på syndoffersdjurets huvud och sedan slakta det till syndoffer på samma plats där man slaktar brännoffersdjuren.
Lev 4:34  Och prästen skall taga något av syndoffrets blod på sitt finger och stryka på brännoffersaltarets horn; men allt det övriga blodet skall han gjuta ut vid foten av altaret.
Lev 4:35  Och allt fettet skall han taga ut, på samma sätt som fettet tages ut ur tackoffersfåret, och prästen skall förbränna det på altaret, ovanpå Herrens eldsoffer. När så prästen för honom bringar försoning för den synd han har begått, då bliver honom förlåtet.
Lev 5:1  Och om någon syndar, i det att han, när han hör edsförpliktelsen och kan vittna om något, vare sig han har sett det eller eljest förnummit det, likväl icke yppar detta och han sålunda bär på missgärning;
Lev 5:2  eller om någon, utan att märka det, kommer vid något orent - vare sig den döda kroppen av ett orent vilddjur, eller den döda kroppen av ett orent boskapsdjur, eller den döda kroppen av något slags orent smådjur - och han så bliver oren och ådrager sig skuld;
Lev 5:3  eller om han, utan att märka det, kommer vid en människas orenhet, det må nu vara vad som helst varigenom hon kan vara oren, och han sedan får veta det och han så ådrager sig skuld;
Lev 5:4  eller om någon, utan att märka det, svär i obetänksamhet med sina läppar något, vare sig ont eller gott - det må nu vara vad som helst som man kan svärja i obetänksamhet - och sedan kommer till insikt därom och han så ådrager sig skuld i något av dessa stycken:
Lev 5:5  så skall han, när han har ådragit sig skuld i något av dessa stycken, bekänna det vari han har syndat
Lev 5:6  och såsom bot för den synd han har begått föra fram åt HERREN ett hondjur av småboskapen, antingen en tacka eller en get, till syndoffer. Och prästen skall bringa försoning för honom, till rening från hans synd.
Lev 5:7  Men om han icke förmår bekosta ett sådant djur, så skall han såsom bot för vad han har syndat bära fram åt Herren två turturduvor eller två unga duvor, en till syndoffer och en till brännoffer.
Lev 5:8  Dem skall han bära fram till prästen, och denne skall först offra den som är avsedd till syndoffer. Han skall vrida huvudet av den invid halsen, dock utan att frånskilja det.
Lev 5:9  Och han skall stänka något av syndoffrets blod på altarets vägg; men det övriga blodet skall utkramas vid foten av altaret. Det är ett syndoffer.
Lev 5:10  Och den andra skall han offra till ett brännoffer, på föreskrivet sätt. När så prästen bringar försoning för honom, till rening från den synd han har begått, då bliver honom förlåtet.
Lev 5:11  Men om han icke kan anskaffa två turturduvor eller två unga duvor, så skall han såsom offer för vad han har syndat bära fram en tiondedels efa fint mjöl till syndoffer, men ingen olja skall han gjuta därpå och ingen rökelse lägga därpå, ty det är ett syndoffer.
Lev 5:12  Och han skall bära det fram till prästen, och prästen skall taga en handfull därav, det som utgör själva altaroffret, och förbränna det på altaret, ovanpå HERRENS eldsoffer. Det är ett syndoffer.
Lev 5:13  När så prästen för honom bringar försoning för den synd han har begått i något av dessa stycken, då bliver honom förlåtet. Och det övriga skall tillhöra prästen, likasom vid spisoffret.
Lev 5:14  Och HERREN talade till Mose och sade:
Lev 5:15  Om någon begår en orättrådighet, i det att han ouppsåtligen försyndar sig genom att undanhålla något som är helgat åt Herren, så skall han såsom bot föra fram åt HERREN till skuldoffer av småboskapen en felfri vädur, efter det värde du bestämmer i silver, till ett visst belopp siklar efter helgedomssikelns vikt.
Lev 5:16  Och han skall giva ersättning för det som han har undanhållit av det helgade och skall lägga femtedelen av värdet därtill; och detta skall han giva åt prästen. När så prästen bringar försoning för honom genom skuldoffersväduren, då bliver honom förlåtet.
Lev 5:17  Och om någon, utan att veta det, syndar, i det att han bryter mot något HERRENS bud genom vilket något förbjudes, och han så ådrager sig skuld och bär på missgärning,
Lev 5:18  så skall han såsom skuldoffer föra fram till prästen av småboskapen en felfri vädur, efter det värde du bestämmer. När så prästen för honom bringar försoning för den synd han har begått ouppsåtligen och utan att veta det, då bliver honom förlåtet.
Lev 5:19  Det är ett skuldoffer, ty han har ådragit sig skuld inför HERREN.
Lev 6:1  Och HERREN talade till Mose och sade:
Lev 6:2  Om någon syndar och begår en orättrådighet mot HERREN, i det att han inför sin nästa nekar angående något som denne har ombetrott honom eller överlämnat i hans hand, eller angående något som han med våld har tagit; eller i det att han med orätt avhänder sin nästa något;
Lev 6:3  eller i det att han, när han har hittat något borttappat, nekar därtill och svär falskt i någon sak, vad det nu må vara, vari en människa kan försynda sig:
Lev 6:4  så skall den som så har syndat Och därmed ådragit sig skuld återställa vad han med våld har tagit eller med orätt tillägnat sig, eller det som har varit honom ombetrott, eller det borttappade som han har hittat,
Lev 6:5  eller vad det må vara, varom han har svurit falskt; han skall ersätta det till dess fulla belopp och lägga femtedelen av värdet därtill. Han skall giva det åt ägaren samma dag han bär fram sitt skuldoffer.
Lev 6:6  Ty sitt skuldoffer skall han föra fram inför HERREN; en felfri vädur av småboskapen, efter det värde du bestämmer, skall han såsom sitt skuldoffer föra fram till prästen.
Lev 6:7  När så prästen bringar försoning för honom inför HERRENS ansikte, då bliver honom förlåtet, vad han än må hava gjort, som har dragit skuld över honom.
Lev 6:8  Och Herren talade till Mose och sade:
Lev 6:9  Bjud Aron och hans söner och säg: Detta är lagen om brännoffret: Brännoffret skall ligga på altarets härd hela natten intill morgonen, och elden på altaret skall därigenom hållas brinnande.
Lev 6:10  Och prästen skall ikläda sig sin livrock av linne och ikläda sig benkläder av linne, för att de må skyla hans kött; därefter skall han taga bort askan vartill elden har förbränt brännoffret på altaret, och lägga den vid sidan av altaret.
Lev 6:11  Sedan skall han taga av sig sina kläder och ikläda sig andra kläder och föra askan bort utanför lägret till en ren plats.
Lev 6:12  Men elden på altaret skall hållas brinnande och får icke slockna; prästen skall var morgon antända ny ved därpå. Och han skall lägga brännoffret därpå och förbränna fettstyckena av tackoffret därpå.
Lev 6:13  Elden skall beständigt hållas brinnande på altaret; den får icke slockna.
Lev 6:14  Och detta är lagen om spisoffret: Arons söner skola bära fram det inför HERRENS ansikte, till altaret.
Lev 6:15  Och prästen skall taga en handfull därav, nämligen av det fina mjölet som hör till spisoffret, och av oljan, därtill all rökelsen som ligger på spisoffret, och detta, som utgör själva altaroffret, skall han förbränna på altaret, till en välbehaglig lukt för HERREN.
Lev 6:16  Och det som är över därav skola Aron och hans söner äta. Osyrat skall det ätas på en helig plats; i förgården till uppenbarelsetältet skola de äta det.
Lev 6:17  Det skall icke bakas med surdeg. Detta är deras del, det som jag har givit dem av mina eldsoffer. Det är högheligt likasom syndoffret och skuldoffret.
Lev 6:18  Allt mankön bland Arons barn må äta det. Det skall vara deras evärdliga rätt av HERRENS eldsoffer, från släkte till släkte. Var och en som kommer därvid bliver helig.
Lev 6:19  Och HERREN talade till Mose och sade:
Lev 6:20  Detta är det offer som Aron och hans söner skola offra åt HERREN på den dag då någon av dem undfår smörjelsen: en tiondedels efa fint mjöl såsom det dagliga spisoffret, hälften om morgonen och hälften om aftonen.
Lev 6:21  På plåt skall det tillredas med olja, och du skall bära fram det hopknådat; och du skall offra det sönderdelat, såsom när man offrar ett spisoffer i stycken, till en välbehaglig lukt för HERREN.
Lev 6:22  Och den präst bland hans söner, som bliver smord i hans ställe, skall göra så. Detta skall vara en evärdlig stadga. Såsom ett heloffer skall det förbrännas åt HERREN.
Lev 6:23  En prästs spisoffer skall alltid vara ett heloffer; det får icke ätas.
Lev 6:24  Och HERREN talade till Mose och sade:
Lev 6:25  Tala till Aron och hans söner och säg: Detta är lagen om syndoffret: På samma plats där brännoffersdjuret slaktas skall ock syndoffersdjuret slaktas, inför HERRENS ansikte. Det är högheligt.
Lev 6:26  Den präst som offrar syndoffret skall äta det; på en helig plats skall det ätas, i förgården till uppenbarelsetältet.
Lev 6:27  Var och en som kommer vid köttet bliver helig. Och om något av blodet stänkes på någons kläder, så skall man avtvå det bestänkta stället på en helig plats.
Lev 6:28  Ett lerkärl vari kokningen har skett skall sönderslås; men har kokningen skett i ett kopparkärl, så skall detta skuras och sköljas med vatten.
Lev 6:29  Allt mankön bland prästerna må äta det. Det är högheligt.
Lev 6:30  Men intet syndoffer av vars blod något bäres in i uppenbarelsetältet till att bringa försoning i helgedomen får ätas; det skall brännas upp i eld.
Lev 7:1  Och detta är lagen om skuldoffret: Det är högheligt.
Lev 7:2  På samma plats där man slaktar brännoffersdjuret skall man slakta skuldoffersdjuret. Och man skall stänka dess blod på altaret runt omkring.
Lev 7:3  Och allt dess fett skall man offra, svansen och det fett som omsluter inälvorna,
Lev 7:4  och båda njurarna med det fett som sitter på dem invid länderna, så ock leverfettet, vilket man skall frånskilja invid njurarna.
Lev 7:5  Och prästen skall förbränna det på altaret till ett eldsoffer åt HERREN. Det är ett skuldoffer.
Lev 7:6  Allt mankön bland prästerna må äta det; på en helig plats skall det ätas; det är högheligt.
Lev 7:7  Vad som gäller om syndoffret skall ock gälla om skuldoffret; samma lag skall gälla för dem båda. Den präst som bringar försoning därmed, honom skall det tillhöra.
Lev 7:8  Och när en präst bär fram brännoffer for någon, skall huden av det framburna brännoffersdjuret tillhöra den prästen.
Lev 7:9  Och ett spisoffer som är bakat i ugn, eller som är tillrett i panna eller på plåt, skall alltid tillfalla den präst som bär fram det.
Lev 7:10  Men ett spisoffer som är begjutet med olja, eller som frambäres torrt, skall alltid tillfalla Arons söner gemensamt, den ene likaväl som den andre.
Lev 7:11  Och detta är lagen om tackoffret, när ett sådant bäres fram åt HERREN:
Lev 7:12  Om någon vill bära fram ett sådant till lovoffer, så skall han, förutom det till lovoffret hörande slaktdjuret, bära fram osyrade kakor, begjutna med olja, och osyrade tunnkakor, smorda med olja, och fint mjöl, hopknådat, i form av kakor, begjutna med olja.
Lev 7:13  Jämte kakor av syrat bröd skall han bära fram detta sitt offer, förutom det slaktdjur som hör till det tackoffer han bär fram såsom lov offer.
Lev 7:14  Av detta offer skall han bära fram en kaka av vart slag, såsom en gärd åt HERREN; den präst som stänker tackoffrets blod på altaret, honom skall den tillhöra.
Lev 7:15  Och köttet av det slaktdjur, som hör till det tackoffer som bäres fram såsom lovoffer, skall ätas samma dag det har offrats; intet därav må lämnas kvar till följande morgon.
Lev 7:16  Om däremot det slaktoffer som någon vill bära fram år ett löftesoffer eller ett frivilligt offer, så skall offerdjuret likaledes ätas samma dag det har offrats; dock må det som har blivit över därav ätas den följande dagen.
Lev 7:17  Bliver ändå något över av offerköttet, skall detta på tredje dagen brännas upp i eld.
Lev 7:18  Om någon på tredje dagen äter av tackoffersköttet, så bliver offret icke välbehagligt; honom som har burit fram det skall det då icke räknas till godo, det skall anses såsom en vederstygglighet. Den som äter därav kommer att bära på missgärning.
Lev 7:19  Ej heller må det kött ätas, som har kommit vid något orent, utan det skall brännas upp i eld. För övrigt må köttet ätas av var och en som är ren.
Lev 7:20  Men den som äter kött av HERRENS tackoffer, medan orenhet låder vid honom, han skall utrotas ur sin släkt.
Lev 7:21  Och om någon har kommit vid något orent - vare sig en människas orenhet, eller ett orent djur, eller vilken oren styggelse det vara må - och han likväl äter kött av HERRENS tackoffer, så skall han utrotas ur sin släkt.
Lev 7:22  Och HERREN talade till Mose och sade:
Lev 7:23  Tala till Israels barn och säg: Intet fett av fäkreatur, får eller getter skolen I äta.
Lev 7:24  Fettet av ett självdött eller ihjälrivet djur må eljest användas till alla slags behov, men äta det skolen I icke.
Lev 7:25  Ty var och en som äter fettet av något djur varav man bär fram eldsoffer åt HERREN, vem det vara må som äter därav, han skall utrotas ur sin släkt.
Lev 7:26  Och intet blod skolen I förtära varken av fåglar eller av boskap, var I än ären bosatta.
Lev 7:27  Var och en som förtär något blod, han skall utrotas ur sin släkt.
Lev 7:28  Och HERREN talade till Mose och sade:
Lev 7:29  Tala till Israels barn och säg: Den som vill offra ett tackoffer åt HERREN, han skall av detta sitt tackoffer bära fram åt HERREN den vederbörliga offergåvan.
Lev 7:30  Med egna händer skall han bära fram HERRENS eldsoffer; fettet jämte bringan skall han bära fram, bringan till att viftas såsom ett viftoffer inför HERRENS ansikte.
Lev 7:31  Och prästen skall förbränna fettet på altaret, men bringan skall tillhöra Aron och hans söner.
Lev 7:32  Också det högra lårstycket skolen I giva åt prästen, såsom en gärd av edra tackoffer.
Lev 7:33  Den bland Arons söner, som offrar tackoffrets blod och fettet, han skall hava det högra lårstycket till sin del.
Lev 7:34  Ty av Israels barns tackoffer tager jag viftoffersbringan och offergärdslåret och giver dem åt prästen Aron och åt hans söner till en evärdlig rätt av Israels barn.
Lev 7:35  Detta är Arons och hans söners ämbetslott av HERRENS eldsoffer, den lott som gavs dem den dag de fördes fram till att bliva HERRENS präster
Lev 7:36  vilken lott, efter HERRENS befallning på den dag då han smorde dem, skulle givas dem av Israels barn, till en evärdlig rätt, släkte efter släkte.
Lev 7:37  Detta är lagen om brännoffret, spisoffret, syndoffret, skuldoffret, handfyllningsoffret och tackoffret,
Lev 7:38  vilken HERREN på Sinai berg gav Mose, på den dag då han bjöd Israels barn att de skulle offra sina offer åt HERREN, i Sinais öken.
Lev 8:1  Och HERREN talade till Mose och sade:
Lev 8:2  "Tag Aron och hans söner jämte honom samt deras kläder och smörjelseoljan, så ock syndofferstjuren och de två vädurarna och korgen med de osyrade bröden.
Lev 8:3  Församla sedan hela menigheten vid ingången till uppenbarelsetältet."
Lev 8:4  Och Mose gjorde såsom HERREN hade bjudit honom, och menigheten församlade sig vid ingången till uppenbarelsetältet.
Lev 8:5  Och Mose sade till menigheten: "Detta är vad HERREN har bjudit mig att göra."
Lev 8:6  Och Mose förde fram Aron och hans söner och tvådde dem med vatten.
Lev 8:7  Och han satte livklädnaden på honom och omgjordade honom med bältet och klädde på honom kåpan och satte på honom efoden och omgjordade honom med efodens skärp och fäste därmed ihop alltsammans på honom.
Lev 8:8  Och han satte på honom bröstskölden och lade urim och tummim in i skölden.
Lev 8:9  Och han satte huvudbindeln på hans huvud och satte på huvudbindeln framtill den gyllene plåten, det heliga diademet, såsom HERREN hade bjudit Mose.
Lev 8:10  Och Mose tog smörjelseoljan och smorde tabernaklet och allt vad däri var och helgade allt;
Lev 8:11  och han stänkte därmed sju gånger på altaret och smorde altaret och alla dess tillbehör och bäckenet jämte dess fotställning, för att helga dem.
Lev 8:12  Och han göt smörjelseolja på Arons huvud och smorde honom för att helga honom.
Lev 8:13  Och Mose förde fram Arons söner och satte livklädnader på dem och omgjordade dem med bälten och band huvor på dem, såsom HERREN hade bjudit Mose.
Lev 8:14  Och han förde fram syndofferstjuren, och Aron och hans söner lade sina händer på syndofferstjurens huvud.
Lev 8:15  Sedan slaktades den, och Mose tog blodet och strök med sitt finger på altarets horn runt omkring och renade altaret; men det övriga blodet göt han ut vid foten av altaret och helgade detta och bragte försoning för det.
Lev 8:16  Och han tog allt det fett som satt på inälvorna, så ock leverfettet och båda njurarna med fettet på dem; och Mose förbrände det på altaret.
Lev 8:17  Men det övriga av tjuren, hans hud och kött och orenlighet, brände han upp i eld utanför lägret såsom HERREN hade bjudit Mose.
Lev 8:18  Och han förde fram brännoffersväduren, och Aron och hans söner lade sina händer på vädurens huvud.
Lev 8:19  Sedan slaktades den, och Mose stänkte blodet på altaret runt omkring;
Lev 8:20  men själva väduren delade han i dess stycken. Och Mose förbrände huvudet och styckena och istret;
Lev 8:21  inälvorna och fötterna tvådde han i vatten. Sedan förbrände Mose hela väduren på altaret. Det var ett brännoffer till en välbehaglig lukt, det var ett eldsoffer åt HERREN, Såsom HERREN hade bjudit Mose.
Lev 8:22  Därefter förde han fram den andra väduren, handfyllningsväduren, och Aron och hans söner lade sina händer på vädurens huvud.
Lev 8:23  Sedan slaktades den, och Mose tog av dess blod och beströk Arons högra örsnibb och tummen på hans högra hand och stortån på hans högra fot.
Lev 8:24  Därefter förde han fram Arons söner. Och Mose beströk med blodet deras högra örsnibb och tummen på deras högra hand och stortån på deras högra fot; men det övriga blodet stänkte Mose på altaret runt omkring.
Lev 8:25  Och han tog fettet, svansen och allt det fett som satt på inälvorna, så ock leverfettet och båda njurarna med fettet på dem, därtill det högra lårstycket.
Lev 8:26  Och ur korgen med de osyrade bröden, som stod inför HERRENS ansikte, tog han en osyrad kaka, en oljebrödskaka och en tunnkaka och lade detta på fettstyckena och det högra lårstycket.
Lev 8:27  Och alltsammans lade han på Arons och hans söners händer och viftade det såsom ett viftoffer inför HERRENS ansikte.
Lev 8:28  Sedan tog Mose det ur deras händer och förbrände det på altaret, ovanpå brännoffret. Det var ett handfyllningsoffer till en välbehaglig lukt, det var ett eldsoffer åt HERREN.
Lev 8:29  Och Mose tog bringan och viftade den såsom ett viftoffer inför HERRENS ansikte; av handfyllningsoffrets vädur fick Mose detta till sin del, såsom HERREN hade bjudit Mose.
Lev 8:30  Och Mose tog av smörjelseoljan och av blodet på altaret och stänkte på Aron - på hans kläder - och likaledes på hans söner och hans söners kläder; han helgade så Aron - hans kläder - och likaledes hans söner och hans söners kläder.
Lev 8:31  Och Mose sade till Aron och till hans söner: "Koken köttet vid ingången till uppenbarelsetältet, och äten det där jämte brödet som är i handfyllningskorgen, såsom jag har bjudit och sagt: Aron och hans söner skola äta det.
Lev 8:32  Men vad som bliver över av köttet; eller av brödet, det skolen I bränna upp i eld.
Lev 8:33  Och under sju dagar skolen I icke gå bort ifrån uppenbarelsetältets ingång, icke förrän edra handfyllningsdagar äro ute, ty sju dagar skall eder handfyllning vara.
Lev 8:34  Och HERREN har bjudit, att såsom det i dag har tillgått, så skall det ock sedan tillgå, på det att försoning må bringas för eder.
Lev 8:35  Vid ingången till uppenbarelsetältet skolen I stanna kvar i sju dygn, dag och natt, och I skolen iakttaga vad HERREN har bjudit eder iakttaga, på det att I icke mån dö; ty så är mig bjudet."
Lev 8:36  Och Aron och hans söner gjorde allt vad HERREN hade bjudit genom Mose.
Lev 9:1  Och på åttonde dagen kallade Mose till sig Aron och hans söner och de äldste i Israel.
Lev 9:2  Och han sade till Aron: "Tag dig en tjurkalv till syndoffer och en vädur till brännoffer, båda felfria, och för dem fram inför HERRENS ansikte.
Lev 9:3  Och tala till Israels barn och säg: Tagen en bock till syndoffer och en kalv och ett lamm, båda årsgamla och felfria, till brännoffer,
Lev 9:4  så ock en tjur och en vädur till tackoffer, att offra inför HERRENS ansikte, därtill ett spisoffer, begjutet med olja; ty i dag uppenbarar sig HERREN för eder."
Lev 9:5  Och de togo det som Mose hade givit dem befallning om och förde det fram inför uppenbarelsetältet; och hela menigheten trädde fram och ställde sig inför HERRENS ansikte.
Lev 9:6  Då sade Mose: "Detta är vad HERREN har bjudit eder göra; så skall HERRENS härlighet visa sig för eder."
Lev 9:7  Och Mose sade till Aron: "Träd fram till altaret och offra ditt syndoffer och ditt brännoffer, och bringa försoning för dig själv och folket; offra sedan folkets offer och bringa försoning för dem, såsom HERREN har bjudit."
Lev 9:8  Då trädde Aron fram till altaret och slaktade sin syndofferskalv.
Lev 9:9  Och Arons söner buro fram blodet till honom, och han doppade sitt finger i blodet och strök på altarets horn, men det övriga blodet göt han ut vid foten av altaret.
Lev 9:10  Och syndoffersdjurets fett, njurar och leverfett förbrände han på altaret, såsom HERREN hade bjudit Mose.
Lev 9:11  Men köttet och huden brände han upp i eld utanför lägret.
Lev 9:12  Sedan slaktade han brännoffersdjuret. Och Arons söner räckte honom blodet, och han stänkte det på altaret runt omkring.
Lev 9:13  Och de räckte honom brännoffersdjuret, delat i sina stycken, och dess huvud, och han förbrände det på altaret.
Lev 9:14  Och han tvådde inälvorna och fötterna och förbrände dem ovanpå brännoffret, på altaret.
Lev 9:15  Därefter förde han fram folkets offer. Han tog folkets syndoffersbock och slaktade honom och offrade honom till syndoffer, på samma sätt som det förra syndoffersdjuret.
Lev 9:16  Och han förde fram brännoffersdjuren och offrade dem på föreskrivet sätt.
Lev 9:17  Och han bar fram spisoffret och tog en handfull därav och förbrände detta på altaret, förutom morgonens brännoffer.
Lev 9:18  Sedan slaktade han tjuren och väduren, som voro folkets tackoffer. Och Arons söner räckte honom blodet, och han stänkte det på altaret runt omkring.
Lev 9:19  Och fettstyckena av tjuren, samt av väduren svansen och vad som omsluter inälvorna, så ock njurarna och leverfettet,
Lev 9:20  dessa fettstycken lade de på bringorna; och han förbrände fettstyckena på altaret.
Lev 9:21  Men bringorna och det högra lårstycket viftade Aron till ett viftoffer inför HERRENS ansikte, såsom Mose hade bjudit.
Lev 9:22  Och Aron lyfte upp sina händer över folket och välsignade det. Därefter steg han ned, sedan han hade offrat syndoffret, brännoffret och tackoffret.
Lev 9:23  Och Mose och Aron gingo in i uppenbarelsetältet; sedan gingo de åter ut och välsignade folket. Då visade sig HERRENS härlighet för allt folket.
Lev 9:24  Och eld gick ut från HERREN och förtärde brännoffret och fettstyckena på altaret. Och allt folket såg detta; då jublade de och föllo ned på sina ansikten.
Lev 10:1  Men Arons söner Nadab och Abihu togo var sitt fyrfat och lade eld i dem och strödde rökelse därpå och buro fram inför HERRENS ansikte främmande eld, annan eld än den han hade givit dem befallning om.
Lev 10:2  Då gick eld ut från HERREN och förtärde dem, så att de föllo döda ned inför HERRENS ansikte.
Lev 10:3  Och Mose sade till Aron: "Detta är vad HERREN har talat och sagt: På dem som stå mig nära vill jag bevisa mig helig, och inför allt folket bevisa mig härlig." Och Aron teg stilla.
Lev 10:4  Och Mose kallade till sig Misael och Elsafan, Arons farbroder Ussiels söner, och sade till dem: "Träden fram och bären edra fränder bort ifrån helgedomen och fören den utanför lägret."
Lev 10:5  Då trädde de fram och buro bort dem i deras livklädnader, utanför lägret, såsom Mose hade sagt.
Lev 10:6  Och Mose sade till Aron och till hans söner Eleasar och Itamar: "I skolen icke hava edert hår oordnat, ej heller riva sönder edra kläder, på det att I icke mån dö och draga förtörnelse över hela menigheten. Men edra bröder, hela Israels hus, må gråta över denna brand som HERREN har upptänt.
Lev 10:7  Och I skolen icke gå bort ifrån uppenbarelsetältets ingång, på det att I icke mån dö; ty HERRENS smörjelseolja är på eder." Och de gjorde såsom Mose hade sagt.
Lev 10:8  Och HERREN talade till Aron och sade:
Lev 10:9  "Varken du själv eller dina söner må dricka vin eller starka drycker, när I skolen gå in i uppenbarelsetältet, på det att I icke mån dö. Det skall vara en evärdlig stadga för eder från släkte till släkte.
Lev 10:10  I skolen skilja mellan heligt och oheligt, mellan orent och rent;
Lev 10:11  och I skolen lära Israels barn alla de stadgar som HERREN har kungjort för dem genom Mose."
Lev 10:12  Och Mose sade till Aron och till Eleasar och Itamar, hans kvarlevande söner: "Tagen det spisoffer som har blivit över av HERRENS eldsoffer, och äten det osyrat vid sidan av altaret, ty det är högheligt.
Lev 10:13  I skolen äta det på en helig plats; ty det är din och dina söners stadgade rätt av HERRENS eldsoffer; så är mig bjudet.
Lev 10:14  Och viftoffersbringan och offergärdslåret skola ätas av dig, och av dina söner och dina döttrar jämte dig, på en ren plats, ty de äro dig givna såsom din och dina söners stadgade rätt av Israels barns tackoffer.
Lev 10:15  Jämte eldsoffren - fettstyckena - skola offergärdslåret och viftoffersbringan bäras fram för att viftas såsom ett viftoffer inför HERRENS ansikte; och de skola såsom en evärdlig rätt tillhöra dig och dina söner jämte dig, såsom HERREN har bjudit."
Lev 10:16  Och Mose frågade efter syndoffersbocken, men den befanns vara uppbränd. Då förtörnades han på Eleasar och Itamar, Arons kvarlevande söner, och sade:
Lev 10:17  "Varför haven I icke ätit syndoffret på den heliga platsen? Det är ju högheligt. Och han har givit eder det, för att I skolen borttaga menighetens missgärning och bringa försoning för dem inför HERRENS ansikte.
Lev 10:18  Se, dess blod har icke blivit inburet i helgedomens inre; därför skullen I på heligt område hava ätit upp köttet, såsom jag hade bjudit."
Lev 10:19  Men Aron sade till Mose: "Se, de hava i dag offrat sitt syndoffer och sitt brännoffer inför HERRENS ansikte, och mig har vederfarits vad du vet. Om jag nu i dag åte syndofferskött, skulle detta vara HERREN välbehagligt?"
Lev 10:20  När Mose hörde detta, var han till freds.
Lev 11:1  Och HERREN talade till Mose och Aron och sade till dem:
Lev 11:2  Talen till Israels barn och sägen: Dessa äro de djur som I fån äta bland alla fyrfotadjur på jorden:
Lev 11:3  alla de fyrfotadjur som hava klövar och hava dem helkluvna, och som idissla, dem fån I äta.
Lev 11:4  Men dessa skolen I icke äta av de idisslande djuren och av dem som hava klövar: kamelen, ty han idisslar väl, men har icke klövar, han skall gälla för eder såsom oren; klippdassen, ty han idisslar väl
Lev 11:5  men har icke klövar, han skall gälla for eder såsom oren; haren, ty han idisslar väl,
Lev 11:6  men har icke klövar, han skall gälla för eder såsom oren; svinet,
Lev 11:7  ty det har väl klövar och har dem helkluvna, men det idisslar icke, det skall gälla för eder såsom orent.
Lev 11:8  Av dessa djurs kött skolen I icke äta, ej heller skolen I komma vid deras döda kroppar; de skola gälla för eder såsom orena.
Lev 11:9  Detta är vad I fån äta av allt det som lever i vattnet: allt det i vattnet, vare sig i sjöar eller i strömmar, som har fenor och fjäll, det fån I äta.
Lev 11:10  Men allt det i sjöar och strömmar, som icke har fenor och fjäll, bland allt det som rör sig i vattnet, bland alla levande varelser i vattnet, det skall vara en styggelse för eder.
Lev 11:11  Ja, de skola vara en styggelse för eder; av deras kött skolen I icke äta, och deras döda kroppar skolen I räkna såsom en styggelse.
Lev 11:12  Allt det i vattnet, som icke har fenor och fjäll, skall vara en styggelse för eder.
Lev 11:13  Och bland fåglarna skolen I räkna dessa såsom en styggelse, de skola icke ätas, de äro en styggelse: örnen, lammgamen, havsörnen,
Lev 11:14  gladan, falken med dess arter,
Lev 11:15  alla slags korpar efter deras arter,
Lev 11:16  strutsen, tahemasfågeln, fiskmåsen, höken med dess arter,
Lev 11:17  ugglan, dykfågeln, uven,
Lev 11:18  tinsemetfågeln, pelikanen,
Lev 11:19  asgamen, hägern, regnpiparen med dess arter, härfågeln och flädermusen.
Lev 11:20  Alla de flygande smådjur som gå på fyra fötter skola vara en styggelse för eder.
Lev 11:21  Av alla flygande smådjur, som gå på fyra fötter fån I allenast äta dem som ovanför sina fötter hava två ben att hoppa med på jorden.
Lev 11:22  Dessa fån I äta bland gräshopporna: arbe med dess arter, soleam med dess arter, hargol med dess arter och hagab med dess arter.
Lev 11:23  Men alla andra flygande smådjur som hava fyra fötter skola vara en styggelse för eder.
Lev 11:24  Genom följande djur ådragen I eder orenhet; var och en som kommer vid deras döda kroppar skall vara oren ända till aftonen,
Lev 11:25  och var och en som har burit bort någon sådan död kropp skall två sina kläder och vara oren ända till aftonen:
Lev 11:26  alla de fyrfotadjur som hava klövar, men icke helkluvna, och som icke idissla, de skola gälla för eder såsom orena. Var och en som kommer vid dem bliver oren.
Lev 11:27  Och alla slags fyrfotade djur som gå på tassar skola gälla för eder såsom orena. Var och en som kommer vid deras döda kroppar skall vara oren ända till aftonen.
Lev 11:28  Och den som har burit bort en sådan död kropp, han skall två sina kläder och vara oren ända till aftonen; de skola gälla för eder såsom orena.
Lev 11:29  Och bland de smådjur som röra sig på jorden skola dessa gälla för eder såsom orena: vesslan, jordråttan, ödlan med dess arter,
Lev 11:30  anakan, koadjuret, letaan, hometdjuret och kameleonten.
Lev 11:31  Dessa äro de som skola gälla för eder såsom orena bland alla smådjur. Var och en som kommer vid dem, sedan de äro döda, skall vara oren ända till aftonen.
Lev 11:32  Och allt varpå något sådant djur faller, sedan det är dött, bliver orent, vare sig det är något slags träkärl, eller det är kläder, eller något av skinn, eller en säck, eller vilken annan sak det vara må, som användes till något behov. Man skall lägga det i vatten, och det skall vara orent ända till aftonen; så bliver det rent.
Lev 11:33  Och om något sådant faller i något slags lerkärl, så bliver allt som är i detta orent, och kärlet skolen I slå sönder.
Lev 11:34  Allt slags mat däri, allt som man äter tillrett med vatten, det bliver orent; och allt slags dryck i något slags kärl, allt som man dricker, det bliver orent därav.
Lev 11:35  Och allt varpå någon sådan död kropp faller bliver orent. Är det en ugn eller en härd, skall den förstöras, ty den bliver oren. Och den skall gälla för eder såsom oren.
Lev 11:36  Men en källa eller en brunn, en plats dit vatten samlar sig, skall förbliva ren; men kommer någon vid själva den döda kroppen, bliver han oren.
Lev 11:37  Och om en sådan död kropp faller på något slags utsädeskorn, något man sår, då förbliver detta rent.
Lev 11:38  Men om vatten har kommit på säden och någon sådan död kropp sedan faller därpå, så skall den gälla för eder såsom oren.
Lev 11:39  Och om något fyrfotadjur som får ätas av eder dör, så skall den som kommer vid dess döda kropp vara oren ända till aftonen.
Lev 11:40  Och den som äter kött av en sådan död kropp, han skall två sina kläder och vara oren ända till aftonen. Och den som har burit bort någon sådan död kropp, han skall två sina kläder och vara oren ända till aftonen.
Lev 11:41  Och alla slags smådjur som röra sig på jorden äro en styggelse; de skola icke ätas.
Lev 11:42  Varken av det som går på buken eller av det som går på fyra eller flera fötter, bland alla de smådjur som röra sig på jorden, skolen I äta något, ty de äro en styggelse.
Lev 11:43  Gören eder icke själva till en styggelse genom något sådant djur, och ådragen eder icke orenhet genom sådana, så att I bliven orenade genom dem.
Lev 11:44  Ty jag är HERREN, eder Gud; och I skolen hålla eder heliga och vara heliga, ty jag är helig. Och I skolen icke ådraga eder orenhet genom något av de smådjur som röra sig på jorden.
Lev 11:45  Ty jag är HERREN, som har fört eder upp ur Egyptens land, för att jag skall vara eder Gud. Så skolen I nu vara heliga, ty jag är helig.
Lev 11:46  Detta är lagen om fyrfotadjuren, och om fåglarna, och om alla slags levande varelser som röra sig i vattnet, och om alla slags smådjur på jorden,
Lev 11:47  för att man skall kunna skilja mellan orent och rent, mellan de djur som få ätas och de djur som icke få ätas.
Lev 12:1  Och HERREN talade till Mose och sade:
Lev 12:2  Tala till Israels barn och säg: När en kvinna föder barn och det är ett gossebarn som hon har fött, så skall hon vara oren i sju dagar; lika många dagar som vid sin månadsrening skall hon vara oren.
Lev 12:3  Och på åttonde dagen skall barnets förhud omskäras.
Lev 12:4  Och sedan skall hon stanna hemma trettiotre dagar, under sitt reningsflöde. Hon skall icke komma vid något heligt och får icke heller komma till helgedomen, förrän hennes reningsdagar äro ute.
Lev 12:5  Men om det är ett flickebarn som hon har fött, så skall hon vara oren i två veckor, på samma sätt som vid sin månadsrening; och sedan skall hon stanna hemma i sextiosex dagar, under sitt reningsflöde.
Lev 12:6  Och när hennes reningsdagar äro ute, vare sig efter son eller efter dotter, skall hon föra fram ett årsgammalt lamm såsom brännoffer, och en ung duva eller en turturduva såsom syndoffer, till uppenbarelsetältets ingång, till prästen.
Lev 12:7  Och han skall offra detta inför HERRENS ansikte och bringa försoning för henne, så bliver hon ren från sitt blodflöde. Detta är lagen om en barnaföderska, när hon har fött ett gossebarn, och när hon har fött ett flickebarn.
Lev 12:8  Och om hon icke förmår bekosta ett får, så skall hon taga två turturduvor eller två unga duvor, en till brännoffer och en till syndoffer. Och prästen skall bringa försoning för henne, så bliver hon ren.
Lev 13:1  Och HERREN talade till Mose och Aron och sade:
Lev 13:2  När någon på sin kropps hud får en upphöjning eller ett utslag eller en ljus fläck, och därav uppstår ett spetälskeartat ont på hans kropps hud, så skall han föras till prästen Aron eller till en av hans söner, prästerna.
Lev 13:3  Om då prästen, när han beser det angripna stället på hans kropps hud, finner att håret på det angripna stället har vitnat, och att det angripna stället visar sig djupare än den övriga huden på kroppen, så är han angripen av spetälska; och sedan prästen har besett honom skall han förklara honom oren.
Lev 13:4  Och om det är en vit fläck som synes på hans kropps hud, men den icke visar sig djupare än den övriga huden, och håret därpå icke har vitnat, så skall prästen hålla den angripne innestängd i sju dagar.
Lev 13:5  Om då prästen, när han på sjunde dagen beser honom, finner att det angripna stället visar sig oförändrat, och att det onda icke har utbrett sig på huden, så skall prästen för andra gången hålla honom innestängd i sju dagar.
Lev 13:6  Om då prästen, när han på sjunde dagen beser honom för andra gången, finner att det angripna stället har bleknat, och att det onda icke har utbrett sig på huden, så skall prästen förklara honom ren, ty då är det ett vanligt utslag, och sedan han har tvått sina kläder, är han ren.
Lev 13:7  Men om utslaget utbreder sig på huden, sedan han har låtit bese sig av prästen för att förklaras ren, och han nu för andra gången låter bese sig av prästen
Lev 13:8  och prästen då, när han beser honom, finner att utslaget har utbrett sig på huden, så skall prästen förklara honom oren, ty då är det spetälska.
Lev 13:9  När någon bliver angripen av spetälska, skall han föras till prästen.
Lev 13:10  Om då prästen, när han beser honom, finner en vit upphöjning på huden, och ser att håret där har vitnat, och att svallkött bildar sig i upphöjningen,
Lev 13:11  så är det gammal spetälska på hans kropps hud, och prästen skall förklara honom oren; han skall då icke stänga honom inne, ty han är oren.
Lev 13:12  Men om spetälskan så har brutit ut på huden, att på den angripne hela huden, från huvud till fötter, överallt där prästen ser, är betäckt av spetälska
Lev 13:13  och prästen alltså, när han beser honom, finner att spetälska betäcker hela hans kropp, så skall han förklara den angripne ren. Hela hans kropp har blivit vit; han är ren.
Lev 13:14  Men så snart svallkött visar sig på honom, är han oren.
Lev 13:15  När prästen ser svallköttet, skall han förklara honom oren; svallköttet är orent, det är spetälska.
Lev 13:16  Men om svallköttet förändrar sig och stället bliver vitt, så skall han komma till prästen.
Lev 13:17  Om då prästen, när han beser honom, finner att det angripna stället har blivit vitt, så skall prästen förklara den angripne ren, han är då ren.
Lev 13:18  När någon på sin kropps hud har haft en bulnad som har blivit läkt,
Lev 13:19  men sedan, på det ställe där bulnaden var, en vit upphöjning eller en rödvit fläck visar sig, så skall han låta bese sig av prästen.
Lev 13:20  Om då prästen, när han beser honom, finner att stället visar sig lägre än den övriga huden, och att håret därpå har vitnat, så skall prästen förklara honom oren; ty då är han angripen av spetälska, som har brutit ut där bulnaden var.
Lev 13:21  Men om prästen, när han beser stället, finner att vitt hår saknas där, och att stället icke är lägre än den övriga huden, och att det är blekt, så skall prästen hålla honom innestängd i sju dagar.
Lev 13:22  Om då det onda utbreder sig på huden, så skall prästen förklara honom oren, ty då är han angripen.
Lev 13:23  Men om den ljusa fläcken bliver oförändrad där den är och icke utbreder sig, då är det ett märke efter bulnaden, och prästen skall förklara honom ren.
Lev 13:24  Men om någon på sin kropps hud får ett brännsår, och om av ärrbildningen i brännsåret sedan bliver en rödvit eller vit fläck
Lev 13:25  och prästen, när han beser stället, finner att håret på fläcken har vitnat, och att den visar sig djupare än den övriga huden, så är mannen angripen av spetälska, som har brutit ut där brännsåret var; och prästen skall förklara honom oren, ty då är han angripen av spetälska.
Lev 13:26  Men om prästen, när han beser stället, finner att vitt hår saknas på den ljusa fläcken, och att stället icke är lägre än den övriga huden, och att det är blekt, så skall prästen hålla honom innestängd i sju dagar.
Lev 13:27  Om då prästen, när han på sjunde dagen beser honom, finner att det onda har utbrett sig på huden, så skall prästen förklara honom oren ty då är han angripen av spetälska.
Lev 13:28  Men om den ljusa fläcken bliver oförändrad där den är och icke utbreder sig på huden och förbliver blek, då är det en upphöjning efter brännsåret, och prästen skall förklara honom ren, ty det är ett märke efter brännsåret.
Lev 13:29  När på en man eller en kvinna något ställe på huvudet eller på hakan bliver angripet
Lev 13:30  och prästen, då han beser det angripna stället, finner att det visar sig djupare än den övriga huden och att gulaktigt tunt hår finnes där, så skall prästen förklara den angripne oren, ty då är det spetälskeskorv, huvud- eller hakspetälska.
Lev 13:31  Men om prästen, när han beser det angripna stället med skorven, finner, att om det än icke visar sig djupare än den övriga huden, svart hår likväl saknas där, så skall prästen hålla den av skorven angripne innestängd i sju dagar.
Lev 13:32  Om då prästen, när han på sjunde dagen beser det angripna stället, finner att skorven icke har utbrett sig, och att där icke finnes något gulaktigt hår, och att skorven icke visar sig djupare än den övriga huden,
Lev 13:33  så skall den sjuke raka sig, utan att dock raka det skorviga stället, och prästen skall för andra gången hålla den skorvsjuke innestängd i sju dagar.
Lev 13:34  Om då prästen, när han på sjunde dagen beser den skorvsjuke, finner att skorven icke har utbrett sig på huden, och att den icke visar sig djupare än den övriga huden, så skall prästen förklara honom ren, och sedan han har tvått sina kläder, är han ren.
Lev 13:35  Men om skorven utbreder sig på huden, sedan han har blivit förklarad ren,
Lev 13:36  och prästen, när han beser honom, finner att skorven har utbrett sig på huden, så behöver prästen icke efterforska om där finnes något gulaktigt hår, ty han är oren.
Lev 13:37  Men om skorven visar sig oförändrad, och svart hår har vuxit upp på stället, då är skorven läkt, och han är ren, och prästen skall förklara honom ren.
Lev 13:38  När någon, man eller kvinna, på sin kropps hud får fläckar, vita fläckar,
Lev 13:39  och prästen, när han beser den angripne, finner att fläckarna på hans kropps hud äro blekvita, då är det ett ofarligt utslag som har kommit fram på huden; han är ren.
Lev 13:40  När på en mans huvud håret utan vidare faller av, är det vanlig bakskallighet; han är ren.
Lev 13:41  Och om håret utan vidare faller av på främre delen av huvudet, så är det vanlig framskallighet; han är ren.
Lev 13:42  Men när på det skalliga stället, baktill eller framtill, en rödvit fläck uppstår, då är det spetälska som har brutit ut på det skalliga stället baktill eller framtill.
Lev 13:43  Om alltså prästen, när han beser honom, finner att den upphöjda fläcken på det skalliga stället, baktill eller framtill, är rödvit, och att den visar sig lik spetälska på den övriga kroppens hud,
Lev 13:44  så är mannen spetälsk, han är oren; prästen skall strax förklara honom oren, ty han är angripen på sitt huvud.
Lev 13:45  Den som är angripen av spetälska skall gå med sönderrivna kläder, han skall hava sitt hår oordnat och skyla sitt skägg, och han skall ropa: "Oren! Oren!"
Lev 13:46  Så länge han är angripen av spetälska, skall han vara oren; oren är han. Han skall bo avskild; utanför lägret skall han hava sin bostad.
Lev 13:47  När en klädnad bliver angripen av spetälska, vare sig klädnaden är av ylle eller av linne,
Lev 13:48  eller när så sker med något vävt eller virkat tyg, vare sig av linne eller av ylle, eller med skinn eller med något, vad det vara må, som är förfärdigat av skinn,
Lev 13:49  och det angripna stället visar sig grönaktigt eller rödaktigt, på klädnaden eller skinnet, eller på det vävda eller virkade tyget, eller på skinnsaken, vad det vara må, då är stället angripet av spetälska och skall visas för prästen.
Lev 13:50  Och när prästen har besett det angripna stället, skall han hava den angripna saken inlåst i sju dagar.
Lev 13:51  Om han då, när han på sjunde dagen beser det angripna stället, finner att skadan har utbrett sig på klädnaden, eller på det vävda eller virkade tyget, eller på skinnet, vadhelst det vara må, som är förfärdigat av skinnet, så är stället angripet av elakartad spetälska; sådant är orent.
Lev 13:52  Och man skall bränna upp klädnaden, eller det vävda eller virkade tyget, vare sig det år av ylle eller av linne, eller skinnsaken som är angripen, vad det vara må; ty det är en elakartad spetälska; allt sådant skall brännas upp i eld.
Lev 13:53  Men om prästen, när han beser stället, finner att fläcken icke har utbrett sig på klädnaden, eller på det vävda eller virkade tyget, eller på skinnsaken, vad det vara må,
Lev 13:54  så skall prästen bjuda att man tvår den sak på vilken det angripna stället finnes, och han skall för andra gången hava den inlåst i sju dagar.
Lev 13:55  Om då prästen, när han efter tvagningen beser det angripna stället, finner att det angripna stället icke har förändrat sitt utseende, så är en sådan sak oren, om ock fläcken icke vidare har utbrett sig; du skall bränna upp den i eld; det är en frätfläck, vare sig den sitter på avigsidan eller på rätsidan.
Lev 13:56  Men om prästen, när han beser det angripna stället, finner att det efter tvagningen har bleknat, så skall han riva bort det från klädnaden eller skinnet, eller från det vävda eller virkade tyget.
Lev 13:57  Om likväl sedan en fläck åter visar sig på klädnaden, eller på det vävda eller virkade tyget, eller på skinnsaken, vad det vara må, så är det spetälska som har brutit ut; den sak på vilken det angripna stället finnes skall du bränna upp i eld.
Lev 13:58  Men om genom tvagningen fläcken har gått bort på klädnaden, eller på det vävda eller virkade tyget, eller på skinnsaken, vad det vara må, så skall det för andra gången tvås, och så bliver det rent.
Lev 13:59  Detta är lagen om det som bliver angripet av spetälska, antingen det är en klädnad av ylle eller linne, eller det är vävt eller virkat tyg, eller någon skinnsak, vad det vara må - den lag efter vilken det skall förklaras rent eller orent.
Lev 14:1  Och HERREN talade till Mose och sade:
Lev 14:2  Detta vare lagen om huru man skall förfara, när den som har haft spetälska skall renas: Han skall föras till prästen;
Lev 14:3  och prästen skall gå ut utanför lägret. Om då prästen, när han beser den spetälske, finner att han är botad från den spetälska varav han var angripen,
Lev 14:4  så skall prästen bjuda att man för dens räkning, som skall renas, tager två levande rena fåglar, cederträ, rosenrött garn och isop.
Lev 14:5  Och prästen skall bjuda att man slaktar den ena fågeln över ett lerkärl med friskt vatten i.
Lev 14:6  Sedan skall han taga den levande fågeln, så ock cederträet, det rosenröda garnet och isopen, och detta alltsammans, jämväl den levande fågeln, skall han doppa i den fågelns blod, som har blivit slaktad över det friska vattnet.
Lev 14:7  Och han skall stänka sju gånger på den som skall renas från spetälskan; och sedan han så har renat honom, skall han slappa den levande fågeln fri ute på marken.
Lev 14:8  Och den som skall renas skall två sina kläder och raka av allt sitt hår och bada sig i vatten, så bliver han ren och får sedan gå in i lägret. Dock skall han stanna utanför sitt tält i sju dagar.
Lev 14:9  Och på sjunde dagen skall han raka av allt sitt hår, både huvudhåret och skägget och ögonbrynen: allt sitt hår skall han raka av. Och han skall två sina kläder och bada sin kropp i vatten, så bliver han ren.
Lev 14:10  Och på åttonde dagen skall han taga två felfria lamm av hankön och ett årsgammalt felfritt lamm av honkön, så ock tre tiondedels efa fint mjöl, begjutet med olja, till spisoffer, och därtill en log olja.
Lev 14:11  Och prästen som förrättar reningen skall ställa den som skall renas och allt det andra fram inför HERRENS ansikte, vid ingången till uppenbarelsetältet.
Lev 14:12  Och prästen skall taga det ena lammet och offra det till ett skuldoffer, jämte tillhörande log olja, och vifta detta såsom ett viftoffer inför HERRENS ansikte.
Lev 14:13  Och man skall slakta lammet på samma plats där man slaktar synd- och brännoffersdjuren, på en helig plats; ty skuldoffret tillhör prästen, likasom syndoffret; det är högheligt.
Lev 14:14  Och prästen skall taga något av skuldoffrets blod, och därmed skall prästen bestryka högra örsnibben på den som skall renas, så ock tummen på hans högra hand och stortån på hans högra fot.
Lev 14:15  Sedan skall prästen taga av tillhörande log olja och gjuta i sin vänstra hand,
Lev 14:16  och prästen skall doppa sitt högra pekfinger i oljan som han har i sin vänstra hand och stänka något av oljan med sitt finger sju gånger inför HERRENS ansikte.
Lev 14:17  Och med det som bliver över av oljan i hans hand skall prästen bestryka högra örsnibben på den som skall renas, så ock tummen på hans högra hand och stortån på hans högra fot, ovanpå skuldoffersblodet.
Lev 14:18  Och det som sedan är över av oljan i prästens hand skall han gjuta på dens huvud, som skall renas; så skall prästen bringa försoning för honom inför HERRENS ansikte.
Lev 14:19  Därefter skall prästen offra syndoffret och bringa försoning för den som skall renas, så att han bliver fri ifrån sin orenhet; sedan skall han slakta brännoffersdjuret.
Lev 14:20  Och prästen skall offra brännoffret på altaret och tillika spisoffret. När så prästen bringar försoning för honom, då bliver han ren.
Lev 14:21  Men om han är fattig och icke kan anskaffa så mycket, så skall han taga allenast ett lamm till skuldoffer, och vifta det för att bringa försoning för sig, och allenast en tiondedels efa fint mjöl, begjutet med olja, till spisoffer, och därtill en log olja,
Lev 14:22  så ock två turturduvor eller två unga duvor, efter som han kan anskaffa; den ena skall vara till syndoffer, den andra till brännoffer.
Lev 14:23  Och han skall, för att förklaras ren, bära allt detta till prästen på åttonde dagen, till uppenbarelsetältets ingång, inför HERRENS ansikte.
Lev 14:24  Och prästen skall taga skuldofferslammet och tillhörande log olja, och detta skall prästen vifta såsom ett viftoffer inför HERRENS ansikte.
Lev 14:25  Och man skall slakta skuldofferslammet, och prästen skall taga av skuldoffrets blod och bestryka högra örsnibben på den som skall renas, så ock tummen på hans högra hand och stortån på hans högra fot.
Lev 14:26  Sedan skall prästen gjuta något av oljan i sin vänstra hand,
Lev 14:27  och prästen skall stänka med sitt högra pekfinger något av oljan som han har i sin vänstra hand sju gånger inför HERRENS ansikte.
Lev 14:28  Och prästen skall med oljan som han har i sin hand bestryka högra örsnibben på den som skall renas, så ock tummen på hans högra hand och stortån på hans högra fot, ovanpå skuldoffersblodet.
Lev 14:29  Och det som är över av oljan i prästens hand skall han gjuta på dens huvud, som skall renas, till att bringa försoning för honom inför HERRENS ansikte.
Lev 14:30  Därefter skall han offra den ena av turturduvorna eller av de unga duvorna, vad han nu har kunnat anskaffa;
Lev 14:31  efter som han har kunnat anskaffa: skall han offra den ena till syndoffer och den andra till brännoffer, tillika med spisoffret. Så skall prästen bringa försoning inför HERRENS ansikte för den som skall renas.
Lev 14:32  Detta är lagen om den som har varit angripen av spetälska, men icke kan anskaffa vad som rätteligen hör till hans rening.
Lev 14:33  Och HERREN talade till Mose och Aron och sade:
Lev 14:34  När I kommen in i Kanaans land, som jag vill giva eder till besittning, och jag låter något hus i det land I fån till besittning bliva angripet av spetälska,
Lev 14:35  så skall husets ägare gå och anmäla det för prästen och säga: "Det synes som om mitt hus vore angripet av spetälska."
Lev 14:36  Då skall prästen bjuda att man, innan prästen går in för att bese det angripna stället, utrymmer huset, för att icke allt som är i huset skall bliva orent. Och därefter skall prästen gå in för att bese huset.
Lev 14:37  Om han då, när han beser det angripna stället, finner att det angripna stället på husets vägg bildar grönaktiga eller rödaktiga fördjupningar, som visa sig lägre än den övriga väggen,
Lev 14:38  så skall prästen gå ut ur huset, till dörren på huset, och stänga huset för sju dagar.
Lev 14:39  Om då prästen, när han på sjunde dagen kommer igen och beser det, finner att fläcken har utbrett sig på husets vägg,
Lev 14:40  så skall prästen bjuda att man bryter ut de stenar som äro angripna, och kastar dem utanför staden på någon oren plats.
Lev 14:41  Men huset skall man skrapa överallt innantill och kasta det avskrapade murbruket utanför staden på någon oren plats.
Lev 14:42  Och man skall taga andra stenar och sätta in dem i de förras ställe och taga annat murbruk och rappa huset därmed.
Lev 14:43  Om likväl en fläck åter kommer fram på huset, sedan man har brutit ut stenarna, och sedan man har skrapat huset, och sedan det har blivit rappat,
Lev 14:44  så skall prästen gå in och bese det, och om han då finner att fläcken har utbrett sig på huset, så är detta en elakartad spetälska på huset, det är orent.
Lev 14:45  Och man skall riva ned huset, med dess stenar och trävirke och allt murbruk på huset, och föra bort alltsammans utanför staden till någon oren plats.
Lev 14:46  Och om någon har gått in i huset under den tid det skulle vara stängt, så skall han vara oren ända till aftonen.
Lev 14:47  Och om någon har legat i huset, skall han två sina kläder, och om någon har ätit i huset, skall också han två sina kläder.
Lev 14:48  Men om prästen, när han går in och beser huset, finner att fläcken icke har utbrett sig på huset, sedan det har blivit rappat, så skall han förklara huset rent, ty då är det onda hävt.
Lev 14:49  Och han skall till husets rening taga två fåglar, cederträ, rosenrött garn och isop.
Lev 14:50  Och han skall slakta den ena fågeln över ett lerkärl med friskt vatten i.
Lev 14:51  Sedan skall han taga cederträet, isopen, det rosenröda garnet och den levande fågeln, och doppa alltsammans i den slaktade fågelns blod och det friska vattnet, och stänka på huset sju gånger.
Lev 14:52  Så skall han rena huset med fågelns blod och det friska vattnet och med den levande fågeln, cederträet, isopen och det rosenröda garnet.
Lev 14:53  Och han skall släppa den levande fågeln fri ute på marken utanför staden. När han så bringar försoning för huset, då bliver det rent.
Lev 14:54  Detta är lagen om allt slags spetälskesjukdom och spetälskeskorv,
Lev 14:55  om spetälska på kläder och på hus,
Lev 14:56  om upphöjningar på huden, utslag och ljusa fläckar,
Lev 14:57  till undervisning om när något är orent eller rent. Detta är lagen om spetälska.
Lev 15:1  Och HERREN talade till Mose och Aron och sade:
Lev 15:2  Talen till Israels barn och sägen till dem:
Lev 15:3  Om någon får flytning ur sitt kött, så är sådan flytning oren. Och angående hans orenhet, medan flytningen varar, gäller följande: Evad hans kött avsöndrar flytningen, eller det tillsluter sig för flytningen, så är han oren.
Lev 15:4  Allt varpå den sjuke ligger bliver orent, och allt varpå han sitter bliver orent.
Lev 15:5  Och den som kommer vid det varpå han har legat skall två sina kläder och bada sig i vatten och vara oren ända till aftonen.
Lev 15:6  Och den som sätter sig på något varpå den sjuke har suttit skall två sina kläder och bada sig i vatten och vara oren ända till aftonen.
Lev 15:7  Och den som kommer vid den sjukes kropp skall två sina kläder och bada sig i vatten och vara oren ända till aftonen.
Lev 15:8  Och om den sjuke spottar på någon som är ren, skall denne två sina kläder och bada sig i vatten och vara oren ända till aftonen.
Lev 15:9  Och allt varpå den sjuke sitter när han färdas någonstädes, bliver orent.
Lev 15:10  Och var och en som kommer vid något, vad det vara må, som har legat under honom skall vara oren ända till aftonen; och den som bär bort något sådant skall två sina kläder och bada sig i vatten och vara oren ända till aftonen.
Lev 15:11  Och var och en som den sjuke kommer vid, utan att hava sköljt sina händer i vatten, skall två sina kläder och bada sig i vatten och vara oren ända till aftonen.
Lev 15:12  Och ett lerkärl som den sjuke kommer vid skall sönderslås; men är det ett träkärl, skall det sköljas med vatten.
Lev 15:13  När den som har flytning bliver ren från sin flytning, skall han, för att förklaras ren, räkna sju dagar och därefter två sina kläder, och sedan skall han bada sin kropp i rinnande vatten, så bliver han ren.
Lev 15:14  Och på åttonde dagen skall han taga sig två turturduvor eller två unga duvor och komma inför HERRENS ansikte, till uppenbarelsetältets ingång, och giva dem åt prästen.
Lev 15:15  Och prästen skall offra dem, den ena till syndoffer och den andra till brännoffer; så skall prästen bringa försoning för honom inför HERRENS ansikte, till rening från hans flytning.
Lev 15:16  Och om en man har haft sädesutgjutning, så skall han bada hela sin kropp i vatten och vara oren ända till aftonen.
Lev 15:17  Och allt slags klädnad och allt av skinn, varpå sådan sädesutgjutning har skett, skall tvås i vatten och vara orent ända till aftonen.
Lev 15:18  Och när en man har legat hos en kvinna och sädesutgjutning har skett, så skola de båda bada sig i vatten och vara orena ända till aftonen.
Lev 15:19  Och när en kvinna har sin flytning, i det att blod avgår ur hennes kött, skall hon vara oren i sju dagar, och var och en som kommer vid henne skall vara oren ända till aftonen.
Lev 15:20  Och allt varpå hon ligger under sin månadsrening bliver orent, och allt varpå hon sitter bliver orent.
Lev 15:21  Och var och en som kommer vid det varpå hon har legat skall två sina kläder och bada sig i vatten och vara oren ända till aftonen.
Lev 15:22  Och var och en som kommer vid något varpå hon har suttit skall två sina kläder och bada sig i vatten och vara oren ända till aftonen.
Lev 15:23  Och om någon sak lägges på det varpå hon har legat eller suttit, och någon då kommer vid denna sak, så skall han vara oren ända till aftonen.
Lev 15:24  Och om en man ligger hos henne, och något av hennes månadsflöde kommer på honom, skall han vara oren i sju dagar, och allt varpå han ligger bliver orent.
Lev 15:25  Och om en kvinna har blodflöde under en längre tid, utan att det är hennes månadsrening, eller om hon har flöde utöver tiden för sin månadsrening, så skall om henne, så länge hennes orena flöde varar, gälla detsamma som under hennes månadsreningstid; hon är oren.
Lev 15:26  Om allt varpå hon ligger, så länge hennes flöde varar, skall gälla detsamma som om det varpå hon ligger under sin månadsrening; och allt varpå hon sitter bliver orent, likasom under hennes månadsrening.
Lev 15:27  Och var och en som kommer vid något av detta bliver oren; han skall två sina kläder och bada sig i vatten och vara oren ända till aftonen.
Lev 15:28  Men om hon bliver ren från sitt flöde, skall hon räkna sju dagar och sedan vara ren.
Lev 15:29  Och på åttonde dagen skall hon taga sig två turturduvor eller två unga duvor och bära dem till prästen, till uppenbarelsetältets ingång.
Lev 15:30  Och prästen skall offra den ena till syndoffer och den andra till brännoffer; så skall prästen bringa försoning för henne inför HERRENS ansikte, till rening från hennes orena flöde.
Lev 15:31  Så skolen I bevara Israels barn från orenhet, på det att de icke må dö i sin orenhet, om de orena mitt tabernakel, som står mitt ibland dem.
Lev 15:32  Detta är lagen om den som har flytning och om den som har sädesutgjutning, så att han därigenom bliver oren,
Lev 15:33  och om den kvinna som har sin månadsrening, och om den som har någon flytning, evad det är man eller kvinna, så ock om en man som ligger hos en oren kvinna.
Lev 16:1  Och HERREN talade till Mose, sedan Arons två söner voro döda, de båda som träffats av döden, när de trädde fram inför HERRENS ansikte.
Lev 16:2  Och HERREN sade till Mose: Säg till din broder Aron att han icke på vilken tid som helst får gå in i helgedomen innanför förlåten, framför nådastolen som är ovanpå arken, på det att han icke må dö; ty i molnskyn vill jag uppenbara mig över nådastolen.
Lev 16:3  Så skall förfaras, när Aron skall gå in i helgedomen: Han skall taga en ungtjur till syndoffer och en vädur till brännoffer;
Lev 16:4  han skall ikläda sig en helig livklädnad av linne och hava benkläder av linne över sitt kött, och han skall omgjorda sig med ett bälte av linne och vira en huvudbindel av linne om sitt huvud; detta är de heliga kläderna, och innan han ikläder sig dem, skall han bada sin kropp i vatten.
Lev 16:5  Och av Israels barns menighet skall han mottaga två bockar till syndoffer och en vädur till brännoffer.
Lev 16:6  Och Aron skall föra fram sin egen syndofferstjur och bringa försoning för sig och sitt hus.
Lev 16:7  Sedan skall han taga de två bockarna och ställa dem inför HERRENS ansikte, vid ingången till uppenbarelsetältet.
Lev 16:8  Och Aron skall draga lott om de två bockarna: en lott för HERREN och en lott för Asasel.
Lev 16:9  Och den bock som lotten bestämmer åt HERREN skall Aron föra fram och offra till syndoffer.
Lev 16:10  Men den bock som lotten bestämmer åt Asasel skall ställas levande inför HERRENS ansikte, för att försoning må bringas för honom, på det att han må släppas fri ut till Asasel i öknen.
Lev 16:11  Aron skall alltså föra fram sin syndofferstjur och bringa försoning för sig och sitt hus, han skall slakta sin syndofferstjur.
Lev 16:12  Sedan skall han taga ett fyrfat fullt med glöd från altaret som står inför HERRENS ansikte, och fylla sina händer med stött välluktande rökelse; och han skall bära in detta innanför förlåten.
Lev 16:13  Och rökelsen skall han lägga på elden inför HERRENS ansikte, så att ett moln av rökelse skyler nådastolen, ovanpå vittnesbördet, på det att han icke må dö.
Lev 16:14  Och han skall taga av tjurens blod och stänka med sitt finger framtill på nådastolen; och framför nådastolen skall han stänka blodet sju gånger med sitt finger.
Lev 16:15  Sedan skall han slakta folkets syndoffersbock och bära in hans blod innanför förlåten; och han skall göra med hans blod såsom han gjorde med tjurens blod: han skall tänka därmed på nådastolen och framför nådastolen.
Lev 16:16  Så skall han bringa försoning för helgedomen och rena den från Israels barns orenheter och överträdelser, vad de än må hava syndat. Och på samma sätt skall han göra ned uppenbarelsetältet, som har sin plats hos dem mitt ibland deras orenheter.
Lev 16:17  Och ingen människa får vara i uppenbarelsetältet, från den stund på han går in för att bringa försoning i helgedomen, ända till dess han har gått ut. Så skall han bringa försoning för sig och sitt hus och för Israels hela församling.
Lev 16:18  Sedan skall han gå ut till altaret som står inför HERRENS ansikte och bringa försoning för det; han skall taga av tjurens blod och av bockens blod och stryka på altarets horn runt omkring,
Lev 16:19  och han skall stänka blodet därpå med sitt finger sju gånger, och rena och helga det från Israels barns orenheter.
Lev 16:20  När han så har fullbordat försoningen för helgedomen, uppenbarelsetältet och altaret, skall han föra fram den levande bocken.
Lev 16:21  Och Aron skall lägga båda sina händer på den levande bockens huvud, och bekänna över honom Israels barns alla missgärningar och alla deras överträdelser, vad de än må hava syndat; han skall lägga dem på bockens huvud och genom en man som hålles redo därtill släppa honom ut i öknen.
Lev 16:22  Så skall bocken bära alla deras missgärningar på sig ut i vildmarken; man skall släppa bocken ute i öknen.
Lev 16:23  Därefter skall Aron gå in i uppenbarelsetältet och taga av sig linnekläderna, som han hade iklätt sig när han gick in i helgedomen; och han skall lämna dem där.
Lev 16:24  Och han skall bada sin kropp i vatten på en helig plats och ikläda sig sina vanliga kläder; sedan skall han gå ut och offra sitt eget brännoffer och folkets brännoffer och bringa försoning för sig och för folket.
Lev 16:25  Och fettet av syndoffersdjuret skall han förbränna på altaret.
Lev 16:26  Men den som släppte bocken ut till Asasel skall två sina kläder och bada sin kropp i vatten; därefter får han gå in i lägret.
Lev 16:27  Och syndofferstjuren och syndoffersbocken, vilkas blod blev inburet för att bringa försoning i helgedomen, skola föras bort utanför lägret, och man skall bränna upp dem i eld med deras hud och kött och orenlighet.
Lev 16:28  Och den som bränner upp detta skall två sina kläder och bada sin kropp i vatten; därefter får han gå in i lägret.
Lev 16:29  Och detta skall vara för eder en evärdlig stadga: I sjunde månaden, på tionde dagen i månaden, skolen I späka eder och icke göra något arbete, varken infödingen eller främlingen som bor ibland eder.
Lev 16:30  Ty på den dagen skall försoning bringas för eder, till att rena eder; från alla edra synder skolen I renas inför HERRENS ansikte.
Lev 16:31  En vilosabbat skall den vara för eder, och I skolen då späka eder. Detta skall vara en evärdlig stadga.
Lev 16:32  Och den präst, som har blivit smord och mottagit handfyllning till att vara präst i sin faders ställe skall bringa denna försoning; han skall ikläda sig linnekläderna, de heliga kläderna,
Lev 16:33  och han skall bringa försoning för det allraheligaste och försoning för uppenbarelsetältet och altaret, och han skall bringa försoning för prästerna och allt folket i församlingen.
Lev 16:34  Detta skall vara för eder en evärdlig stadga, att försoning skall bringas för Israels barn, till rening från alla deras synder, en gång om året. Och han gjorde såsom HERREN hade bjudit Mose.
Lev 17:1  Och HERREN talade till Mose och sade:
Lev 17:2  Tala till Aron och hans söner och alla Israels barn och säg till dem Detta är vad HERREN har bjudit och sagt:
Lev 17:3  Om någon av Israels hus, i lägret eller utanför lägret, slaktar ett fäkreatur eller ett lamm eller en get,
Lev 17:4  utan att föra fram djuret till uppenbarelsetältets ingång för att frambära det såsom en offergåva åt HERREN framför HERRENS tabernakel, så skall detta tillräknas den mannen såsom blodskuld, ty blod har han utgjutit, och den mannen skall utrotas ur sitt folk.
Lev 17:5  Därför skola Israels barn föra sina slaktdjur, som de pläga slakta ute på marken, fram till HERREN, till uppenbarelsetältets ingång, till prästen, och där slakta dem såsom tackoffer åt HERREN.
Lev 17:6  Och prästen skall stänka blodet på HERRENS altare, vid ingången till uppenbarelsetältet, och förbränna fettet till en välbehaglig lukt för HERREN.
Lev 17:7  Och de skola icke mer offra sina slaktoffer åt de onda andar som de i trolös avfällighet löpa efter. Detta skall vara en evärdlig stadga för dem från släkte till släkte.
Lev 17:8  Och du skall säga till dem: Om någon av Israels hus, eller av främlingarna som bo ibland dem, offrar ett brännoffer eller ett slaktoffer
Lev 17:9  och icke för det fram till uppenbarelsetältets ingång för att offra det åt HERREN, så skall den mannen utrotas ur sin släkt.
Lev 17:10  Och om någon av Israels hus, eller av främlingarna som bo ibland dem, förtär något blod, så skall jag vända mitt ansikte mot honom som förtär blodet och utrota honom ur hans folk.
Lev 17:11  Ty allt kötts själ är i blodet, och jag har givit eder det till altaret, till att bringa försoning för edra själar; ty blodet är det som bringar försoning, genom själen som är däri.
Lev 17:12  Därför säger jag till Israels barn: Ingen av eder skall förtära blod; och främlingen som bor ibland eder skall icke heller förtära blod.
Lev 17:13  Och om någon av Israels barn, eller av främlingarna som bo ibland dem, fäller ett villebråd av fyrfotadjur eller en fågel, sådant som får ätas, så skall han låta blodet rinna ut och övertäcka det med jord.
Lev 17:14  Ty så är det med allt kötts själ, att blodet är det som innehåller själen; därför säger jag till Israels barn: I skolen icke förtära något kötts blod. Ty blodet är allt kötts själ; var och en som förtär det skall utrotas.
Lev 17:15  Och var och en som äter ett självdött eller ihjälrivet djur, evad han är inföding eller främling, skall två sina kläder och bada sig i vatten och vara oren ända till aftonen; då bliver han ren.
Lev 17:16  Men om han icke tvår sina kläder och icke badar sin kropp kommer han att bära på missgärning.
Lev 18:1  Och HERREN talade till Mose och sade:
Lev 18:2  Tala till Israels barn och säg till dem: Jag är HERREN, eder Gud.
Lev 18:3  I skolen icke göra såsom man gör i Egyptens land, där I haven bott. Ej heller skolen I göra såsom man gör i Kanaans land, dit jag vill föra eder; I skolen icke vandra efter deras stadgar.
Lev 18:4  Efter mina rätter skolen I göra och mina stadgar skolen I hålla, och skolen vandra efter dem. Jag är HERREN, eder Gud.
Lev 18:5  Ja, I skolen hålla mina stadgar och rätter, ty den människa som gör efter dem skall leva genom dem. Jag är HERREN.
Lev 18:6  Ingen bland eder skall komma vid någon kvinna som år hans nära blodsförvant och blotta hennes blygd. Jag är HERREN.
Lev 18:7  Du skall icke blotta din faders blygd genom att blotta din moders blygd; hon är din moder, du skall icke blotta hennes blygd.
Lev 18:8  Du skall icke blotta någon annan kvinnas blygd, som är din faders hustru, ty det är din faders blygd.
Lev 18:9  Du skall icke blotta din systers blygd, evad hon är din faders dotter eller din moders dotter, evad hon är född hemma eller född ute.
Lev 18:10  Du skall icke blotta din sondotters eller din dotterdotters blygd, ty det är din egen blygd.
Lev 18:11  Du skall icke blotta din faders hustrus dotters blygd, ty hon är av din faders släkt, hon är din syster.
Lev 18:12  Du skall icke blotta din faders systers blygd; hon är din faders nära blodsförvant.
Lev 18:13  Du skall icke blotta din moders systers blygd, ty hon är din moders nära blodsförvant.
Lev 18:14  Du skall icke blotta din faders broders blygd: vid hans hustru skall du icke komma; hon är din faders syster.
Lev 18:15  Du skall icke blotta din svärdotters blygd; hon är din sons hustru, hennes blygd skall du icke blotta.
Lev 18:16  Du skall icke blotta din broders hustrus blygd, ty det är din broders blygd.
Lev 18:17  Du skall icke blotta en kvinnas blygd och tillika hennes dotters; du skall icke heller taga till hustru hennes sondotter eller dotterdotter och blotta dennas blygd, de äro ju nära blodsförvanter; sådant vore en skändlighet.
Lev 18:18  Och du skall icke till hustru taga en kvinna jämte hennes syster, så att du uppväcker fiendskap mellan dem, i det att du blottar den enas blygd och tillika den andras, medan den förra lever.
Lev 18:19  Du skall icke komma vid en kvinna och blotta hennes blygd, när hon är oren under sin månadsrening.
Lev 18:20  Med din nästas hustru skall du icke beblanda dig, så att du genom henne bliver oren.
Lev 18:21  Du skall icke giva någon av dina avkomlingar till offer åt Molok; du skall icke ohelga din Guds namn. Jag är HERREN.
Lev 18:22  Du skall icke ligga hos en man såsom man ligger hos en kvinna; det är en styggelse.
Lev 18:23  Du skall icke beblanda där med något djur, så att du genom detta bliver oren. Och ingen kvinna skall hava att skaffa med något djur, så att hon beblandar sig därmed; det är en vederstygglighet.
Lev 18:24  I skolen icke orena eder med något av allt detta, ty med allt sådant hava de hedningar orenat sig, som jag fördriver för eder.
Lev 18:25  Därigenom har landet blivit orenat, och jag har på det hemsökt dess missgärning, så att landet har utspytt sina inbyggare.
Lev 18:26  Så hållen då I mina stadgar och rätter, och ingen av eder, evad han är inföding eller en främling som bor ibland eder, må göra någon av alla dessa styggelser.
Lev 18:27  Ty alla dessa styggelser hava landets inbyggare, som hava varit där före eder, bedrivit, så att landet har blivit orenat.
Lev 18:28  Gören intet sådant, på det att landet icke må utspy eder, om I så orenen det, likasom det utspyr det folk som har bott där före eder.
Lev 18:29  Ty var och en som gör någon av alla dessa styggelser skall utrotas ur sitt folk, ja, var och en som gör sådant.
Lev 18:30  Iakttagen därför vad jag har bjudit eder iakttaga, så att I icke gören efter någon av de styggeliga stadgar som man har följt före eder, och så orenen eder genom dem. Jag är HERREN, eder Gud.
Lev 19:1  Och HERREN talade till Mose och sade:
Lev 19:2  Tala till Israels barns hela menighet och säg till dem: I skolen vara heliga, ty jag, HERREN, eder Gud, är helig.
Lev 19:3  Var och en av eder frukte sin moder och sin fader. Mina sabbater skolen I hålla. Jag är HERREN, eder Gud.
Lev 19:4  I skolen icke vända eder till avgudar och icke göra eder gjutna gudar. Jag är HERREN, eder Gud.
Lev 19:5  När I viljen offra tackoffer åt HERREN, skolen I offra det på sådant sätt att I bliven välbehagliga.
Lev 19:6  Samma dag I offren det skall det ätas, eller ock den följande dagen; men det som bliver över till tredje dagen skall brännas upp i eld.
Lev 19:7  Om det ätes på tredje dagen, så är det en vederstygglighet; det bliver då icke välbehagligt.
Lev 19:8  Den som äter därav kommer att bära på missgärning, ty han har ohelgat det som var helgat åt HERREN, och han skall utrotas ur sin släkt.
Lev 19:9  När I inbärgen skörden av edert land, skall du icke skörda intill yttersta kanten av din åker, icke heller skall du göra någon axplockning efter din skörd.
Lev 19:10  Och i din vingård skall du icke göra någon efterskörd, och de avfallna druvorna i din vingård skall du icke plocka upp; du skall lämna detta kvar åt den fattige och åt främlingen. Jag är HERREN, eder Gud.
Lev 19:11  I skolen icke stjäla eller ljuga eller begå något svek mot varandra.
Lev 19:12  I skolen icke svärja falskt vid mitt namn; då ohelgar du din Guds namn. Jag är HERREN.
Lev 19:13  Du skall icke med orätt avhända din nästa något, eller taga något ifrån honom med våld. Du skall icke förhålla dagakarlen hans lön över natten till morgonen.
Lev 19:14  Du skall icke uttala förbannelser över en döv, och för en blind skall du icke lägga något varpå han kan falla; du skall frukta din Gud. Jag är HERREN.
Lev 19:15  I skolen icke göra orätt i domen; du skall icke hava anseende till den ringes person, ej heller vara partisk för den mäktige; du skall döma din nästa rätt.
Lev 19:16  Du skall icke gå med förtal bland dina fränder; du skall icke stå efter din nästas blod. Jag är HERREN.
Lev 19:17  Du skall icke hava hat till din broder i ditt hjärta, men väl må du tillrättavisa din nästa, så att du icke för hans skull kommer att bära på synd.
Lev 19:18  Du skall icke hämnas och icke hysa agg mot någon av ditt folk, utan du skall älska din nästa såsom dig själv. Jag är HERREN.
Lev 19:19  Mina stadgar skolen I hålla: Du skall icke låta två slags djur av din boskap para sig med varandra; din åker skall du icke beså med två slags säd; en klädnad av två olika slags garn får icke komma på dig.
Lev 19:20  Om en man har legat hos en kvinna och beblandat sig med henne, och hon är trälinna i en annan mans våld, och hon icke har blivit friköpt eller frigiven, så skola de straffas, men icke dödas, eftersom hon icke var fri.
Lev 19:21  Och han skall föra fram sitt skuldoffer inför HERREN, till uppenbarelsetältets ingång, en skuldoffersvädur.
Lev 19:22  När så prästen medelst skuldoffersväduren bringar försoning för honom inför HERRENS ansikte för den synd han har begått, då bliver den synd han har begått honom förlåten.
Lev 19:23  När I kommen in i landet och planteren träd av olika slag med ätbar frukt, skolen I anse deras frukt såsom deras förhud. I tre år skolen I hålla dem för oomskurna och icke äta av dem;
Lev 19:24  men under det fjärde året skall all deras frukt vara helgad till HERRENS lov,
Lev 19:25  och först under det femte skolen I äta deras frukt. Så skolen I göra, för att de må giva så mycket större avkastning åt eder. Jag är HERREN, eder Gud.
Lev 19:26  I skolen icke äta något med blod i. I skolen icke befatta eder med spådom eller teckentyderi.
Lev 19:27  I skolen icke rundklippa kanten av edert huvudhår, ej heller skall du avstympa kanten av ditt skägg.
Lev 19:28  I skolen icke göra något märke på eder kropp för någon död, ej heller bränna in skrifttecken på eder. Jag är HERREN.
Lev 19:29  Du skall icke ohelga din dotter med att låta henne bliva en sköka, på det att icke landet må förfalla i skökoväsende och bliva uppfyllt av skändlighet.
Lev 19:30  Mina sabbater skolen I hålla, och för min helgedom skolen I hava fruktan. Jag är HERREN.
Lev 19:31  I skolen icke vända eder till andar som tala genom besvärjare eller spåmän. Söken icke sådana, så att I bliven orena genom dem. Jag är HERREN, eder Gud.
Lev 19:32  För ett grått huvud skall du stå upp, och den gamle skall du ära; du skall frukta din Gud. Jag är HERREN.
Lev 19:33  När en främling bor hos eder i edert land, skolen I icke förtrycka honom.
Lev 19:34  Främlingen som bor hos eder skall räknas såsom en inföding bland eder, du skall älska honom såsom dig själv; I haven ju själva varit främlingar i Egyptens land. Jag är HERREN, eder Gud.
Lev 19:35  I skolen icke göra orätt i domen, icke i fråga om mått, vikt eller mål.
Lev 19:36  Riktig våg, riktiga vikter, riktig efa, riktigt hin-mått skolen I hava. Jag är HERREN, eder Gud, som har fört eder ut ur Egyptens land.
Lev 19:37  Så skolen I nu hålla alla mina stadgar och alla mina rätter och göra efter dem. Jag är HERREN.
Lev 20:1  Och HERREN talade till Mose och sade:
Lev 20:2  Du skall ock säga till Israels barn: Om någon av Israels barn, eller av främlingarna som bo i Israel, giver någon av sina avkomlingar åt Molok, så skall han straffas med döden; folket i landet skall stena honom.
Lev 20:3  Och jag skall vända mitt ansikte mot den mannen och utrota honom ur hans folk, därför att han har givit en av sina avkomlingar åt Molok, och därmed orenat min helgedom och ohelgat mitt heliga namn.
Lev 20:4  Om folket i landet ser genom fingrarna med den mannen, när han giver en av sina avkomlingar åt Molok, så att de icke döda honom,
Lev 20:5  då skall jag själv vända mitt ansikte mot den mannen och mot hans släkt; och honom och alla dem som hava följt honom och i trolös avfällighet lupit efter Molok skall jag utrota ur deras folk.
Lev 20:6  Och om någon vänder sig till andar som tala genom besvärjare eller spåmän, för att i trolös avfällighet löpa efter dem, så skall jag vända mitt ansikte mot honom och utrota honom ur hans folk.
Lev 20:7  Så skolen I nu hålla eder heliga, och vara heliga; ty jag är HERREN, eder Gud.
Lev 20:8  Och I skolen hålla mina stadgar och göra efter dem. Jag är HERREN, som helgar eder.
Lev 20:9  Om någon uttalar förbannelser över sin fader eller sin moder, skall han straffas med döden; över sin fader och sin moder har han uttalat förbannelser, blodskuld låder vid honom.
Lev 20:10  Om någon begår äktenskapsbrott med en annan mans hustru, om han begår äktenskapsbrott med sin nästas hustru, så skola de straffas med döden, både mannen och kvinnan som hava begått äktenskapsbrottet.
Lev 20:11  Om någon ligger hos en kvinna som är hans faders hustru, så blottar han sin faders blygd; de skola båda straffas med döden, blodskuld låder vid dem.
Lev 20:12  Om någon ligger hos sin svärdotter, så skola de båda straffas med döden; de hava bedrivit en vederstygglighet, blodskuld låder vid dem.
Lev 20:13  Om en man ligger hos en annan man såsom man ligger hos en kvinna, så göra de båda en styggelse; de skola straffas med döden, blodskuld låder vid dem.
Lev 20:14  Om någon till hustru tager en kvinna och tillika hennes moder, så är det en skändlighet; man skall bränna upp både honom och dem i eld, för att icke någon skändlighet må finnas bland eder.
Lev 20:15  Om en man beblandar sig med något djur, så skall han straffas med döden, och djuret skolen I dräpa.
Lev 20:16  Och om en kvinna kommer vid något djur och beblandar sig därmed, så skall du dräpa både kvinnan och djuret; de skola straffas med döden, blodskuld låder vid dem.
Lev 20:17  Om någon tager till hustru sin syster, sin faders dotter eller sin moders dotter, och ser hennes blygd och hon ser hans blygd, så är det en skamlig gärning, och de skola utrotas inför sitt folks ögon; han har blottat sin systers blygd, han bär på missgärning.
Lev 20:18  Om någon ligger hos en kvinna som har sin månadsrening och blottar hennes blygd, i det att han avtäcker hennes brunn och hon blottar sitt blods brunn, så skola de båda utrotas ur sitt folk.
Lev 20:19  Du skall icke blotta din moders systers eller din faders systers blygd. Ty den så gör avtäcker sin nära blodsförvants blygd; de komma att bära på missgärning.
Lev 20:20  Om någon ligger hos sin farbroders hustru, så blottar han sin farbroders blygd; de komma att bära på synd, barnlösa skola de dö.
Lev 20:21  Om någon tager sin broders hustru, så är det en oren gärning; han blottar då sin broders blygd, barnlösa skola de bliva.
Lev 20:22  Så skolen I nu hålla alla mina stadgar och alla mina rätter och stadgar och göra efter dem, för att landet icke må utspy eder, det land dit jag vill föra eder, så att I fån bo där.
Lev 20:23  Och I skolen icke vandra efter det folks stadgar, som jag vill fördriva för eder; ty just därför att de hava bedrivit allt sådant, har jag blivit led vid dem.
Lev 20:24  Och därför har jag sagt till eder I skolen besitta deras land, ty jag skall giva eder det till besittning, ett land som flyter av mjölk och honung. Jag är HERREN, eder Gud, som har avskilt eder från andra folk.
Lev 20:25  Gören alltså skillnad mellan rena fyrfotadjur och orena, och mellan rena fåglar och orena, så att I icke gören eder själva till styggelse för de fyrfotadjurs eller fåglars skull eller för de kräldjurs, skull på marken, som jag har avskilt, för att I skolen hålla dem för orena.
Lev 20:26  I skolen vara mig heliga, ty jag, HERREN, är helig, och jag har avskilt eder från andra folk, för att I skolen höra mig till.
Lev 20:27  När någon, man eller kvinna, befattar sig med andebesvärjelse eller spådom, skall denne straffas med döden; man skall stena honom, blodskuld låder vid honom.
Lev 21:1  Och HERREN sade till Mose: Säg till prästerna, Arons söner, säg till dem så: En präst får icke ådraga sig orenhet genom någon död bland sina fränder,
Lev 21:2  utom genom sina närmaste blodsförvanter: sin moder, sin fader, sin dotter, sin broder; son, sin dotter, sin broder;
Lev 21:3  så ock genom sin syster, om hon var jungfru och stod honom närmare och icke tillhörde någon man, i sådant fall må han ådraga sig orenhet genom henne.
Lev 21:4  Eftersom han är en herre bland sina fränder, får han icke ådraga sig orenhet och göra sig ohelig.
Lev 21:5  Prästerna skola icke raka någon del av sitt huvud skallig eller avraka kanten av sitt skägg eller rista något märke på sin kropp.
Lev 21:6  De skola vara helgade åt sin Guds och må icke ohelga sin Guds namn, ty de bära fram HERRENS eldsoffer sin Guds spis; därför skola de heliga.
Lev 21:7  Ingen av dem skall taga till hustru en sköka eller en vanärad kvinna, ej heller skall någon taga till hustru en kvinna som har blivit förskjuten av sin man, ty prästen är helgad åt sin Gud.
Lev 21:8  Därför skall du akta honom helig, ty han bär fram din Guds spis; han skall vara dig helig, ty jag, HERREN, som helgar eder, är helig.
Lev 21:9  Om en prästs dotter ohelgar sig genom skökolevnad, så ohelgar hon sin fader; hon skall brännas upp i eld.
Lev 21:10  Den som är överstepräst bland sina bröder, den på vilkens huvud smörjelseoljan har blivit utgjuten, och som har mottagit handfyllning till att ikläda sig prästkläderna, han skall icke hava sitt hår oordnat, ej heller riva sönder sina kläder;
Lev 21:11  och han skall icke gå in till någon död; icke ens genom sin fader eller genom sin moder får han ådraga sig orenhet.
Lev 21:12  Och ur helgedomen skall han icke gå ut, på det att han icke må ohelga sin Guds helgedom, ty hans Guds smörjelseolja, varmed han har blivit invigd, är på honom. Jag är HERREN.
Lev 21:13  Till hustru skall han taga en kvinna som är jungfru.
Lev 21:14  En änka eller en förskjuten hustru eller en vanärad kvinna, en sköka - en sådan får han icke taga, utan en jungfru bland sina fränder skall han taga till hustru,
Lev 21:15  för att han icke må ohelga sin livsfrukt bland sina fränder; ty jag är HERREN, som helgar honom.
Lev 21:16  Och HERREN talade till Mose och sade:
Lev 21:17  Tala till Aron och säg: Av dina avkomlingar i kommande släkten skall ingen som har något lyte träda fram för att frambära sin Guds spis.
Lev 21:18  Ingen skall träda fram, som har något lyte, varken en blind eller en halt, eller en som har lyte i ansiktet, eller som har någon lem för stor,
Lev 21:19  ingen som har brutit arm eller ben,
Lev 21:20  ingen som är puckelryggig eller förkrympt, eller som har fel på ögat, eller som har skabb eller annat utslag, eller som är snöpt.
Lev 21:21  Av prästen Arons avkomlingar skall ingen som har något lyte gå fram för att frambära HERRENS eldsoffer; han har ett lyte, han skall icke gå fram för att frambära sin Guds spis.
Lev 21:22  Sin Guds spis må han äta, både det som är högheligt och det som är heligt,
Lev 21:23  men eftersom han har ett lyte, skall han icke gå in till förlåten, ej heller skall han gå fram till altaret, på det att han icke må ohelga mina heliga ting; ty jag är HERREN, som helgar dem.
Lev 21:24  Och Mose talade detta till Aron och hans söner och alla Israels barn.
Lev 22:1  Och HERREN talade till Mose och sade:
Lev 22:2  Tala till Aron och hans söner och säg att de skola hålla sig ifrån de heliga gåvor som Israels barn bära fram åt mig, på det att de icke må ohelga mitt heliga namn. Jag är HERREN.
Lev 22:3  Säg till dem: Om i kommande släkten någon av edra avkomlingar, medan orenhet låder vid honom, kommer vid de heliga gåvor som Israels barn bära fram åt HERREN, så skall han utrotas ur min åsyn Jag är HERREN.
Lev 22:4  Om någon av Arons avkomlingar är spetälsk eller har flytning, skall han icke äta av de heliga gåvorna, förrän han har blivit ren; ej heller den som kommer vid någon som har blivit oren genom en död, eller den som har haft sädesutgjutning;
Lev 22:5  ej heller den som kommer vid något slags smådjur genom vilket man bliver oren, eller vid en människa genom vilken man bliver oren, på vad sätt denna än må hava blivit oren.
Lev 22:6  Den som kommer vid något sådant, han skall vara oren ända till aftonen, och skall icke äta av de heliga gåvorna, förrän han har badat sin kropp i vatten.
Lev 22:7  Men när solen har gått ned, är han ren, och sedan må han äta av de heliga gåvorna, ty det är hans spis.
Lev 22:8  Ett självdött eller ihjälrivet djur skall han icke äta, så att han därigenom bliver oren. Jag är HERREN.
Lev 22:9  De skola iakttaga vad jag har bjudit dem iakttaga, på det att de icke för det heligas skull må komma att bära på synd och träffas av döden därför att de ohelga det. Jag är HERREN, som helgar dem.
Lev 22:10  Ingen främmande får äta av det heliga; en inhysesman hos prästen eller en hans legodräng skall icke äta av det heliga.
Lev 22:11  Men när en präst har köpt en träl för sina penningar, må denne äta därav, så ock den träl som är född i hans hus; dessa må äta av hans spis.
Lev 22:12  När en prästs dotter har blivit en främmande mans hustru, skall hon icke äta av det heliga som gives till offergärd.
Lev 22:13  Men om en prästs dotter har blivit änka eller blivit förskjuten, och hon är utan livsfrukt, och hon så kommer åter till sin faders hus och är där såsom i sin ungdom, då må hon äta av sin faders spis; men ingen främmande får äta därav.
Lev 22:14  Och om någon ouppsåtligen äter av det heliga, skall han lägga femtedelen därtill och giva prästen ersättning för det heliga.
Lev 22:15  Prästerna skola icke ohelga de heliga gåvorna, det som Israels barn göra såsom gärd åt HERREN,
Lev 22:16  och därigenom draga över dem missgärning och skuld, när de äta av deras heliga gåvor; ty jag är HERREN, som helgar dem.
Lev 22:17  Och HERREN talade till Mose och sade:
Lev 22:18  Tala till Aron och hans söner och alla Israels barn och säg till dem: Om någon av Israels hus eller av främlingarna i Israel vill offra något offer, vare sig det är ett löftesoffer eller ett frivilligt offer som de vilja offra åt HERREN såsom brännoffer, så skolen I göra det på sådant sätt att I bliven välbehagliga;
Lev 22:19  offret skall vara ett felfritt handjur, av fäkreaturen eller av fåren eller av getterna;
Lev 22:20  I skolen icke därtill taga ett djur som har något lyte, ty genom ett sådant bliven I icke välbehagliga.
Lev 22:21  Och när någon vill offra ett tackoffer åt HERREN av fäkreaturen eller av småboskapen, vare sig det gäller att fullgöra ett löfte, eller det gäller ett frivilligt offer, då skall det vara felfritt far att bliva välbehagligt; intet lyte får finnas därpå.
Lev 22:22  Det som är blint eller brutet eller stympat eller sårigt, eller det som har skabb eller annat utslag sådant skolen I icke offra åt HERREN; eldsoffer av sådant skolen I icke lägga på altaret åt HERREN.
Lev 22:23  Ett djur av fäkreaturen eller av småboskapen, som har någon lem för stor eller för liten, må du väl offra såsom frivilligt offer, men såsom löftesoffer bliver det icke välbehagligt.
Lev 22:24  Och I skolen icke offra åt HERREN något som har blivit snöpt genom klämning eller krossning eller avslitning eller utskärning; sådant skolen I icke göra i edert land.
Lev 22:25  Icke heller av en utlännings hand skolen I mottaga och offra sådana djur till eder Guds spis, ty de äro skadade, de hava ett lyte; genom sådana bliven I icke välbehagliga.
Lev 22:26  Och HERREN talade till Mose och sade:
Lev 22:27  När en kalv eller ett får eller en get har blivit född, skall djuret dia sin moder i sju dagar. Men allt ifrån den åttonde dagen är det välbehagligt såsom eldsoffersgåva åt HERREN.
Lev 22:28  I skolen icke slakta något djur, vare sig av fäkreaturen eller av småboskapen, på samma dag som dess avföda.
Lev 22:29  När I viljen offra ett lovoffer åt HERREN, skolen I offra det på sådant sätt att I bliven välbehagliga.
Lev 22:30  Det skall ätas samma dag; I skolen icke lämna något därav kvar till följande morgon. Jag är HERREN.
Lev 22:31  I skolen hålla mina bud och göra efter dem. Jag är HERREN.
Lev 22:32  I skolen icke ohelga mitt heliga namn, ty jag vill bliva helgad bland Israels barn. Jag är HERREN, som helgar eder,
Lev 22:33  han som har fört eder ut ur Egyptens land, för att jag skall vara eder Gud. Jag är HERREN.
Lev 23:1  Och HERREN talade till Mose och sade:
Lev 23:2  Tala till Israels barn och säg till dem: Dessa äro HERRENS högtider, vilka I skolen utlysa såsom heliga sammankomster; mina högtider äro dessa:
Lev 23:3  Sex dagar skall arbete göras, men på sjunde dagen är vilosabbat, en dag för helig sammankomst; intet arbete skolen I då göra. Det är HERRENS sabbat, var I än ären bosatta.
Lev 23:4  Dessa äro HERRENS högtider, de heliga sammankomster som I skolen utlysa på bestämda tider:
Lev 23:5  I första månaden, på fjortonde dagen i månaden, vid aftontiden, är HERRENS påsk.
Lev 23:6  Och på femtonde dagen i samma månad är HERRENS osyrade bröds högtid; då skolen I äta osyrat bröd, i sju dagar.
Lev 23:7  På den första dagen skolen I hålla en helig sammankomst; ingen arbetssyssla skolen I då göra.
Lev 23:8  Och I skolen offra eldsoffer åt HERREN i sju dagar. På den sjunde dagen skall åter hållas en helig sammankomst; ingen arbetssyssla skolen I då göra.
Lev 23:9  Och HERREN talade till Mose och sade:
Lev 23:10  Tala till Israels barn och säg till dem: När I kommen in i det land som jag vill giva eder, och I inbärgen dess skörd, då skolen I bära till prästen den kärve som är förstlingen av eder skörd.
Lev 23:11  Och den kärven skall han vifta inför HERRENS ansikte, för att I mån bliva välbehagliga; dagen efter sabbaten skall prästen vifta den.
Lev 23:12  Och på den dag då I låten vifta kärven skolen I offra ett felfritt årsgammalt lamm till brännoffer åt HERREN,
Lev 23:13  och såsom spisoffer därtill två tiondedels efa fint mjöl, begjutet med olja, ett eldsoffer åt HERREN till en välbehaglig lukt, och såsom drickoffer därtill en fjärdedels hin vin.
Lev 23:14  Och intet av det nya, varken bröd eller rostade ax eller korn av grönskuren säd, skolen I äta förrän på denna samma dag, icke förrän I haven burit fram offergåvan åt eder Gud. Detta skall vara en evärdlig stadga för eder från släkte till släkte, var I än ären bosatta.
Lev 23:15  Sedan skolen I räkna sju fulla veckor från dagen efter sabbaten, från den dag då I buren fram viftofferskärven;
Lev 23:16  femtio dagar skolen I räkna intill dagen efter den sjunde sabbaten; då skolen I bära fram ett offer av den nya grödan åt HERREN.
Lev 23:17  Från de orter där I bon skolen I bära fram viftoffersbröd, två kakor av två tiondedels efa fint mjöl, bakade med surdeg: en förstlingsgåva åt HERREN.
Lev 23:18  Och jämte brödet skolen I föra fram sju felfria årsgamla lamm, en ungtjur och två vädurar, till att offras såsom brännoffer åt HERREN, med tillhörande spisoffer och drickoffer: ett eldsoffer till en välbehaglig lukt för HERREN.
Lev 23:19  Därtill skolen I offra en bock till syndoffer och två årsgamla lamm till tackoffer.
Lev 23:20  Och prästen skall vifta dem såsom ett viftoffer inför HERRENS ansikte, jämte förstlingsbröden som bäras fram tillika med de båda lammen de skola vara helgade åt HERREN och tillhöra prästen.
Lev 23:21  Och till denna samma dag skolen I utlysa en helig sammankomst att hållas av eder; ingen arbetssyssla skolen I då göra. Detta skall vara en evärdlig stadga för eder från släkte till släkte, var I än ären bosatta.
Lev 23:22  Och när I inbärgen skörden av edert land, skall du icke skörda intill yttersta kanten av din åker, icke heller skall du göra någon axplockning efter din skörd, du skall lämna detta kvar åt den fattige och åt främlingen. Jag är HERREN, eder Gud.
Lev 23:23  Och HERREN talade till Mose och sade:
Lev 23:24  Tala till Israels barn och säg: I sjunde månaden, på första dagen i månaden, skolen I hålla sabbatsvila, en högtid med basunklang, till att bringa eder i åminnelse inför HERREN, en helig sammankomst.
Lev 23:25  Ingen arbetssyssla skolen I då göra, och I skolen offra eldsoffer åt HERREN.
Lev 23:26  Och HERREN talade till Mose och sade:
Lev 23:27  Men på tionde dagen i samma sjunde månad är försoningsdagen; då skolen I hålla en helig sammankomst, och I skolen då späka eder; och I skolen offra eldsoffer åt HERREN.
Lev 23:28  Och I skolen intet arbete göra på denna samma dag, ty det är en försoningsdag, då försoning bringas för eder inför HERRENS, eder Guds, ansikte.
Lev 23:29  Och var och en som icke späker sig på denna samma dag skall utrotas ur sin släkt.
Lev 23:30  Och var och en som gör något arbete på denna samma dag, honom skall jag förgöra ur hans folk.
Lev 23:31  Intet arbete skolen I då göra Detta skall vara en evärdlig stadga för eder från släkte till släkte, var I än ären bosatta.
Lev 23:32  En vilosabbat skall den vara för eder, och I skolen då späka eder. På nionde dagen i månaden, om aftonen, skolen I hålla denna eder sabbatsvila, från afton till afton.
Lev 23:33  Och HERREN talade till Mose och sade:
Lev 23:34  Tala till Israels barn och säg: På femtonde dagen i samma sjunde månad är HERRENS lövhyddohögtid, i sju dagar.
Lev 23:35  På den första dagen skall man hålla en helig sammankomst; ingen arbetssyssla skolen I då göra.
Lev 23:36  I sju dagar skolen I offra eldsoffer åt HERREN. På den åttonde dagen skolen I hålla en helig sammankomst och skolen offra eldsoffer åt HERREN. Då är högtidsförsamling; ingen arbetssyssla skolen I då göra.
Lev 23:37  Dessa äro HERRENS högtider, vilka I skolen utlysa såsom heliga sammankomster, och på vilka I skolen offra eldsoffer åt HERREN, brännoffer och spisoffer, slaktoffer och drickoffer, var dag de för den dagen bestämda offren -
Lev 23:38  detta förutom HERRENS sabbater, och förutom edra övriga gåvor, och förutom alla edra löftesoffer, och förutom alla frivilliga offer som I given åt HERREN.
Lev 23:39  Men på femtonde dagen i sjunde månaden, när I inbärgen avkastningen av landet, skolen I fira HERRENS högtid, i sju dagar. På den första dagen är sabbatsvila, på den åttonde dagen är ock sabbatsvila.
Lev 23:40  Och I skolen på den första dagen taga frukt av edra skönaste träd, kvistar av palmer och grenar av lummiga träd och av pilträd, och skolen så vara glada i sju dagar inför HERRENS, eder Guds, ansikte.
Lev 23:41  I skolen fira denna högtid såsom en HERRENS högtid sju dagar om året. Detta skall vara en evärdlig stadga för eder från släkte till släkte; i sjunde månaden skolen I fira den.
Lev 23:42  Då skolen I bo i lövhyddor i sju dagar; alla de som äro infödingar i Israel skola bo i lövhyddor,
Lev 23:43  för att edra efterkommande må veta huru jag lät Israels barn bo i lövhyddor, när jag förde dem ut ur Egyptens land Jag är HERREN, eder Gud.
Lev 23:44  Och Mose talade till Israels barn om dessa HERRENS högtider.
Lev 24:1  Och HERREN talade till Mose och sade:
Lev 24:2  Bjud Israels barn att bära till dig ren olja, av stötta oliver, till ljusstaken, så att lamporna dagligen kunna sättas upp.
Lev 24:3  Utanför den förlåt som hänger framför vittnesbördet, i uppenbarelsetältet, skall Aron beständigt sköta den, från aftonen till morgonen, inför HERRENS ansikte. Detta skall vara en evärdlig stadga för eder från släkte till släkte.
Lev 24:4  Lamporna på den gyllene ljusstaken skall han beständigt sköta inför HERRENS ansikte.
Lev 24:5  Och du skall taga fint mjöl och därav baka tolv kakor; var kaka skall innehålla två tiondedels efa.
Lev 24:6  Och du skall lägga upp dem i två rader, sex i var rad, på det gyllene bordet inför HERRENS ansikte.
Lev 24:7  Och på vardera raden skall du lägga ren rökelse, för att denna må utgöra själva altaroffret av bröden, ett eldsoffer åt HERREN.
Lev 24:8  Sabbatsdag efter sabbatsdag skall man beständigt lägga upp dem inför HERRENS ansikte: en gärd av Israels barn, till ett evigt förbund.
Lev 24:9  De skola tillhöra Aron och hans söner och skola ätas av dem på en helig plats, ty de äro högheliga och äro hans evärdliga rätt av HERRENS eldsoffer.
Lev 24:10  Och en man som var son till en israelitisk kvinna, men till fader hade en egyptisk man, gick ut bland Israels barn; och den israelitiska kvinnans son och en israelitisk man kommo i träta med varandra i lägret.
Lev 24:11  Och den israelitiska kvinnans son smädade Namnet och hädade. Då förde de honom fram till Mose. Och hans moder hette Selomit, dotter till Dibri, av Dans stam.
Lev 24:12  Och de satte honom i förvar, för att de skulle få hans dom bestämd efter HERRENS befallning.
Lev 24:13  Och HERREN talade till Mose och sade:
Lev 24:14  För ut hädaren utanför lägret; sedan må alla som hörde det lägga sina händer på hans huvud, och må så hela menigheten stena honom.
Lev 24:15  Och till Israels barn skall du tala och säga: Om någon hädar sin Gud, kommer han att bära på synd.
Lev 24:16  Och den som smädar HERRENS namn skall straffas med döden; hela menigheten skall stena honom. Evad det är en främling eller en inföding som smädar Namnet, skall han dödas.
Lev 24:17  Om någon slår ihjäl någon människa, skall han straffas med döden;
Lev 24:18  och den som slår ihjäl ett boskapsdjur skall ersätta det: liv för liv.
Lev 24:19  Och om någon vållar att hans nästa får ett lyte, så skall man göra mot honom såsom han själv har gjort:
Lev 24:20  bruten lem för bruten lem, öga för öga, tand för tand; samma lyte han har vållat att en annan fick skall han själv få.
Lev 24:21  Den som slår ihjäl ett boskapsdjur skall ersätta det, och den som slår ihjäl en människa skall dödas.
Lev 24:22  En och samma lag skall gälla för eder, den skall gälla lika väl för främlingen som för infödingen; ty jag är HERREN, eder Gud.
Lev 24:23  Och Mose talade detta till Israels barn; och de förde ut hädaren utanför lägret och stenade honom. Alltså gjorde Israels barn såsom HERREN hade bjudit Mose.
Lev 25:1  Och HERREN talade till Mose på Sinai berg och sade:
Lev 25:2  Tala till Israels barn och säg till dem: När I kommen in i det land som jag vill giva eder, skall landet hålla sabbat åt HERREN.
Lev 25:3  I sex år skall du beså din åker, och i sex år skära din vingård och inbärga avkastningen av landet,
Lev 25:4  men under det sjunde året skall landet hava vilosabbat, en HERRENS sabbat; då skall du icke beså din åker och icke skära din vingård.
Lev 25:5  Vad som växer upp av spillsäden efter din skörd skall du icke skörda, och de druvor som växa på dina oskurna vinträd skall du icke avbärga. Det skall vara ett sabbatsvilans år för landet.
Lev 25:6  Och vad landets sabbat ändå giver skolen I hava till föda: du själv, din tjänare och din tjänarinna, din daglönare och din inhysesman, de som bo hos dig.
Lev 25:7  Din boskap och de vilda djuren i ditt land skola ock hava sin föda av all dess avkastning.
Lev 25:8  Och du skall räkna sju årsveckor, det är sju gånger sju år, så att tiden för de sju årsveckorna bliver fyrtionio år.
Lev 25:9  Då skall du i sjunde månaden, på tionde dagen i månaden, låta blåsa i larmbasun; på försoningsdagen skolen I blåsa i basun över hela edert land.
Lev 25:10  Och I skolen helga det femtionde året och utropa frihet i landet för alla dess inbyggare. Det skall vara ett jubelår för eder; var och en av eder skall då återfå sin arvsbesittning, var och en av eder skall återfå sin släktegendom.
Lev 25:11  Ett jubelår skall detta femtionde år vara för eder; då skolen I icke så något, och vad som då växer upp av spillsäden skolen I icke skörda, och I skolen då icke avbärga edra oskurna vinträd.
Lev 25:12  Ty det är ett jubelår; heligt skall det vara för eder. Från själva marken skolen I hämta eder föda, av dess avkastning.
Lev 25:13  Under ett sådant jubelår skall var och en av eder återfå sin arvsbesittning.
Lev 25:14  Om I alltså säljer något åt eder nästa eller köpen något av eder nästa, skolen I icke göra varandra orätt:
Lev 25:15  efter antalet år från jubelåret skall du betala din nästa, efter antalet årsgrödor skall han få betalning av dig.
Lev 25:16  Alltefter som åren äro flera skall du betala högre pris, och alltefter som åren äro färre skall du betala lägre pris; ty ett visst antal grödor är det han säljer till dig.
Lev 25:17  I skolen icke göra varandra orätt du skall frukta din Gud; ty jag är HERREN, eder Gud.
Lev 25:18  Och I skolen göra efter mina stadgar, och mina rätter skolen I hålla och skolen göra efter dem; då skolen I bo trygga i landet.
Lev 25:19  Och landet skall giva sin frukt, så att I haven nog att äta, och I skolen bo trygga däri.
Lev 25:20  Och om I frågen: "Vad skola vi äta under det sjunde året, om vi icke få så och icke få inbärga vår gröda?",
Lev 25:21  så mån I veta att jag skall bjuda min välsignelse komma över eder under det sjätte året, så att det giver gröda för de tre åren.
Lev 25:22  Och ännu när I under det åttonde året sån, skolen I hava av den gamla grödan att äta; ända till dess att grödan på det nionde året har kommit in, skolen I hava gammalt att äta.
Lev 25:23  När I säljen jord, skolen I icke sälja den för evärdlig tid, ty landet är mitt; I ären ju främlingar och gäster hos mig.
Lev 25:24  I hela det land I fån till besittning skolen I medgiva rätt att återbörda jordegendom.
Lev 25:25  Om din broder råkar i armod och säljer något av sin arvsbesittning, så må hans närmaste bördeman komma till honom och återbörda det brodern har sålt.
Lev 25:26  Och om någon icke har någon bördeman, men han själv kommer i tillfälle att anskaffa vad som behöves för att återbörda,
Lev 25:27  så skall han räkna efter, huru många år som hava förflutit ifrån försäljningen, och betala lösen för de återstående åren åt den man till vilken han sålde, och han skall så återfå sin besittning.
Lev 25:28  Men om han icke förmår anskaffa vad som behöves till att betala honom, så skall det han har sålt förbliva i köparens hand intill jubelåret. Men på jubelåret skall det frånträdas, och han skall då återfå sin besittning.
Lev 25:29  Om någon säljer ett boningshus i en stad som är omgiven med murar, så skall han hava rätt att återbörda det innan ett år har förflutit, sedan han sålde det; hans rätt att återbörda det är då inskränkt till viss tid.
Lev 25:30  Men om det icke har blivit återbördat, förrän hela året är ute, så skall huset, om det ligger i en stad som är omgiven med murar, förbliva köparens och hans efterkommandes egendom för evärdlig tid; det skall då icke frånträdas på jubelåret.
Lev 25:31  Men hus i sådana byar som icke hava murar omkring sig skola räknas till landets åkermark; de skola kunna återbördas, och på jubelåret skola de frånträdas.
Lev 25:32  Dock skola leviterna inom de städer som äro deras arvsbesittning hava evärdlig rätt att återbörda husen i städerna
Lev 25:33  Också om någon annan av leviterna inlöser det sålda huset i den stad där han har sin besittning, skall det dock frånträdas på jubelåret; ty husen i levitstäderna äro leviternas arvsbesittning bland Israels barn.
Lev 25:34  Och ett fält som är utmark omkring någon av deras städer får icke säljas, ty det är deras evärdliga besittning.
Lev 25:35  Om din broder råkar i armod och kommer på obestånd hos dig, så skall du taga dig an honom; såsom en främling eller en inhysesman skall han få leva hos dig.
Lev 25:36  Du skall icke ockra på honom eller taga ränta, ty du skall frukta din Gud, och du skall låta din broder leva hos dig.
Lev 25:37  Du skall icke lämna honom dina penningar på ocker eller lämna honom av dina livsmedel mot ränta.
Lev 25:38  Jag är HERREN, eder Gud, som har fört eder ut ur Egyptens land, för att giva eder Kanaans land och vara eder Gud.
Lev 25:39  Om din broder råkar i armod hos dig och säljer sig åt dig, skall du icke låta honom göra trälarbete;
Lev 25:40  såsom en daglönare och en inhysesman skall han vara hos dig; intill jubelåret skall han tjäna hos dig.
Lev 25:41  Då skall du giva honom fri, honom själv och hans barn med honom; och han skall återfå sin släktegendom, sin fädernebesittning skall han återfå.
Lev 25:42  Ty de äro mina tjänare, som jag har fört ut ur Egyptens land; de skola icke säljas såsom man säljer trälar.
Lev 25:43  Du skall icke med hårdhet bruka din makt över dem; du skall frukta din Gud.
Lev 25:44  Men om du vill skaffa dig en verklig träl eller trälinna, så skall du köpa en sådan träl eller trälinna från hedningarna som bo runt omkring eder.
Lev 25:45  I mån ock köpa sådana ibland barnen till inhysesmännen som bo hos eder och bland personer av deras släkt, som I haven hos eder, och som äro födda i edert land; sådana skola förbliva eder egendom.
Lev 25:46  Och dem mån I hava att lämna såsom arv åt edra barn efter eder, till egendom och besittning; dem kunnen I hava till trälar evärdligen. Men ibland edra bröder, Israels barn, skall ingen med hårdhet bruka sin makt över den andre.
Lev 25:47  Om en främling eller en inhysesman hos dig kommer till välstånd, och en din broder råkar i armod hos honom och säljer sig åt främlingen som bor inhyses hos dig, eller eljest åt någon som tillhör en främlingssläkt,
Lev 25:48  så skall han sedan, efter det att han har sålt sig, kunna lösas ut; någon av hans bröder må lösa honom;
Lev 25:49  eller ock må hans farbroder eller hans farbroders son lösa honom, eller må någon annan nära blodsförvant av hans släkt lösa honom; eller om han kommer i tillfälle därtill, må han själv lösa sig.
Lev 25:50  Därvid skall han, jämte den som har köpt honom, räkna efter, huru lång tid som har förflutit ifrån det år då han sålde sig åt honom till jubelåret; och det pris för vilket han såldes skall uppskattas efter årens antal; hans arbetstid hos honom skall beräknas till samma värde som en daglönares.
Lev 25:51  Om ännu många år äro kvar, skall han såsom lösen för sig betala en motsvarande del av det penningbelopp som han köptes för.
Lev 25:52  Om däremot allenast få år återstå till jubelåret, så skall han räkna efter detta, sig till godo, och betala lösen för sig efter antalet av sina år.
Lev 25:53  Såsom en daglönare som är lejd för år skall man behandla honom ingen må inför dina ögon med hårdhet bruka sin makt över honom.
Lev 25:54  Men om han icke bliver löst på något av de nämnda sätten, så skall han på jubelåret givas fri, han själv och hans barn med honom.
Lev 25:55  Ty Israels barn äro mina tjänare; de äro mina tjänare, som jag har fört ut ur Egyptens land. Jag är HERREN, eder Gud.
Lev 26:1  I skolen icke göra eder några av gudar, ej heller uppresa åt eder något beläte eller någon stod, eller uppsätta i edert land stenar med inhuggna bilder, för att tillbedja vid dem; ty jag är HERREN, eder Gud.
Lev 26:2  Mina sabbater skolen I hålla, och för min helgedom skolen I hava fruktan. Jag är HERREN.
Lev 26:3  Om I vandren efter mina stadgar och hållen mina bud och gören efter dem,
Lev 26:4  så skall jag giva eder regn i rätt tid, så att jorden giver sin gröda och träden på marken bära sin frukt.
Lev 26:5  Och trösktiden skall hos eder räcka intill vinbärgningen, och vinbärgningen skall räcka intill såningstiden, och I skolen hava bröd nog att äta och skolen bo trygga i edert land.
Lev 26:6  Och jag skall skaffa frid i landet, och I skolen få ro, och ingen skall förskräcka eder. Jag skall göra slut på vilddjuren i landet, och intet svärd skall gå fram genom edert land.
Lev 26:7  I skolen jaga edra fiender framför eder, och de skola falla för edra svärd.
Lev 26:8  Fem av eder skola jaga hundra framför sig, och hundra av eder skola jaga tiotusen, och edra fiender skola falla för edra svärd.
Lev 26:9  Och jag skall vända mig till eder och göra eder fruktsamma och för öka eder, och jag skall upprätthålla mitt förbund med eder.
Lev 26:10  Och gammal gröda, som länge har legat inne, skolen I hava att äta; I skolen nödgas skaffa den gamla undan för den nya.
Lev 26:11  Och jag skall uppresa min boning mitt ibland eder, och min själ skall icke försmå eder.
Lev 26:12  Jag skall vandra mitt ibland eder och vara eder Gud, och I skolen vara mitt folk.
Lev 26:13  Jag är HERREN, eder Gud, som förde eder ut ur Egyptens land, för att I icke skullen vara trälar där; och jag har brutit sönder edert ok och låtit eder gå med upprätt huvud.
Lev 26:14  Men om I icke hören mig och icke gören efter alla dessa bud,
Lev 26:15  om I förkasten mina stadgar, och om edra själar försmå mina rätter, så att I icke gören efter alla mina bud, utan bryten mitt förbund,
Lev 26:16  då skall ock jag handla på samma sätt mot eder: jag skall hemsöka eder med förskräckliga olyckor, med tärande sjukdom och feber, så att edra ögon försmäkta och eder själ förtvinar; och I skolen förgäves så eder säd, ty edra fiender skola äta den.
Lev 26:17  Jag skall vända mitt ansikte mot eder, och I skolen bliva slagna av edra fiender; och de som hata eder skola råda över eder, och I skolen fly, om ock ingen förföljer eder.
Lev 26:18  Om I, detta oaktat, icke hören mig, så skall jag tukta eder sjufalt värre för edra synders skull.
Lev 26:19  Jag skall krossa eder stolta makt. Jag skall låta eder himmel bliva såsom järn och eder jord såsom koppar.
Lev 26:20  Och eder möda skall vara förspilld, ty eder jord skall icke giva sin gröda, och träden i landet skola icke bära sin frukt.
Lev 26:21  Om I ändå vandren mig emot och icke viljen höra mig, så skall jag slå eder sjufalt värre, såsom edra synder förtjäna.
Lev 26:22  Jag skall sända över eder vilddjur, som skola döda edra barn och fördärva eder boskap och minska edert eget antal, så att edra vägar bliva öde.
Lev 26:23  Om I, detta oaktat, icke låten varna eder av mig, utan vandren mig emot,
Lev 26:24  så skall också jag vandra eder emot och slå eder sjufalt för edra synders skull.
Lev 26:25  Jag skall låta eder drabbas av ett hämndesvärd, som skall hämnas mitt förbund, och I skolen nödgas församla eder i städerna; men där skall jag sända pest bland eder, och I skolen bliva givna i fiendehand.
Lev 26:26  Jag skall så fördärva edert livsuppehälle, att edert bröd skall kunna bakas i en enda ugn av tio kvinnor, och edert bröd skall lämnas ut efter vikt, och när I äten, skolen I icke bliva mätta.
Lev 26:27  Om I, detta oaktat, icke hören mig, utan vandren mig emot,
Lev 26:28  så skall också jag i vrede vandra eder emot och tukta eder sjufalt för edra synders skull.
Lev 26:29  I skolen nödgas äta edra söners kött och äta edra döttrars kött.
Lev 26:30  Jag skall ödelägga edra offerhöjder och utrota edra solstoder; jag skall kasta edra döda kroppar på edra eländiga avgudars döda kroppar, ty min själ skall försmå eder.
Lev 26:31  Och jag skall göra edra städer till ruiner och föröda edra helgedomar, och jag skall icke mer med välbehag känna lukten av edra offer.
Lev 26:32  Jag skall själv ödelägga landet, så att edra fiender, som bo däri, skola häpna däröver.
Lev 26:33  Men eder skall jag förströ bland hedningarna, och jag skall förfölja eder med draget svärd; så skall edert land bliva en ödemark, och edra städer skola bliva ruiner.
Lev 26:34  Då skall landet få gottgörelse för sina sabbater, då, under hela den tid det ligger öde och I ären i edra fienders land. Ja, då skall landet hålla sabbat och giva gottgörelse för sina sabbater.
Lev 26:35  Hela den tid det ligger öde skall det hålla sabbat och få den vila det icke fick på edra sabbater, då I bodden däri.
Lev 26:36  Och åt dem som bliva kvar av eder skall jag giva försagda hjärtan i deras fienders länder, så att de jagas på flykten av ett prasslande löv som röres av vinden, och fly, såsom flydde de för svärd, och falla, om ock ingen förföljer dem.
Lev 26:37  Och de skola stupa på varandra, likasom för svärd, om ock ingen förföljer dem. Ja, I skolen icke kunna hålla stånd mot edra fiender.
Lev 26:38  I skolen förgås bland hedningarna, och edra fienders land skall förtära eder.
Lev 26:39  Och de som bliva kvar av eder skola försmäkta i edra fienders land, genom sin egen missgärning, och försmäkta tillika genom sina fäders missgärning, likasom dessa hava gjort.
Lev 26:40  Och de skola nödgas bekänna den missgärning de själva hava begått, och den deras fäder hava begått, genom att handla trolöst mot mig, och huru de hava vandrat mig emot
Lev 26:41  - varför också jag måste vandra dem emot och föra dem bort i deras fienders land - ja, då skola deras oomskurna hjärtan nödgas ödmjuka sig, då skola de få umgälla sin missgärning.
Lev 26:42  Och då skall jag tänka på mitt förbund med Jakob, då skall jag ock tänka på mitt förbund med Isak och på mitt förbund med Abraham, och på landet skall jag tänka.
Lev 26:43  Ty landet måste bliva övergivet av dem och så få gottgörelse för sina sabbater genom att bliva öde när folket är borta, och själva skola de få umgälla sin missgärning, därför, ja, därför att de förkastade mina rätter, och därför att deras själar försmådde mina stadgar.
Lev 26:44  Men detta oaktat skall jag, medan de äro i sina fienders land, icke så förkasta eller försmå dem, att jag förgör dem och bryter mitt förbund med dem; ty jag är HERREN, deras Gud.
Lev 26:45  Nej, till fromma för dem skall jag tänka på förbundet med förfäderna, som jag förde ut ur Egyptens land, inför hedningarnas ögon, på det att jag skulle vara deras Gud. Jag är HERREN.
Lev 26:46  Dessa äro de stadgar och rätter och lagar som HERREN fastställde mellan sig och Israels barn, på Sinai berg genom Mose.
Lev 27:1  Och HERREN talade till Mose och sade:
Lev 27:2  Tala till Israels barn och säg till dem: Om någon skall fullgöra ett löfte, ett sådant varvid du har att bestämma värdet på personer som lovas åt HERREN, så gäller följande:
Lev 27:3  Om värdet skall bestämmas för en man som är mellan tjugu och sextio år gammal, så skall du bestämma detta till femtio siklar silver, efter helgedomssikelns vikt.
Lev 27:4  Om frågan gäller en kvinna, så skall du bestämma värdet till trettio siklar.
Lev 27:5  Om frågan gäller någon som är mellan fem år och tjugu år gammal, så skall det värde du bestämmer vara för mankön tjugu siklar och för kvinnkön tio siklar.
Lev 27:6  Om frågan gäller någon som är mellan en månad och fem år gammal, så skall det värde du bestämmer vara för mankön fem siklar silver och för kvinnkön tre siklar silver.
Lev 27:7  Om frågan gäller någon som är sextio år gammal eller därutöver, så skall det värde du bestämmer vara, om det är en man, femton siklar, men för en kvinna skall det vara tio siklar.
Lev 27:8  Är någon i sådant armod att han icke kan betala det värde du bestämmer, så skall han ställas fram inför prästen, och prästen skall då bestämma ett värde för honom; efter vad den som har gjort löftet kan anskaffa skall prästen bestämma värdet för honom.
Lev 27:9  Om frågan gäller boskap, av de lag man får bära fram såsom offer åt HERREN, så skall allt sådant, när man har givit det åt HERREN, vara heligt;
Lev 27:10  man skall icke utväxla eller utbyta det, vare sig ett bättre mot ett sämre eller ett sämre mot ett bättre. Om någon likväl utbyter ett djur mot ett annat, så skall både det förra och det som har blivit lämnat i utbyte vara heligt.
Lev 27:11  Men om frågan gäller något slags orent djur, ett sådant som man icke får bära fram såsom offer åt HERREN, så skall djuret ställas fram inför prästen;
Lev 27:12  och prästen skall bestämma dess värde, alltefter som det är bättre eller sämre. Såsom du - prästen - bestämmer det, så skall det vara.
Lev 27:13  Och om ägaren vill lösa djuret, så skall han till det värde du har bestämt lägga femtedelen av värdet.
Lev 27:14  Om någon helgar sitt hus, för att det skall vara helgat åt HERREN, så skall prästen bestämma dess värde, alltefter som det är bättre eller sämre. Såsom prästen bestämmer dess värde, så skall det förbliva.
Lev 27:15  Och om den som har helgat sitt hus vill lösa det, så skall han till det värde i penningar du har bestämt lägga femtedelen därav; då bliver det hans.
Lev 27:16  Om någon helgar åt HERREN ett stycke åker av sin arvsbesittning så skall du bestämma dess värde efter utsädet därpå: mot var homer utsädeskorn skola svara femtio siklar silver.
Lev 27:17  Om han helgar sin åker ända från jubelåret, så skall det förbliva vid det värde du bestämmer.
Lev 27:18  Men om han helgar sin åker efter jubelåret, då skall prästen åt honom beräkna penningvärdet efter antalet av de år som återstå till nästa jubelår; och ett motsvarande avdrag skall göras på det värde du förut har bestämt.
Lev 27:19  Och om den som har helgat åkern vill lösa den, så skall han till det värde i penningar du har bestämt lägga femtedelen därav; då förbliver den hans.
Lev 27:20  Om han icke löser åkern, men säljer den åt någon annan, så får åkern sedan icke lösas,
Lev 27:21  utan när åkern frånträdes på jubelåret, skall den vara helgad åt HERREN, likasom en tillspillogiven åker; hans arvsbesittning tillfaller då prästen.
Lev 27:22  Om någon helgar åt HERREN en åker som han har köpt, en som icke hör till hans arvsbesittning,
Lev 27:23  så skall prästen åt honom räkna ut beloppet av det bestämda värdet intill jubelåret; och han skall samma dag erlägga detta värde, som du har bestämt; det skall vara helgat åt HERREN.
Lev 27:24  Men på jubelåret skall åkern återgå till den av vilken den har blivit köpt, och vilkens arvejord den är.
Lev 27:25  Och när du bestämmer något värde, skall det alltid bestämmas i helgedomssiklar, sikeln räknad till tjugu gera.
Lev 27:26  Men det som är förstfött ibland boskap, och som tillhör HERREN redan såsom förstfött, det skall ingen helga; vare sig det är ett djur av fäkreaturen eller ett djur av småboskapen, tillhör det redan HERREN
Lev 27:27  Men om frågan gäller något orent djur, så skall man lösa det efter det värde du bestämmer och lägga femtedelen av värdet därtill. Om det icke löses, så skall det säljas efter det värde du bestämmer.
Lev 27:28  Och om frågan gäller något tillspillogivet, vad någon har givit till spillo åt HERREN av sin egendom, det må vara en människa eller ett boskapsdjur eller den åker som är hans arvsbesittning, så får sådant varken säljas eller lösas; allt tillspillogivet är högheligt och tillhör HERREN.
Lev 27:29  En människa som har blivit tillspillogiven får aldrig lösas; en sådan måste dödas.
Lev 27:30  Och all tionde av jorden, vare sig av säden på jorden eller av trädens frukt, tillhör HERREN; den är helgad åt HERREN.
Lev 27:31  Om någon vill lösa något av sin tionde, så skall han lägga femtedelen av värdet därtill.
Lev 27:32  Och vad beträffar tionde av fäkreatur eller av småboskap, allt som går under herdestaven, så skall av allt detta vart tionde djur vara helgat åt HERREN;
Lev 27:33  man skall icke efterforska om det är bättre eller sämre, och man får icke utbyta det. Om någon likväl utbyter djuret, så skall både detta och det som har blivit lämnat i utbyte vara heligt; det får icke lösas.
Lev 27:34  Dessa äro de bud som HERREN på Sinai berg gav Israels barn genom Mose.


\end{document}