\begin{document}

\title{4 Moseboken}


\chapter{1}

\par 1 Och HERREN talade till Mose i Sinais öken, i uppenbarelsetältet, på första dagen i andra månaden av det andra året efter deras uttåg ur Egyptens land; han sade:
\par 2 "Räknen antalet av Israels barn, deras hela menighet, efter deras släkter och efter deras familjer, vart namn räknat särskilt, allt mankön, var person för sig;
\par 3 alla stridbara män i Israel, de män som äro tjugu år gamla eller därutöver, dem skolen I inmönstra efter deras häravdelningar, du och Aron.
\par 4 I skolen därvid taga till eder en man av var stam, den som är huvudman för sin stams familjer.
\par 5 Och dessa äro namnen på de män som skola biträda eder: av Ruben: Elisur, Sedeurs son;
\par 6 av Simeon: Selumiel, Surisaddais son;
\par 7 av Juda: Naheson, Amminadabs son;
\par 8 av Isaskar: Netanel, Suars son;
\par 9 av Sebulon: Eliab, Helons son;
\par 10 av Josefs barn: av Efraim: Elisama, Ammihuds son; av Manasse: Gamliel, Pedasurs son;
\par 11 av Benjamin: Abidan, Gideonis son;
\par 12 av Dan: Ahieser, Ammisaddais son;
\par 13 av Aser: Pagiel, Okrans son;
\par 14 av Gad: Eljasaf, Deguels son;
\par 15 av Naftali: Ahira, Enans son."
\par 16 Dessa voro ombud för menigheten, hövdingar för sina fädernestammar, huvudmän för Israels ätter.
\par 17 Och Mose och Aron togo till sig dessa namngivna män;
\par 18 och sedan de hade församlat hela menigheten på första dagen i andra månaden, blev folket infört i förteckningen efter sina släkter och efter sina familjer, vart namn räknat särskilt, de som voro tjugu år gamla eller därutöver, var person för sig,
\par 19 allt såsom HERREN hade bjudit Mose; och han mönstrade dem i Sinais öken.
\par 20 Och avkomlingarna av Rubens, Israels förstföddes, söner, upptecknade efter sina släkter och efter sina familjer, vart namn räknat särskilt, var person för sig, alla av mankön som voro tjugu år gamla eller därutöver, alla stridbara män,
\par 21 så många av Rubens stam som inmönstrades, utgjorde fyrtiosex tusen fem hundra.
\par 22 Avkomlingarna av Simeons söner, upptecknade efter sina släkter och efter sina familjer, så många som inmönstrades, vart namn räknat särskilt, var person för sig, alla av mankön som voro tjugu år gamla eller därutöver, alla stridbara män,
\par 23 så många av Simeons stam som inmönstrades, utgjorde femtionio tusen tre hundra.
\par 24 Avkomlingarna av Gads söner, upptecknade efter sina släkter och efter sina familjer, vart namn räknat särskilt, de som voro tjugu år gamla eller därutöver, alla stridbara män,
\par 25 så många av Gads stam som inmönstrades, utgjorde fyrtiofem tusen sex hundra femtio.
\par 26 Avkomlingarna av Judas söner, upptecknade efter sina släkter och efter sina familjer, vart namn räknat särskilt, de som voro tjugu år gamla eller därutöver, alla stridbara män,
\par 27 så många av Juda stam som inmönstrades, utgjorde sjuttiofyra tusen sex hundra.
\par 28 Avkomlingarna av Isaskars söner, upptecknade efter sina släkter och efter sina familjer, vart namn räknat särskilt, de som voro tjugu år gamla eller därutöver, alla stridbara män,
\par 29 så många av Isaskars stam son inmönstrades, utgjorde femtiofyra tusen fyra hundra.
\par 30 Avkomlingarna av Sebulons söner upptecknade efter sina släkter och efter sina familjer, vart namn räknat särskilt, de som voro tjugu år gamla eller därutöver, alla stridbara män,
\par 31 så många av Sebulons stam som inmönstrades, utgjorde femtiosju tusen fyra hundra.
\par 32 Avkomlingarna av Josefs söner: Avkomlingarna av Efraims söner, upptecknade efter sina släkter och efter sina familjer, vart namn räknat särskilt, de som voro tjugu år gamla eller därutöver, alla stridbara män,
\par 33 så många av Efraims stam som inmönstrades; utgjorde fyrtio tusen fem hundra.
\par 34 Avkomlingarna av Manasses söner, upptecknade efter sina släkter och efter sina familjer, vart namn räknat särskilt, de som voro tjugu år gamla eller därutöver, alla stridbara män,
\par 35 så många av Manasse stam som inmönstrades, utgjorde trettiotvå tusen två hundra.
\par 36 Avkomlingarna av Benjamins söner, upptecknade efter sina släkter och efter sina familjer, vart namn räknat särskilt, de som voro tjugu år gamla eller därutöver, alla stridbara män,
\par 37 så många av Benjamins stam som inmönstrades, utgjorde trettiofem tusen fyra hundra.
\par 38 Avkomlingarna av Dans söner, upptecknade efter sina släkter och efter sina familjer, vart namn räknat särskilt, de som voro tjugu år gamla eller därutöver, alla stridbara män,
\par 39 så många av Dans stam som inmönstrades, utgjorde sextiotvå tusen sju hundra.
\par 40 Avkomlingarna av Asers söner, upptecknade efter sina släkter och efter sina familjer, vart namn räknat särskilt, de som voro tjugu år gamla eller därutöver, alla stridbara män,
\par 41 så många av Asers stam som inmönstrades, utgjorde fyrtioett tusen fem hundra.
\par 42 Avkomlingarna av Naftalis söner, upptecknade efter sina släkter och efter sina familjer, vart namn räknat särskilt, de som voro tjugu år gamla eller därutöver, alla stridbara män,
\par 43 så många av Naftali stam som inmönstrades, utgjorde femtiotre tusen fyra hundra.
\par 44 Dessa voro de inmönstrade, de som blevo inmönstrade av Mose och Aron och Israels hövdingar, tolv män, som företrädde var och en sin stamfamilj.
\par 45 Och alla de av Israels barn som inmönstrades, efter deras familjer, de som voro tjugu år gamla eller därutöver, alla stridbara män i Israel,
\par 46 alla dessa inmönstrade utgjorde sex hundra tre tusen fem hundra femtio.
\par 47 Men leviterna i sin fädernestam blevo icke inmönstrade med de övriga.
\par 48 Ty HERREN talade till Mose och sade:
\par 49 Levi stam allenast skall du icke inmönstra, och du skall icke räkna antalet av dem med de övriga israeliterna;
\par 50 utan du skall förordna leviterna att förestå vittnesbördets tabernakel med alla dess redskap och alla dess tillbehör. De skola bära tabernaklet och alla dess redskap och göra tjänst därvid; och runt omkring tabernaklet skola de hava sitt läger.
\par 51 När tabernaklet skall bryta upp, skola leviterna nedtaga det, och när tabernaklet skall slås upp, skola leviterna uppsätta det; men om någon främmande kommer därvid, skall han dödas.
\par 52 De övriga israeliterna skola lägra sig var och en i sitt läger, och var och en under sitt baner, efter sina häravdelningar;
\par 53 men leviterna skola lägra sig runt omkring vittnesbördets tabernakel, för att icke förtörnelse må komma över Israels barns menighet; och leviterna skola iakttaga vad som är att iakttaga vid vittnesbördets tabernakel.
\par 54 Och Israels barn gjorde så; de gjorde i alla stycken såsom HERREN hade bjudit Mose.

\chapter{2}

\par 1 Och HERREN talade till Mose och Aron och sade:
\par 2 Israels barn skola lägra sig var och en under sitt baner, vid de fälttecken som höra till deras särskilda familjer; runt omkring uppenbarelsetältet skola de lägra sig så, att de hava det framför sig.
\par 3 På framsidan, österut, skall Juda lägra sig under sitt baner, efter sina häravdelningar: Juda barns hövding Naheson, Amminadabs son
\par 4 med de inmönstrade som utgöra hans här, sjuttiofyra tusen sex hundra man.
\par 5 Bredvid honom skall Isaskars stam lägra sig: Isaskars barns hövding Netanel, Suars son,
\par 6 med de inmönstrade som utgöra hans här, femtiofyra tusen fyra hundra man.
\par 7 Därnäst Sebulons stam: Sebulons barns hövding Eliab, Helons son,
\par 8 med de inmönstrade som utgöra hans här, femtiosju tusen fyra hundra man.
\par 9 De inmönstrade som tillhöra Juda läger utgöra alltså tillsammans ett hundra åttiosex tusen fyra hundra man, delade i sina häravdelningar. De skola vid uppbrott tåga främst.
\par 10 Ruben skall lägra sig under sitt baner söderut, efter sina häravdelningar: Rubens barns hövding Elisur, Sedeurs son,
\par 11 med de inmönstrade som utgöra hans här, fyrtiosex tusen fem hundra man.
\par 12 Bredvid honom skall Simeons stam lägra sig: Simeons barns hövding Selumiel, Surisaddais son,
\par 13 med de inmönstrade som utgöra hans här, femtionio tusen tre hundra man.
\par 14 Därnäst Gads stam: Gads barns hövding Eljasaf, Reguels son
\par 15 med de inmönstrade som utgöra hans här, fyrtiofem tusen sex hundra femtio man.
\par 16 De inmönstrade som tillhöra Rubens läger utgöra alltså tillsammans ett hundra femtioett tusen fyra hundra femtio man, delade i sina häravdelningar. Och de skola vid uppbrott tåga i andra rummet.
\par 17 Sedan skall uppenbarelsetältet med leviternas läger hava sin plats i tåget, mitt emellan de övriga lägren. I den ordning de lägra sig skola de ock tåga, var och en på sin plats, under sina baner.
\par 18 Efraim skall lägra sig under sitt baner västerut, efter sina häravdelningar: Efraims barns hövding Elisama, Ammihuds son,
\par 19 med de inmönstrade som utgöra hans här, fyrtio tusen fem hundra man.
\par 20 Bredvid honom skall Manasse stam lägra sig: Manasse barns hövding Gamliel, Pedasurs son,
\par 21 med de inmönstrade som utgöra hans här, trettiotvå tusen två hundra man.
\par 22 Därnäst Benjamins stam: Benjamins barns hövding Abidan, Gideonis son,
\par 23 med de inmönstrade som utgöra hans här, trettiofem tusen fyra hundra man.
\par 24 De inmönstrade som tillhöra Efraims läger utgöra alltså tillsammans ett hundra åtta tusen ett hundra man, delade i sina häravdelningar. Och de skola vid uppbrott tåga i tredje rummet.
\par 25 Dan skall lägra sig under sitt baner norrut, efter sina häravdelningar: Dans barns hövding Ahieser, Ammisaddais son,
\par 26 med de inmönstrade som utgöra hans här, sextiotvå tusen sju hundra man.
\par 27 Bredvid honom skall Asers stam lägra sig: Asers barns hövding Pagiel, Okrans son,
\par 28 med de inmönstrade som utgöra hans har, fyrtioett tusen fem hundra man.
\par 29 Därnäst Naftali stam: Naftali barns hövding Ahira, Enans son,
\par 30 med de inmönstrade som utgöra hans här, femtiotre tusen fyra hundra man.
\par 31 De inmönstrade som tillhöra Dans läger utgöra alltså tillsammans ett hundra femtiosju tusen sex hundra man. De skola vid uppbrott tåga sist, under sina baner.
\par 32 Dessa voro, efter sina familjer, de av Israels barn som inmönstrades. De som inmönstrades i lägren, efter sina häravdelningar, utgjorde tillsammans sex hundra tre tusen fem hundra femtio man.
\par 33 Men leviterna blevo icke inmönstrade med de övriga israeliterna, ty så hade HERREN: bjudit Mose.
\par 34 Och Israels barn gjorde så; alldeles så, som HERREN hade bjudit Mose, lägrade de sig under sina baner, och så tågade de ock, var och en i sin släkt, efter sin familj.

\chapter{3}

\par 1 Detta är berättelsen om Arons och Moses släkt, vid den tid då HERREN talade med Mose på Sinai berg.
\par 2 Dessa äro namnen på Arons söner: Nadab, den förstfödde, och Abihu, Eleasar och Itamar.
\par 3 Dessa voro namnen på Arons söner, de smorda prästerna, som hade mottagit handfyllning till att vara präster.
\par 4 Men Nadab och Abihu föllo döda ned inför HERRENS ansikte, när de framburo främmande eld inför HERRENS ansikte i Sinais öken; och de hade inga söner. Sedan voro Eleasar och Itamar präster under sin fader Aron.
\par 5 Och HERREN talade till Mose och sade:
\par 6 Levi stam skall du låta få tillträde hit; du skall låta dem stå inför prästen Aron för att betjäna honom.
\par 7 De skola iakttaga vad han har att iakttaga, och vad hela menigheten har att iakttaga, inför uppenbarelsetältet, i det att de förrätta tjänsten vid tabernaklet.
\par 8 Och de skola hava vården om alla uppenbarelsetältets tillbehör, och iakttaga vad Israels barn hava att iakttaga, i det att de förrätta tjänsten vid tabernaklet.
\par 9 Alltså skall du giva leviterna åt Aron och hans söner; de skola vara honom givna såsom gåva av Israels barn.
\par 10 Men Aron och hans söner skall du anbefalla att iakttaga vad som hör till deras prästämbete. Om någon främmande kommer därvid, skall han dödas.
\par 11 Och HERREN talade till Mose och sade:
\par 12 Se, jag har själv bland Israels barn uttagit leviterna i stället för allt förstfött bland Israels barn, allt som öppnar moderlivet, så att leviterna skola tillhöra mig.
\par 13 Ty mig tillhör allt förstfött; på den dag då jag slog allt förstfött i Egyptens land helgade jag åt mig allt förstfött i Israel, såväl människor som boskap. Mig skola de tillhöra. Jag är HERREN.
\par 14 Och HERREN talade till Mose i Sinais öken och sade:
\par 15 Mönstra Levi barn, efter deras familjer och efter deras släkter; alla av mankön som äro en månad gamla eller därutöver skall du inmönstra.
\par 16 och Mose inmönstrade dem efter HERRENS befallning, såsom honom hade blivit bjudet.
\par 17 Och dessa voro Levis söner, efter deras namn: Gerson, Kehat och Merari.
\par 18 Och dessa voro namnen på Gersons söner, efter deras släkter: Libni och Simei.
\par 19 Och Kehats söner efter sina släkter voro Amram och Jishar, Hebron och Ussiel.
\par 20 Och Meraris söner efter sina släkter voro Maheli och Musi. Dessa voro leviternas släkter, efter deras familjer.
\par 21 Från Gerson härstammade libniternas släkt och simeiternas släkt; dessa voro gersoniternas släkter.
\par 22 De av dem som inmönstrades, i det att man räknade alla dem av mankön som voro en månad gamla eller därutöver dessa inmönstrade utgjorde sju tusen fem hundra.
\par 23 Gersoniternas släkter hade sitt läger bakom tabernaklet, västerut.
\par 24 Och hövding för gersoniternas stamfamilj var Eljasaf, Laels son.
\par 25 Och Gersons barn skulle vid uppenbarelsetältet hava vården om själva tabernaklet och dess täckelse, om dess överdrag och om förhänget för ingången till uppenbarelsetältet,
\par 26 vidare om förgårdens omhängen och om förhänget för ingången till förgården, som omgav tabernaklet och altaret, så ock om dess streck - vad arbete nu kunde förekomma därvid.
\par 27 Från Kehat härstammade amramiternas släkt, jishariternas släkt hebroniternas släkt och ussieliternas släkt; dessa voro kehatiternas släkter.
\par 28 När man räknade alla dem av mankön som voro en månad gamla eller därutöver, utgjorde de åtta tusen sex hundra; dessa voro de som skulle hava vården om de heliga föremålen.
\par 29 Kehats barns släkter hade sitt läger vid sidan av tabernaklet, söderut.
\par 30 Och hövding för de kehatitiska släkternas stamfamilj; var Elisafan, Ussiels son.
\par 31 De skulle hava vården om arken, bordet, ljusstaken, altarna och de tillbehör till de heliga föremålen, som begagnades vid gudstjänsten, så ock om förhänget och om allt arbete därvid.
\par 32 Men överhövding över alla leviterna var Eleasar, prästen Arons son; han var förman för dem som skulle hava vården om de heliga föremålen.
\par 33 Från Merari härstammade maheliternas släkt och musiternas släkt; dessa voro merariternas släkter.
\par 34 Och de av dem som inmönstrades, i det att man räknade alla dem av mankön som voro en månad gamla eller därutöver, utgjorde sex tusen två hundra.
\par 35 Och hövding för de meraritiska släkternas stamfamilj var Suriel, Abihails son. De hade sitt läger vid sidan av tabernaklet, norrut.
\par 36 Och Meraris barn fingo till åliggande att hava vården om bräderna till tabernaklet, om dess tvärstänger, stolpar och fotstycken och om alla dess tillbehör och om allt arbete därvid,
\par 37 så ock om stolparna till förgården runt omkring med deras fotstycken, deras pluggar och streck.
\par 38 Men mitt för tabernaklet, på framsidan, mitt för uppenbarelsetältet, österut, hade Mose och Aron och hans söner sitt läger; dessa skulle iakttaga vad som var att iakttaga vid helgedomen, vad Israels barn hade att iakttaga; men om någon främmande kom därvid, skulle han dödas.
\par 39 De inmönstrade av leviterna som Mose och Aron inmönstrade efter deras släkter, enligt HERRENS befallning, alla av mankön som voro en månad gamla eller därutöver, utgjorde tillsammans tjugutvå tusen.
\par 40 Och HERREN sade till Mose: Mönstra allt förstfött av mankön bland Israels barn, alla som äro en månad gamla eller därutöver, och räkna antalet av deras namn.
\par 41 Och tag ut åt mig - ty jag är HERREN - leviterna i stället för allt förstfött bland Israels barn, så ock leviternas boskap i stället för allt förstfött bland Israels barns boskap.
\par 42 Och Mose mönstrade allt förstfött bland Israels barn, såsom HERREN hade bjudit honom.
\par 43 Och de förstfödde av mankön, vart namn räknat särskilt, de som voro en månad gamla eller därutöver, utgjorde, så många som inmönstrades, tillsammans tjugutvå tusen två hundra sjuttiotre.
\par 44 Och HERREN talade till Mose och sade:
\par 45 Du skall uttaga leviterna i stället för allt förstfött bland Israels barn, så ock leviternas boskap i stället för dessas boskap; så att leviterna skola tillhöra mig. Jag är HERREN.
\par 46 Men till lösen för de två hundra sjuttiotre personer med vilka antalet av Israels barns förstfödde överstiger leviternas antal,
\par 47 skall du taga fem siklar för var person; du skall taga upp dessa efter helgedomssikelns vikt, sikeln räknad till tjugu gera.
\par 48 Och du skall giva penningarna åt Aron och hans söner såsom lösen för de övertaliga bland folket.
\par 49 Och Mose tog lösesumman av dem som voro övertaliga, när man räknade dem som voro lösta genom leviterna.
\par 50 Av Israels barns förstfödde tog han penningarna, ett tusen tre hundra sextiofem siklar, efter helgedomssikelns vikt.
\par 51 Och Mose gav lösesumman åt Aron och hans söner, efter HERRENS befallning, såsom HERREN hade bjudit Mose.

\chapter{4}

\par 1 Och HERREN talade till Mose och Aron och sade:
\par 2 Räknen bland Levi barn antalet av Kehats barn, efter deras släkter och efter deras familjer,
\par 3 dem som äro trettio år gamla eller därutöver, ända till femtio år, alla tjänstbara män som kunna förrätta sysslor vid uppenbarelsetältet.
\par 4 Och detta skall vara Kehats barns tjänstgöring vid uppenbarelsetältet: de skola hava hand om de högheliga föremålen.
\par 5 När lägret skall bryta upp, skola Aron och hans söner gå in och taga ned den förlåt som hänger framför arken och med den övertäcka vittnesbördets ark;
\par 6 däröver skola de lägga ett överdrag av tahasskinn och över detta ytterligare breda ett kläde, helt och hållet mörkblått; sedan skola de sätta in stängerna.
\par 7 Och över skådebrödsbordet skola de breda ett mörkblått kläde och ställa därpå faten, skålarna och bägarna, ävensom kannorna till drickoffren; "det beständiga brödet" skall ock läggas därpå.
\par 8 Häröver skola de breda ett rosenrött kläde och betäcka detta med ett överdrag av tahasskinn; sedan skola de sätta in stängerna.
\par 9 Och de skola taga ett mörkblått kläde och därmed övertäcka ljusstaken och dess lampor, lamptänger och brickor, så ock alla tillhörande oljekärl som begagnas under tjänstgöringen därvid;
\par 10 och de skola lägga den med alla dess tillbehör i ett överdrag av tahasskinn och sedan lägga alltsammans på en bår.
\par 11 Över det gyllene altaret skola de likaledes breda ett mörkblått kläde och betäcka detta med ett överdrag av tahasskinn, sedan skola de sätta in stängerna.
\par 12 Och de skola taga alla gudstjänstredskap, som begagnas vid tjänstgöringen i helgedomen, och lägga dem i ett mörkblått kläde och betäcka dem med ett överdrag av tahasskinn, och sedan lägga dem på en bår.
\par 13 Och de skola taga bort askan från altaret och breda över det ett purpurrött kläde
\par 14 och lägga därpå alla tillbehör som begagnas under tjänstgöringen därvid, fyrfaten, gafflarna, skovlarna och skålarna, korteligen, altarets alla tillbehör; och däröver skola de breda ett överdrag av tahasskinn och så sätta in stängerna.
\par 15 Sedan nu Aron och hans söner, när lägret skall bryta upp, så hava övertäckt de heliga föremålen och alla tillbehör till dessa heliga föremål, skola därefter Kehats barn komma för att bära; men de må icke röra vid de heliga föremålen, ty då skola de dö. Detta är vad Kehats barn hava att bära av det som hör till uppenbarelsetältet.
\par 16 Och Eleasars, prästen Arons sons åliggande skall vara att hava vården om oljan till ljusstaken, om den välluktande rökelsen, om det dagliga spisoffret och om smörjelseoljan; hans åliggande skall vara att hava vården om hela tabernaklet och om allt vad däri är, de heliga föremålen och deras tillbehör.
\par 17 Och HERREN talade till Mose och Aron och sade:
\par 18 Låten icke kehatiternas släktgren utrotas ur leviternas stam.
\par 19 Utan gören på följande sätt med dem, för att de må leva och icke dö, när de nalkas de högheliga föremålen: Aron och hans söner skola gå in och anvisa var och en av dem vad han har att göra eller bära
\par 20 men själva må de icke gå in och se de heliga föremålen, icke ens ett ögonblick, ty då skola de dö.
\par 21 Och HERREN talade till Mose och sade:
\par 22 Räkna ock antalet av Gersons barn, efter deras familjer och efter deras släkter.
\par 23 Dem som äro trettio år gamla eller därutöver, ända till femtio år, skall du inmönstra, alla tjänstbara män som kunna förrätta arbete vid uppenbarelsetältet.
\par 24 Detta skall vara gersoniternas släkters tjänstgöring, vad de hava att göra och vad de hava att bära:
\par 25 de skola bära de tygvåder av vilka tabernaklet bildas, uppenbarelsetältets täckelse, dess överdrag och det överdrag av tahasskinn som ligger ovanpå detta, förhänget för ingången till uppenbarelsetältet,
\par 26 vidare omhängena till förgården, förhänget för porten till förgården, som omgiver tabernaklet och altaret, så ock tillhörande streck och alla redskap till arbetet därvid; och allt som härvid är att göra skola de förrätta.
\par 27 På det sätt Aron och hans söner bestämma skall Gersons barns hela tjänstgöring försiggå, i fråga om allt vad de hava att bära och göra; och I skolen överlämna i deras vård allt vad de hava att bära.
\par 28 Detta är den tjänstgöring som Gersons barns släkter skola hava vid uppenbarelsetältet; och vad de hava att iakttaga skola de utföra under ledning av Itamar, prästen Arons son
\par 29 Meraris barn skall du inmönstra, efter deras släkter och efter deras familjer.
\par 30 Dem som äro trettio år gamla eller därutöver, ända till femtio år, skall du inmönstra, alla tjänstbara män som kunna förrätta arbete vid uppenbarelsetältet.
\par 31 Och detta är vad som skall åligga dem att bära, allt vad som hör till deras tjänstgöring vid uppenbarelsetältet: bräderna till tabernaklet, dess tvärstänger, stolpar och fotstycken,
\par 32 så ock stolparna till förgården runt omkring med deras fotstycken, pluggar och streck, korteligen, alla deras tillbehör och allt som hör till arbetet därvid; och I skolen lämna dem uppgift på de särskilda föremål som det åligger dem att bära.
\par 33 Detta skall vara de meraritiska släkternas tjänstgöring, allt vad som hör till deras tjänstgöring vid uppenbarelsetältet; och det skall utföras under ledning av Itamar, prästen Arons son.
\par 34 Och Mose och Aron och menighetens hövdingar inmönstrade Kehats barn, efter deras släkter och efter deras familjer,
\par 35 dem som voro trettio år gamla eller därutöver, ända till femtio år alla tjänstbara män som kunde göra arbete vid uppenbarelsetältet.
\par 36 Och de av dem som inmönstrades, efter deras släkter, utgjorde två tusen sju hundra femtio.
\par 37 Så många voro de av kehatiternas släkter som inmönstrades, summan av dem som skulle göra tjänst vid uppenbarelsetältet, de som Mose och Aron inmönstrade, efter HERRENS befallning genom Mose.
\par 38 Och de av Gersons barn som inmönstrades, efter deras släkter och efter deras familjer,
\par 39 de som voro trettio år gamla eller därutöver, ända till femtio år, alla tjänstbara män som kunde göra arbete vid uppenbarelsetältet,
\par 40 dessa som inmönstrades efter sina släkter och efter sina familjer utgjorde två tusen sex hundra trettio.
\par 41 Så många voro de av Gersons barns släkter som inmönstrades, summan av dem som skulle göra tjänst vid uppenbarelsetältet, de som Mose och Aron inmönstrade, efter HERRENS befallning.
\par 42 Och de av Meraris barns släkter som inmönstrades, efter deras släkter och efter deras familjer,
\par 43 de som voro trettio år gamla eller därutöver, ända till femtio år, alla tjänstbara män som kunde göra arbete vid uppenbarelsetältet,
\par 44 dessa som inmönstrades efter sina släkter utgjorde tre tusen två hundra.
\par 45 Så många voro de av Meraris barns släkter som inmönstrades, de som Mose och Aron inmönstrade efter HERRENS befallning genom Mose.
\par 46 De av leviterna som Mose och Aron och Israels hövdingar inmönstrade, efter deras släkter och efter deras familjer,
\par 47 de som voro trettio år gamla eller därutöver, ända till femtio år, alla som kunde förrätta något tjänstgöringsarbete eller något bärarearbete vid uppenbarelsetältet,
\par 48 dessa som inmönstrades utgjorde tillsammans åtta tusen fem hundra åttio.
\par 49 Efter HERRENS befallning blevo de inmönstrade genom Mose, var och en till det som han hade att göra eller bära, och var och en fick det åliggande som HERREN hade bjudit Mose.

\chapter{5}

\par 1 Och HERREN talade till Mose och sade:
\par 2 Bjud Israels barn att de skaffa bort ur lägret var och en som är spetälsk eller har flytning och var och en som har blivit oren genom någon död.
\par 3 En sådan, vare sig man eller kvinna, skolen I skaffa bort; till något ställe utanför lägret skolen I skicka honom, för att han icke må orena deras läger; jag har ju min boning mitt ibland dem.
\par 4 Och Israels barn gjorde så; de skickade dem till ett ställe utanför lägret; såsom HERREN hade tillsagt Mose, så gjorde Israels barn.
\par 5 Och HERREN talade till Mose och sade:
\par 6 Tala till Israels barn: Om någon, vare sig man eller kvinna, begår någon synd - vad det nu må vara, vari en människa kan försynda sig - i det han gör sig skyldig till en orättrådighet mot HERREN, och denna person alltså ådrager sig skuld,
\par 7 så skall han bekänna den synd han har begått, och ersätta det han har förbrutit sig på till dess fulla belopp och lägga femtedelen av värdet därtill; och detta skall han giva åt den som han har förbrutit sig emot.
\par 8 Men om denne icke har efterlämnat någon bördeman, åt vilken ersättning kan givas för det han har förbrutit sig på, då skall ersättningen för detta givas åt HERREN och tillhöra prästen, utom försoningsväduren, med vilken försoning bringas för den skyldige.
\par 9 Och alla heliga gåvor som Israels barn giva såsom en gärd, vilken de bära fram till prästen, skola tillhöra denne;
\par 10 honom skola allas heliga gåvor tillhöra; vad någon giver åt prästen skall tillhöra denne.
\par 11 Och HERREN talade till Mose och sade:
\par 12 Tala till Israels barn och säg till dem: Om en hustru har svikit sin man och varit honom otrogen,
\par 13 i det att någon annan har legat hos henne och beblandat sig med henne, utan att hennes man har fått veta därav, och utan att hon har blivit röjd, fastän hon verkligen har låtit skända sig; om alltså intet vittne finnes mot henne och hon icke har blivit gripen på bar gärning,
\par 14 men misstankens ande likväl kommer över honom, så att han får misstanke mot sin hustru, och det verkligen är så, att hon har låtit skända sig; eller om misstankens ande kommer över honom, så att han får misstanke mot sin hustru, och detta fastän hon icke har låtit skända sig:
\par 15 så skall mannen föra sin hustru till prästen och såsom offer för henne bära fram en tiondedels efa kornmjöl, men ingen olja skall han gjuta därpå och ingen rökelse lägga därpå, ty det är ett misstankeoffer, ett åminnelseoffer, som bringar en missgärning i åminnelse.
\par 16 Och prästen skall föra henne fram och ställa henne inför HERRENS ansikte.
\par 17 Och prästen skall taga heligt vatten i ett lerkärl, och sedan skall prästen taga något av stoftet på tabernaklets golv och lägga i vattnet.
\par 18 Och prästen skall ställa kvinnan fram inför HERRENS ansikte och lösa upp kvinnans hår och lägga på hennes händer åminnelseoffret, det är misstankeoffret; men prästen själv skall hålla i sin hand det förbannelsebringande olycksvattnet.
\par 19 Därefter skall prästen besvärja kvinnan och säga till henne: "Om ingen har lägrat dig och du icke har svikit din man genom att låta skända dig, så må detta förbannelsebringande olycksvatten icke skada dig.
\par 20 Men om du har svikit din man och låtit skända dig, i det att någon annan än din man har beblandat sig med dig"
\par 21 (prästen besvärjer nu kvinnan med förbannelsens ed, i det han säger till kvinnan:) "Då må HERREN göra dig till ett exempel som man nämner, när man förbannar och svär bland ditt folk; HERREN må då låta din länd förvissna och din buk svälla upp;
\par 22 ja, när du har fått detta förbannelsebringande vatten in i ditt liv, då må det komma din buk att svälla upp och din länd att förvissna." Och kvinnan skall säga: "Amen, amen."
\par 23 Sedan skall prästen skriva upp dessa förbannelser på ett blad och därefter avtvå dem i olycksvattnet
\par 24 och giva kvinnan det förbannelsebringande olycksvattnet att dricka, för att detta förbannelsebringande vatten må bliva henne till olycka, när hon har fått det i sig.
\par 25 Och prästen skall taga misstankeoffret ur kvinnans hand och vifta detta offer inför HERRENS ansikte och bära det fram till altaret.
\par 26 Och prästen skall av offret taga en handfull, det som utgör själva altaroffret, och förbränna det på altaret; därefter skall han giva kvinnan vattnet att dricka.
\par 27 Och när han så har givit henne vattnet att dricka, då skall detta ske: om hon har låtit skända sig och varit sin man otrogen, så skall det förbannelsebringande vattnet, när hon har fått det i sig, bliva henne till olycka, i det att hennes buk sväller upp och hennes länd förvissnar; och kvinnan skall bliva ett exempel som man nämner, när man förbannar bland hennes folk.
\par 28 Men om kvinnan icke har låtit skända sig, utan är ren, då skall hon förbliva oskadd och kunna undfå livsfrukt.
\par 29 Detta är misstankelagen, om huru förfaras skall, när en kvinna har svikit sin man och låtit skända sig,
\par 30 eller när eljest misstankens ande kommer över en man, så att han misstänker sin hustru; han skall då ställa hustrun fram inför HERRENS ansikte, och prästen skall med henne göra allt vad denna lag stadgar.
\par 31 Så skall mannen vara fri ifrån missgärning, men hustrun kommer att bära på missgärning.

\chapter{6}

\par 1 Och HERREN talade till Mose och sade:
\par 2 Tala till Israels barn och säg till dem: Om någon, vare sig man eller kvinna, har att fullgöra ett nasirlöfte, ett löfte att vara HERRENS nasir,
\par 3 så skall han avhålla sig från vin och starka drycker; han skall icke dricka någon syrad dryck av vin eller någon annan syrad stark dryck; intet slags druvsaft skall han dricka, ej heller skall han äta druvor, vare sig friska eller torra.
\par 4 Så länge hans nasirtid varar, skall han icke äta något som kommer av vinträdet, icke ens dess kartar eller späda skott.
\par 5 Så länge hans nasirlöfte varar, skall ingen rakkniv komma på hans huvud; till dess att den tid är ute, under vilken han skall vara HERRENS nasir, skall han vara helig och låte håret växa långt på sitt huvud.
\par 6 Så länge han är HERRENS nasir, skall han icke nalkas någon död.
\par 7 Icke ens genom sin fader eller sin moder, sin broder eller sin syster får han ådraga sig orenhet, om de dö ty han bär på sitt huvud tecknet till att han är sin Guds nasir;
\par 8 så länge hans nasirtid varar, är han helgad åt HERREN.
\par 9 Men om någon oförtänkt och plötsligt dör i hans närhet, och därmed orenar hans huvud, på vilket han bär nasirtecknet, så skall han raka sitt huvud den dag han bliver ren; han skall raka det på sjunde dagen.
\par 10 Och på åttonde dagen skall han bära fram till prästen två turturduvor eller två unga duvor, till uppenbarelsetältets ingång.
\par 11 Och prästen skall offra en till syndoffer och en till brännoffer och bringa försoning för honom, till rening från den synd han har dragit över sig genom den döde; sedan skall han samma dag åter helga sitt huvud;
\par 12 han skall inviga sig till nasir åt HERREN för lika lång tid som han förut hade lovat. Och han skall föra fram ett årsgammalt lamm till skuldoffer. Den förra löftestiden skall vara ogill, därför att hans nasirat blev orenat.
\par 13 Och detta är lagen om en nasir: Den dag hans nasirtid är ute skall han föras fram till uppenbarelsetältets ingång;
\par 14 och han skall såsom sitt offer åt HERREN frambära ett årsgammalt felfritt lamm av hankön till brännoffer och ett årsgammalt felfritt lamm av honkön till syndoffer och en felfri vädur till tackoffer,
\par 15 därjämte en korg med osyrat bröd, kakor av fint mjöl, begjutna med olja, och osyrade tunnkakor, smorda med olja, så ock tillhörande spis offer och drickoffer.
\par 16 Och prästen skall bära fram detta inför HERRENS ansikte och offra hans syndoffer och hans brännoffer.
\par 17 Och väduren skall han offra till tackoffer åt HERREN, jämte korgen med de osyrade bröden; prästen skall ock offra tillhörande spisoffer och drickoffer.
\par 18 Och nasiren skall vid ingången till uppenbarelsetältet raka sitt huvud, på vilket han bär nasirtecknet, och taga sitt huvudhår, sitt nasirtecken, och lägga det på elden som brinner under tackoffret.
\par 19 Och prästen skall taga den kokta vädursbogen, och därjämte ur korgen en osyrad kaka och en osyrad tunnkaka, och lägga detta på nasirens händer, sedan denne har rakat av sig nasirtecknet.
\par 20 Och prästen skall vifta detta såsom ett viftoffer inför HERRENS ansikte; det skall vara helgat åt prästen, jämte viftoffersbringan och offergärdslåret. Sedan får nasiren åter dricka vin.
\par 21 Detta är lagen om den som har avlagt ett nasirlöfte, och om vad han på grund av nasirlöftet skall offra åt HERREN, förutom vad han eljest kan anskaffa; efter innehållet i det löfte han har avlagt skall han göra, enligt lagen om hans nasirat.
\par 22 Och HERREN talade till Mose och sade:
\par 23 Tala till Aron och hans söner och säg: När I välsignen Israels barn, skolen I säga så till dem:
\par 24 HERREN välsigne dig och bevare dig.
\par 25 HERREN låte sitt ansikte lysa över dig och vare dig nådig.
\par 26 HERREN vände sitt ansikte till dig och give dig frid.
\par 27 Så skola de lägga mitt namn på Israels barn, och jag skall då välsigna dem.

\chapter{7}

\par 1 Då nu Mose hade satt upp tabernaklet och smort och helgat det, med alla dess tillbehör, och hade satt upp altaret med alla dess tillbehör, och smort och helgat detta,
\par 2 framburos offergåvor av Israels hövdingar, huvudmännen för stamfamiljerna, det är stamhövdingarna, som stodo i spetsen för de inmönstrade.
\par 3 De förde fram såsom sin offergåva inför HERRENS ansikte sex övertäckta vagnar och tolv oxar: två hövdingar tillhopa en vagn och var hövding en oxe; dessa förde de fram inför tabernaklet.
\par 4 Och HERREN sade till Mose:
\par 5 "Tag emot detta av dem för att bruka det till uppenbarelsetältets tjänst; och lämna det åt leviterna, alltefter beskaffenheten av vars och ens tjänst."
\par 6 Och Mose tog emot vagnarna och oxarna och gav dem åt leviterna.
\par 7 Två vagnar och fyra oxar gav han åt Gersons barn, efter beskaffenheten av deras tjänst;
\par 8 fyra vagnar och åtta oxar gav han åt Meraris barn, efter beskaffenheten av den tjänst de förrättade under ledning av Itamar, prästen Arons son;
\par 9 men åt Kehats barn gav han icke något, ty dem ålåg att hava hand om de heliga föremålen, och dessa skulle bäras på axlarna.
\par 10 Och hövdingarna förde fram skänker till altarets invigning, när det smordes; hövdingarna förde fram dessa sina offergåvor inför altaret.
\par 11 Och HERREN sade till Mose: "Låt hövdingarna, en i sänder, var och en på sin dag, föra fram sina offergåvor till altarets invigning."
\par 12 Och den som på första dagen förde fram sin offergåva var Naheson, Amminadabs son, av Juda stam
\par 13 Hans offergåva var ett silverfat, ett hundra trettio siklar i vikt, och en silverskål om sjuttio siklar, efter helgedomssikelns vikt, båda fulla med fint mjöl, begjutet med olja, till spisoffer,
\par 14 vidare en skål av guld om tio siklar, full med rökelse,
\par 15 vidare en ungtjur, en vädur och ett årsgammalt lamm till brännoffer
\par 16 och en bock till syndoffer,
\par 17 samt till tackoffret två tjurar, fem vädurar, fem bockar och fem årsgamla lamm. Detta var Nahesons, Amminadabs sons, offergåva.
\par 18 På andra dagen förde Netanel, Suars son, hövdingen för Isaskar, fram sin gåva;
\par 19 han framförde såsom sin offergåva ett silverfat, ett hundra trettio siklar i vikt, och en silverskål om sjuttio siklar, efter helgedomssikelns vikt, båda fulla med fint mjöl, begjutet med olja, till spisoffer,
\par 20 vidare en skål av guld om tio siklar, full med rökelse,
\par 21 vidare en ungtjur, en vädur och ett årsgammalt lamm till brännoffer
\par 22 och en bock till syndoffer,
\par 23 samt till tackoffret två tjurar, fem vädurar, fem bockar och fem årsgamla lamm. Detta var Netanels, Suars sons, offergåva.
\par 24 På tredje dagen kom hövdingen för Sebulons barn, Eliab, Helons son;
\par 25 hans offergåva var ett silverfat, ett hundra trettio siklar i vikt, och en silverskål om sjuttio siklar, efter helgedomssikelns vikt, båda fulla med fint mjöl, begjutet med olja, till spisoffer
\par 26 vidare en skål av guld om tio siklar, full med rökelse,
\par 27 vidare en ungtjur, en vädur och ett årsgammalt lamm till brännoffer
\par 28 och en bock till syndoffer,
\par 29 samt till tackoffret två tjurar, fem vädurar, fem bockar och fem årsgamla lamm. Detta var Eliabs, Helons sons, offergåva.
\par 30 På fjärde dagen kom hövdingen för Rubens barn, Elisur, Sedeurs son;
\par 31 hans offergåva var ett silverfat, ett hundra trettio siklar i vikt, och en silverskål om sjuttio siklar, efter helgedomssikelns vikt, båda fulla med fint mjöl, begjutet med olja, till spisoffer,
\par 32 vidare en skål av guld om tio siklar, full med rökelse,
\par 33 vidare en ungtjur, en vädur och ett årsgammalt lamm till brännoffer
\par 34 och en bock till syndoffer,
\par 35 samt till tackoffret två tjurar, fem vädurar, fem bockar och fem årsgamla lamm. Detta var Elisurs, Sedeurs sons, offergåva.
\par 36 På femte dagen kom hövdingen för Simeons barn, Selumiel, Surisaddais son;
\par 37 hans offergåva var ett silverfat, ett hundra trettio siklar i vikt, och en silverskål om sjuttio siklar, efter helgedomssikelns vikt, båda fulla med fint mjöl, begjutet med olja, till spisoffer,
\par 38 vidare en skål av guld om tio siklar, full med rökelse,
\par 39 vidare en ungtjur, en vädur och ett årsgammalt lamm till brännoffer
\par 40 och en bock till syndoffer,
\par 41 samt till tackoffret två tjurar, fem vädurar, fem bockar och fem årsgamla lamm. Detta var Selumiels, Surisaddais sons, offergåva.
\par 42 På sjätte dagen kom hövdingen för Gads barn, Eljasaf, Deguels son;
\par 43 hans offergåva var ett silverfat, ett hundra trettio siklar i vikt, och en silverskål om sjuttio siklar, efter helgedomssikelns vikt, båda fulla med fint mjöl, begjutet med olja, till spisoffer,
\par 44 vidare en skål av guld om tio siklar, full med rökelse,
\par 45 vidare en ungtjur, en vädur och ett årsgammalt lamm till brännoffer
\par 46 och en bock till syndoffer,
\par 47 samt till tackoffret två tjurar, fem vädurar, fem bockar och fem årsgamla lamm. Detta var Eljasafs, Deguels sons, offergåva.
\par 48 På sjunde dagen kom hövdingen för Efraims barn, Elisama, Ammihuds son;
\par 49 hans offergåva var ett silverfat, ett hundra trettio siklar i vikt, och en silverskål om sjuttio siklar, efter helgedomssikelns vikt, båda fulla med fint mjöl, begjutet med olja, till spisoffer,
\par 50 vidare en skål av guld om tio siklar, full med rökelse,
\par 51 vidare en ungtjur, en vädur och ett årsgammalt lamm till brännoffer
\par 52 och en bock till syndoffer,
\par 53 samt till tackoffret två tjurar, fem vädurar, fem bockar och fem årsgamla lamm. Detta var Elisamas, Ammihuds sons, offergåva.
\par 54 På åttonde dagen kom hövdingen för Manasse barn, Gamliel, Pedasurs son;
\par 55 hans offergåva var ett silverfat, ett hundra trettio siklar i vikt, och en silverskål om sjuttio siklar, efter helgedomssikelns vikt, båda fulla med fint mjöl, begjutet med olja, till spisoffer,
\par 56 vidare en skål av guld om tio siklar, full med rökelse,
\par 57 vidare en ungtjur en vädur och ett årsgammalt lamm till brännoffer
\par 58 och en bock till syndoffer,
\par 59 samt till tackoffret två tjurar, fem vädurar, fem bockar och fem årsgamla lamm. Detta var Gamliels Pedasurs sons, offergåva.
\par 60 På nionde dagen kom hövdingen för Benjamins barn, Abidan, Gideonis son;
\par 61 hans offergåva var ett silverfat, ett hundra trettio siklar i vikt, och en silverskål om sjuttio siklar, efter helgedomssikelns vikt, båda fulla med fint mjöl, begjutet med olja, till spisoffer,
\par 62 vidare en skål av guld om tio siklar, full med rökelse,
\par 63 vidare en ungtjur, en vädur och ett årsgammalt lamm till brännoffer
\par 64 och en bock till syndoffer,
\par 65 samt till tackoffret två tjurar, fem vädurar, fem bockar och fem årsgamla lamm. Detta var Abidans, Gideonis sons, offergåva.
\par 66 På tionde dagen kom hövdingen för Dans barn, Ahieser, Ammisaddais son;
\par 67 hans offergåva var ett silverfat, ett hundra trettio siklar i vikt, och en silverskål om sjuttio siklar, efter helgedomssikelns vikt, båda fulla med fint mjöl, begjutet med olja, till spisoffer,
\par 68 vidare en skål av guld om tio siklar, full med rökelse,
\par 69 vidare en ungtjur, en vädur och ett årsgammalt lamm till brännoffer
\par 70 och en bock till syndoffer,
\par 71 samt till tackoffret två tjurar, fem vädurar, fem bockar och fem årsgamla lamm. Detta var Ahiesers, Ammisaddais sons, offergåva.
\par 72 På elfte dagen kom hövdingen för Asers barn, Pagiel, Okrans son;
\par 73 hans offergåva var ett silverfat, ett hundra trettio siklar i vikt, och en silverskål om sjuttio siklar, efter helgedomssikelns vikt, båda fulla med fint mjöl, begjutet med olja, till spisoffer,
\par 74 vidare en skål av guld om tio siklar, full med rökelse,
\par 75 vidare en ungtjur, en vädur och ett årsgammalt lamm till brännoffer
\par 76 och en bock till syndoffer,
\par 77 samt till tackoffret två tjurar, fem vädurar, fem bockar och fem årsgamla lamm. Detta var Pagiels, Okrans sons, offergåva.
\par 78 På tolfte dagen kom hövdingen för Naftali barn, Ahira, Enans son;
\par 79 hans offergåva var ett silverfat, ett hundra trettio siklar i vikt, och en silverskål om sjuttio siklar, efter helgedomssikelns vikt, båda fulla med fint mjöl, begjutet med olja, till spisoffer,
\par 80 vidare en skål av guld om tio siklar, full med rökelse,
\par 81 vidare en ungtjur, en vädur och ett årsgammalt lamm till brännoffer
\par 82 och en bock till syndoffer,
\par 83 samt till tackoffret två tjurar, fem vädurar, fem bockar och fem årsgamla lamm. Detta var Ahiras, Enans sons, offergåva.
\par 84 Detta var vad Israels hövdingar skänkte till altarets invigning, när det smordes: tolv silverfat, tolv silverskålar och tolv guldskålar.
\par 85 Vart fat kom på ett hundra trettio silversiklar och var skål på sjuttio siklar, så att silvret i dessa kärl sammanlagt utgjorde två tusen fyra hundra siklar, efter helgedomssikelns vikt.
\par 86 Av de tolv guldskålarna, som voro fulla med rökelse, vägde var och en tio siklar, efter helgedomssikelns vikt, så att guldet i skålarna sammanlagt utgjorde ett hundra tjugu siklar.
\par 87 Brännoffers-fäkreaturen utgjorde tillsammans tolv tjurar, vartill kommo tolv vädurar, tolv årsgamla lamm, med tillhörande spisoffer, och tolv bockar till syndoffer.
\par 88 Och tackoffers-fäkreaturen utgjorde tillsammans tjugufyra tjurar, vartill kommo sextio vädurar, sextio bockar och sextio årsgamla lamm. Detta var vad som skänktes till altarets invigning, sedan det hade blivit smort.
\par 89 Och när Mose gick in i uppenbarelsetältet för att tala med honom, hörde han rösten tala till sig från nådastolen ovanpå vittnesbördets ark, från platsen mellan de två keruberna; där talade rösten till honom.

\chapter{8}

\par 1 Och HERREN talade till Mose och sade:
\par 2 "Tala till Aron och säg till honom: När du sätter upp lamporna, skall detta ske så, att de sju lamporna kasta sitt sken över platsen framför ljusstaken."
\par 3 Och Aron gjorde så; han satte upp lamporna så, att de kastade sitt sken över platsen framför ljusstaken, såsom HERREN hade bjudit Mose.
\par 4 Och ljusstaken var gjord på följande sätt: den var av guld i drivet arbete; också dess fotställning och blommorna därpå voro i drivet arbete. Efter det mönster som HERREN hade visat Mose hade denne låtit göra ljusstaken.
\par 5 Och HERREN talade till Mose och sade:
\par 6 Du skall bland Israels barn uttaga leviterna och rena dem.
\par 7 Och på följande sätt skall du göra med dem för att rena dem: Du skall stänka reningsvatten på dem; och de skola låta raka hela sin kropp och två sina kläder och skola så rena sig.
\par 8 Sedan skola de taga en ungtjur, med tillhörande spisoffer av fint mjöl, begjutet med olja; därjämte skall du taga en annan ungtjur till syndoffer.
\par 9 Och du skall föra leviterna fram inför uppenbarelsetältet, och du skall församla Israels barns hela menighet
\par 10 Och när du har fört leviterna fram inför HERRENS ansikte, skola Israels barn lägga sina händer på dem.
\par 11 Och Aron skall vifta leviterna inför HERRENS ansikte såsom ett viftoffer från Israels barn, och de skola sedan hava till åliggande att förrätta HERRENS tjänst.
\par 12 Och leviterna skola lägga sina händer på tjurarnas huvuden, och den ena skall du offra till syndoffer och den andra till brännoffer åt HERREN, för att bringa försoning för leviterna.
\par 13 Så skall du ställa leviterna inför Aron och hans söner och vifta dem såsom ett viftoffer åt HERREN.
\par 14 På detta sätt skall du bland Israels barn avskilja leviterna, så att leviterna skola tillhöra mig.
\par 15 Därefter skola leviterna gå in och göra tjänst vid uppenbarelsetältet, sedan du har renat dem och viftat dem såsom ett viftoffer;
\par 16 ty bland Israels barn äro de givna åt mig såsom gåva; i stället för allt som öppnar moderlivet, allt förstfött bland Israels barn, har jag uttagit dem åt mig.
\par 17 Ty mig tillhör allt förstfött bland, Israels barn, både människor och boskap; på den dag då jag slog allt förstfött i Egyptens land helgade jag det åt mig.
\par 18 Och jag har tagit leviterna i stället för allt förstfött bland Israels barn.
\par 19 Och jag har bland Israels barn givit leviterna såsom gåva åt Aron och hans söner, till att förrätta Israels barns tjänst vid uppenbarelsetältet och bringa försoning för Israels barn, på det att ingen hemsökelse må drabba Israels barn, därigenom att Israels barn nalkas helgedomen.
\par 20 Och Mose och Aron och Israels barns hela menighet gjorde så med leviterna; Israels barn gjorde med leviterna i alla stycken såsom HERREN hade bjudit Mose angående dem.
\par 21 Och leviterna renade sig och tvådde sina kläder, och Aron viftade dem såsom ett viftoffer inför HERRENS ansikte, och Aron bragte försoning för dem och renade dem.
\par 22 Därefter gingo leviterna in och förrättade sin tjänst vid uppenbarelsetältet under Aron och hans söner. Såsom HERREN hade bjudit Mose angående leviterna, så gjorde de med dem.
\par 23 och HERREN talade till Mose och sade:
\par 24 Detta är vad som skall gälla angående leviterna: Den som är tjugufem år gammal eller därutöver skall infinna sig och göra tjänst med arbete vid uppenbarelsetältet.
\par 25 Men när leviten bliver femtio år gammal, skall han vara fri ifrån att tjäna med arbete; han skall då icke längre arbeta.
\par 26 Han må betjäna sina bröder vid uppenbarelsetältet med att iakttaga vad som där är att iakttaga; men något bestämt arbete skall han icke förrätta. Så skall du förfara med leviterna i vad som angår deras åligganden.

\chapter{9}

\par 1 Och HERREN talade till Mose Sinais öken, i första månaden av det andra året efter deras uttåg ur Egyptens land; han sade:
\par 2 Israels barn skola ock hålla påsk högtid på den bestämda tiden.
\par 3 På fjortonde dagen i denna månad, vid aftontiden, skolen I hålla den, på bestämd tid. Enligt alla stadgar och föreskrifter därom skolen I hålla den.
\par 4 Så sade då Mose till Israels barn att de skulle hålla påskhögtid.
\par 5 Och de höllo påskhögtid i först månaden, på fjortonde dagen i månaden, vid aftontiden, i Sinais öken Israels barn gjorde i alla stycken såsom HERREN hade bjudit Mose
\par 6 Men där voro några män som hade blivit orena genom en död människa, så att de icke kunde hålla påskhögtid på den dagen; dessa trädde på den dagen fram inför Mose och Aron.
\par 7 Och männen sade till honom: "Vi hava blivit orena genom en död människa; skall det därför förmenas oss att bland Israels barn bära fram HERRENS offergåva på bestämd?"
\par 8 Mose svarade dem: "Stannen så vill jag höra vad HERREN bjuder angående eder."
\par 9 Och HERREN talade till Mose och sade:
\par 10 Tala till Israels barn och säg: Om någon bland eder eller edra efterkommande har blivit oren genom en död, eller är ute på resa långt borta, och han ändå vill hålla HERRENS påskhögtid
\par 11 så skall han hålla den i andra månaden, på fjortonde dagen, vid aftontiden. Med osyrat bröd och bittra örter skall han äta påskalammet.
\par 12 Intet därav skall lämnas kvar till morgonen, och intet ben skall sönderslås därpå. I alla stycken skall påskhögtiden hållas såsom stadgat är därom.
\par 13 Men om någon som är ren, och som icke är ute på resa ändå underlåter att hålla påskhögtid, så skall han utrotas ur sin släkt, eftersom han icke har burit fram HERRENS offergåva på bestämd tid; den mannen bär på synd.
\par 14 Och om någon främling bor hos eder och vill hålla HERRENS påskhögtid, så skall han hålla den enligt den stadga och föreskrift som gäller för påskhögtiden. En och samma stadga skall gälla för eder, lika väl för främlingen som för infödingen i landet.
\par 15 Och på den dag då tabernaklet sattes upp övertäckte molnskyn tabernaklet, vittnesbördets tält, och om aftonen, och sedan ända till morgonen, var det såsom såge man en eld över tabernaklet.
\par 16 Så var det beständigt: molnskyn övertäckte det, och om natten var det såsom såge man en eld.
\par 17 Och så ofta molnskyn höjde sig från tältet, bröto Israels barn strax upp, och på det ställe där molnskyn stannade, där slogo Israels barn läger.
\par 18 Efter HERRENS befallning bröto Israels barn upp, och efter HERRENS befallning slogo de läger. Så länge molnskyn vilade över tabernaklet, lågo de i läger.
\par 19 Och om molnskyn en längre tid förblev över tabernaklet, så iakttogo Israels barn vad HERREN bjöd dem iakttaga och bröto icke upp.
\par 20 Stundom kunde det hända att molnskyn allenast några få dagar stannade över tabernaklet; då lågo de efter HERRENS befallning i läger och bröto sedan upp efter HERRENS befallning.
\par 21 Stundom kunde det ock hända att molnskyn stannade allenast från aftonen till morgonen; när då molnskyn om morgonen höjde sig, bröto de upp; eller om så var, att molnskyn stannade en dag och en natt och sedan höjde sig, så bröto de upp då.
\par 22 Eller om den stannade två dagar, eller en månad, eller vilken tid som helst, så att molnskyn länge förblev vilande över tabernaklet, så lågo Israels barn stilla i läger och bröto icke upp; men när den sedan höjde sig, bröto de upp.
\par 23 Efter HERRENS befallning slogo de läger, och efter HERRENS befallning bröto de upp. Vad HERREN bjöd dem iakttaga, det iakttogo de, efter HERRENS befallning genom Mose.

\chapter{10}

\par 1 Och HERREN talade till Mose och sade:
\par 2 "Gör dig två trumpeter av silver; i drivet arbete skall du göra dem. Dessa skall du bruka, när menigheten skall sammankallas, och när lägren skola bryta upp.
\par 3 När man stöter i dem båda, skall hela menigheten församla sig till dig, vid ingången till uppenbarelsetältet.
\par 4 Men när man stöter allenast i den ena, skola hövdingarna, huvudmännen för Israels ätter, församla sig till dig.
\par 5 Och när I blåsen en larmsignal, skola de läger bryta upp, som ligga österut.
\par 6 Men när I blåsen larmsignal för andra gången, skola de läger bryta upp, som ligga söderut. När lägren skola bryta upp, skall man blåsa larmsignal,
\par 7 men när församlingen skall sammankallas, skolen I icke blåsa larmsignal, utan stöta i trumpeterna.
\par 8 Och Arons söner, prästerna, äro de som skola blåsa i trumpeterna. Detta skall vara en evärdlig stadga för eder från släkte till släkte.
\par 9 Och om I, i edert land, dragen ut till strid mot någon eder ovän som angriper eder, så skolen I blåsa larmsignal med trumpeterna; härigenom skolen I då bringas i åminnelse inför HERRENS, eder Guds, ansikte, och I skolen så bliva frälsta ifrån edra fiender.
\par 10 Och när I haven en glädjedag och haven edra högtider och nymånader, skolen I stöta i trumpeterna, då I offren edra brännoffer och tackoffer; så skola de bringa eder i åminnelse inför eder Guds ansikte. Jag är HERREN, eder Gud."
\par 11 I andra året, i andra månaden, på tjugonde dagen i månaden höjde sig molnskyn från vittnesbördets tabernakel.
\par 12 Då bröto Israels barn upp från Sinais öken och tågade från lägerplats till lägerplats; och molnskyn stannade i öknen Paran.
\par 13 Och när de nu första gången bröto upp, efter HERRENS befallning genom Mose,
\par 14 var Juda barns läger under sitt baner det första som bröt upp, häravdelning efter häravdelning; och anförare för denna här var Naheson, Amminadabs son.
\par 15 Och anförare för den här som utgjordes av Isaskars barns stam var Netanel, Suars son.
\par 16 Och anförare för den här som utgjordes av Sebulons barns stam var Eliab, Helons son.
\par 17 Därefter, sedan tabernaklet hade blivit nedtaget, bröto Gersons barn och Meraris barn upp och buro tabernaklet.
\par 18 Därefter bröt Rubens läger upp under sitt baner, häravdelning efter häravdelning; och anförare för denna här var Elisur, Sedeurs son.
\par 19 Och anförare för den här som utgjordes av Simeons barns stam var Selumiel, Surisaddais son.
\par 20 Och anförare för den här som utgjordes av Gads barns stam var Eljasaf, Deguels son.
\par 21 Därefter bröto kehatiterna upp och buro de heliga tingen, och de andra satte upp tabernaklet, innan dessa hunno fram.
\par 22 Därefter bröt Efraims barns läger upp under sitt baner, häravdelning efter häravdelning; och anförare för denna här var Elisama, Ammihuds son.
\par 23 Och anförare för den här som utgjordes av Manasse barns stam var Gamliel, Pedasurs son.
\par 24 Och anförare for den här som utgjordes av Benjamins barns stam var Abidan, Gideonis son.
\par 25 Därefter bröt Dans barns läger upp under sitt baner, såsom eftertrupp i hela lägertåget, häravdelning efter häravdelning; och anförare för denna här var Ahieser, Ammisaddais son.
\par 26 Och anförare för den här som utgjordes av Asers barns stam var Pagiel, Okrans son.
\par 27 Och anförare för den här som utgjordes av Naftali barns stam var Ahira, Enans son.
\par 28 I denna ordning bröto Israels barn upp, häravdelning efter häravdelning. Och de bröto nu upp.
\par 29 Och Mose sade till Hobab, som var son till midjaniten Reguel, Moses svärfader: "Vi bryta nu upp och tåga till det land om vilket HERREN har sagt: 'Det vill jag giva eder.' Följ du med oss, så vilja vi göra dig gott, ty HERREN har lovat Israel vad gott är.
\par 30 Men han svarade honom: "Jag vill icke följa med, utan jag vill gå hem till mitt land och till min släkt."
\par 31 Då sade han: "Ack nej, övergiv oss icke. Du vet ju bäst var vi kunna lägra oss i öknen; bliv du därför nu vårt öga.
\par 32 Om du följer med oss, skola vi låta också dig få gott av det goda som HERREN gör mot oss.
\par 33 Så bröto de upp och tågade från HERRENS berg tre dagsresor. Och HERRENS förbundsark gick framför dem tre dagsresor, för att utse viloplats åt dem.
\par 34 Och HERRENS molnsky svävade över dem om dagen, när de bröto upp från sitt lägerställe.
\par 35 Och så ofta arken bröt upp, sade Mose: "Stå upp, HERRE; må dina fiender varda förskingrade, och må de som hata dig fly för ditt ansikte."
\par 36 Och när den sattes ned, sade han: "Kom tillbaka, HERRE, till Israels ätters mångtusenden."

\chapter{11}

\par 1 Men folket knorrade, och detta misshagade HERREN. Ty när HERREN hörde det, upptändes hans vrede, och HERRENS eld begynte brinna ibland dem och förtärde de som voro ytterst i lägret.
\par 2 Då ropade folket till Mose, och Mose bad till HERREN, och så stannade elden av.
\par 3 Och detta ställe fick namnet Tabeera, därför att HERRENS eld hade brunnit ibland dem.
\par 4 Och den blandade folkhop som åtföljde dem greps av lystnad; Israels barn själva begynte då ock åter att gråta och sade: "Ack om vi hade kött att äta!
\par 5 Vi komma ihåg fisken som vi åt i Egypten för intet, så ock gurkorna, melonerna, purjolöken, rödlöken och vitlöken.
\par 6 Men nu försmäkta våra själar, ty här finnes alls intet; vi få intet annat se än manna.
\par 7 Men mannat liknade korianderfrö och hade samma utseende som bdelliumharts.
\par 8 Folket gick omkring och samlade sådant, och malde det därefter på handkvarn eller stötte sönder det i mortel, och kokte det sedan i gryta och bakade kakor därav; och det smakade såsom fint bakverk med olja.
\par 9 När daggen om natten föll över lägret, föll ock mannat där.
\par 10 Och Mose hörde huru folket i sina särskilda släkter grät, var och en vid ingången till sitt tält; och HERRENS vrede upptändes storligen, och Mose själv blev misslynt.
\par 11 Och Mose sade till HERREN: "Varför har du gjort så illa mot din tjänare, och varför har jag så litet funnit nåd för dina ögon, att du har lagt på mig bördan av hela detta folk?
\par 12 Är då jag moder eller fader till hela detta folk, eftersom du säger till mig att jag skall bära det i min famn, såsom spenabarnet bäres av sin vårdare, in i det land som du med ed har lovat åt deras fäder?
\par 13 Varifrån skall jag få kött att giva åt hela detta folk? De gråta ju och vända sig mot mig och säga: 'Giv oss kött, så att vi få äta.'
\par 14 Jag förmår icke ensam bära hela detta folk, ty det bliver mig för tungt.
\par 15 Vill du så handla mot mig, så dräp mig hellre med ens, om jag har funnit nåd för dina ögon, och låt mig slippa detta elände."
\par 16 Då sade HERREN till Mose: "Samla ihop åt mig sjuttio män av de äldste i Israel, dem som du vet höra till de äldste i folket och till dess tillsyningsmän; och för dessa fram till uppenbarelsetältet och låt dem ställa sig där hos dig.
\par 17 Där vill jag då stiga ned och tala med dig, och jag vill taga av den ande som är över dig och låta komma över dem; sedan skola de bistå dig med att bära på bördan av folket, så att du slipper bära den ensam.
\par 18 Och till folket skall du säga: Helgen eder till i morgon, så skolen I få kött att äta, eftersom I haven gråtit inför HERREN och sagt: 'Ack om vi hade kött att äta! I Egypten var oss gott att vara!' Så skall nu HERREN giva eder kött att äta.
\par 19 Icke allenast en dag eller två dagar skolen I få äta det, icke allenast fem dagar eller tio dagar eller tjugu dagar,
\par 20 utan en hel månads tid, till dess att det går ut genom näsan på eder och bliver eder vämjeligt; detta därför att I haven förkastat HERREN, som är mitt ibland eder, och haven gråtit inför hans ansikte och sagt: 'Varför drogo vi då ut ur Egypten?'"
\par 21 Mose sade: "Av sex hundra tusen man till fots utgöres det folk som jag har omkring mig, och dock säger du: 'Kött vill jag giva dem, så att de hava att äta en månads tid!'
\par 22 Finnas då får och fäkreatur att slakta åt dem i sådan mängd att det räcker till för dem? Eller skall man samla ihop alla havets fiskar åt dem, så att det räcker till för dem?"
\par 23 HERREN svarade Mose: "Är då HERRENS arm för kort? Du skall nu få se om det som jag har sagt skall vederfaras dig eller icke."
\par 24 Och Mose gick ut och förkunnade för folket vad HERREN hade sagt. Sedan samlade han ihop sjuttio män av de äldste i folket och lät dem ställa sig runt omkring tältet.
\par 25 Då steg HERREN ned i molnskyn och talade till honom, och tog av den ande som var över honom och lät komma över de sjuttio äldste. Då nu anden föll på dem, begynte de profetera, vilket de sedan icke mer gjorde.
\par 26 Och två män hade stannat kvar i lägret; den ene hette Eldad och den andre Medad. Också på dem föll anden, ty de voro bland de uppskrivna, men hade likväl icke gått ut till tältet; och de profeterade i lägret.
\par 27 Då skyndade en ung man bort och berättade detta för Mose och sade: "Eldad och Medad profetera i lägret."
\par 28 Josua, Nuns son, som hade varit Moses tjänare allt ifrån sin ungdom, tog då till orda och sade: "Mose, min herre, förbjud dem det."
\par 29 Men Mose sade till honom: "Skall du så nitälska för mig? Ack att fastmer allt HERRENS folk bleve profeter, därigenom att HERREN läte sin Ande komma över dem!"
\par 30 Sedan gick Mose tillbaka till lägret med de äldste i Israel.
\par 31 Och en stormvind for ut ifrån HERREN, och den förde med sig vaktlar från havet och drev dem över lägret, en dagsresa vitt på vardera sidan, runt omkring lägret, och vid pass två alnar högt över marken.
\par 32 Då stod folket upp och gick hela den dagen och sedan hela natten och hela den följande dagen och samlade ihop vaktlar; det minsta någon samlade var tio homer. Och de bredde ut dem runt omkring lägret.
\par 33 Men under det att de ännu hade köttet mellan tänderna, innan det var förtärt, upptändes HERRENS vrede mot folket, och HERREN anställde ett mycket stort nederlag bland folket.
\par 34 Och detta ställe fick namnet Kibrot-Hattaava, ty där begrov man dem av folket, som hade gripits av lystnad.
\par 35 Från Kibrot-Hattaava bröt folket upp och tågade till Haserot; och i Haserot stannade de.

\chapter{12}

\par 1 Och Mirjam jämte Aron talade illa om Mose för den etiopiska kvinnans skull som han hade tagit till hustru; han hade nämligen tagit en etiopisk kvinna till hustru.
\par 2 Och de sade: "Är då Mose den ende som HERREN talar med? Talar han icke också med oss?" Och HERREN hörde detta.
\par 3 Men mannen Mose var mycket saktmodig, mer än någon annan människa på jorden.
\par 4 Och strax sade HERREN till Mose, Aron och Mirjam: "Gån ut, I tre, till uppenbarelsetältet." Och de gingo ditut, alla tre.
\par 5 Då steg HERREN ned i en molnstod och blev stående vid ingången till tältet; och han kallade på Aron och Mirjam, och de gingo båda ditut.
\par 6 Och han sade: "Hören nu mina ord. Eljest om någon är en profet bland eder, giver jag, HERREN, mig till känna för honom i syner och talar med honom i drömmar.
\par 7 Men så gör jag icke med min tjänare Mose; i hela mitt hus är han betrodd.
\par 8 Muntligen talar jag med honom, öppet och icke i förtäckta ord, och han får skåda HERRENS gestalt. Varför haven I då icke haft försyn för att tala illa om min tjänare Mose?"
\par 9 Och HERRENS vrede upptändes mot dem, och han övergav dem.
\par 10 När så molnskyn drog sig tillbaka från tältet, se, då var Mirjam vit såsom snö av spetälska; när Aron vände sig till Mirjam, fick han se att hon var spetälsk.
\par 11 Då sade Aron till Mose: "Ack, min herre, lägg icke på oss bördan av en synd som vi i vår dårskap hava begått.
\par 12 Låt henne icke bliva lik ett dödfött foster, vars kropp är till hälften förstörd, när det kommer ut ur sin moders liv."
\par 13 Då ropade Mose till HERREN och sade: "O Gud, gör henne åter frisk."
\par 14 HERREN svarade Mose: "Om hennes fader hade spottat henne i ansiktet, skulle hon ju hava fått sitta med skam i sju dagar. Må hon alltså nu hållas innestängd i sju dagar utanför lägret; sedan får hon komma tillbaka dit igen."
\par 15 Så hölls då Mirjam innestängd i sju dagar utanför lägret, och folket bröt icke upp, förrän Mirjam hade kommit tillbaka.

\chapter{13}

\par 1 Därefter bröt folket upp från Haserot och lägrade sig i öknen Paran.
\par 2 Och HERREN talade till Mose och sade:
\par 3 "Sänd åstad några män för att bespeja Kanaans land, som jag vill giva åt Israels barn. En man ur var fädernestam skolen I sända, men allenast sådana som äro hövdingar bland dem."
\par 4 Och Mose sände från öknen Paran åstad sådana män, efter HERRENS befallning; allasammans hörde de till huvudmännen bland Israels barn.
\par 5 Och dessa voro namnen på dem: Av Rubens stam: Sammua, Sackurs son;
\par 6 av Simeons stam: Safat, Horis son;
\par 7 av Juda stam: Kaleb, Jefunnes son;
\par 8 av Isaskars stam: Jigeal, Josefs son;
\par 9 av Efraims stam: Hosea, Nuns son;
\par 10 av Benjamins stam: Palti, Rafus son;
\par 11 av Sebulons stam: Gaddiel, Sodis son;
\par 12 av Josefs stam: av Manasse stam: Gaddi, Susis son;
\par 13 av Dans stam: Ammiel, Gemallis son;
\par 14 av Asers stam: Setur, Mikaels son;
\par 15 av Naftali stam: Nahebi, Vofsis son;
\par 16 av Gads stam; Geuel, Makis son.
\par 17 Dessa voro namnen på de män som Mose sände åstad för att bespeja landet. Men Mose gav Hosea, Nuns son, namnet Josua.
\par 18 Och Mose sände dessa åstad för att bespeja Kanaans land. Och han sade till dem: "Dragen nu upp till Sydlandet, och dragen vidare upp till Bergsbygden.
\par 19 Och sen efter, hurudant landet är, och om folket som bor däri är starkt eller svagt, om det är litet eller stort,
\par 20 och hurudant landet är, vari de bo, om det är gott eller dåligt, och hurudana de platser äro, där de bo, om de bo i läger eller i befästa städer,
\par 21 och hurudant själva landet är, om det är fett eller magert, om träd finnas där eller icke. Varen vid gott mod, och tagen med eder hit av landets frukt." Det var nämligen vid den tid då de första druvorna voro mogna
\par 22 Så drogo de åstad och bespejade landet från öknen Sin ända till Rehob, där vägen går till Hamat.
\par 23 De drogo upp till Sydlandet och kommo till Hebron; där bodde Ahiman, Sesai och Talmai, Anaks avkomlingar. Men Hebron byggdes sju år före Soan i Egypten.
\par 24 Och de kommo till Druvdalen; där skuro de av en kvist med en ensam druvklase på, och denna bars sedan på en stång av två man. Därtill togo de granatäpplen och fikon.
\par 25 Detta ställe blev kallat Druvdalen för den druvklases skull som Israels barn där skuro av.
\par 26 Och efter fyrtio dagar vände de tillbaka, sedan de hade bespejat landet.
\par 27 De gingo åstad och kommo till Mose och Aron och Israels barns hela menighet i öknen Paran, i Kades, och avgåvo sin berättelse inför dem och hela menigheten och visade dem landets frukt.
\par 28 De förtäljde för honom och sade: "Vi kommo till det land dit du sände oss. Och det flyter i sanning av mjölk och honung, och här är dess frukt.
\par 29 Men folket som bor i landet är starkt, och städerna äro välbefästa och mycket stora; ja, vi sågo där också avkomlingar av Anak.
\par 30 Amalekiterna bo i Sydlandet, hetiterna, jebuséerna och amoréerna bo i Bergsbygden, och kananéerna bo vid havet och utmed Jordan."
\par 31 Men Kaleb sökte stilla folket, så att de icke skulle knota emot Mose; han sade: "Låt oss ändå draga ditupp och intaga det, ty förvisso skola vi bliva det övermäktiga."
\par 32 Men de män som hade varit däruppe med honom sade: "Vi kunna icke draga upp mot detta folk, ty de äro oss för starka."
\par 33 Och de talade bland Israels barn illa om landet som de hade bespejat; de sade: "Det land som vi hava genomvandrat och bespejat är ett land som förtär sina inbyggare, och alla människor, som vi där sågo, voro resligt folk.
\par 34 Vi sågo där ock jättarna, Anaks barn, av jättestammen; vi tyckte då att vi själva voro såsom gräshoppor, och sammalunda tyckte de om oss.

\chapter{14}

\par 1 Då begynte hela menigheten ropa och skria, och folket grät den natten.
\par 2 Och alla Israels barn knorrade emot Mose och Aron, och hela menigheten sade till dem: "O att vi hade fått dö i Egyptens land, eller att vi hade fått dö här i öknen!
\par 3 Varför vill då HERREN föra oss in i detta andra land, där vi måste falla för svärd, och där våra hustrur och barn skola bliva fiendens byte? Det vore förvisso bättre för oss att vända tillbaka till Egypten."
\par 4 Och de sade till varandra: "Låt oss välja en anförare och vända tillbaka till Egypten."
\par 5 Då föllo Mose och Aron ned på sina ansikten inför Israels barns hela församlade menighet.
\par 6 Och Josua, Nuns son, och Kaleb, Jefunnes son, vilka voro bland dem som hade bespejat landet, revo sönder sina kläder
\par 7 och sade till Israels barns hela menighet: "Det land som vi hava genomvandrat och bespejat är ett övermåttan gott land.
\par 8 Om HERREN har behag till oss, så skall han föra oss in i det landet och giva det åt oss - ett land som flyter av mjölk och honung.
\par 9 Allenast mån I icke sätta eder upp mot HERREN; och för folket i landet mån I icke frukta, ty de skola bliva såsom en munsbit för oss. Deras beskärm har vikit ifrån dem, men med oss är HERREN; frukten icke för dem."
\par 10 Men hela menigheten ropade att man skulle stena dem. Då visade sig HERRENS härlighet i uppenbarelsetältet för alla Israels barn.
\par 11 Och HERREN sade till Mose: "Huru länge skall detta folk förakta mig, och huru länge skola de framhärda i att icke vilja tro på mig, oaktat alla de tecken jag har gjort bland dem?
\par 12 Jag skall slå dem med pest och förgöra dem, men dig vill jag göra till ett folk, större och mäktigare än detta."
\par 13 Mose sade till HERREN: "Egyptierna hava ju förnummit att du med din kraft har fört detta folk ut ifrån dem hitupp,
\par 14 och de hava omtalat det för inbyggarna här i landet, så att de hava fått höra att du är HERREN mitt ibland detta folk, att du, HERRE, visar dig för dem ansikte mot ansikte, att din molnsky står över dem, och att du går framför dem i en molnstod om dagen och i en eldstod om natten.
\par 15 Om du nu dödade detta folk, alla tillsammans, då skulle hedningarna, som finge höra detta berättas om dig, säga så:
\par 16 'Därför att HERREN icke förmådde föra detta folk in i det land som han med ed hade lovat åt dem, därför har han slaktat dem i öknen.'
\par 17 Nej, må nu Herrens kraft bevisa sig stor, såsom du har talat och sagt:
\par 18 'HERREN är långmodig och stor i mildhet, han förlåter missgärning och överträdelse, fastän han icke låter någon bliva ostraffad, utan hemsöker fädernas missgärning på barn och efterkommande i tredje och fjärde led.'
\par 19 Så tillgiv nu detta folk dess missgärning, enligt din stora nåd, såsom du har låtit din förlåtelse följa detta folk allt ifrån Egypten och ända hit."
\par 20 Då sade HERREN: "Jag vill tillgiva dem efter din bön.
\par 21 Men så sant jag lever, och så sant hela jorden skall bliva full av HERRENS härlighet:
\par 22 av alla de män som hava sett min härlighet och de tecken jag har gjort i Egypten och i öknen, och som dock nu tio gånger hava frestat mig och icke velat höra min röst,
\par 23 av dem skall ingen få se det land som jag med ed har lovat åt deras fäder; ingen av dem som hava föraktat mig skall få se det.
\par 24 Men eftersom i min tjänare Kaleb är en annan ande, så att han i allt har efterföljt mig, därför vill jag låta honom komma in i det land där han nu har varit, och hans avkomlingar skola besitta det.
\par 25 Men då nu amalekiterna och kananéerna bo i dalbygden, så vänden eder i morgon åt annat håll bryten upp och tagen vägen mot öknen, åt Röda havet till."
\par 26 Och HERREN talade till Mose och Aron och sade:
\par 27 "Huru länge skall denna onda menighet fortfara att knorra mot mig? Ty jag har hört huru Israels barn knorra mot mig.
\par 28 Säg nu till dem: 'Så sant jag lever, säger HERREN, jag skall göra med eder såsom I själva haven sagt inför mig.
\par 29 Här i öknen skola edra döda kroppar bliva liggande; så skall det gå eder alla, så många I ären som haren blivit inmönstrade, alla som äro tjugu år gamla eller därutöver, eftersom I haven knorrat mot mig.
\par 30 Sannerligen, ingen av eder skall komma in i det land som jag med upplyft hand har lovat giva eder till boning, ingen förutom Kaleb, Jefunnes son, och Josua, Nuns son.
\par 31 Men edra barn, om vilka I saden att de skulle bliva fiendens byte? dem skall jag låta komma ditin, och de skola lära känna det land som I haven föraktat.
\par 32 I själva däremot - edra döda kroppar skola bliva liggande här i öknen.
\par 33 Och edra barn skola draga omkring såsom herdar i öknen i fyrtio år, och skola bära på bördan av eder trolösa avfällighet, till dess att edra döda kroppar hava förgåtts i öknen.
\par 34 Såsom I under fyrtio dagar haven bespejat landet, så skolen I under fyrtio år - ett år för var dag - komma att bära på edra missgärningar; I skolen då förnimma vad det är att jag tager min hand ifrån eder.'
\par 35 Jag, HERREN, talar; jag skall förvisso göra så med hela denna onda menighet, som har rotat sig samman mot mig; här i öknen skola de förgås, här skola de dö."
\par 36 Och de män som Mose hade sänt åstad för att bespeja landet, och som vid sin återkomst hade förlett hela menigheten att knorra mot honom, därigenom att de talade illa om landet,
\par 37 dessa män som hade talat illa om landet träffades nu av döden genom en hemsökelse, inför HERRENS ansikte.
\par 38 Av de män som hade gått åstad för att bespeja landet blevo dock Josua, Nuns son, och Kaleb, Jefunnes son, vid liv.
\par 39 Och Mose talade detta till alla Israels barn. Då blev folket mycket sorgset.
\par 40 Och de stodo upp bittida följande morgon för att draga åstad upp mot den övre bergsbygden; och de sade: "Se, här äro vi; vi vilja nu draga upp till det land som HERREN har talat om; ty vi hava syndat."
\par 41 Men Mose sade: "Varför viljen I så överträda HERRENS befallning? Det kan ju icke lyckas väl.
\par 42 HERREN är icke med bland eder; dragen därför icke ditupp, på det att I icke mån bliva slagna av edra fiender.
\par 43 Ty amalekiterna och kananéerna skola där möta eder, och I skolen falla för svärd; I haven ju vänt eder bort ifrån HERREN, och HERREN skall därför icke vara med eder."
\par 44 Likväl drogo de i sitt övermod upp mot den övre bergsbygden; men HERRENS förbundsark och Mose lämnade icke lägret.
\par 45 Då kommo amalekiterna och kananéerna, som bodde där i bergsbygden, ned och slogo dem och förskingrade dem och drevo dem ända till Horma.

\chapter{15}

\par 1 Och HERREN talade till Mose och sade:
\par 2 Tala till Israels barn och säg till dem: Om I, när I kommen in i det land som jag vill giva eder, och där I skolen bo,
\par 3 viljen offra ett eldsoffer åt HERREN, ett brännoffer eller ett slaktoffer - vare sig det gäller att fullgöra ett löfte, eller det gäller ett frivilligt offer, eller det gäller edra högtidsoffer - för att bereda HERREN en välbehaglig lukt, genom fäkreatur eller småboskap,
\par 4 så skall den som vill offra åt HERREN ett sådant offer bära fram såsom spisoffer en tiondedels efa fint mjöl, begjutet med en fjärdedels hin olja;
\par 5 och såsom drickoffer skall du offra en fjärdedels hin vin till vart lamm, vare sig det är ett brännoffer som offras, eller det är ett slaktoffer.
\par 6 Men till en vädur skall du såsom spisoffer offra två tiondedels efa fint mjöl, begjutet med en tredjedels hin olja,
\par 7 och såsom drickoffer skall du bära fram en tredjedels hin vin, till er välbehaglig lukt för HERREN.
\par 8 Och när du offrar en ungtjur till brännoffer eller till slaktoffer, vare sig det gäller att fullgöra ett löfte eller det gäller tackoffer åt HERREN.
\par 9 Så skall, jämte ungtjuren, såsom spisoffer frambäras tre tiondedels efa fint mjöl, begjutet med en halv in olja,
\par 10 och såsom drickoffer skall du bära fram en halv hin vin: ett eldsoffer till en välbehaglig lukt för HERREN.
\par 11 Vad här är sagt skall offras till var tjur, var vädur, vart djur av småboskapen, vare sig får eller get.
\par 12 Efter antalet av de djur I offren skolen I offra till vart och ett vad här är sagt, efter deras antal.
\par 13 Var inföding skall offra detta, såsom här är sagt, när han vill offra ett eldsoffer till en välbehaglig lukt för HERREN.
\par 14 Och när en främling som vistas hos eder, eller som i kommande släkten bor ibland eder, vill offra ett eldsoffer till en välbehaglig lukt för HERREN, så skall han offra på samma sätt som I offren.
\par 15 Inom församlingen skall en och samma stadga gälla för eder och för främlingen som bor ibland eder. Detta skall vara en evärdlig stadga för eder från släkte till släkte: I själva och främlingen skolen förfara på samma sätt inför HERRENS ansikte.
\par 16 Samma lag och samma rätt skall gälla för eder och för främlingen som bor hos eder.
\par 17 Och HERREN talade till Mose och sade:
\par 18 Tala till Israels barn och säg till dem: När I kommen in i det land dit jag vill föra eder,
\par 19 skolen I, då I äten av landets bröd, giva åt HERREN en offergärd därav.
\par 20 Såsom förstling av edert mjöl skolen I giva en kaka till offergärd; I skolen giva den, likasom I given en offergärd från eder loge.
\par 21 Av förstlingen av edert mjöl skolen I giva åt HERREN en offergärd, släkte efter släkte.
\par 22 Och om I begån synd ouppsåtligen, i det att I underlåten att göra efter något av dessa bud som HERREN har kungjort för Mose,
\par 23 efter något av det som HERRENS har bjudit eder genom Mose, från den dag då HERREN gav sina bud och allt framgent, släkte efter släkte,
\par 24 så skall, om synden har blivit begången ouppsåtligen, utan att menigheten visste det, hela menigheten offra en ungtjur såsom brännoffer, till en välbehaglig lukt för HERREN, med det spisoffer och det drickoffer som på föreskrivet sätt skall offras därtill, och tillika en bock såsom syndoffer.
\par 25 Och prästen skall bringa försoning för Israels barns hela menighet, och så bliver dem förlåtet; ty det var en ouppsåtlig synd, och de hava burit fram sitt offer, ett eldsoffer åt HERREN, och sitt syndoffer inför HERRENS ansikte för sin ouppsåtliga synd.
\par 26 Ja, så bliver dem förlåtet, Israels barns hela menighet och främlingen som bor ibland dem; ty hela folket var delaktigt i den ouppsåtliga synden.
\par 27 Och om någon enskild syndar ouppsåtligen, skall han såsom syndoffer föra fram en årsgammal get.
\par 28 Och prästen skall bringa försoning för denne som har försyndat sig genom ouppsåtlig synd, inför HERRENS ansikte; på det att honom må bliva förlåtet, när försoning bringas för honom.
\par 29 För infödingen bland Israels barn och för främlingen som bor ibland dem, för eder alla skall gälla en och samma lag, när någon begår synd ouppsåtligen.
\par 30 Men den som begår något med upplyft hand, evad han är inföding eller främling, han hånar HERREN, och han skall utrotas ur sitt folk.
\par 31 Ty HERRENS ord har han föraktat, och mot hans bud har han brutit; utan förskoning skall han utrotas; missgärning vilar på honom.
\par 32 Medan nu Israels barn voro i öknen, ertappades en man med att samla ihop ved på sabbatsdagen.
\par 33 Och de som ertappade honom med att samla ihop ved förde honom fram inför Mose och Aron och hela menigheten.
\par 34 Och då det icke var bestämt vad som borde göras med honom, satte de honom i förvar.
\par 35 Och HERREN sade till Mose: "Mannen skall straffas med döden; hela menigheten skall stena honom utanför lägret."
\par 36 Då förde hela menigheten ut honom utanför lägret och stenade honom till döds, såsom HERREN hade bjudit Mose.
\par 37 Och HERREN talade till Mose och sade:
\par 38 Tala till Israels barn och säg till dem att de och deras efterkommande skola göra sig tofsar i hörnen på sina kläder och sätta ett mörkblått snöre på var hörntofs.
\par 39 Och detta skolen I hava till tofsprydnad, för att I, när I sen därpå, mån tänka på alla HERRENS bud och göra efter dem, och icke sväva omkring efter edra hjärtans och ögons lustar, som I nu löpen efter i trolös avfällighet.
\par 40 Ty jag vill att I skolen tänka på och göra efter alla mina bud och vara helgade åt eder Gud.
\par 41 Jag är HERREN, eder Gud, som har fört eder ut ur Egyptens land, för att jag skall vara eder Gud Jag är HERREN, eder Gud.

\chapter{16}

\par 1 Och Kora, son till Jishar, son till Kehat, son till Levi, samt Datan och Abiram, Eliabs söner, och On, Pelets son, av Rubens söner, dessa togo till sig folk
\par 2 och gjorde uppror emot Mose; och dem följde två hundra femtio män av Israels barn, hövdingar i menigheten, ombud i folkförsamlingen, ansedda män.
\par 3 Och de församlade sig emot Mose och Aron och sade till dem: "Nu må det vara nog. Hela menigheten är ju helig, alla äro det, och HERREN är mitt ibland dem; varför upphäven I eder då över HERRENS församling?"
\par 4 När Mose hörde detta, föll han ned på sitt ansikte.
\par 5 Sedan talade han till Kora och hela hans hop och sade: "I morgon skall HERREN göra kunnigt vem som hör honom till, och vem som är den helige åt vilken han giver tillträde till sig. Och den han utväljer, honom skall han giva tillträde till sig.
\par 6 Gören nu på detta sätt: tagen edra fyrfat, du Kora och hela din hop,
\par 7 och läggen eld i dem och strön rökelse på dem inför HERRENS ansikte i morgon; den man som HERREN då utväljer, han är den helige. Ja, nu må det vara nog, I Levi söner."
\par 8 Ytterligare sade Mose till Kora: Hören nu, I Levi söner.
\par 9 Är det eder icke nog att Israels Gud har avskilt eder från Israels menighet och givit eder tillträde till sig, så att I fån förrätta tjänsten i HERRENS tabernakel och stå inför menigheten och betjäna den?
\par 10 Åt dig, och åt alla dina bröder, Levi söner, jämte dig, har han givit tillträde till sig; och nu stån I också efter prästadömet!
\par 11 Därför, tagen eder till vara, du och hela din hop, I som haven rotat eder samman mot HERREN - ty vad är Aron, att I knorren mot honom?"
\par 12 Och Mose sände och lät kalla till sig Datan och Abiram, Eliabs söner. Men de sade: "vi komma icke.
\par 13 Är det icke nog att du har fört oss hitupp ur ett land som flöt av mjölk och honung, för att låta oss dö I öknen? Vill du nu ock upphäva dig till herre över oss?
\par 14 Ingalunda har du fört oss in i ett land som flyter av mjölk och honung, eller givit oss åkrar och vingårdar till arvedel. Eller tror du att du kan sticka ut ögonen på dessa människor? Nej, vi komma icke."
\par 15 Då blev Mose mycket vred och sade till HERREN: "Se icke till deras offergåva. Icke så mycket som en enda åsna har jag tagit av dem, och ingen av dem har jag gjort något ont."
\par 16 Och Mose sade till Kora: "Du och hela din hop mån inställa eder inför HERRENS ansikte i morgon, du själv och de, så ock Aron.
\par 17 Och var och en av eder må taga sitt fyrfat och lägga rökelse därpå, och sedan bära sitt fyrfat fram inför HERRENS ansikte, två hundra femtio fyrfat; du själv och Aron mån ock taga var sitt fyrfat."
\par 18 Och de togo var och en sitt fyrfat och lade eld därpå och strödde rökelse därpå, och ställde sig vid ingången till uppenbarelsetältet; och Mose och Aron likaså.
\par 19 Och Kora församlade mot dem hela menigheten vid ingången till uppenbarelsetältet. Då visade sig HERRENS härlighet för hela menigheten.
\par 20 och HERREN talade till Mose och Aron och sade:
\par 21 "Skiljen eder från denna menighet, så skall jag i ett ögonblick förgöra dem."
\par 22 Då föllo de ned på sina ansikten och sade: "O Gud, du Gud som råder över allt kötts anda, skall du förtörnas på hela menigheten, därför att en enda man syndar?"
\par 23 Då talade HERREN till Mose och sade:
\par 24 "Tala till menigheten och säg: Dragen eder bort ifrån platsen runt omkring Koras, Datans och Abirams lägerställe."
\par 25 Och Mose stod upp och gick till Datan och Abiram, och de äldste i Israel följde honom.
\par 26 Och han talade till menigheten och sade: "Viken bort ifrån dessa ogudaktiga människors tält, och kommen icke vid något som tillhör dem, på det att I icke mån förgås genom alla deras synder."
\par 27 Då drogo de sig bort ifrån platsen runt omkring Koras, Datans och Abirams lägerställe; men Datan och Abiram hade gått ut och ställt sig vid ingången till sina tält med sina hustrur och barn, både stora och små.
\par 28 Och Mose sade: "Därav skolen I förnimma att det är HERREN som har sänt mig för att göra alla dessa gärningar, och att jag icke har handlat efter eget tycke:
\par 29 om dessa dö på samma sätt som andra människor dö, eller drabbas av hemsökelse på samma sätt som andra människor, så har HERREN icke sänt mig;
\par 30 men om HERREN här låter något alldeles nytt ske, i det att marken öppnar sin mun och uppslukar dem med allt vad de hava, så att de levande fara ned i dödsriket, då skolen I därav veta att dessa människor hava föraktat HERREN."
\par 31 Och just som han hade slutat att tala allt detta, rämnade marken under dem,
\par 32 och jorden öppnade sin mun och uppslukade dem och deras hus och allt Koras folk och alla deras ägodelar;
\par 33 och de foro levande ned i dödsriket, de med allt vad de hade, och jorden övertäckte dem, och så utrotades de ur församlingen.
\par 34 Och hela Israel, som stod runt omkring dem, flydde vid deras rop, ty de fruktade att bliva uppslukade av jorden.
\par 35 Men eld gick ut från HERREN och förtärde de två hundra femtio männen som hade burit fram rökelse.
\par 36 Och HERREN talade till Mose och sade:
\par 37 "Säg till Eleasar, prästen Arons son, att han skall taga fyrfaten ut ur branden, men kasta ut elden i dem långt bort,
\par 38 ty de hava blivit heliga. Och dessa fyrfat - de mäns som genom sin synd förverkade sina liv - dem skall man hamra ut till plåtar för att därmed överdraga altaret; ty de hava varit framburna inför HERRENS ansikte och hava därigenom blivit heliga. Och de skola så vara ett tecken för Israels barn."
\par 39 Då tog prästen Eleasar kopparfyrfaten som de uppbrända männen hade burit fram, och man hamrade ut dem för att därmed överdraga altaret,
\par 40 till en påminnelse för Israels barn att ingen främmande, ingen som icke vore av Arons säd, måtte träda fram för att antända rökelse inför HERRENS ansikte, på det att det icke skulle gå honom såsom det gick Kora och hans hop: allt i enlighet med vad HERREN hade sagt honom genom Mose.
\par 41 Men dagen därefter knorrade Israels barns hela menighet emot Mose och Aron och sade: "Det är I som haven dödat HERRENS folk."
\par 42 Då nu menigheten församlade sig emot Mose och Aron, vände dessa sig mot uppenbarelsetältet och fingo då se molnskyn övertäcka det; och HERRENS härlighet visade sig.
\par 43 Då gingo Mose och Aron fram inför uppenbarelsetältet.
\par 44 Och HERREN talade till Mose och sade:
\par 45 "Gån bort ifrån denna menighet, så skall jag i ett ögonblick förgöra dem." Då föllo de ned på sina ansikten.
\par 46 Och Mose sade till Aron: "Tag ditt fyrfat och lägg eld från altaret därpå och strö rökelse därpå, och bär det så med hast bort till menigheten och bringa försoning för dem; ty förtörnelse har gått ut från HERRENS ansikte, och hemsökelsen har begynt."
\par 47 Då tog Aron det som Mose hade tillsagt honom och skyndade mitt in i församlingen, och se, hemsökelsen hade redan begynt ibland folket, men han lade rökelsen på och bragte försoning för folket.
\par 48 När han så stod mellan de döda och de levande, upphörde hemsökelsen.
\par 49 Men de som hade omkommit genom hemsökelsen utgjorde fjorton tusen sju hundra, förutom dem som hade omkommit för Koras skull.
\par 50 Sedan vände Aron tillbaka till Mose vid uppenbarelsetältets ingång; och hemsökelsen hade upphört.

\chapter{17}

\par 1 Och HERREN talade till Mose och sade:
\par 2 "Tala till Israels barn, och tag av dem, av alla som bland dem äro hövdingar för stamfamiljer, en stav för var stamfamilj, tillsammans tolv stavar. Vars och ens namn skall du skriva på hans stav.
\par 3 Och Arons namn skall du skriva på Levi stav; ty huvudmannen för denna stams familjer skall hava sin särskilda stav.
\par 4 Sedan skall du lägga in dem i uppenbarelsetältet framför vittnesbördet, där jag uppenbarar mig för eder.
\par 5 Då skall ske att den man som jag utväljer, hans stav skall grönska. Och så skall jag göra slut på Israels barns knorrande, så att jag slipper höra huru de knorra mot eder."
\par 6 Och Mose talade till Israels barn, och hövdingarna för deras stamfamiljer gåvo honom alla var och en sin stav, tillsammans tolv stavar; och Arons stav var med bland deras stavar.
\par 7 Och Mose lade stavarna inför HERRENS ansikte i vittnesbördets tält.
\par 8 När nu Mose dagen därefter gick in i vittnesbördets tält, se, då grönskade Arons stav, som var där för Levi hus, den hade knoppar och utslagna blommor och mogna mandlar.
\par 9 Och Mose bar alla stavarna ut från HERRENS ansikte till alla Israels barn; och de sågo på dem och togo var och en sin stav.
\par 10 Och HERREN sade till Mose: "Lägg Arons stav tillbaka framför vittnesbördet, för att den där må förvaras såsom ett tecken för de gensträviga; så skall du göra en ände på deras knorrande, så att jag slipper höra det, på det att de icke må dö."
\par 11 Och Mose gjorde så; såsom HERREN hade bjudit honom, så gjorde han.
\par 12 Och Israels barn ropade till Mose: "Se, vi omkomma, vi förgås, vi förgås allasammans!
\par 13 Var och en som kommer därvid, som kommer vid HERRENS tabernakel, han dör. Skola vi då verkligen alla omkomma?"

\chapter{18}

\par 1 Och HERREN sade till Aron: Du och dina söner, och din faders hus jämte dig, skolen bära den missgärning som vidlåder helgedomen; och du och dina söner jämte dig skolen bära den missgärning som vidlåder edert prästämbete.
\par 2 Men också dina fränder, Levi stam, din faders stam, skall du låta få tillträde dit jämte dig, och de skola hålla sig till dig och betjäna dig, under det att du och dina söner jämte dig gören tjänst inför vittnesbördets tält.
\par 3 Och de skola iakttaga vad du har att iakttaga, och vad som eljest är att iakttaga vid hela tältet; men de må icke komma vid de heliga redskapen eller altaret, på det att icke både de och I mån dö.
\par 4 De skola hålla sig till dig och iakttaga vad som är att iakttaga vid uppenbarelsetältet, under all tjänstgöring vid tältet; men ingen främmande får komma eder nära.
\par 5 Och I skolen iakttaga vad som är att iakttaga vid helgedomen och vid altaret, på det att icke förtörnelse åter må komma över Israels barn.
\par 6 Se, jag har uttagit edra bröder, leviterna, bland Israels barn; en gåva äro de åt eder, givna åt HERREN, till att förrätta tjänsten vid uppenbarelsetältet.
\par 7 Men du och dina söner jämte dig skolen iakttaga vad som hör till edert prästämbete, i allt vad som angår altaret och det som är innanför förlåten, och skolen så göra tjänst. Jag giver eder edert prästämbete såsom en gåvotjänst; men om någon främmande kommer därvid, skall han dödas.
\par 8 Och HERREN talade till Aron: Se, jag giver åt dig vad som skall förvaras av det som gives mig såsom gärd. Av Israels barns alla heliga gåvor giver jag detta till ämbetslott åt dig och dina söner, såsom en evärdlig rätt.
\par 9 Detta skall tillhöra dig av det! högheliga som icke lämnas åt elden: alla deras offergåvor, så ofta de frambära spisoffer eller syndoffer, eller frambära skuldoffer till ersättning åt mig, detta skall såsom högheligt tillhöra dig och dina söner.
\par 10 På en höghelig plats skall du äta detta; allt mankön må äta det, det skall vara dig heligt.
\par 11 Och detta är rad som skall tillhöra dig såsom en gärd av Israels barns gåvor, så ofta de frambära viftoffer; åt dig och åt dina söner och döttrar jämte dig giver jag det till en evärdlig rätt; var och en i ditt hus som är ren må äta det:
\par 12 allt det bästa av olja och allt det bästa av vin och av säd, förstlingen därav, som de giva åt HERREN, detta giver jag åt dig.
\par 13 Förstlingsfrukterna av allt som växer i deras land, vilka de bära fram åt HERREN, skola tillhöra dig; var och en i ditt hus som är ren må äta därav.
\par 14 Allt tillspillogivet i Israel skall tillhöra dig.
\par 15 Allt det som öppnar moderlivet, vad kött det vara må, evad det är människor eller boskap som de föra fram till HERREN, det skall tillhöra dig; dock så, att du tager lösen för det som är förstfött bland människor, och likaledes tager lösen för det som är förstfött bland orena djur.
\par 16 Och vad angår dem som skola lösas, skall du taga lösen för dem, när de äro en månad gamla, och detta efter det värde du har bestämt: fem siklar silver, efter helgedomssikelns vikt, denna räknad till tjugu gera.
\par 17 Men för det som är förstfött bland fäkreatur eller får eller getter må du icke taga lösen; det är heligt. Deras blod skall du stänka på altaret, och deras fett skall du förbränna såsom ett eldsoffer, till en välbehaglig lukt för HERREN.
\par 18 Men deras kött skall tillhöra dig; det skall tillhöra dig likasom viftoffersbringan och det högra lårstycket.
\par 19 Alla heliga gåvor som Israels barn giva åt HERREN såsom gärd, dem giver jag åt dig och åt dina söner och döttrar jämte dig, såsom en evärdlig rätt. Ett evärdligt saltförbund inför HERRENS ansikte skall detta vara för dig och för dina avkomlingar jämte dig.
\par 20 och HERREN sade till Aron: I deras land skall du icke hava någon arvedel, och du skall icke hava någon lott bland dem; jag skall vara din lott och arvedel bland Israels barn.
\par 21 Och se, åt Levi barn giver jag all tionde i Israel till arvedel, såsom lön för den tjänst de förrätta, tjänsten vid uppenbarelsetältet.
\par 22 Men de övriga israeliterna må hädanefter icke komma vid uppenbarelsetältet, ty de skola därigenom komma att bära på synd och så träffas av döden;
\par 23 utan leviterna skola förrätta tjänsten vid uppenbarelsetältet, och de skola bära de missgärningar som begås. Detta skall vara en evärdlig stadga för eder från släkte till släkte; bland Israels barn skola de icke hava någon arvedel.
\par 24 Ty den tionde som Israels barn giva åt HERREN såsom gärd, den giver jag åt leviterna till arvedel. Därför är det som jag säger om dem att de icke skola hava någon arvedel bland Israels barn.
\par 25 Och HERREN talade till Mose och sade:
\par 26 Till leviterna skall du så tala och säga: När I av Israels barn mottagen den tionde som jag har bestämt att I skolen få av dem såsom eder arvedel, då skolen I därav giva en gärd åt HERREN, en tionde av tionden.
\par 27 Och denna eder gärd skall så anses, som när andra giva säd från logen och vin och olja från pressen.
\par 28 På detta sätt skolen ock I av all tionde som I mottagen av Israels barn giva en gärd åt HERREN; och denna HERRENS gärd av tionden skolen I giva åt prästen Aron.
\par 29 Av alla gåvor som I fån skolen giva åt HERREN hela den gärd som tillkommer honom; av allt det bästa av gåvorna skolen I giva den, sådant bland dessa som passar till heliga gåvor.
\par 30 Och du skall säga till dem: När I nu given såsom gärd det bästa av dem, skall denna leviternas gåva så anses, som när andra giva vad loge och press avkasta.
\par 31 I med edert husfolk mån äta det på vilken plats som helst; ty det är eder lön för eder tjänstgöring vid uppenbarelsetältet.
\par 32 När I så given det bästa av dem såsom gärd, skolen I icke för deras skull komma att bära på synd; och då skolen I icke ohelga Israels barns heliga gåvor och så träffas av döden.

\chapter{19}

\par 1 Och HERREN talade till Mose och Aron och sade:
\par 2 Detta är den lagstadga som HERREN har påbjudit: Säg till Israels barn att de skaffa fram till dig en röd, felfri ko, en som icke har något lyte, och som icke har burit något ok.
\par 3 Denna skolen I lämna åt prästen Eleasar; och man skall föra ut henne utanför lägret och slakta henne i hans åsyn.
\par 4 Och prästen Eleasar skall taga något av hennes blod på sitt finger, och stänka med hennes blod sju gånger mot framsidan av uppenbarelsetältet.
\par 5 Sedan skall man bränna upp kon inför hans ögon; hennes hud och kött och blod jämte hennes orenlighet skall man bränna upp.
\par 6 Och prästen skall taga cederträ, isop och rosenrött garn och kasta det i elden vari kon brännes upp.
\par 7 Och prästen skall två sina kläder och bada sin kropp i vatten; därefter får han gå in i lägret. Dock skall prästen vara oren ända till aftonen.
\par 8 Också den som brände upp henne skall två sina kläder i vatten och bada sin kropp i vatten, och vara oren ända till aftonen.
\par 9 Och en man som är ren skall samla ihop askan efter kon och lägga den utanför lägret på en ren plats. Den skall förvaras åt Israels barns menighet, till stänkelsevatten. Det är ett syndoffer.
\par 10 Och mannen som samlade ihop askan efter kon skall två sina kläder och vara oren ända till aftonen. Detta skall vara en evärdlig stadga för Israels barn och för främlingen som bor ibland dem.
\par 11 Den som kommer vid någon död, vid en människas lik, han skall vara oren i sju dagar.
\par 12 Han skall rena sig härmed på tredje dagen och på sjunde dagen, så bliver han ren. Men om han icke renar sig på tredje dagen och på sjunde dagen, så bliver han icke ren.
\par 13 Var och en som kommer vid någon död, vid liket av en människa som har dött, och sedan icke renar sig, han orenar HERRENS tabernakel, och han skall utrotas ur Israel. Därför att stänkelsevatten icke har blivit stänkt på honom, skall han vara oren; orenhet låder alltjämt vid honom.
\par 14 Detta är lagen: När en människa dör i ett tält, skall var och en som kommer in i tältet och var och en som redan är i tältet vara oren i sju dagar.
\par 15 Och alla öppna kärl, alla som icke hava stått överbundna, skola vara orena.
\par 16 Och var och en som ute på marken kommer vid någon som har fallit för svärd eller på annat sätt träffats av döden, eller vid människoben eller vid en grav, han skall vara oren i sju dagar.
\par 17 Och för att rena den som så har blivit oren skall man taga av askan efter det uppbrända syndoffret och gjuta friskt vatten därpå i ett kärl.
\par 18 Och en man som är ren skall taga isop och doppa i vattnet och stänka på tältet och på allt bohaget, och på de personer som hava varit därinne, och på honom som har kommit vid benen eller vid den fallne eller vid den som har dött på annat sätt, eller vid graven.
\par 19 Och mannen som är ren skall på tredje dagen och på sjunde dagen bestänka den som har blivit oren. När så på sjunde dagen hans rening är avslutad, skall han två sina kläder och bada sig i vatten, så bliver han ren om aftonen.
\par 20 Men om någon har blivit oren och sedan icke renar sig, skall han utrotas ur församlingen; ty han har orenat HERRENS helgedom; stänkelsevatten har icke blivit stänkt på honom, han är oren.
\par 21 Och detta skall vara för dem en evärdlig stadga. Mannen som stänkte stänkelsevattnet skall två sina kläder; och om någon annan kommer vid stänkelsevattnet, skall han vara oren ända till aftonen.
\par 22 Och allt som den orene kommer vid skall vara orent, och den som kommer vid honom skall vara oren ända till aftonen.

\chapter{20}

\par 1 Och Israels barn, hela menigheten, kommo in i öknen Sin i den första månaden, och folket stannade i Kades; där dog Mirjam och blev där också begraven.
\par 2 Och menigheten hade intet vatten; då församlade de sig emot Mose och Aron.
\par 3 Och folket begynte tvista med Mose och sade: "O att också vi hade fått förgås, när våra broder förgingos inför HERRENS ansikte!
\par 4 Varför haven I fört HERRENS församling in i denna öken, så att vi och vår boskap måste dö här?
\par 5 Och varför haven I fört oss upp ur Egypten och låtit oss komma till denna svåra plats, där varken säd eller fikonträd eller vinträd eller granatträd växa, och där intet vatten finnes att dricka?"
\par 6 Men Mose och Aron gingo bort ifrån församlingen till uppenbarelsetältets ingång och föllo ned på sina ansikten. Då visade sig HERRENS härlighet för dem.
\par 7 Och HERREN talade till Mose och sade:
\par 8 "Tag staven, och församla menigheten, du med din broder Aron, och talen till klippan inför deras ögon, så skall den giva vatten ifrån sig; så skaffar du fram vatten åt dem ur klippan och giver menigheten och dess boskap att dricka.
\par 9 Då tog Mose stav en från dess plats inför HERRENS ansikte, såsom han hade bjudit honom.
\par 10 Och Mose och Aron sammankallade församlingen framför klippan; där sade han till dem: "Hören nu, I gensträvige; kunna vi väl ur denna klippa skaffa fram vatten åt eder?"
\par 11 Och Mose lyfte upp sin hand och slog på klippan med sin stav två gånger; då kom mycket vatten ut, så att menigheten och dess boskap fick dricka.
\par 12 Men HERREN sade till Mose och Aron: "Eftersom I icke trodden på mig och icke höllen mig helig inför Israels barns ögon, därför skolen I icke få föra denna församling in i det land som jag har givit dem."
\par 13 Detta var Meribas vatten, där Israels barn tvistade med HERREN, och där han bevisade sig helig på dem.
\par 14 Och Mose skickade sändebud från Kades till konungen i Edom och lät säga: "Så säger din broder Israel: Du känner alla de vedermödor som vi hava haft att utstå,
\par 15 huru våra fäder drogo ned till Egypten, och huru vi bodde i Egypten i lång tid, och huru vi och våra fäder blevo illa behandlade av egyptierna.
\par 16 Men vi ropade till HERREN, och han hörde vår röst och sände en ängel som förde oss ut ur Egypten; och se, vi äro nu i Kades, staden som ligger vid gränsen till ditt område.
\par 17 Låt oss tåga genom ditt land. Vi skola icke taga vägen över åkrar och vingårdar, och icke dricka vatten ur brunnarna; stora vägen skola vi gå, utan att vika av vare sig till höger eller till vänster, till dess vi hava kommit igenom ditt område."
\par 18 Men Edom svarade honom: "Du får icke tåga genom mitt land. Om du det gör, skall jag draga ut emot dig med svärd."
\par 19 Men Israels barn sade till honom: "På den allmänna farvägen skola vi draga fram, och om jag eller min boskap dricker av ditt vatten, skall jag betala det. Jag begär ju ingenting: allenast att få tåga vägen fram härigenom."
\par 20 Han svarade: "Nej, du får icke tåga härigenom." Och Edom drog ut mot honom med mycket folk och ned stor makt.
\par 21 Då alltså Edom icke tillstadde Israel att tåga genom sitt område, vek Israel av och gick undan för honom.
\par 22 Och de bröto upp från Kades. Och Israels barn, hela menigheten, kommo till berget Hor.
\par 23 Och HERREN talade till Aron på berget Hor, vid gränsen till Edoms land, och sade:
\par 24 "Aron skall samlas till sina fäder: han skall icke komma in i det land som jag har givit åt Israels barn; ty I voren gensträviga mot min befallning vid Meribas vatten.
\par 25 Tag nu Aron och hans son Eleasar med dig, och för dem upp på berget Hor,
\par 26 och tag av Aron hans kläder och sätt dem på hans son Eleasar. Så skall Aron samlas till sina fäder och I dö där."
\par 27 Och Mose gjorde såsom HERREN hade bjudit; och de stego upp på berget Hor inför hela menighetens ögon.
\par 28 Och Mose tog av Aron hans kläder och satte dem på hans son Eleasar. Och Aron dog där uppe på bergets topp; men Mose och Eleasar stego ned från berget.
\par 29 Och när hela menigheten förnam att Aron hade givit upp andan, begräto de honom i trettio dagar, hela Israels hus.

\chapter{21}

\par 1 Då nu konungen i Arad, kananéen, som bodde i Sydlandet, hörde att Israel var i antågande på Atarimvägen, gav han sig i strid med Israel och tog några av dem till fånga.
\par 2 Då gjorde Israel ett löfte åt HERREN och sade: "Om du giver detta folk i min hand, så skall jag giva deras städer till spillo."
\par 3 Och HERREN hörde Israels röst och gav kananéerna i deras hand, och de gåvo dem och deras städer till spillo; så fick stället namnet Horma.
\par 4 Och de bröto upp från berget Hor och togo vägen åt Röda havet till, för att gå omkring Edoms land. Men under vägen blev folket otåligt.
\par 5 Och folket talade emot Gud och emot Mose och sade: "Varför haven I fört oss upp ur Egypten, så att vi måste dö i öknen? Här finnes ju varken bröd eller vatten, och vår själ vämjes vid den usla föda vi få."
\par 6 Då sände HERREN giftiga ormar bland folket, och dessa stungo folket; och mycket folk i Israel blev dödat.
\par 7 Då kom folket till Mose och sade: "Vi hava syndat därmed att vi talade mot HERREN och mot dig. Bed till HERREN att han tager bort dessa ormar ifrån oss." Och Mose bad för folket.
\par 8 Då sade HERREN till Mose: "Gör dig en orm och sätt upp den på en stång; sedan må var och en som har blivit ormstungen se på den, så skall han bliva vid liv."
\par 9 Då gjorde Mose en orm av koppar och satte upp den på en stång; när sedan någon hade blivit stungen av en orm, såg han upp på kopparormen och blev så vid liv.
\par 10 Och Israels barn bröto upp och lägrade sig i Obot;
\par 11 och från Obot bröto de upp och lägrade sig vid Ije-Haabarim i öknen som ligger framför Moab, österut.
\par 12 Därifrån bröto de upp och lägrade sig i Sereds dal.
\par 13 Därifrån bröto de upp och lägrade sig på andra sidan Arnon, där denna bäck från amoréernas område, där den har runnit upp, flyter fram i öknen; ty Arnon är Moabs gräns och flyter fram mellan Moabs land och amoréernas.
\par 14 Därför heter det i "Boken om HERRENS krig": "Vaheb i Sufa och dalarna där Arnon går fram,
\par 15 och dalarnas sluttning, som sänker sig mot Ars bygd och stöder sig mot Moabs gräns."
\par 16 Därifrån drogo de till Beeri om brunnen där var det som HERREN sade till Mose: "Församla folket, så vill jag giva dem vatten."
\par 17 Då sjöng Israel denna sång: "Flöda, du brunn! Ja, sjungen om den,
\par 18 om brunnen som furstar grävde, som folkets ypperste borrade, med spiran, med sina stavar.
\par 19 Från öknen drogo de till Mattana, från Mattana till Nahaliel, från: Nahaliel till Bamot,
\par 20 från Bamot till den dal som ligger: på Moabs mark uppe på Pisga, där man kan se ut över ödemarken.
\par 21 Och Israel skickade sändebud till Sihon, amoréernas konung, och lät säga:
\par 22 "Låt mig taga genom ditt land. Vi skola icke vika av ifrån vägen in i åkrar eller vingårdar, och icke dricka vatten ur brunnarna. Stora vägen skola vi gå, till dess vi hava kommit igenom ditt område."
\par 23 Men Sihon tillstadde icke Israel att tåga genom sitt område, utan församlade allt sitt folk och drog ut mot Israel i öknen, till dess han kom till Jahas; där gav han sig i strid med Israel.
\par 24 Men Israel slog honom med svärdsegg och intog hans land från Arnon ända till Jabbok, ända till Ammons barns land, ty Ammons barns gräns var befäst.
\par 25 Och Israel intog alla städerna där; och Israel bosatte sig i amoréernas alla städer, i Hesbon och alla underlydande orter.
\par 26 Hesbon var nämligen Sihons, amoréernas konungs, stad, ty denne hade fört krig med den förre konungen i Moab och tagit ifrån honom hela hans land ända till Arnon.
\par 27 Därför säga skalderna: "Kommen till Hesbon! Byggas och befästas skall Sihons stad.
\par 28 Ty eld gick ut från Hesbon, en låga från Sihons stad; den förtärde Ar i Moab, dem som bodde på Arnons höjder.
\par 29 Ve dig, Moab! Förlorat är du, Kemos' folk! Han lät sina söner bliva slagna på flykten och sina döttrar föras bort i fångenskap, bort till amoréernas konung, Sihon.
\par 30 Vi sköto ned dem - förlorat var Hesbon, landet ända till Dibon; vi härjade ända till Nofa, Nofa, som når till Medeba."
\par 31 Så bosatte sig då Israel i amoréernas land.
\par 32 Och Mose sände ut och lät bespeja Jaeser, och de intogo dess underlydande orter; och han fördrev amoréerna som bodde där.
\par 33 Sedan vände de sig åt annat hål. och drogo upp åt Basan till. Och Og, konungen i Basan, drog med allt sitt folk ut till strid mot dem, till Edrei.
\par 34 Men HERREN sade till Mose: "Frukta icke för honom, ty i din hand har jag givit honom och allt hans folk och hans land. Och du skall göra med honom på samma sätt som du gjorde med Sihon, amoréernas konung, som bodde i Hesbon."
\par 35 Och de slogo honom jämte hans söner och allt hans folk, och läto ingen av dem slippa undan. Så intogo de hans land.

\chapter{22}

\par 1 Och Israels barn bröto upp och lägrade sig på Moabs hedar, på andra sidan Jordan mitt emot Jeriko.
\par 2 Och Balak, Sippors son, såg allt vad Israel hade gjort mot amoréerna.
\par 3 Och Moab bävade storligen för folket, därför att det var så talrikt; Moab gruvade sig för Israels barn.
\par 4 Och Moab sade till de äldste i Midjan: "Nu kommer denna hop att äta upp allt som finnes här runt omkring oss, likasom oxen äter upp vad grönt som finnes på marken." Och Balak, Sippors son, var på den tiden konung i Moab.
\par 5 Och han skickade sändebud till Bileam, Beors son, i Petor vid floden, i hans stamfränders land, för att kalla honom till sig; han lät säga: "Se, här är ett folk som har dragit ut ur Egypten; se, det övertäcker marken, och det har lägrat sig mitt emot mig.
\par 6 Så kom nu och förbanna åt mig detta folk, ty det är mig för mäktigt; kanhända skall jag då kunna slå det och förjaga det ur landet. Ty jag vet att den du välsignar, han är välsignad, och den du förbannar, han bliver förbannad."
\par 7 Så gingo nu de äldste i Moab och de äldste i Midjan åstad och hade med sig spådomslön; och de kommo till Bileam och framförde till honom Balaks ord.
\par 8 Och han sade till dem: "Stannen här över natten, så vill jag sedan giva eder svar efter vad HERREN talar till mig." Då stannade Moabs furstar kvar hos Bileam.
\par 9 Och Gud kom till Bileam; han sade: "Vad är det för män som du har hos dig?"
\par 10 Bileam svarade Gud: "Balak, Sippors son, konungen i Moab, har sänt till mig detta bud:
\par 11 'Se, här är det folk som har dragit ut ur Egypten, och det övertäcker marken. Så kom nu och förbanna det åt mig; kanhända skall jag då kunna giva mig i strid med det och förjaga det.'"
\par 12 Då sade Gud till Bileam: "Du skall icke gå med dem; du skall icke förbanna detta folk, ty det är välsignat."
\par 13 Om morgonen, när Bileam hade stått upp, sade han alltså till Balaks furstar: "Gån hem till edert land, ty HERREN vill icke tillstädja mig att följa med eder."
\par 14 Då stodo Moabs furstar upp och gingo hem till Balak och sade: "Bileam vägrade att följa med oss."
\par 15 Men Balak sände ännu en gång åstad furstar, flera och förnämligare än de förra.
\par 16 Och de kommo till Bileam och sade till honom: "Så säger Balak Sippors son: 'Låt dig icke avhållas från att komma till mig;
\par 17 ty jag vill bevisa dig övermåttan stor ära, och allt vad du begär av mig skall jag göra. Kom nu och förbanna åt mig detta folk.'"
\par 18 Då svarade Bileam och sade till Balaks tjänare: "Om Balak än gåve mig så mycket silver och guld som hans hus rymmer, kunde jag dock icke överträda HERRENS min Guds befallning, så att jag gjorde något däremot, vare sig litet eller stort.
\par 19 Men stannen nu också I kvar har över natten, för att jag må förnimma vad HERREN ytterligare kan vilja tala till mig."
\par 20 Och Gud kom till Bileam om natten och sade till honom: "Om dessa män hava kommit för att kalla dig, så stå upp och följ med dem. Men allenast vad jag säger dig skall du göra."
\par 21 Om morgonen, när Bileam hade stått upp, sadlade han alltså sin åsninna och följde med Moabs furstar.
\par 22 Men då han nu följde med, upptändes Guds vrede, och HERRENS ängel ställde sig på vägen för att hindra honom, där han red på sin åsninna, åtföljd av två sina tjänare.
\par 23 När då åsninnan såg HERRENS ängel stå på vägen med ett draget svärd i sin hand, vek hon av ifrån vägen och gick in på åkern; men Bileam slog åsninnan för att driva henne tillbaka in på vägen.
\par 24 Därefter ställde sig HERRENS ängel i en smal gata mellan vingårdarna, där murar funnos på båda sidor.
\par 25 När nu åsninnan såg HERRENS ängel, trängde hon sig mot muren och klämde så Bileams ben mot muren; och han slog henne ännu en gång.
\par 26 Då gick HERRENS ängel längre fram och ställde sig på ett trångt ställe, där ingen utväg fanns att vika undan, vare sig till höger eller till vänster.
\par 27 När åsninnan nu såg HERRENS ängel, lade hon sig ned under Bileam. Då upptändes Bileams vrede och han slog åsninnan med sin stav.
\par 28 Men HERREN öppnade åsninnans mun, och hon sade till Bileam: "Vad har jag gjort dig, eftersom du nu tre gånger har slagit mig?"
\par 29 Bileam svarade åsninnan: "Du har ju handlat skamligt mot mig; om jag hade haft ett svärd i min hand, skulle jag nu hava dräpt dig."
\par 30 Men åsninnan sade till Bileam: "Är icke jag din egen åsninna, som du har ridit på i all din tid intill denna dag? Och har jag någonsin förut plägat göra så mot dig? Han svarade: "Nej."
\par 31 Och HERREN öppnade Bileams ögon, så att han såg HERRENS ängel stå på vägen med ett draget svärd i sin hand. Då bugade han sig och föll ned på sitt ansikte.
\par 32 Och HERRENS ängel sade till honom: "Varför har du nu tre gånger slagit din åsninna? Se, jag har gått ut för att hindra dig, ty denna väg leder till fördärv och är mig emot.
\par 33 Och åsninnan såg mig, och hon har nu tre gånger vikit undan for mig. Om hon icke hade vikit undan för mig, så skulle jag nu hava dräpt dig, men låtit henne leva."
\par 34 Då sade Bileam till HERRENS ängel: "Jag har syndat, ty jag visste icke att du stod mig emot på vägen. Om nu min resa misshagar dig, så vill jag vända tillbaka."
\par 35 Men HERRENS ängel svarade Bileam: "Följ med dessa män; men intet annat än vad jag säger dig skall du tala." Så följde då Bileam med Balaks furstar.
\par 36 När Balak hörde att Bileam kom, gick han honom till mötes till Ir i Moab, vid gränsen där Arnon flyter, vid yttersta gränsen.
\par 37 Och Balak sade till Bileam: "Sände jag icke enträget bud till dig för att kalla dig hit? Varför ville du då icke begiva dig till mig? Skulle jag icke kunna bevisa dig tillräcklig ära?"
\par 38 Bileam svarade Balak: "Du ser nu att jag har kommit till dig. Men det står ingalunda i min egen makt att tala något. Vad Gud lägger i min mun, det måste jag tala."
\par 39 Sedan följde Bileam med Balak, och de kommo till Kirjat-Husot.
\par 40 Och Balak slaktade fäkreatur och småboskap och sände till Bileam och de furstar som voro med honom.
\par 41 Och följande morgon tog Balak Bileam med sig och förde honom upp på Bamot-Baal; och han kunde från denna plats se en del av folket.

\chapter{23}

\par 1 Och Bileam sade till Balak: "Bygg här åt mig sju altaren, och skaffa hit åt mig sju tjurar och sju vädurar."
\par 2 Balak gjorde såsom Bileam sade; och Balak och Bileam offrade en tjur och en vädur på vart altare
\par 3 Därefter sade Bileam till Balak "Stanna kvar vid ditt brännoffer; jag vill gå bort och se om till äventyrs HERREN visar sig för mig; och vad helst han uppenbarar för mig, det skall jag förkunna för dig." Och han gick upp på en kal höjd.
\par 4 Och Gud visade sig för Bileam; då sade denne till honom: "De sju altarna har jag uppfört, och på vart altare har jag offrat en tjur och en vädur."
\par 5 Och HERREN lade i Bileams mun vad han skulle tala; han sade: "Gå tillbaka till Balak och tala så och så."
\par 6 När han nu kom tillbaka till honom, fann han honom stående vid sitt brännoffer tillsammans med alla Moabs furstar.
\par 7 Då hör han upp sin röst och kvad: "Från Aram hämtade mig Balak, från österns berg Moabs konung: 'Kom och förbanna åt mig Jakob, kom och tala ofärd över Israel.
\par 8 Huru kan jag förbanna den gud ej förbannar, och tala ofärd över den som HERREN ej talar ofärd över?
\par 9 Från klippornas topp ser jag ju honom, och från höjderna skådar jag honom: se, det är ett folk som bor för sig självt och icke anser sig likt andra folkslag.
\par 10 Vem kan räkna Jakob, tallös såsom stoftet, eller tälja ens fjärdedelen av Israel? Må jag få dö de rättfärdigas död, och blive mitt slut såsom deras!"
\par 11 Då sade Balak till Bileam: "Vad har du gjort mot mig! Till att förbanna mina fiender hämtade jag dig, och nu har du i stället välsignat dem."
\par 12 Men han svarade och sade: "Skulle jag då icke akta på vad HERREN lägger i min mun, och tala det?"
\par 13 Och Balak sade till honom: "Följ nu med mig till ett annat ställe, varifrån du ser dem; du ser här allenast en del av dem, du ser dem icke allasammans. Från det stället må du förbanna dem åt mig."
\par 14 Och han tog honom med sig till Väktarplanen på toppen av Pisga. Där byggde han sju altaren och offrade en tjur och en vädur på vart altare.
\par 15 Därefter sade han till Balak: "Stanna kvar har vid ditt brännoffer; jag själv vill därborta se till, om något visar sig."
\par 16 Och HERREN visade sig för Bileam och lade i hans mun vad han skulle tala; han sade: "Gå tillbaka till Balak och tala så och så."
\par 17 När han nu kom till honom, fann han honom stående vid sitt brännoffer, och Moabs furstar stodo där med honom. Och Balak frågade honom: "Vad har HERREN talat?"
\par 18 Då hov han upp sin röst och kvad: "Stå upp, Balak, och hör; lyssna till mig, du Sippors son.
\par 19 Gud är icke en människa, så att han kan ljuga, icke en människoson, så att han kan ångra något. Skulle han säga något och icke göra det, tala något och icke fullborda det?
\par 20 Se, att välsigna har jag fått i uppdrag; han har välsignat, och jag kan icke rygga det.
\par 21 Ofärd är icke att skåda i Jakob och olycka icke att se i Israel. HERREN, hans Gud, är med honom, och jubel såsom mot en konung höres där.
\par 22 Det är Gud som har fört dem ut ur Egypten; deras styrka är såsom vildoxars.
\par 23 Ty trolldom båtar intet mot Jakob, ej heller spådom mot Israel. Nej, nu måste sägas om Jakob och om Israel: 'Vad gör icke Gud!'
\par 24 Se, det är ett folk som står upp likt en lejoninna, ett folk som reser sig likasom ett lejon. Det lägger sig ej ned, förrän det har ätit rov och druckit blod av slagna man."
\par 25 Då sade Balak till Bileam: "Om du nu icke vill förbanna dem, så må du åtminstone icke välsigna dem.
\par 26 Men Bileam svarade och sade till Balak: "Sade jag icke till dig: 'Allt vad HERREN säger, det måste jag göra'?"
\par 27 Och Balak sade till Bileam: "Kom, jag vill taga dig med mig till ett annat ställe. Kanhända skall det behaga Gud att du därifrån förbannar dem åt mig.
\par 28 Och Balak tog Bileam med sig upp på toppen av Peor, där man kan se ut över ödemarken.
\par 29 Och Bileam sade till Balak: "Bygg här åt mig sju altaren, och skaffa hit åt mig sju tjurar och sju vädurar.
\par 30 Och Balak gjorde såsom Bileam sade; och han offrade en tjur och en vädur på vart altare.

\chapter{24}

\par 1 Då nu Bileam såg att det var: HERRENS vilja att han skulle välsigna Israel, gick han icke, såsom de förra gångerna, bort och såg efter tecken, utan vände sitt ansikte mot öknen.
\par 2 Och när Bileam lyfte upp sina ögon och såg Israel lägrad efter sina stammar, kom Guds Ande över honom.
\par 3 Och han hov upp sin röst och kvad: "Så säger Bileam, Beors son, så säger mannen med det slutna ögat,
\par 4 så säger han som hör Guds tal, han som skådar syner från den Allsmäktige, i det han sjunker ned och får sina ögon öppnade:
\par 5 Huru sköna äro icke dina tält, du Jakob, dina boningar, du Israel!
\par 6 De likna dalar som utbreda sig vida, de äro såsom lustgårdar invid en ström, såsom aloeträd, planterade av HERREN, såsom cedrar invid vatten.
\par 7 Vatten flödar ur hans ämbar, hans sådd bliver rikligen vattnad. Större än Agag skall hans konung vara, ja, upphöjd bliver hans konungamakt.
\par 8 Det är Gud som har fört honom ut ur Egypten. Hans styrka är såsom en vildoxes. Han skall uppsluka de folk som stå honom emot, deras ben skall han sönderkrossa, och med sina pilar skall han genomborra dem.
\par 9 Han har lagt sig ned, han vilar såsom ett lejon, såsom en lejoninna - vem vågar oroa honom? Välsignad vare den som välsignar dig, och förbannad vare den som förbannar dig!"
\par 10 Då upptändes Balaks vrede mot Bileam, och han slog ihop händerna. Och Balak sade till Bileam: "Till att förbanna mina fiender kallade jag dig hit, och se, du har i stället nu tre gånger välsignat dem.
\par 11 Giv dig nu av hem igen. Jag tänkte att jag skulle få bevisa dig stor ära; men se, HERREN har förmenat dig att bliva ärad."
\par 12 Bileam svarade Balak: "Sade jag icke redan till sändebuden som du skickade till mig:
\par 13 'Om Balak än gåve mig så mycket silver och guld som hans hus rymmer, kunde jag dock icke överträda HERRENS befallning, så att jag efter eget tycke gjorde något, vad det vara må.' Vad HERREN säger, det måste jag tala.
\par 14 Se, jag går nu hem till mitt folk; men jag vill varsko dig om vad detta folk skall göra mot ditt folk i kommande dagar."
\par 15 Och han hov upp sin röst och kvad: "Så säger Bileam, Beors son, så säger mannen med det slutna ögat,
\par 16 så säger han som hör Guds tal och har kunskap från den Högste, han som skådar syner från den Allsmäktige, i det han sjunker ned och får sina ögon öppnade:
\par 17 Jag ser honom, men icke denna tid, jag skådar honom, men icke nära. En stjärna träder fram ur Jakob, och en spira höjer sig ur Israel. Den krossar Moabs tinningar och slår ned alla söner till Set.
\par 18 Edom skall han få till besittning till besittning Seir - sina fienders länder. Ty Israel skall göra mäktiga ting;
\par 19 ur Jakob skall en härskare komma; han skall förgöra i städerna dem som rädda sig dit."
\par 20 Och han fick se Amalek; då hov han upp sin röst och kvad: "En förstling bland folken är Amalek, men på sistone hemfaller han åt undergång."
\par 21 Och han fick se kainéerna; då hov han upp sin röst och kvad: "Fast är din boning, och lagt på klippan är ditt näste.
\par 22 Likväl skall Kain bliva utrotad; ja, Assur skall omsider föra dig i fångenskap."
\par 23 Och han lov åter upp sin röst och kvad: "O ve! Vem skall bliva vid liv, när Gud låter detta ske?
\par 24 Skepp skola komma från kittéernas kust, de skola tukta Assur, tukta Eber; också han skall hemfalla åt undergång."
\par 25 Och Bileam stod upp och vände tillbaka hem; också Balak for sin väg.

\chapter{25}

\par 1 Och medan Israel uppehöll sig i Sittim, begynte folket bedriva otukt med Moabs döttrar.
\par 2 Dessa inbjödo folket till sina gudars offermåltider,
\par 3 Och folket åt och tillbad deras gudar. Och Israel slöt sig till Baal-Peor. Då upptändes HERRENS vrede mot Israel.
\par 4 Och HERREN sade till Mose: "Hämta folkets alla huvudman, och låt upphänga sådana i solen för HERREN, på det att HERRENS vredes glöd må vändas ifrån Israel."
\par 5 Då sade Mose till Israels domare: "Var och en av eder dräpe bland sina män dem som hava slutit sig till Baal-Peor."
\par 6 Nu kom en man av Israels barn och förde in bland sina bröder en midjanitisk kvinna, inför Moses och Israels barns hela menighets ögon, under det att dessa stodo gråtande vid ingången till uppenbarelsetältet.
\par 7 När Pinehas, son till Eleasar, son till prästen Aron, såg detta, stod han upp i menigheten och tog ett spjut i sin hand
\par 8 och följde efter den israelitiske mannen in i tältets sovrum och genomborrade dem båda, såväl den israelitiske mannen som ock kvinnan, i det att han stack henne genom underlivet. Så upphörde hemsökelsen bland Israels barn.
\par 9 Men de som hade omkommit genom hemsökelsen utgjorde tjugufyra tusen.
\par 10 Och HERREN talade till Mose och sade:
\par 11 "Pinehas, han som är son till prästen Arons son Eleasar, har avvänt min vrede från Israels barn, i det att han har nitälskat bland dem såsom jag nitälskar; därför har jag icke i min nitälskan förgjort Israels barn.
\par 12 Säg fördenskull: Se, jag gör med honom ett fridsförbund;
\par 13 och för honom, och för hans avkomlingar efter honom, skall detta vara ett förbund genom vilket han får ett evärdligt prästadöme, till lön för att han nitälskade för sin Gud och bragte försoning för Israels barn."
\par 14 Och den dödade israelitiske mannen, han som dödades jämte den midjanitiska kvinnan, hette Simri, Salus son, och var hövding för en familj bland simeoniterna.
\par 15 Och den dödade midjanitiska kvinnan hette Kosbi, dotter till Sur; denne var stamhövding, hövding för en stamfamilj i Midjan.
\par 16 Och HERREN talade till Mose och sade:
\par 17 "Angripen midjaniterna och slån dem.
\par 18 Ty de hava angripit eder genom de onda råd som de lade mot eder i saken med Peor och i saken med Kosbi, den midjanitiska hövdingdottern, deras syster, vilken dödades på den dag då hemsökelsen drabbade eder för Peors skull."

\chapter{26}

\par 1 Efter denna hemsökelse talade HERREN till Mose och till Eleasar, prästen Arons son, och sade:
\par 2 "Räknen antalet av Israels barn, deras hela menighet, dem som äro tjugu år gamla eller därutöver, efter deras familjer, alla stridbara män i Israel."
\par 3 Och Mose och prästen Eleasar talade till dem på Moabs hedar, vid Jordan mitt emot Jeriko, och sade:
\par 4 "De som äro tjugu år gamla eller därutöver skola räknas." Så hade ju HERREN bjudit Mose och Israels barn, dem som hade dragit ut ur Egyptens land.
\par 5 Ruben var Israels förstfödde. Rubens barn voro: Av Hanok hanokiternas släkt, av Pallu palluiternas släkt,
\par 6 av Hesron hesroniternas släkt, av Karmi karmiternas släkt.
\par 7 Dessa voro rubeniternas släkter. Och de av dem som inmönstrades utgjorde fyrtiotre tusen sju hundra trettio.
\par 8 Men Pallus söner voro Eliab.
\par 9 Och Eliabs söner voro Nemuel, Datan och Abiram; det var den Datan och den Abiram, båda ombud för menigheten, som satte sig upp emot Mose och Aron, tillika med Koras hop, när dessa satte sig upp emot HERREN,
\par 10 varvid jorden öppnade sin mun och uppslukade dem jämte Kora, vid det tillfälle då dennes hop omkom, i det att elden förtärde de två hundra femtio männen, så att de blevo till en varnagel.
\par 11 Men Koras söner omkommo icke.
\par 12 Simeons barn, efter deras släkter, voro: Av Nemuel nemueliternas släkt, av Jamin jaminiternas släkt, av Jakin jakiniternas släkt,
\par 13 av Sera seraiternas släkt, av Saul sauliternas släkt.
\par 14 Dessa voro simeoniternas släkter, tjugutvå tusen två hundra.
\par 15 Gads barn, efter deras släkter, voro: Av Sefon sefoniternas släkt, av Haggi haggiternas släkt, av Suni suniternas släkt,
\par 16 av Osni osniternas släkt, av Eri eriternas släkt,
\par 17 av Arod aroditernas släkt, av Areli areliternas släkt.
\par 18 Dessa voro Gads barns släkter, så många av dem som inmönstrades, fyrtio tusen fem hundra.
\par 19 Judas söner voro Er och Onan; men Er och Onan dogo i Kanaans land.
\par 20 Och Juda barn, efter deras släkter, voro: Av Sela selaniternas släkt, av Peres peresiternas släkt, av Sera seraiternas släkt.
\par 21 Men Peres' barn voro: Av Hesron hesroniternas släkt, av Hamul hamuliternas släkt.
\par 22 Dessa voro Juda släkter, så många av dem som inmönstrades, sjuttiosex tusen fem hundra.
\par 23 Isaskars barn, efter deras släkter, voro: Av Tola tolaiternas släkt, av Puva puniternas släkt,
\par 24 av Jasub jasubiternas släkt, av Simron simroniternas släkt.
\par 25 Dessa voro Isaskars släkter, så många av dem som inmönstrades, sextiofyra tusen tre hundra.
\par 26 Sebulons barn, efter deras släkter, voro: Av Sered serediternas släkt, av Elon eloniternas släkt, av Jaleel jaleeliternas släkt.
\par 27 Dessa voro sebuloniternas släkter, så många av dem som inmönstrades, sextio tusen fem hundra.
\par 28 Josefs barn, efter deras släkter voro Manasse och Efraim.
\par 29 Manasse barn voro: Av Makir makiriternas släkt; men Makir födde Gilead; av Gilead kom gileaditernas släkt.
\par 30 Dessa voro Gileads barn: Av Ieser ieseriternas släkt, av Helek helekiternas släkt,
\par 31 av Asriel asrieliternas släkt, av Sikem sikemiternas släkt,
\par 32 av Semida semidaiternas släkt och av Hefer heferiternas släkt.
\par 33 Men Selofhad, Hefers son, hade inga söner, utan allenast döttrar; och Selofhads döttrar hette Mahela, Noa, Hogla, Milka och Tirsa.
\par 34 Dessa voro Manasse släkter, och de av dem som inmönstrades utgjorde femtiotvå tusen sju hundra.
\par 35 Dessa voro Efraims barn, efter deras släkter: Av Sutela sutelaiternas släkt, av Beker bekeriternas släkt, av Tahan tahaniternas släkt.
\par 36 Men dessa voro Sutelas barn: Av Eran eraniternas släkt
\par 37 Dessa voro Efraims barns släkter, så många av dem som inmönstrades trettiotvå tusen fem hundra. Dessa voro Josefs barn, efter deras släkter.
\par 38 Benjamins barn, efter deras släkter, voro: Av Bela belaiternas släkt, av Asbel asbeliternas släkt, av Ahiram ahiramiternas släkt,
\par 39 av Sefufam sufamiternas släkt, av Hufam hufamiternas släkt.
\par 40 Men Belas söner voro Ard och Naaman; arditernas släkt; av Naaman naamiternas släkt.
\par 41 Dessa voro Benjamins barn, efter deras släkter, och de av dem som inmönstrades utgjorde fyrtiofem tusen sex hundra.
\par 42 Dessa voro Dans barn, efter deras släkter: Av Suham suhamiternas släkt. Dessa voro Dans släkter, efter deras släkter.
\par 43 Suhamiternas släkter, så många av dem som inmönstrades, utgjorde tillsammans sextiofyra tusen fyra hundra.
\par 44 Asers barn, efter deras släkter, voro: Av Jimna Jimnasläkten, av Jisvi jisviternas släkt, av Beria beriaiternas släkt.
\par 45 Av Berias barn: Av Heber heberiternas släkt, av Malkiel malkieliternas släkt.
\par 46 Och Asers dotter hette Sera.
\par 47 Dessa voro Asers barns släkter, så många av dem som inmönstrades, femtiotre tusen fyra hundra.
\par 48 Naftali barn, efter deras släkter, voro: Av Jaseel jaseeliternas släkt, av Guni guniternas släkt,
\par 49 av Jeser jeseriternas släkt, av Sillem sillemiternas släkt.
\par 50 Dessa voro Naftali släkter, efter deras släkter; och de av dem som inmönstrades utgjorde fyrtiofem tusen fyra hundra.
\par 51 Dessa voro de av Israels barn som inmönstrades, sen hundra ett tusen sju hundra trettio.
\par 52 Och HERREN talade till Mose och sade:
\par 53 Åt dessa skall landet utskiftas till arvedel, efter personernas antal.
\par 54 Åt en större stam skall du giva en större arvedel, och åt en mindre stam en mindre arvedel; åt var stam skall arvedel givas efter antalet av dess inmönstrade.
\par 55 Men genom lottkastning skall landet utskiftas. Efter namnen på sina fädernestammar skola de få sina arvedelar.
\par 56 Efter lottens utslag skall var stam större eller mindre, få sin arvedel sig tillskiftad.
\par 57 Och dessa voro de av Levi stam som inmönstrades, efter deras släkter: Av Gerson gersoniternas släkt av Kehat kehatiternas släkt, av Merari merariternas släkt.
\par 58 Dessa voro leviternas släkter: libniternas släkt, hebroniternas släkt maheliternas släkt, musiternas släkt koraiternas släkt. Men Kehat födde Amram.
\par 59 Och Amrams hustru hette Jokebed, Levis dotter, som föddes åt Levi i Egypten; och hon födde åt Amram Aron och Mose och deras syster Mirjam.
\par 60 Och åt Aron föddes Nadab och Abihu, Eleasar och Itamar.
\par 61 Men Nadab och Abihu träffades av döden, när de buro fram främmande eld inför HERRENS ansikte.
\par 62 Och de av dem som inmönstrades utgjorde tjugutre tusen, alla av mankön som voro en månad gamla eller därutöver. De hade nämligen icke blivit inmönstrade bland Israels barn, eftersom icke någon arvedel var given åt dem bland Israels barn.
\par 63 Dessa voro de som inmönstrades av Mose och prästen Eleasar, när dessa mönstrade Israels barn på Moabs hedar, vid Jordan mitt emot Jeriko.
\par 64 Bland dessa var ingen av dem som förut hade blivit inmönstrade av Mose och prästen Aron, när dessa mönstrade Israels barn i Sinais öken,
\par 65 ty om dem hade HERREN sagt: "De skola döden dö i öknen." Därför var ingen kvar av dem, förutom Kaleb, Jefunnes son, och Josua, Nuns son.

\chapter{27}

\par 1 Och Selofhads döttrar trädde fram Selofhads, som var son till Hefer, son till Gilead, son till Makir son till Manasse, av Manasses, Josefs sons, släkter. Och hans döttrar hette Mahela, Noa, Hogla, Milka och Tirsa.
\par 2 Dessa kommo nu inför Mose och prästen Eleasar och stamhövdingarna och hela menigheten, vid ingången till uppenbarelsetältet, och sade:
\par 3 "Vår fader har dött i öknen, men han var icke med i den hop som rotade sig samman mot HERREN, Koras hop, utan han dog genom egen synd, och han hade inga söner.
\par 4 Icke skall nu vår faders namn utplånas ur hans släkt för det att han icke hade någon son? Giv åt oss en besittning ibland vår faders bröder."
\par 5 och Mose bar fram deras sak inför HERREN.
\par 6 Då talade HERREN till Mose och sade:
\par 7 "Selofhads döttrar hava talat rätt. Du skall giva också dem en arvsbesittning bland deras faders bröder genom att låta deras faders arvedel övergå till dem.
\par 8 Och till Israels barn skall du tala och säga: När någon dör utan att efterlämna någon son, skolen I låta hans arvedel övergå till hans dotter.
\par 9 Men om han icke har någon dotter, så skolen I giva hans arvedel åt hans bröder.
\par 10 Har han icke heller några bröder, så skolen I giva hans arvedel åt hans faders bröder.
\par 11 Men om hans fader icke har några broder, så skolen I giva hans arvedel åt närmaste blodsförvant inom hans släkt, och denne skall då taga den i besittning." Detta skall vara en rättsstadga för Israels barn, såsom HERREN har bjudit Mose.
\par 12 Och HERREN sade till Mose: "Stig upp här på Abarimberget, så skall du få se det land som jag har givit åt Israels barn.
\par 13 Men när du har sett det, skall också du samlas till dina fäder, likasom din broder Aron har blivit samlad till sina fäder;
\par 14 detta därför att I, i öknen Sin, när menigheten tvistade med mig, voren gensträviga mot min befallning och icke villen hålla mig helig genom att skaffa fram vatten inför deras ögon." Detta gällde Meribas vatten vid Kades, i öknen Sin.
\par 15 Och Mose talade till HERREN och sade:
\par 16 "Må HERREN, den Gud som råder över allt kötts anda, sätta en man över menigheten,
\par 17 som kan gå i spetsen för dem, när de draga ut eller vända åter, och som kan vara deras ledare och anförare, så att icke HERRENS menighet kommer att likna får som icke hava någon herde."
\par 18 HERREN svarade Mose: "Tag till dig Josua, Nuns son, ty han är en man i vilken ande är, och lägg din hand på honom.
\par 19 Och för honom fram inför prästen Eleasar och hela menigheten, och insätt honom i hans ämbete inför deras ögon,
\par 20 och lägg något av din värdighet på honom, för att Israels barns hela menighet må lyda honom.
\par 21 Och hos prästen Eleasar skall han sedan hava att inställa sig, för att denne genom urims dom må hämta svar åt honom inför HERRENS ansikte. Efter hans ord skola de draga ut och vända åter, han själv och alla Israels barn med honom, hela menigheten."
\par 22 och Mose gjorde såsom HERREN hade bjudit honom; han tog Josua och förde honom fram inför prästen Eleasar och hela menigheten.
\par 23 Och denne lade sina händer på honom och insatte honom i hans ämbete, såsom HERREN hade befallt genom Mose.

\chapter{28}

\par 1 Och HERREN talade till Mose och sade:
\par 2 Bjud Israels barn och säg till dem: Mina offer, det som är min spis av mina eldsoffer, en välbehaglig lukt för mig, dem skolen I akta på, så att I offren dem åt mig på bestämd tid.
\par 3 Och säg till dem: Detta är vad I skolen offra åt HERREN såsom eldsoffer: två årsgamla felfria lamm till brännoffer för var dag beständigt.
\par 4 Det ena lammet skall du offra om morgonen, och det andra lammet skall du offra vid aftontiden,
\par 5 och såsom spisoffer en tiondedels efa fint mjöl, begjutet med en fjärdedels hin olja av stötta oliver.
\par 6 Detta är det dagliga brännoffret, som offrades på Sinai berg, till en välbehaglig lukt, ett eldsoffer åt HERREN.
\par 7 Och såsom drickoffer därtill skall du offra en fjärdedels hin, till det första lammet; i helgedomen skall drickoffer av stark dryck utgjutas åt HERREN.
\par 8 Det andra lammet skall du offra vid aftontiden; med likadant spisoffer och drickoffer som om morgonen skall du offra det: ett eldsoffer till en välbehaglig lukt för HERREN.
\par 9 Men på sabbatsdagen skall du offra två årsgamla felfria lamm, så ock två tiondedels efa fint mjöl, begjutet med olja, såsom spisoffer, samt tillhörande drickoffer.
\par 10 Detta är sabbatsbrännoffret, som skall offras var sabbat, jämte det dagliga brännoffret med tillhörande drickoffer.
\par 11 Och på edra nymånadsdagar skolen I offra till brännoffer åt HERREN två ungtjurar och en vädur och sju årsgamla felfria lamm,
\par 12 så ock tre tiondedels efa fint mjöl, begjutet med olja, såsom spisoffer till var tjur, två tiondedels efa fint mjöl, begjutet med olja, såsom spisoffer till väduren,
\par 13 och en tiondedels efa fint mjöl begjutet med olja, såsom spisoffer till vart lamm: ett brännoffer till en välbehaglig lukt, ett eldsoffer åt HERREN.
\par 14 Och de tillhörande drickoffren skola utgöras av en halv hin vin till var tjur och en tredjedels hin till väduren och en fjärdedels hin till vart lamm. Detta är nymånadsbrännoffret, som skall offras i var och en av årets månader.
\par 15 Tillika skolen I offra en bock till syndoffer åt HERREN; den skall offras jämte det dagliga brännoffret med tillhörande drickoffer.
\par 16 Och i första månaden, på fjortonde dagen i månaden, är HERRENS påsk.
\par 17 Och på femtonde dagen i samma månad är högtid; då skall man äta osyrat bröd, i sju dagar.
\par 18 På den första dagen skall man hålla en helig sammankomst; ingen arbetssyssla skolen I då göra.
\par 19 Och såsom eldsoffer, såsom brännoffer åt HERREN, skolen I offra två ungtjurar och en vädur och sju årsgamla lamm; felfria skola de vara.
\par 20 Och såsom spisoffer därtill skolen I offra fint mjöl, begjutet med olja; tre tiondedels efa skolen I offra till var ungtjur och två tiondedels efa till väduren;
\par 21 en tiondedels efa skall du offra till vart och ett av de sju lammen;
\par 22 tillika skolen I offra en syndoffersbock till att bringa försoning för eder.
\par 23 Förutom morgonens brännoffer, som utgör det dagliga brännoffret, skolen I offra detta.
\par 24 Likadana offer skolen I offra var dag i sju dagar: en eldsoffersspis, till en välbehaglig lukt för HERREN. Jämte det dagliga brännoffret skall detta offras, med tillhörande drickoffer.
\par 25 Och på den sjunde dagen skolen I hålla en helig sammankomst; ingen arbetssyssla skolen I då göra.
\par 26 Och på förstlingsdagen, då I bären fram ett offer av den nya grödan åt HERREN, vid eder veckohögtid, skolen I hålla en helig sammankomst; ingen arbetssyssla skolen I då göra.
\par 27 Såsom brännoffer till en välbehaglig lukt för HERREN skolen I då offra två ungtjurar, en vädur, sju årsgamla lamm,
\par 28 och såsom spisoffer därtill fint mjöl, begjutet med olja: tre tiondedels efa till var tjur, två tiondedels efa till väduren,
\par 29 en tiondedels efa till vart och ett av de sju lammen;
\par 30 tillika skolen I offra en bock till att bringa försoning för eder.
\par 31 Förutom det dagliga brännoffret med tillhörande spisoffer skolen I offra detta - felfria skola djuren vara - och därjämte tillhörande drickoffer.

\chapter{29}

\par 1 Och i sjunde månaden, på första dagen i månaden, skolen I hålla en helig sammankomst; ingen arbetssyssla skolen I då göra. En basunklangens dag skall den vara för eder.
\par 2 Såsom brännoffer till en välbehaglig lukt för HERREN skolen I då offra en ungtjur, en vädur, sju årsgamla felfria lamm,
\par 3 och såsom spisoffer därtill fint mjöl, begjutet med olja: tre tiondedels efa till tjuren, två tiondedels efa till väduren
\par 4 och en tiondedels efa till vart och ett av de sju lammen;
\par 5 tillika skolen I offra en bock såsom syndoffer, till att bringa försoning för eder -
\par 6 detta förutom nymånadsbrännoffret med tillhörande spisoffer, och förutom det dagliga brännoffret med tillhörande spisoffer, och förutom de drickoffer som på föreskrivet sätt skola offras till båda: allt till en välbehaglig lukt, ett eldsoffer åt HERREN.
\par 7 På tionde dagen i samma sjunde månad skolen I ock hålla en helig sammankomst, och I skolen då späka eder; intet arbete skolen I då göra.
\par 8 Och såsom brännoffer till en välbehaglig lukt för HERREN skolen I då offra en ungtjur, en vädur, sju årsgamla lamm - felfria skola de vara -
\par 9 och såsom spisoffer därtill fint mjöl, begjutet med olja: tre tiondedels efa till tjuren, två tiondedels efa till väduren,
\par 10 en tiondedels efa till vart och ett av de sju lammen;
\par 11 tillika skolen I offra en bock såsom syndoffer - detta förutom försoningssyndoffret och det dagliga brännoffret med tillhörande spisoffer, och förutom de drickoffer som höra till båda.
\par 12 På femtonde dagen i sjunde månaden skolen I ock hålla en helig sammankomst; ingen arbetssyssla skolen I då göra. Då skolen I fira en HERRENS högtid, i sju dagar.
\par 13 Och såsom brännoffer, såsom eldsoffer, skolen I då offra till en välbehaglig lukt för HERREN tretton ungtjurar, två vädurar, fjorton årsgamla lamm - felfria skola de vara -
\par 14 och såsom spisoffer därtill fint mjöl, begjutet med olja: tre tiondedels efa till var och en av de tretton tjurarna, två tiondedels efa till var och en av de två vädurarna,
\par 15 en tiondedels efa till vart och ett av de fjorton lammen;
\par 16 tillika skolen I offra en bock såsom syndoffer - detta förutom det dagliga brännoffret med tillhörande spisoffer och drickoffer.
\par 17 Och på den andra dagen: tolv ungtjurar, två vädurar, fjorton årsgamla felfria lamm,
\par 18 med det spisoffer och de drickoffer som skola offras till dem, till tjurarna, vädurarna och lammen, efter deras antal, på föreskrivet sätt,
\par 19 tillika också en bock såsom syndoffer - detta förutom det dagliga brännoffret med tillhörande spisoffer och det drickoffer som hör till dem.
\par 20 Och på den tredje dagen: elva tjurar, två vädurar, fjorton årsgamla felfria lamm,
\par 21 med det spisoffer och de drickoffer som skola offras till dem, till tjurarna, vädurarna och lammen, efter deras antal, på föreskrivet sätt,
\par 22 tillika också en syndoffersbock - detta förutom det dagliga brännoffret med tillhörande spisoffer och drickoffer.
\par 23 Och på den fjärde dagen; tio tjurar, två vädurar, fjorton årsgamla felfria lamm,
\par 24 med det spisoffer och de drickoffer som skola offras till dem, till tjurarna, vädurarna och lammen, efter deras antal, på föreskrivet satt,
\par 25 tillika också en bock såsom syndoffer - detta förutom det dagliga brännoffret med tillhörande spisoffer och drickoffer.
\par 26 Och på den femte dagen: nio tjurar, två vädurar, fjorton årsgamla felfria lamm,
\par 27 med det spisoffer och de drickoffer som skola offras till dem, till tjurarna, vädurarna och lammen, efter deras antal, på föreskrivet sätt,
\par 28 tillika också en syndoffersbock - detta förutom det dagliga brännoffret med tillhörande spisoffer och drickoffer.
\par 29 Och på den sjätte dagen: åtta tjurar, två vädurar, fjorton årsgamla felfria lamm,
\par 30 med det spisoffer och de drickoffer som skola offras till dem, till tjurarna, vädurarna och lammen, efter deras antal, på föreskrivet sätt,
\par 31 tillika också en syndoffersbock - detta förutom det dagliga brännoffret med tillhörande spisoffer och drickoffer.
\par 32 Och på den sjunde dagen: sju tjurar, två vädurar, fjorton årsgamla felfria lamm,
\par 33 med det spisoffer och de drickoffer som skola offras till dem, till tjurarna, vädurarna och lammen, efter deras antal, på föreskrivet sätt,
\par 34 tillika också en syndoffersbock - detta förutom det dagliga brännoffret med tillhörande spisoffer och drickoffer.
\par 35 På den åttonde dagen skolen I hålla en högtidsförsamling; ingen arbetssyssla skolen I då göra.
\par 36 Och såsom brännoffer, såsom eldsoffer, skolen I då offra till en välbehaglig lukt för HERREN en tjur, en vädur, sju årsgamla felfria lamm,
\par 37 med det spisoffer och de drickoffer som skola offras till dem, till tjuren, väduren och lammen, efter deras antal, på föreskrivet sätt,
\par 38 tillika också en syndoffersbock - detta förutom det dagliga brännoffret med tillhörande spisoffer och drickoffer.
\par 39 Dessa offer skolen I offra åt HERREN vid edra högtider, förutom edra löftesoffer och frivilliga offer, dessa må nu vara brännoffer eller spisoffer eller drickoffer eller tackoffer.

\chapter{30}

\par 1 Och Mose sade detta till Israels barn, alldeles såsom HERREN hade bjudit honom.
\par 2 Och Mose talade till Israels barns stamhövdingar och sade: Detta är vad HERREN har bjudit:
\par 3 om någon gör ett löfte åt HERREN, eller svär en ed genom vilken han förbinder sig till återhållsamhet i något stycke, så skall han icke sedan bryta sitt ord; han skall i alla stycken göra vad hans mun har talat.
\par 4 Och om en kvinna, medan hon vistas i sin faders hus och ännu är ung, gör ett löfte åt HERREN och förbinder sig till återhållsamhet i något stycke,
\par 5 och hennes fader hör hennes löfte och huru hon förbinder sig till återhållsamhet, och hennes fader icke säger något till henne därom, så skola alla hennes löften hava gällande kraft, och alla hennes förbindelser till återhållsamhet skola hava gällande kraft.
\par 6 Men om hennes fader samma dag han hör det säger nej därtill, då skola hennes löften och hennes förbindelser till återhållsamhet alla vara utan gällande kraft; och HERREN skall förlåta henne, eftersom hennes fader sade nej till henne.
\par 7 Och om hon bliver gift, och löften då vila på henne, eller något obetänksamt ord från hennes läppar, varmed hon har bundit sig,
\par 8 och hennes man får höra därom, men icke säger något till henne därom samma dag han hör det, så skola hennes löften hava gällande kraft, och hennes förbindelser till återhållsamhet skola hava gällande kraft.
\par 9 Men om hennes man samma dag han får höra det säger nej därtill, då upphäver han därmed hennes givna löfte och det obetänksamma ord från hennes läppar, varmed hon har bundit sig; och HERREN skall förlåta henne det.
\par 10 Men en änkas eller en förskjuten hustrus löfte skall hava gällande kraft för henne, vartill hon än må hava förbundit sig.
\par 11 Och om en kvinna i sin mans hus gör ett löfte, eller med ed förbinder sig till återhållsamhet i något stycke,
\par 12 och hennes man hör det, men icke säger något till henne därom - icke säger nej till henne - så skola alla hennes löften hava gällande kraft, och alla hennes förbindelser till återhållsamhet skola hava gällande kraft.
\par 13 Men om hennes man upphäver dem samma dag han hör dem, då skall allt som hennes läppar hava talat vara utan gällande kraft, det må nu vara löften eller någon förbindelse till återhållsamhet; hennes man har upphävt dem, därför skall HERREN förlåta henne.
\par 14 Åt alla hennes löften och åt alla hennes edliga förbindelser till att späka sig kan hennes man giva gällande kraft, och hennes man kan ock upphäva dem.
\par 15 Men om hennes man icke före påföljande dags ingång säger någonting till henne därom, så giver han gällande kraft åt alla hennes löften och åt alla de förbindelser till återhållsamhet, som vila på henne; han giver dem gällande kraft därigenom att han icke säger något till henne därom samma dag han hör dem.
\par 16 Men om han upphäver dem först någon tid efter det han har hört dem, då kommer han att bära på hennes missgärning.
\par 16 Men om han upphäver dem först någon tid efter det han har hört dem, då kommer han att bära på hennes missgärning.

\chapter{31}

\par 1 Och HERREN talade till Mose och sade:
\par 2 "Kräv ut hämnd för Israels barn midjaniterna; sedan skall du samlas till dina fäder.
\par 3 Då talade Mose till folket och sade: "Låten en del av edra män väpna sig till strid; dessa skola tåga mot Midjan och utföra HERRENS hämnd på Midjan.
\par 4 Tusen man ur var och en särskild av Israels alla stammar skolen I sända ut i striden."
\par 5 Så avlämnades då ur Israels ätter tusen man av var stam: tolv tusen man, väpnade till strid.
\par 6 Och Mose sände dessa, tusen man av var stam, ut i striden; han sände med dem Pinehas, prästen Eleasars son, ut i striden, och denne tog med sig de heliga redskapen och larmtrumpeterna.
\par 7 Och de gingo till strids emot Midjan, såsom HERREN hade bjudit Mose, och dräpte allt mankön.
\par 8 Och jämte andra som då blevo slagna av dem dräptes ock de midjanitiska konungarna Evi, Rekem, Sur, Hur och Reba, fem midjanitiska konungar; Bileam, Beors son, dräpte de ock med svärd.
\par 9 Och Israels barn förde Midjans kvinnor och barn bort såsom fångar; och alla deras dragare och all deras boskap och allt deras övriga gods togo de såsom byte.
\par 10 Och alla deras städer, i de trakter där de bodde, och alla deras tältläger brände de upp i eld.
\par 11 Och de togo med sig allt bytet, och allt vad de hade rövat, både människor och boskap.
\par 12 Och de förde fångarna och det rövade och bytet fram till Mose och prästen Eleasar och Israels barns menighet i lägret på Moabs hedar, som ligga vid Jordan mitt emot Jeriko.
\par 13 Och Mose och prästen Eleasar och alla menighetens hövdingar gingo dem till mötes utanför lägret.
\par 14 Men Mose förtörnades på krigsbefälet, över- och underhövitsmännen, när de kommo tillbaka från sitt krigståg.
\par 15 Mose sade till dem: "Haven I då låtit alla kvinnorna leva?
\par 16 Det var ju de som, på Bileams inrådan, förledde Israels barn till att begå otrohet mot HERREN i saken med Peor, och som därigenom vållade att en hemsökelse kom över HERRENS menighet.
\par 17 Så dräpen nu alla gossebarn, och dräpen alla kvinnor som hava haft med män, med mankön, att skaffa.
\par 18 Men alla flickebarn som icke hava haft med mankön att skaffa, dem mån I låta leva för eder räkning.
\par 19 Själva skolen I nu lägra eder utanför lägret, i sju dagar. Var och en av eder som har dräpt någon människa, och var och en som har kommit vid någon slagen skall rena sig på tredje dagen och på sjunde dagen - såväl I själva som edra fångar.
\par 20 Alla kläder och allt som är förfärdigat av skinn och allt som är gjort av gethår och alla redskap av trä skolen I ock rena åt eder."
\par 21 Och prästen Eleasar sade till stridsmännen som hade deltagit i kriget: Detta är den lagstadga som HERREN har givit Mose:
\par 22 Guld och silver, koppar, järn, tenn och bly,
\par 23 allt sådant som tål eld, skolen I låta gå genom eld, så bliver det rent; dock bör det tillika renas med stänkelsevatten. Men allt som icke tål eld skolen I låta gå genom vatten.
\par 24 Och I skolen två edra kläder på sjunde dagen, så bliven I rena; därefter fån I gå in i lägret.
\par 25 Och HERREN talade till Mose och sade:
\par 26 Över det tagna rovet, både människor och boskap, skall du göra en beräkning, du tillsammans med prästen Eleasar och huvudmännen för menighetens familjer;
\par 27 sedan skall du dela rovet i två delar, mellan de krigare som hava varit med i striden och hela den övriga menigheten.
\par 28 Och du skall låta det krigsfolk som har varit med i striden giva var femhundrade av människor, fäkreatur, åsnor och får såsom skatt åt HERREN.
\par 29 Så mycket skall tagas av den hälft som tillfaller dem, och du skall giva detta åt prästen Eleasar såsom en gärd åt HERREN.
\par 30 Men ur den hälft som tillfaller de övriga israeliterna skall du uttaga var femtionde av människor, sammalunda av fäkreatur, åsnor och får, korteligen, av all boskap, och detta skall du giva åt leviterna, som det åligger att iakttaga vad som är att iakttaga vid HERRENS tabernakel.
\par 31 Och Mose och prästen Eleasar gjorde såsom HERREN hade bjudit Mose.
\par 32 Och rovet, nämligen återstoden av det byte som krigsfolket hade tagit utgjorde: av får sex hundra sjuttiofem tusen,
\par 33 av fäkreatur sjuttiotvå tusen,
\par 34 av åsnor sextioett tusen,
\par 35 och av människor, sådana kvinnor som icke hade haft med mankön att skaffa, tillsammans trettiotvå tusen personer.
\par 36 Och hälften därav, eller den del om tillföll dem som hade varit med striden, utgjorde: av får ett antal av tre hundra trettiosju tusen fem hundra,
\par 37 varav skatten åt HERREN utgjorde sex hundra sjuttiofem får;
\par 38 av fäkreatur trettiosex tusen, varav skatten åt HERREN sjuttiotvå;
\par 39 av åsnor trettio tusen fem hundra, varav skatten åt HERREN sextioen;
\par 40 av människor sexton tusen, varav skatten åt HERREN trettiotvå personer.
\par 41 Och skatten, den för HERREN bestämda gärden, gav Mose åt prästen Eleasar, såsom HERREN hade bjudit Mose.
\par 42 Och den hälft, som tillföll de övriga israeliterna, och som Mose hade avskilt från krigsfolkets,
\par 43 denna hälft, den som tillföll menigheten, utgjorde: av får tre hundra trettiosju tusen fem hundra,
\par 44 av fäkreatur trettiosex tusen,
\par 45 av åsnor trettio tusen fem hundra
\par 46 och av människor sexton tusen.
\par 47 Och ur denna hälft, som tillföll de övriga israeliterna, uttog Mose var femtionde, både av människor och av boskap, och gav detta åt leviterna, som det ålåg att iakttaga vad som var att iakttaga vid HERRENS tabernakel, allt såsom HERREN hade bjudit Mose.
\par 48 Och befälhavarna över härens avdelningar, över- och underhövitsmännen, trädde fram till Mose.
\par 49 Och de sade till Mose: "Dina tjänare hava räknat antalet av de krigsmän som vi hava haft under vårt befäl, och icke en enda fattas bland oss.
\par 50 Därför hava vi nu såsom en offergåva åt HERREN burit fram var och en del som han har kommit över av gyllene klenoder, armband av olika slag, ringar, örhängen och halssmycken, detta för att bringa försoning för oss inför HERRENS ansikte."
\par 51 Och Mose och prästen Eleasar togo emot guldet av dem, alla slags klenoder.
\par 52 Och guldet som gavs såsom gärd åt HERREN av över- och underhövitsmännen utgjorde sammanlagt sexton tusen sju hundra femtio siklar.
\par 53 Manskapet hade tagit byte var och en för sig.
\par 54 Och Mose och prästen Eleasar togo emot guldet av över- och underhövitsmännen och buro in det i uppenbarelsetältet, för att det skulle bringa Israels barn i åminnelse inför HERRENS ansikte.

\chapter{32}

\par 1 Och Rubens barn och Gads barn hade stora och mycket talrika boskapshjordar; och när de sågo Jaesers land och Gileads land, funno de att detta var en trakt för boskap.
\par 2 Då kommo Gads barn och Rubens barn och sade till Mose och prästen Eleasar och menighetens hövdingar:
\par 3 "Atarot, Dibon, Jaeser, Nimra, Hesbon, Eleale, Sebam, Nebo och Beon,
\par 4 det land som HERREN har låtit Israels menighet intaga, är ett land för boskap, och dina tjänare hava boskap."
\par 5 Och de sade ytterligare: "Om vi hava funnit nåd inför dina ögon så må detta land givas åt dina tjänare till besittning. Låt oss slippa att gå över Jordan."
\par 6 Men Mose sade till Gads barn och Rubens barn: "Skolen då I stanna här, under det att edra bröder draga ut i krig?
\par 7 Varför viljen I avvända Israels barns hjärtan från att gå över floden, in i det land som HERREN har givit åt dem?
\par 8 Så gjorde ock edra fäder, när jag sände dem från Kades-Barnea för att bese landet:
\par 9 sedan de hade dragit upp till Druvdalen och besett landet, avvände de Israels barns hjärtan från att gå in i det land som HERREN hade givit åt dem.
\par 10 Och på den dagen upptändes HERRENS vrede, och han svor och sade:
\par 11 'Av de män som hava dragit upp ur Egypten skall ingen som är tjugu år gammal eller därutöver få se det land som jag med ed har lovat åt Abraham, Isak och Jakob - eftersom de icke i allt hava efterföljt mig
\par 12 ingen förutom Kaleb, Jefunnes son, kenaséen, och Josua, Nuns son; ty de hava i allt efterföljt HERREN.'
\par 13 Så upptändes HERRENS vrede mot Israel, och han lät dem driva omkring i öknen i fyrtio år, till dess att hela det släkte hade dött bort som hade gjort vad ont var i HERRENS ögon.
\par 14 Och se, nu haven I trätt i edra fäders fotspår, I, syndiga mäns avföda, och öken så ännu mer HERRENS vredes glöd mot Israel.
\par 15 Då I nu vänden eder bort ifrån honom, skall han låta Israel ännu längre bliva kvar i öknen, och I dragen så fördärv över allt detta folk."
\par 16 Då trädde de fram till honom och sade: "Låt oss här bygga gårdar åt var boskap och städer åt våra kvinnor och barn.
\par 17 Själva vilja vi sedan skyndsamt väpna oss och gå åstad i spetsen för Israels barn, till dess vi hava fört dem dit de skola. Under tiden kunna våra kvinnor och barn bo i de befästa städerna och så vara skyddade mot landets inbyggare.
\par 18 Vi skola icke vända tillbaka hem, förrän Israels barn hava fått var och en sin arvedel.
\par 19 Ty vi vilja icke taga vår arvedel jämte dem, på andra sidan Jordan och längre bort, utan vår arvedel har tillfallit oss här på andra sidan Jordan, på östra sidan."
\par 20 Mose svarade dem: "Om I gören såsom I nu haven sagt, om I väpnen eder inför HERREN till kriget,
\par 21 så att alla edra väpnade män gå över Jordan inför HERREN och stanna där, till dess han har fördrivit sina fiender för sig,
\par 22 om I alltså vänden tillbaka först då landet har blivit HERREN underdånigt, så skolen I vara utan skuld mot HERREN och Israel, och detta land skall då bliva eder besittning inför HERREN.
\par 23 Men om I icke så gören, se, då synden I mot HERREN, och I skolen då komma att förnimma eder synd, ty den skall drabba eder.
\par 24 Byggen eder nu städer åt edra kvinnor och barn, och gårdar åt eder boskap, och gören vad eder mun har talat."
\par 25 Och Gads barn och Rubens barn talade till Mose och sade: "Dina tjänare skola göra såsom min herre bjuder.
\par 26 Våra barn våra hustrur, vår boskap och alla våra dragare skola bliva kvar här i Gileads städer.
\par 27 Men dina tjänare, vi så många som äro väpnade till strid, skola draga ditöver och kämpa inför HERREN, såsom min herre har sagt."
\par 28 Och Mose gav befallning om dem åt prästen Eleasar och åt Josua, Nuns son, och åt huvudmännen för familjerna inom Israels barns stammar.
\par 29 Mose sade till dem: "Om Gads barn och Rubens barn gå över Jordan med eder, så många som äro väpnade till att kämpa inför HERREN, och landet så bliver eder underdånigt, då skolen I åt dem giva landet Gilead till besittning.
\par 30 Men om de icke draga väpnade ditöver med eder, så skola de få sin besittning ibland eder i Kanaans land."
\par 31 Och Gads barn och Rubens barn svarade och sade: "Vad HERREN har sagt till dina tjänare, det vilja vi göra.
\par 32 Vi vilja draga väpnade över till Kanaans land inför HERREN, och så få vår arvsbesittning har på andra sidan Jordan."
\par 33 Så gav då Mose åt dem, åt Gads barn, Rubens barn och ena hälften av Manasses, Josefs sons, stam, Sihons, amoréernas konungs, rike och Ogs rike, konungens i Basan: själva landet med dess städer och dessas områden, landets städer runt omkring.
\par 34 Och Gads barn byggde upp Dibon, Atarot, Aroer,
\par 35 Atrot-Sofan, Jaeser, Jogbeha,
\par 36 Bet-Nimra och Bet-Haran, befästa städer och boskapsgårdar.
\par 37 Och Rubens barn byggde upp Hesbon, Eleale, Kirjataim,
\par 38 Nebo och Baal-Meon - vilkas namn hava ändrats - och Sibma. Och de gåvo namn åt städerna som de byggde upp.
\par 39 Och Makirs, Manasses sons, barn gingo åstad till Gilead och intogo det och fördrevo amoréerna som bodde där.
\par 40 Och Mose gav Gilead åt Makir, Manasses son, och han bosatte sig där.
\par 41 Och Jair, Manasses son, gick åstad och intog deras byar och kallade dem Jairs byar.
\par 42 Och Noba gick åstad och intog Kenat, med underlydande orter, och kallade det Noba, efter sitt eget namn.

\chapter{33}

\par 1 Dessa voro Israels barns lägerplatser, när de drogo ut ur Egyptens land, efter sina härskaror, anförda av Mose och Aron.
\par 2 Och Mose upptecknade på HERRENS befallning deras uppbrottsorter, alltefter som de ändrade lägerplats. Och dessa voro nu deras lägerplatser, alltefter som uppbrottsorterna följde på varandra:
\par 3 De bröto upp från Rameses i första månaden, på femtonde dagen i första månaden. Dagen efter påskhögtiden drogo Israels barn ut med upplyft hand inför alla egyptiers ögon,
\par 4 under det att egyptierna begrovo dem som HERREN hade slagit bland dem, alla de förstfödda, då när HERREN höll dom över deras gudar.
\par 5 Så bröto nu Israels barn upp från Rameses och lägrade sig i Suckot.
\par 6 Och de bröto upp från Suckot och lägrade sig i Etam, där öknen begynte.
\par 7 Och de bröto upp från Etam och vände om till Pi-Hahirot, som ligger mitt emot Baal-Sefon, och lägrade sig framför Migdol.
\par 8 Och de bröto upp från Hahirot och gingo mitt igenom havet in i öknen och tågade så tre dagsresor i Etams öken och lägrade sig i Mara.
\par 9 Och de bröto upp från Mara och kommo till Elim; och i Elim funnos tolv vattenkällor och sjuttio palmträd, och de lägrade sig där.
\par 10 Och de bröto upp från Elim och lägrade sig vid Röda havet.
\par 11 Och de bröto upp från Röda havet och lägrade sig i öknen Sin.
\par 12 Och de bröto upp från öknen Sin och lägrade sig i Dofka.
\par 13 Och de bröto upp från Dofka och lägrade sig i Alus.
\par 14 Och de bröto upp från Alus och lägrade sig i Refidim, och där fanns intet vatten åt folket att dricka.
\par 15 Och de bröto upp från Refidim och lägrade sig i Sinais öken.
\par 16 Och de bröto upp från Sinais öken och lägrade sig i Kibrot-Hattaava.
\par 17 Och de bröto upp från Kibrot-Hattaava och lägrade sig i Haserot.
\par 18 Och de bröto upp från Haserot och lägrade sig i Ritma.
\par 19 Och de bröto upp från Ritma och lägrade sig i Rimmon-Peres.
\par 20 Och de bröto upp från Rimmon-Peres och lägrade sig i Libna.
\par 21 Och de bröto upp från Libna och lägrade sig i Rissa.
\par 22 Och de bröto upp från Rissa och lägrade sig i Kehelata.
\par 23 Och de bröto upp från Kehelata och lägrade sig vid berget Sefer.
\par 24 Och de bröto upp från berget Sefer och lägrade sig i Harada.
\par 25 Och de bröto upp från Harada och lägrade sig i Makhelot.
\par 26 Och de bröto upp från Makhelot och lägrade sig i Tahat.
\par 27 Och de bröto upp från Tahat och lägrade sig i Tera.
\par 28 Och de bröto upp från Tera och lägrade sig i Mitka.
\par 29 Och de bröto upp från Mitka och lägrade sig i Hasmona.
\par 30 Och de bröto upp från Hasmona och lägrade sig i Moserot.
\par 31 Och de bröto upp från Moserot och lägrade sig i Bene-Jaakan.
\par 32 Och de bröto upp från Bene-Jaakan och lägrade sig i Hor-Haggidgad.
\par 33 Och de bröto upp från Hor-Haggidgad och lägrade sig i Jotbata.
\par 34 Och de bröto upp från Jotbata och lägrade sig i Abrona.
\par 35 och de bröto upp från Abrona och lägrade sig i Esjon-Geber.
\par 36 Och de bröto upp från Esjon-Geber och lägrade sig i öknen Sin, det är Kades.
\par 37 Och de bröto upp från Kades och lägrade sig vid berget Hor, på gränsen till Edoms land.
\par 38 Och prästen Aron steg upp på berget Hor, efter HERRENS befallning, och dog där i det fyrtionde året efter Israels barns uttåg ur Egyptens land, i femte månaden, på första dagen i månaden.
\par 39 Och Aron var ett hundra tjugutre år gammal, när han dog på berget Hor.
\par 40 Och konungen i Arad, kananéen, som bodde i Sydlandet i Kanaans land, fick nu höra att Israels barn voro i antågande.
\par 41 Och de bröto upp från berget Hor och lägrade sig i Salmona.
\par 42 Och de bröto upp från Salmona och lägrade sig i Punon.
\par 43 Och de bröto upp från Punon och lägrade sig i Obot.
\par 44 Och de bröto upp från Obot och lägrade sig i Ije-Haabarim vid Moabs gräns.
\par 45 Och de bröto upp från Ijim och lägrade sig i Dibon-Gad.
\par 46 Och de bröto upp från Dibon-Gad och lägrade sig i Almon-Diblataima.
\par 47 Och de bröto upp från Almon-Diblataima och lägrade sig vid Abarimbergen, framför Nebo.
\par 48 Och de bröto upp från Abarimbergen och lägrade sig på Moabs hedar, vid Jordan mitt emot Jeriko.
\par 49 Och deras läger vid Jordan sträckte sig från Bet-Hajesimot ända till Abel-Hassitim på Moabs hedar.
\par 50 Och HERREN talade till Mose på Moabs hedar, vid Jordan mitt emot Jeriko, och sade:
\par 51 Tala till Israels barn och säg till dem: När I haven gått över Jordan, in i Kanaans land,
\par 52 skolen I fördriva landets alla inbyggare för eder, och I skolen förstöra alla deras stenar med inhuggna bilder, och alla deras gjutna beläten skolen I förstöra, och alla deras offerhöjder skolen I ödelägga.
\par 53 Och I skolen intaga landet och bosätta eder där, ty åt eder har jag givit landet till besittning.
\par 54 Och I skolen utskifta landet såsom arvedel åt eder genom lottkastning efter edra släkter; åt en större stam skolen I giva en större arvedel, och åt en mindre stam en mindre arvedel; var och en skall få sin del där lotten bestämmer att han skall hava den; efter edra fädernestammar skolen I utskifta landet såsom arvedel åt eder.
\par 55 Men om I icke fördriven landets inbyggare för eder, så skola de som I låten vara kvar av dem bliva törnen i edra ögon och taggar i edra sidor, och skola tränga eder i landet där I bon.
\par 56 Och då skall jag göra med eder så, som jag hade tänkt göra med dem.

\chapter{34}

\par 1 Och HERREN talade till Mose och sade:
\par 2 Bjud Israels barn och säg till dem: När I kommen till Kanaans land, då är detta det land som skall tillfalla eder såsom arvedel: Kanaans land, så långt dess gränser nå.
\par 3 Edert land skall på södra sidan sträcka sig från öknen Sin utmed Edom; och eder södra gräns skall i öster begynna vid ändan av Salthavet.
\par 4 Sedan skall eder gräns böja sig söder om Skorpionhöjden och gå fram till Sin och gå ut söder om Kades-Barnea. Och den skall gå vidare ut till Hasar-Addar och fram till Asmon.
\par 5 Och från Asmon skall gränsen böja sig mot Egyptens bäck och gå ut vid havet.
\par 6 Och eder gräns i väster skall vara Stora havet; det skall utgöra gränsen. Detta skall vara eder gräns i väster.
\par 7 Och detta skall vara eder gräns i norr: Från Stora havet skolen I draga eder gränslinje fram vid berget Hor.
\par 8 Från berget Hor skolen I draga eder gränslinje dit där vägen går till Hamat, och gränsen skall gå ut vid Sedad.
\par 9 Sedan skall gränsen gå till Sifron och därifrån ut vid Hasar-Enan. Detta skall vara eder gräns i norr.
\par 10 och såsom eder gräns i öster skolen I draga upp en linje från Hasar-Enan fram till Sefam.
\par 11 Och från Sefam skall gränsen gå ned till Haribla, öster om Ain, och gränsen skall gå vidare ned och intill bergsluttningen vid Kinneretsjön, österut.
\par 12 Sedan skall gränsen gå ned till Jordan och ut vid Salthavet. Detta skall vara edert land, med dess gränser runt omkring.
\par 13 Och Mose bjöd Israels barn och sade: Detta är det land som I genom lottkastning skolen utskifta såsom arvedel åt eder, det land om vilket HERREN har bjudit att det skall givas åt de nio stammarna och den ena halva stammen.
\par 14 Ty rubeniternas barns stam, efter dess familjer, och gaditernas barns stam, efter dess familjer, och den andra hälften av Manasse stam, dessa hava redan fått sin arvedel.
\par 15 Dessa två stammar och denna halva stam hava fått sin arvedel på andra sidan Jordan mitt emot Jeriko, österut mot solens uppgång.
\par 16 Och HERREN talade till Mose och sade:
\par 17 Dessa äro namnen på de män som skola åt eder utskifta landet i arvslotter: först och främst prästen Eleasar och Josua, Nuns son;
\par 18 vidare skolen I taga en hövding ur var stam till att utskifta landet,
\par 19 och dessa äro de männens namn: av Juda stam Kaleb, Jefunnes son;
\par 20 av Simeons barns stam Samuel Ammihuds son;
\par 21 av Benjamins stam Elidad, Kislons son;
\par 22 av Dans barns stam en hövding, Bucki, Joglis son;
\par 23 av Josefs barn: av Manasse barns stam en hövding, Hanniel, Efods son,
\par 24 och av Efraims barn stam en hövding, Kemuel, Siftans son;
\par 25 av Sebulons barns stam en hövding, Elisafan, Parnaks son;
\par 26 av Isaskars barns stam en hövding, Paltiel, Assans son;
\par 27 av Asers barns stam en hövding, Ahihud, Selomis son;
\par 28 av Naftali barns stam en hövding, Pedael, Ammihuds son.
\par 29 Dessa äro de som HERREN bjöd att utskifta arvslotterna åt Israels barn i Kanaans land.

\chapter{35}

\par 1 Och HERREN talade till Mose på Moabs hedar, vid Jordan mitt emot Jeriko, och sade:
\par 2 Bjud Israels barn att de av de arvslotter de få till besittning skola åt leviterna giva städer att bo i; utmarker runt omkring dessa städer skolen I ock giva åt leviterna.
\par 3 Städerna skola de själva hava att bo i, men de tillhörande utmarkerna skola vara för deras dragare och deras boskap och alla deras övriga djur.
\par 4 Och städernas utmarker, som I skolen giva åt leviterna, skola sträcka sig tusen alnar från stadsmuren utåt på alla sidor.
\par 5 Och utanför staden skolen I mäta upp på östra sidan två tusen alnar, på västra sidan två tusen alnar, på södra sidan två tusen alnar och på norra sidan två tusen alnar, med staden i mitten. Detta skola de få såsom utmarker till sina städer.
\par 6 Och de städer som I given åt leviterna skola först och främst vara de sex fristäderna, vilka I skolen giva till det ändamålet att en dråpare må kunna fly till dem; vidare skolen I jämte dessa städer giva dem fyrtiotvå andra,
\par 7 så att de städer som I given åt leviterna tillsammans utgöra fyrtioåtta städer, med tillhörande utmarker.
\par 8 Och av dessa städer, som I skolen giva av Israels barns besittningsområde, skolen I taga flera ur den stam som är större, och färre ur den som är mindre. Var stam skall åt leviterna giva ett antal av sina städer, som svarar mot den arvedel han själv har fått.
\par 9 Och HERREN talade till Mose och sade:
\par 10 Tala till Israels barn och säg till dem: När I haven gått över Jordan, in i Kanaans land,
\par 11 skolen I utse åt eder vissa städer, som I skolen hava till fristäder, till vilka en dråpare som ouppsåtligen har dödat någon må kunna fly.
\par 12 Och dessa städer skolen I hava såsom tillflyktsorter undan blodshämnaren, så att dråparen slipper dö, förrän han har stått till rätta inför menigheten.
\par 13 Och de städer som I skolen giva till fristäder skola vara sex.
\par 14 Tre av städerna skolen I giva på andra sidan Jordan, och de tre övriga städerna skolen I giva i själva Kanaans land;
\par 15 dessa skola vara fristäder. Israels barn, såväl som främlingen och inhysesmannen som bor ibland dem skola hava dessa sex städer såsom tillflyktsorter, till vilka var och er som ouppsåtligen har dödat någon må kunna fly.
\par 16 Men om någon slår en annan till döds med ett föremål av järn, så är han en sannskyldig dråpare; en sådan skall straffas med döden.
\par 17 Likaledes, om någon i sin hand har en sten med vilken ett dråpslag kan givas, och han därmed slår en annan till döds, så är han en sannskyldig dråpare; en sådan skall straffas med döden.
\par 18 Eller om någon i sin hand har ett föremål av trä varmed ett dråpslag kan givas, och han därmed slår en annan till döds, så är han en sannskyldig dråpare; en sådan skall straffas med döden.
\par 19 Blodshämnaren må döda den dråparen; varhelst han träffar på honom må han döda honom.
\par 20 Likaledes om någon av hat stöter till en annan, eller med berått mod kastar något på honom; så att han dör,
\par 21 eller av fiendskap slår honom till döds med handen, då skall den som gav slaget straffas med döden, ty han är en sannskyldig dråpare; blodshämnaren må döda den dråparen, varhelst han träffar på honom
\par 22 Men om någon av våda, utan fiendskap, stöter till en annan, eller utan berått mod kastar på honom något föremål, vad det vara må;
\par 23 eller om han, utan att se honom, med någon sten varmed dråpslag kan givas träffar honom, så att han dör, och detta utan att han var hans fiende eller hade för avsikt att skada honom,
\par 24 då skall menigheten döma mellan den som gav slaget och blodshämnaren, enligt här givna föreskrifter.
\par 25 Och menigheten skall rädda dråparen ur blodshämnarens hand, och menigheten skall låta honom vända tillbaka till fristaden dit han hade flytt, och där skall han stanna kvar, till dess den med helig olja smorde översteprästen dör.
\par 26 Men om dråparen går utom området för den fristad dit han har flytt,
\par 27 och blodshämnaren då, när han träffar på honom utom hans fristads område, dräper dråparen, så vilar ingen blodskuld på honom.
\par 28 Ty i sin fristad skall en dråpare stanna kvar, till dess översteprästen dör; men efter översteprästens död må han vända tillbaka till den ort där han har sin besittning.
\par 29 Och detta skall vara en rättsstadga för eder från släkte till släkte, var I än ären bosatta.
\par 30 Om någon slår ihjäl en annan, skall man, efter vittnens utsago, dräpa dråparen; men en enda persons vittnesmål är icke nog för att man skall kunna döma någon till döden.
\par 31 I skolen icke taga lösen för en dråpares liv, om han är skyldig till döden, utan han skall straffas med döden.
\par 32 Ej heller skolen I taga lösen för att den som har flytt till en fristad skall före prästens död få vända tillbaka och bo i landet.
\par 33 I skolen icke ohelga det land där I ären; genom blod ohelgas landet, och försoning kan icke bringas för landet för det blod som har blivit utgjutet däri, annat än genom dens blod, som har utgjutit det.
\par 34 I skolen icke orena landet där I bon, det i vars mitt jag har min boning, ty jag, HERREN, har min boning mitt ibland Israels barn.

\chapter{36}

\par 1 Och huvudmännen för familjerna i Gileads barns släkt - Gileads, som var son till Makir, Manasses son, av Josefs barns släkter - trädde fram och talade inför Mose och de hövdingar som voro huvudmän för Israels barns familjer.
\par 2 De sade: "HERREN har bjudit min herre att genom lottkastning göra landet såsom arvedel åt Israels barn och HERREN har vidare bjudit min herre att giva Selofhads, vår broders, arvedel åt hans döttrar.
\par 3 Men om nu dessa bliva gifta med någon ur Israels barns andra stammar, så tages deras arvedel bort ifrån våra fäders arvedel, under det att den stam de komma att tillhöra får sin arvedel ökad; på detta sätt bliver en del av vår arvslott oss fråntagen.
\par 4 När sedan jubelåret inträder för Israels barn, bliver deras arvedel lagd till den stams arvedel, som de komma att tillhöra, men från vår fädernestams arvedel tages deras arvedel bort."
\par 5 Då bjöd Mose Israels barn, efter HERRENS befallning, och sade: "Josefs barns stam har talat rätt.
\par 6 Detta är vad HERREN bjuder angående Selofhads döttrar; han säger: De må gifta sig med vem de finna för gott, allenast de gifta sig inom en släkt som hör till deras egen fädernestam.
\par 7 Ty en arvedel som tillhör någon, av Israels barn må icke gå över från en stam till en annan, utan Israels barn skola behålla kvar var och en sin fädernestams arvedel.
\par 8 Och när en kvinna som inom någon av Israels barns stammar har kommit i besittning av en arvedel gifter sig, skall det vara med en man av någon släkt som hör till hennes egen fädernestam, så att Israels barn förbliva i besittning var och en av sina fäders arvedel.
\par 9 Ty ingen arvedel må gå över från en stam till en annan, utan Israels barns stammar skola behålla kvar var och en sin arvedel."
\par 10 Selofhads döttrar gjorde såsom HERREN hade bjudit Mose.
\par 11 Mahela, Tirsa, Hogla, Milka och Noa, Selofhads döttrar, gifte sig med sina farbröders söner.
\par 12 De blevo alltså gifta inom Manasses, Josefs sons, barns släkter, och deras arvedel stannade så kvar inom deras fädernesläkts stam.
\par 13 Dessa äro de bud och rätter som HERREN genom Mose gav Israels barn, på Moabs hedar, vid Jordan mitt emot Jeriko.


\end{document}