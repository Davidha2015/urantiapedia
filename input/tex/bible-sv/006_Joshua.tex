\begin{document}

\title{Josua}


\chapter{1}

\par 1 Efter HERRENS tjänare Moses död sade HERREN till Josua, Nuns son, Moses tjänare:
\par 2 "Min tjänare Mose är död; så stå nu upp och gå över denna Jordan, du med allt detta folk, in i det land som jag vill giva dem, giva åt Israels barn.
\par 3 Var ort som eder fot beträder har jag givit eder, såsom jag lovade Mose.
\par 4 Från öknen till Libanon däruppe och ända till den stora floden, floden Frat, över hetiternas land och ända till Stora havet västerut skall edert område sträcka sig.
\par 5 Ingen skall kunna stå dig emot i alla dina livsdagar; såsom jag har varit med Mose, så skall jag ock vara med dig; jag skall icke lämna dig eller övergiva dig.
\par 6 Var frimodig och oförfärad; ty du skall utskifta åt detta folk såsom arv det land som jag med ed har lovat deras fäder att giva dem.
\par 7 Allenast må du vara helt frimodig och oförfärad till att i alla stycken hålla den lag som min tjänare Mose har givit dig och göra efter den; vik icke av därifrån vare sig till höger eller till vänster; på det att du må hava framgång i allt vad du företager dig.
\par 8 Låt icke denna lagbok vara skild från din mun; tänk på den både dag och natt, så att du i alla stycken håller det som är skrivet i den och gör därefter; ty då skola dina vägar vara lyckosamma, och då skall du hava framgång.
\par 9 Se, jag har bjudit dig att vara frimodig och oförfärad; så var nu icke förskräckt eller försagd. Ty HERREN, din Gud, är med dig i allt vad du företager dig."
\par 10 Då bjöd Josua folkets tillsyningsmän och sade:
\par 11 "Gån igenom lägret och bjuden folket och sägen: 'Reden till reskost åt eder; ty om tre dagar skolen I gå över denna Jordan, för att komma in i och taga i besittning det land som HERREN, eder Gud, vill giva eder till besittning.'"
\par 12 Men till rubeniterna och gaditerna och ena hälften av Manasse stam sade Josua:
\par 13 "Tänken på det som HERRENS tjänare Mose bjöd eder, när han sade: 'HERREN, eder Gud, vill låta eder komma till ro och giva eder detta land.'
\par 14 Edra hustrur, edra barn och eder boskap må nu stanna kvar i det land som Mose har givit eder här på andra sidan Jordan; men I själva, så många av eder som äro tappra stridsmän, skolen draga väpnade åstad i spetsen för edra bröder och hjälpa dem,
\par 15 till dess att HERREN har låtit edra bröder komma till ro såväl som eder, när också de hava tagit i besittning det land som HERREN, eder Gud, vill giva dem. Sedan mån I vända tillbaka till det land som skall vara eder besittning; det mån I då taga i besittning, det land som HERRENS tjänare Mose har givit eder här på andra sidan Jordan, på östra sidan."
\par 16 Då svarade de Josua och sade: "Allt vad du har bjudit oss vilja vi göra, och varthelst du sänder oss, dit vilja vi gå.
\par 17 Såsom vi i allt hava lytt Mose, så vilja vi ock lyda dig; allenast må HERREN, din Gud, vara med dig, såsom han var med Mose.
\par 18 Var och en som är gensträvig mot dina befallningar och icke lyssnar till dina ord, vadhelst du bjuder honom, han skall bliva dödad. Allenast må du vara frimodig och oförfärad."

\chapter{2}

\par 1 Josua, Nuns son, sände hemligen ut två spejare från Sittim och sade: "Gån och besen landet och Jeriko." De gingo åstad och kommo in i ett hus där en sköka bodde, vid namn Rahab, och där lade de sig till vila.
\par 2 Men för konungen i Jeriko blev inberättat: "I natt hava några män kommit hit från Israels barn för att utforska landet."
\par 3 Då sände konungen i Jeriko till Rahab och lät säga: "Lämna ut de män som hava kommit till dig och tagit in i ditt hus, ty de hava kommit hit för att utforska hela landet."
\par 4 Men kvinnan tog de båda männen och dolde dem; sedan svarade hon: "Ja, männen kommo till mig, men jag visste icke varifrån de voro;
\par 5 och när porten skulle stängas, sedan det hade blivit mörkt, gingo männen ut, och jag vet icke vart de togo vägen; skynden eder att sätta efter dem, så fån I nog fatt i dem."
\par 6 Men hon hade fört dem upp på taket och gömt dem under linstjälkar, som hon hade där, utbredda på taket.
\par 7 Så satte nu männen efter dem åt Jordan till, bort emot vadställena; och man stängde stadsporten så snart förföljarna hade begivit sig åstad.
\par 8 Men innan de främmande männen hade lagt sig, steg hon upp till dem på taket
\par 9 och sade till dem: "Jag vet att HERREN har givit eder detta land, och att förskräckelse för eder har fallit över oss, ja, att alla landets inbyggare äro i ångest för eder.
\par 10 Ty vi hava hört huru HERREN lät vattnet i Röda havet torka ut framför eder, när I drogen ut ur Egypten, och vad I haven gjort med amoréernas konungar, de två på andra sidan Jordan, Sihon och Og, huru I gåven dem till spillo.
\par 11 Då vi hörde detta, blevo våra hjärtan förfärade, och numera har ingen mod att stå eder emot; ty HERREN, eder Gud, är Gud, uppe i himmelen och nere på jorden.
\par 12 Så loven mig nu med ed vid HERREN, att eftersom jag har gjort barmhärtighet med min faders hus och giva mig ett säkert tecken därpå,
\par 13 och låta min fader och min moder, mina bröder och mina systrar leva, så ock alla som tillhöra dem, och rädda oss från döden."
\par 14 Männen sade till henne: "Med vårt eget liv svara vi för edert, såframt I icke förråden vårt förehavande; när HERREN giver oss landet, skola vi bevisa dig barmhärtighet och trofasthet."
\par 15 Då släppte hon ned dem genom fönstret med ett tåg; ty hennes hus låg invid stadsmuren, så att hon bodde invid själva muren.
\par 16 Och hon sade till dem: "Gån upp i bergsbygden, så att edra förföljare icke träffa på eder; och hållen eder gömda där i tre dagar, till dess edra förföljare hava kommit tillbaka, så kunnen I sedan fortsätta eder färd."
\par 17 Och männen sade till henne: "Vi vilja likväl vara fria ifrån den ed som du nu har tagit av oss,
\par 18 om du, när vi komma in i landet, underlåter att binda detta röda snöre i det fönster genom vilket du har släppt ned oss, och likaledes om du icke har din fader och din moder och dina bröder, alla av din faders hus, samlade hemma hos dig.
\par 19 Dock, om någon går åstad, utom dörrarna till ditt hus, så komme hans blod över hans huvud, och vi äro utan skuld; om däremot någons hand kommer vid en av dem som äro inne i ditt hus, så må dennes blod komma över vårt huvud.
\par 20 Och om du förråder vårt förehavande, så äro vi likaledes fria ifrån den ed som du har tagit av oss."
\par 21 Hon svarade: "Vare det såsom I haven sagt." Och så lät hon dem gå, och de drogo åstad. Men hon band det röda snöret i fönstret.
\par 22 Så drogo de nu åstad och kommo upp i bergsbygden och stannade där i tre dagar, till dess att deras förföljare hade vänt tillbaka; ty dessa hade sökt efter dem överallt på vägarna, men hade icke funnit dem.
\par 23 Sedan vände de båda männen tillbaka och kommo ned från bergsbygden och gingo över floden och kommo så till Josua, Nuns son; och de förtäljde för honom allt vad som hade vederfarits dem.
\par 24 Och de sade till Josua: "HERREN har givit hela landet i vår hand; alla landets inbyggare äro i ångest för oss."

\chapter{3}

\par 1 Bittida följande morgon bröt Josua med alla Israels barn upp från Sittim och kom till Jordan; där stannade de om natten, innan de gingo över.
\par 2 Men efter tre dagar gingo tillsyningsmännen genom lägret
\par 3 och bjödo folket och sade: "Så snart I fån se HERRENS, eder Guds, förbundsark, och att de levitiska prästerna bära den, skolen ock I bryta upp från eder plats och följa efter den
\par 4 - låten dock mellan den och eder vara ett avstånd av vid pass två tusen alnar; närmare mån I icke komma den - på det att I mån kunna veta vilken väg I skolen gå, ty I haven icke förut dragit den vägen fram."
\par 5 Och Josua sade till folket: "Helgen eder, ty i morgon skall HERREN göra under bland eder."
\par 6 Därefter sade Josua till prästerna: "Tagen förbundsarken och dragen åstad framför folket." Då togo de förbundsarken och gingo framför folket.
\par 7 Och HERREN sade till Josua: "I dag skall jag begynna att göra dig stor i hela Israels ögon, på det att de må förnimma, att såsom jag har varit med Mose, så vill jag ock vara med dig.
\par 8 Bjud du nu prästerna som bära förbundsarken och säg: 'Så snart I kommen till den yttersta randen av Jordans vatten, skolen I stanna där, vid Jordan.'"
\par 9 Då sade Josua till Israels barn: "Träden fram hit och hören HERRENS, eder Guds, ord."
\par 10 Och Josua sade: "Härav skolen I förnimma att en levande Gud är mitt ibland eder, och att han förvisso vill fördriva för eder kananéerna, hetiterna, hivéerna, perisséerna, girgaséerna, amoréerna och jebuséerna:
\par 11 förbundsarken, hela jordens Herres förbundsark, drager nu framför eder över Jordan.
\par 12 Väljen alltså ut tolv män ur Israels stammar, en man för var stam.
\par 13 Så snart då prästerna som bära HERRENS, hela jordens Herres, ark stå stilla med sina fötter i Jordans vatten, det vatten som kommer uppifrån, bliva avskuret i sitt lopp, och det skall stå såsom en samlad hög."
\par 14 Folket bröt då upp från sina tält för att gå över Jordan, och prästerna som buro förbundsarken gingo framför folket.
\par 15 När nu de som buro arken kommo till Jordan, så att prästerna, som buro arken, med sina fötter vidrörde yttersta randen av vattnet i Jordan, vilken under hela skördetiden är full över alla sina bräddar,
\par 16 då stannade det vatten som kom uppifrån, och blev stående såsom en samlad hög långt borta, uppe vid Adam, staden som ligger bredvid Saretan; och det vatten som flöt ned mot Hedmarkshavet, Salthavet, blev sålunda helt och hållet avskuret. Och folket gick över mitt emot Jeriko.
\par 17 Men prästerna som buro HERRENS förbundsark stodo orörliga på torr mark mitt i Jordan; och hela Israel gick över på torr mark, till dess att allt folket helt och hållet hade kommit över Jordan.

\chapter{4}

\par 1 Då nu allt folket helt och hållet hade kommit över Jordan, sade HERREN till Josua:
\par 2 "Väljen ut bland folket tolv män, en man ur var stam,
\par 3 och bjuden dem och sägen: 'Tagen här ur Jordan, från den plats där prästerna stodo med sina fötter, tolv stenar, och lyften upp dem och fören dem över med eder, och läggen ned dem på det ställe, där I skolen lägra eder i natt.'"
\par 4 Då kallade Josua till sig de tolv män som han hade utsett bland Israels barn, en man ur var stam.
\par 5 Och Josua sade till dem: "Dragen åstad framför HERRENS, eder Guds, ark, och gån ut mitt i Jordan; och var och en av eder må där lyfta upp en sten på axeln, efter antalet av Israels barns stammar.
\par 6 Detta skall nämligen bliva ett minnesmärke bland eder. När då edra barn i framtiden fråga: 'Vad betyda dessa stenar?',
\par 7 skolen I svara dem så: 'De betyda att Jordans vatten här blev avskuret i sitt lopp, framför HERRENS förbundsark; ja, när den gick över Jordan, blev Jordans vatten avskuret i sitt lopp. Därför skola dessa stenar vara ett åminnelsemärke för Israels barn till evärdlig tid.'"
\par 8 Då gjorde Israels barn såsom Josua bjöd dem; de togo upp tolv stenar ur Jordan, såsom HERREN hade tillsagt Josua, efter antalet av Israels barns stammar; och de förde dem över med sig till lägerstället och lade ned dem där.
\par 9 Tillika reste Josua tolv stenar mitt i Jordan, på samma plats där prästerna som buro förbundsarken hade stått med sina fötter; och de finnas kvar där ännu i dag.
\par 10 Och prästerna som buro arken blevo stående mitt i Jordan, till dess att allt det var fullgjort, som HERREN hade bjudit Josua att tillsäga folket, alldeles i enlighet med vad Mose förut hade bjudit Josua; och folket gick över med hast.
\par 11 Men när allt folket helt och hållet hade kommit över, gick ock HERRENS ark över, jämte prästerna, och tog plats framför folket.
\par 12 Och Rubens barn och Gads barn och ena hälften av Manasse stam drogo väpnade åstad i spetsen för Israels barn, såsom Mose hade tillsagt dem.
\par 13 Det var vid pass fyrtio tusen män som så drogo åstad, väpnade till strid, för att kämpa inför HERREN på Jerikos hedmarker.
\par 14 På den dagen gjorde HERREN Josua stor i hela Israels ögon, och de fruktade honom, såsom de fruktat Mose, så länge denne levde.
\par 15 Och HERREN sade till Josua:
\par 16 "Bjud prästerna som bära vittnesbördets ark att stiga upp ur Jordan."
\par 17 Och Josua bjöd prästerna och sade: "Stigen upp ur Jordan."
\par 18 När då prästerna som buro HERRENS förbundsark stego upp ur Jordan, hade deras fötter knappt hunnit upp på torra landet, förrän Jordans vatten vände tillbaka till sin plats och nådde, såsom förut, upp över alla sina bräddar.
\par 19 Det var på tionde dagen i första månaden som folket steg upp ur Jordan; och de lägrade sig i Gilgal, på gränsen av östra Jerikoområdet.
\par 20 Och de tolv stenarna som de hade tagit ur Jordan reste Josua i Gilgal.
\par 21 Och han sade till Israels barn: "När nu edra barn i framtiden fråga sina fäder: 'Vad betyda dessa stenar?',
\par 22 då skolen I göra det kunnigt för edra barn och säga: 'Israel gick på torr mark över denna Jordan,
\par 23 i det att HERREN, eder Gud, lät vattnet i Jordan torka ut framför eder, till dess I haden gått över den, likasom HERREN, eder Gud, gjorde med Röda havet, som han lät torka ut framför oss, till dess vi hade gått över det;
\par 24 på det att alla folk på jorden må förnimma huru stark HERRENS hand är, så att I frukten HERREN, eder Gud, alltid.'"

\chapter{5}

\par 1 Då nu alla amoréernas konungar på andra sidan Jordan, på västra sidan, och alla kananéernas konungar vid havet hörde huru HERREN hade låtit vattnet i Jordan torka ut framför Israels barn, medan vi gingo över den, blevo deras hjärtan förfärade, och de hade icke längre mod att stå emot Israels barn.
\par 2 Vid den tiden sade HERREN till Josua: "Gör dig stenknivar och omskär åter Israels barn, för andra gången."
\par 3 Då gjorde Josua sig stenknivar och omskar Israels barn vid Förhudshöjden.
\par 4 Och orsaken varför Josua omskar dem var denna: allt det folk av mankön, som hade dragit ut ur Egypten, alla stridbara män, hade dött i öknen under vägen, efter uttåget ur Egypten.
\par 5 Ty väl hade bland folket alla de som voro med under uttåget blivit omskurna, men de bland folket, som voro födda i öknen under vägen, efter uttåget ur Egypten, de voro alla oomskurna.
\par 6 Ty Israels barn vandrade i öknen i fyrtio år, under vilken tid alla stridbara män i folket, som hade dragit ut ur Egypten, förgingos, eftersom de icke hörde HERRENS röst, varför ock HERREN svor att han icke skulle låta dem se det land som han med ed hade lovat deras fäder att giva oss, ett land som flyter av mjölk och honung.
\par 7 Men deras barn, som han hade låtit uppstå i deras ställe, dem omskar nu Josua, ty de hade förhud, eftersom de icke hade blivit omskurna under vägen.
\par 8 Och när allt folket hade blivit omskuret, stannade de kvar där de voro i lägret, till dess de hade blivit läkta.
\par 9 Och HERREN sade till Josua: "I dag har jag avvältrat från eder Egyptens smälek." Och detta ställe fick namnet Gilgal, såsom det heter ännu i dag.
\par 10 Medan nu Israels barn voro lägrade i Gilgal, höllo de påskhögtid den fjortonde dagen i månaden, om aftonen, på Jerikos hedmarker.
\par 11 Och dagen efter påskhögtiden åto de osyrat bröd och rostade ax av landets säd, just på den dagen.
\par 12 Och mannat upphörde dagen därefter, då de nu åto av landets säd, och Israels barn fingo icke manna mer, utan de åto det året av landet Kanaans avkastning.
\par 13 Och medan Josua var vid Jeriko, hände sig att han, i det han lyfte upp sina ögon, fick se en man stå där framför sig med ett draget svärd i sin hand. Då gick Josua fram till honom och frågade honom: "Tillhör du oss eller våra ovänner?"
\par 14 Han svarade: "Nej, jag är hövitsman över HERRENS här, och jag har just nu kommit hit." Då föll Josua ned till jorden på sitt ansikte och bugade sig; sedan sade han till honom: "Vad har min herre att säga till sin tjänare?"
\par 15 Hövitsmannen över HERRENS här sade då till Josua: "Drag dina skor av dina fötter, ty platsen där du står är helig." Och Josua gjorde så.

\chapter{6}

\par 1 Och Jeriko hade sina portar stängda, det höll sig tillstängt för Israels barn; ingen gick ut eller in.
\par 2 Men HERREN sade till Josua: "Se, jag har givit Jeriko med dess konung, med dess tappra stridsmän, i din hand.
\par 3 Tågen nu omkring staden, så många stridbara män I ären, runt omkring staden en gång; så skall du göra i sex dagar.
\par 4 Och sju präster skola bära de sju jubelbasunerna framför arken; men på sjunde dagen skolen I tåga omkring staden sju gånger; och prästerna skola stöta i basunerna.
\par 5 Och när det blåses i jubelhornet med utdragen ton, och I hören basunljudet, skall allt folket upphäva ett stort härskri; då skola stadsmurarna falla på stället, och folket skall draga in över dem, var och en rätt fram."
\par 6 Då kallade Josua, Nuns son, till sig prästerna och sade till dem: "Tagen förbundsarken, och sju präster skola bära sju jubelbasuner framför HERRENS ark."
\par 7 Och till folket blev sagt: "Dragen ut och tågen omkring staden; och den väpnade skaran skall draga framför HERRENS ark."
\par 8 Då nu Josua hade sagt detta till folket, drogo de sju präster som buro jubelbasunerna framför HERREN åstad och stötte i basunerna; och HERRENS förbundsark följde efter dem.
\par 9 Och den väpnade skaran gick framför prästerna som stötte i basunerna, och den övriga hopen slutade tåget och följde efter arken, under det att man alltjämt stötte i basunerna.
\par 10 Men Josua hade bjudit folket och sagt: "I skolen icke upphäva något härskri eller låta höra eder röst eller ens låta något ord utgå av eder mun, förrän den dag då jag säger till eder: 'Häven upp ett härskri'; då skolen I upphäva ett härskri."
\par 11 Och när han så hade låtit bära HERRENS ark omkring staden, runt omkring den en gång, gingo de in i lägret och stannade i lägret över natten.
\par 12 Och följande morgon stod Josua bittida upp, och prästerna togo HERRENS ark.
\par 13 Och de sju präster som buro de sju jubelbasunerna framför HERRENS ark gingo alltjämt och stötte i basunerna; och den väpnade skaran gick framför dem, och den övriga hopen slutade tåget och följde efter HERRENS ark, under det att man alltjämt stötte i basunerna.
\par 14 De tågade också nu på andra dagen en gång omkring staden och återvände sedan till lägret; så gjorde de i sex dagar.
\par 15 Men på sjunde dagen stodo de bittida upp vid morgonrodnadens uppgång och tågade då sju gånger omkring staden på samma sätt; endast denna dag tågade de sju gånger omkring staden.
\par 16 Och när prästerna sjunde gången stötte i basunerna, sade Josua till folket: "Häven upp ett härskri, ty HERREN har givit eder staden.
\par 17 Men staden med allt vad däri är skall givas till spillo åt HERREN; allenast skökan Rahab skall få leva, jämte alla som äro inne i hennes hus, därför att hon gömde de utskickade som vi hade sänt åstad.
\par 18 Men tagen eder väl till vara för det tillspillogivna, så att I icke, sedan I haven givit det till spillo, ändå tagen något av det tillspillogivna och därigenom kommen Israels läger att hemfalla åt tillspillogivning, och så dragen olycka över det.
\par 19 Allt silver och guld och allt som är av koppar eller järn skall vara helgat åt HERREN och ingå till HERRENS skatt."
\par 20 Då hov folket upp ett härskri, och man stötte i basunerna. Ja, när folket hörde basunljudet, hov det upp ett stort härskri; då föllo murarna på stället, och folket drog över dem in i staden, var och en rätt fram; så intogo de staden.
\par 21 Och de gåvo till spillo allt vad som fanns i staden, både män och kvinnor, både unga och gamla, så ock oxar, får och åsnor, och slogo dem med svärdsegg.
\par 22 Men till de båda män som hade bespejat landet sade Josua: "Gån in i skökans hus och fören kvinnan, jämte alla som tillhöra henne, ut därifrån, såsom I med ed haven lovat henne."
\par 23 Då gingo de unga män som hade varit där såsom spejare ditin och förde ut Rahab, jämte hennes fader och moder och hennes bröder och alla som tillhörde henne; hela hennes släkt förde de ut. Och de släppte dem utanför Israels läger.
\par 24 Men staden med allt vad som fanns däri brände de upp i eld; allenast silvret och guldet och det som var av koppar eller järn lade de till skatten i HERRENS hus.
\par 25 Men skökan Rahab och hennes faders hus och alla som tillhörde henne lät Josua leva, och hon fick bo bland Israels folk, intill denna dag; detta därför att hon gömde de utskickade som Josua hade sänt åstad för att bespeja Jeriko.
\par 26 På den tiden lät Josua folket svärja denna ed: "Förbannad vare inför HERREN den man som tager sig före att åter bygga upp denna stad, Jeriko. När han lägger dess grund, må detta kosta honom hans äldste son, och när han sätter upp dess portar, må detta kosta honom hans yngste son."
\par 27 Och HERREN var med Josua, så att ryktet om honom gick ut över hela landet.

\chapter{7}

\par 1 Men Israels barn förgrepo sig trolöst på det tillspillogivna; ty Akan, son till Karmi, son till Sabdi, son till Sera, av Juda stam, tog något av det tillspillogivna. Då upptändes HERRENS vrede mot Israels barn.
\par 2 Och Josua sände från Jeriko några män åstad till Ai, som ligger vid Bet-Aven, öster om Betel, och sade till dem: "Dragen ditupp och bespejen landet." Så drogo då männen upp och bespejade Ai.
\par 3 Och när de kommo tillbaka till Josua, sade de till honom: "Allt folket behöver icke draga ditupp; om vid pass två eller tre tusen man draga upp, skola de nog intaga Ai. Du behöver icke låta allt folket göra sig mödan att tåga dit, ty dess invånare äro få."
\par 4 Alltså fingo vid pass tre tusen man av folket draga ditupp; men dessa måste fly för ajiterna.
\par 5 Och sedan ajiterna hade slagit vid pass trettiosex man av dem, förföljde de de övriga utanför stadsporten ända till Sebarim och slogo dem på sluttningen där. Då blev folkets hjärta förfärat, det blev såsom vatten.
\par 6 Och Josua med de äldste i Israel rev sönder sina kläder, och föll ned på sitt ansikte till jorden framför HERRENS ark och låg där ända till aftonen, och de strödde stoft på sina huvuden.
\par 7 Och Josua sade: "Ack, Herre, HERRE, varför har du då fört detta folk över Jordan, om du vill giva i amoréernas hand och så förgöra oss? O att vi hade beslutit oss för att stanna på andra sidan Jordan!
\par 8 Ack Herre, vad skall jag nu säga, sedan Israel har tagit till flykten för sina fiender?
\par 9 När kananéerna och landets alla övriga inbyggare få höra detta, skola de omringa oss och utrota till och med vårt namn från jorden. Vad vill du då göra för ditt stora namns ära?"
\par 10 Men HERREN svarade Josua: "Stå upp. Varför ligger du så på ditt ansikte?
\par 11 Israel har syndat, de hava överträtt det förbund som jag stadgade för dem; de hava tagit av det tillspillogivna, de hava stulit, de hava ljugit, de hava gömt det bland sitt eget gods.
\par 12 Därför kunna Israels barn icke stå emot sina fiender, utan de måste taga till flykten för sina fiender, ty de äro nu själva hemfallna åt tillspillogivning. Jag vill icke mer vara med eder, om I icke alldeles skaffen bort ifrån eder det tillspillogivna.
\par 13 Stå nu upp och helga folket och säg: Helgen eder till i morgon. Ty så säger HERREN, Israels Gud: Något tillspillogivet finnes hos dig, Israel; du skall icke kunna stå emot dina fiender, förrän I skiljen det tillspillogivna från eder.
\par 14 I morgon skolen I träda fram, den ena stammen efter den andra; i den stam som HERREN då låter träffas av lotten skall den ena släkten efter den andra träda fram; och i den släkt som HERREN låter träffas av lotten skall den ena familjen efter den andra träda fram; och i den familj som HERREN låter träffas av lotten skall den ena mannen efter den andra träda fram.
\par 15 Och den som då träffas av lotten såsom skyldig till förgripelse på det tillspillogivna, han skall brännas upp i eld med allt vad han har, därför att han överträdde HERRENS förbund och gjorde vad som var en galenskap i Israel."
\par 16 Så lät nu Josua bittida följande morgon Israel träda fram, den ena stammen efter den andra; då träffades Juda stam av lotten.
\par 17 När han då lät Juda släkter träda fram, träffade lotten seraiternas släkt; och när han lät seraiternas släkt träda fram, den ena mannen efter den andra, träffades Sabdi av lotten.
\par 18 När han då lät hans familj träda fram, den ena mannen efter den andra, träffade lotten Akan, son till Karmi, son till Sabdi, son till Sera av Juda stam.
\par 19 Då sade Josua till Akan: "Min son, giv ära åt HERREN, Israels Gud, och bekänn, honom till pris: säg mig vad du gjort och dölj intet för mig."
\par 20 Akan svarade Josua och sade: "Det är sant, jag har syndat mot HERREN, Israels Gud, ty så har jag gjort:
\par 21 jag såg ibland bytet en dyrbar mantel från Sinear och två hundra siklar silver och en guldplatta, femtio siklar i vikt, och till detta fick begärelse och tog det; se, det är gömt i jorden, i mitten av mitt tält, och silvret underst."
\par 22 Då sände Josua några män dit för att se efter, och de skyndade till tältet; och de funno det gömt där i hans tält, och silvret underst.
\par 23 Och de togo det ur tältet och buro det till Josua och Israels barns menighet och lade det ned inför HERREN.
\par 24 Då tog Josua och Israels menighet med honom Akan, Seras son, och silvret och manteln och guldplattan, och hans söner och döttrar, hans oxar, åsnor och får, och hans tält, och allt övrigt som han hade, och förde alltsammans upp till Akors dal.
\par 25 Och Josua sade: "Varför drog du olycka över oss? Nu skall ock HERREN i dag låta olycka komma över dig." Och Israels menighet stenade honom; de brände upp dem i eld och kastade stenar på dem.
\par 26 Och de uppkastade över honom ett stort stenröse, som finnes kvar ännu i dag; och så vände sig HERREN ifrån sin vredes glöd. Därav fick det stället namnet Akors dal, såsom det heter ännu i dag.

\chapter{8}

\par 1 Och HERREN sade till Josua: "Frukta icke och var icke förfärad; tag med dig allt krigsfolket och stå upp och drag åstad mot Ai. Se, i din hand har jag givit konungen i Ai med hans folk, hans stad och hans land.
\par 2 Och du skall göra med Ai och dess konung på samma sätt som du gjorde med Jeriko och dess konung; dock mån I behålla rovet därifrån och boskapen, såsom edert byte. Lägg nu ett bakhåll mot staden, på andra sidan därom."
\par 3 Då bröt Josua upp med allt krigsfolket för att draga åstad mot Ai. Och Josua utvalde trettio tusen man, de tappraste stridsmännen, och sände dem ut om natten.
\par 4 Och han bjöd dem och sade: "Given akt: I skolen lägga eder i bakhåll mot staden, på andra sidan därom, men läggen eder icke alltför långt ifrån staden; och hållen eder alla redo.
\par 5 Själv skall jag, med allt det folk som är kvar hos mig, rycka fram mot staden. När de då draga ut mot oss såsom förra gången, vilja vi fly för dem.
\par 6 Då skola de draga efter oss, till dess vi hava lockat dem långt bort ifrån staden; ty de skola tänka: 'De flyr för oss, nu såsom förra gången.'
\par 7 Men under det att vi fly för dem, skolen I bryta fram ifrån bakhållet och intaga staden, ty HERREN, eder Gud, har givit den i eder hand.
\par 8 Och så snart I haven fått staden i edert våld, skolen I tända eld på den; efter HERRENS ord skolen I så göra. Given akt på vad jag nu har bjudit eder."
\par 9 Så sände Josua dem åstad, och de gingo och lade sig i bakhåll mellan Betel och Ai, väster om Ai. Men Josua stannade över natten bland folket.
\par 10 Och bittida följande morgon mönstrade Josua folket och drog så, med de äldste i Israel i spetsen för folket, upp till Ai.
\par 11 Och allt det krigsfolk som var kvar hos honom drog med ditupp och ryckte allt närmare, till dess de kommo mitt emot staden; där lägrade de sig norr om Ai, med dalen mellan sig och Ai.
\par 12 Men han tog vid pass fem tusen man och lade dem i bakhåll mellan Betel och Ai, väster om staden.
\par 13 Och sedan folket hade blivit uppställt, såväl hela lägret, norr om staden, gick Josua den natten fram till mitten av dalen.
\par 14 När konungen i Ai såg detta, skyndade sig männen i staden, han själv med allt sitt folk, och drogo bittida om morgonen ut till strid mot Israel, bort till den utsedda platsen, framför hedmarken; han själv visste nämligen icke att ett bakhåll var lagt mot honom på andra sidan om staden.
\par 15 Och Josua och hela Israel läto slå sig av dem och flydde åt öknen till.
\par 16 Då uppbådades allt folket i staden till att förfölja dem; och under det att de förföljde Josua, blevo de lockade långt bort ifrån staden.
\par 17 Icke en enda man blev kvar i Ai eller i Betel, utan alla drogo ut efter Israel och lämnade staden öppen, i det att de förföljde Israel.
\par 18 Och HERREN sade till Josua: "Räck ut lansen, som du har i din hand, mot Ai, ty jag skall giva det i din hand." Då räckte Josua ut lansen, som han hade i sin hand, mot staden.
\par 19 Och de som lågo i bakhåll bröto med hast upp från sin plats och skyndade åstad, så snart han räckte ut sin hand, och kommo in i staden och intogo den; och de tände strax eld på staden.
\par 20 När då männen från Ai vände sig om, fingo de se röken från staden stiga upp mot himmelen; och de hade ingen utväg att fly, vare sig hit eller dit, då nu det folk som flydde åt öknen vände sig mot sina förföljare.
\par 21 Ty när Josua och hela Israel sågo att de som lågo i bakhåll hade intagit staden, och att röken steg upp från staden, vände de om och angrepo ajiterna.
\par 22 De andra drogo nu också ut från staden emot dem, så att de kommo mitt emellan israeliterna och fingo dem på båda sidor om sig, och dessa nedgjorde dem då och läto ingen av dem slippa undan och rädda sig.
\par 23 Men konungen i Ai blev levande tagen till fånga och förd till Josua.
\par 24 Och när Israel hade dräpt alla Ais invånare ute på fältet, i öknen, dit de hade förföljt dem, och dessa allasammans så hade fallit för svärdsegg och blivit nedgjorda, då vände hela Israel tillbaka till Ai och slog med svärdsegg också dem som voro där.
\par 25 Och de som föllo på den dagen, män och kvinnor, utgjorde tillsammans tolv tusen personer, allt folket i Ai.
\par 26 Ty Josua drog icke tillbaka sin hand, med vilken han hade räckt ut lansen, förrän alla Ais invånare hade blivit givna till spillo.
\par 27 Allenast boskapen och rovet från denna stad togo israeliterna såsom sitt byte, efter den befallning som HERREN hade givit Josua.
\par 28 Och Josua brände upp Ai och gjorde det till en grushög för evärdlig tid, till en ödemark, såsom det är ännu i dag.
\par 29 Och konungen i Ai lät han hänga upp på en påle, där han fick hänga ända till aftonen. Men när solen gick ned, tog man på Josuas befallning hans döda kropp ned från pålen och kastade den vid ingången till stadsporten; och man uppkastade över den ett stort stenröse, som finnes kvar ännu i dag.
\par 30 Då byggde Josua åt HERREN, Israels Gud, ett altare på berget Ebal,
\par 31 såsom HERRENS tjänare Mose hade bjudit Israels barn, och såsom det var föreskrivet i Moses lagbok: ett altare av ohuggna stenar, vid vilka man icke hade kommit med något järn; och på det offrade de brännoffer åt HERREN och slaktade tackoffer.
\par 32 Och han lät där på stenarna sätta en avskrift av Moses lag, den lag som Mose hade skrivit och förelagt Israels barn.
\par 33 Och Israels menighet, med dess äldste och tillsyningsmän och domare, stod på båda sidor om arken, så att de hade framför sig de levitiska prästerna som buro HERRENS förbundsark, menigheten, främlingar såväl som infödingar, den ena hälften vänd mot berget Gerissim och den andra hälften mot berget Ebal, i enlighet med vad HERRENS tjänare Mose hade bjudit, nämligen att man först skulle välsigna Israels folk.
\par 34 Därefter läste han upp alla lagens ord, välsignelsen och förbannelsen, alldeles såsom det var skrivet i lagboken.
\par 35 Icke ett ord av allt det som Mose hade bjudit underlät Josua att uppläsa inför Israels hela församling, med kvinnor och barn, och inför de främlingar som följde med dem.

\chapter{9}

\par 1 Då nu alla de konungar som bodde på andra sidan Jordan, i Bergsbygden, i Låglandet och i hela kustlandet vid Stora havet upp emot Libanon, hörde vad som hade skett - hetiterna, amoréerna, kananéerna, perisséerna, hivéerna och jebuséerna -
\par 2 slöto de sig endräktigt tillhopa för att strida mot Josua och Israel.
\par 3 Men när invånarna i Gibeon hörde vad Josua hade gjort med Jeriko och Ai,
\par 4 togo också de sin tillflykt till list: de gingo åstad och föregåvo sig vara sändebud; de lade utslitna packsäckar på sina åsnor, så ock utslitna, sönderspruckna och hopflickade vinläglar av skinn,
\par 5 och togo utslitna, lappade skor på sina fötter och klädde sig i utslitna kläder, varjämte allt det bröd de togo med sig till reskost var torrt och söndersmulat.
\par 6 Så gingo de till Josua i lägret vid Gilgal och sade till honom och Israels män: "Vi hava kommit hit från ett avlägset land; sluten nu förbund med oss."
\par 7 Men Israels män svarade hivéerna: "Kanhända bon I här mitt ibland oss; huru skulle vi då kunna sluta förbund med eder?"
\par 8 Då sade de till Josua: "Vi vilja bliva dig underdåniga." Josua frågade dem: "Vilka ären I då, och varifrån kommen I?"
\par 9 De svarade honom: "Dina tjänare hava kommit från ett mycket avlägset land för HERRENS, din Guds, namns skull; ty vi hava hört ryktet om honom och allt vad han har gjort i Egypten
\par 10 och allt vad han har gjort med amoréernas konungar, de två på andra sidan Jordan, Sihon, konungen i Hesbon, och Og, konungen i Basan, som bodde i Astarot.
\par 11 Därför sade våra äldste och alla vårt lands inbyggare till oss: 'Tagen reskost med eder och gån dem till mötes och sägen till dem: Vi vilja bliva eder underdåniga, sluten nu förbund med oss.'
\par 12 Detta vårt bröd var nybakat, när vi togo det med oss till reskost hemifrån, den dag vi gåvo oss i väg för att gå till eder; men se, nu är det torrt och söndersmulat.
\par 13 Dessa vinläglar, som voro nya, när vi fyllde dem, se, de äro nu sönderspruckna. Och dessa kläder och skor som vi hava på oss hava blivit utslitna under vår mycket långa resa."
\par 14 Då togo männen av deras reskost, men rådfrågade icke HERRENS mun.
\par 15 Och Josua tillförsäkrade dem fred och slöt ett förbund med dem, att de skulle få leva; och menighetens hövdingar gåvo dem sin ed.
\par 16 Men när tre dagar voro förlidna, sedan de hade slutit förbund med dem, fingo de höra att de voro från grannskapet, ja, att de bodde mitt ibland dem.
\par 17 Då bröto Israels barn upp och kommo på tredje dagen till deras städer; och deras städer voro: Gibeon, Kefira, Beerot och Kirjat-Jearim.
\par 18 Likväl angrepo Israels barn dem icke, eftersom menighetens hövdingar hade givit dem sin ed vid HERREN, Israels Gud. Men hela menigheten knorrade mot hövdingarna.
\par 19 Då sade alla hövdingarna till menigheten: "Vi hava givit dem vår ed vid HERREN, Israels Gud; därför kunna vi nu icke komma vid dem.
\par 20 Detta är vad vi vilja göra med dem, i det att vi låta dem leva, på det att icke förtörnelse må komma över oss, för edens skull som vi hava svurit dem."
\par 21 Och hövdingarna sade till dem att de skulle få leva; men de måste bliva vedhuggare och vattenbärare åt hela menigheten, såsom hövdingarna hade sagt till dem.
\par 22 Och Josua kallade dem till sig och talade till dem och sade: "Varför haven I bedragit oss och sagt: 'Vi bo mycket långt borta från eder', fastän I bon mitt ibland oss?
\par 23 Så varen I därför nu förbannade; I skolen aldrig upphöra att vara trälar, vedhuggare och vattenbärare vid min Guds hus."
\par 24 De svarade Josua och sade: "Det hade blivit berättat för dina tjänare huru HERREN, din Gud, hade tillsagt sin tjänare Mose att han ville giva eder hela detta land och förgöra alla landets inbyggare för eder; därför fruktade vi storligen för våra liv, när I kommen, och så gjorde vi detta.
\par 25 Och se, nu äro vi i din hand. Vad dig synes gott och rätt att göra med oss, det må du göra."
\par 26 Och han gjorde så med dem; han friade dem från Israels barns hand, så att de icke dräpte dem;
\par 27 men tillika bestämde Josua på den dagen att de skulle bliva vedhuggare och vattenbärare och vid HERRENS altare - såsom de äro ännu i dag - på den plats som han skulle utvälja.

\chapter{10}

\par 1 Då nu Adoni-Sedek, konungen i Jerusalem, hörde att Josua hade intagit Ai och givit det till spillo, och att han hade gjort med Ai och dess konung på samma sätt som han hade gjort med Jeriko och dess konung, och att invånarna i Gibeon hade ingått fred med Israel och fingo bo mitt ibland dem,
\par 2 fruktade han och hans folk storligen, ty Gibeon var en stor stad, såsom en av konungastäderna, ja, det var större än Ai, och dess män voro alla tappra.
\par 3 Och Adoni-Sedek, konungen i Jerusalem, sände till Hoham, konungen i Hebron, till Piram, konungen i Jarmut, till Jafia, konungen i Lakis, och till Debir, konungen i Eglon, och lät säga:
\par 4 "Kommen hitupp till mig och hjälpen mig, så att vi kunna slå gibeoniterna, ty de hava ingått fred med Josua och Israels barn."
\par 5 Så församlade sig då de fem amoreiska konungarna, konungen i Jerusalem, konungen i Hebron, konungen i Jarmut, konungen i Lakis, konungen i Eglon, och drogo ditupp med alla sina härar; och de belägrade Gibeon och angrepo det.
\par 6 Men gibeoniterna sände till Josua i lägret vid Gilgal och läto säga: "Drag icke din hand från dina tjänare, utan kom hitupp till oss med hast och undsätt oss och hjälp oss, ty konungarna över amoréerna, som bo i bergsbygden, hava församlat sig mot oss."
\par 7 Då drog Josua ditupp från Gilgal med allt sitt krigsfolk och alla sina tappraste stridsmän.
\par 8 Och HERREN sade till Josua: "Frukta icke för dem, ty jag har givit dem i dina händer; ingen av dem skall kunna stå dig emot."
\par 9 Och Josua kom plötsligt över dem, ty han tågade hela natten, sedan han hade brutit upp från Gilgal.
\par 10 Och HERREN sände en sådan förvirring bland dem, när de fingo se israeliterna, att dessa tillfogade dem ett stort nederlag vid Gibeon; därefter förföljde de dem på vägen upp till Bet-Horon och nedgjorde dem, och drevo dem ända till Aseka och Mackeda.
\par 11 Och när de så, under sin flykt för Israel, hade kommit till den sluttning som går ned från Bet-Horon, lät HERREN stora stenar falla över dem från himmelen, hela vägen ända till Aseka, så att de blevo dödade; de som dödades genom hagelstenarna voro till och med flera än de som Israels barn dräpte med svärd.
\par 12 Och Josua talade till HERREN på den dag då HERREN gav amoréerna i Israels barns våld; han sade inför Israel: "Du sol, stå stilla i Gibeon, du måne, i Ajalons dal!"
\par 13 Då stod solen stilla, och månen blev stående, till dess folket hade tagit hämnd på sina fiender. Detta finnes ju upptecknat i "Den redliges bok". Solen blev stående mitt på himmelen nästan en hel dag och hastade icke att gå ned.
\par 14 Aldrig har någon dag, varken förr eller senare, varit lik denna, i det att HERREN då lydde en mans ord; ty HERREN stridde för Israel.
\par 15 Och Josua med hela Israel vände tillbaka till lägret vid Gilgal.
\par 16 Men de fem konungarna flydde och gömde sig i grottan vid Mackeda.
\par 17 Då blev det inberättat för Josua: "Man har funnit de fem konungarna gömda i grottan till Mackeda."
\par 18 Josua sade: "Vältren stora stenar framför ingången till grottan, och sätten dit folk för att bevaka den.
\par 19 Men I andra, stannen icke, utan förföljen edra fiender, och nedgören dem som bliva efter; låten dem icke komma in i sina städer, ty HERREN, eder Gud har givit dem i eder hand."
\par 20 Då nu Josua och Israels barn hade tillfogat dem ett mycket stort nederlag och nedgjort dem - varvid dock några av dem lyckades rädda sig och komma in i de befästa städerna -
\par 21 vände allt folket välbehållet tillbaka till Josua i lägret vid Mackeda, ty ingen vågade mer ens röra sin tunga mot någon av Israels barn.
\par 22 Då sade Josua: "Öppnen grottan och fören de fem konungarna till mig, ut ur grottan."
\par 23 De gjorde så och förde de fem konungarna ut till honom ur grottan: konungen i Jerusalem, konungen i Hebron, konungen i Jarmut, konungen i Lakis, och konungen i Eglon.
\par 24 När dessa konungar hade blivit förda ut till Josua, kallade Josua till sig alla Israels män och sade till anförarna för krigsfolket som hade dragit med honom: "Träden fram och sätt edra fötter på dessa konungars halsar." Och de trädde fram och satte sina fötter på deras halsar.
\par 25 Sedan sade Josua till dem: "Frukten icke och varen icke försagda, utan varen frimodiga och oförfärade, ty så skall HERREN göra med alla sina fiender som I kommen i strid med."
\par 26 Därefter lät Josua slå dem till döds och hänga upp dem på fem pålar; och de fingo på pålarna ända till aftonen.
\par 27 Men vid solnedgången togos de på Josuas befallning ned från pålarna och kastades in i grottan där de hade varit gömda; och framför ingången till grottan lade man stora stenar, som ligga kvar där ännu i denna dag.
\par 28 Och Josua intog Mackeda på den dagen och slog dess invånare och dess konung med svärdsegg; han gav det till spillo med alla dem som voro därinne och lät ingen slippa undan. Och han gjorde med konungen i Mackeda på samma sätt som han hade gjort med konungen i Jeriko.
\par 29 Därefter drog Josua med hela Israel från Mackeda till Libna och belägrade Libna.
\par 30 Och HERREN gav också det och dess konung i Israels hand; och de slogo dess invånare med svärdsegg, alla dem som voro därinne, och läto ingen därinne slippa undan. Och han gjorde med dess konung på samma sätt som han hade gjort med konungen i Jeriko.
\par 31 Sedan drog Josua med hela Israel från Libna till Lakis och belägrade och angrep det.
\par 32 Och HERREN gav Lakis i Israels hand, så att de intogo det på andra dagen; och de slogo dess invånare med svärdsegg, alla dem som voro därinne - alldeles såsom de hade gjort med Libna.
\par 33 Då drog Horam, konungen i Geser, upp för att hjälpa Lakis; men Josua slog honom och hans folk och lät ingen av dem slippa undan.
\par 34 Och från Lakis drog Josua med hela Israel till Eglon, och de belägrade och angrepo det.
\par 35 Och de intogo det samma dag och slogo dess invånare med svärdsegg, och han gav på den dagen till spillo alla dem som voro därinne - alldeles såsom han hade gjort med Lakis.
\par 36 Sedan drog Josua med hela Israel från Eglon upp till Hebron och belägrade det.
\par 37 Och de intogo det och slogo dess invånare och dess konung med svärdsegg, så ock alla dess lydstäder och alla dem som voro därinne, och han lät ingen slippa undan - alldeles såsom han hade gjort med Eglon. Han gav det till spillo med alla dem som voro därinne.
\par 38 Därefter vände Josua med hela Israel tillbaka till Debir och belägrade det.
\par 39 Och han underkuvade det med dess konung och alla dess lydstäder, och de slogo deras invånare med svärdsegg; de gåvo till spillo alla dem som voro därinne, och han lät ingen slippa undan. Han gjorde med Debir och dess konung på samma sätt som han hade gjort med Hebron, och såsom han hade gjort med Libna och dess konung.
\par 40 Så intog Josua hela landet, Bergsbygden, Sydlandet, Låglandet och Bergssluttningarna, och slog alla konungar där och lät ingen slippa undan; han gav till spillo allt vad andra hade, såsom HERREN, Israels Gud, hade bjudit.
\par 41 Josua intog allt som fanns mellan Kades-Barnea och Gasa, så ock hela landet Gosen ända till Gibeon.
\par 42 Alla dessa konungar och deras land underkuvade Josua på en gång, ty HERREN, Israels Gud, stridde för Israel.
\par 43 Därefter vände Josua med hela Israel tillbaka till lägret vid Gilgal.

\chapter{11}

\par 1 Då nu Jabin, konungen i Hasor, hörde detta, sände han bud till Jobab, konungen i Madon, och till konungen i Simron och konungen i Aksaf
\par 2 och till de konungar som bodde norrut, i Bergsbygden och på Hedmarken, söder om Kinarot, och i Låglandet, så ock i Nafot-Dor, västerut,
\par 3 vidare till kananéerna österut och västerut och till amoréerna, hetiterna, perisséerna och jebuséerna i Bergsbygden, så och till hivéerna nedanför Hermon, i Mispalandet.
\par 4 Dessa drogo nu ut med alla sina härar, en folkskara så talrik som sanden på havets strand, jämte hästar och vagnar i stor myckenhet.
\par 5 Alla dessa konungar rotade sig samman; och de kommo och lägrade sig tillhopa vid Meroms vatten, för att strida mot Israel.
\par 6 Men HERREN sade till Josua: "Frukta icke för dem, ty i morgon vid denna tid vill jag själv giva dem allasammans slagna i Israels våld. På deras hästar skall du avskära fotsenorna, och deras vagnar skall du bränna upp i eld."
\par 7 Och Josua kom med allt sitt krigsfolk plötsligt över dem vid Meroms vatten och anföll dem.
\par 8 Och HERREN gav dem i Israels hand, och de slogo dem och förföljde dem ända till Stora Sidon, till Misrefot-Maim och till Mispedalen, österut; de slogo dem och läto ingen slippa undan.
\par 9 Och Josua gjorde med dem såsom HERREN hade befallt honom: på deras hästar lät han avskära fotsenorna, och deras vagnar lät han bränna upp i eld.
\par 10 Därefter, vid samma tid, vände Josua tillbaka och intog Hasor och slog dess konung med svärd; ty Hasor var fordom huvudstaden för alla dessa riken.
\par 11 Alla de som voro därinne blevo slagna med svärdsegg och givna till spillo, så att intet som anda hade lämnades kvar; och själva Hasor brände han upp i eld.
\par 12 Likaledes underkuvade Josua alla de andra konungastäderna med alla deras konungar, och han slog deras invånare med svärdsegg och gav dem till spillo, såsom HERRENS tjänare Mose hade bjudit.
\par 13 Dock brände Israel icke upp någon av de städer som lågo på höjder, utom Hasor allena, ty det uppbrändes av Josua.
\par 14 Och allt rovet från dessa städer, så ock boskapen, togo Israels barn såsom sitt byte; men alla människor i dem slogo de med svärdsegg, till dess att de hade förgjort dem; de läto intet som anda hade bliva kvar.
\par 15 Såsom HERREN hade bjudit sin tjänare Mose, så hade Mose bjudit Josua, och så gjorde Josua; han underlät icke något av allt det som HERREN hade bjudit Mose.
\par 16 Så intog Josua hela detta land: Bergsbygden, hela Sydlandet och hela landet Gosen, Låglandet och Hedmarken, så ock Israels bergsbygd och dess lågland,
\par 17 landet från Halakberget, som höjer sig mot Seir, ända till Baal-Gad i Libanonsdalen nedanför berget Hermon; och alla konungar där tog han till fånga och slog dem till döds.
\par 18 I lång tid förde Josua krig mot alla dessa konungar.
\par 19 Om man undantager de hivéer som bodde i Gibeon, fanns ingen stad som ingick fred med Israels barn, utan dessa intogo dem alla med strid.
\par 20 Ty från HERREN kom det att de förstockade sina hjärtan och mötte Israel med krig, för att de skulle givas till spillo, och för att nåd icke skulle vederfaras dem; i stället skulle de förgöras, såsom HERREN hade bjudit Mose.
\par 21 Under denna tid drog Josua åstad och utrotade anakiterna i Bergsbygden, i Hebron, Debir och Anab, i hela Juda bergsbygd och i hela Israels bergsbygd; Josua gav dem med deras städer till spillo.
\par 22 I Israels barns land lämnades inga anakiter kvar; allenast i Gasa, Gat och Asdod blevo några kvar.
\par 23 Så intog Josua hela landet, alldeles såsom HERREN hade lovat Mose; och Josua gav det till arvedel åt Israel, efter deras avdelningar och stammar. Och landet hade nu ro från krig.

\chapter{12}

\par 1 Dessa voro de konungar i landet, som Israels barn slogo, och vilkas land de togo i besittning på andra sidan Jordan, på östra sidan, landet från bäcken Arnon ända till berget Hermon, så ock hela Hedmarken på östra sidan:
\par 2 Sihon, amoréernas konung, som bodde i Hesbon och rådde över landet Aroer vid bäcken Arnons strand och från dalens mitt, samt över ena hälften av Gilead ända till bäcken Jabbok, som är Ammons barns gräns,
\par 3 ävensom över Hedmarken ända upp till Kinarotsjön, på östra sidan, och ända ned till Hedmarkshavet, Salthavet, på östra sidan, åt Bet-Hajesimot till, och längre söderut till trakten nedanför Pisgas sluttningar.
\par 4 Vidare intogo de Ogs område, konungens i Basan, vilken var en av de sista rafaéerna och bodde i Astarot och Edrei.
\par 5 Han rådde över Hermons bergsbygd och över Salka och hela Basan ända till gesuréernas och maakatéernas område, så ock över andra hälften av Gilead, till Sihons område, konungens i Hesbon.
\par 6 HERRENS tjänare Mose och Israels barn hade slagit dessa; och HERRENS tjänare Mose hade givit landet till besittning åt rubeniterna, gaditerna och ena hälften av Manasse stam.
\par 7 Och följande voro de konungar i landet, som Josua och Israels barn slogo på andra sidan Jordan, på västra sidan, från Baal-Gad i Libanonsdalen ända till Halakberget, som höjer sig mot Seir. (Josua gav sedan landet till besittning åt Israels stammar, efter deras avdelningar,
\par 8 såväl Bergsbygden, Låglandet, Hedmarken och Bergssluttningarna som ock Öknen och Sydlandet, hetiternas, amoréernas, kananéernas, perisséernas, hivéernas och jebuséernas land.)
\par 9 De voro: konungen i Jeriko en, konungen i Ai, som ligger bredvid Betel, en,
\par 10 konungen i Jerusalem en, konungen i Hebron en,
\par 11 konungen i Jarmut en, konungen i Lakis en,
\par 12 konungen i Eglon en, konungen i Geser en,
\par 13 konungen i Debir en, konungen i Geder en,
\par 14 konungen i Horma en, konungen i Arad en,
\par 15 konungen i Libna en, konungen i Adullam en,
\par 16 konungen i Mackeda en, konungen i Betel en,
\par 17 konungen i Tappua en, konungen i Hefer en,
\par 18 konungen i Afek en, konungen i Lassaron en,
\par 19 konungen i Madon en, konungen i Hasor en,
\par 20 konungen i Simron-Meron en, konungen i Aksaf en,
\par 21 konungen i Taanak en, konungen i Megiddo en,
\par 22 konungen i Kedes en, konungen i Jokneam vid Karmel en,
\par 23 konungen över Dor i Nafat-Dor en, konungen över Goim vid Gilgal en,
\par 24 konungen i Tirsa en - tillsammans trettioen konungar.

\chapter{13}

\par 1 Då nu Josua var gammal och kommen till hög ålder, sade HERREN till honom: "Du är gammal och kommen till hög ålder, men ännu återstår av landet en mycket stor del som skall intagas.
\par 2 Detta är nämligen vad som återstår av landet: alla filistéernas kretsar och hela gesuréernas land.
\par 3 Ty allt som finnes mellan Sihor, öster om Egypten, och Ekrons område norrut räknas till Kananéernas land, nämligen vad filistéernas fem hövdingar innehava - den i Gasa, den i Asdod, den i Askelon, den i Gat och den i Ekron - så ock avéernas område,
\par 4 hela kananéernas land söderut, vidare Meara, som tillhör sidonierna, ända till Afek, ända till amoréernas område.
\par 5 och gebaléernas land samt hela Libanonstrakten österut, från Baal-Gad, nedanför berget Hermon, ända dit där vägen går till Hamat -
\par 6 alla inbyggarna i bergsbygden, från Libanon ända till Misrefot-Maim, alla sidonier: dessa skall jag själv fördriva för Israels barn. Men fördela du genom lottkastning landet åt Israel till arvedel, såsom jag har bjudit dig.
\par 7 Ja, redan nu må du utskifta detta land till arvedel åt de nio stammarna och åt ena hälften av Manasse stam."
\par 8 Jämte Manasse hade ock rubeniterna och gaditerna fått sin arvedel, den som Mose gav dem på andra sidan Jordan, på östra sidan, just såsom HERRENS tjänare Mose gav den åt dem:
\par 9 landet från Aroer, vid bäcken Arnons strand, och från staden i dalens mitt, så ock hela Medebaslätten ända till Dibon,
\par 10 jämte alla övriga städer som hade tillhört Sihon, amoréernas konung, vilken regerade i Hesbon, ända till Ammons barns område,
\par 11 vidare Gilead och gesuréernas och maakatéernas område och hela Hermons bergsbygd och hela Basan ända till Salka,
\par 12 hela Ogs rike i Basan, hans som regerade i Astarot och Edrei, och som levde kvar såsom en av de sista rafaéerna, sedan Mose hade slagit och fördrivit dem.
\par 13 Dock fördrevo Israels barn icke gesuréerna och maakatéerna; därför bodde ock gesuréer och maakatéer kvar bland Israels folk, såsom de göra ännu i dag.
\par 14 (Men åt Levi stam gav han icke någon arvedel. HERRENS, Israels Guds, eldsoffer äro hans arvedel, såsom han har sagt honom.)
\par 15 Mose gav alltså land åt Rubens barns stam, efter deras släkter.
\par 16 De fingo området från Aroer, vid bäcken Arnons strand, och från staden i dalens mitt, så ock hela slätten vid Medeba,
\par 17 Hesbon med alla dess lydstäder på slätten, Dibon, Bamot-Baal, Bet-Baal-Meon,
\par 18 Jahas, Kedemot, Mefaat,
\par 19 Kirjataim, Sibma, Seret-Hassahar på Dalberget,
\par 20 Bet-Peor samt Pisgas sluttningar och Bet-Hajesimot,
\par 21 alla städerna på slätten, hela Sihons rike, amoréernas konungs, hans som regerade i Hesbon, och som hade blivit slagen av Mose jämte de midjanitiska hövdingarna Evi, Rekem, Sur, Hur och Reba, Sihons lydfurstar, som bodde där i landet.
\par 22 Bileam, Beors son, spåmannen, dräptes ock av Israels barn med svärd, jämte andra som då blevo slagna av dem.
\par 23 Och gränsen för Rubens barn var Jordan; den utgjorde gränsen. Detta är Rubens barns arvedel, efter deras släkter, städerna med sina byar.
\par 24 Likaledes gav Mose land åt Gads stam, åt Gads barn, efter deras släkter.
\par 25 De fingo till sitt område Jaeser och alla städer i Gilead och hälften av Ammons barns land, ända till det Aroer som ligger gent emot Rabba,
\par 26 vidare landet från Hesbon ända till Ramat-Hammispe och Betonim, och från Mahanaim ända till Lidebirs område,
\par 27 samt i dalen: Bet-Haram, Bet-Nimra, Suckot och Safon, det övriga av Sihons rike, konungens i Hesbon, intill Jordan, som utgjorde gränsen, upp till ändan av Kinneretsjön, landet på andra sidan Jordan, på östra sidan.
\par 28 Detta är Gads barns arvedel, efter deras släkter, städerna med sina byar.
\par 29 Och Mose gav också land åt ena hälften av Manasse stam, så att denna hälft av Manasse barns stam fick land, efter sina släkter.
\par 30 Deras område utgjordes av landet från Mahanaim, av hela Basan, hela Ogs rike, konungens i Basan, med alla Jairs byar i Basan, sextio städer,
\par 31 alltså ock av halva Gilead jämte Astarot och Edrei, Ogs huvudstäder i Basan; detta gavs åt Makirs, Manasses sons, barn, nämligen åt ena hälften av Makirs barn, efter deras släkter.
\par 32 Dessa voro de arvslotter som Mose utskiftade på Moabs hedar, på andra sidan Jordan mitt emot Jeriko, på östra sidan.
\par 33 Men åt Levi stam gav Mose icke någon arvedel. HERREN, Israels Gud, är deras arvedel, såsom han har sagt dem.

\chapter{14}

\par 1 Och dessa äro de arvslotter som Israels barn fingo i Kanaans land, de som prästen Eleasar och Josua, Nuns son, och huvudmännen för familjerna inom Israels barns stammar utskiftade åt dem,
\par 2 nämligen genom lottkastning om vars och ens arvedel, såsom HERREN hade bjudit genom Mose angående de nio stammarna och den ena halva stammen.
\par 3 Ty de två övriga stammarna och den andra halva stammen hade av Mose fått sin arvedel på andra sidan Jordan, men leviterna hade han icke givit någon arvedel bland dem.
\par 4 Ty Josefs barn utgjorde två stammar, Manasse och Efraim; och åt leviterna gav man icke någon särskild del av landet, utan allenast några städer att bo i, med tillhörande utmarker för deras boskap och deras övriga egendom.
\par 5 Såsom HERREN hade bjudit Mose, så gjorde Israels barn, när de utskiftade landet
\par 6 Men Juda barn trädde fram inför Josua i Gilgal, och kenaséen Kaleb, Jefunnes son, sade till honom: "Du vet själv vad HERREN sade till gudsmannen Mose angående mig och dig i Kades-Barnea.
\par 7 Jag var fyrtio år gammal, när HERRENS tjänare Mose sände mig åstad från Kades-Barnea för att bespeja landet, och jag avgav sedan min berättelse därom inför honom efter bästa förstånd.
\par 8 Mina bröder, som hade varit däruppe med mig, gjorde folkets hjärtan försagda, men jag efterföljde i allt HERREN, min Gud.
\par 9 Då betygade Mose på den dagen med ed och sade: 'Sannerligen, det land som din fot har beträtt skall vara din och dina barns arvedel för evärdlig tid, därför att du i allt har efterföljt HERREN, min Gud.'
\par 10 Och se, nu har HERREN låtit mig leva, såsom han lovade, i ytterligare fyrtiofem år, sedan HERREN talade så till Mose - de år Israel vandrade i öknen; se, jag är nu åttiofem år gammal.
\par 11 Ännu i dag är jag lika stark som jag var den dag då Mose sände mig åstad, ja, sådan min kraft då var, sådan är den ännu, vare sig det gäller att strida eller att vara ledare och anförare.
\par 12 Så giv mig nu denna bergsbygd om vilken HERREN talade på den dagen. Du hörde ju själv då att anakiterna bo där, och att där finnas stora befästa städer; måhända är HERREN med mig, så att jag kan fördriva dem, såsom HERREN har lovat."
\par 13 Då välsignade Josua Kaleb, Jefunnes son, och gav honom Hebron till arvedel.
\par 14 Alltså fick då kenaséen Kaleb, Jefunnes son, Hebron till arvedel, såsom det är ännu i dag, därför att han i allt hade efterföljt HERREN, Israels Gud.
\par 15 Men Hebron hette fordom Kirjat-Arba efter den störste mannen bland anakiterna. Och landet hade nu ro från krig.

\chapter{15}

\par 1 Juda barns stam fick, efter sina släkter, sin lott söderut intill Edoms gräns, intill öknen Sin, längst ned i söder.
\par 2 Och deras södra gräns begynte vid ändan av Salthavet, vid dess sydligaste vik,
\par 3 gick vidare söder om Skorpionhöjden och fram till Sin, drog sig så upp söder om Kades-Barnea, gick därefter framom Hesron och drog sig upp till Addar samt böjde sig sedan mot Karka.
\par 4 Vidare gick den fram till Asmon och därifrån ut till Egyptens bäck; sedan gick gränsen ut vid havet. "Detta", sade han, "skall vara eder gräns i söder."
\par 5 Gränsen i öster var Salthavet ända till Jordans utlopp. Och gränsen på norra sidan begynte vid den vik av detta hav, där Jordan har sitt utlopp.
\par 6 Därifrån drog sig gränsen upp mot Bet-Hogla och gick fram norr om Bet-Haaraba; vidare drog sig gränsen upp till Bohans, Rubens sons, sten.
\par 7 Därefter drog sig gränsen upp till Debir från Akors dal i nordlig riktning mot det Gilgal som ligger mitt emot Adummimshöjden, söder om bäcken; sedan gick gränsen fram till Semeskällans vatten och så ut till Rogelskällan.
\par 8 Vidare drog sig gränsen uppåt Hinnoms sons dal, söder om Jebus' höjd, det är Jerusalem; därefter drog sig gränsen upp till toppen av det berg som ligger gent emot Hinnomsdalen, västerut, i norra ändan av Refaimsdalen.
\par 9 Och från toppen av detta berg drog sig gränsen fram till Neftoavattnets källa och vidare till städerna i Efrons bergsbygd; sedan drog sig gränsen till Baala, det är Kirjat-Jearim.
\par 10 Och från Baala böjde sig gränsen åt väster mot Seirs bergsbygd och gick fram till Jearims bergshöjd, det är Kesalon, norr om denna, och gick så ned till Bet-Semes och framom Timna.
\par 11 Vidare gick gränsen till Ekrons höjd, norrut; därefter drog sig gränsen till Sickeron, gick så framom berget Baala och därifrån ut till Jabneel; sedan gick gränsen ut vid havet.
\par 12 Och gränsen i väster följde Stora havet; det utgjorde gränsen. Dessa voro Juda barns gränser runt omkring, efter deras släkter.
\par 13 Men åt Kaleb, Jefunnes son, gavs, efter HERRENS befallning till Josua, en särskild del bland Juda barn, nämligen Arbas, Anaks faders, stad, det är Hebron.
\par 14 Och Kaleb fördrev därifrån Anaks tre söner, Sesai, Ahiman och Talmai, Anaks avkomlingar.
\par 15 Därifrån drog han upp mot Debirs invånare. Men Debir hette fordom Kirjat-Sefer.
\par 16 Och Kaleb sade: "Åt den som angriper Kirjat-Sefer och intager det vill jag giva min dotter Aksa till hustru."
\par 17 När då Otniel, son till Kenas, Kalebs broder, intog det, gav han honom sin dotter Aksa till hustru.
\par 18 Och när hon kom till honom, intalade hon honom att begära ett stycke åkermark av hennes fader; och hon steg hastigt ned från åsnan. Då sade Kaleb till henne: "Vad önskar du?"
\par 19 Hon sade: "Giv mig en avskedsskänk; eftersom du har gift bort mig till det torra Sydlandet, må du giva mig vattenkällor." Då gav han henne Illiotkällorna och Tatiotkällorna.
\par 20 Detta var nu Juda barns stams arvedel, efter deras släkter.
\par 21 Och de städer som lågo ytterst i Juda barns stam, mot Edoms gräns, i Sydlandet, voro: Kabseel, Eder, Jagur,
\par 22 Kina, Dimona, Adada,
\par 23 Kedes, Hasor och Jitnan,
\par 24 Sif, Telem, Bealot,
\par 25 Hasor-Hadatta, Keriot, Hesron, det är Hasor,
\par 26 Amam, Sema, Molada,
\par 27 Hasar-Gadda, Hesmon, Bet-Pelet,
\par 28 Hasar-Sual, Beer-Seba och Bisjotja,
\par 29 Baala, Ijim, Esem,
\par 30 Eltolad, Kesil, Horma,
\par 31 Siklag, Madmanna, Sansanna,
\par 32 Lebaot, Silhim, Ain och Rimmon - tillsammans tjugunio städer med sina byar.
\par 33 I Låglandet: Estaol, Sorga, Asna,
\par 34 Sanoa och En-Gannim, Tappua och Enam,
\par 35 Jarmut och Adullam, Soko och Aseka,
\par 36 Saaraim, Aditaim, Gedera och Gederotaim - fjorton städer med sina byar;
\par 37 Senan, Hadasa, Migdal-Gad,
\par 38 Dilean, Mispe, Jokteel,
\par 39 Lakis, Boskat, Eglon,
\par 40 Kabbon, Lamas, Kitlis,
\par 41 Gederot, Bet-Dagon, Naama och Mackeda - sexton städer med sina byar;
\par 42 Libna, Eter, Asan,
\par 43 Jifta, Asna, Nesib,
\par 44 Kegila, Aksib och Maresa - nio städer med sina byar;
\par 45 Ekron med underlydande städer och byar;
\par 46 från Ekron till havet allt vad som ligger på sidan om Asdod samt dithörande byar;
\par 47 vidare Asdod med underlydande städer och byar, Gasa med underlydande städer och byar ända till Egyptens bäck och fram till Stora havet, som utgjorde gränsen.
\par 48 Och i Bergsbygden: Samir, Jattir, Soko,
\par 49 Danna, Kirjat-Sanna, det är Debir,
\par 50 Anab, Estemo, Anim,
\par 51 Gosen, Holon och Gilo - elva städer med sina byar;
\par 52 Arab, Ruma, Esean,
\par 53 Janum, Bet-Tappua, Afeka,
\par 54 Humta, Kirjat-Arba, det är Hebron, och Sior - nio städer med sina byar;
\par 55 Maon, Karmel, Sif, Juta,
\par 56 Jisreel, Jokdeam och Sanoa,
\par 57 Kain, Gibea och Timna - tio städer med sina byar;
\par 58 Halhul, Bet-Sur, Gedor,
\par 59 Maarat, Bet-Anot och Eltekon - sex städer med sina byar;
\par 60 Kirjat-Baal, det är Kirjat-Jearim, och Rabba - två städer med sina byar;
\par 61 I Öknen: Bet-Haaraba, Middin, Sekaka,
\par 62 Nibsan, Ir-Hammela och En-Gedi - sex städer med sina byar.
\par 63 Men jebuséerna, som bodde i Jerusalem, kunde Juda barn icke fördriva; därför bodde ock jebuséerna kvar bland Juda barn i Jerusalem, såsom de göra ännu i dag.

\chapter{16}

\par 1 Och lotten föll ut för Josefs barn sålunda: Landet från Jordan vid Jeriko till Jerikos vatten österut, öknen, som från Jeriko höjer sig uppåt Bergsbygden mot Betel.
\par 2 Och gränsen gick vidare från Betel till Lus och så fram till arkiternas område, mot Atarot.
\par 3 Därefter gick den västerut ned till jafletiternas område, ända till Nedre Bet-Horons område och till Geser; sedan gick den ut vid havet.
\par 4 Detta fingo nu Josefs barn, Manasse och Efraim, till arvedel.
\par 5 Efraims barn fingo, efter sina släkter, sina gränser sålunda: Gränsen för deras arvedel i öster gick från Atrot-Addar ända till Övre Bet-Horon.
\par 6 Sedan gick gränsen ut vid havet. I norr var Mikmetat gräns. Därifrån böjde sig gränsen österut till Taanat-Silo. Därefter gick den fram där i öster till Janoa.
\par 7 Från Janoa gick den ned till Atarot och Naara, träffade så Jeriko och gick ut vid Jordan.
\par 8 Från Tappua gick gränsen västerut till Kanabäcken och gick sedan ut vid havet. Detta var Efraims barns stams arvedel, efter deras släkter.
\par 9 Dit hörde ock de städer som avsöndrades åt Efraims barn inom Manasse barns arvedel, alla dessa städer med sina byar.
\par 10 Men de fördrevo icke kananéerna som bodde i Geser; därför bodde ock kananéerna kvar bland Efraims barn, såsom de göra ännu i dag, men de blevo arbetspliktiga tjänare under dem.

\chapter{17}

\par 1 Och Manasse stam fick sin lott sålunda, ty han var Josefs förstfödde: Makir, Manasses förstfödde, Gileads fader, fick Gilead och Basan, ty han var en stridsman.
\par 2 Manasses övriga barn fingo ock land, efter sina släkter: Abiesers barn, Heleks barn, Asriels barn, Sikems barn, Hefers barn, och Semidas barn. Dessa voro Manasses, Josefs sons, manliga avkomlingar, efter deras släkter.
\par 3 Men Selofhad, son till Hefer, son till Gilead, son till Makir, son till Manasse, hade inga söner, utan allenast döttrar; och hans döttrar hette Mahela, Noa, Hogla, Milka och Tirsa.
\par 4 Dessa trädde fram inför prästen Eleasar och Josua, Nuns son, och stamhövdingarna och sade: "HERREN bjöd Mose att giva oss en arvedel bland våra bröder." Då gav man dem, efter HERRENS befallning, en arvedel bland deras faders bröder.
\par 5 Alltså blevo de lotter som tillföllo Manasse tio - förutom Gileads land och Basan på andra sidan Jordan -
\par 6 eftersom Manasses döttrar fingo en arvedel bland hans söner. Men Gileads land hade Manasses övriga barn fått.
\par 7 Och Manasse fick sin gräns bestämd sålunda: Den gick från Aser till Mikmetat, som ligger gent emot Sikem; därefter gick gränsen åt höger, till En-Tappuas inbyggare.
\par 8 (Tappuas land tillföll nämligen Manasse, men själva Tappua, inemot Manasse gräns, tillföll Efraims barn.)
\par 9 Och gränsen gick vidare ned till Kanabäcken, söder om bäcken; men städerna där tillföllo Efraim, fastän de lågo bland Manasse städer. Manasse gräns gick vidare norr om bäcken och gick sedan ut vid havet.
\par 10 Det som låg söder om den tillföll Efraim, men det som låg norr om den tillföll Manasse, och deras gräns var havet; och i norr nådde de till Aser och i öster till Isaskar.
\par 11 Och inom Isaskar och Aser fick Manasse Bet-Sean med underlydande orter, Jibleam med underlydande orter, invånarna i Dor och underlydande orter, invånarna i En-Dor och underlydande orter, invånarna i Taanak och underlydande orter, invånarna i Megiddo och underlydande orter, de tre höjdernas land.
\par 12 Men Manasse barn kunde icke intaga dessa städer, utan kananéerna förmådde hålla sig kvar där i landet.
\par 13 När sedan Israels barn blevo de starkare, gjorde de kananéerna arbetspliktiga under sig; de fördrevo dem icke heller då.
\par 14 Och Josefs barn talade till Josua och sade: "Varför har du givit oss till arvedel allenast en lott och ett skifte, fastän vi äro ett talrikt folk, då ju HERREN hitintills har välsignat oss?"
\par 15 Då svarade Josua dem: "Om du är ett för talrikt folk, så drag upp till skogsbygden och röj dig där mark i perisséernas och rafaéernas land, eftersom Efraims bergsbygd är dig för trång."
\par 16 Men Josefs barn sade: "I bergsbygden finnes icke rum nog för oss; och de kananéer som bo i dalbygden hava allasammans stridsvagnar av järn, både de som bo i Bet-Sean och underlydande orter och de som bo i Jisreels dal."
\par 17 Josua sade till Josefs hus, till Efraim och Manasse: "Du är ett talrikt folk och har stor kraft, därför skall du icke hava allenast en lott;
\par 18 utan du skall få en bergsbygd, som ju ock är en skogsbygd, men som du skall röja upp, så att till och med utkanterna därav skola tillhöra dig. Ty du måste fördriva kananéerna, eftersom de hava stridsvagnar av järn och äro så starka."

\chapter{18}

\par 1 Och Israels barns hela menighet församlade sig i Silo och uppsatte där uppenbarelsetältet, då nu landet var dem underdånigt.
\par 2 Men ännu återstodo av Israels barn sju stammar som icke hade fått sin arvedel sig tillskiftad.
\par 3 Därför sade Josua till Israels barn: "Huru länge viljen I försumma att gå åstad och taga i besittning det land som HERREN, edra fäders Gud, har givit eder?
\par 4 Utsen åt eder tre män för var stam, så skall jag sända dem åstad, för att de må stå upp och draga omkring i landet och sätta upp en beskrivning däröver, efter som vars och ens arvedel skall bliva, och så komma tillbaka till mig.
\par 5 De skola nämligen uppdela det åt sig i sju delar, varvid Juda skall förbliva vid sitt område i söder och Josefs hus förbliva vid sitt område i norr.
\par 6 Och sedan skolen I sätta upp beskrivningen över landet, efter dessa sju delar, och bära den hit till mig, så vill jag kasta lott för eder här inför HERREN, vår Gud.
\par 7 Ty leviterna få ingen särskild del bland eder, utan HERRENS prästadöme är deras arvedel; och Gad och Ruben och ena hälften av Manasse stam hava redan fått sin arvedel på andra sidan Jordan, på östra sidan, den arvedel som HERRENS tjänare Mose gav dem."
\par 8 Och männen stodo upp och gingo åstad; och när de gingo åstad bjöd Josua dem att de skulle sätta upp en beskrivning över landet, i det han sade: "Gån åstad och dragen omkring i landet och sätten upp en beskrivning däröver, och vänden sedan tillbaka till mig, så vill jag kasta lott för eder här inför HERREN i Silo."
\par 9 Så gingo då männen åstad och drogo genom landet och satte upp en beskrivning över det, efter dess sju delar, med dess städer, och kommo så tillbaka till Josua i lägret vid Silo.
\par 10 Sedan kastade Josua lott för dem i Silo inför HERREN, och Josua utskiftade där landet åt Israels barn, efter deras avdelningar.
\par 11 Då nu lotten drogs för Benjamins barns stam, efter deras släkter, föll den ut så, att det område som lotten gav dem låg mellan Juda barns och Josefs barns områden.
\par 12 Deras gräns på norra sidan begynte vid Jordan, och gränsen drog sig så upp mot Jerikos höjd i norr och uppåt bergsbygden västerut och gick så ut i öknen vid Bet-Aven.
\par 13 Därifrån gick gränsen fram till Lus, till höjden söder om Lus, det är Betel; sedan gick gränsen ned till Atrot-Addar över berget söder om Nedre Bet-Horon.
\par 14 Och gränsen drog sig vidare framåt och böjde sig på västra sidan söderut från berget som ligger gent emot Bet-Horon, söder därom, och gick så ut till Kirjat-Baal, det är staden Kirjat-Jearim inom Juda barns område. Detta var västra sidan.
\par 15 Och södra sidan begynte vid ändan av Kirjat-Jearims område, och gränsen gick så åt väster fram till Neftoavattnets källa.
\par 16 Sedan gick gränsen ned till ändan av det berg som ligger gent emot Hinnoms sons dal, norrut i Refaimsdalen, och därefter ned i Hinnomsdalen, på södra sidan om Jebus' höjd, och gick så ned till Rogelskällan.
\par 17 Därefter drog den sig norrut och gick fram till Semeskällan och vidare till Gelilot, som ligger mitt emot Adummimshöjden, och gick så ned till Bohans, Rubens sons, sten.
\par 18 Vidare gick den fram till den höjd som ligger framför Hedmarken, norrut, och så ned till Hedmarken.
\par 19 Sedan gick gränsen fram till Bet-Hoglas höjd, norrut, och så gick gränsen ut till Salthavets norra vik, vid Jordans södra ända. Detta var södra gränsen.
\par 20 Men på östra sidan var Jordan gränsen. Detta var Benjamins barns arvedel med dess gränser runt omkring, efter deras släkter.
\par 21 Och de städer som tillföllo Benjamins barns stam, efter deras släkter, voro: Jeriko, Bet-Hogla, Emek-Kesis,
\par 22 Bet-Haaraba, Semaraim, Betel,
\par 23 Avim, Para, Ofra,
\par 24 Kefar-Haammoni, Ofni och Geba - tolv städer med sina byar;
\par 25 Gibeon, Rama, Beerot,
\par 26 Mispe, Kefira, Mosa,
\par 27 Rekem, Jirpeel, Tarala,
\par 28 Sela, Elef, Jebus, det är Jerusalem, Gibeat och Kirjat - fjorton städer med sina byar. Detta var nu Benjamins barns arvedel, efter deras släkter.

\chapter{19}

\par 1 Och den andra lotten föll ut för Simeon, för Simeons barns stam, efter deras släkter; och de fingo sin arvedel inom Juda barns arvedel.
\par 2 De fingo inom dessas arvedel Beer-Seba, Seba, Molada,
\par 3 Hasar-Sual, Bala, Esem,
\par 4 Eltolad, Betul, Horma,
\par 5 Siklag, Bet-Hammarkabot, Hasar-Susa,
\par 6 Bet-Lebaot och Saruhen - tretton städer med deras byar;
\par 7 Ain, Rimmon, Eter och Asan - fyra städer med deras byar;
\par 8 därtill alla de byar som lågo runt omkring dessa städer, ända till Baalat-Beer, det sydliga Rama. Detta var Simeons barns arvedel, efter deras släkter.
\par 9 Ur Juda barns skifte fingo Simeons barn sin arvedel, ty Juda barns lott var för stor för dem; därför fingo Simeons barn sin arvedel inom deras arvedel.
\par 10 Den tredje lotten drogs ut för Sebulons barn, efter deras släkter; och gränsen för deras arvedel gick ända till Sarid.
\par 11 Därifrån drog sig deras gräns västerut uppåt till Mareala och träffade Dabbeset och träffade vidare dalen som ligger gent emot Jokneam.
\par 12 På andra sidan från Sarid, österut mot solens uppgång, vände den sig åt Kislot-Tabors område och gick vidare till Dobrat och upp till Jafia.
\par 13 Därifrån gick den fram österut mot solens uppgång till Gat-Hefer och Et-Kasin och vidare till det Rimmon som sträcker sig till Nea.
\par 14 Härförbi böjde sig gränsen i norr till Hannaton och gick så ut vid Jifta-Els dal.
\par 15 Och den omfattade Kattat, Nahalal, Simron, Jidala och Bet-Lehem - tolv städer med deras byar.
\par 16 Detta var Sebulons barns arvedel, efter deras släkter, de nämnda städerna med sina byar.
\par 17 För Isaskar föll den fjärde lotten ut, för Isaskars barn, efter deras släkter.
\par 18 Och deras gräns omfattade Jisreel, Kesullot, Sunem,
\par 19 Hafaraim, Sion, Anaharat,
\par 20 Rabbit, Kisjon, Ebes,
\par 21 Remet, En-Gannim, En-Hadda och Bet-Passes;
\par 22 och gränsen träffade Tabor, Sahasuma och Bet-Semes; och deras gräns gick ut vid Jordan - sexton städer med deras byar.
\par 23 Detta var Isaskars barns stams arvedel, efter deras släkter, städerna med sina byar.
\par 24 Den femte lotten föll ut för Asers barns stam, efter deras släkter.
\par 25 Och deras gräns omfattade Helkat, Hali, Beten, Aksaf,
\par 26 Alammelek, Amead och Miseal; och vid havet träffade den Karmel och Sihor-Libnat.
\par 27 Därefter vände den sig åt öster till Bet-Dagon och träffade Sebulon och Jifta-Els dal i norr, vidare Bet-Haemek och Negiel och gick så ut till Kabul i norr.
\par 28 Och den omfattade Ebron, Rehob, Hammon och Kana, ända upp till Stora Sidon.
\par 29 Och gränsen vände sig till Rama och gick fram till den befästa staden Tyrus; sedan vände sig gränsen till Hosa och gick så ut vid havet där landsträckan vid Aksib begynner.
\par 30 Och den omfattade Umma, Afek och Rehob - tjugutvå städer med deras byar.
\par 31 Detta var Asers barns stams arvedel, efter deras släkter, de nämnda städerna med sina byar.
\par 32 För Naftali barn föll den sjätte lotten ut, för Naftali barn, efter deras släkter.
\par 33 Och deras gräns gick från Helef, från terebinten i Saanannim till Adami-Hannekeb och Jabneel, ända till Lackum, och gick så ut vid Jordan.
\par 34 Och gränsen vände sig västerut till Asnot-Tabor och gick vidare därifrån till Huckok; den träffade Sebulon i söder, och Aser träffade den i väster och Juda med Jordan i öster.
\par 35 Och den omfattade de befästa städerna Siddim, Ser och Hammat, Rackat och Kinneret,
\par 36 Adama, Rama, Hasor,
\par 37 Kedes, Edrei, En-Hasor,
\par 38 Jireon och Migdal-El, Horem, Bet-Anat och Bet-Semes - nitton städer med deras byar.
\par 39 Detta var Naftali barns stams arvedel efter deras släkter, städerna med sina byar.
\par 40 För Dans barns stam, efter deras släkter, föll den sjunde lotten ut.
\par 41 Och gränsen för deras arvedel omfattade Sorga, Estaol, Ir-Semes,
\par 42 Saalabbin, Ajalon, Jitla,
\par 43 Elon, Timna, Ekron,
\par 44 Elteke, Gibbeton, Baalat,
\par 45 Jehud, Bene-Berak, Gat-Rimmon,
\par 46 Me-Hajarkon och Harackon, tillika med området framför Jafo.
\par 47 (Men när sedan Dans barns område gick förlorat för dem, drogo Dans barn upp och belägrade Lesem och intogo det och slogo dess invånare med svärdsegg; och sedan de så hade tagit det i besittning, bosatte de sig där och kallade Lesem för Dan, efter Dans, sin faders, namn.)
\par 48 Detta var Dans barns stams arvedel, efter deras släkter, de nämnda städerna med sina byar.
\par 49 När Israels barn så hade utskiftat landet efter dess gränser, gåvo de åt Josua, Nuns son, en särskild arvedel ibland sig.
\par 50 Efter HERRENS befallning gåvo de honom nämligen den stad som han begärde, Timnat-Sera i Efraims bergsbygd; och han bebyggde staden och bosatte sig där.
\par 51 Dessa voro de arvslotter som prästen Eleasar och Josua, Nuns son, och huvudmännen för familjerna inom Israels barns stammar utskiftade genom lottkastning i Silo inför HERRENS ansikte, vid ingången till uppenbarelsetältet. Så avslutade de nu fördelningen av landet.

\chapter{20}

\par 1 Och HERREN talade till Josua och sade:
\par 2 "Tala till Israels barn och säg: Utsen åt eder de fristäder om vilka jag har talat till eder genom Mose,
\par 3 de städer till vilka en dråpare som ouppsåtligen, utan vett och vilja, har dödat någon må kunna fly; och I skolen hava dem såsom tillflyktsorter undan blodshämnaren.
\par 4 Och när någon flyr till en av dessa städer, skall han stanna vid ingången till stadsporten och omtala sin sak för de äldste i den staden; därefter må de taga honom in i staden till sig och giva honom en plats där han får bo ibland dem.
\par 5 Och om blodshämnaren förföljer honom, skola de icke överlämna dråparen i hans hand, eftersom han utan vett och vilja har dödat sin nästa, och utan att förut hava burit hat till honom.
\par 6 Och han skall stanna kvar där i staden, till dess han har stått till rätta inför menigheten, och till dess den dåvarande översteprästen har dött; sedan må dråparen vända tillbaka och komma till sin stad från vilken han har flytt."
\par 7 Så helgade de då därtill Kedes i Galileen, i Naftali bergsbygd, Sikem i Efraims bergsbygd och Kirjat-Arba, det är Hebron, i Juda bergsbygd.
\par 8 Och på andra sidan Jordan mitt emot Jeriko, på östra sidan, utsågo de därtill inom Rubens stam Beser i öknen på slätten, inom Gads stam Ramot i Gilead, och inom Manasse stam Golan i Basan.
\par 9 Dessa voro de städer som för alla Israels barn, och för de främlingar som bodde ibland dem, bestämdes att vara orter till vilka var och en som ouppsåtligen hade dödat någon finge fly, så att han skulle slippa dö för blodshämnarens hand, innan han hade stått till rätta inför menigheten.

\chapter{21}

\par 1 Och huvudmännen för leviternas familjer trädde fram inför prästen Eleasar och Josua, Nuns son, och huvudmännen för familjerna inom Israels barns stammar
\par 2 och talade till dem i Silo i Kanaans land, och sade: "HERREN bjöd genom Mose att man skulle giva oss städer att bo i, med tillhörande utmarker för vår boskap."
\par 3 Så gåvo då Israels barn, efter HERRES befallning, av sina arvslotter åt leviterna följande städer med tillhörande utmarker.
\par 4 För kehatiternas släkter föll lotten ut så, att bland dessa leviter prästen Arons söner genom lotten fingo ur Juda stam, ur simeoniternas stam och ur Benjamins stam tretton städer.
\par 5 Och Kehats övriga barn fingo genom lotten ur Efraims stams släkter, ur Dans stam och ur ena hälften av Manasse stam tio städer.
\par 6 Gersons barn åter fingo genom lotten ur Isaskars stams släkter, ur Asers stam, ur Naftali stam och ur andra hälften av Manasse stam, i Basan, tretton städer.
\par 7 Meraris barn fingo, efter sina släkter, ur Rubens stam, ur Gads stam och ur Sebulons stam tolv städer.
\par 8 Israels barn gåvo nu åt leviterna dessa städer med tillhörande utmarker, genom lottkastning, såsom HERREN hade bjudit genom Mose.
\par 9 Ur Juda barns stam och ur Simeons barns stam gav man följande här namngivna städer:
\par 10 Bland kehatiternas släkter bland Levi barn fingo Arons söner följande, ty dem träffade lotten först:
\par 11 Man gav dem Arbas, Anoks faders, stad, det är Hebron, i Juda bergsbygd, med dess utmarker runt omkring.
\par 12 Men åkerjorden och byarna som hörde till staden gav man till besittning åt Kaleb, Jefunnes son.
\par 13 Åt prästen Arons söner gav man alltså dråparfristaden Hebron med dess utmarker, vidare Libna med dess utmarker,
\par 14 Jattir med dess utmarker, Estemoa med dess utmarker,
\par 15 Holon med dess utmarker, Debir med dess utmarker,
\par 16 Ain med dess utmarker, Jutta med dess utmarker och Bet-Semes med dess utmarker - nio städer ur dessa två stammar;
\par 17 och ur Benjamins stam Gibeon med dess utmarker, Geba med dess utmarker,
\par 18 Anatot med dess utmarker och Almon med dess utmarker - fyra städer.
\par 19 De städer som Arons söner, prästerna, fingo utgjorde alltså tillsammans tretton städer, med tillhörande utmarker.
\par 20 Och Kehats barns släkter av leviterna, nämligen de övriga Kehats barn, fingo ur Efraims stam följande städer, som lotten bestämde åt dem:
\par 21 Man gav dem dråparfristaden Sikem med dess utmarker i Efraims bergsbygd, Geser med dess utmarker,
\par 22 Kibsaim med dess utmarker och Bet-Horon med dess utmarker - fyra städer;
\par 23 och ur Dans stam Elteke med dess utmarker, Gibbeton med dess utmarker,
\par 24 Ajalon med dess utmarker och Gat-Rimmon med dess utmarker - fyra städer;
\par 25 och ur ena hälften av Manasse stam Taanak med dess utmarker och Gat-Rimmon med dess utmarker - två städer.
\par 26 De städer som de övriga Kehats barns släkter fingo utgjorde alltså tillsammans tio, med tillhörande utmarker.
\par 27 Bland leviternas släkter fingo vidare Gersons barn ur ena hälften av Manasse stam dråparfristaden Golan i Basan med dess utmarker och Beestera med dess utmarker - två städer;
\par 28 och ur Isaskars stam Kisjon med dess utmarker, Dobrat med dess utmarker,
\par 29 Jarmut med dess utmarker och En-Gannim med dess utmarker - fyra städer;
\par 30 och ur Asers stam Miseal med dess utmarker, Abdon med dess utmarker,
\par 31 Helkat med dess utmarker och Rehob med dess utmarker - fyra städer;
\par 32 och ur Naftali stam dråparfristaden Kedes i Galileen med dess utmarker, Hammot-Dor med dess utmarker och Kartan med dess utmarker - tre städer.
\par 33 Gersoniternas städer, efter deras släkter, utgjorde alltså tillsammans tretton städer, med tillhörande utmarker.
\par 34 Och de övriga leviterna, Meraris barns släkter, fingo ur Sebulons stam Jokneam med dess utmarker, Karta med dess utmarker,
\par 35 Dimna med dess utmarker och Nahalal med dess utmarker - fyra städer;
\par 36 ur Rubens stam Beser i öknen med dess utmarker, Jahas med dess utmarker,
\par 37 Kedemot med dess utmarker och Mefaat med dess utmarker;
\par 38 och ur Gads stam dråparfristaden Ramot i Gilead med dess utmarker, Mahanaim med dess utmarker,
\par 39 Hesbon med dess utmarker och Jaeser med dess utmarker - tillsammans fyra städer.
\par 40 De städer som dessa de övriga leviternas släkter, Meraris barn, fingo på sin lott, efter sina släkter, utgjorde alltså tillsammans tolv städer.
\par 41 Tillsammans utgjorde levitstäderna inom Israels barns besittningsområde fyrtioåtta städer med tillhörande utmarker.
\par 42 Var och en av dessa städer skulle bestå av själva staden och tillhörande utmarker runt omkring. Så var det med alla dessa städer.
\par 43 Så gav då HERREN åt Israel hela det land som han med ed hade lovat giva åt deras fäder; och de togo det i besittning och bosatte sig där.
\par 44 Och HERREN lät dem hava ro på alla sidor, alldeles såsom han med ed hade lovat deras fäder; och ingen av deras fiender kunde stå dem emot, utan HERREN gav alla deras fiender i deras hand.
\par 45 Intet uteblev av allt det goda som HERREN hade lovat Israels hus; det gick allt i fullbordan.

\chapter{22}

\par 1 Då kallade Josua till sig rubeniterna och gaditerna och ena hälften av Manasse stam
\par 2 och sade till dem: "I haven hållit allt vad HERRENS tjänare Mose har bjudit eder; I haven ock lyssnat till mina ord, vadhelst jag har befallt eder.
\par 3 I haven under denna långa tid, ända till denna dag, icke övergivit edra bröder, och I haven hållit vad HERRENS, eder Guds, bud har befallt eder hålla.
\par 4 Och nu har HERREN, eder Gud, låtit edra bröder komma till ro, såsom han lovade dem; så vänden nu om och gån hem till edra hyddor i det land I haven fått till besittning, det som HERRENS tjänare Mose har givit eder på andra sidan Jordan.
\par 5 Allenast mån I noga hålla och göra efter de bud och den lag som HERRENS tjänare Mose har givit eder, så att I älsken HERREN, eder Gud, och alltid vandren på hans vägar och iakttagen hans bud och hållen eder till honom och tjänen honom av allt edert hjärta och av all eder själ."
\par 6 Och Josua välsignade dem och lät dem gå, och så gingo de hem till sina hyddor.
\par 7 Ty åt ena hälften av Manasse stam hade Mose givit land i Basan, och åt andra hälften hade Josua givit land jämte deras bröder på andra sidan Jordan, på västra sidan. Då nu Josua lät dem gå hem till sina hyddor, välsignade han dem
\par 8 och sade till dem: "Vänden tillbaka till edra hyddor med de stora skatter I haven fått, med boskap i stor myckenhet, med silver, guld, koppar och järn och kläder i stor myckenhet; skiften så med edra bröder bytet från edra fiender."
\par 9 Så vände då Rubens barn och Gads barn och ena hälften av Manasse stam tillbaka, och gingo bort ifrån de övriga israeliterna, bort ifrån Silo i Kanaans land, för att begiva sig till Gileads land, det land de hade fått till besittning, och där de skulle hava sina besittningar, efter HERRENS befallning genom Mose.
\par 10 När så Rubens barn och Gads barn och ena hälften av Manasse stam kommo till stenkretsarna vid Jordan i Kanaans land, byggde de där ett altare vid Jordan, ett ansenligt altare.
\par 11 Och de övriga israeliterna fingo höra sägas: "Se, Rubens barn och Gads barn och ena hälften av Manasse stam hava byggt ett altare mitt emot Kanaans land, i stenkretsarna vid Jordan, på andra sidan om de övriga israeliternas område."
\par 12 När Israels barn hörde detta, församlade sig deras hela menighet i Silo för att draga upp till strid mot dem.
\par 13 Därefter sände Israels barn Pinehas till Rubens barn och Gads barn och ena hälften av Manasse stam, i Gileads land, Pinehas, prästen Eleasars son,
\par 14 och med honom tio hövdingar, en hövding för var stamfamilj inom Israels alla stammar; var och en av dem var huvudman för sin familj inom Israels ätter.
\par 15 Och när dessa kommo till Rubens barn och Gads barn och ena hälften av Manasse stam, i Gileads land, talade de till dem och sade:
\par 16 "Så säger hela HERRENS menighet: Vad är detta för en otrohet som I haven begått mot Israels Gud, då I haven vänt eder bort ifrån HERREN, därigenom att I haven byggt eder ett altare och sålunda nu satt eder upp mot HERREN?
\par 17 Är det icke nog att vi hava begått missgärningen med Peor, från vilken vi ännu i dag icke hava blivit renade, och för vilken en hemsökelse drabbade HERRENS menighet?
\par 18 Viljen I nu ytterligare vända eder bort ifrån HERREN? Om I i dag sätten eder upp mot HERREN, så skall förvisso i morgon hans förtörnelse drabba Israels hela menighet.
\par 19 Men om det land I haven fått till besittning tyckes eder vara orent, så dragen över till det land HERREN har tagit till besittning, där HERRENS tabernakel har sin plats, och haven edra besittningar där bland oss. Sätten eder icke upp mot HERREN och sätten eder icke upp mot oss genom att bygga eder ett altare, ett annat än HERRENS, vår Guds, altare.
\par 20 När Akan, Seras son, hade trolöst förgripit sig på det tillspillogivna, kom icke då förtörnelse över Israels hela menighet, så att han själv icke blev den ende som förgicks genom den missgärningen?
\par 21 Då svarade Rubens barn och Gads barn och ena hälften av Manasse stam och talade till huvudmännen för Israels ätter:
\par 22 "Gud, HERREN Gud, ja, Gud, HERREN Gud, han vet det, och Israel må ock veta det: Sannerligen, om detta har skett i upproriskhet och otrohet mot HERREN - du må då i dag undandraga oss din hjälp! -
\par 23 om vi hava byggt altaret åt oss, därför att vi vilja vända oss bort ifrån HERREN, och om vi vilja offra därpå brännoffer eller spisoffer eller frambära tackoffer därpå, då må HERREN själv utkräva vad vi hava förskyllt.
\par 24 Nej, vi hava sannerligen gjort så av fruktan för vad som kunde hända, i det att vi tänkte att edra barn i framtiden skulle kunna säga till våra barn: 'Vad haven I att göra med HERREN, Israels Gud?
\par 25 HERREN har ju satt Jordan till gräns mellan oss och eder, I Rubens barn och Gads barn; alltså haven I ingen del i HERREN.' Och så skulle edra barn kunna hindra våra barn från att frukta HERREN.
\par 26 Därför sade vi: Må vi gripa oss an och bygga detta altare, men icke till brännoffer eller till slaktoffer,
\par 27 utan till att vara ett vittne mellan oss och eder, och mellan bådas efterkommande efter oss, att vi vilja förrätta HERRENS tjänst inför hans ansikte med våra brännoffer och slaktoffer och tackoffer, så att edra barn i framtiden icke kunna säga till våra barn: 'I haven ingen del i HERREN.'
\par 28 Och vi tänkte: Om det i framtiden händer att de så säga till oss och våra efterkommande, då kunna vi svara: 'Sen på den bild av HERRENS altare, som våra fäder hava gjort, men icke till brännoffer eller till slaktoffer, utan till att vara ett vittne mellan oss och eder.'
\par 29 Bort det, att vi skulle sätta oss upp mot HERREN och nu vända oss bort ifrån HERREN genom att bygga ett altare till brännoffer eller till spisoffer eller slaktoffer, ett annat än HERRENS, vår Guds, altare, som står framför hans tabernakel."
\par 30 Då nu prästen Pinehas och menighetens hövdingar, nämligen huvudmännen för Israels ätter, som voro med honom, hörde vad Rubens barn, Gads barn och Manasse barn talade, behagade det dem.
\par 31 Och Pinehas, prästen Eleasars son, sade till Rubens barn, Gads barn och Manasse barn: "Nu hava vi förnummit att HERREN är mitt ibland oss, därav nämligen, att I icke haven velat begå en sådan otrohet mot HERREN. Därmed haven I ock räddat Israels barn undan HERRENS hand."
\par 32 Därefter vände Pinehas, prästen Eleasars son, jämte hövdingarna tillbaka från Rubens barn och Gads barn, i Gileads land, in i Kanaans land till de övriga israeliterna och avgåvo sin berättelse härom inför dem.
\par 33 Denna behagade Israels barn, och Israels barn lovade Gud; och de tänkte icke mer på att draga upp till strid mot dem, för att fördärva det land där Rubens barn och Gads barn bodde.
\par 34 Och Rubens barn och Gads barn gåvo namn åt altaret; de sade: "Ett vittne är det mellan oss, att HERREN är Gud."

\chapter{23}

\par 1 En lång tid härefter, när HERREN hade låtit Israel få ro för alla dess fiender runt omkring, och då Josua var gammal och kommen till hög ålder,
\par 2 kallade han till sig hela Israel, dess äldste, dess huvudmän, dess domare och tillsyningsmän och sade till dem: "Jag är nu gammal och kommen till hög ålder.
\par 3 Och I haven själva sett allt vad HERREN, eder Gud, gjorde med alla dessa folk, när I drogen härin, ty HERREN eder Gud, stridde själv för eder.
\par 4 Se, dessa folk som ännu äro kvar har jag genom lottkastning fördelat åt eder till arvedel, efter edra stammar, allt ifrån Jordan intill Stora havet västerut, för att icke nämna alla de folk jag har utrotat.
\par 5 Och HERREN, eder Gud, skall själv driva dem undan för eder och förjaga dem för eder, så att I skolen taga deras land i besittning, såsom HERREN, eder Gud, har lovat eder.
\par 6 Men varen I nu ståndaktiga i att hålla och göra allt vad som är föreskrivet i Moses lagbok, så att I icke viken av därifrån vare sig till höger eller till vänster,
\par 7 och icke träden i gemenskap med dessa folk som ännu äro kvar här jämte eder, ej heller nämnen deras gudars namn eller svärjen vid dem eller tjänen och tillbedjen dem.
\par 8 Nej, till HERREN, eder Gud, skolen I hålla eder, såsom I haven gjort ända till denna dag,
\par 9 varför ock HERREN har fördrivit för eder stora och mäktiga folk, så att ingen har kunnat stå eder emot ända till denna dag.
\par 10 En enda man bland eder jagade tusen framför sig, ty HERREN, eder Gud, stridde själv för eder, såsom han hade lovat eder.
\par 11 Så haven nu noga akt på eder själva, så att I älsken HERREN, eder Gud.
\par 12 Ty om I vänden eder bort ifrån honom, och hållen eder till återstoden av dessa folk som ännu äro kvar här jämte eder, och befrynden eder med dem, så att I träden i gemenskap med dem och de med eder,
\par 13 då mån I förvisso veta att HERREN, eder Gud, icke mer skall fördriva dessa folk undan för eder, utan de skola bliva eder till en snara och ett giller och bliva ett gissel för edra sidor och taggar i edra ögon, till dess I bliven utrotade ur detta goda land, som HERREN, eder Gud, har givit eder.
\par 14 Se, jag går nu all världens väg. Så besinnen då av allt edert hjärta och av all eder själ att intet har uteblivit av allt det goda som HERREN, eder Gud, har lovat angående eder; det har allt gått i fullbordan för eder, intet därav har uteblivit.
\par 15 Men likasom allt det goda som HERREN, eder Gud, lovade eder har kommit över eder, så skall ock HERREN låta allt det onda komma över eder, till dess han har förgjort eder ur detta goda land, som HERREN, eder Gud, har givit eder.
\par 16 Om I nämligen överträden HERRENS, eder Guds, förbund, det som han har stadgat för eder, och gån åstad och tjänen andra gudar och tillbedjen dem, så skall HERRENS vrede upptändas mot eder, och I skolen med hast bliva utrotade ur det goda land som han har givit eder."

\chapter{24}

\par 1 Och Josua församlade alla Israels stammar till Sikem; och han kallade till sig de äldste i Israel, dess huvudmän, dess domare och dess tillsyningsmän, och de trädde fram inför Gud.
\par 2 Och Josua sade till allt folket: "Så säger HERREN, Israels Gud: På andra sidan floden bodde edra fäder i forna tider; så gjorde ock Tera, Abrahams och Nahors fader. Och de tjänade där andra gudar.
\par 3 Men jag hämtade eder fader Abraham från andra sidan floden och lät honom vandra omkring i hela Kanaans land. Och jag gjorde hans säd talrik; jag gav honom Isak,
\par 4 och åt Isak gav jag Jakob och Esau. Och jag gav Seirs bergsbygd till besittning åt Esau; men Jakob och hans söner drogo ned till Egypten.
\par 5 Sedan sände jag Mose och Aron och hemsökte Egypten med de gärningar jag där gjorde, och därefter förde jag eder ut.
\par 6 Och när jag förde edra fäder ut ur Egypten och I haden kommit till havet, förföljde egyptierna edra fäder med vagnar och ryttare ned i Röda havet.
\par 7 Då ropade de till HERREN, och han satte ett tjockt mörker mellan eder och egyptierna och lät havet komma över dem, så att det övertäckte dem; ja, I sågen med egna ögon vad jag gjorde med egyptierna. Sedan bodden I i öknen en lång tid.
\par 8 Därefter förde jag eder in i amoréernas land, vilka bodde på andra sidan Jordan, och de inläto sig i strid med eder; men jag gav dem i eder hand, så att I intogen deras land, och jag förgjorde dem för eder.
\par 9 Då uppreste sig Balak, Sippors son, konungen i Moab, och gav sig i strid med Israel. Och han sände och lät kalla till sig Bileam, Beors son, för att denne skulle förbanna eder.
\par 10 Men jag ville icke höra på Bileam, utan han måste välsigna eder, och jag räddade eder ur hans hand.
\par 11 Och när I haden gått över Jordan och kommit till Jeriko, gåvo sig Jerikos borgare i strid med eder, så och amoréerna, perisséerna, kananéerna, hetiterna och girgaséerna, hivéerna och jebuséerna; men jag gav dem i eder hand.
\par 12 Och jag sände getingar framför eder, och genom dessa förjagades amoréernas två konungar för eder, icke genom ditt svärd eller din båge.
\par 13 Och jag gav eder ett land varpå du icke hade nedlagt något arbete, så ock städer som I icke haden byggt, och i dem fingen I bo; och av vingårdar och olivplanteringar som I icke haden planterat fingen I äta.
\par 14 Så frukten nu HERREN och tjänen honom ostraffligt och troget; skaffen bort de gudar som edra fäder tjänade på andra sidan floden och i Egypten, och tjänen HERREN.
\par 15 Men om det misshagar eder att tjäna HERREN, så utväljen åt eder i dag vem I viljen tjäna, antingen de gudar som edra fäder tjänade, när de bodde på andra sidan floden, eller de gudar som dyrkas av amoréerna, i vilkas land I själva bon. Men jag och mitt hus, vi vilja tjäna HERREN."
\par 16 Då svarade folket och sade: "Bort det, att vi skulle övergiva HERREN och tjäna andra gudar!
\par 17 Nej, HERREN är vår Gud, han är den som har fört oss och våra fäder upp ur Egyptens land, ur träldomshuset, och som inför våra ögon har gjort dessa stora under, och bevarat oss på hela den väg vi hava vandrat och bland alla de folk genom vilkas land vi hava dragit fram.
\par 18 HERREN har förjagat för oss alla dessa folk, så ock amoréerna som bodde i landet. Därför vilja vi ock tjäna HERREN; ty an är vår Gud."
\par 19 Josua sade till folket: "I kunnen icke tjäna HERREN, ty han är en helig Gud; han är en nitälskande Gud, han skall icke hava fördrag med edra överträdelser och synder.
\par 20 Om I övergiven HERREN och tjänen främmande gudar, så skall han vända sig bort och låta det gå eder illa och förgöra eder, i stället för att han hittills har låtit det gå eder väl."
\par 21 Men folket sade till Josua: "Icke så, utan vi vilja tjäna HERREN."
\par 22 Då sade Josua till folket: "I ären nu själva vittnen mot eder, att I haven utvalt HERREN åt eder, för att tjäna honom." De svarade: "Ja."
\par 23 Han sade: "Så skaffen nu bort de främmande gudar som I haven bland eder, och böjen edra hjärtan till HERREN, Israels Gud."
\par 24 Folket svarade Josua: "HERREN, vår Gud, vilja vi tjäna, och hans röst vilja vi höra."
\par 25 Så slöt då Josua på den dagen ett förbund med folket och förelade dem lag och rätt i Sikem.
\par 26 Och Josua tecknade upp allt detta i Guds lagbok. Och han tog en stor sten och reste den där, under eken som stod vid HERRENS helgedom.
\par 27 Och Josua sade till allt folket: "Se, denna sten skall vara vittne mot oss, ty den har hört alla de ord som HERREN har talat med oss; den skall vara vittne mot eder, så att I icke förneken eder Gud."
\par 28 Sedan lät Josua folket gå, var och en till sin arvedel.
\par 29 En tid härefter dog HERRENS tjänare Josua, Nuns son, ett hundra tio år gammal.
\par 30 Och man begrov honom på hans arvedels område, i Timnat-Sera i Efraims bergsbygd, norr om berget Gaas.
\par 31 Och Israel tjänade HERREN, så länge Josua levde, och så länge de äldste levde, de som voro kvar efter Josua, och som visste av alla de gärningar HERREN hade gjort för Israel.
\par 32 Och Josefs ben, som Israels barn hade fört upp ur Egypten, begrovo de i Sikem, på det jordstycke som Jakob hade köpt av Hamors, Sikems faders, barn för hundra kesitor; Josefs barn fingo detta till arvedel.
\par 33 Och Eleasar, Arons son, dog, och man begrov honom i hans son Pinehas' stad, Gibea, som hade blivit denne given i Efraims bergsbygd.


\end{document}