\begin{document}

\title{Domarboken}


\chapter{1}

\par 1 Efter Josuas död frågade Israels barn HERREN och sade: "Vem bland oss skall först draga upp mot kananéerna och strida mot dem?"
\par 2 HERREN sade: "Juda skall göra det; se, jag har givit landet i hans hand."
\par 3 Då sade Juda till sin broder Simeon: "Drag upp med mig in i min arvslott, och låt oss strida mot kananéerna, så skall jag sedan tåga med dig in i din arvslott." Så tågade då Simeon med honom.
\par 4 När nu Juda drog ditupp, gav HERREN kananéerna och perisséerna i deras hand, så att de slogo dem vid Besek, tio tusen man.
\par 5 Ty vid Besek träffade de på Adoni-Besek och stridde mot honom och slogo så kananéerna och perisséerna.
\par 6 Och Adoni-Besek flydde, men de förföljde honom och grepo honom och höggo av honom hans tummar och stortår.
\par 7 Då sade Adoni-Besek: "Sjuttio konungar med avhuggna tummar och stortår hämtade upp smulorna under mitt bord; efter mina gärningar har Gud nu vedergällt mig." Sedan förde de honom till Jerusalem, och där dog han.
\par 8 Men Juda barn belägrade Jerusalem och intogo det och slogo dess invånare med svärdsegg; därefter satte de eld på staden.
\par 9 Sedan drogo Juda barn ned för att strida mot de kananéer som bodde i Bergsbygden, i Sydlandet och i Låglandet.
\par 10 Och Juda tågade åstad mot de kananéer som bodde i Hebron - vilket fordom hette Kirjat-Arba - och de slogo Sesai, Ahiman och Talmai.
\par 11 Därifrån tågade de åstad mot Debirs invånare. Men Debir hette fordom Kirjat-Sefer.
\par 12 Och Kaleb sade: "Åt den som angriper Kirjat-Sefer och intager det vill jag giva min dotter Aksa till hustru."
\par 13 När då Otniel, son till Kenas, Kalebs yngre broder, intog det, gav han honom sin dotter Aksa till hustru.
\par 14 Och när hon kom till honom, intalade hon honom att begära ett stycke åkermark av hennes fader; och hon steg hastigt ned från åsnan. Då sade Kaleb till henne: "Vad önskar du?"
\par 15 Hon sade till honom: "Låt mig få en avskedsskänk; eftersom du har gift bort mig till det torra Sydlandet, må du giva mig vattenkällor." Då gav Kaleb henne Illitkällorna och Tatitkällorna.
\par 16 Och kainéens, Moses svärfaders, barn hade dragit upp från Palmstaden med Juda barn till Juda öken, söder om Arad; de gingo åstad och bosatte sig bland folket där.
\par 17 Men Juda tågade åstad med sin broder Simeon, och de slogo de kananéer som bodde i Sefat; och de gåvo staden till spillo; så fick den namnet Horma.
\par 18 Därefter intog Juda Gasa med dess område, Askelon med dess område och Ekron med dess område.
\par 19 Och HERREN var med Juda, så att de intogo bergsbygden; men de kunde icke fördriva dem som bodde i dalbygden, därför att dessa hade stridsvagnar av järn.
\par 20 Och de gåvo Hebron åt Kaleb, såsom Mose hade föreskrivit; och han fördrev därifrån Anaks tre söner.
\par 21 Men jebuséerna, som bodde i Jerusalem, blevo icke fördrivna av Benjamins barn; därför bodde ock jebuséerna kvar bland Benjamins barn i Jerusalem, såsom de göra ännu dag.
\par 22 Så drogo ock männen av Josefs hus upp till Betel, och HERREN var med dem.
\par 23 Och männen av Josefs hus läto bespeja Betel, samma stad som fordom hette Lus.
\par 24 Då fingo deras kunskapare se en man gå ut ur staden, och de sade till honom: "Visa oss var vi kunna komma in i staden, så vilja vi sedan göra barmhärtighet med dig."
\par 25 När han sedan hade visat dem var de kunde komma in i staden, slog de stadens invånare med svärdsegg; men den mannen och hela hans släkt läto de gå.
\par 26 Och mannen begav sig till hetiternas land; där byggde han en stad och gav den namnet Lus, såsom den heter ännu i dag.
\par 27 Men Manasse intog icke Bet-Sean med underlydande orter, ej heller Taanak med underlydande orter; och ej heller fördrevo de invånarna i Dor och underlydande orter, ej heller invånarna i Jibleam och underlydande orter, ej heller invånarna i Megiddo och underlydande orter, utan kananéerna förmådde hålla sig kvar där i landet.
\par 28 När sedan israeliterna blevo de starkare, läto de kananéerna bliva arbetspliktiga under sig; de fördrevo dem icke heller då.
\par 29 Icke heller fördrev Efraim de kananéer som bodde i Geser, utan kananéerna bodde kvar bland dem där i Geser.
\par 30 Sebulon fördrev icke invånarna i Kitron och invånarna i Nahalol, utan kananéerna bodde kvar bland dem, men blevo arbetspliktiga under dem.
\par 31 Aser fördrev icke invånarna i Acko eller invånarna i Sidon, ej heller dem i Alab, Aksib, Helba, Afik och Rehob.
\par 32 Alltså bodde aseriterna bland kananéerna, landets gamla inbyggare; ty de fördrevo dem icke.
\par 33 Naftali fördrev icke invånarna i Bet-Semes, ej heller invånarna i Bet-Anat, utan bodde ibland kananéerna, landets gamla inbyggare; men invånarna i Bet-Semes och Bet-Anat blevo arbetspliktiga åt dem.
\par 34 Men amoréerna trängde undan Dans barn till bergsbygden, ty de tillstadde dem icke att komma ned till dalbygden.
\par 35 Och amoréerna förmådde hålla sig kvar i Har-Heres, Ajalon och Saalbim; men Josefs barns hand blev tung över dem, så att de blevo arbetspliktiga under dessa.
\par 36 Och amoréernas område sträckte sig från Skorpionhöjden, från Sela vidare uppåt.

\chapter{2}

\par 1 Och HERRENS ängel kom från Gilgal upp till Bokim. Och han sade: "Jag förde eder upp ur Egypten och lät eder komma in i det land som jag med ed hade lovat åt edra fäder; och jag sade: 'Jag skall icke bryta mitt förbund med eder till evig tid.
\par 2 I åter skolen icke sluta förbund med detta lands inbyggare; I skolen bryta ned deras altaren.' Men I haven icke velat höra min röst. Vad haven I gjort! -
\par 3 Därför säger jag nu ock: 'Jag vill icke förjaga dem för eder, utan de skola tränga eder i sidorna, och deras gudar skola bliva eder till en snara.'"
\par 4 När HERRENS ängel hade talat dessa ord till alla Israels barn, brast folket ut i gråt.
\par 5 Och de gåvo den platsen namnet Bokim; och de offrade där åt HERREN.
\par 6 Sedan Josua hade låtit folket gå, drogo Israels barn åstad var och en till sin arvedel, för att taga landet i besittning.
\par 7 Och folket tjänade HERREN, så länge Josua levde, och så länge de äldste levde, de som voro kvar efter Josua, dessa som hade sett alla de stora gärningar HERREN hade gjort för Israel.
\par 8 Men HERRENS tjänare Josua, Nuns son, dog, när han var ett hundra tio år gammal.
\par 9 Och man begrov honom på hans arvedels område i Timna-Heres i Efraims bergsbygd, norr om berget Gaas.
\par 10 När sedan också hela det släktet hade blivit samlat till sina fäder, kom ett annat släkte upp efter dem, ett som icke visste av HERREN eller de gärningar som han hade gjort för Israel.
\par 11 Då gjorde Israels barn vad ont var i HERRENS ögon och tjänade Baalerna.
\par 12 De övergåvo HERREN, sina fäders Gud, som hade fört dem ut ur Egyptens land, och följde efter andra gudar, de folks gudar, som bodde omkring dem, och dessa tillbådo de; därmed förtörnade de HERREN.
\par 13 Ty när de övergåvo HERREN och tjänade Baal och Astarterna,
\par 14 upptändes HERRENS vrede mot i Israel, och han gav dem i plundrares hand, och dessa utplundrade dem; han sålde dem i deras fienders hand där runt omkring, så att de icke mer kunde stå emot sina fiender.
\par 15 Varthelst de drogo ut var HERRENS hand emot dem, så att de kommo i olycka, såsom HERREN hade hotat, och såsom HERREN hade svurit att det skulle gå dem, och de kommo i stor nöd.
\par 16 Då lät HERREN domare uppstå, som frälste dem ur deras plundrares hand.
\par 17 Men de hörde icke heller på sina domare, utan lupo i trolös avfällighet efter andra gudar och tillbådo dem; de veko med hast av ifrån den väg som deras fäder hade vandrat, i lydnad för HERRENS bud, och gjorde icke såsom de.
\par 18 När HERREN alltså lät någon domare uppstå bland dem, var han med domaren och frälste dem ur deras fienders hand, så länge domaren levde; ty då de jämrade sig över sina förtryckare och plågare, förbarmade sig HERREN.
\par 19 Men när domaren dog, vände de tillbaka och togo sig till vad fördärvligt var, ännu mer än deras fäder, så att de följde efter andra gudar och tjänade och tillbådo dem; de avstodo icke från sina gärningar och sin hårdnackenhet.
\par 20 Därför upptändes HERRENS vrede mot Israel, så att han sade: "Eftersom detta folk har överträtt det förbund som jag stadgade för deras fäder, och icke har velat höra min röst,
\par 21 därför skall icke heller jag hädanefter fördriva för dem en enda man av de folk som Josua lämnade efter sig, när han dog;
\par 22 ty jag skall med dem sätta Israel på prov, om de vilja hålla HERRENS väg och vandra därpå, såsom deras fäder hava hållit den, eller om de icke vilja det."
\par 23 Alltså lät HERREN dessa folk bliva kvar och fördrev dem icke med hast; han gav dem icke i Josuas hand.

\chapter{3}

\par 1 Dessa voro de folk som HERREN lät bliva kvar, för att genom dem sätta Israel på prov, alla de israeliter nämligen, som icke hade varit med om alla krigen i Kanaan
\par 2 - allenast på det att dessa Israels barns efterkommande skulle få vara med om sådana, för att han så skulle lära dem att föra krig, dock allenast dem som förut icke hade varit med om sådana -:
\par 3 filistéernas fem hövdingar och alla kananéer och sidonier, samt de hivéer som bodde i Libanons bergsbygd, från berget Baal-Hermon ända dit där vägen går till Hamat.
\par 4 Med dessa ville HERREN sätta Israel på prov, för att förnimma om de ville hörsamma de bud som han hade givit deras fäder.
\par 5 Då nu Israels barn bodde: ibland kananéerna, hetiterna, amoréerna, perisséerna, hivéerna och jebuséerna,
\par 6 togo de deras döttrar till hustrur åt sig och gåvo sina döttrar åt deras söner och tjänade deras gudar.
\par 7 Så gjorde Israels barn vad ont var i HERRENS ögon och glömde HERREN, sin Gud, och tjänade Baalerna och Aserorna.
\par 8 Då upptändes HERRENS vrede mot Israel, och han sålde dem i Kusan-Risataims hand, konungens i Aram-Naharaim; och Israels barn måste tjäna Kusan-Risataim i åtta år.
\par 9 Men Israels barn ropade till HERREN, och HERREN lät då bland Israels barn en frälsare uppstå, som frälste dem, nämligen Otniel, son till Kenas, Kalebs yngre broder.
\par 10 HERRENS Ande kom över honom, och han blev domare i Israel, och när han drog ut till strid, gav HERREN Kusan-Risataim, konungen i Aram, i hans hand, så att hans hand blev Kusan-Risataim övermäktig.
\par 11 Och landet hade nu ro i fyrtio år; så dog Otniel, Kenas' son.
\par 12 Men Israels barn gjorde åter vad ont var i HERRENS ögon, då gav HERREN Eglon, konungen i Moab, makt över Israel, eftersom de gjorde vad ont var i HERRENS ögon.
\par 13 Denne förenade med sig Ammons barn och Amalek; sedan tågade han åstad och slog Israel, varefter de intogo Palmstaden.
\par 14 Och Israels barn måste nu tjäna: Eglon, konungen i Moab, i aderton år.
\par 15 Men Israels barn ropade till HERREN, och HERREN lät då bland dem en frälsare uppstå, benjaminiten Ehud, Geras son, en vänsterhänt man. När Israels barn genom honom skulle sända sina skänker till Eglon, konungen i Moab,
\par 16 gjorde sig Ehud ett tveeggat svärd, en fot långt; och han band detta under sina kläder vid sin högra länd.
\par 17 Så överlämnade han skänkerna till Eglon, konungen i Moab. Men Eglon var en mycket fet man.
\par 18 När han nu hade överlämnat skänkerna, lät han folket som hade burit dem gå sin väg.
\par 19 Men själv vände han tillbaka från Belätesplatsen vid Gilgal och lät säga: "Jag har ett hemligt ärende till dig, o konung." När denne då sade: "Lämnen oss i ro", gingo alla de som stodo omkring honom ut därifrån.
\par 20 Men sedan Ehud hade kommit in till honom, där han satt i sommarsalen, som han hade för sig allena, sade Ehud: "Jag har ett ord från Gud att säga dig." Då stod han upp från sin stol.
\par 21 Men Ehud räckte ut sin vänstra hand och tog svärdet från sin högra länd och stötte det i hans buk,
\par 22 så att ock fästet följde med in efter klingan, och klingan omslöts av fettet, ty han drog icke ut svärdet ur hans buk. Därefter gick Ehud ut i försalen;
\par 23 och när han hade kommit ditut, i förhallen, stängde han igen dörrarna till salen efter sig och riglade dem.
\par 24 Sedan, då han hade gått sin väg, kommo Eglons tjänare, och när de fingo se att dörrarna till salen voro riglade, tänkte de: "Förvisso har han något avsides bestyr i sin sommarkammare."
\par 25 Men sedan de hade väntat länge och väl, och han ändå icke öppnade dörrarna till salen, togo de nyckeln och öppnade själva, och se, då låg deras herre död där på golvet.
\par 26 Men Ehud hade flytt undan, medan de dröjde; han hade redan hunnit förbi Belätesplatsen och flydde sedan undan till Seira.
\par 27 Och så snart han hade kommit hem, lät han stöta i basun Efraims bergsbygd; då drogo Israels barn ned från bergsbygden med honom i spetsen för sig.
\par 28 Och han sade till dem: "Följen efter mig, ty HERREN har givit edra fiender, moabiterna, i eder hand." Då drogo de efter honom längre ned och besatte vadställena över Jordan för moabiterna och läto ingen komma över.
\par 29 Där slogo de då moabiterna, vid pass tio tusen man, allasammans ansenligt och tappert folk; icke en enda kom undan.
\par 30 Så blev Moab då kuvat under Israels hand. Och landet hade nu ro i åttio år.
\par 31 Efter honom kom Samgar, Anats son; han slog filistéerna, sex hundra man, med en oxpik. Också han frälste Israel.

\chapter{4}

\par 1 Men Israels barn gjorde åter vad ont var i HERRENS ögon, när Ehud var död.
\par 2 Då sålde HERREN dem i Jabins hand, den kananeiske konungens, som regerade i Hasor. Hans härhövitsman var Sisera, och denne bodde i Haroset-Haggoim.
\par 3 Och Israels barn ropade till HERREN; ty han hade nio hundra stridsvagnar av järn, och han förtryckte Israels barn våldsamt i tjugu år.
\par 4 Men Debora, en profetissa, Lappidots hustru, var på den tiden domarinna i Israel.
\par 5 Hon plägade sitta under Deborapalmen, mellan Rama och Betel i Efraims bergsbygd, och Israels barn drogo ditupp till henne, for att hon skulle skipa rätt.
\par 6 Hon sände nu och lät kalla till sig Barak, Abinoams son, från Kedes i Naftali, och sade till honom: "Se, HERREN, Israels Gud, bjuder: Drag åstad upp på berget Tabor och tag med dig tio tusen man av Naftali barn och Sebulons barn.
\par 7 Ty jag vill draga Sisera, Jabins härhövitsman, med hans vagnar och skaror, till dig vid bäcken Kison och giva honom i din hand."
\par 8 Barak sade till henne: "Om du går med mig, så går jag, men om du icke går med mig, så går icke heller jag."
\par 9 Då svarade hon: "Ja, jag skall gå med dig; dock skall äran då icke bliva din på den väg du nu går, utan HERREN skall sälja Sisera i en kvinnas hand." Så stod Debora upp och gick med Barak till Kedes.
\par 10 Då bådade Barak upp Sebulon och Naftali till Kedes, och tio tusen man följde honom ditupp; Debora gick ock ditupp med honom.
\par 11 Men kainéen Heber hade skilt sig från de övriga kainéerna, Hobabs, Moses svärfaders, barn; och han hade sina tältplatser ända till terebinten i Saannim vid Kedes.
\par 12 Och man berättade för Sisera att Barak, Abinoams son, hade dragit upp på berget Tabor.
\par 13 Då bådade Sisera upp alla sina stridsvagnar, nio hundra vagnar av järn, därtill ock allt folk han hade, att draga från Haroset-Haggoim till bäcken Kison.
\par 14 Men Debora sade till Barak: "Stå upp, ty detta är den dag på vilken HERREN har givit Sisera i din hand; se, HERREN har dragit ut framför dig." Så drog då Barak ned från berget Tabor, och tio tusen man följde honom.
\par 15 Och HERREN sände förvirring över Sisera och alla hans vagnar och hela hans här, så att de veko tillbaka för Baraks svärd; och Sisera steg ned från sin vagn och flydde till fots.
\par 16 Och Barak jagade efter vagnarna och hären ända till Haroset-Haggoim. Och hela Siseras här föll för svärdsegg; icke en enda kom undan.
\par 17 Men Sisera hade under flykten styrt sina steg till Jaels, kainéen Hebers hustrus, tält; ty vänskap rådde mellan Jabin, konungen i Hasor, och kainéen Hebers hus.
\par 18 Då gick Jael ut emot Sisera och sade till honom: "Kom in, min herre, kom in till mig, frukta intet." Så gick han då in till henne i tältet, och hon höljde över honom med ett täcke.
\par 19 Och han sade till henne: "Giv mig litet vatten att dricka, ty jag är törstig." Då öppnade hon mjölkkärlet och gav honom att dricka och höljde sedan åter över honom.
\par 20 Och han sade till henne: "Ställ dig vid ingången till tältet; och kommer någon och frågar dig om någon är här, så svara nej."
\par 21 Men Jael, Hebers hustru, grep en tältplugg och tog en hammare i sin hand, gick därefter sakta in till honom och slog pluggen genom tinningen på honom, så att den gick ned i marken. Så dödades han, där han låg försänkt i tung sömn, medtagen av trötthet.
\par 22 I samma stund kom Barak jagande efter Sisera; då gick Jael ut emot honom och sade till honom: "Kom hit, så skall jag visa dig den man som du söker." När han då gick in till henne, fick han se Sisera ligga död där, med tältpluggen genom tinningen.
\par 23 Så lät Gud på den dagen Jabin, konungen i Kanaan, bliva kuvad av Israels barn.
\par 24 Och Israels barns hand vilade allt tyngre på Jabin, konungen i Kanaan; och till slut förgjorde de Jabin, konungen i Kanaan.

\chapter{5}

\par 1 De sjöngo Debora och Barak, Abinoams son, denna sång:
\par 2 Att härförare förde an i Israel, att folket villigt följde dem - loven HERREN därför!
\par 3 Hören, I konungar; lyssnen, I furstar. Till HERRENS ära vill jag, vill jag sjunga, lovsäga HERREN, Israels Gud.
\par 4 HERRE, när du drog ut från Seir, när du gick fram ifrån Edoms mark, då bävade jorden, då strömmade det från himmelen, då strömmade vatten ned ifrån molnen;
\par 5 bergen skälvde inför HERRENS ansikte, ja, Sinai inför HERRENS, Israels Guds, ansikte.
\par 6 I Samgars dagar, Anats sons, i Jaels dagar lågo vägarna öde; vandrarna måste färdas svåra omvägar.
\par 7 Inga styresmän funnos, inga funnos mer i Israel, förrän du stod upp, Debora, stod upp såsom en moder i Israel.
\par 8 Man valde sig nya gudar; då nådde striden fram till portarna. Men ingen sköld, intet spjut var att se hos de fyrtio tusen i Israel.
\par 9 Mitt hjärta tillhör Israels hövdingar och dem bland folket, som villigt följde;
\par 10 ja, loven HERREN. I som riden på vita åsninnor, I som sitten hemma på mattor, och I som vandren på vägen, talen härom.
\par 11 När man under rop skiftar byte mellan vattenhoarna, då lovprisar man där HERRENS rättfärdiga gärningar, att han i rättfärdighet regerar i Israel. Då drog HERRENS folk ned till portarna.
\par 12 Upp, upp, Debora! Upp, upp, sjung din sång! Stå upp, Barak; tag dig fångar, du Abinoams son.
\par 13 Då satte folkets kvarleva de tappre till anförare, HERREN satte mig till anförare över hjältarna.
\par 14 Från Efraim kommo män som hade rotfäst sig i Amalek; Benjamin följde dig och blandade sig med dina skaror. Ned ifrån Makir drogo hövdingar åstad, och från Sebulon män som buro anförarstav.
\par 15 Furstarna i Isaskar slöto sig till Debora; och likasom Isaskar, så gjorde ock Barak; ned i dalen skyndade man i dennes spår. Bland Rubens ätter höllos stora rådslag.
\par 16 Men varför satt du kvar ibland dina fållor och lyssnade till flöjtspel vid hjordarna? Ja, av Rubens ätter fördes stora överläggningar.
\par 17 Gilead stannade på andra sidan Jordan. Och Dan varför - dröjer han ännu vid skeppen? Aser satt kvar vid havets strand, vid sina vikar stannade han.
\par 18 Men Sebulon var ett folk som prisgav sitt liv åt döden, Naftali likaså, på stridsfältets höjder.
\par 19 Konungar drogo fram och stridde ja, då stridde Kanaans konungar vid Taanak, invid Megiddos vatten; men byte av silver vunno de icke.
\par 20 Från himmelen fördes strid, stjärnorna stridde från sina banor mot Sisera.
\par 21 Bäcken Kison ryckte dem bort, urtidsbäcken, bäcken Kison. Gå fram, min själ, med makt!
\par 22 Då stampade hästarnas hovar, när deras tappra ryttare jagade framåt, framåt.
\par 23 Förbannen Meros, säger HERRENS ängel, ja, förbannen dess inbyggare, därför att de ej kommo HERREN till hjälp, HERREN till hjälp bland hjältarna.
\par 24 Välsignad vare Jael framför andra kvinnor, Hebers hustru, kainéens, välsignad framför alla kvinnor som bo i tält!
\par 25 Vatten begärde han; då gav hon honom mjölk, gräddmjölk bar hon fram i högtidsskålen.
\par 26 Sin hand räckte hon ut efter tältpluggen, sin högra hand efter arbetshammaren med den slog hon Sisera och krossade hans huvud, spräckte hans tinning och genomborrade den.
\par 27 Vid hennes fötter sjönk han ihop, föll omkull och blev liggande; ja, vid hennes fötter sjönk han ihop och föll omkull; där han sjönk ihop, där föll han dödsslagen.
\par 28 Ut genom fönstret skådade hon och ropade, Siseras moder, ut genom gallret: "Varför dröjer väl hans vagn att komma? Varför äro de så senfärdiga, hans vagnshästars fötter?"
\par 29 Då svara de klokaste av hennes hovtärnor, och själv giver hon sig detsamma svaret:
\par 30 "Förvisso vunno de byte, som de nu utskifta: en flicka, ja, två åt envar av männen, byte av praktvävnader för Siseras räkning, byte av praktvävnader, brokiga tyger; en präktig duk, ja, två brokiga dukar för de fångnas halsar."
\par 31 Så må alla dina fiender förgås, o HERRE. Men de som älska honom må likna solen, när den går upp i hjältekraft. Och landet hade nu ro i fyrtio år.

\chapter{6}

\par 1 Men när Israels barn gjorde vad ont var i HERRENS ögon, gav HERREN dem i Midjans hand, i sju år.
\par 2 Och Midjans hand blev Israel så övermäktig, att Israels barn till skydd mot Midjan gjorde sig de hålor som nu äro att se i bergen, så ock grottorna och bergfästena.
\par 3 Så ofta israeliterna hade sått, drogo midjaniterna, amalekiterna och österlänningarna upp emot dem
\par 4 och lägrade sig där och överföllo dem och fördärvade landets gröda ända fram emot Gasa; de lämnade inga livsmedel kvar i Israel, inga får, oxar eller åsnor.
\par 5 Ty de drogo ditupp med sin boskap och sina tält och kommo så talrika som gräshoppor; de själva och deras kameler voro oräkneliga. Och de föllo in i landet för att fördärva det.
\par 6 Så kom Israel i stort elände genom Midjan; då ropade Israels barn till HERREN.
\par 7 Och när Israels barn ropade till HERREN för Midjans skull,
\par 8 sände HERREN en profet till Israels barn. Denne sade till dem: "Så säger HERREN, Israels Gud: Jag själv har fört eder upp ur Egypten och hämtat eder ut ur träldomshuset.
\par 9 Jag har räddat eder från egyptiernas hand och från alla edra förtryckares hand; jag har förjagat dem för eder och givit eder deras land.
\par 10 Och jag sade till eder: Jag är HERREN, eder Gud; I skolen icke frukta de gudar som dyrkas av amoréerna, i vilkas land I bon. Men I villen icke höra min röst.
\par 11 Och HERRENS ängel kom och satte sig under terebinten vid Ofra, som tillhörde abiesriten Joas; dennes son Gideon höll då på att klappa ut vete i vinpressen, för att bärga det undan Midjan.
\par 12 För honom uppenbarade sig nu HERRENS ängel och sade till honom: "HERREN är med dig, du tappre stridsman."
\par 13 Gideon svarade honom: "Ack min herre, om HERREN är med oss, varför har då allt detta kommit över oss? Och var äro alla hans under, om vilka våra fäder hava förtäljt för oss och sagt: 'Se, HERREN har fört oss upp ur Egypten'? Nu har ju HERREN förskjutit oss och givit oss i Midjans våld."
\par 14 Då vände sig HERREN till honom och sade: "Gå åstad i denna din kraft och fräls Israel ur Midjans våld; se, jag har sänt dig."
\par 15 Han svarade honom: "Ack Herre, varmed kan jag frälsa Israel? Min ätt är ju den oansenligaste i Manasse, och jag själv den ringaste i min faders hus."
\par 16 HERREN sade till honom: "Jag vill vara med dig, så att du skall slå Midjan, såsom voro det en enda man."
\par 17 Men han svarade honom: "Om jag har funnit nåd för dina ögon, så låt mig få ett tecken att det är du som talar med mig.
\par 18 Gå icke bort härifrån, förrän jag har kommit tillbaka till dig och hämtat ut min offergåva och lagt fram den för dig." Han sade: "Jag vill stanna, till dess du kommer igen."
\par 19 Då gick Gideon in och tillredde en killing, så ock osyrat bröd av en efa mjöl; därefter lade han köttet i en korg och hällde spadet i en kruka; sedan bar han ut det till honom under terebinten och satte fram det.
\par 20 Men Guds ängel sade till honom: Tag köttet och det osyrade brödet, och lägg det på berghällen där, och gjut spadet däröver." Och han gjorde så.
\par 21 Och HERRENS ängel räckte ut staven som han hade i sin hand och rörde med dess ända vid köttet och det osyrade brödet; då kom eld ut ur klippan och förtärde köttet och det osyrade brödet; och därvid försvann HERRENS ängel ur hans åsyn.
\par 22 Då såg Gideon att det var HERRENS ängel. Och Gideon sade: "Ve mig, Herre, HERRE, eftersom jag nu har sett HERRENS ängel ansikte mot ansikte!"
\par 23 Men HERREN sade till honom: "Frid vare med dig, frukta icke; du skall icke dö."
\par 24 Då byggde Gideon där ett altare åt HERREN och kallade det HERREN är frid; det finnes kvar ännu i dag i det abiesritiska Ofra.
\par 25 Den natten sade HERREN till honom: "Tag den tjur som tillhör din fader och den andra sjuåriga tjuren. Riv sedan ned det Baalsaltare som tillhör din fader, och hugg sönder Aseran som står därinvid.
\par 26 Bygg därefter upp ett altare åt HERREN, din Gud, överst på denna fasta plats, och uppför det på övligt sätt; tag så den andra tjuren och offra den till brännoffer på styckena av Aseran som du har huggit sönder."
\par 27 Då tog Gideon tio av sina tjänare med sig och gjorde såsom HERREN sade sagt till honom. Men eftersom han fruktade att göra det om dagen, av rädsla för sin faders hus och för männen i staden, gjorde han det om natten.
\par 28 Bittida följande morgon fingo mannen i staden se att Baals altare låg nedbrutet, att Aseran därinvid var sönderhuggen, och att den andra tjuren hade blivit offrad såsom brännoffer på det nyuppbyggda altaret.
\par 29 Då sade de till varandra: "Vem har gjort detta?" Och när de frågade och gjorde efterforskningar, fingo de veta att Gideon, Joas' son, hade gjort det.
\par 30 Då sade männen i staden till Joas: "För din son hitut, han måste dö; ty han har brutit ned Baals altare, och han har ock huggit sönder Aseran som stod därinvid."
\par 31 Men Joas svarade alla som stodo omkring honom: "Viljen I utföra Baals sak, viljen I komma honom till hjälp? Den som vill utföra hans sak, han skall bliva dödad innan nästa morgon. Är han Gud, så utföre han själv sin sak, eftersom denne har brutit ned hans altare.
\par 32 Härav kallade man honom då Jerubbaal, i det man sade: "Baal utföre sin sak mot honom, eftersom han har brutit ned hans altare."
\par 33 Och midjaniterna, amalekiterna och österlänningarna hade alla tillhopa församlat sig och gått över floden och lägrat sig i Jisreels dal.
\par 34 Men Gideon hade blivit beklädd med HERRENS Andes kraft; han stötte i basun, och abiesriterna församlade sig och följde efter honom.
\par 35 Och han sände omkring budbärare i hela Manasse, så att ock de övriga församlade sig och följde efter honom; likaledes sände han budbärare till Aser, Sebulon och Naftali, och dessa drogo då också upp, de andra till mötes.
\par 36 Och Gideon sade till Gud: "Om du verkligen vill genom min hand frälsa Israel, såsom du har lovat,
\par 37 så se nu här: jag lägger denna avklippta ull på tröskplatsen; ifall dagg kommer allenast på ullen, under det att marken eljest överallt förbliver torr, då vet jag att du genom min hand skall frälsa Israel, såsom du har lovat."
\par 38 Och det skedde så, ty när han bittida dagen därefter kramade ur ullen, kunde han av den pressa ut så mycket dagg, att en hel skål blev full med vatten.
\par 39 Men Gideon sade till Gud: "Må din vrede icke upptändas mot mig, därför att jag talar ännu en enda gång. Låt mig få försöka blott en gång till med ullen: gör nu så, att allenast ullen förbliver torr, under det att dagg kommer eljest överallt på marken."
\par 40 Och Gud gjorde så den natten; allenast ullen var torr, men eljest hade dagg kommit överallt på marken.

\chapter{7}

\par 1 Bittida följande morgon drog Jerubbaal, det är Gideon, åstad med allt folket som följde honom, och de lägrade sig vid Harodskällan; han hade då midjaniternas läger norr om sig, från Morehöjden ned i dalen.
\par 2 Men HERREN sade till Gideon: "Folket som har följt dig är för talrikt för att jag skulle vilja giva Midjan i deras hand; ty Israel kunde då berömma sig mot mig och säga: 'Min egen hand har frälst mig.'
\par 3 Låt därför nu utropa för folket och säga: Om någon fruktar och är rädd, så må han vända tillbaka hem och skynda bort ifrån Gileads berg." Då vände tjugutvå tusen man av folket tillbaka, så att allenast tio tusen man stannade kvar.
\par 4 Men HERREN sade till Gideon: "Folket är ännu för talrikt; för dem ned till vattnet, så skall jag där göra ett urval av dem åt dig. Den om vilken jag då säger till dig: 'Denne skall gå med dig', han får gå med dig, men var och en om vilken jag säger till dig: 'Denne skall icke gå med dig', han får icke gå med."
\par 5 Så förde han då folket ned till vattnet. Och HERREN sade till Gideon: "Alla som läppja av vattnet, såsom hunden gör, dem skall du ställa för sig, och likaså alla som falla ned på knä för att dricka."
\par 6 Då befanns antalet av dem som hade läppjat av vattnet, genom att med handen föra det till munnen, vara tre hundra man; allt det övriga folket hade fallit ned på knä för att dricka vatten.
\par 7 Och HERREN sade till Gideon: "Med de tre hundra män som hava läppjat av vattnet skall jag frälsa eder och giva Midjan i din hand; allt det andra folket må begiva sig hem, var och en till sitt."
\par 8 Då tog hans folk till sig sitt munförråd och sina basuner, därefter lät han alla de andra israeliterna gå hem, var och en till sin hydda; han behöll allenast de tre hundra männen. Och midjaniternas läger hade han nedanför sig i dalen.
\par 9 Och HERREN sade den natten till honom: "Stå upp och drag ned i lägret, ty jag har givit det i din hand.
\par 10 Men om du fruktar för att draga ditned, så må du gå förut med din tjänare Pura ned till lägret
\par 11 och höra efter, vad man där talar; sedan skall du få mod till att draga ditned och bryta in i lägret." Så gick han då med sin tjänare Pura ned till förposterna i lägret.
\par 12 Och midjaniterna, amalekiterna och alla österlänningarna lågo där i dalen, talrika såsom gräshoppor; och deras kameler voro oräkneliga, talrika såsom sanden på havets strand.
\par 13 Då nu Gideon kom dit, höll en man just på att förtälja en dröm för en annan. Han sade: "Jag har nyss haft en dröm. Jag tyckte att en kornbrödskaka kom rullande in i midjaniternas läger. Den kom ända fram till tältet och slog emot det, så att det föll, och vände upp och ned på det, och tältet blev så liggande."
\par 14 Då svarade den andre och sade: "Detta betyder intet annat än israeliten Gideons, Joas' sons, svärd; Gud har givit Midjan och hela lägret i hans hand."
\par 15 När Gideon hörde denna dröm förtäljas och hörde dess uttydning, föll han ned och tillbad. Därefter vände han tillbaka till Israels läger och sade: "Stån upp, ty HERREN har givit midjaniternas läger i eder hand.
\par 16 Och han delade sina tre hundra män i tre hopar och gav allasammans basuner i händerna, så ock tomma krukor, med facklor inne i krukorna.
\par 17 Och han sade till dem: "Sen på mig och gören såsom jag; så snart jag har kommit till utkanten av lägret, skolen I göra såsom jag gör.
\par 18 När nämligen jag och alla som jag har med mig stöta i basunerna, skolen ock I stöta i basunerna runt omkring hela lägret och ropa: 'För HERREN och för Gideon!'"
\par 19 Så kommo nu Gideon och de hundra män, som han hade med sig, till utkanten av lägret, när den mellersta nattväkten ingick; och man hade just ställt ut vakterna. Då stötte de i basunerna och krossade krukorna som de hade i sina händer.
\par 20 De tre hoparna stötte i basunerna och slogo sönder krukorna; de fattade med vänstra handen i facklorna och med högra handen i basunerna och stötte i dem; och de ropade: "HERRENS och Gideons svärd!"
\par 21 Men de stodo stilla, var och en på sin plats, runt omkring lägret. Då begynte alla i lägret att löpa hit och dit och skria och fly.
\par 22 Och när de stötte i de tre hundra basunerna, vände HERREN den enes svärd mot den andre i hela lägret; och de som voro i lägret flydde ända till Bet-Hasitta, åt Serera till, ända till stranden vid Abel-Mehola, förbi Tabbat.
\par 23 Och åter församlade sig israeliterna, från Naftali och Aser och från hela Manasse, och förföljde midjaniterna.
\par 24 Och Gideon hade sänt omkring budbärare i hela Efraims bergsbygd och låtit säga: "Dragen ned mot Midjan och besätten i deras väg vattendragen ända till Bet-Bara, ävensom Jordan." Så församlade sig alla Efraims män och besatte vattendragen ända till Bet-Bara, ävensom Jordan.
\par 25 Och de togo två midjanitiska hövdingar, Oreb och Seeb, till fånga, och dräpte Oreb vid Orebsklippan, och Seeb dräpte de vid Seebspressen, och förföljde så midjaniterna. Men Orebs och Seebs huvuden förde de över till Gideon på andra sidan Jordan.

\chapter{8}

\par 1 Men Efraims män sade till honom: "Huru har du kunnat handla så mot oss? Varför bådade du icke upp oss, när du drog ut till strid mot Midjan?" Och de foro häftigt ut mot honom.
\par 2 Han svarade dem: "Vad har jag då uträttat i jämförelse med eder? Är icke Efraims efterskörd bättre än Abiesers vinbärgning?
\par 3 I eder hand var det som Gud gav de midjanitiska hövdingarna Oreb och Seeb. Vad har jag kunnat uträtta i jämförelse med eder?" Då han så talade, stillades deras vrede mot honom.
\par 4 När sedan Gideon kom till Jordan, gick han över jämte de tre hundra män som han hade med sig; och de voro trötta av förföljandet.
\par 5 Han sade därför till männen i Suckot: "Given några kakor bröd åt folket som följer mig, ty de äro trötta; se, jag är nu i färd med att förfölja Seba och Salmunna, de midjanitiska konungarna."
\par 6 Men de överste i Suckot svarade: "Har du då redan Seba och Salmunna i ditt våld, eftersom du fordrar att vi skola giva bröd åt din här?"
\par 7 Gideon sade "Nåväl; när HERREN, giver Seba och Salmunna i min hand, skall jag söndertröska edert kött med ökentörnen och tistlar."
\par 8 Så drog han vidare därifrån upp till Penuel och talade på samma sätt till dem som voro där; och männen i Penuel gåvo honom samma svar som männen i Suckot hade givit.
\par 9 Då sade han ock till männen i Penuel: "När jag kommer välbehållen tillbaka, skall jag riva ned detta torn."
\par 10 Men Seba och Salmunna befunno sig i Karkor och hade sin här hos sig, vid pass femton tusen man, allt som var kvar av österlänningarnas hela här; ty de stupade utgjorde ett hundra tjugu tusen svärdbeväpnade män.
\par 11 Och Gideon drog upp på karavanvägen, öster om Noba och Jogbeha, och överföll hären, där den låg sorglös i sitt läger.
\par 12 Och Seba och Salmunna flydde, men han satte efter dem; och han tog de två midjanitiska konungarna Seba och Salmunna till fånga och skingrade hela hären.
\par 13 När därefter Gideon, Joas' son, vände tillbaka från striden, ned från Hereshöjden,
\par 14 fick han fatt på en ung man, en av invånarna i Suckot, och utfrågade denne, och han måste skriva upp åt honom de överste i Suckot och de äldste där, sjuttiosju män.
\par 15 När han sedan kom till männen i Suckot, sade han: "Se här äro nu Seba och Salmunna, om vilka I hånfullt saden till mig: 'Har du redan Seba och Salmunna i ditt våld, eftersom du fordrar att vi skola giva bröd åt dina trötta män?'"
\par 16 Därefter lät han gripa de äldste i staden och tog ökentörnen och tistlar och lät männen i Suckot få känna dem.
\par 17 Och tornet i Penuel rev han ned och dräpte männen i staden.
\par 18 Och till Seba och Salmunna sade han: "Hurudana voro de män som I dräpten på Tabor?" De svarade: "De voro lika dig; var och en såg ut såsom en konungason."
\par 19 Han sade: "Då var det mina bröder, min moders söner. Så sant HERREN lever: om I haden låtit dem leva, skulle jag icke hava dräpt eder."
\par 20 Sedan sade han till Jeter, sin förstfödde: "Stå upp och dräp dem." Men gossen drog icke ut sitt svärd, ty han var försagd, eftersom han ännu var allenast en gosse.
\par 21 Då sade Seba och Salmunna: "Stå upp, du själv, och stöt ned oss; ty sådan mannen är, sådan är ock hans styrka." Så stod då Gideon upp och dräpte Seba och Salmunna. Och han tog för sin räkning de prydnader som sutto på deras kamelers halsar.
\par 22 Och israeliterna sade till Gideon: "Råd du över oss, och såsom du så ock sedan din son och din sonson; ty du har frälst oss ur Midjans hand."
\par 23 Men Gideon svarade dem: "Jag vill icke råda över eder, och min son skall icke heller råda över eder, utan HERREN skall råda över eder."
\par 24 Och Gideon sade ytterligare till dem: "Ett vill jag dock begära av eder: var och en av eder må giva mig den näsring han har fått såsom byte." Ty midjaniterna buro näsringar av guld, eftersom de voro ismaeliter.
\par 25 De svarade: "Ja, vi vilja giva dig dem." Och de bredde ut ett kläde, och var och en kastade på detta den näsring han hade fått såsom byte.
\par 26 Och guldringarna, som han hade begärt, befunnos väga ett tusen sju hundra siklar i guld - detta förutom de halsprydnader, de örhängen och de purpurröda kläder som de midjanitiska konungarna hade burit, och förutom de kedjor som hade suttit på deras kamelers halsar.
\par 27 Och Gideon lät därav göra en efod och satte upp den i sin stad, Ofra; och hela Israel lopp där i trolös avfällighet efter den. Och den blev för Gideon och hans hus till en snara.
\par 28 Så blev nu Midjan kuvat under Israels barn och upplyfte icke mer sitt huvud. Och landet hade ro i fyrtio år, så länge Gideon levde.
\par 29 Men Jerubbaal, Joas' son, gick hem och stannade sedan i sitt hus.
\par 30 Och Gideon hade sjuttio söner, som hade utgått från hans länd, ty han ägde många hustrur.
\par 31 En bihustru som han hade i Sikem födde honom ock en son; denne gav han namnet Abimelek.
\par 32 Och Gideon, Joas' son, dog i en god ålder och blev begraven i sin fader Joas' grav i det abiesritiska Ofra
\par 33 Men när Gideon var död, begynte Israels barn åter i trolös avfällighet löpa efter Baalerna; och de gjorde Baal-Berit till gud åt sig.
\par 34 Israels barn tänkte icke på HERREN, sin Gud, som hade räddat dem från alla deras fienders hand runt omkring.
\par 35 Ej heller visade de Jerubbaals, Gideons, hus någon kärlek, till gengäld för allt det goda som han hade gjort mot Israel.

\chapter{9}

\par 1 Men Abimelek, Jerubbaals son, gick bort till sin moders bröder i Sikem och talade till dem och till alla som voro besläktade med hans morfaders hus, och sade:
\par 2 "Talen så till alla Sikems borgare: Vilket är bäst för eder: att sjuttio män, alla Jerubbaals söner, råda över eder, eller att en enda man råder över eder? Kommen därjämte ihåg att jag är edert kött och ben."
\par 3 Då talade hans moders bröder till hans förmån allt detta inför alla Sikems borgare. Och dessa blevo vunna för Abimelek, ty de tänkte: "Han är ju vår broder."
\par 4 Och de gåvo honom sjuttio siklar silver ur Baal-Berits tempel; för dessa lejde Abimelek löst folk och äventyrare, vilkas anförare han blev.
\par 5 Därefter begav han sig till sin faders hus i Ofra och dräpte där sina bröder, Jerubbaals söner, sjuttio män, och detta på en och samma sten; dock blev Jotam, Jerubbaals yngste son, vid liv, ty han hade gömt sig.
\par 6 Sedan församlade sig alla Sikems borgare och alla som bodde i Millo och gingo åstad och gjorde Abimelek till konung vid Vård-terebinten invid Sikem.
\par 7 När man berättade detta far Jotam, gick han åstad och ställde sig på toppen av berget Gerissim och hov upp sin röst och ropade och sade till dem: "Hören mig, I Sikems borgare, för att Gud ock må höra eder.
\par 8 Träden gingo en gång åstad för att smörja en konung över sig. Och de sade till olivträdet: 'Bliv du konung över oss'
\par 9 Men olivträdet svarade dem: 'Skulle jag avstå från min fetma, som både gudar och människor ära mig för, och gå bort för att svaja över de andra träden?'
\par 10 Då sade träden till fikonträdet: 'Kom du och bliv konung över oss.'
\par 11 Men fikonträdet svarade dem: 'Skulle jag avstå från min sötma och min goda frukt och gå bort för att svaja över de andra träden?'
\par 12 Då sade träden till vinträdet: 'Kom du och bliv konung över oss.'
\par 13 Men vinträdet svarade dem: 'Skulle jag avstå från min vinmust, som gör både gudar och människor glada, och gå bort för att svaja över de andra träden?'
\par 14 Då sade alla träden till törnbusken: 'Kom du och bliv konung över oss.'
\par 15 Törnbusken svarade träden: 'Om det är eder uppriktiga mening att smörja mig till konung över eder, så kommen och tagen eder tillflykt under min skugga; varom icke, så skall eld gå ut ur törnbusken och förtära cedrarna på Libanon.'
\par 16 Så hören nu: om I haven förfarit riktigt och redligt däri att I haven gjort Abimelek till konung, och om I haven förfarit väl mot Jerubbaal och hans hus, och haven vedergällt honom efter hans gärningar -
\par 17 ty I veten att min fader stridde för eder och vågade sitt liv för att rädda eder från Midjans hand,
\par 18 under det att I däremot i dag haven rest eder upp mot min faders hus och dräpt hans söner, sjuttio män, på en och samma sten, och gjort Abimelek, hans tjänstekvinnas son, till konung över Sikems borgare, eftersom han är eder broder -
\par 19 om I alltså denna dag haven förfarit riktigt och redligt mot Jerubbaal och hans hus, då mån I glädja eder över Abimelek, och han må ock glädja sig över eder;
\par 20 varom icke, så må eld gå ut från Abimelek och förtära Sikems borgare och dem som bo i Millo, och från Sikems borgare och från dem som bo i Millo må eld gå ut och förtära Abimelek."
\par 21 Och Jotam skyndade sig undan och flydde bort till Beer, och där bosatte han sig för att vara i säkerhet för sin broder Abimelek.
\par 22 När Abimelek hade härskat över Israel i tre år,
\par 23 sände Gud en tvedräktsande mellan Abimelek och Sikems borgare, så att Sikems borgare avföllo från Abimelek.
\par 24 Detta skedde, för att våldet mot Jerubbaals sjuttio söner skulle bliva hämnat, och för att deras blod skulle komma över deras broder Abimelek som dräpte dem, så ock över Sikems borgare, som lämnade honom understöd, så att han kunde dräpa sina bröder.
\par 25 För att skada honom lade Sikems borgare nu folk i försåt på bergshöjderna, och dessa plundrade alla som drogo vägen fram därförbi. Detta blev berättat för Abimelek.
\par 26 Men Gaal, Ebeds son, kom nu dit med sina bröder, och de drogo in i Sikem. Och Sikems borgare fattade förtroende för honom.
\par 27 Så hände sig en gång att de gingo ut på fältet och avbärgade sina vingårdar och pressade druvorna och höllo en glädjefest, och de gingo därvid in i sin guds hus och åto och drucko, och uttalade förbannelser över Abimelek.
\par 28 Och Gaal, Ebeds son, sade: "Vad är Abimelek, och vad är Sikem, eftersom vi skola tjäna honom? Han är ju Jerubbaals son, och Sebul är hans tillsyningsman. Nej, tjänen män som härstamma från Hamor, Sikems fader. Varför skulle vi tjäna denne?
\par 29 Ack om jag hade detta folk under min vård! Då skulle jag driva bort Abimelek." Och i fråga om Abimelek sade han: "Föröka din här och drag ut."
\par 30 Men när Sebul, hövitsmannen i staden, fick höra vad Gaal, Ebeds son, hade sagt, upptändes hans vrede.
\par 31 Och han sände listeligen bud till Abimelek och lät säga: "Se, Gaal, Ebeds son, och hans bröder hava kommit till Sikem, och de hålla just nu på att uppvigla staden mot dig.
\par 32 Bryt därför nu upp om natten, du med ditt folk, och lägg dig i bakhåll på fältet.
\par 33 Sedan må du i morgon bittida, när solen går upp, störta fram mot staden. När han då med sitt folk drager ut mot dig, må du göra med honom vad tillfället giver vid handen."
\par 34 Då bröt Abimelek med allt sitt folk upp om natten, och de lade sig i bakhåll mot Sikem, i fyra hopar.
\par 35 Och Gaal, Ebeds son, kom ut och ställde sig vid ingången till stadsporten; och i detsamma bröt Abimelek med sitt folk fram ifrån bakhållet.
\par 36 När då Gaal såg folket, sade han till Sebul: "Se, där kommer folk ned från bergshöjderna." Men Sebul svarade honom: "Det är skuggan av bergen, som för dina ögon ser ut såsom människor."
\par 37 Gaal tog åter till orda och sade: "Jo, där kommer folk ned från Mittelhöjden, och en annan hop kommer på vägen från Teckentydarterebinten."
\par 38 Då sade Sebul till honom: "Var är nu din stortalighet, du som sade: 'Vad är Abimelek, eftersom vi skola tjäna honom?' Se, här kommer det folk som du så föraktade. Drag nu ut och strid mot dem.
\par 39 Så drog då Gaal ut i spetsen för Sikems borgare och gav sig i strid med Abimelek.
\par 40 Men Abimelek jagade honom på flykten, och han flydde undan för honom; och många föllo slagna ända fram till stadsporten.
\par 41 Och Abimelek stannade i Aruma; men Sebul drev bort Gaal och hans bröder och lät dem icke längre stanna i Sikem.
\par 42 Dagen därefter gick folket ut på fältet; och man berättade detta för Abimelek.
\par 43 Då tog han sitt folk och delade dem i tre hopar och lade sig i bakhåll på fältet. Och när han fick se att folket gick ut ur staden, bröt han upp och anföll dem och nedgjorde dem.
\par 44 Abimelek och de hopar han hade med sig störtade nämligen fram och ställde sig vid ingången till stadsporten; men de båda andra hoparna störtade fram mot alla som voro på fältet och nedgjorde dem.
\par 45 När så Abimelek hade ansatt staden hela den dagen, intog han den och dräpte det folk som fanns därinne Sedan rev han ned staden och beströdde platsen med salt.
\par 46 När besättningen i Sikems torn hörde detta, begåvo de sig alla till det fasta valvet i El-Berits tempelbyggnad.
\par 47 Och när det blev berättat for Abimelek att hela besättningen i Sikems torn hade församlat sig där,
\par 48 gick han med allt sitt folk upp till berget Salmon; och Abimelek tog en yxa i sin hand och högg av en trädgren och lyfte upp den och lade den på axeln; och han sade till sitt folk: "Gören med hast detsamma som I haven sett mig göra."
\par 49 Då högg också allt folket av var sin gren och följde efter Abimelek, och de lade grenarna intill det fasta valvet och tände upp eld till att förbränna valvet jämte dem som voro där. Så omkommo ock alla de människor som bodde i Sikems torn, vid pass tusen män och kvinnor.
\par 50 Och Abimelek drog åstad till Tebes och belägrade Tebes och intog det.
\par 51 Men mitt i staden var ett starkt torn, och dit flydde alla män och kvinnor, alla borgare i staden, och stängde igen om sig; sedan stego de upp på tornets tak.
\par 52 Och Abimelek kom till tornet och angrep det; och han gick fram till porten på tornet för att bränna upp den i eld.
\par 53 Men en kvinna kastade en kvarnsten ned på Abimeleks huvud och bräckte så hans huvudskål.
\par 54 Då ropade han med hast på sin vapendragare och sade till honom: "Drag ut ditt svärd och döda mig, för att man icke må säga om mig: En kvinna dräpte honom." Då genomborrade hans tjänare honom, så att han dog.
\par 55 När nu israeliterna sågo att Abimelek var död, gingo de hem, var och en till sitt.
\par 56 Alltså lät Gud det onda som Abimelek hade gjort mot sin fader, då han dräpte sina sjuttio bröder, komma tillbaka över honom.
\par 57 Och allt det onda som Sikems män hade gjort lät Gud ock komma tillbaka över deras huvuden. Så gick Jotams, Jerubbaals sons, förbannelse i fullbordan på dem.

\chapter{10}

\par 1 Efter Abimelek uppstod till Israels frälsning Tola, son till Pua, son till Dodo, en man från Isaskar; och han bodde i Samir, i Efraims bergsbygd.
\par 2 Han var domare i Israel i tjugutre år; sedan dog han och blev begraven i Samir.
\par 3 Efter honom uppstod gileaditen Jair. Han var domare i Israel i tjugutvå år.
\par 4 Han hade trettio söner, som plägade rida på trettio åsnor; och de hade trettio städer. Dessa kallar man ännu i dag Jairs byar, och de ligga i Gileads land.
\par 5 Och Jair dog och blev begraven i Kamon.
\par 6 Men Israels barn gjorde åter vad ont var i HERRENS ögon och tjänade Baalerna och Astarterna, så ock Arams, Sidons, Moabs, Ammons barns och filistéernas gudar och övergåvo HERREN och tjänade honom icke.
\par 7 Då upptändes HERRENS vrede mot Israel, och han sålde dem i filistéernas och Ammons barns hand.
\par 8 Och dessa plågade Israels barn och förforo våldsamt mot dem det året; i aderton år gjorde de så mot alla de israeliter som bodde på andra sidan Jordan, i amoréernas land, i Gilead.
\par 9 Därtill gingo Ammons barn över Jordan och gåvo sig i strid också med Juda, Benjamin och Efraims hus, så att Israel kom i stor nöd.
\par 10 Då ropade Israels barn till HERREN och sade: "Vi hava syndat mot dig, ty vi hava övergivit vår Gud och tjänat Baalerna."
\par 11 Men HERREN sade till Israels barn: "Har jag icke frälst eder från egyptierna, amoréerna, Ammons barn och filistéerna?
\par 12 Likaledes bleven I förtryckta av sidonierna, amalekiterna och maoniterna; och när I ropaden till mig, frälste jag eder från deras hand.
\par 13 Men I haven nu övergivit mig och tjänat andra gudar; därför vill jag icke mer frälsa eder.
\par 14 Gån bort och ropen till de gudar som I haven utvalt; må de frälsa eder, om I nu ären i nöd.
\par 15 Då sade Israels barn till HERREN: "Vi hava syndat; gör du med oss alldeles såsom dig täckes. Allenast rädda oss nu denna gång."
\par 16 Därefter skaffade de bort ifrån sig de främmande gudarna och tjänade HERREN. Då kunde han icke längre lida att se Israels vedermöda.
\par 17 Och Ammons barn blevo uppbådade och lägrade sig i Gilead; men Israels barn församlade sig och lägrade sig i Mispa.
\par 18 Då sade folket, nämligen de överste i Gilead, till varandra: "Vem vill begynna striden mot Ammons barn? Den som det vill skall bliva hövding över alla Gileads inbyggare."

\chapter{11}

\par 1 Gileaditen Jefta var en tapper stridsman, men han var son till en sköka; och Jeftas fader var Gilead.
\par 2 Nu födde ock Gileads hustru honom söner; och när dessa hans hustrus söner hade växt upp, drevo de ut Jefta och sade till honom: "Du skall icke taga arv i vår faders hus, ty du är son till en kvinna som icke är hans hustru."
\par 3 Då flydde Jefta bort ifrån sina bröder och bosatte sig i landet Tob; där sällade sig löst folk till Jefta och gjorde strövtåg med honom.
\par 4 Någon tid därefter gåvo Ammons barn sig i strid med Israel.
\par 5 Men när Ammons barn gåvo sig i strid med Israel, gingo de äldste i Gilead åstad för att hämta Jefta från landet Tob.
\par 6 Och de sade till Jefta: "Kom och bliv vår anförare, så vilja vi strida mot Ammons barn."
\par 7 Men Jefta svarade de äldste i Gilead: "I haven ju hatat mig och drivit mig ut ur min faders hus. Huru kunnen I då nu, när I ären i nöd, komma till mig?"
\par 8 De äldste i Gilead sade till Jefta: "Just därför hava vi nu kommit tillbaka till dig, och du måste gå med oss och strida mot Ammons barn; ty du skall bliva hövding över oss, alla Gileads inbyggare."
\par 9 Jefta svarade de äldste i Gilead: "Om I nu fören mig tillbaka för att strida mot Ammons barn och HERREN giver dem i mitt våld, så vill jag ock sedan vara eder hövding."
\par 10 Då sade de äldste i Gilead till Jefta: "HERREN höre vårt avtal. Förvisso skola vi låta det bliva så om du har sagt."
\par 11 Så gick då Jefta med de äldste i Gilead, och folket satte honom till hövding och anförare över sig. Och Jefta uttalade inför HERREN i Mispa allt vad han hade sagt.
\par 12 Och Jefta skickade sändebud till Ammons barns konung och lät säga: "Vad har du med mig att göra, eftersom du har kommit emot mig och angripit mitt land?"
\par 13 Då svarade Ammons barns konung Jeftas sändebud: "När Israel drog upp från Egypten, togo de ju mitt land från Arnon ända till Jabbok och till Jordan; så giv mig nu detta tillbaka i godo."
\par 14 Åter skickade Jefta sändebud till Ammons barns konung
\par 15 och lät säga till honom: "Så säger Jefta: Israel har icke tagit något land vare sig från Moab eller från Ammons barn.
\par 16 Ty när de drogo upp från Egypten och Israel hade tågat genom öknen ända till Röda havet och sedan kommit till Kades,
\par 17 skickade Israel sändebud till konungen i Edom och lät säga: 'Låt mig tåga genom ditt land.' Men konungen i Edom hörde icke därpå. De skickade ock till konungen i Moab, men denne ville icke heller Då stannade Israel i Kades.
\par 18 Därefter tågade de genom öknen och gingo omkring Edoms land och Moabs land och kommo öster om Moabs land och lägrade sig på andra sidan Arnon; de kommo icke in på Moabs område, ty Arnon är Moabs gräns.
\par 19 Sedan skickade Israel sändebud till Sihon, amoréernas konung, konungen i Hesbon; och Israel lät säga till honom: 'Låt oss genom ditt land tåga dit vi skola.'
\par 20 Men Sihon litade icke på Israel och lät dem icke tåga genom sitt land, utan församlade allt sitt folk, och de lägrade sig i Jahas; där inlät han sig i strid med Israel.
\par 21 Men HERREN, Israels Gud, gav Sihon och allt hans folk i Israels hand, så att de slogo dem; och Israel intog hela amoréernas land, ty dessa bodde då i detta land.
\par 22 De intogo hela amoréernas område, från Arnon ända till Jabbok, och från öknen ända till Jordan.
\par 23 Och nu, då HERREN, Israels Gud, har fördrivit amoréerna för sitt folk Israel, skulle du taga deras land i besittning!
\par 24 Är det icke så: vad din gud Kemos giver dig till besittning, det tager du i besittning? Så taga ock vi, närhelst HERREN, vår Gud, fördriver ett folk för oss, deras land i besittning.
\par 25 Menar du att du är så mycket förmer än Balak, Sippors sons konungen i Moab? Han dristade ju icke att inlåta sig i tvist med Israel eller giva sig i strid med dem.
\par 26 När Israel nu i tre hundra år har bott i Hesbon och underlydande orter, i Aror och underlydande orter och i alla städer på båda sidor om Arnon, varför haven I då under hela den tiden icke tagit detta ifrån oss?
\par 27 Jag har icke försyndat mig mot dig, men du gör illa mot mig, då du nu överfaller mig. HERREN, domaren, må i dag döma mellan Israels barn och Ammons barn."
\par 28 Men Ammons barns konung hörde icke på vad Jefta lät säga honom genom sändebuden.
\par 29 Då kom HERRENS Ande över Jefta; och han tågade genom Gilead och Manasse och tågade så genom Mispe i Gilead, och från Mispe i Gilead tågade han fram mot Ammons barn.
\par 30 Och Jefta gjorde ett löfte åt HERREN och sade: "Om du giver Ammons barn i min hand,
\par 31 så lovar jag att vadhelst som ur dörrarna till mitt hus går ut emot mig, när jag välbehållen kommer tillbaka från Ammons barn, det skall höra HERREN till, och det skall jag offra till brännoffer."
\par 32 Så drog nu Jefta åstad mot Ammons barn för att strida mot dem; och HERREN gav dem i hans hand.
\par 33 Och han tillfogade dem ett mycket stort nederlag och intog landet från Aroer ända till fram emot Minnit, tjugu städer, och ända till Abel-Keramim. Alltså blevo Ammons barn kuvade under Israels barn.
\par 34 När sedan Jefta kom hem till sitt hus i Mispa, då gick hans dotter ut emot honom med pukor och dans. Och hon var hans enda barn, han hade utom henne varken son eller dotter.
\par 35 I detsamma han nu fick se henne, rev han sönder sina kläder och ropade: "Ve mig, min dotter, du kommer mig att sjunka till jorden, du drager olycka över mig! Ty jag har öppnat min mun inför HERREN till ett löfte och kan icke taga mitt ord tillbaka."
\par 36 Hon svarade honom: "Min fader, har du öppnat din mun inför HERREN, så gör med mig enligt din muns tal, eftersom HERREN nu har skaffat dig hämnd på dina fiender, Ammons barn."
\par 37 Och hon sade ytterligare till sin fader: "Uppfyll dock denna min begäran: unna mig två månader, så att jag får gå åstad ned på bergen och begråta min jungfrudom med mina väninnor.
\par 38 Han svarade: "Du får gå åstad." Och han tillstadde henne att vara borta i två månader. Då gick hon åstad med sina väninnor och begrät sin jungfrudom på bergen.
\par 39 Men efter två månader vände hon tillbaka till sin fader, och han förfor då med henne efter det löfte han hade gjort. Och hon hade icke känt någon man.
\par 40 Sedan blev det en sedvänja i Israel att Israels döttrar år efter år gingo åstad för att lovprisa gileaditen Jeftas dotter, under fyra dagar vart år.

\chapter{12}

\par 1 Men Efraims män församlade sig och drogo till Safon; och de sade till Jefta: "Varför drog du åstad till strid mot Ammons barn utan att kalla på oss till att tåga med dig? Nu vilja vi bränna upp ditt hus jämte dig själv i eld.
\par 2 Jefta svarade dem: "Jag och mitt folk lågo i svår fejd med Ammons barn; då manade jag eder att komma, men I villen icke frälsa mig ur deras hand.
\par 3 Och när jag såg att I icke villen frälsa mig, tog jag min själ i min hand och drog åstad mot Ammons barn, och HERREN gav dem i min hand. Varför haven I då nu dragit upp emot mig till att strida emot mig?"
\par 4 Och Jefta församlade alla Gileads män och gav sig i strid med Efraim. Och Gileads män slogo efraimiterna; dessa hade nämligen sagt: "Flyktingar ifrån Efraim ären I; Gilead är ett mellanting, varken Efraim eller Manasse."
\par 5 Och gileaditerna besatte vadställena över Jordan för efraimiterna. Då nu någon av de efraimitiska flyktingarna sade: "Låt mig komma över", frågade Gileads män honom: "Är du en efraimit?" Om han då svarade nej,
\par 6 så sade de till honom: "Säg 'schibbolet'." Sade han då "sibbolet", därför att han icke nog lade sig vinn om att uttala ordet rätt, så grepo de honom och höggo ned honom där vid vadställena över Jordan. På detta sätt föllo vid det tillfället fyrtiotvå tusen efraimiter.
\par 7 Och Jefta var domare i Israel i sex år. Sedan dog gileaditen Jefta och blev begraven i en av Gileads städer.
\par 8 Efter honom var Ibsan från Bet-Lehem domare i Israel.
\par 9 Han hade trettio söner, och trettio döttrar gifte han bort; han fick ock trettio döttrar genom att skaffa hustrur åt sina söner utifrån. Och han var domare i Israel i sju år.
\par 10 Sedan dog Ibsan och blev begraven i Bet-Lehem.
\par 11 Efter honom var sebuloniten Elon domare i Israel; i tio år var han domare i Israel.
\par 12 Sedan dog sebuloniten Elon och blev begraven i Ajalon, i Sebulons land.
\par 13 Efter honom var pirgatoniten Abdon, Hillels son, domare i Israel.
\par 14 Han hade fyrtio söner och trettio sonsöner, vilka plägade rida på sjuttio åsnor. Och han var domare i Israel i åtta år.
\par 15 Sedan dog pirgatoniten Abdon, Hillels son, och blev begraven i Pirgaton i Efraims land, i amalekiternas bergsbygd.

\chapter{13}

\par 1 Men Israels barn gjorde åter vad ont var i HERRENS ögon; då gav HERREN dem i filistéernas hand, i fyrtio år.
\par 2 I Sorga levde nu en man av daniternas släkt, vid namn Manoa; hans hustru var ofruktsam och hade icke fött några barn.
\par 3 Men HERRENS ängel uppenbarade sig för hustrun och sade till henne: "Se, du är ofruktsam och har icke fött några barn, men du skall bliva havande och föda en son.
\par 4 Tag dig nu till vara, så att du icke dricker vin eller starka drycker ej heller äter något orent.
\par 5 Ty se, du skall bliva havande och föda en son, på vilkens huvud ingen rakkniv skall komma, ty gossen skall vara en Guds nasir allt ifrån moderlivet; och han skall göra begynnelse till att frälsa Israel ur filistéernas hand."
\par 6 Då gick hustrun in och omtalade detta för sin man och sade: "En gudsman kom till mig; han såg ut såsom en Guds ängel, mycket fruktansvärd. Jag frågade honom icke varifrån han var, och sitt namn lät han mig icke veta.
\par 7 Och han sade till mig: 'Se, du skall bliva havande och föda en son; drick nu icke vin eller starka drycker och ät icke något orent, ty gossen skall vara en Guds nasir, från moderlivet ända till sin död.'"
\par 8 Och Manoa bad till HERREN och sade: "Ack Herre, låt gudsmannen som du sände hit åter komma till oss, för att han må lära oss huru vi skola göra med gossen som skall födas."
\par 9 Och Gud hörde Manoas röst; Guds ängel kom åter till hans hustru, när hon en gång satt ute på marken och hennes man Manoa icke var hos henne.
\par 10 Då skyndade hustrun strax åstad och berättade det för sin man; hon sade till honom: "Mannen som kom till mig häromdagen har uppenbarat sig för mig."
\par 11 Manoa stod upp och följde sin hustru; och när han kom till mannen, frågade han honom: "Är du den man som förut talade med min hustru?" Han svarade: "Ja."
\par 12 Då sade Manoa: "När det som du har sagt går i fullbordan, vad är då att iakttaga med gossen? Hur skall man göra med honom?"
\par 13 HERRENS ängel svarade Manoa "Din hustru skall taga sig till vara för allt varom jag har talat med henne.
\par 14 Hon skall icke äta något som ha vuxit på vinträd, och vin eller starka drycker får hon icke dricka, ej heller får hon äta något orent. Allt vad jag har bjudit henne skall hon hålla."
\par 15 Och Manoa sade till HERRENS ängel: "Låt oss få hålla dig kvar, så vilja vi tillreda en killing och sätta fram för dig."
\par 16 Men HERRENS ängel svarade Manoa: "Om du ock håller mig kvar, skall jag dock icke äta av din mat; men om du vill tillreda ett brännoffer, så offra detta åt HERREN." Ty Manoa förstod icke att det var HERRENS ängel.
\par 17 Och Manoa sade till HERRENS ängel: "Vad är ditt namn? Säg oss det, för att vi må kunna ära dig, när det som du har sagt går i fullbordan."
\par 18 HERRENS ängel sade till honom: "Varför frågar du efter mitt namn? Det är alltför underbart."
\par 19 Och Manoa tog killingen med tillhörande spisoffer och lade upp den på klippan åt HERREN. Då lät han något underbart ske i Manoas och hans hustrus åsyn.
\par 20 När lågan steg upp från altaret mot himmelen, for nämligen HERRENS ängel upp, i lågan från altaret. Då Manoa och hans hustru sågo detta, föllo de ned till jorden på sitt ansikte
\par 21 Sedan visade sig HERRENS ängel icke mer för Manoa och hans hustru. Då förstod Manoa att det hade varit HERRENS ängel.
\par 22 Och Manoa sade till sin hustru: "Nu måste vi dö, eftersom vi hava sett Gud."
\par 23 Men hans hustru svarade honom: "Om HERREN hade velat döda oss, så hade han icke tagit emot något brännoffer och spisoffer av vår hand, och icke låtit oss se allt detta, ej heller hade han nu låtit oss höra sådant."
\par 24 Därefter födde hans hustru en son och gav honom namnet Simson; och gossen växte upp, och HERREN välsignade honom.
\par 25 Och HERRENS Ande begynte att verka på honom, medan han var i Dans läger, mellan Sorga och Estaol.

\chapter{14}

\par 1 När Simson en gång gick ned till Timna, fick han där i Timna se en kvinna, en av filistéernas döttrar.
\par 2 Och när han kom upp därifrån, omtalade han det för sin fader och moder och sade: "Jag har i Timna sett en kvinna, en av filistéernas döttrar; henne mån I nu skaffa mig till hustru."
\par 3 Hans fader och moder sade till honom: "Finnes då ingen kvinna bland dina bröders döttrar och i hela mitt folk, eftersom du vill gå bort för att skaffa dig en hustru från de oomskurna filistéerna?" Simson sade till sin fader: "Skaffa mig denna, ty hon behagar mig."
\par 4 Men hans fader och moder visste icke att detta kom från HERREN, som sökte sak med filistéerna. På den tiden rådde nämligen filistéerna över Israel.
\par 5 Och Simson gick med sin fader och moder ned till Timna; men just som de hade hunnit fram till vingårdarna vid Timna, kom ett ungt lejon rytande emot honom.
\par 6 Då föll HERRENS Ande över honom, och han slet sönder lejonet, såsom hade han slitit sönder en killing, fastän han icke hade någonting i sin hand; men han talade icke om för sin fader och moder vad han hade gjort.
\par 7 När han så kom ditned, talade han med kvinnan; och hon behagade Simson.
\par 8 En tid därefter vände han tillbaka för att hämta henne och vek då av vägen för att se på det döda lejonet; då fick han i lejonets kropp se en bisvärm med honung.
\par 9 Och han skrapade ut honungen i sina händer och åt därav, medan han gick, han kom så till sin fader och moder och gav dem, och de åto. Men han talade icke om för dem att det var ur lejonets kropp han hade skrapat honungen.
\par 10 När nu hans fader kom ned till kvinnan, gjorde Simson där ett gästabud, ty så plägade de unga männen göra.
\par 11 Och när de fingo se honom, skaffade de trettio bröllopssvenner, som skulle vara hos honom.
\par 12 Till dem sade Simson: "Jag vill förelägga eder en gåta; om I under de sju gästabudsdagarna sägen mig lösningen på den och gissen rätt, så skall jag giva eder trettio fina linneskjortor och trettio högtidsdräkter.
\par 13 Men om I icke kunnen säga mig lösningen, så skolen I giva mig trettio fina linneskjortor och trettio högtidsdräkter." De sade till honom: "Förelägg oss din gåta, låt oss höra den."
\par 14 Då sade han till dem: "Från storätaren utgick ätbart, från den grymme kom sötma." Men under tre dagar kunde de icke lösa gåtan.
\par 15 På sjunde dagen sade de då till Simsons hustru: "Locka din man till att säga oss lösningen på gåtan; eljest skola vi bränna upp dig och din faders hus i eld. Icke haven I väl bjudit oss hit för att utarma oss?"
\par 16 Då låg Simsons hustru över honom med gråt och sade: "Du hatar mig allenast och älskar mig alls icke; du har förelagt mina landsmän en gåta, men mig har du icke sagt lösningen på den." Han svarade henne: "Icke ens åt min fader eller min moder har jag sagt lösningen; skulle jag då säga den åt dig?"
\par 17 Men hon låg över honom med gråt under de sju dagar de höllo gästabudet. Och på sjunde dagen sade han henne lösningen, eftersom hon så hårt ansatte honom; sedan sade hon lösningen på gåtan åt sina landsmän.
\par 18 Innan solen gick ned på sjunde dagen, gåvo honom alltså männen i staden det svaret: "Vad är sötare än honung, och vad är grymmare än ett lejon?" Men han sade till dem: "Haden I icke plöjt med min kviga, så haden I icke gissat min gåta."
\par 19 Och HERRENS Ande kom över honom, och han gick ned till Askelon och slog där ihjäl trettio män och tog deras kläder och gav högtidsdräkterna åt dem som hade sagt lösningen på gåtan. Och hans vrede upptändes, och han vände tillbaka upp till sin faders hus.
\par 20 Då blev Simsons hustru given åt den av hans bröllopssvenner, som han hade haft till sin särskilda följesven.

\chapter{15}

\par 1 En tid därefter, medan veteskörden pågick, ville Simson besöka sin hustru, och förde med sig en killing. Och han sade: "Låt mig gå in till min hustru i kammaren." Men hennes fader ville icke tillstädja honom att gå in;
\par 2 hennes fader sade: "Jag höll för säkert att du hade fattat hat till henne, och därför gav jag henne åt din bröllopssven. Men hon har ju en yngre syster, som är fagrare än hon; tag denna i stället för den andra."
\par 3 Men Simson svarade dem: "Denna gång är jag utan skuld gent emot filistéerna, om jag gör dem något ont."
\par 4 Och Simson gick bort och fångade tre hundra rävar; sedan tog han facklor, band så ihop två och två rävar med svansarna och satte in en fackla mitt emellan de två svansarna.
\par 5 Därefter tände han eld på facklorna och släppte djuren in på filistéernas sädesfält och antände så både sädesskylar och oskuren säd, vingårdar och olivplanteringar.
\par 6 Då nu filistéerna frågade efter vem som hade gjort detta, fingo de det svaret: "Det har Simson, timnitens måg, därför att denne tog hans hustru och gav henne åt hans bröllopssven." Då drogo filistéerna åstad och brände upp både henne och hennes fader i eld.
\par 7 Men Simson sade till dem: "Om I beten eder så, skall jag sannerligen icke vila, förrän jag har tagit hämnd på eder."
\par 8 Och han for våldsamt fram med dem, så att de varken kunde gå eller stå. Sedan gick han ned därifrån och bodde i bergsklyftan vid Etam.
\par 9 Då drogo filistéerna upp och lägrade sig i Juda; och de spridde sig i Lehi.
\par 10 Och Juda män sade: "Varför haven I dragit hitupp mot oss?" De svarade: "Vi hava dragit hitupp för att binda Simson och för att göra mot honom såsom han har gjort mot oss."
\par 11 Då drogo tre tusen män från Juda ned till bergsklyftan vid Etam och sade till Simson: "Du vet ju att filistéerna råda över oss; huru har du då kunnat göra så mot oss?" Han svarade dem: "Såsom de hava gjort mot mig, så har jag gjort mot dem."
\par 12 De sade till honom: "Vi hava kommit hitned för att binda dig och sedan lämna dig i filistéernas hand." Simson sade till dem: "Så given mig nu eder ed på att I icke själva viljen stöta ned mig."
\par 13 De svarade honom: "Nej, vi vilja allenast binda dig och sedan lämna dig i deras hand, men vi skola icke döda dig." Så bundo de honom med två nya tåg och förde honom upp, bort ifrån klippan.
\par 14 När han nu kom till Lehi, skriade filistéerna och sprungo emot honom. Då kom HERRENS Ande över honom, och tågen omkring hans armar blevo såsom lintrådar, när de antändas av eld, och banden likasom smälte bort ifrån hans händer.
\par 15 Och han fick fatt i en åsnekäke som ännu var frisk; och han räckte ut sin hand och tog den, och med den slog han ihjäl tusen män.
\par 16 Sedan sade Simson: "Med åsnekäken slog jag en skara, ja, två; med åsnekäken slog jag tusen man."
\par 17 När han hade sagt detta, kastade han käken ifrån sig. Och man kallade den platsen Ramat-Lehi.
\par 18 Men då han därefter blev mycket törstig, ropade han till HERREN och sade: "Du själv har genom din tjänare givit denna stora seger; och nu måste jag dö av törst, och så falla i de oomskurnas hand!"
\par 19 Då lät Gud fördjupningen i Lehi öppna sig, och därur gick ut vatten, så att han kunde dricka; och hans ande kom tillbaka, och han fick liv igen. Därav kallades källan Den ropandes källa i Lehi, såsom den heter ännu i dag.
\par 20 Och han var domare i Israel under filistéernas tid, i tjugu år.

\chapter{16}

\par 1 Och Simson gick till Gasa; där fick han se en sköka och gick in till henne.
\par 2 När då gasiterna fingo höra att Simson hade kommit dit, omringade de platsen och lågo i försåt för honom hela natten vid stadsporten. Men hela natten höllo de sig stilla; de tänkte: "Vi vilja vänta till i morgon, när det bliver dager; då skola vi dräpa honom."
\par 3 Och Simson låg där intill midnatt; men vid midnattstiden stod han upp och grep tag i stadsportens dörrar och i de båda dörrposterna och ryckte loss dem jämte bommen, och lade alltsammans på sina axlar och bar upp det till toppen på det berg som ligger gent emot Hebron.
\par 4 Därefter fattade han kärlek till en kvinna som hette Delila, vid bäcken Sorek.
\par 5 Då kommo filistéernas hövdingar upp till henne och sade till henne: "Locka honom till att uppenbara för dig varav det beror att han är så stark, och huru vi skola bliva honom övermäktiga, så att vi kunna binda honom och kuva honom; vi vilja då giva dig ett tusen ett hundra siklar silver var."
\par 6 Då sade Delila till Simson: "Säg mig varav det beror att du är så stark, och huru man skulle kunna binda och kuva dig."
\par 7 Simson svarade henne: "Om man bunde mig med sju friska sensträngar, som icke hade hunnit torka, så bleve jag svag och vore såsom en vanlig människa."
\par 8 Då buro filistéernas hövdingar till henne sju friska sensträngar, som icke hade hunnit torka; och hon band honom med dem.
\par 9 Men hon hade lagt folk i försåt i den inre kammaren. Sedan ropade hon till honom: "Filistéerna äro över dig, Simson!" Då slet han sönder sensträngarna så lätt som en blångarnssnodd slites sönder, när den kommer intill elden. Alltså hade man ingenting fått veta om hans styrka.
\par 10 Då sade Delila till Simson: "Du har ju bedragit mig och ljugit för mig. Men säg mig nu huru man skulle kunna binda dig."
\par 11 Han svarade henne: "Om man bunde mig med nya tåg, som ännu icke hade blivit begagnade till något, så bleve jag svag och vore såsom en vanlig människa."
\par 12 Då tog Delila nya tåg och band honom med dem och ropade så till honom: "Filistéerna äro över dig, Simson!"; och folk låg i försåt i den inre kammaren. Men han slet tågen av sina armar, såsom hade det varit trådar.
\par 13 Då sade Delila till Simson: "Hittills har du bedragit mig och ljugit för mig; säg mig nu huru man skulle kunna binda dig." Han svarade henne: "Jo, om du vävde in de sju flätorna på mitt huvud i ränningen till din väv."
\par 14 Hon slog alltså fast dem med pluggen och ropade sedan till honom: "Filistéerna äro över dig, Simson!" När han då vaknade upp ur sömnen, ryckte han loss vävpluggen jämte ränningen till väven.
\par 15 Då sade hon till honom: "Huru kan du säga att du har mig kär, du som icke är uppriktig mot mig? Tre gånger har du nu bedragit mig och icke velat säga mig varpå det beror att du är så stark."
\par 16 Då hon nu dag efter dag hårt ansatte honom med denna sin begäran och plågade honom därmed, blev han så otålig att han kunde dö,
\par 17 och yppade så för henne hela sin hemlighet och sade till henne: "Ingen rakkniv har kommit på mitt huvud, ty jag är en Guds nasir allt ifrån min moders liv. Därför, om man rakar håret av mig, viker min styrka ifrån mig, så att jag bliver svag och är såsom alla andra människor."
\par 18 Då nu Delila insåg att han hade yppat för henne hela sin hemlighet, sände hon bud och kallade till sig filistéernas hövdingar; hon lät säga: "Kommen hitupp ännu en gång, ty han har nu yppat för mig hela sin hemlighet." Då kommo filistéernas hövdingar ditupp till henne och förde med sig penningarna.
\par 19 Nu lagade hon så, att han somnade in på hennes knän; och sedan hon hade kallat till sig en man som på hennes befallning skar av de sju flätorna på hans huvud, begynte hon att få makt över honom, och hans styrka vek ifrån honom.
\par 20 Därefter ropade hon: "Filistéerna äro över dig, Simson!" När han då vaknade upp ur sömnen, tänkte han: "Jag gör mig väl fri, nu såsom de förra gångerna, och skakar mig lös"; ty han visste icke att HERREN hade vikit ifrån honom.
\par 21 Men filistéerna grepo honom och stucko ut ögonen på honom. Därefter förde de honom ned till Gasa och bundo honom med kopparfjättrar, och han måste mala i fängelset.
\par 22 Men hans huvudhår begynte åter växa ut, sedan det hade blivit avrakat.
\par 23 Och filistéernas hövdingar församlade sig för att anställa en stor offerfest åt sin gud Dagon och göra sig glada, ty de sade: "Vår gud har givit vår fiende Simson i vår hand."
\par 24 Och när folket såg honom, lovade de likaledes sin gud och sade: "Vår gud har givit vår fiende i vår hand honom som förödde vårt land och slog så många av oss ihjäl."
\par 25 Då nu deras hjärtan hade blivit glada, sade de: "Låt hämta Simson, för att han må förlusta oss." Och Simson blev hämtad ur fängelset och måste vara dem till förlustelse. Och de hade ställt honom mellan pelarna.
\par 26 Men Simson sade till den gosse som höll honom vid handen: "Släpp mig och låt mig komma intill pelarna som huset vilar på, så att jag får luta mig mot dem."
\par 27 Och huset var fullt med män och kvinnor, och filistéernas alla hövdingar voro där; och på taket voro vid pass tre tusen män och kvinnor, som sågo på, huru Simson förlustade dem.
\par 28 Men Simson ropade till HERREN och sade: "Herre, HERRE, tänk på mig och styrk mig allenast denna gång, o Gud, så att jag får taga hämnd på filistéerna för ett av mina båda ögon."
\par 29 Därefter fattade Simson i de båda mittelpelare som huset vilade på, och tog fast tag mot dem; han fattade i den ena med högra handen och i den andra med vänstra.
\par 30 Och Simson sade: "Må jag nu själv dö med filistéerna." Sedan böjde han sig framåt med sådan kraft, att huset föll omkull över hövdingarna och allt folket som fanns där. Och de som han så dödade vid sin död voro flera än de som han hade dödat, medan han levde.
\par 31 Och hans bröder och hela hans familj kommo ditned och togo honom upp med sig och begrovo honom mellan Sorga och Estaol, i hans fader Manoas grav. Han hade då i tjugu år varit domare i Israel.

\chapter{17}

\par 1 I Efraims bergsbygd levde en man som hette Mika.
\par 2 Denne sade till sin moder: "De ett tusen ett hundra silversiklar som blevo dig fråntagna, och för vilkas skull du uttalade en förbannelse, som jag själv hörde, se, de penningarna finnas hos mig. Det var jag som tog dem." Då sade hans moder: "Välsignad vare du, min son, av HERREN!"
\par 3 Så gav han de ett tusen ett hundra silversiklarna tillbaka åt sin moder. Men hans moder sade: "Härmed helgar jag dessa penningar åt HERREN och lämnar dem åt min son, för att han må låta göra en skuren och en gjuten gudabild. Här lämnar jag dem nu tillbaka åt dig."
\par 4 Men han gav penningarna tillbaka åt sin moder. Då tog hans moder två hundra siklar silver och gav dem åt en guldsmed, och denne gjorde därav en skuren och en gjuten gudabild, vilka sedan ställdes in i Mikas hus.
\par 5 Mannen Mika hade så ett gudahus; han lät ock göra en efod och husgudar och insatte genom handfyllning en av sina söner till präst åt sig.
\par 6 På den tiden fanns ingen konung i Israel; var och en gjorde vad honom behagade.
\par 7 I Bet-Lehem i Juda levde då en ung man av Juda släkt; han var levit och bodde där såsom främling.
\par 8 Denne man vandrade bort ifrån sin stad, Bet-Lehem i Juda, för att se om han funne någon annan ort där han kunde bo; och under sin färd kom han till Efraims bergsbygd, fram till Mikas hus.
\par 9 Då frågade Mika honom: "Varifrån kommer du?" Han svarade honom: "Jag är en levit från Bet-Lehem i Juda, och jag är nu stadd på vandring, för att se om jag finner någon annan ort där jag kan bo."
\par 10 Mika sade till honom: "Stanna kvar hos mig, och bliv fader och präst åt mig, så skall jag årligen giva dig tio siklar silver och vad kläder du behöver, och därtill din föda." Då följde leviten med honom.
\par 11 Och leviten gick in på att stanna hos mannen, och denne behandlade den unge mannen såsom sin son.
\par 12 Och Mika insatte leviten genom handfyllning, så att den unge mannen blev präst åt honom; och han var sedan kvar i Mikas hus.
\par 13 Och Mika sade: "Nu vet jag att HERREN skall göra mig gott, eftersom jag har fått leviten till präst."

\chapter{18}

\par 1 På den tiden fanns ingen konung i Israel. Och på den tiden sökte sig daniternas stam en arvedel till att bo i, ty ända dittills hade icke något område tillfallit den såsom arvedel bland Israels övriga stammar.
\par 2 Så sände då Dans barn ur sin släkt fem män, uttagna bland dem, tappra män, från Sorga och Estaol, till att bespeja landet och undersöka det; och de sade till dem: "Gån åstad och undersöken landet." Så kommo de till Efraims bergsbygd, fram till Mikas hus; där stannade de över natten.
\par 3 När de nu voro vid Mikas hus och kände igen den unge levitens sätt att tala, gingo de fram till honom och frågade honom: "Vem har fört dig hit? Och vad gör du på detta ställe, och huru har du det här?"
\par 4 Han omtalade då för dem: "Så och så gjorde Mika med mig; han gav mig lön, och jag blev präst åt honom."
\par 5 Då sade de till honom: "Fråga då Gud, så att vi få veta om den resa som vi äro stadda på skall bliva lyckosam."
\par 6 Prästen svarade dem: "Gån i frid. Den resa som I ären stadda på står under HERRENS beskydd."
\par 7 Då gingo de fem männen vidare och kommo till Lais; och de sågo huru folket därinne bodde i trygghet, på sidoniernas sätt, stilla och trygga, och att ingen gjorde någon skada i landet genom att tillvälla sig makten; och de bodde långt ifrån sidonierna och hade intet att skaffa med andra människor.
\par 8 När de sedan kommo åter till sina bröder i Sorga och Estaol, frågade deras bröder dem: "Vad haven I att säga?"
\par 9 De svarade: "Upp, låt oss draga åstad mot dem! Ty vi hava besett landet och funnit det mycket gott. Skolen då I sitta stilla? Nej, varen ej sena till att tåga åstad, så att I kommen dit och intagen landet.
\par 10 När I kommen dit, kommen I till ett folk som känner sig tryggt, och landet har utrymme nog. Ja, Gud har givit det i eder hand - en ort där ingen brist är på något som jorden kan bära."
\par 11 Så bröto sex hundra man av daniternas släkt, omgjordade med vapen, upp därifrån, nämligen från Sorga och Estaol.
\par 12 De drogo upp och lägrade sig vid Kirjat-Jearim i Juda. Därför kallar man ännu i dag det stället för Dans läger; det ligger bakom Kirjat-Jearim.
\par 13 Därifrån drogo de vidare till Efraims bergsbygd och kommo så fram till Mikas hus.
\par 14 De fem män som hade varit åstad för att bespeja Lais' land togo då till orda och sade till sina bröder: "I mån veta att här i husen finnas en efod och husgudar och en skuren och en gjuten Gudabild. Så betänken nu vad I bören göra."
\par 15 Då drogo de ditfram och kommo till den unge levitens hus, till Mikas hus, och hälsade honom.
\par 16 Men de sex hundra männen av Dans barn ställde sig vid ingången till porten, omgjordade med sina vapen som de voro.
\par 17 Och de fem män som hade varit åstad för att bespeja landet gingo upp och kommo ditin och togo den skurna gudabilden och efoden, så ock husgudarna och den gjutna gudabilden, under det att prästen stod vid ingången till porten jämte de sex hundra vapenomgjordade männen.
\par 18 När nu de fem männen hade gått in i Mikas hus och tagit den skurna gudabilden med efoden och husgudarna och den gjutna gudabilden, sade prästen till dem: "Vad är det I gören!"
\par 19 De svarade honom: "Tig, lägg handen på din mun, och gå med oss och bliv fader och präst åt oss. Vilket är bäst för dig: att vara präst för en enskild mans hus eller att vara präst för en hel stam och släkt i Israel?"
\par 20 Då blev prästens hjärta glatt, och han tog emot efoden och husgudarna och den skurna gudabilden och slöt sig till folket.
\par 21 Sedan vände de sig åt annat håll och gingo vidare, och läto därvid kvinnor och barn och boskapen och det dyrbaraste godset föras främst i tåget.
\par 22 Men när Dans barn hade kommit ett långt stycke väg från Mikas hus, upphunnos de av de män som voro bosatta i närheten av Mikas hus, och som under tiden hade samlat sig.
\par 23 Vid dessas tillrop vände sig nu Dans barn om och frågade Mika: "Vad fattas dig, eftersom du kommer med en sådan hop?"
\par 24 Han svarade: "I haven tagit de gudar som jag har gjort åt mig, därtill ock prästen, och så gån I eder väg. Vad har jag nu mer kvar? Och ändå frågen I mig: 'Vad fattas dig?'!"
\par 25 Men Dans barn sade till honom: "Låt oss icke höra ett ord mer från dig. Eljest kan det hända att några män i förbittring hugga ned eder, och då bliver du orsak till att I förloren livet, både du själv och ditt husfolk."
\par 26 Därefter fortsatte Dans barn sin väg; och när Mika såg att de voro starkare än han, vände han om och drog tillbaka hem igen.
\par 27 Sedan de så hade tagit både vad Mika hade låtit förfärdiga och därtill hans präst, föllo de över folket i Lais, som levde stilla och i trygghet, och slogo dem med svärdsegg; men staden brände de upp i eld.
\par 28 Och ingen kunde komma den till hjälp, ty den låg långt ifrån Sidon, och folket däri hade intet att skaffa med andra människor; den låg i Bet-Rehobs dal. Sedan byggde de åter upp staden och bosatte sig där.
\par 29 Och de gåvo staden namnet Dan efter sin fader Dan, som var son till Israel; förut hade staden hetat Lais.
\par 30 Och Dans barn ställde där upp åt sig den skurna gudabilden; och Jonatan, son till Gersom, Manasses son, och hans söner voro präster åt daniternas stam, ända till dess att landets folk fördes bort i fångenskap.
\par 31 De ställde upp åt sig den skurna gudabild som Mika hade gjort, och de hade denna kvar under hela den tid Guds hus var i Silo.

\chapter{19}

\par 1 På den tiden, då ännu ingen konung fanns i Israel, bodde en levitisk man längst uppe i Efraims bergsbygd. Denne tog till bihustru åt sig en kvinna från Bet-Lehem i Juda.
\par 2 Men hans bihustru blev honom otrogen och gick ifrån honom till sin faders hus i Bet-Lehem i Juda; där uppehöll hon sig en tid av fyra månader.
\par 3 Då stod hennes man upp och begav sig åstad efter henne, för att tala vänligt med henne och så föra henne tillbaka; och han hade med sig sin tjänare och ett par åsnor. Hon förde honom då in i sin faders hus, och när kvinnans fader fick se honom, gick han glad emot honom.
\par 4 Och hans svärfader, kvinnans fader, höll honom kvar, så att han stannade hos honom i tre dagar; de åto och drucko och voro där nätterna över.
\par 5 När de nu på fjärde dagen stodo upp bittida om morgonen och han gjorde sig redo att resa, sade kvinnans fader till sin måg: "Vederkvick dig med ett stycke bröd; sedan mån I resa."
\par 6 Då satte de sig ned och åto båda tillsammans och drucko. Därefter sade kvinnans fader till mannen: "Beslut dig för att stanna här över natten, och låt ditt hjärta vara glatt."
\par 7 Och när mannen ändå gjorde sig redo att resa, bad hans svärfader honom så enträget, att han ännu en gång stannade kvar där över natten.
\par 8 På femte dagen stod han åter upp bittida om morgonen för att resa; då sade kvinnans fader: "Vederkvick dig först, och dröjen så till eftermiddagen." Därefter åto de båda tillsammans.
\par 9 När sedan mannen gjorde sig redo att resa med sin bihustru och sin tjänare, sade hans svärfader, kvinnans fader, till honom: "Se, det lider mot aftonen; stannen kvar över natten, dagen nalkas ju sitt slut; ja, stanna kvar här över natten, och låt ditt hjärta vara glatt. Sedan kunnen I i morgon bittida företaga eder färd, så att du får komma hem till din hydda."
\par 10 Men mannen ville icke stanna över natten, utan gjorde sig redo och reste sin väg, och kom så fram till platsen mitt emot Jebus, det är Jerusalem. Och han hade med sig ett par sadlade åsnor; och hans bihustru följde honom.
\par 11 Då de nu voro vid Jebus och dagen var långt framliden, sade tjänaren till sin herre: "Kom, låt oss taga in i denna jebuséstad och stanna där över natten."
\par 12 Men hans herre svarade honom: "Vi skola icke taga in i en främmande stad, där inga israeliter bo; låt oss draga vidare, fram till Gibea."
\par 13 Och han sade ytterligare till sin tjänare: "Kom, låt oss försöka hinna fram till en av orterna här och stanna över natten i Gibea eller Rama."
\par 14 Så drogo de vidare; och när de voro invid Gibea i Benjamin, gick solen ned.
\par 15 Då togo de in där och kommo för att stanna över natten i Gibea. Och när mannen kom ditin, satte han sig på den öppna platsen i staden, men ingen ville taga emot dem i sitt hus över natten.
\par 16 Men då, om aftonen, kom en gammal man från sitt arbete på fältet, och denne man var från Efraims bergsbygd och bodde såsom främling i Gibea; ty folket där på orten voro benjaminiter.
\par 17 När denne nu lyfte upp sina ögon, fick han se den vägfarande mannen på den öppna platsen i staden. Då sade den gamle mannen: "Vart skall du resa, och varifrån kommer du?"
\par 18 Han svarade honom: "Vi äro på genomresa från Bet-Lehem i Juda till den del av Efraims bergsbygd, som ligger längst uppe; därifrån är jag, och jag har gjort en resa till Bet-Lehem i Juda. Nu är jag på väg till HERRENS hus, men ingen vill här taga emot mig i sitt hus.
\par 19 Jag har både halm och foder åt våra åsnor, så ock bröd och vin åt mig själv och åt din tjänarinna och åt mannen som åtföljer oss, dina tjänare, så att intet fattas oss."
\par 20 Då sade den gamle mannen: "Frid vare med dig! Men låt mig få sörja för allt som kan fattas dig. Härute på den öppna platsen må du icke stanna över natten."
\par 21 Därefter förde han honom till sitt hus och fodrade åsnorna. Och sedan de hade tvått sina fötter, åto de och drucko.
\par 22 Under det att de så gjorde sina hjärtan glada, omringades plötsligt huset av männen i staden, onda män, som bultade på dörren; och de sade till den gamle mannen, som rådde om huset: "För hitut den man som har kommit till ditt hus, så att vi få känna honom."
\par 23 Då gick mannen som rådde om huset ut till dem och sade till dem: "Nej, mina bröder, gören icke så illa. Eftersom nu denne man har kommit in i mitt hus, mån I icke göra en sådan galenskap.
\par 24 Se, jag har en dotter som är jungfru, och han har själv en bihustru. Dem vill jag föra hitut, så kunnen I kränka dem och göra med dem vad I finnen för gott. Men med denne man mån I icke göra någon sådan galenskap.
\par 25 Men männen ville icke höra på honom; då tog mannen sin bihustru och förde henne ut till dem. Och de kände henne och hanterade henne skändligt hela natten ända till morgonen; först när morgonrodnaden gick upp, läto de henne gå.
\par 26 Då kom kvinnan mot morgonen och föll ned vid ingången till mannens hus, där hennes herre var, och låg så, till dess det blev dager.
\par 27 När nu hennes herre stod upp om morgonen och öppnade dörren till huset och gick ut för att fortsätta sin färd, fick han se sin bihustru ligga vid ingången till huset med händerna på tröskeln.
\par 28 Han sade till henne: "Stå upp och låt oss gå." Men hon gav intet svar. Då tog han och lade henne på åsnan; sedan gjorde mannen sig redo och reste hem till sitt.
\par 29 Men när han hade kommit hem, fattade han en kniv och tog sin bihustru och styckade henne, efter benen i hennes kropp, i tolv stycken och sände styckena omkring över hela Israels land.
\par 30 Och var och en som såg detta sade: "Något sådant har icke hänt eller blivit sett allt ifrån den dag då Israels barn drogo upp ur Egyptens land ända till denna dag. Övervägen detta, rådslån och sägen edert ord."

\chapter{20}

\par 1 Då drogo alla Israels barn ut, och menigheten församlade sig såsom en man, från Dan ända till Beer-Seba, så ock från Gileads land, inför HERREN i Mispa.
\par 2 Och de förnämsta i hela folket alla Israels stammar, trädde fram i Guds folks församling: fyra hundra tusen svärdbeväpnade mån till fots.
\par 3 Men Benjamins barn fingo höra att de övriga israeliterna hade dragit upp till Mispa. Och Israels barn sade: "Omtalen huru denna ogärning har tillgått."
\par 4 Då tog den levitiske mannen, den mördade kvinnans man, till orda och sade: "Jag och min bihustru kommo till Gibea i Benjamin för att stanna där över natten.
\par 5 Då blev jag överfallen av Gibeas borgare; de omringade huset om natten för att våldföra sig på mig. Mig tänkte de dräpa, och min bihustru kränkte de, så att hon dog.
\par 6 Då tog jag min bihustru och styckade henne och sände styckena omkring över Israels arvedels hela område, eftersom de hade gjort en sådan skändlighet och galenskap i Israel.
\par 7 Se, nu ären I allasammans här, I Israels barn. Läggen nu fram förslag och råd här på stället."
\par 8 Då stod allt folket upp såsom en man och sade: "Ingen av oss må gå hem till sin hydda, ingen må begiva sig hem till sitt hus.
\par 9 Detta är vad vi nu vilja göra med Gibea: vi skola låta lotten gå över det.
\par 10 På vart hundratal i alla Israels stammar må vi taga ut tio män, och på vart tusental hundra, och på vart tiotusental tusen, för att dessa må skaffa munförråd åt folket, så att folket, när det kommer till Geba i Benjamin, kan göra med staden såsom tillbörligt är för all den galenskap som den har gjort i Israel."
\par 11 Så församlade sig vid staden alla män i Israel, endräktigt såsom en man.
\par 12 Och Israels stammar sände åstad män till alla Benjamins stammar och läto säga; "Vad är det för en ogärning som har blivit begången ibland eder!
\par 13 Lämnen nu ut de onda män som bo i Gibea, så att vi få döda dem och skaffa bort ifrån Israel vad ont är." Men benjaminiterna ville icke lyssna till sina bröders, de övriga israeliternas, ord.
\par 14 I stället församlade sig Benjamins barn från sina städer till Gibea, för att draga ut till strid mot de övriga israeliterna.
\par 15 På den dagen mönstrades Benjamins barn, de utgjorde från dessa städer tjugusex tusen svärdbeväpnade män; vid denna mönstring medräknades icke de som bodde i Gibea, vilka utgjorde sju hundra utvalda män.
\par 16 Bland allt detta folk funnos sju hundra utvalda män som voro vänsterhänta; alla dessa kunde med slungstenen träffa på håret, utan att fela.
\par 17 Och när Israels män - Benjamin frånräknad - mönstrades, utgjorde de fyra hundra tusen svärdbeväpnade män; alla dessa voro krigsmän.
\par 18 Dessa bröto nu upp och drogo åstad till Betel och frågade Gud. Israels barn sade: "Vem bland oss skall först draga ut i striden mot Benjamins barn?" HERREN svarade: "Juda först."
\par 19 Då bröto Israels barn upp följande morgon och lägrade sig framför Gibea.
\par 20 Därefter drogo Israels män ut till strid mot Benjamin; Israels män ställde upp sig till strid mot dem vid Gibea.
\par 21 Men Benjamins barn drogo ut ur Gibea och nedgjorde på den dagen tjugutvå tusen man av Israel.
\par 22 Folket, Israels män, tog dock åter mod till sig och ställde upp sig ännu en gång till strid på samma plats där de hade ställt upp sig första dagen.
\par 23 Israels barn gingo nämligen upp och gräto inför HERRENS ansikte ända till aftonen; och de frågade HERREN: "Skall jag ännu en gång inlåta mig i strid med min broder Benjamins barn?" Och HERREN svarade: "Dragen ut mot honom."
\par 24 När så Israels barn dagen därefter ryckte fram mot Benjamins barn,
\par 25 drog ock Benjamin på andra dagen ut från Gibea mot Israels barn och nedgjorde av dem ytterligare aderton tusen man, allasammans svärdbeväpnade män.
\par 26 Då drogo alla Israels barn upp, allt folket, och kommo till Betel och gräto och stannade där inför HERRENS ansikte och fastade på den dagen ända till aftonen; och de offrade brännoffer och tackoffer inför HERRENS ansikte.
\par 27 Och Israels barn frågade HERREN (ty Guds förbundsark stod på den tiden där,
\par 28 och Pinehas, son till Eleasar, Arons son, gjorde tjänst inför den på den tiden); de sade: "Skall jag ännu en gång draga ut till strid mot min broder Benjamins barn, eller skall jag avstå därifrån?" HERREN svarade: "Dragen upp; ty i morgon skall jag giva honom i din hand."
\par 29 Då lade Israel manskap i bakhåll mot Gibea, runt omkring det.
\par 30 Och därefter drogo Israels barn upp mot Benjamins barn, på tredje dagen, och ställde upp sig i slagordning mot Gibea likasom de förra gångerna.
\par 31 Och Benjamins barn drogo ut mot folket och blevo lockade långt bort ifrån staden; och likasom det hade skett de förra gångerna, fingo de i början slå ihjäl några av folket på vägarna (både på den som går upp till Betel och på den som går till Gibea över fältet), kanhända ett trettiotal av Israels män.
\par 32 Då tänkte Benjamins barn: "De äro slagna av oss, nu likasom förut." Men Israels barn hade träffat det avtalet: "Vi vilja fly och så locka dem långt bort ifrån staden, ut på vägarna.
\par 33 Och alla Israels män hade brutit upp från platsen där de voro, och hade ställt upp sig i slagordning vid Baal-Tamar, under det att de israeliter som lågo i bakhåll bröto fram ifrån sin plats vid Maare-Geba.
\par 34 Så kommo då tio tusen man, utvalda ur hela Israel, fram gent emot Gibea, och striden blev hård, utan att någon visste att olyckan var dem så nära.
\par 35 Och HERREN lät Benjamin bliva slagen av Israel, och Israels barn nedgjorde av Benjamin på den dagen tjugufem tusen ett hundra man, allasammans svärdbeväpnade män.
\par 36 Nu sågo Benjamins barn att de voro slagna. Israels män gåvo nämligen plats åt Benjamin, ty de förlitade sig på bakhållet som de hade lagt mot Gibea.
\par 37 Då skyndade sig de som lågo i bakhåll att falla in i Gibea; de som lågo i bakhåll drogo åstad och slogo alla invånarna i staden med svärdsegg.
\par 38 Men de övriga israeliterna hade träffat det avtalet med dem som lågo i bakhåll, att dessa skulle låta en tjock rök såsom tecken stiga upp från staden.
\par 39 Israels män vände alltså ryggen i striden. Men sedan Benjamin i början hade fått slå ihjäl några av Israels man, kanhända ett trettiotal, och därvid hade tänkt: "Förvisso äro de slagna av oss, nu likasom i den förra striden",
\par 40 då kommo de att vända sig om, vid det att rökpelaren, det avtalade tecknet, begynte stiga upp från staden. Och de fingo nu se hela staden förvandlad i lågor som slogo upp mot himmelen.
\par 41 När då Israels män åter vände om, blevo Benjamins män förskräckta, ty nu sågo de att olyckan var dem nära.
\par 42 Och de vände om för Israels män, bort åt öknen till, men fienderna hunno upp dem; och de som bodde i städerna där nedgjorde dem som hade kommit mitt emellan.
\par 43 De omringade benjaminiterna, de satte efter dem och trampade ned dem på deras viloplats, ända fram emot Gibea, österut.
\par 44 Så föllo av Benjamin aderton tusen man, allasammans tappert folk.
\par 45 Då vände de övriga sig mot öknen och flydde dit, till Rimmons klippa; men de andra gjorde en efterskörd bland dem på vägarna, fem tusen man, och satte så efter dem ända till Gideom och slogo av dem två tusen man.
\par 46 Alltså utgjorde de som på den dagen föllo av Benjamin tillsammans tjugufem tusen svärdbeväpnade män; alla dessa voro tappert folk.
\par 47 Men av dem som vände sig mot öknen och flydde dit, till Rimmons klippa, hunno sex hundra man ditfram; dessa stannade på Rimmons klippa i fyra månader.
\par 48 Emellertid vände Israels män tillbaka till Benjamins barn och slogo dem med svärdsegg, både dem av stadens befolkning, som ännu voro oskadda, och jämväl boskapen, korteligen, allt vad de träffade på; därtill satte de eld på alla städer som de träffade på.

\chapter{21}

\par 1 Men Israels män hade svurit i Mispa och sagt: "Ingen av oss skall giva sin dotter till hustru åt någon benjaminit."
\par 2 Och nu kom folket till Betel och stannade där ända till aftonen inför Guds ansikte; och de brusto ut i bitter gråt
\par 3 och sade: "Varför, o HERRE, Israels Gud, har sådant fått ske i Israel, att i dag en hel stam fattas i Israel?"
\par 4 Dagen därefter stod folket bittida upp och byggde där ett altare och offrade brännoffer och tackoffer.
\par 5 och Israels barn sade: "Finnes någon bland Israels alla stammar, som icke kom upp till HERREN med den övriga församlingen?" Ty man hade svurit en dyr ed, att den som icke komme upp till HERREN i Mispa skulle straffas med döden.
\par 6 Och Israels barn ömkade sig över sin broder Benjamin och sade: "Nu har en hel stam blivit borthuggen från Israel.
\par 7 Vad skola vi göra för dem som äro kvar, så att de kunna få hustrur? Ty själva hava vi ju svurit vid HERREN att icke åt dem giva hustrur av våra döttrar."
\par 8 Och då frågade åter: "Finnes bland Israels stammar någon som icke kom upp till HERREN i Mispa?" Och se, från Jabes i Gilead hade ingen kommit till lägret, till församlingen där.
\par 9 Ty när folket mönstrades, befanns det att ingen av invånarna i Jabes i Gilead var där.
\par 10 Då sände menigheten dit tolv tusen av de tappraste männen och bjöd dessa och sade: "Gån åstad och slån invånarna i Jabes i Gilead med svärdsegg, också kvinnor och barn.
\par 11 Ja, detta är vad I skolen göra: allt mankön och alla de kvinnor som hava haft med mankön att skaffa skolen I giva till spillo.
\par 12 Men bland invånarna i Jabes i Gilead funno de fyra hundra unga kvinnor som voro jungfrur och icke hade haft med män, med mankön, att skaffa. Dessa förde de då till lägret i Silo i Kanaans land.
\par 13 Sedan sände hela menigheten åstad och underhandlade med de benjaminiter som befunno sig på Rimmons klippa, och tillbjöd dem fred.
\par 14 Så vände nu Benjamin tillbaka; och man gav dem till hustrur de kvinnor från Jabes i Gilead, som man hade låtit leva. Men dessa räckte ingalunda till för dem.
\par 15 Och folket ömkade sig över Benjamin, eftersom HERREN hade gjort en rämna bland Israels stammar.
\par 16 Och de äldste i menigheten sade: "Vad skola vi göra med dem som äro kvar, så att de kunna få hustrur? Ty alla kvinnor äro ju utrotade ur Benjamin."
\par 17 Och de sade ytterligare: "De undkomna av Benjamin måste få en besittning, så att icke en stam bliver utplånad ur Israel.
\par 18 Men själva kunna vi icke åt dem giva hustrur av våra döttrar, ty Israels barn hava svurit och sagt: Förbannad vare den som giver en hustru åt Benjamin."
\par 19 Och de sade vidare: "En HERREN högtid plägar ju hållas år efter år i Silo, som ligger norr om Betel, öster om den väg som går från Betel upp till Sikem, och söder om Lebona."
\par 20 Och de bjödo Benjamins barn och sade: "Gån åstad och läggen eder i försåt i vingårdarna.
\par 21 När I då fån se Silos döttrar komma ut för att uppföra sina dansar, skolen I komma fram ur vingårdarna, och var och en av eder skall bland Silos döttrar rycka till sig en som kan bliva hans hustru; därefter skolen I begiva eder hem till Benjamins land.
\par 22 Om sedan deras fäder eller deras bröder komma och beklaga sig för oss, vilja vi säga till dem: 'Förunnen oss dem; ty ingen av oss har tagit sig någon hustru i kriget. I haven ju då icke själva givit dem åt dessa; ty i sådant fall haden I ådragit eder skuld.'"
\par 23 Och Benjamins barn gjorde så och skaffade sig hustrur, lika många som de själva voro, bland de dansande kvinnor som de rövade. Sedan begåvo de sig tillbaka till sin arvedel och byggde åter upp städerna och bosatte sig i dem.
\par 24 Också de övriga israeliterna begåvo sig bort därifrån, var och en till sin stam och sin släkt, och drogo ut därifrån, var och en till sin arvedel.
\par 25 På den tiden fanns ingen konung i Israel; var och en gjorde vad honom behagade.


\end{document}