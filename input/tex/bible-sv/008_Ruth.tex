\begin{document}

\title{Rut}


\chapter{1}

\par 1 På den tid då domarna regerade uppstod hungersnöd i landet. Då drog en man från Bet-Lehem i Juda åstad med sin hustru och sina båda söner för att bosätta sig i Moabs land under någon tid.
\par 2 Mannen hette Elimelek, hans hustru Noomi, och hans båda söner Mahelon och Kiljon; och de voro efratiter, från Bet-Lehem i Juda. Så kommo de nu till Moabs land och vistades där.
\par 3 Och Elimelek, Noomis man, dog; men hon levde kvar med sina båda söner.
\par 4 Dessa skaffade sig moabitiska hustrur; den ena hette Orpa och den andra Rut.
\par 5 Och sedan de hade bott där vid pass tio år, dogo också de båda, Mahelon och Kiljon; men kvinnan levde kvar efter sina båda söner och sin man.
\par 6 Då stod hon upp med sina sonhustrur för att vända tillbaka från Moabs land; ty hon hade hört i Moabs land att HERREN hade sett till sitt folk och givit det bröd.
\par 7 Så begav hon sig, jämte sina båda sonhustrur, från det ställe där hon hade vistats. Men när de nu gingo sin väg fram, för att komma tillbaka till Juda land,
\par 8 sade Noomi till sina båda sonhustrur: "Vänden om och gå hem igen, var och en till sin moder.
\par 9 HERREN bevise godhet mot eder, såsom I haven gjort mot de båda döda och mot mig. HERREN give eder att I mån finna ro, var i sin mans hus." Därefter kysste hon dem. Men de brusto ut i gråt
\par 10 och sade till henne: "Nej, vi vilja följa med dig tillbaka till ditt folk."
\par 11 Men Noomi svarade: "Vänden om, mina döttrar. Varför skullen I gå med mig? Kan väl jag ännu en gång få söner i mitt liv, vilka kunna bliva män åt eder?
\par 12 Vänden om, mina döttrar, och gån hem, ty jag är nu för gammal att överlämna mig åt en man. Och om jag än kunde tänka: 'Jag har ännu hopp', ja, om jag ock redan i natt överlämnade mig åt en man och så verkligen födde söner,
\par 13 icke skullen I därför vänta, till dess att de hade blivit fullvuxna, icke skullen I därför stänga eder inne och förbliva utan män? Bort det, mina döttrar! Jag känner redan bedrövelse nog för eder skull, eftersom HERRENS hand så har drabbat mig."
\par 14 Då brusto de åter ut i gråt. Och Orpa kysste sin svärmoder till avsked, men Rut höll sig alltjämt intill henne.
\par 15 Då sade hon: "Se, din svägerska har vänt tillbaka till sitt folk och till sin gud; vänd ock du tillbaka och följ din svägerska."
\par 16 Men Rut svarade: "Sök icke intala mig att övergiva dig och vända tillbaka ifrån dig. Ty dit du går vill ock jag gå, och där du stannar vill ock jag stanna. Ditt folk är mitt folk, och din Gud är min Gud.
\par 17 Där du dör vill ock jag dö, och där vill jag bliva begraven. HERREN straffe mig nu och framgent, om något annat än döden kommer att skilja mig från dig."
\par 18 Då hon nu såg att denna stod fast i sitt beslut att gå med henne, upphörde hon att tala därom med henne.
\par 19 Så gingo de båda med varandra, till dess att de kommo till Bet-Lehem. Och när de kommo till Bet-Lehem, kom hela staden i rörelse för deras skull, och kvinnorna sade: "Detta är ju Noomi!"
\par 20 Men hon sade till dem: "Kallen mig icke Noomi, utan kallen mig Mara, ty den Allsmäktige har låtit mycken bedrövelse komma över mig.
\par 21 Rik drog jag härifrån, och tomhänt har HERREN låtit mig komma tillbaka. Varför kallen I mig då Noomi, när HERREN har vittnat emot mig, när den Allsmäktige har låtit det gå mig så illa?"
\par 22 Så kom då Noomi tillbaka med sin sonhustru, moabitiskan Rut, i det hon vände tillbaka från Moabs land. Och de kommo till Bet-Lehem, när kornskörden begynte.

\chapter{2}

\par 1 Men Noomi hade en frände på sin mans sida, en rik man av Elimeleks släkt, vid namn Boas.
\par 2 Och moabitiskan Rut sade till Noomi: "Låt mig gå ut på åkern och plocka ax efter någon inför vilkens ögon jag finner nåd." Hon svarade henne: "Ja, gå, min dotter."
\par 3 Då gick hon åstad och kom till en åker och plockade där ax efter skördemännen; och det hände sig så för henne att åkerstycket tillhörde Boas, som var av Elimeleks släkt.
\par 4 Och Boas kom just då dit från Bet-Lehem; och han sade till skördemännen: "HERREN vare med eder." De svarade honom: "HERREN välsigne dig."
\par 5 Och Boas frågade den bland tjänarna, som hade uppsikt över skördemännen: "Vem tillhör den unga kvinnan där?"
\par 6 Tjänaren som hade uppsikt över skördemännen svarade och sade: "Det är en moabitisk kvinna, den kvinna som med Noomi har kommit hit från Moabs land.
\par 7 Hon bad att hon skulle få plocka och hopsamla ax bland kärvarna, efter skördemännen; och så kom hon, och hon har hållit på allt sedan i morse ända till denna stund, utom att hon nyss har vilat något litet därinne."
\par 8 Då sade Boas till Rut: "Hör, min dotter: du skall icke gå bort och plocka ax på någon annan åker, ej heller gå härifrån, utan du skall hålla dig till mina tjänarinnor här.
\par 9 Se efter, var skördemännen arbeta på åkern, och gå efter dem; jag har förbjudit mina tjänare att göra dig något för när. Och om du bliver törstig, så gå till kärlen och drick av det som mina tjänare hämta."
\par 10 Då föll hon ned på sitt ansikte och bugade sig mot jorden och sade till honom: "Varför har jag funnit sådan nåd för dina ögon, att du tager dig an mig, fastän jag är en främling?"
\par 11 Boas svarade och sade till henne: "För mig har blivit berättat allt vad du har gjort mot din svärmoder efter din mans död, huru du har övergivit din fader och din moder och ditt fädernesland, och vandrat åstad till ett folk som du förut icke kände.
\par 12 HERREN vedergälle dig för vad du har gjort; ja, må full lön tillfalla dig från HERREN, Israels Gud, till vilken du har kommit, för att finna tillflykt under hans vingar."
\par 13 Hon sade: "Så må jag då finna nåd för dina ögon, min herre; ty du har tröstat mig och talat vänligt med din tjänarinna, fastän jag icke är såsom någon av dina tjänarinnor."
\par 14 Och när måltidsstunden var inne, sade Boas till henne: "Kom hitfram och ät av brödet och doppa ditt brödstycke i vinet." Då satte hon sig vid sidan av skördemännen; och han lade för henne rostade ax, och hon åt och blev mätt och fick därtill över.
\par 15 Och när hon därefter stod upp för att plocka ax, bjöd Boas sina tjänare och sade: "Låten henne ock få plocka ax mellan kärvarna, och förfördelen henne icke.
\par 16 Ja, I mån till och med draga ut strån ur knipporna åt henne och låta dem ligga, så att hon får plocka upp dem, och ingen må banna henne därför."
\par 17 Så plockade hon ax på åkern ända till aftonen; och när hon klappade ut det som hon hade plockat, var det vid pass en efa korn.
\par 18 Och hon tog sin börda och gick in i staden, och hennes svärmoder fick se vad hon hade plockat. Därefter tog hon fram och gav henne vad hon hade fått över, sedan hon hade ätit sig mätt.
\par 19 Då sade hennes svärmoder till henne: "Var har du i dag plockat ax, och var har du arbetat? Välsignad vare han som har tagit sig an dig!" Då berättade hon för sin svärmoder hos vem hon hade arbetat; hon sade: "Den man som jag i dag har arbetat hos heter Boas."
\par 20 Då sade Noomi till sin sonhustru: "Välsignad vare han av HERREN, därför att han icke har undandragit sig att bevisa godhet både mot de levande och mot de döda!" Och Noomi sade ytterligare till henne: "Den mannen är vår nära frände, en av våra bördemän."
\par 21 Moabitiskan Rut sade: "Han sade ock till mig: 'Håll dig till mina tjänare, ända till dess att de hava inbärgat hela min skörd.'"
\par 22 Då sade Noomi till sin sonhustru Rut: "Ja, det är bäst, min dotter, att du går med hans tjänarinnor, så att man icke behandlar dig illa, såsom det kunde ske på en annan åker."
\par 23 Så höll hon sig då till Boas' tjänarinnor och plockade ax där, till dess både korn- och veteskörden voro avslutade. Men hon bodde hos sin svärmoder.

\chapter{3}

\par 1 Och hennes svärmoder Noomi sade till henne: "Min dotter, jag vill söka skaffa dig ro, för att det må gå dig väl.
\par 2 Så hör då: Boas, med vilkens tjänarinnor du har varit tillsammans, är ju vår frände. Och just i natt kastar han korn på sin tröskplats.
\par 3 Så två dig nu och smörj dig och kläd dig, och gå ned till tröskplatsen. Men laga så, att mannen icke får se dig, förrän han har ätit och druckit.
\par 4 När han då lägger sig, så se efter, var han lägger sig, och gå dit och lyft upp täcket vid hans fötter och lägg dig där; han skall då själv säga dig vad du bör göra."
\par 5 Hon svarade: "Allt vad du säger vill jag göra."
\par 6 Och hon gick ned till tröskplatsen och gjorde alldeles såsom hennes svärmoder hade bjudit henne.
\par 7 Ty när Boas hade ätit och druckit, så att hans hjärta blev glatt, och sedan han hade gått åstad och lagt sig invid sädeshögen, kom hon oförmärkt och lyfte upp täcket vid hans fötter och lade sig där.
\par 8 Vid midnattstiden blev mannen uppskrämd och böjde sig framåt och fick då se en kvinna ligga vid hans fötter.
\par 9 Och han sade: "Vem är du?" Hon svarade: "Jag är Rut, din tjänarinna. Bred ut din mantelflik över din tjänarinna, ty du är min bördeman."
\par 10 Då sade han: "Välsignad vare du av HERREN, min dotter! Du har nu givit ett större bevis på din kärlek än förut, därigenom att du icke har lupit efter unga män, vare sig fattiga eller rika.
\par 11 Så frukta nu icke, min dotter; allt vad du säger vill jag göra dig. Ty allt folket i min stad vet att du är en rättskaffens kvinna.
\par 12 Nu är det visserligen sant att jag är din bördeman; men en annan bördeman finnes, som är närmare än jag.
\par 13 Stanna nu kvar i natt; om han i morgon vill taga dig efter bördesrätt, gott, må han då göra det, så sant HERREN lever. Ligg nu kvar ända till morgonen."
\par 14 Så låg hon vid hans fötter ända till morgonen, men hon fick stå upp, innan ännu någon kunde känna igen den andre; ty han tänkte: "Det får icke bliva känt att kvinnan har kommit hit till tröskplatsen."
\par 15 Och han sade: "Räck hit manteln som du har på dig, och håll fram den." Och hon höll fram den. Då mätte han upp sex mått korn och gav henne att bära; därefter gick hon in i staden.
\par 16 Och när hon kom till sin svärmoder, sade denna: "Huru har det gått för dig, min dotter?" Då berättade hon för henne allt vad mannen hade gjort mot henne;
\par 17 och hon sade: "Dessa sex mått korn gav han mig, i det han sade: 'Du skall icke komma tomhänt hem till din svärmoder.'"
\par 18 Då svarade hon: "Bida, min dotter, till dess du får se huru saken avlöper; ty mannen skall icke giva sig till ro, med mindre han i dag för saken till sitt slut."

\chapter{4}

\par 1 Och Boas hade gått upp till stadsporten och satt sig där. Då hände sig att den bördeman som Boas hade talat om gick där fram; då nämnde han honom vid namn och sade: "Kom hit och sätt dig här." Och han kom och satte sig.
\par 2 Därefter tog Boas till sig tio män av de äldste i staden och sade: "Sätten eder här." Och de satte sig.
\par 3 Sedan sade han till bördemannen: "Det åkerstycke som tillhörde vår broder Elimelek har Noomi sålt, hon som kom tillbaka från Moabs land.
\par 4 Därefter tänkte jag att jag skulle underrätta dig därom och säga: Köp det inför dem som här sitta och inför mitt folks äldste. Om du vill taga det efter bördesrätt, så säg mig det, så att jag får veta det, ty ingen annan äger bördesrätt än du och, näst dig, jag själv." Han sade: "Jag vill taga det efter bördesrätt."
\par 5 Då sade Boas: "När du köper åkern av Noomis hand, då köper du den ock av moabitiskan Rut, den dödes hustru, med skyldighet att uppväcka den dödes namn och fästa det vid hans arvedel."
\par 6 Bördemannen svarade: "Då kan jag icke begagna mig av min bördesrätt, ty jag skulle därmed fördärva min egen arvedel. Börda du åt dig vad jag skulle hava bördat, ty jag kan icke göra det."
\par 7 Men när någon bördade något eller avtalade ett byte, var det fordom sed i Israel att han, till stadfästelse av ett sådant avtal, drog av sig sin sko och gav den åt den andre; och detta gällde såsom ett vittnesbörd i Israel.
\par 8 Så sade nu bördemannen till Boas: "Köp du det"; och han drog därvid av sig sin sko.
\par 9 Då sade Boas till de äldste och till allt folket: "I ären i dag vittnen till att jag nu har köpt av Noomis hand allt vad som har tillhört Kiljon och Mahelon.
\par 10 Därjämte har jag ock köpt moabitiskan Rut, Mahelons hustru, till hustru åt mig, för att uppväcka den dödes namn och fästa det vid hans arvedel, på det att den dödes namn icke må bliva utrotat bland hans bröder eller ur porten till hans stad. I ären i dag vittnen härtill."
\par 11 Och allt folket i stadsporten, så ock de äldste, svarade: "Ja, och HERREN låte den kvinna som nu går in i ditt hus bliva lik Rakel och Lea, de båda som hava byggt upp Israels hus. Och må du förkovra dig storligen i Efrata och göra dig ett namn i Bet-Lehem.
\par 12 Och blive ditt hus såsom Peres' hus, hans som Tamar födde åt Juda, genom de avkomlingar som HERREN skall giva dig med denna unga kvinna."
\par 13 Så tog då Boas Rut till sig, och hon blev hans hustru, och han gick in till henne; och HERREN gav henne livsfrukt, och hon födde en son.
\par 14 Då sade kvinnorna till Noomi: "Lovad vare HERREN, som i dag har så gjort, att det icke fattas dig en bördeman som skall få ett namn i Israel!
\par 15 Han skall bliva dig en tröstare och en försörjare på din ålderdom; ty din sonhustru, som har dig kär, har fött honom, hon som är mer för dig än sju söner."
\par 16 Och Noomi tog barnet och lade det i sin famn och blev dess sköterska.
\par 17 Och grannkvinnorna sade: "Noomi har fått en son"; och de gåvo honom namn, de kallade honom Obed. Han blev fader till Isai, Davids fader.
\par 18 Och detta är Peres' släktregister: Peres födde Hesron;
\par 19 Hesron födde Ram; Ram födde Amminadab;
\par 20 Amminadab födde Naheson; Naheson födde Salma;
\par 21 Salmon födde Boas; Boas födde Obed;
\par 22 Obed födde Isai, och Isai födde David.


\end{document}