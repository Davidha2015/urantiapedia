\begin{document}

\title{1 Samuel}

1Sa 1:1  I Ramataim-Sofim, i Efraims bergsbygd, levde en man som hette Elkana, son till Jeroham, son till Elihu, son till Tohu, son till Suf, en efraimit.
1Sa 1:2  Han hade två hustrur; den ena hette Hanna, den andra Peninna. Och Peninna hade barn, men Hanna var barnlös.
1Sa 1:3  Den mannen begav sig år efter år upp från sin stad för att tillbedja och offra åt HERREN Sebaot i Silo, där Elis båda söner, Hofni och Pinehas, då voro HERRENS präster.
1Sa 1:4  En dag offrade nu Elkana. Och han plägade giva sin hustru Peninna och alla hennes söner och döttrar var sin andel av offret;
1Sa 1:5  men åt Hanna gav han då en dubbelt så stor andel, ty han hade Hanna kär, fastän HERREN hade gjort henne ofruktsam.
1Sa 1:6  Men hennes medtävlerska plägade, för att väcka hennes vrede, mycket retas med henne, därför att HERREN hade gjort henne ofruktsam.
1Sa 1:7  För vart år, så ofta hon hade kommit upp till HERRENS hus, gjorde han på samma sätt, och den andra retades då med henne på samma sätt. Och nu grät hon och åt intet.
1Sa 1:8  Då sade hennes man Elkana till henne: "Hanna, varför gråter du? Varför äter du icke? Varför är du så sorgsen? Är jag icke mer för dig än tio söner?
1Sa 1:9  En gång när de hade ätit och druckit i Silo hände sig, medan prästen Eli satt på sin stol vid dörren till HERRENS tempel, att Hanna stod upp
1Sa 1:10  och i sin djupa bedrövelse begynte bedja till HERREN under bitter gråt.
1Sa 1:11  Och hon gjorde ett löfte och sade: HERRE Sebaot, om du vill se till din tjänarinnas lidande och tänka på mig och icke förgäta din tjänarinna, utan giva din tjänarinna en manlig avkomling, så vill jag giva denne åt HERREN för hela hans liv, och ingen rakkniv skall komma på hans huvud."
1Sa 1:12  När hon nu länge så bad inför HERREN och Eli därvid gav akt på hennes mun
1Sa 1:13  - Hanna talade nämligen i sitt hjärta; allenast hennes läppar rörde sig, men hennes röst hördes icke - då trodde Eli att hon var drucken.
1Sa 1:14  Därför sade Eli till henne: "Huru länge skall du bete dig såsom en drucken? Laga så, att ruset går av dig."
1Sa 1:15  Men Hanna svarade och sade: "Nej, min herre, jag är en hårt prövad kvinna; vin och starka drycker har jag icke druckit, men jag utgöt nu min själ för HERREN.
1Sa 1:16  Anse icke din tjänarinna för en ond kvinna, ty det är mitt myckna bekymmer och min myckna sorg som har drivit mig att tala ända till denna stund."
1Sa 1:17  Då svarade Eli och sade: "Gå i frid. Israels Gud skall giva dig vad du har utbett dig av honom."
1Sa 1:18  Hon sade: "Låt din tjänarinna finna nåd för dina ögon." Så gick kvinnan sin väg och fick sig mat, och hon såg sedan icke mer så sorgsen ut.
1Sa 1:19  Bittida följande morgon, sedan de hade tillbett inför HERREN, vände de tillbaka och kommo hem igen till Rama. Och Elkana kände sin hustru Hanna, och HERREN tänkte på henne,
1Sa 1:20  Och Hanna blev havande och födde en son, när tiden hade gått om; denne gav hon namnet Samuel, "ty", sade hon, "av HERREN har jag utbett mig honom."
1Sa 1:21  När sedan mannen Elkana med hela sitt hus begav sig upp för att offra åt HERREN sitt årliga slaktoffer och sitt löftesoffer,
1Sa 1:22  gick Hanna icke med ditupp, utan sade till sin man: "Jag vill vänta, till dess att gossen har blivit avvand, då skall jag föra honom med mig, för att han må ställas fram inför HERRENS ansikte och sedan stanna där för alltid."
1Sa 1:23  Hennes man Elkana sade till henne: "Gör vad du finner för gott; stanna, till dess du har avvant honom; må HERREN allenast uppfylla sitt ord." Så stannade då hustrun hemma och gav sin son di, till dess hon skulle avvänja honom.
1Sa 1:24  Men sedan hon hade avvant honom, tog hon honom med sig ditupp, jämte tre tjurar, en efa mjöl och en vinlägel; så förde hon honom in i HERRENS hus i Silo. Men gossen var ännu helt ung.
1Sa 1:25  Och de slaktade tjuren och förde så gossen fram till Eli.
1Sa 1:26  Och hon sade: "Hör mig, min herre; så sant du lever, min herre, jag är den kvinna som stod här bredvid dig och bad till HERREN.
1Sa 1:27  Om denne gosse bad jag; nu har HERREN givit mig vad jag utbad mig av honom.
1Sa 1:28  Därför vill ock jag nu giva honom tillbaka åt HERREN; så länge han lever, skall han vara given åt HERREN." Och de tillbådo där HERREN.
1Sa 2:1  Och Hanna bad och sade: "Mitt hjärta fröjdar sig i HERREN; mitt horn är upphöjt genom HERREN. Min mun är vitt upplåten mot mina fiender; ty jag gläder mig över din frälsning.
1Sa 2:2  Ingen är helig såsom HERREN ty ingen finnes förutom dig; ingen klippa är såsom vår Gud.
1Sa 2:3  Fören icke beständigt så mycket högmodigt tal; vad fräckt är gånge icke ut ur eder mun. Ty HERREN är en Gud som vet allt, och hos honom vägas gärningarna.
1Sa 2:4  Hjältarnas bågar äro sönderbrutna, men de stapplande omgjorda sig med kraft.
1Sa 2:5  De som voro mätta måste taga lega för bröd, men de som ledo hunger hungra icke mer. Ja, den ofruktsamma föder sju barn, men den moder som fick många barn vissnar bort.
1Sa 2:6  HERREN dödar och gör levande, han för ned i dödsriket och upp därifrån.
1Sa 2:7  HERREN gör fattig, han gör ock rik; han ödmjukar, men han upphöjer ock.
1Sa 2:8  Han upprättar den ringe ur stoftet, ur dyn lyfter han den fattige upp, ty han vill låta dem sitta bredvid furstar, och en härlig tron giver han dem till arvedel. Ty jordens grundfästen äro HERRENS, och jordkretsen har han ställt på dem.
1Sa 2:9  Sina frommas fötter bevarar han, men de ogudaktiga förgöras i mörkret, ty ingen förmår något genom egen kraft.
1Sa 2:10  De som strida mot HERREN bliva krossade, ovan dem dundrar han i himmelen; ja, HERREN dömer jordens ändar. Men han giver makt åt sin konung, han upphöjer sin smordes horn.
1Sa 2:11  Och Elkana gick hem igen till Rama; gossen däremot gjorde tjänst inför HERREN under prästen Eli.
1Sa 2:12  Men Elis söner voro onda män, de ville icke veta av HERREN.
1Sa 2:13  På följande sätt plägade nämligen prästerna gå till väga med folket: så ofta någon offrade ett slaktoffer, kom prästens tjänare, medan köttet koktes, och hade en treuddig gaffel i sin hand;
1Sa 2:14  den stack han ned i kitteln eller pannan eller krukan eller grytan, och allt vad han så fick upp med gaffeln, det tog prästen. Så gjorde de mot alla israeliter som kommo dit till Silo.
1Sa 2:15  Ja, till och med innan man hade förbränt det feta, kom prästens tjänare och sade till den som offrade: "Giv hit kött, så att jag kan steka det åt prästen, ty han vill icke hava kokt kött av dig, utan rått."
1Sa 2:16  Om då mannen svarade honom: "Först skall man nu förbränna det feta; tag sedan vad dig lyster", så sade han: "Nej, nu strax skall du lämna det, eljest tager jag det med våld."
1Sa 2:17  Och de unga männens synd var så mycket större inför HERREN som folket därigenom lärde sig att förakta HERRENS offer.
1Sa 2:18  Men Samuel gjorde tjänst inför HERRENS ansikte, och var redan såsom gosse iklädd linne-efod.
1Sa 2:19  Därtill plägade hans moder vart år göra åt honom en liten kåpa, som hon hade med sig till honom, när hon jämte sin man begav sig upp för att offra det årliga slaktoffret.
1Sa 2:20  Då plägade Eli välsigna Elkana jämte hans hustru och säga: "HERREN skänke dig ytterligare avkomma med denna kvinna, i stället för den som hon utbad sig genom sin bön till HERREN." Och så gingo de hem igen.
1Sa 2:21  Och HERREN såg till Hanna, och hon blev havande och födde tre söner och två döttrar. Men gossen Samuel växte upp i HERRENS hus.
1Sa 2:22  Då nu Eli, som var mycket gammal, fick höra allt vad hans söner gjorde mot hela Israel, och att de lågo hos de kvinnor som hade tjänstgöring vid ingången till uppenbarelsetältet,
1Sa 2:23  sade han till dem: "Varför gören sådant, allt detta onda som jag hör allt folket här tala om eder?
1Sa 2:24  Icke så, mina söner! Det rykte jag hör vara gängse bland HERRENS folk är icke gott.
1Sa 2:25  Om en människa försyndar sig mot en annan, så kan Gud medla för henne; men om en människa försyndar sig mot HERREN, vem kan då göra sig till medlare för henne?" Men de lyssnade icke till sin faders ord, ty HERREN ville döda dem.
1Sa 2:26  Gossen Samuel däremot växte till i ålder och välbehag både för HERREN och för människor.
1Sa 2:27  Och en gudsman kom till Eli och sade till honom: "Så säger HERREN: Har jag icke uppenbarat mig för din faders hus, när de ännu voro i Egypten och tjänade Faraos hus?
1Sa 2:28  Och har jag icke utvalt honom bland alla Israels stammar till präst åt mig, till att offra på mitt altare och antända rökelse och bära efod inför mitt ansikte? Och gav jag icke åt din faders hus Israels barns alla eldsoffer?
1Sa 2:29  Varför förtrampen I då de slaktoffer och spisoffer som jag har påbjudit i min boning? Och huru kan du ära dina söner mer än mig, så att I göden eder med det bästa av var offergåva som mitt folk Israel bär fram?
1Sa 2:30  Därför säger HERREN, Israels Gud: Väl har jag sagt att ditt och din faders hus skulle få göra tjänst inför mig evärdligen. Men nu säger HERREN: Bort det! Ty dem som ära mig vill jag ock ära, men de som förakta mig skola komma på skam.
1Sa 2:31  Se, dagar skola komma, då jag skall avhugga din arm och din faders hus' arm, så att ingen skall bliva gammal i ditt hus.
1Sa 2:32  Och du skall få se min boning lida nöd, trots allt det goda som vederfares Israel. Och ingen skall någonsin bliva gammal i ditt eget hus.
1Sa 2:33  Dock vill jag icke från mitt altare utrota var man av din slakt, så att jag kommer dina ögon att förtvina och din själ att försmäkta; men alla som växa upp i ditt hus skola dö, när de hava hunnit till manlig ålder.
1Sa 2:34  Och tecknet härtill skall för dig vara det som skall övergå dina båda söner Hofni och Pinehas: på en och samma dag skola de båda dö.
1Sa 2:35  Men jag skall låta en präst uppstå åt mig, som bliver beståndande, en som gör efter vad i mitt hjärta och min själ är; åt honom skall jag bygga ett hus som bliver beståndande, och han skall göra tjänst inför min smorde beständigt.
1Sa 2:36  Och var och en som bliver kvar: av ditt hus skall komma och falla ned för honom, för att få en silverpenning eller en kaka bröd; han skall säga: 'Anställ mig vid någon prästsyssla, så att jag får en bit bröd att äta.'"
1Sa 3:1  Så gjorde nu den unge Samuel tjänst inför HERREN under Eli. Och HERRENS ord var sällsynt på den tiden, profetsyner voro icke vanliga.
1Sa 3:2  Då nu en gång Eli, vilkens ögon hade begynt att bliva skumma, så att han icke kunde se, låg och sov på sin plats,
1Sa 3:3  innan ännu Guds lampa hade slocknat, och medan också Samuel låg och sov, då hände sig i HERRENS tempel, där Guds ark stod,
1Sa 3:4  att HERREN ropade på Samuel Denne svarade: "Här är jag."
1Sa 3:5  Därefter skyndade han till Eli och sade: "Här är jag; du ropade ju på mig." Men han svarade: "Jag har icke ropat; gå tillbaka och lägg dig." Och han gick och lade sig.
1Sa 3:6  Men HERREN ropade ännu en gång på Samuel; och Samuel stod upp och gick till Eli och sade: "Här är jag; du ropade ju på mig." Men han svarade: "Jag har icke ropat, min son; gå tillbaka och lägg dig."
1Sa 3:7  Samuel hade nämligen ännu icke lärt att känna igen HERREN, och ännu hade icke något HERRENS ord blivit uppenbarat för honom.
1Sa 3:8  Men HERREN ropade åter på Samuel, för tredje gången; och han stod upp och gick till Eli och sade: "Här är jag; du ropade ju på mig. Då förstod Eli att det var HERREN som ropade på ynglingen.
1Sa 3:9  Därför sade Eli till Samuel: "Gå och lägg dig; och om han vidare ropar på dig, så säg: 'Tala, HERRE; din tjänare hör." Och Samuel gick och lade sig på sin plats.
1Sa 3:10  Då kom HERREN och ställde sig där och ropade såsom de förra gångerna: "Samuel! Samuel!" Samuel svarade: "Tala, din tjänare hör."
1Sa 3:11  Då sade HERREN till Samuel: "Se, jag skall i Israel göra något som kommer att genljuda i båda öronen på var och en som får höra det.
1Sa 3:12  På den dagen skall jag låta komma över Eli allt vad jag har uttalat över hans hus, det första till det sista.
1Sa 3:13  Ty jag har förkunnat för honom att jag skall vara domare över hans hus till evig tid, därför att han har syndat, i det han visste huru hans söner drogo förbannelse över sig och dock icke höll dem tillbaka.
1Sa 3:14  Därför har jag ock med ed betygat om Elis hus: Sannerligen, Elis hus' missgärning skall icke någonsin kunna försonas, vare sig med slaktoffer eller med någon annat offergåva."
1Sa 3:15  Och Samuel låg kvar ända tills morgonen, då han öppnade dörrarna till HERRENS hus. Och Samuel fruktade för att omtala synen for Eli.
1Sa 3:16  Men Eli ropade på Samuel och sade: "Samuel, min son! Denne svarade: "Här är jag."
1Sa 3:17  Han sade: "Vad var det han ta lade till dig? Dölj det icke för mig. Gud straffe dig nu och framgent, om du döljer for mig något enda ord av det han talade till dig."
1Sa 3:18  Då omtalade Samuel för honom alltsammans och dolde intet för honom. Och han sade: "Han är HERREN; han göre vad honom täckes.
1Sa 3:19  Men Samuel växte upp, och HERREN var med honom och lät intet av allt vad han hade talat falla till jorden.
1Sa 3:20  Och hela Israel, från Dan ända till Beer-Seba, förstod att Samuel var betrodd att vara HERRENS profet.
1Sa 3:21  Och HERREN fortfor att låta se sig i Silo; ty HERREN uppenbarade sig för Samuel i Silo genom HERRENS ord.
1Sa 3:22  Och Samuels ord kom till hela Israel.
1Sa 4:1  Och Israel drog ut till strid mot filistéerna och lägrade sig vid Eben-Haeser, under det filistéerna hade lägrat sig vid Afek.
1Sa 4:2  Filistéerna ställde då upp sig i slagordning mot Israel, och striden utbredde sig, och israeliterna blevo slagna av filistéerna; dessa nedgjorde på slagfältet vid pass fyra tusen man.
1Sa 4:3  När folket kom tillbaka till lägret sade de äldste i Israel: "Varför har HERREN i dag låtit oss bliva slagna av filistéerna? Låt oss hämta hit till oss från Silo HERRENS förbundsark, för att den må komma och vara ibland oss och frälsa oss från våra fienders hand."
1Sa 4:4  Så sände då folket till Silo, och de buro därifrån HERREN Sebaots förbundsark, hans som tronar på keruberna; och Elis båda söner, Hofni och Pinehas, följde därvid med Guds förbundsark.
1Sa 4:5  Då nu HERRENS förbundsark kom in i lägret, hov hela Israel upp ett stort jubelrop, så att det dånade i marken.
1Sa 4:6  När då filistéerna hörde jubelropet, sade de: "Vad betyder detta stora jubelrop i hebréernas läger?" Och de fingo veta att HERRENS ark hade kommit in i lägret.
1Sa 4:7  Då blevo filistéerna förskräckta, ty de tänkte: "Gud har kommit in i lägret." Och de sade: "Ve oss! Något sådant har förut icke hänt.
1Sa 4:8  Ve oss! Vem kan rädda oss från denne väldige Guds hand? Det var denne Gud som slog egyptierna med alla slags plågor i öknen.
1Sa 4:9  Men fatten dock mod och varen män, I filistéer, så att I icke bliven trälar åt hebréerna, såsom de hava varit trälar åt eder. Ja, varen män och striden."
1Sa 4:10  Så stridde nu filistéerna, och israeliterna blevo slagna och flydde var och en till sin hydda, och nederlaget blev mycket stort: av Israel föllo trettio tusen man fotfolk.
1Sa 4:11  Därtill blev Guds ark tagen, och Elis båda söner, Hofni och Pinehas, blevo dödade.
1Sa 4:12  Och en benjaminit sprang från slagfältet och kom till Silo samma dag, med sönderrivna kläder och med jord på sitt huvud.
1Sa 4:13  Och när han kom dit, satt Eli på sin stol vid sidan av vägen och såg utåt, ty hans hjärta bävade av oro för Guds ark. Då nu mannen kom in i staden med budskapet, höjde hela staden upp klagorop.
1Sa 4:14  Och när Eli hörde klagoropet, sade han: "Vad betyder detta larm?" Då kom mannen skyndsamt dit och berättade det för Eli.
1Sa 4:15  Men Eli var nittioåtta år gammal och hans ögon voro starrblinda, så att han icke kunde se.
1Sa 4:16  Och mannen sade till Eli: "Jag är den som har kommit från slagfältet; jag har i dag flytt ifrån slagfältet." Då sade han: "Huru har det gått, min son?"
1Sa 4:17  Budbäraren svarade och sade: "Israel har flytt för filistéerna, mycket folk har också stupat; dina båda söner, Hofni och Pinehas, äro ock döda, och därtill har Guds ark blivit tagen."
1Sa 4:18  När han nämnde om Guds ark, föll Eli baklänges av stolen vid sidan av porten och bröt nacken av sig och dog; ty mannen var gammal och tung. Han hade då varit domare i Israel i fyrtio år.
1Sa 4:19  Och när hans sonhustru, Pinehas' hustru, som var havande och nära att föda, fick höra ryktet om att Guds ark var tagen, och att hennes svärfader och hennes man voro döda, sjönk hon ned och födde sitt barn, ty födslovåndorna kommo över henne.
1Sa 4:20  Och när hon då höll på att dö, sade kvinnorna som stodo omkring henne: "Frukta icke; du har fött en son." Men hon svarade intet och aktade icke därpå.
1Sa 4:21  Och hon kallade gossen I-Kabod, och sade: "Härligheten är borta från Israel." Därmed syftade hon på att Guds ark var tagen, så ock på sin svärfader och sin man.
1Sa 4:22  Hon sade: "Härligheten är borta från Israel", eftersom Guds ark var tagen.
1Sa 5:1  När filistéerna hade tagit Guds ark, förde de den från Eben-Haeser till Asdod.
1Sa 5:2  Där togo filistéerna Guds ark och förde in den i Dagons tempel och ställde den bredvid Dagon.
1Sa 5:3  Men när asdoditerna bittida dagen därefter kommo dit, fingo de se Dagon ligga framstupa på jorden framför HERRENS ark. Då togo de Dagon och satte honom upp igen på hans plats.
1Sa 5:4  Men när de dagen därefter åter kommo dit bittida om morgonen, fingo de ånyo se Dagon ligga framstupa på jorden framför HERRENS ark; och Dagons huvud och hans båda händer lågo avslagna på tröskeln, allenast fiskdelen satt kvar på honom.
1Sa 5:5  I Asdod trampar därför ännu i dag ingen på Dagons tröskel, varken någon av Dagons präster, ej heller någon annan som går in i Dagons tempel.
1Sa 5:6  Och HERRENS hand var tung över asdoditerna; han anställde förödelse bland dem, i det han slog dem med bölder, såväl i Asdod som inom tillhörande områden.
1Sa 5:7  Då nu invånarna i Asdod sågo att så skedde, sade de: "Israels Guds ark får icke stanna hos oss, ty hans hand vilar hårt på oss och på vår gud Dagon."
1Sa 5:8  Och de sände bud och läto församla till sig alla filistéernas hövdingar och sade: "Vad skola vi göra med Israels Guds ark?" De svarade: "Israels Guds ark må flyttas till Gat." Då flyttade de Israels Guds ark dit.
1Sa 5:9  Men sedan de hade flyttat den dit, kom genom HERRENS hand en mycket stor förvirring i staden; han slog invånarna i staden, både små och stora, så att bölder slogo upp på dem.
1Sa 5:10  Då sände de Guds ark till Ekron. Men när Guds ark kom till Ekron, ropade ekroniterna: "De hava flyttat Israels Guds ark till oss för att döda oss och vårt folk."
1Sa 5:11  Och de sände bud och läto för samla alla filistéernas hövdingar och sade: "Sänden bort Israels Guds ark, så att den får komma tillbaka till sin plats igen och icke dödar oss och vårt folk." Ty en dödlig förvirring hade uppstått i hela staden; Guds hand låg mycket tung på den.
1Sa 5:12  De av invånarna som icke dogo blevo slagna med bölder; och ropet från staden steg upp mot himmelen.
1Sa 6:1  Sedan nu HERRENS ark hade varit i filistéernas land i sju månader,
1Sa 6:2  tillkallade filistéerna sina präster och spåmän och sade: "Vad skola vi göra med HERRENS ark? Låten oss veta på vilket sätt vi skola sända den till dess plats igen."
1Sa 6:3  De svarade: "Om I viljen sända bort Israels Guds ark, skolen I icke sända bort den utan skänker; I måsten giva åt honom ett skuldoffer. Då skolen I bliva botade, och det skall då också bliva eder kunnigt varför hans hand icke drager sig tillbaka från eder."
1Sa 6:4  Då frågade de: "Vad för ett skuldoffer skola vi giva åt honom?" De svarade: "Fem bölder av guld och fem jordråttor av guld, lika många som filistéernas hövdingar; ty en och samma hemsökelse har träffat alla, också edra hövdingar.
1Sa 6:5  I skolen göra avbildningar av edra bölder och avbildningar av jordråttorna som fördärva edert land; given så ära åt Israels Gud. Kanhända tager han då bort sin tunga hand från eder, så ock från eder gud och från edert land.
1Sa 6:6  Varför tillsluten I edra hjärtan, såsom egyptierna och Farao tillslöto sina hjärtan? Måste icke dessa, sedan han hade utfört stora gärningar bland dem, släppa israeliterna, så att de fingo gå?
1Sa 6:7  Så gören eder nu en ny vagn, och tagen två kor som giva di, och som icke hava burit något ok, och spännen korna för vagnen, men skiljen deras kalvar ifrån dem och låten dem stanna hemma.
1Sa 6:8  Tagen så HERRENS ark och sätten den på vagnen, och läggen de gyllene klenoder I given honom såsom skuldoffer ned i ett skrin vid sidan av den, och låten den så gå åstad.
1Sa 6:9  Sedan skolen I se efter: om den tager vägen till sitt land, upp mot Bet-Semes, så var det han som gjorde oss allt detta stora onda; men om så icke sker, då veta vi att det icke var hans hand som hemsökte oss. Detta har då träffat oss allenast av en händelse."
1Sa 6:10  Männen gjorde så; de togo två kor som gåvo di och spände dem för vagnen; men deras kalvar behöllo de hemma.
1Sa 6:11  Och de satte HERRENS ark på vagnen, därtill ock skrinet med jordråttorna av guld och med avbildningarna av svulsterna.
1Sa 6:12  Och korna gingo raka vägen fram åt Bet-Semes till; de höllo alltjämt samma stråt och gingo där råmande, utan att vika av vare sig till höger eller till vänster. Och filistéernas hövdingar gingo efter dem ända till Bet-Semes' område.
1Sa 6:13  Men betsemesiterna höllo på med veteskörd i dalen. När de nu lyfte upp sina ögon, fingo de se arken; och de blevo glada, då de sågo den.
1Sa 6:14  Men när vagnen kom till betsemesiten Josuas åker, stannade den där; och där låg en stor sten. Då höggo de sönder trävirket på vagnen och offrade korna till brännoffer åt HERREN.
1Sa 6:15  Leviterna hade nämligen lyft ned HERRENS ark jämte skrinet som stod därbredvid, det vari de gyllene klenoderna funnos, och hade satt detta på den stora stenen. Sedan offrade invånarna i Bet-Semes på den dagen brännoffer och slaktoffer åt HERREN.
1Sa 6:16  Och när filistéernas fem hövdingar hade sett detta, vände de samma dag tillbaka till Ekron.
1Sa 6:17  De svulster av guld som filistéerna gåvo såsom skuldoffer åt HERREN utgjorde: för Asdod en, för Gasa en, för Askelon en, för Gat en, för Ekron en.
1Sa 6:18  Men jordråttorna av guld voro lika många som filistéernas alla städer under de fem hövdingarna, varvid medräknas både befästa städer och landsbygdens byar, intill den stora Sorgestenen, på vilken de satte ned HERRENS ark, och som finnes kvar ännu i dag, på betsemesiten Josuas åker.
1Sa 6:19  Av invånarna i Bet-Semes blevo ock många slagna, därför att de hade sett på HERRENS ark; han slog sjuttio man bland folket, femtio tusen man. Och folket sörjde däröver att HERREN hade slagit så många bland folket.
1Sa 6:20  Och invånarna i Bet-Semes sade: "Vem kan bestå inför HERREN, denne helige Gud? Och till vem skall han draga bort ifrån oss?"
1Sa 6:21  Och de skickade sändebud till dem som bodde i Kirjat-Jearim och läto säga: "Filistéerna hava sänt tillbaka HERRENS ark; kommen hitned och hämten den upp till eder.
1Sa 7:1  Då kommo Kirjat-Jearims män och hämtade HERRENS ark ditupp och förde den in i Abinadabs hus på höjden. Och hans son Eleasar helgade de till att hava vården om HERRENS ark.
1Sa 7:2  Och från den dag då arken fick sin plats i Kirjat-Jearim förflöt en lång tid: tjugu år förgingo; och hela Israels hus suckade nu efter HERREN.
1Sa 7:3  Men Samuel sade till hela Israels hus: "Om I av allt edert hjärta viljen vända om till HERREN, så skaffen bort ifrån eder de främmande gudarna och Astarterna, och rikten edra hjärtan till HERREN och tjänen honom allena, så skall han rädda eder ifrån filistéernas hand."
1Sa 7:4  Då skaffade Israels barn bort Baalerna och Astarterna och tjänade HERREN allena.
1Sa 7:5  Och Samuel sade: "Församlen hela Israel i Mispa, så vill jag där bedja till HERREN för eder."
1Sa 7:6  Då församlade de sig i Mispa och öste upp vatten och göto ut det in. för HERREN och fastade den dagen; och de sade där: "Vi hava syndat mot HERREN." Och Samuel dömde Israels barn i Mispa.
1Sa 7:7  Men när filistéerna hörde att Israels barn hade församlat sig i Mispa, drogo filistéernas hövdingar ditupp mot Israel. Då Israels barn hörde detta, blevo de förskräckta för filistéerna.
1Sa 7:8  Och Israels barn sade till Samuel: "Hör icke upp att ropa för oss till HERREN, vår Gud, att han må frälsa oss ifrån filistéernas hand."
1Sa 7:9  Då tog Samuel ett dilamm och offrade det såsom ett heloffer, till brännoffer åt HERREN; och Samuel ropade till HERREN för Israel, och HERREN bönhörde honom.
1Sa 7:10  Under det att Samuel offrade brännoffret, ryckte nämligen filistéerna fram till strid mot Israel; men HERREN lät ett starkt tordön dundra över filistéerna på den dagen och förvirrade dem, så att de blevo slagna av Israel.
1Sa 7:11  Och Israels män drogo ut från Mispa och förföljde filistéerna och nedgjorde dem, under det att de förföljde dem ända till trakten nedanför Bet-Kar.
1Sa 7:12  Då tog Samuel en sten och reste den mellan Mispa och Sen och gav den namnet Eben-Haeser, i det han sade: "Allt härintill har HERREN hjälpt oss."
1Sa 7:13  Så blevo filistéerna kuvade och kommo icke mer in i Israels land Och HERRENS hand var emot filistéerna, så länge Samuel levde.
1Sa 7:14  Och de städer som filistéerna hade tagit från Israel kommo tillbaka till Israel, allasammans, från Ekron ända till Gat; och det tillhörande området tog Israel också igen ifrån filistéerna. Och mellan Israel och amoréerna blev fred.
1Sa 7:15  Och Samuel var domare i Israel, så länge han levde.
1Sa 7:16  Vart år färdades han omkring till Betel, Gilgal och Mispa; och han dömde Israel på alla dessa platser.
1Sa 7:17  Sedan plägade han vända tillbaka till Rama, ty där var hans hem, och där dömde han eljest Israel där byggde han ock ett altare åt HERREN.
1Sa 8:1  Men när Samuel blev gammal, satte han sina söner till domare över Israel
1Sa 8:2  Hans förstfödde son hette Joel, och hans andre son Abia; de hade sitt domarsäte i Beer-Seba.
1Sa 8:3  Men hans söner vandrade icke på hans väg, utan veko av därifrån och sökte orätt vinning; de togo mutor och vrängde rätten.
1Sa 8:4  Då församlade sig alla de äldste i Israel och kommo till Samuel i Rama.
1Sa 8:5  Och de sade till honom: "Du är ju nu gammal, och dina söner vandra icke på dina vägar. Så sätt nu en konung över oss till att döma oss, såsom alla andra folk hava."
1Sa 8:6  Men det misshagade Samuel, detta att de sade då: "Giv oss en konung, for att han må döma oss." Och Samuel bad till HERREN.
1Sa 8:7  Då sade HERREN till Samuel "Lyssna till folkets ord, och gör allt vad de begära av dig; ty det är icke dig de hava förkastat, nej, mig hava de förkastat, i det de icke vilja att jag skall vara konung över dem.
1Sa 8:8  Såsom de alltid hava gjort, från den dag då jag förde dem upp ur Egypten ända till denna dag, i det att de hava övergivit mig och tjänat andra gudar, så göra de nu ock mot dig.
1Sa 8:9  Så lyssna nu till deras ord. Dock må du högtidligt varna dem och förkunna för dem den konungs rätt, som kommer att regera över dem."
1Sa 8:10  Och Samuel sade till folket, som hade begärt en konung av honom, allt vad HERREN hade talat.
1Sa 8:11  Han sade: "Detta bliver den konungs rätt, som kommer att regera över eder: Edra söner skall han taga och skall sätta dem på sina vagnar och hästar, till sin tjänst, eller ock skola de nödgas löpa framför hans vagnar.
1Sa 8:12  Andra av dem skall han taga och sätta till sina över- och underhövitsmän, och andra skola nödgas plöja hans åkerjord och inbärga hans skörd och förfärdiga hans krigsredskap och hans vagnsredskap.
1Sa 8:13  Edra döttrar skall han taga till salvoberederskor, kokerskor och bagerskor.
1Sa 8:14  Edra bästa åkrar, vingårdar och olivplanteringar skall han taga och skall giva dem åt sina tjänare;
1Sa 8:15  och han skall taga tionde av edra sädesfält och edra vingårdar och giva åt sina hovmän och tjänare.
1Sa 8:16  Därtill skall han taga edra tjänare och edra tjänarinnor och edra bästa ynglingar, så ock edra åsnor, och bruka dem för sitt behov.
1Sa 8:17  Av eder småboskap skall han taga tionde, och I skolen vara hans trälar.
1Sa 8:18  När I då ropen om hjälp för dem konungs skull som I själva haven utvalt åt eder, då skall HERREN icke svara eder.
1Sa 8:19  Men folket ville icke lyssna till Samuels ord, utan sade: "Nej, en konung måste vi hava över oss."
1Sa 8:20  Vi vilja bliva lika alla andra folk; vi vilja hava en konung som dömer oss, och som drager ut i spetsen för oss till att föra våra krig."
1Sa 8:21  Då nu Samuel hörde allt detta som folket sade, framförde han det till HERREN.
1Sa 8:22  Men HERREN sade till Samuel: "Lyssna till deras ord, och sätt en konung över dem." Då sade Samuel till Israels män: "Gån hem, var och en till sin stad."
1Sa 9:1  I Benjamin levde en man som hette Kis, son till Abiel, son till Seror, son till Bekorat, son till Afia, son till en benjaminit; och han var en rik man.
1Sa 9:2  Han hade en son som hette Saul, en ståtlig och fager man; bland Israels barn fanns ingen man som var fagrare än han; han var huvudet högre än allt folket.
1Sa 9:3  Nu hade Kis', Sauls faders, åsninnor kommit bort för honom; därför sade Kis till sin son Saul: "Tag med dig en av tjänarna och stå upp och gå åstad och sök efter åsninnorna."
1Sa 9:4  Då gick han genom Efraims bergsbygd och därefter genom Salisalandet; men de funno dem icke. Så gingo de genom Saalimslandet, men där voro de icke; sedan gick han genom Benjamins land, men de funno dem icke heller där.
1Sa 9:5  När de så hade kommit in i Sufs land, sade Saul till tjänaren som han hade med sig: "Kom, låt oss gå hem igen; min fader kunde eljest, stället för att tänka på åsninnorna, bliva orolig för vår skull."
1Sa 9:6  Men han svarade honom: "Se, i denna stad finnes en gudsman; han är en ansedd man; allt vad han säger, det sker. Låt oss nu gå dit; måhända kan han säga oss något om den färd vi hava företagit oss."
1Sa 9:7  Då sade Saul till sin tjänare: "Men om vi gå dit, vad skola vi då taga med oss åt mannen? Brödet är ju slut i våra ränslar, och vi hava icke heller någon annan gåva att taga med oss åt gudsmannen. Eller vad hava vi väl?"
1Sa 9:8  Tjänaren svarade Saul ännu en gång och sade: "Se, här har jag i min ägo en fjärdedels sikel silver; den vill jag giva åt gudsmannen, för att han må säga oss vilken väg vi böra gå."
1Sa 9:9  (Fordom sade man så i Israel, när man gick för att fråga Gud: "Kom, låt oss gå till siaren." Ty den som man nu kallar profet kallade man fordom siare.)
1Sa 9:10  Saul sade till sin tjänare: "Ditt förslag är gott; kom, låt oss gå." Så gingo de till staden där gudsmannen fanns.
1Sa 9:11  När de nu gingo uppför höjden där staden låg, träffade de några flickor som hade gått ut för att hämta vatten; dem frågade de: "Är siaren här?"
1Sa 9:12  De svarade dem och sade: "Ja, helt nära. Skynda dig nu, ty han har i dag kommit till staden; folket firar nämligen i dag en offerfest på offerhöjden.
1Sa 9:13  Om I nu gån in i staden, träffen I honom, innan han går upp på höjden till måltiden, ty folket äter icke, förrän han kommer. Han skall välsigna offret; först sedan begynna de inbjudna att äta. Gån därför nu ditupp, ty just nu kunnen I träffa honom."
1Sa 9:14  Så gingo de upp till staden. Och just när de kommo in i staden, mötte de Samuel, som var stadd på väg upp till offerhöjden.
1Sa 9:15  Men dagen innan Saul kom hade HERREN uppenbarat för Samuel och sagt:
1Sa 9:16  "I morgon vid denna tid skall jag sända till dig en man från Benjamins land, och honom skall du smörja till furste över mitt folk Israel; han skall frälsa mitt folk ifrån filistéernas hand. Ty jag har sett till mitt folk, eftersom deras rop har kommit till mig."
1Sa 9:17  När nu Samuel fick se Saul, gav HERREN honom den uppenbarelsen: "Se där är den man om vilken jag sade till dig: Denne skall styra mitt folk."
1Sa 9:18  Men Saul gick fram till Samuel i porten och sade: "Säg mig var siaren bor."
1Sa 9:19  Samuel svarade Saul och sade: "Jag är siaren. Gå före mig upp på offerhöjden, ty I skolen äta där med mig i dag. Men i morgon vill jag låta dig gå; och om allt vad du har på hjärtat vill jag giva dig besked.
1Sa 9:20  Och vad angår åsninnorna, som nu i tre dagar hava varit borta för dig, skall du icke bekymra dig för dem, ty de äro återfunna. Vem tillhör för övrigt allt vad härligt är i Israel, om icke dig och hela din faders hus?"
1Sa 9:21  Saul svarade och sade: "Jag är ju en benjaminit, från en av de minsta stammarna i Israel, och min släkt är ju den ringaste bland alla släkter i Benjamins stammar. Varför talar du då till mig på det sättet?"
1Sa 9:22  Men Samuel tog Saul och hans tjänare och förde dem upp i salen och gav dem plats överst bland de inbjudna, vilka voro vid pass trettio män.
1Sa 9:23  Och Samuel sade till kocken: "Giv hit det stycke som jag gav dig, och som jag sade att du skulle förvara hos dig."
1Sa 9:24  Då tog kocken fram lårstycket med vad därtill hörde, och satte det fram för Saul; och Samuel sade: "Se, här sättes nu fram för dig det som har blivit sparat; ät därav. Ty just för denna stund blev det undanlagt åt dig, då när jag sade att jag hade inbjudit folket." Så åt Saul den dagen med Samuel.
1Sa 9:25  Därefter gingo de ned från offerhöjden och in i staden. Sedan samtalade han med Saul uppe på taket.
1Sa 9:26  Men bittida följande dag, när morgonrodnaden gick upp, ropade Samuel uppåt taket till Saul och sade: "Stå upp, så vill jag ledsaga dig till vägs." Då stod Saul upp, och de gingo båda åstad, han och Samuel.
1Sa 9:27  När de så voro på väg ned mot ändan av staden, sade Samuel till Saul: "Säg till tjänaren att han skall gå före oss" - och han fick gå - "men du själv må nu stanna här, så vill jag låta dig höra vad Gud har talat."
1Sa 10:1  Och Samuel tog sin oljeflaska och göt olja på hans huvud och kysste honom och sade: "Se, HERREN har smort dig till furste över sin arvedel.
1Sa 10:2  När du nu går ifrån mig, skall du invid Rakels grav, vid Benjamins gräns, vid Selsa, träffa två män; dessa skola säga till dig: 'Åsninnorna som du gick åstad att söka äro återfunna; din fader tänker därför icke mer på åsninnorna, men han är orolig för eder skull och säger: Vad skall jag göra för att finna min son?'
1Sa 10:3  Och när du har gått därifrån ett stycke fram och kommit till Tabors terebint skall du där möta tre män som äro på väg upp till Gud i Betel. En bär tre killingar, en bar tre brödkakor, och en bär en vinlägel.
1Sa 10:4  Dessa skola hälsa dig och giv dig två bröd, och du skall taga emot vad de giva.
1Sa 10:5  Sedan kommer du till Guds Gibea, där filistéernas fogdar äro. Och när du kommer dit in i staden, skall du träffa på en skara profeter, som komma ned från offerhöjden där, med psaltare, puka, flöjt och harpa före sig, under det att de själva äro i profetisk hänryckning.
1Sa 10:6  Och HERRENS Ande skall komma över dig, så att också du fattas av hänryckning likasom de; och du skall då bliva förvandlad till en annan människa.
1Sa 10:7  När du nu ser att dessa tecken inträffa, då må du göra vad tillfället giver vid handen, ty Gud är med dig.
1Sa 10:8  Sedan må du gå ned före mig till Gilgal, så skall jag komma ditned till dig, för att offra brännoffer och tackoffer; sju dagar skall du vänta till dess jag kommer till dig och förkunnar för dig vad du skall göra.
1Sa 10:9  I det han nu vände sig om för att gå ifrån Samuel, förvandlade Gud hans sinne och gav honom ett annat hjärta; och alla dessa tecken inträffade samma dag.
1Sa 10:10  När de kommo till Gibea, mötte honom där en skara profeter; då kom Guds Ande över honom, så att han, mitt ibland dem, själv fattades av profetisk hänryckning.
1Sa 10:11  Då nu alla som förut kände honom fingo se honom vara i hänryckning likasom profeterna, sade folket sinsemellan: "Vad har skett med Kis' son? Är ock Saul bland profeterna?"
1Sa 10:12  Men en av männen därifrån svarade och sade: "Vem är då dessas fader?" - Härav uppkom ordspråket: "Är ock Saul bland profeterna?"
1Sa 10:13  Men när hans profetiska hänryckning hade upphört, gick han upp på offerhöjden.
1Sa 10:14  Då frågade Sauls farbroder honom och hans tjänare: "Var haven I varit?" Han svarade: "Borta för att söka åsninnorna. Men när vi sågo att de ingenstädes voro att finna, gingo vi till Samuel."
1Sa 10:15  Då sade Sauls farbroder: "Tala om for mig vad Samuel sade till eder."
1Sa 10:16  Saul svarade sin farbroder: "Han omtalade för oss att åsninnorna voro återfunna." Men vad Samuel hade sagt om konungadömet omtalade han icke för honom.
1Sa 10:17  Därefter kallade Samuel folket tillsammans till HERREN, i Mispa.
1Sa 10:18  Och han sade till Israels barn: "Så säger HERREN, Israels Gud: Jag har fört Israel upp ur Egypten, och jag räddade eder icke allenast undan Egypten, utan ock undan alla andra konungadömen som förtryckte eder.
1Sa 10:19  Men nu haven I förkastat eder Gud, som själv frälste eder ur alla edra olyckor och trångmål, och haven sagt till honom: 'Sätt en konung över oss.' Så träden nu fram inför HERREN efter edra stammar och edra ätter."
1Sa 10:20  Därpå lät Samuel alla Israels stammar gå fram; då träffades Benjamin stam av lotten.
1Sa 10:21  När han sedan lät Benjamins stam gå fram efter dess släkter, träffade Matris släkt av lotten; därpå träffades Saul, Kis' son, av lotten, men när de då sökte efter honom, stod han icke att finna.
1Sa 10:22  Då frågade de HERREN ännu en gång: "Har någon mer kommit hit? HERREN svarade: "Han har gömt sig bland trossen."
1Sa 10:23  Då skyndade de dit och hämtad honom därifrån, och när han nu trädde fram bland folket, var han huvudet högre än allt folket.
1Sa 10:24  Och Samuel sade till allt folket: "Här sen I nu den som HERREN har utvalt; ingen är honom lik bland allt folket." Då jublade allt folket och ropade: "Leve konungen!"
1Sa 10:25  Och Samuel kungjorde för folket konungadömets rätt och tecknade upp den i en bok och lade ned den inför HERREN. Sedan lät Samuel allt folket gå hem, var och en till sitt.
1Sa 10:26  Också Saul gick hem till Gibea; och honom följde en härskara av män vilkas hjärtan Gud hade rört.
1Sa 10:27  Men några onda män sade: "Vad hjälp skulle denne kunna giva oss?" Och de föraktade honom och buro icke fram skänker till honom. Men han låtsade som om han icke märkte det.
1Sa 11:1  Och ammoniten Nahas drog upp och belägrade Jabes i Gilead. Då sade alla män i Jabes till Nahas: "Slut fördrag med oss, så vilja vi bliva dig underdåniga."
1Sa 11:2  Men ammoniten Nahas svarade dem: "På det villkoret vill jag sluta fördrag med eder, att jag får sticka ut högra ögat på eder alla och därmed tillfoga hela Israel smälek."
1Sa 11:3  De äldste i Jabes sade till honom: "Giv oss sju dagars uppskov, så att vi kunna skicka sändebud över hela Israels land; om då ingen vill hjälpa oss, så skola vi giva oss åt dig."
1Sa 11:4  Så kommo nu sändebuden till Sauls Gibea och omtalade detta för folket. Då brast allt folket ut i gråt.
1Sa 11:5  Men just då kom Saul gående bakom sina oxar från åkern. Och Saul frågade: "Vad fattas folket, eftersom de gråta?" Och de förtäljde för honom vad mannen från Jabes hade sagt.
1Sa 11:6  Då kom Guds Ande över Saul, när han hörde detta, och hans vrede upptändes högeligen.
1Sa 11:7  Och han tog ett par oxar och styckade dem och sände styckena omkring över hela Israels land med sändebuden och lät säga: "Den som icke drager ut efter Saul och Samuel, med hans oxar skall så göras." Då föll en förskräckelse ifrån HERREN över folket, så att de drogo ut såsom en man.
1Sa 11:8  Och han mönstrade dem i Besek, och Israels barn utgjorde då tre hundra tusen, och Juda män trettio tusen.
1Sa 11:9  Och de sade till sändebuden som hade kommit: "Så skolen I säga till männen i Jabes i Gilead: I morgon skolen I få hjälp, när solen bränner som hetast." Och sändebuden kommo och förkunnade detta för männen i Jabes; och dessa blevo glada däröver.
1Sa 11:10  Nu läto männen i Jabes säga: "I morgon vilja vi giva oss åt eder, och I mån då göra med oss vadhelst I finnen för gott."
1Sa 11:11  Dagen därefter fördelade Saul folket i tre hopar; och de trängde in i lägret vid morgonväkten och nedgjorde ammoniterna, och upphörde först när det var som hetast på dagen. Och de som kommo undan blevo så kringspridda, att icke två av dem kommo undan tillsammans.
1Sa 11:12  Då sade folket till Samuel: "Vilka voro de som sade: 'Skulle Saul bliva konung över oss!' Given hit dessa män, så att vi få döda dem."
1Sa 11:13  Men Saul sade: "På denna dag skall ingen dödas, ty i dag har HERREN givit seger åt Israel."
1Sa 11:14  Och Samuel sade till folket: "Kom, låt oss gå till Gilgal och där förnya konungadömet."
1Sa 11:15  Då gick allt folket till Gilgal och gjorde Saul till konung där, inför HERRENS ansikte, i Gilgal; och de offrade där tackoffer inför HERRENS ansikte. Och Saul och alla Israels män voro där uppfyllda av glädje.
1Sa 12:1  Och Samuel sade till hela Israel: "Se, jag har lyssnat till edra ord och gjort allt vad I haven begärt av mig; jag har satt en konung över eder.
1Sa 12:2  Nu är det eder konung som skall vara eder ledare, nu då jag är gammal och grå; I haven ju redan mina söner ibland eder. Hittills är det jag som har varit eder ledare, från min ungdom ända till denna dag.
1Sa 12:3  Se har står jag, vittnen nu mot mig inför HERREN och inför hans smorde. Har jag tagit någons oxe, eller har jag tagit någons åsna? Har jag förtryckt någon eller övat våld mot någon? Har jag tagit mutor av någon, för att jag skulle se genom fingrarna med honom? Jag vill då giva eder ersättning därför."
1Sa 12:4  De svarade: "Du har icke förtryckt oss, du har icke övat våld mot oss; och från ingen människa har du tagit något."
1Sa 12:5  Då sade han till dem: "HERREN vare vittne mot - eder, och hans smorde vare ock vittne denna dag, att I icke haven funnit något i min hand." De svarade: "Ja, vare det så."
1Sa 12:6  Samuel sade till folket: "Ja, HERREN vare vittne, han som lät Mose och Aron uppstå och förde edra fäder upp ur Egyptens land.
1Sa 12:7  Så träden nu fram, för att jag må gå till rätta med eder inför HERREN angående allt gott som HERREN i sin rättfärdighet har gjort mot eder och mot edra fäder.
1Sa 12:8  När Jakob hade kommit fram till Egypten, ropade edra fäder till HERREN, och HERREN sände Mose och Aron, som förde edra fäder ut ur Egypten och läto dem bosätta sig här i landet.
1Sa 12:9  Men när de glömde HERREN, sin Gud, sålde han dem i Siseras hand, härhövitsmannens i Hasor, och i filistéernas hand och i Moabs konungs hand, och dessa stridde mot dem.
1Sa 12:10  Men då ropade till HERREN och sade: 'Vi hava syndat, ty vi hava övergivit HERREN och tjänat Baalerna och Astarterna; men rädda oss nu från våra fienders hand, så vilja vi tjäna dig.'
1Sa 12:11  Då sände HERREN Jerubbaal och Bedan och Jefta och Samuel och räddade eder från edra fienders hand runt omkring, så att I fingen bo i trygghet.
1Sa 12:12  Men när I sågen att Nahas, Ammons barns konung, kom emot eder, saden I till mig: 'Nej, en konung måste regera över oss', fastän det är HERREN, eder Gud, som är eder konung.
1Sa 12:13  Och se, här är nu den konung I haven utvalt, den som I haven begärt; se, HERREN har satt en konung över eder.
1Sa 12:14  Allenast man I nu frukta HERREN och tjäna honom och höra hans röst och icke vara gensträviga mot HERRENS befallning. Ja, både I och den konung som regerar över eder mån följa HERREN, eder Gud.
1Sa 12:15  Men om I icke hören HERRENS röst, utan ären gensträviga mot HERRENS befallning, då skall HERRENS hand drabba eder likasom edra fäder.
1Sa 12:16  Träden nu ock fram och sen det stora under som HERREN skall göra inför edra ögon.
1Sa 12:17  Nu är ju tiden för veteskörden; men jag vill ropa till HERREN att han må låta det dundra och regna. Så skolen I märka och se huru mycket ont I haven gjort i HERRENS ögon genom eder begäran att få en konung."
1Sa 12:18  Och Samuel ropade till HERREN, och HERREN lät det dundra och regna på den dagen. Då betogs allt folket av stor fruktan för HERREN och för Samuel.
1Sa 12:19  Och allt folket sade till Samuel: "Bed för dina tjänare till HERREN, din Gud, att vi icke må dö, eftersom vi till alla våra andra synder ock hava lagt det onda att vi hava begärt att få en konung."
1Sa 12:20  Samuel sade till folket: "Frukten icke. Väl haven I gjort allt detta onda; men viken nu blott icke av ifrån HERREN, utan tjänen HERREN av allt edert hjärta.
1Sa 12:21  Viken icke av; ty då följen I tomma avgudar, som varken kunna hjälpa eller rädda, eftersom de äro allenast tomhet.
1Sa 12:22  Ty HERREN skall icke förskjuta sitt folk, för sitt stora namns skull, eftersom HERREN har behagat att göra eder till sitt folk.
1Sa 12:23  Vare det ock fjärran ifrån mig att jag skulle så synda mot HERREN, att jag upphörde att bedja för eder! Jag vill fastmer lära eder den goda och rätta vägen.
1Sa 12:24  Allenast frukten HERREN och tjänen honom troget av allt edert hjärta. Ty sen vilka stora ting han har gjort med eder!
1Sa 12:25  Men om I gören vad ont är, så skolen både I och eder konung förgås."
1Sa 13:1  Saul hade varit konung ett år, och när han nu regerade över Israel på andra året,
1Sa 13:2  utvalde han åt sig tre tusen män ur Israel. Av dessa hade Saul själv hos sig två tusen i Mikmas och i Betels bergsbygd, och ett tusen hade Jonatan hos sig i Gibea i Benjamin. Men det övriga folket hade han låtit gå hem, var och en till sin hydda.
1Sa 13:3  Och Jonatan dräpte filistéernas fogde i Geba; och filistéerna fingo höra det. Men Saul lät stöta i basun över hela landet och säga: "Detta må hebréerna höra."
1Sa 13:4  Så fick hela Israel höra omtalas att Saul hade dräpt filistéernas fogde, och att Israel därigenom hade blivit förhatligt för filistéerna. Och folket bådades upp att följa Saul till Gilgal.
1Sa 13:5  Under tiden hade filistéerna församlat sig för att strida mot Israel: trettio tusen vagnar och sex tusen ryttare, och fotfolk så talrikt som sanden på havets strand; och de drogo upp och lägrade sig vid Mikmas, öster om Bet-Aven.
1Sa 13:6  Då nu israeliterna sågo sig vara i nöd, i det att folket svårt ansattes, gömde sig folket i grottor, i skogssnår och bland klippor, i fasta valv och i gropar.
1Sa 13:7  Och somliga av hebréerna gingo över Jordan in i Gads och Gileads land. Men Saul var ännu kvar i Gilgal; och allt folket följde honom med bävan.
1Sa 13:8  När han nu hade väntat sju dagar, intill den tid Samuel hade bestämt men Samuel likväl icke kom till Gilgal, begynte folket skingra sig och gå ifrån honom.
1Sa 13:9  Då sade Saul: "Fören fram till mig brännoffers- och tackoffersdjuren." Därpå frambar han brännoffret.
1Sa 13:10  Men just när han hade slutat att frambära brännoffret, kom Samuel. Då gick Saul honom till mötes för att hälsa honom.
1Sa 13:11  Men Samuel sade: "Vad har du gjort!" Saul svarade: "När jag såg att folket skingrade sig och gick ifrån mig, under det att du icke kom inom den bestämda tiden, fastän filistéerna voro församlade vid Mikmas,
1Sa 13:12  då tänkte jag: Nu komma filistéerna hitned mot mig i Gilgal, och jag har ännu icke bönfallit inför HERREN. Då tog jag mod till mig och offrade brännoffret."
1Sa 13:13  Samuel sade till Saul: "Du har handlat dåraktigt. Du har icke hållit det bud HERREN, din Gud, har givit dig; eljest skulle HERREN hava befäst ditt konungadöme över Israel för evig tid.
1Sa 13:14  Men nu skall ditt konungadöme icke bliva beståndande. HERREN har sökt sig en man efter sitt hjärta, och honom har HERREN förordnat till furste över sitt folk, eftersom du icke har hållit vad HERREN bjöd dig."
1Sa 13:15  Därefter stod Samuel upp och gick från Gilgal till Gibea i Benjamin. Men Saul mönstrade det folk som fanns hos honom: vid pass sex hundra man.
1Sa 13:16  Och Saul och hans son Jonatan stannade i Geba i Benjamin med det folk som fanns hos dem, under det att filistéerna hade lägrat sig vid Mikmas.
1Sa 13:17  Och en härskara, delad i tre hopar, drog ut ur filistéernas läger för att härja: en hop tog vägen till Ofra i Sualslandet,
1Sa 13:18  en hop tog vägen till Bet-Horon, och en hop tog vägen till det område som vetter åt Seboimsdalen, åt öknen till.
1Sa 13:19  Ingen smed fanns då i hela Israels land, ty filistéerna fruktade att hebréerna skulle låta göra sig svärd eller spjut.
1Sa 13:20  Och så måste en israelit alltid begiva sig ned till filistéerna, om han ville låta vässa sin lie eller sin plogbill eller sin yxa eller sin skära,
1Sa 13:21  när det hade blivit något fel med eggen på skärorna eller plogbillarna, eller med gafflarna eller yxorna, eller när oxpikarnas uddar behövde rätas.
1Sa 13:22  Härav kom sig, att när striden skulle stå, ingen enda av Sauls och Jonatans folk hade ett svärd eller ett spjut; allenast Saul själv och hans son Jonatan hade sådana.
1Sa 13:23  Men filistéerna läto en utpost rycka fram till passet vid Mikmas.
1Sa 14:1  Så hände sig nu en dag att Jonatan, Sauls son, sade till sin vapendragare: "Kom, låt oss gå över till filistéernas utpost där på andra sidan." Men han omtalade det icke för sin fader.
1Sa 14:2  Saul vistades då vid Gibeas gräns, under granatträdet i Migron, och folket som han hade hos sig utgjorde vid pass sex hundra man;
1Sa 14:3  och Ahia, son till Ahitub, som var broder till I-Kabod, son till Pinehas, son till Eli, HERRENS präst i Silo, har då efoden. Och folket visste icke om, att Jonatan hade gått bort.
1Sa 14:4  Men i passet, där Jonatan sökte gå över för att komma till filistéernas utpost, låg på vardera sidan en brant klippa; den ena hette Boses och den andra Sene.
1Sa 14:5  Den ena klippan reste sig i norr, mitt emot Mikmas, den andra i söder, mitt emot Geba.
1Sa 14:6  Och Jonatan sade till sin vapendragare: "Kom, låt oss gå över till dessa oomskurnas utpost, kanhända skall HERREN göra något för oss. Ty intet hindrar HERREN att giva seger genom få likasåväl som genom många."
1Sa 14:7  Hans vapendragare svarade honom: "Gör allt vad du har i sinnet. Gå du åstad; jag följer dig vart du vill."
1Sa 14:8  Då sade Jonatan: "Välan, vi skola gå över till männen där och laga så, att de få se oss.
1Sa 14:9  Om de då säga till oss så: 'Stån stilla, till dess vi komma fram till eder', då skola vi stanna där vi äro och icke stiga upp till dem.
1Sa 14:10  Om de däremot säga så: 'Kommen hitupp till oss', då skola vi stiga ditupp, ty då har HERREN givit dem i vår hand; detta skall för oss vara tecknet härtill."
1Sa 14:11  När nu de två hade blivit synliga för filistéernas utpost, sade filistéerna: "Se, hebréerna krypa ut ur hålen där de hava gömt sig."
1Sa 14:12  Därpå ropade utpostens manskap till Jonatan och hans vapendragare och sade: "Kommen hitupp till oss, så skola vi väl lära eder!" Då sade Jonatan till sin vapendragare: "Följ mig ditupp, ty HERREN har givit dem i Israels hand."
1Sa 14:13  Och Jonatan klättrade på händer och fötter uppför, och hans vapendragare följde honom. Och de föllo för Jonatan; och hans vapendragare gick efter honom och gav dem dödsstöten.
1Sa 14:14  I det första anfallet nedgjorde så Jonatan och hans vapendragare vid pass tjugu män, på en sträcka av vid pass ett halvt plogland.
1Sa 14:15  Då uppstod förskräckelse i lägret på fältet och bland allt folket; utposterna och de som hade gått ut för att härja grepos ock av förskräckelse. Och marken darrade, så att en förskräckelse ifrån Gud uppstod.
1Sa 14:16  Och Sauls väktare i Gibea i Benjamin fingo se att hopen var i upplösning, och att man sprang hit och dit.
1Sa 14:17  Då sade Saul till folket som han hade hos sig: "Hållen mönstring och sen efter, vem som har gått ifrån oss." När de då höllo mönstring, funno de att Jonatan och hans vapendragare icke voro där.
1Sa 14:18  Då sade Saul till Ahia: "För hit Guds ark." Ty Guds ark fanns på den tiden bland Israels barn.
1Sa 14:19  Medan Saul ännu talade med prästen, tilltog larmet i filistéernas läger allt mer och mer. Då sade Saul till prästen: "Låt det vara."
1Sa 14:20  Och Saul och allt det folk som han hade hos sig församlade sig och drogo till stridsplatsen; där fingo de se att den ene hade lyft sitt svärd mot den andre, så att en mycket stor förvirring hade uppstått.
1Sa 14:21  Och de hebréer som sedan gammalt lydde under filistéerna, och som hade dragit hitupp med dem och voro här och där i lägret, dessa slöto sig nu ock till de israeliter som anfördes av Saul och Jonatan.
1Sa 14:22  Och när de israeliter som hade gömt sig i Efraims bergsbygd hörde att filistéerna flydde, satte alla dessa också efter dem och deltogo i striden.
1Sa 14:23  Så gav HERREN Israel seger på den dagen, och striden fortsattes ända bortom Bet-Aven.
1Sa 14:24  När nu Israels män på den dagen voro hårt ansträngda, band Saul folket med följande ed: "Förbannad vare den man som förtär någon föda före aftonen, och innan jag har tagit hämnd på mina fiender." Så smakade då ingen av folket någon föda.
1Sa 14:25  Och när de allasammans kommo in i skogsbygden, låg honung på marken.
1Sa 14:26  Men när folket hade kommit in i skogsbygden och fått se den utflutna honungen, vågade dock ingen föra handen upp till munnen, ty folket fruktade för eden.
1Sa 14:27  Jonatan däremot hade icke hört, när hans fader band folket med eden; därför räckte han ut staven som han hade i sin hand och doppade dess ända i honungskakan, och förde så handen till munnen; då kunde hans ögon åter se klart.
1Sa 14:28  Men en man bland folket tog till orda och sade: "Din fader har bundit folket med en dyr ed och sagt: 'Förbannad vare den man som dag förtär någon föda.'" Och folket var uttröttat.
1Sa 14:29  Jonatan svarade: "Min fader har därmed dragit olycka över landet. Sen huru klara mina ögon hava blivit, därför att jag smakade något litet av honungen här.
1Sa 14:30  Huru mycket mer, om folket i dag hade fått äta sig mätta av bytet som de hade tagit från sina fiender - huru mycket större skulle icke då filistéernas nederlag hava blivit!"
1Sa 14:31  Emellertid slogo de filistéerna på den dagen och förföljde dem från Mikmas till Ajalon. Och folket var mycket uttröttat.
1Sa 14:32  Därför kastade sig folket över bytet och tog får, oxar och kalvar och slaktade dem på marken; och folket åt sedan köttet med blodet i.
1Sa 14:33  När man berättade detta för Saul och sade: "Se, folket syndar mot HERREN genom att äta kött med blodet i", utropade han: "I haven handlat brottsligt. Vältren nu fram till mig en stor sten."
1Sa 14:34  Och Saul sade vidare: "Gån ut bland folket och sägen till dem 'Var och en före fram till mig sin oxe och sitt får, och slakten dem här och äten; synden icke mot HERREN genom att äta köttet med blodet i.'" Då förde allt folket, var och en med egen hand, om natten fram sina oxar och slaktade dem där.
1Sa 14:35  Och Saul byggde ett altare åt HERREN; detta var det första altare som han byggde åt HERREN.
1Sa 14:36  Och Saul sade: "Låt oss i natt draga ned och förfölja filistéerna och anställa plundring bland dem, ända till dess det bliver dager i morgon, och låt oss laga så, att ingen av dem bliver kvar." De svarade: "Gör allt vad dig täckes." Men prästen sade: "Låt oss träda fram hit till Gud."
1Sa 14:37  Då frågade Saul Gud: "Skall jag draga ned och förfölja filistéerna? Vill du då giva dem i Israels hand?" Men han gav honom intet svar den dagen.
1Sa 14:38  Då sade Saul: "Kommen hitfram, alla I folkets förnämsta män, för att I mån få veta och se vari den synd består, som i dag har blivit begången.
1Sa 14:39  Ty så sant HERREN lever, han som har givit Israel seger: om den ock vore begången av min son Jonatan, skall han döden dö." Men ingen bland allt folket svarade honom.
1Sa 14:40  Då sade han till hela Israel: "Ställen I eder på ena sidan, så vill jag med min son Jonatan ställa mig på andra sidan." Folket svarade Saul: "Gör vad dig täckes."
1Sa 14:41  Och Saul sade till HERREN, "Israels Gud: "Låt sanningen komma i dagen." Då träffades Jonatan och Saul av lotten, och folket gick fritt.
1Sa 14:42  Saul sade: "Kasten lott mellan mig och min son Jonatan." Då träffades Jonatan av lotten.
1Sa 14:43  Saul sade till Jonatan: "Omtala för mig vad du har gjort." Då omtalade Jonatan det för honom och sade: "Med ändan av staven som jag hade i min hand tog jag litet honung och smakade därpå - och så skall jag nu dö!"
1Sa 14:44  Saul svarade: "Ja, Gud straffe mig nu och framgent: du måste döden dö, Jonatan."
1Sa 14:45  Men folket sade till Saul: "Skulle Jonatan dö, han som har förskaffat Israel denna stora seger? Bort det! Så sant HERREN lever, icke ett hår från hans huvud skall falla till jorden; ty med Guds hjälp har han i dag utfört detta." Och folket köpte Jonatan fri ifrån döden.
1Sa 14:46  Och Saul drog hem, utan att vidare förfölja filistéerna; filistéerna begåvo sig ock hem till sitt.
1Sa 14:47  När Saul nu hade tagit konungadömet över Israel i besittning, förde han krig mot alla sina fiender runt omkring: mot Moab, mot Ammons barn, mot Edom, mot konungarna i Soba och mot filistéerna; och vart han vände sig tuktade han dem.
1Sa 14:48  Han gjorde mäktiga ting och slog Amalek och räddade så Israel från dess plundrares hand.
1Sa 14:49  Sauls söner voro Jonatan, Jisvi och Malki-Sua; och av hans båda döttrar hette den äldre Merab och den yngre Mikal.
1Sa 14:50  Sauls hustru hette Ahinoam, Ahimaas' dotter. Hans härhövitsman hette Abiner, son till Ner, som var Sauls farbroder.
1Sa 14:51  Ty Kis, Sauls fader, och Ner, Abners fader, voro söner till Abiel.
1Sa 14:52  Men kriget mot filistéerna pågick häftigt, så länge Saul levde. Och varhelst Saul såg någon rask och krigsduglig man tog han honom i sin tjänst.
1Sa 15:1  Men Samuel sade till Saul: "Det var mig HERREN sände att smörja dig till konung över sitt folk Israel. Så hör nu HERRENS ord.
1Sa 15:2  Så säger HERREN Sebaot: Jag vill hemsöka Amalek för det som han gjorde mot Israel, att han lade sig i vägen för honom, när han drog upp ur Egypten.
1Sa 15:3  Så drag nu åstad och slå amalekiterna och giv dem till spillo, med allt vad de hava, och skona dem icke, utan döda både män och kvinnor, både barn och spenabarn, både fäkreatur och får, både kameler och åsnor."
1Sa 15:4  Då bådade Saul upp folket och mönstrade dem i Telaim: två hundra tusen man fotfolk, och dessutom tio tusen man från Juda.
1Sa 15:5  När Saul sedan kom till Amaleks stad, lade han ett bakhåll i dalen.
1Sa 15:6  Men till kainéerna lät Saul säga: "Skiljen eder från amalekiterna och dragen ned, för att jag icke må utrota eder tillsammans med dem. I bevisaden ju barmhärtighet mot alla Israels barn, när de drogo ut ur Egypten." Då skilde sig kainéerna från amalekiterna.
1Sa 15:7  Och Saul slog amalekiterna och förföljde dem från Havila fram emot Sur, som ligger öster om Egypten.
1Sa 15:8  Och han tog Agag, Amaleks konung, levande till fånga, och allt folket gav han till spillo, och han slog dem med svärdsegg.
1Sa 15:9  Men Saul och folket skonade Agag, så ock det bästa och det näst bästa av får och fäkreatur jämte lammen, korteligen, allt som var av värde; sådant ville de icke giva till spillo. All boskap däremot, som var dålig och mager, gåvo de till spillo.
1Sa 15:10  Då kom HERRENS ord till Samuel; han sade:
1Sa 15:11  "Jag ångrar att jag har gjort Saul till konung, ty han har vänt sig bort ifrån mig och icke fullgjort mina befallningar." Detta gick Samuel hårt till sinnes, och han ropade till HERREN hela den natten.
1Sa 15:12  Och bittida om morgonen stod Samuel upp och gick för att möta Saul. Då blev det berättat för Samuel att Saul hade kommit till Karmel och där rest åt sig en minnesstod, och att han sedan hade vänt om och dragit därifrån ned till Gilgal.
1Sa 15:13  När nu Samuel kom till Saul, sade Saul till honom: "Välsignad vare du av HERREN. Jag har nu fullgjort HERRENS befallning."
1Sa 15:14  Men Samuel sade: "Vad är det då för ett läte av får som ljuder i mina öron, och vad är det för ett läte av fäkreatur som jag hör?"
1Sa 15:15  Saul svarade: "Från amalekiterna hava de fört dem med sig, ty folket skonade det bästa av fåren och fäkreaturen för att offra det åt HERREN, din Gud; men det övriga hava vi givit till spillo."
1Sa 15:16  Då sade Samuel till Saul: "Håll nu upp, så vill jag förkunna för dig vad HERREN i natt har talat till mig. Han sade till honom: "Tala."
1Sa 15:17  Samuel sade: "Se, fastän du var ringa i dina egna ögon, har du blivit ett huvud för Israels stammar, ty HERREN smorde dig till konung över Israel.
1Sa 15:18  Och HERREN sände dig åstad och sade: 'Gå och giv till spillo amalekiterna, de syndarna, och strid mot dem, till dess att du har förgjort dem.'
1Sa 15:19  Varför har du då icke hört HERRENS röst, utan kastat dig över bytet och gjort vad ont är i HERRENS ögon?"
1Sa 15:20  Saul svarade Samuel: "Jag har ju hört HERRENS röst och gått den väg på vilken HERREN har sänt mig. Jag har fört hit Agag, Amaleks konung, och givit Amalek till spillo.
1Sa 15:21  Men folket tog av bytet far och fäkreatur, det bästa av det tillspillogivna, för att offra det åt HERREN din Gud, i Gilgal."
1Sa 15:22  Då sade Samuel: "Menar du att HERREN har samma behag till brännoffer och slaktoffer som därtill att man hör HERRENS röst? Nej, lydnad är bättre än offer, och hörsamhet bättre än det feta av vädurar.
1Sa 15:23  Ty gensträvighet är trolldomssynd, och motspänstighet är avguderi och husgudsdyrkan. Eftersom du har förkastat HERRENS ord, har han ock förkastat dig, och du skall icke längre vara konung."
1Sa 15:24  Saul sade till Samuel: "Jag har syndat därmed att jag har överträtt HERRENS befallning och handlat emot dina ord; ty jag fruktade för folket och lyssnade till deras ord.
1Sa 15:25  Men förlåt mig nu min synd, och vänd tillbaka med mig, så att jag får tillbedja HERREN."
1Sa 15:26  Samuel sade till Saul: "Jag vänder icke tillbaka med dig; ty då du har förkastat HERRENS ord, har HERREN ock förkastat dig, så att du icke längre får vara konung över Israel."
1Sa 15:27  När nu Samuel vände sig om för att gå, fattade han i hörnet på hans mantel, och den rycktes sönder.
1Sa 15:28  Och Samuel sade till honom: "HERREN har i dag ryckt Israels konungarike från dig och givit det åt en annan, som är bättre än du.
1Sa 15:29  Och den Härlige i Israel ljuger icke och ångrar sig icke; ty han är icke en människa, så att han skulle kunna ångra sig."
1Sa 15:30  Han svarade: "Jag har syndat; men bevisa mig dock nu den äran inför de äldste i mitt folk och inför Israel, att du vänder tillbaka med mig, så att jag får tillbedja HERREN, din Gud."
1Sa 15:31  Då vände Samuel tillbaka och följde med Saul; och Saul tillbad HERREN.
1Sa 15:32  Och Samuel sade: "Fören fram till mig Agag, Amaleks konung." Då gick Agag med glatt mod fram till honom. Och Agag sade: "Välan, snart är dödens bitterhet överstånden."
1Sa 15:33  Men Samuel sade: "Såsom ditt svärd har gjort kvinnor barnlösa så skall ock din moder bliva barnlös framför andra kvinnor." Därpå högg Samuel Agag i stycken inför HERREN, i Gilgal.
1Sa 15:34  Sedan begav sig Samuel till Rama; men Saul drog upp till sitt hem i Sauls Gibea.
1Sa 15:35  Och Samuel ville icke mer se Saul så länge han levde, ty Samuel sörjde över Saul, eftersom HERREN ångrade att han hade gjort Saul till konung över Israel.
1Sa 16:1  Och HERREN sade till Samuel: "Huru länge tänker du sörja över Saul? Jag har ju förkastat honom, ty jag vill icke längre att han skall vara konung över Israel. Fyll ditt horn med olja och gå åstad jag vill sända dig till betlehemiten Isai, ty en av hans söner har jag utsett åt mig till konung."
1Sa 16:2  Men Samuel sade: "Huru skall jag kunna gå dit? Om Saul får höra det, så dräper han mig." HERREN svarade: "Tag en kviga med dig och säg: 'Jag har kommit för att offra åt HERREN.'
1Sa 16:3  Sedan skall du inbjuda Isai till offret, och jag skall då själv låta dig veta vad du bör göra, och du skall smörja åt mig den jag säger dig."
1Sa 16:4  Samuel gjorde vad HERREN hade sagt, och kom så till Bet-Lehem Men när de äldste i staden fingo se honom, blevo de förskräckta och frågade: "Allt står väl rätt till?"
1Sa 16:5  Han svarade: "Ja. Jag har kommit för att offra åt HERREN. Helgen eder och kommen med mig tid offret." Och han helgade Isai och hans söner och inbjöd dem till offret.
1Sa 16:6  När de nu kommo dit och han fick se Eliab, tänkte han: "Förvisso står HERRENS smorde här inför honom."
1Sa 16:7  Men HERREN sade till Samuel "Skåda icke på hans utseende och på hans högväxta gestalt, ty jag har förkastat honom. Ty det är icke såsom en människa ser; en människa ser på det som är för ögonen men HERREN ser till hjärtat."
1Sa 16:8  Då kallade Isai på Abinadab och lät honom gå fram för Samuel. Men han sade: "Icke heller denne har HERREN utvalt."
1Sa 16:9  Då lät Isai Samma gå fram. Men han sade: "Icke heller denne har HERREN utvalt."
1Sa 16:10  På detta sätt lät Isai sju av sina söner gå fram för Samuel; men Samuel sade till Isai: "HERREN har icke utvalt någon av dessa."
1Sa 16:11  Och Samuel frågade Isai: "Är detta alla ynglingarna?" Han svarade: "Ännu återstår den yngste, men han går nu i vall med fåren." Då sade Samuel till Isai: "Sänd åstad och hämta hit honom, ty vi skola icke sätta oss till bords, förrän han kommer hit."
1Sa 16:12  Då sände han åstad och lät hämta honom, och han var ljuslätt och hade sköna ögon och ett fagert utseende. Och HERREN sade: "Stå upp och smörj honom, ty denne är det."
1Sa 16:13  Då tog Samuel sitt oljehorn och smorde honom mitt ibland hans bröder; och HERRENS Ande kom över David, från den dagen och allt framgent. Sedan stod Samuel upp och gick till Rama.
1Sa 16:14  Men sedan HERRENS Ande hade vikit ifrån Saul, kvaldes han av en ond ande från HERREN.
1Sa 16:15  Då sade Sauls tjänare till honom: "Eftersom en ond ande från Gud kväljer dig,
1Sa 16:16  må du, vår herre, tillsäga dina tjänare, som stå inför dig, att de söka upp en man som är kunnig i harpospel, på det att han må spela på harpan, när den onde anden från Gud kommer över dig; så skall det bliva bättre med dig."
1Sa 16:17  Då sade Saul till sina tjänare: "Sen eder för min räkning om efter en man som är skicklig i strängaspel, och fören honom till mig."
1Sa 16:18  En av männen svarade då och sade: "Betlehemiten Isai har en son som jag har funnit vara kunnig i strängaspel, en käck stridsman och en förståndig man, därtill en fager man; och HERREN är med honom."
1Sa 16:19  Så sände då Saul bud till Isai och lät säga: "Sänd till mig din son David, som vaktar fåren."
1Sa 16:20  Då tog Isai en åsna, som han lastade med bröd, vidare en vinlägel och en killing, och sände detta med sin son David till Saul.
1Sa 16:21  Så kom David till Saul och trädde i hans tjänst och blev honom mycket kär, så att han fick bliva hans vapendragare.
1Sa 16:22  Och Saul sände till Isai och lät säga: "Låt David stanna kvar i min tjänst, ty han har funnit nåd för mina ögon."
1Sa 16:23  När nu anden från Gud kom över Saul, tog David harpan och spelade; då kände Saul lindring, och det blev bättre med honom, och den onde anden vek ifrån honom.
1Sa 17:1  Men filistéerna församlade sina härar till strid; de församlade sig vid det Soko som hör till Juda. Och de lägrade sig mellan Soko och Aseka, vid Efes-Dammim.
1Sa 17:2  Saul och Israels män hade ock församlat sig och lägrat sig i Terebintdalen; och de ställde upp sig till strid mot filistéerna.
1Sa 17:3  Filistéerna stodo vid berget på ena sidan, och israeliterna stodo vid berget på andra sidan, så att de hade dalen emellan sig.
1Sa 17:4  Då framträdde ur filistéernas skaror en envigeskämpe vid namn Goljat, från Gat; han var sex alnar och ett kvarter lång.
1Sa 17:5  Han hade en kopparhjälm på sitt huvud och var klädd i ett fjällpansar, och hans pansar hade en vikt av fem tusen siklar koppar.
1Sa 17:6  Och han hade benskenor av koppar och bar en lans av koppar på sin rygg.
1Sa 17:7  Skaftet på hans spjut liknade en vävbom, och spetsen på spjutet höll sex hundra siklar järn. Och hans sköldbärare gick framför honom.
1Sa 17:8  Han trädde nu fram och ropade till Israels här och sade till dem: "Varför dragen I ut och ställen upp eder till strid? Jag står här på filistéernas vägnar, och I ären Sauls tjänare; väljen nu ut åt eder en man som må komma hitned till mig.
1Sa 17:9  Om han förmår strida mot mig och slår ned mig, så skola vi vara eder underdåniga; men om jag bliver hans överman och slår ned honom, så skolen I vara oss underdåniga och tjäna oss."
1Sa 17:10  Och filistéen sade ytterligare: "Jag har i dag smädat Israels här. Skaffen nu hit någon, så att vi få strida med varandra!
1Sa 17:11  Då Saul och hela Israel hörde dessa filistéens ord, blevo de gripna av förfäran och stor fruktan.
1Sa 17:12  Men David var son till den omtalade efratiten från Bet-Lehem i Juda, som hette Isai och hade åtta söner; denne var på Sauls tid en gammal man vid framskriden ålder.
1Sa 17:13  Nu hade Isais tre äldsta söner dragit åstad och följt med Saul ut i kriget. Av dessa hans tre söner, som hade dragit ut i kriget, hette den förstfödde Eliab, hans andre son Abinadab och den tredje Samma.
1Sa 17:14  David var den yngste. De tre äldsta hade nu följt med Saul.
1Sa 17:15  Men David lämnade understundom Saul och gick hem för att vakta sin faders får i Bet-Lehem.
1Sa 17:16  Och filistéen kom fram både bittida och sent; i fyrtio dagar kom han och ställde sig där.
1Sa 17:17  Nu sade Isai en gång till sin son David: "Tag för dina bröders räkning en efa av dessa rostade ax jämte dessa tio bröd, och skaffa detta skyndsamt till dina bröder i lägret.
1Sa 17:18  Och dessa tio ostar skall du föra till deras överhövitsman. Du skall se efter, om det står väl till med dina bröder, och begära av dem en mottagningspant.
1Sa 17:19  Saul och de och alla Israels män äro nämligen i Terebintdalen och strida mot filistéerna."
1Sa 17:20  Bittida följande morgon överlämnade David fåren åt en vaktare, tog med sig vad han skulle och begav sig åstad, såsom Isai hade bjudit honom. När han kom fram till vagnborgen, hov hären, som då skulle draga ut i slagordning, upp sitt härskri.
1Sa 17:21  Och Israel och filistéerna ställde upp sig i slagordning mot varandra.
1Sa 17:22  Då lämnade David ifrån sig sakerna åt trossvaktaren och skyndade bort till hären; och när han kom dit, hälsade han sina bröder.
1Sa 17:23  Under det att han talade med dem, trädde nu envigeskämpen, han som hette Goljat, filistéen ifrån Gat, fram ur filistéernas här och talade såsom förut; och David hörde det.
1Sa 17:24  Och alla Israels män flydde för mannen, när de fingo se honom och fruktade storligen.
1Sa 17:25  Och Israels män sade: "Sen I mannen där, som nu träder upp? Han träder upp för att smäda Israel. Men den man som slår ned honom vill konungen begåva med stor rikedom, och åt honom vill han giva sin dotter, och hans faders hus vill han göra skattefritt i Israel."
1Sa 17:26  Och David sade till de man som stodo bredvid honom: "Vad får den man som slår ned denne filisté och därmed tager bort sådan smälek från Israel? Ty vem är denne oomskurne filisté, som vågar smäda den levande Gudens här?"
1Sa 17:27  Folket upprepade då för honom det som nyss hade blivit sagt; de sade: "Detta får den man som slår ned honom."
1Sa 17:28  Men Eliab, hans äldste broder, hörde huru han talade med männen; då upptändes Eliabs vrede mot David, och han sade: "Varför har du kommit hitned, och åt vem har du överlämnat den lilla fårhjorden där i öknen? Jag känner ditt övermod och ditt hjärtas ondska; för att se på striden är det som du har kommit hitned."
1Sa 17:29  David svarade: "Vad har jag då gjort? Det var ju allenast en fråga."
1Sa 17:30  Sedan vände han sig ifrån honom till en annan och upprepade sin fråga, och folket gav honom samma svar som förut.
1Sa 17:31  Men vad David hade talat blev bekant; och man berättade det för Saul, och denne lät hämta honom.
1Sa 17:32  Och David sade till Saul: "Må ingen låta sitt mod falla. Din tjänare vill gå åstad och strida mot denne filisté."
1Sa 17:33  Saul sade till David: "Icke kan du gå åstad mot denne filisté och strida mot honom; du är du ju allenast en yngling, och han är en stridsman allt ifrån ungdomen."
1Sa 17:34  Men David svarade Saul: "Din tjänare har gått i vall med sin faders får; om då ett lejon eller en björn kom och tog bort ett får av hjorden,
1Sa 17:35  så följde jag efter vilddjuret och slog ned det och ryckte rovet ur munnen på det; och om det då reste sig upp mot mig, så fattade jag det i skägget och slog ned det och dödade det.
1Sa 17:36  Har nu din tjänare slagit ned både lejon och björn, så skall det gå denne oomskurne filisté såsom det gick vart och ett av dessa djur, ty han har smädat den levande Gudens här."
1Sa 17:37  Och David sade ytterligare: "HERREN, som räddade mig undan lejon och björn, han skall ock rädda mig undan denne filisté." Då sade Saul till David: "Gå då åstad; HERREN skall vara med dig."
1Sa 17:38  Och Saul klädde på David sina egna kläder och satte en kopparhjälm på hans huvud och klädde på honom ett pansar.
1Sa 17:39  Och David omgjordade sig med hans svärd utanpå kläderna och prövade på att gå därmed, ty han hade aldrig försökt något sådant. Och David sade till Saul: "Jag kan icke gå så klädd, ty jag har aldrig försökt sådant." Därpå lade David det av sig.
1Sa 17:40  Och han tog sin stav i handen och valde ut åt sig fem släta stenar ur bäcken och lade dem i sin herdeväska och i barmen, och tog sin slunga i handen; därefter gick han fram mot filistéen.
1Sa 17:41  Och filistéen gick framåt och kom David allt närmare, och hans sköld bärare gick framför honom.
1Sa 17:42  Då nu filistéen såg upp och fick se David, föraktade han honom; ty denne var ännu en yngling, ljuslätt och skön.
1Sa 17:43  Och filistéen sade till David: "Menar du att jag är en hund, eftersom du kommer emot mig med käppar?" Och filistéen förbannade David, i det han svor vid sina gudar.
1Sa 17:44  Sedan sade filistéen till David: "Kom hit till mig, så skall jag giva ditt kött åt himmelens fåglar och åt markens djur."
1Sa 17:45  David svarade filistéen: "Du kommer mot mig med svärd och spjut och lans, men jag kommer mot dig i HERREN Sebaots namn, hans som är Israels härs Gud, den härs som du har smädat.
1Sa 17:46  HERREN skall denna dag överlämna dig i min hand, så att jag skall slå ned dig och taga ditt huvud av dig, och jag skall denna dag giva de filisteiska krigarnas döda kroppar åt himmelens fåglar och åt jordens vilda djur; så skola alla länder förnimma att Israel har en Gud.
1Sa 17:47  Och hela denna hop skall förnimma att det icke är genom svärd och spjut som HERREN giver seger; ty striden är HERRENS, och han skall giva eder i vår hand."
1Sa 17:48  När då filistéen gjorde sig redo och gick framåt och närmade sig David, sprang David med hast fram mot hären, filistéen till mötes.
1Sa 17:49  Och David stack sin hand i väskan och tog därur en sten och slungade och träffade filistéen i pannan; och stenen trängde in i pannan, så att han föll omkull med ansiktet mot jorden.
1Sa 17:50  Så övervann David filistéen med slunga och sten och slog filistéen till döds,
1Sa 17:51  utan att David därvid hade något svärd i sin hand. Sedan sprang David fram och ställde sig invid filistéen och fattade i hans svärd; och när han hade dragit det ut ur skidan, gav han honom dödsstöten och högg av hans huvud därmed. När filistéerna nu sågo att deras kämpe var död, flydde de.
1Sa 17:52  Men Israels och Juda man stodo upp och höjde ett härskri och förföljde filistéerna ända dit där vägen går till Gai, och ända intill Ekrons portar; och filistéer föllo och lågo slagna på vägen till Saaraim, och sedan ända till Gat och ända till Ekron.
1Sa 17:53  Sedan Israels barn sålunda häftigt hade förföljt filistéerna, vände de tillbaka och plundrade deras läger.
1Sa 17:54  Och David tog filistéens huvud och förde det till Jerusalem, men hans vapen lade han i sitt tält.
1Sa 17:55  När Saul såg David gå ut mot filistéen, frågade han härhövitsmannen Abner: "Vems son är denne yngling, Abner?" Abner svarade: "Så sant du lever, konung, jag vet det icke."
1Sa 17:56  Då sade konungen: "Hör då efter, vems son den unge mannen är."
1Sa 17:57  När sedan David vände tillbaka, efter att hava slagit ihjäl filistéen, tog Abner honom med sig och förde honom inför Saul, medan han ännu hade filistéens huvud i sin hand.
1Sa 17:58  Då sade Saul till honom: "Vems son är du, yngling" David svarade: "Din tjänare Isais, betlehemitens, son."
1Sa 18:1  Sedan, efter det att David hade talat ut med Saul, fäste sig Jonatans hjärta så vid Davids hjärta, att Jonatan hade honom lika kär som sitt eget liv.
1Sa 18:2  Och Saul tog honom till sig på den dagen och lät honom icke mer vända tillbaka till sin faders hus.
1Sa 18:3  Och Jonatan slöt ett förbund med David, då han nu hade honom lika kär som sitt eget liv.
1Sa 18:4  Och Jonatan tog av sig manteln som han hade på sig och gav den åt David, så ock sina övriga kläder, ända till sitt svärd, sin båge och sitt bälte
1Sa 18:5  Och när David drog ut, hade han framgång överallt dit Saul sände honom; Saul satte honom därför över krigsfolket. Och allt folket fann behag i honom, också de som voro Sauls tjänare.
1Sa 18:6  Och när de kommo hem, då David vände tillbaka, efter att hava slagit ned filistéen, gingo kvinnorna ut från alla Israels städer, under sång och dans, för att möta konung Saul med jubel, med pukor och trianglar.
1Sa 18:7  Och kvinnorna sjöngo med fröjd sålunda: "Saul har slagit sina tusen, men David sina tio tusen."
1Sa 18:8  Då blev Saul mycket vred, ty det talet misshagade honom, och han sade: "Åt David hava de givit tio tusen, och åt mig hava de givit tusen; nu fattas honom allenast konungadömet."
1Sa 18:9  Och Saul såg med ont öga på David från den dagen och allt framgent.
1Sa 18:10  Dagen därefter kom en ond ande från Gud över Saul, så att han rasade i sitt hus; men David spelade på harpan, såsom han dagligen plägade. Och Saul hade sitt spjut i handen.
1Sa 18:11  Och Saul svängde spjutet och tänkte: "Jag skall spetsa David fast vid väggen." Men David böjde sig undan för honom, två gånger.
1Sa 18:12  Och Saul fruktade för David, eftersom HERREN var med honom, sedan han hade vikit ifrån Saul.
1Sa 18:13  Därför avlägsnade Saul honom ifrån sig, i det att han gjorde honom till överhövitsman i sin här; han blev så folkets ledare och anförare.
1Sa 18:14  Och David hade framgång på alla sina vägar, och HERREN var med honom.
1Sa 18:15  Då nu Saul såg att han hade så stor framgång, fruktade han honom än mer.
1Sa 18:16  Men hela Israel och Juda hade David kär, eftersom han var deras ledare och anförare.
1Sa 18:17  Och Saul sade till David: "Se, min äldsta dotter, Merab, vill jag giva dig till hustru; skicka dig allenast såsom en tapper man i min tjänst, och för HERRENS krig." Ty Saul tänkte: "Min hand må icke drabba honom, filistéernas hand må drabba honom."
1Sa 18:18  Men David svarade Saul: "Vem är jag, vilka hava mina levnadsförhållanden varit, och vad är min faders släkt i Israel, eftersom jag skulle bliva konungens måg?"
1Sa 18:19  När tiden kom att Sauls dotter Merab skulle hava givits åt David, blev hon emellertid given till hustru åt meholatiten Adriel. -
1Sa 18:20  Men Sauls dotter Mikal hade David kär. Och när man omtalade detta för Saul, behagade det honom.
1Sa 18:21  Saul tänkte nämligen: "Jag skall giva henne åt honom, för att hon må bliva honom en snara, så att filistéernas hand drabbar honom." Och Saul sade till David: "För andra gången kan du nu bliva min måg."
1Sa 18:22  Och Saul bjöd sina tjänare att de hemligen skulle tala så med David: "Se, konungen har behag till dig, och alla hans tjänare hava dig kär; du hör nu bliva konungens måg."
1Sa 18:23  Och Sauls tjänare talade dessa ord i Davids öron. Men David sade: "Tyckes det eder vara en så ringa sak att bliva konungens måg? Jag är ju en fattig och ringa man."
1Sa 18:24  Detta omtalade Sauls tjänare för honom och sade: "Så har David sagt."
1Sa 18:25  Då tillsade Saul dem att de skulle säga så till David: "Konungen begär ingen annan brudgåva än förhudarna av ett hundra filistéer, för att hämnd så må tagas på konungens fiender." Saul hoppades nämligen att han skulle få David fälld genom filistéernas hand.
1Sa 18:26  När så hans tjänare omtalade för David vad han hade sagt, ville David gärna på det villkoret bliva konungens måg; och innan tiden ännu var förlupen,
1Sa 18:27  stod David upp och drog åstad med sina män och slog av filistéerna två hundra man. Och David tog deras förhudar med sig, och fulla antalet blev överlämnat åt konungen, för att han skulle bliva konungens måg. Och Saul gav honom så sin dotter Mikal till hustru.
1Sa 18:28  Men Saul såg och förstod att HERREN var med David; Och Sauls dotter Mikal hade honom kär.
1Sa 18:29  Då fruktade Saul ännu mer för David, och så blev Saul Davids fiende för hela livet.
1Sa 18:30  Men filistéernas furstar drogo i fält; och så ofta de drogo ut, hade David större framgång än någon annan av Sauls tjänare, så att hans namn blev mycket berömt.
1Sa 19:1  Och Saul talade med sin son Jonatan och med alla sina tjänare om att döda David; men Sauls son Jonatan var David mycket tillgiven.
1Sa 19:2  Därför omtalade Jonatan detta för David och sade: "Min fader Saul söker att döda dig. Tag dig alltså till vara i morgon och håll dig gömd på någon plats där du kan vara dold.
1Sa 19:3  Men själv vill jag gå ut och ställa mig bredvid min fader på marken, där du är, och jag vill tala om dig med min fader; om jag då märker något, skall jag sedan omtala det för dig.
1Sa 19:4  Och Jonatan talade till Davids bästa med sin fader Saul och sade till honom: "Konungen må icke försynda sig på sin tjänare David ty han har icke försyndat sig mot dig, utan vad han har gjort har varit till stort gagn för dig.
1Sa 19:5  Han tog ju sin själ i sin hand och slog ned filistéen, och HERREN gav så hela Israel en stor seger; du har själv sett det och glatt dig däråt. Varför skulle du då försynda dig på oskyldigt blod genom att döda David utan sak?"
1Sa 19:6  Och Saul lyssnade till Jonatans ord; och Saul svor: "Så sant HERREN lever, han skall icke dödas."
1Sa 19:7  Sedan kallade Jonatan David till sig; och Jonatan omtalade för honom allt som hade blivit sagt. Därefter förde Jonatan David till Saul, och han var i hans tjänst såsom förut.
1Sa 19:8  När så kriget åter begynte, drog David ut och stridde mot filistéerna och tillfogade dem ett stort nederlag, så att de flydde för honom.
1Sa 19:9  Men en ond ande från HERREN kom över Saul, där han satt i sitt hus med spjutet i handen, under det att David spelade på harpan.
1Sa 19:10  Då sökte Saul att med spjutet spetsa David fast vid väggen; men denne vek undan för Saul, så att han allenast stötte spjutet in i väggen. Och David flydde och kom undan samma natt.
1Sa 19:11  Emellertid sände Saul till Davids hus några män med uppdrag att vakta på honom och att sedan om morgonen döda honom. Men Mikal, Davids hustru, omtalade detta för honom och sade: "Om du icke i natt räddar ditt liv, så är du i morgon dödens man."
1Sa 19:12  Därefter släppte Mikal ned David genom fönstret; och han begav sig på flykten och kom så undan.
1Sa 19:13  Sedan tog Mikal husguden och lade honom i sängen och satte myggnätet av gethår över huvudgärden och höljde täcket över honom.
1Sa 19:14  När sedan Saul sände sina män med uppdrag att hämta David, sade hon: "Han är sjuk."
1Sa 19:15  Då sände Saul dit männen med uppdrag att skaffa sig tillträde till David själv och sade: "Bären honom i sängen hitupp till mig, så att jag får döda honom."
1Sa 19:16  Men när männen kommo in, fingo de se att det var husguden som låg i sängen, med myggnätet över huvudgärden.
1Sa 19:17  Då sade Saul till Mikal: "Varför har du så bedragit mig och släppt min fiende, så att han har kommit undan?" Mikal svarade Saul: "Han sade till mig: 'Släpp mig; eljest dödar jag dig.'"
1Sa 19:18  När David nu hade flytt och kommit undan, begav han sig till Samuel i Rama och omtalade för denne allt vad Saul hade gjort honom. Och han och Samuel gingo till Najot och stannade där.
1Sa 19:19  Och det blev berättat för Saul att David var i Najot vid Rama.
1Sa 19:20  Då sände Saul dit några män med uppdrag att hämta David. Men när Sauls utskickade fingo se skaran av profeterna i profetisk hänryckning, och fingo se Samuel stå där såsom deras anförare, kom Guds Ande över dem, så att också de fattades av hänryckning.
1Sa 19:21  När man omtalade detta för Saul, sände han dit andra män; men också de fattades av hänryckning. Och när han då ytterligare, för tredje gången, sände dit män med samma uppdrag, fattades också dessa av hänryckning.
1Sa 19:22  Då begav han sig själv till Rama; och när han kom till den stora brunnen i Seku, frågade han: "Var äro Samuel och David?" Man svarade: "De äro i Najot vid Rama."
1Sa 19:23  Då begav han sig dit, till Najot vid Rama. Men Guds Ande kom också över honom, så att han hela vägen gick i profetisk hänryckning, ända till dess att han kom fram till Najot vid Rama.
1Sa 19:24  Då kastade också han av sig sina kläder, i det att också han blev fattad av hänryckning inför Samuel; och han föll ned och låg där naken hela den dagen och hela natten. Därför plägar man säga: "Är ock Saul bland profeterna?"
1Sa 20:1  Men David flydde från Najot vid Rama och kom till Jonatan och sade: "Vad har jag gjort? Vilken missgärning, vilken synd har jag begått mot dig fader, eftersom han står efter mitt liv?"
1Sa 20:2  Han svarade honom: "Bort det! Du skall icke dö. Min fader gör ju intet, varken något viktigt eller något oviktigt, utan att uppenbara det för mig. Varför skulle då min fader dölja detta för mig? Nej, så skall icke ske."
1Sa 20:3  Men David betygade ytterligare med ed och sade: "Din fader vet väl att jag har funnit nåd för dina ögon; därför tänker han: 'Jonatan skall icke få veta detta, på det att han icke må bliva bedrövad.' Men så sant HERREN lever, och så sant du själv lever: det är icke mer än ett steg mellan mig och döden."
1Sa 20:4  Då sade Jonatan till David: "Vadhelst du önskar skall jag göra för dig."
1Sa 20:5  David sade till Jonatan: "I morgon är ju nymånad, och jag skulle då rätteligen sitta till bords med konungen; men låt mig nu gå och gömma mig ute på marken till i övermorgon afton.
1Sa 20:6  Om då din fader saknar mig, så säg: 'David utbad sig tillstånd av mig att få göra ett hastigt besök i sin stad, Bet-Lehem, där hela släkten nu firar sin årliga offerfest.'
1Sa 20:7  Om han då säger: 'Gott!', så kan din tjänare vara trygg. Men om han bliver vred, så märker du därav att han har beslutit min ofärd.
1Sa 20:8  Visa så din nåd mot din tjänare, eftersom du har låtit din tjänare ingå ett HERRENS förbund med dig. Men om det finnes någon missgärning hos mig, så döda mig du, ty varför skulle du föra mig till din fader?"
1Sa 20:9  Då sade Jonatan: "Bort det! Om jag märker att min fader har beslutit att låta ofärd komma över dig, skall jag förvisso omtala det för dig."
1Sa 20:10  Men David sade till Jonatan: "Vem skall omtala för mig detta, eller säga mig om din fader giver dig ett hårt svar?"
1Sa 20:11  Jonatan sade till David: "Kom, låt oss gå ut på marken." Och de gingo båda ut på marken.
1Sa 20:12  Och Jonatan sade till David: "Vid HERREN" Israels Gud: om jag finner att det låter gott för David, när jag i morgon eller i övermorgon vid denna tid utforskar min fader, så skall jag förvisso sända bud till dig och uppenbara det för dig.
1Sa 20:13  HERREN straffe Jonatan nu och framgent, om jag, såframt min fader åstundar din ofärd, icke uppenbarar det för dig och låter dig komma undan, så att du får gå dina färde i trygghet. Och HERREN vare då med dig, såsom han har varit med min fader.
1Sa 20:14  Och nog skall du väl, om jag då ännu är i livet, ja, nog skall du väl bevisa barmhärtighet mot mig, såsom HERREN är barmhärtig, så att jag slipper att dö?
1Sa 20:15  Icke skall du väl någonsin taga bort din barmhärtighet från mitt hus, icke ens då när HERREN har tagit bort alla Davids fiender ifrån jorden?"
1Sa 20:16  Jonatan slöt då ett förbund med Davids hus; och HERREN utkrävde sedan av Davids fiender vad de hade förskyllt.
1Sa 20:17  Och Jonatan besvor David ytterligare vid sin kärlek till honom, ty han hade honom lika kär som han hade sitt eget liv;
1Sa 20:18  Jonatan sade till honom: "I morgon är nymånad, och du skall då saknas, ty din plats kommer ju att stå tom.
1Sa 20:19  Men gå i övermorgon skyndsamt ned till den plats där du gömde dig den dag då ogärningen skulle hava skett, och uppehåll dig bredvid Eselstenen.
1Sa 20:20  Jag vill då själv i dess närhet avskjuta mina tre pilar, såsom sköte jag till måls.
1Sa 20:21  Sedan skall jag skicka min tjänare att gå och söka upp pilarna. Om jag då säger till tjänaren: "Se, pilarna ligga bakom dig, närmare hitåt', så tag du upp dem och kom fram, ty då kan du vara trygg och ingenting är på färde, så sant HERREN lever.
1Sa 20:22  Men om jag säger så till den unge mannen: 'Se, pilarna ligga framför dig, längre bort', så gå dina färde, ty då sänder HERREN dig bort.
1Sa 20:23  Och i fråga om det som jag och du nu hava talat, är HERREN vittne mellan mig och dig till evig tid."
1Sa 20:24  Och David gömde sig ute på marken. Och när nymånaden var inne, satte konungen sig till bords för att äta.
1Sa 20:25  Konungen satte sig på sin vanliga sittplats, platsen vid väggen; och Jonatan stod upp, och Abner satte sig vid Sauls sida. Men Davids plats stod tom.
1Sa 20:26  Saul sade dock intet den dagen, ty han tänkte: "Något har hänt honom; han är nog icke ren, säkerligen är han icke ren."
1Sa 20:27  Men när Davids plats stod tom också dagen efter nymånadsdagen, dagen därefter, sade Saul till sin son Jonatan: "Varför har Isais son varken i går eller i dag kommit till måltiden?"
1Sa 20:28  Jonatan svarade Saul: "David utbad sig tillstånd av mig att få gå till Bet-Lehem;
1Sa 20:29  han sade: 'Låt mig gå, ty vi fira en släktofferfest i staden, och min broder har själv bjudit mig att komma; om jag har funnit nåd för dina ögon, så låt mig nu slippa härifrån för att besöka mina bröder.' Därför har han icke kommit till konungens bord."
1Sa 20:30  Då upptändes Sauls vrede mot Jonatan, och han sade till honom: "Du son till en otuktig kvinna! Visste jag då icke att du hade funnit behag i Isais son, till skam för dig själv och till skam för din moders blygd!
1Sa 20:31  Ty så länge Isais son lever på jorden, är varken du eller din konungamakt säker. Sänd därför nu åstad och låt hämta honom hit till mig, ty han är dödens barn."
1Sa 20:32  Jonatan svarade sin fader Saul och sade till honom: "Varför skall han dödas? Vad har han gjort?"
1Sa 20:33  Då svängde Saul spjutet mot honom för att genomborra honom; och nu märkte Jonatan att hans fader hade beslutit att döda David.
1Sa 20:34  Och Jonatan stod upp från bordet i vredesmod och åt intet på den andra nymånadsdagen, ty han var bedrövad för Davids skull, därför att hans fader hade gjort sådan orätt mot denne.
1Sa 20:35  Följande morgon gick Jonatan ut på marken, vid den tid han hade utsatt för David; och han hade en liten gosse med sig.
1Sa 20:36  Och han sade till gossen: "Spring och sök reda på pilarna som jag skjuter av." Medan nu gossen sprang, sköt han pilen över honom.
1Sa 20:37  Och när gossen kom till det ställe dit Jonatan hade avskjutit pilen, ropade Jonatan efter gossen och sade: "Pilen ligger ju framför dig, längre bort."
1Sa 20:38  Och Jonatan ropade ytterligare efter gossen: "Fort, skynda dig, stanna icke!" Och gossen som Jonatan hade med sig tog upp pilen och kom till sin herre.
1Sa 20:39  Men gossen visste icke varom fråga var; allenast Jonatan och David visste det.
1Sa 20:40  Och Jonatan lämnade sina vapen åt gossen som han hade med sig och sade till honom: "Gå och bär dem in i staden."
1Sa 20:41  Men sedan gossen hade gått, reste David sig upp på södra sidan; och han föll ned till jorden på sitt ansikte och bugade sig tre gånger; och de kysste varandra och gräto med varandra, och David grät överljutt.
1Sa 20:42  Och Jonatan sade till David: "Gå i frid. Blive det såsom vi båda svuro vid HERRENS namn, när vi sade: 'HERREN vare vittne mellan mig och dig, mellan mina efterkommande och dina, till evig tid.'"
1Sa 20:43  Sedan stod han upp och gick sina färde, men Jonatan gick in i staden igen.
1Sa 21:1  Och David kom till prästen Ahimelek i Nob. Men Ahimelek blev förskräckt, när han fick se David, och frågade honom: "Varför kommer du ensam och har ingen med dig?"
1Sa 21:2  David svarade prästen Ahimelek: "Konungen har givit mig ett uppdrag, men han sade till mig: 'Ingen får veta något om det uppdrag vari jag sänder dig, och som jag har givit dig.' Och mina män har jag visat till det och det stället.
1Sa 21:3  Giv mig nu vad du har till hands, fem bröd eller vad som kan finnas."
1Sa 21:4  Prästen svarade David och sade "Vanligt bröd har jag icke till hands; allenast heligt bröd finnes - om eljest dina män hava avhållit sig från kvinnor."
1Sa 21:5  David svarade prästen och sade till honom: "Ja, sannerligen, kvinnor hava på sista tiden varit skilda från oss; när jag drog åstad, voro ock mina mäns tillhörigheter heliga. Därför, om också vårt förehavande är av helt vanligt slag, är det dock i dag heligt, vad våra tillhörigheter angår."
1Sa 21:6  Då gav prästen honom av det heliga; ty där fanns icke något annat bröd än skådebröden, som hade legat inför HERRENS ansikte, men som man hade burit undan, för att lägga fram nybakat bröd samma dag det gamla togs bort.
1Sa 21:7  Men där befann sig den dagen en av Sauls tjänare, som var satt i förvar inför HERREN, en edomé vid namn Doeg, den förnämste av Sauls herdar.
1Sa 21:8  Och David frågade Ahimelek ytterligare: "Har du icke här till hands något spjut eller något svärd? Ty varken mitt svärd eller mina andra vapen tog jag med mig, eftersom konungens uppdrag krävde så stor skyndsamhet."
1Sa 21:9  Prästen svarade: "Jo, det svärd som har tillhört filistéen Goljat, honom som du slog ned i Terebintdalen; det finnes, inhöljt i ett kläde, där bakom efoden. Vill du taga det med dig, så tag det; ty något annat än det har jag icke." David sade: "Dess like finnes icke; giv mig det."
1Sa 21:10  Och David stod upp och flydde samma dag för Saul och kom till Akis, konungen i Gat.
1Sa 21:11  Men Akis' tjänare sade till honom: "Detta är ju David, landets konung! Det är ju till dennes ära man sjunger så under dansen: 'Saul har slagit sina tusen, men David sina tio tusen.'
1Sa 21:12  David lade märke till dessa ord, och han begynte storligen frukta för Akis, konungen i Gat.
1Sa 21:13  Därför ställde han sig vansinnig inför deras ögon och betedde sig såsom en ursinnig, när de ville fasthålla honom, och ritade på dörrarna i porten och lät spotten rinna ned i sitt skägg.
1Sa 21:14  Då sade Akis till sina tjänare: "I sen ju huru vanvettigt mannen beter sig. Varför fören I honom till mig?
1Sa 21:15  Har jag då sådan brist på vanvettiga människor, att I behövden föra denne hit, för att han skulle bete sig vanvettigt inför mig? Skulle en sådan få komma in i mitt hus?"
1Sa 22:1  Då begav sig David därifrån och flydde undan till Adullams grotta. Och när hans bröder och hela hans faders hus fingo höra detta, kommo de ditned till honom.
1Sa 22:2  Och till honom församlade sig alla slags män som voro i något trångmål, alla som ansattes av fordringsägare och alla missnöjda, och han blev deras hövding; vid pass fyra hundra man slöto sig så till honom.
1Sa 22:3  Därifrån begav sig David till Mispe i Moab. Och han sade till konungen i Moab: "Låt min fader och min moder få komma hitöver och vara hos eder till dess jag får veta vad Gud vill göra med mig."
1Sa 22:4  Och han förde dem fram inför konungen i Moab; och de fingo stanna hos denne, så länge David var på borgen.
1Sa 22:5  Men profeten Gad sade till David: "Du skall icke stanna här på borgen; drag bort härifrån och begiv dig in i Juda land." Då drog David bort därifrån och kom till Heretskogen.
1Sa 22:6  Och Saul fick höra att man hade fått spaning på David och de män som voro med honom. Då nu Saul en dag satt i Gibea under tamarisken på höjden, med sitt spjut i handen, under det att alla hans tjänare stodo omkring honom,
1Sa 22:7  sade han till sina tjänare, där de stodo omkring honom: "Hören, I benjaminiter. Skall då också Isais son åt eder alla giva åkrar och vingårdar och göra eder alla till över- och underhövitsmän?
1Sa 22:8  Ty I haven ju alla sammansvurit eder mot mig, och ingen har uppenbarat för mig att min son har slutit förbund med Isais son. Ingen av eder bekymrar sig så mycket om mig, att han har uppenbarat det för mig. Min son har ju uppeggat min tjänare till att stämpla mot mig, såsom nu sker."
1Sa 22:9  Edoméen Doeg, som ock stod där bland Sauls tjänare, svarade då och sade: "Jag har sett Isais son komma till Ahimelek, Ahitubs son, i Nob.
1Sa 22:10  Denne frågade då HERREN för honom och gav honom reskost; han gav honom ock filistéen Goljats svärd."
1Sa 22:11  Då sände konungen och lät kalla till sig prästen Ahimelek, Ahitubs son, och hela hans faders hus, prästerna i Nob. Och de kommo alla till konungen.
1Sa 22:12  Då sade Saul: "Hör mig, du Ahitubs son." Han svarade: "Jag hör dig, min herre."
1Sa 22:13  Saul sade till honom: "Varför haven I sammansvurit eder mot mig, du och Isais son, i det att du har givit honom bröd och svärd och frågat Gud för honom, så att han skulle kunna sätta sig upp mot mig och stämpla mot mig, såsom nu sker?"
1Sa 22:14  Ahimelek svarade konungen och sade: "Vem bland alla dina tjänare är väl så betrodd som David, han som därtill är konungens måg och hövding för din livvakt och högt ärad i ditt hus?
1Sa 22:15  Är det då nu för första gången som jag har frågat Gud för honom? Bort det! Icke må konungen lägga mig, sin tjänare, och hela min faders hus något till last, ty din tjänare visste alls intet om allt detta."
1Sa 22:16  Men konungen sade: "Du måste döden dö, Ahimelek, du själv och hela din faders hus."
1Sa 22:17  Och konungen sade till drabanterna som stodo där omkring honom: "Träden fram och döden HERRENS präster; ty också de hålla med David; och fastän de visste att han flydde, uppenbarade de det icke för mig." Men konungens tjänare ville icke uträcka sina händer till att stöta ned HERRENS präster.
1Sa 22:18  Då sade konungen till Doeg: "Träd du fram och stöt ned prästerna." Edoméen Doeg trädde då fram och stötte ned prästerna och dödade på den dagen åttiofem män som buro linne-efod.
1Sa 22:19  Och invånarna i präststaden Nob blevo slagna med svärdsegg, både män och kvinnor, både barn och spenabarn; också fäkreatur, åsnor och får blevo slagna med svärdsegg.
1Sa 22:20  Allenast en son till Ahimelek, Ahitubs son, vid namn Ebjatar, kom undan, och denne flydde bort till David.
1Sa 22:21  Och Ebjatar omtalade för David att Saul hade dräpt HERRENS präster.
1Sa 22:22  Då sade David till Ebjatar: "Jag förstod redan då att edoméen Doeg, eftersom han var där, skulle omtala allt för Saul. Det är jag som är orsaken till att hela din faders hus har förgåtts.
1Sa 22:23  Bliv kvar hos mig, frukta intet; ty den som står efter mitt liv, han står ock efter ditt liv. Hos mig är du i gott förvar."
1Sa 23:1  Och man berättade för David: "Filistéerna hålla nu på att belägra Kegila, och de plundra logarna."
1Sa 23:2  Då frågade David HERREN: "Skall jag draga åstad och slå dessa filistéer?" HERREN svarade David: "Drag åstad och slå filistéerna och fräls Kegila."
1Sa 23:3  Men Davids män sade till honom: "Vi leva ju i fruktan redan här i Juda. Och nu skulla vi därtill draga åstad till Kegila, mot filistéernas här!"
1Sa 23:4  Då frågade David HERREN ännu en gång, och HERREN svarade honom och sade: "Stå upp och drag ned till Kegila; ty jag vill giva filistéerna i din hand."
1Sa 23:5  Då drog David med sina män till Kegila och stridde mot filistéerna och förde bort deras boskap och tillfogade dem ett stort nederlag. Så frälste David invånarna i Kegila.
1Sa 23:6  När Ebjatar, Ahimeleks son, flydde till David i Kegila, förde han efoden med sig ditned.
1Sa 23:7  Och det blev berättat för Saul att David hade dragit in i Kegila. Då sade Saul: "Gud har förkastat honom och givit honom i min hand, ty han har själv stängt in sig genom att gå in i en stad med portar och bommar."
1Sa 23:8  Därefter bådade Saul upp allt folket till strid, för att draga ned till Kegila och där innesluta David och hans man.
1Sa 23:9  Men när David fick veta att Saul stämplade ont mot honom, sade han till prästen Ebjatar: "Bär hit efoden."
1Sa 23:10  Och David sade: "HERRE, Israels Gud, din tjänare har hört att Saul har i sinnet att komma mot Kegila och fördärva staden för min skull.
1Sa 23:11  Skola Kegilas borgare då utlämna mig åt honom? Skall Saul komma hitned, såsom din tjänare har hört? HERRE, Israels Gud, förkunna det för din tjänare." HERREN svarade: "Han skall komma hitned."
1Sa 23:12  David frågade ytterligare: "Skola Kegilas borgare då utlämna mig och mina man åt Saul?" HERREN svarade: "De skola utlämna eder."
1Sa 23:13  Då bröt David upp med sitt folk, som utgjorde vid pass sex hundra man, och de drogo ut från Kegila och vandrade vart de kunde. När det då blev berättat för Saul att David hade flytt undan från Kegila avstod han från att draga ut.
1Sa 23:14  Så uppehöll sig nu David i öknen på bergfästena; han uppehöll sig bland bergen i öknen Sif. Och Saul sökte alltjämt efter honom, men Gud gav honom icke i hans hand.
1Sa 23:15  Och medan David var i Hores i öknen Sif, förnam han att Saul hade dragit ut för att söka döda honom.
1Sa 23:16  Men Jonatan, Sauls son, stod upp och gick till David i Hores och styrkte hans mod i Gud.
1Sa 23:17  Han sade till honom: "Frukta icke; ty min fader Sauls hand skall icke träffa dig, utan du skall bliva konung över Israel, och jag skall då hava andra platsen, näst efter dig. Detta vet ock min fader Saul."
1Sa 23:18  Sedan slöto de båda ett förbund inför HERREN. Och David stannade kvar i Hores, men Jonatan gick hem igen.
1Sa 23:19  Men några sifiter drogo upp till Saul i Gibea och sade: "David håller sig nu gömd hos oss på bergfästena i Hores, på Hakilahöjden, som ligger söder om ödemarken.
1Sa 23:20  Så drag nu ditned, o konung, så snart det lyster dig att göra det. står sak bliver det då att utlämna honom åt konungen."
1Sa 23:21  Då sade Saul: "Varen välsignade av HERREN, därför att I haven velat spara mig bekymmer.
1Sa 23:22  Men gån nu och skaffen eder ytterligare visshet, och tagen reda på och sen efter, på vilket ställe han nu vistas, och vem som har sett honom där; ty man har sagt mig att han är mycket listig.
1Sa 23:23  Och sen efter och tagen reda på alla gömställen där han kan gömma sig; och kommen så igen till mig, när I haven fått visshet, så vill jag sedan gå med eder. Ty finnes han i landet, skall jag veta att söka upp honom, om jag än måste söka bland alla Juda ätter."
1Sa 23:24  Då stodo de upp och gingo till Sif före Saul. Men David och hans män voro i öknen Maon, på hedmarken, söder om ödemarken.
1Sa 23:25  När nu Saul drog åstad med sina män för att söka efter David, om talade man det för denne, och han drog då ned till klippan och stannade så i öknen Maon. När Saul hörde detta, satte han efter David in i öknen Maon.
1Sa 23:26  Och Saul gick på ena sidan om berget, och David med sina män på andra sidan. Men just som David var stadd på flykt för att komma undan Saul, under det att Saul och hans män sökte kringränna David och hans män för att taga dem till fånga
1Sa 23:27  kom en budbärare till Saul och sade: "Skynda dig och kom, ty filistéerna hava fallit in i landet."
1Sa 23:28  Då upphörde Saul att förfölja David och drog mot filistéerna. Därav fick det stället namnet Sela-Hammalekot.
1Sa 24:1  Men David drog upp därifrån och uppehöll sig sedan på En-Gedis bergfästen.
1Sa 24:2  Och när Saul kom tillbaka från tåget mot filistéerna, omtalade man för honom att David var i En-Gedis öken.
1Sa 24:3  Då tog Saul tre tusen män, utvalda ur hela Israel, och drog åstad för att söka efter David och hans män på Stenbocksklipporna.
1Sa 24:4  Och när han kom till boskapsgårdarna vid vägen, fanns där en grotta; då gick han ditin för något avsides bestyr. Men David och hans män sutto längst inne i grottan.
1Sa 24:5  Då sade Davids män till honom: "Se, detta är den dag om vilken HERREN har sagt till dig: Jag vill nu giva din fiende i din hand, så att du far göra med honom vad du finner för gott." Då stod David upp och skar oförmärkt av en flik på Sauls mantel.
1Sa 24:6  Men därefter slog Davids samvete honom, därför att han hade skurit av fliken på Sauls mantel.
1Sa 24:7  Och han sade till sina män: "HERREN låte det vara fjärran ifrån mig att jag skulle göra detta mot min herre, mot HERRENS smorde, att jag skulle uträcka min hand mot honom; han är ju HERRENS Smorde."
1Sa 24:8  Och David höll sina män tillbaka med stränga ord och tillstadde dem icke att överfalla Saul. Men när Saul hade stått upp och gått ut ur grottan och fortsatt sin färd,
1Sa 24:9  då stod ock David upp och gick ut ur grottan och ropade efter Saul: "Min herre konung!" När då Saul såg sig tillbaka, böjde David sig ned, med ansiktet mot jorden, och bugade sig.
1Sa 24:10  Och David sade till Saul: "Varför hör du på sådana människors ord, som säga att David söker din ofärd?
1Sa 24:11  Du har ju i dag med egna ögon sett hurusom jag skonade dig, när HERREN i dag hade givit dig i min hand i grottan Och man uppmanade mig att dräpa dig; jag tänkte: 'Jag vill icke uträcka min hand mot min herre; han är ju HERRENS Smorde.
1Sa 24:12  Se själv, min fader, ja, se här fliken av din mantel i min hand. Ty därav att jag skar av fliken på din mantel, men icke dräpte dig, må du märka och se att jag icke har velat göra något ont eller begå någon förbrytelse, och att jag icke har försyndat mig mot dig, fastän du traktar efter att taga mitt liv.
1Sa 24:13  HERREN skall döma mellan mig och dig, och HERREN skall hämnas mig på dig, men min hand skall icke röra dig.
1Sa 24:14  Det är såsom det gamla ordspråket säger: 'Från de ogudaktiga kommer vad ogudaktigt är'; därför skall hand icke röra dig."
1Sa 24:15  Efter vem har Israels konung dragit ut? Efter vem är det du jagar? Efter en död hund, efter en enda liten loppa!
1Sa 24:16  Så vare då HERREN domare och döme mellan mig och dig; må han se härtill och utföra min sak, ja, må han döma mig fri ifrån din hand."
1Sa 24:17  När David hade talat dessa ord till Saul, sade Saul: "De är ju din röst, min son David." Och Saul brast ut i gråt.
1Sa 24:18  Och han sade till David: "Du är rättfärdigare än jag, ty du har bevisat mig gott, under det jag har bevisat dig ont.
1Sa 24:19  Du har i dag låtit mig se din godhet mot mig, därigenom att du icke har dräpt mig, fastän HERREN hade överlämnat mig i din hand.
1Sa 24:20  Ty när någon träffar på sin fiende, plägar han då låta honom gå sin väg i ro? HERREN vedergälle dig med sitt goda för vad du denna dag har gjort mig.
1Sa 24:21  Och nu vet jag väl att du skall bliva konung, och att Israels konungadöme skall förbliva i din hand.
1Sa 24:22  Men lova mig nu med ed vid HERREN att du icke utrotar mina avkomlingar efter mig och icke utplånar mitt namn ur min faders hus."
1Sa 24:23  Då svor David Saul denna ed. Därefter drog Saul hem; men David och hans män drogo upp till borgen.
1Sa 25:1  Och Samuel dog, och hela Israel församlade sig och höll dödsklagan efter honom; och de begrovo honom där han bodde i Rama. Och David stod upp och drog ned till öknen Paran.
1Sa 25:2  I Maon fanns då en man som hade sin boskapsskötsel i Karmel, och den mannen var mycket rik; han ägde tre tusen får och ett tusen getter. Och han höll just då på att klippa sina får i Karmel.
1Sa 25:3  Mannen hette Nabal, och hans hustru hette Abigail. Hustrun hade ett gott förstånd och ett skönt utseende; men mannen var hård och ondskefull; och han var en avkomling av Kaleb.
1Sa 25:4  När nu David i öknen fick höra att Nabal klippte sina får,
1Sa 25:5  sände han dit tio unga män; och David sade till männen: "Gån upp till Karmel och begiven eder till Nabal och hälsen honom från mig.
1Sa 25:6  Och I skolen säga till mina bröder där: "Frid vare med dig själv frid vare med ditt hus, och frid vare med allt vad du har.
1Sa 25:7  Jag har nu hört att du håller på med fårklippning. Nu är det så att dina herdar hava vistats i vårt grannskap, utan att vi hava gjort dem något förfång, och utan att något har kommit bort för dem under hela den tid de hava varit i Karmel.
1Sa 25:8  Fråga dina tjänare därom, så skola de själva säga dig det. Låt nu våra män finna nåd för dina ögon. Vi hava ju kommit hit på en glad dag. Giv därför åt dina tjänare och åt din son David vad du kan hava till hands."
1Sa 25:9  När nu Davids män kommo dit, talade de på Davids vägnar till Nabal alldeles såsom det var dem befallt, och sedan väntade de stilla.
1Sa 25:10  Men Nabal svarade Davids tjänare och sade: "Vem är David, vem är Isais son? I denna tid är det många tjänare som rymma från sina herrar.
1Sa 25:11  Skulle jag taga min mat och min dryck och slaktdjuren, som jag har slaktat åt mina fårklippare, och giva detta åt män om vilka jag icke ens vet varifrån de äro?"
1Sa 25:12  Då vände Davids män om och gingo sin väg; och när de hade kommit tillbaka, berättade de for honom allt, såsom det hade tillgått.
1Sa 25:13  Då sade David till sina män: "Var och en omgjorde sig med sitt svärd." Och var och en omgjordade sig med sitt svärd; jämväl David själv omgjordade sig med sitt svärd. Och vid pass fyra hundra man följde med David ditupp, men två hundra stannade vid trossen.
1Sa 25:14  Men en av tjänarna berättade för Abigail, Nabals hustru, och sade: "David har skickat sändebud hit från öknen och låtit hälsa vår herre, men han visade av dem.
1Sa 25:15  Dessa män hava likväl varit oss mycket nyttiga; vi hava aldrig lidit något förfång, och aldrig har något kommit bort för oss under hela den tid vi drogo omkring i deras närhet, medan vi voro därute på marken.
1Sa 25:16  De voro en mur för oss både dag och natt under hela den tid vi vistades i deras grannskap, medan vi vaktade hjorden.
1Sa 25:17  Så betänk nu och se till, vad du bör göra, ty något ont är nog beslutet mot vår herre och över hela hans hus; och han är ju en ond man, så att ingen vågar säga något åt honom."
1Sa 25:18  Då gick Abigail strax och tog två hundra bröd, två vinläglar, fem tillredda får, fem sea-mått rostade ax, ett hundra russinkakor och två hundra fikonkakor, och lastade detta på åsnor.
1Sa 25:19  Och hon sade till sina tjänare: "Gån framför mig, jag vill komma efter eder." Men för sin man Nabal sade hon intet härom.
1Sa 25:20  När hon nu red på sin åsna och kom ned i en hålväg i berget, fick hon se David och hans män komma ned från motsatta sidan, så att hon måste möta dem.
1Sa 25:21  Men David hade sagt: "Förgäves har jag skyddat allt vad den mannen hade i öknen, så att intet av allt vad han ägde har kommit bort; men har har vedergällt mig med ont för gott.
1Sa 25:22  Så sant Gud må straffa Davids fiender nu och framgent, jag skall av allt som tillhör honom icke låta någon av mankön leva kvar till i morgon."
1Sa 25:23  Då nu Abigail fick se David, steg hon strax ned från åsnan och föll ned inför David på sitt ansikte och bugade sig mot jorden.
1Sa 25:24  Hon föll till hans fötter och sade: "På mig vilar denna missgärning, herre. Men låt din tjänarinna få tala inför dig, och hör på din tjänarinnas ord.
1Sa 25:25  Icke må min herre fästa något avseende vid Nabal, den onde mannen, ty vad hans namn betyder, det är han; Nabal heter han, och dårskap bor i honom. Men jag, din tjänarinna, har icke sett de män som du, min herre, sände.
1Sa 25:26  Och nu, min herre, så sant HERREN lever, och så sant du själv lever, du som av HERREN har avhållits från att ådraga dig blodskuld och skaffa dig rätt med egen hand: må det nu gå dina fiender och dem som söka bereda min herre ofärd såsom det må gå Nabal.
1Sa 25:27  Och låt nu dessa hälsningsskänker, som din trälinna har medfört till min herre, givas åt de män som följa min herre.
1Sa 25:28  Förlåt din tjänarinna vad hon har brutit. Ty HERREN skall förvisso åt min herre uppbygga ett hus som bliver beståndande, eftersom min herre för HERRENS krig; och du skall icke bliva skyldig till något ont, så länge du lever.
1Sa 25:29  Och om någon står upp för att förfölja dig och söka döda dig, så må min herres liv vara inknutet i de levandes pung hos HERREN, din Gud; men dina fienders liv må han lägga i sin slunga och slunga det bort.
1Sa 25:30  När nu HERREN gör med min herre allt det goda varom han har talat till dig, och förordnar dig till furste över Israel,
1Sa 25:31  skall alltså detta icke bliva dig en stötesten eller vara till hjärteångest för min herre, att du har utgjutit blod utan sak, och att min herre själv har skaffat sig rätt. Men när HERREN gör min herre gott, så tänk på din tjänarinna."
1Sa 25:32  Då sade David till Abigal: "Välsignad vare HERREN, Israels Gud som i dag har sänt dig mig till mötes!
1Sa 25:33  Och välsignat vare ditt förstånd, och välsignad vare du själv, som i dag har hindrat mig från att ådraga mig blodskuld och skaffa mig rätt med egen hand!
1Sa 25:34  Men så sant HERREN, Israels Gud, lever, han som har avhållit mig från att göra dig något ont: om du icke strax hade kommit mig till mötes, så skulle i morgon, när det hade blivit dager, ingen av mankön hava funnits kvar av Nabals hus."
1Sa 25:35  Därefter tog David emot av henne vad hon hade medfört åt honom; och han sade till henne: "Far i frid hem igen. Se, jag har lyssnat till dina ord och gjort dig till viljes."
1Sa 25:36  När sedan Abigail kom hem till Nabal, höll denne just i sitt hus ett gästabud, som var såsom en konungs gästabud; och Nabals hjärta var glatt i honom, och han var mycket drucken. Därför omtalade hon alls intet för honom förrän om morgonen, när det blev dager.
1Sa 25:37  Men om morgonen, när ruset hade gått av Nabal, omtalade hans hustru för honom vad som hade hänt. Då blev hans hjärta såsom dött i hans bröst, och han blev såsom en sten.
1Sa 25:38  Och vid pass tio dagar därefter slog HERREN Nabal, så att han dog.
1Sa 25:39  När David hörde att Nabal var död, sade han: "Lovad vare HERREN, som på Nabal har hämnats den smälek han tillfogade mig, och som har bevarat sin tjänare från att göra vad ont var, under det att HERREN lät Nabals ondska komma tillbaka över hans eget huvud!" Och David sände åstad och lät säga Abigail att han önskade få henne till sin hustru.
1Sa 25:40  När så Davids tjänare kommo till Abigail i Karmel, talade de till henne och sade: "David har sänt oss till dig för att få dig till hustru åt sig."
1Sa 25:41  Då stod hon upp och föll ned till jorden på sitt ansikte och sade: "Må din tjänarinna bliva en trälinna, som tvår min herres tjänares fötter."
1Sa 25:42  Därefter stod Abigail upp med hast och satte sig på sin åsna, likaledes de fem tärnor som utgjorde hennes följe. Och hon följde med dem som David hade sänt till henne och blev hans hustru.
1Sa 25:43  David hade ock tagit till hustru Ahinoam från Jisreel, så att dessa båda blevo hans hustrur.
1Sa 25:44  Men Saul hade givit sin dotter Mikal, Davids hustru, åt Palti, Lais' son, från Gallim.
1Sa 26:1  Och sifiterna kommo till Saul i Gibea och sade: "David håller sig nu gömd på Hakilahöjden, gent emot ödemarken."
1Sa 26:2  Då bröt Saul upp och drog ned till öknen Sif med tre tusen män utvalda ur Israel, för att söka efter David i öknen Sif.
1Sa 26:3  Och Saul lägrade sig på Hakilahöjden, som ligger gent emot ödemarken, vid vägen. Men David uppehöll sig då i öknen. Och när David förnam att Saul hade kommit efter honom in i öknen,
1Sa 26:4  sände han ut spejare och fick så full visshet om att Saul hade kommit.
1Sa 26:5  Då bröt David upp och begav sig till det ställe där Saul hade lägrat sig; och David såg platsen där Saul låg med sin härhövitsman Abner, Ners son. Saul låg nämligen i vagnborgen, och folket var lägrat runt
1Sa 26:6  Och David tog till orda och sade I till hetiten Ahimelek och till Abisai, Serujas son, Joabs broder: "Vem vill gå med mig ned till Saul i lägret?" Då svarade Abisai: "Jag vill gå med dig ditned."
1Sa 26:7  Så kommo då David och Abisai om natten till folket där, och sågo Saul ligga och sova i vagnborgen, med spjutet nedstött i jorden invid huvudgärden; och Abner och folket lågo runt omkring honom.
1Sa 26:8  Då sade Abisai till David: "Gud har i dag överlämnat din fiende i din hand; så låt mig nu få spetsa honom fast i jorden med spjutet; det skall ske genom en enda stöt, jag skall icke behöva giva honom mer än den."
1Sa 26:9  Men David svarade Abisai: "Du får icke förgöra honom; ty vem har uträckt sin hand mot HERRENS smorde och förblivit ostraffad?"
1Sa 26:10  Och David sade ytterligare: "Så sant HERREN lever, HERREN må själv slå honom, eller ock må hans dödsdag komma i vanlig ordning, eller må han draga ut i strid och så få sin bane;
1Sa 26:11  men HERREN låte det vara fjärran ifrån mig att jag skulle uträcka min hand mot HERRENS smorde. Tag nu likväl spjutet som står vid hans huvudgärd och vattenkruset; och låt oss sedan gå vår väg.
1Sa 26:12  Och David tog spjutet och vattenkruset från Sauls huvudgärd, och sedan gingo de sin väg. Men ingen såg eller märkte det eller ens vaknade, utan allasammans sovo; ty HERREN hade låtit en tung sömn falla över dem.
1Sa 26:13  Sedan, när David hade kommit över på andra sidan, ställde han sig på toppen av berget, långt ifrån så att avståndet var stort mellan dem.
1Sa 26:14  Och David ropade till folket och till Abner, Ners son, och sade: "Vill du icke svara, Abner?" Abner svarade och sade: "Vem är du som så ropar till konungen?"
1Sa 26:15  David sade till Abner: "Du är ju en man som icke har sin like i Israel. Varför har du då icke vakat över din herre, konungen? En av folket har ju kommit in för att förgöra konungen, din herre.
1Sa 26:16  Vad du har gjort är icke väl gjort. Så sant HERREN lever, I haden förtjänat att dö, därför att I icke haven vakat över eder herre, HERRENS smorde. Se nu efter: var äro konungens spjut och vattenkruset som stodo vid hans huvudgärd?"
1Sa 26:17  Då kände Saul igen Davids röst och sade: "Det är ju din röst, min son David." David svarade: "Ja, min herre konung."
1Sa 26:18  Och han sade ytterligare: "Varför jagar min herre så efter sin tjänare? Vad har jag då gjort, och vad för ont är i min hand?
1Sa 26:19  Må nu min herre konungen höra sin tjänares ord: Om det är HERREN som har uppeggat dig emot mig, så låt honom få känna lukten av en offergåva; men om det är människor, så vare de förbannade inför HERREN, därför att de nu hava drivit mig bort, så att jag icke får uppehålla mig i HERRENS arvedel. De säga ju: 'Gå bort och tjäna andra gudar.'
1Sa 26:20  Och må nu icke mitt blod falla på jorden fjärran ifrån HERRENS ansikte, då Israels konung har dragit ut för att söka efter en enda liten loppa, såsom man jagar rapphöns på bergen."
1Sa 26:21  Då sade Saul: "Jag har syndat. Kom tillbaka, min son David; ty jag vill icke mer göra dig något ont, eftersom mitt liv i dag har varit dyrt aktat i dina ögon. Se, ja har handlat i mycket stor dårskap och förvillelse."
1Sa 26:22  David svarade och sade: "Se här är spjutet, o konung; låt nu en av dina män komma hitöver och hämta det.
1Sa 26:23  Och HERREN skall vedergälla var och en för hans rättfärdighet och trofasthet. HERREN gav dig ju dag i min hand, men jag ville icke uträcka min hand mot HERRENS smorde.
1Sa 26:24  Och likasom ditt liv i dag har varit högt aktat i mina ögon, så så ock mitt liv vara högt aktat i HERRENS ögon, så att han räddar mig ur all nöd."
1Sa 26:25  Saul sade till David: "Välsignad vare du, min son David! Vad du företager dig, det skall du ock förmå utföra." Därefter gick David sin väg, och Saul vände tillbaka hem igen.
1Sa 27:1  Men David sade till sig själv: "En dag skall jag nu i alla fall omkomma genom Sauls hand. Ingen annan räddning finnes för mig än att fly undan till filistéernas land; då måste Saul avstå ifrån att vidare söka efter mig över hela Israels område, och så undkommer jag hans hand."
1Sa 27:2  Och David bröt upp och drog med sina sex hundra man över till Akis, Maoks son, konungen i Gat.
1Sa 27:3  Och David stannade hos Akis i Gat med sina män, var och en med sitt husfolk, David med sina båda hustrur, Ahinoam från Jisreel och Abigail, karmeliten Nabals hustru.
1Sa 27:4  Och när det blev berättat för Saul att David hade flytt till Gat, sökte han icke vidare efter honom.
1Sa 27:5  Men David sade till Akis: "Om jag har funnit nåd för dina ögon, så låt mig få min bostad i någon av landsortsstäderna, så att jag får vistas där. Varför skulle din tjänare bo i huvudstaden hos dig?"
1Sa 27:6  Då gav Akis honom samma dag Siklag. Därför hör Siklag ännu i dag under Juda konungar.
1Sa 27:7  Den tid David bodde i filistéernas land var sammanräknat ett år och fyra månader.
1Sa 27:8  Men David drog upp med sina män, och de företogo plundringståg i gesuréernas, girsiternas och amalekiternas land. Ty dessa stammar bodde sedan gammalt där i landet, fram emot Sur och ända intill Egyptens land.
1Sa 27:9  Och så ofta David härjade i landet, lät han varken män eller kvinnor bliva vid liv; men får och fäkreatur och åsnor och kameler och kläder tog han med sig och vände så tillbaka och kom till Akis.
1Sa 27:10  När då Akis sade: "Haven I väl i dag företagit något plundringståg?", svarade David: "Ja, i den del av Sydlandet, som tillhör Juda", eller: "I den del av Sydlandet, som tillhör jerameeliterna", eller: "I den del av Sydlandet, som tillhör kainéerna."
1Sa 27:11  Men att David lät varken män eller kvinnor bliva vid liv och komma till Gat, det skedde därför att han tänkte: "De kunde eljest förråda oss och säga: 'Så och så har David gjort, så har han betett sig under hela den tid han har bott i filistéernas land.'"
1Sa 27:12  Därför trodde Akis David och tänkte: "Han har nu gjort sig förhatlig för sitt folk Israel och kommer att bliva min tjänare för alltid.
1Sa 28:1  Vid den tiden församlade filistéerna sina krigshärar för att strida mot Israel. Och Akis sade till David: "Du må veta att du med dina män nu måste draga ut med mig i härnad."
1Sa 28:2  David svarade Akis: "Välan, då skall du ock få märka vad din tjänare kan uträtta." Akis sade till David "Välan, jag sätter dig alltså till väktare över mitt huvud för beständigt."
1Sa 28:3  Samuel var nu död, och hela Israel hade hållit dödsklagan efter honom; och de hade begravit honom i hans stad, i Rama. Och Saul hade utdrivit andebesvärjare och spåmän ur landet.
1Sa 28:4  Så församlade sig nu filistéerna och kommo och lägrade sig vid Sunem. Då församlade ock Saul hela Israel, och de lägrade sig vid Gilboa.
1Sa 28:5  Men när Saul såg filistéernas läger, fruktade han och förskräcktes högeligen i sitt hjärta.
1Sa 28:6  Och Saul frågade HERREN, men HERREN svarade honom icke, varken genom drömmar eller genom urim eller genom profeter.
1Sa 28:7  Då sade Saul till sina tjänare: "Söken upp åt mig någon andebesvärjerska, så vill jag gå till henne och fråga henne." Hans tjänare svarade honom: "I En-Dor finnes en andebesvärjerska."
1Sa 28:8  Då gjorde Saul sig oigenkännlig och tog på sig andra kläder och gick åstad med två män; och de kommo till kvinnan om natten. Och han sade: "Spå åt mig genom anden, och mana upp åt mig den jag säger dig."
1Sa 28:9  Men kvinnan svarade honom: "Du vet ju själv vad Saul har gjort, huru han har utrotat andebesvärjare och spåmän ur landet. Varför lägger du då ut en snara för mitt liv och vill döda mig?"
1Sa 28:10  Då svor Saul henne en ed vid HERREN och sade: "Så sant HERREN lever, i denna sak skall intet tillräknas dig såsom missgärning."
1Sa 28:11  Kvinnan frågade: "Vem skall jag då mana upp åt dig?" Han svarade: "Mana upp Samuel åt mig."
1Sa 28:12  Men när kvinnan fick se Samuel, gav hon till ett högt rop. Och kvinnan sade till Saul: "Varför har du bedragit mig? Du är ju Saul."
1Sa 28:13  Konungen sade till henne: "Frukta icke. Vad är det då du ser?" Kvinnan svarade Saul: "Jag ser ett gudaväsen komma upp ur jorden."
1Sa 28:14  Han frågade henne: "Huru ser han ut?" Hon svarade: "Det är en gammal man som kommer upp, höljd i en kåpa." Då förstod Saul att det var Samuel, och böjde sig ned med ansiktet mot jorden och bugade sig.
1Sa 28:15  Och Samuel sade till Saul: "Varför har du stört min ro och manat mig upp?" Saul svarade: "Jag är i stor nöd: filistéerna hava begynt krig mot mig, och Gud har vikit ifrån mig och svarar mig icke mer, varken genom profeter eller genom drömmar. Därför har jag kallat dig upp, på det att du må låta mig veta vad jag skall göra."
1Sa 28:16  Men Samuel svarade: "Varför frågar du mig, då nu HERREN har vikit ifrån dig och blivit din fiende?
1Sa 28:17  HERREN har efter sitt behag gjort vad han hade sagt genom mig: HERREN har ryckt riket ur din hand och givit det åt en annan, åt David.
1Sa 28:18  Eftersom du icke hörde HERRENS röst och icke lät Amalek känna hans vredes glöd, därför har HERREN nu gjort dig detta.
1Sa 28:19  HERREN skall giva både dig och Israel i filistéernas hand, och i morgon skall du med dina söner vara hos mig; ja, också Israels läger skall HERREN giva i filistéernas hand."
1Sa 28:20  Då föll Saul strax raklång till jorden; så förfärad blev han över Samuels ord. Också voro hans krafter uttömda, ty på ett helt dygn hade han ingenting ätit.
1Sa 28:21  Men kvinnan gick fram till Saul, och när hon såg huru högeligen förskräckt han var, sade hon till honom: "Se, din tjänarinna lyssnade till din begäran; Jag tog min själ i min hand och hörsammade den önskan du uttalade till mig.
1Sa 28:22  Så lyssna nu också du till dina tjänarinnas ord och låt mig sätta fram litet mat för dig, och ät, så att du hämtar krafter, innan du går dina färde."
1Sa 28:23  Men han vägrade och sade: "Jag vill icke äta." Då bådo honom hans tjänare jämte kvinnan så enträget, att han lyssnade till deras ord; han stod upp från jorden och satte sig på vilobädden.
1Sa 28:24  Och kvinnan hade en gödd kalv i huset; den slaktade hon nu i hast. Därpå tog hon mjöl och knådade det och bakade därav osyrat bröd.
1Sa 28:25  Sedan satte hon fram det för Saul: och hans tjänare, och de åto. Därefter stodo de upp och gingo samma natt sina färde.
1Sa 29:1  Filistéerna församlade nu alla sina härar i Afek, medan israeliterna voro lägrade vid källan i Jisreel.
1Sa 29:2  Då nu filistéernas hövdingar tågade fram med avdelningar på hundra och tusen, och David och hans män därvid tågade sist fram, tillika med Akis,
1Sa 29:3  sade filistéernas furstar: "Vad hava dessa hebréer här att göra?" Men Akis svarade filistéernas furstar: "Denne David är ju Sauls, Israels konungs, tjänare, som nu har varit hos mig över år och dag, och jag har icke funnit något ont hos honom, från den dag han gick över till mig ända till denna dag.
1Sa 29:4  Då blevo filistéernas furstar förtörnade på honom; och filistéernas furstar sade till honom: "Låt mannen vända om och gå tillbaka till den ort du har anvisat honom; han får icke draga med med oss till strid, för att han icke under striden må bliva vår motståndare. Ty varigenom skulle han väl bättre kunna göra sig behaglig för sin herre än genom dessa mäns huvuden?
1Sa 29:5  Han är ju den David till vilkens ära man sjunger så under dansen: 'Saul har slagit sina tusen, men David sina tio tusen.'"
1Sa 29:6  Då kallade Akis David till sig och sade till honom: "Så sant HERREN lever, du är en redlig man, och att du går ut och in här hos mig i lägret är mig välbehagligt, ty jag har icke funnit något ont hos dig, från den dag du kom till mig ända till denna dag; men för hövdingarna är du icke välbehaglig.
1Sa 29:7  Så vänd nu tillbaka och gå i frid, för att du icke må göra något som misshagar filistéernas hövdingar."
1Sa 29:8  David sade till Akis: "Vad har jag då gjort, och vad har du funnit hos din tjänare, från den dag jag kom i din tjänst ända till denna dag, eftersom jag icke får gå åstad och strida mot min herre konungens fiender?"
1Sa 29:9  Akis svarade och sade till David: "Jag vet bäst att du är mig välbehaglig såsom en Guds ängel; men filistéernas furstar säga: 'Han får icke draga upp med oss i striden.'
1Sa 29:10  Så stå nu upp bittida i morgon, jämte din herres tjänare som hava kommit hit med dig; och när morgon haven stått bittida upp, mån I draga edra färde, så snart det har blivit dager."
1Sa 29:11  Då stod David bittida upp med sina män för att om morgonen drag tillbaka till filistéernas land. Men filistéerna drogo upp till Jisreel.
1Sa 30:1  När David med sina män på tredje dagen kom till Siklag, hade amalekiterna infallit i Sydlandet och i Siklag; och de hade intagit Siklag och bränt upp det i eld.
1Sa 30:2  Och kvinnorna som voro därinne, både små och stora, hade de fört bort såsom fångar, utan att döda någon; de hade allenast fört bort dem och gått sin väg.
1Sa 30:3  När nu David med sina män kom till staden och fick se att den var uppbränd i eld, och att deras hustrur jämte deras söner och döttrar voro bortförda såsom fångar,
1Sa 30:4  brast han ut i gråt, så ock hans folk; och de gräto, till dess att de icke förmådde gråta mer.
1Sa 30:5  Davids båda hustrur, Ahinoam från Jisreel och Abigail, karmeliten Nabals hustru, voro också fångna.
1Sa 30:6  Och David kom i stor nöd, ty folket tänkte stena honom; så förbittrat var allt folket, var och en för sina söners och döttrars skull. Men David hämtade styrka hos HERREN, sin Gud.
1Sa 30:7  Och David sade till prästen Ebjatar, Ahimeleks son: "Bär hit till mig efoden." Då bar Ebjatar fram efoden till David.
1Sa 30:8  David frågade nu HERREN: "Skall jag sätta efter denna rövarskara? Kan jag då hinna upp den?" Han svarade honom: "Sätt efter dem; ty du skall förvisso hinna upp dem och skaffa räddning."
1Sa 30:9  Då begav sig David åstad med sina sex hundra man, och de kommo till bäcken Besor; där stannade de som nödgades bliva efter.
1Sa 30:10  Men David fortsatte förföljelsen med fyra hundra man; ty de som sade blivit för trötta, och som därför stannade, utan att gå över bäcken Besor, utgjorde två hundra man.
1Sa 30:11  Sedan träffade de på fältet en egyptisk man; honom togo de med sig till David. Och när de hade givit honom bröd att äta och vatten att dricka
1Sa 30:12  och när de ytterligare hade givit honom ett stycke fikonkaka och två russinkakor att äta, kom livskraften tillbaka i honom igen. På tre dygn hade han nämligen varken ätit eller druckit.
1Sa 30:13  Och David frågade honom: "Vem tillhör du, och varifrån är du?" Han svarade: "Jag är en egyptisk yngling, tjänare åt en amalekitisk man; men min herre övergav mig för tre dagar sedan, därför att jag blev sjuk.
1Sa 30:14  Vi hade nämligen infallit i den del av Sydlandet, som tillhör keretéerna, och i det område som tillhör Juda, och i den del av Sydlandet, som tillhör Kaleb, och vi hade bränt upp Siklag i eld."
1Sa 30:15  David sade till honom: "Vill du föra mig ned till den rövarskaran?" Han svarade: "Lova mig med ed vid Gud att du icke dödar mig eller utlämnar mig åt min herre, så vill jag föra dig ned till den rövarskaran."
1Sa 30:16  Så förde han honom ditned, och de lågo då kringspridda överallt på marken och åto och drucko och förlustade sig med allt det stora byte som de hade tagit ur filistéernas land och ur Juda land.
1Sa 30:17  Och ända från skymningen intill nästa dags afton höll David på med att nedgöra dem; och ingen enda av dem kom undan, utom fyra hundra tjänare som satte sig upp på kamelerna och flydde.
1Sa 30:18  Och David räddade allt vad amalekiterna hade tagit; sina båda hustrur räddade David också.
1Sa 30:19  Ingen saknades, varken liten eller stor, ingens son och ingens dotter, ej heller något av bytet eller något av det som de hade tagit med sig; David förde alltsammans tillbaka.
1Sa 30:20  David tog ock alla får och fäkreatur, och man drev dessa framför den övriga boskapen och ropade: "Detta är Davids byte."
1Sa 30:21  Och när David kom tillbaka till de två hundra man som hade varit för trötta att följa honom, och som därför hade fått stanna kvar vid bäcken Besor, gingo dessa åstad för att möta David och det folk som han hade med sig; då gick David fram till folket och hälsade dem.
1Sa 30:22  Men allahanda onda och illasinnade män, bland dem som hade följt med David, togo till orda och sade: "Eftersom dessa icke följde med oss, skola vi icke giva dem något av bytet som vi hava räddat; var och en av dem må allenast taga sin hustru och sina barn med sig och gå hem."
1Sa 30:23  Men David svarade: "Så skolen I icke göra, mina bröder, med det som HERREN har givit oss, då han bevarade oss och gav i vår hand denna rövarskara, som kom över oss.
1Sa 30:24  Och vem skulle för övrigt härutinnan vilja lyssna till eder? Nej, sådan deras lott är, som draga med i striden, sådan skall deras lott vara, som stanna vid trossen; de skola dela jämnt med varandra."
1Sa 30:25  Och därvid blev det, från den dagen och allt framgent; ty han gjorde detta till lag och rätt i Israel, såsom det är ännu i dag.
1Sa 30:26  När sedan David kom till Siklag, sände han en del av bytet till de äldste i Juda, sina vänner, i det han lät säga: "Detta är en skänk till eder av bytet från HERRENS fiender."
1Sa 30:27  Han sände till de äldste i Betel, de äldste i Ramot i Sydlandet och de äldste i Jattir;
1Sa 30:28  till de äldste i Aroer, de äldste i Sifamot och de äldste i Estemoa;
1Sa 30:29  till de äldste i Rakal, de äldste i jerameeliternas städer och de äldste i kainéernas städer;
1Sa 30:30  till de äldste i Horma, de äldste i Bor-Asan och de äldste i Atak;
1Sa 30:31  till de äldste i Hebron och till alla de orter där David hade vandrat omkring med sina män.
1Sa 31:1  Och filistéerna stridde mot Israel; och Israels män flydde för filistéerna och föllo slagna på berget Gilboa.
1Sa 31:2  Och filistéerna ansatte ivrigt Saul och hans söner. Och filistéerna dödade Jonatan, Abinadab och Malki-Sua, Sauls söner.
1Sa 31:3  När då Saul själv blev häftigt anfallen och bågskyttarna kommo över honom, greps han av stor förskräckelse för skyttarna.
1Sa 31:4  Och Saul sade till sin vapendragare: "Drag ut ditt svärd och genomborra mig därmed, så att icke dessa oomskurna komma och genomborra mig och hantera mig skändligt." Men hans vapendragare ville det icke, ty han fruktade storligen. Då tog Saul själv svärdet och störtade sig därpå.
1Sa 31:5  Men när vapendragaren såg att Saul var död, störtade han sig ock på sitt svärd och följde honom i döden.
1Sa 31:6  Så dogo då med varandra på den dagen Saul och hans tre söner och hans vapendragare, och därjämte alla hans män.
1Sa 31:7  Och när israeliterna på andra sidan dalen och på andra sidan Jordan förnummo att Israels män hade flytt, och att Saul och hans söner voro döda, övergåvo de städerna och flydde; sedan kommo filistéerna och bosatte sig i dem.
1Sa 31:8  Dagen därefter kommo filistéerna för att plundra de slagna och funno då Saul och hans tre söner, där de lågo fallna på berget Gilboa.
1Sa 31:9  Då höggo de av hans huvud och drogo av honom hans vapen och sände dem omkring i filistéernas land och läto förkunna det glada budskapet i sitt avgudahus och bland folket.
1Sa 31:10  Och de lade hans vapen i Astartetemplet, men hans kropp hängde de upp på Bet-Sans mur.
1Sa 31:11  Men när invånarna i Jabes i Gilead hörde vad filistéerna hade gjort med Saul,
1Sa 31:12  stodo de upp, alla stridbara män, och gingo hela natten och togo Sauls och hans söners kroppar ned från Bet-Sans mur, och begåvo sig därefter till Jabes och förbrände dem där.
1Sa 31:13  Sedan togo de deras ben och begrovo dem under tamarisken i Jabes och fastade så i sju dagar.


\end{document}