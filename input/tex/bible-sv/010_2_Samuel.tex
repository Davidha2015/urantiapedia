\begin{document}

\title{2 Samuelsboken}


\chapter{1}

\par 1 Efter Sauls död, när David hade kommit tillbaka från segern över Amalek, och när David sedan i två dagar hade uppehållit sig i Siklag,
\par 2 då hände sig på tredje dagen att en man kom från Sauls läger, med sönderrivna kläder och med jord på sitt huvud. Och när han kom in till David, föll han ned till jorden och bugade sig.
\par 3 David frågade honom: "Varifrån kommer du?" Han svarade honom: "Jag kommer såsom flykting ifrån Israels läger."
\par 4 Då sade David till honom: "Huru har det gått? Säg mig det." Han svarade: "Folket har flytt ur striden, många av folket hava också fallit och dött; Saul och hans son Jonatan äro ock döda."
\par 5 David frågade den unge mannen som berättade detta för honom: "Huru vet du att Saul och hans son Jonatan äro döda?"
\par 6 Den unge mannen som hade framfört underrättelsen till honom svarade: "Jag kom av en händelse upp på berget Gilboa, och där fick jag se Saul stödja sig mot sitt spjut, under det att vagnar och ryttare ansatte honom.
\par 7 När han då vände sig om och fick se mig, ropade han på mig, och jag svarade: 'Här är jag.'
\par 8 Då frågade han mig vem jag var, och jag svarade honom att jag var en amalekit.
\par 9 Sedan sade han till mig: 'Träd fram hit till mig och giv mig dödsstöten, ty jag är gripen av dödens vanmakt, om ock livet ännu alltjämt är kvar i mig.'
\par 10 Då trädde jag fram till honom och dödade honom, ty jag visste ju att han icke skulle kunna överleva sitt fall. Och jag tog diademet som satt på hans huvud, och ett armband som satt på hans arm, och jag bar nu detta hit till min herre."
\par 11 Då fattade David i sina kläder och rev sönder dem; så gjorde ock alla de män som voro där med honom.
\par 12 Och de höllo dödsklagan och gräto och fastade ända till aftonen för Sauls och hans son Jonatans skull, och för HERRENS folks och för Israels hus' skull, därför att de hade fallit för svärd.
\par 13 Och David frågade den unge mannen som hade framfört underrättelsen till honom: "Varifrån är du?" Han svarade: "Jag är son till en amalekit som lever här såsom främling."
\par 14 David sade till honom: "Kände du då ingen fruktan för att uträcka din hand till att förgöra HERRENS smorde?"
\par 15 Och David kallade på en av sina män och sade: "Kom hit och stöt ned honom." Och han slog honom till döds.
\par 16 Och David sade till honom: "Ditt blod komme över ditt huvud, ty din egen mun har vittnat mot dig, i det att du sade: 'Jag har dödat HERRENS smorde.'"
\par 17 Och David sjöng följande klagosång över Saul och hans son Jonatan,
\par 18 och han befallde att man skulle lära Juda barn "Bågsången"; den är upptecknad i "Den redliges bok":
\par 19 "Din härlighet, Israel, ligger slagen på dina höjder. Huru hava icke hjältarna fallit!
\par 20 Förkunnen det icke i Gat, bebåden det ej på Askelons gator, för att filistéernas döttrar icke må glädja sig, de oomskurnas döttrar ej fröjda sig.
\par 21 I Gilboa berg, på eder må ej falla dagg eller regn, ej ses offergärdsskördar. Ty hjältarnas sköld blev där till smälek, Sauls sköld, ej sedan smord med olja.
\par 22 Från slagnas blod, från hjältars hull vek Jonatans båge icke tillbaka, vände Sauls svärd ej omättat åter.
\par 23 Saul och Jonatan, så kära och ljuvliga för varandra i livet, de blevo ej heller skilda i döden, de två, som voro snabbare än örnar, starka mer än lejon.
\par 24 Israels döttrar, gråten över Saul, över honom som klädde eder i scharlakan och praktskrud och prydde edra kläder med gyllene smycken.
\par 25 Huru hava icke hjältarna fallit i striden! Jonatan ligger slagen på dina höjder.
\par 26 Jag sörjer över dig, du min broder Jonatan; mycket ljuvlig var du mig. Dyrbar var mig din kärlek, mer än kvinnokärlek.
\par 27 Huru hava icke hjältarna fallit, de båda stridssvärden förgåtts!"

\chapter{2}

\par 1 Därefter frågade David HERREN: "Skall jag draga upp till någon av Juda städer?" HERREN svarade honom: "Drag upp." Då frågade David: "Vart skall jag draga upp?", Han svarade: "Till Hebron."
\par 2 Så drog då David ditupp jämte sina båda hustrur, Ahinoam från Jisreel och Abigail, karmeliten Nabals hustru.
\par 3 David lät ock sina män draga ditupp, var och en med sitt husfolk; och de bosatte sig i Hebrons städer.
\par 4 Dit kommo nu Juda män och smorde David till konung över Juda hus. När man berättade för David att det var männen i Jabes i Gilead som hade begravit Saul,
\par 5 skickade David sändebud till männen i Jabes i Gilead och lät säga till dem: "Varen välsignade av HERREN, därför att I haven bevisat eder herre Saul den barmhärtighetstjänsten att begrava honom!
\par 6 Så må nu ock HERREN bevisa barmhärtighet och trofasthet mot eder. Själv vill jag också göra eder gott, därför att I haven gjort detta.
\par 7 Varen alltså nu vid gott mod och oförskräckta, fastän eder herre Saul är död; det är nu jag som av Juda hus har blivit smord till konung över dem."
\par 8 Men Abner, Ners son, Sauls härhövitsman, tog Sauls son Is-Boset och förde honom över till Mahanaim
\par 9 och gjorde honom till konung i Gilead och asuréernas land och Jisreel, så ock över Efraim, Benjamin och hela det övriga Israel.
\par 10 Sauls son Is-Boset var fyrtio år gammal, när han blev konung över Israel, och han regerade i två år. Allenast Juda hus höll sig till David.
\par 11 Den tid David var konung i Hebron över Juda hus utgjorde sammanräknat sju år och sex månader.
\par 12 Och Abner, Ners son, drog ut med Sauls son Is-Bosets folk ifrån Mahanaim till Gibeon.
\par 13 Joab, Serujas son, och Davids folk drogo också ut; och de mötte varandra vid Gibeons damm. Där stannade de på var sin sida om dammen.
\par 14 Och Abner sade till Joab: "Må vi låta några unga män stå upp och utföra en krigslek i vår åsyn." Joab svarade: "Må så ske."
\par 15 Då stodo de upp och gingo fram i lika antal: tolv för Benjamin och för Sauls son Is-Boset, och tolv av Davids folk.
\par 16 Och de fattade varandra i huvudet och stötte svärdet i sidan på varandra och föllo så allasammans därför blev detta ställe kallat Helkat-Hassurim vid Gibeon.
\par 17 Sedan begynte en mycket hård strid på den dagen; men Abner och Israels män blevo slagna av Davids folk.
\par 18 Nu funnos där tre söner till Seruja: Joab, Abisai och Asael. Och Asael var snabbfotad såsom en gasell på fältet.
\par 19 Och Asael förföljde Abner, utan att vika undan vare sig till höger eller till vänster från Abner.
\par 20 Då vände Abner sig om och sade: "Är det du, Asael?" Han svarade: "Ja."
\par 21 Då sade Abner till honom: "Vänd dig åt annat håll, åt höger eller åt vänster. Angrip någon av de yngre och försök att taga hans rustning." Men Asael ville icke låta honom vara.
\par 22 Då sade Abner ännu en gång till Asael: "Låt mig vara. Du vill väl icke att jag skall slå dig till jorden? Huru skulle jag sedan kunna se din broder Joab i ansiktet?"
\par 23 När han ändå icke ville låta honom vara, gav Abner honom med bakändan av sitt spjut en stöt i underlivet, så att spjutet gick ut baktill; och han föll ned där och dog på stället. Och var och en som kom till platsen där Asael hade fallit ned och dött stannade där.
\par 24 Och Joab och Abisai förföljde Abner. Men när solen hade gått ned och de hade kommit till Ammahöjden, som ligger gent emot Gia, åt Gibeons öken till,
\par 25 då samlade sig Benjamins barn tillhopa bakom Abner, så att de utgjorde en sluten skara, och intogo en ställning på toppen av en och samma höjd.
\par 26 Och Abner ropade till Joab och sade: "Skall då svärdet få oavlåtligen frossa? Förstår du icke att detta måste leda till ett bittert slut?" Huru länge tänker du dröja, innan du befaller ditt folk att upphöra med att förfölja sina bröder?"
\par 27 Joab svarade: "Så sant Gud lever: om du ingenting hade sagt, då hade folket först i morgon fått draga sig tillbaka och upphöra att förfölja sina bröder."
\par 28 Därefter lät Joab stöta i basunen; då stannade allt folket och förföljde icke mer Israel. Och sedan stridde de icke vidare.
\par 29 Men Abner och hans män tågade genom Hedmarken hela den natten; därefter gingo de över Jordan och tågade vidare hela förmiddagen och kommo så till Mahanaim.
\par 30 Joab åter samlade tillhopa allt folket, sedan han hade upphört att förfölja Abner; då fattades av Davids folk nitton man utom Asael.
\par 31 Davids folk hade däremot slagit till döds tre hundra sextio man av Benjamin och av Abners folk.
\par 32 Och de togo upp Asael och begrovo honom i hans faders grav i Bet-Lehem. Därefter tågade Joab och hans män hela natten och kommo i dagningen till Hebron.

\chapter{3}

\par 1 Kriget mellan Sauls hus och Davids hus blev långvarigt. Därunder blev David allt starkare och starkare, men Sauls hus allt svagare och svagare.
\par 2 I Hebron föddes söner åt David. hans förstfödde var Amnon, som han fick med Ahinoam från Jisreel.
\par 3 Hans andre son var Kilab, som han fick med Abigal, karmeliten Nabals hustru, och den tredje var Absalom, son till Maaka, som var dotter till Talmai, konungen i Gesur.
\par 4 Den fjärde var Adonia, Haggits son, och den femte var Sefatja, Abitals son.
\par 5 Den sjätte var Jitream, som David fick med sin hustru Egla. Dessa föddes åt David i Hebron.
\par 6 Så länge kriget varade mellan Sauls hus och Davids hus, bistod Abner kraftigt Sauls hus.
\par 7 Men Saul hade haft en bihustru vid namn Rispa, Ajas dotter; och Is-Boset sade till Abner: "Varför har du gått in till min faders bihustru?"
\par 8 För dessa hans ord blev Abner mycket vred och sade: "Är jag då ett hundhuvud från Juda land? Just då jag bevisar barmhärtighet mot din fader Sauls hus, mot hans bröder och hans vänner, och icke har låtit dig falla i Davids hand, just då tillvitar du mig att hava begått en missgärning med denna kvinna.
\par 9 Gud straffe Abner nu och framgent, om jag icke hädanefter handlar så mot David som HERREN med ed har lovat honom:
\par 10 jag vill göra så, att konungadömet tages ifrån Sauls hus, och att i stället Davids tron bliver upprest över både Israel och Juda, från Dan ända till Beer-Seba."
\par 11 Då tordes han icke säga ett ord mer åt Abner, av fruktan för honom.
\par 12 Men Abner skickade strax sändebud till David och lät säga: "Vem tillhör landet?", och lät vidare säga: "Slut förbund med mig, så skall jag bistå dig och göra så, att hela Israel går över till dig."
\par 13 Han svarade: "Gott! Jag vill sluta förbund med dig. Men en sak fordrar jag av dig, nämligen att du icke träder fram inför mitt ansikte utan att hit medföra Mikal, Sauls dotter, när du kommer för att träda fram inför mitt ansikte."
\par 14 Därefter skickade David sändebud till Is-Boset, Sauls son, och lät säga: "Giv mig åter min hustru Mikal, som jag förvärvade mig för ett hundra filistéers förhudar."
\par 15 Då sände Is-Boset åstad och lät taga henne ifrån hennes man, Paltiel, Lais' son.
\par 16 Men hennes man gick med henne och följde henne under beständig gråt ända till Bahurim. Här sade Abner till honom: "Vänd om och gå dina färde." Då vände han om.
\par 17 Och Abner hade underhandlat med de äldste i Israel och sagt: "Sedan lång tid tillbaka haven I sökt att få David till konung över eder.
\par 18 Fullborden nu edert uppsåt, ty så har HERREN sagt om David: Genom min tjänare Davids hand skall jag frälsa mitt folk Israel ifrån filistéernas hand och ifrån alla dess fienders hand."
\par 19 Likaledes talade Abner härom med benjaminiterna. Därefter gick Abner ock åstad for att tala med David i Hebron om allt vad Israel och hela Benjamins hus hade funnit lämpligt att svara.
\par 20 När då Abner, åtföljd av tjugu man, kom till David i Hebron, gjorde David ett gästabud för Abner och hans män.
\par 21 Och Abner sade till David: "Jag vill stå upp och gå åstad och församla hela Israel till min herre konungen, för att de må sluta förbund med dig, så att du bliver konung på vad villkor dig lyster." Sedan lät David Abner gå, och han drog bort i frid.
\par 22 Just då kommo Davids folk och Joab hem från ett strövtåg och förde med sig ett stort byte; men Abner var nu icke längre kvar hos David i Hebron, ty denne hade låtit honom gå, och han hade dragit bort i frid.
\par 23 Men när Joab och hela hans här kom hem, berättade man för honom och sade: "Abner, Ners son, kom till konungen, och denne lät honom gå, och han drog bort i frid."
\par 24 Då gick Joab in till konungen och sade: "Vad har du gjort! Då nu Abner hade kommit till dig, varför lät du då honom gå, så att han fritt kunde draga sina färde?
\par 25 Du känner väl Abner, Ners son? Han kom hit för att bedraga dig. Han ville utforska ditt görande och låtande, och utforska allt vad du förehar.
\par 26 Sedan, när Joab hade gått ut från David, sände han bud efter Abner, och sändebuden förde denne tillbaka från Bor-Hassira. Men David visste intet därom.
\par 27 När Abner så hade kommit tillbaka till Hebron, förde Joab honom avsides till mitten av porten, under förevändning att tala enskilt med honom; där sårade han honom till döds med en stöt i underlivet - detta for att hämnas sin broder Asaels blod.
\par 28 När David sedan fick höra detta, sade han: "Jag och mitt konungadöme äro oskyldiga inför HERREN evinnerligen till Abners, Ners sons blod.
\par 29 Må det komma över Joabs huvud och över hela hans faders hus; och må i Joabs hus aldrig fattas män som hava flytning, eller som äro spetälska, eller som stödja sig på krycka, eller som falla för svärd, eller som lida brist på bröd."
\par 30 Så hade nu Joab och hans broder Abisai dräpt Abner, därför att denne hade dödat deras broder Asael vid Gibeon, under striden.
\par 31 Och David sade till Joab och allt folket som var med honom: "Riven sönder edra kläder och höljen eder i sorgdräkt och hållen dödsklagan efter Abner." Och konung David gick själv bakom båren.
\par 32 Så begrovo; de Abner i Hebron; och konungen brast ut i gråt vid Abners grav, och allt folket grät.
\par 33 Och konungen sjöng följande klagosång över Abner: "Måste då Abner dö en gudlös dåres död?
\par 34 Dina händer voro ju ej bundna, dina fötter ej slagna i fjättrar. Du föll såsom man faller för ogärningsmän." Då begrät allt folket honom ännu mer."
\par 35 Och allt folket kom för att förmå David att äta något under dagens lopp; men David betygade med ed och sade: "Gud straffe mig nu och framgent, om jag smakar bröd eller något annat, förrän solen har gått ned."
\par 36 När folket hörde detta, behagade det dem alla väl, likasom allt annat som konungen gjorde behagade allt folket väl.
\par 37 Och allt folket och hela Israel insåg då att konungen ingen del hade haft i att Abner, Ners son, hade blivit dödad.
\par 38 Och konungen sade till sina tjänare: "I veten nogsamt att en furste och en stor man i dag har fallit i Israel.
\par 39 Men jag är ännu svag, fastän jag är smord till konung, och dessa män, Serujas söner, äro starkare än jag. HERREN vedergälle den som ont gör, efter hans ondska."

\chapter{4}

\par 1 Då nu Sauls son hörde att Abner var död i Hebron, sjönk allt hans mod, och hela Israel var förskräckt.
\par 2 Men Sauls son hade till hövitsmän för sina strövskaror två män, av vilka den ene hette Baana och den andre Rekab, söner till Rimmon från Beerot, av Benjamins barn. Ty också Beerot räknas till Benjamin;
\par 3 men beerotiterna flydde till Gittaim och bodde där sedan såsom främlingar, vilket de göra ännu i dag.
\par 4 (Också Jonatan, Sauls son, hade lämnat efter sig en son, som nu var ofärdig i fötterna. Han var nämligen fem år gammal, när budskapet om Saul och Jonatan kom från Jisreel, och då tog hans sköterska honom och flydde; men under hennes bråda flykt föll han omkull och blev därefter halt; och han hette Mefiboset.)
\par 5 Nu gingo beerotiten Rimmons söner Rekab och Baana åstad och kommo till Is-Bosets hus, då det var som hetast på dagen, medan han låg i sin middagssömn.
\par 6 När de så, under förevändning att hämta vete, hade kommit in i det inre av huset, sårade de honom med en stöt i underlivet; därefter flydde Rekab och hans broder Baana undan.
\par 7 De kommo alltså in i huset, när han låg på sin vilobädd i sovkammaren, och sårade honom till döds och höggo huvudet av honom; därpå togo de hans huvud och färdades genom Hedmarken hela natten.
\par 8 Och de förde så Is-Bosets huvud till David i Hebron och sade till konungen: "Se här är Is-Bosets, Sauls sons, din fiendes, huvud, hans som stod efter ditt liv. HERREN har i dag givit min herre konungen hämnd på Saul och hans efterkommande."
\par 9 Då svarade David Rekab och hans broder Baana, beerotiten Rimmons söner, och sade till dem: "Så sant HERREN lever, han som har förlossat mig från all nöd:
\par 10 den som förkunnade för mig och sade: 'Nu är Saul död', och som menade sig vara en glädjebudbärare, honom lät jag gripa och dräpa i Siklag, honom som jag eljest skulle hava givit budbärarlön;
\par 11 huru mycket mer skall jag icke då nu, när ogudaktiga män hava dräpt en oskyldig man i hans eget hus, på hans säng, utkräva hans blod av eder hand och utrota eder från jorden!"
\par 12 På Davids befallning dräpte hans män dem sedan och höggo av deras händer och fötter och hängde upp dem vid dammen i Hebron. Men Is-Bosets huvud togo de, och de begrovo det i Abners grav i Hebron.

\chapter{5}

\par 1 Sedan; kommo alla Israels stammar till David i Hebron och sade så: "Vi äro ju ditt kött och ben.
\par 2 Redan för länge sedan, då Saul ännu var konung över oss, var det du som var ledare och anförare för Israel. Och till dig har HERREN sagt: Du skall vara en herde för mitt folk Israel, ja, du skall vara en furste över Israel."
\par 3 När så alla de äldste i Israel kommo till konungen i Hebron, slöt konung David ett förbund med dem där i Hebron, inför HERREN; och sedan smorde de David till konung över Israel.
\par 4 David var trettio år gammal, när han blev konung, och han regerade i fyrtio år.
\par 5 I Hebron regerade han över Juda i sju år och sex månader, och i Jerusalem regerade han i trettiotre år över hela Israel och Juda.
\par 6 Och konungen drog med sina män till Jerusalem, mot jebuséerna, som bodde där i landet. De sade då till David: "Hitin kommer du icke; blinda och halta skola driva dig bort, de mena att David icke skall komma hitin."
\par 7 Men David intog likväl Sions borg, det är Davids stad.
\par 8 Och David sade på den dagen: "Vemhelst som slår ihjäl en jebusé och tränger fram till vattenledningen, han slår ihjäl just dessa halta och blinda, som David hatar." Därför plägar man säga: "Ingen blind och halt må komma in i huset."
\par 9 Sedan tog David sin boning på borgen och kallade den Davids stad. Där uppförde David byggnader runt omkring, från Millo och vidare inåt.
\par 10 Och David blev allt mäktigare och mäktigare, och HERREN, härskarornas Gud, var med honom.
\par 11 Och Hiram, konungen i Tyrus, skickade sändebud till David med cederträ, därjämte ock timmermän och stenhuggare; och de byggde ett hus åt David.
\par 12 Och David märkte att HERREN hade befäst honom såsom konung över Israel, och att han hade upphöjt hans konungadöme, för sitt folk Israels skull.
\par 13 Och David tog sig ännu flera bihustrur och hustrur från Jerusalem, sedan han hade kommit från Hebron; och åt David föddes ännu flera söner och döttrar.
\par 14 Dessa äro namnen på de söner som föddes åt honom i Jerusalem: Sammua, Sobab, Natan, Salomo,
\par 15 Jibhar, Elisua, Nefeg, Jafia,
\par 16 Elisama, Eljada och Elifelet.
\par 17 Men när filistéerna hörde att David hade blivit smord till konung över Israel, drogo de allasammans upp för att fånga David. När David hörde detta, drog han ned till borgen.
\par 18 Och sedan filistéerna hade kommit fram, spridde de sig i Refaimsdalen.
\par 19 Då frågade David HERREN: "Skall jag draga upp mot filistéerna? Vill du då giva dem i min hand?" HERREN svarade David: "Drag upp; ty jag skall giva filistéerna i din hand.
\par 20 Och David kom till Baal-Perasim, och där slog David dem. Då sade han: "HERREN har brutit ned mina fiender inför mig, likasom en vattenflod bryter ned." Därav fick det stället namnet Baal-Perasim.
\par 21 De lämnade där efter sig sina avgudabilder, och David och hans män togo dessa med sig.
\par 22 Men filistéerna drogo upp ännu en gång och spridde sig i Refaimsdalen.
\par 23 När David då frågade HERREN, svarade han: "Du skall icke draga ditupp; du må kringgå dem bakifrån, så att du kommer över dem från det håll där bakaträden stå.
\par 24 Så snart du sedan hör ljudet av steg i bakaträdens toppar, skynda då raskt fram, ty då har HERREN dragit ut framför dig till att slå filistéernas här."
\par 25 David gjorde såsom HERREN hade bjudit honom; och han slog filistéerna och förföljde dem från Geba ända fram emot Geser.

\chapter{6}

\par 1 Åter församlade David allt utvalt manskap i Israel, trettio tusen man.
\par 2 Och David bröt upp och drog åstad med allt sitt folk ifrån Baale-Juda, för att därifrån föra upp Guds ark, som hade fått sitt namn efter HERREN Sebaot, honom som tronar på keruberna.
\par 3 Och de satte Guds ark på en ny vagn och förde den bort ifrån Abinadabs hus på höjden; och Ussa och Ajo, Abinadabs söner, körde den nya vagnen.
\par 4 Så förde de Guds ark bort ifrån Abinadabs hus på höjden, och följde själva med, och Ajo gick därvid framför arken.
\par 5 Och David och hela Israels hus fröjdade sig inför HERREN, med allahanda instrumenter av cypressträ, med harpor, psaltare, pukor, skallror och cymbaler.
\par 6 Men när de kommo till Nakonslogen, räckte Ussa ut sin hand mot Guds ark och fattade i den, ty oxarna snavade.
\par 7 Då upptändes HERRENS vrede mot Ussa, och Gud slog honom där för hans förseelse, så att han föll ned död där vid Guds ark.
\par 8 Men det gick David hårt till sinne att HERREN så hade brutit ned Ussa; och han kallade det ställe Peres-Ussa, såsom det heter ännu i dag.
\par 9 Och David betogs av sådan fruktan för HERREN på den dagen, att han sade: "Huru skulle jag töras låta HERRENS ark komma till mig?"
\par 10 Därför ville David icke låta flytta in HERRENS ark till sig i David stad, utan lät sätta in den i gatiten Obed-Edoms hus.
\par 11 Sedan blev HERRES ark kvar i gatiten Obed-Edoms hus i tre månader; men HERREN välsignade Obed-Edom och hela hans hus.
\par 12 När det nu blev berättat för konung David att HERREN hade välsignat Obed-Edoms hus och allt vad han hade, för Guds arks skull, då gick David åstad och hämtade Guds ark ur Obed-Edoms hus upp till Davids stad under jubel.
\par 13 Och när de som buro HERRENS ark hade gått sex steg framåt, offrade han en tjur och en gödkalv.
\par 14 Själv dansade David med all makt inför HERREN, och därvid var David iklädd en linne-efod.
\par 15 Så hämtade David och hela Israel HERRENS ark ditupp under jubel och basuners ljud.
\par 16 När då HERRENS ark kom in i Davids stad, blickade Mikal, Sauls dotter, ut genom fönstret, och när hon såg konung David hoppa och dansa inför HERREN fick hon förakt för honom i sitt hjärta.
\par 17 Sedan de hade fört HERRENS ark ditin, ställde de den på dess plats i tältet som David hade slagit upp åt den; och därefter offrade David brännoffer inför HERREN, så ock tackoffer.
\par 18 När David hade offrat brännoffret och tackoffret välsignade han folket i HERREN Sebaots namn.
\par 19 Och åt allt folket, åt var och en i hela hopen av israeliter, både man och kvinna, gav han en kaka bröd, ett stycke kött och en druvkaka. Sedan gick allt folket hem, var och en till sitt.
\par 20 Men när David kom tillbaka för att hälsa sitt husfolk, gick Mikal, Sauls dotter, ut emot honom och sade: "Huru härlig har icke Israels konung visat sig i dag, då han i dag har blottat sig för sina tjänares tjänstekvinnors ögon, såsom löst folk plägar göra!"
\par 21 Då sade David till Mikal: "Inför HERREN, som har utvalt mig framför din fader och hela hans hus, och som har förordnat mig till furste över HERRENS folk, över Israel - inför HERREN fröjdade jag mig.
\par 22 Dock kände jag mig rätteligen för ringa till detta, ja, jag var i mina ögon allt för låg därtill. Skulle jag då söka ära hos tjänstekvinnorna, om vilka du talade?"
\par 23 Och Mikal, Sauls dotter, fick inga barn, så länge hon levde.

\chapter{7}

\par 1 Då nu konungen satt i sitt hus, sedan HERREN hade låtit honom få ro runt omkring för alla hans fiender,
\par 2 sade han till profeten Natan: "Se, jag bor i ett hus av cederträ, under det att Guds ark bor i ett tält.
\par 3 Natan sade till konungen: "Välan, gör allt vad du har i sinnet; ty HERREN är med dig."
\par 4 Men om natten kom HERRENS ord till Natan; han sade:
\par 5 "Gå och säg till min tjänare David: Så säger HERREN: Skulle du bygga mig ett hus att bo i?
\par 6 Jag har ju icke bott i något hus, allt ifrån den dag då jag förde Israels barn upp ur Egypten ända till denna dag, utan jag har flyttat omkring i ett tält, i ett tabernakel.
\par 7 Har jag då någonsin, varhelst jag flyttade omkring med alla Israels barn, talat och sagt så till någon enda av Israels stammar, som jag har förordnat till herde för mitt folk Israel: 'Varför haven I icke byggt mig ett hus av cederträ?'
\par 8 Och nu skall du säga så till min tjänare David: Så säger HERREN Sebaot: Från betesmarken, där du följde fåren, har jag hämtat dig, för att du skulle bliva en furste över mitt folk Israel.
\par 9 Och jag har varit med dig på alla dina vägar och utrotat alla dina fiender för dig. Och jag vill göra dig ett namn, så stort som de störstes namn på jorden.
\par 10 Jag skall bereda en plats åt mitt folk Israel och plantera det, så att det får bo kvar där, utan att vidare bliva oroat. Orättfärdiga människor skola icke mer förtrycka det, såsom fordom skedde,
\par 11 och såsom det har varit allt ifrån den tid då jag förordnade domare över mitt folk Israel; och jag skall låta dig få ro för alla dina fiender. Så förkunnar nu HERREN för dig att HERREN skall uppbygga ett hus åt dig.
\par 12 När din tid är ute och du vilar hos dina fäder, skall jag efter dig upphöja den son som skall utgå ur ditt liv; och jag skall befästa hans konungadöme.
\par 13 Han skall bygga ett hus åt mitt namn, och jag skall befästa hans konungatron för evig tid.
\par 14 Jag skall vara hans fader, och han skall vara min son, så att jag visserligen, om han gör något illa, skall straffa honom med ris, såsom människor pläga tuktas, och med plågor, sådana som hemsöka människors barn;
\par 15 men min nåd skall icke vika ifrån honom, såsom jag lät den vika ifrån Saul, vilken jag lät vika undan för dig.
\par 16 Ditt hus och ditt konungadöme skola bliva beståndande inför dig till evig tid; ja, din tron skall vara befäst för evig tid."
\par 17 Alldeles i överensstämmelse med dessa ord och med denna syn talade nu Natan till David.
\par 18 Då gick konung David in och satte sig ned inför HERRENS ansikte och sade: "Vem är jag, Herre, HERRE, och vad är mitt hus, eftersom du har låtit mig komma härtill?
\par 19 Och detta har ändå synts dig vara för litet, Herre, HERRE; du har ock talat angående din tjänares hus om det som ligger långt fram i tiden. och härom har du talat på människosätt, Herre, HERRE!
\par 20 Vad skall nu David vidare tala till dig? Du känner ju din tjänare, Herre, HERRE.
\par 21 För ditt ords skull och efter ditt hjärta har du gjort allt detta stora och förkunnat det för din tjänare.
\par 22 Därför är du ock stor HERRE Gud, ty ingen är dig lik, och ingen Gud finnes utom dig, efter allt vad vi hava hört med våra öron.
\par 23 Och var finnes på jorden något enda folk likt ditt folk Israel, något folk som en Gud själv har gått åstad att förlossa åt sig till ett folk, för att så göra sig ett namn - ja, för att göra dessa stora ting med eder och dessa fruktansvärda gärningar med ditt land, inför ditt folk, det som du förlossade åt dig från Egypten, från hedningarna och deras gudar.
\par 24 Och du har berett åt dig ditt folk Israel, dig till ett folk för evig tid, och du, HERRE, har blivit deras Gud.
\par 25 Så uppfyll nu, HERRE Gud, för evig tid vad du har talat om din tjänare och om hans hus; gör såsom du har talat.
\par 26 Då skall ditt namn bliva stort till evig tid, så att man skall säga: 'HERREN Sebaot är Gud över Israel.' Och så skall din tjänare Davids hus bestå inför dig.
\par 27 Ty du, HERRE Sebaot, Israels Gud, har uppenbarat för din tjänare och sagt: 'Jag vill bygga dig ett hus.' Därför har din tjänare fått frimodighet att bedja till dig denna bön.
\par 28 Och nu, Herre, HERRE, du är Gud, och dina ord äro sanning; och du du har lovat din tjänare detta goda,
\par 29 så värdes nu välsigna din tjänares hus, så att det förbliver evinnerligen inför dig. Ja, du, Herre, HERRE, har lovat det, och genom din välsignelse skall din tjänares hus bliva välsignat evinnerligen."

\chapter{8}

\par 1 En tid härefter slog David filistéerna och kuvade dem. Därvid bemäktigade sig David huvudstaden och tog den ur filistéernas hand.
\par 2 Han slog ock moabiterna och mätte dem med snöre, i det att han lät dem lägga sig ned på jorden: med två snörlängder mätte han ut den del av dem, som skulle dödas, och med en full snörlängd den del som han låt leva. Så blevo moabiterna David underdåniga och förde till honom skänker.
\par 3 Likaledes slog David Hadadeser, Rehobs son, konungen i Soba, när denne hade dragit åstad för att utsträcka sitt välde till floden.
\par 4 Och David tog till fånga av han folk ett tusen sju hundra ryttare och tjugu tusen man fotfolk; och David lät avskära fotsenorna på alla vagnshästarna, utom på ett hundra hästar, som han skonade.
\par 5 När sedan araméerna från Damaskus kommo för att hjälpa Hadadeser, konungen i Soba, nedgjorde David tjugutvå tusen man av dem.
\par 6 Och David insatte fogdar bland araméerna i Damaskus, och araméerna blevo David underdåniga och förde till honom skänker. Så gav HERREN seger åt David, varhelst han drog fram.
\par 7 Och David tog de gyllene sköldar som Hadadesers tjänare hade burit och förde dem till Jerusalem.
\par 8 Och från Hadadesers städer Beta och Berotai tog konung David koppar i stor myckenhet.
\par 9 Då nu Toi, konungen i Hamat, hörde att David hade slagit Hadadesers hela här,
\par 10 sände han sin son Joram till konung David för att hälsa honom och lyckönska honom, därför att han hade givit sig i strid med Hadadeser och slagit honom; ty Hadadeser hade varit Tois fiende. Och han hade med sig kärl av silver, av guld och av koppar.
\par 11 Också dessa helgade konung David åt HERREN, likasom han hade gjort med det silver och guld han hade tagit från alla de folk som han hade underlagt sig:
\par 12 från araméerna, moabiterna, Ammons barn, filistéerna och amalekiterna, så ock med det byte han hade tagit från Hadadeser, Rehobs son, konungen i Soba.
\par 13 Och när David kom tillbaka från sin seger över araméerna, gjorde han sig ytterligare ett namn i Saltdalen, där han slog aderton tusen man.
\par 14 Och han insatte fogdar i Edom, i hela Edom insatte han fogdar; och alla edoméer blevo David underdåniga. Så gav HERREN seger åt David, varhelst han drog fram.
\par 15 David regerade nu över hela Israel; och David skipade lag och rätt åt allt sitt folk.
\par 16 Joab, Serujas son, hade befälet över krigshären, och Josafat Ahiluds son, var kansler.
\par 17 Sadok, Ahitubs son, och Ahimelek, Ebjatars son, voro präster, och Seraja var sekreterare.
\par 18 Benaja, Jojadas son, hade befälet över keretéerna och peletéerna; dessutom voro Davids söner präster.

\chapter{9}

\par 1 Och David sade: "Finnes ännu någon kvar av Sauls hus, mot vilken jag kan bevisa barmhärtighet för Jonatans skull?"
\par 2 Nu hade Sauls hus haft en tjänare vid namn Siba; honom hämtade man till David. Då sade konungen till honom: "Är du Siba?" Han svarade: "Ja, din tjänare."
\par 3 Konungen frågade: "Finnes ingen kvar av Sauls hus, mot vilken jag kan bevisa barmhärtighet, såsom Gud är barmhärtig?" Siba svarade konungen: "Ännu finnes kvar en son till Jonatan, en som är ofärdig i fötterna."
\par 4 Konungen frågade honom "Var är han?" Siba svarade konungen: "Han är nu i Makirs, Ammiels sons, hus i Lo-Debar."
\par 5 Då sände konung David och lät hämta honom från Makirs, Ammiels sons, hus i Lo-Debar.
\par 6 När så Mefiboset, Sauls son Jonatans son, kom in till David, föll han ned på sitt ansikte och bugade sig. Då sade David: "Mefiboset!" Han svarade: "Ja, din tjänare hör."
\par 7 David sade till honom: "Frukta icke, ty jag vill bevisa barmhärtighet mot dig för din fader Jonatans skull, och jag vill giva dig allt din faders Sauls jordagods tillbaka, och du skall äta vid mitt bord beständigt."
\par 8 Då bugade han sig och sade: "Vad är jag, din tjänare, eftersom du vänder dig till en sådan död hund som jag är?"
\par 9 Därefter tillkallade konungen Siba, Sauls tjänare, och sade till honom; "Allt som Saul och hela hans hus har ägt giver jag åt din herres son.
\par 10 Och du med dina söner och dina tjänare skall bruka jorden åt honom och inbärga skörden, för att din herres son må hava bröd att äta, dock skall Mefiboset, din herres son, beständigt äta vid mitt bord." Siba hade nämligen femton söner och tjugu tjänare.
\par 11 Då sade Siba till konungen: "Din tjänare skall i alla stycken göra såsom min herre konungen bjuder sin tjänare." "Ja", svarade han, "Mefiboset skall äta vid mitt bord, såsom vore han en av konungens söner."
\par 12 Mefiboset hade en liten son, som hette Mika. Och alla som bodde i Sibas hus blevo Mefibosets tjänare.
\par 13 Själv bodde Mefiboset i Jerusalem, eftersom han beständigt skulle äta vid konungens bord. Och han var halt på båda fötterna.

\chapter{10}

\par 1 En tid härefter dog Ammons barns konung, och hans son Hanun blev konung efter honom.
\par 2 Då sade David: "Jag vill bevisa Hanun, Nahas' son, vänskap, likasom hans fader bevisade mig vänskap." Och David sände några av sina tjänare för att trösta honom i hans sorg efter fadern. När så Davids tjänare kommo till Ammons barns land,
\par 3 sade Ammons barns furstar till sin herre Hanun: "Menar du att David därmed att han sänder tröstare till dig vill visa dig att han ärar din fader? Nej, för att undersöka staden, för att bespeja och sedan fördärva den har David sänt sina tjänare till dig."
\par 4 Då tog Hanun Davids tjänare och lät raka av dem halva skägget och skära av deras kläder mitt på, ända uppe vid sätet, och lät dem så gå.
\par 5 När man berättade detta för David, sände han bud emot dem; ty männen voro ju mycket vanärade. Och konungen lät säga: "Stannen i Jeriko, till dess edert skägg hinner växa ut, och kommen så tillbaka."
\par 6 Då nu Ammons barn insågo att de hade gjort sig förhatliga för David, sände de bort och lejde från Aram-Bet-Rehob och Aram-Soba tjugu tusen man fotfolk, av konungen i Maaka ett tusen man och av Tobs män tolv tusen.
\par 7 När David hörde detta, sände han åstad Joab med hela hären, de tappraste krigarna.
\par 8 Och Ammons barn drogo ut och ställde upp sig till strid framför stadsporten; men de från Aram-Soba och Rehob, ävensom Tobs män och maakatéerna, ställde upp sig för sig själva på fältet.
\par 9 Då Joab nu såg att han hade fiender både framför sig och bakom sig, gjorde han ett urval bland allt Israels utvalda manskap och ställde sedan upp sig mot araméerna.
\par 10 Men det övriga folket överlämnade han åt sin broder Absai, vilken med dem ställde upp sig mot Ammons barn.
\par 11 Och han sade: "Om araméerna bliva mig övermäktiga, så skall du komma mig till hjälp; och om Ammons barn bliva dig övermäktiga, så vill jag tåga till din hjälp.
\par 12 Var nu vid gott mod; ja, låt oss visa mod i striden för vårt folk och för vår Guds städer. Sedan må HERREN göra vad honom täckes."
\par 13 Därefter ryckte Joab fram med sitt folk till strid mot araméerna och de flydde för honom.
\par 14 Men när Ammons barn sågo att araméerna flydde, flydde också de för Abisai och begåvo sig in i staden. Då drog Joab bort ifrån Ammons barn och begav sig tillbaka till Jerusalem.
\par 15 Då alltså araméerna sågo att de hade blivit slagna av Israel, församlade de sig allasammans.
\par 16 Och Hadadeser sände bud att de araméer som bodde på andra sidan floden skulle rycka ut; dessa kommo då till Helam, anförda av Sobak, Hadadesers härhövitsman.
\par 17 När detta blev berättat för David, församlade han hela Israel och gick över Jordan och kom till Helam; och araméerna ställde upp sig i slagordning mot David och gåvo sig i strid med honom.
\par 18 Men araméerna flydde för Israel, och David dräpte av araméerna manskapet på sju hundra vagnar, så ock fyrtio tusen ryttare; deras härhövitsman Sobak slog han ock där till döds.
\par 19 Då alltså Hadadesers alla lydkonungar sågo att de hade blivit slagna av israeliterna, ingingo de fred med dem och blevo dem underdåniga. Efter detta fruktade araméerna för att vidare hjälpa Ammons barn.

\chapter{11}

\par 1 Följande år, vid den tid då konungarna plägade draga i fält, sände David åstad Joab och med honom sina tjänare och hela Israel; och de härjade Ammons barns land och belägrade Rabba, medan David stannade kvar i Jerusalem.
\par 2 Då hände sig en afton, när David hade stått upp från sitt läger och gick omkring på konungshusets tak, att han från taket fick se en kvinna som badade; och kvinnan var mycket fager att skåda.
\par 3 David sände då åstad och förfrågade sig om kvinnan, och man sade: "Det är Bat-Seba, Eliams dotter, hetiten Urias hustru."
\par 4 Då sände David några män med uppdrag att hämta henne, och hon kom till honom, och han låg hos henne, när hon hade helgat sig från sin orenhet. Sedan återvände hon hem.
\par 5 Men kvinnan blev havande; hon sände då åstad och lät underrätta David därom och säga: "Jag är havande."
\par 6 Då sände David till Joab detta bud: "Sänd till mig hetiten Uria." Så sände då Joab Uria till David.
\par 7 Och när Uria kom till David, frågade denne om det stod väl till med Joab och med folket, och huru kriget gick.
\par 8 Därefter sade David till Uria: "Gå nu ned till ditt hus och två dina fötter." När då Uria gick ut ur konungens hus, sändes en gåva från konungen efter honom.
\par 9 Men Uria lade sig till vila vid ingången till konungshuset, jämte hans herres alla andra tjänare, och gick icke ned till sitt eget hus.
\par 10 Detta berättade man för David och sade: "Uria har icke gått ned till sitt hus." Då sade David till Uria: "Du kommer ju från resan; varför har du då icke gått ned till ditt hus?"
\par 11 Uria svarade David: "Arken och Israel och Juda bo nu i lägerhyddor, och min herre Joab och min herres tjänare äro lägrade ute på marken: skulle jag då gå in i mitt hus för att äta och dricka och ligga hos min hustru? Så sant du lever, så sant din själ lever: jag vill icke göra så."
\par 12 Då sade David till Uria: "Stanna här också i dag, så vill jag i morgon sända dig åstad." Så stannade då Uria i Jerusalem den dagen och den följande.
\par 13 Och David inbjöd honom till sig och lät honom äta och dricka med sig och gjorde honom drucken. Men om aftonen gick han ut och lade sig på sitt läger tillsammans med sin herres tjänare, och gick icke ned till sitt hus.
\par 14 Följande morgon skrev David ett brev till Joab och sände det med Uria.
\par 15 I brevet skrev han så: "Ställen Uria längst fram, där striden är som häftigast, och dragen eder sedan tillbaka från honom, så att han bliver slagen till döds."
\par 16 Under belägringen av staden skickade då Joab Uria till den plats där han visste att de tappraste männen funnos.
\par 17 Och männen i staden gjorde ett utfall och gåvo sig i strid med Joab, och flera av folket, av Davids tjänare, föllo; också hetiten Uria dödades.
\par 18 Då sände Joab och lät berätta för David allt vad som hade hänt under striden.
\par 19 Och han bjöd budbäraren och sade: "När du har omtalat för konungen allt vad som har hänt under striden,
\par 20 då upptändes kanske konungens vrede, och han säger till dig: 'Varför gingen I under striden så nära intill staden? Vissten I icke att de skulle skjuta uppifrån muren?
\par 21 Vem var det som slog ihjäl Abimelek, Jerubbesets son? Var det icke en kvinna som kastade en kvarnsten ned på honom från muren, så att han dödades, där i Tebes? Varför gingen I då så nära intill muren?' Men då skall du säga: 'Din tjänare Uria, hetiten, är ock död.'"
\par 22 Budbäraren gick åstad och kom och berättade för David allt vad Joab hade sänt honom att säga;
\par 23 budbäraren sade till David: "Männen blevo oss övermäktiga och drogo ut mot oss på fältet, men vi slogo dem tillbaka ända till stadsporten.
\par 24 Då sköto skyttarna uppifrån muren på dina tjänare, så att flera av konungens tjänare dödades; din tjänare Uria, hetiten, är ock död."
\par 25 Då sade David till budbäraren: "Så skall du säga till Joab: 'Låt icke detta förtryta dig, ty svärdet förtär än den ene, än den andre; fortsatt med kraft stadens belägring och förstör den.' Och intala honom så mod."
\par 26 Då nu Urias hustru hörde att hennes man Uria var död, höll hon dödsklagan efter sin man.
\par 27 Och när sorgetiden var förbi, sände David och lät hämta henne hem till sig, och hon blev hans hustru; därefter födde hon honom en son. Men vad David hade gjort misshagade HERREN.

\chapter{12}

\par 1 Och HERREN sände Natan till David. När han kom in till honom, sade han till honom: "Två män bodde i samma stad; den ene var rik och den andre fattig.
\par 2 Den rike hade får och fäkreatur i stor myckenhet.
\par 3 Men den fattige hade icke mer än ett enda litet lamm, som han hade köpt; han uppfödde det, och det växte upp hos honom och hans söner, tillsammans med dem: det åt av hans brödstycke och drack ur hans bägare och låg i hans famn och var för honom såsom en dotter.
\par 4 Så kom en vägfarande till den rike mannen; då nändes han icke taga av sina får och fäkreatur för att tillreda åt den resande som hade kommit till honom, utan han tog den fattige mannens lamm och tillredde det åt mannen som hade kommit till honom."
\par 5 Då upptändes Davids vrede storligen mot den mannen, och han sade till Natan: "Så sant HERREN lever: dödens barn är den man som har gjort detta.
\par 6 Och lammet skall han ersätta fyradubbelt, därför att han gjorde sådant, och eftersom han var så obarmhärtig."
\par 7 Men Natan sade till David: "Du är den mannen. Så säger HERREN, Israels Gud: Jag har smort dig till konung över Israel, och jag har räddat dig ur Sauls hand.
\par 8 Jag har givit dig din herres hus och lagt din herres hustrur i din famn; ja jag har givit dig Israels hus och Juda. Och om detta skulle vara för litet, så vore jag villig att ytterligare giva dig både ett och annat.
\par 9 Varför har du då föraktat HERRENS ord och gjort vad ont är i hans ögon? Hetiten Uria har du låtit slå ihjäl med svärd, och hans hustru har du tagit till hustru åt dig själv; ja, honom har du dräpt med Ammons barns svärd.
\par 10 Så skall nu icke heller svärdet vika ifrån ditt hus till evig tid, därför att du har föraktat mig och tagit hetiten Urias hustru till hustru åt dig.
\par 11 Så säger HERREN: Se, jag skall låta olyckor komma över dig från ditt eget hus, och jag skall taga dina hustrur inför dina ögon och giva dem åt en annan, och han skall ligga hos dina hustrur mitt på ljusa dagen.
\par 12 Ty väl har du gjort sådant i hemlighet, men jag vill låta detta ske inför hela Israel, och det på ljusa dagen."
\par 13 Då sade David till Natan: "Jag har syndat mot HERREN." Natan sade till David: "Så har ock HERREN tillgivit dig din synd; du skall icke dö.
\par 14 Men eftersom du genom denna gärning har kommit HERRENS fiender att förakta honom, skall ock den son som har blivit född åt dig döden dö."
\par 15 Sedan gick Natan hem igen. Och HERREN slog barnet som Urias hustru hade fött åt David, han slog det, så att det blev dödssjukt.
\par 16 Då sökte David Gud för gossens skull; och David höll fasta, och när han kom hem, låg han på bara marken över natten.
\par 17 Då stodo de äldste i hans hus upp och gingo till honom, för att förmå honom att stiga upp från marken; men han ville icke, och han åt icke heller något med dem.
\par 18 Men på sjunde dagen dog barnet. Då fruktade Davids tjänare att om tala för honom att barnet hade dött, ty de tänkte: "När vi talade till honom, medan barnet ännu levde, ville han ju icke lyssna till våra ord. Huru skulle vi då kunna säga till honom att barnet har dött? Han kunde göra något ont."
\par 19 Men när David såg att hans tjänare viskade med varandra, förstod han att barnet hade dött. Då frågade David sina tjänare: "Har barnet dött?" De svarade: "Ja."
\par 20 Då stod David upp från marken och tvådde sig och smorde sig och bytte om kläder och gick in i HERRENS hus och tillbad. Och när han kom hem igen, begärde han att man skulle sätta fram mat åt honom, och han åt.
\par 21 Då sade hans tjänare till honom: "Varför gör du på detta sätt? Medan barnet levde, fastade du och grät för dess skull; men så snart barnet har dött, står du upp och äter!"
\par 22 Han svarade: så länge barnet ännu levde, fastade och grät jag, ty jag tänkte: 'Vem vet, kanhända bliver HERREN mig nådig och låter barnet få leva.'
\par 23 Men nu, när det har dött, varför skulle jag då fasta? Kan jag väl skaffa honom tillbaka igen? Jag går bort till honom, men han kommer icke tillbaka till mig."
\par 24 Och David tröstade sin hustru Bat-Seba och gick in till henne och låg hos henne. Och hon födde en son, åt vilken han gav namnet Salomo. Och HERREN älskade honom
\par 25 och sände ett budskap med profeten Natan, och denne gav honom namnet Jedidja, för HERRENS skull.
\par 26 Och Joab angrep Rabba i Ammons barns land och intog konungastaden.
\par 27 Sedan sände Joab bud till David och lät säga honom: "Jag har angripit Rabba och har redan intagit Vattenstaden.
\par 28 Så församla du nu det övriga folket och belägra staden och intag den, så att det icke bliver jag som intager staden och får bära namnet därför."
\par 29 Då församlade David allt folket och tågade till Rabba och angrep det och intog det.
\par 30 Och han tog deras konungs krona från hans huvud; den vägde en talent guld och var prydd med en dyrbar sten. Den sattes nu på Davids huvud. Och han förde ut byte från staden i stor myckenhet.
\par 31 Och folket därinne förde han ut och lade dem under sågar och tröskvagnar av järn och bilor av järn och överlämnade dem åt Molok. Så gjorde han mot Ammons barns alla städer. Sedan vände David med allt folket tillbaka till Jerusalem.

\chapter{13}

\par 1 Därefter tilldrog sig följande Davids son Absalom hade en skön syster som hette Tamar, och Davids son Amnon fattade kärlek till henne.
\par 2 Ja, Amnon kom för sin syster Tamars skull i en sådan vånda att han blev sjuk; ty hon var jungfru, och det syntes Amnon icke vara möjligt att göra henne något.
\par 3 Men Amnon hade en vän, som hette Jonadab, en son till Davids broder Simea; och Jonadab var en mycket klok man.
\par 4 Denne sade nu till honom: "Varför ser du var morgon så avtärd ut, du konungens son? Vill du icke säga mig det?" Amnon svarade honom: "Jag har fattat kärlek till min broder Absaloms syster Tamar."
\par 5 Jonadab sade till honom: "Lägg dig på din säng och gör dig sjuk. När då din fader kommer för att besöka dig, så säg till honom: 'Låt min syster Tamar komma och giva mig något att äta, men låt henne tillreda maten inför mina ögon; så att jag ser det och kan få den ur hennes hand att äta.'"
\par 6 Då lade Amnon sig och gjorde sig sjuk. När nu konungen kom för att besöka honom, sade Amnon till konungen: "Låt min syster Tamar komma hit och tillaga två kakor inför mina ögon, så att jag kan få dem ur hennes hand att äta."
\par 7 Då sände David bud in i huset till Tamar och lät säga: "Gå till din broder Amnons hus och red till åt honom något att äta."
\par 8 Tamar gick då åstad till sin broder Amnons hus, där denne låg till sängs. Och hon tog deg och knådade den och gjorde därav kakor inför hans ögon och gräddade kakorna.
\par 9 Därefter tog hon pannan och lade upp dem därur inför hans ögon; men han ville icke äta. Och Amnon sade: "Låt alla gå ut härifrån. Då gingo alla ut därifrån.
\par 10 Sedan sade Amnon till Tamar: "Bär maten hitin i kammaren, så att jag får den ur din hand att äta." Då tog Tamar kakorna som hon hade tillrett och bar dem in i kammaren till sin broder Amnon.
\par 11 Men när hon kom fram med dem till honom, för att han skulle äta, fattade han i henne och sade till henne: "Kom hit och ligg hos mig, min syster."
\par 12 Hon sade till honom: "Ack nej, min broder, kränk mig icke; ty sådant får icke ske i Israel. Gör icke en sådan galenskap.
\par 13 Vart skulle jag då taga vägen med min skam? Och du själv skulle ju sedan i Israel hållas för en dåre. Tala nu med konungen; han vägrar nog icke att giva mig åt dig."
\par 14 Men han ville icke lyssna till hennes ord och blev henne övermäktig och kränkte henne och låg hos henne.
\par 15 Men därefter fick Amnon en mycket stor motvilja mot henne; ja, den motvilja han fick mot henne var större än den kärlek han hade haft till henne. Och Amnon sade till henne: "Stå upp och gå din väg."
\par 16 Då sade hon till honom: "Gör dig icke skyldig till ett så svårt brott som att driva bort mig; det vore värre än det andra som du har gjort med mig."
\par 17 Men han ville icke höra på henne, utan ropade på den unge man som han hade till tjänare och sade: "Driven denna kvinna ut härifrån, och rigla du dörren efter henne."
\par 18 Och hon hade en fotsid livklädnad på sig; ty i sådana kåpor voro konungens döttrar klädda, så länge de voro jungfrur. När tjänaren nu hade fört ut Tamar och riglat dörren efter henne,
\par 19 tog hon aska och strödde på sitt huvud, och den fotsida livklädnaden som hon hade på sig rev hon sönder; och hon lade handen på sitt huvud och gick där ropande och klagande.
\par 20 Då sade hennes broder Absalom till henne: "Har din broder Aminon varit hos dig? Tig nu stilla, min syster; han är ju din broder. Lägg denna sak icke så på sinnet." Så stannade då Tamar i sin broder Absaloms hus i svår sorg.
\par 21 Men när konung David fick höra allt detta, blev han mycket vred.
\par 22 Och Absalom talade intet med Amnon, varken gott eller ont, ty Absalom hatade Amnon, därför att denne hade kränkt hans syster Tamar.
\par 23 Två år därefter hade Absalom fårklippning i Baal-Hasor, som ligger vid Efraim. Och Absalom inbjöd då alla konungens söner.
\par 24 Absalom kom till konungen och sade: "Din tjänare skall nu hava fårklippning; jag beder att konungen ville jämte sina tjänare gå med din tjänare."
\par 25 Men konungen svarade Absalom "Nej, min son, vi må icke allasammans gå med, ty vi vilja icke vara dig till besvär." Och fastän han bad honom enträget, ville han icke gå, utan gav honom sin avskedshälsning.
\par 26 Då sade Absalom: "Om du icke vill, så låt dock min broder Amnon gå med oss." Konungen frågade honom: "Varför skall just han gå med dig?"
\par 27 Men Absalom bad honom så enträget, att han lät Amnon och alla de övriga konungasönerna gå med honom.
\par 28 Och Absalom bjöd sina tjänare och sade: "Sen efter, när Amnons hjärta bliver glatt av vinet; och när jag då säger till eder: 'Huggen ned Amnon', så döden honom utan fruktan. Det är ju jag som bjuder eder det, varen frimodiga och skicken eder såsom käcka män."
\par 29 Och Absaloms tjänare gjorde med Amnon såsom Absalom hade bjudit. Då stodo alla konungens söner upp och satte sig var och en på sin mulåsna och flydde.
\par 30 Medan de ännu voro på väg, kom till David ett rykte om att Absalom hade huggit ned alla konungens söner, så att icke en enda av dem fanns kvar.
\par 31 Då stod konungen upp och rev sönder sina kläder och lade sig på marken, under det att alla hans tjänare stodo där med sönderrivna kläder.
\par 32 Men Jonadab, som var son till Davids broder Simea, tog till orda och sade: "Min herre må icke tänka att de hava dödat alla de unga männen, konungens söner, det är Amnon allena som är död. Ty Absaloms uppsyn har bådat olycka ända ifrån den dag då denne kränkte hans syster Tamar.
\par 33 Så må nu min herre konungen icke akta på detta som har blivit sagt, att alla konungens söner äro döda; nej, Amnon allena är död."
\par 34 Emellertid flydde Absalom. - När nu mannen som stod på vakt lyfte upp sina ögon, fick han se mycket folk komma från vägen bakom honom, vid sidan av berget.
\par 35 Då sade Jonadab till konungen: "Se, där komma konungens söner. Såsom din tjänare sade, så har det gått till."
\par 36 Just när han hade sagt detta, kommo konungens söner; och de brusto ut i gråt. Också konungen och alla hans tjänare gräto häftigt och bitterligen.
\par 37 Men Absalom hade flytt och begivit sig till Talmai, Ammihurs son, konungen i Gesur. Och David sörjde hela tiden sin son.
\par 38 Sedan Absalom hade flytt och begivit sig till Gesur, stannade han där i tre år.
\par 39 Och konung David avstod ifrån att draga ut mot Absalom, ty han tröstade sig över att Amnon var död.

\chapter{14}

\par 1 Men Joab, Serujas son, märkte att konungens hjärta var vänt mot Absalom.
\par 2 Då sände Joab till Tekoa och lät därifrån hämta en klok kvinna och sade till henne: "Låtsa att du har sorg, och kläd dig i sorgkläder och smörj dig icke med olja, utan skicka dig såsom en kvinna som i lång tid har haft sorg efter en död.
\par 3 Gå så in till konungen och tala till honom såsom jag säger dig." Joab lade nu orden i hennes mun.
\par 4 Och kvinnan från Tekoa talade med konungen; hon föll ned till jorden på sitt ansikte och bugade sig och sade: "Hjälp, o konung!"
\par 5 Konungen sade till henne: "Vad fattas dig?" Hon svarade: "Ack, jag är änka; min man är död.
\par 6 Och din tjänarinna hade två söner; dessa båda kommo i träta med varandra ute på marken, där ingen fanns, som kunde träda emellan och hindra dem; den ene slog då ned den andre och dödade honom.
\par 7 Och nu har hela släkten rest sig upp mot din tjänarinna, och de säga: 'Giv hit honom som slog ned sin broder, så att vi få döda honom, därför att han har tagit sin broders liv och dräpt honom; på det sättet förgöra vi ock arvingen.' På det att efter min man varken namn eller efterkommande må finnas på jorden, vilja de utsläcka den gnista av mig, som ännu är kvar."
\par 8 Då sade konungen till kvinnan: "Gå hem igen; jag vill giva befallning om dig."
\par 9 Kvinnan från Tekoa sade till konungen: "På mig, o min herre konung, och på min faders hus vile missgärningen, men konungen och hans tron vare utan skuld."
\par 10 Konungen sade: "Om någon säger något åt dig, så för honom till mig; han skall sedan icke mer antasta dig.
\par 11 Hon sade: "Ja, må konungen tänka på HERREN, sin Gud, så att blodshämnaren icke får göra olyckan större, och så att de icke förgöra min son." Då sade han: "Så sant HERREN lever, icke ett hår av din son skall falla på jorden."
\par 12 Men kvinnan sade: "Låt din tjänarinna tala ännu ett ord till min herre konungen." Han sade: "Tala."
\par 13 Då sade kvinnan: "Varför är du då så sinnad mot Guds folk? När konungen talar så, då ligger ju däri att han själv bär på skuld, eftersom konungen icke låter sin förskjutne son komma tillbaka.
\par 14 Vi måste ju alla dö och äro då såsom vatten som spilles på jorden, vilket icke kan samlas upp igen. Men Gud tager icke livet bort, utan han tänker ut vad göras kan, för att den förskjutne icke må förbliva förskjuten och skild från honom.
\par 15 Och att jag nu har kommit för att tala detta till min herre konungen, det har skett därför att folket förskräckte mig. Då tänkte din tjänarinna: Jag vill dock tala med konungen; kanhända skall konungen uppfylla sin trälinnas önskan.
\par 16 Ja, konungen skall lyssna till sin trälinna och rädda mig från den mans hand, som vill förgöra både mig och min son från Guds arvedel.
\par 17 Och din tjänarinna tänkte: Min herre konungens ord skall giva mig ro. Ty min herre konungen är lik Guds ängel däri att han hör allt, både gott och ont. Och nu vare HERREN, din Gud, med dig.
\par 18 Då svarade konungen och sade till kvinnan: "Dölj icke för mig något av det varom jag nu vill fråga dig." Kvinnan sade: "Min herre konungen tale."
\par 19 Då sade konungen: "Har icke Joab sin hand med i allt detta?" Kvinnan svarade och sade: "Så sant du lever, min herre konung: om min Herre konungen talar något, så kan ingen komma undan det, vare sig åt höger eller åt vänster. Ja, det är din tjänare Joab som har bjudit mig detta, och han har lagt i din tjänarinnas mun allt vad jag har sagt.
\par 20 För att giva saken ett annat utseende har din tjänare Joab handlat på detta sätt; men min herre liknar i vishet Guds ängel och vet allt som sker på jorden."
\par 21 Så sade då konungen till Joab: "Välan, jag vill göra såsom du önskar. Gå nu och för tillbaka den unge mannen Absalom."
\par 22 Då föll Joab ned till jorden på sitt ansikte och bugade sig och välsignade konungen; och Joab sade: "I dag märker din tjänare att jag har funnit nåd för dina ögon, min herre konung, eftersom konungen uppfyller sin tjänares önskan."
\par 23 Och Joab stod upp och begav sig till Gesur och förde Absalom till Jerusalem.
\par 24 Men konungen sade: "Han får begiva sig till sitt hus, men han får icke komma inför mitt ansikte. Då begav sig Absalom till sitt hus, och kom icke inför konungens ansikte.
\par 25 Men i hela Israel fanns ingen så skön man som Absalom, ingen som man så mycket prisade: från hans fotblad upp till hans hjässa fanns icke något fel på honom
\par 26 Och när han lät klippa håret på sitt huvud - vid slutet av vart år lät han klippa det, ty det blev honom då så tungt att han måste låta klippa det - så befanns det, att när man vägde håret från hans huvud, då vägde det två hundra siklar, efter konungsvikt.
\par 27 Och åt Absalom föddes tre söner och en dotter, som fick namnet Tamar; hon var en skön kvinna.
\par 28 När Absalom hade bott två hela år i Jerusalem utan att få komma inför konungens ansikte,
\par 29 sände han bud efter Joab, i avsikt att skicka denne till konungen; men han ville icke komma till honom. Och han sände bud ännu en gång, men han ville ändå icke komma.
\par 30 Då sade han till sina tjänare: "I sen att Joab där har ett åkerstycke vid sidan av mitt, och på det har han korn; gån nu dit och tänden eld därpå." Så tände då Absaloms tjänare eld på åkerstycket.
\par 31 Då stod Joab upp och gick hem till Absalom och sade till honom: "Varför hava dina tjänare tänt eld på mitt åkerstycke?"
\par 32 Absalom svarade Joab: "Jag sände ju till dig och lät säga: Kom hit, så att jag kan skicka dig till konungen och låta säga: 'Varför fick jag komma hem från Gesur? Det hade varit bättre för mig, om jag ännu vore kvar där.' Nu vill jag komma inför konungens ansikte; och finnes någon missgärning hos mig, så må han döda mig.
\par 33 Då gick Joab till konungen och sade honom detta. Denne kallade då till sig Absalom, och han kom till konungen; och han föll ned för honom på sitt ansikte och bugade sig till jorden inför konungen. Och konungen kysste Absalom.

\chapter{15}

\par 1 En tid härefter skaffade Absalom sig vagn och hästar, därtill ock femtio man som löpte framför honom.
\par 2 Och Absalom plägade bittida om morgonen ställa sig vid sidan av vägen som ledde till porten, och så ofta någon då var på väg till konungen med en rättssak som han ville hava avdömd, kallade Absalom honom till sig och frågade: "Från vilken stad är du?" När han då svarade: "Din tjänare är från den och den av Israels stammar",
\par 3 sade Absalom till honom: "Din sak är visserligen god och rätt, men du har ingen som hör på dig hos konungen."
\par 4 Och Absalom tillade: "Ack om jag bleve satt till domare i landet! Om då var och en som hade någon rätts- och domssak komme till mig, så skulle jag skaffa honom rättvisa."
\par 5 Och när någon gick fram för att buga sig för honom, räckte han ut sin hand och fattade i honom och kysste honom.
\par 6 På detta sätt gjorde Absalom med alla israeliter som kommo för att få någon sak avdömd hos konungen. Så förledde Absalom Israels män.
\par 7 Fyrtio år voro nu förlidna, då Absalom en gång sade till konungen: "Låt mig begiva mig till Hebron för att där infria det löfte som jag har gjort åt HERREN.
\par 8 Ty din tjänare gjorde ett löfte, när jag bodde i Gesur i Aram; jag sade: 'Om HERREN låter mig komma tillbaka till Jerusalem, så vill jag hålla en gudstjänst åt HERREN.'"
\par 9 Konungen sade till honom: "Gå i frid." Då stod han upp och begav sig till Hebron.
\par 10 Men Absalom sände ut hemliga budbärare till alla Israels stammar och lät säga: "När I hören basunen ljuda, så sägen: 'Nu har Absalom blivit konung i Hebron.'"
\par 11 Och med Absalom hade följt två hundra män från Jerusalem, som voro inbjudna och följde med i all oskuld, utan att veta om någonting.
\par 12 Medan Absalom offrade slaktoffren, sände han också och lät hämta giloniten Ahitofel, Davids rådgivare, från hans stad Gilo. Och sammansvärjningen växte i styrka, och i allt större myckenhet gick folket över till Absalom.
\par 13 Men en budbärare kom till David och sade: "Israels män hava vänt sina hjärtan till Absalom."
\par 14 Då sade David till alla sina tjänare, dem som han hade hos sig i Jerusalem: "Upp, låt oss fly, ty ingen annan räddning finnes för oss undan Absalom. Skynden eder åstad, så att han icke med hast kommer över oss och för olycka över oss och slår stadens invånare med svärdsegg."
\par 15 Konungens tjänare svarade konungen: "Till allt vad min herre konungen behagar äro dina tjänare redo.
\par 16 Då drog konungen ut, och allt hans husfolk följde honom; dock lämnade konungen kvar tio av sina bihustrur för att vakta huset.
\par 17 Så drog då konungen ut, och allt folket följde honom; men de stannade vid Bet-Hammerhak.
\par 18 Och alla hans tjänare tågade förbi på sidan om honom, så ock alla keretéerna och peletéerna; och alla gatiterna, sex hundra man, som hade följt med honom från Gat, tågade likaledes förbi framför konungen.
\par 19 Då sade konungen till gatiten Ittai: "Varför går också du med oss? Vänd om och stanna hos den som nu är konung; du är ju en främling och därtill landsflyktig från ditt hem.
\par 20 I går kom du; skulle jag då i dag låta dig irra omkring med oss på var färd, nu då jag själv går jag vet icke vart? Vänd tillbaka och för dina bröder tillbaka med dig; må nåd och trofasthet bevisas eder."
\par 21 Men Ittai svarade konungen och sade: "Så sant HERREN lever, och så sant min herre konungen lever: på den plats där min herre konungen är, där vill ock din tjänare vara, det må gälla liv eller död."
\par 22 Då sade David till Ittai: "Kom då och drag med." Och gatiten Ittai drog med jämte alla sina män och alla kvinnor och barn som han hade med sig.
\par 23 Och hela landet grät högljutt, när allt folket drog fram. Och då nu konungen gick över bäcken Kidron, gick ock allt folket över och tog vägen åt öknen.
\par 24 Bland de andra såg man ock Sadok jämte alla leviterna, och de buro med sig Guds förbundsark; men de satte ned Guds ark - varvid också Ebjatar kom ditupp - till dess att allt folket hade hunnit draga fram ur staden.
\par 25 Då sade konungen till Sadok: "För Guds ark tillbaka in i staden. Om jag finner nåd för HERRENS ögon, låter han mig komma tillbaka, så att jag åter får se honom och hans boning.
\par 26 Men om han säger så: 'Jag har icke behag till dig' - se, då är jag redo; han göre då med mig såsom honom täckes."
\par 27 Och konungen sade till prästen Sadok: "Du är ju siare; vänd tillbaka till staden i frid. Och din son Ahimaas och Ebjatars son Jonatan, båda edra söner, må följa med eder.
\par 28 Se, jag vill dröja vid färjställena i öknen, till dess att ett budskap kommer från eder med underrättelser till mig.
\par 29 Då förde Sadok och Ebjatar Guds ark tillbaka till Jerusalem och stannade där.
\par 30 Men David gick gråtande uppför Oljeberget med överhöljt huvud ock bara fötter; och allt folket som följde med honom hade ock höljt över sina huvuden och gingo ditupp under gråt.
\par 31 Och när man berättade för David att Ahitofel var med bland dem som hade sammansvurit sig med Absalom, sade David: "HERRE, gör Ahitofels råd till dårskap."
\par 32 När sedan David hade kommit upp på bergstoppen, där man plägade tillbedja Gud, då kom arkiten Husai emot honom, med sönderriven livklädnad och med jord på sitt huvud.
\par 33 David sade till honom: "Om du går med mig, så bliver du mig till besvär.
\par 34 Men om du vänder tillbaka till staden och säger till Absalom: 'Din tjänare vill jag vara, o konung; jag har förut varit din faders tjänare, men nu vill jag vara din tjänare', så kan du gagna mig med att göra Ahitofels råd om intet.
\par 35 Där har du ju ock prästerna Sadok och Ebjatar; allt vad du får höra från konungens hus må du meddela prästerna Sadok och Ebjatar.
\par 36 De hava ju ock där sina båda söner hos sig: Sadok har Ahimaas, och Ebjatar Jonatan; genom dem kunnen I sända mig bud om allt vad I fån höra."
\par 37 Så gick då Husai, Davids vän, in i staden. Och jämväl Absalom drog in i Jerusalem.

\chapter{16}

\par 1 När David hade gått framåt ett litet stycke från bergstoppen, då mötte honom Siba, Mefibosets tjänare, med ett par lastade åsnor, som buro två hundra bröd, ett hundra russinkakor, ett hundra fruktkakor och en vinlägel.
\par 2 Då sade konungen till Siba: "Vad vill du med detta?" Siba svarade: "Åsnorna skola vara för konungens husfolk till att rida på, brödet och fruktkakorna skola tjänarna hava att äta, och vinet skola de törstande hava att dricka i öknen."
\par 3 Konungen sade: "Men var är din herres son?" Siba svarade konungen: "Han är kvar i Jerusalem; ty han tänkte: 'Nu skall Israels hus giva mig tillbaka min faders rike.'"
\par 4 Då sade konungen till Siba: "Se, allt vad Mefiboset äger skall vara ditt." Siba svarade: "Jag faller ned för dig; låt mig finna nåd för dina ögon, min herre konung."
\par 5 När sedan konung David hade kommit till Bahurim, då trädde därifrån ut en man som var besläktad med Sauls hus och hette Simei, Geras son; han trädde fram och for ut i förbannelser.
\par 6 Och han kastade stenar på David och på alla konung Davids tjänare, fastän allt folket och alla hjältarna omgåvo denne, både till höger och till vänster.
\par 7 Och Simeis ord, när han förbannade honom, voro dessa: "Bort, bort, du blodsman, du ogärningsman!
\par 8 HERREN låter nu allt Sauls hus' blod komma tillbaka över dig, du som har blivit konung i hans ställe; HERREN giver nu konungadömet åt din son Absalom. Se, nu har du kommit i den olycka du förtjänade, ty en blodsman är du."
\par 9 Då sade Abisai, Serujas son, till konungen: "Varför skall den döda hunden där få förbanna min herre konungen? Låt mig gå dit och hugga huvudet av honom."
\par 10 Men konungen svarade: "Vad haven I med mig att göra, I Serujas söner? Om han förbannar, och om det är HERREN som har bjudit honom att förbanna David, vem törs då fråga: 'Varför gör du så?'
\par 11 Och David sade ytterligare till Abisai och till alla sina tjänare: "Min son, han som har utgått från mitt eget liv, står mig ju efter livet; med huru mycket mer skäl då denne benjaminit! Låten honom vara, må han förbanna; ty HERREN har befallt honom det.
\par 12 Kanhända skall HERREN se till den orätt mig sker, så att HERREN åter giver mig lycka, till gengäld för den förbannelse som i dag uttalas över mig."
\par 13 Och David gick med sina män vägen fram, under det att Simei gick längs utmed berget, jämsides med honom, och for ut i förbannelser och kastade stenar och grus, där han gick jämsides med honom.
\par 14 När så konungen, med allt folket som följde honom, hade kommit till Ajefim, rastade han där.
\par 15 Men Absalom hade med allt sitt folk, Israels män, kommit till Jerusalem; han hade då också Ahitofel med sig.
\par 16 När nu arkiten Husai, Davids vän, kom till Absalom, ropade Husai till Absalom: "Leve konungen! Leve konungen!"
\par 17 Absalom sade till Husai: "Är det så du visar din kärlek mot din vän? Varför har du icke följt med din vän?"
\par 18 Husai svarade Absalom: "Nej, den som HERREN och detta folk och alla Israels män hava utvalt, honom vill jag tillhöra, och hos honom vill jag stanna.
\par 19 Och dessutom, vilken bör jag tjäna? Bör jag icke tjäna inför hans son? Jo, såsom jag har tjänat inför din fader, så vill jag ock göra det inför dig."
\par 20 Och Absalom sade till Ahitofel: "Given nu ett råd om vad vi skola göra."
\par 21 Ahitofel sade till Absalom: "Gå in till din faders bihustrur, som han har lämnat kvar för att vakta huset. Då får hela Israel höra att du har gjort dig förhatlig för din fader, och så styrkes modet hos alla dem som hålla med dig.
\par 22 Därefter slog man upp ett tält åt Absalom ovanpå taket, och så gick Absalom in till sin faders bihustrur inför hela Israels ögon.
\par 23 Den tiden gällde nämligen ett råd som Ahitofel gav lika mycket som om man hade frågat Gud till råds; så mycket gällde vart råd av Ahitofel både för David och för Absalom.

\chapter{17}

\par 1 Och Ahitofel sade till Absalom: "Låt mig utvälja tolv tusen män, så vill jag bryta upp och förfölja David i natt.
\par 2 Då kan jag komma över honom och förskräcka honom, medan han är utmattad och modlös, och allt hans folk skall då taga till flykten; sedan kan jag döda konungen, när han står där övergiven.
\par 3 Därefter skall jag föra allt folket tillbaka till dig. Ty om det så går den man du söker, så är detta som om alla vände tillbaka; allt folket får då frid."
\par 4 Detta behagade Absalom och alla de äldste i Israel.
\par 5 Likväl sade Absalom: "Kalla ock arkiten Husai hit, så att vi också få höra vad han har att säga.
\par 6 När då Husai kom in till Absalom, sade Absalom till honom: "Så och så har Ahitofel talat. "Skola vi göra såsom han har sagt? Varom icke, så tala du."
\par 7 Husai svarade Absalom: Det råd som Ahitofel denna gång har givit är icke gott."
\par 8 Och Husai sade ytterligare: "Du känner din fader och hans män, och vet att de äro hjältar och bistra såsom en björninna från vilken man har tagit ungarna ute på marken. Och din fader är ju en krigsman som icke vilar med sitt folk under natten.
\par 9 Nu har han säkerligen gömt sig i någon håla eller på något annat ställe. Om nu redan i början några av folket här fölle, så skulle var och en som finge höra talas därom säga att det folk som följer Absalom har lidit ett nederlag;
\par 10 och då skulle till och med den tappraste, den som hade mod såsom ett lejon, bliva högeligen förfärad; ty hela Israel vet att din fader är en hjälte, och att de som följa honom äro tappra män.
\par 11 Därför är nu mitt råd: Låt hela Israel från Dan ända till Beer-Seba församla sig till dig, så talrikt som sanden vid havet; och själv må du draga med i striden.
\par 12 När vi så drabba ihop med honom, varhelst han må påträffas, skola vi slå ned på honom, såsom daggen faller över marken; och då skall intet bliva kvar av honom och alla de män som äro med honom
\par 13 Ja, om han också droge sig tillbaka in i någon stad, så skulle hela Israel kasta linor omkring den staden, och vi skulle draga den ned i dalen, till dess att icke minsta sten vore att finna därav.
\par 14 Då sade Absalom och alla Israels män: "Arkiten Husais råd är bättre än Ahitofels råd." HERREN hade nämligen skickat det så, att Ahitofels goda råd gjordes om intet, för att HERREN skulle låta olycka komma över Absalom.
\par 15 Och Husai sade till prästerna Sadok och Ebjatar: "Det och det rådet har Ahitofel givit Absalom och de äldste i Israel, men jag har givit det och det rådet.
\par 16 Så sänden nu med hast bud och låten säga David: 'Stanna icke över natten vid färjställena i öknen, utan gå hellre över, för att icke konungen och allt hans folk må drabbas av fördärv.'"
\par 17 Nu hade Jonatan och Ahimaas sitt tillhåll vid Rogelskällan, och en tjänstekvinna gick dit med budskap; sedan plägade de själva gå med budskapet till konung David. Ty de tordes icke gå in i staden och visa sig där.
\par 18 Men en gosse fick se dem och berättade det för Absalom. Då gingo båda med hast sin väg och kommo in i en mans hus i Bahurim, som på sin gård hade en brunn; i den stego de ned.
\par 19 Och hans hustru tog ett skynke och bredde ut det över brunnshålet och strödde gryn därpå, så att man icke kunde märka något.
\par 20 Då nu Absaloms tjänare kommo in i huset till hustrun och frågade var Ahimaas och Jonatan voro, svarade hon dem: "De gingo över bäcken där." Då sökte de, men utan att finna, och vände så tillbaka till Jerusalem.
\par 21 Men sedan de hade gått sin väg, stego de andra upp ur brunnen och gingo med sitt budskap till konung David; de sade till David: "Bryten upp och gån med hast över vattnet, ty det och det rådet har Ahitofel givit, till eder ofärd."
\par 22 Då bröt David upp med allt det folk han hade hos sig, och de gingo över Jordan; och om morgonen, när det blev dager, saknades ingen enda, utan alla hade kommit över Jordan.
\par 23 Men när Ahitofel såg att man icke följde hans råd, sadlade han sin åsna och stod upp och for hem till sin stad, och sedan han hade beställt om sitt hus, hängde han sig. Och när han var död, blev han begraven i sin faders grav.
\par 24 Så hade nu David kommit till Mahanaim, när Absalom med alla Israels män gick över Jordan.
\par 25 Men Absalom hade satt Amasa i Joabs ställe över hären. Och Amasa var son till en man vid namn Jitra, en israelit, som hade gått in till Abigal, Nahas' dotter och Serujas, Joabs moders, syster.
\par 26 Och Israel och Absalom lägrade sig i Gileads land.
\par 27 Men när David kom till Mahanaim, hade Sobi, Nahas' son, från Rabba i Ammons barns land, och Makir, Ammiels son, från Lo-Debar, och Barsillai, en gileadit från Rogelim,
\par 28 låtit föra dit sängar, skålar, lerkärl, så ock vete, korn, mjöl och rostade ax, ävensom bönor, linsärter och annat rostat,
\par 29 därjämte honung, gräddmjölk, får och nötostar till mat åt David och hans folk; ty de tänkte: "Folket är hungrigt, trött och törstigt i öknen.

\chapter{18}

\par 1 Och David mönstrade sitt folk och satte över- och underhövitsmän över dem.
\par 2 Därefter lät David folket tåga åstad: en tredjedel under Joabs befäl, en tredjedel under Abisais, Serujas sons, Joabs broders, befäl, och en tredjedel under gatiten Ittais befäl. Och konungen sade till folket: "Jag vill ock själv draga ut med eder."
\par 3 Men folket svarade: "Du får icke draga ut; ty om vi måste fly, aktar ingen på oss, och om hälften av oss bliver dödad, aktar man icke heller på oss, men du är nu så god som tio tusen av oss. Därför är det nu bättre att du står redo att komma oss till hjälp från staden.
\par 4 Då sade konungen till dem: "Vad I ansen vara bäst vill jag göra." Och konungen ställde sig vid sidan av porten, under det att allt folket drog ut i avdelningar på hundra och tusen.
\par 5 Men konungen bjöd Joab, Abisai och Ittai och sade: "Faren nu varligt med den unge mannen Absalom." Och allt folket hörde huru konungen så bjöd alla hövitsmännen angående Absalom.
\par 6 Så drog då folket ut på fältet mot Israel, och striden stod i Efraims skog.
\par 7 Där blev Israels folk slaget av Davids tjänare, och många stupade där på den dagen: tjugu tusen man.
\par 8 Och striden utbredde sig över hela den trakten; och skogen förgjorde mer folk, än svärdet förgjorde på den dagen.
\par 9 Och Absalom kom i Davids tjänares väg. Absalom red då på sin mulåsna; och när mulåsnan kom under en stor terebint med täta grenar, fastnade hans huvud i terebinten, så att han blev hängande mellan himmel och jord, ty mulåsnan som han satt på sprang sin väg.
\par 10 Och en man fick se det och berättade för Joab och sade: "Jag såg där borta Absalom hänga i en terebint."
\par 11 Då sade Joab till mannen som berättade detta för honom: "Om du såg det, varför slog du honom då icke strax till jorden? Jag skulle då gärna hava givit dig tio siklar silver och ett bälte."
\par 12 Men mannen svarade Joab: "Om jag ock finge väga upp tusen siklar silver i mina händer, skulle jag dock icke vilja uträcka min hand mot konungens son, ty konungen bjöd ju dig och Abisai och Ittai, så att vi hörde det: 'Tagen vara, I alla, på den unge mannen Absalom.'
\par 13 Dessutom, om jag lömskt hade förgripit mig på hans liv, så hade du säkerligen lämnat mig i sticket, eftersom intet kan förbliva dolt för konungen."
\par 14 Joab sade: "Jag vill icke på detta sätt förhala tiden med dig." Därefter tog han tre spjut i sin hand och stötte dem i Absaloms bröst, medan denne ännu var vid liv, där han hängde under terebinten.
\par 15 Sedan kommo tio unga män, Joabs vapendragare, ditfram, och av dem blev Absalom till fullo dödad.
\par 16 Och Joab lät stöta i basunen, och folket upphörde att förfölja Israel, ty Joab ville skona folket.
\par 17 Och de togo Absalom och kastade honom i en stor grop i skogen och staplade upp ett mycket stort stenröse över honom. Men hela Israel flydde, var och en till sin hydda.
\par 18 Och Absalom hade, medan han ännu levde, låtit resa åt sig en stod som står i Konungsdalen; ty han tänkte: "Jag har ingen son som kan bevara mitt namns åminnelse. Den stoden hade han uppkallat efter sitt namn, och den heter ännu i dag Absaloms minnesvård.
\par 19 Och Ahimaas, Sadoks son, sade: "Låt mig skynda åstad och förkunna för konungen glädjebudskapet att HERREN har dömt honom fri ifrån hans fienders hand."
\par 20 Men Joab svarade honom: "I dag bliver du ingen glädjebudbärare; en annan dag må du förkunna glädjebudskap, men denna dag förkunnar du icke något glädjebudskap, eftersom nu konungens son är död."
\par 21 Därefter sade Joab till en etiopier: "Gå och berätta för konungen vad du har sett." Då föll etiopiern ned för Joab och skyndade därpå åstad.
\par 22 Men Ahimaas, Sadoks son; sade ännu en gång till Joab: "Låt också mig, vad än må ske, få skynda åstad, efter etiopiern." Joab sade: "Varför vill du skynda åstad, min son, då detta ju icke kan vara ett glädjebudskap som skaffar dig någon lön?"
\par 23 Han svarade: "Vad än må ske vill jag skynda åstad." Då sade han till honom: "Så skynda då." Och Ahimaas skyndade åstad och tog vägen över Jordanslätten och hann om etiopiern.
\par 24 Under tiden satt David inne i porten. Och väktaren gick upp på porttaket invid muren; när han där lyfte upp sina ögon, fick han se en man komma ensam springande.
\par 25 Väktaren ropade och förkunnade det för konungen. Då sade konungen: "Är han ensam, så har han ett glädjebudskap att kungöra." Och han kom allt närmare.
\par 26 Därefter fick väktaren se en annan man komma springande; då ropade väktaren till portvaktaren och sade: "Nu ser jag åter en man komma ensam springande." Konungen sade: "Denne är ock en glädjebudbärare."
\par 27 Och väktaren sade: "Efter sitt sätt att springa tyckes mig den förste vara Ahimaas, Sadoks son." Då sade konungen: "Det är en god man; han kommer säkerligen med ett gott glädjebudskap."
\par 28 Och Ahimaas ropade och sade till konungen: "Allt väl!" Därefter föll han ned till Jorden på sitt ansikte inför konungen och sade: "Lovad vare HERREN, din Gud, som har prisgivit de människor som hade upplyft sin hand mot min herre konungen!"
\par 29 Då frågade konungen: "Står det väl till med den unge mannen Absalom?" Ahimaas svarade: "Jag såg en stor hop folk, när Joab avsände konungens andre tjänare och mig, din tjänare; men jag vet icke vad det var."
\par 30 Konungen sade: "Gå åt sidan och ställ dig där." Då gick han åt sidan och blev stående där
\par 31 Just då kom etiopiern. Och etiopiern sade: "Mottag, min herre konung, det glädjebudskapet att HERREN i dag har dömt dig fri ifrån alla de mäns hand, som hava rest sig upp mot dig."
\par 32 Konungen frågade etiopiern: "Står det väl till med den unge mannen Absalom?" Etiopiern svarade: "Må det så gå med min herre konungens fiender och med alla som resa sig upp mot dig för att göra dig ont, såsom det har gått med den unge mannen.
\par 33 Då blev konungen häftigt upprörd och gick upp i salen över porten och grät. Och under det att han gick, ropade han så: "Min son Absalom, min son, min son Absalom! Ack, att jag hade fått dö i ditt ställe! Absalom, min son, min son!"

\chapter{19}

\par 1 Och det blev berättat för Joab att konungen grät och sörjde Absalom.
\par 2 Och segern blev på den dagen förbytt till sorg för allt folket, eftersom folket på den dagen fick höra sägas att konungen var bedrövad för sin sons skull.
\par 3 Och folket smög sig på den dagen in i staden, såsom människor pläga göra, vilka hava vanärat sig, därigenom att de hava flytt under striden.
\par 4 Men konungen hade skylt sitt ansikte; och konungen klagade med hög röst: "Min son Absalom! Absalom, min son, min son!"
\par 5 Då gick Joab in i huset till konungen och sade: "Du kommer i dag alla dina tjänares ansikten att rodna av skam, fastän de i dag hava räddat både ditt eget liv och dina söners och döttrars liv och dina hustrurs liv och dina bihustrurs liv.
\par 6 Ty du älskar ju dem som hata dig, och hatar dem som älska dig. I dag har du nämligen gjort kunnigt att dina hövitsmän och tjänare äro intet for dig, ty i dag märker jag, att om Absalom vore vid liv, men alla vi andra i dag hade omkommit, så skulle detta hava varit dig mer till behag.
\par 7 Men stå nu upp, och gå ut och tala vänligt med dina tjänare; ty jag svär vid HERREN, att om du icke gör det, så skall icke en enda man stanna kvar hos dig över denna natt, och detta skall för dig bliva en större olycka än alla de olyckor som hava övergått dig från din ungdom ända till nu."
\par 8 Då stod konungen upp och satte sig i porten. Och man gjorde kunnigt för allt folket och sade: "Konungen sitter nu i porten." Då kom allt folket inför konungen. Men Israel hade flytt, var och en till sin hydda.
\par 9 Och allt folket i alla Israels stammar begynte därefter förebrå varandra och säga: "Konungen har räddat oss från vara fienders hand och hjälpt oss ifrån filistéernas hand, och nu har han måst fly ur landet för Absalom.
\par 10 Men Absalom, som vi hade smort till konung över oss, har blivit dödad i striden. Varför sägen I då icke ett ord om att föra konungen tillbaka?"
\par 11 Under tiden hade konung David sänt bud till prästerna Sadok och Ebjatar och låtit säga: "Talen så till de äldste i Juda: 'Varför skolen I vara de sista att hämta konungen tillbaka hem? Ty vad hela Israel talar har redan kommit för konungen, där han bor.
\par 12 I ären ju mina bröder, I ären ju mitt kött och ben. Varför skolen I då vara de sista att hämta konungen tillbaka?'
\par 13 Och till Amasa skolen I säga: 'Är du icke mitt kött och ben? Gud straffe mig nu och framgent, om du icke för all din tid skall bliva härhövitsman hos mig i Joabs ställe.'"
\par 14 Härigenom vann han alla Juda mäns hjärtan utan undantag, så att de sände detta budskap till konungen: "Vänd tillbaka, du själv med alla dina tjänare."
\par 15 Då vände konungen tillbaka och kom till Jordan; men Juda hade kommit till Gilgal för att möta konungen och föra konungen över Jordan.
\par 16 Också Simei, Geras son, benjaminiten, som var från Bahurim, skyndade sig och drog ned med Juda män for att möta konung David.
\par 17 Och med honom följde tusen man från Benjamin, ävensom Siba, vilken hade varit tjänare i Sauls hus, jämte hans femton söner och tjugu tjänare. Dessa hade nu hastat ned till Jordan före konungen.
\par 18 Och färjan gick över för att överföra konungens familj, och för att användas efter hans gottfinnande. Men Simei, Geras son, föll ned inför konungen, när han skulle fara över Jordan,
\par 19 och sade till konungen: "Må min herre icke tillräkna mig min missgärning, och icke tänka på huru illa din tjänare gjorde på den dag då min herre konungen drog ut från Jerusalem; må konungen icke akta därpå.
\par 20 Ty din tjänare inser att jag då försyndade mig; därför har jag nu i dag först av hela Josefs hus kommit hitned för att möta min herre konungen.
\par 21 Då tog Abisai, Serujas son, till orda och sade: "Skulle icke Simei dödas för detta? Han har ju förbannat HERRENS smorde."
\par 22 Men David svarade: "Vad haven I med mig att göra, I Serujas söner, eftersom I i dag ären mig till hinders? Skulle väl i dag någon dödas i Israel? Vet jag då icke att jag i dag har blivit konung över Israel?"
\par 23 Därefter sade konungen till Simei: "Du skall icke dö." Och konungen gav honom sin ed därpå.
\par 24 Mefiboset, Sauls son, hade ock kommit ned för att möta konungen. Han hade varken ansat sina fötter eller sitt skägg, ej heller hade han låtit två sina kläder allt ifrån den dag då konungen drog bort, ända till den dag då han kom igen i frid.
\par 25 När han nu kom till Jerusalem för att möta konungen, sade konungen till honom: "Varför följde du icke med mig, Mefiboset?"
\par 26 Han svarade: "Min herre konung, min tjänare bedrog mig. Ty din tjänare sade: 'Jag vill sadla min åsna och sätta mig på den och så begiva mig till konungen'; din tjänare är ju halt.
\par 27 Men han har förtalat din tjänare hos min herre konungen. Min herre konungen är ju dock såsom Guds ängel; så gör nu vad dig täckes.
\par 28 Ty hela min faders hus förtjänade intet annat än döden av min herre konungen, och likväl lät du din tjänare sitta bland dem som få äta vid ditt bord. Vad har jag då rätt att ytterligare begära, och varom kan jag väl ytterligare ropa till konungen?"
\par 29 Konungen sade till honom: "Varför ordar du ytterligare härom? Jag säger att du och Siba skolen dela jordagodset."
\par 30 Då sade Mefiboset till konungen: "Han må gärna taga alltsammans, sedan nu min herre konungen har kommit hem igen i frid."
\par 31 Gileaditen Barsillai hade ock farit ned från Rogelim och drog sedan med konungen till Jordan, för att få ledsaga honom över Jordan.
\par 32 Barsillai var då mycket gammal: åttio år. Han hade sörjt för konungens behov, medan denne uppehöll sig i Mahanaim, ty han var en mycket rik man.
\par 33 Konungen sade nu till Barsillai: "Du skall draga med mig, så skall jag sörja för dina behov hemma hos mig i Jerusalem."
\par 34 Men Barsillai svarade konungen: "Huru många år kan jag väl ännu hava att leva, eftersom jag skulle följa med konungen upp till Jerusalem?
\par 35 Jag är nu åttio år gammal; kan jag då känna skillnad mellan bättre och sämre, eller har väl din tjänare någon smak för vad jag äter eller för vad jag dricker? Eller kan jag ännu njuta av att höra sångare och sångerskor sjunga? Varför skulle din tjänare då ytterligare bliva min herre konungen till besvär?
\par 36 Allenast för en stund vill din tjänare fara med konungen över Jordan. Varför skulle väl konungen giva mig en sådan vedergällning?
\par 37 Låt din tjänare vända tillbaka, så att jag får dö i min stad, där jag har min faders och min moders grav. Men se här är din tjänare Kimham, låt honom få draga med min herre konungen; och gör för honom vad dig täckes."
\par 38 Då sade konungen: "Så må då Kimham draga med mig, och jag skall göra för honom vad du vill. Och allt vad du begär av mig skall jag göra dig."
\par 39 Därefter gick allt folket över Jordan, och konungen själv gick också över. Och konungen kysste Barsillai och tog avsked av honom. Sedan vände denne tillbaka hem igen.
\par 40 Så drog nu konungen till Gilgal, och Kimham följde med honom, så ock allt Juda folk. Och de, jämte hälften av Israels folk, förde konungen ditöver.
\par 41 Men då kommo alla de övriga israeliterna till konungen och sade till honom: "Varför hava våra bröder, Juda män, fått hemligen bemäktiga sig dig och föra konungen och hans familj, tillika med alla Davids män, över Jordan?"
\par 42 Alla Juda män svarade Israels män: "Konungen står ju oss närmast; varför vredgens I då häröver? Hava vi levat på konungen eller skaffat oss någon vinning genom honom?"
\par 43 Då svarade Israels män Juda män och sade: "Tio gånger större del än I hava vi i den som är konung, alltså ock i David. Varför haven I då ringaktat oss? Och voro icke vi de som först talade om att hämta vår konung tillbaka?" Men Juda män läto ännu hårdare ord falla än Israels män.

\chapter{20}

\par 1 Nu hände sig att där fanns en illasinnad man vid namn Seba, Bikris son, en benjaminit. Denne stötte i basun och sade: "Vi hava ingen del i David och ingen arvslott i Isais son. Israel drage hem, var och en till sin hydda."
\par 2 Då övergåvo alla Israels män David och följde Seba, Bikris son; men Juda män höllo sig till sin konung och följde honom från Jordan ända till Jerusalem.
\par 3 Så kom David hem igen till Jerusalem. Och konungen tog då de tio bihustrur som han hade lämnat kvar för att vakta huset, och satte in dem i ett särskilt hus till att där förvaras; och han gav dem underhåll, men gick icke in till dem. Där förblevo de nu instängda till sin dödsdag och levde redan under hans livstid såsom änkor.
\par 4 Och konungen sade till Amasa: "Båda upp åt mig Juda män inom tre dagar, och inställ dig sedan själv här."
\par 5 Amasa begav sig då åstad för att uppbåda Juda; men när han dröjde utöver den tid som hade blivit honom förelagd,
\par 6 sade David till Abisai: "Nu kommer Seba, Bikris son, att bliva farligare för oss än Absalom. Tag du din herres tjänare och sätt efter honom, så att han icke bemäktigar sig några befästa städer och tillfogar oss för stor skada."
\par 7 Alltså drogo Joabs män tillika med keretéerna och peletéerna och alla hjältarna ut efter honom; de drogo ut från Jerusalem för att sätta efter Seba, Bikris son.
\par 8 Men när de hade hunnit till den stora stenen vid Gibeon, kom Amasa emot dem. Joab var då klädd i livrocken som plägade utgöra hans dräkt, och ovanpå den hade han ett bälte, med ett svärd i skidan, bundet över sina länder; men när han gick fram, föll det ut.
\par 9 Och Joab sade till Amasa: "Står det väl till med dig, min broder?" Därvid fattade Joab Amasa i skägget med högra handen såsom för att kyssa honom.
\par 10 Och då Amasa icke tog sig till vara för det svärd som Joab hade i sin andra hand, gav denne honom därmed en stöt i underlivet, så att hans inälvor runno ut på jorden. Så dog han, utan att den andre behövde giva honom någon ytterligare stöt. Därefter fortsatte Joab och hans broder Abisai att förfölja Seba, Bikris son.
\par 11 Men en av Joabs tjänare stod kvar därbredvid och ropade: "Var och en som är Joabs vän och håller med David, han följe efter Joab."
\par 12 Nu låg Amasa sölad i sitt blod mitt på vägen; och mannen såg allt folket stannade. Då förde han Amasa undan från vägen in på åkern och kastade ett kläde över honom, eftersom han såg huru alla de som kommo därförbi stannade.
\par 13 Så snart han var bortskaffad från vägen, drogo alla förbi och följde Joab för att sätta efter Seba, Bikris son.
\par 14 Denne drog emellertid genom alla Israels stammar till Abel och Bet-Maaka och genom hela Habberim; och folk samlade sig och följde honom ända ditin.
\par 15 Men de kommo och belägrade honom där i Abel vid Bet-Hammaaka och kastade upp mot staden en vall, som reste sig inemot yttermuren. Och allt Joabs folk arbetade på att förstöra muren och kullstöta den.
\par 16 Då ropade en klok kvinna från staden: "Hören! Hören! Sägen till Joab att han kommer hit, så att jag får tala med honom."
\par 17 När han då kom fram till kvinnan, frågade hon: "Är du Joab?" Han svarade: "Ja." Hon sade till honom: "Hör din tjänarinnas ord." Han svarade: "Jag hör."
\par 18 Då sade hon: "Fordom plägade man säga så: 'I Abel skall man fråga till råds'; sedan kunde man utföra sina planer.
\par 19 Vi äro de fridsammaste och trognaste i Israel, och du söker att förgöra en stad som är en moder i Israel. Varför vill du förstöra HERRENS arvedel?"
\par 20 Joab svarade och sade: "Bort det, bort det, att jag skulle vilja förstöra och fördärva!
\par 21 Det är icke så, utan en man från Efraims bergsbygd vid namn Seba, Bikris son, har rest sig upp mot konung David; utlämnen allenast honom, så vill jag draga bort ifrån staden." Kvinnan svarade Joab: "Hans huvud skall strax bliva utkastat till dig över muren."
\par 22 Sedan vände sig kvinnan med sitt kloka råd till allt folket, och de höggo huvudet av Seba, Bikris son, och kastade ut det till Joab. Då stötte denne i basunen, och krigsfolket skingrade sig och drog bort ifrån staden, var och en till sin hydda. Och Joab vände tillbaka till konungen i Jerusalem.
\par 23 Joab hade nu befälet över hela krigshären i Israel, och Benaja, Jojadas son, hade befälet över keretéerna och peletéerna.
\par 24 Adoram hade uppsikten över de allmänna arbetena, och Josafat, Ahiluds son, var kansler.
\par 25 Seja var sekreterare, och Sadok och Ebjatar voro präster.
\par 26 Dessutom var ock jairiten Ira präst hos David.

\chapter{21}

\par 1 Men under Davids tid uppstod en hungersnöd, som varade oavbrutet i tre år; då sökte David HERRENS ansikte. HERREN svarade: "För Sauls och hans blodbefläckade hus' skull sker detta, därför att han dödade gibeoniterna.
\par 2 Då kallade konungen till sig gibeoniterna och talade med dem. Men gibeoniterna voro icke israeliter, utan en kvarleva av amoréerna och fastän Israels barn hade givit dem sin ed, hade Saul, i sin nitälskan för Israels barn och för Juda, försökt att nedgöra dem.
\par 3 David sade nu till gibeoniterna: "Vad skall jag göra för eder, och varmed skall jag bringa försoning, så att I välsignen HERRENS arvedel?"
\par 4 Gibeoniterna svarade honom: "Vi fordra icke silver och guld av Saul och hans hus, ej heller hava vi rätt att döda någon man i Israel." Han frågade: "Vad begären I då att jag skall göra för eder?"
\par 5 De svarade konungen: "Den man som ville förgöra oss, och som stämplade mot oss, för att vi skulle bliva utrotade och icke mer hava bestånd någonstädes inom Israels land,
\par 6 av hans söner må sju utlämnas till oss, så att vi få upphänga dem för HERREN i Sauls, HERRENS utvaldes, Gibea." Konungen sade: "Jag skall utlämna dem."
\par 7 Men konungen skonade Mefiboset, Sauls son Jonatans son, för den ed vid HERREN, som de, David och Jonatan, Sauls son, hade svurit varandra.
\par 8 Däremot tog konungen de två söner, Armoni och Mefiboset, som Rispa, Ajas dotter, hade fött åt Saul, och de fem söner som Mikal, Sauls dotter, hade fött åt meholatiten Adriel, Barsillais son
\par 9 och överlämnade dem åt gibeoniterna, och dessa upphängde dem på berget inför HERREN, så att de omkommo, alla sju på en gång. Och det var under de första skördedagarna, när kornskörden begynte, som de blevo dödade.
\par 10 Då tog Rispa, Ajas dotter, sin sorgdräkt och hade den till sitt läger ovanpå klippan från det att skörden begynte, ända till dess att vattnet strömmade ned över dem från himmelen; och hon tillstadde icke himmelens fåglar att slå ned på dem om dagen, ej heller markens vilda djur att göra det om natten.
\par 11 När det blev berättat för David vad Rispa, Ajas dotter, Sauls bihustru, hade gjort
\par 12 begav sig David åstad och hämtade Sauls och hans son Jonatans ben från borgarna i Jabes i Gilead. Dessa hade nämligen i hemlighet tagit deras kroppar bort ifrån den öppna platsen i Bet-San, där filistéerna hade hängt upp dem, när filistéerna slogo Saul på Gilboa.
\par 13 Och då han hade fört Sauls och hans son Jonatans ben upp därifrån, samlade man ock ihop de upphängdas ben.
\par 14 Sedan begrov man Sauls och hans son Jonatans ben i Benjamins land, i Sela, i hans fader Kis' grav; man gjorde allt vad konungen hade bjudit. Och därefter hörde Gud landets bön.
\par 15 Åter uppstod krig mellan filistéerna och Israel. Och David drog ned med sina tjänare, och de stridde mot filistéerna. Men David blev trött;
\par 16 och Jisbo-Benob, en av rafaéernas avkomlingar, vilkens lans vägde tre hundra siklar koppar, och som var iklädd en ny rustning, tänkte då döda David.
\par 17 Men Abisai, Serujas son, kom honom till hjälp och slog filistéen till döds. Då besvuro Davids män honom att han icke mer skulle draga ut med dem i striden, så att han icke utsläckte Israels lampa.
\par 18 Därefter stod åter en strid med filistéerna vid Gob; husatiten Sibbekai slog då ned Saf, en av rafaéernas avkomlingar.
\par 19 Åter stod en strid med filistéerna vid Gob; Elhanan, Jaare-Oregims son, betlehemiten, slog då ned gatiten Goljat, som hade ett spjut vars skaft liknade en vävbom.
\par 20 Åter stod en strid vid Gat. Där var en reslig man som hade sex fingrar på var hand och sex tår på var fot, eller tillsammans tjugufyra; han var ock en avkomling av rafaéerna.
\par 21 Denne smädade Israel; då blev han nedgjord av Jonatan, son till Simeai, Davids broder.
\par 22 Dessa fyra voro avkomlingar av rafaéerna i Gat; och de föllo för Davids och hans tjänares hand.

\chapter{22}

\par 1 Och David talade till HERREN denna sångs ord, när HERREN hade räddat honom från alla hans fienders hand och från Sauls hand.
\par 2 Han sade: HERRE, du mitt bergfäste, min borg och min räddare,
\par 3 Gud, du min klippa, till vilken jag tager min tillflykt, min sköld och min frälsnings horn, mitt värn och min tillflykt, min frälsare, du som frälsar mig från våldet!
\par 4 HERREN, den högtlovade, åkallar jag, och från mina fiender bliver jag frälst.
\par 5 Ty dödens bränningar omvärvde mig, fördärvets strömmar förskräckte mig,
\par 6 dödsrikets band omslöto mig, dödens snaror föllo över mig.
\par 7 Men jag åkallade HERREN i min nöd, ja, jag gick med min åkallan till min Gud. Och han hörde från sin himmelska boning min röst, och mitt rop kom till hans öron.
\par 8 Då skalv jorden och bävade, himmelens grundvalar darrade; de skakades, ty hans vrede var upptänd.
\par 9 Rök steg upp från hans näsa och förtärande eld från hans mun, eldsglöd ljungade från honom.
\par 10 Och han sänkte himmelen och for ned och töcken var under hans fötter.
\par 11 Han for på keruben och flög, han sågs komma på vindens vingar
\par 12 Och han gjorde mörker till en hydda som omslöt honom: vattenhopar, tjocka moln.
\par 13 Ur glansen framför honom ljungade eldsglöd.
\par 14 HERREN dundrade från himmelen den Högste lät höra sin röst.
\par 15 Han sköt pilar och förskingrade dem, ljungeld och förvirrade dem.
\par 16 Havets bäddar kommo i dagen, jordens grundvalar blottades, för HERRENS näpst, för hans vredes stormvind.
\par 17 Han räckte ut sin hand från höjden och fattade mig, han drog mig upp ur de stora vattnen.
\par 18 Han räddade mig från min starke fiende, från mina ovänner, ty de voro mig övermäktiga.
\par 19 De överföllo mig på min olyckas dag, men HERREN blev mitt stöd.
\par 20 Han förde mig ut på rymlig plats han räddade mig, ty han hade behag till mig.
\par 21 HERREN lönar mig efter min rättfärdighet; efter mina händers renhet vedergäller han mig.
\par 22 Ty jag höll mig på HERRENS vägar och avföll icke från min Gud i ogudaktighet;
\par 23 nej, alla hans rätter hade jag för ögonen, och från hans stadgar vek jag icke av.
\par 24 Så var jag ostrafflig för honom och tog mig till vara för missgärning.
\par 25 Därför vedergällde mig HERREN efter min rättfärdighet, efter min renhet inför hans ögon.
\par 26 Mot den fromme bevisar du dig from, mot en ostrafflig hjälte bevisar du dig ostrafflig.
\par 27 Mot den rene bevisar du dig ren, men mot den vrånge bevisar du dig avog.
\par 28 och du frälsar ett betryckt folk, men dina ögon äro emot de stolta, till att ödmjuka dem.
\par 29 Ja, du, HERRE, är min lampa; ty HERREN gör mitt mörker ljust.
\par 30 Ja, med dig kan jag nedslå härskaror, med min Gud stormar jag murar.
\par 31 Guds väg är ostrafflig, HERRENS tal är luttrat. En sköld är han för alla som taga sin tillflykt till honom.
\par 32 Ty vem är Gud förutom HERREN, och vem är en klippa förutom vår Gud?
\par 33 Gud, du som var mitt starka värn och ledde den ostrafflige på hans väg,
\par 34 du som gjorde hans fötter såsom hindens och ställde mig på mina höjder,
\par 35 du som lärde mina händer att strida och mina armar att spänna kopparbågen!
\par 36 Du gav mig din frälsnings sköld och din bönhörelse gjorde mig stor,
\par 37 du skaffade rum för mina steg, där jag gick, och mina fötter vacklade icke.
\par 38 Jag förföljde mina fiender och förgjorde dem; jag vände icke tillbaka, förrän jag hade gjort ände på dem.
\par 39 Ja, jag gjorde ände på dem och slog dem, så att de icke mer reste sig; de föllo under mina fötter.
\par 40 Du omgjordade mig med kraft till striden, du böjde mina motståndare under mig.
\par 41 Mina fiender drev du på flykten för mig, dem som hatade mig förgjorde jag.
\par 42 De sågo sig omkring, men det fanns ingen som frälste; efter HERREN, men han svarade dem icke.
\par 43 Och jag stötte dem sönder till stoft på jorden, jag krossade och förtrampade dem såsom orenlighet på gatan.
\par 44 Du räddade mig ur mitt folks strider, du bevarade mig till ett huvud över hedningar; folkslag som jag ej kände blevo mina tjänare.
\par 45 Främlingar visade mig underdånighet; vid blotta ryktet hörsammade de mig.
\par 46 Ja, främlingarnas mod vissnade bort; de omgjordade sig och övergåvo sina borgar.
\par 47 HERREN lever! Lovad vare min klippa, upphöjd vare Gud, min frälsnings klippa!
\par 48 Gud, som har givit mig hämnd och lagt folken under mig;
\par 49 du som har fört mig ut från mina fiender och upphöjt mig över mina motståndare, räddat mig från våldets man!
\par 50 Fördenskull vill jag tacka dig, HERRE, bland hedningarna, och lovsjunga ditt namn.
\par 51 Ty du giver din konung stor seger och gör nåd mot din smorde, mot David och hans säd till evig tid.

\chapter{23}

\par 1 Dessa voro Davids sista ord: Så säger David, Isais son, så säger den man som blev högt upphöjd, Jakobs Guds smorde, Israels ljuvlige sångare:
\par 2 HERRENS Ande har talat genom mig, och hans ord är på min tunga;
\par 3 Israels Gud har så sagt, Israels klippa har så talat till mig: "Den som råder över människorna rätt, den som råder i Guds fruktan,
\par 4 han är lik morgonens ljus, när solen går upp, en morgon utan moln, då jorden grönskar genom solsken efter regn.
\par 5 Ja, är det icke så med mitt hus inför Gud? Han har ju upprättat med mig ett evigt förbund, i allo stadgat och betryggat. Ja, visst skall han låta all frälsning och glädje växa upp åt mig.
\par 6 Men de onda äro allasammans lika bortkastade törnen, som man ej vill taga i med handen.
\par 7 Och måste man röra vid dem, så rustar man sig med järn och med spjutskaft, och bränner sedan upp dem i eld på stället.
\par 8 Dessa äro namnen på Davids hjältar: Joseb-Bassebet, en takemonit, den förnämste bland kämparna, han som svängde sitt spjut över åtta hundra som hade blivit slagna på en gång.
\par 9 Och näst honom kom Eleasar, son till Dodi, son till en ahoait. Han var en av de tre hjältar som voro med David, när de blevo smädade av filistéerna, som där hade församlat sig till strid; Israels män drogo sig då tillbaka.
\par 10 Men han höll stånd och högg in på filistéerna, till dess att hans hand blev så trött att den var såsom faststelnad vid svärdet; och HERREN beredde så en stor seger på den dagen. Sedan hade folket allenast att vända om och följa med honom för att plundra.
\par 11 Och efter honom kom Samma, son till Age, en hararit. En gång hade filistéerna församlat sig, så att de utgjorde en hel skara. Och där var ett åkerstycke, fullt med linsärter. Och folket flydde för filistéerna.
\par 12 Då ställde han sig mitt på åkerstycket och försvarade det och slog filistéerna; och HERREN beredde så en stor seger.
\par 13 En gång drogo tre av de trettio förnämsta männen ned och kommo vid skördetiden till David vid Adullams grotta, medan en skara filistéer var lägrad i Refaimsdalen.
\par 14 Men David var då på borgen, under det att en filisteisk utpost fanns i Bet-Lehem.
\par 15 Och David greps av lystnad och sade: "Ack att någon ville giva mig vatten att dricka från brunnen vid Bet-Lehems stadsport!"
\par 16 Då bröto de tre hjältarna sig igenom filistéernas läger och hämtade vatten ur brunnen vid Bet-Lehems stadsport och togo det och buro det till David. Men han ville icke dricka det, utan göt ut det såsom ett drickoffer åt HERREN.
\par 17 Han sade nämligen: "Bort det, HERRE, att jag skulle göra detta! Skulle jag dricka de mäns blod, som gingo åstad med fara för sina liv?" Och han ville icke dricka det. Sådana ting hade de tre hjältarna gjort.
\par 18 Abisai, broder till Joab, Serujas son, var den förnämste av tre andra; han svängde en gång sitt spjut över tre hundra som hade blivit slagna. Och han hade ett stort namn bland de tre.
\par 19 Han var visserligen mer ansedd än någon annan i detta tretal, och han var de andras hövitsman, men upp till de tre första kom han dock icke.
\par 20 Och Benaja, son till Jojada, som var son till en tapper, segerrik man från Kabseel; han slog ned de två Arielerna i Moab, och det var han som en snövädersdag steg ned och slog ihjäl lejonet i brunnen.
\par 21 Han slog ock ned den egyptiske mannen som var så ansenlig att skåda. Fastän egyptiern hade ett spjut i handen, gick han ned mot honom, väpnad allenast med sin stav. Och han ryckte spjutet ur egyptierns hand och dräpte honom med hans eget spjut.
\par 22 Sådana ting hade Benaja, Jojadas son, gjort. Och han hade ett stort namn bland de tre hjältarna.
\par 23 Han var mer ansedd än någon av de trettio, men upp till de tre första kom han icke. Och David insatte honom i sin livvakt.
\par 24 Till de trettio hörde: Asael, Joabs broder; Elhanan, Dodos son, från Bet-Lehem;
\par 25 haroditen Samma; haroditen Elika;
\par 26 peletiten Heles; tekoaiten Ira, Ickes' son;
\par 27 anatotiten Abieser; husatiten Mebunnai;
\par 28 ahoaiten Salmon; netofatiten Maherai;
\par 29 netofatiten Heleb, Baanas son; Ittai, Ribais son, från Gibea i Benjamins barns stam;
\par 30 Benaja, en pirgatonit; Hiddai från Gaas' dalar;
\par 31 arabatiten Abi-Albon; barhumiten Asmavet;
\par 32 saalboniten Eljaba; Bene-Jasen; Jonatan;
\par 33 harariten Samma; arariten Ahiam, Sarars son;
\par 34 Elifelet, son till Ahasbai, maakatitens son; giloniten Eliam, Ahitofels son;
\par 35 Hesro från Karmel; arabiten Paarai;
\par 36 Jigeal, Natans son, från Soba; gaditen Bani;
\par 37 ammoniten Selek; beerotiten Naharai, vapendragare åt Joab, Serujas son;
\par 38 jeteriten Ira; jeteriten Gareb;
\par 39 hetiten Uria. Tillsammans utgjorde de trettiosju.

\chapter{24}

\par 1 Men HERRENS vrede upptändes åter mot Israel, så att han uppeggade David mot dem och sade: "Gå åstad och räkna Israel och Juda."
\par 2 Då sade konungen till Joab, hövitsmannen för hans här: "Far igenom alla Israels stammar, från Dan ända till Beer-Seba, och anställen en folkräkning, så att jag får veta huru stor folkmängden är."
\par 3 Joab svarade konungen: "Må HERREN, din Gud, föröka detta folk hundrafalt, huru talrikt det än är, och må min herre konungen få se detta med egna ögon. Men varför har min herre konungen fått lust till sådant?"
\par 4 Likväl blev konungens befallning gällande, trots Joab och härens andra hövitsmän; alltså drog Joab jämte härens andra hövitsmän ut i konungens tjänst för att anställa folkräkning i Israel.
\par 5 Och de gingo över Jordan och lägrade sig vid Aroer, på högra sidan om staden i Gads dal, och åt Jaeser till.
\par 6 Därifrån kommo de till Gilead och Tatim-Hodsis land; sedan kommo de till Dan-Jaan och så runt omkring till Sidon.
\par 7 Därefter kommo de till Tyrus' befästningar och till hivéernas och kananéernas alla städer; slutligen drogo de till Beer-Seba i Juda sydland.
\par 8 Och sedan de så hade farit igenom hela landet, kommo de efter nio månader och tjugu dagar hem till Jerusalem.
\par 9 Och Joab uppgav för konungen vilken slutsumma folkräkningen utvisade: i Israel funnos åtta hundra tusen stridbara, svärdbeväpnade män, och Juda män voro fem hundra tusen.
\par 10 Men Davids samvete slog honom, sedan han hade låtit räkna folket, och David sade till HERREN: "Jag har syndat storligen i vad jag har gjort; men tillgiv nu, HERRE, din tjänares missgärning, ty jag har handlat mycket dåraktigt."
\par 11 Då nu David stod upp om morgonen, hade HERRENS ord kommit till profeten Gad, Davids siare; han hade sagt:
\par 12 "Gå och tala till David: Så säger HERREN: Tre ting förelägger jag dig; välj bland dem ut åt dig ett som du vill att jag skall göra dig."
\par 13 Då gick Gad in till David och förkunnade detta för honom. Han sade till honom: "Vill du att hungersnöd under sju år skall komma i ditt land? Eller att du i tre månader skall nödgas fly för dina ovänner, medan de förfölja dig? Eller att pest i tre dagar skall hemsöka ditt land? Betänk nu och eftersinna vilket svar jag skall giva honom som har sänt mig."
\par 14 David svarade Gad: "Jag är i stor vånda. Men låt oss då falla i HERRENS hand, ty hans barmhärtighet är stor; i människohand vill jag icke falla."
\par 15 Så lät då HERREN pest komma i Israel, från morgonen intill den bestämda tiden; därunder dogo av folket, ifrån Dan ända till Beer-Seba, sjuttio tusen män.
\par 16 Men när ängeln räckte ut sin hand över Jerusalem för att fördärva det, ångrade HERREN det onda, och han sade till ängeln, folkets fördärvare: "Det är nog; drag nu din hand tillbaka." Och HERRENS ängel var då vid jebuséen Araunas tröskplats.
\par 17 Men när David fick se ängeln som slog folket, sade han till HERREN så: "Det är ju jag som har syndat, det är jag som har gjort illa; men dessa, min hjord, vad hava de gjort? Må din hand vända sig mot mig och min faders hus."
\par 18 Och Gad kom till David samma dag och sade till honom: "Gå åstad och res ett altare åt HERREN på jebuséen Araunas tröskplats."
\par 19 Och David gick åstad efter Gads ord, såsom HERREN hade bjudit.
\par 20 När Arauna nu blickade ut och fick se att konungen och hans tjänare kommo till honom, gick han ut och föll ned till jorden på sitt ansikte för konungen.
\par 21 Och Arauna sade: "Varför kommer min herre konungen till sin tjänare?" David svarade: "För att köpa tröskplatsen av dig och där bygga ett altare åt HERREN; och må så hemsökelsen upphöra bland folket."
\par 22 Då sade Arauna till David: "Min herre konungen tage till sitt offer vad honom täckes. Se här äro fäkreaturen till brännoffer, och här äro tröskvagnarna, jämte fäkreaturens ok, till ved.
\par 23 Alltsammans, o konung, giver Arauna åt konungen." Och Arauna sade ytterligare till konungen: "Må HERREN, din Gud, vara dig nådig."
\par 24 Men konungen svarade Arauna: "Nej, jag vill köpa det av dig för ett bestämt pris; ty jag vill icke offra åt HERREN, min Gud, brännoffer som jag har fått för intet." Och David köpte tröskplatsen och fäkreaturen för femtio siklar silver.
\par 25 Och David byggde där ett altare åt HERREN och offrade brännoffer och tackoffer. Och HERREN lyssnade till landets bön, och hemsökelsen upphörde bland Israel.


\end{document}