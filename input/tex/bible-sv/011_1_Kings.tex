\begin{document}

\title{1 Kings}

1Ki 1:1  Konung David var nu gammal och kommen till hög ålder; och ehuru man höljde täcken över honom, kunde han dock icke hålla sig varm.
1Ki 1:2  Då sade hans tjänare till honom: "Må man för min herre konungens räkning söka upp en ung kvinna, en jungfru, som kan bliva konungens tjänarinna och sköta honom. Om hon får ligga i din famn, så bliver min herre konungen varm"
1Ki 1:3  Så sökte de då över hela Israels land efter en skön flicka; och de funno Abisag från Sunem och förde henne till konungen.
1Ki 1:4  Hon var en mycket skön flicka, och hon skötte nu konungen och betjänade honom, men konungen hade intet umgänge med henne.
1Ki 1:5  Men Adonia, Haggits son, hov sig upp och sade: "Det är jag som skall bliva konung." Och han skaffade sig vagnar och ryttare, därtill ock femtio man som löpte framför honom.
1Ki 1:6  Hans fader hade aldrig velat bedröva honom med att säga: "Varför gör du så?" Han var ock mycket fager; och hans moder hade fött honom näst efter Absalom.
1Ki 1:7  Och han begynte underhandla med Joab, Serujas son, och med prästens Ebjatar, och dessa slöto sig till Adonia och understödde honom.
1Ki 1:8  Men prästen Sadok och Benaja, Jojadas son, samt profeten Natan, Simei, Rei och Davids hjältar höllo icke med Adonia.
1Ki 1:9  Och Adonia slaktade får och fäkreatur och gödkalvar vid Soheletstenen, som ligger vid Rogelskällan; och han inbjöd dit alla sina bröder, konungens söner, och alla de Juda män som voro i konungens tjänst.
1Ki 1:10  Men profeten Natan, Benaja, hjältarna och sin broder Salomo inbjöd han icke.
1Ki 1:11  Då sade Natan så till Bat-Seba, Salomos moder: "Du har väl hört att Adonia, Haggits son, har blivit konung, utan att vår herre David vet därom?
1Ki 1:12  Men jag vill nu giva dig ett råd, för att du må kunna rädda ditt liv och din son Salomos liv.
1Ki 1:13  Gå in till konung David och säg till honom: 'Har du icke, min herre konung, själv med ed lovat din tjänarinna och sagt: Din son Salomo skall bliva konung efter mig; han skall sitta på min tron? Varför har då Adonia blivit konung?'
1Ki 1:14  Och medan du ännu är där och talar med konungen, skall jag efter dig komma in och bekräfta dina ord."
1Ki 1:15  Så gick då Bat-Seba in till konungen, i kammaren. Konungen var nu mycket gammal; och Abisag från Sunem betjänade konungen.
1Ki 1:16  Och Bat-Seba bugade sig och föll ned för konungen. Då frågade konungen: "Vad önskar du?"
1Ki 1:17  Hon sade till honom: "Min herre, du har ju själv lovat din tjänarinna med en ed vid HERREN, din Gud: 'Din son Salomo skall bliva konung efter mig; han skall sitta på min tron.'
1Ki 1:18  Men se, nu har Adonia blivit konung, fastän du, min herre konung, ännu icke har fått veta det.
1Ki 1:19  Och han har slaktat tjurar och gödkalvar och får i myckenhet, och han har inbjudit alla konungens söner och prästen Ebjatar och härhövitsmannen Joab; men din tjänare Salomo har han icke inbjudit.
1Ki 1:20  På dig, min herre konung, äro nu hela Israels ögon riktade, i förväntan att du skall kungöra för dem vem som skall sitta på min herre konungens tron efter honom.
1Ki 1:21  Eljest torde hända, att när min herre konungen har gått till vila hos sina fäder, då bliva jag och min son Salomo hållna såsom brottslingar."
1Ki 1:22  Medan hon ännu höll på att tala med konungen, kom profeten Natan.
1Ki 1:23  Och man anmälde det för konungen och sade: "Profeten Natan är här." När han så kom inför konungen, föll han ned till jorden på sitt ansikte för konungen.
1Ki 1:24  Och Natan sade: "Min herre konung, är det väl du som har sagt att Adonia skall bliva konung efter dig, och att han skall sitta på din tron?
1Ki 1:25  Ty han har i dag gått ned och slaktat tjurar och gödkalvar och får i myckenhet, och har inbjudit alla konungens söner och härhövitsmännen och prästen Ebjatar, och de hålla nu på med att äta och dricka hos honom; och de ropa: 'Leve konung Adonia!'
1Ki 1:26  Men mig, din tjänare, och prästen Sadok och Benaja, Jojadas son, och din tjänare Salomo har han icke inbjudit.
1Ki 1:27  Kan väl detta hava utgått från min herre konungen, utan att du har låtit dina tjänare vet vem som skall sitta på min herre konungens tron efter honom?"
1Ki 1:28  Då svarade konung David och sade: "Kallen hit till mig Bat-Seba." När hon nu kom inför konungen och stod inför konungen,
1Ki 1:29  betygade konungen med ed och sade: "Så sant HERREN lever, han som har förlossat mig från all nöd:
1Ki 1:30  såsom jag lovade dig med ed vid HERREN, Israels Gud, då jag sade: 'Din son Salomo skall bliva konung efter mig; han skall sitta på min tron i mitt ställe', så vill jag denna dag göra."
1Ki 1:31  Då bugade sig Bat-Seba, med ansiktet mot jorden, och föll ned för konungen och sade: "Må min herre, konung David, leva evinnerligen!"
1Ki 1:32  Och konung David sade: "Kallen till mig prästen Sadok och profeten Natan och Benaja, Jojadas son. När dessa kommo inför konungen,
1Ki 1:33  sade konungen till dem: "Tagen eder herres tjänare med eder och sätten min son Salomo på min egen mulåsna och fören honom med till Gihon.
1Ki 1:34  Där må prästen Sadok och profeten Natan smörja honom till konung över Israel; sedan skolen I stöta i basun och ropa: 'Leve konung Salomo!'
1Ki 1:35  Därefter skolen I följa honom hitupp, och när han kommer hit, skall han sätta sig på min tron, och så skall han vara konung i mitt ställe. Ty det är honom jag har förordnat att vara furste över Israel och Juda."
1Ki 1:36  Då svarade Benaja, Jojadas son, konungen och sade: "Amen. Så bjude ock HERREN, min herre konungens Gud.
1Ki 1:37  Såsom HERREN har varit med min herre konungen, så vare han ock med Salomo. Ja, må han göra hans tron ännu mäktigare än min herres, konung Davids, tron."
1Ki 1:38  Så gingo nu prästen Sadok och profeten Natan och Benaja, Jojadas son, ditned, jämte keretéerna och peletéerna, och satte Salomo på konung Davids mulåsna och förde honom till Gihon.
1Ki 1:39  Och prästen Sadok tog oljehornet ur tältet och smorde Salomo. Därefter stötte de i basun, och allt folket ropade: "Leve konung Salomo!"
1Ki 1:40  Sedan följde allt folket honom upp, under det att de blåste på flöjter och visade sin glädje med ett så stort jubel, att jorden kunde rämna av deras rop.
1Ki 1:41  Men Adonia och alla de inbjudna som han hade hos sig hörde detta, just då de hade slutat att äta. När Joab nu hörde basunljudet, sade han: "Varför höres detta larm från staden?"
1Ki 1:42  Medan han ännu talade, kom Jonatan, prästen Ebjatars son; och Adonia sade: "Kom hit, ty du är en rättskaffens man och har nog ett gott glädjebudskap att förkunna."
1Ki 1:43  Jonatan svarade och sade till Adonia: "Nej, vår herre, konung David, har gjort Salomo till konung.
1Ki 1:44  Och konungen har med honom sänt åstad prästen Sadok och profeten Natan och Benaja, Jojadas son, jämte keretéerna och peletéerna, och de hava satt honom på konungens mulåsna.
1Ki 1:45  Därefter hava prästen Sadok och profeten Natan i Gihon smort honom till konung, och sedan hava de dragit upp därifrån under jubel, och hela staden har kommit i rörelse. Härav kommer det buller som I haven hört.
1Ki 1:46  Salomo sitter nu ock på konungatronen.
1Ki 1:47  Vidare hava konungens tjänare kommit och lyckönskat vår herre konung David, och sagt: 'Din Gud låte Salomos namn bliva ännu större än ditt namn, och hans tron ännu mäktigare än din tron.' Och konungen har tillbett, nedböjd på sin säng;
1Ki 1:48  ja, konungen har sagt så: 'Lovas vare HERREN, Israels Gud, som i dag har satt en efterträdare på min tron, så att jag med egna ögon har fått se det!"
1Ki 1:49  Då blevo alla de inbjudna som voro hos Adonia förskräckta och stodo upp och gingo bort, var och en sin väg.
1Ki 1:50  Men Adonia fruktade så för Salomo, att han stod upp och gick bort och fattade i hornen på altaret.
1Ki 1:51  Och det blev berättat för Salomo: "Se, Adonia fruktar för konung Salomo; därför har han fattat i hornen på altaret och sagt: 'Konung Salomo måste lova mig i dag med ed att han icke skall döda sin tjänare med svärd.'"
1Ki 1:52  Då sade Salomo: "Om han vill vara en rättskaffens man, så skall icke ett hår av hans huvud falla till jorden; men om något ont bliver funnet hos honom, så skall han dö."
1Ki 1:53  Därefter sände konung Salomo åstad och lät hämta honom från altaret; och han kom och föll ned för konung Salomo. Då sade Salomo till honom: "Gå hem till ditt."
1Ki 2:1  Då nu tiden tillstundade att David skulle dö, bjöd han sin son Salomo och sade:
1Ki 2:2  "Jag går nu all världens väg; så var då frimodig och visa dig såsom en man.
1Ki 2:3  Och håll vad HERREN, din Gud, bjuder dig hålla, så att du vandrar på hans vägar och håller hans stadgar, hans bud och rätter och vittnesbörd, såsom det är skrivet i Moses lag, på det att du må hava framgång i allt vad du gör, och överallt dit du vänder dig;
1Ki 2:4  så att HERREN får uppfylla det ord som han talade om mig, då han sade: 'Om dina barn hava akt på sin väg, så att de vandra inför mig i trohet och av allt sitt hjärta och av all sin själ, då' - sade han - 'skall på Israels tron aldrig saknas en avkomling av dig.'
1Ki 2:5  Vidare: du vet väl vad Joab, Serujas son, har gjort mot mig, huru han gjorde mot de två härhövitsmännen i Israel, Abner, Ners son, och Amasa, Jeters son, huru han dräpte dem, så att han i fredstid utgöt blod, likasom hade det varit krig, och, likasom hade det varit krig, lät blod komma på bältet som han hade omkring sina länder, och på skorna som han hade på sina fötter.
1Ki 2:6  Så gör nu efter din vishet, och låt icke hans grå hår få med frid fara ned i dödsriket.
1Ki 2:7  Men mot gileaditen Barsillais söner skall du bevisa godhet, så att de få vara med bland dem som äta vid ditt bord; ty på sådant sätt bemötte de mig, när jag flydde för din broder Absalom.
1Ki 2:8  Vidare har du hos dig Simei, Geras son, benjaminiten från Bahurim, som for ut mot mig i gruvliga förbannelser på den dag då jag gick till Mahanaim, men som sedan kom ned till Jordan mig till mötes, varvid jag med en ed vid HERREN lovade honom och sade: 'Jag skall icke döda dig med svärd.'
1Ki 2:9  Men nu må du icke låta honom bliva ostraffad, ty du är en vis man och vet väl vad du bör göra med honom, så att du låter hans grå hår med blod fara ned i dödsriket."
1Ki 2:10  och David gick till vila hos sina fäder och blev begraven i Davids stad.
1Ki 2:11  Den tid David regerade över Israel var fyrtio år; i Hebron regerade han i sju år, och i Jerusalem regerade han i trettiotre år.
1Ki 2:12  Och Salomo satte sig på sin fader Davids tron, och han konungamakt blev starkt befäst.
1Ki 2:13  Men Adonia, Haggits son, kom till Bat-Seba, Salomos moder. Hon frågade då: "Har du gott att meddela?" Han svarade: "Ja."
1Ki 2:14  Därefter sade han: "Jag har något att tala med dig om." Hon svarade: "Tala."
1Ki 2:15  Då sade han: "Du vet själv att konungadömet tillhörde mig, och att hela Israel fäste sina blickar på mig, i förväntan att jag skulle bliva konung. Men så gick konungadömet ifrån mig och blev min broders; genom HERRENS skickelse blev det hans.
1Ki 2:16  Nu har jag en enda bön till dig. Visa icke bort mig." Hon svarade honom: "Tala."
1Ki 2:17  Då sade han: "Säg till konung Salomo - dig visar han ju icke bort - att han giver mig Abisag från Sunem till hustru."
1Ki 2:18  Bat-Seba svarade: "Gott! Jag skall själv tala med konungen om dig."
1Ki 2:19  Så gick då Bat-Seba in till konung Salomo för att tala med honom om Adonia. Då stod konungen upp och gick emot henne och bugade sig för henne och satte sig därefter på sin stol; man ställde ock fram en stol åt konungens moder, och hon satte sig på hans högra sida.
1Ki 2:20  Därefter sade hon: "Jag har en enda liten bön till dig. Visa icke bort mig." Konungen svarade henne: "Framställ din bön, min moder; jag vill ingalunda visa bort dig."
1Ki 2:21  Då sade hon: "Låt giva Abisag från Sunem åt din broder Adonia till hustru."
1Ki 2:22  Men konung Salomo svarade och sade till sin moder: "Varför begär du endast Abisag från Sunem åt Adonia? Du kunde lika gärna begära konungadömet åt honom - han är ju min äldste broder - ja, åt honom och åt prästen Ebjatar och åt Joab, Serujas son."
1Ki 2:23  Och konung Salomo betygade med ed vid HERREN och sade: "Gud straffe mig nu och framgent, om icke Adonia med sitt liv skall få umgälla att han har talat detta.
1Ki 2:24  Och nu, så sant HERREN lever, han som har utsett mig och uppsatt mig på min fader Davids tron, och som, enligt sitt löfte, har uppbyggt åt mig ett hus: i dag skall Adonia dödas."
1Ki 2:25  Därefter sände konung Salomo åstad och lät utföra detta genom Benaja, Jojadas son; denne stötte ned honom, så att han dog.
1Ki 2:26  Och till prästen Ebjatar sade konungen: "Gå bort till ditt jordagods i Anatot, ty du har förtjänat döden; men i dag vill jag icke döda dig, eftersom du har burit Herrens, HERRENS ark framför min fader David, och eftersom du med min fader har lidit allt vad han har fått lida.
1Ki 2:27  Så drev Salomo bort Ebjatar och lät honom icke längre vara HERRENS präst, för att HERRENS ord skulle uppfyllas, det som han hade talat i Silo över Elis hus.
1Ki 2:28  Då nu ryktet härom kom till Joab - som ju hade slutit sig till Adonia, om han ock icke hade slutit sig till Absalom - flydde han till HERRENS tält och fattade i hornen på altaret.
1Ki 2:29  Men när det blev berättat för konung Salomo att Joab hade flytt till HERRENS tält, och att han stod invid altaret, sände Salomo åstad Benaja, Jojadas son, och sade: "Gå och stöt ned honom."
1Ki 2:30  När Benaja så kom till HERRENS tält, sade han till honom: "Så säger konungen: Gå bort härifrån." Men han svarade: "Nej; här vill jag dö." När Benaja framförde detta till konungen och sade: "Så och så har Joab sagt, så har han svarat mig",
1Ki 2:31  sade konungen till honom: "Gör såsom han har sagt, stöt ned honom och begrav honom, så att du befriar mig och min faders hus från skulden för det blod som Joab utan sak har utgjutit.
1Ki 2:32  Och må HERREN låta hans blod komma tillbaka över hans eget huvud, därför att han stötte ned två män som voro rättfärdigare och bättre än han själv, och dräpte dem med svärd, utan att min fader David visste det, nämligen Abner, Ners son, härhövitsmannen i Israel, och Amasa, Jeters son, härhövitsmannen i Juda.
1Ki 2:33  Ja, deras blod skall komma tillbaka över Joabs och hans efterkommandes huvud för evigt. Men åt David och hans efterkommande, hans hus och hans tron skall HERREN giva frid till evig tid.
1Ki 2:34  Så gick då Benaja, Jojadas son, ditupp och stötte ned honom och dödade honom; och han blev begraven där han bodde i öknen.
1Ki 2:35  Och konungen satte Benaja, Jojadas son, i hans ställe över hären; och prästen Sadok hade konungen satt i Ebjatars ställe.
1Ki 2:36  Därefter sände konungen och lät kalla till sig Simei och sade till honom: "Bygg dig ett hus i Jerusalem och bo där, och därifrån får du icke gå ut, varken hit eller dit.
1Ki 2:37  Ty det må du veta, att på den dag du går ut och går över bäcken Kidron skall du döden dö. Ditt blod kommer då över ditt eget huvud."
1Ki 2:38  Simei sade till konungen: "Vad du har talat är gott; såsom min herre konungen har sagt, så skall din tjänare göra." Och Simei bodde i Jerusalem en lång tid.
1Ki 2:39  Men tre år därefter hände sig att två tjänare flydde ifrån Simei till Akis, Maakas son, konungen i Gat. Och man berättade för Simei och sade: "Dina tjänare äro i Gat."
1Ki 2:40  Då stod Simei upp och sadlade sin åsna och begav sig till Akis i Gat för att söka efter sina tjänare. Simei begav sig alltså åstad och hämtade sina tjänare från Gat.
1Ki 2:41  Men när det blev berättat för Salomo att Simei hade begivit sig från Jerusalem till Gat och kommit tillbaka,
1Ki 2:42  sände konungen och lät kalla till sig Simei och sade till honom: "Har jag icke bundit dig med ed vid HERREN och varnat dig och sagt till dig: 'Det må du veta, att på den dag du går ut och begiver dig hit eller dit skall du döden dö'? Och du svarade mig: 'Vad du har sagt är gott, och jag har hört det.'
1Ki 2:43  Varför har du då icke aktat på din ed vid HERREN och på det bud som jag har givit dig?"
1Ki 2:44  Och konungen sade ytterligare till Simei: "Du känner själv allt det onda som ditt hjärta vet med sig att du har gjort min fader David. HERREN skall nu låta din ondska komma tillbaka över ditt eget huvud.
1Ki 2:45  Men konung Salomo skall bliva välsignad, och Davids tron skall bliva befäst inför HERREN till evig tid."
1Ki 2:46  På konungens befallning gick därefter Benaja, Jojadas son, fram och stötte ned honom, så att han dog. Och konungadömet blev befäst i Salomos hand.
1Ki 3:1  Och Salomo befryndade sig med Farao, konungen i Egypten; han tog Faraos dotter till hustru och förde henne in i Davids stad, och där fick hon bo, till dess han hade byggt sitt hus färdigt, så ock HERRENS hus och muren runt omkring Jerusalem.
1Ki 3:2  Emellertid offrade folket på höjderna, eftersom ännu vid denna tid intet hus hade blivit byggt åt HERRENS namn.
1Ki 3:3  Och Salomo älskade HERREN och vandrade efter sin fader Davids stadgar, utom att han frambar offer på höjderna och tände offereld där.
1Ki 3:4  Och konungen begav sig till Gibeon för att offra där, ty detta var den förnämsta offerhöjden; tusen brännoffer offrade Salomo på altaret där.
1Ki 3:5  I Gibeon uppenbarade sig nu HERREN för Salomo i en dröm om natten; Gud sade: "Bed mig om vad du vill att jag skall giva dig."
1Ki 3:6  Salomo svarade: "Du har gjort stor nåd med din tjänare, min fader David, eftersom han vandrade inför dig i trohet, rättfärdighet och rättsinnighet mot dig. Och du bevarade åt honom denna stora nåd och gav honom en son till efterträdare på hans tron, såsom ju nu har skett.
1Ki 3:7  Ja, nu har du, HERRE, min Gud, gjort din tjänare till konung efter min fader David; men jag är en helt ung man, som icke rätt förstår att vara ledare och anförare.
1Ki 3:8  Och din tjänare är här bland ditt folk, det som du har utvalt, ett folk som är så talrikt att det icke kan räknas eller täljas för sin myckenhets skull.
1Ki 3:9  Så giv nu din tjänare ett hörsamt hjärta, så att han kan vara domare för ditt folk och skilja mellan gott och ont; ty vem förmår väl eljest att vara domare för detta ditt stora folk?"
1Ki 3:10  Detta, att Salomo bad om sådant, täcktes Herren.
1Ki 3:11  Och Gud sade till honom: "Eftersom du har bett om sådant och icke bett om ett långt liv, ej heller bett om rikedom eller bett om dina fienders liv, utan har bett om att få förstånd till att akta på vad rätt är,
1Ki 3:12  se, därför vill jag göra såsom du önskar; se, jag giver dig ett så vist och förståndigt hjärta, att din like icke har funnits före dig, och att din like ej heller skall uppstå efter dig.
1Ki 3:13  Därtill giver jag dig ock vad du icke har bett om, nämligen både rikedom och ära, så att i all din tid ingen konung skall vara din like.
1Ki 3:14  Och om du vandrar på mina vägar, så att du håller mina stadgar och bud, såsom din fader David gjorde, då skall jag låta dig länge leva."
1Ki 3:15  Därefter vaknade Salomo och fann att det var en dröm. Och när han kom till Jerusalem, trädde han fram inför Herrens förbundsark och offrade brännoffer och frambar tackoffer; och därefter gjorde han ett gästabud för alla sina tjänare.
1Ki 3:16  Vid den tiden kommo två skökor till konungen och trädde fram inför honom.
1Ki 3:17  Och den ena kvinnan sade: "Hör mig, herre. Jag och denna kvinna bo i samma hus. Och jag födde barn där i huset hos henne.
1Ki 3:18  Sedan, på tredje dagen efter det jag hade fött mitt barn, födde ock denna kvinna ett barn. Och vi voro tillsammans, utan att någon främmande var hos oss i huset; allenast vi båda voro i huset.
1Ki 3:19  Men en natt dog denna kvinnas son, ty hon hade legat ihjäl honom.
1Ki 3:20  Då stod hon upp om natten och tog min son från min sida, under det att din tjänarinna sov, och lade honom i sin famn, men sin döde son lade hon i min famn.
1Ki 3:21  När jag då om morgonen reste mig upp för att giva min son di, fick jag se att han var död. Men när jag såg nogare på honom om morgonen, fick jag se att det icke var min son, den som jag hade fött."
1Ki 3:22  Då sade den andra kvinnan: "Det är icke så. Min son är den som lever, och din son är den som är död." Men den första svarade: "Det är icke så. Din son är den som är död, och min son är den som lever." Så tvistade de inför konungen.
1Ki 3:23  Då sade konungen: "Den ena säger: 'Denne, den som lever, är min son, och din son är den som är död.' Och den andra säger: 'Det är icke så. Din son är den som är död, och min son är den som lever.'"
1Ki 3:24  Därefter sade konungen: "Tagen hit ett svärd." Och när man hade burit svärdet fram till konungen,
1Ki 3:25  sade konungen: "Huggen det levande barnet i två delar, och given den ena hälften åt den ena och den andra hälften åt den andra."
1Ki 3:26  Men då sade den kvinna vilkens son det levande barnet var till konungen - ty hennes hjärta upprördes av kärlek till sonen - hon sade: "Hör mig, herre; given henne det levande barnet; döden det icke." Men den andra sade: "Må det vara varken mitt eller ditt; huggen det itu."
1Ki 3:27  Då tog konungen till orda och sade: "Given henne det levande barnet; döden det icke. Hon är dess moder."
1Ki 3:28  När nu hela Israel fick höra talas om den dom som konungen hade fällt, häpnade de över konungen, ty de sågo att Guds vishet var i honom till att skipa rätt.
1Ki 4:1  Konung Salomo var nu konung över hela Israel.
1Ki 4:2  Och dessa voro hans förnämsta män: Asarja, Sadoks son, var präst;
1Ki 4:3  Elihoref och Ahia, Sisas söner, voro sekreterare; Josafat, Ahiluds son, var kansler;
1Ki 4:4  Benaja, Jojadas son, var överbefälhavare; Sadok och Ebjatar voro präster;
1Ki 4:5  Asarja, Natans son, var överfogde; Sabud, Natans son, en präst, var konungens vän;
1Ki 4:6  Ahisar var överhovmästare; Adoniram, Abdas son, hade uppsikten över de allmänna arbetena.
1Ki 4:7  Och Salomo hade satt över hela Israel tolv fogdar, som skulle sörja för vad konungen och hans hus behövde; var och en hade årligen sin månad, då han skulle sörja för dessa behov.
1Ki 4:8  Och följande voro deras namn: Ben-Hur i Efraims bergsbygd;
1Ki 4:9  Ben-Deker i Makas, Saalbim, Bet-Semes, Elon, Bet-Hanan;
1Ki 4:10  Ben-Hesed i Arubbot, vilken hade Soko och hela Heferlandet;
1Ki 4:11  Ben-Abinadab i hela Nafat-Dor - denne fick Salomos dotter Tafat till hustru -;
1Ki 4:12  Baana, Ahiluds son, i Taanak och Megiddo och i hela den del av Bet-Sean, som ligger på sidan om Saretan, nedanför Jisreel, från Bet-Sean ända till Abel-Mehola och bortom Jokmeam;
1Ki 4:13  Ben-Geber i Ramot i Gilead; han hade Manasses son Jairs byar, som ligga i Gilead; han hade ock landsträckan Argob, som ligger i Basan, sextio stora städer med murar och kopparbommar;
1Ki 4:14  Ahinadab, Iddos son, i Mahanaim;
1Ki 4:15  Ahimaas i Naftali; också han hade tagit en dotter av Salomo, Basemat, till hustru;
1Ki 4:16  Baana, Husais son, i Aser och Alot;
1Ki 4:17  Josafat, Paruas son, i Isaskar;
1Ki 4:18  Simei, Elas son, i Benjamin;
1Ki 4:19  Geber, Uris son, i Gileads land, det land som hade tillhört Sihon, amoréernas konung, och Og, konungen i Basan; ty allenast en enda fogde fanns i det landet.
1Ki 4:20  Juda och Israel voro då talrika, så talrika som sanden vid havet; och man åt och drack och var glad.
1Ki 4:21  Så var nu Salomo herre över alla riken ifrån floden till filistéernas land och ända ned till Egyptens gräns; de förde skänker till Salomo och voro honom underdåniga, så länge han levde.
1Ki 4:22  Och vad Salomo för var dag behövde av livsmedel var: trettio korer fint mjöl och sextio korer vanligt mjöl,
1Ki 4:23  tio gödda oxar, tjugu valloxar och hundra far, förutom hjortar, gaseller, dovhjortar och gödda fåglar.
1Ki 4:24  Ty han rådde över hela landet på andra sidan floden, ifrån Tifsa ända till Gasa, över alla konungar på andra sidan floden; och han hade fred på alla sidor, runt omkring,
1Ki 4:25  Så att Juda och Israel sutto i trygghet, var och en under sitt vinträd och sitt fikonträd, ifrån Dan ända till Beer-Seba, så länge Salomo levde.
1Ki 4:26  Och Salomo hade fyrtio tusen spann vagnshästar och tolv tusen ridhästar.
1Ki 4:27  Och de nämnda fogdarna sörjde var sin månad för konung Salomos behov, och för allas som hade tillträde till konung Salomos bord; de läto intet fattas.
1Ki 4:28  Och kornet och halmen för hästarna och travarna förde de, var och en i sin ordning, till det ställe där han uppehöll sig.
1Ki 4:29  Och Gud gav Salomo vishet och förstånd i mycket rikt mått och så mycken insikt, att den kunde liknas vid sanden på havets strand,
1Ki 4:30  så att Salomos vishet var större än alla österlänningars vishet och all Egyptens vishet.
1Ki 4:31  Han var visare än alla andra människor, visare än esraiten Etan och Heman och Kalkol och Darda, Mahols söner; och ryktet om honom gick ut bland alla folk runt omkring.
1Ki 4:32  Han diktade tre tusen ordspråk, och hans sånger voro ett tusen fem.
1Ki 4:33  Han talade om träden, från cedern på Libanon ända till isopen, som växer fram ur väggen. Han talade ock om fyrfotadjuren, om fåglarna, om kräldjuren och om fiskarna.
1Ki 4:34  Och från alla folk kom man för att höra Salomos visdom, från alla konungar på jorden, som hade hört talas om hans visdom.
1Ki 5:1  Och Hiram, konungen i Tyrus, sände sina tjänare till Salomo, sedan han hade fått höra att denne hade blivit smord till konung efter sin fader; ty Hiram hade alltid varit Davids vän.
1Ki 5:2  Och Salomo sände till Hiram och lät säga:
1Ki 5:3  "Du vet själv att min fader David icke kunde bygga något hus åt HERRENS, sin Guds, namn, för de krigs skull med vilka fienderna runt omkring ansatte honom, till dess att HERREN lade dem under hans fötter
1Ki 5:4  Men nu har HERREN, min Gud, låtit mig få ro på alla sidor; ingen motståndare finnes, och ingen olycka är på färde.
1Ki 5:5  Därför tänker jag nu på att bygga ett hus åt HERRENS, min Guds, namn, såsom HERREN talade till min fader David, i det han sade: 'Din son, den som jag skall sätta på din tron efter dig, han skall bygga huset åt mitt namn.'
1Ki 5:6  Så bjud nu att man hugger åt mig cedrar på Libanon. Härvid skola mina tjänare vara dina tjänare behjälpliga; och jag vill giva dig betalning för dina tjänares arbete, alldeles såsom du själv begär. Ty du vet själv att bland oss icke finnes någon som är så skicklig att hugga virke som sidonierna."
1Ki 5:7  Då nu Hiram hörde Salomos ord, blev han mycket glad; och han sade: "Lovad vare HERREN i dag, han som har givit David en så vis son till att regera över detta talrika folk!"
1Ki 5:8  Och Hiram sände till Salomo och lät säga: "Jag har hört det budskap du har sänt till mig. Jag vill göra allt vad du begär i fråga om cederträ och cypressträ.
1Ki 5:9  Mina tjänare skola föra virket från Libanon ned till havet, och jag skall låta lägga det i flottar på havet och föra det till det ställe som du anvisar mig, och lossa det där; men du må själv avhämta det. Du åter skall göra vad jag begär, nämligen förse mitt hus med livsmedel."
1Ki 5:10  Så gav då Hirom åt Salomo cederträ och cypressträ, så mycket han begärde.
1Ki 5:11  Men Salomo gav åt Hiram tjugu tusen korer vete, till föda för hans hus, och tjugu korer olja av stötta oliver. Detta gav Salomo åt Hiram för vart år.
1Ki 5:12  Och HERREN hade givit Salomo vishet, såsom han hade lovat honom. Och vänskap rådde mellan Hiram och Salomo; och de slöto förbund med varandra.
1Ki 5:13  Och konung Salomo bådade upp arbetsfolk ur hela Israel, och arbetsfolket utgjorde trettio tusen man.
1Ki 5:14  Dessa sände han till Libanon, tio tusen i vår månad, skiftevis, så att de voro en månad på Libanon och två månader hemma; och Adoniram hade uppsikten över de allmänna arbetena.
1Ki 5:15  Och Salomo hade sjuttio tusen män som buro bördor, och åttio tusen som höggo sten i bergen,
1Ki 5:16  förutom de överfogdar som av Salomo voro anställda över arbetet, tre tusen tre hundra, vilka hade befälet över folket som utförde arbetet.
1Ki 5:17  Och på konungens befallning bröto de stora och dyrbara stenar, för att husets grund skulle kunna läggas med huggen sten,
1Ki 5:18  Och Salomos byggningsmän och Hiroms byggningsmän och männen från Gebal höggo och tillredde både det trävirke och de stenar som behövdes till att bygga huset.
1Ki 6:1  I det fyra hundra åttionde året efter Israels barns uttåg ur Egyptens land, i det fjärde året av Salomos regering över Israel, i månaden Siv, det är den andra månaden, begynte han bygga huset åt HERREN.
1Ki 6:2  Huset som konung Salomo byggde åt HERREN var sextio alnar långt, tjugu alnar brett och trettio alnar högt.
1Ki 6:3  Förhuset framför tempelsalen var tjugu alnar långt, framför husets kortsida, och tio alnar brett, där det låg framför huset.
1Ki 6:4  Och han gjorde fönster på huset, slutna fönster, med bjälkramar.
1Ki 6:5  Och runt omkring huset, utmed dess vägg, uppförde han en ytterbyggnad, som gick runt omkring husets väggar, både utmed tempelsalen och utmed koret; och han gjorde däri sidokamrar runt omkring.
1Ki 6:6  Den nedersta våningen i ytterbyggnaden var fem alnar bred, den mellersta sex alnar bred och den tredje sju alnar bred; ty han hade gjort avsatser på huset runt omkring utvändigt, för att icke behöva göra fästhål i husets väggar.
1Ki 6:7  Och när huset uppfördes, byggdes det av sten som hade blivit färdighuggen vid stenbrottet; alltså hördes varken hammare eller yxa eller andra järnverktyg vid huset, när det byggdes.
1Ki 6:8  Dörren till mellersta sidokammaren hade sin plats på husets södra sida, och genom en trappgång kom man upp till den mellersta våningen, och från den mellersta våningen upp till den tredje.
1Ki 6:9  Så byggde han huset och fullbordade det. Och han panelade huset med inläggningar och med cederplankor i rader.
1Ki 6:10  Och i ytterbyggnaden utmed hela huset byggde han våningarna fem alnar höga; och den var fäst vid huset med cederbjälkar.
1Ki 6:11  Och HERRENS ord kom till Salomo; han sade:
1Ki 6:12  "Med detta hus som du nu bygger skall så ske: om du vandrar efter mina stadgar och gör efter mina rätter och håller alla mina bud och vandrar efter dem, så skall jag på dig uppfylla mitt ord, det som jag talade till din fader David:
1Ki 6:13  jag skall bo mitt ibland Israels barn och skall icke övergiva mitt folk Israel."
1Ki 6:14  Så byggde nu Salomo huset och fullbordade det.
1Ki 6:15  Han täckte husets väggar invändigt med bräder av cederträ. Från husets golv ända upp till takbjälkarna överklädde han det med trä invändigt; husets golv överklädde han med bräder av cypressträ.
1Ki 6:16  Och han täckte de tjugu alnarna i det innersta av huset med bräder av cederträ, från golvet ända upp till bjälkarna; så inrättade han rummet därinne åt sig till ett kor: det allraheligaste.
1Ki 6:17  Och fyrtio alnar mätte den del av huset, som utgjorde tempelsalen därframför.
1Ki 6:18  Och innantill hade huset en beläggning av cederträ med utsirningar i form av gurkfrukter och blomsterband; alltsammans var där av cederträ, ingen sten syntes.
1Ki 6:19  Och ett kor inredde han i det inre av huset för att där ställa HERRENS förbundsark.
1Ki 6:20  Och framför koret, som var tjugu alnar långt, tjugu alnar brett och tjugu alnar högt, och som han överdrog med fint guld, satte han ett altare, överklätt med cederträ.
1Ki 6:21  Och Salomo överdrog det inre av huset med fint guld. Och med kedjor av guld stängde han för koret; och jämväl detta överdrog han med guld.
1Ki 6:22  Alltså överdrog han hela huset med guld, till dess att hela huset var helt och hållet överdraget med guld. Han överdrog ock med guld hela det altare som hörde till koret.
1Ki 6:23  Och till koret gjorde han två keruber av olivträ. Den ena av dem var tio alnar hög;
1Ki 6:24  och den kerubens ena vinge var fem alnar, och kerubens andra vinge var ock fem alnar, så att det var tio alnar från den ena vingspetsen till den andra.
1Ki 6:25  Den andra keruben var ock tio alnar. Båda keruberna hade samma mått och samma form:
1Ki 6:26  den ena keruben var tio alnar hög och likaså den andra keruben.
1Ki 6:27  Och han ställde keruberna i de innersta av huset, och keruberna bredde ut sina vingar, så att den enas ena vinge rörde vid den ena väggen och den andra kerubens ena vinge rörde vid den andra väggen; och mitt i huset rörde deras båda andra vingar vid varandra.
1Ki 6:28  Och han överdrog keruberna med guld.
1Ki 6:29  Och alla husets väggar runt omkring utsirade han med snidverk i form av keruber, palmer och blomsterband; så både i det inre rummet och i det yttre.
1Ki 6:30  Och husets golv överdrog han med guld; så både i det inre rummet och i det yttre.
1Ki 6:31  För ingången till koret gjorde han dörrar av olivträ. Dörrinfattningen hade formen av en femkant.
1Ki 6:32  Och de båda dörrarna av olivträ prydde han med utsirningar i form av keruber, palmer och blomsterband, och överdrog dem med guld; han lade ut guldet över keruberna och palmerna.
1Ki 6:33  Likaså gjorde han för ingången till tempelsalen dörrposter av olivträ, i fyrkant,
1Ki 6:34  och två dörrar av cypressträ, var dörr bestående av två dörrhalvor som kunde vridas.
1Ki 6:35  Och han utsirade dem med keruber, palmer och blomsterband, och överdrog dem med guld, som lades jämnt över snidverken.
1Ki 6:36  Vidare byggde han den inre förgårdsmuren av tre varv huggna stenar och ett varv huggna bjälkar av cederträ.
1Ki 6:37  I det fjärde året blev grunden lagd till HERRENS hus, i månaden Siv.
1Ki 6:38  Och i det elfte året, månaden Bul, det är den åttonde månaden, var huset färdigt till alla sina delar alldeles såsom det skulle vara. Han byggde alltså därpå i sju år.
1Ki 7:1  Men på sitt eget hus byggde Salomo i tretton år, innan han fick hela sitt hus färdigt.
1Ki 7:2  Han byggde Libanonskogshuset, hundra alnar långt, femtio alnar brett och trettio alnar högt, med fyra rader pelare av cederträ och med huggna bjälkar av cederträ ovanpå pelarna.
1Ki 7:3  Det hade ock ett tak av cederträ över sidokamrarna, vilka vilade på pelarna, som tillsammans voro fyrtiofem, femton i var rad.
1Ki 7:4  Och det hade bjälklag i tre rader; och fönsteröppningarna sutto mitt emot varandra i tre omgångar.
1Ki 7:5  Alla dörröppningar och dörrposter voro fyrkantiga, av bjälkar; och fönsteröppningarna sutto alldeles mitt emot varandra i tre omgångar.
1Ki 7:6  Vidare gjorde han pelarförhuset, femtio alnar långt och trettio alnar brett, och framför detta också ett förhus med pelare, och med ett trapphus framför dessa.
1Ki 7:7  Och han gjorde tronförhuset, där han skulle skipa rätt, domsförhuset; det var belagt med cederträ från golv till tak.
1Ki 7:8  Och hans eget hus, där han själv skulle bo, på den andra gården, innanför förhuset, var byggt på samma sätt. Salomo byggde ock ett hus, likadant som detta förhus, åt Faraos dotter, som han hade tagit till hustru.
1Ki 7:9  Allt detta var av dyrbara stenar, avmätta såsom byggnadsblock och sågade med såg invändigt och utvändigt, alltsammans, ända ifrån grunden upp till taklisterna; och likaså allt därutanför, ända till den stora förgårdsmuren.
1Ki 7:10  Och grunden var lagd med dyrbara och stora stenar, stenar av tio alnars längd och av åtta alnars längd.
1Ki 7:11  Därovanpå lågo dyrbara stenar, avmätta såsom byggnadsblock, ävensom cederbjälkar.
1Ki 7:12  Och den stora förgårdsmuren där runt omkring var uppförd av tre varv huggna stenar och ett varv huggna bjälkar av cederträ. Så var det ock med den inre förgårdsmuren till HERRENS hus, så jämväl med husets förhus.
1Ki 7:13  Och konung Salomo sände och lät hämta Hiram från Tyrus.
1Ki 7:14  Denne var son till en änka av Naftali stam, och hans fader var en tyrisk man, en kopparsmed; han hade konstskicklighet och förstånd och kunskap i fullt mått till att utföra alla slags arbeten av koppar. Han kom nu till konung Salomo och utförde alla hans arbeten.
1Ki 7:15  Han förfärdigade de båda pelarna av koppar. Aderton alnar hög var den ena pelaren, och en tolv alnar lång tråd mätte omfånget av den andra pelaren.
1Ki 7:16  Han gjorde ock två pelarhuvuden, gjutna av koppar, till att sätta ovanpå pelarna; vart pelarhuvud var fem alnar högt.
1Ki 7:17  Nätlika utsirningar, som bildade ett nätverk, hängprydnader i form av kedjor funnos på pelarhuvudena som sutto ovanpå pelarna, sju på vart pelarhuvud.
1Ki 7:18  Och han gjorde pelarna så, att två rader gingo runt omkring över det ena av de nätverk som tjänade till att betäcka pelarhuvudena, vilka höjde sig över granatäpplena; och likadant gjorde han på det andra pelarhuvudet.
1Ki 7:19  Och pelarhuvudena som sutto ovanpå pelarna inne i förhuset voro utformade till liljor, och mätte fyra alnar.
1Ki 7:20  På båda pelarna funnos pelarhuvuden, också ovantill invid den bukformiga delen inemot nätverket. Och granatäpplena voro två hundra, i rader runt omkring, över det andra pelarhuvudet.
1Ki 7:21  Pelarna ställde han upp vid förhuset till tempelsalen. Åt den pelare han ställde upp på högra sidan gav han namnet Jakin, och åt den han ställde upp på vänstra sidan gav han namnet Boas.
1Ki 7:22  Överst voro pelarna utformade till liljor. Så blev då arbetet med pelarna fullbordat.
1Ki 7:23  Han gjorde ock havet, i gjutet arbete. Det var tio alnar från den ena kanten till den andra, runt allt omkring, och fem alnar högt; och ett trettio alnar långt snöre mätte dess omfång.
1Ki 7:24  Och under kanten voro gurklika sirater, som omgåvo det runt omkring - tio alnar brett som det var - så att de gingo runt omkring havet. De gurklika siraterna sutto i två rader, och de voro gjutna i ett stycke med det övriga.
1Ki 7:25  Det stod på tolv oxar, tre vända mot norr, tre vända mot väster, tre vända mot söder och tre vända mot öster; havet stod ovanpå dessa, och deras bakdelar voro alla vända inåt.
1Ki 7:26  Dess tjocklek var en handsbredd; och dess kant var gjord såsom kanten på en bägare, i form av en utslagen lilja. Det rymde två tusen bat.
1Ki 7:27  Vidare gjorde han de tio bäckenställen, av koppar. Vart ställ var fyra alnar långt, fyra alnar brett och tre alnar högt.
1Ki 7:28  Och på följande sätt voro dessa ställ gjorda. De voro försedda med sidolister, vilka sidolister hade sin plats mellan hörnlisterna.
1Ki 7:29  På dessa sidolister mellan hörnlisterna funnos avbildade lejon, tjurar och keruber, och likaså på hörnlisterna upptill. Under lejonen och tjurarna sutto nedhängande blomsterslingor.
1Ki 7:30  Vart ställ hade fyra hjul av koppar med axlar av koppar; och dess fyra fötter voro försedda med bärarmar. Dessa bärarmar voro gjutna till att sitta under bäckenet, och mitt för var och en sutto blomsterslingor.
1Ki 7:31  Sin öppning hade det inom kransstycket, som höjde sig en aln uppåt. öppningen i detta var rund; det var så gjort, att det kunde tjäna såsom underlag, och det mätte en och en halv aln. Också på dess öppning funnos utsirningar. Men sidolisterna därtill voro fyrkantiga, icke runda.
1Ki 7:32  De fyra hjulen sutto under sidolisterna, och hjulens hållare voro fästa vid bäckenstället. Vart hjul mätte en och en halv aln.
1Ki 7:33  Hjulen voro gjorda såsom vagnshjul; och deras hållare, deras ringar, deras ekrar och deras navar voro allasammans gjutna.
1Ki 7:34  Fyra bärarmar funnos på vart ställ, i de fyra hörnen; bärarmarna voro gjorda i ett stycke med sitt ställ.
1Ki 7:35  Överst på vart ställ var en helt och hållet rund uppsats, en halv aln hög; och ovantill på vart ställ sutto dess hållare, så ock dess sidolister gjorda i ett stycke därmed.
1Ki 7:36  Och på hållarnas ytor och på sidolisterna inristade han keruber, lejon och palmer, alltefter som utrymme fanns på var och en, så ock blomsterslingor runt omkring.
1Ki 7:37  På detta sätt gjorde han de tio bäckenställen; de voro alla gjutna på samma sätt och hade samma mått och samma form
1Ki 7:38  Han gjorde ock tio bäcken, av koppar. Fyrtio bat rymde vart bäcken, och vart bäcken mätte fyra alnar; till vart och ett av de tio bäckenställen gjordes ett bäcken.
1Ki 7:39  Och han ställde fem av bäckenställen på högra sidan om huset och fem på vänstra sidan om huset. Och havet ställde han på högra sidan om huset, åt sydost.
1Ki 7:40  Hirom gjorde dessa bäcken, så ock skovlarna och skålarna. Så förde Hiram allt det arbete till slut, som han fick utföra åt konung Salomo för HERRENS hus:
1Ki 7:41  nämligen två pelare, och de två klotformiga pelarhuvuden som sutto ovanpå pelarna, och de två nätverk som skulle betäcka de båda klotformiga pelarhuvuden som sutto ovanpå pelarna,
1Ki 7:42  därjämte de fyra hundra granatäpplena till de båda nätverken, två rader granatäpplen till vart nätverk, för att de båda klotformiga pelarhuvuden som sutto uppe på pelarna så skulle bliva betäckta,
1Ki 7:43  Vidare de tio bäckenställen och de tio bäckenen på bäckenställen,
1Ki 7:44  så ock havet, som var allenast ett, och de tolv oxarna under havet,
1Ki 7:45  vidare askkärlen, skovlarna och skålarna, korteligen, alla redan nämnda föremål som Hiram gjorde åt konung Salomo för HERRENS hus. Allt var av polerad koppar.
1Ki 7:46  På Jordanslätten lät konungen gjuta det i lerformar, mellan Suckot och Saretan.
1Ki 7:47  Och för den övermåttan stora myckenhetens skull lämnade Salomo alla föremålen ovägda, så att kopparens vikt icke blev utrönt.
1Ki 7:48  Salomo gjorde ock alla övriga föremål som skulle finnas i HERRENS hus: det gyllene altaret, det gyllene bordet som skådebröden skulle ligga på,
1Ki 7:49  så ock ljusstakarna, fem på högra sidan och fem på vänstra framför koret, av fint guld, med blomverket, lamporna och lamptängerna av guld,
1Ki 7:50  vidare faten, knivarna, de båda slagen av skålar och fyrfaten, av fint guld, äntligen de gyllene gångjärnen till de dörrar som ledde till det innersta av huset, det allraheligaste, och till de dörrar i huset, som ledde till tempelsalen.
1Ki 7:51  Sedan allt det arbete som konung Salomo lät utföra på HERRENS hus var färdigt, förde Salomo ditin vad hans fader David hade helgat åt HERREN: silvret, guldet och kärlen; detta lade han in i skattkamrarna i HERRENS hus.
1Ki 8:1  Därefter församlade Salomo de äldste i Israel, alla huvudmännen för stammarna, Israels barns familjehövdingar, till konung Salomo i Jerusalem, för att hämta HERRENS förbundsark upp från Davids stad, det är Sion.
1Ki 8:2  Så församlade sig då till konung Salomo alla Israels män under högtiden i månaden Etanim, det är den sjunde månaden.
1Ki 8:3  När då alla de äldste i Israel hade kommit tillstädes, lyfte prästerna upp arken.
1Ki 8:4  Och de hämtade HERRENS ark och uppenbarelsetältet ditupp, jämte alla heliga föremål som funnos i tältet; prästerna och leviterna hämtade det ditupp.
1Ki 8:5  Och konung Salomo stod framför arken jämte Israels hela menighet, som hade församlats till honom; och de offrade därvid småboskap och fäkreatur i sådan myckenhet, att de icke kunde täljas eller räknas.
1Ki 8:6  Och prästerna buro in HERRENS förbundsark till dess plats i husets kor, i det allraheligaste, till platsen under kerubernas vingar.
1Ki 8:7  Ty keruberna bredde ut sina vingar fram över den plats där arken stod, så att arken och dess stänger ovantill betäcktes av keruberna.
1Ki 8:8  Och stängerna voro så långa, att deras ändar väl kunde ses från helgedomen framför koret, men däremot icke voro synliga längre ute. Och de hava blivit kvar där ända till denna dag.
1Ki 8:9  I arken fanns intet annat än de två stentavlor som Mose hade lagt ned däri vid Horeb, när HERREN slöt förbund med Israels barn, sedan de hade dragit ut ur Egyptens land.
1Ki 8:10  Men när prästerna gingo ut ur helgedomen, uppfyllde molnskyn HERRENS hus,
1Ki 8:11  så att prästerna för molnskyns skull icke kunde stå där och göra tjänst; ty HERRENS härlighet uppfyllde HERRENS hus.
1Ki 8:12  Då sade Salomo: "HERREN har sagt att han vill bo i töcknet.
1Ki 8:13  Jag har nu byggt ett hus till boning åt dig, berett en plats där du må förbliva till evig tid."
1Ki 8:14  Sedan vände konungen sig om och välsignade Israels hela församling, under det att Israels hela församling förblev stående.
1Ki 8:15  Han sade: "Lovad vare HERREN, Israels Gud, som med sin hand har fullbordat vad han med sin mun lovade min fader David, i det han sade:
1Ki 8:16  'Från den dag då jag förde mitt folk Israel ut ur Egypten har jag icke i någon av Israels stammar utvalt en stad, till att i den bygga ett hus där mitt namn skulle vara; men David har jag utvalt till att råda över mitt folk Israel.'
1Ki 8:17  Och min fader David hade väl i sinnet att bygga ett hus åt HERRENS, Israels Guds, namn;
1Ki 8:18  men HERREN sade till min fader David: 'Då du nu har i sinnet att bygga ett hus åt mitt namn, så gör du visserligen väl däri att du har detta i sinnet;
1Ki 8:19  dock skall icke du få bygga detta hus, utan din son, den som har utgått från din länd, han skall bygga huset åt mitt namn.'
1Ki 8:20  Och HERREN har uppfyllt det löfte han gav; ty jag har kommit upp min fader Davids ställe och sitter nu på Israels tron, såsom HERREN lovade, och jag har byggt huset åt HERRENS, Israels Guds, namn.
1Ki 8:21  Och där har jag tillrett ett rum för arken, i vilken förvaras det förbund som HERREN slöt med våra fäder, när han förde dem ut ur Egyptens land."
1Ki 8:22  Därefter trädde Salomo fram för HERRENS altare inför Israels hela församling, och uträckte sina händer mot himmelen
1Ki 8:23  och sade: "HERRE, Israels Gud, ingen gud är dig lik, uppe i himmelen eller nere på jorden, du som håller förbund och bevarar nåd mot dina tjänare, när de vandra inför dig av allt sitt hjärta,
1Ki 8:24  du som har hållit vad du lovade din tjänare David, min fader; ty vad du med din mun lovade, det fullbordade du med din hand, såsom nu har skett.
1Ki 8:25  Så håll nu ock, HERRE, Israels Gud, vad du lovade din tjänare David, min fader, i det att du sade: 'Aldrig skall den tid komma, då på Israels tron icke inför mig sitter en avkomling av dig, om allenast dina barn hava akt på sin väg, så att de vandra inför mig, såsom du har vandrat inför mig.'
1Ki 8:26  Så låt nu, o Israels Gud, de ord som du har talat till din tjänare David, min fader, bliva sanna.
1Ki 8:27  Men kan då Gud verkligen bo på jorden? Himlarna och himlarnas himmel rymma dig ju icke; huru mycket mindre då detta hus som jag har byggt!
1Ki 8:28  Men vänd dig ändå till din tjänares bön och åkallan, HERRE, min Gud, så att du hör på det rop och den bön som din tjänare nu uppsänder till dig,
1Ki 8:29  och låter dina ögon natt och dag vara öppna och vända mot detta hus - den plats varom du har sagt: 'Mitt namn skall vara där' - så att du ock hör den bön som din tjänare beder, vänd mot denna plats.
1Ki 8:30  Ja, hör på den åkallan som din tjänare och ditt folk Israel uppsända, vända mot denna plats. Må du höra den och låta den komma upp till himmelen, där du bor; och när du hör, så må du förlåta.
1Ki 8:31  Om någon försyndar sig mot sin nästa och man ålägger honom en ed och låter honom svärja, och han så kommer och svär inför ditt altare i detta hus,
1Ki 8:32  må du då höra det i himmelen och utföra ditt verk och skaffa dina tjänare rätt, i det att du dömer den skyldige skyldig och låter hans gärningar komma över hans huvud, men skaffar rätt åt den som har rätt och låter honom få efter hans rättfärdighet.
1Ki 8:33  Om ditt folk Israel bliver slaget av en fiende, därför att de hava syndat mot dig, men de omvända sig till dig och prisa ditt namn och bedja och åkalla dig i detta hus,
1Ki 8:34  må du då höra det i himmelen och förlåta ditt folk Israels synd och låta dem komma tillbaka till det land som du har givit åt deras fäder.
1Ki 8:35  Om himmelen bliver tillsluten, så att regn icke faller, därför att de hava syndat mot dig, men de då bedja, vända mot denna plats, och prisa ditt namn och omvända sig från sin synd, när du bönhör dem,
1Ki 8:36  må du då höra det i himmelen och förlåta dina tjänares och ditt folk Israels synd, i det att du lär dem den goda väg som de skola vandra; och må du låta det regna över ditt land, det som du har givit åt ditt folk till arvedel.
1Ki 8:37  Om hungersnöd uppstår i landet, om pest uppstår, om sot eller rost, om gräshoppor eller gräsmaskar komma, om fienden tränger folket i det land där deras städer stå, eller om någon annan plåga eller sjukdom kommer, vilken det vara må,
1Ki 8:38  och om då någon bön och åkallan höjes från någon människa, vilken det vara må, eller från hela ditt folk Israel, när de var för sig känna plågan därav i sitt hjärta och så uträcka sina händer mot detta hus,
1Ki 8:39  må du då höra det i himmelen, där du bor, och förlåta och utföra ditt verk, i det att du giver var och en efter alla hans gärningar, eftersom du känner hans hjärta - ty du allena känner alla människors hjärtan -
1Ki 8:40  på det att de alltid må frukta dig, så länge de leva i det land som du har givit åt våra fäder.
1Ki 8:41  Också om en främling, en som icke är av ditt folk Israel, kommer ifrån fjärran land för ditt namns skull
1Ki 8:42  - ty man skall ock där höra talas om ditt stora namn och din starka hand och din uträckta arm - om någon sådan kommer och beder, vänd mot detta hus,
1Ki 8:43  må du då i himmelen, där du bor, höra det och göra allt varom främlingen ropar till dig, på det att alla jordens folk må känna ditt namn och frukta dig, likasom ditt folk Israel gör, och förnimma att detta hus som jag har byggt är uppkallat efter ditt namn.
1Ki 8:44  Om ditt folk drager ut i strid mot sin fiende, på den väg du sänder dem, och de då bedja till HERREN, vända i riktning mot den stad som du har utvalt, och mot det hus som jag har byggt åt ditt namn,
1Ki 8:45  må du då i himmelen höra deras bön och åkallan och skaffa dem rätt.
1Ki 8:46  Om de synda mot dig - eftersom ingen människa finnes, som icke syndar - och du bliver vred på dem och giver dem i fiendens våld, så att man tager dem till fånga och för dem bort till fiendens land, fjärran eller nära,
1Ki 8:47  men de då besinna sig i det land där de äro i fångenskap, och omvända sig och åkalla dig i landet där man håller dem fångna och säga: 'Vi hava syndat och gjort illa, vi hava varit ogudaktiga',
1Ki 8:48  om de så omvända sig till dig av allt sitt hjärta och av all sin själ, i sina fienders land - deras som hava fört dem i fångenskap - och bedja till dig, vända i riktning mot sitt land, det som du har givit åt deras fäder, och mot den stad som du har utvalt, och mot det hus som jag har byggt åt ditt namn,
1Ki 8:49  må du då i himmelen, där du bor, höra deras bön och åkallan och skaffa dem rätt
1Ki 8:50  och förlåta ditt folk vad de hava syndat mot dig, och alla de överträdelser som de hava begått mot dig, och låta dem finna barmhärtighet inför dem som hålla dem fångna, så att dessa förbarma sig över dem.
1Ki 8:51  Ty de äro ju ditt folk och din arvedel, som du har fört ut ur Egypten, den smältugnen.
1Ki 8:52  Ja, låt dina ögon vara öppna och vända till din tjänares och ditt folk Israels åkallan, så att du hör på dem, så ofta de ropa till dig.
1Ki 8:53  Ty du har själv avskilt dem åt dig till arvedel bland alla folk på Jorden, såsom du talade genom din tjänare Mose, när du förde våra fäder ut ur Egypten, o Herre, HERRE."
1Ki 8:54  När Salomo hade slutat att med dessa ord bedja och åkalla HERREN, stod han upp från HERRENS altare, där han hade legat på sina knän med händerna uträckta mot himmelen,
1Ki 8:55  och trädde fram och välsignade Israels hela församling med hög röst och sade:
1Ki 8:56  "Lovad vare HERREN, som har givit sitt folk Israel ro, alldeles såsom han har sagt! Alls intet har uteblivit av allt det goda som han lovade genom sin tjänare Mose.
1Ki 8:57  Så vare då HERREN, vår Gud, med oss, såsom han har varit med våra fäder. Han må icke övergiva oss och förskjuta oss,
1Ki 8:58  utan böja våra hjärtan till sig, så att vi alltid vandra på hans vägar och hålla hans bud och stadgar och rätter, dem som han har givit våra fäder.
1Ki 8:59  Och må dessa mina ord, med vilka jag har bönfallit inför HERRENS ansikte, vara nära HERREN, vår Gud, dag och natt, så att han skaffar rätt åt sin tjänare och rätt åt sitt folk Israel, efter var dags behov;
1Ki 8:60  på det att alla folk på jorden må förnimma att HERREN är Gud, och ingen annan.
1Ki 8:61  Och må edra hjärtan vara hängivna åt HERREN, vår Gud, så att I alltjämt vandren efter hans stadgar och hållen hans bud, såsom I nu gören."
1Ki 8:62  Och konungen jämte hela Israel offrade slaktoffer inför HERRENS ansikte.
1Ki 8:63  Till det tackoffer som Salomo offrade åt HERREN tog han tjugutvå tusen tjurar och ett hundra tjugu tusen av småboskapen. Så invigdes HERRENS hus av konungen och alla Israels barn.
1Ki 8:64  På samma dag helgade konungen den mellersta delen av förgården framför HERRENS hus, ty där offrade han brännoffret, spisoffret och fettstyckena av tackoffret, eftersom kopparaltaret, som stod inför HERRENS ansikte, var för litet för att brännoffret, spisoffret och fettstyckena av tackoffret skulle kunna rymmas där.
1Ki 8:65  Vid detta tillfälle firade Salomo högtiden, och med honom hela Israel - en stor församling ifrån hela landet, allt ifrån det ställe där vägen går till Hamat ända till Egyptens bäck - inför HERRENS, vår Guds, ansikte i sju dagar och åter sju dagar, tillsammans fjorton dagar.
1Ki 8:66  På åttonde dagen lät han folket gå, och de togo avsked av konungen. Sedan gingo de till sina hyddor, fulla av glädje och fröjd över allt det goda som HERREN hade gjort mot sin tjänare David och sitt folk Israel.
1Ki 9:1  Då nu Salomo hade byggt HERRENS hus färdigt, så ock konungshuset, ävensom allt annat som han hade känt åstundan och lust att utföra,
1Ki 9:2  uppenbarade sig HERREN för andra gången för Salomo, likasom han förut hade uppenbarat sig för honom i Gibeon.
1Ki 9:3  Och HERREN sade till honom "Jag har hört den bön och åkallan som du har uppsänt till mig; detta hus som du har byggt har jag helgat, till att där fästa mitt namn för evig tid. Och mina ögon och mitt hjärta skola vara där alltid.
1Ki 9:4  Om du nu vandrar inför mig, såsom din fader David vandrade, med ostraffligt hjärta och i redlighet, så att du gör allt vad jag har bjudit dig och håller mina stadgar och rätter
1Ki 9:5  då skall jag upprätthålla din konungatron över Israel evinnerligen, såsom jag lovade angående din fader David, när jag sade: 'Aldrig skall på Israels tron saknas en avkomling av dig.'
1Ki 9:6  Men om I och edra barn vänden om och övergiven mig, och icke hållen de bud och stadgar som jag har förelagt eder, utan gån bort och tjänen andra gudar och tillbedjen dem,
1Ki 9:7  då skall jag utrota Israel ur det land som jag har givit dem; och det hus som jag har helgat åt mitt namn skall jag förkasta ifrån mitt ansikte; och Israel skall bliva ett ordspråk och en visa bland alla folk.
1Ki 9:8  Och huru upphöjt detta hus nu än må vara, skall då var och en som går därförbi bliva häpen och vissla. Och när man frågar: 'Varför har HERREN gjort så mot detta land och detta hus?',
1Ki 9:9  då skall man svara: 'Därför att de övergåvo HERREN, sin Gud, som hade fört deras fäder ut ur Egyptens land, och höllo sig till andra gudar och tillbådo dem och tjänade dem, därför har HERREN låtit allt detta onda komma över dem.'"
1Ki 9:10  När de tjugu år voro förlidna, under vilka Salomo byggde på de två husen, HERRENS hus och konungshuset,
1Ki 9:11  gav konung Salomo tjugu städer i Galileen åt Hiram, konungen i Tyrus, som hade försett honom med cederträ, cypressträ och guld, så mycket han begärde.
1Ki 9:12  Men när Hiram från Tyrus begav sig ut för att bese de städer som Salomo hade givit honom, behagade de honom icke,
1Ki 9:13  utan han sade: "Vad är detta för städer som du har givit mig, min broder?" Och han kallade dem Kabuls land, såsom de heta ännu i dag.
1Ki 9:14  Men Hiram sände till konungen ett hundra tjugu talenter guld.
1Ki 9:15  Och på följande sätt förhöll det sig med det arbetsfolk som konung Salomo bådade upp för att bygga HERRENS hus och hans eget hus och Millo, ävensom Jerusalems murar, så ock Hasor, Megiddo och Geser.
1Ki 9:16  (Farao, konungen i Egypten, hade nämligen dragit upp och intagit Geser och bränt upp det i eld och dräpt de kananéer som bodde i staden, varefter han hade givit den till hemgift åt sin dotter, Salomos hustru.
1Ki 9:17  Men Salomo byggde upp Geser, ävensom Nedre Bet-Horon,
1Ki 9:18  så ock Baalat och Tamar i öknen där i landet,
1Ki 9:19  vidare alla Salomos förrådsstäder, vagnsstäderna och häststäderna, och vad annat Salomo kände åstundan att bygga i Jerusalem, på Libanon och eljest i hela det land som lydde under hans välde.)
1Ki 9:20  Allt det folk som fanns kvar av amoréerna, hetiterna, perisséerna hivéerna och jebuséerna, korteligen, alla de som icke voro av Israels barn -
1Ki 9:21  deras avkomlingar, så många som funnos kvar i landet efter dem, i det Israels barn icke hade förmått giva dem till spillo, dessa pålade Salomo att vara arbetspliktiga tjänare, såsom de äro ännu i dag.
1Ki 9:22  Men av Israels barn gjorde Salomo ingen till träl, utan de blevo krigare och blevo hans tjänare och hövitsman och kämpar, eller uppsyningsmän över hans vagnar och ridhästar.
1Ki 9:23  Överfogdarna över Salomos arbeten voro fem hundra femtio; dessa hade befälet över folket som utförde arbetet.
1Ki 9:24  Men så snart Faraos dotter hade flyttat upp från Davids stad till det hus som han hade byggt åt henne, byggde han ock Millo.
1Ki 9:25  Och Salomo offrade tre gånger om året brännoffer och tackoffer på det altare som han hade byggt åt HERREN, och tände därjämte rökelsen inför HERRENS ansikte. Så hade han då gjort huset färdigt.
1Ki 9:26  Konung Salomo byggde ock en flotta i Esjon-Geber, som ligger vid Elot, på stranden av Röda havet, i Edoms land.
1Ki 9:27  På denna flotta sände Hiram av sitt folk sjökunnigt skeppsmanskap, som åtföljde Salomos folk.
1Ki 9:28  De foro till Ofir och hämtade därifrån guld, fyra hundra tjugu talenter, som de förde till konung Salomo.
1Ki 10:1  När drottningen av Saba fick höra ryktet om Salomo och vad han hade gjort för HERRENS namn, kom hon för att sätta honom på prov med svåra frågor.
1Ki 10:2  Hon kom till Jerusalem med ett mycket stort följe, med kameler, som buro välluktande kryddor och guld i stor myckenhet, så ock ädla stenar. Och när hon kom inför Salomo, förelade hon honom allt vad hon hade i tankarna.
1Ki 10:3  Men Salomo gav henne svar på alla hennes frågor; intet var förborgat för konungen, utan han kunde giva henne svar på allt.
1Ki 10:4  När nu drottningen av Saba såg all Salomos vishet, och såg huset som han hade byggt,
1Ki 10:5  och såg rätterna på hans bord, och såg huru hans tjänare sutto där, och huru de som betjänade honom utförde sina åligganden, och huru de voro klädda, och vidare såg hans munskänkar, och när hon såg brännoffren som han offrade i HERRENS hus, då blev hon utom sig av förundran.
1Ki 10:6  Och hon sade till konungen: "Sant var det tal som jag hörde i mitt land om dig och om din vishet.
1Ki 10:7  Jag ville icke tro vad man sade, förrän jag själv kom och med egna ögon fick se det; men nu finner jag att det icke ens till hälften har blivit omtalat för mig. Du har långt mer vishet och rikedom, än jag genom ryktet hade hört.
1Ki 10:8  Sälla äro dina män, sälla äro dessa dina tjänare, som beständigt få stå inför dig och höra din visdom.
1Ki 10:9  Lovad vare HERREN, din Gud, som har funnit sådant behag i dig, att han har satt dig på Israels tron! Ja, därför att HERREN älskar Israel evinnerligen, därför har han satt dig till konung, för att du skall skipa lag och rätt."
1Ki 10:10  Och hon gav åt konungen ett hundra tjugu talenter guld, så ock välluktande kryddor i stor myckenhet, därtill ädla stenar; en så stor myckenhet av välluktande kryddor, som drottningen av Saba gav åt konung Salomo, har aldrig mer blivit införd.
1Ki 10:11  När Hirams flotta hämtade guld från Ofir, hemförde också den från Ofir almugträ i stor myckenhet, ävensom ädla stenar.
1Ki 10:12  Av almugträet lät konungen göra tillbehör till HERRENS hus och till konungshuset, så ock harpor och psaltare för sångarna. Så mycket almugträ har sedan intill denna dag icke införts eller blivit sett i landet.
1Ki 10:13  Konung Salomo åter gav åt drottningen av Saba allt vad hon åstundade och begärde, och skänkte henne i sin konungsliga frikostighet också annat därutöver. Sedan vände hon om och for till sitt land igen med sina tjänare.
1Ki 10:14  Det guld som årligen inkom till Salomo vägde sex hundra sextiosex talenter,
1Ki 10:15  förutom det som inkom genom kringresande handelsmän och genom krämares köpenskap, så ock från Erebs alla konungar och från ståthållarna i landet.
1Ki 10:16  Och konung Salomo lät göra två hundra stora sköldar av uthamrat guld och använde till var sådan sköld sex hundra siklar guld;
1Ki 10:17  likaledes tre hundra mindre sköldar av uthamrat guld och använde till var sådan sköld tre minor guld; och konungen satte upp dem i Libanonskogshuset.
1Ki 10:18  Vidare lät konungen göra en stor tron av elfenben och överdrog den med fint guld.
1Ki 10:19  Tronen hade sex trappsteg, och tronens ryggstycke var ovantill avrundat; på båda sidor om sitsen voro armstöd, och två lejon stodo utmed armstöden;
1Ki 10:20  och tolv lejon stodo där på de sex trappstegen, på båda sidor. Något sådant har aldrig blivit förfärdigat i något annat rike.
1Ki 10:21  Och alla konung Salomos dryckeskärl voro av guld, och alla kärl i Libanonskogshuset voro av fint guld; av silver fanns intet, det aktades icke för något i Salomos tid.
1Ki 10:22  Ty konungen hade en egen Tarsisflotta på havet jämte Hirams flotta; en gång vart tredje år kom Tarsisflottan hem och förde med sig guld och silver, elfenben, apor och påfåglar.
1Ki 10:23  Och konung Salomo blev större än någon annan konung på jorden, både i rikedom och i vishet.
1Ki 10:24  Från alla länder kom man för att besöka Salomo och höra den vishet som Gud hade nedlagt i hans hjärta.
1Ki 10:25  Och var och en förde med sig skänker: föremål av silver och av guld, kläder, vapen, välluktande kryddor, hästar och mulåsnor. Så skedde år efter år.
1Ki 10:26  Salomo samlade ock vagnar och ridhästar, så att han hade ett tusen fyra hundra vagnar och tolv tusen ridhästar; dem förlade han dels i vagnsstäderna, dels i Jerusalem, hos konungen själv.
1Ki 10:27  Och konungen styrde så, att silver blev lika vanligt i Jerusalem som stenar, och cederträ lika vanligt som mullbärsfikonträ i Låglandet.
1Ki 10:28  Och hästarna som Salomo lät anskaffa infördes från Egypten; ett antal kungliga uppköpare hämtade ett visst antal av dem till bestämt pris.
1Ki 10:29  Var vagn som hämtades upp från Egypten och infördes kostade sex hundra siklar silver, och var häst ett hundra femtio. Sammalunda infördes ock genom deras försorg sådana till hetiternas alla konungar och till konungarna i Aram.
1Ki 11:1  Men konung Salomo hade utom Faraos dotter många andra utländska kvinnor som han älskade: moabitiskor, ammonitiskor, edomeiskor, sidoniskor och hetitiskor,
1Ki 11:2  kvinnor av de folk om vilka HERREN hade lagt till Israels barn: "I skolen icke inlåta eder med dem, och de få icke inlåta sig med eder; de skola förvisso eljest förleda edra hjärtan att avfalla till deras gudar." Till dessa höll sig Salomo och älskade dem.
1Ki 11:3  Han hade sju hundra furstliga gemåler och tre hundra bihustrur. Dessa kvinnor förledde hans hjärta till avfall.
1Ki 11:4  Ja, när Salomo blev gammal, förledde kvinnorna hans hjärta att avfalla till andra gudar, så att hans hjärta icke förblev hängivet åt HERREN, hans Gud, såsom hans fader Davids hjärta hade varit.
1Ki 11:5  Så kom Salomo att följa efter Astarte, sidoniernas gudinna, och Milkom, ammoniternas styggelse.
1Ki 11:6  Och Salomo gjorde vad ont var i HERRENS ögon och följde icke i allt efter HERREN, såsom hans fader David hade gjort.
1Ki 11:7  Salomo byggde nämligen då en offerhöjd åt Kemos, moabiternas styggelse, på berget öster om Jerusalem, och likaså en åt Molok, Ammons barns styggelse.
1Ki 11:8  På samma sätt gjorde han för alla sina utländska kvinnor, så att de fingo tända offereld och frambära offer åt sina gudar.
1Ki 11:9  Och HERREN blev vred på Salomo, därför att hans hjärta hade avfallit från HERREN, Israels Gud, som dock två gånger hade uppenbarat sig för honom,
1Ki 11:10  och som hade givit honom ett särskilt bud angående denna sak, att han icke skulle följa efter andra gudar, ett HERRENS bud som han icke hade hållit.
1Ki 11:11  Därför sade HERREN till Salomo: "Eftersom det är så med dig, och eftersom du icke har hållit det förbund och de stadgar som jag har givit dig, skall jag rycka riket ifrån dig och giva det åt din tjänare.
1Ki 11:12  Men för din fader Davids skull vill jag icke göra detta i din tid; först ur din sons hand skall jag rycka det.
1Ki 11:13  Dock skall jag icke rycka hela riket ifrån honom, utan en stam skall jag giva åt din son, för min tjänare Davids skull och för Jerusalems skull, som jag har utvalt."
1Ki 11:14  Och HERREN lät en motståndare till Salomo uppstå i edoméen Hadad. Denne var av konungasläkten i Edom.
1Ki 11:15  Ty när David var i strid med Edom, och härhövitsmannen Joab drog upp för att begrava de slagna och därvid förgjorde allt mankön i Edom
1Ki 11:16  - ty Joab och hela Israel stannade där i sex månader, till dess att han hade utrotat allt mankön i Edom -
1Ki 11:17  då flydde Adad jämte några edomeiska män som hade varit i hans faders tjänst, och de togo vägen till Egypten; Hadad var då en ung gosse.
1Ki 11:18  De begav sig åstad från Midjan och kommo till Paran; och de togo folk med sig från Paran och kommo så till Egypten, till Farao, konungen i Egypten. Denne gav honom ett hus och anslog ett underhåll åt honom och gav honom land.
1Ki 11:19  Och Hadad fann mycken nåd för Faraos ögon, så att denne gav honom till hustru en syster till sin gemål, en syster till drottning Tapenes.
1Ki 11:20  Denna syster till Tapenes födde åt honom sonen Genubat, och Tapenes lät avvänja honom i Faraos hus; sedan vistades Genubat i Faraos hus bland Faraos söner.
1Ki 11:21  Då nu Hadad i Egypten hörde att David hade gått till vila hos sina fäder, och att härhövitsmannen Joab var död, sade han till Farao: "Låt mig fara hem till mitt land."
1Ki 11:22  Men Farao sade till honom: "Vad fattas dig här hos mig, eftersom du vill fara till ditt land?" Han svarade: "Hindra mig icke, utan låt mig gå.
1Ki 11:23  Och Gud lät ännu en motståndare till honom uppstå i Reson, Eljadas son, som hade flytt ifrån sin herre, Hadadeser, konungen i Soba.
1Ki 11:24  När David sedan anställde blodbadet ibland dem, samlade denne folk omkring sig och blev hövitsman för en strövskara; dessa drogo därefter till Damaskus och slogo sig ned där och gjorde sig till herrar i Damaskus.
1Ki 11:25  Denne var nu under Salomos hela livstid Israels motståndare och gjorde det skada, han såväl som Hadad. Han avskydde Israel; och han blev konung över Aram.
1Ki 11:26  Och en av Salomos tjänare hette Jerobeam; han var son till Nebat, en efraimit, från Sereda, och hans moder hette Seruga och var änka. Denne reste sig upp mot konungen.
1Ki 11:27  Orsaken varför han reste sig upp mot konungen var följande. Salomo byggde då på Millo; han ville befästa det blottade stället på sin fader Davids stad.
1Ki 11:28  Nu var Jerobeam en dugande man; och då Salomo såg att den unge mannen var driftig i sitt arbete, satte han honom över allt det arbete som ålåg Josefs hus.
1Ki 11:29  Vid den tiden hände sig en gång att Jerobeam hade begivit sig ut ur Jerusalem; då kom profeten Ahia från Silo emot honom på vägen, där han gick klädd i en ny mantel; och de båda voro ensamma på fältet.
1Ki 11:30  Och Ahia fattade i den nya manteln som han hade på sig och ryckte sönder den i tolv stycken.
1Ki 11:31  Därefter sade han till Jerobeam: "Tag här tio stycken för dig. Ty så säger HERREN, Israels Gud: Se, jag vill rycka riket ur Salomos hand och giva tio av stammarna åt dig;
1Ki 11:32  den ena stammen skall han få behålla för min tjänare Davids skull och för Jerusalems skull, den stads som jag har utvalt ur alla Israels stammar.
1Ki 11:33  Så skall ske, därför att de hava övergivit mig och tillbett Astarte, sidoniernas gudinna, och Kemos, Moabs gud, och Milkom, Ammons barns gud, och icke vandrat på mina vägar och icke gjort vad rätt är i mina ögon, efter mina stadgar och rätter, såsom hans fader David gjorde.
1Ki 11:34  Dock skall jag icke taga ifrån honom själv det samlade riket, utan jag vill låta honom förbliva furste, så länge han lever, för min tjänare Davids skull, som jag utvalde, därför att han höll mina bud och stadgar.
1Ki 11:35  Men från hans son skall jag taga konungadömet och giva det åt dig, nämligen de tio stammarna.
1Ki 11:36  En stam skall jag giva åt hans son, så att min tjänare David alltid har en lampa inför mitt ansikte i Jerusalem, den stad som jag har utvalt åt mig, till att där fästa mitt namn.
1Ki 11:37  Dig vill jag alltså taga och vill låta dig regera över allt vad dig lyster; du skall bliva konung över Israel.
1Ki 11:38  Om du nu hörsammar allt vad jag bjuder dig och vandrar på mina vägar och gör vad rätt är i mina ögon, så att du håller mina stadgar och bud, såsom min tjänare David gjorde, så skall jag vara med dig och bygga åt dig ett hus som bliver beståndande, såsom jag byggde ett hus åt David, och jag skall giva Israel åt dig. -
1Ki 11:39  Ja, för den sakens skull skall jag ödmjuka Davids säd, dock icke för alltid."
1Ki 11:40  Och Salomo sökte tillfälle att döda Jerobeam; men Jerobeam stod upp och flydde till Egypten, till Sisak, konungen i Egypten. Och han stannade i Egypten till Salomos död.
1Ki 11:41  Vad nu mer är att säga om Salomo, om allt vad han gjorde och om hans vishet, det finnes upptecknat i Salomos krönika.
1Ki 11:42  Den tid Salomo regerade i Jerusalem över hela Israel var fyrtio år.
1Ki 11:43  Och Salomo gick till vila hos sina fäder och blev begraven i sin fader Davids stad. Och hans son Rehabeam blev konung efter honom.
1Ki 12:1  Och Rehabeam drog till Sikem, ty hela Israel hade kommit till Sikem för att göra honom till konung.
1Ki 12:2  När Jerobeam, Nebats son, hörde detta - han var då ännu kvar i Egypten, dit han hade flytt för konung Salomo; Jerobeam bodde alltså i Egypten,
1Ki 12:3  men de sände ditbort och läto kalla honom åter - då kom han tillstädes jämte Israels hela församling och talade till Rehabeam och sade:
1Ki 12:4  "Din fader gjorde vårt ok för svårt; men lätta nu du det svåra arbete och det tunga ok som din fader lade på oss, så vilja vi tjäna dig."
1Ki 12:5  Han svarade dem: "Gån bort och vänten ännu tre dagar, och kommen så tillbaka till mig." Och folket gick.
1Ki 12:6  Då rådförde sig konung Rehabeam med de gamle som hade varit i tjänst hos hans fader Salomo, medan denne ännu levde; han sade: "Vilket svar råden I mig att giva detta folk?"
1Ki 12:7  De svarade honom och sade: "Om du i dag underkastar dig detta folk och bliver dem till tjänst, om du lyssnar till deras bön och talar goda ord till dem, så skola de för alltid bliva dina tjänare."
1Ki 12:8  Men han aktade icke på det råd som de gamle hade givit honom, utan rådförde sig med de unga män som hade vuxit upp med honom, och som nu voro i hans tjänst.
1Ki 12:9  Han sade till dem: "Vilket svar råden I oss att giva detta folk som har talat till mig och sagt: 'Lätta det ok som din fader har lagt på oss'?"
1Ki 12:10  De unga männen som hade vuxit upp med honom svarade honom då och sade: "Så bör du säga till detta folk som har talat till dig och sagt: 'Din fader gjorde vårt ok tungt, men lätta du det för oss' - så bör du tala till dem: 'Mitt minsta finger är tjockare än min faders länd.
1Ki 12:11  Så veten nu, att om min fader har belastat eder med ett tungt ok, så skall jag göra edert ok ännu tyngre; har min fader tuktat eder med ris, så skall jag tukta eder med skorpiongissel.'"
1Ki 12:12  Så kom nu Jerobeam med allt folket till Rehabeam på tredje dagen, såsom konungen hade befallt, i det han sade: "Kommen tillbaka till mig på tredje dagen."
1Ki 12:13  Då gav konungen folket ett hårt svar; ty han aktade icke på det råd som de gamle hade givit honom.
1Ki 12:14  Han talade till dem efter de unga männens råd och sade: "Har min fader gjort edert ok tungt, så skall jag göra edert ok ännu tyngre; har min fader tuktat eder med ris, så skall jag tukta eder med skorpiongissel."
1Ki 12:15  Alltså hörde konungen icke på folket; ty det var så skickat av HERREN, för att hans ord skulle uppfyllas, det som HERREN hade talat till Jerobeam, Nebats son, genom Ahia från Silo.
1Ki 12:16  Då nu hela Israel förnam att konungen icke ville höra på dem, gav folket konungen detta svar: "Vad del hava vi i David? Ingen arvslott hava vi i Isais son. Drag hem till dina hyddor, Israel. Se nu själv om ditt hus, du David." Därefter drog Israel hem till sina hyddor.
1Ki 12:17  Allenast över de israeliter som bodde i Juda städer förblev Rehabeam konung.
1Ki 12:18  Och när konung Rehabeam sände åstad Adoram, som hade uppsikten över de allmänna arbetena, stenade hela Israel denne till döds; och konung Rehabeam själv måste med hast stiga upp i sin vagn och fly till Jerusalem.
1Ki 12:19  Så avföll Israel från Davids hus och har varit skilt därifrån ända till denna dag.
1Ki 12:20  Men när hela Israel hörde att Jerobeam hade kommit tillbaka, sände de och läto kalla honom till folkförsamlingen och gjorde honom till konung över hela Israel; ingen höll sig till Davids hus, utom Juda stam allena.
1Ki 12:21  Och när Rehabeam kom till Jerusalem, församlade han hela Juda hus och Benjamins stam, ett hundra åttio tusen utvalda krigare, för att de skulle strida mot Israels hus och återvinna konungadömet åt Rehabeam, Salomos son.
1Ki 12:22  Men Guds ord kom till gudsmannen Semaja;
1Ki 12:23  han sade: "Säg till Rehabeam, Salomos son, Juda konung, och till hela Juda hus och Benjamin och till det övriga folket:
1Ki 12:24  Så säger HERREN: I skolen icke draga upp och strida mot edra bröder, Israels barn. Vänden tillbaka hem, var och en till sitt, ty vad som har skett har kommit från mig." Och de lyssnade till HERRENS ord och vände om och gingo sin väg, såsom HERREN hade befallt.
1Ki 12:25  Men Jerobeam befäste Sikem i Efraims bergsbygd och bosatte sig där. Därifrån drog han åstad och befäste Penuel.
1Ki 12:26  Och Jerobeam sade vid sig själv: "Såsom nu är, kan riket komma tillbaka till Davids hus.
1Ki 12:27  Ty om folket här får draga upp och anställa slaktoffer i HERRENS hus i Jerusalem, så kan folkets hjärta vända tillbaka till deras herre Rehabeam, Juda konung; ja, då kunna de dräpa mig och vända tillbaka till Rehabeam, Juda konung."
1Ki 12:28  Sedan nu konungen hade överlagt härom, lät han göra två kalvar av guld. Därefter sade han till folket: "Nu må det vara nog med edra färder upp till Jerusalem. Se, här är din Gud, Israel, han som har fört dig upp ur Egyptens land."
1Ki 12:29  Och han ställde upp den ena i Betel, och den andra satte han upp i Dan.
1Ki 12:30  Detta blev en orsak till synd; folket gick ända till Dan för att träda fram inför den ena av dem.
1Ki 12:31  Han byggde också upp offerhöjdshus och gjorde till präster allahanda män ur folket, sådana som icke voro av Levi barn.
1Ki 12:32  Och Jerobeam anordnade en högtid i åttonde månaden, på femtonde dagen i månaden, lik högtiden Juda, och steg då upp till altaret; så gjorde han i Betel för att offra åt de kalvar som han hade låtit göra. Och de män som han hade gjort till offerhöjdspräster lät han göra tjänst i Betel.
1Ki 12:33  Till det altare som han hade gjort i Betel steg han alltså upp på femtonde dagen i åttonde månaden, den månad som han av eget påfund hade valt. Han anordnade nämligen då en högtid för Israels barn och steg upp till altaret för att där tända offereld.
1Ki 13:1  Men då kom på HERRENS befallning en gudsman från Juda till Betel, just när Jerobeam stod vid altaret för att där tända offereld.
1Ki 13:2  Och mannen ropade mot altaret på HERRENS befallning och sade: "Altare! Altare! Så säger HERREN: Se, åt Davids hus skall födas en son vid namn Josia, han skall på dig slakta offerhöjdsprästerna som antända offereld på dig, och människoben skall man då bränna upp på dig."
1Ki 13:3  På samma gång angav han ett tecken, i det han sade: "Detta är tecknet på att det är HERREN som har talat: se, altaret skall rämna, och askan därpå skall spillas ut."
1Ki 13:4  När konung Jerobeam hörde dessa ord, som gudsmannen ropade mot altaret i Betel, räckte han ut sin hand från altaret och sade: "Gripen honom." Men handen som han hade räckt ut mot honom förvissnade, och han kunde icke draga den tillbaka till sig igen.
1Ki 13:5  Och altaret rämnade, och askan på altaret spilldes ut; det var det tecken som gudsmannen på HERRENS befallning hade angivit.
1Ki 13:6  Då tog konungen till orda och sade till gudsmannen: "Bönfall inför HERREN, din Gud, och bed för mig att jag må kunna draga min hand tillbaka till mig igen." Och gudsmannen bönföll inför HERREN; och konungen kunde då draga sin hand tillbaka till sig igen, och den var likadan som förut.
1Ki 13:7  Då talade konungen till gudsmannen: "Kom hem med mig och vederkvick dig; sedan vill jag giva dig en gåva."
1Ki 13:8  Men gudsmannen svarade konungen: "Om du än vill giva mig hälften av vad som finnes i ditt hus, så kommer jag dock icke med dig; här på orten vill jag varken äta eller dricka.
1Ki 13:9  Ty så har HERREN genom sitt ord bjudit mig och sagt: Du skall varken äta eller dricka, och ej heller vända tillbaka samma väg du har gått hit."
1Ki 13:10  Därefter gick han sina färde en annan väg och vände icke tillbaka samma väg han hade kommit till Betel.
1Ki 13:11  Men i Betel bodde en gammal profet. Dennes son kom och förtäljde för honom allt vad gudsmannen den dagen hade gjort i Betel, huru han hade talat till konungen. När de hade förtäljt detta för sin fader,
1Ki 13:12  frågade deras fader dem vilken väg han hade gått. Och hans söner visste vilken väg gudsmannen som kom från Juda hade gått.
1Ki 13:13  Då sade han till sina söner: "Sadlen åsnan åt mig." När de då hade sadlat åsnan åt honom, satte han sig på den
1Ki 13:14  och begav dig åstad efter gudsmannen och fann honom sittande under terebinten; och han frågade honom: "Är du den gudsman som har kommit från Juda?" Han svarade: "Ja."
1Ki 13:15  Då sade han till honom: "Kom med mig hem och ät med mig."
1Ki 13:16  Men han svarade: "Jag kan icke vända om med dig och följa dig, och jag vill icke äta eller dricka med dig här på orten;
1Ki 13:17  ty så har blivit mig sagt genom HERRENS ord: Du skall varken äta eller dricka där; du skall icke heller gå tillbaka samma väg du har gått dit."
1Ki 13:18  Han sade till honom: "Jag är ock en profet såsom du, och en ängel har talat till mig på HERRENS befallning och sagt: 'För honom tillbaka med dig hem och giv honom att äta och dricka.'" Men häri ljög han för honom.
1Ki 13:19  Då vände han tillbaka med honom och åt i hans hus och drack.
1Ki 13:20  Men under det att de sutto till bords, kom HERRENS ord till profeten som hade fört honom tillbaka.
1Ki 13:21  Och han ropade till gudsmannen som hade kommit från Juda och sade: "Så säger HERREN: Därför att du har varit gensträvig mot HERRENS ord och icke hållit det bud som HERREN, din Gud, har givit dig,
1Ki 13:22  utan vänt tillbaka och ätit och druckit på den ort där han hade förbjudit, dig att äta och dricka, därför skall din döda kropp icke komma i dina fäders grav.
1Ki 13:23  Sedan han nu hade ätit och druckit, sadlade han åsnan åt honom, åt profeten som han hade fört tillbaka.
1Ki 13:24  Och denne begav sig åstad; men ett lejon kom emot honom på vägen och dödade honom. Sedan låg hans döda kropp utsträckt där på vägen, under det att åsnan stod bredvid den; och lejonet stod också bredvid den döda kroppen.
1Ki 13:25  Då nu folk som gick därförbi fick se den döda kroppen ligga utsträckt på vägen och lejonet stå bredvid den döda kroppen, gingo de in i staden där den gamle profeten bodde och omtalade det där.
1Ki 13:26  När profeten, som hade fört honom tillbaka från hans väg, hörde det, sade han: "Det är gudsmannen, han som var gensträvig mot HERRENS ord; därför har HERREN givit honom i lejonets våld, och det har krossat och dödat honom, i enlighet med det ord som HERREN hade talat till honom."
1Ki 13:27  Därefter tillsade han sina söner att de skulle sadla åsnan åt honom; och de sadlade den.
1Ki 13:28  Så begav han sig åstad och fann den döda kroppen liggande utsträckt på vägen och åsnan och lejonet stående bredvid den döda kroppen; lejonet hade icke ätit av den döda kroppen och ej heller krossat åsnan.
1Ki 13:29  Då tog profeten upp gudsmannens döda kropp och lade den på åsnan och förde den tillbaka; och den gamle profeten begav sig in i sin stad för att hålla dödsklagan och begrava honom.
1Ki 13:30  Och han lade hans döda kropp i sin egen grav; och de höllo dödsklagan efter honom och ropade: "Ack ve, min broder!"
1Ki 13:31  Då han nu hade begravit honom, sade han till sina söner: "När jag dör, så begraven mig i den grav där gudsmannen ligger begraven; läggen mina ben vid sidan av hans ben.
1Ki 13:32  Ty förvisso skall det ord gå i fullbordan, som han på HERRENS befallning ropade mot altaret i Betel och mot alla offerhöjdshus i Samariens städer."
1Ki 13:33  Dock vände Jerobeam efter detta icke om från sin onda väg, utan gjorde åter allahanda man ur folket till offerhöjdspräster; vem som hade lust därtill fick av honom mottaga handfyllning till att vara offerhöjdspräst.
1Ki 13:34  På detta sätt blev han för Jerobeams hus en orsak till synd, och en orsak till att det blev utplånat och utrotat från jorden.
1Ki 14:1  Vid den tiden blev Abia, Jerobeams son, sjuk.
1Ki 14:2  Då sade Jerobeam till sin hustru: "Stå upp och förkläd dig, så att ingen kan märka att du är Jerobeams hustru, och gå till Silo, ty där bor profeten Ahia, han som förkunnade om mig att jag skulle bliva konung över detta folk.
1Ki 14:3  Och tag med dig tio bröd, därtill smått bakverk och en kruka honung, och gå in till honom; han skall då förkunna för dig huru det skall gå med gossen."
1Ki 14:4  Jerobeams hustru gjorde så; hon stod upp och gick till Silo och kom till Ahias hus. Och Ahia kunde icke se, ty hans ögon voro starrblinda av ålderdom.
1Ki 14:5  Men HERREN hade sagt till Ahia: "Just nu kommer Jerobeams hustru för att förfråga sig hos dig om sin son, ty han är sjuk; så och så skall du tala till henne. Men när hon kommer, skall hon ställa sig främmande.
1Ki 14:6  Då nu Ahia hörde ljudet av hennes steg, när hon kom i dörren, sade han: "Kom in, du Jerobeams hustru. Varför ställer du dig främmande? Jag har ju fått uppdrag att giva dig ett hårt budskap.
1Ki 14:7  Gå och säg Jerobeam: Så säger HERREN, Israels Gud: Se, jag har upphöjt dig ur folket och satt dig till furste över mitt folk Israel
1Ki 14:8  och har ryckt riket från Davids hus och givit det åt dig. Men du har icke varit sådan som min tjänare David, som höll mina bud och följde efter mig av allt sitt hjärta, så att han gjorde allenast vad rätt var i mina ögon;
1Ki 14:9  utan du har gjort mer ont än alla som hava varit före dig och har gått bort och gjort dig andra gudar, nämligen gjutna beläten, för att förtörna mig, och har kastat mig bakom din rygg.
1Ki 14:10  Därför skall jag låta olycka komma över Jerobeams hus och utrota allt mankön av Jerobeams hus, både små och stora i Israel; och jag skall bortsopa Jerobeams hus, såsom man sopar bort orenlighet, till dess det bliver en ände därpå.
1Ki 14:11  Den av Jerobeams hus, som dör i staden, skola hundarna äta upp, och den som dör ute på marken, skola himmelens fåglar äta upp. Ty så har HERREN talat.
1Ki 14:12  Så stå du nu upp och gå hem igen. När din fot träder in i staden, skall barnet dö.
1Ki 14:13  Och hela Israel skall hålla dödsklagan efter honom, och man skall begrava honom; ty av Jerobeams hus skall allenast han komma i en grav, därför att i Jerobeams hus dock hos honom blev funnet något som var gott inför HERREN, Israels Gud.
1Ki 14:14  Men HERREN skall låta en konung över Israel uppstå åt sig, en konung som skall utrota Jerobeams hus. Detta är den dagen; och vad skall icke nu ske!
1Ki 14:15  HERREN skall slå Israel, så att det bliver likt vassen, som vaggar hit och dit i vattnet. Och han skall rycka upp Israel ur detta goda land, som han har givit åt deras fäder, och skall förströ dem på andra sidan floden, därför att de hava gjort sig Aseror och därmed förtörnat HERREN.
1Ki 14:16  Och han skall prisgiva Israel för de synders skull som Jerobeam har begått, och genom vilka han har kommit Israel att synda."
1Ki 14:17  Då stod Jerobeams hustru upp och gick sin väg och kom till Tirsa; och just som hon beträdde husets tröskel, gav gossen upp andan.
1Ki 14:18  Och man begrov honom, och hela Israel höll dödsklagan efter honom, i enlighet med det ord som HERREN hade talat genom sin tjänare, profeten Ahia.
1Ki 14:19  Vad nu mer är att säga om Jerobeam, om hans krig och om hans regering, det finnes upptecknat i Israels konungars krönika.
1Ki 14:20  Den tid Jerobeam regerade var tjugutvå år. Så gick han till vila hos sina fäder; och hans son Nadab blev konung efter honom.
1Ki 14:21  Men Rehabeam, Salomos son, var konung i Juda. Fyrtioett år gammal var Rehabeam, när han blev konung, och han regerade sjutton år i Jerusalem, den stad som HERREN hade utvalt ur alla Israels stammar, till att där fästa sitt namn. Hans moder hette Naama, ammonitiskan.
1Ki 14:22  Och Juda gjorde vad ont var i HERRENS ögon; med de synder som de begingo retade de honom långt mer, än deras fäder hade gjort.
1Ki 14:23  Ty också de byggde sig offerhöjder och reste stoder och Aseror på alla höga kullar och under alla gröna träd;
1Ki 14:24  ja, också tempelbolare funnos i landet. De gjorde efter alla styggelser hos de folk som HERREN hade fördrivit för Israels barn.
1Ki 14:25  Men i konung Rehabeams femte regeringsår drog Sosak, konungen i Egypten, upp mot Jerusalem.
1Ki 14:26  Och han tog skatterna i HERRENS hus och skatterna i konungshuset; alltsammans tog han. Han tog ock alla de gyllene sköldar som Salomo hade låtit göra.
1Ki 14:27  I deras ställe lät konung Rehabeam göra sköldar av koppar, och dessa lämnade han i förvar åt hövitsmännen för drabanterna som höllo vakt vid ingången till konungshuset.
1Ki 14:28  Och så ofta konungen gick till HERRENS hus, buro drabanterna dem; sedan förde de dem tillbaka till drabantsalen.
1Ki 14:29  Vad nu mer är att säga om Rehabeam och om allt vad han gjorde, det finnes upptecknat i Juda konungars krönika.
1Ki 14:30  Men Rehabeam och Jerobeam lågo i krig med varandra, så länge de levde.
1Ki 14:31  Och Rehabeam gick till vila hos sina fäder och blev begraven hos sina fäder i Davids stad. Hans moder hette Naama, ammonitiskan. Och hans son Abiam blev konung efter honom.
1Ki 15:1  I konung Jerobeams, Nebats sons, adertonde regeringsår blev Abiam konung över Juda.
1Ki 15:2  Han regerade tre år i Jerusalem. Hans moder hette Maaka, Abisaloms dotter.
1Ki 15:3  Och han vandrade i alla de synder som hans fader hade begått före honom, och hans hjärta var icke hängivet åt HERREN, hans Gud, såsom hans fader Davids hjärta hade varit.
1Ki 15:4  Allenast för Davids skull lät HERREN, hans Gud, honom få en lampa i Jerusalem, i det att han uppsatte hans son efter honom och lät Jerusalem hava bestånd -
1Ki 15:5  detta därför att David gjorde vad rätt var i HERRENS ögon och icke vek ifrån något som han bjöd honom, så länge han levde, utom i saken med hetiten Uria.
1Ki 15:6  Men Rehabeam och Jerobeam lågo i krig med varandra, så länge den förre levde.
1Ki 15:7  Vad nu mer är att säga om Abiam och om allt vad han gjorde, det finnes upptecknat i Juda konungars krönika. Men Abiam och Jerobeam lågo i krig med varandra.
1Ki 15:8  Och Abiam gick till vila hos sina fäder, och man begrov honom i Davids stad. Och hans son Asa blev konung efter honom.
1Ki 15:9  I Jerobeams, Israels konungs, tjugonde regeringsår blev Asa konung över Juda.
1Ki 15:10  Han regerade fyrtioett år i Jerusalem. Hans moder hette Maaka, Abisaloms dotter.
1Ki 15:11  Och Asa gjorde vad rätt var i HERRENS ögon, såsom hans fader David hade gjort
1Ki 15:12  Han drev ut tempelbolarna ur landet och skaffade bort alla de eländiga avgudabeläten som hans fader hade låtit göra.
1Ki 15:13  Ja, sin moder Maaka avsatte han från hennes drottningsvärdighet, därför att hon hade satt upp en styggelse åt Aseran; Asa högg nu ned styggelsen och brände upp den i Kidrons dal.
1Ki 15:14  Men offerhöjderna blevo icke avskaffade; dock var Asas hjärta hängivet åt HERREN, så länge han levde.
1Ki 15:15  Och han förde in i HERRENS hus både vad hans fader och vad han själv hade helgat åt HERREN: silver, guld och kärl.
1Ki 15:16  Men Asa och Baesa, Israels konung, lågo i krig med varandra, så länge de levde.
1Ki 15:17  Baesa, Israels konung, drog upp mot Juda och begynte befästa Rama, för att hindra att någon komme vare sig till eller ifrån Asa, Juda konung.
1Ki 15:18  Då tog Asa allt silver och guld som fanns kvar i skattkamrarna i HERRENS hus, ävensom skatterna i konungshuset, och lämnade detta åt sina tjänare; därefter sände konung Asa dem till Ben-Hadad, son till Tabrimmon, son till Hesjon, konungen i Aram, som bodde i Damaskus, och lät säga:
1Ki 15:19  "Ett förbund består ju mellan mig och dig, såsom det var mellan min fader och din fader. Se, här sänder jag dig skänker av silver och guld, så bryt då nu ditt förbund med Baesa, Israels konung, för att han må lämna mig i fred."
1Ki 15:20  Och Ben-Hadad lyssnade till konung Asa och sände sina krigshövitsmän mot Israels städer och förhärjade Ijon, Dan, Abel-Bet-Maaka och hela Kinarot jämte hela Naftali land.
1Ki 15:21  När Baesa hörde detta, avstod han från att befästa Rama och höll sig sedan stilla i Tirsa.
1Ki 15:22  Men konung Asa bådade upp hela Juda, ingen fritagen; och de förde bort stenar och trävirke som Baesa använde till att befästa Rama. Därmed befäste nu konung Asa Geba i Benjamin, så ock Mispa.
1Ki 15:23  Allt vad mer är att säga om Asa, om alla hans bedrifter, om allt vad han gjorde och om de städer han byggde, det finnes upptecknat i Juda konungars krönika. Men på sin ålderdom fick han en sjukdom i sina fötter.
1Ki 15:24  Och Asa gick till vila hos sina fäder och blev begraven hos sina fäder i sin fader Davids stad. Och hans son Josafat blev konung efter honom.
1Ki 15:25  Men Nadab, Jerobeams son, blev konung över Israel i Asas, Juda konungs, andra regeringsår, och han regerade över Israel i två år.
1Ki 15:26  Han gjorde vad ont var i HERRENS ögon och vandrade på sin faders väg och i den synd genom vilken denne hade kommit Israel att synda.
1Ki 15:27  Men Baesa, Ahias son, av Isaskar hus, anstiftade en sammansvärjning mot honom, och Baesa dräpte honom vid Gibbeton, som tillhörde filistéerna; Nadab med hela Israel höll nämligen på med att belägra Gibbeton.
1Ki 15:28  I Asas, Juda konungs, tredje regeringsår var det som Baesa dödade honom, och han blev så själv konung i hans ställe.
1Ki 15:29  Och när han hade blivit konung förgjorde han hela Jerobeams hus; han lät intet som anda hade bliva kvar av Jerobeams hus, utan utrotade det, i enighet med det ord som HERREN hade talat genom sin tjänare Ahia från Silo -
1Ki 15:30  detta för de synders skull som Jerobeam hade begått, och genom vilka han kom Israel att synda, så att han därmed förtörnade HERREN, Israels Gud.
1Ki 15:31  Vad nu mer är att säga om Nadab och om allt vad han gjorde det finnes upptecknat i Israels konungars krönika.
1Ki 15:32  Men Asa och Baesa, Israels konung, lågo i krig med varandra, länge de levde.
1Ki 15:33  I Asas, Juda konungs, tredje regeringsår blev Baesa, Ahias son, konung över hela Israel i Tirsa och regerade i tjugufyra år.
1Ki 15:34  Han gjorde vad ont var i HERRENS ögon och vandrade på Jerobeams väg och i den synd genom vilken denne hade kommit Israel att synda.
1Ki 16:1  Och HERRENS ord kom till Jehu, Hananis son, mot Baesa; han sade:
1Ki 16:2  "Se, jag har lyft dig upp ur stoftet och satt dig till furste över mitt folk Israel. Men du har vandrat på Jerobeams väg och kommit mitt folk Israel att synda, så att de hava förtörnat mig genom sina synder.
1Ki 16:3  Därför vill jag bortsopa Baesa och hans hus; ja, jag vill göra med ditt hus såsom jag gjorde med Jerobeams, Nebats sons, hus.
1Ki 16:4  Den av Baesas hus, som dör i staden, skola hundarna äta upp, och den av hans hus, som dör ute på marken, skola himmelens fåglar äta upp."
1Ki 16:5  Vad nu mer är att säga om Baesa, om vad han gjorde och om hans bedrifter, det finnes upptecknat i Israels konungars krönika.
1Ki 16:6  Och Baesa gick till vila hos sina fäder och blev begraven i Tirsa. Och hans son Ela blev konung efter honom.
1Ki 16:7  Men genom profeten Jehu, Hananis son, hade HERRENS ord kommit till Baesa och hans hus, icke allenast för allt det onda som han hade gjort i HERRENS ögon, då han förtörnade honom genom sina händers verk, så att det måste gå honom såsom det gick Jerobeams hus, utan ock därför att han hade förgjort detta.
1Ki 16:8  I Asas, Juda konungs, tjugusjätte regeringsår blev Ela, Baesas son, konung över Israel i Tirsa och regerade i två år.
1Ki 16:9  Men hans tjänare Simri, som var hövitsman för den ena hälften av stridsvagnarna, anstiftade en sammansvärjning mot honom. Och en gång, då han i Tirsa hade druckit sig drucken i Arsas hus, överhovmästarens i Tirsa,
1Ki 16:10  kom Simri dit och slog honom till döds - det var i Asas, Juda konungs, tjugusjunde regeringsår - och han själv blev så konung i hans ställe.
1Ki 16:11  Och när han hade blivit konung och intagit sin tron, förgjorde han hela Baesas hus, utan att låta någon av mankön bliva kvar, varken hans blodsförvanter eller hans vänner.
1Ki 16:12  Så utrotade Simri hela Baesas hus, i enlighet med det ord som HERREN hade talat till Baesa genom profeten Jehu -
1Ki 16:13  detta för alla de synders skull som Baesa och hans son Ela hade begått, och genom vilka de hade kommit Israel att synda, så att de förtörnade HERREN, Israels Gud, med de fåfängliga avgudar som de dyrkade.
1Ki 16:14  Vad nu mer är att säga om Ela och om allt vad han gjorde, det finnes upptecknat i Israels konungars krönika.
1Ki 16:15  I Asas, Juda konungs, tjugusjunde regeringsår blev Simri konung och regerade i sju dagar, i Tirsa. Folket höll då på att belägra Gibbeton, som tillhörde filistéerna.
1Ki 16:16  Medan nu folket höll på med belägringen, fingo de höra sägas "Simri har anstiftat en sammansvärjning; han har ock dräpt konungen." Då gjorde hela Israel samma dag Omri, den israelitiske härhövitsmannen, till konung, i lägret.
1Ki 16:17  Därefter drog Omri med hela Israel upp från Gibbeton, och de angrepo Tirsa.
1Ki 16:18  Men när Simri såg att staden var intagen, gick han in i konungshusets palatsbyggnad och brände upp konungshuset jämte sig själv i eld och omkom så -
1Ki 16:19  detta för de synders skull som han hade begått, i det att han gjorde vad ont var i HERRENS ögon och vandrade på Jerobeams väg och i den synd som denne hade gjort, och genom vilken han hade kommit Israel att synda.
1Ki 16:20  Vad nu mer är att säga om Simri och om den sammansvärjning som han anstiftade, det finnes upptecknat i Israels konungars krönika.
1Ki 16:21  Nu delade sig Israels folk i två hälfter; den ena hälften av folket höll sig till Tibni, Ginats son, och ville göra honom till konung, och den andra hälften höll sig till Omri.
1Ki 16:22  Men den del av folket som höll sig till Omri, fick överhanden över den del som höll sig till Tibni, Ginats son. Och när Tibni var död, blev Omri konung.
1Ki 16:23  I Asas, Juda konungs, trettioförsta regeringsår blev Omri konung över Israel och regerade i tolv år; i Tirsa regerade han i sex år.
1Ki 16:24  Han köpte berget Samaria av Semer för två talenter silver; och han bebyggde berget och kallade staden som han byggde där Samaria, efter Semer, den man som hade varit bergets ägare.
1Ki 16:25  Men Omri gjorde vad ont var i HERRENS ögon; han gjorde mer ont än någon av dem som hade varit före honom.
1Ki 16:26  Han vandrade i allt på Jerobeams, Nebats sons, väg och i de synder genom vilka denne hade kommit Israel att synda, så att de förtörnade HERREN, Israels Gud, med de fåfängliga avgudar de dyrkade.
1Ki 16:27  Vad nu mer är att säga om Omri, om vad han gjorde och om de bedrifter han utförde, det finnes upptecknat i Israels konungars krönika.
1Ki 16:28  Och Omri gick till vila hos sina fäder och blev begraven i Samaria. Och hans son Ahab blev konung efter honom.
1Ki 16:29  Ahab, Omris son, blev konung över Israel i Asas, Juda konungs, trettioåttonde regeringsår; sedan regerade Ahab, Omris son, i tjugutvå år över Israel i Samaria.
1Ki 16:30  Men Ahab, Omris son, gjorde vad ont var i HERRENS ögon, mer än någon av dem som hade varit före honom.
1Ki 16:31  Det var honom icke nog att vandra i Jerobeams, Nebats sons, synder; han tog ock till hustru Isebel, dotter till Etbaal, sidoniernas konung, och gick så åstad och tjänade Baal och tillbad honom.
1Ki 16:32  Och han reste ett altare åt Baal i Baalstemplet som han hade byggt i Samaria.
1Ki 16:33  Därtill lät Ahab göra Aseran. Så gjorde Ahab mer till att förtörna HERREN, Israels Gud, än någon av de Israels konungar som hade varit före honom.
1Ki 16:34  Under hans tid byggde beteliten Hiel åter upp Jeriko. Men när han lade dess grund, kostade det honom hans äldste son Abiram, och när han satte upp dess portar, kostade det honom hans yngste son Segib - i enlighet med det ord som HERREN hade talat genom Josua, Nuns son.
1Ki 17:1  Och tisbiten Elia, en man som förut hade uppehållit sig i Gilead, sade till Ahab: "Så sant HERREN, Israels Gud, lever, han vilkens tjänare jag är, under dessa år skall varken dagg eller regn falla, med mindre jag säger det."
1Ki 17:2  Och HERRENS ord kom till honom; han sade:
1Ki 17:3  "Gå bort härifrån och begiv dig österut, och göm dig vid bäcken Kerit, som österifrån rinner ut i Jordan.
1Ki 17:4  Din dryck skall du få ur bäcken, och korparna har jag bjudit att där förse dig med föda."
1Ki 17:5  Då gick han bort och gjorde såsom HERREN hade befallt; han gick bort och uppehöll sig vid bäcken Kerit, som österifrån rinner ut i Jordan.
1Ki 17:6  Och korparna förde till honom bröd och kött om morgonen, och bröd och kött om aftonen, och sin dryck fick han ur bäcken.
1Ki 17:7  Men efter någon tid torkade bäcken ut, därför att det icke regnade i landet.
1Ki 17:8  Då kom HERRENS ord till honom; han sade:
1Ki 17:9  "Stå upp och gå till Sarefat, som hör till Sidon, och uppehåll dig där. Se, jag har där bjudit en änka att förse dig med föda."
1Ki 17:10  Han stod upp och gick till Sarefat. Och när han kom till stadsporten, fick han där se en änka som samlade ved. Då ropade han till henne och sade: "Hämta litet vatten åt mig i kärlet, så att jag får dricka."
1Ki 17:11  När hon nu gick för att hämta det, ropade han efter henne och sade: "Tag ock med dig ett stycke bröd åt mig."
1Ki 17:12  Men hon svarade: "Så sant HERREN, din Gud, lever, jag äger icke en kaka bröd, utan allenast en hand full mjöl i krukan och litet olja i kruset. Och se, här har jag samlat ihop ett par vedpinnar, och jag går nu hem och tillreder det åt mig och min son, för att vi må äta det och sedan dö."
1Ki 17:13  Då sade Elia till henne: "Frukta icke; gå och gör såsom du har sagt. Men red först till en liten kaka därav åt mig, och bär ut den till mig; red sedan till åt dig och din son.
1Ki 17:14  Ty så säger HERREN, Israels Gud: Mjölet i krukan skall icke taga slut, och oljan i kruset skall icke tryta, intill den dag då HERREN låter det regna på jorden."
1Ki 17:15  Då gick hon åstad och gjorde såsom Elia hade sagt. Och hon hade sedan att äta, hon själv och han och hennes husfolk, en lång tid.
1Ki 17:16  Mjölet i krukan tog icke slut, och oljan i kruset tröt icke, i enlighet med det ord som HERREN hade talat genom Elia.
1Ki 17:17  Men härefter hände sig, att kvinnans, hans värdinnas, son blev sjuk; hans sjukdom blev mycket svår, så att han till slut icke mer andades.
1Ki 17:18  Då sade hon till Elia: "Vad har du med mig att göra, du gudsman? Du har kommit till mig, för att min missgärning skulle bliva ihågkommen, så att min son måste dö."
1Ki 17:19  Men han sade till henne: "Giv mig din son." Och han tog honom ur hennes famn och bar honom upp i salen där han bodde och lade honom på sin säng.
1Ki 17:20  Och han ropade till HERREN och sade: "HERRE, min Gud, har du väl kunnat göra så illa mot denna änka, vilkens gäst jag är, att du har dödat hennes son?"
1Ki 17:21  Därefter sträckte han sig ut över gossen tre gånger och ropade till HERREN och sade: "HERRE, min Gud, låt denna gosses själ komma tillbaka in i honom."
1Ki 17:22  Och HERREN hörde Elias röst, och gossens själ kom tillbaka in i honom, och han fick liv igen.
1Ki 17:23  Och Elia tog gossen och bar honom från salen ned i huset och gav honom åt hans moder. Och Elia sade: "Se, din son lever."
1Ki 17:24  Då sade kvinnan till Elia: "Nu vet jag att du är en gudsman, och att HERRENS ord i din mun är sanning."
1Ki 18:1  En lång tid härefter, på tredje året, kom HERRENS ord till Elia; han sade: "Gå åstad och träd fram för Ahab, så skall jag sedan låta det regna på jorden."
1Ki 18:2  Då gick Elia åstad för att träda fram för Ahab. Men hungersnöden var då stor i Samaria.
1Ki 18:3  Och Ahab kallade till sig Obadja, sin överhovmästare; men Obadja dyrkade HERREN med stor iver.
1Ki 18:4  Och när Isebel utrotade HERRENS profeter, hade Obadja tagit ett hundra profeter och gömt dem, femtio man åt gången, i en grotta och försett dem med mat och dryck.
1Ki 18:5  Ahab sade nu till Obadja: "Far igenom landet till alla vattenkällor och alla bäckar. Kanhända skola vi finna gräs, så att vi kunna behålla hästar och mulåsnor vid liv och slippa att slakta ned någon boskap."
1Ki 18:6  Och de fördelade mellan sig landet som de skulle draga i genom. Ahab for en väg för sig, och Obadja for en annan väg för sig.
1Ki 18:7  När nu Obadja färdades sin väg fram, fick han se Elia komma emot sig. Och han kände igen denne och föll ned på Sitt ansikte och sade: "Är du här, min herre Elia?"
1Ki 18:8  Han svarade honom: "Ja. Gå och säg till din herre: 'Elia är här.'"
1Ki 18:9  Då sade han: "Varmed har jag försyndat mig, eftersom du vill giva din tjänare i Ahabs hand och låta honom döda mig?
1Ki 18:10  Så sant HERREN, din Gud, lever, det finnes icke något folk eller något rike dit min herre icke har sänt för att söka efter dig; och om man har svarat: 'Han är icke här', så har han av det riket eller det folket tagit en ed, att man icke har funnit dig.
1Ki 18:11  Och nu säger du: 'Gå och säg till din herre: Elia är här!'
1Ki 18:12  Om nu, när jag går ifrån dig, HERRENS Ande skulle rycka bort dig, jag vet icke vart, och jag likväl komme med ditt budskap till Ahab, så skulle han dräpa mig, när han icke funne dig. Och dock har ju jag, din tjänare, fruktat HERREN allt ifrån min ungdom.
1Ki 18:13  Har det icke blivit berättat för min herre vad jag gjorde, när Isebel dräpte HERRENS profeter, huru jag gömde ett hundra av HERRENS profeter, femtio man och åter femtio, i en grotta och försåg dem med mat och dryck?
1Ki 18:14  Och nu säger du: 'Gå och säg till din herre: Elia är här!' - för att han skall dräpa mig."
1Ki 18:15  Men Elia svarade: "Så sant HERREN Sebaot lever, han vilkens tjänare jag är, redan i dag skall jag träda fram för honom."
1Ki 18:16  Då gick Obadja Ahab till mötes och förkunnade detta för honom; och Ahab begav sig åstad för att möta Elia.
1Ki 18:17  Och när Ahab fick se Elia, sade Ahab till honom: "Är du här, du som drager olycka över Israel?"
1Ki 18:18  Han svarade: "Det är icke jag, som drager olycka över Israel, utan du och din faders hus, därmed att I övergiven HERRENS bud, och därmed att du följer efter Baalerna.
1Ki 18:19  Men sänd nu bort och församla hela Israel till mig på berget Karmel, jämte Baals fyra hundra femtio profeter och Aserans fyra hundra profeter, som äta vid Isebels bord."
1Ki 18:20  Då sände Ahab omkring bland Israels barn och lät församla profeterna på berget Karmel.
1Ki 18:21  Och Elia trädde fram för allt folket och sade: "Huru länge viljen I halta på båda sidor? Är det HERREN som är Gud, så följen efter honom; men om Baal är det, så följen efter honom." Och folket svarade honom icke ett ord.
1Ki 18:22  Då sade Elia till folket: "Jag allena är kvar såsom HERRENS Profet, och Baals profeter äro fyra hundra femtio man.
1Ki 18:23  Må man nu giva oss två tjurar, och må de välja ut åt sig den ena tjuren och stycka den och lägga den på veden, utan att tända eld därpå, så vill jag reda till den andra tjuren och lägga den på veden, utan att tända eld därpå.
1Ki 18:24  Därefter mån I åkalla eder guds namn, men själv vill jag åkalla HERRENS namn. "Den gud som då svarar med eld, han vare Gud." Allt folket svarade och sade. "Ditt förslag är gott."
1Ki 18:25  Då sade Elia till Baals profeter: "Väljen ut åt eder den ena tjuren och reden till den, I först, ty I ären flertalet; åkallen därefter eder guds namn, men eld fån I icke tända."
1Ki 18:26  Då togo de den tjur som han gav dem och redde till den; sedan åkallade de Baals namn från morgonen ända till middagen och ropade: "Baal, svara oss." Men icke ett ljud hördes, och ingen svarade. Och alltjämt haltade de åstad kring altaret som man hade gjort.
1Ki 18:27  När det så blev middag, gäckades Elia med dem och sade: "Ropen ännu högre, ty visserligen är han en gud, men han har väl något att begrunda, eller ock har han gått avsides, eller är han på resa; kanhända sover han, men då skall han väl vakna."
1Ki 18:28  Då ropade de ännu högre och ristade sig, såsom deras sed var, med svärd och spjut, så att blodet kom ut på dem.
1Ki 18:29  När det sedan hade blivit eftermiddag, fattades de av profetiskt raseri, och höllo så på ända till den tid då spisoffret frambäres. Men icke ett ljud hördes, ingen svarade, och ingen tycktes heller akta på dem.
1Ki 18:30  Och Elia sade till allt folket: "Träden hitfram till mig." Så trädde nu allt folket fram till honom. Då satte han åter i stånd HERRENS altare, som hade blivit nedrivet.
1Ki 18:31  Elia tog tolv stenar, lika många som Jakobs söners stammar - den mans, till vilken detta HERRENS ord hade kommit: "Israel skall vara ditt namn."
1Ki 18:32  Och han byggde av stenarna ett altare i HERRENS namn och gjorde omkring altaret en grav, stor nog för ett utsäde av två sea-mått.
1Ki 18:33  Därefter lade han upp veden, styckade tjuren och lade den på veden.
1Ki 18:34  Sedan sade han: "Fyllen fyra krukor med vatten, och gjuten ut vattnet över brännoffret och veden." Han sade ytterligare: "Gören så ännu en gång." Och de gjorde så för andra gången. Därefter sade han: "Gören så för tredje gången." Och de gjorde så för tredje gången.
1Ki 18:35  Och vattnet flöt runt omkring altaret; och han lät fylla också graven med vatten.
1Ki 18:36  Då nu tiden var inne att frambära spisoffret, trädde profeten Elia fram och sade: "HERRE, Abrahams, Isaks och Israels Gud, låt det i dag bliva kunnigt att du är Gud i Israel, och att jag är din tjänare, och att det är på din befallning jag har gjort allt detta."
1Ki 18:37  Svara mig, HERRE, svara mig, så att detta folk förnimmer att det är du, HERRE, som är Gud, i det att du vänder om deras hjärtan."
1Ki 18:38  "Då föll HERRENS ed ned och förtärde brännoffret, veden, stenarna och jorden, och uppslickade vattnet som var i graven.
1Ki 18:39  När allt folket såg detta, föllo de ned på sina ansikten och sade: "HERREN är det som är Gud! HERREN är det som är Gud!"
1Ki 18:40  Men Elia sade till dem: "Gripen Baals profeter; låten ingen av dem komma undan." Och de grepo dem. Och Elia lät föra dem ned till bäcken Kison och slakta dem där.
1Ki 18:41  Och Elia sade till Ahab: "Begiv dig ditupp, ät och drick, ty jag hör bruset av regn."
1Ki 18:42  Då begav sig Ahab ditupp för att äta och dricka. Men Elia steg upp på Karmels topp, hukade sig ned mot jorden och sänkte sitt ansikte mellan sina knän.
1Ki 18:43  Och han sade till sin tjänare "Gå upp och skåda ut åt havet." Denne gick då upp och skådade ut, men sade: "Jag ser ingenting." Så tillsade han honom sju gånger att gå tillbaka.
1Ki 18:44  När han då kom dit sjunde gången sade han: "Nu ser jag ett litet moln, icke större än en mans hand, stiga upp ur havet." Då sade han: "Gå upp och säg till Ahab: Spänn för och far ned, så att regnet icke håller dig kvar."
1Ki 18:45  Och i ett ögonblick förmörkades himmelen av moln och storm, och ett starkt regn föll. Och Ahab steg upp i sin vagn och for till Jisreel.
1Ki 18:46  Men HERRENS hand hade kommit över Elia, så att han omgjorde sina länder och sprang framför Ahab ända inemot Jisreel.
1Ki 19:1  Men när Ahab berättade för Isebel allt vad Elia hade gjort, och huru han hade dräpt alla profeterna med svärd,
1Ki 19:2  Sände Isebel en budbärare till Elia och lät säga: "Gudarna straffe mig nu och framgent om jag icke i morgon vid denna tid låter det gå med ditt liv såsom det gick med alla dessas liv."
1Ki 19:3  När han förnam detta, stod han upp och begav sig i väg för att rädda sitt liv, och han kom så till Beer-Seba, som hör till Juda; där lämnade han kvar sin tjänare.
1Ki 19:4  Men själv gick han ut i öknen en dagsresa. Där satte han sig under en ginstbuske; och han önskade sig döden och sade: "Det är nog; tag nu mitt liv, HERRE, ty jag är icke förmer än mina fäder."
1Ki 19:5  Därefter lade han sig att sova under en ginstbuske. Men se, då rörde en ängel vid honom och sade till honom: "Stå upp och ät."
1Ki 19:6  När han då såg upp, fick han vid sin huvudgärd se ett bröd, sådant som bakas på glödande stenar, och ett krus med vatten. Och han åt och drack och lade sig åter ned.
1Ki 19:7  Men HERRENS ängel rörde åter vid honom, för andra gången, och sade: "Stå upp och ät, ty eljest bliver vägen dig för lång."
1Ki 19:8  Då stod han upp och åt och drack, och gick så, styrkt av den maten, i fyrtio dagar och fyrtio nätter, ända till Guds berg Horeb.
1Ki 19:9  Där gick han in i en grotta, och i den stannade han över natten. Då kom HERRENS ord till honom; han sade till honom: "Vad vill du här, Elia?"
1Ki 19:10  Han svarade: "Jag har nitälskat för HERREN, härskarornas Gud. Ty Israels barn hava övergivit ditt förbund, rivit ned dina altaren och dräpt dina profeter med svärd; jag allena är kvar, och de stå efter att taga mitt liv."
1Ki 19:11  Han sade: "Gå ut och ställ dig på berget inför HERREN." Då gick HERREN fram där, och en stor och stark storm, som ryckte loss berg och bröt sönder klippor, gick före HERREN; men icke var HERREN i stormen. Efter stormen kom en jordbävning; men icke var HERREN i jordbävningen.
1Ki 19:12  Efter jordbävningen kom en eld; men icke var HERREN i elden. Efter elden kom ljudet av en sakta susning.
1Ki 19:13  Så snart Elia hörde detta, skylde han sitt ansikte med manteln och gick ut och ställde sig vid ingången till grottan. Då kom en röst till honom och sade: "Vad vill du här, Elia?"
1Ki 19:14  Han svarade: "Jag har nitälskat för HERREN, härskarornas Gud. Ty Israels barn hava övergivit ditt förbund, rivit ned dina altaren och dräpt dina profeter med svärd; jag allena är kvar, och de stå efter att taga mitt liv."
1Ki 19:15  HERREN sade till honom: "Gå nu tillbaka igen, och tag vägen till Damaskus' öken, och gå in och smörj Hasael till konung över Aram.
1Ki 19:16  Och Jehu, Nimsis son, skall du smörja till konung över Israel. Och till profet i ditt ställe skall du smörja Elisa, Safats son, från Abel-Mehola.
1Ki 19:17  Och så skall ske: den som kommer undan Hasaels svärd, honom skall Jehu döda, och den som kommer undan Jehus svärd, honom skall Elisa döda.
1Ki 19:18  Men jag skall låta sju tusen män bliva kvar i Israel, alla de knän som icke hava böjt sig för Baal, och var mun som icke har givit honom hyllningskyss."
1Ki 19:19  När han sedan gick därifrån, träffade han på Elisa, Safats son, som höll på att plöja; tolv par oxar gingo framför honom, och själv körde han det tolfte paret. Och Elia gick fram till honom och kastade sin mantel över honom.
1Ki 19:20  Då släppte han oxarna och skyndade efter Elia och sade: "Låt mig först få kyssa min fader och min moder, så vill jag sedan följa dig." Han sade till honom: "Välan, du må gå tillbaka igen; du vet ju vad jag har gjort med dig."
1Ki 19:21  Då lämnade han honom och gick tillbaka och tog sina båda oxar och slaktade dem, och med oxarnas ok kokade han deras kött; detta gav han åt folket, och de åto. Därefter stod han upp och följde Elia och blev hans tjänare.
1Ki 20:1  Och Ben-Hadad, konungen i Aram, samlade hela sin här; han hade med sig trettiotvå konungar jämte hästar och vagnar. Han drog upp och belägrade Samaria och ansatte det.
1Ki 20:2  Och han skickade sändebud in i staden till Ahab, Israels konung,
1Ki 20:3  och lät säga honom: "Så säger Ben-Hadad: Ditt silver och ditt guld tillhör mig, och det bästa du har av kvinnor och barn tillhör mig ock."
1Ki 20:4  Israels konung svarade och sade: "Såsom du har sagt, min herre konung: jag själv och allt vad jag har tillhör dig."
1Ki 20:5  Men sändebuden kommo tillbaka och sade: "Så säger Ben-Hadad: Jag har ju sänt till dig och låtit säga: 'Ditt silver och ditt guld, dina kvinnor och dina barn skall du giva mig.'
1Ki 20:6  Och nu skall jag sannerligen i morgon vid denna tid sända mina tjänare till dig, för att de må genomsöka ditt hus och dina tjänares hus; och allt som är dina ögons lust skola de taga med sig och föra bort."
1Ki 20:7  Då kallade Israels konung till sig alla de äldste i landet och sade: "Märken och sen huru denne står efter vårt fördärv. Ty när han sände till mig och begärde mina kvinnor och mina barn, mitt silver och mitt guld, vägrade jag ju icke att giva honom det."
1Ki 20:8  Alla de äldste och allt folket sade till honom: "Hör icke på honom och gör honom icke till viljes."
1Ki 20:9  Så svarade han då Ben-Hadads sändebud: "Sägen till min herre konungen: Allt, varom du förra gången sände bud till din tjänare, det vill jag foga mig i; men detta kan jag icke foga mig i." Och sändebuden vände tillbaka med detta svar.
1Ki 20:10  Då sände Ben-Hadad till honom och lät säga: "Gudarna straffe mig nu och framgent, om Samarias grus skall räcka till att fylla händerna på allt det folk som följer mig."
1Ki 20:11  Men Israels konung svarade och sade: "Sägen så: Icke må den som omgjordar sig med svärdet berömma sig likt den som spänner det av sig."
1Ki 20:12  Så snart Ben-Hadad hörde detta svar, där han satt och drack med konungarna i lägerhyddorna, sade han till sina tjänare: "Gören eder redo." Och de gjorde sig redo till att angripa staden.
1Ki 20:13  Då trädde en profet fram till Ahab, Israels konung, och sade: "Så säger HERREN: Ser du hela denna stor hop? Se, jag vill i dag giva den i din hand, på det att du må förnimma att jag är HERREN."
1Ki 20:14  Då frågade Ahab: "Genom vem? Han svarade: "Så säger HERREN: Genom landshövdingarnas män." Han frågade ytterligare: "Vem skall begynna striden?" Han svarade: "Du själv."
1Ki 20:15  Så mönstrade han då landshövdingarnas män, och de voro två hundra trettiotvå, därefter mönstrade han allt folket, alla Israels barn, sju tusen man.
1Ki 20:16  Och vid middagstiden gjorde de ett utfall, just när Ben-Hadad höll på att dricka sig drucken i lägerhyddorna, tillsammans med de trettiotvå konungar som sade kommit honom till hjälp.
1Ki 20:17  Landshövdingarnas män drogo först ut. Och de kunskapare som Ben-Hadad sände ut underrättade honom om att folk kom ut från Samaria.
1Ki 20:18  Då sade han: "Om de hava dragit ut i fredlig avsikt, så gripen dem levande; och om de hava dragit ut till strid, så gripen dem ock levande."
1Ki 20:19  Men när dessa - landshövdingarnas män och hären som följde dem - hade kommit ut ur staden,
1Ki 20:20  höggo de ned var och en sin man, och araméerna flydde, och Israel förföljde dem. Och Ben-Hadad, konungen i Aram, kom undan på en häst, jämte några ryttare.
1Ki 20:21  Och Israels konung drog ut och slog både ryttarhären och vagnshären och tillfogade araméerna ett stort nederlag.
1Ki 20:22  Men profeten trädde fram till Israels konung och sade till honom: "Grip dig nu an; och betänk och se till, vad du bör göra, ty nästa år kommer konungen i Aram att åter draga upp mot dig."
1Ki 20:23  Men den arameiske konungens tjänare sade till honom: "Deras gud är en bergsgud; därför hava de blivit oss övermäktiga. Låt oss nu strida mot dem på slätten, så skola vi förvisso bliva dem övermäktiga.
1Ki 20:24  Och vidare måste du göra så: avsätt var och en av konungarna från hans plats, och insätt ståthållare i deras ställe.
1Ki 20:25  Skaffa dig sedan själv en här, lika stor som den du har förlorat, med lika många hästar och lika många vagnar, och låt oss sedan strida mot dem på slätten, så skola vi förvisso bliva dem övermäktiga." Och han lyssnade till deras ord och gjorde så.
1Ki 20:26  Följande år mönstrade Ben-Hadad araméerna och drog så upp till Afek för att strida mot Israel.
1Ki 20:27  Israels barn hade ock blivit mönstrade och försedda med livsmedel och tågade därefter emot dem. Och Israels barn lägrade sig gent emot dem, lika två små gethjordar, under det att araméerna uppfyllde landet.
1Ki 20:28  Då trädde gudsmannen fram och sade till Israels konung: "Så säger HERREN: Därför att araméerna hava sagt: 'HERREN är en bergsgud och icke en dalgud', därför giver jag hela denna stora hop i din hand, på det att I mån förnimma att jag är HERREN."
1Ki 20:29  Och de voro lägrade mitt emot varandra i sju dagar. På sjunde dagen kom det till strid, och Israels barn slogo då av araméerna hundra tusen man fotfolk, detta på en enda dag.
1Ki 20:30  De återstående flydde in i staden Afek; men stadsmuren föll ned över tjugusju tusen man, dem som återstodo. Ben-Hadad flydde också och kom in i staden och sprang från kammare till kammare.
1Ki 20:31  Då sade hans tjänare till honom: "Vi hava hört att konungarna av Israels hus äro nådiga konungar. Låt oss därför sätta säcktyg om våra länder och rep om våra huvuden och giva oss åt Israels konung; kanhända låter han dig då få leva."
1Ki 20:32  Och de bundo säcktyg omkring sina länder och rep omkring sina huvuden och kommo så till Israels konung och sade: "Din tjänare Ben-Hadad beder: 'Låt mig få leva.'" Han svarade: "Är han ännu vid liv, han min broder?"
1Ki 20:33  Männen, som i detta hans ord sågo ett gott varsel, skyndade att taga fasta därpå och sade: "Ja, din broder är Ben-Hadad." Han sade: "Gån och hämten honom hit." Då gav sig Ben-Hadad åt honom, och han lät honom stiga upp i sin vagn.
1Ki 20:34  Och Ben-Hadad sade till honom: "De städer som min fader tog från din fader vill jag giva tillbaka, och du skall för din räkning få inrätta handelskvarter i Damaskus, såsom min fader fick göra i Samaria." "Välan", sade Ahab, "på sådana villkor vill jag giva dig fri." Och han slöt ett fördrag med honom och gav honom fri.
1Ki 20:35  Och en av profetlärjungarna sade på HERRENS befallning till en annan: "Slå till mig." Men mannen vägrade att slå honom.
1Ki 20:36  Då sade han till honom: "Eftersom du icke har lyssnat till HERRENS röst, därför skall ett lejon slå ned dig, när du går ifrån mig." Och när han gick sin väg ifrån honom, kom ett lejon emot honom och slog ned honom.
1Ki 20:37  Sedan träffade han en annan man och sade: "Slå till mig." Och mannen slog honom så hårt, att sår uppstod därav.
1Ki 20:38  Därefter gick profeten och ställde sig i konungens väg, sedan han hade gjort sig oigenkännlig genom att sätta en bindel över ögonen.
1Ki 20:39  När nu konungen kom därfram, ropade han till konungen och sade: "Din tjänare hade givit sig ut i striden, då i detsamma en man kom därifrån och förde till mig en annan man och sade: 'Vakta denne man; om han kommer bort, skall det gå dig såsom det skulle hava gått honom, eller ock måste du betala en talent silver.'
1Ki 20:40  Nu hände sig, under det din tjänare hade att syssla än här än där, att mannen kom undan." Israels konung sade till honom: "Din dom är given; du har ju själv avkunnat den."
1Ki 20:41  Då tog han skyndsamt bort bindeln från sina ögon, och Israels konung kände igen honom och såg att han var en av profeterna.
1Ki 20:42  Och han sade till konungen: "Så säger HERREN: Därför att du har släppt ur din hand den man som av mig var given till spillo, skall det gå dig såsom det skulle hava gått honom, och ditt folk såsom det har gått hans folk."
1Ki 20:43  Och Israels konung begav sig hem, missmodig och vred, och kom till Samaria.
1Ki 21:1  Därefter hände sig följande. Jisreeliten Nabot hade en vingård i Jisreel bredvid Ahabs palats, konungens i Samaria.
1Ki 21:2  Och Ahab talade till Nabot och sade: "Låt mig få din vingård för att därav göra mig en köksträdgård, eftersom den ligger så nära intill mitt hus; jag vill giva dig en bättre vingård i stället, eller om dig så behagar, vill jag giva dig penningar såsom betalning för den."
1Ki 21:3  Men Nabot svarade Ahab: "HERREN låte det vara fjärran ifrån mig att jag skulle låta dig få mina fäders arvedel."
1Ki 21:4  Då gick Ahab hem till sitt, missmodig och vred för det svars skull som jisreeliten Nabot hade givit honom, när denne sade: "Jag vill icke låta dig få mina fäders arvedel." Och han lade sig på sin säng och vände bort sitt ansikte och åt intet.
1Ki 21:5  Då kom hans hustru Isebel in till honom och frågade honom: "Varför är du så missmodig, och varför äter du intet?"
1Ki 21:6  Han svarade henne: "Därför att när jag talade till jisreeliten Nabot och sade till honom: 'Låt mig få din vingård för penningar, eller om du så önskar, vill jag giva dig en annan vingård i stället', då svarade han: 'Jag vill icke låta dig få min vingård.'
1Ki 21:7  Då sade hans hustru Isebel honom: "Är det du som nu regerar över Israel? Stå upp och ät och var vid gott mod; jag skall skaffa dig jisreeliten Nabot vingård.
1Ki 21:8  Därefter skrev hon ett brev i Ahabs namn och satte sigill under det med hans signetring, och sände så brevet till de äldste och förnämsta i Nabots stad, de som bodde där jämte honom.
1Ki 21:9  Och hon skrev i brevet så: "Lysen ut en fasta, och låten Nabot sitta längst fram bland folket.
1Ki 21:10  Och låten så två onda män sätta sig mitt emot honom, och låten dem vittna emot honom och säga: 'Du har talat förgripligt mot Gud och konungen.' Fören så ut honom och stenen honom till döds."
1Ki 21:11  Och de äldsta och förnämsta männen i staden, de som bodde där i hans stad, handlade i enlighet med det bud som Isebel hade sänt dem, och såsom det var skrivet i brevet som hon hade sänt till dem.
1Ki 21:12  De lyste ut en fasta och läto Nabot sitta längst fram bland folket.
1Ki 21:13  Och de två onda männen kommo och satte sig mitt emot honom; och de onda männen vittnade mot Nabot inför folket och sade: Nabot har talat förgripligt mot Gud och konungen." Då förde man honom utanför staden och stenade honom till döds.
1Ki 21:14  Därefter sände de bud till Isebel och läto säga: "Nabot har blivit stenad till döds."
1Ki 21:15  Så snart Isebel hörde att Nabot var stenad till döds, sade hon till Ahab: "Stå upp och tag jisreeliten Nabots vingård i besittning, den som han vägrade att låta dig få för penningar; ty Nabot är icke längre vid liv, utan han är död."
1Ki 21:16  Så snart Ahab hörde att Nabot var död, stod han upp och begav sig åstad ned till jisreeliten Nabots vingård för att taga den i besittning.
1Ki 21:17  Men HERRENS ord kom till tisbiten Elia; han sade:
1Ki 21:18  "Stå upp, gå åstad och möt Ahab, Israels konung, som bor i Samaria. Du träffar honom i Nabots vingård, dit han har gått ned för att taga den i besittning.
1Ki 21:19  Och du skall tala till honom och säga: 'Så säger HERREN: Har du till redan hunnit att både dräpa och tillträda arvet?' Därefter skall du tala till honom och säga: 'Så säger HERREN: På samma ställe där hundarna hava slickat Nabots blod skola hundarna slicka också ditt blod.'"
1Ki 21:20  Ahab sade till Elia: "Har du äntligen funnit mig, du min fiende?" Han svarade: "Ja, jag har funnit dig. Eftersom du har sålt dig till att göra vad ont är i HERRENS ögon,
1Ki 21:21  därför skall jag ock låta vad ont är komma över dig och skall bortsopa dig, och av Ahabs hus skall jag utrota allt mankön, både små och stora i Israel.
1Ki 21:22  Och jag skall göra med ditt hus såsom jag gjorde med Jerobeams, Nebats sons, hus, och såsom jag gjorde med Baesas, Ahias sons, hus, därför att du har förtörnat mig och kommit Israel att synda.
1Ki 21:23  Också om Isebel har HERREN talat och sagt: Hundarna skola äta upp Isebel invid Jisreels murar.
1Ki 21:24  Ja, den av Ahabs hus, som dör i staden, skola hundarna äta upp, och den som dör ute på marken skola himmelens fåglar äta upp."
1Ki 21:25  (Också har ingen varit såsom Ahab, han som sålde sig till att göra vad ont var i HERRENS ögon, när hans hustru Isebel uppeggade honom därtill.
1Ki 21:26  Mycken styggelse förövade han, i det han följde efter de eländiga avgudarna, alldeles såsom amoréerna hade gjort, vilka HERREN fördrev för Israels barn.)
1Ki 21:27  Men när Ahab hörde de orden, rev han sönder sina kläder och svepte säcktyg om sin kropp och fastade; och han låg höljd i säcktyg och gick tyst omkring.
1Ki 21:28  Då kom HERRENS ord till tisbiten Elia; han sade:
1Ki 21:29  "Har du sett huru Ahab ödmjukar sig inför mig? Därför att han så ödmjukar sig inför mig, skall jag icke låta olyckan komma i hans tid; först i hans sons tid skall jag låta olyckan komma över hans hus."
1Ki 22:1  Och de sutto i ro i tre år, under vilka intet krig var mellan Aram och Israel.
1Ki 22:2  Men i det tredje året for Josafat, Juda konung, ned till Israels konung.
1Ki 22:3  Och Israels konung sade till sina tjänare: "I veten ju att Ramot i Gilead tillhör oss. Och likväl sitta vi stilla och taga det icke ifrån konungen i Aram."
1Ki 22:4  Och han frågade Josafat: Vill du draga med mig för att belägra Ramot i Gilead?" Josafat svarade Israels konung: "Jag såsom du, mitt folk såsom ditt folk, mina hästar såsom dina hästar!"
1Ki 22:5  Men Josafat sade ytterligare till Israels konung: "Fråga dock först HERREN härom."
1Ki 22:6  Då församlade Israels konung profeterna, vid pass fyra hundra män, och frågade dem: "Skall jag draga åstad mot Ramot i Gilead för att belägra det, eller skall jag avstå därifrån?" De svarade: "Drag ditupp; Herren skall giva det i konungens hand."
1Ki 22:7  Men Josafat sade: "Finnes här ingen annan HERRENS profet, så att vi kunna fråga genom honom?"
1Ki 22:8  Israels konung svarade Josafat: "Här finnes ännu en man, Mika, Jimlas son, genom vilken vi kunna fråga HERREN; men han är mig förhatlig, ty han profeterar aldrig lycka åt mig, utan allenast olycka." Josafat sade: "Konungen säge icke så."
1Ki 22:9  Då kallade Israels konung till sig en hovman och sade: "Skaffa skyndsamt hit Mika, Jimlas son."
1Ki 22:10  Israels konung och Josafat, Juda konung, sutto nu var och en på sin tron, iklädda sina skrudar, på en tröskplats vid Samarias port, under det att alla profeterna profeterade inför dem.
1Ki 22:11  Då gjorde sig Sidkia, Kenaanas son, horn av järn och sade: "Så säger HERREN: Med dessa skall du stånga araméerna, så att de förgöras."
1Ki 22:12  Och alla profeterna profeterade på samma sätt och sade: "Drag upp mot Ramot i Gilead, så skall du bliva lyckosam; HERREN skall giva det i konungens hand."
1Ki 22:13  Och budet som hade gått för att tillkalla Mika talade till honom och sade: "Det är så, att profeterna med en mun lova konungen lycka; låt nu dina ord stämma överens med vad de hava talat, och lova också du lycka."
1Ki 22:14  Men Mika svarade: "Så sant HERREN lever, jag skall allenast tala det som HERREN säger till mig."
1Ki 22:15  När han sedan kom till konungen, frågade konungen honom: "Mika, skola vi draga åstad till Ramot i Gilead för att belägra det, eller skola vi avstå därifrån?" Han svarade honom: "Drag ditupp, så skall du bliva lyckosam; HERREN skall giva det i konungens hand."
1Ki 22:16  Men konungen sade till honom: "Huru många gånger skall jag besvärja dig att icke tala till mig annat än sanning i HERRENS namn?"
1Ki 22:17  Då sade han: "Jag såg hela Israel förskingrat på bergen, likt får som icke hava någon herde. Och HERREN sade: 'Dessa hava icke någon herre; må de vända tillbaka hem i frid, var och en till sitt.'"
1Ki 22:18  Då sade Israels konung till Josafat: "Sade jag dig icke att denne aldrig profeterar lycka åt mig, utan allenast olycka?"
1Ki 22:19  Men han sade: "Hör alltså HERRENS ord. Jag såg HERREN sitta på sin tron och himmelens hela härskara stå där hos honom, på hans högra sida och på hans vänstra.
1Ki 22:20  Och HERREN sade: 'Vem vill locka Ahab att draga upp mot Ramot i Gilead, för att han må falla där?' Då sade den ene så och den andre så.
1Ki 22:21  Slutligen kom anden fram och ställde sig inför HERREN och sade: 'Jag vill locka honom därtill.' HERREN frågade honom: 'På vad sätt?'
1Ki 22:22  Han svarade: 'Jag vill gå ut och bliva en lögnens ande i alla hans profeters mun.' Då sade han: 'Du må försöka att locka honom därtill, och du skall också lyckas; gå ut och gör så.'
1Ki 22:23  Och se, nu har HERREN lagt en lögnens ande i alla dessa dina profeters mun, medan HERREN ändå har beslutit att olycka skall komma över dig."
1Ki 22:24  Då trädde Sidkia, Kenaanas son, fram och gav Mika ett slag på kinden och sade: "På vilken väg har då HERRENS Ande gått bort ifrån mig för att tala med dig?"
1Ki 22:25  Mika svarade: "Du skall få se det på den dag då du nödgas springa från kammare till kammare för att gömma dig."
1Ki 22:26  Men Israels konung sade: "Tag Mika och för honom tillbaka till Amon, hövitsmannen i staden, och till Joas, konungasonen.
1Ki 22:27  Och säg: Så säger konungen: Sätten denne i fängelse och bespisen honom med fångkost, till dess jag kommer välbehållen hem."
1Ki 22:28  Mika svarade: "Om du kommer välbehållen tillbaka, så har HERREN icke talat genom mig." Och han sade ytterligare: "Hören detta, I folk, allasammans."
1Ki 22:29  Så drog nu Israels konung jämte Josafat, Juda konung, upp till Ramot i Gilead.
1Ki 22:30  Och Israels konung sade till Josafat: "Jag vill förkläda mig, när jag drager ut i striden, men du må vara klädd i dina egna kläder." Så förklädde sig Israels konung, när han drog ut i striden.
1Ki 22:31  Men konungen i Aram hade bjudit och sagt till sina trettiotvå vagnshövitsmän: "I skolen icke giva eder i strid med någon, vare sig liten eller stor, utom med Israels konung allena."
1Ki 22:32  När då hövitsmännen över vagnarna fingo se Josafat, tänkte de: "Förvisso är detta Israels konung", och vände sig därför till anfall mot honom. Då gav Josafat upp ett rop.
1Ki 22:33  Så snart nu hövitsmännen över vagnarna märkte att det icke var Israels konung, vände de om och läto honom vara.
1Ki 22:34  Men en man som spände sin båge och sköt på måfå träffade Israels konung i en fog på rustningen. Då sade denne till sin körsven: "Sväng om vagnen och för mig ut ur hären, ty jag är sårad."
1Ki 22:35  Och striden blev på den dagen allt häftigare, och konungen stod upprätt i sin vagn, vänd mot araméerna; men om aftonen gav han upp andan. Och blodet från såret hade runnit ned i vagnen.
1Ki 22:36  Och vid solnedgången gick ett rop genom hären: "Var och en till sin stad igen! Var och en till sitt land igen!"
1Ki 22:37  Så dödades då konungen och blev förd till Samaria; och man begrov konungen där i Samaria.
1Ki 22:38  Och när man sköljde vagnen i dammen i Samaria, slickade hundarna hans blod, och skökorna badade sig däri - såsom HERREN hade sagt.
1Ki 22:39  Vad nu mer är att säga om Ahab och om allt vad han gjorde, om elfenbenshuset som han byggde, och om alla de städer som han byggde, det finnes upptecknat i Israels konungars krönika.
1Ki 22:40  Och Ahab gick till vila hos sina fäder. Och hans son Ahasja blev konung efter honom.
1Ki 22:41  Men Josafat, Asas sons blev konung över Juda i Ahabs, Israels konungs, fjärde regeringsår.
1Ki 22:42  Trettiofem år gammal var Josafat, när han blev konung, och han regerade tjugufem år i Jerusalem. Hans moder hette Asuba, Silhis dotter.
1Ki 22:43  Och han vandrade i allt på sin fader Asas väg, utan att vika av ifrån den; han gjorde nämligen vad rätt var i HERRENS ögon.
1Ki 22:44  Dock blevo offerhöjderna icke avskaffade, utan folket fortfor att frambära offer och tända offereld på höjderna.
1Ki 22:45  Och Josafat höll fred med Israels konung.
1Ki 22:46  Vad nu mer är att säga om Josafat och om de bedrifter han utförde och om hans krig, det finnes upptecknat i Juda konungars krönika.
1Ki 22:47  Han utrotade ock ur landet de tempelbolare som ännu funnos där, vilka hade lämnats kvar i hans fader Asas tid.
1Ki 22:48  I Edom fanns då ingen konung, utan en ståthållare regerade där.
1Ki 22:49  Och Josafat hade låtit bygga Tarsis-skepp, som skulle gå till Ofir för att hämta guld; men de kommo aldrig åstad, ty de ledo skeppsbrott vid Esjon-Geber.
1Ki 22:50  Då sade Ahasja, Ahabs son, till Josafat: "Låt mitt folk fara med ditt folk på skeppen." Men Josafat ville icke.
1Ki 22:51  Och Josafat gick till vila hos sina fäder och blev begraven hos sina fäder i sin fader Davids stad. Och hans son Joram blev konung efter honom.
1Ki 22:52  Ahasja, Ahabs son, blev konung över Israel i Samaria i Josafats, Juda konungs, sjuttonde regeringsår, och han regerade över Israel i två år.
1Ki 22:53  Han gjorde vad ont var i HERRENS ögon och vandrade på sin faders och sin moders väg och på Jerobeams, Nebats sons, väg, hans som hade kommit Israel att synda.
1Ki 22:54  Och han tjänade Baal och tillbad honom och förtörnade HERREN, Israels Gud, alldeles såsom hans fader hade gjort.


\end{document}