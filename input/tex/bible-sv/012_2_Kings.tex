\begin{document}

\title{2 Kings}

2Ki 1:1  Efter Ahabs död avföll Moab från Israel.
2Ki 1:2  Och Ahasja störtade ned genom gallret i sin Övre sal i Samaria och skadade sig, så att han blev sjuk. Då skickade han åstad sändebud och sade till dem: "Gån och frågen Baal-Sebub, guden i Ekron, om jag skall tillfriskna från denna sjukdom."
2Ki 1:3  Men HERRENS ängel hade talat så till tisbiten Elia: "Stå upp och gå emot konungens i Samaria sändebud och tala så till dem: 'Är det därför att ingen Gud finnes i Israel som I gån åstad och frågen Baal-Sebub, guden i Ekron?
2Ki 1:4  Därför att I gören detta, säger HERREN så: Du skall icke komma upp ur den säng i vilken du har lagt dig, ty du skall döden dö.'" Och Elia gick.
2Ki 1:5  När sedan sändebuden kommo tillbaka till konungen, frågade han dem: "Varför kommen I tillbaka?"
2Ki 1:6  De svarade honom: "En man kom emot oss och sade till oss: 'Gån tillbaka till konungen, som har sänt eder, och talen så till honom: Så säger HERREN: Är det därför att ingen Gud finnes i Israel som du sänder bud för att fråga Baal-Sebub, guden i Ekron? Därför att du så gör skall du icke komma upp ur den säng i vilken du har lagt dig, ty du skall döden dö.'"
2Ki 1:7  Då frågade han dem: "Huru såg den mannen ut, som kom emot eder och talade till eder på detta sätt?"
2Ki 1:8  De svarade honom: "Mannen bar en hårmantel och var omgjordad med en lädergördel om sina länder." Då sade han: "Det var tisbiten Elia."
2Ki 1:9  Och han sände till honom en underhövitsman med femtio man. Och när denne kom upp till honom, där han satt på toppen av berget, sade han till honom: "Du gudsman, konungen befaller dig att komma ned."
2Ki 1:10  Men Elia svarade och sade till underhövitsmannen: "Om jag är en gudsman, så komme eld ned från himmelen och förtäre dig och dina femtio." Då kom eld ned från himmelen och förtärde honom och hans femtio.
2Ki 1:11  Och han sände åter till honom en annan underhövitsman med femtio man. Denne tog till orda och sade till honom: "Du gudsman, så säger konungen: Kom strax ned."
2Ki 1:12  Men Elia svarade och sade till dem: "Om jag är en gudsman, så komme eld ned från himmelen och förtäre dig och dina femtio." Då kom Guds eld ned från himmelen och förtärde honom och hans femtio.
2Ki 1:13  Åter sände han åstad en tredje underhövitsman med femtio man Och denne tredje underhövitsman drog ditupp, och när han kom fram, föll han ned på sina knän för Elia och bad honom och sade till honom: "Du gudsman, låt mitt liv och dessa dina femtio tjänares liv vara något aktat i dina ögon.
2Ki 1:14  Se, eld har kommit ned från himmelen och förtärt de första två underhövitsmännen med deras femtio man; men låt nu mitt liv vara något aktat i dina ögon.
2Ki 1:15  Och HERRENS ängel sade till Elia: "Gå ned med honom, frukta icke för honom." Då stod han upp och gick med honom ned till konungen
2Ki 1:16  Och han sade till denne: "Så säger HERREN: Eftersom du skickade sändebud för att fråga Baal-Sebub, guden i Ekron - likasom om i Israel icke funnes någon Gud, som du kunde fråga härom - fördenskull skall du icke få komma upp ur den säng i vilken du har lagt dig, ty du skall döden dö."
2Ki 1:17  Och han dog, i enlighet med det HERRENS ord som Elia hade talat; och Joram blev konung efter honom, i Jorams, Josafats sons, Juda konungs, andra regeringsår. Han hade nämligen ingen son.
2Ki 1:18  Vad nu mer är att säga om Ahasja, om vad han gjorde, det finnes upptecknat i Israels konungars krönika.
2Ki 2:1  Vid den tid då HERREN ville upptaga Elia till himmelen i en stormvind gingo Elia och Elisa från Gilgal.
2Ki 2:2  Och Elia sade till Elisa: "Stanna här, ty HERREN har sänt mig till Betel." Men Elisa svarade: "Så sant HERREN lever, och så sant du själv lever, jag lämnar dig icke. Och de gingo ned till Betel.
2Ki 2:3  Då kommo profetlärjungarna i Betel ut till Elisa och sade till honom: "Vet du att HERREN i dag vill taga din herre ifrån dig, upp över ditt huvud?" Han svarade: "Ja, jag vet det; tigen stilla."
2Ki 2:4  Och Elia sade till honom: "Elisa, stanna här, ty HERREN har sänt mig till Jeriko." Men han svarade: "Så sant HERREN lever, och så sant du själv lever, jag lämnar dig icke." Och de kommo till Jeriko.
2Ki 2:5  Då gingo profetlärjungarna i Jeriko fram till Elisa och sade till honom: "Vet du att HERREN i dag vill taga din herre ifrån dig, upp över ditt huvud?" Han svarade: "Ja, jag vet det; tigen stilla."
2Ki 2:6  Och Elia sade till honom: "Stanna här, ty HERREN har sänt mig till Jordan." Men han svarade: "Så sant HERREN lever, och så sant du själv lever, jag lämnar dig icke." Och de gingo båda åstad.
2Ki 2:7  Men femtio män av profetlärjungarna gingo ock åstad och ställde sig på något avstånd, längre bort, under det att de båda stodo vid Jordan.
2Ki 2:8  Och Elia tog sin mantel och vek ihop den och slog på vattnet; då delade sig detta åt två sidor. Och de gingo så båda på torr mark därigenom.
2Ki 2:9  När de hade kommit över, sade Elia till Elisa: "Bed mig om vad jag skall göra för dig, innan jag bliver tagen ifrån dig." Elisa sade "Må en dubbel arvslott av din ande falla mig till."
2Ki 2:10  Han svarade: "Du har bett om något svårt. Men om du ser mig, när jag bliver tagen ifrån dig, då kommer det dock att så ske dig; varom icke, så sker det ej."
2Ki 2:11  Under det att de nu gingo och talade, syntes plötsligt en vagn eld, med hästar av eld, och skilde de båda från varandra; och Elia for i stormvinden upp till himmelen.
2Ki 2:12  Och Elisa såg det och ropade: "Min fader, min fader! Du som för Israel är både vagnar och ryttare!" Sedan såg han honom icke mer. Och han fattade i sina kläder och rev sönder dem i två stycken.
2Ki 2:13  Därefter tog han upp Elias mantel, som hade fallit av denne, och vände så om och ställde sig vid Jordans strand.
2Ki 2:14  Och han tog Elias mantel, som hade fallit av denne, och slog på vattnet och sade: "Var är HERREN, Elias Gud?" Då nu också Elisa slog på vattnet, delade det sig åt två sidor, och han gick över.
2Ki 2:15  När profetlärjungarna, som voro vid Jeriko på något avstånd, sågo detta, sade de: "Elias ande vilar på Elisa." Och de kommo honom till mötes och bugade sig ned till jorden för honom.
2Ki 2:16  Och de sade till honom: "Se, bland dina tjänare finnas femtio raska män; låt dessa gå och söka efter din herre. Kanhända har HERRENS Ande lyft upp honom och kastat honom på något berg eller i någon dal." Men han svarade: "Sänden ingen åstad.
2Ki 2:17  Men när de länge och väl enträget hade bett honom därom, sade han: "Så sänden då åstad." Då sände de åstad femtio män; och dessa sökte efter honom i tre dagar, men funno honom icke.
2Ki 2:18  När de sedan kommo tillbaka till honom, medan han ännu vistades i Jeriko, sade han till dem: "Sade jag icke till eder att I icke skullen gå?"
2Ki 2:19  Och männen i staden sade till Elisa: "Stadens läge är ju gott, såsom min herre ser, men vattnet är dåligt, och därav komma missfall i landet."
2Ki 2:20  Han sade: "Hämten hit åt mig en ny skål och läggen salt däri." Och de hämtade en åt honom.
2Ki 2:21  Därefter gick han ut till vattenkällan och kastade salt däri och sade: "Så säger HERREN: Jag har nu gjort detta vatten sunt; död och missfall skola icke mer komma därav.
2Ki 2:22  Och vattnet blev sunt, och har förblivit så ända till denna dag, i enlighet med det ord Elisa talade.
2Ki 2:23  Därifrån begav han sig upp till Betel. Och under det han var på väg ditupp, kom en skara gossar ut ur staden; och de begynte driva gäck med honom och ropade till honom: "Upp med dig, du flintskalle! Upp med dig, du flintskalle!"
2Ki 2:24  När han då vände sig om och fick se dem, uttalade han en förbannelse över dem i HERRENS namn. Då kommo två björninnor ut ur skogen och sleto sönder fyrtiotvå av barnen.
2Ki 2:25  Därifrån gick han till berget Karmel och vände sedan därifrån tillbaka till Samaria.
2Ki 3:1  Joram, Ahabs son, blev konung över Israel i Samaria i Josafats, Juda konungs, adertonde regeringsår, och han regerade i tolv år.
2Ki 3:2  Han gjorde vad ont var i HERRENS ögon; dock icke såsom hans fader och moder, ty han skaffade bort den Baalsstod som hans fader hade låtit göra.
2Ki 3:3  Dock höll han fast vid de Jerobeams, Nebats sons, synder genom vilka denne hade kommit Israel att synda; från dessa avstod han icke.
2Ki 3:4  Mesa, konungen i Moab, som ägde mycken boskap, hade i skatt till konungen i Israel erlagt hundra tusen lamm och ull av hundra tusen vädurar.
2Ki 3:5  Men när Ahab var död, avföll konungen i Moab från konungen i Israel.
2Ki 3:6  Då drog konung Joram ut från Samaria och mönstrade hela Israel.
2Ki 3:7  Därefter sände han åstad bud till Josafat, konungen i Juda, och lät säga honom: "Konungen i Moab har avfallit från mig. Vill du draga med mig för att strida mot Moab?" Han svarade: "Ja, jag vill draga ditupp - jag såsom du, mitt folk såsom ditt folk, mina hästar såsom dina hästar!"
2Ki 3:8  Och han frågade: "Vilken väg skola vi draga ditupp?" Han svarade "Vägen genom Edoms öken."
2Ki 3:9  Så drogo då konungen i Israel konungen i Juda och konungen i Edom åstad; men när de hade färdats sju dagsresor, fanns intet vatten för hären och för djuren som de hade med sig.
2Ki 3:10  Då sade Israels konung: "Ack att HERREN skulle kalla tillhopa dessa tre konungar för att giva dem i Moabs hand!"
2Ki 3:11  Men Josafat sade: "Finnes här ingen HERRENS profet, så att vi kunna fråga HERREN genom honom?" Då svarade en av Israels konungs tjänare och sade: "Elisa, Safats son, finnes här, han som plägade gjuta vatten på Elias händer."
2Ki 3:12  Josafat sade: "Hos honom är HERRENS ord." Israels konung och Josafat och Edoms konung gingo då ned till honom.
2Ki 3:13  Men Elisa sade till Israels konung: "Vad har du med mig att göra? Gå du till din faders profeter och till din moders profeter." Israels konung svarade honom: "Bort det, att HERREN skulle hava kallat tillhopa dessa tre konungar för att giva dem i Moabs hand!"
2Ki 3:14  Då sade Elisa: "Så sant HERREN Sebaot lever, han vilkens tjänare jag är: om jag icke hade undseende för Josafat, Juda konung, så skulle jag icke akta på dig eller se till dig.
2Ki 3:15  Men hämten nu hit åt mig en harpospelare." Så ofta harpospelaren spelade, kom nämligen HERRENS hand över honom.
2Ki 3:16  Och han sade: "Så säger HERREN: Gräven i denna dal grop vid grop.
2Ki 3:17  Ty så säger HERREN: I skolen icke märka någon vind, ej heller se något regn, men likväl skall denna dal bliva full med vatten, så att både I själva skolen hava att dricka och eder boskap och edra övriga djur.
2Ki 3:18  Dock anser HERREN icke ens detta vara nog, utan han vill ock giva Moab i eder hand.
2Ki 3:19  Och I skolen intaga alla befästa städer och alla andra ansenliga städer, I skolen fälla alla nyttiga träd och kasta igen alla vattenkällor, och alla bördiga åkerstycken skolen I fördärva med stenar."
2Ki 3:20  Och se, om morgonen, vid den tid då spisoffret frambäres, strömmade vatten till från Edomssidan, så att landet fylldes med vatten.
2Ki 3:21  Moabiterna hade nu allasammans hört att konungarna hade dragit upp för att strida mot dem, och alla de som voro vid vapenför ålder eller därutöver blevo uppbådade och stodo nu vid gränsen.
2Ki 3:22  Men bittida om morgonen, när solen gick upp och lyste på vattnet, sågo moabiterna vattnet framför sig rött såsom blod.
2Ki 3:23  Då sade de: "Det är blod! Konungarna hava helt visst råkat i strid och därvid dräpt varandra. Nu till plundring, Moab!"
2Ki 3:24  Men när de kommo till Israels läger, bröto israeliterna fram och slogo moabiterna, så att de flydde för dem. Och de drogo in i landet och slogo ytterligare moabiterna.
2Ki 3:25  Och städerna förstörde de, och på alla bördiga åkerstycken kastade de var och en sin sten, till dess de hade överhöljt dem, och alla vattenkällor täppte de till, och alla nyttiga träd fällde de, så att de till slut lämnade kvar allenast stenarna av Kir-Hareset. Men när slungkastarna omringade staden och besköto den
2Ki 3:26  och Moabs konung såg att han icke kunde hålla stånd i striden, tog han med sig sju hundra svärdbeväpnade män för att slå sig igenom till Edoms konung; men de kunde det icke.
2Ki 3:27  Då tog han sin förstfödde son, den som skulle bliva konung efter honom, och offrade denne på muren till ett brännoffer. Då drabbades Israel av svår hemsökelse, så att de måste bryta upp och lämna honom i fred och vända tillbaka till sitt land igen.
2Ki 4:1  Och en kvinna som var hustru till en av profetlärjungarna ropade till Elisa och sade: "Min man, din tjänare, har dött, och du vet att din tjänare fruktade HERREN; nu kommer hans fordringsägare och vill taga mina båda söner till trälar.
2Ki 4:2  Elisa sade till henne: "Vad kan jag göra för dig? Säg mig, vad har du i huset?" Hon svarade: "Din tjänarinna har intet annat i huset än en flaska smörjelseolja."
2Ki 4:3  Då sade han: "Gå och låna dig kärl utifrån av alla dina grannar, tomma kärl, men icke för få.
2Ki 4:4  Gå så in, och stäng igen dörren om dig och dina söner, och gjut i alla dessa kärl; och när ett kärl är fullt, så flytta undan det."
2Ki 4:5  Då gick hon ifrån honom. Och sedan hon hade stängt igen dörren om sig och sina söner, buro de fram kärlen till henne, och hon göt i.
2Ki 4:6  Och när kärlen voro fulla, sade hon till sin son: "Bär fram åt mig ännu ett kärl." Men han svarade henne: "Här finnes intet kärl mer. Då stannade oljan av.
2Ki 4:7  Och hon kom och berättade detta för gudsmannen. Då sade han: "Gå och sälj oljan, och betala din skuld. Sedan må du med dina söner leva av det som bliver över."
2Ki 4:8  En dag kom Elisa över till Sunem. Där bodde en rik kvinna, som nödgade honom att äta hos sig; och så ofta han sedan kom ditöver, tog han in där och åt.
2Ki 4:9  Då sade hon en gång till sin man: "Se, jag har förnummit att han som beständigt kommer hitöver är en helig gudsman.
2Ki 4:10  Så låt oss nu mura upp ett litet rum på taket och där sätta in åt honom en säng, ett bord, en stol och en ljusstake, så att han kan få taga in där, när han kommer till oss."
2Ki 4:11  Så kom han dit en dag och fick då taga in i rummet och ligga där.
2Ki 4:12  Och han sade till sin tjänare Gehasi: "Kalla hit sunemitiskan." Då kallade han dit henne, och hon infann sig där hos tjänaren.
2Ki 4:13  Ytterligare tillsade han honom: "Säg till henne: 'Se, du har haft allt detta besvär för oss. Vad kan nu jag göra för dig? Har du något att andraga hos konungen eller hos härhövitsmannen?'" Men hon svarade: "Nej; jag bor ju här mitt ibland mitt folk."
2Ki 4:14  Sedan frågade han: "Vad kan jag då göra för henne?" Gehasi svarade: "Jo, hon har ingen son, och hennes man är gammal."
2Ki 4:15  Så sade han då: "Kalla henne hitin." Då kallade han dit henne, och hon stannade i dörren.
2Ki 4:16  Och han sade: "Nästa år vid just denna tid skall du hava en son i famnen." Hon svarade: "Nej, min herre, du gudsman, inbilla icke din tjänarinna något sådant."
2Ki 4:17  Men kvinnan blev havande och födde en son följande år, just vid den tid som Elisa hade sagt henne.
2Ki 4:18  Och när gossen blev större, hände sig en dag att han gick ut till sin fader hos skördemännen.
2Ki 4:19  Då begynte han klaga för sin fader: "Mitt huvud! Mitt huvud!" Denne sade till sin tjänare: "Tag honom och bär honom till hans moder.
2Ki 4:20  Han tog honom då och förde honom till hans moder. Och han satt i hennes knä till middagstiden; då gav han upp andan.
2Ki 4:21  Men hon gick upp och lade honom på gudsmannens säng och stängde igen om honom och gick ut.
2Ki 4:22  Därefter kallade hon på sin man och sade: "Sänd till mig en av tjänarna med en åsninna, så vill jag skynda till gudsmannen; sedan kommer jag strax tillbaka."
2Ki 4:23  Han sade: "Varför vill du i dag fara till honom? Det är ju varken nymånad eller sabbat." Hon svarade: "Oroa dig icke!"
2Ki 4:24  Sedan lät hon sadla åsninnan och sade till sin tjänare: "Driv på framåt, och gör icke något uppehåll i min färd, förrän jag säger dig till."
2Ki 4:25  Så begav hon sig åstad och kom till gudsmannen på berget Karmel. Då nu gudsmannen fick se henne på något avstånd, sade han till sin tjänare Gehasi: "Se, där är sunemitiskan.
2Ki 4:26  Skynda nu emot henne och fråga henne: 'Allt står väl rätt till med dig och med din man och med gossen?'" Hon svarade: "Ja."
2Ki 4:27  Men när hon kom upp till gudsmannen på berget, fattade hon om hans fötter. Då gick Gehasi fram och ville driva henne undan; men gudsmannen sade: "Låt henne vara, ty hennes själ är bedrövad; men HERREN hade fördolt detta för mig och icke låtit mig få veta det."
2Ki 4:28  Och hon sade: "Hade jag väl bett min herre om en son? Sade jag icke fastmer att du icke skulle inbilla mig något?"
2Ki 4:29  Då sade han till Gehasi: "Omgjorda dina länder och tag min stav i din hand och gå åstad; om du möter någon, så hälsa icke på honom, och om någon hälsar på dig, så besvara icke hans hälsning. Och lägg sedan min stav på gossens ansikte."
2Ki 4:30  Men gossens moder sade: "Så sant HERREN lever, och så sant du själv lever, jag släpper dig icke." Då stod han upp och följde med henne.
2Ki 4:31  Men Gehasi hade redan gått före dem och lagt staven på gossens ansikte; dock hördes icke ett ljud, och intet spår av förnimmelse kunde märkas. Då vände han om och gick honom till mötes och berättade det för honom och sade: "Gossen har icke vaknat upp."
2Ki 4:32  Och när Elisa kom in i huset, fick han se att gossen låg död på hans säng.
2Ki 4:33  Då gick han in och stängde igen dörren om dem båda och bad till HERREN.
2Ki 4:34  Och han steg upp i sängen och lade sig över gossen, så att han hade sin mun på hans mun, sina ögon på hans ögon och sina händer på hans händer. När han så lutade sig ned över gossen, blev kroppen varm.
2Ki 4:35  Därefter gick han åter fram och tillbaka i rummet och steg så åter upp i sängen och lutade sig ned över honom. Då nös gossen, ända till sju gånger. Och därpå slog gossen upp ögonen.
2Ki 4:36  Sedan ropade han på Gehasi och sade: "Kalla hit sunemitiskan." Då kallade han in henne, och när hon kom in till honom, sade han: "Tag din son."
2Ki 4:37  Då kom hon fram och föll ned för hans fötter och bugade sig mot jorden. Därefter tog hon sin son och gick ut.
2Ki 4:38  Och Elisa kom åter till Gilgal, medan hungersnöden var i landet. När då profetlärjungarna sutto där inför honom, sade han till sin tjänare "Sätt på den stora grytan och koka något till soppa åt profetlärjungarna."
2Ki 4:39  Och en av dem gick ut på marken för att plocka något grönt; då fick han se en vild slingerväxt, och av den plockade han något som liknade gurkor, sin mantel full. När han sedan kom in, skar han sönder dem och lade dem i soppgrytan; ty de kände icke till dem.
2Ki 4:40  Och de öste upp åt männen, för att de skulle äta. Men så snart de hade begynt äta av soppan, gåvo de upp ett rop och sade: "Döden är i grytan, du gudsman!" Och de kunde icke äta.
2Ki 4:41  Då sade han: "Skaffen hit mjöl." Detta kastade han i grytan. Därefter sade han: "Ös upp åt folket och låt dem äta." Och intet skadligt fanns nu mer i grytan.
2Ki 4:42  Och en man kom från Baal-Salisa och förde med sig åt gudsmannen förstlingsbröd; tjugu kornbröd, och ax av grönskuren säd i sin påse. Då sade han: "Giv det åt folket att äta."
2Ki 4:43  Men hans tjänare sade: "Huru skall jag kunna sätta fram detta för hundra män?" Han sade: "Giv det åt folket att äta; ty så säger HERREN: De skola äta och få över.
2Ki 4:44  Då satte han fram det för dem. Och de åto och fingo över, såsom HERREN hade sagt.
2Ki 5:1  Naaman, den arameiske konungens härhövitsman, hade stort anseende hos sin herre och var högt aktad, ty genom honom hade HERREN givit seger åt Aram; och han var en tapper stridsman, men spetälsk.
2Ki 5:2  Nu hade araméerna, en gång då de drogo ut på strövtåg, fört med sig såsom fånge ur Israels land en ung flicka, som kom i tjänst hos Naamans hustru.
2Ki 5:3  Denna sade till sin fru: "Ack att min herre vore hos profeten i Samaria, så skulle denne nog befria honom från hans spetälska!"
2Ki 5:4  Då gick hon åstad och berättade detta för sin herre och sade: "Så och så har flickan ifrån Israels land sagt.
2Ki 5:5  Konungen i Aram svarade: "Far dit, så skall jag sända brev till konungen i Israel." Så for han då och tog med sig tio talenter silver och sex tusen siklar guld, så ock tio högtidsdräkter.
2Ki 5:6  Och han överlämnade brevet till Israels konung, och däri stod det: "Nu, när detta brev kommer dig till handa, må du veta att jag har sänt till dig min tjänare Naaman, för att du må befria honom från hans spetälska."
2Ki 5:7  När Israels konung hade läst brevet, rev han sönder sina kläder och sade: "Är jag då Gud, så att jag skulle kunna döda och göra levande, eftersom denne sänder bud till mig att jag skall befria en man från hans spetälska? Märken nu och sen huru han söker sak med mig."
2Ki 5:8  Men när gudsmannen Elisa hörde att Israels konung hade rivit sönder sina kläder, sände han till konungen och lät säga: "Varför har du rivit sönder dina kläder? Låt honom komma till mig, så skall han förnimma att en profet finnes i Israel."
2Ki 5:9  Så kom då Naaman med sina hästar och vagnar och stannade vid dörren till Elisas hus.
2Ki 5:10  Då sände Elisa ett bud ut till honom och lät säga: "Gå bort och bada dig sju gånger i Jordan, så skall ditt kött åter bliva sig likt, och du skall bliva ren."
2Ki 5:11  Men Naaman blev vred och for sin väg, i det han sade: "Jag tänkte att han skulle gå ut till mig och träda fram och åkalla HERRENS, sin Guds, namn och föra sin hand fram och åter över stället och så taga bort spetälskan.
2Ki 5:12  Äro icke Damaskus' floder, Abana och Parpar, bättre än alla vatten Israel? Då kunde jag ju lika gärna bada mig i dem för att bliva ren." Så vände han om och for sin väg i vrede.
2Ki 5:13  Men hans tjänare gingo fram och talade till honom och sade: "Min fader, om profeten hade förelagt dig något svårt, skulle du då icke hava gjort det? Huru mycket mer nu, då han allenast har sagt till dig: 'Bada dig, så bliver du ren'!
2Ki 5:14  Då for han ned och doppade sig i Jordan sju gånger, såsom gudsmannen hade sagt; och hans kött blev då åter sig likt, friskt såsom en ung gosses kött, och han blev ren.
2Ki 5:15  Därefter vände han tillbaka till gudsmannen med hela sin skara och gick in och trädde fram för honom och sade: "Se, nu vet jag att ingen Gud finnes på hela jorden utom i Israel. Så tag nu emot en tacksamhetsskänk av din tjänare."
2Ki 5:16  Men han svarade: "Så sant HERREN lever, han vilkens tjänare jag är, jag vill icke taga emot den." Och fastän han enträget bad honom att taga emot den, ville han icke.
2Ki 5:17  Då sade Naaman: "Om du icke vill detta, så låt då din tjänare få så mycket jord som ett par mulåsnor kunna bära. Ty din tjänare vill icke mer offra brännoffer och slaktoffer åt andra gudar, utan allenast åt HERREN.
2Ki 5:18  Detta må dock HERREN förlåta din tjänare: när min herre går in i Rimmons tempel för att där böja knä, och han då stöder sig vid min hand, och jag också böjer knä där i Rimmons tempel, må då HERREN förlåta din tjänare, när jag så böjer knä i Rimmons tempel.
2Ki 5:19  Han sade till honom: "Far i frid." Men när hän hade lämnat honom och farit ett stycke väg framåt,
2Ki 5:20  tänkte Gehasi, gudsmannen Elisas tjänare: "Se, min herre har släppt denne Naaman från Aram, utan att taga emot av honom vad han hade fört med sig. Så sant HERREN lever, jag vill skynda efter honom och söka få något av honom."
2Ki 5:21  Så gav sig då Gehasi åstad efter Naaman. Men när Naaman såg någon skynda efter sig, steg han med hast ned från vagnen och gick emot honom och sade: "Allt står väl rätt till?"
2Ki 5:22  Han svarade: "Ja; men min herre har sänt mig och låter säga: 'Just nu hava två unga män, profetlärjungar, kommit till mig från Efraims bergsbygd; giv dem en talent silver och två högtidsdräkter.'"
2Ki 5:23  Naaman svarade: "Värdes taga två talenter." Och han bad honom enträget och knöt så in två talenter silver i två pungar och tog fram två högtidsdräkter, och lämnade detta åt två av sina tjänare, och dessa buro det framför honom.
2Ki 5:24  Men när han kom till kullen, tog han det ur deras hand och lade det i förvar i huset; sedan lät han männen gå sin väg.
2Ki 5:25  Därefter gick han in och trädde fram för sin herre. Då frågade Elisa honom: "Varifrån kommer du, Gehasi?" Han svarade: "Din tjänare har ingenstädes varit."
2Ki 5:26  Då sade han till honom: "Menar du att jag icke i min ande var med, när en man vände om från sin vagn och gick emot dig? Är det nu tid att du skaffar dig silver och skaffar dig kläder, så ock olivplanteringar, vingårdar, får och fäkreatur, tjänare och tjänarinnor,
2Ki 5:27  nu då Naamans spetälska kommer att låda vid dig och vid dina efterkommande för evigt?" Så gick denne ut ifrån honom, vit såsom snö av spetälska.
2Ki 6:1  Profetlärjungarna sade till Elisa: "Se, rummet där vi sitta inför dig är för trångt för oss."
2Ki 6:2  Låt oss därför gå till Jordan och därifrån hämta var sin timmerstock, så att vi där kunna bygga oss ett annat hus att sitta i." Han svarade: "Gån åstad."
2Ki 6:3  Men en av dem sade: "Värdes själv gå med dina tjänaren." Han varade: "Ja, jag skall gå med."
2Ki 6:4  Så gick han med dem. Och när de kommo till Jordan, begynte de hugga ned träd.
2Ki 6:5  Men under det att en av dem höll på att fälla en stock, föll yxjärnet i vattnet. Då gav han upp ett rop och sade: "Ack, min herre, yxan var ju lånad."
2Ki 6:6  Gudsmannen frågade: "Var föll den i?" Och han visade honom stället. Då högg han av ett stycke trä och kastade det i där och fick så järnet att flyta upp.
2Ki 6:7  Sedan sade han: "Tag nu upp det." Då räckte mannen ut sin hand och tog det.
2Ki 6:8  Och konungen i Aram låg i krig med Israel. Men när han rådförde sig med sina tjänare och sade: "På det och det stället vill jag lägra mig",
2Ki 6:9  då sände gudsmannen bud till Israels konung och lät säga: "Tag dig till vara för att tåga fram vid det stället, ty araméerna ligga där."
2Ki 6:10  Då sände Israels konung till det ställe som gudsmannen hade angivit för honom och varnat honom för; och han tog sig till vara där. Detta skedde icke allenast en gång eller två gånger.
2Ki 6:11  Häröver blev konungen i Aram mycket orolig; och han kallade till sig sina tjänare och sade till dem: "Kunnen I icke säga mig vem av de våra det är som håller med Israels konung?"
2Ki 6:12  Då svarade en av hans tjänare: "Icke så, min herre konung; men Elisa, profeten i Israel, kungör för Israels konung vart ord som du talar i din sovkammare."
2Ki 6:13  Han sade: "Gån och sen till, var han finnes, så att jag kan sända åstad och gripa honom." Och man berättade för honom att han var i Dotan.
2Ki 6:14  Då sände han dit hästar och vagnar och en stor här; och de kommo dit om natten och omringade staden.
2Ki 6:15  När nu gudsmannens tjänare bittida om morgonen stod upp och gick ut, fick han se att en här hade lägrat sig runt omkring staden med hästar och vagnar. Då sade tjänaren till honom: "Ack, min herre, huru skola vi nu göra?"
2Ki 6:16  Han svarade: "Frukta icke; ty de som äro med oss äro flera än de som äro med dem."
2Ki 6:17  Och Elisa bad och sade: "HERRE, öppna hans ögon, så att han ser." Då öppnade HERREN tjänarens ögon, och han fick se att berget var fullt med hästar och vagnar av eld, runt omkring Elisa.
2Ki 6:18  När de nu drogo ned mot honom, bad Elisa till HERREN och sade: "Slå detta folk med blindhet." Då slog han dem med blindhet, såsom Elisa bad.
2Ki 6:19  Och Elisa sade till dem: "Detta är icke den rätta vägen eller den rätta staden. Följen mig, så skall jag föra eder till den man som I söken." Därefter förde han dem till Samaria.
2Ki 6:20  Men när de kommo till Samaria, sade Elisa: "HERRE, öppna dessas ögon, så att de se." Då öppnade HERREN deras ögon, och de fingo se att de voro mitt i Samaria.
2Ki 6:21  När då Israels konung såg dem, sade han till Elisa: "Skall jag hugga ned dem, min fader, skall jag hugga ned dem?"
2Ki 6:22  Han svarade: "Du skall icke hugga ned dem. Du plägar ju icke ens hugga ned dem som du har tagit till fånga med svärd och båge. Sätt fram för dem mat och dryck och låt dem äta och dricka, och låt dem sedan gå till sin herre igen"
2Ki 6:23  Då tillredde han åt dem en stor måltid, och när de hade ätit och druckit, lät han dem gå; och de gingo till sin herre igen. Sedan kommo icke vidare några arameiska strövskaror in i Israels land.
2Ki 6:24  Därefter hände sig att Ben-Hadad, konungen i Aram, samlade hela sin här och drog upp och belägrade Samaria.
2Ki 6:25  Och medan de belägrade Samaria, uppstod där en så stor hungersnöd, att man betalade åttio siklar silver för ett åsnehuvud och fem siklar silver för en fjärdedels kab duvoträck.
2Ki 6:26  Och en gång då Israels konung gick omkring på muren ropade en kvinna till honom och sade: "Hjälp, min herre konung!"
2Ki 6:27  Han svarade: "Hjälper icke HERREN dig, varifrån skall då jag kunna skaffa hjälp åt dig? Från logen eller från vinpressen?"
2Ki 6:28  Och konungen frågade henne: "Vad fattas dig?" Hon svarade: "Kvinnan där sade till mig: 'Giv hit din son, så att vi få äta honom i dag, så skola vi äta min son i morgon.'
2Ki 6:29  Så kokade vi min son och åto upp honom. Nästa dag sade jag till henne: 'Giv nu hit din son, så att vi få äta honom.' Men då gömde hon undan sin son."
2Ki 6:30  När konungen hörde kvinnans ord, rev han sönder sina kläder, där han gick på muren. Då fick folket se att han hade säcktyg inunder, närmast kroppen.
2Ki 6:31  Och han sade: "Gud straffe mig nu och framgent, om Elisas, Safats sons, huvud i dag får sitta kvar på honom."
2Ki 6:32  Så sände han då dit en man före sig, under det att Elisa satt i sitt hus och de äldste sutto där hos honom. Men innan den utskickade hann fram till honom, sade han till de äldste: "Sen I huru denne mördarson sänder hit en man för att taga mitt huvud? Men sen nu till, att I stängen igen dörren, när den utskickade kommer; och spärren så vägen för honom med den. Jag hör nu ock ljudet av hans herres steg efter honom."
2Ki 6:33  Medan han ännu talade med dem, kom den utskickade ned till honom. Och denne sade: "Se, detta är en olycka som kommer från HERREN; huru skall jag då längre kunna hoppas på HERREN?"
2Ki 7:1  Men Elisa svarade: "Hören HERRENS ord. Så säger HERREN: I morgon vid denna tid skall man få ett sea-mått fint mjöl för en sikel, så ock två sea-mått korn för en sikel, i Samarias port."
2Ki 7:2  Den kämpe vid vilkens hand konungen stödde sig svarade då gudsmannen och sade: "Om HERREN också gjorde fönster på himmelen, huru skulle väl detta kunna ske?" Han sade: "Du skall få se det med egna ögon, men du skall icke få äta därav."
2Ki 7:3  Utanför stadsporten uppehöllo sig då fyra spetälska män. Dessa sade till varandra: "Varför skola vi stanna kvar här, till dess vi dö?"
2Ki 7:4  Om vi besluta oss för att gå in i staden nu då hungersnöd är i staden, så skola vi dö där, och om vi stanna här, skola vi ock dö. Välan då, låt oss gå över till araméernas läger; låta de oss leva, så få vi leva, och döda de oss, så må vi dö."
2Ki 7:5  Så stodo de då upp i skymningen för att gå in i araméernas läger. Men när de kommo till utkanten av araméernas läger, se, då fanns ingen människa där.
2Ki 7:6  Ty Herren hade låtit ett dån av vagnar och hästar höras i araméernas läger, ett dån såsom av en stor här, så att de hade sagt till varandra: "Förvisso har Israels konung lejt hjälp mot oss av hetiternas och egyptiernas konungar, för att dessa skola komma över oss."
2Ki 7:7  Därför hade de brutit upp och flytt i skymningen och hade övergivit sina tält, sina hästar och åsnor, lägret sådant det stod; de hade flytt för att rädda sina liv.
2Ki 7:8  När de spetälska nu kommo till utkanten av lägret, gingo de in i ett tält och åto och drucko, och togo därur silver och guld och kläder, och gingo så bort och gömde det. Sedan vände de tillbaka och gingo in i ett annat tält och togo vad där fanns, och gingo så bort och gömde det.
2Ki 7:9  Men därefter sade de till varandra: "Vi bete oss icke rätt. I dag kunna vi frambära ett glädjebudskap. Men om vi nu tiga och vänta till i morgon, när det bliver dager, så skall det tillräknas oss såsom missgärning. Välan då, låt oss gå och berätta detta i konungens hus.
2Ki 7:10  Så gingo de åstad och ropade an vakten vid stadsporten och berättade för dem och sade: "Vi kommo till araméernas läger, men där fanns ingen människa, och icke ett ljud av någon människa hördes; där stodo allenast hästarna och åsnorna bundna och tälten såsom de pläga stå."
2Ki 7:11  Detta ropades sedan ut av dem som höllo vakt vid porten, och man förkunnade det också inne i konungens hus.
2Ki 7:12  Konungen stod då upp om natten och sade till sina tjänare: "Jag vill säga eder vad araméerna hava för händer mot oss. De veta att vi lida hungersnöd, därför hava de gått ut ur lägret och gömt sig ute på marken, i det de tänka att de skola gripa oss levande, när vi nu gå ut ur staden, och att de så skola komma in i staden."
2Ki 7:13  Men en av hans tjänare svarade och sade: "Låt oss taga fem av de återstående hästarna, dem som ännu finnas kvar härinne - det skall ju eljest gå dem såsom det går hela hopen av israeliter som ännu äro kvar härinne, eller såsom det har gått hela hopen av israeliter som redan hava omkommit - och låt oss sända åstad och se efter."
2Ki 7:14  Så tog man då två vagnar med hästar för, och konungen sände dem åstad efter araméernas här och sade: "Faren åstad och sen efter."
2Ki 7:15  Dessa foro nu efter dem ända till Jordan; och se, hela vägen var full med kläder och andra saker som araméerna hade kastat ifrån sig, när de hastade bort. Och de utskickade kommo tillbaka och berättade detta för konungen.
2Ki 7:16  Då drog folket ut och plundrade Araméernas läger; och nu fick man ett sea-mått fint mjöl för en sikel och likaså två sea-mått korn för en sikel, såsom HERREN hade sagt.
2Ki 7:17  Och den kämpe vid vilkens hand konungen plägade stödja sig hade av honom blivit satt till att hålla ordning vid stadsporten; men folket trampade honom till döds i porten, detta i enlighet med gudsmannens ord, vad denne hade sagt, när konungen kom ned till honom.
2Ki 7:18  Ty när gudsmannen sade till konungen: "I morgon vid denna tid skall man i Samarias port få två sea-mått korn för en sikel och likaså ett sea-mått fint mjöl för en sikel",
2Ki 7:19  då svarade kämpen gudsmannen och sade: "Om HERREN också gjorde fönster på himmelen, huru skulle väl något sådant kunna ske?" Då sade han: "Du skall få se det med egna ögon, men du skall icke få äta därav."
2Ki 7:20  Så gick det honom ock, ty folket trampade honom till döds i porten.
2Ki 8:1  Och Elisa talade till den kvinna vilkens son han hade gjort levande, han sade: "Stå upp och drag bort med ditt husfolk och vistas var du kan, ty HERREN har bjudit hungersnöden komma, och den har redan kommit in i landet och skall räcka i sju år."
2Ki 8:2  Då stod kvinnan upp och gjorde såsom gudsmannen sade; hon drog bort med sitt husfolk och vistades i filistéernas land i sju år.
2Ki 8:3  Men när de sju åren voro förlidna, kom kvinnan tillbaka ifrån filistéernas land; och hon gick åstad för att anropa konungen om att återfå sitt hus och sin åker.
2Ki 8:4  Och konungen höll då på att tala med Gehasi, gudsmannens tjänare, och sade: "Förtälj för mig alla de stora ting som Elisa har gjort."
2Ki 8:5  Och just som han förtäljde för konungen huru han hade gjort en död levande, då kom den kvinna vilkens son han hade gjort levande och anropade konungen om att återfå sitt hus och sin åker. Då sade Gehasi: "Min herre konung, detta är kvinnan, och detta är hennes son, den som Elisa har gjort levande."
2Ki 8:6  Då frågade konungen kvinnan, och hon förtäljde allt för honom. Sedan lät konungen henne få en hovman med sig och sade: "Skaffa tillbaka allt vad som tillhör henne, och därtill all avkastning av åkern, från den dag då hon lämnade landet ända till nu."
2Ki 8:7  Och Elisa kom till Damaskus, under det att Ben-Hadad, konungen i Aram, låg sjuk. När man nu berättade för denne att gudsmannen hade kommit dit,
2Ki 8:8  sade konungen till Hasael: "Tag skänker med dig och gå gudsmannen till mötes, och fråga HERREN genom honom om jag skall tillfriskna från denna sjukdom.
2Ki 8:9  Så gick då Hasael honom till mötes och tog med sig skänker, allt det bästa som fanns i Damaskus, så mycket som fyrtio kameler kunde bära. Och han kom och trädde fram för honom och sade: "Din son Ben-Hadad, konungen i Aram, har sänt mig till dig och låter fråga: 'Skall jag tillfriskna från denna sjukdom?'"
2Ki 8:10  Elisa svarade honom: "Gå och säg till honom: 'Du skall tillfriskna.' Men HERREN har uppenbarat för mig att han likväl skall dö."
2Ki 8:11  Och gudsmannen stirrade framför sig och betraktade honom länge och väl; därefter begynte han gråta.
2Ki 8:12  Då sade Hasael: "Varför gråter min herre?" Han svarade: "Därför att jag vet huru mycket ont du skall göra Israels barn: du skall sätta eld på deras fästen, deras unga män skall du dräpa med svärd, deras späda barn skall du krossa, och deras havande kvinnor skall du upprista."
2Ki 8:13  Hasael sade: "Vad är väl din tjänare, den hunden, eftersom han skulle kunna göra så stora ting?" Elisa svarade: "HERREN har uppenbarat för mig att du skall bliva konung över Aram."
2Ki 8:14  Och han gick ifrån Elisa och kom till sin herre. Då frågade denne honom: "Vad sade Elisa till dig?" Han svarade: "Han sade till mig att du skall tillfriskna."
2Ki 8:15  Men dagen därefter tog han täcket och doppade det i vatten och bredde ut det över hans ansikte; detta blev hans död. Och Hasael blev konung efter honom.
2Ki 8:16  I Jorams, Ahabs sons, Israels konungs, femte regeringsår, medan Josafat var konung i Juda, blev Joram, Josafats son, konung i Juda.
2Ki 8:17  Han var trettiotvå år gammal, när han blev konung, och han regerade åtta år i Jerusalem.
2Ki 8:18  Men han vandrade på Israels konungars väg, såsom Ahabs hus hade gjort, ty en dotter till Ahab var hans hustru; han gjorde vad ont var i HERRENS ögon.
2Ki 8:19  Dock ville HERREN icke fördärva Juda, för sin tjänare Davids skull, enligt sitt löfte till honom, att han skulle låta honom och hans söner hava en lampa för alltid.
2Ki 8:20  I hans tid avföll Edom från Juda välde och satte en egen konung över sig.
2Ki 8:21  Då drog Joram över till Sair med alla sina stridsvagnar. Och om natten gjorde han ett anfall på edoméerna, som hade omringat honom, och slog dem och hövitsmannen över deras vagnar, men folket flydde till sina hyddor.
2Ki 8:22  Så avföll Edom från Juda välde, och det har varit skilt därifrån ända till denna dag. Vid just samma tid avföll ock Libna.
2Ki 8:23  Vad nu mer är att säga om Joram och om allt vad han gjorde, det finnes upptecknat i Juda konungars krönika.
2Ki 8:24  Och Joram gick till vila hos sina fäder och blev begraven hos sina fäder i Davids stad. Och hans son Ahasja blev konung efter honom.
2Ki 8:25  I Jorams, Ahabs sons, Israels konungs, tolfte regeringsår blev Ahasja, Jorams son, konung i Juda.
2Ki 8:26  Tjugutvå år gammal var Ahasja, när han blev konung, och han regerade ett år i Jerusalem. Hans moder hette Atalja, dotter till Omri, Israels konung.
2Ki 8:27  Han vandrade på Ahabs hus' väg och gjorde vad ont var i HERRENS ögon likasom Ahabs hus; han var ju nära besläktad med Ahabs hus.
2Ki 8:28  Och han drog åstad med Joram, Ahabs son, och stridde mot Hasael, konungen i Aram, vid Ramot Gilead. Men Joram blev sårad av araméerna.
2Ki 8:29  Då vände konung Joram tillbaka, för att i Jisreel låta hela sig från de sår som araméerna hade tillfogat honom vid Rama, i striden mot Hasael, konungen i Aram. Och Ahasja, Jorams son, Juda konung, for ned för att besöka Joram, Ahabs son, i Jisreel, eftersom denne låg sjuk.
2Ki 9:1  Profeten Elisa kallade till sig en av profetlärjungarna och sade till honom: "Omgjorda dina länder och tag denna oljeflaska med dig, och gå till Ramot i Gilead.
2Ki 9:2  Och när du har kommit dit, så sök upp Jehu, son till Josafat, son till Nimsi, och gå in och bed honom stå upp, där han sitter bland sina bröder, och för honom in i den innersta kammaren.
2Ki 9:3  Tag så oljeflaskan och gjut olja på hans huvud och säg: 'Så säger HERREN: Jag har smort dig till konung över Israel.' Öppna sedan dörren och fly, utan att dröja."
2Ki 9:4  Så gick då den unge mannen, profetens tjänare, åstad till Ramot i Gilead.
2Ki 9:5  Och när han kom dit, fick han se härens hövitsmän sitta där. Då sade han: "Jag har ett ärende till dig, hövitsman." Jehu frågade: "Till vem av oss alla här?" Han svarade: "Till dig själv, hövitsman."
2Ki 9:6  Då stod han upp och gick in i huset; och han göt oljan på hans huvud och sade till honom: "Så säger HERREN Israels Gud: Jag har smort dig till konung över HERRENS folk, över Israel.
2Ki 9:7  Och du skall förgöra Ahabs, din herres, hus; ty jag vill på Isebel hämnas mina tjänare profeterna blod, ja, alla HERRENS tjänares blod.
2Ki 9:8  Och Ahabs hela hus skall förgås jag skall utrota allt mankön av Ahab hus, både små och stora i Israel.
2Ki 9:9  Och jag skall göra med Ahab hus såsom jag gjorde med Jerobeams, Nebats sons, hus, och såsom jag gjorde med Baesas, Ahias sons, hus.
2Ki 9:10  Och hundarna skola äta upp Isebel på Jisreels åkerfält, och ingen skall begrava henne." Därefter öppnade han dörren och flydde.
2Ki 9:11  När sedan Jehu åter kom ut till sin herres tjänare, frågade man honom: "Allt står väl rätt till? Varför kom denne vanvetting till dig?" Han svarade dem: "I kännen ju den mannen och hans tal."
2Ki 9:12  Men de sade: "Du vill bedraga oss; säg oss sanningen." Då sade han: "Så och så talade han till mig och sade: 'Så säger HERREN: Jag har smort dig till konung över Israel.'"
2Ki 9:13  Strax tog då var och en av dem sin mantel och lade den under honom på själva trappan; och de stötte i basun och ropade: "Jehu har blivit konung."
2Ki 9:14  Och Jehu, son till Josafat, son till Nimsi, anstiftade nu en sammansvärjning mot Joram. (Joram hade då med hela Israel legat vid Ramot i Gilead för att försvara det mot Hasael, konungen i Aram;
2Ki 9:15  men själv hade konung Joram vänt tillbaka, för att i Jisreel låta hela sig från de sår som araméerna hade tillfogat honom under hans strid mot Hasael, konungen i Aram.) Och Jehu sade: "Om I så viljen, så låten ingen slippa ut ur staden, som kan gå åstad och berätta detta i Jisreel."
2Ki 9:16  Och Jehu steg upp i sin vagn och for till Jisreel, ty Joram låg sjuk där; och Ahasja, Juda konung, hade farit ditned för att besöka Joram.
2Ki 9:17  När nu väktaren som stod på tornet i Jisreel fick se Jehus skara, då han kom, sade han: "Jag ser en skara." Då bjöd Joram att man skulle taga en ryttare och sända honom dem till mötes och låta honom fråga om allt stode rätt till.
2Ki 9:18  Ryttaren red honom då till mötes och sade: "Konungen låter fråga 'Allt står väl rätt till?'" Då svarade Jehu: "Vad kommer den saken dig vid? Vänd, och följ efter mig." Och väktaren berättade och sade: "Den utskickade har hunnit fram till dem, men han kommer icke tillbaka."
2Ki 9:19  Då sände han en annan ryttare. när denne hade hunnit fram till dem, sade han: "Konungen låter fråga: 'Allt står väl rätt till?'" Jehu svarade: "Vad kommer den saken dig vid? Vänd, och följ efter mig."
2Ki 9:20  Väktaren berättade åter och sade: "Han har hunnit fram till dem men han kommer icke tillbaka. På deras sätt att fara fram ser det ut som vore det Jehu, Nimsis son, ty han far fram såsom en vanvetting."
2Ki 9:21  Då sade Joram: "Spänn för." Och man spände för hans vagn. Och Joram, Israels konung, for nu ut med Ahasja, Juda konung, var och en i sin vagn; de foro ut för att möta Jehu. Och de träffade tillsammans med honom på jisreeliten Nabots åkerstycke.
2Ki 9:22  När Joram nu fick se Jehu, sade han: "Allt står väl rätt till, Jehu?" Denne svarade: "Huru skulle det kunna stå rätt till, så länge som du tål din moder Isebels avgudiska väsen och hennes många trolldomskonster?"
2Ki 9:23  Då svängde Joram om vagnen och flydde, i det han ropade till Ahasja: "Förräderi, Ahasja!"
2Ki 9:24  Men Jehu hade fattat bågen i sin hand och sköt Joram i ryggen, att pilen gick ut genom hjärtat, och han sjönk ned i sin vagn.
2Ki 9:25  Därefter sade han till sin livkämpe Bidkar: "Tag honom och kasta ut honom på jisreeliten Nabots åkerstycke; kom ihåg huru HERREN, när jag och du bredvid varandra redo bakom hans fader Ahab, om denne uttalade den utsagan:
2Ki 9:26  'Sannerligen, så visst som jag i går såg Nabots och hans söners blod, säger HERREN, skall jag just på detta åkerstycke vedergälla dig, säger HERREN.' Tag därför honom nu och kasta ut honom här på åkerstycket, i enlighet med HERRENS ord."
2Ki 9:27  När Ahasja, Juda konung, såg detta, flydde han åt Trädgårdshuset till. Men Jehu jagade efter honom och ropade: "Skjuten ned också honom i vagnen." Så skedde ock på Gurhöjden vid Jibleam; men han flydde vidare till Megiddo och dog där.
2Ki 9:28  Sedan förde hans tjänare honom i vagnen till Jerusalem; och man begrov honom i hans grav hos hans fäder, i Davids stad.
2Ki 9:29  Ahasja hade blivit konung över Juda i Jorams, Ahabs sons, elfte regeringsår.
2Ki 9:30  Så kom nu Jehu till Jisreel. När Isebel fick höra detta, sminkade hon sig kring ögonen och smyckade sitt huvud och såg ut genom fönstret.
2Ki 9:31  Och när Jehu kom in genom porten, ropade hon: "Allt står väl rätt till, du, Simri, som har dräpt din herre?"
2Ki 9:32  Han lyfte sitt ansikte upp mot fönstret och sade: "Vem håller med mig? Vem?" Då sågo två eller tre hovmän ut, ned på honom.
2Ki 9:33  Han sade: "Störten ned henne." Och de störtade ned henne, så att hennes blod stänkte på väggen och på hästarna; och han körde över henne.
2Ki 9:34  Därefter gick han in och åt och drack. Sedan sade han: "Tagen vara på henne, den förbannade, och begraven henne, ty hon är dock en konungadotter."
2Ki 9:35  Men när de då gingo åstad för att begrava henne, funno de av henne intet annat än huvudskålen, fötterna och händerna.
2Ki 9:36  och de vände tillbaka och berättade detta för honom. Då sade han: "Detta är vad HERREN talade genom sin tjänare tisbiten Elia, i det han sade: 'På Jisreels åkerfält skola hundarna äta upp Isebels kött;
2Ki 9:37  och Isebels döda kropp skall ligga såsom gödsel på marken på Jisreels åkerfält, så att ingen skall kunna säga: Detta är Isebel.'"
2Ki 10:1  Men Ahab hade sjuttio söner i Samaria. Och Jehu skrev brev och sände till Samaria, till de överste i Jisreel, de äldste, och till de fostrare som Ahab hade utsett;
2Ki 10:2  han skrev: "Nu, när detta brev kommer eder till handa, I som haven eder herres söner hos eder, och som haven vagnarna och hästarna hos eder, och därtill en befäst stad och vapen,
2Ki 10:3  mån I utse den som är bäst och lämpligast av eder herres söner och sätta honom på hans faders tron och strida för eder herres hus."
2Ki 10:4  Men de blevo övermåttan förskräckta och sade: "De två konungarna hava ju icke kunnat hålla stånd mot honom; huru skulle då vi kunna hålla stånd!"
2Ki 10:5  Och överhovmästaren och hövdingen över staden och de äldste och konungasönernas fostrare sände till Jehu och läto säga: "Vi äro dina tjänare; allt vad du säger oss villa vi göra. Vi vilja icke göra någon till konung; gör vad dig täckes."
2Ki 10:6  Då skrev han ett annat brev till dem, vari det stod: "Om I hållen med mig och viljen lyssna till mina ord, så tagen huvudena av eder herres söner och kommen i morgon vid denna tid till mig i Jisreel." De sjuttio konungasönerna bodde nämligen hos de store i staden, vilka fostrade dem.
2Ki 10:7  Då nu brevet kom dem till handa, togo de konungasönerna och slaktade dem, alla sjuttio, och lade deras huvuden i korgar och sände dem till honom i Jisreel.
2Ki 10:8  När då ett bud kom och berättade för honom att de hade fört dit konungasönernas huvuden, sade han: "Läggen dem till i morgon i två högar vid ingången till porten."
2Ki 10:9  Och om morgonen gick han ut och ställde sig där och sade till allt folket: "I ären utan skuld. Det är jag som har anstiftat sammansvärjningen mot min herre och dräpt honom; men vem har slagit ihjäl alla dessa?
2Ki 10:10  Märken nu huru intet av HERRENS ord faller till jorden, intet som HERREN har talat mot Ahabs hus. Ja, HERREN har gjort vad han har sagt genom sin tjänare Elia."
2Ki 10:11  Sedan, dräpte Jehu alla som voro kvar av Ahabs hus i Jisreel, så ock alla hans store och hans förtrogne och hans präster; han lät ingen slippa undan.
2Ki 10:12  Därefter stod han upp och begav sig åstad till Samaria; men under vägen, när Jehu kom till Bet-Eked-Haroim,
2Ki 10:13  träffade han på Ahasjas, Juda konungs, bröder. Han frågade dem: "Vilka ären I?" De svarade: "Vi äro Ahasjas bröder, och vi äro på väg ned för att hälsa på konungasönerna och konungamoderns söner."
2Ki 10:14  Han sade: "Gripen dem levande." Då grepo de dem levande och slaktade dem och kastade dem i Bet-Ekeds brunn, alla fyrtiotvå; han lät ingen av dem bliva kvar.
2Ki 10:15  När han sedan begav sig därifrån, träffade han Jonadab, Rekabs son, som kom honom till mötes; och han hälsade på honom och sade till honom: "Är du lika redligt sinnad mot mig som jag är mot dig?" Jonadab svarade: "Ja." "Är det så", sade han, "så räck mig din hand." Då räckte han honom sin hand; och han lät honom stiga upp till sig i vagnen.
2Ki 10:16  Och han sade: "Far med mig och se huru jag nitälskar för HERREN." Så körde man åstad med honom i hans vagn.
2Ki 10:17  Och när han kom till Samaria, dräpte han alla som voro kvar av Ahabs hus i Samaria och förgjorde det så, i enlighet med det ord som HERREN hade talat till Elia.
2Ki 10:18  Och Jehu församlade allt folket och sade till dem: "Ahab har tjänat Baal litet; Jehu skall tjäna honom mycket.
2Ki 10:19  Så kallen nu hit till mig alla Baals profeter, alla hans tjänare och alla hans präster - ingen får saknas - ty jag har ett stort offer åt Baal i sinnet; var och en som saknas skall mista livet." Men Jehu gjorde så med led list, i avsikt att utrota Baals tjänare.
2Ki 10:20  Därefter sade Jehu: "Pålysen en helig högtidsförsamling åt Baal." Då lyste man ut en sådan.
2Ki 10:21  Och Jehu sände bud över hela Israel, och alla Baals tjänare kommo; Ingen underlät att komma. Och de gingo in i Baals tempel, och Baals tempel blev fullt, ifrån den ena ändan till den andra.
2Ki 10:22  Sedan sade han till föreståndaren för klädkammaren: "Tag fram kläder åt alla Baals tjänare." Och han tog fram kläderna åt dem.
2Ki 10:23  Därefter gick Jehu in i Baals tempel med Jonadab, Rekabs son. Och han sade till Baals tjänare: "Sen nu noga efter, att här bland eder icke finnes någon HERRENS tjänare, utan allenast sådana som tjäna Baal.
2Ki 10:24  De gingo alltså in för att offra slaktoffer och brännoffer. Men Jehu hade därutanför ställt åttio man och sagt: "Om någon slipper undan av de män som jag nu överlämnar i edra händer, så skall liv givas för liv."
2Ki 10:25  Och när man hade offrat brännoffret, sade Jehu till drabanterna och kämparna: "Gån in och slån ned dem; låten ingen komma ut." Och de slogo dem med svärdsegg, och drabanterna och kämparna kastade undan deras kroppar. Därefter gingo de in i det inre av Baals tempel
2Ki 10:26  och kastade ut stoderna ur Baals tempel och brände upp dem.
2Ki 10:27  Och själva Baalsstoden bröto de ned; de bröto ock ned Baals tempel och gjorde därav avträden, som finnas kvar ännu i dag.
2Ki 10:28  Så utrotade Jehu Baal ur Israel.
2Ki 10:29  Men från de Jerobeams, Nebats sons, synder genom vilka denne hade kommit Israel att synda, från dem avstod icke Jehu, icke från guldkalvarna i Betel och Dan.
2Ki 10:30  Och HERREN sade till Jehu: "Därför att du har väl utfört vad rätt var i mina ögon, och gjort mot Ahabs hus allt vad jag hade i sinnet, därför skola dina söner till fjärde led sitta på Israels tron.
2Ki 10:31  Men Jehu tog dock icke i akt att vandra efter HERRENS, Israels Guds, lag av allt sitt hjärta; han avstod icke från de Jerobeams synder genom vilka denne hade kommit Israel att synda.
2Ki 10:32  Vid denna tid begynte HERREN skära bort stycken från Israel, ty Hasael slog israeliterna utefter hela deras gräns
2Ki 10:33  och intog östra sidan om Jordan hela landet Gilead, gaditernas, rubeniternas och manassiternas land, området från Aroer vid bäcken Arnon, både Gilead och Basan.
2Ki 10:34  Vad nu mer är att säga om Jehu, om allt vad han gjorde och om alla hans bedrifter, det finnes upptecknat i Israels konungars krönika.
2Ki 10:35  Och Jehu gick till vila hos sina fäder, och man begrov honom i Samaria. Och hans son Joahas blev konung efter honom.
2Ki 10:36  Den tid Jehu regerade över Israel Samaria var tjuguåtta år.
2Ki 11:1  När Atalja, Ahasjas moder, förnam att hennes son var död, stod hon upp och förgjorde hela konungasläkten.
2Ki 11:2  Men just när konungabarnen skulle dödas, tog Joseba, konung Jorams dotter, Ahasjas syster, Joas, Ahasjas son, och skaffade honom jämte hans amma hemligen undan, in i sovkammaren; där höll man honom dold för Atalja, så att han icke blev dödad.
2Ki 11:3  Sedan var han hos henne i HERRENS hus, där han förblev gömd i sex år, medan Atalja regerade i landet.
2Ki 11:4  Men i det sjunde året sände Jojada åstad och lät hämta karéernas och drabanternas underhövitsmän och förde dem in till sig i HERRENS hus; och sedan han hade gjort en överenskommelse med dem och tagit en ed av dem i HERRENS hus visade han dem konungens son.
2Ki 11:5  Därefter bjöd han dem och sade: "Detta är vad I skolen göra: en tredjedel av eder, I som haven att inträda i vakthållningen på sabbaten, skall hålla vakt i konungshuset
2Ki 11:6  och en tredjedel vid Surporten och en tredjedel vid porten bakom drabanterna; så skolen hålla vakt vid huset var i sin ordning.
2Ki 11:7  Men de båda andra avdelningarna av eder, nämligen alla som hava att avgå från vakthållningen på sabbaten, de skola hålla vakt i HERRENS hus hos konungen.
2Ki 11:8  I skolen ställa eder runt omkring; konungen, var och en med sina vapen i handen; och om någon vill tränga sig inom leden, skall han dödas. Och I skolen följa konungen, vare sig han går ut eller in.
2Ki 11:9  Underhövitsmännen gjorde allt vad prästen Jojada hade bjudit dem; var och en av dem tog sina män, både de som skulle inträda i vakthållningen på sabbaten och de som skulle avgå därifrån på sabbaten, och de kommo så till prästen Jojada.
2Ki 11:10  Och prästen gav åt underhövitsmännen det spjut och de sköldar som hade tillhört konung David, och som funnos i HERRENS hus.
2Ki 11:11  Och drabanterna ställde upp sig, var och en med sina vapen i handen, från husets södra sida till husets norra sida, mot altaret och mot huset, runt omkring konungen.
2Ki 11:12  Därefter förde han ut konungasonen och satte på honom kronan och gav honom vittnesbördet; och de gjorde honom till konung och smorde honom. Och de klappade i händerna och ropade: "Leve konungen!"
2Ki 11:13  När Atalja nu hörde drabanternas och folkets rop, gick hon in i HERRENS hus till folket.
2Ki 11:14  Där fick hon då se konungen stå vid pelaren, såsom övligt var, och hövitsmännen och trumpetblåsarna bredvid konungen, och fick höra huru hela folkmängden jublade och stötte i trumpeterna. Då rev Atalja sönder sina kläder och utropade: "Sammansvärjning! Sammansvärjning!"
2Ki 11:15  Men prästen Jojada gav underhövitsmännen som anförde skaran denna befallning: "Fören henne ut mellan leden, och om någon följer henne, så må han dödas med svärd." Prästen ville nämligen förhindra att hon dödades i HERRENS hus.
2Ki 11:16  Alltså grepo de henne, och när hon hade kommit till den plats där hästarna plägade föras in i konungshuset, dödades hon där.
2Ki 11:17  Och Jojada slöt det förbundet mellan HERREN, konungen och folket, att de skulle vara ett HERRENS folk; han slöt ock ett förbund mellan konungen och folket.
2Ki 11:18  Och hela folkmängden begav sig till Baals tempel och rev ned det och förstörde i grund dess altaren och dess bilder; och Mattan, Baals präst, dräpte de framför altarna. Därefter ställde prästen ut vakter vid HERRENS hus.
2Ki 11:19  Och han tog med sig underhövitsmännen jämte karéerna och drabanterna och hela folkmängden, och de förde konungen ned från HERRENS hus och gingo in i konungshuset genom Drabantporten; och han satte sig på konungatronen.
2Ki 11:20  Och hela folkmängden gladde sig, och staden förblev lugn. Men Atalja hade de dödat med svärd i konungshuset.
2Ki 11:21  Joas var sju år gammal, när han blev konung.
2Ki 12:1  I Jehus sjunde regeringsår blev Joas konung, och han regerade fyrtio år i Jerusalem. Hans moder hette Sibja, från Beer-Seba.
2Ki 12:2  Och Joas gjorde vad rätt var HERRENS ögon, så länge han levde, prästen Jojada hade varit hans lärare.
2Ki 12:3  Dock blevo offerhöjderna icke avskaffade, utan folket fortfor att frambära offer och tända offereld på höjderna.
2Ki 12:4  Och Joas sade till prästerna: "Alla penningar vilka såsom heliga gåvor inflyta till HERRENS hus, gångbara penningar, sådana som utgöra lösen för personer, efter det värde som för var och en bestämmes, och alla penningar som någon av sitt hjärta manas att bära till HERRENS hus,
2Ki 12:5  dem skola prästerna taga emot, var och en av sina bekanta, och de skola därmed sätta i stånd vad som är förfallet på HERRENS hus, överallt där något förfallet finnes."
2Ki 12:6  Men i konung Joas' tjugutredje regeringsår hade prästerna ännu icke satt i stånd vad som var förfallet på huset.
2Ki 12:7  Då kallade konung Joas till sig prästen Jojada och de övriga prästerna och sade till dem: "Varför sätten I icke i stånd vad som är förfallet på huset? Nu fån I icke längre taga emot penningar av edra bekanta, utan I skolen lämna dem ifrån eder till det som är förfallet på huset."
2Ki 12:8  Och prästerna samtyckte till att icke taga emot penningar av folket, och ej heller befatta sig med att sätta i stånd vad som var förfallet på huset.
2Ki 12:9  Då tog prästen Jojada en kista och borrade ett hål på locket och ställde den bredvid altaret, på högra sidan, när man går in i HERRENS hus. Och prästerna som höllo vakt vid tröskeln lade dit alla penningar som inflöto till HERRENS hus.
2Ki 12:10  Men så snart de då märkte att mycket penningar fanns i kistan, gick konungens sekreterare ditupp jämte översteprästen, och de knöto in och räknade sedan de penningar som funnos i HERRENS hus.
2Ki 12:11  Därefter överlämnades de uppvägda penningarna åt de män som förrättade arbete såsom tillsyningsmän vid HERRENS hus, och dessa betalade ut dem åt de timmermän och byggningsmän som arbetade på HERRENS hus,
2Ki 12:12  och åt murarna och stenhuggarna, så ock till inköp av trävirke och huggen sten för att sätta i stånd vad som var förfallet på HERRENS hus, korteligen, till alla utgifter för att sätta huset i stånd.
2Ki 12:13  Men man gjorde inga silverfat för HERRENS hus, ej heller knivar, skålar, trumpeter eller andra föremål av guld eller av silver, för de penningar som inflöto till HERRENS hus,
2Ki 12:14  utan man gav dem åt arbetarna, och dessa satte därför HERRENS hus i stånd.
2Ki 12:15  Och man höll icke någon räkenskap med de män åt vilka penningarna överlämnades, för att de skulle giva dem åt arbetarna, utan de fingo handla på heder och tro.
2Ki 12:16  Men skuldoffers- och syndofferspenningarna gingo icke till HERRENS hus utan tillföllo prästerna.
2Ki 12:17  På den tiden drog Hasael, konungen i Aram, upp och belägrade Gat och intog det, därefter ställde Hasael sitt tåg upp mot Jerusalem.
2Ki 12:18  Då tog Joas, Juda konung, allt vad hans fäder Josafat, Joram och Ahasja, Juda konungar, hade helgat åt HERREN, och vad han själv hade helgat åt HERREN, och allt guld som fanns i skattkamrarna i HERRENS hus och i konungshuset, och sände det till Hasael, konungen i Aram, och då lämnade denne Jerusalem i fred.
2Ki 12:19  Vad nu mer är att säga om Joas och om allt vad han gjorde, det finnes upptecknat i Juda konungars krönika.
2Ki 12:20  Och hans tjänare uppreste sig och sammansvuro sig och dräpte Joas i Millobyggnaden, som sträcker sig ned mot Silla.
2Ki 12:21  Det var hans tjänare Josakar, Simeats son, och Josabad, Somers son, som slogo honom till döds. Och man begrov honom hos hans fäder i Davids stad. Och hans son Amasja blev konung efter honom
2Ki 13:1  I Joas', Ahasjas sons, Juda konungs, tjugutredje regeringsår blev Joahas, Jehus son, konung över Israel i Samaria och regerade i sjutton år.
2Ki 13:2  Han gjorde vad ont var i HERRENS ögon och följde efter de Jerobeams, Nebats sons, synder genom vilka denne hade kommit Israel att synda; från dem avstod han icke.
2Ki 13:3  Då upptändes HERRENS vrede mot Israel, och han gav dem i Hasaels, den arameiske konungens, hand och i Ben-Hadads, Hasaels sons, hand hela denna tid.
2Ki 13:4  (Men Joahas bönföll inför HERREN, och HERREN hörde honom, eftersom han såg Israels betryck, då nu konungen i Aram förtryckte dem.
2Ki 13:5  Och HERREN gav åt Israel en frälsare, så att de blevo räddade ur araméernas hand; sedan bodde Israels barn i sina hyddor såsom förut.
2Ki 13:6  Dock avstodo de icke från de Jerobeams hus' synder genom vilka denne hade kommit Israel att synda, utan vandrade i dem. Aseran fick också stå kvar i Samaria.)
2Ki 13:7  Ty han hade icke låtit Joahas behålla mer folk än femtio ryttare, tio vagnar och tio tusen man fotfolk; så illa förgjorde dem konungen i Aram; han slog dem, så att de blevo såsom stoft, när man tröskar.
2Ki 13:8  Vad nu mer är att säga om Joahas, om allt vad han gjorde och om hans bedrifter, det finnes upptecknat i Israels konungars krönika.
2Ki 13:9  Och Joahas gick till vila hos sina fäder, och man begrov honom i Samaria. Och hans son Joas blev konung efter honom.
2Ki 13:10  I Joas', Juda konungs, trettiosjunde regeringsår blev Joas, Joahas' son, konung över Israel i Samaria och regerade i sexton år.
2Ki 13:11  Han gjorde vad ont var i HERRENS ögon; han avstod icke från någon av de Jerobeams, Nebats sons, synder genom vilka denne hade kommit Israel att synda, utan vandrade i dem.
2Ki 13:12  Vad nu mer är att säga om Joas, om allt vad han gjorde och om hans bedrifter, om hans krig mot Amasja, Juda konung, det finnes upptecknat i Israels konungars krönika.
2Ki 13:13  Och Joas gick till vila hos sina fäder, och Jerobeam besteg hans tron. Och Joas blev begraven i Samaria, hos Israels konungar.
2Ki 13:14  Men när Elisa låg sjuk i den sjukdom varav han dog, kom Joas, Israels konung, ned till honom. Och han satt hos honom gråtande och sade: "Min fader, min fader! Du som för Israel är både vagnar och ryttare!"
2Ki 13:15  Då sade Elisa till honom: "Hämta en båge och pilar." Och han hämtade åt honom en båge och pilar.
2Ki 13:16  Då sade han till Israels konung: "Fatta i bågen med din hand." Och när han hade gjort detta, lade Elisa sina händer på konungens händer.
2Ki 13:17  Därefter sade han: "Öppna fönstret mot öster." Och när han hade öppnat det, sade Elisa: "Skjut." Och han sköt. Då sade han: "En HERRENS segerpil, en segerpil mot Aram! Du skall slå araméerna vid Afek, så att de förgöras."
2Ki 13:18  Därefter sade han: "Tag pilarna." Och när han hade tagit dem, sade han till Israels konung: "Slå på jorden." Då slog han tre gånger och sedan höll han upp.
2Ki 13:19  Då blev gudsmannen vred på honom och sade: "Du skulle slagit fem eller sex gånger, ty då skulle du hava slagit araméerna så, att de hade blivit förgjorda; men nu kommer du att slå araméerna allenast tre gånger."
2Ki 13:20  Så dog då Elisa, och man begrov honom. Men moabitiska strövskaror plägade falla in i landet, vid årets ingång.
2Ki 13:21  Så hände sig, att just när några höllo på att begrava en man, fingo de se en strövskara; då kastade de mannen i Elisas grav. När då mannen kom i beröring med Elisas ben, fick han liv igen och reste sig upp på sina fötter.
2Ki 13:22  Och Hasael, konungen i Aram, hade förtryckt Israel, så länge Joahas levde.
2Ki 13:23  Men HERREN blev dem nådig och förbarmade sig över dem och vände sig till dem, för det förbunds skull som han hade slutit med Abraham, Isak och Jakob; ty han ville icke fördärva dem, och han hade ännu icke kastat dem bort ifrån sitt ansikte.
2Ki 13:24  Och Hasael, konungen i Aram, dog, och hans son Ben-Hadad blev konung efter honom.
2Ki 13:25  Då tog Joas, Joahas' son, tillbaka från Ben-Hadad, Hasaels son, de städer som denne i krig hade tagit ifrån hans fader Joahas. Tre gånger slog Joas honom och återtog så Israels städer.
2Ki 14:1  I Joas', Joahas' sons, Israels konungs, andra regeringsår blev Amasja, Joas' son, konung i Juda.
2Ki 14:2  Han var tjugufem år gammal, när han blev konung, och han regerade tjugunio år i Jerusalem. Hans moder hette Joaddin, från Jerusalem.
2Ki 14:3  Han gjorde vad rätt var i HERRENS ögon, dock icke såsom hans fader David; men han gjorde i allt såsom hans fader Joas hade gjort.
2Ki 14:4  Offerhöjderna blevo likväl icke avskaffade, utan folket fortfor att frambära offer och tända offereld på höjderna.
2Ki 14:5  Och sedan konungadömet hade blivit befäst i hans hand, lät han dräpa dem av sina tjänare, som hade dräpt hans fader, konungen.
2Ki 14:6  Men mördarnas barn dödade han icke, i enlighet med vad föreskrivet var i Moses lagbok, där HERREN hade bjudit och sagt: "Föräldrarna skola icke dödas för sina barns skull, och barnen skola icke dödas för sina föräldrars skull, utan var och en skall dö genom sin egen synd."
2Ki 14:7  Han slog edoméerna i Saltdalen, tio tusen man, och intog Sela med strid och gav det namnet Jokteel, såsom det heter ännu i dag.
2Ki 14:8  Därefter skickade Amasja sändebud till Joas, son till Joahas, son till Jehu, Israels konung, och lät säga: "Kom, låt oss drabba samman med varandra."
2Ki 14:9  Men Joas, Israels konung, sände då till Amasja, Juda konung, och lät svara: "Törnbusken på Libanon sände en gång bud till cedern på Libanon och lät säga: 'Giv din dotter åt min son till hustru.' Men sedan gingo markens djur på Libanon fram över törnbusken och trampade ned den.
2Ki 14:10  Du har slagit Edom, och däröver förhäver du dig i ditt hjärta. Men låt dig nöja med den äran, och stanna hemma. Varför utmanar du olyckan, dig själv och Juda med dig till fall?"
2Ki 14:11  Men Amasja ville icke höra härpå, och Joas, Israels konung, drog då upp, och de drabbade samman med varandra, han och Amasja, Juda konung, vid det Bet-Semes som hör till Juda.
2Ki 14:12  Och Juda män blevo slagna av Israels män och flydde, var och en till sin hydda.
2Ki 14:13  Och Amasja, Juda konung, son till Joas, son till Ahasja, blev tagen till fånga i Bet-Semes av Joas, Israels konung. Och när de kommo till Jerusalem, bröt han ned ett stycke av Jerusalems mur vid Efraimsporten, och därifrån ända till Hörnporten, fyra hundra alnar.
2Ki 14:14  Och han tog allt guld och silver och alla kärl som funnos i HERRENS hus och i konungshusets skattkamrar, därtill ock gisslan, och vände så tillbaka till Samaria.
2Ki 14:15  Vad nu mer är att säga om Joas, om vad han gjorde och om hans bedrifter och om hans krig mot Amasja, Juda konung, det finnes upptecknat i Israels konungars krönika.
2Ki 14:16  Och Joas gick till vila hos sina fäder och blev begraven i Samaria, hos Israels konungar. Och hans son Jerobeam blev konung efter honom.
2Ki 14:17  Men Amasja, Joas' son, Juda konung, levde i femton år efter Joas', Joahas' sons, Israels konungs, död.
2Ki 14:18  Vad nu mer är att säga om Amasja, det finnes upptecknat i Juda konungars krönika.
2Ki 14:19  Och en sammansvärjning anstiftades mot honom i Jerusalem, så att han måste fly till Lakis. Då sändes män efter honom till Lakis, och dessa dödade honom där.
2Ki 14:20  Sedan förde man honom därifrån på hästar; och han blev begraven i Jerusalem hos sina fäder i Davids stad.
2Ki 14:21  Och allt folket i Juda tog Asarja som då var sexton år gammal, och gjorde honom till konung i hans fader Amasjas ställe.
2Ki 14:22  Det var han som befäste Elat; ock han lade det åter under Juda, sedan konungen hade gått till vila hos sina fäder.
2Ki 14:23  I Amasjas, Joas' sons, Juda konungs, femtonde regeringsår blev Jerobeam, Joas' son, konung över Israel i Samaria och regerade i fyrtioett år.
2Ki 14:24  Han gjorde vad ont var i HERRENS ögon; han avstod icke från någon av de Jerobeams, Nebats sons, synder genom vilka denne hade kommit Israel att synda.
2Ki 14:25  Han återvann Israels område, från det ställe där vägen går till Hamat ända till Hedmarkshavet, i enlighet med det ord som HERREN, Israels Gud, hade talat genom sin tjänare profeten Jona, Amittais son, från Gat-Hahefer.
2Ki 14:26  Ty HERREN såg att Israels betryck var mycket svårt, och att det var ute med alla och envar, och att Israel icke hade någon hjälpare.
2Ki 14:27  Och HERREN hade ännu icke beslutit att utplåna Israels namn under himmelen; därför frälste han dem genom Jerobeam, Joas' son.
2Ki 14:28  Vad nu mer är att säga om Jerobeam, om allt vad han gjorde och om hans bedrifter och hans krig, så ock om huru han åt Israel återvann den del av Damaskus och Hamat, som en gång hade tillhört Juda, det finnes upptecknat i Israels konungars krönika.
2Ki 14:29  Och Jerobeam gick till vila hos sina fäder, Israels konungar. Och hans son Sakarja blev konung efter honom.
2Ki 15:1  I Jerobeams, Israels konungs, tjugusjunde regeringsår blev Asarja, Amasjas son, konung i Juda.
2Ki 15:2  Han var sexton år gammal, när han blev konung, och han regerade femtiotvå år i Jerusalem. Hans moder hette Jekolja, från Jerusalem.
2Ki 15:3  Han gjorde vad rätt var i HERRENS ögon, alldeles såsom hans fader Amasja hade gjort.
2Ki 15:4  Dock blevo offerhöjderna icke avskaffade, utan folket fortfor att frambära offer och tända offereld på höjderna.
2Ki 15:5  Men HERREN hemsökte konungen, så att han blev spetälsk för hela sitt liv; och han bodde sedan i ett särskilt hus. Jotam, konungens son, förestod då hans hus och dömde folket i landet.
2Ki 15:6  Vad nu mer är att säga om Asarja och om allt vad han gjorde, det finnes upptecknat i Juda konungars krönika.
2Ki 15:7  Och Asarja gick till vila hos sina fäder, och man begrov honom hos hans fäder i Davids stad. Och hans son Jotam blev konung efter honom.
2Ki 15:8  I Asarjas, Juda konungs, trettioåttonde regeringsår blev Sakarja, Jerobeams son, konung över Israel i Samaria och regerade i sex månader.
2Ki 15:9  Han gjorde vad ont var i HERRENS ögon, såsom hans fäder hade gjort; han avstod icke från de Jerobeams, Nebats sons, synder genom vilka denne hade kommit Israel att synda.
2Ki 15:10  Och Sallum, Jabes' son, anstiftade en sammansvärjning mot honom och slog honom till döds i folkets åsyn, och blev så konung i hans ställe
2Ki 15:11  Vad nu mer är att säga om Sakarja, det finnes upptecknat i Israels konungars krönika.
2Ki 15:12  Så, uppfylldes det ord som HERREN, hade talat till Jehu, när han sade: "Dina söner till fjärde led skola sitta på Israels tron." Det skedde så.
2Ki 15:13  Sallum, Jabes' son, blev konung i Ussias, Juda konungs, trettionionde regeringsår, och han regerade en månads tid i Samaria.
2Ki 15:14  Men då drog Menahem, Gadis son, upp från Tirsa och kom till Samaria och slog Sallum, Jabes' son, till döds i Samaria, och blev så konung i hans ställe
2Ki 15:15  Vad nu mer är att säga om Sallum och om den sammansvärjning han anstiftade, det finnes upptecknat i Israels konungars krönika.
2Ki 15:16  Vid den tiden förhärjade Menahem Tifsa med allt vad därinne var, så ock hela dess område, från Tirsa; ty staden hade icke öppnat portarna, därför härjade han den, och alla dess havande kvinnor lät han upprista.
2Ki 15:17  I Asarjas, Juda konungs, trettionionde regeringsår blev Menahem, Gadis son, konung över Israel och regerade i tio år, i Samaria.
2Ki 15:18  Han gjorde vad ont var i HERRENS ögon; han avstod icke, så länge han levde, från de Jerobeams Nebats sons, synder genom vilka denne hade kommit Israel att synda.
2Ki 15:19  Och Pul, konungen i Assyrien, föll in i landet; då gav Menahem åt Pul tusen talenter silver, för att han skulle understödja honom och befästa konungadömet i hans hand.
2Ki 15:20  Och de penningar som Menahem skulle giva åt konungen i Assyrien tog han ut genom att lägga skatt på alla rika män i Israel, en skatt av femtio siklar silver på var och en. Så vände då konungen i Assyrien tillbaka och stannade icke där i landet.
2Ki 15:21  Vad nu mer är att säga om Menahem och om allt vad han gjorde, det finnes upptecknat i Israels konungars krönika.
2Ki 15:22  Och Menahem gick till vila hos sina fäder. Och hans son Pekaja blev konung efter honom.
2Ki 15:23  I Asarjas, Juda konungs, femtionde regeringsår blev Pekaja, Menahems son, konung över Israel i Samaria och regerade i två år.
2Ki 15:24  Han gjorde vad ont var i HERRENS ögon; han avstod icke från de Jerobeams, Nebats sons, synder genom vilka denne hade kommit Israel att synda.
2Ki 15:25  Och Peka, Remaljas son, hans livkämpe, anstiftade en sammansvärjning mot honom och dräpte honom i Samaria, i konungshusets palatsbyggnad, han tillika med Argob och Arje; därvid hade han med sig femtio gileaditer. Så dödade han honom och blev konung i hans ställe.
2Ki 15:26  Vad nu mer är att säga om Pekaja och om allt vad han gjorde, det finnes upptecknat i Israels konungars krönika.
2Ki 15:27  I Asarjas, Juda konungs, femtioandra regeringsår blev Peka, Remaljas son, konung över Israel i Samaria och regerade i tjugu år.
2Ki 15:28  Han gjorde vad ont var i HERRENS Ögon; han avstod icke från de Jerobeams; Nebats sons, synder genom vilka denne hade kommit Israel att synda.
2Ki 15:29  I Pekas, Israels konungs, tid kom Tiglat-Pileser, konungen i Assyrien, och intog Ijon, Abel-Bet-Maaka, Janoa, Kedes, Hasor, Gilead och Galileen, hela Naftali land, och förde folket bort till Assyrien.
2Ki 15:30  Och Hosea, Elas son, anstiftade en sammansvärjning mot Peka, Remaljas son, och slog honom till döds och blev så konung i hans ställe, i Jotams, Ussias sons, tjugonde regeringsår.
2Ki 15:31  Vad nu mer är att säga om Peka och om allt vad han gjorde, det finnes upptecknat i Israels konungars krönika.
2Ki 15:32  I Pekas, Remaljas sons, Israels konungs, andra regeringsår blev Jotam, Ussias son, konung i Juda.
2Ki 15:33  Han var tjugufem år gammal, när han blev konung, och han regerade sexton år i Jerusalem. Hans moder hette Jerusa, Sadoks dotter.
2Ki 15:34  Han gjorde vad rätt var i HERRENS ögon; han gjorde alldeles såsom hans fader Ussia hade gjort.
2Ki 15:35  Dock blevo offerhöjderna icke avskaffade, utan folket fortfor att frambära offer och tända offereld på höjderna. Han byggde Övre porten till HERRENS hus.
2Ki 15:36  Vad nu mer är att säga om Jotam, om vad han gjorde, det finnes upptecknat i Juda konungars krönika.
2Ki 15:37  Vid den tiden begynte HERREN att låta Juda hemsökas av Resin, konungen i Aram, och av Peka, Remaljas son.
2Ki 15:38  Och Jotam gick till vila hos sina fäder och blev begraven hos sina fäder i sin fader Davids stad. Och hans son Ahas blev konung efter honom.
2Ki 16:1  I Pekas, Remaljas sons, sjuttonde regeringsår blev Ahas, Jotams son, konung i Juda.
2Ki 16:2  Tjugu år gammal var Ahas, när han blev konung, och han regerade sexton år i Jerusalem. Han gjorde icke vad rätt var i HERRENS, sin Guds, ögon, såsom hans fader David,
2Ki 16:3  utan vandrade på Israels konungars väg; ja, han lät ock sin son gå genom eld, efter den styggeliga seden hos de folk som HERREN hade fördrivit för Israels barn.
2Ki 16:4  Och han frambar offer och tände offereld på höjderna och kullarna och under alla gröna träd.
2Ki 16:5  På den tiden drogo Resin, konungen i Aram, och Peka, Remaljas son, Israels konung, upp för att erövra Jerusalem; och de inneslöto Ahas, men kunde icke erövra staden.
2Ki 16:6  Vid samma tid återvann Resin, konungen i Aram, Elat åt Aram och jagade Juda män bort ifrån Elot. Därefter kommo araméerna till Elat och bosatte sig där, och där bo de ännu i dag.
2Ki 16:7  Men Ahas skickade sändebud till Tiglat-Pileser, konungen i Assyrien, och lät säga: "Jag är din tjänare och din son. Drag hitupp och fräls mig från Arams konung och från Israels konung, ty de hava överfallit mig."
2Ki 16:8  Och Ahas tog det silver och guld som fanns i HERRENS hus och i konungshusets skattkamrar, och sände det såsom skänk till konungen i Assyrien.
2Ki 16:9  Och konungen i Assyrien lyssnade till honom: konungen i Assyrien drog upp mot Damaskus och intog det och förde bort folket till Kir och dödade Resin.
2Ki 16:10  Sedan for konung Ahas till Damaskus för att där möta Tiglat-Pileser, konungen i Assyrien. Och när konung Ahas fick se altaret i Damaskus, sände han till prästen Uria en avteckning av altaret och en mönsterbild till ett sådant, alldeles såsom det var gjort.
2Ki 16:11  Sedan byggde prästen Uria altaret; alldeles efter den föreskrift som konung Ahas hade sänt till honom från Damaskus gjorde prästen Uria det färdigt, till dess konung Ahas kom tillbaka från Damaskus.
2Ki 16:12  När så konungen efter sin hemkomst från Damaskus fick se altaret, trädde han fram till altaret och steg upp till det.
2Ki 16:13  Därefter förbrände han sitt brännoffer och sitt spisoffer och utgöt sitt drickoffer; och blodet av det tackoffer som han offrade stänkte han på altaret.
2Ki 16:14  Men kopparaltaret, som stod inför HERRENS ansikte, flyttade han undan från husets framsida, från platsen mellan det nya altaret och HERRENS hus, och ställde det på norra sidan om detta altare.
2Ki 16:15  Och konung Ahas bjöd prästen Uria och sade: "På det stora altaret skall du förbränna morgonens brännoffer och aftonens spisoffer, ävensom konungens brännoffer jämte hans spisoffer, så ock brännoffer, spisoffer och drickoffer för allt folket i landet; och allt blodet, såväl av brännoffer som av slaktoffer, skall du stänka därpå. Men vad Jag skall göra med kopparaltaret, det vill jag närmare betänka."
2Ki 16:16  Och prästen Uria gjorde alldeles såsom konung Ahas bjöd honom.
2Ki 16:17  Konung Ahas lösbröt ock sidolisterna på bäckenställen och tog bort bäckenet från dem; och havet lyfte han ned från kopparoxarna som stodo därunder och ställde det på ett stengolv.
2Ki 16:18  Och den täckta sabbatsgången som man hade byggt vid huset, så ock konungens yttre ingångsväg, förlade han inom HERRENS hus, för den assyriske konungens skull.
2Ki 16:19  Vad nu mer är att säga om Ahas, om vad han gjorde, det finnes upptecknat i Juda konungars krönika.
2Ki 16:20  Och Ahab gick till vila hos sina fäder och blev begraven hos sina fäder i Davids stad. Och hans son Hiskia blev konung efter honom.
2Ki 17:1  I Ahas', Juda konungs, tolfte regeringsår blev Hosea, Elas son, konung i Samaria över Israel och regerade i nio år.
2Ki 17:2  Han gjorde vad ont var i HERRENS; ögon, dock icke såsom de israelitiska konungar som hade varit före honom
2Ki 17:3  Mot honom drog den assyriske konungen Salmaneser upp; och Hosea måste bliva honom underdånig och giva honom skänker.
2Ki 17:4  Men sedan märkte konungen i Assyrien att Hosea förehade stämplingar, i det att han skickade sändebud till So, konungen i Egypten och icke, såsom förut, vart år sände skänker till konungen i Assyrien Då lät konungen i Assyrien spärra in honom och hålla honom bunden i fängelse.
2Ki 17:5  Ty konungen i Assyrien drog upp och angrep hela landet, och drog upp mot Samaria och belägrade det i tre år.
2Ki 17:6  I Hoseas nionde regeringsår intog konungen i Assyrien Samaria och förde Israel bort till Assyrien och lät dem bo i Hala och vid Habor - en ström i Gosan - och i Mediens städer.
2Ki 17:7  Israels barn hade ju syndat mot HERREN, sin Gud, honom som hade fört dem upp ur Egyptens land, undan Faraos, den egyptiske konungens, hand, och de hade fruktat andra gudar.
2Ki 17:8  De hade ock vandrat efter de folks stadgar, som HERREN hade fördrivit för Israels barn, och efter de stadgar som Israels konungar hade uppgjort.
2Ki 17:9  Ja, Israels barn hade bedrivit otillbörliga ting mot HERREN, sin Gud; de hade byggt sig offerhöjder på alla sina boningsorter, vid väktartornen såväl som i de befästa städerna.
2Ki 17:10  De hade rest stoder och Aseror åt sig på alla höga kullar och under alla gröna träd.
2Ki 17:11  Där hade de på alla offerhöjder tänt offereld, likasom de folk som HERREN hade drivit bort för dem, och hade gjort onda ting, så att de förtörnade HERREN.
2Ki 17:12  De hade tjänat de eländiga avgudarna, fastän HERREN hade sagt till dem: "I skolen icke göra så.
2Ki 17:13  Och HERREN hade varnat både Israel och Juda genom alla sina profeter och siare och sagt: "Vänden om från edra onda vägar och hållen mina bud och stadgar - efter hela den lag som jag gav edra fäder - så ock vad jag har låtit säga eder genom mina tjänare profeterna."
2Ki 17:14  Men de ville icke höra, utan voro hårdnackade såsom deras fäder, vilka icke trodde på HERREN, sin Gud.
2Ki 17:15  De förkastade hans stadgar och det förbund som han hade slutit med deras fäder, och de förordningar som han hade givit dem, och följde efter fåfängliga avgudar och bedrevo fåfänglighet, likasom de folk som voro omkring dem, fastän HERREN hade förbjudit dem att göra såsom dessa.
2Ki 17:16  De övergåvo HERRENS, sin Guds, alla bud och gjorde sig gjutna beläten, två kalvar; de gjorde sig ock Aseror och tillbådo himmelens hela härskara och tjänade Baal.
2Ki 17:17  Och de läto sina söner och döttrar gå genom eld och befattade sig med spådom och övade trolldom de sålde sig till att göra vad ont var i HERRENS ögon och förtörnade honom därmed.
2Ki 17:18  Därför blev ock HERREN mycket vred på Israel och försköt dem från sitt ansikte, så att icke något annat blev kvar än Juda stam allena.
2Ki 17:19  Dock höll icke heller Juda HERRENS, sin Guds, bud, utan vandrade efter de stadgar som Israel hade uppgjort.
2Ki 17:20  Så förkastade då HERREN all Israels säd och tuktade dem och gav dem i plundrares hand, till dess att han kastade dem bort ifrån sitt ansikte.
2Ki 17:21  Ty när han hade ryckt Israel från Davids hus och de hade gjort Jerobeam, Nebats son, till konung, förförde Jerobeam Israel till att avfalla från HERREN och kom dem att begå en stor synd.
2Ki 17:22  Och Israels barn vandrade i alla de synder son Jerobeam hade gjort; de avstodo icke från dem.
2Ki 17:23  Men till slut försköt HERREN Israel från sitt ansikte, såsom har hade hotat genom alla sina tjänare profeterna. Så blev Israel; bortfört från sitt land till Assyrien, där de äro ännu i dag.
2Ki 17:24  Och konungen i Assyrien lät folk komma från Babel, Kuta, Ava, Hamat och Sefarvaim och bosätta sig i Samariens städer, i Israels barns ställe. Så togo då dessa Samarien i besittning och bosatte sig i dess städer.
2Ki 17:25  Men då de under den första tiden av sin vistelse där icke fruktade HERREN, sände HERREN bland dem lejon, som anställde förödelse bland dem.
2Ki 17:26  Och man omtalade detta för konungen i Assyrien och sade: "De folk som du har fört bort och låtit bosätta sig i Samariens städer veta icke huru landets Gud skall dyrkas därför har han sänt lejon ibland dem, och dessa döda dem nu, eftersom de icke veta huru landets Gud skall dyrkas."
2Ki 17:27  Då bjöd konungen i Assyrien och sade: "Låten en av de präster som I haven fört bort därifrån fara dit; må de fara dit och bosätta sig där. Och må han lära dem huru landets Gud skall dyrkas."
2Ki 17:28  Så kom då en av de präster som: de hade fört bort ifrån Samarien och bosatte sig i Betel; och han lärde dem huru de skulle frukta HERREN.
2Ki 17:29  Väl gjorde sig vart folk sin egen gud och ställde upp denne i de offerhöjdshus som samariterna hade uppbyggt, vart folk för sig, i de städer där det bodde.
2Ki 17:30  Folket ifrån Babel gjorde sig en Suckot-Benot, folket ifrån Kut gjorde sig en Nergal, och folket ifrån Hamat gjorde sig en Asima;
2Ki 17:31  aviterna gjorde sig en Nibhas och en Tartak, och sefarviterna brände upp sina barn i eld åt Adrammelek och Anammelek, Sefarvaims gudar.
2Ki 17:32  Men de fruktade också HERREN. Och de gjorde män ur sin egen krets till offerhöjdspräster åt sig, och dessa offrade åt dem i offerhöjdshusen.
2Ki 17:33  Så fruktade de visserligen HERREN, men de tjänade därjämte sina egna gudar, på samma sätt som de folk ifrån vilka man hade fört bort dem.
2Ki 17:34  Ännu i dag göra de likasom förut: de frukta icke HERREN och göra icke efter de stadgar och den rätt som de hava fått, icke efter den lag och de bud som HERREN har givit Jakobs barn, den mans åt vilken han gav namnet Israel.
2Ki 17:35  Ty HERREN slöt ett förbund med dem och bjöd dem och sade: "I skolen icke frukta andra gudar, ej heller tillbedja dem eller tjäna dem eller offra åt dem.
2Ki 17:36  Nej, HERREN allena, som förde eder upp ur Egyptens land med stor makt och uträckt arm, honom skolen I frukta, honom skolen I tillbedja och åt honom skolen I offra.
2Ki 17:37  Och de stadgar och rätter, den lag och de bud som han har föreskrivit eder, dem skolen I hålla och göra i all tid; men I skolen icke frukta andra gudar.
2Ki 17:38  Det förbund som jag har slutit med eder skolen I icke förgäta; I skolen icke frukta andra gudar.
2Ki 17:39  Allenast HERREN, eder Gud, skolen I frukta, så skall han rädda eder ur alla edra fienders hand."
2Ki 17:40  Men de ville icke höra härpå, utan gjorde likasom förut.
2Ki 17:41  Så fruktade då dessa folk HERREN, men tjänade därjämte sina beläten. Också deras barn och deras barnbarn göra ännu i dag såsom deras fäder gjorde.
2Ki 18:1  I Hoseas, Elas sons, Israels konungs, tredje regeringsår blev Hiskia, Ahas' son, konung i Juda.
2Ki 18:2  Han var tjugufem år gammal, när han blev konung, och han regerade tjugunio år i Jerusalem. Hans moder hette Abi, Sakarjas dotter.
2Ki 18:3  Han gjorde vad rätt var i HERRENS ögon, alldeles såsom hans fader David hade gjort.
2Ki 18:4  Han avskaffade offerhöjderna, slog sönder stoderna och högg ned Aseran. Han krossade ock den kopparorm som Mose hade gjort; ty ända till denna tid hade Israels barn tänt offereld åt denne. Man kallade honom Nehustan.
2Ki 18:5  På HERREN, Israels Gud, förtröstade han, så att ingen var honom lik bland alla Juda konungar efter honom, ej heller bland dem som hade varit före honom.
2Ki 18:6  Han höll sig till HERREN och vek icke av ifrån honom, utan höll hans bud, dem som HERREN hade givit Mose.
2Ki 18:7  Och HERREN var med honom, så att han hade framgång i allt vad han företog sig. Han avföll från konungen i Assyrien och upphörde att vara honom underdånig.
2Ki 18:8  Han slog ock filistéerna och intog deras land ända till Gasa med dess område, såväl väktartorn som befästa städer.
2Ki 18:9  I konung Hiskias fjärde regeringsår, som var Hoseas, Elas sons, Israels konungs, sjunde regeringsår, drog Salmaneser, konungen i Assyrien, upp mot Samaria och belägrade det.
2Ki 18:10  Och de intogo det efter tre år, i Hiskias sjätte regeringsår; under detta år, som var Hoseas, Israels konungs, nionde regeringsår, blev Samaria intaget.
2Ki 18:11  Och konungen i Assyrien förde Israel bort till Assyrien och förflyttade dem till Hala och till Habor, en ström i Gosan, och till Mediens städer -
2Ki 18:12  detta därför att de icke hörde HERRENS, sin Guds, röst, utan överträdde hans förbund, allt vad HERRENS tjänare Mose hade bjudit; de ville varken höra eller göra det.
2Ki 18:13  Och i konung Hiskias fjortonde regeringsår drog Sanherib, konungen i Assyrien, upp och angrep alla befästa städer i Juda och intog dem.
2Ki 18:14  Då sände Hiskia, Juda konung, till Assyriens konung i Lakis och lät säga: "Jag har försyndat mig; vänd om och lämna mig i fred. Vad du lägger på mig vill jag bära." Då ålade konungen i Assyrien Hiskia, Juda konung, att betala tre hundra talenter silver och trettio talenter guld.
2Ki 18:15  Och Hiskia gav ut alla de penningar som funnos i HERRENS hus och i konungshusets skattkamrar.
2Ki 18:16  Vid detta tillfälle lösbröt ock Hiskia från dörrarna till HERRENS tempel och från dörrposterna den beläggning som Hiskia, Juda konung, hade överdragit dem med, och gav detta åt konungen i Assyrien.
2Ki 18:17  Men konungen i Assyrien sände från Lakis åstad Tartan, Rab-Saris och Rab-Sake med en stor här mot konung Hiskia i Jerusalem. Dessa drogo då upp och kommo till Jerusalem; och när de hade dragit ditupp och kommit fram, stannade de vid Övre dammens vattenledning, på vägen till Valkarfältet.
2Ki 18:18  Och de begärde att få tala med konungen. Då gingo överhovmästaren Eljakim, Hilkias son, och sekreteraren Sebna och kansleren Joa, Asafs son, ut till dem.
2Ki 18:19  Och Rab-Sake sade till dem: "Sägen till Hiskia: Så säger den store konungen, konungen i Assyrien: Vad är det för en förtröstan som du nu har hängivit dig åt?
2Ki 18:20  Du menar väl att allenast munväder behövs för att veta råd och hava makt att föra krig. På vem förtröstar du då, eftersom du har satt dig upp mot mig?
2Ki 18:21  Du förtröstar val nu på den bräckta rörstaven Egypten, men se, när någon stöder sig på den, går den in i hans hand och genomborrar den. Ty sådan är Farao, konungen i Egypten, för alla som förtrösta på honom.
2Ki 18:22  Eller sägen I kanhända till mig: 'Vi förtrösta på HERREN, vår Gud'? Var det då icke hans offerhöjder och altaren Hiskia avskaffade, när han sade till Juda och Jerusalem: 'Inför detta altare skolen I tillbedja, har i Jerusalem'?
2Ki 18:23  Men ingå nu ett vad med min herre, konungen i Assyrien: jag vill giva dig två tusen hästar, om du kan skaffa dig ryttare till dem.
2Ki 18:24  Huru skulle du då kunna slå tillbaka en enda ståthållare, en av min herres ringaste tjänare? Och du sätter din förtröstan till Egypten, i hopp om att så få vagnar och ryttare!
2Ki 18:25  Menar du då att jag utan HERRENS vilja har dragit upp till detta ställe för att fördärva det? Nej, det är HERREN som har sagt till mig: Drag upp mot detta land och fördärva det."
2Ki 18:26  Då sade Eljakim, Hilkias son, och Sebna och Joa till Rab-Sake: "Tala till dina tjänare på arameiska, ty vi förstå det språket, och tala icke med oss på judiska inför folket som står på muren."
2Ki 18:27  Men Rab-Sake svarade dem: "Är det då till din herre och till dig som min herre har sänt mig att tala dessa ord? Är det icke fastmer till de män som sitta på muren, och som jämte eder skola nödgas äta sin egen träck och dricka sitt eget vatten?"
2Ki 18:28  Därefter trädde Rab-Sake närmare och ropade med hög röst på judiska och talade och sade: "Hören den store konungens, den assyriske konungens, ord.
2Ki 18:29  Så säger konungen: Låten icke Hiskia bedraga eder, ty han förmår icke rädda eder ur min hand
2Ki 18:30  Och låten icke Hiskia förleda eder att förtrösta på HERREN, därmed att han säger: 'HERREN skall förvisso rädda oss, och denna stad skall icke bliva given i den assyriske konungens hand.'
2Ki 18:31  Hören icke på Hiskia. Ty så säger konungen i Assyrien: Gören upp i godo med mig och given eder åt mig, så skolen I få äta var och en av sitt vinträd och av sitt fikonträd och dricka var och en ur sin brunn,
2Ki 18:32  till dess jag kommer och hämtar eder till ett land som är likt edert eget land, ett land med säd och vin, ett land med bröd och vingårdar, ett land med ädla olivträd och honung; så skolen I få leva och icke dö. Men hören icke på Hiskia; ty han vill förleda eder, när han säger: 'HERREN skall rädda oss.'
2Ki 18:33  Har väl någon av de andra folkens gudar någonsin räddat sitt land ur den assyriske konungens hand?
2Ki 18:34  Var äro Hamats och Arpads gudar? Var äro Sefarvaims, Henas och Ivas gudar? Eller hava de räddat Samaria ur min hand?
2Ki 18:35  Vilken bland andra länders alla gudar har väl räddat sitt land ur min hand, eftersom I menen att HERREN skall rädda Jerusalem ur min hand?"
2Ki 18:36  Men folket teg och svarade honom icke ett ord, ty konungen hade så bjudit och sagt: "Svaren honom icke."
2Ki 18:37  Och överhovmästaren Eljakim, Hilkias son, och sekreteraren Sebna och kansleren Joa, Asafs son, kommo till Hiskia med sönderrivna kläder och berättade för honom vad Rab-Sake hade sagt.
2Ki 19:1  Då nu konung Hiskia hörde detta, rev han sönder sina kläder och höljde sig i sorgdräkt och gick in i HERRENS hus.
2Ki 19:2  Och överhovmästaren Eljakim och sekreteraren Sebna och de äldste bland prästerna sände han, höljda i sorgdräkt, till profeten Jesaja, Amos' son.
2Ki 19:3  Och de sade till denne: "Så säger Hiskia: En nödens, tuktans och smälekens dag är denna dag, ty fostren hava väl kommit fram till födseln, men kraft att föda finnes icke.
2Ki 19:4  Kanhända skall HERREN, din Gud, höra alla Rab-Sakes ord, med vilka hans herre, konungen i Assyrien, har sänt honom till att smäda den levande Guden, så att han straffar honom för dessa ord, som han, HERREN, din Gud, har hört. Så bed nu en bön för den kvarleva som ännu finnes."
2Ki 19:5  När nu konung Hiskias tjänare kommo till Jesaja,
2Ki 19:6  sade Jesaja till dem: "Så skolen I säga till eder herre: Så säger HERREN: Frukta icke för de ord som du har hört, dem med vilka den assyriske konungens tjänare hava hädat mig.
2Ki 19:7  Se, jag skall låta en sådan ande komma in i honom, att han, på grund av ett rykte som han skall få höra, vänder tillbaka till sitt land; och jag skall låta honom falla för svärd i hans eget land.
2Ki 19:8  Och Rab-Sake vände tillbaka och fann den assyriske konungen upptagen med att belägra Libna; ty han hade hört att han hade brutit upp från Lakis.
2Ki 19:9  Men när Sanherib fick höra säga om Tirhaka, konungen i Etiopien, att denne hade dragit ut för att strida mot honom, skickade han åter sändebud till Hiskia och sade:
2Ki 19:10  "Så skolen I säga till Hiskia, Juda konung: Låt icke din Gud, som du förtröstar på, bedraga dig, i det att du tänker: 'Jerusalem skall icke bliva givet i den assyriske konungens hand.'
2Ki 19:11  Du har nu hört vad konungarna i Assyrien hava gjort med alla andra länder, huru de hava givit dem till spillo. Och du skulle nu bliva räddad!
2Ki 19:12  Hava väl de folk som mina fäder fördärvade, Gosan, Haran, Resef och Edens barn i Telassar, blivit räddade av sina gudar?
2Ki 19:13  Var är Hamats konung och Arpads konung och konungen över Sefarvaims stad, över Hena och Iva?"
2Ki 19:14  När Hiskia hade mottagit brevet av sändebuden och läst det, gick han upp i HERRENS hus, och där bredde Hiskia ut det inför HERRENS ansikte.
2Ki 19:15  Och Hiskia bad inför HERRENS ansikte och sade: "HERRE, Israels Gud, du som tronar på keruberna, du allena är Gud, den som råder över alla riken på jorden; du har gjort himmel och jord.
2Ki 19:16  HERRE, böj ditt öra härtill och hör; HERRE, öppna dina ögon och se. Ja, hör Sanheribs ord, det budskap varmed han har smädat den levande Guden.
2Ki 19:17  Det är sant, HERRE, att konungarna i Assyrien hava förött folken och deras land.
2Ki 19:18  Och de hava kastat deras gudar i elden; ty dessa voro inga gudar, utan verk av människohänder, trä och sten; därför kunde de förgöra dem.
2Ki 19:19  Men fräls oss nu, HERRE, vår Gud, ur hans hand, så att alla riken på jorden förnimma att du, HERRE, allena är Gud."
2Ki 19:20  Då sände Jesaja, Amos' son, bud till Hiskia och låt säga: "Så säger HERREN, Israels Gud: Det varom du har bett mig angående Sanherib, konungen i Assyrien, det har jag hört.
2Ki 19:21  Så är nu detta det ord som HERREN har talat om honom: Hon föraktar dig och bespottar dig, jungfrun dottern Sion; hon skakar huvudet efter dig, dottern Jerusalem.
2Ki 19:22  vem har du smädat och hädat, och mot vem har du upphävt din röst? Alltför högt har du upplyft dina ögon - Ja, mot Israels Helige.
2Ki 19:23  Genom dina sändebud smädade du Herren, när du sade: 'Med mina många vagnar drog jag upp på bergens höjder, längst upp på Libanon; jag högg ned dess höga cedrar och väldiga cypresser; jag trängde fram till dess innersta gömslen, dess frodigaste skog;
2Ki 19:24  jag grävde brunnar och drack ut främmande vatten, och med min fot uttorkade jag alla Egyptens strömmar.'
2Ki 19:25  Har du icke hört att jag för länge sedan beredde detta? Av ålder bestämde jag ju så; och nu har jag fört det fram: du fick makt att ödelägga befästa städer till grusade stenhopar.
2Ki 19:26  Deras invånare blevo maktlösa, de förfärades och stodo med skam. Det gick dem såsom gräset på marken och gröna örter, såsom det som växer på taken, och säd som förbrännes, förrän strået har vuxit upp.
2Ki 19:27  Om du sitter eller går ut eller går in, så vet jag det, och huru du rasar mot mig.
2Ki 19:28  Men då du nu så rasar mot mig, och då ditt övermod har nått till mina öron, skall jag sätta min krok i din näsa och mitt betsel i din mun och föra dig tillbaka samma väg som du har kommit på.
2Ki 19:29  Och detta skall för dig vara tecknet: man skall detta år äta vad som växer upp av spillsäd, och nästa år självvuxen säd; men det tredje året skolen I få så och skörda och plantera vingårdar och äta deras frukt.
2Ki 19:30  Och den räddade skara av Juda hus, som bliver kvar, skall åter skjuta rot nedtill och bära frukt upptill.
2Ki 19:31  Ty från Jerusalem skall utgå en kvarleva, en räddad skara från Sions berg. HERRENS nitälskan skall göra detta.
2Ki 19:32  Därför säger HERREN så om konungen i Assyrien: Han skall icke komma in i denna stad och icke skjuta någon pil ditin; han skall icke mot den föra fram någon sköld eller kasta upp någon vall mot den.
2Ki 19:33  Samma väg han kom skall han vända tillbaka, och in i denna stad skall han icke komma, säger HERREN.
2Ki 19:34  Ty jag skall beskärma och frälsa denna stad för min och min tjänare Davids skull."
2Ki 19:35  Och samma natt gick HERRENS ängel ut och slog i assyriernas läger ett hundra åttiofem tusen man; och när man bittida följande morgon kom ut, fick man se döda kroppar ligga där överallt
2Ki 19:36  Då bröt Sanherib, konungen i Assyrien, upp och tågade tillbaka; och han stannade sedan i Nineve.
2Ki 19:37  Men när han en gång tillbad i sin gud Nisroks tempel, blev han dräpt med svärd av Adrammelek och Sareser; därefter flydde dessa undan till Ararats land. Och hans son Esarhaddon blev konung efter honom.
2Ki 20:1  Vid den tiden blev Hiskia dödssjuk; och profeten Jesaja, Amos' son, kom till honom och sade till honom: "Så säger HERREN: Beställ om ditt hus; ty du måste dö och skall icke tillfriskna."
2Ki 20:2  Då vände han sitt ansikte mot väggen och bad till HERREN och sade:
2Ki 20:3  "Ack HERRE, tänk dock på huru jag har vandrat inför dig i trohet och med hängivet hjärta och gjort vad gott är i dina ögon." Och Hiskia grät bitterligen.
2Ki 20:4  Men innan Jesaja hade hunnit ut ur den inre staden, kom HERRENS ord till honom; han sade:
2Ki 20:5  "Vänd om och säg till Hiskia, fursten över mitt folk: Så säger HERREN, din fader Davids Gud: Jag har hört din bön, jag har sett dina tårar. Se, jag vill göra dig frisk; i övermorgon skall du få gå upp i HERRENS hus.
2Ki 20:6  Och jag skall föröka din livstid med femton år; jag skall ock rädda dig och denna stad ur den assyriske konungens hand. Ja, jag skall beskärma denna stad, för min skull och för min tjänare Davids skull.
2Ki 20:7  Och Jesaja sade: "Hämten hit en fikonkaka." Då hämtade man en sådan och lade den på bulnaden. Och han tillfrisknade.
2Ki 20:8  Och Hiskia sade till Jesaja: "Vad för ett tecken gives mig därpå att HERREN skall göra mig frisk, så att jag i övermorgon får gå upp i HERRENS hus?"
2Ki 20:9  Jesaja svarade: "Detta skall för dig vara tecknet från HERREN därpå att HERREN skall göra vad han har lovat: skuggan har gått tio steg framåt; skall den nu gå tio steg tillbaka?"
2Ki 20:10  Hiskia sade: "Det är lätt för skuggan att sträcka sig tio steg framåt. Nej, låt skuggan gå tio steg tillbaka."
2Ki 20:11  Då ropade profeten Jesaja till HERREN, och han lät skuggan på Ahas' solvisare gå tillbaka de tio steg som den redan hade lagt till rygga.
2Ki 20:12  Vid samma tid sände Berodak-Baladan, Baladans son, konungen i Babel, brev och skänker till Hiskia, ty han hade sport att Hiskia hade varit sjuk.
2Ki 20:13  Och när Hiskia hade hört på dem, visade han dem hela sitt förrådshus, sitt silver och guld, sina välluktande kryddor och sina dyrbara oljor, och hela sitt tyghus och allt vad som fanns i hans skattkamrar. Intet fanns i Hiskias hus eller eljest i hans ägo, som han icke visade dem.
2Ki 20:14  Men profeten Jesaja kom till konung Hiskia och sade till honom: "Vad hava dessa män sagt, och varifrån hava de kommit till dig?" Hiskia svarade: "De hava kommit ifrån fjärran land, ifrån Babel."
2Ki 20:15  Han sade vidare: "Vad hava de sett i ditt hus?" Hiskia svarade: "Allt som är i mitt hus hava de sett; intet finnes i mina skattkamrar, som jag icke har visat dem."
2Ki 20:16  Då sade Jesaja till Hiskia: "Hör HERRENS ord:
2Ki 20:17  Se, dagar skola komma, då allt som finnes i ditt hus, och som dina fäder hava samlat ända till denna dag skall föras bort till Babel; intet skall bliva kvar, säger HERREN.
2Ki 20:18  Och söner till dig, de som skola utgå av dig, och som du skall föda, dem skall man taga, och de skola bliva hovtjänare i den babyloniske konungens palats."
2Ki 20:19  Hiskia sade till Jesaja: "Gott är det HERRENS ord som du har talat." Och han sade ytterligare: "Ja, om nu blott frid och trygghet få råda i min tid."
2Ki 20:20  Vad nu mer är att säga om Hiskia och om alla hans bedrifter, och om huru han anlade dammen och vattenledningen och ledde vatten in i staden, det finnes upptecknat Juda konungars krönika.
2Ki 20:21  Och Hiskia gick till vila hos sina fäder. Och hans son Manasse blev konung efter honom.
2Ki 21:1  Manasse var tolv år gammal, när han blev konung, och han regerade femtiofem år i Jerusalem. Hans moder hette Hefsi-Ba.
2Ki 21:2  Han gjorde vad ont var i HERRENS ögon, efter den styggeliga seden hos de folk som HERREN hade fördrivit för Israels barn.
2Ki 21:3  Han byggde åter upp de offerhöjder som hans fader Hiskia hade förstört, och reste altaren åt Baal och gjorde en Asera, likasom Ahab, Israels konung, hade gjort, och tillbad och tjänade himmelens hela härskara.
2Ki 21:4  Ja, han byggde altaren i HERRENS hus, det om vilket HERREN hade sagt: "Vid Jerusalem vill jag fästa mitt namn."
2Ki 21:5  Han byggde altaren åt himmelens hela härskara på de båda förgårdarna till HERRENS hus.
2Ki 21:6  Han lät ock sin son gå genom eld och övade teckentyderi och svartkonst och skaffade sig andebesvärjare och spåmän och gjorde mycket som var ont i HERRENS ögon, så att han förtörnade honom.
2Ki 21:7  Och Aserabelätet som han hade låtit göra satte han i det hus om vilket HERREN hade sagt till David och till hans son Salomo: "Vid detta hus och vid Jerusalem som jag har utvalt bland alla Israels stammar, vill jag fästa mitt namn för evig tid
2Ki 21:8  Och jag skall icke mer låta Israel vandra flyktig bort ifrån det land som jag har givit åt deras fäder, om de allenast hålla och göra allt vad jag har bjudit dem, och det alldeles efter den lag som min tjänare Mose har givit dem."
2Ki 21:9  Men de lyssnade icke härtill, och Manasse förförde dem, så att de gjorde mer ont än de folk som HERREN hade förgjort för Israels barn.
2Ki 21:10  Då talade HERREN genom sina tjänare profeterna och sade:
2Ki 21:11  "Eftersom Manasse, Juda konung, har bedrivit dessa styggelser och så gjort mer ont, än allt vad amoréerna som voro före honom hava gjort, så att han med sina eländiga avgudar har kommit också Juda att synda,
2Ki 21:12  därför säger HERREN, Israels Gud, så: 'Se, jag skall låta en sådan olycka komma över Jerusalem och Juda, att det skall genljuda i båda öronen på var och en som får höra det.
2Ki 21:13  Och mot Jerusalem skall jag bruka det mätsnöre som jag brukade mot Samaria, och det sänklod som jag brukade mot Ahabs hus; och jag skall skölja Jerusalem tomt, såsom man sköljer ett fat och, sedan man har sköljt det, vänder det upp och ned.
2Ki 21:14  Och jag skall förskjuta kvarlevan av min arvedel och giva dem i deras fienders hand, så att de skola bliva ett rov och ett byte för alla sina fiender -
2Ki 21:15  detta därför att de hava gjort vad ont är i mina ögon och beständigt förtörnat mig, från den dag då deras fader drogo ut ur Egypten ända till denna dag.'"
2Ki 21:16  Därtill utgöt ock Manasse oskyldigt blod i så stor myckenhet, att han därmed uppfyllde Jerusalem från den ena ändan till den andra - detta förutom den särskilda syns genom vilken han kom Juda att synda och göra vad ont var i HERRENS ögon.
2Ki 21:17  Vad nu mer är att säga om Manasse och om allt vad han gjorde så ock om den synd han begick det finnes upptecknat i Juda konungars krönika.
2Ki 21:18  Och Manasse gick till vila hos sina fäder och blev begraven i trädgården till sitt hus, i Ussas trädgård. Och hans son Amon blev konung efter honom.
2Ki 21:19  Amon var tjugutvå år gammal när han blev konung, och han regerade två år i Jerusalem. Hans moder hette Mesullemet, Harus' dotter, från Jotba.
2Ki 21:20  Han gjorde vad ont var i HERRENS ögon, såsom hans fader Manasse hade gjort.
2Ki 21:21  Han vandrade i allt på samma väg som hans fader hade vandrat, och tjänade och tillbad de eländiga avgudar som hans fader hade tjänat.
2Ki 21:22  Han övergav HERREN, sina fäders Gud, och vandrade icke på HERRENS väg.
2Ki 21:23  Och Amons tjänare sammansvuro sig mot honom och dödade konungen hemma i hans hus.
2Ki 21:24  Men folket i landet dräpte alla som hade sammansvurit sig mot konung Amon. Därefter gjorde folket i landet hans son Josia till konung efter honom.
2Ki 21:25  Vad nu mer är att säga om Amon, om vad han gjorde, det finnes upptecknat i Juda konungars krönika.
2Ki 21:25  Vad nu mer är att säga om Amon, om vad han gjorde, det finnes upptecknat i Juda konungars krönika.
2Ki 22:1  Josia var åtta år gammal, när han blev konung, och han regerade trettioett år i Jerusalem. Hans moder hette Jedida, Adajas dotter, från Boskat.
2Ki 22:2  Han gjorde vad rätt var i HERRENS ögon och vandrade i allt på sin fader Davids väg och vek icke av vare sig till höger eller till vänster.
2Ki 22:3  I sitt adertonde regeringsår sände konung Josia sekreteraren Safan, son till Asalja, Mesullams son, åstad till HERRENS hus och sade:
2Ki 22:4  "Gå upp till översteprästen Hilkia, och bjud honom att göra i ordning de penningar som hava influtit till HERRENS hus, sedan de hava blivit insamlade ifrån folket av dem som hålla vakt vid tröskeln.
2Ki 22:5  Och han skall överlämna dem åt de män som förrätta arbete såsom tillsyningsmän vid HERRENS hus, och dessa skola giva dem åt de män som arbeta vid HERRENS hus, för att sätta i stånd vad som är förfallet på huset,
2Ki 22:6  nämligen åt timmermännen, byggningsmännen och murarna, så ock till att inköpa trävirke och huggen sten för att sätta huset i stånd.
2Ki 22:7  Dock skall man icke hålla någon räkenskap med dem angående de penningar som överlämnas åt dem, utan de skola få handla på heder och tro."
2Ki 22:8  Och översteprästen Hilkia sade till sekreteraren Safan: "Jag har funnit lagboken i HERRENS hus." Och Hilkia gav boken åt Safan, och han läste den.
2Ki 22:9  Därefter gick sekreteraren Safan in till konungen och avgav sin berättelse inför konungen; han sade: "Dina tjänare hava tömt ut de penningar som funnos i templet, och hava överlämnat dem åt de män som förrätta arbete såsom tillsyningsmän vid HERRENS hus."
2Ki 22:10  Vidare berättade sekreteraren Safan för konungen och sade: "Prästen Hilkia har givit mig en bok." Och Safan föreläste den för konungen.
2Ki 22:11  När konungen nu hörde lagbokens ord, rev han sönder sina kläder.
2Ki 22:12  Och konungen bjöd prästen Hilkia och Ahikam, Safans son, och Akbor, Mikajas son, och sekreteraren Safan och Asaja, konungens tjänare, och sade:
2Ki 22:13  "Gån och frågen HERREN för mig och för folket, ja, för hela Juda, angående det som står i denna bok som nu har blivit funnen. Ty stor är HERRENS vrede, den som är upptänd mot oss, därför att våra fäder icke hava velat lyssna till denna boks ord och icke hava gjort allt som är oss föreskrivet.
2Ki 22:14  Då gingo prästen Hilkia och Ahikam, Akbor, Safan och Asaja till profetissan Hulda, hustru åt Sallum, klädkammarvaktaren, som var son till Tikva, Harhas' son; hon bodde i Jerusalem, i Nya staden. Och de talade med henne.
2Ki 22:15  Då sade hon till dem: "Så säger HERREN, Israels Gud: Sägen till den man som har sänt eder till mig:
2Ki 22:16  Så säger HERREN: Se, över denna plats och över dess invånare skall jag låta olycka komma, allt vad som står i den bok som Juda konung har läst -
2Ki 22:17  detta därför att de hava övergivit mig och tänt offereld åt andra gudar, och så hava förtörnat mig med alla sina händers verk. Min vrede skall upptändas mot denna plats och skall icke bliva utsläckt.
2Ki 22:18  Men till Juda konung, som har sänt eder för att fråga HERREN, till honom skolen I säga så: Så säger HERRES, Israels Gud,
2Ki 22:19  angående de ord som du har hört: Eftersom ditt hjärta blev bevekt och du ödmjukade dig inför HERREN, när du hörde vad jag har talat mot denna plats och mot dess invånare, nämligen att de skola bliva ett föremål för häpnad och ett exempel som man nämner, när man förbannar, och eftersom du rev sönder dina kläder och grät inför mig, fördenskull har jag ock hört dig, säger HERREN.
2Ki 22:20  Därför vill jag samla dig till dina fäder, så att du får samlas till dem i din grav med frid; och dina ögon skola slippa att se all den olycka som jag skall låta komma över denna plats." Och de vände till baka till konungen med detta svar.
2Ki 23:1  Då sände konungen åstad män som församlade till honom alla de äldste i Juda och Jerusalem.
2Ki 23:2  Och konungen gick upp i HERRENS hus, och alla Juda män och alla Jerusalems invånare följde honom, också prästerna och profeterna, ja, allt folket, ifrån den minste till den störste. Och han läste upp för dem allt vad som stod i förbundsboken, som hade blivit funnen i HERRENS hus
2Ki 23:3  Och konungen trädde fram till pelaren och slöt inför HERRENS ansikte det förbundet, att de skulle följa efter HERREN och hålla hans bud, hans vittnesbörd och hans stadgar, av allt hjärta och av all själ, och upprätthålla detta förbunds ord, dem som voro skrivna i denna bok. Och allt folket trädde in i förbundet.
2Ki 23:4  Därefter bjöd konungen översteprästen Hilkia och prästerna näst under honom, så ock dem som höllo vakt vid tröskeln, att de skulle föra bort ur HERRENS tempel alla de redskap som voro gjorda åt Baal och Aseran och åt himmelens hela härskara. Och han lät bränna upp dem utanför Jerusalem på Kidrons fält, men askan efter dem lät han föra till Betel.
2Ki 23:5  Han avsatte ock de avgudapräster som Juda konungar hade tillsatt, för att tända offereld på offerhöjderna i Juda städer och runt omkring Jerusalem, så ock dem som tände offereld åt Baal, åt solen, åt månen, åt stjärnbilderna och åt himmelens hela härskara.
2Ki 23:6  Och han tog Aseran ur HERRENS hus och förde bort den utanför Jerusalem till Kidrons dal, och brände upp den där i Kidrons dal, han stötte sönder den till stoft och kastade stoftet på den allmänna begravningsplatsen.
2Ki 23:7  Vidare rev han ned tempelbolarhusen som funnos i HERRENS hus, dem i vilka kvinnor vävde tyg till tält åt Aseran.
2Ki 23:8  Och han lät föra alla prästerna bort ifrån Juda städer och orenade de offerhöjder där prästerna hade tänt offereld, från Geba ända till Beer-Seba; och han bröt ned offerhöjderna vid stadsportarna, både den som låg vid ingången till stadshövitsmannen Josuas port och den som låg till vänster, när man gick in genom stadsporten.
2Ki 23:9  Dock fingo offerhöjdsprästerna icke stiga upp till HERRENS altare i Jerusalem; de fingo allenast äta osyrat bröd ibland sina bröder.
2Ki 23:10  Han orenade ock Tofet i Hinnoms barns dal, för att ingen skulle låta sin son eller dotter gå genom eld, till offer åt Molok.
2Ki 23:11  Och han skaffade bort de hästar som Juda konungar hade invigt åt solen och ställt upp så, att man icke kunde gå in i HERRENS hus, vid hovmannen Netan-Meleks kammare i Parvarim; och solens vagnar brände han upp i eld.
2Ki 23:12  Och altarna på taket över Ahas' sal, vilka Juda konungar hade låtit göra, och de altaren som Manasse hade låtit göra på de båda förgårdarna till HERRENS hus, dem bröt konungen ned; sedan skyndade han bort därifrån och kastade stoftet av dem i Kidrons dal.
2Ki 23:13  Och offerhöjderna öster om Jerusalem och söder om Fördärvets berg, vilka Salomo, Israels konung, hade byggt åt Astarte, sidoniernas styggelse, åt Kemos, Moabs styggelse, och åt Milkom, Ammons barns skändlighet, dem orenade konungen.
2Ki 23:14  Han slog sönder stoderna och högg ned Aserorna; och platsen där de hade stått fyllde han med människoben.
2Ki 23:15  Också altaret i Betel, den offerhöjd som Jerobeam, Nebats son, hade byggt upp, han som kom Israel att synda, också detta altare med offerhöjden bröt han ned; därefter brände han upp offerhöjden och stötte sönder den till stoft och brände tillika upp Aseran.
2Ki 23:16  När då Josia såg sig om och fick se gravarna som voro där på berget, sände han åstad och lät hämta benen ur gravarna och brände upp dem på altaret och orenade det så - i enlighet med det HERRENS ord som hade blivit förkunnat av gudsmannen som förkunnade att detta skulle ske.
2Ki 23:17  Och han frågade: "Vad är det för en vård som jag ser där?" Folket i staden svarade honom: "Det är den gudsmans grav, som kom från Juda och ropade mot altaret i Betel att det skulle ske, som du nu har gjort."
2Ki 23:18  Då sade han: "Låten honom vara; ingen må röra hans ben." Så lämnade man då hans ben i fred, och tillika benen av den profet som hade kommit dit från Samarien.
2Ki 23:19  Därjämte skaffade Josia bort alla de offerhöjdshus i Samariens städer, som Israels konungar hade byggt upp, och med vilka de hade kommit förtörnelse åstad; och han gjorde med dem alldeles på samma sätt som han hade gjort i Betel.
2Ki 23:20  Och alla offerhöjdspräster som funnos där slaktade han på altarna och brände människoben ovanpå dem. Därefter vände han tillbaka till Jerusalem.
2Ki 23:21  Och konungen bjöd allt folket och sade: "Hållen HERRENS, eder Guds, påskhögtid, såsom det är föreskrivet i denna förbundsbok."
2Ki 23:22  Ty en sådan påskhögtid hade icke blivit hållen sedan den tid då domarna dömde Israel, icke under Israels konungars och Juda konungars hela tid.
2Ki 23:23  Först i konung Josias adertonde regeringsår hölls en sådan HERRENS påskhögtid i Jerusalem.
2Ki 23:24  Därjämte skaffade Josia bort andebesvärjarna och spåmännen, husgudarna och de eländiga avgudarna, och alla styggelser som voro att se i Juda land och i Jerusalem, på det att han skulle upprätthålla lagens ord, dem som voro skrivna i den bok som prästen Hilkia hade funnit i HERRENS hus.
2Ki 23:25  Ingen konung lik honom hade funnits före honom, ingen som så av allt sitt hjärta och av all sin själ och av all sin kraft hade vänt sig till HERREN, i enlighet med Moses hela lag; och efter honom uppstod ej heller någon som var honom lik.
2Ki 23:26  Dock vände HERREN sig icke ifrån sin stora vredes glöd, då nu hans vrede hade blivit upptänd mot Juda, för allt det varmed Manasse hade förtörnat honom.
2Ki 23:27  Och HERREN sade: "Också Juda vill jag förskjuta ifrån mitt ansikte, likasom jag har förskjutit Israel; ja, jag vill förkasta Jerusalem, denna stad som jag hade utvalt, så ock det hus varom jag hade sagt: Mitt namn skall vara där."
2Ki 23:28  Vad nu mer är att säga om Josia och om allt vad han gjorde, det finnes upptecknat i Juda konungars krönika.
2Ki 23:29  I hans tid drog Farao Neko, konungen i Egypten, upp mot konungen i Assyrien, till floden Frat. Då tågade konung Josia emot honom, men blev dödad av honom vid Megiddo, under första sammandrabbningen.
2Ki 23:30  Och hans tjänare förde hans döda kropp i en vagn bort ifrån Megiddo till Jerusalem och begrovo honom i hans grav. Men folket i landet tog Josias son Joahas och smorde honom och gjorde honom till konung efter hans fader.
2Ki 23:31  Joahas var tjugutre år gammal, när han blev konung, och han regerade tre månader i Jerusalem. Hans moder hette Hamutal, Jeremias dotter, från Libna.
2Ki 23:32  Han gjorde vad ont var i HERRENS ögon, alldeles såsom hans fäder hade gjort.
2Ki 23:33  Och Farao Neko lät sätta honom i fängelse i Ribla i Hamats land och gjorde så slut på hans regering i Jerusalem; och han pålade landet en skatt av ett hundra talenter silver och en talent guld.
2Ki 23:34  Och Farao Neko gjorde Josias son Eljakim till konung i hans fader Josias ställe och förändrade hans namn till Jojakim. Men Joahas tog han med sig, och denne kom så till Egypten och dog där.
2Ki 23:35  Och Jojakim betalade ut silvret och guldet åt Farao; men han måste skattlägga landet för att kunna utbetala dessa penningar enligt Faraos befallning. Efter som var och en av folket i landet blev uppskattad, indrevs från dem silvret och guldet, för att sedan utbetalas åt Farao Neko.
2Ki 23:36  Jojakim var tjugufem år gammal, när han blev konung; och han regerade elva år i Jerusalem. Hans moder hette Sebida, Pedajas dotter, från Ruma.
2Ki 23:37  Han gjorde vad ont var i HERRENS ögon, alldeles såsom hans fäder hade gjort.
2Ki 24:1  I hans tid drog Nebukadnessar, konungen i Babel, upp, och Jojakim blev honom underdånig och förblev så i tre år; men sedan avföll han från honom.
2Ki 24:2  Då sände HERREN över honom kaldéernas; araméernas, moabiternas och Ammons barns härskaror; han sände dem över Juda för att förgöra det - i enlighet med det ord som HERREN hade talat genom sina tjänare profeterna.
2Ki 24:3  Ja, det var efter HERRENS ord som detta kom över Juda, i det att han försköt det från sitt ansikte för de synder Manasse hade begått och för allt vad denne hade förövat,
2Ki 24:4  jämväl för det oskyldiga blod som han utgöt, ty han uppfyllde Jerusalem med oskyldigt blod; det ville HERREN icke förlåta.
2Ki 24:5  Vad nu mer är att säga om Jojakim och om allt vad han gjorde, det finnes upptecknat i Juda konungars krönika.
2Ki 24:6  Och Jojakim gick till vila hos sina fäder. Och hans son Jojakin blev konung efter honom.
2Ki 24:7  Därefter drog konungen i Egypten icke vidare ut ur sitt land, ty konungen i Babel hade intagit allt som tillhörde konungen i Egypten, från Egyptens bäck ända till floden Frat."
2Ki 24:8  Jojakin var aderton år gammal, när han blev konung, och han regerade tre månader i Jerusalem. Hans moder hette Nehusta, Elnatans dotter, från Jerusalem.
2Ki 24:9  Han gjorde vad ont var i HERRENS ögon, alldeles såsom hans fader hade gjort.
2Ki 24:10  På den tiden drogo den babyloniske konungen Nebukadnessars tjänare upp till Jerusalem, och staden blev belägrad.
2Ki 24:11  Och Nebukadnessar, konungen i Babel, kom till staden, medan hans tjänare belägrade den.
2Ki 24:12  Då gav sig Jojakin, Juda konung, åt konungen i Babel, med sin moder och med sina tjänare, sina hövitsmän och hovmän; och konungen i Babel tog honom så till fånga i sitt åttonde regeringsår.
2Ki 24:13  Och han förde bort därifrån alla skatter i HERRENS hus och skatterna i konungshuset, han lösbröt ock beläggningen från alla gyllene föremål som Salomo, Israels konung, hade låtit göra för HERRENS tempel, detta i enlighet med vad HERREN hade hotat.
2Ki 24:14  Och han förde bort i fångenskap hela Jerusalem, alla hövitsmän och alla tappra stridsmän; tio tusen förde han bort, jämväl alla timmermän och smeder. Inga andra lämnades kvar än de ringaste av folket i landet.
2Ki 24:15  Han förde Jojakin bort till Babel; därjämte förde han konungens moder, konungens hustrur och hans hovmän samt de mäktige i landet såsom fångar bort ifrån Jerusalem till Babel,
2Ki 24:16  så ock alla stridsmännen, sju tusen, och timmermännen och smederna, ett tusen, allasammans raska och krigsdugliga män. Dessa fördes nu av den babyloniske konungen i fångenskap till Babel.
2Ki 24:17  Men konungen i Babel gjorde hans farbroder Mattanja till konung i hans ställe och förändrade dennes namn till Sidkia.
2Ki 24:18  Sidkia var tjuguett år gammal, när han blev konung, och han regerade elva år i Jerusalem. Hans moder hette Hamital, Jeremias dotter, från Libna.
2Ki 24:19  Han gjorde vad ont var i HERRENS ögon, alldeles såsom Jojakim hade gjort.
2Ki 24:20  Ty på grund av HERRENS vrede skedde vad som skedde med Jerusalem och Juda, till dess att han kastade dem bort ifrån sitt ansikte. Och Sidkia avföll från konungen i Babel.
2Ki 25:1  Då, i hans nionde regeringsår, i tionde månaden; på tionde dagen i månaden, kom Nebukadnessar, konungen i Babel, med hela sin här till Jerusalem och belägrade det; och de byggde en belägringsmur runt omkring det.
2Ki 25:2  Så blev staden belägrad och förblev så ända till konung Sidkias elfte regeringsår.
2Ki 25:3  Men på nionde dagen i månaden var hungersnöden så stor i staden, att mängden av folket icke hade något att äta.
2Ki 25:4  Och staden stormades, och allt krigsfolket flydde om natten genom porten mellan de båda murarna (den port som ledde till den kungliga trädgården), medan kaldéerna lågo runt omkring staden; och folket tog vägen åt Hedmarken till.
2Ki 25:5  Men kaldéernas här förföljde konungen, och de hunno upp honom på Jerikos hedmarker, sedan hela hans här hade övergivit honom och skingrat sig.
2Ki 25:6  Och de grepo konungen och förde honom till den babyloniske konungen i Ribla; där höll man rannsakning och dom med honom.
2Ki 25:7  Och Sidkias barn slaktade man inför hans ögon, och på Sidkia själv stack man ut ögonen; och man fängslade honom med kopparfjättrar och förde honom till Babel.
2Ki 25:8  I femte månaden, på sjunde dagen i månaden, detta i den babyloniske konungen Nebukadnessars nittonde regeringsår, kom den babyloniske konungens tjänare Nebusaradan, översten för drabanterna, till Jerusalem.
2Ki 25:9  Denne brände upp HERRENS hus och konungshuset: ja, alla hus i Jerusalem, i synnerhet alla de förnämas hus, brände han upp i eld.
2Ki 25:10  Och murarna runt omkring Jerusalem brötos ned av hela den här av kaldéer, som översten för drabanterna hade med sig,
2Ki 25:11  och återstoden av folket, dem som voro kvar i staden, och de överlöpare som hade gått över till konungen i Babel, så ock den övriga hopen, dem förde Nebusaradan, översten för drabanterna, bort i fångenskap.
2Ki 25:12  Men av de ringaste i landet lämnade översten för drabanterna några kvar till vingårdsmän och åkermän.
2Ki 25:13  Kopparpelarna i HERRENS hus, bäckenställen och kopparhavet i HERRENS hus slogo kaldéerna sönder och förde kopparen till Babel.
2Ki 25:14  Och askkärlen, skovlarna, knivarna, skålarna och alla kopparkärl som hade begagnats vid gudstjänsten togo de bort.
2Ki 25:15  Likaledes tog översten för drabanterna bort fyrfaten och offerskålarna, allt som var av rent guld eller av rent silver.
2Ki 25:16  Vad angår de två pelarna, havet, som var allenast ett, och bäckenställen, som Salomo hade låtit göra till HERRENS hus, så kunde kopparen i alla dessa föremål icke vägas.
2Ki 25:17  Aderton alnar hög var den ena pelaren, och ovanpå den var ett pelarhuvud av koppar, och pelarhuvudet var tre alnar högt, och ett nätverk och granatäpplen funnos på pelarhuvudet runt omkring, alltsammans av koppar; och likadant var det på den andra pelaren, över nätverket.
2Ki 25:18  Och översten för drabanterna tog översteprästen Seraja jämte Sefanja, prästen näst under honom, så ock de tre som höllo vakt vid tröskeln,
2Ki 25:19  och från staden tog han en hovman, den som var anförare för krigsfolket, och fem av konungens närmaste män, som påträffades i staden, så ock sekreteraren, den härhövitsman som plägade utskriva folket i landet till krigstjänst, och sextio andra män av landets folk, som påträffades i staden -
2Ki 25:20  dessa tog Nebusaradan, översten för drabanterna, och förde dem till den babyloniske konungen i Ribla.
2Ki 25:21  Och konungen i Babel lät avliva dem där, i Ribla i Hamats land. blev Juda bortfört från sitt land.
2Ki 25:22  Men över det folk som blev kvar i Juda land, det folk som Nebukadnessar, konungen i Babel, lät bliva kvar där, satte han Gedalja, son till Ahikam, son till Safan.
2Ki 25:23  När då alla krigshövitsmännen jämte sina män fingo höra att konungen i Babel hade satt Gedalja över landet, kommo de till Gedalja i Mispa, nämligen Ismael, Netanjas son, Johanan, Kareas son, netofatiten Seraja, Tanhumets son, och Jaasanja, maakatitens son, med sina män.
2Ki 25:24  Och Gedalja gav dem och deras män sin ed och sade till dem: "Frukten icke för kaldéernas tjänare. Stannen kvar i landet, och tjänen konungen i Babel, så skall det gå eder väl."
2Ki 25:25  Men i sjunde månaden kom Ismael, son till Netanja, son till Elisama, av konungslig börd, och hade med sig tio män, och de slogo ihjäl Gedalja, så ock de judar och kaldéer som voro hos honom i Mispa.
2Ki 25:26  Då bröt allt folket upp, från den minste till den störste, tillika med krigshövitsmännen, och begav sig till Egypten; ty de fruktade för kaldéerna.
2Ki 25:27  Men i det trettiosjunde året sedan Jojakin, Juda konung, hade blivit bortförd i fångenskap, i tolfte månaden, på tjugusjunde dagen i månaden, tog Evil-Merodak, konungen i Babel - samma år han blev konung - Jojakin, Juda konung, till nåder och befriade honom ur fängelset;
2Ki 25:28  Och han talade vänligt med honom och gav honom främsta platsen bland de konungar som voro hos honom i Babel.
2Ki 25:29  Han fick lägga av sin fångdräkt och beständigt äta vid hans bord, så länge han levde.
2Ki 25:30  Och ett ständigt underhåll gavs honom från konungen, visst för var dag, så länge han levde.


\end{document}