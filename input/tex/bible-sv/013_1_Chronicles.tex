\begin{document}

\title{1 Krönikeboken}


\chapter{1}

\par 1 Adam, Set, Enos,
\par 2 Kenan, Mahalalel, Jered,
\par 3 Hanok, Metusela, Lemek,
\par 4 Noa, Sem, Ham och Jafet.
\par 5 Jafets söner voro Gomer, Magog, Madai, Javan, Tubal, Mesek och Tiras.
\par 6 Gomers söner voro Askenas, Difat och Togarma.
\par 7 Javans söner voro Elisa och Tarsisa, kittéerna och rodanéerna.
\par 8 Hams söner voro Kus, Misraim, Put och Kanaan.
\par 9 Kus' söner voro Seba, Havila, Sabta, Raema och Sabteka. Raemas söner voro Saba och Dedan.
\par 10 Men Kus födde Nimrod; han var den förste som upprättade ett välde på jorden.
\par 11 Och Misraim födde ludéerna, anaméerna, lehabéerna, naftuhéerna,
\par 12 patroséerna, kasluhéerna, från vilka filistéerna hava utgått, och kaftoréerna.
\par 13 Och Kanaan födde Sidon, som var hans förstfödde, och Het,
\par 14 så ock jebuséerna, amoréerna och girgaséerna,
\par 15 hivéerna, arkéerna, sinéerna,
\par 16 arvadéerna, semaréerna och hamatéerna.
\par 17 Sems söner voro Elam, Assur, Arpaksad, Lud och Aram, så ock Us, Hul, Geter och Mesek.
\par 18 Arpaksad födde Sela, och Sela födde Eber.
\par 19 Men åt Eber föddes två söner; den ene hette Peleg, ty i hans tid blev jorden fördelad; och hans broder hette Joktan.
\par 20 Och Joktan födde Almodad, Selef, Hasarmavet, Jera,
\par 21 Hadoram, Usal, Dikla,
\par 22 Ebal, Abimael, Saba,
\par 23 Ofir, Havila och Jobab; alla dessa voro Joktans söner.
\par 24 Sem, Arpaksad, Sela,
\par 25 Eber, Peleg, Regu,
\par 26 Serug, Nahor, Tera,
\par 27 Abram, det är Abraham
\par 28 Abrahams söner voro Isak och Ismael.
\par 29 Detta är deras släkttavla: Nebajot, Ismaels förstfödde, vidare Kedar, Adbeel och Mibsam,
\par 30 Misma och Duma, Massa, Hadad och Tema,
\par 31 Jetur, Nafis och Kedma. Dessa voro Ismaels söner.
\par 32 Och de söner som Ketura, Abrahams bihustru, födde voro Simran, Joksan, Medan, Midjan, Jisbak och Sua. Joksans söner voro Saba och Dedan.
\par 33 Och Midjans söner voro Efa, Efer, Hanok, Abida och Eldaa. Alla dessa voro Keturas söner.
\par 34 Och Abraham födde Isak. Isaks söner voro Esau och Israel.
\par 35 Esaus söner voro Elifas, Reguel, Jeus, Jaelam och Kora.
\par 36 Elifas' söner voro Teman och Omar, Sefi och Gaetam, Kenas, Timna och Amalek
\par 37 Reguels söner voro Nahat, Sera, Samma och Missa.
\par 38 Men Seirs söner voro Lotan, Sobal, Sibeon, Ana, Dison, Eser och Disan.
\par 39 Lotans söner voro Hori och Homam; och Lotans syster var Timna.
\par 40 Sobals söner voro Aljan, Manahat och Ebal, Sefi och Onam. Och Sibeons söner voro Aja och Ana.
\par 41 Anas söner voro Dison. Och Disons söner voro Hamran, Esban, Jitran och Keran.
\par 42 Esers söner voro Bilhan, Saavan, Jaakan. Disans söner voro Us och Aran.
\par 43 Och dessa voro de konungar som regerade i Edoms land, innan ännu någon israelitisk konung var konung där: Bela, Beors son, och hans stad hette Dinhaba.
\par 44 När Bela dog, blev Jobab, Seras son, från Bosra, konung efter honom.
\par 45 När Jobab dog, blev Husam från temanéernas land konung efter honom.
\par 46 När Husam dog, blev Hadad, Bedads son, konung efter honom, han som slog midjaniterna på Moabs mark; och hans stad hette Avit.
\par 47 När Hadad dog, blev Samla från Masreka konung efter honom.
\par 48 När Samla dog, blev Saul, från Rehobot vid floden, konung efter honom.
\par 49 När Saul dog, blev Baal-Hanan, Akbors son, konung efter honom
\par 50 När Baal-Hanan dog, blev Hadad konung efter honom; och hans stad hette Pagi, och hans hustru hette Mehetabel, dotter till Matred, var dotter till Me-Sahab.
\par 51 Men när Hadad hade dött, voro dessa Edoms stamfurstar: fursten Timna, fursten Alja, fursten Jetet,
\par 52 fursten Oholibama, fursten Ela, fursten Pinon,
\par 53 fursten Kenas, fursten Teman, fursten Mibsar,
\par 54 fursten Magdiel, fursten Iram. Dessa voro Edoms stamfurstar.

\chapter{2}

\par 1 Dessa voro Israels söner: Ruben, Simeon, Levi och Juda, Isaskar och Sebulon,
\par 2 Dan, Josef och Benjamin, Naftali, Gad och Aser.
\par 3 Judas söner voro Er, Onan och Sela; dessa tre föddes åt honom av Suas dotter, kananeiskan. Men Er, Judas förstfödde, misshagade HERREN; därför dödade han honom.
\par 4 Och Tamar, hans sonhustru, födde åt honom Peres och Sera, så att Judas söner voro tillsammans fem.
\par 5 Peres' söner voro Hesron och Hamul.
\par 6 Seras söner voro Simri, Etan, Heman, Kalkol och Dara, tillsammans fem.
\par 7 Men Karmis söner voro Akar, som drog olycka över Israel, när han trolöst förgrep sig på det tillspillogivna.
\par 8 Och Etans söner voro Asarja.
\par 9 Och de söner som föddes åt Hesron voro Jerameel, Ram och Kelubai.
\par 10 Och Ram födde Amminadab, och Amminadab födde Naheson, hövding för Juda barn.
\par 11 Naheson födde Salma, och Salma födde Boas.
\par 12 Boas födde Obed, och Obed födde Isai.
\par 13 Isai födde Eliab, som var hans förstfödde, Abinadab, den andre, och Simea, den tredje,
\par 14 Netanel, den fjärde, Raddai, den femte,
\par 15 Osem, den sjätte, David, den sjunde.
\par 16 Och deras systrar voro Seruja och Abigail. Och Serujas söner voro Absai, Joab och Asael, tillsammans tre.
\par 17 Och Abigail födde Amasa, och Amasas fader var ismaeliten Jeter.
\par 18 Och Kaleb, Hesrons son, födde ett barn av kvinnkön, Asuba, därtill ock Jeriot; och dessa voro henne söner: Jeser, Sobab och Ardon.
\par 19 Och när Asuba dog, tog Kaleb Efrat till hustru åt sig, och hon födde åt honom Hur.
\par 20 Och Hur födde Uri, och Uri födde Besalel.
\par 21 Därefter gick Hesron in till Makirs, Gileads faders, dotter; henne tog han till hustru, när han var sextio år gammal. Och hon födde åt honom Segub.
\par 22 Och Segub födde Jair; denne hade tjugutre städer i Gileads land.
\par 23 Men gesuréerna och araméerna togo ifrån dem Jairs byar jämte Kenat med underlydande orter, sextio städer. Alla dessa voro söner till Makir, Gileads fader.
\par 24 Och sedan Hesron hade dött i Kaleb-Efrata, födde Hesrons hustru Abia åt honom Ashur, Tekoas fader.
\par 25 Och Jerameels, Hesrons förstföddes, söner voro Ram, den förstfödde, vidare Buna, Oren och Osem samt Ahia.
\par 26 Men Jerameel hade en annan hustru som hette Atara; hon var moder till Onam.
\par 27 Och Rams, Jerameels förstföddes, söner voro Maas, Jamin och Eker.
\par 28 Onams söner voro Sammai och Jada; och Sammais söner voro Nadab och Abisur.
\par 29 Och Abisurs hustru hette Abihail; hon födde åt honom Aban och Molid.
\par 30 Nadabs söner voro Seled och Appaim. Seled dog barnlös.
\par 31 Men Appaims söner voro Jisei; Jiseis söner voro Sesan; Sesans söner voro Alai.
\par 32 Jadas, Sammais broders, söner voro Jeter och Jonatan. Jeter dog barnlös.
\par 33 Men Jonatans söner voro Pelet och Sasa. Dessa voro Jerameels söner.
\par 34 Men Sesan hade inga söner, utan allenast döttrar. Nu hade Sesan en egyptisk tjänare som hette Jarha.
\par 35 Och Sesan gav sin dotter till hustru åt sin tjänare Jarha, och hon födde åt honom Attai.
\par 36 Attai födde Natan, och Natan födde Sabad.
\par 37 Sabad födde Eflal, och Eflal födde Obed.
\par 38 Obed födde Jehu, och Jehu födde Asarja.
\par 39 Asarja födde Heles, och Heles födde Eleasa.
\par 40 Eleasa födde Sisamai, och Sisamai födde Sallum.
\par 41 Sallum födde Jekamja, och Jekamja födde Elisama.
\par 42 Och Kalebs, Jerameels broders, söner voro Mesa, hans förstfödde, som var Sifs fader, och Maresas, Hebrons faders, söner.
\par 43 Men Hebrons söner voro Kora, Tappua, Rekem och Sema.
\par 44 Sema födde Raham, Jorkeams fader. Men Rekem födde Sammai.
\par 45 Sammais son var Maon, och Maon var Bet-Surs fader.
\par 46 Och Efa, Kalebs bihustru, födde Haran, Mosa och Gases; och Haran födde Gases.
\par 47 Och Jadais söner voro Regem, Jotam, Gesan, Pelet, Efa och Saaf.
\par 48 Kalebs bihustru Maaka födde Seber och Tirhana.
\par 49 Hon födde ock Saaf, Madmannas fader, Seva, Makbenas fader och Gibeas fader. Och Kalebs dotter var Aksa.
\par 50 Dessa voro Kalebs söner: Hurs, Efratas förstföddes, son var Sobal Kirjat-Jearims fader,
\par 51 vidare Salma, Bet-Lehems fader, och Haref, Bet-Gaders fader.
\par 52 Söner till Sobal, Kirjat-Jearims fader, voro Haroe och hälften av Hammenuhot-släkten.
\par 53 Men Kirjat-Jearims släkter voro jeteriterna, putiterna, sumatiterna och misraiterna. Från dem utgingo sorgatiterna och estaoliterna.
\par 54 Salmas söner voro Bet-Lehem och netofatiterna, Atrot-Bet-Joab, så ock hälften av manahatiterna, sorgiterna.
\par 55 Och de skriftlärdes släkter, deras som bodde i Jaebes, voro tireatiterna, simeatiterna, sukatiterna. Dessa voro de kainéer som härstammade från Hammat, fader till Rekabs släkt.

\chapter{3}

\par 1 Dessa voro de söner som föddes åt David i Hebron: Amnon, den förstfödde, av Ahinoam från Jisreel; Daniel, den andre, av Abigail från Karmel;
\par 2 Absalom, den tredje, son till Maaka, som var dotter till Talmai, konungen i Gesur; Adonia, den fjärde, Haggits son;
\par 3 Sefatja, den femte, av Abital; Jitream, den sjätte, av hans hustru Egla.
\par 4 Dessa sex föddes åt honom i Hebron, där han regerade i sju år och sex månader. I Jerusalem åter regerade han i trettiotre år.
\par 5 Och dessa söner föddes åt honom i Jerusalem: Simea, Sobab, Natan och Salomo, tillsammans fyra, av Bat-Sua, Ammiels dotter;
\par 6 vidare Jibhar, Elisama, Elifelet,
\par 7 Noga, Nefeg, Jafia,
\par 8 Elisama, Eljada och Elifelet, tillsammans nio.
\par 9 Detta var alla Davids söner, förutom sönerna med bihustrurna; och Tamar var deras syster.
\par 10 Salomos son var Rehabeam. Hans son var Abia; hans son var Asa; hans son var Josafat.
\par 11 Hans son var Joram; hans son var Ahasja; hans son var Joas.
\par 12 Hans son var Amasja; hans son var Asarja; hans son var Jotam.
\par 13 Hans son var Ahas; hans son var Hiskia; hans son var Manasse.
\par 14 Hans son var Amon; hans son var Josia.
\par 15 Josias söner voro Johanan den förstfödde, Jojakim, den andre, Sidkia, den tredje, Sallum, den fjärde.
\par 16 Jojakims söner voro hans son Jekonja och dennes son Sidkia.
\par 17 Jekonjas söner voro Assir och dennes son Sealtiel,
\par 18 vidare Malkiram, Pedaja, Senassar, Jekamja, Hosama och Nedabja.
\par 19 Pedajas söner voro Serubbabel och Simei. Serubbabels söner voro Mesullam och Hananja, och deras syster var Selomit,
\par 20 vidare Hasuba, Ohel, Berekja, Hasadja och Jusab-Hesed, tillsammans fem.
\par 21 Hananjas söner voro Pelatja och Jesaja, vidare Refajas söner, Arnans söner, Obadjas söner och Sekanjas söner.
\par 22 Sekanjas söner voro Semaja, Semajas söner voro Hattus, Jigeal, Baria, Nearja och Safat, tillsammans sex.
\par 23 Nearjas söner voro Eljoenai, Hiskia och Asrikam, tillsammans tre.
\par 24 Eljoenais söner voro Hodauja, Eljasib, Pelaja, Ackub, Johanan, Delaja och Anani, tillsammans sju.

\chapter{4}

\par 1 Judas söner voro Peres, Hesron, Karmi, Hur och Sobal.
\par 2 Och Reaja, Sobals son, födde Jahat, och Jahat födde Ahumai och Lahad. Dessa voro sorgatiternas släkter.
\par 3 Och dessa voro Abi-Etams söner: Jisreel, Jisma och Jidbas, och deras syster hette Hasselelponi,
\par 4 vidare Penuel, Gedors fader, och Eser, Husas fader. Dessa voro söner till Hur, Efratas förstfödde, Bet-Lehems fader.
\par 5 Och Ashur, Tekoas fader, hade två hustrur, Helea och Naara.
\par 6 Naara födde åt honom Ahussam, Hefer, Timeni och ahastariterna. Dessa voro Naaras söner.
\par 7 Och Heleas söner voro Seret, Jishar och Etnan.
\par 8 Och Kos födde Anub och Hassobeba, så ock Aharhels, Harums sons, släkter.
\par 9 Men Jaebes var mer ansedd än sina bröder; hans moder hade givit honom namnet Jaebes, i det hon sade: "Jag har fött honom med smärta."
\par 10 Och Jaebes åkallade Israels Gud och sade: "O att du ville välsigna mig och utvidga mitt område och låta din hand vara med mig! O att du ville avvända vad ont är, så att jag sluppe att känna någon smärta!" Och Gud lät det ske, som han begärde.
\par 11 Och Kelub, Suhas broder, födde Mehir; han var Estons fader.
\par 12 Och Eston födde Bet-Rafa, Pasea och Tehinna, Ir-Nahas' fader. Dessa voro männen från Reka.
\par 13 Och Kenas söner voro Otniel och Seraja. Otniels söner voro Hatat.
\par 14 Och Meonotai födde Ofra. Och Seraja födde Joab, fader till Timmermansdalens släkt, ty dessa voro timmermän.
\par 15 Och Kalebs, Jefunnes sons, söner voro Iru, Ela och Naam, så ock Elas söner och Kenas.
\par 16 Och Jehallelels söner voro Sif och Sifa, Tirja och Asarel.
\par 17 Och Esras son var Jeter, vidare Mered, Efer och Jalon. Och kvinnan blev havande och födde Mirjam, Sammai och Jisba, Estemoas fader.
\par 18 Och hans judiska hustru födde Jered, Gedors fader, och Heber, Sokos fader, och Jekutiel, Sanoas fader. Men de andra voro söner till Bitja, Faraos dotter, som Mered hade tagit till hustru.
\par 19 Och söner till Hodias hustru, Nahams syster, voro Kegilas fader, garmiten, och maakatiten Estemoa.
\par 20 Och Simons söner voro Amnon och Rinna, Ben-Hanan och Tilon. Och Jiseis söner voro Sohet och Sohets son.
\par 21 Söner till Sela, Judas son, voro Er, Lekas fader, och Laeda, Maresas fader, och de släkter som tillhörde linnearbetarnas hus, av Asbeas hus,
\par 22 vidare Jokim och männen i Koseba samt Joas och Saraf, som blevo herrar över Moab, så ock Jasubi-Lehem. Men detta tillhör en avlägsen tid.
\par 23 Dessa voro krukmakarna och invånarna i Netaim och Gedera; de bodde där hos konungen och voro i hans arbete.
\par 24 Simeons söner voro Nemuel och Jamin, Jarib, Sera och Saul.
\par 25 Hans son var Sallum; hans son var Mibsam; hans son var Misma.
\par 26 Mismas söner voro hans son Hammuel, dennes son Sackur och dennes son Simei.
\par 27 Och Simei hade sexton söner och sex döttrar; men hans bröder hade icke många barn. Och deras släkt i sin helhet förökade sig icke så mycket som Juda barn.
\par 28 Och de bodde i Beer-Seba, Molada och Hasar-Sual,
\par 29 i Bilha, i Esem och i Tolad,
\par 30 i Betuel, i Horma och i Siklag,
\par 31 i Bet-Markabot, i Hasar-Susim, i Bet-Birei och i Saaraim. Dessa voro deras städer, till dess att David blev konung.
\par 32 Och deras byar voro Etam och Ain, Rimmon, Token och Asan - fem städer;
\par 33 därtill alla deras byar, som lågo runt omkring dessa städer, ända till Baal. Dessa voro deras boningsorter; och de hade sitt särskilda släktregister.
\par 34 Vidare: Mesobab, Jamlek och Josa, Amasjas son,
\par 35 och Joel och Jehu, son till Josibja, son till Seraja, son till Asiel,
\par 36 och Eljoenai, Jaakoba, Jesohaja, Asaja, Adiel, Jesimiel och Benaja,
\par 37 så ock Sisa, son till Sifei, son till Allon, son till Jedaja, son till Simri son till Semaja.
\par 38 Dessa nu nämnda voro hövdingar i sina släkter, och deras familjer utbredde sig och blevo talrika.
\par 39 Och de drogo fram mot Gedor, ända till östra sidan av dalen, för att söka bete för sin boskap.
\par 40 Och de funno fett och gott bete, och landet hade utrymme nog, och där var stilla och lugnt, ty de som förut bodde där voro hamiter.
\par 41 Men dessa som här hava blivit upptecknade vid namn kommo i Hiskias, Juda konungs, tid och förstörde deras tält och slogo de meiniter som funnos där och gåvo dem till spillo, så att de nu icke mer äro till, och bosatte sig i deras land; ty där fanns bete för deras boskap.
\par 42 Och av dem, av Simeons barn, drogo fem hundra man till Seirs bergsbygd; och Pelatja, Nearja, Refaja och Ussiel, Jiseis söner, stodo i spetsen för dem.
\par 43 Och de slogo den sista kvarlevan av amalekiterna; sedan bosatte de sig där och bo där ännu i dag.

\chapter{5}

\par 1 Och Rubens söner, Israels förstföddes - han var nämligen den förstfödde, men därför att han oskärade sin faders bädd, blev hans förstfödslorätt given åt Josefs, Israels sons, söner, dock icke så, att denne skulle upptagas i släktregistret såsom den förstfödde;
\par 2 ty väl var Juda den mäktigaste bland sina bröder, och furste blev en av hans avkomlingar, men förstfödslorätten blev dock Josefs -
\par 3 Rubens, Israels förstföddes, söner voro Hanok och Pallu, Hesron och Karmi.
\par 4 Joels söner voro hans son Semaja, dennes son Gog, dennes son Simei,
\par 5 dennes son Mika, dennes son Reaja, dennes son Baal,
\par 6 så ock dennes son Beera, som Tillegat-Pilneeser, konungen i Assyrien, förde bort i fångenskap; han var hövding för rubeniterna.
\par 7 Och hans bröder voro, efter sina släkter, när de upptecknades i släktregistret efter sin ättföljd: Jegiel, huvudmannen, Sakarja
\par 8 och Bela, son till Asas, son till Sema, son till Joel; han bodde i Aroer, och hans boningsplatser nådde ända till Nebo och Baal-Meon.
\par 9 Och österut nådde hans boningsplatser ända fram till öknen som sträcker sig ifrån floden Frat; ty de hade stora boskapshjordar i Gileads land.
\par 10 Men i Sauls tid förde de krig mot hagariterna, och dessa föllo för deras hand; då bosatte de sig i deras hyddor utefter hela östra sidan av Gilead.
\par 11 Och Gads barn hade sina boningsplatser gent emot dem i landet Basan ända till Salka:
\par 12 Joel, huvudmannen, och Safam därnäst, och vidare Jaanai och Safat i Basan.
\par 13 Och deras bröder voro, efter sina familjer, Mikael, Mesullam, Seba, Jorai, Jaekan, Sia och Eber, tillsammans sju.
\par 14 Dessa voro söner till Abihail, son till Huri, son till Jaroa, son till Gilead, son till Mikael, son till Jesisai, son till Jado, son till Bus.
\par 15 Men Ahi, son till Abdiel, son till Guni, var huvudman för deras familjer.
\par 16 Och de bodde i Gilead i Basan och underlydande orter, så ock på alla Sarons utmarker, så långt de sträckte sig.
\par 17 Alla dessa blevo upptecknade i släktregistret i Jotams, Juda konungs, och i Jerobeams, Israels konungs, tid.
\par 18 Rubens barn och gaditerna och ena hälften av Manasse stam, de av dem, som voro krigsdugliga och buro sköld och svärd och spände båge och voro stridskunniga, utgjorde fyrtiofyra tusen sju hundra sextio stridbara män.
\par 19 Och de förde krig mot hagariterna och mot Jetur, Nafis och Nodab.
\par 20 Och seger beskärdes dem i striden mot dessa, så att hagariterna och alla som voro med dem blevo givna i deras hand; ty de ropade till Gud under striden, och han bönhörde dem, därför att de förtröstade på honom.
\par 21 Och såsom byte förde de bort deras boskapshjordar, femtio tusen kameler, två hundra femtio tusen får och två tusen åsnor, så ock ett hundra tusen människor.
\par 22 Ty många hade fallit slagna, eftersom striden var av Gud. Sedan bosatte de sig i deras land och bodde där ända till fångenskapen.
\par 23 Halva Manasse stams barn bodde ock där i landet, från Basan ända till Baal-Hermon och Senir och Hermons berg, och de voro talrika.
\par 24 Och dessa voro huvudmän för sina familjer: Efer, Jisei, Eliel, Asriel, Jeremia, Hodauja och Jadiel, tappra stridsmän, namnkunniga män, huvudmän för sina familjer.
\par 25 Men de blevo otrogna mot sina fäders Gud, i det att de i trolös avfällighet lupo efter de gudar som dyrkades av de folk där i landet, som Gud hade förgjort för dem.
\par 26 Då uppväckte Israels Gud den assyriske konungen Puls ande och den assyriske konungen Tillegat-Pilnesers ande och lät folket föras bort i fångenskap, såväl rubeniterna och gaditerna som ena hälften av Manasse stam, och lät dem komma till Hala, Habor, Hara och Gosans ström, där de äro ännu i dag.

\chapter{6}

\par 1 Levis söner voro Gerson, Kehat och Merari.
\par 2 Kehats söner voro Amram, Jishar, Hebron och Ussiel.
\par 3 Amrams barn voro Aron, Mose och Mirjam. Arons söner voro Nadab och Abihu, Eleasar och Itamar.
\par 4 Eleasar födde Pinehas, Pinehas födde Abisua.
\par 5 Abisua födde Bucki, och Bucki födde Ussi.
\par 6 Ussi födde Seraja, och Seraja födde Merajot.
\par 7 Merajot födde Amarja, och Amarja födde Ahitub.
\par 8 Ahitub födde Sadok, och Sadok födde Ahimaas.
\par 9 Ahimaas födde Asarja, och Asarja födde Johanan.
\par 10 Johanan födde Asarja; det var han som var präst i det tempel som Salomo byggde i Jerusalem.
\par 11 Asarja födde Amarja, och Amarja födde Ahitub.
\par 12 Ahitub födde Sadok, och Sadok födde Sallum.
\par 13 Sallum födde Hilkia, och Hilkia födde Asarja.
\par 14 Asarja födde Seraja, och Seraja födde Josadak.
\par 15 Men Josadak måste gå med i fångenskap, när HERREN lät Juda och Jerusalem föras bort genom Nebukadnessar.
\par 16 Levis söner voro Gersom, Kehat och Merari.
\par 17 Och dessa voro namnen på Gersoms söner: Libni och Simei.
\par 18 Och Kehats söner voro Amram, Jishar, Hebron och Ussiel.
\par 19 Meraris söner voro Maheli och Musi. Dessa voro leviternas släkter, efter deras fäder.
\par 20 Från Gersom härstammade hans son Libni, dennes son Jahat, dennes son Simma,
\par 21 dennes son Joa, dennes son Iddo, dennes son Sera, dennes son Jeaterai.
\par 22 Kehats söner voro hans son Amminadab, dennes son Kora, dennes son Assir,
\par 23 dennes son Elkana, dennes son Ebjasaf, dennes son Assir,
\par 24 dennes son Tahat, dennes son Uriel, dennes son Ussia och dennes son Saul.
\par 25 Elkanas söner voro Amasai och Ahimot.
\par 26 Hans son var Elkana; hans son var Elkana-Sofai; hans son var Nahat.
\par 27 Hans son var Eliab; hans son var Jeroham; hans son var Elkana.
\par 28 Och Samuels söner voro Vasni, den förstfödde, och Abia.
\par 29 Meraris söner voro Maheli, dennes son Libni, dennes son Simei, dennes son Ussa,
\par 30 dennes son Simea, dennes son Haggia, dennes son Asaja.
\par 31 Och dessa voro de som David anställde för att ombesörja sången i HERRENS hus, sedan arken hade fått en vilostad.
\par 32 De gjorde tjänst inför uppenbarelsetältets tabernakel såsom sångare, till dess att Salomo byggde HERRENS hus i Jerusalem; de stodo där och förrättade sin tjänst, såsom det var föreskrivet för dem.
\par 33 Dessa voro de som så tjänstgjorde, och dessa voro deras söner: Av kehatiternas barn: Heman, sångaren, son till Joel, son till Samuel,
\par 34 son till Elkana, son till Jeroham, son till Eliel, son till Toa,
\par 35 son till Sif, son till Elkana, son till Mahat, son till Amasai,
\par 36 son till Elkana, son till Joel, son till Asarja, son till Sefanja,
\par 37 son till Tahat, son till Assir, son till Ebjasaf, son till Kora,
\par 38 son till Jishar, son till Kehat, son till Levi, son till Israel;
\par 39 vidare hans broder Asaf, som hade sin plats på hans högra sida, Asaf, son till Berekja, son till Simea,
\par 40 son till Mikael, son till Baaseja, son till Malkia,
\par 41 son till Etni, son till Sera, son till Adaja,
\par 42 son till Etan, son till Simma, son till Simei,
\par 43 son till Jahat, son till Gersom, son till Levi.
\par 44 Och deras bröder, Meraris barn stodo på den vänstra sidan: Etan son till Kisi, son till Abdi, son till Malluk,
\par 45 son till Hasabja, son till Amasja, son till Hilkia,
\par 46 son till Amsi, son till Bani, son till Semer,
\par 47 son till Maheli, son till Musi, son till Merari, son till Levi.
\par 48 Och deras bröder, de övriga leviterna, hade blivit givna till allt slags tjänstgöring vid tabernaklet, Guds hus.
\par 49 Men Aron och hans söner ombesörjde offren på brännoffersaltaret och på rökelsealtaret, och skulle utföra all förrättning i det allraheligaste och bringa försoning för Israel, alldeles såsom Mose, Guds tjänare, hade bjudit.
\par 50 Och dessa voro Arons söner: hans son Eleasar, dennes son Pinehas, dennes son Abisua,
\par 51 dennes son Bucki, dennes son Ussi, dennes son Seraja,
\par 52 dennes son Merajot, dennes son Amarja, dennes son Ahitub,
\par 53 dennes son Sadok, dennes son Ahimaas.
\par 54 Och dessa voro deras boningsorter, efter deras tältläger inom deras område: Åt Arons söner av kehatiternas släkt - ty dem träffade nu lotten -
\par 55 åt dem gav man Hebron i Juda land med dess utmarker runt omkring.
\par 56 Men åkerjorden och byarna som hörde till staden gav man åt Kaleb, Jefunnes son.
\par 57 Åt Arons söner gav man alltså fristäderna Hebron och Libna med dess utmarker, vidare Jattir och Estemoa med dess utmarker.
\par 58 Hilen med dess utmarker, Debir med dess utmarker,
\par 59 Asan med dess utmarker och Bet-Semes med dess utmarker;
\par 60 och ur Benjamins stam Geba med dess utmarker, Alemet med dess utmarker och Anatot med dess utmarker, så att deras städer tillsammans utgjorde tretton städer, efter deras släkter.
\par 61 Och Kehats övriga barn fingo ur en stamsläkt, nämligen den stamhalva som utgjorde ena hälften av Manasse stam, genom lottkastning tio städer.
\par 62 Gersoms barn åter fingo, efter sina släkter, ur Isaskars stam, ur Asers stam, ur Naftali stam och ur Manasse stam i Basan tretton städer.
\par 63 Meraris barn fingo, efter sina släkter, ur Rubens stam, ur Gads stam och ur Sebulons stam genom lottkastning tolv städer.
\par 64 Så gåvo Israels barn åt leviterna dessa städer med deras utmarker.
\par 65 Genom lottkastning gåvo de åt dem ur Juda barns stam, ur Simeons barns stam och ur Benjamins barns stam dessa städer, som de namngåvo.
\par 66 Och bland Kehats barns släkter fingo några följande städer ur Efraims stam såsom sitt område:
\par 67 Man gav dem fristäderna Sikem med dess utmarker i Efraims bergsbygd, Geser med dess utmarker,
\par 68 Jokmeam med dess utmarker, Bet-Horon med dess utmarker;
\par 69 vidare Ajalon med dess utmarker och Gat-Rimmon med dess utmarker;
\par 70 och ur ena hälften av Manasse stam Aner med dess utmarker och Bileam med dess utmarker. Detta tillföll Kehats övriga barns släkt.
\par 71 Gersoms barn fingo ur den släkt som utgjorde ena hälften av Manasse stam Golan i Basan med dess utmarker och Astarot med dess utmarker;
\par 72 och ur Isaskars stam Kedes med dess utmarker, Dobrat med dess utmarker,
\par 73 Ramot med dess utmarker och Anem med dess utmarker;
\par 74 och ur Asers stam Masal med dess utmarker, Abdon med dess utmarker,
\par 75 Hukok med dess utmarker och Rehob med dess utmarker;
\par 76 och ur Naftali stam Kedes i Galileen med dess utmarker, Hammon med dess utmarker och Kirjataim med dess utmarker.
\par 77 Meraris övriga barn fingo ur Sebulons stam Rimmono med dess utmarker och Tabor med dess utmarker,
\par 78 och på andra sidan Jordan mitt emot Jeriko, öster om Jordan, ur Rubens stam Beser i öknen med dess utmarker, Jahas med dess utmarker,
\par 79 Kedemot med dess utmarker och Mefaat med dess utmarker;
\par 80 och ur Gads stam Ramot i Gilead med dess utmarker, Mahanaim med dess utmarker,
\par 81 Hesbon med dess utmarker och Jaeser med dess utmarker.

\chapter{7}

\par 1 Och Isaskars söner voro Tola och Pua, Jasib och Simron, tillsammans fyra.
\par 2 Tolas söner voro Ussi, Refaja, Jeriel, Jamai, Jibsam och Samuel, huvudmän för sina familjer, ättlingar av Tola, tappra stridsmän, upptecknade efter sin ättföljd. I Davids tid var deras antal tjugutvå tusen sex hundra.
\par 3 Ussis söner voro Jisraja, och Jisrajas söner voro Mikael, Obadja och Joel samt Jissia, tillhopa fem, allasammans huvudmän.
\par 4 Och med dem följde stridbara härskaror, trettiosex tusen man, efter sin ättföljd och sina familjer; ty de hade många hustrur och barn.
\par 5 Och deras bröder i alla Isaskars släkter voro tappra stridsmän; åttiosju tusen utgjorde tillsammans de som voro upptecknade i deras släktregister.
\par 6 Benjamins söner voro Bela, Beker och Jediael, tillsammans tre.
\par 7 Belas söner voro Esbon, Ussi, Ussiel, Jerimot och Iri, tillsammans fem, huvudmän för sina familjer, tappra stridsmän; de som voro upptecknade i deras släktregister utgjorde tjugutvå tusen trettiofyra.
\par 8 Bekers söner voro Semira, Joas, Elieser, Eljoenai, Omri, Jeremot, Abia, Anatot och Alemet. Alla dessa voro Bekers söner.
\par 9 De som voro upptecknade i deras släktregister, efter sin ättföljd, efter huvudmannen för sina familjer, tappra stridsmän, utgjorde tjugu tusen två hundra.
\par 10 Jediaels söner voro Bilhan; Bilhans söner voro Jeus, Benjamin, Ehud, Kenaana, Setan, Tarsis och Ahisahar.
\par 11 Alla dessa voro Jediaels söner, upptecknade efter huvudmännen för sina familjer, tappra stridsmän, sjutton tusen två hundra stridbara krigsmän.
\par 12 Och Suppim och Huppim voro Irs söner. - Men Husim voro Ahers söner.
\par 13 Naftalis söner voro Jahasiel, Guni, Jeser och Sallum, Bilhas söner.
\par 14 Manasses söner voro Asriel, som kvinnan födde; hans arameiska bihustru födde Makir, Gileads fader.
\par 15 Och Makir tog hustru åt Huppim och Suppim. Hans syster hette Maaka. Och den andre hette Selofhad. Och Selofhad hade döttrar.
\par 16 Och Maaka, Makirs hustru, födde en son och gav honom namnet Peres, men hans broder hette Seres. Hans söner voro Ulam och Rekem.
\par 17 Ulams söner voro Bedan. Dessa voro söner till Gilead, son till Makir, son till Manasse.
\par 18 Och hans syster var Hammoleket; hon födde Is-Hod, Abieser och Mahela.
\par 19 Och Semidas söner voro Ajan, Sekem, Likhi och Aniam.
\par 20 Och Efraims söner voro Sutela, dennes son Bered, dennes son Tahat, dennes son Eleada, dennes son Tahat,
\par 21 dennes son Sabad och dennes son Sutela, så ock Eser och Elead. Och män från Gat, som voro födda där i landet, dräpte dem, därför att de hade dragit ned för att taga deras boskapshjordar.
\par 22 Då sörjde Efraim, deras fader, i lång tid, och hans bröder kommo för att trösta honom.
\par 23 Och han gick in till sin hustru, och hon blev havande och födde en son; och han gav honom namnet Beria, därför att det hade skett under en olyckstid för hans hus.
\par 24 Hans dotter var Seera; hon byggde Nedre och Övre Bet-Horon, så ock Ussen-Seera.
\par 25 Och hans son var Refa; hans son var Resef, ävensom Tela; hans son var Tahan.
\par 26 Hans son var Laedan; hans son var Ammihud; hans son var Elisama.
\par 27 Hans son var Non; hans son var Josua.
\par 28 Och deras besittning och deras boningsorter voro Betel med underlydande orter, österut Naaran och västerut Geser med underlydande orter, vidare Sikem med underlydande orter, ända till Aja med underlydande orter.
\par 29 Men i Manasse barns ägo voro Bet-Sean med underlydande orter, Taanak med underlydande orter, Megiddo med underlydande orter, Dor med underlydande orter. Här bodde nu Josefs, Israels sons, barn.
\par 30 Asers söner voro Jimna, Jisva, Jisvi och Beria; och deras syster var Sera.
\par 31 Berias söner voro Heber och Malkiel; han var Birsaits fader.
\par 32 Och Heber födde Jaflet, Somer och Hotam, så ock Sua, deras syster.
\par 33 Och Jaflets söner voro Pasak, Bimhal och Asvat. Dessa voro Jaflets söner.
\par 34 Semers söner voro Ahi och Rohaga, Jaba och Aram.
\par 35 Hans broder Helems söner voro Sofa, Jimna, Seles och Amal.
\par 36 Sofas söner voro Sua, Harnefer, Sual, Beri och Jimra,
\par 37 Beser, Hod, Samma, Silsa, Jitran och Beera.
\par 38 Jeters söner voro Jefunne, Pispa och Ara.
\par 39 Och Ullas söner voro Ara, Hanniel och Risja.
\par 40 Alla dessa voro Asers söner, huvudmän för sina familjer, utvalda tappra stridsmän, huvudmän bland hövdingarna; och de som voro upptecknade i deras släktregister såsom dugliga till krigstjänst utgjorde ett antal av tjugusex tusen man.

\chapter{8}

\par 1 Och Benjamin födde Bela, sin förstfödde, Asbel, den andre, och Ahara, den tredje,
\par 2 Noha, den fjärde, och Rafa, den femte.
\par 3 Bela hade följande söner: Addar, Gera, Abihud,
\par 4 Abisua, Naaman, Ahoa,
\par 5 Gera, Sefufan och Huram.
\par 6 Och dessa voro Ehuds söner, och de voro familjehuvudmän för dem som bodde i Geba, och som blevo bortförda till Manahat,
\par 7 dit Gera jämte Naaman och Ahia förde bort dem: han födde Ussa och Ahihud.
\par 8 Och Saharaim födde barn i Moabs land, sedan han hade skilt sig från sina hustrur, Husim och Baara;
\par 9 med sin hustru Hodes födde han där Jobab, Sibja, Mesa, Malkam,
\par 10 Jeus, Sakeja och Mirma. Dessa voro hans söner, huvudmän för familjer.
\par 11 Med Husim hade han fött Abitub och Elpaal.
\par 12 Och Elpaals söner voro Eber, Miseam och Semed. Han var den som byggde Ono och Lod med underlydande orter.
\par 13 Beria och Sema - vilka voro familjehuvudmän för Ajalons invånare och förjagade Gats invånare -
\par 14 så ock Ajo, Sasak och Jeremot.
\par 15 Och Sebadja, Arad, Eder,
\par 16 Mikael, Jispa och Joha voro Berias söner.
\par 17 Och Sebadja, Mesullam, Hiski, Heber,
\par 18 Jismerai, Jislia och Jobab voro Elpaals söner.
\par 19 Och Jakim, Sikri, Sabdi,
\par 20 Elienai, Silletai, Eliel,
\par 21 Adaja, Beraja och Simrat voro Simeis söner.
\par 22 Och Jispan, Eber, Eliel,
\par 23 Abdon, Sikri, Hanan,
\par 24 Hananja, Elam, Antotja,
\par 25 Jifdeja och Peniel voro Sasaks söner.
\par 26 Och Samserai, Seharja, Atalja,
\par 27 Jaaresja, Elia och Sikri voro Jerohams söner.
\par 28 Dessa vore huvudman för familjer, huvudmän efter sin ättföljd; de bodde i Jerusalem.
\par 29 I Gibeon bodde Gibeons fader, vilkens hustru hette Maaka.
\par 30 Och hans förstfödde son var Abdon; vidare Sur, Kis, Baal, Nadab,
\par 31 Gedor, Ajo och Seker.
\par 32 Men Miklot födde Simea. Också dessa bodde jämte sina bröder i Jerusalem, gent emot sina bröder.
\par 33 Och Ner födde Kis, Kis födde Saul, och Saul födde Jonatan, Malki-Sua, Abinadab och Esbaal.
\par 34 Jonatans son var Merib-Baal, och Merib-Baal födde Mika.
\par 35 Mikas söner voro Piton, Melek, Taarea och Ahas.
\par 36 Ahas födde Joadda, Joadda födde Alemet, Asmavet och Simri, och Simri födde Mosa.
\par 37 Mosa födde Binea. Hans son var Rafa; hans son var Eleasa; hans son var Asel.
\par 38 Och Asel hade sex söner, och dessa hette Asrikam, Bokeru, Ismael, Searja, Obadja och Hanan. Alla dessa voro Asels söner.
\par 39 Och hans broder Eseks söner voro Ulam, hans förstfödde, Jeus, den andre, och Elifelet, den tredje.
\par 40 Och Ulams söner voro tappra stridsmän, som voro skickliga i att spänna båge; och de hade många söner och sonsöner: ett hundra femtio. Alla dessa voro av Benjamins barn

\chapter{9}

\par 1 Och hela Israel blev upptecknat i släktregister, och de finnas uppskrivna i boken om Israels konungar. Och Juda fördes i fångenskap bort till Babel för sin otrohets skull.
\par 2 Men de förra invånarna som bodde där de hade sin arvsbesittning, i sina städer, utgjordes av vanliga israeliter, präster, leviter och tempelträlar.
\par 3 I Jerusalem bodde en del av Juda barn, av Benjamins barn och av Efraims och Manasse barn, nämligen:
\par 4 Utai, son till Ammihud, son till Omri, son till Imri, son till Bani, av Peres', Judas sons, barn;
\par 5 av siloniterna Asaja, den förstfödde, och hans söner;
\par 6 av Seras barn Jeguel och deras broder, sex hundra nittio;
\par 7 av Benjamins barn Sallu, son till Mesullam, son till Hodauja, son till Hassenua,
\par 8 vidare Jibneja, Jerohams son, och Ela, son till Ussi, son till Mikri, och Mesullam, son till Sefatja, son till Reguel, son till Jibneja,
\par 9 så ock deras bröder, efter deras ättföljd, nio hundra femtiosex. Alla dessa män voro huvudmän för familjer, var och en för sin familj.
\par 10 Och av prästerna: Jedaja, Jojarib och Jakin,
\par 11 vidare Asarja, son till Hilkia, son till Mesullam, son till Sadok, son till Merajot, son till Ahitub, fursten i Guds hus,
\par 12 vidare Adaja, son till Jeroham, son till Pashur, son till Malkia, vidare Maasai, son till Adiel, son till Jasera, son till Mesullam, son till Mesillemit, son till Immer,
\par 13 så ock deras bröder, huvudmän för sina familjer, ett tusen sju hundra sextio, dugande män i de sysslor som hörde till tjänstgöringen i Guds hus.
\par 14 Och av leviterna: Semaja, som till Hassub, son till Asrikam, son till Hasabja, av Meraris barn,
\par 15 vidare Bakbackar, Heres och Galal, så ock Mattanja, son till Mika, son till Sikri, son till Asaf,
\par 16 vidare Obadja, son till Semaja, son till Galal, son till Jedutun, så ock Berekja, son till Asa, son till Elkana, som bodde i netofatiternas byar.
\par 17 Och dörrvaktarna: Sallum, Ackub, Talmon och Ahiman med sina bröder; men Sallum var huvudmannen.
\par 18 Och ända till nu göra de tjänst vid Konungsporten, på östra sidan. Dessa voro dörrvaktarna i Levi barns läger.
\par 19 Men Sallum, son till Kore, son till Ebjasaf, son till Kora, hade jämte sina bröder, dem som voro av hans familj, koraiterna, till tjänstgöringssyssla att hålla vakt vid tältets trösklar; deras fäder hade nämligen i HERRENS läger hållit vakt vid ingången.
\par 20 Och Pinehas, Eleasars son, hade förut varit furste över dem - med honom vare HERREN!
\par 21 Sakarja, Meselemjas son, var dörrvaktare vid ingången till uppenbarelsetältet.
\par 22 Alla dessa voro utvalda till dörrvaktare vid trösklarna: två hundra tolv. De blevo i sina byar upptecknade i släktregistret. David och siaren Samuel hade tillsatt dem att tjäna på heder och tro.
\par 23 De och deras söner stodo därför vid portarna till HERRENS hus, tälthuset, och höllo vakt.
\par 24 Efter de fyra väderstrecken hade dörrvaktarna sina platser: i öster, väster, norr och söder.
\par 25 Och deras bröder, de som fingo bo i sina byar, skulle var sjunde dag, alltid på samma timme, infinna sig hos dem.
\par 26 Ty på heder och tro voro dessa fyra anställda såsom förmän för dörrvaktarna. Detta var nu leviterna. De hade ock uppsikten över kamrarna och förvaringsrummen i Guds hus.
\par 27 Och de vistades om natten runt omkring Guds hus, ty dem ålåg att hålla vakt, och de skulle öppna dörrarna var morgon.
\par 28 Somliga av dem hade uppsikten över de kärl som användes vid tjänstgöringen. De buro nämligen in dem, efter att hava räknat dem, och buro sedan ut dem, efter att åter hava räknat dem.
\par 29 Och somliga av dem voro förordnade till att hava uppsikten över de andra kärlen, över alla andra helgedomens kärl, så ock över det fina mjölet och vinet och oljan och rökelsen och de välluktande kryddorna.
\par 30 Men somliga av prästernas söner beredde salvan av de välluktande kryddorna.
\par 31 Och Mattitja, en av leviterna, koraiten Sallums förstfödde, hade på heder och tro uppsikten över bakverket.
\par 32 Och somliga av deras bröder, kehatiternas söner, hade uppsikten över skådebröden och skulle tillreda dem för var sabbat.
\par 33 Men de andra, nämligen sångarna, huvudmän för levitiska familjer, vistades i kamrarna, fria ifrån annan tjänstgöring, ty dag och natt voro de upptagna av sina egna sysslor.
\par 34 Dessa voro huvudmännen för de levitiska familjerna, huvudman efter sin ättföljd; de bodde i Jerusalem.
\par 35 I Gibeon bodde Gibeons fader Jeguel, vilkens hustru hette Maaka.
\par 36 Och hans förstfödde son var Abdon; vidare Sur, Kis, Baal, Ner, Nadab
\par 37 Gedor, Ajo, Sakarja och Miklot.
\par 38 Men Miklot födde Simeam. Också de bodde jämte sina bröder i Jerusalem, gent emot sina bröder.
\par 39 Och Ner födde Kis, Kis födde Saul, och Saul födde Jonatan, Malki-Sua, Abinadab och Esbaal.
\par 40 Jonatans son var Merib-Baal, och Merib-Baal födde Mika.
\par 41 Mikas söner voro Piton, Melek och Taharea.
\par 42 Ahas födde Jaera, Jaera födde Alemet, Asmavet och Simri, och Simri födde Mosa.
\par 43 Mosa födde Binea. Hans son var Refaja; hans son var Eleasa; hans son var Asel.
\par 44 Och Asel hade sex söner, och dessa hette Asrikam, Bokeru, Ismael, Searja, Obadja och Hanan. Dessa voro Asels söner.

\chapter{10}

\par 1 Och filistéerna stridde mot Israel; och Israels män flydde för filistéerna och föllo slagna på berget Gilboa.
\par 2 Och filistéerna ansatte ivrigt Saul och hans söner. Och filistéerna dödade Jonatan, Abinadab och Malki-Sua, Sauls söner.
\par 3 När då Saul själv blev häftigt anfallen och bågskyttarna kommo över honom, greps han av förskräckelse för skyttarna.
\par 4 Och Saul sade till sin vapendragare: "Drag ut ditt svärd och genomborra mig därmed, så att icke dessa oomskurna komma och hantera mig skändligt." Men hans vapendragare ville det icke, ty han fruktade storligen. Då tog Saul själv svärdet och störtade sig därpå.
\par 5 Men när vapendragaren såg att Saul var död, störtade han sig ock på sitt svärd och dog.
\par 6 Så dogo då Saul och hans tre söner; och alla som hörde till hans hus dogo på samma gång.
\par 7 Och när alla israeliterna i dalen förnummo att deras här hade flytt, och att Saul och hans söner voro döda, övergåvo de sina städer och flydde; sedan kommo filistéerna och bosatte sig i dem.
\par 8 Dagen därefter kommo filistéerna för att plundra de slagna och funno då Saul och hans söner, där de lågo fallna på berget Gilboa.
\par 9 Och de plundrade honom och togo med sig hans huvud och hans vapen och sände dem omkring i filistéernas land och läto förkunna det glada budskapet för sina avgudar och för folket.
\par 10 Och de lade hans vapen i sitt gudahus, men hans huvudskål hängde de upp i Dagons tempel.
\par 11 Men när allt folket i Jabes i Gilead hörde allt vad filistéerna hade gjort med Saul,
\par 12 stodo de upp, alla stridbara män, och togo Sauls och hans söners lik och förde dem till Jabes; och de begrovo deras ben under terebinten i Jabes och fastade så i sju dagar.
\par 13 Detta blev Sauls död, därför att han hade begått otrohet mot HERREN, i det att han icke hade hållit HERRENS ord, så ock därför att han hade frågat en ande och sökt svar hos en sådan.
\par 14 Han hade icke sökt svar hos HERREN; därför dödade HERREN honom. Och sedan överflyttade han konungadömet på David, Isais son.

\chapter{11}

\par 1 Då församlade sig hela Israel till David i Hebron och sade: "Vi äro ju ditt kött och ben.
\par 2 Redan för länge sedan, redan då Saul ännu var konung, var det du som var ledare och anförare för Israel. Och till dig har HERREN, din Gud, sagt: Du skall vara en herde för mitt folk Israel, ja, du skall vara en furste över mitt folk Israel."
\par 3 När så alla de äldste i Israel kommo till konungen i Hebron, slöt David ett förbund med dem där i Hebron, inför HERREN; och sedan smorde de David till konung över Israel, i enlighet med HERRENS ord genom Samuel.
\par 4 Och David drog med hela Israel till Jerusalem, det är Jebus; där befunno sig jebuséerna, som ännu bodde kvar i landet.
\par 5 Och invånarna i Jebus sade till David: "Hitin kommer du icke." Men David intog likväl Sions borg, det är Davids stad
\par 6 Och David sade: "Vemhelst som först slår ihjäl en jebusé, han skall bliva hövding och anförare." Och Joab, Serujas son, kom först ditupp och blev så hövding.
\par 7 Sedan tog David sin boning i bergfästet; därför kallade man det Davids stad.
\par 8 Och han uppförde befästningsverk runt omkring staden, från Millo och allt omkring; och Joab återställde det övriga av staden.
\par 9 Och David blev allt mäktigare och mäktigare, och HERREN Sebaot var med honom
\par 10 Och dessa äro de förnämsta bland Davids hjältar, vilka gåvo honom kraftig hjälp att bliva konung, de jämte hela Israel, och så skaffade honom konungaväldet, enligt HERRENS ord angående Israel.
\par 11 Detta är förteckningen på Davids hjältar: Jasobeam, son till en hakmonit, den förnämste bland kämparna, han som svängde sitt spjut över tre hundra som hade blivit slagna på en gång.
\par 12 Och efter honom kom ahoaiten Eleasar, son till Dodo; han var en av de tre hjältarna.
\par 13 Han var med David vid Pas-Dammim, när filistéerna där hade församlat sig till strid. Och där var ett åkerstycke, fullt med korn. Och folket flydde för filistéerna.
\par 14 Då ställde de sig mitt på åkerstycket och försvarade det och slogo filistéerna; och HERREN lät dem så vinna en stor seger.
\par 15 En gång drogo tre av de trettio förnämsta männen ned över klippan till David vid Adullams grotta, medan en avdelning filistéer var lägrad i Refaimsdalen.
\par 16 Men David var då på borgen, under det att en filisteisk utpost fanns i Bet-Lehem.
\par 17 Och David greps av lystnad och sade: "Ack att någon ville giva mig vatten att dricka från brunnen vid Bet-Lehems stadsport!"
\par 18 Då bröto de tre sig igenom filistéernas läger och hämtade vatten ur brunnen vid Bet-Lehems stadsport och togo det och buro det till David. Men David ville icke dricka det, utan göt ut det såsom ett drickoffer åt HERREN.
\par 19 Han sade nämligen: "Gud låte det vara fjärran ifrån mig att jag skulle göra detta! Skulle jag dricka dessa mäns blod, som hava vågat sina liv? Ty med fara för sina liv hava de burit det hit." Och han ville icke dricka det. Sådana ting hade de tre hjältarna gjort.
\par 20 Absai, Joabs broder, var den förnämste av tre andra; han svängde en gång sitt spjut över tre hundra som hade blivit slagna. Och han hade ett stort namn bland de tre.
\par 21 Han var dubbelt mer ansedd än någon annan i detta tretal, och han var deras hövitsman, men upp till de tre första kom han dock icke.
\par 22 Vidare Benaja, son till Jojada, som var son till en tapper, segerrik man från Kabseel; han slog ned de två Arielerna i Moab, och det var han som en snövädersdag steg ned och slog ihjäl lejonet i brunnen.
\par 23 Han slog ock ned den egyptiske mannen som var så reslig: fem alnar lång. Fastän egyptiern i handen hade ett spjut som liknade en vävbom, gick han ned mot honom, väpnad allenast med sin stav. Och han ryckte spjutet ur egyptiern hand och dräpte honom med hans eget spjut.
\par 24 Sådana ting hade Benaja, Jojadas son, gjort. Och han hade ett stort namn bland de tre hjältarna.
\par 25 Ja, han var mer ansedd än någon av de trettio, men upp till de tre första kom han icke. Och David satte honom till anförare för sin livvakt.
\par 26 De tappra hjältarna voro: Asael, Joabs broder, Elhanan, Dodos son, från Bet-Lehem;
\par 27 haroriten Sammot; peloniten Heles;
\par 28 tekoaiten Ira, Ickes' son; anatotiten Abieser;
\par 29 husatiten Sibbekai; ahoaiten Ilai;
\par 30 netofatiten Maherai; netofatiten Heled, Baanas son;
\par 31 Itai, Ribais son, från Gibea i Benjamins barns stam; pirgatoniten Benaja;
\par 32 Hurai från Gaas' dalar; arabatiten Abiel;
\par 33 baharumiten Asmavet; saalboniten Eljaba;
\par 34 gisoniten Bene-Hasem; harariten Jonatan, Sages son;
\par 35 harariten Ahiam, Sakars son; Elifal, Urs son;
\par 36 mekeratiten Hefer; peloniten Ahia;
\par 37 Hesro från Karmel; Naarai, Esbais son;
\par 38 Joel, broder till Natan; Mibhar, Hagris son;
\par 39 ammoniten Selek; berotiten Naherai, vapendragare åt Joab, Serujas son;
\par 40 jeteriten Ira; jeteriten Gareb;
\par 41 hetiten Uria; Sabad, Alais son;
\par 42 rubeniten Adina, Sisas son, en huvudman bland rubeniterna, och jämte honom trettio andra;
\par 43 Hanan, Maakas son, och mitniten Josafat;
\par 44 astarotiten Ussia; Sama och Jeguel, aroeriten Hotams söner;
\par 45 Jediael, Simris son, och hans broder Joha, tisiten;
\par 46 Eliel-Hammahavim samt Jeribai och Josauja, Elnaams söner, och moabiten Jitma;
\par 47 slutligen Eliel, Obed och Jaasiel-Hammesobaja.

\chapter{12}

\par 1 Och dessa voro de som kommo till David i Siklag, medan han ännu höll sig undan för Saul, Kis' son; de hörde till de hjältar som bistodo honom under kriget.
\par 2 De voro väpnade med båge och skickliga i att, både med höger och med vänster hand, slunga stenar och avskjuta pilar från bågen. Av Sauls stamfränder, benjaminiterna, kommo:
\par 3 Ahieser, den förnämste, och Joas, gibeatiten Hassemaas söner; Jesuel och Pelet, Asmavets söner; Beraka; anatotiten Jehu;
\par 4 gibeoniten Jismaja, en av de trettio hjältarna, anförare för de trettio; Jeremia; Jahasiel; Johanan; gederatiten Josabad;
\par 5 Eleusai; Jerimot; Bealja; Semarja; harufiten Sefatja;
\par 6 koraiterna Elkana, Jissia, Asarel, Joeser och Jasobeam;
\par 7 Joela och Sebadja, söner till Jeroham, av strövskaran.
\par 8 Och av gaditerna avföllo några och gingo till David i bergfästet i öknen, tappra män, krigsmän skickliga att strida, rustade med sköld och spjut; de hade en uppsyn såsom lejon och voro snabba såsom gaseller på bergen:
\par 9 Eser, den förnämste, Obadja, den andre, Eliab, den tredje,
\par 10 Masmanna, den fjärde, Jeremia, den femte,
\par 11 Attai, den sjätte, Eliel, den sjunde,
\par 12 Johanan, den åttonde, Elsabad, den nionde,
\par 13 Jeremia, den tionde, Makbannai, den elfte.
\par 14 Dessa hörde till Gads barn och till de förnämsta i hären; den ringaste av dem var ensam så god som hundra, men den ypperste så god som tusen.
\par 15 Dessa voro de som i första månaden gingo över Jordan, när den var full över alla sina bräddar, och som förjagade alla dem som bodde i dalarna, åt öster och åt väster.
\par 16 Av Benjamins och Juda barn kommo några män till David ända till bergfästet.
\par 17 Då gick David ut emot dem och tog till orda och sade till dem: "Om I kommen till mig i fredlig avsikt och viljen bistå mig, så är mitt hjärta redo till förening med eder; men om I kommen för att förråda mig åt mina ovänner, fastän ingen orätt är i mina händer, då må våra fäders Gud se därtill och straffa det."
\par 18 Men Amasai, den förnämste bland de trettio, hade blivit beklädd med andekraft, och han sade: "Dina äro vi, David, och med dig stå vi, du Isais son. Frid vare med dig, frid, och frid vare med dem som bistå dig ty din Gud har bistått dig!" Och David tog emot dem och gav dem plats bland de förnämsta i sin skara.
\par 19 Från Manasse gingo några över till David, när han med filistéerna drog ut i strid mot Saul, dock fingo de icke bistå dessa; ty när filistéernas hövdingar hade rådplägat, skickade de bort honom, i det de sade: "Det gäller huvudet för oss, om han går över till sin herre Saul.
\par 20 När han då drog till Siklag, gingo dessa från Manasse över till honom: Adna, Josabad, Jediael, Mikael, Josabad, Elihu och Silletai, huvudmän för de ätter som tillhörde Manasse.
\par 21 Dessa bistodo David mot strövskaran, ty de voro allasammans tappra stridsmän och blevo hövitsmän i hären.
\par 22 Dag efter dag kommo nämligen allt flera till David för att bistå honom, så att hans läger blev övermåttan stort.
\par 23 Detta är de tal som angiva summorna av det väpnade krigsfolk som kom till David i Hebron, för att efter HERRENS befallning flytta Sauls konungamakt över på honom:
\par 24 Juda barn, som buro sköld och spjut, sex tusen åtta hundra, väpnade till strid;
\par 25 av Simeons barn tappra krigsmän, sju tusen ett hundra;
\par 26 av Levi barn fyra tusen sex hundra;
\par 27 därtill Jojada, fursten inom Arons släkt, och med honom tre tusen sju hundra;
\par 28 så ock Sadok, en tapper yngling, med sin familj, tjugutvå hövitsmän;
\par 29 av Benjamins barn, Sauls stamfränder, tre tusen (ty ännu vid den tiden höllo de flesta av dem troget med Sauls hus);
\par 30 av Efraims barn tjugu tusen åtta hundra, tappra stridsmän, namnkunniga män i sina familjer;
\par 31 av ena hälften av Manasse stam aderton tusen namngivna män, som kommo för att göra David till konung;
\par 32 av Isaskars barn kommo män som väl förstodo tidstecknen och insågo vad Israel borde göra, två hundra huvudmän, därtill alla deras stamfränder under deras befäl;
\par 33 av Sebulon stridbara män, rustade till krig med alla slags vapen, femtio tusen, som samlades endräktigt;
\par 34 av Naftali ett tusen hövitsmän, och med dem trettiosju tusen, väpnade med sköld och spjut;
\par 35 av daniterna krigsrustade män, tjuguåtta tusen sex hundra;
\par 36 av Aser stridbara män, rustade till krig, fyrtio tusen;
\par 37 och från andra sidan Jordan, av rubeniterna, gaditerna och andra hälften av Manasse stam, ett hundra tjugu tusen, väpnade med alla slags vapen som brukas vid krigföring.
\par 38 Alla dessa krigsmän, ordnade till strid, kommo i sina hjärtans hängivenhet till Hebron för att göra David till konung över hela Israel. Också hela det övriga Israel var enigt i att göra David till konung.
\par 39 Och de voro där hos David i tre dagar och åto och drucko, ty deras bröder hade försett dem med livsmedel.
\par 40 De som bodde närmast dem, ända upp till Isaskar, Sebulon och Naftali, tillförde dem ock på åsnor, kameler, mulåsnor och oxar livsmedel i myckenhet till föda: mjöl, fikonkakor och russinkakor, vin och olja, fäkreatur och småboskap; ty glädje rådde i Israel.

\chapter{13}

\par 1 Och David rådförde sig med över- och underhövitsmännen, med alla furstarna.
\par 2 Sedan sade David till Israels hela församling: "Om I så finnen för gott, och om detta är från HERREN, vår Gud, så låt oss sända bud åt alla håll till våra övriga bröder i alla Israels landsändar, och därjämte till prästerna och leviterna i de städer kring vilka de hava sina utmarker, att de må församla sig till oss;
\par 3 och låt oss flytta vår Guds ark till oss, ty i Sauls tid frågade vi icke efter den."
\par 4 Och hela församlingen svarade att man skulle göra så, ty förslaget behagade hela folket.
\par 5 Så församlade då David hela Israel, från Sihor i Egypten ända dit där vägen går till Hamat, för att hämta Guds ark från Kirjat-Jearim.
\par 6 Och David drog med hela Israel upp till Baala, det är Kirjat-Jearim, som hör till Juda, för att därifrån föra upp Guds, HERRENS, ark, hans som tronar på keruberna, och efter vilken den hade fått sitt namn.
\par 7 Och de satte Guds ark på en ny vagn och förde den bort ifrån Abinadabs hus; och Ussa och Ajo körde vagnen.
\par 8 Och David och hela Israel fröjdade sig inför Gud av all makt, med sånger och med harpor, psaltare, pukor, cymbaler och trumpeter.
\par 9 Men när de kommo till Kidonslogen, räckte Ussa ut sin hand för att fatta I arken, ty oxarna snavade.
\par 10 Då upptändes HERRENS vrede mot Ussa, och därför att han hade räckt ut sin hand mot arken, slog han honom, så att han föll ned död där inför Gud.
\par 11 Men det gick David hårt till sinnes att HERREN så hade brutit ned Ussa; och han kallade det stället Peres-Ussa, såsom det heter ännu i dag.
\par 12 Och David betogs av sådan fruktan för Gud på den dagen, att han sade: "Huru skulle jag töras låta föra Guds ark till mig?"
\par 13 Därför lät David icke flytta in arken till sig i Davids stad, utan lät sätta in den i gatiten Obed-Edoms hus.
\par 14 Sedan blev Guds ark kvar vid Obed-Edoms hus, där den stod i sitt eget hus, i tre månader; men HERREN välsignade Obed-Edoms hus och allt vad som hörde honom till.

\chapter{14}

\par 1 Och Hiram, konungen i Tyrus, skickade sändebud till David med cederträ, därjämte ock murare och timmermän, för att de skulle bygga honom ett hus.
\par 2 Och David märkte att HERREN hade befäst honom såsom konung över Israel; ty han hade låtit hans rike bliva övermåttan upphöjt, för sitt folk Israels skull.
\par 3 Och David tog sig ännu flera hustrur i Jerusalem, och David födde ännu flera söner och döttrar.
\par 4 Dessa äro namnen på de söner som han fick i Jerusalem: Sammua, Sobab, Natan, Salomo,
\par 5 Jibhar, Elisua, Elpelet,
\par 6 Noga, Nefeg, Jafia,
\par 7 Elisama, Beeljada och Elifelet.
\par 8 Men när filistéerna hörde att David hade blivit smord till konung över hela Israel, drogo de allasammans upp för att fånga David. När David hörde detta, drog han ut mot dem.
\par 9 Då nu filistéerna hade fallit in i Refaimsdalen och där företogo plundringståg,
\par 10 frågade David Gud: "Skall jag draga upp mot filistéerna? Vill du då giva dem i min hand?" HERREN svarade honom: "Drag upp; jag vill giva dem i din hand."
\par 11 Och de drogo upp till Baal-Perasim, och där slog David dem. Då sade David: "Gud har brutit ned mina fiender genom min hand, likasom en vattenflod bryter ned." Därav fick det stället namnet Baal-Perasim.
\par 12 De lämnade där efter sig sina gudar; och David befallde att dessa skulle brännas upp i eld.
\par 13 Men filistéerna företogo ännu en gång plundringståg i dalen.
\par 14 När David då åter frågade Gud, svarade Gud honom: "Du skall icke draga upp efter dem; du må kringgå dem på en omväg, så att du kommer över dem från det håll där bakaträden stå.
\par 15 Så snart du sedan hör ljudet av steg i bakaträdens toppar, drag då ut till strid, ty då har Gud dragit ut framför dig till att slå filistéernas här."
\par 16 David gjorde såsom Gud hade bjudit honom; och de slogo filistéernas här och förföljde dem från Gibeon ända till Geser.
\par 17 Och ryktet om David gick ut i alla länder, och HERREN lät fruktan för honom komma över alla folk.

\chapter{15}

\par 1 Och han uppförde åt sig hus i Davids stad; sedan beredde han en plats åt Guds ark och slog upp ett tält åt den.
\par 2 Därvid befallde David: "Inga andra än leviterna må bära Guds ark; ty dem har HERREN utvalt till att bära Guds ark och till att göra tjänst inför honom för evärdlig tid."
\par 3 Och David församlade hela Israel till Jerusalem för att hämta HERRENS ark upp till den plats som han hade berett åt den.
\par 4 Och David samlade tillhopa Arons barn och leviterna;
\par 5 av Kehats barn: Uriel, deras överste, och hans bröder, ett hundra tjugu;
\par 6 av Meraris barn: Asaja, deras överste, och hans bröder, två hundra tjugu;
\par 7 av Gersoms barn: Joel, deras överste, och hans bröder, ett hundra trettio;
\par 8 av Elisafans barn: Semaja, deras överste, och hans bröder, två hundra;
\par 9 av Hebrons barn: Eliel, deras överste, och hans bröder, åttio;
\par 10 av Ussiels barn: Amminadab, deras överste, och hans bröder, ett hundra tolv.
\par 11 Och David kallade till sig prästerna Sadok och Ebjatar jämte leviterna Uriel, Asaja, Joel, Semaja, Eliel och Amminadab.
\par 12 Och han sade till dem: "I ären huvudmän för leviternas familjer. Helgen eder tillika med edra bröder, och hämten så HERRENS, Israels Guds, ark upp till den plats som jag har berett åt den.
\par 13 Ty därför att I förra gången icke voren tillstädes var det som HERREN, vår Gud, bröt ned en av oss, till straff för att vi icke sökte honom så, som tillbörligt var."
\par 14 Då helgade prästerna och leviterna sig till att hämta upp HERRENS, Israels Guds, ark.
\par 15 Och såsom Mose hade bjudit i enlighet med HERRENS ord, buro nu Levi barn Guds ark med stänger, som vilade på deras axlar.
\par 16 Och David sade till de översta bland leviterna att de skulle förordna sina bröder sångarna till tjänstgöring med musikinstrumenter, psaltare, harpor och cymbaler, som de skulle låta ljuda, under det att de höjde glädjesången.
\par 17 Leviterna förordnade då Heman, Joels son, och av hans bröder Asaf, Berekjas son, och av dessas bröder, Meraris barn, Etan, Kusajas son,
\par 18 och jämte dem deras bröder av andra ordningen Sakarja, Ben, Jaasiel, Semiramot, Jehiel, Unni, Eliab, Benaja, Maaseja, Mattitja, Elifalehu, Mikneja, Obed-Edom och Jegiel, dörrvaktarna.
\par 19 Och sångarna, Heman, Asaf och Etan, skulle slå kopparcymbaler.
\par 20 Sakarja, Asiel, Semiramot, Jehiel, Unni, Eliab, Maaseja och Benaja skulle spela på psaltare, till Alamót.
\par 21 Mattitja, Elifalehu, Mikneja, Obed-Edom, Jegiel och Asasja skulle leda sången med harpor, till Seminit.
\par 22 Kenanja, leviternas anförare, när de buro, skulle undervisa i att bära, ty han var kunnig i sådant.
\par 23 Berekja och Elkana skulle vara dörrvaktare vid arken.
\par 24 Sebanja, Josafat, Netanel, Amasai, Sakarja, Benaja och Elieser, prästerna, skulle blåsa i trumpeter framför Guds ark. Slutligen skulle Obed-Edom och Jehia vara dörrvaktare vid arken.
\par 25 Så gingo då David och de äldste i Israel och överhövitsmännen åstad för att hämta HERRENS förbundsark upp ur Obed-Edoms hus, under jubel.
\par 26 Och då Gud skyddade leviterna som buro HERRENS förbundsark, offrade man sju tjurar och sju vädurar.
\par 27 Därvid var David klädd i en kåpa av fint linne; så voro ock alla leviterna som buro arken, så ock sångarna och Kenanja, som anförde sångarna, när de buro. Och därjämte bar David en linne-efod.
\par 28 Och hela Israel hämtade upp HERRENS förbundsark under jubel och basuners ljud; och man blåste i trumpeter och slog cymbaler och lät psaltare och harpor ljuda.
\par 29 När då HERRENS förbundsark kom till Davids stad, blickade Mikal, Sauls dotter, ut genom fönstret, och då hon såg konung David dansa och göra sig glad, fick hon förakt för honom i sitt hjärta.

\chapter{16}

\par 1 Sedan de hade fört Guds ark ditin, ställde de den i tältet som David hade slagit upp åt den, och framburo därefter brännoffer och tackoffer inför Guds ansikte.
\par 2 När David hade offrat brännoffret och tackoffret, välsignade han folket i HERRENS namn.
\par 3 Och åt var och en av alla israeliterna, både man och kvinna, gav han en kaka bröd, ett stycke kött och en druvkaka.
\par 4 Och han förordnade vissa leviter till att göra tjänst inför HERRENS ark, för att de skulle prisa, tacka och lova HERREN, Israels Gud:
\par 5 Asaf såsom anförare, näst efter honom Sakarja, och vidare Jegiel, Semiramot, Jehiel, Mattitja, Eliab, Benaja, Obed-Edom och Jegiel med psaltare och harpor; och Asaf skulle slå cymbaler.
\par 6 Men prästerna Benaja och Jahasiel skulle beständigt stå med sina trumpeter framför Guds förbundsark.
\par 7 På den dagen var det som David först fastställde den ordningen att man genom Asaf och hans bröder skulle tacka HERREN på detta sätt:
\par 8 "Tacken HERREN, åkallen hans namn, gören hans gärningar kunniga bland folken.
\par 9 Sjungen till hans ära, lovsägen honom, talen om alla hans under.
\par 10 Berömmen eder av hans heliga namn; glädje sig av hjärtat de som söka HERREN.
\par 11 Frågen efter HERREN och hans makt, söken hans ansikte beständigt.
\par 12 Tänken på de underbara verk som han har gjort, på hans under och hans muns domar,
\par 13 I Israels, hans tjänares, säd, I Jakobs barn, hans utvalda.
\par 14 Han är HERREN, vår Gud; över hela jorden gå hans domar.
\par 15 Tänken evinnerligen på hans förbund, intill tusen släkten på vad han har stadgat,
\par 16 på det förbund han slöt med Abraham och på hans ed till Isak.
\par 17 Han fastställde det för Jakob till en stadga, för Israel till ett evigt förbund;
\par 18 han sade: 'Åt dig vill jag giva Kanaans land, det skall bliva eder arvedels lott.'
\par 19 Då voren I ännu en liten hop, I voren ringa och främlingar därinne.
\par 20 Och de vandrade åstad ifrån folk till folk ifrån ett rike bort till ett annat.
\par 21 Han tillstadde ingen att göra dem skada, han straffade konungar för deras skull:
\par 22 'Kommen icke vid mina smorda, och gören ej mina profeter något ont.'
\par 23 Sjungen till HERRENS ära, alla länder, båden glädje var dag, förkunnen hans frälsning.
\par 24 Förtäljen bland hedningarna hans ära, bland alla folk hans under.
\par 25 Ty stor är HERREN och högt lovad, och fruktansvärd är han mer än alla gudar.
\par 26 Ty folkens alla gudar äro avgudar, men HERREN är den som har gjort himmelen.
\par 27 Majestät och härlighet äro inför hans ansikte, makt och fröjd i hans boning.
\par 28 Given åt HERREN, I folkens släkter, given åt HERREN ära och makt;
\par 29 given åt HERREN hans namns ära, bären fram skänker och kommen inför hans ansikte, tillbedjen HERREN i helig skrud.
\par 30 Bäven för hans ansikte, alla länder; se, jordkretsen står fast och vacklar icke.
\par 31 Himmelen vare glad, och jorden fröjde sig, och bland hedningarna säge man: 'HERREN är nu konung!'
\par 32 Havet bruse och allt vad däri är, marken glädje sig och allt som är därpå;
\par 33 ja, då juble skogens träd inför HERREN, ty han kommer för att döma jorden.
\par 34 Tacken HERREN, ty han är god, ty hans nåd varar evinnerligen,
\par 35 och sägen: 'Fräls oss, du vår frälsnings Gud, församla oss och rädda oss från hedningarna, så att vi få prisa ditt heliga namn och berömma oss av ditt lov.'
\par 36 Lovad vare HERREN, Israels Gud, från evighet till evighet!" Och allt folket sade: "Amen", och lovade HERREN.
\par 37 Och han gav där, inför HERRENS förbundsark, åt Asaf och hans bröder uppdraget att beständigt göra tjänst inför arken, var dag med de för den dagen bestämda sysslorna.
\par 38 Men Obed-Edom och deras bröder voro sextioåtta; och Obed-Edom, Jedituns son, och Hosa gjorde han till dörrvaktare.
\par 39 Och prästen Sadok och hans bröder, prästerna, anställde han inför HERRENS tabernakel, på offerhöjden i Gibeon,
\par 40 för att de beständigt skulle offra åt HERREN brännoffer på brännoffersaltaret, morgon och afton, och göra allt vad som var föreskrivet i HERRENS lag, den som han hade givit åt Israel;
\par 41 och jämte dem Heman och Jedutun och de övriga namngivna utvalda, på det att de skulle tacka HERREN, därför att hans nåd varar evinnerligen.
\par 42 Och hos dessa, nämligen Heman och Jedutun, förvarades trumpeter och cymbaler åt dem som skulle spela, så ock andra instrumenter som hörde till gudstjänsten. Och Jedutuns söner gjorde han till dörrvaktare.
\par 43 Sedan gick allt folket hem, var och en till sitt; men David vände om för att hälsa sitt husfolk.

\chapter{17}

\par 1 Då nu David satt i sitt hus, sade han till profeten Natan: "Se, jag bor i ett hus av cederträ, under det att HERRENS förbundsark står under ett tält."
\par 2 Natan sade till David: "Gör allt vad du har i sinnet; ty Gud är med dig."
\par 3 Men om natten kom Guds ord till Natan; han sade:
\par 4 "Gå och säg till min tjänare David: Så säger HERREN: Icke du skall bygga mig det hus som jag skall bo i.
\par 5 Jag har ju icke bott i något hus, från den dag då jag förde Israel hitupp ända till denna dag, utan jag har flyttat ifrån tält till tält, ifrån tabernakel till tabernakel.
\par 6 Har jag då någonsin, varhelst jag flyttade omkring med hela Israel, talat och sagt så till någon enda av Israels domare, som jag har förordnat till herde för mitt folk: 'Varför haven I icke byggt mig ett hus av cederträ?'
\par 7 Och nu skall du säga så till min tjänare David: Så säger HERREN Sebaot: Från betesmarken, där du följde fåren, har jag hämtat dig, för att du skulle bliva en furste över mitt folk Israel.
\par 8 Och jag har varit med dig på alla dina vägar och utrotat alla dina fiender för dig. Och jag vill göra dig ett namn, sådant som de störstes namn på jorden.
\par 9 Jag skall bereda en plats åt mitt folk Israel och plantera det, så att det får bo kvar där, utan att vidare bliva oroat. Orättfärdiga människor skola icke mer föröda det, såsom fordom skedde,
\par 10 och såsom det har varit allt ifrån den tid då jag förordnade domare över mitt folk Israel; och jag skall kuva alla dina fiender. Så förkunnar jag nu för dig att HERREN skall bygga ett hus åt dig.
\par 11 Ty det skall ske, att när din tid är ute och du går till dina fäder skall jag efter dig upphöja din son, en av dina avkomlingar; och jag skall befästa hans konungamakt.
\par 12 Han skall bygga ett hus åt mig, och jag skall befästa hans tron för evig tid.
\par 13 Jag skall vara hans fader, och han skall vara min son; och min nåd skall jag icke låta vika ifrån honom, såsom jag lät den vika ifrån din företrädare.
\par 14 Jag skall hålla honom vid makt i mitt hus och i mitt rike för evig tid, och hans tron skall vara befäst för evig tid."
\par 15 Alldeles i överensstämmelse med dessa ord och med denna syn talade nu Natan till David.
\par 16 Då gick konung David in och satte sig ned inför HERRENS ansikte och sade: "Vem är jag, HERRE Gud, och vad är mitt hus, eftersom du har låtit mig komma härtill?
\par 17 Och detta har likväl synts dig vara för litet, o Gud; du har talat angående din tjänares hus om det som ligger långt fram i tiden. Ja, du har sett till mig på människosätt, for att upphöja mig, HERRE Gud.
\par 18 Vad skall nu David vidare säga till dig om den ära du har bevisat din tjänare? Du känner ju din tjänare.
\par 19 HERRE, för din tjänares skull och efter ditt hjärta har du gjort allt detta stora och förkunnat alla dessa stora ting.
\par 20 HERRE, ingen är dig lik, och ingen Gud finnes utom dig, efter allt vad vi hava hört med våra öron.
\par 21 Och var finnes på jorden något enda folk som är likt ditt folk Israel, vilket Gud själv har gått åstad att förlossa åt sig till ett folk - för att så göra dig ett stort och fruktansvärt namn, i det att du förjagade hedningarna för ditt folk, det som du hade förlossat ifrån Egypten?
\par 22 Och du har gjort ditt folk Israel till ett folk åt dig för evig tid, och du, HERRE, har blivit deras Gud
\par 23 Så må nu, HERRE, vad du har talat om din tjänare och om hans hus bliva fast för evig tid; gör såsom du har talat.
\par 24 Då skall ditt namn anses fast och bliva stort till evig tid, så att man skall säga: 'HERREN Sebaot, Israels Gud, är Gud över Israel.' Och så skall din tjänare Davids hus bestå inför dig.
\par 25 Ty du, min Gud, har uppenbarat för din tjänare att du skall bygga honom ett hus; därför har din tjänare dristat att bedja inför dig.
\par 26 Och nu, HERRE, du är Gud; och då du har lovat din tjänare detta goda,
\par 27 så må du nu ock värdigas välsigna din tjänares hus, så att det förbliver evinnerligen inför dig. Ty vad du, HERRE, välsignar, det är välsignat evinnerligen."

\chapter{18}

\par 1 En tid härefter slog David filistéerna och kuvade dem. Därvid tog han Gat med underlydande orter ur filistéernas hand.
\par 2 Han slog ock moabiterna; så blevo moabiterna David underdåniga och förde till honom skänker.
\par 3 Likaledes slog David Hadareser, konungen i Soba, vid Hamat, när denne hade dragit åstad för att befästa sitt välde vid floden Frat.
\par 4 Och David tog ifrån honom ett tusen vagnar och tog till fånga sju tusen ryttare och tjugu tusen man fotfolk; och David lät avskära fotsenorna på alla vagnshästarna, utom på ett hundra hästar, som han skonade.
\par 5 När sedan araméerna från Damaskus kommo för att hjälpa Hadareser, konungen i Soba, nedgjorde David tjugutvå tusen man av dem.
\par 6 Och David insatte fogdar bland araméerna i Damaskus; och araméerna blevo David underdåniga och förde till honom skänker. Så gav HERREN seger åt David, varhelst han drog fram.
\par 7 Och David tog de gyllene sköldar som Hadaresers tjänare hade burit och förde dem till Jerusalem.
\par 8 Och från Hadaresers städer Tibhat och Kun tog David koppar i stor myckenhet; därav gjorde sedan Salomo kopparhavet, pelarna och kopparkärlen.
\par 9 Då nu Tou, konungen i Hamat, hörde att David hade slagit Hadaresers, konungens i Soba, hela här,
\par 10 sände han sin son Hadoram till konung David för att hälsa honom och lyckönska honom, därför att han hade givit sig i strid med Hadareser och slagit honom; ty Hadareser hade varit Tous fiende. Han sände ock alla slags kärl av guld, silver och koppar.
\par 11 Också dessa helgade konung David åt HERREN, likasom han hade gjort med det silver och guld han hade hemfört från alla andra folk: från edoméerna, moabiterna, Ammons barn, filistéerna och amalekiterna.
\par 12 Och sedan Absai, Serujas son, hade slagit edoméerna i Saltdalen, aderton tusen man,
\par 13 insatte han fogdar i Edom; och alla edoméer blevo David underdåniga. Så gav HERREN seger åt David, varhelst han drog fram.
\par 14 David regerade nu över hela Israel; och han skipade lag och rätt åt allt sitt folk.
\par 15 Joab, Serujas son, hade befälet över krigshären, och Josafat, Ahiluds son, var kansler.
\par 16 Sadok, Ahitubs son, och Abimelek, Ebjatars son, voro präster, och Sausa var sekreterare.
\par 17 Benaja, Jojadas son, hade befälet över keretéerna och peletéerna; men Davids söner voro de förnämste vid konungens sida.

\chapter{19}

\par 1 En tid härefter dog Nahas, Ammons barns konung, och hans son blev konung efter honom.
\par 2 Då sade David: "Jag vill bevisa Hanun, Nahas' son, vänskap, eftersom hans fader bevisade mig vänskap." Och David skickade sändebud för att trösta honom i hans sorg efter fadern. När så Davids tjänare kommo till Ammons barns land, till Hanun, för att trösta honom,
\par 3 sade Ammons barns furstar till Hanun: "Menar du att David därmed att han sänder tröstare till dig vill visa dig att han ärar din fader? Nej, för att undersöka och fördärva och bespeja landet hava hans tjänare kommit till dig."
\par 4 Då tog Hanun Davids tjänare och lät raka dem och skära av deras kläder mitt på, ända uppe vid sätet, och lät dem så gå.
\par 5 Och man kom och berättade för David vad som hade hänt männen; då sände han bud emot dem, ty männen voro ju mycket vanärade. Och konungen lät säga: "Stannen i Jeriko, till dess edert skägg hinner växa ut, och kommen så tillbaka."
\par 6 Då nu Ammons barn insågo att de hade gjort sig förhatliga för David, sände Hanun och Ammons barn ett tusen talenter silver för att leja sig vagnar och ryttare från Aram-Naharaim, från Aram-Maaka och från Soba.
\par 7 De lejde sig trettiotvå tusen vagnar, ävensom hjälp av konungen i Maaka med hans folk; dessa kommo och lägrade sig framför Medeba. Ammons barn församlade sig ock från sina städer och kommo för att strida.
\par 8 När David hörde detta, sände han åstad Joab med hela hären, de tappraste krigarna.
\par 9 Och Ammons barn drogo ut och ställde upp sig till strid vid ingången till staden; men de konungar som hade kommit dit ställde upp sig för sig själva på fältet.
\par 10 Då Joab nu såg att han hade fiender både framför sig och bakom sig, gjorde han ett urval bland allt Israels utvalda manskap och ställde sedan upp sig mot araméerna.
\par 11 Men det övriga folket överlämnade han åt sin broder Absai, och dessa fingo ställa upp sig mot Ammons barn.
\par 12 Och han sade: "Om araméerna bliva mig övermäktiga, så skall du komma mig till hjälp; och om Ammons barn bliva dig övermäktiga, så vill jag hjälpa dig.
\par 13 Var nu vid gott mod; ja, låt oss visa mod i striden för vårt folk och för vår Guds städer. Sedan må HERREN göra vad honom täckes.
\par 14 Därefter ryckte Joab fram med sitt folk till strid mot araméerna, och de flydde för honom.
\par 15 Men när Ammons barn sågo att araméerna flydde, flydde också de för hans broder Absai och begåvo sig in i staden. Då begav sig Joab till Jerusalem.
\par 16 Då alltså araméerna sågo att de hade blivit slagna av Israel, sände de bud att de araméer som bodde på andra sidan floden skulle rycka ut, anförda av Sofak, Hadaresers härhövitsman.
\par 17 När detta blev berättat för David, församlade han hela Israel och gick över Jordan, och då han kom fram till dem, ställde han upp sig i slagordning mot dem; och när David hade ställt upp sig till strid mot araméerna, gåvo dessa sig i strid med honom.
\par 18 Men araméerna flydde undan för Israel, och David dräpte av araméerna manskapet på sju tusen vagnar, så ock fyrtio tusen man fotfolk; härhövitsmannen Sofak dödade han ock.
\par 19 Då, alltså Hadaresers tjänare sågo att de hade blivit slagna av israeliterna, ingingo de fred med David och blevo honom underdåniga. Efter detta ville araméerna icke vidare hjälpa Ammons barn.

\chapter{20}

\par 1 Följande år, vid den tid då konungarna plägade draga i fält, tågade Joab ut med krigshären och härjade Ammons barns land, och kom så och belägrade Rabba, medan David stannade kvar i Jerusalem. Och Joab intog Rabba och förstörde det.
\par 2 Och David tog deras konungs krona från hans huvud, den befanns väga en talent guld och var prydd med en dyrbar sten. Den sattes nu på Davids huvud. Och han förde ut byte från staden i stor myckenhet.
\par 3 Och folket därinne förde han ut och söndersargade dem med sågar och tröskvagnar av järn och med bilor. Så gjorde David mot Ammons barns alla städer. Sedan vände David med allt folket tillbaka till Jerusalem.
\par 4 Därefter uppstod en strid med filistéerna vid Geser; husatiten Sibbekai slog då ned Sippai, en av rafaéernas avkomlingar; så blevo de kuvade.
\par 5 Åter stod en strid med filistéerna; Elhanan, Jaurs son, slog då ned Lami, gatiten Goljats broder, som hade ett spjut vars skaft liknade en vävbom.
\par 6 Åter stod en strid vid Gat. Där var en reslig man som hade sex fingrar och sex tår, tillsammans tjugufyra; han var ock en avkomling av rafaéerna.
\par 7 Denne smädade Israel; då blev han nedgjord av Jonatan, son till Simea, Davids broder.
\par 8 Dessa voro avkomlingar av rafaéerna i Gat; och de föllo för Davids och hans tjänares hand.

\chapter{21}

\par 1 Men Satan trädde upp mot Israel och uppeggade David till att räkna Israel.
\par 2 Då sade David till Joab och till folkets andra hövitsman: "Gån åstad och räknen Israel, från Beer-Seba ända till Dan, och given mig besked därom, så att jag får veta huru många de äro."
\par 3 Joab svarade: "Må HERREN än vidare föröka sitt folk hundrafalt. Äro de då icke, min herre konung, allasammans min herres tjänare? Varför begär då min herre sådant? Varför skulle man därmed draga skuld över Israel?
\par 4 Likväl blev konungens befallning gällande, trots Joab. Alltså drog Joab ut och for omkring i hela Israel, och kom så hem igen till Jerusalem.
\par 5 Och Joab uppgav för David vilken slutsumma folkräkningen utvisade: i Israel funnos tillsammans elva hundra tusen svärdbeväpnade män, och i Juda funnos fyra hundra sjuttio tusen svärdbeväpnade man.
\par 6 Men Levi och Benjamin hade han icke räknat jämte de andra, ty konungens befallning var en styggelse för Joab.
\par 7 Vad som hade skett misshagade Gud, och han hemsökte Israel.
\par 8 Då sade David till Gud: "Jag har syndat storligen däri att jag har gjort detta; men tillgiv nu din tjänares missgärning, ty jag har handlat mycket dåraktigt."
\par 9 Men HERREN talade till Gad, Davids siare, och sade:
\par 10 "Gå och tala till David och säg: Så säger HERREN: Tre ting lägger jag fram för dig; välj bland dem ut åt dig ett som du vill att jag skall göra dig."
\par 11 Då gick Gad in till David och sade till honom: "Så säger HERREN:
\par 12 Tag vilketdera du vill: antingen hungersnöd i tre år, eller förödelse i tre månader genom dina ovänners anfall, utan att du kan undkomma dina fienders svärd, eller HERRENS svärd och pest i landet under tre dagar, i det att HERRENS ängel sprider fördärv inom hela Israels område. Eftersinna nu vilket svar jag skall giva honom som har sänt mig."
\par 13 David svarade Gad: "Jag är i stor vånda. Men låt mig då falla i HERRENS hand, ty hans barmhärtighet är mycket stor; i människohand vill jag icke falla."
\par 14 Så lät då HERREN pest komma i Israel, så att sjuttio tusen män av Israel föllo.
\par 15 Och Gud sände en ängel mot Jerusalem till att fördärva det. Men när denne höll på att fördärva, såg HERREN därtill och ångrade det onda, så att han sade till ängeln, Fördärvaren: "Det är nog; drag nu din hand tillbaka." Och HERRENS ängel stod då vid jebuséen Ornans tröskplats.
\par 16 När nu David lyfte upp sina ögon och fick se HERRENS ängel stående mellan jorden och himmelen med ett blottat svärd i sin hand, uträckt över Jerusalem, då föllo han och de äldste, höljda i sorgdräkt, ned på sina ansikten.
\par 17 Och David sade till Gud: "Det var ju jag som befallde att folket skulle räknas. Det är då jag som har syndat och gjort vad ont är; men dessa, min hjord, vad hava de gjort? HERRE, min Gud, må din hand vända sig mot mig och min faders hus, men icke mot ditt folk, så att det bliver hemsökt."
\par 18 Men HERRENS ängel befallde Gad att säga till David att David skulle gå åstad och resa ett altare åt HERREN på jebuséen Ornans tröskplats.
\par 19 Och David gick åstad på grund av det ord som Gad hade talat i HERRENS namn.
\par 20 Då Ornan nu vände sig om, fick han se ängeln; och hans fyra söner som voro med honom, gömde sig. Men Ornan höll på att tröska vete.
\par 21 Och David kom till Ornan; när då Ornan såg upp och fick se David, gick han fram ifrån tröskplatsen och föll ned till jorden på sitt ansikte för David.
\par 22 Och David sade till Ornan: "Giv mig den plats där du tröskar din säd, så att jag där kan bygga ett altare åt HERREN; giv mig den för full betalning; och må så hemsökelsen upphöra bland folket."
\par 23 Då sade Ornan till David: "Tag den, och må sedan min herre konungen göra vad honom täckes. Se, här giver jag dig fäkreaturen till brännoffer och tröskvagnarna till ved och vetet till spisoffer; alltsammans giver jag."
\par 24 Men konung David svarade Ornan: "Nej, jag vill köpa det för full betalning; ty jag vill icke taga åt HERREN det som är ditt, och offra brännoffer som jag har fått för intet."
\par 25 Och David gav åt Ornan för platsen sex hundra siklar guld, i full vikt.
\par 26 Och David byggde där ett altare åt HERREN och offrade brännoffer och tackoffer. Han ropade till HERREN, och han svarade honom med eld från himmelen på brännoffersaltaret.
\par 27 Och på HERRENS befallning stack ängeln sitt svärd tillbaka i skidan.
\par 28 Då, när David förnam att HERREN hade bönhört honom på jebuséen Ornans tröskplats, offrade han där.
\par 29 Men HERRENS tabernakel, som Mose hade låtit göra i öknen, stod jämte brännoffersaltaret, vid den tiden på offerhöjden i Gibeon.
\par 30 Dock vågade David icke komma inför Guds ansikte för att söka honom; så förskräckt var han för HERRENS ängels svärd.

\chapter{22}

\par 1 Och David sade: "Här skall HERREN Guds hus stå, och här altaret för Israels brännoffer."
\par 2 Och David befallde att man skulle samla tillhopa de främlingar som funnos i Israels land; och han anställde hantverkare, som skulle hugga ut stenar för att därmed bygga Guds hus.
\par 3 Och David anskaffade järn i myckenhet till spikar på dörrarna i portarna och till krampor, så ock koppar i sådan myckenhet att den icke kunde vägas,
\par 4 och cederbjälkar i otalig mängd; ty sidonierna och tyrierna förde cederträ i myckenhet till David.
\par 5 David tänkte nämligen: "Min son Salomo är ung och späd, men huset som skall byggas åt HERREN måste göras övermåttan stort, så att det bliver namnkunnigt och prisat i alla länder; jag vill därför skaffa förråd åt honom." Så skaffade David förråd i myckenhet före sin död.
\par 6 Och han kallade till sig sin son Salomo och bjöd honom att bygga ett hus åt HERREN, Israels Gud.
\par 7 Och David sade till sin son Salomo: "Jag hade själv i sinnet att bygga ett hus åt HERRENS, min Guds, namn.
\par 8 Men HERRENS ord kom till mig; han sade: Du har utgjutit blod i myckenhet och fört stora krig; du skall icke bygga ett hus åt mitt namn, eftersom du har utgjutit så mycket blod på jorden, i min åsyn.
\par 9 Men se, åt dig skall födas en son; han skall bliva en fridsäll man, och jag skall låta honom få fred med alla sina fiender runt omkring; ty Salomo skall han heta, och frid och ro skall jag låta vila över Israel i hans dagar.
\par 10 Han skall bygga ett hus åt mitt namn; han skall vara min son, och jag skall vara hans fader. Och jag skall befästa hans konungatron över Israel för evig tid.
\par 11 Så vare nu HERREN med dig, min son; må du bliva lyckosam och få bygga HERRENS, din Guds, hus, såsom han har lovat om dig.
\par 12 Må HERREN allenast giva dig klokhet och förstånd, när han sätter dig till härskare över Israel, och förhjälpa dig till att hålla HERRENS, din Guds, lag.
\par 13 Då skall du bliva lyckosam, om du håller och gör efter de stadgar och rätter som HERREN har bjudit Mose att ålägga Israel. Var frimodig och oförfärad; frukta icke och var icke försagd.
\par 14 Och se, trots mitt betryck har jag nu anskaffat till HERRENS hus ett hundra tusen talenter guld och tusen gånger tusen talenter silver, därtill av koppar och järn mer än som kan vägas, ty så mycket är det; trävirke och sten har jag ock anskaffat, och mer må du själv anskaffa.
\par 15 Arbetare har du ock i myckenhet hantverkare, stenhuggare och timmermän, och därtill allahanda folk som är kunnigt i allt slags annat arbete.
\par 16 På guldet, silvret, kopparen och järnet kan ingen räkning hållas. Upp då och gå till verket; och vare HERREN med dig!"
\par 17 Därefter bjöd David alla Israels furstar att de skulle understödja hans son Salomo; han sade:
\par 18 "HERREN, eder Gud, är ju med eder och har låtit eder få ro på alla sidor; ty han har givit landets förra inbyggare i min hand, och landet har blivit HERREN och hans folk underdånigt.
\par 19 Så vänden nu edert hjärta och eder själ till att söka HERREN, eder Gud; och stån upp och byggen HERREN Guds helgedom, så att man kan föra HERRENS förbundsark och vad annat som hör till Guds helgedom in i det hus som skall byggas åt HERRENS namn."

\chapter{23}

\par 1 Och när David blev gammal och levnadsmätt, gjorde han sin son Salomo till konung över Israel.
\par 2 Och han församlade alla Israels furstar, så ock prästerna och leviterna.
\par 3 Och leviterna blevo räknade, de nämligen som voro trettio år gamla eller därutöver; och deras antal, antalet av alla personer av mankön, utgjorde trettioåtta tusen.
\par 4 "Av dessa", sade han, "skola tjugufyra tusen förestå sysslorna vid HERRENS hus, och sex tusen vara tillsyningsmän och domare;
\par 5 fyra tusen skola vara dörrvaktare och fyra tusen skola lovsjunga HERREN till de instrumenter som jag har låtit göra för lovsången."
\par 6 Och David delade dem i avdelningar efter Levis söner, Gerson Kehat och Merari.
\par 7 Till gersoniterna hörde Laedan och Simei.
\par 8 Laedans söner voro Jehiel, huvudmannen, Setam och Joel, tillsammans tre.
\par 9 Simeis söner voro Selomot, Hasiel och Haran, tillsammans tre. Dessa voro huvudmän för Laedans familjer.
\par 10 Och Simeis söner voro Jahat, Sina, Jeus och Beria. Dessa voro Simeis söner, tillsammans fyra.
\par 11 Jahat var huvudmannen, och Sisa var den andre. Men Jeus och Beria hade icke många barn; därför fingo de utgöra allenast en familj, en ordning.
\par 12 Kehats söner voro Amram, Jishar, Hebron och Ussiel, tillsammans fyra.
\par 13 Amrams söner voro Aron och Mose. Och Aron blev jämte sina söner för evärdlig tid avskild till att helgas såsom höghelig, till att för evärdlig tid antända rökelse inför HERREN och göra tjänst inför honom och välsigna i hans namn.
\par 14 Men gudsmannen Moses söner räknades till Levi stam.
\par 15 Moses söner voro Gersom och Elieser.
\par 16 Gersoms söner voro Sebuel, huvudmannen.
\par 17 Och Eliesers söner voro Rehabja, huvudmannen. Elieser hade inga andra söner; men Rehabjas söner voro övermåttan talrika.
\par 18 Jishars söner voro Selomit, huvudmannen.
\par 19 Hebrons söner voro Jeria, huvudmannen, Amarja, den andre, Jahasiel, den tredje, och Jekameam, den fjärde.
\par 20 Ussiels söner voro Mika, huvudmannen, och Jissia, den andre.
\par 21 Meraris söner voro Maheli och Musi. Mahelis söner voro Eleasar och Kis.
\par 22 När Eleasar dog, lämnade han inga söner efter sig, utan allenast döttrar; men Kis' söner, deras fränder, togo dessa till hustrur.
\par 23 Musis söner voro Maheli, Eder och Jeremot, tillsammans tre.
\par 24 Dessa voro Levi barn, efter deras familjer, huvudmännen för familjerna, så många av dem som inmönstrades, vart namn räknat särskilt, var person för sig, de som kunde förrätta sysslor vid tjänstgöringen i HERRENS hus, nämligen de som voro tjugu år gamla eller därutöver.
\par 25 Ty David sade: "HERREN, Israels Gud, har låtit sitt folk komma till ro, och han har nu sin boning i Jerusalem till evig tid;
\par 26 därför behöva icke heller leviterna mer bära tabernaklet och alla redskap till tjänstgöringen därvid."
\par 27 (Enligt berättelsen om Davids sista tid räknades nämligen av Levi barn de som voro tjugu år gamla eller därutöver.)
\par 28 De fingo i stället sin plats vid Arons söners sida för tjänstgöringen i HERRENS hus, i vad som rörde förgårdarna och kamrarna och reningen av allt heligt och sysslorna vid tjänstgöringen i Guds hus,
\par 29 vare sig det gällde skådebröden eller det fina mjölet till spisoffret eller de osyrade tunnkakorna eller plåtarna eller det hopknådade mjölet, eller något mått och mål,
\par 30 eller att var morgon göra tjänst genom att tacka och lova HERREN, och likaledes var afton,
\par 31 eller att offra alla brännoffer åt HERREN på sabbaterna, vid nymånaderna och vid högtiderna, till bestämt antal och såsom det var föreskrivet för dem, beständigt, inför HERRENS ansikte.
\par 32 De skulle iakttaga vad som var att iakttaga vid uppenbarelsetältet och vid det heliga, vad Arons söner, deras bröder, hade att iakttaga vid tjänstgöringen i HERRENS hus.

\chapter{24}

\par 1 Och Arons söner hade följande avdelningar: Arons söner voro Nadab och Abihu, Eleasar och Itamar.
\par 2 Men Nadab och Abihu dogo före sin fader; och de hade inga söner. Så blevo allenast Eleasar och Itamar präster.
\par 3 Och David jämte Sadok, av Eleasars söner, och Ahimelek, av Itamars söner, indelade dem och bestämde den ordning i vilken de skulle göra tjänst.
\par 4 Då nu Eleasars söner befunnos hava flera huvudmän än Itamars söner, indelade man dem så, att Eleasars söner fingo sexton huvudmän för sina familjer och Itamars söner åtta huvudmän för sina familjer.
\par 5 Man indelade dem genom lottkastning, de förra såväl som de senare, ty helgedomens furstar och Guds furstar togos både av Eleasars söner och av Itamars söner.
\par 6 Och Semaja, Netanels son, sekreteraren, av Levi stam, tecknade upp dem i närvaro av konungen, furstarna och prästen Sadok och Ahimelek, Ebjatars son, och i närvaro av huvudmännen för prästernas och leviternas familjer. Lotterna drogos skiftevis för Eleasars och för Itamars familjer.
\par 7 Den första lotten föll ut för Jojarib, den andra för Jedaja,
\par 8 den tredje för Harim, den fjärde för Seorim,
\par 9 den femte för Malkia, den sjätte för Mijamin,
\par 10 den sjunde för Hackos, den åttonde för Abia,
\par 11 den nionde för Jesua, den tionde för Sekanja,
\par 12 den elfte för Eljasib, den tolfte för Jakim,
\par 13 den trettonde för Huppa, den fjortonde för Jesebab,
\par 14 den femtonde för Bilga, den sextonde för Immer,
\par 15 den sjuttonde för Hesir, den adertonde för Happisses,
\par 16 den nittonde för Petaja, den tjugonde för Hesekiel,
\par 17 den tjuguförsta för Jakin, den tjuguandra för Gamul,
\par 18 den tjugutredje for Delaja, den tjugufjärde för Maasja.
\par 19 Detta blev den ordning i vilken de skulle göra tjänst, när de gingo in i HERRENS hus, såsom det var föreskrivet för dem genom deras fader Aron, i enlighet med vad HERREN, Israels Gud, hade bjudit honom.
\par 20 Vad angår de övriga Levi barn, så hörde till Amrams barn Subael, till Subaels barn Jedeja,
\par 21 till Rehabja, det är till Rehabjas barn, huvudmannen Jissia,
\par 22 till jishariterna Selomot, till Selomots barn Jahat.
\par 23 Och benajiter voro Jeria, Amarja, den andre, Jahasiel, den tredje, och Jekameam, den fjärde.
\par 24 Ussiels barn voro Mika; till Mikas barn hörde Samur.
\par 25 Mikas broder var Jissia; till Jissias barn hörde Sakarja.
\par 26 Meraris barn voro Maheli och Musi, Jaasia-Benos söner.
\par 27 Meraris barn voro dessa av Jaasia-Beno, och vidare Soham, Sackur och Ibri.
\par 28 Mahelis son var Eleasar, men denne hade inga söner.
\par 29 Till Kis, det är Kis' barn, hörde Jerameel.
\par 30 Men Musis barn voro Maheli, Eder och Jerimot. Dessa voro leviternas barn, efter deras familjer.
\par 31 Också dessa kastade lott likasåväl som deras bröder, Arons söner, i närvaro av konung David, Sadok, Ahimelek och huvudmännen för prästernas och leviternas familjer, huvudmännen för familjerna likasåväl som deras yngsta bröder.

\chapter{25}

\par 1 Och David jämte härhövitsmännen avskilde till tjänstgöring Asafs, Hemans och Jedutuns söner, som hade profetisk anda till att spela på harpor, psaltare och cymbaler. Och detta är förteckningen på dem, på de män som fingo denna tjänstgöring till åliggande.
\par 2 Av Asafs söner: Sackur, Josef, Netanja och Asarela, Asafs söner, under ledning av Asaf, som hade profetisk anda till att spela, under konungens ledning.
\par 3 Av Jedutun: Jedutuns söner Gedalja, Seri, Jesaja, Hasabja och Mattitja, tillsammans sex, med harpor, under ledning av sin fader Jedutun, som hade profetisk anda till att spela tack- och lovsånger till HERREN.
\par 4 Av Heman: Hemans söner Buckia, Mattanja, Ussiel, Sebuel och Jerimot, Hananja, Hanani, Eliata, Giddalti och Romamti-Eser, Josbekasa, Malloti, Hotir, Mahasiot.
\par 5 Alla dessa voro söner till Heman, som var konungens siare, enligt det löfte Gud hade givit, att han ville upphöja hans horn; därför gav Gud Heman fjorton söner och tre döttrar.
\par 6 Alla dessa stodo var och en under sin faders ledning, när de utförde sången i HERRENS hus till cymbaler, psaltare och harpor och så gjorde tjänst i Guds hus; de stodo under konungens, Asafs, Jedutuns och Hemans ledning.
\par 7 Och antalet av dem jämte deras bröder, av dem som hade blivit undervisade i sången till HERREN ära, alla de däri kunniga, utgjorde två hundra åttioåtta.
\par 8 Och de kastade lott om tjänstgöringen, alla, den minste likasåväl som den störste, den kunnige jämte lärjungen.
\par 9 Den första lotten kom ut för Asaf och föll på Josef; den andre blev Gedalja, han själv med sina bröder och söner, tillsammans tolv;
\par 10 den tredje blev Sackur, med sin söner och bröder, tillsammans tolv
\par 11 den fjärde lotten kom ut för Jisri, med hans söner och bröder, tillsammans tolv;
\par 12 den femte blev Netanja, med sina söner och bröder, tillsammans tolv;
\par 13 den sjätte blev Buckia, med sina söner och bröder, tillsammans tolv;
\par 14 den sjunde blev Jesarela, med sina söner och bröder, tillsammans tolv;
\par 15 den åttonde blev Jesaja, med sin söner och bröder, tillsammans tolv
\par 16 den nionde blev Mattanja, med sina söner och bröder, tillsammans tolv;
\par 17 den tionde blev Simei, med sina söner och bröder, tillsammans tolv
\par 18 den elfte blev Asarel, med sin söner och bröder, tillsammans tolv
\par 19 den tolfte lotten kom ut för Hasabja, med hans söner och bröder tillsammans tolv;
\par 20 den trettonde blev Subael, med sina söner och bröder, tillsammans tolv;
\par 21 den fjortonde blev Mattitja, med sina söner och bröder, tillsammans tolv;
\par 22 den femtonde lotten kom ut för Jeremot, med hans söner och bröder, tillsammans tolv;
\par 23 den sextonde för Hananja, med hans söner och bröder, tillsammans tolv;
\par 24 den sjuttonde för Josbekasa, med hans söner och bröder, tillsammans tolv;
\par 25 den adertonde för Hanani, med hans söner och bröder, tillsammans tolv;
\par 26 den nittonde för Malloti, med hans söner och bröder, tillsammans tolv;
\par 27 den tjugonde för Elijata, med hans söner och bröder, tillsammans tolv;
\par 28 den tjuguförsta för Hotir, med hans söner och bröder, tillsammans tolv;
\par 29 den tjuguandra för Giddalti, med hans söner och bröder, tillsammans tolv;
\par 30 den tjugutredje för Mahasiot, med hans söner och bröder, tillsammans tolv;
\par 31 den tjugufjärde för Romamti-Eser, med hans söner och bröder, tillsammans tolv.

\chapter{26}

\par 1 Vad angår dörrvaktarnas avdelningar, så hörde till koraiterna Meselemja, Kores son, av Asafs barn.
\par 2 Och Meselemja hade söner: Sakarja var den förstfödde, Jediael den andre, Sebadja den tredje, Jatniel den fjärde,
\par 3 Elam den femte, Johanan den sjätte, Eljoenai den sjunde.
\par 4 Och Obed-Edom hade söner: Semaja var den förstfödde, Josabad den andre, Joa den tredje, Sakar den fjärde, Netanel den femte,
\par 5 Ammiel den sjätte, Isaskar den sjunde, Peulletai den åttonde; ty Gud hade välsignat honom.
\par 6 Åt hans son Semaja föddes ock söner, som blevo furstar inom sin familj, ty de voro dugande män.
\par 7 Semajas söner voro Otni, Refael och Obed, Elsabad och hans bröder, dugliga män, Elihu och Semakja.
\par 8 Alla dessa hörde till Obed-Edoms avkomlingar, de själva och deras söner och bröder, dugliga och kraftfulla män i tjänsten, tillsammans sextiotvå avkomlingar av Obed-Edom.
\par 9 Meselemja hade ock söner och bröder, dugliga män, tillsammans aderton.
\par 10 Och Hosa, av Meraris barn, hade söner: Simri var huvudmannen, ty visserligen var han icke förstfödd, men hans fader insatte honom till huvudman;
\par 11 Hilkia var den andre, Tebalja den tredje, Sakarja den fjärde. Hosas söner och bröder voro tillsammans tretton.
\par 12 Dessa avdelningar av dörrvaktarna, nämligen dessa deras huvudmän, fingo nu, likasåväl som deras bröder, sina åligganden för att göra tjänst i HERRENS hus.
\par 13 Och om var port kastade de lott, den minste såväl som den störste, efter sina familjer.
\par 14 Den lott som angav öster föll då på Selemja; och för hans son Sakarja, en rådklok man, kastade man lott, och för honom kom ut den lott som angav norr;
\par 15 för Obed-Edom den lott som angav söder, under det att hans söner fingo på sin del förrådshuset;
\par 16 för Suppim och för Hosa den lott som angav platsen västerut, vid Salleketporten, där vägen höjer sig uppåt, det ena vaktstället invid det andra.
\par 17 Österut voro sex leviter, norrut fyra för var dag, söderut fyra för var dag, och vid förrådshuset två i sänder;
\par 18 vid Parbar västerut voro fyra vid vägen och två vid själva Parbar.
\par 19 Dessa voro dörrvaktarnas avdelningar, av koraiternas barn och av Meraris barn.
\par 20 Och av leviterna hade Ahia uppsikten över Guds hus skatter och vården om de förråd som utgjordes av vad som hade blivit helgat åt HERREN.
\par 21 Laedans barn, nämligen gersoniternas barn av Laedans släkt, huvudmannen för gersoniten Laedans familj, jehieliterna,
\par 22 det är jehieliternas barn, Setam och hans broder Joel, hade uppsikten över skatterna i HERRENS hus.
\par 23 Vad angår amramiterna, jishariterna, hebroniterna och ossieliterna,
\par 24 så var Sebuel, son till Gersom, son till Mose, överuppsyningsman över skatterna.
\par 25 Och hans bröder av Eliesers släkt voro dennes son Rehabja, dennes son Jesaja, dennes son Joram, dennes son Sikri och dennes son Selomot.
\par 26 Denne Selomot och hans bröder hade uppsikten över alla förråd som utgjordes av vad som hade blivit helgat åt HERREN av konung David, så ock av huvudmännen för familjerna, ävensom av över- och underhövitsmännen och av härhövitsmännen.
\par 27 Från krigen och av bytet hade de helgat detta för att hålla HERRENS hus vid makt;
\par 28 likaledes allt vad siaren Samuel och Saul, Kis' son, och Abner, Ners son, och Joab, Serujas son, hade helgat - korteligen, var och en som helgade något lämnade det under Selomits och hans bröders vård.
\par 29 Av jishariterna togos Kenanja och hans söner till de världsliga sysslorna i Israel, till att vara tillsyningsmän och domare.
\par 30 Av hebroniterna togos Hasabja och hans bröder, dugliga män, ett tusen sju hundra, till ämbetsförvaltningen i Israel på andra sidan Jordan, på västra sidan, till alla slags sysslor åt HERREN och till konungens tjänst.
\par 31 För hebroniterna var Jeria huvudman, för hebroniterna efter deras ättföljd och familjer. (I Davids fyrtionde regeringsår anställdes undersökning rörande dem; och bland dem funnos då dugande män i Jaeser i Gilead.)
\par 32 Hans bröder, dugliga män, vore två tusen sju hundra, huvudmän för familjer. Dem satte konung David över rubeniterna, gaditerna och ena hälften av Manasse stam, för att ombesörja alla Guds och konungens angelägenheter.

\chapter{27}

\par 1 Och detta är förteckningen på Israels barn, efter deras antal med huvudmännen för deras familjer och med över- och underhövitsmännen och med deras tillsyningsmän vilka tjänade konungen i allt som rörde krigsfolkets avdelningar, vilka avdelningar kommo och avgingo skiftevis för var och en av årets alla månader, var avdelning tjugufyra tusen man stark.
\par 2 Över den första avdelningen, den som tjänstgjorde under första månaden, hade Jasobeam, Sabdiels son, befälet. Och i hans avdelning voro tjugufyra tusen.
\par 3 Han hörde till Peres' barn och var huvudanförare för alla härhövitsmän som tjänstgjorde under första månaden.
\par 4 Över den andra månadens avdelning hade ahoaiten Dodai befälet, det var hans avdelning; där var ock fursten Miklot. Och i hans avdelning voro tjugufyra tusen.
\par 5 Den tredje härhövitsmannen, den som tjänstgjorde under tredje månaden, var Benaja, prästen Jojadas son, såsom huvudanförare. Och i hans avdelning voro tjugufyra tusen.
\par 6 Denne Benaja var en hjälte bland de trettio och hade befälet över de trettio. Och vid hans avdelning var hans son Ammisabad.
\par 7 Den fjärde, den som tjänstgjorde under fjärde månaden, var Asael, Joabs broder, och efter honom hans son Sebadja. Och i hans avdelning voro tjugufyra tusen
\par 8 Den femte, den som tjänstgjorde under femte månaden, var hövitsmannen Samhut, jisraiten. Och i hans avdelning voro tjugufyra tusen.
\par 9 Den sjätte, den som tjänstgjorde under sjätte månaden, var tekoaiten Ira, Ickes' son. Och i hans avdelning voro tjugufyra tusen.
\par 10 Den sjunde, den som tjänstgjorde under sjunde månaden, var peloniten Heles, av Efraims barn. Och i hans avdelning voro tjugufyra tusen.
\par 11 Den åttonde, den som tjänstgjorde under åttonde månaden, var husatiten Sibbekai, som hörde till seraiterna. Och i hans avdelning voro tjugufyra tusen.
\par 12 Den nionde, den som tjänstgjorde under nionde månaden, var anatotiten Abieser, som hörde till benjaminiterna. Och i hans avdelning voro tjugufyra tusen.
\par 13 Den tionde, den som tjänstgjorde under tionde månaden, var netofatiten Maherai, som hörde till seraiterna. Och i hans avdelning voro tjugufyra tusen.
\par 14 Den elfte, den som tjänstgjorde under elfte månaden, var pirgatoniten Benaja, av Efraims barn. Och i hans avdelning voro tjugufyra tusen.
\par 15 Den tolfte, den som tjänstgjorde under tolfte månaden, var netofatiten Heldai, som hörde till Otniels släkt. Och i hans avdelning voro tjugufyra tusen.
\par 16 Och Israels stamhövdingar voro dessa: furste för rubeniterna var Elieser, Sikris son; för simeoniterna Sefatja, Maakas son;
\par 17 för Levi Hasabja, Kemuels son; för Arons släkt Sadok;
\par 18 för Juda Elihu, en av Davids bröder; för Isaskar Omri, Mikaels son;
\par 19 för Sebulon Jismaja, Obadjas son; för Naftali Jerimot, Asriels son;
\par 20 för Efraims barn Hosea, Asasjas son; för ena hälften av Manasse stam Joel, Pedajas son;
\par 21 för andra hälften av Manasse, den i Gilead, Jiddo, Sakarjas son; för Benjamin Jaasiel, Abners son;
\par 22 för Dan Asarel, Jerohams son. Dessa voro Israels stamhövdingar.
\par 23 Men David tog i förteckningen icke upp dem som voro under tjugu år, ty HERREN hade lovat att han ville föröka Israel såsom stjärnorna på himmelen.
\par 24 Joab, Serujas son, begynte räkningen, men fullbordade den icke, ty genom den kom förtörnelse över Israel; och antalet togs icke upp i någon förteckning i konung Davids krönika.
\par 25 Uppsikten över konungens skatter hade Asmavet, Adiels son; över förråden på fälten, i städerna och byarna och fästningstornen Jonatan, Ussias son;
\par 26 över dem som arbetade på fältet med jordbruket Esri, Kelubs son;
\par 27 över vingårdarna ramatiten Simei; över de vinförråd som man hade samlat i vingårdarna sifmiten Sabdi;
\par 28 över olivplanteringarna och mullbärsfikonträden i Låglandet gaderiten Baal-Hanan; över oljeförråden Joas.
\par 29 Över de fäkreatur som betade i Saron saroniten Sitrai, och över fäkreaturen i dalarna Safat, Adlais son;
\par 30 över kamelerna ismaeliten Obil; över åsninnorna meronotiten Jedeja;
\par 31 över småboskapen hagariten Jasis. Alla dessa voro uppsyningsmän över konung Davids ägodelar.
\par 32 Men Jonatan, Davids farbroder, var rådgivare; han var en förståndig och skriftlärd man. Jehiel, Hakmonis son, var anställd hos konungens söner.
\par 33 Ahitofel var konungens rådgivare, och arkiten Husai var konungens vän.
\par 34 Efter Ahitofel kom Jojada, Benajas son, och Ebjatar. Och Joab var konungens härhövitsman.

\chapter{28}

\par 1 Och David församlade till Jerusalem alla Israels hövdingar, stamhövdingarna och häravdelningarnas hövitsmän, dem som voro i konungens tjänst, och över- och underhövitsmännen och uppsyningsmännen över alla konungens och hans söners ägodelar och boskap, så ock hovmännen och hjältarna och alla tappra stridsmän.
\par 2 Och konung David stod upp från sin plats och sade: "Hören mig, mina bröder och mitt folk. Jag hade själv i sinnet att bygga ett hus till vilostad för HERRENS förbundsark och för vår Guds fotapall, och jag hade skaffat förråd till byggnadsverket.
\par 3 Men Gud sade till mig: 'Du skall icke bygga ett hus åt mitt namn, ty du är en krigsman och har utgjutit blod.'
\par 4 Dock utvalde HERREN, Israels Gud mig ur hela min faders hus till att vara konung över Israel evärdligen. Ty Juda utvalde han till furste, och i Juda hus min faders hus, och bland min faders söner hade han behag till mig, så att han gjorde mig till konung över hela Israel.
\par 5 Och bland alla mina söner - ty HERREN har givit mig många söner - utvalde han min son Salomo till att sitta på HERRENS konungatron och härska över Israel.
\par 6 Och han sade till mig: 'Din son Salomo är den som skall bygga mitt hus och mina förgårdar; ty honom har jag utvalt till min son, och jag skall vara hans fader.
\par 7 Och jag skall befästa hans konungamakt för evigt, om han är ståndaktig i att göra efter mina bud och rätter, såsom han nu gör.'
\par 8 Och nu säger jag inför hela Israel, HERRENS församling, och inför vår Gud, som hör det: Hållen och akten på alla HERRENS, eder Guds, bud, så att I fån besitta det goda landet och lämna det såsom arv åt edra barn efter eder till evärdlig tid.
\par 9 Och du, min son Salomo, må lära känna din faders Gud och tjäna honom med hängivet hjärta och med villig själ; ty HERREN rannsakar alla hjärtan och förstår alla uppsåt och tankar. Om du söker honom, så låter han sig finnas av dig, men om du övergiver honom, då förkastar han dig evinnerligen.
\par 10 Så se nu till; ty HERREN har utvalt dig att bygga ett hus till helgedomen. Var frimodig och gå till verket."
\par 11 Och David gav åt sin son Salomo en mönsterbild av förhuset och tempelbyggnaderna, och av förrådskamrarna, de övre salarna och de inre rummen, och av nådastolens boning;
\par 12 vidare en mönsterbild av allt som han hade tänkt ut i sitt sinne rörande förgårdarna till HERRENS hus, och rörande alla kamrarna runt omkring för Guds hus' skatter och för de förråd som utgjordes av vad som hade blivit helgat åt HERREN;
\par 13 vidare föreskrifter rörande prästernas och leviternas avdelningar och alla sysslor som skulle förekomma vid tjänstgöringen i HERRENS hus, och rörande alla kärl som skulle användas vid tjänstgöringen i HERRENS hus,
\par 14 och rörande guldet, med uppgift på den vikt i guld, som kom på vart särskilt kärl till tjänstgöringen, och rörande alla kärl av silver, med uppgift på den vikt som kom på vart särskilt kärl till tjänstgöringen.
\par 15 Och han angav vikten på de gyllene ljusstakarna med tillhörande lampor av guld, med uppgift på vikten i var särskild ljusstake med dess lampor, så ock rörande silverljusstakarna, med uppgift på vikten i var ljusstake med dess lampor, alltefter beskaffenheten av den tjänstförrättning vid vilken ljusstaken skulle användas;
\par 16 likaledes rörande vikten på guldet till skådebrödsborden, vart bord för sig, och rörande silvret till silverborden.
\par 17 Och han gav honom föreskrifter rörande gafflarna och skålarna och kannorna av rent guld, och rörande de gyllene bägarna, med uppgift på vikten i var särskild bägare, och rörande silverbägarna, med uppgift på vikten i var särskild bägare;
\par 18 likaså rörande rökelsealtaret av rent guld, med uppgift på vikten; så ock en mönsterbild av vagnen, de gyllene keruberna, som skulle breda ut sina vingar och övertäcka HERRENS förbundsark.
\par 19 "Om alltsammans", sade han, "har HERREN undervisat mig genom en skrift av sin hand, om allt som skall utföras enligt mönsterbilden."
\par 20 Och David sade till sin son Salomo: "Var frimodig och oförfärad och gå till verket; frukta icke och var icke försagd. Ty HERREN Gud, min Gud, skall vara med dig. Han skall icke lämna dig och icke övergiva dig, till dess att allt som skall utföras för tjänstgöringen i HERRENS hus har blivit fullbordat.
\par 21 Och se, här äro prästernas och leviternas avdelningar, som skola förrätta allt slags tjänst i Guds hus. Och till allt som skall utföras har du hos dig allahanda villigt folk, utrustat med vishet till allt slags arbete; därjämte äro hövdingarna och allt folket redo till allt vad du befaller."

\chapter{29}

\par 1 Och konung David sade till hela församlingen: "Min son Salomo den ende som Gud har utvalt, är ung och späd, och arbetet är stort, ty denna borg är icke avsedd för en människa, utan för HERREN Gud.
\par 2 Därför har jag, så vitt jag har förmått, för min Guds hus anskaffat guld till det som skall vara av guld, silver till det som skall vara av silver, koppar till det som skall vara av koppar, järn till det som skall vara av järn, och trä till det som skall vara av trä, dessutom onyxstenar och andra infattningsstenar, svartglänsande och brokiga stenar, korteligen, alla slags dyrbara stenar, så ock marmor i myckenhet.
\par 3 Och därjämte, eftersom jag har min Guds hus kärt, giver jag nu vad jag själv äger i guld och silver till min Guds hus, utöver allt vad jag förut har anskaffat för det heliga huset:
\par 4 tre tusen talenter guld, guld från Ofir, och sju tusen talenter renat silver till att därmed överdraga byggnadernas väggar,
\par 5 till att göra av guld vad som skall vara av guld, och till att göra av silver vad som skall vara av silver, ja, till allt slags arbete som utföres av konstnärer. Vill då någon annan nu i dag frivilligt fylla sin hand med gåvor åt HERREN?"
\par 6 Då kommo frivilligt familjehövdingarna och Israels stamhövdingar, så ock över- och underhövitsmännen och tillika uppsyningsmännen över konungens arbeten,
\par 7 och de gåvo till arbetet på Guds hus fem tusen talenter guld, tio tusen dariker, tio tusen talenter silver, aderton tusen talenter koppar och ett hundra tusen talenter järn.
\par 8 Och var och en som hade ädla stenar i sin ägo gav dem till skatten i HERRENS hus, under gersoniten Jehiels vård.
\par 9 Då gladde sig folket över deras frivilliga gåvor, ty av hängivet hjärta buro de fram sina frivilliga gåvor åt HERREN; konung David gladde sig ock högeligen.
\par 10 Och David lovade HERREN inför hela församlingen; David sade: "Lovad vare du, HERRE, vår fader Israels Gud, från evighet till evighet!
\par 11 Dig, HERRE, tillhör storhet och makt och härlighet och glans och majestät, ja, allt vad i himmelen och på jorden är. Ditt, o HERRE, är riket, och du har upphöjt dig till ett huvud över allt.
\par 12 Rikedom och ära komma från dig, du råder över allt, och i din hand är kraft och makt; det står i din hand att göra vad som helst stort och starkt.
\par 13 Så tacka vi dig nu, vår Gud, och lova ditt härliga namn.
\par 14 Ty vad är väl jag, och vad är mitt folk, att vi själva skulle förmå att giva sådana frivilliga gåvor? Nej, från dig kommer allt, och ur din hand hava vi givit det åt dig.
\par 15 Ty vi äro främlingar hos dig och gäster såsom alla våra fäder; såsom en skugga äro våra dagar på jorden, och intet är här att lita på.
\par 16 HERRE, vår Gud, alla dessa håvor som vi hava anskaffat för att bygga dig ett hus åt ditt heliga namn - från din hand hava de kommit, och ditt är alltsammans.
\par 17 Och jag vet, min Gud, att du prövar hjärtat och har behag till vad rätt är. Med rättsinnigt hjärta har jag burit fram alla dessa frivilliga gåvor; och nu har jag ock sett med glädje huru ditt folk, som står har, har burit fram åt dig sina frivilliga gåvor.
\par 18 HERRE, Abrahams, Isaks och Israels, våra fäders, Gud, låt evinnerligen ditt folks hjärtas håg och tankar vara redo till sådant, och vänd deras hjärtan till dig.
\par 19 Och giv min son Salomo ett hängivet hjärta, så att han håller dina bud, dina vittnesbörd och dina stadgar, och utför allt detta och bygger denna borg, vartill jag har skaffat förråd."
\par 20 Därefter sade David till hela församlingen: "Loven HERREN, eder Gud." Då lovade hela församlingen HERREN, sina fäders Gud, och de bugade sig och föllo ned för HERREN och för konungen.
\par 21 Och dagen efter denna dag slaktade de slaktoffer åt HERREN och offrade brännoffer åt HERREN: tusen tjurar, tusen vädurar och tusen lamm med tillhörande drickoffer, därtill slaktoffer i myckenhet för hela Israel.
\par 22 Och de åto och drucko inför HERRENS ansikte på den dagen med stor glädje. Och de gjorde för andra gången Salomo, Davids son, till konung; de smorde honom till en HERRENS furste, och Sadok till präst.
\par 23 Och så satt Salomo på HERRENS tron såsom konung efter sin fader David, och han blev lyckosam; och hela Israel lydde honom.
\par 24 Och alla hövdingarna och hjältarna och därjämte alla konung Davids söner underkastade sig konung Salomo.
\par 25 Och HERREN gjorde Salomo övermåttan stor inför hela Israel, och lät hans konungsliga härlighet bliva större än någons som före honom hade varit konung över Israel.
\par 26 Men David, Isais son, hade regerat över hela Israel.
\par 27 Den tid han regerade över Israel var fyrtio år; i Hebron regerade han i sju år, och i Jerusalem regerade han i trettiotre år.
\par 28 Och han dog i en god ålder, mätt på att leva och mätt på rikedom och ära. Och hans son Salomo blev konung efter honom.
\par 29 Och vad som är att säga om konung David, om hans första tid såväl som om hans sista, det finnes upptecknat i siaren Samuels krönika, i profeten Natans krönika och i siaren Gads krönika,
\par 30 tillika med hela hans regering och hans bedrifter och de skickelser som övergingo honom och Israel och alla andra länder och riken.


\end{document}