\begin{document}

\title{2 Chronicles}

2Ch 1:1  Salomo, Davids son, befäste sig nu i sin konungamakt, i det att HERREN, hans Gud, var med honom och gjorde honom övermåttan stor.
2Ch 1:2  Och sedan Salomo hade låtit kallelse utgå till hela Israel, till över- och underhövitsmännen, till domarna och till alla hövdingar i hela Israel, huvudmännen för familjerna,
2Ch 1:3  begav han sig med hela denna församling till offerhöjden i Gibeon, ty där stod Guds uppenbarelsetält, som HERRENS tjänare Mose hade gjort i öknen.
2Ch 1:4  Guds ark däremot hade David hämtat från Kirjat-Jearim upp till den plats som David hade berett åt den, ty han hade åt den slagit upp ett tält i Jerusalem.
2Ch 1:5  Men kopparaltaret, som Besalel, son till Uri, son till Hur, hade gjort, det hade man ställt upp framför HERRENS tabernakel; och Salomo och församlingen gingo dit för att fråga honom.
2Ch 1:6  Där offrade nu Salomo inför HERRENS ansikte på kopparaltaret, som stod vid uppenbarelsetältet; han offrade på det tusen brännoffer.
2Ch 1:7  Och om natten uppenbarade sig Gud för Salomo; han sade till honom: "Bed mig om vad du vill att jag skall giva dig."
2Ch 1:8  Salomo svarade Gud: "Du har gjort stor nåd med min fader David och har låtit mig bliva konung efter honom.
2Ch 1:9  Så låt nu, HERRE Gud, ditt ord till min fader David visa sig vara sant; ty du har själv gjort mig till konung över ett folk som är så talrikt som stoftet på jorden.
2Ch 1:10  Giv mig nu vishet och förstånd till att vara detta folks ledare och anförare; ty vem skulle eljest kunna vara domare för detta ditt stora folk?"
2Ch 1:11  Då sade Gud till Salomo: "Eftersom du är så till sinnes, och icke har bett om rikedom, skatter och ära eller om dina ovänners liv, och ej heller bett om långt liv, utan har bett om vishet och förstånd, så att du kan vara domare för mitt folk, över vilket jag har gjort dig till konung,
2Ch 1:12  därför vare vishet och förstånd dig givna; därtill vill jag ock giva dig rikedom och skatter och ära, så att ingen konung före dig har haft och ej heller någon efter dig skall hava så mycket därav."
2Ch 1:13  Sedan nu Salomo hade varit vid offerhöjden i Gibeon, begav han sig från uppenbarelsetältet till Jerusalem och regerade där över Israel.
2Ch 1:14  Och Salomo samlade vagnar och ridhästar, så att han hade ett tusen fyra hundra vagnar och tolv tusen ridhästar; dem förlade han dels i vagnsstäderna, dels i Jerusalem, hos konungen själv.
2Ch 1:15  Och konungen styrde så, att silver och guld blev lika vanligt i Jerusalem som stenar, och cederträ lika vanligt som mullbärsfikonträ i Låglandet.
2Ch 1:16  Och hästarna som Salomo lät anskaffa infördes från Egypten; ett antal kungliga uppköpare hämtade ett visst antal av dem till bestämt pris.
2Ch 1:17  Var vagn som de hämtade upp från Egypten och införde kostade sex hundra siklar silver, och var häst ett hundra femtio. Sammalunda infördes ock genom deras försorg sådana till hetiternas alla konungar och till konungarna i Aram.
2Ch 2:1  Och Salomo tänkte nu på att bygga ett hus åt HERRENS namn och ett hus åt sig själv till konungaboning.
2Ch 2:2  Därför avräknade Salomo sjuttio tusen män till att vara bärare, åttio tusen man till att hugga sten i bergen, och tre tusen sex hundra till att hava uppsikt över de andra.
2Ch 2:3  Och Salomo sände till Huram, konungen i Tyrus, och lät säga: "Visa samma vänskap mot mig som mot min fader David, till vilken du sände cederträ, för att han skulle bygga sig ett hus att bo i.
2Ch 2:4  Nu vill jag bygga ett hus åt HERRENS, min Guds, namn och helga det åt honom, för att man där må antända välluktande rökelse inför hans ansikte, och hava skådebröden beständigt upplagda, och offra brännoffer morgon och afton, på sabbaterna, vid nymånaderna och vid HERRENS, vår Guds, högtider; ty så är det för evärdlig tid stadgat för Israel.
2Ch 2:5  Och det hus som jag vill bygga skall vara stort, ty vår Gud är större än alla andra gudar.
2Ch 2:6  Vem förmår väl att bygga honom ett hus? Himlarna och himlarnas himmel rymma honom ju icke. Vem är då jag, att jag skulle kunna bygga honom ett hus, om icke för att antända rökelse inför hans ansikte?
2Ch 2:7  Så sänd mig nu en konstförfaren man som kan arbeta i guld, silver, koppar och järn, så ock i purpurrött, karmosinrött och mörkblått garn, och som är skicklig i att utföra snidverk, tillsammans med de konstförfarna män som jag har hos mig här i Juda och Jerusalem, och som min fader David har anställt.
2Ch 2:8  Och sänd mig cederträ, cypressträ och algumträ från Libanon, ty jag vet att dina tjänare äro skickliga i att hugga virke på Libanon; och mina tjänare äro redo att vara dina tjänare behjälpliga.
2Ch 2:9  Må du skaffa mig virke i myckenhet, ty huset som jag vill bygga skall vara stort och härligt.
2Ch 2:10  Och jag är villig att åt timmermännen som hugga virket giva, för dina tjänares räkning, tjugu tusen korer tröskat vete, tjugu tusen korer korn, tjugu tusen bat vin och tjugu tusen bat olja."
2Ch 2:11  Härpå svarade Huram, konungen i Tyrus, i ett brev som han sände till Salomo: "Därför att HERREN älskar sitt folk, har han satt dig till konung över dem."
2Ch 2:12  Och Huram skrev ytterligare: "Lovad vare HERREN, Israels Gud, himmelens och jordens skapare, han som har givit konung David en vis son, så utrustad med klokhet och förstånd, att han kan bygga ett hus åt HERREN och ett hus åt sig själv till konungaboning!
2Ch 2:13  Så sänder jag nu en konstförfaren och förståndig man, nämligen Huram-Abi.
2Ch 2:14  Han är son till en av Dans döttrar, och hans fader är en tyrisk man; han är skicklig att arbeta i guld och silver, i koppar, järn, sten och trä, så ock i purpurrött, mörkblått, vitt och karmosinrött garn, och tillika att utföra alla slags snidverk och att väva alla slags konstvävnader; honom må du låta utföra arbetet tillsammans med dina och min herres, din fader Davids, konstförfarna män.
2Ch 2:15  Må alltså nu min herre sända till sina tjänare vetet och kornet, oljan och vinet som han har talat om.
2Ch 2:16  Då vilja vi hugga virke på Libanon, så mycket du behöver, och flotta det till dig på havet till Jafo; men därifrån må du själv låta föra det upp till Jerusalem."
2Ch 2:17  Och Salomo lät räkna alla främmande män i Israels land, likasom hans fader David förut hade anställt en räkning av dem. Och de befunnos vara ett hundra femtiotre tusen sex hundra.
2Ch 2:18  Av dem utsåg han sjuttio tusen till att vara bärare, åttio tusen till att hugga sten i bergen, och tre tusen sex hundra till att hava uppsikt över folket och hålla det till arbete.
2Ch 3:1  Och Salomo begynte att bygga HERRENS hus i Jerusalem, på berget Moria, där hans fader David hade fått sin uppenbarelse, och där han nu själv hade berett rum, på det ställe som David hade utsett, nämligen på jebuséen Ornans tröskplats.
2Ch 3:2  Han begynte att bygga på andra dagen i andra månaden, i sitt fjärde regeringsår.
2Ch 3:3  När Salomo då skulle bygga Guds hus, lade han grunden så, att det blev sextio alnar långt och tjugu alnar brett, efter det gamla alnmåttet.
2Ch 3:4  Förhuset, som låg framför långhuset, framför husets kortsida, mätte tjugu alnar, och dess höjd var ett hundra tjugu, och han överdrog det innantill med rent guld.
2Ch 3:5  Huvudbyggnaden beklädde han med cypressträ, detta åter beklädde han med bästa guld, och prydde det med palmer och kedjeverk.
2Ch 3:6  Därjämte smyckade han huset med dyrbara stenar. Men guldet var från Parvaim.
2Ch 3:7  Och han beklädde huset, bjälkarna, trösklarna, ävensom väggarna och dörrarna däri med guld, och lät inrista keruber på väggarna.
2Ch 3:8  Vidare tillredde han det rum som skulle vara det allraheligaste; det låg utefter husets kortsida och var tjugu alnar långt och tjugu alnar brett. Och han beklädde det med bästa guld, sex hundra talenter i vikt.
2Ch 3:9  Och spikarna däri vägde femtio siklar i guld. De övre salarna beklädde han ock med guld.
2Ch 3:10  Och till det rum som var det allraheligaste gjorde han två keruber, i bildhuggeriarbete, och man överdrog dem med guld.
2Ch 3:11  Längden på kerubernas vingar tillsammans var tjugu alnar. Den enas ena vinge, fem alnar lång, rörde vid husets ena vägg, och hans andra vinge, fem alnar lång, rörde vid den andra kerubens vinge.
2Ch 3:12  Och den andra kerubens ena vinge, fem alnar lång, rörde vid husets andra vägg, och hans andra vinge, fem alnar lång, nådde intill den första kerubens vinge.
2Ch 3:13  Alltså bredde dessa keruber ut sina vingar tjugu alnar vitt, under det att de stodo på sina fötter, med ansiktena vända inåt.
2Ch 3:14  Och han gjorde förlåten av mörkblått, purpurrött, karmosinrött och vitt garn och prydde den med keruber.
2Ch 3:15  Och han gjorde två pelare till att stå framför huset, trettiofem alnar höga; och huvudet som satt ovanpå var och en av dem var fem alnar.
2Ch 3:16  Och han gjorde kedjor till koret och satte ock sådana upptill på pelarna. Och vidare gjorde han hundra granatäpplen och satte dem på kedjorna.
2Ch 3:17  Och pelarna ställde han upp framför tempelsalen, den ena på högra sidan och den andra på vänstra; åt den högra gav han namnet Jakin och åt den vänstra namnet Boas.
2Ch 4:1  Och han gjorde ett altare av koppar, tjugu alnar långt, tjugu alnar brett och tio alnar högt.
2Ch 4:2  Han gjorde ock havet, i gjutet arbete. Det var tio alnar från den ena kanten till den andra, runt allt omkring, och fem alnar högt; och ett trettio alnar långt snöre mätte dess omfång.
2Ch 4:3  Och runt omkring nedantill voro bilder som föreställde oxar, och omgåvo det runt omkring - tio alnar brett som det var - så att de omslöto havet runt omkring; oxarna bildade två rader och voro gjutna i ett stycke med det övriga.
2Ch 4:4  Det stod ock på tolv oxar, tre vända mot norr, tre vända mot väster, tre vända mot söder och tre vända mot öster; havet stod ovanpå dessa, och deras bakdelar voro alla vända inåt.
2Ch 4:5  Dess tjocklek var en handsbredd; och dess kant var gjord såsom kanten på en bägare, i form av en utslagen lilja. Det rymde och höll tre tusen bat.
2Ch 4:6  Vidare gjorde han tio bäcken och ställde fem på högra sidan och fem på vänstra, för att brukas vid tvagning; i dem skulle man nämligen skölja vad som hörde till brännoffret. Men havet var för prästerna till att två sig i.
2Ch 4:7  Vidare gjorde han de gyllene ljusstakarna, tio till antalet, sådana de skulle vara, och ställde dem i tempelsalen, fem på högra sidan och fem på vänstra.
2Ch 4:8  Vidare gjorde han tio bord och satte dem i tempelsalen, fem på högra sidan och fem på vänstra. Han gjorde ock ett hundra skålar av guld.
2Ch 4:9  Och han gjorde prästernas förgård och den stora yttre förgården, så ock dörrar till denna förgård; och dörrarna överdrog han med koppar.
2Ch 4:10  Och havet ställde han på högra sidan, åt sydost.
2Ch 4:11  Dessutom gjorde Huram askkärlen, skovlarna och skålarna. Så förde Hiram det arbete till slut, som han fick utföra åt konung Salomo för Guds hus:
2Ch 4:12  nämligen två pelare, och de två kloten och pelarhuvudena ovanpå pelarna, och de två nätverk som skulle betäcka de båda klotformiga pelarhuvuden som sutto ovanpå pelarna,
2Ch 4:13  och därjämte de fyra hundra granatäpplena till de båda nätverken, två rader granatäpplen till vart nätverk, för att de båda klotformiga pelarhuvuden som sutto uppe på pelarna så skulle bliva betäckta.
2Ch 4:14  Vidare gjorde han bäckenställen och gjorde tillika bäckenen på bäckenställen,
2Ch 4:15  så ock havet, som var allenast ett, och de tolv oxarna därunder.
2Ch 4:16  Och askkärlen, skovlarna och gafflarna och alla dithörande föremål gjorde Huram-Abiv åt konung Salomo till HERRENS hus. Allt var av blank koppar.
2Ch 4:17  På Jordanslätten lät konungen gjuta det i lerformar, mellan Suckot och Sereda.
2Ch 4:18  Och Salomo lät göra en så stor myckenhet av alla dessa föremål, att kopparens vikt icke kunde utrönas.
2Ch 4:19  Alltså gjorde Salomo alla föremål som skulle finnas i Guds hus: det gyllene altaret, borden som skådebröden skulle ligga på,
2Ch 4:20  så ock ljusstakarna med sina lampor, som skulle tändas på föreskrivet sätt, framför koret, av fint guld,
2Ch 4:21  med blomverket, lamporna och lamptängerna av guld - allt av yppersta guld;
2Ch 4:22  vidare knivarna, de båda slagen av skålar och fyrfaten, av fint guld. Och vad angår ingångarna i huset, så voro både de dörrar i dess innersta, som ledde till det allraheligaste, och de dörrar i huset, som ledde till tempelsalen, gjorda av guld.
2Ch 5:1  Sedan allt det arbete som Salomo lät utföra för HERRENS hus var färdigt, förde Salomo ditin vad hans fader David hade helgat åt HERREN: silvret, guldet och alla kärlen; detta lade han in i skattkamrarna i Guds hus.
2Ch 5:2  Därefter församlade Salomo de äldste i Israel, alla huvudmännen för stammarna, Israels barns familjehövdingar, till Jerusalem, för att hämta HERRENS förbundsark upp från Davids stad, det är Sion.
2Ch 5:3  Så församlade sig då till konungen alla Israels män under högtiden, den som firades i sjunde månaden.
2Ch 5:4  När då alla de äldste i Israel hade kommit tillstädes, lyfte leviterna upp arken.
2Ch 5:5  Och de hämtade arken och uppenbarelsetältet ditupp, jämte alla heliga föremål som funnos i tältet; de levitiska prästerna hämtade det ditupp.
2Ch 5:6  Och konung Salomo stod framför arken jämte Israels hela menighet, som hade församlats till honom; och de offrade därvid småboskap och fäkreatur i sådan myckenhet, att de icke kunde täljas eller räknas.
2Ch 5:7  Och prästerna buro in HERRENS förbundsark till dess plats i husets kor, i det allraheligaste, till platsen under kerubernas vingar.
2Ch 5:8  Keruberna höllo nämligen sina vingar utbredda över den plats där arken stod, så att arken och dess stänger ovantill övertäcktes av keruberna.
2Ch 5:9  Och stängerna voro så långa, att deras ändar, som sköto ut från arken, väl kunde ses framför koret, men däremot icke voro synliga längre ute. Och den har blivit kvar där ända till denna dag.
2Ch 5:10  I arken fanns intet annat än de två tavlor som Mose hade lagt dit vid Horeb, när HERREN slöt förbund med Israels barn, sedan de hade dragit ut ur Egypten.
2Ch 5:11  Men när prästerna gingo ut ur helgedomen (ty alla präster som funnos där hade helgat sig, utan avseende på vilken avdelning de tillhörde;
2Ch 5:12  och leviterna, samtliga sångarna, Asaf, Heman och Jedutun med sina söner och bröder, stodo, klädda i vitt linne, med cymbaler, psaltare och harpor öster om altaret, och jämte dem ett hundra tjugu präster som blåste i trumpeter;
2Ch 5:13  och trumpetblåsarna och sångarna stämde på en gång och enhälligt upp HERRENS lov och pris), och när man nu lät trumpeter och cymbaler och andra instrumenter ljuda och begynte lova HERREN, därför att han är god, och därför att hans nåd varar evinnerligen, då blev huset, HERRENS hus, uppfyllt av en molnsky,
2Ch 5:14  så att prästerna för molnskyns skull icke kunde stå där och göra tjänst; ty HERRENS härlighet uppfyllde Guds hus.
2Ch 6:1  Då sade Salomo: "HERREN har sagt att han vill bo i töcknet.
2Ch 6:2  Men jag har byggt ett hus till boning åt dig och berett en plats där du må förbliva till evig tid."
2Ch 6:3  Sedan vände konungen sig om och välsignade Israels hela församling, under det att Israels hela församling förblev stående.
2Ch 6:4  Han sade: "Lovad vare HERREN, Israels Gud, som med sina händer har fullbordat vad han med sin mun lovade min fader David, i det han sade:
2Ch 6:5  'Från den dag då jag förde mitt folk ut ur Egyptens land har jag icke i någon av Israels stammar utvalt en stad, till att i den bygga ett hus där mitt namn skulle vara, ej heller har jag utvalt någon man till att vara en furste över mitt folk Israel;
2Ch 6:6  men Jerusalem har jag nu utvalt, för att mitt namn skall vara där, och David har jag utvalt till att råda över mitt folk Israel.'
2Ch 6:7  Och min fader David hade väl i sinnet att bygga ett hus åt HERRENS, Israels Guds, namn;
2Ch 6:8  men HERREN sade till min fader David: 'Då du nu har i sinnet att bygga ett hus åt mitt namn, så gör du visserligen väl däri att du har detta i sinnet;
2Ch 6:9  dock skall icke du få bygga detta hus, utan din son, den som har utgått från din länd, han skall bygga huset åt mitt namn.'
2Ch 6:10  Och HERREN har uppfyllt det löfte han gav; ty jag har kommit upp i min fader Davids ställe och sitter nu på Israels tron, såsom HERREN lovade, och jag har byggt huset åt HERRENS, Israels Guds, namn.
2Ch 6:11  Och där har jag satt arken, i vilken förvaras det förbund som HERREN slöt med Israels barn."
2Ch 6:12  Därefter trädde han fram för HERRENS altare inför Israels hela församling och uträckte sina händer.
2Ch 6:13  Ty Salomo hade gjort en talarstol av koppar, fem alnar lång, fem alnar bred och tre alnar hög, och ställt den mitt på den yttre förgården, på den stod han nu. Och han föll ned på sina knän inför Israels hela församling, och uträckte sina händer mot himmelen
2Ch 6:14  och sade: "HERRE, Israels Gud, ingen gud är dig lik, i himmelen eller på jorden, du som håller förbund och bevarar nåd mot dina tjänare, när de vandra inför dig av allt sitt hjärta,
2Ch 6:15  du som har hållit vad du lovade din tjänare David, min fader; ty vad du med din mun lovade, det fullbordade du med din hand, så som nu har skett.
2Ch 6:16  Så håll nu ock, HERRE, Israels Gud, vad du lovade din tjänare David, min fader, i det att du sade: 'Aldrig skall den tid komma, då på Israels tron icke inför mig sitter en avkomling av dig, om allenast dina barn hava akt på sin väg, så att de vandra efter min lag, såsom du har vandrat inför mig.'
2Ch 6:17  Så låt nu, HERRE, Israels Gud, det ord som du har talat till din tjänare David bliva sant.
2Ch 6:18  Men kan då Gud verkligen bo på jorden bland människorna? Himlarna och himlarnas himmel rymma dig ju icke; huru mycket mindre då detta hus som jag har byggt!
2Ch 6:19  Men vänd dig ändå till din tjänares bön och åkallan, HERRE, min Gud, så att du hör på det rop och den bön som din tjänare uppsänder till dig
2Ch 6:20  och låter dina ögon dag och natt vara öppna och vända mot detta hus - den plats varom du har sagt att du där vill fästa ditt namn - så att du ock hör den bön som din tjänare beder, vänd mot denna plats.
2Ch 6:21  Ja, hör på de böner som din tjänare och ditt folk Israel uppsända, vända mot denna plats. Må du höra dem från himmelen, där du bor; och när du hör, så må du förlåta.
2Ch 6:22  Om någon försyndar sig mot sin nästa och man ålägger honom en ed och låter honom svärja, och han så kommer och svär inför ditt altare i detta hus,
2Ch 6:23  må du då höra det från himmelen och utföra ditt verk och skaffa dina tjänare rätt, i det att du vedergäller den skyldige och låter hans gärningar komma över hans huvud, men skaffar rätt åt den som har rätt och låter honom få efter hans rättfärdighet.
2Ch 6:24  Och om ditt folk Israel bliver slaget av en fiende, därför att de hava syndat mot dig, men de omvända sig och prisa ditt namn och bedja och åkalla inför ditt ansikte i detta hus
2Ch 6:25  må du då höra det från himmelen och förlåta ditt folk Israels synd och låta dem komma tillbaka till det land som du har givit åt dem och deras fäder.
2Ch 6:26  Om himmelen bliver tillsluten, så att regn icke faller, därför att de hava syndat mot dig, men de då bedja, vända mot denna plats, och prisa ditt namn och omvända sig från sin synd, när du bönhör dem,
2Ch 6:27  må du då höra det i himmelen och förlåta dina tjänares och ditt folk Israels synd, i det att du lär dem den goda väg som de skola vandra; och må du låta det regna över ditt land, det som du har givit åt ditt folk till arvedel.
2Ch 6:28  Om hungersnöd uppstår i landet, om pest uppstår, om sot och rost, om gräshoppor och gräsmaskar komma, om fienderna tränga folket i det land där deras städer stå, eller om någon annan plåga och sjukdom kommer, vilken det vara må,
2Ch 6:29  och om då någon bön och åkallan höjes från någon människa, vilken det vara må, eller ock från hela ditt folk Israel, när de var för sig känna den plåga och smärta som har drabbat dem, och de så uträcka sina händer mot detta hus,
2Ch 6:30  må du då höra det från himmelen, där du bor, och förlåta och giva var och en efter alla hans gärningar, eftersom du känner hans hjärta - ty du allena känner människornas hjärtan -
2Ch 6:31  på det att de alltid må frukta dig och vandra på dina vägar, så länge de leva i det land som du har givit åt våra fäder.
2Ch 6:32  Också om en främling, en som icke är av ditt folk Israel, kommer ifrån fjärran land, för ditt stora namns och din starka hands och din uträckta arms skull, om någon sådan kommer och beder, vänd mot detta hus,
2Ch 6:33  må du då från himmelen, där du bor, höra det och göra allt varom främlingen ropar till dig, på det att alla jordens folk må känna ditt namn och frukta dig, likasom ditt folk Israel gör, och förnimma att detta hus som jag har byggt är uppkallat efter ditt namn.
2Ch 6:34  Om ditt folk drager ut till strid mot sina fiender, på den väg du sänder dem, och de då bedja till dig, vända i riktning mot denna stad som du har utvalt och mot det hus som jag har byggt åt ditt namn,
2Ch 6:35  må du då från himmelen höra deras bön och åkallan och skaffa dem rätt.
2Ch 6:36  Om de synda mot dig - eftersom ingen människa finnes, som icke syndar - och du bliver vred på dem och giver dem i fiendens våld, så att man tager dem till fånga och för dem bort till något annat land, fjärran eller nära,
2Ch 6:37  men de då besinna sig i det land där de äro i fångenskap, och omvända sig och åkalla dig i fångenskapens land och säga: 'Vi hava syndat, vi hava gjort illa och varit ogudaktiga',
2Ch 6:38  om de så omvända sig till dig av allt sitt hjärta och av all sin själ, i fångenskapens land, dit man har fört dem i fångenskap, och bedja, vända i riktning mot sitt land, det som du har givit åt deras fäder, och mot den stad som du har utvalt, och mot det hus som jag har byggt åt ditt namn,
2Ch 6:39  må du då från himmelen, där du bor, höra deras bön och åkallan och skaffa dem rätt och förlåta ditt folk vad de hava syndat mot dig.
2Ch 6:40  Ja, min Gud, låt nu dina ögon vara öppna och dina öron akta på vad som bedes på denna plats.
2Ch 6:41  Ja: Stå upp, HERRE Gud, och kom till din vilostad, du och din makts ark. Dina präster, HERRE Gud, vare klädda i frälsning, och dina fromma glädje sig över ditt goda.
2Ch 6:42  HERRE Gud, visa icke tillbaka din smorde; tänk på den nåd du har lovat din tjänare David.
2Ch 7:1  När Salomo hade slutat sin bön, kom eld ned från himmelen och förtärde brännoffret och slaktoffren, och HERRENS härlighet uppfyllde huset.
2Ch 7:2  Och prästerna kunde icke gå in i HERRENS hus, eftersom HERRENS härlighet uppfyllde HERRENS hus.
2Ch 7:3  Då nu alla Israels barn sågo huru elden kom ned, och sågo HERRENS härlighet över huset, föllo de ned på den stenlagda gården, med ansiktena mot jorden, och tillbådo HERREN och tackade honom, därför att han är god, och därför att hans nåd varar evinnerligen.
2Ch 7:4  Och konungen och allt folket offrade slaktoffer inför HERRENS ansikte.
2Ch 7:5  Konung Salomo offrade såsom slaktoffer tjugutvå tusen tjurar och ett hundra tjugu tusen av småboskapen. Så invigdes Guds hus av konungen och allt folket.
2Ch 7:6  Och prästerna stodo där i sina tjänstförrättningar, och leviterna stodo med HERRENS musikinstrumenter, som konung David hade låtit göra, för att de med dem skulle tacka HERREN, därför att hans nåd varar evinnerligen; David lät nämligen dem utföra lovsången. Men prästerna stodo mitt emot dem och blåste i trumpeter, medan hela Israel förblev stående.
2Ch 7:7  Och Salomo helgade den mellersta delen av förgården framför HERRENS hus; ty där offrade han brännoffren och fettstyckena av tackoffret eftersom kopparaltaret som Salomo hade låtit göra icke kunde rymma brännoffret, spisoffret och fettstyckena.
2Ch 7:8  Tid detta tillfälle firade Salomo högtiden i sju dagar, och med honom hela Israel, en mycket stor församling ifrån hela landet, allt ifrån det ställe där vägen går till Hamat ända till Egyptens bäck.
2Ch 7:9  Och på åttonde dagen höllo de högtidsförsamling. Ty altarets invigning firade de i sju dagar och högtiden i sju dagar.
2Ch 7:10  Men på tjugutredje dagen i sjunde månaden lät han folket gå hem till sina hyddor; och de voro fulla av glädje och fröjd över det goda som HERREN hade gjort mot David och Salomo och mot sitt folk Israel.
2Ch 7:11  Så fullbordade Salomo HERRENS hus och konungshuset; och allt vad Salomo hade haft i sinnet att utföra i HERRENS hus och i sitt eget hus hade lyckats honom väl.
2Ch 7:12  Och HERREN uppenbarade sig för Salomo om natten och sade till honom: "Jag har hört din bön och utvalt denna plats åt mig till offerplats.
2Ch 7:13  Om jag tillsluter himmelen, så att regn icke faller, om jag bjuder gräshoppor att fördärva landet, eller om jag sänder pest bland mitt folk,
2Ch 7:14  men mitt folk, det som är uppkallat efter mitt namn, då ödmjukar sig och beder och söker mitt ansikte och omvänder sig från sina onda vägar, så vill jag höra det från himmelen och förlåta deras synd och skaffa bot åt deras land.
2Ch 7:15  Så skola nu mina ögon vara öppna och mina öron akta på vad som bedes på denna plats.
2Ch 7:16  Och nu har jag utvalt och helgat detta hus, för att mitt namn skall vara där till evig tid. Och mina ögon och mitt hjärta skola vara där alltid.
2Ch 7:17  Om du nu vandrar inför mig, såsom din fader David vandrade, så att du gör allt vad jag har bjudit dig och håller mina stadgar och rätter,
2Ch 7:18  då skall jag upprätthålla din konungatron, såsom jag lovade din fader David, när jag sade: 'Aldrig skall den tid komma, då en avkomling av dig icke råder över Israel.'
2Ch 7:19  Men om I vänden om och övergiven de stadgar och bud som jag har förelagt eder, och gån bort och tjänen andra gudar och tillbedjen dem,
2Ch 7:20  då skall jag rycka upp dem som så göra ur mitt land, det som jag har givit dem; och detta hus som jag har helgat åt mitt namn skall jag förkasta ifrån mitt ansikte; och jag skall göra det till ett ordspråk och en visa bland alla folk.
2Ch 7:21  Och över detta hus, som har varit så upphöjt, skall då var och en som går därförbi bliva häpen. Och när någon frågar: 'Varför har HERREN gjort så mot detta land och detta hus?',
2Ch 7:22  då skall man svara: 'Därför att de övergåvo HERREN, sina fäders Gud, som hade fört dem ut ur Egyptens land, och höllo sig till andra gudar och tillbådo dem och tjänade dem, därför har han låtit allt detta onda komma över dem.'"
2Ch 8:1  När de tjugu år voro förlidna, under vilka Salomo byggde på HERRENS hus och på sitt eget hus,
2Ch 8:2  byggde Salomo upp de städer som Huram hade givit honom och lät Israels barn bosätta sig i dem.
2Ch 8:3  Och Salomo drog till Hamat-Soba och bemäktigade sig det.
2Ch 8:4  Och han byggde upp Tadmor i öknen och alla de förrådsstäder som i Hamat äro byggda av honom.
2Ch 8:5  Vidare byggde han upp Övre Bet-Horon och Nedre Bet-Horon och gjorde dem till fasta städer med murar, portar och bommar,
2Ch 8:6  så ock Baalat och alla Salomos förrådsstäder, ävensom alla vagnsstäderna och häststäderna, och allt annat som Salomo kände åstundan att bygga i Jerusalem, på Libanon och eljest i hela det land som lydde under hans välde.
2Ch 8:7  Allt det folk som fanns kvar av hetiterna, amoréerna, perisséerna, hivéerna och jebuséerna, korteligen, alla de som icke voro av Israel -
2Ch 8:8  deras avkomlingar, så många som funnos kvar i landet efter dem, i det att Israels barn icke hade utrotat dem, dessa pålade Salomo att vara arbetspliktiga, såsom de äro ännu i dag.
2Ch 8:9  Men somliga av Israels barn gjorde Salomo icke till trälar vid de arbeten han utförde, utan de blevo krigare och hövitsmän för hans kämpar, eller uppsyningsmän över hans vagnar och ridhästar.
2Ch 8:10  Och konung Salomos överfogdar voro två hundra femtio; dessa hade befälet över folket.
2Ch 8:11  Och Salomo lät Faraos dotter flytta upp från Davids stad till det hus som han hade byggt åt henne; ty han sade: "Jag vill icke att någon kvinna skall bo i Davids, Israels konungs, hus, ty det är en helig plats, eftersom HERRENS ark har kommit dit."
2Ch 8:12  Nu offrade Salomo brännoffer åt HERREN på HERRENS, altare, det som han hade byggt framför förhuset;
2Ch 8:13  han offrade var dag de för den dagen bestämda offren, efter Moses bud, på sabbaterna, vid nymånaderna och vid högtiderna tre gånger om året, nämligen vid det osyrade brödets högtid, vid veckohögtiden och vid lövhyddohögtiden.
2Ch 8:14  Och efter sin fader Davids anordning fastställde han de avdelningar i vilka prästerna skulle tjänstgöra, ävensom leviternas åligganden, att de skulle utföra lovsången och betjäna prästerna - var dag de för den dagen bestämda åliggandena - så ock huru dörrvaktarna, efter sina avdelningar, skulle hålla vakt vid de särskilda portarna; ty så hade gudsmannen David bjudit.
2Ch 8:15  Och man vek icke av ifrån vad konungen hade bjudit angående prästerna och leviterna, varken i fråga om någon annan angelägenhet eller i fråga om förråden.
2Ch 8:16  Så utfördes allt Salomos arbete, först intill den dag då grunden lades till HERRENS hus, och sedan intill dess det blev fullbordat. Och så var då HERRENS hus färdigt.
2Ch 8:17  Vid denna tid drog Salomo till Esjon-Geber och till Elot, på havsstranden, i Edoms land.
2Ch 8:18  Och Huram sände till honom skepp genom sitt folk, och därjämte av sitt folk sjökunnigt manskap. De foro med Salomos folk till Ofir och hämtade därifrån fyra hundra femtio talenter guld, som de förde till konung Salomo.
2Ch 9:1  När drottningen av Saba fick höra ryktet om Salomo, kom hon för att i Jerusalem sätta Salomo på prov med svåra frågor. Hon kom med ett mycket stort följe och förde med sig kameler, som buro välluktande kryddor och guld i myckenhet, så ock ädla stenar. Och när hon kom inför konung Salomo, förelade hon honom allt vad hon hade i tankarna.
2Ch 9:2  Men Salomo gav henne svar på alla hennes frågor; intet var förborgat för Salomo, utan han kunde giva henne svar på allt.
2Ch 9:3  När nu drottningen av Saba såg Salomos vishet, och såg huset som han hade byggt,
2Ch 9:4  och såg rätterna på hans bord och såg huru hans tjänare sutto där, och huru de som betjänade honom utförde sina åligganden, och huru de voro klädda, och vidare såg hans munskänkar, och huru de voro klädda, och när hon såg den trappgång på vilken han gick upp till HERRENS hus, då blev hon utom sig av förundran.
2Ch 9:5  Och hon sade till konungen: "Sant var det tal som jag hörde i mitt land om dig och om din vishet.
2Ch 9:6  Jag ville icke tro vad man sade förrän jag själv kom och med egna ögon fick se det; men nu finner jag att vidden av din vishet icke ens till hälften har blivit omtalad för mig. Du är vida förmer, än jag genom ryktet hade hört.
2Ch 9:7  Sälla äro dina män, och sälla äro dessa dina tjänare, som beständigt få stå inför dig och höra din visdom.
2Ch 9:8  Lovad vare HERREN, din Gud, som har funnit sådant behag i dig, att han har satt dig på sin tron till att vara konung inför HERREN, din Gud! Ja, därför att din Gud älskar Israel och vill hålla det vid makt evinnerligen, därför har han satt dig till konung över dem, för att du skall skipa lag och rätt."
2Ch 9:9  Och hon gav åt konungen ett hundra tjugu talenter guld, så och välluktande kryddor i stor myckenhet, därtill ädla stenar; sådana välluktande kryddor som de vilka drottningen av Saba gav åt konung Salomo hava eljest icke funnits.
2Ch 9:10  När Hirams folk och Salomos folk hämtade guld från Ofir, hemförde också de algumträ och ädla stenar.
2Ch 9:11  Av algumträet lät konungen göra tillbehör till HERRENS hus och till konungshuset, så ock harpor och psaltare för sångarna. Sådant hade aldrig förut blivit sett i Juda land.
2Ch 9:12  Konung Salomo åter gav åt drottningen av Saba allt vad hon åstundade och begärde, förutom vad som svarade emot det hon hade medfört åt konungen. Sedan vände hon om och for till sitt land igen med sina tjänare.
2Ch 9:13  Det guld som årligen inkom till Salomo vägde sex hundra sextiosex talenter,
2Ch 9:14  förutom det som infördes genom kringresande handelsmän och andra köpmän; också Arabiens alla konungar och ståthållarna i landet förde guld och silver till Salomo.
2Ch 9:15  Och konung Salomo lät göra två hundra stora sköldar av uthamrat guld och använde till var sådan sköld sex hundra siklar uthamrat guld;
2Ch 9:16  likaledes tre hundra mindre sköldar av uthamrat guld och använde till var sådan sköld tre hundra siklar guld; och konungen satte upp dem i Libanonskogshuset.
2Ch 9:17  Vidare lät konungen göra en stor tron av elfenben och överdrog den med rent guld.
2Ch 9:18  Tronen hade sex trappsteg och en pall av guld, fastsatta vid tronen; på båda sidor om sitsen voro armstöd, och två lejon stodo utmed armstöden;
2Ch 9:19  och tolv lejon stodo där på de sex trappstegen, på båda sidor. Något sådant har aldrig blivit förfärdigat i något annat rike.
2Ch 9:20  Och alla konung Salomos dryckeskärl voro av guld, och alla kärl i Libanonskogshuset voro av fint guld; silver aktades icke för något i Salomos tid.
2Ch 9:21  Ty konungen hade skepp som gingo till Tarsis med Hurams folk; en gång vart tredje år kommo Tarsis-skeppen hem och förde med sig guld och silver, elfenben, apor och påfåglar.
2Ch 9:22  Och konung Salomo blev större än någon annan konung på jorden, både i rikedom och i vishet.
2Ch 9:23  Alla konungar på jorden kommo för att besöka Salomo och höra den vishet som Gud hade nedlagt i hans hjärta.
2Ch 9:24  Och var och en av dem förde med sig skänker: föremål av silver och av guld, kläder, vapen, välluktande kryddor, hästar och mulåsnor. Så skedde år efter år.
2Ch 9:25  Och Salomo hade fyra tusen spann hästar med vagnar och tolv tusen ridhästar; dem förlade han dels i vagnsstäderna, dels i Jerusalem, hos konungen själv.
2Ch 9:26  Och han var herre över alla konungar ifrån floden ända till filistéernas land och sedan ända ned till Egyptens gräns.
2Ch 9:27  Och konungen styrde så, att silver blev lika vanligt i Jerusalem som stenar, och cederträ lika vanligt som mullbärsfikonträ i Låglandet.
2Ch 9:28  Och hästar infördes till Salomo från Egypten och från alla andra länder.
2Ch 9:29  Vad nu vidare är att säga om Salomo, om hans första tid såväl som om hans sista, det finnes upptecknat i profeten Natans krönika, i siloniten Ahias profetia och i siaren Jedais syner om Jerobeam, Nebats son.
2Ch 9:30  Salomo regerade i Jerusalem över hela Israel i fyrtio år.
2Ch 9:31  Och Salomo gick till vila hos sina fäder, och man begrov honom i hans fader Davids stad. Och hans son Rehabeam blev konung efter honom.
2Ch 10:1  Och Rehabeam drog till Sikem, ty hela Israel hade kommit till Sikem för att göra honom till konung.
2Ch 10:2  När Jerobeam, Nebats son, hörde detta, där han var i Egypten - dit hade han nämligen flytt för konung Salomo - vände han tillbaka från Egypten.
2Ch 10:3  Och de sände bort och läto kalla honom åter. Då kom Jerobeam tillstädes jämte hela Israel och talade till Rehabeam och sade:
2Ch 10:4  "Din fader gjorde vårt ok för svårt; men lätta nu du det svåra arbete och det tunga ok som din fader lade på oss, så vilja vi tjäna dig."
2Ch 10:5  Han svarade dem: "Vänten ännu tre dagar, och kommen så tillbaka till mig." Och folket gick.
2Ch 10:6  Då rådförde sig konung Rehabeam med de gamle som hade varit i tjänst hos hans fader Salomo, medan denne ännu levde; han sade: "Vilket svar råden I mig att giva detta folk?"
2Ch 10:7  De svarade honom och sade: "Om du visar dig god mot detta folk och är nådig mot dem och talar goda ord till dem, så skola de för alltid bliva dina tjänare."
2Ch 10:8  Men han aktade icke på det råd som de gamle hade givit honom, utan rådförde sig med de unga män som hade vuxit upp med honom, och som nu voro i hans tjänst.
2Ch 10:9  Han sade till dem: "Vilket svar råden I oss att giva detta folk som har talat till mig och sagt: 'Lätta det ok som din fader har lagt på oss'?"
2Ch 10:10  De unga männen som hade vuxit upp med honom svarade honom då och sade: "Så bör du säga till folket som har talat till dig och sagt: 'Din fader gjorde vårt ok tungt, men lätta du det för oss' - så bör du säga till dem: 'Mitt minsta finger är tjockare än min faders länd.
2Ch 10:11  Så veten nu, att om min fader har belastat eder med ett tungt ok, så skall jag göra edert ok ännu tyngre; har min fader tuktat eder med ris, så skall jag göra det med skorpiongissel.'"
2Ch 10:12  Så kom nu Jerobeam med allt folket till Rehabeam på tredje dagen, såsom konungen hade befallt, i det han sade: "Kommen tillbaka till mig på tredje dagen."
2Ch 10:13  Då gav konungen dem ett hårt svar; ty konung Rehabeam aktade icke på de gamles råd.
2Ch 10:14  Han talade till dem efter de unga männens råd och sade: "Jag skall göra edert ok tungt, ja, jag skall göra det ännu tyngre än förut; har min fader tuktat eder med ris, så skall jag göra det med skorpiongissel."
2Ch 10:15  Alltså hörde konungen icke på folket; ty det var så skickat av Gud, för att HERRENS ord skulle uppfyllas, det som han hade talat till Jerobeam, Nebats son, genom Ahia från Silo.
2Ch 10:16  Då nu hela Israel förnam att konungen icke ville höra på dem, gav folket konungen detta svar: "Vad del hava vi i David? Ingen arvslott hava vi i Isais son. Israel drage hem, var och en till sin hydda. Se nu själv om ditt hus, du David." Därefter drog hela Israel hem till sina hyddor.
2Ch 10:17  Allenast över de israeliter som bodde i Juda städer förblev Rehabeam konung.
2Ch 10:18  Och när konung Rehabeam sände åstad Hadoram, som hade uppsikten över de allmänna arbetena, stenade Israels barn denne till döds; och konung Rehabeam själv måste med hast stiga upp i sin vagn och fly till Jerusalem.
2Ch 10:19  Så avföll Israel från Davids hus och har varit skilt därifrån ända till denna dag.
2Ch 11:1  Och när Rehabeam kom till Jerusalem, församlade han Juda hus och Benjamin, ett hundra åttio tusen utvalda krigare, för att de skulle strida mot Israel och återvinna konungadömet åt Rehabeam.
2Ch 11:2  Men HERRENS ord kom till gudsmannen Semaja; han sade:
2Ch 11:3  "Säg till Rehabeam, Salomos son, Juda konung, och till alla israeliter i Juda och Benjamin:
2Ch 11:4  Så säger HERREN: I skolen icke draga upp och strida mot edra bröder. Vänden tillbaka hem, var och en till sitt, ty vad som har skett har kommit från mig." Och de lyssnade till HERRENS ord och vände om och drogo icke mot Jerobeam.
2Ch 11:5  Men Rehabeam bodde i Jerusalem, och han befäste städer i Juda och gjorde dem till fasta platser.
2Ch 11:6  Han befäste Bet-Lehem, Etam, Tekoa,
2Ch 11:7  Bet-Sur, Soko, Adullam
2Ch 11:8  Gat, Maresa, Sif,
2Ch 11:9  Adoraim, Lakis, Aseka,
2Ch 11:10  Sorga, Ajalon och Hebron, alla i Juda och Benjamin, och gjorde dem till fasta städer.
2Ch 11:11  Och han gjorde deras befästningar starka och tillsatte hövdingar i dem och lade in i dem förråd av mat, olja och vin;
2Ch 11:12  var och en särskild av dessa städer försåg han med sköldar och spjut; han befäste dem mycket starkt. Och Juda och Benjamin förblevo under hans välde.
2Ch 11:13  Och prästerna och leviterna i hela Israel gingo över till honom från alla sina områden;
2Ch 11:14  ty leviterna övergåvo sina utmarker och sina andra besittningar och begåvo sig till Juda och Jerusalem, eftersom Jerobeam med sina söner drev dem bort ifrån deras tjänst såsom HERRENS präster,
2Ch 11:15  och anställde åt sig andra präster för offerhöjderna och för de onda andarna och för kalvarna som han hade låtit göra.
2Ch 11:16  Och dem följde ifrån alla Israels stammar de som vände sina hjärtan till att söka HERREN, Israels Gud; dessa kommo till Jerusalem för att offra åt HERREN, sina fäders Gud.
2Ch 11:17  I tre år befäste de så konungamakten i Juda och gjorde Rehabeams, Salomos sons, välde starkt; ty i tre år vandrade de på Davids och Salomos väg.
2Ch 11:18  Och Rehabeam tog till hustru åt sig Mahalat, dotter till Jerimot, Davids son, och till Abihail, Eliabs, Isais sons, dotter.
2Ch 11:19  Hon födde åt honom sönerna Jeus, Semarja och Saham.
2Ch 11:20  Och efter henne tog han till hustru Maaka, Absaloms dotter. Hon födde åt honom Abia, Attai, Sisa och Selomit.
2Ch 11:21  Och Rehabeam hade Maaka, Absaloms dotter, kärare än alla sina andra hustrur och bihustrur - ty han hade tagit aderton hustrur och sextio bihustrur - och han födde tjuguåtta söner och sextio döttrar.
2Ch 11:22  Och Rehabeam satte Abia, Maakas son, till huvud och furste bland sina bröder, ty han hade i sinnet att göra honom till konung.
2Ch 11:23  Och på lämpligt sätt fördelade han alla Judas och Benjamins landskap och alla fasta städer mellan några av sina söner och gav dem rikligt underhåll; han skaffade dem ock hustrur i mängd.
2Ch 12:1  När Rehabeams konungamakt nu hade blivit befäst och han hade blivit mäktig, övergav han HERRENS lag, han jämte hela Israel.
2Ch 12:2  Men i konung Rehabeams femte regeringsår drog Sisak, konungen i Egypten, upp mot Jerusalem, därför att de hade varit otrogna mot HERREN;
2Ch 12:3  han kom med ett tusen två hundra vagnar och sextio tusen ryttare, och ingen kunde räkna det folk som följde honom från Egypten: libyer, suckéer och etiopier.
2Ch 12:4  Och han intog de fasta städerna i Juda och kom ända till Jerusalem.
2Ch 12:5  Och profeten Semaja hade kommit till Rehabeam och till Juda furstar, som hade församlat sig i Jerusalem av fruktan för Sisak; och han sade till dem: "Så säger HERREN: I haven övergivit mig, därför har ock jag övergivit eder och givit eder i Sisaks hand."
2Ch 12:6  Då ödmjukade sig Israels furstar och konungen själv och sade: "HERREN är rättfärdig."
2Ch 12:7  När nu HERREN såg att de ödmjukade sig, kom HERRENS ord till Semaja; han sade: "Eftersom de hava ödmjukat sig, vill jag icke fördärva dem; jag skall låta dem med knapp nöd komma undan, och min vrede skall icke bliva utgjuten över Jerusalem genom Sisaks hand.
2Ch 12:8  Dock skola de nödgas bliva honom underdåniga, för att de må lära sig förstå vilken skillnad det är mellan att tjäna mig och att tjäna främmande konungadömen."
2Ch 12:9  Så drog nu Sisak, konungen i Egypten, upp mot Jerusalem. Och han tog skatterna i HERRENS hus och skatterna i konungshuset; alltsammans tog han. Han tog ock de gyllene sköldar som Salomo hade låtit göra.
2Ch 12:10  I deras ställe lät konung Rehabeam göra sköldar av koppar, och dessa lämnade han i förvar åt hövitsmännen för drabanterna som höllo vakt vid ingången till konungshuset.
2Ch 12:11  Och så ofta konungen gick till HERRENS hus, gingo ock drabanterna och buro dem; sedan förde de dem tillbaka till drabantsalen.
2Ch 12:12  Därför att nu Rehabeam ödmjukade sig, vände sig HERRENS vrede ifrån honom, så att han icke alldeles fördärvade honom. Också fanns ännu något gott i Juda.
2Ch 12:13  Alltså befäste konung Rehabeam sitt välde i Jerusalem och fortsatte att regera. Rehabeam var nämligen fyrtioett år gammal, när han blev konung, och han regerade sjutton år i Jerusalem, den stad som HERREN hade utvalt ur alla Israel stammar, till att där fästa sitt namn. Hans moder hette Naama, ammonitiskan.
2Ch 12:14  Och han gjorde vad ont var, ty han vände icke sitt hjärta till att söka HERREN.
2Ch 12:15  Men vad som är att säga om Rehabeam, om hans första tid såväl som om hans sista, det finnes upptecknat i profeten Semajas och siaren Iddos krönikor, enligt släktregistrens sätt. Och Rehabeam och Jerobeam lågo i krig med varandra, så länge de levde.
2Ch 12:16  Men Rehabeam gick till vila hos sina fäder och blev begraven i Davids stad. Och hans son Abia blev konung efter honom.
2Ch 13:1  I konung Jerobeams adertonde regeringsår blev Abia konung över Juda.
2Ch 13:2  Han regerade tre år i Jerusalem. Hans moder hette Mikaja, Uriels dotter, från Gibea. Men Abia och Jerobeam lågo i krig med varandra.
2Ch 13:3  Och Abia begynte kriget med en här av tappra krigsmän, fyra hundra tusen utvalda män; men Jerobeam ställde upp sig till strid mot honom med åtta hundra tusen utvalda tappra stridsmän.
2Ch 13:4  Och Abia steg upp på berget Semaraim i Efraims bergsbygd och sade: "Hören mig, du, Jerobeam, och I, hela Israel.
2Ch 13:5  Skullen I icke veta att det är HERREN, Israels Gud, som har givit åt David konungadömet över Israel för evig tid, åt honom själv och hans söner, genom ett saltförbund?
2Ch 13:6  Men Jerobeam, Nebats son, Salomos, Davids sons, tjänare, uppreste sig och avföll från sin herre.
2Ch 13:7  Och till honom församlade sig löst folk, onda män, och de blevo Rehabeam, Salomos son, för starka, eftersom Rehabeam ännu var ung och försagd och därför icke kunde stå dem emot.
2Ch 13:8  Och nu menen I eder kunna stå emot HERRENS konungadöme, som tillhör Davids söner, eftersom I ären en stor hop och haven hos eder de guldkalvar som Jerobeam har låtit göra åt eder till gudar.
2Ch 13:9  Haven I icke fördrivit HERRENS präster, Arons söner, och leviterna, och själva gjort eder präster, såsom de främmande folken göra? Vemhelst som kommer med en ungtjur och sju vädurar for att taga handfyllning, han får bliva präst åt dessa gudar, som icke äro gudar.
2Ch 13:10  Men vi hava HERREN till vår Gud, och vi hava icke övergivit honom. Vi hava präster av Arons söner, som göra tjänst inför HERREN, och leviter, som sköta tempelsysslorna;
2Ch 13:11  och de förbränna åt HERREN brännoffer var morgon och var afton och antända välluktande rökelse och lägga upp bröd på det gyllene bordet och tända var afton den gyllene ljusstaken med dess lampor. Ty vi hålla vad HERREN, vår Gud, har bjudit oss hålla, men I haven övergivit honom.
2Ch 13:12  Och se, vi hava Gud i spetsen för oss, och vi hava hans präster med larmtrumpeterna för att blåsa till strid mot eder. I Israels barn, striden icke mot HERREN, edra fäders Gud; ty då skall det icke gå eder väl."
2Ch 13:13  Men Jerobeam hade låtit kringgå dem och lagt ett bakhåll för att falla dem i ryggen; så stodo de nu mitt emot Juda män och hade sitt bakhåll bakom dem.
2Ch 13:14  När då Juda män vände sig om, fingo de se att de hade fiender både framför sig och bakom sig. Då ropade de till HERREN, och prästerna blåste i trumpeterna.
2Ch 13:15  Därefter hovo Juda män upp ett härskri; och när Juda män hovo upp sitt härskri, lät Gud Jerobeam och hela Israel bliva slagna av Abia och Juda.
2Ch 13:16  Och Israels barn flydde för Juda, och Gud gav dem i deras hand.
2Ch 13:17  Och Abia med sitt folk anställde ett stort nederlag bland dem, så att fem hundra tusen unga män av Israel föllo slagna.
2Ch 13:18  Alltså blevo Israels barn på den tiden kuvade; men Juda barn voro starka, ty de stödde sig på HERREN sina fäders Gud.
2Ch 13:19  Och Abia förföljde Jerobeam och tog ifrån honom några städer: Betel med underlydande orter, Jesana med underlydande orter och Efron med underlydande orter.
2Ch 13:20  Och Jerobeam förmådde ingenting mer, så länge Abia levde; och han blev hemsökt av HERREN, så att han dog.
2Ch 13:21  Men Abia befäste sitt välde; och han tog sig fjorton hustrur och födde tjugutvå söner och sexton döttrar.
2Ch 13:22  Vad nu mer är att säga om Abia, om hans företag och om annat som rör honom, det finnes upptecknat i profeten Iddos "Utläggning".
2Ch 14:1  Och Abia gick till vila hos sina fäder, och man begrov honom i Davids stad. Och hans son Asa blev konung efter honom. Under hans tid hade landet ro i tio år.
2Ch 14:2  Och Asa gjorde vad gott och rätt var i HERRENS, sin Guds, ögon.
2Ch 14:3  Han skaffade bort de främmande altarna och offerhöjderna och slog sönder stoderna och högg ned Aserorna.
2Ch 14:4  Och han uppmanade Juda att söka HERREN, sina fäders Gud, och hålla lagen och budorden.
2Ch 14:5  Ur alla Juda städer skaffade han bort offerhöjderna och solstoderna; och riket hade ro under honom.
2Ch 14:6  Och han byggde fasta städer i Juda, eftersom landet hade ro och han under dessa år icke hade något krig; ty HERREN hade givit honom lugn.
2Ch 14:7  Han sade nämligen till Juda: "Låt oss bygga dessa städer och förse dem runt omkring med murar och torn, med portar och bommar, medan vi ännu hava landet i vår makt, därför att vi hava sökt HERREN, vår Gud; ty vi hava sökt honom, och han har låtit oss få lugn på alla sidor." Så byggde de då, och allt gick väl.
2Ch 14:8  Och Asa hade en här som var väpnad med stora sköldar och med spjut, och som utgjordes av tre hundra tusen man från Juda, vartill kommo två hundra åttio tusen man från Benjamin, som voro väpnade med små sköldar och spände båge. Alla dessa voro tappra stridsmän.
2Ch 14:9  Men Sera från Etiopien drog ut mot dem med en här av tusen gånger tusen man och tre hundra vagnar; och han kom till Maresa.
2Ch 14:10  Och Asa drog ut mot honom, och de ställde upp sig till strid i Sefatas dal vid Maresa.
2Ch 14:11  Och Asa ropade till HERREN, sin Gud, och sade: "HERRE, förutom dig finnes ingen som kan hjälpa i striden mellan den starke och den svage. Så hjälp oss, HERRE, vår Gud, ty på dig stödja vi oss, och i ditt namn hava vi kommit hit mot denna hop. HERRE, du är vår Gud; mot dig förmår ju ingen människa något."
2Ch 14:12  Och HERREN lät etiopierna bliva slagna av Asa och Juda, så att etiopierna flydde.
2Ch 14:13  Och Asa och hans folk förföljde dem ända till Gerar; och av etiopierna föllo så många, att ingen av dem kom undan med livet, ty de blevo nedgjorda av HERREN och hans här. Och folket tog byte i stor myckenhet.
2Ch 14:14  Och de intogo alla städer runt omkring Gerar, ty en förskräckelse ifrån HERREN hade kommit över dessa; och de plundrade alla städerna, ty i dem fanns mycket att plundra.
2Ch 14:15  Till och med boskapsskjulen bröto de ned och förde bort småboskap i myckenhet och kameler, och vände så tillbaka till Jerusalem.
2Ch 15:1  Och över Asarja, Odeds son, kom Guds Ande.
2Ch 15:2  Han gick ut mot Asa och sade till honom: "Hören mig, du, Asa, och I, hela Juda och Benjamin. HERREN är med eder, när I ären med honom, och om I söken honom, så låter han sig finnas av eder; men om I övergiven honom, så övergiver han ock eder.
2Ch 15:3  En lång tid var ju Israel utan den sanne Guden, utan präster som undervisade dem, och utan någon lag.
2Ch 15:4  Men i sin nöd omvände de sig till HERREN, Israels Gud, och när de sökte honom, lät han sig finnas av dem.
2Ch 15:5  Under de tiderna fanns ingen trygghet, när man gick ut eller in; utan stor förvirring rådde bland alla dem som bodde här i länderna,
2Ch 15:6  och folk drabbade samman med folk och stad med stad; ty Gud förvirrade dem med allt slags nöd.
2Ch 15:7  Men varen I frimodiga, låten icke modet falla, ty edert verk skall få sin lön."
2Ch 15:8  När Asa hörde dessa ord och denna profetia av profeten Oded, tog han mod till sig och skaffade bort styggelserna ur Judas och Benjamins hela land och ur de städer som han hade tagit i Efraims bergsbygd, och upprättade åter HERRENS altare, det som stod framför HERRENS förhus.
2Ch 15:9  Och han församlade hela Juda och Benjamin, så ock de främlingar ifrån Efraim, Manasse och Simeon, som bodde ibland dem; ty många från Israel hade gått över till honom, när de sågo att HERREN, hans Gud, var med honom.
2Ch 15:10  Och de församlade sig till Jerusalem i tredje månaden av Asas femtonde regeringsår,
2Ch 15:11  och offrade på den dagen åt HERREN sju hundra tjurar och sju tusen djur av småboskapen, uttagna av det byte som de hade fört med sig.
2Ch 15:12  Och de ingingo det förbundet att de skulle söka HERREN, sina fäders Gud, av allt sitt hjärta och av all sin själ,
2Ch 15:13  och att var och en som icke sökte HERREN, Israels Gud, han skulle bliva dödad, liten eller stor, man eller kvinna.
2Ch 15:14  Och de gåvo HERREN sin ed med hög röst och under jubel, och under det att trumpeter och basuner ljödo.
2Ch 15:15  Och hela Juda gladde sig över eden; ty de hade svurit den av allt sitt hjärta, och de sökte HERREN med hela sin vilja, och han lät sig finnas av dem, och han lät dem få ro på alla sidor.
2Ch 15:16  Konung Asa avsatte ock sin moder Maaka från hennes drottningsvärdighet, därför att hon hade satt upp en styggelse åt Aseran; Asa högg nu ned styggelsen och krossade de och brände upp den i Kidrons dal.
2Ch 15:17  Men offerhöjderna blevo icke avskaffade ur Israel; dock var Asas hjärta gudhängivet, så länge han levde.
2Ch 15:18  Och han förde in i Guds hus både vad hans fader och vad han själv hade helgat åt HERREN: silver, guld och kärl.
2Ch 15:19  Och intet krig uppstod förrän i Asas trettiofemte regeringsår.
2Ch 16:1  I Asas trettiosjätte regeringsår drog Baesa, Israels konung, upp mot Juda och begynte befästa Rama, för att hindra att någon komme vare sig från eller till Asa, Juda konung.
2Ch 16:2  Då tog Asa silver och guld ur skattkamrarna i HERRENS hus och i konungshuset, och sände det till Ben-Hadad, konungen i Aram, som bodde i Damaskus, och lät säga:
2Ch 16:3  "Ett förbund består ju mellan mig och dig, såsom det var mellan min fader och din fader. Se, här sänder jag dig silver och guld; så bryt då ditt förbund med Baesa, Israels konung, för att han må lämna mig i fred."
2Ch 16:4  Och Ben-Hadad lyssnade till konung Asa och sände sina krigshövitsmän mot Israels städer, och de förhärjade Ijon, Dan och Abel-Maim samt alla förrådshus i Naftali städer.
2Ch 16:5  När Baesa hörde detta, avstod han från att befästa Rama och lät sina arbeten där upphöra.
2Ch 16:6  Men konung Asa tog med sig hela Juda, och de förde bort ifrån Rama stenar och trävirke som Baesa använde till att befästa det. Därmed befäste han så Geba och Mispa.
2Ch 16:7  Vid samma tid kom siaren Hanani till Asa, Juda konung, och sade till honom: "Eftersom du stödde dig på konungen i Aram och icke stödde dig på HERREN, din Gud, därför har den arameiske konungens här sluppit undan din hand.
2Ch 16:8  Voro icke etiopierna och libyerna en väldig här, med vagnar och ryttare i stor myckenhet? Men därför att du då stödde dig på HERREN, gav han dem i din hand.
2Ch 16:9  TY HERRENS ögon överfara hela jorden, för att han med sin kraft skall bistå dem som med sina hjärtan hängiva sig åt honom. Härutinnan har du handlat dåraktigt. Därför skall du hädanefter hava ständiga strider."
2Ch 16:10  Men Asa blev förtörnad på siaren och satte honom i stockhuset; så förbittrad var han på honom för vad han hade sagt. Vid samma tid förfor Asa ock våldsamt mot andra av folket.
2Ch 16:11  Men vad som är att säga om Asa, om hans första tid såväl som om hans sista, det finnes upptecknat i boken om Judas och Israels konungar.
2Ch 16:12  Och i sitt trettionionde regeringsår fick Asa en sjukdom i sina fötter, och sjukdomen blev övermåttan svår; men oaktat sin sjukdom sökte han icke HERREN, utan allenast läkares hjälp.
2Ch 16:13  Och Asa gick till vila hos sina fäder och dog i sitt fyrtioförsta regeringsår.
2Ch 16:14  Och man begrov honom i den grav som han hade låtit hugga ut åt sig i Davids stad; och man lade honom på en bädd som man hade fyllt med vällukter och kryddor av olika slag, konstmässigt beredda, och anställde till hans ära en mycket stor förbränning.
2Ch 17:1  Och hans son Josafat blev konung; efter honom. Han befäste sitt välde mot Israel.
2Ch 17:2  Han lade in krigsfolk i alla Juda fasta städer och lade in besättningar i Juda land och i de Efraims städer som hans fader Asa hade intagit.
2Ch 17:3  Och HERREN var med Josafat, ty han vandrade på sin fader Davids första vägar och sökte icke Baalerna,
2Ch 17:4  utan sökte sin faders Gud och vandrade efter hans bud och gjorde icke såsom Israel.
2Ch 17:5  Därför befäste HERREN konungadömet i hans hand, och hela Juda gav skänker åt Josafat, så att hans rikedom och ära blev stor.
2Ch 17:6  Och då hans frimodighet växte på HERRENS vägar, skaffade han också bort offerhöjderna och Aserorna ur Juda.
2Ch 17:7  Och i sitt tredje regeringsår sände han ut sina hövdingar Ben-Hail, Obadja, Sakarja, Netanel och Mikaja, till att undervisa i Juda städer,
2Ch 17:8  och med dem några leviter, nämligen leviterna Semaja, Netanja, Sebadja, Asael, Semiramot, Jonatan, Adonia, Tobia och Tob-Adonia; och de hade med sig prästerna Elisama och Joram.
2Ch 17:9  Dessa undervisade nu i Juda och hade HERRENS lagbok med sig; de foro omkring i alla Juda städer och undervisade bland folket.
2Ch 17:10  Och en förskräckelse ifrån HERREN kom över alla riken i de länder som lågo omkring Juda, så att de icke vågade kriga mot Josafat.
2Ch 17:11  Och en del av filistéerna förde skänker till Josafat och gåvo silver i skatt. Därtill förde ock araberna till honom småboskap, sju tusen sju hundra vädurar och sju tusen sju undra bockar.
2Ch 17:12  Så blev Josafat allt mäktigare och till slut övermåttan mäktig. Och han byggde borgar och förrådsstäder i Juda.
2Ch 17:13  Han hade stora upplag i Juda städer; och krigsfolk, tappra stridsmän, hade han i Jerusalem.
2Ch 17:14  Och detta var ordningen bland dem, efter deras familjer. Till Juda hörde följande överhövitsmän: hövitsmannen Adna och med honom tre hundra tusen tappra stridsmän;
2Ch 17:15  därnäst hövitsmannen Johanan och med honom två hundra åttio tusen;
2Ch 17:16  därnäst Amasja, Sikris son, som frivilligt hade givit sig i HERRENS tjänst, och med honom två hundra tusen tappra stridsmän.
2Ch 17:17  Men från Benjamin voro: Eljada, en tapper stridsman, och med honom två hundra tusen, väpnade med båge och sköld;
2Ch 17:18  därnäst Josabad och med honom ett hundra åttio tusen, rustade till strid.
2Ch 17:19  Dessa voro de som gjorde tjänst hos konungen; därtill kommo de som konungen hade förlagt i de befästa städerna i hela Juda.
2Ch 18:1  När Josafat nu hade kommit till stor rikedom och ära, befryndade han sig med Ahab.
2Ch 18:2  Och efter några års förlopp for han ned till Ahab i Samaria. Och Ahab lät för honom och folket som han hade med sig slakta får och fäkreatur i myckenhet; och han sökte intala honom att draga upp mot Ramot i Gilead.
2Ch 18:3  Ahab, Israels konung, frågade alltså Josafat, Juda konung: "Vill du draga med mig mot Ramot i Gilead?" Han svarade honom: "Jag såsom du, och mitt folk såsom ditt folk! Jag vill följa med dig i striden."
2Ch 18:4  Men Josafat sade ytterligare till Israels konung: "Fråga dock först HERREN härom."
2Ch 18:5  Då församlade Israels konung profeterna, fyra hundra män, och frågade dem: "Skola vi draga åstad till Ramot i Gilead för att belägra det, eller skall jag avstå därifrån?" De svarade: "Drag ditupp; Gud skall giva det i konungens hand."
2Ch 18:6  Men Josafat sade: "Finnes här ingen annan HERRENS profet, så att vi kunna fråga genom honom?"
2Ch 18:7  Israels konung svarade Josafat: "Här finnes ännu en man, Mika, Jimlas son, genom vilken vi kunna fråga HERREN; men han är mig förhatlig, ty han profeterar aldrig lycka åt mig, utan beständigt allenast olycka." Josafat sade: "Konungen säge icke så."
2Ch 18:8  Då kallade Israels konung till sig en hovman och sade: "Skaffa skyndsamt hit Mika, Jimlas son."
2Ch 18:9  Israels konung och Josafat, Juda konung, sutto nu var och en på sin tron, iklädda sina skrudar; de sutto på en tröskplats vid Samarias port, under det att alla profeterna profeterade inför dem.
2Ch 18:10  Då gjorde sig Sidkia, Kenaanas son, horn av järn och sade: "Så säger HERREN: Med dessa skall du stånga araméerna, så att de förgöras."
2Ch 18:11  Och alla profeterna profeterade på samma sätt och sade: "Drag upp mot Ramot i Gilead, så skall du bliva lyckosam; HERREN skall giva det i konungens hand."
2Ch 18:12  Och budet som hade gått för att kalla på Mika talade till honom och sade: "Det är så, att profeterna med en mun lova konungen lycka; så låt nu ock ditt tal stämma överens med deras, och lova också du lycka."
2Ch 18:13  Men Mika svarade: "Så sant HERREN lever, jag skall allenast tala det som min Gud säger."
2Ch 18:14  När han sedan kom till konungen, frågade konungen honom: "Mika, skola vi draga åstad till Ramot i Gilead för att belägra det, eller skall jag avstå därifrån?" Han svarade: "Dragen ditupp, så skolen I bliva lyckosamma; de skola bliva givna i eder hand."
2Ch 18:15  Men konungen sade till honom: "Huru många gånger skall jag besvärja dig att icke tala till mig annat än sanning i HERRENS namn?"
2Ch 18:16  Då sade han: "Jag såg hela Israel förskingrat på bergen, likt får som icke hava någon herde. Och HERREN sade: 'Dessa hava icke någon herre; må de vända tillbaka hem i frid, var och en till sitt.'"
2Ch 18:17  Då sade Israels konung till Josafat: "Sade jag dig icke att denne aldrig profeterar lycka åt mig, utan allenast olycka?"
2Ch 18:18  Men han sade: "Hören alltså HERRENS ord. Jag såg HERREN sitta på sin tron och himmelens hela härskara stå på hans högra sida och på hans vänstra.
2Ch 18:19  Och HERREN sade: 'Vem vill locka Ahab, Israels konung, att draga upp mot Ramot i Gilead, för att han må falla där?' Då sade den ene så och den andre så.
2Ch 18:20  Slutligen kom anden fram och ställde sig inför HERREN och sade: 'Jag vill locka honom därtill.' HERREN frågade honom: 'På vad sätt?'
2Ch 18:21  Han svarade: 'Jag vill gå ut och bliva en lögnens ande i alla hans profeters mun." Då sade han: 'Du må försöka att locka honom därtill och du skall också lyckas; gå ut och gör så.'
2Ch 18:22  Och se, nu har HERREN lagt en lögnens ande i dessa dina profeters mun, medan HERREN ändå har beslutit att olycka skall komma över dig."
2Ch 18:23  Då trädde Sidkia, Kenaanas son, fram och gav Mika ett slag på kinden och sade: "På vilken väg har då HERRENS Ande gått bort ifrån mig för att tala med dig?"
2Ch 18:24  Mika svarade: "Du skall få se det på den dag då du nödgas springa från kammare till kammare för att gömma dig."
2Ch 18:25  Men Israels konung sade: "Tagen Mika och fören honom tillbaka till Amon, hövitsmannen i staden, och till Joas, konungasonen.
2Ch 18:26  Och sägen: Så säger konungen: Sätten denne i fängelse och bespisen honom med fångkost, till dess jag kommer välbehållen tillbaka."
2Ch 18:27  Mika svarade: "Om du kommer välbehållen tillbaka, så har HERREN icke talat genom mig." Och han sade ytterligare: "Hören detta, I folk, allasammans."
2Ch 18:28  Så drog nu Israels konung jämte Josafat, Juda konung, upp till Ramot i Gilead.
2Ch 18:29  Och Israels konung sade till Josafat: "Jag vill förkläda mig, när jag drager ut i striden, men du må vara klädd i dina egna kläder." Så förklädde sig Israels konung, när de drogo ut i striden.
2Ch 18:30  Men konungen i Aram hade bjudit och sagt till sina vagnshövitsmän: "I skolen icke giva eder i strid med någon, vare sig liten eller stor, utom med Israels konung allena."
2Ch 18:31  När då hövitsmannen över vagnarna fingo se Josafat, tänkte de: "Detta är Israels konung", och omringade honom därför, i avsikt att anfalla honom. Då gav Josafat upp ett rop, och HERREN hjälpte honom, Gud vände dem bort ifrån honom.
2Ch 18:32  Så snart nämligen hövitsmännen över vagnarna märkte att det icke var Israels konung, vände de om och läto honom vara.
2Ch 18:33  Men en man som spände sin båge och sköt på måfå träffade Israels konung i en fog på rustningen. Då sade denne till sin körsven: "Sväng om vagnen och för mig ut ur hären, ty jag är sårad."
2Ch 18:34  Och striden blev på den dagen allt häftigare, och Israels konung höll sig ända till aftonen upprätt i sin vagn, vänd mot araméerna; men vid den tid då solen gick ned gav han upp andan.
2Ch 19:1  Men Josafat, Juda konung, vände välbehållen hem igen till Jerusalem.
2Ch 19:2  Då gick siaren Jehu, Hananis son, ut mot konung Josafat och sade till honom: "Skall man då hjälpa den ogudaktige? Skall du då älska dem som hata HERREN? För vad du har gjort vilar nu HERRENS förtörnelse över dig.
2Ch 19:3  Dock har något gott blivit funnet hos dig, ty du har utrotat Aserorna ur landet och har vänt ditt hjärta till att söka Gud."
2Ch 19:4  Och Josafat stannade nu i Jerusalem, men sedan drog han åter ut bland folket, ifrån Beer-Seba ända till Efraims bergsbygd, och förde dem tillbaka till HERREN, deras fäders Gud.
2Ch 19:5  Och han anställde domare i landet, i alla Juda befästa städer, särskilda för var stad.
2Ch 19:6  Och han sade till dessa domare: "Sen till, vad I gören; ty I dömen icke människodom, utan HERRENS dom, och han är närvarande, så ofta I dömen.
2Ch 19:7  Låten alltså nu fruktan för HERREN vara över eder. Given akt på vad I gören; ty hos HERREN, vår Gud, finnes ingen orätt, och han har icke anseende till personen, ej heller tager han mutor."
2Ch 19:8  Också i Jerusalem hade Josafat anställt några av leviterna och prästerna och några av huvudmännen för Israels familjer till att döma HERRENS dom och avgöra rättstvister. När de sedan vände tillbaka till Jerusalem,
2Ch 19:9  bjöd han dem och sade: "Så skolen I göra i HERRENS fruktan, redligt och med hängivet hjärta.
2Ch 19:10  Och så ofta någon rättssak drages inför eder av edra bröder, som bo i sina städer, det må gälla dom i en blodssak eller eljest tillämpning av lag och bud, stadgar och rätter, då skolen I varna dem, så att de icke ådraga sig skuld inför HERREN, varigenom förtörnelse kommer över eder och edra bröder. Så skolen I göra, för att I icke mån ådraga eder skuld.
2Ch 19:11  Och se, översteprästen Amarja skall vara eder förman i alla HERRENS saker, och Sebadja, Ismaels son, fursten för Juda hus, i alla konungens saker; och leviterna skola vara tillsyningsmän under eder. Varen nu ståndaktiga i vad I gören, och HERREN skall vara med den som är god."
2Ch 20:1  Därefter kommo Moabs barn och Ammons barn och med dem en del av ammoniterna för att strida mot Josafat.
2Ch 20:2  Och man kom och berättade detta för Josafat och sade: "En stor hop kommer mot dig från landet på andra sidan havet, från Aram, och de äro redan i Hasason-Tamar (det är En-Gedi)."
2Ch 20:3  Då blev Josafat förskräckt och vände sin håg till att söka HERREN; och han lät lysa ut en fasta över hela Juda.
2Ch 20:4  Och Juda församlade sig för att söka hjälp hos HERREN; ja, från alla Juda städer kom man för att söka HERREN.
2Ch 20:5  Och Josafat trädde upp i Juda mäns och Jerusalems församling i HERRENS hus, framför den nya förgården,
2Ch 20:6  och sade: "HERRE, våra fäders Gud, är icke du Gud i himmelen och den som råder över alla hednafolkens riken? I din hand är kraft och makt; och ingen finnes, som kan stå dig emot.
2Ch 20:7  Var det icke du, vår Gud, som fördrev detta lands inbyggare för ditt folk Israel och gav det åt Abrahams, din väns, säd för evig tid?
2Ch 20:8  De fingo bo där, och de byggde dig där en helgedom åt ditt namn, i det de sade:
2Ch 20:9  'Om något ont kommer över oss, svärd, straffdom eller pest eller hungersnöd, så vilja vi träda upp inför detta hus och inför dig, ty ditt namn är i detta hus; och vi vilja ropa till dig i vår nöd, och du skall då höra och hjälpa.'
2Ch 20:10  Se därför nu huru Ammons barn och Moab och folket i Seirs bergsbygd - genom vilkas område du icke tillstadde Israel att gå, när de kommo från Egyptens land, varför de ock togo en omväg bort ifrån dem och icke förgjorde dem -
2Ch 20:11  se huru dessa nu vedergälla oss, i det att de komma för att förjaga oss ur det land som är din besittning, och som du har givit oss till besittning.
2Ch 20:12  Du, vår Gud, skall du icke hålla dom över dem? Ty vi förmå intet mot denna stora hop som kommer emot oss, och själva veta vi icke vad vi skola göra, utan till dig se våra ögon."
2Ch 20:13  Och hela Juda stod där inför HERREN med sina späda barn, sina hustrur och söner.
2Ch 20:14  Då kom HERRENS Ande mitt i församlingen över Jahasiel, son till Sakarja, son till Benaja, son till Jegiel, son till Mattanja, en levit, av Asafs söner,
2Ch 20:15  och han sade: "Akten härpå, alla I av Juda, och I Jerusalems invånare, och du konung Josafat. Så säger HERREN till eder: Frukten icke och varen icke förfärade för denna stora hop, ty striden är icke eder, utan Guds.
2Ch 20:16  Dragen i morgon ned mot dem. De draga då upp på Hassishöjden, och I skolen träffa dem vid andan av dalen, framför Jeruels öken.
2Ch 20:17  Men därvid bliver det icke eder sak att strida. I skolen allenast träda fram och stå stilla och se på, huru HERREN frälsar eder, I av Juda och Jerusalem. Frukten icke och varen icke förfärade. Dragen i morgon ut mot dem, och HERREN skall vara med eder."
2Ch 20:18  Då böjde Josafat sig ned med ansiktet mot jorden, och alla Juda män och Jerusalems invånare föllo ned för HERREN och tillbådo HERREN.
2Ch 20:19  Och de av leviterna, som tillhörde kehatiternas och koraiternas barn, stodo upp och lovade HERREN, Israels Gud, med hög och stark röst.
2Ch 20:20  Men bittida följande morgon drogo de ut till Tekoas öken. Och när de drogo ut, trädde Josafat fram och sade: "Hören mig, I av Juda och I Jerusalems invånare. Haven tro på HERREN, eder Gud, så skolen I hava ro. Och tron på hans profeter, så skolen I bliva lyckosamma."
2Ch 20:21  Och sedan han hade rådfört sig med folket, ställde han upp män som skulle sjunga till HERRENS ära och lova honom i helig skrud, under det att de drogo ut framför den väpnade hären; de skulle sjunga: "Tacken HERREN, ty hans nåd varar evinnerligen."
2Ch 20:22  Och just som de begynte med sången och lovet, lät HERREN ett angrepp ske bakifrån på Ammons barn och Moab och folket ifrån Seirs bergsbygd, dem som hade kommit mot Juda; och de blevo slagna.
2Ch 20:23  Och Ammons barn och Moab reste sig mot folket ifrån Seirs bergsbygd och gåvo dem till spillo och förgjorde dem; och när de hade gjort ände på folket ifrån Seir, hjälptes de åt att nedgöra varandra.
2Ch 20:24  När sedan Juda män kommo upp på höjden, varifrån man kunde se ut över öknen, och vände sig mot fiendernas hop, fingo de se dessa ligga döda på jorden, och ingen hade undkommit.
2Ch 20:25  Och när Josafat begav sig dit med sitt folk för att plundra och taga byte från dem, funno de där en myckenhet av gods och av döda kroppar och av dyrbara ting; och de togo for sig så mycket att de icke kunde bära det. Och de fortsatte plundringen i tre dagar; så stort var bytet.
2Ch 20:26  Men på fjärde dagen församlade de sig i Berakadalen; där lovade de HERREN, och därav fick det stället namnet Berakadalen, såsom det heter ännu i dag.
2Ch 20:27  Därefter vände alla Judas och Jerusalems män, med Josafat i spetsen, glada tillbaka igen till Jerusalem; ty HERREN hade berett dem glädje genom vad som hade skett med deras fiender.
2Ch 20:28  Och de drogo in i Jerusalem med psaltare, harpor och trumpeter och tågade till HERRENS hus.
2Ch 20:29  Och en förskräckelse ifrån Gud kom över alla de främmande rikena, när de hörde att HERREN hade stritt mot Israels fiender.
2Ch 20:30  Och Josafats rike hade nu ro, ty hans Gud lät honom få lugn på alla sidor.
2Ch 20:31  Så regerade Josafat över Juda. han var trettiofem år gammal, när han blev konung, och han regerade tjugufem år i Jerusalem. Hans moder hette Asuba, Silhis dotter.
2Ch 20:32  Och han vandrade på sin fader Asas väg, utan att vika av ifrån den; han gjorde nämligen vad rätt var i HERRENS ögon.
2Ch 20:33  Dock blevo offerhöjderna icke avskaffade, och ännu hade folket icke vänt sina hjärtan till sina fäders Gud.
2Ch 20:34  Vad nu mer är att säga om Josafat, om hans första tid såväl som om hans sista, det finnes upptecknat i Jehus, Hananis sons, krönika, som är upptagen i boken om Israels konungar.
2Ch 20:35  Men sedan förband sig Josafat, Juda konung, med Ahasja, Israels konung, fastän denne var ogudaktig i sina gärningar;
2Ch 20:36  han förband sig med honom för att bygga skepp som skulle gå till Tarsis. Och de byggde skepp i Esjon-Geber.
2Ch 20:37  Då profeterade Elieser, Dodavahus son, från Maresa, mot Josafat; han sade: "Därför att du har förbundit dig med Ahasja, skall HERREN låta ditt företag bliva om intet." Och somliga av skeppen ledo skeppsbrott, så att de icke kunde gå till Tarsis.
2Ch 21:1  Och Josafat gick till vila hos sina fäder och blev begraven hos sina fäder i Davids stad. Och hans son Joram blev konung efter honom.
2Ch 21:2  Denne hade bröder, söner till Josafat: Asarja, Jehiel, Sakarja, Asarjahu, Mikael och Sefatja; alla dessa voro söner till Josafat, Israels konung.
2Ch 21:3  Och deras fader gav dem stora skänker i silver och guld och dyrbarheter, därtill ock fasta städer i Juda; men konungadömet hade han givit åt Joram, ty denne var den förstfödde.
2Ch 21:4  När Joram nu hade övertagit sin faders konungadöme och befäst sig däri, dräpte han alla sina bröder med svärd, så ock några av Israels furstar,
2Ch 21:5  Joram var trettiotvå år gammal, är han blev konung, och han regerade åtta år i Jerusalem.
2Ch 21:6  Men han vandrade på Israels konungars väg, såsom Ahabs hus hade gjort, ty en dotter till Ahab var hans hustru; han gjorde vad ont var i HERRENS ögon.
2Ch 21:7  Dock ville HERREN icke fördärva Davids hus, för det förbunds skull som han hade slutit med David, och enligt sitt löfte, att han skulle låta honom och hans söner hava en lampa för alltid.
2Ch 21:8  I hans tid avföll Edom från Juda välde och satte en egen konung över sig.
2Ch 21:9  Då drog Joram dit med sina hövitsmän och med alla sina stridsvagnar. Och om natten gjorde han ett anfall på edoméerna, som hade omringat honom, och slog dem och hövitsmännen över deras vagnar.
2Ch 21:10  Så avföll Edom från Juda välde, och det har varit skilt därifrån ända till denna dag. Vid samma tid avföll ock Libna från hans välde, därför att han hade övergivit HERREN, sina fäders Gud.
2Ch 21:11  Också han uppförde offerhöjder på bergen i Juda och förledde så Jerusalems invånare till trolös avfällighet och förförde Juda.
2Ch 21:12  Men en skrivelse kom honom till handa från profeten Elia, så lydande: "Så säger HERREN, din fader Davids Gud: Se, du har icke vandrat på din fader Josafats vägar eller på Asas, Juda konungs, vägar,
2Ch 21:13  utan du har vandrat på Israels konungars väg och förlett Juda och Jerusalems invånare till trolös avfällighet, på samma sätt som Ahabs hus förledde till avfällighet; du har också dräpt dina bröder, dem som hörde till din faders hus, och som voro bättre än du.
2Ch 21:14  Därför skall HERREN låta en stor hemsökelse drabba ditt folk, så ock dina barn och dina hustrur och allt vad du äger;
2Ch 21:15  och själv skall du träffas av svår sjukdom, en sjukdom i dina inälvor, så svår att dina inälvor, efter år och dagar, skola falla ut i följd av sjukdomen."
2Ch 21:16  Och HERREN uppväckte mot Joram filistéernas ande och de arabers som bodde närmast etiopierna;
2Ch 21:17  och de drogo upp mot Juda och bröto in där och förde bort allt gods som fanns i konungens hus, därtill ock hans söner och hustrur, så att han icke hade kvar någon av sina söner förutom Joahas, sin yngste son.
2Ch 21:18  Och efter allt detta hemsökte HERREN honom med en obotlig sjukdom i inälvorna.
2Ch 21:19  Och efter år och dagar, när två år voro förlidna, föllo hans inälvor ut i följd av sjukdomen, och han dog i svåra plågor; men hans folk anställde ingen förbränning till hans ära, såsom de hade gjort efter hans fäder.
2Ch 21:20  Han var trettiotvå år gammal, när han blev konung, och han regerade åtta år i Jerusalem. Och han gick bort utan att bliva saknad, och man begrov honom i Davids stad, men icke i konungagravarna.
2Ch 22:1  Och Jerusalems invånare gjorde Ahasja, hans yngste son, till konung efter honom; ty alla de äldre hade blivit dräpta av den rövarskara som med araberna hade kommit till lägret. Så blev då Ahasja, Jorams son, konung i Juda.
2Ch 22:2  Fyrtiotvå år gammal var Ahasja, är han blev konung, ock han regerade ett år i Jerusalem. Hans moder hette Atalja, Omris dotter.
2Ch 22:3  Också han vandrade på Ahabs hus' vägar, ty hans moder var hans rådgiverska i ogudaktighet.
2Ch 22:4  Han gjorde vad ont var i HERRENS ögon likasom Ahabs hus; ty därifrån tog han, efter sin faders död, sina rådgivare, till sitt eget fördärv.
2Ch 22:5  Det var ock deras råd han följde, när han drog åstad med Joram, Ahabs son, Israels konung, och stridde mot Hasael, konungen i Aram, vid Ramot i Gilead. Men Joram blev sårad av araméerna.
2Ch 22:6  Då vände han tillbaka, för att i Jisreel låta hela sig från de sår som han hade fått vid Rama, i striden mot Hasael, konungen i Aram. Och Asarja, Jorams son, Juda konung, for ned för att besöka Joram, Ahabs son, i Jisreel, eftersom denne låg sjuk.
2Ch 22:7  Men till Ahasjas fördärv var det av Gud bestämt att han skulle komma till Joram. Ty när han hade kommit dit, for han med Joram för att möta Jehu, Nimsis son, som HERREN hade smort till att utrota Ahabs hus.
2Ch 22:8  Så hände sig att Jehu, när han utförde straffdomen över Ahabs hus, träffade på de Juda furstar och de brorsöner till Ahasja, som voro i Ahasjas tjänst, och dräpte dem.
2Ch 22:9  Sedan sökte han efter Ahasja; och man grep denne, där han höll sig gömd i Samaria, och förde honom till Jehu och dödade honom. Men därefter begrovo de honom, ty de sade: "Han var dock son till Josafat, som sökte HERREN av allt sitt hjärta." Och av Ahasjas hus fanns sedan ingen dom förmådde övertaga konungadömet.
2Ch 22:10  När nu Atalja, Ahasjas moder, förnam att hennes son var död, stod hon upp och förgjorde hela konungasläkten i Juda hus.
2Ch 22:11  Men just när konungabarnen skulle dödas, tog konungadottern Josabeat Joas, Ahasjas son, och skaffade honom hemligen undan, i det att han förde honom jämte hans amma in i sovkammaren; där höll Josabeat, konung Jorams dotter, prästen Jojadas hustru - som ju ock var Ahasjas syster - honom dold för Atalja, så att denna icke fick döda honom.
2Ch 22:12  Sedan var han hos dem i Guds hus, där han förblev gömd i sex år, medan Atalja regerade i landet.
2Ch 23:1  Men i det sjunde året tog Jojada mod till sig och förband sig med underhövitsmännen Asarja, Jerohams son, Ismael, Johanans son, Asarja, Obeds son, Maaseja, Adajas son, och Elisafat, Sikris son.
2Ch 23:2  Dessa foro därefter omkring i Juda och församlade leviterna ur alla Juda städer, så ock huvudmännen för Israels familjer. Och när de kommo till Jerusalem,
2Ch 23:3  Slöt hela församlingen i Guds hus ett förbund med konungen. Och Jojada sade till dem: "Konungens son skall nu vara konung, såsom HERREN har talat angående Davids söner.
2Ch 23:4  Detta är alltså vad I skolen göra: en tredjedel av eder, nämligen de präster och leviter som hava att inträda i vakthållningen på sabbaten, skall stå på vakt vid trösklarna
2Ch 23:5  och en tredjedel vid konungshuset och en tredjedel vid Jesodporten; och allt folket skall vara på förgårdarna till HERRENS hus.
2Ch 23:6  Dock må ingen annan än prästerna och de tjänstgörande leviterna gå in i HERRENS hus; dessa må gå in, ty de äro heliga. Men allt det övriga folket skall iakttaga vad HERREN har bjudit dem iakttaga.
2Ch 23:7  Och leviterna skola ställa sig runt omkring konungen, var och en med sina vapen i handen; och om någon vill tränga sig in i huset, skall han dödas. Och I skolen följa konungen, vare sig han går in eller ut."
2Ch 23:8  Leviterna och hela Juda gjorde allt vad prästen Jojada hade bjudit dem, var och en av dem tog sina män, både de som skulle inträda i vakthållningen på sabbaten och de som skulle avgå därifrån på sabbaten, ty prästen Jojada lät ingen avdelning vara fri ifrån tjänstgöring.
2Ch 23:9  Och prästen Jojada gav åt underhövitsmännen de spjut och de sköldar av olika slag, som hade tillhört konung David, och som funnos i Guds hus.
2Ch 23:10  Och han ställde upp allt folket, var och en med sitt vapen i handen, från husets södra sida till husets norra sida, mot altaret och mot huset, runt omkring konungen.
2Ch 23:11  Därefter förde de ut konungasonen och satte på honom kronan och gåvo honom vittnesbördet och gjorde honom till konung; och Jojada och hans söner smorde honom och ropade: "Leve konungen!"
2Ch 23:12  När Atalja nu hörde folkets rop, då de skyndade fram och hyllade konungen, gick hon in i HERRENS hus till folket.
2Ch 23:13  Där fick hon då se konungen stå vid sin pelare, nära ingången, och hövitsmännen och trumpetblåsarna bredvid konungen, och fick höra huru hela folkmängden jublade och stötte i trumpeterna, och huru sångarna med sina instrumenter ledde hyllningssången. Då rev Atalja sönder sina kläder och ropade: "Sammansvärjning! Sammansvärjning!"
2Ch 23:14  Men prästen Jojada lät underhövitsmännen som anförde skaran träda fram, och han sade till dem: "Fören henne ut mellan leden, och om någon följer henne, så må han dödas med svärd." Prästen förbjöd dem nämligen att döda henne i HERRENS hus.
2Ch 23:15  Alltså grepo de henne, och när hon hade kommit fram dit där Hästporten för in i konungshuset, dödade de henne där.
2Ch 23:16  Och Jojada slöt ett förbund mellan sig och allt folket och konungen, att de skulle vara ett HERRENS folk.
2Ch 23:17  Och allt folket begav sig till Baals tempel och rev ned det och slog sönder dess altaren och bilder; och Mattan, Baals präst, dräpte de framför altarna.
2Ch 23:18  Därefter ställde Jojada ut vakter vid HERRENS hus och betrodde detta värv åt de levitiska prästerna, dem som David hade indelat i klasser för tjänstgöringen i HERRENS hus, till att offra brännoffer åt HERREN, såsom det var föreskrivet i Moses lag, med jubel och sång, efter Davids anordning.
2Ch 23:19  Och han ställde dörrvaktarna vid portarna till HERRENS hus, för att ingen skulle komma in, som på något sätt var oren.
2Ch 23:20  Och han tog med sig underhövitsmännen och de förnämsta och mäktigaste bland folket och hela folkmängden och förde konungen ned från HERRENS hus, och de gingo in i konungshuset genom Övre porten; och de satte konungen på konungatronen.
2Ch 23:21  Och hela folkmängden gladde sig, och staden förblev lugn. Men Atalja hade de dödat med svärd.
2Ch 24:1  Joas var sju år gammal, när han blev konung, och han regerade fyrtio år i Jerusalem. Hans moder hette Sibja, från Beer-Seba.
2Ch 24:2  Och Joas gjorde vad rätt var i HERRENS ögon, så länge prästen Jojada levde.
2Ch 24:3  Och Jojada tog åt honom två hustrur, och han födde söner och döttrar.
2Ch 24:4  Därefter blev Joas betänkt på att upphjälpa HERRENS hus.
2Ch 24:5  Och han församlade prästerna och leviterna och sade till dem: "Faren vart år ut till Juda städer, och samlen från hela Israel in penningar till att sätta eder Guds hus i stånd; och I skolen bedriva denna sak med skyndsamhet." Men leviterna skyndade sig icke.
2Ch 24:6  Då kallade konungen till sig översteprästen Jojada och sade till honom: "Varför har du icke tillhållit leviterna att från Juda och Jerusalem indriva den skatt som HERRENS tjänare Mose pålade, och som Israels församling skulle erlägga till vittnesbördets tabernakel?
2Ch 24:7  Ty Ataljas, den ogudaktiga kvinnans, söner hava fördärvat Gud hus; ja, allt som var helgat till HERRENS hus hava de använt till Baalerna."
2Ch 24:8  På konungens befallning gjorde man därefter en kista och ställde den utanför porten till HERREN hus.
2Ch 24:9  Och man lät utropa i Juda och Jerusalem att den skatt som Guds tjänare Mose hade pålagt Israel i öknen skulle erläggas åt HERREN.
2Ch 24:10  Och alla furstarna och allt folket buro fram penningar med glädje och kastade dem i kistan, till dess att allt var insamlat.
2Ch 24:11  Och när tid blev att genom leviternas försorg föra kistan till de granskningsmän som konungen hade förordnat, och dessa då märkte att mycket penningar fanns i den, då kommo konungens sekreterare och översteprästens tillsyningsman och tömde kistan och buro den sedan tillbaka till dess plats. Så gjorde de gång efter annan och samlade in penningar i myckenhet.
2Ch 24:12  Därefter lämnade konungen och Jojada dessa åt den som skulle utföra arbetet på HERRENS hus, och lejde stenhuggare och timmermän till att upphjälpa HERRENS hus, så ock järn- och kopparsmeder till att sätta HERRENS hus i stånd.
2Ch 24:13  Och de som utförde arbetet bedrevo det så, att arbetet gick framåt under deras händer, Och de återställde Guds hus i dess förra skick och satte det i gott stånd.
2Ch 24:14  Och när de hade slutat, buro de återstoden av penningarna till konungen och Jojada; och man gjorde därav kärl till HERRENS hus, kärl till gudstjänsten och offren, skålar och andra kärl av guld och silver Och man offrade brännoffer i HERRENS hus beständigt, så länge Jojada levde.
2Ch 24:15  Men Jojada blev gammal och mätt på att leva och dog så; ett hundra trettio år gammal var han vid sin död.
2Ch 24:16  Och man begrov honom i Davids stad bland konungarna, därför att han hade gjort vad gott var mot Israel och mot Gud och hans hus.
2Ch 24:17  Men efter Jojadas död kommo Juda furstar och föllo ned för konungen; då lyssnade konungen till dem.
2Ch 24:18  Och de övergåvo HERRENS, sina fäders Guds, hus och tjänade Aserorna och avgudarna. Då kom förtörnelse över Juda och Jerusalem genom den skuld de så ådrogo sig.
2Ch 24:19  Och profeter sändes ibland dem för att omvända dem till HERREN; och dessa varnade dem, men de lyssnade icke därtill.
2Ch 24:20  Men Sakarja, prästen Jojadas son, hade blivit beklädd med Guds Andes kraft, och han trädde fram inför folket och sade till dem: "Så säger Gud: Varför överträden I HERRENS bud, eder själva till ingen fromma? Eftersom I haven övergivit HERREN, har han ock övergivit eder."
2Ch 24:21  Då sammansvuro de sig mot honom och stenade honom, enligt konungens befallning, på förgården till HERRENS hus.
2Ch 24:22  Ty konung Joas tänkte icke på den kärlek som Jojada, dennes fader, hade bevisat honom, utan dräpte hans son. Men denne sade i sin dödsstund: "Må HERREN se detta och utkräva det."
2Ch 24:23  Och när året hade gått till ända, drog araméernas här upp mot honom, och de kommo till Juda och Jerusalem och utrotade ur folket alla folkets furstar. Och allt byte som de togo sände de till konungen i Damaskus.
2Ch 24:24  Ty fastän araméernas här som då ryckte an utgjorde allenast en ringa skara, gav HERREN likväl i deras hand en mycket talrik här, därför att folket hade övergivit HERREN, sina fäders Gud. Så fingo de utföra straffdomen över Joas.
2Ch 24:25  Och när dessa drogo bort ifrån honom - ty de lämnade honom kvar illa sjuk - sammansvuro sig hans tjänare mot honom, därför att han hade utgjutit prästen Jojadas söners blod, och dräpte honom på hans säng; detta blev hans död. Och man begrov honom i Davids stad; dock begrov man honom icke i konungagravarna.
2Ch 24:26  Och de som sammansvuro sig mot honom voro Sabad, son till ammonitiskan Simeat, och Josabad, son till moabitiskan Simrit.
2Ch 24:27  Men om hans söner, och om de många profetior som förkunnades mot honom, och om huru Guds hus åter upprättades, härom är skrivet i "Utläggning av Konungaboken". Och hans son Amasja blev konung efter honom.
2Ch 25:1  Amasja var tjugufem år gammal, när han blev konung, och han regerade tjugunio år i Jerusalem. Hans moder hette Joaddan, från Jerusalem.
2Ch 25:2  Han gjorde vad rätt var i HERRENS ögon, dock icke av fullt hängivet hjärta.
2Ch 25:3  Och sedan hans konungadöme hade blivit befäst, lät han dräpa dem av sina tjänare, som hade dödat hans fader, konungen.
2Ch 25:4  Men deras barn dödade han icke, utan handlade i enlighet med vad föreskrivet var i Moses lagbok, där HERREN hade bjudit och sagt: "Föräldrarna skola icke dö för sina barns skull, och barnen skola icke dö för sina föräldrars skull, utan var och en skall dö genom sin egen synd."
2Ch 25:5  Och Amasja församlade Juda barn och lät dem ställa upp sig efter sina familjer, efter sina över- och under- hövitsmän, hela Juda och Benjamin. Därefter inmönstrade han dem som voro tjugu år gamla eller därutöver, och fann dem utgöra tre hundra tusen utvalda stridbara män, som kunde föra spjut och sköld.
2Ch 25:6  Därtill lejde han för hundra talenter silver ett hundra tusen tappra stridsmän ur Israel.
2Ch 25:7  Men en gudsman kom till honom och sade: "O konung, låt icke Israels här draga åstad med dig, ty HERREN är icke med Israel, icke med hela hopen av Efraims barn;
2Ch 25:8  utan du själv må allena draga åstad. Grip verket an, gå frimodigt ut i striden. Gud skall eljest låta dig komma på fall genom fienden; ty Gud förmår både att hjälpa och att stjälpa."
2Ch 25:9  Amasja sade till gudsmannen: "Men huru skall det då gå med de hundra talenterna som jag har givit åt skaran från Israel?" Gudsmannen svarade: "HERREN kan väl giva dig mer än det."
2Ch 25:10  Då avskilde Amasja den skara som hade kommit till honom från Efraim och lät dem gå hem igen. Häröver blevo dessa högeligen förgrymmade på Juda och vände tillbaka hem i vredesmod.
2Ch 25:11  Men Amasja tog mod till sig och tågade ut med sitt folk och drog till Saltdalen och nedgjorde där av Seirs barn tio tusen.
2Ch 25:12  Och Juda barn togo andra tio tusen till fånga levande; dem förde de upp på spetsen av en klippa och störtade dem ned från klippspetsen, så att de alla krossades.
2Ch 25:13  Men de som tillhörde den skara som Amasja hade sänt tillbaka, och som icke hade fått gå med honom ut i striden, företogo plundringståg i Juda städer, från Samaria ända till Bet-Horon; och de nedgjorde tre tusen av invånarna och togo stort byte.
2Ch 25:14  När sedan Amasja kom tillbaka från sin seger över edoméerna, förde han med sig Seirs barns gudar och ställde upp dem till gudar åt sig; och han tillbad inför dem och tände offereld åt dem.
2Ch 25:15  Då upptändes HERRENS vrede mot Amasja, och han sände till honom en profet; denne sade till honom: "Varför söker du detta folks gudar, som ju icke hava kunnat rädda sitt eget folk ur din hand?"
2Ch 25:16  När denne så talade till honom, svarade han honom: "Hava vi satt dig till konungens rådgivare? Håll upp, om du icke vill att man skall dräpa dig." Då höll profeten upp och sade: "Jag förstår nu att Gud har beslutit att fördärva dig, eftersom du gör på detta sätt och icke vill höra på mitt råd."
2Ch 25:17  Och sedan Amasja, Juda konung, hade hållit rådplägning, sände han till Joas, son till Joahas, son till Jehu, Israels konung, och lät säga: "Kom, låt oss drabba samman med varandra."
2Ch 25:18  Men Joas, Israels konung, sände då till Amasja, Juda konung, och lät svara: "Törnbusken på Libanon sände en gång bud till cedern på Libanon och lät säga: 'Giv din dotter åt min son till hustru.' Men sedan gingo markens djur på Libanon fram över törnbusken och trampade ned den.
2Ch 25:19  Du tänker på huru du har slagit Edom, och däröver förhäver du dig ditt hjärta och vill vinna ännu mer ära. Men stanna nu hemma. Varför utmanar du olyckan, dig själv och Juda med dig till fall?"
2Ch 25:20  Men Amasja ville icke höra härpå, ty Gud skickade det så, för att de skulle bliva givna i fiendehand, eftersom de hade sökt Edoms gudar.
2Ch 25:21  Så drog då Joas, Israels konung, upp, och de drabbade samman med varandra, han och Amasja, Juda konung, vid det Bet-Semes som hör till Juda.
2Ch 25:22  Och Juda män blevo slagna av Israels män och flydde, var och en till sin hydda.
2Ch 25:23  Och Amasja, Juda konung, son till Joas, son till Joahas, blev tagen till fånga i Bet-Semes av Joas, Israels konung. Och när denne hade fört honom till Jerusalem, bröt han ned ett stycke av Jerusalems mur, från Efraimsporten ända till Poneporten, fyra hundra alnar.
2Ch 25:24  Och han tog allt guld och silver och alla kärl som funnos i Guds hus, hos Obed-Edom, och konungshusets skatter, därtill ock gisslan, och vände så tillbaka till Samaria.
2Ch 25:25  Men Amasja, Joas' son, Juda konung, levde i femton år efter Joas', Joahas' sons, Israels konungs, död.
2Ch 25:26  Vad nu mer är att säga om Amasja, om hans första tid såväl som om hans sista, det finnes upptecknat i boken om Judas och Israels konungar.
2Ch 25:27  Och från den tid då Amasja vek av ifrån HERREN begynte man anstifta en sammansvärjning mot honom i Jerusalem, så att han måste fly till Lakis. Då sändes män efter honom till Lakis, och dessa dödade honom där.
2Ch 25:28  Sedan förde man honom därifrån på hästar och begrov honom hos hans fäder i Juda huvudstad.
2Ch 26:1  Och allt folket i Juda tog Ussia, som då var sexton år gammal, och gjorde honom till konung i hans fader Amasjas ställe.
2Ch 26:2  Det var han som befäste Elot, och han lade det åter under Juda, sedan konungen hade gått till vila hos sina fäder.
2Ch 26:3  Ussia var sexton år gammal, när han blev konung, och han regerade femtiotvå år i Jerusalem. Hans moder hette Jekilja, från Jerusalem.
2Ch 26:4  Han gjorde vad rätt var i HERRENS ögon, alldeles såsom hans fader Amasja hade gjort.
2Ch 26:5  Och han sökte Gud, så länge Sakarja levde, han som aktade på Guds syner. Och så länge han sökte HERREN, lät Gud det gå honom väl.
2Ch 26:6  Han drog ut och stridde mot filistéerna och bröt ned Gats, Jabnes och Asdods murar; och han byggde städer på Asdods område och annorstädes i filistéernas land.
2Ch 26:7  Och Gud hjälpte honom mot filistéerna och mot de araber som bodde i Gur-Baal och mot maoniterna.
2Ch 26:8  Och ammoniterna måste giva skänker åt Ussia, och ryktet om honom sträckte sig ända till Egypten, ty han blev övermåttan mäktig.
2Ch 26:9  Och Ussia byggde torn i Jerusalem över Hörnporten och över Dalporten och över Vinkeln och befäste dem.
2Ch 26:10  Han byggde ock torn i öknen och högg ut många brunnar, ty han hade mycken boskap, både i låglandet och på slätten. Jordbruks- och vingårdsarbetare hade han i bergsbygden och på de bördiga fälten, ty han var en vän av åkerbruk.
2Ch 26:11  Och Ussia hade en krigshär som drog ut till strid i avdelade skaror, med en mansstyrka som hade blivit fastställd vid mönstring genom sekreteraren Jeguel och tillsyningsmannen Maaseja, under överinseende av Hananja, en av konungens hövitsmän.
2Ch 26:12  Hela antalet av de tappra stridsmän som voro huvudmän för familjerna var två tusen sex hundra.
2Ch 26:13  Under deras befäl stod en krigshär av tre hundra sju tusen fem hundra män, som stridde med kraft och mod och voro konungens hjälp mot fienden.
2Ch 26:14  Och Ussia försåg hela denna här med sköldar, spjut, hjälmar, pansar och bågar, så ock med slungstenar.
2Ch 26:15  Och han lät i Jerusalem göra krigsredskap, konstmässigt uttänkta, till att sätta upp på tornen och på murarnas hörn, för att med dem avskjuta pilar och stora stenar. Och ryktet om honom gick ut vida omkring, ty underbart hjälptes han fram till makt.
2Ch 26:16  Men när han nu var så mäktig, blev hans hjärta högmodigt, så att han gjorde vad fördärvligt var; han förbröt sig trolöst mot HERREN, sin Gud, i det att han gick in i HERRENS tempel för att antända rökelse på rökelsealtaret.
2Ch 26:17  Då gick prästen Asarja ditin efter honom, åtföljd av åttio HERRENS präster, oförskräckta män.
2Ch 26:18  Dessa trädde fram mot konung Ussia och sade till honom: "Det hör icke dig till, Ussia, att antända rökelse åt HERREN, utan det tillhör prästerna, Arons söner, som är helgade till att antända rökelse. Gå ut ur helgedomen, ty du har begått en förbrytelse, och HERREN Gud skall icke låta detta lända dig till ära."
2Ch 26:19  Då for Ussia ut i vrede, där han stod med ett rökelsekar i sin hand för att antända rökelse. Men just som han for ut mot prästerna, slog spetälska ut på hans panna, i prästernas närvaro, inne i HERRENS hus, bredvid rökelsealtaret.
2Ch 26:20  Och när översteprästen Asarja och alla prästerna vände sig till honom och fingo se att han var spetälsk i pannan, drevo de honom strax ut därifrån. Själv skyndade han också ut, eftersom HERREN så hemsökte honom.
2Ch 26:21  Sedan var konung Ussia spetälsk för hela sitt liv och bodde i ett särskilt hus såsom spetälsk, ty han var utesluten från HERRENS hus. Hans son Jotam förestod då konungens hus och dömde folket i landet.
2Ch 26:22  Vad nu mer är att säga om Ussia, om hans första tid såväl som om hans sista, det har profeten Jesaja, Amos' son, tecknat upp.
2Ch 26:23  Och Ussia gick till vila hos sina fäder, och man begrov honom hos hans fäder, ute på konungagravens mark, detta med tanke därpå att han hade varit spetälsk. Och hans son Jotam blev konung efter honom.
2Ch 27:1  Jotam var tjugufem år gammal när han blev konung, och han regerade sexton år i Jerusalem. Hans moder hette Jerusa, Sadoks dotter.
2Ch 27:2  Han gjorde vad rätt var i HERRENS ögon, alldeles såsom hans fader Ussia hade gjort, vartill kom att han icke trängde in i HERRENS tempel; men folket gjorde ännu vad fördärvligt var.
2Ch 27:3  Han byggde Övre porten till HERRENS hus, och på Ofelmuren utförde han stora byggnadsarbeten.
2Ch 27:4  Därtill byggde han städer i Juda bergsbygd, och i skogarna byggde han borgar och torn.
2Ch 27:5  Och när han så kom i strid med Ammons barns konung, blev han dem övermäktig, så att Ammons barn det året måste giva honom ett hundra talenter silver, tio tusen korer vete och tio tusen korer korn. Lika mycket måste Ammons barn erlägga åt honom också nästa år och året därpå.
2Ch 27:6  Så mäktig blev Jotam, därför att han vandrade ståndaktigt inför HERREN, sin Gud.
2Ch 27:7  Vad nu mer är att säga om Jotam och om alla hans krig och andra företag, det finnes upptecknat i boken om Israels och Juda konungar.
2Ch 27:8  Han var tjugufem år gammal, när han blev konung, och han regerade sexton år i Jerusalem.
2Ch 27:9  Och Jotam gick till vila hos sina fäder, och man begrov honom i Davids stad. Och hans son Ahas blev konung efter honom.
2Ch 28:1  Ahas var tjugu år gammal när han blev konung, och han regerade sexton år i Jerusalem. Han gjorde icke vad rätt var i HERRENS ögon, såsom hans fader David,.
2Ch 28:2  utan vandrade på Israels konungars väg; ja, han lät ock göra gjutna beläten åt Baalerna.
2Ch 28:3  Och själv tände han offereld i Hinnoms sons dal och brände upp sina barn i eld, efter den styggeliga seden hos de folk som HERREN hade fördrivit för Israels barn.
2Ch 28:4  Och han frambar offer och tände offereld på höjderna och kullarna och under alla gröna träd.
2Ch 28:5  Därför gav HERREN, hans Gud, honom i den arameiske konungens hand; de slogo honom och togo av hans folk en stor hop fångar och förde dem till Damaskus. Han blev ock given i Israels konungs hand, så att denne tillfogade honom ett stort nederlag.
2Ch 28:6  Ty Peka, Remaljas son, dräpte av Juda ett hundra tjugu tusen man på en enda dag, allasammans stridbara män. Detta skedde därför att de hade övergivit HERREN, sina fäders Gud.
2Ch 28:7  Och Sikri, en tapper man från Efraim, dräpte Maaseja, konungasonen, och Asrikam, slottshövdingen, och Elkana, konungens närmaste man.
2Ch 28:8  Och Israels barn bortförde från sina bröder två hundra tusen fångar, nämligen deras hustrur, söner och döttrar, och togo därjämte mycket byte från dem och förde bytet till Samaria.
2Ch 28:9  Men där var en HERRENS profet som hette Oded; denne gick ut mot hären, när den kom till Samaria, och sade till dem: "Se, i sin vrede över Juda har HERREN, edra fäders Gud, givit dem i eder hand, men I haven dräpt dem med en hätskhet som har nått upp till himmelen.
2Ch 28:10  Och nu tänken I göra Judas och Jerusalems barn till trälar och trälinnor åt eder. Därmed dragen I ju allenast skuld över eder själva inför HERREN, eder Gud.
2Ch 28:11  Så hören mig nu: Sänden tillbaka fångarna som I haven tagit från edra bröder; ty HERRENS vrede är upptänd mot eder."
2Ch 28:12  Några av huvudmännen bland Efraims barn, nämligen Asarja, Johanans son, Berekja, Mesillemots son, Hiskia, Sallums son, och Amasa, Hadlais son, stodo då upp och gingo emot dem som kommo från kriget
2Ch 28:13  och sade till dem: "I skolen icke föra dessa fångar hitin; ty I förehaven något som drager skuld över oss inför HERREN, och varigenom I ytterligare föröken våra synder och vår skuld. Vår skuld är ju redan stor nog, och vrede är upptänd mot Israel."
2Ch 28:14  Då lämnade krigsfolket ifrån sig fångarna och bytet inför de överste och hela församlingen.
2Ch 28:15  Och de nämnda männen stodo upp och togo sig an fångarna. Alla som voro nakna bland dem klädde de upp med vad de hade tagit såsom byte; de gåvo dem kläder och skor, mat och dryck, och smorde dem med olja, och alla som icke orkade gå läto de sätta sig upp på åsnor, och förde dem så till Jeriko, Palmstaden, till deras bröder där. Sedan vände de tillbaka till Samaria.
2Ch 28:16  Vid samma tid sände konung Ahas bud till konungarna i Assyrien, med begäran att de skulle hjälpa honom.
2Ch 28:17  Ty förutom allt annat hade edoméerna kommit och slagit Juda och tagit fångar.
2Ch 28:18  Och filistéerna hade fallit in i städerna i Juda lågland och sydland och hade intagit Bet-Semes, Ajalon och Gederot, så ock Soko med underlydande orter, Timna med underlydande orter och Gimso med underlydande orter, och hade bosatt sig i dem.
2Ch 28:19  Ty HERREN ville förödmjuka Juda, för Ahas', den israelitiske konungens, skull, därför att denne hade vållat oordning i Juda och varit otrogen mot HERREN.
2Ch 28:20  Men Tillegat-Pilneeser, konungen i Assyrien, drog emot honom och angrep honom, i stället för att understödja honom.
2Ch 28:21  Ty fastän Ahas plundrade HERRENS hus och konungshuset och de överstes hus och gav allt åt konungen i Assyrien, så hjälpte det honom dock icke.
2Ch 28:22  Och i sin nöd försyndade sig samme konung Ahas ännu mer genom otrohet mot HERREN.
2Ch 28:23  Han offrade nämligen åt gudarna i Damaskus, som hade slagit honom; ty han tänkte: "Eftersom de arameiska konungarnas gudar hava förmått hjälpa dem, vill jag offra åt dessa gudar, för att de ock må hjälpa mig." Men i stället var det dessa som kommo honom och hela Israel på fall.
2Ch 28:24  Ahas samlade ihop de kärl som funnos i Guds hus och bröt sönder kärlen i Guds hus och stängde igen dörrarna till HERRENS hus, och gjorde sig altaren i vart hörn i Jerusalem.
2Ch 28:25  Och i var och en av Juda städer uppförde han offerhöjder för att där tända offereld åt andra gudar, och han förtörnade så HERREN, sina fäders Gud.
2Ch 28:26  Vad nu mer är att säga om honom och om alla hans företag, under hans första tid såväl som under hans sista, det finnes upptecknat i boken om Judas och Israels konungar.
2Ch 28:27  Och Ahas gick till vila hos sina fäder, och man begrov honom i Jerusalem, inne i själva staden; de lade honom nämligen icke i Israels konungars gravar. Och hans son Hiskia blev konung efter honom.
2Ch 29:1  Hiskia var tjugufem år gammal, när han blev konung, och han regerade tjugunio år i Jerusalem. Hans moder hette Abia, Sakarjas dotter.
2Ch 29:2  Han gjorde vad rätt var i HERRENS ögon, alldeles såsom hans fader David hade gjort.
2Ch 29:3  I sitt första regeringsår, i första månaden, öppnade han dörrarna till HERRENS hus och satte dem i stånd
2Ch 29:4  Och han lät hämta prästerna och leviterna och församlade dem på den öppna platsen mot öster.
2Ch 29:5  Och han sade till dem: "Hören mig, I leviter. Helgen nu eder själva, och helgen HERRENS, edra fäders Guds, hus, och skaffen orenheten ut ur helgedomen.
2Ch 29:6  Ty våra fäder voro otrogna och gjorde vad ont var i HERRENS, vår Guds, ögon och övergåvo honom; de vände sitt ansikte bort ifrån HERRENS boning och vände honom ryggen.
2Ch 29:7  De stängde ock igen dörrarna till förhuset, släckte ut lamporna, tände ingen rökelse och offrade inga brännoffer i helgedomen åt Israels Gud.
2Ch 29:8  Därför har HERRENS förtörnelse kommit över Juda och Jerusalem, och han har gjort dem till en varnagel, till ett föremål för häpnad och begabberi, såsom I sen med egna ögon.
2Ch 29:9  Ja, därför hava ock våra fäder fallit för svärd, och våra söner och döttrar och hustrur hava fördenskull kommit i fångenskap.
2Ch 29:10  Men nu har jag i sinnet att sluta ett förbund med HERREN, Israels Gud, för att hans vredes glöd må vända sig ifrån oss.
2Ch 29:11  Så varen nu icke försumliga, mina barn, ty eder har HERREN utvalt till att stå inför hans ansikte och göra tjänst inför honom, till att vara hans tjänare och antända rökelse åt honom."
2Ch 29:12  Då stodo leviterna upp: Mahat, Amasais son, och Joel, Asarjas son, av kehatiternas barn; av Meraris barn Kis, Abdis son, och Asarja, Jehallelels son; av gersoniterna Joa, Simmas son, och Eden, Joas son;
2Ch 29:13  av Elisafans barn Simri och Jeguel; av Asafs barn Sakarja och Mattanja;
2Ch 29:14  av Hemans barn Jehuel och Simei; av Jedutuns barn Semaja och Ussiel.
2Ch 29:15  Dessa församlade nu sina bröder och helgade sig och gingo, såsom konungen hade bjudit i kraft av HERRENS ord, sedan in för att rena HERRENS hus.
2Ch 29:16  Men prästerna gingo in i det inre av HERRENS hus för att rena det, och all orenhet som de funno i HERRENS tempel buro de ut på förgården till HERRENS hus; där togo leviterna emot den och buro ut den i Kidrons dal.
2Ch 29:17  De begynte att helga templet på första dagen i första månaden, och på åttonde dagen i månaden hade de hunnit till HERRENS förhus och helgade sedan HERRENS hus under åtta dagar; och på sextonde dagen i första månaden hade de fullgjort sitt arbete.
2Ch 29:18  Då gingo de in till konung Hiskia och sade: "Vi hava renat hela HERRENS hus och brännoffersaltaret med alla dess tillbehör och skådebrödsbordet med alla dess tillbehör.
2Ch 29:19  Och alla de kärl som konung Ahas under sin regering i sin otrohet förkastade, dem hava vi återställt och helgat, och de stå nu framför HERRENS altare."
2Ch 29:20  Då lät konung Hiskia bittida om morgonen församla de överste i staden och gick upp i HERRENS hus.
2Ch 29:21  Och man förde fram sju tjurar, sju vädurar och sju lamm, så ock sju bockar till syndoffer för riket och för helgedomen och för Juda; och han befallde Arons söner, prästerna, att offra detta på HERRENS altare.
2Ch 29:22  Då slaktade de fäkreaturen, och prästerna togo upp blodet och stänkte det på altaret; därefter slaktade de vädurarna och stänkte blodet på altaret; sedan slaktade de lammen och stänkte blodet på altaret.
2Ch 29:23  Därefter förde de syndoffersbockarna fram inför konungen och församlingen, och de lade sina händer på dem.
2Ch 29:24  Och prästerna slaktade dem och läto deras blod såsom syndoffer komma på altaret, till försoning för hela Israel; ty konungen hade befallt att offra dessa brännoffer och syndoffer för hela Israel.
2Ch 29:25  Och han lät leviterna ställa upp sig till tjänstgöring i HERRENS hus med cymbaler, psaltare och harpor, såsom David och Gad, konungens siare, och profeten Natan hade bjudit; ty budet härom var givet av HERREN genom hans profeter.
2Ch 29:26  Och leviterna ställde upp sig med Davids instrumenter, och prästerna med trumpeterna.
2Ch 29:27  Och Hiskia befallde att man skulle offra brännoffret på altaret; och på samma gång som offret begynte, begynte ock HERRENS sång ljuda jämte trumpeterna, och detta under ledning av Davids, Israels konungs, instrumenter.
2Ch 29:28  Och hela församlingen föll ned, under det att sången sjöngs och trumpeterna skallade - allt detta ända till dess brännoffret var fullbordat.
2Ch 29:29  Och när de hade offrat brännoffret, knäböjde konungen och alla som voro där tillstädes med honom, och tillbådo.
2Ch 29:30  Och konung Hiskia och de överste befallde leviterna att lova HERREN med Davids och siaren Asafs ord; och de sjöngo hans lov med glädje och böjde sig ned och tillbådo.
2Ch 29:31  Och Hiskia tog till orda och sade: "I haven nu tagit handfyllning till HERRENS tjänst. Så träden nu hit och fören fram slaktoffer och lovoffer till HERRENS hus." Då förde församlingen fram slaktoffer och lovoffer, och var och en som av sitt hjärta manades därtill offrade brännoffer.
2Ch 29:32  Antalet av de brännoffersdjur som församlingen förde fram var sjuttio tjurar, ett hundra vädurar och två hundra lamm, alla dessa till brännoffer åt HERREN.
2Ch 29:33  Och tackoffren utgjordes av sex hundra tjurar och tre tusen djur av småboskapen.
2Ch 29:34  Men prästerna voro för få, så att de icke kunde draga av huden på alla brännoffersdjuren; därför understöddes de av sina bröder leviterna, till dess detta göromål var fullgjort, och till dess prästerna hade helgat sig. Ty i fråga om att helga sig hade leviterna visat sig mer rättsinniga än prästerna.
2Ch 29:35  Också var antalet stort av brännoffer, vartill kommo fettstyckena från tackoffren, så ock de drickoffer som hörde till brännoffren. Så blev det ordnat med tjänstgöringen i HERRENS hus.
2Ch 29:36  Och Hiskia och allt folket gladde sig över vad Gud hade berett åt folket; ty helt oväntat hade detta kommit till stånd.
2Ch 30:1  Därefter sände Hiskia ut bud till hela Israel och Juda och skrev också brev till Efraim och Manasse, att de skulle komma till HERRENS hus i Jerusalem för att hålla HERRENS, Israels Guds, påskhögtid.
2Ch 30:2  Och konungen och hans förnämsta män och hela församlingen i Jerusalem enade sig om att hålla påskhögtiden i andra månaden;
2Ch 30:3  ty de kunde icke hålla den nu genast, eftersom prästerna ännu icke hade helgat sig i tillräckligt antal och folket icke hade hunnit församla sig till Jerusalem.
2Ch 30:4  Därför syntes det konungen och hela församlingen rätt att göra så.
2Ch 30:5  Och de beslöto att låta utropa i hela Israel, från Beer-Seba ända till Dan, att man skulle komma och hålla HERRENS, Israels Guds, påskhögtid i Jerusalem; ty man hade icke eljest hållit den samfällt, såsom föreskrivet var.
2Ch 30:6  Så begåvo sig då ilbuden åstad med breven från konungen och hans förnämsta män och drogo genom hela Israel och Juda, enligt konungens befallning, och sade: "I Israels barn, vänden om till HERREN, Abrahams, Isaks och Israels Gud, på det att han må vända om till den kvarleva av eder, som har räddats undan de assyriska konungarnas hand.
2Ch 30:7  Och varen icke såsom edra fäder och bröder, som voro otrogna mot HERREN, sina fäders Gud, så att han prisgav dem åt förödelse, såsom I själva haven sett.
2Ch 30:8  Varen alltså nu icke hårdnackade såsom edra fäder, utan underkasten eder HERREN och kommen till hans helgedom, den som han har helgat för evig tid, och tjänen HERREN, eder Gud, på det att hans vredes glöd må vända sig ifrån eder.
2Ch 30:9  Ty om I vänden om till HERREN, skola edra bröder och edra barn finna barmhärtighet inför dem som hålla dem fångna, så att de få vända tillbaka till detta land; ty HERREN, eder Gud, är nådig och barmhärtig, och han skall icke vända sitt ansikte ifrån eder, om I vänden om till honom."
2Ch 30:10  Och ilbuden foro ifrån stad till stad i Efraims och Manasse land och ända till Sebulon; men man gjorde spe av dem och bespottade dem.
2Ch 30:11  Dock funnos några i Aser, Manasse och Sebulon, som ödmjukade sig och kommo till Jerusalem.
2Ch 30:12  Också i Juda verkade Guds hand, så att han gav dem ett endräktigt hjärta till att göra efter vad konungen och de överste hade bjudit i kraft av HERRENS ord.
2Ch 30:13  Och mycket folk kom tillhopa i Jerusalem för att hålla det osyrade brödets högtid i andra månaden, en mycket stor församling.
2Ch 30:14  Och de stodo upp och skaffade bort de altaren som funnos i Jerusalem; också alla offereldsaltarna skaffade de bort och kastade dem i Kidrons dal.
2Ch 30:15  Och de slaktade påskalammet på fjortonde dagen i andra månaden; prästerna och leviterna, som nu kände blygsel och därför hade helgat sig, förde därvid fram brännoffer till HERRENS hus.
2Ch 30:16  Och de inställde sig till tjänstgöring på sina platser, såsom det var föreskrivet för dem, efter gudsmannen Moses lag; och prästerna stänkte med blodet, sedan de hade tagit emot det av leviterna.
2Ch 30:17  Ty många funnos i församlingen, som icke hade helgat sig; därför måste leviterna slakta påskalammen för alla som icke voro rena, och så helga dem åt HERREN.
2Ch 30:18  Det var nämligen en myckenhet av folket, många från Efraim och Manasse, Isaskar och Sebulon, som icke hade renat sig, utan åto påskalammet på annat sätt än föreskrivet var. Men Hiskia hade bett för dem och sagt: "HERREN, den gode, förlåte var och en
2Ch 30:19  som har vänt sitt hjärta till att söka Gud, HERREN, sina fäders Gud, om han än icke är ren efter helgedomens ordning."
2Ch 30:20  Och HERREN hörde Hiskia och skonade folket.
2Ch 30:21  Så höllo Israels barn, de som då voro tillstädes i Jerusalem, det osyrade brödets högtid i sju dagar med stor glädje; och leviterna och prästerna lovade HERREN var dag med kraftiga instrumenter, HERREN till ära.
2Ch 30:22  Och Hiskia talade vänligt till alla de leviter som voro väl förfarna i HERRENS tjänst. Och de åto av högtidsoffren under de sju dagarna, i det att de offrade tackoffer och prisade HERREN, sina faders Gud.
2Ch 30:23  Och hela församlingen enade sig om att hålla högtid under ännu sju dagar; och så höll man högtid med glädje också under de sju dagarna.
2Ch 30:24  Ty Hiskia, Juda konung, hade såsom offergärd givit åt församlingen ett tusen tjurar och av småboskapen sju tusen djur, och de överste hade såsom offergärd givit åt församlingen ett tusen tjurar och av småboskapen tio tusen djur. Och ett stort antal präster helgade sig.
2Ch 30:25  Och hela Juda församling gladde sig med prästerna och leviterna, så ock hela församlingen av dem som hade kommit från Israel, ävensom de främlingar som hade kommit från Israels land, eller som bodde i Juda.
2Ch 30:26  Och i Jerusalem var stor glädje; ty alltsedan Salomos, Davids sons, Israels konungs, tid hade icke något sådant som detta skett i Jerusalem.
2Ch 30:27  Och de levitiska prästerna stodo upp och välsignade folket, och deras röst blev hörd, och deras bön kom till himmelen, hans heliga boning.
2Ch 31:1  När nu allt detta var till ända, drogo alla israeliter som hade varit där tillstädes ut till Juda städer och slogo sönder stoderna, höggo ned Aserorna och bröto ned offerhöjderna och altarna i hela Juda och Benjamin och i Efraim och Manasse, till dess att de hade gjort ände på dem; sedan vände alla Israels barn tillbaka till sina städer, var och en till sin egendom.
2Ch 31:2  Och Hiskia förordnade om prästernas och leviternas avdelningar, alltefter som de tillhörde den ena eller den andra avdelningen, så att var och en av såväl prästerna som leviterna fick sitt bestämda göromål, när brännoffer och tackoffer skulle offras, till att därvid göra tjänst och tacka och lovsjunga i portarna till HERRENS läger.
2Ch 31:3  Och konungen anslog en del av sin egendom till brännoffren, nämligen till att offra brännoffer morgon och afton, och till att offra brännoffer på sabbaterna, vid nymånaderna och vid högtiderna, såsom det var föreskrivet i HERRENS lag.
2Ch 31:4  Och han befallde folket som bodde i Jerusalem att giva prästerna och leviterna deras del, för att de skulle kunna hålla fast vid HERRENS lag.
2Ch 31:5  Och när denna befallning blev känd, gåvo Israels barn rikligen en förstling av säd, vin, olja och honung och av all markens avkastning; och tionde av allt förde de fram i myckenhet.
2Ch 31:6  Och de av Israels och Juda barn, som bodde i Juda städer, förde ock fram tionde av fäkreatur och små boskap, så ock tionde av de heliga gåvor som helgades åt HERRENS, deras Gud, och lade upp dem i särskilda högar.
2Ch 31:7  I tredje månaden begynte de att lägga upp högarna, och i sjunde månaden hade de slutat därmed.
2Ch 31:8  När då Hiskia och de överste kommo och sågo högarna, prisade de HERREN och hans folk Israel.
2Ch 31:9  Och Hiskia frågade prästerna och leviterna om högarna.
2Ch 31:10  Då svarade honom översteprästen Asarja, av Sadoks hus, och sade; "Alltsedan man begynte föra fram offergärden till HERRENS hus, hava vi ätit och blivit mätta och dock fått mycket kvar; ty HERREN har välsignat sitt folk, och vad som är kvar är denna stora rikedom."
2Ch 31:11  Och Hiskia befallde att man skulle inreda förrådskamrar i HERRENS hus, och man inredde sådana.
2Ch 31:12  Och i dem förde man in offergärden och tionden och de heliga gåvorna, allt på heder och tro. Och överuppsyningsman däröver var leviten Konanja, och hans närmaste man var hans broder Simei.
2Ch 31:13  Men Jehiel, Asasja, Nahat, Asael, Jerimot, Josabad, Eliel, Jismakja, Mahat och Benaja voro tillsyningsmän under Konanja och hans broder Simei, efter förordnande av konung Hiskia och Asarja, fursten i Guds hus.
2Ch 31:14  Och leviten Kore, Jimnas son, som var dörrvaktare på östra sidan, hade uppsikten över de frivilliga gåvorna åt Gud och skulle fördela HERRENS offergärd och det högheliga av offren.
2Ch 31:15  Och under honom sattes Eden, Minjamin, Jesua, Semaja, Amarja och Sekanja till förtroendemän i präststäderna för att ombesörja utdelningen åt sina bröder, efter deras avdelningar, åt den minste såväl som åt den störste.
2Ch 31:16  Härifrån voro undantagna alla sådana i sina släktregister upptecknade personer av mankön, från tre års ålder och därutöver, som skulle infinna sig i HERRENS hus, där var dag de för den dagen bestämda sysslorna skulle utföras genom dem som hade tjänstgöringen, med de särskilda åligganden de hade efter sina avdelningar.
2Ch 31:17  Och vad angick prästernas släktregister, så var det uppgjort efter deras familjer; och av leviterna voro de upptagna, som voro tjugu år gamla eller därutöver, efter sina särskilda åligganden, alltefter sina avdelningar.
2Ch 31:18  Och i släktregistret skulle de vara upptecknade jämte alla sina späda barn, hustrur, söner och döttrar, så många de voro. Ty på heder och tro skulle de förvalta det heliga såsom heligt.
2Ch 31:19  Och för dem av Arons söner, prästerna, som bodde på sina städers utmarker, voro i var särskild stad namngivna män tillsatta, som åt allt mankön bland prästerna och åt alla de leviter som voro upptecknade i släktregistret skulle utdela vad dem tillkom.
2Ch 31:20  Så förfor Hiskia i hela Juda, och han gjorde inför HERREN, sin Gud, vad gott och rätt och sant var.
2Ch 31:21  Och allt som han företog sig, när han nu sökte sin Gud, allt, vare sig det angick tjänstgöringen i Guds hus eller det angick lagen och budorden, det gjorde han av allt sitt hjärta, och det lyckades honom väl.
2Ch 32:1  Sedan han hade utfört detta och bevisat sådan trohet, kom Sanherib, konungen i Assyrien, och drog in i Juda och belägrade dess befästa städer och tänkte erövra dem åt sig.
2Ch 32:2  Då nu Hiskia såg att Sanherib kom, i avsikt att belägra Jerusalem,
2Ch 32:3  rådförde han sig med sina förnämsta män och sina hjältar om att täppa för vattnet i de källor som lågo utom staden; och de hjälpte honom härmed.
2Ch 32:4  Mycket folk församlades och täppte till alla källorna och dämde för bäcken som flöt mitt igenom trakten, ty de sade: "När de assyriska konungarna komma, böra de icke finna vatten i sådan myckenhet."
2Ch 32:5  Och han tog mod till sig och byggde upp muren överallt där den var nedbruten, och byggde tornen högre, och förde upp en annan mur därutanför, och befäste Millo i Davids stad, och lät göra skjutvapen i myckenhet, så ock sköldar.
2Ch 32:6  Och han tillsatte krigshövitsmän över folket och församlade dem till sig på den öppna platsen vid stadsporten, och talade uppmuntrande till dem och sade:
2Ch 32:7  "Varen frimodiga och oförfärade, frukten icke och varen icke förskräckta för konungen i Assyrien och för hela den hop han har med sig; ty med oss är en som är större än den som är med honom.
2Ch 32:8  Med honom är en arm av kött, men med oss är HERREN, vår Gud, och han skall hjälpa oss och föra våra krig. Och folket tryggade sig vid Hiskias, Juda konungs, ord.
2Ch 32:9  Därefter sände Sanherib, konungen i Assyrien - som nu med hela sin härsmakt låg framför Lakis - sina tjänare till Jerusalem, till Hiskia, Juda konung, och till alla dem av Juda, som voro i Jerusalem, och lät säga:
2Ch 32:10  "Så säger Sanherib, konungen i Assyrien: Varpå förtrösten I, eftersom I stannen kvar i det belägrade Jerusalem?
2Ch 32:11  Se, Hiskia uppeggar eder, så att I kommen att dö genom hunger och törst; han säger: 'HERREN, vår Gud, skall rädda oss ur den assyriske konungens hand.'
2Ch 32:12  Har icke denne samme Hiskia avskaffat hans offerhöjder och altaren och sagt till Juda och Jerusalem: 'Inför ett enda altare skolen I tillbedja, och på detta skolen I tända offereld'?
2Ch 32:13  Veten I icke vad jag och mina fäder hava gjort med andra länders alla folk? Hava väl de gudar som dyrkas av folken i dessa andra länder någonsin förmått rädda sina länder ur min hand?
2Ch 32:14  Ja, vilket bland alla dessa folk som mina fäder hava givit till spillo har väl haft någon gud som har förmått rädda sitt folk ur min hand eftersom I menen att eder Gud förmår rädda eder ur min hand!"
2Ch 32:15  Nej, låten nu icke Hiskia så bedraga och uppegga eder, och tron honom icke; ty ingen gud hos något folk eller i något rike har förmått rädda sitt folk ur min hand eller ur mina fäders hand. Huru mycket mindre skall då eder Gud kunna rädda eder ur min hand!"
2Ch 32:16  Och hans tjänare talade ännu mer mot HERREN Gud och mot hans tjänare Hiskia.
2Ch 32:17  Han hade ock skrivit ett brev vari han smädade HERREN, Israels Gud, och talade mot honom så: "Lika litet som de gudar som dyrkas av folken i de andra länderna hava kunnat rädda sina folk ur min hand, lika litet skall Hiskias Gud kunna rädda sitt folk ur min hand."
2Ch 32:18  Och till Jerusalems folk, dem som stodo på muren, ropade de med hög röst på judiska för att göra dem modlösa och förskräckta, så att man sedan skulle kunna intaga staden.
2Ch 32:19  Och de talade om Jerusalems Gud på samma sätt som om de främmande folkens gudar, vilka äro verk av människohänder.
2Ch 32:20  Men vid allt detta bådo konung Hiskia och profeten Jesaja, Amos' son, och ropade till himmelen.
2Ch 32:21  Då sände HERREN en ängel, som förgjorde alla de tappra stridsmännen och furstarna och hövitsmännen i den assyriske konungens läger, så att han med skam måste draga tillbaka till sitt land. Och när han en gång gick in i sin guds hus, blev han där nedhuggen med svärd av sina egna söner.
2Ch 32:22  Så frälste HERREN Hiskia och Jerusalems invånare ur Sanheribs, den assyriske konungens, hand och ur alla andras hand; och han beskyddade dem på alla sidor.
2Ch 32:23  Och många förde skänker till HERREN i Jerusalem och dyrbara gåvor till Hiskia, Juda konung; och han blev härefter högt aktad av alla folk.
2Ch 32:24  Vid den tiden blev Hiskia dödssjuk. Då bad han till HERREN, och han svarade honom och gav honom ett undertecken.
2Ch 32:25  Dock återgäldade Hiskia icke det goda som hade blivit honom bevisat, utan hans hjärta blev högmodigt; därför kom förtörnelse över honom och över Juda och Jerusalem.
2Ch 32:26  Men då Hiskia ödmjukade sig, mitt i sitt hjärtas högmod, och Jerusalems invånare med honom, drabbade HERRENS förtörnelse dem icke, så länge Hiskia levde.
2Ch 32:27  Och Hiskias rikedom och härlighet var mycket stor; han hade byggt sig skattkamrar för silver och guld och ädla stenar, och för välluktande kryddor, och för sköldar och för allahanda dyrbara håvor av andra slag,
2Ch 32:28  så ock förrådshus för vad som kom in av säd, vin och olja, ävensom stall för allt slags boskap; och hjordar hade han skaffat för sina fållor.
2Ch 32:29  Och han hade byggt sig städer och förvärvat sig stor rikedom på får och fäkreatur; ty Gud hade givit honom mycket stora ägodelar.
2Ch 32:30  Det var ock Hiskia som täppte till Gihonsvattnets övre källa och ledde vattnet nedåt, väster om Davids stad. Och Hiskia var lyckosam i allt vad han företog sig.
2Ch 32:31  Jämväl när från Babels furstar de sändebud kommo, som voro skickade till honom för att fråga efter det under som hade skett i landet, övergav Gud honom allenast för att pröva honom, på det att han skulle förnimma allt vad som var i hans hjärta.
2Ch 32:32  Vad nu mer är att säga om Hiskia och om hans fromma gärningar, det finnes upptecknat i "Profeten Jesajas, Amos' sons, syner", i boken om Judas och Israels konungar.
2Ch 32:33  Och Hiskia gick till vila hos sina fäder, och han begrov honom på den plats där man går upp till Davids hus' gravar; och hela Juda och Jerusalems invånare bevisade honom ära vid hans död. Och hans son Manasse blev konung efter honom.
2Ch 33:1  Manasse var tolv år gammal, när han blev konung, och han regerade femtiofem år i Jerusalem.
2Ch 33:2  Han gjorde vad ont var i HERRENS ögon, efter den styggeliga seden hos de folk som HERREN hade fördrivit för Israels barn.
2Ch 33:3  Han byggde åter upp de offerhöjder som hans fader Hiskia hade brutit ned, och reste altaren åt Baalerna och gjorde Aseror, och tillbad och tjänade himmelens hela härskara.
2Ch 33:4  Ja, han byggde altaren i HERRENS hus, det om vilket HERREN hade sagt: "I Jerusalem skall mitt namn vara till evig tid."
2Ch 33:5  Han byggde altaren åt himmelens hela härskara på de båda förgårdarna till HERRENS hus.
2Ch 33:6  Han lät ock sina barn gå genom eld i Hinnoms sons dal och övade teckentyderi, svartkonst och trolldom och skaffade sig andebesvärjare och spåmän och gjorde mycket som var ont i HERRENS ögon, så att han förtörnade honom.
2Ch 33:7  Och avgudabelätet som han hade låtit göra satte han i Guds hus, om vilket Gud hade sagt till David och till hans son Salomo: "Vid detta hus och vid Jerusalem, som jag har utvalt bland alla Israels stammar, vill jag fästa mitt namn för evig tid.
2Ch 33:8  Och jag skall icke mer låta Israel vandra bort ifrån det land som jag har bestämt åt edra fäder, om de allenast hålla och göra allt vad jag har bjudit dem, alldeles efter den lag, de stadgar och rätter som de hava fått genom Mose."
2Ch 33:9  Men Manasse förförde Juda och Jerusalems invånare, så att de gjorde mer ont än de folk som HERREN hade förgjort för Israels barn.
2Ch 33:10  Och HERREN talade till Manasse och hans folk, men de aktade icke därpå.
2Ch 33:11  Då lät HERREN den assyriske konungens härhövitsmän komma över dem; de slogo Manasse i bojor och fängslade honom med kopparfjättrar och förde honom till Babel.
2Ch 33:12  Men när han nu var i nöd, bön föll han inför HERREN, sin Gud, och ödmjukade sig storligen för sina fäders Gud.
2Ch 33:13  Och när han så bad till honom, lät han beveka sig och hörde hans bön och lät honom komma tillbaka till Jerusalem såsom konung. Och då besinnade Manasse att HERREN är Gud.
2Ch 33:14  Därefter byggde han en yttre mur till Davids stad västerut mot Gihon i dalen, intill Fiskporten, och runt omkring Ofel, och gjorde den mycket hög. Och han insatte krigshövitsmän i alla befästa städer i Juda.
2Ch 33:15  Och han skaffade bort de främmande gudarna och avgudabelätet ur HERRENS hus, så ock alla de altaren som han hade byggt på det berg där HERRENS hus stod och i Jerusalem, och kastade dem utanför staden.
2Ch 33:16  Och han upprättade HERRENS altare och offrade tackoffer och lovoffer därpå, och uppmanade Juda att tjäna HERREN, Israels Gud.
2Ch 33:17  Men folket offrade ännu på höjderna, dock allenast åt HERREN, sin Gud.
2Ch 33:18  Vad nu mer är att säga om Manasse och om hans bön till sin Gud och om de ord som siarna talade till honom i HERRENS, Israels Guds, namn, det står i Israels konungars krönika.
2Ch 33:19  Och om hans bön och huru han blev bönhörd, och om all hans synd och otrohet, och om de platser på vilka han byggde offerhöjder och ställde upp sina Aseror och beläten, innan han ödmjukade sig, härom är skrivet i Hosais krönika.
2Ch 33:20  Och Manasse gick till vila hos sina fäder, och man begrov honom där han bodde. Och hans son Amon blev konung efter honom.
2Ch 33:21  Amon var tjugutvå år gammal, när han blev konung, och han regerade två år i Jerusalem.
2Ch 33:22  Han gjorde vad ont var i HERRENS ögon, såsom hans fader Manasse hade gjort; åt alla de beläten som hans fader Manasse hade låtit göra offrade Amon, och han tjänade dem.
2Ch 33:23  Men han ödmjukade sig icke för HERREN, såsom hans fader Manasse hade gjort, utan denne Amon hopade skuld på skuld.
2Ch 33:24  Och hans tjänare sammansvuro sig mot honom och dödade honom hemma i hans hus.
2Ch 33:25  Men folket i landet dräpte alla som hade sammansvurit sig mot konung Amon. Därefter gjorde folket i landet hans son Josia till konung efter honom.
2Ch 34:1  Josia var åtta år gammal, när han blev konung, och han regerade trettioett år i Jerusalem.
2Ch 34:2  Han gjorde vad rätt var i HERRENS ögon och vandrade på sin fader Davids vägar och vek icke av vare sig till höger eller till vänster.
2Ch 34:3  I sitt åttonde regeringsår, medan han ännu var en yngling, begynte han att söka sin fader Davids Gud; och i det tolfte året begynte han att rena Juda och Jerusalem från offerhöjderna och Aserorna och från de skurna och gjutna belätena.
2Ch 34:4  Men Baalsaltarna brötos ned i hans åsyn, och solstoderna som voro uppställda på dem högg han ned, och Aserorna och de skurna och gjutna belätena slog han sönder och krossade dem till stoft och strödde ut stoftet på de mäns gravar, som hade offrat åt dem.
2Ch 34:5  Och prästernas ben brände han upp på deras altaren. Så renade han Juda och Jerusalem.
2Ch 34:6  Och i Manasses, Efraims och Simeons städer ända till Naftali genomsökte han överallt husen.
2Ch 34:7  Och sedan han hade brutit ned altarna och krossat Aserorna och belätena sönder till stoft och huggit ned alla solstoder i hela Israels land, vände han tillbaka till Jerusalem.
2Ch 34:8  Och i sitt adertonde regeringsår, medan han höll på med att rena landet och templet, sände han Safan, Asaljas son, och Maaseja, hövitsmannen i staden, och kansleren Joa, Joahas' son, för att sätta HERRENS, sin Guds, hus i stånd.
2Ch 34:9  Och de gingo till översteprästen Hilkia och avlämnade de penningar som hade influtit till Guds hus, sedan de av de leviter som höllo vakt vid tröskeln hade blivit insamlade från Manasse, Efraim och hela det övriga Israel, så ock från hela Juda och Benjamin och från Jerusalems invånare;
2Ch 34:10  de överlämnade dem åt de män som förrättade arbete såsom tillsyningsmän vid HERRENS hus. Sedan gåvos penningarna av dessa män, som förrättade arbete och hade befattning vid HERRENS hus med att laga huset och sätta det i stånd,
2Ch 34:11  de gåvos åt timmermännen och byggningsmännen, till att inköpa huggen sten och trävirke till stockar, för att man därmed skulle timra upp de hus som Juda konungar hade förstört.
2Ch 34:12  Och männen fingo vid sitt arbete handla på heder och tro; och tillsyningsmän över dem och föreståndare för arbetet voro Jahat och Obadja, leviter av Meraris barn, och Sakarja och Mesullam, av kehatiternas barn, så ock alla de leviter som voro kunniga på musikinstrumenter.
2Ch 34:13  De hade ock tillsynen över bärarna, så att föreståndare funnos för alla arbetarna vid de särskilda göromålen. Av leviterna togos ock skrivare, uppsyningsmän och dörrvaktare.
2Ch 34:14  När de nu togo ut penningarna som hade influtit till HERRENS hus, fann prästen Hilkia HERRENS lagbok, den som hade blivit given genom Mose
2Ch 34:15  Då tog Hilkia till orda och sade till sekreteraren Safan: "Jag har funnit lagboken i HERRENS hus." Och Hilkia gav boken åt Safan.
2Ch 34:16  Och Safan bar boken till konungen och avgav därjämte sin berättelse inför konungen och sade: "Allt vad dina tjänare hava fått i uppdrag att göra, det göra de.
2Ch 34:17  Och de hava tömt ut de penningar som funnos i HERRENS hus, och hava överlämnat dem åt tillsyningsmännen och åt arbetarna."
2Ch 34:18  Vidare berättade sekreteraren Safan för konungen och sade: "Prästen Hilkia har givit mig en bok." Och Safan föreläste därur för konungen
2Ch 34:19  När konungen nu hörde lagens ord, rev han sönder sina kläder.
2Ch 34:20  Och konungen bjöd Hilkia och Ahikam, Safans son, och Abdon, Mikas son, och sekreteraren Safan och Asaja, konungens tjänare, och sade:
2Ch 34:21  "Gån och frågen HERREN för mig och för dem som äro kvar av Israel och Juda, angående det som står i den bok som nu har blivit funnen. Ty stor är HERRENS vrede, den som är utgjuten över oss, därför att våra fäder icke hava hållit HERRENS ord och icke hava gjort allt som är föreskrivet i denna bok."
2Ch 34:22  Då gick Hilkia, tillika med andra som konungen sände åstad, till profetissan Hulda, hustru åt Sallum, klädkammarvaktaren, som var son till Tokehat, Hasras son; hon bodde i Jerusalem, i Nya staden. Och de talade med henne såsom dem bjudet var.
2Ch 34:23  Då svarade hon dem: "Så säger HERREN, Israels Gud: Sägen till den man som har sänt eder till mig:
2Ch 34:24  Så säger HERREN: Se, över denna plats och över dess invånare skall jag låta olycka komma, alla de förbannelser som äro skrivna i den bok som man har föreläst för Juda konung -
2Ch 34:25  detta därför att de hava övergivit mig och tänt offereld åt andra gudar, och så hava förtörnat mig med alla sina händers verk. Min vrede skall utgjutas över denna plats och skall icke bliva utsläckt.
2Ch 34:26  Men till Juda konung, som har sänt eder för att fråga HERREN, till honom skolen I säga så: Så säger HERREN, Israels Gud, angående de ord som du har hört:
2Ch 34:27  Eftersom ditt hjärta blev bevekt och du ödmjukade dig inför Gud, när du hörde hans ord mot denna plats och mot dess invånare, ja, ödmjukade dig inför mig och rev sönder dina kläder och grät inför mig, fördenskull har jag ock hört dig, säger HERREN.
2Ch 34:28  Se, jag vill samla dig till dina fäder, så att du får samlas till dem i din grav med frid, och dina ögon skola slippa att se all den olycka som jag skall låta komma över denna plats och dess invånare." Och de vände tillbaka till konungen med detta svar.
2Ch 34:29  Då sände konungen åstad och lät församla alla de äldste i Juda och Jerusalem.
2Ch 34:30  Och konungen gick upp i HERRENS hus med alla Juda män och Jerusalems invånare, också prästerna och leviterna, ja, allt folket, ifrån den störste till den minste. Och han läste upp för dem allt vad som stod i förbundsboken, som hade blivit funnen i HERRENS hus.
2Ch 34:31  Och konungen trädde fram på sin plats och slöt inför HERRENS ansikte det förbundet, att de skulle följa efter HERREN och hålla hans bud, hans vittnesbörd och hans stadgar, av allt sitt hjärta och av all sin själ, och göra efter förbundets ord, dem som voro skrivna i denna bok.
2Ch 34:32  Och han lät alla som funnos i Jerusalem och Benjamin träda in i förbundet Och Jerusalems invånare gjorde efter Guds, sina fäders Guds, förbund.
2Ch 34:33  Och Josia skaffade bort alla styggelser ur Israels barns alla landområden, och tillhöll alla dem som funnos i Israel att tjäna HERREN, sin Gud. Så länge han levde, veko de icke av ifrån HERREN, sina fäders Gud.
2Ch 35:1  Därefter höll Josia HERRENS påskhögtid i Jerusalem; man slaktade påskalammet på fjortonde dagen i första månaden.
2Ch 35:2  Och han fastställde prästernas åligganden och styrkte dem till tjänstgöringen i HERRENS hus.
2Ch 35:3  Och han sade till leviterna som undervisade hela Israel, och som voro helgade åt HERREN: "Sätten den heliga arken i det hus som Salomo, Davids son, Israels konung, har byggt. Den skall icke mer vara en börda på edra axlar. Tjänen nu HERREN, eder Gud, och hans folk Israel.
2Ch 35:4  Gören eder redo efter edra familjer, i edra avdelningar, enligt vad David, Israels konung, har föreskrivit, och enligt hans son Salomos föreskrifter,
2Ch 35:5  och inställen eder i helgedomen, ordnade efter edra bröders, det meniga folkets, familjeskiften, så att en avdelning av en levitisk familj kommer på vart skifte.
2Ch 35:6  Och slakten påskalammet och helgen eder och reden till det för edra bröder, så att I gören efter HERRENS ord genom Mose."
2Ch 35:7  Och Josia gav åt det meniga folket såsom offergärd småboskap, dels lamm och dels killingar, till ett antal av trettio tusen, alltsammans till påskoffer, åt alla som voro där tillstädes, så ock tre tusen fäkreatur, detta allt av konungens enskilda egendom.
2Ch 35:8  Och hans förnämsta män gåvo efter sin fria vilja offergåvor åt folket, åt prästerna och leviterna. Hilkia, Sakarja och Jehiel, furstarna i Guds hus, gåvo åt prästerna två tusen sex hundra lamm och killingar till påskoffer, så ock tre hundra fäkreatur.
2Ch 35:9  Men Konanja och hans bröder, Semaja och Netanel, jämte Hasabja, Jegiel och Josabad, de översta bland leviterna, gåvo åt leviterna såsom offergärd fem tusen lamm och killingar till påskoffer, så ock fem hundra fäkreatur.
2Ch 35:10  Så blev det då ordnat för gudstjänsten; och prästerna inställde sig till tjänstgöring på sina platser och likaledes leviterna, efter sina avdelningar, såsom konungen hade bjudit.
2Ch 35:11  Därefter slaktade de påskalammet, och prästerna stänkte med blodet som de togo emot av leviterna; och dessa drogo av huden.
2Ch 35:12  Och de avskilde brännoffersstyckena och delade ut dem åt det meniga folket, efter deras familjeskiften, för att de skulle offra dem åt HERREN, såsom det var föreskrivet i Moses bok. På samma sätt gjorde de ock med fäkreaturen.
2Ch 35:13  Och de stekte påskalammet på eld, på föreskrivet sätt; men tackoffersköttet kokade de i grytor, pannor och kittlar och delade ut det med hast åt allt det meniga folket.
2Ch 35:14  Sedan redde de till åt sig själva och åt prästerna; ty prästerna, Arons söner, voro upptagna ända till natten med att offra brännoffret och fettstyckena; därför måste leviterna reda till både åt sig och åt prästerna, Arons söner.
2Ch 35:15  Och sångarna, Asafs barn, stodo på sin plats, såsom David och Asaf och Heman och konungens siare Jedutun hade bjudit, och dörrvaktarna stodo var och en vid sin port; de behövde icke gå ifrån sin tjänstgöring, ty deras bröder, de andra leviterna, redde till åt dem.
2Ch 35:16  Så blev allt ordnat för HERRENS tjänst på den dagen, i det att man höll påskhögtid och offrade brännoffer på HERRENS altare, såsom konung Josia hade bjudit.
2Ch 35:17  De israeliter som voro där tillstädes höllo nu påskhögtid och firade det osyrade brödets högtid i sju dagar.
2Ch 35:18  En påskhögtid lik denna hade icke blivit hållen i Israel sedan profeten Samuels tid; ty ingen av Israels konungar hade hållit en sådan påskhögtid som den vilken nu hölls av Josia jämte prästerna och leviterna och hela Juda och dem av Israel, som voro där tillstädes, jämväl Jerusalems invånare.
2Ch 35:19  I Josias adertonde regeringsår hölls denna påskhögtid.
2Ch 35:20  Efter allt detta, sedan Josia hade försatt templet i gott stånd, drog Neko, konungen i Egypten, upp för att strida vid Karkemis, som ligger vid Frat; och Josia drog ut mot honom.
2Ch 35:21  Då skickade denne sändebud till honom och lät säga: "Vad har du med mig att göra, du Juda konung? Det är icke mot dig jag nu kommer, utan mot min arvfiende, och Gud har befallt mig att skynda. Hör upp att trotsa Gud, som är med mig, och tag dig till vara, så att han icke fördärvar dig."
2Ch 35:22  Men i stället för att vända om och lämna honom i fred förklädde Josia sig och gick att strida mot honom, utan att höra på Nekos ord, som dock kommo från Guds mun. Och det kom till strid på Megiddos slätt.
2Ch 35:23  Men skyttarnas skott träffade konung Josia; och konungen sade till sina tjänare: "Bären mig undan, ty jag är svårt sårad."
2Ch 35:24  Då buro hans tjänare honom från stridsvagnen och satte honom i hans andra vagn och förde honom till Jerusalem; och han gav upp andan och blev begraven där hans fäder voro begravna. Och hela Juda och Jerusalem sörjde Josia.
2Ch 35:25  Och Jeremia sjöng en klagosång över Josia. Och alla sångare och sångerskor talade sedan i sina klagosånger om Josia, såsom man gör ännu i dag; och dessa sånger blevo allmänt gängse i Israel. De finnas upptecknade bland "Klagosångerna".
2Ch 35:26  Vad nu mer är att säga om Josia och om de fromma gärningar han gjorde, efter vad föreskrivet var i HERRENS lag,
2Ch 35:27  och om annat som han företog sig under sin första tid såväl som under sin sista, det finnes upptecknat i boken om Israels och Juda konungar
2Ch 36:1  Och folket i landet tog Josias son Joahas och gjorde honom till konung i Jerusalem efter hans fader.
2Ch 36:2  Joahas var tjugutre år gammal, när han blev konung, och han regerade tre månader i Jerusalem.
2Ch 36:3  Konungen i Egypten avsatte honom i Jerusalem och pålade landet en skatt av ett hundra talenter silver och en talent guld.
2Ch 36:4  Och konungen i Egypten gjorde hans broder Eljakim till konung över Juda och Jerusalem och förändrade hans namn till Jojakim men hans broder Joahas, honom tog Neko med sig, och han förde honom till Egypten.
2Ch 36:5  Jojakim var tjugufem år gammal när han blev konung, och han regerade elva år i Jerusalem. Han gjorde vad ont var i HERRENS, sin Guds, ögon.
2Ch 36:6  Och Nebukadnessar, konungen i Babel, drog upp mot honom och fängslade honom med kopparfjättrar och förde honom bort till Babel.
2Ch 36:7  Och en del av kärlen i HERRENS hus förde Nebukadnessar till Babel, och han satte in dem i sitt tempel i Babel.
2Ch 36:8  Vad nu mer är att säga om Jojakim och om de styggelser som han gjorde, och om vad han eljest har befunnits vara skyldig till, det finnes upptecknat i boken om Israels och Juda konungar. Och hans son Jojakin blev konung efter honom.
2Ch 36:9  Jojakin var åtta år gammal, när han blev konung, och han regerade tre månader och tio dagar i Jerusalem. Han gjorde vad ont var i HERRENS ögon.
2Ch 36:10  Och vid följande års början sände konung Nebukadnessar och lät hämta honom till Babel, tillika med de dyrbara kärlen i HERRENS hus; och han gjorde hans broder Sidkia till konung över Juda och Jerusalem.
2Ch 36:11  Sidkia var tjuguett år gammal, när han blev konung, och han regerade elva år i Jerusalem.
2Ch 36:12  Han gjorde vad ont var i HERRENS, sin Guds, ögon; han ödmjukade sig icke för profeten Jeremia, som talade HERRENS ord.
2Ch 36:13  Han avföll från konung Nebukadnessar, som hade tagit ed av honom vid Gud. Och han var hårdnackad och förstockade sitt hjärta, så att han icke omvände sig till HERREN, Israels Gud.
2Ch 36:14  Alla de översta bland prästerna och folket försyndade sig ock storligen i otrohet mot Gud med hedningarnas alla styggelser och orenade HERRENS hus, som han hade helgat i Jerusalem.
2Ch 36:15  Och HERREN, deras faders Gud, skickade sina budskap till dem titt och ofta genom sina sändebud, ty han ömkade sig över sitt folk och sin boning.
2Ch 36:16  Men de begabbade Guds sändebud och föraktade hans ord och bespottade hans profeter, till dess HERRENS vrede över hans folk växte så, att ingen bot mer fanns.
2Ch 36:17  Då sände han emot dem kaldéernas konung, och denne dräpte deras unga män med svärd i deras helgedomshus och skonade varken ynglingar eller jungfrur, ej heller gamla och gråhårsmän; allt blev givet i hans hand.
2Ch 36:18  Och alla kärl i Guds hus, både stora och små, och skatterna i HERRENS hus, så ock konungens och hans förnämsta mäns skatter, allt förde han till Babel.
2Ch 36:19  Och man brände upp Guds hus och bröt ned Jerusalems mur, och alla dess palats brände man upp i eld och förstörde alla de dyrbara föremål som funnos där.
2Ch 36:20  Och dem som hade undsluppit svärdet förde han bort i fångenskap till Babel, och de blevo tjänare åt honom och åt hans söner, till dess att perserna kommo till väldet -
2Ch 36:21  för att HERRENS ord genom Jeremias mun skulle uppfyllas - alltså till dess att landet hade fått gottgörelse för sina sabbater. Ty medan det låg öde, hade det sabbat - till dess att sjuttio år hade gått till ända.
2Ch 36:22  Men i den persiske konungens Kores' första regeringsår uppväckte HERREN - för att HERRENS ord genom Jeremias mun skulle fullbordas - den persiske konungen Kores' ande, så att denne lät utropa över hela sitt rike och tillika skriftligen kungöra följande:
2Ch 36:23  "Så säger Kores, konungen i Persien: Alla riken på jorden har HERREN, himmelens Gud, givit mig; och han har anbefallt mig att bygga honom ett hus i Jerusalem i Juda. Vemhelst nu bland eder, som tillhör hans folk, med honom vare HERREN, hans Gud, och han drage ditupp."


\end{document}