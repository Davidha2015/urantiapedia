\begin{document}

\title{Nehemiah}

Neh 1:1  Nehemjas, Hakaljas sons, berättelse. I månaden Kisleu, i det tjugonde året, när jag var i Susans borg,
Neh 1:2  hände sig att Hanani, en av mina bröder, och några andra män kommo från Juda. Och jag frågade dem om judarna, den räddade skara som fanns kvar efter fångenskapen, och om Jerusalem.
Neh 1:3  De sade till mig: "De kvarblivna, de som efter fångenskapen finnas kvar i hövdingdömet, lida stor nöd och smälek, och Jerusalems mur är nedbruten, och dess portar äro uppbrända i eld."
Neh 1:4  När jag hade hört detta, satt jag gråtande och sörjande i flera dagar och fastade och bad inför himmelens Gud.
Neh 1:5  Och jag sade: "Ack HERRE, himmelens Gud, du store och fruktansvärde Gud, du som håller förbund och bevarar nåd mot dem som älska dig och hålla dina bud,
Neh 1:6  låt ditt öra akta härpå, och låt dina ögon vara öppna, och hör din tjänares bön, den som jag nu beder inför dig både dag och natt, för Israels barn, dina tjänare, i det att jag bekänner Israels barns synder, dem som vi hava begått mot dig; ty också jag och min faders hus hava syndat.
Neh 1:7  Vi hava svårt förbrutit oss mot dig; vi hava icke hållit de bud och stadgar och rätter som du gav din tjänare Mose.
Neh 1:8  Men tänk på det ord som du gav din tjänare Mose, när du sade: 'Om I ären otrogna, så skall jag förströ eder bland folken;
Neh 1:9  men om I vänden om till mig och hållen mina bud och gören efter dem, då vill jag, om än edra fördrivna vore vid himmelens ända, likväl församla dem därifrån och låta dem komma till den plats som jag har utvalt till boning åt mitt namn.'
Neh 1:10  De äro ju dina tjänare och ditt folk, som du har förlossat genom din stora kraft och din starka hand.
Neh 1:11  Ack Herre, låt ditt öra akta på din tjänares bön, ja, på vad dina tjänare bedja, de som vilja frukta ditt namn; låt nu din tjänare vara lyckosam och låt honom finna barmhärtighet inför denne man." Jag var då munskänk hos konungen.
Neh 2:1  I månaden Nisan, i Artasastas tjugonde regeringsår, vid ett tillfälle då vin stod framsatt för konungen, tog jag vinet och gav det åt honom. Och jag hade icke förr visat mig sorgsen inför honom;
Neh 2:2  men nu sade konungen till mig: "Varför ser du så sorgsen ut? Du är ju icke sjuk; du måste hava någon hjärtesorg." Då blev jag övermåttan häpen.
Neh 2:3  Och jag sade till konungen: "Må konungen leva evinnerligen! Skulle jag icke se sorgsen ut, då den stad där mina fäders gravar äro ligger öde och dess portar äro förtärda av eld?"
Neh 2:4  Konungen sade till mig: "Vad är det då du begär?" Då bad jag en bön till himmelens Gud
Neh 2:5  och sade till konungen: "Om det så täckes konungen, och om du finner behag i din tjänare, så beder jag att du ville låta mig fara till Juda, till den stad där mina fäders gravar äro, på det att jag åter må bygga upp den."
Neh 2:6  Då frågade konungen mig, allt under det att drottningen satt vid hans sida: "Huru länge kan din resa räcka, och när kan du komma tillbaka?" Då det nu alltså täcktes konungen att låta mig fara, uppgav jag för honom en bestämd tid.
Neh 2:7  Och jag sade till konungen: "Om det så täckes konungen, så må brev givas mig till ståthållarna i landet på andra sidan floden, att de låta mig fara därigenom, till dess jag kommer till Juda,
Neh 2:8  så ock ett brev till Asaf, uppsyningsmannen över den kungliga skogsparken, att han låter mig få virke för att därmed timra upp portarna till borgen som hör till templet, ävensom virke till stadsmuren, så ock till det hus där jag själv skall hava min bostad." Och konungen beviljade mig detta, eftersom min Guds goda hand var över mig.
Neh 2:9  När jag så kom till ståthållarna i landet på andra sidan floden, gav jag dem konungens brev. Och konungen hade sänt med mig härhövitsmän och ryttare.
Neh 2:10  Men då horoniten Sanballat och Tobia, den ammonitiske tjänstemannen, hörde detta, förtröt det dem högeligen att någon hade kommit för att se Israels barn till godo.
Neh 2:11  När jag sedan hade kommit till Jerusalem och varit där i tre dagar,
Neh 2:12  stod jag upp om natten jämte några få män, utan att hava omtalat för någon människa vad min Gud ingav mig i hjärtat att göra för Jerusalem; och det djur som jag red på var det enda jag hade med mig.
Neh 2:13  Och jag drog om natten ut genom Dalporten fram emot Drakkällan och Dyngporten och besåg Jerusalems murar, huru de voro nedbrutna, och huru dess portar voro förtärda av eld.
Neh 2:14  Och jag drog vidare till Källporten och till Konungsdammen, men där var det icke möjligt för djuret att komma fram med mig.
Neh 2:15  Då begav jag mig uppför dalen om natten och besåg muren och vände sedan åter in genom Dalporten och kom så tillbaka.
Neh 2:16  Och föreståndarna hade icke fått veta vart jag hade gått, och vad jag ville göra, ty jag hade ännu icke omtalat något för judarna, prästerna, ädlingarna, föreståndarna och de övriga, som skulle få med arbetet att göra.
Neh 2:17  Men nu sade jag till dem: "I sen själva i vilken nöd vi äro, huru Jerusalem ligger öde, och huru dess portar äro uppbrända i eld. Välan då, låt oss bygga upp Jerusalems mur, för att vi icke längre må vara till smälek."
Neh 2:18  Och jag omtalade för dem huru min Guds hand hade varit mig nådig, så ock vad konungen hade lovat mig. Då sade de: "Vi vilja stå upp och bygga." Och de togo mod till sig för det goda verket.
Neh 2:19  Men när horoniten Sanballat och Tobia, den ammonitiske tjänstemannen, och araben Gesem hörde detta, bespottade de oss och visade förakt för oss; och de sade: "Vad är det I gören? Viljen I sätta eder upp mot konungen?"
Neh 2:20  Då gav jag dem detta svar: "Himmelens Gud skall låta det gå oss väl, och vi, hans tjänare, vilja stå upp och bygga; men I haven ingen del eller rätt eller åminnelse i Jerusalem."
Neh 3:1  Och översteprästen Eljasib och hans bröder, prästerna, stodo upp och byggde Fårporten, vilken de helgade, och i vilken de sedan satte in dörrarna. Vidare byggde de ända fram till Hammeatornet, som de helgade, och vidare fram till Hananeltornet.
Neh 3:2  Därbredvid byggde Jerikos män; och därbredvid byggde Sackur, Imris son.
Neh 3:3  Fiskporten byggdes av Hassenaas barn; de timrade upp den och satte in dess dörrar, dess riglar och bommar.
Neh 3:4  Därbredvid arbetade Meremot, son till Uria, son till Hackos, på att sätta muren i stånd; därbredvid arbetade Mesullam, son till Berekja, son till Mesesabel; och därbredvid arbetade Sadok, Baanas son.
Neh 3:5  Därbredvid arbetade tekoaiterna, men de förnämsta bland dem ville icke böja sin hals till att tjäna sin Herre.
Neh 3:6  Gamla porten sattes i stånd av Jojada, Paseas son, och Mesullam, Besodjas son; de timrade upp den och satte in dess dörrar, dess riglar och bommar.
Neh 3:7  Därbredvid arbetade gibeoniten Melatja och meronotiten Jadon jämte männen från Gibeon och Mispa, som lydde under ståthållaren i landet på andra sidan floden.
Neh 3:8  Därbredvid arbetade Ussiel, Harhajas son, jämte guldsmederna; och därbredvid arbetade Hananja, en av salvoberedarna. Det nästföljande stycket av Jerusalem lät man vara, ända till Breda muren.
Neh 3:9  Därbredvid arbetade Refaja, Hurs son, hövdingen över ena hälften av Jerusalems område.
Neh 3:10  Därbredvid arbetade Jedaja, Harumafs son, mitt emot sitt eget hus; och därbredvid arbetade Hattus, Hasabnejas son.
Neh 3:11  En annan sträcka sattes i stånd av Malkia, Harims son, och av Hassub, Pahat-Moabs son, och därjämte Ugnstornet.
Neh 3:12  Därbredvid arbetade Sallum, Hallohes' son, hövdingen över andra hälften av Jerusalems område, han själv med sina döttrar.
Neh 3:13  Dalporten sattes i stånd av Hanun och Sanoas invånare; de byggde upp den och satte in dess dörrar, dess riglar och bommar. De byggde ock ett tusen alnar på muren, ända fram till Dyngporten.
Neh 3:14  Och Dyngporten sattes i stånd av Malkia, Rekabs son, hövdingen över Bet-Hackerems område; han byggde upp den och satte in dess dörrar, dess riglar och bommar.
Neh 3:15  Och Källporten sattes i stånd av Sallun, Kol-Hoses son, hövdingen över Mispas område; han byggde upp den och lade tak därpå och satte in dess dörrar, dess riglar och bommar. Han byggde ock muren vid Vattenledningsdammen, invid den kungliga trädgården, ända fram till trapporna som föra ned från Davids stad.
Neh 3:16  Därnäst sattes ett stycke i stånd av Nehemja, Asbuks son, hövdingen över ena hälften av Bet-Surs område, nämligen stycket ända fram till platsen mitt emot Davidsgravarna och vidare fram till den grävda dammen och till Hjältehuset.
Neh 3:17  Därnäst arbetade leviterna under Rehum, Banis son; därbredvid arbetade Hasabja, hövdingen över ena hälften av Kegilas område, för sitt område.
Neh 3:18  Därnäst arbetade deras bröder under Bavai, Henadads son, hövdingen över andra hälften av Kegilas område.
Neh 3:19  Därbredvid sattes en annan sträcka i stånd av Eser, Jesuas son, hövdingen över Mispa, nämligen från platsen mitt emot uppgången till tyghuset i Vinkeln.
Neh 3:20  Därnäst sattes, under ivrigt arbete, en annan sträcka i stånd av Baruk, Sabbais son, från Vinkeln ända fram till ingången till översteprästen Eljasibs hus.
Neh 3:21  Därnäst sattes en annan sträcka i stånd av Meremot, son till Uria, son till Hackos, från ingången till Eljasibs hus ända dit där Eljasibs hus slutar.
Neh 3:22  Därnäst arbetade prästerna, männen från Jordanslätten.
Neh 3:23  Därnäst arbetade Benjamin och Hassub, mitt emot sitt eget hus; därnäst arbetade Asarja, son till Maaseja, son till Ananja, utmed sitt eget hus.
Neh 3:24  Därnäst sattes en annan sträcka i stånd av Binnui, Henadads son, från Asarjas hus ända fram till Vinkeln och vidare fram till Hörnet.
Neh 3:25  Palal, Usais son, satte i stånd stycket från platsen mitt emot Vinkeln och det torn som skjuter ut från det övre konungshuset, vid fängelsegården; därnäst kom Pedaja, Pareos' son.
Neh 3:26  (Men tempelträlarna bodde på Ofel ända fram till platsen mitt emot Vattenporten mot öster och det utskjutande tornet.)
Neh 3:27  Därnäst sattes en annan sträcka i stånd av tekoaiterna, från platsen mitt emot det stora utskjutande tornet ända fram till Ofelmuren.
Neh 3:28  Ovanför Hästporten arbetade prästerna, var och en mitt emot sitt eget hus.
Neh 3:29  Därnäst arbetade Sadok, Immers son, mitt emot sitt eget hus; och därnäst arbetade Semaja, Sekanjas son, som hade vakten vid Östra porten.
Neh 3:30  Därnäst sattes en annan sträcka i stånd av Hananja, Selemjas son, och Hanun, Salafs sjätte son; därnäst arbetade Mesullam, Berekjas son, mitt emot sin tempelkammare.
Neh 3:31  Därnäst sattes ett stycke i stånd av Malkia, en av guldsmederna, ända fram till tempelträlarnas och köpmännens hus, mitt emot Mönstringsporten och vidare fram till Hörnsalen.
Neh 3:32  Och mellan Hörnsalen och Fårporten arbetade guldsmederna och köpmännen på att sätta muren i stånd.
Neh 4:1  När nu Sanballat hörde att vi höllo på att bygga upp muren, vredgades han och blev högeligen förtörnad. Och han bespottade judarna
Neh 4:2  och talade så inför sina bröder och inför Samariens krigsfolk: "Vad är det dessa vanmäktiga judar göra? Skall man låta dem hållas? Skola de få offra? Skola de kanhända i sinom tid fullborda sitt verk? Skola de kunna giva liv åt stenarna i grushögarna, där de ligga förbrända?"
Neh 4:3  Och ammoniten Tobia, som stod bredvid honom sade: "Huru de än bygga, skall dock en räv komma deras stenmur att rämna, blott han hoppar upp på den."
Neh 4:4  Hör, vår Gud, huru föraktade vi äro. Låt deras smädelser falla tillbaka på deras egna huvuden. Ja, låt dem bliva utplundrade i ett land dit de föras såsom fångar.
Neh 4:5  Överskyl icke deras missgärningar, och låt deras synd icke varda utplånad ur din åsyn, eftersom de hava varit de byggande till förargelse.
Neh 4:6  Och vi byggde på muren, och hela muren blev hopfogad till sin halva höjd; och folket arbetade med gott mod.
Neh 4:7  Men när Sanballat och Tobia och araberna, ammoniterna och asdoditerna hörde att man alltjämt höll på med att laga upp Jerusalems murar, och att rämnorna begynte igentäppas, då blevo de mycket vreda.
Neh 4:8  Och de sammansvuro sig allasammans att gå åstad och angripa Jerusalem och störa folket i deras arbete.
Neh 4:9  Då bådo vi till vår Gud; och vi läto hålla vakt mot dem både dag och natt för att skydda oss mot dem.
Neh 4:10  Men judarna sade: "Bärarnas kraft sviker, och gruset är alltför mycket; vi förmå icke mer att bygga på muren."
Neh 4:11  Våra ovänner åter sade: "Innan de få veta eller se något, skola vi stå mitt ibland dem och dräpa dem; så skola vi göra slut på arbetet."
Neh 4:12  När nu de judar som bodde i deras grannskap kommo och från alla håll uppmanade oss, väl tio gånger, att vi skulle draga oss tillbaka till dem,
Neh 4:13  då ställde jag upp folket i de lägsta och mest öppna delarna av staden bakom muren; jag ställde upp dem efter släkter, med sina svärd, spjut och bågar.
Neh 4:14  Och sedan jag hade besett allt, stod jag upp och sade till ädlingarna och föreståndarna och det övriga folket: "Frukten icke för dem; tänken på Herren, den store och fruktansvärde, och striden för edra bröder, edra söner och döttrar, edra hustrur och edra hus."
Neh 4:15  Sedan våra fiender sålunda hade fått förnimma att saken var oss bekant, och att Gud hade gjort deras råd om intet, kunde vi alla vända tillbaka till muren, var och en till sitt arbete.
Neh 4:16  Från den dagen var ena hälften av mina tjänare sysselsatt med arbetet, under det att andra hälften stod väpnad med sina spjut, sköldar, bågar och pansar, medan furstarna stodo bakom hela Juda hus.
Neh 4:17  De som byggde på muren och de som lassade på och buro bördor gjorde sitt arbete med den ena handen, och med den andra höllo de vapnet.
Neh 4:18  Och de som byggde hade var och en sitt svärd bundet vid sin länd, under det att de byggde; och bredvid mig stod en basunblåsare.
Neh 4:19  Jag hade nämligen sagt till ädlingarna och föreståndarna och det övriga folket: "Arbetet är stort och vidsträckt, och vi äro spridda över muren, långt ifrån varandra.
Neh 4:20  Där I nu hören basunen ljuda, dit skolen I församla eder till oss; vår Gud skall strida för oss."
Neh 4:21  Så gjorde ock vi vårt arbete, under det att hälften av folket stod väpnad med sina spjut från morgonrodnadens uppgång, till dess att stjärnorna kommo fram.
Neh 4:22  Vid samma tid sade jag ock till folket att var och en med sin tjänare skulle stanna över natten inne i Jerusalem, så att vi om natten kunna hava dem till vakt och om dagen till arbete.
Neh 4:23  Och varken jag eller mina bröder eller mina tjänare eller de som gjorde vakt hos mig lade av kläderna; vapnen höllos av var och en för lika nödvändiga som vatten.
Neh 5:1  Och männen av folket med sina hustrur hovo upp ett stort rop mot sina judiska bröder.
Neh 5:2  Några sade: "Vi med våra söner och döttrar äro många; låt oss få säd, så att vi hava att äta och kunna bliva vid liv."
Neh 5:3  Och några sade: "Våra åkrar, vingårdar och hus måste vi pantsätta; låt oss få säd till att stilla vår hunger."
Neh 5:4  Och andra sade: "Vi hava måst låna penningar på våra åkrar och vingårdar till skatten åt konungen.
Neh 5:5  Nu äro ju våra kroppar lika goda som våra bröders kroppar, och våra barn lika goda som deras barn; men ändå måste vi giva våra söner och döttrar i träldom, ja, några av våra döttrar hava redan blivit givna i träldom, utan att vi förmå göra något därvid, eftersom våra åkrar och vingårdar äro i andras händer."
Neh 5:6  När jag nu hörde deras rop och hörde dessa ord, blev jag mycket vred.
Neh 5:7  Och sedan jag hade gått till råds med mig själv, förebrådde jag ädlingarna och föreståndarna och sade till dem: "Det är ocker I bedriven mot varandra." Därefter sammankallade jag en stor folkförsamling emot dem.
Neh 5:8  Och jag sade till dem: "Vi hava efter förmåga friköpt våra judiska bröder som voro sålda åt hedningarna. Skolen nu I sälja edra bröder? Skola de behöva sälja sig åt oss?" Då tego de och hade intet att svara.
Neh 5:9  Och jag sade: "Vad I gören är icke rätt. I borden ju vandra i vår Guds fruktan, så att våra fiender, hedningarna, ej finge orsak att smäda oss.
Neh 5:10  Också jag och mina bröder och mina tjänare hava penningar och säd att fordra av dem; låt oss nu avstå från vår fordran.
Neh 5:11  Given dem redan i dag tillbaka deras åkrar, vingårdar, olivplanteringar och hus, och skänken efter den ränta på penningarna, på säden, på vinet och oljan, som I haven att fordra av dem."
Neh 5:12  De svarade: "Vi vilja giva det tillbaka och icke utkräva något av dem; vi vilja göra såsom du har sagt." Och jag tog en ed av dem, sedan jag hade tillkallat prästerna, att de skulle göra så.
Neh 5:13  Därjämte skakade jag fånget på min mantel och sade: "Var och en som icke håller detta sitt ord, honom må Gud så skaka bort ifrån hans hus och hans gods; ja, varde han så utskakad och tom på allt." Och hela församlingen sade: "Amen", och lovade HERREN. Därefter gjorde folket såsom det var sagt.
Neh 5:14  Ytterligare är att nämna att från den dag då jag förordnades att vara ståthållare över dem i Juda land, alltså från Artasastas tjugonde regeringsår ända till hans trettioandra, tolv hela år, varken jag eller mina bröder åto av ståthållarkosten.
Neh 5:15  De förra ståthållarna, de som hade varit före mig, hade betungat folket och tagit av dem mat och vin till ett värde av mer än fyrtio siklar silver, och jämväl deras tjänare hade förfarit hårt mot folket. Men så gjorde icke jag, ty jag fruktade Gud.
Neh 5:16  Dessutom höll jag i att arbeta på muren, och ingen åker köpte vi oss; och alla mina tjänare voro församlade vid arbetet där.
Neh 5:17  Och av judarna och deras föreståndare åto ett hundra femtio man vid mitt bord, förutom dem som kommo till oss ifrån folken runt omkring oss.
Neh 5:18  Och vad som tillreddes för var dag, nämligen en oxe och sex utsökta får, förutom fåglar, det tillreddes på min bekostnad; och var tionde dag anskaffades mycket vin av alla slag. Men likväl krävde jag icke ut ståthållarkosten, eftersom arbetet tyngde så svårt på folket.
Neh 5:19  Tänk, min Gud, på allt vad jag har gjort för detta folk, och räkna mig det till godo!
Neh 6:1  När nu Sanballat och Tobia och araben Gesem och våra övriga fiender hörde att jag hade byggt upp muren, och att det icke mer fanns någon rämna i den - om jag ock vid den tiden ännu icke hade satt in dörrar i portarna -
Neh 6:2  då sände Sanballat och Gesem bud till mig och läto säga: "Kom, låt oss träda tillsammans i Kefirim i Onos dal." De tänkte nämligen göra mig något ont.
Neh 6:3  Men jag skickade bud till dem och lät säga: "Jag har ett stort arbete för händer och kan icke komma ned. Arbetet kan ju icke vila, såsom dock måste ske, om jag lämnade det och komme ned till eder."
Neh 6:4  Och de sände samma bud till mig fyra gånger; men var gång gav jag dem samma svar som förut.
Neh 6:5  Då sände Sanballat för femte gången till mig sin tjänare med samma bud, och denne hade nu med sig ett öppet brev.
Neh 6:6  Däri var skrivet: "Det förljudes bland folken, och påstås jämväl av Gasmu, att du och judarna haven i sinnet att avfalla, och att det är därför du bygger upp muren, ja, att du vill bliva deras konung - sådant säger man.
Neh 6:7  Du lär ock hava beställt profeter som i Jerusalem skola utropa och förkunna att du är konung i Juda. Eftersom nu konungen nog får höra talas härom, därför må du nu komma, så att vi få rådslå med varandra."
Neh 6:8  Då sände jag bud till honom och lät svara: "Intet av det du säger har någon grund, utan det är dina egna påfund."
Neh 6:9  De ville nämligen alla skrämma oss, i tanke att vi då skulle förlora allt mod till arbetet, och att detta så skulle bliva ogjort. Styrk du nu i stället mitt mod!
Neh 6:10  Men jag gick hem till Semaja, son till Delaja, son till Mehetabel; han höll sig då inne. Och han sade: "Låt oss tillsammans gå till Guds hus, in i templet, och sedan stänga igen templets dörrar. Ty de skola komma för att dräpa dig; om natten skola de komma för att dräpa dig."
Neh 6:11  Men jag svarade: "Skulle en man sådan som jag vilja fly? Eller kan väl en man av mitt slag gå in i templet och dock bliva vid liv? Nej, jag vill icke gå dit."
Neh 6:12  Jag förstod nämligen att Gud icke hade sänt honom, utan att han förebådade mig sådant, blott därför att Tobia och Sanballat hade lejt honom.
Neh 6:13  Han var lejd, för att jag skulle låta skrämma mig till att göra såsom han sade och därmed försynda mig; på detta sätt ville de framkalla ont rykte om mig, för att sedan kunna smäda mig.
Neh 6:14  Tänk, min Gud på Tobia, ävensom Sanballat, efter dessa hans gärningar, så ock på profetissan Noadja och de andra profeterna som ville skrämma mig!
Neh 6:15  Och muren blev färdig på tjugufemte dagen i månaden Elul, efter femtiotvå dagar.
Neh 6:16  När nu alla våra fiender hörde detta, betogos de, alla de kringboende folken, av fruktan, och sågo att de hade kommit illa till korta; ty de förstodo nu att detta arbete var vår Guds verk.
Neh 6:17  Vid denna tid sände ock Juda ädlingar många brev till Tobia, och brev från Tobia ankommo ock till dem.
Neh 6:18  Ty många i Juda voro genom ed förbundna med honom; han var nämligen måg till Sekanja, Aras son, och hans son Johanan hade tagit till hustru en dotter till Mesullam, Berekjas son.
Neh 6:19  Dessa plägade också inför mig tala gott om honom, och vad jag sade buro de fram till honom. Tobia sände ock brev för att skrämma mig.
Neh 7:1  När nu muren var uppbyggd, satte jag in dörrarna; och dörrvaktare, sångare och leviter blevo anställda.
Neh 7:2  Och till befälhavare över Jerusalem satte jag min broder Hanani jämte Hananja, hövitsman i borgen, ty denne hölls för en pålitlig man och var gudfruktig mer än många andra.
Neh 7:3  Och jag sade till dem: "Jerusalems portar må icke öppnas, förrän solen är högt uppe; och medan vakten ännu står kvar, skall man stänga dörrarna och sätta bommarna för. Och I skolen ställa ut vakter av Jerusalems invånare, var och en på hans post, så att envar får stå framför sitt eget hus."
Neh 7:4  Och staden var vidsträckt och stor, men där fanns icke mycket folk, och husen voro icke uppbyggda.
Neh 7:5  Och min Gud ingav mig i hjärtat att jag skulle församla ädlingarna, föreståndarna och folket för att upptecknas i släktregister. Då fann jag släktförteckningen över dem som först hade dragit upp, och jag fann däri så skrivet:
Neh 7:6  "Dessa voro de män från hövdingdömet, som drogo upp ur den landsflykt och fångenskap till vilken de hade blivit bortförda av Nebukadnessar, konungen i Babel, och som vände tillbaka till Jerusalem och till Juda, var och en till sin stad,
Neh 7:7  i det att de följde med Serubbabel, Jesua, Nehemja, Asarja, Raamja, Nahamani, Mordokai, Bilsan, Misperet, Bigvai, Nehum och Baana. Detta var antalet män av Israels meniga folk:
Neh 7:8  Pareos' barn: två tusen ett hundra sjuttiotvå;
Neh 7:9  Sefatjas barn: tre hundra sjuttiotvå;
Neh 7:10  Aras barn: sex hundra femtiotvå;
Neh 7:11  Pahat-Moabs barn, av Jesuas och Joabs barn: två tusen åtta hundra aderton;
Neh 7:12  Elams barn: ett tusen två hundra femtiofyra;
Neh 7:13  Sattus barn: åtta hundra fyrtiofem;
Neh 7:14  Sackais barn: sju hundra sextio;
Neh 7:15  Binnuis barn: sex hundra fyrtioåtta;
Neh 7:16  Bebais barn: sex hundra tjuguåtta;
Neh 7:17  Asgads barn: två tusen tre hundra tjugutvå;
Neh 7:18  Adonikams barn: sex hundra sextiosju;
Neh 7:19  Bigvais barn: två tusen sextiosju;
Neh 7:20  Adins barn: sex hundra femtiofem;
Neh 7:21  Aters barn av Hiskia: nittioåtta;
Neh 7:22  Hasums barn: tre hundra tjuguåtta;
Neh 7:23  Besais barn: tre hundra tjugufyra;
Neh 7:24  Harifs barn: ett hundra tolv;
Neh 7:25  Gibeons barn: nittiofem;
Neh 7:26  männen från Bet-Lehem och Netofa: ett hundra åttioåtta;
Neh 7:27  männen från Anatot: ett hundra tjuguåtta;
Neh 7:28  männen från Bet-Asmavet: fyrtiotvå;
Neh 7:29  männen från Kirjat-Jearim, Kefira och Beerot: sju hundra fyrtiotre;
Neh 7:30  männen från Rama och Geba: sex hundra tjuguen;
Neh 7:31  männen från Mikmas: ett hundra tjugutvå;
Neh 7:32  männen från Betel och Ai: ett hundra tjugutre;
Neh 7:33  männen från det andra Nebo: femtiotvå;
Neh 7:34  den andre Elams barn: ett tusen två hundra femtiofyra;
Neh 7:35  Harims barn: tre hundra tjugu;
Neh 7:36  Jerikos barn: tre hundra fyrtiofem;
Neh 7:37  Lods, Hadids och Onos barn: sju hundra tjuguen;
Neh 7:38  Senaas barn: tre tusen nio hundra trettio.
Neh 7:39  Av prästerna: Jedajas barn av Jesuas hus: nio hundra sjuttiotre;
Neh 7:40  Immers barn: ett tusen femtiotvå;
Neh 7:41  Pashurs barn: ett tusen två hundra fyrtiosju;
Neh 7:42  Harims barn: ett tusen sjutton.
Neh 7:43  Av leviterna: Jesuas barn av Kadmiel, av Hodevas barn: sjuttiofyra;
Neh 7:44  av sångarna: Asafs barn: ett hundra fyrtioåtta;
Neh 7:45  av dörrvaktarna: Sallums barn, Aters barn, Talmons barn, Ackubs barn, Hatitas barn, Sobais barn: ett hundra trettioåtta.
Neh 7:46  Av tempelträlarna: Sihas barn, Hasufas barn, Tabbaots barn,
Neh 7:47  Keros' barn, Sias barn, Padons barn,
Neh 7:48  Lebanas barn, Hagabas barn, Salmais barn,
Neh 7:49  Hanans barn, Giddels barn, Gahars barn,
Neh 7:50  Reajas barn, Resins barn, Nekodas barn,
Neh 7:51  Gassams barn, Ussas barn, Paseas barn,
Neh 7:52  Besais barn, Meunims barn, Nefusesims barn,
Neh 7:53  Bakbuks barn, Hakufas barn, Harhurs barn,
Neh 7:54  Basluts barn, Mehidas barn, Harsas barn,
Neh 7:55  Barkos' barn, Siseras barn, Temas barn,
Neh 7:56  Nesias barn, Hatifas barn.
Neh 7:57  Av Salomos tjänares barn: Sotais barn, Soferets barn, Peridas barn,
Neh 7:58  Jaalas barn, Darkons barn, Giddels barn,
Neh 7:59  Sefatjas barn, Hattils barn, Pokeret-Hassebaims barn, Amons barn.
Neh 7:60  Tempelträlarna och Salomos tjänares barn utgjorde tillsammans tre hundra nittiotvå.
Neh 7:61  Och dessa voro de som drogo åstad från Tel-Mela, Tel-Harsa, Kerub, Addon och Immer, men som icke kunde uppgiva sina familjer och sin släkt, och huruvida de voro av Israel:
Neh 7:62  Delajas barn, Tobias barn, Nekodas barn, sex hundra fyrtiotvå.
Neh 7:63  Och av prästerna: Habajas barn, Hackos' barn, Barsillais barn, hans som tog en av gileaditen Barsillais döttrar till hustru och blev uppkallad efter deras namn.
Neh 7:64  Dessa sökte efter sina släktregister, men man kunde icke finna dem; därför blevo de såsom ovärdiga uteslutna från prästadömet.
Neh 7:65  Och ståthållaren tillsade dem att de icke skulle få äta av det högheliga, förrän en präst uppstode med urim och tummim.
Neh 7:66  Hela församlingen utgjorde sammanräknad fyrtiotvå tusen tre hundra sextio,
Neh 7:67  förutom deras tjänare och tjänarinnor, som voro sju tusen tre hundra trettiosju. Och till dem hörde två hundra fyrtiofem sångare och sångerskor.
Neh 7:68  De hade sju hundra trettiosex hästar, två hundra fyrtiofem mulåsnor.
Neh 7:69  Och de hade fyra hundra trettiofem kameler och sex tusen sju hundra tjugu åsnor.
Neh 7:70  Och somliga bland huvudmännen för familjerna gåvo skänker till arbetet. Ståthållaren gav till kassan i guld ett tusen dariker, därtill femtio skålar och fem hundra trettio prästerliga livklädnader.
Neh 7:71  Och somliga bland huvudmännen för familjerna gåvo till arbetskassan i guld tjugu tusen dariker och i silver två tusen två hundra minor.
Neh 7:72  Och det övriga folkets gåvor utgjorde i guld tjugu tusen dariker och i silver två tusen minor, så ock sextiosju prästerliga livklädnader.
Neh 7:73  Och prästerna, leviterna, dörrvaktarna, sångarna, en del av meniga folket samt tempelträlarna, korteligen hela Israel, bosatte sig i sina städer."
Neh 8:1  När sjunde månaden nalkades och Israels barn voro bosatta i sina städer, församlade sig folket, alla såsom en man, på den öppna platsen framför Vattenporten; och de bådo Esra, den skriftlärde, att hämta fram Moses lagbok, den som HERREN hade givit åt Israel.
Neh 8:2  Då framlade prästen Esra lagen för församlingen, för både män och kvinnor, alla som kunde förstå vad de hörde; detta var på första dagen i sjunde månaden.
Neh 8:3  Och han föreläste därur vid den öppna platsen framför Vattenporten, från dagningen till middagen, för män och kvinnor, dem som kunde förstå det; och allt folket lyssnade till lagboken.
Neh 8:4  Och Esra, den skriftlärde, stod på en hög träställning som man hade gjort för det ändamålet; och bredvid honom stodo Mattitja, Sema, Anaja, Uria, Hilkia och Maaseja på hans högra sida, och till vänster om honom Pedaja, Misael, Malkia, Hasum, Hasbaddana, Sakarja och Mesullam.
Neh 8:5  Och Esra öppnade boken, så att allt folket såg det, ty han stod högre än allt folket; och när han öppnade den, stod allt folket upp.
Neh 8:6  Och Esra lovade den store HERREN Gud, och allt folket svarade: "Amen, Amen", med uppräckta händer; och de böjde sig ned och tillbådo HERREN med ansiktet mot jorden.
Neh 8:7  Och Jesua, Bani, Serebja, Jamin, Ackub, Sabbetai, Hodia, Maaseja, Kelita, Asarja, Josabad, Hanan, Pelaja och de andra leviterna undervisade folket i lagen, medan folket stod där, var och en på sin plats.
Neh 8:8  Och de föreläste tydligt ur boken, ur Guds lag; och de utlade meningen, så att man förstod det som lästes.
Neh 8:9  Och Nehemja, han som var ståthållare, och prästen Esra, den skriftlärde, och leviterna, som undervisade folket, sade till allt folket: "Denna dag är helgad åt HERREN, eder Gud; sörjen icke och gråten icke." Ty allt folket grät, när de hörde lagens ord.
Neh 8:10  Och han sade ytterligare till dem: "Gån bort och äten eder bästa mat och dricken edert sötaste vin, och sänden omkring gåvor därav till dem som icke hava något tillrett åt sig, ty denna dag är helgad åt vår Herre. Och varen icke bedrövade, ty fröjd i HERREN är eder starkhet."
Neh 8:11  Också leviterna lugnade allt folket och sade: "Varen stilla, ty dagen är helig; varen icke bedrövade."
Neh 8:12  Och allt folket gick bort och åt och drack; de sände ock omkring gåvor av den mat de hade tillagat och gjorde sig mycket glada; ty de hade aktat på det som man hade kungjort för dem.
Neh 8:13  Dagen därefter församlade sig huvudmännen för hela folkets familjer, så ock prästerna och leviterna, till Esra, den skriftlärde, för att giva närmare akt på lagens ord.
Neh 8:14  Och de funno skrivet i lagen att HERREN genom Mose hade bjudit att Israels barn skulle bo i lövhyddor under högtiden i sjunde månaden,
Neh 8:15  och att man skulle kungöra och låta utropa i alla deras städer och i Jerusalem och säga: "Gån ut på bergen och hämten löv av olivträd, planterade eller vilda, och löv av myrten, palmträd och andra lummiga träd, och gören lövhyddor, såsom det är föreskrivet."
Neh 8:16  Då gick folket ut och hämtade sådant och gjorde sig hyddor på tak och på gårdar, var och en åt sig, så ock på gårdarna till Guds hus och på den öppna platsen vid Vattenporten och på den öppna platsen vid Efraimsporten.
Neh 8:17  Och hela församlingen, så många som hade kommit tillbaka ifrån fångenskapen, gjorde sig lövhyddor och bodde i dessa hyddor. Ty från Jesuas, Nuns sons, dagar ända till den dagen hade Israels barn icke gjort så. Och där rådde mycket stor glädje.
Neh 8:18  Och man föreläste ur Guds lagbok var dag, från den första dagen till den sista. Och de höllo högtid i sju dagar, och på åttonde dagen hölls en högtidsförsamling på föreskrivet sätt.
Neh 9:1  Men på tjugufjärde dagen i samma månad församlade sig Israels barn och höllo fasta och klädde sig i sorgdräkt och strödde jord på sina huvuden.
Neh 9:2  Och de som voro av Israels släkt avskilde sig från alla främlingar och trädde så fram och bekände sina synder och sina fäders missgärningar.
Neh 9:3  Och de stodo upp, var och en på sin plats, och man föreläste ur HERRENS, deras Guds, lagbok under en fjärdedel av dagen; och under en annan fjärdedel bekände de sina synder och tillbådo HERREN, sin Gud.
Neh 9:4  Och Jesua och Bani, Kadmiel, Sebanja, Bunni, Serebja, Bani och Kenani trädde upp på leviternas upphöjning och ropade med hög röst till HERREN, sin Gud.
Neh 9:5  Och leviterna Jesua och Kadmiel, Bani, Hasabneja, Serebja, Hodia, Sebanja och Petaja sade: "Stån upp och loven HERREN, eder Gud, från evighet till evighet. Ja, lovat vare ditt härliga namn, som är upphöjt över allt lov och pris.
Neh 9:6  Du allena är HERREN. Du har gjort himlarna och himlarnas himmel och hela deras härskara, jorden och allt vad därpå är, haven och allt vad som är i dem, och det är du som behåller det allt vid liv; och himmelens härskara tillbeder dig.
Neh 9:7  Du är HERREN Gud, som utvalde Abram och förde honom ut från det kaldeiska Ur och gav honom namnet Abraham.
Neh 9:8  Och du fann hans hjärta fast i tron inför dig, och du slöt med honom det förbundet att du skulle giva åt hans säd kananéernas, hetiternas, amoréernas, perisséernas, jebuséernas och girgaséernas land, ja, giva det åt dem; och du uppfyllde dina ord, ty du är rättfärdig.
Neh 9:9  Och du såg till våra fäders betryck i Egypten och hörde deras rop vid Röda havet.
Neh 9:10  Du gjorde tecken och under på Farao och på alla hans tjänare och på allt folket i hans land; ty du förnam att dessa handlade övermodigt mot dem, och du gjorde dig ett namn, som är detsamma än i dag.
Neh 9:11  Havet klöv du itu för dem, så att de gingo mitt igenom havet på torr mark; men deras förföljare lät du sjunka i djupet såsom stenar, i väldiga vatten.
Neh 9:12  Du ledde dem om dagen med en molnstod, och om natten med en eldstod, för att lysa dem på den väg de skulle gå.
Neh 9:13  Och du steg ned på berget Sinai och talade till dem från himmelen och gav dem rättfärdiga rätter och riktiga lagar, goda stadgar och bud.
Neh 9:14  Du gav dem kunskap om din heliga sabbat och gav dem bud och stadgar och lag genom din tjänare Mose.
Neh 9:15  Och du gav dem bröd från himmelen, när de hungrade, och lät vatten komma ut ur klippan, när de törstade; och du tillsade dem att gå och taga i besittning det land som du med upplyft hand hade lovat giva åt dem.
Neh 9:16  Men våra fäder, de voro övermodiga; de voro hårdnackade, så att de icke hörde på dina bud.
Neh 9:17  De ville icke höra och tänkte icke på de under som du hade gjort med dem, utan voro hårdnackade och valde i sin gensträvighet en anförare, för att vända tillbaka till sin träldom. Men du är en förlåtande Gud, nådig och barmhärtig, långmodig och stor i mildhet; och du övergav dem icke.
Neh 9:18  Nej, fastän de gjorde åt sig en gjuten kalv och sade: 'Detta är din Gud, han som har fört dig upp ur Egypten', och fastän de gjorde sig skyldiga till stora hädelser,
Neh 9:19  så övergav du dem likväl icke i öknen, efter din stora barmhärtighet. Molnstoden vek om dagen icke ifrån dem, utan ledde dem på vägen, ej heller eldstoden om natten, utan lyste dem på den väg de skulle gå.
Neh 9:20  Din gode Ande sände du att undervisa dem, och ditt manna förvägrade du icke deras mun, och vatten gav du dem, när de törstade.
Neh 9:21  I fyrtio år försörjde du dem i öknen, så att intet fattades dem; deras kläder blevo icke utslitna, och deras fötter svullnade icke.
Neh 9:22  Och du gav dem riken och folk och utskiftade lotter åt dem på skilda håll; och de intogo Sihons land, det land som tillhörde konungen i Hesbon, och det land som tillhörde Og, konungen i Basan
Neh 9:23  Och du lät deras barn bliva talrika såsom stjärnorna på himmelen, och förde dem in i det land varom du hade sagt till deras fäder att de skulle komma dit och taga det i besittning.
Neh 9:24  Så kommo då barnen och togo landet i besittning, och du kuvade för dem landets inbyggare, kananéerna, och gav dessa i deras hand, både konungarna och folken där i landet, så att de gjorde med dem vad de ville.
Neh 9:25  Och de intogo befästa städer och ett bördigt land och kommo i besittning av hus, fulla med allt gott, och av uthuggna brunnar, vingårdar, olivplanteringar och fruktträd i myckenhet; och de åto och blevo mätta och feta och gjorde sig glada dagar av ditt myckna goda.
Neh 9:26  Men de blevo gensträviga och satte sig upp mot dig och kastade din lag bakom sin rygg och dräpte dina profeter, som varnade dem och ville omvända dem till dig; och de gjorde sig skyldiga till stora hädelser.
Neh 9:27  Då gav du dem i deras ovänners hand, så att dessa förtryckte dem; men när de i sin nöds tid ropade till dig, hörde du det från himmelen, och efter din stora barmhärtighet gav du dem frälsare, som frälste dem ur deras ovänners hand.
Neh 9:28  När de så kommo till ro, gjorde de åter vad ont var inför dig. Då överlämnade du dem i deras fienders hand, så att dessa fingo råda över dem; men när de åter ropade till dig, då hörde du det från himmelen och räddade dem efter din barmhärtighet, många gånger.
Neh 9:29  Och du varnade dem och ville omvända dem till din lag; men de voro övermodiga och hörde icke på dina bud, utan syndade mot dina rätter, om vilka det gäller att den människa som gör efter dem får leva genom dem; de spjärnade emot i gensträvighet och voro hårdnackade och ville icke höra.
Neh 9:30  Du hade fördrag med dem i många år och varnade dem med din Ande genom dina profeter, men de lyssnade icke därtill; då gav du dem i de främmande folkens hand.
Neh 9:31  Men i din stora barmhärtighet gjorde du icke alldeles ände på dem och övergav dem icke; ty du är en nådig och barmhärtig Gud.
Neh 9:32  Och nu, vår Gud, du store, väldige och fruktansvärde Gud, du som håller förbund och bevarar nåd, nu må du icke akta för ringa all den vedermöda som har träffat oss, våra konungar, våra furstar, våra präster, våra fäder och hela ditt folk, ifrån de assyriska konungarnas dagar ända till denna dag.
Neh 9:33  Nej, du är rättfärdig vid allt det som har kommit över oss; ty du har visat dig trofast, men vi hava varit ogudaktiga.
Neh 9:34  Och våra konungar, våra furstar, våra präster och våra fäder hava icke gjort efter din lag och icke aktat på dina bud och på de varningar som du har låtit komma till dem.
Neh 9:35  Och fastän de sutto i sitt eget rike i det myckna goda som du hade givit dem, och i det rymliga och bördiga land som du hade upplåtit för dem, hava de ändå icke tjänat dig och icke omvänt sig från sina onda gärningar.
Neh 9:36  Se, vi äro nu andras tjänare; i det land som du gav åt våra fäder, för att de skulle äta dess frukt och dess goda, just där äro vi andras tjänare,
Neh 9:37  och sin rika avkastning giver det åt de konungar som du för våra synders skull har satt över oss. Och de råda över våra kroppar och vår boskap såsom de vilja, och vi äro i stor nöd."
Neh 9:38  På grund av allt detta slöto vi ett fast förbund och uppsatte det skriftligen; och på skrivelsen, som försågs med sigill, stodo våra furstars, våra leviters och våra prästers namn.
Neh 10:1  Följande namn stodo på skrivelserna som buro sigillen: Nehemja, ståthållaren, Hakaljas son, och Sidkia,
Neh 10:2  Seraja, Asarja, Jeremia,
Neh 10:3  Pashur, Amarja, Malkia,
Neh 10:4  Hattus, Sebanja, Malluk,
Neh 10:5  Harim, Meremot, Obadja,
Neh 10:6  Daniel, Ginneton, Baruk,
Neh 10:7  Mesullam, Abia, Mijamin,
Neh 10:8  Maasja, Bilgai, Semaja; dessa voro prästerna.
Neh 10:9  Och leviterna voro: Jesua, Asanjas son, Binnui, av Henadads barn, Kadmiel,
Neh 10:10  så ock deras bröder: Sebanja, Hodia, Kelita, Pelaja, Hanan,
Neh 10:11  Mika, Rehob, Hasabja,
Neh 10:12  Sackur, Serebja, Sebanja,
Neh 10:13  Hodia, Bani och Beninu.
Neh 10:14  Folkets huvudmän voro: Pareos, Pahat-Moab, Elam, Sattu, Bani,
Neh 10:15  Bunni, Asgad, Bebai,
Neh 10:16  Adonia, Bigvai, Adin,
Neh 10:17  Ater, Hiskia, Assur,
Neh 10:18  Hodia, Hasum, Besai,
Neh 10:19  Harif, Anatot, Nobai,
Neh 10:20  Magpias, Mesullam, Hesir,
Neh 10:21  Mesesabel, Sadok, Jaddua,
Neh 10:22  Pelatja, Hanan, Anaja,
Neh 10:23  Hosea, Hananja, Hassub,
Neh 10:24  Hallohes, Pilha, Sobek,
Neh 10:25  Rehum, Hasabna, Maaseja,
Neh 10:26  Ahia, Hanan, Anan,
Neh 10:27  Malluk, Harim och Baana.
Neh 10:28  Och det övriga folket, prästerna, leviterna, dörrvaktarna, sångarna, tempelträlarna och alla de som hade avskilt sig från de främmande folken och vänt sig till Guds lag, så ock deras hustrur, söner och döttrar, alla som hade kommit till moget förstånd,
Neh 10:29  dessa slöto sig till sina förnämligare bröder och gingo ed och svuro att de skulle vandra efter Guds lag, den som hade blivit given genom Guds tjänare Mose, och att de skulle hålla och göra efter alla HERRENS, vår HERRES, bud och rätter och stadgar,
Neh 10:30  att vi icke skulle giva våra döttrar åt de främmande folken, ej heller taga deras döttrar till hustrur åt våra söner.
Neh 10:31  Och när de främmande folken förde in handelsvaror eller något slags säd till salu på sabbatsdagen, skulle vi icke köpa det av dem på sabbat eller helgdag; och vi skulle låta vart sjunde år vara friår och då avstå från alla slags krav.
Neh 10:32  Och vi fastställde för oss den förpliktelsen att såsom vår gärd årligen erlägga en tredjedels sikel till tjänsten i vår Guds hus,
Neh 10:33  nämligen till skådebröden, och till det dagliga brännoffret, och till offren på sabbaterna, vid nymånaderna och högtiderna, och till tackoffren, och till syndoffren för Israels försoning, och till allt arbete i vår Guds hus.
Neh 10:34  Och vi, prästerna, leviterna och folket, kastade lott angående vedoffret, huru man årligen skulle föra det till vår Guds hus på bestämda tider, efter våra familjer, för att antändas på HERRENS, vår Guds, altare, såsom det är föreskrivet i lagen.
Neh 10:35  Och vi skulle årligen föra till HERRENS hus förstlingen av vår mark, och förstlingen av all frukt på alla slags träd,
Neh 10:36  och de förstfödda av våra söner och av vår boskap, såsom det är föreskrivet i lagen; vi skulle föra till vår Guds hus de förstfödda både av våra fäkreatur och av vår småboskap, till prästerna som gjorde tjänst i vår Guds hus.
Neh 10:37  Och förstlingen av vårt mjöl och våra offergärder, så ock av allt slags trädfrukt, av vin och olja skulle vi föra till prästerna, in i kamrarna i vår Guds hus, och tionden av vår jord till leviterna; ty det var leviterna som skulle uppbära tionden i alla de städer vid vilka vi brukade jorden.
Neh 10:38  Och en präst, en av Arons söner, skulle vara med leviterna, när leviterna uppburo tionden; och själva skulle leviterna föra tionden av sin tionde upp till vår Guds hus, in i förrådshusets kamrar.
Neh 10:39  Ty såväl de övriga israeliterna som Levi barn skulle föra sin offergärd av säd, vin och olja in i dessa kamrar, där helgedomens kärl och de tjänstgörande prästerna, ävensom dörrvaktarna och sångarna voro. Alltså skulle vi icke försumma vår Guds hus.
Neh 11:1  Och folkets furstar bodde i Jerusalem; men det övriga folket kastade lott, för att så var tionde man skulle utses att bo i Jerusalem, den heliga staden, medan nio tiondedelar skulle bo i de andra städerna.
Neh 11:2  Och folket välsignade alla de män som frivilligt bosatte sig i Jerusalem.
Neh 11:3  Och de huvudmän i hövdingdömet, som bodde i Jerusalem, bodde var och en där han hade sin arvsbesittning, i sin stad; vanliga israeliter, präster, leviter och tempelträlar, så ock Salomos tjänares barn.
Neh 11:4  I Jerusalem bodde en del av Juda barn och en del av Benjamins barn, nämligen: Av Juda barn: Ataja, son till Ussia, son till Sakarja, son till Amarja, son till Sefatja, son till Mahalalel, av Peres' barn,
Neh 11:5  så ock Maaseja, son till Baruk, son till Kol-Hose, son till Hasaja, son till Adaja, son till Jojarib, son till Sakarja, silonitens son.
Neh 11:6  Peres' barn som bodde i Jerusalem utgjorde tillsammans fyra hundra sextioåtta stridbara män.
Neh 11:7  Och Benjamins barn voro dessa: Sallu, son till Mesullam, son till Joed, son till Pedaja, son till Kolaja, son till Maaseja, son till Itiel, son till Jesaja,
Neh 11:8  och näst honom Gabbai och Sallai, nio hundra tjuguåtta.
Neh 11:9  Joel, Sikris son, var tillsyningsman över dem, och Juda, Hassenuas son, var den andre i befälet över staden.
Neh 11:10  Av prästerna: Jedaja, Jojaribs son, Jakin
Neh 11:11  samt Seraja, son till Hilkia, son till Mesullam, son till Sadok, son till Merajot, son till Ahitub, fursten i Guds hus,
Neh 11:12  så ock deras bröder, som förrättade sysslorna i huset, åtta hundra tjugutvå; vidare Adaja, son till Jeroham, son till Pelalja, son till Amsi, son till Sakarja, son till Pashur, son till Malkia,
Neh 11:13  så ock hans bröder, huvudmän för familjer, två hundra fyrtiotvå; vidare Amassai, son till Asarel, son till Asai, son till Mesillemot, son till Immer,
Neh 11:14  så ock deras bröder, dugande män, ett hundra tjuguåtta; och tillsyningsman över dem var Sabdiel, Haggedolims son.
Neh 11:15  Och av leviterna: Semaja, son till Hassub, son till Asrikam, son till Hasabja, son till Bunni,
Neh 11:16  så ock Sabbetai och Josabad, som hade uppsikten över de yttre sysslorna vid Guds hus och hörde till leviternas huvudmän,
Neh 11:17  vidare Mattanja, son till Mika, son till Sabdi, son till Asaf, sånganföraren, som vid bönen tog upp lovsången, och Bakbukja, den av hans bröder, som var närmast efter honom, och Abda, son till Sammua, son till Galal, son till Jeditun.
Neh 11:18  Leviterna i den heliga staden utgjorde tillsammans två hundra åttiofyra.
Neh 11:19  Och dörrvaktarna, Ackub, Talmon och deras bröder, som höllo vakt vid portarna, voro ett hundra sjuttiotvå.
Neh 11:20  Och de övriga israeliterna, prästerna och leviterna bodde i alla de andra städerna i Juda, var och en i sin arvedel.
Neh 11:21  Men tempelträlarna bodde på Ofel, och Siha och Gispa hade uppsikten över tempelträlarna.
Neh 11:22  Och tillsyningsman bland leviterna i Jerusalem vid sysslorna i Guds hus var Ussi, son till Bani, son till Hasabja, son till Mattanja, son till Mika, av Asafs barn, sångarna.
Neh 11:23  Ty ett kungligt påbud var utfärdat angående dem, och en bestämd utanordning var för var dag fastställd för sångarna.
Neh 11:24  Och Petaja, Mesesabels son, av Seras, Judas sons, barn, gick konungen till handa i var sak som rörde folket.
Neh 11:25  Och i byarna med tillhörande utmarker bodde ock en del av Juda barn: i Kirjat-Arba och underlydande orter, I Dibon och underlydande orter, i Jekabseel och dess byar,
Neh 11:26  vidare i Jesua, Molada, Bet-Pelet
Neh 11:27  och Hasar-Sual, så ock i Beer-Seba och underlydande orter,
Neh 11:28  i Siklag, ävensom i Mekona och underlydande orter,
Neh 11:29  i En-Rimmon, Sorga, Jarmut,
Neh 11:30  Sanoa, Adullam och deras byar, i Lakis med dess utmarker, i Aseka och underlydande orter; och de hade sina boningsorter från Beer- Seba ända till Hinnoms dal.
Neh 11:31  Och Benjamins barn hade sina boningsorter från Geba: i Mikmas och Aja, så ock i Betel och underlydande orter,
Neh 11:32  i Anatot, Nob, Ananja,
Neh 11:33  Hasor, Rama, Gittaim,
Neh 11:34  Hadid, Seboim, Neballat,
Neh 11:35  Lod, Ono, och Timmermansdalen.
Neh 11:36  Och av leviterna blevo några avdelningar från Juda räknade till Benjamin.
Neh 12:1  Och dessa voro de präster och leviter som drogo upp med Serubbabel, Sealtiels son, och Jesua: Seraja, Jeremia, Esra,
Neh 12:2  Amarja, Malluk, Hattus,
Neh 12:3  Sekanja, Rehum, Meremot,
Neh 12:4  Iddo, Ginnetoi, Abia,
Neh 12:5  Mijamin, Maadja, Bilga,
Neh 12:6  Semaja, Jojarib, Jedaja,
Neh 12:7  Sallu, Amok, Hilkia och Jedaja. Dessa voro huvudmän för prästerna och för sina bröder i Jesuas tid.
Neh 12:8  Och leviterna voro: Jesua, Binnui, Kadmiel, Serebja, Juda och Mattanja, som jämte sina bröder förestod lovsången;
Neh 12:9  vidare Bakbukja och Unno, deras bröder, som hade sina platser mitt emot dem, så att var avdelning hade sin tjänstgöring.
Neh 12:10  Och Jesua födde Jojakim, och Jojakim födde Eljasib, och Eljasib Jojada,
Neh 12:11  och Jojada födde Jonatan, och Jonatan födde Jaddua.
Neh 12:12  Och i Jojakims tid voro huvudmännen för prästernas familjer följande: för Seraja Meraja, för Jeremia Hananja,
Neh 12:13  för Esra Mesullam, för Amarja Johanan,
Neh 12:14  för Malluki Jonatan, för Sebanja Josef,
Neh 12:15  för Harim Adna, för Merajot Helkai,
Neh 12:16  för Iddo Sakarja, för Ginneton Mesullam,
Neh 12:17  för Abia Sikri, för Minjamin, för Moadja Piltai,
Neh 12:18  för Bilga Sammua, för Semaja Jonatan,
Neh 12:19  för Jojarib Mattenai, för Jedaja Ussi,
Neh 12:20  för Sallai Kallai, för Amok Eber,
Neh 12:21  för Hilkia Hasabja, för Jedaja Netanel.
Neh 12:22  I Eljasibs, Jojadas, Johanans och Jadduas tid blevo huvudmännen för leviternas familjer upptecknade, ävenså prästerna under persern Darejaves' regering.
Neh 12:23  Huvudmännen för Levi barns familjer äro upptecknade i krönikeboken, ända till Johanans, Eljasibs sons, tid.
Neh 12:24  Och leviternas huvudmän voro Hasabja, Serebja och Jesua, Kadmiels son, samt deras bröder, som stodo mitt emot dem för att lova och tacka, såsom gudsmannen David hade bjudit, den ena tjänstgörande avdelningen jämte den andra.
Neh 12:25  Mattanja, Bakbukja, Obadja, Mesullam, Talmon och Ackub höllo såsom dörrvaktare vakt över förrådshusen vid portarna.
Neh 12:26  Dessa levde i Jojakims, Jesuas sons, Josadaks sons, tid, och i Nehemjas, ståthållarens, och i prästen Esras, den skriftlärdes, tid.
Neh 12:27  Och när Jerusalems mur skulle invigas, uppsökte man leviterna på alla deras orter och förde dem till Jerusalem för att hålla invignings- och glädjehögtid under tacksägelse och sång, med cymbaler, psaltare och harpor.
Neh 12:28  Då församlade sig sångarnas barn såväl från nejden runt omkring Jerusalem som från netofatiternas byar,
Neh 12:29  ävensom från Bet-Haggilgal och från Gebas och Asmavets utmarker; ty sångarna hade byggt sig byar runt omkring Jerusalem.
Neh 12:30  Och prästerna och leviterna renade sig och renade sedan folket, portarna och muren.
Neh 12:31  Och jag lät Juda furstar stiga upp på muren. Därefter anordnade jag två stora lovsångskörer och högtidståg; den ena kören gick till höger ovanpå muren, fram till Dyngporten.
Neh 12:32  Och dem följde Hosaja och ena hälften av Juda furstar
Neh 12:33  samt Asarja, Esra och Mesullam,
Neh 12:34  Juda, Benjamin, Semaja och Jeremia,
Neh 12:35  ävensom några av prästerna söner med trumpeter, vidare Sakarja, son till Jonatan, son till Semaja, son till Mattanja, son till Mikaja, son till Sackur, son till Asaf,
Neh 12:36  så ock hans bröder Semaja, Asarel, Milalai, Gilalai, Maai, Netanel och Juda samt Hanani, med gudsmannen Davids musikinstrumenter; och Esra, den skriftlärde, gick i spetsen för dem.
Neh 12:37  Och de gingo över Källporten och rakt fram uppför trapporna till Davids stad, på trappan i muren ovanför Davids hus, ända fram till Vattenporten mot öster.
Neh 12:38  Och efter den andra lovsångskören, som gick åt motsatt håll, följde jag med andra hälften av folket, ovanpå muren, upp genom Ugnstornet ända till Breda muren,
Neh 12:39  vidare över Efraimsporten, Gamla porten och Fiskporten och genom Hananeltornet, ända fram till Fårporten; och de stannade vid Fängelseporten.
Neh 12:40  Sedan trädde de båda lovsångskörerna upp i Guds hus, och likaså jag och ena hälften av föreståndarna jämte mig,
Neh 12:41  så ock prästerna Eljakim, Maaseja, Minjamin, Mikaja, Eljoenai, Sakarja och Hananja, med trumpeterna,
Neh 12:42  och Maaseja, Semaja, Eleasar, Ussi, Johanan, Malkia, Elam och Eser. Och sångarna läto sången ljuda under Jisrajas anförarskap.
Neh 12:43  Och de offrade på den dagen stora offer och voro glada, ty Gud hade berett dem stor glädje; också kvinnor och barn voro glada. Och glädjen från Jerusalem hördes vida omkring.
Neh 12:44  Vid samma tid tillsattes män som skulle förestå förrådskamrarna där offergärder, förstling och tionde nedlades; de skulle i dem hopsamla från stadsåkrarna det som efter lagen tillkom prästerna och leviterna. Ty glädje rådde i Juda över att prästerna och leviterna nu gjorde sin tjänst.
Neh 12:45  Dessa iakttogo nu vad som var att iakttaga vid gudstjänsten och vid reningarna, och likaså gjorde sångarna och dörrvaktarna sin tjänst, såsom David och hans son Salomo hade bjudit.
Neh 12:46  Ty redan i fordom tid, på Davids och Asafs tid, hans som var anförare för sångarna, sjöngos lov- och tacksägelsesånger till Gud.
Neh 12:47  Och nu under Serubbabels och Nehemjas tid gav hela Israel åt sångarna och dörrvaktarna vad som tillkom dem för var dag; och man gav åt leviterna deras helgade andel, och leviterna gåvo åt Arons söner deras helgade andel.
Neh 13:1  Vid samma tid föreläste man ur Moses bok för folket, och man fann däri skrivet att ingen ammonit eller moabit någonsin skulle få komma in i Guds församling,
Neh 13:2  därför att de icke hade kommit Israels barn till mötes med mat och dryck, utan hade lejt Bileam emot dem till att förbanna dem; fastän vår Gud förvandlade förbannelsen till välsignelse.
Neh 13:3  Och när de hade hört lagen, avskilde de allt slags främmande folk från Israel.
Neh 13:4  Men en tid förut hade prästen Eljasib, som var satt att förestå kammaren i vår Guds hus, och som var en frände till Tobia,
Neh 13:5  åt denne inrett en stor kammare, där man förut plägade lägga in spisoffret, rökelsen och kärlen och den tionde av säd, vin och olja, som var bestämd åt leviterna, sångarna och dörrvaktarna, så ock offergärden åt prästerna.
Neh 13:6  Men under allt detta var jag icke i Jerusalem; ty i den babyloniske konungen Artasastas trettioandra regeringsår hade jag återkommit till konungen. Men sedan jag efter någon tid hade utbett mig tillstånd av konungen,
Neh 13:7  begav jag mig till Jerusalem. Och när jag där förnam det onda som Eljasib hade gjort till förmån för Tobia, då han hade inrett åt honom en kammare i förgårdarna till Guds hus,
Neh 13:8  misshagade detta mig högeligen; och jag lät kasta allt Tobias bohag ut ur kammaren.
Neh 13:9  Därefter tillsade jag att man skulle rena kamrarna, och jag lät åter ställa in i dem Guds hus' kärl, så ock spisoffret och rökelsen.
Neh 13:10  Och när jag vidare fick veta att man icke hade givit åt leviterna vad dem tillkom, varför ock leviterna och sångarna, i stället för att förrätta sina sysslor, hade avvikit var och en till sitt jordagods,
Neh 13:11  då förebrådde jag föreståndarna detta och sade: "Varför har Guds hus blivit så försummat?" Och jag hämtade dem tillhopa och lät dem inställa sig på sina platser.
Neh 13:12  Och hela Juda förde fram till förrådshusen sin tionde av säd, vin och olja;
Neh 13:13  och jag satte prästen Selemja och Sadok, den skriftlärde, och Pedaja, en av leviterna, till förvaltare över förrådshusen och gav dem till biträde Hanan, son till Sackur, son till Mattanja; ty dessa voro ansedda såsom pålitliga män, och de skulle nu ombesörja utdelningen åt sina bröder.
Neh 13:14  Tänk fördenskull på mig, min Gud, och låt icke de fromma gärningar bliva utplånade, som jag har gjort för min Guds hus och för tjänstgöringen där!
Neh 13:15  Vid samma tid såg jag i Juda huru man trampade vinpressarna på sabbaten och förde hem säd, som man lastade på åsnor, så ock vin, druvor och fikon och annat lastgods av olika slag, och huru man förde sådant till Jerusalem på sabbatsdagen; och jag varnade dem, när de sålde dessa livsförnödenheter.
Neh 13:16  Och tyrierna, som vistades där, förde in fisk och alla slags varor och sålde dem på sabbaten till judarna, och detta i Jerusalem.
Neh 13:17  Då förebrådde jag Juda ädlingar detta och sade till dem: "Huru kunnen I handla så illa och därmed ohelga sabbatsdagen?
Neh 13:18  Var det icke därför att edra fäder gjorde sådant som vår Gud lät all denna olycka komma över oss och över denna stad? Och nu dragen I ännu större vrede över Israel genom att så ohelga sabbaten."
Neh 13:19  Och så snart det begynte bliva mörkt i Jerusalems portar före sabbaten, tillsade jag att man skulle stänga dörrarna; jag tillsade ock att man icke skulle öppna dem förrän efter sabbaten. Och jag ställde några av mina tjänare på vakt vid portarna, för att intet lastgods skulle kunna föras in på sabbatsdagen.
Neh 13:20  Då stannade köpmän och försäljare av alla slags varor utanför Jerusalem över natten, och det både en och två gånger.
Neh 13:21  Men jag varnade dem och sade till dem: "Varför stannen I över natten framför muren? Om I ännu en gång gören så, skall jag låta min hand drabba eder."
Neh 13:22  Och jag tillsade leviterna att de skulle rena sig och komma och hålla vakt vid portarna, för att sabbatsdagen måtte hållas helig. Tänk ock därför på mig, min Gud, och hav misskund med mig efter din stora nåd!
Neh 13:23  På den tiden såg jag också judiska män som hade tagit till sig asdoditiska, ammonitiska och moabitiska kvinnor.
Neh 13:24  Och deras barn talade till hälften asdoditiska - ty judiska kunde de icke tala riktigt - eller ock något av de andra folkens tungomål.
Neh 13:25  Då förebrådde jag dem detta och uttalade förbannelser över dem, ja, några av dem slog jag och ryckte jag i skägget. Och jag besvor dem vid Gud och sade: "I skolen icke giva edra döttrar åt deras söner, ej heller skolen I av deras döttrar taga hustrur åt edra söner eller åt eder själva.
Neh 13:26  Var det icke med sådant som Salomo, Israels konung, försyndade sig? Det fanns bland de många folken ingen konung som var hans like, ty han var älskad av sin Gud, och Gud satte honom till konung över hela Israel. Likväl kommo de främmande kvinnorna också honom att synda.
Neh 13:27  Och nu skulle vi om eder få höra att I haven gjort allt detta stora onda och varit otrogna mot vår Gud, i det att I haven tagit till eder främmande kvinnor!"
Neh 13:28  Och en son till Jojada, översteprästen Eljasibs son, var måg till horoniten Sanballat; honom drev jag bort ifrån mig.
Neh 13:29  Tänk på dem, min Gud, därför att de hava befläckat prästadömet och prästadömets och leviternas förbund!
Neh 13:30  Så renade jag folket ifrån allt främmande väsen; och jag fastställde vad prästerna och leviterna skulle iakttaga, var och en i sin syssla,
Neh 13:31  och huru vedoffret på bestämda tider skulle avlämnas, och huru med förstlingsgåvorna skulle förfaras. Tänk härpå, min Gud, och räkna mig det till godo!


\end{document}