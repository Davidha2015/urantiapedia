\begin{document}

\title{Esther}

Est 1:1  I Ahasveros' tid - den Ahasveros' som regerade från Indien ända till Etiopien, över ett hundra tjugusju hövdingdömen -
Est 1:2  under den tiden, medan konung Ahasveros satt på konungatronen i Susans borg, tilldrog sig följande.
Est 1:3  I sitt tredje regeringsår gjorde han ett gästabud för alla sina furstar och tjänare, varvid Persiens och Mediens härförare och hans förnämsta män och furstarna i hövdingdömena voro samlade inför honom.
Est 1:4  Och han lät dem under många dagar se sin konungsliga härlighet och rikedom och sin storhets glans och prakt - under ett hundra åttio dagar.
Est 1:5  Och när dessa dagar hade gått till ända, gjorde konungen ett sju dagars gästabud för allt det folk som fanns i Susans borg, både stora och små, i den inhägnade trädgård som hörde till konungapalatset.
Est 1:6  Där hängde tapeter av linne, bomull och mörkblått tyg, uppsatta med vita och purpurröda snören i ringar av silver och på pelare av vit marmor. Soffor av guld och silver stodo på ett golv som var inlagt med grön och vit marmor och med pärlglänsande och svart sten.
Est 1:7  Och dryckerna sattes fram i gyllene kärl, det ena icke likt det andra, och konungsligt vin fanns i myckenhet, såsom det hövdes hos en konung.
Est 1:8  Och när man drack, gällde den lagen att intet tvång skulle råda; ty konungen hade befallt alla sina hovmästare att de skulle rätta sig efter vars och ens önskan.
Est 1:9  Samtidigt gjorde ock Vasti, drottningen, ett gästabud för kvinnorna i konung Ahasveros' kungliga palats.
Est 1:10  När då på sjunde dagen konungens hjärta var glatt av vinet, befallde han Mehuman, Bisseta, Harebona, Bigeta, Abageta, Setar och Karkas, de sju hovmän som gjorde tjänst hos konung Ahasveros,
Est 1:11  att de skulle föra drottning Vasti, prydd med kunglig krona, inför konungen, för att han skulle låta folken och furstarna se hennes skönhet, ty hon var fager att skåda.
Est 1:12  Men drottning Vasti ville icke komma, fastän konungen befallde henne det genom hovmännen. Då blev konungen mycket förtörnad. och hans vrede upptändes.
Est 1:13  Och konungen frågade de vise som voro kunniga i tidstecknens tydning (ty konungens ärenden plägade så läggas fram för alla i lag och rätt kunniga;
Est 1:14  och han hade vid sin sida Karsena, Setar, Admata, Tarsis, Meres, Marsena och Memukan, de sju furstar i Persien och Medien, som voro konungens närmaste män och innehade främsta platsen i riket); han frågade:
Est 1:15  "Vad skall man efter lag göra med drottning Vasti, då hon nu icke har gjort vad konung Ahasveros befallde genom hovmännen?"
Est 1:16  Memukan svarade inför konungen och furstarna: "Icke mot konungen allena har drottning Vasti gjort illa, utan mot alla furstar och alla folk i alla konung Ahasveros' hövdingdömen.
Est 1:17  Ty vad drottningen har gjort skall komma ut bland alla kvinnor, och skall leda till att de förakta sina män, då de ju kunna säga: 'Konung Ahasveros befallde att man skulle föra drottning Vasti inför honom, men hon kom icke.'
Est 1:18  Ja, redan i dag skola furstinnorna i Persien och Medien, när de få höra vad drottningen har gjort, åberopa detta inför alla konungens furstar, och därav skall komma förakt och förtret mer än nog.
Est 1:19  Om det så täckes konungen, må han därför låta en kunglig befallning utgå - och må denna upptecknas i Persiens och Mediens lagar, så att den bliver orygglig - att Vasti icke mer skall få komma inför konung Ahasveros' ansikte; och hennes konungsliga värdighet give konungen åt en annan, som är bättre än hon.
Est 1:20  När så den förordning som konungen utfärdar bliver kunnig i hela hans rike, så stort det är, då skola alla kvinnor giva sina män tillbörlig ära, både stora och små."
Est 1:21  Detta tal behagade konungen och furstarna, och konungen gjorde såsom Memukan hade sagt.
Est 1:22  Skrivelser blevo sända till alla konungens hövdingdömen, till vart hövdingdöme med dess skrift och till vart folk på dess tungomål, att envar man skulle vara herre i sitt hus och tala sitt folks tungomål.
Est 2:1  Efter en tids förlopp, sedan konung Ahasveros' vrede hade lagt sig, tänkte han åter på Vasti och vad hon hade gjort, och vad som var beslutet om henne.
Est 2:2  Då sade konungens män som betjänade honom: "Må man för konungens räkning söka upp unga och fagra jungfrur,
Est 2:3  och må konungen i sitt rikes alla hövdingdömen förordna vissa män som samla tillhopa alla dessa unga och fagra jungfrur till fruhuset i Susans borg och överlämna dem åt konungens hovman Hege, kvinnovaktaren, och man give dem vad nödigt är till deras beredelse.
Est 2:4  Och den kvinna som konungen finner behag i blive drottning i Vastis ställe." Detta tal behagade konungen, och han gjorde så.
Est 2:5  I Susans borg fanns då en judisk man som hette Mordokai, son till Jair, son till Simei, son till Kis, en benjaminit;
Est 2:6  denne hade blivit bortförd från Jerusalem med de fångar som fördes bort tillsammans med Jekonja, Juda konung, när denne fördes bort av Nebukadnessar, konungen i Babel.
Est 2:7  Han var fosterfader åt Hadassa, som ock kallades Ester, hans farbroders dotter; ty hon hade varken fader eller moder. Hon var en flicka med skön gestalt, fager att skåda; och efter hennes faders och moders död hade Mordokai upptagit henne såsom sin egen dotter.
Est 2:8  Då nu konungens befallning och påbud blev kunnigt, och många unga kvinnor samlades tillhopa till Susans borg och överlämnades åt Hegai, blev ock Ester hämtad till konungshuset och överlämnad åt kvinnovaktaren Hegai.
Est 2:9  Och flickan behagade honom och fann nåd inför honom; därför skyndade han att giva henne vad nödigt var till hennes beredelse, så ock den kost hon skulle hava, ävensom att giva henne från konungshuset de sju tärnor som utsågos åt henne. Och han lät henne med sina tärnor flytta in i den bästa delen av fruhuset.
Est 2:10  Men om sitt folk och sin släkt hade Ester icke yppat något, ty Mordokai hade förbjudit henne att yppa något därom.
Est 2:11  Och Mordokai gick var dag fram och åter utanför gården till fruhuset, för att få veta huru det stod till med Ester, och vad som vederfors henne.
Est 2:12  Nu var det så, att när ordningen kom till den ena eller andra av de unga kvinnorna att gå in till konung Ahasveros, sedan med henne hade förfarits i tolv månader såsom det var påbjudet om kvinnorna (så lång tid åtgick nämligen till att bereda dem: sex månader med myrraolja och sex månader med välluktande kryddor och annat som var nödigt till kvinnornas beredelse),
Est 2:13  när alltså en kvinna gick in till konungen, då fick hon taga med sig ifrån fruhuset till konungshuset allt vad hon begärde.
Est 2:14  Och sedan hon om aftonen hade gått ditin, skulle hon om morgonen, när hon gick tillbaka, gå in i det andra fruhuset och överlämnas åt konungens hovman Saasgas, som hade vakten över bihustrurna. Hon fick sedan icke mer komma in till konungen, om icke konungen hade funnit sådant behag i henne, att hon uttryckligen blev kallad till honom.
Est 2:15  Då nu ordningen att gå in till konungen kom till Ester, dotter till Abihail, farbroder till Mordokai, som hade upptagit henne till sin dotter, begärde hon intet annat än det som konungens hovman Hegai, kvinnovaktaren, rådde henne till. Och Ester fann nåd för allas ögon, som sågo henne.
Est 2:16  Ester blev hämtad till konung Ahasveros i hans kungliga palats i tionde månaden, det är månaden Tebet, i hans sjunde regeringsår.
Est 2:17  Och Ester blev konungen kärare än alla de andra kvinnorna, och hon fann nåd och ynnest inför honom mer än alla de andra jungfrurna, så att han satte en kunglig krona på hennes huvud och gjorde henne till drottning i Vastis ställe.
Est 2:18  Och konungen gjorde ett stort gästabud för alla sina furstar och tjänare, ett gästabud till Esters ära; och han beviljade skattelindring åt sina hövdingdömen och delade ut skänker, såsom det hövdes en konung.
Est 2:19  När sedermera jungfrur för andra gången samlades tillhopa och Mordokai satt i konungens port
Est 2:20  (men Ester hade, såsom Mordokai bjöd henne, icke yppat något om sin släkt och sitt folk, ty Ester gjorde efter Mordokais befallning, likasom när hon var under hans vård),
Est 2:21  vid den tiden, under det att Mordokai satt i konungens port, blevo Bigetan och Teres, två av de hovmän hos konungen, som höllo vakt vid tröskeln, förbittrade på konung Ahasveros och sökte tillfälle att bära hand på honom.
Est 2:22  Härom fick Mordokai kunskap, och han berättade det för drottning Ester; därefter omtalade Ester det för konungen på Mordokais vägnar.
Est 2:23  Saken blev nu undersökt och så befunnen; och de blevo båda upphängda på trä. Och detta upptecknades i krönikan, för konungen.
Est 3:1  En tid härefter upphöjde konung Ahasveros agagiten Haman, Hammedatas son, till hög värdighet och gav honom främsta platsen bland alla de furstar som voro hos honom.
Est 3:2  Och alla konungens tjänare som voro i konungens port böjde knä och föllo ned för Haman, ty så hade konungen bjudit om honom. Men Mordokai böjde icke knä och föll icke ned för honom.
Est 3:3  Då sade konungens tjänare som voro i konungens port till Mordokai: "Varför överträder du konungens bud?"
Est 3:4  Och när de dag efter dag hade sagt så till honom, utan att han lyssnade till dem, berättade de det för Haman, för att se om Mordokais förklaring skulle få gälla: ty han hade berättat för dem att han var en jude.
Est 3:5  När nu Haman såg att Mordokai icke böjde knä eller föll ned för honom, uppfylldes han med vrede.
Est 3:6  Men det syntes honom för ringa att bära hand allenast på Mordokai, sedan man berättat för honom av vilket folk Mordokai var, utan Haman sökte tillfälle att utrota alla judar som funnos i Ahasveros' hela rike, därför att de voro Mordokais landsmän.
Est 3:7  I första månaden, det är månaden Nisan, i Ahasveros' tolfte regeringsår, kastades pur, det är lott, inför Haman om var särskild dag och var särskild månad intill tolfte månaden, det är månaden Adar.
Est 3:8  Och Haman sade till konung Ahasveros: "Här finnes ett folk som bor kringspritt och förstrött bland de andra folken i ditt rikes alla hövdingdömen. Deras lagar äro olika alla andra folks, och de göra icke efter konungens lagar; därför är det icke konungen värdigt att låta dem vara.
Est 3:9  Om det så täckes konungen, må fördenskull en skrivelse utfärdas, att man skall förgöra dem. Tio tusen talenter silver skall jag då kunna väga upp åt tjänstemännen till att läggas in i konungens skattkamrar."
Est 3:10  Då tog konungen ringen av sin hand och gav den åt agagiten Haman, Hammedatas son, judarnas ovän.
Est 3:11  Därefter sade konungen till Haman: "Silvret vare dig skänkt, och med folket må du göra såsom du finner för gott."
Est 3:12  Så blevo då konungens sekreterare tillkallade på trettonde dagen i första månaden, och en skrivelse, alldeles sådan som Haman ville, utfärdades till konungens satraper och till ståthållarna över de särskilda hövdingdömena och till furstarna över de särskilda folken, till vart hövdingdöme med dess skrift och till vart folk på dess tungomål. I konung Ahasveros' namn utfärdades skrivelsen, och den beseglades med konungens ring.
Est 3:13  Sedan kringsändes med ilbud brev till alla konungens hövdingdömen, att man skulle utrota, dräpa och förgöra judarna, både unga och gamla, både barn och kvinnor, alla på en och samma dag, nämligen på trettonde dagen i tolfte månaden, det är månaden Adar, varvid ock deras ägodelar såsom byte skulle givas till plundring.
Est 3:14  I skrivelsen stod att i vart särskilt hövdingdöme ett påbud, öppet för alla folk, skulle utfärdas, som innehöll att de skulle vara redo den dagen.
Est 3:15  Och på grund av konungens befallning drogo ilbuden med hast åstad, så snart påbudet hade blivit utfärdat i Susans borg. Men konungen och Haman satte sig ned till att dricka, under det att bestörtning rådde i staden Susan.
Est 4:1  När Mordokai fick veta allt vad som hade skett, rev han sönder sina kläder och klädde sig i säck och aska, och gick så ut i staden och uppgav högljudda och bittra klagorop.
Est 4:2  Och han begav sig till konungens port och stannade framför den, ty in i konungens port fick ingen komma, som var klädd i sorgdräkt.
Est 4:3  Och i vart hövdingdöme dit konungens befallning och påbud kom blev stor sorg bland judarna, och de fastade, gräto och klagade, ja, de flesta satte sig i säck och aska.
Est 4:4  När nu Esters tjänarinnor och hovmän kommo och berättade detta för henne, blev drottningen högeligen förskräckt; och hon skickade ut kläder till Mordokai, för att man skulle kläda honom i dem och taga av honom sorgdräkten; men han tog icke emot dem.
Est 4:5  Då kallade Ester till sig Hatak, en av de hovmän som konungen hade anställt i hennes tjänst, och bjöd honom att gå till Mordokai, för att få veta vad som var på färde, och varför han gjorde så.
Est 4:6  När då Hatak kom ut till Mordokai på den öppna platsen i staden framför konungens port,
Est 4:7  berättade Mordokai för honom allt vad som hade hänt honom, och uppgav beloppet av den penningsumma som Haman hade lovat väga upp till konungens skattkamrar, för att han skulle få förgöra judarna.
Est 4:8  Och en avskrift av det skrivna påbud som hade blivit utfärdat i Susan om att de skulle utrotas lämnade han honom ock, för att han skulle visa Ester den och berätta allt för henne, och ålägga henne att gå in till konungen och bedja honom om misskund och söka nåd hos honom för sitt folk.
Est 4:9  Och Hatak kom och berättade för Ester vad Mordokai hade sagt.
Est 4:10  Då bjöd Ester Hatak att gå till Mordokai och säga:
Est 4:11  "Alla konungens tjänare och folket i konungens hövdingdömen veta, att om någon, vare sig man eller kvinna, går in till konungen på den inre gården utan att vara kallad, så gäller för var och en samma lag: att han skall dödas, såframt icke konungen räcker ut mot honom den gyllene spiran, till tecken på att han får leva. Men jag har icke på trettio dagar varit kallad att komma till konungen."
Est 4:12  När man nu berättade för Mordokai vad Ester hade sagt,
Est 4:13  sade Mordokai att man skulle giva Ester detta svar: "Tänk icke att du ensam bland alla judar skall slippa undan, därför att du är i konungens hus.
Est 4:14  Nej, om du tiger stilla vid detta tillfälle, så skall nog hjälp och räddning beredas judarna från något annat håll, men du och din faders hus, I skolen förgöras. Vem vet om du icke just för en sådan tid som denna har kommit till konungslig värdighet?"
Est 4:15  Då lät Ester giva Mordokai detta svar:
Est 4:16  "Gå åstad och församla alla judar som finnas i Susan, och hållen fasta för mig; I skolen icke äta eller dricka något under tre dygn, vare sig dag eller natt. Jag med mina tärnor vill ock sammalunda fasta; därefter vill jag gå in till konungen, fastän det är emot lagen. Och skall jag gå förlorad, så må det då ske."
Est 4:17  Och Mordokai gick bort och gjorde alldeles såsom Ester hade bjudit honom.
Est 5:1  På tredje dagen klädde Ester sig i konungslig skrud och trädde in på den inre gården till konungshuset, mitt emot själva konungshuset; konungen satt då på sin konungatron i det kungliga palatset, mitt emot palatsets dörr.
Est 5:2  När nu konungen såg drottning Ester stå på gården, fann hon nåd för hans ögon, så att konungen räckte ut mot Ester den gyllene spira, som han hade i sin hand; då gick Ester fram och rörde vid ändan av spiran.
Est 5:3  Och konungen sade till henne: "Vad önskar du, drottning Ester, och vad är din begäran? Gällde den ock hälften av riket, så skall den beviljas dig."
Est 5:4  Ester svarade: "Om det så täckes konungen, må konungen jämte Haman i dag komma till ett gästabud, som jag har tillrett för honom."
Est 5:5  Då sade konungen: "Skynden att hämta hit Haman, för att så må ske, som Ester har begärt." Så kommo då konungen och Haman till gästabudet, som Ester hade tillrett.
Est 5:6  Och när vinet dracks, sade konungen till Ester: "Vad är din bön? Den vare dig beviljad. Och vad är din begäran? Gällde den ock hälften av riket, så skall den uppfyllas."
Est 5:7  Ester svarade och sade: "Min bön och min begäran är:
Est 5:8  om jag har funnit nåd för konungens ögon, och det täckes konungen att bevilja min bön och uppfylla min begäran, så må konungen och Haman komma till ännu ett gästabud, som jag vill tillreda för dem; då skall jag i morgon göra såsom konungen har befallt."
Est 5:9  Och Haman gick därifrån den dagen, glad och väl till mods. Men när han fick se Mordokai i konungens port och denne varken stod upp eller ens rörde sig för honom, då uppfylldes Haman med vrede mot Mordokai.
Est 5:10  Men Haman betvang sig och gick hem; därefter sände han och lät hämta sina vänner och sin hustru Seres.
Est 5:11  Och Haman talade för dem om sin rikedom och härlighet och om sina många barn och om all den storhet, som konungen hade givit honom, och om huru konungen i allt hade upphöjt honom över de andra furstarna och konungens övriga tjänare.
Est 5:12  Och Haman sade ytterligare: "Icke heller har drottning Ester låtit någon annan än mig komma med konungen till det gästabud, som hon hade tillrett; och jämväl i morgon är jag bjuden till henne, jämte konungen.
Est 5:13  Men vid allt detta kan jag dock icke vara till freds, så länge jag ser juden Mordokai sitta i konungens port."
Est 5:14  Då sade hans hustru Seres och alla hans vänner till honom: "Låt resa upp en påle, femtio alnar hög, och bed i morgon konungen, att Mordokai må bliva upphängd därpå; då kan du glad komma med konungen till gästabudet." Detta behagade Haman, och han lät resa upp pålen.
Est 6:1  Den natten kunde konungen icke sova; därför lät han hämta krönikan, där minnesvärda händelser voro upptecknade, och man föreläste ur den för konungen.
Est 6:2  Då fann man där skrivet, att Mordokai hade berättat, hurusom Bigetana och Teres, två av de hovmän, som höllo vakt vid tröskeln, hade sökt tillfälle att bära hand på konung Ahasveros.
Est 6:3  Konungen frågade: "Vilken ära och upphöjelse har vederfarits Mordokai för detta?" Konungens män, som betjänade honom, svarade: "Intet sådant har vederfarits honom."
Est 6:4  Då sade konungen: "Är någon nu tillstädes på gården?" Och Haman hade just kommit in på den yttre gården till konungshuset för att bedja konungen, att Mordokai måtte bliva upphängd på den påle, som han hade låtit sätta upp för hans räkning.
Est 6:5  Så svarade honom då konungens tjänare: "Ja, Haman står därute på gården." Konungen sade: "Låt honom komma in."
Est 6:6  När då Haman kom in, sade konungen till honom: "Huru skall man göra med den man, som konungen vill ära?" Men Haman tänkte i sitt hjärta: "Vem skulle konungen vilja bevisa ära mer än mig?"
Est 6:7  Därför sade Haman till konungen: "Om konungen vill ära någon,
Est 6:8  så skall man hämta en konungslig klädnad, som konungen själv har burit, och en häst, som konungen själv har ridit på, och på vilkens huvud en kunglig krona är fäst;
Est 6:9  och man skall överlämna klädnaden och hästen åt en av konungens förnämsta furstar, och klädnaden skall sättas på den man, som konungen vill ära, och man skall föra honom ridande på hästen fram på den öppna platsen i staden och utropa framför honom: 'Så gör man med den man, som konungen vill ära.'"
Est 6:10  Då sade konungen till Haman: "Skynda dig att taga klädnaden och hästen, såsom du har sagt, och gör så med juden Mordokai, som sitter i konungens port. Underlåt intet av allt vad du har sagt."
Est 6:11  Så tog då Haman klädnaden och hästen och satte klädnaden på Mordokai och förde honom ridande fram på den öppna platsen i staden och utropade framför honom: "Så gör man med den man, som konungen vill ära."
Est 6:12  Och Mordokai vände tillbaka till konungens port; men Haman skyndade hem, sörjande och med överhöljt huvud.
Est 6:13  Och när Haman förtäljde för sin hustru Seres och alla sina vänner vad som hade hänt honom, sade hans vise män och hans hustru Seres till honom: "Om Mordokai, som du har begynt att stå tillbaka för, är av judisk börd, så förmår du intet mot honom, utan skall komma alldeles till korta för honom."
Est 6:14  Medan de ännu så talade med honom, kommo konungens hovmän för att skyndsamt hämta Haman till gästabudet, som Ester hade tillrett.
Est 7:1  Så kommo då konungen och Haman till gästabudet hos drottning Ester.
Est 7:2  Och när vinet dracks, sade konungen till Ester, också nu på andra dagen: "Vad är din bön, drottning Ester? Den vare dig beviljad. Och vad är din begäran? Gällde den ock hälften av riket, så skall den uppfyllas."
Est 7:3  Drottning Ester svarade och sade: "Om jag har funnit nåd för dina ögon, o konung, och det så täckes konungen, så blive mitt liv mig skänkt på min bön, och mitt folks på min begäran.
Est 7:4  Ty vi äro sålda, jag och mitt folk, till att utrotas, dräpas och förgöras. Om vi allenast hade blivit sålda till trälar och trälinnor, så skulle jag hava tegat; ty den olyckan vore icke sådan, att vi borde besvära konungen därmed."
Est 7:5  Då svarade konung Ahasveros och sade till drottning Ester: "Vem är den, och var är den, som har fördristat sig att så göra?"
Est 7:6  Ester sade: "En hätsk och illvillig man är det: den onde Haman där." Då blev Haman förskräckt för konungen och drottningen.
Est 7:7  Och konungen stod upp i vrede och lämnade gästabudet och gick ut i palatsets trädgård; men Haman trädde fram för att bedja drottning Ester om sitt liv, ty han såg, att konungen hade beslutit hans ofärd.
Est 7:8  När konungen därefter kom tillbaka till gästabudssalen från palatsets trädgård, hade Haman sjunkit ned mot den soffa, där Ester satt; då sade konungen: "Vill han ock öva våld mot drottningen, härinne i min närvaro?" Knappt hade detta ord gått över konungens läppar, förrän man höljde över Hamans ansikte.
Est 7:9  Och Harebona, en av hovmännen hos konungen, sade: "Vid Hamans hus står redan en påle, femtio alnar hög, som Haman låtit resa upp för Mordokai, vilkens ord en gång var konungen till sådant gagn." Då sade konungen: "Hängen upp honom på den."
Est 7:10  Så hängde de upp Haman på den påle, som han hade låtit sätta upp för Mordokai. Sedan lade sig konungens vrede.
Est 8:1  Samma dag gav konung Ahasveros åt drottning Ester Hamans, judarnas oväns, hus. Och Mordokai fick tillträde till konungen, ty Ester hade nu omtalat, vad han var för henne.
Est 8:2  Och konungen tog av sig ringen, som han hade låtit taga ifrån Haman, och gav den åt Mordokai. Och Ester satte Mordokai över Hamans hus.
Est 8:3  Och Ester talade ytterligare inför konungen, i det att hon föll ned för hans fötter; hon bönföll honom gråtande, att han skulle avvända agagiten Hamans onda råd och det anslag, som denne hade förehaft mot judarna.
Est 8:4  Då räckte konungen ut den gyllene spiran mot Ester; och Ester stod upp och trädde fram inför konungen
Est 8:5  och sade: "Om det så täckes konungen, och om jag har funnit nåd inför honom, och det synes konungen vara riktigt och jag är honom till behag, så må en skrivelse utfärdas för att återkalla de brev, som innehöllo agagiten Hamans, Hammedatas sons, anslag, och som han skrev för att förgöra judarna i alla konungens hövdingdömen.
Est 8:6  Ty huru skulle jag kunna uthärda att se den olycka, som eljest träffade mitt folk? Ja, huru skulle jag kunna uthärda att se mina landsmän förgöras?"
Est 8:7  Då sade konung Ahasveros till drottning Ester och till juden Mordokai: "Se, Hamans hus har jag givit åt Ester, och han själv har blivit upphängd på en påle, därför att han ville bära hand på judarna.
Est 8:8  Men utfärden nu ock I en skrivelse angående judarna i konungen namn, såsom I finnen för gott, och beseglen den med konungens ring. Ty en skrivelse, som är utfärdad i konungens namn och beseglad med konungens ring, kan icke återkallas."
Est 8:9  Så blevo nu strax konungens sekreterare tillkallade, på tjugutredje dagen i tredje månaden, det är månaden Sivan, och en skrivelse, alldeles sådan som Mordokai ville, utfärdades till judarna och till satraperna, ståthållarna och furstarna i hövdingdömena, från Indien ända till Etiopien, ett hundra tjugusju hövdingdömen, till vart hövdingdöme med dess skrift och till vart folk på dess tungomål, jämväl till judarna med deras skrift och på deras tungomål.
Est 8:10  Han utfärdade skrivelsen i konung Ahasveros' namn och beseglade den med konungens ring. Därefter kringsände han brev med ilbud till häst, som redo på kungliga travare från stuterierna,
Est 8:11  att konungen tillstadde judarna i var särskild stad att församla sig till försvar för sitt liv och att i vart folk och hövdingdöme utrota, dräpa och förgöra alla väpnade skaror, som angrepe dem, ävensom barn och kvinnor, varvid deras ägodelar såsom byte skulle givas till plundring,
Est 8:12  detta på en och samma dag i alla konung Ahasveros' hövdingdömen, nämligen på trettonde dagen i tolfte månaden, det är månaden Adar.
Est 8:13  I skrivelsen stod, att i vart särskilt hövdingdöme ett påbud, öppet för alla folk, skulle utfärdas, som innehöll, att judarna skulle vara redo till den dagen att hämnas på sina fiender.
Est 8:14  Och på grund av konungens befallning drogo ilbuden på de kungliga travarna skyndsamt och med hast åstad, så snart påbudet hade blivit utfärdat i Susans borg.
Est 8:15  Men Mordokai gick ut från konungen i konungslig klädnad av mörkblått och vitt tyg och med en stor gyllene krona och en mantel av vitt och purpurrött tyg, under det att staden Susan jublade och var glad.
Est 8:16  För judarna hade nu uppgått ljus och glädje, fröjd och ära.
Est 8:17  Och i vart hövdingdöme och i var stad, dit konungens befallning och påbud kom, blev glädje och fröjd bland judarna, och de höllo gästabud och högtid. Och många ur de främmande folken blevo judar, ty förskräckelse för judarna hade fallit över dem.
Est 9:1  På trettonde dagen i tolfte månaden, det är månaden Adar, den dag då konungens befallning och påbud skulle verkställas, och då judarnas fiender hade hoppats att bliva dem övermäktiga - fastän det vände sig så, att judarna i stället skulle bliva sina motståndare övermäktiga -
Est 9:2  på den dagen församlade sig judarna i sina städer, i alla konung Ahasveros' hövdingdömen, för att kasta sig över dem, som sökte deras ofärd; och ingen kunde stå dem emot, ty förskräckelse för dem hade fallit över alla folk.
Est 9:3  Och alla furstarna i hövdingdömena och satraperna och ståthållarna och konungens tjänstemän understödde judarna, ty förskräckelse för Mordokai hade fallit över dem.
Est 9:4  Ty Mordokai var nu stor i konungens hus, och hans rykte gick ut i alla hövdingdömen, eftersom denne Mordokai lev allt större och större.
Est 9:5  Och judarna anställde med sina svärd ett nederlag överallt bland sina fiender och dräpte och förgjorde dem och förforo såsom de ville med sina motståndare.
Est 9:6  I Susans borg dräpte och förgjorde judarna fem hundra män.
Est 9:7  Och Parsandata, Dalefon, Aspata,
Est 9:8  Porata, Adalja, Aridata,
Est 9:9  Parmasta, Arisai, Aridai och Vajsata,
Est 9:10  judarnas ovän Hamans, Hammedatas sons, tio söner dräpte de; men till plundring räckte de icke ut sin hand.
Est 9:11  Samma dag fick konungen veta huru många som hade blivit dräpta i Susans borg.
Est 9:12  Då sade konungen till drottning Ester: "I Susans borg hava judarna dräpt och förgjort fem hundra män utom Hamans tio söner; vad skola de då icke hava gjort i konungens övriga hövdingdömen? Vad är nu din bön? Den vare dig beviljad. Och vad är ytterligare din begäran? Den skall uppfyllas."
Est 9:13  Ester svarade: "Om det så täckes konungen, så må det också i morgon tillstädjas de judar, som äro i Susan, att göra efter påbudet för i dag; och må Hamans tio söner bliva upphängda på pålen."
Est 9:14  Då befallde konungen, att så skulle ske, och påbudet blev utfärdat i Susan; därefter blevo Hamans tio söner upphängda.
Est 9:15  och de judar, som voro i Susan, församlade sig också på fjortonde dagen i månaden Adar och dräpte i Susan tre hundra män; men till plundring räckte de icke ut sin hand.
Est 9:16  Och de övriga judarna, de som voro i konungen hövdingdömen, församlade sig till försvar för sitt liv och skaffade sig ro för sina fiender, i det att dräpte sjuttiofem tusen av dessa sina motståndare; men till plundring räckte de icke ut sin hand.
Est 9:17  Detta skedde på trettonde dagen i månaden Adar; men på fjortonde dagen vilade de och firade den såsom en gästabuds- och glädjedag.
Est 9:18  De judar åter, som voro i Susan, hade församlat sig både den trettonde dagen och på den fjortonde; men de vilade på den femtonde dagen och firade den såsom en gästabuds- och glädjedag.
Est 9:19  Därför fira judarna på landsbygden, de som bo i landsortsstäderna, den fjortonde dagen i månaden Adar såsom en glädje-, gästabuds- och högtidsdag, på vilken de sända gåvor till varandra av den mat de hava tillagat.
Est 9:20  Och Mordokai tecknade upp dessa händelser och sände skrivelser till alla judar i konung Ahasveros' hövdingdömen, både nära och fjärran,
Est 9:21  och stadgade såsom lag för dem, att de alltid, år efter år, skulle fira den fjortonde och den femtonde dagen i månaden Adar,
Est 9:22  eftersom det var på dessa dagar som judarna hade fått ro för sina fiender, och eftersom i denna månad deras bedrövelse hade blivit förvandlad till glädje och deras sorg till högtid. Därför skulle de fira dessa dagar såsom gästabuds- och glädjedagar, på vilka de skulle sända gåvor till varandra av den mat de hade tillagat, så ock skänker till de fattiga.
Est 9:23  Och judarna antogo såsom sed, vad de nu hade begynt att göra, det varom Mordokai hade skrivit till dem -
Est 9:24  detta eftersom agagiten Haman, Hammedatas son, alla judars ovän, hade förehaft sitt anslag mot judarna till att förgöra dem och hade kastat pur, det är lott, till att plötsligt överfalla och förgöra dem;
Est 9:25  varemot konungen, när han hade fått veta detta, hade givit befallning och utfärdat en skrivelse om att det onda anslag, som denne hade förehaft mot judarna, skulle vända tillbaka på hans eget huvud, så att han själv och hans söner hade blivit upphängda på pålen.
Est 9:26  Fördenskull blevo dessa dagar kallade purim efter ordet pur; och fördenskull, i anledning av allt som stod i detta brev, och vad de själva härav hade sett, och vad som hade vederfarits dem,
Est 9:27  stadgade judarna och antogo såsom orygglig sed för sig och sina efterkommande och för alla, som slöto sig till dem, att alltid, år efter år, fira dessa båda dagar, efter föreskriften om dem och på den för dem bestämda tiden,
Est 9:28  och att dessa dagar skulle ihågkommas och firas i alla tider, i var släkt, i vart hövdingdöme och i var stad, så att dessa purimsdagar oryggligt skulle hållas bland judarna och deras åminnelse icke upphöra bland deras efterkommande.
Est 9:29  Men drottning Ester, Abihails dotter, och juden Mordokai uppsatte ånyo en skrivelse, i eftertryckliga ordalag, för att stadga såsom en lag, vad som föreskrevs i detta nya brev om purim.
Est 9:30  Och skrivelser, vänligt och välvilligt avfattade, utsändes till alla judar i de ett hundra tjugusju hövdingdömena i Ahasveros' rike,
Est 9:31  för att stadga såsom lag, att de skulle fira dessa purimsdagar på deras bestämda tider så, som juden Mordokai och drottning Ester stadgade för dem, och så, som de stadgade för sig själva och sina efterkommande, nämligen med föreskrivna fastor och övliga klagorop.
Est 9:32  Alltså blevo genom Esters befallning dessa föreskrifter om purim stadgade såsom lag; och den tecknades upp i en bok.
Est 10:1  Och konung Ahasveros tog skatt både av fastlandet och av öarna i havet.
Est 10:2  Och allt vad han i sin makt och sin väldighet gjorde, ävensom berättelsen om den storhet, till vilken konungen upphöjde Mordokai, det finnes upptecknat i de mediska och persiska konungarnas krönika.
Est 10:3  Ty juden Mordokai var konung Ahasveros' närmaste man, och han var stor bland judarna och älskad av alla sina bröder, eftersom han sökte sitt folks bästa och lade sig ut för alla sina landsmän till deras välfärd.


\end{document}