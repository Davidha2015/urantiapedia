\begin{document}

\title{Job}


\chapter{1}

\par 1 I Us' land levde en man som hette Job; han var en ostrafflig och redlig man, som fruktade Gud och flydde det onda.
\par 2 Åt honom föddes sju söner och tre döttrar;
\par 3 och han ägde sju tusen får, tre tusen kameler, fem hundra par oxar och fem hundra åsninnor, därtill tjänare i stor mängd. Så var denne man mäktigare än någon annan i Österlandet.
\par 4 Och hans söner hade för sed att gå åstad och hålla gästabud, den ena dagen i den enes hus, den andra dagen i den andres; de sände då och inbjödo sina tre systrar att äta och dricka tillsammans med dem.
\par 5 När så en omgång av gästabudsdagar var till ända, sände Job efter dem för att helga dem; bittida om morgonen offrade han då ett brännoffer för var och en av dem. Ty Job tänkte "Kanhända hava mina barn syndat och i sina hjärtan talat förgripligt om Gud". Så gjorde Job för var gång.
\par 6 Men nu hände sig en dag att Guds söner kommo och trädde fram inför HERREN, och Åklagaren kom också med bland dem.
\par 7 Då frågade HERREN Åklagaren: "Varifrån kommer du?" Åklagaren svarade HERREN och sade: "Från en vandring utöver jorden och från en färd omkring på den."
\par 8 Då sade HERREN till Åklagaren: "Har du givit akt på min tjänare Job? Ty på jorden finnes icke hans like i ostrafflighet och redlighet, ingen som så fruktar Gud och flyr det onda."
\par 9 Åklagaren svarade HERREN och sade: "Är det då för intet som Job fruktar Gud?
\par 10 Du har ju på allt sätt beskärmat honom och hans hus och allt vad han äger; du har välsignat hans händers verk, och hans boskapshjordar hava utbrett sig i landet.
\par 11 Man räck ut din hand och kom vid detta allt som han äger; förvisso skall han då mitt i ansiktet tala förgripliga ord mot dig."
\par 12 HERREN sade till Åklagaren: "Välan, allt vad han äger vare givet i din hand; allenast mot honom själv må du icke räcka ut din hand." Så gick Åklagaren bort ifrån HERRENS ansikte.
\par 13 När nu en dag hans söner och döttrar höllo måltid och drucko vin i den äldste broderns hus,
\par 14 Kom en budbärare till Job och sade: "Oxarna gingo för plogen, och åsninnorna betade därbredvid;
\par 15 då föllo sabéerna in och rövade bort dem, och folket slogo de med svärdsegg. Jag var den ende som kom undan, för att jag skulle underrätta dig därom."
\par 16 Medan denne ännu talade, kom åter en och sade: "Guds eld föll ifrån himmelen och slog ned bland småboskapen och folket och förtärde dem. Jag var den ende som kom undan, för att jag skulle underrätta dig därom."
\par 17 Medan denne ännu talade, kom åter en och sade: Kaldéerna ställde upp sitt manskap i tre hopar och föllo så över kamelerna och rövade bort dem, och folket slogo de med svärdsegg. Jag var den ende som kom undan, för att jag skulle underrätta dig därom."
\par 18 Under det att denne ännu talade, kom åter en annan och sade: Dina söner och döttrar höllo måltid och drucko vin i den äldste broderns hus;
\par 19 då kom en stark storm fram över öknen och tog tag i husets fyra hörn, och det föll omkull över folket, så att de förgingos. Jag var den ende som kom undan, för att jag skulle underrätta dig därom."
\par 20 Då stod Job upp och rev sönder sin mantel och skar av håret på sitt huvud. Och han föll ned till jorden och tillbad
\par 21 och sade: "Naken kom jag ur min moders liv, och naken skall jag vända åter dit; HERREN gav, och HERREN tog. Lovat vare HERRENS namn!"
\par 22 Vid allt detta syndade Job icke och talade intet lasteligt mot Gud.

\chapter{2}

\par 1 Åter hände sig en dag att Guds söner kommo och trädde fram inför HERREN; och Åklagaren kom också med bland dem och trädde fram inför HERREN.
\par 2 Då frågade HERREN Åklagaren: "Varifrån kommer du?" Åklagaren svarade HERREN och sade: "Från en vandring utöver jorden och från en färd omkring på den."
\par 3 Då sade HERREN till Åklagaren: "Har du givit akt på min tjänare Job? Ty på jorden finnes icke hans like i ostrafflighet och redlighet, ingen som så fruktar Gud och flyr det onda; och ännu håller han fast vid sin ostrafflighet. Så har du då uppeggat mig mot honom till att utan sak fördärva honom."
\par 4 Åklagaren svarade HERREN och sade: "Hud för hud; allt vad man äger giver man ju för att själv slippa undan.
\par 5 Men räck ut din hand och kom vid hans kött och ben; förvisso skall han då mitt i ansiktet tala förgripliga ord mot dig."
\par 6 HERREN sade till Åklagaren: "Välan, han vare given i din hand; allenast hans liv må du skona."
\par 7 Så gick Åklagaren bort ifrån HERRENS ansikte och slog Job med svåra bulnader, ifrån fotbladet ända till hjässan.
\par 8 Och han tog sig en lerskärva att skrapa sig med, där han satt mitt i askan.
\par 9 Då sade hans hustru till honom: "Håller du ännu fast vid din ostrafflighet? Tala fritt ut om Gud, och dö."
\par 10 Man han svarade henne: "Du talar såsom en dåraktig kvinna skulle tala. Om vi taga emot det goda av Gud, skola vi då icke också taga emot det onda?" Vid allt detta syndade Job icke med sina läppar.
\par 11 Men tre vänner till Job fingo höra om alla de olyckor som hade träffat honom, och de kommo så, var och en från sin ort; Elifas från Teman, Bildad från Sua och Sofar från Naama. Och de avtalade med varandra att de skulle begiva sig åstad för att ömka honom och trösta honom.
\par 12 Men när de, ännu på avstånd, lyfte upp sina ögon och sågo att de icke mer kunde känna igen honom, brusto de ut i gråt och revo sönder sina mantlar och kastade stoft mot himmelen, ned över sina huvuden.
\par 13 Sedan sutto de med honom på jorden i sju dagar och sju nätter, utan att någon av dem talade ett ord till honom, eftersom de sågo att hans plåga var mycket stor.

\chapter{3}

\par 1 Därefter upplät Job sin mun och förbannade sin födelsedag;
\par 2 Job tog till orda och sade:
\par 3 Må den dag utplånas, på vilken jag föddes, och den natt som sade: "Ett gossebarn är avlat."
\par 4 Må den dagen vändas i mörker, må Gud i höjden ej fråga efter den och intet dagsljus lysa däröver.
\par 5 Mörkret och dödsskuggan börde den åter, molnen lägre sig över den; förskräcke den allt som kan förmörka en dag.
\par 6 Den natten må gripas av tjockaste mörker; ej må den få fröjda sig bland årets dagar, intet rum må den finna inom månadernas krets.
\par 7 Ja, ofruktsam blive den natten, aldrig höje sig jubel under den.
\par 8 Må den förbannas av dem som besvärja dagar, av dem som förmå mana upp Leviatan.
\par 9 Må dess grynings stjärnor förmörkas, efter ljus må den bida, utan att det kommer, morgonrodnadens ögonbryn må den aldrig få se;
\par 10 eftersom den ej tillslöt dörrarna till min moders liv, ej lät olyckan förbliva dold för mina ögon.
\par 11 Varför fick jag ej dö strax i modersskötet, förgås vid det jag kom ut ur min moders liv?
\par 12 Varför funnos knän mig till mötes, och varför bröst, där jag fick di?
\par 13 Hade så icke skett, låge jag nu i ro, jag finge då sova, jag njöte då min vila,
\par 14 vid sidan av konungar och rådsherrar i landet, män som byggde sig palatslika gravar,
\par 15 ja, vid sidan av furstar som voro rika på guld och hade sina hus uppfyllda av silver;
\par 16 eller vore jag icke till, lik ett nedgrävt foster, lik ett barn som aldrig fick se ljuset.
\par 17 Där hava ju de ogudaktiga upphört att rasa, där få de uttröttade komma till vila;
\par 18 där hava alla fångar fått ro, de höra där ingen pådrivares röst.
\par 19 Små och stora äro där varandra lika, trälen har där blivit fri ifrån sin herre.
\par 20 Varför skulle den olycklige skåda ljuset? Ja, varför gives liv åt dem som plågas så bittert,
\par 21 åt dem som vänta efter döden, utan att den kommer, och spana därefter mer än efter någon skatt,
\par 22 åt dem som skulle glädjas - ja, intill jubel - och fröjda sig, allenast de funne sin grav;
\par 23 varför åt en man vilkens väg är höljd i mörker, åt en man så kringstängd av Gud?
\par 24 Suckan har ju blivit mitt dagliga bröd, och såsom vatten strömma mina klagorop.
\par 25 ty det som ingav mig förskräckelse, det drabbar mig nu, och vad jag fruktade för, det kommer över mig.
\par 26 Jag får ingen rast, ingen ro, ingen vila; ångest kommer över mig.

\chapter{4}

\par 1 Därefter tog Elifas från Teman till orda och sade:
\par 2 Misstycker du, om man dristar tala till dig? Vem kan hålla tillbaka sina ord?
\par 3 Se, många har du visat till rätta, och maktlösa händer har du stärkt;
\par 4 dina ord hava upprättat den som stapplade, och åt vacklande knän har du givit kraft.
\par 5 Men nu, då det gäller dig själv, bliver du otålig, när det är dig det drabbar, förskräckes du.
\par 6 Skulle då icke din gudsfruktan vara din tillförsikt och dina vägars ostrafflighet ditt hopp?
\par 7 Tänk efter: när hände det att en oskyldig fick förgås? och var skedde det att de redliga måste gå under?
\par 8 Nej, så har jag sett det gå, att de som plöja fördärv och de som utså olycka, de skörda och sådant;
\par 9 för Guds andedräkt förgås de och för en fnysning av hans näsa försvinna de.
\par 10 Ja, lejonets skri och rytarens röst måste tystna, och unglejonens tänder brytas ut;
\par 11 Det gamla lejonet förgås, ty det finner intet rov, och lejoninnans ungar bliva förströdda.
\par 12 Men till mig smög sakta ett ord, mitt öra förnam det likasom en viskning,
\par 13 När tankarna svävade om vid nattens syner och sömnen föll tung på människorna,
\par 14 då kom en förskräckelse och bävan över mig, med rysning fyllde den alla ben i min kropp.
\par 15 En vindpust for fram över mitt ansikte, därvid reste sig håren på min kropp.
\par 16 Och något trädde inför mina ögon, en skepnad vars form jag icke skönjde; och jag hörde en susning och en röst:
\par 17 "Kan då en människa hava rätt mot Gud eller en man vara ren inför sin skapare?
\par 18 Se, ej ens på sina tjänare kan han förlita sig, jämväl sina änglar måste han tillvita fel;
\par 19 huru mycket mer då dem som bo i hyddor av ler, dem som hava sin grundval i stoftet! De krossas sönder så lätt som mal;
\par 20 när morgon har bytts till afton, ligga de slagna; innan man aktar därpå, hava de förgåtts för alltid.
\par 21 Ja, deras hyddas fäste ryckes bort för dem, oförtänkt måste de dö."

\chapter{5}

\par 1 Ropa fritt; vem finnes, som svarar dig, och till vilken av de heliga kan du vända dig?
\par 2 Se, dåren dräpes av sin grämelse, och den fåkunnige dödas av sin bitterhet.
\par 3 Jag såg en dåre, fast var han rotad, men plötsligt måste jag ropa ve över hans boning.
\par 4 Ty hans barn gå nu fjärran ifrån frälsning, de förtrampas i porten utan räddning.
\par 5 Av hans skörd äter vem som är hungrig, den rövas bort, om och hägnad med törnen; efter hans rikedom gapar ett giller.
\par 6 Ty icke upp ur stoftet kommer fördärvet, ej ur marken skjuter olyckan upp;
\par 7 nej, människan varder född till olycka, såsom eldgnistor måste flyga mot höjden.
\par 8 Men vore det nu jag, så sökte jag nåd hos Gud, åt Gud hemställde jag min sak,
\par 9 åt honom som gör stora och outrannsakliga ting, under, flera än någon kan räkna,
\par 10 åt honom som låter regnet falla på jorden och sänder vatten ned över markerna,
\par 11 när han vill upphöja de ringa och förhjälpa de sörjande till frälsning.
\par 12 Han är den som gör de klokas anslag om intet, så att deras händer intet uträtta med förnuft;
\par 13 han fångar de visa i deras klokskap och låter de illfundiga förhasta sig i sina rådslag:
\par 14 mitt på dagen råka de ut för mörker och famla mitt i ljuset, likasom vore det natt.
\par 15 Så frälsar han från deras tungors svärd, han frälsar den fattige ur den övermäktiges hand.
\par 16 Den arme kan så åter hava ett hopp, och orättfärdigheten måste tillsluta sin mun.
\par 17 Ja, säll är den människa som Gud agar; den Allsmäktiges tuktan må du icke förkasta.
\par 18 Ty om han och sargar, så förbinder han ock, om han slår, så hela ock hans händer.
\par 19 Sex gånger räddar han dig ur nöden, ja, sju gånger avvändes olyckan från dig.
\par 20 I hungerstid förlossar han dig från döden och i krig undan svärdets våld.
\par 21 När tungor svänga gisslet, gömmes du undan; du har intet att frukta, när förhärjelse kommer.
\par 22 Ja, åt förhärjelse och dyr tid kan du då le, för vilddjur behöver du ej heller känna fruktan;
\par 23 ty med markens stenar står du i förbund, och med djuren på marken har du ingått fred.
\par 24 Och du får se huru din hydda står trygg; när du synar din boning, saknas intet däri.
\par 25 Du får ock se huru din ätt förökas, huru din avkomma bliver såsom markens örter.
\par 26 I graven kommer du, när du har hunnit din mognad, såsom sädesskylen bärgas, då dess tid är inne.
\par 27 Se, detta hava vi utrannsakat, och så är det; hör därpå och betänk det väl.

\chapter{6}

\par 1 Då tog Job till orda och sade:
\par 2 Ack att min grämelse bleve vägd och min olycka lagd jämte den på vågen!
\par 3 Se, tyngre är den nu än havets sand, därför kan jag icke styra mina ord.
\par 4 Ty den Allsmäktiges pilar hava träffat mig, och min ande indricker deras gift; ja, förskräckelser ifrån Gud ställa sig upp mot mig.
\par 5 Icke skriar vildåsnan, när hon har friskt gräs, icke råmar oxen, då han står vid sitt foder?
\par 6 Men vem vill äta den mat som ej har smak eller sälta, och vem finner behag i slemörtens saft?
\par 7 Så vägrar nu min själ att komma vid detta, det är för mig en vämjelig spis.
\par 8 Ack att min bön bleve hörd, och att Gud ville uppfylla mitt hopp!
\par 9 O att det täcktes Gud att krossa mig, att räcka ut sin hand och avskära mitt liv!
\par 10 Då funnes ännu för mig någon tröst, jag kunde då jubla, fastän plågad utan förskoning; jag har ju ej förnekat den Heliges ord.
\par 11 Huru stor är då min kraft, eftersom jag alltjämt bör hoppas? Och vad väntar mig för ände, eftersom jag skall vara tålig?
\par 12 Min kraft är väl ej såsom stenens, min kropp är väl icke av koppar?
\par 13 Nej, förvisso gives ingen hjälp för mig, var utväg har blivit mig stängd.
\par 14 Den förtvivlade borde ju röna barmhärtighet av sin vän, men se, man övergiver den Allsmäktiges fruktan,
\par 15 Mina bröder äro trolösa, de äro såsom regnbäckar, ja, lika bäckarnas rännilar, som snart sina ut,
\par 16 som väl kunna gå mörka av vinterns flöden, när snön har fallit och gömt sig i dem,
\par 17 men som åter försvinna, när de träffas av hettan, och torka bort ifrån sin plats, då värmen kommer.
\par 18 Vägfarande där i trakten vika av till dem, men de finna allenast ödslighet och måste förgås.
\par 19 Temas vägfarande skådade dithän, Sabas köpmanståg hoppades på dem;
\par 20 men de kommo på skam i sin förtröstan, de sågo sig gäckade, när de hade hunnit ditfram.
\par 21 Ja, likaså ären I nu ingenting värda, handfallna stån I av förfäran och förskräckelse.
\par 22 Har jag då begärt att I skolen giva mig gåvor, taga av edert gods för att lösa mig ut,
\par 23 att I skolen rädda mig undan min ovän, köpa mig fri ur våldsverkares hand?
\par 24 Undervisen mig, så vill jag tiga, lären mig att förstå vari jag har farit vilse.
\par 25 Gott är förvisso uppriktigt tal, men tillrättavisning av eder, vad båtar den?
\par 26 Haven I då i sinnet att hålla räfst med ord, och skall den förtvivlade få tala för vinden?
\par 27 Då kasten I väl också lott om den faderlöse, då lären I väl köpslå om eder vän!
\par 28 Dock, må det nu täckas eder att akta på mig; icke vill jag ljuga eder mitt i ansiktet.
\par 29 Vänden om! Må sådan orätt icke ske; ja, vänden ännu om, ty min sak är rättfärdig!
\par 30 Skulle väl orätt bo på min tunga, och min mun, skulle den ej förstå vad fördärvligt är?

\chapter{7}

\par 1 En stridsmans liv lever ju människan på jorden, och hennes dagar äro såsom dagakarlens dagar.
\par 2 Hon är lik en träl som flämtar efter skugga, lik en dagakarl som får bida efter sin lön.
\par 3 Så har jag fått till arvedel månader av elände; nätter av vedermöda hava blivit min lott.
\par 4 Så snart jag har lagt mig, är min fråga: "När skall jag då få stå upp?" Ty aftonen synes mig så lång; jag är övermätt av oro, innan morgonen har kommit.
\par 5 Med förruttnelsens maskar höljes min kropp, med en skorpa lik jord; min hud skrymper samman och faller sönder.
\par 6 Mina dagar fly snabbare än vävarens spole; de försvinna utan något hopp.
\par 7 Tänk därpå att mitt liv är en fläkt, att mitt öga icke mer skall få se någon lycka.
\par 8 Den nu ser mig, hans öga skall ej vidare skåda mig; bäst din blick vilar på mig, är jag icke mer.
\par 9 Såsom ett moln som har försvunnit och gått bort, så är den som har farit ned i dödsriket; han kommer ej åter upp därifrån.
\par 10 Aldrig mer vänder han tillbaka till sitt hus, och hans plats vet icke av honom mer.
\par 11 Därför vill jag nu icke lägga band på min mun, jag vill taga till orda i min andes ångest, jag vill klaga i min själs bedrövelse.
\par 12 Icke är jag väl ett hav eller ett havsvidunder, så att du måste sätta ut vakt mot mig?
\par 13 När jag hoppas att min bädd skall trösta mig, att mitt läger skall lindra mitt bekymmer,
\par 14 då förfärar du mig genom drömmar, och med syner förskräcker du mig.
\par 15 Nej, hellre vill jag nu bliva kvävd, hellre dö än vara blott knotor!
\par 16 Jag är led vid detta; aldrig kommer jag åter till liv. Låt mig vara; mina dagar äro ju fåfänglighet.
\par 17 Vad är då en människa, att du gör så stor sak av henne, aktar på henne så noga,
\par 18 synar henne var morgon, prövar henne vart ögonblick?
\par 19 Huru länge skall det dröja, innan du vänder din blick ifrån mig, lämnar mig i fred ett litet andetag?
\par 20 Om jag än har syndar, vad skadar jag därmed dig, du människornas bespejare? Varför har du satt mig till ett mål för dina angrepp och låtit mig bliva en börda för mig själv?
\par 21 Varför vill du icke förlåta mig min överträdelse, icke tillgiva mig min missgärning? Nu måste jag ju snart gå till vila i stoftet; om du söker efter mig, så är jag icke mer.

\chapter{8}

\par 1 Därefter tog Bildad från Sua till orda och sade:
\par 2 Huru länge vill du hålla på med sådant tal och låta din muns ord komma såsom en väldig storm?
\par 3 Skulle väl Gud kunna kränka rätten? Kan den Allsmäktige kränka rättfärdigheten?
\par 4 Om dina barn hava syndat mot honom och han gav dem i sina överträdelsers våld,
\par 5 så vet, att om du själv söker Gud och beder till den Allsmäktige om misskund,
\par 6 då, om du är ren och rättsinnig, ja, då skall han vakna upp till din räddning och upprätta din boning, så att du bor där i rättfärdighet;
\par 7 och så skall din första tid synas ringa, då nu din sista tid har blivit så stor.
\par 8 Ty fråga framfarna släkten, och akta på vad fäderna hava utrönt
\par 9 - vi själva äro ju från i går och veta intet, en skugga äro våra dagar på jorden;
\par 10 men de skola undervisa dig och säga dig det, ur sina hjärtan skola de hämta fram svar:
\par 11 "Icke kan röret växa högt, där marken ej är sank, eller vassen skjuta i höjden, där vatten ej finnes?
\par 12 Nej, bäst den står grön, ej mogen för skörd, måste den då vissna, före allt annat gräs.
\par 13 Så går det alla som förgäta Gud; den gudlöses hopp måste varda om intet.
\par 14 Ty hans tillförsikt visar sig bräcklig och hans förtröstan lik spindelns väv.
\par 15 Han förlitar sig på sitt hus, men det har intet bestånd; han tryggar sig därvid, men det äger ingen fasthet.
\par 16 Lik en frodig planta växer han i solens sken, ut över lustgården sträcka sig hans skott;
\par 17 kring stenröset slingra sig hans rötter, mellan stenarna bryter han sig fram.
\par 18 Men när så Gud rycker bort honom från hans plats, då förnekar den honom: 'Aldrig har jag sett dig.'
\par 19 Ja, så går det med hans levnads fröjd, och ur mullen få andra växa upp."
\par 20 Se, Gud föraktar icke den som är ostrafflig, han håller ej heller de onda vid handen.
\par 21 Så bida då, till dess han fyller din mun med löje och dina läppar med jubel.
\par 22 De som hata dig varda då höljda med skam, och de ogudaktigas hyddor skola ej mer vara till.

\chapter{9}

\par 1 Därefter tog Job till orda och sade:
\par 2 Ja, förvisso vet jag att så är; huru skulle en människa kunna hava rätt mot Gud?
\par 3 Vill han gå till rätta med henne, så kan hon ej svara honom på en sak bland tusen.
\par 4 Han som är så vis i förstånd och så väldig i kraft, vem kan trotsa honom och dock slippa undan;
\par 5 honom som oförtänkt flyttar bort berg och omstörtar dem i sin vrede;
\par 6 honom som kommer jorden att vackla från sin plats, och dess pelare bäva därvid;
\par 7 honom som befaller solen, så går hon icke upp, och som sätter stjärnorna under försegling;
\par 8 honom som helt allena spänner ut himmelen och skrider fram över havets toppar;
\par 9 honom som har gjort Karlavagnen och Orion, Sjustjärnorna och söderns Stjärngemak;
\par 10 honom som gör stora och outrannsakliga ting och under, flera än någon kan räkna?
\par 11 Se, han far förbi mig, innan jag hinner att se det, han drager framom mig, förrän jag bliver honom varse.
\par 12 Se, han griper sitt rov; vem kan hindra honom? Vem kan säga till honom: "Vad gör du?"
\par 13 Gud, han ryggar icke sin vrede; för honom har Rahabs följe måst böja sig;
\par 14 huru skulle jag då våga svara honom, välja ut ord till att tala med honom?
\par 15 Nej, om jag än hade rätt, tordes jag dock ej svara; jag finge anropa min motpart om misskund.
\par 16 Och om han än svarade mig på mitt rop, så kunde jag ej tro att han lyssnade till min röst.
\par 17 Ty med storm hemsöker han mig och slår mig med sår på sår, utan sak.
\par 18 Han unnar mig icke att hämta andan; nej, med bedrövelser mättar han mig.
\par 19 Gäller det försteg i kraft: "Välan, jag är redo!", gäller det rätt: "Vem ställer mig till ansvar?"
\par 20 Ja, hade jag än rätt, så dömde min mun mig skyldig; vore jag än ostrafflig, så läte han mig synas vrång.
\par 21 Men ostrafflig är jag! Jag aktar ej mitt liv, jag frågar icke efter, om jag får leva.
\par 22 Det må gå som det vill, nu vare det sagt: han förgör den ostrafflige jämte den ogudaktige.
\par 23 Om en landsplåga kommer med plötslig död, så bespottar han de oskyldigas förtvivlan.
\par 24 Jorden är given i de ogudaktigas hand, och täckelse sätter han för dess domares ögon. Är det ej han som gör det, vem är det då?
\par 25 Min dagar hasta undan snabbare än någon löpare, de fly bort utan att hava sett någon lycka;
\par 26 de ila åstad såsom en farkost av rör, såsom en örn, när han störtar sig ned på sitt byte.
\par 27 Om jag än besluter att förgäta mitt bekymmer, att låta min sorgsenhet fara och göra mig glad,
\par 28 Så måste jag dock bäva för alla mina kval; jag vet ju att du icke skall döma mig fri.
\par 29 Nej, såsom skyldig måste jag stå där; varför skulle jag då göra mig fåfäng möda?
\par 30 Om jag än tvår mig i snö och renar mina händer i lutsalt,
\par 31 så skall du dock sänka mig ned i pölen, så att mina kläder måste vämjas vid mig.
\par 32 Ty han är ej min like, så att jag vågar svara honom, ej en sådan, att vi kunna gå till doms med varandra;
\par 33 ingen skiljeman finnes mellan oss, ingen som har myndighet över oss båda.
\par 34 Må han blott vända av från mig sitt ris, och må fruktan för honom ej förskräcka mig;
\par 35 då skall jag tala utan att rädas för honom, ty jag vet med min själv att jag icke är en sådan.

\chapter{10}

\par 1 Min själ är led vid livet. Jag vill giva fritt lopp åt min klagan, jag vill tala i min själs bedrövelse.
\par 2 Jag vill säga till Gud: Döm mig icke skyldig; låt mig veta varför du söker sak mot mig.
\par 3 Anstår det dig att öva våld, att förkasta dina händers verk, medan du låter ditt ljus lysa över de ogudaktigas rådslag?
\par 4 Har du då ögon som en varelse av kött, eller ser du såsom människor se?
\par 5 Är din ålder som en människas ålder, eller äro dina år såsom en mans tider,
\par 6 eftersom du letar efter missgärning hos mig och söker att hos mig finna synd,
\par 7 du som dock vet att jag icke är skyldig, och att ingen finnes, som kan rädda ur din hand?
\par 8 Dina händer hava danat och gjort mig, helt och i allo; och nu fördärvar du mig!
\par 9 Tänk på huru du formade mig såsom lera; och nu låter du mig åter varda till stoft!
\par 10 Ja, du utgöt mig såsom mjölk, och såsom ostämne lät du mig stelna.
\par 11 Med hud och kött beklädde du mig, av ben och senor vävde du mig samman.
\par 12 Liv och nåd beskärde du mig, och genom din vård bevarades min ande.
\par 13 Men därvid gömde du i ditt hjärta den tanken, jag vet att du hade detta i sinnet:
\par 14 om jag syndade, skulle du vakta på mig och icke lämna min missgärning ostraffad.
\par 15 Ve mig, om jag befunnes vara skyldig! Men vore jag än oskyldig, så finge jag ej lyfta mitt huvud, jag skulle mättas av skam och skåda min ofärd.
\par 16 Höjde jag det likväl, då skulle du såsom ett lejon jaga mig och alltjämt bevisa din undermakt på mig.
\par 17 Nya vittnen mot mig skulle du då föra fram och alltmer låta mig känna din förtörnelse; med skaror efter skaror skulle du ansätta mig.
\par 18 Varför lät du mig då komma ut ur modersskötet? Jag borde hava förgåtts, innan något öga såg mig,
\par 19 hava blivit såsom hade jag aldrig varit till; från moderlivet skulle jag hava förts till graven.
\par 20 Kort är ju min tid; må han då låta mig vara, lämna mig i fred, så att jag får en flyktig glädje,
\par 21 innan jag går hädan, för att aldrig komma åter, bort till mörkrets och dödsskuggans land,
\par 22 till det land vars dunkel är såsom djupa vatten, dit där dödsskugga och förvirring råder, ja, där dagsljuset självt är såsom djupa vatten.

\chapter{11}

\par 1 Därefter tog Sofar från Naama till orda och sade:
\par 2 Skall sådant ordflöde bliva utan svar och en så stortalig man få rätt?
\par 3 Skall ditt lösa tal nödga män till tystnad, så att du får bespotta, utan att någon kommer dig att blygas?
\par 4 Och skall du så få säga: "Vad jag lär är rätt, och utan fläck har jag varit inför dina ögon"?
\par 5 Nej, om allenast Gud ville tala och upplåta sina läppar till att svara dig,
\par 6 om han ville uppenbara dig sin visdoms lönnligheter, huru han äger förstånd, ja, i dubbelt mått, då insåge du att Gud, dig till förmån, har lämnat åt glömskan en del av din missgärning.
\par 7 Men kan väl du utrannsaka Guds djuphet eller fatta den Allsmäktiges fullkomlighet?
\par 8 Hög såsom himmelen är den - vad kan du göra? djupare än dödsriket - vad kan du förstå?
\par 9 Dess längd sträcker sig vidare än jorden, och i bredd överträffar den havet.
\par 10 När han vill fara fram och spärra någon inne eller kalla någon till doms, vem kan då hindra honom?
\par 11 Han är ju den som känner lögnens män, fördärv upptäcker han, utan att leta därefter.
\par 12 Men lika lätt kan en dåraktig man få förstånd, som en vildåsnefåle kan födas till människa.
\par 13 Om du nu rätt bereder ditt hjärta och uträcker dina händer till honom,
\par 14 om du skaffar bort det fördärv som kan låda vid din hand och ej låter orättfärdighet bo i dina hyddor,
\par 15 ja, då får du upplyfta ditt ansikte utan skam, du står fast och har intet att frukta.
\par 16 Ja, då skall du förgäta din olycka, blott minnas den såsom vatten som har förrunnit.
\par 17 Ditt liv skall då stråla klarare än middagens sken; och kommer mörker på, så är det som en gryning till morgon.
\par 18 Du kan då vara trygg, ty du äger ett hopp; du spanar omkring dig och går sedan trygg till vila.
\par 19 Ja, du får då ligga i ro, utan att någon förskräcker dig, och många skola söka din ynnest.
\par 20 Men de ogudaktigas ögon skola försmäkta; ingen tillflykt skall mer finnas för dem, och deras hopp skall vara att få giva upp andan.

\chapter{12}

\par 1 Därefter tog Job till orda och sade:
\par 2 Ja, visst ären I det rätta folket, och med eder kommer visheten att dö ut!
\par 3 Dock, jämväl jag har förstånd så gott som I, icke står jag tillbaka för eder; ty vem är den som ej begriper slikt?
\par 4 Så måste jag då vara ett åtlöje för min vän, jag som fick svar, så snart jag ropade till Gud; man ler åt en som är rättfärdig och ostrafflig!
\par 5 Ja, med förakt ses olyckan av den som står säker; förakt väntar dem vilkas fötter vackla.
\par 6 Men förhärjares hyddor åtnjuta frid, och trygghet få sådana som trotsa Gud, de som hava sin gud i sin hand.
\par 7 Men fråga du boskapen, den må undervisa dig, och fåglarna under himmelen, de må upplysa dig;
\par 8 eller tala till jorden, hon må undervisa dig, fiskarna i havet må giva dig besked.
\par 9 Vem kan icke lära genom allt detta att det är HERRENS hand som har gjort det?
\par 10 I hans han är ju allt levandes själ och alla mänskliga varelsers anda.
\par 11 Skall icke öra pröva orden, likasom munnen prövar matens smak?
\par 12 Vishet tillkommer ju de gamle och förstånd dem som länge hava levat.
\par 13 Hos Honom finnes vishet och makt, hos honom råd och förstånd.
\par 14 Se, vad han river ned, det bygges ej upp; för den han spärrar inne kan ingen upplåta.
\par 15 Han håller vattnen tillbaka - se, se då bliver där torrt, han släpper dem lösa, då fördärva de landet.
\par 16 Hos honom är kraft och klokhet, den förvillade och förvillaren äro båda i hans hand.
\par 17 Rådsherrar utblottar han, han för dem i landsflykt, och domare gör han till dårar.
\par 18 Han upplöser konungars välde och sätter fångbälte om deras höfter.
\par 19 Präster utblottar han, han för dem i landsflykt, och de säkrast rotade kommer han på fall.
\par 20 Välbetrodda män berövar han målet och avhänder de äldste deras insikt.
\par 21 Han utgjuter förakt över furstar och lossar de starkes gördel.
\par 22 Han blottar djupen, så att de ej höljas av mörker, dödsskuggan drager han fram i ljuset.
\par 23 Han låter folkslag växa till - och förgör dem; han utvidgar deras gränser, men för dem sedan bort.
\par 24 Stamhövdingar i landet berövar han förståndet, han leder dem vilse i väglösa ödemarker.
\par 25 De famla i mörkret och hava intet ljus, han kommer dem att ragla såsom druckna.

\chapter{13}

\par 1 Ja, alltsammans har mitt öga sett, mitt öra har hört det och nogsamt givit akt.
\par 2 Vad I veten, det vet också jag; icke står jag tillbaka för eder.
\par 3 Men till den Allsmäktige vill jag nu tala, det lyster mig att gå till rätta med Gud.
\par 4 Dock, I ären män som spinna ihop lögn, allasammans hopsätten I fåfängligt tal.
\par 5 Om I ändå villen alldeles tiga! Det kunde tillräknas eder som vishet.
\par 6 Hören nu likväl mitt klagomål, och akten på mina läppars gensagor.
\par 7 Viljen I försvara Gud med orättfärdigt tal och honom till förmån bruka oärligt tal?
\par 8 Skolen I visa eder partiska för honom eller göra eder till sakförare för Gud?
\par 9 Icke kan sådant ändas väl, när han håller räfst med eder? Eller kunnen I gäckas med honom, såsom man kan gäckas med en människa?
\par 10 Nej, förvisso skall han straffa eder, om I visen en hemlig partiskhet.
\par 11 Sannerligen, hans majestät skall då förskräcka eder, och fruktan för honom skall falla över eder.
\par 12 Edra tänkespråk skola då bliva visdomsord av aska, edra försvarsverk varda såsom vallar av ler.
\par 13 Tigen nu för min, så skall jag tala, gånge så över mig vad det vara må.
\par 14 Ja, huru det än går, vill jag fatta mitt kött mellan tänderna och taga min själ i min hand.
\par 15 Må han dräpa mig, jag hoppas intet annat; min vandel vill jag ändå hålla fram inför honom.
\par 16 Redan detta skall lända mig till frälsning, ty ingen gudlös dristar komma inför honom.
\par 17 Hören, hören då mina ord, och låten min förklaring tränga in i edra öron.
\par 18 Se, här lägger jag saken fram; jag vet att jag skall befinnas hava rätt.
\par 19 Eller gives det någon som kan vederlägga mig? Ja, då vill jag tiga - och dö.
\par 20 Allenast två ting må du ej göra mot mig, så behöver jag ej dölja mig inför ditt ansikte:
\par 21 din hand må du ej låta komma mig när, och fruktan för dig må icke förskräcka mig.
\par 22 Sedan må du åklaga, och jag vill svara, eller ock skall jag tala, och du må gendriva mig.
\par 23 Huru är det alltså med mina missgärningar och synder? Låt mig få veta min överträdelse och synd.
\par 24 Varför döljer du ditt ansikte och aktar mig såsom din fiende?
\par 25 Vill du skrämma ett löv som drives av vinden, vill du förfölja ett borttorkat strå?
\par 26 Du skriver ju bedrövelser på min lott och giver mig till arvedel min ungdoms missgärningar;
\par 27 du sätter mina fötter i stocken, du vaktar på alla vägar, för mina fotsulor märker du ut stegen.
\par 28 Och detta mot en som täres bort lik murket trä, en som liknar en klädnad sönderfrätt av mal!

\chapter{14}

\par 1 Människan, av kvinna född, lever en liten tid och mättas av oro;
\par 2 lik ett blomster växer hon upp och vissnar bort, hon flyr undan såsom skuggan och har intet bestånd.
\par 3 Och till att vakta på en sådan upplåter du dina ögon, ja, du drager mig till doms inför dig.
\par 4 Som om en ren skulle kunna framgå av en oren! Sådant kan ju aldrig ske.
\par 5 Äro nu människans dagar oryggligt bestämda, hennes månaders antal fastställt av dig, har du utstakat en gräns som hon ej kan överskrida,
\par 6 vänd då din blick ifrån henne och unna henne ro, låt henne njuta en dagakarls glädje av sin dag.
\par 7 För ett träd finnes ju kvar något hopp; hugges det än ned, kan det åter skjuta skott, och telningar behöva ej fattas därpå.
\par 8 Om än dess rot tynar hän i jorden och dess stubbe dör bort i mullen,
\par 9 så kan det grönska upp genom vattnets ångor och skjuta grenar lik ett nyplantat träd.
\par 10 Men om en man dör, så ligger han där slagen; om en människa har givit upp andan, var finnes hon då mer?
\par 11 Såsom när vattnet har förrunnit ur en sjö, och såsom när en flod har sinat bort och uttorkat,
\par 12 så ligger mannen där och står ej mer upp, han vaknar icke åter, så länge himmelen varar; aldrig väckes han upp ur sin sömn.
\par 13 Ack, att du ville gömma mig i dödsriket, fördölja mig, till dess din vrede hade upphört, staka ut för mig en tidsgräns och sedan tänka på mig -
\par 14 fastän ju ingen kan få liv, när han en gång är död! Då skulle jag hålla min stridstid ut, ända till dess att min avlösning komme.
\par 15 Du skulle då ropa på mig, och jag skulle svara dig; efter dina händers verk skulle du längta;
\par 16 ja, du skulle då räkna mina steg, du skulle ej akta på min synd.
\par 17 I en förseglad pung låge då min överträdelse, och du överskylde min missgärning.
\par 18 Men såsom själva berget faller och förvittrar, och såsom klippan flyttas ifrån sin plats,
\par 19 såsom stenar nötas sönder genom vattnet, och såsom mullen sköljes bort av dess flöden, så gör du ock människans hopp om intet.
\par 20 Du slår henne ned för alltid, och hon far hädan; du förvandlar hennes ansikte och driver henne bort.
\par 21 Om hennes barn komma till ära, så känner hon det icke; om de sjunka ned till ringhet, så aktar hon dock ej på dem.
\par 22 Hennes kropp känner blott sin egen plåga, hennes själ blott den sorg hon själv får förnimma.

\chapter{15}

\par 1 Därefter tog Elifas från Teman till orda och sade:
\par 2 Skall en vis man tala så i vädret och fylla upp sitt bröst med östanvind?
\par 3 Skall han försvara sin sak med haltlöst tal, med ord som ingenting bevisa?
\par 4 Än mer, du gör gudsfruktan om intet och kommer med klagolåt inför Gud.
\par 5 Ty din ondska lägger dig orden i munnen, och ditt behag står till illfundigt tal.
\par 6 Så dömes du nu skyldig av din mun, ej av mig, dina egna läppar vittna emot dig.
\par 7 Var du den första människa som föddes, och fick du liv, förrän höjderna funnos?
\par 8 Blev du åhörare i Guds hemliga råd och fick så visheten i ditt våld?
\par 9 Vad vet du då, som vi icke veta? Vad förstår du, som ej är oss kunnigt?
\par 10 Gråhårsman och åldring finnes också bland oss, ja, en som övergår din fader i ålder.
\par 11 Försmår du den tröst som Gud har att bjuda, och det ord som i saktmod talas med dig?
\par 12 Vart föres du hän av ditt sinne, och varför välva dina ögon så,
\par 13 i det du vänder ditt raseri mot Gud och öser ut ord ur din mun?
\par 14 Vad är en människa, att hon skulle vara ren? Vad en av kvinna född, att han skulle vara rättfärdig?
\par 15 Se, ej ens på sina heliga kan han förlita sig, och himlarna äro icke rena inför hans ögon;
\par 16 huru mycket mindre då den som är ond och fördärvad, den man som läskar sig med orättfärdighet såsom med vatten!
\par 17 Jag vill kungöra dig något, så hör nu mig; det som jag har skådat vill jag förtälja,
\par 18 vad visa män hava gjort kunnigt, lagt fram såsom ett arv ifrån sina fäder,
\par 19 ifrån dem som allena fingo landet till gåva, och bland vilka ingen främling ännu hade trängt in:
\par 20 Den ogudaktige har ångest i alla sina dagar, under de år, helt få, som beskäras en våldsverkare.
\par 21 Skräckröster ljuda i hans öron; när han är som tryggast, kommer förhärjaren över honom.
\par 22 Han har intet hopp om räddning ur mörkret, ty svärdet lurar på honom.
\par 23 Såsom flykting söker han sitt bröd: var är det? Han förnimmer att mörkrets dag är för handen.
\par 24 Ångest och trångmål förskräcka honom, han nedslås av dem såsom av en stridsrustad konung.
\par 25 Ty mot Gud räckte han ut sin hand, och mot den Allsmäktige förhävde han sig;
\par 26 han stormade mot honom med trotsig hals, med sina sköldars ryggar i sluten hop;
\par 27 han höljde sitt ansikte med fetma och samlade hull på sin länd;
\par 28 han bosatte sig i städer, dömda till förstöring, i hus som ej fingo bebos, ty till stenhopar voro de bestämda.
\par 29 Därför bliver han ej rik, och hans gods består ej, hans skördar luta ej tunga mot jorden.
\par 30 Han kan icke undslippa mörkret; hans telningar skola förtorka av hetta, och själv skall han förgås genom Guds muns anda.
\par 31 I sin förvillelse må han ej lita på vad fåfängligt är, ty fåfänglighet måste bliva hans lön.
\par 32 I förtid skall hans mått varda fyllt, och hans krona skall ej grönska mer.
\par 33 Han bliver lik ett vinträd som i förtid mister sina druvor, lik ett olivträd som fäller sina blommor.
\par 34 Ty den gudlöses hus förbliver ofruktsamt, såsom eld förtär hyddor där mutor tagas.
\par 35 Man går havande med olycka och föder fördärv; den livsfrukt man alstrar är ett sviket hopp.

\chapter{16}

\par 1 Därefter tog Job till orda och sade:
\par 2 Över nog har jag fått höra av sådant; usla tröstare ären I alla.
\par 3 Är det nu slut på detta tal i vädret, eller eggar dig ännu något till gensvar?
\par 4 Jag kunde väl ock tala, jag såsom I; ja, jag ville att I voren i mitt ställe! Då kunde jag hopsätta ord mot eder och skaka mot eder mitt huvud till hån.
\par 5 Med munnen kunde jag då styrka eder och med läpparnas ömkan bereda eder lindring.
\par 6 Om jag nu talar, så lindras därav ej min plåga; och tiger jag, icke släpper den mig ändå.
\par 7 Nej, nu har all min kraft blivit tömd; du har ju förött hela mitt hus.
\par 8 Och att du har hemsökt mig, det gäller såsom vittnesbörd; min sjukdom får träda upp och tala mot mig.
\par 9 I vrede söndersliter och ansätter man mig, man biter sina tänder samman emot mig; ja, min ovän vässer mot mig sina blickar.
\par 10 Man spärrar upp munnen mot mig, smädligt slår man mig på mina kinder; alla rota sig tillsammans emot mig.
\par 11 Gud giver mig till pris åt orättfärdiga människor och kastar mig i de ogudaktigas händer.
\par 12 Jag satt i god ro, då krossade han mig; han grep mig i nacken och slog mig i smulor. Han satte mig upp till ett mål för sina skott;
\par 13 från alla sidor träffa mig hans pilar, han genomborrar mina njurar utan förskoning, min galla gjuter han ut på jorden.
\par 14 Han bryter ned mig med stöt på stöt, han stormar emot mig såsom en kämpe.
\par 15 Säcktyg bär jag hopfäst över min hud, och i stoftet har jag måst sänka mitt horn,
\par 16 Mitt anlete är glödande rött av gråt, och på mina ögonlock är dödsskugga lägrad.
\par 17 Och detta, fastän våld ej finnes i mina händer, och fastän min bön är ren!
\par 18 Du jord, överskyl icke mitt blod, och låt för mitt rop ingen vilostad finnas.
\par 19 Se, redan nu har jag i himmelen mitt vittne, och i höjden den som skall tala för mig.
\par 20 Mina vänner hava mig nu till sitt åtlöje, därför skådar mitt öga med tårar till Gud,
\par 21 Ja, må han här skaffa rätt åt en man mot Gud och åt ett människobarn mot dess nästa.
\par 22 Ty få äro de år som skola upprinna, innan jag vandrar den väg där jag ej mer kommer åter.

\chapter{17}

\par 1 Min livskraft är förstörd, mina dagar slockna ut, bland gravar får jag min lott.
\par 2 Ja, i sanning är jag omgiven av gäckeri, och avoghet får mitt öga ständigt skåda hos dessa!
\par 3 Så ställ nu säkerhet och borgen för mig hos dig själv; vilken annan vill giva mig sitt handslag?
\par 4 Dessas hjärtan har du ju tillslutit för förstånd, därför skall du icke låta dem triumfera.
\par 5 Den som förråder sina vänner till plundring, på hans barn skola ögonen försmäkta.
\par 6 Jag är satt till ett ordspråk bland folken; en man som man spottar i ansiktet är jag.
\par 7 Därför är mitt öga skumt av grämelse, och mina lemmar äro såsom en skugga allasammans.
\par 8 De redliga häpna över sådant, och den oskyldige uppröres av harm mot den gudlöse.
\par 9 Men den rättfärdige håller fast vid sin väg, och den som har rena händer bemannar sig dess mer.
\par 10 Ja, gärna mån I alla ansätta mig på nytt, jag lär ändå bland eder ej finna någon vis.
\par 11 Mina dagar äro förlidna, sönderslitna äro mina planer, vad som var mitt hjärtas begär.
\par 12 Men natten vill man göra till dag, ljuset skulle vara nära, nu då mörker bryter in.
\par 13 Nej, huru jag än bidar, bliver dödsriket min boning, i mörkret skall jag bädda mitt läger;
\par 14 till graven måste jag säga: "Du är min fader", till förruttnelsens maskar: "Min moder", "Min syster".
\par 15 Vad bliver då av mitt hopp, ja, mitt hopp, vem får skåda det?
\par 16 Till dödsrikets bommar far det ned, då jag nu själv går till vila i stoftet.

\chapter{18}

\par 1 Därefter tog Bildad från Sua till orda och sade:
\par 2 Huru länge skolen I gå på jakt efter ord? Kommen till förstånd; sedan må vi talas vid.
\par 3 Varför skola vi aktas såsom oskäliga djur, räknas i edra ögon såsom ett förstockat folk?
\par 4 Du som i din vrede sliter sönder dig själv, menar du att dör din skull jorden skall bliva öde och klippan flyttas bort från sin plats?
\par 5 Nej, den ogudaktiges ljus skall slockna ut, och lågan av hans eld icke giva något sken.
\par 6 Ljuset skall förmörkas i hans hydda, och lampan slockna ut för honom.
\par 7 Hans väldiga steg skola stäckas, hans egna rådslag bringa honom på fall.
\par 8 Ty han rusar med sina fötter in i nätet, försåten lura, där han vandrar fram;
\par 9 snaran griper honom om hälen, och gillret tager honom fatt;
\par 10 garn till att fånga honom äro lagda på marken och snärjande band på hans stig.
\par 11 Från alla sidor ängsla honom förskräckelser, de jaga honom, varhelst han går fram.
\par 12 Olyckan vill uppsluka honom, och ofärd står redo, honom till fall.
\par 13 Under hans hud frätas hans lemmar bort, ja, av dödens förstfödde bortfrätas hans lemmar.
\par 14 Ur sin hydda, som han förtröstar på, ryckes han bort, och till förskräckelsernas konung vandrar han hän.
\par 15 I hans hydda får främlingar bo, och svavel utströs över hans boning.
\par 16 Nedantill förtorkas hans rötter, och ovantill vissnar hans krona bort.
\par 17 Hans åminnelse förgås ifrån jorden, hans namn lever icke kvar i världen.
\par 18 Från ljus stötes han ned i mörker och förjagas ifrån jordens krets.
\par 19 Utan barn och avkomma bliver han i sitt folk, och ingen i hans boningar skall slippa undan.
\par 20 Över hans ofärdsdag häpna västerns folk, och österns män gripas av rysning.
\par 21 Ja, så sker det med den orättfärdiges hem, så går det dens hus, som ej vill veta av Gud.

\chapter{19}

\par 1 Därefter tog Job till orda och sade:
\par 2 Huru länge skolen I bedröva min själ och krossa mig sönder med edra ord?
\par 3 Tio gånger haven I nu talat smädligt mot mig och kränkt mig utan all försyn.
\par 4 Om så är, att jag verkligen har farit vilse, då är förvillelsen min egen sak.
\par 5 Men viljen I ändå verkligen förhäva eder mot mig, och påstån I att smäleken har drabbat mig med skäl,
\par 6 så veten fastmer att Gud har gjort mig orätt och att han har omsnärjt mig med sitt nät.
\par 7 Se, jag klagar över våld, men får intet svar; jag ropar, men får icke rätt.
\par 8 Min väg har han spärrat, så att jag ej kommer fram, och över mina stigar breder han mörker.
\par 9 Min ära har han avklätt mig, och från mitt huvud har han tagit bort kronan.
\par 10 Från alla sidor bryter han ned mig, så att jag förgås; han rycker upp mitt hopp, såsom vore det ett träd.
\par 11 Sin vrede låter han brinna mot mig och aktar mig såsom sina ovänners like.
\par 12 Hans skaror draga samlade fram och bereda sig väg till anfall mot mig; de lägra sig runt omkring min hydda.
\par 13 Långt bort ifrån mig har han drivit mina fränder; mina bekanta äro idel främlingar mot mig.
\par 14 Mina närmaste hava dragit sig undan, och mina förtrogna hava förgätit mig.
\par 15 Mitt husfolk och mina tjänstekvinnor akta mig såsom främling; en främmande man har jag blivit i deras ögon.
\par 16 Kallar jag på min tjänare, så svarar han icke; ödmjukt måste jag bönfalla hos honom.
\par 17 Min andedräkt är vidrig för min hustru, jag väcker leda hos min moders barn.
\par 18 Till och med de små barnen visa mig förakt; så snart jag står upp, tala de ohöviskt emot mig.
\par 19 Ja, en styggelse är jag för alla dem jag umgicks med; de som voro mig kärast hava vänt sig emot mig.
\par 20 Benen i min kropp tränga ut i hud och hull; knappt tandköttet har jag fått behålla kvar.
\par 21 Haven misskund, haven misskund med mig, I mina vänner, då nu Guds hand så har hemsökt mig.
\par 22 Varför skolen I förfölja mig, I såsom Gud, och aldrig bliva mätta av mitt kött?
\par 23 Ack att mina ord skreves upp, ack att de bleve upptecknade i en bok,
\par 24 ja, bleve med ett stift av järn och med bly för evig tid inpräglade i klippan!
\par 25 Dock, jag vet att min förlossare lever, och att han till slut skall stå fram över stoftet.
\par 26 Och sedan denna min sargade hud är borta, skall jag fri ifrån mitt kött få skåda Gud.
\par 27 Ja, honom skall jag få skåda, mig till hjälp, för mina ögon skall jag se honom, ej såsom en främling; därefter trånar jag i mitt innersta.
\par 28 Men när I tänken: "huru skola vi icke ansätta honom!" - såsom vore skulden att finna hos mig -
\par 29 då mån I taga eder till vara för svärdet, ty vreden hör till de synder som straffas med svärd; så mån I då besinna att en dom skall komma.

\chapter{20}

\par 1 Därefter tog Sofar från Naama till orda och sade:
\par 2 På sådant tal giva mina tankar mig ett svar, än mer, då jag nu är så upprörd i mitt inre.
\par 3 Smädlig tillrättavisning måste jag höra, och man svarar mig med munväder på förståndigt tal.
\par 4 Vet du då icke att så har varit från evig tid, från den stund då människor sattes på jorden:
\par 5 att de ogudaktigas jubel varar helt kort och den gudlöses glädje ett ögonblick?
\par 6 Om än hans förhävelse stiger upp till himmelen och hans huvud når intill molnen,
\par 7 Så förgås han dock för alltid och aktas lik sin träck; de som sågo honom måste fråga: "Var är han?"
\par 8 Lik en dröm flyger han bort, och ingen finner honom mer; han förjagas såsom en syn om natten.
\par 9 Det öga som såg honom ser honom icke åter, och hans plats får ej skåda honom mer.
\par 10 Hans barn måste gottgöra hans skulder till de arma, hans händer återbära hans vinning.
\par 11 Bäst ungdomskraften fyller hans ben, skall den ligga i stoftet med honom.
\par 12 Om än ondskan smakar ljuvligt i hans mun, så att han gömmer den under sin tunga,
\par 13 är rädd om den och ej vill gå miste därom, utan håller den förvarad inom sin gom,
\par 14 så förvandlas denna kost i hans inre, bliver huggormsetter i hans liv.
\par 15 Den rikedom han har slukat måste han utspy; av Gud drives den ut ur hans buk.
\par 16 Ja, huggormsgift kommer han att dricka, av etterormens tunga bliver han dräpt.
\par 17 Ingen bäck får vederkvicka hans syn, ingen ström med flöden av honung och gräddmjölk.
\par 18 Sitt fördärv måste han återbära, han får ej njuta därav; hans fröjd svarar ej mot den rikedom han har vunnit.
\par 19 Ty mot de arma övade han våld och lät dem ligga där; han rev till sig hus som han ej kan hålla vid makt.
\par 20 Han visste ej av någon ro för sin buk, men han skall icke rädda sig med sina skatter.
\par 21 Intet slapp undan hans glupskhet, därför äger och hans lycka intet bestånd.
\par 22 Mitt i hans överflöd påkommer honom nöd, och envar eländig vänder då mot honom sin hand.
\par 23 Ja, så måste ske, för att hans buk må bliva fylld; sin vredes glöd skall Gud sända över honom och låta den tränga såsom ett regn in i hans kropp.
\par 24 Om han flyr undan för vapen av järn, så genomborras han av kopparbågens skott.
\par 25 När han då drager i pilen och den kommer ut ur hans rygg, när den ljungande udden kommer fram ur hans galla, då falla dödsfasorna över honom.
\par 26 Idel mörker är förvarat åt hans skatter; till mat gives honom eld som brinner utan pust, den förtär vad som är kvar i hans hydda.
\par 27 Himmelen lägger hans missgärning i dagen, och jorden reser sig upp emot honom.
\par 28 Vad som har samlats i hans hus far åter sin kos, likt förrinnande vatten, på vredens dag.
\par 29 Sådan lott får en ogudaktig människa av Gud, sådan arvedel har av Gud blivit bestämd åt henne.

\chapter{21}

\par 1 Därefter tog Job till orda och sade:
\par 2 Hören åtminstone på mina ord; låten det vara den tröst som I given mig.
\par 3 Haven fördrag med mig, så att jag får tala; sedan jag har talat, må du bespotta.
\par 4 Är då min klagan, såsom när människor eljest klaga? Eller huru skulle jag kunna vara annat än otålig?
\par 5 Akten på mig, så skolen I häpna och nödgas lägga handen på munnen.
\par 6 Ja, när jag tänker därpå, då förskräckes jag själv, och förfäran griper mitt kött.
\par 7 Varför få de ogudaktiga leva, ja, med åldern växa till i rikedom?
\par 8 De se sina barn leva kvar hos sig, och sin avkomma hava de inför sina ögon.
\par 9 Deras hus stå trygga, ej hemsökta av förskräckelse; Gud låter sitt ris icke komma vid dem.
\par 10 När deras boskap parar sig, är det icke förgäves; lätt kalva deras kor, och icke i otid.
\par 11 Sina barn släppa de ut såsom en hjord, deras piltar hoppa lustigt omkring.
\par 12 De stämma upp med pukor och harpor, och glädja sig vid pipors ljud.
\par 13 De förnöta sina dagar i lust, och ned till dödsriket fara de i frid.
\par 14 Och de sade dock till Gud: "Vik ifrån oss, dina vägar vilja vi icke veta av.
\par 15 Vad är den Allsmäktige, att vi skulle tjäna honom? och vad skulle det hjälpa oss att åkalla honom?"
\par 16 Det är sant, i deras egen hand står ej deras lycka, och de ogudaktigas rådslag vare fjärran ifrån mig!
\par 17 Men huru ofta utslocknar väl de ogudaktigas lampa, huru ofta händer det att ofärd kommer över dem, och att han tillskiftar dem lotter i vrede?
\par 18 De borde ju bliva såsom halm för vinden, lika agnar som stormen rycker bort.
\par 19 "Gud spar åt hans barn att lida för hans ondska." Ja, men honom själv borde han vedergälla, så att han finge känna det.
\par 20 Med egna ögon borde han se sitt fall, och av den Allsmäktiges vrede borde han få dricka.
\par 21 Ty vad frågar han efter sitt hus, när han själv är borta, när hans månaders antal har nått sin ände?
\par 22 "Skall man då lära Gud förstånd, honom som dömer över de högsta?"
\par 23 Ja, den ene får dö i sin välmaktstid, där han sitter i allsköns frid och ro;
\par 24 hans stävor hava fått stå fulla med mjölk, och märgen i hans ben har bevarat sin saft.
\par 25 Den andre måste dö med bedrövad själ, och aldrig fick han njuta av någon lycka.
\par 26 Tillsammans ligga de så i stoftet, och förruttnelsens maskar övertäcka dem.
\par 27 Se, jag känner väl edra tankar och de funder med vilka I viljen nedslå mig.
\par 28 I spörjen ju: "Vad har blivit av de höga herrarnas hus, av hyddorna när de ogudaktiga bodde?"
\par 29 Haven I då ej frågat dem som vida foro, och akten I ej på deras vittnesbörd:
\par 30 att den onde bliver sparad på ofärdens dag och bärgad undan på vredens dag?
\par 31 Vem vågar ens förehålla en sådan hans väg? Vem vedergäller honom, vad han än må göra?
\par 32 Och när han har blivit bortförd till graven, så vakar man sedan där vid kullen.
\par 33 Ljuvligt får han vilja under dalens torvor. I hans spår drager hela världen fram; före honom har och otaliga gått.
\par 34 Huru kunnen I då bjuda mig så fåfänglig tröst? Av edra svar står allenast trolösheten kvar.

\chapter{22}

\par 1 Därefter tog Elifas från Teman till orda och sade:
\par 2 Kan en man bereda Gud något gagn, så att det länder honom till gagn, om någon är förståndig?
\par 3 Har den Allsmäktige någon båtnad av att du är rättfärdig, eller någon vinning av att du vandrar ostraffligt?
\par 4 Är det för din gudsfruktans skull som han straffar dig, och som han går med dig till doms?
\par 5 Har då icke din ondska varit stor, och voro ej dina missgärningar utan ände?
\par 6 Jo, du tog pant av din broder utan sak, du plundrade de utblottade på deras kläder.
\par 7 Åt den försmäktande gav du intet vatten att dricka, och den hungrige nekade du bröd.
\par 8 För den väldige ville du upplåta landet, och den myndige skulle få bo däri,
\par 9 men änkor lät du gå med tomma händer, och de faderlösas armar blevo krossade.
\par 10 Därför omgives du nu av snaror och förfäras av plötslig skräck.
\par 11 ja, av ett mörker där du intet ser, och av vattenflöden som övertäcka dig.
\par 12 I himmelens höjde är det ju Gud som har sin boning, och du ser stjärnorna däruppe, huru högt de sitta;
\par 13 därför tänker du: "Vad kan Gud veta? Skulle han kunna döma, han som bor bortom töcknet?
\par 14 Molnen äro ju ett täckelse, så att han intet ser; och på himlarunden är det han har sin gång."
\par 15 Vill du då hålla dig på forntidens väg, där fördärvets män gingo fram,
\par 16 de män som bortrycktes, innan deras tid var ute, och såsom en ström flöt deras grundval bort,
\par 17 de män som sade till Gud: "Vik ifrån oss", ty vad skulle den Allsmäktige kunna göra dem?
\par 18 Det var ju dock han som uppfyllde deras hus med sitt goda. De ogudaktigas rådslag vare fjärran ifrån mig!
\par 19 De rättfärdiga skola se det och glädja sig, och den oskyldige skall få bespotta dem:
\par 20 "Ja, nu äro förvisso våra motståndare utrotade, och deras överflöd har elden förtärt."
\par 21 Men sök nu förlikning och frid med honom; därigenom skall lycka falla dig till.
\par 22 Tag emot undervisning av hans mun, och förvara hans ord i ditt hjärta.
\par 23 Om du omvänder dig till den Allsmäktige, så bliver du upprättad; men orättfärdighet må du skaffa bort ur din hydda.
\par 24 Ja kasta din gyllene skatt i stoftet och Ofirs-guldet ibland bäckens stenar,
\par 25 så bliver den Allsmäktige din gyllene skatt, det ädlaste silver varder han för dig.
\par 26 Ja, då skall du hava din lust i den Allsmäktige och kunna upplyfta ditt ansikte till Gud.
\par 27 När du då beder till honom, skall han höra dig, och de löften du gör skall du få infria.
\par 28 Allt vad du besluter skall då lyckas för dig, och ljus skall skina på dina vägar.
\par 29 Om de leda mot djupet och du då beder: "Uppåt!", så frälsar han mannen som har ödmjukat sig.
\par 30 Ja han räddar och den som ej är fri ifrån skuld; genom dina händers renhet räddas en sådan.

\chapter{23}

\par 1 Därefter tog Job till orda och sade:
\par 2 Också i dag vill min klaga göra uppror. Min hand kännes matt för min suckans skull.
\par 3 Om jag blott visste huru jag skulle finna honom, huru jag kunde komma dit där han bor!
\par 4 Jag skulle då lägga fram för honom min sak och fylla min mun med bevis.
\par 5 Jag ville väl höra vad han kunde svara mig, och förnimma vad han skulle säga till mig.
\par 6 Icke med övermakt finge han bekämpa mig, nej, han borde allenast lyssna till mig.
\par 7 Då skulle hans motpart stå här såsom en redlig man, ja, då skulle jag för alltid komma undan min domare.
\par 8 Men går jag mot öster, så är han icke där; går jag mot väster, så varsnar jag honom ej;
\par 9 har han något att skaffa i norr, jag skådar honom icke; döljer han sig i söder, jag ser honom ej heller där.
\par 10 Han vet ju vilken väg jag har vandrat; han har prövat mig, och jag har befunnits lik guld.
\par 11 Vid hans spår har min for hållit fast, hans väg har jag följt, utan att vika av.
\par 12 Från hans läppars bud har jag icke gjort något avsteg; mer än egna rådslut har jag aktat hans muns tal.
\par 13 Men hans vilja är orygglig; vem kan hindra honom? Vad honom lyster, det gör han ock.
\par 14 Ja, han giver mig fullt upp min beskärda del, och mycket av samma slag har han ännu i förvar.
\par 15 Därför gripes jag av förskräckelse för hans ansikte; när jag betänker det, fruktar jag för honom.
\par 16 Det är Gud som har gjort mitt hjärta försagt, den Allsmäktige är det som har vållat min förskräckelse,
\par 17 ty jag fick icke förgås, innan mörkret kom, dödsnatten undanhöll han mig.

\chapter{24}

\par 1 Varför har den Allsmäktige inga räfstetider i förvar? varför få hans vänner ej skåda hans hämndedagar?
\par 2 Se, råmärken flyttar man undan, rövade hjordar driver man i bet;
\par 3 de faderlösas åsna för man bort och tager änkans ko i pant.
\par 4 Man tränger de fattiga undan från vägen, de betryckta i landet måste gömma sig med varandra.
\par 5 Ja, såsom vildåsnor måste de leva i öknen; dit gå de och möda sig och söka något till täring; hedmarken är det bröd de hava åt sina barn.
\par 6 På fältet få de till skörd vad boskap plägar äta, de hämta upp det sista i den ogudaktiges vingård.
\par 7 Nakna ligga de om natten, berövade sina kläder; de hava intet att skyla sig med i kölden.
\par 8 Av störtskurar från bergen genomdränkas de; de famna klippan, ty de äga ej annan tillflykt.
\par 9 Den faderlöse slites från sin moders bröst, och den betryckte drabbas av utpantning.
\par 10 Nakna måste de gå omkring, berövade sina kläder, hungrande nödgas de bära på kärvar.
\par 11 Inom sina förtryckares murar måste de bereda olja, de få trampa vinpressar och därvid lida törst.
\par 12 Utstötta ur människors samfund jämra de sig, ja, från dödsslagnas själar uppstiger ett rop. Men Gud aktar ej på vad förvänt som sker.
\par 13 Andra hava blivit fiender till ljuset; de känna icke dess vägar och hålla sig ej på dess stigar.
\par 14 Vid dagningen står mördaren upp för att dräpa den betryckte och fattige; och om natten gör han sig till tjuvars like.
\par 15 Äktenskapsbrytarens öga spejar efter skymningen, han tänker: "Intet öga får känna igen mig", och sätter så ett täckelse framför sitt ansikte.
\par 16 När det är mörkt, bryta sådana sig in i husen, men under dagen stänga de sig inne; ljuset vilja de icke veta av.
\par 17 Ty det svarta mörkret räknas av dem alla såsom morgon, med mörkrets förskräckelser äro de ju förtrogna.
\par 18 "Men hastigt", menen I, "ryckes en sådan bort av strömmen, förbannad bliver hans del i landet; till vingårdarna får han ej mer styra sina steg.
\par 19 Såsom snövatten förtäres av torka och hetta, så förtär dödsriket den som har syndat.
\par 20 Hans moders liv förgäter honom, maskar frossa på honom, ingen finnes, som bevarar hans minne; såsom ett träd brytes orättfärdigheten av.
\par 21 Så går det, när någon plundrar den ofruktsamma, som intet föder, och när någon icke gör gott mot änkan."
\par 22 Ja, men han uppehåller ock våldsmännen genom sin kraft, de få stå upp, när de redan hade förlorat hoppet om livet;
\par 23 han giver dem trygghet, så att de få vila, och hans ögon vaka över deras vägar.
\par 24 När de hava stigit till sin höjd, beskäres dem en snar hädanfärd, de sjunka då ned och dö som alla andra; likasom axens toppar vissna de bort.
\par 25 Är det ej så, vem vill då vederlägga mig, vem kan göra mina ord om intet?

\chapter{25}

\par 1 Därefter tog Bildad från Sua till orda och sade:
\par 2 Hos honom är väldighet och förskräckande makt, hos honom, som skapar frid i sina himlars höjd.
\par 3 Vem finnes, som förmår räkna hans skaror? Och vem överstrålas ej av hans ljus?
\par 4 Huru skulle då en människa kunna hava rätt mot Gud eller en av kvinna född kunna befinnas ren?
\par 5 Se, ej ens månen skiner nog klart, ej ens stjärnorna äro rena i hans ögon;
\par 6 huru mycket mindre då människan, det krypet, människobarnet, den masken!

\chapter{26}

\par 1 Därefter tog Job till orda och sade:
\par 2 Vilken hjälp har du ej skänkt den vanmäktige, huru har du ej stärkt den maktlöses arm!
\par 3 Vilka råd har du ej givit den ovise, och vilket överflöd av klokhet har du ej lagt i dagen!
\par 4 Vem gav dig kraft att tala sådana ord, och vems ande var det som kom till orda ur dig?
\par 5 Dödsrikets skuggor gripas av ångest, djupets vatten och de som bo däri.
\par 6 Dödsriket ligger blottat för honom, och avgrunden har intet täckelse.
\par 7 Han spänner ut nordanrymden över det tomma och hänger upp jorden på intet.
\par 8 Han samlar vatten i sina moln såsom i ett knyte, och skyarna brista icke under bördan.
\par 9 Han gömmer sin tron för vår åsyn, han omhöljer den med sina skyar.
\par 10 En rundel har han välvt såsom gräns för vattnen, där varest ljus ändas i mörker.
\par 11 Himmelens pelare skälva, de gripas av förfäran vid hans näpst.
\par 12 Med sin kraft förskräckte han havet, och genom sitt förstånd sönderkrossade han Rahab.
\par 13 Blott han andades, blev himmelen klar; hans hand genomborrade den snabba ormen.
\par 14 Se, detta är allenast utkanterna av hans verk; en sakta viskning är allt vad vi förnimma därom. Hans allmakts dunder, vem skulle kunna fatta det?

\chapter{27}

\par 1 Åter hov Job upp sin röst och kvad:
\par 2 Så sant Gud lever, han som har förhållit mig min rätt, den Allsmäktige, som har vållat min själs bedrövelse:
\par 3 aldrig, så länge ännu min ande är i mig och Guds livsfläkt är kvar i min näsa,
\par 4 aldrig skola mina läppar tala vad orättfärdigt är, och min tunga bära fram oärligt tal.
\par 5 Bort det, att jag skulle giva eder rätt! Intill min död låter jag min ostrafflighet ej tagas ifrån mig.
\par 6 Vid min rättfärdighet håller jag fast och släpper den icke, mitt hjärta förebrår mig ej för någon av mina dagar.
\par 7 Nej, såsom ogudaktig må min fiende stå där och min motståndare såsom orättfärdig.
\par 8 Ty vad hopp har den gudlöse när hans liv avskäres, när hans själ ryckes bort av Gud?
\par 9 Månne Gud skall höra hans rop, när nöden kommer över honom?
\par 10 Eller kan en sådan hava sin lust i den Allsmäktige, kan han åkalla Gud alltid?
\par 11 Jag vill undervisa eder om huru Gud går till väga; huru den Allsmäktige tänker, vill jag icke fördölja.
\par 12 Dock, I haven ju själva allasammans skådat det; huru kunnen I då hängiva eder åt så fåfängliga tankar?
\par 13 Hören vad den ogudaktiges lott bliver hos Gud, vilken arvedel våldsverkaren får av den Allsmäktige:
\par 14 Om hans barn bliva många, så är vinningen svärdets; hans avkomlingar få ej bröd att mätta sig med.
\par 15 De som slippa undan läggas i graven genom pest, och hans änkor kunna icke hålla sin klagogråt.
\par 16 Om han ock hopar silver såsom stoft och lägger kläder på hög såsom lera,
\par 17 så är det den rättfärdige som får kläda sig i vad han lägger på hög, och den skuldlöse kommer att utskifta silvret.
\par 18 Det hus han bygger bliver så förgängligt som malen, det skall likna skjulet som vaktaren gör sig.
\par 19 Rik lägger han sig och menar att intet skall tagas bort; men när han öppnar sina ögon, är ingenting kvar.
\par 20 Såsom vattenfloder taga förskräckelser honom fatt, om natten rövas han bort av stormen.
\par 21 Östanvinden griper honom, så att han far sin kos, den rycker honom undan från hans plats.
\par 22 Utan förskoning skjuter Gud sina pilar mot honom; för hans hand måste han flykta med hast.
\par 23 Då slår man ihop händerna, honom till hån; man visslar åt honom på platsen där han var.

\chapter{28}

\par 1 Silvret har ju sin gruva, sin fyndort har guldet, som man renar;
\par 2 järn hämtas upp ur jorden, och stenar smältas till koppar.
\par 3 Man sätter då gränser för mörkret, och rannsakar ned till yttersta djupet,
\par 4 Där spränger man schakt långt under markens bebyggare, där färdas man förgäten djupt under vandrarens fot, där hänger man svävande, fjärran ifrån människor.
\par 5 Ovan ur jorden uppväxer bröd, men därnere omvälves den såsom av eld.
\par 6 Där, bland dess stenar, har safiren sitt fäste, guldmalm hämtar man ock där.
\par 7 Stigen ditned är ej känd av örnen, och falkens öga har ej utspanat den;
\par 8 den har ej blivit trampad av stolta vilddjur, intet lejon har gått därfram.
\par 9 Ja, där bär man hand på hårda stenen; bergen omvälvas ända ifrån rötterna.
\par 10 In i klipporna bryter man sig gångar, där ögat får se allt vad härligt är.
\par 11 Vattenådror täppas till och hindras att gråta. Så dragas dolda skatter fram i ljuset.
\par 12 Men visheten, var finnes hon, och var har förståndet sin boning?
\par 13 Priset för henne känner ingen människa; hon står ej att finna i de levandes land.
\par 14 Djupet säger: "Hon är icke här", och havet säger: "Hos mig är hon icke."
\par 15 Hon köper icke för ädlaste metall, med silver gäldas ej hennes värde.
\par 16 Hon väges icke upp med guld från Ofir, ej med dyrbar onyx och safir.
\par 17 Guld och glas kunna ej liknas vid henne; hon får ej i byte mot gyllene klenoder.
\par 18 Koraller och kristall må icke ens nämnas; svårare är förvärva vishet än pärlor.
\par 19 Etiopisk topas kan ej liknas vid henne; hon väges icke upp med renaste guld.
\par 20 Ja, visheten, varifrån kommer väl hon, och var har förståndet sin boning?
\par 21 Förborgad är hon för alla levandes ögon, för himmelens fåglar är hon fördold;
\par 22 avgrunden och döden giva till känna; "Blott hörsägner om henne förnummo våra öron."
\par 23 Gud, han är den som känner vägen till henne, han är den som vet var hon har sin boning.
\par 24 Ty han förmår skåda till jordens ändar, allt vad som finnes under himmelen ser han.
\par 25 När han mätte ut åt vinden dess styrka och avvägde vattnen efter mått,
\par 26 när han stadgade en lag för regnet och en väg för tordönets stråle,
\par 27 då såg han och uppenbarade henne, då lät han henne stå fram, då utforskade han henne.
\par 28 Och till människorna sade han så: "Se Herrens fruktan, det är vishet, och att fly det onda är förstånd."

\chapter{29}

\par 1 Åter hov Job upp sin röst och kvad:
\par 2 Ack att jag vore såsom i forna månader, såsom i de dagar då Gud gav mig sitt beskydd,
\par 3 då hans lykta sken över mitt huvud och jag vid hans ljus gick fram genom mörkret!
\par 4 Ja, vore jag såsom i min mognads dagar, då Guds huldhet vilade över min hydda,
\par 5 då ännu den Allsmäktige var med mig och mina barn stodo runt omkring mig,
\par 6 då mina fötter badade i gräddmjölk och klippan invid mig göt ut bäckar av olja!
\par 7 När jag då gick upp till porten i staden och intog mitt säte på torget,
\par 8 då drogo de unga sig undan vid min åsyn, de gamla reste sig upp och blevo stående.
\par 9 Då höllo hövdingar tillbaka sina ord och lade handen på munnen;
\par 10 furstarnas röst ljöd då dämpad, och deras tunga lådde vid gommen.
\par 11 Ja, vart öra som hörde prisade mig då säll, och vart öga som såg bar vittnesbörd om mig;
\par 12 ty jag räddade den betryckte som ropade, och den faderlöse, den som ingen hjälpare hade.
\par 13 Den olyckliges välsignelse kom då över mig, och änkans hjärta uppfyllde jag med jubel.
\par 14 I rättfärdighet klädde jag mig, och den var såsom min klädnad; rättvisa bar jag såsom mantel och huvudbindel.
\par 15 Ögon blev jag då åt den blinde, och fötter var jag åt den halte.
\par 16 Jag var då en fader för de fattiga, och den okändes sak redde jag ut.
\par 17 Jag krossade den orättfärdiges käkar och ryckte rovet undan hans tänder.
\par 18 Jag tänkte då: "I mitt näste skall jag få dö, mina dagar skola bliva många såsom sanden.
\par 19 Min rot ligger ju öppen för vatten, och i min krona faller nattens dagg.
\par 20 Min ära bliver ständigt ny, och min båge föryngras i min hand."
\par 21 Ja, på mig hörde man då och väntade, man lyssnade under tystnad på mitt råd.
\par 22 Sedan jag hade talat, talade ingen annan; såsom ett vederkvickande flöde kommo mina ord över dem.
\par 23 De väntade på mig såsom på regn, de spärrade upp sina munnar såsom efter vårregn.
\par 24 När de misströstade, log jag emot dem, och mitt ansiktes klarhet kunde de icke förmörka.
\par 25 Täcktes jag besöka dem, så måste jag sitta främst; jag tronade då såsom en konung i sin skara, lik en man som har tröst för de sörjande.

\chapter{30}

\par 1 Och nu le de åt mig, människor som äro yngre till åren än jag, män vilkas fäder jag aktade ringa, ja, ej ens hade velat sätta bland mina vallhundar.
\par 2 Vad skulle de också kunna gagna mig med sin hjälp, dessa människor som sakna all manlig kraft?
\par 3 Utmärglade äro de ju av brist och svält; de gnaga sin föda av torra öknen, som redan i förväg är öde och ödslig.
\par 4 Saltörter plocka de där bland snåren, och ginströtter är vad de hava till mat.
\par 5 Ur människors samkväm drives de ut, man ropar efter dem såsom efter tjuvar.
\par 6 I gruvliga klyftor måste de bo, i hålor under jorden och i bergens skrevor.
\par 7 Bland snåren häva de upp sitt tjut, under nässlor ligga de skockade,
\par 8 en avföda av dårar och ärelöst folk, utjagade ur landet med hugg och slag.
\par 9 Och för sådana har jag nu blivit en visa, de hava mig till ämne för sitt tal;
\par 10 med avsky hålla de sig fjärran ifrån mig, de hava ej försyn för att spotta åt mig.
\par 11 Nej, mig till plåga, lossa de alla band, alla tyglar kasta de av inför mig.
\par 12 Invid min högra sida upphäver sig ynglet; mina fötter vilja de stöta undan. De göra sig vägar som skola leda till min ofärd.
\par 13 Stigen framför mig hava de rivit upp. De göra sitt bästa till att fördärva mig, de som dock själva äro hjälplösa.
\par 14 Såsom genom en bred rämna bryta de in; de vältra sig fram under murarnas brak.
\par 15 Förskräckelser välvas ned över mig. Såsom en storm bortrycka de min ära, och såsom ett moln har min välfärd farit bort.
\par 16 Och nu utgjuter sig min själ inom mig, eländesdagar hålla mig fast.
\par 17 Natten bortfräter benen i min kropp, och kvalen som gnaga mig veta ej av vila.
\par 18 Genom övermäktig kraft har mitt kroppshölje blivit vanställt, såsom en livklädnad hänger det omkring mig.
\par 19 I orenlighet har jag blivit nedstjälpt, och själv är jag nu lik stoft och aska.
\par 20 Jag ropar till dig, men du svarar mig icke; jag står här, men de bespejar mig allenast.
\par 21 Du förvandlas för mig till en grym fiende, med din starka hand ansätter du mig.
\par 22 Du lyfter upp mig i stormvinden och för mig hän, och i bruset låter du mig försmälta av ångest.
\par 23 Ja, jag förstår att du vill föra mig till döden, till den boning dit allt levande församlas.
\par 24 Men skulle man vid sitt fall ej få sträcka ut handen, ej ropa efter hjälp, när ofärd har kommit?
\par 25 Grät jag ej själv över den som hade hårda dagar, och ömkade sig min själ ej över den fattige?
\par 26 Se, jag väntade mig lycka, men olycka kom; jag hoppades på ljus, men mörker kom.
\par 27 Därför sjuder mitt innersta och får ingen ro, eländesdagar hava ju mött mig.
\par 28 Med mörknad hud går jag, fastän ej bränd av solen; mitt i församlingen står jag upp och skriar.
\par 29 En broder har jag blivit till schakalerna, och en frände är jag vorden till strutsarna.
\par 30 Min hud har svartnat och lossnat från mitt kött, benen i min kropp äro förbrända av hetta.
\par 31 I sorgelåt är mitt harpospel förbytt, mina pipors klang i högljudd gråt.

\chapter{31}

\par 1 Ett förbund slöt jag med mina ögon: aldrig skulle jag skåda efter någon jungfru.
\par 2 Vilken lott finge jag eljest av Gud i höjden, vilken arvedel av den Allsmäktige därovan?
\par 3 Ofärd kommer ju över de orättfärdiga, och olycka drabbar ogärningsmän.
\par 4 Ser icke han mina vägar, räknar han ej alla mina steg?
\par 5 Har jag väl umgåtts med lögn, och har min fot varit snar till svek?
\par 6 Nej, må jag vägas på en riktig våg, så skall Gud förnimma min ostrafflighet.
\par 7 Hava mina steg vikit av ifrån vägen, har mitt hjärta följt efter mina ögon, eller låder vid min händer en fläck?
\par 8 Då må en annan äta var jag har sått, och vad jag har planterat må ryckas upp med roten.
\par 9 Har mitt hjärta låtit dåra sig av någon kvinna, så att jag har stått på lur vid min nästas dörr?
\par 10 Då må min hustru mala mjöl åt en annan, och främmande män må då famntaga henne.
\par 11 Ja, sådant hade varit en skändlighet, en straffbar missgärning hade det varit,
\par 12 en eld som skulle förtära intill avgrunden och förhärja till roten all min gröda.
\par 13 Har jag kränkt min tjänares eller tjänarinnas rätt, när de hade någon tvist med mig?
\par 14 Vad skulle jag då göra, när Gud stode upp, och när han hölle räfst, vad kunde jag då svara honom?
\par 15 Han som skapade mig skapade ju och dem i moderlivet, han, densamme, har berett dem i modersskötet.
\par 16 Har jag vägrat de arma vad de begärde eller låtit änkans ögon försmäkta?
\par 17 Har jag ätit mitt brödstycke allena, utan att den faderlöse och har fått äta därav?
\par 18 Nej, från min ungdom fostrades han hos mig såsom hos en fader, och från min moders liv var jag änkors ledare.
\par 19 Har jag kunnat se en olycklig gå utan kläder, se en fattig ej äga något att skyla sig med?
\par 20 Måste ej fastmer hans länd välsigna mig, och fick han ej värma sig i ull av mina lamm?
\par 21 Har jag lyft min hand mot den faderlöse, därför att jag såg mig hava medhåll i porten?
\par 22 Då må min axel lossna från sitt fäste och min arm brytas av ifrån sin led.
\par 23 Jag måste då frukta ofärd ifrån Gud och skulle stå maktlös inför hans majestät.
\par 24 Har jag satt mitt hopp till guldet och kallat guldklimpen min förtröstan?
\par 25 Var det min glädje att min rikedom blev så stor, och att min hand förvärvade så mycket?
\par 26 Hände det, när jag såg solljuset, huru det sken, och månen, huru härligt den gick fram,
\par 27 att mitt hjärta hemligen lät dåra sig, så att jag med handkyss gav dem min hyllning?
\par 28 Nej, också det hade varit en straffbar missgärning; därmed hade jag ju förnekat Gud i höjden.
\par 29 Har jag glatt mig åt min fiendes ofärd och fröjdats, när olycka träffade honom?
\par 30 Nej, jag tillstadde ej min mun att synda så, ej att med förbannelse begära hans liv.
\par 31 Och kan mitt husfolk icke bevittna att envar fick mätta sig av kött vid mitt bord?
\par 32 Främlingen behövde ej stanna över natten på gatan, mina dörrar lät jag stå öppna utåt vägen.
\par 33 Har jag på människovis skylt mina överträdelser och gömt min missgärning i min barm,
\par 34 av fruktan för den stora hopen och av rädsla för stamfränders förakt, så att jag teg och ej gick utom min dörr?
\par 35 Ack att någon funnes, som ville höra mig! Jag har sagt mitt ord. Den Allsmäktige må nu svara mig; ack att jag finge min vederparts motskrift!
\par 36 Sannerligen, jag skulle då bära den högt på min skuldra, såsom en krona skulle jag fästa den på mig.
\par 37 Jag ville då göra honom räkenskap för alla mina steg, lik en furste skulle jag då träda inför honom.
\par 38 Har min mark höjt rop över mig, och hava dess fåror gråtit med varandra?
\par 39 Har jag förtärt dess gröda obetald eller utpinat dess brukares liv?
\par 40 Då må törne växa upp för vete, och ogräs i stället för korn. Slut på Jobs tal.

\chapter{32}

\par 1 De tre männen upphörde nu att svara Job, eftersom han höll sig själv för rättfärdig.
\par 2 Då blev Elihu, Barakels son, från Bus, av Rams släkt, upptänd av vrede. Mot Job upptändes han av vrede, därför att denne menade sig hava rätt mot Gud;
\par 3 och mot hans tre vänner upptändes hans vrede, därför att de icke funno något svar varmed de kunde vederlägga Job.
\par 4 Hittills hade Elihu dröjt att tala till Job, därför att de andra voro äldre till åren än han.
\par 5 Men då nu Elihu såg att de tre männen icke mer hade något att svara, upptändes hans vrede.
\par 6 Så tog då Elihu, Barakels son, från Bus, till orda och sade; Ung till åren är jag, I däremot ären gamla. Därför höll jag mig tillbaka och var försagd och lade ej fram för eder min mening.
\par 7 Jag tänkte: "Må åldern tala, och må årens mängd förkunna visdom."
\par 8 Dock, på anden i människorna kommer det an, den Allsmäktiges livsfläkt giver dem förstånd.
\par 9 Icke de åldriga äro alltid visast, icke de äldsta förstå bäst vad rätt är.
\par 10 Därför säger jag nu: Hör mig; jag vill lägga fram min mening, också jag.
\par 11 Se, jag väntade på vad I skullen tala, jag lyssnade efter förstånd ifrån eder, efter skäl som I skullen draga fram.
\par 12 Ja, noga aktade jag på eder. Men se, ingen fanns, som vederlade Job, ingen bland eder, som kunde svara på hans ord.
\par 13 Nu mån I icke säga: "Vi möttes av vishet; Gud, men ingen människa, kan nedslå denne."
\par 14 Skäl mot min mening har han icke lagt fram, ej heller skall jag bemöta honom med edra bevis.
\par 15 Se, nu stå de bestörta och svara ej mer, målet i munnen hava de mist.
\par 16 Och jag skulle vänta, då de nu intet kunna säga, då de stå där och ej mer hava något svar!
\par 17 Nej, också jag vill svara i min ordning, jag vill lägga fram min mening, också jag.
\par 18 Ty, fullt upp har jag av skäl, anden i mitt inre vill spränga mig sönder.
\par 19 Ja, mitt inre är såsom instängt vin, likt en lägel med nytt vin är det nära att brista.
\par 20 Så vill jag då tala och skaffa mig luft, jag vill upplåta mina läppar och svara.
\par 21 Jag får ej hava anseende till personen, och jag skall ej till någon tala inställsamma ord.
\par 22 Nej, jag förstår ej att tala inställsamma ord; huru lätt kunde ej eljest min skapare rycka mig bort!

\chapter{33}

\par 1 Men hör nu, Job, mina ord, och lyssna till allt vad jag vill säga.
\par 2 Se, jag upplåter nu mina läppar, min tunga tager till orda i min mun.
\par 3 Ur ett redbart hjärta framgår mitt tal, och vad mina läppar förstå säga de ärligt ut.
\par 4 Guds ande är det som har gjort mig, den Allsmäktiges fläkt beskär mig liv.
\par 5 Om du förmår, så må du nu svara mig; red dig till strid mot mig, träd fram.
\par 6 Se, jag är likställd med dig inför Gud, jag är danad av en nypa ler, också jag.
\par 7 Ja, fruktan för mig behöver ej förskräcka dig, ej heller kan min myndighet trycka dig ned.
\par 8 Men nu sade du så inför mina öron, så ljödo de ord jag hörde:
\par 9 "Ren är jag och fri ifrån överträdelse, oskyldig är jag och utan missgärning;
\par 10 men se, han finner på sak mot mig, han aktar mig såsom sin fiende.
\par 11 Han sätter mina fötter i stocken, vaktar på alla mina vägar."
\par 12 Nej, häri har du orätt, svarar jag dig. Gud är ju förmer än en människa.
\par 13 Huru kan du gå till rätta med honom, såsom gåve han aldrig svar i sin sak?
\par 14 Både på ett sätt och på två talar Gud, om man också ej aktar därpå.
\par 15 I drömmen, i nattens syn, när sömnen har fallit tung över människorna och de vila i slummer på sitt läger,
\par 16 då öppnar han människornas öron och sätter inseglet på sina varningar till dem,
\par 17 när han vill avvända någon från en ogärning eller hålla högmodet borta ifrån en människa.
\par 18 Så bevarar han hennes själ från graven och hennes liv ifrån att förgås genom vapen.
\par 19 Hon bliver ock agad genom plågor på sitt läger och genom ständig oro, allt intill benen.
\par 20 Hennes sinne får leda vid maten, och hennes själ vid den föda hon älskade.
\par 21 Hennes hull förtvinar, till dess intet är att se, ja, hennes ben täras bort intill osynlighet.
\par 22 Så nalkas hennes själ till graven och hennes liv hän till dödens makter.
\par 23 Men om en ängel då finnes, som vakar över henne, en medlare, någon enda av de tusen, och denne får lära människan hennes plikt,
\par 24 då förbarmar Gud sig över henne och säger; "Fräls henne, så att hon slipper fara ned i graven; lösepenningen har jag nu fått."
\par 25 Hennes kropp får då ny ungdomskraft, hon bliver åter såsom under sin styrkas dagar.
\par 26 När hon då beder till Gud, är han henne nådig och låter henne se sitt ansikte med jubel; han giver så den mannen hans rättfärdighet åter.
\par 27 Så får denne då sjunga inför människorna och säga: "Väl syndade jag, och väl kränkte jag rätten, dock vederfors mig ej vad jag hade förskyllt;
\par 28 ty han förlossade min själ, så att den undslapp graven, och mitt liv får nu med lust skåda ljuset."
\par 29 Se, detta allt kommer Gud åstad, både två gånger och tre, för den mannen,
\par 30 till att rädda hans själ från graven, så att han får njuta av de levandes ljus.
\par 31 Akta nu härpå, du Job, och hör mig; tig, så att jag får tala.
\par 32 Dock, har du något att säga, så svara mig; tala, ty gärna gåve jag dig rätt.
\par 33 Varom icke, så är det du som må höra på mig; du må tiga, så att jag får lära dig vishet.

\chapter{34}

\par 1 Och Elihu tog till orda och sade:
\par 2 Hören, I vise, mina ord; I förståndige, lyssnen till mig.
\par 3 Örat skall ju pröva orden, och munnen smaken hos det man vill äta.
\par 4 Må vi nu utvälja åt oss vad rätt är, samfällt söka förstå vad gott är.
\par 5 Se, Job har sagt: "Jag är oskyldig. Gud har förhållit mig min rätt.
\par 6 Fastän jag har rätt, måste jag stå såsom lögnare; dödsskjuten är jag, jag som intet har brutit."
\par 7 Var finnes en man som är såsom Job? Han läskar sig med bespottelse såsom med vatten,
\par 8 han gör sig till ogärningsmäns stallbroder och sällar sig till ogudaktiga människor.
\par 9 Ty han säger: "Det gagnar en man till intet, om han håller sig väl med Gud."
\par 10 Hören mig därför, I förståndige män: Bort det, att Gud skulle begå någon orätt, att den Allsmäktige skulle göra vad orättfärdigt är!
\par 11 Nej, han vedergäller var människa efter hennes gärningar och lönar envar såsom hans vandel har förtjänat.
\par 12 Ty Gud gör i sanning intet som är orätt, den Allsmäktige kan icke kränka rätten.
\par 13 Vem har bjudit honom att vårda sig om jorden, och vem lade på honom bördan av hela jordens krets?
\par 14 Om han ville tänka allenast på sig själv och åter draga till sig sin anda och livsfläkt,
\par 15 då skulle på en gång allt kött förgås, och människorna skulle vända åter till stoft.
\par 16 Men märk nu väl och hör härpå, lyssna till vad mina ord förkunna.
\par 17 Skulle den förmå regera, som hatade vad rätt är? Eller fördömer du den som är den störste i rättfärdighet?
\par 18 Får man då säga till en konung: "Du ogärningsman", eller till en furste: "Du ogudaktige"?
\par 19 Gud har ju ej anseende till någon hövdings person, han aktar den rike ej för mer än den fattige, ty alla äro de hans händers verk.
\par 20 I ett ögonblick omkomma de, mitt i natten: folkhopar gripas av bävan och förgås, de väldige ryckas bort, utan människohand.
\par 21 Ty hans ögon vakta på var mans vägar, och alla deras steg, dem ser han.
\par 22 Intet mörker finnes och ingen skugga så djup, att ogärningsmän kunna fördölja sig däri.
\par 23 Ty länge behöver Gud ej vakta på en människa, innan hon måste stå till doms inför honom.
\par 24 Han krossar de väldige utan rannsakning och låter så andra träda fram i deras ställe.
\par 25 Ja, han märker väl vad de göra, han omstörtar dem om natten och låter dem förgås.
\par 26 Såsom ogudaktiga tuktar han dem öppet, inför människors åsyn,
\par 27 eftersom de veko av ifrån honom och ej aktade på alla hans vägar.
\par 28 De bragte så den armes rop inför honom, och rop av betryckta fick han höra.
\par 29 Vem vågar då fördöma, om han stillar larmet? Ja, vem vill väl skåda honom, om han döljer sitt ansikte, för ett folk eller för en enskild man,
\par 30 när han vill rycka makten ifrån gudlösa människor och hindra dem att bliva snaror för folket?
\par 31 Kan man väl säga till Gud: "Jag måste lida, jag som ändå intet har förbrutit.
\par 32 Visa mig du vad som går över mitt förstånd; om jag har gjort något orätt, vill jag då ej göra så mer."
\par 33 Skall då han, för ditt klanders skull, giva vedergällning såsom du vill? Du själv, och icke jag, må döma därom; ja, tala du ut vad du menar.
\par 34 Men kloka män skola säga så till mig, visa män, när de få höra mig:
\par 35 "Job talar utan någon insikt, hans ord äro utan förstånd."
\par 36 Så må nu Job utstå prövningar allt framgent, då han vill försvara sig på ogärningsmäns sätt.
\par 37 Till sin synd lägger han ju uppenbar ondska, oss till hån slår han ihop sina händer och talar stora ord mot Gud.

\chapter{35}

\par 1 Och Elihu tog till orda och sade:
\par 2 Menar du att sådant är riktigt? Kan du påstå att du har rätt mot Gud,
\par 3 du som frågar vad rättfärdighet gagnar dig, vad den båtar dig mer än synd?
\par 4 Svar härpå vill jag giva dig, jag ock dina vänner med dig.
\par 5 Skåda upp mot himmelen och se, betrakta skyarna, som gå där högt över dig.
\par 6 Om du syndar, vad gör du väl honom därmed? Och om dina överträdelser äro många, vad skadar du honom därmed?
\par 7 Eller om du är rättfärdig, vad giver du honom, och vad undfår han av din hand?
\par 8 Nej, för din like kunde din ogudaktighet något betyda och för en människoson din rättfärdighet.
\par 9 Väl klagar man, när våldsgärningarna äro många, man ropar om hjälp mot de övermäktigas arm;
\par 10 men ingen frågar: "Var är min Gud, min skapare, han som låter lovsånger ljuda mitt i natten,
\par 11 han som giver oss insikt framför markens djur och vishet framför himmelens fåglar?"
\par 12 Därför är det man får ropa utan svar om skydd mot de ondas övermod.
\par 13 Se, på fåfängliga böner hör icke Gud, den Allsmäktige aktar icke på slikt;
\par 14 allra minst, när du påstår att du icke får skåda honom, att du måste vänta på honom, fastän saken är uppenbar.
\par 15 Och nu menar du att hans vrede ej håller någon räfst, och att han föga bekymrar sig om människors övermod?
\par 16 Ja, till fåfängligt tal spärrar Job upp sin mun, utan insikt talar han stora ord.

\chapter{36}

\par 1 Vidare sade Elihu:
\par 2 Bida ännu litet, så att jag får giva dig besked, ty ännu något har jag att säga till Guds försvar.
\par 3 Min insikt vill jag hämta vida ifrån, och åt min skapare vill jag skaffa rätt.
\par 4 Ja, förvisso skola mina ord icke vara lögn; en man med fullgod insikt har du framför dig.
\par 5 Se, Gud är väldig, men han försmår dock ingen, han som är så väldig i sitt förstånds kraft.
\par 6 Den ogudaktige låter han ej bliva vid liv, men åt de arma skaffar han rätt.
\par 7 Han tager ej sina ögon från de rättfärdiga; de få trona i konungars krets, för alltid låter han dem sitta där i höghet.
\par 8 Och om de läggas bundna i kedjor och fångas i eländets snaror,
\par 9 så vill han därmed visa dem vad de hava gjort, och vilka överträdelser de hava begått i sitt högmod;
\par 10 han vill då öppna deras öra för tuktan och mana dem att vända om ifrån fördärvet.
\par 11 Om de då höra på honom och underkasta sig, så få de framleva sina dagar i lycka och sina år i ljuvlig ro.
\par 12 Men höra de honom ej, så förgås de genom vapen och omkomma, när de minst tänka det.
\par 13 Ja, de som med gudlöst hjärta hängiva sig åt vrede och icke anropa honom, när han lägger dem i band,
\par 14 deras själ skall i deras ungdom ryckas bort av döden, och deras liv skall dela tempelbolares lott.
\par 15 Genom lidandet vill han rädda den lidande, och genom betrycket vill han öppna hans öra.
\par 16 Så sökte han ock draga dig ur nödens gap, ut på en rymlig plats, där intet trångmål rådde; och ditt bord skulle bliva fullsatt med feta rätter.
\par 17 Men nu bär du till fullo ogudaktighetens dom; ja, dom och rättvisa hålla dig nu fast.
\par 18 Ty vrede borde ej få uppegga dig under din tuktans tid, och huru svårt du än har måst plikta, borde du ej därav ledas vilse.
\par 19 Huru kan han lära dig bedja, om icke genom nöd och genom allt som nu har prövat din kraft?
\par 20 Du må ej längta så ivrigt efter natten, den natt då folken skola ryckas bort ifrån sin plats.
\par 21 Tag dig till vara, så att du ej vänder dig till vad fördärvligt är; sådant behagar dig ju mer än att lida.
\par 22 Se, Gud är upphöjd genom sin kraft. Var finnes någon mästare som är honom lik?
\par 23 Vem har föreskrivit honom hans väg, och vem kan säga: "Du gör vad orätt är?"
\par 24 Tänk då på att upphöja hans gärningar, dem vilka människorna besjunga
\par 25 och som de alla skåda med lust, de dödliga, om de än blott skönja dem i fjärran.
\par 26 Ja, Gud är för hög för vårt förstånd, hans år äro flera än någon kan utrannsaka.
\par 27 Se, vattnets droppar drager han uppåt, och de sila ned såsom regn, där hans dimma går fram;
\par 28 skyarna gjuta dem ut såsom en ström, låta dem drypa ned över talrika människor.
\par 29 Ja, kan någon fatta molnens utbredning, braket som utgår från hans hydda?
\par 30 Se, sitt ljungeldsljus breder han ut över molnen, och själva havsgrunden höljer han in däri.
\par 31 Ty så utför han sina domar över folken; så bereder han ock näring i rikligt mått.
\par 32 I ljungeldsljus höljer han sina händer och sänder det ut mot dem som begynna strid.
\par 33 Budskap om honom bär hans dunder; själva boskapen bebådar hans antåg.

\chapter{37}

\par 1 Ja, vid sådant förskräckes mitt hjärta, bävande spritter det upp.
\par 2 Hören, hören huru hans röst ljuder vred, hören dånet som går ut ur hans mun.
\par 3 Han sänder det åstad, så långt himmelen når, och sina ljungeldar bort till jordens ändar.
\par 4 Efteråt ryter så dånet, när han dundrar med sin väldiga röst; och på ljungeldarna spar han ej, då hans röst låter höra sig.
\par 5 Ja, underbart dundrar Gud med sin röst, stora ting gör han, utöver vad vi förstå.
\par 6 Se, åt snön giver han bud: "Fall ned till jorden", så ock åt regnskuren, åt sitt regnflödes mäktiga skur.
\par 7 Därmed fjättrar han alla människors händer, så att envar som han har skapat kan lära därav.
\par 8 Då draga sig vilddjuren in i sina gömslen, och i sina kulor lägga de sig till ro.
\par 9 Från Stjärngemaket kommer då storm och köld genom nordanhimmelens stjärnor;
\par 10 med sin andedräkt sänder Gud frost, och de vida vattnen betvingas.
\par 11 Skyarna lastar han ock med väta och sprider omkring sina ljungeldsmoln.
\par 12 De måste sväva än hit, än dit, alltefter hans rådslut och de uppdrag de få, vadhelst han ålägger dem på jordens krets.
\par 13 Än är det som tuktoris, än med hjälp åt hans jord, än är det med nåd som han låter dem komma.
\par 14 Lyssna då härtill, du Job; stanna och betänk Guds under.
\par 15 Förstår du på vad sätt Gud styr deras gång och låter ljungeldarna lysa fram ur sina moln?
\par 16 Förstår du lagen för skyarnas jämvikt, den Allvises underbara verk?
\par 17 Förstår du huru kläderna bliva dig så heta, när han låter jorden domna under sunnanvinden?
\par 18 Kan du välva molnhimmelen så som han, så fast som en spegel av gjuten metall?
\par 19 Lär oss då vad vi skola säga till honom; för vårt mörkers skull hava vi intet att lägga fram.
\par 20 Ej må det bebådas honom att jag vill tala. Månne någon begär sitt eget fördärv?
\par 21 Men synes icke redan skenet? Strålande visar han sig ju mellan skyarna, där vinden har gått fram och sopat dem undan.
\par 22 I guldglans kommer han från norden. Ja, Gud är höljd i fruktansvärt majestät;
\par 23 den Allsmäktige kunna vi icke fatta, honom som är så stor i kraft, honom som ej kränker rätten, ej strängaste rättfärdighet.
\par 24 Fördenskull frukta människorna honom; men de självkloka - dem alla aktar han ej på.

\chapter{38}

\par 1 Och HERREN svarade Job ur stormvinden och sade:
\par 2 Vem är du som stämplar vishet såsom mörker, i det att du talar så utan insikt?
\par 3 Omgjorda nu såsom ej man dina länder; jag vill fråga dig, och du må giva mig besked.
\par 4 Var var du, när jag lade jordens grund? Säg det, om du har ett så stort förstånd.
\par 5 Vem har fastställt hennes mått - du vet ju det? Och vem spände sitt mätsnöre ut över henne?
\par 6 Var fingo hennes pelare sina fästen, och vem var det som lade hennes hörnsten,
\par 7 medan morgonstjärnorna tillsammans jublade och alla Guds söner höjde glädjerop?
\par 8 Och vem satte dörrar för havet, när det föddes och kom ut ur moderlivet,
\par 9 när jag gav det moln till beklädnad och lät töcken bliva dess linda,
\par 10 när jag åt det utstakade min gräns och satte bom och dörrar därför,
\par 11 och sade: "Härintill skall du komma, men ej vidare, här skola dina stolta böljor lägga sig"?
\par 12 Har du i din tid bjudit dagen att gry eller anvisat åt morgonrodnaden dess plats,
\par 13 där den skulle fatta jorden i dess flikar, så att de ogudaktiga skakades bort därifrån?
\par 14 Då ändrar den form såsom leran under signetet, och tingen stå fram såsom klädda i skrud;
\par 15 då berövas de ogudaktiga sitt ljus, och den arm som lyftes för högt brytes sönder.
\par 16 Har du stigit ned till havets källor och vandrat omkring på djupets botten?
\par 17 Hava dödens portar avslöjat sig för dig, ja, såg du dödsskuggans portar?
\par 18 Har du överskådat jordens vidder? Om du känner allt detta, så låt höra.
\par 19 Vet du vägen dit varest ljuset bor, eller platsen där mörkret har sin boning,
\par 20 så att du kan hämta dem ut till deras gräns och finna stigarna som leda till deras hus?
\par 21 Visst kan du det, ty så tidigt blev du ju född, så stort är ju dina dagars antal!
\par 22 Har du varit framme vid snöns förrådshus? Och haglets förrådshus, du såg väl dem
\par 23 - de förråd som jag har sparat till hemsökelsens tid, till stridens och drabbningens dag?
\par 24 Vet du vägen dit varest ljuset delar sig, dit där stormen sprider sig ut över jorden?
\par 25 Vem har åt regnflödet öppnat en ränna och banat en väg för tordönets stråle,
\par 26 till att sända regn över länder där ingen bor, över öknar, där ingen människa finnes,
\par 27 till att mätta ödsliga ödemarker och giva växt åt gräsets brodd?
\par 28 Säg om regnet har någon fader, och vem han är, som födde daggens droppar?
\par 29 Ur vilken moders liv är det isen gick fram, och vem är hon som födde himmelens rimfrost?
\par 30 Se, vattnet tätnar och bliver likt sten, så ytan sluter sig samman över djupet.
\par 31 Knyter du tillhopa Sjustjärnornas knippe? Och förmår du att lossa Orions band?
\par 32 Är det du som, när tid är, för himmelstecknen fram, och som leder Björninnan med hennes ungar?
\par 33 Ja, förstår du himmelens lagar, och ordnar du dess välde över jorden?
\par 34 Kan du upphöja din röst till molnen och förmå vattenflöden att övertäcka dig?
\par 35 Kan du sända ljungeldar åstad, så att de gå, så att de svara dig: "Ja vi äro redo"?
\par 36 Vem har lagt vishet i de mörka molnen, och vem gav förstånd åt järtecknen i luften?
\par 37 Vem håller med sin vishet räkning på skyarna? Och himmelens läglar, vem häller ut dem,
\par 38 medan mullen smälter såsom malm och jordkokorna klibbas tillhopa?

\chapter{39}

\par 1 Är det du som jagar upp rov åt lejoninnan och stillar de unga lejonens hunger,
\par 2 när de trycka sig ned i sina kulor eller ligga på lur i snåret?
\par 3 Vem är det som skaffar mat åt korpen, när hans ungar ropar till Gud, där de sväva omkring utan föda?
\par 4 Vet du tiden för stengetterna att föda, vakar du över när hindarna bör kalva?
\par 5 Räknar du månaderna som de skola gå dräktiga, ja, vet du tiden för dem att föda?
\par 6 De böja sig ned, de avbörda sig sina foster, hastigt göra de sig fria ifrån födslovåndan.
\par 7 Deras ungar frodas och växa till på marken, så springa de sin väg och vända ej tillbaka.
\par 8 Vem har skänkt vildåsnan hennes frihet, vem har lossat den skyggas band?
\par 9 Se, hedmarken gav jag henne till hem, och saltöknen blev hennes boning.
\par 10 Hon ler åt larmet i staden, hon hör ingen pådrivares rop.
\par 11 Vad hon spanar upp på berget har hon till bete, hon letar efter allt som är grönt.
\par 12 Skall vildoxen finnas hågad att tjäna dig och att stanna över natten invid din krubba?
\par 13 Kan du tvinga vildoxen att gå i fåran efter töm och förmå honom att i ditt spår harva markerna jämna?
\par 14 Kan du lita på honom, då ju hans kraft är så stor, kan du betro åt honom ditt arbetes frukt?
\par 15 Överlåter du åt honom att föra hem din säd och att hämta den tillhopa till din loge?
\par 16 Strutshonans vingar flaxa med fröjd, men vad modersömhet visa väl hennes pennor, hennes fjädrar?
\par 17 Åt jorden överlåter hon ju sina ägg och ruvar dem ovanpå sanden.
\par 18 Hon bryr sig ej om att en fot kan krossa dem, att ett vilddjur kan trampa dem sönder.
\par 19 Hård är hon mot sin avkomma, såsom vore den ej hennes; att hennes avel kan gå under, det bekymrar henne ej.
\par 20 Ty Gud har gjort henne glömsk för vishet, han har ej tilldelat henne förstånd.
\par 21 Men när det gäller, piskar hon sig själv upp till språng; då ler hon åt både häst och man.
\par 22 Är det du som giver åt hästen hans styrka och kläder hans hals med brusande man?
\par 23 Är det du som lär honom gräshoppans språng? Hans stolta frustning, en förskräckelse är den!
\par 24 Han skrapar marken och fröjdar sig i sin kraft och rusar så fram mot väpnade skaror.
\par 25 Han ler åt fruktan och känner ej förfäran, han ryggar icke tillbaka för svärd.
\par 26 Omkring honom ljuder ett rassel av koger, av ljungande spjut och lans.
\par 27 Han skakas och rasar och uppslukar marken, han kan icke styra sig, när basunen har ljudit.
\par 28 För var basunstöt frustar han: Huj! Ännu i fjärran vädrar han striden, anförarnas rop och larmet av härskrin.
\par 29 Är det ett verk av ditt förstånd, att falken svingar sig upp och breder ut sina vingar till flykt mot söder?
\par 30 Eller är det på ditt bud som örnen stiger så högt och bygger sitt näste i höjden?
\par 31 På klippan bor han, där har han sitt tillhåll, på klippans spets och på branta berget.
\par 32 Därifrån spanar han efter sitt byte, långt bort i fjärran skådar hans ögon.
\par 33 Hans ungar frossa på blod, och där slagna ligga, där finner man honom.
\par 34 Så svarade nu HERREN Job och sade:
\par 35 Vill du tvista med den Allsmäktige, du mästare? Svara då, du som så klagar på Gud!
\par 36 Job svarade HERREN och sade:
\par 37 Nej, därtill är jag för ringa; vad skulle jag svara dig? Jag måste lägga handen på munnen.
\par 38 En gång har jag talat, och nu säger jag intet mer; ja, två gånger, men jag gör det icke åter.

\chapter{40}

\par 1 Och HERREN talade till Job ur stormvinden och sade:
\par 2 Omgjorda såsom en man dina länder; jag vill fråga dig, och du må giva mig besked.
\par 3 Vill du göra min rätt om intet och döma mig skyldig, för att själv stå såsom rättfärdig?
\par 4 Har du en sådan arm som Gud, och förmår du dundra med din röst såsom han?
\par 5 Pryd dig då med ära och höghet, kläd dig i majestät och härlighet.
\par 6 Gjut ut din vredes förgrymmelse, ödmjuka med en blick allt vad högt är.
\par 7 Ja, kuva med en blick allt vad högt är, slå ned de ogudaktiga på stället.
\par 8 Göm dem i stoftet allasammans, ja, fjättra deras ansikten i mörkret.
\par 9 Då vill jag prisa dig, också jag, för segern som din högra hand har berett dig.
\par 10 Se, Behemot, han är ju mitt verk såväl som du. Han lever av gräs såsom en oxe.
\par 11 Och se vilken kraft han äger i sina länder, vilken styrka han har i sin buks muskler.
\par 12 Han bär sin svans så styv som en ceder, ett konstrikt flätverk äro senorna i hans lår.
\par 13 Hans benpipor äro såsom rör av koppar, benen i hans kropp likna stänger av järn.
\par 14 Förstlingen är han av vad Gud har gjort; hans skapare själv har givit honom hans skära.
\par 15 Ty foder åt honom frambära bergen, där de vilda djuren alla hava sin lek.
\par 16 Under lotusträd lägger han sig ned, i skygdet av rör och vass.
\par 17 Lotusträd giva honom tak och skugga, pilträd hägna honom runt omkring.
\par 18 Är floden än så våldsam, så ängslas han dock icke; han är trygg, om ock en Jordan bryter fram mot hans gap.
\par 19 Vem kan fånga honom, när han är på sin vakt, vem borrar en snara genom hans nos?
\par 20 Kan du draga upp Leviatan med krok och med en metrev betvinga hans tunga?
\par 21 Kan du sätta en sävhank i hans nos eller borra en hake genom hans käft?
\par 22 Menar du att han skall slösa på dig många böner eller tala till dig med mjuka ord?
\par 23 Att han skall vilja sluta fördrag med dig, så att du finge honom till din träl för alltid?
\par 24 Kan du hava honom till leksak såsom en fågel och sätta honom i band åt dina tärnor?
\par 25 Pläga fiskarlag köpslå om honom och stycka ut hans kropp mellan krämare?
\par 26 Kan du skjuta hans hud full med spjut och hans huvud med fiskharpuner?
\par 27 Ja, försök att bära hand på honom du skall minnas den striden och skall ej föra så mer.
\par 28 Nej, den sådant vågar, hans hopp bliver sviket, han fälles till marken redan vid hans åsyn.

\chapter{41}

\par 1 Så oförvägen är ingen, att han törs reta denne. Vem vågar då sätta sig upp mot mig själv?
\par 2 Vem har först givit mig något, som jag alltså bör betala igen? Mitt är ju allt vad som finnes under himmelen.
\par 3 Jag vill ej höra upp att tala om hans lemmar, om huru väldig han är, och huru härligt han är danad.
\par 4 Vem mäktar rycka av honom hans pansar? Vem vågar sig in mellan hans käkars par?
\par 5 Hans gaps dörrar, vem vill öppna dem? Runtom hans tänder bor ju förskräckelse.
\par 6 Stolta sitta på honom sköldarnas rader; hopslutna äro de med fast försegling.
\par 7 Tätt fogar sig den ena intill den andra, icke en vindfläkt tränger in mellan dem.
\par 8 Var och en håller ihop med den nästa, de gripa in i varandra och skiljas ej åt.
\par 9 När han fnyser, strålar det av ljus; hans blickar äro såsom morgonrodnadens ögonbryn.
\par 10 Bloss fara ut ur hans gap, eldgnistor springa fram därur.
\par 11 Från hans näsborrar utgår rök såsom ur en sjudande panna på bränslet.
\par 12 Hans andedräkt framgnistrar eldkol, och lågor bryta fram ur hans gap.
\par 13 På hans hals har kraften sin boning, och framför honom stapplar försagdhet.
\par 14 Själva det veka på hans buk är ett stadigt fogverk, det sitter orubbligt, såsom gjutet på honom.
\par 15 Hans hjärta är fast såsom sten, fast såsom bottenstenen i kvarnen.
\par 16 När han reser sig, bäva hjältar, av ångest mista de all sans.
\par 17 Angripes han med ett svärd, så håller det ej stånd, ej heller spjut eller pil eller pansar.
\par 18 Han aktar järn såsom halm och koppar såsom murket trä.
\par 19 Bågskott skrämma honom ej bort, slungstenar förvandlas för honom till strå;
\par 20 ja, stridsklubbor aktar han såsom strå, han ler åt rasslet av lansar.
\par 21 På sin buk bär han skarpa eggar, spår såsom av en tröskvagn ristar han i dyn.
\par 22 Han gör djupet sjudande som en gryta, likt en salvokokares kittel förvandlar han vattnet.
\par 23 Bakom honom strålar vägen av ljus, djupet synes bära silverhår.
\par 24 Ja, på jorden finnes intet som är honom likt, otillgänglig för fruktan skapades han.
\par 25 På allt vad högt är ser han med förakt, konung är han över alla stolta vilddjur.

\chapter{42}

\par 1 Job svarade HERREN och sade:
\par 2 Ja, jag vet att du förmår allt, och att intet som du besluter är dig för svårt.
\par 3 Vem var då jag som i oförstånd gav vishet namn av mörker? Jag ordade ju om vad jag icke begrep, om det som var mig för underbart och det jag ej kunde förstå.
\par 4 Men hör nu, så vill jag tala; jag vill fråga dig, och du må giva mig besked.
\par 5 Blott hörsägner hade jag förnummit om dig, men nu har jag fått se dig med egna ögon.
\par 6 Därför tager jag det tillbaka och ångrar mig, i stoft och aska.
\par 7 Sedan HERREN hade talat så till Job, sade han till Elifas från Teman: "Min vrede är upptänd mot dig och dina båda vänner, därför att I icke haven talat om mig vad rätt är, såsom min tjänare Job har gjort.
\par 8 Så tagen eder nu sju tjurar och sju vädurar, och gån till min tjänare Job och offren dem såsom brännoffer för eder; ock låten min tjänare Job bedja för eder. Till äventyrs skall jag då, av nåd mot honom, avstå från att göra något förskräckligt mot eder, till straff därför att I icke haven talat om mig vad rätt är, såsom min tjänare Job har gjort."
\par 9 Då gingo Elifas från Teman, Bildad från Sua och Sofar från Naaman åstad och gjorde såsom HERREN hade tillsagt dem; och HERREN tog nådigt emot Jobs bön.
\par 10 Och då nu Job bad för sina vänner, upprättade HERREN åter honom själv; HERREN gav Job dubbelt igen mot vad han förut hade haft.
\par 11 Och alla hans bröder och systrar och alla hans forna bekanta kommo till honom och höllo måltid med honom i hans hus, och ömkade honom för alla de olyckor som HERREN hade låtit komma över honom. Och de gåvo honom vardera en kesita och en guldring.
\par 12 Och HERREN välsignade slutet av Jobs levnad ännu mer än begynnelsen, så att han fick fjorton tusen får, sex tusen kameler, ett tusen par oxar och ett tusen åsninnor.
\par 13 Och han fick sju söner och tre döttrar.
\par 14 Den första dottern kallade han Jemima, den andra Kesia och den tredje Keren-Happuk.
\par 15 Och så sköna kvinnor som Jobs döttrar funnos icke i hela landet; och deras fader gav dem arvedel bland deras bröder.
\par 16 Och Job levde därefter ett hundra fyrtio år, och fick se sina barn och barnbarn i fyra led.
\par 17 Sedan dog Job, gammal och mätt på att leva.


\end{document}