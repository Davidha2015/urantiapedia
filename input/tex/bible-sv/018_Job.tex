\begin{document}

\title{Job}

Job 1:1  I Us' land levde en man som hette Job; han var en ostrafflig och redlig man, som fruktade Gud och flydde det onda.
Job 1:2  Åt honom föddes sju söner och tre döttrar;
Job 1:3  och han ägde sju tusen får, tre tusen kameler, fem hundra par oxar och fem hundra åsninnor, därtill tjänare i stor mängd. Så var denne man mäktigare än någon annan i Österlandet.
Job 1:4  Och hans söner hade för sed att gå åstad och hålla gästabud, den ena dagen i den enes hus, den andra dagen i den andres; de sände då och inbjödo sina tre systrar att äta och dricka tillsammans med dem.
Job 1:5  När så en omgång av gästabudsdagar var till ända, sände Job efter dem för att helga dem; bittida om morgonen offrade han då ett brännoffer för var och en av dem. Ty Job tänkte "Kanhända hava mina barn syndat och i sina hjärtan talat förgripligt om Gud". Så gjorde Job för var gång.
Job 1:6  Men nu hände sig en dag att Guds söner kommo och trädde fram inför HERREN, och Åklagaren kom också med bland dem.
Job 1:7  Då frågade HERREN Åklagaren: "Varifrån kommer du?" Åklagaren svarade HERREN och sade: "Från en vandring utöver jorden och från en färd omkring på den."
Job 1:8  Då sade HERREN till Åklagaren: "Har du givit akt på min tjänare Job? Ty på jorden finnes icke hans like i ostrafflighet och redlighet, ingen som så fruktar Gud och flyr det onda."
Job 1:9  Åklagaren svarade HERREN och sade: "Är det då för intet som Job fruktar Gud?
Job 1:10  Du har ju på allt sätt beskärmat honom och hans hus och allt vad han äger; du har välsignat hans händers verk, och hans boskapshjordar hava utbrett sig i landet.
Job 1:11  Man räck ut din hand och kom vid detta allt som han äger; förvisso skall han då mitt i ansiktet tala förgripliga ord mot dig."
Job 1:12  HERREN sade till Åklagaren: "Välan, allt vad han äger vare givet i din hand; allenast mot honom själv må du icke räcka ut din hand." Så gick Åklagaren bort ifrån HERRENS ansikte.
Job 1:13  När nu en dag hans söner och döttrar höllo måltid och drucko vin i den äldste broderns hus,
Job 1:14  Kom en budbärare till Job och sade: "Oxarna gingo för plogen, och åsninnorna betade därbredvid;
Job 1:15  då föllo sabéerna in och rövade bort dem, och folket slogo de med svärdsegg. Jag var den ende som kom undan, för att jag skulle underrätta dig därom."
Job 1:16  Medan denne ännu talade, kom åter en och sade: "Guds eld föll ifrån himmelen och slog ned bland småboskapen och folket och förtärde dem. Jag var den ende som kom undan, för att jag skulle underrätta dig därom."
Job 1:17  Medan denne ännu talade, kom åter en och sade: Kaldéerna ställde upp sitt manskap i tre hopar och föllo så över kamelerna och rövade bort dem, och folket slogo de med svärdsegg. Jag var den ende som kom undan, för att jag skulle underrätta dig därom."
Job 1:18  Under det att denne ännu talade, kom åter en annan och sade: Dina söner och döttrar höllo måltid och drucko vin i den äldste broderns hus;
Job 1:19  då kom en stark storm fram över öknen och tog tag i husets fyra hörn, och det föll omkull över folket, så att de förgingos. Jag var den ende som kom undan, för att jag skulle underrätta dig därom."
Job 1:20  Då stod Job upp och rev sönder sin mantel och skar av håret på sitt huvud. Och han föll ned till jorden och tillbad
Job 1:21  och sade: "Naken kom jag ur min moders liv, och naken skall jag vända åter dit; HERREN gav, och HERREN tog. Lovat vare HERRENS namn!"
Job 1:22  Vid allt detta syndade Job icke och talade intet lasteligt mot Gud.
Job 2:1  Åter hände sig en dag att Guds söner kommo och trädde fram inför HERREN; och Åklagaren kom också med bland dem och trädde fram inför HERREN.
Job 2:2  Då frågade HERREN Åklagaren: "Varifrån kommer du?" Åklagaren svarade HERREN och sade: "Från en vandring utöver jorden och från en färd omkring på den."
Job 2:3  Då sade HERREN till Åklagaren: "Har du givit akt på min tjänare Job? Ty på jorden finnes icke hans like i ostrafflighet och redlighet, ingen som så fruktar Gud och flyr det onda; och ännu håller han fast vid sin ostrafflighet. Så har du då uppeggat mig mot honom till att utan sak fördärva honom."
Job 2:4  Åklagaren svarade HERREN och sade: "Hud för hud; allt vad man äger giver man ju för att själv slippa undan.
Job 2:5  Men räck ut din hand och kom vid hans kött och ben; förvisso skall han då mitt i ansiktet tala förgripliga ord mot dig."
Job 2:6  HERREN sade till Åklagaren: "Välan, han vare given i din hand; allenast hans liv må du skona."
Job 2:7  Så gick Åklagaren bort ifrån HERRENS ansikte och slog Job med svåra bulnader, ifrån fotbladet ända till hjässan.
Job 2:8  Och han tog sig en lerskärva att skrapa sig med, där han satt mitt i askan.
Job 2:9  Då sade hans hustru till honom: "Håller du ännu fast vid din ostrafflighet? Tala fritt ut om Gud, och dö."
Job 2:10  Man han svarade henne: "Du talar såsom en dåraktig kvinna skulle tala. Om vi taga emot det goda av Gud, skola vi då icke också taga emot det onda?" Vid allt detta syndade Job icke med sina läppar.
Job 2:11  Men tre vänner till Job fingo höra om alla de olyckor som hade träffat honom, och de kommo så, var och en från sin ort; Elifas från Teman, Bildad från Sua och Sofar från Naama. Och de avtalade med varandra att de skulle begiva sig åstad för att ömka honom och trösta honom.
Job 2:12  Men när de, ännu på avstånd, lyfte upp sina ögon och sågo att de icke mer kunde känna igen honom, brusto de ut i gråt och revo sönder sina mantlar och kastade stoft mot himmelen, ned över sina huvuden.
Job 2:13  Sedan sutto de med honom på jorden i sju dagar och sju nätter, utan att någon av dem talade ett ord till honom, eftersom de sågo att hans plåga var mycket stor.
Job 3:1  Därefter upplät Job sin mun och förbannade sin födelsedag;
Job 3:2  Job tog till orda och sade:
Job 3:3  Må den dag utplånas, på vilken jag föddes, och den natt som sade: "Ett gossebarn är avlat."
Job 3:4  Må den dagen vändas i mörker, må Gud i höjden ej fråga efter den och intet dagsljus lysa däröver.
Job 3:5  Mörkret och dödsskuggan börde den åter, molnen lägre sig över den; förskräcke den allt som kan förmörka en dag.
Job 3:6  Den natten må gripas av tjockaste mörker; ej må den få fröjda sig bland årets dagar, intet rum må den finna inom månadernas krets.
Job 3:7  Ja, ofruktsam blive den natten, aldrig höje sig jubel under den.
Job 3:8  Må den förbannas av dem som besvärja dagar, av dem som förmå mana upp Leviatan.
Job 3:9  Må dess grynings stjärnor förmörkas, efter ljus må den bida, utan att det kommer, morgonrodnadens ögonbryn må den aldrig få se;
Job 3:10  eftersom den ej tillslöt dörrarna till min moders liv, ej lät olyckan förbliva dold för mina ögon.
Job 3:11  Varför fick jag ej dö strax i modersskötet, förgås vid det jag kom ut ur min moders liv?
Job 3:12  Varför funnos knän mig till mötes, och varför bröst, där jag fick di?
Job 3:13  Hade så icke skett, låge jag nu i ro, jag finge då sova, jag njöte då min vila,
Job 3:14  vid sidan av konungar och rådsherrar i landet, män som byggde sig palatslika gravar,
Job 3:15  ja, vid sidan av furstar som voro rika på guld och hade sina hus uppfyllda av silver;
Job 3:16  eller vore jag icke till, lik ett nedgrävt foster, lik ett barn som aldrig fick se ljuset.
Job 3:17  Där hava ju de ogudaktiga upphört att rasa, där få de uttröttade komma till vila;
Job 3:18  där hava alla fångar fått ro, de höra där ingen pådrivares röst.
Job 3:19  Små och stora äro där varandra lika, trälen har där blivit fri ifrån sin herre.
Job 3:20  Varför skulle den olycklige skåda ljuset? Ja, varför gives liv åt dem som plågas så bittert,
Job 3:21  åt dem som vänta efter döden, utan att den kommer, och spana därefter mer än efter någon skatt,
Job 3:22  åt dem som skulle glädjas - ja, intill jubel - och fröjda sig, allenast de funne sin grav;
Job 3:23  varför åt en man vilkens väg är höljd i mörker, åt en man så kringstängd av Gud?
Job 3:24  Suckan har ju blivit mitt dagliga bröd, och såsom vatten strömma mina klagorop.
Job 3:25  ty det som ingav mig förskräckelse, det drabbar mig nu, och vad jag fruktade för, det kommer över mig.
Job 3:26  Jag får ingen rast, ingen ro, ingen vila; ångest kommer över mig.
Job 4:1  Därefter tog Elifas från Teman till orda och sade:
Job 4:2  Misstycker du, om man dristar tala till dig? Vem kan hålla tillbaka sina ord?
Job 4:3  Se, många har du visat till rätta, och maktlösa händer har du stärkt;
Job 4:4  dina ord hava upprättat den som stapplade, och åt vacklande knän har du givit kraft.
Job 4:5  Men nu, då det gäller dig själv, bliver du otålig, när det är dig det drabbar, förskräckes du.
Job 4:6  Skulle då icke din gudsfruktan vara din tillförsikt och dina vägars ostrafflighet ditt hopp?
Job 4:7  Tänk efter: när hände det att en oskyldig fick förgås? och var skedde det att de redliga måste gå under?
Job 4:8  Nej, så har jag sett det gå, att de som plöja fördärv och de som utså olycka, de skörda och sådant;
Job 4:9  för Guds andedräkt förgås de och för en fnysning av hans näsa försvinna de.
Job 4:10  Ja, lejonets skri och rytarens röst måste tystna, och unglejonens tänder brytas ut;
Job 4:11  Det gamla lejonet förgås, ty det finner intet rov, och lejoninnans ungar bliva förströdda.
Job 4:12  Men till mig smög sakta ett ord, mitt öra förnam det likasom en viskning,
Job 4:13  När tankarna svävade om vid nattens syner och sömnen föll tung på människorna,
Job 4:14  då kom en förskräckelse och bävan över mig, med rysning fyllde den alla ben i min kropp.
Job 4:15  En vindpust for fram över mitt ansikte, därvid reste sig håren på min kropp.
Job 4:16  Och något trädde inför mina ögon, en skepnad vars form jag icke skönjde; och jag hörde en susning och en röst:
Job 4:17  "Kan då en människa hava rätt mot Gud eller en man vara ren inför sin skapare?
Job 4:18  Se, ej ens på sina tjänare kan han förlita sig, jämväl sina änglar måste han tillvita fel;
Job 4:19  huru mycket mer då dem som bo i hyddor av ler, dem som hava sin grundval i stoftet! De krossas sönder så lätt som mal;
Job 4:20  när morgon har bytts till afton, ligga de slagna; innan man aktar därpå, hava de förgåtts för alltid.
Job 4:21  Ja, deras hyddas fäste ryckes bort för dem, oförtänkt måste de dö."
Job 5:1  Ropa fritt; vem finnes, som svarar dig, och till vilken av de heliga kan du vända dig?
Job 5:2  Se, dåren dräpes av sin grämelse, och den fåkunnige dödas av sin bitterhet.
Job 5:3  Jag såg en dåre, fast var han rotad, men plötsligt måste jag ropa ve över hans boning.
Job 5:4  Ty hans barn gå nu fjärran ifrån frälsning, de förtrampas i porten utan räddning.
Job 5:5  Av hans skörd äter vem som är hungrig, den rövas bort, om och hägnad med törnen; efter hans rikedom gapar ett giller.
Job 5:6  Ty icke upp ur stoftet kommer fördärvet, ej ur marken skjuter olyckan upp;
Job 5:7  nej, människan varder född till olycka, såsom eldgnistor måste flyga mot höjden.
Job 5:8  Men vore det nu jag, så sökte jag nåd hos Gud, åt Gud hemställde jag min sak,
Job 5:9  åt honom som gör stora och outrannsakliga ting, under, flera än någon kan räkna,
Job 5:10  åt honom som låter regnet falla på jorden och sänder vatten ned över markerna,
Job 5:11  när han vill upphöja de ringa och förhjälpa de sörjande till frälsning.
Job 5:12  Han är den som gör de klokas anslag om intet, så att deras händer intet uträtta med förnuft;
Job 5:13  han fångar de visa i deras klokskap och låter de illfundiga förhasta sig i sina rådslag:
Job 5:14  mitt på dagen råka de ut för mörker och famla mitt i ljuset, likasom vore det natt.
Job 5:15  Så frälsar han från deras tungors svärd, han frälsar den fattige ur den övermäktiges hand.
Job 5:16  Den arme kan så åter hava ett hopp, och orättfärdigheten måste tillsluta sin mun.
Job 5:17  Ja, säll är den människa som Gud agar; den Allsmäktiges tuktan må du icke förkasta.
Job 5:18  Ty om han och sargar, så förbinder han ock, om han slår, så hela ock hans händer.
Job 5:19  Sex gånger räddar han dig ur nöden, ja, sju gånger avvändes olyckan från dig.
Job 5:20  I hungerstid förlossar han dig från döden och i krig undan svärdets våld.
Job 5:21  När tungor svänga gisslet, gömmes du undan; du har intet att frukta, när förhärjelse kommer.
Job 5:22  Ja, åt förhärjelse och dyr tid kan du då le, för vilddjur behöver du ej heller känna fruktan;
Job 5:23  ty med markens stenar står du i förbund, och med djuren på marken har du ingått fred.
Job 5:24  Och du får se huru din hydda står trygg; när du synar din boning, saknas intet däri.
Job 5:25  Du får ock se huru din ätt förökas, huru din avkomma bliver såsom markens örter.
Job 5:26  I graven kommer du, när du har hunnit din mognad, såsom sädesskylen bärgas, då dess tid är inne.
Job 5:27  Se, detta hava vi utrannsakat, och så är det; hör därpå och betänk det väl.
Job 6:1  Då tog Job till orda och sade:
Job 6:2  Ack att min grämelse bleve vägd och min olycka lagd jämte den på vågen!
Job 6:3  Se, tyngre är den nu än havets sand, därför kan jag icke styra mina ord.
Job 6:4  Ty den Allsmäktiges pilar hava träffat mig, och min ande indricker deras gift; ja, förskräckelser ifrån Gud ställa sig upp mot mig.
Job 6:5  Icke skriar vildåsnan, när hon har friskt gräs, icke råmar oxen, då han står vid sitt foder?
Job 6:6  Men vem vill äta den mat som ej har smak eller sälta, och vem finner behag i slemörtens saft?
Job 6:7  Så vägrar nu min själ att komma vid detta, det är för mig en vämjelig spis.
Job 6:8  Ack att min bön bleve hörd, och att Gud ville uppfylla mitt hopp!
Job 6:9  O att det täcktes Gud att krossa mig, att räcka ut sin hand och avskära mitt liv!
Job 6:10  Då funnes ännu för mig någon tröst, jag kunde då jubla, fastän plågad utan förskoning; jag har ju ej förnekat den Heliges ord.
Job 6:11  Huru stor är då min kraft, eftersom jag alltjämt bör hoppas? Och vad väntar mig för ände, eftersom jag skall vara tålig?
Job 6:12  Min kraft är väl ej såsom stenens, min kropp är väl icke av koppar?
Job 6:13  Nej, förvisso gives ingen hjälp för mig, var utväg har blivit mig stängd.
Job 6:14  Den förtvivlade borde ju röna barmhärtighet av sin vän, men se, man övergiver den Allsmäktiges fruktan,
Job 6:15  Mina bröder äro trolösa, de äro såsom regnbäckar, ja, lika bäckarnas rännilar, som snart sina ut,
Job 6:16  som väl kunna gå mörka av vinterns flöden, när snön har fallit och gömt sig i dem,
Job 6:17  men som åter försvinna, när de träffas av hettan, och torka bort ifrån sin plats, då värmen kommer.
Job 6:18  Vägfarande där i trakten vika av till dem, men de finna allenast ödslighet och måste förgås.
Job 6:19  Temas vägfarande skådade dithän, Sabas köpmanståg hoppades på dem;
Job 6:20  men de kommo på skam i sin förtröstan, de sågo sig gäckade, när de hade hunnit ditfram.
Job 6:21  Ja, likaså ären I nu ingenting värda, handfallna stån I av förfäran och förskräckelse.
Job 6:22  Har jag då begärt att I skolen giva mig gåvor, taga av edert gods för att lösa mig ut,
Job 6:23  att I skolen rädda mig undan min ovän, köpa mig fri ur våldsverkares hand?
Job 6:24  Undervisen mig, så vill jag tiga, lären mig att förstå vari jag har farit vilse.
Job 6:25  Gott är förvisso uppriktigt tal, men tillrättavisning av eder, vad båtar den?
Job 6:26  Haven I då i sinnet att hålla räfst med ord, och skall den förtvivlade få tala för vinden?
Job 6:27  Då kasten I väl också lott om den faderlöse, då lären I väl köpslå om eder vän!
Job 6:28  Dock, må det nu täckas eder att akta på mig; icke vill jag ljuga eder mitt i ansiktet.
Job 6:29  Vänden om! Må sådan orätt icke ske; ja, vänden ännu om, ty min sak är rättfärdig!
Job 6:30  Skulle väl orätt bo på min tunga, och min mun, skulle den ej förstå vad fördärvligt är?
Job 7:1  En stridsmans liv lever ju människan på jorden, och hennes dagar äro såsom dagakarlens dagar.
Job 7:2  Hon är lik en träl som flämtar efter skugga, lik en dagakarl som får bida efter sin lön.
Job 7:3  Så har jag fått till arvedel månader av elände; nätter av vedermöda hava blivit min lott.
Job 7:4  Så snart jag har lagt mig, är min fråga: "När skall jag då få stå upp?" Ty aftonen synes mig så lång; jag är övermätt av oro, innan morgonen har kommit.
Job 7:5  Med förruttnelsens maskar höljes min kropp, med en skorpa lik jord; min hud skrymper samman och faller sönder.
Job 7:6  Mina dagar fly snabbare än vävarens spole; de försvinna utan något hopp.
Job 7:7  Tänk därpå att mitt liv är en fläkt, att mitt öga icke mer skall få se någon lycka.
Job 7:8  Den nu ser mig, hans öga skall ej vidare skåda mig; bäst din blick vilar på mig, är jag icke mer.
Job 7:9  Såsom ett moln som har försvunnit och gått bort, så är den som har farit ned i dödsriket; han kommer ej åter upp därifrån.
Job 7:10  Aldrig mer vänder han tillbaka till sitt hus, och hans plats vet icke av honom mer.
Job 7:11  Därför vill jag nu icke lägga band på min mun, jag vill taga till orda i min andes ångest, jag vill klaga i min själs bedrövelse.
Job 7:12  Icke är jag väl ett hav eller ett havsvidunder, så att du måste sätta ut vakt mot mig?
Job 7:13  När jag hoppas att min bädd skall trösta mig, att mitt läger skall lindra mitt bekymmer,
Job 7:14  då förfärar du mig genom drömmar, och med syner förskräcker du mig.
Job 7:15  Nej, hellre vill jag nu bliva kvävd, hellre dö än vara blott knotor!
Job 7:16  Jag är led vid detta; aldrig kommer jag åter till liv. Låt mig vara; mina dagar äro ju fåfänglighet.
Job 7:17  Vad är då en människa, att du gör så stor sak av henne, aktar på henne så noga,
Job 7:18  synar henne var morgon, prövar henne vart ögonblick?
Job 7:19  Huru länge skall det dröja, innan du vänder din blick ifrån mig, lämnar mig i fred ett litet andetag?
Job 7:20  Om jag än har syndar, vad skadar jag därmed dig, du människornas bespejare? Varför har du satt mig till ett mål för dina angrepp och låtit mig bliva en börda för mig själv?
Job 7:21  Varför vill du icke förlåta mig min överträdelse, icke tillgiva mig min missgärning? Nu måste jag ju snart gå till vila i stoftet; om du söker efter mig, så är jag icke mer.
Job 8:1  Därefter tog Bildad från Sua till orda och sade:
Job 8:2  Huru länge vill du hålla på med sådant tal och låta din muns ord komma såsom en väldig storm?
Job 8:3  Skulle väl Gud kunna kränka rätten? Kan den Allsmäktige kränka rättfärdigheten?
Job 8:4  Om dina barn hava syndat mot honom och han gav dem i sina överträdelsers våld,
Job 8:5  så vet, att om du själv söker Gud och beder till den Allsmäktige om misskund,
Job 8:6  då, om du är ren och rättsinnig, ja, då skall han vakna upp till din räddning och upprätta din boning, så att du bor där i rättfärdighet;
Job 8:7  och så skall din första tid synas ringa, då nu din sista tid har blivit så stor.
Job 8:8  Ty fråga framfarna släkten, och akta på vad fäderna hava utrönt
Job 8:9  - vi själva äro ju från i går och veta intet, en skugga äro våra dagar på jorden;
Job 8:10  men de skola undervisa dig och säga dig det, ur sina hjärtan skola de hämta fram svar:
Job 8:11  "Icke kan röret växa högt, där marken ej är sank, eller vassen skjuta i höjden, där vatten ej finnes?
Job 8:12  Nej, bäst den står grön, ej mogen för skörd, måste den då vissna, före allt annat gräs.
Job 8:13  Så går det alla som förgäta Gud; den gudlöses hopp måste varda om intet.
Job 8:14  Ty hans tillförsikt visar sig bräcklig och hans förtröstan lik spindelns väv.
Job 8:15  Han förlitar sig på sitt hus, men det har intet bestånd; han tryggar sig därvid, men det äger ingen fasthet.
Job 8:16  Lik en frodig planta växer han i solens sken, ut över lustgården sträcka sig hans skott;
Job 8:17  kring stenröset slingra sig hans rötter, mellan stenarna bryter han sig fram.
Job 8:18  Men när så Gud rycker bort honom från hans plats, då förnekar den honom: 'Aldrig har jag sett dig.'
Job 8:19  Ja, så går det med hans levnads fröjd, och ur mullen få andra växa upp."
Job 8:20  Se, Gud föraktar icke den som är ostrafflig, han håller ej heller de onda vid handen.
Job 8:21  Så bida då, till dess han fyller din mun med löje och dina läppar med jubel.
Job 8:22  De som hata dig varda då höljda med skam, och de ogudaktigas hyddor skola ej mer vara till.
Job 9:1  Därefter tog Job till orda och sade:
Job 9:2  Ja, förvisso vet jag att så är; huru skulle en människa kunna hava rätt mot Gud?
Job 9:3  Vill han gå till rätta med henne, så kan hon ej svara honom på en sak bland tusen.
Job 9:4  Han som är så vis i förstånd och så väldig i kraft, vem kan trotsa honom och dock slippa undan;
Job 9:5  honom som oförtänkt flyttar bort berg och omstörtar dem i sin vrede;
Job 9:6  honom som kommer jorden att vackla från sin plats, och dess pelare bäva därvid;
Job 9:7  honom som befaller solen, så går hon icke upp, och som sätter stjärnorna under försegling;
Job 9:8  honom som helt allena spänner ut himmelen och skrider fram över havets toppar;
Job 9:9  honom som har gjort Karlavagnen och Orion, Sjustjärnorna och söderns Stjärngemak;
Job 9:10  honom som gör stora och outrannsakliga ting och under, flera än någon kan räkna?
Job 9:11  Se, han far förbi mig, innan jag hinner att se det, han drager framom mig, förrän jag bliver honom varse.
Job 9:12  Se, han griper sitt rov; vem kan hindra honom? Vem kan säga till honom: "Vad gör du?"
Job 9:13  Gud, han ryggar icke sin vrede; för honom har Rahabs följe måst böja sig;
Job 9:14  huru skulle jag då våga svara honom, välja ut ord till att tala med honom?
Job 9:15  Nej, om jag än hade rätt, tordes jag dock ej svara; jag finge anropa min motpart om misskund.
Job 9:16  Och om han än svarade mig på mitt rop, så kunde jag ej tro att han lyssnade till min röst.
Job 9:17  Ty med storm hemsöker han mig och slår mig med sår på sår, utan sak.
Job 9:18  Han unnar mig icke att hämta andan; nej, med bedrövelser mättar han mig.
Job 9:19  Gäller det försteg i kraft: "Välan, jag är redo!", gäller det rätt: "Vem ställer mig till ansvar?"
Job 9:20  Ja, hade jag än rätt, så dömde min mun mig skyldig; vore jag än ostrafflig, så läte han mig synas vrång.
Job 9:21  Men ostrafflig är jag! Jag aktar ej mitt liv, jag frågar icke efter, om jag får leva.
Job 9:22  Det må gå som det vill, nu vare det sagt: han förgör den ostrafflige jämte den ogudaktige.
Job 9:23  Om en landsplåga kommer med plötslig död, så bespottar han de oskyldigas förtvivlan.
Job 9:24  Jorden är given i de ogudaktigas hand, och täckelse sätter han för dess domares ögon. Är det ej han som gör det, vem är det då?
Job 9:25  Min dagar hasta undan snabbare än någon löpare, de fly bort utan att hava sett någon lycka;
Job 9:26  de ila åstad såsom en farkost av rör, såsom en örn, när han störtar sig ned på sitt byte.
Job 9:27  Om jag än besluter att förgäta mitt bekymmer, att låta min sorgsenhet fara och göra mig glad,
Job 9:28  Så måste jag dock bäva för alla mina kval; jag vet ju att du icke skall döma mig fri.
Job 9:29  Nej, såsom skyldig måste jag stå där; varför skulle jag då göra mig fåfäng möda?
Job 9:30  Om jag än tvår mig i snö och renar mina händer i lutsalt,
Job 9:31  så skall du dock sänka mig ned i pölen, så att mina kläder måste vämjas vid mig.
Job 9:32  Ty han är ej min like, så att jag vågar svara honom, ej en sådan, att vi kunna gå till doms med varandra;
Job 9:33  ingen skiljeman finnes mellan oss, ingen som har myndighet över oss båda.
Job 9:34  Må han blott vända av från mig sitt ris, och må fruktan för honom ej förskräcka mig;
Job 9:35  då skall jag tala utan att rädas för honom, ty jag vet med min själv att jag icke är en sådan.
Job 10:1  Min själ är led vid livet. Jag vill giva fritt lopp åt min klagan, jag vill tala i min själs bedrövelse.
Job 10:2  Jag vill säga till Gud: Döm mig icke skyldig; låt mig veta varför du söker sak mot mig.
Job 10:3  Anstår det dig att öva våld, att förkasta dina händers verk, medan du låter ditt ljus lysa över de ogudaktigas rådslag?
Job 10:4  Har du då ögon som en varelse av kött, eller ser du såsom människor se?
Job 10:5  Är din ålder som en människas ålder, eller äro dina år såsom en mans tider,
Job 10:6  eftersom du letar efter missgärning hos mig och söker att hos mig finna synd,
Job 10:7  du som dock vet att jag icke är skyldig, och att ingen finnes, som kan rädda ur din hand?
Job 10:8  Dina händer hava danat och gjort mig, helt och i allo; och nu fördärvar du mig!
Job 10:9  Tänk på huru du formade mig såsom lera; och nu låter du mig åter varda till stoft!
Job 10:10  Ja, du utgöt mig såsom mjölk, och såsom ostämne lät du mig stelna.
Job 10:11  Med hud och kött beklädde du mig, av ben och senor vävde du mig samman.
Job 10:12  Liv och nåd beskärde du mig, och genom din vård bevarades min ande.
Job 10:13  Men därvid gömde du i ditt hjärta den tanken, jag vet att du hade detta i sinnet:
Job 10:14  om jag syndade, skulle du vakta på mig och icke lämna min missgärning ostraffad.
Job 10:15  Ve mig, om jag befunnes vara skyldig! Men vore jag än oskyldig, så finge jag ej lyfta mitt huvud, jag skulle mättas av skam och skåda min ofärd.
Job 10:16  Höjde jag det likväl, då skulle du såsom ett lejon jaga mig och alltjämt bevisa din undermakt på mig.
Job 10:17  Nya vittnen mot mig skulle du då föra fram och alltmer låta mig känna din förtörnelse; med skaror efter skaror skulle du ansätta mig.
Job 10:18  Varför lät du mig då komma ut ur modersskötet? Jag borde hava förgåtts, innan något öga såg mig,
Job 10:19  hava blivit såsom hade jag aldrig varit till; från moderlivet skulle jag hava förts till graven.
Job 10:20  Kort är ju min tid; må han då låta mig vara, lämna mig i fred, så att jag får en flyktig glädje,
Job 10:21  innan jag går hädan, för att aldrig komma åter, bort till mörkrets och dödsskuggans land,
Job 10:22  till det land vars dunkel är såsom djupa vatten, dit där dödsskugga och förvirring råder, ja, där dagsljuset självt är såsom djupa vatten.
Job 11:1  Därefter tog Sofar från Naama till orda och sade:
Job 11:2  Skall sådant ordflöde bliva utan svar och en så stortalig man få rätt?
Job 11:3  Skall ditt lösa tal nödga män till tystnad, så att du får bespotta, utan att någon kommer dig att blygas?
Job 11:4  Och skall du så få säga: "Vad jag lär är rätt, och utan fläck har jag varit inför dina ögon"?
Job 11:5  Nej, om allenast Gud ville tala och upplåta sina läppar till att svara dig,
Job 11:6  om han ville uppenbara dig sin visdoms lönnligheter, huru han äger förstånd, ja, i dubbelt mått, då insåge du att Gud, dig till förmån, har lämnat åt glömskan en del av din missgärning.
Job 11:7  Men kan väl du utrannsaka Guds djuphet eller fatta den Allsmäktiges fullkomlighet?
Job 11:8  Hög såsom himmelen är den - vad kan du göra? djupare än dödsriket - vad kan du förstå?
Job 11:9  Dess längd sträcker sig vidare än jorden, och i bredd överträffar den havet.
Job 11:10  När han vill fara fram och spärra någon inne eller kalla någon till doms, vem kan då hindra honom?
Job 11:11  Han är ju den som känner lögnens män, fördärv upptäcker han, utan att leta därefter.
Job 11:12  Men lika lätt kan en dåraktig man få förstånd, som en vildåsnefåle kan födas till människa.
Job 11:13  Om du nu rätt bereder ditt hjärta och uträcker dina händer till honom,
Job 11:14  om du skaffar bort det fördärv som kan låda vid din hand och ej låter orättfärdighet bo i dina hyddor,
Job 11:15  ja, då får du upplyfta ditt ansikte utan skam, du står fast och har intet att frukta.
Job 11:16  Ja, då skall du förgäta din olycka, blott minnas den såsom vatten som har förrunnit.
Job 11:17  Ditt liv skall då stråla klarare än middagens sken; och kommer mörker på, så är det som en gryning till morgon.
Job 11:18  Du kan då vara trygg, ty du äger ett hopp; du spanar omkring dig och går sedan trygg till vila.
Job 11:19  Ja, du får då ligga i ro, utan att någon förskräcker dig, och många skola söka din ynnest.
Job 11:20  Men de ogudaktigas ögon skola försmäkta; ingen tillflykt skall mer finnas för dem, och deras hopp skall vara att få giva upp andan.
Job 12:1  Därefter tog Job till orda och sade:
Job 12:2  Ja, visst ären I det rätta folket, och med eder kommer visheten att dö ut!
Job 12:3  Dock, jämväl jag har förstånd så gott som I, icke står jag tillbaka för eder; ty vem är den som ej begriper slikt?
Job 12:4  Så måste jag då vara ett åtlöje för min vän, jag som fick svar, så snart jag ropade till Gud; man ler åt en som är rättfärdig och ostrafflig!
Job 12:5  Ja, med förakt ses olyckan av den som står säker; förakt väntar dem vilkas fötter vackla.
Job 12:6  Men förhärjares hyddor åtnjuta frid, och trygghet få sådana som trotsa Gud, de som hava sin gud i sin hand.
Job 12:7  Men fråga du boskapen, den må undervisa dig, och fåglarna under himmelen, de må upplysa dig;
Job 12:8  eller tala till jorden, hon må undervisa dig, fiskarna i havet må giva dig besked.
Job 12:9  Vem kan icke lära genom allt detta att det är HERRENS hand som har gjort det?
Job 12:10  I hans han är ju allt levandes själ och alla mänskliga varelsers anda.
Job 12:11  Skall icke öra pröva orden, likasom munnen prövar matens smak?
Job 12:12  Vishet tillkommer ju de gamle och förstånd dem som länge hava levat.
Job 12:13  Hos Honom finnes vishet och makt, hos honom råd och förstånd.
Job 12:14  Se, vad han river ned, det bygges ej upp; för den han spärrar inne kan ingen upplåta.
Job 12:15  Han håller vattnen tillbaka - se, se då bliver där torrt, han släpper dem lösa, då fördärva de landet.
Job 12:16  Hos honom är kraft och klokhet, den förvillade och förvillaren äro båda i hans hand.
Job 12:17  Rådsherrar utblottar han, han för dem i landsflykt, och domare gör han till dårar.
Job 12:18  Han upplöser konungars välde och sätter fångbälte om deras höfter.
Job 12:19  Präster utblottar han, han för dem i landsflykt, och de säkrast rotade kommer han på fall.
Job 12:20  Välbetrodda män berövar han målet och avhänder de äldste deras insikt.
Job 12:21  Han utgjuter förakt över furstar och lossar de starkes gördel.
Job 12:22  Han blottar djupen, så att de ej höljas av mörker, dödsskuggan drager han fram i ljuset.
Job 12:23  Han låter folkslag växa till - och förgör dem; han utvidgar deras gränser, men för dem sedan bort.
Job 12:24  Stamhövdingar i landet berövar han förståndet, han leder dem vilse i väglösa ödemarker.
Job 12:25  De famla i mörkret och hava intet ljus, han kommer dem att ragla såsom druckna.
Job 13:1  Ja, alltsammans har mitt öga sett, mitt öra har hört det och nogsamt givit akt.
Job 13:2  Vad I veten, det vet också jag; icke står jag tillbaka för eder.
Job 13:3  Men till den Allsmäktige vill jag nu tala, det lyster mig att gå till rätta med Gud.
Job 13:4  Dock, I ären män som spinna ihop lögn, allasammans hopsätten I fåfängligt tal.
Job 13:5  Om I ändå villen alldeles tiga! Det kunde tillräknas eder som vishet.
Job 13:6  Hören nu likväl mitt klagomål, och akten på mina läppars gensagor.
Job 13:7  Viljen I försvara Gud med orättfärdigt tal och honom till förmån bruka oärligt tal?
Job 13:8  Skolen I visa eder partiska för honom eller göra eder till sakförare för Gud?
Job 13:9  Icke kan sådant ändas väl, när han håller räfst med eder? Eller kunnen I gäckas med honom, såsom man kan gäckas med en människa?
Job 13:10  Nej, förvisso skall han straffa eder, om I visen en hemlig partiskhet.
Job 13:11  Sannerligen, hans majestät skall då förskräcka eder, och fruktan för honom skall falla över eder.
Job 13:12  Edra tänkespråk skola då bliva visdomsord av aska, edra försvarsverk varda såsom vallar av ler.
Job 13:13  Tigen nu för min, så skall jag tala, gånge så över mig vad det vara må.
Job 13:14  Ja, huru det än går, vill jag fatta mitt kött mellan tänderna och taga min själ i min hand.
Job 13:15  Må han dräpa mig, jag hoppas intet annat; min vandel vill jag ändå hålla fram inför honom.
Job 13:16  Redan detta skall lända mig till frälsning, ty ingen gudlös dristar komma inför honom.
Job 13:17  Hören, hören då mina ord, och låten min förklaring tränga in i edra öron.
Job 13:18  Se, här lägger jag saken fram; jag vet att jag skall befinnas hava rätt.
Job 13:19  Eller gives det någon som kan vederlägga mig? Ja, då vill jag tiga - och dö.
Job 13:20  Allenast två ting må du ej göra mot mig, så behöver jag ej dölja mig inför ditt ansikte:
Job 13:21  din hand må du ej låta komma mig när, och fruktan för dig må icke förskräcka mig.
Job 13:22  Sedan må du åklaga, och jag vill svara, eller ock skall jag tala, och du må gendriva mig.
Job 13:23  Huru är det alltså med mina missgärningar och synder? Låt mig få veta min överträdelse och synd.
Job 13:24  Varför döljer du ditt ansikte och aktar mig såsom din fiende?
Job 13:25  Vill du skrämma ett löv som drives av vinden, vill du förfölja ett borttorkat strå?
Job 13:26  Du skriver ju bedrövelser på min lott och giver mig till arvedel min ungdoms missgärningar;
Job 13:27  du sätter mina fötter i stocken, du vaktar på alla vägar, för mina fotsulor märker du ut stegen.
Job 13:28  Och detta mot en som täres bort lik murket trä, en som liknar en klädnad sönderfrätt av mal!
Job 14:1  Människan, av kvinna född, lever en liten tid och mättas av oro;
Job 14:2  lik ett blomster växer hon upp och vissnar bort, hon flyr undan såsom skuggan och har intet bestånd.
Job 14:3  Och till att vakta på en sådan upplåter du dina ögon, ja, du drager mig till doms inför dig.
Job 14:4  Som om en ren skulle kunna framgå av en oren! Sådant kan ju aldrig ske.
Job 14:5  Äro nu människans dagar oryggligt bestämda, hennes månaders antal fastställt av dig, har du utstakat en gräns som hon ej kan överskrida,
Job 14:6  vänd då din blick ifrån henne och unna henne ro, låt henne njuta en dagakarls glädje av sin dag.
Job 14:7  För ett träd finnes ju kvar något hopp; hugges det än ned, kan det åter skjuta skott, och telningar behöva ej fattas därpå.
Job 14:8  Om än dess rot tynar hän i jorden och dess stubbe dör bort i mullen,
Job 14:9  så kan det grönska upp genom vattnets ångor och skjuta grenar lik ett nyplantat träd.
Job 14:10  Men om en man dör, så ligger han där slagen; om en människa har givit upp andan, var finnes hon då mer?
Job 14:11  Såsom när vattnet har förrunnit ur en sjö, och såsom när en flod har sinat bort och uttorkat,
Job 14:12  så ligger mannen där och står ej mer upp, han vaknar icke åter, så länge himmelen varar; aldrig väckes han upp ur sin sömn.
Job 14:13  Ack, att du ville gömma mig i dödsriket, fördölja mig, till dess din vrede hade upphört, staka ut för mig en tidsgräns och sedan tänka på mig -
Job 14:14  fastän ju ingen kan få liv, när han en gång är död! Då skulle jag hålla min stridstid ut, ända till dess att min avlösning komme.
Job 14:15  Du skulle då ropa på mig, och jag skulle svara dig; efter dina händers verk skulle du längta;
Job 14:16  ja, du skulle då räkna mina steg, du skulle ej akta på min synd.
Job 14:17  I en förseglad pung låge då min överträdelse, och du överskylde min missgärning.
Job 14:18  Men såsom själva berget faller och förvittrar, och såsom klippan flyttas ifrån sin plats,
Job 14:19  såsom stenar nötas sönder genom vattnet, och såsom mullen sköljes bort av dess flöden, så gör du ock människans hopp om intet.
Job 14:20  Du slår henne ned för alltid, och hon far hädan; du förvandlar hennes ansikte och driver henne bort.
Job 14:21  Om hennes barn komma till ära, så känner hon det icke; om de sjunka ned till ringhet, så aktar hon dock ej på dem.
Job 14:22  Hennes kropp känner blott sin egen plåga, hennes själ blott den sorg hon själv får förnimma.
Job 15:1  Därefter tog Elifas från Teman till orda och sade:
Job 15:2  Skall en vis man tala så i vädret och fylla upp sitt bröst med östanvind?
Job 15:3  Skall han försvara sin sak med haltlöst tal, med ord som ingenting bevisa?
Job 15:4  Än mer, du gör gudsfruktan om intet och kommer med klagolåt inför Gud.
Job 15:5  Ty din ondska lägger dig orden i munnen, och ditt behag står till illfundigt tal.
Job 15:6  Så dömes du nu skyldig av din mun, ej av mig, dina egna läppar vittna emot dig.
Job 15:7  Var du den första människa som föddes, och fick du liv, förrän höjderna funnos?
Job 15:8  Blev du åhörare i Guds hemliga råd och fick så visheten i ditt våld?
Job 15:9  Vad vet du då, som vi icke veta? Vad förstår du, som ej är oss kunnigt?
Job 15:10  Gråhårsman och åldring finnes också bland oss, ja, en som övergår din fader i ålder.
Job 15:11  Försmår du den tröst som Gud har att bjuda, och det ord som i saktmod talas med dig?
Job 15:12  Vart föres du hän av ditt sinne, och varför välva dina ögon så,
Job 15:13  i det du vänder ditt raseri mot Gud och öser ut ord ur din mun?
Job 15:14  Vad är en människa, att hon skulle vara ren? Vad en av kvinna född, att han skulle vara rättfärdig?
Job 15:15  Se, ej ens på sina heliga kan han förlita sig, och himlarna äro icke rena inför hans ögon;
Job 15:16  huru mycket mindre då den som är ond och fördärvad, den man som läskar sig med orättfärdighet såsom med vatten!
Job 15:17  Jag vill kungöra dig något, så hör nu mig; det som jag har skådat vill jag förtälja,
Job 15:18  vad visa män hava gjort kunnigt, lagt fram såsom ett arv ifrån sina fäder,
Job 15:19  ifrån dem som allena fingo landet till gåva, och bland vilka ingen främling ännu hade trängt in:
Job 15:20  Den ogudaktige har ångest i alla sina dagar, under de år, helt få, som beskäras en våldsverkare.
Job 15:21  Skräckröster ljuda i hans öron; när han är som tryggast, kommer förhärjaren över honom.
Job 15:22  Han har intet hopp om räddning ur mörkret, ty svärdet lurar på honom.
Job 15:23  Såsom flykting söker han sitt bröd: var är det? Han förnimmer att mörkrets dag är för handen.
Job 15:24  Ångest och trångmål förskräcka honom, han nedslås av dem såsom av en stridsrustad konung.
Job 15:25  Ty mot Gud räckte han ut sin hand, och mot den Allsmäktige förhävde han sig;
Job 15:26  han stormade mot honom med trotsig hals, med sina sköldars ryggar i sluten hop;
Job 15:27  han höljde sitt ansikte med fetma och samlade hull på sin länd;
Job 15:28  han bosatte sig i städer, dömda till förstöring, i hus som ej fingo bebos, ty till stenhopar voro de bestämda.
Job 15:29  Därför bliver han ej rik, och hans gods består ej, hans skördar luta ej tunga mot jorden.
Job 15:30  Han kan icke undslippa mörkret; hans telningar skola förtorka av hetta, och själv skall han förgås genom Guds muns anda.
Job 15:31  I sin förvillelse må han ej lita på vad fåfängligt är, ty fåfänglighet måste bliva hans lön.
Job 15:32  I förtid skall hans mått varda fyllt, och hans krona skall ej grönska mer.
Job 15:33  Han bliver lik ett vinträd som i förtid mister sina druvor, lik ett olivträd som fäller sina blommor.
Job 15:34  Ty den gudlöses hus förbliver ofruktsamt, såsom eld förtär hyddor där mutor tagas.
Job 15:35  Man går havande med olycka och föder fördärv; den livsfrukt man alstrar är ett sviket hopp.
Job 16:1  Därefter tog Job till orda och sade:
Job 16:2  Över nog har jag fått höra av sådant; usla tröstare ären I alla.
Job 16:3  Är det nu slut på detta tal i vädret, eller eggar dig ännu något till gensvar?
Job 16:4  Jag kunde väl ock tala, jag såsom I; ja, jag ville att I voren i mitt ställe! Då kunde jag hopsätta ord mot eder och skaka mot eder mitt huvud till hån.
Job 16:5  Med munnen kunde jag då styrka eder och med läpparnas ömkan bereda eder lindring.
Job 16:6  Om jag nu talar, så lindras därav ej min plåga; och tiger jag, icke släpper den mig ändå.
Job 16:7  Nej, nu har all min kraft blivit tömd; du har ju förött hela mitt hus.
Job 16:8  Och att du har hemsökt mig, det gäller såsom vittnesbörd; min sjukdom får träda upp och tala mot mig.
Job 16:9  I vrede söndersliter och ansätter man mig, man biter sina tänder samman emot mig; ja, min ovän vässer mot mig sina blickar.
Job 16:10  Man spärrar upp munnen mot mig, smädligt slår man mig på mina kinder; alla rota sig tillsammans emot mig.
Job 16:11  Gud giver mig till pris åt orättfärdiga människor och kastar mig i de ogudaktigas händer.
Job 16:12  Jag satt i god ro, då krossade han mig; han grep mig i nacken och slog mig i smulor. Han satte mig upp till ett mål för sina skott;
Job 16:13  från alla sidor träffa mig hans pilar, han genomborrar mina njurar utan förskoning, min galla gjuter han ut på jorden.
Job 16:14  Han bryter ned mig med stöt på stöt, han stormar emot mig såsom en kämpe.
Job 16:15  Säcktyg bär jag hopfäst över min hud, och i stoftet har jag måst sänka mitt horn,
Job 16:16  Mitt anlete är glödande rött av gråt, och på mina ögonlock är dödsskugga lägrad.
Job 16:17  Och detta, fastän våld ej finnes i mina händer, och fastän min bön är ren!
Job 16:18  Du jord, överskyl icke mitt blod, och låt för mitt rop ingen vilostad finnas.
Job 16:19  Se, redan nu har jag i himmelen mitt vittne, och i höjden den som skall tala för mig.
Job 16:20  Mina vänner hava mig nu till sitt åtlöje, därför skådar mitt öga med tårar till Gud,
Job 16:21  Ja, må han här skaffa rätt åt en man mot Gud och åt ett människobarn mot dess nästa.
Job 16:22  Ty få äro de år som skola upprinna, innan jag vandrar den väg där jag ej mer kommer åter.
Job 17:1  Min livskraft är förstörd, mina dagar slockna ut, bland gravar får jag min lott.
Job 17:2  Ja, i sanning är jag omgiven av gäckeri, och avoghet får mitt öga ständigt skåda hos dessa!
Job 17:3  Så ställ nu säkerhet och borgen för mig hos dig själv; vilken annan vill giva mig sitt handslag?
Job 17:4  Dessas hjärtan har du ju tillslutit för förstånd, därför skall du icke låta dem triumfera.
Job 17:5  Den som förråder sina vänner till plundring, på hans barn skola ögonen försmäkta.
Job 17:6  Jag är satt till ett ordspråk bland folken; en man som man spottar i ansiktet är jag.
Job 17:7  Därför är mitt öga skumt av grämelse, och mina lemmar äro såsom en skugga allasammans.
Job 17:8  De redliga häpna över sådant, och den oskyldige uppröres av harm mot den gudlöse.
Job 17:9  Men den rättfärdige håller fast vid sin väg, och den som har rena händer bemannar sig dess mer.
Job 17:10  Ja, gärna mån I alla ansätta mig på nytt, jag lär ändå bland eder ej finna någon vis.
Job 17:11  Mina dagar äro förlidna, sönderslitna äro mina planer, vad som var mitt hjärtas begär.
Job 17:12  Men natten vill man göra till dag, ljuset skulle vara nära, nu då mörker bryter in.
Job 17:13  Nej, huru jag än bidar, bliver dödsriket min boning, i mörkret skall jag bädda mitt läger;
Job 17:14  till graven måste jag säga: "Du är min fader", till förruttnelsens maskar: "Min moder", "Min syster".
Job 17:15  Vad bliver då av mitt hopp, ja, mitt hopp, vem får skåda det?
Job 17:16  Till dödsrikets bommar far det ned, då jag nu själv går till vila i stoftet.
Job 18:1  Därefter tog Bildad från Sua till orda och sade:
Job 18:2  Huru länge skolen I gå på jakt efter ord? Kommen till förstånd; sedan må vi talas vid.
Job 18:3  Varför skola vi aktas såsom oskäliga djur, räknas i edra ögon såsom ett förstockat folk?
Job 18:4  Du som i din vrede sliter sönder dig själv, menar du att dör din skull jorden skall bliva öde och klippan flyttas bort från sin plats?
Job 18:5  Nej, den ogudaktiges ljus skall slockna ut, och lågan av hans eld icke giva något sken.
Job 18:6  Ljuset skall förmörkas i hans hydda, och lampan slockna ut för honom.
Job 18:7  Hans väldiga steg skola stäckas, hans egna rådslag bringa honom på fall.
Job 18:8  Ty han rusar med sina fötter in i nätet, försåten lura, där han vandrar fram;
Job 18:9  snaran griper honom om hälen, och gillret tager honom fatt;
Job 18:10  garn till att fånga honom äro lagda på marken och snärjande band på hans stig.
Job 18:11  Från alla sidor ängsla honom förskräckelser, de jaga honom, varhelst han går fram.
Job 18:12  Olyckan vill uppsluka honom, och ofärd står redo, honom till fall.
Job 18:13  Under hans hud frätas hans lemmar bort, ja, av dödens förstfödde bortfrätas hans lemmar.
Job 18:14  Ur sin hydda, som han förtröstar på, ryckes han bort, och till förskräckelsernas konung vandrar han hän.
Job 18:15  I hans hydda får främlingar bo, och svavel utströs över hans boning.
Job 18:16  Nedantill förtorkas hans rötter, och ovantill vissnar hans krona bort.
Job 18:17  Hans åminnelse förgås ifrån jorden, hans namn lever icke kvar i världen.
Job 18:18  Från ljus stötes han ned i mörker och förjagas ifrån jordens krets.
Job 18:19  Utan barn och avkomma bliver han i sitt folk, och ingen i hans boningar skall slippa undan.
Job 18:20  Över hans ofärdsdag häpna västerns folk, och österns män gripas av rysning.
Job 18:21  Ja, så sker det med den orättfärdiges hem, så går det dens hus, som ej vill veta av Gud.
Job 19:1  Därefter tog Job till orda och sade:
Job 19:2  Huru länge skolen I bedröva min själ och krossa mig sönder med edra ord?
Job 19:3  Tio gånger haven I nu talat smädligt mot mig och kränkt mig utan all försyn.
Job 19:4  Om så är, att jag verkligen har farit vilse, då är förvillelsen min egen sak.
Job 19:5  Men viljen I ändå verkligen förhäva eder mot mig, och påstån I att smäleken har drabbat mig med skäl,
Job 19:6  så veten fastmer att Gud har gjort mig orätt och att han har omsnärjt mig med sitt nät.
Job 19:7  Se, jag klagar över våld, men får intet svar; jag ropar, men får icke rätt.
Job 19:8  Min väg har han spärrat, så att jag ej kommer fram, och över mina stigar breder han mörker.
Job 19:9  Min ära har han avklätt mig, och från mitt huvud har han tagit bort kronan.
Job 19:10  Från alla sidor bryter han ned mig, så att jag förgås; han rycker upp mitt hopp, såsom vore det ett träd.
Job 19:11  Sin vrede låter han brinna mot mig och aktar mig såsom sina ovänners like.
Job 19:12  Hans skaror draga samlade fram och bereda sig väg till anfall mot mig; de lägra sig runt omkring min hydda.
Job 19:13  Långt bort ifrån mig har han drivit mina fränder; mina bekanta äro idel främlingar mot mig.
Job 19:14  Mina närmaste hava dragit sig undan, och mina förtrogna hava förgätit mig.
Job 19:15  Mitt husfolk och mina tjänstekvinnor akta mig såsom främling; en främmande man har jag blivit i deras ögon.
Job 19:16  Kallar jag på min tjänare, så svarar han icke; ödmjukt måste jag bönfalla hos honom.
Job 19:17  Min andedräkt är vidrig för min hustru, jag väcker leda hos min moders barn.
Job 19:18  Till och med de små barnen visa mig förakt; så snart jag står upp, tala de ohöviskt emot mig.
Job 19:19  Ja, en styggelse är jag för alla dem jag umgicks med; de som voro mig kärast hava vänt sig emot mig.
Job 19:20  Benen i min kropp tränga ut i hud och hull; knappt tandköttet har jag fått behålla kvar.
Job 19:21  Haven misskund, haven misskund med mig, I mina vänner, då nu Guds hand så har hemsökt mig.
Job 19:22  Varför skolen I förfölja mig, I såsom Gud, och aldrig bliva mätta av mitt kött?
Job 19:23  Ack att mina ord skreves upp, ack att de bleve upptecknade i en bok,
Job 19:24  ja, bleve med ett stift av järn och med bly för evig tid inpräglade i klippan!
Job 19:25  Dock, jag vet att min förlossare lever, och att han till slut skall stå fram över stoftet.
Job 19:26  Och sedan denna min sargade hud är borta, skall jag fri ifrån mitt kött få skåda Gud.
Job 19:27  Ja, honom skall jag få skåda, mig till hjälp, för mina ögon skall jag se honom, ej såsom en främling; därefter trånar jag i mitt innersta.
Job 19:28  Men när I tänken: "huru skola vi icke ansätta honom!" - såsom vore skulden att finna hos mig -
Job 19:29  då mån I taga eder till vara för svärdet, ty vreden hör till de synder som straffas med svärd; så mån I då besinna att en dom skall komma.
Job 20:1  Därefter tog Sofar från Naama till orda och sade:
Job 20:2  På sådant tal giva mina tankar mig ett svar, än mer, då jag nu är så upprörd i mitt inre.
Job 20:3  Smädlig tillrättavisning måste jag höra, och man svarar mig med munväder på förståndigt tal.
Job 20:4  Vet du då icke att så har varit från evig tid, från den stund då människor sattes på jorden:
Job 20:5  att de ogudaktigas jubel varar helt kort och den gudlöses glädje ett ögonblick?
Job 20:6  Om än hans förhävelse stiger upp till himmelen och hans huvud når intill molnen,
Job 20:7  Så förgås han dock för alltid och aktas lik sin träck; de som sågo honom måste fråga: "Var är han?"
Job 20:8  Lik en dröm flyger han bort, och ingen finner honom mer; han förjagas såsom en syn om natten.
Job 20:9  Det öga som såg honom ser honom icke åter, och hans plats får ej skåda honom mer.
Job 20:10  Hans barn måste gottgöra hans skulder till de arma, hans händer återbära hans vinning.
Job 20:11  Bäst ungdomskraften fyller hans ben, skall den ligga i stoftet med honom.
Job 20:12  Om än ondskan smakar ljuvligt i hans mun, så att han gömmer den under sin tunga,
Job 20:13  är rädd om den och ej vill gå miste därom, utan håller den förvarad inom sin gom,
Job 20:14  så förvandlas denna kost i hans inre, bliver huggormsetter i hans liv.
Job 20:15  Den rikedom han har slukat måste han utspy; av Gud drives den ut ur hans buk.
Job 20:16  Ja, huggormsgift kommer han att dricka, av etterormens tunga bliver han dräpt.
Job 20:17  Ingen bäck får vederkvicka hans syn, ingen ström med flöden av honung och gräddmjölk.
Job 20:18  Sitt fördärv måste han återbära, han får ej njuta därav; hans fröjd svarar ej mot den rikedom han har vunnit.
Job 20:19  Ty mot de arma övade han våld och lät dem ligga där; han rev till sig hus som han ej kan hålla vid makt.
Job 20:20  Han visste ej av någon ro för sin buk, men han skall icke rädda sig med sina skatter.
Job 20:21  Intet slapp undan hans glupskhet, därför äger och hans lycka intet bestånd.
Job 20:22  Mitt i hans överflöd påkommer honom nöd, och envar eländig vänder då mot honom sin hand.
Job 20:23  Ja, så måste ske, för att hans buk må bliva fylld; sin vredes glöd skall Gud sända över honom och låta den tränga såsom ett regn in i hans kropp.
Job 20:24  Om han flyr undan för vapen av järn, så genomborras han av kopparbågens skott.
Job 20:25  När han då drager i pilen och den kommer ut ur hans rygg, när den ljungande udden kommer fram ur hans galla, då falla dödsfasorna över honom.
Job 20:26  Idel mörker är förvarat åt hans skatter; till mat gives honom eld som brinner utan pust, den förtär vad som är kvar i hans hydda.
Job 20:27  Himmelen lägger hans missgärning i dagen, och jorden reser sig upp emot honom.
Job 20:28  Vad som har samlats i hans hus far åter sin kos, likt förrinnande vatten, på vredens dag.
Job 20:29  Sådan lott får en ogudaktig människa av Gud, sådan arvedel har av Gud blivit bestämd åt henne.
Job 21:1  Därefter tog Job till orda och sade:
Job 21:2  Hören åtminstone på mina ord; låten det vara den tröst som I given mig.
Job 21:3  Haven fördrag med mig, så att jag får tala; sedan jag har talat, må du bespotta.
Job 21:4  Är då min klagan, såsom när människor eljest klaga? Eller huru skulle jag kunna vara annat än otålig?
Job 21:5  Akten på mig, så skolen I häpna och nödgas lägga handen på munnen.
Job 21:6  Ja, när jag tänker därpå, då förskräckes jag själv, och förfäran griper mitt kött.
Job 21:7  Varför få de ogudaktiga leva, ja, med åldern växa till i rikedom?
Job 21:8  De se sina barn leva kvar hos sig, och sin avkomma hava de inför sina ögon.
Job 21:9  Deras hus stå trygga, ej hemsökta av förskräckelse; Gud låter sitt ris icke komma vid dem.
Job 21:10  När deras boskap parar sig, är det icke förgäves; lätt kalva deras kor, och icke i otid.
Job 21:11  Sina barn släppa de ut såsom en hjord, deras piltar hoppa lustigt omkring.
Job 21:12  De stämma upp med pukor och harpor, och glädja sig vid pipors ljud.
Job 21:13  De förnöta sina dagar i lust, och ned till dödsriket fara de i frid.
Job 21:14  Och de sade dock till Gud: "Vik ifrån oss, dina vägar vilja vi icke veta av.
Job 21:15  Vad är den Allsmäktige, att vi skulle tjäna honom? och vad skulle det hjälpa oss att åkalla honom?"
Job 21:16  Det är sant, i deras egen hand står ej deras lycka, och de ogudaktigas rådslag vare fjärran ifrån mig!
Job 21:17  Men huru ofta utslocknar väl de ogudaktigas lampa, huru ofta händer det att ofärd kommer över dem, och att han tillskiftar dem lotter i vrede?
Job 21:18  De borde ju bliva såsom halm för vinden, lika agnar som stormen rycker bort.
Job 21:19  "Gud spar åt hans barn att lida för hans ondska." Ja, men honom själv borde han vedergälla, så att han finge känna det.
Job 21:20  Med egna ögon borde han se sitt fall, och av den Allsmäktiges vrede borde han få dricka.
Job 21:21  Ty vad frågar han efter sitt hus, när han själv är borta, när hans månaders antal har nått sin ände?
Job 21:22  "Skall man då lära Gud förstånd, honom som dömer över de högsta?"
Job 21:23  Ja, den ene får dö i sin välmaktstid, där han sitter i allsköns frid och ro;
Job 21:24  hans stävor hava fått stå fulla med mjölk, och märgen i hans ben har bevarat sin saft.
Job 21:25  Den andre måste dö med bedrövad själ, och aldrig fick han njuta av någon lycka.
Job 21:26  Tillsammans ligga de så i stoftet, och förruttnelsens maskar övertäcka dem.
Job 21:27  Se, jag känner väl edra tankar och de funder med vilka I viljen nedslå mig.
Job 21:28  I spörjen ju: "Vad har blivit av de höga herrarnas hus, av hyddorna när de ogudaktiga bodde?"
Job 21:29  Haven I då ej frågat dem som vida foro, och akten I ej på deras vittnesbörd:
Job 21:30  att den onde bliver sparad på ofärdens dag och bärgad undan på vredens dag?
Job 21:31  Vem vågar ens förehålla en sådan hans väg? Vem vedergäller honom, vad han än må göra?
Job 21:32  Och när han har blivit bortförd till graven, så vakar man sedan där vid kullen.
Job 21:33  Ljuvligt får han vilja under dalens torvor. I hans spår drager hela världen fram; före honom har och otaliga gått.
Job 21:34  Huru kunnen I då bjuda mig så fåfänglig tröst? Av edra svar står allenast trolösheten kvar.
Job 22:1  Därefter tog Elifas från Teman till orda och sade:
Job 22:2  Kan en man bereda Gud något gagn, så att det länder honom till gagn, om någon är förståndig?
Job 22:3  Har den Allsmäktige någon båtnad av att du är rättfärdig, eller någon vinning av att du vandrar ostraffligt?
Job 22:4  Är det för din gudsfruktans skull som han straffar dig, och som han går med dig till doms?
Job 22:5  Har då icke din ondska varit stor, och voro ej dina missgärningar utan ände?
Job 22:6  Jo, du tog pant av din broder utan sak, du plundrade de utblottade på deras kläder.
Job 22:7  Åt den försmäktande gav du intet vatten att dricka, och den hungrige nekade du bröd.
Job 22:8  För den väldige ville du upplåta landet, och den myndige skulle få bo däri,
Job 22:9  men änkor lät du gå med tomma händer, och de faderlösas armar blevo krossade.
Job 22:10  Därför omgives du nu av snaror och förfäras av plötslig skräck.
Job 22:11  ja, av ett mörker där du intet ser, och av vattenflöden som övertäcka dig.
Job 22:12  I himmelens höjde är det ju Gud som har sin boning, och du ser stjärnorna däruppe, huru högt de sitta;
Job 22:13  därför tänker du: "Vad kan Gud veta? Skulle han kunna döma, han som bor bortom töcknet?
Job 22:14  Molnen äro ju ett täckelse, så att han intet ser; och på himlarunden är det han har sin gång."
Job 22:15  Vill du då hålla dig på forntidens väg, där fördärvets män gingo fram,
Job 22:16  de män som bortrycktes, innan deras tid var ute, och såsom en ström flöt deras grundval bort,
Job 22:17  de män som sade till Gud: "Vik ifrån oss", ty vad skulle den Allsmäktige kunna göra dem?
Job 22:18  Det var ju dock han som uppfyllde deras hus med sitt goda. De ogudaktigas rådslag vare fjärran ifrån mig!
Job 22:19  De rättfärdiga skola se det och glädja sig, och den oskyldige skall få bespotta dem:
Job 22:20  "Ja, nu äro förvisso våra motståndare utrotade, och deras överflöd har elden förtärt."
Job 22:21  Men sök nu förlikning och frid med honom; därigenom skall lycka falla dig till.
Job 22:22  Tag emot undervisning av hans mun, och förvara hans ord i ditt hjärta.
Job 22:23  Om du omvänder dig till den Allsmäktige, så bliver du upprättad; men orättfärdighet må du skaffa bort ur din hydda.
Job 22:24  Ja kasta din gyllene skatt i stoftet och Ofirs-guldet ibland bäckens stenar,
Job 22:25  så bliver den Allsmäktige din gyllene skatt, det ädlaste silver varder han för dig.
Job 22:26  Ja, då skall du hava din lust i den Allsmäktige och kunna upplyfta ditt ansikte till Gud.
Job 22:27  När du då beder till honom, skall han höra dig, och de löften du gör skall du få infria.
Job 22:28  Allt vad du besluter skall då lyckas för dig, och ljus skall skina på dina vägar.
Job 22:29  Om de leda mot djupet och du då beder: "Uppåt!", så frälsar han mannen som har ödmjukat sig.
Job 22:30  Ja han räddar och den som ej är fri ifrån skuld; genom dina händers renhet räddas en sådan.
Job 23:1  Därefter tog Job till orda och sade:
Job 23:2  Också i dag vill min klaga göra uppror. Min hand kännes matt för min suckans skull.
Job 23:3  Om jag blott visste huru jag skulle finna honom, huru jag kunde komma dit där han bor!
Job 23:4  Jag skulle då lägga fram för honom min sak och fylla min mun med bevis.
Job 23:5  Jag ville väl höra vad han kunde svara mig, och förnimma vad han skulle säga till mig.
Job 23:6  Icke med övermakt finge han bekämpa mig, nej, han borde allenast lyssna till mig.
Job 23:7  Då skulle hans motpart stå här såsom en redlig man, ja, då skulle jag för alltid komma undan min domare.
Job 23:8  Men går jag mot öster, så är han icke där; går jag mot väster, så varsnar jag honom ej;
Job 23:9  har han något att skaffa i norr, jag skådar honom icke; döljer han sig i söder, jag ser honom ej heller där.
Job 23:10  Han vet ju vilken väg jag har vandrat; han har prövat mig, och jag har befunnits lik guld.
Job 23:11  Vid hans spår har min for hållit fast, hans väg har jag följt, utan att vika av.
Job 23:12  Från hans läppars bud har jag icke gjort något avsteg; mer än egna rådslut har jag aktat hans muns tal.
Job 23:13  Men hans vilja är orygglig; vem kan hindra honom? Vad honom lyster, det gör han ock.
Job 23:14  Ja, han giver mig fullt upp min beskärda del, och mycket av samma slag har han ännu i förvar.
Job 23:15  Därför gripes jag av förskräckelse för hans ansikte; när jag betänker det, fruktar jag för honom.
Job 23:16  Det är Gud som har gjort mitt hjärta försagt, den Allsmäktige är det som har vållat min förskräckelse,
Job 23:17  ty jag fick icke förgås, innan mörkret kom, dödsnatten undanhöll han mig.
Job 24:1  Varför har den Allsmäktige inga räfstetider i förvar? varför få hans vänner ej skåda hans hämndedagar?
Job 24:2  Se, råmärken flyttar man undan, rövade hjordar driver man i bet;
Job 24:3  de faderlösas åsna för man bort och tager änkans ko i pant.
Job 24:4  Man tränger de fattiga undan från vägen, de betryckta i landet måste gömma sig med varandra.
Job 24:5  Ja, såsom vildåsnor måste de leva i öknen; dit gå de och möda sig och söka något till täring; hedmarken är det bröd de hava åt sina barn.
Job 24:6  På fältet få de till skörd vad boskap plägar äta, de hämta upp det sista i den ogudaktiges vingård.
Job 24:7  Nakna ligga de om natten, berövade sina kläder; de hava intet att skyla sig med i kölden.
Job 24:8  Av störtskurar från bergen genomdränkas de; de famna klippan, ty de äga ej annan tillflykt.
Job 24:9  Den faderlöse slites från sin moders bröst, och den betryckte drabbas av utpantning.
Job 24:10  Nakna måste de gå omkring, berövade sina kläder, hungrande nödgas de bära på kärvar.
Job 24:11  Inom sina förtryckares murar måste de bereda olja, de få trampa vinpressar och därvid lida törst.
Job 24:12  Utstötta ur människors samfund jämra de sig, ja, från dödsslagnas själar uppstiger ett rop. Men Gud aktar ej på vad förvänt som sker.
Job 24:13  Andra hava blivit fiender till ljuset; de känna icke dess vägar och hålla sig ej på dess stigar.
Job 24:14  Vid dagningen står mördaren upp för att dräpa den betryckte och fattige; och om natten gör han sig till tjuvars like.
Job 24:15  Äktenskapsbrytarens öga spejar efter skymningen, han tänker: "Intet öga får känna igen mig", och sätter så ett täckelse framför sitt ansikte.
Job 24:16  När det är mörkt, bryta sådana sig in i husen, men under dagen stänga de sig inne; ljuset vilja de icke veta av.
Job 24:17  Ty det svarta mörkret räknas av dem alla såsom morgon, med mörkrets förskräckelser äro de ju förtrogna.
Job 24:18  "Men hastigt", menen I, "ryckes en sådan bort av strömmen, förbannad bliver hans del i landet; till vingårdarna får han ej mer styra sina steg.
Job 24:19  Såsom snövatten förtäres av torka och hetta, så förtär dödsriket den som har syndat.
Job 24:20  Hans moders liv förgäter honom, maskar frossa på honom, ingen finnes, som bevarar hans minne; såsom ett träd brytes orättfärdigheten av.
Job 24:21  Så går det, när någon plundrar den ofruktsamma, som intet föder, och när någon icke gör gott mot änkan."
Job 24:22  Ja, men han uppehåller ock våldsmännen genom sin kraft, de få stå upp, när de redan hade förlorat hoppet om livet;
Job 24:23  han giver dem trygghet, så att de få vila, och hans ögon vaka över deras vägar.
Job 24:24  När de hava stigit till sin höjd, beskäres dem en snar hädanfärd, de sjunka då ned och dö som alla andra; likasom axens toppar vissna de bort.
Job 24:25  Är det ej så, vem vill då vederlägga mig, vem kan göra mina ord om intet?
Job 25:1  Därefter tog Bildad från Sua till orda och sade:
Job 25:2  Hos honom är väldighet och förskräckande makt, hos honom, som skapar frid i sina himlars höjd.
Job 25:3  Vem finnes, som förmår räkna hans skaror? Och vem överstrålas ej av hans ljus?
Job 25:4  Huru skulle då en människa kunna hava rätt mot Gud eller en av kvinna född kunna befinnas ren?
Job 25:5  Se, ej ens månen skiner nog klart, ej ens stjärnorna äro rena i hans ögon;
Job 25:6  huru mycket mindre då människan, det krypet, människobarnet, den masken!
Job 26:1  Därefter tog Job till orda och sade:
Job 26:2  Vilken hjälp har du ej skänkt den vanmäktige, huru har du ej stärkt den maktlöses arm!
Job 26:3  Vilka råd har du ej givit den ovise, och vilket överflöd av klokhet har du ej lagt i dagen!
Job 26:4  Vem gav dig kraft att tala sådana ord, och vems ande var det som kom till orda ur dig?
Job 26:5  Dödsrikets skuggor gripas av ångest, djupets vatten och de som bo däri.
Job 26:6  Dödsriket ligger blottat för honom, och avgrunden har intet täckelse.
Job 26:7  Han spänner ut nordanrymden över det tomma och hänger upp jorden på intet.
Job 26:8  Han samlar vatten i sina moln såsom i ett knyte, och skyarna brista icke under bördan.
Job 26:9  Han gömmer sin tron för vår åsyn, han omhöljer den med sina skyar.
Job 26:10  En rundel har han välvt såsom gräns för vattnen, där varest ljus ändas i mörker.
Job 26:11  Himmelens pelare skälva, de gripas av förfäran vid hans näpst.
Job 26:12  Med sin kraft förskräckte han havet, och genom sitt förstånd sönderkrossade han Rahab.
Job 26:13  Blott han andades, blev himmelen klar; hans hand genomborrade den snabba ormen.
Job 26:14  Se, detta är allenast utkanterna av hans verk; en sakta viskning är allt vad vi förnimma därom. Hans allmakts dunder, vem skulle kunna fatta det?
Job 27:1  Åter hov Job upp sin röst och kvad:
Job 27:2  Så sant Gud lever, han som har förhållit mig min rätt, den Allsmäktige, som har vållat min själs bedrövelse:
Job 27:3  aldrig, så länge ännu min ande är i mig och Guds livsfläkt är kvar i min näsa,
Job 27:4  aldrig skola mina läppar tala vad orättfärdigt är, och min tunga bära fram oärligt tal.
Job 27:5  Bort det, att jag skulle giva eder rätt! Intill min död låter jag min ostrafflighet ej tagas ifrån mig.
Job 27:6  Vid min rättfärdighet håller jag fast och släpper den icke, mitt hjärta förebrår mig ej för någon av mina dagar.
Job 27:7  Nej, såsom ogudaktig må min fiende stå där och min motståndare såsom orättfärdig.
Job 27:8  Ty vad hopp har den gudlöse när hans liv avskäres, när hans själ ryckes bort av Gud?
Job 27:9  Månne Gud skall höra hans rop, när nöden kommer över honom?
Job 27:10  Eller kan en sådan hava sin lust i den Allsmäktige, kan han åkalla Gud alltid?
Job 27:11  Jag vill undervisa eder om huru Gud går till väga; huru den Allsmäktige tänker, vill jag icke fördölja.
Job 27:12  Dock, I haven ju själva allasammans skådat det; huru kunnen I då hängiva eder åt så fåfängliga tankar?
Job 27:13  Hören vad den ogudaktiges lott bliver hos Gud, vilken arvedel våldsverkaren får av den Allsmäktige:
Job 27:14  Om hans barn bliva många, så är vinningen svärdets; hans avkomlingar få ej bröd att mätta sig med.
Job 27:15  De som slippa undan läggas i graven genom pest, och hans änkor kunna icke hålla sin klagogråt.
Job 27:16  Om han ock hopar silver såsom stoft och lägger kläder på hög såsom lera,
Job 27:17  så är det den rättfärdige som får kläda sig i vad han lägger på hög, och den skuldlöse kommer att utskifta silvret.
Job 27:18  Det hus han bygger bliver så förgängligt som malen, det skall likna skjulet som vaktaren gör sig.
Job 27:19  Rik lägger han sig och menar att intet skall tagas bort; men när han öppnar sina ögon, är ingenting kvar.
Job 27:20  Såsom vattenfloder taga förskräckelser honom fatt, om natten rövas han bort av stormen.
Job 27:21  Östanvinden griper honom, så att han far sin kos, den rycker honom undan från hans plats.
Job 27:22  Utan förskoning skjuter Gud sina pilar mot honom; för hans hand måste han flykta med hast.
Job 27:23  Då slår man ihop händerna, honom till hån; man visslar åt honom på platsen där han var.
Job 28:1  Silvret har ju sin gruva, sin fyndort har guldet, som man renar;
Job 28:2  järn hämtas upp ur jorden, och stenar smältas till koppar.
Job 28:3  Man sätter då gränser för mörkret, och rannsakar ned till yttersta djupet,
Job 28:4  Där spränger man schakt långt under markens bebyggare, där färdas man förgäten djupt under vandrarens fot, där hänger man svävande, fjärran ifrån människor.
Job 28:5  Ovan ur jorden uppväxer bröd, men därnere omvälves den såsom av eld.
Job 28:6  Där, bland dess stenar, har safiren sitt fäste, guldmalm hämtar man ock där.
Job 28:7  Stigen ditned är ej känd av örnen, och falkens öga har ej utspanat den;
Job 28:8  den har ej blivit trampad av stolta vilddjur, intet lejon har gått därfram.
Job 28:9  Ja, där bär man hand på hårda stenen; bergen omvälvas ända ifrån rötterna.
Job 28:10  In i klipporna bryter man sig gångar, där ögat får se allt vad härligt är.
Job 28:11  Vattenådror täppas till och hindras att gråta. Så dragas dolda skatter fram i ljuset.
Job 28:12  Men visheten, var finnes hon, och var har förståndet sin boning?
Job 28:13  Priset för henne känner ingen människa; hon står ej att finna i de levandes land.
Job 28:14  Djupet säger: "Hon är icke här", och havet säger: "Hos mig är hon icke."
Job 28:15  Hon köper icke för ädlaste metall, med silver gäldas ej hennes värde.
Job 28:16  Hon väges icke upp med guld från Ofir, ej med dyrbar onyx och safir.
Job 28:17  Guld och glas kunna ej liknas vid henne; hon får ej i byte mot gyllene klenoder.
Job 28:18  Koraller och kristall må icke ens nämnas; svårare är förvärva vishet än pärlor.
Job 28:19  Etiopisk topas kan ej liknas vid henne; hon väges icke upp med renaste guld.
Job 28:20  Ja, visheten, varifrån kommer väl hon, och var har förståndet sin boning?
Job 28:21  Förborgad är hon för alla levandes ögon, för himmelens fåglar är hon fördold;
Job 28:22  avgrunden och döden giva till känna; "Blott hörsägner om henne förnummo våra öron."
Job 28:23  Gud, han är den som känner vägen till henne, han är den som vet var hon har sin boning.
Job 28:24  Ty han förmår skåda till jordens ändar, allt vad som finnes under himmelen ser han.
Job 28:25  När han mätte ut åt vinden dess styrka och avvägde vattnen efter mått,
Job 28:26  när han stadgade en lag för regnet och en väg för tordönets stråle,
Job 28:27  då såg han och uppenbarade henne, då lät han henne stå fram, då utforskade han henne.
Job 28:28  Och till människorna sade han så: "Se Herrens fruktan, det är vishet, och att fly det onda är förstånd."
Job 29:1  Åter hov Job upp sin röst och kvad:
Job 29:2  Ack att jag vore såsom i forna månader, såsom i de dagar då Gud gav mig sitt beskydd,
Job 29:3  då hans lykta sken över mitt huvud och jag vid hans ljus gick fram genom mörkret!
Job 29:4  Ja, vore jag såsom i min mognads dagar, då Guds huldhet vilade över min hydda,
Job 29:5  då ännu den Allsmäktige var med mig och mina barn stodo runt omkring mig,
Job 29:6  då mina fötter badade i gräddmjölk och klippan invid mig göt ut bäckar av olja!
Job 29:7  När jag då gick upp till porten i staden och intog mitt säte på torget,
Job 29:8  då drogo de unga sig undan vid min åsyn, de gamla reste sig upp och blevo stående.
Job 29:9  Då höllo hövdingar tillbaka sina ord och lade handen på munnen;
Job 29:10  furstarnas röst ljöd då dämpad, och deras tunga lådde vid gommen.
Job 29:11  Ja, vart öra som hörde prisade mig då säll, och vart öga som såg bar vittnesbörd om mig;
Job 29:12  ty jag räddade den betryckte som ropade, och den faderlöse, den som ingen hjälpare hade.
Job 29:13  Den olyckliges välsignelse kom då över mig, och änkans hjärta uppfyllde jag med jubel.
Job 29:14  I rättfärdighet klädde jag mig, och den var såsom min klädnad; rättvisa bar jag såsom mantel och huvudbindel.
Job 29:15  Ögon blev jag då åt den blinde, och fötter var jag åt den halte.
Job 29:16  Jag var då en fader för de fattiga, och den okändes sak redde jag ut.
Job 29:17  Jag krossade den orättfärdiges käkar och ryckte rovet undan hans tänder.
Job 29:18  Jag tänkte då: "I mitt näste skall jag få dö, mina dagar skola bliva många såsom sanden.
Job 29:19  Min rot ligger ju öppen för vatten, och i min krona faller nattens dagg.
Job 29:20  Min ära bliver ständigt ny, och min båge föryngras i min hand."
Job 29:21  Ja, på mig hörde man då och väntade, man lyssnade under tystnad på mitt råd.
Job 29:22  Sedan jag hade talat, talade ingen annan; såsom ett vederkvickande flöde kommo mina ord över dem.
Job 29:23  De väntade på mig såsom på regn, de spärrade upp sina munnar såsom efter vårregn.
Job 29:24  När de misströstade, log jag emot dem, och mitt ansiktes klarhet kunde de icke förmörka.
Job 29:25  Täcktes jag besöka dem, så måste jag sitta främst; jag tronade då såsom en konung i sin skara, lik en man som har tröst för de sörjande.
Job 30:1  Och nu le de åt mig, människor som äro yngre till åren än jag, män vilkas fäder jag aktade ringa, ja, ej ens hade velat sätta bland mina vallhundar.
Job 30:2  Vad skulle de också kunna gagna mig med sin hjälp, dessa människor som sakna all manlig kraft?
Job 30:3  Utmärglade äro de ju av brist och svält; de gnaga sin föda av torra öknen, som redan i förväg är öde och ödslig.
Job 30:4  Saltörter plocka de där bland snåren, och ginströtter är vad de hava till mat.
Job 30:5  Ur människors samkväm drives de ut, man ropar efter dem såsom efter tjuvar.
Job 30:6  I gruvliga klyftor måste de bo, i hålor under jorden och i bergens skrevor.
Job 30:7  Bland snåren häva de upp sitt tjut, under nässlor ligga de skockade,
Job 30:8  en avföda av dårar och ärelöst folk, utjagade ur landet med hugg och slag.
Job 30:9  Och för sådana har jag nu blivit en visa, de hava mig till ämne för sitt tal;
Job 30:10  med avsky hålla de sig fjärran ifrån mig, de hava ej försyn för att spotta åt mig.
Job 30:11  Nej, mig till plåga, lossa de alla band, alla tyglar kasta de av inför mig.
Job 30:12  Invid min högra sida upphäver sig ynglet; mina fötter vilja de stöta undan. De göra sig vägar som skola leda till min ofärd.
Job 30:13  Stigen framför mig hava de rivit upp. De göra sitt bästa till att fördärva mig, de som dock själva äro hjälplösa.
Job 30:14  Såsom genom en bred rämna bryta de in; de vältra sig fram under murarnas brak.
Job 30:15  Förskräckelser välvas ned över mig. Såsom en storm bortrycka de min ära, och såsom ett moln har min välfärd farit bort.
Job 30:16  Och nu utgjuter sig min själ inom mig, eländesdagar hålla mig fast.
Job 30:17  Natten bortfräter benen i min kropp, och kvalen som gnaga mig veta ej av vila.
Job 30:18  Genom övermäktig kraft har mitt kroppshölje blivit vanställt, såsom en livklädnad hänger det omkring mig.
Job 30:19  I orenlighet har jag blivit nedstjälpt, och själv är jag nu lik stoft och aska.
Job 30:20  Jag ropar till dig, men du svarar mig icke; jag står här, men de bespejar mig allenast.
Job 30:21  Du förvandlas för mig till en grym fiende, med din starka hand ansätter du mig.
Job 30:22  Du lyfter upp mig i stormvinden och för mig hän, och i bruset låter du mig försmälta av ångest.
Job 30:23  Ja, jag förstår att du vill föra mig till döden, till den boning dit allt levande församlas.
Job 30:24  Men skulle man vid sitt fall ej få sträcka ut handen, ej ropa efter hjälp, när ofärd har kommit?
Job 30:25  Grät jag ej själv över den som hade hårda dagar, och ömkade sig min själ ej över den fattige?
Job 30:26  Se, jag väntade mig lycka, men olycka kom; jag hoppades på ljus, men mörker kom.
Job 30:27  Därför sjuder mitt innersta och får ingen ro, eländesdagar hava ju mött mig.
Job 30:28  Med mörknad hud går jag, fastän ej bränd av solen; mitt i församlingen står jag upp och skriar.
Job 30:29  En broder har jag blivit till schakalerna, och en frände är jag vorden till strutsarna.
Job 30:30  Min hud har svartnat och lossnat från mitt kött, benen i min kropp äro förbrända av hetta.
Job 30:31  I sorgelåt är mitt harpospel förbytt, mina pipors klang i högljudd gråt.
Job 31:1  Ett förbund slöt jag med mina ögon: aldrig skulle jag skåda efter någon jungfru.
Job 31:2  Vilken lott finge jag eljest av Gud i höjden, vilken arvedel av den Allsmäktige därovan?
Job 31:3  Ofärd kommer ju över de orättfärdiga, och olycka drabbar ogärningsmän.
Job 31:4  Ser icke han mina vägar, räknar han ej alla mina steg?
Job 31:5  Har jag väl umgåtts med lögn, och har min fot varit snar till svek?
Job 31:6  Nej, må jag vägas på en riktig våg, så skall Gud förnimma min ostrafflighet.
Job 31:7  Hava mina steg vikit av ifrån vägen, har mitt hjärta följt efter mina ögon, eller låder vid min händer en fläck?
Job 31:8  Då må en annan äta var jag har sått, och vad jag har planterat må ryckas upp med roten.
Job 31:9  Har mitt hjärta låtit dåra sig av någon kvinna, så att jag har stått på lur vid min nästas dörr?
Job 31:10  Då må min hustru mala mjöl åt en annan, och främmande män må då famntaga henne.
Job 31:11  Ja, sådant hade varit en skändlighet, en straffbar missgärning hade det varit,
Job 31:12  en eld som skulle förtära intill avgrunden och förhärja till roten all min gröda.
Job 31:13  Har jag kränkt min tjänares eller tjänarinnas rätt, när de hade någon tvist med mig?
Job 31:14  Vad skulle jag då göra, när Gud stode upp, och när han hölle räfst, vad kunde jag då svara honom?
Job 31:15  Han som skapade mig skapade ju och dem i moderlivet, han, densamme, har berett dem i modersskötet.
Job 31:16  Har jag vägrat de arma vad de begärde eller låtit änkans ögon försmäkta?
Job 31:17  Har jag ätit mitt brödstycke allena, utan att den faderlöse och har fått äta därav?
Job 31:18  Nej, från min ungdom fostrades han hos mig såsom hos en fader, och från min moders liv var jag änkors ledare.
Job 31:19  Har jag kunnat se en olycklig gå utan kläder, se en fattig ej äga något att skyla sig med?
Job 31:20  Måste ej fastmer hans länd välsigna mig, och fick han ej värma sig i ull av mina lamm?
Job 31:21  Har jag lyft min hand mot den faderlöse, därför att jag såg mig hava medhåll i porten?
Job 31:22  Då må min axel lossna från sitt fäste och min arm brytas av ifrån sin led.
Job 31:23  Jag måste då frukta ofärd ifrån Gud och skulle stå maktlös inför hans majestät.
Job 31:24  Har jag satt mitt hopp till guldet och kallat guldklimpen min förtröstan?
Job 31:25  Var det min glädje att min rikedom blev så stor, och att min hand förvärvade så mycket?
Job 31:26  Hände det, när jag såg solljuset, huru det sken, och månen, huru härligt den gick fram,
Job 31:27  att mitt hjärta hemligen lät dåra sig, så att jag med handkyss gav dem min hyllning?
Job 31:28  Nej, också det hade varit en straffbar missgärning; därmed hade jag ju förnekat Gud i höjden.
Job 31:29  Har jag glatt mig åt min fiendes ofärd och fröjdats, när olycka träffade honom?
Job 31:30  Nej, jag tillstadde ej min mun att synda så, ej att med förbannelse begära hans liv.
Job 31:31  Och kan mitt husfolk icke bevittna att envar fick mätta sig av kött vid mitt bord?
Job 31:32  Främlingen behövde ej stanna över natten på gatan, mina dörrar lät jag stå öppna utåt vägen.
Job 31:33  Har jag på människovis skylt mina överträdelser och gömt min missgärning i min barm,
Job 31:34  av fruktan för den stora hopen och av rädsla för stamfränders förakt, så att jag teg och ej gick utom min dörr?
Job 31:35  Ack att någon funnes, som ville höra mig! Jag har sagt mitt ord. Den Allsmäktige må nu svara mig; ack att jag finge min vederparts motskrift!
Job 31:36  Sannerligen, jag skulle då bära den högt på min skuldra, såsom en krona skulle jag fästa den på mig.
Job 31:37  Jag ville då göra honom räkenskap för alla mina steg, lik en furste skulle jag då träda inför honom.
Job 31:38  Har min mark höjt rop över mig, och hava dess fåror gråtit med varandra?
Job 31:39  Har jag förtärt dess gröda obetald eller utpinat dess brukares liv?
Job 31:40  Då må törne växa upp för vete, och ogräs i stället för korn. Slut på Jobs tal.
Job 32:1  De tre männen upphörde nu att svara Job, eftersom han höll sig själv för rättfärdig.
Job 32:2  Då blev Elihu, Barakels son, från Bus, av Rams släkt, upptänd av vrede. Mot Job upptändes han av vrede, därför att denne menade sig hava rätt mot Gud;
Job 32:3  och mot hans tre vänner upptändes hans vrede, därför att de icke funno något svar varmed de kunde vederlägga Job.
Job 32:4  Hittills hade Elihu dröjt att tala till Job, därför att de andra voro äldre till åren än han.
Job 32:5  Men då nu Elihu såg att de tre männen icke mer hade något att svara, upptändes hans vrede.
Job 32:6  Så tog då Elihu, Barakels son, från Bus, till orda och sade; Ung till åren är jag, I däremot ären gamla. Därför höll jag mig tillbaka och var försagd och lade ej fram för eder min mening.
Job 32:7  Jag tänkte: "Må åldern tala, och må årens mängd förkunna visdom."
Job 32:8  Dock, på anden i människorna kommer det an, den Allsmäktiges livsfläkt giver dem förstånd.
Job 32:9  Icke de åldriga äro alltid visast, icke de äldsta förstå bäst vad rätt är.
Job 32:10  Därför säger jag nu: Hör mig; jag vill lägga fram min mening, också jag.
Job 32:11  Se, jag väntade på vad I skullen tala, jag lyssnade efter förstånd ifrån eder, efter skäl som I skullen draga fram.
Job 32:12  Ja, noga aktade jag på eder. Men se, ingen fanns, som vederlade Job, ingen bland eder, som kunde svara på hans ord.
Job 32:13  Nu mån I icke säga: "Vi möttes av vishet; Gud, men ingen människa, kan nedslå denne."
Job 32:14  Skäl mot min mening har han icke lagt fram, ej heller skall jag bemöta honom med edra bevis.
Job 32:15  Se, nu stå de bestörta och svara ej mer, målet i munnen hava de mist.
Job 32:16  Och jag skulle vänta, då de nu intet kunna säga, då de stå där och ej mer hava något svar!
Job 32:17  Nej, också jag vill svara i min ordning, jag vill lägga fram min mening, också jag.
Job 32:18  Ty, fullt upp har jag av skäl, anden i mitt inre vill spränga mig sönder.
Job 32:19  Ja, mitt inre är såsom instängt vin, likt en lägel med nytt vin är det nära att brista.
Job 32:20  Så vill jag då tala och skaffa mig luft, jag vill upplåta mina läppar och svara.
Job 32:21  Jag får ej hava anseende till personen, och jag skall ej till någon tala inställsamma ord.
Job 32:22  Nej, jag förstår ej att tala inställsamma ord; huru lätt kunde ej eljest min skapare rycka mig bort!
Job 33:1  Men hör nu, Job, mina ord, och lyssna till allt vad jag vill säga.
Job 33:2  Se, jag upplåter nu mina läppar, min tunga tager till orda i min mun.
Job 33:3  Ur ett redbart hjärta framgår mitt tal, och vad mina läppar förstå säga de ärligt ut.
Job 33:4  Guds ande är det som har gjort mig, den Allsmäktiges fläkt beskär mig liv.
Job 33:5  Om du förmår, så må du nu svara mig; red dig till strid mot mig, träd fram.
Job 33:6  Se, jag är likställd med dig inför Gud, jag är danad av en nypa ler, också jag.
Job 33:7  Ja, fruktan för mig behöver ej förskräcka dig, ej heller kan min myndighet trycka dig ned.
Job 33:8  Men nu sade du så inför mina öron, så ljödo de ord jag hörde:
Job 33:9  "Ren är jag och fri ifrån överträdelse, oskyldig är jag och utan missgärning;
Job 33:10  men se, han finner på sak mot mig, han aktar mig såsom sin fiende.
Job 33:11  Han sätter mina fötter i stocken, vaktar på alla mina vägar."
Job 33:12  Nej, häri har du orätt, svarar jag dig. Gud är ju förmer än en människa.
Job 33:13  Huru kan du gå till rätta med honom, såsom gåve han aldrig svar i sin sak?
Job 33:14  Både på ett sätt och på två talar Gud, om man också ej aktar därpå.
Job 33:15  I drömmen, i nattens syn, när sömnen har fallit tung över människorna och de vila i slummer på sitt läger,
Job 33:16  då öppnar han människornas öron och sätter inseglet på sina varningar till dem,
Job 33:17  när han vill avvända någon från en ogärning eller hålla högmodet borta ifrån en människa.
Job 33:18  Så bevarar han hennes själ från graven och hennes liv ifrån att förgås genom vapen.
Job 33:19  Hon bliver ock agad genom plågor på sitt läger och genom ständig oro, allt intill benen.
Job 33:20  Hennes sinne får leda vid maten, och hennes själ vid den föda hon älskade.
Job 33:21  Hennes hull förtvinar, till dess intet är att se, ja, hennes ben täras bort intill osynlighet.
Job 33:22  Så nalkas hennes själ till graven och hennes liv hän till dödens makter.
Job 33:23  Men om en ängel då finnes, som vakar över henne, en medlare, någon enda av de tusen, och denne får lära människan hennes plikt,
Job 33:24  då förbarmar Gud sig över henne och säger; "Fräls henne, så att hon slipper fara ned i graven; lösepenningen har jag nu fått."
Job 33:25  Hennes kropp får då ny ungdomskraft, hon bliver åter såsom under sin styrkas dagar.
Job 33:26  När hon då beder till Gud, är han henne nådig och låter henne se sitt ansikte med jubel; han giver så den mannen hans rättfärdighet åter.
Job 33:27  Så får denne då sjunga inför människorna och säga: "Väl syndade jag, och väl kränkte jag rätten, dock vederfors mig ej vad jag hade förskyllt;
Job 33:28  ty han förlossade min själ, så att den undslapp graven, och mitt liv får nu med lust skåda ljuset."
Job 33:29  Se, detta allt kommer Gud åstad, både två gånger och tre, för den mannen,
Job 33:30  till att rädda hans själ från graven, så att han får njuta av de levandes ljus.
Job 33:31  Akta nu härpå, du Job, och hör mig; tig, så att jag får tala.
Job 33:32  Dock, har du något att säga, så svara mig; tala, ty gärna gåve jag dig rätt.
Job 33:33  Varom icke, så är det du som må höra på mig; du må tiga, så att jag får lära dig vishet.
Job 34:1  Och Elihu tog till orda och sade:
Job 34:2  Hören, I vise, mina ord; I förståndige, lyssnen till mig.
Job 34:3  Örat skall ju pröva orden, och munnen smaken hos det man vill äta.
Job 34:4  Må vi nu utvälja åt oss vad rätt är, samfällt söka förstå vad gott är.
Job 34:5  Se, Job har sagt: "Jag är oskyldig. Gud har förhållit mig min rätt.
Job 34:6  Fastän jag har rätt, måste jag stå såsom lögnare; dödsskjuten är jag, jag som intet har brutit."
Job 34:7  Var finnes en man som är såsom Job? Han läskar sig med bespottelse såsom med vatten,
Job 34:8  han gör sig till ogärningsmäns stallbroder och sällar sig till ogudaktiga människor.
Job 34:9  Ty han säger: "Det gagnar en man till intet, om han håller sig väl med Gud."
Job 34:10  Hören mig därför, I förståndige män: Bort det, att Gud skulle begå någon orätt, att den Allsmäktige skulle göra vad orättfärdigt är!
Job 34:11  Nej, han vedergäller var människa efter hennes gärningar och lönar envar såsom hans vandel har förtjänat.
Job 34:12  Ty Gud gör i sanning intet som är orätt, den Allsmäktige kan icke kränka rätten.
Job 34:13  Vem har bjudit honom att vårda sig om jorden, och vem lade på honom bördan av hela jordens krets?
Job 34:14  Om han ville tänka allenast på sig själv och åter draga till sig sin anda och livsfläkt,
Job 34:15  då skulle på en gång allt kött förgås, och människorna skulle vända åter till stoft.
Job 34:16  Men märk nu väl och hör härpå, lyssna till vad mina ord förkunna.
Job 34:17  Skulle den förmå regera, som hatade vad rätt är? Eller fördömer du den som är den störste i rättfärdighet?
Job 34:18  Får man då säga till en konung: "Du ogärningsman", eller till en furste: "Du ogudaktige"?
Job 34:19  Gud har ju ej anseende till någon hövdings person, han aktar den rike ej för mer än den fattige, ty alla äro de hans händers verk.
Job 34:20  I ett ögonblick omkomma de, mitt i natten: folkhopar gripas av bävan och förgås, de väldige ryckas bort, utan människohand.
Job 34:21  Ty hans ögon vakta på var mans vägar, och alla deras steg, dem ser han.
Job 34:22  Intet mörker finnes och ingen skugga så djup, att ogärningsmän kunna fördölja sig däri.
Job 34:23  Ty länge behöver Gud ej vakta på en människa, innan hon måste stå till doms inför honom.
Job 34:24  Han krossar de väldige utan rannsakning och låter så andra träda fram i deras ställe.
Job 34:25  Ja, han märker väl vad de göra, han omstörtar dem om natten och låter dem förgås.
Job 34:26  Såsom ogudaktiga tuktar han dem öppet, inför människors åsyn,
Job 34:27  eftersom de veko av ifrån honom och ej aktade på alla hans vägar.
Job 34:28  De bragte så den armes rop inför honom, och rop av betryckta fick han höra.
Job 34:29  Vem vågar då fördöma, om han stillar larmet? Ja, vem vill väl skåda honom, om han döljer sitt ansikte, för ett folk eller för en enskild man,
Job 34:30  när han vill rycka makten ifrån gudlösa människor och hindra dem att bliva snaror för folket?
Job 34:31  Kan man väl säga till Gud: "Jag måste lida, jag som ändå intet har förbrutit.
Job 34:32  Visa mig du vad som går över mitt förstånd; om jag har gjort något orätt, vill jag då ej göra så mer."
Job 34:33  Skall då han, för ditt klanders skull, giva vedergällning såsom du vill? Du själv, och icke jag, må döma därom; ja, tala du ut vad du menar.
Job 34:34  Men kloka män skola säga så till mig, visa män, när de få höra mig:
Job 34:35  "Job talar utan någon insikt, hans ord äro utan förstånd."
Job 34:36  Så må nu Job utstå prövningar allt framgent, då han vill försvara sig på ogärningsmäns sätt.
Job 34:37  Till sin synd lägger han ju uppenbar ondska, oss till hån slår han ihop sina händer och talar stora ord mot Gud.
Job 35:1  Och Elihu tog till orda och sade:
Job 35:2  Menar du att sådant är riktigt? Kan du påstå att du har rätt mot Gud,
Job 35:3  du som frågar vad rättfärdighet gagnar dig, vad den båtar dig mer än synd?
Job 35:4  Svar härpå vill jag giva dig, jag ock dina vänner med dig.
Job 35:5  Skåda upp mot himmelen och se, betrakta skyarna, som gå där högt över dig.
Job 35:6  Om du syndar, vad gör du väl honom därmed? Och om dina överträdelser äro många, vad skadar du honom därmed?
Job 35:7  Eller om du är rättfärdig, vad giver du honom, och vad undfår han av din hand?
Job 35:8  Nej, för din like kunde din ogudaktighet något betyda och för en människoson din rättfärdighet.
Job 35:9  Väl klagar man, när våldsgärningarna äro många, man ropar om hjälp mot de övermäktigas arm;
Job 35:10  men ingen frågar: "Var är min Gud, min skapare, han som låter lovsånger ljuda mitt i natten,
Job 35:11  han som giver oss insikt framför markens djur och vishet framför himmelens fåglar?"
Job 35:12  Därför är det man får ropa utan svar om skydd mot de ondas övermod.
Job 35:13  Se, på fåfängliga böner hör icke Gud, den Allsmäktige aktar icke på slikt;
Job 35:14  allra minst, när du påstår att du icke får skåda honom, att du måste vänta på honom, fastän saken är uppenbar.
Job 35:15  Och nu menar du att hans vrede ej håller någon räfst, och att han föga bekymrar sig om människors övermod?
Job 35:16  Ja, till fåfängligt tal spärrar Job upp sin mun, utan insikt talar han stora ord.
Job 36:1  Vidare sade Elihu:
Job 36:2  Bida ännu litet, så att jag får giva dig besked, ty ännu något har jag att säga till Guds försvar.
Job 36:3  Min insikt vill jag hämta vida ifrån, och åt min skapare vill jag skaffa rätt.
Job 36:4  Ja, förvisso skola mina ord icke vara lögn; en man med fullgod insikt har du framför dig.
Job 36:5  Se, Gud är väldig, men han försmår dock ingen, han som är så väldig i sitt förstånds kraft.
Job 36:6  Den ogudaktige låter han ej bliva vid liv, men åt de arma skaffar han rätt.
Job 36:7  Han tager ej sina ögon från de rättfärdiga; de få trona i konungars krets, för alltid låter han dem sitta där i höghet.
Job 36:8  Och om de läggas bundna i kedjor och fångas i eländets snaror,
Job 36:9  så vill han därmed visa dem vad de hava gjort, och vilka överträdelser de hava begått i sitt högmod;
Job 36:10  han vill då öppna deras öra för tuktan och mana dem att vända om ifrån fördärvet.
Job 36:11  Om de då höra på honom och underkasta sig, så få de framleva sina dagar i lycka och sina år i ljuvlig ro.
Job 36:12  Men höra de honom ej, så förgås de genom vapen och omkomma, när de minst tänka det.
Job 36:13  Ja, de som med gudlöst hjärta hängiva sig åt vrede och icke anropa honom, när han lägger dem i band,
Job 36:14  deras själ skall i deras ungdom ryckas bort av döden, och deras liv skall dela tempelbolares lott.
Job 36:15  Genom lidandet vill han rädda den lidande, och genom betrycket vill han öppna hans öra.
Job 36:16  Så sökte han ock draga dig ur nödens gap, ut på en rymlig plats, där intet trångmål rådde; och ditt bord skulle bliva fullsatt med feta rätter.
Job 36:17  Men nu bär du till fullo ogudaktighetens dom; ja, dom och rättvisa hålla dig nu fast.
Job 36:18  Ty vrede borde ej få uppegga dig under din tuktans tid, och huru svårt du än har måst plikta, borde du ej därav ledas vilse.
Job 36:19  Huru kan han lära dig bedja, om icke genom nöd och genom allt som nu har prövat din kraft?
Job 36:20  Du må ej längta så ivrigt efter natten, den natt då folken skola ryckas bort ifrån sin plats.
Job 36:21  Tag dig till vara, så att du ej vänder dig till vad fördärvligt är; sådant behagar dig ju mer än att lida.
Job 36:22  Se, Gud är upphöjd genom sin kraft. Var finnes någon mästare som är honom lik?
Job 36:23  Vem har föreskrivit honom hans väg, och vem kan säga: "Du gör vad orätt är?"
Job 36:24  Tänk då på att upphöja hans gärningar, dem vilka människorna besjunga
Job 36:25  och som de alla skåda med lust, de dödliga, om de än blott skönja dem i fjärran.
Job 36:26  Ja, Gud är för hög för vårt förstånd, hans år äro flera än någon kan utrannsaka.
Job 36:27  Se, vattnets droppar drager han uppåt, och de sila ned såsom regn, där hans dimma går fram;
Job 36:28  skyarna gjuta dem ut såsom en ström, låta dem drypa ned över talrika människor.
Job 36:29  Ja, kan någon fatta molnens utbredning, braket som utgår från hans hydda?
Job 36:30  Se, sitt ljungeldsljus breder han ut över molnen, och själva havsgrunden höljer han in däri.
Job 36:31  Ty så utför han sina domar över folken; så bereder han ock näring i rikligt mått.
Job 36:32  I ljungeldsljus höljer han sina händer och sänder det ut mot dem som begynna strid.
Job 36:33  Budskap om honom bär hans dunder; själva boskapen bebådar hans antåg.
Job 37:1  Ja, vid sådant förskräckes mitt hjärta, bävande spritter det upp.
Job 37:2  Hören, hören huru hans röst ljuder vred, hören dånet som går ut ur hans mun.
Job 37:3  Han sänder det åstad, så långt himmelen når, och sina ljungeldar bort till jordens ändar.
Job 37:4  Efteråt ryter så dånet, när han dundrar med sin väldiga röst; och på ljungeldarna spar han ej, då hans röst låter höra sig.
Job 37:5  Ja, underbart dundrar Gud med sin röst, stora ting gör han, utöver vad vi förstå.
Job 37:6  Se, åt snön giver han bud: "Fall ned till jorden", så ock åt regnskuren, åt sitt regnflödes mäktiga skur.
Job 37:7  Därmed fjättrar han alla människors händer, så att envar som han har skapat kan lära därav.
Job 37:8  Då draga sig vilddjuren in i sina gömslen, och i sina kulor lägga de sig till ro.
Job 37:9  Från Stjärngemaket kommer då storm och köld genom nordanhimmelens stjärnor;
Job 37:10  med sin andedräkt sänder Gud frost, och de vida vattnen betvingas.
Job 37:11  Skyarna lastar han ock med väta och sprider omkring sina ljungeldsmoln.
Job 37:12  De måste sväva än hit, än dit, alltefter hans rådslut och de uppdrag de få, vadhelst han ålägger dem på jordens krets.
Job 37:13  Än är det som tuktoris, än med hjälp åt hans jord, än är det med nåd som han låter dem komma.
Job 37:14  Lyssna då härtill, du Job; stanna och betänk Guds under.
Job 37:15  Förstår du på vad sätt Gud styr deras gång och låter ljungeldarna lysa fram ur sina moln?
Job 37:16  Förstår du lagen för skyarnas jämvikt, den Allvises underbara verk?
Job 37:17  Förstår du huru kläderna bliva dig så heta, när han låter jorden domna under sunnanvinden?
Job 37:18  Kan du välva molnhimmelen så som han, så fast som en spegel av gjuten metall?
Job 37:19  Lär oss då vad vi skola säga till honom; för vårt mörkers skull hava vi intet att lägga fram.
Job 37:20  Ej må det bebådas honom att jag vill tala. Månne någon begär sitt eget fördärv?
Job 37:21  Men synes icke redan skenet? Strålande visar han sig ju mellan skyarna, där vinden har gått fram och sopat dem undan.
Job 37:22  I guldglans kommer han från norden. Ja, Gud är höljd i fruktansvärt majestät;
Job 37:23  den Allsmäktige kunna vi icke fatta, honom som är så stor i kraft, honom som ej kränker rätten, ej strängaste rättfärdighet.
Job 37:24  Fördenskull frukta människorna honom; men de självkloka - dem alla aktar han ej på.
Job 38:1  Och HERREN svarade Job ur stormvinden och sade:
Job 38:2  Vem är du som stämplar vishet såsom mörker, i det att du talar så utan insikt?
Job 38:3  Omgjorda nu såsom ej man dina länder; jag vill fråga dig, och du må giva mig besked.
Job 38:4  Var var du, när jag lade jordens grund? Säg det, om du har ett så stort förstånd.
Job 38:5  Vem har fastställt hennes mått - du vet ju det? Och vem spände sitt mätsnöre ut över henne?
Job 38:6  Var fingo hennes pelare sina fästen, och vem var det som lade hennes hörnsten,
Job 38:7  medan morgonstjärnorna tillsammans jublade och alla Guds söner höjde glädjerop?
Job 38:8  Och vem satte dörrar för havet, när det föddes och kom ut ur moderlivet,
Job 38:9  när jag gav det moln till beklädnad och lät töcken bliva dess linda,
Job 38:10  när jag åt det utstakade min gräns och satte bom och dörrar därför,
Job 38:11  och sade: "Härintill skall du komma, men ej vidare, här skola dina stolta böljor lägga sig"?
Job 38:12  Har du i din tid bjudit dagen att gry eller anvisat åt morgonrodnaden dess plats,
Job 38:13  där den skulle fatta jorden i dess flikar, så att de ogudaktiga skakades bort därifrån?
Job 38:14  Då ändrar den form såsom leran under signetet, och tingen stå fram såsom klädda i skrud;
Job 38:15  då berövas de ogudaktiga sitt ljus, och den arm som lyftes för högt brytes sönder.
Job 38:16  Har du stigit ned till havets källor och vandrat omkring på djupets botten?
Job 38:17  Hava dödens portar avslöjat sig för dig, ja, såg du dödsskuggans portar?
Job 38:18  Har du överskådat jordens vidder? Om du känner allt detta, så låt höra.
Job 38:19  Vet du vägen dit varest ljuset bor, eller platsen där mörkret har sin boning,
Job 38:20  så att du kan hämta dem ut till deras gräns och finna stigarna som leda till deras hus?
Job 38:21  Visst kan du det, ty så tidigt blev du ju född, så stort är ju dina dagars antal!
Job 38:22  Har du varit framme vid snöns förrådshus? Och haglets förrådshus, du såg väl dem
Job 38:23  - de förråd som jag har sparat till hemsökelsens tid, till stridens och drabbningens dag?
Job 38:24  Vet du vägen dit varest ljuset delar sig, dit där stormen sprider sig ut över jorden?
Job 38:25  Vem har åt regnflödet öppnat en ränna och banat en väg för tordönets stråle,
Job 38:26  till att sända regn över länder där ingen bor, över öknar, där ingen människa finnes,
Job 38:27  till att mätta ödsliga ödemarker och giva växt åt gräsets brodd?
Job 38:28  Säg om regnet har någon fader, och vem han är, som födde daggens droppar?
Job 38:29  Ur vilken moders liv är det isen gick fram, och vem är hon som födde himmelens rimfrost?
Job 38:30  Se, vattnet tätnar och bliver likt sten, så ytan sluter sig samman över djupet.
Job 38:31  Knyter du tillhopa Sjustjärnornas knippe? Och förmår du att lossa Orions band?
Job 38:32  Är det du som, när tid är, för himmelstecknen fram, och som leder Björninnan med hennes ungar?
Job 38:33  Ja, förstår du himmelens lagar, och ordnar du dess välde över jorden?
Job 38:34  Kan du upphöja din röst till molnen och förmå vattenflöden att övertäcka dig?
Job 38:35  Kan du sända ljungeldar åstad, så att de gå, så att de svara dig: "Ja vi äro redo"?
Job 38:36  Vem har lagt vishet i de mörka molnen, och vem gav förstånd åt järtecknen i luften?
Job 38:37  Vem håller med sin vishet räkning på skyarna? Och himmelens läglar, vem häller ut dem,
Job 38:38  medan mullen smälter såsom malm och jordkokorna klibbas tillhopa?
Job 39:1  Är det du som jagar upp rov åt lejoninnan och stillar de unga lejonens hunger,
Job 39:2  när de trycka sig ned i sina kulor eller ligga på lur i snåret?
Job 39:3  Vem är det som skaffar mat åt korpen, när hans ungar ropar till Gud, där de sväva omkring utan föda?
Job 39:4  Vet du tiden för stengetterna att föda, vakar du över när hindarna bör kalva?
Job 39:5  Räknar du månaderna som de skola gå dräktiga, ja, vet du tiden för dem att föda?
Job 39:6  De böja sig ned, de avbörda sig sina foster, hastigt göra de sig fria ifrån födslovåndan.
Job 39:7  Deras ungar frodas och växa till på marken, så springa de sin väg och vända ej tillbaka.
Job 39:8  Vem har skänkt vildåsnan hennes frihet, vem har lossat den skyggas band?
Job 39:9  Se, hedmarken gav jag henne till hem, och saltöknen blev hennes boning.
Job 39:10  Hon ler åt larmet i staden, hon hör ingen pådrivares rop.
Job 39:11  Vad hon spanar upp på berget har hon till bete, hon letar efter allt som är grönt.
Job 39:12  Skall vildoxen finnas hågad att tjäna dig och att stanna över natten invid din krubba?
Job 39:13  Kan du tvinga vildoxen att gå i fåran efter töm och förmå honom att i ditt spår harva markerna jämna?
Job 39:14  Kan du lita på honom, då ju hans kraft är så stor, kan du betro åt honom ditt arbetes frukt?
Job 39:15  Överlåter du åt honom att föra hem din säd och att hämta den tillhopa till din loge?
Job 39:16  Strutshonans vingar flaxa med fröjd, men vad modersömhet visa väl hennes pennor, hennes fjädrar?
Job 39:17  Åt jorden överlåter hon ju sina ägg och ruvar dem ovanpå sanden.
Job 39:18  Hon bryr sig ej om att en fot kan krossa dem, att ett vilddjur kan trampa dem sönder.
Job 39:19  Hård är hon mot sin avkomma, såsom vore den ej hennes; att hennes avel kan gå under, det bekymrar henne ej.
Job 39:20  Ty Gud har gjort henne glömsk för vishet, han har ej tilldelat henne förstånd.
Job 39:21  Men när det gäller, piskar hon sig själv upp till språng; då ler hon åt både häst och man.
Job 39:22  Är det du som giver åt hästen hans styrka och kläder hans hals med brusande man?
Job 39:23  Är det du som lär honom gräshoppans språng? Hans stolta frustning, en förskräckelse är den!
Job 39:24  Han skrapar marken och fröjdar sig i sin kraft och rusar så fram mot väpnade skaror.
Job 39:25  Han ler åt fruktan och känner ej förfäran, han ryggar icke tillbaka för svärd.
Job 39:26  Omkring honom ljuder ett rassel av koger, av ljungande spjut och lans.
Job 39:27  Han skakas och rasar och uppslukar marken, han kan icke styra sig, när basunen har ljudit.
Job 39:28  För var basunstöt frustar han: Huj! Ännu i fjärran vädrar han striden, anförarnas rop och larmet av härskrin.
Job 39:29  Är det ett verk av ditt förstånd, att falken svingar sig upp och breder ut sina vingar till flykt mot söder?
Job 39:30  Eller är det på ditt bud som örnen stiger så högt och bygger sitt näste i höjden?
Job 39:31  På klippan bor han, där har han sitt tillhåll, på klippans spets och på branta berget.
Job 39:32  Därifrån spanar han efter sitt byte, långt bort i fjärran skådar hans ögon.
Job 39:33  Hans ungar frossa på blod, och där slagna ligga, där finner man honom.
Job 39:34  Så svarade nu HERREN Job och sade:
Job 39:35  Vill du tvista med den Allsmäktige, du mästare? Svara då, du som så klagar på Gud!
Job 39:36  Job svarade HERREN och sade:
Job 39:37  Nej, därtill är jag för ringa; vad skulle jag svara dig? Jag måste lägga handen på munnen.
Job 39:38  En gång har jag talat, och nu säger jag intet mer; ja, två gånger, men jag gör det icke åter.
Job 40:1  Och HERREN talade till Job ur stormvinden och sade:
Job 40:2  Omgjorda såsom en man dina länder; jag vill fråga dig, och du må giva mig besked.
Job 40:3  Vill du göra min rätt om intet och döma mig skyldig, för att själv stå såsom rättfärdig?
Job 40:4  Har du en sådan arm som Gud, och förmår du dundra med din röst såsom han?
Job 40:5  Pryd dig då med ära och höghet, kläd dig i majestät och härlighet.
Job 40:6  Gjut ut din vredes förgrymmelse, ödmjuka med en blick allt vad högt är.
Job 40:7  Ja, kuva med en blick allt vad högt är, slå ned de ogudaktiga på stället.
Job 40:8  Göm dem i stoftet allasammans, ja, fjättra deras ansikten i mörkret.
Job 40:9  Då vill jag prisa dig, också jag, för segern som din högra hand har berett dig.
Job 40:10  Se, Behemot, han är ju mitt verk såväl som du. Han lever av gräs såsom en oxe.
Job 40:11  Och se vilken kraft han äger i sina länder, vilken styrka han har i sin buks muskler.
Job 40:12  Han bär sin svans så styv som en ceder, ett konstrikt flätverk äro senorna i hans lår.
Job 40:13  Hans benpipor äro såsom rör av koppar, benen i hans kropp likna stänger av järn.
Job 40:14  Förstlingen är han av vad Gud har gjort; hans skapare själv har givit honom hans skära.
Job 40:15  Ty foder åt honom frambära bergen, där de vilda djuren alla hava sin lek.
Job 40:16  Under lotusträd lägger han sig ned, i skygdet av rör och vass.
Job 40:17  Lotusträd giva honom tak och skugga, pilträd hägna honom runt omkring.
Job 40:18  Är floden än så våldsam, så ängslas han dock icke; han är trygg, om ock en Jordan bryter fram mot hans gap.
Job 40:19  Vem kan fånga honom, när han är på sin vakt, vem borrar en snara genom hans nos?
Job 40:20  Kan du draga upp Leviatan med krok och med en metrev betvinga hans tunga?
Job 40:21  Kan du sätta en sävhank i hans nos eller borra en hake genom hans käft?
Job 40:22  Menar du att han skall slösa på dig många böner eller tala till dig med mjuka ord?
Job 40:23  Att han skall vilja sluta fördrag med dig, så att du finge honom till din träl för alltid?
Job 40:24  Kan du hava honom till leksak såsom en fågel och sätta honom i band åt dina tärnor?
Job 40:25  Pläga fiskarlag köpslå om honom och stycka ut hans kropp mellan krämare?
Job 40:26  Kan du skjuta hans hud full med spjut och hans huvud med fiskharpuner?
Job 40:27  Ja, försök att bära hand på honom du skall minnas den striden och skall ej föra så mer.
Job 40:28  Nej, den sådant vågar, hans hopp bliver sviket, han fälles till marken redan vid hans åsyn.
Job 41:1  Så oförvägen är ingen, att han törs reta denne. Vem vågar då sätta sig upp mot mig själv?
Job 41:2  Vem har först givit mig något, som jag alltså bör betala igen? Mitt är ju allt vad som finnes under himmelen.
Job 41:3  Jag vill ej höra upp att tala om hans lemmar, om huru väldig han är, och huru härligt han är danad.
Job 41:4  Vem mäktar rycka av honom hans pansar? Vem vågar sig in mellan hans käkars par?
Job 41:5  Hans gaps dörrar, vem vill öppna dem? Runtom hans tänder bor ju förskräckelse.
Job 41:6  Stolta sitta på honom sköldarnas rader; hopslutna äro de med fast försegling.
Job 41:7  Tätt fogar sig den ena intill den andra, icke en vindfläkt tränger in mellan dem.
Job 41:8  Var och en håller ihop med den nästa, de gripa in i varandra och skiljas ej åt.
Job 41:9  När han fnyser, strålar det av ljus; hans blickar äro såsom morgonrodnadens ögonbryn.
Job 41:10  Bloss fara ut ur hans gap, eldgnistor springa fram därur.
Job 41:11  Från hans näsborrar utgår rök såsom ur en sjudande panna på bränslet.
Job 41:12  Hans andedräkt framgnistrar eldkol, och lågor bryta fram ur hans gap.
Job 41:13  På hans hals har kraften sin boning, och framför honom stapplar försagdhet.
Job 41:14  Själva det veka på hans buk är ett stadigt fogverk, det sitter orubbligt, såsom gjutet på honom.
Job 41:15  Hans hjärta är fast såsom sten, fast såsom bottenstenen i kvarnen.
Job 41:16  När han reser sig, bäva hjältar, av ångest mista de all sans.
Job 41:17  Angripes han med ett svärd, så håller det ej stånd, ej heller spjut eller pil eller pansar.
Job 41:18  Han aktar järn såsom halm och koppar såsom murket trä.
Job 41:19  Bågskott skrämma honom ej bort, slungstenar förvandlas för honom till strå;
Job 41:20  ja, stridsklubbor aktar han såsom strå, han ler åt rasslet av lansar.
Job 41:21  På sin buk bär han skarpa eggar, spår såsom av en tröskvagn ristar han i dyn.
Job 41:22  Han gör djupet sjudande som en gryta, likt en salvokokares kittel förvandlar han vattnet.
Job 41:23  Bakom honom strålar vägen av ljus, djupet synes bära silverhår.
Job 41:24  Ja, på jorden finnes intet som är honom likt, otillgänglig för fruktan skapades han.
Job 41:25  På allt vad högt är ser han med förakt, konung är han över alla stolta vilddjur.
Job 42:1  Job svarade HERREN och sade:
Job 42:2  Ja, jag vet att du förmår allt, och att intet som du besluter är dig för svårt.
Job 42:3  Vem var då jag som i oförstånd gav vishet namn av mörker? Jag ordade ju om vad jag icke begrep, om det som var mig för underbart och det jag ej kunde förstå.
Job 42:4  Men hör nu, så vill jag tala; jag vill fråga dig, och du må giva mig besked.
Job 42:5  Blott hörsägner hade jag förnummit om dig, men nu har jag fått se dig med egna ögon.
Job 42:6  Därför tager jag det tillbaka och ångrar mig, i stoft och aska.
Job 42:7  Sedan HERREN hade talat så till Job, sade han till Elifas från Teman: "Min vrede är upptänd mot dig och dina båda vänner, därför att I icke haven talat om mig vad rätt är, såsom min tjänare Job har gjort.
Job 42:8  Så tagen eder nu sju tjurar och sju vädurar, och gån till min tjänare Job och offren dem såsom brännoffer för eder; ock låten min tjänare Job bedja för eder. Till äventyrs skall jag då, av nåd mot honom, avstå från att göra något förskräckligt mot eder, till straff därför att I icke haven talat om mig vad rätt är, såsom min tjänare Job har gjort."
Job 42:9  Då gingo Elifas från Teman, Bildad från Sua och Sofar från Naaman åstad och gjorde såsom HERREN hade tillsagt dem; och HERREN tog nådigt emot Jobs bön.
Job 42:10  Och då nu Job bad för sina vänner, upprättade HERREN åter honom själv; HERREN gav Job dubbelt igen mot vad han förut hade haft.
Job 42:11  Och alla hans bröder och systrar och alla hans forna bekanta kommo till honom och höllo måltid med honom i hans hus, och ömkade honom för alla de olyckor som HERREN hade låtit komma över honom. Och de gåvo honom vardera en kesita och en guldring.
Job 42:12  Och HERREN välsignade slutet av Jobs levnad ännu mer än begynnelsen, så att han fick fjorton tusen får, sex tusen kameler, ett tusen par oxar och ett tusen åsninnor.
Job 42:13  Och han fick sju söner och tre döttrar.
Job 42:14  Den första dottern kallade han Jemima, den andra Kesia och den tredje Keren-Happuk.
Job 42:15  Och så sköna kvinnor som Jobs döttrar funnos icke i hela landet; och deras fader gav dem arvedel bland deras bröder.
Job 42:16  Och Job levde därefter ett hundra fyrtio år, och fick se sina barn och barnbarn i fyra led.
Job 42:17  Sedan dog Job, gammal och mätt på att leva.


\end{document}