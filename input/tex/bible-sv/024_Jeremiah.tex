\begin{document}

\title{Jeremiah}

Jer 1:1  Detta är vad som talades av Jeremia, Hilkias son, en av prästerna i Anatot i Benjamins land.
Jer 1:2  Till honom kom HERRENS ord i Josias, Amons sons, Juda konungs, tid, i hans trettonde regeringsår,
Jer 1:3  Och sedan i Jojakims, Josias sons, Juda konungs, tid, intill slutet av Sidkias, Josias sons, Juda konungs, elfte regeringsår, då Jerusalems invånare i femte månaden fördes bort i fångenskap.
Jer 1:4  HERRENS ord kom till mig; han sade:
Jer 1:5  "Förrän jag danade dig i moderlivet, utvalde jag dig, och förrän du utgick ur modersskötet, helgade jag dig; jag satte dig till en profet för folken."
Jer 1:6  Men jag svarade: "Ack Herre HERRE! Se, jag förstår icke att tala, ty jag är för ung.
Jer 1:7  Då sade HERREN till mig: "Säg icke: 'Jag är för ung', utan gå åstad vart jag än sänder dig, och tala vad jag än bjuder dig.
Jer 1:8  Frukta icke för dem; ty jag är med dig och vill hjälpa dig, säger HERREN."
Jer 1:9  Och HERREN räckte ut sin hand och rörde vid min mun; och HERREN sade till mig: "Se, jag lägger mina ord i din mun.
Jer 1:10  Ja, jag sätter dig i dag över folk och riken, för att du skall upprycka och nedbryta, förgöra och fördärva, uppbygga och plantera."
Jer 1:11  Och HERRENS ord kom till mig; han sade: "Vad ser du, Jeremia?" Tag svarade: "Jag ser en gren av ett mandelträd."
Jer 1:12  Och HERREN sade till mig: "Du har sett rätt, ty jag skall vaka över mitt ord och låta det gå i fullbordan."
Jer 1:13  Och HERRENS ord kom till mig för andra gången; han sade: "Vad ser du?" Jag svarade: "Jag ser en sjudande gryta; den synes åt norr till."
Jer 1:14  Och HERREN sade till mig: "Ja, från norr skall olyckan bryta in över alla landets inbyggare.
Jer 1:15  Ty se, jag skall kalla på alla folkstammar i rikena norrut, säger HERREN; och de skola komma och resa upp var och en sitt säte vid ingången till Jerusalems portar och mot alla dess murar, runt omkring, och mot alla Juda städer.
Jer 1:16  Och jag skall gå till rätta med dem för all deras ondska, därför att de hava övergivit mig och tänt offereld åt andra gudar och tillbett sina händers verk.
Jer 1:17  Så omgjorda nu du dina länder, och stå upp och tala till dem allt vad Jag bjuder dig. Var icke förfärad för dem, på det att jag icke må låta vad förfärligt är komma över dig inför dem.
Jer 1:18  Ty se, jag själv gör dig i dag till en fast stad och till en järnpelare och en kopparmur mot hela landet, mot Juda konungar, mot dess furstar, mot dess präster och mot det meniga folket,
Jer 1:19  så att de icke skola bliva dig övermäktiga, om de vilja strida mot dig; ty jag är med dig, säger HERREN, och jag vill hjälpa dig."
Jer 2:1  Och HERRENS ord kom till mig; han sade:
Jer 2:2  Gå åstad och predika för Jerusalem och säg: Så säger HERREN: Jag kommer ihåg, dig till godo, din ungdoms kärlek, huru du älskade mig under din brudtid, huru du följde mig i öknen, i landet där man intet sår.
Jer 2:3  Ja, en HERRENS heliga egendom är Israel, förstlingen av hans skörd; alla som vilja äta därav ådraga sig skuld, olycka kommer över dem, säger HERREN.
Jer 2:4  Hören HERRENS ord, I av Jakobs hus, I alla släkter av Israels hus.
Jer 2:5  Så säger HERREN: Vad orätt funno edra fäder hos mig, eftersom de gingo bort ifrån mig och följde efter fåfängliga avgudar och bedrevo fåfänglighet?
Jer 2:6  De frågade icke: "Var är HERREN, han som förde oss upp ur Egyptens land, han som ledde oss i öknen, det öde och oländiga landet, torrhetens och dödsskuggans land, det land där ingen vägfarande färdades, och där ingen människa bodde?"
Jer 2:7  Och jag förde eder in i det bördiga landet, och I fingen äta av dess frukt och dess goda. Men när I haden kommit ditin, orenaden I mitt land och gjorden min arvedel till en styggelse.
Jer 2:8  Prästerna frågade icke: "Var är HERREN?" De som hade lagen om händer ville icke veta av mig, och herdarna avföllo från mig; profeterna profeterade i Baals namn och följde efter sådana som icke kunde hjälpa.
Jer 2:9  Därför skall jag än vidare gå till rätta med eder, säger HERREN, ja, ännu med edra barnbarn skall jag gå till rätta.
Jer 2:10  Dragen bort till kittéernas öländer och sen efter, sänden bud till Kedar och forsken noga efter; sen till, om något sådant där har skett.
Jer 2:11  Har väl något hednafolk bytt bort sina gudar? Och dock äro dessa inga gudar. Men mitt folk har bytt bort sin ära mot en avgud som icke kan hjälpa.
Jer 2:12  Häpnen häröver, I himlar; förskräckens och bäven storligen, säger HERREN.
Jer 2:13  Ty mitt folk har begått en dubbel synd: mig hava de övergivit, en källa med friskt vatten, och de hava gjort sig brunnar, usla brunnar, som icke hålla vatten.
Jer 2:14  Är väl Israel en träl eller en hemfödd slav, eftersom han så har lämnats till plundring?
Jer 2:15  Lejon ryta mot honom, de låta höra sitt skri. De göra hans land till en ödemark, hans städer brännas upp, så att ingen kan bo i dem.
Jer 2:16  Till och med Nofs och Tapanhes' barn avbeta dina berg.
Jer 2:17  Men är det ej du själv som vållar dig detta, därmed att du övergiver HERREN din Gud, när han vill leda dig på den rätta vägen?
Jer 2:18  Varför vill du nu gå till Egypten och dricka av Sihors vatten? Och varför vill du gå till Assyrien och dricka av flodens vatten?
Jer 2:19  Det är din ondska som bereder dig tuktan, det är din avfällighet som ådrager dig straff. Märk därför och besinna vilken olycka och sorg det har med sig att du övergiver HERREN, din Gud, och icke vill frukta mig, säger Herren, HERREN Sebaot.
Jer 2:20  Ty för länge sedan bröt du sönder ditt ok och slet av dina band och sade: "Jag vill ej tjäna." Och på alla höga kullar och under alla gröna träd lade du dig ned för att öva otukt.
Jer 2:21  Jag hade ju planterat dig såsom ett ädelt vinträd av alltigenom äkta art; huru har du då kunnat förvandlas för mig till vilda rankor av ett främmande vinträd?
Jer 2:22  Ja, om du ock tvår dig med lutsalt och tager än så mycken såpa, så förbliver dock din missgärning oren inför mig, säger Herren, HERREN.
Jer 2:23  Huru kan du säga: "Jag har ej orenat mig, jag har icke följt efter Baalerna"? Besinna vad du har bedrivit i dalen, ja, betänk vad du har gjort. Du är lik ett ystert kamelsto, som löper hit och dit.
Jer 2:24  Du är lik en vildåsna, fostrad i öknen, en som flåsar i sin brunst, och vars brånad ingen kan stävja; om någon vill till henne, behöver han ej löpa sig trött; när hennes månad kommer, träffar man henne.
Jer 2:25  Akta din fot, så att den icke tappar skon, och din strupe, så att den ej bliver torr av törst. Men du svarar: "Du mödar dig förgäves. Nej, jag älskar de främmande, och efter dem vill jag följa."
Jer 2:26  Såsom tjuven står där med skam, när han ertappas, så skall Israels hus komma på skam, med sina konungar, och furstar, med sina präster och profeter,
Jer 2:27  dessa som säga till trästycket: "Du är min fader", och säga till stenen: "Du har fött mig." Ty de vända ryggen till mig och icke ansiktet; men när olycka är på färde, ropa de: "Upp och fräls oss!"
Jer 2:28  Var äro då dina gudar, de som du gjorde åt dig? Må de stå upp. Kunna de frälsa dig i din olyckas tid? Ty så många som dina städer äro, så många hava dina gudar blivit, du Juda.
Jer 2:29  Huru kunnen I gå till rätta med mig? I haven ju alla avfallit från mig, säger HERREN.
Jer 2:30  Förgäves har jag slagit edra barn; de hava icke velat taga emot tuktan. Edert svärd har förtärt edra profeter, såsom vore det ett förhärjande lejon.
Jer 2:31  Du onda släkte, giv akt på HERRENS ord. Har jag då för Israel varit en öken eller ett mörkrets land, eftersom mitt folk säger: "Vi hava gjort oss fria, vi vilja ej mer komma till dig"?
Jer 2:32  Icke förgäter en jungfru sina smycken eller en brud sin gördel? Men mitt folk har förgätit mig sedan urminnes tid.
Jer 2:33  Huru skickligt går du icke till väga, när du söker älskog! Därför har du ock blivit förfaren på det ondas vägar.
Jer 2:34  Ja, på dina mantelflikar finner man blod av arma och oskyldiga, som du har dödat, icke därför att de ertappades vid inbrott, nej, därför att din håg står till allt sådant.
Jer 2:35  Och dock säger du: "Jag går fri ifrån straff; hans vrede mot mig har förvisso upphört." Nej, jag vill gå till rätta med dig, om du än säger: "Jag har icke syndat."
Jer 2:36  Varför har du nu så brått att vandra åstad på en annan väg? Också med Egypten skall du komma på skam, likasom du kom på skam med Assyrien.
Jer 2:37  Också därifrån skall du få gå din väg, med händerna på huvudet. Ty HERREN förkastar dem som du förlitar dig på, och du skall icke bliva lyckosam med dem.
Jer 3:1  Det är sagt: Om en man skiljer sig från sin hustru, och hon så går bort ifrån honom och bliver en annan mans hustru, icke får han då åter komma tillbaka till henne? Bleve icke då det landet ohelgat? Och du, som har bedrivit otukt med så många älskare, du vill ändå få komma tillbaka till mig! säger HERREN.
Jer 3:2  Lyft upp dina ögon till höjderna och se: var lät du icke skända dig? Vid vägarna satt du och spejade efter dem, såsom en arab i öknen, och ohelgade landet genom din otukt och genom din ondska.
Jer 3:3  Väl blevo regnskurarna förhållna, och intet vårregn föll; men du hade en äktenskapsbryterskas panna, du ville icke blygas.
Jer 3:4  Och ändå har du nyss ropat till mig: "Min fader!", Min ungdoms vän är du!"
Jer 3:5  "Skulle han kunna behålla vrede evinnerligen, skulle han framhärda så för alltid?" Så talar du och gör dock vad ont är, ja, fullbordar det ock.
Jer 3:6  Och HERREN sade till mig i konung Josias tid: Har du sett vad Israel, den avfälliga kvinnan, har gjort? Hon gick upp på alla höga berg och bort under alla gröna träd och bedrev där otukt.
Jer 3:7  Och jag tänkte att sedan hon hade gjort allt detta, skulle hon vända tillbaka till mig. Men hon vände icke tillbaka. Och hennes syster Juda, den trolösa kvinnan, såg det.
Jer 3:8  Och jag såg, att fastän jag hade skilt mig från Israel, den avfälliga, och givit henne skiljebrev just för hennes äktenskapsbrotts skull, så skrämdes dock hennes syster Juda den trolösa, icke därav, utan gick likaledes åstad och bedrev otukt
Jer 3:9  och ohelgade så landet genom sin lättfärdiga otukt, i det hon begick äktenskapsbrott med sten och trä.
Jer 3:10  Ja, oaktat allt detta vände hennes syster Juda, den trolösa, icke tillbaka till mig av fullt hjärta, utan allenast med skrymteri, säger HERREN.
Jer 3:11  Och HERREN sade till mig: Israel, den avfälliga, har bevisat sig rättfärdigare än Juda, den trolösa.
Jer 3:12  Gå bort och predika så norrut och säg: Vänd om, Israel, du avfälliga, säger HERREN, så vill jag icke längre med ogunst se på eder; ty jag är nådig, säger HERREN, jag behåller icke vrede evinnerligen.
Jer 3:13  Allenast må du besinna din missgärning, att du har varit avfällig från HERREN, din Gud, och lupit hit och dit till främmande gudar under alla gröna träd; ja, I haven icke velat höra min röst, säger HERREN.
Jer 3:14  Vänden om, I avfälliga barn, säger HERREN, ty jag är eder rätte herre; så vill jag hämta eder, en från var stad och två från var släkt, och föra eder till Sion.
Jer 3:15  Och jag vill giva eder herdar efter mitt hjärta, och de skola föra eder i bet med förstånd och insikt.
Jer 3:16  Och det skall ske, att när I på den tiden föröken eder och bliven fruktsamma i landet, säger HERREN, då skall man icke mer tala om HERRENS förbundsark eller tänka på den; man skall icke komma ihåg den eller sakna den, och man skall icke göra någon ny sådan.
Jer 3:17  Utan på den tiden skall man kalla Jerusalem "HERRENS tron"; och dit skola församla sig alla hednafolk, till HERRENS namn i Jerusalem. Och de skola icke mer vandra efter sina onda hjärtans hårdhet.
Jer 3:18  På den tiden skall Juda hus gå till Israels hus, och tillsammans skola de komma från nordlandet in i det land som jag gav edra fäder till arvedel.
Jer 3:19  Jag tänkte: "Vilken plats skall jag ej förläna dig bland barnen, och vilket ljuvligt land skall jag icke giva dig, den allra härligaste arvedel bland folken!" Och jag tänkte: "Då skolen I kalla mig fader och icke mer vika bort ifrån mig."
Jer 3:20  Men såsom när en hustru är trolös mot sin make, så haven I av Israels hus varit trolösa mot mig, säger HERREN.
Jer 3:21  Därför höras rop på höjderna, gråt och böner av Israels barn; ty de hava gått på förvända vägar och förgätit HERREN, sin Gud.
Jer 3:22  Så vänden nu om, I avfälliga barn, så vill jag hela eder från edert avfall. Ja se, vi komma till dig, ty du är HERREN, vår Gud.
Jer 3:23  Sannerligen, bedrägligt var vårt hopp till höjderna, blott tomt larm gåvo oss bergen. Sannerligen, det är hos HERREN, vår Gud, som frälsning finnes för Israel.
Jer 3:24  Men skändlighetsguden har förtärt frukten av våra fäders arbete, allt ifrån vår ungdom, deras får och fäkreatur, deras söner och döttrar.
Jer 3:25  Så vilja vi nu ligga här i vår skam, och blygd må hölja oss. Ty mot HERREN, vår Gud, hava vi syndat, vi och våra fäder, ifrån vår ungdom ända till denna dag; vi hava icke velat höra HERRENS, vår Guds, röst.
Jer 4:1  om du omvänder dig, Israel, säger HERREN, skall du få vända tillbaka till mig; och om du skaffar bort dina styggelser från min åsyn, skall du slippa vandra flyktig omkring.
Jer 4:2  Då skall du svärja i sanning, rätt och rättfärdighet: "Så sant HERREN lever", och hednafolken skola välsigna sig i honom och berömma sig av honom.
Jer 4:3  Ja, så säger HERREN till Juda män och till Jerusalem: Bryten eder ny mark, och sån ej bland törnen.
Jer 4:4  Omskären eder åt HERREN; skaffen bort edert hjärtas förhud, I Juda män och I Jerusalems invånare. Eljest skall min vrede bryta fram såsom en eld och brinna så, att ingen kan utsläcka den, för edert onda väsendes skull.
Jer 4:5  Förkunnen i Juda, kungören i Jerusalem och påbjuden, ja, stöten i basun i landet, ropen ut med hög röst och sägen: "Församlen eder och låt oss fly in i de befästa städerna."
Jer 4:6  Resen upp ett baner som visar åt Sion, bärgen edert gods och dröjen icke ty jag skall låta olycka komma från norr, med stor förstöring.
Jer 4:7  Ett lejon drager fram ur sitt snår och en folkfördärvare bryter upp, han går ut ur sin boning, för att göra ditt land till en ödemark; då bliva dina städer förstörda, så att ingen kan bo i dem.
Jer 4:8  Så höljen eder nu i sorgdräkt, klagen och jämren eder, ty HERRENS vredes glöd upphör icke över oss.
Jer 4:9  På den tiden, säger HERREN, skall det vara förbi med konungens och furstarnas mod, och prästerna skola bliva förfärade och profeterna stå häpna.
Jer 4:10  Men jag sade: "Ack Herre, HERRE, svårt bedrog du sannerligen detta folk och Jerusalem, då du sade: "Det skall gå eder väl." Svärdet är ju nära att taga vårt liv.
Jer 4:11  På den tiden skall det sägas om detta folk och om Jerusalem: En brännande vind från höjderna i öknen kommer emot dottern mitt folk, icke en sådan vind som passar, när man kastar säd eller rensar korn;
Jer 4:12  nej, en våldsammare vind än som så låter jag komma. Ja, nu vill jag gå till rätta med dem!
Jer 4:13  Se, såsom ett moln kommer han upp och såsom en stormvind äro hans vagnar; hans hästar äro snabbare än örnar, ve oss, vi äro förlorade!
Jer 4:14  Så två nu ditt hjärta rent från ondska, Jerusalem, för att du må bliva frälst. Huru länge skola fördärvets tankar bo i ditt bröst?
Jer 4:15  Från Dan höres ju en budbärare ropa, och från Efraims bergsbygd en som bådar fördärv.
Jer 4:16  Förkunnen för folken, ja, kungören över Jerusalem att en belägringshär kommer ifrån fjärran land och häver upp sitt rop mot Juda städer.
Jer 4:17  Såsom väktare kring ett åkerfält samla de sig runt omkring henne, därför att hon har varit gensträvig mot mig, säger HERREN.
Jer 4:18  Ja, ditt eget leverne och dina egna varningar vålla dig detta; det är din ondskas frukt att det bliver dig så bittert, och att plågan träffar dig ända in i hjärtat.
Jer 4:19  I mitt innersta våndas jag, i mitt hjärtas djup. Mitt hjärta klagar i mig, jag kan icke tiga, ty basunljud hör du, min själ, och krigiskt härskri.
Jer 4:20  Olycka efter olycka ropas ut, ja, hela landet bliver förött; plötsligt bliva mina hyddor förödda, i ett ögonblick mina tält.
Jer 4:21  Huru länge skall jag se stridsbaneret och höra basunljud?
Jer 4:22  Ja, mitt folk är oförnuftigt, de vilja ej veta av mig. De äro dåraktiga barn och hava intet förstånd. Visa äro de till att göra vad ont är, men att göra vad gott är förstå de ej.
Jer 4:23  Jag såg på jorden, och se, den var öde och tom, och upp mot himmelen, och där lyste intet ljus.
Jer 4:24  Jag såg på bergen, och se, de bävade, och alla höjder vacklade.
Jer 4:25  Jag såg mig om, och då fanns där ingen människa, och alla himmelens fåglar hade flytt bort.
Jer 4:26  Jag såg mig om, och då var det bördiga landet en öken, och alla dess städer voro nedbrutna, för HERRENS ansikte, för hans vredes glöd,
Jer 4:27  Ty så säger HERREN: Hela landet skall bliva en ödemark, om jag än ej alldeles vill göra ände därpå.
Jer 4:28  Därför sörjer jorden, och himmelen därovan kläder sig i sorgdräkt, därför att jag så har talat och beslutit och ej kan ångra det eller taga det tillbaka.
Jer 4:29  För larmet av ryttare och bågskyttar tager hela staden till flykten. Man giver sig in i skogssnåren och upp bland klipporna. Alla städer äro övergivna, ingen människa bor mer i dem.
Jer 4:30  Vad vill du göra i din förödelse? Om du än kläder dig i scharlakan, om du än pryder dig med gyllene smycken om du än söker förstora dina ögon genom smink, så gör du dig dock skön förgäves. Dina älskare förakta dig, ja, de stå efter ditt liv.
Jer 4:31  Ty jag hör rop såsom av en barnaföderska, nödrop såsom av en förstföderska. Det är dottern Sion som ropar; hon flämtar, hon räcker ut sina händer: "Ack, ve mig! I mördares våld försmäktar min själ."
Jer 5:1  Gån omkring på gatorna i Jerusalem, och sen till och given akt; söken på dess torg om I finnen någon, om där är någon som gör rätt och beflitar sig om sanning; då vill jag förlåta staden.
Jer 5:2  Men säga de än: "Så sant HERREN lever", så svärja de dock falskt.
Jer 5:3  HERRE, är det ej sanning dina ögon söka? Du slog dem, men de kände ingen sveda. Du förgjorde dem, men de ville ej taga emot tuktan. De gjorde sina pannor hårdare än sten, de ville icke omvända sig.
Jer 5:4  Då tänkte jag: "Detta är allenast de ringa i folket; de äro dåraktiga, ty de känna icke HERRENS väg, sin Guds rätt.
Jer 5:5  Jag vill nu gå till de stora och tala med dem; de måste ju känna HERRENS väg, sin Guds rätt." Men ock dessa hade alla brutit sönder oket och slitit av banden.
Jer 5:6  Därför bliva de slagna av lejonet från skogen och fördärvade av vargen från hedmarken; pantern lurar vid deras städer, och envar som vågar sig ut därifrån bliver ihjälriven. Ty många äro deras överträdelser och talrika deras avfällighetssynder.
Jer 5:7  Huru skulle jag då kunna förlåta dig? Dina barn hava ju övergivit mig och svurit vid gudar som icke äro gudar. Jag gav dem allt till fyllest, men de blevo mig otrogna och samlade sig i skaror till skökohuset.
Jer 5:8  De likna välfödda, ystra hästar; de vrenskas var och en efter sin nästas hustru.
Jer 5:9  Skulle jag icke för sådant hemsöka dem? säger HERREN. Och skulle icke min själ hämnas på ett sådant folk som detta är?
Jer 5:10  Stormen då hennes murar och för stören dem, dock utan att alldeles göra ände på henne. Riven bort hennes vinrankor, de äro ju icke HERRENS.
Jer 5:11  Ty både Israels hus och Juda hus hava varit mycket trolösa mot mig, säger HERREN.
Jer 5:12  De hava förnekat HERREN och sagt: "Han betyder intet. Olycka skall icke komma över oss, svärd och hunger skola vi icke se.
Jer 5:13  Men profeterna skola försvinna såsom en vind, och han som säges tala är icke i dem; dem själva skall det så gå."
Jer 5:14  Därför säger HERREN, härskarornas Gud: Eftersom I fören ett sådant tal, se, därför skall jag göra mina ord i din mun till en eld, och detta folk till ved, och elden skall förtära dem.
Jer 5:15  Se, jag skall låta komma över eder, I av Israels hus, ett folk ifrån fjärran land, säger HERREN, ett starkt folk, ett urgammalt folk, ett folk vars tungomål du icke känner, och vars tal du icke förstår.
Jer 5:16  Deras koger är en öppen grav; de äro allasammans hjältar.
Jer 5:17  De skola förtära din skörd och ditt bröd, de skola förtära dina söner och döttrar, de skola förtära dina får och fäkreatur, de skola förtära dina vinträd och fikonträd. Dina befästa städer, som du förlitar dig på, dem skola de förstöra med svärd.
Jer 5:18  Dock vill jag på den tiden, säger HERREN, icke alldeles göra ände på eder.
Jer 5:19  Och om I då frågen: "Varför har HERREN, vår Gud, gjort oss allt detta?", så skall du svara dem: "Såsom I haven övergivit mig och tjänat främmande gudar i edert eget land, så skolen I nu få tjäna främlingar i ett land som icke är edert."
Jer 5:20  Förkunnen detta i Jakobs hus, kungören det i Juda och sägen:
Jer 5:21  Hör detta, du dåraktiga och oförståndiga folk, I som haven ögon, men icke sen, I som haven öron, men icke hören.
Jer 5:22  Skullen I icke frukta mig, säger HERREN, skullen I icke bäva för mig, for mig som har satt stranden till en damm for havet, till en evärdlig gräns, som det icke kan överskrida, så att dess böljor, huru de än svalla, ändå intet förmå, och huru de än brusa, likväl icke kunna överskrida den?
Jer 5:23  Men detta folk har ett gensträvigt och upproriskt hjärta; de hava avfallit och gått sin väg.
Jer 5:24  De sade icke i sina hjärtan: "Låtom oss frukta HERREN, vår Gud, honom som giver regn i rätt tid, både höst och vår, och som ständigt beskär oss de bestämda skördeveckorna."
Jer 5:25  Edra missgärningar hava nu fört dessa i olag, och edra synder hava förhållit för eder detta goda.
Jer 5:26  Ty bland mitt folk finnas ogudaktiga människor: de ligga i försåt, likasom fågelfängaren ligger på lur, de sätta ut giller till att fånga människor.
Jer 5:27  Såsom när en bur är full av fåglar, så äro deras hus fulla av svek. Därigenom hava de blivit så stora och rika; de hava blivit feta och skinande.
Jer 5:28  För sina ogärningar veta de icke av någon gräns, de hålla icke rätten vid makt, icke den faderlöses rätt, till att främja den; och i den fattiges sak fälla de icke rätt dom.
Jer 5:29  Skulle jag icke för sådant hemsöka dem? säger HERREN. Skulle icke min själ hämnas på ett sådant folk som detta är?
Jer 5:30  Förfärliga och gruvliga ting ske i landet.
Jer 5:31  Profeterna profetera lögn, och prästerna styra efter deras råd; och mitt folk vill så hava det. Men vad skolen I göra, när änden på detta kommer?
Jer 6:1  Bärgen edert gods ut ur Jerusalem, I Benjamins barn, stöten i basun i Tekoa, och resen upp ett högt baner ovanför Bet-Hackerem; ty en olycka hotar från norr, med stor förstöring.
Jer 6:2  Hon som är så fager och förklemad, dottern Sion, henne skall jag förgöra.
Jer 6:3  Herdar skola komma över henne med sina hjordar; de skola slå upp sina tält runt omkring henne, avbeta var och en sitt stycke.
Jer 6:4  Ja, invigen eder till strid mot henne. "Upp, låt oss draga åstad, medan middagsljuset varar! Ack att dagen redan lider till ända! Ack att aftonens skuggor förlängas!
Jer 6:5  Välan, så låt oss draga ditupp om natten och förstöra hennes palatser."
Jer 6:6  Ty så säger HERREN Sebaot: Fällen träd och kasten upp vallar emot Jerusalem. Hon är staden som skall hemsökas, hon som i sig har idel förtryck
Jer 6:7  Likasom en brunn låter vatten välla fram, så låter hon ondska framvälla. Våld och förödelse hör man där, sår och slag äro beständigt inför min åsyn.
Jer 6:8  Låt varna dig, Jerusalem, så att min själ ej vänder sig ifrån dig, så att jag icke gör dig till en ödemark, till ett obebott land.
Jer 6:9  Så säger HERREN Sebaot: En efterskörd, likasom på ett vinträd, skall man hålla på kvarlevan av Israel. Räck ut din hand åter och åter, såsom när man plockar av druvor från rankorna.
Jer 6:10  Men inför vem skall jag tala och betyga för att bliva hörd? Se, deras öron äro oomskurna, så att de icke kunna höra. Ja, HERRENS ord har blivit till smälek bland dem; de hava intet behag därtill.
Jer 6:11  Därför är jag uppfylld av HERRENS vrede, jag förmår icke hålla den inne. Utgjut den över barnen på gatan och över alla de unga männens samkväm; ja, både man och kvinna skola drabbas därav, jämväl den gamle och den som har fyllt sina dagars mått.
Jer 6:12  Och deras hus skola gå över i främmandes ägo så ock deras åkrar och deras hustrur, ty jag vill uträcka min hand mot landets inbyggare, säger HERREN.
Jer 6:13  Ty alla, både små och stora, söka där orätt vinning, och både profeter och präster fara allasammans med lögn,
Jer 6:14  de taga det lätt med helandet av mitt folks skada; de säga: "Allt står väl till, allt står väl till", och dock står icke allt väl till.
Jer 6:15  De skola komma på skam, ty de övade styggelse. Likväl känna de alls icke någon skam och veta icke av någon blygsel. Därför skola de falla bland de andra; när min hemsökelse träffar dem, skola de komma på fall, säger HERREN,
Jer 6:16  Så sade HERREN: "Ställen eder vid vägarna och sen till, och frågen efter forntidens stigar, frågen vilken väg som är den goda vägen, och vandren på den, så skolen I finna ro för edra själar." Men de svarade: "Vi vilja icke vandra på den."
Jer 6:17  Och när jag då satte väktare över eder och sade: "Akten på basunens ljud", svarade de: "Vi vilja icke akta därpå."
Jer 6:18  Hören därför, I hednafolk, och märk, du menighet, vad som sker bland dem.
Jer 6:19  Ja hör, du jord: Se, jag skall låta olycka komma över detta folk, såsom en frukt av deras anslag, eftersom de icke akta på mina ord, utan förkasta min lag.
Jer 6:20  Vad frågar jag efter rökelse, komme den ock från Saba, eller efter bästa kalmus ifrån fjärran land? Edra brännoffer täckas mig icke, och edra slaktoffer behaga mig icke.
Jer 6:21  Därför säger HERREN så: Se, jag skall lägga stötestenar för detta folk; och genom dem skola både fader och söner komma på fall, den ene borgaren skall förgås med den andre.
Jer 6:22  Så säger HERREN: Se, ett folk kommer från nordlandet, ett stort folk reser sig vid jordens yttersta ända.
Jer 6:23  De föra båge och lans, de äro grymma och utan förbarmande. Dånet av dem är såsom havets brus, och på sina hästar rida de fram, rustade såsom kämpar till strid, mot dig, du dotter Sion.
Jer 6:24  När vi höra ryktet om dem, sjunka våra händer ned, ängslan griper oss, ångest lik en barnaföderskas.
Jer 6:25  Gå icke ut på marken, och vandra ej på vägen, ty fienden bär svärd; skräck från alla sidor!
Jer 6:26  Du dotter mitt folk, höll dig i sorgdräkt, vältra dig i aska, höj sorgelåt likasom efter ende sonen, och håll bitter dödsklagan; ty plötsligt kommer förhärjaren över oss.
Jer 6:27  Jag har satt dig till en proberare i mitt folk - såväl som till ett fäste - på det att du må lära känna och pröva deras väg.
Jer 6:28  De äro allasammans avfälliga och gensträviga, de gå med förtal, de äro koppar och järn, allasammans äro de fördärvliga människor.
Jer 6:29  Blåsbälgen pustar, men ur elden kommer allenast bly fram; allt luttrande är förgäves, slagget bliver ändå icke frånskilt.
Jer 6:30  "Ett silver som må kastas bort", så kan man kalla dem, ty HERREN har förkastat dem.
Jer 7:1  Detta är det ord som kom till Jeremia från HERREN; han sade:
Jer 7:2  Ställ dig i porten till HERRENS hus, och predika där detta ord och säg: Hören HERRENS ord, I alla av Juda, som gån in genom dessa portar för att tillbedja HERREN.
Jer 7:3  Så säger HERREN Sebaot, Israels Gud: Bättren edert leverne och edert väsende, så vill jag låta eder bo kvar på denna plats.
Jer 7:4  Förliten eder icke på lögnaktigt tal, när man säger: "Här är HERRENS tempel, HERRENS tempel, HERRENS tempel!"
Jer 7:5  Nej, om I bättren edert leverne och edert väsende, om I dömen rätt mellan man och man,
Jer 7:6  om I upphören att förtrycka främlingen, den faderlöse och änkan, att utgjuta oskyldigt blod på denna plats och att följa efter andra gudar, eder själva till olycka,
Jer 7:7  då vill jag för evärdliga tider låta eder bo på denna plats, i det land som jag har givit åt edra fäder.
Jer 7:8  Men se, I förliten eder på lögnaktigt tal, som icke kan hjälpa.
Jer 7:9  Huru är det? I stjälen, mörden och begån äktenskapsbrott, I svärjen falskt, I tänden offereld åt Baal och följen efter andra gudar, som I icke kännen;
Jer 7:10  sedan kommen I hit och träden fram inför mitt ansikte i detta hus, som är uppkallat efter mitt namn, och sägen: "Med oss är ingen nöd" - för att därefter fortfara med alla dessa styggelser.
Jer 7:11  Hållen I det då för en rövarkula, detta hus, som är uppkallat efter mitt namn? Ja, sannerligen, också jag anser det så, säger HERREN.
Jer 7:12  Gån bort till den plats i Silo, där jag först lät mitt namn bo, och sen huru jag har gjort med den, för mitt folk Israels ondskas skull.
Jer 7:13  Och eftersom I haven gjort alla dessa gärningar, säger HERREN, och icke haven velat höra, fastän jag titt och ofta har talat till eder, och icke haven velat svara, fastän jag har ropat på eder,
Jer 7:14  därför vill jag nu med detta hus, som är uppkallat efter mitt namn, och som I förliten eder på, och med denna plats, som jag har givit åt eder och edra fäder, göra såsom jag gjorde med Silo.
Jer 7:15  Och jag skall kasta eder bort ifrån mitt ansikte, såsom jag har bortkastat alla edra bröder all Efraims släkt.
Jer 7:16  Så må du nu icke bedja för detta folk eller frambära någon klagan och förbön för dem eller lägga dig ut för dem hos mig, ty jag vill icke höra dig.
Jer 7:17  Ser du icke vad de göra i Juda städer och på Jerusalems gator?
Jer 7:18  Barnen samla tillhopa ved, fäderna tända upp eld och kvinnorna knåda deg, allt för att baka offerkakor åt himmelens drottning; och drickoffer utgjuta de åt andra gudar, mig till sorg.
Jer 7:19  Men är det då mig som de bereda sorg därmed, säger HERREN, och icke fastmer sig själva, så att de komma på skam?
Jer 7:20  Därför säger Herren, HERREN så: Se, min vrede och förtörnelse skall utgjuta sig över denna plats, över både människor och djur, över både träden på marken och frukten på jorden; och den skall brinna och icke bliva utsläckt.
Jer 7:21  Så säger HERREN Sebaot, Israels Gud: Läggen edra brännoffer tillhopa med edra slaktoffer och äten så kött.
Jer 7:22  Ty på den tid då jag förde edra fäder ut ur Egyptens land gav jag dem icke någon befallning eller något bud angående brännoffer och slaktoffer;
Jer 7:23  utan detta var det bud jag gav dem: "Hören min röst, så vill jag vara eder Gud, och I skolen vara mitt folk; och vandren i allt på den väg som jag bjuder eder, på det att det må gå eder väl."
Jer 7:24  Men de ville icke höra eller böja sitt öra till mig, utan vandrade efter sina egna rådslag, i sina onda hjärtans hårdhet, och veko tillbaka i stället för att gå framåt.
Jer 7:25  Allt ifrån den dag då edra fäder drogo ut ur Egyptens land ända till nu har jag dag efter dag, titt och ofta, sänt till eder alla mina tjänare profeterna.
Jer 7:26  Men man ville icke höra mig eller böja sitt öra till mig; de voro hårdnackade och gjorde ännu mer ont än deras fäder.
Jer 7:27  Och om du än säger dem allt detta, så skola de dock icke höra dig; och om du än ropar till dem, så skola de dock icke svara dig.
Jer 7:28  Säg därför till dem: "Detta är det folk som icke vill höra HERRENS, sin Guds, röst eller taga emot tuktan. Sanningen är försvunnen och utrotad ur deras mun."
Jer 7:29  Skär av dig ditt huvudhår och kasta det bort, och stäm upp en klagosång på höjderna. Ty HERREN har förkastat och förskjutit detta släkte, som har uppväckt hans vrede.
Jer 7:30  Juda barn hava ju gjort vad ont är i mina ögon, säger HERREN; de hava satt upp sina styggelser i det hus som är uppkallat efter mitt namn, och de hava så orenat det.
Jer 7:31  Och Tofethöjderna i Hinnoms sons dal hava de byggt upp, för att där uppbränna sina söner och döttrar i eld, fastän jag aldrig har bjudit eller ens tänkt mig något sådant.
Jer 7:32  Se, därför skola dagar komma, säger HERREN, då man icke mer skall säga "Tofet" eller "Hinnoms sons dal", utan "Dråpdalen", och då man skall begrava i Tofet, därför att ingen annan plats finnes.
Jer 7:33  Ja, detta folks döda kroppar skola bliva mat åt himmelens fåglar och markens djur, och ingen skall skrämma bort dem.
Jer 7:34  Och i Juda städer och på Jerusalems gator skall jag göra slut på fröjderop och glädjerop, på rop för brudgum och rop för brud, ty landet skall bliva ödelagt.
Jer 8:1  på den tiden, säger HERREN, skall man kasta Juda konungars och furstars ben, och prästernas och profeternas ben, och Jerusalems invånares ben ut ur deras gravar
Jer 8:2  och kringströ dem inför solen och månen och himmelens hela härskara, som de hava älskat, tjänat och efterföljt, sökt och tillbett; man skall icke sedan samla dem tillhopa eller begrava dem, utan de skola bliva gödsel på marken.
Jer 8:3  Och alla kvarblivna, de som lämnas kvar av detta onda släkte, skola hellre vilja dö än leva, vilka än de orter må vara, dit dessa kvarlämnade bliva fördrivna av mig, säger HERREN Sebaot.
Jer 8:4  Du skall ock säga till dem: Så säger HERREN: Om någon faller, står han ju upp igen; om någon går bort, vänder han ju tillbaka.
Jer 8:5  Varför går det då bort i beständig avfällighet, detta folk i Jerusalem? Varför hålla de fast vid sitt svek och vilja icke vända tillbaka?
Jer 8:6  Jag har givit akt och hört huru de tala vad orätt är; ingen enda finnes, som ångrar sin ondska, ingen säger: "Vad har jag gjort!" Alla löpa de bort, lika hästar som rusa åstad i striden.
Jer 8:7  Till och med hägern under himmelen känner ju sin bestämda tid, och turturduvan, svalan och tranan taga i akt tiden för sin återkomst; mitt folk däremot känner ej HERRENS rätter.
Jer 8:8  Huru kunnen I då säga: "Vi äro visa och hava HERRENS lag ibland oss"? Icke så, de skriftlärdes lögnpenna har förvandlat den i lögn.
Jer 8:9  Sådana visa skola komma på skam, komma till korta och bliva snärjda. De hava ju förkastat HERRENS ord, vari äro de då visa?
Jer 8:10  Så skall jag nu giva deras hustrur åt andra och deras åkrar åt erövrare Ty alla, både små och stora, söka orätt vinning; både profeter och präster fara allasammans med lögn,
Jer 8:11  de taga det lätt med helandet av dottern mitt folks skada; de säga: "Allt står väl till, allt står väl till", och dock står icke allt väl
Jer 8:12  De skola komma på skam, övade styggelse. Likväl känna de alls icke skam och veta icke av att blygas. Därför skola de falla bland de andra; när hemsökelsen träffar dem, skola de komma på fall, säger HERREN.
Jer 8:13  Jag skall bortrycka och förgöra dem, säger HERREN. Inga druvor växa på vinträden, och inga fikon på fikonträden, utan till och med löven äro vissnade: De bud jag gav dem överträda de.
Jer 8:14  Varför sitta vi här stilla? Församlen eder och låt oss fly in i de befästa städerna och förgås där; ty HERREN, vår Gud, vill förgöra oss, han giver oss gift att dricka därför att vi syndade mot HERREN.
Jer 8:15  V bida efter frid, men intet gott kommer, efter en tid då vi skulle bliva helade, men se, förskräckelse kommer.
Jer 8:16  Från Dan hör man frustandet av hans hästar; för hans hingstars gnäggande bävar hela landet. De komma och förtära landet med allt vad däri är, staden med dem som bo däri.
Jer 8:17  Ty se, jag sänder emot eder ormar, basilisker, mot vilka ingen besvärjelse hjälper, och de skola stinga eder, säger HERREN.
Jer 8:18  Var skall jag finna vederkvickelse i min sorg? Mitt hjärta är sjukt i mig.
Jer 8:19  Hör, dottern mitt folk ropar i fjärran land: "Finnes då icke HERREN i Sion? Är dennes konung icke mer där?" Ja, varför hava de förtörnat mig med sina beläten, med sina främmande avgudar?
Jer 8:20  Skördetiden är förbi, sommaren är till ända, och ingen frälsning har kommit oss till del.
Jer 8:21  Jag är förkrossad, därför att dottern mitt folk så krossas, jag går sörjande, häpnad har gripit mig.
Jer 8:22  Finnes då ingen balsam i Gilead, finnes ingen läkare där? Eller varför bliver dottern mitt folk icke helad från sina sår?
Jer 9:1  Ack att mitt huvud vore en vattenbrunn och mina ögon en tårekälla, så att jag kunde gråta dag och natt över de slagna hos dottern mitt folk!
Jer 9:2  Ack att jag hade ett härbärge i öknen, så att jag kunde övergiva mitt folk och draga bort ifrån dem! Ty de äro allasammans äktenskapsbrytare, en församling av trolösa.
Jer 9:3  Sin tungas båge spänna de till att avskjuta lögner, och till sanning bruka de icke sin makt i landet. Nej, de gå från ogärning till ogärning, men mig vilja de ej veta av, säger HERREN.
Jer 9:4  Var och en tage sig till vara för sin vän, och ingen förlite sig på någon sin broder; ty den ene brodern gör allt för att bedraga den andre, och den ene vännen går omkring och förtalar den andre.
Jer 9:5  Var och en handlar svikligt mot sin vän, och ingen talar vad sant är; de öva sina tungor i att tala lögn de arbeta sig trötta med att göra illa.
Jer 9:6  Du bor mitt ibland falskhet; i sin falskhet vilja de ej veta av mig, säger HERREN.
Jer 9:7  Därför säger HERREN Sebaot så. Se, jag måste luttra och pröva dem; ty vad annat kan jag göra, då nu dottern mitt folk är sådan?
Jer 9:8  Deras tunga är en mördande pil; vad den talar är svek. Med munnen tala de vänligt till sin nästa, men i hjärtat lägga de försåt för honom.
Jer 9:9  Skulle jag icke för sådant hemsöka dem? säger HERREN. Skulle icke min själ hämnas på ett sådant folk som detta är?
Jer 9:10  Över bergen vill jag gråta och sjunga sorgesång; jag vill höja klagosång över betesmarkerna i öknen. Ty de äro förbrända, så att ingen går där fram och inga läten av boskap där höras; både himmelens fåglar och fyrfotadjuren hava flytt och äro borta.
Jer 9:11  Jag skall göra Jerusalem till en stenhop, till en boning för schakaler, och Juda städer till en ödemark, där ingen bor.
Jer 9:12  Vem är en vis man, så att han förstår detta? Och till vem har HERRENS mun talat, så att han kan förklara detta: varför landet har blivit så fördärvat, förbränt såsom en öken, där ingen går fram?
Jer 9:13  Och HERREN svarade: Jo, därför att de hava övergivit min lag, den som jag förelade dem, och icke hava hört min röst och vandrat efter den
Jer 9:14  utan vandrat efter sina egna hjärtans hårdhet och efterföljt Baalerna, såsom deras fader lärde dem.
Jer 9:15  Därför säger HERREN Sebaot, Israels Gud, så: Se, jag skall giva detta folk malört att äta och gift att dricka.
Jer 9:16  Och jag skall förströ dem bland folk som varken de eller deras fäder hava känt, och skall sända svärdet efter dem, till dess att jag har gjort ände på dem.
Jer 9:17  Så säger HERREN Sebaot: Given akt; tillkallen gråterskor, för att de må komma, och sänden efter förfarna kvinnor, och låten dem komma.
Jer 9:18  Låten dem med hast stämma upp sorgesång över oss, så att våra ögon flyta i tårar och vatten strömmar från våra ögonlock.
Jer 9:19  Ty sorgesång höres ljuda från Sion: Huru har ej förstörelse drabbat oss! Vi hava kommit illa på skam, vi måste ju övergiva landet, ty våra boningar hava de slagit ned.
Jer 9:20  Ja, hören, I kvinnor, HERRENS ord, och edert öra fatte hans muns tal. Lären edra döttrar sorgesång; ja, lären varandra klagosång.
Jer 9:21  Ty döden stiger in genom vara fönster, han kommer in i våra palats; han utrotar barnen från gatan och ynglingarna från torgen.
Jer 9:22  Ja, tala: Så säger HERREN: Och människornas döda kroppar ligga såsom gödsel på marken och såsom kärvar efter skördemannen, vilka ingen samlar upp.
Jer 9:23  Så säger HERREN: Den vise berömme sig icke av sin vishet, den starke berömme sig icke av sin styrka, den rike berömme sig icke av sin rikedom.
Jer 9:24  Nej, den som vill berömma sig, han berömme sig därav att han har förstånd till att känna mig: att jag är HERREN, som gör nåd, rätt och rättfärdighet på jorden. Ty till sådana har jag behag, säger HERREN.
Jer 9:25  Se, dagar skola komma, säger HERREN, då jag skall hemsöka alla omskurna som dock äro oomskurna:
Jer 9:26  Egypten, Juda, Edom, Ammons barn, Moab och alla ökenbor med kantklippt hår. Ty hednafolken äro alla oomskurna, och hela Israels hus har ett oomskuret hjärta.
Jer 10:1  Hören det ord som HERREN talar till eder, I av Israels hus. Så säger HERREN:
Jer 10:2  I skolen icke vänja eder vid hedningarnas sätt och icke förfäras för himmelens tecken, därför att hedningarna förfäras för dem.
Jer 10:3  Ty vad folken predika är fåfängliga avgudar. Se, av ett stycke trä från skogen hugger man ut dem, och konstnärens händer tillyxa dem;
Jer 10:4  med silver och guld pryder man dem och fäster dem med spikar och hammare, för att de icke skola falla omkull.
Jer 10:5  Lika fågelskrämmor på ett gurkfält stå de där och kunna ej tala; man måste bära dem, ty de kunna ej gå. Frukten då icke för dem, ty de kunna ej göra något ont; och att göra något gott, det förmå de ej heller.
Jer 10:6  Men dig, HERRE, är ingen lik; du är stor, ditt namn är stort i makt.
Jer 10:7  Vem skulle icke frukta dig, du folkens konung? Sådant tillkommer ju dig. Ty bland folkens alla vise och i alla deras riken finnes ingen som är dig lik.
Jer 10:8  Nej, allasammans äro de oförnuftiga och dårar. Avgudadyrkan är att dyrka trä,
Jer 10:9  silverplåt, hämtad från Tarsis, guld, fört ifrån Ufas, arbetat av en konstnär, av en guldsmeds händer. I blått och rött purpurtyg stå de klädda, allasammans blott verk av konstförfarna män.
Jer 10:10  Men HERREN är en sann Gud, han är en levande Gud och en evig konung; för hans förtörnelse bävar jorden, och folken kunna icke uthärda hans vrede.
Jer 10:11  Så skolen I säga till dem: De gudar som icke hava gjort himmel och jord, de skola utrotas från jorden och ej få finnas under himmelen.
Jer 10:12  Han har gjort jorden genom sin kraft, han har berett jordens krets genom sin vishet, och genom sitt förstånd har han utspänt himmelen.
Jer 10:13  När han vill låta höra sin röst, då brusa himmelens vatten, då låter han regnskyar stiga upp från jordens ända; han låter ljungeldar komma med regn och för vinden ut ur dess förvaringsrum.
Jer 10:14  Såsom dårar stå då alla människor där och begripa intet; guldsmederna komma då alla på skam med sina beläten, ty deras gjutna beläten äro lögn, och ingen ande är i dem.
Jer 10:15  De äro fåfänglighet, en tillverkning att le åt; när hemsökelsen kommer över dem, måste de förgås.
Jer 10:16  Men sådan är icke han som är Jakobs del; nej, det är han som har skapat allt, och Israel är hans arvedels stam. HERREN Sebaot är hans namn.
Jer 10:17  Samlen edert gods och fören det bort ur landet, I som sitten under belägring.
Jer 10:18  Ty så säger HERREN: Se, denna gång skall jag slunga bort landets inbyggare; jag skall bereda dem ångest, så att de förnimma det.
Jer 10:19  Ve mig, jag är sönderkrossad! Oläkligt är mitt sår. Men jag säger: Ja, detta är min plåga, jag måste bära den!
Jer 10:20  Mitt tält är förstört, och mina tältstreck äro alla avslitna. Mina barn äro borta, de finnas icke mer; ingen är kvar, som kan slå upp mitt tält och sätta upp mina tältdukar.
Jer 10:21  Ty herdarna voro oförnuftiga, de frågade icke efter HERREN; därför hade de ingen framgång, och hela deras hjord blev förskingrad.
Jer 10:22  Lyssna, något höres! Se, det nalkas! Ett stort dån kommer från nordlandet för att göra Juda städer till en ödemark, till en boning för schakaler.
Jer 10:23  Jag vet det, HERRE: människans väg beror ej av henne, det står icke i vandrarens makt att rätt styra sina steg.
Jer 10:24  Så tukta mig, HERRE likväl med måtta; icke i din vrede, på det att du ej må göra mig till intet.
Jer 10:25  Utgjut din förtörnelse över hedningarna, som icke känna dig, och över de släkter som ej åkalla ditt namn. ty de hava uppätit Jakob, ja, uppätit och gjort ände på honom, och hans boning hava de förött.
Jer 11:1  Detta är det ord som kom till Jeremia från HERREN; han sade:
Jer 11:2  "Hören detta förbunds ord, och talen till Juda män och till Jerusalems invånare;
Jer 11:3  säg till dem: Så säger HERREN, Israels Gud: Förbannad vare den man som icke hör detta förbunds ord,
Jer 11:4  det som jag bjöd edra fader på den tid då jag förde dem ut ur Egyptens land, den smältugnen, i det jag sade: Hören min röst och gören detta, alldeles såsom jag bjuder eder, så skolen I vara mitt folk, och jag skall vara eder Gud,
Jer 11:5  på det att jag må hålla den ed som jag har svurit edra fäder: att giva dem ett land som flyter av mjölk och honung, såsom ock nu har skett." Och jag svarade och sade: "Ja, amen, HERRE."
Jer 11:6  Och HERREN sade till mig: Predika allt detta i Juda städer och på gatorna i Jerusalem och säg: Hören detta förbunds ord och gören efter dem.
Jer 11:7  Ty både på den dag då jag förde edra fäder ut ur Egyptens land och sedan ända till denna dag har jag varnat dem, ja, titt och ofta har jag varnat dem och sagt: "Hören min röst";
Jer 11:8  men de ville icke höra eller böja sitt öra därtill, utan vandrade var och en i sitt onda hjärtas hårdhet. Därför lät jag ock komma över dem allt vad jag hade sagt i det förbund som jag bjöd dem hålla, men som de dock icke höllo.
Jer 11:9  Och HERREN sade till mig: Jag vet huru Juda män och Jerusalems invånare hava sammansvurit sig.
Jer 11:10  De hava vänt tillbaka till sina förfäders missgärningar, deras som icke ville höra mina ord. Själva hava de så följt efter andra gudar och tjänat dem. Ja, Israels hus och Juda hus hava brutit det förbund som jag slöt med deras fäder.
Jer 11:11  Därför säger HERREN så: Se, jag skall låta en olycka komma över dem, som de icke skola kunna undkomma; och när de då ropa till mig, skall jag icke höra dem.
Jer 11:12  Och om så Juda städer och Jerusalems invånare gå bort och ropa till de gudar åt vilka de pläga tända offereld, så skola dessa alls icke kunna frälsa dem i deras olyckas tid.
Jer 11:13  Ty så många som dina städer äro, så många hava dina gudar blivit, du Juda; och så många som gatorna äro i Jerusalem, så många altaren haven I satt upp åt skändlighetsguden: altaren till att tända offereld åt Baal.
Jer 11:14  Så må du nu icke bedja för detta folk eller frambära någon klagan och förbön för dem; ty jag vill icke höra, när de ropa till mig för sin olyckas skull.
Jer 11:15  Vad har min älskade att göra i mitt hus, då hon, ja, hela hopen, övar sådan skändlighet? Kan heligt kött komma såsom offer från dig? När du får bedriva din ondska, då fröjdar du dig ju.
Jer 11:16  "Ett grönskande olivträd, prytt med sköna frukter", så kallade HERREN dig; men nu har han med stort och väldigt dån tänt upp en eld omkring det trädet, så att dess grenar fördärvas.
Jer 11:17  Ja, HERREN Sebaot, han som planterade dig, har beslutit olycka över dig, för den ondskas skull som Israels och Juda hus hava bedrivit till att förtörna mig, i det att de hava tänt offereld åt Baal.
Jer 11:18  HERREN kungjorde det för mig, så att jag fick veta det; ja, du lät mig se vad de förehade.
Jer 11:19  Själv var jag såsom ett menlöst lamm som föres bort till att slaktas; jag visste ej att de förehade anslag mot mig: "Låt oss fördärva trädet med dess frukt, låt oss utrota honom ur de levandes land, så att man icke mer kommer ihåg hans namn."
Jer 11:20  Men HERREN Sebaot är en rättfärdig domare, som prövar njurar och hjärta. Så låt mig då få se din hämnd på dem, ty för dig har jag lagt fram min sak.
Jer 11:21  Därför säger HERREN så om Anatots män, dem som stå efter ditt liv och säga: "Profetera icke i HERRENS namn, om du icke vill dö för vår hand"
Jer 11:22  ja, därför säger HERREN Sebaot så: Se, jag skall hemsöka dem; deras unga män skola dö genom svärd, deras söner och döttrar skola dö genom hunger.
Jer 11:23  Och intet skall bliva kvar av dem; ty jag skall låta olycka drabba Anatots män, när deras hemsökelses år kommer.
Jer 12:1  HERRE, om jag vill gå till rätta med dig, så behåller du dock rätten. Likväl måste jag tala med dig om vad rätt är. Varför går det de ogudaktiga så väl? Varför hava alla trolösa så god lycka?
Jer 12:2  Du planterar dem, och de slå rot; de växa och bära frukt. Nära är du i deras mun, men fjärran är du från deras innersta.
Jer 12:3  Men du, HERRE, känner mig; du ser mig och prövar huru mitt hjärta är mot dig. Ryck dem bort såsom får till att slaktas, och invig dem till en dödens dag.
Jer 12:4  Huru länge skall landet ligga sörjande och gräset på marken allestädes förtorka, så att både fyrfotadjur och fåglar förgås för inbyggarnas ondskas skull, under det att dessa säga: "Han skall icke se vår undergång"
Jer 12:5  Om du icke orkar löpa i kapp med fotgängare, huru vill du då taga upp tävlan med hästar? Och om du nu känner dig trygg i ett fredligt land, huru skall det gå dig bland Jordanbygdens snår?
Jer 12:6  Se, till och med dina bröder och din faders hus äro ju trolösa mot dig; till och med dessa ropa med full hals bakom din rygg. Du må icke tro på dem, om de ock tala vänligt till dig.
Jer 12:7  Jag har övergivit mitt hus, förskjutit min arvedel; det som var kärast för min själ lämnade jag i fiendehand.
Jer 12:8  Hon som är min arvedel blev mot mig såsom ett lejon i skogen; hon har höjt sin röst mot mig, därför har jag fattat hat till henne.
Jer 12:9  Skall min arvedel vara mot mig såsom en brokig rovfågel - då må ock rovfåglar komma emot henne från alla sidor. Upp, samlen tillhopa alla markens djur, och låten dem komma för att äta!
Jer 12:10  Herdar i mängd fördärva min vingård och förtrampa min åker; de göra min sköna åker till en öde öken. Man gör den till en ödemark;
Jer 12:11  sörjande och öde ligger den framför mig. Hela landet ödelägges, ty ingen finnes, som vill akta på.
Jer 12:12  Över alla höjder i öknen rycka förhärjare fram, ja, HERRENS svärd förtär allt, från den ena ändan av landet till den andra; intet kött kan finna räddning.
Jer 12:13  De hava sått vete, men skördat tistel; de hava mödat sig fåfängt. Ja, I skolen komma på skam med eder gröda för HERRENS glödande vredes skull.
Jer 12:14  Så säger HERREN om alla de onda grannar som förgripa sig på den arvedel jag har givit åt mitt folk Israel: Se, jag skall rycka dem bort ur deras land, och Juda hus skall jag rycka undan ifrån dem.
Jer 12:15  Men därefter, sedan jag har ryckt dem bort, skall jag åter förbarma mig över dem och låta dem komma tillbaka, var och en till sin arvedel och var och en till sitt land.
Jer 12:16  Om de då rätt lära sig mitt folks vägar, så att de svärja vid mitt namn: "Så sant HERREN lever", likasom de förut lärde mitt folk att svärja vid Baal, då skola de bliva upprättade mitt ibland mitt folk.
Jer 12:17  Men om de icke vilja höra, så skall jag alldeles bortrycka och förgöra det folket, säger HERREN.
Jer 13:1  Så sade HERREN till mig: "Gå bort och köp dig en linnegördel, och sätt den omkring dina länder, men låt den icke komma i vatten."
Jer 13:2  Och jag köpte en gördel, såsom HERREN hade befallt, och satte den omkring mina länder.
Jer 13:3  Då kom HERRENS ord till mig för andra gången; han sade:
Jer 13:4  "Tag gördeln som du har köpt, och som du bär omkring dina länder, och stå upp och gå bort till Frat, och göm den där i en stenklyfta."
Jer 13:5  Och jag gick bort och gömde den vid Frat, såsom HERREN hade bjudit mig.
Jer 13:6  Sedan, en lång tid därefter, sade HERREN till mig: "Stå upp och gå bort till Frat, och hämta därifrån den gördel som jag bjöd dig gömma där."
Jer 13:7  Och jag gick bort till Frat och grävde upp gördeln och hämtade fram den från det ställe där jag hade gömt den. Och se, gördeln var fördärvad, så att den icke mer dugde till något.
Jer 13:8  Då kom HERRENS ord till mig; han sade:
Jer 13:9  Så säger HERREN: På samma sätt skall jag sända fördärv över Judas och Jerusalems stora högmod.
Jer 13:10  Detta onda folk, som icke vill höra mitt ord, utan vandrar i sitt hjärtas hårdhet och följer efter andra gudar och tjänar och tillbeder dem, det skall bliva såsom denna gördel vilken icke duger till något.
Jer 13:11  Ty likasom en mans gördel sluter sig tätt omkring hans länder, så lät jag hela Israels hus och hela Juda hus sluta sig till mig, säger HERREN, på det att de skulle vara mitt folk och bliva mig till berömmelse, lov och ära; men de ville icke höra.
Jer 13:12  Säg därför till dem detta ord: Så säger HERREN, Israels Gud: Alla vinkärl äro till för att fyllas med vin. Och när de då säga till dig: "Skulle vi icke veta att alla vinkärl äro till för att fyllas med vin?",
Jer 13:13  så svara dem: Så säger HERREN: Se, jag skall fylla detta lands alla inbyggare, konungarna som sitta på Davids tron, och prästerna och profeterna, ja, alla Jerusalems invånare, så att de bliva druckna.
Jer 13:14  Och jag skall krossa dem, den ene mot den andre, både fäder och barn, säger HERREN. Jag skall icke hava någon misskund, icke skona och icke förbarma mig, så att jag avstår från att fördärva dem.
Jer 13:15  Hören och lyssnen härtill, varen icke övermodiga; ty HERREN har talat.
Jer 13:16  Given HERREN, eder Gud, ära, förrän han låter mörkret komma, och förrän edra fötter snubbla på bergen, när det skymmer; ty det ljus I förbiden skall han byta i dödsskugga och göra till töcken.
Jer 13:17  Men om I icke hören härpå, så måste min själ i lönndom sörja över sådant övermod, och mitt öga måste bitterligen gråta och flyta i tårar, därför att HERRENS hjord då bliver bortförd i fångenskap.
Jer 13:18  Säg till konungen och konungamodern: Sätten eder lågt ned, ty den härlighetens krona som prydde edert huvud har fallit av eder.
Jer 13:19  Städerna i Sydlandet äro tillslutna, och ingen finnes, som öppnar dem; hela Juda är bortfört i fångenskap, ja, bortfört helt och hållet.
Jer 13:20  Lyften upp edra ögon och sen huru de komma norrifrån. Var är nu hjorden som var dig given, den hjord som var din ära?
Jer 13:21  Vad vill du säga, när han sätter till herrar över dig män som du själv har lärt att komma till dig såsom älskare? Skulle du då icke gripas av vånda såsom en kvinna i barnsnöd?
Jer 13:22  Men om du säger i ditt hjärta: "Varför har det gått mig så?", så vet: for din stora missgärnings skull blev ditt mantelsläp upplyft och dina fötter nesligt blottade.
Jer 13:23  Kan väl en etiopier förvandla sin hud eller en panter sina fläckar? Då skullen också I kunna göra något gott, I som ären så övade i ondska.
Jer 13:24  Välan, jag vill förskingra dem såsom strå som far bort för öknens vind.
Jer 13:25  Detta skall vara din lott och din beskärda del från mig, säger HERREN, därför att du har förgätit mig och förlitat dig på lögn.
Jer 13:26  Därför skall jag ock draga upp ditt mantelsläp över ditt ansikte, så att man får se din skam.
Jer 13:27  Din otukt, ditt vrenskande, ditt skändliga otuktsväsen - på höjderna, på fältet har jag sett dina styggelser. Ve dig, Jerusalem! Du kommer icke att bliva ren - på huru lång tid ännu?
Jer 14:1  Detta är det HERRENS ord som kom till Jeremia angående torkan.
Jer 14:2  Juda ligger sörjande, dess portar äro förfallna, likasom i sorgdräkt luta de mot jorden, och ett klagorop stiger upp från Jerusalem.
Jer 14:3  Stormännen där sända de små efter vatten, men när de komma till dammarna, finna de intet vatten; de måste vända tillbaka med tomma kärl. De stå där med skam och blygd och måste hölja över sina huvuden.
Jer 14:4  För markens skull, som ligger vanmäktig, därför att intet regn faller på jorden, stå åkermännen med skam och måste hölja över sina huvuden.
Jer 14:5  Ja, också hinden på fältet övergiver sin nyfödda kalv, därför att intet grönt finnes
Jer 14:6  Och vildåsnorna stå på höjderna och flämta såsom schakaler; deras ögon försmäkta, därför att gräset är borta.
Jer 14:7  Om än våra missgärningar vittna emot oss, så hjälp dock, HERRE, för ditt namns skull. Ty vår avfällighet är stor; mot dig hava vi syndat.
Jer 14:8  Du Israels hopp dess frälsare i nödens tid, varför är du såsom en främling i landet, lik en vägfarande som slår upp sitt tält allenast för en natt?
Jer 14:9  Varför är du lik en rådlös man, lik en hjälte som icke kan hjälpa? Du bor ju dock mitt ibland oss, HERRE, och vi äro uppkallade efter ditt namn; så övergiv oss då icke.
Jer 14:10  Så säger HERREN om detta folk: På detta sätt driva de gärna omkring, de hålla icke sina fötter i styr. Därför har HERREN intet behag till dem; nej, han kommer nu ihåg deras missgärning och hemsöker deras synder.
Jer 14:11  Och HERREN sade till mig: Du må icke bedja om något gott för detta folk.
Jer 14:12  Ty om de än fasta, så vill jag dock icke höra deras rop, och om de än offra brännoffer och spisoffer så har jag intet behag till dem, utan vill förgöra dem med svärd, hungersnöd och pest.
Jer 14:13  Då sade jag: "Ack Herre, HERRE! Profeterna säga ju till dem: I skolen icke se något svärd, ej heller skall hungersnöd träffa eder, nej, en varaktig frid skall jag giva eder på denna plats."
Jer 14:14  Men HERREN sade till mig: Profeterna profetera lögn i mitt namn; jag har icke sänt dem eller givit dem någon befallning eller talat till dem. Lögnsyner och tomma spådomar och fåfängligt tal och sina egna hjärtans svek är det de profetera för eder.
Jer 14:15  Därför säger HERREN så om de profeter som profetera i mitt namn, fastän jag icke har sänt dem, och som säga att svärd och hungersnöd icke skola komma i detta land: Jo, genom svärd och hunger skola dessa profeter förgås.
Jer 14:16  Och folket som de profetera för, både män och hustrur, både söner och döttrar, skola komma att ligga på Jerusalems gator, slagna av hunger och svärd, och ingen skall begrava dem; och jag skall utgjuta deras ondska över dem.
Jer 14:17  Men du skall säga till dem detta ord: Mina ögon flyta i tårar natt och dag och få ingen ro, ty jungfrun, dottern mitt folk har drabbats av stor förstöring, av ett svårt och oläkligt sår.
Jer 14:18  Om jag går ut på marken, se, då ligga där svärdsslagna män; och kommer jag in i staden, så mötes jag där av hungerns plågor. Ja, både profeter och präster nödgas draga från ort till ort, till ett land som de icke känna.
Jer 14:19  Har du då alldeles förkastat Juda? Har din själ begynt försmå Sion? Eller varför har du slagit oss så, att ingen kan hela oss? Vi bida efter frid, men intet gott kommer, efter en tid då vi skulle bliva helade, men se, förskräckelse kommer.
Jer 14:20  HERRE, vi känna vår ogudaktighet, våra fäders missgärning, ty vi hava syndat mot dig.
Jer 14:21  För ditt namns skull, förkasta oss icke, låt din härlighets tron ej bliva föraktad; kom ihåg ditt förbund med oss, och bryt det icke.
Jer 14:22  Finnas väl bland hedningarnas fåfängliga avgudar sådana som kunna giva regn? Eller kan himmelen av sig själv låta regnskurar falla? Är det icke dig, HERRE, vår Gud, som vi måste förbida? Det är ju du som har gjort allt detta.
Jer 15:1  Men HERREN sade till mig; Om än Mose och Samuel trädde inför mig, så skulle min själ dock icke vända sig till detta folk. Driv dem bort ifrån mitt ansikte och låt dem gå.
Jer 15:2  Och om de fråga dig: "Vart skola vi gå?", så skall du svara dem: Så säger HERREN: I pestens våld den som hör pesten till, i svärdets våld den som hör svärdet till, i hungerns våld den som hör hungern till, i fångenskapens våld den som hör fångenskapen till.
Jer 15:3  Fyra slags hemsökelser skall jag låta komma över dem, säger HERREN: svärdet, som skall dräpa dem, hundarna, som skola släpa bort dem, himmelens fåglar och vilddjuren på marken, som skola äta upp och fördärva dem.
Jer 15:4  Och jag skall göra dem till en varnagel för alla riken på jorden, till straff för det som Manasse, Hiskias son, Juda konung, har gjort i Jerusalem.
Jer 15:5  Ty vem kan hava misskund med dig, Jerusalem, och vem kan ömka dig, och vem kan vilja komma för att fråga om det står väl till med dig?
Jer 15:6  Du själv försköt mig, säger HERREN; du gick din väg bort. Därför uträckte jag mot dig min hand och fördärvade dig; jag hade tröttnat att förbarma mig.
Jer 15:7  Ja, jag kastade dem med kastskovel vid landets portar, jag gjorde föräldrarna barnlösa, jag förgjorde mitt folk, då de ej ville vända om från sina vägar.
Jer 15:8  Deras änkor blevo genom mig talrikare än sanden i havet; över mödrarna till deras unga lät jag förhärjare komma mitt på ljusa dagen; plötsligt lät jag ångest och förskräckelse falla över dem.
Jer 15:9  Om en moder än hade sju söner, måste hon dock giva upp andan i sorg; hennes sol gick ned, medan det ännu var dag, hon måste bliva till skam och blygd. Och vad som är kvar av dem skall jag giva till pris åt deras fienders svärd, säger HERREN
Jer 15:10  "Ve mig, min moder, att du har fött mig, mig som är till kiv och träta för hela landet! Jag har icke drivit ocker, ej heller har någon behövt ockra på mig; likväl förbanna de mig alla."
Jer 15:11  Men HERREN svarade: "Sannerligen, jag skall styrka dig och låta det gå dig väl. Sannerligen, jag skall så göra, att dina fiender komma och bönfalla inför dig i olyckans och nödens tid.
Jer 15:12  Kan man bryta sönder järn, järn från norden, eller koppar?" -
Jer 15:13  Ditt gods och dina skatter skall jag lämna till plundring, och det utan betalning, till straff för allt vad du har syndat i hela ditt land.
Jer 15:14  Och jag skall låta dina fiender föra dig in i ett land som du icke känner. Ty min vredes eld är upptänd; mot eder skall det brinna.
Jer 15:15  HERRE, du vet det. Tänk på mig och låt dig vårda om mig, och skaffa mig hämnd på mina förföljare; tag mig icke bort, du som är långmodig. Betänk huru jag bär smälek för din skull
Jer 15:16  När jag fick dina ord, blevo de min spis, ja, dina ord blevo för mig mitt hjärtas fröjd och glädje; ty jag är uppkallad efter ditt namn, HERRE, härskarornas Gud.
Jer 15:17  Jag har icke suttit i gycklares samkväm och förlustat mig där; för din hands skull har jag måst sitta ensam, ty du har uppfyllt mig med förgrymmelse.
Jer 15:18  Varför skall jag då plågas så oavlåtligt, och varför är mitt sår så ohelbart? Det vill ju icke läkas. Ja, du bliver för mig såsom en försinande bäck, så som ett vatten som ingen kan lita på.
Jer 15:19  Därför säger HERREN så: Om du vänder åter, så vill jag låta dig komma åter och bliva min tjänare. Och om du frambär ädel metall utan slagg, så skall du få tjäna mig såsom mun. Dessa skola då vända åter till dig, men du skall icke vända åter till dem.
Jer 15:20  Och jag skall göra dig inför detta folk till en fast kopparmur, så att de icke skola bliva dig övermäktiga, om de vilja strida mot dig; ty jag är med dig och vill frälsa dig och vill hjälpa dig, säger HERREN.
Jer 15:21  Jag skall hjälpa dig ut ur de ondas våld och skall förlossa dig ur våldsverkarnas hand.
Jer 16:1  Och HERRENS ord kom till mig; han sade:
Jer 16:2  Du skall icke taga dig någon hustru eller skaffa dig några söner och döttrar på denna plats.
Jer 16:3  Ty så säger HERREN om de söner och döttrar som bliva I födda på denna plats, och om mödrarna som hava fött dem, och om fäderna som hava avlat dem i detta land:
Jer 16:4  Av svåra sjukdomar skola de dö; man skall icke hålla dödsklagan efter dem eller begrava dem, utan de skola bliva gödsel på marken. Och genom svärd och hunger skola de förgås, och deras döda kroppar skola bliva mat åt himmelens fåglar och markens djur. I
Jer 16:5  Ty så säger HERREN: Du skall icke gå in i något sorgehus och icke begiva dig åstad för att hålla dödsklagan, ej heller ömka dem; ty jag har tagit bort min frid ifrån detta folk, säger HERREN, ja, min nåd och barmhärtighet.
Jer 16:6  Och både stora och små skola dö i detta land, utan att bliva begravna; och man skall icke hålla dödsklagan efter dem, och ingen skall för deras skull rista märken på sig eller raka sitt huvud.
Jer 16:7  Man skall icke bryta bröd åt någon, för att trösta honom i sorgen efter en död, och icke giva någon tröstebägaren att dricka, när han har förlorat fader eller moder.
Jer 16:8  Och i gästabudshus skall du icke heller gå in för att sitta med dem och äta och dricka.
Jer 16:9  Ty så säger HERREN Sebaot, Israels Gud: Se, inför edra ögon, och medan I ännu leven, skall jag på denna plats göra slut på fröjderop och glädjerop, på rop för brudgum och rop för brud. I
Jer 16:10  När du nu förkunnar alla dessa ord för detta folk och de då fråga dig: "Varför har HERREN uttalat över oss all denna stora olycka? Och vari består den missgärning och synd som vi hava begått mot HERREN, vår Gud?",
Jer 16:11  då skall du svara dem: "Jo, edra fäder övergåvo mig, säger HERREN, och följde efter andra gudar och tjänade och tillbådo dem; ja, mig övergåvo de och höllo icke min lag.
Jer 16:12  Och I själva haven gjort ännu mer ont, än edra fäder gjorde; ty se, I vandren var och en efter sitt onda hjärtas hårdhet, och I viljen icke höra mig.
Jer 16:13  Därför skall jag ock slunga eder bort ur detta land, till ett land son varken I eller edra fäder haven känt, och där Skolen I få tjäna andra gudar både dag och natt; ty jag skall icke hava någon misskund med eder."
Jer 16:14  Se, därför skola dagar komma, säger HERREN, då man icke mer skall säga: "Så sant HERREN lever, han som har fört Israels barn upp ur Egyptens land",
Jer 16:15  utan: "Så sant HERREN lever, han som har fört Israels barn upp ur nordlandet, och ur alla andra länder till vilka han hade drivit dem bort." Ty jag skall föra dem tillbaka till deras land, det som jag gav åt deras fäder.
Jer 16:16  Se, jag skall sända bud efter många fiskare, säger HERREN, och de skola fiska upp dem; och sedan skall jag sända bud efter många jägare, och de skola jaga dem ned från alla berg och alla höjder och ut ur stenklyftorna.
Jer 16:17  Ty mina ögon äro riktade på alla deras vägar; de kunna icke gömma sig för mitt ansikte och deras missgärning är icke fördold för mina ögon.
Jer 16:18  Och först skall jag i dubbelt mått vedergälla dem för deras missgärning och synd, för att de hava oskärat mitt land, i det att de hava uppfyllt min arvedel med sina styggeliga och skändliga avgudars döda kroppar.
Jer 16:19  HERRE, du min starkhet och mitt värn, du min tillflykt på nödens dag, till dig skola hedningarna komma från jordens ändar och skola säga: "Allenast lögn hava våra fäder fått i arv. fåfängliga avgudar, av vilka ingen kan hjälpa.
Jer 16:20  Kan väl en människa göra sig gudar? Nej, de gudarna äro inga gudar.
Jer 16:21  Därför vill jag nu denna gång låta dem förnimma det, jag vill låta dem känna min hand och min makt, för att de må veta att mitt namn är HERREN.
Jer 17:1  Juda synd är uppskriven med järnstift, med diamantgriffel; den är inristad på deras hjärtas tavla och på edra altarens horn,
Jer 17:2  så visst som deras barn vid gröna träd och på höga kullar komma ihåg sina altaren och Aseror.
Jer 17:3  Du mitt berg på fältet, ditt gods, ja, alla dina skatter skall jag lämna till plundring, så ock dina offerhöjder, till straff för vad du har syndat i hela ditt land.
Jer 17:4  Och du skall nödgas avstå - och detta genom din egen förskyllan - från den arvedel som jag har givit dig; och jag skall låta dig tjäna dina fiender i ett land som du icke känner. Ty I haven upptänt min vredes eld, och den skall brinna till evig tid.
Jer 17:5  Så säger HERREN: Förbannad är den man som förtröstar på människor och sätter kött sig till arm och med sitt hjärta viker av ifrån HERREN.
Jer 17:6  Han skall bliva såsom en torr buske på hedmarken och skall icke få se något gott komma, utan skall bo på förbrända platser i öknen, i ett land med salthedar, där ingen bor.
Jer 17:7  Men välsignad är den man som förtröstar på HERREN, den som har HERREN till sin förtröstan.
Jer 17:8  Han är lik ett träd som är planterat vid vatten, och som sträcker ut sina rötter till bäcken; ty om än hetta kommer, så förskräckes det icke, utan bevarar sina löv grönskande; och om ett torrt år kommer, så sörjer det icke och upphör ej heller att bara frukt.
Jer 17:9  Ett illfundigt och fördärvat ting är hjärtat framför allt annat; vem kan förstå det?
Jer 17:10  Dock, Jag, HERREN, utrannsakar hjärtat och prövar njurarna, och giver så åt var och en efter hans vägar, efter hans gärningars frukt.
Jer 17:11  Lik en rapphöna som ruvar på ägg, vilka hon ej själv har lagt, är den som samlar rikedom med orätt; i sina halva dagar måste han lämna den och vid sitt slut skall han stå såsom en dåre.
Jer 17:12  En härlighetens tron, en urgammal höjd är vår helgedoms plats.
Jer 17:13  HERREN är Israels hopp; alla som övergiva dig komma på skam. de som vika av ifrån mig likna en skrift i sanden; ty de hava övergivit HERREN, källan med det friska vattnet.
Jer 17:14  Hela du mig, HERRE, så varde jag helad; fräls mig du, så varder jag frälst. Ty du är mitt lov.
Jer 17:15  Se, dessa säga till mig: "Vad bliver av HERRENS ord? Må det fullbordas!"
Jer 17:16  Det är ju så, att jag ej har undandragit mig herdekallet i din efterföljd, och fördärvets dag har jag icke åstundat; du vet det själv. Vad mina läppar hava uttalat, det har talats inför ditt ansikte.
Jer 17:17  Så bliv då icke till skräck för mig; du som är min tillflykt på olyckans dag.
Jer 17:18  Låt dem som förfölja mig komma på skam, men låt icke mig komma på skam; låt dem bliva förfärade, men låt ej mig bliva förfärad. Låt en olycksdag komma över dem, och krossa dem i dubbelt mått.
Jer 17:19  Så sade HERREN till mig: Gå åstad och ställ dig i Menighetsporten, där Juda konungar gå in och gå ut, och sedan i Jerusalems alla andra portar;
Jer 17:20  och säg till dem: Hören HERRENS ord, I Juda konungar med hela Juda, och I alla Jerusalems invånare som gån in genom dessa portar.
Jer 17:21  Så säger HERREN: Tagen eder val till vara för att på sabbatsdagen bära någon börda eller föra in någon sådan genom Jerusalems portar.
Jer 17:22  Och fören icke på sabbatsdagen någon börda ut ur edra hus, och gören ej heller något annat arbete, utan helgen sabbatsdagen, såsom jag bjöd edra fäder,
Jer 17:23  fastän de icke ville höra eller böja sitt öra därtill, utan voro hårdnackade, så att de icke hörde eller togo emot tuktan.
Jer 17:24  Men om I viljen höra mig, säger HERREN, så att I på sabbatsdagen icke fören någon börda in genom denna stads portar, utan helgen sabbatsdagen, så att I på den icke gören något arbete,
Jer 17:25  då skola konungar och furstar som komma att sitta på Davids tron få draga in genom denna stads portar, på vagnar och hästar, följda av sina furstar, av Juda man och Jerusalems invånare; och denna stad skall då förbliva bebodd evinnerligen.
Jer 17:26  Och från Juda städer, från Jerusalems omnejd och från Benjamins land, från Låglandet, Bergsbygden och Sydlandet skall man komma och frambära brännoffer, slaktoffer, spisoffer och rökelse och frambära lovoffer till HERRENS hus.
Jer 17:27  Men om I icke hören mitt bud att helga sabbaten och att icke bära någon börda in genom Jerusalems portar på sabbatsdagen, då skall jag tända eld på dess portar, och elden skall förtära Jerusalems palatser och skall icke kunna utsläckas.
Jer 18:1  Detta är det ord som kom till Jeremia från HERREN; han sade
Jer 18:2  "Stå upp och gå ned till krukmakarens hus; där vill jag låta dig höra mina ord."
Jer 18:3  Då gick jag ned till krukmakarens hus och fann honom upptagen med arbete på krukmakarskivan.
Jer 18:4  Och när kärlet som krukmakare höll på att göra av leret misslyckades i hans hand, begynte han omigen, och gjorde därav ett annat kärl så, som han ville hava det gjort.
Jer 18:5  Och HERRENS ord kom till mig han sade:
Jer 18:6  Skulle jag icke kunna göra med eder, I är Israels hus, såsom denne krukmakare gör? säger HERREN Jo, såsom leret är i krukmakarens hand, så ären ock I i min hand, I av Israels hus.
Jer 18:7  Den ena gången hotar jag ett folk och ett rike att jag vill upprycka, nedbryta och förgöra det;
Jer 18:8  men om då det folket omvänder sig från det onda väsende mot vilket jag vände mitt hot, så ångrar jag det onda som jag hade tänkt att göra dem.
Jer 18:9  En annan gång lovar jag ett folk och ett rike att jag vill uppbygga och plantera det;
Jer 18:10  men om det då gör vad ont är i mina ögon och icke hör min röst, så ångrar jag det goda som jag hade sagt att jag ville göra dem.
Jer 18:11  Så säg du nu till Juda man och Jerusalems invånare: Så säger HERREN: Se, jag bereder åt eder en olycka, och jag har i sinnet ett anslag mot eder. Vänden därför om, var och en från sin onda väg, och bättren edert leverne och edert väsende.
Jer 18:12  Men de skola svara: "Du mödar dig förgäves. Vi vilja följa vara egna tankar och göra var och er efter sitt onda hjärtas hårdhet.
Jer 18:13  Därför säger HERREN så: Frågen efter bland hednafolken om någon har hört något sådant. Alltför gruvliga ting har jungfrun Israel bedrivit.
Jer 18:14  Övergiver då Libanons snö sin upphöjda klippa, eller sina de friska vatten ut, som strömma ifrån fjärran,
Jer 18:15  eftersom mitt folk förgäter mig och tänder offereld åt avgudar? Se, av dem skola de bringas på fall, när de gå sin gamla stråt och vandra på villostigar, på obanade vägar.
Jer 18:16  Så göra de sitt land till ett föremål för häpnad, för begabberi evinnerligen; alla som gå där fram skola häpna och skaka huvudet.
Jer 18:17  Såsom en östanvind skall jag förskingra dem, när fienden kommer; jag skall visa dem ryggen och icke ansiktet, på deras ofärds dag..
Jer 18:18  Men de sade: "Kom, låt oss tänka ut något anslag mot Jeremia. Ty prästerna skola icke komma till korta med undervisning, ej heller de vise med råd, ej heller profeterna med förkunnelse. Ja, kom, låt oss fälla honom med vara tungor, vi behöva alls icke akta på vad han säger."
Jer 18:19  HERRE, akta du på mig, och hör rad mina motståndare tala.
Jer 18:20  Skall man få vedergälla gott med ont, eftersom dessa hava grävt en grop för mitt liv? Tänk på huru jag har stått inför ditt ansikte för att mana gott för dem, till att avvända från dem din vrede.
Jer 18:21  Därför må du överlämna deras barn åt hungersnöden och giva dem själva till pris åt svärdet, så att deras hustrur bliva barnlösa och änkor, deras män dräpta av pesten, och deras ynglingar slagna med svärd striden.
Jer 18:22  Må klagorop höras från deras hus, i det att du plötsligt låter rövarskaror komma över dem. Ty de hava grävt en grop för att fånga mig, och snaror hava de lagt ut för mina fötter.
Jer 18:23  Men du, HERRE, känner alla deras mordiska anslag mot mig; så må du då icke förlåta dem deras missgärning eller utplåna deras synd ur din åsyn. Må de bringas på fall inför dig; ja, utför ditt verk mot dem på din vredes tid.
Jer 19:1  Så sade HERREN: Gå åstad och köp dig en lerkruka av krukmakaren; och tag med dig några av de äldste i folket och av de äldste bland prästerna,
Jer 19:2  och gå ut till Hinnoms sons dal, som ligger framför Lerskärvsporten, och predika där de ord som jag skall tala till dig.
Jer 19:3  Du skall säga: "Hören HERRENS ord, I Juda konungar och I Jerusalems invånare: Så säger HERREN Sebaot, Israels Gud: Se, jag skall låta en sådan olycka komma över denna plats, att det skall genljuda i öronen på var och en som får höra det.
Jer 19:4  Eftersom de hava övergivit mig och icke aktat denna plats, utan där tänt offereld åt andra gudar, som varken de själva eller deras fäder eller Juda konungar hava känt, och eftersom de hava uppfyllt denna plats med oskyldigas blod,
Jer 19:5  och byggt sina Baalshöjder, för att där bränna upp sina barn i eld, till brännoffer åt Baal, fastän jag aldrig har bjudit eller talat om eller ens tänkt mig något sådant,
Jer 19:6  se, därför skola dagar komma, säger HERREN, då man icke mer skall kalla denna plats 'Tofet' eller 'Hinnoms sons dal', utan 'Dråpdalen'.
Jer 19:7  Och då skall jag på denna plats göra om intet Judas och Jerusalems råd, och jag skall låta dem falla för deras fienders svärd och för de mäns hand, som stå efter deras liv och jag skall giva deras döda kroppar till mat åt himmelens fåglar och markens djur.
Jer 19:8  Och jag skall göra denna stad till ett föremål för häpnad och begabberi; alla som gå där från skola häpna och vissla vid tanken på alla dess plågor.
Jer 19:9  Och jag skall låta dem äta sina egna söners och döttrars kött, ja, den ene skall nödgas äta den andres kött. I sådan nöd och sådant trångmål skola de komma genom sina fiender och genom dem som stå efter deras liv."
Jer 19:10  Och du skall slå sönder krukan inför de mäns ögon, som hava gått med dig,
Jer 19:11  och du skall säga till dem: "Så säger HERREN Sebaot: Jag skall sönderslå detta folk och denna stad, på samma sätt som man slår sönder ett krukmakarkärl, så att det icke kan bliva helt igen; och man skall begrava i Tofet, därför att ingen annan plats finnes att begrava på.
Jer 19:12  Så skall jag göra med denna plats, säger HERREN, och med dess invånare; jag skall göra denna stad lik Tofet.
Jer 19:13  Och husen i Jerusalem och Juda konungars hus, de orena, skola bliva såsom Tofetplatsen, ja, alla de hus på vilkas tak man har tänt offereld åt himmelens hela härskara och utgjutit drickoffer åt andra gudar."
Jer 19:14  När sedan Jeremia kom igen från Tofet, dit HERREN hade sänt honom för att profetera, ställde han sig i förgården till HERRENS hus och sade till allt folket:
Jer 19:15  "Så säger HERREN Sebaot, Israels Gud: Se, över denna stad med alla dess lydstäder skall jag låta all den olycka komma, som jag har beslutit över den - detta därför att de hava varit hårdnackade och icke velat höra mina ord."
Jer 20:1  Då nu Pashur, Immers son, prästen, som var överuppsyningsman i HERRENS hus, hörde Jeremia profetera detta,
Jer 20:2  lät han hudflänga profeten Jeremia och satte honom i stocken i Övre Benjaminsporten till HERRENS hus.
Jer 20:3  Men när Pashur dagen därefter släppte Jeremia lös ur stocken, sade Jeremia till honom: "Pashur är icke det namn varmed HERREN benämner dig, utan Magor-Missabib;
Jer 20:4  ty så säger HERREN: Se, jag skall göra dig till skräck såväl för dig själv som för alla dina vänner; och de skola falla för sina fienders svärd, i din egen åsyn. Och hela Juda skall jag giva i den babyloniske konungens hand, och han skall föra dem bort till Babel och dräpa dem med svärd.
Jer 20:5  Och jag skall giva denna stads alla rikedomar, allt dess gods och alla dyrbarheter däri, ja, Juda konungars alla skatter skall jag giva i deras fienders hand; och de skola göra det till sitt byte och taga det och föra det till Babel.
Jer 20:6  Och du själv, Pashur, skall gå i fångenskap, med alla som bo i ditt hus. Du skall komma till Babel; där skall du dö, och där skall du begravas, Sjungen till HERRENS ära, jämte alla dina vänner, för vilka du har profeterat lögn."
Jer 20:7  Du, HERRE, övertalade mig, och jag lät mig övertalas; du grep mig och blev mig övermäktig. Så har jag blivit ett ständigt åtlöje; var man bespottar mig.
Jer 20:8  Ty' så ofta jag talar, måste jag klaga; jag måste ropa över våld och förtryck, ty HERRENS ord har blivit mig till smälek och hån beständigt.
Jer 20:9  Men när jag sade: "Jag vill icke tänka på honom eller vidare tala i hans namn", då blev det i mitt hjärta såsom brunne där en eld, instängd i mitt innersta; jag mödade mig med att uthärda den, men jag kunde det icke.
Jer 20:10  Ty jag hör mig förtalas av många; skräck från alla sidor! "Anklagen honom!" "Ja, vi vilja anklaga honom!" Alla som hava varit mina vänner vakta på att jag skall falla: "Kanhända skall han låta locka sig, så att vi bliva honom övermäktiga och få taga hämnd på honom."
Jer 20:11  Men HERREN är med mig såsom en väldig hjälte; därför skola mina förföljare komma på fall och intet förmå. Ja, de skola storligen komma på skam, därför att de ej hade förstånd; de skola drabbas av en evig blygd, som icke skall varda förgäten
Jer 20:12  Ty HERREN Sebaot prövar med rättfärdighet, han ser njurar och hjärta. Så skall jag då få se din hämnd på dem, ty för dig har jag lagt fram min sak.
Jer 20:13  Sjungen till HERRENS ära, loven HERREN; ty han räddar den fattiges själv ur de ondas hand.
Jer 20:14  Förbannad vare den dag på vilken jag föddes; utan välsignelse blive den dag då min moder födde mig.
Jer 20:15  Förbannad vare den man som förkunnade för min fader: "Ett gossebarn är dig fött", och så gjorde honom stor glädje.
Jer 20:16  Gånge det den mannen såsom det gick de städer som HERREN omstörtade utan förbarmande. Må han få höra klagorop om morgonen och härskri om middagen.
Jer 20:17  därför att han icke dräpte mig strax i moderlivet, så att min moder fick bliva min grav och hennes liv vara havande för evigt.
Jer 20:18  Varför kom jag ut ur moderlivet och fick se olycka och bedrövelse, så att mina dagar måste försvinna i skam?
Jer 21:1  Detta är det ord som kom till Jeremia från HERREN, när konung Sidkia sände till honom Pashur, Malkias son, och prästen Sefanja, Maasejas son, och lät säga:
Jer 21:2  "Fråga HERREN för oss, då nu Nebukadressar, konungen i Babel, har angripit oss; kanhända vill HERREN handla med oss i enlighet med alla sina förra under, så att denne lämnar oss i fred.
Jer 21:3  Jeremia svarade dem: Så skolen I säga till Sidkia:
Jer 21:4  Så säger HERREN, Israels Gud Se, de vapen i eder hand, med vilka I utanför muren striden mot konungen i Babel och kaldéerna, som belägra eder, dem skall jag vända om och skall famla dem inne i denna stad.
Jer 21:5  Och jag skall själv strida mot eder med uträckt hand och stark arm, vrede och harm och stor förtörnelse.
Jer 21:6  Och jag skall slå dem som bo i denna stad, både människor och djur; i svår pest skola de dö.
Jer 21:7  Och därefter, säger HERREN, skall jag låta Sidkia, Juda konung, och hans tjänare och folket, dem som i denna stad äro kvar efter pesten svärdet och hungersnöden, falla i Nebukadressars, den babyloniske konungens, hand och i deras fienders hand, i de mäns hand, som stå efter deras liv. Och han skall slå dem med svärdsegg; han skall icke skona dem och icke hava någon misskund eller något förbarmande.
Jer 21:8  Och till detta folk skall du säga Så säger HERREN: Se, jag förelägger eder vägen till livet och vägen till döden.
Jer 21:9  Den som stannar kvar i denna stad, han skall dö genom svärd eller hunger eller pest, men den som går ut och giver sig åt kaldéerna, som belägra eder, han skall få leva och vinna sitt liv såsom ett byte.
Jer 21:10  Ty jag har vänt mitt ansikte mot denna stad, till dess olycka och icke till dess lycka, säger HERREN. Den skall bliva given i den babyloniske konungens hand, och han skall bränna upp den i eld.
Jer 21:11  Och till Juda konungs hus skall du säga: Hören HERRENS ord:
Jer 21:12  I av Davids hus, så säger HERREN: "Fällen var morgon rätt dom, och rädden den plundrade ur förtryckarens hand, för att icke min vrede må bryta fram såsom en eld och brinna så, att ingen kan utsläcka den" - detta för deras onda väsendes skull.
Jer 21:13  Se, jag skall vända mig mot dig, du som bor i dalen, du bergfäste på slätten, säger HERREN, ja, mot eder som sägen: "Vem kan falla över oss, och vem kan tränga in i våra boningar?"
Jer 21:14  Jag skall hemsöka eder efter edra gärningars frukt, säger HERREN. Ja, jag skall tända upp en eld i deras skog, och den skall förtära allt där runt omkring.
Jer 22:1  Så sade HERREN: Gå ned till Juda konungs hus och tala där följande ord;
Jer 22:2  säg: Hör HERRENS ord, du Juda konung, som sitter på Davids tron, hör det du med dina tjänare och ditt folk, I som gån in genom dessa portar.
Jer 22:3  Så säger HERREN: Öven rätt och rättfärdighet, och rädden den plundrade ur förtryckarens hand; förorätten icke främlingen, den faderlöse och änkan, gören icke övervåld mot dem, och utgjuten icke oskyldigt blod på denna plats.
Jer 22:4  Ty om I gören efter detta ord, så skola konungar som komma att sitta på Davids tron få draga in genom portarna till detta hus, på vagnar och hästar, följda av sina tjänare och sitt folk.
Jer 22:5  Men om I icke hören dessa ord, då har jag svurit vid mig själv, säger HERREN, att detta hus skall bliva ödelagt.
Jer 22:6  Ty så säger HERREN om Juda konungs hus: Väl är du för mig såsom ett Gilead, såsom Libanons topp; men jag skall sannerligen göra dig till en öken, till obebodda städer.
Jer 22:7  Och jag skall inviga fördärvare till att komma över dig, var och en med sina vapen, och de skola hugga ned dina väldiga cedrar och kasta dem i elden
Jer 22:8  Och många folk skola gå fram vid denna stad, och man skall fråga varandra: "Varför har HERREN gjort så mot denna stora stad?"
Jer 22:9  Och man skall då svara Därför att de övergåvo HERREN sin Guds, förbund och tillbådo andra gudar och tjänade dem."
Jer 22:10  Gråten icke över en död man, och ömken honom icke; men gråten bitterligen över honom som har måst vandra bort, ty han skall icke mer komma tillbaka och återse sitt fädernesland.
Jer 22:11  Ty så säger HERREN om Sallum, Josias son, Juda konung, som blev konung efter sin fader Josia, och som har dragit bort ifrån denna plats: Han skall icke mer komma hit tillbaka,
Jer 22:12  utan på den ort dit han har blivit bortförd i fångenskap, där skall han dö; detta land skall han icke mer få återse.
Jer 22:13  Ve dig, du som bygger ditt hus med orättfärdighet och dina salar med orätt, du som låter din nästa arbeta för intet och icke giver honom hans lön,
Jer 22:14  du som säger: "Jag vill bygga mig ett stort hus med rymliga salar", och så gör åt dig vida fönster och belägger huset med cederträ och målar det rött med dyrbar färg!
Jer 22:15  Kallar du det att vara konung, att du ävlas med att bygga cederhus? Din fader åt ju och drack, dock övade han rätt och rättfärdighet; och då gick det honom väl.
Jer 22:16  Han skaffade den betryckte och fattige rätt; och då gick det väl. Är icke detta att känna mig? säger HERREN.
Jer 22:17  Men dina ögon och ditt hjärta stå allenast efter vinning och efter att utgjuta den oskyldiges blod och att öva förtryck och våld.
Jer 22:18  Därför säger HERREN så om Jojakim, Josias son, Juda konung: Man skall ej hålla dödsklagan efter honom och ropa: "Ack ve, min broder! Ack ve, syster!" Man skall ej hålla dödsklagan efter honom och ropa: "Ack ve, herre! Ack ve, huru härlig han var!"
Jer 22:19  Såsom man begraver en åsna, så skall han begravas; han skall släpas ut och kastas bort, långt utanför Jerusalems portar.
Jer 22:20  Stig upp på Libanon och ropa, häv upp din röst i Basan, och ropa från Abarim, ty alla dina älskare äro krossade.
Jer 22:21  Jag talade till dig, när det gick dig väl, men du sade: "Jag vill icke höra." Sådan har din väg varit allt ifrån din ungdom, att du icke har velat höra min röst.
Jer 22:22  Alla dina herdar skola nu få en stormvind till sin herde, och dina älskare måste gå i fångenskap. Ja, då skall du komma på skam och få blygas för all din ondskas skull.
Jer 22:23  Du som bor på Libanon, du som har ditt näste i cedrarna, huru skall du icke jämra dig, när vånda kommer över dig, ångest lik en barnaföderskas!
Jer 22:24  Så sant jag lever, säger HERREN, om du, Konja, Jojakims son, Juda konung, än vore en signetring på min högra hand, så skulle jag dock rycka dig därifrån.
Jer 22:25  Och jag skall giva dig i de mäns hand, som stå efter ditt liv, och i de mäns hans som du fruktar för, nämligen i Nebukadressars, den babyloniske konungens, hand och i kaldéernas hand.
Jer 22:26  Och dig och din moder, den som har fött dig, skall jag slunga bort till ett annat land, där I icke ären födda; och där skolen I dö.
Jer 22:27  Till det land dit deras själ längtar att återvända, dit skola de icke få vända åter.
Jer 22:28  Är då han, denne Konja, ett föraktligt, krossat beläte eller ett värdelöst kärl? Eller varför hava de blivit bortslungade, han och hans avkomlingar, och kastade bort till ett land som de icke hava känt?
Jer 22:29  O land, land, land, hör HERRENS ord!
Jer 22:30  Så säger HERREN: Tecknen upp denne man såsom barnlös, såsom en man som ingen lycka har haft i sina livsdagar. Ty ingen av hans avkomlingar skall vara så lyckosam att han får sitta på Davids tron och i framtiden råda över Juda.
Jer 23:1  Ve över de herdar som fördärva och förskingra fåren i min hjord! säger HERREN.
Jer 23:2  Därför säger HERREN, Israels Gud, så om de herdar som föra mitt folk i bet: Det är I som haven förskingrat mina får och drivit bort dem och underlåtit att söka deras bästa. Men se, nu skall jag hemsöka eder för edert onda väsendes skull, säger HERREN.
Jer 23:3  Och jag skall själv församla kvarlevan av mina får ur alla de länder till vilka jag har drivit dem bort, och skall föra dem tillbaka till deras betesmarker, och de skola bliva fruktsamma och föröka sig.
Jer 23:4  Och jag skall låta herdar uppstå åt dem, vilka skola föra dem i bet; och de skola icke mer behöva frukta eller förskräckas och skola icke mer drabbas av hemsökelse, säger HERREN.
Jer 23:5  Se, dagar skola komma, säger HERREN, då jag skall låta en rättfärdig telning uppstå åt David. Han skall regera såsom konung och hava framgång, och han skall skaffa rätt och rättfärdighet på jorden.
Jer 23:6  I hans dagar skall Juda varda frälst och Israel bo i trygghet; och detta skall vara det namn han skall få: HERREN vår rättfärdighet.
Jer 23:7  Se, därför skola dagar komma, säger HERREN, då man icke mer skall säga: "Så sant HERREN lever, han som har fört Israels barn upp ur Egyptens land",
Jer 23:8  utan: "Så sant HERREN lever, han som har fört upp Israels hus släkt och hämtat dem ut ur nordlandet och ur alla andra länder till vilka jag hade drivit dem bort." Och så skola de få bo i sitt land.
Jer 23:9  Om profeterna. Mitt hjärta vill brista i mitt bröst, alla ben i min kropp äro vanmäktiga. Jag är såsom en drucken man, en man överväldigad av vin, inför HERREN och inför hans heliga ord.
Jer 23:10  Ty landet är fullt av äktenskapsbrytare, under förbannelse ligger landet sörjande, och betesmarkerna i öknen äro förtorkade; man hastar till vad ont är och har sin styrka i orättrådighet.
Jer 23:11  Ty både profeter och präster äro gudlösa; ända inne i mitt hus har jag mött deras ondska, säger HERREN.
Jer 23:12  Därför skall deras väg bliva för dem såsom en slipprig stig i mörkret, de skola på den stöta emot och falla. Ty jag vill låta olycka drabba dem, när deras hemsökelses är kommer, säger HERREN.
Jer 23:13  Väl såg jag ock hos Samarias profeter vad förvänt var; de profeterade i Baals namn och förde mitt folk Israel vilse.
Jer 23:14  Men hos Jerusalems profeter har jag sett de gruvligaste ting: de leva i äktenskapsbrott och fara med lögn; de styrka modet hos dem som göra ont, så att ingen vill omvända sig från sin ondska. De äro alla för mig såsom Sodom, och stadens invånare såsom Gomorras.
Jer 23:15  Därför säger HERREN Sebaot så om profeterna: Se, jag skall giva dem malört att äta och gift att dricka, ty från profeterna i Jerusalem har gudlöshet gått ut över hela landet.
Jer 23:16  Så säger HERREN Sebaot: Hören icke på de profeters ord, som profetera för eder, ty de bedraga eder; sina egna hjärtans syner tala de, icke vad som kommer från HERRENS mun.
Jer 23:17  De säga alltjämt till dem som förakta mig: "HERREN har så talat: Det skall gå eder väl." Och till var och en som vandrar i sitt hjärtas hårdhet säga de: "Ingen olycka skall komna över eder."
Jer 23:18  Vilken av dem har då fått tillträde till HERRENS råd, så att han kan förnimma och höra hans ord? Och vilken har aktat på hans ord och lyssnat därtill?
Jer 23:19  Se, en stormvind från HERREN är här, hans förtörnelse bryter fram, en virvlande storm! Över de ogudaktigas huvuden virvlar den ned.
Jer 23:20  Och HERRENS vrede skall icke upphöra, förrän han har utfört och fullbordat sitt hjärtas tankar; i kommande dagar skolen I förvisso förnimma det.
Jer 23:21  Jag sände icke dessa profeter, utan själva lupo de åstad; jag talade icke till dem, utan själva profeterade de.
Jer 23:22  Om de verkligen hade tillträde till mitt råd, så borde de förkunna mina ord för mitt folk och förmå dem att vända om från sin onda väg och sitt onda väsende.
Jer 23:23  Ar jag väl en Gud allenast på nära håll, säger HERREN, och icke en Gud också i fjärran?
Jer 23:24  Eller skulle någon kunna gömma sig på ett så lönnligt ställe att jag icke skulle se honom? säger HERREN. Är jag icke den som uppfyller himmel och jord? säger HERREN.
Jer 23:25  Jag har hört vad profeterna säga, de som profetera lögn i mitt namn; de säga: "Jag har haft en dröm, jag har haft en dröm."
Jer 23:26  Huru länge skall detta vara? Hava de något i sinnet, dessa profeter som profetera lögn, och som äro profeter genom sina egna hjärtans svek,
Jer 23:27  dessa som tänka att de genom sina drömmar; dem som de förtälja för varandra, skola komma mitt folk att förgäta mitt namn, likasom deras fäder glömde mitt namn för Baal?
Jer 23:28  Den profet som har haft en dröm, han må förtälja sin dröm; men den som bar undfått mitt ord, han må tala mitt ord i sanning. Vad har halmen att skaffa med säden? säger HERREN.
Jer 23:29  Är icke mitt ord såsom en eld, säger HERREN, och likt en hammare som krossar sönder klippor?
Jer 23:30  Se, därför skall jag komma över profeterna, säger HERREN, dessa som stjäla mina ord, den ene från den andre;
Jer 23:31  ja, jag skall komma över profeterna, säger HERREN, dessa som frambära sin egen tungas ord, men säga: "Så säger HERREN."
Jer 23:32  Ja, jag skall komma över dem som profetera lögndrömmar, säger HERREN, och som, när de förtälja dem, föra mitt folk vilse med sina lögner och sin stortalighet, fastän jag icke har sänt dem eller givit dem något uppdrag, och fastän de alls icke kunna hjälpa detta folk, säger HERREN.
Jer 23:33  Om nu detta folk eller en profet eller en präst gör dig denna fråga: "Vad förkunnar HERRENS tunga?", så skall du säga till dem vad som är den verkliga "tungan", och att jag därför skall kasta eder bort, säger HERREN.
Jer 23:34  Och den profet eller den präst eller den av folket, som säger "HERRENS tunga", den mannen och hans hus skall jag hemsöka.
Jer 23:35  Nej, så skolen I fråga varandra och säga eder emellan: "Vad har HERREN svarat?", eller: "Vad har HERREN talat?"
Jer 23:36  Men Om HERRENS "tunga" mån I icke mer orda; ty en tunga skall då vars och ens eget ord bliva för honom, eftersom I förvänden den levande Gudens, HERRENS Sebaots, vår Guds, ord.
Jer 23:37  Så skall du säga till profeten: "Vad har HERREN svarat dig?", eller: vad har HERREN talat?"
Jer 23:38  Men om I sägen "HERRENS tunga", då säger HERREN så: "Eftersom I sägen detta ord 'HERRENS tunga', fastän jag har sänt bud till eder och låtit säga: I skolen icke säga 'HERRENS tunga',
Jer 23:39  därför skall jag nu alldeles förgäta eder och kasta eder bort ifrån mitt ansikte, med den stad som jag har givit åt eder och edra fäder.
Jer 23:40  Och jag skall låta en evig smälek komma över eder, och en evig blygd, som icke skall varda förgäten."
Jer 24:1  HERREN lät mig se följande syn: Jag fick se två korgar med fikon uppställda framför HERRENS tempel; och det var efter det att Nebukadressar, konungen i Babel, hade fört bort ifrån Jerusalem Jekonja, Jojakims son, konungen i Juda, så ock Juda furstar, jämt timmermännen och smederna, och låtit dem komma till Babel.
Jer 24:2  I den ena korgen funnos mycket goda fikon, sådana som fikon ifrån förstlingsskörden äro; och i de andra korgen funnos mycket usla fikon, så usla att de icke kunde ätas.
Jer 24:3  Och HERREN sade till mig: "Vad ser du, Jeremia?" Jag svarade "Fikon; och de goda fikonen är mycket goda, men de usla fikonen äro mycket usla, så usla att de icke kunna ätas."
Jer 24:4  Och HERRENS ord kom till mig han sade:
Jer 24:5  Så säger HERREN, Israels Gud: Såsom man med välbehag ser på de goda fikonen, så vill jag med välbehag se till de bortförda av Juda, dem som jag från denna plats har sänt bort till kaldéernas land.
Jer 24:6  Jag skall med välbehag vända mitt öga till dem och låta dem komma tillbaka till detta land. Jag skall uppbygga dem och icke slå ned dem; jag skall plantera dem och icke upprycka dem.
Jer 24:7  Och jag skall giva dem hjärtan till att känna att jag är HERREN; och de skola vara mitt folk, och jag skall vara deras Gud. Ty de skola omvända sig till mig av allt sitt hjärta.
Jer 24:8  Men såsom man gör med usla fikon, som äro så usla att de icke kunna ätas, likaså, säger HERREN, skall jag göra med Sidkia, Juda konung, och med hans furstar och med kvarlevan i Jerusalem, ja, både med dem som hava blivit kvar här i landet och med dem som hava bosatt sig i Egyptens land.
Jer 24:9  På alla orter dit jag fördriver dem skall jag göra dem till en varnagel och en skräckbild för alla riken jorden, till en smälek, till ett ordspråk och en visa, och till ett exempel som man nämner, när man förbannar.
Jer 24:10  Och jag skall sända bland dem svärd, hungersnöd och pest, till dess att de bliva utrotade ur det land som jag har givit åt dem och deras fäder.
Jer 25:1  Detta är det ord som kom till Jeremia angående hela Juda folk, i Jojakims, Josias sons, Juda konungs, fjärde regeringsår, vilket var Nebukadressars, den babyloniske konungens, första regeringsår.
Jer 25:2  Och detta ord talade profeten Jeremia till hela Juda folk och till alla Jerusalems invånare; han sade:
Jer 25:3  Allt ifrån Josias, Amons sons, Juda konungs, trettonde regeringsår ända till denna dag, eller nu under tjugutre år, har HERRENS ord kommit till mig; men fastän jag titt och ofta har talat till eder, haven I icke velat höra.
Jer 25:4  Och fastän HERREN titt och ofta har sänt till eder alla sina tjänare profeterna, haven I icke velat höra. I böjden icke edra öron till att höra,
Jer 25:5  när de sade: "Vänden om, var och en från sin onda väg och sitt onda väsende, så skolen I för evärdliga tider få bo kvar i det land som HERREN har givit åt eder och edra fäder.
Jer 25:6  Och följen icke efter andra gudar, så att I tjänen och tillbedjen dem; och förtörnen mig icke genom edra händers verk, på det att jag icke må låta olycka komma över eder.
Jer 25:7  I villen icke höra på mig, säger HERREN, och så förtörnaden I mig genom edra händers verk, eder själva till olycka.
Jer 25:8  Därför säger HERREN Sebaot så: Eftersom I icke villen höra mina ord,
Jer 25:9  därför skall jag sända åstad och hämta alla nordens folkstammar, säger HERREN, och skall sända bud till min tjänare Nebukadressar, konungen i Babel; och jag skall låta dem komma över detta land och dess inbyggare, så ock över alla folken här runt omkring. Och dem skall jag giva till spillo, och skall göra dem till ett föremål för häpnad och begabberi, och låta deras land bliva ödemarker för evärdlig tid.
Jer 25:10  Och jag skall i dem göra slut på fröjderop och glädjerop, på rop för brudgum och rop för brud, på buller av kvarn och ljus från lampa.
Jer 25:11  Ja, hela detta land skall bliva ödelagt och förött, och dessa folk skola vara Babels konung underdåniga i sjuttio år.
Jer 25:12  Men när sjuttio år äro till ända skall jag hemsöka konungen i Babel och folket där för deras missgärning, säger HERREN, och hemsöka kaldéernas land och göra det till en ödemark för evärdlig tid.
Jer 25:13  Och jag skall på det landet låta alla de ord fullbordas, som jag har talat mot det, allt vad som är skrivet i denna bok, och vad Jeremia har profeterat mot alla dessa folk.
Jer 25:14  Ty också dem skola mäktiga folk och stora konungar göra sig underdåniga, och jag skall vedergälla dem efter deras gärningar och deras händers verk.
Jer 25:15  Ty så sade HERREN, Israels Gud, till mig: "Tag denna kalk med vredesvin ur min hand, och giv alla de folk till vilka jag sänder dig att dricka därur.
Jer 25:16  Må de dricka, så att de ragla och mista sansen, när det svärd kommer, som jag skall sända ibland dem."
Jer 25:17  Och jag tog kalken ur HERRENS hand och gav alla de folk att dricka, till vilka HERREN sände mig,
Jer 25:18  nämligen Jerusalem med Juda städer och med dess konungar och furstar, för att så göra dem till en ödemark, och till ett föremål för häpnad och begabberi, och till ett exempel som man nämner, när man förbannar, såsom ock nu har skett;
Jer 25:19  vidare Farao, konungen i Egypten, med hans tjänare, hans furstar och allt hans folk,
Jer 25:20  så ock allt Erebs folk med alla konungar i Us' land och alla konungar i filistéernas land, både Askelon och Gasa och Ekron och kvarlevan i Asdod;
Jer 25:21  vidare Edom, Moab och Ammons barn;
Jer 25:22  vidare alla konungar i Tyrus, alla konungar i Sidon och konungarna i kustländerna på andra sidan havet;
Jer 25:23  vidare Dedan, Tema, Bus och alla dem som hava kantklippt hår;
Jer 25:24  vidare alla konungar i Arabien och alla konungar över Erebs folk, som bo i öknen,
Jer 25:25  så ock alla konungar i Simri, alla konungar i Elam och alla konungar i Medien,
Jer 25:26  slutligen alla konungar i nordlandet - både dem som bo nära och dem som bo fjärran, den ene såväl som den andre - och alla övriga riken i världen, utöver jordens yta. Och Sesaks konung skall dricka efter dem.
Jer 25:27  Och du skall säga till dem: Så säger HERREN Sebaot, Israels Gud: Dricken, så att I bliven druckna, och spyn, och fallen omkull utan att kunna stå upp; ja, fallen, när det svärd kommer, som jag skall sända bland eder. -
Jer 25:28  Men om de icke vilja taga emot kalken ur din hand och dricka, så säg till dem: Så säger HERREN Sebaot: I måsten dricka.
Jer 25:29  Ty se, med den stad som är uppkallad efter mitt namn skall jag begynna hemsökelsen. Skullen då I bliva ostraffade? Nej, I skolen icke bliva ostraffade, utan jag skall båda upp ett svärd mot jordens alla inbyggare, säger HERREN Sebaot.
Jer 25:30  Och du skall profetera för dem allt detta och säga till dem: HERREN upphäver ett rytande från höjden och från sin heliga boning låter han höra sin röst; ja, han upphäver ett högt rytande över sin ängd och höjer skördeskri, såsom en vintrampare, över alla jordens inbyggare.
Jer 25:31  Dånet höres intill jordens ända, ty HERREN har sak med folken, han går till rätta med allt kött; de ogudaktiga giver han till pris åt svärdet, säger HERREN.
Jer 25:32  Så säger HERREN Sebaot: Se, en olycka går fram ifrån det ena folket till det andra, och ett stort oväder stiger upp från jordens yttersta ända.
Jer 25:33  Och de som bliva slagna av HERREN på den tiden skola ligga strödda från jordens ena ända till den andra; man skall icke hålla dödsklagan efter dem eller samla dem tillhopa och begrava dem, utan de skola bliva gödsel på marken.
Jer 25:34  Jämren eder, I herdar, och klagen; vältren eder på marken, I väldige i hjorden; ty tiden är inne, att I skolen slaktas. I skolen bliva förskingrade, I skolen komma på fall, såsom det händer jämväl ett dyrbart kärl.
Jer 25:35  Då finnes icke mer någon undflykt för herdarna, icke mer någon räddning för de väldige i hjorden.
Jer 25:36  Hör huru herdarna ropa, huru de väldige i hjorden jämra sig! Ty HERREN ödelägger deras betesmark,
Jer 25:37  och de fredliga ängderna förgöras genom HERRENS vredes glöd.
Jer 25:38  Han drager ut såsom ett lejon ur sitt snår. Ja, deras land bliver en ödemark under förhärjelsens glöd, under hans vredes glöd.
Jer 26:1  I begynnelsen av Jojakims, Josias sons, Juda konungs, regering kom detta ord från HERREN; han sade:
Jer 26:2  Så säger HERREN: Ställ dig i förgården till HERRENS hus och tala mot alla Juda städer, från vilka man kommer för att tillbedja i HERRENS hus, tala alla de ord som jag har bjudit dig tala till dem; tag intet därifrån.
Jer 26:3  Kanhända skola de då höra och vända om, var och en från sin onda väg; då vill jag ångra det onda som jag har i sinnet att göra med dem för deras onda väsendes skull.
Jer 26:4  Du skall säga till dem: Så säger HERREN: Om I icke viljen höra mig och vandra efter den lag som jag har förelagt eder,
Jer 26:5  och höra vad mina tjänare profeterna tala - de som jag titt och ofta sänder till eder, fastän I icke viljen höra -
Jer 26:6  då skall jag göra med detta hus såsom jag gjorde med Silo, och skall låta denna stad för alla jordens folk bliva ett exempel som man nämner, när man förbannar.
Jer 26:7  Och prästerna och profeterna och allt folket hörde Jeremia tala dessa ord i HERRENS hus.
Jer 26:8  Och när Jeremia hade slutat att tala allt vad HERREN hade bjudit honom tala till allt folket, grepo honom prästerna och profeterna och allt folket och sade: "Du måste döden dö.
Jer 26:9  Huru djärves du profetera i HERRENS namn och säga: 'Det skall gå detta hus likasom det gick Silo, och denna stad skall ödeläggas, så att ingen mer bor däri'?" Och allt folket församlade sig mot Jeremia i HERRENS hus.
Jer 26:10  Då nu Juda furstar hörde detta, gingo de från konungshuset upp till HERRENS hus och satte sig vid ingången till HERRENS nya port.
Jer 26:11  Då sade prästerna och profeterna till furstarna och till allt folket sålunda: "Denne man förtjänar döden, ty han har profeterat mot denna stad, såsom I haven hört med egna öron."
Jer 26:12  Men Jeremia svarade alla furstarna och allt folket och sade: "Det är HERREN som har sänt mig att profetera mot detta hus och denna stad allt det som I haven hört.
Jer 26:13  Så bättren nu edert leverne och edert väsende, och hören HERRENS, eder Guds, röst; då vill HERREN ångra det onda som han har talat mot eder.
Jer 26:14  Och vad mig angår, så är jag i eder hand; gören med mig vad eder gott och rätt synes.
Jer 26:15  Men det skolen I veta, att om I döden mig, så dragen I oskyldigt blod över eder och över denna stad och dess invånare; ty det är i sanning HERREN som har sänt mig till eder att tala allt detta inför eder."
Jer 26:16  Då sade furstarna och allt folket till prästerna och profeterna: "Denne man förtjänar icke döden, ty i HERRENS, vår Guds, namn har han talat till oss.
Jer 26:17  Och några av de äldste i landet stodo upp och sade till folkets hela församling sålunda:
Jer 26:18  "Morastiten Mika profeterade i Hiskias, Juda konungs, tid och sade till hela Juda folk: 'Så säger HERREN Sebaot: Sion skall varda upplöjt till en åker, och Jerusalem skall bliva en stenhop och tempelberget en skogbevuxen höjd.'
Jer 26:19  Men lät väl Hiskia, Juda konung, med hela Juda, döda honom? Fruktade han icke i stället HERREN och bönföll inför honom, så att HERREN ångrade det onda som han hade beslutit över dem, medan tvärtom vi nu stå färdiga att draga över oss själva så mycket ont?"
Jer 26:20  Där var ock en annan man, Uria, Semajas son, från Kirjat-Hajearim, som profeterade i HERRENS namn; och han profeterade mot denna stad och detta land alldeles såsom Jeremia hade gjort.
Jer 26:21  När då konung Jojakim med alla sina hjältar och alla furstar hörde vad han sade, ville han döda honom. Men när Uria fick höra härom, blev han förskräckt och flydde och kom till Egypten.
Jer 26:22  Då sände konung Jojakim några män till Egypten, nämligen Elnatan, Akbors son, och några andra med honom, in i Egypten.
Jer 26:23  Och dessa hämtade Uria ut ur Egypten och förde honom till konung Jojakim; och denne lät dräpa honom med svärd, och lät så kasta hans döda kropp på den allmänna begravningsplatsen.
Jer 26:24  Men Ahikam, Safans son, höll sin hand över Jeremia, så att man icke lämnade honom i folkets hand till att dödas.
Jer 27:1  I begynnelsen av Jojakims, Josias sons, Juda konungs, regering kom detta ord till Jeremia från HERREN;
Jer 27:2  han sade: Så har HERREN sagt till mig: Gör dig band och ok och sätt detta på din hals.
Jer 27:3  Sänd det sedan till konungen i Edom, konungen i Moab, konungen över Ammons barn, konungen i Tyrus och konungen i Sidon, genom de sändebud som hava kommit till Sidkia, Juda konung, i Jerusalem.
Jer 27:4  Och bjud dem med dessa ord framföra sitt budskap till sina herrar: Så säger HERREN Sebaot, Israels Gud: Så skolen I säga till edra herrar:
Jer 27:5  Jag är den som genom min stora kraft och min uträckta arm har gjort jorden, med de människor och djur som äro på jorden; och jag giver den åt vem jag vill.
Jer 27:6  Så giver jag nu alla dessa länder i min tjänare Nebukadnessars, den babyloniske konungens, hand; ja ock markens djur giver jag honom, för att de må tjäna honom.
Jer 27:7  Och alla folk skola vara honom och hans son och hans sonson underdåniga, till dess att också för hans land tiden är inne, att mäktiga folk och stora konungar skola göra honom sig underdånig.
Jer 27:8  Och det folk och det rike som icke vill vara honom, Nebukadnessar, konungen i Babel, underdånigt, och som icke vill giva sin hals under den babyloniske konungens ok, det folket skall jag hemsöka med svärd, hungersnöd och pest, säger HERREN, till dess att jag har förgjort dem genom hans hand.
Jer 27:9  Därför mån I icke höra på edra profeter och spåman, på edra drömmar, på edra teckentydare och trollkarlar, när dessa säga till eder: "I skolen icke komma att tjäna konungen i Babel";
Jer 27:10  ty de profetera lögn för eder, och komma så åstad att I bliven förda långt undan från edert land, i det jag måste driva eder bort, så att I förgåns.
Jer 27:11  Men det folk som böjer sin hals under den babyloniske konungens ok och tjänar honom, det skall jag låta få ro i sitt land, säger HERREN, så att de kunna bruka det och bo däri.
Jer 27:12  Till Sidkia, Juda konung, talade jag på alldeles samma sätt; jag sade: Böjen eder hals under den babyloniske konungens ok, och tjänen honom och hans folk, så skolen I få leva.
Jer 27:13  Icke viljen I dö, du och ditt folk, genom svärd, hunger och pest, såsom HERREN har sagt att det skall ske med det folk som icke vill tjäna konungen i Babel?
Jer 27:14  Hören alltså icke på de profeters ord, som säga till eder "I skolen icke komma att tjäna konungen i Babel"; ty de profetera lögn för eder.
Jer 27:15  Jag har icke sänt dem, säger HERREN; det är de själva som profetera lögn i mitt namn, och de komma så åstad att jag måste driva eder bort, så att I förgåns, jämte de profeter som profetera för eder.
Jer 27:16  Och till prästerna och till hela detta folk talade jag och sade: Så säger HERREN: Hören icke på edra profeters ord, när de profetera för eder och säga: "Se, de kärl som höra till HERRENS hus skola nu snart föras tillbaka från Babel"; ty de profetera lögn för eder.
Jer 27:17  Hören icke på dem, utan tjänen konungen i Babel, så skolen I få leva. Icke viljen I att denna stad skall bliva ödelagd?
Jer 27:18  Om de verkligen äro profeter och hava HERRENS ord, så må de lägga sig ut hos HERREN Sebaot, för att de kärl som ännu äro kvar i HERRENS hus och i Juda konungs hus och i Jerusalem icke också må föras bort till Babel
Jer 27:19  Ty så säger HERREN Sebaot om pelarna och havet och bäckenställen och det övriga som ännu är kvar här i staden,
Jer 27:20  därför att Nebukadnessar, konungen i Babel, icke tog det med sig, när han förde bort Jekonja, Jojakims son, Juda konung, från Jerusalem till Babel, jämte alla ädlingar i Juda och Jerusalem -
Jer 27:21  ja, så säger HERREN Sebaot, Israels Gud, om det som ännu är kvar här i HERRENS hus och i Juda konungs hus och i Jerusalem:
Jer 27:22  Till Babel skall det föras, och där skall det förbliva ända till den dag då jag ser därtill, säger HERREN, och för det upp till denna plats igen.
Jer 28:1  Men samma år, i begynnelsen av Sidkias, Juda konungs, regering, i femte månaden av hans fjärde regeringsår, talade profeten Hananja, Assurs son, från Gibeon, så till mig i HERRENS hus, i prästernas och allt folkets närvaro; han sade:
Jer 28:2  "Så säger HERREN Sebaot, Israels Gud: Jag skall sönderbryta den babyloniske konungens ok.
Jer 28:3  Inom två års tid skall jag föra tillbaka till denna plats alla de kärl i HERRENS hus, som Nebukadnessar, konungen i Babel, har tagit bort ifrån denna plats och fört till Babel.
Jer 28:4  Och Jekonja, Jojakims son, Juda konung, och alla fångar ifrån Juda, som hava kommit till Babel, skall jag föra tillbaka till denna plats, säger HERREN; ty jag skall sönderbryta den babyloniske konungens ok."
Jer 28:5  Men profeten Jeremia svarade profeten Hananja, i närvaro av prästerna och allt det folk som stod i HERRENS hus;
Jer 28:6  profeten Jeremia sade: "Amen. Så göre HERREN. Det som du har profeterat må HERREN uppfylla, i det att han för tillbaka från Babel till denna plats de kärl som funnos i HERRENS hus, så ock alla fångarna.
Jer 28:7  Men hör dock detta ord som jag vill tala inför dig och allt folket.
Jer 28:8  Forna tiders profeter, de som hava varit före mig och dig, hava mot mäktiga länder och stora riken profeterat om krig, olycka och pest.
Jer 28:9  Därför, om nu en profet profeterar om lycka, så kan man först då när den profetens ord går i fullbordan veta att han är en profet som HERREN i sanning har sänt."
Jer 28:10  Då tog profeten Hananja oket från profeten Jeremias hals och bröt sönder det.
Jer 28:11  Och Hananja sade i allt folkets närvaro: "Så säger HERREN: Just så skall jag inom två års tid bryta sönder den babyloniske konungen Nebukadnessars ok och taga det från alla folkens hals." Men profeten Jeremia gick sin väg.
Jer 28:12  Sedan, efter det att profeten Hananja hade brutit sönder oket och tagit det från profeten Jeremias hals, kom HERRENS ord till Jeremia; han sade:
Jer 28:13  "Gå åstad och säg till Hananja: Så säger HERREN: Ett ok av trä har du brutit sönder, men i dess ställe har du skaffat ett ok av järn.
Jer 28:14  Ty så säger HERREN Sebaot, Israels Gud: Ett ok av järn skall jag sätta på alla dessa folks hals, för att de må tjäna Nebukadnessar, konungen i Babel; ty honom skola de tjäna. Ja ock markens djur har jag givit honom."
Jer 28:15  Och profeten Jeremia sade ytterligare till profeten Hananja: "Hör, du Hananja: HERREN har icke sänt dig; du har förlett detta folk att sätta sin lit till lögn.
Jer 28:16  Därför säger HERREN så: Se, jag skall taga dig bort ifrån jorden. I detta år skall du dö, eftersom du har predikat avfall från HERREN."
Jer 28:17  Och samma år, i sjunde månaden, dog profeten Hananja.
Jer 29:1  Detta är vad som stod i det brev som profeten Jeremia sände från Jerusalem till de äldste som ännu levde kvar i fångenskapen, och till prästerna och profeterna och allt folket, dem som Nebukadnessar hade fört bort ifrån Jerusalem till Babel,
Jer 29:2  sedan konung Jekonja hade givit sig fången i Jerusalem, jämte konungamodern och hovmännen, Judas och Jerusalems furstar, så ock timmermännen och smederna.
Jer 29:3  Han sände brevet genom Eleasa, Safans son, och Gemarja, Hilkias son, när Sidkia, Juda konung, sände dessa till Babel, till Nebukadnessar, konungen i Babel; det lydde så:
Jer 29:4  Så säger HERREN Sebaot, Israels Gud, till alla de fångar som jag har låtit föra bort ifrån Jerusalem till Babel:
Jer 29:5  Byggen hus och bon i dem; planteren trädgårdar och äten deras frukt.
Jer 29:6  Tagen hustrur, och föden söner och döttrar; och tagen hustrur åt edra söner och given edra döttrar åt män, och må dessa föda söner och döttrar; och föröken eder där, och förminskens icke.
Jer 29:7  Och söken den stads bästa, dit jag har fört eder bort i fångenskap, och bedjen för den till HERREN; ty då det går den väl, så går det ock eder val.
Jer 29:8  Ty så säger HERREN Sebaot, Israels Gud: Låten icke bedraga eder av de profeter som äro bland eder, ej heller av edra spåman, och akten icke på de drömmar som I drömmen.
Jer 29:9  Ty man profeterar lögn för eder i mitt namn; jag har icke sänt dem, säger HERREN.
Jer 29:10  Ty så säger HERREN: Först när sjuttio år hava gått till ända i Babel, skall jag se till eder och uppfylla på eder mitt löftesord att föra eder tillbaka till denna plats.
Jer 29:11  Jag vet väl vilka tankar jag har för eder, säger HERREN, nämligen fridens tankar och icke ofärdens, till att giva eder en framtid och ett hopp.
Jer 29:12  Och I skolen åkalla mig och gå åstad och bedja till mig, och jag vill höra på eder.
Jer 29:13  I skolen söka mig, och I skolen ock finna mig, om I frågen efter mig av allt edert hjärta.
Jer 29:14  Ty jag vill låta mig finnas av eder, säger HERREN; och jag skall åter upprätta eder och skall församla eder från alla de folk och alla de arter till vilka jag har drivit eder bort, säger HERREN; och jag skall låta eder komma tillbaka till denna plats, varifrån jag har låtit föra eder bort i fångenskap.
Jer 29:15  Detta skriver jag, därför att I sägen: "HERREN har låtit profeter uppstå åt oss i Babel."
Jer 29:16  Ty så säger HERREN om den konung som sitter på Davids tron, och om allt det folk som bor i denna stad, edra bröder som icke hava med eder gått bort i fångenskap,
Jer 29:17  ja, så säger HERREN Sebaot: Se, jag skall sända mot dem svärd, hungersnöd och pest, och låta dem räknas lika med odugliga fikon, som äro så usla att man icke kan äta dem.
Jer 29:18  Ja, jag skall förfölja dem med svärd, hungersnöd och pest, och göra dem till en varnagel för alla riken på jorden, till ett exempel som man nämner, när man förbannar, till ett föremål för häpnad, begabberi och smälek bland alla de folk till vilka jag skall driva dem bort -
Jer 29:19  detta därför att de icke ville höra mina ord, säger HERREN, när jag titt och ofta sände till dem mina tjänare profeterna. Ty I villen ju icke höra, säger HERREN.
Jer 29:20  Men hören nu I HERRENS ord, alla I fångna som jag från Jerusalem har sänt bort till Babel:
Jer 29:21  Så säger HERREN Sebaot, Israels Gud, om Ahab, Kolajas son, och om Sidkia, Maasejas son, som i mitt namn profetera lögn för eder: Se, jag skall giva dem i Nebukadressars, den babyloniske konungens, hand, och han skall låta dräpa dem inför edra ögon.
Jer 29:22  Och alla fångar ifrån Juda, som äro i Babel, skola från dem hämta ett förbannelsens ord; de skola "HERREN göre med dig såsom med Sidkia och Ahab, vilka Babels konung lät steka i eld."
Jer 29:23  De hava ju gjort vad som är ens galenskap i Israel, de hava begått äktenskapsbrott med varandras hustrur och hava fört lögnaktigt tal i mitt namn, sådant som jag icke hade bjudit dem. Jag är den som vet det och betygar det, säger HERREN.
Jer 29:24  Och till nehelamiten Semaja skall du säga sålunda:
Jer 29:25  Så säger HERREN Sebaot, Israels Gud: Du har i ditt namn sänt brev till allt folket i Jerusalem och till prästen Sefanja, Maasejas son, och till alla de andra prästerna, så lydande:
Jer 29:26  "HERREN har satt dig till präst i prästen Jojadas ställe, för att i HERRENS hus skall finnas tillsyningsmän över alla vanvettingar som profetera, så att du kan sätta sådana i stock och halsjärn.
Jer 29:27  Varför har du då icke näpst Jeremia från Anatot, som profeterar för eder?
Jer 29:28  Därigenom att du har underlåtit detta har han kunnat sända bud till oss i Babel och låta säga: 'Ännu är lång tid kvar, byggen eder hus och bon i dem, och planteren trädgårdar och äten deras frukt.'"
Jer 29:29  Och prästen Sefanja har läst upp detta brev för profeten Jeremia.
Jer 29:30  Och nu har HERRENS ord kommit till Jeremia, han har sagt:
Jer 29:31  Sänd bud till alla de fångna och låt säga dem: Så säger HERREN om nehelamiten Semaja: Eftersom Semaja, utan att vara sänd av mig, har profeterat för eder och förlett eder att sätta eder lit till lögn,
Jer 29:32  därför säger HERREN så: Se, jag skall hemsöka nehelamiten Semaja och hans avkomlingar. Ingen av dem skall få bo ibland detta folk, och han skall icke få se det goda som jag vill göra med mitt folk, säger HERREN. Ty han har predikat avfall från HERREN.
Jer 30:1  Detta är det ord som kom till Jeremia från HERREN; han sade:
Jer 30:2  Så säger HERREN, Israels Gud: Teckna upp åt dig i en bok alla de ord som jag har talat till dig.
Jer 30:3  Ty se, dagar skola komma, säger HERREN, då jag åter skall upprätta mitt folk, Israel och Juda, säger HERREN, och låta dem komma tillbaka till det land som jag har givit åt deras fäder; och de skola taga det i besittning.
Jer 30:4  Och detta är vad HERREN har talat om Israel och Juda.
Jer 30:5  Så säger HERREN: Ett förfärans rop fingo vi höra; förskräckelse utan någon räddning!
Jer 30:6  Frågen efter och sen till: pläga då män föda barn? Eller varför ser jag alla män hålla sina händer på länderna såsom kvinnor i barnsnöd, och varför hava alla ansikten blivit så dödsbleka?
Jer 30:7  Ve! Detta är en stor dag, en sådan att ingen är den lik. Ja, en tid av nöd är inne för Jakob; dock skall han bliva frälst därur.
Jer 30:8  Och det skall ske på den tiden, Säger HERREN Sebaot, att jag skall bryta sönder oket och taga det från din hals och slita av dina band. Ja, inga främmande skola längre tvinga honom att tjäna sig,
Jer 30:9  utan han skall få tjäna HERREN, sin Gud, och David, sin konung, ty honom skall jag låta uppstå åt dem.
Jer 30:10  Så frukta nu icke, du min tjänare Jakob, säger HERREN. och var ej förfärad du Israel; ty se, jag skall frälsa dig ur det avlägsna landet, och dina barn ur deras fångenskaps land. Och Jakob skall få komma tillbaka och leva i ro och säkerhet, och ingen skall förskräcka honom.
Jer 30:11  Ty jag är med dig, säger HERREN, till att frälsa dig. Ja, jag skall göra ände på alla de folk; bland vilka jag har förstrött dig; men på dig vill jag icke alldeles göra ände, jag vill blott tukta dig med måtta; ty alldeles ostraffad kan jag ju ej låta dig bliva.
Jer 30:12  Ty så säger HERREN: Ohelbar är din skada, oläkligt det sår du har fått.
Jer 30:13  Ingen tager sig an din sak, så att han sköter ditt sår; ingen helande läkedom finnes för dig.
Jer 30:14  Alla dina älskare hava förgätit dig; de fråga icke efter dig. Ty såsom man slår en fiende, så har jag slagit dig, med grym tuktan, därför att din missgärning var så stor och dina synder så många.
Jer 30:15  Huru kan du klaga över din skada, över att bot ej finnes för din plåga? Därför att din missgärning var så stor och dina synder så många, har jag gjort dig detta.
Jer 30:16  Så skola då alla dina uppätare nu bliva uppätna, och alla dina ovänner skola allasammans gå i fångenskap; dina skövlare skola varda skövlade, och alla dina plundrare skall jag lämna till plundring.
Jer 30:17  Ty jag vill hela dina sår och läka dig från de slag du har fått, säger HERREN, då man nu kallar dig "den fördrivna", "det Sion som ingen frågar efter".
Jer 30:18  Så säger HERREN: Se, jag skall åter upprätta Jakobs hyddor och förbarma mig över hans boningar; staden skall åter bliva uppbyggd på sin höjd, och palatset skall stå på sin rätta plats.
Jer 30:19  Ifrån folket skall ljuda tacksägelse och rop av glada människor. Jag skall föröka dem, och de skola icke förminskas; jag skall låta dem komma till ära, och de skola ej aktas ringa.
Jer 30:20  Hans söner skola varda såsom fordom, hans menighet skall bestå inför mig, jag skall hemsöka alla hans förtryckare.
Jer 30:21  Hans väldige skall stamma från honom själv, och hans herre skall utgå från honom själv, och honom skall jag låta komma mig nära och nalkas mig; ty vilken annan vill våga sitt liv med att nalkas mig? säger HERREN.
Jer 30:22  Och I skolen vara mitt folk och jag skall vara eder Gud..
Jer 30:23  Se, en stormvind från HERREN är här, hans förtörnelse bryter fram, en härjande storm! Över de ogudaktigas huvuden virvlar den ned.
Jer 30:24  HERRENS vredes glöd skall icke upphöra, förrän han har utfört och fullbordat sitt hjärtas tankar; i kommande dagar skolen I förnimma det.
Jer 31:1  På den tiden, säger HERREN, skall jag vara alla Israels släkters Gud, och de skola vara mitt folk.
Jer 31:2  Så säger HERREN: Det folk som undslipper svärdet finner nåd i öknen; Israel får draga åstad dit där det får ro.
Jer 31:3  Fjärran ifrån uppenbarade sig HERREN för mig: "Ja, med evig kärlek har jag älskat dig; därför låter jag min nåd förbliva över dig.
Jer 31:4  Ännu en gång skall jag upprätta dig, så att du varder upprättad, du jungfru Israel; ännu en gång skall du få utrusta dig med puka och draga ut i dans bland dem som göra sig glada.
Jer 31:5  Ännu en gång skall du få plantera vingårdar på Samariens berg, och planteringsmännen skola själva skörda frukten.
Jer 31:6  Ty en dag kommer, då vaktare skola ropa på Efraims berg: 'Upp, låt oss draga till Sion, upp till HERREN, vår Gud.'"
Jer 31:7  Ty så säger HERREN: Jublen i glädje över Jakob, höjen fröjderop över honom som är huvudet bland folken, Låten lovsång ljuda och sägen: HERRE, giv frälsning åt ditt folk, åt kvarlevan av Israel."
Jer 31:8  Ja, jag skall föra dem från nordlandet och församla dem från jordens yttersta ända - bland dem både blinda och halta, både havande kvinnor och barnaföderskor; i en stor skara skola de komma hit tillbaka.
Jer 31:9  Under gråt skola de komma, men jag skall leda dem, där de gå bedjande fram; Jag skall föra dem till vattenbäckar, på en jämn väg, där de ej skola stappla. Ty jag har blivit en fader för Israel, och Efraim är min förstfödde son.
Jer 31:10  Hören HERRENS ord, I hednafolk, och förkunnen det i havsländerna i fjärran; sägen: Han som förskingrade Israel skall ock församla det och bevara det, såsom en herde sin hjord.
Jer 31:11  Ty HERREN skall förlossa Jakob och lösköpa honom ur den övermäktiges hand.
Jer 31:12  Och de skola komma och jubla på Sions höjd och strömma dit där HERRENS goda är, dit där man får säd, vin och olja och unga hjordar av får och fä; deras själ skall vara lik en vattenrik trädgård, och de skola icke vidare försmäkta.
Jer 31:13  Då skola jungfrurna förlusta sig med dans; unga och gamla skola glädja sig tillsammans. Jag skall förvandla deras sorg i fröjd, trösta dem och glädja dem efter deras bedrövelse.
Jer 31:14  Och prästerna skall jag vederkvicka med feta rätter; och mitt folk skall bliva mättat av mitt goda, säger HERREN.
Jer 31:15  Så säger HERREN: Ett rop höres i Rama, klagan och bitter gråt; det är Rakel som begråter sina barn, hon vill icke låta trösta sig i sorgen över att hennes barn icke mer äro till.
Jer 31:16  Men så säger HERREN: Hör upp med din högljudda gråt, och låt dina ögon icke mer fälla tårar; ty ditt verk skall få sin lön, säger HERREN, och de skola vända tillbaka från sina fienders land.
Jer 31:17  Ja, det finnes ett hopp för din framtid, säger HERREN; dina barn skola vända tillbaka till sitt land.
Jer 31:18  Jag har nogsamt hört huru Efraim klagar: "Du har tuktat mig, ja, jag har blivit tuktad såsom en otämd kalv; tag mig nu åter, så att jag får vända åter; du är ju HERREN, min Gud.
Jer 31:19  Ty sedan jag har vänt mitt sinne, ångrar jag mig, och sedan jag har kommit till besinning, slår jag mig på länden; jag både blyges och skämmes, då jag nu bär min ungdoms smälek."
Jer 31:20  Är då Efraim for mig en så dyrbar son, är han mitt älsklingsbarn, eftersom jag alltjämt tänker på honom, huru ofta jag än har måst hota honom? Ja, så mycket ömkar sig mitt hjärta över honom; jag måste förbarma mig över honom, säger HERREN.
Jer 31:21  Sätt upp vägmärken för dig, res åt dig vägvisare; giv akt på vägen, på stigen där du vandrade. Och vänd så tillbaka, du jungfru Israel, vänd tillbaka till dessa dina städer.
Jer 31:22  Huru länge skall du göra bukter hit och dit, du avfälliga dotter? Se, HERREN vill skapa något nytt i landet: det bliver nu kvinnan som tager mannen i sitt beskärm.
Jer 31:23  Så säger HERREN Sebaot, Israels Gud: I Juda land med dess städer skall man ännu en gång, när jag åter har upprättat det, få säga det ordet: "HERREN välsigne dig, du rättfärdighetens boning, du heliga berg."
Jer 31:24  Och Juda folk med alla sina städer skall samlat bo däri, åkermän jämte vandrande herdar.
Jer 31:25  Ty jag skall vederkvicka trötta själar, och alla försmäktande själar skall jag mätta.
Jer 31:26  (Härvid uppvaknade jag och såg mig om, och min sömn hade varit ljuvlig.)
Jer 31:27  Se, dagar skola komma, säger HERREN, då jag skall beså Israels land och Juda land med säd av människor och med säd av djur.
Jer 31:28  Och likasom jag har vakat över dem till att upprycka, nedbryta, fördärva, förgöra och plåga, så vill jag nu vaka över dem till att uppbygga och plantera, säger HERREN.
Jer 31:29  På den tiden skall man icke mer säga: "Fäderna hava ätit sura druvor, och barnens tänder bliva ömma därav."
Jer 31:30  Nej, var och en skall dö genom sin egen missgärning; var man som äter sura druvor, hans tänder skola bliva ömma därav.
Jer 31:31  Se, dagar skola komma, säger HERREN, då jag skall sluta ett nytt förbund med Israels hus och med Juda hus;
Jer 31:32  icke ett sådant förbund som det jag slöt med deras fäder på den dag då jag tog dem vid handen till att föra dem ut ur Egyptens land det förbund med mig, som de bröto, fastän jag var deras rätte herre, säger HERREN.
Jer 31:33  Nej, detta är det förbund som jag skall sluta med Israels hus i kommande dagar, säger HERREN: Jag skall lägga min lag i deras bröst och i deras hjärtan skall jag skriva den, och jag skall vara deras Gud, och de skola vara mitt folk.
Jer 31:34  Då skola de icke mer behöva undervisa varandra, icke den ene brodern den andre, och säga: "Lär känna HERREN"; ty de skola alla känna mig, från den minste bland dem till den störste, säger HERREN. Ty jag skall förlåta deras missgärning, och deras synd skall jag icke mer komma ihåg.
Jer 31:35  Så säger HERREN, han som har satt solen till att lysa om dagen och månen och stjärnorna till att lysa om natten, i ordnad gång, han som rör upp havet, så att dess böljor brusa, han vilkens namn är HERREN Sebaot:
Jer 31:36  Först när denna ordning icke mer består inför mig, säger HERREN, först då skall Israels släkt upphöra att inför mig alltjämt vara ett folk.
Jer 31:37  Ja, så säger HERREN: Först när himmelen varder uppmätt därovan och jordens grundvalar utrannsakade därnere, först då skall jag förkasta all Israels släkt, till straff för allt vad de hava gjort, säger HERREN.
Jer 31:38  Se, dagar skola komma, säger HERREN, då staden åter skall varda uppbyggd till HERRENS ära, från Hananeltornet intill Hörnporten.
Jer 31:39  Och mätsnöret skall vidare dragas rätt fram mot Garebshöjden och skall sedan vändas mot Goa.
Jer 31:40  Och hela lik- och askdalen och alla fälten intill bäcken Kidron och intill hörnet vid Hästporten österut skola vara helgade åt HERREN. Aldrig mer skall där tima någon omstörtning eller någon förstöring.
Jer 32:1  Detta är det ord som från HERREN kom till Jeremia i Sidkias, Juda konungs, tionde regeringsår, vilket var Nebukadressars adertonde regeringsår.
Jer 32:2  Vid den tiden belägrade den babyloniske konungens här Jerusalem, och profeten Jeremia låg då fången i fängelsegården i Juda konungs hus.
Jer 32:3  Ty Sidkia, Juda konung, hade låtit spärra in honom, i det han sade: "Huru djärves du profetera och säga: 'Så säger HERREN: Se, jag skall giva denna stad i de babyloniske konungens hand, och han skall intaga den.
Jer 32:4  Och Sidkia, Juda konung, skall icke undkomma kaldéernas hand, utan skall förvisso bliva given i den babyloniske konungens hand, så att han nödgas muntligen tala med honom och stå inför honom, öga mot öga.
Jer 32:5  Och Sidkia skall av honom föras till Babel och skall förbliva där, till dess jag ser till honom, säger HERREN. När I striden mot kaldéerna, skolen I icke hava någon framgång."
Jer 32:6  Och Jeremia sade: "HERRENS ord kom till mig; han sade:
Jer 32:7  Se, Hanamel, din farbroder Sallums son, skall komma till dig och säga: 'Köp du min åker i Anatot, ty du har såsom bördeman rätt att köpa den.'"
Jer 32:8  Och Hanamel, min farbroders son, kom till mig i fängelsegården, såsom HERREN hade sagt, och sade till mig: "Köp min åker i Anatot, i Benjamins land, ty du har arvsrätt därtill och är bördeman; så köp den då åt dig." Då förstod jag att det var HERRENS ord
Jer 32:9  och köpte åkern av Hanamel, min farbroders son, i Anatot, och vägde upp penningarna åt honom, sjutton siklar silver.
Jer 32:10  Jag skrev ett köpebrev och förseglade det och tillkallade vittnen och vägde upp penningarna på en våg.
Jer 32:11  Och jag tog köpebrevet, såväl det förseglade, som innehöll avtalet och de särskilda bestämmelserna, som ock det öppna brevet,
Jer 32:12  och gav köpebrevet åt Baruk, son till Neria, son till Mahaseja, i närvaro av min frände Hanamel och de vittnen som hade underskrivit köpebrevet, och alla andra judar som voro tillstädes i fängelsegården.
Jer 32:13  Och jag bjöd Baruk, i deras närvaro, och sade:
Jer 32:14  Så säger HERREN Sebaot, Israels Gud: Tag du dessa brev, både detta förseglade köpebrev och detta öppna brev, och lägg dem i ett lerkärl, för att de må vara i behåll för lång tid.
Jer 32:15  Ty så säger HERREN Sebaot, Israels Gud: Ännu en gång skall man komma att i detta land köpa hus och åkrar och vingårdar."
Jer 32:16  Och sedan jag hade givit köpebrevet åt Baruk, Nerias son, bad jag till HERREN och sade:
Jer 32:17  "Ack Herre, HERRE, du är ju den som har gjort himmel och jord genom din stora kraft och din uträckta arm. Intet är så underbart att du icke skulle förmå det,
Jer 32:18  du som gör nåd med tusenden och vedergäller fädernas missgärning i deras barns sköte efter dem; du store och väldige Gud, vilkens namn är HERREN Sebaot;
Jer 32:19  du som är stor i råd och mäktig i gärningar; du vilkens ögon äro öppna över människobarnens alla vägar, så att du giver åt var och en efter hans vägar och efter hans gärningars frukt;
Jer 32:20  du som gjorde tecken och under i Egyptens land, och som har gjort sådana intill denna dag, både i Israel och bland andra människor, och som har gjort dig ett namn, som är detsamma än i dag.
Jer 32:21  Du förde ditt folk Israel ut ur Egyptens land med tecken och under, med stark hand och uträckt arm, och genom stor förskräckelse.
Jer 32:22  Och du gav dem detta land, som du med ed hade lovat deras fäder att giva dem, ett land som flyter av mjölk och honung.
Jer 32:23  Och de kommo och togo det i besittning, men de ville icke höra din röst och vandrade icke efter din lag; de gjorde intet av det du hade bjudit dem att göra. Därför lät du all denna olycka vederfaras dem.
Jer 32:24  Se, belägringsvallarna gå redan så långt fram mot staden, att man kan intaga den och genom svärd, hungersnöd och pest är staden given i de kaldeiska belägrarnas hand. Vad du hotade med, det har skett, och du har det nu inför dina ögon.
Jer 32:25  Och likväl, fastän staden är given i kaldéernas hand, sade du, Herre HERRE, till mig: 'Köp du åkern för penningar, och tag vittnen därpå'!"
Jer 32:26  Och HERRENS ord kom till Jeremia han sade: Se,
Jer 32:27  jag är HERREN, allt kötts Gud; skulle något vara så underbart att jag icke förmådde det?
Jer 32:28  Därför säger HERREN så: Se, jag vill giva denna stad i kaldéernas och Nebukadressars, den babyloniske konungens, hand, och han skall intaga den.
Jer 32:29  Och kaldéerna, som belägra denna stad, skola komma och tända eld på staden och bränna upp den, tillika med de hus på vilkas tak man har tänt offereld åt Baal och utgjutit drickoffer åt andra gudar, till att förtörna mig.
Jer 32:30  Ty allt ifrån sin ungdom hava Israels barn och Juda barn allenast gjort vad ont är i mina ögon; ja, Israels barn hava med sina händers verk berett mig allenast förtörnelse, säger HERREN.
Jer 32:31  Ty allt ifrån den dag då denna stad byggdes ända till nu har den uppväckt min vrede och förtörnelse, så att jag måste förkasta den från mitt ansikte,
Jer 32:32  för all den ondskas skull som Israels barn och Juda barn med sina konungar, furstar, präster och profeter, både Juda män och Jerusalems invånare, hava bedrivit, till att förtörna mig.
Jer 32:33  De vände ryggen till mig och icke ansiktet; och fastän de titt och ofta blevo varnade, ville de icke höra och taga emot tuktan.
Jer 32:34  De satte upp sina styggelser i det hus som är uppkallat efter mitt namn och orenade det så;
Jer 32:35  och Baalshöjderna i Hinnoms sons dal byggde de upp, för att där offra sina söner och döttrar åt Molok, fastän jag aldrig hade bjudit dem att göra sådan styggelse eller ens tänkt mig något sådant; och så förledde de Juda till synd.
Jer 32:36  Men så säger nu HERREN, Israels Gud, om denna stad, som I menen vara genom svärd, hungersnöd och pest given i den babyloniske konungens hand:
Jer 32:37  Se, jag skall församla dem ur alla de länder till vilka jag i min vrede och harm och stora förtörnelse har fördrivit dem, och jag skall föra dem tillbaka till denna plats och låta dem bo bär i trygghet.
Jer 32:38  Och de skola vara mitt folk, och jag skall vara deras Gud.
Jer 32:39  Och jag skall giva dem alla ett och samma hjärta och lära dem en och samma väg, så att de frukta mig beständigt; för att det må gå dem väl, och deras barn efter dem.
Jer 32:40  Och jag skall sluta med dem ett evigt förbund, så att jag icke upp hör att följa dem och göra dem gott; och min fruktan skall jag ingiva i deras hjärtan, så att de icke vika av ifrån mig.
Jer 32:41  Och jag skall hava min fröjd i att göra dem gott, och skall plantera dem i detta land med trofasthet, av allt mitt hjärta och all min själ.
Jer 32:42  Ty så säger HERREN: Likasom jag har låtit all denna stora olycka komma över detta folk, så skall jag ock låta allt det goda som jag lovade dem komma dem till del.
Jer 32:43  Och man skall komma att köpa åkrar i detta land, om vilket I sägen att det är en ödemark, där varken människor eller djur kunna bo, och att det är givet i kaldéernas hand.
Jer 32:44  Ja, åkrar skall man köpa för penningar, och man skall skriva och försegla köpebrev och tillkalla vittnen i Benjamins land, i Jerusalems omnejd och i Juda städer, både i Bergsbygdens och i Låglandets och i Sydlandets städer; ty jag skall åter upprätta dem, säger HERREN.
Jer 33:1  Och HERRENS ord kom till Jeremia för andra gången, medan han ännu var inspärrad i fängelsegården; han sade:
Jer 33:2  Så säger HERREN, han som ock utför sitt verk, HERREN, som bereder det för att låta det komma till stånd, han vilkens namn är HERREN:
Jer 33:3  Ropa till mig, så vill jag svara dig och förkunna för dig stora och förunderliga ting, som du icke känner.
Jer 33:4  Ty så säger HERREN, Israels Gud, om husen i denna stad och om Juda konungars hus, som nu brytas ned för belägringsvallarna och värden:
Jer 33:5  Man har kommit hitin för att strida med kaldéerna, och man skall så fylla husen med döda kroppar av människor som jag har slagit i min vrede och förtörnelse, människor som genom all sin ondska hava vållat att jag har måst dölja mitt ansikte för denna stad.
Jer 33:6  Dock, jag skall hela dess sår och skaffa läkedom och läka dem, och jag skall låta dem skåda frid och trygghet i överflöd.
Jer 33:7  Och jag skall åter upprätta Juda och Israel och uppbygga dem, så att de bliva såsom förut.
Jer 33:8  Och jag skall rena dem från all missgärning varmed de hava syndat mot mig, och förlåta alla missgärningar genom vilka de hava syndat mot mig och avfallit från mig.
Jer 33:9  Och staden skall bliva mig till fröjd och berömmelse, och till lov och ära inför alla jordens folk, när de få höra allt det goda som jag gör med dem; och de skola förskräckas och darra vid åsynen av all den lycka och all den framgång som jag bereder henne.
Jer 33:10  Så säger HERREN: På denna plats, om vilken I sägen att den är så öde att varken människor eller djur kunna bo där, ja, här i Juda städer och på Jerusalems gator, som äro så ödelagda att inga människor, inga invånare, inga djur där finnas,
Jer 33:11  här skall man ännu en gång höra fröjderop och glädjerop, rop för brudgum och rop för brud, rop av människor som säga: "Tacken HERREN Sebaot, ty HERREN är god, ty hans nåd varar evinnerligen", och av människor som frambära lovoffer i HERRENS hus. Ty jag vill åter upprätta landet, så att det bliver såsom förut, säger HERREN.
Jer 33:12  Så säger HERREN Sebaot: På denna plats, som nu är så öde att varken människor eller ens djur kunna bo här, ja ock i alla hithörande städer, här skola åter en gång finnas betesmarker där herdar kunna låta sina hjordar lägra sig.
Jer 33:13  I Bergsbygdens, Låglandets och Sydlandets städer, i Benjamins land i Jerusalems omnejd och i andra Juda städer skola ännu en gång hjordar draga fram, förbi herdar som räkna dem, säger HERREN.
Jer 33:14  Se, dagar skola komma, säger HERREN, då jag skall uppfylla det löftesord som jag har talat om Israels hus och angående Juda hus.
Jer 33:15  I de dagarna och på den tiden skall jag låta en rättfärdig telning växa upp åt David. Han skall skaffa rätt och rättfärdighet på jorden.
Jer 33:16  I de dagarna skall Juda varda frälst och Jerusalem bo i trygghet; och man skall kalla det så: HERREN vår rättfärdighet.
Jer 33:17  Ty så säger HERREN: Aldrig skall den tid komma, då icke en avkomling av David sitter på Israels hus' tron,
Jer 33:18  aldrig den tid då icke en avkomling av de levitiska prästerna gör tjänst inför mig och alla dagar bär fram brännoffer och förbränner spisoffer och anställer slaktoffer.
Jer 33:19  Och HERRENS ord kom till Jeremia; han sade:
Jer 33:20  Så säger HERREN: Först när I gören om intet mitt förbund med dagen och mitt förbund med natten, så att det icke bliver dag och natt i rätt tid,
Jer 33:21  först då skall mitt förbund med min tjänare David bliva om intet, så att icke längre en avkomling av honom sitter såsom konung på hans tron, och först då mitt förbund med de levitiska prästerna, som göra tjänst åt mig.
Jer 33:22  Lika oräknelig som himmelens härskara är, och lika otalig som sanden är i havet, lika talrik skall jag låta min tjänare Davids säd bliva och lika många leviterna, som göra tjänst åt mig.
Jer 33:23  Och HERRENS ord kom till Jeremia; han sade:
Jer 33:24  Har du icke märkt huru detta folk talar och säger: "De båda släkter som HERREN utvalde, dem har han förkastat"? Och så säga de föraktligt om mitt folk att det icke mer synes dem vara ett folk.
Jer 33:25  Men så säger HERREN: Om mitt förbund med dag och natt icke är beståndande, och om jag icke har stadgat en fast ordning för himmel och jord,
Jer 33:26  allenast då skall jag förkasta Jakobs och Davids, min tjänares, säd, så att jag icke mer av hans säd tager dem som skola råda över Abrahams, Isaks och Jakobs säd. Ty jag skall åter upprätta dem och förbarma mig över dem.
Jer 34:1  Detta är det ord som kom till Jeremia från HERREN, när Nebukadressar, konungen i Babel, med hela sin här och med alla de riken på jorden, som lydde under hans välde, och med alla folk angrep Jerusalem och alla dess lydstäder; han sade:
Jer 34:2  Så säger HERREN, Israels Gud: Gå åstad och säg till Sidkia, Juda konung, ja, säg till honom: Så säger HERREN: Se, jag skall giva denna stad i den babyloniske konungens hand, och han skall bränna upp den i eld.
Jer 34:3  Och du själv skall icke kunna undkomma hans hand, utan skall förvisso bliva gripen och given i hans hand, så att du nödgas stå inför konungen i Babel, öga mot öga; och han skall muntligen tala med dig, och du skall komma till Babel.
Jer 34:4  Men hör HERRENS ord, du Sidkia, Juda konung: Så säger HERREN om dig: Du skall icke dö genom svärd.
Jer 34:5  Nej, i frid skall du dö; och likasom man har anställt förbränning till dina fäders, de förra konungarnas, ära, deras som hava varit före dig, så skall man ock anställa förbränning till din ära och hålla dödsklagan efter dig: "Ack ve, Herre!" Ty detta har jag talat, säger HERREN.
Jer 34:6  Och profeten Jeremia talade till Sidkia, Juda konung, allt detta i Jerusalem,
Jer 34:7  under det att den babyloniske konungens här belägrade Jerusalem och allt som ännu återstod av städer i Juda, nämligen Lakis och Aseka; ty dessa voro de enda av Juda städer, som ännu voro kvar och voro befästa.
Jer 34:8  Detta är det ord som kom till Jeremia från HERREN, sedan konung Sidkia hade slutit ett förbund med allt folket i Jerusalem därom att de bland sig skulle utropa frihet,
Jer 34:9  så att var och en skulle släppa sin träl och sin trälinna fria, om det var en hebreisk man eller kvinna, på det att icke den ene juden skulle hava den andre till träl.
Jer 34:10  Och detta hörsammades av alla furstarna och allt folket, av dem som hade varit med om förbundet och lovat att var och en skulle släppa sin träl och sin trälinna fria, så att han icke mer skulle hava dem till trälar; de hörsammade det och släppte dem.
Jer 34:11  Men sedermera ändrade de sig och togo tillbaka de trälar och trälinnor som de hade släppt fria, och gjorde dem åter till trälar och trälinnor.
Jer 34:12  Då kom HERRENS ord till Jeremia från HERREN; han sade:
Jer 34:13  Så säger HERREN, Israels Gud: Jag själv slöt ett förbund med edra fäder på den tid då jag förde dem ut ur Egyptens land, ur träldomshuset; jag sade:
Jer 34:14  "När sju år äro förlidna, skall var och en av eder släppa sin broder, hebréen, som har sålt sig åt dig och tjänat dig i sex år; du skall då släppa honom fri ur din tjänst." Dock ville edra fäder icke höra på mig eller böja sina öron därtill.
Jer 34:15  Men I haven nyss vänt om och gjort vad rätt är i mina ögon, i det att I haven utropat frihet var och en för sin broder. Och I haven härom slutit ett förbund inför mitt ansikte, i det hus som är uppkallat efter mitt namn.
Jer 34:16  Men nu haven I åter ändrat eder och ohelgat mitt namn och tagit tillbaka var och en sin träl och sin trälinna, dem som I haden släppt fria till att gå vart de ville; ja, I haven nu åter gjort dem till edra trälar och trälinnor.
Jer 34:17  Därför säger HERREN så: I haven icke hört på mig och utropat frihet var och en för sin broder och sin nästa. Så utropar då jag, säger HERREN, för eder frihet att hemfalla åt svärd, pest och hungersnöd; ja, jag skall göra eder till en varnagel för alla riken på jorden.
Jer 34:18  Och de män som hava överträtt mitt förbund och icke hållit förpliktelserna vid det förbund de slöto inför mitt ansikte - vid kalven som av dem blev huggen i två stycken, mellan vilka de gingo -
Jer 34:19  dessa män, nämligen Judas och Jerusalems furstar, hovmännen och prästerna och allt folket i landet, som gingo mellan styckena av kalven,
Jer 34:20  dem skall jag giva i deras fienders hand, i de mäns hand, som stå efter deras liv; och deras döda kroppar skola bliva mat åt himmelens fåglar och markens djur.
Jer 34:21  Och Sidkia, Juda konung, med hans furstar skall jag giva i deras fienders hand, i de mäns hand, som stå efter deras liv, och i händerna på den babyloniske konungens här, som nu har dragit bort ifrån eder.
Jer 34:22  Se, jag skall giva dem befallning, säger HERREN, att de åter skola draga mot denna stad och belägra den; och de skola då intaga den och bränna upp den i eld. Och Juda städer skall jag göra till en ödemark, där ingen bor.
Jer 35:1  Detta är det ord som kom till Jeremia från HERREN i Jojakims, Josias sons, Juda konungs, tid; han sade
Jer 35:2  Gå bort till rekabiternas släkt och tala med dem, och för dem till HERRENS hus, in i en av kamrarna, och giv dem vin att dricka.
Jer 35:3  Då tog jag med mig Jaasanja, son till Jeremia, son till Habassinja, jämte hans bröder och alla hans söner och rekabiternas hela övriga släkt,
Jer 35:4  och förde dem till HERRENS hus, in i den kammare som innehades av sönerna till gudsmannen Hanan, Jigdaljas son, den kammare som ligger bredvid furstarnas, ovanom dörrvaktaren Maasejas, Sallums sons, kammare.
Jer 35:5  Och jag satte fram för rekabiternas släkt kannor, fulla med vin, så ock bägare, och sade till dem: "Dricken vin?"
Jer 35:6  Men de svarade: "Vi dricka icke vin. Ty vår fader Jonadab, Rekabs son, har bjudit oss och sagt: 'I och edra barn skolen aldrig dricka vin;
Jer 35:7  och hus skolen I icke bygga, och säd skolen I icke så, och vingårdar skolen I icke plantera, ej heller äga sådana, utan I skolen bo i tält i all eder tid, för att I mån länge leva i det land där I bon såsom främlingar.'
Jer 35:8  Och vi hava hörsammat vår fader Jonadabs, Rekabs sons, befallning, i allt vad han har bjudit oss, så att vi med våra hustrur och våra söner och döttrar aldrig dricka vin,
Jer 35:9  ej heller bygga hus till att bo i, ej heller äga vingårdar eller åkrar eller säd.
Jer 35:10  Vi hava alltså bott i tält och hava hörsammat och gjort allt vad vår fader Jonadab har bjudit oss.
Jer 35:11  Men när Nebukadressar, konungen i Babel, drog upp och föll in i landet, sade vi: 'Välan, vi vilja begiva oss till Jerusalem, undan kaldéernas och araméernas här.' Och så bosatte vi oss i Jerusalem."
Jer 35:12  Och HERRENS ord kom till Jeremia; han sade:
Jer 35:13  Så säger HERREN Sebaot, Israels Gud: Gå åstad och säg till Juda män och till Jerusalems invånare: Skolen I då icke taga emot tuktan, så att I hören mina ord, säger HERREN?
Jer 35:14  Det bud som Jonadab, Rekabs son, gav sina barn, att de icke skulle dricka vin, det har blivit iakttaget, och ännu i dag dricka de icke vin, av hörsamhet mot sin faders bud. Men själv har jag titt och ofta talat till eder, och I haven dock icke hörsammat mig.
Jer 35:15  Och titt och ofta har jag sänt till eder alla mina tjänare profeterna och låtit säga: "Vänden om, var och en från sin onda väg, och bättren edert väsende, och följen icke efter andra gudar, så att I tjänen dem; då skolen I få bo i det land som jag har givit åt eder och edra fäder." Men I böjden icke edert öra därtill och hörden icke på mig.
Jer 35:16  Eftersom nu detta folk icke har hörsammat mig, såsom Jonadabs, Rekabs sons, barn hava iakttagit det bud som deras fader gav dem,
Jer 35:17  därför säger HERREN, härskarornas Gud, Israels Gud, så: Se, över Juda och över alla Jerusalems invånare skall jag låta all den olycka komma, som jag har förkunnat över dem, därför att de icke hörde, när jag talade till dem, och icke svarade, när jag kallade på dem.
Jer 35:18  Och till rekabiternas släkt sade Jeremia: Så säger HERREN Sebaot, Israels Gud: Därför att I haven hörsammat eder fader Jonadabs bud och hållit alla hans bud och i alla stycken gjort såsom han har bjudit eder,
Jer 35:19  därför säger HERREN Sebaot, Israels Gud, så: Aldrig skall den tid komma, då icke en avkomling av Jonadab, Rekabs son, står inför mitt ansikte.
Jer 36:1  I Jojakims, Josias sons, Juda konungs, fjärde regeringsår kom detta ord till Jeremia från HERREN; han sade:
Jer 36:2  Tag dig en bokrulle och teckna däri upp allt vad jag har talat till dig angående Israel och Juda och alla hednafolk, från den dag då jag först talade till dig i Josias tid ända till denna dag.
Jer 36:3  Kanhända skall Juda hus, när de höra all den olycka som jag har i sinnet att göra dem, vända om, var och en från sin onda väg, och så skall jag förlåta dem deras missgärning och synd.
Jer 36:4  Då kallade Jeremia till sig Baruk, Nerias son; och efter Jeremias diktamen tecknade Baruk i en bokrulle upp alla de ord som HERREN hade talat till honom.
Jer 36:5  Och Jeremia bjöd Baruk och sade: "Jag är själv under tvång, så att jag icke kan begiva mig till HERRENS hus.
Jer 36:6  Men gå du dit; och ur den rulle som du har skrivit efter min diktamen må du därpå fastedagen läsa UPP HERRENS ord inför folket i HERRENS hus. Inför hela Juda, så många som komma in från sina städer, må du ock läsa upp dem.
Jer 36:7  Kanhända skola de då bönfalla inför HERREN och vända om, var och en från sin onda väg. Ty stor är den vrede och förtörnelse som HERREN har uttalat över detta folk."
Jer 36:8  Och Baruk, Nerias son, gjorde alldeles såsom profeten Jeremia hade bjudit honom: i HERRENS hus läste han ur boken upp HERRENS ord.
Jer 36:9  I Jojakims, Josias sons, Juda konungs, femte regeringsår, i nionde månaden, utlystes nämligen en fasta inför HERREN, vilken hölls av allt folket i Jerusalem och av allt det folk som från Juda städer hade kommit till Jerusalem.
Jer 36:10  Då läste Baruk ur boken upp Jeremias ord; han läste upp dem i HERRENS hus, i sekreteraren Gemarjas, Safans sons, kammare på den övre förgården, vid ingången till nya porten på HERRENS hus, inför allt folket..
Jer 36:11  När nu Mika; son till Gemarja, son till Safan, hade hört alla HERRENS ord uppläsas ur boken,
Jer 36:12  gick han ned till konungshuset och in i sekreterarens kammare; där sutto då alla furstarna: sekreteraren Elisama, Delaja, Semajas son, Elnatan, Akbors son, Gemarja, Safans son, Sidkia, Hananjas son, och alla de andra furstarna.
Jer 36:13  Och Mika omtalade för dem allt vad han hade hört Baruk läsa upp ur boken inför folket.
Jer 36:14  Då sände alla furstarna Jehudi, son till Netanja, son till Selemja, Kusis son, åstad till Baruk och läto säga honom: "Tag med dig den rulle varur du har läst inför folket, och kom hit." Och Baruk, Nerias son, tog rullen med sig och kom till dem.
Jer 36:15  Då sade de till honom: "Sätt dig ned och läs den inför oss." Och Baruk läste inför dem.
Jer 36:16  När de då hörde allt som stod där, sågo de med förskräckelse på varandra och sade till Baruk: "Vi måste omtala för konungen allt som står här."
Jer 36:17  Och de frågade Baruk och sade: "Tala om för oss huru det skedde att du efter hans diktamen tecknade upp allt detta."
Jer 36:18  Baruk svarade dem: "Han dikterade för mig allt detta, och jag tecknade upp det i boken med bläck."
Jer 36:19  Då sade furstarna till Baruk: "Gå och göm dig, du jämte Jeremia, och låten ingen veta var I ären."
Jer 36:20  Därefter, sedan de hade lämnat rullen i förvar i sekreteraren Elisamas kammare, gingo de in till konungen på förgården och omtalade så allt för konungen.
Jer 36:21  Då sände konungen Jehudi att hämta rullen; och denne hämtade den från sekreteraren Elisamas kammare. Sedan läste Jehudi upp den inför konungen och inför alla furstarna, som stodo omkring konungen.
Jer 36:22  Konungen bodde då i vinterhuset, ty det var den nionde månaden. Och kolpannan stod påtänd framför honom;
Jer 36:23  och så ofta Jehudi hade läst tre eller fyra spalter, skar han av rullen med pennkniven och kastade stycket på elden i kolpannan, ända till dess att hela rullen var förtärd av elden i kolpannan.
Jer 36:24  Och varken konungen själv eller någon av hans tjänare blev förskräckt eller rev sönder sina kläder, när de hörde allt detta som upplästes.
Jer 36:25  Och fastän Elnatan, Delaja och Gemarja bådo konungen att han icke skulle bränna upp rullen, lyssnade han icke till dem.
Jer 36:26  I stället bjöd konungen Jerameel, konungasonen, och Seraja, Asriels son, och Selemja, Abdeels son, att de skulle gripa skrivaren Baruk och profeten Jeremia. Men HERREN gömde dem undan.
Jer 36:27  Men sedan konungen hade bränt upp rullen med det som Baruk efter Jeremias diktamen hade skrivit däri, kom HERRENS ord till Jeremia; han sade:
Jer 36:28  Tag dig nu åter en annan rulle och teckna däri upp allt vad som förut stod i den förra rullen, den som Jojakim, Juda konung, brände upp.
Jer 36:29  Men angående Jojakim, Juda konung, skall du säga: Så säger HERREN: Du har bränt upp denna rulle och sagt: "Huru kunde du skriva däri att konungen i Babel förvisso skall komma och fördärva detta land, och göra slut på både människor och djur däri?"
Jer 36:30  Därför säger HERREN så om Jojakim, Juda konung: Ingen ättling av honom skall sitta på Davids tron; och hans egen döda kropp skall komma att ligga utkastad, prisgiven åt hettan om dagen och åt kölden om natten.
Jer 36:31  Och jag skall hemsöka honom och hans avkomlingar och hans tjänare för deras missgärnings skull, och över dem och över Jerusalems invånare och över Juda män skall jag låta all den olycka komma, som jag har förkunnat över dem, fastän de icke hava velat höra.
Jer 36:32  Då tog Jeremia en annan rulle och gav den åt skrivaren Baruk, Nerias son; och efter Jeremias diktamen tecknade denne däri upp allt vad som hade stått i den bok som Jojakim, Juda konung, hade bränt upp i eld. Och till detta lades ytterligare mycket annat av samma slag.
Jer 37:1  Och Sidkia, Josias son, blev konung i stället för Konja, Jojakims son; ty Nebukadressar, konungen i Babel, gjorde honom till konung i Juda land.
Jer 37:2  Men varken han eller hans tjänare eller folket i landet hörda på HERRENS ord, dem som han talade genom profeten Jeremia.
Jer 37:3  Dock sände konung Sidkia åstad Jehukal, Selemjas son, och prästen Sefanja, Maasejas son, till profeten Jeremia och lät säga: "Bed för oss till HERREN, vår Gud."
Jer 37:4  Jeremia gick då ännu ut och in bland folket, ty man hade ännu icke satt honom i fängelse.
Jer 37:5  Och Faraos här hade då dragit ut från Egypten; och när kaldéerna, som belägrade Jerusalem, hade fått höra ryktet därom, hade de dragit sig tillbaka från Jerusalem.
Jer 37:6  Då kom HERRENS ord till profeten Jeremia; han sade:
Jer 37:7  Så säger HERREN, Israels Gud: Så skolen I svara Juda konung, som har sänt eder till mig för att fråga mig: "Se, Faraos här, som har dragit ut till eder hjälp, skall vända tillbaka till sitt land Egypten.
Jer 37:8  Sedan skola kaldéerna komma tillbaka och belägra denna stad och de skola då intaga den och bränna upp den i eld.
Jer 37:9  Därför säger HERREN så: Bedragen icke eder själva med att tänka: 'Kaldéerna skola nu en gång för alla draga bort ifrån oss'; ty de skola icke draga bort.
Jer 37:10  Nej, om I än så slogen kaldéernas hela här, när de strida mot eder, att allenast några svårt sårade män blevo kvar av dem, så skulle dessa resa sig upp, var och en i sitt tält, och skulle bränna upp denna stad i eld.
Jer 37:11  Men när kaldéernas här hade dragit sig tillbaka från Jerusalem för Faraos har,
Jer 37:12  ville Jeremia lämna Jerusalem och begiva sig till Benjamins land, för att där taga i besittning en jordlott bland folket.
Jer 37:13  När han då kom till Benjaminsporten, stod där såsom vakthavande en man vid namn Jiria, son till Selemja, son till Hananja; denne grep profeten Jeremia och sade: "Du vill gå över till kaldéerna."
Jer 37:14  Jeremia svarade: "Det är icke sant; jag vill icke gå över till kaldéerna", men ingen hörde på honom. Och Jiria grep Jeremia och förde honom till furstarna.
Jer 37:15  Och furstarna förtörnades på Jeremia och läto hudflänga honom och satte honom i häkte i sekreteraren Jonatans hus, ty detta hade de gjort till fängelse.
Jer 37:16  Men när Jeremia hade kommit i fängelsehålan, ned i fångvalven, och suttit där en lång tid,
Jer 37:17  sände konung Sidkia och lät hämta honom; och hemma hos sig frågade konungen honom hemligen och sade: "Har något ord kommit från HERREN Jeremia svarade: "Ja"; och han tillade: "Du skall bliva given i den babyloniske konungens hand.
Jer 37:18  Därefter frågade Jeremia konung Sidkia: "Varmed har jag försyndat mig mot dig och dina tjänare och detta folk, eftersom I haven satt mig i fängelse?
Jer 37:19  Och var äro nu edra profeter, som profeterade för eder och sade: 'Konungen i Babel skall icke komma över eder och över detta land'?
Jer 37:20  Så hör mig nu, herre konung; värdes upptaga min bön: sänd mig icke tillbaka till sekreteraren Jonatans hus, på det att jag icke må dö där."
Jer 37:21  På konung Sidkias befallning satte man då Jeremia i förvar i fängelsegården, och gav honom en kaka bröd om dagen från Bagargatan, till dess att det var slut på allt brödet i staden. Så stannade Jeremia i fängelsegården.
Jer 38:1  Men Sefatja, Mattans son, och Gedalja, Pashurs son, och Jukal, Selemjas son, och Pashur, Malkias son, hörde huru Jeremia talade till allt folket och sade:
Jer 38:2  "Så säger HERREN: Den som stannar kvar i denna stad, han skall dö genom svärd eller hunger eller pest, men den som giver sig åt kaldéerna, han skall få leva, ja, han skall vinna sitt liv såsom ett byte och få leva.
Jer 38:3  Ty så säger HERREN: Denna stad skall förvisso bliva given i händerna på den babyloniske konungens här, och han skall intaga den.
Jer 38:4  Då sade furstarna till konungen: "Denne man bör dödas, eftersom han gör folket modlöst, både det krigsfolk som ännu är kvar här i staden och jämväl allt det övrig: folket, i det att han talar sådan: ord till dem. Ty denne man söker icke folkets välfärd, utan dess olycka.
Jer 38:5  Konung Sidkia svarade: "Välan han är i eder hand; ty konungen förmår intet mot eder."
Jer 38:6  Då togo de Jeremia och kastad honom i konungasonen Malkias brunn på fängelsegården; de släppte Jeremia ditned med tåg. I brunnen var intet vatten, men dy, och Jeremia sjönk ned i dyn..
Jer 38:7  När nu etiopiern Ebed-Melek, en hovman, som befann sig i konungshuset, under det att konungen uppehöll sig i Benjaminsporten, fick höra att de hade sänkt Jeremia ned i brunnen,
Jer 38:8  begav han sig åstad från konungshuset och talade till konungen och sade:
Jer 38:9  "Min herre konung, dessa män hava handlat illa i allt vad de hava gjort mot profeten Jeremia; ty de hava kastat honom i brunnen, där han strax måste dö av hunger, då nu intet bröd finnes i staden.
Jer 38:10  Då bjöd konungen etiopiern Ebed-Melek och sade: "Tag med dig härifrån trettio män, och drag profeten Jeremia upp ur brunnen, innan han dör."
Jer 38:11  Så tog då Ebed-Melek männen med sig och begav sig till konungshuset, till rummet under skattkammaren, och hämtade därifrån trasor av sönderrivna och utslitna kläder och lät sänka ned dem med tåg till Jeremia i brunnen.
Jer 38:12  Och etiopiern Ebed-Melek sade till Jeremia: "Lägg trasorna av de sönderrivna och utslitna kläderna under dina armar, mellan dem och tågen." Och Jeremia gjorde så.
Jer 38:13  Sedan drogo de med tågen Jeremia upp ur brunnen. Men Jeremia måste stanna i fängelsegården.
Jer 38:14  Därefter sände konung Sidkia åstad och lät hämta profeten Jeremia till sig vid tredje ingången till HERRENS hus. Och konungen sade till Jeremia: "Jag vill fråga dig något dölj intet för mig."
Jer 38:15  Jeremia sade till Sidkia: "Om jag säger dig något, så kommer du förvisso att låta döda mig; och om jag giver dig ett råd, så hör du icke på mig."
Jer 38:16  Då gav konung Sidkia Jeremia sin ed, hemligen, och sade: "Så sant HERREN lever, han som har givit oss detta vårt liv: jag skall icke låta döda dig, ej heller skall jag lämna dig i händerna på dessa män som stå efter ditt liv."
Jer 38:17  Då sade Jeremia till Sidkia: "Så säger HERREN, härskarornas Gud, Israels Gud: Om du giver dig åt den babyloniske konungens furstar, så skall du få leva, och denna stad skall då icke bliva uppbränd i eld, utan du och ditt hus skolen få leva.
Jer 38:18  Men om du icke giver dig åt den babyloniske konungens furstar, då skall denna stad bliva given i kaldéernas hand, och de skola bränna upp den i eld, och du själv skall icke undkomma deras hand."
Jer 38:19  Konung Sidkia svarade Jeremia: "Jag rädes för de judar som hava gått över till kaldéerna; kanhända skall man lämna mig i deras händer, och de skola då hantera mig skändligt."
Jer 38:20  Jeremia sade: "Man skall icke göra det. Hör blott HERRENS röst i vad jag säger dig, så skall det gå dig väl, och du skall få leva.
Jer 38:21  Men om du vägrar att giva dig, så är detta vad HERREN har uppenbarat för mig:
Jer 38:22  Se, alla de kvinnor som äro kvar i Juda konungs hus skola då föras ut till den babyloniske konungens furstar; och kvinnorna skola klaga: 'Dina vänner sökte förleda dig, och de fingo makt med dig. Dina fötter fastnade i dyn; då drogo de sig undan'
Jer 38:23  Och alla dina hustrur och dina barn skall man föra ut till kaldéerna, och du själv skall icke undkomma deras hand, utan skall varda gripen av den babyloniske konungens hand och bliva en orsak till att denna stad brännes upp i eld."
Jer 38:24  Då sade Sidkia till Jeremia: "Låt ingen få veta vad här har blivit talat; eljest måste du dö.
Jer 38:25  Och om furstarna få höra att jag har talat med dig, och de komma till dig och säga till dig: 'Låt oss veta vad du har sagt till konungen; dölj intet för oss, så skola vi icke döda dig; säg oss ock vad konungen har sagt till dig' -
Jer 38:26  då skall du svara dem: 'Jag bönföll inför konungen att han icke skulle sända mig tillbaka till Jonatans hus för att dö där.'"
Jer 38:27  Och alla furstarna kommo till Jeremia och frågade honom; men han svarade dem alldeles såsom konungen hade bjudit honom. Då tego de och gingo bort ifrån honom, eftersom ingen hade hört huru det verkligen hade gått till.
Jer 38:28  Men Jeremia fick stanna i fängelsegården ända till den dag då Jerusalem blev intaget. Sedan nu Jerusalem hade blivit intaget
Jer 39:1  - efter det att Nebukadressar, konungen i Babel, med hela sin här hade kommit till Jerusalem och begynt belägra det i Sidkia, Juda konungs, nionde regeringsår, i tionde månaden,
Jer 39:2  och efter det att staden; hade blivit stormad i Sidkias elfte regeringsår, i fjärde månaden, på nionde dagen månaden
Jer 39:3  drogo alla den babyloniske konungens furstar därin och stannade i Mellersta porten: nämligen Nergal Sareser, Samgar-Nebo, Sarsekim, överste hovmannen, Nergal-Sareser, överste magern, och alla den babyloniske konungens övriga furstar.
Jer 39:4  Och när Sidkia, Juda konung, med allt sitt krigsfolk fick se dem, flydde de och drogo om natten ut ur staden, på den väg som ledde till den kungliga trädgården, genom porten mellan de båda murarna; och han tog vägen bort åt Hedmarken till.
Jer 39:5  Men kaldéernas här förföljde dem och de hunno upp Sidkia på Jeriko hedmarker. Och de togo fatt honom och förde honom till Nebukadressar, den babyloniske konungen, i Ribla i Hamats land; där höll denne rannsakning och dom med honom.
Jer 39:6  Och den babyloniske konungen lät i Ribla slakta Sidkias barn inför hans ögon; också alla andra Juda ädlingar lät konungen i Babel slakta.
Jer 39:7  Och på Sidkia själv lät han sticka ut ögonen och lät fängsla honom med kopparfjättrar, för att föra honom till Babel.
Jer 39:8  Och kaldéerna brände upp i eld både konungens hus och folkets hus och bröto ned Jerusalems murar.
Jer 39:9  Och återstoden av folket, dem som voro kvar i staden, och de över löpare som hade gått över till honom, och vad som för övrigt var kvar av folket, dem förde Nebusaradan, översten för drabanterna, bort till Babel.
Jer 39:10  Men av de ringaste bland folket, av dem som ingenting hade, lämnade Nebusaradan, översten för drabanterna, några kvar i Juda land, och gav dem samtidigt vingårdar och åkerfält.
Jer 39:11  Och Nebukadressar, konungen Babel, gav genom Nebusaradan, översten för drabanterna, befallning angående Jeremia och sade:
Jer 39:12  "Tag honom och se i honom till godo, och gör honom icke något ont, utan gör med honom efter som han själv begär av dig."
Jer 39:13  Då sände Nebusaradan, översten för drabanterna, och Nebusasban, överste hovmannen, och Nergal-Sareser, överste magern, och alla den babyloniske konungens övriga väldige -
Jer 39:14  dessa sände bort och läto hämta Jeremia ifrån fängelsegården och lämnade honom åt Gedalja, son till Ahikam, son till Safan, på det att denne skulle föra honom hem; så fick han stanna där bland folket.
Jer 39:15  Men HERRENS ord hade kommit till Jeremia, medan han var inspärrad i fängelsegården; han hade sagt:
Jer 39:16  Gå och säg till etiopiern Ebed-Melek: Så säger HERREN Sebaot, Israels Gud: Se, vad jag har förkunnat, det skall jag låta komma över denna stad, till dess olycka och icke till dess lycka, och det skall uppfyllas i din åsyn på den dagen.
Jer 39:17  Men dig skall jag rädda på den dagen, säger HERREN, och du skall icke bliva given i de mäns hand, som du fruktar för.
Jer 39:18  Ty jag skall förvisso låta dig komma undan, och du skall icke falla för svärd, utan vinna ditt liv såsom ett byte, därför att du har förtröstat på mig, säger HERREN.
Jer 40:1  Detta är det ord som kom till Jeremia från HERREN, sedan Nebusaradan, översten för drabanterna, hade släppt honom lös från Rama; denne lät nämligen hämta honom, där han låg bunden med kedjor bland alla andra fångar ifrån Jerusalem och Juda, som skulle föras bort till Babel.
Jer 40:2  Översten för drabanterna lät alltså hämta Jeremia och sade till honom: "HERREN, din Gud, hade förkunnat denna olycka över denna plats;
Jer 40:3  och HERREN har låtit den komma och har gjort såsom han hade sagt. I haden ju syndat mot HERREN och icke hört hans röst, och därför har detta vederfarits eder.
Jer 40:4  Och se, nu löser jag dig i dag ur kedjorna som dina händer hava varit bundna med. Om du är sinnad att komma med mig till Babel, så kom, och jag skall då se dig till godo; men om du icke är sinnad att komma med mig till Babel, så gör det icke. Se, hela landet ligger öppet för dig; dit dig synes gott och rätt att gå, dit må du gå."
Jer 40:5  Och då han ännu dröjde att vända tillbaka, tillade han: "Vänd tillbaka till Gedalja, son till Ahikam, son till Safan, som konungen i Babel har satt över Juda städer, och stanna hos honom bland folket. Eller gå åt vilket annat håll som helst dit det behagar dig att gå." Och översten för drabanterna gav honom vägkost och skänker och lät honom gå.
Jer 40:6  Så begav sig då Jeremia till Gedalja, Ahikams son, i Mispa och stannade hos honom bland folket som var kvar i landet.
Jer 40:7  När då alla krigshövitsmännen på landsbygden jämte sina män fingo höra att konungen i Babel hade satt Gedalja, Ahikams son, över landet, och att han hade anförtrott åt honom män kvinnor och barn, och dem av de ringaste i landet, som man icke hade fört bort till Babel,
Jer 40:8  kommo de till Gedalja i Mispa, nämligen Ismael, Netanjas son, Johanan och Jonatan, Kareas söner, Seraja, Tanhumets son, netofatiten Ofais söner och Jesanja, maakatitens son, med sina män.
Jer 40:9  Och Gedalja, son till Ahikam, son till Safan, gav dem och deras män sin ed och sade: "Frukten icke för att tjäna kaldéerna. Stannen kvar i landet, och tjänen konungen i Babel, så skall det gå eder väl.
Jer 40:10  Se, själv stannar jag kvar i Mispa, för att vara till tjänst åt kaldéer som komma till oss; men I mån insamla vin och frukt och olja och lägga det i edra kärl, och stanna kvar i de städer som I haven tagit i besittning."
Jer 40:11  Då nu också alla de judar som voro i Moabs och Ammons barns och Edoms land, och de som voro i andra länder hörde att konungen i Babel hade låtit några av judarna bliva kvar, och att han hade satt över dem Gedalja, son till Ahikam, son till Safan,
Jer 40:12  vände alla dessa judar tillbaka från alla de orter dit de hade blivit fördrivna, och kommo till Juda land, till Gedalja i Mispa. Och de inbärgade vin och frukt i stor myckenhet.
Jer 40:13  Men Johanan, Kareas son, och alla krigshövitsmännen på landsbygden kommo till Gedalja i Mispa
Jer 40:14  och sade till honom: "Du vet väl att Baalis, Ammons barns konung, har sänt hit Ismael, Netanjas son, för att slå ihjäl dig?" Men Gedalja, Ahikams son, trodde dem icke.
Jer 40:15  Och Johanan, Kareas son, sade i hemlighet till Gedalja i Mispa: "Låt mig gå åstad och dräpa Ismael, Netanjas son; ingen skall få veta det. Varför skulle han få slå ihjäl dig och så bliva en orsak till att vi judar, som hava församlats till dig, allasammans förskingras, och vad som är kvar av Juda förgås?"
Jer 40:16  Men Gedalja, Ahikams son, sade till Johanan, Kareas son: "Du får icke göra detta; ty vad du säger om Ismael är icke sant."
Jer 41:1  Men i sjunde månaden kom Ismael, son till Netanja, son till Elisama, av konungslig börd och en av konungens väldige, med tio män till Gedalja, Ahikams son, i Mispa, och de höllo måltid tillsammans i Mispa.
Jer 41:2  Och Ismael, Netanjas son, jämte de tio män som voro med honom, överföll då Gedalja, son till Ahikam, son till Safan, och slog honom till döds med svärd, honom som konungen i Babel hade satt över landet.
Jer 41:3  Därjämte dräpte Ismael alla de judar som voro hos Gedalja i Mispa, så ock alla de kaldéer som funnos där, och som tillhörde krigsfolket.
Jer 41:4  Dagen efter den då han hade dödat Gedalja, och innan ännu någon visste av detta,
Jer 41:5  kom en skara av åttio män från Sikem, Silo och Samaria; de hade rakat av sig skägget och rivit sönder sina kläder och ristat märken på sig, och hade med sig spisoffer och rökelse till att frambära i HERRENS hus.
Jer 41:6  Och Ismael, Netanjas son, gick ut emot dem från Mispa, gråtande utan uppehåll. Och när han mötte dem, sade han till dem: "Kommen in till Gedalja, Ahikams son."
Jer 41:7  Men när de hade kommit in i staden, blevo de nedstuckna av Ismael, Netanjas son, och de män som voro med honom, och kastade i brunnen.
Jer 41:8  Men bland dem funnos tio män som sade till Ismael: "Döda oss icke; ty vi hava förråd av vete, korn, olja och honung gömda på landsbygden." Då lät han dem vara och dödade dem icke med de andra.
Jer 41:9  Och brunnen i vilken Ismael kastade kropparna av alla de män som han hade dräpt, när han dräpte Gedalja, var densamma som konung Asa hade låtit göra, när Baesa, Israels konung, anföll honom; denna fylldes nu av Ismael, Netanjas son, med ihjälslagna män.
Jer 41:10  Därefter bortförde Ismael såsom fångar allt det folk som var kvar i Mispa, konungadöttrarna och allt annat folk som hade lämnats kvar i Mispa, och som Nebusaradan, översten för drabanterna, hade anförtrott åt Gedalja, Ahikams son; dem bortförde Ismael, Netanjas son; såsom fångar och drog åstad bort till Ammons barn.
Jer 41:11  Men när Johanan, Kareas son, och alla de krigshövitsmän som voro med honom fingo höra om allt det onda som Ismael, Netanjas son, hade gjort,
Jer 41:12  togo de alla sina män och gingo åstad för att strida mot Ismael, Netanjas son; och de träffade på honom vid det stora vattnet i Gibeon.
Jer 41:13  Då nu hela skaran av dem som Ismael förde med sig fick se Johanan, Kareas son, och alla de krigshövitsmän som voro med honom, blevo de glada;
Jer 41:14  och de vände om, hela skaran av dem som Ismael hade bortfört såsom fångar ifrån Mispa, och gåvo sig åstad tillbaka till Johanan, Kareas son.
Jer 41:15  Men Ismael, Netanjas son, räddade sig med åtta män undan Johanan och begav sig till Ammons barn.
Jer 41:16  Och Johanan, Kareas son, och alla de krigshövitsmän som voro med honom togo med sig allt som var kvar av folket, dem av Mispas invånare, som han hade vunnit tillbaka från Ismael, Netanjas son, sedan denne hade dräpt Gedalja, Ahikams son: både krigsmän och kvinnor och barn och hovmän, som han hade hämtat tillbaka från Gibeon.
Jer 41:17  Och de drogo åstad; men i Kimhams härbärge invid Bet-Lehem stannade de, för att sedan draga vidare och komma till Egypten,
Jer 41:18  undan kaldéerna; ty de fruktade för dessa, eftersom Ismael, Netanjas son, hade dräpt Gedalja, Ahikams son, vilken konungen i Babel hade satt över landet.
Jer 42:1  Då trädde alla krigshövitsmännen fram, jämte Johanan, Kareas son, och Jesanja, Hosajas son, så ock allt folket, både små och stora,
Jer 42:2  och sade till profeten Jeremia: "Värdes upptaga vår bön: bed för oss till HERREN, din Gud, för hela denna kvarleva - ty vi äro blott några få, som hava blivit kvar av många; du ser med egna ögon att det är så med oss.
Jer 42:3  Må så HERREN, din Gud, kungöra för oss vilken väg vi böra gå, och vad vi hava att göra."
Jer 42:4  Profeten Jeremia svarade dem: "Jag vill lyssna till eder. Ja, jag vill bedja till HERREN, eder Gud, såsom I haven begärt. Och vadhelst HERREN svarar eder skall jag förkunna för eder; intet skall jag undanhålla för eder."
Jer 42:5  Då sade de till Jeremia: "HERREN vare ett sannfärdigt och osvikligt vittne mot oss, om vi icke i alla stycken göra efter det ord varmed HERREN, din Gud, sänder dig till oss.
Jer 42:6  Det må vara gott eller ont, så vilja vi höra HERRENS, vår Guds, röst, hans som vi sända dig till; på det att det må gå oss väl, när vi höra HERRENS, vår Guds, röst."
Jer 42:7  Och tio dagar därefter kom HERRENS ord till Jeremia.
Jer 42:8  Då kallade han till sig Johanan, Kareas son, och alla de krigshövitsmän som voro med honom, och allt folket, både små och stora,
Jer 42:9  och sade till dem: Så säger HERREN, Israels Gud, han som I haven sänt mig till, för att jag skulle hos honom bönfalla för eder:
Jer 42:10  Om I stannen kvar i detta land, så skall jag uppbygga eder och ej mer slå eder ned; jag skall plantera eder och ej mer upprycka eder. Ty jag ångrar det onda som jag har gjort eder.
Jer 42:11  Frukten icke mer för konungen i Babel, som I nu frukten för, frukten icke för honom, säger HERREN. Ty jag är med eder och vill frälsa eder och rädda eder ur hans hand.
Jer 42:12  Jag vill låta eder finna barmhärtighet; ja, han skall bliva barmhärtig mot eder och låta eder vända tillbaka till edert land.
Jer 42:13  Men om I sägen: "Vi vilja icke stanna i detta land", om I alltså icke hören HERRENS, eder Guds, röst,
Jer 42:14  utan tänken: "Nej, vi vilja begiva oss till Egyptens land, där vi slippa att se krig och höra basunljud och hungra efter bröd, där vilja vi bo" -
Jer 42:15  välan, hören då HERRENS ord, I kvarblivna av Juda: Så säger HERREN Sebaot, Israels Gud: Om I verkligen ställen eder färd till Egypten och kommen dit, för att bo där såsom främlingar,
Jer 42:16  så skall svärdet, som I frukten för, hinna upp eder där i Egyptens land, och hungersnöden, som I rädens för, skall följa efter eder dit till Egypten, och där skolen I dö.
Jer 42:17  Ja, de människor som ställa sin färd till Egypten, för att bo där, skola alla dö genom svärd, hunger och pest, och ingen av dem skall slippa undan och kunna rädda sig från den olycka som jag skall låta komma över dem.
Jer 42:18  Ty så säger HERREN Sebaot, Israels Gud: Likasom min vrede och förtörnelse har utgjutit sig över Jerusalems invånare, så skall ock min förtörnelse utgjuta sig över eder, om I begiven eder till Egypten, och I skolen bliva ett exempel som man nämner, när man förbannar, och ett föremål för häpnad, bannande och smälek, och I skolen aldrig mer få se denna ort.
Jer 42:19  Ja, HERREN säger till eder, I kvarblivna av Juda: Begiven eder icke till Egypten. Märken väl att jag i dag har varnat eder.
Jer 42:20  Ty I bedrogen eder själva, när I sänden mig till HERREN, eder Gud, och saden: "Bed för oss till HERREN, vår Gud; och vadhelst HERREN, vår Gud, säger, det må du förkunna för oss, så vilja vi göra det."
Jer 42:21  Jag har nu i dag förkunnat det för eder. Men I haven icke velat höra HERRENS, eder Guds, röst, i allt det varmed han har sänt mig till eder.
Jer 42:22  Så veten nu att I skolen dö genom svärd, hunger och pest, på den ort dit I åstunden att komma, för att bo där såsom främlingar.
Jer 43:1  Men när Jeremia hade talat till allt folket alla HERRENS, deras Guds, ord, med vilka HERREN, deras Gud, hade sänt honom till dem, allt som sagt är,
Jer 43:2  då svarade Asarja, Hosajas son, och Johanan, Kareas son, och alla de övriga fräcka männen - dessa svarade Jeremia: "Det är icke sant vad du säger; HERREN, vår Gud, har icke sänt dig och låtit säga: 'I skolen icke begiva eder till Egypten, för att bo där såsom främlingar.'
Jer 43:3  Nej, det är Baruk, Nerias son, som uppeggar dig mot oss, på det att vi må bliva givna i kaldéernas hand, för att dessa skola döda oss eller föra oss bort till Babel."
Jer 43:4  Och varken Johanan, Kareas son, eller någon av krigshövitsmännen eller någon av folket ville höra HERRENS röst och stanna kvar i Juda land.
Jer 43:5  I stället togo Johanan, Kareas son, och alla krigshövitsmännen med sig alla de kvarblivna av Juda, dem som från alla de folk till vilka de hade varit fördrivna hade kommit tillbaka, för att bo i Juda land,
Jer 43:6  både män, kvinnor och barn, där till konungadöttrarna och alla andra som Nebusaradan, översten för drabanterna, hade lämnat kvar hos Gedalja, son till Ahikam, son till Safan, jämväl profeten Jeremia och Baruk, Nerias son,
Jer 43:7  och begåvo sig till Egyptens land, ty de ville icke höra HERRENS röst. Och de kommo så fram till Tapanhes.
Jer 43:8  Och HERRENS ord kom till Jeremia i Tapanhes; han sade:
Jer 43:9  Tag dig några stora stenar och mura in dem i murbruket, där tegelgolvet lägges, vid ingången till Faraos hus i Tapanhes; gör detta inför judiska mäns ögon
Jer 43:10  och säg till dem: Så säger HERREN Sebaot, Israels Gud: Se, jag skall sända åstad och hämta min tjänare Nebukadressar, konungen i Babel, och hans tron skall jag sätta upp ovanpå de stenar som jag har låtit mura in har, och han skall på dem breda ut sin tronmatta.
Jer 43:11  Ty han skall komma och slå Egyptens land och giva i pestens våld den som hör pesten till, i fångenskapens våld den som hör fångenskapen till, i svärdets våld den som hör svärdet till.
Jer 43:12  Och jag skall tända eld på Egyptens gudahus, och han skall bränna upp dem och föra gudarna bort. Och han skall rensa Egyptens land från ohyra, likasom en herde rensar sin mantel; sedan skall han draga därifrån i god ro.
Jer 43:13  Och han skall slå sönder stoderna i Bet-Semes i Egyptens land, och Egyptens gudahus skall han bränna upp i eld.
Jer 44:1  Detta är det ord som kom till Jeremia angående alla de judar som bodde i Egyptens land, dem som bodde i Migdol, Tapanhes, Nof och Patros' land; han sade:
Jer 44:2  "Så säger HERREN Sebaot, Israels Gud: I haven sett all den olycka som jag har låtit komma över Jerusalem och över alla Juda städer - Se, de äro nu ödelagda, och ingen bor i dem;
Jer 44:3  detta för den ondskas skull som de bedrevo till att förtörna mig, i det att de gingo bort och tände offereld och tjänade andra gudar, som varken I själva eller edra fäder haden känt.
Jer 44:4  Och titt och ofta sände jag till eder alla mina tjänare profeterna och lät säga: 'Bedriven icke denna styggelse, som jag hatar.'
Jer 44:5  Men de ville icke höra eller böja sitt öra därtill, så att de omvände sig från sin ondska och upphörde att tända offereld åt andra gudar.
Jer 44:6  Därför blev min förtörnelse och vrede utgjuten, och den brann i Juda städer och på Jerusalems gator, så att de blevo ödelagda och förödda, såsom de nu äro.
Jer 44:7  Och nu säger HERREN, härskarornas Gud, Israels Gud, så: Varför bereden I eder själva stor olycka? I utroten ju ur Juda både man och kvinna, både barn och spenabarn bland eder, så att ingen kvarleva av eder kommer att återstå;
Jer 44:8  I förtörnen ju mig genom edra händers verk, i det att I tänden offereld åt andra gudar i Egyptens land, dit I haven kommit, för att bo där såsom främlingar. Härav måste ske att I varden utrotade, och bliven ett exempel som man nämner, när man förbannar, och ett föremål för smälek bland alla jordens folk.
Jer 44:9  Haven I förgätit edra fäders onda gärningar och Juda konungars onda gärningar och deras hustrurs onda gärningar och edra egna onda gärningar och edra hustrurs onda gärningar, vad de gjorde i Juda land och på Jerusalems gator?
Jer 44:10  Ännu i dag äro de icke ödmjukade; de frukta intet och vandra icke efter min lag och mina stadgar, dem som jag förelade eder och edra fäder.
Jer 44:11  Därför säger HERREN Sebaot, Israels Gud, så: Se, jag skall vända mitt ansikte mot eder till eder olycka, till att utrota hela Juda.
Jer 44:12  Och jag skall gripa de kvarblivna av Juda, som hava ställt sin färd till Egyptens land, för att bo där såsom främlingar. Och de skola allasammans förgås, i Egyptens land skola de falla; genom svärd och hunger skola de förgås, både små och stora, ja, genom svärd och hunger skola de dö. Och de skola bliva ett exempel som man nämner, när man förbannar, och ett föremål för häpnad, bannande och smälek.
Jer 44:13  Och jag skall hemsöka dem som bo i Egyptens land, likasom jag hemsökte Jerusalem, med svärd, hunger och pest.
Jer 44:14  Och bland de kvarblivna av Juda, som hava kommit för att bo såsom främlingar där i Egyptens land, skall ingen kunna rädda sig och slippa undan, så att han kan vända tillbaka till Juda land, dit de dock åstunda att få vända tillbaka, för att bo där. Nej, de skola icke få vända tillbaka dit, förutom några få som bliva räddade."
Jer 44:15  Då svarade alla männen - vilka väl visste att deras hustrur tände offereld åt andra gudar - och alla kvinnorna, som stodo där i en stor hop, så ock allt folket som bodde i Egyptens land, i Patros, de svarade Jeremia och sade:
Jer 44:16  "I det som du har talat till oss i HERRENS namn vilja vi icke hörsamma dig,
Jer 44:17  utan vi vilja göra allt vad vår mun har lovat, nämligen tända offereld åt himmelens drottning och utgjuta drickoffer åt henne, såsom vi och våra fader, våra konungar och furstar gjorde i Juda städer och på Jerusalems gator. Då hade vi bröd nog, och det gick oss väl, och vi sågo icke till någon olycka.
Jer 44:18  Men från den stund då vi upphörde att tända offereld åt himmelens drottning och utgjuta drickoffer åt henne hava vi lidit brist på allt, och förgåtts genom svärd och hunger.
Jer 44:19  Och när vi nu tända offereld åt himmelens drottning och utgjuta drickoffer åt henne, är det då utan våra mäns samtycke som vi åt henne göra offerkakor, vilka äro avbilder av henne, och som vi utgjuta drickoffer åt henne?"
Jer 44:20  Men Jeremia sade till allt folket, till männen och kvinnorna och allt folket, som hade givit honom detta svar, han sade:
Jer 44:21  "Förvisso har HERREN kommit ihåg och tänkt på huru I haven tänt offereld i Juda städer och på. Jerusalems gator, både I själva och edra fäder, både edra konungar och furstar och folket i landet.
Jer 44:22  Och HERREN kunde icke längre hava fördrag med eder för edert onda väsendes skull, och för de styggelsers skull som I bedreven, utan edert land blev ödelagt och ett föremål för häpnad och förbannelse, så att ingen kunde bo där, såsom vi nu se.
Jer 44:23  Därför att I tänden offereld och syndaden mot HERREN och icke villen höra HERRENS röst eller vandra efter hans lag, efter hans stadgar och vittnesbörd, därför har denna olycka träffat eder, såsom vi nu se".
Jer 44:24  Och Jeremia sade ytterligare till allt folket och till alla kvinnorna: "Hören HERRENS ord, I alla av Juda, som ären i Egyptens land,
Jer 44:25  Så säger HERREN Sebaot, Israels Gud: I och edra hustrur haven med edra händer fullgjort vad I taladen med eder mun, när I saden: 'Förvisso vilja vi fullgöra de löften som vi gjorde, att tända offereld åt himmelens drottning och utgjuta drickoffer åt henne.' Välan, I mån hålla edra löften och fullgöra edra löften;
Jer 44:26  men hören då också HERRENS ord, I alla av Juda, som bon i Egyptens land: Se, jag svär vid mitt stora namn, säger HERREN, att i hela Egyptens land mitt namn icke mer skall varda nämnt av någon judisk mans mun, så att han säger: 'Så sant Herren, HERREN lever.'
Jer 44:27  Ty se, jag skall vaka över dem, till deras olycka, och icke till deras lycka, och alla män av Juda, som äro i Egyptens land, skola förgås genom svärd och hunger, till dess att de hava fått en ände.
Jer 44:28  Och allenast några som undkomma svärdet skola få vända tillbaka från Egyptens land till Juda land, en ringa hop. Och så skola alla kvarblivna av Juda, som hava kommit till Egyptens land, för att bo där såsom främlingar, få förnimma vilkens ord det är som bliver beståndande, mitt eller deras.
Jer 44:29  Och detta skall för eder vara tecknet till att jag skall hemsöka eder på denna ort, säger HERREN, och I skolen så förnimma att mina ord om eder förvisso skola bliva beståndande, eder till olycka:
Jer 44:30  Så säger HERREN: Se, jag skall giva Farao Hofra, konungen i Egypten i hans fienders hand och i de mäns hand, som stå efter hans liv, likasom jag har givit Sidkia, Juda konung, i Nebukadressars, den babyloniske konungens, hand, hans som var hans fiende, och som stod efter hans liv."
Jer 45:1  Detta är det ord som profeten Jeremia talade till Baruk, Nerias son, när denne efter Jeremias diktamen tecknade upp dessa tal i en bok, under Jojakims, Josias sons, Juda konungs, fjärde regeringsår; han sade:
Jer 45:2  Så säger HERREN, Israels Gud, om dig, Baruk:
Jer 45:3  Du säger: "Ve mig, ty HERREN har lagt ny sorg till min förra plåga! Jag är så trött av suckande och finner ingen ro."
Jer 45:4  Men så skall du svara honom: Så säger HERREN: Se, vad jag har byggt upp, det måste jag riva ned, och vad jag har planterat, det måste jag rycka upp; och detta gäller hela jorden.
Jer 45:5  Och du begär stora ting för dig! Begär icke något sådant; ty se, jag skall låta olycka komma över all kött, säger HERREN, men dig skall jag låta vinna ditt liv såsom ett byte, till vilken ort du än må gå.
Jer 46:1  Detta är vad som kom till profeten Jeremia såsom HERRENS ord om hednafolken.
Jer 46:2  Om Egypten, angående den egyptiske konungen Farao Nekos här, som stod invid floden Frat, vid Karkemis, och som blev slagen av Nebukadressar, konungen i Babel, i Jojakims, Josias sons, Juda konungs, fjärde regeringsår.
Jer 46:3  Reden till sköld och skärm, och rycken fram till strid.
Jer 46:4  Spännen för hästarna och bestigen springarna, och ställen upp eder, med hjälmarna på. Gören spjuten blanka, ikläden eder pansaren.
Jer 46:5  Men varav kommer detta som jag nu ser? De äro förfärade. De vika tillbaka; deras hjältar bliva slagna. De taga till flykten utan att vända sig om. Skräck från alla sidor! säger HERREN.
Jer 46:6  Ej ens den snabbaste kan fly undan, ej ens hjälten kan rädda sig. Norrut, invid floden Frat, där stappla de och falla.
Jer 46:7  Vem är denne som stiger upp såsom Nilfloden, denne vilkens vatten svalla såsom strömmar?
Jer 46:8  Det är Egypten som stiger upp såsom Nilfloden, och såsom strömmar svalla hans vatten. Han säger: "Jag vill stiga upp och övertäcka landet; jag vill fördärva städerna och dem som bo därinne."
Jer 46:9  Ja, dragen ditupp, I hästar; stormen fram, I vagnar. Må hjältarna tåga fram, etiopier och putéer, rustade med sköldar, och ludéer, rustade med bågar, bågar som de spänna.
Jer 46:10  Ty detta är Herrens; HERREN Sebaots, dag, en hämndedag, då han skall hämnas på sina motståndare; nu skall svärdet frossa sig mätt och dricka sig rusigt av deras blod. Ty ett slaktoffer vill Herren, HERREN Sebaot, anställa i nordlandet vid floden Frat.
Jer 46:11  Drag upp till Gilead och hämta balsam, du jungfru dotter Egypten. Men förgäves skaffar du dig läkemedel i mängd; du kan icke bliva helad.
Jer 46:12  Folken få höra om din skam, och av dina klagorop bliver jorden full; ty den ene hjälten stapplar på den andre, och de falla båda tillsammans.
Jer 46:13  Detta är det ord som HERREN talade till profeten Jeremia om att Nebukadressar, konungen i Babel, skulle komma och slå Egyptens land:
Jer 46:14  Förkunnen i Egypten och kungören i Migdol, ja, kungören i Nof, så ock i Tapanhes, och sägen: "Träd fram och gör dig redo, ty svärdet frossar runt omkring dig."
Jer 46:15  Varför äro dina väldige slagna till marken? De kunde ej hålla stånd, ty HERREN stötte dem bort.
Jer 46:16  Han kom många att stappla, och så föllo de, den ene över den andre; de ropade: "Upp, låt oss vända tillbaka till vårt folk och till vårt fädernesland, undan det härjande svärdet."
Jer 46:17  Ja, man ropar där: "Farao är förlorad, Egyptens konung! Han har förfelat sin tid."
Jer 46:18  Så sant jag lever, säger konungen, han vilkens namn är HERREN Sebaot, en skall komma, väldig såsom Tabor ibland bergen, såsom Karmel vid havet.
Jer 46:19  Så reden nu till åt eder, I dottern Egyptens inbyggare, vad man behöver, när man skall gå i landsflykt. Ty Nof skall bliva en ödemark och varda uppbränt, så att ingen kan bo där.
Jer 46:20  En skön kviga är Egypten; men en broms kommer farande norrifrån.
Jer 46:21  Också de legoknektar hon har i sitt land, lika gödda kalvar, ja, också de vända då om och fly allasammans, de kunna icke hålla stånd. Ty deras ofärds dag har kommit över dem, deras hemsökelses tid.
Jer 46:22  Tyst smyger hon undan såsom en krälande orm, ty med härsmakt draga de fram, och med yxor komma de över henne, såsom gällde det att hugga ved.
Jer 46:23  De fälla hennes skog, säger HERREN, ty ogenomtränglig är den; talrikare äro de än gräshoppor, ja, de kunna ej räknas.
Jer 46:24  På skam kommer dottern Egypten; hon bliver given i nordlandsfolkets hand.
Jer 46:25  Så säger HERREN Sebaot, Israels Gud: Se, jag skall hemsöka Amon från No, så ock Farao och Egypten med dess gudar och dess konungar, ja, både Farao och dem som förlita sig på honom.
Jer 46:26  Och jag skall giva dem i de mans hand, som stå efter deras liv, i Nebukadressars, den babyloniske konungens, och i hans tjänares hand. Men därefter skall landet bliva bebott såsom i forna dagar, säger HERREN.
Jer 46:27  Så frukta då icke, du min tjänare Jakob, och var ej förfärad, du Israel; ty se, jag skall frälsa dig ur det avlägsna landet, och dina barn ur deras fångenskaps land. Och Jakob skall få komma tillbaka och leva i ro och säkerhet, och ingen skall förskräcka honom.
Jer 46:28  Ja, frukta icke, du min tjänare Jakob, säger HERREN, ty jag är med dig. Och jag skall göra ände på alla de folk till vilka jag har drivit dig bort; men på dig vill jag ej alldeles göra ände, jag vill blott tukta dig med måtta; ty alldeles ostraffad kan jag ju ej låta dig bliva.
Jer 47:1  Detta är vad som kom till profeten Jeremia såsom HERRENS ord om filistéerna, förrän Farao hade intagit Gasa.
Jer 47:2  Så säger HERREN: Se, vatten stiga upp norrifrån och växa till en översvämmande ström; de översvämma landet och allt vad däri är, städerna med dem som bo därinne. Och människorna ropa, alla landets inbyggare jämra sig
Jer 47:3  När bullret höres av hans hingstars hovslag, när hans vagnar dåna, när hans hjuldon rassla, då se ej fäderna sig om efter barnen, så maktlösa stå de
Jer 47:4  inför den dag som kommer med fördärv över alla filistéer, med undergång för alla dem som äro kvar till att försvara Tyrus och Sidon. Ty HERREN skall fördärva filistéerna, kvarlevan från Kaftors ö.
Jer 47:5  Skallighet stundar för Gasa, det är förbi med Askelon, med kvarlevan i deras dalbygd. Huru länge skall du rista märken på dig?
Jer 47:6  Ack ve! Du HERRENS svärd, när skall du äntligen få ro, Drag dig tillbaka i din skida, vila dig och var stilla.
Jer 47:7  Dock, huru skulle det kunna få ro, då det är HERRENS bud det utför? Mot Askelon, mot Kustlandet vid havet, mot dem har han bestämt det.
Jer 48:1  Om Moab. Så säger HERREN Sebaot, Israels Gud: Ve över Nebo, ty det är förstört! Kirjataim har kommit på skam och är intaget, fästet har kommit på skam och ligger krossat.
Jer 48:2  Moabs berömmelse är icke mer. I Hesbon förehar man onda anslag mot det: "Upp, låt oss utrota det, så att det icke mer är ett folk." Också du, Madmen, skall förgöras, svärdet skall följa dig i spåren.
Jer 48:3  Klagorop höras från Horonaim, förödelse och stort brak.
Jer 48:4  Ja, Moab ligger förstört; högljutt klaga dess barn.
Jer 48:5  Uppför Halluhots höjd stiger man under gråt, och på vägen ned till Horonaim höras ångestfulla klagorop över förstörelsen.
Jer 48:6  Flyn, rädden edra liv, och bliven som torra buskar i öknen.
Jer 48:7  Ty därför att du förlitar dig på dina verk och dina skatter, skall ock du bliva intagen; och Kemos skall gå bort i fångenskap och hans präster och furstar med honom.
Jer 48:8  Och en förhärjare skall komma över var stad, så att ingen stad skall kunna rädda sig; dalen skall bliva förstörd och slätten ödelagd, såsom HERREN har sagt.
Jer 48:9  Given vingar åt Moab, ty flygande måste han fly bort. Hans städer skola bliva mark, och ingen skall bo i dem.
Jer 48:10  Förbannad vare den som försumligt utför HERRENS verk, förbannad vare den som dröjer att bloda sitt svärd.
Jer 48:11  I säkerhet har Moab levat från sin ungdom och har legat i ro på sin drägg; han har icke varit tömd ur ett kärl i ett annat, icke vandrat bort i fångenskap; därför har hans smak behållit sig, och hans lukt har ej förvandlats.
Jer 48:12  Se, därför skola dagar komma, säger HERREN, då jag skall sända till honom vintappare, som skola tappa honom och tömma hans kärl och krossa hans krukor.
Jer 48:13  Då skall Moab komma på skam med Kemos, likasom Israels hus kom på skam med Betel, som det förlitade sig på.
Jer 48:14  Huru kunnen I säga: "Vi äro hjältar och tappra män i striden"?
Jer 48:15  Moab skall ändå bliva förstört, dess städer skola gå upp i rök, och dess utvalda unga manskap måste ned till att slaktas; så säger konungen, han vilkens namn är HERREN Sebaot.
Jer 48:16  Snart kommer Moabs ofärd, och hans olycka hastar fram med fart.
Jer 48:17  Ömken honom, I alla som bon omkring honom, I alla som kännen hans namn. Sägen: "Huru sönderbruten är icke den starka spiran, den präktiga staven!"
Jer 48:18  Stig ned från din härlighet och sätt dig på torra marken, du dottern Dibons folk; ty Moabs förhärjare drager upp mot dig och förstör dina fästen.
Jer 48:19  Ställ dig vid vägen och spela omkring dig, du Aroers folk; fråga männen som fly och kvinnorna som söka rädda sig, säg: "Vad har hänt?"
Jer 48:20  Moab har kommit på skam, ja, det är krossat; jämren eder och ropen Förkunnen vid Arnon att Moab är förstört.
Jer 48:21  Domen har kommit över slättlandet, över Holon, Jahas och Mofaat,
Jer 48:22  över Dibon, Nebo och Bet-Diblataim,
Jer 48:23  över Kirjataim, Bet-Gamul och Bet-Meon,
Jer 48:24  över Keriot och Bosra och över alla andra städer i Moabs land, vare sig de ligga fjärran eller nära.
Jer 48:25  Avhugget är Moabs horn, och hans arm är sönderbruten, säger HERREN.
Jer 48:26  Gören honom drucken, ty han har förhävt sig mot HERREN; ja, må Moab ragla omkull i sina egna spyor och bliva till åtlöje, också han.
Jer 48:27  Eller var icke Israel till ett åtlöje för dig? Blev han då ertappad bland tjuvar, eftersom du skakar huvudet, så ofta du talar om honom?
Jer 48:28  Övergiven edra städer och byggen bo i klipporna, I Moabs inbyggare, och bliven lika duvor som bygga sina nästen bortom klyftans gap.
Jer 48:29  Vi hava hört om Moabs högmod, det övermåttan höga, om hans stolthet, högmod och högfärd och hans hjärtas förhävelse.
Jer 48:30  Jag känner, säger HERREN, hans övermod och opålitlighet, hans lösa tal och opålitliga handlingssätt.
Jer 48:31  Därför måste jag jämra mig för Moabs skull; över hela Moab måste jag klaga. Över Kir-Heres' män må man sucka.
Jer 48:32  Mer än Jaeser gråter, måste jag gråta över dig, du Sibmas vinträd, du vars rankor gingo över havet och nådde till Jaesers hav; mitt i din sommar och din vinbärgning har ju en förhärjare slagit ned.
Jer 48:33  Glädje och fröjd är nu avbärgad från de bördiga fälten och från Moabs land. På vinet i pressarna har jag gjort slut; man trampar ej mer vin under skördeskri, skördeskriet är intet skördeskri mer.
Jer 48:34  Från Hesbon, jämmerstaden, ända till Eleale, ända till Jahas upphäver man rop, och från Soar ända till Horonaim, till Eglat-Selisia; ty också Nimrims vatten bliva torr ökenmark.
Jer 48:35  Och jag skall i Moab så göra, säger HERREN, att ingen mer frambär offer på offerhöjden och ingen mer tänder offereld åt sin gud.
Jer 48:36  Därför klagar mitt hjärta såsom en flöjt över Moab, ja, mitt hjärta klagar såsom en flöjt över Kir-Heres' män: vad de hava kvar av sitt förvärv går ju förlorat.
Jer 48:37  Ty alla huvuden äro skalliga och alla skägg avskurna; på alla händer äro sårmärken och omkring länderna säcktyg.
Jer 48:38  På alla Moabs tak och på dess torg höres allenast dödsklagan, ty jag har krossat Moab såsom ett värdelöst kärl, säger HERREN.
Jer 48:39  Huru förfärad är han icke! I mån jämra eder. Huru vänder icke Moab ryggen till med blygd! Ja, Moab bliver ett åtlöje och en skräck för alla dem som bo däromkring.
Jer 48:40  Ty så säger HERREN: Se, en som liknar en örn svävar fram och breder ut sina vingar över Moab.
Jer 48:41  Keriot bliver intaget, bergfästena bliva erövrade. Och Moabs hjältars hjärtan bliva på den dagen såsom en kvinnas hjärta, när hon är barnsnöd.
Jer 48:42  Ja, Moab skall förgöras så att det icke mer är ett folk, ty det har förhävt sig mot HERREN.
Jer 48:43  Faror, fallgropar och fällor vänta eder, I Moabs inbyggare, säger HERREN.
Jer 48:44  Om någon flyr undan faran, så störtar han i fallgropen, och om han kommer upp ur fallgropen, så fångas han i fällan. Ty jag skall låta ett hemsökelsens år komma över dem, över Moab, säger HERREN.
Jer 48:45  I Hesbons skugga stanna de, det är ute med flyktingarnas kraft. Ty eld gick ut från Hesbon, en låga från Sihons land; och den förtärde Moabs tinning, hjässan på stridslarmets söner.
Jer 48:46  Ve dig, Moab! Förlorat är Kemos' folk. Ty dina söner äro tagna till fånga, och dina döttrar förda bort i fångenskap.
Jer 48:47  Men i kommande dagar skall jag åter upprätta Moab, säger HERREN. Så långt om domen över Moab.
Jer 49:1  Om Ammons barn. Så säger HERREN: Har Israel nu inga barn, eller har han ingen arvinge mer? Eller varför har Malkam tagit arv, efter Gad, och varför bor hans folk i dess städer?
Jer 49:2  Se, därför skola dagar komma, säger HERREN då jag skall låta höra ett härskri mot Rabba i Ammons barns land; och då skall det bliva en öde grushög, och dess lydstäder skola brännas upp i eld; och Israel skall då taga arv efter dem som hava tagit hans arv, säger HERREN.
Jer 49:3  Jämra dig, du Hesbon, ty Ai är förstört; ropen, I Rabbas döttrar. Höljen eder i sorgdräkt, klagen, och gån omkring i gårdarna; ty Malkam måste vandra bort i fångenskap, och hans präster och furstar med honom.
Jer 49:4  Varför berömmer du dig av dina dalar, av att din dal flödar över, du avfälliga dotter? Du som förlitar dig på dina skatter och säger: "Vem skall väl komma åt mig?",
Jer 49:5  se, jag skall låta förskräckelse komma över dig från alla dem som bo omkring dig, säger Herren, HERREN Sebaot. Och I skolen varda bortdrivna, var och en åt sitt håll och ingen skall församla de flyktande.
Jer 49:6  Men därefter skall jag åter upprätta Ammons barn, säger HERREN.
Jer 49:7  Om Edom. Så säger HERREN Sebaot: Finnes då ingen vishet mer i Teman? Har all rådighet försvunnit ifrån de förståndiga? Är deras vishet uttömd?
Jer 49:8  Flyn, vänden om, gömmen eder djupt nere, I Dedans inbyggare. Ty över Esau skall jag låta ofärd komma på hans hemsökelses tid.
Jer 49:9  När vinbärgare komma över dig, skola de icke lämna kvar någon efterskörd. När tjuvar komma om natten, skola de fördärva så mycket dem lyster.
Jer 49:10  Ty jag skall blotta Esau, jag skall uppenbara hans gömslen, och han skall icke lyckas hålla sig dold; fördärv skall drabba hans barn, hans bröder och grannar, och han skall icke mer vara till.
Jer 49:11  Bekymra dig ej om dina faderlösa, jag vill behålla dem vid liv; och må dina änkor förtrösta på mig.
Jer 49:12  Ty så säger HERREN: Se, de som icke hade förskyllt att dricka kalken, de nödgas att dricka den; skulle då du bliva ostraffad? Nej, du skall icke bliva ostraffad, utan skall nödgas att dricka den.
Jer 49:13  Ty vid mig själv har jag svurit, säger HERREN, att Bosra skall bliva ett föremål för häpnad och smälek; det skall förödas och bliva ett exempel som man nämner, när man förbannar; och alla dess lydstäder skola bliva ödemarker för evärdlig tid.
Jer 49:14  Ett budskap har jag hört från HERREN, och en budbärare är utsänd bland folken: "Församlen eder och kommen emot det, och stån upp till strid.
Jer 49:15  Ty se, jag skall göra dig ringa bland folken, föraktad bland människorna.
Jer 49:16  Den förfäran du väckte har bedragit dig, ja, ditt hjärtas övermod, där du sitter ibland bergsklyftorna och håller dig fast högst uppe på höjden. Om du än byggde ditt näste så högt uppe som örnen, så skulle jag dock störta dig ned därifrån, säger HERREN.
Jer 49:17  Och Edom skall bliva ett föremål för häpnad; alla som gå där fram skola häpna och vissla vid tanken på alla dess plågor.
Jer 49:18  Likasom när Sodom och Gomorra med sina grannstäder omstörtades, säger HERREN, så skall ingen mer bo där och intet människobarn där vistas.
Jer 49:19  Se, lik ett lejon som drager upp från Jordanbygdens snår och bryter in på frodiga betesmarker skall jag i ett ögonblick jaga dem bort därifrån; och den som jag utväljer skall jag sätta till herde över dem. Ty vem är min like, och vem kan ställa mig till ansvar? Och vilken är den herde som kan bestå inför mig?
Jer 49:20  Hören därför det råd som HERREN har lagt mot Edom, och de tankar som han har mot Temans inbyggare: Ja, herdegossarna skola sannerligen släpas bort; sannerligen, deras betesmark skall häpna över dem.
Jer 49:21  Vid dånet av deras fall bävar jorden; man skriar så, att ljudet höres ända borta vid Röda havet.
Jer 49:22  Se, en som liknar en örn lyfter sig och svävar fram och breder ut sina vingar över Bosra. Och Edoms hjältars hjärtan bliva på den dagen såsom en kvinnas hjärta, när hon är i barnsnöd.
Jer 49:23  Om Damaskus. Hamat och Arpad komma på skam; ty ett ont budskap få de höra, och de betagas av ångest. I havet råder oro; det kan ej vara stilla.
Jer 49:24  Damaskus förlorar modet, det vänder sig om till flykt, ty skräck har fattat det; ångest och vånda har gripit det, lik en barnaföderskas.
Jer 49:25  Varför lät man den icke vara, den berömda staden, min glädjes stad?
Jer 49:26  Så måste nu dess unga män falla på dess gator, och alla dess stridsmän förgöras på den dagen, säger HERREN Sebaot.
Jer 49:27  Och jag skall tända eld på Damaskus' murar, och elden skall förtära Ben-Hadads palatser.
Jer 49:28  Om Kedar och Hasors riken, som blevo slagna av Nebukadressar, konungen i Babel. Så säger HERREN: Upp, ja, dragen åstad upp mot Kedar, och fördärven Österlandets söner.
Jer 49:29  Deras hyddor och deras hjordar må man taga, deras tält och allt deras bohag och deras kameler må föras bort ifrån dem och man må ropa över dem; "Skräck från alla sidor!"
Jer 49:30  Flyn, ja, flykten med hast, gömmen eder djupt nere, I Hasors inbyggare, säger HERREN, ty Nebukadressar, konungen i Babel, har lagt råd mot eder och tänkt ut mot eder ett anslag.
Jer 49:31  Upp, säger HERREN, ja, dragen ditupp mot ett fredligt folk, som bor där i trygghet, utan både portar och bommar, i sin avskilda boning.
Jer 49:32  Deras kameler skola bliva edert byte, och deras myckna boskap skall bliva edert rov; och jag skall förströ dem åt alla väderstreck, männen med det kantklippta håret och från alla sidor skall jag låta ofärd komma över dem, säger HERREN.
Jer 49:33  Och Hasor skall bliva en boning för schakaler en ödemark till evärdlig tid; ingen skall mer bo där och intet människobarn där vistas.
Jer 49:34  Detta är vad som kom till profeten Jeremia såsom HERRENS ord om Elam, i begynnelsen av Sidkias, Juda konungs, regering; han sade:
Jer 49:35  Så säger HERREN Sebaot: Se, jag skall bryta sönder Elams båge, deras yppersta makt.
Jer 49:36  Och från himmelens fyra ändar skall jag låta fyra vindar komma mot Elam, och skall förströ dess folk åt alla dessa väderstreck; och intet folk skall finnas, dit icke de fördrivna ifrån Elam skola komma.
Jer 49:37  Och jag skall göra elamiterna förfärade för sina fiender och för dem som stå efter deras liv, och jag skall låta olycka komma över dem, min vredes glöd, säger HERREN. Jag skall sända svärdet efter dem, till dess att jag har gjort ände på dem.
Jer 49:38  Och jag skall sätta upp min tron i Elam och förgöra där både konung och furstar, säger HERREN.
Jer 49:39  Men i kommande dagar skall jag åter upprätta Elam, säger HERREN.
Jer 50:1  Detta är det ord som HERREN talade om Babel, om kaldéernas land, genom profeten Jeremia.
Jer 50:2  Förkunnen detta bland folken och kungören det, och resen upp ett baner; kungören det, döljen det icke. Sägen: Babel är intaget, Bel har kommit på skam, Merodak är krossad, ja, dess avgudar hava kommit på skam, dess eländiga avgudar äro krossade.
Jer 50:3  Ty ett folk drager upp mot det norrifrån, som skall göra dess land till en ödemark, så att ingen kan bo däri; både människor och djur skola fly bort.
Jer 50:4  I de dagarna och på den tiden, säger HERREN, skola Israels barn komma vandrande tillsammans med Juda barn; under gråt skola de gå åstad och söka HERREN, sin Gud.
Jer 50:5  De skola fråga efter Sion; hitåt skola deras ansikten vara vända: "Kommen! Må man nu hålla fast vid HERREN i ett evigt förbund, som aldrig varder förgätet."
Jer 50:6  En vilsekommen hjord var mitt folk. Deras herdar hade fört dem vilse och läto dem irra omkring på bergen. Så strövade de från berg till höjd och glömde sin rätta lägerplats.
Jer 50:7  Alla som träffade på dem åto upp dem, och deras ovänner sade: "Vi ådraga oss ingen skuld därmed." Så skedde, därför att de hade syndat mot HERREN, rättfärdighetens boning, mot HERREN, deras fäders hopp.
Jer 50:8  Flyn ut ur Babel, dragen bort ifrån kaldéernas land, och bliven lika bockar som hasta framför hjorden.
Jer 50:9  Ty se, jag skall uppväcka från nordlandet en hop av stora folk och föra dem upp mot Babel, och de skola rusta sig till strid mot det; från det hållet skall det bliva intaget. Deras pilar skola vara såsom en lyckosam hjältes, som icke vänder tillbaka utan seger.
Jer 50:10  Och Kaldeen skall lämnas till plundring; dess plundrare skola alla få nog, säger HERREN.
Jer 50:11  Ja, om I än glädjens och fröjden eder, I som skövlen min arvedel, om I än hoppen såsom kvigor på tröskplatsen och frusten såsom hingstar,
Jer 50:12  eder moder skall dock komma storligen på skam; hon som har fött eder skall få blygas. Se, bland folken skall hon bliva den yttersta - en öken, ett torrt land och en hedmark!
Jer 50:13  För HERRENS förtörnelses skull måste det ligga obebott och alltigenom vara en ödemark. Alla som gå fram vid Babel skola häpna och vissla vid tanken på alla dess plågor.
Jer 50:14  Rusten eder till strid mot Babel från alla sidor, I som spännen båge; skjuten på henne, sparen icke på pilarna; ty mot HERREN har hon syndat.
Jer 50:15  Höjen segerrop över henne på alla sidor: "Hon har måst giva sig; fallna äro hennes stödjepelare, nedrivna hennes murar!" Detta är ju HERRENS hämnd, så hämnens då på henne. Såsom hon har gjort, så mån I göra mot henne.
Jer 50:16  Utroten ur Babel både dem som så och dem som i skördens tid föra lien. Undan det härjande svärdet må envar nu vända om till sitt folk och envar fly hem till sitt land.
Jer 50:17  Israel var ett vilsekommet får som jagades av lejon. Först åts det upp av konungen i Assyrien, och sist har Nebukadressar, konungen i Babel, gnagt dess ben.
Jer 50:18  Därför säger HERREN Sebaot, Israels Gud, så: Se, jag skall hemsöka konungen i Babel och hans land, likasom jag har hemsökt konungen i Assyrien.
Jer 50:19  Och jag skall föra Israel tillbaka till hans betesmarker, och han skall få gå bet på Karmel och i Basan; och på Efraims berg och i Gilead skall han få äta sig mätt.
Jer 50:20  I de dagarna och på den tider säger HERREN, skall man söka efter Israels missgärning, och den skall icke mer vara till, och efter Juda synder, och de skola icke mer bliva funna; ty jag skall förlåta dem som jag låter leva kvar.
Jer 50:21  Drag ut mot Merataims land och mot inbyggarna i Pekod. Förfölj dem och döda dem och giv dem till spillo, säger HERREN, och gör i alla stycken såsom jag har befallt dig.
Jer 50:22  Krigsrop höras i landet, och stort brak.
Jer 50:23  Huru sönderbruten och krossad är den icke, den hammare som slog hela jorden! Huru har icke Babel blivit till häpnad bland folken!
Jer 50:24  Jag lade ut en snara för dig, och så blev du fångad, Babel, förrän du visste därav; du blev ertappad och gripen, ty det var med HERREN som du hade givit dig i strid.
Jer 50:25  HERREN öppnade sin rustkammare och tog fram sin vredes vapen. Ty ett verk hade Herren, HERREN Sebaot, att utföra i kaldéernas land.
Jer 50:26  Ja, kommen över det från alla sidor, öppnen dess förrådskammare, kasten i en hög vad där finnes, såsom man gör med säd, och given det till spillo; låten intet därav bliva kvar.
Jer 50:27  Nedgören alla dess tjurar, fören dem ned till att slaktas. Ve dem, ty deras dag har kommit, deras hemsökelses tid!
Jer 50:28  Hör huru de fly och söka rädda sig ur Babels land, för att i Sion förkunna HERRENS, vår Guds, hämnd, hämnden för hans tempel.
Jer 50:29  Båden upp mot Babel folk i mängd, allt vad bågskyttar heter; lägren eder runt omkring det, låten ingen undkomma. Vedergällen det efter dess gärningar; gören mot det alldeles såsom det självt har gjort. Ty mot HERREN har det handlat övermodigt, mot Israels Helige.
Jer 50:30  Därför skola dess unga man falla på dess gator, och alla dess stridsmän skola förgöras på den dagen, säger HERREN.
Jer 50:31  Se, jag skall vända mig mot dig, du övermodige, säger Herren, HERREN Sebaot, ty din dag har kommit, den tid då jag vill hemsöka dig.
Jer 50:32  Då skall den övermodige stappla och falla, och ingen skall kunna upprätta honom. Och jag skall tända eld på hans städer, och elden skall förtära allt omkring honom.
Jer 50:33  Så säger HERREN Sebaot: Förtryckta äro Israels barn, och Juda barn jämte dem. Alla de som hava fart dem i fångenskap hålla dem fast och vilja icke släppa dem.
Jer 50:34  Men deras förlossare är stark; HERREN Sebaot är hans namn. Han skall förvisso utföras deras sak, så att han skaffar ro åt jorden - men oro åt Babels invånare.
Jer 50:35  Svärd komme över kaldéerna, säger HERREN, över Babels invånare, över dess furstar och dess visa män!
Jer 50:36  Svärd komme över lögnprofeterna, så att de stå där såsom dårar! Svärd komme över dess hjältar, så att de bliva förfärade!
Jer 50:37  Svärd komme över dess hästar och vagnar och över allt främmande folk därinne, så att de bliva såsom kvinnor! Svärd komme över dess skatter, så att de bliva tagna såsom byte!
Jer 50:38  Torka komme över dess vatten, så att de bliva uttorkade! Ty det är ett belätenas land, och skräckgudar dyrka de såsom vanvettiga människor.
Jer 50:39  Därför skola nu schakaler bo där tillsammans med andra ökendjur, och strutsar skola där få sin boning. Aldrig mer skall det bliva bebyggt, från släkte till släkte skall det vara obebott.
Jer 50:40  Likasom när Sodom och Gomorra med sina grannstäder omstörtades av Gud, säger HERREN, så skall ingen mer bo där och intet människobarn där vistas.
Jer 50:41  Se, ett folk kommer norrifrån; ett stort folk och många konungar resa sig och komma från jordens yttersta ända.
Jer 50:42  De föra båge och lans, de äro grymma och utan förbarmande. Dånet av dem är såsom havets brus, och på sina hästar rida de fram, rustade såsom kämpar till strid, mot dig, du dotter Babel.
Jer 50:43  När konungen i Babel hör ryktet om dem, sjunka hans händer ned; ängslan griper honom, ångest lik en barnaföderskas.
Jer 50:44  Se, lik ett lejon som drager upp från Jordanbygdens snår och bryter in på frodiga betesmarker skall jag i ett ögonblick jaga dem bort därifrån; och den som jag utväljer skall jag sätta till herde över dem. Ty vem är min like, och vem kan ställa mig till ansvar? Och vilken är den herde som kan bestå inför mig?
Jer 50:45  Hören därför det råd som HERREN har lagt mot Babel, och de tankar som han har mot kaldéernas land: Ja, herdegossarna skola sannerligen släpas bort; sannerligen, deras betesmark skall häpna över dem.
Jer 50:46  När man ropar: "Babel är intaget", då bävar jorden, och ett skriande höres bland folken.
Jer 51:1  Så säger HERREN: Se, jag skall uppväcka mot Babel och mot Leb-Kamais inbyggare en fördärvares ande.
Jer 51:2  Och jag skall sända främlingar mot Babel, och de skola kasta det med kastskovlar och ödelägga dess land. Ja, från alla sidor skola de komma emot det på olyckans dag.
Jer 51:3  Skyttar skola spänna sina bågar mot dem som där spänna båge, och mot dem som där yvas i pansar. Skonen icke dess unga män, given hela dess här till spillo.
Jer 51:4  Dödsslagna män skola då falla i kaldéernas land och genomborrade man på dess gator.
Jer 51:5  Ty Israel och Juda äro icke änkor som hava blivit övergivna av sin Gud, av HERREN Sebaot, därför att deras land var fullt av skuld mot Israels Helige.
Jer 51:6  Flyn ut ur Babel; må var och en söka rädda sitt liv, så att I icke förgås genom dess missgärning. Ty detta är för HERREN en hämndens tid, då han vill vedergälla det vad det har gjort.
Jer 51:7  Babel var i HERRENS hand en gyllene kalk som gjorde hela jorden drucken. Av dess vin drucko folken, och så blevo folken såsom vanvettiga.
Jer 51:8  Men plötsligt är nu Babel fallet och krossat. Jämren eder över henne, hämten balsam för hennes plåga, om hon till äventyrs kan helas.
Jer 51:9  "Ja, vi hava sökt hela Babel, men hon har icke kunnat helas; låt oss lämna henne och gå var och en till sitt land. Ty hennes straffdom räcker upp till himmelen och når allt upp till skyarna.
Jer 51:10  HERREN har låtit vår rätt gå fram; kom, låt oss förtälja i Sion HERRENS, vår Guds, verk."
Jer 51:11  Vässen pilarna, fatten sköldarna. HERREN har uppväckt de mediska konungarnas ande; ty hans tankar äro vända mot Babel till att fördärva det. Ja, HERRENS hämnd är här, hämnden för hans tempel.
Jer 51:12  Resen upp ett baner mot Babels murar, hållen sträng vakt, ställen ut väktare, läggen bakhåll; ty HERREN har fattat sitt beslut, och han gör vad han har talat mot Babels invånare.
Jer 51:13  Du som bor vid stora vatten och är så rik på skatter, din ände har nu kommit, din vinningslystnads mått är fyllt.
Jer 51:14  HERREN Sebaot har svurit vid sig själv: sannerligen, om jag än har uppfyllt dig med människor så talrika som gräshoppor, så skall man dock få upphäva skördeskri över dig.
Jer 51:15  Han har gjort jorden genom sin kraft, han har berett jordens krets genom sin vishet, och genom sitt förstånd har han utspänt himmelen.
Jer 51:16  När han vill låta höra sin röst, då brusa himmelens vatten, då låter han regnskyar stiga upp från jordens ända; han låter ljungeldar komma med regn och för vinden ut ur dess förvaringsrum.
Jer 51:17  Såsom dårar stå då alla människor där och begripa intet; guldsmederna komma då alla på skam med sina beläten, ty deras gjutna beläten äro lögn, och ingen ande är i dem.
Jer 51:18  De äro fåfänglighet, en tillverkning att le åt; när hemsökelsen kommer över dem, måste de förgås.
Jer 51:19  Men sådan är icke han som är Jakobs del; nej, det är han som har skapat allt, och särskilt sin arvedels stam. HERREN Sebaot är hans namn.
Jer 51:20  Du var min hammare, mitt stridsvapen; med dig krossade jag folk, med dig fördärvade jag riken.
Jer 51:21  Med dig krossade jag häst och ryttare; med dig krossade jag vagn och körsven.
Jer 51:22  Med dig krossade jag man och kvinna; med dig krossade jag gammal och ung; med dig krossade jag yngling och jungfru.
Jer 51:23  Med dig krossade jag herden och hans hjord; med dig krossade jag åkermannen och hans oxpar; med dig krossade jag ståthållare och landshövding.
Jer 51:24  Men nu skall jag vedergälla Babel och alla Kaldeens inbyggare allt det onda som de hava förövat mot Sion, inför edra ögon, säger HERREN.
Jer 51:25  Se, jag skall vända mig mot dig, du fördärvets berg, säger HERREN, du som fördärvade hela jorden; och jag skall uträcka min hand mot dig och vältra dig ned från klipporna och göra dig till ett förbränt berg,
Jer 51:26  så att man icke av dig skall kunna taga vare sig hörnsten eller grundsten, utan du skall bliva en ödemark för evärdlig tid, säger HERREN.
Jer 51:27  Resen upp ett baner på jorden, stöten i basun ibland folken, invigen folk till strid mot det, båden upp mot det riken, både Ararats, Minnis och Askenas', tillsätten hövdingar mot det, dragen ditupp med hästar som likna borstiga gräshoppor.
Jer 51:28  Invigen folk till strid mot det: Mediens konungar, dess ståthållare och alla dess landshövdingar, och hela det land som lyder under deras välde.
Jer 51:29  Då darrar jorden och bävar, ty nu fullbordas vad HERREN tänkte mot Babel: att han ville göra Babels land till en ödemark, där ingen skulle bo.
Jer 51:30  Babels hjältar upphöra att strida, de sitta stilla i sina fästen; deras styrka har försvunnit, de hava blivit såsom kvinnor. Man har tänt eld på dess boningar; dess bommar äro sönderbrutna.
Jer 51:31  Löparna löpa mot varandra, den ene budbäraren korsar den andres väg, med bud till konungen i Babel om att hela hans stad år intagen,
Jer 51:32  att vadställena äro besatta och dammarna förbrända i eld och krigsmännen gripna av skräck.
Jer 51:33  Ty så säger HERREN Sebaot, Israels Gud: Dottern Babel är såsom en tröskplats, när man just har trampat till den; ännu en liten tid, och skördetiden kommer för henne.
Jer 51:34  Uppätit mig och förgjort mig har han, Nebukadressar, konungen i Babel. Han har gjort mig till ett tomt kärl; lik en drake har han uppslukat mig, han har fyllt sin buk med mina läckerheter och drivit mig bort.
Jer 51:35  "Den orätt mig har skett och det som har vederfarits mitt kött, det komme över Babel", så må Sions invånare säga; och "Mitt blod komme över Kaldeens inbyggare", så må Jerusalem säga.
Jer 51:36  Därför säger HERREN så: Se, jag skall utföra din sak och utkräva din hämnd. Jag skall låta dess hav sina bort och dess brunn uttorka,
Jer 51:37  och Babel skall bliva en stenhop, en boning för schakaler, ett föremål för häpnad och begabberi, så att ingen kan bo där.
Jer 51:38  Alla ryta de nu såsom lejon; de skria såsom lejonungar.
Jer 51:39  Men när de äro som mest upptända, skall jag tillreda åt dem ett gästabud; jag skall göra dem druckna, så att de jubla. Så skola de somna in i en evig sömn, ur vilken de aldrig skola uppvakna, säger HERREN.
Jer 51:40  Jag skall föra dem ned till att slaktas såsom lamm, likasom vädurar och bockar.
Jer 51:41  Huru har icke Sesak blivit intaget och hon som var hela jordens berömmelse erövrad! Huru har icke Babel blivit ett föremål för häpnad bland folken!
Jer 51:42  Havet steg upp över Babel; av dess brusande böljor blev det övertäckt.
Jer 51:43  Så blev av dess städer en ödemark, ett torrt land och en hedmark, ett land där ingen bor, och där intet människobarn går fram.
Jer 51:44  Ja, jag skall hemsöka Bel i Babel och taga ut ur hans gap vad han har slukat; och folken skola icke mer strömma till honom. Babels murar skola ock falla.
Jer 51:45  Dragen ut därifrån, mitt folk; må var och en söka rädda sitt liv undan HERRENS vredes glöd.
Jer 51:46  Varen icke försagda i edra hjärtan, och frukten icke för de olycksbud som höras i landet, om än ett olycksbud kommer det ena året och sedan nästa år ett nytt olycksbud, och om än våld råder på jorden och härskare står mot härskare.
Jer 51:47  Se, därför skola dagar komma, då jag skall hemsöka Babels beläten, och då hela dess land skall stå med skam och alla skola falla slagna därinne.
Jer 51:48  Då skola himmel och jord jubla över Babel, de och allt vad i dem är, då nu förhärjarna komma över det norrifrån, säger HERREN.
Jer 51:49  Ja, I slagna av Israel, också Babel måste falla, likasom för Babel människor föllo slagna över hela jorden.
Jer 51:50  I som haven lyckats rädda eder undan svärdet, gån åstad, stannen icke. Kommen ihåg HERREN, i fjärran land, och tänken på Jerusalem.
Jer 51:51  Vi stå här med skam, ja vi måste höra smädelse; blygsel höljer vårt ansikte, ty främlingar hava kastat sig över vad heligt som fanns i HERRENS hus.
Jer 51:52  Se, därför skola dagar komma, säger HERREN, då jag skall hemsöka dess beläten, och då slagna män skola jämra sig i hela dess land.
Jer 51:53  Om Babel än stege upp till himmelen, och om det gjorde sin befästning än så hög och stark så skulle dock förhärjare ifrån mig komma över det, säger HERREN.
Jer 51:54  Klagorop höras från Babel, och stort brak från kaldéernas land.
Jer 51:55  Ty HERREN förhärjar Babel och gör slut på det stora larmet därinne. Och deras böljor brusa såsom stora vatten; dånet av dem ljuder högt.
Jer 51:56  Ty över det, över Babel, kommer en förhärjare, och dess hjältar tagas till fånga, deras bågar brytas sönder. Se, HERREN är en vedergällningens Gud; han lönar till fullo.
Jer 51:57  Ja, jag skall göra dess furstar druckna, så ock dess visa män, dess ståthållare, dess landshövdingar och dess hjältar, och de skola somna in i en evig sömn, ur vilken de aldrig skola uppvakna, säger konungen, han vilkens namn är HERREN Sebaot.
Jer 51:58  Så säger HERREN Sebaot: Det vida Babels murar skola i grund omstörtas, och dess höga portar skola brännas upp i eld. Så möda sig folken för det som skall bliva till intet, och folkslagen arbeta sig trötta för det som skall förbrännas av elden.
Jer 51:59  Detta är vad profeten Jeremia bjöd Seraja, son till Neria, son till Mahaseja, när denne begav sig till Babel med Sidkia, Juda konung, i hans fjärde regeringsår. Seraja var nämligen den som hade bestyret med lägerplatserna.
Jer 51:60  Och Jeremia tecknade i en och samma bok upp alla de olyckor som skulle komma över Babel, allt detta som nu är skrivet om Babel.
Jer 51:61  Jeremia sade till Seraja: "När du kommer till Babel, så se till, att du läser upp allt detta.
Jer 51:62  Och du skall säga: 'HERRE, du har själv talat om denna ort att du vill fördärva den, så att ingen mer skall bo där, varken någon människa eller något djur; ty den skall vara en ödemark för evärdlig tid.'
Jer 51:63  Och när du har läst upp boken till slut, så bind en sten vid den och kasta den ut i Frat,
Jer 51:64  och säg: 'På detta sätt skall Babel sjunka ned och icke mer komma upp, för den olyckas skull som jag skall låta komma över det, mitt under deras ävlan.'" Så långt Jeremias ord.
Jer 52:1  Sidkia var tjuguett år gammal, när han blev konung, och han regerade elva år i Jerusalem. Hans moder hette Hamital, Jeremias dotter, från Libna.
Jer 52:2  Han gjorde vad ont var i HERRENS ögon, alldeles såsom Jojakim hade gjort.
Jer 52:3  Ty på grund av HERRENS vrede skedde vad som skedde med Jerusalem och Juda, till dess att han kastade dem bort ifrån sitt ansikte. Och Sidkia avföll från konungen i Babel.
Jer 52:4  Då, i hans nionde regeringsår, i tionde månaden, på tionde dagen i månaden, kom Nebukadressar, konungen i Babel, med hela sin här till Jerusalem, och de belägrade det; och de byggde en belägringsmur runt omkring det.
Jer 52:5  Så blev staden belägrad och förblev så ända till konung Sidkias elfte regeringsår.
Jer 52:6  Men i fjärde månaden, på nionde dagen i månaden, var hungersnöden så stor i staden, att mängden av folket icke hade något att äta.
Jer 52:7  Och staden stormades, och allt krigsfolket flydde och drog ut ur staden om natten genom porten mellan de båda murarna (den port som ledde till den kungliga trädgården), medan kaldéerna lågo runt omkring staden; och de togo vägen åt Hedmarken till.
Jer 52:8  Men kaldéernas här förföljde konungen, och de hunno upp Sidkia på Jerikos hedmarker, sedan hela hans här hade övergivit honom och skingrat sig.
Jer 52:9  Och de grepo konungen och förde honom till den babyloniske konungen i Ribla i Hamats land; där höll denne rannsakning och dom med honom.
Jer 52:10  Och konungen i Babel lät slakta Sidkias barn inför hans ögon; därjämte lät han ock slakta alla Juda furstar i Ribla.
Jer 52:11  Och på Sidkia själv lät han sticka ut ögonen och lät fängsla honom med kopparfjättrar. Och konungen i Babel förde honom därefter till Babel och lät honom sitta i fängelsehuset ända till hans dödsdag.
Jer 52:12  I femte månaden, på tionde dagen i månaden, detta i den babyloniske konungen Nebukadressars nittonde regeringsår, kom Nebusaradan, översten för drabanterna; denne var den babyloniske konungens förtroendeman vid Jerusalem.
Jer 52:13  Han brände upp HERRENS hus och konungshuset; ja, alla hus i Jerusalem, i synnerhet alla de förnämas hus, brände han upp i eld.
Jer 52:14  Och alla murar runt omkring Jerusalem brötos ned av hela den här av kaldéer, som översten för drabanterna hade med sig.
Jer 52:15  Och en del av de ringaste bland folket och den övriga återstoden av folket, dem som voro kvar i staden, och de överlöpare som hade gått över till konungen i Babel, så ock det hantverksfolk som fanns kvar, dem förde Nebusaradan, översten för drabanterna, bort i fångenskap.
Jer 52:16  Men av de ringaste i landet lämnade Nebusaradan, översten för drabanterna, några kvar till vingårdsmän och åkermän.
Jer 52:17  Kopparpelarna i HERRENS hus, bäckenställen och kopparhavet i HERRENS hus slogo kaldéerna sönder och förde all kopparen till Babel.
Jer 52:18  Och askkärlen, skovlarna, knivarna, de båda slagen av skålar och alla kopparkärl som hade begagnats vid gudstjänsten togo de bort.
Jer 52:19  Likaledes tog översten för drabanterna bort faten, fyrfaten, offerskålarna, askkärlen, ljusstakarna, de andra skålarna och bägarna, allt vad som var av rent guld eller av rent silver.
Jer 52:20  Vad angår de två pelarna, havet som var allenast ett, och de tolv kopparoxarna under bäckenställen, som konung Salomo hade låtit göra till HERRENS hus, så kunde kopparen i alla dessa föremål icke vägas.
Jer 52:21  Och vad pelarna angår, så var den ena pelaren aderton alnar hög, och en tolv alnar lång tråd mätte dess omfång, och den var fyra finger tjock och ihålig.
Jer 52:22  Och ovanpå den var ett pelarhuvud av koppar; och detta ena pelarhuvud var fem alnar högt, och ett nätverk och granatäpplen funnos på pelarhuvudet runt omkring, alltsammans av koppar. Och likadant var det på den andra pelaren med fina granatäpplen.
Jer 52:23  Och granatäpplena voro nittiosex utåt; men tillsammans voro granatäpplena på nätverket runt omkring ett hundra.
Jer 52:24  Och översten för drabanterna tog översteprästen Seraja jämte Sefanja prästen näst under honom, så ock de tre som höllo vakt vid tröskeln,
Jer 52:25  och från staden tog han en hovman, den som var anförare för krigsfolket, och sju av konungens närmaste män, som påträffades i staden, så ock härhövitsmannens sekreterare, som plägade utskriva folket i landet till krigstjänst, och sextio andra män av landets folk, som påträffades i staden -
Jer 52:26  dessa tog Nebusaradan, översten för drabanterna, och förde dem till den babyloniske konungen i Ribla.
Jer 52:27  Och konungen i Babel lät avliva dem där, i Ribla i Hamats land. Så blev Juda bortfört från sitt land.
Jer 52:28  Detta är antalet av dem som Nebukadressar förde bort: i det sjunde året tre tusen tjugutre judar,
Jer 52:29  och i Nebukadressars adertonde: regeringsår åtta hundra trettiotvå personer från Jerusalem.
Jer 52:30  Men i Nebukadressars tjugutredje regeringsår bortförde Nebusaradan, översten för drabanterna, av judarna sju hundra fyrtiofem personer. Hela antalet utgjorde fyra tusen sex hundra personer.
Jer 52:31  Men i det trettiosjunde året sedan Jojakin, Juda konung, hade blivit bortförd i fångenskap, i tolfte månaden, på tjugufemte dagen i månaden, tog Evil-Merodak, konungen i Babel - samma år han blev konung - Jojakin, Juda konung, till nåder och förde honom ut ur fängelset.
Jer 52:32  Och han talade vänligt med honom och gav honom översta platsen bland de konungar som voro hos honom i Babel.
Jer 52:33  Han fick lägga av sin fångdräkt och beständigt äta vid hans bord, så länge han levde.
Jer 52:34  Och ett ständigt underhåll gavs honom från konungen i Babel, visst för var dag, ända till hans dödsdag, så länge han levde.


\end{document}