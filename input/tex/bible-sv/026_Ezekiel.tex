\begin{document}

\title{Hesekiel}


\chapter{1}

\par 1 I det trettionde året, på femte dagen i fjärde månaden, när jag var bland de fångna vid strömmen Kebar, öppnades himmelen, och jag såg en syn från Gud.
\par 2 På femte dagen i månaden, när femte året gick, efter att konung Jojakin hade blivit bortförd i fångenskap,
\par 3 kom HERRENS ord till prästen Hesekiel, Busis son, i kaldéernas land vid strömmen Kebar, och HERRENS hand kom där över honom.
\par 4 Och jag fick se en stormvind komma norrifrån, ett stort moln med flammande eld, och ett sken omgav det; och mitt däri, mitt i elden, syntes något som var såsom glänsande malm.
\par 5 Och mitt däri syntes något som liknade fyra väsenden, och dessa sågo ut på följande sätt: de liknade människor,
\par 6 men vart väsende hade fyra ansikten, och vart och ett av dem hade fyra vingar,
\par 7 och deras ben voro raka och deras fötter såsom fötterna på en kalv och de glimmade såsom glänsande koppar.
\par 8 Och de hade människohänder under sina vingar på alla fyra sidorna. Och med de fyras ansikten och vingar förhöll det sig så:
\par 9 deras vingar slöto sig intill varandra; och när de gingo, behövde de icke vända sig, utan gingo alltid rakt fram.
\par 10 Och deras ansikten liknade människoansikten, och alla fyra hade lejonansikten på högra sidan, och alla fyra hade tjuransikten på vänstra sidan, och alla fyra hade ock örnansikten.
\par 11 Så var det med deras ansikten. Och deras vingar voro utbredda upptill; vart väsende hade två vingar med vilka de slöto sig intill varandra, och två som betäckte deras kroppar.
\par 12 Och de gingo alltid rakt fram; vart anden ville gå, dit gingo de, och när de gingo, behövde de icke vända sig.
\par 13 Och väsendena voro till sitt utseende lika eldsglöd, som brunno likasom bloss, under det att elden for omkring mellan väsendena; och den gav ett sken ifrån sig, och ljungeldar foro ut ur elden.
\par 14 Och väsendena hastade fram och tillbaka likasom blixtar.
\par 15 När jag nu såg på väsendena, fick jag se ett hjul stå på jorden, invid väsendena, vid var och en av deras fyra framsidor.
\par 16 Och det såg ut som om hjulen voro gjorda av något som liknade krysolit, och alla fyra voro likadana; och det såg vidare ut som om de voro så gjorda, att ett hjul var insatt i ett annat.
\par 17 När de skulle gå, kunde de gå åt alla fyra sidorna, de behövde icke vända sig, när de gingo.
\par 18 Och deras lötar voro höga och förskräckliga, och på alla fyra voro lötarna fullsatta med ögon runt omkring.
\par 19 Och när väsendena gingo, gingo ock hjulen invid dem, och när väsendena lyfte sig upp över jorden lyfte sig ock hjulen.
\par 20 Vart anden ville gå, dit gingo de, ja, varthelst anden ville gå; och hjulen lyfte sig jämte dem, ty väsendenas ande var i hjulen.
\par 21 När väsendena gingo, gingo ock dessa; när de stodo stilla, stodo ock dessa stilla; när de lyfte sig upp över jorden, lyfte sig ock hjulen jämte dem, ty väsendenas ande var i hjulen.
\par 22 Och över väsendenas huvuden syntes något som liknade ett himlafäste, till utseendet såsom underbar kristall, utspänt ovanpå deras huvuden.
\par 23 Och under fästet voro deras vingar utbredda rätt emot varandra Vart särskilt väsende hade två vingar med vilka det kunde betäcka sin kropp.
\par 24 Och när de gingo, lät dånet av deras vingar i mina öron såsom dånet av stora vatten, såsom den Allsmäktiges röst; ja, det var ett väldigt dån, likt dånet från en härskara. Men när de stodo stilla, höllo de sina vingar nedsänkta.
\par 25 Och ovan fästet, som vilade på deras huvuden, dånade det; när de då stodo stilla, höllo de sina vingar nedsänkta.
\par 26 Och ovanpå fästet, som vilade på deras huvuden, syntes något som såg ut att vara av safirsten, och som liknade en tron; och ovanpå det som liknade en tron satt en som till utseendet liknade en människa,
\par 27 Och jag såg något som var såsom glänsande malm och omgivet runt omkring av något som såg ut såsom eld, ända ifrån det som såg ut att vara hans länder och sedan allt uppåt. Men nedåt från det som såg ut att vara hans länder såg jag något som såg ut såsom eld; och ett sken omgav honom.
\par 28 Såsom bågen som synes i skyn, när det regnar, så såg skenet ut där runt omkring. Så såg det ut, som tycktes mig vara HERRENS härlighet; och när jag såg det, föll jag ned på mitt ansikte, och jag hörde rösten av en som talade

\chapter{2}

\par 1 Och han sade till mig: "Du människobarn, stå upp på dina fötter, så vill jag tala med dig."
\par 2 När han så talade till mig, kom en andekraft i mig och reste upp mig på mina fötter; och jag hörde på honom som talade till mig.
\par 3 Och han sade till mig: "Du människobarn, jag sänder dig till Israels barn, de avfälliga hedningarna som hava avfallit från mig; de och deras fäder hava begått överträdelser mot mig allt intill den dag som i dag är.
\par 4 Till barnen med hårda pannor och förstockade hjärtan sänder jag dig, och du skall säga till dem: 'Så säger Herren, HERREN'
\par 5 Och evad de höra därpå eller icke - ty de äro ett gensträvigt släkte - så skola de dock förnimma att en profet har varit ibland dem.
\par 6 Och du, människobarn, frukta icke för dem, och frukta icke för deras ord, fastän du omgives av tistlar och törnen och bor ibland skorpioner. Nej, frukta icke för deras ord, och var icke förfärad för dem själva, då de nu äro ett gensträvigt släkte.
\par 7 Utan tala mina ord till dem, evad de höra på dem eller icke, då de nu äro så gensträviga.
\par 8 Men du, människobarn, hör nu vad jag talar till dig; var icke gensträvig såsom detta gensträviga släkte. Öppna din mun och ät vad jag giver dig."
\par 9 Och jag fick se en hand uträckas mot mig, och i den såg jag en bokrulle.
\par 10 Och denna breddes ut framför mig, och den var fullskriven innan och utan; och där voro uppskrivna klagosånger, suckan och verop.

\chapter{3}

\par 1 Och han sade till mig: "Du människobarn, ät vad du här finner, ät upp denna rulle, och gå sedan åstad och tala till Israels hus."
\par 2 Då öppnade jag min mun, och han gav mig rullen att äta.
\par 3 Och han sade till mig: "Du människobarn, du måste mätta din buk och fylla dina inälvor med den rulle som jag nu giver dig." Och jag åt, och den var i min mun söt såsom honung.
\par 4 Och han sade till mig: "Du människobarn, gå bort till Israels hus och tala till dem med mina ord.
\par 5 Ty du bliver ju icke sänd till ett folk med obegripligt språk och trög tunga, utan till Israels hus,
\par 6 icke till mångahanda folk med obegripligt språk och trög tunga, vilkas tal du icke förstår; sannerligen, sände jag dig till sådana, så skulle de höra på dig
\par 7 Men Israels hus vill icke höra på dig, ty de vilja icke höra på mig; hela Israels hus har hårda pannor och förhärdade hjärtan.
\par 8 Men se, jag gör ditt ansikte hårt såsom deras ansikten, och din panna hård såsom deras pannor.
\par 9 Ja, jag gör din panna hård såsom diamant, hårdare än flinta. Du skall icke frukta för dem och icke förfäras för dem, då de nu äro ett gensträvigt släkte."
\par 10 Och han sade till mig: "Du människobarn, allt vad jag talar till dig skall du upptaga i ditt hjärta och höra med dina öron.
\par 11 Och gå bort till dina fångna landsmän, och tala till dem och säg till dem: 'Så säger Herren, HERREN' - evad de nu höra därpå eller icke."
\par 12 Och en andekraft lyfte upp mig, och jag hörde bakom mig ljudet av ett väldigt dån: "Lovad vare HERRENS härlighet, där varest den är!",
\par 13 så ock ljudet av väsendenas vingar, som rörde vid varandra, och ljudet av hjulen jämte dem och ljudet av ett väldigt dån.
\par 14 Och en andekraft lyfte upp mig och förde mig bort, och jag färdades åstad, bedrövad och upprörd i min ande, och HERRENS hand var stark över mig.
\par 15 Och jag kom till de fångna i Tel-Abib, till dem som bodde vid strömmen Kebar, till den plats där de bodde; och jag satt där ibland dem i sju dagar, försänkt i djup sorg.
\par 16 Men efter sju dagar kom HERRENS ord till mig; han sade:
\par 17 "Du människobarn, jag har satt dig till en väktare för Israels hus, för att du å mina vagnar skall varna dem, när du hör ett ord från min mun.
\par 18 Om jag säger till den ogudaktige: 'Du måste dö' och du då icke varnar honom, ja, om du icke säger något till att varna den ogudaktige för hans ogudaktiga väg och rädda hans liv, då skall väl den ogudaktige dö genom sin missgärning, men hans blod skall jag utkräva av din hand.
\par 19 Men om du varnar den ogudaktige och han likväl icke vänder om från sin ogudaktighet och sin ogudaktiga väg, då skall visserligen han dö genom sin missgärning, men du själv har räddat din själ.
\par 20 Och om en rättfärdig man vänder om från sin rättfärdighet och gör vad orätt är, så skall jag lägga en stötesten i hans väg, och han skall dö. Om du då icke har varnat honom, så skall han väl dö genom sin synd, och den rättfärdighet som han förr har övat skall icke varda ihågkommen, men hans blod skall jag utkräva av din hand.
\par 21 Men om du har varnat den rättfärdige, för att han, den rättfärdige, icke skall synda, och han så avhåller sig från synd, då skall han förvisso få leva, därför att han lät varna sig, och du själv har då räddat din själ."
\par 22 Och HERRENS hand kom där över mig, och han sade till mig: "Stå upp och gå ut på slätten; där skall jag tala med dig."
\par 23 Då stod jag upp och gick ut på slätten; och se, där stod HERRENS härlighet, alldeles sådan som jag hade sett den vid strömmen Kebar; och jag föll ned på mitt ansikte.
\par 24 Men en andekraft kom i mig och reste upp mig på mina fötter. Och han talade med mig och sade till mig: "Gå och stäng dig inne i ditt hus.
\par 25 Och se, du människobarn, bojor skola läggas på dig, och du skall bliva bunden med sådana, så att du icke kan gå ut bland de andra.
\par 26 Och jag skall låta din tunga låda vid din gom, så att du bliver stum och icke kan bestraffa dem, då de nu äro ett gensträvigt släkte.
\par 27 Men när jag talar med dig, skall jag upplåta din mun, så att du kan säga till dem: 'Så säger Herren, HERREN.' Den som då vill höra, han höre, och den som icke vill, han höre icke, då de nu äro ett gensträvigt släkte."

\chapter{4}

\par 1 Och du, människobarn, tag dig en tegeltavla och lägg den framför dig och rista på den in en stad, nämligen Jerusalem.
\par 2 Och res upp bålverk mot den och bygg en belägringsmur mot den och kasta upp en vall mot den och slå upp läger mot den och sätt upp murbräckor mot den runt omkring.
\par 3 Och tag dig en järnplåt och sätt upp den såsom en järnvägg mellan dig och staden; och vänd så ditt ansikte emot den och håll den belägrad och ansätt den. Detta skall vara ett tecken för Israels hus.
\par 4 Och lägg du dig på din vänstra sida och lägg Israels hus' missgärning ovanpå; lika många dagar som du ligger så, skall du bära på deras missgärning.
\par 5 Jag skall låta deras missgärningsår för dig motsvaras av ett lika antal dagar, nämligen av tre hundra nittio dagar; så länge skall du bara på Israels hus' missgärning.
\par 6 Och sedan, när du har fullgjort detta, skall du lägga dig på din högra sida och bära på Juda hus' missgärning; i fyrtio dagar, var dag svarande mot ett år, skall denna min föreskrift gälla för dig.
\par 7 Och du skall vända ditt ansikte och din blottade arm mot det belägrade Jerusalem och profetera mot det.
\par 8 Och se, jag skall lägga bojor på dig, så att du icke kan vända dig från den ena sidan på den andra, förrän dina belägringsdagar äro slut.
\par 9 Och tag dig vete, korn, bönor, linsärter, hirs och spält och lägg detta i ett och samma kärl och baka dig bröd därav; lika många dagar som du ligger på ena sidan, alltså tre hundra nittio dagar, skall detta vara vad du har att äta.
\par 10 Den mat som du får att äta skall du äta efter vikt, tjugu siklar om dagen; detta skall du hava att äta från en viss timme ena dagen till samma timme nästa dag
\par 11 Du skall ock dricka vatten efter mått, nämligen en sjättedels hin; så mycket skall du hava att dricka från en viss timme ena dagen till samma timme nästa dag.
\par 12 Tillredd såsom kornkakor skall maten ätas av dig, och du skall tillreda den inför deras ögon på bränsle av människoträck.
\par 13 Och HERREN tillade: "Likaså skola Israels barn äta sitt bröd orent bland hedningarna, till vilka jag skall fördriva dem."
\par 14 Men jag svarade: "Ack, Herre, HERRE! Se, jag har ännu aldrig blivit orenad. Jag har aldrig, från min ungdom och intill nu, ätit något självdött eller ihjälrivet djur; och sådant kött som räknas för vederstyggligt har aldrig kommit i min mun."
\par 15 Då sade han till mig: "Välan, jag vill låta dig taga kospillning i stället för människoträck; vid sådan må du baka ditt bröd."
\par 16 Och han sade åter till mig: "Du människobarn, se, jag vill fördärva livsuppehället för Jerusalem, så att de skola äta bröd efter vikt, och det med oro, och dricka vatten efter mått, och det med förfäran;
\par 17 ja, så att de lida brist på bröd och vatten och gripas av förfäran, den ene med den andre, och försmäkta genom sin missgärning.

\chapter{5}

\par 1 Och du, människobarn, tag dig ett skarpt svärd och bruka det såsom rakkniv, och för det över ditt huvud och din haka; tag dig så en vågskål och dela det avrakade håret.
\par 2 En tredjedel skall du bränna upp i eld mitt i staden, när belägringsdagarna hava gått till ända; en tredjedel skall du taga ut och slå den med svärdet där runt omkring; och en tredjedel skall du strö ut för vinden, och mitt svärd skall jag draga ut efter dem.
\par 3 Men några få strån skall du taga undan därifrån, och dem skall du knyta in i flikarna av din mantel.
\par 4 Och av dessa strån skall du återigen taga några och kasta dem i elden och bränna upp dem i eld. Härifrån skall en eld gå ut över hela Israels hus.
\par 5 Så säger Herren, HERREN: Detta år Jerusalem, som jag har satt mitt ibland hednafolken, med länder runt däromkring.
\par 6 Men det var gensträvigt mot mina rätter på ett ännu ogudaktigare sätt än hednafolken, och var ännu mer gensträvigt mot mina stadgar än länderna runt däromkring; ty de förkastade mina rätter och vandrade icke efter mina stadgar.
\par 7 Därför säger Herren, HERREN så: Eftersom I haven rasat värre än hednafolken runt omkring eder, och icke haven vandrat efter mina stadgar och icke gjort efter mina rätter, ja, icke ens gjort efter de hednafolks rätter, som bo runt omkring eder,
\par 8 därför säger Herren, HERREN så: Se, fördenskull skall jag också komma över dig och skipa rätt mitt ibland dig inför hednafolkens ögon;
\par 9 jag skall göra med dig vad jag aldrig förr har gjort, och sådant som jag aldrig mer vill göra, för alla dina styggelsers skull.
\par 10 Därför skola i dig föräldrar äta sina barn, och barn sina föräldrar; och jag skall skipa rätt i dig och strö ut för alla vindar allt som bliver kvar av dig.
\par 11 Ja, så sant jag lever, säger Herren, HERREN: sannerligen, därför att du har orenat min helgedom med alla dina skändligheter och alla dina styggelser, skall jag också utan skonsamhet vända bort mitt öga och icke hava någon misskund.
\par 12 En tredjedel av dig skall dö av pest och förgås av hunger i dig, en tredjedel skall falla för svärd runt omkring dig; och en tredjedel skall jag strö ut för alla vindar, och mitt svärd skall jag draga ut efter dem.
\par 13 Ja, min vrede skall få uttömma sig, och jag skall släcka min förtörnelse på dem och hämnas på dem; och när jag så uttömmer min förtörnelse på dem, skola de förnimma att jag, HERREN, har talat i min nitälskan.
\par 14 Och jag skall låta dig bliva en ödemark och en smälek bland folken runt omkring dig, inför var mans ögon, som går där fram.
\par 15 Ja, det skall bliva till smälek och hån, till varnagel och skräck för folken runt omkring dig, när jag så skipar rätt i dig med vrede och förtörnelse och förtörnelses tuktan. Jag, HERREN, har talat.
\par 16 När jag sänder bland dem hungerns onda pilar, som bliva till fördärv, ja, när jag sänder dessa till att fördärva eder och så låter eder hunger bliva allt värre, då skall jag förstöra för eder edert livsuppehälle.
\par 17 Jag skall sända över eder hungersnöd och vilddjur, som skola döda edra barn; och pest och blodsutgjutelse skall gå över dig, och svärd skall jag låta komma över dig. Jag, HERREN, har talat.

\chapter{6}

\par 1 Och HERRENS ord kom till mig; han sade:
\par 2 Du människobarn, vänd ditt ansikte mot Israels berg och profetera mot dem
\par 3 och säg: I Israels berg, hören Herrens, HERRENS ord: Så säger Herren, HERREN till bergen och höjderna, till bäckarna och dalarna: Se, jag skall låta svärd komma över eder och förstöra edra offerhöjder.
\par 4 Och edra altaren skola varda förödda och edra solstoder sönderkrossade, och dem av eder, som bliva slagna, skall jag låta bliva kastade inför edra eländiga avgudar.
\par 5 Och jag skall låta Israels barns döda kroppar ligga där inför deras eländiga avgudar, och jag skall förströ edra ben runt omkring edra altaren.
\par 6 Var I än ären bosatta skola städerna bliva öde och offerhöjderna ödelagda, så att edra altaren stå öde och förödda, och edra eländiga avgudar bliva sönderslagna och få en ände, och edra solstoder bliva nedhuggna, och edra verk utplånade.
\par 7 Dödsslagna män skola då falla bland eder; och I skolen förnimma att jag är HERREN.
\par 8 Och om jag låter några leva kvar, så att somliga av eder, när I bliven förströdda i länderna, räddas undan svärdet ute bland folken,
\par 9 så skola dessa edra räddade ute bland folken, där de äro i fångenskap, tänka på mig, när jag har krossat deras trolösa hjärtan, som veko av ifrån mig, och deras ögon, som i trolös avfällighet skådade efter deras eländiga avgudar; och de skola känna leda vid sig själva för det onda som de hava gjort med alla sina styggelser.
\par 10 Och de skola förnimma att jag är HERREN. Det är icke ett tomt ord att jag skall låta denna olycka komma över dem.
\par 11 Så säger Herren, HERREN: Slå dina händer tillsammans, och stampa med dina fötter, och ropa ack och ve över alla de onda styggelserna i Israels hus, ty genom svärd, hunger och pest måste de falla.
\par 12 Den som är långt borta skall dö av pest, och den som är nära skall falla för svärd, och den som bliver kvar och varder bevarad skall dö av hunger; så skall jag uttömma min vrede på dem.
\par 13 Och I skolen förnimma att jag är HERREN, när deras slagna män ligga där mitt ibland sina eländiga avgudar, runt omkring sina altaren, på alla höga kullar, på alla bergstoppar, under alla gröna träd och under alla lummiga terebinter, varhelst de hava låtit en välbehaglig lukt uppstiga till alla sina oländiga avgudar.
\par 14 Och jag skall uträcka min hand mot dem och göra landet mer öde och tomt än öknen vid Dibla, var de än äro bosatta; och de skola förnimma att jag är HERREN.

\chapter{7}

\par 1 Och HERRENS ord kom till mig; han sade:
\par 2 Du människobarn, så säger Herren, HERREN till Israels land: Änden! Ja, änden kommer över landets fyra hörn.
\par 3 Nu kommer änden över dig, ty jag skall sända min vrede mot dig och döma dig efter dina gärningar och låta alla dina styggelser komma över dig.
\par 4 Jag skall icke visa dig någon skonsamhet och icke hava någon misskund; nej, jag skall låta dina gärningar komma över dig, och dina styggelser skola vila på dig. Och I skolen förnimma att jag är HERREN.
\par 5 Så säger Herren, HERREN: Se, en olycka kommer, en olycka ensam i sitt slag!
\par 6 En ände kommer, ja, änden kommer, den vaknar upp och kommer över dig.
\par 7 Ja se, det kommer! Nu kommer ordningen till dig, du folk som bor här i landet; din stund kommer, förvirringens dag är nära, då intet skördeskri mer skall höras på bergen.
\par 8 Nu skall jag snart utgjuta min förtörnelse över dig och uttömma min vrede på dig, och döma dig efter dina gärningar och låta alla dina styggelser komma över dig.
\par 9 Jag skall icke visa någon skonsamhet och icke hava någon misskund, jag skall giva dig efter dina gärningar, och dina styggelser skola vila på dig. Och I skolen förnimma att jag, HERREN, är den som slår.
\par 10 Se, dagen är inne; se, det kommer! Ordningen går sin gång, riset blomstrar upp, övermodet grönskar;
\par 11 våldet reser sig till ett ris för ogudaktigheten. Då bliver intet kvar av dem, intet av hela deras hop, intet av deras gods, och till intet bliver deras härlighet.
\par 12 Stunden kommer, dagen nalkas; köparen må icke glädja sig, och säljaren må icke sörja, ty vredesglöd kommer över hela hopen därinne.
\par 13 Säljaren skall icke få tillbaka vad han har sålt, om han ens får förbliva vid liv. Ty profetian om hela hopen därinne skall icke ryggas och ingen som lever i missgärning skall kunna hålla stånd.
\par 14 Man stöter i basun och rustar allt i ordning, men ingen drager ut till strid; ty min vredesglöd går fram över hela hopen därinne.
\par 15 Ute härjar svärdet och därinne pest och hungersnöd, den som är ute på marken dör genom svärdet, och den som är i staden, honom förtär hungersnöd och pest.
\par 16 Och om några av dem bliva räddade, så skola de söka sin tillflykt i bergen och vara lika klyftornas duvor, som allasammans klaga. Så skall det gå var och en genom hans missgärning.
\par 17 Alla händer skola sjunka ned, och alla knän skola bliva såsom vatten.
\par 18 Människorna skola kläda sig i sorgdräkt, och förfäran skall övertäcka dem, alla ansikten skola höljas av skam, och alla huvuden skola bliva skalliga.
\par 19 Man skall kasta sitt silver ut på gatorna och akta sitt guld såsom orenlighet. Deras silver och guld skall icke kunna rädda dem på HERRENS vredes dag, de skola icke kunna mätta sig därmed eller därmed fylla sin buk; ty det har varit för dem en stötesten till missgärning.
\par 20 Dess sköna glans brukade man till högfärd, ja, de gjorde därav sina styggeliga bilder, sina skändliga avgudar. Därför skall jag göra det till orenlighet för dem.
\par 21 Jag skall giva det såsom byte i främlingars hand och såsom rov åt de ogudaktigaste på jorden, för att de må ohelga det.
\par 22 Och jag skall vända bort mitt ansikte ifrån dem, så att man får ohelga min klenod; våldsmän skola få draga därin och ohelga den.
\par 23 Gör kedjorna redo; ty landet är fullt av blodsdomar, och staden är full av orätt.
\par 24 Och jag skall låta de värsta hednafolk komma och taga deras hus i besittning. Så skall jag göra slut på de fräckas övermod, och deras helgedomar skola varda oskärade.
\par 25 Förskräckelse skall komma, och när de söka räddning, skall ingen vara att finna.
\par 26 Den ena olyckan skall komma efter den andra, det ena sorgebudet skall följa det andra. Man skall få tigga profeterna om syner, prästerna skola komma till korta med sin undervisning och de äldste med sina råd.
\par 27 Konungen skall sörja, hövdingarna skola kläda sig i förskräckelse, och folket i landet skall stå där med darrande händer. Jag skall göra med den efter deras gärningar och skipa rätt åt dem såsom rätt är åt dem; och de skola förnimma att jag är HERREN.

\chapter{8}

\par 1 Och i sjätte året, i sjätte månaden, på femte dagen i månaden, när jag satt i mitt hus och de äldste i Juda sutto hos mig, kom Herrens, HERRENS hand där över mig.
\par 2 Och jag fick se något som till utseendet liknade eld; allt, ända ifrån det som såg ut att vara hans länder och sedan allt nedåt, var eld. Men från hans länder och sedan allt uppåt syntes något som liknade strålande ljus, och som var såsom glänsande malm.
\par 3 Och han räckte ut något som var bildat såsom en hand och fattade mig vid en lock av mitt huvudhår; och en andekraft lyfte mig upp mellan himmel och jord och förde mig, i en syn från Gud, till Jerusalem, dit där man går in till den inre förgården genom den port som vetter åt norr, där varest avgudabelätet, det som hade uppväckt Guds nitälskan, hade sin plats.
\par 4 Och se, där syntes Israels Guds härlighet, alldeles sådan som jag hade sett den på slätten.
\par 5 Och han sade till mig: "Du människobarn, lyft upp dina ögon mot norr." När jag nu lyfte upp mina ögon mot norr, fick jag se avgudabelätet, det som hade uppväckt Guds nitälskan, stå där norr om altarporten, vid själva ingången.
\par 6 Och han sade till mig: "Du människobarn, ser du vad de göra här? Stora äro de styggelser som Israels hus här bedriver, så att jag måste draga långt bort ifrån min helgedom; men du skall få se ännu flera, större styggelser."
\par 7 Sedan förde han mig till förgårdens ingång, och jag fick där se ett hål i väggen.
\par 8 Och han sade till mig: "Du människobarn, bryt igenom väggen." Då bröt jag igenom väggen och fick nu se en dörr.
\par 9 Och han sade till mig: "Gå in och se vilka onda styggelser de här bedriva."
\par 10 När jag nu kom in, fick jag se allahanda bilder av vederstyggliga kräldjur och fyrfotadjur, så ock Israels hus' alla eländiga avgudar, inristade runt omkring på väggarna.
\par 11 Och framför dem stodo sjuttio av: de äldste i Israels hus, och Jaasanja, Safans son, stod mitt ibland dem, och var och en av dem hade sitt rökelsekar i handen, och vällukt steg upp från rökelsemolnet.
\par 12 Och han sade till mig: "Du människobarn, ser du vad de äldste i Israels hus bedriva i mörkret, var och en i sin avgudakammare? Ty de säga: 'HERREN ser oss icke, HERREN har övergivit landet.'"
\par 13 Därefter sade han till mig: "Du skall få se ännu flera, större styggelser som dessa bedriva."
\par 14 Och han förde mig fram mot ingången till norra porten på HERRENS hus, och se, där sutto kvinnor som begräto Tammus.
\par 15 Och han sade till mig: "Du människobarn, ser du detta? Men du skall få se ännu flera styggelser, större än dessa."
\par 16 Och han förde mig till den inre förgården till HERRENS hus, och se, vid ingången till HERRENS tempel, mellan förhuset och altaret, stodo vid pass tjugufem män, som vände ryggarna åt HERRENS tempel och ansiktena åt öster, och som tillbådo solen i öster.
\par 17 Och han sade till mig: "Du människobarn, ser du detta? Är det icke nog för Juda hus att bedriva de styggelser som de här hava bedrivit, eftersom de nu ock hava uppfyllt landet med orätt och åter hava förtörnat mig? Se nu huru de sätta vinträdskvisten för näsan!
\par 18 Därför skall också jag utföra mitt: verk i vrede; jag skall icke visa någon skonsamhet och icke hava någon misskund. Och om de än ropa med hög röst inför mig, skall jag dock icke höra dem."

\chapter{9}

\par 1 Och jag hörde honom ropa med hög röst och säga: "Kommen hit med hemsökelser över staden, och var och en have sitt mordvapen i handen."
\par 2 Och se, då kommo sex män från övre porten, den som vetter åt norr och var och en hade sin stridshammare i handen; och bland dem fanns en man som var klädd i linnekläder och hade ett skrivtyg vid sin länd. Och de kommo och ställde sig vid sidan av kopparaltaret.
\par 3 Och Israels Guds härlighet hade lyft sig från keruben, som den vilade på, och hade flyttat sig till tempelhusets tröskel, och ropade nu till mannen som var klädd i linnekläderna och hade skrivtyget vid sin länd;
\par 4 HERREN sade till honom: "Gå igenom Jerusalems stad, och teckna med ett tecken på pannan de män som sucka och jämra sig över alla styggelser som bedrivas därinne."
\par 5 Och till de andra hörde jag honom säga: "Dragen fram i staden efter honom och slån ned folket; visen ingen skonsamhet och haven ingen misskund.
\par 6 Både åldringar och ynglingar och jungfrur, både barn och kvinnor skolen I dräpa och förgöra, men I mån icke komma vid någon som har tecknet på sig, och I skolen begynna vid min helgedom." Och de begynte med de äldste, med de män som stodo framför tempelhuset.
\par 7 Han sade nämligen till dem: "Orenen tempelhuset, och fyllen upp förgårdarna med slagna; dragen sedan ut." Och de drogo ut och slogo ned folket i staden.
\par 8 Då nu jag blev lämnad kvar, när de så slogo folket, föll jag ned på mitt ansikte och ropade och sade: "Ack, Herre, HERRE, vill du då förgöra hela kvarlevan av Israel, eftersom du så utgjuter din vrede över Jerusalem?"
\par 9 Han sade till mig: "Israels och Juda hus' missgärning är alltför stor; landet är uppfyllt med orätt, och staden är full av lagvrängning. Ty de säga: 'HERREN har övergivit landet, HERREN ser det icke.'
\par 10 Därför skall icke heller jag visa någon skonsamhet eller hava någon misskund, utan skall låta deras gärningar komma över deras huvuden."
\par 11 Och mannen som var klädd i linnekläderna och hade skrivtyget vid sin länd kom nu tillbaka och gav besked och sade: "Jag har gjort såsom du bjöd mig."

\chapter{10}

\par 1 Och jag fick se att på fästet, som vilade på kerubernas huvuden, fanns något som tycktes vara av safirsten, något som till utseendet liknade en tron; detta syntes ovanpå dem.
\par 2 Och han sade till mannen som var klädd i linnekläderna, han sade: "Gå in mellan rundlarna, in under keruben, och tag dina händer fulla med eldsglöd från platsen mellan keruberna, och strö ut dem över staden." Och jag såg honom gå.
\par 3 Och keruberna stodo till höger om huset, när mannen gick ditin, och molnet uppfyllde den inre förgården.
\par 4 Men HERRENS härlighet höjde sig upp från keruben och flyttade sig till husets tröskel; och huset uppfylldes då av molnet, och förgården blev full av glansen från HERRENS härlighet.
\par 5 Och dånet av kerubernas vingar hördes ända till den yttre förgården, likt Gud den Allsmäktiges röst, då han talar.
\par 6 Och när han nu bjöd mannen som var klädd i linnekläderna och sade: "Tag eld från platsen mellan rundlarna, inne mellan keruberna", då gick denne ditin och ställde sig bredvid ett av hjulen.
\par 7 Då räckte keruben där ut sin hand, mellan de andra keruberna, till elden som brann mellan keruberna, och tog därav och lade i händerna på honom som var klädd i linnekläderna; och denne tog det och gick så ut.
\par 8 Och under vingarna på keruberna så syntes något som var bildat såsom en människohand.
\par 9 Och jag fick se fyra hjul stå invid keruberna ett hjul invid var kerub och det såg ut som om hjulen voro av något som liknade krysolitsten.
\par 10 De sågo alla fyra likadana ut, och ett hjul tycktes vara insatt i ett annat.
\par 11 När de skulle gå, kunde de gå åt alla fyra sidorna, de behövde icke vända sig, när de gingo. Ty åt det håll dit den främste begav sig gingo de andra efter, utan att de behövde vända sig, när de gingo.
\par 12 Och hela deras kropp, deras rygg, deras händer och deras vingar, så ock hjulen, voro fulla med ögon runt omkring; de fyra hade nämligen var sitt hjul.
\par 13 Och jag hörde att hjulen kallades "rundlar".
\par 14 Och var och en hade fyra ansikten; det första ansiktet var en kerubs, det andra en människas, det tredje ett lejons, det fjärde en örns.
\par 15 Och keruberna höjde sig upp; de var samma väsenden som jag hade sett vid strömmen Kebar.
\par 16 Och när keruberna gingo, gingo ock hjulen invid dem; och när keruberna lyfte sina vingar för att höja sig över jorden, skilde sig hjulen icke ifrån dem.
\par 17 När de stodo stilla, stodo ock dessa stilla, och när de höjde sig, höjde sig ock dessa med dem, ty väsendenas ande var i dem.
\par 18 Och HERRENS härlighet flyttade I sig bort ifrån husets tröskel och stannade över keruberna.
\par 19 Då såg jag huru keruberna lyfte sina vingar och höjde sig från jorden, när de begåvo sig bort, och hjulen jämte dem; och de stannade vid ingången till östra porten på HERRENS hus, och Israels Guds härlighet vilade ovanpå dem.
\par 20 Det var samma väsenden som jag: hade sett under Israels Gud vid strömmen Kebar, och jag märkte att det var keruber.
\par 21 Var och en hade fyra ansikten och fyra vingar, och under deras vingar var något som liknade människohänder.
\par 22 Och deras ansikten voro likadana som de ansikten jag hade sett vid strömmen Kebar, så sågo de ut, och sådana voro de. Och de gingo alla rakt fram.

\chapter{11}

\par 1 Och en andekraft lyfte upp mig och förde mig till östra porten på HERRENS hus, den som vetter åt öster. Där fick jag se tjugufem män stå vid ingången till porten; och jag såg bland dem Jaasanja, Assurs son, och Pelatja, Benajas son, som voro furstar i folket.
\par 2 Och han sade till mig: "Du människobarn, det är dessa män som tänka ut vad fördärvligt är och råda till vad ont är, här i staden;
\par 3 det är de som säga: 'Hus byggas icke upp så snart. Här är grytan, och vi äro köttet.
\par 4 Profetera därför mot dem, ja, profetera, du människobarn."
\par 5 Då föll HERRENS Ande över mig, han sade till mig: "Säg: Så säger HERREN: Sådant sägen I, I av Israels hus, och edra hjärtans tankar känner jag väl.
\par 6 Många ligga genom eder slagna här i staden; I haven uppfyllt dess gator med slagna.
\par 7 Därför säger Herren, HERREN så: De slagna vilkas fall I haven vållat i staden, de äro köttet, och den är grytan; men eder själva skall man föra bort ur den.
\par 8 I frukten för svärd, och svärd skall jag ock låta komma över eder, säger Herren, HERREN.
\par 9 Jag skall föra eder bort härifrån och giva eder i främlingars hand; och jag skall hålla dom över eder.
\par 10 För svärd skolen I falla; vid Israels gräns skall jag döma eder. Och I skolen förnimma att jag är HERREN.
\par 11 Staden skall icke vara en gryta för eder, och I skolen icke vara köttet i den; nej, vid Israels gräns skall jag döma eder.
\par 12 Då skolen I förnimma att jag är HERREN, I som icke haven vandrat efter mina stadgar och icke haven gjort efter mina rätter, utan haven gjort efter de hednafolks rätter, som bo runt omkring eder."
\par 13 Medan jag så profeterade, hade Pelatja, Benajas son, uppgivit andan. Då föll jag ned på mitt ansikte och ropade med hög röst och sade: "Ack, Herre, HERRE, vill du då alldeles göra ände på kvarlevan av Israel?"
\par 14 Och HERRENS ord kom till mig; han sade:
\par 15 "Du människobarn, dina bröder, ja, dina bröder dina nära fränder och hela Israels hus, alla de till vilka Jerusalems invånare säga: 'Hållen eder borta från HERREN; det är åt oss som landet har blivit givet till besittning' -
\par 16 om dem skall du alltså säga: Så säger Herren, HERREN: Ja, väl har jag fört dem långt bort ibland folken och förstrött dem i länderna, och med nöd har jag varit för dem en helgedom i de länder dit de hava kommit;
\par 17 men därför skall du nu säga: Så säger Herren, HERREN: Jag skall församla eder ifrån folkslagen och hämta eder tillhopa från de länder dit I haven blivit förströdda, och skall giva eder Israels land.
\par 18 Och när de hava kommit dit, skola de skaffa bort därifrån alla de skändliga och styggeliga avgudar som nu finnas där.
\par 19 Och jag skall giva dem alla ett och samma hjärta, och en ny ande skall jag låta komma i deras bröst; jag skall taga bort stenhjärtat ur deras kropp och giva dem ett hjärta av kött,
\par 20 så att de vandra efter mina stadgar och hålla mina rätter och göra efter dem, och de skola vara mitt folk, och jag skall vara deras Gud.
\par 21 Men de vilkas hjärtan efterfölja de skändliga och styggeliga avgudarnas hjärtan, deras gärningar skall jag låta komma över deras huvuden, säger Herren, HERREN."
\par 22 Och keruberna, följda av hjulen, lyfte sina vingar, och Israels Guds härlighet vilade ovanpå dem.
\par 23 Och HERRENS härlighet höjde sig och lämnade staden och stannade på berget öster om staden.
\par 24 Men mig hade en andekraft lyft upp och fört bort till de fångna i Kaldeen, så hade skett i synen, genom Guds Ande. Sedan försvann för mig den syn jag hade fått se.
\par 25 Och jag talade till de fångna alla de ord som HERREN hade uppenbarat för mig.

\chapter{12}

\par 1 Och HERRENS ord kom till mig; han sade;
\par 2 "Du människobarn, du bor mitt i det gensträviga släktet, bland människor som hava ögon att se med, men dock icke se, och öron att höra med, men dock icke höra, eftersom de äro ett så gensträvigt släkte.
\par 3 Så red nu till åt dig, du människobarn, vad man behöver, när man skall gå i landsflykt. Och vandra i deras åsyn åstad på ljusa dagen, ja, vandra i deras åsyn åstad från det ställe där du nu bor bort till en annan ort - om de till äventyrs ville akta därpå, då de nu äro ett så gensträvigt släkte.
\par 4 För ut ditt bohag, på ljusa dagen och i deras åsyn, såsom skulle du gå i landsflykt, och vandra så i deras åsyn själv åstad på aftonen, såsom landsflyktiga pläga.
\par 5 Gör dig i deras åsyn en öppning i väggen, och för bohaget ut genom den.
\par 6 Lyft det sedan i deras åsyn upp på axeln och för bort det, när det har blivit alldeles mörkt; och betäck ditt ansikte, så att du icke ser landet. Ty jag gör dig till ett tecken för Israels hus."
\par 7 Och jag gjorde såsom han bjöd mig; på ljusa dagen förde jag ut mitt bohag, såsom skulle jag gå i landsflykt. Sedan, om aftonen, gjorde jag mig med handen en öppning i väggen, och när det hade blivit alldeles mörkt, förde jag det ut genom den och bar det så på axeln, i deras åsyn.
\par 8 Och HERRENS ord kom till mig den följande morgonen; han sade:
\par 9 Du människobarn, säkert har Israels hus, det gensträviga släktet, frågat dig: "Vad är det du gör?"
\par 10 Så svara dem nu: Så säger Herren, HERREN: Denna utsaga gäller fursten i Jerusalem och alla dem av Israels hus, som äro därinne.
\par 11 Säg: Jag är ett tecken för eder; såsom jag har gjort, så skall det gå dem: de skola vandra bort i landsflykt och fångenskap.
\par 12 Och fursten som de hava ibland sig skall lyfta upp sin börda på axeln, när det har blivit alldeles mörkt, och skall så draga ut. Man skall göra en öppning i väggen, så att han genom den kan bära ut sin börda; och han skall betäcka sitt ansikte, så att han icke ser landet med sitt öga.
\par 13 Och jag skall breda ut mitt nät över honom, och han skall bliva fångad i min snara; och jag skall föra honom till Babel i kaldéernas land, som han dock icke skall se; och där skall han dö.
\par 14 Och alla som äro omkring honom, till hans hjälp, och alla hans härskaror skall jag förströ åt alla väderstreck, och mitt svärd skall jag draga ut efter dem.
\par 15 Och de skola förnimma att jag är HERREN, när jag förskingrar dem bland folken och förströr dem i länderna.
\par 16 Men några få av dem skall jag låta bliva kvar efter svärd, hungersnöd och pest, för att de bland de folk till vilka de komma skola kunna förtälja om alla sina styggelser; och de skola förnimma att jag är HERREN.
\par 17 Och HERRENS ord kom till mig; han sade:
\par 18 Du människobarn, ät nu ditt bröd med bävan, och drick ditt vatten darrande och med oro.
\par 19 Och säg till folket i landet: Så säger Herren, HERREN om Jerusalems invånare i Israels land: De skola äta sitt bröd med oro och dricka sitt vatten med förfäran; så skall landet bliva ödelagt och plundrat på allt vad däri är, för den orätts skull som alla dess inbyggare hava övat.
\par 20 Och de städer som nu äro bebodda skola komma att ligga öde, och landet skall bliva en ödemark; och I skolen förnimma att jag är HERREN.
\par 21 Och HERRENS ord kom till mig; han sade:
\par 22 Du människobarn, vad är det för ett ordspråk I haven i Israels land, när I sägen: "Tiden går, och av alla profetsynerna bliver intet"?
\par 23 Säg nu till dem: Så säger Herren, HERREN: Jag skall göra slut på det ordspråket, så att man icke mer skall bruka det i Israel. Tala i stället så till dem: "Tiden kommer snart, med alla profetsynernas fullbordan."
\par 24 Ty inga falska profetsyner och inga lögnaktiga spådomar skola mer finnas i Israels hus;
\par 25 nej, jag, HERREN, skall tala det ord som jag vill tala, och det skall fullbordas, utan att länge fördröjas. Ja, du gensträviga släkte, i edra dagar skall jag tala ett ord och skall ock fullborda det, säger Herren, HERREN.
\par 26 Och HERRENS ord kom till mig; han sade:
\par 27 Du människobarn, se, Israels hus säger: "Den syn som han skådar gäller dagar som icke komma så snart; han profeterar om tider som ännu äro långt borta."
\par 28 Säg därför till dem: Så säger Herren, HERREN: Intet av vad jag har talat skall längre fördröjas; vad jag talar, det skall ske, säger Herren, HERREN.

\chapter{13}

\par 1 Och HERRENS ord kom till mig; han sade:
\par 2 Du människobarn, profetera mot Israels profeterande profeter; säg till dem som profetera efter sina egna hjärtans ingivelser: Hören HERRENS ord.
\par 3 Så säger Herren, HERREN: Ve eder, I dåraktiga profeter, som följen eder egen ande och syner som I icke haven sett! -
\par 4 Lika rävar på öde platser äro dina profeter, Israel.
\par 5 I haven icke trätt fram i gapet eller fört upp någon mur omkring Israels hus, så att det har kunnat bestå i striden på HERRENS dag.
\par 6 Nej, deras syner voro falskhet och deras spådomar lögn, fastän de sade "Så har HERREN sagt." HERREN hade ju icke sänt dem, men de hoppades att deras tal ändå skulle gå i fullbordan.
\par 7 Ja, förvisso var det falska syner som I skådaden och lögnaktiga spådomar som I uttaladen, fastän I saden: "Så har HERREN sagt." Jag hade ju icke talat något sådant.
\par 8 Därför säger Herren, HERREN så: Eftersom edert tal är falskhet och edra syner äro lögn, se, därför skall jag komma över eder, säger Herren, HERREN.
\par 9 Och min hand skall drabba profeterna som skåda falska syner och spå lögnaktiga spådomar. De skola icke få en plats i mitt folks församling och skola icke bliva upptagna i förteckningen på Israels hus, ej heller skola de få komma till Israels land; och I skolen förnimma att jag är Herren, HERREN.
\par 10 Eftersom, ja, eftersom de förde mitt folk vilse, i det att de sade: "Allt står väl till", och dock stod icke allt väl till, och eftersom de, när folket bygger upp en mur, vitmena den,
\par 11 därför må du säga till dessa vitmenare att den måste falla. Ett slagregn skall komma - ja, I skolen fara ned, I hagelstenar, och du skall bryta ned den, du stormvind!
\par 12 Och när så väggen faller, då skall man förvisso säga till eder: "Var är nu vitmeningen som I ströken på?"
\par 13 Därför säger Herren, HERREN så; Jag skall i min förtörnelse låta en stormvind bryta lös, ett slagregn skall komma genom min vrede, och hagelstenar genom min förtörnelse, så att det bliver en ände därpå.
\par 14 Och jag skall förstöra väggen som I beströken med vitmening, jag skall slå den till jorden, så att dess grundval bliver blottad. Och när den faller, skolen I förgås därinne; och I skolen förnimma att jag är HERREN.
\par 15 Och jag skall uttömma min förtörnelse på väggen och på dem som hava bestrukit den med vitmening; och så skall jag säga till eder: Det är ute med väggen, det är ute med dess vitmenare,
\par 16 med Israels profeter, som profeterade om Jerusalem och skådade syner, det till behag, om att allt stod väl till, och dock stod icke allt väl till, säger Herren, HERREN.
\par 17 Och du, människobarn, vänd ditt ansikte mot dina landsmaninnor som profetera efter sina egna hjärtans ingivelser; profetera mot dem
\par 18 och säg: Så säger Herren, HERREN: Ve eder som syn bindlar till alla handleder och gören slöjor till alla huvuden, både ungas och gamlas, för att så fånga själar! Skullen I få fånga själar bland mitt folk och döma somliga själar till liv, eder till vinning,
\par 19 I som för några nävar korn och några bitar bröd ohelgen mig hos mitt folk, därmed att I dömen till döden själar som icke skola dö, och dömen till liv själar som icke skola leva, i det att I ljugen för mitt folk, som gärna hör lögn?
\par 20 Nej, och därför säger Herren, HERREN så: Se, jag skall väl nå edra bindlar, i vilka I fången själarna såsom fåglar, och skall slita dem från edra armar; och jag skall giva själarna fria, de själar som I haven fångat såsom fåglar.
\par 21 Och jag skall slita sönder edra slöjor och rädda mitt folk ur eder hand, och de skola icke mer vara ett byte i eder hand; och I skolen förnimma att jag är HERREN.
\par 22 Eftersom I genom lögnaktigt tal haven gjort den rättfärdige försagd i hjärtat, honom som jag ingalunda ville plåga, men däremot haven styrkt den ogudaktiges mod, så att han icke vänder om från sin onda väg och räddar sitt liv,
\par 23 därför skolen I icke få fortsätta att skåda falska syner och att öva spådom; utan jag skall rädda mitt folk ur eder hand, och I skolen förnimma att jag är HERREN.

\chapter{14}

\par 1 Och några av de äldste i Israel kommo till mig och satte sig ned hos mig.
\par 2 Då kom HERRENS ord till mig; han sade:
\par 3 Du människobarn, dessa män hava låtit sina eländiga avgudar få insteg i sina hjärtan och hava ställt upp framför sig vad som är dem en stötesten till missgärning. Skulle jag väl låta fråga mig av sådana?
\par 4 Nej; tala därför med dem och säg till dem: Så säger Herren, HERREN: Var och en av Israels hus, som låter sina eländiga avgudar få insteg i sitt hjärta och ställer upp framför sig vad som är honom en stötesten till missgärning, och så kommer till profeten, honom skall jag, HERREN, giva svar såsom han har förtjänat genom sina många eländiga avgudar.
\par 5 Så skall jag gripa Israels barn i hjärtat, därför att de allasammans hava vikit bort ifrån mig genom sina eländiga avgudar.
\par 6 Säg därför till Israels hus: Så säger Herren, HERREN: Vänden om, ja, vänden eder bort ifrån edra eländiga avgudar, vänden edra ansikten bort ifrån alla edra styggelser.
\par 7 Ty om någon av Israels hus, eller av främlingarna som bo i Israel, viker bort ifrån mig, och låter sina eländiga avgudar få insteg i sitt hjärta och ställer upp framför sig vad som är honom en stötesten till missgärning, och så kommer till profeten, för att denne skall fråga mig för honom, så vill jag, HERREN, själv giva honom svar:
\par 8 jag skall vända mitt ansikte mot den mannen och göra honom till ett tecken och till ett ordspråk, och utrota honom ur mitt folk; och I skolen förnimma att jag är HERREN.
\par 9 Men om profeten låter förföra sig och talar något ord, så har jag, HERREN, låtit den profeten bliva förförd; och jag skall uträcka min hand mot honom och förgöra honom ur mitt folk Israel.
\par 10 Och de skola båda bära på sin missgärning: profetens missgärning skall räknas lika med den frågandes missgärning -
\par 11 på det att Israels barn icke mer må gå bort ifrån mig och fara vilse, ej heller mer orena sig med alla sina överträdelser, utan vara mitt folk, såsom jag skall vara deras Gud, säger Herren, HERREN.
\par 12 Och HERRENS ord kom till mig; han sade:
\par 13 Du människobarn, om ett land syndade mot mig och beginge otrohet, så att jag måste uträcka min hand mot det och fördärva dess livsuppehälle och sända hungersnöd över det och utrota därur både människor och djur,
\par 14 och om då därinne funnes dessa tre män: Noa, Daniel och Job, så skulle de genom sin rättfärdighet rädda allenast sina egna liv, säger Herren, HERREN.
\par 15 Om jag läte vilddjur draga fram genom landet och göra det folktomt, så att det bleve så öde att ingen vågade draga där fram för djuren skull,
\par 16 då skulle, så sant jag lever, säger Herren, HERREN, dessa tre män, om de vore därinne, icke kunna rädda vare sig söner eller döttrar; allenast de själva skulle räddas, men landet måste bliva öde.
\par 17 Eller om jag läte svärd komma över det landet, i det att jag sade: "Svärdet fare fram genom landet!", och jag så utrotade därur både människor och djur,
\par 18 och om då dessa tre män vore därinne, så skulle de, så sant jag lever, säger Herren, HERREN, icke kunna rädda söner eller döttrar; allenast de själva skulle räddas.
\par 19 Eller om jag sände pest i det landet och utgöte min vrede däröver i blod, för att utrota därur både människor och djur,
\par 20 och om då Noa, Daniel och Job vore därinne, så skulle de, så sant jag lever, säger Herren, HERREN, icke kunna rädda vare sig son eller dotter; de skulle genom sin rättfärdighet rädda allenast sina egna liv.
\par 21 Och så säger Herren, HERREN: Men huru mycket värre bliver det icke, när jag på en gång sänder mina fyra svåra straffdomar: svärd, hungersnöd, vilddjur och pest, över Jerusalem, för att utrota därur både människor och djur!
\par 22 Likväl skola några räddade bliva kvar där, några söner och döttrar, som skola föras bort. Och se, dessa skola draga bort till eder; och när I fån se deras vandel och deras gärningar, då skolen I trösta eder för den olycka som jag har låtit komma över Jerusalem, ja, för allt som jag har låtit komma över det.
\par 23 De skola vara eder till tröst, när I sen deras vandel och deras gärningar; I skolen då förstå att jag icke utan sak har gjort allt vad jag har gjort mot det, säger Herren, HERREN.

\chapter{15}

\par 1 Och HERRENS ord kom till mig; han sade:
\par 2 Du människobarn, varutinnan är vinstockens trä förmer än annat trä, vinstockens, vars rankor växa upp bland skogens andra träd?
\par 3 Tager man väl virke därav till att förfärdiga något nyttigt? Gör man ens därav en plugg för att på den hänga upp någonting?
\par 4 Och om det nu därtill har varit livet till mat åt elden, så att dess båda ändar hava blivit förtärda av eld, och vad däremellan finnes är svett, duger det då till något nyttigt?
\par 5 Icke ens medan det ännu var oskadat, kunde man förfärdiga något nyttigt därav; huru mycket mindre kan man förfärdiga något nyttigt därav, sedan det endels har blivit förtärt av elden och endels är svett!
\par 6 Därför säger Herren, HERREN så: Såsom det händer med vinstockens trä bland annat trä från skogen, att Jag lämnar det till mat åt elden, så skall jag ock göra med Jerusalems invånare.
\par 7 Jag skall vända mitt ansikte mot dem; ur elden hava de kommit undan, men eld skall dock förtära dem. Och I skolen förnimma att jag är HERREN, när jag vänder mitt ansikte mot dem.
\par 8 Och jag skall göra landet till en ödemark, därför att de hava varit otrogna, säger Herren, HERREN.

\chapter{16}

\par 1 Och HERRENS ord kom till mig; han sade:
\par 2 Du människobarn, förehåll Jerusalem dess styggelser
\par 3 och säg: Så säger Herren, HERREN till Jerusalem: Från Kanaans land stammar du, och där är du född; din fader var en amoré och din moder en hetitisk kvinna.
\par 4 Och vid din födelse gick det så till. När du föddes, skar ingen av din navelsträng, och du blev icke rentvagen med vatten, ej heller ingniden med salt och lindad.
\par 5 Ingen såg på dig med så mycken ömkan, att han villa göra något sådant med dig eller visa dig någon misskund, utan man kastade ut dig på öppna fältet den dag du föddes; så ringa aktade man ditt liv.
\par 6 Då gick jag förbi där du låg och fick se dig sprattla i ditt blod, och jag sade till dig: "Du skall få bliva vid liv, du som ligger där i ditt blod." Ja, jag sade till dig: "Du skall få bliva vid liv, du som ligger där i ditt blod;
\par 7 ja, jag skall föröka dig till många tusen, såsom växterna äro på marken." Och du sköt upp och blev stor och mycket fager; dina bröst hade höjt sig, och ditt hår hade växt, men du var ännu naken och blottad.
\par 8 Då gick jag åter förbi där du var och fick se att din tid var inne, din älskogstid; och jag bredde min mantel över dig och betäckte din blygd. Och så gav jag dig min ed och ingick förbund med dig, säger Herren, HERREN, och du blev min.
\par 9 Och jag tvådde dig med vatten och sköljde blodet av dig, och smorde dig med olja,
\par 10 och klädde på dig brokigt vävda kläder och satte på dig skor av tahasskinn och en huvudbindel av fint linne och en slöja av silke.
\par 11 Och jag prydde dig med smycken: jag satte armband på dina armar och en kedja om din hals,
\par 12 jag satte en ring i din näsa och örhängen i dina öron och en härlig krona på ditt huvud.
\par 13 Så blev du prydd med guld och silver, och dina kläder voro av fint linne, av siden och av tyg i brokig vävnad. Fint mjöl, honung och olja fick du äta. Du blev övermåttan skön, och så vart du omsider en drottning.
\par 14 Och ryktet om dig gick ut bland folken för din skönhets skull, ty den var fullkomlig genom de härliga prydnader som jag hade satt på dig, säger Herren, HERREN.
\par 15 Men du förlitade dig på din skönhet och bedrev otukt, sedan du nu hade fått sådant rykte; du slösade din otukt på var och en som gick där fram: det vore ju något för honom.
\par 16 Och du tog dina kläder och gjorde dig med dem brokiga offerhöjder och bedrev på dessa otukt, sådana gärningar som eljest aldrig någonsin hava förekommit, ej heller mer skola göras.
\par 17 Och du tog dina härliga smycken, det guld och silver som jag hade givit dig, och gjorde dig så mansbilder, med vilka du bedrev otukt.
\par 18 Och du tog dina brokigt vävda kläder och höljde dem i dessa; och min olja och min rökelse satte du fram för dem.
\par 19 Och det bröd som jag hade givit dig - ty fint mjöl, olja och honung hade jag ju låtit dig få att äta - detta satte du fram för dem till en välbehaglig lukt; ja, därhän kom det, säger Herren, HERREN.
\par 20 Och du tog dina söner och döttrar, dem som du hade fött åt mig, och offrade dessa åt dem till spis. Var det då icke nog att du bedrev otukt?
\par 21 Skulle du också slakta mina söner och giva dem till pris såsom offer åt dessa?
\par 22 Och vid alla dina styggelser och din otukt tänkte du icke på din ungdoms dagar, då du var naken och blottad och låg där sprattlande i ditt blod.
\par 23 Och sedan du hade bedrivit all denna ondska - ve, ve dig! säger Herren, HERREN -
\par 24 byggde du dig kummel och gjorde dig höjdaltaren på alla öppna platser.
\par 25 I alla gathörn byggde du dig höjd altaren och lät din skönhet skända och spärrade ut benen åt alla som gingo där fram; ja, du bedrev mycken otukt.
\par 26 Du bedrev otukt med egyptierna, dina grannar med det stora köttet, ja, mycken otukt till att förtörna mig.
\par 27 Men se, då uträckte jag min hand mot dig och minskade ditt underhåll och gav dig till pris åt dina fiender, filistéernas döttrar, som blygdes över ditt skändliga väsende.
\par 28 Men sedan bedrev du otukt med assyrierna, ty du hade ännu icke blivit mätt; ja, du bedrev otukt med dem och blev ändå icke mätt.
\par 29 Du gick med din otukt ända bort till krämarlandet, kaldéernas land; men icke ens så blev du mätt.
\par 30 Huru älskogskrankt var icke ditt hjärta, säger Herren, HERREN, eftersom du gjorde allt detta, sådana gärningar som allenast den fräckaste sköka kan göra.
\par 31 Med dina döttrar uppförde du åt dig kummel i alla gathörn och höjdaltaren på alla öppna platser. Men däri var du olik andra skökor, att du försmådde skökolön,
\par 32 du äktenskapsbryterska, som i stället för den man du hade tog andra män till dig.
\par 33 Åt alla andra skökor måste man giva skänker, men här var det du som gav skänker åt alla dina älskare och mutade dem, för att de skulle komma till dig från alla håll och bedriva otukt med dig.
\par 34 Så gjorde du vid din otukt tvärt emot vad andra kvinnor göra; efter dig lopp ingen för att bedriva otukt, men du gav skökolön, utan att själv få någon skökolön; du gjorde tvärt emot andra.
\par 35 Hör därför HERRENS ord, du sköka.
\par 36 Så säger Herren, HERREN: Eftersom du har varit så frikostig med din skam och blottat din blygd i otukt med din älskare, därför, och för alla dina vederstyggliga eländiga avgudars skull och för dina söners blods skull, dina söners, som du gav åt dessa,
\par 37 se, därför skall jag församla alla dina älskare, dem som du har varit till behag, ja, alla dem som du har älskat mer eller mindre; dem skall jag församla mot dig från alla håll och blotta din blygd inför dem, så att de få se all din blygd.
\par 38 Och jag skall döma dig efter den lag som gäller för äktenskapsbryterskor och blodsutgjuterskor, och skall låta dig bliva ett blodigt offer för min vrede och nitälskan.
\par 39 Och jag skall giva dig i deras hand, och de skola slå ned dina kummel och bryta ned dina höjdaltaren, och slita av dig kläderna och taga ifrån dig dina härliga smycken och låta dig ligga naken och blottad.
\par 40 Och de skola sammankalla en församling mot dig, och man skall stena dig och hugga sönder dig med svärd;
\par 41 och dina hus skall man bränna upp i eld. Så skall man hålla dom över dig inför många kvinnors ögon. Och så skall jag göra slut på din otukt, och du skall icke mer kunna giva någon skökolön.
\par 42 Och jag skall släcka min vrede på dig, så att min nitälskan kan vika ifrån dig, och så att jag får ro och slipper att mer förtörnas.
\par 43 Eftersom du icke tänkte på din ungdoms dagar, utan var avog mot mig i allt detta, se, därför skall också jag låta dina gärningar komma över ditt huvud, säger Herren, HERREN, på det att du icke mer må lägga sådan skändlighet till alla dina andra styggelser.
\par 44 Se, alla som bruka ordspråk skola på dig tillämpa det ordspråket: "Sådan moder, sådan dotter."
\par 45 Ja, du är din moders dotter, hennes som övergav sin man och sina barn; du är dina systrars syster deras som övergåvo sina män och sina barn; eder moder var en hetitisk kvinna och eder fader en amoré.
\par 46 Din större syster var Samaria med sina döttrar, hon som bodde norrut från dig; och din mindre syster, som bodde söderut från dig, var Sodom med sina döttrar.
\par 47 Men du nöjde dig icke med att vandra på deras vägar och att göra efter deras styggelser; inom kort bedrev du värre ting än de, på alla dina vägar.
\par 48 Så sant jag lever, säger Herren, HERREN: din syster Sodom och hennes döttrar hava icke gjort vad du och dina döttrar haven gjort.
\par 49 Se, detta var din syster Sodoms missgärning: fastän höghet, överflöd och tryggad ro hade blivit henne och hennes döttrar beskärd, understödde hon likväl icke den arme och fattige.
\par 50 Tvärtom blevo de högfärdiga och bedrevo vad styggeligt var inför mig; därför försköt jag dem, när jag såg detta.
\par 51 Ej heller Samaria har syndat hälften så mycket som du. Du har gjort så många flera styggelser än dessa, att du genom alla de styggelser du har bedrivit har kommit dina systrar att synas rättfärdiga.
\par 52 Så må också du nu bara din skam, du som nu kan lända dina systrar till ursäkt; ty därigenom att du har bedrivit ännu vederstyggligare synder än de, stå nu såsom rättfärdiga i jämförelse med dig. Ja, blygs och bär din skam över att du så har kommit dina systrar att synas rättfärdiga.
\par 53 Därför skall jag ock åter upp rätta dem, Sodom med hennes döttrar och Samaria med hennes döttrar. Dig skall jag ock åter upprätta mitt ibland dem,
\par 54 för att du må bära din skam och skämmas för allt vad du har gjort, och därmed bliva dem till tröst.
\par 55 Och med dina systrar skall så ske: Sodom och hennes döttrar skola åter bliva vad de fordom voro, och Samaria och hennes döttrar skola åter bliva vad de fordom voro Också du själv och dina döttrar skolen åter bliva vad I fordom voren.
\par 56 Men om du förr icke ens hördes nämna din syster Sodom, under din höghetstid,
\par 57 innan ännu din egen ondska hade blivit uppenbarad - såsom den blev på den tid då du vart till smälek för Arams döttrar och för alla de kringboende filistéernas döttrar, som hånade dig på alla sidor -
\par 58 Så måste du nu själv bära på din skändlighet och dina styggelser, säger HERREN.
\par 59 Ty så säger Herren, HERREN: Jag har handlat med dig efter dina gärningar, ty du hade ju föraktat eden och brutit förbundet.
\par 60 Men jag vill nu tänka på det förbund som jag slöt med dig i din ungdoms dagar, och upprätta med dig ett evigt förbund.
\par 61 Då skall du tänka tillbaka på dina vägar och skämmas, när du får taga till dig dina systrar, de större jämte de mindre; ty jag skall giva dem åt dig till döttrar, dock icke för din trohet i förbundet.
\par 62 Men jag skall upprätta mitt förbund med dig, och du skall förnimma att jag är HERREN;
\par 63 och så skall du tänka därpå och blygas, så att du av skam icke mer kan upplåta din mun, då när jag förlåter dig allt vad du har gjort, säger Herren, HERREN.

\chapter{17}

\par 1 Och HERRENS ord kom till mig; han sade:
\par 2 Du människobarn, förelägg Israels hus en gåta, och tala till det en liknelse;
\par 3 säg: Så säger Herren, HERREN: Den stora örnen med de stora vingarna och de långa pennorna, han som är så full med brokiga fjädrar, han kom till Libanon och tog bort toppen på cedern.
\par 4 Han bröt av dess översta kvist och förde den till krämarlandet och satte den i en köpmansstad.
\par 5 Sedan tog han en telning som växte i landet och planterade den i fruktbar jordmån; han tog den och satte den bland pilträd, på ett ställe där mycket vatten fanns.
\par 6 Och den fick växa upp och bliva ett utgrenat vinträd, dock med låg stam, för att dess rankor skulle vända sig till honom och dess rötter vara under honom. Den blev alltså ett vinträd som bar grenar och sköt skott.
\par 7 Men där var ock en annan stor örn med stora vingar och fjädrar i mängd; och se, till denne böjde nu vinträdet längtansfullt sina grenar, och från platsen där det var planterat sträckte det sina rankor mot honom, för att han skulle vattna det.
\par 8 Och dock var det planterat i god jordmån, på ett ställe där mycket vatten fanns, så att det kunde få grenar och bära frukt och bliva ett härligt vinträd.
\par 9 Säg vidare: Så säger Herren, HERREN: Kan det nu gå det väl? Skall man icke rycka upp dess rötter och riva av dess frukt, så att det förtorkar, och så att alla blad som hava vuxit ut därpå förtorka? Och sedan skall varken stor kraft eller mycket folk behövas för att flytta det bort ifrån dess rötter.
\par 10 Visst står det fast planterat, men kan det gå det väl? Skall det icke alldeles förtorka, när östanvinden når det, ja, förtorka på den plats där det har vuxit upp?
\par 11 Och HERRENS ord kom till mig, han sade:
\par 12 Säg till det gensträviga släktet Förstån I icke vad detta betyder? Så säg då: Se, konungen i Babel kom till Jerusalem och tog dess konung och dess furstar och hämtade dem till sig i Babel.
\par 13 Och han tog en ättling av konungahuset och slöt förbund med honom och lät honom anlägga ed. Men de mäktige i landet hade han fört bort med sig,
\par 14 för att landet skulle bliva ett oansenligt rike, som icke kunde uppresa sig, och som skulle nödgas hålla förbundet med honom, om det ville bestå.
\par 15 Men han avföll från honom och skickade sina sändebud till Egypten, för att man där skulle giva honom hästar och mycket folk. Kan det gå den väl, som så gör? Kan han undkomma? Kan den som bryter förbund undkomma?
\par 16 Så sant jag lever, säger Herren, HERREN: där den konung bor, som gjorde honom till konung, den vilkens ed han likväl föraktade, och vilkens förbund han bröt, där, hos honom i Babel, skall han sannerligen dö.
\par 17 Och Farao skall icke med stor härsmakt och mycket folk bistå honom i kriget, när en vall kastas upp och en belägringsmur bygges, till undergång för många människor.
\par 18 Eftersom han föraktade eden och bröt förbundet och gjorde allt detta fastän han hade givit sitt löfte, därför skall han icke undkomma.
\par 19 Ja, därför säger Herren, HERREN så: Så sant jag lever, jag skall förvisso låta min ed, som han har föraktat, och mitt förbund, som han har brutit, komma över hans huvud.
\par 20 Och jag skall breda ut mitt nät över honom, och han skall bliva fångad i min snara; och jag skall föra honom till Babel och där hålla dom över honom, för den otrohets skull som han har begått mot mig.
\par 21 Och alla flyktingar ur alla hans härskaror skola falla för svärd, och om några bliva räddade, så skola de varda förströdda åt alla väderstreck. Och I skolen förnimma att jag, HERREN, har talat.
\par 22 Så säger Herren, HERREN: Jag vill ock själv taga en kvist av toppen på den höga cedern och sätta den; av dess översta skott skall jag avbryta en späd kvist och själv plantera den på ett högt och brant berg.
\par 23 På Israels stolta berg skall jag plantera den, och den skall bära grenar och få frukt och bliva en härlig ceder. Och allt vad fåglar heter av alla slag skall bo under den; de skola bo i skuggan av dess grenar.
\par 24 Och alla träd på marken skola förnimma att det är jag, HERREN, som förödmjukar höga träd och upphöjer låga träd, som låter friska träd förtorka och gör torra träd grönskande. Jag, HERREN, har talat det, och jag fullbordar det också.

\chapter{18}

\par 1 Och HERRENS ord kom till mig; han sade:
\par 2 Vad orsak haven I till att bruka detta ordspråk i Israels land: "Fäderna äta sura druvor, och barnens tänder bliva ömma därav"?
\par 3 Så sant jag lever, säger Herren, HERREN, I skolen ingen orsak mer hava att bruka detta ordspråk i Israel.
\par 4 Se, alla själar äro mina, faderns själ såväl som sonens är min; den som syndar, han skall dö.
\par 5 Om nu en man är rättfärdig och övar rätt och rättfärdighet,
\par 6 om han icke håller offermåltid på bergen, ej heller upplyfter sina ögon till Israels hus' eländiga avgudar, om han icke skändar sin nästas hustru, ej heller kommer vid en kvinna under hennes orenhets tid,
\par 7 om han icke förtrycker någon, utan giver tillbaka den pant han har fått för skuld, om han icke tager rov, utan giver sitt bröd åt den hungrige och kläder den nakne,
\par 8 om han icke ockrar eller tager ränta, om han håller sin hand tillbaka från vad orätt är och fäller rätta domar människor emellan -
\par 9 ja, om han så vandrar efter mina stadgar och håller mina rätter, i det att han gör vad redligt är, då är han rättfärdig och skall förvisso få leva, säger Herren, HERREN.
\par 10 Men om han så föder en son som bliver en våldsverkare, vilken utgjuter blod eller gör allenast något av allt detta
\par 11 som han själv icke gjorde, en som håller offermåltid på bergen, skändar sin nästas hustru,
\par 12 förtrycker den arme och fattige, tager rov, icke giver pant tillbaka, upplyfter sina ögon till de eländiga avgudarna, bedriver vad styggeligt är,
\par 13 ockrar och tager ränta - skulle då denne få leva? Nej, han skall icke få leva, utan eftersom han bedriver sådana styggelser, skall han straffas med döden; hans blod skall komma över honom.
\par 14 Och om sedan denne föder en son, vilken ser alla de synder som hans fader begår, och vid åsynen av dem själv tager sig till vara för att göra sådant,
\par 15 en som icke håller offermåltid på bergen, icke upplyfter sina ögon till Israels hus' eländiga avgudar, icke skändar sin nästas hustru,
\par 16 en som icke förtrycker någon, icke fordrar pant eller tager rov, utan giver sitt bröd åt den hungrige och kläder den nakne,
\par 17 en som icke förgriper sig på den arme, ej heller ockrar eller tager ränta, utan gör efter mina rätter och vandrar efter mina stadgar, då skall denne icke dö genom sin faders missgärning, utan skall förvisso få leva.
\par 18 Hans fader däremot, som begick våldsgärningar och rövade från sin broder och gjorde bland sina fränder det som icke var gott, se, han måste dö genom sin missgärning.
\par 19 Huru kunnen I nu fråga: "Varför skulle icke sonen bära på sin faders missgärning?" Jo, sonen övade ju rätt och rättfärdighet och höll alla mina stadgar och gjorde efter dem; därför skall han förvisso få leva.
\par 20 Den som syndar, han skall dö; en son skall icke bära på sin faders missgärning, och en fader skall icke bära på sin sons missgärning. Över den rättfärdige skall hans rättfärdighet komma, och över den ogudaktige skall hans ogudaktighet komma.
\par 21 Men om den ogudaktige omvänder sig från alla de synder som han har begått, och håller alla mina stadgar och övar rätt och rättfärdighet, då skall han förvisso leva och icke dö.
\par 22 Ingen av de överträdelser han har begått skall du tillräknas honom; genom den rättfärdighet han har övat skall han få leva.
\par 23 Menar du att jag har lust till den ogudaktiges död, säger Herren, HERREN, och icke fastmer därtill att han vänder om från sin väg och får leva?
\par 24 Men om den rättfärdige vänder om från sin rättfärdighet och gör vad orätt är, alla sådana styggelser som den ogudaktige gör - skulle han då få leva, om han gör så? Nej, intet av all den rättfärdighet han har övat skall då ihågkommas, utan genom den otrohet han har begått och den synd han har övat skall han dö.
\par 25 Men nu sägen I: "Herrens väg är icke alltid densamma." Hören då, I av Israels hus: Skulle verkligen min väg icke alltid vara densamma? Är det icke fastmer eder väg som icke alltid är densamma?
\par 26 Om den rättfärdige vänder om från sin rättfärdighet och gör vad orätt är, så måste han dö till straff därför; genom det orätta som han gör måste han dö.
\par 27 Men om den ogudaktige vänder om från den ogudaktighet som han har övat, och i stället övar rätt och rättfärdighet, då får han behålla sin själ vid liv.
\par 28 Ja, eftersom han kom till insikt och vände om från alla de överträdelser han hade begått, skall han förvisso leva och icke dö.
\par 29 Och ändå säga de av Israels hus: "Herrens väg är icke alltid densamma"! Skulle verkligen mina vägar icke alltid vara desamma, I av Israels hus? Är det icke fastmer eder väg som icke alltid är densamma?
\par 30 Alltså: jag skall döma var och en av eder efter hans vägar, I av Israels hus, säger Herren, HERREN. Vänden om, ja, vänden eder bort ifrån alla edra överträdelser, för att eder missgärning icke må bliva eder till en stötesten.
\par 31 Kasten bort ifrån eder alla de överträdelser som I haven begått, och skaffen eder ett nytt hjärta och en ny ande; ty icke viljen I väl dö, I av Israels hus?
\par 32 Jag har ju ingen lust till någons död, säger Herren HERREN. Omvänden eder därför, så fån I leva.

\chapter{19}

\par 1 Men du, stäm upp en klagosång över Israels furstar;
\par 2 säg: Huru var icke din moder en lejoninna! Bland lejon låg hon; hon födde upp sina ungar bland kraftiga lejon.
\par 3 Så födde hon upp en av sina ungar, så att han blev ett kraftigt lejon; han lärde sig att taga rov, människor åt han upp.
\par 4 Men folken fingo höra om honom och han blev fångad i deras grop; och man förde honom med krok i nosen till Egyptens land.
\par 5 När hon nu såg att hon fick vänta förgäves, och att hennes hopp blev om intet, då tog hon en annan av sina ungar och gjorde denne till ett kraftigt lejon.
\par 6 Stolt gick han omkring bland lejonen, ja, han blev ett kraftigt lejon; han lärde sig att taga rov, människor åt han upp.
\par 7 Han våldförde deras änkor, deras städer förödde han. Och landet med vad däri var blev förfärat vid dånet av hans rytande.
\par 8 Då bådade man upp folk mot honom runt omkring från länderna; och de bredde ut sitt nät för honom, och han blev fångad i deras grop
\par 9 Sedan satte de honom i en bur, med krok i nosen, och förde honom till konungen Babel Där satte man honom in i fasta borgar, för att hans röst ej mer skulle höras bort till Israels berg.
\par 10 Medan de levde i ro, var din moder såsom ett vinträd, planterat vid vatten. Och det blev ett fruktsamt träd, rikt på skott, genom det myckna vattnet.
\par 11 Det fick starka grenar, tjänliga till härskarspiror, och dess stam växte hög, omgiven av lövverk, så att det syntes vida, ty det var högt och rikt på rankor.
\par 12 Då ryckte man upp det i vrede, och det blev kastat på jorden, och stormen från öster förtorkade dess frukt. Dess starka grenar brötos av och torkade bort, elden fick förtära dem.
\par 13 Nu är det utplanterat i öknen, i ett torrt och törstande land.
\par 14 Och eld har gått ut från dess yppersta gren och har förtärt dess frukt. Så finnes där nu ingen stark gren kvar, ingen härskarspira! En klagosång är detta, och den har fått tjäna såsom klagosång.

\chapter{20}

\par 1 I sjunde året, på tionde dagen i femte månaden, kommo några av de äldste i Israel för att fråga HERREN; och de satte sig ned hos mig.
\par 2 Då kom HERRENS ord till mig han sade:
\par 3 Du människobarn, tala med de äldste i Israel och säg till dem: Så säger Herren, HERREN: Haven I kommit för att fråga mig? Så sant jag lever, jag låter icke fråga mig av eder, säger Herren, HERREN.
\par 4 Men vill du döma dem, ja, vill du döma, du människobarn, så förehåll dem deras fäders styggelser
\par 5 och säg till dem: Så säger Herren, HERREN: På den dag då jag utvalde Israel, då upplyfte jag min hand till ed inför Jakobs hus' barn och gjorde mig känd för dem i Egyptens land; jag upplyfte min hand till ed inför dem och sade: "Jag är HERREN, eder Gud.
\par 6 På den dagen lovade jag dem med upplyft hand att föra dem ut ur Egyptens land, till det land som jag hade utsett åt dem, ett land som skulle flyta av mjölk och honung, och som vore härligast bland alla länder.
\par 7 Och jag sade till dem: "Var och en av eder kaste bort sina ögons styggelser, och ingen orene sig på Egyptens eländiga avgudar; jag är HERREN, eder Gud."
\par 8 Men de voro gensträviga mot mig och ville icke höra på mig; de kastade icke bort var och en sina ögons styggelser, och de övergåvo icke Egyptens eländiga avgudar. Då tänkte jag på att utgjuta min förtörnelse över dem och att uttömma min vrede på dem mitt i Egyptens land.
\par 9 Men vad jag gjorde, det gjorde jag för mitt namns skull, för att detta icke skulle bliva vanärat i de folks ögon, bland vilka de levde, och i vilkas åsyn jag gjorde mig känd för dem, i det jag förde dem ut ur Egyptens land.
\par 10 Så förde jag dem då ut ur Egyptens land och lät dem komma in i öknen.
\par 11 Och jag gav dem mina stadgar och kungjorde för dem mina rätter; den människa som gör efter dem får leva genom dem.
\par 12 Jag gav dem ock mina sabbater, till att vara ett tecken mellan mig och dem, för att man skulle veta att jag är HERREN, som helgar dem.
\par 13 Men Israels hus var gensträvigt mot mig i öknen; de vandrade icke efter mina stadgar, utan föraktade mina rätter, fastän den människa som gör efter dem får leva genom dem; de ohelgade ock svårt mina sabbater. Då tänkte jag på att utgjuta min förtörnelse över dem i öknen och så förgöra dem.
\par 14 Men vad jag gjorde, det gjorde jag för mitt namns skull, för att detta icke skulle bliva vanärat i de folks ögon, i vilkas åsyn jag hade fört dem ut.
\par 15 Likväl upplyfte jag min hand inför dem i öknen och svor att jag icke skulle låta dem komma in i det land som jag hade givit dem, ett land som skulle flyta av mjölk och honung, och som vore härligast bland alla länder -
\par 16 detta därför att de föraktade mina rätter och icke vandrade efter mina stadgar, utan ohelgade mina sabbater, i det att deras hjärtan följde efter deras eländiga avgudar.
\par 17 Men jag visade dem skonsamhet och fördärvade dem icke; jag gjorde icke alldeles ände på dem i öknen.
\par 18 Och jag sade till deras barn i öknen: "I skolen icke vandra efter edra fäders stadgar och icke hålla deras rätter, ej heller orena eder på deras eländiga avgudar.
\par 19 Jag är HERREN, eder Gud; vandren efter mina stadgar och håller mina rätter och gören efter dem.
\par 20 Och helgen mina sabbater, och må de vara ett tecken mellan mig och eder, för att man må veta att jag är HERREN, eder Gud.
\par 21 Men deras barn voro gensträviga mot mig; de vandrade icke efter mina stadgar och höllo icke mina rätter, så att de gjorde efter dem fastän den människa som gör efter dem får leva genom dem; de ohelgade ock mina sabbater. Då tänkte jag på att utgjuta min förtörnelse över dem och att uttömma min vrede på dem i öknen.
\par 22 Men jag drog min hand tillbaka, och vad jag gjorde, det gjorde jag för mitt namns skull, för att detta icke skulle bliva vanärat i de folks ögon, i vilkas åsyn jag hade fört dem ut.
\par 23 Likväl upplyfte jag min hand inför dem i öknen och svor att förskingra dem bland folken och förströ dem i länderna,
\par 24 eftersom de icke gjorde efter mina rätter, utan föraktade mina stadgar och ohelgade mina sabbater, och eftersom deras ögon hängde vid deras fäders eländiga avgudar.
\par 25 Därför gav jag dem ock stadgar som icke voro till deras båtnad, och rätter genom vilka de icke kunde bliva vid liv.
\par 26 Och jag lät dem orena sig med sina offerskänker, med att låta allt som öppnade moderlivet gå genom eld, ty jag ville slå dem med förfäran, på det att de skulle förstå att jag är HERREN.
\par 27 Tala därför till Israels hus, du människobarn, och säg till dem: Så säger Herren, HERREN: Också därmed hava edra fäder hädat mig, att de hava begått otrohet mot mig.
\par 28 När jag hade låtit dem komma in i det land som jag med upplyft hand hade lovat att giva dem, och när de så där fingo se någon hög kulle eller något lummigt träd, då offrade de där sina slaktoffer och framburo där sina offergåvor, mig till förtörnelse, och läto där sina offers välbehagliga lukt uppstiga och utgöto där sina drickoffer..
\par 29 Då sade jag till dem: "Vad är detta för en offerhöjd, denna som I kommen till?" Därav fick en sådan plats namnet "offerhöjd", såsom man säger ännu i dag.
\par 30 Säg därför till Israels hus: Så säger Herren, HERREN: Skolen då I orena eder på samma sätt som edra fäder gjorde, och i trolös avfällighet löpa efter deras styggelser?
\par 31 I orenen eder ännu i dag på alla edra eländiga avgudar, i det att I frambären åt dem edra offerskänker och låten edra barn gå genom eld. Skulle jag då låta fråga mig av eder, I av Israels hus? Nej, så sant jag lever, säger Herren, HERREN, ja låter icke fråga mig av eder.
\par 32 Och förvisso skall icke det få ske som har kommit eder i sinnet, då I tänken: "Vi vilja bliva såsom hedningarna, såsom folken i andra länder: vi vilja tjäna trä och sten.
\par 33 Så sant jag lever, säger Herren, HERREN, med stark hand och uträckt arm och utgjuten förtörnelse skall jag sannerligen regera över eder.
\par 34 Och med stark hand och uträckt arm och utgjuten förtörnelse skall jag föra eder ut ifrån folken och församla eder från de länder i vilka I ären förströdda..
\par 35 Och jag skall föra eder in i Folkens öken, och där skall jag gå till rätta med eder, ansikte mot ansikte.
\par 36 Likasom jag gick till rätta med edra fäder i öknen vid Egyptens land, så skall jag ock gå till rätta med eder, säger Herren, HERREN.
\par 37 Och jag skall låta eder draga fram under staven och tvinga eder in i förbundets band.
\par 38 Och jag skall rensa bort ifrån eder dem som sätta sig upp emot mig och avfalla från mig, jag skall skaffa bort dem ur det land där de nu bo, men in i Israels land skola de icke få komma; och I skolen förnimma att jag är HERREN.
\par 39 Men hören nu, I av Israels hus: Så säger Herren, HERREN: Välan, gån åstad och tjänen edra eländiga avgudar, var och en dem han har. Sedan skolen I förvisso komma att höra på mig, och I skolen då icke mer ohelga mitt heliga namn med edra offerskänker och edra eländiga avgudar.
\par 40 Ty på mitt heliga berg, på Israels höga berg, säger Herren, HERREN där skall hela Israels hus tjäna mig, så många därav som finnas i landet; där skall jag finna behag i dem, där skall jag hava lust till edra offergärder och till förstlingen av edra gåvor, vadhelst I viljen helga.
\par 41 Vid den välbehagliga lukten skall jag finna behag i eder, när tiden kommer, att jag för eder ut ifrån folken och församlar eder från de länder i vilka I ären förströdda. Och jag skall bevisa mig helig på eder inför folkens ögon.
\par 42 Ja, I skolen förnimma att jag är HERREN, när jag låter eder komma in i Israels land, det land som jag med upplyft hand lovade att giva åt edra fäder.
\par 43 Och där skolen I tänka tillbaka på edra vägar och på alla de gärningar som I orenaden eder med; och I skolen känna leda vid eder själva för allt det onda som I haven gjort.
\par 44 Och I skolen förnimma att jag är HERREN, när jag så handlar med eder, för mitt namns skull och icke efter edra onda vägar och edra skändliga gärningar, I av Israels hus, säger Herren, HERREN.
\par 45 Och HERRENS ord kom till mig; han sade:
\par 46 Du människobarn, vänd ditt ansikte söderut och predika mot söder; ja, profetera mot skogslandet söderut;
\par 47 säg till skogen söderut: Hör HERRENS ord: Så säger Herren, HERREN: Se, jag skall tända upp en eld i dig, och den skall förtära alla träd i dig, både de friska och de torra; den flammande lågan skall icke kunna släckas, och av den skola allas ansikten förbrännas, allas mellan söder och norr.
\par 48 Och allt kött skall se att jag, HERREN, har upptänt den; den skall icke kunna släckas.
\par 49 Och jag sade: "Ack, Herre, HERRE! Dessa säga om mig: 'Denne talar ju gåtor.'"

\chapter{21}

\par 1 Och HERRENS ord kom till mig; han sade:
\par 2 Du människobarn, vänd ditt ansikte mot Jerusalem och predika mot helgedomarna, ja, profetera mot Israels land.
\par 3 Och säg till Israels land: Så säger HERREN: Se, jag skall vända mig mot dig och draga ut mitt svärd ur skidan och utrota ur dig både rättfärdiga och ogudaktiga.
\par 4 Ja, eftersom jag skall utrota ur dig både rättfärdiga och ogudaktiga, därför skall mitt svärd fara ut ur skidan och vända sig mot allt kött mellan söder och norr;
\par 5 och allt kött skall förnimma att jag, HERREN, har dragit ut mitt svärd ur skidan; det skall icke mer stickas in igen.
\par 6 Men du, människobarn, må sucka, ja, du må sucka inför deras ögon, som om dina länder skulle brista sönder i din bittra smärta.
\par 7 Och när de fråga dig: "Varför suckar du?", då skall du svara: "För ett olycksbud, som när det kommer, skall göra att alla hjärtan förfäras och alla händer sjunka ned och alla sinnen omtöcknas och alla knän bliva såsom vatten. Se, det kommer, ja, det fullbordas! säger Herren, HERREN."
\par 8 Och HERRENS ord kom till mig; han sade:
\par 9 Du människobarn, profetera och säg: Så säger HERREN: Säg: Ett svärd, ja, ett svärd har blivit vässt och har blivit fejat.
\par 10 Det har blivit vässt, för att det skall anställa ett slaktande; det har blivit fejat, för att det skall blixtra. Eller skola vi få fröjd därav? Fröjd av det som bliver ett tuktoris för min son, ett för vilket intet trä kan bestå!
\par 11 Nej, han har lämnat det till att fejas, för att det skall fattas i handen; svärdet har blivit vässt och fejat för att sättas i en dråpares hand.
\par 12 Ropa och jämra dig, du människobarn, ty det drabbar mitt folk, det drabbar alla Israels hövdingar. De äro med mitt folk hemfallna åt svärdet; därför må du slå dig på länden.
\par 13 Ty rannsakning har redan skett; huru skulle det då vara möjligt att riset icke drabbade, det ris för vilket intet kan bestå? säger Herren, HERREN.
\par 14 Men du, människobarn, profetera och slå händerna tillsammans; må svärdet fördubblas, ja, bliva såsom tre, må det bliva ett mordsvärd, ett mordsvärd jämväl för den störste, svärdet som drabbar dem från alla håll.
\par 15 Ja, för att deras hjärtan må försmälta av ångest, och för att många må falla, skall jag sända det blänkande svärdet mot alla deras portar. Ack, det är gjort likt en blixt, det är draget för att slakta!
\par 16 Hugg lös med all makt åt höger, måtta åt vänster, varthelst din egg kan bliva riktad.
\par 17 Också jag skall slå mina händer tillsammans och släcka min vrede. Jag, HERREN, har talat.
\par 18 Och HERRENS ord kom till mig; han sade:
\par 19 Du människobarn, märk ut åt dig två vägar på vilka den babyloniske konungens svärd kan gå fram; låt båda gå ut från ett och samma land. Skär så ut en vägvisare, skär ut den för den plats där stadsvägarna skilja sig.
\par 20 Märk ut såsom det håll dit svärdet kan gå dels Rabba i Ammons barns land, dels Juda med det befästa Jerusalem.
\par 21 Ty konungen i Babel står redan vid vägskälet där de båda vägarna begynna; han vill låta spå åt sig, han skakar pilarna, han rådfrågar sina husgudar, han ser på levern.
\par 22 I sin högra hand får han då ut lotten "Jerusalem", för att han där skall sätta upp murbräckor, öppna sin mun till krigsrop, upphäva sin röst till härskri, för att han där skall sätta upp murbräckor mot portarna, kasta upp en vall och bygga en belägringsmur. -
\par 23 Detta synes dem vara en falsk spådom: de hava ju heliga eder. Men han uppväcker minnet av deras missgärning, och så ryckas de bort.
\par 24 Därför säger Herren, HERREN så: Eftersom I haven uppväckt minnet av eder missgärning i det att edra överträdelser hava blivit uppenbara, så att eder syndfullhet visar sig i allt vad I gören, ja, eftersom minnet av eder har blivit uppväckt, därför skolen I komma att med makt ryckas bort.
\par 25 Och du, dödsdömde, ogudaktige furste över Israel, du vilkens dag kommer, när din missgärning har nått sin gräns,
\par 26 så säger Herren, HERREN: Tag av dig huvudbindeln, lyft av dig kronan. Det som nu är skall icke förbliva vad det är; vad lågt är skall upphöjas, och vad högt är skall förödmjukas.
\par 27 Omstörtas, omstörtas, omstörtas skall detta av mig; också detta skall vara utan bestånd, till dess han kommer, som har rätt därtill, den som jag har givit det åt.
\par 28 Och du, människobarn, profetera Och såg: Så säger Herren, HERREN om Ammons barn och om deras smädelser: Säg: Ett svärd, ja, ett svärd är draget, det är fejat för att slakta för att varda mättat och för att blixtra,
\par 29 mitt under det att man skådar åt dig falska profetsyner och spår åt dig lögnaktiga spådomar om att du skall sättas på de dödsdömda ogudaktigas hals, vilkas dag kommer, när missgärningen har nått sin gräns.
\par 30 Må det stickas i skidan igen. I den trakt där du är skapad, i det land varifrån du stammar, där skall jag döma dig.
\par 31 Jag skall Utgjuta min vrede över dig, jag skall mot dig blåsa upp min förgrymmelses eld; och jag skall giva dig till pris åt vilda människor, åt män som äro mästare i att fördärva.
\par 32 Du skall bliva till mat åt elden, ditt blod skall flyta i landet; ingen skall mer tänka på dig. Ty jag, HERREN, har talat.

\chapter{22}

\par 1 Och HERRENS ord kom till mig; han sade:
\par 2 Du människobarn, vill du döma ja, vill du döma blodstaden? Förehåll henne då alla hennes styggelser
\par 3 och säg: Så säger Herren, HERREN: Du stad som utgjuter dina invånares blod, så att din stund måste komma, du som gör eländiga avgudar åt dig och så bliver orenad!
\par 4 Genom det blod som du har utgjutit har du ådragit dig skuld, och genom de eländiga avgudar som du har gjort har du orenat dig; så har du påskyndat dina dagars slut och nu hunnit gränsen för dina år. Därför skall jag låta dig bliva till smälek för folken och till spott för alla länder..
\par 5 Ja, både nära och fjärran skall man bespotta dig, du vilkens namn är skändat, du förvirringens stad.
\par 6 Se, hos dig trotsa Israels hövdingar var och en på sin arm, om det gäller att utgjuta blod.
\par 7 Över fader och moder uttalar man förbannelser hos dig; mot främlingen övar man våld hos dig; den faderlöse och änkan förtrycker man hos dig.
\par 8 Mina heliga ting föraktar du, och mina sabbater ohelgar du.
\par 9 Förtalare finnas hos dig, om det gäller att utgjuta blod. Man håller hos dig offermåltider på bergen; man bedriver hos dig vad skändligt är.
\par 10 Man blottar sin faders blygd hos dig; man kränker hos dig kvinnan, när hon har sin orenhets tid.
\par 11 Man bedriver styggelse, var och en med sin nästas hustru; ja, man orenar i skändlighet sin sons hustru; man kränker hos dig sin syster, sin faders dotter.
\par 12 Man tager hos dig mutor för att utgjuta blod; ja, du ockrar och tager ränta och skinnar din nästa med våld, och mig förgäter du, säger Herren, HERREN.
\par 13 Men se, jag slår mina händer tillsammans i harm över det skinneri du övar, och i harm över det blod som du utgjuter hos dig.
\par 14 Menar du att ditt mod skall bestå, eller att dina händer skola vara starka nog, när tiden kommer, att jag utför mitt verk på dig? Jag HERREN, har talat, och jag fullbordar det också.
\par 15 Jag skall förskingra dig bland folken och förströ dig i länderna; så skall jag taga bort ifrån dig all din orenhet.
\par 16 Du skall bliva vanärad inför folkens ögon, genom din egen skuld; och du skall förnimma att jag är HERREN.
\par 17 Och HERRENS ord kom till mig; han sade:
\par 18 Du människobarn, Israels hus har för mig blivit slagg; de äro allasammans blott koppar, tenn, järn och bly i ugnen; de äro ett silver som kan räknas för slagg.
\par 19 Därför säger Herren, HERREN så: Eftersom I allasammans haven blivit slagg, se, därför skall jag hopsamla eder i Jerusalem.
\par 20 Likasom man hopsamlar silver, koppar, järn, bly och tenn i ugnen och där blåser upp eld under det och smälter det, så skall jag i min vrede och förtörnelse hopsamla eder och kasta eder i ugnen och smälta eder.
\par 21 Ja, jag skall samla eder tillhopa; och blåsa upp min förgrymmelses eld under eder, för att I man smältas däri.
\par 22 Likasom silver smältes i ugnen, så skolen I smältas däri; och I skolen förnimma att det är jag, HERREN, som utgjuter min förtörnelse över eder.
\par 23 Och HERRENS ord kom till mig; han sade: Du människobarn, säg till dem:
\par 24 Du är ett land som icke bliver renat, icke varder sköljt av regn på vredens dag.
\par 25 De profeter som där finnas hava sammansvurit sig och blivit såsom rytande, rovgiriga lejon; de äta upp själar, de riva till sig gods och dyrbarheter och göra många till änkor därinne.
\par 26 Prästerna där våldföra min lag och ohelga mina heliga ting; de göra ingen åtskillnad mellan heligt och oheligt och undervisa icke om skillnaden mellan rent och orent. De tillsluta sina ögon för mina sabbater, och så bliver jag ohelgad mitt ibland dem.
\par 27 Furstarna därinne äro såsom rovgiriga vargar; de utgjuta blod och förgöra själar för att skaffa sig vinning.
\par 28 De profeter som de hava tjäna dem såsom vitmenare; de skåda åt dem falska profetsyner och spå åt dem lögnaktiga spådomar; de säga: "Så säger Herren, HERREN", och det fastän HERREN icke har talat.
\par 29 Folket i landet begår våldsgärningar och tager rov; den arme och fattige förtrycka de, och mot främlingen öva de våld, utan lag och rätt.
\par 30 Jag söker bland dem efter någon som skulle kunna uppföra en mur och träda fram i gapet inför mig till försvar för landet, på det att jag icke må fördärva det; men jag finner ingen.
\par 31 Därför utgjuter jag min vrede över dem och gör ände på dem med min förgrymmelses eld. Deras gärningar skall jag låta komma över deras huvuden, säger Herren, HERREN.

\chapter{23}

\par 1 Och HERRENS ord kom till mig; han sade:
\par 2 Du människobarn, det var en gång två kvinnor, döttrar till en och samma moder.
\par 3 Dessa bedrevo otukt i Egypten; de gjorde det redan i sin ungdom. Där kramades deras bröst, och där smekte man deras jungfruliga barm.
\par 4 Den äldre hette Ohola, och hennes syster Oholiba. Därefter blevo de mina, och födde söner och döttrar. Och om deras namn är att veta att Ohola är Samaria, och Oholiba Jerusalem.
\par 5 Men Ohola bedrev otukt i stället för att hålla sig till mig; hon upptändes av lusta till sina älskare hennes grannar assyrierna,
\par 6 där de kommo klädda i mörkblå purpur och voro ståthållare och landshövdingar, vackra unga män allasammans, ryttare som redo på hästar.
\par 7 Hon gav åt dem sin trolösa älskog, åt Assurs alla yppersta söner; och varhelst hon upptändes av lusta, där orenade hon sig på alla deras eländiga avgudar
\par 8 Men ändå uppgav hon icke sin otukt med egyptierna, som hade fått ligga hos henne i hennes ungdom, och som hade smekt hennes jungfruliga barm och slösat på henne sin otukt.
\par 9 Därför gav jag henne till pris åt hennes älskare, åt Assurs söner, till vilka hon var upptänd av lusta.
\par 10 Och sedan dessa hade blottat hennes blygd, förde de bort hennes söner och döttrar och dräpte henne själv med svärd; så blev hon en varnagel för andra kvinnor, då nu dom blev hållen över henne.
\par 11 Men fastän hennes syster Oholiba såg detta, upptändes hon av lusta ännu värre och drev sin otukt ännu längre än systern.
\par 12 Hon upptändes av lusta till Assurs söner; de voro ju ståthållare och landshövdingar och voro hennes grannar, de kommo klädda i präktig dräkt, ryttare som redo på hästar, vackra unga män allasammans.
\par 13 Och jag såg att också hon orenade sig; båda gingo de samma väg.
\par 14 Men denna drev sin otukt ännu längre. Ty när hon fick se mansbilder inristade i väggen, beläten av kaldéer, som man hade inristat och målat röda med dyrbar färg,
\par 15 framställda med gördlar kring sina länder och med ståtliga huvudbonader, allasammans lika kämpar, ja, när hon fick se dessa bilder av Babels söner, av de män som hade sitt fädernesland i Kaldeen;
\par 16 då upptändes hon av lusta till dem, strax när hon såg dem för sina ögon. Och hon sände bud till dem i Kaldeen;
\par 17 och Babels söner kommo till henne och lågo hos henne i älskog och orenade henne genom sin otukt. Först sedan hon hade blivit orenad av dem, vände sig hennes själ ifrån dem.
\par 18 Men när hon så öppet bedrev sin otukt och blottade sin blygd, då vände sig min själ ifrån henne, likasom den hade vänt sig ifrån hennes syster.
\par 19 Dock drev hon sin otukt ännu längre: hon mindes sin ungdoms dagar, då hon bedrev otukt i Egyptens land;
\par 20 och så upptändes hon åter av lusta till bolarna där, som hade kött såsom åsnor och flöde såsom hästar.
\par 21 Ja, din håg stod åter till din ungdoms skändlighet, när egyptierna smekte din barm, därför att du hade så ungdomliga bröst.
\par 22 Därför, du Oholiba, säger Herren, HERREN så: Se, jag skall uppväcka mot dig dina älskare, dem som din själ har vänt sig ifrån, och jag skall låta dem komma över dig från alla sidor,
\par 23 Babels söner och alla kaldéer, pekodéer, soéer och koéer och alla Assurs söner med dem, vackra unga män, ståthållare och landshövdingar allasammans, kämpar och berömliga män, som rida på hästar allasammans.
\par 24 De skola komma över dig med vagnar och hjuldon i mängd och med skaror av folk; rustade med skärmar och sköldar och klädda hjälmar skola de anfalla dig från alla sidor. Och jag skall överlämna domen åt dem, och de skola döma dig efter sina rätter.
\par 25 Jag skall låta min nitälskan gå över dig, så att de fara grymt fram mot dig; de skola skära av dig näsa och öron, och de som bliva kvar av dig skola falla för svärd. Man skall föra bort dina söner och döttrar, och vad som bliver kvar av dig skall förtäras av eld.
\par 26 Man skall slita av dig dina kläder och taga ifrån dig dina härliga smycken.
\par 27 Så skall jag göra slut på din skändlighet och på den otukt som du begynte öva i Egyptens land; och du skall icke mer lyfta upp dina ögon till dem och icke mer tänka på Egypten.
\par 28 Ty så säger Herren, HERREN: Se, jag vill giva dig till pris åt dem som du nu hatar, åt dem som din själ har vänt sig bort ifrån.
\par 29 Och de skola fara fram mot dig såsom fiender, och skola taga ifrån dig allt vad du har förvärvat och lämna dig naken och blottad; ja, din otuktiga blygd skall varda blottad, med din skändlighet och din otukt.
\par 30 Detta skall man göra dig, därför att du i otukt lopp efter hedningarna och orenade dig på deras eländiga avgudar.
\par 31 Du vandrade på din systers väg; därför skall jag sätta i din hand samma kalk som gavs åt henne.
\par 32 Ja, så säger Herren, HERREN: Du skall nödgas dricka din systers kalk, så djup och så vid som den är, och den skall bringa dig åtlöje och smälek i fullt mått.
\par 33 Du skall bliva drucken och bliva full av bedrövelse, ty en ödeläggelsens och förödelsens kalk är din syster Samarias kalk.
\par 34 Du skall nödgas dricka ut den till sista droppen, ja ock slicka dess skärvor, och du skall sarga ditt bröst. Ty jag har talat, säger Herren, HERREN.
\par 35 Därför säger Herren, HERREN så: Eftersom du har förgätit mig och kastat mig bakom din rygg, därför måste du ock bära på din skändlighet och din otukt.
\par 36 Och HERREN sade till mig: Du människobarn, vill du döma Ohola och Oholiba? Förehåll dem då deras styggelser.
\par 37 Ty de hava begått äktenskapsbrott, och blod låder vid deras händer. Ja, med sina eländiga avgudar hava de begått äktenskapsbrott; och till mat åt dem hava de offrat sina barn, dem som de hade fött åt mig.
\par 38 Därtill gjorde de mig detta: samma dag som de orenade min helgedom ohelgade de ock mina sabbater.
\par 39 Ty samma dag som de slaktade sina barn åt de eländiga avgudarna gingo de in i min helgedom och ohelgade den. Se, sådant hava de gjort i mitt hus.
\par 40 Än mer, de sände bud efter män som skulle komma fjärran ifrån; budbärare skickades till dem, och se, de kommo, de män för vilka du hade tvått dig och sminkat dina ögon och prytt dig med smycken.
\par 41 Och du satt på en härlig vilobädd, med ett dukat bord framför, och du hade där ställt fram min rökelse och min olja.
\par 42 Sorglöst larm hördes därinne, och till de män ur hopen, som voro där, hämtade man ytterligare in dryckesbröder från öknen. Och dessa satte armband på kvinnornas armar och härliga kronor på deras huvuden.
\par 43 Då sade jag: "Skall hon, den utlevade, få hålla i med att begå äktenskapsbrott? Skall man alltjämt få bedriva otukt med henne, då hon är en sådan?"
\par 44 Ty man gick in till henne, såsom man går in till en sköka; ja, så gick man in till Ohola och till Oholiba, de skändliga kvinnorna.
\par 45 Men rättfärdiga man skola döma dem efter den lag som gäller för äktenskapsbryterskor och blodsutgjuterskor; ty äktenskapsbryterskor äro de, och blod låder vid deras händer.
\par 46 Ja, så säger Herren, HERREN: Må man sammankalla en församling mot dem och prisgiva dem åt misshandling och plundring.
\par 47 Och församlingen skall stena dem och hugga dem i stycken med svärd, och dräpa deras söner och döttrar, och bränna upp deras hus i eld.
\par 48 Så skall jag göra slut på skändligheten i landet, och alla kvinnor må låta varna sig, så att de icke bedriva sådan skändlighet som I.
\par 49 Och man skall låta eder skändlighet komma över eder, och I skolen få bära på de synder I haven begått med edra eländiga avgudar; och I skolen förnimma att jag är Herren, HERREN.

\chapter{24}

\par 1 Och HERRENS ord kom till mig i nionde året, på tionde dagen i tionde månaden; han sade:
\par 2 Du människobarn, skriv upp åt dig namnet på denna dag, just denna dag; ty konungen i Babel har på just denna dag ryckt fram mot Jerusalem.
\par 3 Och tala till det gensträviga släktet i en liknelse; säg till dem: Så säger Herren, HERREN: Sätt på grytan, och när du har satt på den, så gjut vatten däri.
\par 4 Lägg sedan köttstyckena tillhopa däri, allahanda goda stycken, av låret och bogen; och fyll den så med de bästa märgbenen.
\par 5 Tag härtill det bästa av hjorden; och lägg bränsle under den för att koka benen. Låt den koka starkt, så att ock benen bliva kokta i den.
\par 6 Så säger nu Herren, HERREN: Ve över blodstaden, den rostiga grytan, varifrån rosten icke har kunnat tagas bort! Det ena köttstycket efter det andra har man redan tagit ut därur, utan att kasta lott om ordningen.
\par 7 Ty det blod hon har utgjutit är ännu kvar därinne; på kala klippan lät hon det rinna ned; hon utgöt det icke på sådan mark att mullen har kunnat skyla det.
\par 8 För att vreden skulle hava sin gång, och för att jag skulle utkräva hämnd, lät jag det blod hon utgöt komma på kala klippan, där det icke kunde skylas.
\par 9 Därför säger Herren, HERREN så: Ve över blodstaden! Jag skall nu ytterligare öka på bränslet därunder.
\par 10 Ja, lägg på mer ved, tänd upp eld, låt köttet bliva förstört och spadet koka in och benen bliva förbrända.
\par 11 Och låt den sedan stå tom på eldsglöden, till dess att den bliver så upphettad att dess koppar glödgas och orenligheten smältes bort därur och rosten försvinner.
\par 12 Tung möda har den kostat, och ändå har dess myckna rost icke gått bort. Så må nu dess röst komma i elden!
\par 13 Därför att din orenhet är så skändlig, och därför att du icke blev ren. huru jag än sökte rena dig, därför skall du nu icke mer bliva fri ifrån din orenhet, förrän jag har släckt min vrede på dig.
\par 14 Jag, HERREN, har talat. Det kommer! Jag skall fullborda det! Jag skall icke släppa efter och icke skona och icke ångra mig. Efter dina vägar och dina gärningar skall man döma dig, säger Herren, HERREN.
\par 15 Och HERRENS ord kom till mig; han sade:
\par 16 Du människobarn, se, genom en plötslig död skall jag taga ifrån dig den som är dina ögons lust, men du må icke hålla dödsklagan eller gråta eller fälla tårar.
\par 17 Tyst må du jämra dig; men du skall icke hålla sorgefest såsom efter en död. Nej, sätt på dig din huvudbindel och tag skor på dina fötter; skyl icke ditt skägg, och ät icke det särskilda bröd som eljest är övligt.
\par 18 Sedan talade jag nästa morgon till folket, men på aftonen dog min hustru; och följande morgon gjorde jag såsom mig var befallt.
\par 19 Då sade folket till mig: "Vill du icke omtala för oss vad det betyder att du så gör?"
\par 20 Jag svarade dem: HERRENS ord kom till mig; han sade:
\par 21 Säg till Israels hus: Så säger Herren, HERREN: Se, jag vill ohelga min helgedom, eder stolta härlighet, edra ögons lust och eder själs längtan. Och edra söner och döttrar, som I haven måst övergiva, skola falla för svärd.
\par 22 Då skolen I komma att göra såsom jag har gjort: I skolen icke skyla skägget och icke äta det övliga brödet.
\par 23 Och I skolen behålla huvudbindlarna på edra huvuden och skorna på edra fötter; I skolen icke hålla dödsklagan eller gråta, utan skolen sitta där försmäktande genom edra missgärningar och sucka med varandra.
\par 24 Hesekiel skall vara ett tecken för eder; alldeles såsom han gör skolen I komma att göra. När detta händer, skolen I förnimma att jag är Herren, HERREN.
\par 25 Men du, människobarn, må veta att på den tid då jag tager ifrån dem deras värn, deras härliga fröjd, deras ögons lust och deras själs begär, deras söner och döttrar,
\par 26 på den tiden skall en räddad flykting komma till dig och förkunna detta.
\par 27 Och då när flyktingen är där, skall din mun upplåtas, och du skall tala och icke mer vara stum; och du skall vara ett tecken för dem, och de skola förnimma att jag är HERREN.

\chapter{25}

\par 1 Och HERRENS ord kom till mig; han sade:
\par 2 Du människobarn, vänd ditt ansikte mot Ammons barn och profetera mot dem.
\par 3 Och säg till Ammons barn: Hören Herrens, HERRENS ord: Så säger Herren, HERREN: Eftersom du ropar: "Rätt så!" över min helgedom, som har blivit oskärad, och över Israels land, som har blivit ödelagt, och över Juda folk, som har måst vandra bort i fångenskap,
\par 4 se, därför vill jag giva dig till besittning åt österlänningarna, så att de få slå upp sina tältläger i dig och sätta upp sina boningar i dig; de skola få äta din frukt, och de skola dricka din mjölk.
\par 5 Och jag skall göra Rabba till en betesmark för kameler och Ammons barns land till en lägerplats för får; och I skolen förnimma att Jag är HERREN.
\par 6 Ty så säger Herren, HERREN: Eftersom du klappar i händerna och stampar med fötterna och i ditt sinnes hela övermod gläder dig vid Israels lands ofärd,
\par 7 se, därför skall jag uträcka min hand mot dig och giva dig till byte åt hedningarna och utrota dig ifrån folken och utplåna dig ur länderna; jag skall förgöra dig, och du skall förnimma att jag är HERREN.
\par 8 Så säger Herren, HERREN: Eftersom Moab och Seir säga: "Se, nu är det med Juda hus likasom med alla andra folk",
\par 9 se, därför skall jag lägga Moabs bergsluttning öppen och förstöra dess städer, Ja, dess städer så många de äro, vad härligast är i landet, Bet-Hajesimot, Baal-Meon och Kirjatama.
\par 10 Åt österlänningarna skall jag giva det till besittning, likasom jag skall göra med Ammons barns land, så att man icke mer tänker på Ammons barn ibland folken.
\par 11 Ja, över Moab skall jag hålla dom, och de skola förnimma att jag är HERREN.
\par 12 Så säger Herren, HERREN: Eftersom Edom har handlat så hämndgirigt mot Juda hus och ådragit sig svår skuld genom sin hämnd på dem,
\par 13 därför säger Herren, HERREN så: Jag skall uträcka min hand mot Edom och utrota därur både människor och djur. Och jag skall göra det till en ödemark ända från Teman, och ända borta i Dedan skola de falla för svärd.
\par 14 Och jag skall utföra min hämnd på Edom genom mitt folk Israel, och dessa skola göra med Edom efter min vrede och förtörnelse; och det skall så få känna min hämnd, säger Herren, HERREN.
\par 15 Så säger Herren, HERREN: Eftersom filistéerna hava handlat så hämndgirigt, ja, eftersom de i sitt sinnes övermod hava velat utkräva hämnd och i sin eviga fiendskap hava velat bereda fördärv,
\par 16 därför säger Herren, HERREN så: Se, jag vill uträcka min hand mot filistéerna och utrota keretéerna och förgöra vad som är kvar av Kustlandet vid havet.
\par 17 Och jag skall taga stor hämnd på dem och tukta dem i förtörnelse. Och när jag låter min hämnd drabba dem, då skola de förnimma att jag är HERREN.

\chapter{26}

\par 1 Och i elfte året, på första dagen i månaden, kom HERRENS ord till mig; han sade:
\par 2 Du människobarn, eftersom Tyrus sade om Jerusalem: "Rätt så, uppbruten är nu folkens port, den är öppnad för mig; jag bliver rik, nu då hon är förödd",
\par 3 därför säger Herren, HERREN så: Se, jag skall komma över dig, Tyrus, och jag skall upphäva många folk mot dig, likasom havet upphäver sina böljor.
\par 4 De skola förstöra Tyrus' murar och riva ned dess torn. Så skall jag sopa bort själva dess grus och förvandla staden till en kal klippa.
\par 5 En torkplats för fisknät skall den vara ute i havet; ty jag har talat, säger Herren, HERREN. Ja, den skall bliva ett byte för folken;
\par 6 och dess döttrar på fastlandet skola dräpas med svärd. De skola förnimma att jag är HERREN.
\par 7 Ty så säger Herren, HERREN: Se, jag vill låta Nebukadressar, konungen i Babel, konungarnas konung, komma norrifrån över Tyrus, med hästar och vagnar och ryttare och med en stor hop folk.
\par 8 Dina döttrar på fastlandet skall han dräpa med svärd; han skall bygga en belägringsmur mot dig och kasta upp mot dig en vall och resa ett sköldtak mot dig.
\par 9 Sin murbräckas stötar skall han rikta mot dina murar och skall med sina krigsredskap bryta ned dina torn.
\par 10 Hans hästar äro så många att dammet skall överhölja dig. Vid dånet av hans ryttare och av hans hjuldon och vagnar skola dina murar darra, när han drager in genom dina portar, såsom man drager in i en erövrad stad.
\par 11 Med sina hästars hovar skall han trampa sönder alla dina gator; ditt folk skall han dräpa med svärd, och dina stolta stoder skola störta till jorden.
\par 12 Man skall röva dina skatter och plundra dina handelsvaror; man skall riva dina murar och bryta ned dina sköna hus; och stenarna, trävirket och gruset skall man kasta i havet.
\par 13 Jag skall göra slut på dina sångers buller, och man skall icke mer höra klangen av dina harpor.
\par 14 Ja, jag skall göra dig till en kal klippa en torkplats för fisknät skall du bliva; aldrig mer skall du varda uppbyggd. Ty jag, HERREN, har talat, säger Herren, HERREN.
\par 15 Så säger Herren, HERREN till Tyrus: Sannerligen, vid dånet av ditt fall, när de slagna jämra sig, vid det att man dräper och mördar i dig, skola havsländerna bäva.
\par 16 Och alla hövdingar vid havet skola stiga ned från sina troner, de skola lägga bort sina mantlar och taga av sig sina brokigt vävda kläder; förskräckelse bliver deras klädnad, och nere på jorden skola de sitta; deras förskräckelse varder ständigt ny, och de häpna över ditt öde.
\par 17 De stämma upp en klagosång över dig och säga om dig: Huru har du icke blivit förstörd, du havsfolkens tillhåll, du högtprisade stad, du som var så mäktig på havet, där du låg med dina invånare, vilka fyllde människorna med skräck för alla som bodde i dig!
\par 18 Nu förskräckas havsländerna på ditt falls dag, och öarna i havet förfäras vid din undergång.
\par 19 Ty så säger Herren, HERREN: När jag gör dig till en ödelagd stad, lik någon stad som ingen bebor, ja, när jag låter djupet upphäva sig mot dig och de stora vattnen betäcka dig,
\par 20 då störtar jag dig ned till dem som hava farit ned i graven, till folk som levde för länge sedan; och lik en längesedan ödelagd plats får du ligga där i jordens djup, hos dem som hava farit ned i graven. Så skall du förbliva obebodd, medan jag gör härliga ting i de levandes land.
\par 21 Jag skall låta dig taga en ande med förskräckelse, så att man aldrig i evighet skall finna dig, huru man än söker efter dig, säger Herren, HERREN.

\chapter{27}

\par 1 Och HERRENS ord kom till mig; han sade:
\par 2 Du människobarn, stäm upp en klagosång över Tyrus;
\par 3 säg till Tyrus: Du som bor vid havets portar och driver köpenskap med folken, hän till många havsländer, så säger Herren, HERREN: O Tyrus, du säger själv: "Jag är skönhetens fullhet."
\par 4 Ja, dig som har ditt rike ute i havet, dig gjorde dina byggningsmän fullkomlig i skönhet.
\par 5 Av cypress från Senir timrade de allt plankverk på dig; de hämtade en ceder från Libanon för att göra din mast.
\par 6 Av ekar från Basan tillverkade de dina åror. Ditt däck prydde de med elfenben i ädelt trä från kittéernas öländer.
\par 7 Ditt segel var av fint linne, med brokig vävnad från Egypten, och det stod såsom ditt baner. Mörkblått och purpurrött tyg från Elisas öländer hade du till soltält.
\par 8 Sidons och Arvads invånare voro roddare åt dig; de förfarna män du själv hade, o Tyrus, dem tog du till skeppare.
\par 9 Gebals äldste och dess förfarnaste män tjänade dig med att bota dina läckor. Alla havets skepp med sina sjömän tjänade dig vid ditt varubyte.
\par 10 Perser, ludéer och putéer funnos i din här och voro ditt krigsfolk. Sköldar och hjälmar hängde de upp i dig; dessa gåvo dig glans.
\par 11 Arvads söner stodo med din här runt om på dina murar, gamadéer hade sin plats i dina torn. Sina stora sköldar hängde de upp runt om på dina murar; de gjorde din skönhet fullkomlig.
\par 12 Tarsis var din handelsvän, ty du var rik på allt slags gods silver, järn, tenn och bly gavs dig såsom betalning.
\par 13 Javan, Tubal och Mesek, de drevo köpenskap med dig; trälar och kopparkärl gåvo de dig i utbyte..
\par 14 Vagnshästar, ridhästar och mulåsnor gåvos åt dig såsom betalning från Togarmas land.
\par 15 Dedans söner drevo köpenskap med dig ja, många havsländer drevo handel i din tjänst; elfenben och ebenholts tillförde de dig såsom hyllningsgåvor.
\par 16 Aram var din handelsvän, ty du var rik på konstarbeten; karbunkelstenar, purpurrött tyg, brokiga vävnader och fint linne. koraller och rubiner gåvo de dig såsom betalning.
\par 17 Juda och Israels land drevo köpenskap med dig; vete från Minnit, bakverk och honung, olja och balsam gåvo de dig i utbyte.
\par 18 Damaskus var din handelsvän, ty du var rik på konstarbeten, ja, på allt slags gods; de kommo med vin från Helbon och med ull från Sahar.
\par 19 Vedan och Javan gåvo dig spånad såsom betalning; konstsmitt järn och kassia och kalmus fick du i utbyte.
\par 20 Dedan drev köpenskap hos dig med sadeltäcken att rida på.
\par 21 Araberna och Kedars alla furstar, de drevo handel i din tjänst; med lamm och vädurar och bockar drevo de handel hos dig.
\par 22 Sabas och Raemas köpmän drevo köpenskap med dig; kryddor av allra yppersta slag och alla slags ädla stenar och guld gåvo de dig såsom betalning.
\par 23 Haran, Kanne och Eden, Sabas köpmän, Assur och Kilmad drevo köpenskap med dig.
\par 24 De drevo köpenskap hos dig med sköna kläder, med mörkblå, brokigt vävda mantlar, med mångfärgade täcken, med välspunna, starka tåg, på din marknad.
\par 25 Tarsis-skepp foro åstad med dina bytesvaror. Så fylldes du med gods och blev tungt lastad, där du låg i havet.
\par 26 Och dina roddare förde dig åstad, ut på de vida vattnen. Då kom östanvinden och krossade dig. där du låg i havet.
\par 27 Ditt gods, dina handels- och bytesvaror, dina sjömän och skeppare, dina läckors botare och dina bytesmäklare, allt krigsfolk på dig, allt manskap som fanns ombord på dig, de sjunka nu ned i havet, på ditt falls dag.
\par 28 Vid dina skeppares klagorop bäva markerna,
\par 29 och alla som ro med åror övergiva sina skepp; sjömän och alla skeppare på havet begiva sig i land.
\par 30 De ropa högt över ditt öde och klaga bittert; de strö stoft på sina huvuden och vältra sig i aska.
\par 31 De raka sig skalliga för din skull och hölja sig i sorgdräkt; de gråta över dig i bitter sorg, under bitter klagan.
\par 32 Med jämmer stämma de upp en klagosång om dig, en klagosång över ditt öde: "Vem var såsom Tyrus, hon som nu ligger i det tysta ute i havet?"
\par 33 Där dina handelsvaror sattes i land från havet mättade du många folk; med ditt myckna gods och dina många bytesvaror riktade du jordens konungar.
\par 34 Men nu, då du har förlist och försvunnit ifrån havet, ned i vattnens djup, nu hava dina bytesvaror och allt ditt manskap sjunkit med dig.
\par 35 Havsländernas alla inbyggare häpna över ditt öde, deras konungar stå rysande, med förfäran i sina ansikten.
\par 36 Köpmännen ute bland folken vissla åt dig; du har tagit en ände med förskräckelse till evig tid.

\chapter{28}

\par 1 Och HERRENS ord kom till mig; han sade:
\par 2 Du människobarn, säg till fursten i Tyrus: Så säger Herren, HERREN: Eftersom ditt hjärta är så högmodigt och du säger: "Jag är en gud, ja, på ett gudasäte tronar jag mitt ute i havet", du som dock är en människa och icke en gud, huru mycket du än i ditt hjärta tycker dig vara en gud -
\par 3 och sant är att du är visare än Daniel; ingen hemlighet är förborgad för dig;
\par 4 genom din vishet och ditt förstånd har du skaffat dig rikedom, guld och silver har du skaffat dig i dina förrådshus;
\par 5 och genom den stora vishet varmed du drev din köpenskap har du ökat din rikedom, och så har ditt hjärta blivit högmodigt för din rikedoms skull -
\par 6 därför säger Herren, HERREN så: Eftersom du i ditt hjärta tycker dig vara en gud,
\par 7 se, därför skall jag låta främlingar komma över dig, de grymmaste folk; och de skola draga ut sina svärd mot din visdoms skönhet och skola oskära din glans.
\par 8 De skola störta dig ned i graven, och du skall dö såsom en dödsslagen man, mitt ute i havet.
\par 9 Månne du då skall säga till din dråpare: "Jag är en gud", du som ej är en gud, utan en människa, i dens våld, som slår dig till döds?
\par 10 Såsom de oomskurna dö, så skall du dö, för främlingars hand. Ty jag har talat, säger Herren, HERREN.
\par 11 Och HERRENS ord kom till mig han sade:
\par 12 Du människobarn, stäm upp en klagosång över konungen i Tyrus och säg till honom: Så säger Herren, HERREN: Du var ypperst bland härliga skapelser, full med vishet och fullkomlig i skönhet.
\par 13 I Eden, Guds lustgård, bodde du, höljd i alla slags ädla stenar: karneol, topas och kalcedon, krysolit, onyx och jaspis, safir, karbunkel och smaragd, jämte guld; du var prydd med smycken och klenoder, beredda den dag då du skapades.
\par 14 Du var en kerub, som skuggade vida, och jag hade satt dig att vara på det heliga gudaberget, du fick där gå omkring bland gnistrande stenar.
\par 15 Lyckosam var du på dina vägar från den dag då du skapades, till dess att orättfärdighet blev funnen hos dig.
\par 16 Men under din myckna köpenskap blev ditt inre fyllt med orätt, och du föll i synd. Då förvisade jag dig från gudaberget och förgjorde dig, du vittskuggande kerub; du fick ej stanna bland de gnistrande stenarna.
\par 17 Eftersom ditt hjärta högmodades över din skönhet och du förspillde din vishet för ditt pråls skull, därför slog jag dig ned till jorden och gav dig till pris åt konungarna, så att de fingo se sin lust på dig.
\par 18 Genom dina många missgärningar vid din orättrådiga köpenskap ohelgade du dina helgedomar. Därför lät jag eld gå ut ifrån dig, och av den blev du förtärd. Jag lät dig ligga såsom aska på jorden inför alla som besökte dig.
\par 19 Alla som kände dig bland folken häpnade över ditt öde. Du tog en ände med förskräckelse för evig tid.
\par 20 Och HERRENS ord kom till mig; han sade:
\par 21 Du människobarn, vänd ditt ansikte mot Sidon och profetera mot det
\par 22 och säg: Så säger Herren, HERREN: Se, jag skall komma över dig, Sidon, och förhärliga mig i dig. Ja, att jag är HERREN, det skall man förnimma, när jag håller dom över henne och bevisar mig helig på henne.
\par 23 Och jag skall sända över henne pest och blod på hennes gator, och dödsslagna män skola falla därinne för ett svärd som skall drabba henne från alla sidor; och man skall förnimma att jag är HERREN.
\par 24 Sedan skall för Israels hus icke mer finnas någon stingande tagg eller något sårande törne bland alla de grannfolk som nu håna dem; och man skall förnimma att jag är Herren, HERREN.
\par 25 Ja, så säger Herren, HERREN: När jag församlar Israels barn från de folk bland vilka de äro förströdda, då skall jag bevisa mig helig på dem inför folkens ögon, och de skola sedan få bo i sitt land, det som jag har givit åt min tjänare Jakob.
\par 26 De skola bo där i trygghet och bygga hus och plantera vingårdar ja, de skola bo i trygghet, när jag håller dom över alla som håna dem på alla sidor; och de skola förnimma att jag är HERREN, deras Gud.

\chapter{29}

\par 1 I tionde året, på tolfte dagen i tionde månaden, kom HERRENS ord till mig; han sade:
\par 2 Du människobarn, vänd ditt ansikte mot Farao, konungen i Egypten, och profetera mot honom och mot hela Egypten.
\par 3 Tala och säg: Så säger Herren, HERREN: Se, jag skall komma över dig, Farao, du Egyptens konung, du stora drake, som ligger där i dina strömmar och säger: "Min Nilflod är min; själv har jag gjort mig."
\par 4 Jag skall sätta krokar i dina käftar och låta fiskarna i dina strömmar fastna vid dina fjäll, och så skall jag draga dig upp ur dina strömmar med alla de fiskar i dina strömmar, som hänga fast vid dina fjäll.
\par 5 Och jag skall kasta dig ut i öknen med alla fiskarna ifrån dina strömmar; du skall falla på marken och ej tagas bort därifrån eller upphämtas, ty åt markens djur och himmelens fåglar vill jag giva dig till mat;
\par 6 och alla Egyptens inbyggare skola förnimma att jag är HERREN. Ty de äro en rörstav för Israels barn;
\par 7 ja, när dessa fatta i dig med handen, går du sönder och sårar envar av dem i sidan; och när de stödja sig på dig, brytes du av och lämnar dem alla med vacklande länder.
\par 8 Därför säger Herren, HERREN så: Se, jag vill låta svärd komma över dig, och jag skall utrota ur dig både människor och djur.
\par 9 Och Egyptens land skall bliva förött och ödelagt, och man skall förnimma att jag är HERREN. Detta därför att han sade: "Nilfloden är min; själv har jag gjort den.
\par 10 Ja, därför skall jag komma över dig och dina strömmar, och göra Egyptens land till en ödemark, ett ödelagt land, från Migdol till Sevene, fram till Etiopiens gräns.
\par 11 Ingen människofot skall gå där fram, och ingen fot av något boskapsdjur skall gå där fram; och det skall ligga obebott i fyrtio år.
\par 12 Och jag skall göra Egyptens land till en ödemark bland ödelagda länder, och dess städer skola ligga öde bland förhärjade städer i fyrtio år; och jag skall förskingra egyptierna bland folken och förströ dem i länderna.
\par 13 Ty så säger Herren, HERREN: När fyrtio år äro förlidna, skall jag församla egyptierna från de folk bland vilka de äro förskingrade.
\par 14 Och jag skall åter upprätta Egypten och låta egyptierna komma tillbaka till Patros' land, varifrån de stamma. Där skola de bliva ett oansenligt rike,
\par 15 ja, ett rike oansenligare än andra riken, så att det icke mer skall kunna upphäva sig över folken; jag skall låta dem bliva så få att de icke kunna råda över folken.
\par 16 Och Israels barn skola icke mer sätta sitt hopp till dem som allenast uppväcka minnet av deras missgärning, när de vända sig till dem; och de skola förnimma att jag är Herren, HERREN.
\par 17 I tjugusjunde året, på första dagen i första månaden, kom HERRENS ord till mig; han sade:
\par 18 Du människobarn, Nebukadressar konungen i Babel, har låtit sin här förrätta ett svårt arbete mot Tyrus; alla huvuden hava blivit skalliga och alla skuldror sönderskavda. Men han och hans här hava icke fått någon lön från Tyrus för det arbete som han har förrättat mot det.
\par 19 Därför säger Herren, HERREN så: Se, jag vill giva Egyptens land åt Nebukadressar, konungen i Babel; och han skall föra bort dess rikedomar och taga rov därifrån och göra byte där, och detta skall hans här få till lön.
\par 20 Såsom en vedergällning för hans arbete giver jag honom Egyptens land; ty för min räkning hava de utfört sitt verk, säger Herren, HERREN.
\par 21 På den tiden skall jag låta ett horn växa upp åt Israels hus, och du skall få upplåta din mun mitt ibland dem; och de skola förnimma att jag är HERREN.

\chapter{30}

\par 1 Och HERRENS ord kom till mig; han sade:
\par 2 Du människobarn, profetera och säg: Så säger Herren, HERREN: Jämren eder: "Ack ve, vilken dag!"
\par 3 Ty dagen är nära, HERRENS dag är nära; en molnhöljd dag är det, hednafolkens stund är inne.
\par 4 Ett svärd kommer över Egypten, och Etiopien fattas av ångest, när de slagna falla i Egypten och dess rikedomar föras bort och dess grundvalar upprivas.
\par 5 Etiopier, putéer och ludéer, och hela hopen av främmande folk, och kubéer och förbundslandets söner skola med dem falla för svärd.
\par 6 Så säger HERREN: Ja, Egyptens försvarare skola falla, och dess stolta makt skall störtas ned; från Migdol till Sevene skola de som bo där falla för svärd, säger Herren, HERREN.
\par 7 Och deras land skall ligga öde bland ödelagda länder, och städerna där skola vara bland förhärjade städer.
\par 8 Och man skall förnimma att jag är HERREN, när jag tänder eld på Egypten och låter alla dess hjälpare varda krossade.
\par 9 På den dagen skola sändebud draga ut från mig på skepp, för att injaga skräck hos Etiopien mitt i dess trygghet; och man skall där fattas av ångest på Egyptens dag; ty se, det kommer!
\par 10 Så säger Herren, HERREN: Ja, jag skall göra slut på Egyptens rikedomar genom Nebukadressar, konungen i Babel.
\par 11 Han och hans folk med honom, de grymmaste hedningar, skola hämtas dit till att fördärva landet; de skola draga sina svärd mot Egypten och uppfylla landet med slagna.
\par 12 Och jag skall göra strömmarna till torr mark och sälja landet i onda mäns hand. Jag skall ödelägga landet med allt vad däri är, genom främmande män. Jag, HERREN, har talat.
\par 13 Så säger Herren, HERREN: Jag skall ock förstöra de eländiga avgudarna och göra slut på avgudarna i Nof, och ur Egyptens land skall ingen furste mer uppstå; och jag skall låta fruktan komma över Egyptens land.
\par 14 Jag skall ödelägga Patros och tända eld på Soan och hålla dom över No.
\par 15 Och jag skall utgjuta min vrede över Sin, Egyptens värn, och utrota den larmande hopen i No.
\par 16 Ja, jag skall tända eld på Egypten, Sin skall gripas av ångest, No skall bliva intaget och Nof överfallas på ljusa dagen.
\par 17 Avens och Pi-Besets unga män skola falla för svärd, och själva skola de vandra bort i fångenskap.
\par 18 I Tehafnehes bliver dagen mörk, när jag där bryter sönder Egyptens ok och dess stolta makt där får en ände; ja, ett moln skall övertäcka det, och dess döttrar skola vandra bort i fångenskap.
\par 19 Jag skall hålla dom över Egypten, och man skall förnimma att jag är HERREN.
\par 20 I elfte året, på sjunde dagen i första månaden, kom HERRENS ord till mig; han sade:
\par 21 Du människobarn, jag har brutit sönder Faraos, den egyptiske konungens, arm; och se, den har icke blivit förbunden, man har icke brukat läkemedel, icke lindat den, icke lagt på den förband, för att åter göra den stark nog till att föra svärdet.
\par 22 Därför säger Herren, HERREN så: Se, jag skall komma över Farao, konungen i Egypten, och bryta sönder hans armar, både den som ännu är stark och den som redan är sönderbruten, och skall låta svärdet falla ur hans hand.
\par 23 Och jag skall förskingra egyptierna bland folken och förströ dem i länderna.
\par 24 Den babyloniske konungens armar skall jag stärka, och jag skall sätta milt svärd i hans hand; men Faraos armar skall jag bryta sönder, så att han upphäver jämmerrop inför honom, såsom en dödsslagen kämpe gör.
\par 25 Ja, jag skall stärka den babyloniske konungens armar, men Faraos armar skola sjunka ned; och man skall förnimma att jag är HERREN, när jag sätter mitt svärd i den babyloniske konungens hand, för att han skall svänga det mot Egyptens land.
\par 26 Och jag skall förskingra egyptierna bland folken och förströ dem i länderna; och de skola förnimma att jag är HERREN.

\chapter{31}

\par 1 I elfte året, på första dagen i tredje månaden, kom HERRENS ord till mig; han sade:
\par 2 Du människobarn, säg till Farao, konungen i Egypten, och till hans larmande hop: Vem kan förliknas med dig i din storhet?
\par 3 Se, du är ett ädelt träd, en ceder på Libanon, med sköna grenar och skuggrik krona och hög stam, en som med sin topp räcker upp bland molnen.
\par 4 Vatten gåvo den växt, djupets källor gjorde den hög. Ty med sina strömmar omflöto de platsen där den var planterad; först sedan sände de sina flöden till alla andra träd på marken.
\par 5 Så fick den högre stam än alla träd på marken, den fick talrika kvistar och långa grenar, genom det myckna vatten den hade, när den sköt skott.
\par 6 Alla himmelens fåglar byggde sig nästen bland dess kvistar, under dess grenar födde alla markens djur sina ungar, och i dess skugga bodde allahanda stora folk.
\par 7 Och den blev skön genom sin storhet och genom sina grenars längd, där den stod med sin rot invid stora vatten.
\par 8 Ingen ceder i Guds lustgård gick upp emot denna, ingen cypress hade kvistar som kunde förliknas med dennas, ingen lönn bar grenar, jämförliga med dennas; nej, intet träd i Guds lustgård liknade den i skönhet
\par 9 Så skön hade jag låtit den bliva, i dess rikedom på grenar, att alla Edens träd i Guds lustgård måste avundas den.
\par 10 Därför säger Herren, HERREN så Eftersom du växte så hög, ja, eftersom ditt träd sträckte sin topp upp bland molnen och förhävde sig i sitt hjärta över att det var så högt,
\par 11 därför skall jag prisgiva det åt en som är väldig bland folken. Han skall förvisso utföra sitt verk därpå, ty för dess ogudaktighets skull har jag förkastat det.
\par 12 Ja, främlingar hava fått hugga ned det, de grymmaste folk, och hava låtit det ligga. Dess kvistar hava nu fallit på bergen och i alla dalar; dess grenar hava blivit avbrutna och kastade i alla landets bäckar, och alla folk på jorden hava måst draga bort därifrån och försaka dess skugga och låta det ligga.
\par 13 På dess kullfallna stam bo alla himmelens fåglar, och på dess grenar lägra sig alla markens djur.
\par 14 Så sker, för att ett träd som växer vid vatten aldrig skall yvas över sin höjd och sträcka sin topp upp bland molnen; ja, för att icke ens de väldigaste av dem skola stå och yvas, intet träd som har haft vatten att dricka. Ty de äro allasammans hemfallna åt döden och måste ned i jordens djup, till att vara där bland människors barn, hos dem som hava farit ned i graven.
\par 15 Så säger Herren, HERREN: På den dag då det for ned till dödsriket lät jag djupet för dess skull hölja sig i sorgdräkt; jag hämmade strömmarna där, och de stora vattnen höllos tillbaka. Jag lät Libanon för dess skull kläda sig i svart, och alla träd på marken förtvinade i sorg över det.
\par 16 Genom dånet av dess fall kom jag folken att bäva, när jag störtade det ned i dödsriket, till dem som hade farit ned i graven. Men då tröstade sig i jordens djup alla Edens träd, de yppersta och bästa på Libanon, alla de som hade haft vatten att dricka.
\par 17 Också de hade, såsom det trädet, måst fara ned till dödsriket, till dem som voro slagna med svärd; dit foro ock de som hade varit dess stöd och hade bott i dess skugga bland folken.
\par 18 Kan nu något bland Edens träd förliknas med dig i härlighet och storhet? Och dock skall du, såsom Edens träd, störtas ned i jordens djup och ligga där bland oomskurna, hos dem som äro slagna med svärd. Så skall det gå Farao och hela hans larmande hop, säger Herren, HERREN.

\chapter{32}

\par 1 I tolfte året, på första dagen i tolfte månaden, kom HERRENS ord till mig; han sade:
\par 2 Du människobarn, stäm upp en klagosång över Farao, konungen i Egypten, och säg till honom: Det är förbi med dig, du lejon bland folken! Och du var dock lik draken i havet, där du for fram i dina strömmar och rörde upp vattnet med dina fötter och grumlade dess strömmar.
\par 3 Så säger nu Herren, HERREN: Jag skall breda ut mitt nät över dig genom skaror av många folk, och de skola draga upp dig i mitt garn.
\par 4 Jag skall kasta dig upp på jorden, jag skall slunga dig bort på marken och låta alla himmelens fåglar slå ned på dig och låta de vilda djuren på hela jorden mätta sig med dig.
\par 5 Jag skall kasta ditt kött på bergen och fylla dalarna med ditt stora skrov.
\par 6 Och landet som du har nedsölat skall jag vattna med ditt blod ända upp till bergen, och bäckarna skola bliva fulla av dig.
\par 7 Och när jag utsläcker dig, skall jag övertäcka himmelen och förmörka dess stjärnor; jag skall övertäcka solen med moln, och månens ljus skall icke lysa mer.
\par 8 Alla ljus på himmelen skall jag förmörka för din skull och låta mörker komma över ditt land, säger Herren, HERREN.
\par 9 Och många folks hjärtan skall jag slå med skräck, när jag gör din undergång bekant bland folkslagen, ja, i länder som du icke känner.
\par 10 Jag skall komma många folk att häpna för din skull, och deras konungar skola för din skull gripas av bävan, när jag i deras åsyn svänger mitt svärd; vart ögonblick skola de frukta, envar för sitt liv, på ditt falls dag.
\par 11 Ty så säger Herren, HERREN: Den babyloniske konungens svärd skall komma över dig.
\par 12 Jag skall låta din larmande hop falla för hjältars svärd, grymmast bland hedningar äro de alla. De skola föröda Egyptens härlighet, och hela dess larmande hop skall förgöras;
\par 13 jag skall utrota all dess boskap, den som betar vid det myckna vattnet. Av människofot skall det icke mer röras upp, ej heller röras upp av boskapsklövar.
\par 14 Sedan skall jag låta deras vatten sjunka undan och deras strömmar flyta bort såsom olja, säger Herren, HERREN,
\par 15 i det jag gör Egyptens land till en ödslig ödemark och berövar landet allt vad däri är, när jag nu slår alla dess inbyggare, så att man förnimmer att jag är HERREN.
\par 16 Detta är en klagosång som man skall sjunga, ja, folkens döttrar skola sjunga den; de skola sjunga den över Egypten med hela dess larmande hop, säger Herren, HERREN.
\par 17 I tolfte året, på femtonde dagen i månaden, kom HERRENS ord till mig; han sade:
\par 18 Du människobarn, sjung sorgesång över Egyptens larmande hop. Bjud henne att såsom döttrarna av de väldigaste folk fara ned i jordens djup, till dem som redan hava farit ned i graven.
\par 19 Finnes någon så ringa att du är förmer än hon? Nej, far du ned och låt dig bäddas bland de oomskurna.
\par 20 Bland män som äro slagna med svärd skola ock dina falla. Svärdet är redo; släpen bort henne med hela hennes larmande hop.
\par 21 Mäktiga hjältar skola tala till Farao ur dödsriket, till honom och till hans hjälpare: "Ja, de hava måst fara hitned, och nu ligga de där, de oomskurna, slagna med svärd.".
\par 22 Där ligger redan Assur med hela sin skara; runt omkring honom har denna sin gravplats. Allasammans ligga de där slagna, fallna för svärd.
\par 23 Sin grav har han fått längst ned i underjorden, och runt omkring honom ligger hans skara begraven. Allasammans ligga de slagna, fallna för svärd, de man som en gång utbredde skräck i de levandes land.
\par 24 Där ligger Elam med hela sin larmande hop, vilande runt omkring hans grav. Allasammans ligga de slagna, männen som föllo för svärd, och som oomskurna måste fara, ned i jordens djup, desamma som en gång utbredde skräck omkring sig i de levandes land; nu måste de bära sin skam bland de andra som hava farit ned i graven.
\par 25 Ja, bland slagna har han fått sitt läger med hela sin larmande hop; runt omkring honom har denna sin gravplats. Allasammans ligga de där oomskurna, slagna med svärd; en gång utbredde sig ju skräck omkring dem i de levandes land, men de måste nu bära sin skam bland dem som hava farit ned i graven. Ja, bland slagna har han fått sin plats.
\par 26 Där ligger Mesek-Tubal med hela sin larmande hop; runt omkring honom har denna sin gravplats. Allasammans ligga de där oomskurna, slagna med svärd; en gång utbredde de ju skräck omkring sig i de levandes land.
\par 27 Men dessa fallna män ur de oomskurnas hop, de få icke vila bland hjältarna, bland dem som hava farit ned till dödsriket i sin krigiska rustning och fått sina svärd lagda under sina huvuden. Nej, deras missgärningar hava kommit över deras ben. De utbredde ju skräck i de levandes land, såsom hjältar göra.
\par 28 Ja, också du skall bliva krossad bland de oomskurna och få ligga bland dem som äro slagna med svärd.
\par 29 Där ligger Edom med sina konungar och alla sina hövdingar; huru mäktiga de än voro, hava de nu fått sin plats bland dem som äro slagna med svärd; de måste ligga bland de oomskurna, bland dem som hava farit ned i graven.
\par 30 Där ligga Nordlandets furstar allasammans, med alla sidonier, ty de hava måst fara ned till de slagna, de hava kommit på skam, trots den skräck de utbredde genom sina väldiga gärningar. Och de ligga där oomskurna bland dem som hava blivit slagna med svärd; de måste bära sin skam bland dem som hava farit ned i graven.
\par 31 Dem skall nu Farao få se, och han skall så trösta sig över hela sin larmande hop. Ja, Farao och hela hans har äro slagna med svärd, säger Herren, HERREN.
\par 32 Ty väl utbredde jag skräck för honom i de levandes land, men nu måste han, Farao, med hela sin larmande hop, låta sig bäddas bland de oomskurna, hos dem som äro slagna med svärd, säger Herren, HERREN.

\chapter{33}

\par 1 Och HERRENS ord kom till mig; han sade:
\par 2 Du människobarn, tala till dina landsmän och säg till dem: Om jag vill låta svärdet komma över ett land, och folket i landet har utsett bland sig en man som det har gjort till sin väktare,
\par 3 och denne ser svärdet komma över landet och stöter i basunen och varnar folket,
\par 4 men den som får höra basunljudet ända icke låter varna sig, och svärdet sedan kommer och tager honom bort, då kommer hans blod över hans eget huvud.
\par 5 Ty han hörde ju basunljudet, men lät icke varna sig; därför kommer hans blod över honom själv. Om han hade låtit varna sig, så hade han räddat sitt liv. -
\par 6 Men om väktaren ser svärdet komma och icke stöter i basunen och folket så icke bliver varnat, och svärdet sedan kommer och tager bort någon bland dem, då bliver visserligen denne borttagen genom sin egen missgärning, men hans blod skall jag utkräva av väktarens hand.
\par 7 Dig, du människobarn, har jag satt till en väktare för Israels hus, för att du å mina vägnar skall varna dem, när du hör ett ord från min mun.
\par 8 Om jag säger till den ogudaktige: "Du ogudaktige, du måste dö", och du då icke säger något till att varna den ogudaktige för hans väg, så skall väl den ogudaktige dö genom sin missgärning, men hans blod skall jag utkräva av din hand.
\par 9 Men om du varnar den ogudaktige för hans väg, på det att han må vända om ifrån den, och han likväl icke vänder om ifrån sin väg, då skall visserligen han dö genom sin missgärning, men du själv har räddat din själ.
\par 10 Och du, människobarn, säg till Israels hus: I sägen så: "Våra överträdelser och synder tynga på oss och vi försmäkta genom dem. Huru kunna vi då bliva vid liv?"
\par 11 Men svara dem: Så sant jag lever, säger Herren, HERREN, jag har ingen lust till den ogudaktiges död, utan fastmer därtill att den ogudaktige vänder om från sin väg och får leva. Så vänden då om, ja, vänden om från edra onda vägar; ty icke viljen I väl dö, I av Israels hus?
\par 12 Men du, människobarn, säg till dina landsmän: Den rättfärdiges rättfärdighet skall icke rädda honom, när han begår överträdelser; och den ogudaktige skall icke komma på fall genom sin ogudaktighet, när han vänder om från sin ogudaktighet, lika litet som den rättfärdige skall kunna leva genom sin rättfärdighet, när han syndar.
\par 13 Om jag säger till den rättfärdige att han skall få leva, och han sedan i förlitande på sin rättfärdighet gör vad orätt är, så skall intet ihågkommas av all hans rättfärdighet, utan genom det orätta som han gör skall han dö.
\par 14 Och om jag säger till den ogudaktige: "Du måste dö", och han sedan vänder om från sin synd och övar rätt och rättfärdighet,
\par 15 så att han, den ogudaktige, give tillbaka den pant han har fått och ersätter vad han har rövat och vandrar efter livets stadgar, så att han icke gör vad orätt är, då skall han förvisso leva och icke dö.
\par 16 Ingen av de synder han har begått skall då tillräknas honom; han har övat rätt och rättfärdighet, därför skall han förvisso få leva. -
\par 17 Men nu säga dina landsmän: "Herrens väg är icke alltid densamma", då det fastmer är deras egen väg som icke alltid är densamma.
\par 18 Om den rättfärdige vänder om från sin rättfärdighet och gör vad orätt är, så måste han just därför dö.
\par 19 Men om den ogudaktige vänder om från sin ogudaktighet och övar rätt och rättfärdighet, då skall han just därför få leva.
\par 20 Och ändå sägen I: "Herrens väg är icke alltid densamma." Jo, jag skall döma var och en av eder efter hans vägar, I av Israels hus.
\par 21 I det tolfte året sedan vi hade blivit bortförda i fångenskap, på femte dagen i tionde månaden, kom en flykting ifrån Jerusalem till mig med budskapet: "Staden är intagen."
\par 22 Nu hade på aftonen före flyktingens ankomst HERRENS hand kommit över mig; men på morgonen öppnade han åter min mun, just före mannens ankomst, så att jag, då nu min mun blev öppnad, upphörde att vara stum.
\par 23 Och HERRENS ord kom till mig; han sade:
\par 24 Du människobarn, de som bo ibland ruinerna där borta i Israels land säga "Abraham var en ensam man, och han fick dock landet till besittning. Vi äro många, oss måste väl landet då vara givet till besittning!"
\par 25 Säg därför till dem: Så säger Herren, HERREN: I äten kött med blodet i, I upplyften edra ögon till edra eländiga avgudar, och I utgjuten blod; och likväl skullen I få hava landet till besittning!
\par 26 I trotsen på edra svärd, I bedriven vad styggeligt är, I skänden varandras hustrur; och likväl skullen I få hava landet till besittning!
\par 27 Nej; så skall du säga till dem: Så säger Herren, HERREN: Så sant jag lever, de som bo där bland ruinerna skola falla för svärd; och dem som bo på landsbygden skall jag giva till mat åt de vilda djuren, och de som bo i bergfästen eller i grottor skola dö genom pest.
\par 28 Jag skall göra landet öde och tomt, och dess stolta makt skall få en ände; och Israels berg skola ödeläggas, så att ingen går där fram.
\par 29 Och de skola förnimma att jag är HERREN, när jag gör landet öde och tomt, för alla de styggelsers skull som de hava bedrivit.
\par 30 Men du, människobarn, dina landsmän, som orda om dig invid väggarna och i ingångarna till husen, de tala sinsemellan, den ene med den andre, och säga: "Kom, låt oss höra vad det är för ett ord som nu utgår från HERREN."
\par 31 Och de komma till dig, såsom gällde det en folkförsamling, och sätta sig hos dig såsom mitt folk; och de höra dina ord, men göra icke efter dem. Ty väl hopgöra de med munnen ljuvliga ord, men deras hjärtan stå blott efter egen vinning.
\par 32 Och se, du är för dem, såsom när någon som har vacker röst och spelar väl sjunger en kärleksvisa; de höra väl dina ord, men göra icke efter dem.
\par 33 Men när det kommer - ty se det kommer! - då skola de förnimma att en profet har varit ibland dem.

\chapter{34}

\par 1 Och HERRENS ord kom till mig; han sade:
\par 2 Du människobarn, profetera mot Israels herdar, profetera och säg till dem, till herdarna: Så säger Herren, HERREN: Ve eder, I Israels herdar, som haven sörjt allenast för eder själva! Var det då icke för hjorden som herdarna borde sörja?
\par 3 I stället åten I upp det feta, med ullen klädden I eder, det gödda slaktaden I; men om hjorden vårdaden I eder icke.
\par 4 De svaga stärkten I icke, det sjuka heladen I icke, det sargade förbunden I icke, det fördrivna förden I icke tillbaka, det förlorade uppsökten I icke, utan med förtryck och hårdhet fören I fram mot dem.
\par 5 Så blevo de förskingrade, därför att de icke hade någon herde, de blevo till mat åt alla markens djur, ja, de blevo förskingrade.
\par 6 Mina får gå nu vilse på alla berg och alla höga kullar; över hela jorden äro mina får förskingrade, utan att någon frågar efter dem eller uppsöker dem.
\par 7 Hören därför HERRENS ord, I herdar:
\par 8 Så sant jag lever, säger Herren, HERREN, sannerligen, eftersom mina får hava lämnats till rov, ja, eftersom mina får hava blivit till mat åt alla markens djur, då de nu icke hava någon herde, och eftersom mina herdar icke fråga efter mina får ja, eftersom herdarna sörja för sig själva och icke sörja för mina får,
\par 9 därför, I herdar: Hören HERRENS ord:
\par 10 Så säger Herren, HERREN: Se, jag skall komma över herdarna och utkräva mina får ur deras hand och göra slut på deras herdetjänst; och herdarna skola då icke mer kunna sörja för sig själva, ty jag skall rädda mina får ur deras gap, så att de icke bliva till mat åt dem.
\par 11 Ty så säger Herren, HERREN: Se, jag skall själv taga mig an mina får och leta dem tillsammans.
\par 12 Likasom en herde letar tillsammans sin hjord, när hans får äro förströdda omkring honom, så skall ock jag leta tillsammans mina får och rädda dem från alla de orter till vilka de förskingrades på en dag av moln och töcken.
\par 13 Och jag skall föra dem ut ifrån folken och församla dem ur länderna, och skall låta dem komma till sitt eget land och föra dem i bet på Israels berg, vid bäckarna och var man eljest kan bo i landet.
\par 14 På goda betesplatser skall jag föra dem i bet, på Israels höga berg skola de få sina betesmarker; där skola de lägra sig på goda betesmarker, och fett bete skola de hava på Israels berg.
\par 15 Jag skall själv föra mina får i bet och själv utse lägerplatser åt dem, säger Herren, HERREN.
\par 16 Det förlorade skall jag uppsöka, det fördrivna skall jag föra tillbaka, det sargade skall jag förbinda, och det svaga skall jag stärka. Men det feta och det starka skall jag förgöra; ja, jag skall sköta det såsom rätt är.
\par 17 Men I, mina får, så säger Herren, HERREN: Se, jag vill döma mellan får och får, mellan vädurar och bockar.
\par 18 Är det eder icke nog att I fån beta på den bästa betesplatsen, eftersom I med edra fötter trampen ned vad som är kvar på eder betesplats? Och är det eder icke nog att I fån dricka det klaraste vattnet, eftersom I med edra fötter grumlen vad som har lämnats kvar?
\par 19 Skola mina får beta av det som edra fötter hava trampat ned, och dricka vad edra fötter hava grumlat?
\par 20 Nej; därför säger Herren, HERREN så till dem: Se, jag skall själv döma mellan de feta fåren och de magra fåren.
\par 21 Eftersom I med sida och bog stöten undan alla de svaga och med edra horn stången dem, till dess att I haven drivit dem ut och förskingrat dem,
\par 22 därför skall jag frälsa mina får, så att de icke mer bliva till rov, och skall döma mellan får och får.
\par 23 Och jag skall låta en herde uppstå, gemensam för dem alla, och han skall föra dem i bet, nämligen min tjänare David; ja, han skall föra dem i bet, han skall vara deras herde.
\par 24 Jag, HERREN, skall vara deras Gud, men min tjänare David skall vara hövding bland dem. Jag, HERREN, har talat.
\par 25 Och jag skall med dem sluta ett fridsförbund; jag skall göra ände på vilddjuren i landet, så att man i trygghet kan bo mitt i öknen och sova i skogarna.
\par 26 Och jag skall låta dem själva och landet runt omkring min höjd bliva till välsignelse. Jag skall låta regn falla i rätt tid; regnskurar till välsignelse skall det bliva.
\par 27 Träden på marken skola bära sin frukt, och jorden skall giva sin gröda, och själva skola de bo i sitt land i trygghet; och de skola förnimma att jag är HERREN, när jag bryter sönder deras ok och räddar dem från de människors hand, som hava hållit dem i träldom.
\par 28 De skola sedan icke mer bliva ett byte för folken, och markens djur skola ej äta upp dem, utan de skola bo i trygghet, och ingen skall förskräcka dem.
\par 29 Och jag skall åt dem låta en plantering växa upp, som skall bliva dem till berömmelse; och de som bo i landet skola icke mer ryckas bort av hunger, ej heller skola de mer lida smälek av folken.
\par 30 Och de skola förnimma att jag, HERREN, deras Gud, är med dem, och att de, Israels hus, äro mitt folk, säger Herren, HERREN.
\par 31 Ja, I ären mina får, I ären får i min hjord, människor som I ären, och jag är eder Gud, säger Herren, HERREN.

\chapter{35}

\par 1 Och HERRENS ord kom till mig; han sade:
\par 2 Du människobarn, vänd ditt ansikte mot Seirs berg och profetera mot det;
\par 3 säg till det: Så säger Herren, HERREN: Se, jag skall komma över dig, du Seirs berg, och uträcka min hand mot dig och göra dig öde och tomt.
\par 4 Jag skall göra dina städer till ruiner, och självt skall du bliva öde; och du skall förnimma att jag är HERREN.
\par 5 Eftersom du har hyst en evig fiendskap mot Israels barn och givit dem till pris åt svärdet under deras ofärds tid, den tid då missgärningen hade nått sin gräns,
\par 6 därför, så sant jag lever, säger Herren, HERREN, skall jag förvandla dig till blod, och blod skall förfölja dig; eftersom du icke har hatat blod, skall blod förfölja dig.
\par 7 Ja, jag skall göra Seirs berg tomt och öde och utrota därifrån envar som färdas där, fram eller tillbaka.
\par 8 Och jag skall uppfylla dess berg med dess slagna män; ja, på dina höjder, i dina dalar och vid alla dina bäckar skola svärdsslagna män falla.
\par 9 Jag skall göra dig till en ödemark för evärdlig tid, och dina städer skola icke mer bliva bebodda; och I skolen förnimma att jag är HERREN.
\par 10 Eftersom du sade: "De båda folken och de båda länderna skola bliva mina, vi skola taga dem i besittning" - detta fastän HERREN bodde där -
\par 11 därför, så sant jag lever, säger Herren, HERREN, skall jag utföra mitt verk med samma vrede och nitälskan varmed du i din hätskhet har utfört ditt verk mot dem; och jag skall göra mig känd bland dem, när jag dömer dig.
\par 12 Och du skall förnimma att jag är HERREN. Jag har hört alla de smädelser som du har talat mot Israels berg, i det du har sagt: "Det är en ödemark; de äro givna åt oss till mat."
\par 13 Ja, I spärraden upp munnen mot mig och togen den full av ord mot mig; jag har väl hört det.
\par 14 Så säger Herren, HERREN: Till hela jordens glädje skall jag göra dig till en ödemark.
\par 15 Därför att du gladde dig åt att Israels hus' arvedel blev ödelagd, därför skall jag göra likaså med dig. Du skall bliva en ödemark, du Seirs berg, du hela Edom, så långt du sträcker dig; och man skall förnimma att jag är HERREN.

\chapter{36}

\par 1 Och du, människobarn, profetera om Israels berg och säg: I Israels berg, hören HERRENS ord.
\par 2 Så säger Herren, HERREN: Eftersom fienden säger om eder: "Rätt så, de urgamla offerhöjderna hava nu blivit vår besittning",
\par 3 därför må du profetera och säga: Så säger Herren, HERREN: Eftersom, ja, eftersom man har förött eder och fikar efter eder från alla sidor, för att I måtten tillfalla de övriga folken såsom deras besittning och eftersom I ären så utsatta för onda tungors hån och folks förtal,
\par 4 därför, I Israels berg, mån I nu höra Herrens, HERRENS ord: Så säger Herren, HERREN till bergen och höjderna, till bäckarna och dalarna, till de förödda ruinerna och de övergivna städerna, som hava lämnats till rov och spott åt de övriga folken runt omkring,
\par 5 ja, därför säger Herren, HERREN så: Sannerligen, i brinnande nitälskan talar jag mot de övriga folken och mot Edom, så långt det sträcker sig, ja, mot dessa som med hela sitt hjärtas glädje och i sitt sinnes övermod hava tillägnat sig mitt land såsom besittning, för att driva ut dess inbyggare och göra det till sitt byte.
\par 6 Profetera alltså om Israels land och säg till bergen och höjderna, till bäckarna och dalarna: Så säger Herren, HERREN: Se, i nitälskan och vrede är det som jag talar, eftersom I liden sådan smälek av folken.
\par 7 Därför säger Herren, HERREN så Jag upplyfter min hand och betygar: Sannerligen, folken runt omkring eder skola själva få lida smälek.
\par 8 Men I, Israels berg, I skolen åter grönska och bära frukt åt mitt folk Israel, ty snart skola de komma åter.
\par 9 Ty se, jag skall komma till eder, jag skall vända mig till eder, och I skolen bliva brukade och besådda.
\par 10 Och jag skall församla på eder människor i myckenhet, alla Israels barn, så många de äro; och städerna skola ånyo bliva bebodda och ruinerna åter byggas upp.
\par 11 Ja, jag skall församla på eder människor och boskap i myckenhet, och de skola föröka sig och bliva fruktsamma. Jag skall låta eder bliva bebodda, alldeles såsom I fordom voren; ja, jag skall göra eder ännu mer gott, än I förut fingen röna; och I skolen förnimma att jag är HERREN.
\par 12 Jag skall låta människor åter vandra fram över eder, nämligen mitt folk Israel; de skola hava dig till besittning, och du skall vara deras arvedel; och du skall icke vidare döda deras barn.
\par 13 Så säger Herren, HERREN: Eftersom man säger till dig: "Du är en människoäterska, du har dödat ditt eget folks barn",
\par 14 därför skall du nu icke mer få äta upp människor och icke mer få döda ditt folks barn, säger Herren, HERREN.
\par 15 Jag skall icke mer låta dig höra smälek av folken, och du skall icke mer nödgas bära folkslagens förakt; ej heller skall du mer bringa ditt folk på fall, säger Herren, HERREN.
\par 16 Och HERRENS ord kom till mig; han sade:
\par 17 Du människobarn när Israels barn ännu bodde i sitt land, då orenade de det genom sitt väsende och sina gärningar; såsom en kvinnas orenhet var deras väsende för mig.
\par 18 Då utgöt jag min vrede över dem, för det blods skull som de hade utgjutit över landet, och därför att de hade orenat det med sina eländiga avgudar.
\par 19 Jag förskingrade dem bland folken, och de blevo förströdda i länderna; efter deras väsende och deras gärningar dömde jag dem.
\par 20 Men till vilka folk de än kommo vanärade de mitt heliga namn, i det att man sade om dem: "Detta är HERRENS folk, och de hava likväl måst draga ut ur sitt land.
\par 21 Då ville jag skona mitt heliga namn, som Israels barn vanärade bland de folk till vilka de kommo.
\par 22 Säg därför till Israels barn: Så säger Herren, HERREN: Icke för eder skull gör jag detta, I Israels barn, utan för mitt heliga namns skull, som I haven vanärat bland de folk till vilka I haven kommit.
\par 23 Jag vill nu helga mitt stora namn, som har blivit vanärat bland folken, i det att I haven vanärat det bland dem; och folken skola förnimma att jag är HERREN, säger Herren, HERREN, när jag bevisar mig helig på eder inför deras ögon.
\par 24 Ty jag skall hämta eder ifrån folken och församla eder ifrån alla länder och föra eder till edert land.
\par 25 Och jag skall stänka rent vatten på eder, så att I bliven rena; jag skall rena eder från all eder orenhet och från alla edra eländiga avgudar.
\par 26 Och jag skall giva eder ett nytt hjärta och låta en ny ande komma i edert bröst; jag skall taga bort stenhjärtat ur eder kropp och giva eder ett hjärta av kött.
\par 27 Jag skall låta min Ande komma i edert bröst och så göra, att I vandren efter mina stadgar och hållen mina rätter och gören efter dem.
\par 28 Så skolen I få bo i det land som jag gav åt edra fäder, och I skolen vara mitt folk, och jag skall vara eder Gud.
\par 29 Och jag skall frälsa eder från all eder orenhet. Och jag skall kalla fram säden och låta den bliva ymnig och skall icke mer låta någon hungersnöd komma över eder.
\par 30 Ja, ymnig skall jag låta trädens frukt och markens gröda bliva, för att I icke mer skolen lida hungersnödens smälek bland folken.
\par 31 Då skolen I tänka på edra onda vägar och på edra gärningar, som icke voro goda; och I skolen känna leda vid eder själva för edra missgärningars och styggelsers skull.
\par 32 Men icke för eder skull gör jag detta, säger Herren, HERREN; det vare eder kunnigt. I mån skämmas och blygas för edra vägar, I Israels barn.
\par 33 Så säger Herren, HERREN: När jag har renat eder från alla edra missgärningar, då skall jag låta städerna ånyo bliva bebodda, och då skola ruinerna åter byggas upp,
\par 34 och det förödda landet skall åter bliva brukat, i stället för att det har legat såsom en ödemark inför var man som har gått där fram.
\par 35 Och då skall man säga: "Det landet som var så förött har nu blivit såsom Edens lustgård, och städerna som voro så ödelagda, förödda och förstörda, de äro nu bebodda och befästa."
\par 36 Då skola de folk som äro kvar runt omkring eder förnimma att jag, HERREN, nu åter har byggt upp det som var förstört och ånyo planterat det som var förött. Jag, HERREN, har talat det, och jag fullbordar det också.
\par 37 Så säger Herren, HERREN; Också på detta sätt vill jag bönhöra Israels barn, så vill jag handla med dem: jag skall där föröka människorna, så att de bliva såsom fårhjordar.
\par 38 Såsom hjordar av offerdjur, såsom fårhjordar i Jerusalem vid dess högtider, så skola de hjordar av människor vara, som skola uppfylla de ödelagda städerna. Och man skall förnimma att jag är HERREN.

\chapter{37}

\par 1 HERRENS hand kom över mig, och genom HERRENS Ande fördes jag åstad och sattes ned mitt på slätten, som nu låg full med ben.
\par 2 Och han förde mig fram runt omkring dem, och jag såg att de lågo där i stor myckenhet utöver dalen och jag såg att de voro alldeles förtorkade.
\par 3 Och han sade till mig: "Du människobarn, kunna väl dessa ben åter bliva levande?" Jag svarade: "Herre, HERRE, du vet det."
\par 4 Då sade han till mig: "Profetera över dessa ben och säg till dem: I förtorkade ben, hören HERRENS ord;
\par 5 Så säger Herren, HERREN till dessa ben: Se, jag skall låta ande komma in i eder, så att I åter bliven levande.
\par 6 Jag skall fästa senor vid eder och låta kött växa på eder och övertäcka eder med hud och giva eder ande, så att I åter bliven levande; och I skolen förnimma att jag är HERREN."
\par 7 Och jag profeterade, såsom det hade blivit mig bjudet. Och när jag nu profeterade, hördes ett rassel, och där blev ett gny, och benen kommo åter tillhopa, så att det ena benet fogades till det andra.
\par 8 Och jag såg huru senor och kött växte på dem, och huru de övertäcktes med hud därovanpå; men ingen ande var ännu i dem.
\par 9 Då sade han till mig: "Profetera och tala till anden, ja, profetera, du människobarn, och säg till anden: Så säger Herren, HERREN: Kom, du ande, från de fyra väderstrecken och blås på dessa dräpta, så att de åter bliva levande."
\par 10 Och jag profeterade, såsom han hade bjudit mig. Då kom anden in i dem, och de blevo åter levande och reste sig upp på sina fötter, en övermåttan stor skara.
\par 11 Och han sade till mig: "Du människobarn, dessa ben, de äro alla Israels barn. Se, de säga: 'Våra ben äro förtorkade, vårt hopp har blivit om intet, det är ute med oss.'
\par 12 Profetera därför och säg till dem: Så säger Herren, HERREN: Se, jag vill öppna edra gravar och hämta eder, mitt folk, upp ur edra gravar och låta eder komma till Israels land.
\par 13 Och I skolen förnimma att jag är HERREN, när jag öppnar edra gravar och hämtar eder, mitt folk, upp ur edra gravar.
\par 14 Och jag skall låta min ande komma in i eder, så att I åter bliven levande, och jag skall låta eder få bo i edert land; och I skolen förnimma att jag, HERREN, har tala det, och att jag också har fullborda det, säger HERREN."
\par 15 Och HERRENS ord kom till mig han sade:
\par 16 Du människobarn, tag dig en trästav och skriv på den: "För Juda och hans fränder bland Israels barn." Tag sedan en annan trästav och skriv på den: "En stav för Josef, Efraim, och för hans fränder av hela Israels hus."
\par 17 Foga dem sedan tillhopa med varandra till en enda stav, så att de bliva förenade till ett i din hand.
\par 18 När då dina landsmän säga till dig: "Förklara för oss vad du menar härmed",
\par 19 så svara dem: "Så säger Herren, HERREN: Se, jag vill taga Josefs stav, den som är i Efraims hand, vilken stav ock gäller för de stammar av Israel, som äro hans fränder, och intill denna vill jag lägga Judas stav, båda tillhopa, och så göra dem till en enda stav, så att de bliva ett i min hand."
\par 20 Och stavarna som du har skrivit på skall du hålla i din hand inför deras ögon.
\par 21 Och du skall tala till dem: Så säger Herren, HERREN: Se, jag skall hämta Israels barn ut ifrån de folk till vilka de hava måst vandra bort; jag skall samla dem tillhopa från alla håll och föra dem in i deras land.
\par 22 Och jag skall göra dem till ett enda folk i landet, på Israels berg; en och samma konung skola de alla hava; de skola icke mer vara två folk och icke mer vara delade i två riken.
\par 23 Sedan skola de icke mer orena sig med sina eländiga avgudar och styggelser och med alla slags överträdelser. Och jag skall frälsa dem och hämta dem från alla orter där de hava syndat, och skall rena dem, så att de bliva mitt folk, och jag skall vara deras Gud.
\par 24 Och min tjänare David skall vara konung över dem, och de skola så alla hava en och samma herde; och de skola vandra efter mina rätter och hålla mina stadgar och göra efter dem.
\par 25 Så skola de få bo i det land som jag gav åt min tjänare Jakob, det vari edra fäder bodde. De skola själva få bo där, så ock deras barn och deras barnbarn till evig tid; och min tjänare David skall vara deras hövding evinnerligen.
\par 26 Och jag skall med dem sluta ett fridsförbund; ett evigt förbund med dem skall det vara. Jag skall insätta dem och föröka dem och låta min helgedom stå bland dem evinnerligen.
\par 27 Ja, min boning skall vara hos dem, och jag skall vara deras Gud, och de skola vara mitt folk.
\par 28 Så skola folken förnimma att jag är HERREN, som helgar Israel, då nu min helgedom förbliver ibland dem evinnerligen.

\chapter{38}

\par 1 Och HERRENS ord kom till mig; han sade:
\par 2 Du människobarn, vänd ditt ansikte mot Gog i Magogs land, mot hövdingen över Ros, Mesek och Tubal, och profetera mot honom
\par 3 och säg: Så säger Herren, HERREN: Se, jag skall komma över dig, Gog, du hövding över Ros, Mesek och Tubal.
\par 4 Jag skall locka dig åstad, jag skall sätta krokar i dina käftar och föra dig ut med hela din här, hästar och ryttare, allasammans i präktig rustning, en stor skara, väpnad med skärmar och sköldar, och allasammans med svärd i hand.
\par 5 Perser, etiopier och putéer är, med dem, allasammans med sköld och hjälm,
\par 6 Gomer och alla dess härskaror, Togarmas folk ifrån den yttersta norden och alla dess härskaror; ja, många folk har du med dig.
\par 7 Rusta dig och gör dig redo med alla de skaror som hava församlat sig till dig; och bliv du deras hövitsman.
\par 8 När lång tid har gått, skall du bliva uppbådad; i kommande år skall du få tåga in i ett land som då har fått ro efter svärdet, och vart folk då har blivit hopsamlat från många andra folk, ja, upp till Israels berg, som så länge lågo öde, men vilkas folk då har blivit hämtat fram ifrån de andra folken, så att alla nu bo där i trygghet.
\par 9 Dit skall du draga upp, du skall komma såsom ett oväder och vara såsom ett moln som övertäcker landet, du med alla dina härskaror och med många folk som följa dig.
\par 10 Så säger Herren, HERREN: På den tiden skola planer uppstå i ditt hjärta, och du skall tänka ut onda anslag.
\par 11 Du skall säga: "Jag vill draga upp mot det obefästa landet, jag vill komma över dessa säkra, som bo där i trygghet, ja, som allasammans bo där utan murar och varken hava bommar eller portar."
\par 12 Ty du vill taga rov och göra byte och vända din hand mot ödemarker som nu åter äro bebyggda, och mot ett folk som har blivit hopsamlat från hedningarna, och som nu förvärvar sig boskap och gods, där det bor på jordens mittelhöjd.
\par 13 Saba och Dedan och Tarsis' köpmän och alla dess unga lejon skola då utfråga dig: "Har du kommit för att taga rov, har du församlat dina skaror till att göra byte till att föra bort silver och guld, till att taga boskap och gods, ja, till att taga stort rov?"
\par 14 Profetera därför, du människobarn och säg till Gog: Så säger Herren HERREN: Se, på den tiden, när mitt folk Israel åter bor i trygghet, då skall du förnimma det.
\par 15 Du skall då komma från ditt land längst uppe i norr, du själv och många folk med dig, allasammans ridande på hästar, en stor skara, en talrik här.
\par 16 Du skall draga upp mot mitt folk Israel och komma såsom ett moln för att övertäcka landet. I kommande dagar skall detta ske; jag skall då låta dig komma över mitt land, för att folken skola lära känna mig, när jag inför deras ögon bevisar mig helig på dig, du Gog.
\par 17 Så säger Herren, HERREN: Du är ju den om vilken jag i forna tider talade genom mina tjänare, Israels profeter, som i de tiderna, år efter år, profeterade om att jag skulle låta dig komma över dem.
\par 18 Men på den dagen, den dag då Gog kommer över Israels land, säger Herren, HERREN, då skall jag giva luft åt min vrede.
\par 19 Ja, i min nitälskan och min vredes eld betygar jag det: på den dagen skall det förvisso bliva en stor jordbävning i Israels land.
\par 20 Då skola de bäva för mig, både fiskarna i havet och fåglarna under himmelen och djuren på marken och alla kräldjur som röra sig på jorden och alla människor på jordens yta. Och bergen skola slås ned och klipporna störta omkull och alla murar falla till jorden.
\par 21 Och jag skall båda upp svärd mot honom på alla mina berg, säger Herren, HERREN; den enes svärd skall vara vänt mot den andres.
\par 22 Och jag skall gå till rätta med honom medelst pest och blod; och slagregn och hagelstenar, eld och svavel skall jag låta regna över honom och hans härskaror och över de många folk som följa honom.
\par 23 Så skall jag bevisa mig stor och helig och göra mig känd inför många folks ögon; och de skola förnimma att jag är HERREN.

\chapter{39}

\par 1 Och du, människobarn, profetera mot Gog och säg: Så säger Herren, HERREN: Se, jag skall komma över dig, Gog, du hövding över Ros, Mesek och Tubal.
\par 2 Jag skall locka dig åstad och leda dig fram och föra dig från landet längst uppe i norr och låta dig komma till Israels berg.
\par 3 Där skall jag slå bågen ur din vänstra hand och låta pilarna falla ur din högra hand.
\par 4 På Israels berg skall du falla, med alla dina härskaror och med de folk som följa dig; jag skall giva dig till mat åt rovfåglar av alla slag och åt markens djur.
\par 5 Ute på marken skall du falla. Ty jag har talat, säger Herren, HERREN.
\par 6 Och jag skall sända eld över Magog och över dem som bo trygga i havsländerna; och de skola förnimma att jag är HERREN.
\par 7 Och jag skall göra mitt heliga namn kunnigt bland mitt folk Israel, jag skall icke mer låta mitt heliga namn bliva ohelgat; och folken skola förnimma att jag är HERREN, helig i Israel.
\par 8 Se, det kommer, ja, det fullbordas! säger Herren, HERREN. Detta är den dag om vilken jag har talat.
\par 9 Sedan skola invånarna i Israels städer gå ditut och taga rustningar, sköldar och skärmar, bågar och pilar, handpåkar och spjut såsom bränsle till att elda med, och de skola elda därmed i sju år.
\par 10 De skola icke behöva hämta trä från marken eller hugga ved i skogarna, ty de skola elda med rustningarna. Så skola de taga rov av sina rövare och plundra sina plundrare, säger Herren, HERREN.
\par 11 På den tiden skall jag där i Israel giva åt Gog en plats till grav, nämligen "De framtågandes dal" öster om havet, och den skall stänga vägen för andra som vilja tåga där fram. Där skall man begrava Gog och hela hans larmande hop, och man skall kalla den "Gogs larmande hops dal".
\par 12 Och i sju månader skola Israels barn hålla på med att begrava dem, för att rena landet.
\par 13 Allt folket i landet skall hålla på med begravandet, och detta skall lända dem till berömmelse. Så skall ske på den tid då jag förhärligar mig, säger Herren, HERREN.
\par 14 Och man skall avskilja män som beständigt skola genomvandra landet och begrava dem som tågade där fram, och som ännu ligga kvar ovan jord, och de skola så rena landet; efter sju månaders förlopp skola dessa begynna sitt letande.
\par 15 När så någon av dessa män, som genomvandra landet, på sin färd får se människoben, då skall han sätta upp en vård därbredvid, till dess att dödgrävarna hinna begrava dem i "Gogs larmande hops dal".
\par 16 Där skall ock finnas en stad med namnet Hamona. På detta sätt skola de rena landet.
\par 17 Du människobarn, så säger Herren, HERREN: Säg till alla slags fåglar och till alla markens djur: Församlen eder och kommen hit; samlen eder tillhopa från alla håll till mitt slaktoffer, till ett stort slaktoffer som jag vill anställa åt eder på Israels berg; I skolen få äta kött och dricka blod.
\par 18 I skolen få äta kött av hjältar och dricka blod av jordens hövdingar: av vädurar och lamm och bockar och tjurar, allasammans gödda i Basan.
\par 19 I skolen få äta eder mätta av fett och dricka eder druckna av blod från det slaktoffer som jag anställer åt eder.
\par 20 Ja, mätten eder vid mitt bord av ridhästar och vagnshästar, av hjältar och allt slags krigsfolk, säger Herren, HERREN.
\par 21 Och jag skall uppenbara min härlighet bland folken, så att alla folk skola se den dom som jag har utfört, och se huru jag har låtit min hand drabba dem.
\par 22 Och Israels barn skola förnimma att jag, HERREN, är deras Gud, från den dagen och allt framgent.
\par 23 Och folken skola förnimma att Israels barn blevo bortförda i fångenskap för sin missgärnings skull, eftersom de voro trolösa mot mig, så att jag måste fördölja mitt ansikte för dem; och jag gav dem då i deras ovänners hand, så att de allasammans föllo för svärd.
\par 24 Efter deras orenhet och deras överträdelser handlade jag med dem och fördolde mitt ansikte för dem.
\par 25 Därför säger Herren, HERREN så: Nu skall jag åter upprätta Jakob och förbarma mig över hela Israels hus och nitälska för mitt heliga namn.
\par 26 Och de skola förgäta sin skam och all den otrohet som de hava begått mot mig, då de nu få bo i trygghet i sitt land, utan att någon förskräcker dem.
\par 27 Ja, när jag låter dem vända tillbaka ifrån folkslagen och församlar dem från deras fienders länder, då skall jag bevisa mig helig på dem inför många folks ögon.
\par 28 Och de skola förnimma att jag är HERREN, deras Gud, ty om jag än drev dem bort i fångenskap bland folken, så samlade jag dem sedan tillhopa till deras land och lät ingen enda av dem bliva kvar därute;
\par 29 och jag skall därefter icke mer fördölja mitt ansikte för dem, ty jag skall utgjuta min Ande över Israels hus, säger Herren, HERREN'.

\chapter{40}

\par 1 I det tjugufemte året sedan vi hade blivit bortförda i fångenskap, vid årets begynnelse, på tionde dagen i månaden, i det fjortonde året sedan staden hade blivit intagen, på just den dagen kom HERRENS hand över mig, och han förde mig ditbort.
\par 2 I en syn från Gud förde han mig till Israels land och satte mig ned på ett mycket högt berg, och på detta var likasom en stad byggd söderut.
\par 3 Och dit förde han mig, och se, där stod en man vilkens utseende var såsom koppar; han hade ett linnesnöre i sin hand, så ock en mätstång; och han stod vid porten.
\par 4 Och mannen talade till mig: "Du människobarn, se med dina ögon och hör med dina öron, och akta på allt som jag kommer att visa dig, ty du har blivit förd hit, för att jag skall visa dig det; förkunna för Israels hus allt vad du får se."
\par 5 Och jag såg att en mur gick utomkring huset, runt omkring det. Och mätstången som mannen hade i sin hand var sex alnar lång, var aln en handsbredd längre än en vanlig aln. Och han mätte murbyggnadens bredd: den var en stång, och dess höjd: den var en stång.
\par 6 Därefter gick han till en port som låg mot öster och steg uppför dess trappsteg; och han mätte portens ena tröskel: den var en stång bred och sedan den andra tröskeln: den var en stång bred.
\par 7 Och var vaktkammare var en stång lång och en stång bred, och avståndet mellan vaktkamrarna var fem alnar; och porttröskeln invid portens förhus på inre sidan mätte en stång.
\par 8 Och han mätte upp portens förhus på inre sidan: det mätte en stång.
\par 9 Han mätte upp portens förhus det höll åtta alnar, och dess murpelare: de höllo två alnar. Och portens förhus låg på inre sidan.
\par 10 Och vaktkamrarna i porten mot öster voro tre på var sida, alla tre lika stora; och murpelarna på båda sidorna voro lika stora.
\par 11 Och han mätte portöppningens bredd: den var tio alnar, och portens längd: den var tretton alnar.
\par 12 Och framför vaktkamrarna var en avskrankning, som höll en aln; en aln höll ock avskrankningen på motsatta sidan; och var vaktkammare, på vardera sidan, höll sex alnar.
\par 13 Och han mätte porten från den ena vaktkammarens tak till den andras: den var tjugufem alnar bred; och dörr låg mot dörr.
\par 14 Och han tog upp murpelarna till sextio alnar; och intill förgårdens murpelare sträckte sig porten runt omkring.
\par 15 Och avståndet mellan ingångsportens framsida och förhusets framsida vid den inre portöppningen var femtio alnar.
\par 16 Och slutna fönster funnos till vaktkamrarna och till deras murpelare invändigt i porten runt omkring, och likaledes i förhusen; fönstren sutto runt omkring invändigt, och murpelarna voro prydda med palmer.
\par 17 Och han förde mig till den yttre förgården, och jag såg att där voro tempelkamrar och ett stengolv, anlagt runt omkring förgården; trettio tempelkamrar voro uppbyggda på stengolvet.
\par 18 Och stengolvet gick utefter portarnas sidoväggar, så att det motsvarade portarnas längd; detta var det nedre stengolvet.
\par 19 Och han mätte avståndet från den nedre portens framsida till den inre förgårdens yttre framsida: det var hundra alnar, både på östra sidan och på norra.
\par 20 Sedan mätte han ock längden och bredden på den port som låg mot norr på den yttre förgården.
\par 21 Också den hade tre vaktkamrar på var sida och likaledes murpelare och förhus, lika stora som den förra portens; den var femtio alnar lång och tjugufem alnar bred.
\par 22 Och fönstren, förhuset och palmerna däri voro lika stora som i den port som låg mot öster; och man steg upp till den på sju trappsteg, och dess förhus låg framför dessa.
\par 23 Och en port till den inre förgården fanns mitt emot denna port, det var i norr såsom i öster; och han mätte avståndet från den ena porten till den andra: det var hundra alnar.
\par 24 Därefter lät han mig gå till södra sidan, och jag såg att också på södra sidan fanns en port. Och han mätte dess murpelare och förhus; de voro lika stora som de andra.
\par 25 Och fönster funnos på den och på dess förhus runt omkring, likadana som de andra fönstren. Den var femtio alnar lång och tjugufem alnar bred.
\par 26 Och trappan ditupp utgjordes av sju trappsteg, och dess förhus låg framför dessa; och den var prydd med palmer på sina murpelare, på båda sidor.
\par 27 Och en port till den inre förgården fanns ock på södra sidan; och han mätte avståndet från den ena porten till den andra på södra sidan: det var hundra alnar.
\par 28 Därefter förde han mig till den inre förgården genom södra porten. Och han mätte den södra porten den var lika stor som de andra
\par 29 Och dess vaktkamrar, murpelare och förhus voro lika stora som de andra, och fönster funnos på den och på dess förhus runt omkring. Den var femtio alnar lång och tjugu fem alnar bred.
\par 30 Och förhus funnos runt omkring tjugufem alnar långa och fem alnar breda.
\par 31 Och dess förhus låg utåt den yttre förgården, och dess murpelare voro prydda med palmer; och uppgången därtill utgjordes av åtta trappsteg.
\par 32 Sedan förde han mig till den inre förgårdens östra sida och mätte porten där; den var lika stor som de andra.
\par 33 Och dess vaktkamrar, murpelare och förhus voro lika stora som de andra, och fönster funnos på den och på dess förhus runt omkring. Den var femtio alnar lång och tjugufem alnar bred.
\par 34 Och dess förhus låg mot den yttre förgården, och dess murpelare voro prydda med palmer på båda sidor; och uppgången därtill utgjordes av åtta trappsteg.
\par 35 Därefter förde han mig till den norra porten och mätte den; den var lika stor som de andra.
\par 36 Så ock dess vaktkamrar, murpelare och förhus, och fönster funnos på den runt omkring. Den var femtio alnar lång och tjugufem alnar bred.
\par 37 Och dess murpelare stodo vid den yttre förgården, och dess murpelare voro prydda med palmer på båda sidor; och uppgången därtill utgjordes av åtta trappsteg.
\par 38 Och en tempelkammare med sin ingång fanns vid murpelarna, i portarna; där skulle man skölja brännoffren.
\par 39 Och i portens förhus stodo två bord på var sida, och på dem skulle man slakta brännoffers-, syndoffers- och skuldoffersdjuren.
\par 40 Och vid den yttre sidovägg som låg norrut, när man steg upp till portens ingång, stodo två bord; och vid den andra sidoväggen på porten förhus stodo ock två bord.
\par 41 Alltså stodo vid portens sidoväggar fyra bord på var sida, eller tillsammans åtta bord, på vilka man skulle slakta.
\par 42 Och för brännoffret stodo där fyra bord av huggna stenar, en och en halv aln långa, en och en halv aln breda och en aln höga; på dessa skulle man lägga de redskap som man slaktade brännoffers- och slaktoffersdjuren med.
\par 43 Och dubbelkrokar, en handsbredd långa, voro fästa innantill runt om kring; och på borden skulle offerköttet läggas.
\par 44 Och utanför den inre porten funnos för sångarna, på den inre gården, tempelkamrar, som lågo vid den norra portens sidovägg, med sin framsida åt söder; och en annan låg vid den östra portens sidovägg med sin framsida åt norr.
\par 45 Och han talade till mig: "Denna tempelkammare, vars framsida ligger mot söder, är för de präster som förrätta tjänsten inne i huset.
\par 46 Och den tempelkammare vars framsida ligger mot norr är för de präster som förrätta tjänsten vid altaret, alltså för Sadoks söner, vilka äro de av Levi barn, som få träda fram till HERREN för att göra tjänst inför honom."
\par 47 Och han mätte förgården; den var hundra alnar lång och hundra alnar bred, en liksidig fyrkant; och altaret stod framför huset.
\par 48 Sedan förde han mig till huset förhus. Och han mätte för husets murpelare: de höllo fem alnar var på sin sida, så ock porten bredd: den var på var sida tre alnar.
\par 49 Förhuset var tjugu alnar lång och elva alnar brett, nämligen vid trappstegen på vilka man steg ditupp. Och vid murpelarna stodo pelare, en på var sida.

\chapter{41}

\par 1 Därefter förde han mig till tempelsalen. Och han mätte murpelarna de voro sex alnar breda, var på sin sida - tabernaklets bredd.
\par 2 Och ingången var tio alnar bred, och sidoväggarna vid ingången voro på var sida fem alnar. Sedan mätte han salens längd: den var fyrtio alnar, och dess bredd: den var tjugu alnar.
\par 3 Därefter gick han in i det innersta rummet. Och han mätte murpelarna vid ingången: de höllo två alnar, och ingången: den höll sex alnar, och ingångens bredd: den var sju alnar.
\par 4 Och han mätte dess längd: den var tjugu alnar, och dess bredd: den var tjugu alnar, framför tempelsalen. Och han sade till mig: "Detta är det allraheligaste."
\par 5 Därefter mätte han husets mur: den höll sex alnar, och sidokamrarnas bredd: den var fyra alnar, runt omkring hela huset.
\par 6 Och sidokamrarna lågo den ena ovanför den andra, i trettiotre omgångar; och på den mur som sträckte sig innanför sidokamrarna runt omkring funnos avsatser, på vilka de skulle hava sitt fäste; ty i själva husväggen skulle de icke vara infästa.
\par 7 Härigenom blevo sidokamrarna, där de lågo kring huset, bredare alltefter som de lågo högre upp. Ty husets kringbyggnad sträckte sig med övervåning ovanpå övervåning runt omkring huset. Därför växte bredden inåt, alltefter som våningen låg högre upp. Och från den nedersta våningen steg man så upp i den översta genom den mellersta.
\par 8 Och jag såg att huset låg på en upphöjd fot, som sträckte sig runt däromkring; sidokamrarnas grundvalar voro nämligen en hel stång höga, sex alnar till kanten.
\par 9 Sidokamrarnas yttermur var fem alnar tjock; och den plats som blev fri tillhörde husets sidokamrar.
\par 10 Och mellanrummet bort till tempelkamrarna var tjugu alnar brett runt omkring hela huset.
\par 11 Och ingångarna till sidokamrarna lågo utåt den fria platsen, en ingång mot norr och en ingång åt söder; den fria platsen var fem alnar bred, runt omkring.
\par 12 Och den byggnad som låg invid den avsöndrade platsen på västra sidan var sjuttio alnar bred, och byggnadens mur var fem alnar tjock runt omkring och nittio alnar lång.
\par 13 Och han mätte huset: det var hundra alnar långt. Och den avsöndrade platsen jämte byggnaden med dess murar utgjorde en längd av hundra alnar.
\par 14 Och bredden på husets framsida jämte den avsöndrade platsen åt öster utgjorde en längd av hundra alnar.
\par 15 Och han mätte längden på den byggnad som låg invid den avsöndrade platsen, på dennas baksida, och mätte avsatserna på dess framvägg åt båda sidor - de höllo hundra alnar - vidare det inre tempelrummet och förgårdsförhusen,
\par 16 trösklarna och de slutna fönstren, avsatserna på framväggen runt omkring, i deras tre våningar, platsen invid tröskeln - som var av polerat trä - runt omkring,
\par 17 så ock avståndet från marken upp till fönstren. Och fönstren voro täckta. Men ovanför dörren, både in emot det inre rummet och utåt, och eljest utefter hela väggen runt omkring, innantill och utantill, funnos avmätta fält,
\par 18 på vilka framställdes keruber och palmer, var palm mellan två keruber. Och var kerub hade två ansikten:
\par 19 ett människoansikte åt palmen på ena sidan, och ett lejonansikte åt palmen på andra sidan; så var gjort på hela huset runt omkring.
\par 20 Från marken ända upp över ingången funnos keruber och palmer framställda, så ock på tempelsalens väggar.
\par 21 Tempelsalens dörröppning var fyrkantig, och framsidan av det heligaste hade sitt givna utseende.
\par 22 Altaret var av trä, tre alnar högt och två alnar långt, och det hade hörn; och dess långsidor och väggar voro av trä. Och han talade till mig: "Detta är det bord som skall stå inför HERRENS ansikte."
\par 23 Och både tempelsalen och det heligaste hade dubbeldörrar.
\par 24 Och var dörr hade två dörrskivor, två vridbara dörrskivor: den ena dörren hade två dörrskivor, och likaledes den andra två.
\par 25 Och på dem, på dörrarna till tempelsalen, funnos framställda keruber och palmer, likasom på väggarna; och på förhusets framsida, utantill, var ett trapphus av trä.
\par 26 Och slutna fönster och palmer funnos på förhusets sidoväggar, på båda sidor, så ock i husets sidokamrar och i trapphusen.

\chapter{42}

\par 1 Och han lät mig gå ut på den yttre förgården den väg som gick åt norr, och förde mig därefter till den byggnad med tempelkamrar, som låg invid den avsöndrade platsen och tillika invid murbyggnaden norrut,
\par 2 till långsidan, som mätte hundra alnar, med sin ingång i norr; men bredden var femtio alnar.
\par 3 Ut emot den tjugu alnar breda platsen på den inre förgården och ut emot stengolvet på den yttre förgården lågo avsatserna på det ena husets framvägg mitt emot avsatserna på det andra husets framvägg, i tre våningar.
\par 4 Och framför tempelkamrarna gick en tio alnar bred gång till den inre förgården, en alnsbred väg; och ingångarna lågo mot norr.
\par 5 Men de översta tempelkamrarna voro mindre än de andra, ty avsatserna på framväggen togo bort mer rum från dem än från de nedersta och mellersta kamrarna i byggnaden.
\par 6 Ty kamrarna lågo i tre våningar och hade inga pelare, såsom förgårdarna hade; därför blevo de översta våningens kamrar mer indragna än den nedersta och den mellersta våningens.
\par 7 Och en yttre skiljemur gick utmed tempelkamrarna åt den yttre förgården till, framför tempelkamrarna, och den var femtio alnar lång.
\par 8 Ty längden på tempelkammarbyggnaden utåt den yttre förgården var femtio alnar, men åt templet till hundra alnar.
\par 9 Och nedanför dessa tempelkamrar var ingången från öster, när man ville komma till dem från den yttre förgården.
\par 10 Där förgårdens skiljemur var som tjockast, lågo ock på östra sidan tempelkamrar invid den avsöndrade platsen och tillika invid murbyggnaden.
\par 11 Och en väg gick framför dem, likadan som vägen framför tempelkamrarna på norra sidan; och de hade samma längd och bredd. Och alla utgångar här voro såsom där, både i fråga om övriga anordningar och i fråga om själva dörröppningarna.
\par 12 Och såsom det var med dörröppningarna på tempelkamrarna vid södra sidan, så fanns också har en dörröppning, vid vilken en väg begynte, en väg som gick utefter den behöriga skiljemuren, och som låg österut, när man gick in i tempelkamrarna.
\par 13 Och han sade till mig: "De norra och södra tempelkamrarna invid den avsöndrade platsen skola vara de heliga tempelkamrar i vilka prästerna, som få träda fram inför HERREN, skola äta det högheliga; där skola de förvara det högheliga, såväl spisoffer som syndoffer och skuldoffer ty det är en helig plats.
\par 14 När prästerna - som från helgedomen icke strax få begiva sig till den yttre förgården - hava kommit ditin, skola de där lämna kvar de kläder i vilka de hava gjort tjänst, ty dessa äro heliga; först när de hava iklätt sig andra kläder, få de träda ut på den plats som är för folket."
\par 15 När han nu hade slutat uppmätningen av det inre huset, lät han mig gå ut till den port som låg mot öster. Och han mätte platsen runt omkring.
\par 16 Han mätte med sin mätstång åt östra sidan: den höll efter mätstången fem hundra stänger runt omkring.
\par 17 Han mätte åt norra sidan: den höll efter mätstången fem hundra stänger runt omkring.
\par 18 Han mätte ock åt södra sidan: den höll efter mätstången fem hundra stänger.
\par 19 Han vände sig mot västra sidan och mätte med mätstången fem hundra stänger.
\par 20 Åt alla fyra sidorna mätte han platsen. Den var omgiven av en mur, som utefter platsens längd höll fem hundra stänger och utefter dess bredd fem hundra stänger. Och den skulle skilja det heliga från det som icke var heligt.

\chapter{43}

\par 1 Och han lät mig gå åstad till porten, den port som vette åt öster.
\par 2 Då såg jag Israels Guds härlighet komma österifrån, och dånet därvid var såsom dånet av stora vatten, och jorden lyste av hans härlighet.
\par 3 Och den syn som jag då såg var likadan som den jag såg, när jag kom för att fördärva staden; det var en syn likadan som den jag såg vid strömmen Kebar. Och jag föll ned på mitt ansikte.
\par 4 Och HERRENS härlighet kom in i huset genom den port som låg mot öster.
\par 5 Och en andekraft lyfte upp mig och förde mig in på den inre förgården, och jag såg att HERRENS härlighet uppfyllde huset.
\par 6 Då hörde jag en röst tala till mig från huset, under det att en man stod bredvid mig.
\par 7 Den sade till mig: Du människobarn, detta är den plats där min tron är, den plats där mina fötter skola stå, där jag vill bo ibland Israels barn evinnerligen. Och Israels hus skall icke mer orena mitt eviga namn, varken de själva eller deras konungar, med sin trolösa avfällighet, med sina konungars döda kroppar och med sina offerhöjder:
\par 8 de som satte sin tröskel invid min tröskel, och sin dörrpost vid sidan av min dörrpost, så att allenast muren var mellan mig och dem, och som så orenade mitt heliga namn med de styggelser de bedrevo, varför jag ock förgjorde dem i min vrede.
\par 9 Men nu skola de skaffa sin trolösa avfällighet och sina konungars döda kroppar långt bort ifrån mig, så att jag kan bo ibland dem evinnerligen.
\par 10 Men du, människobarn, förkunna nu för Israels barn om detta hus, på det att de må blygas för sina missgärningar. Må de mäta det härliga byggnadsverket.
\par 11 Om de då blygas för allt vad de hava gjort, så kungör för dem och teckna för deras ögon upp husets form och inredning, dess utgångar och ingångar, alla dess former och alla stadgar därom, alla dess former och alla lagar därom, så att de akta på hela dess form och alla stadgar därom och göra efter dem.
\par 12 Detta är lagen om huset: på toppen av berget skall hela dess område runt omkring vara högheligt. Ja, detta är lagen om huset.
\par 13 Men dessa voro måtten på altaret i alnar, var aln en handsbredd längre an en vanlig aln: Dess bottenram var en aln hög och en aln bred, och kanten på ramen, runt omkring utmed randen, var ett kvarter hög; detta var altarets underlag.
\par 14 Avståndet från bottenramen vid marken upp till den nedre avsatsen var två alnar, och bredden var en aln. Avståndet från den mindre avsatsen upp till den större var fyra alnar, och bredden var en aln.
\par 15 Altarhärden höll fyra alnar; och från altarhärden stodo de fyra hornen uppåt.
\par 16 Och altarhärden var tolv alnar lång och tolv alnar bred, så att dess fyra sidor bildade en liksidig fyrkant.
\par 17 Och avsatsen var fjorton alnar lång och fjorton alnar bred, utefter sina fyra sidor. Kanten runt omkring den höll en halv aln, och dess bottenram sträckte sig en aln runt omkring. Och altarets trappsteg vette åt öster.
\par 18 Och han sade till mig: "Du människobarn, så säger Herren, HERREN: Dessa äro stadgarna om altaret för den dag då det bliver färdigt, så att man kan offra brännoffer och stänka blod därpå.
\par 19 Då skall du åt de levitiska prästerna giva en ungtjur till syndoffer, åt dem som äro av Sadoks säd, och som få träda fram till mig, säger Herren, HERREN, för att göra tjänst inför mig.
\par 20 Och du skall taga något av dess blod och stryka på altarets fyra hörn och på avsatsens fyra hörn och på kanten runt omkring; så skall du rena det och bringa försoning för det.
\par 21 Sedan skall du taga syndofferstjuren, och utanför helgedomen skall den brännas upp på en därtill bestämd plats, som hör till huset.
\par 22 Och nästa dag skall du föra fram en felfri bock till syndoffer; och man skall rena altaret med den, på samma sätt som man renade det med tjuren.
\par 23 När du så har fullbordat reningen, skall du föra fram en felfri ungtjur och en felfri vädur av småboskapen.
\par 24 Dem skall du föra fram inför HERREN, och prästerna skola strö salt på dem och offra dem såsom brännoffer åt HERREN.
\par 25 Under sju dagar skall du dagligen offra en syndoffersbock; och en ungtjur och en vädur av småboskapen, båda felfria, skall man likaledes offra.
\par 26 Under sju dagar skall man sålunda bringa försoning för altaret och rena det och inviga det.
\par 27 Men sedan dessa dagar hava gått till ända, skola prästerna på åttonde dagen och allt framgent offra på altaret edra brännoffer och tackoffer; och jag skall då hava behag till eder, säger Herren, HERREN".

\chapter{44}

\par 1 Därefter förde han mig tillbaka mot helgedomens yttre port, den kom vette åt öster; den var nu stängd.
\par 2 Och HERREN sade till mig: "Denna port skall förbliva stängd och icke mer öppnas, och ingen skall gå in genom den, ty HERREN, Israels Gud, har gått in genom den; därför skall den vara stängd.
\par 3 Dock skall fursten, eftersom han är furste, få sitta där och hålla måltid inför HERRENS ansikte; han skall då gå in genom portens förhus, och samma väg skall han gå ut igen."
\par 4 Därefter förde han mig genom norra porten till platsen framför huset; och jag fick se huru HERRENS härlighet uppfyllde HERRENS hus. Då föll jag ned på mitt ansikte.
\par 5 och HERREN sade till mig: Du människobarn, akta på och se med dina ögon, och hör med dina öron allt vad jag nu talar med dig om alla stadgar angående HERRENS hus och om alla lagar som röra det; och giv akt på huru man går in i huset genom alla helgedomens utgångar.
\par 6 Och säg till Israels hus, det gensträviga: Så säger Herren, HERREN: Nu må det vara nog med alla de styggelser I haven bedrivit, I av Israels hus,
\par 7 I som haven låtit främlingar med oomskuret hjärta och oomskuret kött komma in i min helgedom och vara där, så att mitt hus har blivit ohelgat, under det att I framburen min spis, fett och blod. Så har mitt förbund blivit brutet, för att icke nämna alla edra andra styggelser.
\par 8 I haven icke själva förrättat tjänsten vid mina heliga föremål, utan haven satt andra till att åt eder förrätta tjänsten i min helgedom.
\par 9 Så säger Herren, HERREN: Ingen främling med oomskuret hjärta och oomskuret kött får komma in i min helgedom, ingen av de främlingar som finnas bland Israels barn.
\par 10 Utan de leviter som gingo bort ifrån mig, när Israel for vilse de som då själva foro vilse och gingo bort ifrån mig och följde sina eländiga avgudar, de skola bära på sin missgärning
\par 11 och skola i min helgedom bestrida vakttjänstgöringen vid husets portar och annan tjänstgöring i huset; de skola slakta brännoffer och slaktoffer åt folket, och skola stå inför dem till att betjäna dem.
\par 12 Eftersom de betjänade dem inför deras eländiga avgudar och så blevo för Israels hus en stötesten till missgärning, därför betygar jag om dem med upplyft hand, säger Herren, HERREN, att de skola få bära på sin missgärning.
\par 13 De skola icke få nalkas mig, till att förrätta prästerlig tjänst inför mig, eller till att nalkas något av mina heliga föremål, nämligen de högheliga, utan de skola bära på sin skam och på de styggeliga synder som de hava bedrivit.
\par 14 Och jag skall sätta dem till att förrätta tjänsten i huset vid allt tjänararbete där, allt som där skall utföras.
\par 15 Men de levitiska präster, nämligen Sadoks söner, som förrättade tjänsten vid min helgedom, när de övriga israeliterna foro vilse och gingo bort ifrån mig, de skola få träda fram till mig för att göra tjänst inför mig; de skola stå inför mitt ansikte för att offra åt mig fett och blod, säger Herren, HERREN.
\par 16 De skola gå in i min helgedom, och de skola träda fram till mitt bord för att göra tjänst inför mig och förrätta vad som är att förrätta åt mig.
\par 17 Och när de komma in i den inre förgårdens portar, skola de ikläda sig linnekläder; de få icke hava på sig något av ylle, när de göra tjänst i den inre förgårdens portar och inne i huset.
\par 18 De skola hava huvudbonader av linne på sina huvuden, och benkläder av linne omkring sina länder; de skola icke omgjorda sig med något som framkallar svett.
\par 19 Och när de sedan gå ut på den yttre förgården, till folket på den yttre förgården, skola de taga av sig de kläder i vilka de hava gjort tjänst, och skola lämna dem kvar i helgedomens tempelkamrar och ikläda sig andra kläder, för att de icke må göra folket heligt med sina kläder.
\par 20 De skola icke raka huvudet, men skola icke heller låta håret växa fritt, utan skola klippa sitt huvudhår kort.
\par 21 Och vin får ingen präst dricka, när han har kommit in på den inre förgården.
\par 22 En änka eller en frånskild kvinna får han icke taga till hustru åt sig, utan allenast en jungfru av Israels barns släkt; dock får han taga en änka, om hon är änka efter en präst.
\par 23 Och de skola lära mitt folk att skilja mellan heligt och oheligt och undervisa dem om skillnaden mellan orent och rent.
\par 24 Och i rättssaker skola de uppträda såsom domare och skola avdöma dem efter mina rätter. Och mina lagar och stadgar skola de iakttaga vid alla mina högtider, och mina sabbater skola de hålla heliga.
\par 25 Ingen av dem får orena sig genom att gå in till någon död människa; allenast genom fader eller moder eller son eller dotter eller broder, eller genom en syster som icke har tillhört någon man må han ådraga sig orenhet.
\par 26 Men när han därefter har blivit ren, skall man räkna för honom ytterligare sju dagar;
\par 27 och på den dag då han går in i helgedomen, på den inre förgården, för att göra tjänst i helgedomen, då skall han bära fram ett syndoffer för sig, säger Herren, HERREN.
\par 28 Och deras arvedel skall bestå däri att jag själv skall vara deras arvedel. Och I skolen icke giva dem någon besittning i Israel, ty jag själv är deras besittning.
\par 29 Spisoffret och syndoffret och skuldoffret få de äta, och allt tillspillogivet i Israel skall höra dem till.
\par 30 Och det första av alla förstlingsfrukter av alla slag, och alla offergärder av alla slag, vadhelst I frambären såsom offergärd, detta skall höra prästerna till; och förstlingen av edert mjöl skolen I giva åt prästen, för att du må bringa välsignelse över ditt hus.
\par 31 Intet självdött eller ihjälrivet djur, vare sig fågel eller boskapsdjur, få prästerna äta.

\chapter{45}

\par 1 Och när I genom lottkastning fördelen landet till arvedel, då skolen I åt HERREN giva en offergärd, en helig del av landet, i längd tjugufem tusen alnar och i bredd tio tusen; detta stycke skall vara heligt till hela sitt omfång runt omkring.
\par 2 Härav skall tagas till helgedomen en liksidig fyrkant, fem hundra alnar i längd och fem hundra i bredd, runt omkring, och till utmark där runt omkring femtio alnar.
\par 3 Av det tillmätta stycket skall du alltså avmäta ett område, tjugufem tusen alnar i längd och tio tusen i bredd; där skall helgedomen, det högheliga, ligga.
\par 4 Detta skall vara en helig del av landet, och det skall tillhöra prästerna, som göra tjänst i helgedomen, dem som få träda fram till att göra tjänst inför HERREN; detta skall vara en plats åt dem för deras hus, så ock en helig plats för helgedomen.
\par 5 Och ett stycke, tjugufem tusen alnar i längd och tio tusen i bredd, skall tillhöra leviterna, som göra tjänst i huset, såsom deras besittning, med tjugu tempelkamrar.
\par 6 Och åt staden skolen I giva till besittning ett område, fem tusen alnar i bredd och tjugufem tusen i längd, motsvarande det heliga offergärdsområdet; det skall höra hela Israels hus till.
\par 7 Och fursten skall på båda sidor om det heliga offergärdsområdet och stadens besittning få ett område, beläget invid det heliga offergärdsområdet och stadens besittning, dels på västra sidan, västerut, dels ock på östra sidan, österut, och I längd motsvarande en stamlotts utsträckning från västra gränsen till östra.
\par 8 Detta skall han hava till sitt land, till besittning i Israel. Och mina furstar skola då icke mer förtrycka mitt folk, utan skola låta Israels hus få behålla sitt land efter sina stammar.
\par 9 Så säger Herren, HERREN: Nu må det vara nog, I Israels furstar. Skaffen bort våld och förtryck, och öven rätt och rättfärdighet; hören upp att driva mitt folk ifrån hus och hem, säger Herren, HERREN.
\par 10 Riktig våg, riktig efa, riktigt bat-mått skolen I hava.
\par 11 Efan och bat-måttet skola hålla samma mått, så att bat-måttet rymmer tiondedelen av en homer, och likaledes efan tiondedelen av en homer; ty efter homern skall man bestämma måtten.
\par 12 Sikeln skall innehålla tjugu gera; tjugu siklar, tjugufem siklar, femton siklar skall minan innehålla hos eder.
\par 13 Detta är den offergärd I skolen giva: en sjättedels efa av var homer vete och en sjättedels efa av var homer korn;
\par 14 vidare den stadgade gärden av olja, räknat efter bat av olja: en tiondedels bat av var kor (som är ett mått på tio bat och lika med en homer, ty tio bat utgöra en homer);
\par 15 vidare av småboskapen från Israels betesmarker ett djur på vart tvåhundratal, till spisoffer, brännoffer och tackoffer, för att bringa försoning för folket, säger Herren, HERREN.
\par 16 Allt folket i landet skall vara förpliktat till denna offergärd åt fursten i Israel.
\par 17 Men fursten skall det åligga att frambära brännoffer, spisoffer och drickoffer på festerna, nymånaderna och sabbaterna, vid alla Israels hus' högtider. Han skall anskaffa syndoffer, spisoffer, brännoffer och tackoffer till att bringa försoning för Israels hus.
\par 18 Så säger Herren, HERREN: På första dagen i första månaden skall du taga en felfri ungtjur och rena helgedomen.
\par 19 Och prästen skall taga något av syndoffrets blod och stryka på husets dörrpost och på altaravsatsens fyra hörn och på dörrposten till den inre förgårdens port.
\par 20 Så skall du ock göra på sjunde dagen i månaden, om så är, att någon har syndat ouppsåtligen och av fåkunnighet; på detta sätt skolen I bringa försoning för huset.
\par 21 På fjortonde dagen i första månaden skolen I fira påskhögtid; i sju dagar skolen I hålla högtid, och man skall då äta osyrat bröd.
\par 22 På den dagen skall fursten för sig själv och för allt folket i landet offra en tjur till syndoffer.
\par 23 Men sedan, under högtidens sju dagar, skall han dagligen under de sju dagarna offra såsom brännoffer åt HERREN sju tjurar och sju vädurar, alla felfria, och såsom syndoffer en bock dagligen.
\par 24 Och såsom spisoffer skall han offra en efa till var tjur och en efa till var vädur, jämte en hin olja till var efa.
\par 25 På femtonde dagen i sjunde månaden skall han vid högtiden frambära likadana offer under de sju dagarna, likadana syndoffer, brännoffer och spisoffer och lika mycket olja.

\chapter{46}

\par 1 Så säger Herren, HERREN: Den inre förgårdens port, den som vetter åt öster, skall vara stängd under de sex arbetsdagarna, men på sabbatsdagen skall den öppnas; likaledes skall den öppnas på nymånadsdagen.
\par 2 Och då skall fursten utifrån gå in genom portens förhus och ställa sig vid portens dörrpost; och när prästerna offra hans brännoffer och hans tackoffer, skall han tillbedja på portens tröskel och därefter gå ut. Men porten skall icke stängas förren om aftonen.
\par 3 Och folket i landet skall på sabbater och nymånader tillbedja inför HERREN vid ingången till samma port.
\par 4 Och brännoffret som fursten skall frambära åt HERREN skall på sabbatsdagen utgöras av sex felfria lamm och en felfri vädur.
\par 5 Och såsom spisoffer skall han frambära en efa till väduren, men till lammen såsom spisoffer så mycket han vill giva, jämte en hin olja till var efa.
\par 6 Men på nymånadsdagen skall han frambära en felfri ungtjur, sex lamm och en vädur, allasammans felfria.
\par 7 Och såsom spisoffer skall han offra en efa till tjuren och en efa till väduren, och till lammen så mycket han vill anskaffa, jämte en hin olja till var efa.
\par 8 Och när fursten vill gå in, skall han gå in genom portens förhus, och samma väg skall han gå ut igen.
\par 9 Men när folket i landet kommer inför HERRENS ansikte vid högtiderna, då skall den som har gått in genom norra porten för att tillbedja gå ut genom södra porten, och den som har gått in genom södra porten skall gå ut genom norra porten; ingen skall gå tillbaka genom samma port som han har kommit in igenom, utan man skall gå ut genom den motsatta.
\par 10 Och fursten skall gå in tillsammans med de andra, när de gå in; och när de gå ut, skola de gå ut tillsammans.
\par 11 Men vid fester och högtider skall spisoffret utgöras av en efa till var tjur och en efa till var vädur, och till lammen av så mycket han vill giva, jämte en hin olja till var efa.
\par 12 Och när fursten vill offra ett frivilligt offer, vare sig ett brännoffer eller ett tackoffer såsom frivilligt offer åt HERREN, då skall man öppna åt honom den port som vetter åt öster, och han skall offra sitt brännoffer och sitt tackoffer alldeles så, som han plägar offra på sabbatsdagen; och därefter skall han gå ut och sedan han har gått ut, skall man stänga porten.
\par 13 Du skall dagligen offra såsom brännoffer åt HERREN ett felfritt årsgammalt lamm; var morgon skall du offra ett sådant.
\par 14 Och såsom spisoffer skall du därtill offra var morgon en sjättedels efa, så ock en tredjedels hin olja för att fukta mjölet - detta såsom spisoffer åt HERREN, såsom evärdlig rätt för beständigt.
\par 15 I skolen offra lammet och spisoffret och oljan var morgon såsom dagligt brännoffer.
\par 16 Så säger Herren, HERREN: Om fursten giver någon av sina söner en gåva, så bliver det dennes arvedel, det skall höra hans söner till; de skola besitta det såsom arv.
\par 17 Men om han av sin arvedel giver något såsom gåva åt någon av sina tjänare, så skall detta tillhöra denne intill friåret; då skall det återgå till fursten. Hans arvedel är det ju, och hans söner skall det tillfalla.
\par 18 Fursten må icke taga något av folkets arvedel och så kränka den i deras besittningsrätt; allenast av sin egen besittning må han giva arvedelar åt sina söner, för att ingen av mitt folk skall bliva undanträngd från sin särskilda besittning.
\par 19 Och han förde mig genom den ingång som låg vid sidan av porten till de heliga tempelkamrar som voro bestämda för prästerna, och som vette åt norr; och jag såg att där var en plats längst uppe i väster.
\par 20 Och han sade till mig: "Detta är den plats där prästerna skola koka skuldoffret och syndoffret, och där de skola baka spisoffret, för att icke behöva bära ut det på den yttre förgården och så göra folket heligt."
\par 21 Därefter lät han mig gå ut på den yttre förgården och förde mig omkring till förgårdens fyra hörn; och jag såg då att i vart och ett av förgårdens hörn fanns en gård.
\par 22 I förgårdens fyra hörn funnos kringstängda gårdar, fyrtio alnar långa och trettio alnar breda; dessa fyra hörngårdar voro lika stora.
\par 23 Och runt omkring inuti dem gick en mur, runt omkring i alla fyra; och nedtill vid muren runt omkring hade man inrättat eldstäder till kokning.
\par 24 Och han sade till mig: "Detta är de kök i vilka husets tjänare skola koka folkets slaktoffer."

\chapter{47}

\par 1 Därefter förde han mig tillbaka till husets ingång, och där fick jag se vatten rinna fram under husets tröskel på östra sidan, ty husets framsida låg mot öster; och vattnet flöt ned under husets södra sidovägg, söder om altaret.
\par 2 Sedan lät han mig gå ut genom norra porten och förde mig omkring på en yttre väg till den yttre porten, den som vette åt öster. Där fick jag se vatten välla fram på södra sidan.
\par 3 Sedan gick mannen, med ett mätsnöre i handen, ett stycke mot öster och mätte därvid upp tusen alnar och lät mig så gå över vattnet, och vattnet räckte mig där till fotknölarna.
\par 4 Åter mätte han upp tusen alnar och lät mig så gå över vattnet, och vattnet räckte mig där till knäna. Åter mätte han upp tusen alnar och lät mig så gå över vattnet, som där räckte mig upp till länderna.
\par 5 Ännu en gång mätte han upp tusen alnar, och nu var det en ström som jag icke kunde gå över. Ty vattnet gick så högt att man måste simma; det var en ström som man icke kunde gå över.
\par 6 Och han sade till mig: "Nu har du ju sett det, du människobarn?" Sedan förde han mig tillbaka upp på strömmens strand.
\par 7 Och när han hade fört mig dit tillbaka, fick jag se träd i stor myckenhet stå på strömmens strand, på båda sidor.
\par 8 Och han sade till mig: "Detta vatten rinner fram mot Östra kretsen och flyter ned på Hedmarken och faller därefter ut i havet. Vattnet som fick bryta fram går alltså till havet, och så bliver vattnet där sunt.
\par 9 Och överallt dit den dubbla strömmen kommer, där upplivas alla levande varelser som röra sig i stim, och fiskarna bliva där mycket talrika; ty när detta vatten kommer dit, bliver havsvattnet sunt, och allt får liv, där strömmen kommer.
\par 10 Och fiskare skola stå utmed den från En-Gedi ända till En-Eglaim, och såsom ett enda fiskeläge skall den sträckan vara. Där skola finnas fiskar av olika slag i stor myckenhet, alldeles såsom i Stora havet.
\par 11 Men gölar och dammar där skola icke bliva sunda, utan skola tjäna till saltberedning.
\par 12 Och vid strömmen, på dess båda stränder, skola allahanda fruktträd växa upp, vilkas löv icke skola vissna, och vilkas frukt icke skall taga slut, utan var månad skola träden bara ny frukt, ty deras vatten kommer från helgedomen. Och deras frukter skola tjäna till föda och deras löv till läkedom."
\par 13 Så säger Herren, HERREN: Dessa äro de gränser efter vilka I skolen utskifta landet såsom arvedel åt Israels tolv stammar (varvid Josef får mer än en lott).
\par 14 I skolen få det till arvedel, den ene såväl som den andre, därför att jag med upplyft hand har lovat att giva det åt edra fader; så skall nu detta land tillfalla eder såsom arvsegendom.
\par 15 Detta skall vara landets gräns på norra sidan: från Stora havet längs Hetlonsvägen, dit fram där vägen går till Sedad,
\par 16 Hamat, Berota, Sibraim, som ligger mellan Damaskus' och Hamat områden, det mellersta Haser, som ligger invid Haurans område.
\par 17 Så skall gränsen gå från havet till Hasar-Enon vid Damaskus' område och vidare allt längre norrut och upp mot Hamats område. Detta är norra sidan.
\par 18 Och på östra sidan skall gränsen begynna mellan Hauran och Damaskus och gå mellan Gilead och Israels land och utgöras av Jordan; från nordgränsen nedåt, utmed Östra havet, skolen I mäta ut den. Detta är östra sidan.
\par 19 Och på sydsidan, söderut, skall gränsen gå från Tamar till Meribots vatten vid Kades, till bäcken, fram till Stora havet. Detta är sydsidan, söderut.
\par 20 Och på västra sidan skall gränsen utgöras av Stora havet och gå från sydgränsen till en punkt mitt emot det ställe där vägen går till Hamat. Detta är västra sidan.
\par 21 Och I skolen utskifta detta land åt eder efter Israels stammar.
\par 22 I skolen utdela det genom lottkastning till arvedel åt eder själva och åt främlingarna som bo ibland eder och hava fött barn ibland eder. Ty de skola av eder hållas lika med infödda israeliter; de skola tillfalla eder såsom en arvedel bland Israels stammar.
\par 23 I den stam där främlingen bor, där skolen I giva honom hans arvedel, säger Herren, HERREN.

\chapter{48}

\par 1 Och dessa äro namnen på stammarna. Vid norra gränsen längs efter Hetlonsvägen, dit fram där vägen går till Hamat, vidare bort mot Hasar-Enan - med Damaskus' område jämte Hamat i norr - där skall Dan hava en lott, så att hela sträckan från östra sidan till västra tillhör honom.
\par 2 Och närmast Dans område skall Aser hava en lott, från östra sidan till västra.
\par 3 Och närmast Asers område skall Naftali hava en lott, från östra sidan till västra.
\par 4 Och närmast Naftalis område skall Manasse hava en lott, från östra sidan till västra.
\par 5 Och närmast Manasses område skall Efraim hava en lott, från östra sidan till västra.
\par 6 Och närmast Efraims område skall Ruben hava en lott, från östra sidan till västra.
\par 7 Och närmast Rubens område skall Juda hava en lott, från östra sidan till västra.
\par 8 Och närmast Juda område skall från östra sidan till västra sträcka sig det offergärdsområde som I skolen giva såsom gärd, tjugufem tusen alnar i bredd, och i längd lika med en stamlotts längd från östra sidan till västra; och helgedomen skall ligga där i mitten.
\par 9 Det offergärdsområde som I skolen giva såsom gärd åt HERREN skall vara i längd tjugufem tusen alnar och i bredd tio tusen.
\par 10 Och av detta heliga offergärdsområde skall ett stycke tillhöra prästerna, i norr tjugufem tusen alnar, i väster tio tusen i bredd, i öster likaledes tio tusen i bredd och i söder tjugufem tusen i längd; och HERRENS helgedom skall ligga där i mitten.
\par 11 Det skall tillhöra prästerna, dem som hava blivit helgade bland Sadoks söner, dem som hava förrättat tjänsten åt mig, och som icke, såsom leviterna gjorde, foro vilse, när de övriga israeliterna foro vilse.
\par 12 Därför skall en särskild offergärdsdel av den från landet avtagna offergärden tillhöra dem såsom ett högheligt område invid leviternas.
\par 13 Men leviterna skola få ett område motsvarande prästernas, i längd tjugufem tusen alnar och i bredd tiotusen - längden överallt tjugufem tusen och bredden tio tusen.
\par 14 Och de få icke sälja något därav; det bästa landet må man icke byta bort eller eljest överlåta åt någon annan, ty det är helgat åt HERREN.
\par 15 Men de fem tusen alnar som bliva över på bredden invid de tjugufem tusen skola utgöra ett icke heligt område för staden, dels till att bo på, dels såsom utmark; och staden skall ligga där i mitten.
\par 16 Och detta är måttet på den: norra sidan fyra tusen fem hundra alnar, södra sidan fyra tusen fem hundra, på östra sidan fyra tusen fem hundra, och västra sidan fyra tusen fem hundra.
\par 17 Och staden skall hava en utmark, som norrut är två hundra femtio alnar, söderut två hundra femtio, österut två hundra femtio och västerut två hundra femtio.
\par 18 Och vad som bliver över på långsidan utmed det heliga offergärdsområdet, nämligen tio tusen alnar österut och tio tusen västerut - ty det skall sträcka sig utmed det heliga offergärdsområdet - av detta skall avkastningen tjäna till föda åt stadens bebyggare.
\par 19 Alla stadens bebyggare från alla Israels stammar skola bruka det.
\par 20 Hela offergärdsområdet skall alltså vara tjugufem tusen alnar i längd och tjugufem tusen i bredd; det heliga offergärdsområde som I given såsom gärd skall bilda en fyrkant, stadens besittning inberäknad.
\par 21 Och fursten skall få vad som bliver över på båda sidor om det heliga offergärdsområdet och stadens besittning, nämligen landet invid det tjugufem tusen alnar breda offergärdsområdet, ända till östra gränsen, och likaledes västerut landet utefter det tjugufem tusen alnar breda området, ända till västra gränsen. Dessa områden, motsvarande stamlotterna, skola tillhöra fursten. Och det heliga offergärdsområdet med det heliga huset skall ligga mitt emellan dem.
\par 22 Med sin gräns å ena sidan mot leviternas besittning, å andra sidan mot stadens, skall detta område ligga mitt emellan furstens besittningar. Och furstens besittningar skola ligga mellan Juda område och Benjamins område.
\par 23 Därefter skola de återstående stammarna komma. Först skall Benjamin hava en lott från östra sidan till västra.
\par 24 Och närmast Benjamins område skall Simeon hava en lott, från östra sidan till västra.
\par 25 Och närmast Simeons område skall Isaskar hava en lott, från östra sidan till västra.
\par 26 Och närmast Isaskars område skall Sebulon hava en lott, från östra sidan till västra.
\par 27 Och närmast Sebulons område skall Gad hava en lott, från östra sida till västra.
\par 28 Och närmast Gads område, på dess sydsida, söderut, skall gränser gå från Tamar över Meribas vatten vid Kades till bäcken, fram emot Stora havet.
\par 29 Detta är det land som I genom lottkastning skolen utdela åt Israels stammar till arvedel; och detta skall vara deras stamlotter, säger Herren, HERREN.
\par 30 Och följande utgångar skall staden hava: På norra sidan skall den hålla ett mått av fyra tusen fem hundra alnar,
\par 31 och av stadens portar, uppkallade efter Israels stammars namn, skola tre ligga i norr: den första Rubens port, den andra Juda port, den tredje Levi port.
\par 32 Och på östra sidan skall den ock. hålla fyra tusen fem hundra alnar och hava tre portar: den första Josefs port, den andra Benjamins port, den tredje Dans port.
\par 33 Sammalunda skall ock södra sidan hålla ett mått av fyra tusen fem hundra alnar och hava tre portar: den första Simeons port, den andra Isaskars port, den tredje Sebulons port.
\par 34 Västra sidan skall hålla fyra tusen fem hundra alnar och hava tre portar: den första Gads port, den andra Asers port, den tredje Naftali port.
\par 35 Runt omkring skall den hålla aderton tusen alnar. Och stadens namn skall allt framgent vara: Här är HERREN.


\end{document}