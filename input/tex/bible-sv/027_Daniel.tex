\begin{document}

\title{Daniel}

Dan 1:1  I Jojakims, Juda konungs, tredje regeringsår kom Nebukadnessar, konungen i Babel, mot Jerusalem och belägrade det.
Dan 1:2  Och Herren gav Jojakim, Juda konung, i hans hand, så ock en del av kärlen i Guds hus; och han förde dem till Sinears land, in i sin guds hus. Och kärlen förde han in i sin guds skattkammare.
Dan 1:3  Och konungen befallde Aspenas, sin överste hovman, att han skulle av Israels barn taga till sig ynglingar av konungslig släkt eller av förnäm börd,
Dan 1:4  sådana som icke hade något lyte, utan voro fagra att skåda och utrustade med förstånd till att inhämta allt slags visdom, kloka och läraktiga ynglingar, som kunde bliva dugliga att tjäna i konungens palats; dem skulle han låta undervisa i kaldéernas skrift och tungomål.
Dan 1:5  Och konungen bestämde åt dem ett visst underhåll för var dag, av konungens egen mat och av det vin han själv drack, och befallde att man skulle uppfostra dem i tre år; när den tiden vore förliden, skulle de få göra tjänst hos konungen.
Dan 1:6  Bland dessa voro nu Daniel, Hananja, Misael och Asarja, av Juda barn.
Dan 1:7  Men överste hovmannen gav dem andra namn: Daniel kallade han Beltesassar, Hananja Sadrak, Misael Mesak och Asarja Abed-Nego.
Dan 1:8  Men Daniel lät sig angeläget vara att icke orena sig med konungens mat eller med vinet som denne drack av; och han bad överste hovmannen att han icke skulle nödgas orena sig.
Dan 1:9  Och Gud lät Daniel finna nåd och barmhärtighet inför överste hovmannen.
Dan 1:10  Men överste hovmannen sade till Daniel: "Jag fruktar att min herre, konungen, som har beställt om eder mat och dryck, då skall finna edra ansikten magrare än de ynglingars som äro jämnåriga med eder, och att I så skolen draga skuld över mitt huvud inför konungen."
Dan 1:11  Då sade Daniel till hovmästaren som av överste hovmannen hade blivit satt till att hava uppsikt över Daniel, Hananja, Misael och Asarja:
Dan 1:12  "Gör ett försök med dina tjänare i tio dagar, och låt giva oss grönsaker att äta och vatten att dricka.
Dan 1:13  Sedan må du jämföra vårt utseende med de ynglingars som hava ätit av konungens mat; och efter vad du då anser må du göra med dina tjänare."
Dan 1:14  Och han lyssnade till denna deras begäran och gjorde ett försök med dem i tio dagar.
Dan 1:15  Och efter de tio dagarnas förlopp befunnos de vara fagrare att skåda och stadda vid bättre hull än alla de ynglingar som hade ätit av konungens mat.
Dan 1:16  Då lät hovmästaren dem allt fortfarande slippa den mat som hade varit bestämd för dem och det vin som de skulle hava druckit, och gav dem grönsaker.
Dan 1:17  Åt dessa fyra ynglingar gav nu Gud kunskap och insikt i allt slags skrift och visdom; och Daniel fick förstånd på alla slags syner och drömmar.
Dan 1:18  Och när den tid var förliden, efter vilken de, enligt konungens befallning, skulle föras fram för honom, blevo de av överste hovmannen förda inför Nebukadnessar
Dan 1:19  När då konungen talade med dem, fanns bland dem alla ingen som kunde förliknas med Daniel, Hananja, Misael och Asarja; och de fingo så göra tjänst hos konungen.
Dan 1:20  Och närhelst konungen tillfrågade dem i en sak som fordrade vishet i förståndet, fann han dem vara tio gånger klokare än någon av de spåmän och besvärjare som funnos i hela hans rike.
Dan 1:21  Och Daniel fortfor så intill konung Kores' första regeringsår.
Dan 2:1  I sitt andra regeringsår hade Nebukadnessar drömmar av vilka han blev orolig till sinnes, och sömnen vek bort ifrån honom.
Dan 2:2  Då lät konungen tillkalla sina spåmän, besvärjare, trollkarlar och kaldéer, för att de skulle giva konungen till känna vad han hade drömt, och de kommo och trädde fram för konungen.
Dan 2:3  Och konungen sade till dem: "Jag har haft en dröm, och jag är orolig till sinnes och ville veta vad jag har drömt."
Dan 2:4  Då talade kaldéerna till konungen på arameiska: "Må du leva evinnerligen, o konung! Förtälj drömmen för dina tjänare, så skola vi meddela uttydningen."
Dan 2:5  Konungen svarade och sade till kaldéerna: "Nej, mitt oryggliga beslut är, att om I icke sägen mig drömmen och dess uttydning, skolen I huggas i stycken, och edra hus skola göras till platser för orenlighet.
Dan 2:6  Men om I meddelen drömmen och dess uttydning, så skolen I få gåvor och skänker och stor ära av mig. Meddelen mig alltså nu drömmen och dess uttydning."
Dan 2:7  De svarade för andra gången och sade: "Konungen må förtälja drömmen för sina tjänare, så skola vi meddela uttydningen."
Dan 2:8  Konungen svarade och sade: "Jag märker nogsamt att I viljen vinna tid, eftersom I sen att mitt beslut är oryggligt,
Dan 2:9  att om I icke sägen mig drömmen, domen över eder icke kan bliva annat än en. Ja, I haven kommit överens om att inför mig föra lögnaktigt och bedrägligt tal, i hopp att tiderna skola förändra sig. Sägen mig alltså nu vad jag har drömt, så märker jag att I ock kunnen meddela mig uttydningen därpå."
Dan 2:10  Då svarade kaldéerna konungen och sade: "Det finnes ingen människa på jorden, som förmår meddela konungen det som han vill veta; aldrig har ju heller någon konung, huru stor och mäktig han än var, begärt sådant som detta av någon spåman eller besvärjare eller kaldé.
Dan 2:11  Ty det som konungen begär är alltför svårt, och ingen finnes, som kan meddela konungen det, förutom gudarna; och de hava icke sin boning ibland de dödliga.
Dan 2:12  Då blev konungen vred och mycket förtörnad och befallde att man skulle förgöra alla de vise i Babel.
Dan 2:13  När alltså påbudet härom hade blivit utfärdat och man skulle döda de vise, sökte man ock efter Daniel och hans medbröder för att döda dem.
Dan 2:14  Då vände sig Daniel med kloka och förståndiga ord till Arjok, översten för konungens drabanter, vilken hade dragit ut för att döda de vise i Babel.
Dan 2:15  Han tog till orda och frågade Arjok, konungens hövitsman: "Varför har detta stränga påbud blivit utfärdat av konungen?" Då omtalade Arjok för Daniel vad som var på färde.
Dan 2:16  Och Daniel gick in och bad konungen att tid måtte beviljas honom, så skulle han meddela konungen uttydningen.
Dan 2:17  Därefter gick Daniel hem och omtalade för Hananja, Misael och Asarja, sina medbröder, vad som var på färde,
Dan 2:18  och han uppmanade dem att bedja himmelens Gud om förbarmande, så att denna hemlighet bleve uppenbarad, på det att icke Daniel och hans medbröder måtte förgöras tillika med de övriga vise Babel.
Dan 2:19  Då blev hemligheten uppenbarad för Daniel i en syn om natten. Och Daniel lovade himmelens Gud därför;
Dan 2:20  Daniel hov upp sin röst och sade: "Lovat vare Guds namn från evighet till evighet! Ty vishet och makt höra honom till.
Dan 2:21  Han låter tider och stunder omskifta, han avsätter konungar och tillsätter konungar, han giver åt de visa deras vishet och åt de förståndiga deras förstånd.
Dan 2:22  Han uppenbarar det som är djupt och förborgat, han vet vad i mörkret är, och hos honom bor ljuset.
Dan 2:23  Dig, mina fäders Gud, tackar och prisar jag för att du har givit mig vishet och förmåga, och för att du nu har uppenbarat för mig det vi bådo dig om; ty det som konungen ville veta har du uppenbarat för oss."
Dan 2:24  I följd härav gick Daniel in till Arjok, som av konungen hade fått befallning att förgöra de vise i Babel; han gick åstad och sade till honom så: "De vise i Babel må du icke förgöra. För mig in till konungen, så skall jag meddela konungen uttydningen."
Dan 2:25  Då förde Arjok med hast Daniel inför konungen och sade till honom så: "Jag har bland de judiska fångarna funnit en man som kan säga konungen uttydningen."
Dan 2:26  Konungen svarade och sade till Daniel, som hade fått namnet Beltesassar: "Förmår du säga mig den dröm som jag har haft och dess uttydning?"
Dan 2:27  Daniel svarade konungen och sade: "Den hemlighet som konungen begär att få veta kunna inga vise, besvärjare, spåmän eller stjärntydare meddela konungen.
Dan 2:28  Men det finnes en Gud i himmelen, som kan uppenbara hemligheter, och han har låtit konung Nebukadnessar veta vad som skall ske i kommande dagar. Detta var din dröm och den syn du hade på ditt läger:
Dan 2:29  När du, o konung, låg på ditt läger, uppstego hos dig tankar på vad som skall ske i framtiden; och han som uppenbarar hemligheter lät dig veta vad som skall ske.
Dan 2:30  Och för mig har denna hemlighet blivit uppenbarad, icke i kraft av någon vishet som jag äger framför alla andra levande varelser, utan på det att uttydningen må bliva kungjord för konungen, så att du förstår ditt hjärtas tankar.
Dan 2:31  Du, o konung, såg i din syn en stor bildstod stå framför dig, och den stoden var hög och dess glans övermåttan stor, och den var förskräcklig att skåda.
Dan 2:32  Bildstodens huvud var av bästa guld, dess bröst och armar voro av silver, dess buk och länder av koppar; dess ben voro av järn,
Dan 2:33  dess fötter delvis av järn och delvis av lera.
Dan 2:34  Medan du nu betraktade den, blev en sten lösriven, dock icke genom människohänder, och den träffade bildstoden på fötterna, som voro av järn och lera, och krossade dem.
Dan 2:35  Då blev på en gång alltsammans krossat, järnet, leran, kopparen, silvret och guldet, och det blev såsom agnar på en tröskloge om sommaren, och vinden förde bort det, så att man icke mer kunde finna något spår därav. Men av stenen som hade träffat bildstoden blev ett stort berg, som uppfyllde hela jorden.
Dan 2:36  Detta var drömmen; och vi vilja nu säga konungen uttydningen:
Dan 2:37  Du, o konung, konungarnas konung, åt vilken himmelens Gud har givit rike, väldighet, makt och ära,
Dan 2:38  och i vilkens hand han har givit människors barn och djuren på marken och fåglarna under himmelen, varhelst varelser bo, och som han har satt till herre över allasammans, du är det gyllene huvudet.
Dan 2:39  Men efter dig skall uppstå ett annat rike, ringare än ditt, och därefter ännu ett tredje rike, ett som är av koppar, och det skall råda över hela jorden.
Dan 2:40  Ett fjärde rike skall ock uppstå och vara starkt såsom järn, ty järnet krossar och sönderslår ju allt; och såsom järnet förstör allt annat, så skall ock detta krossa och förstöra.
Dan 2:41  Men att du såg fötterna och tårna vara delvis av krukmakarlera och delvis av järn, det betyder att det skall vara ett söndrat rike, dock så att det har något av järnets fasthet, ty du såg ju järn vara där, blandat med lerjord.
Dan 2:42  Och att tårna på fötterna voro delvis av järn och delvis av lera, det betyder att riket skall vara delvis starkt och delvis svagt.
Dan 2:43  Och att du såg järnet vara blandat med lerjord, det betyder att väl en beblandning där skall äga rum genom människosäd, men att delarna likväl icke skola hålla ihop med varandra, lika litet som järn kan förbinda sig med lera.
Dan 2:44  Men i de konungarnas dagar skall himmelens Gud upprätta ett rike som aldrig i evighet skall förstöras och vars makt icke skall bliva överlämnad åt något annat folk. Det skall krossa och göra en ände på alla dessa andra riken, men självt skall det bestå evinnerligen;
Dan 2:45  ty du såg ju att en sten blev lösriven från berget, dock icke genom människohänder, och att den krossade järnet, kopparen, leran, silvret och guldet. Så har en stor Gud uppenbarat för konungen vad som skall ske i framtiden, och drömmen är viss, och dess uttydning är tillförlitlig."
Dan 2:46  Då föll konung Nebukadnessar på sitt ansikte och tillbad inför Daniel, och befallde att man skulle offra åt honom spisoffer och rökoffer.
Dan 2:47  Och konungen svarade Daniel och sade: "I sanning, eder Gud är en Gud över andra gudar och en herre över konungar och en uppenbarare av hemligheter, eftersom du har kunnat uppenbara denna hemlighet."
Dan 2:48  Därefter upphöjde konungen Daniel och gav honom många stora skänker och satte honom till herre över hela Babels hövdingdöme och till högste föreståndare för alla de vise i Babel.
Dan 2:49  Och på Daniels bön förordnade konungen Sadrak, Mesak och Abed-Nego att förvalta Babels hövdingdöme; men Daniel själv stannade vid konungens hov.
Dan 3:1  Konung Nebukadnessar lät göra en gyllene bildstod, sextio alnar hög och sex alnar bred; den lät han ställa upp på Duraslätten i Babels hövdingdöme.
Dan 3:2  Och konung Nebukadnessar sände åstad och lät församla satraper, landshövdingar och ståthållare, fogdar, skattmästare, domare, lagtolkare och alla andra makthavande i hövdingdömena, för att de skulle komma till invigningen av den bildstod som konung Nebukadnessar hade låtit ställa upp.
Dan 3:3  Då församlade sig satraperna, landshövdingarna och ståthållarna, fogdarna, skattmästarna, domarna, lagtolkarna och alla andra makthavande i hövdingdömena till invigningen av den bildstod som konung Nebukadnessar hade låtit ställa upp och när de så stodo framför den bildstod som Nebukadnessar hade låtit ställa upp,
Dan 3:4  utropade en härold med hög röst: "Detta vare eder befallt, I folk och stammar och tungomål:
Dan 3:5  När I hören ljudet av horn, pipor, cittror, sambukor, psaltare, säckpipor och allahanda andra instrumenter, skolen I falla ned och tillbedja den gyllene bildstod som konung Nebukadnessar har låtit ställa upp.
Dan 3:6  Men den som icke faller ned och tillbeder, han skall i samma stund kastas i den brinnande ugnen."
Dan 3:7  Så snart nu allt folket hörde ljudet av horn, pipor, cittror, sambukor, psaltare och allahanda andra instrumenter, föllo de alltså ned, alla folk och stammar och tungomål, och tillbådo den gyllene bildstod som konung Nebukadnessar hade låtit ställa upp.
Dan 3:8  Men strax därefter kommo några kaldeiska män fram och anklagade judarna.
Dan 3:9  De togo till orda och sade till konung Nebukadnessar: Må du leva evinnerligen, o konung!
Dan 3:10  Du, o konung, har givit befallning att alla människor, när de hörde ljudet av horn, pipor, cittror, sambukor, psaltare, säckpipor och allahanda andra instrumenter, skulle falla ned och tillbedja den gyllene bildstoden,
Dan 3:11  och att var och en som icke fölle ned och tillbåde skulle kastas i den brinnande ugnen.
Dan 3:12  Men nu äro här några judiska män, Sadrak, Mesak och Abed-Nego, vilka du har förordnat att förvalta Babels hövdingdöme. Dessa män hava icke aktat på dig, o konung. De dyrka icke dina gudar; och den gyllene bildstod som du har låtit ställa upp tillbedja de icke."
Dan 3:13  Då befallde Nebukadnessar i vrede och förbittring att man skulle föra fram Sadrak, Mesak och Abed-Nego. Och när man hade fört fram männen inför konungen,
Dan 3:14  talade Nebukadnessar till dem och sade: "Är det av förakt som I, Sadrak, Mesak och Abed-Nego, icke dyrken mina gudar och icke tillbedjen den gyllene bildstod som jag har låtit ställa upp?
Dan 3:15  Välan, allt må vara gott, om I ären redo, att när I hören ljudet av horn, pipor, cittror, sambukor, psaltare, säckpipor och allahanda andra instrumenter, falla ned och tillbedja den bildstod som jag har låtit göra. Men om I icke tillbedjen, då skolen I i samma stund bliva kastade i den brinnande ugnen; och vilken är väl den gud som då kan rädda eder ur min hand?"
Dan 3:16  Då svarade Sadrak, Mesak och Abed-Nego och sade till konungen: "O Nebukadnessar, vi behöva icke giva dig något svar på detta.
Dan 3:17  Om vår Gud, den som vi dyrka, förmår rädda oss, så skall han ock rädda oss ur den brinnande ugnen och ur din hand, o konung.
Dan 3:18  Men om han icke vill det, så må du veta, o konung, att vi ändå icke dyrka dina gudar, och att vi icke vilja tillbedja den gyllene bildstod som du har låtit ställa upp."
Dan 3:19  Då uppfylldes Nebukadnessar av vrede mot Sadrak, Mesak och Abed-Nego, så att hans ansikte förvandlades. Och han hov upp sin röst och befallde att man skulle göra ugnen sju gånger hetare, än man någonsin hade sett den vara.
Dan 3:20  Och några handfasta män i hans här fingo befallning att binda Sadrak, Mesak och Abed-Nego och kasta dem i den brinnande ugnen.
Dan 3:21  Så blevo dessa med sina underkläder, livrockar, mössor och andra kläder bundna och kastade i den brinnande ugnen.
Dan 3:22  Men eftersom konungens befallning hade varit så sträng, och ugnen därför hade blivit så övermåttan starkt upphettad, blevo de män som förde Sadrak, Mesak och Abed-Nego ditupp själva dödade av eldslågorna,
Dan 3:23  vid det att de tre männen Sadrak, Mesak och Abed-Nego bundna kastades ned i den brinnande ugnen.
Dan 3:24  Då blev konung Nebukadnessar förskräckt och stod upp med hast och frågade sina rådsherrar och sade: "Var det icke tre män som vi läto kasta bundna i elden? De svarade och sade till konungen: "Jo förvisso, o konung."
Dan 3:25  Han fortfor och sade: "Och ändå ser jag nu fyra män, som gå lösa och lediga inne i elden, och ingen skada har skett dem; och den fjärde ser så ut, som vore han en gudason."
Dan 3:26  Därefter trädde Nebukadnessar fram till öppningen på den brinnande ugnen och hov upp sin röst och sade: "Sadrak, Mesak och Abed-Nego, I den högste Gudens tjänare kommen hitut." Då gingo Sadrak, Mesak och Abed-Nego ut ur elden.
Dan 3:27  Och satraperna, landshövdingarna och ståthållarna och konungens rådsherrar församlade sig där, och fingo då se att elden icke hade haft någon makt över männens kroppar, och att håret på deras huvuden icke var svett, och att deras kläder icke hade blivit skadade; ja, man kunde icke ens känna lukten av något bränt på dem.
Dan 3:28  Då hov Nebukadnessar upp sin röst och sade: "Lovad vare Sadraks, Mesaks och Abed-Negos Gud, som sände sin ängel och räddade sina tjänare, vilka så förtröstade på honom, att de överträdde konungens befallning och vågade sina liv för att icke nödgas dyrka eller tillbedja någon annan gud än sin egen Gud!
Dan 3:29  Och härmed giver jag nu befallning att vilken som helst av alla folk och stammar och tungomål, som säger något otillbörligt om Sadraks, Mesaks och Abed-Negos Gud, han skall huggas i stycken, och hans hus skall göras till en plats för orenlighet; ty ingen gud finnes, som så kan hjälpa som denne.
Dan 3:30  Därefter lät konungen Sadrak, Mesak och Abed-Nego komma till stor ära och makt i Babels hövdingdöme.
Dan 3:31  Konung Nebukadnessar till alla folk och stammar och tungomål som finnas på hela jorden. Mycken frid vare med eder!
Dan 3:32  Jag har funnit för gott att härmed kungöra de tecken och under som den högste Guden har gjort med mig.
Dan 3:33  Ty stora äro förvisso hans tecken, och mäktiga äro hans under. Hans rike är ett evigt rike, och hans välde varar från släkte till släkte.
Dan 4:1  Jag, Nebukadnessar, satt i god ro i mitt hus och levde lycklig i mitt palats.
Dan 4:2  Då hade jag en dröm som förskräckte mig; jag ängslades genom drömbilder på mitt läger och genom en syn som jag såg.
Dan 4:3  Därför gav jag befallning att man skulle hämta alla de vise i Babel till mig, för att de skulle säga mig drömmens uttydning.
Dan 4:4  Så kommo nu spåmännen, besvärjarna, kaldéerna och stjärntydarna, och jag förtäljde drömmen för dem, men de kunde icke säga mig dess uttydning.
Dan 4:5  Slutligen kom ock Daniel inför mig, han som hade fått namnet Beltesassar efter min guds namn, och i vilken heliga gudars ande är; och jag förtäljde drömmen för honom sålunda:
Dan 4:6  "Beltesassar, du som är den överste bland spåmännen, du om vilken jag vet att heliga gudars ande är i dig, och att ingen hemlighet är dig för svår, säg mig vad jag såg i min dröm, och vad den betyder.
Dan 4:7  Detta var den syn jag hade på mitt läger: Jag såg i min syn ett träd stå mitt på jorden, och det var mycket högt.
Dan 4:8  Ja, stort och väldigt var trädet, och så högt att det räckte upp till himmelen och syntes allt intill jordens ända.
Dan 4:9  Dess lövverk var skönt, och de bar mycken frukt, så att det hade föda åt alla. Markens djur funno skugga därunder, och himmelens fåglar bodde på dess grenar, och allt kött hade sin föda därav.
Dan 4:10  Vidare såg jag, i den syn jag hade på mitt läger, huru en helig ängel steg ned från himmelen.
Dan 4:11  Han ropade med hög röst och sade: 'Huggen ned trädet och skären av dess grenar, riven bort dess lövverk och förströn dess frukt, så att djuren som ligga därunder fara sin väg och fåglarna flyga bort ifrån dess grenar.
Dan 4:12  Dock må stubben med rötterna lämnas kvar i jorden, bunden med kedjor av järn och koppar, bland markens gräs; av himmelens dagg skall han vätas och hava sin lott med djuren bland markens örter.
Dan 4:13  Hans hjärta skall förvandlas, så att det icke mer är en människas, och ett djurs hjärta skall givas åt honom, och sju tider skola så gå fram över honom.
Dan 4:14  Så är det förordnat genom änglarnas rådslut, och så är det befallt om denna sak av de heliga, för att de levande skola besinna att den Högste råder över människors riken och giver dem åt vem han vill, ja, upphöjer den lägste bland människor till att härska över dem.'
Dan 4:15  Sådan var den dröm som jag, konung Nebukadnessar, hade. Och du, Beltesassar, må nu säga uttydningen; ty ingen av de vise i mitt rike kan säga mig uttydningen, men du kan det väl, ty heliga gudars ande är i dig."
Dan 4:16  Då stod Daniel, som också hade namnet Beltesassar, en stund häpen, uppfylld av oroliga tankar. Men konungen tog åter till orda och sade: "Beltesassar, låt icke drömmen och vad den betyder förskräcka dig. Beltesassar svarade och sade: "Min herre, o att drömmen gällde dem som hata dig, och dess betydelse dina fiender!
Dan 4:17  Trädet som du såg, vilket var så stort och väldigt och så högt att det räckte upp till himmelen och syntes över hela jorden,
Dan 4:18  och som hade ett så skönt lövverk och bar mycken frukt, så att det hade föda åt alla, trädet under vilket markens djur bodde, och på vars grenar himmelens fåglar hade sina nästen,
Dan 4:19  det är du själv, o konung, du som har blivit så stor och väldig, du vilkens storhet har vuxit, till dess att den har nått upp till himmelen, och vilkens välde sträcker sig till jordens ända.
Dan 4:20  Men att konungen såg en helig ängel stiga ned från himmelen, vilken sade: 'Huggen ned trädet och förstören det; dock må stubben med rötterna lämnas kvar i jorden, bunden med kedjor av järn och koppar, bland markens gräs; av himmelens dagg skall han vätas och hava sin lott med markens djur, till dess att sju tider hava gått fram över honom',
Dan 4:21  detta betyder följande, o konung, och detta är den Högstes rådslut, som har drabbat min herre konungen:
Dan 4:22  Du skall bliva utstött från människorna och nödgas bo ibland markens djur och äta gräs såsom en oxe och vätas av himmelens dagg; och sju tider skola så gå fram över dig, till dess du besinnar att den Högste råder över människors riken och giver dem åt vem han vill.
Dan 4:23  Men att det befalldes att trädets stubbe med rötterna skulle lämnas kvar, det betyder att du skall återfå ditt rike, när du har besinnat att det är himmelen som har makten.
Dan 4:24  Därför, o konung, må du låta mitt råd täckas dig: gör dig fri ifrån dina synder genom att göra gott, och ifrån dina missgärningar genom att öva barmhärtighet mot de fattiga, om till äventyrs din lycka så kunde bliva beståndande."
Dan 4:25  Allt detta drabbade också konung Nebukadnessar.
Dan 4:26  Tolv månader därefter, när konungen en gång gick omkring på taket av det kungliga palatset i Babel,
Dan 4:27  hov han upp sin röst och sade: "Se, detta är det stora Babel, som jag har byggt upp till ett konungasäte genom min väldiga makt, min härlighet till ära!"
Dan 4:28  Medan ordet ännu var i konungens mun, kom en röst från himmelen: "Dig, konung Nebukadnessar, vare det sagt: Ditt rike har blivit taget ifrån dig;
Dan 4:29  du skall bliva utstött från människorna och nödgas bo ibland markens djur och äta gräs såsom en oxe; och sju tider skola så gå fram över dig, till dess du besinnar att den Högste råder över människors riken och giver dem åt vem han vill."
Dan 4:30  I samma stund gick det ordet i fullbordan på Nebukadnessar; han blev utstött från människorna och måste äta gräs såsom en oxe, och av himmelens dagg vättes hans kropp, till dess att hans hår växte och blev såsom örnfjädrar, och till dess att hans naglar blevo såsom fågelklor.
Dan 4:31  Men när tiden var förliden, upplyfte jag, Nebukadnessar, mina ögon till himmelen och fick åter mitt förstånd. Då lovade jag den Högste, jag prisade och ärade honom som lever evinnerligen, honom vilkens välde är ett evigt välde, och vilkens rike varar från släkte till släkte,
Dan 4:32  honom mot vilken alla som bo på jorden äro att akta såsom intet, ty han gör vad han vill både med himmelens här och med dem som bo på jorden, och ingen kan stå emot hans hand eller säga till honom: "Vad gör du?"
Dan 4:33  Så fick jag då på den tiden åter mitt förstånd, och jag fick tillbaka min härlighet och glans, mitt rike till ära; och mina rådsherrar och stormän sökte upp mig. Och jag blev åter insatt i mitt rike, och ännu större makt blev mig given.
Dan 4:34  Därför prisar nu jag, Nebukadnessar, och upphöjer och ärar himmelens konung, ty alla hans gärningar äro sanning, och hans vägar äro rätta, och dem som vandra i högmod kan han ödmjuka.
Dan 5:1  Konung Belsassar gjorde ett stort gästabud för sina tusen stormän och höll dryckeslag med de tusen.
Dan 5:2  Medan nu Belsassar var under vinets välde, befallde han att man skulle bära fram de kärl av guld och silver, som hans fader Nebukadnessar hade tagit ur templet i Jerusalem; ur dem skulle så konungen och hans stormän, hans gemåler och bihustrur dricka.
Dan 5:3  Då bar man fram de gyllene kärl som hade blivit tagna ur tempelsalen i Guds hus i Jerusalem; och konungen och hans stormän, hans gemåler och bihustrur drucko ur dem.
Dan 5:4  Medan de så drucko vin, prisade de sina gudar av guld och silver, av koppar, järn, trä och sten.
Dan 5:5  Då visade sig i samma stund fingrar såsom av en människohand, vilka mitt emot den stora ljusstaken skrevo på den vitmenade väggen i konungens palats; och konungen såg handen som skrev.
Dan 5:6  Då vek färgen bort ifrån konungens ansikte, och han uppfylldes av oroliga tankar, så att hans länder skälvde och hans knän slogo emot varandra.
Dan 5:7  Och konungen ropade med hög röst och befallde att man skulle hämta besvärjarna, kaldéerna ock stjärntydarna. Och konungen lät säga så till de vise i Babel: "Vemhelst som kan läsa denna skrift och meddela mig dess uttydning, han skall bliva klädd i purpur, och den gyllene kedjan skall hängas om hans hals, och han skall bliva den tredje herren i riket."
Dan 5:8  Då kommo alla konungens vise tillstädes, men de kunde icke läsa skriften eller säga konungen dess uttydning.
Dan 5:9  Då blev konung Belsassar ännu mer förskräckt, och färgen vek bort ifrån hans ansikte, och hans stormän stodo bestörta.
Dan 5:10  Men när konungens och hans stormäns tal kom för konungamodern, begav hon sig till gästabudssalen; där tog hon till orda och sade: "Må du leva evinnerligen, o konung! Låt icke oroliga tankar uppfylla dig, och må färgen icke vika bort ifrån ditt ansikte.
Dan 5:11  I ditt rike finnes en man i vilken heliga gudars ande är. I din faders dagar befanns han hava insikt och förstånd och vishet, lik gudars vishet; och din fader, konung Nebukadnessar, satte honom till den överste bland spåmännen, besvärjarna, kaldéerna och stjärntydarna; ja, detta gjorde din fader konungen,
Dan 5:12  eftersom en övermåttan hög ande och klokhet och förstånd och skicklighet att uttyda drömmar och lösa gåtor och reda ut invecklade ting fanns hos denne Daniel, åt vilken konungen hade givit namnet Beltesassar. Låt därför nu tillkalla Daniel; han skall meddela uttydningen."
Dan 5:13  När så Daniel hade blivit hämtad till konungen, talade denne till Daniel och sade: "Du är ju Daniel, en av de judiska fångar som min fader konungen förde hit från Juda?
Dan 5:14  Jag har hört sägas om dig att gudars ande är i dig, och att du har befunnits hava insikt och förstånd och övermåttan stor vishet.
Dan 5:15  Nu är det så, att de vise och besvärjarna hava blivit hämtade hit till mig för att läsa denna skrift och säga mig dess uttydning; men de kunna icke meddela mig någon uttydning därpå.
Dan 5:16  Men om dig har jag hört att du kan giva uttydningar och reda ut invecklade ting. Om du alltså nu kan läsa skriften och säga mig dess uttydning, så skall du bliva klädd i purpur, och den gyllene kedjan skall hängas om din hals, och du skall bliva den tredje herren i riket."
Dan 5:17  Då svarade Daniel och sade till konungen: "Dina gåvor må du själv behålla, och dina skänker må du giva åt en annan; dem förutan skall jag läsa skriften för konungen och säga honom uttydningen:
Dan 5:18  Åt din fader Nebukadnessar, o konung, gav den högste Guden rike, storhet, ära och härlighet;
Dan 5:19  och för den storhets skull som han hade givit honom darrade alla folk och stammar och tungomål, i förskräckelse för honom. Vem han ville dödade han, och vem han ville lät han leva; vem han ville upphöjde han, och vem han ville ödmjukade han.
Dan 5:20  Men när hans hjärta förhävde sig och hans ande blev stolt och övermodig, då störtades han från sin konungatron, och hans ära togs ifrån honom.
Dan 5:21  Han blev utstött från människors barn, och hans hjärta blev likt ett djurs, och han måste bo ibland vildåsnor och äta gräs såsom en oxe, och av himmelens dagg vättes hans kropp - detta till dess han besinnade att den högste Guden råder över människors riken och upphöjer vem han vill till att härska över dem.
Dan 5:22  Men du, Belsassar, hans son, som har vetat allt detta, har ändå icke ödmjukat ditt hjärta,
Dan 5:23  utan förhävt dig mot himmelens Herre och låtit bära fram inför dig kärlen från hans hus; och du och dina stormän, dina gemåler och bihustrur haven druckit vin ur dem och du har därunder prisat dina gudar av silver och guld, av koppar, järn, trä och sten, som varken se eller höra eller veta något. Men den Gud som har i sitt våld din ande och alla dina vägar, honom har du icke ärat.
Dan 5:24  Därför har nu av honom denna hand blivit sänd och denna skrift blivit tecknad.
Dan 5:25  Och så lyder den skrift som här är tecknad: Mene mene tekel u-farsin.
Dan 5:26  Och detta är uttydningen därpå: Mene, det betyder: Gud har räknat ditt rikes dagar och gjort ände på det
Dan 5:27  Tekel, det betyder: du är vägd på en våg och befunnen för lätt.
Dan 5:28  Peres, det betyder: ditt rike har blivit styckat och givet åt meder och perser."
Dan 5:29  Då befallde Belsassar att man skulle kläda Daniel i purpur, och att den gyllene kedjan skulle hängas om hans hals, och att man skulle utropa om honom att han skulle vara den tredje herren i riket.
Dan 5:30  Samma natt blev Belsassar, kaldéernas konung, dödad.
Dan 5:31  Och Darejaves av Medien mottog riket, när han var sextiotvå år gammal.
Dan 6:1  Darejaves fann för gott att sätta över riket ett hundra tjugu satraper, för att sådana skulle finnas överallt i riket.
Dan 6:2  Och över dem satte han tre furstar, av vilka Daniel var en; inför dessa skulle satraperna avlägga räkenskap, så att konungen icke lede något men.
Dan 6:3  Men Daniel gjorde sig bemärkt framför de andra furstarna och satraperna, ty en övermåttan hög ande var i honom, och konungen var betänkt på att sätta honom över hela riket.
Dan 6:4  Då sökte de andra furstarna och satraperna att finna någon sak mot Daniel i det som angick riket. Men de kunde icke finna någon sådan sak eller något som var orätt, eftersom han var trogen i sin tjänst; ingen försummelse och intet orätt var att finna hos honom.
Dan 6:5  Då sade männen: "Vi lära icke finna någon sak mot denne Daniel, om vi icke till äventyrs kunna finna en sådan i hans gudsdyrkan."
Dan 6:6  Därefter skyndade furstarna och satraperna in till konungen och sade till honom så: "Må du leva evinnerligen, konung Darejaves!
Dan 6:7  Alla rikets furstar, landshövdingarna och satraperna, rådsherrarna och ståthållarna hava rådslagit om att en kunglig förordning borde utfärdas och ett förbud stadgas, av det innehåll att vilken som helst som under trettio dagar vänder sig med bön till någon annan, vare sig gud eller människa, än till dig, o konung, han skall kastas i lejongropen.
Dan 6:8  Så låt nu, o konung, härom utfärda ett förbud och sätta upp en skrivelse, som efter Mediens och Persiens oryggliga lag icke kan återkallas."
Dan 6:9  I överensstämmelse härmed lät då konung Darejaves sätta upp en skrivelse och utfärda ett förbud.
Dan 6:10  Men så snart Daniel hade fått veta att skrivelsen var uppsatt, gick han in i sitt hus, varest han i sin övre sal hade fönster som voro öppna i riktning mot Jerusalem. Där föll han tre gånger om dagen ned på sina knän och bad och tackade sin Gud, såsom han förut hade plägat göra.
Dan 6:11  När männen nu skyndade till, funno de Daniel bedjande och åkallande sin Gud.
Dan 6:12  Därefter gingo de till konungen och frågade honom angående det kungliga förbudet: "Har du icke låtit sätta upp ett förbud, av det innehåll att vilken som helst som under trettio dagar vänder sig med bön till någon annan, vare sig gud eller människa, än till dig, o konung, han skall kastas i lejongropen?" Konungen svarade och sade: "Jo, och det påbudet står fast efter Mediens och Persiens oryggliga lag."
Dan 6:13  Då svarade de och sade till konungen: "Daniel, en av de judiska fångarna, aktar varken på dig eller på det förbud som du har låtit sätta upp, utan förrättar sin bön tre gånger om dagen."
Dan 6:14  När konungen hörde detta, blev han mycket bedrövad och gjorde sig bekymmer över huru han skulle kunna rädda Daniel; ända till solnedgången mödade han sig med att söka en utväg att hjälpa honom.
Dan 6:15  Då skyndade männen till konungen och sade till honom: "Vet, o konung, att det är en Mediens och Persiens lag att intet förbud och ingen förordning som konungen utfärdar kan återkallas."
Dan 6:16  Då lät konungen hämta Daniel och kasta honom i lejongropen och konungen talade till Daniel och sade: "Din Gud, den som du så oavlåtligen dyrkar, han må rädda dig."
Dan 6:17  Och man förde fram en sten och lade den över gropens öppning, och konungen förseglade den med sitt eget och med sina stormäns signet, för att ingen förändring skulle kunna göras i det som nu hade skett med Daniel.
Dan 6:18  Därefter gick konungen hem till sitt palats och tillbragte hela natten under fasta och lät inga kvinnor komma inför sig; och sömnen flydde honom.
Dan 6:19  Sedan om morgonen, när det dagades, stod konungen upp och gick med hast till lejongropen.
Dan 6:20  Och när han hade kommit nära intill gropen, ropade han på Daniel med ängslig röst; konungen talade till Daniel och sade: "Daniel, du den levande Gudens tjänare, har väl din Gud, den som du så oavlåtligen dyrkar, kunnat rädda dig från lejonen?"
Dan 6:21  Då svarade Daniel konungen: "Må du leva evinnerligen, o konung!
Dan 6:22  Min Gud har sänt sin ängel och tillslutit lejonens gap, så att de icke hava gjort mig någon skada. Ty jag har inför honom befunnits oskyldig; ej heller har jag förbrutit mig mot dig, o konung.
Dan 6:23  Då blev konungen mycket glad, och befallde att man skulle taga Daniel upp ur gropen. Och när Daniel hade blivit tagen upp ur gropen, kunde man icke upptäcka någon skada på honom; ty han hade trott på sin Gud.
Dan 6:24  Sedan lät konungen hämta de män som hade anklagat Daniel, och han lät kasta dem i lejongropen, med deras barn och hustrur; och innan de ännu hade hunnit till bottnen i gropen, föllo lejonen över dem och krossade alla deras ben.
Dan 6:25  Därefter lät konung Darejaves skriva till alla folk och stammar och tungomål som funnos på hela jorden: "Mycken frid vare med eder!
Dan 6:26  Härmed giver jag befallning att man inom mitt rikes hela område skall bäva och frukta för Daniels Gud. Ty han är den levande Guden, som förbliver evinnerligen; och hans rike är sådant att det icke kan förstöras, och hans välde består intill änden.
Dan 6:27  Han är en räddare och hjälpare, och han gör tecken och under i himmelen och på jorden, han som har räddat Daniel ur lejonens våld."
Dan 6:28  Och denne Daniel steg i ära och makt under Darejaves' och under persern Kores' regeringar.
Dan 7:1  I den babyloniske konungen Belsassars första regeringsår hade Daniel en dröm och såg en syn på sitt läger. Sedan tecknade han upp drömmen och meddelade huvudsumman av dess innehåll.
Dan 7:2  Detta är Daniels berättelse: Jag hade en syn om natten, och såg i den huru himmelens fyra vindar stormade fram mot det stora havet.
Dan 7:3  Och fyra stora djur stego upp ur havet, det ena icke likt det andra.
Dan 7:4  Det första liknade ett lejon, men det hade vingar såsom en örn. Medan jag ännu såg härpå, rycktes vingarna av djuret, och det restes upp från jorden, så att det blev ställt på två fötter såsom en människa, och ett mänskligt hjärta blev givet åt det.
Dan 7:5  Sedan fick jag se ännu ett djur, det andra i ordningen; det var likt en björn, och det reste upp sin ena sida, och det hade tre revben i sitt gap, mellan tänderna. Och till det djuret blev så sagt: "Stå upp och sluka mycket kött."
Dan 7:6  Därefter fick jag se ett annat djur, som liknade en panter, men på sina sidor hade det fyra fågelvingar; och djuret hade fyra huvuden, och välde blev givet åt det.
Dan 7:7  Därefter fick jag i min syn om natten se ett fjärde djur, övermåttan förskräckligt, fruktansvärt och starkt; det hade stora tänder av järn, det uppslukade och krossade, och vad som blev kvar trampade det under fötterna; det var olikt alla de förra djuren och hade tio horn.
Dan 7:8  Men under det att jag betraktade hornen, fick jag se huru mellan dem ett annat horn sköt upp, ett litet, för vilket tre av de förra hornen blevo bortstötta; och se, det hornet hade ögon lika människoögon, och en mun som talade stora ord.
Dan 7:9  Medan jag ännu såg härpå, blevo troner framsatta, och en som var gammal satte sig ned. Hans klädnad var snövit, och håret på hans huvud var såsom ren ull; hans tron var av eldslågor, och hjulen därpå voro av flammande eld.
Dan 7:10  En flod av eld strömmade ut från honom, tusen gånger tusen voro hans tjänare, och tio tusen gånger tio tusen stodo där till hans tjänst. Så satte man sig ned till doms, och böcker blevo upplåtna.
Dan 7:11  Medan jag nu såg härpå, skedde det att, för de stora ords skull som hornet talade - medan jag ännu såg härpå - djuret dödades och dess kropp förstördes och kastades i elden för att brännas upp.
Dan 7:12  Från de övriga djuren togs ock deras välde, ty deras livslängd var bestämd till tid och stund.
Dan 7:13  Sedan fick jag, i min syn om natten, se huru en som liknade en människoson kom med himmelens skyar; och han nalkades den gamle och fördes fram inför honom.
Dan 7:14  Åt denne gavs välde och ära och rike, och alla folk och stammar och tungomål måste tjäna honom. Hans välde är ett evigt välde, som icke skall tagas ifrån honom, och hans rike skall icke förstöras.
Dan 7:15  Då kände jag, Daniel, min ande oroas i sin boning, och den syn som jag hade haft förskräckte mig.
Dan 7:16  Jag gick fram till en av dem som stodo där och bad honom om en tillförlitlig förklaring på allt detta Och han svarade mig och sade mig uttydningen därpå:
Dan 7:17  "De fyra stora djuren betyda att fyra konungar skola uppstå på jorden.
Dan 7:18  Men sedan skola den Högstes heliga undfå riket och taga det i besittning för evig tid, ja, för evigheters evighet."
Dan 7:19  Därefter ville jag hava tillförlitlig förklaring angående det fjärde djuret, som var olikt alla de andra, det som var så övermåttan förskräckligt och hade tänder av järn och klor av koppar, det som uppslukade och krossade och sedan trampade under fötterna vad som blev kvar;
Dan 7:20  så ock angående de tio hornen på dess huvud, och angående det nya hornet, det som sedan sköt upp, och för vilket tre andra föllo av, det hornet som hade Ögon, och en mun som talade stora ord, det som var större att skåda än de övriga,
Dan 7:21  det hornet som jag ock hade sett; föra krig mot de heliga och bliva dem övermäktigt,
Dan 7:22  till dess att den gamle kom och rätt blev skipad åt den Högstes heliga och tiden var inne, då de heliga fingo taga riket i besittning.
Dan 7:23  Då svarade han så: "Det fjärde djuret betyder att ett fjärde rike skall uppstå på jorden, ett som är olikt alla de andra rikena. Det skall uppsluka hela jorden och förtrampa och krossa den.
Dan 7:24  Och de tio hornen betyda att tio konungar skola uppstå i det riket; och efter dem skall uppstå en annan, som skall vara olik de förra, och som skall slå ned tre konungar.
Dan 7:25  Och denne skall upphäva sitt tal mot den Högste och föröda den Högstes heliga; han skall sätta sig i sinnet att förändra heliga tider och lagar; och de skola givas i hans hand under en tid, och tider, och en halv tid.
Dan 7:26  Men dom skall bliva hållen, och hans välde skall tagas ifrån honom och fördärvas och förgöras i grund.
Dan 7:27  Men rike och välde och storhet, utöver alla riken under himmelen, skall givas åt den Högstes heligas folk. Dess rike skall vara ett evigt rike, och alla välden skola tjäna och lyda det."
Dan 7:28  Här slutar berättelsen. Men jag, Daniel, uppfylldes av många oroliga tankar, och färgen vek bort ifrån mitt ansikte; men jag bevarade i mitt hjärta vad som hade hänt.
Dan 8:1  I konung Belsassars tredje regeringsår såg jag, Daniel, en syn, en som kom efter den jag förut hade sett.
Dan 8:2  Då jag nu i denna syn såg till, tyckte jag mig vara i Susans borg i hövdingdömet Elam; och då jag vidare såg till i synen, fann jag mig vara vid floden Ulai.
Dan 8:3  Och när jag lyfte upp mina ögon, fick jag se en vädur stå framför floden, och han hade två horn; och båda hornen voro höga, men det ena var högre än det andra, och detta som var högre sköt sist upp.
Dan 8:4  Jag såg väduren stöta med hornen västerut och norrut och söderut, och intet djur kunde stå honom emot, och ingen kunde rädda ur hans våld; han for fram såsom han ville och företog sig stora ting.
Dan 8:5  Och när jag vidare gav akt, fick jag se en bock komma västerifrån och gå fram över hela jorden, dock utan att röra vid jorden; och bocken hade ett ansenligt horn i pannan.
Dan 8:6  Och han nalkades väduren med de båda hornen, den som jag hade sett stå framför floden, och sprang emot honom i väldig vrede.
Dan 8:7  Jag såg honom komma ända inpå väduren och störta över honom i förbittring, och han stötte till väduren och krossade hans båda horn, så att väduren icke hade någon kraft att stå emot honom. Sedan slog han honom till jorden och trampade på honom; och ingen fanns, som kunde rädda väduren ur hans våld.
Dan 8:8  Och bocken företog sig mycket stora ting. Men när han hade blivit som starkast, brast det stora hornet sönder, och fyra andra ansenliga horn sköto upp i dess ställe, åt himmelens fyra väderstreck.
Dan 8:9  Och från ett av dem gick ut ett nytt horn, i begynnelsen litet, och det växte övermåttan söderut och österut och åt "det härliga landet" till.
Dan 8:10  Och det växte ända upp till himmelens härskara och kastade några av denna härskara, av stjärnorna, ned till jorden och trampade på dem.
Dan 8:11  Ja, till och med mot härskarornas furste företog han sig stora ting: han tog bort ifrån honom det dagliga offret, och hans helgedoms boning slogs ned.
Dan 8:12  Jämte det dagliga offret bliver ock en härskara prisgiven, för överträdelses skull. Och det slår sanningen ned till jorden och lyckas väl i vad det företager sig.
Dan 8:13  Sedan hörde jag en av de heliga tala, och en annan helig frågade denne som talade: "Huru lång tid avser synen om det dagliga offret, och om överträdelsen som kommer åstad förödelse, och om förtrampandet av både helgedom och härskara?"
Dan 8:14  Då svarade han mig: "Två tusen tre hundra aftnar och morgnar; därefter skall helgedomen komma till sin rätt igen."
Dan 8:15  När nu jag, Daniel, hade sett denna syn och sökte att förstå den, fick jag se en som såg ut såsom en man stå framför mig.
Dan 8:16  Och mitt över Ulai hörde jag rösten av en människa som ropade och sade: "Gabriel, uttyd synen för denne."
Dan 8:17  Då kom han intill platsen där jag stod, men jag blev förskräckt, när han kom, och föll ned på mitt ansikte. Och han sade till mig: "Giv akt härpå, du människobarn; ty synen syftar på ändens tid."
Dan 8:18  Medan han så talade med mig, låg jag i vanmakt, med mitt ansikte mot jorden; men han rörde vid mig och reste upp mig igen.
Dan 8:19  Därefter sade han: "Se, jag vill kungöra för dig vad som skall ske, när det lider mot slutet med vreden ty på ändens tid syftar detta.
Dan 8:20  Väduren som du såg, han med de två hornen, betyder Mediens och Persiens konungar.
Dan 8:21  Men bocken är Javans konung, och det stora hornet i hans panna är den förste konungen.
Dan 8:22  Men att det brast sönder, och att fyra andra uppstodo i dess ställe, det betyder att fyra riken skola uppstå av hans folk, dock icke jämlika med honom i kraft.
Dan 8:23  Och vid slutet av deras välde, när överträdarna hava fyllt sitt mått, skall en fräck och arglistig konung uppstå;
Dan 8:24  han skall bliva stor i kraft, dock icke jämlik med den förre i kraft och han skall komma åstad så stort fördärv att man måste förundra sig; och han skall lyckas väl och få fullborda sitt uppsåt. Ja, han skall fördärva många, och jämväl de heligas folk.
Dan 8:25  Därigenom att han är så klok, skall han lyckas så väl med sitt svek, han skall föresätta sig stora ting, oförtänkt skall han fördärva många. Ja, mot furstarnas furste skall han sätta sig upp; men utan människohand skall han då varda krossad.
Dan 8:26  Och synen angående aftnarna och morgnarna, varom nu är talat, är sanning. Men göm du den synen, ty den syftar på en avlägsen framtid."
Dan 8:27  Men jag, Daniel, blev maktlös och låg sjuk en tid. Sedan stod jag upp och förrättade min tjänst hos konungen; och jag var häpen över synen, men ingen förstod den.
Dan 9:1  I Darejaves', Ahasveros' sons, första regeringsår - hans som var av medisk släkt, men som hade blivit upphöjd till konung över kaldéernas rike -
Dan 9:2  i dennes första regeringsår kom jag, Daniel, att i skrifterna lägga märke till det antal år, som HERREN hade angivit för profeten Jeremia, när han sade att han ville låta sjuttio år gå till ända, medan Jerusalem låg öde.
Dan 9:3  Då vände jag mitt ansikte till Herren Gud med ivrig bön och åkallan, och fastade därvid i säck och aska.
Dan 9:4  Jag bad till HERREN, min Gud, och bekände och sade: "Ack Herre, du store och fruktansvärde Gud, du som håller förbund och bevarar nåd mot dem som älska dig och hålla dina bud!
Dan 9:5  Vi hava syndat och gjort illa och varit ogudaktiga och avfälliga; vi hava vikit av ifrån dina bud och rätter.
Dan 9:6  Vi hava icke hörsammat dina tjänare profeterna, som talade i ditt namn till våra konungar, furstar och fader och till allt folket i landet.
Dan 9:7  Du, Herre, är rättfärdig, men vi måste blygas, såsom vi ock nu göra, vi Juda man och Jerusalems invånare, ja, hela Israel, både de som bo nära och de som bo fjärran i alla andra länder dit du har fördrivit dem, därför att de voro otrogna mot dig.
Dan 9:8  Ja, Herre, vi med våra konungar, furstar och fäder måste blygas, därför att vi hava syndat mot dig.
Dan 9:9  Men hos Herren, vår Gud, är barmhärtighet och förlåtelse. Ty vi voro avfälliga från
Dan 9:10  och hörde icke HERRENS, vår Guds, röst, så att vi vandrade efter hans lagar, dem som han förelade oss genom sina tjänare profeterna.
Dan 9:11  Nej, hela Israel överträdde din lag och vek av, utan att höra din röst. Därför utgöt sig ock över oss den förbannelse som han hade svurit att sända, och som står skriven i Moses, Guds tjänares, lag; ty vi hade ju syndat mot honom.
Dan 9:12  Han höll sina ord, vad han hade talat mot oss, och mot domarna som dömde oss; och han lät en så stor olycka komma över oss, att ingenstädes under himmelen något sådant har skett, som det som nu har skett i Jerusalem.
Dan 9:13  I enlighet med vad som står skrivet i Moses lag kom all denna olycka över oss, men ändå sökte vi icke att blidka HERREN, vår Gud, genom att omvända oss från våra missgärningar och akta på din sanning.
Dan 9:14  Därför vakade ock HERREN över att olyckan drabbade oss; ty HERREN, vår Gud, är rättfärdig i alla de gärningar som han gör, men hörde icke hans röst.
Dan 9:15  Och nu, Herre, vår Gud, du som förde ditt folk ut ur Egyptens land med stark hand, och så gjorde dig ett namn, som är detsamma än i dag! Vi hava syndat, vi hava varit ogudaktiga.
Dan 9:16  Men Herre, låt, för all din rättfärdighets skull, din vrede och förtörnelse vända sig ifrån din stad Jerusalem, ditt heliga berg; ty genom våra synder och genom våra faders missgärningar hava Jerusalem och ditt folk blivit till smälek för alla som bo omkring oss.
Dan 9:17  Och hör nu, du vår Gud, din tjänares bön och åkallan, och låt ditt ansikte lysa över din ödelagda helgedom, för Herrens skull.
Dan 9:18  Böj, min Gud, ditt öra härtill och hör; öppna dina ögon och se vilken förödelse som har övergått oss, och se till staden som är uppkallad efter ditt namn. Ty icke i förlitande på vad rättfärdigt vi hava gjort bönfalla vi inför dig, utan i förlitande på din stora barmhärtighet.
Dan 9:19  O Herre, hör, o Herre, förlåt; o Herre, akta härpå, och utför ditt verk utan att dröja - för din egen skull, min Gud, ty din stad och ditt folk äro uppkallade efter ditt namn."
Dan 9:20  Medan jag ännu så talade och bad och bekände min egen och mitt folk Israels synd och inför HERREN, min Gud, frambar min förbön för min Guds heliga berg -
Dan 9:21  medan jag alltså ännu så talade i min bön, kom Gabriel till mig i flygande hast, den man som jag förut hade sett i min syn; och det var vid tiden för aftonoffret.
Dan 9:22  Han undervisade mig och talade till mig och sade: "Daniel, jag har nu begivit mig hit för att lära dig förstånd.
Dan 9:23  Redan när du begynte din bön, utgick befallning, och jag har kommit för att giva dig besked, ty du är högt benådad. Så giv nu akt på ordet, och akta på synen.
Dan 9:24  Sjuttio veckor äro bestämda över ditt folk och över din heliga stad, innan en gräns sättes för överträdelsen och synderna få en ände och missgärningen varder försonad och en evig rättfärdighet framhavd, och innan syn och profetia beseglas och en höghelig helgedom bliver smord.
Dan 9:25  Så vet nu och förstå: Från den tid då ordet om att Jerusalem åter skulle byggas upp utgick, till dess en smord, en furste, kommer, skola sju veckor förgå; och under sextiotvå veckor skall det åter byggas upp med sina gator och sina vallgravar, om ock i tider av trångmål.
Dan 9:26  Men efter de sextiotvå veckorna skall en som är smord förgöras, utan att någon efterföljer honom. Och staden och helgedomen skall en anryckande furstes folk förstöra; men själv skall denne få sin ände i störtfloden. Och intill änden skall strid vara; förödelse är oryggligt besluten.
Dan 9:27  Och han skall med många sluta ett starkt förbund för en vecka, och för en halv vecka skola genom honom slaktoffer och spisoffer vara avskaffade; och på styggelsens vinge skall förödaren komma. Detta skall fortgå, till dess att förstöring och oryggligt besluten straffdom utgjuter sig över förödaren."
Dan 10:1  I den persiske konungen Kores' tredje regeringsår fick Daniel, som ock kallades Beltesassar, en uppenbarelse; den uppenbarelsen är sanning och bådar stor vedermöda. Och han aktade på uppenbarelsen och lade märke till synen.
Dan 10:2  Jag, Daniel, hade då gått sörjande tre veckors tid.
Dan 10:3  Jag åt ingen smaklig mat, kött och vin kommo icke i min mun, ej heller smorde jag min kropp med olja, förrän de tre veckorna hade gått till ända.
Dan 10:4  På tjugufjärde dagen i första månaden, när jag var vid stranden av den stora floden, nämligen Hiddekel,
Dan 10:5  fick jag, då jag lyfte upp mina ögon, se en man stå där, klädd i linnekläder och omgjordad kring sina länder med ett bälte av guld från Ufas.
Dan 10:6  Hans kropp var såsom av krysolit hans ansikte liknade en ljungeld hans ögon voro såsom eldbloss, han armar och fötter såsom glänsande koppar; och ljudet av hans tal var såsom ett väldigt dån.
Dan 10:7  Och jag, Daniel, var den ende som såg synen; de män som voro med mig sågo den icke, men en stor förskräckelse föll över dem, så att de flydde bort och gömde sig.
Dan 10:8  Så blev jag allena kvar, och när jag såg den stora synen, förgick all min kraft; färgen vek bort ifrån mitt ansikte, så att det blev dödsblekt, och jag hade ingen kraft mer kvar.
Dan 10:9  Då hörde jag ljudet av hans tal; och på samma gång jag hörde ljudet av hans tal, där jag låg i vanmakt på mitt ansikte, med ansiktet mot jorden,
Dan 10:10  rörde en hand vid mig och hjälpte mig, så att jag skälvande kunde resa mig på mina knän och händer.
Dan 10:11  Sedan sade han till mig: "Daniel, du högt benådade man, giv akt på de ord som jag vill tala till dig, och res dig upp på dina fötter; ty jag har nu blivit sänd till dig." När han så talade till mig, reste jag mig bävande upp.
Dan 10:12  Och han sade till mig: "Frukta icke, Daniel, ty redan ifrån första dagen, då när du vände ditt hjärta till att söka förstånd och till att ödmjuka dig inför din Gud, hava dina ord varit hörda; och jag har nu kommit för dina ords skull.
Dan 10:13  Fursten för Persiens rike stod mig emot under tjuguen dagar; men då kom Mikael, en av de förnämsta furstarna, mig till hjälp, under det att jag förut hade stått där allena mot Persiens konungar.
Dan 10:14  Och nu har jag kommit för att undervisa dig om vad som skall hända ditt folk i kommande dagar; ty också detta är en syn som syftar på framtiden."
Dan 10:15  Under det han så talade till mig, böjde jag mitt ansikte mot jorden och var stum.
Dan 10:16  Men se, han som var lik en människa rörde vid mina läppar. Då upplät jag min mun och talade och sade till honom som stod framför mig: "Min herre, vid den syn jag såg har jag känt mig gripen av vånda, och jag har ingen kraft mer kvar.
Dan 10:17  Huru skulle också min herres tjänare, en sådan som jag, kunna tala med en sådan som min herre är? Jag har nu ingen kraft mer i mig och förmår icke mer att andas."
Dan 10:18  Då rörde han som såg ut såsom en människa åter vid mig och styrkte mig.
Dan 10:19  Han sade: "Frukta icke, du högt benådade man; frid vare med dig, var stark, ja, var stark." När han så talade med mig, kände jag mig styrkt och sade: "Tala, min herre, ty du har nu styrkt mig."
Dan 10:20  Då sade han: "Kan du nu förstå varför jag har kommit till dig? Men jag måste strax vända tillbaka för att strida mot fursten för Persien, och när jag är fri ifrån honom, kommer fursten för Javan.
Dan 10:21  Dock vill jag förkunna för dig vad som är upptecknat i sanningens bok. Och ingen enda står mig bi mot dessa, förutom Mikael, eder furste.
Dan 11:1  Och jag stod vid hans sida såsom hans stöd och värn i medern Darejaves' första regeringsår.
Dan 11:2  Och nu skall jag förkunna för dig vad visst är. Se, ännu tre konungar skola uppstå i Persien, och den fjärde skall förvärva sig större rikedomar än någon av de andra, och när han har blivit som starkast genom sina rikedomar, skall han uppbjuda all sin makt mot Javans rike.
Dan 11:3  Sedan skall en väldig konung uppstå, och han skall härska med stor makt och göra vad han vill.
Dan 11:4  Men knappt har han uppstått, så skall hans rike brista sönder och bliva delat efter himmelens förra väderstreck; och det skall icke tillfalla hans avkomlingar eller förbliva lika mäktigt som när han hade makten; ty hans rike skall omstörtas och tillfalla andra än dem.
Dan 11:5  Och konungen i Söderlandet skall bliva mäktig, så ock en av hans furstar; ja, denne skall bliva en ännu mäktigare härskare än han själv, och hans herradöme skall bliva stort.
Dan 11:6  Och efter några år skola de förbinda sig med varandra, och Söderlandskonungens dotter skall draga till konungen i Nordlandet för att komma åstad förlikning. Men hon skall icke kunna behålla den makt hon vinner, ej heller skall han och hans makt bliva beståndande; utan hon skall bliva given till pris, hon jämte dem som läto henne draga dit, både hennes fader och den man som i sin tid tog henne till sig.
Dan 11:7  Men av telningarna från hennes rot skall en stiga upp på hans plats; denne skall draga mot Nordlandskonungens här och tränga in i hans fäste och göra med folket vad han vill och behålla övermakten.
Dan 11:8  Deras gudar och beläten och deras dyrbara håvor, både silver och guld, skall han ock föra såsom byte till Egypten. Sedan skall han i några år lämna Nordlandskonungen i ro.
Dan 11:9  Däremot skall denne tränga in i Söderlandskonungens rike, men han skall få vända tillbaka till sitt land igen.
Dan 11:10  Och hans söner skola rusta sig till strid och samla en väldig krigs här; och den skall rycka fram och svämma över och utbreda sig; och den skall komma igen, och striden skall föras ända fram till hans fäste.
Dan 11:11  Då skall konungen i Söderlandet resa sig i förbittring och draga ut och strida mot konungen i Nordlandet; och denne skall ställa upp en stor härskara, men den härskaran skall varda given i den andres hand.
Dan 11:12  När då härskaran är sin kos, växer hans övermod; men om han än han slagit ned tiotusenden, får han dock icke makten.
Dan 11:13  Konungen i Nordlandet skall ställa upp en ny härskara, större än den förra; och efter en tid av några år skall han komma med en stor krigshär och stora förråd.
Dan 11:14  Vid samma tid skola många andra resa sig mot konungen i Söderlandet; våldsmän av ditt eget folk skola ock upphäva sig, för att synen skall fullbordas; men dessa skola falla.
Dan 11:15  Och konungen i Nordlandet skall rycka an och kasta upp vallar och intaga en välbefäst stad; och Söderlandets makt skall icke kunna hålla stånd, dess utvalda krigsfolk skall icke hava någon kraft till motstånd.
Dan 11:16  Och han som rycker emot honom skall göra vad han vill, och ingen skall kunna stå emot honom; han skall sätta sig fast i "det härliga landet", och förstöring skall komma genom hans hand.
Dan 11:17  Han skall rycka an med hela sitt rikes makt; dock är han hågad för förlikning, och en sådan skall han komma åstad. En av sina döttrar skall han giva åt honom till hustru, henne till fördärv. Men detta skall icke hava något bestånd och icke vara honom till gagn.
Dan 11:18  Därefter skall han vända sig mot öländerna och intaga många; men en härförare skall göra slut på hans smädelser, ja, låta hans smädelser vända tillbaka över honom själv.
Dan 11:19  Då skall han vända sig till sitt eget lands fästen; men han skall vackla och falla och sedan icke mer finnas till.
Dan 11:20  Och på hans plats skall uppstå en annan, en som låter en fogde draga igenom det land som är hans rikes prydnad; men efter några dagar skall han störtas, dock icke genom vrede, ej heller i krig.
Dan 11:21  Och på hans plats skall uppstå en föraktlig man, åt vilken konungavärdighet icke var ämnad; oförtänkt skall han komma och bemäktiga sig riket genom ränker.
Dan 11:22  Och översvämmande härar skola svämmas bort för honom och krossas, så ock förbundets furste.
Dan 11:23  Från den stund då man förbinder sig med honom skall han bedriva svek. Han skall draga åstad och få övermakten, med allenast litet folk.
Dan 11:24  Oförtänkt skall han falla in i landets bördigaste trakter, och skall göra ting som hans fäder och hans fäders fäder icke hade gjort; byte och rov och gods skall han strö ut åt sitt folk; och mot fästena skall han förehava anslag, intill en viss tid.
Dan 11:25  Och han skall uppbjuda sin kraft; och sitt mod emot konungen i Söderlandet och komma med en stor här, men konungen i Söderlandet skall ock rusta sig till strid, med en mycket stor och talrik här; dock skall han icke kunna hålla stånd, för de anslags skull som göras mot honom.
Dan 11:26  De som äta hans bröd skola störta honom. Och hans här skall svämma över, och många skola bliva slagna och falla.
Dan 11:27  Båda konungarna skola hava ont i sinnet, där de sitta tillhopa vid samma bord, skola de tala lögn, men det skall icke hava någon framgång; ty ännu dröjer änden, intill den bestämda tiden.
Dan 11:28  Han skall vända tillbaka till sitt land med stora förråd, och han skall lägga planer mot det heliga förbundet; och när han har fullbordat dem, skall han vända tillbaka till sitt land.
Dan 11:29  På bestämd tid skall han sedan åter draga åstad mot Söderlandet, men denna senare gång skall det ej gå såsom den förra.
Dan 11:30  Ty skepp från Kittim skola komma emot honom, och han skall förlora modet. Då skall han vända om och rikta sin vrede mot det heliga förbundet och giva den fritt lopp. Och när han har kommit hem, skall han lyssna till dem som hava övergivit det heliga förbundet.
Dan 11:31  Och härar, utsända av honom, skola komma och oskära helgedomens fäste och avskaffa det dagliga offret och ställa upp förödelsens styggelse.
Dan 11:32  Och dem som hava kränkt förbundet skall han med hala ord locka till helt avfall; men de av folket, som känna sin Gud, skola stå fasta och hålla ut.
Dan 11:33  Och de förståndiga bland folket skola lära många insikt; men de skola bliva hemsökta med svärd och eld, med fångenskap och plundring, till en tid;
Dan 11:34  dock skall under hemsökelsen en liten seger beskäras dem, och många skola då av skrymteri sluta sig till dem.
Dan 11:35  Hemsökelsen skall träffa somliga av de förståndiga, för att en luttring skall ske bland dem, så att de varda renade och tvagna till ändens tid; ty ännu dröjer denna, intill den bestämda tiden.
Dan 11:36  Och konungen skall göra vad han vill och skall förhäva sig och uppträda stormodigt mot allt vad gud heter; ja, mot gudars Gud skall han tala sådant att man måste förundra sig. Och allt skall lyckas honom väl, till dess att vredens tid är ute, då när det har skett, som är oryggligt beslutet.
Dan 11:37  På sina fäders gudar skall han icke akta, ej heller skall han akta på den som är kvinnors lust eller på någon annan Gud, utan han skall uppträda stormodigt mot dem alla.
Dan 11:38  Men fästenas gud skall han i stället ära; en gud som hans fäder icke hava känt skall han ära med guld och silver och ädla stenar och andra dyrbara ting.
Dan 11:39  Och mot starka fästen skall han med en främmande guds hjälp göra vad honom lyster; dem som erkänna denne skall han bevisa stor ära, han skall sätta dem att råda över många, och han skall utskifta jord åt dem till belöning.
Dan 11:40  Men på ändens tid skall konungen i Söderlandet drabba samman med honom; och konungen i Nordlandet skall storma fram mot denne med vagnar och ryttare och många skepp, och skall falla in i främmande länder och svämma över och utbreda sig.
Dan 11:41  Han skall ock falla in i "det härliga landet", och många andra länder skola bliva hemsökta; men dessa skola komma undan hans hand: Edom och Moab och huvuddelen av Ammons barn.
Dan 11:42  Ja, han skall uträcka sin hand mot främmande länder, och Egyptens land skall icke slippa undan;
Dan 11:43  han skall bemäktiga sig skatter av guld och silver och allahanda dyrbara ting i Egypten; och libyer och etiopier skola följa honom åt.
Dan 11:44  Då skall han från öster och norr få höra rykten som förskräcka honom; och kan skall draga ut i stor vrede för att förgöra många och giva dem till spillo.
Dan 11:45  Och sina palatstält skall han slå upp mellan havet och helgedomens härliga berg. Men han går till sin undergång, och ingen skall finnas, som hjälper honom."
Dan 12:1  På den tiden skall Mikael träda upp, den store fursten som står såsom försvarare för dina landsmän; och då kommer en tid av nöd, vars like icke har funnits, allt ifrån den dag då människor blevo till och ända till den tiden. Men på den tiden skola av ditt folk alla de varda frälsta, som finnas skrivna i boken.
Dan 12:2  Och många av dem som sova i mullen skola uppvakna, somliga till evigt liv, och somliga till smälek och evig blygd.
Dan 12:3  De förståndiga skola då lysa, såsom fästet lyser, och de som hava fört de många till rättfärdighet såsom stjärnor, alltid och evinnerligen.
Dan 12:4  Men du, Daniel, må gömma dessa ord och försegla denna skrift intill ändens tid; många komma att rannsaka den, och insikten skall så växa till."
Dan 12:5  När nu jag, Daniel, såg till, fick jag se två andra stå där, en på flodens ena strand, och en på dess andra strand.
Dan 12:6  Och en av dem sade till mannen som var klädd i linnekläder, och som stod ovanför flodens vatten: "Huru länge dröjer det, innan änden kommer med dessa förunderliga ting?"
Dan 12:7  Och jag hörde på mannen som var klädd i linnekläder, och som stod ovanför flodens vatten, och han lyfte sin högra hand och sin vänstra hand upp mot himmelen och svor vid honom som lever evinnerligen att efter en tid, och tider, och en halv tid, och när det heliga folkets makt hade blivit krossad i grund, då skulle allt detta varda fullbordat.
Dan 12:8  Och jag hörde detta, men förstod det icke; och jag frågade: "Min herre, vad bliver slutet på allt detta?"
Dan 12:9  Då sade han: "Gå, Daniel, ty dessa ord skola förbliva gömda och förseglade intill ändens tid.
Dan 12:10  Många skola varda renade och tvagna och luttrade, men de ogudaktiga skola öva sin ogudaktighet, och ingen ogudaktig skall förstå detta; men de förståndiga skola förstå det.
Dan 12:11  Och från den tid då det dagliga offret bliver avskaffat och förödelsens styggelse uppställd skola ett tusen två hundra nittio dagar förgå.
Dan 12:12  Säll är den som förbidar och hinner fram till ett tusen tre hundra trettiofem dagar.
Dan 12:13  Men gå du åstad mot ändens tid; sedan du har vilat, skall du uppstå till din del, vid dagarnas ände."


\end{document}