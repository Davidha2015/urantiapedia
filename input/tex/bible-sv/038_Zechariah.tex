\begin{document}

\title{Sakarja}


\chapter{1}

\par 1 I åttonde månaden av Darejaves' andra regeringsår kom HERRENS ord till Sakarja, son till Berekja, son till Iddo, profeten, han sade:
\par 2 Svårt förtörnad var HERREN på edra fäder.
\par 3 Säg därför nu till folket så säger HERREN Sebaot: Vänden om till mig, säger HERREN Sebaot, så vill jag vända om till eder, säger HERREN Sebaot.
\par 4 Varen icke såsom edra fäder, för vilka forna tiders profeter predikade och sade: "Så säger HERREN Sebaot: Vänden om från edra onda vägar och edra onda gärningar"; men de ville icke höra och aktade icke på mig säger HERREN.
\par 5 Edra fäder, var äro de? Och profeterna, leva de kvar evinnerligen?
\par 6 Nej, men mina ord och mina rådslut, de som jag betrodde åt mina tjänare profeterna, de träffade ju edra fäder, så att de måste vända om och säga: "Såsom HERREN Sebaot hade beslutit att göra med oss, och såsom våra vägar och våra gärningar förtjänade, så har han ock gjort med oss."
\par 7 På tjugufjärde dagen i elfte månaden, det är månaden Sebat, i Darejeves' andra regeringsår, kom HERRENS ord till Sakarja, son till Berekja, son till Iddo, profeten; han sade:
\par 8 Jag hade en syn om natten: Jag fick se en man som red på en röd häst; och han höll stilla bland myrtenträden i dalsänkningen. Och bakom honom stodo andra hästar, röda, bruna och vita.
\par 9 Då frågade jag: "Vad betyda dessa, min herre?" Ängeln som talade med mig svarade mig: "Jag vill låta dig förstå vad dessa betyda."
\par 10 Och mannen som stod bland myrtenträden tog till orda och sade: "Det är dessa som HERREN har sänt ut till att fara omkring på jorden."
\par 11 Och själva togo de till orda och sade till HERRENS ängel, som stod bland myrtenträden: "Vi hava farit omkring på jorden och hava funnit hela jorden lugn och stilla."
\par 12 Då tog HERRENS ängel åter till orda och sade: "HERRE Sebaot, huru länge skall det dröja, innan du förbarmar dig över Jerusalem och Juda städer? Du har ju nu varit vred på dem i sjuttio år."
\par 13 Och HERREN svarade ängeln som talade med mig goda och tröstliga ord;
\par 14 och ängeln som talade med mig sade sedan till mig: "Predika och säg: Så säger HERREN Sebaot: Jag har stor nitälskan för Jerusalem och Sion;
\par 15 och jag är storligen förtörnad på hednafolken, som sitta så säkra; ty när jag var allenast litet förtörnad, hjälpte de ytterligare till att fördärva.
\par 16 Därför säger HERREN så: Jag vill åter vända mig till Jerusalem i barmhärtighet; mitt hus skall där bliva uppbyggt, säger HERREN Sebaot, och mätsnöret skall spännas över Jerusalem.
\par 17 Ytterligare må du predika och säga: Så säger HERREN Sebaot: Ännu en gång skola mina städer få njuta överflöd av goda håvor; ja, HERREN skall ännu en gång trösta Sion, och ännu en gång skall han utvälja Jerusalem."
\par 18 Sedan lyfte jag upp mina ögon och fick då se fyra horn.
\par 19 Då frågade jag ängeln som talade med mig: "Vad betyda dessa?" Han svarade mig: "Detta är de horn som hava förstrött Juda, Israel och Jerusalem."
\par 20 Sedan lät HERREN mig se fyra smeder.
\par 21 Då frågade jag: "I vad ärende hava dessa kommit?" Han svarade: "De förra voro de horn som förströdde Juda, så att ingen kunde upplyfta sitt huvud; men nu hava dessa kommit för att injaga skräck hos dem, och för att slå av hornen på de hednafolk som hava lyft sitt horn mot Juda land, till att förströ dess inbyggare."

\chapter{2}

\par 1 Sedan lyfte jag upp mina ögon och fick då se en man som hade ett mätsnöre i sin hand.
\par 2 Då frågade jag: "Vart går du?" Han svarade mig: "Till att mäta Jerusalem, för att se huru brett och huru långt det skall bliva."
\par 3 Då fick jag se ängeln som talade ned mig komma fram, och en annan ängel kom emot honom.
\par 4 Och han sade till denne: "Skynda åstad och tala till den unge mannen där och säg: 'Jerusalem skall ligga såsom en obefäst plats; så stor myckenhet av människor och djur skall finnas därinne.
\par 5 Men jag själv, säger HERREN, skall vara en eldsmur däromkring, och jag skall bevisa mig härlig därinne.'"
\par 6 Upp, upp! Flyn bort ur nordlandet, säger HERREN, I som av mig haven blivit förströdda åt himmelens fyra väderstreck, säger HERREN.
\par 7 Upp, Sion! Rädda dig, du som nu bor hos dottern Babel!
\par 8 Ty så säger HERREN Sebaot, han som har sänt mig åstad för att förhärliga sig, så säger han om hednafolken, vilka plundrade eder (ty den som rör vid eder, han rör vid hans ögonsten):
\par 9 "Se, jag lyfter min hand mot dem, och de skola bliva ett byte för sina forna trälar; och I skolen så förnimma att HERREN Sebaot har sänt mig."
\par 10 Jubla och gläd dig, du dotter Sion; ty se, jag kommer för att aga min boning i dig, säger HERREN.
\par 11 Och då skola många hednafolk sluta sig till HERREN och bliva mitt folk. Ja, jag skall taga min boning i dig; och du skall förnimma att HERREN Sebaot har sänt mig till dig.
\par 12 Och HERREN skall hava Juda till I sin arvedel i det heliga landet, och ännu en gång skall han utvälja Jerusalem.
\par 13 Allt kött vare stilla inför HERREN; ty han har stått upp och trätt fram ur sin heliga boning.

\chapter{3}

\par 1 Sedan lät han mig se översteprästen Josua stående inför HERRENS ängel; och Åklagaren stod vid hans högra sida för att anklaga honom.
\par 2 Men HERREN sade till Åklagaren: "HERREN skall näpsa dig, du Åklagare; ja, HERREN skall näpsa dig, han som har utvalt Jerusalem. Är då icke denne en brand, ryckt ur elden?"
\par 3 Och Josua var klädd i orena kläder, där han stod inför ängeln
\par 4 Och denne tog till orda och sade till dem som stodo där såsom hans tjänare: "Tagen av honom de orena kläderna." Och till honom själv sade han: "Se, jag har tagit bort ifrån dig din missgärning, och man skall nu kläda dig i högtidskläder."
\par 5 Då sade jag: "Må man ock sätta en ren bindel på hans huvud." Och de satte en ren bindel på hans huvud och klädde på honom andra kläder, medan HERRENS ängel stod därbredvid.
\par 6 Och HERRENS ängel betygade för Josua och sade:
\par 7 "Så säger HERREN Sebaot: Om du vandrar på mina vägar och håller vad jag har bjudit dig hålla, så skall du ock få styra mitt hus och vakta mina gårdar; och jag skall låta dig hava din gång bland dessa som här göra tjänst.
\par 8 Hör härpå, Josua, du överstepräst, med dina medbröder, som sitta här inför dig (ty dessa män skola vara ett tecken): Se, jag vill låta min tjänare Telningen komma;
\par 9 ty se, i den sten som jag har lagt inför Josua - över vilken ena sten sju ögon vaka - i den skall jag inrista den inskrift som hör därtill, säger HERREN Sebaot; och jag skall utplåna detta lands missgärning på en enda dag.
\par 10 På den tiden, säger HERREN Sebaot, skall var och en av eder kunna inbjuda den andre till gäst under sitt vinträd och fikonträd."

\chapter{4}

\par 1 Sedan blev jag av ängeln som talade med mig åter uppväckt, likasom när någon väckes ur sömnen.
\par 2 Och han sade till mig: "Vad ser du?" Jag svarade: "Jag ser en ljusstake, alltigenom av guld, med sin oljeskål ovantill och med sina sju lampor; och sju rör gå till de särskilda lamporna därovantill.
\par 3 Och två olivträd sträcka sig över den, ett på högra sidan om skålen och ett på vänstra."
\par 4 Sedan frågade jag och sade till ängeln som talade med mig: "Vad betyda dessa ting, min herre?"
\par 5 Men ängeln som talade med mig svarade och sade till mig: "Förstår du då icke vad de betyda?" Jag svarade: "Nej, min herre."
\par 6 Då talade han och sade till mig: "Detta är HERRENS ord till Serubbabel: Icke genom någon människas styrka eller kraft skall det ske, utan genom min Ande, säger HERREN Sebaot.
\par 7 Vilket du än må vara, du stora berg som reser dig mot Serubbabel, så skall du ändå förvandlas till jämn mark. Ty han skall få föra fram slutstenen under jubelrop: 'Nåd, nåd må vila över den!'"
\par 8 Vidare kom HERRENS ord till mig; han sade:
\par 9 "Serubbabels händer hava lagt grunden till detta hus; hans händer skola ock få fullborda det. Och du skall förnimma att HERREN Sebaot har sänt mig till eder.
\par 10 Ty vem är den som vill förakta: den ringa begynnelsens dag, när dessa sju glädjas över att se murlodet i Serubbabels hand, dessa HERRENS ögon, som överfara hela jorden?"
\par 11 Och jag frågade och sade till honom: "Vad betyda dessa två olivträd, det på högra och det på vänstra sidan om ljusstaken?"
\par 12 Och ytterligare frågade jag och sade till honom: "Vad betyda de två olivkvistar som sträcka sig intill de två gyllene rännor genom vilka den gyllene oljan ledes ditned?"
\par 13 Då sade han till mig: "Förstår du då icke vad de betyda?" Jag svarade: "Nej, min herre."
\par 14 Då sade han: "Dessa äro de två oljesmorda som stå såsom tjänare inför hela jordens Herre."

\chapter{5}

\par 1 När jag sedan åter lyfte upp mina ögon, fick jag se en bokrulle flyga fram.
\par 2 Och han sade till mig: "Vad ser du?" Jag svarade: "Jag ser en bokrulle flyga fram; den är tjugu alnar lång och tio alnar bred."
\par 3 Då sade han till mig: "Detta är Förbannelsen, som går ut över hela landet; ty i kraft av den skall var och en som stjäl varda bortrensad härifrån, och i kraft av den skall var och en som svär varda bortrensad härifrån.
\par 4 Jag har låtit den gå ut, säger HERREN Sebaot, för att den skall komma in i tjuvens hus och in i dens hus, som svär falskt vid mitt namn; och den skall stanna där i huset och fräta upp det med både trävirke och stenar."
\par 5 Sedan kom ängeln som talade med mig fram och sade till mig: "Lyft upp dina ögon och se vad det är som där kommer fram.
\par 6 Och jag frågade: "Vad är det?" Han svarade: "Det är en sädesskäppa som kommer fram." Ytterligare sade han: "Så är det beställt med dem i hela landet."
\par 7 Jag fick då se huru en rund skiva av bly lyfte sig; och nu syntes där en kvinna som satt i skäppan.
\par 8 Därefter sade han: "Detta är Ogudaktigheten." Och så stötte han henne åter ned i skäppan och slog igen blylocket över dess öppning.
\par 9 När jag sedan lyfte upp mina ögon, fick jag se två kvinnor komma fram; och vinden fyllde deras vingar, ty de hade vingar lika hägerns. Och de lyfte upp skäppan mellan jord och himmel.
\par 10 Då frågade jag ängeln som talade med mig: "Vart föra de skäppan?"
\par 11 Han svarade mig: "De hava i sinnet att bygga ett hus åt henne i Sinears land; och när det är färdigt, skall hon där bliva nedsatt på sin plats."

\chapter{6}

\par 1 När jag sedan åter lyfte upp mina ögon, fick jag se fyra vagnar komma fram mellan två berg, och bergen voro av koppar.
\par 2 För den första vagnen voro röda hästar, för den andra vagnen svarta hästar,
\par 3 för den tredje vagnen vita hästar, och för den fjärde vagnen fläckiga hästar, starkare än de andra.
\par 4 Då tog jag till orda och frågade ängeln som talade med mig: "Vad betyda dessa, min herre?"
\par 5 Ängeln svarade och sade till mig: "Det är himmelens fyra vindar, vilka nu draga ut, sedan de hava fått träda inför hela jordens Herre.
\par 6 Vagnen med de svarta hästarna drager mot nordlandet, de vita följa efter dem, och de fläckiga draga mot sydlandet."
\par 7 Men när de som voro starkast skulle draga ut, voro de ivriga att få fara omkring på jorden. Då sade han: "Ja, I mån fara omkring på jorden." Och de foro åstad omkring på jorden.
\par 8 Därefter ropade han på mig och talade till mig och sade: "Se, genom dem som draga ut mot nordlandet skall jag släcka min vrede på nordlandet."
\par 9 Och HERRENS ord kom till mig; han sade:
\par 10 Tag emot av Heldai gåvorna från de landsflyktiga, från Tobia och Jedaja; du själv må redan samma dag gå åstad bort till Josias, Sefanjas sons, hus, dit dessa hava kommit från Babel.
\par 11 När du så har fått silver och guld, skall du därav låta göra kronor; och du skall sätta en på översteprästen Josuas, Josadaks sons, huvud.
\par 12 Och du skall säga till honom: "Så säger HERREN Sebaot: Se, där är en man som skall kallas Telningen. Under honom skall det gro, och han skall bygga upp HERRENS tempel.
\par 13 Ja, han skall bygga upp HERRENS tempel och förvärva majestät och sitta på sin tron och regera; och en präst skall han vara på sin tron; och fridens rådslag skola vara mellan båda.
\par 14 Men kronorna skola förvaras i HERRENS tempel, till en åminnelse av Helem, Tobia och Jedaja och Hen, Sefanjas son.
\par 15 Och fjärran ifrån skall man komma och bygga på HERRENS tempel; och I skolen så förnimma att HERREN Sebaot har sänt mig till eder. Och så skall ske, om I troget hören HERRENS, eder Guds, röst."

\chapter{7}

\par 1 Och i konung Darejaves' fjärde regeringsår hände sig att HERRENS ord kom till Sakarja, på fjärde dagen i nionde månaden, det är Kisleu.
\par 2 Ty från Betel hade då Sareser och Regem-Melek med sina män blivit sända för att bönfalla inför HERREN;
\par 3 de skulle nämligen fråga prästerna i HERREN Sebaots hus och profeterna sålunda: "Skola vi framgent hålla gråtodag och späka oss i femte månaden, såsom vi hava gjort nu i så många år?"
\par 4 Så kom då HERREN Sebaots ord till mig; han sade:
\par 5 Säg till allt folket i landet och till prästerna: När I nu under sjuttio år haven hållit faste- och klagodagar i femte och i sjunde månaden, har det då varit åt mig som I haven hållit fasta?
\par 6 Och när I äten och dricken, är det icke då för eder egen räkning som I äten och dricken?
\par 7 Haven I icke förnummit de ord som HERREN lät predika genom forna tiders profeter, medan Jerusalem ännu tronade i god ro med sina städer omkring sig, och medan Sydlandet och Låglandet ännu voro bebodda?
\par 8 Vidare kom HERRENS ord till Sakarja; han sade:
\par 9 Så sade ju HERREN Sebaot: "Dömen rätta domar, och bevisen varandra kärlek och barmhärtighet.
\par 10 Förtrycken icke änkan och den faderlöse, främlingen och den fattige, och tänken icke i edra hjärtan ut ont mot varandra."
\par 11 Men de ville icke akta därpå, utan spjärnade emot i gensträvighet och tillslöto sina öron för att slippa att höra.
\par 12 ja, de gjorde sina hjärtan hårda såsom diamant, för att slippa att höra den lag och de ord som HERREN Sebaot genom sin Ande hade sänt, förmedelst forna tiders profeter. Därför utgick stor vrede från HERREN Sebaot.
\par 13 Och det skedde, att likasom de icke hade velat höra, när han ropade, så sade nu HERREN Sebaot: "Jag vill icke höra, när de ropa;
\par 14 utan jag skall förströ dem såsom genom en stormvind bland allahanda hednafolk som de icke känna." Så har nu landet blivit öde efter dem, så att ingen färdas där, vare sig fram eller tillbaka; ty de hava så gjort det ljuvliga landet till en ödemark.

\chapter{8}

\par 1 Vidare kom HERREN Sebaots ord; han sade:
\par 2 Så säger HERREN Sebaot: Jag har stor nitälskan för Sion, ja, med stor vrede nitälskar jag för henne.
\par 3 Så säger HERREN: Jag vill vända: åter till Sion och taga min boning i Jerusalem; och Jerusalem skall kallas "den trogna staden", och HERREN Sebaots berg "det heliga berget".
\par 4 Så säger HERREN Sebaot: Ännu en gång skola gamla män och gamla kvinnor vara att finna på gatorna i Jerusalem, var och en med sin stav i handen, för hög ålders skull.
\par 5 Och stadens gator skola vara fulla av gossar och flickor, som leka där på gatorna,
\par 6 Så säger HERREN Sebaot. Om än sådant på den tiden kan komma att synas alltför underbart för kvarlevan av detta folk, icke måste det väl därför synas alltför underbart också för mig? säger HERREN Sebaot.
\par 7 Så säger HERREN Sebaot: Se, jag skall frälsa mitt folk ut ur både österland och västerland
\par 8 och låta dem komma och bosätta sig i Jerusalem; och de skola vara mitt folk, och jag skall vara deras Gud, i sanning och rättfärdighet.
\par 9 Så säger HERREN Sebaot: Fatten mod, I som i denna tid hören dessa ord av samma profeters mun, som talade på den tid då grunden lades till HERREN Sebaots hus, templet som skulle byggas upp.
\par 10 Ty före den tiden var människornas arbete lönlöst och djurens arbete fåfängt; ingen hade någon ro för sina ovänner, vare sig han gick ut eller in, ty jag eggade alla människor mot varandra.
\par 11 Men nu är jag icke mer så sinnad mot kvarlevan av detta folk som jag var i forna dagar, säger HERREN Sebaot.
\par 12 Ty nu skall friden skaffa utsäde, vinträdet skall giva sin frukt, jorden skall giva sin gröda, och himmelen skall giva sin dagg; och jag skall låta kvarlevan av detta folk få allt detta till sin arvedel.
\par 13 Och det skall ske, att likasom I, både Juda hus och Israels hus, haven varit ett exempel som man har nämnt, när man förbannade bland hedningarna, så skolen I nu, då jag har frälsat eder, tvärtom bliva nämnda, när man välsignar. Frukten icke, utan fatten mod.
\par 14 Ty så säger HERREN Sebaot: Likasom jag, när edra fäder förtörnade mig, beslöt att göra eder ont, säger HERREN Sebaot, och icke sedan ångrade det,
\par 15 så har jag tvärtom i denna tid beslutit att göra gott mot Jerusalem och Juda hus; frukten icke.
\par 16 Men detta är vad I skolen göra: Talen sanning med varandra; domen rätta och fridsamma domar i edra portar.
\par 17 Tänken icke i edra hjärtan ut ont mot varandra, och älsken icke falska eder; ty allt sådant hatar jag, säger HERREN.
\par 18 Och HERREN Sebaots ord kom till mig; han sade:
\par 19 Så säger HERREN Sebaot: Fastedagarna i fjärde, femte, sjunde och tionde månaden skola för Juda hus bliva till fröjd och glädje och till sköna högtider. Men älsken sanning och frid.
\par 20 Så säger HERREN Sebaot: Ännu en gång skall det ske att folk skola komma hit och många städers invånare;
\par 21 och invånarna i den ena staden skola gå till den andra och säga: "Upp, låtom oss gå åstad och bönfalla inför HERREN och söka HERREN Sebaot; jag själv vill ock gå åstad."
\par 22 Ja, många folk och mäktiga hednafolk skola komma och söka HERREN Sebaot i Jerusalem och bönfalla inför HERREN,
\par 23 Så säger HERREN Sebaot: På den tiden skall det ske att tio män av allahanda tungomål som talas bland hednafolken skola fatta en judisk man i mantelfliken och säga: "Låt oss gå med eder, ty vi hava hört att Gud är med eder."

\chapter{9}

\par 1 Detta är en utsaga som innehåller HERRENS ord över Hadraks land; också i Damaskus skall den slå ned - ty HERREN har sitt öga på andra människor såväl som på Israels alla stammar.
\par 2 Den drabbar ock Hamat, som gränsar därintill, så ock Tyrus och Sion, där man är så vis.
\par 3 Tyrus byggde sig ett fäste; hon hopade silver såsom stoft och guld såsom orenlighet på gatan.
\par 4 Men se, Herren skall åter göra henne fattig, han skall bryta hennes murar ned i havet, och hon själv skall förtäras av eld.
\par 5 Askelon må se det med fruktan, och Gasa med stor bävan, så ock Ekron, ty dess hopp skall komma på skam. Gasa mister sin konung, Askelon bliver en obebodd plats
\par 6 Asdod skall bebos av en vanbördig hop; så skall jag utrota filistéernas stolthet.
\par 7 Men när jag har ryckt blodmaten ur deras mun och styggelserna undan deras tänder då skall också av dem bliva en kvarleva åt vår Gud, de skola bliva såsom stamfurstar i Juda, och Ekrons folk skall bliva såsom jebuséerna.
\par 8 Och jag skall slå upp mitt läger till ett värn för mitt hus, mot härar som komma eller gå, och ej mer skall någon plågare komma över dem; ty jag vaktar nu med öppna ögon.
\par 9 Fröjda dig storligen, du dotter Sion; höj jubelrop, du dotter Jerusalem. Se, din konung kommer till dig; rättfärdig och segerrik är han. Han kommer fattig, ridande på en åsna, på en åsninnas fåle.
\par 10 Så skall jag utrota vagnar Efraim och hästar ur Jerusalem; ja, stridens bågar skola utrotas, och han skall tala frid till folken. Och hans herradöme skall nå från hav till hav, och ifrån floden intill jordens ändar.
\par 11 För ditt förbundsblods skull vill jag ock släppa dina fångar fria ur gropen där intet vatten finnes.
\par 12 Så vänden då åter till edert fäste, I fångar som nu haven ett hopp. Ja, det vare eder förkunnat i dag att jag vill giva eder dubbelt igen.
\par 13 Ty jag skall spänna Juda såsom min båge och lägga Efraim såsom pil på den, och dina söner, du Sion, skall jag svänga såsom spjut mot dina söner, du Javan, och göra dig lik en hjältes svärd.
\par 14 Ja, HERREN skall uppenbara sig i höjden, och såsom en ljungeld skall hans pil fara ut; Herren, HERREN skall stöta i basunen, och med sunnanstormar skall han draga fram.
\par 15 HERREN Sebaot, han skall beskärma dem; de skola uppsluka sina fiender och trampa deras slungstenar under fötterna; under stridslarm skola de svälja sina fiender såsom vin, till dess de själva äro så fulla av blod som offerskålar och altarhörn.
\par 16 Och HERREN, deras Gud, skall på den tiden giva dem seger, ty de äro ju det folk som han har till sin hjord. Ja, ädelstenar äro de i en krona, strålande över hans land.
\par 17 Huru stor bliver icke deras lycka, huru stor deras härlighet! Av deras säd skola ynglingar blomstra upp, och jungfrur av deras vin.

\chapter{10}

\par 1 HERREN mån I bedja om regn, när vårregnets tid är inne; HERREN är det som sänder ljungeldarna. Regn skall han då giva människorna i ymnigt mått, gröda på marken åt var och en..
\par 2 Men husgudarnas tal är fåfänglighet, spåmännens syner äro lögn, tomma drömmar är vad de tala, och den tröst de giva är ett intet. Därför måste folket draga hädan såsom en fårhjord och fara illa, där det går utan herde.
\par 3 Mot herdarna är min vrede upptänd och bockarna skall jag hemsöka. Ja, HERREN Sebaot skall vårda sig om sin hjord, Juda hus, och skall så göra den till en stolt stridshäst åt sig.
\par 4 Från den hjorden skall en hörnsten komma, från den en stödjepelare, från den en båge till striden, från den allt vad styresman heter.
\par 5 Och de skola vara lika hjältar, som gå fram i striden, likasom trampade de i orenlighet på gatan. Ja, strida skola de, ty HERREN är med dem, och ryttarna på sina hästar skola då komma på skam.
\par 6 Jag skall giva styrka åt Juda hus, och åt Josefs hus skall jag giva seger. Jag skall i min barmhärtighet låta dem komma tillbaka, och det skall bliva såsom hade jag aldrig förkastat dem. Ty jag är HERREN, deras Gud, och skall bönhöra dem.
\par 7 Och Efraims män skola bliva lika hjältar, och deras hjärtan skola glädja sig såsom av vin. Deras barn skola ock se det och bliva glada; deras hjärtan skola fröjda sig i HERREN.
\par 8 Jag skall locka på dem och samla dem tillhopa, ty jag förlossar dem; och de skola bliva lika talrika som de fordom voro.
\par 9 Och när jag planterar ut dem bland folken, skola de tänka på mig i fjärran land; och med sina barn skola de få leva och komma tillbaka.
\par 10 Ja, från Egyptens land skall jag låta dem komma tillbaka och från Assur skall jag församla dem. Sedan skall jag föra dem till Gileads land och till Libanon; och där skall icke finnas rum nog för dem.
\par 11 Han skall draga fram genom havet på en trång väg, böljorna i havet skall han slå ned, och alla Nilflodens djup skola torka ut. Så skall Assurs stolthet bliva nedbruten och spiran tagen ifrån Egypten.
\par 12 Men dem skall jag göra starka i HERREN, och i hans namn skola de gå fram, säger HERREN.

\chapter{11}

\par 1 Öppna dina dörrar, Libanon, ty eld skall nu förtära dina cedrar.
\par 2 Jämra dig, du cypress, ty cedern måste falla, de härliga träden skola förödas. Jämren eder, I Basans ekar, ty skogen, den ogenomträngliga, varder fälld.
\par 3 Hör huru herdarna jämra sig, när deras härlighet bliver förödd! Hör huru de unga lejonen ryta, när Jordanbygdens snår varda förödda!
\par 4 Så sade HERREN, min Gud: "Bliv en herde för slaktfåren,
\par 5 ty deras köpare slakta dem utan all försyn, och deras säljare säga: 'Lovad vare HERREN att jag bliver allt rikare.' Ej heller skonas de av sina egna herdar."
\par 6 "Se, jag vill nu icke mer skona landets inbyggare", säger HERREN, "utan låta människorna falla i varandras hand och i sina konungars hand, och dessa skola fördärva landet, och jag skall icke rädda någon ur deras hand."
\par 7 Så blev jag en herde för slaktfåren, de arma fåren. Och jag tog mig två stavar, den ena kallade jag Ljuvlig ro, den andra kallade jag Endräkt; och jag vaktade så fåren
\par 8 Men sedan jag inom en månad hade förgjort de tre herdarna, blev jag led vid fåren, likasom ock deras sinne var avogt mot mig.
\par 9 Då sade jag: "Jag vill icke mer vara eder herde. Vad som vill dö, det må dö, och vad som vill förgås, det må förgås; och de som sedan bliva kvar må äta varandras kött."
\par 10 Så tog jag min stav Ljuvlig ro och bröt sönder den, till att upplösa det förbund som jag hade slutit med alla folk.
\par 11 Och när detta nu på den dagen blev upplöst, förnummo de arma fåren, som aktade på mig, att det var HERRENS ord.
\par 12 Därefter sade jag till dem: "Om I så finnen för gott, så given mig min lön; varom icke, må det så vara." Och de vägde upp trettio siklar silver såsom lön åt mig.
\par 13 Då sade HERREN till mig: "Kasta det åt krukmakaren" - det härliga pris vartill jag hade blivit värderad av dem! Och jag tog de trettio silversiklarna och kastade dem i HERRENS hus åt krukmakaren.
\par 14 Därefter bröt jag sönder min andra stav, Endräkt, till att upplösa broderskapet mellan Juda och Israel.
\par 15 Och HERREN sade till mig: "Tag dig nu redskap såsom en oförnuftig herde;
\par 16 ty se, jag vill låta en herde uppstå i landet, som icke vårdar sig om de får som hålla på att förgås, icke uppsöker det förskingrade, icke helar det sargade, icke sörjer för det som är helbrägda, utan allenast äter köttet av de feta och river sönder klövarna på dem."
\par 17 Ve över denne ovärdige herde, som övergiver sin hjord! Må ett svärd träffa hans arm och hans högra öga! Må hans arm alldeles förtvina och hans högra öga förmörkas i grund!

\chapter{12}

\par 1 Detta är en utsaga som innehåller HERRENS ord över Israel. Så säger HERREN, han som har utspänt himmelen och grundat jorden och danat människans ande i henne:
\par 2 Se, jag skall göra Jerusalem till an berusningens kalk för alla folk runt omkring; jämväl över Juda skall det komma, när Jerusalem bliver belägrat.
\par 3 Och det skall ske på den tiden att jag skall göra Jerusalem till en lyftesten för alla folk; var och en som försöker lyfta den skall illa sarga sig därpå. Och alla jordens folk skola församla sig mot det.
\par 4 På den tiden, säger HERREN, skall jag slå alla hästar med förvirring och deras ryttare med vanvett. Men över Juda hus skall jag upplåta mina ögon, när jag bland hednafolken slår alla hästar med blindhet.
\par 5 Då skola Juda stamfurstar säga i sina hjärtan: "Jerusalems invånare äro vår styrka, genom HERREN Sebaot, sin Gud."
\par 6 På den tiden skall jag låta Juda stamfurstar bliva såsom brinnande fyrfat bland ved, och såsom eldbloss bland halmkärvar, så att de förbränna alla folk runt omkring, både åt höger och åt vänster; men Jerusalem skall framgent trona på sin plats, där Jerusalem nu är.
\par 7 Och först skall HERREN giva seger åt Juda hyddor, för att icke Davids hus och Jerusalems invånare skola tillräkna sig större ära än Juda.
\par 8 På den tiden skall HERREN beskärma Jerusalems invånare; den skröpligaste bland dem skall på den tiden vara såsom David, och Davids hus skall vara såsom ett gudaväsen, såsom HERRENS ängel framför dem.
\par 9 Och jag skall på den tiden sätta mig i sinnet att förgöra alla folk som komma mot Jerusalem.
\par 10 Men över Davids hus och över Jerusalems invånare skall jag utgjuta en nådens och bönens ande, så att de se upp till mig, och se vem de hava stungit. Och de skola hålla dödsklagan efter honom, såsom man håller dödsklagan efter ende sonen, och skola bittert sörja honom, såsom man sörjer sin förstfödde.
\par 11 Ja, på den tiden skall i Jerusalem hållas stor dödsklagan, sådan den var, som hölls i Hadadrimmon på Megiddons slätt.
\par 12 Och släkterna i landet skola hålla dödsklagan var för sig: Davids hus släkt för sig, och dess kvinnor för sig, Natans hus' släkt för sig, och dess kvinnor för sig,
\par 13 Levi hus' släkt för sig, och dess kvinnor för sig, Simeis släkt för sig, och dess kvinnor för sig;
\par 14 så ock alla övriga släkter var för sig, och deras kvinnor för sig.

\chapter{13}

\par 1 På den tiden skola Davids hus och Jerusalems invånare få en öppen brunn, till att avtvå sin synd och orenhet.
\par 2 Och det skall ske på den tiden, säger HERREN Sebaot, att jag skall utrota avgudarnas namn ur landet, så att de icke mer skola nämnas; profeterna och orenhetens ande skall jag ock skaffa bort ur landet.
\par 3 Och det skall ske, att om någon därefter uppträder såsom profet, så skola hans egna föräldrar, hans fader och moder, säga till honom: "Du kan icke få leva, du som talar lögn i HERRENS namn." Och hans egna föräldrar, hans fader och moder, skola stinga ihjäl honom, när han vill profetera.
\par 4 Och det skall ske på den tiden att alla profeter skola blygas för sina syner, när de vilja profetera; och för att icke bliva röjda skola de icke mer kläda sig i mantel av hår.
\par 5 Och var och en av dem skall säga: "Jag är ingen profet, en åkerman är jag; redan i min ungdom blev jag köpt till träl."
\par 6 Och om man då frågar honom: "Vad är det för sår du har på din kropp?" så skall han svara: "Dem har jag fått därhemma, hos mina närmaste."
\par 7 Svärd, upp mot min herde, mot den man som fick stå mig nära! säger HERREN Sebaot. Må herden bliva slagen, så att fåren förskingras; ty jag vill nu vända min hand mot de svaga.
\par 8 Och det skall ske i hela landet, säger HERREN, att två tredjedelar där skola utrotas och förgås; allenast en tredjedel skall där lämnas kvar.
\par 9 Och den tredjedelen skall jag låta gå genom eld; jag skall luttra dem, såsom man luttrar silver, och pröva dem, såsom man prövar guld. Så skola de åkalla mitt namn, och jag skall bönhöra dem. Jag skall säga: "Detta är mitt folk." Och det skall svara: "HERREN är min Gud."

\chapter{14}

\par 1 Se, en dag skall komma, en HERRENS dag, då man i dig skall utskifta byte.
\par 2 Ty jag skall församla alla folk till strid mot Jerusalem; och staden skall intagas, och husen skola plundras och kvinnorna skändas. Och hälften av folket i staden skall föras bort i fångenskap. Men återstoden därav skall icke bliva utrotad ur staden;
\par 3 ty HERREN skall draga ut och strida mot de folken, såsom han stridde förr på drabbningens dag.
\par 4 Och han skall den dagen stå med sina fötter på Oljeberget, gent emot Jerusalem, österut; och Oljeberget skall rämna mitt itu, mot öster och väster, till en mycket stor dal, i det att ena hälften av berget viker undan mot norr, och andra hälften därav mot söder.
\par 5 Och I skolen fly ned i dalen mellan mina berg, ty dalen mellan bergen skall räcka ända till Asel; I skolen fly, såsom I flydden för jordbävningen i Ussias, Juda konungs, tid. Då skall HERREN, min Gud, komma, ja, du själv och alla heliga med dig.
\par 6 Och det skall ske på den dagen att ljuset skall bliva borta, ty himlaljusen skola förmörkas.
\par 7 Och det bliver en dag som är ensam i sitt slag, och som är känd av HERREN, en dag då det varken är dag eller natt, en dag då det bliver ljust, när aftonen kommer.
\par 8 Och det skall ske på den tiden att rinnande vatten skola utgå från Jerusalem, ena hälften mot Östra havet och andra hälften mot Västra havet; både sommar och vinter skall det vara så.
\par 9 Och HERREN skall då vara konung över hela jorden; ja, på den tiden skall HERREN vara en, och hans namn ett.
\par 10 Hela landet från Geba till Rimmon, söder om Jerusalem, skall då förvandlas till en slättmark; men själva staden skall trona på sin höjd, och sträcka sig från Benjaminsporten ända till den plats där den förra porten stod, till Hörnporten och Hananeltornet och till de kungliga vinpressarna;
\par 11 och folket skall bo där i ro och icke mer givas till spillo, ty Jerusalem skall trona i trygghet.
\par 12 Men denna hemsökelse skall HERREN låta drabba alla de folk som drogo ut för att strida mot Jerusalem: han skall låta deras kött ruttna, medan de ännu stå på sina fötter; deras ögon skola ruttna i sina hålor och deras tunga skall ruttna i deras mun.
\par 13 Och det skall ske på den tiden att HERREN skall sända en stor förvirring bland dem; de skola bära hand på varandra, och den enes hand skall lyftas mot den andres.
\par 14 Också Juda skall strida mot Jerusalem. Och skatter skola samlas tillhopa från alla folk runt omkring: guld, silver och kläder i stor myckenhet.
\par 15 En likadan hemsökelse skall ock drabba hästar, mulåsnor, kameler, åsnor och alla andra djur som finnas där i lägren.
\par 16 Och det skall ske att alla överblivna ur alla de folk som kommo mot Jerusalem skola år efter år draga ditupp, för att tillbedja konungen HERREN Sebaot, och för att fira lövhyddohögtiden.
\par 17 Men om någon av jordens folkstammar icke drager upp till Jerusalem, för att tillbedja konungen HERREN Sebaot, då skall över den icke komma något regn.
\par 18 Om Egyptens folkstam icke drager åstad och kommer ditupp, så skall ej heller över den komma regn. Detta bliver den hemsökelse som HERREN skall låta drabba de folk som icke draga upp för att fira lövhyddohögtiden.
\par 19 Ja, så skall Egypten drabbas av sin synd, så skola ock alla andra folk drabbas av sin synd, om de icke draga upp för att fira lövhyddohögtiden.
\par 20 På den tiden skall på hästarnas bjällror stå att läsa: "Helgad åt HERREN", och grytorna i HERRENS hus skola vara såsom offerskålarna framför altaret.
\par 21 Och var gryta i Jerusalem och Juda skall vara helgad åt HERREN Sebaot, så att var och en som vill offra kan komma och taga en sådan och koka i den. Och ingen kanané skall mer finnas i HERREN Sebaots hus, på den tiden.


\end{document}