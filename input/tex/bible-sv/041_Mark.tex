\begin{document}

\title{Markus}


\chapter{1}

\par 1 Detta är begynnelsen av evangelium om Jesus Kristus, Guds Son.
\par 2 Så är skrivet hos profeten Esaias: "Se, jag sänder ut min ängel framför dig, och han skall bereda vägen för dig.
\par 3 Hör rösten av en som ropar i öknen: 'Bereden vägen för Herren, gören stigarna jämna för honom.'"
\par 4 I enlighet härmed uppträdde Johannes döparen i öknen och predikade bättringens döpelse till syndernas förlåtelse.
\par 5 Och hela judiska landet och alla Jerusalems invånare gingo ut till honom och läto döpa sig av honom i floden Jordan, och bekände därvid sina synder.
\par 6 Och Johannes hade kläder av kamelhår och bar en lädergördel om sina länder och levde av gräshoppor och vildhonung.
\par 7 Och han predikade och sade: "Efter mig kommer den som är starkare än jag; jag är icke ens värdig att böja mig ned för att upplösa hans skorem.
\par 8 Jag döper eder med vatten, men han skall döpa eder med helig ande."
\par 9 Och det hände sig vid den tiden att Jesus kom från Nasaret i Galileen. Och han lät döpa sig i Jordan av Johannes.
\par 10 Och strax då han steg upp ur vattnet, såg han himmelen dela sig och Anden såsom en duva sänka sig ned över honom.
\par 11 Och en röst kom från himmelen: "Du är min älskade Son; i dig har jag funnit behag."
\par 12 Strax därefter förde Anden honom ut i öknen.
\par 13 Och han var i öknen i fyrtio dagar och frestades av Satan och levde bland vilddjuren; och änglarna betjänade honom.
\par 14 Men sedan Johannes hade blivit satt i fängelse, kom Jesus till Galileen och predikade Guds evangelium
\par 15 och sade: "Tiden är fullbordad, och Guds rike är nära; gören bättring, och tron evangelium."
\par 16 När han nu gick fram utmed Galileiska sjön, fick han se Simon och Simons broder Andreas kasta ut nät i sjön, ty de voro fiskare.
\par 17 Och Jesus sade till dem: "Följen mig, så skall jag göra eder till människofiskare."
\par 18 Strax lämnade de näten och följde honom.
\par 19 När han hade gått litet längre fram, fick han se Jakob, Sebedeus' son, och Johannes, hans broder, där de sutto i båten, också de, och ordnade sina nät.
\par 20 Och strax kallade han dem till sig; och de lämnade sin fader Sebedeus med legodrängarna kvar i båten och följde honom.
\par 21 Sedan begåvo de sig in i Kapernaum; och strax, på sabbaten, gick han in i synagogan och undervisade.
\par 22 Och folket häpnade över hans förkunnelse; ty han förkunnade sin lära för dem med makt och myndighet, och icke såsom de skriftlärde.
\par 23 Strax härefter befann sig i deras synagoga en man som var besatt av en oren ande. Denne ropade
\par 24 och sade: "Vad har du med oss att göra, Jesus från Nasaret? Har du kommit för att förgöra oss? Jag vet vem du är, du Guds Helige."
\par 25 Men Jesus tilltalade honom strängt och sade: "Tig, och far ut ur honom."
\par 26 Då slet och ryckte den orene anden honom och ropade med hög röst och for ut ur honom.
\par 27 Och alla häpnade, så att de begynte fråga varandra och säga: "Vad är detta? Det är ju en ny lära, med makt och myndighet. Till och med de orena andarna befaller han, och de lyda honom."
\par 28 Och ryktet om honom gick strax ut överallt i hela den kringliggande trakten av Galileen.
\par 29 Och strax då de hade kommit ut ur synagogan, begåvo de sig med Jakob och Johannes till Simons och Andreas' hus.
\par 30 Men Simons svärmoder låg sjuk i feber, och de talade strax med honom om henne.
\par 31 Då gick han fram och tog henne vid handen och reste upp henne; och febern lämnade henne, och hon betjänade dem.
\par 32 Men när solen hade gått ned och det hade blivit afton, förde man till honom alla som voro sjuka eller besatta;
\par 33 och hela staden var församlad utanför dörren.
\par 34 Och han botade många som ledo av olika slags sjukdomar; och han drev ut många onda andar, men tillstadde icke de onda andarna att tala, eftersom de kände honom.
\par 35 Och bittida om morgonen, medan det ännu var mörkt, stod han upp och gick åstad bort till en öde trakt, och bad där.
\par 36 Men Simon och de som voro med honom skyndade efter honom.
\par 37 Och när de funno honom, sade de till honom: "Alla fråga efter dig."
\par 38 Då sade han till dem: "Låt oss draga bort åt annat håll, till de närmaste småstäderna, för att jag också där må predika; ty därför har jag begivit mig ut."
\par 39 Och han gick åstad och predikade i hela Galileen, i deras synagogor, och drev ut de onda andarna.
\par 40 Och en spetälsk man kom fram till honom och föll på knä och bad honom och sade till honom: "Vill du, så kan du göra mig ren."
\par 41 Då förbarmade han sig och räckte ut handen och rörde vid honom och sade till honom: "Jag vill; bliv ren."
\par 42 Och strax vek spetälskan ifrån honom, och han blev ren.
\par 43 Sedan vände Jesus strax bort honom med stränga ord
\par 44 och sade till honom: "Se till, att du icke säger något härom för någon; men gå bort och visa dig för prästen, och frambär för din rening det offer som Moses har påbjudit, till ett vittnesbörd för dem."
\par 45 Men när han kom ut, begynte han ivrigt förkunna och utsprida vad som hade skett, så att Jesus icke mer kunde öppet gå in i någon stad, utan måste hålla sig ute i öde trakter; och dit kom man till honom från alla håll.

\chapter{2}

\par 1 Några dagar därefter kom han åter till Kapernaum; och när det spordes att han var hemma,
\par 2 församlade sig så mycket folk, att icke ens platsen utanför dörren mer kunde rymma dem; och han förkunnade ordet för dem.
\par 3 Då kommo de till honom med en lam man, som bars dit av fyra män.
\par 4 Och då de för folkets skull icke kunde komma fram till honom med mannen, togo de bort taket över platsen där han var; och sedan de så hade gjort en öppning, släppte de ned sängen, som den lame låg på.
\par 5 När Jesus såg deras tro, sade han till den lame: "Min son, dina synder förlåtas dig."
\par 6 Nu sutto där några skriftlärde, och dessa tänkte i sina hjärtan:
\par 7 "Huru kan denne tala så? Han hädar ju. Vem kan förlåta synder, utom Gud allena?"
\par 8 Strax förnam då Jesus i sin ande att de tänkte så vid sig själva; och han sade till dem: "Huru kunnen I tänka sådant i edra hjärtan?
\par 9 Vilket är lättare, att säga till den lame: 'Dina synder förlåtas dig' eller att säga: 'Stå upp, tag din säng och gå'?
\par 10 Men för att I skolen veta att Människosonen har makt här på jorden att förlåta synder,
\par 11 så säger jag dig" (och härmed vände han sig till den lame): "Stå upp, tag din säng och gå hem."
\par 12 Då stod han upp och tog strax sin säng och gick ut i allas åsyn, så att de alla uppfylldes av häpnad och prisade Gud och sade: "Sådant hava vi aldrig sett."
\par 13 Åter begav han sig ut och gick längs med sjön. Och allt folket kom till honom, och han undervisade dem.
\par 14 När han nu gick där fram, fick han se Levi, Alfeus' son, sitta vid tullhuset. Och han sade till denne: "Följ mig." Då steg han upp och följde honom.
\par 15 När Jesus därefter låg till bords i hans hus, voro där såsom bordsgäster, jämte Jesus och hans lärjungar, också många publikaner och syndare; ty många sådana funnos bland dem som följde honom.
\par 16 Men när de skriftlärde bland fariséerna sågo att han åt med publikaner och syndare, sade de till hans lärjungar: "Huru kan han äta med publikaner och syndare?"
\par 17 När Jesus hörde detta, sade han till dem: "Det är icke de friska som behöva läkare, utan de sjuka. Jag har icke kommit för att kalla rättfärdiga, utan för att kalla syndare."
\par 18 Och Johannes' lärjungar och fariséerna höllo fasta. Och man kom och sade till honom: "Varför fasta icke dina lärjungar, då Johannes' lärjungar och fariséernas lärjungar fasta?"
\par 19 Jesus svarade dem: "Kunna väl bröllopsgästerna fasta, medan brudgummen ännu är hos dem? Nej, så länge de hava brudgummen hos sig, kunna de icke fasta.
\par 20 Men den tid skall komma, då brudgummen tages ifrån dem, och då, på den tiden, skola de fasta. -
\par 21 Ingen syr en lapp av okrympt tyg på en gammal mantel; om någon så gjorde, skulle det isatta nya stycket riva bort ännu mer av den gamla manteln, och hålet skulle bliva värre.
\par 22 Ej heller slår någon nytt vin i gamla skinnläglar; om någon så gjorde, skulle vinet spränga sönder läglarna, så att både vinet och läglarna fördärvades. Nej, nytt vin slår man i nya läglar."
\par 23 Och det hände sig på sabbaten att han tog vägen genom ett sädesfält; och hans lärjungar begynte rycka av axen, medan de gingo.
\par 24 Då sade fariséerna till honom: "Se! Huru kunna de på sabbaten göra vad som icke är lovligt?"
\par 25 Han svarade dem: "Haven I aldrig läst vad David gjorde, när han själv och de som följde honom kommo i nöd och blevo hungriga:
\par 26 huru han då, på den tid Abjatar var överstepräst, gick in i Guds hus och åt skådebröden - fastän det ju icke är lovligt för andra än för prästerna att äta sådant bröd - och huru han jämväl gav åt dem som följde honom?"
\par 27 Därefter sade han till dem: "Sabbaten blev gjord för människans skull, och icke människan för sabbatens skull.
\par 28 Så år då Människosonen herre också över sabbaten."

\chapter{3}

\par 1 Och han gick åter in i en synagoga. Där var då en man som hade en förvissnad hand.
\par 2 Och de vaktade på honom, för att se om han skulle bota denne på sabbaten; de ville nämligen få något att anklaga honom för.
\par 3 Då sade han till mannen som hade den förvissnade handen: "Stå upp, och kom fram."
\par 4 Sedan sade han till dem: "Vilketdera är lovligt på sabbaten: att göra vad gott är, eller att göra vad ont är, att rädda någons liv, eller att döda?" Men de tego.
\par 5 Då såg han sig omkring på dem med vrede, bedrövad över deras hjärtans förstockelse, och sade till mannen: "Räck ut din hand." Och han räckte ut den; och hans hand blev frisk igen. -
\par 6 Då gingo fariséerna bort och fattade strax, tillsammans med herodianerna, det beslutet att de skulle förgöra honom.
\par 7 Och Jesus drog sig med sina lärjungar undan till sjön, och en stor hop folk följde honom från Galileen.
\par 8 Och från Judeen och Jerusalem och Idumeen och från landet på andra sidan Jordan och från trakterna omkring Tyrus och Sidon kom en stor hop folk till honom, när de fingo höra huru stora ting han gjorde.
\par 9 Och han tillsade sina lärjungar att en båt skulle hållas tillreds åt honom, för folkets skull, för att de icke skulle tränga sig inpå honom.
\par 10 Ty han botade många och blev därför överlupen av alla som hade någon plåga och fördenskull ville röra vid honom.
\par 11 Och när de orena andarna sågo honom, föllo de ned för honom och och ropade och sade: "Du är Guds Son."
\par 12 Men han förbjöd dem strängeligen, åter och åter, att röja honom.
\par 13 Och han gick upp på berget och kallade till sig några som han själv utsåg; och de kommo till honom.
\par 14 Så förordnade han tolv som skulle följa honom, och som han ville sända ut till att predika,
\par 15 och de skulle hava makt att bota sjuka och driva ut onda andar.
\par 16 Han förordnade alltså dessa tolv: Simon, åt vilken han gav tillnamnet Petrus;
\par 17 vidare Jakob, Sebedeus' son, och Johannes, Jakobs broder, åt vilka han gav tillnamnet Boanerges (det betyder tordönsmän);
\par 18 vidare Andreas och Filippus och Bartolomeus och Matteus och Tomas och Jakob, Alfeus' son, och Taddeus och Simon ivraren
\par 19 och Judas Iskariot, densamme som förrådde honom.
\par 20 Och när han kom hem, församlade sig folket åter, så att de icke ens fingo tillfälle att äta.
\par 21 Då nu hans närmaste fingo höra härom, gingo de åstad för att taga vara på honom; ty de menade att han var från sina sinnen.
\par 22 Och de skriftlärde som hade kommit ned från Jerusalem sade att han var besatt av Beelsebul, och att det var med de onda andarnas furste som han drev ut de onda andarna.
\par 23 Då kallade han dem till sig och sade till dem i liknelser: "Huru skulle Satan kunna driva ut Satan?
\par 24 Om ett rike har kommit i strid med sig självt, så kan det riket ju icke hava bestånd;
\par 25 och om ett hus har kommit i strid med sig självt, så skall icke heller det huset kunna äga bestånd.
\par 26 Om alltså Satan har satt sig upp mot sig själv och kommit i strid med sig själv, så kan han icke äga bestånd, utan det är då ute med honom. -
\par 27 Nej, ingen kan gå in i en stark mans hus och plundra honom på hans bohag, såframt han icke förut har bundit den starke. Först därefter kan han plundra hans hus.
\par 28 Sannerligen säger jag eder: Alla andra synder skola bliva människors barn förlåtna, ja ock alla andra hädelser, huru hädiskt de än må tala;
\par 29 men den som hädar den helige Ande, han får icke någonsin förlåtelse, utan är skyldig till evig synd."
\par 30 De hade ju nämligen sagt att han var besatt av en oren ande.
\par 31 Så kommo nu hans moder och hans bröder; och de stannade därutanför och sände bud in till honom för att kalla honom ut.
\par 32 Och mycket folk satt där omkring honom; och man sade till honom: "Se, din moder och dina bröder stå härutanför och fråga efter dig."
\par 33 Då svarade han dem och sade: Vilken är min moder, och vilka äro mina bröder?"
\par 34 Och han såg sig omkring på dem som sutto där runt omkring honom, och han sade: "Se här är min moder, och här äro mina bröder!
\par 35 Den som gör Guds vilja, den är min broder och min syster och min moder."

\chapter{4}

\par 1 Och han begynte åter undervisa vid sjön. Och där församlade sig en stor hop folk omkring honom. Därför steg han i en båt; och han satt i den ute på sjön, under det att allt folket stod på land utmed sjön.
\par 2 Och han undervisade dem mycket i liknelser och sade till dem i sin undervisning:
\par 3 "Hören! En såningsman gick ut för att så.
\par 4 Då hände sig, när han sådde, att somt föll vid vägen, och fåglarna kommo och åto upp det.
\par 5 Och somt föll på stengrund, där det icke hade mycket jord, och det kom strax upp, eftersom det icke hade djup jord;
\par 6 men när solen hade gått upp, förbrändes det, och eftersom det icke hade någon rot, torkade det bort.
\par 7 Och somt föll bland törnen, och törnena sköto upp och förkvävde det, så att det icke gav någon frukt.
\par 8 Men somt föll i god jord, och det sköt upp och växte och gav frukt och bar trettiofalt och sextiofalt och hundrafalt."
\par 9 Och han tillade: "Den som har öron till att höra, han höre."
\par 10 När han sedan hade dragit sig undan ifrån folket, frågade honom de tolv, och med dem de andra som följde honom, om liknelserna.
\par 11 Då sade han till dem: "Åt eder är Guds rikes hemlighet given, men åt dem som stå utanför meddelas alltsammans i liknelser,
\par 12 för att de 'med seende ögon skola se, och dock intet förnimma, och med hörande öron höra, och dock intet förstå, så att de icke omvända sig och undfå förlåtelse'."
\par 13 Sedan sade han till dem: "Förstån I icke denna liknelse, huru skolen I då kunna fatta alla de andra liknelserna? -
\par 14 Vad såningsmannen sår är ordet.
\par 15 Och att säden såddes vid vägen, det är sagt om dem i vilka ordet väl bliver sått, men när de hava hört det, kommer strax Satan ock tager bort ordet som såddes i dem.
\par 16 Sammalunda förhåller det sig med det som sås på stengrunden: det är sagt om dem, som när de få höra ordet, väl strax taga emot det med glädje,
\par 17 men icke hava någon rot i sig, utan bliva beståndande allenast till en tid; när sedan bedrövelse eller förföljelse påkommer för ordets skull, då komma de strax på fall.
\par 18 Annorlunda förhåller det sig med det som sås bland törnena: det är sagt om dem som väl höra ordet,
\par 19 men låta tidens omsorger och rikedomens bedrägliga lockelse, och begärelser efter andra ting, komma därin och förkväva ordet, så att det bliver utan frukt.
\par 20 Men att det såddes i den goda jorden, det är sagt om dem som både höra ordet och taga emot det, och som bära frukt, trettiofalt och sextiofalt och hundrafalt."
\par 21 Och han sade till dem: "Icke tager man väl fram ett ljus, för att det skall sättas under skäppan eller under bänken; man gör det ju, för att det skall sättas på ljusstaken.
\par 22 Ty intet är fördolt, utom för att det skall bliva uppenbarat; ej heller har något blivit undangömt, utom för att det skall komma i dagen.
\par 23 Om någon har öron till att höra, så höre han."
\par 24 Och han sade till dem: "Akten på vad I hören. Med det mått som I mäten med skall ock mätas åt eder, och ännu mer skall bliva eder tilldelat.
\par 25 Ty den som har, åt honom skall varda givet; men den som icke har, från honom skall tagas också det han har."
\par 26 Och han sade: "Så är det med Guds rike, som när en man sår säd i jorden;
\par 27 och han sover, och han vaknar, och nätter och dagar gå, och säden skjuter upp och växer i höjden, han vet själv icke huru.
\par 28 Av sig själv bär jorden frukt, först strå och sedan ax, och omsider finnes fullbildat vete i axet.
\par 29 När så frukten är mogen, låter han strax lien gå, ty skördetiden är då inne."
\par 30 Och han sade: "Vad skola vi likna Guds rike vid, eller med vilken liknelse skola vi framställa det?
\par 31 Det är såsom ett senapskorn, som när det lägges ned i jorden, är minst av alla frön på jorden;
\par 32 men sedan det är nedlagt, skjuter det upp och bliver störst bland alla kryddväxter och får så stora grenar, att himmelens fåglar kunna bygga sina nästen i dess skugga."
\par 33 I många sådana liknelser förkunnade han ordet för dem, efter deras förmåga att fatta det;
\par 34 och utan liknelse talade han icke till dem. Men för sina lärjungar uttydde han allt, när de voro allena.
\par 35 Samma dag, om aftonen, sade han till dem: "Låt oss fara över till andra stranden."
\par 36 Så läto de folket gå och togo honom med i båten, där han redan förut var; och jämväl andra båtar följde med honom.
\par 37 Då kom en häftig stormvind, och vågorna slogo in i båten, så att båten redan begynte fyllas.
\par 38 Men han själv låg i bakstammen och sov, lutad mot huvudgärden. Då väckte de honom och sade till honom: "Mästare, frågar du icke efter att vi förgås?"
\par 39 När han så hade vaknat, näpste han vinden och sade till sjön: "Tig, var stilla." Och vinden lade sig, och det blev alldeles lugnt.
\par 40 Därefter sade han till dem: "Varför rädens I? Haven I ännu ingen tro?"
\par 41 Och de hade blivit mycket häpna och sade till varandra: "Vem är då denne, eftersom både vinden och sjön äro honom lydiga?"

\chapter{5}

\par 1 Så kommo de över till gerasenernas land, på andra sidan sjön.
\par 2 Och strax då han hade stigit ur båten, kom en man, som var besatt av en oren ande, emot honom från gravarna där;
\par 3 han hade nämligen sitt tillhåll bland gravarna. Och icke ens med kedjor kunde man numera fängsla honom;
\par 4 ty väl hade han många gånger blivit fängslad med fotbojor och kedjor, men han hade slitit itu kedjorna och brutit sönder fotbojorna, och ingen kunde få makt med honom.
\par 5 Och han vistades alltid, dag och natt, bland gravarna och på bergen och skriade och sargade sig själv med stenar.
\par 6 När denne nu fick se Jesus på avstånd, skyndade han fram och föll ned för honom
\par 7 och ropade med hög röst och sade: "Vad har du med mig att göra, Jesus, du Guds, den Högstes, Son? Jag besvär dig vid Gud, plåga mig icke."
\par 8 Jesus skulle nämligen just säga till honom: "Far ut ur mannen, du orena ande."
\par 9 Då frågade han honom: "Vad är ditt namn?" Han svarade honom: "Legion är mitt namn, ty vi äro många."
\par 10 Och han bad honom enträget att icke driva dem bort ifrån den trakten.
\par 11 Nu gick där vid berget en stor svinhjord i bet.
\par 12 Och de bådo honom och sade: "Sänd oss åstad in i svinen; låt oss få fara in i dem."
\par 13 Och han tillstadde dem det. Då gåvo sig de orena andarna åstad och foro in i svinen. Och hjorden, vid pass två tusen svin, störtade sig utför branten ned i sjön och drunknade i sjön.
\par 14 Men de som vaktade dem flydde och berättade härom i staden och på landsbygden; och folket gick åstad för att se vad det var som hade skett.
\par 15 När de då kommo till Jesus, fingo de se den som hade varit besatt, mannen som hade haft legionen i sig, sitta där klädd och vid sina sinnen; och de betogos av häpnad.
\par 16 Och de som hade åsett händelsen förtäljde för dem vad som hade vederfarits den besatte, och vad som hade skett med svinen.
\par 17 Då begynte folket bedja honom att han skulle gå bort ifrån deras område.
\par 18 När han sedan steg i båten, bad honom mannen som hade varit besatt, att han skulle få följa honom.
\par 19 Men han tillstadde honom det icke, utan sade till honom: "Gå hem till de dina, och berätta för dem huru stora ting Herren har gjort med dig, och huru han har förbarmat sig över dig."
\par 20 Då gick han åstad och begynte förkunna i Dekapolis huru stora ting Jesus hade gjort med honom; och alla förundrade sig.
\par 21 Och när Jesus hade farit över i båten, tillbaka till andra stranden, församlade sig mycket folk omkring honom, där han stod vid sjön.
\par 22 Då kom en synagogföreståndare, vid namn Jairus, dit; och när denne fick se honom, föll han ned för hans fötter
\par 23 och bad honom enträget och sade: "Min dotter ligger på sitt yttersta. Kom och lägg händerna på henne, så att hon bliver hulpen och får leva."
\par 24 Då gick han med mannen; och honom följde mycket folk, som trängde sig inpå honom.
\par 25 Nu var där en kvinna som hade haft blodgång i tolv år,
\par 26 och som hade lidit mycket hos många läkare och kostat på sig allt vad hon ägde, utan att det hade varit henne till något gagn; snarare hade det blivit värre med henne.
\par 27 Hon hade fått höra om Jesus och kom nu i folkhopen, bakom honom, och rörde vid hans mantel.
\par 28 Ty hon tänkte: "Om jag åtminstone får röra vid hans kläder, så bliver jag hulpen."
\par 29 Och strax uttorkade hennes blods källa, och hon kände i sin kropp att hon var botad från sin plåga.
\par 30 Men strax då Jesus inom sig förnam vilken kraft som hade gått ut ifrån honom, vände han sig om i folkhopen och frågade: "Vem rörde vid mina kläder?"
\par 31 Hans lärjungar sade till honom: "Du ser huru folket tränger på, och ändå frågar du: 'Vem rörde vid mig?'"
\par 32 Då såg han sig omkring för att få se den som hade gjort detta.
\par 33 Men kvinnan fruktade och bävade, ty hon visste vad som hade skett med henne; och hon kom fram och föll ned för honom och sade honom hela sanningen.
\par 34 Då sade han till henne: "Min dotter, din tro har hjälpt dig. Gå i frid, och var botad från din plåga."
\par 35 Medan han ännu talade, kommo några från synagogföreståndarens hus och sade: "Din dotter är död; du må icke vidare göra mästaren omak."
\par 36 Men när Jesus märkte vad som talades, sade han till synagogföreståndaren: "Frukta icke, tro allenast."
\par 37 Och han tillstadde ingen att följa med, utom Petrus och Jakob och Johannes, Jakobs broder.
\par 38 Så kommo de till synagogföreståndarens hus, och han fick där se en hop människor som höjde klagolåt och gräto och jämrade sig högt.
\par 39 Och han gick in och sade till dem: "Varför klagen I och gråten? Flickan är icke död, hon sover."
\par 40 Då hånlogo de åt honom. Men han visade ut dem allasammans; och han tog med sig allenast flickans fader och moder och dem som hade fått följa med honom, och gick in dit där flickan låg.
\par 41 Och han tog flickan vid handen och sade till henne: "Talita, kum" (det betyder: "Flicka, jag säger dig, stå upp").
\par 42 Och strax stod flickan upp och begynte gå omkring (hon var nämligen tolv år gammal); och de blevo strax uppfyllda av stor häpnad.
\par 43 Men han förbjöd dem strängeligen att låta någon få veta vad som hade skett. Därefter tillsade han att man skulle giva henne något att äta.

\chapter{6}

\par 1 Och han gick bort därifrån och begav sig till sin fädernestad; och hans lärjungar följde honom.
\par 2 Och när det blev sabbat, begynte han undervisa i synagogan. Och folket häpnade, när de hörde honom; de sade: "Varifrån har han fått detta? Och vad är det för vishet som har blivit honom given? Och dessa stora kraftgärningar som göras genom honom, varifrån komma de?
\par 3 Är då denne icke timmermannen, han som är Marias son och broder till Jakob och Joses och Judas och Simon? Och bo icke hans systrar här hos oss?" Så blev han för dem en stötesten.
\par 4 Då sade Jesus till dem: "En profet är icke föraktad utom i sin fädernestad och bland sina fränder och i sitt eget hus."
\par 5 Och han kunde icke där göra någon kraftgärning, utom att han botade några få sjuka, genom att lägga händerna på dem.
\par 6 Och han förundrade sig över deras otro. Sedan gick han omkring i byarna, från den ena byn till den andra, och undervisade.
\par 7 Och han kallade till sig de tolv och sände så ut dem, två och två, och gav dem makt över de orena andarna.
\par 8 Och han bjöd dem att icke taga något med sig på vägen, utom allenast en stav: icke bröd, icke ränsel, icke penningar i bältet.
\par 9 Sandaler finge de dock hava på fötterna, men de skulle icke bära dubbla livklädnader.
\par 10 Och han sade till dem: "När I haven kommit in i något hus, så stannen där, till dess I lämnen den orten.
\par 11 Och om man på något ställe icke tager emot eder och icke hör på eder, så gån bort därifrån, och skudden av stoftet som är under edra fötter, till ett vittnesbörd mot dem."
\par 12 Och de gingo ut och predikade att man skulle göra bättring;
\par 13 och de drevo ut många onda andar och smorde många sjuka med olja och botade dem.
\par 14 Och konung Herodes fick höra om honom, ty hans namn hade blivit känt. Man sade: "Det är Johannes döparen, som har uppstått från de döda, och därför verka dessa krafter i honom."
\par 15 Men andra sade: "Det är Elias." Andra åter sade: "Det är en profet, lik de andra profeterna."
\par 16 Men när Herodes hörde detta, sade han: "Det är Johannes, den som jag lät halshugga. Han bar uppstått från de döda."
\par 17 Herodes hade nämligen sänt åstad och låtit gripa Johannes och binda honom och sätta honom i fängelse, för Herodias', sin broder Filippus' hustrus, skull. Ty henne hade Herodes tagit till äkta,
\par 18 och Johannes hade då sagt till honom: "Det är icke lovligt för dig att hava din broders hustru."
\par 19 Därför hyste nu Herodes agg till honom och ville döda honom, men han hade icke makt därtill.
\par 20 Ty Herodes förstod att Johannes var en rättfärdig och helig man, och han fruktade för honom och gav honom sitt beskydd. Och när han hade hört honom, blev han betänksam i många stycken; och han hörde honom gärna.
\par 21 Men så kom en läglig dag, i det att Herodes på sin födelsedag gjorde ett gästabud för sina stormän och för krigsöverstarna och de förnämsta männen i Galileen.
\par 22 Då gick Herodias' dotter ditin och dansade; och hon behagade Herodes och hans bordsgäster. Och konungen sade till flickan: "Begär av mig vadhelst du vill, så skall jag giva dig det."
\par 23 Ja, han lovade henne detta med ed och sade: "Vadhelst du begär av mig, det skall jag giva dig, ända till hälften av mitt rike."
\par 24 Då gick hon ut och frågade sin moder: "Vad skall jag begära?" Hon svarade: "Johannes döparens huvud."
\par 25 Och strax skyndade hon in till konungen och framställde sin begäran och sade: "Jag vill att du nu genast giver mig på ett fat Johannes döparens huvud."
\par 26 Då blev konungen mycket bekymrad, men för edens och för bordsgästernas skull ville han icke avvisa henne.
\par 27 Alltså sände konungen strax en drabant med befallning att hämta hans huvud. Och denne gick åstad och halshögg honom i fängelset
\par 28 och bar sedan fram hans huvud på ett fat och gav det åt flickan, och flickan gav det åt sin moder.
\par 29 Men när hans lärjungar fingo höra härom, kommo de och togo hans döda kropp och lade den i en grav.
\par 30 Och apostlarna församlade sig hos Jesus och omtalade för honom allt vad de hade gjort, och allt vad de hade lärt folket.
\par 31 Då sade han till dem: "Kommen nu I med mig bort till en öde trakt, där vi få vara allena, och vilen eder något litet." Ty de fingo icke ens tid att äta; så många voro de som kommo och gingo.
\par 32 De foro alltså i båten bort till en öde trakt, där de kunde vara allena.
\par 33 Men man såg dem fara sin väg, och många fingo veta det; och från alla städer strömmade då människor tillsammans dit landvägen och kommo fram före dem.
\par 34 När han så steg i land, fick han se att där var mycket folk. Då ömkade han sig över dem, eftersom de voro "lika får som icke hade någon herde"; och han begynte undervisa dem i mångahanda stycken.
\par 35 Men när det redan var långt lidet på dagen, trädde hans lärjungar fram till honom och sade: "Trakten är öde, och det är redan långt lidet på dagen.
\par 36 Låt dem skiljas åt, så att de kunna gå bort i gårdarna och byarna häromkring och köpa sig något att äta."
\par 37 Men han svarade och sade till dem: "Given I dem att äta." De svarade honom: "Skola vi då gå bort och köpa bröd för två hundra silverpenningar och giva dem att äta?"
\par 38 Men han sade till dem: "Huru många bröd haven I? Gån och sen efter." Sedan de hade gjort så, svarade de: "Fem, och därtill två fiskar."
\par 39 Då befallde han dem att låta alla i skilda matlag lägga sig ned i gröna gräset.
\par 40 Och de lägrade sig där i skilda hopar, hundra eller femtio i var.
\par 41 Därefter tog han de fem bröden och de två fiskarna och såg upp till himmelen och välsignade dem. Och han bröt bröden och gav dem åt lärjungarna, för att de skulle lägga fram åt folket; också de två fiskarna delade han mellan dem alla.
\par 42 Och de åto alla och blevo mätta.
\par 43 Sedan samlade man upp överblivna brödstycken, tolv korgar fulla, därtill ock kvarlevor av fiskarna.
\par 44 Och det var fem tusen män som hade ätit.
\par 45 Strax därefter nödgade han sina lärjungar att stiga i båten och i förväg fara över till Betsaida på andra stranden, medan han själv tillsåg att folket skildes åt.
\par 46 Och när han hade tagit avsked av folket, gick han därifrån upp på berget för att bedja.
\par 47 När det så hade blivit afton, var båten mitt på sjön, och han var ensam kvar på land.
\par 48 Och han såg dem vara hårt ansatta, där de rodde fram, ty vinden låg emot dem. Vid fjärde nattväkten kom han då till dem, gående på sjön, och skulle just gå förbi dem.
\par 49 Men när de fingo se honom gå på sjön, trodde de att det var en vålnad och ropade högt;
\par 50 ty de sågo honom alla och blevo förfärade. Men han begynte strax tala med dem och sade till dem: "Varen vid gott mod; det är jag, varen icke förskräckta."
\par 51 Därefter steg han upp till dem i båten, och vinden lade sig. Och de blevo uppfyllda av stor häpnad;
\par 52 ty de hade icke kommit till förstånd genom det som hade skett med bröden, utan deras hjärtan voro förstockade.
\par 53 När de hade farit över till andra stranden, kommo de till Gennesarets land och lade till där.
\par 54 Och när de stego ur båten, kände man strax igen honom;
\par 55 och man skyndade omkring med bud i hela den trakten, och folket begynte då överallt bära de sjuka på sängar dit där man hörde att han var.
\par 56 Och varhelst han gick in i någon by eller någon stad eller någon gård, där lade man de sjuka på de öppna platserna. Och de bådo honom att åtminstone få röra vid hörntofsen på hans mantel; och alla som rörde vid den blevo hulpna.

\chapter{7}

\par 1 Och fariséerna, så ock några skriftlärde som hade kommit från Jerusalem, församlade sig omkring honom;
\par 2 och de fingo då se några av hans lärjungar äta med "orena", det är otvagna, händer.
\par 3 Nu är det så med fariséerna och alla andra judar, att de icke äta något utan att förut, till åtlydnad av de äldstes stadgar, noga hava tvagit sina händer,
\par 4 likasom de icke heller, när de komma från torget, äta något utan att förut hava tvagit sig; många andra stadgar finnas ock, som de av ålder pläga hålla, såsom att skölja bägare och träkannor och kopparskålar.
\par 5 Därför frågade honom nu fariséerna och de skriftlärde: "Varför vandra icke dina lärjungar efter de äldstes stadgar, utan äta med orena händer?"
\par 6 Men han svarade dem: "Rätt profeterade Esaias om eder, I skrymtare, såsom det är skrivet: 'Detta folk ärar mig med sina läppar, men deras hjärtan äro långt ifrån mig;
\par 7 och fåfängt dyrka de mig, eftersom de läror de förkunna äro människobud.'
\par 8 I sätten Guds bud å sido och hållen människors stadgar."
\par 9 Ytterligare sade han till dem: "Rätt så; I upphäven Guds bud för att hålla edra egna stadgar!
\par 10 Moses har ju sagt: 'Hedra din fader och din moder' och 'Den som smädar sin fader eller sin moder, han skall döden dö.'
\par 11 Men I sägen: om en son säger till sin fader eller sin moder: 'Vad du av mig kunde hava fått till hjälp, det giver jag i stället såsom korban' (det betyder offergåva),
\par 12 då kunnen I icke tillstädja honom att vidare göra något för sin fader eller sin moder.
\par 13 På detta sätt gören I Guds budord om intet genom edra fäderneärvda stadgar. Och mycket annat sådant gören I."
\par 14 Därefter kallade han åter folket till sig och sade till dem: "Hören mig alla och förstån.
\par 15 Intet som utifrån går in i människan kan orena henne, men vad som går ut ifrån människan, detta är det som orenar henne."
\par 16 Den som har öron till att höra, han höre.
\par 17 När han sedan hade lämnat folket och kommit inomhus, frågade hans lärjungar honom om detta bildliga tal.
\par 18 Han svarade dem: "Ären då också I så utan förstånd? Insen I icke att intet som utifrån går in i människan kan orena henne,
\par 19 eftersom det icke går in i hennes hjärta, utan ned i buken, och har sin naturliga utgång?" Härmed förklarade han all mat för ren.
\par 20 Och han tillade: "Vad som går ut ifrån människan, detta är det som orenar människan.
\par 21 Ty inifrån, från människornas hjärtan, utgå deras onda tankar, otukt, tjuveri, mord,
\par 22 äktenskapsbrott, girighet, ondska, svek, lösaktighet, avund, hädelse, övermod, oförsynt väsende.
\par 23 Allt detta onda går inifrån ut, och det orenar människan."
\par 24 Och han stod upp och begav sig bort därifrån till Tyrus' område. Där gick han in i ett hus och ville icke att någon skulle få veta det. Dock kunde han icke förbliva obemärkt,
\par 25 utan en kvinna, vilkens dotter var besatt av en oren ande, kom, strax då hon hade fått höra om honom, och föll ned för hans fötter;
\par 26 det var en grekisk kvinna av syrofenicisk härkomst. Och hon bad honom att han skulle driva ut den onde anden ur hennes dotter.
\par 27 Men han sade till henne: "Låt barnen först bliva mättade; det är ju otillbörligt att taga brödet från barnen och kasta det åt hundarna."
\par 28 Hon svarade och sade till honom: "Ja, Herre; också äta hundarna under bordet allenast av barnens smulor."
\par 29 Då sade han till henne: "För det ordets skull säger jag dig: Gå; den onde anden har farit ut ur din dotter."
\par 30 Och när hon kom hem, fann hon flickan ligga på sängen och såg att den onde anden hade farit ut.
\par 31 Sedan begav han sig åter bort ifrån Tyrus' område och tog vägen över Sidon och kom, genom Dekapolis' område, till Galileiska sjön.
\par 32 Och man förde till honom en som var döv och nästan stum och bad honom att lägga handen på denne.
\par 33 Då tog han honom avsides ifrån folket och satte sina fingrar i hans öron och spottade och rörde vid hans tunga
\par 34 och såg upp till himmelen, suckade och sade till honom: "Effata" (det betyder: "Upplåt dig").
\par 35 Då öppnades hans öron, och hans tungas band löstes, och han talade redigt och klart.
\par 36 Och Jesus förbjöd dem att omtala detta för någon; men ju mer han förbjöd dem, dess mer förkunnade de det.
\par 37 Och folket häpnade övermåttan och sade: "Allt har han väl beställt: de döva låter han höra och de stumma tala."

\chapter{8}

\par 1 Då vid samma tid åter mycket folk hade kommit tillstädes, och de icke hade något att äta, kallade han sina lärjungar till sig och sade till dem:
\par 2 "Jag ömkar mig över folket, ty det är redan tre dagar som de hava dröjt kvar hos mig, och de hava intet att äta.
\par 3 Om jag nu låter dem fastande gå ifrån mig hem, så uppgivas de på vägen; somliga av dem hava ju kommit långväga ifrån."
\par 4 Då svarade hans lärjungar honom: "Varifrån skall man här i en öken kunna få bröd till att mätta dessa med?"
\par 5 Han frågade dem: "Huru många bröd haven I?" De svarade: "Sju."
\par 6 Då tillsade han folket att lägra sig på marken. Ock han tog de sju bröden, tackade Gud och bröt dem och gav åt sina lärjungar, för att de skulle lägga fram dem; och de lade fram åt folket.
\par 7 De hade ock några få småfiskar; och när han hade välsignat dessa, bjöd han att man likaledes skulle lägga fram dem.
\par 8 Så åto de och blevo mätta. Och man samlade sedan upp sju korgar med överblivna stycken.
\par 9 Men antalet av dem som voro tillstädes var vid pass fyra tusen. Sedan lät han dem skiljas åt.
\par 10 Och strax därefter steg han i båten med sina lärjungar och for till trakten av Dalmanuta.
\par 11 Och fariséerna kommo ditut och begynte disputera med honom; de ville sätta honom på prov och begärde av honom något tecken från himmelen.
\par 12 Då suckade han ur sin andes djup och sade: "Varför begär detta släkte ett tecken? Sannerligen säger jag eder: Åt detta släkte skall intet tecken givas."
\par 13 Så lämnade han dem och steg åter i båten och for över till andra stranden.
\par 14 Och de hade förgätit att taga med sig bröd; icke mer än ett enda bröd hade de med sig i båten.
\par 15 Och han bjöd dem och sade: "Sen till, att I tagen eder till vara för fariséernas surdeg och för Herodes' surdeg."
\par 16 Då talade de med varandra om att de icke hade bröd med sig.
\par 17 Men när han märkte detta, sade han till dem: "Varför talen I om att I icke haven bröd med eder? Fatten och förstån I då ännu ingenting? Äro edra hjärtan så förstockade?
\par 18 I haven ju ögon; sen I då icke? I haven ju öron; hören I då icke?
\par 19 Och kommen I icke ihåg huru många korgar fulla av stycken I samladen upp, när jag bröt de fem bröden åt de fem tusen?" De svarade honom: "Tolv."
\par 20 "Och när jag bröt de sju bröden åt de fyra tusen, huru många korgar fulla av stycken samladen I då upp?" De svarade: "Sju."
\par 21 Då sade han till dem: "Förstån I då ännu ingenting?"
\par 22 Därefter kommo de till Betsaida. Och man förde till honom en som var blind och bad honom att han skulle röra vid denne.
\par 23 Då tog han den blinde vid handen och ledde honom utanför byn; sedan spottade han på hans ögon och lade händerna på honom och frågade honom: "Ser du något?"
\par 24 Han såg då upp och svarade: "Jag kan urskilja människorna; jag ser dem gå omkring, men de likna träd."
\par 25 Därefter lade han åter händerna på hans ögon, och nu såg han tydligt och var botad och kunde jämväl på långt håll se allting klart.
\par 26 Och Jesus bjöd honom gå hem och sade: "Gå icke ens in i byn."
\par 27 Och Jesus gick med sina lärjungar bort till byarna vid Cesarea Filippi. På vägen dit frågade han sina lärjungar och sade till dem: "Vem säger folket mig vara?"
\par 28 De svarade och sade: "Johannes döparen; andra säga dock Elias, andra åter säga: 'Det är en av profeterna.'"
\par 29 Då frågade han dem: "Vem sägen då I mig vara?" Petrus svarade och sade till honom: "Du är Messias."
\par 30 Då förbjöd han dem strängeligen att för någon säga detta om honom.
\par 31 Sedan begynte han undervisa dem om att Människosonen måste lida mycket, och att han skulle bliva förkastad av de äldste och översteprästerna och de skriftlärde, och att han skulle bliva dödad, men att han tre dagar därefter skulle uppstå igen.
\par 32 Och han talade detta i oförtäckta ordalag. Då tog Petrus honom avsides och begynte ivrigt motsäga honom.
\par 33 Men han vände sig om, och när han då såg sina lärjungar, talade han strängt till Petrus och sade: "Gå bort, Satan, och stå mig icke i vägen; ty dina tankar äro icke Guds tankar, utan människotankar."
\par 34 Och han kallade till sig folket jämte sina lärjungar och sade till dem: "Om någon vill efterfölja mig, så försake han sig själv och tage sitt kors på sig; så följe han mig.
\par 35 Ty den som vill bevara sitt liv, han skall mista det; men den som mister sitt liv, för min och för evangelii skull, han skall bevara det.
\par 36 Och vad hjälper det en människa, om hon vinner hela världen, men förlorar sin själ?
\par 37 Och vad kan en människa giva till lösen för sin själ?
\par 38 Den som blyges för mig och för mina ord, i detta trolösa och syndiga släkte, för honom skall ock Människosonen blygas, när han kommer i sin Faders härlighet med de heliga änglarna."

\chapter{9}

\par 1 Ytterligare sade han till dem: "Sannerligen säger jag eder: Bland dem som här stå finnas några som icke skola smaka döden, förrän de få se Guds rike vara kommet i sin kraft."
\par 2 Sex dagar därefter tog Jesus med sig Petrus och Jakob och Johannes och förde dem ensamma upp på ett högt berg, där de voro allena. Och hans utseende förvandlades inför dem;
\par 3 och hans kläder blevo glänsande och mycket vita, så att ingen valkare på jorden kan göra kläder så vita.
\par 4 Och för dem visade sig Elias jämte Moses, och dessa samtalade med Jesus.
\par 5 Då tog Petrus till orda och sade till Jesus: "Rabbi, här är oss gott att vara; låt oss göra tre hyddor, åt dig en och åt Moses en och åt Elias en."
\par 6 Han visste nämligen icke vad han skulle säga; så stor var deras förskräckelse.
\par 7 Då kom en sky som överskyggde dem, och ur skyn kom en röst: "Denne är min älskade Son; hören honom."
\par 8 Och plötsligt märkte de, när de sågo sig omkring, att där icke mer fanns någon hos dem utom Jesus allena.
\par 9 Då de sedan gingo ned från berget, bjöd han dem att de icke, förrän Människosonen hade uppstått från de döda, skulle för någon omtala vad de hade sett.
\par 10 Och de lade märke till det ordet och begynte tala med varandra om vad som kunde menas med att han skulle uppstå från de döda.
\par 11 Och de frågade honom och sade: "De skriftlärde säga ju att Elias först måste komma?"
\par 12 Han svarade dem: "Elias måste visserligen först komma och upprätta allt igen. Men huru kan det då vara skrivet om Människosonen att han skall lida mycket och bliva föraktad?
\par 13 Dock, jag säger eder att Elias redan har kommit; och de förforo mot honom alldeles såsom de ville, och såsom det var skrivet att det skulle gå honom."
\par 14 När de därefter kommo till lärjungarna, sågo de att mycket folk var samlat omkring dem, och att några skriftlärde disputerade med dem.
\par 15 Och strax då allt folket fick se honom, blevo de mycket häpna och skyndade fram och hälsade honom.
\par 16 Då frågade han dem: "Varom disputeren I med dem?"
\par 17 Och en man i folkhopen svarade honom: "Mästare, jag har fört till dig min son, som är besatt av en stum ande.
\par 18 Och varhelst denne får fatt i honom kastar han omkull honom, och fradgan står gossen om munnen, och han gnisslar med tänderna och bliver såsom livlös. Nu bad jag dina lärjungar att de skulle driva ut honom, men de förmådde det icke."
\par 19 Då svarade han dem och sade: "O du otrogna släkte, huru länge måste jag vara hos eder? Huru länge måste jag härda ut med eder? Fören honom till mig."
\par 20 Och de förde honom till Jesus. Och strax då han fick se Jesus, slet och ryckte anden honom, och han föll ned på jorden och vältrade sig, under det att fradgan stod honom om munnen.
\par 21 Jesus frågade då hans fader: "Huru länge har det varit så med honom?" Han svarade: "Alltsedan han var ett litet barn;
\par 22 och det har ofta hänt att han har kastat honom än i elden, än i vattnet, för att förgöra honom. Men om du förmår något, så förbarma dig över oss och hjälp oss."
\par 23 Då sade Jesus till honom: "Om jag förmår, säger du. Allt förmår den som tror."
\par 24 Strax ropade gossens fader och sade: "Jag tror! Hjälp min otro."
\par 25 Men när Jesus såg att folk strömmade tillsammans dit, tilltalade han den orene anden strängt och sade till honom: "Du stumme och döve ande, jag befaller dig: Far ut ur honom, och kom icke mer in i honom."
\par 26 Då skriade han och slet och ryckte gossen svårt och for ut; och gossen blev såsom död, så att folket menade att han verkligen var död.
\par 27 Men Jesus tog honom vid handen och reste upp honom; och han stod då upp.
\par 28 När Jesus därefter hade kommit inomhus, frågade hans lärjungar honom, då de nu voro allena: "Varför kunde icke vi driva ut honom?"
\par 29 Han svarade dem: "Detta slag kan icke drivas ut genom något annat än bön och fasta."
\par 30 Och de gingo därifrån och vandrade genom Galileen; men han ville icke att någon skulle få veta det.
\par 31 Han undervisade nämligen sina lärjungar och sade till dem: "Människosonen skall bliva överlämnad i människors händer, och man skall döda honom; men tre dagar efter det att han har blivit dödad skall han uppstå igen."
\par 32 Och de förstodo icke vad han sade, men de fruktade att fråga honom.
\par 33 Och de kommo till Kapernaum. Och när han hade kommit dit där han bodde, frågade han dem: "Vad var det I samtaladen om på vägen?"
\par 34 Men de tego, ty de hade på vägen talat med varandra om vilken som vore störst.
\par 35 Då satte han sig ned och kallade till sig de tolv och sade till dem: "Om någon vill vara den förste, så vare han den siste av alla och allas tjänare."
\par 36 Och han tog ett barn och ställde det mitt ibland dem; sedan tog han det upp i famnen och sade till dem:
\par 37 "Den som tager emot ett sådant barn i mitt namn, han tager emot mig, och den som tager emot mig, han tager icke emot mig, utan honom som har sänt mig."
\par 38 Johannes sade till honom: "Mästare, vi sågo huru en man som icke följer oss drev ut onda andar genom ditt namn; och vi ville hindra honom, eftersom han icke följde oss."
\par 39 Men Jesus sade: "Hindren honom icke; ty ingen som genom mitt namn har gjort en kraftgärning kan strax därefter tala illa om mig.
\par 40 Ty den som icke är emot oss, han är för oss.
\par 41 Ja, den som giver eder en bägare vatten att dricka, därför att I hören Kristus till - sannerligen säger jag eder: Han skall ingalunda gå miste om sin lön.
\par 42 Och den som förför en av dessa små som tro, för honom vore det bättre, om en kvarnsten hängdes om hans hals och han kastades i havet.
\par 43 Om nu din hand är dig till förförelse, så hugg av den. Det är bättre för dig att ingå i livet lytt, än att hava båda händerna i behåll och komma till Gehenna, till den eld som icke utsläckes.
\par 44 Där 'deras mask icke dör och elden icke utsläckes'.
\par 45 Och om din fot är dig till förförelse, så hugg av den. Det är bättre för dig att ingå i livet halt, än att hava båda fötterna i behåll och kastas i Gehenna.
\par 46 Där 'deras mask icke dör och elden icke utsläckes'.
\par 47 Och om ditt öga är dig till förförelse, så riv ut det. Det är bättre för dig att ingå i Guds rike enögd, än att hava båda ögonen i behåll och kastas i Gehenna,
\par 48 där 'deras mask icke dör och elden icke utsläckes'.
\par 49 Ty var människa måste saltas med eld. Ty varje offer skall med salt saltas.
\par 50 Saltet är en god sak; men om saltet mister sin sälta, varmed skolen I då återställa dess kraft? - Haven salt i eder och hållen frid inbördes."

\chapter{10}

\par 1 Och han stod upp och begav sig därifrån, genom landet på andra sidan Jordan, till Judeens område. Och mycket folk församlades åter omkring honom, och åter undervisade han dem, såsom hans sed var.
\par 2 Då ville några fariséer snärja honom, och de trädde fram och frågade honom om det vore lovligt för en man att skilja sig från sin hustru.
\par 3 Men han svarade och sade till dem: "Vad har Moses bjudit eder?"
\par 4 De sade: "Moses tillstadde att en man fick skriva skiljebrev åt sin hustru och så skilja sig från henne."
\par 5 Då sade Jesus till dem: "För edra hjärtans hårdhets skull skrev han åt eder detta bud.
\par 6 Men redan vid världens begynnelse 'gjorde Gud dem till man och kvinna'.
\par 7 'Fördenskull skall en man övergiva sin fader och sin moder.
\par 8 Och de tu skola varda ett kött.' Så äro de icke mer två, utan ett kött.
\par 9 Vad nu Gud har sammanfogat, det må människan icke åtskilja."
\par 10 När de sedan hade kommit hem, frågade hans lärjungar honom åter om detsamma.
\par 11 Och han svarade dem: "Den som skiljer sig från sin hustru och tager sig en annan hustru, han begår äktenskapsbrott mot henne.
\par 12 Och om en hustru skiljer sig från sin man och tager sig en annan man, då begår hon äktenskapsbrott.
\par 13 Och man bar fram barn till honom, för att han skulle röra vid dem; men lärjungarna visade bort dem.
\par 14 När Jesus såg detta, blev han misslynt och sade till dem: "Låten barnen komma till mig, och förmenen dem det icke; ty Guds rike hör sådana till.
\par 15 Sannerligen säger jag eder: Den som icke tager emot Guds rike såsom ett barn, han kommer aldrig ditin."
\par 16 Och han tog dem upp i famnen och lade händerna på dem och välsignade dem.
\par 17 När han sedan begav sig åstad för att fortsätta sin väg, skyndade en man fram och föll på knä för honom och frågade honom: "Gode Mästare, vad skall jag göra för att få evigt liv till arvedel?"
\par 18 Jesus sade till honom: "Varför kallar du mig god? Ingen är god utom Gud allena.
\par 19 Buden känner du: 'Du skall icke dräpa', 'Du skall icke begå äktenskapsbrott', 'Du skall icke stjäla', 'Du skall icke bära falskt vittnesbörd', 'Du skall icke undanhålla någon vad honom tillkommer', Hedra din fader och din moder.'"
\par 20 Då svarade han honom: "Mästare, allt detta har jag hållit från min ungdom."
\par 21 Då såg Jesus på honom och fick kärlek till honom och sade till honom: "Ett fattas dig: gå bort och sälj allt vad du äger och giv åt de fattiga; då skall du få en skatt i himmelen. Och kom sedan och följ mig."
\par 22 Men han blev illa till mods vid det talet och gick bedrövad bort, ty han hade många ägodelar.
\par 23 Då såg Jesus sig omkring och sade till sina lärjungar: "Huru svårt är det icke för dem som hava penningar att komma in i Guds rike!"
\par 24 Men lärjungarna häpnade vid hans ord. Då tog Jesus åter till orda och sade till dem: "Ja, mina barn, huru svårt är det icke att komma in i Guds rike!
\par 25 Det är lättare för en kamel att komma igenom ett nålsöga, än för den som är rik att komma in i Guds rike."
\par 26 Då blevo de ännu mer häpna och sade till varandra: "Vem kan då bliva frälst?"
\par 27 Jesus såg på dem och sade: "För människor är det omöjligt, men icke för Gud, ty för Gud är allting möjligt."
\par 28 Då tog Petrus till orda och sade till honom: "Se, vi hava övergivit allt och följt dig."
\par 29 Jesus svarade: "Sannerligen säger jag eder: Ingen som för min och evangelii skull har övergivit hus, eller bröder eller systrar, eller moder eller fader, eller barn, eller jordagods,
\par 30 ingen sådan finnes, som icke skall få hundrafalt igen: redan här i tiden hus, och bröder och systrar, och mödrar och barn, och jordagods, mitt under förföljelser, och i den tillkommande tidsåldern evigt liv.
\par 31 Men många som äro de första skola bliva de sista, medan de sista bliva de första."
\par 32 Och de voro på vägen upp till Jerusalem. Och Jesus gick före dem, och de gingo där bävande; och de som följde med dem voro uppfyllda av fruktan. Då tog han åter till sig de tolv och begynte tala till dem om vad som skulle övergå honom:
\par 33 "Se, vi gå nu upp till Jerusalem, och Människosonen skall bliva överlämnad åt översteprästerna och de skriftlärde, och de skola döma honom till döden och överlämna honom åt hedningarna,
\par 34 och dessa skola begabba honom och bespotta honom och gissla honom och döda honom; men tre dagar därefter skall han uppstå igen."
\par 35 Då trädde Jakob och Johannes, Sebedeus' söner, fram till honom och sade till honom: "Mästare, vi skulle vilja att du läte oss få vad vi nu tänka begära av dig."
\par 36 Han frågade dem: "Vad viljen I då att jag skall låta eder få?"
\par 37 De svarade honom: "Låt den ene av oss i din härlighet få sitta på din högra sida, och den andre på din vänstra."
\par 38 Men Jesus sade till dem: "I veten icke vad I begären. Kunnen I dricka den kalk som jag dricker, eller genomgå det dop som jag genomgår?"
\par 39 De svarade honom: "Det kunna vi." Då sade Jesus till dem: "Ja, den kalk jag dricker skolen I få dricka, och det dop jag genomgår skolen I genomgå,
\par 40 men platsen på min högra sida och platsen på min vänstra tillkommer det icke mig att bortgiva, utan de skola tillfalla dem för vilka så är bestämt."
\par 41 När de tio andra hörde detta, blevo de misslynta på Jakob och Johannes.
\par 42 Då kallade Jesus dem till sig och sade till dem: "I veten att de som räknas för folkens furstar uppträda mot dem såsom herrar, och att deras mäktige låta dem känna sin myndighet.
\par 43 Men så är det icke bland eder; utan den som vill bliva störst bland eder, han vare de andras tjänare,
\par 44 och den som vill vara främst bland eder, han vare allas dräng.
\par 45 Också Människosonen har ju kommit, icke för att låta tjäna sig, utan för att tjäna och giva sitt liv till lösen för många."
\par 46 Och de kommo till Jeriko. Men när han åter gick ut ifrån Jeriko, följd av sina lärjungar och en ganska stor hop folk, satt där vid vägen en blind tiggare, Bartimeus, Timeus' son.
\par 47 När denne hörde att det var Jesus från Nasaret, begynte han ropa och säga: "Jesus, Davids son, förbarma dig över mig."
\par 48 Och många tillsade honom strängeligen att han skulle tiga; men han ropade ännu mycket mer: "Davids son, förbarma dig över mig."
\par 49 Då stannade Jesus och sade: "Kallen honom hit." Och de kallade på den blinde och sade till honom: "Var vid gott mod, stå upp; han kallar dig till sig."
\par 50 Då kastade han av sig sin mantel och stod upp med hast och kom fram till Jesus.
\par 51 Och Jesus talade till honom och sade: "Vad vill du att jag skall göra dig?" Den blinde svarade honom: "Rabbuni, låt mig få min syn."
\par 52 Jesus sade till honom: "Gå; din tro har hjälpt dig." Och strax fick han sin syn och följde honom på vägen.

\chapter{11}

\par 1 När de nu nalkades Jerusalem och voro nära Betfage och Betania vid Oljeberget, sände han åstad två av sina lärjungar
\par 2 och sade till dem: "Gån in i byn som ligger mitt framför eder, så skolen I, strax då I kommen ditin, finna en åsnefåle stå där bunden, som ännu ingen människa har suttit på; lösen den och fören den hit.
\par 3 Och om någon frågar eder varför I gören detta, så skolen I svara: 'Herren behöver den, men han skall strax sända den tillbaka hit."
\par 4 Då gingo de åstad och funno en åsnefåle stå där bunden utanför en port vid vägen, och de löste den.
\par 5 Och några som stodo där bredvid sade till dem: "Vad gören I? Varför lösen I fålen?"
\par 6 Men de svarade dem såsom Jesus hade bjudit. Då lät man dem vara.
\par 7 Och de förde fålen till Jesus och lade sina mantlar på den, och han satte sig upp på den.
\par 8 Och många bredde ut sina mantlar på vägen, andra åter skuro av kvistar och löv på fälten och strödde på vägen.
\par 9 Och de som gingo före och de som följde efter ropade: "Hosianna! Välsignad vare han som kommer, i Herrens namn.
\par 10 Välsignat vare vår fader Davids rike, som nu kommer. Hosianna i höjden!"
\par 11 Så drog han in i Jerusalem och kom in i helgedomen; och när han hade sett sig omkring överallt och det redan var sent på dagen, gick han med de tolv ut till Betania.
\par 12 När de dagen därefter voro på väg tillbaka från Betania, blev han hungrig.
\par 13 Och då han på avstånd fick se ett fikonträd som hade löv, gick han dit för att se om han till äventyrs skulle finna något därpå; men när han kom fram till det, fann han intet annat än löv, det var icke då fikonens tid.
\par 14 Då talade han och sade till trädet: "Aldrig någonsin mer äte någon frukt av dig." Och hans lärjungar hörde detta.
\par 15 När de sedan kommo fram till Jerusalem, gick han in i helgedomen och begynte driva ut dem som sålde och köpte i helgedomen. Och han stötte omkull växlarnas bord och duvomånglarnas säten;
\par 16 han tillstadde icke heller att man bar någonting genom helgedomen.
\par 17 Och han undervisade dem och sade: "Det är ju skrivet: 'Mitt hus skall kallas ett bönehus för alla folk.' Men I haven gjort det till en rövarkula."
\par 18 Då översteprästerna och de skriftlärde fingo höra härom, sökte de efter tillfälle att förgöra honom; ty de fruktade för honom, eftersom allt folket häpnade över hans undervisning.
\par 19 När det blev afton, begåvo de sig ut ur staden.
\par 20 Men då de nu på morgonen åter gingo där fram, fingo de se fikonträdet vara förtorkat ända från roten.
\par 21 Då kom Petrus ihåg vad som hade skett och sade till honom: "Rabbi, se, fikonträdet som du förbannade är förtorkat."
\par 22 Jesus svarade och sade till dem: "Haven tro på Gud.
\par 23 Sannerligen säger jag eder: Om någon säger till detta berg: 'Häv dig upp, och kasta dig i havet' och därvid icke tvivlar i sitt hjärta, utan tror att det han säger skall ske, då skall det ske honom så.
\par 24 Därför säger jag eder: Allt vad I bedjen om och begären, tron att det är eder givet; och det skall ske eder så.
\par 25 Och när I stån och bedjen, så förlåten, om I haven något emot någon, för att också eder Fader, som är i himmelen, må förlåta eder edra försyndelser."
\par 26 Men om I icke förlåten, så skall ej heller eder Fader, som är i himmelen, förlåta edra försyndelser.
\par 27 Så kommo de åter till Jerusalem. Och medan han gick omkring i helgedomen, kommo översteprästerna och de skriftlärde och de äldste fram till honom;
\par 28 och de sade till honom: "Med vad myndighet gör du detta? Och vem har givit dig myndighet att göra detta?"
\par 29 Jesus svarade dem: "Jag vill ställa en fråga till eder; svaren mig på den, så skall ock jag säga eder med vad myndighet jag gör detta.
\par 30 Johannes' döpelse, var den från himmelen eller från människor? Svaren mig härpå."
\par 31 Då överlade de med varandra och sade: "Om vi svara: 'Från himmelen', så frågar han: 'Varför trodden I honom då icke?'
\par 32 Eller skola vi svara: 'Från människor'?" - det vågade de icke av fruktan för folket, ty alla höllo före att Johannes verkligen var en profet.
\par 33 De svarade alltså Jesus och sade: "Vi veta det icke." Då sade Jesus till dem: "Så säger icke heller jag eder med vad myndighet jag gör detta."

\chapter{12}

\par 1 Och han begynte tala till dem i liknelser: "En man planterade en vingård och satte stängsel däromkring och högg ut ett presskar och byggde ett vakttorn; därefter lejde han ut den åt vingårdsmän och for utrikes.
\par 2 När sedan rätta tiden var inne, sände han en tjänare till vingårdsmännen, för att denne av vingårdsmännen skulle uppbära någon del av vingårdens frukt.
\par 3 Men de togo fatt på honom och misshandlade honom och läto honom gå tomhänt tillbaka.
\par 4 Åter sände han till dem en annan tjänare. Honom slogo de i huvudet och skymfade.
\par 5 Sedan sände han åstad ännu en annan, men denne dräpte de. Likaså gjorde de med många andra: somliga misshandlade de, och andra dräpte de.
\par 6 Nu hade han ock en enda son, vilken han älskade. Honom sände han slutligen åstad till dem, ty han tänkte: 'De skola väl hava försyn för min son.'
\par 7 Men vingårdsmännen sade till varandra: 'Denne är arvingen; kom, låt oss dräpa honom, så bliver arvet vårt.'
\par 8 Och de togo fatt på honom och dräpte honom och kastade honom ut ur vingården. -
\par 9 Vad skall nu vingårdens herre göra? Jo, han skall komma och förgöra vingårdsmännen och lämna vingården åt andra.
\par 10 Haven I icke läst detta skriftens ord: 'Den sten som byggningsmännen förkastade, den har blivit en hörnsten;
\par 11 av Herren har den blivit detta, och underbar är den i våra ögon'?"
\par 12 De hade nu gärna velat gripa honom, men de fruktade för folket; ty de förstodo att det var om dem som han hade talat i denna liknelse. Så läto de honom vara och gingo sin väg.
\par 13 Därefter sände de till honom några fariséer och herodianer, för att dessa skulle fånga honom genom något hans ord.
\par 14 Dessa kommo nu och sade till honom: "Mästare, vi veta att du är sannfärdig och icke frågar efter någon, ty du ser icke till personen, utan lär om Guds väg vad sant är. Är det lovligt att giva kejsaren skatt, eller är det icke lovligt? Skola vi giva skatt, eller icke giva?"
\par 15 Men han förstod deras skrymteri och sade till dem: "Varför söken I att snärja mig? Tagen hit en penning, så att jag får se den."
\par 16 Då lämnade de fram en sådan. Därefter frågade han dem: "Vems bild och överskrift är detta?" De svarade honom: "Kejsarens."
\par 17 Då sade Jesus till dem: "Så given kejsaren vad kejsaren tillhör, och Gud vad Gud tillhör." Och de förundrade sig högeligen över honom.
\par 18 Sedan kommo till honom några av sadducéerna, vilka mena att det icke gives någon uppståndelse. Dessa frågade honom och sade:
\par 19 "Mästare, Moses har givit oss den föreskriften, att om någon har en broder som dör, och som efterlämnar hustru, men icke lämnar barn efter sig, så skall han taga sin broders hustru till äkta och skaffa avkomma åt sin broder.
\par 20 Nu voro här sju bröder. Den förste tog sig en hustru, men dog utan att lämna någon avkomma efter sig.
\par 21 Då tog den andre i ordningen henne, men också han dog utan att lämna någon avkomma efter sig; sammalunda den tredje.
\par 22 Så skedde med alla sju: ingen av dem lämnade någon avkomma efter sig. Sist av alla dog ock hustrun.
\par 23 Vilken av dem skall nu vid uppståndelsen, när de uppstå, få henne till hustru? De hade ju alla sju tagit henne till hustru."
\par 24 Jesus svarade dem: "Visar icke eder fråga att I faren vilse och varken förstån skrifterna, ej heller Guds kraft?
\par 25 Efter uppståndelsen från de döda taga män sig icke hustrur, ej heller givas hustrur åt män, utan de äro då såsom änglarna i himmelen.
\par 26 Men vad nu det angår, att de döda uppstå, haven I icke läst i Moses' bok, på det ställe där det talas om törnbusken, huru Gud sade till honom så: 'Jag är Abrahams Gud och Isaks Gud och Jakobs Gud'?
\par 27 Han är en Gud icke för döda, utan för levande. I faren mycket vilse."
\par 28 Då trädde en av de skriftlärde fram, en som hade hört deras ordskifte och förstått att han hade svarat dem väl. Denne frågade honom: "Vilket är det förnämsta av alla buden?"
\par 29 Jesus svarade: "Det förnämsta är detta: 'Hör, Israel! Herren, vår Gud, Herren är en.
\par 30 Och du skall älska Herren, din Gud, av allt ditt hjärta och av all din själ och av allt ditt förstånd och av all din kraft.'
\par 31 Därnäst kommer detta: 'Du skall älska din nästa såsom dig själv.' Intet annat bud är större än dessa."
\par 32 Då svarade den skriftlärde honom: "Mästare, du har i sanning rätt i vad du säger, att han är en, och att ingen annan är än han.
\par 33 Och att älska honom av allt sitt hjärta och av allt sitt förstånd och av all sin kraft och att älska sin nästa såsom sig själv, det är 'förmer än alla brännoffer och slaktoffer'."
\par 34 Då nu Jesus märkte att han hade svarat förståndigt, sade han till honom: "Du är icke långt ifrån Guds rike." Sedan dristade sig ingen att vidare ställa någon fråga på honom.
\par 35 Medan Jesus undervisade i helgedomen, framställde han denna fråga: "Huru kunna de skriftlärde säga att Messias är Davids son?
\par 36 David själv har ju sagt genom den helige Andes ingivelse: 'Herren sade till min herre: Sätt dig på min högra sida, till dess jag har lagt dina fiender dig till en fotapall.'
\par 37 Så kallar nu David själv honom 'herre'; huru kan han då vara hans son?" Och folkskarorna hörde honom gärna.
\par 38 Och han undervisade dem och sade till dem: "Tagen eder till vara för de skriftlärde, som gärna gå omkring i fotsida kläder och gärna vilja bliva hälsade på torgen
\par 39 och gärna sitta främst i synagogorna och på de främsta platserna vid gästabuden -
\par 40 detta under det att de utsuga änkors hus, medan de för syns skull hålla långa böner. De skola få en dess hårdare dom."
\par 41 Och han satte sig mitt emot offerkistorna och såg huru folket lade ned penningar i offerkistorna. Och många rika lade dit mycket.
\par 42 Men en fattig änka kom och lade ned två skärvar, det är ett öre.
\par 43 Då kallade han sina lärjungar till sig och sade till dem: "Sannerligen säger jag eder: Denna fattiga änka lade dit mer än alla de andra som lade något i offerkistorna.
\par 44 Ty dessa lade alla dit av sitt överflöd, men hon lade dit av sitt armod allt vad hon hade, så mycket som fanns i hennes ägo."

\chapter{13}

\par 1 Då han nu gick ut ur helgedomen, sade en av hans lärjungar till honom: "Mästare, se hurudana stenar och hurudana byggnader!"
\par 2 Jesus svarade honom: "Ja, du ser nu dessa stora byggnader; men här skall förvisso icke lämnas sten på sten; allt skall bliva nedbrutet."
\par 3 När han sedan satt på Oljeberget, mitt emot helgedomen, frågade honom Petrus och Jakob och Johannes och Andreas, då de voro allena:
\par 4 "Säg oss när detta skall ske, och vad som bliver tecknet till att tiden är inne, då allt detta skall gå i fullbordan."
\par 5 Då begynte Jesus tala till dem och sade: "Sen till, att ingen förvillar eder.
\par 6 Många skola komma under mitt namn och säga: 'Det är jag' och skola förvilla många.
\par 7 Men när I fån höra krigslarm och rykten om krig, så förloren icke besinningen; sådant måste komma, men därmed är ännu icke änden inne.
\par 8 Ja, folk skall resa sig upp mot folk och rike mot rike, och det skall bliva jordbävningar på den ena orten efter den andra, och hungersnöd skall uppstå; detta är begynnelsen till 'födslovåndorna'.
\par 9 Men tagen I eder till vara. Man skall då draga eder inför domstolar, och I skolen bliva gisslade i synagogor och ställas fram inför landshövdingar och konungar, för min skull, till ett vittnesbörd för dem.
\par 10 Men evangelium måste först bliva predikat för alla folk.
\par 11 När man nu för eder åstad och drager eder inför rätta, så gören eder icke förut bekymmer om vad I skolen tala; utan vad som bliver eder givet i den stunden, det mån I tala. Ty det är icke I som skolen tala, utan den helige Ande.
\par 12 Och den ene brodern skall då överlämna den andre till att dödas, ja ock fadern sitt barn; och barn skola sätta sig upp mot sina föräldrar och skola döda dem.
\par 13 Och I skolen bliva hatade av alla, för mitt namns skull. Men den som är ståndaktig intill änden, han skall bliva frälst.
\par 14 Men när I fån se 'förödelsens styggelse' stå där han icke borde stå - den som läser detta, han give akt därpå - då må de som äro i Judeen fly bort till bergen,
\par 15 och den som är på taket må icke stiga ned och gå in för att hämta något ur sitt hus,
\par 16 och den som är ute på marken må icke vända tillbaka för att hämta sin mantel.
\par 17 Och ve de som äro havande, eller som giva di på den tiden!
\par 18 Men bedjen att det icke må ske om vintern.
\par 19 Ty den tiden skall bliva 'en tid av vedermöda, så svår att dess like icke har förekommit allt ifrån världens begynnelse, från den tid då Gud skapade världen, intill nu', ej heller någonsin skall förekomma.
\par 20 Och om Herren icke förkortade den tiden, så skulle intet kött bliva frälst; men för de utvaldas skull, för de människors skull, som han har utvalt, har han förkortat den tiden.
\par 21 Och om någon då säger till eder: 'Se här är Messias', eller: 'Se där är han', så tron det icke.
\par 22 Ty människor som falskeligen säga sig vara Messias skola uppstå, så ock falska profeter, och de skola göra tecken och under, för att, om möjligt, förvilla de utvalda.
\par 23 Men tagen I eder till vara. Jag har nu sagt eder allt förut.
\par 24 Men på den tiden, efter den vedermödan, skall solen förmörkas och månen upphöra att giva sitt sken,
\par 25 och stjärnorna skola falla ifrån himmelen, och makterna i himmelen skola bäva.
\par 26 Och då skall man få se 'Människosonen komma i skyarna' med stor makt och härlighet.
\par 27 Och han skall då sända ut sina änglar och församla sina utvalda från de fyra väderstrecken, från jordens ända till himmelens ända.
\par 28 Ifrån fikonträdet mån I här hämta en liknelse. När dess kvistar begynna att få save och löven spricka ut, då veten I att sommaren är nära.
\par 29 Likaså, när I sen detta ske, då kunnen I ock veta att han är nära och står för dörren.
\par 30 Sannerligen säger jag eder: Detta släkte skall icke förgås, förrän allt detta sker.
\par 31 Himmel och jord skola förgås, men mina ord skola icke förgås.
\par 32 Men om den dagen och den stunden vet ingen något, icke änglarna i himmelen, icke ens Sonen - ingen utom Fadern.
\par 33 Tagen eder till vara, vaken; ty I veten icke när tiden är inne.
\par 34 Såsom när en man reser utrikes och lämnar sitt hus och giver sina tjänare makt och myndighet däröver, åt var och en hans särskilda syssla, och därvid ock bjuder portvaktaren att vaka.
\par 35 likaså bjuder jag eder: Vaken; ty I veten icke när husets herre kommer, om han kommer på aftonen eller vid midnattstiden eller i hanegället eller på morgonen;
\par 36 vaken, så att han icke finner eder sovande, när han oförtänkt kommer.
\par 37 Men vad jag säger till eder, det säger jag till alla: Vaken!"

\chapter{14}

\par 1 Två dagar därefter var det påsk och det osyrade brödets högtid. Och översteprästerna och de skriftlärde sökte efter tillfälle att gripa honom med list och döda honom.
\par 2 De sade nämligen: "Icke under högtiden, för att ej oroligheter skola uppstå bland folket."
\par 3 Men när han var i Betania, i Simon den spetälskes hus, och där låg till bords, kom en kvinna som hade med sig en alabasterflaska med smörjelse av dyrbar äkta nardus. Och hon bröt sönder flaskan och göt ut smörjelsen över hans huvud.
\par 4 Några som voro där blevo då misslynta och sade till varandra: "Varför skulle denna smörjelse förspillas?
\par 5 Man hade ju kunnat sälja den för mer än tre hundra silverpenningar och giva dessa åt de fattiga." Och de talade hårda ord till henne.
\par 6 Men Jesus sade: "Låten henne vara. Varför oroen I henne? Det är en god gärning som hon har gjort mot mig.
\par 7 De fattiga haven I ju alltid ibland eder, och närhelst I viljen kunnen I göra dem gott, men mig haven I icke alltid.
\par 8 Vad hon kunde, det gjorde hon. Hon har i förväg smort min kropp såsom en tillredelse till min begravning.
\par 9 Och sannerligen säger jag eder: Varhelst i hela världen evangelium bliver predikat, där skall ock det som hon nu har gjort bliva omtalat, henne till åminnelse."
\par 10 Och Judas Iskariot, han som var en av de tolv, gick bort till översteprästerna och ville förråda honom åt dem.
\par 11 När de hörde detta, blevo de glada och lovade att giva honom en summa penningar. Sedan sökte han efter tillfälle att förråda honom, då lägligt var.
\par 12 På första dagen i det osyrade brödets högtid, när man slaktade påskalammet, sade hans lärjungar till honom: "Vart vill du att vi skola gå och reda till, så att du kan äta påskalammet?"
\par 13 Då sände han åstad två av sina lärjungar och sade till dem: "Gån in i staden; där skolen I möta en man som bär en kruka vatten. Följen honom.
\par 14 Och sägen till husbonden i det hus där han går in: 'Mästaren frågar: Var finnes härbärget där jag skall äta påskalammet med mina lärjungar?'
\par 15 Då skall han visa eder en stor sal i övre våningen, tillredd och ordnad för måltid; reden till åt oss där."
\par 16 Och lärjungarna begåvo sig i väg och kommo in i staden och funno det så som han hade sagt dem; och de redde till påskalammet.
\par 17 När det sedan hade blivit afton, kom han dit med de tolv.
\par 18 Och medan de lågo till bords och åto, sade Jesus: "Sannerligen säger jag eder: En av eder skall förråda mig, 'den som äter med mig'."
\par 19 Då begynte de bedrövas och fråga honom, den ene efter den andre: "Icke är det väl jag?"
\par 20 Och han sade till dem: "Det är en av de tolv, den som jämte mig doppar i fatet.
\par 21 Ja, Människosonen skall gå bort, såsom det är skrivet om honom; men ve den människa genom vilken Människosonen bliver förrådd! Det hade varit bättre för den människan, om hon icke hade blivit född."
\par 22 Medan de nu åto, tog han ett bröd och välsignade det och bröt det och gav åt dem och sade: "Tagen detta; detta är min lekamen."
\par 23 Och han tog en kalk och tackade Gud ock gav åt dem; och de drucko alla därav.
\par 24 Och han sade till dem: "Detta är mitt blod, förbundsblodet, som varder utgjutet för många.
\par 25 Sannerligen säger jag eder: Jag skall icke mer dricka av det som kommer från vinträd, förrän på den dag då jag dricker det nytt i Guds rike."
\par 26 När de sedan hade sjungit lovsången, gingo de ut till Oljeberget.
\par 27 Då sade Jesus till dem: "I skolen alla komma på fall; ty det är skrivet: 'Jag skall slå herden, och fåren skola förskingras.'
\par 28 Men efter min uppståndelse skall jag före eder gå till Galileen."
\par 29 Då svarade Petrus honom: "Om än alla andra komma på fall, så skall dock jag det icke."
\par 30 Jesus sade till honom: "Sannerligen säger jag dig: Redan i denna natt, förrän hanen har galit två gånger, skall du tre gånger förneka mig."
\par 31 Då försäkrade han ännu ivrigare: "Om jag än måste dö med dig, så skall jag dock icke förneka dig." Sammalunda sade ock alla de andra. Och de kommo till ett ställe som kallades Getsemane.
\par 32 Då sade han till sina lärjungar: "Bliven kvar här, medan jag beder."
\par 33 Och han tog med sig Petrus och Jakob och Johannes; och han begynte bäva och ängslas.
\par 34 Och han sade till dem: "Min själ är djupt bedrövad, ända till döds; stannen kvar här och vaken."
\par 35 Därefter gick han litet längre bort och föll ned på jorden och bad, att om möjligt vore, den stunden skulle bliva honom besparad.
\par 36 Och han sade: "Abba, Fader, allt är möjligt för dig. Tag denna kalk ifrån mig. Dock icke vad jag vill, utan vad du vill!"
\par 37 Sedan kom han tillbaka och fann dem sovande. Då sade han till Petrus: "Simon, sover du? Förmådde du då icke vaka en kort stund?
\par 38 Vaken, och bedjen att I icke mån komma i frestelse. Anden är villig, men köttet är svagt."
\par 39 Och han gick åter bort och bad och sade samma ord.
\par 40 När han sedan kom tillbaka, fann han dem åter sovande, ty deras ögon voro förtyngda. Och de visste icke vad de skulle svara honom.
\par 41 För tredje gången kom han tillbaka och sade då till dem: "Ja, I soven ännu alltjämt och vilen eder! Det är nog. Stunden är kommen. Människosonen skall nu bliva överlämnad i syndarnas händer.
\par 42 Stån upp, låt oss gå; se, den som förråder mig är nära."
\par 43 Och i detsamma, medan han ännu talade, kom Judas, en av de tolv, och jämte honom en folkskara med svärd och stavar, utsänd från översteprästerna och de skriftlärde och de äldste.
\par 44 Men förrädaren hade kommit överens med dem om ett tecken och sagt: "Den som jag kysser, den är det; honom skolen I gripa och föra bort under säker bevakning."
\par 45 Och när han nu kom dit, trädde han strax fram till honom och sade: "Rabbi!" och kysste honom häftigt.
\par 46 Då grepo de Jesus och togo honom fången.
\par 47 Men en av dem som stodo där bredvid drog sitt svärd och högg till översteprästens tjänare och högg så av honom örat.
\par 48 Och Jesus talade till dem och sade: "Såsom mot en rövare haven I gått ut med svärd och stavar för att fasttaga mig.
\par 49 Var dag har jag varit ibland eder i helgedomen och undervisat, utan att I haven gripit mig. Men skrifterna skulle ju fullbordas."
\par 50 Då övergåvo de honom alla och flydde.
\par 51 Och bland dem som hade följt med honom var en ung man, höljd i ett linnekläde, som var kastat över blotta kroppen; honom grepo de.
\par 52 Men han lämnade linneklädet kvar och flydde undan naken.
\par 53 Så förde de nu Jesus bort till översteprästen, och där församlade sig alla översteprästerna och de äldste och de skriftlärde.
\par 54 Och Petrus följde honom på avstånd ända in på översteprästens gård; där satt han sedan tillsammans med tjänarna och värmde sig vid elden.
\par 55 Och översteprästerna och hela Stora rådet sökte efter något vittnesbörd mot Jesus, för att kunna döda honom; men de funno intet.
\par 56 Ty väl vittnade många falskt mot honom, men vittnesbörden stämde icke överens.
\par 57 Och några stodo upp och vittnade falskt mot honom och sade:
\par 58 "Vi hava själva hört honom säga: 'Jag skall bryta ned detta tempel, som är gjort med händer, och skall sedan på tre dagar bygga upp ett annat, som icke är gjort med händer.'"
\par 59 Men icke ens i det stycket stämde deras vittnesbörd överens.
\par 60 Då stod översteprästen upp ibland dem och frågade Jesus och sade: "Svarar du intet? Huru är det med det som dessa vittna mot dig?"
\par 61 Men han teg och svarade intet. Åter frågade översteprästen honom och sade till honom: "Är du Messias, den Högtlovades Son?"
\par 62 Jesus svarade: "Jag är det. Och I skolen få se Människosonen sitta på Maktens högra sida och komma med himmelens skyar."
\par 63 Då rev översteprästen sönder sina kläder och sade: "Vad behöva vi mer några vittnen?
\par 64 I hörden hädelsen. Vad synes eder?" Då dömde de alla honom skyldig till döden.
\par 65 Och några begynte spotta på honom; och sedan de hade höljt över hans ansikte, slogo de honom på kinderna med knytnävarna och sade till honom: "Profetera." Också rättstjänarna slogo honom på kinderna.
\par 66 Medan nu Petrus befann sig därnere på gården, kom en av översteprästens tjänstekvinnor dit.
\par 67 Och när hon fick se Petrus, där han satt och värmde sig, såg hon på honom och sade: "Också du var med nasaréen, denne Jesus."
\par 68 Men han nekade och sade: "Jag varken vet eller förstår vad du menar." Sedan gick han ut på den yttre gården.
\par 69 När tjänstekvinnan då fick se honom där, begynte hon åter säga till dem som stodo bredvid: "Denne är en av dem."
\par 70 Då nekade han åter. Litet därefter sade återigen de som stodo där bredvid till Petrus: "Förvisso är du en av dem; du är ju också en galilé."
\par 71 Då begynte han förbanna sig och svärja: "Jag känner icke den man som I talen om."
\par 72 Och i detsamma gol hanen för andra gången. Då kom Petrus ihåg Jesu ord, huru han hade sagt till honom: "Förrän hanen har galit två gånger, skall du tre gånger förneka mig." Och han brast ut i gråt.

\chapter{15}

\par 1 Sedan nu översteprästerna, tillsammans med de äldste och de skriftlärde, hela Stora rådet, på morgonen hade fattat sitt beslut, läto de strax binda Jesus och förde honom bort och överlämnade honom åt Pilatus.
\par 2 Då frågade Pilatus honom: "Är du judarnas konung?" Han svarade honom och sade: "Du säger det själv."
\par 3 Och översteprästerna framställde många anklagelser mot honom.
\par 4 Pilatus frågade honom då åter och sade: "Svarar du intet? Du hör ju huru mycket det är som de anklaga dig för."
\par 5 Men Jesus svarade intet mer, så att Pilatus förundrade sig.
\par 6 Nu plägade han vid högtiden giva dem en fånge lös, den som de begärde.
\par 7 Och där fanns då en man, han som kallades Barabbas, vilken satt fängslad jämte de andra som hade gjort upplopp och under upploppet begått dråp.
\par 8 Folket kom ditupp och begynte begära att han skulle göra åt dem såsom han plägade göra.
\par 9 Pilatus svarade dem och sade: "Viljen I att jag skall giva eder 'judarnas konung' lös?"
\par 10 Han förstod nämligen att det var av avund som översteprästerna hade dragit Jesus inför rätta.
\par 11 Men översteprästerna uppeggade folket till att begära att han hellre skulle giva dem Barabbas lös.
\par 12 När alltså Pilatus åter tog till orda och frågade dem: "Vad skall jag då göra med den som I kallen 'judarnas konung'?",
\par 13 så skriade de åter: "Korsfäst honom!"
\par 14 Men Pilatus frågade dem: "Vad ont har han då gjort?" Då skriade de ännu ivrigare: "Korsfäst honom!"
\par 15 Och eftersom Pilatus ville göra folket till viljes, gav han dem Barabbas lös; men Jesus lät han gissla och utlämnade honom sedan till att korsfästas.
\par 16 Och krigsmännen förde honom in i palatset, eller pretoriet, och kallade tillhopa hela den romerska vakten.
\par 17 Och de klädde på honom en purpurfärgad mantel och vredo samman en krona av törnen och satte den på honom.
\par 18 Sedan begynte de hälsa honom: "Hell dig, judarnas konung!"
\par 19 Och de slogo honom i huvudet med ett rör och spottade på honom; därvid böjde de knä och gåvo honom sin hyllning.
\par 20 Och när de hade begabbat honom, klädde de av honom den purpurfärgade manteln och satte på honom hans egna kläder och förde honom ut för att korsfästa honom.
\par 21 Och en man som kom utifrån marken gick där fram, Simon från Cyrene, Alexanders och Rufus' fader; honom tvingade de att gå med och bära hans kors.
\par 22 Och de förde honom till Golgataplatsen (det betyder huvudskalleplatsen).
\par 23 Och de räckte honom vin, blandat med myrra, men han tog icke emot det.
\par 24 Och de korsfäste honom och delade sedan hans kläder mellan sig, genom att kasta lott om vad var och en skulle få.
\par 25 Och det var vid tredje timmen som de korsfäste honom.
\par 26 Och den överskrift som man hade satt upp över honom, för att angiva vad han var anklagad för, hade denna lydelse: "Judarnas konung."
\par 27 Och de korsfäste med honom två rövare, den ene på hans högra sida och den andre på hans vänstra.
\par 28 Och så fullbordades detta skriftens ord: "Han blev räknad bland ogärningsmän."
\par 29 Och de som gingo där förbi bespottade honom och skakade huvudet och sade: "Tvi dig, du som 'bryter ned templet och bygger upp det igen inom tre dagar'!
\par 30 Hjälp dig nu själv, och stig ned från korset."
\par 31 Sammalunda talade ock översteprästerna, jämte de skriftlärde, begabbande ord med varandra och sade: "Andra har han hjälpt; sig själv kan han icke hjälpa.
\par 32 Han som är Messias, Israels konung, han stige nu ned från korset, så att vi få se det och tro." Också de män som voro korsfästa med honom smädade honom.
\par 33 Men vid sjätte timmen kom över hela landet ett mörker, som varade ända till nionde timmen.
\par 34 Och vid nionde timmen ropade Jesus med hög röst: "Eloi, Eloi, lema sabaktani?"; det betyder: "Min Gud, min Gud, varför har du övergivit mig?"
\par 35 Då några av dem som stodo där bredvid hörde detta, sade de: "Hör, han kallar på Elias."
\par 36 Men en av dem skyndade fram och fyllde en svamp med ättikvin och satte den på ett rör och gav honom att dricka, i det han sade: "Låt oss se om Elias kommer och tager honom ned."
\par 37 Men Jesus ropade med hög röst och gav upp andan.
\par 38 Då rämnade förlåten i templet i två stycken, uppifrån och ända ned.
\par 39 Men när hövitsmannen, som stod där mitt emot honom, såg att han på sådant sätt gav upp andan, sade han: "Förvisso var denne man Guds Son."
\par 40 Också några kvinnor stodo där på avstånd och sågo vad som skedde. Bland dessa voro jämväl Maria från Magdala och den Maria som var Jakob den yngres och Joses' moder, så ock Salome
\par 41 - vilka hade följt honom och tjänat honom, medan han var i Galileen - därtill många andra kvinnor, de som med honom hade vandrat upp till Jerusalem.
\par 42 Det var nu tillredelsedag (det är dagen före sabbaten), och det hade blivit afton.
\par 43 Josef från Arimatea, en ansedd rådsherre och en av dem som väntade på Guds rike, tog därför nu mod till sig och gick in till Pilatus och utbad sig att få Jesu kropp.
\par 44 Då förundrade sig Pilatus över att Jesus redan skulle vara död, och han kallade till sig hövitsmannen och frågade honom om det var länge sedan han hade dött.
\par 45 Och när han av hövitsmannen hade fått veta huru det var, skänkte han åt Josef hans döda kropp.
\par 46 Denne köpte då en linneduk och tog honom ned och svepte honom i linneduken och lade honom i en grav som var uthuggen i en klippa; sedan vältrade han en sten för ingången till graven.
\par 47 Men Maria från Magdala och den Maria som var Joses' moder sågo var han lades.

\chapter{16}

\par 1 Och när sabbaten var förliden, köpte Maria från Magdala och den Maria som var Jakobs moder och Salome välluktande kryddor, för att sedan gå åstad och smörja honom.
\par 2 Och bittida om morgonen på första veckodagen kommo de till graven, redan vid soluppgången.
\par 3 Och de sade till varandra: "Vem skall åt oss vältra bort stenen från ingången till graven?"
\par 4 Men när de sågo upp, fingo de se att stenen redan var bortvältrad. Den var nämligen mycket stor.
\par 5 Och när de hade kommit in i graven, fingo de se en ung man sitta där på högra sidan, klädd i en vit fotsid klädnad; och de blevo förskräckta.
\par 6 Men han sade till dem: "Varen icke förskräckta. I söken Jesus från Nasaret, den korsfäste. Han är uppstånden, han är icke här. Se där är platsen där de lade honom.
\par 7 Men gån bort och sägen till hans lärjungar, och särskilt till Petrus: 'Han skall före eder gå till Galileen; där skolen I få se honom, såsom han bar sagt eder.'"
\par 8 Då gingo de ut och flydde bort ifrån graven, ty bävan och bestörtning hade kommit över dem. Och i sin fruktan sade de intet till någon.
\par 9 Men efter sin uppståndelse visade han sig på första veckodagens morgon först för Maria från Magdala, ur vilken han hade drivit ut sju onda andar.
\par 10 Hon gick då och omtalade det för dem som hade följt med honom, och som nu sörjde och gräto.
\par 11 Men när dessa hörde sägas att han levde och hade blivit sedd av henne, trodde de det icke.
\par 12 Därefter uppenbarade han sig i en annan skepnad för två av dem, medan de voro stadda på vandring utåt landsbygden.
\par 13 Också dessa gingo bort och omtalade det för de andra; men icke heller dem trodde man.
\par 14 Sedan uppenbarade han sig också för de elva, när de lågo till bords; och han förebrådde dem då deras otro och deras hjärtans hårdhet, i det att de icke hade trott dem som hade sett honom vara uppstånden.
\par 15 Och han sade till dem: "Gån ut i hela världen och prediken evangelium för allt skapat.
\par 16 Den som tror och bliver döpt, han skall bliva frälst; men den som icke tror, han skall bliva fördömd.
\par 17 Och dessa tecken skola åtfölja dem som tro: genom mitt namn skola de driva ut onda andar, de skola tala nya tungomål,
\par 18 ormar skola de taga i händerna, och om de dricka något dödande gift, så skall det alls icke skada dem; på sjuka skola de lägga händerna, och de skola då bliva friska."
\par 19 Därefter, sedan Herren Jesus hade talat med dem, blev han upptagen i himmelen och satte sig på Guds högra sida.
\par 20 Men de gingo ut och predikade allestädes. Och Herren verkade med dem och stadfäste ordet genom de tecken som åtföljde det.


\end{document}