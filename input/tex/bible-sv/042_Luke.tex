\begin{document}

\title{Luke}

Luk 1:1  Alldenstund många andra hava företagit sig att om de händelser, som bland oss hava timat, avfatta berättelser,
Luk 1:2  i enlighet med vad som har blivit oss meddelat av dem som själva voro åsyna vittnen och ordets tjänare,
Luk 1:3  så har ock jag, sedan jag grundligt har efterforskat allt ända ifrån begynnelsen, beslutit mig för att i följd och ordning skriva därom till dig, ädle Teofilus,
Luk 1:4  så att du kan inse huru tillförlitliga de stycken äro, i vilka du har blivit undervisad.
Luk 1:5  På den tid då Herodes var konung över Judeen levde en präst vid namn Sakarias, av Abias' "dagsavdelning". Denne hade till hustru en av Arons döttrar som hette Elisabet.
Luk 1:6  De voro båda rättfärdiga inför Gud och vandrade ostraffligt efter alla Herrens bud och stadgar.
Luk 1:7  Men de hade inga barn, ty Elisabet var ofruktsam; och båda voro de komna till hög ålder.
Luk 1:8  Medan han nu en gång, när ordningen kom till hans avdelning, gjorde prästerlig tjänst inför Gud,
Luk 1:9  hände det sig, vid den övliga lottningen om de prästerliga sysslorna, att det tillföll honom att gå in i Herrens tempel och tända rökelsen.
Luk 1:10  Och hela menigheten stod utanför och bad, medan rökoffret förrättades.
Luk 1:11  Då visade sig för honom en Herrens ängel, stående på högra sidan om rökelsealtaret.
Luk 1:12  Och när Sakarias såg honom, blev han förskräckt, och fruktan föll över honom.
Luk 1:13  Men ängeln sade till honom: "Frukta icke, Sakarias; ty din bön är hörd, och din hustru Elisabet skall föda dig en son, och honom skall du giva namnet Johannes.
Luk 1:14  Och han skall bliva dig till glädje och fröjd, och många skola glädja sig över hans födelse.
Luk 1:15  Ty han skall bliva stor inför Herren. Vin och starka drycker skall han icke dricka, och redan i sin moders liv skall han bliva uppfylld av helig ande.
Luk 1:16  Och många av Israels barn skall han omvända till Herren, deras Gud.
Luk 1:17  Han skall gå framför honom i Elias' ande och kraft, för att 'vända fädernas hjärtan till barnen' och omvända de ohörsamma till de rättfärdigas sinnelag, så att han skaffar åt Herren ett välberett folk."
Luk 1:18  Då sade Sakarias till ängeln: "Varav skall jag veta detta? Jag är ju själv gammal, och min hustru är kommen till hög ålder."
Luk 1:19  Ängeln svarade och sade till honom: "Jag är Gabriel, som står inför Gud, och jag är utsänd för att tala till dig och förkunna dig detta glada budskap.
Luk 1:20  Och se, ända till den dag då detta sker skall du vara mållös och icke kunna tala, därför att du icke trodde mina ord, vilka dock i sin tid skola fullbordas."
Luk 1:21  Och folket stod och väntade på Sakarias och förundrade sig över att han så länge dröjde i templet;
Luk 1:22  och när han kom ut, kunde han icke tala till dem. Då förstodo de att han hade sett någon syn i templet. Och han tecknade åt dem och förblev stum.
Luk 1:23  Och när tiden för hans tjänstgöring hade gått till ända, begav han sig hem.
Luk 1:24  Men efter den tiden blev hans hustru Elisabet havande och höll sig dold i fem månader;
Luk 1:25  och hon sade: "Så har Herren gjort med mig nu, då han har sett till min smälek bland människorna, för att borttaga den."
Luk 1:26  I sjätte månaden blev ängeln Gabriel sänd av Gud till en stad i Galileen som hette Nasaret,
Luk 1:27  till en jungfru som var trolovad med en man vid namn Josef, av Davids hus; och jungfruns namn var Maria.
Luk 1:28  Och ängeln kom in till henne och sade: "Hell dig, du högtbenådade! Herren är med dig."
Luk 1:29  Men hon blev mycket förskräckt vid hans ord och tänkte på vad denna hälsning månde innebära.
Luk 1:30  Då sade ängeln till henne: "Frukta icke, Maria; ty du har funnit nåd för Gud.
Luk 1:31  Se, du skall bliva havande och föda en son, och honom skall du giva namnet Jesus.
Luk 1:32  Han skall bliva stor och kallas den Högstes Son, och Herren Gud skall giva honom hans fader Davids tron.
Luk 1:33  Och han skall vara konung över Jakobs hus till evig tid, och på hans rike skall ingen ände vara."
Luk 1:34  Då sade Maria till ängeln: "Huru skall detta ske? Jag vet ju icke av någon man."
Luk 1:35  Ängeln svarade och sade till henne: "Helig ande skall komma över dig, och kraft från den Högste skall överskygga dig; därför skall ock det heliga som varder fött kallas Guds Son.
Luk 1:36  Och se, jämväl din fränka Elisabet har blivit havande och skall föda en son, nu på sin ålderdom; och detta är sjätte månaden för henne, som säges vara ofruktsam.
Luk 1:37  Ty för Gud kan intet vara omöjligt."
Luk 1:38  Då sade Maria: "Se, jag är Herrens tjänarinna; ske mig såsom du har sagt." Och ängeln lämnade henne.
Luk 1:39  En av de närmaste dagarna stod Maria upp och begav sig skyndsamt till en stad i Judeen, uppe i bergsbygden.
Luk 1:40  Och hon trädde in i Sakarias' hus och hälsade Elisabet.
Luk 1:41  När då Elisabet hörde Marias hälsning, spratt barnet till i hennes liv; och Elisabet blev fylld av helig ande
Luk 1:42  och brast ut och ropade högt och sade: "Välsignad vare du bland kvinnor, och välsignad din livsfrukt!
Luk 1:43  Men varför sker mig detta, att min Herres moder kommer till mig?
Luk 1:44  Se, när ljudet av din hälsning nådde mina öron, spratt barnet till av fröjd i mitt liv.
Luk 1:45  Och salig är du, som trodde att det skulle fullbordas, som blev dig sagt från Herren."
Luk 1:46  Då sade Maria: "Min själ prisar storligen Herren,
Luk 1:47  och min ande fröjdar sig i Gud, min Frälsare.
Luk 1:48  Ty han har sett till sin tjänarinnas ringhet; och se, härefter skola alla släkten prisa mig salig.
Luk 1:49  Ty den Mäktige har gjort stora ting med mig, och heligt är hans namn.
Luk 1:50  Hans barmhärtighet varar från släkte till släkte över dem som frukta honom.
Luk 1:51  Han har utfört väldiga gärningar med sin arm, han har förskingrat dem som tänkte övermodiga tankar i sina hjärtan.
Luk 1:52  Härskare har han störtat från deras troner, och ringa män har han upphöjt;
Luk 1:53  hungriga har han mättat med sitt goda, och rika har han skickat bort med tomma händer.
Luk 1:54  Han har tagit sig an sin tjänare Israel och tänkt på att bevisa barmhärtighet
Luk 1:55  mot Abraham och mot hans säd till evig tid, efter sitt löfte till våra fäder."
Luk 1:56  Och Maria stannade hos henne vid pass tre månader och vände därefter hem igen.
Luk 1:57  Så var nu för Elisabet tiden inne, då hon skulle föda; och hon födde en son.
Luk 1:58  Och när hennes grannar och fränder fingo höra att Herren hade bevisat henne så stor barmhärtighet, gladde de sig med henne.
Luk 1:59  Och på åttonde dagen kommo de för att omskära barnet; och de ville kalla honom Sakarias, efter hans fader.
Luk 1:60  Men hans moder tog till orda och sade: "Ingalunda; han skall heta Johannes."
Luk 1:61  Då sade de till henne: "I din släkt finnes ju ingen som har det namnet."
Luk 1:62  Och de frågade hans fader genom tecken vad han ville att barnet skulle heta.
Luk 1:63  Då begärde han en tavla och skrev dessa ord: "Johannes är hans namn." Och alla förundrade sig.
Luk 1:64  Men i detsamma öppnades hans mun, och hans tunga löstes, och han talade och lovade Gud.
Luk 1:65  Och deras grannar betogos alla av häpnad, och ryktet om allt detta gick ut över Judeens hela bergsbygd.
Luk 1:66  Och alla som hörde det lade märke därtill och sade: "Vad månde väl varda av detta barn?" Också var ju Herrens hand med honom.
Luk 1:67  Och hans fader Sakarias blev uppfylld av helig ande och profeterade och sade:
Luk 1:68  "Lovad vare Herren, Israels Gud, som har sett till sitt folk och berett det förlossning,
Luk 1:69  och som har upprättat åt oss ett frälsningens horn i sin tjänare Davids hus,
Luk 1:70  såsom han hade lovat genom sin forntida heliga profeters mun.
Luk 1:71  Ty han ville frälsa oss från våra ovänner och ur alla våra motståndares hand,
Luk 1:72  och så göra barmhärtighet med våra fäder och tänka på sitt heliga förbund,
Luk 1:73  vad han med ed hade lovat för vår fader Abraham,
Luk 1:74  Han ville beskära oss att få tjäna honom utan fruktan, frälsta ur våra ovänners hand,
Luk 1:75  ja, att göra tjänst inför honom i helighet och rättfärdighet i alla våra dagar.
Luk 1:76  Och du, barn, skall bliva kallad den Högstes profet, ty du skall gå framför Herren och bereda vägar för honom,
Luk 1:77  till att giva hans folk kunskap om frälsning, i det att deras synder bliva dem förlåtna.
Luk 1:78  Så skall ske för vår Guds förbarmande kärleks skull, som skall låta ett ljus gå upp och skåda ned till oss från höjden,
Luk 1:79  för att 'skina över dem som sitta i mörker och dödsskugga' och så styra våra fötter in på fridens väg."
Luk 1:80  Och barnet växte upp och blev allt starkare i anden. Och han vistades i öknen, intill den dag då han skulle träda fram för Israel.
Luk 2:1  Och det hände sig vid den tiden att från kejsar Augustus utgick ett påbud att hela världen skulle skattskrivas.
Luk 2:2  Detta var den första skattskrivningen, och den hölls, när Kvirinius var landshövding över Syrien.
Luk 2:3  Då färdades alla var och en till sin stad, för att låta skattskriva sig.
Luk 2:4  Så gjorde ock Josef; och eftersom han var av Davids hus och släkt, for han från staden Nasaret i Galileen upp till Davids stad, som heter Betlehem, i Judeen,
Luk 2:5  för att låta skattskriva sig jämte Maria, sin trolovade, som var havande.
Luk 2:6  Medan de voro där, hände sig att tiden var inne, då hon skulle föda.
Luk 2:7  Och hon födde sin förstfödde son och lindade honom och lade honom i en krubba, ty det fanns icke rum för dem i härbärget.
Luk 2:8  I samma nejd voro då några herdar ute på marken och höllo vakt om natten över sin hjord.
Luk 2:9  Då stod en Herrens ängel framför dem, och Herrens härlighet kringstrålade dem; och de blevo mycket förskräckta.
Luk 2:10  Men ängeln sade till dem: "Varen icke förskräckta. Se, jag bådar eder en stor glädje, som skall vederfaras allt folket.
Luk 2:11  Ty i dag har en Frälsare blivit född åt eder i Davids stad, och han är Messias, Herren.
Luk 2:12  Och detta skall för eder vara tecknet: I skolen finna ett nyfött barn, som ligger lindat i en krubba."
Luk 2:13  I detsamma sågs där jämte ängeln en stor hop av den himmelska härskaran, och de lovade Gud och sade:
Luk 2:14  "Ära vare Gud i höjden, och frid på jorden, bland människor till vilka han har behag!"
Luk 2:15  När så änglarna hade farit ifrån herdarna upp i himmelen, sade dessa till varandra: "Låt oss nu gå till Betlehem och se det som där har skett, och som Herren har kungjort för oss."
Luk 2:16  Och de skyndade åstad dit och funno Maria och Josef, och barnet som låg i krubban.
Luk 2:17  Och när de hade sett det, omtalade de vad som hade blivit sagt till dem om detta barn.
Luk 2:18  Och alla som hörde det förundrade sig över vad herdarna berättade för dem.
Luk 2:19  Men Maria gömde och begrundade allt detta i sitt hjärta.
Luk 2:20  Och herdarna vände tillbaka och prisade och lovade Gud för allt vad de hade fått höra och se, alldeles såsom det blivit dem sagt.
Luk 2:21  När sedan åtta dagar hade gått till ända och han skulle omskäras, gavs honom namnet Jesus, det namn som hade blivit nämnt av ängeln, förrän han blev avlad i sin moders liv.
Luk 2:22  Och när deras reningsdagar hade gått till ända, de som voro föreskrivna i Moses' lag, förde de honom upp till Jerusalem för att bära honom fram inför Herren,
Luk 2:23  enligt den föreskriften i Herrens lag, att "allt mankön som öppnar moderlivet skall räknas såsom helgat åt Herren",
Luk 2:24  så ock för att offra "ett par duvor eller två unga turturduvor", såsom stadgat var i Herrens lag.
Luk 2:25  I Jerusalem levde då en man, vid namn Simeon, en rättfärdig och from man, som väntade på Israels tröst; och helig ande var över honom.
Luk 2:26  Och av den helige Ande hade han fått den uppenbarelsen att han icke skulle se döden, förrän han hade fått se Herrens Smorde.
Luk 2:27  Han kom nu genom Andens tillskyndelse till helgedomen. Och när föräldrarna buro in barnet Jesus, för att så göra med honom som sed var efter lagen,
Luk 2:28  då tog också han honom i sin famn och lovade Gud och sade:
Luk 2:29  "Herre, nu låter du din tjänare fara hädan i frid, efter ditt ord,
Luk 2:30  ty mina ögon hava sett din frälsning,
Luk 2:31  vilken du har berett till att skådas av alla folk:
Luk 2:32  ett ljus som skall uppenbaras för hedningarna, och en härlighet som skall givas åt ditt folk Israel."
Luk 2:33  Och hans fader och moder förundrade sig över det som sades om honom.
Luk 2:34  Och Simeon välsignade dem och sade till Maria, hans moder: "Se, denne är satt till fall eller upprättelse för många i Israel, och till ett tecken som skall bliva motsagt.
Luk 2:35  Ja, också genom din själ skall ett svärd gå. Så skola många hjärtans tankar bliva uppenbara."
Luk 2:36  Där fanns ock en profetissa, Hanna, Fanuels dotter, av Asers stam. Hon var kommen till hög ålder; i sju år hade hon levat med sin man, från den tid då hon var jungfru,
Luk 2:37  och hon var nu änka, åttiofyra år gammal. Och hon lämnade aldrig helgedomen, utan tjänade där Gud med fastor och böner, natt och dag.
Luk 2:38  Hon kom också i samma stund tillstädes och prisade Gud och talade om honom till alla dem som väntade på förlossning för Jerusalem.
Luk 2:39  Och när de hade fullgjort allt som var stadgat i Herrens lag, vände de tillbaka till sin stad, Nasaret i Galileen.
Luk 2:40  Men barnet växte upp och blev allt starkare och uppfylldes av vishet; och Guds nåd var över honom.
Luk 2:41  Nu plägade hans föräldrar årligen vid påskhögtiden begiva sig till Jerusalem.
Luk 2:42  När han var tolv år gammal, gingo de också dit upp, såsom sed var vid högtiden.
Luk 2:43  Men när de hade varit med om alla högtidsdagarna och vände hem igen, stannade gossen Jesus kvar i Jerusalem, utan att hans föräldrar lade märke därtill.
Luk 2:44  De menade att han var med i ressällskapet och vandrade så en dagsled och sökte efter honom bland fränder och vänner.
Luk 2:45  När de då icke funno honom, vände de tillbaka till Jerusalem och sökte efter honom.
Luk 2:46  Och efter tre dagar funno de honom i helgedomen, där han satt mitt ibland lärarna och hörde på dem och frågade dem;
Luk 2:47  och alla som hörde honom blevo uppfyllda av häpnad över hans förstånd och hans svar.
Luk 2:48  När de nu fingo se honom där, förundrade de sig högeligen; och hans moder sade till honom: "Min son, varför gjorde du oss detta? Se, din fader och jag hava sökt efter dig med stor oro."
Luk 2:49  Då sade han till dem: "Varför behövden I söka efter mig? Vissten I då icke att jag bör vara där min Fader bor?"
Luk 2:50  Men de förstodo icke det som han talade till dem.
Luk 2:51  Så följde han med dem och kom ned till Nasaret; och han var dem underdånig. Och hans moder gömde allt detta i sitt hjärta.
Luk 2:52  Och Jesus växte till i ålder och vishet och nåd inför Gud och människor.
Luk 3:1  I femtonde året av kejsar Tiberius' regering, när Pontius Pilatus var landshövding i Judeen, och Herodes var landsfurste i Galileen, och hans broder Filippus landsfurste i Itureen och Trakonitis-landet, och Lysanias landsfurste i Abilene,
Luk 3:2  på den tid då Hannas var överstepräst jämte Kaifas - då kom Guds befallning till Johannes, Sakarias' son, i öknen;
Luk 3:3  och han gick åstad och predikade i hela trakten omkring Jordan bättringens döpelse till syndernas förlåtelse.
Luk 3:4  Så uppfylldes vad som var skrivet i profeten Esaias' utsagors bok: "Hör rösten av en som ropar i öknen: Bereden vägen för Herren, gören stigarna jämna för honom.
Luk 3:5  Alla dalar skola fyllas och alla berg och höjder sänkas; vad krokigt är skall bliva rak väg, och vad oländigt är skall bliva släta stigar;
Luk 3:6  och allt kött skall se Guds frälsning.'"
Luk 3:7  Han sade nu till folket som kom ut för att låta döpa sig av honom: "I huggormars avföda, vem har ingivit eder att söka komma undan den tillstundande vredesdomen?
Luk 3:8  Bären då ock sådan frukt som tillhör bättringen. Och sägen icke vid eder själva: 'Vi hava ju Abraham till fader'; Ty jag säger eder att Gud av dessa stenar kan uppväcka barn åt Abraham.
Luk 3:9  Redan är också yxan satt till roten på träden; så bliver då vart träd som icke bär god frukt avhugget och kastat på elden.
Luk 3:10  Och folket frågade honom och sade: "Vad skola vi då göra?"
Luk 3:11  Han svarade och sade till dem: "Den som har två livklädnader, han dele med sig åt den som icke har någon; och en som har matförråd, han göre sammalunda."
Luk 3:12  Så kommo ock publikaner för att låta döpa sig, och de sade till honom: "Mästare, vad skola vi göra?"
Luk 3:13  Han svarade dem: "Kräven icke ut mer än vad som är eder föreskrivet."
Luk 3:14  Också krigsmän frågade honom och sade: "Vad skola då vi göra? Han svarade dem: "Tilltvingen eder icke penningar av någon, genom hot eller på annat otillbörligt sätt, utan låten eder nöja med eder sold."
Luk 3:15  Och folket gick där i förbidan, och alla undrade i sina hjärtan om Johannes icke till äventyrs vore Messias.
Luk 3:16  Men Johannes tog till orda och sade: till dem alla: "Jag döper eder med vatten, men den som kommer, som är starkare än jag, den vilkens skorem jag icke är värdig att upplösa; han skall döpa eder i helig ande och eld.
Luk 3:17  Han har sin kastskovel i handen, ty han vill noga rensa sin loge och samla in vetet i sin lada; men agnarna skall han bränna upp i en eld som icke utsläckes."
Luk 3:18  Så förmanade han folket också i många andra stycken och förkunnade evangelium för dem.
Luk 3:19  Men när han hade förehållit Herodes, landsfursten, hans synd i fråga om hans broders hustru Herodias och förehållit honom allt det onda som han eljest hade gjort,
Luk 3:20  lade Herodes till allt annat också det att han inspärrade Johannes i fängelse.
Luk 3:21  När nu allt folket lät döpa sig och jämväl Jesus blev döpt, så skedde därvid, medan han bad, att himmelen öppnades,
Luk 3:22  och den helige Ande sänkte sig ned över honom I lekamlig skepnad såsom en duva; och från himmelen kom en röst: "Du är min älskade Son; i dig har jag funnit behag."
Luk 3:23  Och Jesus var vid pass trettio år gammal, när han begynte sitt verk. Och man menade att han var son av Josef, som var son av Eli,
Luk 3:24  som var son av Mattat, som var son av Levi, som var son av Melki, som var son av Jannai, som var son av Josef,
Luk 3:25  som var son av Mattatias, som var son av Amos, som var son av Naum, som var son av Esli, som var son av Naggai,
Luk 3:26  som var son av Maat, som var son av Mattatias, som var son av Semein, som var son av Josek, som var son av Joda,
Luk 3:27  som var son av Joanan, som var son av Resa, som var son av Sorobabel, som var son av Salatiel, som var son av Neri,
Luk 3:28  som var son av Melki, som var son av Addi, som var son av Kosam, som var son av Elmadam, som var son av Er,
Luk 3:29  som var son av Jesus, som var son av Elieser, som var son av Jorim, som var son av Mattat, som var son av Levi,
Luk 3:30  som var son av Simeon, som var son av Judas, som var son av Josef, som var son av Jonam, som var son av Eljakim,
Luk 3:31  som var son av Melea, som var son av Menna, som var son av Mattata, som var son av Natam, som var son av David,
Luk 3:32  som var son av Jessai, som var son av Jobed, som var son av Boos, som var son av Sala, som var son av Naasson,
Luk 3:33  som var son av Aminadab, som var son av Admin, som var son av Arni, som var son av Esrom, som var son av Fares, som var son av Judas,
Luk 3:34  som var son av Jakob, som var son av Isak, som var son av Abraham, som var son av Tara, som var son av Nakor,
Luk 3:35  som var son av Seruk, som var son av Ragau, som var son av Falek, som var son av Eber, som var son av Sala,
Luk 3:36  som var son av Kainam, som var son av Arfaksad, som var son av Sem, som var son av Noa, som var son av Lamek,
Luk 3:37  som var son av Matusala, som var son av Enok, som var son av Jaret som var son av Maleleel, som var son av Kainan,
Luk 3:38  som var son av Enos, som var son av Set, som var son av Adam, som var son av Gud.
Luk 4:1  Sedan vände Jesus tillbaka från Jordan, full av helig ande, och fördes genom Anden omkring i öknen
Luk 4:2  och frestades av djävulen under fyrtio dagar. Och under de dagarna åt han intet; men när de hade gått till ända, blev han hungrig.
Luk 4:3  Då sade djävulen till honom: "Är du Guds Son, så bjud denna sten att bliva bröd."
Luk 4:4  Jesus svarade honom: "Det är skrivet: 'Människan skall leva icke allenast av bröd.'"
Luk 4:5  Och djävulen förde honom upp på en höjd och visade honom i ett ögonblick alla riken i världen
Luk 4:6  och sade till honom: "Åt dig vill jag giva makten över allt detta med dess härlighet; ty åt mig har den blivit överlämnad, och åt vem jag vill kan jag giva den.
Luk 4:7  Om du alltså tillbeder inför mig, så skall den hel och hållen höra dig till."
Luk 4:8  Jesus svarade och sade till honom: "Det är skrivet: 'Herren, din Gud, skall du tillbedja, och honom allena skall du tjäna.'"
Luk 4:9  Och han förde honom till Jerusalem och ställde honom uppe på helgedomens mur och sade till honom: "Är du Guds Son, så kasta dig ned härifrån;
Luk 4:10  det är ju skrivet: 'Han skall giva sina änglar befallning om dig, att de skola väl bevara dig';
Luk 4:11  så ock: 'De skola bära dig på händerna, så att du icke stöter din fot mot någon sten.'"
Luk 4:12  Då svarade Jesus och sade till honom: "Det är sagt: 'Du skall icke fresta Herren, din Gud.'"
Luk 4:13  När djävulen så hade slutat med alla sina frestelser, vek han ifrån honom, intill läglig tid.
Luk 4:14  Och Jesus vände i Andens kraft tillbaka till Galileen; och ryktet om honom gick ut i hela den kringliggande trakten.
Luk 4:15  Och han undervisade i deras synagogor och blev prisad av alla.
Luk 4:16  Så kom han till Nasaret, där han var uppfödd. Och på sabbatsdagen gick han, såsom hans sed var, in i synagogan: och där stod han upp till att föreläsa.
Luk 4:17  Då räckte man åt honom profeten Esaias' bok; och när han öppnade boken, fick han se det ställe där det stod skrivet:
Luk 4:18  "Herrens Ande är över mig, ty han har smort mig. Han har satt mig till att förkunna glädjens budskap för de fattiga, till att predika frihet för de fångna och syn för de blinda, ja, till att giva de förtryckta frihet
Luk 4:19  och till att predika ett nådens år från Herren."
Luk 4:20  Sedan lade han ihop boken och gav den tillbaka åt tjänaren och satte sig ned. Och alla som voro i synagogan hade sina ögon fästa på honom.
Luk 4:21  Då begynte han tala och sade till dem: "I dag är detta skriftens ord fullbordat inför edra öron."
Luk 4:22  Och de gåvo honom alla sitt vittnesbörd och förundrade sig över de nådens ord som utgingo från hans mun, och sade: "Är då denne icke Josefs son?"
Luk 4:23  Då sade han till dem: "Helt visst skolen I nu vända mot mig det ordet: 'Läkare, bota dig själv' och säga: 'Sådana stora ting som vi hava hört vara gjorda i Kapernaum, sådana må du göra också här i din fädernestad.'"
Luk 4:24  Och han tillade: "Sannerligen säger jag eder: Ingen profet bliver i sitt fädernesland väl mottagen.
Luk 4:25  Men jag säger eder, såsom sant är: I Israel funnos många änkor på Elias' tid då himmelen var tillsluten i tre år och sex månader, och stor hungersnöd kom över hela landet -
Luk 4:26  och likväl blev Elias icke sänd till någon av dessa, utan allenast till en änka i Sarepta i Sidons land.
Luk 4:27  Och många spetälska funnos i Israel på profeten Elisas tid; och likväl blev ingen av dessa gjord ren, utan allenast Naiman från Syrien."
Luk 4:28  När de som voro i synagogan hörde detta, uppfylldes de alla av vrede
Luk 4:29  och stodo upp och drevo honom ut ur staden och förde honom ända fram till branten av det berg som deras stad var byggd på, och ville störta honom därutför.
Luk 4:30  Men han gick sin väg mitt igenom hopen och vandrade vidare.
Luk 4:31  Och han kom ned till Kapernaum, en stad i Galileen, och undervisade folket på sabbaten.
Luk 4:32  Och de häpnade över hans undervisning, ty han talade med makt och myndighet.
Luk 4:33  Och i synagogan var en man som var besatt av en oren ond ande. Denne ropade med hög röst:
Luk 4:34  "Bort härifrån! Vad har du med oss att göra, Jesus från Nasaret? Har du kommit för att förgöra oss? Jag vet vem du är, du Guds Helige."
Luk 4:35  Men Jesus tilltalade honom strängt och sade: "Tig och far ut ur honom." Då kastade den onde anden omkull mannen mitt ibland dem och for ut ur honom, utan att hava gjort honom någon skada.
Luk 4:36  Och häpnad kom över dem alla, och de talade med varandra och sade: "Vad är det med dennes ord? Med myndighet och makt befaller han ju de orena andarna, och de fara ut."
Luk 4:37  Och ryktet om honom spriddes åt alla håll i den kringliggande trakten.
Luk 4:38  Men han stod upp och gick ut ur synagogan och kom in i Simons hus. Och Simons svärmoder var ansatt av en svår feber, och de bådo honom för henne.
Luk 4:39  Då trädde han fram och lutade sig över henne och näpste febern, och den lämnade henne; och strax stod hon upp och betjänade dem.
Luk 4:40  Men när solen gick ned, förde alla till honom sina sjuka, sådana som ledo av olika slags sjukdomar. Och han lade händerna på var och en av dem och botade dem.
Luk 4:41  Onda andar blevo ock utdrivna ur många, och de ropade därvid och sade: "Du är Guds Son." Men han tilltalade dem strängt och tillsade dem att icke säga något, eftersom de visste att han var Messias.
Luk 4:42  Och när det åter hade blivit dag, gick han åstad bort till en öde trakt. Men folket sökte efter honom; och när de kommo fram till honom, ville de hålla honom kvar och hindra honom att gå sin väg.
Luk 4:43  Men han sade till dem: "Också för de andra städerna måste jag förkunna evangelium om Guds rike, ty därtill har jag blivit utsänd."
Luk 4:44  Och han predikade i synagogorna i Judeen.
Luk 5:1  Då nu en gång folket, för att höra Guds ord, trängde sig inpå honom där han stod vid Gennesarets sjö,
Luk 5:2  fick han se två båtar ligga vid sjöstranden; men de som fiskade hade gått i land och höllo på att skölja sina nät.
Luk 5:3  Då steg han i en av båtarna, den som tillhörde Simon, och bad honom lägga ut något litet från land. Sedan satte han sig ned och undervisade folket från båten.
Luk 5:4  Och när han hade slutat att tala, sade han till Simon: "Lägg ut på djupet; och kasten där ut edra nät till fångst."
Luk 5:5  Då svarade Simon och sade: "Mästare, vi hava arbetat hela natten och fått intet; men på ditt ord vill jag kasta ut näten."
Luk 5:6  Och när de hade gjort så, fingo de en stor hop fiskar i sina nät; och näten gingo sönder.
Luk 5:7  Då vinkade de åt sina kamrater i den andra båten, att dessa skulle komma och hjälpa dem. Och de kommo och fyllde upp båda båtarna, så att de begynte sjunka.
Luk 5:8  När Simon Petrus såg detta, föll han ned för Jesu knän och sade: "Gå bort ifrån mig, Herre; jag är en syndig människa."
Luk 5:9  Ty för detta fiskafänges skull hade han och alla som voro med honom betagits av häpnad,
Luk 5:10  jämväl Jakob och Johannes, Sebedeus' söner, som deltogo med Simon i fisket. Men Jesus sade till Simon: "Frukta icke; härefter skall du fånga människor."
Luk 5:11  Och de förde båtarna i land och lämnade alltsammans och följde honom.
Luk 5:12  Och medan han var i en av städerna, hände sig, att där kom en man som var full av spetälska. När denne fick se Jesus, föll han ned på sitt ansikte och bad honom och sade: "Herre, vill du, så kan du göra mig ren."
Luk 5:13  Då räckte han ut handen och rörde vid honom och sade: "Jag vill; bliv ren." Och strax vek spetälskan ifrån honom.
Luk 5:14  Och han förbjöd honom att omtala detta för någon, men tillade: "Gå åstad och visa dig för prästen och frambär för din rening ett offer, såsom Moses har påbjudit, till ett vittnesbörd för dem."
Luk 5:15  Men ryktet om honom spridde sig dess mer; och mycket folk samlade sig för att höra honom och för att bliva botade från sina sjukdomar.
Luk 5:16  Men han drog sig undan till öde trakter och bad.
Luk 5:17  Nu hände sig en dag, då han undervisade folket, att där sutto några fariséer och laglärare - sådana hade nämligen kommit dit från alla byar i Galileen och Judeen och från Jerusalem - och Herrens kraft verkade, så att sjuka blevo botade av honom.
Luk 5:18  Då kommo några män dit med en lam man, som de buro på en säng; och de försökte komma in med honom för att lägga honom ned framför Jesus.
Luk 5:19  Men då de för folkets skull icke kunde finna något annat sätt att komma in med honom stego de upp på taket och släppte honom tillika med sängen ned genom tegelbeläggningen, mitt ibland dem, framför Jesus.
Luk 5:20  När han såg deras tro, sade han: "Min vän, dina synder äro dig förlåtna."
Luk 5:21  Då begynte de skriftlärde och fariséerna tänka så: "Vad är denne för en, som talar så hädiska ord? Vem kan förlåta synder utom Gud allena?"
Luk 5:22  Men Jesus förnam deras tankar och svarade och sade till dem: "Vad är det I tänken i edra hjärtan?
Luk 5:23  Vilket är lättare att säga: 'Dina synder äro dig förlåtna' eller att säga: 'Stå upp och gå'?
Luk 5:24  Men för att I skolen veta, att Människosonen har makt här på jorden att förlåta synder, så säger jag dig" (och härmed vände han sig till den lame): "Stå upp, tag din säng och gå hem."
Luk 5:25  Då stod han strax upp i deras åsyn och tog sängen, som han hade legat på, och gick hem, prisande Gud.
Luk 5:26  Och de grepos alla av bestörtning och prisade Gud; och de sade, fulla av häpnad: "Vi hava i dag sett förunderliga ting."
Luk 5:27  Sedan begav han sig därifrån. Och han fick se en publikan, vid namn Levi, sitta vid tullhuset. Och han sade till denne: "Följ mig."
Luk 5:28  Då lämnade han allt och stod upp och följde honom.
Luk 5:29  Och Levi gjorde i sitt hus ett stort gästabud för honom; och en stor hop publikaner och andra voro bordsgäster där jämte dem.
Luk 5:30  Men fariséerna - särskilt de skriftlärde bland dem - knorrade mot hans lärjungar och sade: "Huru kunnen I äta och dricka med publikaner och syndare?"
Luk 5:31  Då svarade Jesus och sade till dem: "De är icke de friska som behöva läkare, utan de sjuka.
Luk 5:32  Jag har icke kommit för att kalla rättfärdiga, utan syndare, till bättring.
Luk 5:33  Och de sade till honom: "Johannes' lärjungar fasta ofta och hålla böner, sammalunda ock fariséernas; men dina lärjungar äta och dricka."
Luk 5:34  Jesus svarade dem: "Icke kunnen I väl ålägga bröllopsgästerna att fasta, medan brudgummen ännu är hos dem?
Luk 5:35  Men en annan tid skall komma, och då, när brudgummen tages ifrån dem, då, på den tiden, skola de fasta." -
Luk 5:36  Han framställde ock för dem denna liknelse: "Ingen river av en lapp från en ny mantel och sätter den på en gammal mantel; om någon så gjorde, skulle han icke allenast riva sönder den nya manteln, utan därtill komme, att lappen från den nya manteln icke skulle passa den gamla.
Luk 5:37  Ej heller slår någon nytt vin i gamla skinnläglar; om någon så gjorde, skulle det nya vinet spränga sönder läglarna, och vinet skulle spillas ut, jämte det att läglarna fördärvades.
Luk 5:38  Nej, nytt vin bör man slå i nya läglar. -
Luk 5:39  Och ingen som har druckit gammalt vin vill sedan gärna hava nytt; ty han tycker, att det gamla är bättre."
Luk 6:1  Och det hände sig på en sabbat att han tog vägen genom ett sädesfält; och hans lärjungar ryckte av axen och gnuggade sönder dem med händerna och åto.
Luk 6:2  Då sade några av fariséerna; "Huru kunnen I göra vad som icke är lovligt att göra på sabbaten?
Luk 6:3  Jesus svarade och sade till dem: "Haven I icke läst om det som David gjorde, när han själv och de som följde honom blevo hungriga:
Luk 6:4  huru han då gick in i Guds hus och tog skådebröden och åt, och jämväl gav åt dem som följde honom, fastän det ju icke är lovligt för andra än allenast för prästerna att äta sådant bröd?"
Luk 6:5  Därefter sade han till dem: "Människosonen är herre över sabbaten."
Luk 6:6  På en annan sabbat hände sig att han gick in i synagogan och undervisade. Där var då en man vilkens högra hand var förvissnad.
Luk 6:7  Och de skriftlärde och fariséerna vaktade på honom, för att se om han botade någon på sabbaten; de ville nämligen finna något att anklaga honom för.
Luk 6:8  Men han förstod deras tankar och sade till mannen som hade den förvissnade handen: "Stå upp, och träd fram." Då stod han upp och trädde fram.
Luk 6:9  Sedan sade Jesus till dem: "Jag vill göra eder en fråga. Vilketdera är lovligt på sabbaten: att göra vad gott är, eller att göra vad ont är, att rädda någons liv, eller att förgöra det?"
Luk 6:10  Och han såg sig omkring på dem alla och sade till mannen: "Räck ut din hand." Och han gjorde så; och hans hand blev frisk igen.
Luk 6:11  Men de blevo såsom ursinniga och talade med varandra om vad de skulle kunna företaga sig mot Jesus.
Luk 6:12  Så hände sig på den tiden att han gick åstad upp på berget för att bedja; och han blev kvar där över natten i bön till Gud.
Luk 6:13  Men när det blev dag, kallade han till sig sina lärjungar och utvalde bland dem tolv, som han ock benämnde apostlar:
Luk 6:14  Simon, vilken han ock gav namnet Petrus, och Andreas, hans broder; vidare Jakob och Johannes och Filippus och Bartolomeus
Luk 6:15  och Matteus och Tomas och Jakob, Alfeus' son, och Simon, som kallades ivraren;
Luk 6:16  vidare Judas, Jakobs son, och Judas Iskariot, den som blev en förrädare.
Luk 6:17  Dessa tog han nu med sig och steg åter ned och stannade på en jämn plats; och en stor skara av hans lärjungar var där församlad, så ock en stor hop folk ifrån hela Judeen och Jerusalem, och från kuststräckan vid Tyrus och Sidon.
Luk 6:18  Dessa hade kommit för att höra honom och för att bliva botade från sina sjukdomar. Och jämväl de som voro kvalda av orena andar blevo botade.
Luk 6:19  Och allt folket sökte att få röra vid honom, ty kraft gick ut ifrån honom och botade alla.
Luk 6:20  Och han lyfte upp sina ögon och säg på sina lärjungar och sade: "Saliga ären I, som ären fattiga, ty eder hör Guds rike till.
Luk 6:21  Saliga ären I, som nu hungren, ty I skolen bliva mättade. Saliga ären I, som nu gråten, ty I skolen le.
Luk 6:22  Saliga ären I, när människorna för Människosonens skull hata eder och förskjuta och smäda eder och kasta bort edert namn såsom något ont.
Luk 6:23  Glädjens på den dagen, ja, springen upp av fröjd, ty se, eder lön är stor i himmelen. På samma satt gjorde ju deras fäder med profeterna.
Luk 6:24  Men ve eder, I som ären rika, ty I haven fått ut eder hugnad!
Luk 6:25  Ve eder, som nu ären mätta, ty I skolen hungra! Ve eder, som nu len, ty I skolen sörja och gråta!
Luk 6:26  Ve eder, när alla människor tala väl om eder! På samma sätt gjorde ju deras fader i fråga om de falska profeterna.
Luk 6:27  Men till eder, som hören mig, säger jag: Älsken edra ovänner, gören gott mot dem som hata eder,
Luk 6:28  välsignen dem som förbanna eder, bedjen för dem som förorätta eder.
Luk 6:29  Om någon slår dig på den ena kinden, så håll ock fram den andra åt honom; och om någon tager manteln ifrån dig, så förvägra honom icke heller livklädnaden.
Luk 6:30  Giv åt var och en som beder dig; och om någon tager ifrån dig vad som är ditt, så kräv det icke igen.
Luk 6:31  Såsom I viljen att människorna skola göra mot eder, så skolen I ock göra mot dem.
Luk 6:32  Om I älsken dem som älska eder, vad tack kunnen I få därför? Också syndare älska ju dem av vilka de bliva älskade.
Luk 6:33  Och om I gören gott mot dem som göra eder gott, vad tack kunnen I få därför? Också syndare göra ju detsamma.
Luk 6:34  Och om I lånen åt dem av vilka I kunnen hoppas att själva få något, vad tack kunnen I få därför? Också syndare låna ju åt syndare för att få lika igen.
Luk 6:35  Nej, älsken edra ovänner, och gören gott och given lån utan att hoppas på någon gengäld. Då skall eder lön bliva stor, och då skolen I vara den Högstes barn; ty han är mild mot de otacksamma och onda.
Luk 6:36  Varen barmhärtiga, såsom eder Fader är barmhärtig.
Luk 6:37  Dömen icke, så skolen I icke bliva dömda; fördömen icke, så skolen I icke bliva fördömda. Förlåten, och eder skall bliva förlåtet.
Luk 6:38  Given, och eder skall bliva givet. Ett gott mått, väl packat, skakat och överflödande, skall man giva eder i skötet; ty med det mått som I mäten med skall ock mätas åt eder igen."
Luk 6:39  Han framställde ock för dem denna liknelse: "Kan väl en blind leda en blind? Falla de icke då båda i gropen?
Luk 6:40  Lärjungen är icke förmer än sin mästare; när någon bliver fullärd, så bliver han allenast sin mästare lik.
Luk 6:41  Huru kommer det till, att du ser grandet i din broders öga, men icke bliver varse bjälken i ditt eget öga?
Luk 6:42  Huru kan du säga till din broder: 'Broder, låt mig taga ut grandet i ditt öga', du som icke ser bjälken i ditt eget öga? Du skrymtare, tag först ut bjälken ur ditt eget öga; därefter må du se till, att du kan tala ut grandet i din broders öga.
Luk 6:43  Ty intet gott träd finnes, som bär dålig frukt, och lika litet finnes något dåligt träd som bär god frukt;
Luk 6:44  vart och ett träd kännes ju igen på sin frukt. Icke hämtar man väl fikon ifrån törnen, ej heller skördar man vindruvor av törnbuskar.
Luk 6:45  En god människa bär ur sitt hjärtas goda förråd fram vad gott är, och en ond människa bär ur sitt onda förråd fram vad ont är; ty vad hennes hjärta är fullt av, det talar hennes mun. -
Luk 6:46  Men varför ropen I till mig: 'Herre, Herre', och gören dock icke vad jag säger?
Luk 6:47  Var och en som kommer till mig och hör mina ord och gör efter dem, vem han är lik, det skall jag visa eder.
Luk 6:48  Han är lik en man som ville bygga ett hus och som då grävde djupt och lade dess grund på hälleberget. När sedan översvämning kom, störtade sig vattenströmmen mot det huset, men den förmådde dock icke skaka det, eftersom det var så byggt.
Luk 6:49  Men den som hör och icke gör, han är lik en man som byggde ett hus på blotta jorden, utan att lägga någon grund. Och vattenströmmen störtade sig emot det, och strax föll det samman, och det husets fall blev stort."
Luk 7:1  När han nu hade talat allt detta till slut inför folket, gick han in i Kapernaum.
Luk 7:2  Men där var en hövitsman som hade en tjänare, vilken låg sjuk och var nära döden; och denne var högt skattad av honom.
Luk 7:3  Då han nu fick höra om Jesus, sände han till honom några av judarnas äldste och bad honom komma och bota hans tjänare.
Luk 7:4  När dessa kommo till Jesus, bådo de honom enträget och sade: "Han är värd att du gör honom detta,
Luk 7:5  ty han har vårt folk kärt, och det är han som har byggt synagogan åt oss."
Luk 7:6  Då gick Jesus med dem. Men när han icke var långt ifrån hövitsmannens hus, sände denne några av sina vänner och lät säga till honom: "Herre, gör dig icke omak; ty jag är icke värdig att du går in under mitt tak.
Luk 7:7  Därför har jag ej heller aktat mig själv värdig att komma till dig Men säg ett ord, så bliver min tjänare frisk.
Luk 7:8  Jag är ju själv en man som står under andras befäl; jag har ock krigsmän under mig, och om jag säger till en av dem: 'Gå', så går han, eller till en annan: 'Kom', så kommer han, och om jag säger till min tjänare: 'Gör det', då gör han så."
Luk 7:9  När Jesus hörde detta, förundrade han sig över honom och vände sig om och sade till folket som följde honom: "Jag säger eder: Icke ens i Israel har jag funnit så stor tro."
Luk 7:10  Och de som hade blivit utsända gingo hem igen och funno tjänaren vara frisk.
Luk 7:11  Därefter begav han sig till en stad som hette Nain; och med honom gingo hans lärjungar och mycket folk.
Luk 7:12  Och se, då han kom nära stadsporten, bars där ut en död, och han var sin moders ende son, och hon var änka; och en ganska stor hop folk ifrån staden gick med henne.
Luk 7:13  När Herren fick se henne, ömkade han sig över henne och sade till henne: "Gråt icke."
Luk 7:14  Och han gick fram och rörde vid båren, och de som buro stannade. Och han sade: "Unge man, jag säger dig: Stå upp."
Luk 7:15  Då satte sig den döde upp och begynte tala. Och han gav honom åt hans moder.
Luk 7:16  Och alla betogos av häpnad och prisade Gud och sade: Den stor profet har uppstått ibland oss" och: "Gud har sett till sitt folk."
Luk 7:17  Och detta tal om honom gick ut i hela Judeen och i hela landet däromkring.
Luk 7:18  Och allt detta fick Johannes höra berättas av sina lärjungar.
Luk 7:19  Då kallade Johannes till sig två av sina lärjungar och sände dem till Herren med denna fråga: "Är du den som skulle komma, eller skola vi förbida någon annan?"
Luk 7:20  När mannen kommo fram till honom, sade de: "Johannes döparen har sänt oss till dig och låter fråga: 'Är du den som skulle komma, eller skola vi förbida någon annan?'"
Luk 7:21  Just då höll Jesus på med att bota många som ledo av sjukdomar och plågor, eller som voro besatta av onda andar, och åt många blinda gav han deras syn.
Luk 7:22  Och han svarade och sade till männen: "Gån tillbaka och omtalen för Johannes vad I haven sett och hört: blinda få sin syn, halta gå, spetälska bliva rena, döva höra, döda uppstå, 'för fattiga förkunnas glädjens budskap'.
Luk 7:23  Och salig är den för vilken jag icke bliver en stötesten."
Luk 7:24  När sedan Johannes' sändebud hade gått sin väg, begynte han tala till folket om Johannes: "Varför var det I gingen ut i öknen? Var det för att se ett rör som drives hit och dit av vinden?
Luk 7:25  Eller varför gingen I ut? Var det för att se en människa klädd i fina kläder? De som bära präktiga kläder och leva i kräslighet, dem finnen I ju i konungapalatsen.
Luk 7:26  Varför gingen I då ut? Var det för att se en profet? Ja, jag säger eder: Ännu mer än en profet är han.
Luk 7:27  Han är den om vilken det är skrivet: 'Se, jag sänder ut min ängel framför dig, och han skall bereda vägen för dig.'
Luk 7:28  Jag säger eder: Bland dem som äro födda av kvinnor har ingen varit större än Johannes; men den som är minst i Guds rike är likväl större än han.
Luk 7:29  Så gav ock allt folket som hörde honom Gud rätt, jämväl publikanerna, och läto döpa sig med Johannes' dop.
Luk 7:30  Men fariséerna och de lagkloke föraktade Guds rådslut i fråga om dem själva och läto icke döpa sig av honom.
Luk 7:31  Vad skall jag då likna detta släktes människor vid? Ja, vad äro de lika?
Luk 7:32  De äro lika barn som sitta på torget och ropa till varandra och säga: 'Vi hava spelat för eder, och I haven icke dansat; vi hava sjungit sorgesång, och I haven icke gråtit.'
Luk 7:33  Ty Johannes döparen har kommit, och han äter icke bröd och dricker ej heller vin, och så sägen I: 'Han är besatt av en ond ande.'
Luk 7:34  Människosonen har kommit, och han både äter och dricker, och nu sägen I: 'Se, vilken frossare och vindrinkare han är, en publikaners och syndares vän!'
Luk 7:35  Men Visheten har fått rätt av alla sina barn."
Luk 7:36  Och en farisé inbjöd honom till en måltid hos sig; och han gick in i fariséens hus och lade sig till bords.
Luk 7:37  Nu fanns där i staden en synderska; och när denna fick veta att han låg till bords i fariséens hus, gick hon dit med en alabasterflaska med smörjelse
Luk 7:38  och stannade bakom honom vid hans fötter, gråtande, och begynte väta hans fötter med sina tårar och torkade dem med sitt huvudhår och kysste ivrigt hans fötter och smorde dem med smörjelsen.
Luk 7:39  Men när fariséen som hade inbjudit honom såg detta, sade han vid sig själv: "Vore denne en profet, så skulle han känna till, vilken och hurudan denna kvinna är, som rör vid honom; han skulle då veta att hon är en synderska."
Luk 7:40  Då tog Jesus till orda och sade till honom: "Simon, jag har något att säga dig." Han svarade: "Mästare, säg det.
Luk 7:41  "En man som lånade ut penningar hade två gäldenärer. Den ene var skyldig honom fem hundra silverpenningar, den andre femtio.
Luk 7:42  Men då de icke kunde betala, efterskänkte han skulden för dem båda. Vilken av dem kommer nu att älska honom mest?"
Luk 7:43  Simon svarade och sade: "Jag menar den åt vilken han efterskänkte mest." Då sade han till honom: "Rätt dömde du"
Luk 7:44  Och så vände han sig åt kvinnan och sade till Simon: "Ser du denna kvinna? När jag kom in i ditt hus, gav du mig intet vatten till mina fötter, men hon har vätt mina fötter med sina tårar och torkat dem med sitt hår.
Luk 7:45  Du gav mig ingen hälsningskyss, men ända ifrån den stund då jag kom hitin, har hon icke upphört att ivrigt kyssa mina fötter.
Luk 7:46  Du smorde icke mitt huvud med olja, men hon har smort mina fötter med smörjelse.
Luk 7:47  Fördenskull säger jag dig: Hennes många synder äro henne förlåtna; hon har ju ock visat mycken kärlek. Men den som får litet förlåtet, han älskar ock litet."
Luk 7:48  Sedan sade han till henne: "Dina synder äro dig förlåtna."
Luk 7:49  Då begynte de som voro bordsgäster jämte honom att säga vid sig själva: "Vem är denne, som till och med förlåter synder?"
Luk 7:50  Men han sade till kvinnan: "Din tro har frälst dig. Gå i frid."
Luk 8:1  Därefter vandrade han igenom landet, från stad till stad och från by till by, och predikade och förkunnade evangelium om Guds rike. Och med honom följde de tolv,
Luk 8:2  så ock några kvinnor som hade blivit befriade från onda andar och botade från sjukdomar: Maria, som kallades Magdalena, ur vilken sju onda andar hade blivit utdrivna,
Luk 8:3  och Johanna, hustru till Herodes' fogde Kusas, och Susanna och många andra som tjänade dem med sina ägodelar.
Luk 8:4  Då nu mycket folk kom tillhopa, i det att inbyggarna i de särskilda städerna begåvo sig ut till honom, sade han i en liknelse:
Luk 8:5  "En såningsman gick ut för att så sin säd. Och när han sådde, föll somt vid vägen och blev nedtrampat, och himmelens fåglar åto upp det.
Luk 8:6  Och somt föll på stengrund, och när det hade vuxit upp, torkade det bort, eftersom det icke där hade någon fuktighet.
Luk 8:7  Och somt föll bland törnen, och törnena växte upp tillsammans därmed och förkvävde det.
Luk 8:8  Men somt föll i god jord, och när det hade vuxit upp, bar det hundrafaldig frukt." Sedan han hade talat detta, sade han med hög röst: "Den som har öron till att höra, han höre."
Luk 8:9  Då frågade hans lärjungar honom vad denna liknelse betydde.
Luk 8:10  Han sade: "Eder är givet att lära känna Guds rikes hemligheter, men åt de andra meddelas de i liknelser, för att de med seende ögon intet skola se och med hörande öron intet förstå'.
Luk 8:11  Så är nu detta liknelsens mening: Säden är Guds ord.
Luk 8:12  Och att den såddes vid vägen, det är sagt om dem som hava hört ordet, men sedan kommer djävulen och tager bort det ur deras hjärtan, för att de icke skola komma till tro och bliva frälsta.
Luk 8:13  Och att den såddes på stengrunden det är sagt om dem, som när de få höra ordet, taga emot det med glädje, men icke hava någon rot; de tro allenast till en tid, och i frestelsens stund avfalla de.
Luk 8:14  Och att den föll bland törnena, det är sagt om dem, som när de hava hört ordet, gå bort och låta sig förkvävas av rikedomens omsorger och njutandet av livets goda och så icke föra något fram till mognad.
Luk 8:15  Men att den föll i den goda jorden, det är sagt om dem, som när de hava hört ordet, behålla det i rättsinniga och goda hjärtan och bära frukt i ståndaktighet.
Luk 8:16  Ingen tänder ett ljus och gömmer det sedan under ett kärl eller sätter det under en bänk, utan man sätter det på en ljusstake, för att de som komma in skola se skenet.
Luk 8:17  Ty intet är fördolt, som icke skall bliva uppenbart, ej heller är något undangömt, som icke skall bliva känt och komma i dagen.
Luk 8:18  Akten fördenskull på huru I hören. Ty den som har, åt honom skall varda givet; men den som icke har, från honom skall tagas också det han menar sig hava."
Luk 8:19  Och hans moder och hans bröder kommo och sökte honom, men för folkets skull kunde de icke komma in till honom.
Luk 8:20  Då sade man till honom: "Din moder och dina broder stå härutanför och vilja träffa dig."
Luk 8:21  Men han svarade och sade till dem: "Min moder och mina bröder äro dessa, som höra Guds ord och göra det."
Luk 8:22  En dag steg han med sina lärjungar i en båt och sade till dem: "Låt oss fara över sjön till andra sidan." Och de lade ut.
Luk 8:23  Och medan de seglade fram, somnade han. Men en stormvind for ned över sjön, och deras båt begynte fyllas med vatten, så att de voro i fara.
Luk 8:24  Då gingo de fram och väckte upp honom och sade: "Mästare, Mästare, vi förgås." När han så hade vaknat, näpste han vinden och vattnets vågor, och de stillades, och det blev lugnt.
Luk 8:25  Därefter sade han till dem: "Var är eder tro?" Men de hade blivit häpna och förundrade sig och sade till varandra: "Vem är då denne? Han befaller ju både vindarna och vattnet, och de lyda honom."
Luk 8:26  Så foro de över till gerasenernas land, som ligger mitt emot Galileen.
Luk 8:27  Och när han hade stigit i land, kom en man från staden emot honom, en som var besatt av onda andar, och som under ganska lång tid icke hade haft kläder på sig och icke bodde i hus, utan bland gravarna.
Luk 8:28  Då nu denne fick se Jesus, skriade han och föll ned för honom och sade med hög röst: "Vad har du med mig att göra, Jesus, du Guds, den Högstes, son? Jag beder dig, plåga mig icke."
Luk 8:29  Jesus skulle nämligen just bjuda den orene anden att fara ut ur mannen. Ty i lång tid hade han farit svårt fram med mannen; och väl hade denne varit fängslad med kedjor och fotbojor och hållits i förvar, men han hade slitit sönder bojorna och hade av den onde anden blivit driven ut i öknarna.
Luk 8:30  Jesus frågade honom: "Vad är ditt namn?" Han svarade: "Legion." Ty det var många onda andar som hade farit in i honom.
Luk 8:31  Och dessa bådo Jesus att han icke skulle befalla dem att fara ned i avgrunden.
Luk 8:32  Nu gick där en ganska stor svinhjord i bet på berget. Och de bådo honom att han ville tillstädja dem att fara in i svinen. Och han tillstadde dem det.
Luk 8:33  Då gåvo sig de onda andarna åstad ut ur mannen och foro in i svinen. Och hjorden störtade sig utför branten ned i sjön och drunknade.
Luk 8:34  Men när herdarna sågo vad som hade skett, flydde de och berättade härom i staden och på landsbygden.
Luk 8:35  Och folket gick ut för att se vad som hade skett. När de då kommo till Jesus, funno de mannen, ur vilken de onda andarna hade blivit utdrivna, sitta invid Jesu fötter, klädd och vid sina sinnen; och de betogos av häpnad.
Luk 8:36  Och de som hade sett händelsen omtalade för dem huru den besatte hade blivit botad.
Luk 8:37  Allt folket ifrån den kringliggande trakten av gerasenernas land bad då Jesus att han skulle gå bort ifrån dem, ty de voro gripna av stor förskräckelse. Så steg han då i en båt för att vända tillbaka.
Luk 8:38  Och mannen, ur vilken de onda andarna hade blivit utdrivna, bad honom att få följa honom. Men Jesus tillsade honom att gå, med de orden:
Luk 8:39  "Vänd tillbaka hem, och förtälj huru stora ting Gud har gjort med dig." Då gick han bort och förkunnade i hela staden huru stora ting Jesus hade gjort med honom.
Luk 8:40  När Jesus kom tillbaka, mottogs han av folket; ty alla väntade de på honom.
Luk 8:41  Då kom där en man, vid namn Jairus, som var föreståndare för synagogan. Denne föll ned för Jesu fötter och bad honom att han skulle komma till hans hus;
Luk 8:42  ty han hade ett enda barn, en dotter, vid pass tolv år gammal, som låg för döden. Men under det att han var på väg dit, trängde folket hårt på honom.
Luk 8:43  Nu var där en kvinna som hade haft blodgång i tolv år och icke hade kunnat botas av någon.
Luk 8:44  Hon närmade sig honom bakifrån och rörde vid hörntofsen på hans mantel, och strax stannade hennes blodgång
Luk 8:45  Men Jesus frågade: "Vem var det som rörde vid mig?" Då alla nekade till att hava gjort det, sade Petrus: "Mästare, hela folkhopen trycker och tränger dig ju."
Luk 8:46  Men Jesus sade: "Det var någon som rörde vid mig; ty jag kände att kraft gick ut ifrån mig."
Luk 8:47  Då nu kvinnan såg att hon icke hade blivit obemärkt, kom hon fram bävande och föll ned för honom och omtalade inför allt folket varför hon hade rört vid honom, och huru hon strax hade blivit frisk.
Luk 8:48  Då sade han till henne: "Min dotter, din tro har hjälpt dig. Gå i frid."
Luk 8:49  Medan han ännu talade, kom någon från synagogföreståndarens hus och sade: "Din dotter är död; du må icke vidare göra mästaren omak."
Luk 8:50  Men när Jesus hörde detta, sade han till honom: "Frukta icke; tro allenast, så får hon liv igen."
Luk 8:51  Och när han hade kommit fram till hans hus, tillstadde han ingen att gå med ditin, utom Petrus och Johannes och Jakob och därtill flickans fader och moder.
Luk 8:52  Och alla gräto och jämrade sig över henne. Men han sade: "Gråten icke; hon är icke död, hon sover."
Luk 8:53  Då hånlogo de åt honom, ty de visste ju att hon var död.
Luk 8:54  Men han tog henne vid handen och sade med hög röst: "Flicka, stå upp."
Luk 8:55  Då kom hennes ande igen, och hon stod strax upp. Och han tillsade att man skulle giva henne något att äta.
Luk 8:56  Och hennes föräldrar blevo uppfyllda av häpnad; men han förbjöd dem att för någon omtala vad som hade skett.
Luk 9:1  Och han kallade tillhopa de tolv och gav dem makt och myndighet över alla onda andar, så ock makt att bota sjukdomar.
Luk 9:2  Och han sände ut dem till att predika Guds rike och till att bota sjuka.
Luk 9:3  Och han sade till dem: "Tagen intet med eder på vägen, varken stav eller ränsel eller bröd eller penningar, och haven icke heller dubbla livklädnader.
Luk 9:4  Och när I haven kommit in något hus, så stannen där, till dess I lämnen den orten.
Luk 9:5  Och om man någonstädes icke tager emot eder, så gån bort ifrån den staden, och skudden stoftet av edra fötter, till ett vittnesbörd mot dem."
Luk 9:6  Och de gingo ut och vandrade igenom landet, från by till by, och förkunnade evangelium och botade sjuka allestädes.
Luk 9:7  Men när Herodes, landsfursten, fick höra om allt detta som skedde visste han icke vad han skulle tro. Ty somliga sade: "Det är Johannes, som har uppstått från de döda."
Luk 9:8  Men andra sade: "Det är Elias, som har visat sig." Andra åter sade: "Det är någon av de gamla profeterna, som har uppstått."
Luk 9:9  Men Herodes själv sade: "Johannes har jag låtit halshugga. Vem är då denne, som jag hör sådant om?" Och han sökte efter tillfälle att få se honom.
Luk 9:10  Och apostlarna kommo tillbaka och förtäljde för Jesus huru stora ting de hade gjort. De tog han dem med sig och drog sig undan till en stad som hette Betsaida, där de kunde vara allena.
Luk 9:11  Men när folket fick veta detta, gingo de efter honom. Och han lät dem komma till sig och talade till dem om Guds rike; och dem som behövde botas gjorde han friska.
Luk 9:12  Men dagen begynte nalkas sitt slut. Då trädde de tolv fram och sade till honom: "Låt folket skiljas åt, så att de kunna gå bort i byarna och gårdarna häromkring och skaffa sig härbärge och få mat; vi äro ju här i en öde trakt."
Luk 9:13  Men han sade till dem: "Given I dem att äta." De svarade: "Vi hava icke mer än fem bröd och två fiskar, såframt vi icke skola gå bort och köpa mat åt allt detta folk."
Luk 9:14  Där voro nämligen vid pass fem tusen män. Då sade han till sina lärjungar: "Låten dem lägga sig ned i matlag, femtio eller så omkring i vart."
Luk 9:15  Och de gjorde så och läto dem alla lägga sig ned.
Luk 9:16  Därefter tog han de fem bröden och de två fiskarna och säg upp till himmelen och välsignade dem. Och han bröt bröden och gav åt lärjungarna, för att de skulle lägga fram åt folket.
Luk 9:17  Och de åto alla och blevo mätta. sedan samlade man upp de stycken som hade blivit över efter dem, tolv korgar.
Luk 9:18  När han en gång hade dragit sig undan och var stadd i byn, voro hans lärjungar hos honom. Och han frågade dem och sade: "Vem säger folket mig vara?"
Luk 9:19  De svarade och sade: "Johannes döparen; dock säga andra Elias; andra åter säga: 'Det är någon av de gamla profeterna, som har uppstått.'"
Luk 9:20  Då frågade han dem: "Vem sägen då I mig vara?" Petrus svarade och sade: "Guds Smorde."
Luk 9:21  Då förbjöd han dem strängeligen att säga detta till någon.
Luk 9:22  Och han sade: "Människosonen måste lida mycket, och han skall bliva förkastad av de äldste och översteprästerna och de skriftlärda och skall bliva dödad, men på tredje dagen skall han uppstå igen."
Luk 9:23  Och han sade till alla: "Om någon vill efterfölja mig, så försake han sig själv och tage sitt kors på sig var dag; så följe han mig.
Luk 9:24  Ty den som vill bevara sitt liv han skall mista det; men den som mister sitt liv, för min skull, han skall bevara det.
Luk 9:25  Och vad hjälper det en människa om hon vinner hela världen, men mister sig själv eller själv går förlorad?
Luk 9:26  Den som blyges för mig och för mina ord, för honom skall Människosonen blygas, när han kom mer i sin och min Faders och de heliga änglarnas härlighet.
Luk 9:27  Men sannerligen säger jag eder: Bland dem som här stå finnas några som icke skola smaka döden, förrän de få se Guds rike."
Luk 9:28  Vid pass åtta dagar efter det att han hade talat detta tog han Petrus och Johannes och Jakob med sig och gick upp på berget för att bedja.
Luk 9:29  Och under det att han bad, blev hans ansikte förvandlat, och hans kläder blevo skinande vita.
Luk 9:30  Och de, två män stodo där och samtalade med honom, och dessa voro Moses och Elias.
Luk 9:31  De visade sig i härlighet och talade om hans bortgång, vilken han skulle fullborda i Jerusalem.
Luk 9:32  Men Petrus och de som voro med honom voro förtyngda av sömn; då de sedan vaknade, sågo de hans härlighet och de båda männen, som stodo hos honom.
Luk 9:33  När så dessa skulle skiljas ifrån honom, sade Petrus till Jesus: "Mästare, har är oss gott att vara; låt oss göra tre hyddor, en åt dig och en åt Moses och en åt Elias." Han visste nämligen icke vad han sade.
Luk 9:34  Medan han så talade, kom en sky och överskyggde dem; och de blevo förskräckta, när de trädde in i skyn.
Luk 9:35  Och ur skyn kom en röst som sade: "Denne är min Son, den utvalde; hören honom."
Luk 9:36  Och i detsamma som rösten kom, funno de Jesus vara där allena. - Och de förtego detta och omtalade icke för någon på den tiden något av vad de hade sett.
Luk 9:37  När de dagen därefter gingo ned från berget, hände sig att mycket folk kom honom till mötes.
Luk 9:38  Då ropade en man ur folkhopen och sade: "Mästare, jag beder dig, se till min son, ty han är mitt enda barn.
Luk 9:39  Det är så, att en ande plägar gripa fatt i honom, och strax skriar han då, och anden sliter och rycker honom, och fradgan står honom om munnen. Och det är med knapp nöd han släpper honom, sedan han har sönderbråkat honom.
Luk 9:40  Nu bad jag dina lärjungar att de skulle driva ut honom, men de kunde det icke."
Luk 9:41  Då svarade Jesus och sade: "O du otrogna och vrånga släkte, huru länge måste jag vara hos eder och härda ut med eder? För hit din son."
Luk 9:42  Men ännu medan denne var på väg fram, kastade den onde anden omkull honom och slet och ryckte honom. Då tilltalade Jesus den orene anden strängt och gjorde gossen frisk och gav honom tillbaka åt hans fader.
Luk 9:43  Och alla häpnade över Guds stora makt. Då nu alla förundrade sig över alla de gärningar som han gjorde, sade han till sina lärjungar:
Luk 9:44  "Tagen emot dessa ord med öppna öron: Människosonen skall bliva överlämnad i människors händer.
Luk 9:45  Men de förstodo icke detta som han sade, och det var förborgat för dem, så att de icke kunde fatta det; dock fruktade de att fråga honom om det som han hade sagt.
Luk 9:46  Och bland dem uppstod tanken på vilken av dem som vore störst.
Luk 9:47  Men Jesus förstod deras hjärtans tankar och tog ett barn och ställde det bredvid sig
Luk 9:48  och sade till dem: "Den som tager emot detta barn i mitt namn, han tager emot mig, och den som tager emot mig, han tager emot honom som har sänt mig. Ty den som är minst bland eder alla, han är störst.
Luk 9:49  Och Johannes tog till orda och sade: "Mästare, sågo huru en man drev ut onda andar genom ditt namn; och du ville hindra honom, eftersom han icke följde med oss."
Luk 9:50  Men Jesus sade till honom: "Hindren honom icke; ty den som icke är emot eder, han är för eder."
Luk 9:51  Då nu tiden var inne att han skulle bliva upptagen, beslöt han att ställa sin färd till Jerusalem.
Luk 9:52  Och han sände budbärare framför sig; och de gingo åstad och kommo in i en samaritisk by för att reda till åt honom.
Luk 9:53  Men folket där tog icke emot honom, eftersom han var stadd på färd till Jerusalem.
Luk 9:54  När de båda lärjungarna Jakob i och Johannes förnummo detta, sade de: "Herre, vill du att vi skola bedja att eld kommer ned från himmelen och förtär dem?"
Luk 9:55  Då vände han sig om och tillrättavisade dem. och sade: I veten icke vilken andes barn I ären.
Luk 9:56  Ty Människosonen har icke kommit för att fördärva själar, utan för att frälsa dem." Och de gingo till en annan by.
Luk 9:57  Medan de nu färdades fram på vägen, sade någon till honom: "Jag vill följa dig, varthelst du går.
Luk 9:58  Då svarade Jesus honom: "Rävarna hava kulor, och himmelens fåglar hava nästen; men Människosonen har ingen plats där han kan vila sitt huvud."
Luk 9:59  Och till en annan sade han: "Föl; mig." Men denne svarade: "Tillstäd mig att först gå bort och begrava min fader."
Luk 9:60  Då sade han till honom: "Låt de döda begrava sina döda; men gå du åstad och förkunna Guds rike."
Luk 9:61  Åter en annan sade: "Jag vill följa dig, Herre, men tillstäd mig att först taga avsked av dem som höra till mitt hus."
Luk 9:62  Då svarade Jesus honom: "Ingen som ser sig tillbaka, sedan han har satt sin hand till plogen, är skickad för Guds rike."
Luk 10:1  Därefter utsåg Herren sjuttiotvå andra och sände ut dem framför sig, två och två, till var stad och ort dit han själv tänkte komma
Luk 10:2  "Skörden är mycken, men arbetarna äro få. Bedjen fördenskull skördens Herre att han sänder ut arbetare till sin skörd.
Luk 10:3  Gån åstad. Se, jag sänder eder såsom lamm mitt in ibland ulvar.
Luk 10:4  Bären ingen penningpung, ingen ränsel, inga skor, och hälsen icke på någon under vägen.
Luk 10:5  Men när I kommen in i något hus, så sägen först: 'Frid vare över detta hus.'
Luk 10:6  Om då någon finnes därinne, som är frid värd, så skall den frid I tillönsken vila över honom; varom icke, så skall den vända tillbaka över eder själva.
Luk 10:7  Och stannen kvar i det huset, och äten och dricken vad de hava att giva, ty arbetaren är värd sin lön. Gån icke ur hus i hus.
Luk 10:8  Och när I kommen in i någon stad där man tager emot eder, så äten vad som sättes fram åt eder,
Luk 10:9  och boten de sjuka som finnas där, och sägen till dem: 'Guds rike är eder nära.'
Luk 10:10  Men när I kommen in i någon stad där man icke tager emot eder, så gån ut på dess gator och sägen:
Luk 10:11  'Till och med det stoft som låder vid våra fötter ifrån eder stad skaka vi av oss åt eder. Men det mån I veta, att Guds rike är nära.'
Luk 10:12  Jag säger eder att det för Sodom skall på 'den dagen' bliva drägligare än för den staden.
Luk 10:13  Ve dig, Korasin! Ve dig, Betsaida! Ty om de kraftgärningar som äro gjorda i eder hade blivit gjorda i Tyrus och Sidon, så skulle de för länge sedan hava suttit i säck och aska och gjort bättring.
Luk 10:14  Men också skall det vid domen bliva drägligare för Tyrus och Sidon än för eder.
Luk 10:15  Och du. Kapernaum, skall väl du bliva upphöjt till himmelen? Nej, ned till dödsriket måste du fara. -
Luk 10:16  Den som hör eder, han hör mig, och den som förkastar eder, han förkastar mig; men den som förkastar mig, han förkastar honom som har sänt mig."
Luk 10:17  Och de sjuttiotvå kommo tillbaka, uppfyllda av glädje, och sade: "Herre, också de onda andarna äro oss underdåniga genom ditt namn."
Luk 10:18  Då sade han till dem: "Jag såg Satan falla ned från himmelen såsom en ljungeld.
Luk 10:19  Se, jag har givit eder makt att trampa på ormar och skorpioner och att förtrampa all ovännens härsmakt, och han skall icke kunna göra eder någon skada.
Luk 10:20  Dock, glädjens icke över att änglarna äro eder underdåniga, utan glädjens över att edra namn äro skrivna i himmelen."
Luk 10:21  I samma stund uppfylldes han av fröjd genom den helige Ande och sade: "Jag prisar dig, Fader, du himmelens och jordens Herre, för att du väl har dolt detta för de visa och kloka, men uppenbarat det för de enfaldiga. Ja, Fader; så har ju varit ditt behag.
Luk 10:22  Allt har av min Fader blivit för trott åt mig. Och ingen känner vem Sonen är utom Fadern, ej heller vem Fadern är, utom Sonen och den för vilken Sonen vill göra honom känd."
Luk 10:23  Sedan vände han sig till lärjungarna, när han var allena med dem och sade: "Saliga äro de ögon som se det I sen.
Luk 10:24  Ty jag säger eder: Många profeter och konungar ville se det som I sen men fingo dock icke se det, och höra det som I hören, men fingo dock icke höra det."
Luk 10:25  Men en lagklok stod upp och ville snärja honom och sade: "Mästare, vad skall jag göra för att få evigt liv till arvedel?"
Luk 10:26  Då sade han till honom: "Vad är skrivet i lagen? Huru läser du?"
Luk 10:27  Han svarade och sade: "'Du skall älska Herren, din Gud, av allt ditt hjärta och av all din själ och av all din kraft och av allt ditt förstånd och din nästa såsom dig själv.'"
Luk 10:28  Han sade till honom: "Rätt svarade du. Gör det, så får du leva,
Luk 10:29  Då ville han rättfärdiga sig och sade till Jesus: "Vilken är då min nästa?"
Luk 10:30  Jesus svarade och sade: "En man begav sig från Jerusalem ned till Jeriko, men råkade ut för rövare, som togo ifrån honom hans kläder och därtill slogo honom; därefter gingo de sin väg och läto honom ligga där halvdöd.
Luk 10:31  Så hände sig att en präst färdades samma väg; och när han fick se honom, gick han förbi.
Luk 10:32  Likaledes ock en levit: när denne kom till det stället och fick se honom, gick han förbi.
Luk 10:33  Men en samarit, som färdades samma väg, kom också dit där han låg; och när denne fick se honom, ömkade han sig över honom
Luk 10:34  och gick fram till honom och göt olja och vin i hans sår och förband dem. Sedan lyfte han upp honom på sin åsna och förde honom till ett härbärge och skötte honom.
Luk 10:35  Morgonen därefter tog han fram två silverpenningar och gav dem åt värden och sade: 'Sköt honom och vad du mer kostar på honom skall jag betala dig, när jag kommer tillbaka.' -
Luk 10:36  Vilken av dessa tre synes dig nu hava visat sig vara den mannens nästa, som hade fallit i rövarhänder?"
Luk 10:37  Han svarade: "Den som bevisade honom barmhärtighet." Då sade Jesus till honom: "Gå du och gör sammalunda."
Luk 10:38  När de nu voro på vandring, gick han in i en by, och en kvinna, vid namn Marta, tog emot honom i sitt hus.
Luk 10:39  Och hon hade en syster, som hette Maria; denna satte sig ned vid Herrens fötter och hörde på hans ord.
Luk 10:40  Men Marta var upptagen av mångahanda bestyr för att tjäna honom. Och hon gick fram och sade: "Herre, frågar du icke efter att min syster har lämnat alla bestyr åt mig allena? Säg nu till henne att hon hjälper mig."
Luk 10:41  Då svarade Herren och sade till henne: "Marta, Marta, du gör dig bekymmer och oro för mångahanda,
Luk 10:42  men allenast ett är nödvändigt. Maria har utvalt den goda delen, och den skall icke tagas ifrån henne."
Luk 11:1  När han en gång uppehöll sig på ett ställe för att bedja och hade slutat sin bön, sade en av hans lärjungar till honom: "Herre, lär oss att bedja, såsom ock Johannes lärde sina lärjungar."
Luk 11:2  Då sade han till dem: "När I bedjen, skolen I säga så: 'Fader, helgat varde ditt namn; tillkomme ditt rike;
Luk 11:3  vårt dagliga bröd giv oss var dag;
Luk 11:4  och förlåt oss våra synder, ty också vi förlåta var och en som är oss något skyldig; och inled oss icke i frestelse.'"
Luk 11:5  Ytterligare sade han till dem: "Om någon av eder har en vän och mitt i natten kommer till denne och säger till honom: 'Käre vän, låna mig tre bröd;
Luk 11:6  ty en av mina vänner har kommit resande till mig, och jag har intet att sätta fram åt honom'
Luk 11:7  så svarar kanske den andre inifrån huset och säger: 'Gör mig icke omak; dörren är redan stängd, och både jag och mina barn hava gått till sängs; jag kan icke stå upp och göra dig något.'
Luk 11:8  Men jag säger eder: Om han än icke, av det skälet att han är hans vän, vill stå upp och giva honom något, så kommer han likväl, därför att den andre är så påträngande, att stå upp och giva honom så mycket han behöver.
Luk 11:9  Likaså säger jag till eder: Bedjen, och eder skall varda givet; söken, och I skolen finna; klappen, och för eder skall varda upplåtet.
Luk 11:10  Ty var och en som beder, han får; och den som söker, han finner; och för den som klappar skall varda upplåtet.
Luk 11:11  Finnes bland eder någon fader, som när hans son beder honom om en fisk, i stallet för en fisk räcker honom en orm,
Luk 11:12  eller som räcker honom en skorpion, när han beder om ett ägg?
Luk 11:13  Om nu I, som ären onda, förstån att giva edra barn goda gåvor, huru mycket mer skall icke då den himmelske Fadern giva helig ande åt dem som bedja honom!"
Luk 11:14  Och han drev ut en ond ande som var dövstum. Och när den onde anden hade blivit utdriven, talade den dövstumme; och folket förundrade sig.
Luk 11:15  Men några av dem sade: "Det är med Beelsebul, de onda andarnas furste, som han driver ut de onda andarna."
Luk 11:16  Och några andra ville sätta honom på prov och begärde av honom ett tecken från himmelen.
Luk 11:17  Men han förstod deras tankar och sade till dem: "Vart rike som har kommit i strid med sig självt bliver förött, så att hus faller på hus.
Luk 11:18  Om nu Satan har kommit i strid med sig själv, huru kan då hans rike hava bestånd? I sägen ju att det är med Beelsebul som jag driver ut de onda andarna.
Luk 11:19  Men om det är med Beelsebul som jag driver ut de onda andarna, med vem driva då edra egna anhängare ut dem? De skola alltså vara edra domare.
Luk 11:20  Om det åter är med Guds finger som jag driver ut de onda andarna, så har ju Guds rike kommit till eder. -
Luk 11:21  När en stark man, fullt väpnad, bevakar sin gård, då äro hans ägodelar fredade.
Luk 11:22  Men om någon som är starkare än han angriper honom och övervinner honom, så tager denne ifrån honom alla vapnen, som han förtröstade på, och skiftar ut bytet efter honom.
Luk 11:23  Den som icke är med mig, han är emot mig, och den som icke församlar med mig, han förskingrar.
Luk 11:24  När en oren ande har farit ut ur en människa, vandrar han omkring i ökentrakter och söker efter ro. Men då han icke finner någon, säger han: 'Jag vill vända tillbaka till mitt hus, som jag gick ut ifrån.'
Luk 11:25  Och när han kommer dit och finner det fejat och prytt,
Luk 11:26  då går han åstad och tager med sig sju andra andar, som äro värre än han själv, och de gå ditin och bo där; och så bliver för den människan det sista värre än det första."
Luk 11:27  När han sade detta, hov en kvinna in folkhopen upp sin röst och ropade till honom: "Saligt är det moderssköte som har burit dig, och de bröst som du har diat."
Luk 11:28  Men han svarade: "Ja, saliga är de som höra Guds ord och gömma det."
Luk 11:29  Men när folket strömmade till tog han till orda och sade: "Detta släkte är ett ont släkte Det begär ett tecken, men intet annat tecken skall givas det än Jonas' tecken.
Luk 11:30  Ty såsom Jonas blev ett tecken för nineviterna, så skall ock Människosonen vara ett tecken för detta släkte.
Luk 11:31  Drottningen av Söderlandet skall vid domen träda fram tillsammans med detta släktes män och bliva dem till dom. Ty hon kom från jordens ända för att höra Salomos visdom; och se, har är vad som är mer än Salomo.
Luk 11:32  Ninevitiska män skola vid domen träda fram tillsammans med detta släkte och bliva det till dom. Ty de gjorde bättring vid Jonas' predikan; och se, här är vad som är mer än Jonas.
Luk 11:33  Ingen tänder ett ljus och sätter det på en undangömd plats eller under skäppan, utan man sätter det på ljusstaken, för att de som komma in skola se skenet.
Luk 11:34  Ditt öga är kroppens lykta. När ditt öga är friskt, då har ock hela din kropp ljus; men när det är fördärvat, då är ock din kropp höljd i mörker.
Luk 11:35  Se därför till, att ljuset som du har i dig icke är mörker.
Luk 11:36  Om så hela din kropp får ljus och icke har någon del höljd i mörker, då har den ljus i sin helhet, såsom när en lykta lyser dig med sitt klara sken."
Luk 11:37  Under det att han så talade, inbjöd en farisé honom till måltid hos sig; och han gick ditin och tog plats vid bordet.
Luk 11:38  Men när fariséen såg att han icke tvådde sig före måltiden, förundrade han sig.
Luk 11:39  Då sade Herren till honom: "I fariséer, I gören nu det yttre av bägaren och fatet rent, medan edert inre är fullt av rofferi och ondska.
Luk 11:40  I dårar, har icke han som har gjort det yttre också gjort det inre?
Luk 11:41  Given fastmer såsom allmosa vad inuti kärlet är; först då bliver allt hos eder rent.
Luk 11:42  Men ve eder, I fariséer, som given tionde av mynta och ruta och alla slags kryddväxter, men icke akten på rätten och på kärleken till Gud! Det ena borden I göra, men icke underlåta det andra.
Luk 11:43  Ve eder, I fariséer, som gärna viljen sitta främst i synagogorna och gärna viljen bliva hälsade på torgen!
Luk 11:44  Ve eder, I som ären lika gravar som ingen kan märka, och över vilka människorna gå fram utan att veta det!"
Luk 11:45  Då tog en av de lagkloke till orda och sade till honom: "Mästare, när du så talar, skymfar du också oss."
Luk 11:46  Han svarade: "Ja, ve ock eder, I lagkloke, som på människorna läggen bördor, svåra att bära, men själva icke viljen med ett enda finger röra vid de bördorna!
Luk 11:47  Ve eder, I som byggen upp profeternas gravar, deras som edra fäder dräpte!
Luk 11:48  Så bären I då vittnesbörd om edra fäders gärningar och gillen dem; ty om de dräpte profeterna, så byggen I upp deras gravar.
Luk 11:49  Därför har ock Guds vishet sagt: 'Jag skall sända till dem profeter och apostlar, och somliga av dem skola de dräpa, och andra skola de förfölja.
Luk 11:50  Och så skall av detta släkte utkrävas alla profeters blod, allt det som är utgjutet från världens begynnelse,
Luk 11:51  ända ifrån Abels blod intill Sakarias' blod, hans som förgjordes mellan altaret och templet.' Ja, jag säger eder: Det skall utkrävas av detta släkte.
Luk 11:52  Ve eder, I lagkloke, som haven tagit bort nyckeln till kunskapen! Själva haven I icke kommit ditin och för dem som ville komma dit haven I lagt hinder."
Luk 11:53  När han inför allt folket sade detta till dem, blevo fariséerna och de lagkloke mycket förbittrade och gåvo sig i strid med honom om många stycken;
Luk 11:54  de sökte nämligen efter tillfälle att kunna anklaga honom.
Luk 12:1  Då nu otaligt mycket folk var församlat omkring honom, så att de trampade på varandra, tog han till orda och sade, närmast till sina lärjungar: "Tagen eder till vara för fariséernas surdeg, det är för skrymteri.
Luk 12:2  Intet är förborgat, som icke skall bliva uppenbarat, och intet är fördolt, som icke skall bliva känt.
Luk 12:3  Därför skall allt vad I haven sagt i mörkret bliva hört i ljuset, och vad I haven viskat i någons öra i kammaren, det skall bliva utropat på taken.
Luk 12:4  Men jag säger eder, mina vänner: Frukten icke för dem som väl kunna dräpa kroppen, men sedan icke hava makt att göra något mer.
Luk 12:5  Jag vill lära eder vem I skolen frukta: frukten honom som har makt att, sedan han har dräpt, också kasta i Gehenna. Ja, jag säger eder: Honom skolen I frukta. -
Luk 12:6  Säljas icke fem sparvar för två skärvar? Och icke en av dem är förgäten hos Gud.
Luk 12:7  Men på eder äro till och med huvudhåren allasammans räknade. Frukten icke; I ären mer värda än många sparvar.
Luk 12:8  Och jag säger eder: Var och en som bekänner mig inför människorna, honom skall ock Människosonen kännas vid inför Guds änglar.
Luk 12:9  Men den som förnekar mig inför människorna, han skall ock bliva förnekad inför Guds änglar.
Luk 12:10  Och om någon säger något mot Människosonen, så skall det bliva honom förlåtet; men den som hädar den helige Ande, honom skall det icke bliva förlåtet.
Luk 12:11  Men när man drager eder fram inför synagogor och överheter och myndigheter, så gören eder icke bekymmer för huru eller varmed I skolen försvara eder, eller vad I skolen säga;
Luk 12:12  ty den helige Ande skall i samma stund lära eder vad I skolen säga."
Luk 12:13  Och en man i folkhopen sade till honom: "Mästare, säg till min broder att han skiftar arvet med mig."
Luk 12:14  Men han svarade honom: "Min vän, vem har satt mig till domare eller skiftesman över eder?"
Luk 12:15  Därefter sade han till dem: "Sen till, att I tagen eder till vara för allt slags girighet; ty en människas liv beror icke därpå att hon har överflöd på ägodelar."
Luk 12:16  Och han framställde för dem en liknelse; han sade: "Det var en rik man vilkens åkrar buro ymniga skördar.
Luk 12:17  Och han tänkte vid sig själv och sade: 'Vad skall jag göra? Jag har ju icke rum nog för att inbärga min skörd.'
Luk 12:18  Därefter sade han: 'Så vill jag göra: jag vill riva ned mina lador och bygga upp större, och i dem skall jag samla in all min gröda och allt mitt goda.
Luk 12:19  Sedan vill jag säga till min själ: Kära själ, du har mycket gott för varat för många år; giv dig nu ro, ät, drick och var glad.
Luk 12:20  Men Gud sade till honom: 'Du dåre, i denna natt skall din själ utkrävas av dig; vem skall då få vad du har samlat i förråd?' -
Luk 12:21  Så går det den som samlar skatter åt sig själv, men icke är rik inför Gud."
Luk 12:22  Och han sade till sina lärjungar: "Därför säger jag eder: Gören eder icke bekymmer för edert liv, vad I skolen äta, ej heller för eder kropp, vad I skolen kläda eder med.
Luk 12:23  Livet är ju mer än maten, och kroppen mer än kläderna.
Luk 12:24  Given akt på korparna: de så icke, ej heller skörda de, och de hava varken visthus eller lada; och likväl föder Gud dem. Huru mycket mer ären icke I än fåglarna!
Luk 12:25  Vilken av eder kan med allt sitt bekymmer lägga en aln till sin livslängd?
Luk 12:26  Förmån I nu icke ens det som minst är, varför gören I eder då bekymmer för det övriga?
Luk 12:27  Given akt på liljorna, huru de varken spinna eller väva; och likväl säger jag eder att icke ens Salomo i all sin härlighet var så klädd som en av dem.
Luk 12:28  Kläder nu Gud så gräset på marken, vilket i dag står och i morgon kastas i ugnen, huru mycket mer skall han då icke kläda eder, I klentrogne!
Luk 12:29  Söken därför icke heller I efter vad I skolen äta, eller vad I skolen dricka, och begären icke vad som är för högt.
Luk 12:30  Efter allt detta söka ju hedningarna i världen, och eder Fader vet att I behöven detta.
Luk 12:31  Nej, söken efter hans rike, så skall också detta andra tillfalla eder.
Luk 12:32  Frukta icke, du lilla hjord; ty det har behagat eder Fader att giva eder riket.
Luk 12:33  Säljen vad I ägen och given allmosor; skaffen eder penningpungar som icke nötas ut, en outtömlig skatt i himmelen, dit ingen tjuv når, och där man icke fördärvar.
Luk 12:34  Ty där eder skatt är, där komma ock edra hjärtan att vara.
Luk 12:35  Haven edra länder omgjordade och edra lampor brinnande.
Luk 12:36  Och varen I lika tjänare som vänta på att deras herre skall bryta upp från bröllopet, för att strax kunna öppna för honom, när han kommer och klappar.
Luk 12:37  Saliga äro de tjänare som deras herre finner vakande, när han kommer. Sannerligen säger jag eder: Han skall fästa upp sin klädnad och låta dem taga plats vid bordet och själv gå fram och betjäna dem.
Luk 12:38  Och vare sig han kommer under den andra nattväkten eller under den tredje och finner dem så göra - saliga äro de då.
Luk 12:39  Men det förstån I väl, att om husbonden visste vilken stund tjuven skulle komma, så tillstadde han icke att någon bröt sig in i hans hus.
Luk 12:40  Så varen ock I redo ty i en stund då I icke vänten det skall Människosonen komma."
Luk 12:41  Då sade Petrus: "Herre, är det om oss som du talar i denna liknelse, eller är det om alla?"
Luk 12:42  Herren svarade: "Finnes någon trogen och förståndig förvaltare, som av sin herre kan sättas över hans husfolk, för att i rätt tid giva dem deras bestämda kost -
Luk 12:43  salig är då den tjänaren, om hans herre, när han kommer, finner honom göra så.
Luk 12:44  Sannerligen säger jag eder: Han skall sätta honom över allt vad han äger.
Luk 12:45  Men om så är, att tjänaren säger i sitt hjärta: 'Min herre kommer icke så snart', och han begynner att slå de andra tjänarna och tjänarinnorna och att äta och dricka så att han bliver drucken,
Luk 12:46  då skall den tjänarens herre komma på en dag då han icke väntar det, och i en stund då han icke tänker sig det, och han skall låta hugga honom i stycken och låta honom få sin del med de otrogna. -
Luk 12:47  Och den tjänare som hade fått veta sin herres vilja, men icke redde till eller gjorde efter hans vilja, han skall bliva straffad med många slag.
Luk 12:48  Men den som, utan att hava fått veta hans vilja, gjorde vad som val slag värt, han skall bliva straffad med allenast få slag. Var och en åt vilken mycket är givet, av honom skall mycket varda utkrävt och den som har blivit betrodd med mycket, av honom skall man fordra dess mera.
Luk 12:49  Jag har kommit för att tända en eld på jorden; och huru gärna ville jag icke att den redan brunne!
Luk 12:50  Men jag måste genomgå ett dop; och huru ängslas jag icke, till dess att det är fullbordat!
Luk 12:51  Menen I att jag har kommit för att skaffa frid på jorden? Nej, säger jag eder, fastmer söndring.
Luk 12:52  Ty om fem finnas i samma hus, skola de härefter vara söndrade från varandra, så att tre stå mot två Och två mot tre:
Luk 12:53  fadern mot sin son och sonen mot sin fader, modern mot sin dotter och dottern mot sin moder, svärmodern mot sin sonhustru och sonhustrun mot sin svärmoder.
Luk 12:54  Han hade också till folket: "När I sen ett moln stiga upp i väster, sägen I strax: 'Nu kommer regn'; och det sker så.
Luk 12:55  Och när I sen sunnanvind blåsa, sägen I: 'Nu kommer stark hetta'; och detta sker.
Luk 12:56  I skrymtare, jordens och himmelens utseende förstån I att tyda; huru kommer det då till, att I icke kunnen tyda denna tiden?
Luk 12:57  Varför låten I icke edert eget inre döma om vad rätt är?
Luk 12:58  När du går till en överhetsperson med din motpart, så gör dig under vägen all möda att bliva förlikt med denne, så att han icke drager dig fram inför domaren; då händer att domaren överlämnar dig åt rättstjänaren, och att rättstjänaren kastar dig i fängelse.
Luk 12:59  Jag säger dig: Du skall icke slippa ut därifrån, förrän du har betalt ända till den yttersta skärven."
Luk 13:1  Vid samma tid kommo några och berättade för honom om de galiléer vilkas blod Pilatus hade blandat med deras offer.
Luk 13:2  Då svarade han och sade till dem: "Menen I att dessa galiléer voro större syndare än alla andra galiléer, eftersom de fingo lida sådant?
Luk 13:3  Nej, säger jag eder; men om I icke gören bättring, skolen I alla sammalunda förgås.
Luk 13:4  Eller de aderton som dödades, när tornet i Siloam föll på dem, menen I att de voro mer brottsliga än alla andra människor som bo i Jerusalem?
Luk 13:5  Nej, säger jag eder; men om I icke gören bättring, skolen I alla sammalunda förgås."
Luk 13:6  Och han framställde denna liknelse: "En man hade ett fikonträd planterat i sin vingård; och han kom och sökte frukt därpå, men fann ingen.
Luk 13:7  Då sade han till vingårdsmannen: 'Se, nu i tre år har jag kommit och sökt frukt på detta fikonträd, utan att finna någon. Hugg bort det. Varför skall det därjämte få utsuga jorden?'
Luk 13:8  Men vingårdsmannen svarade och sade till honom: 'Herre, låt det stå kvar också detta år, för att jag under tiden må gräva omkring det och göda det;
Luk 13:9  kanhända skall det så till nästa å bära frukt. Varom icke, så må du då hugga bort det.'"
Luk 13:10  När han en gång på sabbaten undervisade i en synagoga,
Luk 13:11  var där en kvinna som i aderton år hade varit besatt av en sjukdomsande, och hon var så hopkrumpen, att hon alls icke kunde räta upp sin kropp.
Luk 13:12  Då nu Jesus fick se henne, kallade han henne till sig och sade till henne "Kvinna, du är fri ifrån din sjuk dom",
Luk 13:13  och han lade därvid händerna på henne. Och strax rätade hon upp sig och prisade Gud.
Luk 13:14  Men det förtröt synagogföreståndaren att Jesus på sabbaten botade sjuka; och han tog till orda och sade till folket: "Sex dagar finnas, på vilka man bör arbeta. På dem mån I alltså komma och låta bota eder, men icke på sabbatsdagen.
Luk 13:15  Då svarade Herren honom och sade: "I skrymtare, löser icke var och en av eder på sabbaten sin oxe eller åsna från krubban och leder den bort för att vattna den?
Luk 13:16  Och denna kvinna, en Abrahams dotter, som Satan har hållit bunden nu i aderton år, skulle då icke hon på sabbatsdagen få lösas från sin boja?"
Luk 13:17  När han sade detta, blygdes alla hans motståndare; och allt folket gladde sig över alla de härliga gärningar som gjordes av honom.
Luk 13:18  Så sade han då: "Vad är Guds rike likt, ja, vad skall jag likna det vid?
Luk 13:19  Det är likt ett senapskorn som en man tager och lägger ned i sin trädgård, och det växer och bliver ett träd, och himmelens fåglar bygga sina nästen på dess grenar."
Luk 13:20  Ytterligare sade han: "Vad skall jag likna Guds rike vid?
Luk 13:21  Det är likt en surdeg som en kvinna tager och blandar in i tre skäppor mjöl, till dess alltsammans bliver syrat."
Luk 13:22  Och han vandrade från stad till stad och från by till by och undervisade folket, under det att han fortsatte sin färd till Jerusalem.
Luk 13:23  Och någon frågade honom: "Herre, är det allenast få som bliva frälsta?" Då svarade han dem:
Luk 13:24  "Kämpen för att komma in genom den trånga dörren; ty många, säger jag eder, skola försöka att komma in och skola dock icke förmå det.
Luk 13:25  Om husbonden har stått upp och tillslutit dörren, och I sedan kommen och ställen eder därutanför och klappen på dörren och sägen: 'Herre, låt upp för oss', så skall han svara och säga till eder: 'Jag vet icke varifrån I ären.'
Luk 13:26  Och I skolen då säga: 'Vi hava ju ätit och druckit med dig, och du har undervisat på våra gator.'
Luk 13:27  Men han skall svara: 'Jag säger eder: Jag vet icke varifrån I ären; gån bort ifrån mig, alla I ogärningsmän.'
Luk 13:28  Där skall då bliva gråt och tandagnisslan, när I fån se Abraham, Isak och Jakob och alla profeterna vara i Guds rike, men finnen eder själva utkastade.
Luk 13:29  Ja, människor skola komma från öster och väster, från norr och söder och bliva bordsgäster i Guds rike.
Luk 13:30  Och se, då skola somliga som äro de sista bliva de första, och somliga som äro de första bliva de sista.
Luk 13:31  I samma stund kommo några fariséer fram och sade till honom: "Begiv dig åstad bort härifrån; ty Herodes vill dräpa dig."
Luk 13:32  Då svarade han dem: "Gån och sägen den räven, att jag i dag och i morgon driver ut onda andar och botar sjuka, och att jag först på tredje dagen är färdig.
Luk 13:33  Men jag måste vandra i dag och i morgon och i övermorgon, ty det är icke i sin ordning att en profet förgöres annorstädes än i Jerusalem. -
Luk 13:34  Jerusalem, Jerusalem, du som dräper profeterna och stenar dem som äro sända till dig! Huru ofta har jag icke velat församla dina barn, likasom hönan församlar sina kycklingar under sina vingar! Men haven icke velat.
Luk 13:35  Se, edert hus skall komma att stå övergivet. Men jag säger eder: I skolen icke se mig, förrän den tid kommen, då I sägen: 'Välsignad vare han som kommer, i Herrens namn.'
Luk 14:1  När han på en sabbat hade kommit in till en av de förnämligaste fariséerna för att intaga en måltid, hände sig, medan man där vaktade på honom,
Luk 14:2  att en vattusiktig man kom dit och stod framför honom.
Luk 14:3  Då tog Jesus till orda och sade till de lagkloke och fariséerna: "Är det lovligt att bota sjuka på sabbaten, eller är det icke lovligt?"
Luk 14:4  Men de tego. Då tog han mannen vid handen och gjorde honom frisk och lät honom gå.
Luk 14:5  Sedan sade han till dem: "Om någon av eder har en åsna eller en oxe som faller i en brunn, går han icke då strax och drager upp den, jämväl på sabbatsdagen?"
Luk 14:6  Och de förmådde icke svara något härpå.
Luk 14:7  Då han nu märkte huru gästerna utvalde åt sig de främsta platserna, framställde han för dem en liknelse; han sade till dem:
Luk 14:8  "När du av någon har blivit bjuden till bröllop, så tag icke den främsta platsen vid bordet. Ty kanhända finnes bland gästerna någon som är mer ansedd än du,
Luk 14:9  och då kommer till äventyrs den som har bjudit både dig och honom och säger till dig: 'Giv plats åt denne'; och så måste du med skam intaga den nedersta platsen.
Luk 14:10  Nej, när du har blivit bjuden, så gå och tag den nedersta platsen vid bordet. Ty det kan då hända att den som har bjudit dig säger till dig, när han kommer: 'Min vän, stig högre upp.' Då vederfares dig heder inför alla de andra bordsgästerna.
Luk 14:11  Ty var och en som upphöjer sig, han skall bliva förödmjukad, och den som ödmjukar sig, han skall bliva upphöjd."
Luk 14:12  Han sade ock till den som hade bjudit honom: "När du gör ett gästabud, på middagen eller på aftonen, så inbjud icke dina vänner eller dina bröder eller dina fränder, ej heller rika grannar, så att de bjuda dig igen och du därigenom får vedergällning.
Luk 14:13  Nej, när du gör gästabud, så bjud fattiga, krymplingar, halta, blinda.
Luk 14:14  Salig är du då; ty eftersom de icke förmå vedergälla dig, skall du få din vedergällning vid de rättfärdigas uppståndelse."
Luk 14:15  Då nu en av de andra vid bordet hörde detta, sade denne till honom: "Salig är den som får bliva bordsgäst i Guds rike!"
Luk 14:16  Då sade han till honom: "En man tillredde ett stort gästabud och bjöd många.
Luk 14:17  Och när gästabudet skulle hållas, sände han ut sin tjänare och lät säga till dem som voro bjudna: 'Kommen, ty nu är allt redo.'
Luk 14:18  Men de begynte alla strax ursäkta sig. Den förste sade till honom: 'Jag har köpt ett jordagods, och jag måste gå ut och bese det. Jag beder dig, tag emot min ursäkt.'
Luk 14:19  En annan sade: 'Jag har köpt fem par oxar, och jag skall nu gå åstad och försöka dem. Jag beder dig, tag emot min ursäkt.'
Luk 14:20  Åter en annan sade: 'Jag har tagit mig hustru, och därför kan jag icke komma.'
Luk 14:21  Och tjänaren kom igen och omtalade detta för sin herre. Då blev husbonden vred och sade till sin tjänare: 'Gå strax ut på gator och gränder i staden, och för hitin fattiga och krymplingar och blinda och halta.'
Luk 14:22  Sedan sade tjänaren: 'Herre, vad du befallde har blivit gjort, men har är ännu rum.'
Luk 14:23  Då sade herren till tjänaren: 'Gå ut på vägar och stigar, och nödga människorna att komma hitin, så att mitt hus bliver fullt.
Luk 14:24  Ty jag säger eder att ingen av de män som voro bjudna skall smaka sin måltid.'"
Luk 14:25  Och mycket folk gick med honom; och han vände sig om och sade till dem:
Luk 14:26  "Om någon kommer till mig, och han därvid ej hatar sin fader och sin moder, och sin hustru och sina barn, och sina bröder och systrar, därtill ock sitt eget liv, så kan han icke vara min lärjunge.
Luk 14:27  Den som icke bär sitt kors och efterföljer mig, han kan icke vara min lärjunge.
Luk 14:28  Ty om någon bland eder vill bygga ett torn, sätter han sig icke då först ned och beräknar kostnaden och mer till, om han äger vad som behöves för att bygga det färdigt?
Luk 14:29  Eljest, om han lade grunden, men icke förmådde fullborda verket skulle ju alla som finge se det begynna att begabba honom
Luk 14:30  och säga: 'Den mannen begynte bygga, men förmådde icke fullborda sitt verk.'
Luk 14:31  Eller om en konung vill draga ut i krig för att drabba samman med en annan konung, sätter han sig icke då först ned och tänker efter, om han förmår att med tio tusen möta den som kommer emot honom med tjugu tusen?
Luk 14:32  Om så icke är, måste han ju, medan den andre ännu är långt borta, skicka sändebud och underhandla om fred.
Luk 14:33  Likaså kan ingen av eder vara min lärjunge, om han icke försakar allt vad han äger. -
Luk 14:34  Så är väl saltet en god sak, men om till och med saltet mister sin sälta, varmed skall man då återställa dess kraft?
Luk 14:35  Varken för jorden eller för gödselhögen är det tjänligt; man kastar ut det. Den som har öron till att höra, han höre."
Luk 15:1  Och till honom kom allt vad publikaner och syndare hette för att höra honom.
Luk 15:2  Men fariséerna och de skriftlärde knorrade och sade: "Denne tager emot syndare och äter med dem."
Luk 15:3  Då framställde han för dem denna liknelse; han sade:
Luk 15:4  "Om ibland eder finnes en man som har hundra får, och han förlorar ett av dem, lämnar han icke då de nittionio i öknen och går och söker efter det förlorade, till dess han finner det?
Luk 15:5  Och när han har funnit det, lägger han det på sina axlar med glädje.
Luk 15:6  Och när han kommer hem, kallar han tillhopa sina vänner och grannar och säger till dem: 'Glädjens med mig, ty jag har funnit mitt får, som var förlorat.'
Luk 15:7  Jag säger eder att likaså bliver mer glädje i himmelen över en enda syndare som gör bättring, än över nittionio rättfärdiga som ingen bättring behöva.
Luk 15:8  Eller om en kvinna har tio silverpenningar, och hon tappar bort en av dem, tänder hon icke då upp ljus och sopar huset och söker noga, till dess hon finner den?
Luk 15:9  Och när hon har funnit den, kallar hon tillhopa sina väninnor och grannkvinnor och säger: 'Glädjens med mig, ty jag har funnit den penning som jag hade tappat bort.'
Luk 15:10  Likaså, säger jag eder, bliver glädje hos Guds änglar över en enda syndare som gör bättring.
Luk 15:11  Ytterligare sade han: "En man hade två söner.
Luk 15:12  Och den yngre av dem sade till fadern: 'Fader, giv mig den del av förmögenheten, som faller på min lott.' Då skiftade han sina ägodelar mellan dem.
Luk 15:13  Och icke lång tid därefter lade den yngre sonen allt sitt tillhopa och for långt bort till ett främmande land. Där levde han i utsvävningar och förfor så sin förmögenhet.
Luk 15:14  Men sedan han hade slösat bort allt, kom en svär hungersnöd över det landet, och han begynte lida nöd.
Luk 15:15  Då gick han bort och gav sig under en man där i landet, och denne sände honom ut på sina marker för att vakta svin.
Luk 15:16  Och han åstundade att få fylla sin buk med de fröskidor som svinen åto; men ingen gav honom något.
Luk 15:17  Då kom han till besinning och sade: 'Huru många legodrängar hos min fader hava icke bröd i överflöd, medan jag har förgås av hunger!
Luk 15:18  Jag vill stå upp och gå till min fader och säga till honom: Fader, jag har syndat mot himmelen och inför dig;
Luk 15:19  jag är icke mer värd att kallas din son. Låt mig bliva såsom en av dina legodrängar.'
Luk 15:20  Så stod han upp och gick till sin fader. Och medan han ännu var långt borta, fick hans fader se honom och ömkade sig över honom och skyndade emot honom och föll honom om halsen och kysste honom innerligt.
Luk 15:21  Men sonen sade till honom: 'Fader, jag har syndat mot himmelen och inför dig; jag är icke mer värd att kallas din son.'
Luk 15:22  Då sade fadern till sina tjänare 'Skynden eder att taga fram den yppersta klädnaden och kläden honom i den, och sätten en ring på hans hand och skor på hans fötter.
Luk 15:23  Och hämten den gödda kalven och slakten den, så vilja vi äta och gör oss glada.
Luk 15:24  Ty denne min son var död, men har fått liv igen; han var förlorad men är återfunnen.' Och de begynte göra sig glada.
Luk 15:25  Men hans äldre son var ute på marken. När denne nu vände tillbaka och hade kommit nära huset fick han höra spel och dans.
Luk 15:26  Då kallade han till sig en av tjänarna och frågade vad detta kunde betyda.
Luk 15:27  Denne svarade honom: 'Din broder har kommit hem; och då nu din fader har fått honom välbehållen tillbaka, har han låtit slakta den gödda kalven.'
Luk 15:28  Då blev han vred och ville icke gå in. Hans fader gick då ut och talade vanligt med honom.
Luk 15:29  Men han svarade och sade till sin fader: 'Se, i så många år har jag nu tjänat dig, och aldrig har jag överträtt något ditt bud; och lik väl har du åt mig aldrig givit ens en killing, för att jag skulle kunna göra mig glad med mina vänner.
Luk 15:30  Men när denne din son, som har förtärt dina ägodelar tillsammans med skökor, nu har kommit tillbaka, så har du för honom låtit slakta den gödda kalven.'
Luk 15:31  Då sade han till honom: 'Min son, du är alltid hos mig, och all mitt är ditt.
Luk 15:32  Men nu måste vi fröjda oss och vara glada; ty denne din broder var död, men har fått liv igen, han var förlorad, men är återfunnen.'"
Luk 16:1  Han sade också till sina lärjungar: "En rik man hade en förvaltare som hos honom blev angiven för förskingring av hans ägodelar.
Luk 16:2  Då kallade han honom till sig och sade till honom: 'Vad är det jag hör om dig? Gör räkenskap för din förvaltning; ty du kan icke längre få vara förvaltare.'
Luk 16:3  Men förvaltaren sade vid sig själv: 'Vad skall jag göra, då min herre nu tager ifrån mig förvaltningen? Gräva orkar jag icke; att tigga blyges jag för.
Luk 16:4  Dock, nu vet jag vad jag skall göra, för att man må upptaga mig i sina hus, när jag bliver avsatt ifrån förvaltningen.'
Luk 16:5  Och han kallade till sig sin herres gäldenärer, var och en särskilt. Och han frågade den förste: 'Huru mycket är du skyldig min herre?'
Luk 16:6  Han svarade: 'Hundra fat olja.' Då sade han till honom: 'Tag här ditt skuldebrev, och sätt dig nu strax ned och skriv femtio.'
Luk 16:7  Sedan frågade han en annan: 'och du, huru mycket är du skyldig?' Denne svarade: 'hundra tunnor vete.' Då sade han till honom: 'Tag här ditt skuldebrev och skriv åttio.'
Luk 16:8  Och husbonden prisade den orättrådige förvaltaren för det att han hade handlat klokt. Ty denna tidsålders barn skicka sig klokare mot sitt släkte än ljusets barn.
Luk 16:9  Och jag säger eder: Skaffen eder vänner medelst den orättrådige Mamons goda, för att de, när detta har tagit slut, må taga emot eder i de eviga hyddorna.
Luk 16:10  Den som är trogen i det minsta, han är ock trogen i vad mer är, och den som är orättrådig i det minsta, han är ock orättrådig i vad mer är.
Luk 16:11  Haven I nu icke varit trogna, när det gällde den orättrådige Mamons goda, vem vill då betro eder det sannskyldiga goda?
Luk 16:12  Och haven I icke varit trogna, när det gällde vad som tillhörde en annan, vem vill då giva eder det som hör eder till?
Luk 16:13  Ingen som tjänar kan tjäna två herrar; ty antingen kommer han då att hata den ene och älska den andre, eller kommer han att hålla sig till den förre och förakta den senare. I kunnen icke tjäna både Gud och Mamon."
Luk 16:14  Allt detta hörde nu fariséerna, som voro penningkära, och de drevo då gäck med honom.
Luk 16:15  Men han sade till dem: "I hören till dem som göra sig rättfärdiga inför människorna. Men Gud känner edra hjärtan; ty det som bland människor är högt är en styggelse inför Gud.
Luk 16:16  Lagen och profeterna hava haft sin tid intill Johannes. Sedan dess förkunnas evangelium om Guds rike, och var man vill storma ditin.
Luk 16:17  Men snarare kunna himmel och jord förgås, än en enda prick av lagen kan falla bort.
Luk 16:18  Var och en som skiljer sig från sin hustru och tager sig en annan hustru, han begår äktenskapsbrott. Och den som tager till hustru en kvinna som är skild från sin man, han begår äktenskapsbrott.
Luk 16:19  Det var en rik man som klädde sig i purpur och fint linne och levde var dag i glädje och prakt.
Luk 16:20  Men en fattig man, vid namn Lasarus, låg vid hans port, full av sår,
Luk 16:21  och åstundade att få stilla sin hunger med vad som kunde falla ifrån den rike mannens bord. Ja, det gick så långt att hundarna kommo och slickade hans sår.
Luk 16:22  Så hände sig att den fattige dog och blev förd av änglarna till Abrahams sköte. Också den rike dog och blev begraven.
Luk 16:23  När han nu låg i dödsriket och plågades, lyfte han upp sina ögon och fick se Abraham långt borta och Lasarus i hans sköte.
Luk 16:24  Då ropade han och sade: 'Fader Abraham, förbarma dig över mig, och sänd Lasarus att doppa det yttersta av sitt finger i vatten och svalka min tunga, ty jag pinas svårt i dessa lågor.'
Luk 16:25  Men Abraham svarade: 'Min son kom ihåg att du, medan du levde, fick ut ditt goda och Lasarus däremot vad ont var; nu åter får han här hugnad, under det att du pinas.
Luk 16:26  Och till allt detta kommer, att ett stort svalg är satt mellan oss och eder, för att de som vilja begiva sig över härifrån till eder icke skola kunna det, och för att ej heller någon därifrån skall kunna komma över till oss."
Luk 16:27  Då sade han: 'Så beder jag dig då, fader, att du sänder honom till min faders hus,
Luk 16:28  där jag har fem bröder, och låter honom varna dem, så att icke också de komma till detta pinorum.'
Luk 16:29  Men Abraham sade: 'De hava Moses och profeterna; dem må de lyssna till.'
Luk 16:30  Han svarade: 'Nej, fader Abraham; men om någon kommer till dem från de döda, så skola de göra bättring.'
Luk 16:31  Då sade han till honom: 'Lyssna de icke till Moses och profeterna, så skola de icke heller låta övertyga sig, om någon uppstår från de döda.'"
Luk 17:1  Och han sade till sina lärjungar: "Det är icke annat möjligt än att förförelser måste komma, men ve den genom vilken de komma!
Luk 17:2  För honom vore det bättre att en kvarnsten hängdes om hans hals och han kastades i havet, än att han skulle förföra en av dessa små.
Luk 17:3  Tagen eder till vara! Om din broder försyndat sig, så tillrättavisa honom; och om han då ångrar sig, så förlåt honom.
Luk 17:4  Ja, om han sju gånger om dagen försyndar sig mot dig och sju gånger kommer tillbaka till dig och säger: 'Jag ångrar mig' så skall du förlåta honom."
Luk 17:5  Och apostlarna sade till Herren: "Föröka vår tro."
Luk 17:6  Då sade Herren: "Om I haden tro, vore den ock blott såsom ett senapskorn, så skullen I kunna säga till detta mullbärsfikonträd: 'Ryck dig upp med rötterna, och plantera dig i havet', och det skulle lyda eder.
Luk 17:7  Om någon bland eder har en tjänare som kör plogen eller vaktar boskap, icke säger han väl till tjänaren, när denne kommer hem från marken: 'Gå du nu strax till bords'?
Luk 17:8  Säger han icke fastmer till honom: 'Red till min måltid, och fäst så upp din klädnad och betjäna mig, medan jag äter och dricker; sedan må du själv äta och dricka'?
Luk 17:9  Icke tackar han väl tjänaren för att denne gjorde det som blev honom befallt?
Luk 17:10  Sammalunda, när I haven gjort allt som har blivit eder befallt, då skolen I säga: 'Vi äro blott ringa tjänare: vi hava endast gjort vad vi voro pliktiga att göra.'"
Luk 17:11  Då han nu var stadd på sin färd till Jerusalem, tog han vägen mellan Samarien och Galileen.
Luk 17:12  Och när han kom in i en by, möttes han av tio spetälska män. Dessa stannade på avstånd
Luk 17:13  och ropade och sade: "Jesus, Mästare, förbarma dig över oss."
Luk 17:14  När han fick se dem, sade han till dem: "Gån och visen eder för prästerna." Och medan de voro på väg dit, blevo de rena.
Luk 17:15  Och en av dem vände tillbaka, när han såg att han hade blivit botad, och prisade Gud med hög röst
Luk 17:16  och föll ned på mitt ansikte för Jesu fötter och tackade honom. Och denne var en samarit.
Luk 17:17  Då svarade Jesus och sade: "Blevo icke alla tio gjorda rena? Var äro de nio?
Luk 17:18  Fanns då ibland dem ingen som vände tillbaka för att prisa Gud, utom denne främling?"
Luk 17:19  Och han sade till honom: "Stå upp och gå dina färde. Din tro har frälst dig."
Luk 17:20  Och då han blev tillfrågad av fariséerna när Guds rike skulle komma, svarade han dem och sade: "Guds rike kommer icke på sådant sätt att det kan förnimmas med ögonen,
Luk 17:21  ej heller skall man kunna säga: 'Se här är det', eller: 'Där är det.' Ty se, Guds rike är invärtes i eder."
Luk 17:22  Och han sade till lärjungarna: "Den tid skall komma, då I gärna skullen vilja se en av Människosonens dagar, men I skolen icke få det.
Luk 17:23  Väl skall man då säga till eder: 'Se där är han', eller: 'Se här är han'; men gån icke dit, och löpen icke därefter.
Luk 17:24  Ty såsom ljungelden, när den ljungar fram, lyser från himmelens ena ända till den andra, så skall det vara med Människosonen på hand dag.
Luk 17:25  Men först måste han lida mycket och bliva förkastad av detta släkte.
Luk 17:26  Och såsom det skedde på Noas tid, så skall det ock ske i Människosonens dagar:
Luk 17:27  människorna åto och drucko, män togo sig hustrur, och hustrur gåvos åt män, ända till den dag då Noa gick in i arken; då kom floden och förgjorde dem allasammans.
Luk 17:28  Likaledes, såsom det skedde på Lots tid: människorna åto och drucko, köpte och sålde, planterade och byggde,
Luk 17:29  men på den dag då Lot gick ut från Sodom regnade eld och svavel ned från himmelen och förgjorde dem allasammans,
Luk 17:30  alldeles på samma sätt skall det ske den dag då Människosonen uppenbaras.
Luk 17:31  Den som den dagen är på taket och har sitt bohag inne i huset, han må icke stiga ned för att hämta det; ej heller må den som är ute på marken vända tillbaka.
Luk 17:32  Kommen ihåg Lots hustru.
Luk 17:33  Den som står efter att vinna sitt liv, han skall mista det; men den som mister det, han skall rädda det.
Luk 17:34  Jag säger eder: Den natten skola två ligga i samma säng; den ene skall bliva upptagen, den andre skall lämnas kvar.
Luk 17:35  Två kvinnor skola mala tillhopa; den ena skall bliva upptagen, den andra skall lämnas kvar."
Luk 17:36  "Två skola vara tillsammans ute på marken; den ene skall bliva upptagen, och den andre skall lämnas kvar."
Luk 17:37  Då frågade de honom: "Var då, Herre?" Han svarade dem: "Där den döda kroppen är, dit skola ock rovfåglarna församla sig."
Luk 18:1  Och han framställde för dem en liknelse, för att lära dem att de alltid borde bedja, utan att förtröttas.
Luk 18:2  Han sade: "I en stad fanns en domare som icke fruktade Gud och ej heller hade försyn för någon människa.
Luk 18:3  I samma stad fanns ock en änka som åter och åter kom till honom och sade: 'Skaffa mig rätt av min motpart.'
Luk 18:4  Till en tid ville han icke. Men omsider sade han vid sig själv: 'Det må nu vara, att jag icke fruktar Gud och ej heller har försyn för någon människa;
Luk 18:5  likväl, eftersom denna änka är mig så besvärlig, vill jag ändå skaffa henne rätt, för att hon icke med sina ideliga besök skall alldeles pina ut mig.'"
Luk 18:6  Och Herren tillade: "Hören vad den orättfärdige domaren här säger.
Luk 18:7  Skulle då Gud icke skaffa rätt åt sina utvalda, som ropa till honom dag och natt, och skulle han icke hava tålamod med dem?
Luk 18:8  Jag säger eder: Han skall snart skaffa dem rätt. Men skall väl Människosonen, när han kommer, finna tro här på jorden?"
Luk 18:9  Ytterligare framställde han denna liknelse för somliga som förtröstade på sig själva och menade sig vara rättfärdiga, under det att de föraktade andra:
Luk 18:10  "Två män gingo upp i helgedomen för att bedja; den ene var en farisé och den andre en publikan.
Luk 18:11  Fariséen trädde fram och bad så för sig själv: 'Jag tackar dig, Gud, för att jag icke är såsom andra människor, rövare, orättrådiga, äktenskapsbrytare, ej heller såsom denne publikan.
Luk 18:12  Jag fastar två gånger i veckan; jag giver tionde av allt vad jag förvärvar.'
Luk 18:13  Men publikanen stod långt borta och ville icke ens lyfta sina ögon upp mot himmelen, utan slog sig för sitt bröst och sade: 'Gud, misskunda dig över mig syndare.' -
Luk 18:14  Jag säger eder: Denne gick hem igen rättfärdig mer an den andre. Ty var och en som upphöjer sig, han skall bliva förödmjukad, men den som ödmjukar sig, han skall bliva upphöjd."
Luk 18:15  Man bar fram till honom också späda barn, för att han skulle röra vid dem; men när hans lärjungar sågo detta, visade de bort dem.
Luk 18:16  Då kallade Jesus barnen till sig, i det han sade: "Låten barnen komma till mig, och förmenen dem det icke; ty Guds rike hör sådana till.
Luk 18:17  Sannerligen säger jag eder: Den som icke tager emot Guds rike såsom ett barn, han kommer aldrig ditin."
Luk 18:18  Och en överhetsperson frågade honom och sade: "Gode Mästare, vad skall jag göra för att få evigt liv till arvedel?"
Luk 18:19  Jesus sade till honom: "Varför kallar du mig god? Ingen är god utom Gud allena.
Luk 18:20  Buden känner du: 'Du skall icke begå äktenskapsbrott', 'Du skall icke dräpa', 'Du skall icke stjäla', 'Du skall icke bära falskt vittnesbörd', 'Hedra din fader och din moder.'"
Luk 18:21  Då svarade han: "Allt detta har jag hållit från min ungdom."
Luk 18:22  När Jesus hörde detta, sade han till honom: "Ett återstår dig ännu: sälj allt vad du äger och dela ut åt de fattiga; då skall du få en skatt i himmelen. Och kom sedan och följ mig."
Luk 18:23  Men när han hörde detta, blev han djupt bedrövad, ty han var mycket rik.
Luk 18:24  Då nu Jesus såg huru det var med honom, sade han: "Huru svårt är det icke för dem som hava penningar att komma in i Guds rike!
Luk 18:25  Ja, det är lättare för en kamel att komma in genom ett nålsöga, än för den som är rik att komma in i Guds rike."
Luk 18:26  Då sade de som hörde detta: "Vem kan då bliva frälst?"
Luk 18:27  Men han svarade: "Vad som är omöjligt för människor, det är möjligt för Gud."
Luk 18:28  Då sade Petrus: "Se, vi hava övergivit allt som var vårt och hava följt dig."
Luk 18:29  Han svarade dem: "Sannerligen säger jag eder: Ingen som för Guds rikes skull har övergivit hus, eller hustru, eller bröder, eller föräldrar eller barn,
Luk 18:30  ingen sådan finnes, som icke skall mångfaldigt igen redan här i tiden, och i den tillkommande tidsåldern evigt liv."
Luk 18:31  Och han tog till sig de tolv och sade till dem: "Se, vi gå nu upp till Jerusalem, och allt skall fullbordas, som genom profeterna är skrivet om Människosonen.
Luk 18:32  Ty han skall bliva överlämnad åt hedningarna och bliva begabbad och skymfad och bespottad,
Luk 18:33  och de skola gissla honom och döda honom; men på tredje dagen skall han uppstå igen."
Luk 18:34  Och de förstodo intet härav; ja, detta som han talade var dem så fördolt, att de icke fattade vad som sades.
Luk 18:35  Då han nu nalkades Jeriko, hände sig att en blind man satt vid vägen och tiggde.
Luk 18:36  När denne hörde en hop människor gå där fram, frågade han vad det var.
Luk 18:37  Och man omtalade för honom att det var Jesus från Nasaret som kom på vägen.
Luk 18:38  Då ropade han och sade: "Jesus, Davids son, förbarma dig över mig."
Luk 18:39  Och de som gingo framför tillsade honom strängeligen att han skulle tiga; men han ropade ännu mycket mer: "Davids son, förbarma dig över mig."
Luk 18:40  Då stannade Jesus och bjöd att mannen skulle ledas fram till honom. Och när han hade kommit fram, frågade han honom:
Luk 18:41  "Vad vill du att jag skall göra dig?" Han svarade: "Herre, låt mig få min syn."
Luk 18:42  Jesus sade till honom: "Hav din syn; din tro har hjälpt dig."
Luk 18:43  Och strax fick han sin syn och följde honom och prisade Gud. Och allt folket som såg detta lovade Gud.
Luk 19:1  Och han kom in i Jeriko och gick fram genom staden.
Luk 19:2  Där fanns en man, vid namn Sackeus, som var förman för publikanerna och en rik man.
Luk 19:3  Denne ville gärna veta vem som var Jesus och ville se honom, men han kunde det icke för folkets skull, ty han var liten till växten.
Luk 19:4  Då skyndade han i förväg och steg upp i ett mullbärsfikonträd för att få se honom, ty han skulle komma den vägen fram.
Luk 19:5  När Jesus nu kom till det stället, såg han upp och sade till honom: "Sackeus, skynda dig ned, ty i dag måste jag gästa i ditt hus."
Luk 19:6  Och han skyndade sig ned och tog emot honom med glädje.
Luk 19:7  Men alla som sågo det knorrade och sade: "Han har gått in för att gästa hos en syndare."
Luk 19:8  Men Sackeus trädde fram och sade till Herren: "Herre, hälften av mina ägodelar giver jag nu åt de fattiga; och om jag har utkrävt för mycket av någon, så giver jag fyradubbelt igen.
Luk 19:9  Och Jesus sade om honom: "I dag har frälsning vederfarits detta hus, eftersom också han är en Abrahams son.
Luk 19:10  Ty Människosonen har kommit för att uppsöka och frälsa det som var förlorat."
Luk 19:11  Medan de hörde härpå, framställde han ytterligare en liknelse, eftersom han var nära Jerusalem och de nu menade att Guds rike strax skulle uppenbaras.
Luk 19:12  Han sade alltså: "En man av förnämlig släkt tänkte fara bort till ett avlägset land för att utverka åt sig konungslig värdighet; sedan skulle han komma tillbaka.
Luk 19:13  Och han kallade till sig tio sin tjänare och gav dem tio pund och sade till dem: 'Förvalten dessa, till dess jag kommer tillbaka.'
Luk 19:14  Men hans landsmän hatade honom och sände, efter hans avfärd, åstad en beskickning och läto säga: 'Vi vilja icke att denne skall bliva konung över oss."
Luk 19:15  När han sedan kom tillbaka, efter att hava utverkat åt sig den konungsliga värdigheten, lät han kalla till sig de tjänare åt vilka han hade givit penningarna, ty han ville veta vad var och en genom sin förvaltning hade förvärvat.
Luk 19:16  Då kom den förste fram och sade: 'Herre, ditt pund har givit i vinst tio pund.'
Luk 19:17  Han svarade honom: 'Rätt så, du gode tjänare! Eftersom du har varit trogen i en mycket ringa sak, skall du få makt och myndighet över tio städer.'
Luk 19:18  Därefter kom den andre i ordningen och sade: 'Herre, ditt pund har avkastat fem pund.'
Luk 19:19  Då sade han jämväl till denne: 'Så vare ock du satt över fem städer.'
Luk 19:20  Och den siste kom fram och sade: 'Herre, se här är ditt pund; jag har haft det förvarat i en duk.
Luk 19:21  Ty jag fruktade för dig, eftersom du är en sträng man; du vill taga upp vad du icke har lagt ned, och skörda vad du icke har sått.'
Luk 19:22  Han sade till honom: 'Efter dina egna ord vill jag döma dig, du onde tjänare. Du visste alltså att jag är en sträng man, som vill taga upp vad jag icke har lagt ned, och skörda vad jag icke har sått?
Luk 19:23  Varför satte du då icke in mina penningar i en bank? Då hade jag, när jag kom hem, fått uppbära dem med ränta.'
Luk 19:24  Och han sade till dem som stodo vid hans sida: 'Tagen ifrån honom hans pund, och given det åt den som har de tio punden.'
Luk 19:25  De sade till honom: 'Herre, han har ju redan tio pund.'
Luk 19:26  Han svarade: 'Jag säger eder: Var och en som har, åt honom skall varda givet; men den som icke har, från honom skall tagas också det han har.
Luk 19:27  Men dessa mina ovänner, som icke ville hava mig till konung över sig, fören dem hit huggen ned dem här inför mig."
Luk 19:28  Sedan Jesus hade sagt detta, gick han framför de andra upp mot Jerusalem.
Luk 19:29  När han då nalkades Betfage och Betania, vid det berg som kallas Oljeberget, sände han åstad två av lärjungarna
Luk 19:30  och sade: "Gån in i byn som ligger här mitt framför. Och när I kommen ditin, skolen I finna en åsnefåle stå där bunden, som ingen människa någonsin har suttit på, lösen den och fören den hit.
Luk 19:31  Och om någon frågar eder varför I lösen den skolen I svara så: 'Herren behöver den.'"
Luk 19:32  Och de som hade blivit utsända gingo åstad och funno det så som han hade sagt dem.
Luk 19:33  Och när de löste fålen, frågade ägaren dem: "Varför lösen I fålen?"
Luk 19:34  De svarade: "Herren behöver den."
Luk 19:35  Och de förde fålen till Jesus och lade sina mantlar på den och läto Jesus sätta sig därovanpå.
Luk 19:36  Och där han färdades fram bredde de ut sina mantlar under honom på vägen.
Luk 19:37  Och då han var nära foten av Oljeberget, begynte hela lärjungaskaran i sin glädje att med hög röst lova Gud för alla de kraftgärningar som de hade sett;
Luk 19:38  och de sade: "Välsignad vare han som kommer, konungen, i Herrens namn. Frid vare i himmelen och ära i höjden!"
Luk 19:39  Och några fariséer som voro med i folkhopen sade till honom: "Mästare, förbjud dina lärjungar att ropa så."
Luk 19:40  Men han svarade och sade: "Jag säger eder: Om dessa tiga, skola stenarna ropa."
Luk 19:41  Då han nu kom närmare och fick se staden, begynte han gråta över den
Luk 19:42  och sade: "O att du i dag hade insett, också du, vad din frid tillhör! Men nu är det fördolt för dina ögon.
Luk 19:43  Ty den tid skall komma över dig, då dina fiender skola omgiva dig med belägringsvall och innesluta dig och tränga dig på alla sidor.
Luk 19:44  Och de skola slå ned dig till jorden, tillika med dina barn, som äro i dig, och skola icke lämna kvar i dig sten på sten, därför att du icke aktade på den tid då du var sökt."
Luk 19:45  Och han gick in i helgedomen och begynte driva ut dem som sålde därinne;
Luk 19:46  och han sade till dem: "Det är skrivet: 'Och mitt hus skall vara ett bönehus.' Men I haven gjort det till en rövarkula."
Luk 19:47  Och han undervisade var dag i helgedomen. Och översteprästerna och de skriftlärde och folkets förnämste män sökte efter tillfälle att förgöra honom;
Luk 19:48  men de kunde icke finna någon utväg därtill, ty allt folket höll sig till honom och hörde honom.
Luk 20:1  Och en dag, då han undervisade folket i helgedomen och förkunnade evangelium, trädde översteprästerna och de skriftlärde, tillika med de äldste, fram
Luk 20:2  och talade till honom och sade: "Säg oss, med vad myndighet gör du detta? Och vem är det som har rivit dig sådan myndighet?"
Luk 20:3  Han svarade och sade till dem: "Också jag vill ställa en fråga till eder; svaren mig på den.
Luk 20:4  Johannes' döpelse, var den från himmelen eller från människor?"
Luk 20:5  Då överlade de med varandra och sade: "Om vi svara: 'Från himmelen' så frågar han: 'Varför trodden I honom då icke?'
Luk 20:6  Men om vi svara: 'Från människor', då kommer allt folket att stena oss, ty de äro förvissade om att Johannes var en profet."
Luk 20:7  De svarade alltså att de icke visste varifrån den var.
Luk 20:8  Då sade Jesus till dem: "Så säger icke heller jag eder med vad myndighet jag gör detta."
Luk 20:9  Och han framställde för folket denna liknelse: "En man planterade en vingård och lejde ut den åt vingårdsmän och för utrikes för lång tid.
Luk 20:10  När sedan rätta tiden var inne, sände han en tjänare till vingårdsmännen, för att de åt denne skulle lämna någon del av vingårdens frukt. Men vingårdsmännen misshandlade honom och läto honom gå tomhänt bort.
Luk 20:11  Ytterligare sände han en annan tjänare. Också honom misshandlade och skymfade de och läto honom gå tomhänt bort.
Luk 20:12  Ytterligare sände han en tredje. Men också denne slogo de blodig och drevo bort honom.
Luk 20:13  Då sade vingårdens herre: 'Vad skall jag göra? Jo, jag vill sända min älskade son; för honom skola de väl ändå hava försyn.'
Luk 20:14  Men när vingårdsmannen fingo se honom, överlade de med varandra och sade: 'Denne är arvingen; låt oss dräpa honom, för att arvet må bliva vårt.'
Luk 20:15  Och de förde honom ut ur vingården och dräpte honom. "Vad skall nu vingårdens herre göra med dem?
Luk 20:16  Jo, han skall komma och förgöra de vingårdsmännen och lämna vingården åt andra." När de hörde detta, sade de: "Bort det!"
Luk 20:17  Då såg han på dem och sade: "Vad betyder då detta skriftens ord: 'Den sten som byggningsmännen förkastade, den har blivit en hörnsten'?
Luk 20:18  Var och en som faller på den stenen, han skall bliva krossad; men den som stenen faller på, honom skall den söndersmula."
Luk 20:19  Och de skriftlärde och översteprästerna hade gärna velat i samma stund gripa honom, men de fruktade för folket. Ty de förstodo att det var om dem som han hade talat i denna liknelse.
Luk 20:20  Och de vaktade på honom och sände ut några som försåtligen skulle låtsa sig vara rättsinniga män, för att dessa skulle fånga honom genom något hans ord, så att de skulle kunna överlämna honom åt överheten, i landshövdingens våld.
Luk 20:21  Dessa frågade honom och sade: "Mästare, vi veta att du talar och undervisar rätt och icke har anseende till personen, utan lär om Guds väg vad sant är.
Luk 20:22  Är det lovligt för oss att giva kejsaren skatt, eller är det icke lovligt?"
Luk 20:23  Men han märkte deras illfundighet och sade till dem:
Luk 20:24  "Låten mig se en penning. Vems bild och överskrift bär den?" De svarade: "Kejsarens."
Luk 20:25  Då sade han till dem: "Given alltså kejsaren vad kejsaren tillhör, och Gud vad Gud tillhör."
Luk 20:26  Och de förmådde icke fånga honom genom något hans ord inför folket, utan förundrade sig över hans svar och tego.
Luk 20:27  Därefter trädde några sadducéer fram och ville påstå att det icke gives någon uppståndelse. Dessa frågade honom
Luk 20:28  och sade: "Mästare, Moses har givit oss den föreskriften, att om någon har en broder som är gift, men dör barnlös, så skall han taga sin broders hustru till akta och skaffa avkomma åt sin broder.
Luk 20:29  Nu voro här sju bröder. Den förste tog sig en hustru, men dog barnlös.
Luk 20:30  Då tog den andre i ordningen henne
Luk 20:31  och därefter den tredje; sammalunda alla sju. Men de dogo alla, utan att någon av dem lämnade barn efter sig.
Luk 20:32  Slutligen dog ock hustrun.
Luk 20:33  Vilken av dem skall då vid uppståndelsen få kvinnan till hustru? De hade ju alla sju tagit henne till hustru."
Luk 20:34  Jesus svarade dem: "Med den nuvarande tidsålderns barn är det så, att män taga sig hustrur, och hustrur givas åt män;
Luk 20:35  men de som bliva aktade värdiga att få del i den nya tidsåldern och i uppståndelsen från de döda, med dem är det så, att varken män tag sig hustrur, eller hustrur givas män.
Luk 20:36  De kunna ju ej heller mer dö ty de äro lika änglarna och äro, Guds söner, eftersom de hava blivit delaktiga av uppståndelsen.
Luk 20:37  Men att de döda uppstå, det har ock Moses, på det ställe där det talas om törnbusken, givit till känna, när han kallar Herren 'Abrahams Gud och Isaks Gud och Jakobs Gud';
Luk 20:38  Och han är en Gud icke för döda, utan för levande, ty för honom leva alla."
Luk 20:39  Då svarade några av de skriftlärde och sade: "Mästare, du har talat rätt."
Luk 20:40  De dristade sig nämligen icke att vidare ställa någon fråga på honom.
Luk 20:41  Men han sade till dem: "Huru kan man säga att Messias är Davids son?
Luk 20:42  David själv säger ju i Psalmernas bok: Herren sade till min herre: Sätt dig på min högra sida,
Luk 20:43  till dess jag har lagt dina fiender dig till en fotapall.'
Luk 20:44  David kallar honom alltså 'herre'; huru kan han då vara hans son?"
Luk 20:45  Och han sade till sina lärjungar, så att allt folket hörde det:
Luk 20:46  "Tagen eder till vara för de skriftlärde, som gärna gå omkring i fotsida kläder och gärna vilja bliva hälsade på torgen och gärna sitta främst i synagogorna och på de främsta platserna vid gästabuden -
Luk 20:47  detta under det att de utsuga änkors hus, medan de för syns skull hålla långa baner. Del skola få en dess hårdare dom."
Luk 21:1  Och när han såg upp, fick han se huru de rika lade ned sina gåvor i offerkistorna.
Luk 21:2  Därvid fick han ock se huru en fattig änka lade ned två skärvar.
Luk 21:3  Då sade han: "Sannerligen säger jag eder: Denna fattiga änka lade dit mer än alla de andra.
Luk 21:4  Ty det var av sitt överflöd som alla dessa lade ned något bland gåvorna, men hon lade dit av sitt armod allt vad hon hade i sin ägo."
Luk 21:5  Och då några talade om helgedomen, huru den var uppförd av härliga stenar och prydd med helgedomsskänker, sade han:
Luk 21:6  "Dagar skola komma, då av allt detta som I nu sen icke skall lämnas sten på sten, utan allt skall bliva nedbrutet."
Luk 21:7  Då frågade de honom och sade: "Mästare, när skall detta ske? Och vad bliver tecknet till att tiden är inne, då detta kommer att ske?"
Luk 21:8  Han svarade: "Sen till, att I icke bliven förvillade. Ty många skola komma under mitt namn och säga: 'Det är jag' och: 'Tiden är nära'. Men följen dem icke.
Luk 21:9  Och när I fån höra krigslarm och upprorslarm, så bliven icke förfärade; ty sådant måste först komma, men därmed är icke strax änden inne."
Luk 21:10  Därefter sade han till dem: "Folk skall resa sig upp mot folk och rike mot rike;
Luk 21:11  och det skall bliva stora jordbävningar, så ock hungersnöd och farsoter på den ena orten efter den andra, och skräcksyner skola visa sig och stora tecken på himmelen.
Luk 21:12  Men före allt detta skall man gripa eder, man skall förfölja eder och draga eder inför synagogorna och sätta eder i fängelse och föra eder fram inför konungar och landshövdingar, för mitt namns skull.
Luk 21:13  Så skolen I få tillfälle att frambära vittnesbörd.
Luk 21:14  Märken därför noga att I icke förut mån göra eder bekymmer för huru I skolen försvara eder.
Luk 21:15  Ty jag skall giva eder sådana ord och sådan vishet, att ingen av edra vedersakare skall kunna stå emot eller säga något emot.
Luk 21:16  I skolen bliva förrådda till och med av föräldrar och bröder och fränder och vänner; och somliga av eder skall man döda.
Luk 21:17  Och I skolen bliva hatade av alla för mitt namns skull.
Luk 21:18  Men icke ett hår på edra huvuden skall gå förlorat.
Luk 21:19  Genom att vara ståndaktiga skolen I vinna edra själar.
Luk 21:20  Men när I fån se Jerusalem omringas av krigshärar, då skolen I veta att dess ödeläggelse är nära.
Luk 21:21  Då må de som äro i Judeen fly bort till bergen, och de som äro inne i staden må draga ut därifrån och de som äro ute på landsbygden må icke gå ditin.
Luk 21:22  Ty detta är en hämndens tid, då allt som är skrivet skall uppfyllas.
Luk 21:23  Ve dem som äro havande, eller som giva di på den tiden! Ty stor nöd skall då komma i landet, och en vredesdom över detta folk.
Luk 21:24  Och de skola falla för svärdsegg och bliva bortförda i fångenskap till allahanda hednafolk; och Jerusalem skall bliva förtrampat av hedningarna, till dess att hedningarnas tider äro fullbordade.
Luk 21:25  Och tecken skola ske i solen och månen och i stjärnorna, och på jorden skall ångest komma över folken, och de skola stå rådlösa vid havets och vågornas dån,
Luk 21:26  då nu människor uppgiva andan av förskräckelse och ängslan för det som skall övergå världen; ty himmelens makter skola bäva.
Luk 21:27  Och då skall man få se 'Människosonen komma i en sky', med stor makt och härlighet.
Luk 21:28  Men när detta begynner ske, då mån I resa eder upp och upplyfta edra huvuden, ty då nalkas eder förlossning."
Luk 21:29  Och han framställde för dem en liknelse: "Sen på fikonträdet och på alla andra träd.
Luk 21:30  När I fån se att de skjuta knopp, då veten I av eder själva att sommaren redan är nära.
Luk 21:31  Likaså, när I sen detta ske, då kunnen I ock veta att Guds rike är nära.
Luk 21:32  Sannerligen säger jag eder: Detta släkte skall icke förgås, förrän allt detta sker.
Luk 21:33  Himmel och jord skola förgås, men mina ord skola aldrig förgås.
Luk 21:34  Men tagen eder till vara för att låta edra hjärtan förtyngas av omåttlighet och dryckenskap och timliga omsorger, så att den dagen kommer på eder oförtänkt;
Luk 21:35  ty såsom en snara skall den komma över hela jordens alla inbyggare.
Luk 21:36  Men vaken alltjämt, och bedjen att I mån kunna undfly allt detta som skall komma, och kunna bestå inför Människosonen."
Luk 21:37  Och han undervisade om dagarna i helgedomen, men om aftnarna gick han ut till det berg som kallas Oljeberget och stannade där över natten.
Luk 21:38  Och allt folket kom bittida om morgonen till honom i helgedomen för att höra honom.
Luk 22:1  Det osyrade brödets högtid, som ock kallas påsk, var nu nära.
Luk 22:2  Och översteprästerna och de skriftlärde sökte efter tillfälle att röja honom ur vägen. De fruktade nämligen för folket.
Luk 22:3  Men Satan for in i Judas, som kallades Iskariot, och som var en av de tolv.
Luk 22:4  Denne gick bort och talade med översteprästerna och befälhavarna för tempelvakten om huru han skulle överlämna honom åt dem.
Luk 22:5  Då blevo de glada och förklarade sig villiga att giva honom en summa penningar.
Luk 22:6  Och han gick in på deras anbud och sökte sedan efter lägligt tillfälle att förråda honom åt dem, utan att någon folkskockning uppstod.
Luk 22:7  Så kom nu den dag i det osyrad brödets högtid, då man skulle slakta påskalammet.
Luk 22:8  Då sände han åstad Petrus och Johannes och sade: "Gån åstad och reden till åt oss, så att vi kunna äta påskalammet."
Luk 22:9  De frågade honom: Var vill du att vi skola reda till det?"
Luk 22:10  Han svarade dem: När I kommen in i staden, skolen I möta en man som bär en kruka vatten. Följen honom till det hus där han går in.
Luk 22:11  Och sägen till husbonden i det huset: 'Mästaren frågar dig: Var finnes härbärget där jag skall äta påskalammet med mina lärjungar?'
Luk 22:12  Då skall han visa eder en stor sal i övre våningen, ordnad för måltid; reden till där."
Luk 22:13  Och de gingo åstad och funno det så som han hade sagt dem; och de redde till påskalammet.
Luk 22:14  Och när stunden var inne, lade han sig till bords, och apostlarna med honom.
Luk 22:15  Och han sade till dem: "Jag har högeligen åstundat att äta detta påskalamm med eder, förrän mitt lidande begynner;
Luk 22:16  ty jag säger eder att jag icke mer skall fira denna högtid, förrän den kommer till fullbordan i Guds rike."
Luk 22:17  Och han lät giva sig en kalk och tackade Gud och sade: "Tagen detta och delen eder emellan;
Luk 22:18  ty jag säger eder att jag härefter icke, förrän Guds rike kommer, skall dricka av det som kommer från vinträd."
Luk 22:19  Sedan tog han ett bröd och tackade Gud och bröt det och gav åt dem och sade: "Detta är min lekamen, som varder utgiven för eder. Gören detta till min åminnelse."
Luk 22:20  Sammalunda tog han ock kalken, efter måltiden, och sade: "Denna kalk är det nya förbundet, i mitt blod, som varder utgjutet för eder.
Luk 22:21  Men se, den som förråder mig, hans hand är med mig på bordet.
Luk 22:22  Ty Människosonen skall gå bort, såsom förut är bestämt; men ve den människa genom vilken han bliver förrådd!"
Luk 22:23  Och de begynte tala med varandra om vilken av dem det väl kunde vara som skulle göra detta.
Luk 22:24  En tvist uppstod ock mellan dem om vilken av dem som skulle räknas för den störste.
Luk 22:25  Då sade han till dem: "Konungarna uppträda mot sina folk såsom härskare, och de som hava myndighet över folken låta kalla sig 'nådige herrar'.
Luk 22:26  Men så är det icke med eder; utan den som är störst bland eder, han vare såsom den yngste, och den som är den förnämste, han vare såsom en tjänare.
Luk 22:27  Ty vilken är större: den som ligger till bords eller den som tjänar? Är det icke den som ligger till bords? Och likväl är jag här ibland eder såsom en tjänare. -
Luk 22:28  Men I ären de som hava förblivit hos mig i mina prövningar;
Luk 22:29  och såsom min Fader har överlåtit konungslig makt åt mig, så överlåter jag likadan makt åt eder,
Luk 22:30  så att I skolen få äta och dricka vid mitt bord i mitt rike och sitta på troner såsom domare över Israels tolv släkter.
Luk 22:31  Simon, Simon! Se, Satan har begärt att få eder i sitt våld, för att kunna sålla eder såsom vete;
Luk 22:32  men jag har bett för dig, att din tro icke må bliva om intet. Och när du en gång har omvänt dig, så styrk dina bröder."
Luk 22:33  Då sade han till honom: "Herre, jag är redo att med dig både gå i fängelse och gå i döden."
Luk 22:34  Men han svarade: "Jag säger dig, Petrus: I dag skall icke hanen gala, förrän du tre gånger har förnekat mig och sagt att du icke känner mig."
Luk 22:35  Ytterligare sade han till dem: "När jag sände eder åstad utan penningpung, utan ränsel, utan skor, icke fattades eder då något?" De svarade: "Intet."
Luk 22:36  Då sade han till dem: "Nu åter må den som har en penningpung taga den med sig, och den som har en ränsel, han göre sammalunda; och den som icke har något svärd, han sälje sin mantel och köpe sig ett sådant.
Luk 22:37  Ty jag säger eder att på mig måste fullbordas detta skriftens ord: 'Han blev räknad bland ogärningsmän'. Ja, det som är förutsagt om mig, det går nu i fullbordan"
Luk 22:38  Då sade de: "Herre, se här äro två svärd." Han svarade dem: "Det är nog."
Luk 22:39  Och han gick ut och begav sig till Oljeberget, såsom hans sed var; och hans lärjungar följde honom.
Luk 22:40  Men när han hade kommit till platsen, sade han till dem: "Bedjen att I icke mån komma i frestelse."
Luk 22:41  Sedan gick han bort ifrån dem, vid pass ett stenkast, och föll ned på sina knän och bad
Luk 22:42  och sade: "Fader, om det är din vilja, så tag denna kalk ifrån mig. Dock, ske icke min vilja, utan din."
Luk 22:43  Då visade sig för honom en ängel från himmelen, som styrkte honom.
Luk 22:44  Men han hade kommit i svår ångest och bad allt ivrigare, och hans svett blev såsom blodsdroppar, som föllo ned på jorden.
Luk 22:45  När han sedan stod upp från bönen och kom tillbaka till lärjungarna, fann han dem insomnade av bedrövelse.
Luk 22:46  Då sade han till dem: "Varför soven I? Stån upp, och bedjen att I icke mån komma i frestelse."
Luk 22:47  Och se, medan han ännu talade, kom en folkskara; och en av de tolv, den som hette Judas, gick framför dem. Och han trädde fram till Jesus för att kyssa honom.
Luk 22:48  Men Jesus sade till honom: "Judas, förråder du Människosonen med en kyss?"
Luk 22:49  Då nu de som voro med Jesus sågo vad som var på färde, frågade de: "Herre, skola vi hugga till med svärd?"
Luk 22:50  Och en av dem högg till översteprästens tjänare och högg så av honom högra örat.
Luk 22:51  Då svarade Jesus och sade: "Låten det gå så långt." Och han rörde vid hans öra och helade honom.
Luk 22:52  Sedan sade Jesus till dem som hade kommit emot honom, till översteprästerna och befälhavarna för tempelvakten och de äldste: "Såsom mot en rövare haven I gått ut med svärd och stavar.
Luk 22:53  Fastän jag var dag har varit med eder i helgedomen, haven I icke sträckt ut edra händer emot mig men detta är eder stund, och nu råder mörkrets makt."
Luk 22:54  Så grepo de honom och förde honom åstad in i översteprästens hus. Och Petrus följde efter på avstånd.
Luk 22:55  Och de tände upp en eld mitt på gården och satte sig där tillsammans, och Petrus satte sig ibland dem.
Luk 22:56  Men en tjänstekvinna, som fick se honom, där han satt vid elden fäste ögonen på honom och sade: "Också denne var med honom.
Luk 22:57  Men han nekade och sade: "Kvinna, jag känner honom icke."
Luk 22:58  Kort därefter fick en annan, en av mannen, se honom och sade: "Också du är en av dem." Men Petrus svarade: "Nej, det är jag icke."
Luk 22:59  Vid pass en timme därefter kom en annan som bedyrade och sade: "Förvisso var också denne med honom; han är ju ock en galilé."
Luk 22:60  Då svarade Petrus: "Jag förstår icke vad du menar." Och i detsamma, medan han ännu talade, gol hanen.
Luk 22:61  Då vände Herren sig om och såg på Petrus; och Petrus kom då ihåg Herrens ord, huru han hade sagt till honom: "Förrän hanen i dag har galit, skall du tre gånger förneka mig."
Luk 22:62  Och han gick ut och grät bitterligen.
Luk 22:63  Och de män som höllo Jesus fången begabbade honom och misshandlade honom.
Luk 22:64  De höljde över honom och frågade honom och sade: "Profetera: vem var det som slog dig?"
Luk 22:65  Många andra smädliga ord talade de ock mot honom.
Luk 22:66  Men när det blev dag, församlade sig folkets äldste, överstepräster och skriftlärde, och läto föra honom inför sitt Stora råd
Luk 22:67  och sade: "Är du Messias, så säg oss det." Men han svarade dem: "Om jag säger eder det, så tron I det icke.
Luk 22:68  Och om jag frågar, så svaren I icke.
Luk 22:69  Men härefter skall Människosonen sitta på den gudomliga Maktens högra sida."
Luk 22:70  Då sade de alla: "Så är du då Guds Son?" Han svarade dem: "I sägen det själva, att jag är det."
Luk 22:71  Då sade de: "Vad behöva vi mer något vittnesbörd? Vi hava ju själva nu hört det av hans egen mun."
Luk 23:1  Och de stodo upp, hela hopen, och förde honom till Pilatus.
Luk 23:2  Där begynte de anklaga honom och sade: "Vi hava funnit att denne man förleder vårt folk och vill förhindra att man giver kejsaren skatt, och att han säger sig vara Messias, en konung."
Luk 23:3  Då frågade Pilatus honom och sade: Är du judarnas konung?" Han svarade honom och sade: "Du säger det själv."
Luk 23:4  Men Pilatus sade till översteprästerna och till folket: "Jag finner intet brottsligt hos denne man."
Luk 23:5  Då blevo de ännu ivrigare och sade: "Han uppviglar med sin lära folket i hela Judeen, allt ifrån Galileen och ända hit."
Luk 23:6  När Pilatus hörde detta, frågade han om mannen var från Galileen.
Luk 23:7  Och då han fick veta att han var från det land som lydde under Herodes' välde, sände han honom bort till Herodes, som under dessa dalar också var i Jerusalem.
Luk 23:8  När Herodes fick se Jesus, blev han mycket glad, ty han hade sedan lång tid velat se honom; han hade nämligen hört talas om honom, och han hoppades nu att få se honom göra något tecken.
Luk 23:9  Men fastän han ställde ganska många frågor på Jesus, svarade denne honom intet.
Luk 23:10  Och översteprästerna och de skriftlärde stodo där och anklagade honom häftigt.
Luk 23:11  Men Herodes och hans krigsfolk bemötte honom med förakt och begabbade honom; och sedan de hade satt på honom en lysande klädnad, sände de honom tillbaka till Pilatus.
Luk 23:12  Och Herodes och Pilatus blevo den dagen vänner med varandra; Förut hade nämligen dem emellan rått ovänskap.
Luk 23:13  Sedan kallade Pilatus tillhopa översteprästerna och rådsherrarna och folket
Luk 23:14  och sade till dem: "I haven fört till mig denne man och sagt att han förleder folket; och jag har nu i eder närvaro anställt rannsakning med honom, men icke funnit honom skyldig till något av det som I anklagen honom för.
Luk 23:15  Och ej heller Herodes har funnit honom skyldig; han har ju sänt honom tillbaka till oss. I sen alltså att denne icke har gjort något som förtjänar döden.
Luk 23:16  Därför vill jag giva honom lös, medan jag har tuktat honom."
Luk 23:17  Men han var tvungen att giva dem en fånge lös vid högtiden.
Luk 23:18  Då skriade hela hopen och sade: "Hav bort denne, och giv oss Barabbas lös."
Luk 23:19  (Denne man hade blivit kastad i fängelse på grund av ett upplopp, som hade ägt rum i staden, och för ett dråps skull.)
Luk 23:20  Åter talade Pilatus till dem, ty han önskade att kunna giva Jesus lös.
Luk 23:21  Men de ropade emot honom: "Korsfäst, korsfäst honom!"
Luk 23:22  Då talade han till dem för tredje gången och frågade: "Vad ont har denne då gjort? Jag har icke funnit honom skyldig till något som förtjänar döden Därför vill jag giva honom lös, sedan jag har tuktat honom."
Luk 23:23  Men de lågo över honom med höga rop och begärde att han skulle låta korsfästa honom; och deras rop blevo honom övermäktiga.
Luk 23:24  Då dömde Pilatus att så skulle ske, som de begärde.
Luk 23:25  Och han lösgav den man de begärde, den som hade blivit kastad i fängelse för upplopp och dråp; men Jesus utlämnade han, för att med honom skulle ske efter deras vilja.
Luk 23:26  När de sedan förde bort honom, fingo de fatt en man, Simon från Cyrene, som kom utifrån marken; på honom lade de korset, för att han skulle bära det efter Jesus.
Luk 23:27  Men en stor hop folk följde med honom, bland dem också kvinnor som jämrade sig och gräto över honom.
Luk 23:28  Då vände Jesus sig om till dem och sade: "I Jerusalems döttrar, gråten icke över mig, utan gråten över eder själva och över edra barn.
Luk 23:29  Ty se, den tid skall komma, då man skall säga: 'Saliga äro de ofruktsamma, de moderliv som icke hava fött barn, och de bröst som icke hava givit di.'
Luk 23:30  Då skall man begynna säga till: bergen: 'Fallen över oss', och till höjderna: 'Skylen oss.'
Luk 23:31  Ty om han gör så med det friska trädet, vad skall icke då ske med det torra!"
Luk 23:32  Jämväl två andra, två ogärningsmän, fördes ut för att avlivas tillika med honom.
Luk 23:33  Och när de hade kommit till den plats som kallades "Huvudskallen" korsfäste de honom där, så ock ogärningsmännen, den ene på högra sidan och den andre på vänstra.
Luk 23:34  Men Jesus sade. "Fader, förlåt dem; ty de veta icke vad de göra. Och de delade hans kläder mellan sig och kastade lott om dem. -
Luk 23:35  Men folket stod och såg därpå. Och jämväl rådsherrarna drevo gäck med honom och sade: "Andra har han hjälpt; nu må han hjälpa sig själv, om han är Guds Smorde, den utvalde."
Luk 23:36  Också krigsmännen gingo fram och begabbade honom och räckte honom ättikvin
Luk 23:37  och sade: "Är du judarnas konung, så hjälp dig själv."
Luk 23:38  Men över honom hade man ock satt upp en överskrift: "Denne är judarnas konung."
Luk 23:39  Och en av de ogärningsmän som voro där upphängda smädade honom och sade: "Du är ju Messias; hjälp då dig själv och oss."
Luk 23:40  Då tillrättavisade honom den andre och svarade och sade: "Fruktar icke heller du Gud, du som är under samma dom?
Luk 23:41  Oss vederfares detta med all rätt, ty vi lida vad våra gärningar äro värda, men denne man har intet ont gjort."
Luk 23:42  Sedan sade han: "Jesus, tänk på mig, när du kommer i ditt rike."
Luk 23:43  Han svarade honom: "Sannerligen säger jag dig: I dag skall du vara med mig i paradiset."
Luk 23:44  Det var nu omkring sjätte timmen; då kom över hela landet ett mörker, som varade ända till nionde timmen,
Luk 23:45  i det att solen miste sitt sken. Och förlåten i templet rämnade mitt itu.
Luk 23:46  Och Jesus ropade med hög röst och sade: "Fader, i dina händer befaller jag min ande." Och när han hade sagt detta, gav han upp andan.
Luk 23:47  Men när hövitsmannen såg vad som skedde, prisade han Gud och sade: "Så var då denne verkligen en rättfärdig man!"
Luk 23:48  Och när allt folket, de som hade kommit tillsammans för att se härpå, sågo vad som skedde, slogo de sig för bröstet och vände hem igen.
Luk 23:49  Men alla hans vänner stodo på avstånd och sågo detta, bland dem också några kvinnor, de som hade följt med honom från Galileen.
Luk 23:50  Nu var där en rådsherre, vi namn Josef, en god och rättfärdig man,
Luk 23:51  som icke hade samtyckt till deras rådslag och gärning. Han var från Arimatea, en stad i Judeen; och han väntade på Guds rike.
Luk 23:52  Denne gick till Pilatus och utbad sig att få Jesu kropp.
Luk 23:53  Och han tog ned den och svepte den i en linneduk. Sedan lade han den i en grav som var uthuggen i klippan, och där ännu ingen hade varit lagd.
Luk 23:54  Det var då tillredelsedag, och sabbatsdagen begynte ingå.
Luk 23:55  Och de kvinnor, som med honom hade kommit från Galileen, följde efter och sågo graven och sågo huru hans kropp lades ned däri.
Luk 23:56  Sedan vände de hem igen och redde till välluktande kryddor och smörjelse; men på sabbaten voro de stilla, efter lagens bud.
Luk 24:1  Men på första veckodagen kommo de, tidigt i själva dagbräckningen, till graven med de välluktande kryddor som de hade tillrett.
Luk 24:2  Och de funno stenen vara bortvältrad från graven.
Luk 24:3  Då gingo de ditin, men funno icke Herren Jesu kropp.
Luk 24:4  När de nu icke visste vad de skulle tänka härom, se, då stodo två man framför dem i skinande kläder.
Luk 24:5  Och de blevo förskräckta och böjde sina ansikten ned mot jorden. Då sade mannen till dem "Varför söken I den levande bland de döda?
Luk 24:6  Han är icke har, han är uppstånden. Kommen ihåg vad han talade till eder, medan han ännu var i Galileen, huru han sade:
Luk 24:7  'Människosonen måste bliva överlämnad i syndiga människors händer och bliva korsfäst; men på tredje dagen skall han uppstå igen.'"
Luk 24:8  Då kommo de ihåg hans ord.
Luk 24:9  Och de vände tillbaka från graven och omtalade allt detta för de elva och för alla de andra. -
Luk 24:10  Kvinnorna voro Maria från Magdala och Johanna och den Maria som var Jakobs moder. Och jämväl de andra kvinnorna instämde med dem och sade detsamma till apostlarna.
Luk 24:11  Deras ord syntes dock för dessa vara löst tal, och de trodde dem icke.
Luk 24:12  Men Petrus stod upp och skyndade till graven; och när han lutade sig ditin såg han där allenast linnebindlarna. Sedan gick han hem till sitt, uppfylld av förundran över det som hade skett.
Luk 24:13  Men två av dem voro samma dag stadda på vandring till en by som hette Emmaus, och som låg sextio stadiers väg från Jerusalem.
Luk 24:14  Och de samtalade med varandra om allt detta som hade skett.
Luk 24:15  Medan de nu samtalade och överlade med varandra, nalkades Jesus själv och gick med dem.
Luk 24:16  Men deras ögon voro tillslutna, så att de icke kände igen honom.
Luk 24:17  Och han sade till dem: "Vad är det I talen om med varandra, medan I gån här?" Då stannade de och sågo bedrövade ut.
Luk 24:18  Och den ene, som hette Kleopas, svarade och sade till honom: "Du är väl en främling i Jerusalem, den ende som icke har hört vad där har skett i dessa dagar?"
Luk 24:19  Han frågade dem: "Vad då?" De svarade honom: "Det som har skett med Jesus från Nasaret, vilken var en profet, mäktig i gärningar och ord inför Gud och allt folket:
Luk 24:20  huru nämligen våra överstepräster och rådsherrar hava utlämnat honom till att dömas till döden och hava korsfäst honom.
Luk 24:21  Men vi hoppades att han var den som skulle förlossa Israel. Och likväl, till allt detta kommer att det redan är tredje dagen sedan detta skedde.
Luk 24:22  Men nu hava därjämte några av våra kvinnor gjort oss häpna; ty sedan de bittida på morgonen hade varit vid graven
Luk 24:23  och icke funnit hans kropp, kommo de igen och sade att de till och med hade sett en änglasyn, och änglarna hade sagt att han levde.
Luk 24:24  Och när några av dem som voro. med oss gingo bort till graven, funno de det vara så som kvinnorna hade sagt, men honom själv sågo de icke."
Luk 24:25  Då sade han till dem: "O, huru oförståndiga ären I icke och tröghjärtade till att tro på allt vad profeterna hava talat!
Luk 24:26  Måste icke Messias lida detta, för I att så ingå i sin härlighet?"
Luk 24:27  Och han begynte att genomgå Moses och alla profeterna och uttydde för dem vad som i alla skrifterna var sagt om honom.
Luk 24:28  När de nu nalkades byn dit de voro på väg, ställde han sig som om han ville gå vidare.
Luk 24:29  Men de nödgade honom och sade: "Bliv kvar hos oss, ty det lider mot aftonen, och dagen nalkas redan sitt slut." Då gick han ditin och stannade kvar hos dem.
Luk 24:30  Och när han nu låg till bords med dem, tog han brödet och välsignade och bröt det och räckte åt dem.
Luk 24:31  Därvid öppnades deras ögon, så att de kände igen honom. Men då försvann han ur deras åsyn.
Luk 24:32  Och de sade till varandra: "Voro icke våra hjärtan brinnande i oss, när han talade med oss på vägen och uttydde skrifterna för oss?"
Luk 24:33  Och i samma stund stodo de upp och vände tillbaka till Jerusalem; och de funno där de elva församlade, så ock de andra som hade slutit sig till dem.
Luk 24:34  Och dessa sade: "Herren är verkligen uppstånden, och han har visat sig för Simon."
Luk 24:35  Då förtäljde de själva vad som hade skett på vägen, och huru han hade blivit igenkänd av dem, när han bröt brödet.
Luk 24:36  Medan de nu talade härom, stod han själv mitt ibland dem och sade till dem: "Frid vare med eder.
Luk 24:37  Då blevo de förfärade och uppfylldes av fruktan och trodde att det var en ande de sågo.
Luk 24:38  Men han sade till dem: "Varför ären I så förskräckta, och varför uppstiga tvivel i edra hjärtan?
Luk 24:39  Sen här mina händer och mina fötter, och sen att det är jag själv; ja, tagen på mig och sen. En ande har ju icke kött och ben, såsom I sen mig hava."
Luk 24:40  Och när han hade sagt detta, visade han dem sina händer och sina fötter.
Luk 24:41  Men då de ännu icke trodde, för glädjes skull, utan allenast förundrade sig, sade han till dem: "Haven I här något att äta?"
Luk 24:42  Då räckte de honom ett stycke stekt fisk och något av en honungskaka;
Luk 24:43  och han tog det och åt därav i deras åsyn.
Luk 24:44  Och han sade till dem: "Det är såsom jag sade till eder, medan jag ännu var bland eder, att allt måste fullbordas, som är skrivet om mig i Moses' lag och hos profeterna och i psalmerna."
Luk 24:45  Därefter öppnade han deras sinnen, så att de förstodo skrifterna.
Luk 24:46  Och han sade till dem: "Det är så skrivet, att Messias skulle lida och på tredje dagen uppstå från de döda,
Luk 24:47  och att bättring till syndernas förlåtelse i hans namn skulle predikas bland alla folk, och först i Jerusalem.
Luk 24:48  I kunnen vittna härom.
Luk 24:49  Och se, jag vill sända till eder vad min Fader har utlovat. Men I skolen stanna kvar här i staden, till dess I från höjden bliven beklädda med kraft."
Luk 24:50  Sedan förde han dem ut till Betania; och där lyfte han upp sina händer och välsignade dem.
Luk 24:51  Och medan han välsignade dem, försvann han ifrån dem och blev upptagen till himmelen.
Luk 24:52  Då tillbådo de honom och vände sedan tillbaka till Jerusalem, uppfyllda av stor glädje.
Luk 24:53  Och de voro sedan alltid i helgedomen och lovade Gud.


\end{document}