\begin{document}

\title{Johannes}


\chapter{1}

\par 1 I begynnelsen var Ordet, och Ordet var hos Gud, och Ordet var Gud.
\par 2 Detta var i begynnelsen hos Gud.
\par 3 Genom det har allt blivit till, och utan det har intet blivit till, som är till.
\par 4 I det var liv, och livet var människornas ljus.
\par 5 Och ljuset lyser i mörkret, och mörkret har icke fått makt därmed.
\par 6 En man uppträdde, sänd av Gud; hans namn var Johannes.
\par 7 Han kom såsom ett vittne, för att vittna om ljuset, på det att alla skulle komma till tro genom honom.
\par 8 Icke var han ljuset, men han skulle vittna om ljuset.
\par 9 Det sanna ljuset, det som lyser över alla människor, skulle nu komma i världen.
\par 10 I världen var han, och genom honom hade världen blivit till, men världen ville icke veta av honom.
\par 11 Han kom till sitt eget, och hans egna togo icke emot honom.
\par 12 Men åt alla dem som togo emot honom gav han makt att bliva Guds barn, åt dem som tro på hans namn;
\par 13 och de hava blivit födda, icke av blod, ej heller av köttslig vilja, ej heller av någon mans vilja, utan av Gud.
\par 14 Och Ordet vart kött och tog sin boning ibland oss, och vi sågo hans härlighet, vi sågo likasom en enfödd Sons härlighet från sin Fader, och han var full av nåd och sanning.
\par 15 Johannes vittnar om honom, han ropar och säger: "Det var om denne jag sade: 'Den som kommer efter mig, han är före mig; ty han var förr än jag.'"
\par 16 Av hans fullhet hava vi ju alla fått, ja, nåd utöver nåd;
\par 17 ty genom Moses blev lagen given, men nåden och sanningen hava kommit genom Jesus Kristus.
\par 18 Ingen har någonsin sett Gud; den enfödde Sonen, som är i Faderns sköte, han har kungjort vad Gud är.
\par 19 Och detta är vad Johannes vittnade, när judarna hade sänt till honom präster och leviter från Jerusalem för att fråga honom vem han var.
\par 20 Han svarade öppet och förnekade icke; han sade öppet: "Jag är icke Messias."
\par 21 Åter frågade de honom: "Vad är du då? Är du Elias?" Han svarade: "Det är jag icke." - "Är du Profeten?" Han svarade: "Nej."
\par 22 Då sade de till honom: "Vem är du då? Säg oss det, så att vi kunna giva dem svar, som hava sänt oss. Vad säger du om dig själv?"
\par 23 Han svarade: "Jag är rösten av en som ropar i öknen: 'Jämnen vägen för Herren', såsom profeten Esaias sade."
\par 24 Och männen voro utsända ifrån fariséerna.
\par 25 Och de frågade honom och sade till honom: "Varför döper du då, om du icke är Messias, ej heller Elias, ej heller Profeten?"
\par 26 Johannes svarade dem och sade: "Jag döper i vatten; men mitt ibland eder står en som I icke kännen:
\par 27 han som kommer efter mig (eller: vilken har varit före mig), vilkens skorem jag icke är värdig att upplösa."
\par 28 Detta skedde i Betania, på andra sidan Jordan, där Johannes döpte.
\par 29 Dagen därefter såg han Jesus nalkas; då sade han: "Se, Guds Lamm, som borttager världens synd!
\par 30 Om denne var det som jag sade: 'Efter mig kommer en man som är före mig; ty han var förr än jag.'
\par 31 Och jag kände honom icke; men för att han skall bliva uppenbar för Israel, därför är jag kommen och döper i vatten."
\par 32 Och Johannes vittnade och sade: "Jag såg Anden såsom en duva sänka sig ned från himmelen; och han förblev över honom.
\par 33 Och jag kände honom icke; men den som sände mig till att döpa i vatten, han sade till mig: 'Den över vilken du får se Anden sänka sig ned och förbliva, han är den som döper i helig ande.'
\par 34 Och jag har sett det, och jag har vittnat att denne är Guds Son."
\par 35 Dagen därefter stod Johannes åter där med två av sina lärjungar.
\par 36 När då Jesus kom gående, såg Johannes på honom och sade: "Se, Guds Lamm!"
\par 37 Och de två lärjungarna hörde hans ord och följde Jesus.
\par 38 Då vände sig Jesus om, och när han såg att de följde honom, frågade han dem: "Vad viljen I?" De svarade honom: "Rabbi" (det betyder mästare) "var bor du?"
\par 39 Han sade till dem: "Kommen och sen." Då gingo de med honom och sågo var han bodde; och de stannade den dagen hos honom. - Detta skedde vid den tionde timmen.
\par 40 En av de två som hade hört var Johannes sade, och som hade följt Jesus, var Andreas, Simon Petrus' broder.
\par 41 Denne träffade först sin broder Simon och sade till honom: "Vi hava funnit Messias" (det betyder detsamma som Kristus).
\par 42 Och han förde honom till Jesus. Då såg Jesus på honom och sade: "Du är Simon, Johannes' son; du skall heta Cefas" (det betyder detsamma som Petrus).
\par 43 Dagen därefter ville Jesus gå därifrån till Galileen, och han träffade då Filippus. Och Jesus sade till honom: "Följ mig."
\par 44 Och Filippus var från Betsaida, Andreas' och Petrus' stad.
\par 45 Filippus träffade Natanael och sade till honom: "Den som Moses har skrivit om i lagen och som profeterna hava skrivit om, honom hava vi funnit, Jesus, Josefs son, från Nasaret."
\par 46 Natanael sade till honom: "Kan något gott komma från Nasaret?" Filippus svarade honom: "Kom och se."
\par 47 När nu Jesus såg Natanael nalkas, sade han om honom: "Se, denne är en rätt israelit, i vilken icke finnes något svek."
\par 48 Natanael frågade honom: "Huru kunna du känna mig?" Jesus svarade och sade till honom: "Förrän Filippus kallade dig, såg jag dig, där du var under fikonträdet."
\par 49 Natanael svarade honom: "Rabbi, du är Guds Son, du är Israels konung."
\par 50 Jesus svarade och sade till honom: "Eftersom jag sade dig att jag såg dig under fikonträdet, tror du? Större ting än vad detta är skall du få se."
\par 51 Därefter sade han till honom: "Sannerligen, sannerligen säger jag eder: I skolen få se himmelen öppen och Guds änglar fara upp och fara ned över Människosonen."

\chapter{2}

\par 1 På tredje dagen var ett bröllop i Kana i Galileen, och Jesu moder var där.
\par 2 Också Jesus och hans lärjungar blevo bjudna till bröllopet.
\par 3 Och vinet begynte taga slut. Då sade Jesu moder till honom: "De hava intet vin."
\par 4 Jesus svarade henne: "Låt mig vara, moder; min stund är ännu icke kommen."
\par 5 Hans moder sade då till tjänarna: "Vadhelst han säger till eder, det skolen I göra."
\par 6 Nu stodo där sex stenkrukor, sådana som judarna hade för sina reningar; de rymde två eller tre bat-mått var.
\par 7 Jesus sade till dem: "Fyllen krukorna med vatten." Och de fyllde dem ända till brädden.
\par 8 Sedan sade han till dem: "Ösen nu upp och bären till övertjänaren." Och de gjorde så.
\par 9 Och övertjänaren smakade på vattnet, som nu hade blivit vin; och han visste icke varifrån det hade kommit, vilket däremot tjänarna visste, de som hade öst upp vattnet. Då kallade övertjänaren på brudgummen.
\par 10 och sade till honom: "Man brukar eljest alltid sätta fram det goda vinet, och sedan, när gästerna hava fått för mycket, det som är sämre. Du har gömt det goda vinet ända tills nu."
\par 11 Detta var det första tecknet som Jesus gjorde. Han gjorde det i Kana i Galileen och uppenbarade så sin härlighet; och hans lärjungar trodde på honom.
\par 12 Därefter begav han sig ned till Kapernaum med sin moder och sina bröder och sina lärjungar; och där stannade de några få dagar.
\par 13 Judarnas påsk var nu nära, och Jesus begav sig då upp till Jerusalem.
\par 14 Och när han fick i helgedomen se huru där sutto män som sålde fäkreatur och får och duvor, och huru växlare sutto där.
\par 15 Då gjorde han sig ett gissel av tåg och drev dem alla ut ur helgedomen, med får och fäkreatur, och slog ut växlarnas penningar och stötte omkull deras bord.
\par 16 Och till duvomånglarna sade han: "Tagen bort detta härifrån; gören icke min Faders hus till ett marknadshus."
\par 17 Hans lärjungar kommo då ihåg att det var skrivet: "Nitälskan för ditt hus skall förtära mig."
\par 18 Då togo judarna till orda och sade till honom: "Vad för tecken låter du oss se, eftersom du gör på detta sätt?"
\par 19 Jesus svarade och sade till dem: "Bryten ned detta tempel, så skall jag inom tre dagar låta det uppstå igen."
\par 20 Då sade judarna: "I fyrtiosex år har man byggt på detta tempel, och du skulle låta det uppstå igen inom tre dagar?"
\par 21 Men det var om sin kropps tempel han talade.
\par 22 Sedan, när han hade uppstått från de döda, kommo hans lärjungar ihåg att han hade sagt detta; och de trodde då skriften och det ord som Jesus hade sagt.
\par 23 Medan han nu var i Jerusalem, under påsken, vid högtiden, kommo många till tro på hans namn, när de sågo de tecken som han gjorde.
\par 24 Men själv betrodde sig Jesus icke åt dem, eftersom han kände alla
\par 25 och icke behövde någon annans vittnesbörd om människorna; ty av sig själv visste han vad i människan var.

\chapter{3}

\par 1 Men bland fariséerna var en man som hette Nikodemus, en av judarnas rådsherrar.
\par 2 Denne kom till Jesus om natten och sade till honom: "Rabbi, vi veta att det är från Gud du har kommit såsom lärare; ty ingen kan göra sådana tecken som du gör, om icke Gud är med honom."
\par 3 Jesus svarade och sade till honom: "Sannerligen, sannerligen säger jag dig: Om en människa icke bliver född på nytt, så kan hon icke få se Guds rike."
\par 4 Nikodemus sade till honom: "Huru kan en människa födas, när hon är gammal? Icke kan hon väl åter gå in i sin moders liv och födas?"
\par 5 Jesus svarade: "Sannerligen, sannerligen säger jag dig: Om en människa icke bliver född av vatten och ande, så kan hon icke komma in i Guds rike.
\par 6 Det som är fött av kött, det är kött; och det som är fött av Anden, det är ande.
\par 7 Förundra dig icke över att jag sade dig att I måsten födas på nytt.
\par 8 Vinden blåser vart den vill, och du hör dess sus, men du vet icke varifrån den kommer, eller vart den far; så är det med var och en som är född av Anden."
\par 9 Nikodemus svarade och sade till honom: "Huru kan detta ske?"
\par 10 Jesus svarade och sade till honom: "Är du Israels lärare och förstår icke detta?
\par 11 Sannerligen, sannerligen säger jag dig: Vad vi veta, det tala vi, och vad vi hava sett, det vittna vi om, men vårt vittnesbörd tagen I icke emot.
\par 12 Tron i icke, när jag talar till eder om jordiska ting, huru skolen I då kunna tro, när jag talar till eder om himmelska ting?
\par 13 Och likväl har ingen stigit upp till himmelen, utom den som steg ned från himmelen, Människosonen, som var i himmelen.
\par 14 Och såsom Moses upphöjde ormen i öknen, så måste Människosonen bliva upphöjd,
\par 15 så att var och en som tror skall i honom hava evigt liv.
\par 16 Ty så älskade Gud världen, att han utgav sin enfödde Son, på det att var och en som tror på honom skall icke förgås, utan hava evigt liv.
\par 17 Ty icke sände Gud sin Son i världen för att döma världen, utan för att världen skulle bliva frälst genom honom.
\par 18 Den som tror på honom, han bliver icke dömd, men den som icke tror, han är redan dömd, eftersom han icke tror på Guds enfödde Sons namn.
\par 19 Och detta är domen, att när ljuset hade kommit i världen, människorna dock älskade mörkret mer än ljuset, eftersom deras gärningar voro onda,
\par 20 Ty var och en som gör vad ont är, han hatar ljuset och kommer icke till ljuset, på det att hans gärningar icke skola bliva blottade.
\par 21 Men den som gör sanningen, han kommer till ljuset, för att det skall bliva uppenbart att hans gärningar äro gjorda i Gud."
\par 22 Därefter begav sig Jesus med sina lärjungar till den judiska landsbygden, och där vistades han med dem och döpte.
\par 23 Men också Johannes döpte, i Enon, nära Salim, ty där fanns mycket vatten; och folket kom dit och lät döpa sig.
\par 24 Johannes hade nämligen ännu icke blivit kastad i fängelse.
\par 25 Då uppstod mellan Johannes' lärjungar och en jude en tvist om reningen.
\par 26 Och de kommo till Johannes och sade till honom: "Rabbi, se, den som var hos dig på andra sidan Jordan, den som du har vittnat om, han döper, och alla komma till honom."
\par 27 Johannes svarade och sade: "En människa kan intet taga, om det icke bliver henne givet från himmelen."
\par 28 I kunnen själva giva mig det vittnesbördet att jag sade: 'Icke är jag Messias; jag är allenast sänd framför honom.'
\par 29 Brudgum är den som har bruden; men brudgummens vän, som står där och hör honom, han gläder sig storligen åt brudgummens röst. Den glädjen är mig nu given i fullt mått.
\par 30 Det är såsom sig bör att han växer till, och att jag förminskas. -
\par 31 Den som kommer ovanifrån, han är över alla; den som är från jorden, han är av jorden, och av jorden talar han. Ja, den som kommer från himmelen, han är över alla,
\par 32 och vad han har sett och hört, det vittnar han om; och likväl tager ingen emot hans vittnesbörd.
\par 33 Men om någon tager emot hans vittnesbörd, så bekräftar han därmed att Gud är sannfärdig.
\par 34 Ty den som Gud har sänt, han talar Guds ord; Gud giver nämligen icke Anden efter mått.
\par 35 Fadern älskar Sonen, och allt har han givit i hans hand.
\par 36 Den som tror på Sonen, han har evigt liv; men den som icke hörsammar Sonen, han skall icke få se livet, utan Guds vrede förbliver över honom."

\chapter{4}

\par 1 Men Herren fick nu veta att fariséerna hade hört hurusom Jesus vann flera lärjungar och döpte flera än Johannes;
\par 2 dock var det icke Jesus själv som döpte, utan hans lärjungar.
\par 3 Då lämnade han Judeen och begav sig åter till Galileen.
\par 4 Därvid måste han taga vägen genom Samarien.
\par 5 Så kom han till en stad i Samarien som hette Sykar, nära det jordstycke som Jakob gav åt sin son Josef.
\par 6 Och där var Jakobs brunn. Eftersom nu Jesus var trött av vandringen, satte han sig strax ned vid brunnen. Det var vid den sjätte timmen.
\par 7 Då kom en samaritisk kvinna för att hämta vatten. Jesus sade till henne: "Giv mig att dricka."
\par 8 Hans lärjungar hade nämligen gått in i staden för att köpa mat.
\par 9 Då sade den samaritiska kvinnan till honom: "Huru kan du, som är jude, bedja mig, som är en samaritisk kvinna, om något att dricka?" Judarna hava nämligen ingen umgängelse med samariterna.
\par 10 Jesus svarade och sade till henne: "Förstode du Guds gåva, och vem den är som säger till dig: 'Giv mig att dricka', så skulle i stället du hava bett honom, och han skulle då hava givit dig levande vatten."
\par 11 Kvinnan sade till honom: "Herre, du har ju intet att hämta upp vatten med, och brunnen är djup. Varifrån får du då det friska vattnet?"
\par 12 Icke är du väl förmer än vår fader Jakob, som gav oss brunnen och själv med sina barn och sin boskap drack ur den?"
\par 13 Jesus svarade och sade till henne: "Var och en som dricker av detta vatten, han bliver törstig igen;
\par 14 men den som dricker av det vatten som jag giver honom, han skall aldrig någonsin törsta, utan det vatten jag giver honom skall bliva i honom en källa vars vatten springer upp med evigt liv."
\par 15 Kvinnan sade till honom: "Herre, giv mig det vattnet, så att jag icke mer behöver törsta och komma hit för att hämta vatten."
\par 16 Han sade till henne: "Gå och hämta din man, och kom sedan tillbaka."
\par 17 Kvinnan svarade och sade: "Jag har ingen man." Jesus sade till henne: "Du har rätt i vad du säger, att du icke har någon man."
\par 18 Ty fem män har du haft, och den du nu har är icke din man; däri sade du sant.
\par 19 Då sade kvinnan till honom: "Herre, jag ser att du är en profet.
\par 20 Våra fäder hava tillbett på detta berg, men I sägen att i Jerusalem den plats finnes, där man bör tillbedja."
\par 21 Jesus sade till henne: "Tro mig, kvinna: den tid kommer, då det varken är på detta berg eller i Jerusalem som I skolen tillbedja Fadern.
\par 22 I tillbedjen vad I icke kännen, vi tillbedja vad vi känna - ty frälsningen kommer från judarna -
\par 23 men den tid skall komma, ja, den är redan inne, då sanna tillbedjare skola tillbedja Fadern i ande och sanning; ty sådana tillbedjare vill Fadern hava.
\par 24 Gud är ande, och de som tillbedja måste tillbedja i ande och sanning."
\par 25 Kvinnan sade till honom: "Jag vet att Messias skall komma, han som ock kallas Kristus; när han kommer, skall han förkunna oss allt."
\par 26 Jesus svarade henne: "Jag, som talar med dig, är den du nu nämnde."
\par 27 I detsamma kommo hans lärjungar; och de förundrade sig över att han talade med en kvinna. Dock frågade ingen vad han ville henne, eller varför han talade med henne.
\par 28 Men kvinnan lät sin kruka stå och gick in i staden och sade till folket:
\par 29 "Kommen och sen en man som har sagt mig allt vad jag har gjort. Månne icke han är Messias?"
\par 30 Då gingo de ut ur staden och kommo till honom.
\par 31 Under tiden bådo lärjungarna honom och sade: "Rabbi, tag och ät."
\par 32 Men han svarade dem: "Jag har mat att äta som I icke veten om."
\par 33 Då sade lärjungarna till varandra: "Kan väl någon hava burit mat till honom?"
\par 34 Jesus sade till dem: "Min mat är att göra dens vilja, som har sänt mig, och att fullborda hans verk."
\par 35 I sägen ju att det ännu är fyra månader innan skördetiden kommer. Men se, jag säger eder: Lyften upp edra ögon, och sen på fälten, huru de hava vitnat till skörd.
\par 36 Redan nu får den som skördar uppbära sin lön och samla in frukt till evigt liv; så kunna den som sår och den som skördar tillsammans glädja sig.
\par 37 Ty här sannas det ordet, att en är den som sår och en annan den som skördar.
\par 38 Jag har sänt eder att skörda, där I icke haven arbetat. Andra hava arbetat, och I haven fått gå in i deras arbete."
\par 39 Och många samariter från den staden kommo till tro på honom för kvinnans ords skull, då hon vittnade att han hade sagt henne allt vad hon hade gjort.
\par 40 När sedan samariterna kommo till honom, både de honom att stanna kvar hos dem. Så stannade han där i två dagar.
\par 41 Och långt flera kommo då till tro för hans egna ords skull.
\par 42 Och de sade till kvinnan: "Nu är det icke mer för dina ords skull som vi tro, ty vi hava nu själva hört honom, och vi veta nu att han i sanning är världens Frälsare."
\par 43 Men efter de två dagarna gick han därifrån till Galileen.
\par 44 Ty Jesus vittnade själv att en profet icke är aktad i sitt eget fädernesland.
\par 45 När han nu kom till Galileen, togo galiléerna vänligt emot honom, eftersom de hade sett allt vad han hade gjort i Jerusalem vid högtiden. Också de hade nämligen varit där vid högtiden.
\par 46 Så kom han åter till Kana i Galileen, där han hade gjort vattnet till vin. I Kapernaum fanns då en man i konungens tjänst, vilkens son låg sjuk.
\par 47 När han nu hörde att Jesus hade kommit från Judeen till Galileen, begav han sig åstad till honom och bad att han skulle komma ned och bota hans son; ty denne låg för döden.
\par 48 Då sade Jesus till honom: "Om I icke sen tecken och under, så tron I icke."
\par 49 Mannen sade till honom: "Herre, kom ned, förrän mitt barn dör."
\par 50 Jesus svarade honom: "Gå, din son får leva." Då trodde mannen det ord som Jesus sade till honom, och gick.
\par 51 Och medan han ännu var på vägen hem, mötte honom hans tjänare och sade: "Din son kommer att leva."
\par 52 Då frågade han dem vid vilken timme det hade blivit bättre med honom. De svarade honom: "I går vid den sjunde timmen lämnade febern honom."
\par 53 Då märkte fadern att det hade skett just den timme då Jesus sade till honom: "Din son får leva." Och han kom till tro, så ock hela hans hus.
\par 54 Detta var nu åter ett tecken, det andra i ordningen som Jesus gjorde, sedan han hade kommit från Judeen till Galileen.

\chapter{5}

\par 1 Därefter inföll en av judarnas högtider, och Jesus for upp till Jerusalem.
\par 2 Vid Fårporten i Jerusalem ligger en damm, på hebreiska kallad Betesda, och invid den finnas fem pelargångar.
\par 3 I dessa lågo många sjuka, blinda, halta, förtvinade. som väntade på att vattnet skulle uppröras.
\par 4 Ty en ängel steg tidtals ned i dammen och upprörde vattnet. Den som nu först steg ned i vattnet, sedan det hade blivit upprört, han blev frisk, med vilken jukdom han än var behäftad.
\par 5 Där fanns nu en man som hade varit sjuk i trettioåtta år.
\par 6 Då Jesus fick se denne, där han låg, och fick veta att han redan lång tid hade varit sjuk, sade han till honom: "Vill du bliva frisk?"
\par 7 Den sjuke svarade honom: "Herre, jag har ingen som hjälper mig ned i dammen, när vattnet har kommit i rörelse; och så stiger en annan ditned före mig, medan jag ännu är på väg."
\par 8 Jesus sade till honom: "Stå upp, tag din säng och gå."
\par 9 Och strax blev mannen frisk och tog sin säng och gick. Men det var sabbat den dagen.
\par 10 Därför sade judarna till mannen som hade blivit botad: "Det är sabbat; det är icke lovligt för dig att bära sängen."
\par 11 Men han svarade dem: "Den som gjorde mig frisk, han sade till mig: 'Tag din säng och gå.'"
\par 12 Då frågade de honom: "Vem var den mannen som sade till dig att du skulle taga sin säng och gå?"
\par 13 Men mannen som hade blivit botad visste icke vem det var; ty Jesus hade dragit sig undan, eftersom mycket folk var där på platsen. -
\par 14 Sedan träffade Jesus honom i helgedomen och sade till honom: "Se, du har blivit frisk; synda icke härefter, på det att icke något värre må vederfaras dig."
\par 15 Mannen gick då bort och omtalade för judarna, att det var Jesus som hade gjort honom frisk.
\par 16 Därför förföljde nu judarna Jesus (och sökte att döda honom.), eftersom han gjorde sådant på sabbaten.
\par 17 Men han svarade dem: "Min Fader verkar ännu alltjämt; så verkar ock jag."
\par 18 Och därför stodo judarna ännu mer efter att döda honom, eftersom han icke allenast ville göra sabbaten om intet, utan ock kallade Gud sin Fader och gjorde sig själv lik Gud.
\par 19 Då talade Jesus åter och sade till dem: "Sannerligen, sannerligen säger jag eder: Sonen kan icke göra något av sig själv, utan han gör allenast vad han ser Fadern göra; ty vad han gör, det gör likaledes ock Sonen.
\par 20 Ty Fadern älskar Sonen och låter honom se allt vad han själv gör; och större gärningar, än dessa äro, skall han låta honom se, så att I skolen förundra eder.
\par 21 Ty såsom Fadern uppväcker döda och gör dem levande, så gör ock Sonen levande vilka han vill.
\par 22 Icke heller dömer Fadern någon, utan all dom har han överlåtit åt Sonen,
\par 23 för att alla skola ära Sonen såsom de ära Faderns. Den som icke ärar Sonen, han ärar icke heller Fadern, som har sänt honom.
\par 24 Sannerligen, sannerligen säger jag eder: Den som hör mina ord och tror honom som har sänt mig, han har evigt liv och kommer icke under någon dom, utan har övergått från döden till livet.
\par 25 Sannerligen säger jag eder: Den stund kommer, jag, den är redan inne, så de döda skola höra Guds Sons röst, och de som höra den skola bliva levande.
\par 26 Ty såsom Fadern har liv i sig själv, så har han ock givit åt Sonen att hava liv i sig själv.
\par 27 Och han har givit honom makt att hålla dom, eftersom han är Människoson.
\par 28 Förundren eder icke över detta. Ty den stund kommer, då alla som äro i gravarna skola höra hans röst
\par 29 och gå ut ur dem: de som hava gjort vad gott är skola uppstå till liv, och de som hava gjort vad ont är skola uppstå till dom.
\par 30 Jag kan icke göra något av mig själv. Såsom jag hör, så dömer jag; och min dom är rättvis, ty jag söker icke min vilja, utan dens vilja, som har sänt mig.
\par 31 Om jag själv vittnar om mig, så gäller icke mitt vittnesbörd.
\par 32 Men det är en annan som vittnar om mig, och jag vet att hans vittnesbörd om mig är sant.
\par 33 I haven sänt bud till Johannes, och han har vittnat för sanningen,
\par 34 Dock, det är icke av någon människa som jag tager emot vittnesbörd om mig; men jag säger detta, för att I skolen bliva frälsta.
\par 35 Han var den brinnande, skinande lampan, och för en liten stund villen I fröjdas i dess ljus.
\par 36 Men jag har ett vittnesbörd om mig, som är förmer än Johannes' vittnesbörd: de gärningar som Fadern har givit mig att fullborda, just de gärningar som jag gör, de vittna om mig, att Fadern har sänt mig.
\par 37 Ja, Fadern, som har sänt mig, han har själv vittnat om mig. Hans röst haven I aldrig någonsin hört, ej heller haven I sett hans gestalt,
\par 38 och hans ord haven I icke låtit förbliva i eder. Ty den han har sänt, honom tron I icke.
\par 39 I rannsaken skrifterna, därför att I menen eder i dem hava evigt liv; och det är dessa som vittna om mig.
\par 40 Men I viljen icke komma till mig för att få liv.
\par 41 Jag tager icke emot pris av människor;
\par 42 men jag känner eder och vet att I icke haven Guds kärlek i eder.
\par 43 Jag har kommit i min Faders namn, och I tagen icke emot mig; kommer en annan i sitt eget namn, honom skolen I nog mottaga.
\par 44 Huru skullen I kunna tro, I som tagen emot pris av varandra och icke söken det pris som kommer från honom som allena är Gud?
\par 45 Menen icke att det är jag som skall anklaga eder hos Fadern. Den som anklagar eder är Moses, han till vilken I sätten edert hopp.
\par 46 Trodden I Moses, så skullen I ju tro mig, ty om mig har han skrivit.
\par 47 Men tron I icke hans skrifter, huru skolen I då kunna tro mina ord?"

\chapter{6}

\par 1 Därefter for Jesus över Galileiska sjön, "Tiberias' sjö".
\par 2 Och mycket folk följde efter honom, därför att de sågo de tecken som han gjorde med de sjuka.
\par 3 Men Jesus gick upp på berget och satte sig där med sina lärjungar.
\par 4 Och påsken, judarnas högtid, var nära.
\par 5 Då nu Jesus lyfte upp sina ögon och såg att mycket folk kom till honom, sade han till Filippus: "Varifrån skola vi köpa bröd, så att dessa få äta?"
\par 6 Men detta sade han för att sätta honom på prov, ty själv visste han vad han skulle göra.
\par 7 Filippus svarade honom: "Bröd för två hundra silverpenningar vore icke nog för att var och en skulle få ett litet stycke."
\par 8 Då sade till honom en annan av hans lärjungar, Andreas, Simon Petrus' broder:
\par 9 "Här är en gosse som har fem kornbröd och två fiskar; men vad förslår det för så många?"
\par 10 Jesus sade: "Låten folket lägga sig här." Och på det stället var mycket gräs. Då lägrade sig männen där, och deras antal var vid pass fem tusen.
\par 11 Därefter tog Jesus bröden och tackade Gud och delade ut åt dem som hade lagt sig ned där, likaledes ock av fiskarna, så mycket de ville hava.
\par 12 Och när de voro mätta, sade han till sina lärjungar: "Samlen tillhopa de överblivna styckena, så att intet förfares."
\par 13 Då samlade de dem tillhopa och fyllde tolv korgar med stycken, som av de fem kornbröden hade blivit över efter dem som hade ätit.
\par 14 Då nu människorna hade det tecken som han hade gjort, sade de: "Denne är förvisso Profeten som skulle komma i världen."
\par 15 När då Jesus märkte att de tänkte komma och med våld föra honom med sig och göra honom till konung, drog han sig åter undan till berget, helt allena.
\par 16 Men när det blev afton, gingo hans lärjungar ned till sjön
\par 17 och stego i en båt för att fara över sjön till Kapernaum. Det hade då redan blivit mörkt, och Jesus hade ännu icke kommit till dem;
\par 18 och sjön gick hög, ty det blåste hårt.
\par 19 När de så hade rott vid pass tjugufem eller trettio stadier, fingo de se Jesus komma gående på sjön och nalkas båten. Då blevo de förskräckta.
\par 20 Men han sade till dem: "Det är jag; varen icke förskräckta."
\par 21 De ville då taga honom upp i båten; och strax var båten framme vid landet dit de foro.
\par 22 Dagen därefter hände sig detta. Folket som stod kvar på andra sidan sjön hade lagt märke till att där icke fanns mer än en enda båt, och att Jesus icke hade stigit i den båten med sina lärjungar, utan att lärjungarna hade farit bort allena.
\par 23 Andra båtar hade likväl kommit från Tiberias och lagt till nära det ställe där folket bespisades efter det att Herren hade uttalat tacksägelsen.
\par 24 När alltså folket nu såg att Jesus icke var där, ej heller hans lärjungar, stego de själva i båtarna och foro till Kapernaum för att söka efter Jesus.
\par 25 Och då de funno honom där på andra sidan sjön, frågade de honom: "Rabbi, när kom du hit?"
\par 26 Jesus svarade dem och sade: "Sannerligen, sannerligen säger jag eder: I söken mig icke därför att I haven sett tecken, utan därför att I fingen äta av bröden och bleven mätta.
\par 27 Verken icke för att få den mat som förgås, utan för att få den mat som förbliver och har med sig evigt liv, den som Människosonen skall giva eder; ty honom har Fadern, Gud själv, låtit undfå sitt insegel."
\par 28 Då sade de till honom: "Vad skola vi göra för att utföra Guds gärningar?"
\par 29 Jesus svarade och sade till dem: "Detta är Guds gärning, att I tron på den han har sänt."
\par 30 De sade till honom: "Vad för tecken gör du då? Låt oss se något tecken, så att vi kunna tro dig. Vilken gärning utför du?
\par 31 Våra fäder fingo äta manna i öknen, såsom det är skrivet: 'Han gav dem bröd från himmelen att äta.'"
\par 32 Då svarade Jesus dem: "Sannerligen, sannerligen säger jag eder: Det är icke Moses som har givit eder brödet från himmelen, men det är min Fader som giver eder det rätta brödet från himmelen.
\par 33 Ty Guds bröd är det bröd som kommer ned från himmelen och giver världen liv."
\par 34 Då sade de till honom: "Herre, giv oss alltid det brödet."
\par 35 Jesus svarade: "Jag är livets bröd. Den som kommer till mig, han skall aldrig hungra, och den som tror på mig, han skall aldrig törsta.
\par 36 Men det är såsom jag har sagt eder: fastän I haven sett mig, tron I dock icke.
\par 37 Allt vad min Fader giver mig, det kommer till mig; och den som kommer till mig, honom skall jag sannerligen icke kasta ut.
\par 38 Ty jag har kommit ned från himmelen, icke för att göra min vilja, utan för att göra dens vilja, som har sänt mig.
\par 39 Och detta är dens vilja, som har sänt mig, att jag icke skall låta någon enda gå förlorad av dem som han har givit mig, utan att jag skall låta dem uppstå på den yttersta dagen.
\par 40 Ja, detta är min Faders vilja, att var och en som ser Sonen och tror på honom, han skall hava evigt liv, och att jag skall låta honom uppstå på den yttersta dagen."
\par 41 Då knorrade judarna över honom, därför att han hade sagt: "Jag är det bröd som har kommit ned från himmelen."
\par 42 Och de sade: "Är denne icke Jesus, Josefs son, vilkens fader och moder vi känna? Huru kan han då säga: 'Jag har kommit ned från himmelen'?"
\par 43 Jesus svarade och sade till dem: "Knorren icke eder emellan.
\par 44 Ingen kan komma till mig, om icke Fadern, som har sänt mig, drager honom; och jag skall låta honom uppstå på den yttersta dagen.
\par 45 Det är skrivet hos profeterna: 'De skola alla hava fått lärdom av Gud.' Var och en som har lyssnat till Fadern och lärt av honom, han kommer till mig.
\par 46 Icke som om någon skulle hava sett Fadern, utom den som är från Gud; han har sett Fadern.
\par 47 Sannerligen, sannerligen säger jag eder: Den som tror, han har evigt liv.
\par 48 Jag är livets bröd.
\par 49 Edra fäder åto manna i öknen, och de dogo.
\par 50 Men med det bröd som kommer ned från himmelen är det så, att om någon äter därav, så skall han icke dö.
\par 51 Jag är det levande brödet som har kommit ned från himmelen. Om någon äter av det brödet, så skall han leva till evig tid. Och det bröd som jag skall giva är mitt kött; och jag giver det, för att världen skall leva."
\par 52 Då tvistade judarna med varandra och sade: "Huru skulle denne kunna giva oss sitt kött att äta?"
\par 53 Jesus sade då till dem: "Sannerligen, sannerligen säger jag eder: Om I icke äten Människosonens kött och dricken hans blod, så haven I icke liv i eder.
\par 54 Den som äter mitt kött och dricker mitt blod, han har evigt liv, och jag skall låta honom uppstå på den yttersta dagen.
\par 55 Ty mitt kött är sannskyldig mat, och mitt blod är sannskyldig dryck.
\par 56 Den som äter mitt kött och dricker mitt blod, han förbliver i mig, och jag förbliver i honom.
\par 57 Såsom Fadern, han som är den levande, har sänt mig, och såsom jag lever genom Fadern, så skall ock den som äter mig leva genom mig.
\par 58 Så är det med det bröd som har kommit ned från himmelen. Det är icke såsom det fäderna fingo äta, vilka sedan dogo; den som äter detta bröd, han skall leva till evig tid."
\par 59 Detta sade han, när han undervisade i synagogan i Kapernaum.
\par 60 Många av hans lärjungar, som hörde detta, sade då: "Detta är ett hårt tal; vem står ut med att höra på honom?"
\par 61 Men Jesus visste inom sig att hans lärjungar knorrade över detta; och han sade till dem: "Är detta för eder en stötesten?
\par 62 Vad skolen I då säga, om I fån se Människosonen uppstiga dit där han förut var? -
\par 63 Det är anden som gör levande; köttet är till intet gagneligt. De ord som jag har talat till eder äro ande och äro liv.
\par 64 Men bland eder finnas några som icke tro." Jesus visste nämligen från begynnelsen vilka de voro som icke trodde, så ock vilken den var som skulle förråda honom.
\par 65 Och han tillade: "Fördenskull har jag sagt eder att ingen kan komma till mig, om det icke bliver honom givet av Fadern."
\par 66 För detta tals skull drogo sig många av hans lärjungar tillbaka, så att de icke längre vandrade med honom.
\par 67 Då sade Jesus till de tolv: "Icke viljen väl också I gå bort?"
\par 68 Simon Petrus svarade honom: "Herre, till vem skulle vi gå? Du har det eviga livets ord,
\par 69 och vi tro och förstå att du är Guds helige."
\par 70 Jesus svarade dem: "Har icke jag själv utvalt eder, I tolv? Och likväl är en av eder en djävul."
\par 71 Detta sade han om Judas, Simon Iskariots son; ty det var denne som skulle förråda honom, och han var en av de tolv.

\chapter{7}

\par 1 Därefter vandrade Jesus omkring i Galileen, ty i Judeen ville han icke vandra omkring, då nu judarna stodo efter att döda honom.
\par 2 Men judarnas lövhyddohögtid var nu nära.
\par 3 Då sade hans bröder till honom: "Begiv dig härifrån och gå till Judeen, så att också dina lärjungar få se de gärningar som du gör.
\par 4 Ty ingen som vill vara känd bland människor utför sitt verk i hemlighet. Då du nu gör sådana gärningar, så träd öppet fram för världen."
\par 5 Det var nämligen så, att icke ens hans bröder trodde på honom.
\par 6 Då sade Jesus till dem: "Min tid är ännu icke kommen, men för eder är tiden alltid läglig.
\par 7 Världen kan icke hata eder, men mig hatar hon, eftersom jag vittnar om henne, att hennes gärningar äro onda.
\par 8 Gån I upp till högtiden; jag är icke stadd på väg upp till denna högtid, ty min tid är ännu icke fullbordad."
\par 9 Detta sade han till dem och stannade så kvar i Galileen.
\par 10 Men när hans bröder hade gått upp till högtiden, då gick också han ditupp, dock icke öppet, utan likasom i hemlighet.
\par 11 Och judarna sökte efter honom under högtiden och sade: "Var är han?"
\par 12 Och bland folket talades i tysthet mycket om honom. Somliga sade: "Han är en rättsinnig man", men andra sade: "Nej, han förvillar folket."
\par 13 Dock talade ingen öppet om honom, av fruktan för judarna.
\par 14 Men när redan halva högtiden var förliden, gick Jesus upp i helgedomen och undervisade.
\par 15 Då förundrade sig judarna och sade: "Varifrån har denne sin lärdom, han som icke har fått undervisning?"
\par 16 Jesus svarade dem och sade: "Min lära är icke min, utan hans som har sänt mig.
\par 17 Om någon vill göra hans vilja, så skall han förstå om denna lära är från Gud, eller om jag talar av mig själv.
\par 18 Den som talar av sig själv, han söker sin egen ära; men den som söker dens ära, som har sänt honom, han är sannfärdig, och orättfärdighet finnes icke i honom. -
\par 19 Har icke Moses givit eder lagen? Och likväl fullgör ingen av eder lagen. Varför stån I efter att döda mig?"
\par 20 Folket svarade: "Du är besatt av en ond ande. Vem står efter att döda dig?"
\par 21 Jesus svarade och sade till dem: "En gärning allenast gjorde jag, och alla förundren I eder över den.
\par 22 Moses har givit eder omskärelsen - icke som om den vore ifrån Moses, ty den är ifrån fäderna - och så omskären I människor också på en sabbat.
\par 23 Om nu en människa undfår omskärelsen på en sabbat, för att Moses' lag icke skall göras om intet, huru kunnen I då vredgas på mig, därför att jag på en sabbat gjorde en människa hel och frisk?
\par 24 Dömen icke efter skenet, utan dömen en rätt dom."
\par 25 Då sade några av folket i Jerusalem: "Är det icke denne som de stå efter att döda?
\par 26 Och ändå får han tala fritt, utan att de säga något till honom. Hava då rådsherrarna verkligen blivit förvissade om att denne är Messias?
\par 27 Dock, denne känna vi, och vi veta varifrån han är; men när Messias kommer, känner ingen varifrån han är."
\par 28 Då sade Jesus med hög röst, där han undervisade i helgedomen: "Javäl, I kännen mig, och I veten varifrån jag är. Likväl har jag icke kommit av mig själv, men han som har sänt mig är en som verkligen har myndighet att sända, han som I icke kännen.
\par 29 Men jag känner honom, ty från honom är jag kommen, och han har sänt mig."
\par 30 Då ville de gripa honom; dock kom ingen med sin hand vid honom, ty hans stund var ännu icke kommen.
\par 31 Men många av folket trodde på honom, och de sade: "Icke skall väl Messias, när han kommer, göra flera tecken än denne har gjort?"
\par 32 Sådant fingo fariséerna höra folket i tysthet tala om honom. Då sände översteprästerna och fariséerna ut rättstjänare för att gripa honom.
\par 33 Men Jesus sade: "Ännu en liten tid är jag hos eder; sedan går jag bort till honom som har sänt mig.
\par 34 I skolen då söka efter mig, men I skolen icke finna mig, och där jag är, dit kunnen I icke komma."
\par 35 Då sade judarna till varandra: "Vart tänker denne gå, eftersom vi icke skola kunna finna honom? Månne han tänker gå till dem som bo kringspridda bland grekerna? Tänker han då undervisa grekerna?
\par 36 Vad betyder det ord som han sade: 'I skolen söka efter mig, men I skolen icke finna mig, och där jag är, dit kunnen I icke komma'?"
\par 37 På den sista dagen i högtiden, som ock var den förnämsta, stod Jesus där och ropade och sade: "Om någon törstar, så komme han till mig och dricke.
\par 38 Den som tror på mig, av hans innersta skola strömmar av levande vatten flyta fram, såsom skriften säger."
\par 39 Detta sade han om Anden, vilken de som trodde på honom skulle undfå; ty ande var då ännu icke given, eftersom Jesus ännu icke hade blivit förhärligad.
\par 40 Några av folket, som hörde dessa ord, sade då: "Denne är förvisso Profeten."
\par 41 Andra sade: "Han är Messias." Andra åter sade: "Icke kommer väl Messias från Galileen?
\par 42 Säger icke skriften att Messias skall komma av Davids säd och från den lilla staden Betlehem, där David bodde?
\par 43 Så uppstodo för hans skull stridiga meningar bland folket,
\par 44 och somliga av dem ville gripa honom; dock kom ingen med sin hand vid honom.
\par 45 När sedan rättstjänarna kommo tillbaka till översteprästerna och fariséerna, frågade dessa dem: "Varför haven I icke fört honom hit?"
\par 46 Tjänarna svarade: "Aldrig har någon människa talat, som den mannen talar."
\par 47 Då svarade fariséerna dem: "Haven nu också I blivit förvillade?
\par 48 Har då någon av rådsherrarna trott på honom? Eller någon av fariséerna?
\par 49 Nej; men detta folk, som icke känner lagen, det är förbannat.
\par 50 Då sade Nikodemus till dem, han som förut hade besökt honom, och som själv var en av dem:
\par 51 "Icke dömer väl vår lag någon, utan att man först har förhört honom och utrönt vad han förehar?"
\par 52 De svarade och sade till honom: "Kanske också du är från Galileen? Rannsaka, så skall du finna att ingen profet kommer från Galileen."
\par 53 Och de gingo hem, var och en till sitt. om natten.

\chapter{8}

\par 1 Och Jesus gick ut till Oljeberget.
\par 2 Men i dagbräckningen kom han åter till helgedomen (och allt folket kom till honom, och han satte sig och lärde dem.).
\par 3 Då förde översteprästerna och fariséerna dit en kvinna som hade blivit beträdd med äktenskapsbrott; och när de hade lett henne fram,
\par 4 sade de till honom: "Mästare, denna kvinna har på bar gärning blivit beträdd med äktenskapsbrott.
\par 5 Nu bjuder Moses i lagen att sådana skola stenas. Vad säger då du?"
\par 6 Detta sade de för att snärja honom, på det att de skulle få något att anklaga honom för. Då böjde Jesus sig ned och skrev med fingret på jorden.
\par 7 Men när de stodo fast vid sin fråga, reste han sig upp och sade till dem: "Den av eder som är utan synd, han kaste första stenen på henne."
\par 8 Sedan böjde han sig åter ned och skrev på jorden.
\par 9 När de hörde detta (och kände sig överbevisade av samvetet.), gingo de ut, den ene efter den andre, först de äldsta, och Jesus blev lämnad allena med kvinnan, som stod där kvar.
\par 10 Då såg Jesus upp och sade till kvinnan: "Var äro de andra? Har ingen dömt dig?"
\par 11 Hon svarade: "Herre, ingen." Då sade han till henne: "Icke heller jag dömer dig. Gå, och synda icke härefter."
\par 12 Åter talade Jesus till dem och sade: "Jag är världens ljus; den som följer mig, han skall förvisso icke vandra i mörkret, utan skall hava livets ljus."
\par 13 Då sade fariséerna till honom: "Du vittnar om dig själv; ditt vittnesbörd gäller icke."
\par 14 Jesus svarade och sade till dem: "Om jag än vittnar om mig själv, så gäller dock mitt vittnesbörd, ty jag vet varifrån jag har kommit, och vart jag går; men I veten icke varifrån jag kommer, eller vart jag går.
\par 15 I dömen efter köttet; jag dömer ingen.
\par 16 Och om jag än dömer, så är min dom en rätt dom, ty jag är därvid icke ensam, utan med mig är han som har sänt mig.
\par 17 I eder lag är ju ock skrivet att vad två människor vittna, det gäller såsom sant.
\par 18 Här är nu jag som vittnar om mig; om mig vittnar också Fadern, som har sänt mig.
\par 19 Då sade de till honom: "Var är då din Fader?" Jesus svarade: "I kännen varken mig eller min Fader. Om I känden mig, så känden I ock min Fader."
\par 20 Det var på det ställe där offerkistorna stodo som han talade dessa ord, medan han undervisade i helgedomen; men ingen bar hand på honom, ty hans stund var ännu icke kommen.
\par 21 Åter sade han till dem: "Jag går bort, och I skolen då söka efter mig; men I skolen dö i eder synd. Dig jag går, dit kunnen I icke komma."
\par 22 Då sade judarna: "Icke vill han väl dräpa sig själv, eftersom han säger: 'Dit jag går, dit kunnen I icke komma'?"
\par 23 Men han svarade dem: "I ären härnedifrån, jag är ovanifrån; I ären av denna världen, jag är icke av denna världen.
\par 24 Därför sade jag till eder att I skullen dö i edra synder; ty om I icke tron att jag är den jag är, så skolen I dö i edra synder."
\par 25 Då frågade de honom: "Vem är du då?" Jesus svarade dem: "Det som jag redan från begynnelsen har uttalat för eder.
\par 26 Mycket har jag ännu att tala och att döma i fråga om eder. Men han som har sänt mig är sannfärdig, och vad jag har hört av honom, det talar jag ut inför världen."
\par 27 Men de förstodo icke att det var om Fadern som han talade till dem.
\par 28 Då sade Jesus: "När I haven upphöjt Människosonen, då skolen I första att jag är den jag är, och att jag icke gör något av mig själv, utan talar detta såsom Fadern har lärt mig.
\par 29 Och han som har sänt mig är med mig; han har icke lämnat mig allena, eftersom jag alltid gör vad honom behagar."
\par 30 När han talade detta, kommo många till tro på honom.
\par 31 Då sade Jesus till de judar som hade satt tro till honom: "Om I förbliven i mitt ord, så ären I i sanning mina lärjungar;
\par 32 Och I skolen då förstå sanningen, och sanningen skall göra eder fria."
\par 33 De svarade honom: "Vi äro Abrahams säd och hava aldrig varit trälar under någon. Huru kan du då säga: 'I skolen bliva fria'?"
\par 34 Jesus svarade dem: "Sannerligen, sannerligen säger jag eder: Var och en som gör synd, han är syndens träl.
\par 35 Men trälen får icke förbliva i huset för alltid; sonen får förbliva där för alltid.
\par 36 Om nu Sonen gör eder fria, så bliven i verkligen fria.
\par 37 Jag vet att I ären Abrahams säd; men I stån efter att döda mig, eftersom mitt ord icke får någon ingång i eder.
\par 38 Jag talar vad jag har sett hos min Fader; så gören ock I vad I haven hört av eder fader."
\par 39 De svarade och sade till honom: "Vår fader är ju Abraham." Jesus svarade till dem: "Ären I Abrahams barn, så gören ock Abrahams gärningar.
\par 40 Men nu stån I efter att döda mig, en man som har sagt eder sanningen, såsom jag har hört den av Gud. Så handlade icke Abraham.
\par 41 Nej, I gören eder faders gärningar." De sade till honom: "Vi äro icke födda i äktenskapsbrott. Vi hava Gud till fader och ingen annan."
\par 42 Jesus svarade dem: "Vore Gud eder fader, så älskaden I ju mig, ty från Gud har jag utgått, och från honom är jag kommen. Ja, jag har icke kommit av mig själv, utan det är han som har sänt mig.
\par 43 Varför fatten I då icke vad jag talar? Jo, därför att I icke 'stån ut med' att höra på mitt ord.
\par 44 I haven djävulen till eder fader, och vad eder fader har begär till, det viljen i göra. Han har varit en mandråpare från begynnelsen, och i sanningen står han icke, ty sanning finnes icke i honom. När han talar lögn, då talar han av sitt eget, ty han är en lögnare, ja, lögnens fader.
\par 45 Men mig tron I icke, just därför att jag talar sanning.
\par 46 Vilken av eder kan överbevisa mig om någon synd? Om jag alltså talar sanning, varför tron I mig då icke?
\par 47 Den som är av Gud, han lyssnar till Guds ord; och det är därför att I icke ären av Gud som I icke lyssnen därtill.
\par 48 Judarna svarade och sade till honom: "Hava vi icke rätt, då vi säga att du är en samarit och är besatt av en ond ande?"
\par 49 Jesus svarade: "Jag är icke besatt av någon ond ande; fastmer hedrar jag min Fader. I åter skymfen mig.
\par 50 Men jag söker icke min egen ära; en finnes dock som söker den och som dömer.
\par 51 Sannerligen, sannerligen säger jag eder: Den som håller mitt ord, han skall aldrig någonsin se döden."
\par 52 Judarna sade till honom: "Nu förstå vi att du är besatt av en ond ande. Abraham har dött, så ock profeterna, och likväl säger du: 'Den som håller mitt ord, han skall aldrig någonsin smaka döden.'
\par 53 Icke är väl du förmer än vår Fader Abraham? Och han har ju dött. Profeterna hava också dött. Till vem gör du då dig själv?"
\par 54 Jesus svarade: "Om jag själv ville skaffa mig ära, så vore min ära intet; men det är min Fader som förlänar mig ära, han som I säger vara eder Gud.
\par 55 Dock, I haven icke lärt känna honom, men jag känner honom; och om jag sade att jag icke kände honom, så bleve jag en lögnare likasom I; men jag känner honom och håller hans ord.
\par 56 Abraham, eder fader, fröjdade sig över att han skulle få se min dag. Han fick se den och blev glad."
\par 57 Då sade judarna till honom: "Femtio år gammal är du icke ännu, och Abraham har du sett!"
\par 58 Jesus sade till dem: "Sannerligen, sannerligen säger jag eder: Förrän Abraham blev till, är jag."
\par 59 Då togo de upp stenar för att kasta på honom. Men Jesus gömde sig undan och gick sedan ut ur helgedomen.

\chapter{9}

\par 1 När han nu gick vägen fram, fick han se en man som var född blind.
\par 2 Då frågade hans lärjungar honom och sade: "Rabbi, vilken har syndat, denne eller hans föräldrar, så att han har blivit född blind?"
\par 3 Jesus svarade: "Det är varken denne som har syndat eller hans föräldrar, utan så har skett, för att Guds gärningar skulle uppenbaras på honom.
\par 4 Medan dagen varar, måste vi göra dens gärningar, som har sänt mig; natten kommer, då ingen kan verka.
\par 5 Så länge jag är i världen, är jag världens ljus."
\par 6 När han hade sagt detta, spottade han på jorden och gjorde en deg av spotten och lade degen på mannens ögon
\par 7 och sade till honom: "Gå bort och två dig i dammen Siloam" (det betyder utsänd). Mannen gick då dit och tvådde sig; och när han kom igen, kunde han se.
\par 8 Då sade grannarna och andra som förut hade sett honom såsom tiggare: "Är detta icke den man som att och tiggde?"
\par 9 Somliga svarade: "Det är han." Andra sade: "Nej, men han är lik honom." Själv sade han: "Jag är den mannen."
\par 10 Och de frågade honom: "Huru blevo då dina ögon öppnade?"
\par 11 Han svarade: "Den man som heter Jesus gjorde en deg och smorde därmed mina ögon och sade till mig: 'Gå bort till Siloam och två dig.' Jag gick då dit och tvådde mig, och så fick jag min syn."
\par 12 De frågade honom: "Var är den mannen?" Han svarade: "Det vet jag icke."
\par 13 Då förde de honom, mannen som förut hade varit blind, bort till fariséerna.
\par 14 Och det var sabbat den dag då Jesus gjorde degen och öppnade hans ögon.
\par 15 När nu jämväl fariséerna i sin ordning frågade honom huru han hade fått sin syn, svarade han dem: "Han lade en deg på mina ögon, och jag fick två mig, och nu kan jag se."
\par 16 Då sade några av fariséerna: "Den mannen är icke från Gud, eftersom han icke håller sabbaten." Andra sade: "Huru skulle någon som är en syndare kunna göra sådana tecken?" Så funnos bland dem stridiga meningar.
\par 17 Då frågade de åter den blinde: "Vad säger du själv om honom, då det ju var dina ögon han öppnade?" Han svarade: "En profet är han."
\par 18 Men judarna trodde icke att han hade varit blind och fått sin syn, förrän de hade kallat till sig mannen föräldrar, hans som hade fått sin syn.
\par 19 Dem frågade de och sade: "Är detta eder son, den som I sägen vara född blind? Huru kommer det då till, att han nu kan se?"
\par 20 Då svarade han föräldrar och sade: "Att denne är vår son, och att han föddes blind, det veta vi.
\par 21 Men huru han nu kan se, det veta vi icke, ej heller veta vi vem som har öppnat hans ögon. Frågen honom själv; han är gammal nog, han må själv tala för sig."
\par 22 Detta sade hans föräldrar, därför att de fruktade judarna; ty judarna hade redan kommit överens om att den som bekände Jesus vara Messias, han skulle utstötas ur synagogan.
\par 23 Därför var det som hans föräldrar sade: "Han är gammal nog; frågen honom själv."
\par 24 Då kallade de för andra gången till sig mannen som hade varit blind och sade till honom: "Säg nu sanningen, Gud till pris. Vi veta att denne man är en syndare."
\par 25 Han svarade: "Om han är en syndare vet jag icke; ett vet jag: att jag, som var blind, nu kan se."
\par 26 Då frågade de honom: "Vad gjorde han med dig? På vad sätt öppnade han dina ögon?"
\par 27 Han svarade dem: "Jag har ju redan sagt eder det, men I hörden icke på mig. Varför viljen I då åter höra det? Kanske viljen också I bliva hans lärjungar?"
\par 28 Då bannade de honom och sade: "Du är själv hans lärjunge; vi äro Moses' lärjungar.
\par 29 Till Moses har Gud talat, det veta vi; men varifrån denne är, det veta vi icke."
\par 30 Mannen svarade och sade till dem: "Ja, däri ligger det förunderliga, att I icke veten varifrån han är, och ändå har han öppnat mina ögon.
\par 31 Vi veta ju att Gud icke hör syndare, men också att om någon är gudfruktig och gör hans vilja, då hör han honom.
\par 32 Aldrig förut har man hört att någon har öppnat ögonen på en som föddes blind.
\par 33 Vore denne icke från Gud, så kunde han intet göra."
\par 34 De svarade och sade till honom: "Du är hel och hållen född i synd, och du vill undervisa oss!" Och så drevo de ut honom.
\par 35 Jesus fick sedan höra att de hade drivit ut honom, och när han så träffade honom, sade han: "Tror du på Människosonen?"
\par 36 Han svarade och sade: "Herre, vem är han då? Säg mig det, så att jag kan tro på honom."
\par 37 Jesus sade till honom: "Du har sett honom; det är han som talar med dig."
\par 38 Då sade han: "Herre, jag tror." Och han föll ned för honom.
\par 39 Och Jesus sade: "Till en dom har jag kommit hit i världen, för att de som icke se skola varda seende, och för att de som se skola varda blinda."
\par 40 När några fariséer som voro i hans närhet hörde detta, sade de till honom: "Äro då kanske också vi blinda?"
\par 41 Jesus svarade dem: "Voren I blinda, så haden I icke synd. Men nu sägen I: 'Vi se', därför står eder synd kvar."

\chapter{10}

\par 1 "Sannerligen, sannerligen säger jag eder: Den som icke går in i fårahuset genom dörren, utan stiger in någon annan väg, han är en tjuv och en rövare.
\par 2 Men den som går in genom dörren, han är fårens herde.
\par 3 För honom öppnar dörrvaktaren, och fåren lyssna till hans röst, och han kallar sina får vid namn och för dem ut.
\par 4 Och när han har släppt ut alla sina får, går han framför dem, och fåren följa honom, ty de känna hans röst.
\par 5 Men en främmande följa de alls icke, utan fly bort ifrån honom, ty de känna icke de främmandes röst."
\par 6 Så talade Jesus till dem i förtäckta ord; men de förstodo icke vad det var som han talade till dem.
\par 7 Åter sade Jesus till dem: "Sannerligen, sannerligen säger jag eder: Jag är dörren in till fåren.
\par 8 Alla de som hava kommit före mig äro tjuvar och rövare, men fåren hava icke lyssnat till dem.
\par 9 Jag är dörren; den som går in genom mig, han skall bliva frälst, och han skall få gå ut och in och skall finna bete.
\par 10 Tjuven kommer allenast för att stjäla och slakta och förgöra. Jag har kommit, för att de skola hava liv och hava över nog.
\par 11 Jag är den gode herden. En god herde giver sitt liv för fåren.
\par 12 Men den som är lejd och icke är herden själv, när han, den som fåren icke tillhöra, ser ulven komma, då övergiver han fåren och flyr, och ulven rövar bort dem och förskingrar dem.
\par 13 Han är ju lejd och frågar icke efter fåren.
\par 14 Jag är den gode herden, och jag känner mina får, och mina får känna mig,
\par 15 såsom Fadern känner mig, och såsom jag känner Fadern; och jag giver mitt liv för fåren.
\par 16 Jag har ock andra får, som icke höra till detta fårahus; också dem måste jag draga till mig, och de skola lyssna till min röst. Så skall det bliva en hjord och en herde.
\par 17 Därför älskar Fadern mig, att jag giver mitt liv - för att sedan taga igen det.
\par 18 Ingen tager det ifrån mig, utan jag giver det av fri vilja. Jag har makt att giva det, och jag har makt att taga igen det. Det budet har jag fått av min Fader."
\par 19 För dessa ords skull uppstodo åter stridiga meningar bland judarna.
\par 20 Många av dem sade: "Han är besatt av en ond ande och är från sina sinnen. Varför hören I på honom?"
\par 21 Andra åter sade: "Sådana ord talar icke den som är besatt. Icke kan väl en ond ande öppna blindas ögon?"
\par 22 Därefter inföll tempelinvigningens högtid i Jerusalem. Det var nu vinter,
\par 23 och Jesus gick fram och åter i Salomos pelargång i helgedomen.
\par 24 Då samlade sig judarna omkring honom och sade till honom: "Huru länge vill du hålla oss i ovisshet? Om du är Messias, så säg oss det öppet."
\par 25 Jesus svarade dem: "Jag har sagt eder det, men I tron mig icke. De gärningar som jag gör i min Faders namn, de vittna om mig.
\par 26 Men I tron mig icke, ty I ären icke av mina får.
\par 27 Mina får lyssna till min röst, och jag känner dem, och de följa mig.
\par 28 Och jag giver dem evigt liv, och de skola aldrig någonsin förgås, och ingen skall rycka dem ur min hand.
\par 29 Min Fader, som har givit mig dem, är större än alla, och ingen kan rycka dem ur min Faders hand.
\par 30 Jag och Fadern äro ett."
\par 31 Då togo judarna åter upp stenar för att stena honom.
\par 32 Men Jesus sade till dem: "Många goda gärningar, som komma från min Fader, har jag låtit eder se. För vilken av dessa gärningar är det som I viljen stena mig?"
\par 33 Judarna svarade honom: "Det är icke för någon god gärnings skull som vi vilja stena dig, utan därför att du hädar och gör dig själv till Gud, du som är en människa."
\par 34 Jesus svarade dem: "Det är ju så skrivet i eder lag: 'Jag har sagt att I ären gudar'.
\par 35 Om han nu har kallat för gudar dem som Guds ord kom till - och skriften kan ju icke bliva om intet -
\par 36 huru kunnen då I, på den grund att jag har sagt mig vara Guds Son, anklaga mig för hädelse, mig som Fadern har helgat och sänt i världen?
\par 37 Gör jag icke min Faders gärningar, så tron mig icke.
\par 38 Men gör jag dem, så tron gärningarna, om I än icke tron mig; då skolen I fatta och förstå att Fadern är i mig, och att jag är i Fadern."
\par 39 Då ville de åter gripa honom, men han gick sin väg, undan deras händer.
\par 40 Sedan begav han sig åter bort till det ställe på andra sidan Jordan, där Johannes först hade döpt, och stannade kvar där.
\par 41 Och många kommo till honom. Och de sade: "Väl gjorde Johannes intet tecken, men allt vad Johannes sade om denne var sant."
\par 42 Och många kommo där till tro på honom.

\chapter{11}

\par 1 Och en man vid namn Lasarus låg sjuk; han var från Betania, den by där Maria och hennes syster Marta bodde.
\par 2 Det var den Maria som smorde Herren med smörjelse och torkade hans fötter med sitt hår. Och nu låg hennes broder Lasarus sjuk.
\par 3 Då sände systrarna bud till Jesus och läto säga: "Herre, se, han som du har så kär ligger sjuk."
\par 4 När Jesus hörde detta, sade han: "Den sjukdomen är icke till döds, utan till Guds förhärligande, så att Guds Son genom den bliver förhärligad."
\par 5 Och Jesus hade Marta och hennes syster och Lasarus kära.
\par 6 När han nu hörde att denne låg sjuk, stannade han först två dagar där han var;
\par 7 men därefter sade han till lärjungarna: "Låt oss gå tillbaka till Judeen."
\par 8 Lärjungarna sade till honom: "Rabbi, nyligen ville judarna stena dig, och åter går du dit?"
\par 9 Jesus svarade: "Dagen har ju tolv timmar; den som vandrar om dagen, han stöter sig icke, ty han ser då denna världens ljus.
\par 10 Men den som vandrar om natten, han stöter sig, ty han har då intet som lyser honom."
\par 11 Sedan han hade talat detta, sade han ytterligare till dem: "Lasarus, vår vän, har somnat in; men jag går för att väcka upp honom ur sömnen."
\par 12 Då sade hans lärjungar till honom: "Herre, sover han, så bliver han frisk igen."
\par 13 Men Jesus hade talat om hans död; de åter menade att han talade om vanlig sömn.
\par 14 Då sade Jesus öppet till dem: "Lasarus är död.
\par 15 Och för eder skull, för att I skolen tro, gläder jag mig över att jag icke var där. Men låt oss nu gå till honom."
\par 16 Då sade Tomas, som kallades Didymus, till de andra lärjungarna: "Låt oss gå med, för att vi må dö med honom."
\par 17 När så Jesus kom dit, fann han att den döde redan hade legat fyra dagar i graven.
\par 18 Nu låg Betania nära Jerusalem, vid pass femton stadier därifrån,
\par 19 och många judar hade kommit till Marta och Maria för att trösta dem i sorgen över deras broder.
\par 20 Då nu Maria fick höra att Jesus kom, gick hon honom till mötes; men Maria satt kvar hemma.
\par 21 Och Marta sade till Jesus: "Herre, hade du varit här, så vore min broder icke död.
\par 22 Men jag vet ändå att allt vad du beder Gud om, det skall Gud giva dig."
\par 23 Jesus sade till henne: "Din broder skall stå upp igen."
\par 24 Marta svarade honom: "Jag vet att han skall stå upp, vid uppståndelsen på den yttersta dagen."
\par 25 Jesus svarade till henne: "Jag är uppståndelsen och livet. Den som tror på mig, han skall leva, om han än dör;
\par 26 och var och en som lever och tror på mig, han skall aldrig någonsin dö. Tror du detta?"
\par 27 Hon svarade honom: "Ja, Herre, jag tror att du är Messias, Guds Son, han som skulle komma i världen."
\par 28 När hon hade sagt detta, gick hon bort och kallade på Maria, sin syster, och sade hemligen till henne: "Mästaren är här och kallar dig till sig."
\par 29 När hon hörde detta, stod hon strax upp och gick åstad till honom.
\par 30 Men Jesus hade ännu icke kommit in i byn, utan var kvar på det ställe där Marta hade mött honom.
\par 31 Då nu de judar, som voro inne i huset hos Maria för att trösta henne, sågo att hon så hastigt stod upp och gick ut, följde de henne, i tanke att hon gick till graven för att gråta där.
\par 32 När så Maria kom till det ställe där Jesus var och fick se honom, föll hon ned för hans fötter och sade till honom: "Herre, hade du varit där, så vore min broder icke död."
\par 33 Då nu Jesus såg henne gråta och såg jämväl att de judar, som hade kommit med henne, gräto, upptändes han i sin ande och blev upprörd
\par 34 och frågade: "Var haven I lagt honom?" De svarade honom: "Herre, kom och se." Och Jesus grät.
\par 35 Då sade judarna: "Se huru kär han hade honom!"
\par 36 Men somliga av dem sade:
\par 37 "Kunde icke han, som öppnade den blindes ögon, ock hava så gjort att denne icke hade dött?"
\par 38 Då upptändes Jesus åter i sitt innersta och gick bort till graven. Den var urholkad i berget, och en sten låg framför ingången.
\par 39 Jesus sade: "Tagen bort stenen." Då sade den dödes syster Marta till honom: "Herre, han luktar redan, ty han har varit död i fyra dygn."
\par 40 Jesus svarade henne: "Sade jag dig icke, att om du trodde, skulle du få se Guds härlighet?"
\par 41 Då togo de bort stenen. Och Jesus lyfte upp sina ögon och sade: "Fader, jag tackar dig för att du har hört mig.
\par 42 Jag visste ju förut att du alltid hör mig; men för folkets skull, som står här omkring, säger jag detta, för att de skola tro att det är du som har sänt mig."
\par 43 När han hade sagt detta, ropade han med hög röst: "Lasarus, kom ut."
\par 44 Och han som hade varit död kom ut, med händer och fötter inlindade i bindlar och med ansiktet inhöljt i en duk. Jesus sade till dem: "Lösen honom, och låten honom gå."
\par 45 Många judar, som hade kommit till Maria och hade sett vad Jesus hade gjort, trodde då på honom.
\par 46 Men några av dem gingo bort till fariséerna och omtalade för dem vad Jesus hade gjort.
\par 47 Då sammankallade översteprästerna och fariséerna en rådsförsamling och sade: "Vad skola vi taga oss till? Denne man gör ju många tecken.
\par 48 Om vi skola låta honom så fortfara, skola alla tro på honom, och romarna komma då att taga ifrån oss både land och folk."
\par 49 Men en av dem, Kaifas, som var överstepräst för det året, sade till dem: "I förstån intet,
\par 50 och I besinnen icke huru mycket bättre det är för eder att en man dör för folket, än att hela folket förgås."
\par 51 Detta sade han icke av sig själv, utan genom profetisk ingivelse, eftersom han var överstepräst för det året; ty Jesus skulle dö för folket.
\par 52 Ja, icke allenast "för folket"; han skulle dö också för att samla och förena Guds förskingrade barn.
\par 53 Från den dagen var deras beslut fattat att döda honom.
\par 54 Så vandrade då Jesus icke längre öppet bland judarna, utan drog sig undan till en stad som hette Efraim, på landsbygden, i närheten av öknen; där stannade han kvar med sina lärjungar.
\par 55 Men judarnas påsk var nära, och många begåvo sig då, före påsken, från landsbygden upp till Jerusalem för att helga sig.
\par 56 Och de sökte efter Jesus och sade till varandra, där de stodo i helgedom: "Vad menen I? Skall han då alls icke komma till högtiden?"
\par 57 Och översteprästerna och fariséerna hade utfärdat påbud om att den som finge veta var han fanns skulle giva det till känna, för att de måtte kunna gripa honom.

\chapter{12}

\par 1 Sex dagar före påsk kom nu Jesus till Betania, där Lasarus bodde, han som av Jesus hade blivit uppväckt från de döda.
\par 2 Där gjorde man då för honom ett gästabud, och Marta betjänade dem, men Lasarus var en av dem som lågo till bords jämte honom.
\par 3 Då tog Maria ett skålpund smörjelse av dyrbar äkta nardus och smorde därmed Jesu fötter; sedan torkade hon hans fötter med sitt hår. Och huset uppfylldes med vällukt av smörjelsen.
\par 4 Men Judas Iskariot, en av hans lärjungar, den som skulle förråda honom, sade då:
\par 5 "Varför sålde man icke hellre denna smörjelse för tre hundra silverpenningar och gav dessa åt de fattiga?"
\par 6 Detta sade han, icke därför, att han frågade efter de fattiga, utan därför, att han var en tjuv och plägade taga vad som lades i penningpungen, vilken han hade om hand.
\par 7 Men Jesus sade: "Låt henne vara; må hon få fullgöra detta för min begravningsdag.
\par 8 De fattiga haven I ju alltid ibland eder, men mig haven I icke alltid."
\par 9 Nu hade det blivit känt för den stora hopen av judarna att Jesus var där, och de kommo dit, icke allenast för hans skull, utan ock för att se Lasarus, som han hade uppväckt från de döda.
\par 10 Då beslöto översteprästerna att döda också Lasarus.
\par 11 Ty för hans skull gingo många judar bort och trodde på Jesus.
\par 12 När dagen därefter det myckna folk som hade kommit till högtiden fick höra att Jesus var på väg till Jerusalem,
\par 13 togo de palmkvistar och gingo ut för att möta honom och ropade: "Hosianna! Välsignad vare han som kommer, i Herrens namn, han som är Israels konung."
\par 14 Och Jesus fick sig en åsnefåle och satte sig upp på den, såsom det är skrivet:
\par 15 "Frukta icke, du dotter Sion. Se, din konung kommer, sittande på en åsninnas fåle."
\par 16 Detta förstodo hans lärjungar icke då strax, men när Jesus hade blivit förhärligad, då kommo de ihåg att detta var skrivet om honom, och att man hade gjort detta med honom.
\par 17 Så gav nu folket honom sitt vittnesbörd, de som hade varit med honom, när han kallade Lasarus ut ur graven och uppväckte honom från de döda.
\par 18 Därför kom också det övriga folket emot honom, eftersom de hörde att han hade gjort det tecknet.
\par 19 Då sade fariséerna till varandra: "I sen att I alls intet kunnen uträtta; hela världen löper ju efter honom."
\par 20 Nu voro där ock några greker, av dem som plägade fara upp för att tillbedja under högtiden.
\par 21 Dessa kommo till Filippus, som var från Betsaida i Galileen, och bådo honom och sade: "Herre, vi skulle vilja se Jesus."
\par 22 Filippus gick och sade detta till Andreas; Andreas och Filippus gingo och sade det till Jesus.
\par 23 Jesus svarade dem och sade: "Stunden är kommen att Människosonen skall förhärligas.
\par 24 Sannerligen, sannerligen säger jag eder: Om icke vetekornet faller i jorden och dör, så förbliver det ett ensamt korn; men om det dör, så bär det mycken frukt.
\par 25 Den som älskar sitt liv, han mister det, men den som hatar sitt liv i denna världen, han skall behålla det och skall hava evigt liv.
\par 26 Om någon vill tjäna mig, så följe han mig; och där jag är, där skall också min tjänare få vara. Om någon tjänar mig, så skall min Fader ära honom.
\par 27 Nu är min själ i ångest; vad skall jag väl säga? Fader, fräls mig undan denna stund. Dock, just därför har jag kommit till denna stund.
\par 28 Fader, förhärliga ditt namn." Då kom en röst från himmelen: "Jag har redan förhärligat det, och jag skall ytterligare förhärliga det."
\par 29 Folket, som stod där och hörde detta, sade då: "Det var ett tordön." Andra sade: "Det var en ängel som talade med honom."
\par 30 Då svarade Jesus och sade: "Denna röst kom icke för min skull, utan för eder skull."
\par 31 Nu går en dom över denna världen, nu skall denna världens furste utkastas.
\par 32 Och när jag har blivit upphöjd från jorden, skall jag draga alla till mig."
\par 33 Med dessa ord gav han till känna på vad sätt han skulle dö.
\par 34 Då svarade folket honom: "Vi hava hört av lagen att Messias skall stanna kvar för alltid. Huru kan du då säga att Människosonen måste bliva upphöjd? Vad är väl detta för en Människoson?"
\par 35 Jesus sade till dem: "Ännu en liten tid är ljuset ibland eder. Vandren medan I haven ljuset, på det att mörkret icke må få makt med eder; den som vandrar i mörkret, han vet ju icke var han går.
\par 36 Tron på ljuset, medan I haven ljuset, så att I bliven ljusets barn." Detta talade Jesus och gick sedan bort och dolde sig för dem.
\par 37 Men fastän han hade gjort så många tecken inför dem, trodde de icke på honom.
\par 38 Ty det ordet skulle fullbordas, som profeten Esaias säger: "Herre, vem trodde, vad som predikades för oss, och för vem var Herrens arm uppenbar?"
\par 39 Alltså kunde de icke tro; Esaias säger ju ytterligare:
\par 40 "Han har förblindat deras ögon och förstockat deras hjärtan, så att de icke kunna se med sina ögon eller förstå med sina hjärtan och omvända sig och bliva helade av mig."
\par 41 Detta kunde Esaias säga, eftersom han hade sett hans härlighet, när han talade med honom. -
\par 42 Dock funnos jämväl bland rådsherrarna många som trodde på honom; men för fariséernas skulle ville de icke bekänna det, för att de icke skulle bliva utstötta ur synagogan.
\par 43 Ty de skattade högre att bliva ärade av människor än att bliva ärade av Gud.
\par 44 Men Jesus sade med hög röst: "Den som tror på mig, han tror icke på mig, utan på honom som har sänt mig.
\par 45 Och den som ser mig, han ser honom som har sänt mig.
\par 46 Såsom ett ljus har jag kommit i världen, för att ingen av dem som tro på mig skall förbliva i mörkret.
\par 47 Om någon hör mina ord, men icke håller dem, så dömer icke jag honom; ty jag har icke kommit för att döma världen, utan för att frälsa världen.
\par 48 Den som förkastar mig och icke tager emot mina ord, han har dock en domare över sig; det ord som jag har talat, det skall döma honom på den yttersta dagen.
\par 49 Ty jag har icke talat av mig själv, utan Fadern, som har sänt mig, han har bjudit mig vad jag skall säga, och vad jag skall tala.
\par 50 Och jag vet att hans bud är evigt liv; därför, vad jag talar, det talar jag såsom Fadern har sagt mig."

\chapter{13}

\par 1 Före påskhögtiden hände sig detta. Jesus visste att stunden var kommen för honom att gå bort ifrån denna världen till Fadern; och såsom han allt hittills hade älskat sina egna här i världen, så gav han dem nu ett yttersta bevis på sin kärlek.
\par 2 De höllo nu aftonmåltid, och djävulen hade redan ingivit Judas Iskariot, Simons son, i hjärtat att förråda Jesus.
\par 3 Och Jesus visste att Fadern hade givit allt i hans händer, och att han hade gått ut från Gud och skulle gå till Gud.
\par 4 Men han stod upp från måltiden och lade av sig överklädnaden och tog en linneduk och band den om sig.
\par 5 Sedan slog han vatten i ett bäcken och begynte två lärjungarnas fötter och torkade dem med linneduken som han hade bundit om sig.
\par 6 Så kom han till Simon Petrus. Denne sade då till honom: "Herre, skulle du två mina fötter?"
\par 7 Jesus svarade och sade till honom: "Vad jag gör förstår du icke nu, men framdeles skall du fatta det."
\par 8 Petrus sade till honom: "Aldrig någonsin skall du två mina fötter!" Jesus svarade honom: "Om jag icke tvår dig, så har du ingen del med mig."
\par 9 Då sade Simon Petrus till honom: "Herre, icke allenast mina fötter, utan ock händer och huvud!"
\par 10 Jesus svarade honom: "Den som är helt tvagen, han behöver allenast två fötterna; han är ju i övrigt hel och hållen ren. Så ären ock I rena - dock icke alla."
\par 11 Han visste nämligen vem det var som skulle förråda honom; därför sade han att de icke alla voro rena.
\par 12 Sedan han nu hade tvagit deras fötter och tagit på sig överklädnaden och åter lagt sig ned vid bordet, sade han till dem: "Förstån I vad jag har gjort med eder?
\par 13 I kallen mig 'Mästare' och 'Herre', och I säger rätt, ty jag är så.
\par 14 Har nu jag, eder Herre och Mästare, tvagit edra fötter, så ären ock I pliktiga att två varandras fötter.
\par 15 Jag har ju givit eder ett föredöme, för att I skolen göra såsom jag har gjort mot eder.
\par 16 Sannerligen, sannerligen säger jag eder: Tjänaren är icke förmer än sin herre, ej heller sändebudet förmer än den som har sänt honom.
\par 17 Då I veten detta, saliga ären I, om I ock gören det.
\par 18 Jag talar icke om eder alla; jag vet vilka jag har utvalt. Men detta skriftens ord skulle ju fullbordas: 'Den som åt mitt bröd, han lyfte mot mig sin häl.'
\par 19 Redan nu, förrän det sker, säger jag eder det, för att I, när det har skett, skolen tro att jag är den jag är.
\par 20 Sannerligen, sannerligen säger jag eder: Den som tager emot den jag sänder, han tager emot mig; och den som tager emot mig, han tager emot honom som har sänt mig."
\par 21 När Jesus hade sagt detta, blev han upprörd i sin ande och betygade och sade: "Sannerligen, sannerligen säger jag eder: En av eder skall förråda mig."
\par 22 Då sågo lärjungarna på varandra och undrade vilken han talade om.
\par 23 Nu var där bland lärjungarna en som låg till bords invid Jesu bröst, den lärjunge som Jesus älskade.
\par 24 Åt denne gav då Simon Petrus ett tecken och sade till honom: "Säg vilken det är som han talar om."
\par 25 Han lutade sig då mot Jesu bröst och frågade honom: "Herre, vilken är det?"
\par 26 Då svarade Jesus: "Det är den åt vilken jag räcker brödstycket som jag nu doppar." Därvid doppade han brödstycket och räckte det åt Judas, Simon Iskariots son.
\par 27 Då, när denne hade tagit emot brödstycket, for Satan in i honom. Och Jesus sade till honom: "Gör snart vad du gör."
\par 28 Men ingen av dem som lågo där till bords förstod varför han sade detta till honom.
\par 29 Ty eftersom Judas hade penningpungen om hand, menade några att Jesus hade velat säga till honom: "Köp vad vi behöva till högtiden", eller ock att han hade tillsagt honom att giva något åt de fattiga.
\par 30 Då han nu hade tagit emot brödstycket, gick han strax ut; och det var natt.
\par 31 Och när han hade gått ut, sade Jesus: "Nu är Människosonen förhärligad, och Gud är förhärligad i honom.
\par 32 Är nu Gud förhärligad i honom, så skall ock Gud förhärliga honom i sig själv, och han skall snart förhärliga honom.
\par 33 Kära barn, allenast en liten tid är jag ännu hos eder; I skolen sedan söka efter mig, men det som jag sade till judarna: 'Dit jag går, dit kunnen I icke komma', detsamma säger jag nu ock till eder.
\par 34 Ett nytt bud giver jag eder, att I skolen älska varandra; ja, såsom jag har älskat eder, så skolen ock I älska varandra.
\par 35 Om I haven kärlek inbördes, så skola alla därav förstå att I ären mina lärjungar."
\par 36 Då frågade Simon Petrus honom: "Herre, vart går du?" Jesus svarade: "Dit jag går, dit kan du icke nu följa mig; men framdeles skall du följa mig."
\par 37 Petrus sade till honom: "Herre, varför kan jag icke följa dig nu? Mitt liv vill jag giva för dig."
\par 38 Jesus svarade: "Ditt liv vill du giva för mig? Sannerligen, sannerligen säger jag dig: Hanen skall icke gala, förrän du tre gånger har förnekat mig."

\chapter{14}

\par 1 "Edra hjärtan vare icke oroliga. Tron på Gud; tron ock på mig.
\par 2 I min Faders hus äro många boningar; om så icke voro, skulle jag nu säga eder att jag går bort för att bereda eder rum.
\par 3 Och om jag än går bort för att bereda eder rum, så skall jag dock komma igen och taga eder till mig; ty jag vill att där jag är, där skolen I ock vara.
\par 4 Och vägen som leder dit jag går, den veten I."
\par 5 Tomas sade till honom: "Herre, vi veta icke vart du går; huru kunna vi då veta vägen?"
\par 6 Jesus svarade honom: "Jag är vägen och sanningen och livet; ingen kommer till Fadern utom genom mig.
\par 7 Haden I känt mig, så haden I ock känt min Fader; nu kännen I honom och haven sett honom."
\par 8 Filippus sade till honom: "Herre, låt oss se Fadern, så hava vi nog."
\par 9 Jesus svarade honom: "Så lång tid har jag varit hos eder, och du har icke lärt känna mig, Filippus? Den som har sett mig, han har sett Fadern. Huru kan du då säga: 'Låt oss se Fadern'?
\par 10 Tror du icke att jag är i Fadern, och att Fadern är i mig? De ord jag talar till eder talar jag icke av mig själv. Och gärningarna, dem gör Fadern, som bor i mig; de äro hans verk.
\par 11 Tron mig; jag är i Fadern, och Fadern i mig. Varom icke, så tron för själva gärningarnas skull.
\par 12 Sannerligen, sannerligen säger jag eder: Den som tror på mig, han skall ock själv göra de gärningar som jag gör; och ännu större än dessa skall han göra. Ty jag går till Fadern,
\par 13 och vadhelst I bedjen om i mitt namn, det skall jag göra, på det att Fadern må bliva förhärligad i Sonen.
\par 14 Ja, om I bedjen om något i mitt namn, så skall jag göra det.
\par 15 Älsken I mig, så hållen I mina bud,
\par 16 och jag skall bedja Fadern, och han skall giva eder en annan Hjälpare, som för alltid skall vara hos eder:
\par 17 sanningens Ande, som världen icke kan taga emot, ty hon ser honom icke och känner honom icke. Men I kännen honom, ty han bor hos eder och skall vara i eder.
\par 18 Jag skall icke lämna eder faderlösa; jag skall komma till eder.
\par 19 Ännu en liten tid, och världen ser mig icke mer, men I sen mig. Ty jag lever; I skolen ock leva.
\par 20 På den dagen skolen I förstå att jag är i min Fader, och att I ären i mig, och att jag är i eder.
\par 21 Den som har mina bud och håller dem, han är den som älskar mig; och den som älskar mig, han skall bliva älskad av min Fader, och jag skall älska honom och jag skall uppenbara mig för honom."
\par 22 Judas - icke han som kallades Iskariot - sade då till honom: "Herre, varav kommer det att du tänker uppenbara dig för oss, men icke för världen?"
\par 23 Jesus svarade och sade till honom: "Om någon älskar mig, så håller han mitt ord; och min Fader skall älska honom, och vi skola komma till honom och taga vår boning hos honom.
\par 24 Den som icke älskar mig, han håller icke mina ord; och likväl är det ord som I hören icke mitt, utan Faderns, som har sänt mig.
\par 25 Detta har jag talat till eder, medan jag ännu är kvar hos eder.
\par 26 Men Hjälparen, den helige Ande, som Fadern skall sända i mitt namn, han skall lära eder allt och påminna eder om allt vad jag har sagt eder.
\par 27 Frid lämnar jag efter mig åt eder, min frid giver jag eder; icke giver jag eder den såsom världen giver. Edra hjärtan vare icke oroliga eller försagda.
\par 28 I hörden att jag sade till eder: 'Jag går bort, men jag kommer åter till eder.' Om I älskaden mig, så skullen I ju glädjas över att jag går bort till Fadern, ty Fadern är större än jag.
\par 29 Och nu har jag sagt eder det, förrän det sker, på det att I mån tro, när det har skett.
\par 30 Härefter talar jag icke mycket med eder, ty denna världens furste kommer. I mig finnes intet som hör honom till;
\par 31 men detta sker, för att världen skall förstå att jag älskar Fadern och gör såsom Fadern har bjudit mig. Stån upp, låt oss gå härifrån."

\chapter{15}

\par 1 "Jag är det sanna vinträdet, och min Fader är vingårdsmannen.
\par 2 Var gren i mig, som icke bär frukt, den tager han bort; och var och en som bär frukt, den rensar han, för att den skall bära mer frukt.
\par 3 I ären redan nu rena, i kraft av det ord som jag har talat till eder.
\par 4 Förbliven i mig, så förbliver ock jag i eder. Såsom grenen icke kan bära frukt av sig själv, utan allenast om den förbliver i vinträdet, så kunnen I det ej heller, om I icke förbliven i mig.
\par 5 Jag är vinträdet, I ären grenarna. Om någon förbliver i mig, och jag i honom, så bär han mycken frukt; ty mig förutan kunnen I intet göra.
\par 6 Om någon icke förbliver i mig, så kastas han ut såsom en avbruten gren och förtorkas; och man samlar tillhopa sådana grenar och kastar dem i elden, och de brännas upp.
\par 7 Om I förbliven i mig, och mina ord förbliva i eder, så mån I bedja om vadhelst I viljen, och det skall vederfaras eder.
\par 8 Därigenom bliver min Fader förhärligad, att i bären mycken frukt och bliven mina lärjungar.
\par 9 Såsom Fadern har älskat mig, så har ock jag älskat eder; förbliven i min kärlek.
\par 10 Om I hållen mina bud, så förbliven I i min kärlek, likasom jag har hållit min Faders bud och förbliver i hans kärlek.
\par 11 Detta har jag talat till eder, för att min glädje skall bo i eder, och för att eder glädje skall bliva fullkomlig.
\par 12 Detta är mitt bud, att I skolen älska varandra, såsom jag har älskat eder.
\par 13 Ingen har större kärlek, än att han giver sitt liv för sina vänner.
\par 14 I ären mina vänner, om I gören vad jag bjuder eder.
\par 15 Jag kallar eder nu icke längre tjänare, ty tjänaren får icke veta vad hans herre gör; vänner kallar jag eder, ty allt vad jag har hört av min Fader har jag kungjort för eder.
\par 16 I haven icke utvalt mig, utan jag har utvalt eder; och jag har bestämt om eder att I skolen gå åstad och bära frukt, sådan frukt som bliver beståndande, på det att Fadern må giva eder vadhelst I bedjen honom om i mitt namn.
\par 17 Ja, det bjuder jag eder, att I skolen älska varandra.
\par 18 Om världen hatar eder, så betänken att hon har hatat mig förr än eder.
\par 19 Voren I av världen, så älskade ju världen vad henne tillhörde; men eftersom I icke ären av världen, utan av mig haven blivit utvalda och tagna ut ur världen, därför hatar världen eder.
\par 20 Kommen ihåg det ord som jag sade till eder: 'Tjänaren är icke förmer än sin herre.' Hava de förföljt mig, så skola de ock förfölja eder; hava de hållit mitt ord, så skola de ock hålla edert.
\par 21 Men allt detta skola de göra mot eder för mitt namns skull, eftersom de icke känna honom som har sänt mig.
\par 22 Hade jag icke kommit och talat till dem, så skulle de icke hava haft synd; men nu hava de ingen ursäkt för sin synd.
\par 23 Den som hatar mig, han hatar ock min Fader.
\par 24 Hade jag icke bland dem gjort sådana gärningar, som ingen annan har gjort, så skulle de icke hava haft synd; men nu hava de sett dem, och hava likväl hatat både mig och min Fader.
\par 25 Men det ordet skulle ju fullbordas, som är skrivet i deras lag: 'De hava hatat mig utan sak.'
\par 26 Dock, när Hjälparen kommer, som jag skall sända eder ifrån Fadern, sanningens Ande, som utgår ifrån Fadern, då skall han vittna om mig.
\par 27 Också I kunnen vittna, eftersom I haven varit med mig från begynnelsen."

\chapter{16}

\par 1 "Detta har jag talat till eder, för att I icke skolen komma på fall.
\par 2 Man skall utstöta eder ur synagogorna; ja, den tid kommer, då vemhelst som dräper eder skall mena sig därmed förrätta offertjänst åt Gud.
\par 3 Och så skola de göra, därför att de icke hava lärt känna Fadern, ej heller mig.
\par 4 Men detta har jag talat till eder, för att I, när den tiden är inne, skolen komma ihåg att jag har sagt eder det. Jag sade eder det icke från begynnelsen, ty jag var ju hos eder.
\par 5 Och nu går jag bort till honom som har sänt mig; och ingen av eder frågar mig vart jag går.
\par 6 Men edra hjärtan äro uppfyllda av bedrövelse, därför att jag har sagt eder detta.
\par 7 Dock säger jag eder sanningen: Det är nyttigt för eder att jag går bort, ty om jag icke ginge bort, så komme icke Hjälparen till eder; men då jag nu går bort, skall jag sända honom till eder.
\par 8 Och när han kommer, skall han låta världen få veta sanningen i fråga om synd och rättfärdighet och dom:
\par 9 i fråga om synd, ty de tro icke på mig;
\par 10 i fråga om rättfärdighet, ty jag går till Fadern, och I sen mig icke mer;
\par 11 i fråga om dom, ty denna världens furste är nu dömd.
\par 12 Jag hade ännu mycket att säga eder, men I kunnen icke nu bära det.
\par 13 Men när han kommer, som är sanningens Ande, då skall han leda eder fram till hela sanningen. Ty han skall icke tala av sig själv, utan vad han hör, allt det skall han tala; och han skall förkunna för eder vad komma skall.
\par 14 Han skall förhärliga mig, ty av mitt skall han taga och skall förkunna det för eder.
\par 15 Allt vad Fadern har, det är mitt; därför sade jag att han skall taga av mitt och förkunna det för eder.
\par 16 En liten tid, och I sen mig icke mer; och åter en liten tid, och I fån se mig, ty jag går till Fadern."
\par 17 Då sade några av hans lärjungar till varandra: "Vad är detta som han säger till oss: 'En liten tid, och I sen mig icke; och åter en liten tid, och I fån se mig', så ock: 'Jag går till Fadern'?"
\par 18 De sade alltså: "Vad är detta som han säger: 'En liten tid'? Vi förstå icke vad han talar."
\par 19 Då märkte Jesus att de ville fråga honom, och han sade till dem: "I talen med varandra om detta som jag sade: 'En liten tid, och I sen mig icke; och åter en liten tid, och I fån se mig.'
\par 20 Sannerligen, sannerligen säger jag eder: I skolen bliva bedrövade, men eder bedrövelse skall vändas i glädje.
\par 21 När en kvinna föder barn, har hon bedrövelse, ty hennes stund är kommen; men när hon har fött barnet, kommer hon icke mer ihåg sin vedermöda, ty hon gläder sig över att en människa är född till världen.
\par 22 Så haven ock I nu bedrövelse; men jag skall se eder åter, och då skola edra hjärtan glädja sig, och ingen skall taga eder glädje ifrån eder.
\par 23 Och på den dagen skolen I icke fråga mig om något. Sannerligen, sannerligen säger jag eder: Vad I bedjen Fadern om, det skall han giva eder i mitt namn.
\par 24 Hittills haven I icke bett om något i mitt namn; bedjen, och I skolen få, för att eder glädje skall bliva fullkomlig.
\par 25 Detta har jag talat till eder i förtäckta ord; den tid kommer, då jag icke mer skall tala till eder i förtäckta ord, utan öppet förkunna för eder om Fadern.
\par 26 På den dagen skolen I bedja i mitt namn. Och jag säger eder icke att jag skall bedja Fadern för eder,
\par 27 ty Fadern själv älskar eder, eftersom I haven älskat mig och haven trott att jag är utgången från Gud.
\par 28 Ja, jag har gått ut ifrån Fadern och har kommit i världen; åter lämnar jag världen och går till Fadern."
\par 29 Då sade hans lärjungar: "Se, nu talar du öppet och brukar inga förtäckta ord.
\par 30 Nu veta vi att du vet allt, och att det icke är behövligt för dig att man frågar dig; därför tro vi att du är utgången från Gud."
\par 31 Jesus svarade dem: "Nu tron I?
\par 32 Se, den stund kommer, ja, den är redan kommen, så I skolen förskingras, var och en åt sitt håll, och lämna mig allena. Dock, jag är icke allena, ty Fadern är med mig.
\par 33 Detta har jag talat till eder, för att I skolen hava frid i mig. I världen liden i betryck; men varen vid gott mod, jag har övervunnit världen."

\chapter{17}

\par 1 Sedan Jesus hade talat detta, lyfte han upp sina ögon mot himmelen och sade: "Fader, stunden är kommen; förhärliga din Son, på det att din Son må förhärliga dig,
\par 2 eftersom du har givit honom makt över allt kött, för att han skall giva evigt liv åt alla dem som du har givit åt honom.
\par 3 Och detta är evigt liv, att de känna dig, den enda sanne Guden, och den du har sänt, Jesus Kristus.
\par 4 Jag har förhärligat dig på jorden, genom att fullborda det verk som du har givit mig att utföra.
\par 5 Och nu, Fader, förhärliga du mig hos dig själv, med den härlighet som jag hade hos dig, förrän världen var till.
\par 6 Jag har uppenbarat ditt namn för de människor som du har tagit ut ur världen och givit åt mig. De voro dina, och du har givit dem åt mig, och de hava hållit ditt ord.
\par 7 Nu hava de förstått att allt vad du har givit åt mig, det kommer från dig.
\par 8 Ty de ord som du har givit åt mig har jag givit åt dem: och de hava tagit emot dem och hava i sanning förstått att jag är utgången från dig, och de tro att du har sänt mig.
\par 9 Jag beder för dem; det är icke för världen jag beder, utan för dem som du har givit åt mig, ty de äro dina
\par 10 - såsom allt mitt är ditt, och ditt är mitt - och jag är förhärligad i dem.
\par 11 Jag är nu icke längre kvar i världen, men de äro kvar i världen, när jag går till dig. Helige Fader, bevara dem i ditt namn - det som du har förtrott åt mig - för att de må vara ett, likasom vi äro ett.
\par 12 Medan jag var hos dem, bevarade jag dem i ditt namn, det som du har förtrott åt mig; jag vakade över dem, och ingen av dem gick i fördärvet, ingen utom fördärvets man, ty skriften skulle ju fullbordas.
\par 13 Nu går jag till tid; dock talar jag detta, medan jag ännu är här i världen, för att de skola hava min glädje fullkomlig i sig.
\par 14 Jag har givit dem ditt ord; och världen har hatat dem, eftersom de icke äro av världen, likasom icke heller jag är av världen.
\par 15 Jag beder icke att du skall taga dem bort ur världen, utan att du skall bevara dem från det onda.
\par 16 De äro icke av världen, likasom icke heller jag är av världen.
\par 17 Helga dem i sanningen; ditt ord är sanning.
\par 18 Såsom du har sänt mig i världen, så har ock jag sänt dem i världen.
\par 19 Och jag helgar mig till ett offer för dem, på det att ock de må vara i sanning helgade.
\par 20 Men icke för dessa allenast beder jag, utan ock för dem som genom deras ord komma till tro på mig;
\par 21 jag beder att de alla må vara ett, och att, såsom du, Fader, är i mig, och jag i dig, också de må vara i oss, för att världen skall tro att du har sänt mig.
\par 22 Och den härlighet som du har givit mig, den har jag givit åt dem, för att de skola vara ett, såsom vi äro ett
\par 23 - jag i dem, och du i mig - ja, för att de skola vara fullkomligt förenade till ett, så att världen kan förstå att du har sänt mig, och att du har älskat dem, såsom du har älskat mig.
\par 24 Fader, jag vill att där jag är, där skola ock de som du har givit mig vara med mig, så att de få se min härlighet, som du har givit mig; ty du har älskat mig före världens begynnelse.
\par 25 Rättfärdige Fader, världen har icke lärt känna dig, men jag känner dig, och dessa hava förstått att du har sänt mig.
\par 26 Och jag har kungjort för dem ditt namn och skall kungöra det, på det att den kärlek, som du har älskat mig med, må vara i dem, och jag i dem."

\chapter{18}

\par 1 När Jesus hade sagt detta, begav han sig med sina lärjungar därifrån och gick över bäcken Kidron till andra sidan. Där var en örtagård, och i den gick han in med sina lärjungar.
\par 2 Men också Judas, han som förrådde honom, kände till det stället, ty där hade Jesus och hans lärjungar ofta kommit tillsammans.
\par 3 Och Judas tog nu med sig den romerska vakten, så ock några av översteprästernas och fariséernas tjänare, och kom dit med bloss och lyktor och vapen.
\par 4 Och Jesus, som visste allt vad som skulle övergå honom, gick fram och sade till dem: "Vem söken I?"
\par 5 De svarade honom: "Jesus från Nasaret." Jesus sade till dem: "Det är jag." Och Judas, förrädaren, stod också där ibland dem.
\par 6 När Jesus nu sade till dem: "Det är jag", veko de tillbaka och föllo till marken.
\par 7 Åter frågade han dem då: "Vem söken I?" De svarade: "Jesus från Nasaret."
\par 8 Jesus sade: "Jag har sagt eder att det är jag; om det alltså är mig I söken, så låten dessa gå."
\par 9 Ty det ordet skulle fullbordas, som han hade sagt: "Av dem som du har givit mig har jag icke förlorat någon."
\par 10 Och Simon Petrus, som hade ett svärd, drog ut det och högg till översteprästens tjänare och högg så av honom högra örat; och tjänarens namn var Malkus.
\par 11 Då sade Jesus till Petrus: "Stick ditt svärd i skidan. Skulle jag icke dricka den kalk som min Fader har givit mig?"
\par 12 Den romerska vakten med sin överste och de judiska rättstjänarna grepo då Jesus och bundo honom
\par 13 och förde honom bort, först till Hannas; denne var nämligen svärfader till Kaifas, som var överstepräst det året.
\par 14 Och det var Kaifas som under rådplägningen hade sagt till judarna, att det vore bäst om en man finge dö för folket.
\par 15 Och Simon Petrus jämte en annan lärjunge följde efter Jesus. Den lärjungen var bekant med översteprästen och gick med Jesus in på översteprästens gård;
\par 16 men Petrus stod utanför vid porten. Den andre lärjungen, den som var bekant med översteprästen, gick då ut och talade med portvakterskan och fick så föra Petrus ditin.
\par 17 Tjänstekvinnan som vaktade porten sade därvid till Petrus: "Är icke också du en av den mannens lärjungar?" Han svarade: "Nej, det är jag icke."
\par 18 Men tjänarna och rättsbetjänterna hade gjort upp en koleld, ty det var kallt, och de stodo där och värmde sig; bland dem stod också Petrus och värmde sig.
\par 19 Översteprästen frågade nu Jesus om hans lärjungar och om hans lära.
\par 20 Jesus svarade honom: "Jag har öppet talat för världen, jag har alltid undervisat i synagogan eller i helgedomen, på ställen där alla judar komma tillsammans; hemligen har jag intet talat.
\par 21 Varför frågar du då mig? Dem som hava hört mig må du fråga om vad jag har talat till dem. De veta ju vad jag har sagt."
\par 22 När Jesus sade detta, gav honom en av rättstjänarna, som stod där bredvid, ett slag på kinden och sade: "Skall du så svara översteprästen?"
\par 23 Jesus svarade honom: "Har jag talat orätt, så bevisa att det var orätt; men har jag talat rätt, varför slår du mig då?"
\par 24 Och Hannas sände honom bunden till översteprästen Kaifas.
\par 25 Men Simon Petrus stod och värmde sig. Då sade de till honom: "Är icke också du en av hans lärjungar?" Han nekade och sade: "Det är jag icke."
\par 26 Då sade en av översteprästens tjänare, en frände till den som Petrus hade huggit örat av: "Såg jag icke själv att du var med honom i örtagården?"
\par 27 Då nekade Petrus åter. Och i detsamma gol hanen.
\par 28 Sedan förde de Jesus från Kaifas till pretoriet; och det var nu morgon. Men själva gingo de icke in i pretoriet, för att de icke skulle bliva orenade, utan skulle kunna äta påskalammet.
\par 29 Då gick Pilatus ut till dem och sade: "Vad haven I för anklagelse att frambära mot denne man?"
\par 30 De svarade och sade till honom: "Vore han icke en illgärningsman, så hade vi icke överlämnat honom åt dig."
\par 31 Då sade Pilatus till dem: "Tagen I honom, och dömen honom efter eder lag." Judarna svarade honom: "För oss är det icke lovligt att avliva någon."
\par 32 Ty Jesu ord skulle fullbordas, det som han hade sagt för att giva till känna på vad sätt han skulle dö.
\par 33 Pilatus gick åter in i pretoriet och kallade Jesus till sig och sade till honom: "Är du judarnas konung?"
\par 34 Jesus svarade: "Säger du detta av dig själv, eller hava andra sagt dig det om mig?"
\par 35 Pilatus svarade: "Jag är väl icke en jude! Ditt eget folk och översteprästerna hava överlämnat dig åt mig. Vad har du gjort?"
\par 36 Jesus svarade: "Mitt rike är icke av denna världen. Om mitt rike vore av denna världen, så hade väl mina tjänare kämpat för att jag icke skulle bliva överlämnad åt judarna. Men nu är mitt rike icke av denna världen."
\par 37 Så sade Pilatus till honom: "Så är du dock en konung?" Jesus svarade: "Du säger det själv, att jag är en konung. Ja, därtill är jag född, och därtill har jag kommit i världen, att jag skall vittna för sanningen. Var och en som är av sanningen, han hör min röst."
\par 38 Pilatus sade till honom: "Vad är sanning?" När han hade sagt detta, gick han åter ut till judarna och sade till dem: "Jag finner honom icke skyldig till något brott.
\par 39 Nu är det en sedvänja hos eder, att jag vid påsken skall giva eder en fånge lös. Viljen I då att jag skall giva eder 'judarnas konung' lös?"
\par 40 Då skriade de åter och sade: "Icke honom, utan Barabbas." Men Barabbas var en rövare.

\chapter{19}

\par 1 Så tog då Pilatus Jesus och lät gissla honom.
\par 2 Och krigsmännen vredo samman en krona av törnen och satte den på hans huvud och klädde på honom en purpurfärgad mantel.
\par 3 Sedan trädde de fram till honom och sade: "Hell dig, du judarnas konung!" och slogo honom på kinden.
\par 4 Åter gick Pilatus ut och sade till folket: "Se, jag vill föra honom ut till eder, på det att I mån förstå att jag icke finner honom skyldig till något brott."
\par 5 Och Jesus kom då ut, klädd i törnekronan och den purpurfärgade manteln. Och han sade till dem: "Se mannen!"
\par 6 Då nu översteprästerna och rättstjänarna fingo se honom, skriade de: "Korsfäst! Korsfäst!" Pilatus sade till dem: "Tagen I honom, och korsfästen honom; jag finner honom icke skyldig till något brott."
\par 7 Judarna svarade honom: "Vi hava själva en lag, och efter den lagen måste han dö, ty han har gjort sig till Guds Son."
\par 8 När Pilatus hörde dem tala så, blev hans fruktan ännu större.
\par 9 Och han gick åter in i pretoriet och frågade Jesus: "Varifrån är du?" Men Jesus gav honom intet svar.
\par 10 Då sade Pilatus till honom: "Svarar du mig icke? Vet du då icke att jag har makt att giva dig lös och makt att korsfästa dig?"
\par 11 Jesus svarade honom: "Du hade alls ingen makt över mig, om den icke vore dig given ovanifrån. Därför har den större synd, som har överlämnat mig åt dig."
\par 12 Från den stunden sökte Pilatus efter någon utväg att giva honom lös. Men judarna ropade och sade: "Giver du honom lös, så är du icke kejsarens vän. Vemhelst som gör sig till konung, han sätter sig upp mot kejsaren."
\par 13 När Pilatus hörde de orden, lät han föra ut Jesus och satte sig på domarsätet, på en plats som kallades Litostroton, på hebreiska Gabbata.
\par 14 Och det var tillredelsedagen före påsken, vid sjätte timmen. Och han sade till judarna: "Se här är eder konung!"
\par 15 Då skriade de: "Bort med honom! Bort med honom! Korsfäst honom!" Pilatus sade till dem: "Skall jag korsfästa eder konung?" Översteprästerna svarade: "Vi hava ingen annan konung än kejsaren."
\par 16 Då gjorde han dem till viljes och bjöd att han skulle korsfästas. Och de togo Jesus med sig.
\par 17 Och han bar själv sitt kors och kom så ut till det ställe som kallades Huvudskalleplatsen, på hebreiska Golgata.
\par 18 Där korsfäste de honom, och med honom två andra, en på vardera sidan, och Jesus i mitten.
\par 19 Men Pilatus lät ock göra en överskrift och sätta upp den på korset; och den lydde så: "Jesus från Nasaret, judarnas konung."
\par 20 Den överskriften läste många av judarna, ty det ställe där Jesus var korsfäst låg nära staden: och den var avfattad på hebreiska, på latin och på grekiska.
\par 21 Då sade judarnas överstepräster till Pilatus: "Skriv icke: 'Judarnas konung', utan skriv att han har sagt sig vara judarnas konung."
\par 22 Pilatus svarade: "Vad jag har skrivit, det har jag skrivit."
\par 23 Då nu krigsmännen hade korsfäst Jesus, togo de hans kläder och delade dem i fyra delar, en del åt var krigsman. Också livklädnaden togo de. Men livklädnaden hade inga sömmar, utan var vävd i ett stycke, uppifrån och alltigenom.
\par 24 Därför sade de till varandra: "Låt oss icke skära sönder den, utan kasta lott om vilken den skall tillhöra." Ty skriftens ord skulle fullbordas: "De delade mina kläder mellan sig och kastade lott om min klädnad." Så gjorde nu krigsmännen.
\par 25 Men vid Jesu kors stodo hans moder och hans moders syster, Maria, Klopas' hustru, och Maria från Magdala.
\par 26 När Jesus nu fick se sin moder och bredvid henne den lärjunge som han älskade, sade han till sin moder: "Moder, se din son."
\par 27 Sedan sade han till lärjungen: "Se din moder." Och från den stunden tog lärjungen henne hem till sig.
\par 28 Eftersom nu Jesus visste att allt annat redan var fullbordat, sade han därefter, då ju skriften skulle i allt uppfyllas: "Jag törstar."
\par 29 Där stod då en kärl som var fullt av ättikvin. Med det vinet fyllde de en svamp, som de satte på en isopsstängel och förde till hans mun.
\par 30 Och när Jesus hade tagit emot vinet, sade han: "Det är fullbordat." Sedan böjde han ned huvudet och gav upp andan.
\par 31 Men eftersom det var tillredelsedag och judarna icke ville att kropparna skulle bliva kvar på korset över sabbaten (det var nämligen en stor sabbatsdag), bådo de Pilatus att han skulle låta sönderslå de korsfästas ben och taga bort kropparna.
\par 32 Så kommo då krigsmännen och slogo sönder den förstes ben och sedan den andres som var korsfäst med honom.
\par 33 När de därefter kommo till Jesus och sågo honom redan vara död, slogo de icke sönder hans ben;
\par 34 men en av krigsmännen stack upp han sida med ett spjut, och strax kom därifrån ut blod och vatten.
\par 35 Och den som har sett detta, han har vittnat därom, för att ock I skolen tro; och hans vittnesbörd är sant, och han vet att han talar sanning.
\par 36 Ty detta skedde, för att skriftens ord skulle fullbordas: "Intet ben skall sönderslås på honom."
\par 37 Och åter ett annat skriftens ord lyder så: "De skola se upp till honom som de hava stungit."
\par 38 Men Josef från Arimatea, som var en Jesu lärjunge - fastän i hemlighet, av fruktan för judarna - kom därefter och bad Pilatus att få taga Jesu kropp; och Pilatus tillstadde honom det. Då gick han åstad och tog hans kropp.
\par 39 Och jämväl Nikodemus kom dit, han som första gången hade besökt honom om natten; denne förde med sig en blandning av myrra och aloe, vid pass hundra skålpund.
\par 40 Och de togo Jesu kropp och omlindade den med linnebindlar och lade dit de välluktande kryddorna, såsom judarna hava för sed vid tillredelse till begravning.
\par 41 Men invid det ställe där han hade blivit korsfäst var en örtagård, och i örtagården fanns en ny grav, som ännu ingen hade varit lagd i.
\par 42 Där lade de nu Jesus, eftersom det var judarnas tillredelsedag och graven låg nära.

\chapter{20}

\par 1 Men på första veckodagen, medan det ännu var mörkt, kom Maria från Magdala dit till graven och fick se stenen vara borttagen från graven.
\par 2 Då skyndade hon därifrån och kom till Simon Petrus och till den andre lärjungen, den som Jesus älskade, och sade till dem: "De hava tagit Herren bort ur graven, och vi veta icke var de hava lagt honom."
\par 3 Då begåvo sig Petrus och den andre lärjungen åstad på väg till graven.
\par 4 Och de sprungo båda på samma gång; men den andre lärjungen sprang fortare än Petrus och kom först fram till graven.
\par 5 Och när han lutade sig ditin, så han linnebindlarna ligga där; dock gick han icke in.
\par 6 Sedan, efter honom, kom ock Simon Petrus dit. Han gick in i graven och fick så se huru bindlarna lågo där,
\par 7 och huru duken som hade varit höljd över hans huvud icke låg tillsammans med bindlarna, utan för sig själv på ett särskilt ställe, hopvecklad.
\par 8 Då gick ock den andre lärjungen ditin, han som först hade kommit till graven; och han såg och trodde.
\par 9 De hade nämligen ännu icke förstått skriftens ord, att han skulle uppstå från de döda.
\par 10 Och lärjungarna gingo så hem till sitt igen.
\par 11 Men Maria stod och grät utanför graven. Och under det hon grät, lutade hon sig in i graven
\par 12 och fick då se två änglar i vita kläder sitta där Jesu kropp hade legat, den ene vid huvudets plats, den andre vid fötternas.
\par 13 Och de sade till henne: "Kvinna, varför gråter du?" Hon svarade dem: "De hava tagit bort min Herre, och jag vet icke var de hava lagt honom."
\par 14 Vid det hon sade detta, vände hon sig om och fick se Jesus stå där; men hon visste icke att det var Jesus.
\par 15 Jesus sade till henne: "Kvinna, varför gråter du? Vem söker du?" Hon trodde att det var örtagårdsmästaren och svarade honom: "Herre, om det är du som har burit bort honom, så säg mig var du har lagt honom, så att jag kan hämta honom."
\par 16 Jesus sade till henne: "Maria!" Då vände hon sig om och sade till honom på hebreiska: "Rabbuni!" (det betyder mästare).
\par 17 Jesus sade till henne: "Rör icke vid mig; jag har ju ännu icke farit upp till Fadern. Men gå till mina bröder, och säg till dem att jag far upp till min Fader och eder Fader, till min Gud och eder Gud."
\par 18 Maria från Magdala gick då och omtalade för lärjungarna att hon hade sett Herren, och att han hade sagt detta till henne.
\par 19 På aftonen samma dag, den första veckodagen, medan lärjungarna av fruktan för judarna voro samlade inom stängda dörrar, kom Jesus och stod mitt ibland dem och sade till dem: "Frid vare med eder!"
\par 20 Och när han hade sagt detta, visade han dem sina händer och sin sida. Och lärjungarna blevo glada, när de sågo Herren.
\par 21 Åter sade Jesus till dem: "Frid vare med eder! Såsom Fadern har sänt mig, så sänder ock jag eder."
\par 22 Och när han hade sagt detta, andades han på dem och sade till dem: "Tagen emot helig ande!
\par 23 Om I förlåten någon hans synder, så äro de honom förlåtna; och om I binden någon i hans synder, så är han bunden i dem."
\par 24 Men Tomas, en av de tolv, han som kallades Didymus, var icke med dem, när Jesus kom.
\par 25 Då nu de andra lärjungarna sade till honom att de hade sett Herren, svarade han dem: "Om jag icke ser hålen efter spikarna i hans händer och sticker mitt finger i hålen efter spikarna och sticker min hand i hans sida, så kan jag icke tro det."
\par 26 Åtta dagar därefter voro hans lärjungar åter därinne, och Tomas var med bland dem. Då kom Jesus, medan dörrarna voro stängda, och stod mitt ibland de, och sade: "Frid vare med eder!"
\par 27 Sedan sade han till Tomas: "Räck hit dit finger, se här äro mina händer; och räck hit din hand, och stick den i min sida. Och tvivla icke, utan tro."
\par 28 Tomas svarade och sade till honom: "Min Herre och min Gud!"
\par 29 Jesus sade till honom: "Eftersom du har sett mig, tror du? Saliga äro de som icke se och dock tro."
\par 30 Ännu många andra tecken, som icke äro uppskrivna i denna bok, gjorde Jesus i sina lärjungars åsyn.
\par 31 Men dessa hava blivit uppskrivna, för att I skolen tro att Jesus är Messias, Guds Son, och för att I genom tron skolen hava liv i hans namn.

\chapter{21}

\par 1 Därefter uppenbarade sig Jesus åter för lärjungarna, vid Tiberias' sjö; och vid den uppenbarelsen gick så till:
\par 2 Simon Petrus och Tomas, som kallades Didymus, och Natanael, han som var från Kana i Galileen, och Sebedeus' söner voro tillsammans, och med dem två andra av hans lärjungar.
\par 3 Simon Petrus sade då till dem: "Jag vill gå åstad och fiska." De sade till honom: "Vi gå också med dig." Så begåvo de sig åstad och stego i båten. Men den natten fingo de intet.
\par 4 När det sedan hade blivit morgon, stod Jesus där på stranden; dock visste lärjungarna icke att det var Jesus.
\par 5 Och Jesus sade till dem: "Mina barn, haven I något att äta?" De svarade honom: "Nej."
\par 6 Han sade till dem: "Kasten ut nätet på högra sidan om båten, så skolen I få." Då kastade de ut; och nu fingo de en så stor hop fiskar, att de icke förmådde draga upp nätet.
\par 7 Den lärjunge som Jesus älskade sade då till Petrus: "Det är Herren." När Simon Petrus hörde att det var Herren, tog han på sig sin överklädnad - ty han var oklädd - och gav sig i sjön.
\par 8 Men de andra lärjungarna kommo med båten och drogo efter sig nätet med fiskarna; de voro nämligen icke längre från land än vid pass två hundra alnar.
\par 9 När de sedan hade stigit i land, sågo de glöd ligga där och fisk, som låg därpå, och bröd.
\par 10 Jesus sade till dem: "Tagen hit av de fiskar som I nu fingen."
\par 11 Då steg Simon Petrus i båten och drog nätet upp på land, och det var fullt av stora fiskar, ett hundra femtiotre stycken. Och fastän de voro så många, hade nätet icke gått sönder.
\par 12 Därefter sade Jesus till dem: "Kommen hit och äten." Och ingen av lärjungarna dristade sig att fråga honom vem han var, ty de förstodo att det var Herren.
\par 13 Jesus gick då fram och tog brödet och gav dem, likaledes ock av fiskarna.
\par 14 Detta var nu tredje gången som Jesus uppenbarade sig för sina lärjungar, sedan han hade uppstått från de döda.
\par 15 När de hade ätit, sade Jesus till Simon Petrus: "Simon, Johannes' son, älskar du mig mer än dessa göra?" Han svarade honom: "Ja, Herre; du vet att jag har dig kär." Då sade han till honom: "Föd mina lamm."
\par 16 Åter frågade han honom, för andra gången: "Simon, Johannes' son, älskar du mig?" Han svarade honom: "Ja, Herre; du vet att jag har dig kär." Då sade han till honom: "Var en herde för mina får."
\par 17 För tredje gången frågade han honom: "Simon, Johannes' son, har du mig kär?" Petrus blev bedrövad över att han för tredje gången frågade honom: "Har du mig kär?" Och han svarade honom: "Herre, du vet allting; du vet att jag har dig kär." Då sade Jesus till honom: "Föd mina får.
\par 18 Sannerligen, sannerligen säger jag dig: När du var yngre, omgjordade du dig själv och gick vart du ville; men när du bliver gammal, skall du nödgas sträcka ut dina händer, och en annan skall omgjorda dig och föra dig dit du icke vill."
\par 19 Detta sade han för att giva till känna med hurudan död Petrus skulle förhärliga Gud. Och sedan han hade sagt detta, sade han till honom: "Följ mig."
\par 20 När Petrus vände sig om, fick han se att den lärjunge som Jesus älskade följde med, densamme som under aftonmåltiden hade lutat sig mot hans bröst och frågat honom: "Herre, vilken är det som skall förråda dig?"
\par 21 Då nu Petrus såg den lärjungen, frågade han Jesus: "Herre, huru bliver det då med denne?"
\par 22 Jesus svarade honom: "Om jag vill att han skall leva kvar, till dess jag kommer, vad kommer det dig vid? Följ du mig."
\par 23 Så kom det talet ut ibland bröderna, att den lärjungen icke skulle dö. Men Jesus hade icke sagt till honom att han icke skulle dö, utan allenast: "Om jag vill att han skall leva kvar, till dess jag kommer, vad kommer det dig vid?"
\par 24 Det är den lärjungen som vittnar om detta, och som har skrivit detta; och vi veta att hans vittnesbörd är sant.
\par 25 Ännu mycket annat var det som Jesus gjorde; och om allt detta skulle uppskrivas, det ena med det andra, så tror jag att icke ens hela världen skulle kunna rymma de böcker som då bleve skrivna.


\end{document}