\begin{document}

\title{Romarbrevet}


\chapter{1}

\par 1 Paulus, Jesu Kristi tjänare, kallad till apostel, avskild till att förkunna Guds evangelium,
\par 2 vilket Gud redan förut genom sina profeter hade i heliga skrifter utlovat,
\par 3 evangelium om hans Son, vilken såsom människa i köttet är född av Davids säd
\par 4 och såsom helig andevarelse är med kraft bevisad vara Guds Son, allt ifrån uppståndelsen från de döda, ja, evangelium om Jesus Kristus, vår Herre,
\par 5 genom vilken vi hava fått nåd och apostlaämbete för att, hans namn till ära, upprätta trons lydnad bland alla hednafolk,
\par 6 bland vilka jämväl I ären, I som ären kallade och Jesu Kristi egna -
\par 7 jag, Paulus, hälsar alla Guds älskade som bo i Rom, dem som äro kallade och heliga. Nåd vare med eder och frid ifrån Gud, vår Fader, och Herren Jesus Kristus.
\par 8 Först och främst tackar jag min Gud genom Jesus Kristus för eder alla, därför att man i hela världen talar om eder tro.
\par 9 Ty Gud, som jag i min ande tjänar såsom förkunnare av evangelium om hans Son, han är mitt vittne, han vet huru oavlåtligt jag tänker på eder
\par 10 och i mina böner alltid beder att jag dock nu omsider må få ett gynnsamt tillfälle att komma till eder, om Gud så vill.
\par 11 Ty jag längtar efter att se eder, för att jag må kunna meddela eder någon andlig nådegåva till att styrka eder;
\par 12 jag menar: för att jag i eder krets må tillsammans med eder få hämta hugnad ur vår gemensamma tro, eder och min.
\par 13 Jag vill säga eder, mina bröder, att jag ofta har haft i sinnet att komma till eder, för att också bland eder få skörda någon frukt, såsom bland övriga hednafolk; dock har jag allt hittills blivit hindrad.
\par 14 Både mot greker och mot andra folk, både mot visa och mot ovisa har jag förpliktelser.
\par 15 Därför är jag villig att förkunna evangelium också för eder som bon i Rom.
\par 16 Ty jag blyges icke för evangelium; ty det är en Guds kraft till frälsning för var och en som tror, först och främst för juden, så ock för greken.
\par 17 Rättfärdighet från Gud uppenbaras nämligen däri, av tro till tro; så är ock skrivet: "Den rättfärdige skall leva av tro."
\par 18 Ty Guds vrede uppenbarar sig från himmelen över all ogudaktighet och orättfärdighet hos människor som i orättfärdighet undertrycka sanningen.
\par 19 Vad man kan känna om Gud är nämligen uppenbart bland dem; Gud har ju uppenbarat det för dem.
\par 20 Ty hans osynliga väsen, hans eviga makt och gudomshärlighet hava ända ifrån världens skapelse varit synliga, i det att de kunna förstås genom hans verk. Så äro de då utan ursäkt.
\par 21 Ty fastän de hade lärt känna Gud, prisade och tackade de honom dock icke såsom Gud, utan förföllo till fåfängliga tankar; och så blevo deras oförståndiga hjärtan förmörkade.
\par 22 När de berömde sig av att vara visa, blevo de dårar
\par 23 och bytte bort den oförgänglige Gudens härlighet mot beläten, som voro avbilder av förgängliga människor, ja ock av fåglar och fyrfotadjur och krälande djur.
\par 24 Därför prisgav Gud dem i deras hjärtans begärelser åt orenhet, så att de med varandra skändade sina kroppar.
\par 25 De hade ju bytt bort Guds sanning mot lögn och tagit sig för att dyrka och tjäna det skapade framför Skaparen, honom som är högtlovad i evighet, amen.
\par 26 Fördenskull gav Gud dem till pris åt skamliga lustar: deras kvinnor utbytte det naturliga umgänget mot ett onaturligt;
\par 27 sammalunda övergåvo ock männen det naturliga umgänget med kvinnan och upptändes i lusta till varandra och bedrevo styggelse, man med man. Så fingo de på sig själva uppbära sin villas tillbörliga lön.
\par 28 Och eftersom de icke hade aktat det något värt att taga vara på sin kunskap om Gud, gav Gud dem till pris åt ett ovärdigt sinnelag, till att bedriva otillbörliga ting.
\par 29 Så hava de blivit uppfyllda av allt slags orättfärdighet, ondska, girighet, elakhet; de äro fulla av avund, mordlust, trätlystnad, svek, vrångsinthet;
\par 30 de äro örontasslare, förtalare, styggelser för Gud, våldsverkare, övermodiga, stortaliga, illfundiga, olydiga mot sina föräldrar,
\par 31 oförståndiga, trolösa, utan kärlek till sina egna, utan barmhärtighet mot andra.
\par 32 Och fastän de väl veta vad Gud har stadgat såsom rätt, att nämligen de som handla så förtjäna döden, är det dem icke nog att själva så göra, de giva ock sitt bifall åt andra som handla likaså.

\chapter{2}

\par 1 Därför är du utan ursäkt, du människa, vem du än är, som dömer. Ty därmed att du dömer en annan fördömer du dig själv, eftersom du, som dömer den andre, själv handlar på samma sätt.
\par 2 Och vi veta att Guds dom verkligen kommer över dem som handla så.
\par 3 Men du menar väl detta, att du skall kunna undfly Guds dom, du människa, som dömer dem som handla så, och dock gör detsamma som de?
\par 4 Eller föraktar du hans godhets, skonsamhets och långmodighets rikedom, utan att förstå att denna Guds godhet vill föra dig till bättring?
\par 5 Genom din hårdhet och ditt hjärtas obotfärdighet samlar du ju över dig vrede, som skall drabba dig på vredens dag, då när det bliver uppenbart att Gud är en rättvis domare.
\par 6 Ty "han skall vedergälla var och en efter hans gärningar".
\par 7 Evigt liv skall han giva åt dem som med uthållighet i att göra det goda söka härlighet och ära och oförgänglighet.
\par 8 Men över dem som äro genstridiga och icke lyda sanningen, utan lyda orättfärdigheten, över dem kommer vrede och förtörnelse.
\par 9 Ja, bedrövelse och ångest skall komma över den människas själ, som gör det onda, först och främst över judens, så ock över grekens.
\par 10 Men härlighet och ära och frid skall tillfalla var och en som gör det goda, först och främst juden, så ock greken.
\par 11 Ty hos Gud finnes intet anseende till personen;
\par 12 alla de som utan lag hava syndat skola ock utan lag förgås, och alla de som med lag hava syndat skola genom lag bliva dömda.
\par 13 Ty icke lagens hörare äro rättfärdiga inför Gud, men lagens görare skola förklaras rättfärdiga.
\par 14 Ty då hedningarna, som icke hava lag, av naturen göra vad lagen innehåller, så äro dessa, utan att hava lag, sig själv en lag,
\par 15 då de ju sålunda visa att lagens verk äro skrivna i deras hjärtan. Därom utgöra också deras egna samveten ett vittnesbörd, så ock, i den inbördes umgängelsen, deras tankar, när dessa anklaga eller ock försvara dem.
\par 16 Ja, så skall det befinnas vara på den dag då Gud, enligt det evangelium jag förkunnar, genom Kristus Jesus dömer över vad som är fördolt hos människorna.
\par 17 Du kallar dig jude och förlitar dig på lagen och berömmer dig av Gud.
\par 18 Du känner ock hans vilja, och eftersom du har fått din undervisning ur lagen, kan du döma om vad rättast är;
\par 19 och du tilltror dig att vara en ledare för blinda, ett ljus för människor som vandra i mörker,
\par 20 en uppfostrare för oförståndiga, en lärare för enfaldiga, eftersom du i lagen har uttrycket för kunskapen och sanningen.
\par 21 Du som vill lära andra, du lär icke dig själv! Du som predikar att man icke skall stjäla, du begår själv stöld!
\par 22 Du som säger att man icke skall begå äktenskapsbrott, du begår själv sådant brott! Du som håller avgudarna för styggelser, du gör dig själv skyldig till tempelrån!
\par 23 Du som berömmer dig av lagen, du vanärar Gud genom att överträda lagen,
\par 24 ty, såsom det är skrivet, "för eder skull varder Guds namn smädat bland hedningarna".
\par 25 Ty väl är omskärelse till gagn, om du håller lagen; men om du är en lagöverträdare, så är du med din omskärelse dock oomskuren.
\par 26 Om nu den oomskurne håller lagens stadgar, skall han då icke, fastän han är oomskuren, räknas såsom omskuren?
\par 27 Jo, och han som i följd av sin härkomst är oomskuren, men ändå fullgör lagen, skall bliva dig till dom, dig som äger lagens bokstav och omskärelsen, men likväl är en lagöverträdare.
\par 28 Ty den är icke jude, som är det i utvärtes måtto, ej heller är det omskärelse, som sker utvärtes på köttet.
\par 29 Nej, den är jude, som är det i invärtes måtto, och omskärelse är hjärtats omskärelse, en som sker i Anden, och icke i kraft av bokstaven; och han har sin berömmelse, icke från människor, utan från Gud.

\chapter{3}

\par 1 Vilket företräde hava då judarna, eller vad gagn hava de av omskärelsen?
\par 2 Jo, ett stort företräde, på allt sätt; först och främst det, att de hava blivit betrodda med Guds löftesord.
\par 3 Ty vad betyder det, om några av dem blevo trolösa? Kan då deras trolöshet göra Guds trofasthet om intet?
\par 4 Bort det! Må Gud stå såsom sannfärdig, om ock "var människa är en lögnare". Så är ju skrivet: "På det att du må finnas rättfärdig i dina ord och få rätt, när man sätter sig till doms över dig."
\par 5 Men är det nu så, att vår orättfärdighet tjänar till att bevisa Guds rättfärdighet, vad skola vi då säga? Kan väl Gud, han som låter vredesdomen drabba, vara orättfärdig? (Jag talar såsom vore det fråga om en människa.)
\par 6 Bort det! Huru skulle Gud då kunna döma världen?
\par 7 Och å andra sidan, om Guds sannfärdighet genom min lögnaktighet ännu mer har trätt i dagen, honom till ära, varför skall då jag likväl dömas såsom syndare?
\par 8 Och varför skulle vi icke "göra vad ont är, för att gott måtte komma därav", såsom man, för att smäda oss, påstår att vi göra, och såsom några föregiva att vi lära? - Sådana få med rätta sin dom.
\par 9 Huru är det alltså? Äro vi då något förmer än de andra? Ingalunda. Redan härförut har jag ju måst anklaga både judar och greker för att allasammans vara under synd.
\par 10 Så är ock skrivet: "Ingen rättfärdig finnes, icke en enda.
\par 11 Ingen förståndig finnes, ingen finnes som söker Gud.
\par 12 Nej, alla hava de avvikit, allasammans hava de blivit odugliga, ingen finnes som gör vad gott är, det finnes ingen enda.
\par 13 En öppen grav är deras strupe, sina tungor bruka de till svek. Huggormsgift är inom deras läppar.
\par 14 Deras mun är full av förbannelse och bitterhet.
\par 15 Deras fötter äro snara, när det gäller att utgjuta blod.
\par 16 Förödelse och elände är på deras vägar,
\par 17 och fridens väg känna de icke.
\par 18 Guds fruktan är icke för deras ögon."
\par 19 Nu veta vi att allt vad lagen säger, det talar den till dem som hava lagen, för att var mun skall bliva tillstoppad och hela världen stå med skuld inför Gud;
\par 20 ty av laggärningar bliver intet kött rättfärdigt inför honom. Vad som kommer genom lagen är kännedom om synden.
\par 21 Men nu har, utan lag, en rättfärdighet från Gud blivit uppenbarad, en som lagen och profeterna vittna om,
\par 22 en rättfärdighet från Gud genom tro på Jesus Kristus, för alla dem som tro. Ty här är ingen åtskillnad.
\par 23 Alla hava ju syndat och äro i saknad av härligheten från Gud;
\par 24 och de bliva rättfärdiggjorda utan förskyllan, av hans nåd, genom förlossningen i Kristus Jesus,
\par 25 honom som Gud har ställt fram såsom ett försoningsmedel genom tro, i hans blod. Så ville Gud - då han i sin skonsamhet hade haft fördrag med de synder som förut hade blivit begångna - nu visa att han dock var rättfärdig.
\par 26 Ja, så ville han i den tid som nu är lämna beviset för att han är rättfärdig. Härigenom skulle han både själv befinnas vara rättfärdig och göra den rättfärdig, som låter det bero på tro på Jesus.
\par 27 Huru bliver det då med vår berömmelse? Den är utestängd. Genom vilken lag? Månne genom en gärningarnas lag? Nej, genom en trons lag.
\par 28 Vi hålla nämligen före att människan bliver rättfärdig genom tro, utan laggärningar.
\par 29 Eller är Gud allenast judarnas Gud? Är han icke ock hedningarnas? Jo, förvisso också hedningarnas,
\par 30 så visst som Gud är en, han som skall göra de omskurna rättfärdiga av tro, så ock de oomskurna genom tron.
\par 31 Göra vi då vad lag är om intet genom tron? Bort det! Vi göra tvärtom lag gällande.

\chapter{4}

\par 1 Vad skola vi då säga om Abraham, vår stamfader efter köttet?
\par 2 Om Abraham blev rättfärdig av gärningar, så har han ju något att berömma sig av. Dock icke inför Gud.
\par 3 Ty vad säger skriften? "Abraham trodde Gud, och det räknades honom till rättfärdighet."
\par 4 Den som håller sig till gärningar, honom bliver lönen tillräknad icke på grund av nåd, utan på grund av förtjänst.
\par 5 Men den som icke håller sig till gärningar, utan tror på honom som gör den ogudaktige rättfärdig, honom räknas hans tro till rättfärdighet.
\par 6 Så prisar ock David den människa salig, som Gud tillräknar rättfärdighet, utan gärningar:
\par 7 "Saliga äro de vilkas överträdelser äro förlåtna, och vilkas synder äro överskylda.
\par 8 Salig är den man som Herren icke tillräknar synd."
\par 9 Gäller nu detta ordet "salig" de omskurna allenast eller ock de oomskurna? Vi säga ju att "tron räknades Abraham till rättfärdighet".
\par 10 Huru blev den honom då tillräknad? Skedde det sedan han hade blivit omskuren, eller medan han ännu var oomskuren? Det skedde icke sedan han hade blivit omskuren, utan medan han ännu var oomskuren.
\par 11 Och han undfick omskärelsens tecken såsom ett insegel på den rättfärdighet genom tron, som han hade, medan han ännu var oomskuren. Ty så skulle han bliva en fader för alla oomskurna som tro, och så skulle rättfärdighet tillräknas dem.
\par 12 Han skulle ock bliva en fader för omskurna, nämligen för sådana som icke allenast äro omskurna, utan ock vandra i spåren av den tro som vår fader Abraham hade, medan han ännu var oomskuren.
\par 13 Det var nämligen icke genom lag som Abraham och hans säd undfick det löftet att han skulle få världen till arvedel; det var genom rättfärdighet av tro.
\par 14 Ty om de som låta det bero på lag skola få arvedelen, så är tron till intet nyttig, och löftet är gjort om intet.
\par 15 Vad lagen kommer åstad är ju vredesdom; men där ingen lag finnes, där finnes icke heller någon överträdelse.
\par 16 Därför måste det bero på tro, för att det skulle vara av nåd, så att löftet kunde bliva beståndande för all hans säd, icke blott för dem som hörde till lagens folk, utan ock för dem som allenast hade Abrahams tro. Han är ju allas vår fader,
\par 17 enligt detta skriftens ord: "Jag har bestämt dig till att bliva en fader till många folk"; han är detta inför den Gud som han trodde, inför honom som gör de döda levande och kallar på de ting som icke äro till, likasom voro de till.
\par 18 Och där ingen förhoppning fanns, där hoppades han ändå och trodde; och han kunde så bliva "en fader till många folk", efter vad som var förutsagt: "Så skall din säd bliva."
\par 19 Och han försvagades icke i sin tro, när han betänkte huru hans egen kropp var såsom död - han var ju omkring hundra år gammal - och huru jämväl Saras moderliv var såsom dött.
\par 20 Han tvivlade icke på Guds löfte i otro, utan blev fastmer starkare i sin tro; ty han ärade Gud
\par 21 och var fullt viss om att vad Gud hade lovat, det var han också mäktig att hålla.
\par 22 Därför räknades det honom ock till rättfärdighet.
\par 23 Men att det så tillräknades honom, det är skrivet icke såsom gällde det allenast honom,
\par 24 utan det skulle gälla också oss; ty det skall tillräknas jämväl oss, oss som tro på honom som från de döda uppväckte Jesus, vår Herre,
\par 25 vilken utgavs för våra synders skull och uppväcktes för vår rättfärdiggörelses skull.

\chapter{5}

\par 1 Då vi nu hava blivit rättfärdiggjorda av tro, hava vi frid med Gud genom vår Herre Jesus Kristus
\par 2 - genom vilken vi ock hava fått tillträde till den nåd vari vi nu stå - och vi berömma oss i hoppet om Guds härlighet.
\par 3 Och icke det allenast, vi till och med berömma oss av våra lidanden, eftersom vi veta att lidandet verkar ståndaktighet,
\par 4 och ståndaktigheten beprövad fasthet, och fastheten hopp,
\par 5 och hoppet låter oss icke komma på skam; ty Guds kärlek är utgjuten i våra hjärtan genom den helige Ande, vilken har blivit oss given.
\par 6 Ty medan vi ännu voro svaga, led Kristus, när tiden var inne, döden för oss ogudaktiga.
\par 7 Näppeligen vill ju eljest någon dö ens för en rättfärdig man - om nu ock till äventyrs någon kan hava mod att dö för den som har gjort honom gott -
\par 8 men Gud bevisar sin kärlek till oss däri att Kristus dog för oss, medan vi ännu voro syndare.
\par 9 Så mycket mer skola vi därför, sedan vi nu hava blivit rättfärdiggjorda i och genom hans blod, också genom honom bliva frälsta undan vredesdomen.
\par 10 Ty om vi, medan vi voro Guds ovänner, blevo försonade med honom genom hans Sons död, så skola vi, sedan vi hava blivit försonade, ännu mycket mer bliva frälsta i och genom hans liv.
\par 11 Och icke det allenast; vi berömma oss ock av Gud genom vår Herre Jesus Kristus, genom vilken vi nu hava undfått försoningen.
\par 12 Därför är det så: Genom en enda människa har synden kommit in i världen och genom synden döden; och så har döden kommit över alla människor, eftersom de alla hava syndat.
\par 13 Ty synd fanns i världen redan innan lagen fanns. Men synd tillräknas icke där ingen lag finnes;
\par 14 och dock har under tiden från Adam till Moses döden haft väldet också över dem som icke hade syndat genom en överträdelse, i likhet med vad Adam gjorde, han som är en förebild till den som skulle komma.
\par 15 Likväl är det icke så med nådegåvan, som det var med syndafallet. Ty om genom en endas fall de många hava blivit döden underlagda, så har ännu mycket mer Guds nåd och gåvan i och genom nåd - vilken, också den, är kommen genom en enda människa, Jesus Kristus - blivit på ett överflödande sätt de många beskärd.
\par 16 Och med gåvan är det icke såsom det var med det som kom genom denne ene som syndade: domen kom genom en enda, och ledde till en fördömelsedom, men nådegåvan kom i följd av mångas fall, och ledde till en rättfärdiggörelsedom.
\par 17 Och om döden på grund av en endas fall kom till konungavälde genom denne ene, så skola ännu mycket mer de som undfå den överflödande nåden och rättfärdighetsgåvan få konungsligt välde i liv, också det genom en enda, Jesus Kristus. -
\par 18 Alltså, likasom det, som kom genom en endas fall, för alla människor ledde till en fördömelsedom, så leder det, som kom genom rättfärdiggörelsedomen förmedelst en enda, för alla människor till en rättfärdiggörelse som medför liv.
\par 19 Ty såsom genom en enda människas olydnad de många fingo stå såsom syndare, så skola ock genom en endas lydnad de många stå såsom rättfärdiga.
\par 20 Men lagen har därjämte kommit in, för att fallet skulle bliva så mycket större; dock, där synden blev större, där överflödade nåden mycket mer.
\par 21 Ty såsom synden hade utövat sitt välde i och genom döden, så skulle nu ock nåden genom rättfärdighet utöva sitt välde till evigt liv, och det genom Jesus Kristus, vår Herre.

\chapter{6}

\par 1 Vad skola vi då säga? Skola vi förbliva i synden, för att nåden skall bliva så mycket större?
\par 2 Bort det! Vi som hava dött från synden, huru skulle vi ännu kunna leva i den?
\par 3 Veten I då icke att vi alla som hava blivit döpta till Kristus Jesus, vi hava blivit döpta till hans död?
\par 4 Och vi hava så, genom detta dop till döden, blivit begravna med honom, för att, såsom Kristus uppväcktes från de döda genom Faderns härlighet, också vi skola vandra i ett nytt väsende, i liv.
\par 5 Ty om vi hava vuxit samman med honom genom en lika död, så skola vi ock vara sammanvuxna med honom genom en lika uppståndelse.
\par 6 Vi veta ju detta, att vår gamla människa har blivit korsfäst med honom, för att syndakroppen skall göras om intet, så att vi icke mer tjäna synden.
\par 7 Ty den som är död, han är friad ifrån synden.
\par 8 Hava vi nu dött med Kristus, så tro vi att vi ock skola leva med honom,
\par 9 eftersom vi veta att Kristus, sedan han har uppstått från de döda, icke mer dör; döden råder icke mer över honom.
\par 10 Ty hans död var en död från synden en gång för alla, men hans liv är ett liv för Gud.
\par 11 Så mån ock I hålla före att I ären döda från synden och leven för Gud, i Kristus Jesus.
\par 12 Låten därför icke synden hava väldet i edra dödliga kroppar, så att I lyden deras begärelser.
\par 13 Och ställen icke edra lemmar i syndens tjänst, att vara orättfärdighetsvapen, utan ställen eder själva i Guds tjänst, såsom de där från döden hava kommit till livet, och edra lemmar i Guds tjänst, att vara rättfärdighetsvapen.
\par 14 Ty synden skall icke råda över eder, eftersom I icke stån under lagen, utan under nåden.
\par 15 Huru är det alltså? Skola vi synda, eftersom vi icke stå under lagen, utan under nåden? Bort det!
\par 16 I veten ju, att när I ställen eder i någons tjänst för att lyda honom, så ären I tjänare under denne, som I sålunda lyden, vare sig det är under synden, vilket leder till död, eller under lydnaden, vilket leder till rättfärdighet.
\par 17 Men Gud vare tack för att den tid är förbi, då I voren syndens tjänare, och för att I haven blivit av hjärtat lydiga, så att I följen den lära som har givits eder till mönsterbild,
\par 18 och för att I, när I nu haven gjorts fria ifrån synden, haven blivit tjänare under rättfärdigheten -
\par 19 om jag nu får tala på människosätt, för eder köttsliga svaghets skull. Ja, likasom I förr ställden edra lemmar i orenhetens och orättfärdighetens tjänst, till orättfärdighet, så mån I nu ställa edra lemmar i rättfärdighetens tjänst, till helgelse.
\par 20 Medan I voren syndens tjänare, voren I ju fria ifrån rättfärdighetens tjänst;
\par 21 men vilken frukt skördaden I då därav? Jo, det som I nu blygens för; änden på sådant är ju döden.
\par 22 Men nu, då I haven gjorts fria ifrån synden och blivit Guds tjänare, nu skörden I frukten av detta: I varden helgade; och änden bliver att I undfån evigt liv.
\par 23 Ty den lön som synden giver är döden, men den gåva som Gud av nåd giver är evigt liv, i Kristus Jesus, vår Herre.

\chapter{7}

\par 1 Eller veten I icke, mina bröder - jag talar ju till sådana som känna lagen - att lagen råder över en människa för så lång tid som hon lever?
\par 2 Så är ju en gift kvinna genom lag bunden vid sin man, så länge denne lever; men om mannen dör, då är hon löst från den lag som band henne vid mannen.
\par 3 Alltså, om hon giver sig åt en annan man, medan hennes man lever, så kallas hon äktenskapsbryterska; men om mannen dör, då är hon fri ifrån lagen, så att hon icke är äktenskapsbryterska, om hon giver sig åt en annan man.
\par 4 Så haven ock I, mina bröder, genom Kristi kropp blivit dödade från lagen för att tillhöra en annan, nämligen honom som har uppstått från de döda, på det att vi må bära frukt åt Gud.
\par 5 Ty medan vi ännu voro i ett köttsligt väsende, voro de syndiga lustar, som uppväcktes genom lagen, verksamma i våra lemmar till att bära frukt åt döden.
\par 6 Men nu äro vi lösta från lagen, i det att vi hava dött från det varunder vi förr höllos fångna; och så tjäna vi nu i Andens nya väsende, och icke i bokstavens gamla väsende.
\par 7 Vad skola vi då säga? Är lagen synd? Bort det! Men synden skulle jag icke hava lärt känna, om icke genom lagen; ty jag hade icke vetat av begärelsen, om icke lagen hade sagt: "Du skall icke hava begärelse."
\par 8 Men då nu synden fick tillfälle, uppväckte den genom budordet allt slags begärelse i mig. Ty utan lag är synden död.
\par 9 Jag levde en gång utan lag; men när budordet kom, fick synden liv,
\par 10 och jag hemföll åt döden. Så befanns det att budordet, som var givet till liv, det blev mig till död;
\par 11 ty då synden fick tillfälle, förledde den mig genom budordet och dödade mig genom det.
\par 12 Alltså är visserligen lagen helig, och budordet heligt och rättfärdigt och gott.
\par 13 Har då verkligen det som är gott blivit mig till död? Bort det! Men synden har blivit det, för att så skulle varda uppenbart att den var synd, i det att den genom något som självt var gott drog över mig död; och så skulle synden bliva till övermått syndig, genom budordet.
\par 14 Vi veta ju att lagen är andlig, men jag är av köttslig natur, såld till träl under synden.
\par 15 Ty jag kan icke fatta att jag handlar såsom jag gör; jag gör ju icke vad jag vill, men vad jag hatar, det gör jag.
\par 16 Om jag nu gör det som jag icke vill, så giver jag mitt bifall åt lagen och vidgår att den är god.
\par 17 Så är det nu icke mer jag som gör sådant, utan synden, som bor i mig.
\par 18 Ty jag vet att i mig, det är i mitt kött, bor icke något gott; viljan är väl tillstädes hos mig, men att göra det goda förmår jag icke.
\par 19 Ja, det goda som jag vill gör jag icke; men det onda som jag icke vill, det gör jag.
\par 20 Om jag alltså gör vad jag icke vill, så är det icke mer jag som gör det, utan synden, som bor i mig.
\par 21 Så finner jag nu hos mig, som har viljan att göra det goda, den lagen, att det onda fastmer är tillstädes hos mig.
\par 22 Ty efter min invärtes människa har jag min lust i Guds lag;
\par 23 men i mina lemmar ser jag en annan lag, en som ligger i strid med den lag som är i min håg, en som gör mig till fånge under syndens lag, som är i mina lemmar.
\par 24 Jag arma människa! Vem skall frälsa mig från denna dödens kropp? -
\par 25 Gud vare tack, genom Jesus Kristus, vår Herre! Alltså tjänar jag, sådan jag är i mig själv, visserligen med min håg Guds lag, men med köttet tjänar jag syndens lag.

\chapter{8}

\par 1 Så finnes nu ingen fördömelse för dem som äro i Kristus Jesus.
\par 2 Ty livets Andes lag har i Kristus Jesus gjort mig fri ifrån syndens och dödens lag.
\par 3 Ty det som lagen icke kunde åstadkomma, i det den var försvagad genom köttet, det gjorde Gud, då han, för att borttaga synden, sände sin Son i syndigt kötts gestalt och fördömde synden i köttet.
\par 4 Så skulle lagens krav uppfyllas i oss, som vandra icke efter köttet, utan efter Anden.
\par 5 Ty de som äro köttsliga, de hava sitt sinne vänt till vad köttet tillhör; men de som äro andliga, de hava sitt sinne vänt till vad Anden tillhör.
\par 6 Och köttets sinne är död, medan Andens sinne är liv och frid.
\par 7 Köttets sinne är nämligen fiendskap mot Gud, eftersom det icke är Guds lag underdånigt, ej heller kan vara det.
\par 8 Men de som äro i ett köttsligt väsende kunna icke behaga Gud.
\par 9 I åter ären icke i ett köttsligt väsende, utan i ett andligt, om eljest Guds Ande bor i eder; men den som icke har Kristi Ande, han hör icke honom till.
\par 10 Om nu Kristus är i eder, så är väl kroppen hemfallen åt döden, för syndens skull, men Anden är liv, för rättfärdighetens skull.
\par 11 Och om dens Ande, som uppväckte Jesus från de döda, bor i eder, så skall han som uppväckte Kristus Jesus från de döda göra också edra dödliga kroppar levande, genom sin Ande, som bor i eder.
\par 12 Alltså, mina bröder, hava vi icke någon förpliktelse mot köttet, så att vi skola leva efter köttet.
\par 13 Ty om i leven efter köttet, så skolen I dö; men om I genom ande döden kroppens gärningar, så skolen I leva.
\par 14 Ty alla de som drivas av Guds Ande, de äro Guds barn.
\par 15 I haven ju icke fått en träldomens ande, så att I åter skullen känna fruktan; I haven fått en barnaskapets ande, i vilken vi ropa: "Abba! Fader!"
\par 16 Anden själv vittnar med vår ande att vi äro Guds barn.
\par 17 Men äro vi barn, så äro vi ock arvingar, nämligen Guds arvingar och Kristi medarvingar, om vi eljest lida med honom, för att också med honom bliva förhärligade.
\par 18 Ty jag håller före att denna tidens lidanden intet betyda, i jämförelse med den härlighet som kommer att uppenbaras på oss.
\par 19 Ty skapelsens trängtan sträcker sig efter Guds barns uppenbarelse.
\par 20 Skapelsen har ju blivit lagd under förgängligheten, icke av eget val, utan för dens skull, som lade den därunder; dock så, att en förhoppning skulle finnas,
\par 21 att också skapelsen en gång skall bliva frigjord ifrån sin träldom under förgängelsen och komma till den frihet som tillhör Guds barns härlighet.
\par 22 Vi veta ju att ännu i denna stund hela skapelsen samfällt suckar och våndas.
\par 23 Och icke den allenast; också vi själva, som hava fått Anden såsom förstlingsgåva, också vi sucka inom oss och bida efter barnaskapet, vår kropps förlossning.
\par 24 Ty i hoppet äro vi frälsta. Men ett hopp som man ser fullbordat är icke mer ett hopp; huru kan någon hoppas det som han redan ser?
\par 25 Om vi nu hoppas på det som vi icke se, så bida vi därefter med ståndaktighet.
\par 26 Så kommer ock Anden vår svaghet till hjälp; ty vad vi rätteligen böra bedja om, det veta vi icke, men Anden själv manar gott för oss med outsägliga suckar.
\par 27 Och han som rannsakar hjärtan, han vet vad Anden menar, ty det är efter Guds behag som han manar gott för de heliga.
\par 28 Men vi veta att för dem som älska Gud samverkar allt till det bästa, för dem som äro kallade efter hans rådslut.
\par 29 Ty dem som förut hava blivit kända av honom, dem har han ock förut bestämt till att bliva hans Sons avbilder, honom lika, så att denne skulle bliva den förstfödde bland många bröder.
\par 30 Och dem som han har förut bestämt, dem kallar han ock, och dem som han har kallat, dem rättfärdiggör han ock, och dem som han har rättfärdiggjort, dem förhärligar han ock.
\par 31 Vad skola vi nu säga härom? Är Gud för oss, vem kan då vara emot oss?
\par 32 Han som icke har skonat sin egen Son, utan utgivit honom för oss alla, huru skulle han kunna annat än också skänka oss allt med honom?
\par 33 Vem vill anklaga Guds utvalda? Gud är den som rättfärdiggör.
\par 34 Vem är den som vill fördöma? Kristus Jesus är den som har dött, ja, än mer, den som har uppstått; och han sitter på Guds högra sida, han manar ock gott för oss.
\par 35 Vem skulle kunna skilja oss från Kristi kärlek? Månne bedrövelse eller ångest eller förföljelse eller hunger eller nakenhet eller fara eller svärd?
\par 36 Så är ju skrivet: "För din skull varda vi dödade hela dagen; vi hava blivit aktade såsom slaktfår."
\par 37 Nej, i allt detta vinna vi en härlig seger genom honom som har älskat oss.
\par 38 Ty jag är viss om att varken död eller liv, varken änglar eller andefurstar, varken något som nu är eller något som skall komma,
\par 39 varken någon makt i höjden eller någon makt i djupet, ej heller något annat skapat skall kunna skilja oss från Guds kärlek i Kristus Jesus, vår Herre.

\chapter{9}

\par 1 Jag talar sanning i Kristus, jag ljuger icke - därom bär mitt samvete mig vittnesbörd i den helige Ande -
\par 2 när jag säger att jag har stor bedrövelse och oavlåtligt kval i mitt hjärta.
\par 3 Ja, jag skulle önska att jag själv vore förbannad och bortkastad från Kristus, om detta kunde gagna mina bröder, mina fränder efter köttet.
\par 4 De äro ju israeliter, dem tillhöra barnaskapet och härligheten och förbunden och lagstiftningen och tempeltjänsten och löftena.
\par 5 Dem tillhöra ock fäderna, och från dem är Kristus kommen efter köttet, han som är över allting, Gud, högtlovad i evighet, amen.
\par 6 Detta säger jag icke som om Guds löftesord skulle hava blivit om intet. Ty "Israel", det är icke detsamma som alla de som härstamma från Israel.
\par 7 Ej heller äro de alla "barn", därför att de äro Abrahams säd. Nej, det heter: "Genom Isak är det som säd skall uppkallas efter dig."
\par 8 Detta vill säga: Icke de äro Guds barn, som äro barn efter köttet, men de som äro barn efter löftet, de räknas för säd.
\par 9 Ty ett löftesord var det ordet: "Vid denna tid skall jag komma tillbaka, och då skall Sara hava en son."
\par 10 Än mer: så skedde ock, när Rebecka genom en och samme man, nämligen vår fader Isak, blev moder till sina barn.
\par 11 Ty förrän dessa voro födda, och innan de ännu hade gjort vare sig gott eller ont, blev det ordet henne sagt - för att Guds utkorelse-rådslut skulle bliva beståndande, varvid det icke skulle bero på någons gärningar, utan på honom som kallar -
\par 12 det ordet: "Den äldre skall tjäna den yngre."
\par 13 Så är ock skrivet: "Jakob älskade jag, men Esau hatade jag."
\par 14 Vad skola vi då säga? Kan väl orättfärdighet finnas hos Gud? Bort det!
\par 15 Han säger ju till Moses: "Jag skall vara barmhärtig mot den jag vill vara barmhärtig emot, och jag skall förbarma mig över den jag vill förbarma mig över."
\par 16 Alltså beror det icke på någon människas vilja eller strävan, utan på Guds barmhärtighet.
\par 17 Ty skriften säger till Farao: "Just därtill har jag låtit dig uppstå, att jag skall visa min makt på dig, och att mitt namn skall varda förkunnat på hela jorden."
\par 18 Alltså är han barmhärtig mot vem han vill, och vem han vill förhärdar han.
\par 19 Nu torde du säga till mig: "Vad har han då att förebrå oss? Kan väl någon stå emot hans vilja?"
\par 20 O människa, vem är då du, som vill träta med Gud? Icke skall verket säga till sin mästare: "Varför gjorde du mig så?"
\par 21 Har icke krukmakaren den makten över leret, att han av samma lerklump kan göra ett kärl till hedersamt bruk, ett annat till mindre hedersamt?
\par 22 Men om nu Gud, när han ville visa sin vrede och uppenbara sin makt, likväl i stor långmodighet hade fördrag med "vredens kärl", som voro färdiga till fördärv, vad har du då att säga?
\par 23 Och om han gjorde detta för att tillika få uppenbara sin härlighets rikedom på "barmhärtighetens kärl", som han förut hade berett till härlighet?
\par 24 Och till att vara sådana har han ock kallat oss, icke allenast dem som äro av judisk börd, utan jämväl dem som äro av hednisk.
\par 25 Så säger han ock hos Oseas: "Det folk som icke var mitt folk, det skall jag kalla 'mitt folk', och henne som jag icke älskade skall jag kalla 'min älskade'.
\par 26 Och det skall ske att på den ort där det sades till dem: 'I ären icke mitt folk', där skola de kallas 'den levande Gudens barn'."
\par 27 Men Esaias utropar om Israel: "Om än Israels barn vore till antalet såsom sanden i havet, så skall dock allenast en kvarleva bliva frälst.
\par 28 Ty dom skall Herren hålla på jorden, en slutdom, som avgör saken med hast."
\par 29 Och det är såsom redan Esaias har sagt: "Om Herren Sebaot icke hade lämnat en avkomma kvar åt oss, då vore vi såsom Sodom, vi vore Gomorra lika."
\par 30 Vad skola vi då säga? Jo, att hedningarna, som icke foro efter rättfärdighet, hava vunnit rättfärdighet, nämligen den rättfärdighet som kommer av tro,
\par 31 under det att Israel, som for efter en rättfärdighetslag, icke har kommit till någon sådan lag.
\par 32 Varför? Därför att de icke sökte den på trons väg, utan såsom något som skulle vinnas på gärningarnas väg. De stötte sig mot stötestenen,
\par 33 såsom det är skrivet: "Se, jag lägger i Sion en stötesten och en klippa som skall bliva dem till fall; men den som tror på den skall icke komma på skam."

\chapter{10}

\par 1 Mina bröder, mitt hjärtas åstundan och min bön till Gud för dem är att de må bliva frälsta.
\par 2 Ty det vittnesbördet giver jag dem, att de nitälska för Gud. Dock göra de detta icke med rätt insikt.
\par 3 De förstå nämligen icke rättfärdigheten från Gud, utan söka att komma åstad en sin egen rättfärdighet och hava icke givit sig under rättfärdigheten från Gud.
\par 4 Ty lagen har fått sin ände i Kristus, till rättfärdighet för var och en som tror.
\par 5 Moses skriver ju om den rättfärdighet som kommer av lagen, att den människa som övar sådan rättfärdighet skall leva genom den.
\par 6 Men den rättfärdighet som kommer av tro säger så: "Du behöver icke fråga i ditt hjärta: 'Vem vill fara upp till himmelen (nämligen för att hämta Kristus ned)?'
\par 7 ej heller: 'Vem vill fara ned till avgrunden (nämligen för att hämta Kristus upp ifrån de döda?'"
\par 8 Vad säger den då? "Ordet är dig nära, i din mun och i ditt hjärta (nämligen ordet om tron, det som vi predika)."
\par 9 Ty om du med din mun bekänner Jesus vara Herre och i ditt hjärta tror att Gud har uppväckt honom från de döda, då bliver du frälst.
\par 10 Ty genom hjärtats tro bliver man rättfärdig, och genom munnens bekännelse bliver man frälst.
\par 11 Skriften säger ju: "Ingen som tror på honom skall komma på skam."
\par 12 Det är ingen åtskillnad mellan jude och grek; alla hava ju en och samme Herre, och han har rikedomar att giva åt alla som åkalla honom.
\par 13 Ty "var och en som åkallar Herrens namn, han skall varda frälst".
\par 14 Men huru skulle de kunna åkalla den som de icke hava kommit till tro på? Och huru skulle de kunna tro den som de icke hava hört? Och huru skulle de kunna höra, om ingen predikade?
\par 15 Och huru skulle predikare kunna komma, om de icke bleve sända? Så är och skrivet: "Huru ljuvliga äro icke fotstegen av de män som frambära gott budskap!"
\par 16 Dock, icke alla hava blivit evangelium lydiga. Esaias säger ju: "Herre, vem trodde vad som predikades för oss?"
\par 17 Alltså kommer tron av predikan, men predikan i kraft av Kristi ord.
\par 18 Jag frågar då: Hava de kanhända icke hört predikas? Jo, visserligen; det heter ju: "Deras tal har gått ut över hela jorden, och deras ord till världens ändar."
\par 19 Jag frågar då vidare: Har Israel kanhända icke förstått det? Redan Moses säger: "Jag skall uppväcka eder avund mot ett folk som icke är ett folk; mot ett hednafolk utan förstånd skall jag reta eder till vrede."
\par 20 Och Esaias går så långt, att han säger: "Jag har låtit mig finnas av dem som icke sökte mig, jag har låtit mig bliva uppenbar för dem som icke frågade efter mig."
\par 21 Men om Israel säger han: "Hela dagen har jag uträckt mina händer till ett ohörsamt och gensträvigt folk."

\chapter{11}

\par 1 Så frågar jag nu: Har då Gud förskjutit sitt folk? Bort det! Jag är ju själv en israelit, av Abrahams säd och av Benjamins stam.
\par 2 Gud har icke förskjutit sitt folk, som redan förut hade blivit känt av honom. Eller veten I icke vad skriften säger, där den talar om Elias, huru denne inför Gud träder upp mot Israel med dessa ord:
\par 3 "Herre, de hava dräpt dina profeter och rivit ned dina altaren; jag allena är kvar, och de stå efter mitt liv"?
\par 4 Och vad får han då för svar av Gud? "Jag har låtit bliva kvar åt mig sju tusen män, som icke hava böjt knä för Baal."
\par 5 Likaså finnes ock, i den tid som nu är, en kvarleva, i kraft av en utkorelse som har skett av nåd.
\par 6 Men har den skett av nåd, så har den icke skett på grund av gärningar; annars vore nåd icke mer nåd.
\par 7 Huru är det alltså? Vad Israel står efter, det har det icke fått; allenast de utvalda hava fått det, medan de andra hava blivit förstockade.
\par 8 Så är ju skrivet: "Gud har givit dem en sömnaktighetens ande, ögon som de icke kunna se med och öron som de icke kunna höra med; så är det ännu i dag."
\par 9 Och David säger: "Må deras bord bliva dem till en snara, så att de bliva fångade; må det bliva dem till ett giller, så att de få sin vedergällning.
\par 10 Må deras ögon förmörkas, så att de icke se; böj deras rygg alltid."
\par 11 Så frågar jag nu: Var det då för att de skulle komma på fall som de stapplade? Bort det! Men genom deras fall har frälsningen kommit till hedningarna, för att de själva skola "uppväckas till avund".
\par 12 Och har nu redan deras fall varit till rikedom för världen, och har deras fåtalighet varit till rikedom för hedningarna, huru mycket mer skall icke deras fulltalighet så bliva!
\par 13 Men till eder, I som ären av hednisk börd, säger jag: Eftersom jag nu är en hedningarnas apostel, håller jag mitt ämbete högt -
\par 14 om jag till äventyrs så skulle kunna "uppväcka avund" hos dem som äro mitt kött och blod och frälsa några bland dem.
\par 15 Ty om redan deras förkastelse hade med sig världens försoning, vad skall då deras upptagande hava med sig, om icke liv från de döda?
\par 16 Om förstlingsbrödet är heligt, så är ock hela degen helig; och om roten är helig, så äro ock grenarna heliga.
\par 17 Men om nu några av grenarna hava brutits bort, och du, som är av ett vilt olivträd, har blivit inympad bland grenarna och med dem har fått delaktighet i det äkta olivträdets saftrika rot,
\par 18 så må du icke därför förhäva dig över grenarna. Nej, om du skulle vilja förhäva dig, så besinna att det icke är du som bär roten, utan att roten bär dig.
\par 19 Nu säger du kanhända: "Det var för att jag skulle bliva inympad som en del grenar brötos bort."
\par 20 Visserligen. För sin otros skull blevo de bortbrutna, och du får vara kvar genom din tro. Hav då inga högmodiga tankar, utan lev i fruktan.
\par 21 Ty har Gud icke skonat de naturliga grenarna, så skall han icke heller skona dig.
\par 22 Se alltså här Guds godhet och stränghet: Guds stränghet mot dem som föllo och hans godhet mot dig, om du nämligen håller dig fast vid hans godhet; annars bliver också du borthuggen.
\par 23 Men jämväl de andra skola bliva inympade, om de icke hålla fast vid sin otro; Gud är ju mäktig att åter inympa dem.
\par 24 Ty om du har blivit borthuggen från ditt av naturen vilda olivträd och mot naturen inympats i ett ädelt olivträd, huru mycket snarare skola då icke dessa kunna inympas i sitt eget äkta olivträd, det som de efter naturen tillhöra!
\par 25 Ty för att I, mina bröder, icke skolen hålla eder själva för kloka, vill jag yppa för eder denna hemlighet: Förstockelse har drabbat en del av Israel och skall fortfara intill dess hedningarna i fulltalig skara hava kommit in;
\par 26 och så skall hela Israel bliva frälst, såsom det är skrivet: "Från Sion skall förlossaren komma, han skall skaffa bort all ogudaktighet från Jakob.
\par 27 Och när jag borttager deras synder, då skall detta vara det förbund, som jag gör med dem."
\par 28 Se vi nu på evangelium, så äro de hans ovänner, för eder skull; men se vi på utkorelsen, så äro de hans älskade, för fädernas skull.
\par 29 Ty sina nådegåvor och sin kallelse kan Gud icke ångra.
\par 30 Såsom I förut voren ohörsamma mot Gud, men nu genom dessas ohörsamhet haven fått barmhärtighet,
\par 31 så hava nu ock dessa varit ohörsamma, för att de, genom den barmhärtighet som har vederfarits eder, också själva skola få barmhärtighet.
\par 32 Ty Gud har givit dem alla till pris åt ohörsamhet, för att sedan förbarma sig över dem alla.
\par 33 O, vilket djup av rikedom och vishet och kunskap hos Gud! Huru outgrundliga äro icke hans domar, och huru outrannsakliga hans vägar!
\par 34 Ty "vem har lärt känna Herrens sinne, eller vem har varit hans rådgivare?
\par 35 Eller vem har först givit honom något, som han alltså bör betala igen?"
\par 36 Av honom och genom honom och till honom är ju allting. Honom tillhör äran i evighet, amen.

\chapter{12}

\par 1 Så förmanar jag nu eder, mina bröder, vid Guds barmhärtighet, att frambära edra kroppar till ett levande, heligt och Gud välbehagligt offer - eder andliga tempeltjänst.
\par 2 Och skicken eder icke efter denna tidsålders väsende, utan förvandlen eder genom edert sinnes förnyelse, så att I kunnen pröva vad som är Guds vilja, vad som är gott och välbehagligt och fullkomligt.
\par 3 Ty i kraft av den nåd som har blivit mig given, tillsäger jag var och en av eder att icke hava högre tankar om sig än tillbörligt är, utan tänka blygsamt, i överensstämmelse med det mått av tro som Gud har tilldelat var och en.
\par 4 Ty såsom vi i en och samma kropp hava många lemmar, men alla lemmarna icke hava samma förrättning,
\par 5 så utgöra ock vi, fastän många, en enda kropp i Kristus, men var för sig äro vi lemmar, varandra till tjänst.
\par 6 Och vi hava olika gåvor, alltefter den nåd som har blivit oss given. Har någon profetians gåva, så bruke han den efter måttet av sin tro;
\par 7 har någon fått en tjänst, så akte han på tjänsten; är någon satt till lärare, så akte han på sitt lärarkall;
\par 8 är någon satt till att förmana, så akte han på sin plikt att förmana. Den som delar ut gåvor, han göre det med gott hjärta; den som är satt till föreståndare, han vare det med nit; den som övar barmhärtighet, han göre det med glädje.
\par 9 Eder kärlek vare utan skrymtan; avskyn det onda, hållen fast vid det goda.
\par 10 Älsken varandra av hjärtat i broderlig kärlek; söken överträffa varandra i inbördes hedersbevisning.
\par 11 Varen icke tröga, där det gäller nit; varen brinnande i anden, tjänen Herren.
\par 12 Varen glada i hoppet, tåliga i bedrövelsen, uthålliga i bönen.
\par 13 Tagen del i de heligas behov. Varen angelägna om att bevisa gästvänlighet.
\par 14 Välsignen dem som förfölja eder; välsignen, och förbannen icke.
\par 15 Glädjens med dem som äro glada, gråten med dem som gråta.
\par 16 Varen ens till sinnes med varandra. Haven icke edert sinne vänt till vad högt är, utan hållen eder till det som är ringa. Hållen icke eder själva för kloka.
\par 17 Vedergällen ingen med ont för ont. Vinnläggen eder om vad gott är inför var man.
\par 18 Hållen frid med alla människor, om möjligt är, och så mycket som på eder beror.
\par 19 Hämnens icke eder själva, mina älskade, utan lämnen rum för vredesdomen; ty det är skrivet: "Min är hämnden, jag skall vedergälla det, säger Herren."
\par 20 Fastmer, "om din ovän är hungrig, så giv honom att äta, om han är törstig, så giv honom att dricka; ty om du så gör, samlar du glödande kol på hans huvud."
\par 21 Låt dig icke övervinnas av det onda, utan övervinn det onda med det goda.

\chapter{13}

\par 1 Var och en vare underdånig den överhet som han har över sig. Ty ingen överhet finnes, som icke är av Gud; all överhet som finnes är förordnad av Gud.
\par 2 Därför, den som sätter sig upp mot överheten, han står emot vad Gud har förordnat; men de som stå emot detta, de skola få sin dom.
\par 3 Ty de som hava väldet äro till skräck, icke för dem som göra vad gott är, utan för dem som göra vad ont är. Vill du vara utan fruktan för överheten, så gör vad gott är; du skall då bliva prisad av den,
\par 4 ty överheten är en Guds tjänare, dig till fromma. Men gör du vad ont är, då må du frukta; ty överheten bär icke svärdet förgäves, utan är en Guds tjänare, en hämnare, till att utföra vredesdomen över den som gör vad ont är.
\par 5 Därför måste man vara den underdånig, icke allenast för vredesdomens skull, utan ock för samvetets skull.
\par 6 Fördenskull betalen I ju ock skatt; ty överheten förrättar Guds tjänst och är just för detta ändamål ständigt verksam.
\par 7 Så given åt alla vad I ären dem skyldiga; skatt åt den som skatt tillkommer, tull åt den som tull tillkommer, fruktan åt den som fruktan tillkommer, heder åt den som heder tillkommer.
\par 8 Varen ingen något skyldiga - utom när det gäller kärlek till varandra; ty den som älskar sin nästa, han har uppfyllt lagen.
\par 9 De buden: "Du skall icke begå äktenskapsbrott", "Du skall icke dräpa", "Du skall icke stjäla", "Du skall icke hava begärelse" och vilka andra bud som helst, de sammanfattas ju alla i det ordet: "Du skall älska din nästa såsom dig själv."
\par 10 Kärleken gör intet ont mot nästan; alltså är kärleken lagens uppfyllelse.
\par 11 Akten på allt detta, så mycket mer som I veten vad tiden lider, att stunden nu är inne för eder att vakna upp ur sömnen. Ty frälsningen är oss nu närmare, än då vi kommo till tro.
\par 12 Natten är framskriden, och dagen är nära. Låtom oss därför avlägga mörkrets gärningar och ikläda oss ljusets vapenrustning.
\par 13 Låtom oss föra en hövisk vandel, såsom om dagen, icke med vilt leverne och dryckenskap, icke i otukt och lösaktighet, icke i kiv och avund.
\par 14 Ikläden eder fastmer Herren Jesus Kristus, och haven icke sådan omsorg om köttet, att onda begärelser därav uppväckas.

\chapter{14}

\par 1 Om någon är svag i tron, så upptagen honom dock vänligt, utan att döma över andras betänkligheter.
\par 2 Den ene har tro till att äta vad som helst, under det att den som är svag allenast äter vad som växer på jorden.
\par 3 Den som äter må icke förakta den som icke äter. Ej heller må den som icke äter döma den som äter; ty Gud har upptagit honom,
\par 4 och vem är du som dömer en annans tjänare? Om han står eller faller, det kommer allenast hans egen herre vid; men han skall väl bliva stående, ty Herren är mäktig att hålla honom stående.
\par 5 Den ene gör skillnad mellan dag och dag, den andre håller alla dagar för lika; var och en vare fullt viss i sitt sinne.
\par 6 Om någon särskilt aktar på någon dag, så gör han detta för Herren, och om någon äter, så gör han detta för Herren; han tackar ju Gud. Så ock, om någon avhåller sig från att äta, gör han detta för Herren, och han tackar Gud.
\par 7 Ty ingen av oss lever för sig själv, och ingen dör för sig själv.
\par 8 Leva vi, så leva vi för Herren; dö vi, så dö vi för Herren. Evad vi leva eller dö, höra vi alltså Herren till.
\par 9 Ty därför har Kristus dött och åter blivit levande, att han skall vara herre över både döda och levande.
\par 10 Men du, varför dömer du din broder? Och du åter, varför föraktar du din broder? Vi skola ju alla en gång stå inför Guds domstol.
\par 11 Ty det är skrivet: "Så sant jag lever, säger Herren, för mig skola alla knän böja sig, och alla tungor skola prisa Gud."
\par 12 Alltså skall var och en av oss inför Gud göra räkenskap för sig själv.
\par 13 Låtom oss därför icke mer döma varandra. Dömen hellre så, att ingen må för sin broder lägga en stötesten eller något som bliver honom till fall.
\par 14 Jag vet väl och är i Herren Jesus viss om att intet i sig självt är orent; allenast om någon håller något för orent, så är det för honom orent.
\par 15 Om nu genom din mat bekymmer vållas din broder, så vandrar du icke mer i kärleken. Bliv icke genom din mat till fördärv för den som Kristus har lidit döden för.
\par 16 Låten alltså icke det goda som I haven fått bliva utsatt för smädelse.
\par 17 Ty Guds rike består icke i mat och dryck, utan i rättfärdighet och frid och glädje i den helige Ande.
\par 18 Den som häri tjänar Kristus, han är välbehaglig för Gud och håller provet inför människor.
\par 19 Vi vilja alltså fara efter det som länder till frid och till inbördes uppbyggelse.
\par 20 Bryt icke för mats skull ned Guds verk. Väl är allting rent, men om ätandet för någon är en stötesten, så bliver det för den människan till ondo;
\par 21 du gör väl i att avhålla dig från att äta kött och dricka vin och från annat som för din broder bliver en stötesten.
\par 22 Den tro du har må du hava för dig själv inför Gud. Salig är den som icke måste döma sig själv, när det gäller något som han har prövat vara rätt.
\par 23 Men om någon hyser betänkligheter och likväl äter, då är han dömd, eftersom det icke sker av tro. Ty allt som icke sker av tro, det är synd.

\chapter{15}

\par 1 Vi som äro starka äro pliktiga att bära de svagas skröpligheter och att icke leva oss själva till behag.
\par 2 Var och en av oss må leva sin nästa till behag, honom till fromma och honom till uppbyggelse.
\par 3 Kristus levde ju icke sig själv till behag, utan med honom skedde såsom det är skrivet: "Dina smädares smädelser hava fallit över mig."
\par 4 Ty allt vad som fordom har blivit skrivet, det är skrivet oss till undervisning, för att vi, genom ståndaktighet och genom den tröst som skrifterna giva, skola bevara vårt hopp.
\par 5 Och ståndaktighetens och tröstens Gud give eder att vara ens till sinnes med varandra i Kristi Jesu efterföljelse,
\par 6 så att I endräktigt och med en mun prisen vår Herres, Jesu Kristi, Gud och Fader.
\par 7 Därför må den ene av eder vänligt upptaga den andre, såsom Kristus, Gud till ära, har upptagit eder.
\par 8 Vad jag vill säga är detta: För de omskurna har Kristus blivit en tjänare, till ett vittnesbörd om Guds sannfärdighet, för att bekräfta de löften som hade givits åt fäderna;
\par 9 hedningarna åter hava fått prisa Gud för hans barmhärtighets skull. Så är ock skrivet: "Fördenskull vill jag prisa dig bland hedningarna och lovsjunga ditt namn."
\par 10 Och åter heter det: "Jublen, I hedningar, med hans folk";
\par 11 så ock: "Loven Herren, alla hedningar, ja, honom prise alla folk."
\par 12 Så säger ock Esaias: "Telningen från Jessais rot skall komma, ja, han som skall stå upp för att råda över hedningarna; på honom skola hedningarna hoppas."
\par 13 Men hoppets Gud uppfylle eder med all glädje och frid i tron, så att I haven ett överflödande hopp i den helige Andes kraft.
\par 14 Jag är väl redan nu viss om att I, mina bröder, av eder själva ären fulla av godhet, uppfyllda med all kunskap, i stånd jämväl att förmana varandra.
\par 15 Dock har jag, på ett delvis något dristigt sätt, skrivit till eder med ytterligare påminnelser, detta i kraft av den nåd som har blivit mig given av Gud:
\par 16 att jag nämligen skall förrätta Kristi Jesu tjänst bland hedningarna och vara en prästerlig förvaltare av Guds evangelium, så att hedningarna bliva ett honom välbehagligt offer, helgat i den helige Ande.
\par 17 Alltså är det i Kristus Jesus som jag har något att berömma mig av i fråga om min tjänst inför Gud.
\par 18 Ty jag skall icke drista mig att orda om något annat än vad Kristus, för att göra hedningarna lydaktiga, har verkat genom mig, med ord och med gärning,
\par 19 genom kraften i tecken och under, genom Andens kraft. Så har jag, från Jerusalem och runt omkring ända till Illyrien, överallt förkunnat evangelium om Kristus.
\par 20 Och jag har härvid satt min ära i att icke förkunna evangelium, där Kristi namn redan var känt, ty jag ville icke bygga på en annans grundval;
\par 21 utan så har skett, som skrivet är: "De för vilka intet har varit förkunnat om honom skola få se, och de som intet hava hört skola förstå."
\par 22 Det är också härigenom som jag så många gånger har blivit förhindrad att komma till eder.
\par 23 Men då jag nu icke mer har något att uträtta i dessa trakter och under ganska många år har längtat efter att komma till eder,
\par 24 vill jag besöka eder, när jag begiver mig till Spanien. Jag hoppas nämligen att på genomresan få se eder och att därefter av eder bliva utrustad för färden dit, sedan jag först i någon mån har fått min längtan efter eder stillad.
\par 25 Men nu far jag till Jerusalem med understöd åt de heliga.
\par 26 Macedonien och Akaja hava nämligen känt sig manade att göra ett sammanskott åt dem bland de heliga i Jerusalem, som leva i fattigdom.
\par 27 Ja, därtill hava de känt sig manade; de stå också i skuld hos dem. Ty om hedningarna hava fått del i deras andliga goda, så äro de å sin sida skyldiga att vara dem till tjänst med sitt lekamliga goda. -
\par 28 När jag så har fullgjort detta och lämnat i deras händer vad som har blivit insamlat, ämnar jag därifrån begiva mig till Spanien och taga vägen genom eder stad.
\par 29 Och jag vet, att när jag kommer till eder, kommer jag med Kristi välsignelse i fullt mått.
\par 30 Och nu uppmanar jag eder, mina bröder, vid vår Herre Jesus Kristus och vid vår kärlek i Anden, att bistå mig i min kamp, genom att bedja för mig till Gud,
\par 31 att jag må bliva frälst undan de ohörsamma i Judeen, och att det understöd som jag för med mig till Jerusalem må bliva väl mottaget av de heliga.
\par 32 Så skall jag, om Gud vill, med glädje komma till eder och vederkvicka mig tillsammans med eder.
\par 33 Fridens Gud vare med eder alla. Amen.

\chapter{16}

\par 1 Jag anbefaller åt eder vår syster Febe, som är församlingstjänarinna i Kenkrea.
\par 2 Så mottagen då henne i Herren, såsom det höves de heliga, och bistån henne i allt vari hon kan behöva eder; ty hon har själv varit ett stöd för många och jämväl för mig.
\par 3 Hälsen Priska och Akvila, mina medarbetare i Kristus Jesus.
\par 4 De hava ju vågat sitt liv för mig; och icke allenast jag tackar dem därför, utan också alla hednaförsamlingar.
\par 5 Hälsen ock den församling som kommer tillhopa i deras hus. Hälsen Epenetus, min älskade broder, som är förstlingen av dem som i provinsen Asien hava kommit till Kristus.
\par 6 Hälsen Maria, som har arbetat så mycket för eder.
\par 7 Hälsen Andronikus och Junias, mina landsmän och medfångar, som hava ett så gott anseende bland apostlarna, och som längre än jag hava varit i Kristus.
\par 8 Hälsen Ampliatus, min älskade broder i Herren.
\par 9 Hälsen Urbanus, vår medarbetare i Kristus, och Stakys, min älskade broder.
\par 10 Hälsen Apelles, den i Kristus beprövade. Hälsen dem som höra till Aristobulus' hus.
\par 11 Hälsen Herodion, min landsman. Hälsen dem av Narcissus' hus, som äro i Herren.
\par 12 Hälsen Tryfena och Tryfosa, som arbeta i Herren. Hälsen Persis, den älskade systern, som har så mycket arbetat i Herren.
\par 13 Hälsen Rufus, den i Herren utvalde, och hans moder, som också för mig har varit en moder.
\par 14 Hälsen Asynkritus, Flegon, Hermes, Patrobas, Hermas och de bröder som äro tillsammans med dem.
\par 15 Hälsen Filologus och Julia, Nereus och hans syster och Olympas och alla de heliga som äro tillsammans med dem.
\par 16 Hälsen varandra med en helig kyss. Alla Kristi församlingar hälsa eder.
\par 17 Men jag förmanar eder, mina bröder, att hava akt på dem som vålla tvedräkt och kunna bliva eder till fall, i strid med den lära som I haven inhämtat; dragen eder ifrån dem.
\par 18 Ty sådana tjäna icke vår Herre Kristus, utan sin egen buk; och genom sina milda ord och sitt fagra tal bedraga de oskyldiga människors hjärtan.
\par 19 Eder lydnad är ju känd av alla. Över eder gläder jag mig därför; men jag skulle önska att I voren visa i fråga om det goda, och menlösa i fråga om det onda.
\par 20 Och fridens Gud skall snart låta Satan bliva krossad under edra fötter. Vår Herres, Jesu Kristi, nåd vare med eder.
\par 21 Timoteus, min medarbetare, hälsar eder: så göra ock Lucius och Jason och Sosipater, mina landsmän.
\par 22 Jag, Tertius, som har nedskrivit detta brev, hälsar eder i Herren.
\par 23 Gajus, min och hela församlingens värd, hälsar eder. Erastus, stadens kamrerare, och brodern Kvartus hälsar eder.
\par 24 Vår Herres, Jesu Kristi, nåd vare med eder alla. Amen.
\par 25 Men honom som förmår styrka eder, enligt det evangelium jag förkunnar och min predikan om Jesus Kristus, ja, enligt den nu avslöjade hemlighet som förut under evärdliga tider har varit outtalad,
\par 26 men som nu har blivit uppenbarad och, enligt den eviga Gudens befallning, blivit, med stöd av profetiska skrifter, kungjord bland alla hedningar, för att bland dem upprätta trons lydnad -
\par 27 honom, den ende vise Guden, tillhör äran, genom Jesus Kristus, i evigheternas evigheter. Amen.


\end{document}