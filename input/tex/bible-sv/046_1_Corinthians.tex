\begin{document}

\title{1 Corinthians}

1Co 1:1  Paulus, genom Guds vilja kallad till Kristi Jesu apostel, så ock brodern Sostenes,
1Co 1:2  hälsar den Guds församling som finnes i Korint, de i Kristus Jesus helgade, dem som äro kallade och heliga, jämte alla andra som åkalla vår Herres, Jesu Kristi, namn, på alla orter där de eller vi bo.
1Co 1:3  Nåd vare med eder och frid ifrån Gud, vår Fader, och Herren Jesus Kristus.
1Co 1:4  Jag tackar Gud alltid för eder skull, för den Guds nåd som har blivit eder given i Kristus Jesus,
1Co 1:5  att I haven i honom blivit rikligen begåvade i alla stycken, i fråga om allt vad tal och kunskap heter.
1Co 1:6  Så har ju ock vittnesbördet om Kristus blivit befäst hos eder,
1Co 1:7  så att I icke stån tillbaka i fråga om någon nådegåva, medan I vänten på vår Herres, Jesu Kristi, uppenbarelse.
1Co 1:8  Han skall ock göra eder ståndaktiga intill änden, så att I ären ostraffliga på vår Herres, Jesu Kristi, dag.
1Co 1:9  Gud är trofast, han genom vilken I haven blivit kallade till gemenskap med hans Son, Jesus Kristus, vår Herre.
1Co 1:10  Men jag förmanar eder, mina bröder, vid vår Herres, Jesu Kristi, namn, att alla vara eniga i edert tal och att icke låta söndringar finnas bland eder, utan hålla fast tillhopa i samma sinnelag och samma tänkesätt.
1Co 1:11  Det har nämligen av Kloes husfolk blivit mig berättat om eder, mina bröder, att tvister hava uppstått bland eder.
1Co 1:12  Härmed menar jag att bland eder den ene säger: "Jag håller mig till Paulus", den andre: "Jag håller mig till Apollos", en annan: "Jag håller mig till Cefas", åter en annan: "Jag håller mig till Kristus." -
1Co 1:13  Är då Kristus delad? Icke blev väl Paulus korsfäst för eder? Och icke bleven I väl döpta i Paulus' namn?
1Co 1:14  Jag tackar Gud för att jag icke har döpt någon bland eder utom Krispus och Gajus,
1Co 1:15  så att ingen kan säga att I haven blivit döpta i mitt namn.
1Co 1:16  Dock, jag har döpt också Stefanas' husfolk; om jag eljest har döpt någon vet jag icke.
1Co 1:17  Ty Kristus har icke sänt mig till att döpa, utan till att förkunna evangelium, och detta icke med en visdom som består i ord, för att Kristi kors icke skall berövas sin kraft.
1Co 1:18  Ty talet om korset är visserligen en dårskap för dem som gå förlorade, men för oss som bliva frälsta är det en Guds kraft.
1Co 1:19  Det är ju skrivet: "Jag skall göra de visas vishet om intet, och de förståndigas förstånd skall jag slå ned."
1Co 1:20  Ja, var äro de visa? Var äro de skriftlärda? Var äro denna tidsålders klyftiga män? Har icke Gud gjort denna världens visdom till dårskap?
1Co 1:21  Jo, eftersom världen icke genom sin visdom lärde känna Gud i hans visdom, behagade det Gud att genom den dårskap han lät predikas frälsa dem som tro.
1Co 1:22  Ty judarna begära tecken, och grekerna åstunda visdom,
1Co 1:23  vi åter predika en korsfäst Kristus, en som för judarna är en stötesten och för hedningarna en dårskap,
1Co 1:24  men som för de kallade, vare sig judar eller greker, är en Kristus som är Guds kraft och Guds visdom.
1Co 1:25  Ty Guds dårskap är visare än människor, och Guds svaghet är starkare än människor.
1Co 1:26  Ty betänken, mina bröder, huru det var vid eder kallelse: icke många som voro visa efter köttet blevo kallade, icke många mäktiga, icke många av förnämlig släkt.
1Co 1:27  Men det som för världen var dåraktigt, det utvalde Gud, för att han skulle låta de visa komma på skam.
1Co 1:28  Och det som i världen var svagt, det utvalde Gud, för att han skulle låta det starka komma på skam. Och det som i världen var ringa och föraktat, det utvalde Gud - ja, det som ingenting var - för att han skulle göra det till intet, som någonting var.
1Co 1:29  Ty han ville icke att något kött skulle kunna berömma sig inför Gud.
1Co 1:30  Men hans verk är det, att I ären i Kristus Jesus, som för oss har blivit till visdom från Gud, till rättfärdighet och helgelse och till förlossning,
1Co 1:31  för att så skall ske, som det är skrivet: "Den som vill berömma sig, han berömme sig av Herren."
1Co 2:1  När jag kom till eder, mina bröder, var det också icke med höga ord eller hög visdom som jag kom och frambar för eder Guds vittnesbörd.
1Co 2:2  Ty jag hade beslutit mig för, att medan jag var bland eder icke veta om något annat än Jesus Kristus, och honom såsom korsfäst.
1Co 2:3  Och jag uppträdde hos eder i svaghet och med fruktan och mycken bävan.
1Co 2:4  Och mitt tal och min predikan framställdes icke med övertalande visdomsord, utan med en bevisning i ande och kraft;
1Co 2:5  ty eder tro skulle icke vara grundad på människors visdom, utan på Guds kraft.
1Co 2:6  Visdom tala vi dock bland dem som äro fullmogna, men en visdom som icke tillhör denna tidsålder eller denna tidsålders mäktige, vilkas makt bliver till intet.
1Co 2:7  Nej, vi tala Guds hemliga visdom, den fördolda, om vilken Gud, redan före tidsåldrarnas begynnelse, har bestämt att den skall bliva oss till härlighet,
1Co 2:8  och som ingen av denna tidsålders mäktige har känt; ty om de hade känt den, så hade de icke korsfäst härlighetens Herre.
1Co 2:9  Vi tala - såsom det heter i skriften - "vad intet öga har sett och intet öra har hört, och vad ingen människas hjärta har kunnat tänka, vad Gud har berett åt dem som älska honom".
1Co 2:10  Ty för oss har Gud uppenbarat det genom sin Ande. Anden utrannsakar ju allt, ja ock Guds djuphet.
1Co 2:11  Ty vilken människa vet vad som är i en människa, utom den människans egen ande? Likaså känner ingen vad som är i Gud, utom Guds Ande.
1Co 2:12  Men vi hava icke fått världens ande, utan den Ande som är av Gud, för att vi skola veta vad som har blivit oss skänkt av Gud.
1Co 2:13  Om detta tala vi ock, icke med sådana ord som mänsklig visdom lär oss, utan med sådana ord som Anden lär oss; vi hava ju att tyda andliga ting för andliga människor.
1Co 2:14  Men en "själisk" människa tager icke emot vad som hör Guds Ande till. Det är henne en dårskap, och hon kan icke förstå det, ty det måste utgrundas på ett andligt sätt.
1Co 2:15  Den andliga människan åter kan utgrunda allt, men själv kan hon icke utgrundas av någon.
1Co 2:16  Ty "vem har lärt känna Herrens sinne, så att han skulle kunna undervisa honom?" Men vi hava Kristi sinne.
1Co 3:1  Och jag kunde icke tala till eder, mina bröder, såsom till andliga människor, utan måste tala såsom till människor av köttslig natur, såsom till dem som ännu äro barn i Kristus.
1Co 3:2  Mjölk gav jag eder att dricka; fast föda gav jag eder icke, ty det fördrogen I då ännu icke. Ja, icke ens nu fördragen I det,
1Co 3:3  eftersom I ännu haven ett köttsligt sinne. Ty om avund och kiv finnes bland eder, haven I icke då ett köttsligt sinne, och vandren I icke då på vanligt människosätt?
1Co 3:4  När den ene säger: "Jag håller mig till Paulus" och den andre: "Jag håller mig till Apollos", ären I icke då lika hopen av människor?
1Co 3:5  Vad är då Apollos? Vad är Paulus? Allenast tjänare, genom vilka I haven kommit till tro; och de äro det i mån av vad Herren har beskärt åt var och en av dem.
1Co 3:6  Jag planterade, Apollos vattnade, men Gud gav växten.
1Co 3:7  Alltså kommer det icke an på den som planterar, ej heller på den som vattnar, utan på Gud, som giver växten.
1Co 3:8  Den som planterar och den som vattnar - den ene är såsom den andre, dock så, att var och en skall få sin särskilda lön efter sitt särskilda arbete.
1Co 3:9  Ty vi äro Guds medarbetare; I ären ett Guds åkerfält, en Guds byggnad.
1Co 3:10  Efter den Guds nåd som blev mig given lade jag grunden såsom en förfaren byggmästare, och en annan bygger nu vidare därpå. Men var och en må se till, huru han bygger därpå.
1Co 3:11  Ty en annan grund kan ingen lägga, än den som är lagd, nämligen Jesus Kristus;
1Co 3:12  men om någon bygger på den grunden med guld, silver och dyrbara stenar eller med trä, hö och strå,
1Co 3:13  så skall det en gång visa sig huru det är med vars och ens verk. "Den dagen" skall göra det kunnigt; ty den skall uppenbaras i eld, och hurudant vars och ens verk är, det skall elden pröva.
1Co 3:14  Om det byggnadsverk, som någon har uppfört på den grunden, bliver beståndande, så skall han undfå lön;
1Co 3:15  men om hans verk brännes upp, så skall han gå miste om lönen. Själv skall han dock bliva frälst, men såsom igenom eld.
1Co 3:16  Veten I icke att I ären ett Guds tempel och att Guds Ande bor i eder?
1Co 3:17  Om nu någon fördärvar Guds tempel, så skall Gud fördärva honom; ty Guds tempel är heligt, och det templet ären I.
1Co 3:18  Ingen bedrage sig själv. Om någon bland eder menar sig vara vis genom denna tidsålders visdom, så blive han en dåre, för att han skall kunna bliva vis.
1Co 3:19  Ty denna världens visdom är dårskap inför Gud. Det är ju skrivet: "Han fångar de visa i deras klokskap";
1Co 3:20  så ock: "Herren känner de visas tankar, han vet att de äro fåfängliga."
1Co 3:21  Så berömme sig då ingen av människor. Allt hör ju eder till;
1Co 3:22  det må vara Paulus eller Apollos eller Cefas eller hela världen, det må vara liv eller död, vad som nu är, eller vad som skall komma, alltsammans hör eder till.
1Co 3:23  Men I hören Kristus till, och Kristus hör Gud till.
1Co 4:1  Såsom Kristi tjänare och såsom förvaltare av Guds hemligheter, så må man anse oss.
1Co 4:2  Vad man nu därutöver söker hos förvaltare är att en sådan må befinnas vara trogen.
1Co 4:3  För mig betyder det likväl föga att I - eller överhuvud någon mänsklig domstol - sätten eder till doms över mig. Ja, jag vill icke ens sätta mig till doms över mig själv.
1Co 4:4  Ty väl vet jag intet med mig, men därigenom är jag icke rättfärdigad; det är Herren som sitter till doms över mig.
1Co 4:5  Dömen därför icke förrän tid är, icke förrän Herren kommer, han som skall draga fram i ljuset vad som är fördolt i mörker och uppenbara alla hjärtans rådslag. Och då skall var och en undfå av Gud den berömmelse som honom tillkommer.
1Co 4:6  Detta, mina bröder, har jag nu för eder skull så framställt, som gällde det mig och Apollos; ty jag vill att I skolen i fråga om oss lära eder detta: "Icke utöver vad skrivet är." Jag vill icke att I skolen stå emot varandra, uppblåsta var och en över sin lärare.
1Co 4:7  Vem säger då att du har något företräde? Och vad äger du, som du icke har fått dig givet? Men har du nu fått dig givet vad du har, huru kan du då berömma dig, såsom om du icke hade fått det dig givet?
1Co 4:8  I ären kantänka redan mätta, I haven redan blivit rika; oss förutan haven I blivit sannskyldiga konungar! Ja, jag skulle önska att I verkligen haden blivit konungar, så att vi kunde få bliva edra medkonungar.
1Co 4:9  Mig tyckes nämligen att Gud har ställt oss apostlar här såsom de ringaste bland alla, såsom livdömda män; ett skådespel hava vi ju blivit för världen, för både änglar och människor.
1Co 4:10  Vi äro dårar för Kristi skull, men I ären kloka i Kristus; vi äro svaga, men I ären starka; I ären ärade, men vi äro föraktade.
1Co 4:11  Ännu i denna stund lida vi både hunger och törst, vi måste gå nakna, vi få uppbära hugg och slag, vi hava intet stadigt hemvist,
1Co 4:12  vi måste möda oss och arbeta med våra händer. Vi bliva smädade och välsigna likväl; vi lida förföljelse och härda dock ut;
1Co 4:13  man talar illa om oss, men vi tala goda ord. Vi hava blivit såsom världens avskum, såsom var mans avskrap, och vi äro så ännu alltjämt.
1Co 4:14  Detta skriver jag, icke för att komma eder att blygas, utan såsom en förmaning till mina älskade barn.
1Co 4:15  Ty om I än haden tio tusen uppfostrare i Kristus, så haven I dock icke många fäder; det var ju jag som i Kristus Jesus genom evangelium födde eder till liv.
1Co 4:16  Därför förmanar jag eder: Bliven mina efterföljare.
1Co 4:17  Just för denna saks skull sänder jag nu till eder Timoteus, min älskade och trogne son i Herren; han skall påminna eder om huru jag går till väga i Kristus, i enlighet med den lära jag förkunnar allestädes, i alla församlingar.
1Co 4:18  Nu är det väl så, att somliga hava blivit uppblåsta, under förmenande att jag icke skulle komma till eder.
1Co 4:19  Men om Herren så vill, skall jag snart komma till eder; och då skall jag lära känna, icke dessa uppblåsta människors ord, utan deras kraft.
1Co 4:20  Ty Guds rike består icke i ord, utan i kraft.
1Co 4:21  Vilketdera viljen I nu: skall jag komma till eder med ris eller i kärlek och saktmods ande?
1Co 5:1  Det förljudes såväl att överhuvud otukt bedrives bland eder, som ock att sådan otukt förekommer, som man icke ens finner bland hedningarna, nämligen att en son har sin faders hustru.
1Co 5:2  Och ändå ären I uppblåsta och haven icke fastmer blivit uppfyllda av sådan sorg, att I haven drivit ut ur eder krets den som har gjort detta.
1Co 5:3  Jag, som väl till kroppen är frånvarande, men till anden närvarande, har för min del redan, såsom vore jag närvarande, fällt domen över den som har förövat en sådan ogärning:
1Co 5:4  i Herren Jesu namn skola vi komma tillsammans, I och min ande, med vår Herre Jesu kraft,
1Co 5:5  och överlämna den mannen åt Satan till köttets fördärv, för att anden skall bliva frälst på Herren Jesu dag.
1Co 5:6  Det är icke väl beställt med eder berömmelse. Veten I icke att litet surdeg syrar hela degen?
1Co 5:7  Rensen bort den gamla surdegen, så att I bliven en ny deg. I ären ju osyrade; ty vi hava ock ett påskalamm, som är slaktat, nämligen Kristus.
1Co 5:8  Låtom oss därför hålla högtid, icke med gammal surdeg, icke med elakhetens och ondskans surdeg, utan med renhetens och sanningens osyrade bröd.
1Co 5:9  Jag skrev till eder i mitt brev att I icke skullen hava något umgänge med otuktiga människor -
1Co 5:10  detta icke sagt i allmänhet, om alla denna världens otuktiga människor eller om giriga och roffare eller om avgudadyrkare; annars måsten I ju rymma ur världen.
1Co 5:11  Nej, då jag skrev så till eder, menade jag, att om någon som kallades broder vore en otuktig människa eller en girig eller en avgudadyrkare eller en smädare eller en drinkare eller en roffare, så skullen I icke hava något umgänge med en sådan eller äta tillsammans med honom.
1Co 5:12  Ty icke tillkommer det väl mig att döma dem som äro utanför? Dem som äro innanför haven I ju att döma;
1Co 5:13  dem som äro utanför skall Gud döma. "I skolen driva ut ifrån eder den som är ond."
1Co 6:1  Huru kan någon av eder taga sig för, att när han har sak med en annan, gå till rätta icke inför de heliga, utan inför de orättfärdiga?
1Co 6:2  Veten I då icke att de heliga skola döma världen? Men om nu I skolen sitta till doms över världen, ären I då icke goda nog att döma i helt ringa mål?
1Co 6:3  I veten ju att vi skola döma änglar; huru mycket mer böra vi icke då kunna döma i timliga ting?
1Co 6:4  Och likväl, när I nu haven före något mål som gäller sådana ting, sätten I till domare just dem som äro ringa aktade i församlingen!
1Co 6:5  Eder till blygd säger jag detta. Är det då så omöjligt att bland eder finna någon vis man, som kan bliva skiljedomare mellan sina bröder?
1Co 6:6  Måste i stället den ene brodern gå till rätta med den andre, och det inför de otrogna?
1Co 6:7  Överhuvud är redan det en brist hos eder, att I gån till rätta med varandra. Varför liden I icke hellre orätt? Varför låten I icke hellre andra göra eder skada?
1Co 6:8  I stället gören I nu själva orätt och skada, och detta mot bröder.
1Co 6:9  Veten I då icke att de orättfärdiga icke skola få Guds rike till arvedel? Faren icke vilse. Varken otuktiga människor eller avgudadyrkare eller äktenskapsbrytare, varken de som låta bruka sig till synd mot naturen eller de som själva öva sådan synd,
1Co 6:10  varken tjuvar eller giriga eller drinkare eller smädare eller roffare skola få Guds rike till arvedel.
1Co 6:11  Sådana voro ock somliga bland eder, men I haven låtit två eder rena, I haven blivit helgade, I haven blivit rättfärdiggjorda i Herrens, Jesu Kristi, namn och i vår Guds Ande.
1Co 6:12  "Allt är mig lovligt"; ja, men icke allt är nyttigt. "Allt är mig lovligt"; ja, men jag bör icke låta något få makt över mig.
1Co 6:13  Maten är för buken och buken för maten, men bådadera skall Gud göra till intet. Däremot är kroppen icke för otukt, utan för Herren, och Herren för kroppen;
1Co 6:14  och Gud, som har uppväckt Herren, skall ock genom sin kraft uppväcka oss.
1Co 6:15  Veten I icke att edra kroppar äro Kristi lemmar? Skall jag nu taga Kristi lemmar och göra dem till en skökas lemmar? Bort det!
1Co 6:16  Veten I då icke att den som håller sig till en sköka, han bliver en kropp med henne? Det heter ju: "De tu skola varda ett kött."
1Co 6:17  Men den som håller sig till Herren, han är en ande med honom.
1Co 6:18  Flyn otukten. All annan synd som en människa kan begå är utom kroppen; men den som bedriver otukt, han syndar på sin egen kropp.
1Co 6:19  Veten I då icke att eder kropp är ett tempel åt den helige Ande, som bor i eder, och som I haven undfått av Gud, och att I icke ären edra egna?
1Co 6:20  I ären ju köpta, och betalning är given. Så förhärligen då Gud i eder kropp.
1Co 7:1  Vad nu angår det I haven skrivit om, så svarar jag detta: En man gör visserligen väl i att icke komma vid någon kvinna;
1Co 7:2  men för att undgå otuktssynder må var man hava sin egen hustru, och var kvinna sin egen man.
1Co 7:3  Mannen give sin hustru vad han är henne pliktig, sammalunda ock hustrun sin man.
1Co 7:4  Hustrun råder icke själv över sin kropp, utan mannen; sammalunda råder ej heller mannen över sin kropp, utan hustrun.
1Co 7:5  Dragen eder icke undan från varandra, om icke möjligen, med bådas samtycke, till en tid, för att I skolen hava ledighet till bönen. Kommen sedan åter tillsammans, så att Satan icke frestar eder, då I nu icke kunnen leva återhållsamt.
1Co 7:6  Detta säger jag likväl såsom en tillstädjelse, icke såsom en befallning.
1Co 7:7  Jag skulle dock vilja att alla människor vore såsom jag. Men var och en har fått sin särskilda nådegåva från Gud, den ene så, den andre så.
1Co 7:8  Till de ogifta åter och till änkorna säger jag att de göra väl, om de förbliva i samma ställning som jag.
1Co 7:9  Men kunna de icke leva återhållsamt, så må de gifta sig; ty det är bättre att gifta sig än att vara upptänd av begär.
1Co 7:10  Men dem som äro gifta bjuder jag - dock icke jag, utan Herren; En hustru må icke skilja sig från sin man
1Co 7:11  (om hon likväl skulle skilja sig, så förblive hon ogift eller förlike sig åter med mannen), ej heller må en man förskjuta sin hustru.
1Co 7:12  Till de andra åter säger jag själv, icke Herren: Om någon som hör till bröderna har en hustru som icke är troende, och denna är villig att leva tillsammans med honom, så må han icke förskjuta henne.
1Co 7:13  Likaså, om en hustru har en man som icke är troende, och denne är villig att leva tillsammans med henne, så må hon icke förskjuta mannen.
1Co 7:14  Ty den icke troende mannen är helgad i och genom sin hustru, och den icke troende hustrun är helgad i och genom sin man, då han är en broder; annars vore ju edra barn orena, men nu äro de heliga. -
1Co 7:15  Om däremot den icke troende vill skiljas, så må han få skiljas. En broder eller syster är i sådana fall intet tvång underkastad, och Gud har kallat oss till att leva i frid.
1Co 7:16  Ty huru kan du veta, du hustru, om du skall frälsa din man? Eller du man, huru vet du om du skall frälsa din hustru?
1Co 7:17  Må allenast var och en vandra den väg fram, som Herren har bestämt åt honom, var och en i den ställning vari Gud har kallat honom. Den ordningen stadgar jag för alla församlingar.
1Co 7:18  Har någon blivit kallad såsom omskuren, så göre han sig icke åter lik de oomskurna; har någon blivit kallad såsom oomskuren, så låte han icke omskära sig.
1Co 7:19  Det kommer icke an på om någon är omskuren eller oomskuren; allt beror på huruvida han håller Guds bud.
1Co 7:20  Var och en förblive i den kallelse vari han var, när han blev kallad.
1Co 7:21  Har du blivit kallad såsom träl, så låt detta icke gå dig till sinnes; dock, om du kan bliva fri, så begagna dig hellre därav.
1Co 7:22  Ty den träl som har blivit kallad till att vara i Herren, han är en Herrens frigivne; sammalunda är ock den frie, som har blivit kallad, en Kristi livegne.
1Co 7:23  I ären köpta, och betalningen är given; bliven icke människors trälar.
1Co 7:24  Ja, mina bröder, var och en förblive inför Gud i den ställning vari han har blivit kallad.
1Co 7:25  Vad vidare angår dem som äro jungfrur, så har jag icke att åberopa någon befallning av Herren, utan giver allenast ett råd, såsom en som genom Herrens barmhärtighet har blivit förtroende värd.
1Co 7:26  Jag menar alltså, med tanke på den nöd som står för dörren, att den människa gör väl, som förbliver såsom hon är.
1Co 7:27  Är du bunden vid hustru, så sök icke att bliva lös. Är du utan hustru, så sök icke att få hustru.
1Co 7:28  Om du likväl skulle gifta dig, så syndar du icke därmed; ej heller syndar en jungfru, om hon gifter sig. Dock komma de som så göra att draga över sig lekamliga vedermödor; och jag skulle gärna vilja skona eder.
1Co 7:29  Men det säger jag, mina bröder: Tiden är kort; därför må härefter de som hava hustrur vara såsom hade de inga,
1Co 7:30  och de som gråta såsom gräte de icke, och de som glädja sig såsom gladde de sig icke, och de som köpa något såsom finge de icke behålla det,
1Co 7:31  och de som bruka denna världen såsom gjorde de icke något bruk av den. Ty den nuvarande världsordningen går mot sitt slut;
1Co 7:32  och jag skulle gärna vilja att I voren fria ifrån omsorger. Den man som icke är gift ägnar nämligen sin omsorg åt vad som hör Herren till, huru han skall behaga Herren;
1Co 7:33  men den gifte mannen ägnar sin omsorg åt vad som hör världen till, huru han skall behaga sin hustru,
1Co 7:34  och så är hans hjärta delat. Likaså ägnar den kvinna, som icke längre är gift eller som är jungfru, sin omsorg åt vad som hör Herren till, att hon må vara helig till både kropp och ande; men den gifta kvinnan ägnar sin omsorg åt vad som hör världen till, huru hon skall behaga sin man.
1Co 7:35  Detta säger jag till eder egen nytta, och icke för att lägga något band på eder, utan för att I skolen föra en hövisk vandel och stadigt förbliva vid Herren.
1Co 7:36  Men om någon menar sig handla otillbörligt mot sin ogifta dotter därmed att hon får bliva överårig, då må han göra såsom han vill, om det nu måste så vara; han begår därmed ingen synd. Må hon få gifta sig.
1Co 7:37  Om däremot någon är fast i sitt sinne och icke bindes av något nödtvång, utan kan följa sin egen vilja, och så i sitt sinne är besluten att låta sin ogifta dotter förbliva såsom hon är, då gör denne väl.
1Co 7:38  Alltså: den som gifter bort sin dotter, han gör väl; och den som icke gifter bort henne, han gör ännu bättre.
1Co 7:39  En hustru är bunden så länge hennes man lever; men när hennes man är avsomnad, står det henne fritt att gifta sig med vem hon vill, blott det sker i Herren.
1Co 7:40  Men lyckligare är hon, om hon förbliver såsom hon är. Så är min mening, och jag tror att också jag har Guds Ande.
1Co 8:1  Vad åter angår kött från avgudaoffer, så känna vi nog det talet: "Alla hava vi 'kunskap'." "Kunskapen" uppblåser, men kärleken uppbygger.
1Co 8:2  Om någon menar sig hava fått någon "kunskap", så har han ännu icke fått kunskap på sådant sätt som han borde hava.
1Co 8:3  Men den som älskar Gud, han är känd av honom.
1Co 8:4  Vad alltså angår ätandet av kött från avgudaoffer, så säger jag detta: Vi veta visserligen att ingen avgud finnes till i världen, och att det icke finnes mer än en enda Gud.
1Co 8:5  Ty om ock några så kallade gudar skulle finnas, vare sig i himmelen eller på jorden - och det finnes ju många "gudar" och många "herrar" -
1Co 8:6  så finnes dock för oss allenast en enda Gud: Fadern, av vilken allt är, och till vilken vi själva äro, och en enda Herre: Jesus Kristus, genom vilken allt är, och genom vilken vi själva äro.
1Co 8:7  Dock, icke alla hava denna kunskap, utan somliga, som äro vana att ännu alltjämt tänka på avguden, äta köttet såsom avgudaofferskött. Och eftersom deras samvete är svagt, bliver det härigenom befläckat.
1Co 8:8  Men maten skall icke avgöra vår ställning till Gud. Avhålla vi oss från att äta, så bliva vi icke därigenom sämre; äta vi, så bliva vi icke därigenom bättre.
1Co 8:9  Sen likväl till, att denna eder frihet icke till äventyrs bliver en stötesten för de svaga.
1Co 8:10  Ty om någon får se dig, som har undfått "kunskap", ligga till bords i ett avgudahus, skall då icke hans samvete, om han är svag, därav "bliva uppbyggt" på det sätt att han äter köttet från avgudaoffer?
1Co 8:11  Genom din "kunskap" går ju då den svage förlorad - han, din broder, som Kristus har lidit döden för.
1Co 8:12  Om I på sådant sätt synden mot bröderna och såren deras svaga samveten, då synden I mot Kristus själv.
1Co 8:13  Därför, om maten kan bliva min broder till fall, så vill jag sannerligen hellre för alltid avstå från att äta kött, på det att jag icke må bliva min broder till fall.
1Co 9:1  Är jag icke fri? Är jag icke en apostel? Har jag icke sett Jesus, vår Herre? Ären icke I mitt verk i Herren?
1Co 9:2  Om jag icke för andra är en apostel, så är jag det åtminstone för eder, ty I själva ären i Herren inseglet på mitt apostlaämbete.
1Co 9:3  Detta är mitt försvar mot dem som sätta sig till doms över mig.
1Co 9:4  Skulle vi kanhända icke hava rätt att få mat och dryck?
1Co 9:5  Skulle vi icke hava rätt att få såsom hustru föra med oss på våra resor någon som är en syster, vi likaväl som de andra apostlarna och Herrens bröder och särskilt Cefas?
1Co 9:6  Eller äro jag och Barnabas de enda som icke hava rätt att vara fritagna ifrån kroppsarbete?
1Co 9:7  Vem tjänar någonsin i krig på egen sold? Vem planterar en vingård och äter icke dess frukt? Eller vem vaktar en hjord och förtär icke mjölk från hjorden?
1Co 9:8  Icke talar jag väl detta därför att människor pläga så tala? Säger icke själva lagen detsamma?
1Co 9:9  I Moses' lag är ju skrivet: "Du skall icke binda munnen till på oxen som tröskar." Månne det är om oxarna som Gud har sådan omsorg?
1Co 9:10  Eller säger han det icke i alla händelser med tanke på oss? Jo, för vår skull blev det skrivet, att den som plöjer bör plöja med en förhoppning, och att den som tröskar bör göra det i förhoppning om att få sin del.
1Co 9:11  Om vi hava sått åt eder ett utsäde av andligt gott, är det då för mycket, om vi få inbärga från eder en skörd av lekamligt gott?
1Co 9:12  Om andra hava en viss rättighet över eder, skulle då icke vi än mer hava det? Och likväl hava vi icke gjort bruk av den rättigheten, utan vi fördraga allt, för att icke lägga något hinder i vägen för Kristi evangelium.
1Co 9:13  I veten ju att de som förrätta tjänsten i helgedomen få sin föda ifrån helgedomen, och att de som äro anställda vid altaret få sin del, när altaret får sin.
1Co 9:14  Så har ock Herren förordnat att de som förkunna evangelium skola hava sitt uppehälle av evangelium.
1Co 9:15  Men jag för min del har icke gjort bruk av någon sådan förmån. Detta skriver jag nu icke, för att jag själv skall få någon sådan; långt hellre ville jag dö. Nej, ingen skall göra min berömmelse om intet.
1Co 9:16  Ty om jag förkunnar evangelium, så är detta ingen berömmelse för mig. Jag måste ju så göra; och ve mig, om jag icke förkunnade evangelium!
1Co 9:17  Gör jag det av egen drift, så har jag rätt till lön; men då jag nu icke gör det av egen drift, så är den syssla som jag är betrodd med allenast en livegen förvaltares. -
1Co 9:18  Vilken är alltså min lön? Jo, just den, att när jag förkunnar evangelium, så gör jag detta utan kostnad för någon, i det att jag avstår från att göra bruk av den rättighet jag har såsom förkunnare av evangelium.
1Co 9:19  Ty fastän jag är fri och oberoende av alla, har jag dock gjort mig till allas tjänare, för att jag skall vinna dess flera.
1Co 9:20  För judarna har jag blivit såsom en jude, för att kunna vinna judar; för dom som stå under lagen har jag, som själv icke står under lagen, blivit såsom stode jag under lagen, för att kunna vinna dem som stå under lagen.
1Co 9:21  För dem som äro utan lag har jag, som icke är utan Guds lag, men är i Kristi lag, blivit såsom vore jag utan lag, för att jag skall vinna dem som äro utan lag.
1Co 9:22  För de svaga har jag blivit svag, för att kunna vinna de svaga; för alla har jag blivit allt, för att jag i alla händelser skall frälsa några.
1Co 9:23  Men allt gör jag för evangelii skull, för att också jag skall bliva delaktig av dess goda.
1Co 9:24  I veten ju, att fastän de som löpa på tävlingsbanan allasammans löpa, så vinner allenast en segerlönen. Löpen såsom denne, för att I mån vinna lönen.
1Co 9:25  Men alla som vilja deltaga i en sådan tävlan pålägga sig återhållsamhet i alla stycken: dessa för att vinna en förgänglig segerkrans, men vi för att vinna en oförgänglig.
1Co 9:26  Jag för min del löper alltså icke såsom gällde det ett ovisst mål; jag kämpar icke likasom en man som hugger i vädret.
1Co 9:27  Fastmer tuktar jag min kropp och kuvar den, för att jag icke, när jag predikar för andra, själv skall komma till korta vid provet.
1Co 10:1  Ty jag vill säga eder detta, mina bröder: Våra fäder voro alla under molnskyn och gingo alla genom havet;
1Co 10:2  alla blevo de i molnskyn och i havet döpta till Moses;
1Co 10:3  alla åto de samma andliga mat,
1Co 10:4  och alla drucko de samma andliga dryck - de drucko nämligen ur en andlig klippa, som åtföljde dem, och den klippan var Kristus.
1Co 10:5  Men de flesta av dem hade Gud icke behag till; de blevo ju nedgjorda i öknen.
1Co 10:6  Detta skedde oss till en varnagel, för att vi icke skulle hava begärelse till det onda, såsom de hade begärelse därtill.
1Co 10:7  Ej heller skolen I bliva avgudadyrkare, såsom somliga av dem blevo; så är ju skrivet: "Folket satte sig ned till att äta och dricka, och därpå stodo de upp till all leka."
1Co 10:8  Låtom oss icke heller bedriva otukt, såsom somliga av dem gjorde, varför ock tjugutre tusen föllo på en enda dag.
1Co 10:9  Låtom oss icke heller fresta Kristus, såsom somliga av dem gjorde, varför de ock blevo dödade av ormarna.
1Co 10:10  Knorren icke heller, såsom somliga av dem gjorde, varför de ock blevo dödade av "Fördärvaren".
1Co 10:11  Men detta vederfors dem för att tjäna till en varnagel, och det blev upptecknat till lärdom för oss, som hava tidernas ände inpå oss.
1Co 10:12  Därför, den som menar sig stå, han må se till, att han icke faller.
1Co 10:13  Inga andra frestelser hava mött eder än sådana som vanligen möta människor. Och Gud är trofast; han skall icke tillstädja att I bliven frestade över eder förmåga, utan när han låter frestelsen komma, skall han ock bereda en utväg därur, så att I kunnen härda ut i den.
1Co 10:14  Alltså, mina älskade, undflyn avgudadyrkan.
1Co 10:15  Jag säger detta till eder såsom till förståndiga människor; själva mån I döma om det som jag säger.
1Co 10:16  Välsignelsens kalk, över vilken vi uttala välsignelsen, är icke den en delaktighet av Kristi blod? Brödet, som vi bryta, är icke det en delaktighet av Kristi kropp?
1Co 10:17  Eftersom det är ett enda bröd, så äro vi, fastän många, en enda kropp, ty alla få vi vår del av detta ena bröd.
1Co 10:18  Sen på det lekamliga Israel: äro icke de som äta av offren delaktiga i altaret?
1Co 10:19  Vad vill jag då säga härmed? Månne att avgudaofferskött är någonting, eller att en avgud är någonting?
1Co 10:20  Nej, det vill jag säga, att vad hedningarna offra, det offra de åt onda andar och icke åt Gud; och jag vill icke att I skolen hava någon gemenskap med de onda andarna.
1Co 10:21  I kunnen icke dricka Herrens kalk och tillika onda andars kalk; I kunnen icke hava del i Herrens bord och tillika i onda andars bord.
1Co 10:22  Eller vilja vi reta Herren? Äro då vi starkare än han?
1Co 10:23  "Allt är lovligt"; ja, men icke allt är nyttigt. "Allt är lovligt"; ja, men icke allt uppbygger.
1Co 10:24  Ingen söke sitt eget bästa, utan envar den andres.
1Co 10:25  Allt som säljes i köttboden mån I äta; I behöven icke för samvetets skull göra någon undersökning därom.
1Co 10:26  Ty "jorden är Herrens, och allt vad därpå är".
1Co 10:27  Om någon av dem som icke äro troende bjuder eder till sig och I viljen gå till honom, så mån I äta av allt som sättes fram åt eder; I behöven icke för samvetets skull göra någon undersökning därom.
1Co 10:28  Men om någon då säger till eder: "Detta är offerkött", så skolen I avhålla eder från att äta, för den mans skull, som gav saken till känna, och för samvetets skull -
1Co 10:29  jag menar icke ditt eget samvete, utan den andres; ty varför skulle jag låta min frihet dömas av en annans samvete?
1Co 10:30  Om jag äter därav med tacksägelse, varför skulle jag då bliva smädad för det som jag tackar Gud för?
1Co 10:31  Alltså, vare sig I äten eller dricken, eller vadhelst annat I gören, så gören allt till Guds ära.
1Co 10:32  Bliven icke för någon till en stötesten, varken för judar eller för greker eller för Guds församling;
1Co 10:33  varen såsom jag, som i alla stycken fogar mig efter alla och icke söker min egen nytta, utan de mångas, för att de skola bliva frälsta.
1Co 11:1  Varen I mina efterföljare, såsom jag är Kristi.
1Co 11:2  Jag prisar eder för det att I i alla stycken haven mig i minne och hållen fast vid mina lärdomar, såsom de äro eder givna av mig.
1Co 11:3  Men jag vill att I skolen inse detta, att Kristus är envar mans huvud, och att mannen är kvinnans huvud, och att Gud är Kristi huvud.
1Co 11:4  Var och en man som har sitt huvud betäckt, när han beder eller profeterar, han vanärar sitt huvud.
1Co 11:5  Men var kvinna som beder eller profeterar med ohöljt huvud, hon vanärar sitt huvud, ty det är då alldeles som om hon hade sitt hår avrakat.
1Co 11:6  Om en kvinna icke vill hölja sig, så kan hon lika väl låta skära av sitt hår; men eftersom det är en skam för en kvinna att låta skära av sitt hår eller att låta raka av det, så må hon hölja sig.
1Co 11:7  En man är icke pliktig att hölja sitt huvud, eftersom han är Guds avbild och återspeglar hans härlighet, då kvinnan däremot återspeglar mannens härlighet.
1Co 11:8  Ty mannen är icke av kvinnan, utan kvinnan av mannen.
1Co 11:9  Icke heller skapades mannen för kvinnans skull, utan kvinnan för mannens skull.
1Co 11:10  Därför bör kvinnan på sitt huvud hava en "makt", för änglarnas skull.
1Co 11:11  Dock är det i Herren så, att varken kvinnan är till utan mannen, eller mannen utan kvinnan.
1Co 11:12  Ty såsom kvinnan är av mannen, så är ock mannen genom kvinnan; men alltsammans är av Gud. -
1Co 11:13  Dömen själva: höves det en kvinnan att ohöljd bedja till Gud?
1Co 11:14  Lär icke själva naturen eder att det länder en man till vanheder, om han har långt hår,
1Co 11:15  men att det länder en kvinna till ära, om hon har långt hår? Håret är ju henne givet såsom slöja.
1Co 11:16  Om nu likväl någon vill vara genstridig, så mån han veta att vi för vår del icke hava en sådan sedvänja, ej heller andra Guds församlingar.
1Co 11:17  Detta bjuder jag eder nu. Men vad jag icke kan prisa är att I kommen tillsammans, icke till förbättring, utan till försämring.
1Co 11:18  Ty först och främst hör jag sägas att vid edra församlingsmöten söndringar yppa sig bland eder. Och till en del tror jag att så är.
1Co 11:19  Ty partier måste ju finnas bland eder, för att det skall bliva uppenbart vilka bland eder som hålla provet.
1Co 11:20  När I alltså kommen tillsammans med varandra, kan ingen Herrens måltid hållas;
1Co 11:21  ty vid måltiden tager var och en i förväg själv den mat han har medfört, och så får den ene hungra, medan den andre får för mycket.
1Co 11:22  Haven I då icke edra hem, där I kunnen äta ock dricka? Eller är det så, att I förakten Guds församling och viljen komma dem att blygas, som intet hava? Vad skall jag då säga till eder? Skall jag prisa eder? Nej, i detta stycke prisar jag eder icke.
1Co 11:23  Ty jag har från Herren undfått detta, som jag ock har meddelat eder: I den natt, då Herren Jesus blev förrådd, tog han ett bröd
1Co 11:24  och tackade Gud och bröt det och sade: "Tagen, äten. Detta är min lekamen, som varder utgiven för eder. Gören detta till min åminnelse."
1Co 11:25  Sammalunda tog han ock kalken, efter måltiden, och sade: "Denna kalk är det nya förbundet, i mitt blod. Så ofta I dricken den, så gören detta till min åminnelse."
1Co 11:26  Ty så ofta I äten detta bröd och dricken kalken, förkunnen I Herrens död, till dess att han kommer.
1Co 11:27  Den som nu på ett ovärdigt sätt äter detta bröd eller dricker Herrens kalk, han försyndar sig på Herrens lekamen och blod.
1Co 11:28  Pröve då människan sig själv, och äte så av brödet och dricke av kalken.
1Co 11:29  Ty den som äter och dricker, utan att göra åtskillnad mellan Herrens lekamen och annan spis, han äter och dricker en dom över sig.
1Co 11:30  Därför finnas ock bland eder många som äro svaga och sjuka, och ganska många äro avsomnade.
1Co 11:31  Om vi ginge till doms med oss själva, så bleve vi icke dömda.
1Co 11:32  Men då vi nu bliva dömda, så är detta en Herrens tuktan, som drabbar oss, för att vi icke skola bliva fördömda tillika med världen.
1Co 11:33  Alltså, mina bröder, när I kommen tillsammans för att hålla måltid, så vänten på varandra.
1Co 11:34  Om någon är hungrig, då må han äta hemma, så att eder sammankomst icke bliver eder till en dom. Om det övriga skall jag förordna, när jag kommer.
1Co 12:1  Vad nu angår dem som hava andliga gåvor, så vill jag säga eder, mina bröder, huru med dem förhåller sig.
1Co 12:2  I veten att I, medan I voren hedningar, läten eder blindvis föras bort till de stumma avgudarna.
1Co 12:3  Därför vill jag nu förklara för eder, att likasom ingen som talar i Guds Ande säger: "Förbannad vare Jesus", så kan ej heller någon säga: "Jesus är Herre" annat än i den helige Ande.
1Co 12:4  Nådegåvorna äro mångahanda, men Anden är en och densamme.
1Co 12:5  Tjänsterna äro mångahanda, men Herren är en och densamme.
1Co 12:6  Kraftverkningarna äro mångahanda, men Gud är en och densamme, han som verkar allt i alla.
1Co 12:7  Men de gåvor i vilka Anden uppenbarar sig givas åt var och en så, att de kunna bliva till nytta.
1Co 12:8  Så gives genom Anden åt den ene att tala visdomens ord, åt en annan att efter samme Ande tala kunskapens ord,
1Co 12:9  åt en annan gives tro i samme Ande, åt en annan givas helbrägdagörelsens gåvor i samme ene Ande,
1Co 12:10  åt en annan gives gåvan att utföra kraftgärningar, åt en annan att profetera, åt en annan att skilja mellan andar, åt en annan att tala tungomål på olika sätt, åt en annan att uttyda, när någon talar tungomål.
1Co 12:11  Men allt detta verkar densamme ene Anden, i det han, alltefter sin vilja, tilldelar åt var och en någon särskild gåva.
1Co 12:12  Ty likasom kroppen är en och likväl har många lemmar, och likasom kroppens alla lemmar, fastän de äro många, likväl utgöra en enda kropp, likaså är det med Kristus.
1Co 12:13  Ty i en och samme Ande äro vi alla döpta till att utgöra en och samma kropp, vare sig vi äro judar eller greker, vare sig vi äro trälar eller fria; och alla hava vi fått en och samme Ande utgjuten över oss.
1Co 12:14  Kroppen utgöres ju icke heller av en enda lem, utan av många.
1Co 12:15  Om foten ville säga: "Jag är icke hand, därför hör jag icke till kroppen", så skulle den icke dess mindre höra till kroppen.
1Co 12:16  Och om örat ville säga: "Jag är icke öga, därför hör jag icke till kroppen", så skulle det icke dess mindre höra till kroppen.
1Co 12:17  Om hela kroppen vore öga, var funnes då hörseln? Och om den hel och hållen vore öra, var funnes då lukten?
1Co 12:18  Men nu har Gud insatt lemmarna i kroppen, var och en av dem på det sätt som han har velat.
1Co 12:19  Om åter allasammans utgjorde en enda lem, var funnes då själva kroppen?
1Co 12:20  Men nu är det så, att lemmarna äro många, och att kroppen dock är en enda.
1Co 12:21  Ögat kan icke säga till handen: "Jag behöver dig icke", ej heller huvudet till fötterna: "Jag behöver eder icke."
1Co 12:22  Nej, just de kroppens lemmar som tyckas vara svagast äro som mest nödvändiga.
1Co 12:23  Och de delar av kroppen, som tyckas oss vara mindre hedersamma, dem bekläda vi med så mycket större heder; och dem som vi blygas för, dem skyla vi med så mycket större blygsamhet,
1Co 12:24  under det att de andra icke behöva något sådant. Men när Gud sammanfogade kroppen av olika delar och därvid lät den ringare delen få en så mycket större heder,
1Co 12:25  så skedde detta, för att söndring icke skulle uppstå i kroppen, utan alla lemmar endräktigt hava omsorg om varandra.
1Co 12:26  Om nu en lem lider, så lida alla de andra lemmarna med den; om åter en lem äras, så glädja sig alla de andra lemmarna med den.
1Co 12:27  Men nu ären I Kristi kropp och hans lemmar, var och en i sin mån.
1Co 12:28  Och Gud har i församlingen satt först och främst några till apostlar, för det andra några till profeter, för det tredje några till lärare, vidare några till att utföra kraftgärningar, ytterligare några till att hava helbrägdagörelsens gåvor, eller till att taga sig an de hjälplösa, eller till att vara styresmän, eller till att på olika sätt tala tungomål.
1Co 12:29  Icke äro väl alla apostlar? Icke äro väl alla profeter? Icke äro väl alla lärare? Icke utföra väl alla kraftgärningar?
1Co 12:30  Icke hava väl alla helbrägdagörelsens gåvor? Icke tala väl alla tungomål? Icke kunna väl alla uttyda?
1Co 12:31  Men varen ivriga att undfå de nådegåvor som äro de största. Och nu vill jag ytterligare visa eder en väg, en övermåttan härlig väg.
1Co 13:1  Om jag talade både människors och änglars tungomål, men icke hade kärlek, så vore jag allenast en ljudande malm eller en klingande cymbal.
1Co 13:2  Och om jag hade profetians gåva och visste alla hemligheter och ägde all kunskap, och om jag hade all tro, så att jag kunde förflytta berg, men icke hade kärlek, så vore jag intet.
1Co 13:3  Och om jag gåve bort allt vad jag ägde till bröd åt de fattiga, ja, om jag offrade min kropp till att brännas upp, men icke hade kärlek, så vore detta mig till intet gagn.
1Co 13:4  Kärleken är tålig och mild. Kärleken avundas icke, kärleken förhäver sig icke, den uppblåses icke.
1Co 13:5  Den skickar sig icke ohöviskt, den söker icke sitt, den förtörnas icke, den hyser icke agg för en oförrätts skull.
1Co 13:6  Den gläder sig icke över orättfärdigheten, men har sin glädje i sanningen.
1Co 13:7  Den fördrager allting, den tror allting, den hoppas allting, den uthärdar allting.
1Co 13:8  Kärleken förgår aldrig. Men profetians gåva, den skall försvinna, och tungomålstalandet, det skall taga slut, och kunskapen, den skall försvinna.
1Co 13:9  Ty vår kunskap är ett styckverk, och vårt profeterande är ett styckverk;
1Co 13:10  men när det kommer, som är fullkomligt, då skall det försvinna, som är ett styckverk.
1Co 13:11  När jag var barn, talade jag såsom ett barn, mitt sinne var såsom ett barns, jag hade barnsliga tankar; men sedan jag blev man, har jag lagt bort vad barnsligt var.
1Co 13:12  Nu se vi ju på ett dunkelt sätt, såsom i en spegel, men då skola vi se ansikte mot ansikte. Nu är min kunskap ett styckverk, men då skall jag känna till fullo, såsom jag själv har blivit till fullo känd.
1Co 13:13  Så bliva de då beståndande, tron, hoppet, kärleken, dessa tre; men störst bland dem är kärleken.
1Co 14:1  Faren efter kärleken, men varen ock ivriga att undfå de andliga gåvorna, framför allt profetians gåva.
1Co 14:2  Ty den som talar tungomål, han talar icke för människor, utan för Gud; ingen förstår honom ju, han talar i andehänryckning hemlighetsfulla ord.
1Co 14:3  Men den som profeterar, han talar för människor, dem till uppbyggelse och förmaning och tröst.
1Co 14:4  Den som talar tungomål uppbygger allenast sig själv, men den som profeterar, han uppbygger en hel församling.
1Co 14:5  Jag skulle väl vilja att I alla taladen tungomål, men ännu hellre ville jag att I profeteraden. Den som profeterar är förmer än den som talar tungomål, om nämligen den senare icke därjämte uttyder sitt tal, så att församlingen får någon uppbyggelse.
1Co 14:6  Ja, mina bröder, om jag komme till eder och talade tungomål, vad gagn gjorde jag eder därmed, såframt jag icke därjämte genom mitt tal meddelade eder antingen någon uppenbarelse eller någon kunskap eller någon profetia eller någon undervisning?
1Co 14:7  Gäller det icke jämväl om livlösa ting som giva ljud ifrån sig, det må nu vara en flöjt eller en harpa, att vad som spelas på dem icke kan uppfattas, om de icke giva ifrån sig toner som kunna skiljas från varandra?
1Co 14:8  Likaså, om den signal som basunen giver är otydlig, vem gör sig då redo till strid?
1Co 14:9  Detsamma gäller nu för eder; om I icke med edra tungor frambringen begripliga ord, huru skall man då kunna förstå vad I talen? Då bliver det ju ett tal i vädret.
1Co 14:10  Det finnes här i världen olika språk, vem vet huru många, och bland dem finnes intet vars ljud äro utan mening.
1Co 14:11  Men om jag nu icke förstår språket, så bliver jag en främling för den som talar, och den som talar bliver en främling för mig.
1Co 14:12  Detta gäller ock för eder; när I ären ivriga att undfå andliga gåvor, så må eder strävan efter att dessa hos eder skola överflöda hava församlingens uppbyggelse till mål.
1Co 14:13  Därför må den som talar tungomål bedja om att han ock må kunna uttyda.
1Co 14:14  Ty om jag talar tungomål, när jag beder, så beder visserligen min ande, men mitt förstånd kommer ingen frukt åstad.
1Co 14:15  Vad följer då härav? Jo, jag skall väl bedja med anden, men jag skall ock bedja med förståndet; jag skall väl lovsjunga med anden, men jag skall ock lovsjunga med förståndet.
1Co 14:16  Eljest, om du lovar Gud med anden, huru skola de som sitta på de olärdas plats då kunna säga sitt "amen" till din tacksägelse? De förstå ju icke vad du säger.
1Co 14:17  Om än din tacksägelse är god, så bliva de andra dock icke uppbyggda därav. -
1Co 14:18  Gud vare tack, jag talar tungomål mer än I alla;
1Co 14:19  och dock vill jag hellre i församlingen tala fem ord med mitt förstånd, till undervisning jämväl för andra, än tio tusen ord i tungomål.
1Co 14:20  Mina bröder, varen icke barn till förståndet; nej varen barn i ondskan, men varen fullmogna till förståndet.
1Co 14:21  Det är skrivet i lagen: "Genom människor med främmande tungomål och genom främlingars läppar skall jag tala till detta folk, men icke ens så skola de höra på mig, säger Herren."
1Co 14:22  Alltså äro "tungomålen" ett tecken, ej för dem som tro, utan för dem som icke tro; profetian däremot är ett tecken, ej för dem som icke tro, utan för dem som tro.
1Co 14:23  Om nu hela församlingen komme tillhopa till gemensamt möte, och alla där talade tungomål, och så några som vore olärda komme ditin, eller några som icke trodde, skulle då icke dessa säga att I voren ifrån edra sinnen?
1Co 14:24  Om åter alla profeterade, och så någon som icke trodde, eller som vore olärd komme ditin, då skulle denne känna sig avslöjad av alla och av alla utrannsakad.
1Co 14:25  Vad som vore fördolt i hans hjärta bleve då uppenbart, och så skulle han falla ned på sitt ansikte och tillbedja Gud och betyga att "Gud verkligen är i eder".
1Co 14:26  Vad följer då härav, mina bröder? Jo, när I kommen tillsammans, så har var och en något särskilt att meddela: den ene har en psalm, den andre något till undervisning, en annan åter någon uppenbarelse, en talar tungomål, en annan uttyder; allt detta må nu ske så, att det länder till uppbyggelse.
1Co 14:27  Vill man tala tungomål, så må för var gång två eller högst tre få tala, och av dessa en i sänder, och en må uttyda det.
1Co 14:28  Är ingen uttydare tillstädes, så må de tiga i församlingen och tala allenast för sig själva och för Gud.
1Co 14:29  Av dem som vilja profetera må två eller tre få tala, och de andra må döma om det som talas.
1Co 14:30  Men om någon annan som sitter där får en uppenbarelse, då må den förste tiga.
1Co 14:31  Ty I kunnen alla få profetera, den ene efter den andre, så att alla bliva undervisade och alla förmanade;
1Co 14:32  och profeters andar äro profeterna underdåniga.
1Co 14:33  Gud är ju icke oordningens Gud, utan fridens.
1Co 14:34  Såsom kvinnorna tiga i alla andra de heligas församlingar, så må de ock tiga i edra församlingar. Det är dem icke tillstatt att tala, utan de böra underordna sig, såsom lagen bjuder.
1Co 14:35  Vilja de hava upplysning om något, så må de hemma fråga sina män; ty det är en skam för en kvinna att tala i församlingen. -
1Co 14:36  Eller är det från eder som Guds ord har utgått? Eller har det kommit allenast till eder?
1Co 14:37  Om någon menar sig vara en profet eller en man med andegåva, så må han ock inse att vad jag skriver till eder är Herrens bud.
1Co 14:38  Men vill någon icke inse detta, så vare det hans egen sak.
1Co 14:39  Alltså, mina bröder, varen ivriga att undfå profetians gåva och förmenen ej heller någon att tala tungomål.
1Co 14:40  Men låten allt tillgå på höviskt sätt och med ordning.
1Co 15:1  Mina bröder, jag vill påminna eder om det evangelium som jag förkunnade för eder, som I jämväl togen emot, och som I ännu stån kvar i,
1Co 15:2  genom vilket I ock bliven frälsta; jag vill påminna eder om huru jag förkunnade det för eder, såframt I eljest hållen fast därvid - om nu icke så är att I förgäves haven kommit till tro.
1Co 15:3  Jag meddelade eder ju såsom ett huvudstycke vad jag själv hade undfått: att Kristus dog för våra synder, enligt skrifterna,
1Co 15:4  och att han blev begraven, och att han har uppstått på tredje dagen, enligt skrifterna,
1Co 15:5  och att han visade sig för Cefas och sedan för de tolv.
1Co 15:6  Därefter visade han sig för mer än fem hundra bröder på en gång, av vilka de flesta ännu leva kvar, medan några äro avsomnade.
1Co 15:7  Därefter visade han sig för Jakob och sedan för alla apostlarna.
1Co 15:8  Allra sist visade han sig också för mig, som är att likna vid ett ofullgånget foster.
1Co 15:9  Ty jag är den ringaste bland apostlarna, ja, icke ens värdig att kallas apostel, jag som har förföljt Guds församling.
1Co 15:10  Men genom Guds nåd är jag vad jag är, och hans nåd mot mig har icke varit fåfäng, utan jag har arbetat mer än de alla - dock icke jag, utan Guds nåd, som har varit med mig.
1Co 15:11  Det må nu vara jag eller de andra, så är det på det sättet vi predika, och på det sättet I haven kommit till tro.
1Co 15:12  Om det nu predikas om Kristus att han har uppstått från de döda, huru kunna då somliga bland eder säga att det icke finnes någon uppståndelse från de döda?
1Co 15:13  Om det åter icke finnes någon uppståndelse från de döda, då har icke heller Kristus uppstått.
1Co 15:14  Men om Kristus icke har uppstått, då är ju vår predikan fåfäng, då är ock eder tro fåfäng;
1Co 15:15  då befinnas vi ock vara falska Guds vittnen, eftersom vi hava vittnat mot Gud att han har uppväckt Kristus, som han icke har uppväckt, om det är sant att döda icke uppstå.
1Co 15:16  Ja, om döda icke uppstå, så har ej heller Kristus uppstått.
1Co 15:17  Men om Kristus icke har uppstått, så är eder tro förgäves; I ären då ännu kvar i edra synder.
1Co 15:18  Då hava ju ock de gått förlorade, som hava avsomnat i Kristus.
1Co 15:19  Om vi i detta livet hava i Kristus haft vårt hopp, och därav intet bliver, då äro vi de mest ömkansvärda av alla människor.
1Co 15:20  Men nu har Kristus uppstått från de döda, såsom förstlingen av de avsomnade.
1Co 15:21  Ty eftersom döden kom genom en människa, så kom ock genom en människa de dödas uppståndelse.
1Co 15:22  Och såsom i Adam alla dö, så skola ock i Kristus alla göras levande.
1Co 15:23  Men var och en i sin ordning: Kristus såsom förstlingen, därnäst, vid Kristi tillkommelse, de som höra honom till.
1Co 15:24  Därefter kommer änden, då när han överlämnar riket åt Gud och Fadern, sedan han från andevärldens alla furstar och alla väldigheter och makter har tagit all deras makt.
1Co 15:25  Ty han måste regera "till dess han har lagt alla sina fiender under sina fötter".
1Co 15:26  Sist bland hans fiender bliver ock döden berövad all sin makt;
1Co 15:27  ty "allt har han lagt under hans fötter". Men när det heter att "allt är honom underlagt", då är uppenbarligen den undantagen, som har lagt allt under honom.
1Co 15:28  Och sedan allt har blivit Sonen underlagt, då skall ock Sonen själv giva sig under den som har lagt allt under honom. Och så skall Gud bliva allt i alla.
1Co 15:29  Vad kunna annars de som låta döpa sig för de dödas skull vinna därmed? Om så är att döda alls icke uppstå, varför låter man då döpa sig för deras skull?
1Co 15:30  Och varför undsätta vi oss själva var stund för faror?
1Co 15:31  Ty - så sant jag i Kristus Jesus, vår Herre, kan berömma mig av eder, mina bröder - jag lider döden dag efter dag.
1Co 15:32  Om jag hade tänkt såsom människor pläga tänka, när jag i Efesus kämpade mot vilddjuren, vad gagnade mig då det jag gjorde? Om döda icke uppstå - "låtom oss då äta och dricka, ty i morgon måste vi dö".
1Co 15:33  Faren icke vilse: "För goda seder dåligt sällskap är fördärv."
1Co 15:34  Vaknen upp till rätt nykterhet, och synden icke. Somliga finnas ju, som leva i okunnighet om Gud; eder till blygd säger jag detta.
1Co 15:35  Nu torde någon fråga: "På vad sätt uppstå då de döda, och med hurudan kropp skola de träda fram?"
1Co 15:36  Du oförståndige! Det frö du sår, det får ju icke liv, om det icke först har dött.
1Co 15:37  Och när du sår, då är det du sår icke den växt som en gång skall komma upp, utan ett naket korn, kanhända ett vetekorn, kanhända något annat.
1Co 15:38  Men Gud giver det en kropp, en sådan som han vill, och åt vart frö dess särskilda kropp.
1Co 15:39  Icke allt kött är av samma slag, utan människors har sin art, boskapsdjurs kött en annan art, fåglars kött åter en annan, fiskars återigen en annan.
1Co 15:40  Så finnas ock både himmelska kroppar och jordiska kroppar, men de himmelska kropparnas härlighet är av ett slag, de jordiska kropparnas av ett annat slag.
1Co 15:41  En härlighet har solen, en annan härlighet har månen, åter en annan härlighet hava stjärnorna; ja, den ena stjärnan är icke lik den andra i härlighet. -
1Co 15:42  Så är det ock med de dödas uppståndelse: vad som bliver sått förgängligt, det uppstår oförgängligt;
1Co 15:43  vad som bliver sått i ringhet, det uppstår i härlighet; vad som bliver sått i svaghet, det uppstår i kraft;
1Co 15:44  här sås en "själisk" kropp, där uppstår en andlig kropp. Så visst som det finnes en "själisk" kropp, så visst finnes det ock en andlig.
1Co 15:45  Så är ock skrivet: "Den första människan, Adam, blev en levande varelse med själ." Den siste Adam åter blev en levandegörande ande.
1Co 15:46  Men icke det andliga är det första, utan det "själiska"; sedan kommer det andliga.
1Co 15:47  Den första människan var av jorden och jordisk, den andra människan är av himmelen.
1Co 15:48  Sådan som den jordiska var, sådana äro ock de jordiska; och sådan som den himmelska är, sådana äro ock de himmelska.
1Co 15:49  Och såsom vi hava burit den jordiskas gestalt, så skola vi ock bära den himmelskas gestalt.
1Co 15:50  Mina bröder, vad jag nu vill säga är detta, att kött och blod icke kunna få Guds rike till arvedel; ej heller får förgängligheten oförgängligheten till arvedel.
1Co 15:51  Se, jag säger eder en hemlighet: Vi skola icke alla avsomna, men alla skola vi bliva förvandlade,
1Co 15:52  och det i ett nu, i ett ögonblick, vid den sista basunens ljud. Ty basunen skall ljuda, och de döda skola uppstå till oförgänglighet, och då skola vi bliva förvandlade.
1Co 15:53  Ty detta förgängliga måste ikläda sig oförgänglighet, och detta dödliga ikläda sig odödlighet.
1Co 15:54  Men när detta förgängliga har iklätt sig oförgänglighet, och detta dödliga har iklätt sig odödlighet, då skall det ord fullbordas, som står skrivet: "Döden är uppslukad och seger vunnen."
1Co 15:55  Du död, var är din seger? Du död, var är din udd?
1Co 15:56  Dödens udd är synden, och syndens makt kommer av lagen.
1Co 15:57  Men Gud vare tack, som giver oss segern genom vår Herre Jesus Kristus!
1Co 15:58  Alltså, mina älskade bröder, varen fasta, orubbliga, alltid överflödande i Herrens verk, eftersom I veten att edert arbete icke är fåfängt i Herren.
1Co 16:1  Vad nu angår insamlingen till de heliga, så mån I förfara på samma sätt som jag har förordnat för församlingarna i Galatien.
1Co 16:2  Var och en av eder må spara ihop vad han får tillfälle till, och på första dagen i var vecka må han lägga av detta hemma hos sig, så att insamlingen icke göres först vid min ankomst.
1Co 16:3  Men när jag kommer, skall jag sända åstad de män som I själva pröven vara lämpliga, med brev till Jerusalem, för att där frambära eder kärleksgåva.
1Co 16:4  Och om saken befinnes vara värd att också jag reser, så skola de få åtfölja mig.
1Co 16:5  Jag tänker nämligen komma till eder, sedan jag har farit genom Macedonien. Ty Macedonien vill jag allenast fara igenom,
1Co 16:6  men hos eder skall jag kanhända stanna något, möjligen vintern över, för att I därefter mån hjälpa mig till vägs, dit jag kan vilja begiva mig.
1Co 16:7  Jag vill icke besöka eder nu strax, på genomresa, ty jag hoppas att någon tid få stanna hos eder, om Herren så tillstädjer.
1Co 16:8  Men i Efesus vill jag stanna ända till pingst.
1Co 16:9  Ty en dörr till stor och fruktbärande verksamhet har öppnats för mig; jag har ock många motståndare.
1Co 16:10  Men när Timoteus kommer, så sen till, att han utan fruktan må kunna vistas hos eder. Han utför ju Herrens verk, han såväl som jag;
1Co 16:11  må därför ingen förakta honom. Hjälpen honom sedan till vägs i frid, så att han kommer åter till mig; ty jag väntar honom med bröderna.
1Co 16:12  Vad angår brodern Apollos, så har jag ivrigt uppmanat honom att med de andra bröderna begiva sig till eder. Han var dock alls icke hågad att komma just nu; men när det bliver honom lägligt, skall han komma.
1Co 16:13  Vaken, stån fasta i tron, skicken eder såsom män, varen starka.
1Co 16:14  Låten allt hos eder ske i kärlek.
1Co 16:15  Mina bröder, jag vill giva eder en förmaning: I kännen ju Stefanas' husfolk och veten att de äro förstlingen i Akaja, och att de hava ägnat sig åt de heligas tjänst;
1Co 16:16  därför mån I å eder sida underordna eder under dessa män och under envar som bistår dem i deras arbete och själv gör sig möda.
1Co 16:17  Jag gläder mig över att Stefanas och Fortunatus och Akaikus hava kommit hit, ty dessa hava givit mig ersättning för vad jag har måst sakna genom att vara skild från eder;
1Co 16:18  de hava vederkvickt min ande såväl som eder ande. Så lären eder nu att rätt uppskatta sådana män.
1Co 16:19  Församlingarna i provinsen Asien hälsar eder. Akvila och Priska, tillika med den församling som kommer tillhopa i deras hus, hälsa eder mycket i Herren.
1Co 16:20  Ja, alla bröderna hälsa eder. Hälsen varandra med en helig kyss.
1Co 16:21  Här skriver jag, Paulus, min hälsning med egen hand.
1Co 16:22  Om någon icke har Herren kär, så vare han förbannad. Marana, ta!
1Co 16:23  Herren Jesu nåd vare med eder.
1Co 16:24  Min kärlek är med eder alla, i Kristus Jesus.


\end{document}