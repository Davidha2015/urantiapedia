\begin{document}

\title{Galatians}

Gal 1:1  Paulus, apostel, icke från människor, ej heller genom någon människa, utan genom Jesus Kristus och genom Gud, Fadern, som har uppväckt honom från de döda -
Gal 1:2  jag, jämte alla de bröder som äro här med mig, hälsar församlingarna i Galatien.
Gal 1:3  Nåd vare med eder och frid ifrån Gud, vår Fader, och ifrån Herren Jesus Kristus,
Gal 1:4  som har utgivit sig själv för våra synder, för att rädda oss från den nuvarande onda tidsåldern, efter vår Guds och Faders vilja.
Gal 1:5  Honom tillhör äran i evigheternas evigheter. Amen.
Gal 1:6  Det förundrar mig att I så hastigt avfallen från honom, som har kallat eder till att vara i Kristi nåd, och vänden eder till ett nytt evangelium.
Gal 1:7  Likväl är detta icke något annat "evangelium"; det är allenast så, att några finnas som vålla förvirring bland eder och vilja förvända Kristi evangelium.
Gal 1:8  Men om någon, vore det ock vi själva eller en ängel från himmelen, förkunnar evangelium i strid mot vad vi hava förkunnat för eder, så vare han förbannad.
Gal 1:9  Ja, såsom vi förut hava sagt, så säger jag nu åter: Om någon förkunnar evangelium för eder i strid mot vad I haven undfått, så vare han förbannad.
Gal 1:10  Är det då människor som jag nu söker vinna för mig, eller är det Gud? Står jag verkligen efter att "vara människor till behag"? Nej, om jag ännu ville vara människor till behag, så vore jag icke en Kristi tjänare.
Gal 1:11  Ty det vill jag säga eder, mina bröder, att det evangelium som har blivit förkunnat av mig icke är någon människolära.
Gal 1:12  Det är ju icke heller av någon människa som jag har undfått det eller blivit undervisad däri, utan genom en uppenbarelse från Jesus Kristus.
Gal 1:13  I haven ju hört huru det var med mig, medan jag ännu vandrade i judiskt väsende: att jag då övermåttan våldsamt förföljde Guds församling och ville utrota den,
Gal 1:14  ja, att jag gick längre i judiskt väsende än många av mina samtida landsmän och ännu ivrigare nitälskade för mina fäders stadgar.
Gal 1:15  Men när han, som allt ifrån min moders liv har avskilt mig, och som genom sin nåd har kallat mig,
Gal 1:16  täcktes i mig uppenbara sin Son, för att jag bland hedningarna skulle förkunna evangelium om honom, då begav jag mig strax åstad; jag rådförde mig icke med någon människa,
Gal 1:17  ej heller for jag upp till Jerusalem, till dem som före mig voro apostlar. I stället for jag bort till Arabien och vände så åter tillbaka till Damaskus.
Gal 1:18  Först sedan, tre år därefter, for jag upp till Jerusalem, för att lära känna Cefas, och jag stannade hos honom femton dagar.
Gal 1:19  Men av de andra apostlarna såg jag ingen; allenast Jakob, Herrens broder, såg jag.
Gal 1:20  Och Gud vet att jag icke ljuger i vad jag här skriver till eder.
Gal 1:21  Därefter for jag till Syriens och Ciliciens bygder.
Gal 1:22  Men för de kristna församlingarna i Judeen var jag personligen okänd.
Gal 1:23  De hörde allenast huru man sade: "Han som förut förföljde oss, han förkunnar nu evangelium om den tro som han förr ville utrota."
Gal 1:24  Och de prisade Gud för min skull.
Gal 2:1  Efter fjorton års förlopp for jag sedan åter upp till Jerusalem, åtföljd av Barnabas; jag tog då också Titus med mig.
Gal 2:2  Men det var på grund av en uppenbarelse som jag for dit. Och för bröderna där framlade jag det evangelium som jag predikar bland hedningarna; särskilt framlade jag det för de män som stodo högst i anseende - detta av oro för att mitt strävande nu vore förgäves eller förut hade varit det.
Gal 2:3  Men icke ens Titus, min följeslagare, som var grek, blev nödgad att låta omskära sig.
Gal 2:4  Det var nämligen så, att några falska bröder hade kommit att upptagas i församlingen, dit de hade smugit sig in för att bespeja vår frihet, den som vi hava i Kristus Jesus, varefter de ville trälbinda oss.
Gal 2:5  Dock gåvo vi icke ens ett ögonblick vika för dem genom en sådan underkastelse; ty vi ville att evangelii sanning skulle bliva bevarad hos eder.
Gal 2:6  Och vad angår dem som ansågos något vara - hurudana de nu voro, det kommer icke mig vid; Gud har icke anseende till personen - så sökte icke dessa män, de som stodo högst i anseende, att pålägga mig några nya förpliktelser.
Gal 2:7  Tvärtom; de sågo att jag hade blivit betrodd med att förkunna evangelium för de oomskurna, likasom Petrus hade fått de omskurna på sin del -
Gal 2:8  ty densamme som hade stått Petrus bi vid hans apostlaverksamhet bland de omskurna, han hade ock stått mig bi bland hedningarna -
Gal 2:9  och när de nu förnummo vilken nåd som hade blivit mig given, räckte de mig och Barnabas handen till samarbete, både Jakob och Cefas och Johannes, de män som räknades för själva stödjepelarna; vi skulle verka bland hedningarna, och de bland de omskurna.
Gal 2:10  Allenast skulle vi tänka på de fattiga; och just detta har jag också vinnlagt mig om att göra.
Gal 2:11  Men när Cefas kom till Antiokia, trädde jag öppet upp mot honom, ty han hade befunnits skyldig till en försyndelse.
Gal 2:12  Förut hade han nämligen ätit tillsammans med hedningarna; men så kommo några män dit från Jakob, och efter deras ankomst drog han sig tillbaka och höll sig undan, av fruktan för de omskurna.
Gal 2:13  Till samma skrymteri gjorde sig också de andra judarna skyldiga, och så blev till och med Barnabas indragen i deras skrymteri.
Gal 2:14  Men när jag såg att de icke vandrade med fasta steg, enligt evangelii sanning, sade jag till Cefas i allas närvaro: "Om du, som är en jude, kan leva efter hednisk sed i stället för efter judisk, varför vill du då nödga hedningarna att leva efter judiskt sätt?"
Gal 2:15  Vi för vår del äro väl på grund av vår härkomst judar och icke "hedniska syndare";
Gal 2:16  men då vi nu veta att en människa icke bliver rättfärdig av laggärningar, utan genom tro på Kristus Jesus, hava också vi satt vår tro till Jesus Kristus, för att vi skola bliva rättfärdiga av tro på Kristus, och icke av laggärningar. Ty av laggärningar bliver intet kött rättfärdigt.
Gal 2:17  Men om nu jämväl vi, i det vi sökte att bliva rättfärdiga i Kristus, hava befunnits vara syndare, då är ju Kristus en syndatjänare? Bort det!
Gal 2:18  Om så vore att jag byggde upp igen detta samma som jag redan har brutit ned, då bevisade jag därmed, att jag var en överträdare.
Gal 2:19  Ty jag för min del har genom lagen dött bort ifrån lagen, för att jag skall leva för Gud. Jag är korsfäst med Kristus,
Gal 2:20  och nu lever icke mer jag, utan Kristus lever i mig; och det liv som jag nu lever i köttet, det lever jag i tron på Guds Son, som har älskat mig och utgivit sig själv för mig.
Gal 2:21  Jag förkastar icke Guds nåd. Om rättfärdighet kunde vinnas genom lagen, då hade ju Kristus icke behövt lida döden.
Gal 3:1  I oförståndige galater! Vem har så dårat eder, I som dock haven fått Jesus Kristus målad för edra ögon såsom korsfäst?
Gal 3:2  Allenast det vill jag att I skolen svara mig på: Kom det sig av laggärningar att I undfingen Anden, eller kom det sig därav att I lyssnaden i tro?
Gal 3:3  Ären I så oförståndiga? I, som haven begynt i Anden, viljen I nu sluta i köttet?
Gal 3:4  Haven I då upplevat så mycket förgäves - om det nu verkligen har varit förgäves?
Gal 3:5  Alltså, att han som förlänade eder Anden och utförde kraftgärningar bland eder gjorde detta, kom det sig av laggärningar eller därav att I lyssnaden i tro,
Gal 3:6  i enlighet med det ordet: "Abraham trodde Gud, och det räknades honom till rättfärdighet"?
Gal 3:7  Så mån I nu veta att de som låta det bero på tro, de äro Abrahams barn.
Gal 3:8  Och eftersom skriften förutsåg att det var av tro som hedningarna skulle bliva rättfärdiggjorda av Gud, så gav den i förväg åt Abraham detta glada budskap: "I dig skola alla folk varda välsignade."
Gal 3:9  Alltså bliva de som låta det bero på tro välsignade tillika med Abraham, honom som trodde.
Gal 3:10  Ty alla de som låta det bero på laggärningar, de äro under förbannelse. Det är nämligen skrivet: "Förbannad vare var och en som icke förbliver vid allt som är skrivet i lagens bok, och icke gör därefter."
Gal 3:11  Och att ingen i kraft av lag bliver rättfärdig inför Gud, det är uppenbart, eftersom det heter: "Den rättfärdige skall leva av tro."
Gal 3:12  Men i lagen beror det icke på tro; tvärtom heter det: "Den som gör efter dessa stadgar skall leva genom dem."
Gal 3:13  Kristus friköpte oss från lagens förbannelse, när han blev en förbannelse för vår skull. Det är ju skrivet: "Förbannad är var och en som är upphängd på trä."
Gal 3:14  Vi friköptes, för att den välsignelse som hade givits åt Abraham skulle i Jesus Kristus komma också hedningarna till del, så att vi genom tron skulle undfå den utlovade Anden.
Gal 3:15  Mina bröder, jag vill taga ett exempel från vad som gäller bland människor. Icke ens när fråga är om en människas testamentsförordnande, kan någon upphäva det eller lägga något därtill, sedan det en gång har vunnit gällande kraft.
Gal 3:16  Nu gåvos löftena åt Abraham, så ock åt hans "säd". Det heter icke: "och åt dem som komma av din säd", såsom när det talas om många; utan det heter, såsom när det talas om en enda: "och åt din säd", vilken är Kristus.
Gal 3:17  Vad jag alltså vill säga är detta: Ett förordnande som Gud redan hade givit gällande kraft kan icke genom en lag, som utgavs fyra hundra trettio år därefter, hava blivit ogiltigt, så att löftet därmed har gjorts om intet.
Gal 3:18  Om det nämligen vore på grund av lag som arvet skulle undfås, så vore det icke på grund av löfte. Men åt Abraham har Gud skänkt det genom ett löfte.
Gal 3:19  Vartill tjänade då lagen? Jo, på det att överträdelserna skulle komma i dagen, blev den efteråt given, för att gälla till dess att "säden" skulle komma, han åt vilken löftet hade blivit givet; och den utgavs genom änglar och överlämnades i en medlares hand.
Gal 3:20  Men den som är medlare är icke medlare för allenast en enda. Men Gud är en.
Gal 3:21  Är då lagen emot Guds löften? Bort det! Om en lag hade blivit given, som kunde göra levande, då skulle rättfärdigheten verkligen komma av lagen.
Gal 3:22  Men nu har skriften inneslutit alltsammans under synd, för att det som var utlovat skulle, av tro på Jesus Kristus, komma dem till del som tro.
Gal 3:23  Men förrän tron kom, voro vi inneslutna under lagen och höllos i förvar under den, i förbidan på den tro som en gång skulle uppenbaras.
Gal 3:24  Så har lagen blivit vår uppfostrare till Kristus, för att vi skola bliva rättfärdiga av tro.
Gal 3:25  Men sedan tron har kommit, stå vi icke mer under uppfostrare.
Gal 3:26  Alla ären I Guds barn genom tron, i Kristus Jesus;
Gal 3:27  ty I alla, som haven blivit döpta till Kristus, haven iklätt eder Kristus.
Gal 3:28  Här är icke jude eller grek, här är icke träl eller fri, här är icke man och kvinna: alla ären I ett i Kristus Jesus.
Gal 3:29  Hören I nu Kristus till, så ären I därmed ock Abrahams säd, arvingar enligt löftet.
Gal 4:1  Vad jag vill säga är detta: Så länge arvingen är barn, finnes ingen skillnad mellan honom och en träl, fastän han är herre över alla ägodelarna;
Gal 4:2  ty han står under förmyndare och förvaltare, intill den tid som fadern har bestämt.
Gal 4:3  Sammalunda höllos ock vi, när vi voro barn, i träldom under världens "makter".
Gal 4:4  Men när tiden var fullbordad, sände Gud sin Son, född av kvinna och ställd under lagen,
Gal 4:5  för att han skulle friköpa dem som stodo under lagen, så att vi skulle få söners rätt.
Gal 4:6  Och eftersom I nu ären söner, har han sänt i våra hjärtan sin Sons Ande, som ropar: "Abba! Fader!"
Gal 4:7  Så är du nu icke mer träl, utan son; och är du son, så är du ock arvinge, insatt därtill av Gud.
Gal 4:8  Förut, innan I ännu känden Gud, voren I trälar under gudar som till sitt väsende icke voro gudar.
Gal 4:9  Men nu, sedan I haven lärt känna Gud och, vad mer är, haven blivit kända av Gud, huru kunnen I nu vända tillbaka till de svaga och arma "makter", under vilka I åter på nytt viljen bliva trälar?
Gal 4:10  I akten ju på dagar och på månader och på särskilda tider och år. -
Gal 4:11  Jag är bekymrad för eder och fruktar att jag till äventyrs har arbetat förgäves för eder.
Gal 4:12  Jag beder eder, mina bröder: Bliven såsom jag är, eftersom jag har blivit såsom I voren. I haven icke gjort mig något för när.
Gal 4:13  I veten ju att det var på grund av kroppslig svaghet som jag första gången kom att förkunna evangelium för eder.
Gal 4:14  Och fastän mitt kroppsliga tillstånd då väl hade kunnat innebära en frestelse för eder, så sågen I det ändå icke med ringaktning eller leda, utan togen emot mig såsom en Guds ängel, ja, såsom Kristus Jesus själv.
Gal 4:15  När hör man eder nu prisa eder saliga? Det vittnesbördet kan jag nämligen giva eder, att I då, om så hade varit möjligt, skullen hava rivit ut edra ögon och givit dem åt mig.
Gal 4:16  Så har jag då blivit eder ovän därigenom att jag säger eder sanningen!
Gal 4:17  Man söker med iver att vinna eder för sig, men icke med en god iver; nej, del vilja avspärra eder från andra, för att I med så mycket större iver skolen hålla eder till dem.
Gal 4:18  Och det är nu gott att I bliven omfattade med ivrig omsorg, i en god sak, alltid, och icke allenast när jag är tillstädes hos eder,
Gal 4:19  I mina barn, som jag nu åter med vånda måste föda till livet, intill dess att Kristus har tagit gestalt i eder.
Gal 4:20  Jag skulle önska att jag just nu vore hos eder och kunde göra min röst rätt bevekande. Ty jag vet mig knappt någon råd med eder.
Gal 4:21  Sägen mig, I som viljen stå under lagen: haven I icke hört vad lagen säger?
Gal 4:22  Det är ju skrivet att Abraham fick två söner, en med sin tjänstekvinna, och en med sin fria hustru.
Gal 4:23  Men tjänstekvinnans son är född efter köttet, då däremot den fria hustruns son är född i kraft av löftet.
Gal 4:24  Dessa ord hava en djupare mening; ty de båda kvinnorna beteckna två förbund. Av dessa kommer det ena från berget Sina och föder sina barn till träldom, och detta har sin förebild i Agar.
Gal 4:25  Berget Sina kallas nämligen i Arabien för Agar och svarar emot det nuvarande Jerusalem, ty detta lever med sina barn i träldom.
Gal 4:26  Men det Jerusalem som är därovan, det är fritt, och det är vår moder.
Gal 4:27  Så är ju skrivet: "Jubla, du ofruktsamma, du som icke föder barn; brist ut och ropa, du som icke bliver moder, Ty den ensamma skall hava många barn, flera än den som har man."
Gal 4:28  Och I, mina bröder, ären löftets barn, likasom Isak var.
Gal 4:29  Men likasom förr i tiden den son som var född efter köttet förföljde den som var född efter Anden, så är det ock nu.
Gal 4:30  Dock, vad säger skriften? "Driv ut tjänstekvinnan och hennes son; ty tjänstekvinnans son skall förvisso icke ärva med den fria hustruns son."
Gal 4:31  Alltså, mina bröder, vi äro icke barn av en tjänstekvinna, utan av den fria hustrun.
Gal 5:1  För att vi skola vara fria, har Kristus frigjort oss. Stån därför fasta, och låten icke något nytt träldomsok läggas på eder.
Gal 5:2  Se, jag säger eder, jag Paulus, att om I låten omskära eder, så bliver Kristus eder till intet gagn.
Gal 5:3  Och för var och en som låter omskära sig betygar jag nu åter att han är pliktig att fullgöra hela lagen.
Gal 5:4  I haven kommit bort ifrån Kristus, I som viljen bliva rättfärdiga i kraft av lagen; I haven fallit ur nåden.
Gal 5:5  Vi vänta nämligen genom ande, av tro, den rättfärdighet som är vårt hopp.
Gal 5:6  Ty i Kristus Jesus betyder det intet huruvida någon är omskuren eller oomskuren; allt beror på huruvida han har en tro som är verksam genom kärlek.
Gal 5:7  I begynten edert lopp väl. Vem har nu lagt hinder i eder väg, så att I icke mer lyden sanningen?
Gal 5:8  Till sådant kom ingen maning från honom som har kallat eder.
Gal 5:9  Litet surdeg syrar hela degen.
Gal 5:10  För min del har jag i Herren den tillförsikten till eder, att I icke häri skolen tänka på annat sätt; men den som vållar förvirring bland eder, han skall bära sin dom, vem han än må vara.
Gal 5:11  Om så vore, mina bröder, att jag själv ännu predikade omskärelse, varför skulle jag då ännu alltjämt lida förföljelse? Då vore ju korsets stötesten röjd ur vägen. -
Gal 5:12  Jag skulle önska att de män som uppvigla eder läte omskära sig ända till avstympning.
Gal 5:13  I ären ju kallade till frihet, mina bröder; bruken dock icke friheten så, att köttet får något tillfälle. Fastmer mån I tjäna varandra genom kärleken.
Gal 5:14  Ty hela lagens uppfyllelse ligger i ett enda budord, nämligen detta: "Du skall älska din nästa såsom dig själv."
Gal 5:15  Men om I bitens inbördes och äten på varandra, så mån I se till, att I icke bliven uppätna av varandra.
Gal 5:16  Vad jag vill säga är detta: Vandren i ande, så skolen I förvisso icke göra vad köttet har begärelse till.
Gal 5:17  Ty köttet har begärelse mot Anden, och Anden mot köttet; de två ligga ju i strid med varandra, för att hindra eder att göra vad I viljen.
Gal 5:18  Men om I drivens av ande, så stån I icke under lagen.
Gal 5:19  Men köttets gärningar äro uppenbara: de äro otukt, orenhet, lösaktighet,
Gal 5:20  avgudadyrkan, trolldom, ovänskap, kiv, avund, vrede, genstridighet, tvedräkt, partisöndring,
Gal 5:21  missunnsamhet, mord, dryckenskap, vilt leverne och annat sådant, varom jag säger eder i förväg, såsom jag redan förut har sagt, att de som göra sådant, de skola icke få Guds rike till arvedel.
Gal 5:22  Andens frukt åter är kärlek, glädje, frid, tålamod, mildhet, godhet, trofasthet,
Gal 5:23  saktmod, återhållsamhet. Mot sådant är icke lagen.
Gal 5:24  Och de som höra Kristus Jesus till hava korsfäst sitt kött tillika med dess lustar och begärelser.
Gal 5:25  Om vi nu hava liv genom ande, så låtom oss ock vandra i ande.
Gal 5:26  Låtom oss icke söka fåfänglig ära, i det att vi utmana varandra och avundas varandra.
Gal 6:1  Mina bröder, om så händer att någon ertappas med att begå en försyndelse, då mån I, som ären andliga människor, upprätta honom i saktmods ande. Och du må hava akt på dig själv, att icke också du bliver frestad.
Gal 6:2  Bären varandras bördor; så uppfyllen I Kristi lag.
Gal 6:3  Ty om någon tycker sig något vara, fastän han intet är, så bedrager han sig själv.
Gal 6:4  Må var och en pröva sina egna gärningar; han skall då tillmäta sig berömmelse allenast efter vad han själv är, och icke efter vad andra äro.
Gal 6:5  Ty var och en har sin egen börda att bära.
Gal 6:6  Den som får undervisning i ordet, han låte den som undervisar honom få del med sig i allt gott.
Gal 6:7  Faren icke vilse. Gud låter icke gäcka sig. Ty vad människan sår, det skall hon ock skörda.
Gal 6:8  Den som sår i sitt kötts åker, han skall av köttet skörda förgängelse, men den som sår i Andens åker, han skall av Anden skörda evigt liv.
Gal 6:9  Och låtom oss icke förtröttas att göra vad gott är; ty om vi icke uppgivas, så skola vi, när tiden är inne, få inbärga vår skörd.
Gal 6:10  Må vi alltså, medan vi hava tillfälle, göra vad gott är mot var man, och först och främst mot dem som äro våra medbröder i tron.
Gal 6:11  Sen här med vilka stora bokstäver jag egenhändigt skriver till eder!
Gal 6:12  Alla de som eftersträva ett gott anseende här i köttet, de vilja nödga eder till omskärelse, detta allenast för att de själva skola undgå att bliva förföljda för Kristi kors' skull.
Gal 6:13  Ty icke ens dessa omskurna själva hålla lagen. Nej, det är för att kunna berömma sig av edert kött som de vilja att I skolen låta omskära eder.
Gal 6:14  Men vad mig angår, så vare det fjärran ifrån mig att berömma mig av något annat än av vår Herres, Jesu Kristi, kors, genom vilket världen för mig är korsfäst, och jag för världen.
Gal 6:15  Ty det kommer icke an på om någon är omskuren eller oomskuren; allt beror på huruvida han är en ny skapelse.
Gal 6:16  Och över alla dem som komma att vandra efter detta rättesnöre, över dem vare frid och barmhärtighet, ja, över Guds Israel.
Gal 6:17  Må nu ingen härefter vålla mig oro; ty jag bär Jesu märken på min kropp.
Gal 6:18  Vår Herres, Jesu Kristi, nåd vare med eder ande, mina bröder. Amen.


\end{document}