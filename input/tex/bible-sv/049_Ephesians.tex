\begin{document}

\title{Ephesians}

Eph 1:1  Paulus, genom Guds vilja Kristi Jesu apostel, hälsar de heliga som bo i Efesus, de i Kristus Jesus troende.
Eph 1:2  Nåd vare med eder och frid ifrån Gud, vår Fader, och Herren Jesus Kristus.
Eph 1:3  Välsignad vare vår Herres, Jesu Kristi, Gud och Fader, som i Kristus har välsignat oss med all den himmelska världens andliga välsignelse,
Eph 1:4  såsom han ju, förrän världens grund var lagd, har utvalt oss i honom till att vara heliga och ostraffliga inför sig.
Eph 1:5  Ty i sin kärlek förutbestämde han oss till barnaskap hos sig, genom Jesus Kristus, efter sin viljas behag,
Eph 1:6  den nådeshärlighet till pris, varmed han har benådat oss i den älskade.
Eph 1:7  I honom hava vi förlossning genom hans blod, förlåtelse för våra synder, efter hans nåds rikedom.
Eph 1:8  Och denna nåd har han i överflödande mått låtit komma oss till del, med all vishet och allt förstånd,
Eph 1:9  i det att han för oss har kungjort sin viljas hemlighet, enligt det beslut som han efter sitt behag hade fattat inom sig själv,
Eph 1:10  om en ordning som i tidernas fullbordan skulle komma till stånd, det beslutet att i Kristus sammanfatta allt som finnes i himmelen och på jorden.
Eph 1:11  I honom hava vi ock undfått vår arvslott, vi som förut voro bestämda därtill genom dens beslut, som verkar allting efter sin egen viljas råd.
Eph 1:12  Så skulle vi, hans härlighet till pris, vara de som i Kristus redan i förväg hava ägt ett hopp.
Eph 1:13  I honom haven jämväl I, sedan I haven fått höra sanningens ord, eder frälsnings evangelium, ja, i honom haven I, sedan I nu ock haven kommit till tron, såsom ett insegel undfått den utlovade helige Ande,
Eph 1:14  vilken är en underpant på vårt arv, till förvissning om att hans egendomsfolk skall förlossas, hans härlighet till pris.
Eph 1:15  Sedan jag fick höra om eder tro i Herren Jesus och om eder kärlek till alla de heliga,
Eph 1:16  har därför jag å min sida icke upphört att tacka Gud för eder, när jag tänker på eder i mina böner.
Eph 1:17  Och min bön är att vår Herres, Jesu Kristi, Gud, härlighetens Fader, må giva eder en visdomens och uppenbarelsens ande till kunskap om sig,
Eph 1:18  och att han må upplysa edra hjärtans ögon, så att I förstån hurudant det hopp är, vartill han har kallat eder, huru rikt på härlighet hans arv är bland de heliga,
Eph 1:19  och huru översvinnligt stor hans makt är på oss som tro - allt i enlighet med den väldiga styrkas kraft,
Eph 1:20  varmed han har verkat i Kristus, i det att han uppväckte honom från de döda och satte honom på sin högra sida i den himmelska världen,
Eph 1:21  över alla andevärldens furstar och väldigheter och makter och herrar, ja, över allt som kan nämnas, icke allenast i denna tidsålder, utan ock i den tillkommande.
Eph 1:22  "Allt lade han under hans fötter." Och honom gav han åt församlingen till att vara ett huvud över allting -
Eph 1:23  åt församlingen, ty den är hans kropp och är uppfylld av honom som uppfyller allt i alla.
Eph 2:1  Så har han ock gjort eder levande, eder som voren döda genom de överträdelser och synder
Eph 2:2  i vilka I förut vandraden, efter denna världs och tidsålders sätt, i det I följden fursten över luftens härsmakt, över den andemakt som nu är verksam i de ohörsamma.
Eph 2:3  Bland dessa voro förut också vi allasammans, där vi vandrade i vårt kötts begärelser och gjorde vad köttet och sinnet ville; och vi voro genom vår natur hemfallna åt vredesdomen, vi likasom de andra.
Eph 2:4  Men Gud, som är rik på barmhärtighet, har, för den stora kärleks skull, varmed han har älskat oss,
Eph 2:5  gjort oss levande med Kristus, oss som voro döda genom våra synder. Av nåd ären I frälsta!
Eph 2:6  Ja, han har uppväckt oss med honom och satt oss med honom i den himmelska världen, i Kristus Jesus,
Eph 2:7  för att i de kommande tidsåldrarna bevisa sin nåds översvinnliga rikedom, genom godhet mot oss i Kristus Jesus.
Eph 2:8  Ty av nåden ären I frälsta genom tro - och det icke av eder själva, Guds gåva är det -
Eph 2:9  icke av gärningar, för att ingen skall berömma sig.
Eph 2:10  Ty hans verk äro vi, skapade i Kristus Jesus till goda gärningar, vilka Gud förut har berett, för att vi skola vandra i dem.
Eph 2:11  Kommen därför ihåg att I förut, I som voren hedningar i köttet och bleven kallade oomskurna av dem som kallas omskurna, efter den omskärelse som med händer göres på köttet -
Eph 2:12  kommen ihåg att I på den tiden, då när I voren utan Kristus, voren utestängda från medborgarskap i Israel och främmande för löftets förbund, utan hopp och utan Gud i världen.
Eph 2:13  Nu däremot, då I ären i Kristus Jesus, haven I, som förut voren fjärran, kommit nära, i och genom Kristi blod.
Eph 2:14  Ty han är vår frid, han som av de båda har gjort ett och brutit ned den skiljemur som stod emellan oss, nämligen ovänskapen.
Eph 2:15  Ty i sitt kött gjorde han om intet budens stadgelag, för att han skulle av de två i sig skapa en enda ny människa och så bereda frid,
Eph 2:16  och för att han skulle åt dem båda, förenade i en enda kropp, skaffa försoning med Gud, sedan han genom korset hade i sin person dödat ovänskapen.
Eph 2:17  Och han har kommit och har förkunnat det glada budskapet om frid för eder, I som voren fjärran, och om frid för dem som voro nära.
Eph 2:18  Ty genom honom hava vi, de ena såväl som de andra, i en och samme Ande tillträde till Fadern.
Eph 2:19  Alltså ären I nu icke mer främlingar och gäster, utan I haven medborgarskap med de heliga och ären Guds husfolk,
Eph 2:20  uppbyggda på apostlarnas och profeternas grundval, där hörnstenen är Kristus Jesus själv,
Eph 2:21  i vilken allt det som uppbygges bliver sammanslutet och så växer upp till ett heligt tempel i Herren.
Eph 2:22  I honom bliven också I med de andra uppbyggda till en Guds boning, i Anden.
Eph 3:1  Fördenskull böjer jag mina knän, jag Paulus, som till gagn för eder, I hedningar, är Kristi Jesu fånge.
Eph 3:2  I haven väl hört om det nådesuppdrag av Gud, som är mig givet för eder räkning,
Eph 3:3  huru genom uppenbarelse den hemlighet blev för mig kungjord, varom jag ovan har i korthet skrivit.
Eph 3:4  Och när I läsen detta, kunnen I därav förstå vilken insikt jag har i Kristi hemlighet,
Eph 3:5  som under förgångna släktens tider icke hade blivit kungjord för människors barn, såsom den nu genom andeingivelse har blivit uppenbarad för hans heliga apostlar och profeter.
Eph 3:6  Jag menar den hemligheten, att hedningarna i Kristus Jesus äro våra medarvingar och jämte oss lemmar i en och samma kropp och jämte oss delaktiga i löftet - detta genom evangelium,
Eph 3:7  vars tjänare jag har blivit i följd av den Guds nåds gåva som blev mig given genom hans mäktiga kraft.
Eph 3:8  Ja, åt mig, den ringaste bland alla heliga, blev den nåden given att för hedningarna förkunna evangelium om Kristi outrannsakliga rikedom,
Eph 3:9  och att lägga i dagen huru det rådslut har blivit utfört, som tidsåldrarna igenom hade såsom en hemlighet varit fördolt i Gud, alltings skapare.
Eph 3:10  Ty Gud ville att hans mångfaldiga visdom nu, i och genom församlingen, skulle bliva kunnig för furstarna och väldigheterna i den himmelska världen.
Eph 3:11  Sådant hade hans beslut varit från tidsåldrarnas begynnelse, det som han utförde i Kristus Jesus, vår Herre.
Eph 3:12  Och i honom kunna vi med tillförsikt frimodigt träda fram, genom tron på honom.
Eph 3:13  Därför beder jag eder att icke fälla modet vid mina lidanden för eder; de lända ju eder till ära.
Eph 3:14  Fördenskull böjer jag mina knän för Fadern -
Eph 3:15  honom från vilken allt vad fader heter i himmelen och på jorden har sitt namn -
Eph 3:16  och beder att han ville efter sin härlighets rikedom förläna eder, att I genom hans Ande växen till i kraft till eder invärtes människa,
Eph 3:17  och att Kristus genom tron må bo i edra hjärtan, och att I mån vara rotade och grundade i kärleken,
Eph 3:18  så att I, tillika med alla de heliga, till fullo förmån fatta vad bredden och längden och höjden och djupet är
Eph 3:19  och så lära känna Kristi kärlek, som övergår all kunskap. Ty så skolen I bliva helt uppfyllda av all Guds fullhet.
Eph 3:20  Men honom, som förmår göra mer, ja, långt mer än allt vad vi bedja eller tänka, efter den kraft som är verksam i oss,
Eph 3:21  honom tillhör äran i församlingen och i Kristus Jesus alla släkten igenom i evigheternas evighet, amen.
Eph 4:1  Så förmanar jag nu eder, jag som är en fånge i Herren, att föra en vandel som är värdig den kallelse I haven undfått,
Eph 4:2  med all ödmjukhet och allt saktmod, med tålamod, så att I haven fördrag med varandra i kärlek
Eph 4:3  och vinnläggen eder om att bevara Andens enhet genom fridens band:
Eph 4:4  en kropp och en Ande, likasom I ock bleven kallade till att leva i ett och samma hopp, det som tillhör eder kallelse -
Eph 4:5  en Herre, en tro, ett dop, en Gud, som är allas Fader,
Eph 4:6  han som är över alla, genom alla och i alla.
Eph 4:7  Men åt var och en särskild av oss blev nåden given, alltefter som Kristus tillmätte honom sin gåva.
Eph 4:8  Därför heter det: "Han for upp i höjden, han tog fångar, han gav människorna gåvor."
Eph 4:9  Men detta ord "han for upp", vad innebär det, om icke att han förut hade farit hit ned till jordens lägre rymder?
Eph 4:10  Den som for ned, han är ock den som for upp över alla himlar, för att han skulle uppfylla allt.
Eph 4:11  Och han gav oss somliga till apostlar, somliga till profeter, somliga till evangelister, somliga till herdar och lärare.
Eph 4:12  Ty han ville göra de heliga skickliga till att utföra sitt tjänarvärv, att uppbygga Kristi kropp,
Eph 4:13  till dess att vi allasammans komma fram till enheten i tron och i kunskapen om Guds Son, till manlig mognad, och så bliva fullvuxna, intill Kristi fullhet.
Eph 4:14  Så skulle vi icke mer vara barn, icke såsom havets vågor drivas omkring av vart vindkast i läran, vid människornas bedrägliga spel, när de illfundigt söka främja villfarelsens listiga anslag.
Eph 4:15  Nej, vi skulle då hålla oss till sanningen, och i alla stycken i kärlek växa upp till honom som är huvudet, Kristus.
Eph 4:16  Ty från honom hämtar hela kroppen sin tillväxt, till att bliva uppbyggd i kärlek, i det att den sammanslutes och får sammanhållning genom det bistånd var led giver, med en kraft som är avmätt efter var särskild dels uppgift.
Eph 4:17  Jag tillsäger eder alltså och uppmanar eder allvarligt i Herren, att icke mer vandra såsom hedningarna i sitt sinnes fåfänglighet vandra,
Eph 4:18  hedningarna, vilka, i följd av den okunnighet som råder hos dem genom deras hjärtans förstockelse, äro förmörkade till förståndet och bortkomna från det liv som är av Gud.
Eph 4:19  Ty i sin försoffning hava de överlämnat sig åt lösaktighet, så att de i girighet bedriva alla slags orena gärningar.
Eph 4:20  Men I haven icke fått en sådan undervisning om Kristus,
Eph 4:21  om I eljest så haven hört om honom och så blivit lärda i honom, som sanning är i Jesus:
Eph 4:22  att I - då detta nu krävdes på grund av eder förra vandel - haven avlagt den gamla människan, som fördärvar sig genom att följa sina begärelsers bedrägliga lockelser,
Eph 4:23  och nu förnyens genom Anden som bor i edert sinne,
Eph 4:24  och att I haven iklätt eder den nya människan, som är skapad till likhet med Gud i sanningens rättfärdighet och helighet.
Eph 4:25  Läggen därför bort lögnen, och talen sanning med varandra, eftersom vi äro varandras lemmar.
Eph 4:26  "Vredgens, men synden icke"; låten icke solen gå ned över eder vrede,
Eph 4:27  och given icke djävulen något tillfälle.
Eph 4:28  Den som har stulit, han stjäle icke mer, utan arbete hellre, och uträtte med sina händer vad gott är, så att han har något varav han kan dela med sig åt den som lider brist.
Eph 4:29  Låten intet ohöviskt tal utgå ur eder mun, utan allenast det som är gott, till uppbyggelse, där sådan behöves, så att det bliver till välsignelse för dem som höra det.
Eph 4:30  Och bedröven icke Guds helige Ande, vilken I haven undfått såsom ett insegel, för förlossningens dag.
Eph 4:31  All bitterhet och häftighet och vrede, allt skriande och smädande, ja, allt vad ondska heter vare fjärran ifrån eder.
Eph 4:32  Varen i stället goda och barmhärtiga mot varandra, och förlåten varandra, såsom Gud i Kristus har förlåtit eder.
Eph 5:1  Bliven alltså Guds efterföljare, såsom hans älskade barn,
Eph 5:2  och vandren i kärlek, såsom Kristus älskade eder och utgav sig själv för oss till en gåva och ett offer, "Gud till en välbehaglig lukt".
Eph 5:3  Men otukt och orenhet, av vad slag det vara må, och girighet skolen I, såsom det anstår heliga, icke ens låta nämnas bland eder,
Eph 5:4  ej heller ohöviskt väsende och dåraktigt tal och gyckel; sådant är otillbörligt. Låten fastmer tacksägelse höras.
Eph 5:5  Ty det bören I veta, och det insen I också själva, att ingen otuktig eller oren människa har arvedel i Kristi och Guds rike, ej heller någon girig, ty en sådan är en avgudadyrkare.
Eph 5:6  Låten ingen bedraga eder med tomma ord; ty för sådana synder kommer Guds vrede över de ohörsamma.
Eph 5:7  Haven alltså ingen del i sådant.
Eph 5:8  I voren ju förut mörker, men nu ären I ljus i Herren; vandren då såsom ljusets barn.
Eph 5:9  Ty ljusets frukt består i allt vad godhet och rättfärdighet och sanning är.
Eph 5:10  Ja, vandren så, i det att I pröven vad som är välbehagligt för Herren.
Eph 5:11  Och haven ingen delaktighet i mörkrets gärningar, som icke giva någon frukt, utan avslöjen dem fastmer.
Eph 5:12  Vad av sådana människor i hemlighet förövas, därom är det skamligt till och med att tala;
Eph 5:13  men alltsammans bliver uppenbart, när det avslöjas genom ljuset. Ty varhelst något bliver uppenbart, där är ljus.
Eph 5:14  Därför heter det: "Vakna upp, du som sover, och stå upp ifrån de döda, så skall Kristus lysa fram för dig."
Eph 5:15  Sen därför noga till, huru I vandren: att I vandren icke såsom ovisa människor, utan såsom visa;
Eph 5:16  och tagen väl i akt vart lägligt tillfälle. Ty tiden är ond.
Eph 5:17  Varen alltså icke oförståndiga, utan förstån vad som är Herrens vilja.
Eph 5:18  Och dricken eder icke druckna av vin; ty därav kommer ett oskickligt leverne. Låten eder fastmer uppfyllas av ande,
Eph 5:19  och talen till varandra i psalmer och lovsånger och andliga visor, och sjungen och spelen till Herrens ära i edra hjärtan,
Eph 5:20  och tacken alltid Gud och Fadern för allt, i vår Herres, Jesu Kristi, namn.
Eph 5:21  Underordnen eder varandra i Kristi fruktan.
Eph 5:22  I hustrur, underordnen eder edra män, såsom I underordnen eder Herren;
Eph 5:23  ty en man är sin hustrus huvud, såsom Kristus är församlingens huvud, han som ock är denna sin kropps Frälsare.
Eph 5:24  Ja, såsom församlingen underordnar sig Kristus, så skola ock hustrurna i allt underordna sig sina män.
Eph 5:25  I män, älsken edra hustrur, såsom Kristus har älskat församlingen och utgivit sig själv för henne
Eph 5:26  till att helga henne, genom att rena henne medelst vattnets bad, i kraft av ordet.
Eph 5:27  Ty så ville han själv ställa fram församlingen inför sig i härlighet, utan fläck och skrynka och annat sådant; fastmer skulle hon vara helig och ostrafflig.
Eph 5:28  På samma sätt äro männen pliktiga att älska sina hustrur, då dessa ju äro deras egna kroppar; den som älskar sin hustru, han älskar sig själv.
Eph 5:29  Ingen har någonsin hatat sitt eget kött; i stället när och omhuldar man det, såsom Kristus gör med församlingen,
Eph 5:30  eftersom vi äro lemmar av hans kropp.
Eph 5:31  "Fördenskull skall en man övergiva sin fader och sin moder och hålla sig till sin hustru, och de tu skola varda ett kött." -
Eph 5:32  Den hemlighet som ligger häri är stor; jag säger detta med tanke på Kristus och församlingen.
Eph 5:33  Dock gäller också om eder att var och en skall älska sin hustru såsom sig själv; men hustrun å sin sida skall visa sin man vördnad.
Eph 6:1  I barn, varen edra föräldrar lydiga i Herren, ty detta är rätt och tillbörligt.
Eph 6:2  "Hedra din fader och din moder." Det är ju först detta bud som har ett löfte med sig:
Eph 6:3  "för att det må gå dig väl och du må länge leva på jorden".
Eph 6:4  Och I fäder, reten icke edra barn till vrede, utan fostren dem i Herrens tukt och förmaning.
Eph 6:5  I tjänare, varen edra jordiska herrar lydiga, med fruktan och bävan, av uppriktigt hjärta, såsom gällde det Kristus;
Eph 6:6  icke med ögontjänst, av begär att behaga människor, utan såsom Kristi tjänare, som av hjärtat göra Guds vilja;
Eph 6:7  och gören eder tjänst med villighet, såsom tjänaden I Herren och icke människor.
Eph 6:8  I veten ju att vad gott var och en gör, det skall han få igen av Herren, vare sig han är träl eller fri.
Eph 6:9  Och I herrar, handlen på samma sätt mot dem, och upphören att bruka hårda ord; I veten ju att i himmelen finnes den som är Herre över både dem och eder, och att hos honom icke finnes anseende till personen.
Eph 6:10  För övrigt, bliven allt starkare i Herren och i hans väldiga kraft.
Eph 6:11  Ikläden eder hela Guds vapenrustning, så att I kunnen hålla stånd emot djävulens listiga angrepp.
Eph 6:12  Ty den kamp vi hava att utkämpa är en kamp icke mot kött och blod, utan mot furstar och väldigheter och världshärskare, som råda här i mörkret, mot ondskans andemakter i himlarymderna.
Eph 6:13  Tagen alltså på eder hela Guds vapenrustning, så att I kunnen stå emot på den onda dagen och, sedan I haven fullgjort allt, behålla fältet.
Eph 6:14  Stån därför omgjordade kring edra länder med sanningen, och "varen iklädda rättfärdighetens pansar",
Eph 6:15  och haven såsom skor på edra fötter den beredvillighet som fridens evangelium giver.
Eph 6:16  Och tagen alltid trons sköld, varmed I skolen kunna utsläcka den ondes alla brinnande pilar.
Eph 6:17  Och låten giva eder "frälsningens hjälm" och Andens svärd, som är Guds ord.
Eph 6:18  Gören detta under ständig åkallan och bön, så att I alltjämt bedjen i Anden och fördenskull vaken, under ständig uthållighet och ständig bön för alla de heliga.
Eph 6:19  Bedjen ock för mig, att min mun må upplåtas, och att det jag skall tala må bliva mig givet, så att jag frimodigt kungör evangelii hemlighet,
Eph 6:20  för vars skull jag är ett sändebud i kedjor; ja, bedjen att jag må frimodigt tala därom med de rätta orden.
Eph 6:21  Men för att ock I skolen få veta något om mig, huru det går mig, kommer Tykikus, min älskade broder och trogne tjänare i Herren, att underrätta eder om allt.
Eph 6:22  Honom sänder jag till eder, just för att I skolen få veta huru det är med oss, och för att han skall hugna edra hjärtan.
Eph 6:23  Frid vare med bröderna och kärlek tillika med tro, från Gud, Fadern, och Herren Jesus Kristus.
Eph 6:24  Nåd vare med alla som älska vår Herre Jesus Kristus - nåd i oförgängligt liv.


\end{document}