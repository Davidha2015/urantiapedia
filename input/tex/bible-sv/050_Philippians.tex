\begin{document}

\title{Filipperbrevet}


\chapter{1}

\par 1 Paulus och Timoteus, Kristi Jesu tjänare, hälsa alla de heliga i Kristus Jesus som bo i Filippi, tillika med församlingsföreståndare och församlingstjänare.
\par 2 Nåd vare med eder och frid ifrån Gud, vår Fader, och Herren Jesus Kristus.
\par 3 Jag tackar min Gud, så ofta jag tänker på eder,
\par 4 i det jag alltid i alla mina böner med glädje beder för eder alla.
\par 5 Jag tackar honom för att I, allt ifrån första dagen intill nu, haven deltagit i arbetet för evangelium.
\par 6 Och jag har den tillförsikten, att han som i eder har begynt ett gott verk, han skall ock fullborda det, intill Kristi Jesu dag.
\par 7 Och det är ju rätt och tillbörligt att jag tänker så om eder alla, eftersom jag, både när jag ligger i bojor, och när jag försvarar och befäster evangelium, har eder i mitt hjärta såsom alla med mig delaktiga i nåden.
\par 8 Ty Gud är mitt vittne, han vet huru jag längtar efter eder alla med Kristi Jesu kärlek.
\par 9 Och därom beder jag, att eder kärlek må allt mer och mer överflöda av kunskap och förstånd i allt,
\par 10 så att I kunnen döma om vad rättast är, på det att I mån bliva rena och för ingen till stötesten, i väntan på Kristi dag,
\par 11 och bliva rika på rättfärdighetens frukt, vilken kommer genom Jesus Kristus, Gud till ära och pris.
\par 12 Jag vill att I, mina bröder, skolen veta att det som har vederfarits mig snarare har länt till evangelii framgång.
\par 13 Det har nämligen så blivit uppenbart för alla i pretoriet och för alla andra, att det är i Kristus som jag bär mina bojor;
\par 14 och de flesta av bröderna hava genom mina bojor blivit så frimodiga i Herren, att de med allt större dristighet våga oförskräckt förkunna Guds ord.
\par 15 Somliga finnas väl ock, som av avund och trätlystnad predika Kristus, men det finnes också andra som göra det av god vilja.
\par 16 Dessa senare göra det av kärlek, eftersom de veta att jag är satt till att försvara evangelium.
\par 17 De förra åter förkunna Kristus av genstridighet, icke med rent sinne, i tanke att de skola tillskynda mig ytterligare bedrövelse i mina bojor.
\par 18 Vad mer? Kristus bliver dock på ena eller andra sättet förkunnad, det må nu ske för syns skull eller i uppriktighet; och däröver gläder jag mig. Ja, jag skall ock framgent få glädja mig;
\par 19 ty jag vet att detta skall lända mig till frälsning, genom eder förbön och därigenom att Jesu Kristi Ande förlänas mig.
\par 20 Det är nämligen min trängtan och mitt hopp att jag i intet skall komma på skam, utan att Kristus, nu såsom alltid, skall av mig med all frimodighet bliva förhärligad i min kropp, det må ske genom liv eller genom död.
\par 21 Ty att leva, det är för mig Kristus, och att dö, det är för mig en vinning.
\par 22 Men om det att leva i köttet för mig är att utföra ett arbete som bär frukt, vilketdera skall jag då välja? Det kan jag icke säga.
\par 23 Jag drages åt båda hållen. Ty väl åstundar jag att bryta upp och vara hos Kristus, vilket ju vore mycket bättre;
\par 24 men att jag lever kvar i köttet är för eder skull mer av nöden.
\par 25 Och då jag är förvissad härom, vet jag att jag skall leva kvar och förbliva hos eder alla, eder till förkovran och glädje i tron,
\par 26 för att eder berömmelse skall överflöda i Kristus Jesus, i fråga om mig, därigenom att jag ännu en gång kommer till eder.
\par 27 Fören allenast en sådan vandel som är värdig Kristi evangelium, så att jag - vare sig jag kommer och besöker eder, eller jag förbliver frånvarande - får höra om eder att I stån fasta i en och samme Ande och endräktigt kämpen tillsammans för tron på evangelium,
\par 28 utan att i något stycke låta skrämma eder av motståndarna. Ty att I så skicken eder är för dem ett vittnesbörd om att de själva gå mot fördärvet, men att I skolen bliva frälsta, och detta av Gud.
\par 29 Åt eder har ju förunnats icke allenast att tro på Kristus, utan ock att lida för hans skull,
\par 30 i det att I haven samma kamp som I förr sågen mig hava och nu hören att jag har.

\chapter{2}

\par 1 Om nu förmaning i Kristus, om uppmuntran i kärlek, om gemenskap i Anden, om hjärtlig godhet och barmhärtighet betyda något,
\par 2 gören då min glädje fullkomlig, i det att I ären ens till sinnes, uppfyllda av samma kärlek, endräktiga, liksinnade,
\par 3 fria ifrån genstridighet och ifrån begär efter fåfänglig ära. Fasthellre må var och en i ödmjukhet akta den andre förmer än sig själv.
\par 4 Och sen icke var och en på sitt eget bästa, utan var och en också på andras.
\par 5 Varen så till sinnes som Kristus Jesus var,
\par 6 han som var till i Guds-skepnad, men icke räknade jämlikheten med Gud såsom ett byte,
\par 7 utan utblottade sig själv, i det han antog tjänare-skepnad, när han kom i människogestalt. Så befanns han i utvärtes måtto vara såsom en människa
\par 8 och ödmjukade sig och blev lydig intill döden, ja, intill döden på korset.
\par 9 Därför har ock Gud upphöjt honom över allting och givit honom det namn som är över alla namn
\par 10 för att i Jesu namn alla knän skola böja sig, deras som äro i himmelen, och deras som äro på jorden, och deras som äro under jorden,
\par 11 och för att alla tungor skola bekänna, Gud, Fadern, till ära, att Jesus Kristus är Herre.
\par 12 Därför, mina älskade, såsom I alltid förut haven varit lydiga, så mån I också nu med fruktan och bävan arbeta på eder frälsning, och det icke allenast såsom I gjorden, då jag var närvarande, utan ännu mycket mer nu, då jag är frånvarande.
\par 13 Ty Gud är den som verkar i eder både vilja och gärning, för att hans goda vilja skall ske.
\par 14 Gören allt utan att knorra och tveka,
\par 15 så att I bliven otadliga och rena, Guds ostraffliga barn mitt ibland "ett vrångt och avogt släkte", inom vilket I lysen såsom himlaljus i världen,
\par 16 i det att I hållen fast vid livets ord. Bliven mig så till berömmelse på Kristi dag, till ett vittnesbörd om att jag icke har strävat förgäves och icke förgäves har arbetat.
\par 17 Men om än mitt blod bliver utgjutet såsom ett drickoffer, när jag förrättar min tempeltjänst och därvid frambär offret av eder tro, så gläder jag mig dock och deltager i allas eder glädje.
\par 18 Sammalunda mån ock I glädjas och deltaga i min glädje.
\par 19 Jag hoppas nu i Herren Jesus att snart kunna sända Timoteus till eder, så att ock jag får känna hugnad genom det som jag då hör om eder.
\par 20 Ty jag har ingen av samma sinne som han, ingen som av så uppriktigt hjärta kommer att hava omsorg om eder.
\par 21 Allasammans söka de sitt eget, icke vad som hör Kristus Jesus till.
\par 22 Men hans beprövade trohet kännen I; I veten huru han med mig har verkat i evangelii tjänst, såsom en son tjänar sin fader.
\par 23 Honom hoppas jag alltså kunna sända, så snart jag har fått se huru det går med min sak.
\par 24 Och i Herren är jag viss om att jag också själv snart skall få komma.
\par 25 Emellertid har jag funnit det nödvändigt att sända brodern Epafroditus, min medarbetare och medkämpe, tillbaka till eder, honom som I haven skickat hit, för att å edra vägnar överlämna åt mig vad jag kunde behöva.
\par 26 Ty han längtar efter eder alla och har ingen ro, därför att I haven hört honom vara sjuk.
\par 27 Han har också verkligen varit sjuk, ja, nära döden, men Gud förbarmade sig över honom; och icke allenast över honom, utan också över mig, för att jag icke skulle få bedrövelse på bedrövelse.
\par 28 Därför är jag så mycket mer angelägen att sända honom, både för att I skolen få glädjen att återse honom, och för att jag själv därigenom skall få lättnad i min bedrövelse.
\par 29 Tagen alltså emot honom i Herren, med all glädje, och hållen sådana män i ära.
\par 30 Ty för Kristi verks skull var han nära döden, i det han satte sitt liv på spel, för att giva mig ersättning för den tjänst som jag måste sakna från eder personligen.

\chapter{3}

\par 1 För övrigt, mina bröder, glädjen eder i Herren. Att skriva till eder detsamma som förut, det räknar jag icke för något besvär, och det är för eder tryggare.
\par 2 Given akt på de hundarna, given akt på de onda arbetarna, given akt på "de sönderskurna".
\par 3 Ty vi äro "de omskurna", vi som genom Guds Ande tjäna Gud och berömma oss av Kristus Jesus och icke förtrösta på köttet -
\par 4 fastän jag för min del väl också kunde hava skäl att förtrösta på köttet. Ja, om någon menar sig kunna förtrösta på köttet, så kan jag det ännu mer,
\par 5 jag som blev omskuren, när jag var åtta dagar gammal, jag som är av Israels folk och av Benjamins stam, en hebré, född av hebréer, jag som i fråga om lagen har varit en farisé,
\par 6 i fråga om nitälskan varit en församlingens förföljare, i fråga om rättfärdighet - den som vinnes i kraft av lagen - varit en ostrafflig man.
\par 7 Men allt det som var mig en vinning, det har jag för Kristi skull räknat såsom en förlust.
\par 8 Ja, jag räknar i sanning allt såsom förlust mot det som är långt mer värt: kunskapen om Kristus Jesus, min Herre. Ty det är för hans skull som jag har gått förlustig alltsammans och nu räknar det såsom avskräde, på det att jag må vinna Kristus
\par 9 och bliva funnen i honom, icke med min egen rättfärdighet, den som kommer av lag, utan med den rättfärdighet som kommer genom tro på Kristus, rättfärdigheten av Gud, på grund av tron.
\par 10 Ty jag vill lära känna honom och hans uppståndelses kraft och få känna delaktighet i hans lidanden, i det jag bliver honom lik genom en död sådan som hans,
\par 11 om jag så skulle kunna nå fram till uppståndelsen från de döda.
\par 12 Icke som om jag redan hade vunnit det eller redan hade blivit fullkomlig, men jag far efter att vinna det, eftersom jag själv har blivit vunnen av Kristus Jesus.
\par 13 Ja, mina bröder, jag håller icke före att jag ännu har vunnit det, men ett gör jag: jag förgäter det som är bakom mig och sträcker mig mot det som är framför mig
\par 14 och jagar mot målet, för att få den segerlön som hålles framför oss genom Guds kallelse ovanifrån, i Kristus Jesus.
\par 15 Må därför vi alla som äro "fullkomliga" hava ett sådant tänkesätt. Men om så är, att I i något stycke haven andra tankar, så skall Gud också däröver giva eder klarhet.
\par 16 Dock, såvitt vi redan hava hunnit något framåt, så låtom oss vandra vidare på samma väg.
\par 17 Mina bröder, varen ock I mina efterföljare, och sen på dem som vandra på samma sätt som jag, eftersom I ju haven oss till föredöme.
\par 18 Ty det är såsom jag ofta har sagt eder och nu åter måste säga under tårar: många vandra såsom fiender till Kristi kors,
\par 19 och deras ände är fördärv; de hava buken till sin Gud och söka sin ära i det som är deras skam, och deras sinne är vänt till det som hör jorden till.
\par 20 Vi åter hava vårt medborgarskap i himmelen, och därifrån vänta vi ock Herren Jesus Kristus såsom Frälsare,
\par 21 vilken skall så förvandla vår förnedringskropp, att den bliver lik hans härlighetskropp - genom den kraft varmed han ock kan underlägga sig allt.

\chapter{4}

\par 1 Därför, mina älskade och efterlängtade bröder, min glädje och min krona, stån fasta i Herren med detta sinne, I mina älskade.
\par 2 Evodia förmanar jag, och Syntyke förmanar jag att de skola vara ens till sinnes i Herren.
\par 3 Ja, också till dig, min Synsygus - du som med rätta bär det namnet - har jag en bön: Var dessa kvinnor till hjälp, ty jämte mig hava de kämpat i evangelii tjänst, de såväl som Klemens och mina andra medarbetare, vilkas namn äro skrivna i livets bok.
\par 4 Glädjen eder i Herren alltid. Åter vill jag säga: Glädjen eder.
\par 5 Låten edert saktmod bliva kunnigt för alla människor. Herren är nära!
\par 6 Gören eder intet bekymmer, utan låten i allting edra önskningar bliva kunniga inför Gud, genom åkallan och bön, med tacksägelse.
\par 7 Så skall Guds frid, som övergår allt förstånd, bevara edra hjärtan och edra tankar, i Kristus Jesus.
\par 8 För övrigt, mina bröder, vad sant är, vad värdigt, vad rätt, vad rent är, vad som är älskligt och värt att akta, ja, allt vad dygd heter, och allt som förtjänar att prisas - tänken på allt sådant.
\par 9 Detta, som I haven lärt och inhämtat och haven hört av mig och sett hos mig, det skolen I göra; och så skall fridens Gud vara med eder.
\par 10 Det har varit för mig en stor glädje i Herren att I nu omsider haven kommit i en så god ställning, att I haven kunnat tänka på mitt bästa. Dock, I tänkten nog också förut därpå, men I haden icke tillfälle att göra något.
\par 11 Icke som om jag härmed ville säga att något har fattats mig; ty jag har lärt mig att vara nöjd med de omständigheter i vilka jag är.
\par 12 Jag vet att finna mig i ringhet, jag vet ock att finna mig i överflöd. Med vilken ställning och vilka förhållanden som helst är jag förtrogen: jag kan vara mätt, och jag kan vara hungrig; jag kan hava överflöd, och jag kan lida brist.
\par 13 Allt förmår jag i honom som giver mig kraft.
\par 14 Dock gjorden I väl däri att I visaden mig deltagande i mitt betryck.
\par 15 I veten ju ock själva, I filipper, att under evangelii första tid, då när jag hade dragit bort ifrån Macedonien, ingen annan församling än eder trädde i sådan förbindelse med mig, att räkning kunde föras över "utgivet och mottaget".
\par 16 Ty medan jag ännu var i Tessalonika, sänden I mig både en och två gånger vad jag behövde. -
\par 17 Icke som om jag skulle åstunda själva gåvan; nej, vad jag åstundar är en sådan frukt därav, som rikligen kommer eder själva till godo.
\par 18 Jag har nu fått ut allt, och det i överflödande mått. Jag har fullt upp, sedan jag av Epafroditus har mottagit eder gåva, "en välbehaglig lukt", ett offer som täckes Gud och behagar honom väl.
\par 19 Så skall ock min Gud, efter sin rikedom, i fullt mått och på ett härligt sätt i Kristus Jesus giva eder allt vad I behöven.
\par 20 Men vår Gud och Fader tillhör äran i evigheternas evigheter. Amen.
\par 21 Hälsen var och en av de heliga i Kristus Jesus. De bröder som äro här hos mig hälsa eder.
\par 22 Alla de heliga hälsa eder, först och främst de som höra till kejsarens hus.
\par 23 Herrens, Jesu Kristi, nåd vare med eder ande.


\end{document}