\begin{document}

\title{Colossians}

Col 1:1  Paulus, genom Guds vilja Kristi Jesu apostel, så ock brodern Timoteus,
Col 1:2  hälsar de heliga som bo i Kolosse, de troende bröderna i Kristus. Nåd vare med eder och frid ifrån Gud, vår Fader.
Col 1:3  Vi tacka Gud, vår Herres, Jesu Kristi, Fader, alltid för eder i våra böner,
Col 1:4  ty vi hava hört om eder tro i Kristus Jesus och om den kärlek som I haven till alla de heliga;
Col 1:5  vi tacka honom för det hopps skull, som är förvarat åt eder i himmelen. Om detta hopp haven I redan förut fått höra, genom sanningens ord i det evangelium
Col 1:6  som har kommit till eder, likasom det ock är att finna överallt i världen och där bär frukt och växer till, på samma sätt som det har gjort bland eder, allt ifrån den dag, då I hörden det och lärden i sanning känna Guds nåd.
Col 1:7  Det var ju en sådan undervisning I mottogen av Epafras, vår älskade medtjänare, som i vårt ställe är eder en trogen Kristi tjänare;
Col 1:8  det är också han som för oss har omtalat eder kärlek i Anden.
Col 1:9  Allt ifrån den dag då vi fingo höra härom, hava vi därför, å vår sida, icke upphört att bedja för eder och bönfalla om att I mån bliva uppfyllda av kunskap om Guds vilja, i allt slags andlig vishet och andligt förstånd.
Col 1:10  Så skolen I kunna föra en vandel som är värdig Herren, honom i allt till behag, och genom kunskapen om Gud bära frukt och växa till i allt gott verk.
Col 1:11  Och genom hans härliga makt skolen I på allt sätt uppfyllas av kraft till att bevisa ståndaktighet och tålamod i allt;
Col 1:12  och I skolen med glädje tacka Fadern, som har gjort eder skickliga till delaktighet i den arvslott som de heliga hava i ljuset.
Col 1:13  Ty han har frälst oss från mörkrets välde och försatt oss i sin älskade Sons rike.
Col 1:14  I honom hava vi förlossningen, förlåtelsen för våra synder,
Col 1:15  i honom som är den osynlige Gudens avbild och förstfödd före allt skapat.
Col 1:16  Ty i honom skapades allt i himmelen och på jorden, synligt såväl som osynligt, både tronänglar och herrar och furstar och väldigheter i andevärlden. Alltsammans har blivit skapat genom honom och till honom.
Col 1:17  Ja, han är till före allt annat, och alltsammans äger bestånd i honom.
Col 1:18  Och han är huvudet för kroppen, det är församlingen, han som är begynnelsen, den förstfödde ifrån de döda. Så skulle han i allt vara den främste.
Col 1:19  Ty det behagade Gud att låta all fullhet taga sin boning i honom
Col 1:20  och att genom honom försona allt med sig, sedan han genom blodet på hans kors hade berett frid. Ja, genom honom skulle så ske med allt vad på jorden och i himmelen är.
Col 1:21  Också åt eder, som förut voren bortkomna från honom och genom edert sinnelag hans fiender, i det att I gjorden vad ont var,
Col 1:22  också åt eder har han nu skaffat försoning i hans jordiska kropp, genom hans död, för att kunna ställa eder fram inför sig heliga och obefläckade och ostraffliga -
Col 1:23  om I nämligen förbliven i tron, väl grundade och fasta, utan att låta rubba eder från det hopp som tillbjudes oss i evangelium, det evangelium som I haven hört, och som blivit predikat bland allt skapat under himmelen, det vars tjänare jag, Paulus, har blivit.
Col 1:24  Nu gläder jag mig mitt i mina lidanden för eder; och vad som fattas i det mått av Kristus-bedrövelser som jag i mitt kött måste utstå, det uppfyller jag nu för hans kropp, som är församlingen.
Col 1:25  Ty dennas tjänare har jag blivit, i enlighet med det uppdrag av Gud, som har blivit mig givet, att jag nämligen överallt skall för eder förkunna Guds ord,
Col 1:26  den hemlighet som tidsåldrar och släkten igenom hade varit fördold, men som nu har blivit uppenbarad för hans heliga.
Col 1:27  Ty för dem ville Gud kungöra huru rik på härlighet den är bland hedningarna, denna hemlighet, vilken är "Kristus i eder, vårt härlighetshopp".
Col 1:28  Och honom förkunna vi för vår del, i det vi förmana var människa och undervisa var människa med all vishet, för att kunna ställa fram var människa såsom fullkomlig i Kristus.
Col 1:29  Och för det målet arbetar och kämpar jag, i enlighet med hans kraft, som mäktigt verkar i mig.
Col 2:1  Jag vill nämligen, att I skolen veta, vilken kamp jag har att utstå för eder och för församlingen i Laodicea och för alla de andra som icke personligen hava sett mitt ansikte.
Col 2:2  Ty jag önskar, att deras hjärtan skola få hugnad, därigenom att de slutas tillsammans i kärlek och komma till en full förståndsvisshets hela rikedom, till en rätt kunskap om Guds hemlighet, vilken är Kristus;
Col 2:3  ty i honom finnas visdomens och kunskapens alla skatter fördolda.
Col 2:4  Detta säger jag, för att ingen skall bedraga eder med skenfagert tal.
Col 2:5  Ty om jag ock till kroppen är frånvarande, så är jag dock i anden hos eder och gläder mig, när jag ser den ordning, som råder bland eder, och när jag ser fastheten i eder tro på Kristus.
Col 2:6  Såsom I nu haven mottagit Kristus Jesus, Herren, så vandren i honom
Col 2:7  och varen rotade i honom och låten eder uppbyggas i honom och befästas i tron, i enlighet med den undervisning I haven fått, och överflöden i tacksägelse.
Col 2:8  Sen till, att ingen får bortföra eder såsom ett segerbyte genom sin tomma och bedrägliga "vishetslära", i det att han åberopar fäderneärvda människomeningar och håller sig till världens "makter" och icke till Kristus.
Col 2:9  Ty i honom bor gudomens hela fullhet lekamligen,
Col 2:10  och i honom haven I blivit delaktiga av den fullheten, i honom som är huvudet för alla andevärldens furstar och väldigheter.
Col 2:11  I honom haven I ock blivit omskurna genom en omskärelse, som icke skedde med händer, en som bestod däri att I bleven avklädda eder köttsliga kropp; jag menar omskärelsen i Kristus.
Col 2:12  I haven ju med honom blivit begravna i dopet; I haven ock i dopet blivit uppväckta med honom, genom tron på Guds kraft, hans som uppväckte honom från de döda.
Col 2:13  Ja, också eder som voren döda genom edra synder och genom edert kötts oomskurenhet, också eder har han gjort levande med honom; ty han har förlåtit oss alla våra synder.
Col 2:14  Han har nämligen utplånat den handskrift som genom sina stadgar anklagade oss och låg oss i vägen; den har han skaffat undan genom att nagla den fast vid korset.
Col 2:15  Han har avväpnat andevärldens furstar och väldigheter och låtit dem bliva till skam inför alla, i det att han i honom har triumferat över dem.
Col 2:16  Låten därför ingen döma eder i fråga om mat och dryck eller angående högtid eller nymånad eller sabbat.
Col 2:17  Sådant är allenast en skuggbild av vad som skulla komma, men verkligheten själv finnes hos Kristus.
Col 2:18  Låten icke segerlönen tagas ifrån eder av någon som har sin lust i "ödmjukhet" och ängladyrkan och gör sig stor med sina syner, någon som utan orsak är uppblåst genom sitt köttsliga sinne
Col 2:19  och icke håller sig till honom som är huvudet, honom från vilken hela kroppen vinner sin tillväxt i Gud, i det att den av sina ledgångar och senor får bistånd och sammanhållning.
Col 2:20  Om I nu haven dött med Kristus och så blivit frigjorda ifrån världens "makter", varför låten I då allahanda stadgar läggas på eder, likasom levden I ännu i världen:
Col 2:21  "Det skall du icke taga i", "Det skall du icke smaka", "Det skall du icke komma vid",
Col 2:22  och detta när det gäller ting, som alla äro bestämda till att gå under genom förbrukning - allt till åtlydnad av människobud och människoläror?
Col 2:23  Visserligen har allt detta fått namn om sig att vara "vishet", eftersom däri ligger ett självvalt gudstjänstväsende och ett slags "ödmjukhet" och en kroppens späkning; men ingalunda ligger däri "en viss heder", det tjänar allenast till att nära det köttsliga sinnet.
Col 3:1  Om I alltså ären uppståndna med Kristus, så söken det som är därovan, där varest Kristus är och sitter på Guds högra sida.
Col 3:2  Ja, haven edert sinne vänt till det som är därovan, icke till det som är på jorden.
Col 3:3  Ty I haven dött, och edert liv är fördolt med Kristus i Gud.
Col 3:4  När Kristus, han som är vårt liv, bliver uppenbarad, då skolen ock I med honom bliva uppenbarade i härlighet.
Col 3:5  Så döden nu edra lemmar, som höra jorden till: otukt, orenhet, lusta, ond begärelse, så ock girigheten, som ju är avgudadyrkan;
Col 3:6  ty för sådant kommer Guds vrede.
Col 3:7  I de synderna vandraden också I förut, då I ännu haden edert liv i dem.
Col 3:8  Men nu skolen också I lägga bort alltsammans; vrede, häftighet, ondska, smädelse och skamligt tal ur eder mun;
Col 3:9  I skolen icke ljuga på varandra. I haven ju avklätt eder den gamla människan med hennes gärningar
Col 3:10  och iklätt eder den nya, den som förnyas till sann kunskap och så bliver en avbild av honom som har skapat henne.
Col 3:11  Och därvid kommer det icke an på om någon är grek eller jude, omskuren eller oomskuren, barbar eller skyt, träl eller fri; nej, Kristus är allt och i alla.
Col 3:12  Så kläden eder nu såsom Guds utvalda, hans heliga och älskade, i hjärtlig barmhärtighet, godhet, ödmjukhet, saktmod, tålamod.
Col 3:13  Och haven fördrag med varandra och förlåten varandra, om någon har något att förebrå en annan. Såsom Herren har förlåtit eder, så skolen ock I förlåta.
Col 3:14  Men över allt detta skolen I ikläda eder kärleken, ty den är fullkomlighetens sammanhållande band.
Col 3:15  Och låten Kristi frid regera i edra hjärtan; ty till att äga den ären I ock kallade såsom lemmar i en och samma kropp. Och varen tacksamma.
Col 3:16  Låten Kristi ord rikligen bo ibland eder; undervisen och förmanen varandra i all vishet, med psalmer och lovsånger och andliga visor, och sjungen med tacksägelse till Guds ära i edra hjärtan.
Col 3:17  Och allt, vadhelst I företagen eder i ord eller gärning, gören det allt i Herren Jesu namn och tacken Gud, Fadern, genom honom.
Col 3:18  I hustrur, underordnen eder edra män, såsom tillbörligt är i Herren.
Col 3:19  I män, älsken edra hustrur och varen icke bittra mot dem.
Col 3:20  I barn, varen edra föräldrar lydiga i allt, ty detta är välbehagligt i Herren.
Col 3:21  I fäder, reten icke edra barn, på det att de icke må bliva klenmodiga.
Col 3:22  I tjänare, varen i allt edra jordiska herrar lydiga, icke med ögontjänst, av begär att behaga människor, utan av uppriktigt hjärta, i Herrens fruktan.
Col 3:23  Vadhelst I gören, gören det av hjärtat, såsom tjänaden I Herren och icke människor.
Col 3:24  I veten ju, att I till vedergällning skolen av Herren få eder arvedel; den herre I tjänen är Kristus.
Col 3:25  Den som gör orätt, han skall få igen den orätt han har gjort, utan anseende till personen.
Col 4:1  I herrar, given edra tjänare, vad rätt och billigt är; I veten ju, att också I haven en herre i himmelen.
Col 4:2  Varen uthålliga i bönen och vaken i den under tacksägelse.
Col 4:3  Och bedjen jämväl för oss, att Gud må åt oss öppna en dörr för ordet, så att vi få förkunna Kristi hemlighet, den hemlighet, för vars skull jag också är en fånge;
Col 4:4  ja, bedjen, att jag må uppenbara den med de rätta orden.
Col 4:5  Skicken eder visligt mot dem som stå utanför och tagen väl i akt vart lägligt tillfälle.
Col 4:6  Edert tal vare alltid välbehagligt, kryddat med salt; I bören förstå, huru I skolen svara var och en.
Col 4:7  Om allt vad mig angår skall min älskade broder Tykikus, min trogne tjänare och min medtjänare i Herren, underrätta eder.
Col 4:8  Honom sänder jag till eder, just för att I skolen få veta, huru det är med oss, och för att han skall hugna edra hjärtan.
Col 4:9  Tillika sänder jag Onesimus, min trogne och älskade broder, eder landsman. De skola underrätta eder om allting här.
Col 4:10  Aristarkus, min medfånge, hälsar eder; så gör ock Markus, Barnabas syskonbarn. Angående honom haven I redan fått föreskrifter; och om han kommer till eder, så tagen vänligt emot honom.
Col 4:11  Också Jesus, som kallas Justus, hälsar eder. Av de omskurna äro dessa mina enda medarbetare för Guds rike, och de hava varit mig till hugnad.
Col 4:12  Epafras, eder landsman, hälsar eder, en Kristi Jesu tjänare, som i sina böner alltid kämpar för eder, för att I skolen stå fasta och vara fullkomliga och fullt vissa i allt som är Guds vilja.
Col 4:13  Ty jag giver honom det vittnesbördet, att han har stor möda för eder likasom ock för dem som bo i Laodicea och i Hierapolis.
Col 4:14  Lukas, läkaren, den älskade brodern, hälsar eder; så gör ock Demas.
Col 4:15  Hälsen bröderna i Laodicea, så ock Nymfas tillika med den församling som kommer tillhopa i hans hus.
Col 4:16  Sedan detta brev har blivit uppläst hos eder, så sörjen för att det ock bliver uppläst i laodicéernas församling och att jämväl I fån läsa det brev, som kommer från Laodicea.
Col 4:17  Sägen ock detta till Arkippus: "Hav akt på det ämbete, som du har undfått i Herren, så att du fullgör, vad därtill hör."
Col 4:18  Här skriver jag, Paulus, min hälsning med egen hand. Tänken på mina bojor. Nåd vare med eder.


\end{document}