\begin{document}

\title{2 Tessalonikerbrevet}


\chapter{1}

\par 1 Paulus och Silvanus och Timoteus hälsa tessalonikernas församling i Gud, vår Fader, och Herren Jesus Kristus.
\par 2 Nåd vare med eder och frid ifrån Gud, Fadern, och Herren Jesus Kristus.
\par 3 Vi äro pliktiga att alltid tacka Gud för eder, käre bröder, såsom tillbörligt är, därför att eder tro så mäktigt tillväxer, och den kärlek I haven till varandra mer och mer förökas hos eder alla och hos envar av eder.
\par 4 Därför kunna vi själva i Guds församlingar berömma oss av eder, i fråga om eder ståndaktighet och eder tro under alla edra förföljelser, och under de lidanden I måsten uthärda.
\par 5 Sådant är ett vittnesbörd om att Guds dom bliver rättvis. Så skolen I aktas värdiga Guds rike; för dess skull är det ock som I liden.
\par 6 Guds rättfärdighet kräver ju att de som vålla eder lidande få lidande till vedergällning,
\par 7 men att I som utstån lidanden fån hugnad tillsammans med oss, när Herren Jesus uppenbarar sig från himmelen med sin makts änglar,
\par 8 "i lågande eld", och låter straffet drabba dem som icke känna Gud, och dem som icke äro vår Herre Jesu evangelium lydiga.
\par 9 Dessa skola då bliva straffade med evigt fördärv, bort undan Herrens ansikte och hans överväldigande härlighet,
\par 10 när han kommer för att förhärligas i sina heliga och visa sig underbar i alla dem som hava kommit till att tro; ty det vittnesbörd vi hava framburit till eder haven I trott. Så skall ske på den dagen.
\par 11 Fördenskull bedja vi ock alltid för eder, att vår Gud må akta eder värdiga sin kallelse, och att han må med kraft fullborda i eder allt vad en god vilja kan åstunda, och vad tro kan verka,
\par 12 så att vår Herre Jesu namn bliver förhärligat i eder, och I i honom, efter vår Guds och Herrens, Jesu Kristi, nåd.

\chapter{2}

\par 1 I fråga om vår Herres, Jesu Kristi, tillkommelse, och huru vi skola församlas till honom, bedja vi eder, käre bröder,
\par 2 att I icke - vare sig genom någon "andeingivelse" eller på grund av något ord eller något brev, som förmenas komma från oss - så hastigt låten eder bringas ur fattningen och förloren besinningen, som om Herrens dag redan stode för dörren.
\par 3 Låten ingen bedraga eder om vad sätt det vara må. Ty först måste avfallet hava skett och "Laglöshetens människa", fördärvets man, hava trätt fram,
\par 4 vedersakaren, som upphäver sig över allt vad gud heter, och allt som kallas heligt, så att han tager sitt säte i Guds tempel och föregiver sig vara Gud.
\par 5 Kommen I icke ihåg att jag sade eder detta, medan jag ännu var hos eder?
\par 6 Och I veten vad det är som nu håller honom tillbaka, så att han först när hans tid är inne kan träda fram.
\par 7 Redan är ju laglöshetens hemlighet verksam; allenast måste den som ännu håller tillbaka först skaffas ur vägen.
\par 8 Sedan skall "den Laglöse" träda fram, och honom skall då Herren Jesus döda med sin muns anda och tillintetgöra genom sin tillkommelses uppenbarelse -
\par 9 honom som efter Satans tillskyndelse kommer med lögnens alla kraftgärningar och tecken och under
\par 10 och med orättfärdighetens alla bedrägliga konster, för att bedraga dem som gå förlorade, till straff därför att de icke gåvo kärleken till sanningen rum, så att de kunde bliva frälsta.
\par 11 Därför sänder ock Gud över dem villfarelsens makt, så att de sätta tro till lögnen,
\par 12 för att de skola bliva dömda, alla dessa som icke hava satt tro till sanningen, utan funnit behag i orättfärdigheten.
\par 13 Men vi för vår del äro pliktiga att alltid tacka Gud för eder käre bröder, I Herrens älskade, därför att Gud från begynnelsen har utvalt eder till frälsning, i helgelse i Anden och i tro på sanningen.
\par 14 Härtill har han ock genom vårt evangelium kallat eder, för att I skolen bliva delaktiga av vår Herres, Jesu Kristi, härlighet.
\par 15 Stån alltså fasta, käre bröder, och hållen eder vid de lärdomar som I haven mottagit av oss, vare sig muntligen eller genom brev.
\par 16 Och vår Herre Jesus Kristus själv och Gud, vår Fader, som har älskat oss och i nåd berett oss en evig hugnad och givit oss ett gott hopp,
\par 17 han hugne edra hjärtan, och styrke eder till allt vad gott är, både i gärning och i ord.

\chapter{3}

\par 1 För övrigt, käre bröder, bedjen för oss, att Herrens ord må hava framgång och komma till ära hos andra likasom hos eder,
\par 2 så ock att vi må bliva frälsta ifrån vanartiga och onda människor. Ty tron är icke var mans.
\par 3 Men Herren är trofast, och han skall styrka eder och bevara eder från det onda.
\par 4 Och vi hava den tillförsikten till eder i Herren, att I både nu gören och framgent skolen göra vad vi bjuda eder.
\par 5 Ja, Herren styre edra hjärtan till Guds kärlek och Kristi ståndaktighet.
\par 6 Men vi bjuda eder, käre bröder, i vår Herres, Jesu Kristi, namn, att I dragen eder ifrån var broder som för en oordentlig vandel och icke lever efter de lärdomar han har mottagit av oss.
\par 7 I veten ju själva huru man bör efterfölja oss. Ty vi förhöllo oss icke oordentligt bland eder,
\par 8 ej heller åto vi någons bröd för intet; tvärtom åto vi vårt bröd under arbete och möda, och vi strävade natt och dag, för att icke bliva någon av eder till tunga.
\par 9 Icke som om vi ej hade haft rätt därtill, men vi ville låta eder i oss få ett föredöme, för att I skullen efterfölja oss.
\par 10 Redan när vi voro hos eder, gåvo vi ju eder det budet: om någon icke vill arbeta, så skall han icke heller äta.
\par 11 Vi höra nämligen att somliga bland eder föra en oordentlig vandel och icke arbeta, utan allenast syssla med sådant som icke kommer dem vid.
\par 12 Sådana människor bjuda och förmana vi i Herren Jesus Kristus, att de arbeta i stillhet, så att de kunna äta sitt eget bröd.
\par 13 Och I, käre bröder, mån icke förtröttas att göra vad gott är.
\par 14 Men om någon icke lyder vad vi hava sagt i detta brev, så märken ut för eder den mannen, och haven intet umgänge med honom, på det att han må blygas.
\par 15 Hållen honom dock icke för en ovän, utan förmanen honom såsom en broder.
\par 16 Men fridens Herre själv give eder sin frid alltid och på allt sätt. Herren vare med eder alla.
\par 17 Här skriver jag, Paulus, min hälsning med egen hand. Detta är ett kännetecken i alla mina brev; så skriver jag.
\par 18 Vår Herres, Jesu Kristi, nåd vare med eder alla.


\end{document}