\begin{document}

\title{Hebrews}

Heb 1:1  Sedan Gud fordom många gånger och på många sätt hade talat till fäderna genom profeterna,
Heb 1:2  har han nu, på det yttersta av denna tid, talat till oss genom sin Son, som han har insatt till arvinge av allt, genom vilken han ock har skapat världen.
Heb 1:3  Och eftersom denne är hans härlighets återsken och hans väsens avbild och genom sin makts ord bär allt, har han - sedan han hade utfört en rening från synderna - satt sig på Majestätets högra sida i höjden.
Heb 1:4  Och han har blivit så mycket större än änglarna som det namn han har ärvt är förmer än deras.
Heb 1:5  Ty till vilken av änglarna har han någonsin sagt: "Du är min Son, jag har i dag fött dig"? eller: "Jag skall vara hans Fader, och han skall vara min Son"?
Heb 1:6  Likaså säger han, med tanke på den tid då han åter skall låta den förstfödde inträda i världen: "Och alla Guds änglar skola tillbedja honom."
Heb 1:7  Och medan han om änglarna säger: "Han gör sina änglar till vindar och sina tjänare till eldslågor",
Heb 1:8  säger han om Sonen: "Gud, din tron förbliver alltid och evinnerligen, och rättvisans spira är ditt rikes spira.
Heb 1:9  Du har älskat rättfärdighet och hatat orättfärdighet; därför, o Gud, har din Gud smort dig med glädjens olja mer än dina medbröder";
Heb 1:10  så ock: "Du, Herre, lade i begynnelsen jordens grund, och himlarna äro dina händers verk;
Heb 1:11  de skola förgås, men du förbliver; de skola alla nötas ut såsom en klädnad,
Heb 1:12  och såsom en mantel skall du hoprulla dem; såsom en klädnad skola de ock bytas om. Men du är densamme, och dina år skola icke hava någon ände."
Heb 1:13  Och till vilken av änglarna har han någonsin sagt: "Sätt dig på min högra sida, till dess jag har lagt dina fiender dig till en fotapall"?
Heb 1:14  Äro de icke allasammans tjänsteandar, som sändas ut till tjänst för deras skull som skola få frälsning till arvedel?
Heb 2:1  Därför böra vi så mycket mer akta på det som vi hava hört, så att vi icke gå förlorade.
Heb 2:2  Ty om det ord som talades genom änglar blev beståndande, och all överträdelse och olydnad fick sin rättvisa lön,
Heb 2:3  huru skola då vi kunna undkomma, om vi icke taga vara på en sådan frälsning? - en frälsning som ju först förkunnades genom Herren och sedan bekräftades för oss av dem som hade hört honom,
Heb 2:4  varjämte Gud själv ytterligare gav sitt vittnesbörd genom tecken och under och allahanda kraftgärningar, och genom att utdela helig ande, allt efter sin vilja.
Heb 2:5  Ty det var icke under änglars välde som han lade den tillkommande världen, den som vi tala om.
Heb 2:6  Däremot har någon någonstädes betygat och sagt: "Vad är en människa, att du tänker på henne, eller en människoson, att du låter dig vårda om honom?
Heb 2:7  En liten tid lät du honom vara ringare än änglarna, men krönte honom sedan med härlighet och ära och satte honom till herre över dina händers verk;
Heb 2:8  allt lade du under hans fötter." När han underlade honom allting, undantog han nämligen intet från att bliva honom underlagt - om vi ock ännu icke se allting vara honom underlagt.
Heb 2:9  Men honom som en liten tid hade blivit gjord "ringare än änglarna", honom, Jesus, se vi för sitt dödslidandes skull hava blivit krönt med härlighet och ära, för att det genom Guds nåd skulle komma alla till godo att han smakade döden.
Heb 2:10  Ty den för vilkens skull allting är, och genom vilken allting är, honom hövdes det, att när han ville föra många sina barn till härlighet, genom lidanden fullkomna deras frälsnings hövding.
Heb 2:11  Han som helgar och de som bliva helgade hava nämligen alla en och samme Fader. Fördenskull blyges han icke för att kalla dem bröder;
Heb 2:12  han säger ju: "Jag skall förkunna ditt namn för mina bröder, mitt i församlingen skall jag prisa dig";
Heb 2:13  så ock: "Jag vill sätta min förtröstan till honom"; så ock: "Se här äro jag och barnen som Gud har givit mig."
Heb 2:14  Då nu barnen hade blivit delaktiga av kött och blod, blev ock han på ett liknande sätt delaktig därav, för att han genom sin död skulle göra dens makt om intet, som hade döden i sitt våld, det är djävulen,
Heb 2:15  och göra alla dem fria, som av fruktan för döden hela sitt liv igenom hade varit hemfallna till träldom.
Heb 2:16  Ty det är ju icke änglar som han tager sig an; det är Abrahams säd som han tager sig an.
Heb 2:17  Därför måste han i allt bliva lik sina bröder, för att han skulle bliva barmhärtig och en trogen överstepräst i sin tjänst inför Gud, till att försona folkets synder.
Heb 2:18  Ty därigenom att han har lidit, i det han själv blev frestad, kan han hjälpa dem som frestas.
Heb 3:1  Därför, I helige bröder, I som haven blivit delaktiga av en himmelsk kallelse, skolen I akta på vår bekännelses apostel och överstepräst, Jesus,
Heb 3:2  huru han var trogen mot den som hade insatt honom, likasom Moses var "trogen i hela hans hus".
Heb 3:3  Ty han har blivit aktad värdig så mycket större härlighet än Moses, som uppbyggaren av ett hus åtnjuter större ära än själva huset.
Heb 3:4  Vart och ett hus bygges ju av någon, men Gud är den som har byggt allt.
Heb 3:5  Och väl var Moses "trogen i hela hans hus", såsom "tjänare", till ett vittnesbörd om vad som framdeles skulle förkunnas;
Heb 3:6  men Kristus var trogen såsom "son", en son satt över hans hus. Och hans hus äro vi, såframt vi intill änden hålla fast vår frimodighet och vår berömmelse i hoppet.
Heb 3:7  Så säger den helige Ande: "I dag, om I fån höra hans röst,
Heb 3:8  mån I icke förhärda edra hjärtan, såsom när de förbittrade mig på frestelsens dag i öknen,
Heb 3:9  där edra fäder frestade mig och prövade mig, fastän de hade sett mina verk i fyrtio år.
Heb 3:10  Därför blev jag förtörnad på det släktet och sade: 'Alltid fara de vilse med sina hjärtan.' Men de ville icke veta av mina vägar.
Heb 3:11  Så svor jag då i min vrede: De skola icke komma in i min vila."
Heb 3:12  Sen därför till, mina bröder, att icke hos någon bland eder finnes ett ont otroshjärta, så att han avfaller från den levande Guden,
Heb 3:13  utan förmanen varandra alla dagar, så länge det heter "i dag", på det att ingen av eder må bliva förhärdad genom syndens makt att bedraga.
Heb 3:14  Ty vi hava blivit delaktiga av Kristus, såframt vi eljest intill änden hålla fast vår första tillförsikt.
Heb 3:15  När det nu säges: "I dag, om I fån höra hans röst, mån I icke förhärda edra hjärtan, såsom när de förbittrade mig",
Heb 3:16  vilka voro då de som förbittrade honom, fastän de hade hört hans ord? Var det icke alla de som under Moses hade dragit ut ur Egypten?
Heb 3:17  Och vilka voro de som han var förtörnad på i fyrtio år? Var det icke de som hade syndat, de "vilkas kroppar föllo i öknen"?
Heb 3:18  Och vilka gällde den ed som han svor, att de "icke skulle komma in i hans vila", vilka, om icke dem som hade varit ohörsamma?
Heb 3:19  Så se vi då att det var för otros skull som de icke kunde komma ditin.
Heb 4:1  Eftersom nu ett löfte att få komma in i hans vila ännu står kvar, må vi alltså med fruktan se till, att icke någon bland eder en gång befinnes hava blivit efter på vägen.
Heb 4:2  Ty det glada budskapet hava vi mottagit såväl som de; men för dem blev det löftesord de fingo höra till intet gagn, eftersom det icke genom tron hade blivit upptaget i dem som hörde det.
Heb 4:3  Vi som hava kommit till tro, vi få ju komma in i vilan. Det heter också: "Så svor jag då i min vrede: De skola icke komma in i min vila", och detta fastän hans verk stodo där färdiga allt ifrån den tid då världen var skapad.
Heb 4:4  Ty om den sjunde dagen heter det någonstädes så: "Och Gud vilade på sjunde dagen från alla sina verk";
Heb 4:5  här åter heter det: "De skola icke komma in i min vila."
Heb 4:6  Eftersom det alltså står kvar att några skola få komma in i den, och eftersom de som först mottogo det glada budskapet för sin ohörsamhets skull icke kommo ditin,
Heb 4:7  så bestämmer han genom ordet "i dag" åter en viss dag, nu då han så lång tid därefter säger hos David, såsom förut är nämnt: "I dag, om I fån höra hans röst, mån I icke förhärda edra hjärtan."
Heb 4:8  Ty om Josua hade fört dem in i vilan, så skulle Gud icke hava talat om en annan, senare dag.
Heb 4:9  Alltså står en sabbatsvila ännu åter för Guds folk.
Heb 4:10  Ty den som har kommit in i hans vila, han har funnit vila från sina verk, likasom Gud från sina.
Heb 4:11  Så låtom oss nu med all flit sträva efter att få komma in i den vilan, för att ingen må, såsom de, falla och bliva ett varnande exempel på ohörsamhet.
Heb 4:12  Ty Guds ord är levande och kraftigt och skarpare än något tveeggat svärd, och tränger igenom, så att det åtskiljer själ och ande, märg och ben; och det är en domare över hjärtats uppsåt och tankar.
Heb 4:13  Intet skapat är fördolt för honom, utan allt ligger blottat och uppenbart för hans ögon; och inför honom skola vi göra räkenskap.
Heb 4:14  Eftersom vi nu hava en stor överstepräst, som har farit upp genom himlarna, nämligen Jesus, Guds Son, så låtom oss hålla fast vid bekännelsen.
Heb 4:15  Ty vi hava icke en sådan överstepräst som ej kan hava medlidande med våra svagheter, utan en som har varit frestad i allting, likasom vi, dock utan synd.
Heb 4:16  Låtom oss därför med frimodighet gå fram till nådens tron, för att vi må undfå barmhärtighet och finna nåd, till hjälp i rätt tid.
Heb 5:1  Ty var och en som skall bliva överstepräst uttages bland människor och tillsättes för att till människors bästa göra tjänst inför Gud, genom att frambära gåvor och offer för synder.
Heb 5:2  Och han kan hava undseende med de okunniga och vilsefarande, just därför att han själv är behäftad med svaghet
Heb 5:3  och, för denna sin svaghets skull, måste offra för sina egna synder likaväl som för folkets.
Heb 5:4  Och ingen tager sig själv denna värdighet, utan han måste, såsom Aron, kallas därtill av Gud.
Heb 5:5  Så tog sig icke heller Kristus själv äran att bliva överstepräst, utan den äran tillföll honom genom den som sade till honom: "Du är min Son, jag har i dag fött dig",
Heb 5:6  likasom han ock på ett annat ställe säger: "Du är en präst till evig tid, efter Melkisedeks sätt."
Heb 5:7  Och med starkt rop och tårar frambar han, under sitt kötts dagar, böner och åkallan till den som kunde frälsa honom från döden; och han blev bönhörd och tagen ur sin ångest.
Heb 5:8  Så lärde han, fastän han var "Son", lydnad genom sitt lidande;
Heb 5:9  och när han hade blivit fullkomnad, blev han, för alla dem som äro honom lydiga, upphovet till evig frälsning
Heb 5:10  och hälsades av Gud såsom överstepräst "efter Melkisedeks sätt".
Heb 5:11  Härom hava vi mycket att säga, mycket som är svårt att göra tydligt i ord, eftersom I haven blivit så tröga till att höra.
Heb 5:12  Ty fastän det kunde vara på tiden att I själva voren lärare, behöves det snarare att man nu åter undervisar eder i de allra första grunderna av Guds ord; det har kommit därhän med eder, att I behöven mjölk i stället för stadig mat.
Heb 5:13  Men om någon är sådan att han ännu måste leva av mjölk, då är han oskicklig att förstå en undervisning om rättfärdighet; han är ju ännu ett barn.
Heb 5:14  Ty den stadiga maten tillhör de fullmogna, dem som genom vanan hava sina sinnen övade till att skilja mellan gott och ont.
Heb 6:1  Låtom oss därför lämna bakom oss de första grunderna av läran om Kristus och gå framåt mot det som hör till fullkomningen; låtom oss icke åter lägga grunden med bättring från döda gärningar och med tro på Gud,
Heb 6:2  med undervisning om dop och handpåläggning, om de dödas uppståndelse och en evig dom.
Heb 6:3  Ja, detta vilja vi göra, såframt Gud eljest tillstädjer det.
Heb 6:4  Ty dem till vilka ljuset en gång har kommit, och som hava smakat den himmelska gåvan och blivit delaktiga av helig ande,
Heb 6:5  och som hava fått smaka det goda gudsordet och den tillkommande tidsålderns krafter,
Heb 6:6  men som ändå hava avfallit - dem är det omöjligt att återföra till ny bättring, eftersom de på nytt korsfästa Guds Son åt sig och utsätta honom för bespottelse.
Heb 6:7  Det är ju så, att den jord som indricker regnet, när det titt och ofta strömmar ned däröver, och som framalstrar växter, dem till gagn för vilkas räkning den brukas, den jorden får välsignelse från Gud.
Heb 6:8  Den åter som bär törne och tistel, den är ingenting värd och är förbannelsen nära, och slutet bliver att den avbrännes med eld.
Heb 6:9  Men i fråga om eder, I älskade, äro vi vissa om vad bättre är, och vad som länder till frälsning, om vi ock nu tala på detta sätt.
Heb 6:10  Ty Gud är icke orättvis, så att han förgäter vad I haven verkat, och vilken kärlek I bevisaden mot hans namn, då I tjänaden de heliga, såsom I ännu gören.
Heb 6:11  Men vår åstundan är att var och en av eder visar samma nit att intill änden bevara full visshet i sitt hopp,
Heb 6:12  så att I icke bliven tröga, utan bliven efterföljare åt dem som genom tro och tålamod få till arvedel vad utlovat är.
Heb 6:13  Ty när Gud gav löftet åt Abraham, svor han vid sig själv - eftersom han icke hade någon högre att svärja vid -
Heb 6:14  och sade: "Sannerligen, jag skall rikligen välsigna dig och storligen föröka dig."
Heb 6:15  Och när denne tåligt förbidade, fick han så vad utlovat var.
Heb 6:16  Människor svärja ju vid den som är högre än de, och eden tjänar dem till bekräftelse och gör en ände på all tvist.
Heb 6:17  Därför, när Gud ville för dem som skulle få till arvedel vad löftet innebar ännu kraftigare bevisa oryggligheten av sitt rådslut, lade han därtill en ed.
Heb 6:18  Så skulle vi genom två oryggliga utsagor, i vilka Gud omöjligen kunde ljuga, undfå en kraftig uppmuntran, vi som hava sökt vår räddning i att hålla fast vid det hopp som ligger framför oss.
Heb 6:19  I det hoppet hava vi ett säkert och fast själens ankare, som når innanför förlåten,
Heb 6:20  dit Jesus, såsom vår förelöpare, har gått in för oss, i det han blev en överstepräst "efter Melkisedeks sätt, till evig tid".
Heb 7:1  Denne Melkisedek, som var konung i Salem och präst åt Gud den Högste - han som gick Abraham till mötes, när denne var stadd på återvägen, sedan han hade slagit konungarna, och som välsignade honom,
Heb 7:2  varvid Abraham å sin sida gav honom tionde av allt; denne, som när man uttyder vad han kallas, är först och främst "rättfärdighetens konung", men därjämte ock "Salems konung", det är "fridens konung",
Heb 7:3  denne som står där utan fader, utan moder och utan släktledning, utan begynnelse på sina dagar och utan ände på sitt liv och likställes med Guds Son - denne förbliver en präst för beständigt.
Heb 7:4  Och sen nu huru stor han är, denne åt vilken vår stamfader Abraham gav tionde av det förnämsta bytet.
Heb 7:5  Medan de av Levi söner, som undfå prästämbetet, hava befallning att enligt lagen taga tionde av folket, det är av sina bröder, fastän dessa hava utgått från Abrahams länd,
Heb 7:6  tog denne, som icke var av deras släkt, tionde av Abraham och välsignade honom, densamme som hade fått löftena.
Heb 7:7  Nu lär ingen kunna neka att det plägar vara den ringare som mottager välsignelse av den som står högre.
Heb 7:8  Och medan det här är dödliga människor som taga tionde, är det där en som får det vittnesbördet att han förbliver levande.
Heb 7:9  Genom Abraham har på visst sätt också Levi, som tager tionde, fått giva tionde;
Heb 7:10  ty han var ännu i sin stamfaders länd, när Melkisedek gick denne till mötes.
Heb 7:11  Vore det nu så, att fullkomning kunde vinnas genom det levitiska prästadömet - och på detta var ju folkets lagstiftning byggd - varför hade det då behövts att en präst av annat slag, "efter Melkisedeks sätt", skulle uppstå, en som icke nämnes "efter Arons sätt"?
Heb 7:12  (Om prästadömet förändras, måste ju med nödvändighet också lagen förändras.)
Heb 7:13  Den som detta säges om hörde nämligen till en annan stam, en stam från vilken ingen har utgått, som har gjort tjänst vid altaret.
Heb 7:14  Ty det är en känd sak att han som är vår Herre har trätt fram ur Juda stam; och med avseende på den har Moses icke talat något om präster.
Heb 7:15  Och ännu mycket tydligare blir detta, då nu en präst av annat slag uppstår, lik Melkisedek däri,
Heb 7:16  att han har blivit präst icke på grund av en lag som stadgar härstamning efter köttet, utan på grund av en kraft som kommer av oförgängligt liv.
Heb 7:17  Han får nämligen det vittnesbördet: "Du är en präst till evig tid, efter Melkisedeks sätt."
Heb 7:18  Så upphäves nu visserligen en föregående stadga, därför att den var svag och gagnlös -
Heb 7:19  eftersom lagen icke kunde åstadkomma något fullkomligt - men ett bättre hopp sättes i stället, ett hopp genom vilket vi få nalkas Gud.
Heb 7:20  Och i så måtto som detta icke har kommit till stånd utan edlig bekräftelse - det är nämligen så, att medan de andra hava blivit präster utan edlig bekräftelse,
Heb 7:21  har denne blivit det med sådan bekräftelse, genom den som sade till honom: "Herren har svurit och skall icke ångra sig: 'Du är en präst till evig tid'" -
Heb 7:22  i så måtto är också det förbund bättre, som har Jesus till löftesman.
Heb 7:23  Och medan de förra prästerna hava måst bliva flera, därför att de genom döden hindrades från att förbliva i sin tjänst,
Heb 7:24  har däremot denne ett oförgängligt prästadöme, eftersom han förbliver "till evig tid".
Heb 7:25  Därför kan han ock till fullo frälsa dem som genom honom komma till Gud, ty han lever alltid för att mana gott för dem.
Heb 7:26  En sådan överstepräst hövdes oss också att hava, en som vore helig, oskyldig, obesmittad, skild från syndare och upphöjd över himmelen,
Heb 7:27  en som icke var dag behövde frambära offer, såsom de andra översteprästerna, först för sina egna synder och sedan för folkets; detta gjorde han nämligen en gång för alla, när han offrade sig själv.
Heb 7:28  Ty lagen insätter till överstepräster människor som äro behäftade med svaghet, men det löftesord, som efter lagens utgivande gavs under edlig bekräftelse, insätter en "Son" som är fullkomnad "till evig tid".
Heb 8:1  Men en huvudpunkt i vad vi vilja säga är detta: Vi hava en överstepräst som sitter på högra sidan om Majestätets tron i himmelen,
Heb 8:2  för att göra tjänst i det allraheligaste, i det sannskyldiga tabernaklet, vilket Herren har upprättat, och icke någon människa.
Heb 8:3  Ty var och en som bliver överstepräst tillsättes för att frambära gåvor och offer; därför måste också denne hava något att frambära.
Heb 8:4  Om han nu vore på jorden, så vore han icke ens präst, då andra där finnas, som efter lagens bud hava att frambära gåvorna,
Heb 8:5  i det att de tjäna i den helgedom som är en avbild och en skugga av den himmelska. Om en sådan fick ock Moses befallning genom en uppenbarelse, när han skulle förfärdiga tabernaklet. "Se till", heter det, "att du gör allt efter den mönsterbild som har blivit dig visad på berget."
Heb 8:6  Men nu har denne fått ett så mycket förnämligare ämbete, som han är medlare för ett bättre förbund, vars ordning vilar på bättre löften.
Heb 8:7  Ty om det förra förbundet hade varit utan brist, så skulle väl plats icke hava sökts för ett annat.
Heb 8:8  Men nu förebrår Gud dem och säger: "Se, dagar skola komma, säger Herren, då jag skall sluta ett nytt förbund med Israels hus och med Juda hus;
Heb 8:9  icke ett sådant förbund som det jag gjorde med deras fäder, på den dag då jag tog dem vid handen till att föra dem ut ur Egyptens land. Ty de förblevo icke i mitt förbund, och därför frågade icke heller jag efter dem, säger Herren.
Heb 8:10  Nej, detta är det förbund som jag skall sluta med Israels hus i kommande dagar, säger Herren: Jag skall lägga mina lagar i deras sinnen, och i deras hjärtan skall jag skriva dem, och jag skall vara deras Gud, och de skola vara mitt folk.
Heb 8:11  Då skall den ene medborgaren aldrig behöva undervisa den andre, icke den ene brodern den andre och säga: 'Lär känna Herren'; ty de skola alla känna mig, från den minste bland dem till den störste.
Heb 8:12  Ty jag skall i nåd förlåta deras missgärningar, och deras synder skall jag aldrig mer komma ihåg."
Heb 8:13  När han säger "ett nytt förbund", har han därmed givit till känna att det förra är föråldrat; men det som föråldras och bliver gammalt, det är nära att försvinna.
Heb 9:1  Nu hade visserligen också det förra förbundet sina gudstjänststadgar och sin jordiska helgedom.
Heb 9:2  Ty i tabernaklet inrättades ett främre rum, vari stodo ljusstaken och bordet med skådebröden; och detta rum kallas "det heliga".
Heb 9:3  Men bakom den andra förlåten var ett annat rum i tabernaklet, ett som kallas "det allraheligaste",
Heb 9:4  med ett gyllene rökelsealtare och förbundets ark, på alla sidor överdragen med guld. I denna förvarades ett gyllene ämbar med mannat, så ock Arons stav, som hade grönskat, och därtill förbundets tavlor.
Heb 9:5  Därovanpå stodo härlighetskeruber, som överskyggde nådastolen. Men om vart särskilt av dessa föremål är nu icke tillfälle att tala.
Heb 9:6  Så blev detta inrättat. Och i det främre tabernakelrummet gå prästerna ständigt in och förrätta vad som hör till gudstjänsten,
Heb 9:7  men i det andra går allenast översteprästen in en gång om året, och då aldrig utan blod; och han frambär blodet för sig själv och för folkets ouppsåtliga synder.
Heb 9:8  Därmed giver den helige Ande till känna att vägen till det allraheligaste ännu icke har blivit uppenbarad, så länge det främre tabernakelrummet fortfarande äger bestånd.
Heb 9:9  Ty detta är en sinnebild som avser den nuvarande tiden, och i enlighet härmed frambäras gåvor och offer, vilka dock icke kunna fullkomna, efter samvetets krav, den som förrättar sin gudstjänst,
Heb 9:10  utan allenast äro utvärtes stadgar - de såväl som föreskrifterna om mat och dryck och allahanda tvagningar - stadgar pålagda intill dess tiden vore inne för en bättre ordning.
Heb 9:11  Men Kristus kom såsom överstepräst för det tillkommande goda; och genom det större och fullkomligare tabernakel som icke är gjort med händer, det är, som icke tillhör den skapelse som nu är,
Heb 9:12  gick han, icke med bockars och kalvars blod, utan med sitt eget blod, en gång för alla in i det allraheligaste och vann en evig förlossning.
Heb 9:13  Ty om redan blod av bockar och tjurar och aska av en ko, stänkt på dem som hava blivit orenade, helgar till utvärtes renhet,
Heb 9:14  huru mycket mer skall icke Kristi blod - då han nu genom evig ande har framburit sig själv såsom ett felfritt offer åt Gud - rena våra samveten från döda gärningar till att tjäna den levande Guden!
Heb 9:15  Så är han medlare för ett nytt förbund, på det att de som voro kallade skulle få det utlovade eviga arvet, därigenom att en led döden till förlossning ifrån överträdelserna under det förra förbundet.
Heb 9:16  Ty där ett testamente finnes, där måste det styrkas att den som har gjort testamentet är död.
Heb 9:17  Först genom döden bliver ju ett testamente giltigt, varemot det icke äger gällande kraft, så länge den som har gjort det ännu lever.
Heb 9:18  Därför har icke heller det förra förbundet blivit invigt utan blod.
Heb 9:19  Ty sedan alla buden, såsom de lyda i lagen, hade blivit av Moses kungjorda för allt folket, tog han blod av kalvar och bockar, tillika med vatten och röd ull och isop, och bestänkte såväl själva boken som allt folket
Heb 9:20  och sade: "Detta är förbundets blod, det förbunds som Gud har stadgat för eder."
Heb 9:21  Likaledes stänkte han ock blod på tabernaklet och på alla de ting som hörde till gudstjänsten.
Heb 9:22  Så renas enligt lagen nästan allting med blod, och utan att blod utgjutes gives ingen förlåtelse.
Heb 9:23  Alltså var det nödvändigt att avbilderna av de himmelska tingen renades genom sådana medel; men de himmelska tingen själva måste renas genom bättre offer än dessa.
Heb 9:24  Ty Kristus har icke gått in i ett allraheligaste som är gjort med händer, och som allenast är en efterbildning av det sannskyldiga, utan han har gått in i själva himmelen, för att nu träda fram inför Guds ansikte, oss till godo.
Heb 9:25  Ej heller har han gått ditin, för att många gånger offra sig själv, såsom översteprästen år efter år går in i det allraheligaste, med blod som icke är hans eget.
Heb 9:26  Han hade annars måst lida många gånger allt ifrån världens begynnelse. I stället har han uppenbarats en enda gång, nu vid tidernas ände, för att genom offret av sig själv utplåna synden.
Heb 9:27  Och såsom det är människorna förelagt att en gång dö och sedan dömas,
Heb 9:28  så skall Kristus, sedan han en gång har blivit offrad för att bära mångas synder, för andra gången, utan synd, låta sig ses av dem som bida efter honom, till frälsning.
Heb 10:1  Ty lagen innehåller en skugga av det tillkommande goda, men framställer icke tingen i deras verkliga gestalt; därför kan den aldrig genom de offer som ständigt frambäras, år efter år på samma sätt, fullkomna dem som framträda med sådana.
Heb 10:2  Annars skulle man väl hava upphört att offra, då ju de som så förrättade sin gudstjänst icke mer kunde veta med sig någon synd, sedan de en gång hade blivit renade.
Heb 10:3  Men just i offren ligger en årlig påminnelse om synd.
Heb 10:4  Ty omöjligt är att tjurars och bockars blod skulle kunna borttaga synder.
Heb 10:5  Därför säger han vid sitt inträde i världen: "Slaktoffer och spisoffer begärde du icke, men en kropp beredde du åt mig;
Heb 10:6  i brännoffer och syndoffer fann du icke behag.
Heb 10:7  Då sade jag: 'Se, jag kommer - i bokrullen är skrivet om mig - för att göra din vilja, o Gud.'"
Heb 10:8  Sedan han först har sagt: "Slaktoffer och spisoffer, brännoffer och syndoffer begärde du icke, och i sådana fann du icke behag" - och dock frambäras de efter lagen -
Heb 10:9  säger han vidare: "Se, jag kommer för att göra din vilja." Så tager han bort det förra, för att sätta det andra i stället.
Heb 10:10  Och i kraft av denna "vilja" hava vi blivit helgade, därigenom att Jesu Kristi "kropp" en gång för alla har blivit offrad.
Heb 10:11  Och alla andra präster stå dag efter dag i sin tjänst och frambära gång på gång enahanda offer, som dock aldrig kunna borttaga synder;
Heb 10:12  men sedan denne har framburit ett enda offer för synderna, sitter han för beständigt på Guds högra sida
Heb 10:13  och väntar nu allenast på att "hans fiender skola bliva lagda honom till en fotapall".
Heb 10:14  Ty med ett enda offer har han för beständigt fullkomnat dem som bliva helgade.
Heb 10:15  Härom vittnar jämväl den helige Ande för oss. Ty sedan Herren hade sagt:
Heb 10:16  "Detta är det förbund som jag skall sluta med dem i kommande dagar", säger han: "Jag skall lägga mina lagar i deras hjärtan, och i deras sinnen skall jag skriva dem";
Heb 10:17  och vidare: "Deras synder och deras orättfärdiga gärningar skall jag aldrig mer komma ihåg."
Heb 10:18  Men där förlåtelse för dessa är given, där behöves icke mer något offer för synd.
Heb 10:19  Eftersom vi nu, mina bröder, hava en fast tillförsikt att få gå in i det allraheligaste i och genom Jesu blod,
Heb 10:20  i det att han åt oss har invigt en ny och levande väg ditin genom förlåten - det är genom sitt kött -
Heb 10:21  och eftersom vi hava en stor överstepräst över Guds hus,
Heb 10:22  så låtom oss med uppriktiga hjärtan gå fram i full trosvisshet, bestänkta till våra hjärtan och därigenom renade från ett ont samvete, och till kroppen tvagna med rent vatten.
Heb 10:23  Låtom oss oryggligt hålla fast vid hoppets bekännelse, ty den som har givit oss löftet, han är trofast.
Heb 10:24  Och låtom oss akta på varandra för att uppliva varandra till kärlek och goda gärningar;
Heb 10:25  låtom oss icke övergiva vår församlingsgemenskap, såsom somliga hava för sed, utan må vi förmana varandra - detta så mycket mer som I sen huru "dagen" nalkas.
Heb 10:26  Ty om vi med berått mod synda, sedan vi hava undfått kunskapen om sanningen, så återstår icke mer något offer för våra synder,
Heb 10:27  utan allenast en förskräcklig väntan på dom och glöden av en eld som skall förtära motståndarna.
Heb 10:28  Den som föraktar Moses' lag, han skall "efter två eller tre vittnens utsago" dödas utan barmhärtighet;
Heb 10:29  huru mycket svårare straff tron I icke då att den skall anses värd, som förtrampar Guds Son och aktar förbundets blod för orent - det i vilket han har blivit helgad - och som smädar nådens Ande!
Heb 10:30  Vi veta ju vem han är som sade: "Min är hämnden; jag skall vedergälla det", och åter: "Herren skall döma sitt folk."
Heb 10:31  Det är förskräckligt att falla i den levande Gudens händer.
Heb 10:32  Men kommen ihåg den förgångna tiden, då I, sedan ljuset hade kommit till eder, ståndaktigt uthärdaden mången lidandets kamp
Heb 10:33  och dels själva genom smälek och misshandling bleven gjorda till ett skådespel för världen, dels leden med andra som fingo genomgå sådant.
Heb 10:34  Ty I haven delat de fångnas lidanden och med glädje underkastat eder att bliva berövade edra ägodelar. I vissten nämligen att I haven en egendom som är bättre och bliver beståndande.
Heb 10:35  Så kasten nu icke bort eder frimodighet, som ju har med sig stor lön.
Heb 10:36  I behöven nämligen ståndaktighet för att kunna göra Guds vilja och få vad utlovat är.
Heb 10:37  Ty "ännu en helt liten tid, så kommer den som skall komma, och han skall icke dröja;
Heb 10:38  och min rättfärdige skall leva av tro. Men om någon drager sig undan, så finner min själ icke behag i honom".
Heb 10:39  Dock, vi höra icke till dem som draga sig undan, sig själva till fördärv; vi höra till dem som tro och så vinna sina själar.
Heb 11:1  Men tron är en fast tillförsikt om det som man hoppas, en övertygelse om ting som man icke ser.
Heb 11:2  På grund av den fingo ju de gamle sitt vittnesbörd.
Heb 11:3  Genom tron förstå vi att världen har blivit fullbordad genom ett ord av Gud, så att det man ser icke har blivit till av något synligt.
Heb 11:4  Genom tron frambar Abel åt Gud ett bättre offer än Kain, och genom den fick han det vittnesbördet att han var rättfärdig, i det Gud själv gav vittnesbörd om hans offergåvor; och genom tron talar han ännu, fastän han är död.
Heb 11:5  Genom tron togs Enok bort, för att han icke skulle se döden; och "man såg honom icke mer, ty Gud tog honom bort". Förrän han togs bort, fick han nämligen det vittnesbördet att han hade täckts Gud;
Heb 11:6  men utan tro är det omöjligt att täckas Gud. Ty den som vill komma till Gud måste tro att han är till, och att han lönar dem som söka honom.
Heb 11:7  Genom tron var det som Noa, sedan han hade fått uppenbarelse om något som man ännu icke såg, i from förtröstan byggde en ark för att rädda sitt hus; och genom den blev han världen till dom och fick till arvedel den rättfärdighet som hör tron till.
Heb 11:8  Genom tron var Abraham lydig, när han blev kallad, och han drog så ut till det land som han skulle få till arvedel; han drog ut, fastän han icke visste vart han skulle komma.
Heb 11:9  Genom tron bosatte han sig såsom främling i det utlovade landet, likasom i ett främmande land, och bodde i tält med Isak och Jakob, som voro hans medarvingar till samma löfte.
Heb 11:10  Ty han väntade på "staden med de fasta grundvalarna", vars byggmästare och skapare är Gud.
Heb 11:11  Genom tron fick jämväl Sara, fastän överårig, kraft att bliva stammoder för en avkomma, i det hon höll den för trovärdig, som hade givit löftet.
Heb 11:12  Därför föddes ock av en och samme man, en som var så gott som död, avkomlingar "så talrika som stjärnorna på himmelen och som sanden på havets strand, som man icke kan räkna".
Heb 11:13  I tron dogo alla dessa, innan de ännu hade fått vad utlovat var; de hade allenast sett det i fjärran och hade hälsat det och bekänt sig vara "gäster och främlingar" på jorden.
Heb 11:14  De som så tala giva ju därmed till känna att de söka efter ett fädernesland.
Heb 11:15  Och om de hade menat det land som de hade gått ut ifrån, så hade de haft tillfälle att vända tillbaka dit.
Heb 11:16  Men nu stod deras håg till ett bättre, nämligen det himmelska. Därför blyges icke Gud för att kallas deras Gud; ty han har berett åt dem en stad.
Heb 11:17  Genom tron var det som Abraham frambar Isak såsom offer, när han blev satt på prov; ja, sin ende son frambar han såsom offer, han som hade mottagit löftena,
Heb 11:18  han till vilken det hade blivit sagt: "Genom Isak är det som säd skall uppkallas efter dig."
Heb 11:19  Ty han tänkte på att Gud var mäktig att till och med uppväcka från de döda; från de döda fick han honom ock tillbaka, liknelsevis talat.
Heb 11:20  Genom tron gav jämväl Isak sin välsignelse åt Jakob och Esau för kommande tider.
Heb 11:21  Genom tron välsignade den döende Jakob Josefs båda söner och tillbad, lutad mot ändan av sin stav.
Heb 11:22  Genom tron talade Josef, när han låg för döden, om Israels barns uttåg och gav befallning om vad som skulle göras med hans ben.
Heb 11:23  Genom tron blev Moses vid sin födelse dold av sina föräldrar och hölls av dem gömd i tre månader, eftersom de sågo att "det var ett vackert barn"; och de läto icke förskräcka sig av konungens påbud.
Heb 11:24  Genom tron försmådde Moses, sedan han hade blivit stor, att kallas Faraos dotterson.
Heb 11:25  Han ville hellre utstå lidande med Guds folk än för en kort tid leva i syndig njutning;
Heb 11:26  han höll nämligen Kristi smälek för en större rikedom än Egyptens skatter, ty han hade sin blick riktad på lönen.
Heb 11:27  Genom tron övergav han Egypten, utan att låta förskräcka sig av konungens vrede; ty därigenom att han likasom såg den Osynlige kunde han härda ut.
Heb 11:28  Genom tron har han ock förordnat om påsken och blodbestrykningen, på det att "Fördärvaren", som förgjorde allt förstfött, icke skulle komma vid dem.
Heb 11:29  Genom tron drogo de fram genom Röda havet likasom på torra landet; men när egyptierna försökte gå samma väg, dränktes de.
Heb 11:30  Genom tron föllo Jerikos murar, sedan man i sju dagar hade gått runt omkring dem.
Heb 11:31  Genom tron undgick skökan Rahab att förgås tillsammans med de ohörsamma, eftersom hon hade tagit emot spejarna såsom vänner.
Heb 11:32  Och vad skall jag nu vidare säga? Tiden bleve mig ju för kort, ifall jag skulle förtälja om Gideon, Barak, Simson och Jefta, om David och Samuel och profeterna,
Heb 11:33  om dessa som genom tron besegrade konungariken, övade rättfärdighet, fingo löften uppfyllda, tillstoppade lejons gap,
Heb 11:34  dämpade eldens kraft, undkommo svärdets egg, blevo starka från att hava varit svaga, blevo väldiga i krig och drevo främmande härar på flykten.
Heb 11:35  Kvinnor funnos som fingo igen sina döda genom deras uppståndelse. Andra läto sig läggas på sträckbänk och ville icke taga emot någon befrielse, i hopp om en så mycket bättre uppståndelse.
Heb 11:36  Andra åter underkastade sig begabberi och gisselslag, därtill ock bojor och fängelse;
Heb 11:37  de blevo stenade, marterade, söndersågade, dödade med svärd. De gingo omkring höljda i fårskinn och gethudar, nödställda, misshandlade, plågade,
Heb 11:38  dessa människor som världen icke var värdig att hysa; de irrade omkring i öknar och bergstrakter och levde i hålor och jordkulor.
Heb 11:39  Och fastän alla dessa genom tron hava fått sitt vittnesbörd, undfingo de ändå icke vad utlovat var;
Heb 11:40  ty Gud hade åt oss utsett något bättre, på det att de icke oss förutan skulle bliva fullkomnade.
Heb 12:1  Alltså, då vi nu hava omkring oss en så stor hop av vittnen, må ock vi lägga av allt som är oss till hinder, och särskilt synden, som så hårt omsnärjer oss, och med uthållighet löpa framåt i den tävlingskamp som är oss förelagd.
Heb 12:2  Och må vi därvid se på Jesus, trons hövding och fullkomnare, på honom, som i stället för att taga den glädje som låg framför honom, utstod korsets lidande och aktade smäleken för intet, och som nu sitter på högra sidan om Guds tron.
Heb 12:3  Ja, på honom, som har utstått så mycken gensägelse av syndare, på honom mån I tänka, så att I icke tröttnen och uppgivens i edra själar.
Heb 12:4  Ännu haven I icke stått emot ända till blods, i eder kamp mot synden.
Heb 12:5  Och I haven alldeles förgätit den förmaningens röst som talar med eder, såsom man talar med söner: "Min son, förkasta icke Herrens aga, och giv dig icke över, när du tuktas av honom.
Heb 12:6  Ty den Herren älskar, den agar han, och han straffar med riset var son som han har kär."
Heb 12:7  Det är till eder fostran som I fån utstå lidande; Gud handlar med eder såsom med söner. Ty var finnes den son som icke bliver agad av sin fader?
Heb 12:8  Om I lämnadens utan aga, medan alla andra finge sin del därav, så voren I oäkta söner och icke äkta.
Heb 12:9  Vidare: vi hava haft köttsliga fäder, som agade oss, och vi visade försyn för dem; skola vi då icke mycket mer vara underdåniga andarnas Fader, så att vi få leva?
Heb 12:10  De förra agade oss ju för en kort tid, såsom det syntes dem gott, men denne agar oss för vårt verkliga gagn, för att vi skola få del av hans helighet.
Heb 12:11  Väl synes alla aga för tillfället vara icke till glädje, utan till sorg; men efteråt bär den, för dem som hava blivit fostrade därmed, en fridsfrukt som är rättfärdighet.
Heb 12:12  Alltså: "stärken maktlösa händer och vacklande knän",
Heb 12:13  och "gören räta stigar för edra fötter", så att den fot som haltar icke vrides ur led, utan fastmer bliver botad.
Heb 12:14  Faren efter frid med alla och efter helgelse; ty utan helgelse får ingen se Herren.
Heb 12:15  Och sen till, att ingen går miste om Guds nåd, och att ingen giftig rot skjuter skott och bliver till fördärv, så att menigheten därigenom bliver besmittad;
Heb 12:16  sen till, att ingen är en otuktig människa eller ohelig såsom Esau, han som för en enda maträtt sålde sin förstfödslorätt.
Heb 12:17  I veten ju att han ock sedermera blev avvisad, när han på grund av arvsrätt ville få välsignelsen; han kunde nämligen icke vinna någon ändring, fastän han med tårar sökte därefter.
Heb 12:18  Ty I haven icke kommit till ett berg som man kan taga på, ett som "brann i eld", icke till "töcken och mörker" och storm,
Heb 12:19  icke till "basunljud" och till en "röst" som talade så, att de som hörde den bådo att intet ytterligare skulle talas till dem.
Heb 12:20  Ty de kunde icke härda ut med det påbud som gavs dem: "Också om det är ett djur som kommer vid berget, skall det stenas."
Heb 12:21  Och så förskräcklig var den syn man såg, att Moses sade: "Jag är förskräckt och bävar."
Heb 12:22  Nej, I haven kommit till Sions berg och den levande Gudens stad, det himmelska Jerusalem, och till änglar i mångtusental,
Heb 12:23  till en högtidsskara och församling av förstfödda söner som äro uppskrivna i himmelen, och till en domare som är allas Gud, och till fullkomnade rättfärdigas andar,
Heb 12:24  och till ett nytt förbunds medlare, Jesus, och till ett stänkelseblod som talar bättre än Abels blod.
Heb 12:25  Sen till, att I icke visen ifrån eder honom som talar. Ty om dessa icke kunde undfly, när de visade ifrån sig honom som här på jorden kungjorde Guds vilja, huru mycket mindre skola då vi kunna det, om vi vända oss ifrån honom som kungör sin vilja från himmelen?
Heb 12:26  Hans röst kom då jorden att bäva; men nu har han lovat och sagt: "Ännu en gång skall jag komma icke allenast jorden, utan ock himmelen att bäva."
Heb 12:27  Dessa ord "ännu en gång" giva till känna att de ting, som kunna bäva, skola, eftersom de äro skapade, bliva förvandlade, för att de ting, som icke kunna bäva, skola bliva beståndande.
Heb 12:28  Därför, då vi nu skola undfå ett rike som icke kan bäva, så låtom oss vara tacksamma. På det sättet tjäna vi Gud, honom till behag, med helig fruktan och räddhåga.
Heb 12:29  Ty "vår Gud är en förtärande eld".
Heb 13:1  Förbliven fasta i broderlig kärlek.
Heb 13:2  Förgäten icke att bevisa gästvänlighet; ty genom gästvänlighet hava några, utan att veta det, fått änglar till gäster.
Heb 13:3  Tänken på dem som äro fångna, likasom voren I deras medfångar, och tänken på dem som utstå misshandling, eftersom också I själva haven en kropp.
Heb 13:4  Äktenskapet må hållas i ära bland alla, och äkta säng bevaras obesmittad; ty otuktiga människor och äktenskapsbrytare skall Gud döma.
Heb 13:5  Varen i eder handel och vandel fria ifrån penningbegär; låten eder nöja med vad I haven. Ty han har själv sagt: "Jag skall icke lämna dig eller övergiva dig."
Heb 13:6  Alltså kunna vi dristigt säga: "Herren är min hjälpare, jag skall icke frukta; vad kunna människor göra mig?"
Heb 13:7  Tänken på edra lärare, som hava talat Guds ord till eder; sen huru de slutade sin levnad, och efterföljen deras tro.
Heb 13:8  Jesus Kristus är densamme i går och i dag, så ock i evighet.
Heb 13:9  Låten eder icke vilseföras av allahanda främmande läror. Ty det är gott att bliva styrkt i sitt hjärta genom nåd och icke genom offermåltider; de som hava befattat sig med sådant hava icke haft något gagn därav.
Heb 13:10  Vi hava ett altare, från vilket de som göra tjänst vid tabernaklet icke hava rätt att få något att äta.
Heb 13:11  Det är ju så, att kropparna av de djur, vilkas blod översteprästen bär in i det allraheligaste till försoning för synd, "brännas upp utanför lägret".
Heb 13:12  Därför var det ock utanför stadsporten som Jesus utstod sitt lidande, för att han genom sitt eget blod skulle helga folket.
Heb 13:13  Låtom oss alltså gå ut till honom "utanför lägret" och bära hans smälek.
Heb 13:14  Ty vi hava här ingen varaktig stad, utan söka efter den tillkommande staden.
Heb 13:15  Så låtom oss då genom honom alltid till Gud "frambära ett lovets offer", det är "en frukt ifrån läppar" som prisa hans namn.
Heb 13:16  Men förgäten icke att göra gott och dela med eder, ty sådana offer har Gud behag till.
Heb 13:17  Varen edra lärare hörsamma, och böjen eder för dem; ty de vaka över edra själar, eftersom de skola avlägga räkenskap. Må de då kunna göra det med glädje, och icke med suckan, ty detta vore eder icke nyttigt.
Heb 13:18  Bedjen för oss; ty vi tro oss hava ett gott samvete, eftersom vi söka att i alla stycken föra en god vandel.
Heb 13:19  Och jag uppmanar eder att göra detta, så mycket mer som jag hoppas att därigenom dess snarare bliva återgiven åt eder.
Heb 13:20  Men fridens Gud, som från de döda har återfört vår Herre Jesus, vilken genom ett evigt förbunds blod är den store herden för fåren,
Heb 13:21  han fullkomne eder i allt vad gott är, så att I gören hans vilja; och han verke i oss vad som är välbehagligt inför honom, genom Jesus Kristus. Honom tillhör äran i evigheternas evigheter. Amen.
Heb 13:22  Jag beder eder, mina bröder: tagen icke illa upp dessa förmaningens ord; jag har ju ock skrivit till eder helt kort.
Heb 13:23  Veten att vår broder Timoteus har blivit lösgiven. Om han snart kommer hit, vill jag tillsammans med honom besöka eder.
Heb 13:24  Hälsen alla edra lärare och alla de heliga. De italiska bröderna hälsa eder.
Heb 13:25  Nåd vare med eder alla.


\end{document}