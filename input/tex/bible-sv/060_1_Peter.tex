\begin{document}

\title{1 Petrusbrevet}


\chapter{1}

\par 1 Petrus, Jesu Kristi apostel, hälsar de utvalda främlingar som bo kringspridda i Pontus, Galatien, Kappadocien, provinsen Asien och Bitynien,
\par 2 utvalda enligt Guds, Faderns, försyn, i helgelse i Anden, till lydnad och till bestänkelse med Jesu Kristi blod. Nåd och frid föröke sig hos eder.
\par 3 Lovad vare vår Herres, Jesu Kristi, Gud och Fader, som efter sin stora barmhärtighet har genom Jesu Kristi uppståndelse från de döda fött oss på nytt till ett levande hopp,
\par 4 till ett oförgängligt, obesmittat och ovanskligt arv, som i himmelen är förvarat åt eder,
\par 5 I som med Guds makt bliven genom tro bevarade till en frälsning som är beredd för att uppenbaras i den yttersta tiden.
\par 6 Därför mån I fröjda eder, om I ock nu en liten tid, där så måste ske, liden bedrövelse under allahanda prövningar,
\par 7 för att, om eder tro håller provet - vilket är mycket mer värt än guldet, som förgås, men som dock genom eld bliver beprövat - detta må befinnas lända eder till pris, härlighet och ära vid Jesu Kristi uppenbarelse.
\par 8 Honom älsken I utan att hava sett honom; och fastän I ännu icke sen honom, tron I dock på honom och fröjden eder över honom med outsäglig och härlig glädje,
\par 9 då I nu ären på väg att vinna det som är målet för eder tro, nämligen edra själars frälsning.
\par 10 Angående denna frälsning hava profeter ivrigt forskat och rannsakat, de som profeterade om den nåd som skulle vederfaras eder.
\par 11 De hava rannsakat för att finna vilken och hurudan tid det var som Kristi Ande i dem hänvisade till, när han förebådade de lidanden som skulle vederfaras Kristus, och den härlighet som därefter skulle följa.
\par 12 Och det blev uppenbarat för dem att det icke var sig själva, utan eder, som de tjänade härmed. Om samma ting har nu en förkunnelse kommit till eder genom de män som i helig ande, nedsänd från himmelen, hava för eder predikat evangelium; och i de tingen åstunda jämväl änglar att skåda in.
\par 13 Omgjorden därför edert sinnes länder och varen nyktra; och sätten med full tillit edert hopp till den nåd som bjudes eder i och med Jesu Kristi uppenbarelse.
\par 14 Då I nu haven kommit till lydnad, så följen icke de begärelser som I förut, under eder okunnighets tid, levden i,
\par 15 utan bliven heliga i all eder vandel, såsom han som har kallat eder är helig.
\par 16 Det är ju skrivet: "I skolen vara heliga, ty jag är helig."
\par 17 Och om I såsom Fader åkallen honom som utan anseende till personen dömer var och en efter hans gärningar, så vandren ock i fruktan under denna edert främlingsskaps tid.
\par 18 I veten ju att det icke är med förgängliga ting, med silver eller guld, som I haven blivit "lösköpta" från den vandel I förden i fåfänglighet, efter fädernas sätt,
\par 19 utan med Kristi dyra blod, såsom med blodet av ett felfritt lamm utan fläck.
\par 20 Så var förutsett om honom före världens begynnelse; men först nu i de yttersta tiderna har han blivit uppenbarad för eder skull,
\par 21 I som genom honom tron på Gud, vilken uppväckte honom från de döda och gav honom härlighet, så att eder tro nu ock kan vara ett hopp till Gud.
\par 22 Renen edra själar, i lydnad för sanningen, till oskrymtad broderlig kärlek, och älsken varandra av hjärtat med uthållig kärlek,
\par 23 I som ären födda på nytt, icke av någon förgänglig säd, utan av en oförgänglig: genom Guds levande ord, som förbliver.
\par 24 Ty "allt kött är såsom gräs och all dess härlighet såsom gräsets blomster; gräset torkar bort, och blomstret faller av,
\par 25 men Herrens ord förbliver evinnerligen". Och det är detta ord som har blivit förkunnat för eder såsom ett glatt budskap.

\chapter{2}

\par 1 Så läggen då bort all ondska och allt svek så ock skrymteri och avund och allt förtal.
\par 2 Och då I nu ären nyfödda barn, så längten efter att få den andliga oförfalskade mjölken, på det att I genom den mån växa upp till frälsning,
\par 3 om I annars haven "smakat att Herren är god".
\par 4 Och kommen till honom, den levande stenen, som väl av människor är förkastad, men inför Gud är "utvald och dyrbar";
\par 5 och låten eder själva såsom levande stenar uppbyggas till ett andligt hus, så att I bliven ett "heligt prästerskap", som skall frambära andliga offer, vilka genom Jesus Kristus äro välbehagliga för Gud.
\par 6 Det heter nämligen på ett ställe i skriften: "Se, jag lägger i Sion en utvald, dyrbar hörnsten, och den som tror på den skall icke komma på skam."
\par 7 För eder, I som tron, är stenen alltså dyrbar, men för sådana som icke tro "har den sten som byggningsmännen förkastade blivit en hörnsten",
\par 8 som är "en stötesten och en klippa till fall". Eftersom de icke hörsamma ordet, stöta de sig; så var det ock bestämt om dem.
\par 9 I åter ären "ett utvalt släkte, ett konungsligt prästerskap, ett heligt folk, ett egendomsfolk", för att I skolen förkunna hans härliga gärningar, hans som har kallat eder från mörkret till sitt underbara ljus.
\par 10 I som förut "icke voren ett folk", men nu ären "ett Guds folk", I som "icke haden fått någon barmhärtighet", men nu "haven fått barmhärtighet".
\par 11 Mina älskade, jag förmanar eder såsom "gäster och främlingar" att taga eder till vara för de köttsliga begärelserna, vilka föra krig mot själen.
\par 12 Och fören en god vandel bland hedningarna, på det att dessa, om de i någon sak förtala eder såsom illgärningsmän, nu i stället, när de skåda edra goda gärningar, må för dessas skull prisa Gud på den dag då han söker dem.
\par 13 Varen underdåniga all mänsklig ordning för Herrens skull, vare sig det är konungen, såsom den överste härskaren,
\par 14 eller det är landshövdingarna, som ju äro sända av honom för att straffa dem som göra vad ont är och för att prisa dem som göra vad gott är.
\par 15 Ty så är Guds vilja, att I med goda gärningar skolen stoppa munnen till på oförståndiga och fåkunniga människor.
\par 16 I ären ju fria, dock icke som om I haden friheten för att därmed överskyla ondskan, utan såsom Guds tjänare.
\par 17 Bevisen var man ära, älsken bröderna, "frukten Gud, ären konungen".
\par 18 I tjänare, underordnen eder edra herrar med all fruktan, icke allenast de goda och milda, utan också de obilliga.
\par 19 Ty det är välbehagligt för Gud, om någon, med honom för ögonen, tåligt uthärdar sina vedervärdigheter, när han får lida oförskylt.
\par 20 Ty vad berömligt är däri att I bevisen tålamod, när I för edra synders skull fån uppbära hugg och slag? Men om I bevisen tålamod, när I fån lida för goda gärningars skull, då är detta välbehaglig för Gud.
\par 21 Ty därtill ären I kallade, då ju Kristus själv led för eder och efterlämnade åt eder en förebild, på det att I skullen följa honom och vandra i hans fotspår.
\par 22 "Han hade ingen synd gjort, och intet svek fanns i hans mun.
\par 23 När han blev smädad, smädade han icke igen, och när han led, hotade han icke, utan överlämnade sin sak åt honom som dömer rättvist.
\par 24 Och "våra synder bar han" i sin kropp upp på korsets trä, för att vi skulle dö bort ifrån synderna och leva för rättfärdigheten; och "genom hans sår haven I blivit helade".
\par 25 Ty I "gingen vilse såsom får", men nu haven I vänt om till edra själars herde och vårdare.

\chapter{3}

\par 1 Sammalunda, i hustrur, underordnen eder edra män, för att också de män, som till äventyrs icke hörsamma ordet; må genom sina hustrurs vandel bliva vunna utan ord,
\par 2 när de skåda den rena vandel som I fören i fruktan.
\par 3 Eder prydnad vare icke den utvärtes prydnaden, den som består i hårflätningar och påhängda gyllene smycken eller i eder klädedräkt.
\par 4 Den vare fastmer hjärtats fördolda människa, smyckad med den saktmodiga och stilla andens oförgängliga väsende; ty detta är dyrbart inför Gud.
\par 5 På sådant sätt prydde sig ju ock fordom de heliga kvinnorna, de som satte sitt hopp till Gud och underordnade sig sina män.
\par 6 Så var Sara lydig mot Abraham och kallade honom "herre"; och hennes barn haven I blivit, om I gören vad gott är, och icke låten eder förskräckas av något.
\par 7 Sammalunda skolen I ock, I män, på förståndigt sätt leva tillsammans med edra hustrur, då ju hustrun är det svagare kärlet; och eftersom de äro edra medarvingar till livets nåd, skolen I bevisa dem all ära, på det att edra böner icke må bliva förhindrade.
\par 8 Varen slutligen alla endräktiga, medlidsamma, kärleksfulla mot bröderna, barmhärtiga, ödmjuka.
\par 9 Vedergällen icke ont med ont, icke smädelse med smädelse, utan tvärtom välsignen; därtill ären I ju ock kallade, att I skolen få välsignelse till arvedel.
\par 10 Ty "den som vill älska livet och se goda dagar, han avhålle sin tunga från det som är ont och sina läppar från att tala svek;
\par 11 han vände sig bort ifrån det som är ont, och göre vad gott är, han söke friden, och trakte därefter
\par 12 Ty Herrens ögon äro vända till de rättfärdiga, och hans öron till deras bön. Men Herrens ansikte är emot dem som göra det onda".
\par 13 Och vem är den som kan göra eder något ont, om I nitälsken för det som är gott?
\par 14 Skullen I än få lida för rättfärdighets skull, så ären I dock saliga. "Hysen ingen fruktan för dem, och låten eder icke förskräckas;
\par 15 nej, Herren, Kristus, skolen I hålla helig i edra hjärtan." Och I skolen alltid vara redo att svara var och en som av eder begär skäl för det hopp som är i eder, dock med saktmod och i fruktan
\par 16 och med ett gott samvete, så att de som smäda eder goda vandel i Kristus komma på skam, i fråga om det som de förtala eder för.
\par 17 Ty det är bättre att lida för goda gärningar, om så skulle vara Guds vilja, än att lida för onda.
\par 18 Kristus själv led ju en gång döden för synder; rättfärdig led han för orättfärdiga, på det att han skulle föra oss till Gud. Ja, han blev dödad till köttet, men till anden blev han gjord levande.
\par 19 I anden gick han gick han ock åstad och predikade för de andar som höllos i fängelse,
\par 20 för sådana som fordom voro ohörsamma, när Guds långmodighet gav dem anstånd i Noas tid, då när en ark byggdes, i vilken några få - allenast åtta personer - blevo frälsta genom vatten.
\par 21 Efter denna förebild bliven nu och I frälsta genom vatten - nämligen genom ett dop som icke betyder att man avtvår kroppslig orenhet, utan betyder att man anropar Gud om ett gott samvete - i kraft av Jesu Kristi uppståndelse,
\par 22 hans som har farit upp till himmelen, och som nu sitter på Guds högra sida, sedan änglar och väldige och makter i andevärlden hava blivit honom underlagda.

\chapter{4}

\par 1 Då nu Kristus har lidit till köttet, så väpnen ock I eder med samma sinne; ty den som har lidit till köttet har icke längre något att skaffa med synd.
\par 2 Och leven sedan, under den tid som återstår eder här i köttet, icke mer efter människors onda begärelser, utan efter Guds vilja.
\par 3 Ty det är nog, att I under den framfarna tiden haven gjort hedningarnas vilja och levat i lösaktighet och onda begärelser, i fylleri, vilt leverne och dryckenskap och i allahanda skamlig avgudadyrkan,
\par 4 varför de ock förundra sig och smäda eder, då I nu icke löpen med till samma liderlighetens pöl.
\par 5 Men de skola göra räkenskap inför honom som är redo att döma levande och döda.
\par 6 Ty att evangelium blev förkunnat jämväl för döda, det skedde, för att dessa, om de än till köttet blevo dömda, såsom alla människor dömas, likväl till anden skulle få leva, så som Gud lever
\par 7 Men änden på allting är nu nära. Varen alltså besinningsfulla och nyktra, så att I kunnen bedja.
\par 8 Och varen framför allt uthålliga i eder kärlek till varandra, ty "kärleken överskyler en myckenhet av synder".
\par 9 Varen gästvänliga mot varandra utan knot,
\par 10 och tjänen varandra, var och en med den nådegåva han har undfått, såsom goda förvaltare av Guds mångfaldiga nåd.
\par 11 Om någon talar, så vare hans tal i enlighet med Guds ord, om någon har en tjänst, så sköte han den efter måttet av den kraft som Gud förlänar, så att Gud i allt bliver ärad genom Jesus Kristus. Honom tillhör äran och väldet i evigheternas evigheter, amen.
\par 12 Mina älskade, förundren eder icke över den luttringseld som är tänd bland eder, och som I till eder prövning måsten genomgå, och menen icke att därmed något förunderligt vederfares eder;
\par 13 utan ju mer I fån dela Kristi lidanden, dess mer mån I glädja eder, för att I ock mån kunna glädjas och fröjda eder vid hans härlighets uppenbarelse.
\par 14 Saliga ären I, om I för Kristi namns skull bliven smädade, ty härlighetens Ande, Guds Ande, vilar då över eder.
\par 15 Det må nämligen icke ske, att någon av eder får lida såsom dråpare eller tjuv eller illgärningsman, eller därför att han blandar sig i vad honom icke vidkommer.
\par 16 Men om någon får lida för att han är en kristen, då må han icke blygas, utan prisa Gud för detta namns skull.
\par 17 Ty tiden är inne att domen skall begynna, och det på Guds hus; men om begynnelsen sker med oss, vad bliver då änden för dem som icke hörsamma Guds evangelium?
\par 18 Och om den rättfärdige med knapp nöd bliver frälst, "huru skall det då gå den ogudaktige och syndaren?"
\par 19 Alltså, de som efter Guds vilja få lida, de må anbefalla sina själar åt sin trofaste Skapare, allt under det att de göra vad gott är.

\chapter{5}

\par 1 Till de äldste bland eder ställer jag nu denna förmaning, jag som själv är en av de äldste och en som vittnar om Kristi lidanden, och som jämväl har del i den härlighet som kommer att uppenbaras:
\par 2 Varen herdar för Guds hjord, som I haven i eder vård, varen det icke av tvång, utan av fri vilja, icke för slem vinnings skull, utan med villigt hjärta.
\par 3 Uppträden icke såsom herrar över edra församlingar, utan bliven föredömen för hjorden.
\par 4 Då skolen I, när Överherden uppenbaras, undfå härlighetens oförvissneliga segerkrans.
\par 5 Så skolen I ock, I yngre, å eder sida underordna eder de äldre. Ikläden eder alla, i umgängelsen med varandra, ödmjukheten såsom en tjänardräkt. Ty "Gud står emot de högmodiga, men de ödmjuka giver han nåd".
\par 6 Ödmjuken eder alltså under Guds mäktiga hand, för att han må upphöja eder i sinom tid.
\par 7 Och "kasten alla edra bekymmer på honom", ty han har omsorg om eder.
\par 8 Varen nyktra och vaken. Eder vedersakare, djävulen, går omkring såsom ett rytande lejon och söker vem han må uppsluka.
\par 9 Stån honom emot, fasta i tron, och veten att samma lidanden vederfaras edra bröder här i världen.
\par 10 Men all nåds Gud, som har kallat eder till sin eviga härlighet i Kristus, sedan I en liten tid haven lidit, han skall fullkomna, stödja, styrka och stadfästa eder.
\par 11 Honom tillhör väldet i evigheternas evigheter. Amen.
\par 12 Genom Silvanus, eder trogne broder - för en sådan håller jag honom nämligen - har jag nu i korthet skrivit detta, för att förmana eder, och för att betyga att den nåd I stån i är Guds rätta nåd.
\par 13 Församlingen i Babylon, utvald likasom eder församling, hälsar eder. Så gör ock min son Markus.
\par 14 Hälsen varandra med en kärlekens kyss. Frid vare med eder alla som ären i Kristus.


\end{document}