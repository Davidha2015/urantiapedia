\begin{document}

\title{Levililer}


\chapter{1}

\par 1 RAB Musa'yi çagirip Bulusma Çadiri'ndan ona söyle seslendi:
\par 2 "Israil halkiyla konus, onlara de ki, `Içinizden biri RAB'be sunu olarak bir hayvan sunacagi zaman, sigir ya da davar sunmali.
\par 3 "`Eger yakmalik sunu* sigirsa, kusursuz ve erkek olmali. RAB'bin sunuyu kabul etmesi için onu Bulusma Çadiri'nin giris bölümünde sunmali.
\par 4 Elini yakmalik sununun basina koymali. Sunu kisinin günahlarinin bagislanmasi için kabul edilecektir.
\par 5 Bogayi RAB'bin önünde kesmeli. Harun soyundan gelen kâhinler boganin kanini getirip Bulusma Çadiri'nin giris bölümündeki sunagin her yanina dökecekler.
\par 6 Sonra kisi yakmalik sunuyu yüzüp parçalara ayirmali.
\par 7 Kâhin Harun'un ogullari sunakta ates yakip üzerine odun dizecekler.
\par 8 Hayvanin basini, iç yagini, parçalarini sunakta yanan odunlarin üzerine yerlestirecekler.
\par 9 Kisi hayvanin iskembesini, bagirsaklarini ve ayaklarini yikayacak. Kâhin de hepsini yakmalik sunu, yakilan sunu ve RAB'bi hosnut eden koku olarak sunagin üzerinde yakacaktir.
\par 10 "`Eger kisi yakmalik sunu olarak davar, yani koyun ya da keçi sunmak istiyorsa, sunusu kusursuz ve erkek olmali.
\par 11 Onu sunagin kuzeyinde, RAB'bin önünde kesmeli. Harun soyundan gelen kâhinler kani sunagin her yanina dökecekler.
\par 12 Kisi basini, iç yagini kesip hayvani parçalara ayirmali. Kâhin bunlari sunakta yanan odunlarin üzerine yerlestirecek.
\par 13 Kisi hayvanin iskembesini, bagirsaklarini, ayaklarini yikamali. Kâhin bunlari sunak üzerinde yakarak sunmali. Bu yakmalik sunu, yakilan sunu ve RAB'bi hosnut eden kokudur.
\par 14 "`Eger kisi yakmalik sunu olarak RAB'be kus sunmak istiyorsa, kumru ya da güvercin sunmali.
\par 15 Kâhin sunuyu sunaga getirecek, basini ayirip sunagin üzerinde yakacak. Kusun kani sunagin yan tarafindan akitilacak.
\par 16 Kâhin kusun kursagini pisligiyle birlikte çikarip sunagin dogusundaki küllüge atacak.
\par 17 Kanatlarini tutarak kusu ikiye bölecek, ama tümüyle ayirmayacak. Sonra kusu sunakta yanan odunlarin üstünde yakmali. Bu yakmalik sunu, yakilan sunu ve RAB'bi hosnut eden kokudur."

\chapter{2}

\par 1 "'Biri RAB'be tahil sunusu* getirdigi zaman, sunusu ince undan olmali. Üzerine zeytinyagi dökerek ve günnük koyarak
\par 2 sunuyu Harun soyundan gelen kâhinlere götürmeli. Kâhin avuç dolusu ince un, zeytinyagi ve bütün günnügü alip sunagin üzerinde anma payi olarak yakacak. Bu yakilan sunu ve RAB'bi hosnut eden kokudur.
\par 3 Tahil sunusundan artakalan Harun'la ogullarina birakilmali. RAB için yakilan bir sunu oldugundan çok kutsaldir.
\par 4 "'Eger firinda pisirilmis tahil sunusu sunuyorsan, zeytinyagiyla yogrulmus ince undan yapilmis mayasiz pideler ya da üzerine yag sürülmüs mayasiz yufkalar olmali.
\par 5 Eger sunu sacda pisirilmis tahil sunusu ise, zeytinyagiyla yogrulmus mayasiz ince undan yapilmali.
\par 6 Onu sunarken parçalara ayirip üzerine zeytinyagi dökeceksin. Bu tahil sunusudur.
\par 7 Eger sunu tavada pisirilmis tahil sunusu ise, ince un ve zeytinyagiyla yogrulmus olmali.
\par 8 Böyle yapilmis tahil sunusunu RAB'be sunmak için getirip kâhine vereceksin. Kâhin de onu sunaga götürecek.
\par 9 Anma payi olarak tahil sunusundan bir parça alip yakilan sunu ve RAB'bi hosnut eden koku olarak sunak üzerinde yakacak.
\par 10 Tahil sunusundan artakalan Harun'la ogullarina birakilmali. RAB için yakilan bir sunu oldugundan çok kutsaldir.
\par 11 "'RAB'be sunacaginiz tahil sunularinin hiçbirine maya katilmamali. Çünkü RAB için yakilan sunu içinde hiçbir zaman maya ya da bal yakilmamali.
\par 12 Bunlari ilk ürünlerinizin sunusu olarak RAB'be sunabilirsiniz. Ancak RAB'bi hosnut eden koku olarak sunak üzerinde sunulmamalari gerekir.
\par 13 Bütün tahil sunularini tuzlayacaksiniz. Tanri'nin sizinle yaptigi antlasmayi simgeleyen tuzu tahil sunularindan hiç eksik etmeyeceksiniz. Bütün sunulara tuz katacaksiniz.
\par 14 "'Eger RAB'be ilk ürünlerin tahil sunusunu getiriyorsan, kavrulup dövülmüs, taze devsirilmis bugday basaklari sunacaksin.
\par 15 Üzerine zeytinyagi ve günnük koyacaksin. Tahil sunusudur bu.
\par 16 Kâhin biraz dövülmüs bugday ve zeytinyagi alip günnügün tümüyle birlikte anma payi olarak yakacak. RAB için yakilan sunudur bu."

\chapter{3}

\par 1 "'Eger biri esenlik kurbani olarak sigir sunmak istiyorsa, RAB'be erkek ya da disi, kusursuz bir hayvan sunmali.
\par 2 Elini sununun basina koyup onu Bulusma Çadiri'nin giris bölümünde kesmeli. Harun soyundan gelen kâhinler kani sunagin her yanina dökecekler.
\par 3 Kisi esenlik sunusunun* bazi parçalarini RAB için yakilan sunu olarak sunmali. Sununun bagirsak ve iskembe yaglarini,
\par 4 böbreklerini, böbrek üstü yaglarini, karacigerden böbreklere uzanan perdeyi ayiracak.
\par 5 Harun'un ogullari sunakta yanan odunlarin üzerinde duran yakmalik sununun* üzerinde bunlari yakacak. Yakilan sunu, RAB'bi hosnut eden kokudur.
\par 6 "'Eger kisi esenlik kurbani olarak RAB'be davar sunmak istiyorsa, erkek ya da disi, sunusu kusursuz olmali.
\par 7 Eger kuzu sunmak istiyorsa, RAB'bin önünde sunmali.
\par 8 Elini sununun basina koyup onu Bulusma Çadiri'nin önünde kesmeli. Harun'un ogullari kani sunagin her yanina dökecekler.
\par 9 Kisi esenlik kurbaninin bazi parçalarini RAB için yakilan sunu olarak sunmali. Yagini almali, kuyruk sokumunun dibinden bütün kuyruk yagini kesmeli, bagirsak ve iskembe yaglarini,
\par 10 böbreklerini, böbrek üstü yaglarini, karacigerden böbreklere uzanan perdeyi ayirmali.
\par 11 Kâhin bunlari sunagin üzerinde yakacak. RAB için yakilan yiyecek sunusudur bu.
\par 12 "'Eger sunusu keçi ise, onu RAB'bin önünde sunmali.
\par 13 Elini sununun basina koyup onu Bulusma Çadiri'nin önünde kesmeli. Harun'un ogullari kani sunagin her yanina dökecekler.
\par 14 RAB için yakilan sunu olarak sunudan sunlari ayirip sunmali: Bagirsak ve iskembe yaglarini, böbrekleri, böbrek üstü yaglarini, karacigerden böbreklere uzanan perdeyi.
\par 16 Kâhin bütün bunlari sunagin üzerinde yakacak. Yakilan yiyecek sunusudur bu. Kokusu RAB'bi hosnut eder. Yagin tümü RAB'be aittir.
\par 17 Hayvan yagi ve kan yemeyeceksiniz. Yasadiginiz her yerde kusaklar boyunca bu kural hep geçerli olacak."

\chapter{4}

\par 1 RAB Musa'ya söyle dedi:
\par 2 "Israil halkina söyle: 'Biri buyruklarimdan birinde yasakladigim bir seyi yapar, bilmeden günah islerse;
\par 3 meshedilmis* kâhin günah isleyerek halkini da suçlu kilarsa, isledigi günahtan ötürü RAB'be günah sunusu* olarak kusursuz bir boga sunmali.
\par 4 Bogayi Bulusma Çadiri'nin giris bölümüne, RAB'bin önüne getirip elini onun basina koymali ve RAB'bin huzurunda onu kesmeli.
\par 5 Meshedilmis kâhin boga kaninin birazini Bulusma Çadiri'na götürecek.
\par 6 Parmagini kana batirip En Kutsal Yer'in* perdesi önünde, RAB'bin huzurunda yedi kez serpecek.
\par 7 Sonra çadirda, RAB'bin huzurunda, buhur sunaginin boynuzlarina sürecek. Boganin artakalan kanini çadirin giris bölümündeki yakmalik sunu* sunaginin dibine dökecek.
\par 8 Günah sunusu olarak adanan boganin bütün yagini alacak. Bagirsak ve iskembe yaglarini,
\par 9 böbrekleri, böbrek üstü yaglarini, karacigerden böbreklere uzanan perdeyi,
\par 10 esenlik kurbani olarak sunulan sigirda oldugu gibi ayiracak. Bunlari yakmalik sunu sunagi üzerinde yakacak.
\par 11 Boganin artakalan parçalarini; derisini, etinin tümünü, basini, ayaklarini, iskembesini, bagirsaklarini, gübresini ordugahin disinda küllerin döküldügü temiz bir yere götürecek; küllerin üzerinde odun atesiyle yakacak.
\par 13 "'Eger bütün Israil toplulugu bilmeden günah isler, RAB'bin buyruklarindan birinde yasaklanmis olani yaparsa durum gözden kaçsa bile suçlu sayilir.
\par 14 Isledigi günah açiga çikinca, topluluk günah sunusu olarak bir boga sunmali, onu Bulusma Çadiri'nin önüne getirmeli.
\par 15 RAB'bin huzurunda toplulugun ileri gelenleri ellerini boganin basina koyacak ve boga RAB'bin huzurunda kesilecek.
\par 16 Meshedilmis kâhin boganin kanini Bulusma Çadiri'na götürecek.
\par 17 Parmagini kana batirip RAB'bin huzurunda, perdenin önünde yedi kez serpecek.
\par 18 Sonra çadirda RAB'bin huzurunda bulunan sunagin boynuzlarina sürecek. Boganin artakalan kanini çadirin giris bölümündeki yakmalik sunu sunaginin dibine dökecek.
\par 19 Boganin bütün yagini alip sunagin üzerinde yakacak.
\par 20 Günah sunusu olarak sunulan bogaya yaptiginin aynisini yapacak. Böylece kâhin halkin günahlarini bagislatacak ve halk bagislanacak.
\par 21 Ilk bogayi yaktigi gibi bunu da ordugahin disina çikarip yakacak. Toplulugun günah sunusudur bu.
\par 22 "'Önderlerden biri günah isler, bilmeden Tanrisi RAB'bin buyruklarindan birinde yasak olani yaparsa, suçlu sayilir.
\par 23 Isledigi günah kendisine açiklanirsa, sunu olarak kusursuz bir teke getirmeli.
\par 24 Elini tekenin basina koymali ve yakmalik sunularin kesildigi yerde RAB'bin huzurunda onu kesmeli. Bu bir günah sunusudur.
\par 25 Kâhin günah sunusunun kanina parmagini batirip yakmalik sunu sunaginin boynuzlarina sürecek. Artakalan kani yakmalik sunu sunaginin dibine dökecek.
\par 26 Tekenin bütün yagini esenlik kurbaninin yagi gibi sunak üzerinde yakacak. Kâhin kisinin günahini bagislatacak ve kisi bagislanacak.
\par 27 "'Eger halktan biri RAB'bin buyruklarindan birinde yasak olani yapar, bilmeden günah islerse, suçlu sayilir.
\par 28 Isledigi günah kendisine açiklanirsa, günahindan ötürü sunu olarak kusursuz bir disi keçi getirmeli.
\par 29 Elini günah sunusunun basina koymali ve yakmalik sunularin kesildigi yerde onu kesmeli.
\par 30 Kâhin sununun kanina parmagini batirip yakmalik sunu sunaginin boynuzlarina sürecek. Artakalan kani sunagin dibine dökecek.
\par 31 Kisi keçinin bütün yagini, esenlik kurbaninda oldugu gibi ayiracak. Kâhin RAB'bi hosnut eden koku olarak onu sunakta yakacak, kisinin günahini bagislatacak ve kisi bagislanacak.
\par 32 "'Eger biri günah sunusu olarak bir kuzu getirirse, kuzu disi ve kusursuz olmali.
\par 33 Elini günah sunusunun basina koyacak ve yakmalik sunularin kesildigi yerde onu günah sunusu olarak kesecek.
\par 34 Kâhin sununun kanina parmagini batirip yakmalik sunu sunaginin boynuzlarina sürecek. Artakalan kani sunagin dibine dökecek.
\par 35 Esenlik kurbani kuzusunda oldugu gibi kisi sununun bütün yagini ayirmali. Kâhin RAB için yakilan sunularin üzerinde hepsini sunakta yakacak. Kâhin kisinin günahini bagislatacak ve kisi bagislanacak.

\chapter{5}

\par 1 "'Lanetlenecegini bile bile gördügüne ya da bildigine taniklik etmeyen kisi günah islemis olur ve suçunun cezasini çekecektir.
\par 2 "'Biri bilmeden kirli sayilan herhangi bir seye, yabanil, evcil ya da küçük bir hayvan lesine dokunursa, kirlenmis olur ve suçlu sayilir.
\par 3 "'Biri bilmeden kirli sayilan bir insana ya da insandan kaynaklanan kendisini kirletecek herhangi bir seye dokunursa, ne yaptigini anladigi an suçlu sayilacaktir.
\par 4 "'Biri hangi konuda olursa olsun, kötülük ya da iyilik yapmak için, düsünmeden ve ne yaptigini bilmeden ant içerse, bunu anladigi an suçlu sayilacaktir.
\par 5 "'Kisi bu suçlardan birini isledigi zaman, günahini itiraf etmeli.
\par 6 Günahinin bedeli olarak RAB'be bir suç sunusu* getirmeli. Bu sunu küçükbas hayvanlardan olmali. Disi bir kuzu ya da keçi olabilir. Kâhin kisinin günahini bagislatacaktir.
\par 7 "'Eger kuzu alacak gücü yoksa, suçuna karsilik biri günah sunusu*, öbürü yakmalik sunu* olmak üzere RAB'be iki kumru ya da iki güvercin sunmali.
\par 8 Bunlari kâhine getirmeli. Kâhin önce günah sunusunu sunacak. Kusun boynunu kirmali, ama basini koparmamali.
\par 9 Sununun kanindan birazini sunagin yan yüzüne serpmeli. Artakalan kan sunagin dibine akitilmali. Bu günah sunusudur.
\par 10 Kâhin bundan sonra ikinci kusu yakmalik sunu olarak kurallara göre sunacak. Kâhin kisinin günahini bagislatacak ve kisi bagislanacak.
\par 11 "'Eger iki kumru ya da iki güvercin alacak gücü yoksa, günahina karsilik günah sunusu olarak onda bir efa ince un getirmeli. Üzerine zeytinyagi dökmemeli, günnük de koymamali; çünkü bu günah sunusudur.
\par 12 Onu kâhine vermeli. Kâhin anma payi olarak bir avuç dolusu alip sunakta, RAB için yakilan sunularin üzerinde yakacak. Bu günah sunusudur.
\par 13 Kâhin kisinin günahini bagislatacak ve kisi bagislanacak. Tahil sunusunda* oldugu gibi, artakalan un kâhinin olacaktir."
\par 14 RAB Musa'ya söyle dedi:
\par 15 "Eger biri RAB'be adanmis nesnelere el uzatir, bilmeden günah islerse, suç sunusu* olarak RAB'be küçükbas hayvanlardan kusursuz bir koç getirmeli. Degeri gümüs sekelle, kutsal yerin sekeliyle ölçülmeli.
\par 16 Adanmis nesneler konusunda isledigi günahin karsiligini ödemeli ve beste birini üzerine ekleyip kâhine vermeli. Kâhin suç sunusu olan koçla kisinin günahini bagislatacak ve kisi bagislanacak.
\par 17 "Eger biri günah isler, RAB'bin buyruklarindan birinde yasak olani yaparsa, bilmeden yapsa bile, suç islemis olur; suçunun cezasini çekecektir.
\par 18 Kâhine suç sunusu olarak küçükbas hayvanlardan belli degeri olan kusursuz bir koç getirmeli. Kâhin kisinin bilmeden isledigi günahi bagislatacak ve kisi bagislanacak.
\par 19 Bu suç sunusudur. Kisi gerçekten RAB'be karsi suç islemistir."

\chapter{6}

\par 1 RAB Musa'ya söyle dedi:
\par 2 "Eger biri günah isler, RAB'be ihanet eder, kendisine emanet edilen, rehin birakilan ya da çalinti bir mal konusunda komsusunu aldatir ya da ona haksizlik ederse,
\par 3 kayip bir esya bulup yalan söylerse, yalan yere ant içerse, yani insanlarin isleyebilecegi bu suçlardan birini islerse,
\par 4 günah islemis olur ve suçlu sayilir. Çaldigi ya da haksizlikla ele geçirdigi seyi, kendisine emanet edilen ya da buldugu kayip esyayi,
\par 5 ya da hakkinda yalan yere ant içtigi seyi, üzerine beste birini de ekleyerek, suç sunusunu getirdigi gün sahibine geri vermeli.
\par 6 RAB'be suç sunusu olarak kâhine belli degeri olan kusursuz bir koç getirmeli.
\par 7 Kâhin RAB'bin huzurunda onun günahini bagislatacak; isledigi suç ne olursa olsun kisi bagislanacak."
\par 8 RAB Musa'ya söyle dedi:
\par 9 "Harun'la ogullarina buyruk ver: 'Yakmalik sunu* yasasi sudur: Yakmalik sunu bütün gece, sabaha kadar sunaktaki atesin üzerinde kalacak. Sunagin üzerindeki ates sönmeyecek.
\par 10 Kâhin keten giysisini, donunu giyecek. Sunagin üzerindeki yakmalik sunudan kalan külü toplayip sunagin yanina koyacak.
\par 11 Giysilerini degistirdikten sonra külü ordugahin disinda temiz bir yere götürecek.
\par 12 Sunagin üzerindeki ates sürekli yanacak, hiç sönmeyecek. Kâhin her sabah atese odun atacak, yakmalik sununun parçalarini odunlarin üzerine dizecek, onun üzerinde de esenlik sunularinin* yagini yakacak.
\par 13 Sunagin üzerindeki ates sürekli yanacak, hiç sönmeyecek."
\par 14 "'Tahil sunusu* yasasi sudur: Harun'un ogullari onu sunagin önünde RAB'be sunacaklar.
\par 15 Kâhin üzerindeki günnükle birlikte tahil sunusunun ince unundan ve zeytinyagindan bir avuç alip anma payi ve RAB'bi hosnut eden koku olarak sunakta yakacak.
\par 16 Artakalani Harun'la ogullari yiyecekler. Onu kutsal bir yerde, Bulusma Çadiri'nin avlusunda mayasiz ekmek olarak yemeliler.
\par 17 Mayayla pisirilmemeli. Bunu yakilan sunulardan, kâhinlerin payi olarak verdim. Suç sunusu*, günah sunusu* gibi bu da çok kutsaldir.
\par 18 Harun soyundan gelen her erkek ondan yiyebilir. RAB için yakilan sunularda onlarin kusaklar boyunca sonsuza dek paylari olacak. Sunulara her dokunan kutsal sayilacak."
\par 19 RAB Musa'ya söyle dedi:
\par 20 "Harun kâhin olarak meshedildigi* gün, Harun'la ogullari tahil sunusu olarak RAB'be yarisi sabah, yarisi aksam olmak üzere, onda bir efa ince un sunacaklar. Bu sürekli bir sunu olacak.
\par 21 Zeytinyagiyla iyice yogrulup sacda pisirilecek. Tahil sunusunu getirip RAB'bi hosnut eden koku olarak pismis parçalar halinde sunacaklar.
\par 22 Bunu Harun soyundan gelen meshedilmis kâhin RAB'be sunacak. Sürekli bir kural olacak bu. Sununun tümü yakilacak.
\par 23 Kâhinin sundugu her tahil sunusu tümüyle yakilmali, hiç yenmemeli."
\par 24 RAB Musa'ya söyle dedi:
\par 25 "Harun'la ogullarina de ki, 'Günah sunusu* yasasi sudur: Günah sunusu yakmalik sununun* kesildigi yerde, RAB'bin huzurunda kesilecek. Çok kutsaldir.
\par 26 Hayvani sunan kâhin onu kutsal bir yerde, Bulusma Çadiri'nin giris bölümünde yiyecek.
\par 27 Sununun etine her dokunan kutsal sayilacak. Kani birinin giysisine siçrarsa, giysi kutsal bir yerde yikanmali.
\par 28 Içinde etin haslandigi çömlek kirilmali. Ancak tunç* bir kapta haslanmissa, kap iyice ovulup suyla durulanmali.
\par 29 Kâhinler soyundan gelen her erkek bu sunuyu yiyebilir. Çok kutsaldir.
\par 30 Ama kutsal yerde günah bagislatmak için kani Bulusma Çadiri'na getirilen günah sunusunun eti yenmeyecek, yakilacaktir."

\chapter{7}

\par 1 "'Çok kutsal olan suç sunusunun* yasasi sudur:
\par 2 Suç sunusu yakmalik sununun* kesildigi yerde kesilecek ve kani sunagin her yanina dökülecek.
\par 3 Hayvanin bütün yagi alinacak, kuyruk yagi, bagirsak ve iskembe yaglari, böbrekleri, böbrek üstü yaglari, karacigerden böbreklere uzanan perde ayrilacak.
\par 5 Kâhin bunlarin hepsini sunak üzerinde, RAB için yakilan sunu olarak yakacak. Bu suç sunusudur.
\par 6 Kâhinler soyundan gelen her erkek bu sunuyu yiyebilir. Sunu kutsal bir yerde yenecek, çünkü çok kutsaldir.
\par 7 "'Suç ve günah sunulari* için ayni yasa geçerlidir. Et, sunuyu sunarak günahi bagislatan kâhinindir.
\par 8 Yakmalik sununun derisi de sunuyu sunan kâhinindir.
\par 9 Firinda, tavada ya da sacda pisirilen her tahil sunusu* onu sunan kâhinin olacak.
\par 10 Zeytinyagiyla yogrulmus ya da kuru tahil sunulari da Harunogullari'na aittir. Aralarinda esit olarak bölüsülecektir."
\par 11 "'RAB'be sunulacak esenlik kurbaninin yasasi sudur:
\par 12 Eger adam sunusunu RAB'be sükretmek için sunuyorsa, sunusunun yanisira zeytinyagiyla yogrulmus mayasiz pideler, üzerine zeytinyagi sürülmüs mayasiz yufkalar ve iyice karistirilmis ince undan yagla yogrulmus mayasiz pideler de sunacak.
\par 13 RAB'be sükretmek için, esenlik sunusunu* mayali ekmek pideleriyle birlikte sunacak.
\par 14 Her sunudan birini RAB'be bagis sunusu olarak sunacak ve o sunu esenlik sunusunun kanini sunaga döken kâhinin olacak.
\par 15 RAB'be sükretmek için sunulan esenlik kurbaninin eti, sununun sunuldugu gün yenecek, sabaha birakilmayacak.
\par 16 "'Biri gönülden verilen bir sunu ya da diledigi adagi sunmak istiyorsa, kurbanin eti sununun sunuldugu gün yenecek, artakalirsa ertesi güne birakilabilecek.
\par 17 Ancak üçüncü güne birakilan kurban eti yakilacak.
\par 18 Esenlik kurbaninin eti üçüncü gün yenirse sunu kabul edilmeyecek, geçerli sayilmayacak. Çünkü et kirlenmis sayilir ve her yiyen suçunun cezasini çekecektir.
\par 19 "'Kirli sayilan herhangi bir seye dokunan et yenmemeli, yakilmalidir. Öteki etlere gelince, temiz sayilan bir insan o etlerden yiyebilir.
\par 20 Ama biri kirli sayildigi sürece RAB'be sunulan esenlik kurbaninin etini yerse, halkin arasindan atilacak.
\par 21 Ayrica kirli sayilan herhangi bir seye, insandan kaynaklanan bir kirlilige, kirli bir hayvana ya da kirli ve igrenç bir seye dokunup da RAB'be sunulan esenlik kurbaninin etinden yiyen biri halkin arasindan atilacak." Yag ve Kan Yenmemeli
\par 22 RAB Musa'ya söyle dedi:
\par 23 "Israil halkina de ki, 'Ister sigir, ister koyun ya da keçi yagi olsun, hayvan yagi yemeyeceksiniz.
\par 24 Kendiliginden ölen ya da yabanil hayvanlarin parçaladigi bir hayvanin yagi baska seyler için kullanilabilir, ama hiçbir zaman yenmemeli.
\par 25 Kim yakilan ve RAB'be sunulan hayvanlardan birinin yagini yerse, halkimin arasindan atilacak.
\par 26 Nerede yasarsaniz yasayin, hiçbir kusun ya da hayvanin kanini yemeyeceksiniz.
\par 27 Kan yiyen herkes halkimin arasindan atilacak." Kâhinlerin Payi
\par 28 RAB Musa'ya söyle dedi:
\par 29 "Israil halkina de ki, 'RAB'be esenlik kurbani sunmak isteyen biri, esenlik kurbaninin bir parçasini RAB'be sunmali.
\par 30 RAB için yakilan sunusunu kendi eliyle getirmeli. Hayvanin yagini dösüyle birlikte getirecek ve dös RAB'bin huzurunda sallamalik bir sunu olarak sallanacak.
\par 31 Kâhin yagi sunagin üzerinde yakacak, ama dös Harun'la ogullarinin olacak.
\par 32 Esenlik kurbanlarinizin sag budunu bagis olarak kâhine vereceksiniz.
\par 33 Harunogullari arasinda esenlik sunusunun* kanini ve yagini kim sunuyorsa, sag but onun payi olacak.
\par 34 Israil halkinin sundugu esenlik kurbanlarindan sallamalik dösü ve bagis olarak sunulan budu aldim. Israil halkinin payi olarak bunlari sonsuza dek Kâhin Harun'la ogullarina verdim."
\par 35 Harun'la ogullari kâhin atandiklari gün RAB için yakilan sunulardan paylarina bu düstü.
\par 36 RAB onlari meshettigi* gün Israil halkina buyruk vermisti. Adagin bu parçalari gelecek kusaklar boyunca onlarin payi olacakti.
\par 37 Yakmalik, tahil, suç, günah, atanma sunularinin ve esenlik kurbanlarinin yasasi budur.
\par 38 RAB, bu buyrugu çölde, Sina Dagi'nda Israil halkindan kendisine sunu sunmalarini istedigi gün Musa'ya vermisti.

\chapter{8}

\par 1 RAB Musa'ya söyle dedi:
\par 2 "Harun'la ogullarini, kâhin giysilerini, mesh yagini, günah sunusu* olarak sunulacak bogayi, iki koçu ve mayasiz ekmek sepetini Bulusma Çadiri'nin giris bölümüne getir. Bütün toplulugu da oraya çagir."
\par 4 Musa RAB'bin buyrugunu yerine getirdi. Herkes Bulusma Çadiri'nin önünde toplandi.
\par 5 Musa topluluga, "Simdi RAB'bin buyrugunu yerine getirecegim" dedi.
\par 6 Harun'la ogullarini öne çikarip yikadi.
\par 7 Harun'a mintani giydirdi, beline kusagi bagladi, üzerine kaftani, onun üzerine de efodu* giydirdi. Ustaca dokunmus seridiyle efodu bagladi.
\par 8 Üzerine gögüslügü* takti, gögüslügün içine Urim ile Tummim'i* koydu.
\par 9 Basina sarigi sardi, ön kismina kutsal taci, altin levhayi koydu. Musa her seyi RAB'bin buyurdugu gibi yapti.
\par 10 Sonra mesh yagini aldi, Tanri'nin Konutu'nu ve içindeki her seyi meshederek* kutsal kildi.
\par 11 Yagi yedi kez sunagin üzerine serpti; sunagi, sunagin bütün aletlerini, kazani ve ayakliklarini kutsal kilmak için meshetti.
\par 12 Harun'u kutsal kilmak için basina yag dökerek meshetti.
\par 13 Harun'un ogullarini öne çikardi, onlara mintan giydirdi, bellerine kusak bagladi, baslarina baslik koydu. Musa her seyi RAB'bin buyurdugu gibi yapti.
\par 14 Sonra günah sunusu olarak sunulacak bogayi getirdi. Harun'la ogullari ellerini boganin basina koydular.
\par 15 Musa bogayi kesti. Sunagi pak kilmak için kanini parmagiyla sunagin boynuzlarina çepeçevre sürdü. Artan kani sunagin dibine döktü. Böylece sunagi arindirip kutsal kildi.
\par 16 Musa hayvanin bagirsak ve iskembe yaglarini, karaciger perdesini, böbreklerini ve böbrek yaglarini sunagin üzerinde yakti.
\par 17 Boganin geri kalan kismini da -derisini, etini, gübresini- ordugahin disinda yakti. Musa her seyi RAB'bin buyurdugu gibi yapti.
\par 18 Sonra yakmalik sunu* olarak sunulacak koçu getirdi. Harun'la ogullari ellerini koçun basina koydular.
\par 19 Musa koçu kesti, kanini sunagin her yanina döktü.
\par 20 Koçu parçalara ayirip parçalari, basini ve iç yagini yakti.
\par 21 Bagirsaklarini, iskembesini, ayaklarini yikadi ve koçun tümünü sunagin üzerinde yakti. Bu bir yakmalik sunu, RAB'bi hosnut eden koku, yakilan sunuydu. Musa her seyi RAB'bin buyurdugu gibi yapti.
\par 22 Sonra öteki koçu, atanma sunusu olarak sunulacak koçu getirdi. Harun'la ogullari ellerini koçun basina koydular.
\par 23 Musa koçu kesti. Kanini Harun'un sag kulak memesine, sag elinin ve sag ayaginin bas parmaklarina sürdü.
\par 24 Sonra Harun'un ogullarini öne çikardi. Onlarin da sag kulak memelerine, sag ellerinin ve ayaklarinin bas parmaklarina kan sürdü. Artan kani sunagin her yanina döktü.
\par 25 Hayvanin yagini, kuyruk yagini, bagirsak ve iskembe yaglarini, karaciger perdesini, böbreklerini, böbrek yaglarini ve sag budunu aldi.
\par 26 Sonra RAB'bin huzurunda bulunan mayasiz ekmek sepetinden bir ekmek, yagli pide ve yufka alip hayvanin yaglarinin ve sag budunun üzerine koydu.
\par 27 Hepsini Harun'la ogullarinin eline verdi. Bunlari RAB'bin huzurunda sallamalik sunu olarak salladi.
\par 28 Sonra ellerinden alip sunakta yakmalik sununun üzerinde yakti. Bunlar atanma sunusu, RAB'bi hosnut eden koku ve RAB için yakilan sunuydu.
\par 29 Bundan sonra Musa dösü aldi ve sallamalik sunu olarak RAB'bin huzurunda salladi. Atanma sunusu olarak sunulan koçtan Musa'nin payina düsen buydu. Musa her seyi RAB'bin buyurdugu gibi yapti.
\par 30 Musa mesh yagini ve sunagin üzerindeki kani alip Harun'la ogullarinin ve giysilerinin üzerine serpti. Böylece Harun'u, ogullarini ve giysilerini kutsal kilmis oldu.
\par 31 Sonra Harun'la ogullarina, "Eti Bulusma Çadiri'nin giris bölümünde haslayin" dedi, "'Eti Harun'la ogullari yiyecek diye buyurmustum. Atanma sunularinin bulundugu sepetteki ekmekle birlikte onu orada yiyin.
\par 32 Etten ve ekmekten artani yakin.
\par 33 Atanma günleriniz doluncaya kadar, yedi gün boyunca Bulusma Çadiri'nin giris bölümünden ayrilmayin. Çünkü atanmaniz yedi gün sürecek.
\par 34 Bugün yapilan her seyi günahlarinizi bagislatmak için RAB buyurdu.
\par 35 Yedi gün boyunca gece gündüz Bulusma Çadiri'nin giris bölümünde bekleyecek, RAB'bin buyrugunu yerine getireceksiniz. Öyle ki, ölmeyesiniz. Bana böyle buyruk verildi."
\par 36 Böylece Harun'la ogullari RAB'bin Musa araciligiyla verdigi bütün buyruklari yerine getirdiler.

\chapter{9}

\par 1 Sekizinci gün Musa Harun'la ogullarini ve Israil ileri gelenlerini çagirdi.
\par 2 Harun'a, "Kendin için günah sunusu* olarak kusursuz bir erkek buzagi, yakmalik sunu* olarak da kusursuz bir koç al, RAB'be sun" dedi,
\par 3 "Sonra Israil halkina de ki, 'Günah sunusu olarak bir teke, yakmalik sunu olarak da bir yasinda kusursuz bir buzagi ile bir kuzu alin.
\par 4 RAB'bin huzurunda esenlik sunusu* olarak kurban edilmek üzere bir sigir ve bir koçla birlikte zeytinyagiyla yogrulmus tahil sunusu* getirin. Çünkü RAB bugün size görünecektir."
\par 5 Musa'nin buyurduklari Bulusma Çadiri'nin önüne getirildi. Herkes yaklasip RAB'bin huzurunda toplandi.
\par 6 Musa, "RAB sunlari yapmanizi buyuruyor, o zaman RAB'bin yüceligi size görünecektir" dedi.
\par 7 Sonra Harun'a, "Sunaga git, günah ve yakmalik sunularini sun" dedi, "Hem kendinin, hem de halkin günahlarini bagislat. RAB'bin buyurdugu gibi halkin sunusunu sun, günahlarini bagislat."
\par 8 Böylece Harun sunaga gidip kendisi için günah sunusu olarak sunulacak buzagiyi kesti.
\par 9 Ogullari buzaginin kanini ona getirdiler. Harun parmagini kana batirip sunagin boynuzlarina sürdü. Artan kani sunagin dibine döktü.
\par 10 RAB'bin Musa'ya verdigi buyruga uygun olarak günah sunusunun yagini, böbreklerini, karacigerinin perdesini sunakta yakti.
\par 11 Etiyle derisini ise ordugahin disinda yakti.
\par 12 Sonra yakmalik sunuyu kesti. Ogullari sununun kanini kendisine verdiler. O da kani sunagin her yanina döktü.
\par 13 Sununun bütün parçalarini ve basini Harun'a verdiler. Harun hepsini sunagin üzerinde yakti.
\par 14 Sununun iskembesini, bagirsaklarini, ayaklarini yikayip sunakta yakmalik sununun üzerinde yakti.
\par 15 Bundan sonra Harun halkin sunusunu getirdi. Halkin günahlari için sunulacak tekeyi kesti ve ilk sunu gibi bunu da günah sunusu olarak sundu.
\par 16 Yakmalik sunuyu da kurallara uygun biçimde sundu.
\par 17 Sonra tahil sunusunu getirdi. Bir avuç alip her sabah sunulan yakmalik sunuya ek olarak sunagin üzerinde yakti.
\par 18 Halk için esenlik kurbanlari olarak sunulacak sigirla koçu da kesti. Ogullari sunularin kanini kendisine verdiler. O da kani sunagin her yanina döktü.
\par 19 Sigirla koçun yaglarini, kuyruk yagini, bagirsak ve iskembe yaglarini, böbreklerini ve karacigerlerinin perdesini
\par 20 döslerin üzerine koydular. Harun yaglari sunakta yakti.
\par 21 Musa'nin buyurdugu gibi dösleri ve sag budu sallamalik sunu olarak RAB'bin huzurunda salladi.
\par 22 Harun günah, yakmalik, esenlik sunularini sunduktan sonra ellerini halka dogru uzatarak onlari kutsadi ve asagiya indi.
\par 23 Musa'yla Harun Bulusma Çadiri'na girdiler. Disari çikinca halki kutsadilar. O zaman RAB'bin yüceligi halka göründü.
\par 24 RAB bir ates gönderdi. Ates sunagin üzerindeki yakmalik sunuyu, yaglari yakip küle çevirdi. Bunu gören halkin tümü sevinçle haykirarak yüzüstü yere kapandi.

\chapter{10}

\par 1 Harun'un ogullari Nadav'la Avihu buhurdanlarini alip içlerine ates, atesin üstüne de buhur koydular. RAB'bin buyruklarina aykiri bir ates sundular.
\par 2 RAB bir ates gönderdi. Ates onlari yakip yok etti. RAB'bin huzurunda öldüler.
\par 3 Musa Harun'a söyle dedi: "RAB demisti ki, 'Bana hizmet edenler kutsalligima saygi duyacak Ve halkin tümü beni yüceltecek." Harun hiçbir sey söylemedi.
\par 4 Musa Harun'un amcasi Uzziel'in ogullarini, Misael'le Elsafan'i çagirdi, "Gelin, kardeslerinizi kutsal yerin önünden kaldirip ordugahin disina çikarin" dedi.
\par 5 Geldiler ve Musa'nin buyurdugu gibi cesetleri üzerlerindeki mintanlariyla ordugahin disina çikardilar.
\par 6 Sonra Musa Harun'la ogullari Elazar'la Itamar'a, "Saçlarinizi dagitmayin, giysilerinizi yirtmayin" dedi, "Yoksa ölürsünüz ve RAB bütün topluluga öfkelenir. Ama kardesleriniz, bütün Israil halki RAB'bin atesle yok ettigi bu insanlar için yas tutsun.
\par 7 Bulusma Çadiri'nin giris bölümünden ayrilmayin, yoksa ölürsünüz. Çünkü RAB'bin mesh yagiyla kutsandiniz." Harun'la ogullari Musa'nin dedigine uydular.
\par 8 RAB Harun'a söyle dedi:
\par 9 "Sen ve ogullarin Bulusma Çadiri'na sarap ya da herhangi bir içki içip girmeyin, yoksa ölürsünüz. Kusaklar boyunca bir kural olsun bu.
\par 10 Kutsalla bayagi olani, kirliyle temizi birbirinden ayirt etmelisiniz.
\par 11 RAB'bin Musa araciligiyla Israil halkina bildirdigi bütün kurallari onlara ögretmelisiniz."
\par 12 Musa Harun'a ve sag kalan ogullari Elazar'la Itamar'a söyle dedi: "RAB için yakilan sunulardan artan tahil sunusunu* alin, mayasiz ekmek yapip sunagin yaninda yiyin. Çünkü çok kutsaldir.
\par 13 Onu kutsal bir yerde yemelisiniz. Çünkü RAB için yakilan sunulardan senin ve ogullarinin payidir bu. Bana böyle buyruk verildi.
\par 14 Sallamalik dösle bagis olarak sunulan budu ise ogullarin ve kizlarinla birlikte temiz bir yerde yemelisin. Çünkü bunlar Israil halkinin sundugu esenlik kurbanlarindan senin ve çocuklarinin payi olarak ayrildi. Bagis olarak sunulan butla sallamalik dösü, yakilacak sunu yaglariyla birlikte getirip RAB'bin önünde sallamalik sunu olarak sunacaklar. RAB'bin buyrugu uyarinca bunlar sonsuza dek senin ve çocuklarinin payi olacak."
\par 16 Musa günah sunusu* olarak sunulacak tekeyi sorusturdu, yakilmis oldugunu ögrenince, Harun'un sag kalan ogullari Elazar'la Itamar'a çok öfkelendi, "Neden günah sunusunu kutsal bir yerde yemediniz?" diye sordu, "O çok kutsaldir. Toplulugun suçunu üstlenmesi ve günahlarini bagislatmaniz için RAB onu size vermisti.
\par 18 Tekenin kani kutsal çadira getirilmemis. Buyurdugum gibi tekeyi kesinlikle kutsal yerde yemeniz gerekirdi."
\par 19 Harun, "Halk bugün RAB'be günah sunusu ve yakmalik sunu* sundu" diye yanitladi, "Benim basima ise bunlar geldi. Günah sunusunu bugün yemis olsaydim, RAB bundan hosnut olur muydu?"
\par 20 Musa yaniti uygun buldu.

\chapter{11}

\par 1 RAB Musa'yla Harun'a söyle dedi:
\par 2 "Israil halkina deyin ki, 'Karada yasayan hayvanlardan sunlarin etini yiyebilirsiniz:
\par 3 Çatal ve yarik tirnakli, gevis getiren hayvanlarin tümü.
\par 4 Ancak gevis getiren ve çatal tirnakli olan hayvanlardan etini yememeniz gerekenler sunlardir: Deve gevis getirir, ama çatal tirnakli degildir. Sizin için kirli sayilir.
\par 5 Kaya tavsani* gevis getirir, ama çatal tirnakli degildir. Sizin için kirli sayilir.
\par 6 Tavsan gevis getirir, ama çatal tirnakli degildir. Sizin için kirli sayilir.
\par 7 Domuz çatal ve yarik tirnaklidir, ama gevis getirmez. Sizin için kirli sayilir.
\par 8 Bu hayvanlarin etini yemeyecek, lesine dokunmayacaksiniz, sizin için kirlidir.
\par 9 "'Suda yasayan hayvanlardan sunlarin etini yiyebilirsiniz: Denizde, akarsularda yasayan pullu ve yüzgeçli canlilarin etini yiyebilirsiniz.
\par 10 Denizdeki ve akarsulardaki bütün pulsuz ve yüzgeçsiz canlilar -suda toplu halde yasayanlar ve ötekiler- sizin için igrenç sayilir.
\par 11 Bunlar sizin için igrenç sayilacak. Etlerini yemeyecek, leslerinden tiksineceksiniz.
\par 12 Suda yasayan bütün pulsuz ve yüzgeçsiz canlilar sizin için igrenç sayilacak.
\par 13 "'Tiksindirici kuslarin etini yemeyecek, sunlari igrenç sayacaksiniz: Kartal, kuzu kartali, kara akbaba,
\par 14 çaylak, dogan türleri,
\par 15 bütün karga türleri,
\par 16 baykus, puhu, marti, atmaca türleri,
\par 17 kukumav, karabatak, büyük baykus,
\par 18 peçeli baykus, ishakkusu, akbaba,
\par 19 leylek, balikçil türleri, ibibik, yarasa.
\par 20 "'Dört ayakli ve kanatli böceklerin hepsi sizin için igrençtir.
\par 21 Ama dört ayakli ve kanatli olup ayaklarini siçramak için kullanan bazilarinin etini yiyebilirsiniz.
\par 22 Sunlari yiyeceksiniz: Bütün çekirge türleri, küçük çekirge, circirböcegi, agustosböcegi.
\par 23 Öbür dört ayakli, kanatli böceklerin hepsi sizin için igrenç sayilir.
\par 24 "'Sizi kirletecek seyler sunlardir: Asagidaki hayvanlarin lesine dokunan aksama kadar kirli sayilacaktir.
\par 25 Kim asagidaki hayvanlarin lesini tasirsa giysilerini yikayacak ve aksama kadar kirli sayilacaktir.
\par 26 Çatal tirnakli ama tirnagi yarik olmayan ve gevis getirmeyen her hayvan sizin için kirlidir. Bunlara dokunan da kirlenmis sayilir.
\par 27 Dört ayakli hayvanlardan pençelerini yere basarak yürüyenler sizin için kirlidir. Bunlarin lesine dokunanlar aksama kadar kirli sayilacaktir.
\par 28 Bunlarin lesini tasiyanlar giysilerini yikayacak ve aksama kadar kirli sayilacaktir. Çünkü bu hayvanlar sizin için kirlidir.
\par 29 "'Küçük kara hayvanlari içinde sizin için kirli sayilanlar sunlardir: Gelincik, fare, bütün kertenkele türleri -geko, varan, duvar kertenkelesi, düz keler- bukalemun.
\par 31 Sizin için kirli sayilan küçük kara hayvanlari bunlardir. Bunlarin lesine dokunan aksama kadar kirli sayilacaktir.
\par 32 Bunlardan birinin lesi neyin üzerine düserse onu da kirletir. Ister tahta kap, ister giysi, ister deri, ister çul olsun suya konmalidir. Aksama kadar kirli sayilacak ve aksam temizlenmis olacaktir.
\par 33 Bunlardan biri toprak kabin içine düserse, kabin içindekiler kirli sayilacaktir. Toprak kap kirilmalidir.
\par 34 Toprak kaptaki sulu yiyecek ve her içecek kirli sayilacaktir.
\par 35 Bunlardan birinin lesi neyin üzerine düserse onu da kirletir. Üzerine düstügü ister firin olsun, ister ocak, parçalanmalidir. Çünkü onlar kirlidir ve sizin için kirli sayilacaktir.
\par 36 Ancak kaynak ya da su sarnici temiz sayilacaktir; ama bunlarin lesine dokunan kirli sayilacaktir.
\par 37 Eger bu hayvanlardan birinin lesi ekin tohumunun üzerine düserse, o tohum temiz sayilacaktir.
\par 38 Ama suya konmus tohumun içine düserse, tohum sizin için kirlidir.
\par 39 "'Eti yenen hayvanlardan biri ölürse, lesine dokunan aksama kadar kirli sayilacaktir.
\par 40 Hayvanin lesinden yiyen giysilerini yikayacak ve aksama kadar kirli sayilacaktir. Lesi tasiyan da giysilerini yikayacak ve aksama kadar kirli sayilacaktir.
\par 41 "'Bütün küçük kara hayvanlari igrençtir. Yenmeyecektir.
\par 42 Ister karni üzerinde sürünen, ister dört ayakli ya da çok ayakli canlilar olsun, bunlarin hiçbirini yemeyeceksiniz. Çünkü bunlar igrençtir.
\par 43 Bunlarin hiçbiriyle kendinizi kirletmeyin, igrenç duruma sokmayin, kirli duruma düsmeyin.
\par 44 Tanriniz RAB benim. Kendinizi kutsayin ve kutsal olun. Çünkü ben kutsalim. Küçük kara hayvanlarinin hiçbiriyle kendinizi kirletmeyin.
\par 45 Tanriniz olmak için sizi Misir'dan çikaran RAB benim. Kutsal olun, çünkü ben kutsalim.
\par 46 "'Kirli olani temizden, eti yeneni eti yenmeyenden ayirt edebilmeniz için hayvanlar, kuslar, suda toplu halde yasayan bütün canlilar ve küçük kara hayvanlariyla ilgili yasa budur."

\chapter{12}

\par 1 RAB Musa'ya söyle dedi:
\par 2 "Israil halkina de ki, 'Bir kadin hamile kalip erkek çocuk dogurursa, âdet gördügü günlerde oldugu gibi yedi gün kirli sayilacaktir.
\par 3 Çocuk sekizinci gün sünnet edilmeli.
\par 4 Kadin kanamasindan paklanmak için otuz üç gün bekleyecek. Pak sayilmasi için geçmesi gereken bu günler doluncaya dek kutsal bir seye dokunmayacak, tapinaga girmeyecek.
\par 5 Ancak, kiz çocuk dogurursa, âdet gördügü günler gibi iki hafta kirli sayilacaktir. Kanamasindan paklanmak için altmis alti gün bekleyecektir.
\par 6 "'Erkek ya da kiz çocuk doguran kadinin temiz sayilmasi için geçmesi gereken günler dolunca, yakmalik sunu* olarak bir yasinda bir kuzu, günah sunusu* olarak bir güvercin ya da bir kumru getirip Bulusma Çadiri'nin giris bölümünde kâhine verecektir.
\par 7 Kâhin bunu RAB'bin huzurunda sunacak ve kadini aritacak. Böylece kadin kanamasindan temizlenmis sayilacak. Erkek ya da kiz doguran kadinla ilgili yasa budur.
\par 8 "'Eger kadinin kuzu alacak gücü yoksa, biri yakmalik sunu, öbürü günah sunusu olmak üzere, iki kumru ya da iki güvercin yavrusu getirecek. Kâhin kadini aritacak ve kadin temiz sayilacaktir."

\chapter{13}

\par 1 RAB Musa'ya ve Kâhin Harun'a söyle dedi:
\par 2 "Bedeninde deri hastaligina dönüsebilecek sis, kabuk ya da parlak leke bulunan kisi Harun'a, ya da Harun'un kâhin ogullarindan birine götürülecek.
\par 3 Kâhin derideki yaraya bakacak, yarada kil agarmasi varsa ve yara derine inmisse, kisi deri hastaligina yakalanmis demektir. Hastaya bakan kâhin onu kirli ilan edecektir.
\par 4 Derideki parlak leke beyazsa, derine inmemisse, üzerindeki killar agarmamissa, kâhin hastayi yedi gün kapali bir yerde tutacak.
\par 5 Yedinci gün yaraya bakacak; yara ilerlememis, deri yüzeyine yayilmamissa, hastayi yedi gün daha kapali bir yerde tutacak.
\par 6 Yedinci gün hastaya bir daha bakacak; yara solmus, deri yüzeyine yayilmamissa, hastayi temiz ilan edecek. Yara yalnizca kabuktur. Hasta giysilerini yikayip temiz sayilacaktir.
\par 7 Ancak temiz sayilmak için kâhine muayene olduktan sonra derisindeki kabuk yayilirsa, yine kâhine görünmelidir.
\par 8 Kâhin kisiye bakacak, derisindeki kabuk yayilmissa onu kirli ilan edecek. Kisi deri hastaligina yakalanmis demektir.
\par 9 "Deri hastaligina yakalanan kisi kâhine götürülecek.
\par 10 Kâhin ona bakacak. Derideki sis beyazlasmis, üzerindeki killar agarmissa, siskin yarada kizil et görünüyorsa,
\par 11 bu müzmin bir deri hastaligidir. Kâhin kisiyi kirli ilan edecek, ama kapali yerde tutmayacaktir. Çünkü kisi zaten kirlenmistir.
\par 12 "Eger deri hastaligi yayilip kâhinin görebildigi kadariyla tepeden tirnaga hastanin bütün bedenini kaplamissa,
\par 13 kâhin hastaya bakacak ve bedenini hastalik saran kisiyi temiz ilan edecektir. Yaralar beyazlasmis ve temizdir.
\par 14 Ama kizil et görülüyorsa, kisi kirli sayilacaktir.
\par 15 Kâhin kizil et görürse, kisiyi kirli ilan edecektir. Kizil et kirlidir, deri hastaligidir.
\par 16 Eger kizil et iyilesir, beyazlasirsa, hasta yine kâhine görünmelidir.
\par 17 Kâhin hastaya bakacak, yara beyazlasmissa, yarayi temiz ilan edecek. Kisi temiz sayilacak.
\par 18 "Derideki çiban iyilesmis,
\par 19 ama bir süre sonra çibanin yerinde beyaz bir sis, ya da kirmizimsi-beyaz parlak bir leke olusmussa, kâhine göstermeli.
\par 20 Kâhin hastaya bakacak, görünen yara derine inmis, üzerindeki killar agarmissa, kisiyi kirli ilan edecektir. Çibanda bas gösteren bir deri hastaligidir bu.
\par 21 Ama kâhin hastaya baktiginda yarada beyaz kil yoksa, yara derine inmemis ve yara solmussa, kisiyi yedi gün kapali bir yerde tutacaktir.
\par 22 Eger derideki yara yayiliyorsa, kâhin kisiyi kirli ilan edecektir; çünkü kisi hastaliga yakalanmis demektir.
\par 23 Parlak leke geçmemis ama yayilmamissa, bu çiban kabugudur. Kâhin kisiyi temiz ilan edecektir.
\par 24 "Deride ates yanigi varsa ve et kirmizimsi-beyaz ya da beyaz, parlak bir leke haline gelmisse,
\par 25 kâhin yaraya bakmali. Parlak lekenin üzerindeki killar agarmis, leke derine inmisse, yanikta deri hastaligi var demektir. Kâhin kisiyi kirli ilan edecektir, çünkü kisi deri hastaligina yakalanmistir.
\par 26 Ama parlak lekede kil agarmasi yoksa, leke derine inmemis ve solmussa, kâhin hastayi yedi gün kapali bir yerde tutacaktir.
\par 27 Yedinci gün kisiye yine bakacak, eger leke deriye yayilmissa, onu kirli ilan edecek. Çünkü kisi deri hastaligina yakalanmistir.
\par 28 Eger derideki parlak leke geçmemis, ama yayilmamis ve solmussa, yaniktan dogan bir sisliktir. Kâhin kisiyi temiz ilan edecektir. Çünkü yanik yarasinin kabugudur bu.
\par 29 "Bir erkegin ya da kadinin basinda ya da çenesinde yara varsa,
\par 30 kâhin yaraya bakmali. Yara derine inmisse ve üzerinde ince sari tüyler varsa, hastayi kirli ilan edecektir. Çünkü hasta uyuzdur. Bas ya da çenede görülen bir deri hastaligidir.
\par 31 Ancak kâhin bu tür bir yaraya baktigi zaman, yara derine inmemisse, üzerinde siyah kil yoksa, hastayi yedi gün kapali bir yerde tutacak.
\par 32 Yedinci gün yaraya yine bakacak; uyuz deriye yayilmamissa, üzerinde sari kil yoksa, uyuz derine inmemisse,
\par 33 hasta tiras olacak, ama uyuz olan yerlere dokunmayacaktir. Kâhin hastayi yedi gün daha kapali bir yerde tutacak.
\par 34 Yedinci gün uyuza yine bakacak; uyuz deriye yayilmamissa, derine inmemisse, hastayi temiz ilan edecektir. Hasta giysilerini yikayacak ve temiz sayilacaktir.
\par 35 Ama hasta temiz ilan edildikten sonra uyuz derisine yayilirsa,
\par 36 kâhin ona yeniden bakmali. Uyuz yayilmissa, yarada sari kil olup olmamasina bakilmaksizin kisi kirli sayilacaktir.
\par 37 Ama kâhine göre uyuz ilerlememis, üzerinde siyah kil bitmisse, hastalik geçmis demektir. Kisi temizdir. Kâhin onu temiz ilan edecektir.
\par 38 "Bir erkegin ya da kadinin derisinde beyaz parlak lekeler varsa,
\par 39 kâhin ona bakmali. Derideki lekeler beyaz ve solgunsa, sadece deride çikan kirmizi lekelerdir. Kisi temizdir.
\par 40 "Eger adamin saçi dökülmüsse, sadece keldir. Temiz sayilir.
\par 41 Saçinin önü dökülmüsse alni açilmis demektir. Temiz sayilir.
\par 42 Ama saçi dökülmüs ya da alni açilmis adamin basinda veya alninda kirmizimsi-beyaz yaralar çikmissa, adam deri hastaligina yakalanmis demektir.
\par 43 Kâhin adama bakacak. Saçi dökülmüs bas ya da alindaki sisler deri hastaliginin yol açtigi yaralar gibi kirmizimsi-beyazsa,
\par 44 adam deri hastaligina yakalanmistir. Kirlidir. Basindaki yaradan ötürü kâhin kesinlikle onu kirli ilan edecektir.
\par 45 "Böyle bir hastaliga yakalanan kisinin giysileri yirtik, saçlari daginik olmali; kisi agzini örtüp, 'Kirliyim! Kirliyim! diye bagirmali.
\par 46 Hastaligi devam ettigi sürece kirli sayilacaktir, çünkü kirlenmistir. Halktan uzak, ordugahin disinda yasamalidir." Küfle Ilgili Kurallar
\par 47 "Yün ya da keten bir giyside,
\par 48 yün ya da keten bir kumasta, deri ya da deri esyada küf görülürse,
\par 49 giyside, deride, deri esyada, örgü ipinde, kumasta görülen küf yesilimsi ya da kirmizimsi bir renk almissa, kâhine gösterilmelidir.
\par 50 Kâhin ona bakacak ve yedi gün kapali bir yerde tutacak.
\par 51 Yedinci gün ona yine bakacak. Eger kumastaki, giysideki veya herhangi bir amaçla kullanilan deri esyadaki küf yayilmissa, bu tehlikeli bir küftür. Kirli sayilacaktir.
\par 52 Üzerinde küf bulunan yün ya da keten giysi, kumas ya da her türlü deri esya kâhin tarafindan yakilacaktir. Çünkü önü alinamaz bir küftür ve yakilmalidir.
\par 53 "Kâhin giyside, kumasta ya da herhangi bir deri esyada bulunan küfün yayilmadigini görürse,
\par 54 buyruk verecek ve küflü esya yikanacak. Kâhin onu yedi gün daha kapali bir yerde tutacak.
\par 55 Küflü esya yikandiktan sonra yeniden kâhine gösterilmeli. Küf yayilmasa bile rengi degismemisse, kirli sayilacak. Yakilmalidir. Gerek iç yüzünü, gerekse dis yüzünü küf kemirmistir.
\par 56 "Küflü esya yikandiktan sonra, küfte bir solma varsa, kâhin giysideki, derideki ya da kumastaki küflü kismi yirtacak.
\par 57 Eger kumasta, giyside ya da herhangi bir deri esyada yine küf görülürse, küf yayiliyor demektir; esyayi yakacaksin.
\par 58 Yikanan giysiden, kumastan ya da deri bir esyadan küf geçmisse, yeniden yikanacak. Sonra temiz sayilacaktir."
\par 59 Küfün bulastigi yün ya da keten giysiyi, kumasi veya deri esyayi kirli ya da temiz ilan etmenin yasasi budur.

\chapter{14}

\par 1 RAB Musa'ya söyle dedi:
\par 2 "Deri hastasinin temiz kilinacagi gün su yasa geçerlidir: Hasta kâhine götürülecek.
\par 3 Kâhin hastaya ordugahin disinda bakacak. Hastalik iyilesmisse,
\par 4 pak kilinacak kisi için iki temiz, canli kus, sedir agaci, kirmizi iplik ve mercanköskotu getirilmesini buyuracak.
\par 5 Kâhinin buyruguyla kuslardan biri toprak bir kapta, akarsuyun üzerinde kesilecek.
\par 6 Sonra kâhin canli kusu, sedir agacini, kirmizi ipligi ve mercanköskotunu akarsuyun üzerinde kesilen kusun kanina batiracak
\par 7 ve pak kilinacak kisinin üzerine yedi kez serpecek, onu temiz ilan edip canli kusu kira salacak.
\par 8 Pak kilinacak kisi giysilerini yikayacak, bütün killarini tiras edecek ve yikanacak. Bundan sonra pak sayilacak. Artik ordugaha girebilir, ama yedi gün çadirinin disinda kalmali.
\par 9 Yedinci gün saçini, sakalini, kaslarini, bedenindeki bütün killari tiras edecek. Giysilerini yikayacak, kendisi de yikandiktan sonra temiz sayilacak.
\par 10 "Sekizinci gün kusursuz iki erkek kuzu, bir yasinda kusursuz bir disi kuzu, tahil sunusu* olarak zeytinyagiyla yogrulmus onda üç efa ince un ve bir log zeytinyagi getirecek.
\par 11 Paklama isiyle görevli kâhin bütün bunlari ve pak kilacagi kisiyi RAB'bin huzurunda, Bulusma Çadiri'nin giris bölümünde bekletecek.
\par 12 Sonra kâhin erkek kuzulardan birini alip bir log*fg* zeytinyagiyla birlikte RAB'be suç sunusu* olarak sunacak. Sallamalik sunu olarak bunlari RAB'bin huzurunda sallayacak.
\par 13 Erkek kuzuyu günah sunusunun* ve yakmalik sununun* kesildigi kutsal yerde kesecek. Çünkü günah sunusu gibi suç sunusu da kâhine aittir. Çok kutsaldir.
\par 14 Kâhin pak kilinacak kisinin sag kulak memesine, sag elinin ve sag ayaginin bas parmagina suç sunusunun kanindan sürecek.
\par 15 Sonra bir log*fg* zeytinyagindan biraz alarak kendi sol avucuna dökecek.
\par 16 Sag elinin parmagini zeytinyagina batirip RAB'bin huzurunda yedi kez serpecek.
\par 17 Pak kilinacak kisinin sag kulak memesine, sag elinin ve sag ayaginin bas parmagina, suç sunusunun kani üzerine avucunda kalan yagdan sürecek.
\par 18 Ayrica pak kilinacak kisinin basina avucunda kalan yagi sürerek RAB'bin huzurunda onu aritacak.
\par 19 Sonra günah sunusunu sunarak pak kilinacak kisiyi kirliliginden aritacak ve yakmalik sunuyu kesecek.
\par 20 Yakmalik sunuyla tahil sunusunu sunakta sunacak. Böylece kâhin kisiyi aritacak ve kisi temiz sayilacaktir.
\par 21 "Eger kisi yoksulsa ve bunlari alacak gücü yoksa, arinmak üzere sallamalik suç sunusu olarak bir erkek kuzu, tahil sunusu olarak zeytinyagiyla yogrulmus onda bir efa*fh* ince un ve bir log zeytinyagi alacak.
\par 22 Gücü oraninda biri günah sunusu, öbürü yakmalik sunu olmak üzere iki kumru ya da iki güvercin sunacak.
\par 23 Pak kilinmak için sekizinci gün hepsini RAB'bin huzuruna, Bulusma Çadiri'nin giris bölümüne getirip kâhine verecek.
\par 24 Kâhin suç sunusu olan erkek kuzuyla bir log zeytinyagini alip sallamalik sunu olarak RAB'bin huzurunda sallayacak.
\par 25 Suç sunusu olan kuzuyu kesecek. Sununun kanini pak kilinan kisinin sag kulak memesine, sag elinin ve sag ayaginin bas parmagina sürecek.
\par 26 Sonra kendi sol avucuna biraz zeytinyagi dökecek.
\par 27 Sag parmagini sol avucundaki yaga batirarak RAB'bin huzurunda yedi kez serpecek.
\par 28 Avucundaki yagdan pak kilinacak kisinin sag kulak memesine, sag elinin ve sag ayaginin bas parmagina, suç sunusunun kanini sürdügü yerlere sürecek.
\par 29 RAB'bin huzurunda pak kilinacak kisiyi aritmak üzere avucunda kalan yagi basina sürecek.
\par 30 Sonra kisinin gücü oraninda aldigi kumrulardan ya da güvercinlerden birini günah sunusu, öbürünü yakmalik sunu olarak tahil sunusuyla birlikte sunacak. Kâhin böylece pak kilinan kisiyi RAB'bin huzurunda aritacak."
\par 32 Deri hastasi olup da temiz kilinmaya parasal gücü yetmeyen kisiler için bu yasa geçerlidir.
\par 33 RAB Musa'yla Harun'a söyle dedi:
\par 34 "Size mülk olarak verecegim Kenan ülkesine gittiginiz zaman, ülkenizdeki bir eve küf hastaligi gönderirsem,
\par 35 ev sahibi gidip kâhine, 'Evimde küfe benzer bir hastalik gördüm diye haber vermeli.
\par 36 Kâhin küfe bakmaya gitmeden önce, evdeki her seyin kirli sayilmamasi için evin bosaltilmasini buyuracak. Sonra evi görmeye gidecek.
\par 37 Duvarlara yayilan küfe bakacak. Eger küf yesilimsi ya da kirmizimsi lekeler halindeyse ve duvarin içine islemisse,
\par 38 kâhin evi terk edecek ve yedi gün süreyle kapali tutacak.
\par 39 Yedinci gün geri dönecek ve eve yine bakacak. Eger küf duvarlara yayilmissa,
\par 40 küflü taslari söküp kentin disina, kirli sayilan bir yere atmalari için buyruk verecek.
\par 41 Evin içindeki bütün sivayi kazdiracak. Moloz kentin disina, kirli sayilan bir yere dökülecek.
\par 42 Sökülen taslarin yerine baska taslar koyup evi yeniden sivayacaklar.
\par 43 "Taslari sökülüp sivasi kazinan evde yeni siva yapildiktan sonra küf yine ortaya çikarsa,
\par 44 kâhin gidip eve bakacak. Küf yayilmissa, önü alinamaz demektir. Ev kirli sayilir.
\par 45 Yikilmalidir. Taslari, keresteleri, bütün harci kent disina, kirli sayilan bir yere atilmalidir.
\par 46 Evin kapali oldugu günlerde, eve giren biri aksama kadar kirli sayilacak.
\par 47 O evde yatan ya da yemek yiyen biri giysilerini yikamali.
\par 48 "Ev sivandiktan sonra kâhin eve girip bakacak, küf yayilmamissa evi temiz ilan edecek. Çünkü küf geçmis demektir.
\par 49 Evi paklamak için iki kus, sedir agaci, kirmizi iplik ve mercanköskotu alacak.
\par 50 Kuslardan birini toprak bir kapta, akarsuyun üzerinde kesecek.
\par 51 Sedir agacini, mercanköskotunu, kirmizi ipligi, canli kusu alip kesilen kusun kanina ve akarsuya batiracak. Yedi kez eve serpecek.
\par 52 Böylece kusun kani, akarsu, canli kus, sedir agaci, mercanköskotu ve kirmizi iplikle evi paklamis olacak.
\par 53 Sonra canli kusu kent disina, kira salacak. Böylece evin kirliligini bagislatacak ve ev temiz sayilacaktir."
\par 54 Her türlü deri hastaligi, uyuz, giysiye ya da eve bulasan küf, sis, kabuk ya da parlak lekelerle ilgili yasa budur.
\par 57 Bunlarin ne zaman kirli, ne zaman temiz oldugu bu yasaya göre bilinebilir. Deri hastaligi yasasi budur.

\chapter{15}

\par 1 RAB Musa'yla Harun'a söyle dedi:
\par 2 "Israil halkina deyin ki, 'Adamin erkeklik organinda akinti varsa, akinti kirlidir.
\par 3 Akinti ister devam etsin, ister kesilsin adami kirletir. Akintinin neden oldugu kirlilikler sunlardir:
\par 4 Üzerinde yattigi her yatak ve oturdugu her sey kirli sayilacaktir.
\par 5 Kim yatagina dokunursa, giysilerini yikayacak, yikanacak, aksama kadar kirli sayilacaktir.
\par 6 Adamin üzerine oturdugu bir esyaya oturan da giysilerini yikayacak, yikanacak, aksama kadar kirli sayilacaktir.
\par 7 Kim akintisi olan adamin bedenine dokunursa giysilerini yikayacak, yikanacak, aksama kadar kirli sayilacaktir.
\par 8 Eger akintisi olan adam temiz bir adama tükürürse, o kisi giysilerini yikayacak, yikanacak, aksama kadar kirli sayilacaktir.
\par 9 Akintisi olan adamin bindigi her eyer kirli sayilacaktir.
\par 10 Adamin üzerine oturdugu ya da yattigi herhangi bir esyaya dokunan, aksama kadar kirli sayilacaktir. Bu esyalari tasiyan herkes giysilerini yikayacak, yikanacak, aksama kadar kirli sayilacaktir.
\par 11 Akintisi olan adam ellerini yikamadan kime dokunursa o kisi giysilerini yikayacak, yikanacak, aksama kadar kirli sayilacaktir.
\par 12 Akintisi olan adamin dokundugu toprak kap parçalanacak, tahta kap ise suyla çalkalanacaktir.
\par 13 "'Eger adamin akintisi kesilirse, paklanmak için yedi gün bekleyecek. Sonra giysilerini yikayacak, akarsuda yikanacak ve temiz sayilacak.
\par 14 Sekizinci gün iki kumru ya da iki güvercin alip RAB'bin huzuruna, Bulusma Çadiri'nin giris bölümüne gelecek ve bunlari kâhine verecek.
\par 15 Kâhin birini günah sunusu*, ötekini yakmalik sunu* olarak sunacak. Böylece akintisi olan adami RAB'bin huzurunda aritacak.
\par 16 "'Eger bir adamdan meni akarsa, bedeninin tümünü yikayacak ve aksama kadar kirli sayilacaktir.
\par 17 Üzerine meni bulasan her giysi ya da deri esya yikanacak, aksama kadar kirli sayilacaktir.
\par 18 Bir adam kadinla cinsel iliskide bulunurken menisi akarsa, ikisi de yikanacak ve aksama kadar kirli sayilacaklardir.
\par 19 "'Âdet gördügü için kan kaybeden kadin yedi gün kirli sayilacak. Ona dokunan da aksama kadar kirli sayilacak.
\par 20 Âdet gördügü günlerde kadinin üzerinde yattigi ya da oturdugu her sey kirli sayilacaktir.
\par 21 Kim kadinin yatagina dokunursa, giysilerini yikayacak, yikanacak, aksama kadar kirli sayilacaktir.
\par 22 Kim kadinin üzerine oturdugu herhangi bir seye dokunursa, o da giysilerini yikayacak, yikanacak, aksama kadar kirli sayilacaktir.
\par 23 Kadinin yatagindaki ya da oturdugu seyin üzerindeki herhangi bir esyaya dokunan herkes aksama kadar kirli sayilacaktir.
\par 24 Âdet gören kadinin kirliligi onunla yatan adama da bulasir. Adam yedi gün kirli kalir ve yattigi her yatak kirli sayilir.
\par 25 "'Eger bir kadinin âdet günleri disinda uzun süreli bir kanamasi varsa, ya da kanamasi âdet günlerinden sonra da devam ediyorsa, kanamasi oldugu sürece âdet günlerinde oldugu gibi kirli sayilir.
\par 26 Kanamasi oldugu sürece, âdet günlerinde oldugu gibi, yattigi her yatak ve üzerine oturdugu her sey kirli sayilacaktir.
\par 27 Kim bunlara dokunursa kirli sayilacak. Giysilerini yikayacak, yikanacak, aksama kadar kirli kalacaktir.
\par 28 "'Ama kanama durursa, kadin yedi gün bekleyecek, sonra temiz sayilacaktir.
\par 29 Sekizinci gün iki kumru ya da iki güvercin alip Bulusma Çadiri'nin giris bölümüne getirecek ve bunlari kâhine verecek.
\par 30 Kâhin birini günah sunusu, ötekini yakmalik sunu olarak sunacak. Böylece kadini kanamasindan dogan kirlilikten RAB'bin huzurunda aritacak.
\par 31 "'Israil halkini kirliliginden arindiracaksin. Öyle ki, aralarinda bulunan konutumu kirletip kirlilik içinde ölmesinler."
\par 32 Akintisi olan, bosalarak kirlenen adam, âdet gören kadin, akintisi olan erkek ya da kadin ve kirli sayilan kadinla yatan erkekle ilgili yasa budur.

\chapter{16}

\par 1 RAB'bin huzuruna yaklastiklari için ölen Harun'un iki oglunun ölümünden sonra RAB Musa'ya söyle dedi: "Agabeyin Harun'a de ki, perdenin arkasindaki En Kutsal Yer'e* ikide bir girmesin, Antlasma Sandigi'nin* üzerindeki Bagislanma Kapagi'na yaklasmasin. Yoksa ölür. Çünkü ben kapagin üstünde, bulut içinde görünüyorum.
\par 3 Harun En Kutsal Yer'e ancak günah sunusu* olarak bir boga, yakmalik sunu* olarak da bir koç sunarak girebilir.
\par 4 Kutsal keten mintan, keten don giyecek, keten kusak baglayacak, keten sarik saracak. Bunlar kutsal giysilerdir. Bunlari giymeden önce yikanacak.
\par 5 Israil toplulugu günah sunusu olarak Harun'a iki teke, yakmalik sunu olarak bir koç verecek.
\par 6 "Harun bogayi kendisi için günah sunusu olarak sunacak. Böylece kendisinin ve ailesinin günahlarini bagislatacak.
\par 7 Sonra iki tekeyi alip RAB'bin huzuruna, Bulusma Çadiri'nin giris bölümüne götürecek.
\par 8 Ikisi üzerine kura çekecek. Biri RAB için, biri Azazel için.
\par 9 Harun kurada RAB'be düsen tekeyi getirip günah sunusu olarak sunacak.
\par 10 Azazel'e düsen tekeyi ise halkin günahlarini bagislatmak için canli olarak RAB'be sunacak. Onu çöle salip Azazel'e gönderecek.
\par 11 "Harun kendisi için günah sunusu olarak bogayi getirecek. Böylece kendisinin ve ailesinin günahlarini bagislatacak. Bu günah sunusunu kendisi için kesecek.
\par 12 RAB'bin huzurunda bulunan sunagin üzerindeki korlari buhurdana koyup iki avuç dolusu ince ögütülmüs güzel kokulu buhurla perdenin arkasina geçecek.
\par 13 Orada, RAB'bin huzurunda buhuru korlarin üzerine koyacak; buhurun dumani Levha Sandigi'nin üzerindeki Bagislanma Kapagi'ni kaplayacak. Öyle ki, Harun ölmesin.
\par 14 Sonra boganin kanini alip parmagiyla kapagin üzerine, doguya dogru serpecek. Kapagin önünde yedi kez bunu yineleyecek.
\par 15 "Bundan sonra, halk için günah sunusu olarak tekeyi kesecek. Kanini perdenin arkasina götürecek. Boganin kaniyla yaptigi gibi tekenin kanini da Bagislanma Kapagi'nin üzerine ve önüne serpecek.
\par 16 Böylece En Kutsal Yer'i Israil halkinin kirliliklerinden, isyanlarindan, bütün günahlarindan arindiracak. Bulusma Çadiri için de ayni seyi yapacak. Çünkü kirli insanlarin arasinda bulunuyor.
\par 17 Harun kendisi, ailesi ve bütün Israil toplulugunun günahlarini bagislatmak için En Kutsal Yer'e girdiginde, disari çikincaya kadar Bulusma Çadiri'nda hiç kimse bulunmayacak.
\par 18 Harun RAB'bin huzurunda bulunan sunaga çikip sunagi arindiracak, boganin ve tekenin kanini sunagin boynuzlarina çepeçevre sürecek.
\par 19 Kani parmagiyla yedi kez sunaga serpecek. Böylece sunagi Israil halkinin kirliliginden arindirip kutsal kilacak.
\par 20 "Harun En Kutsal Yer'i, Bulusma Çadiri'ni, sunagi arindirdiktan sonra, canli tekeyi sunacak.
\par 21 Iki elini tekenin basina koyacak, Israil halkinin bütün suçlarini, isyanlarini, günahlarini açiklayarak bunlari tekenin basina aktaracak. Sonra bu is için atanan bir adamla tekeyi çöle gönderecek.
\par 22 Teke Israil halkinin bütün suçlarini yüklenerek issiz bir ülkeye tasiyacak. Adam tekeyi çöle salacak.
\par 23 "Sonra Harun Bulusma Çadiri'na girecek. En Kutsal Yer'e girerken giydigi keten giysileri çikarip orada birakacak.
\par 24 Kutsal bir yerde yikanip kendi giysilerini giyecek. Sonra çikip kendisi ve halk için getirilen yakmalik sunulari sunacak, kendisinin ve halkin günahlarini bagislatacak.
\par 25 Günah sunusunun yagini sunakta yakacak.
\par 26 "Tekeyi Azazel'e gönderen adam giysilerini yikayip kendisi de yikandiktan sonra ordugaha girecek.
\par 27 Günah sunusu olarak sunulan ve kanlari günahlari bagislatmak için En Kutsal Yer'e getirilen boga ile teke ordugahin disina çikarilacak. Derileri, etleri, gübreleri yakilacak.
\par 28 Bunlari yakan kisi giysilerini yikayip kendisi de yikandiktan sonra ordugaha girecek.
\par 29 "Asagidakiler sizin için sürekli bir yasa olacak: Yedinci ayin* onuncu günü isteklerinizi denetleyeceksiniz. Gerek Israilliler'den, gerekse aranizda yasayan yabancilardan hiç kimse çalismayacak.
\par 30 Çünkü o gün, Kâhin Harun sizi pak kilmak için günahlarinizi bagislatacaktir. RAB'bin huzurunda bütün günahlarinizdan arinacaksiniz.
\par 31 O gün Sabat'tir*, sizin için dinlenme günüdür. Isteklerinizi denetleyeceksiniz. Bu sürekli bir yasadir.
\par 32 Babasinin meshedip* kendi yerine atadigi kâhin günahlari bagislatacak. Kutsal keten giysileri giyecek.
\par 33 En Kutsal Yer'i, Bulusma Çadiri'ni, sunagi arindiracak; kâhinlerin ve bütün toplulugun günahlarini bagislatacak.
\par 34 "Bu sizin için sürekli bir yasadir: Israil halkinin bütün günahlarini yilda bir kez bagislatmak için verildi." Ve Harun RAB'bin Musa'ya buyurdugu gibi yapti.

\chapter{17}

\par 1 RAB Musa'ya söyle dedi:
\par 2 "Harun'la ogullarina ve bütün Israil halkina de ki, 'RAB'bin buyrugu sudur:
\par 3 Israilliler'den kim ordugahin içinde ya da disinda bir sigir, bir kuzu ya da keçi kurban eder
\par 4 ve onu Bulusma Çadiri'nin giris bölümüne, RAB'bin Konutu'nun önüne, RAB'be sunmak üzere getirmezse, kan dökmüs sayilacak ve halkin arasindan atilacaktir.
\par 5 Öyle ki, Israilliler açik kirlarda kestikleri kurbanlari RAB'bin huzuruna, Bulusma Çadiri'nin giris bölümüne, kâhine getirsinler ve esenlik sunusu* olarak RAB'be kurban etsinler.
\par 6 Kâhin sununun kanini Bulusma Çadiri'nin giris bölümünde RAB'bin sunagi üzerine dökecek, yagini da RAB'bi hosnut eden koku olarak yakacak.
\par 7 Israil halki taptigi teke ilahlara artik kurban kesmeyecek. Bu yasa kusaklar boyunca geçerli olacak.
\par 8 "Onlara de ki, 'Israil halkindan ya da aralarinda yasayan yabancilardan kim yakmalik sunu* veya kurban sunar da,
\par 9 onu RAB'be sunmak için Bulusma Çadiri'nin giris bölümüne getirmezse, halkin arasindan atilacaktir.
\par 10 "'Israil halkindan ya da aralarinda yasayan yabancilardan kim kan yerse, ona öfkeyle bakacagim ve halkimin arasindan atacagim.
\par 11 Çünkü canlilara yasam veren kandir. Ben onu size sunakta kendinizi günahtan bagislatmaniz için verdim. Kan yasam karsiligi günah bagislatir.
\par 12 Bundan dolayi Israil halkina, Sizlerden ya da aranizda yasayan yabancilardan hiç kimse kan yemeyecek, dedim.
\par 13 "'Israil halkindan ya da aralarinda yasayan yabancilardan kim eti yenen bir hayvan veya kus avlarsa, kanini akitip toprakla örtecektir.
\par 14 Çünkü canlilara yasam veren kandir. Bundan dolayi Israil halkina, Hiçbir etin kanini yemeyeceksiniz, dedim. Çünkü her canliya yasam veren kandir. Onu yiyen halkin arasindan atilacaktir.
\par 15 "'Yerli olsun, yabanci olsun ölü buldugu ya da yabanil hayvanlarin parçaladigi bir hayvanin lesini yiyen herkes giysilerini yikayacak, kendisi de yikanacak, aksama kadar kirli sayilacaktir. Ancak bundan sonra temiz sayilacaktir.
\par 16 Eger giysilerini yikamaz ve yikanmazsa suçunun cezasini çekecektir."

\chapter{18}

\par 1 RAB Musa'ya söyle dedi:
\par 2 "Israil halkina de ki, 'Tanriniz RAB benim.
\par 3 Misir'da bir süre yasadiniz; onlarin törelerine göre yasamayacaksiniz. Sizleri Kenan ülkesine götürüyorum. Onlar gibi de yasamayacaksiniz. Onlarin kurallarina uymayacaksiniz.
\par 4 Benim kurallarimi yerine getirecek, ilkelerime göre yasayacaksiniz. Tanriniz RAB benim.
\par 5 Kurallarima, ilkelerime sarilin. Çünkü onlari yerine getiren onlar sayesinde yasayacaktir. RAB benim.
\par 6 "'Hiçbiriniz cinsel iliskide bulunmak için yakin akrabasina yaklasmayacak. RAB benim.
\par 7 Annenle cinsel iliskide bulunarak babanin namusuna dokunmayacaksin. O senin annendir. Onunla iliski kurmayacaksin.
\par 8 Babanin karisiyla cinsel iliski kurmayacaksin. Babanin namusudur o.
\par 9 Annenden ya da babandan olan, ister seninle ayni evde dogmus olsun, ister olmasin üvey kizkardeslerinden biriyle cinsel iliski kurmayacaksin.
\par 10 Kizinin ya da oglunun kiziyla cinsel iliski kurmayacaksin. Çünkü onlarin namusu senin namusundur.
\par 11 Babanin evlendigi kadindan dogan kizla cinsel iliski kurmayacaksin. Çünkü o babandan olmadir, senin kizkardesin sayilir.
\par 12 Halanla cinsel iliski kurmayacaksin. Çünkü o babanin yakin akrabasidir.
\par 13 Teyzenle cinsel iliski kurmayacaksin. Çünkü o annenin yakin akrabasidir.
\par 14 Amcanin namusuna dokunmayacaksin. Karisina yaklasmayacaksin, çünkü o senin yengendir.
\par 15 Gelininle cinsel iliski kurmayacaksin. Çünkü oglunun karisidir. Onunla iliski kurmayacaksin.
\par 16 Kardesinin karisiyla cinsel iliski kurmayacaksin. Çünkü o kardesinin namusudur.
\par 17 Bir kadinin hem kendisiyle, hem kiziyla cinsel iliski kurmayacaksin. Kadinin kizinin ya da oglunun kiziyla cinsel iliski kurmayacaksin. Çünkü onlar kadinin yakin akrabasidir. Onlara yaklasmak alçakliktir.
\par 18 Karin yasadigi sürece onun kizkardesini kuma olarak almayacak ve onunla cinsel iliski kurmayacaksin.
\par 19 "'Âdet gördügü için kirli sayilan bir kadinla cinsel iliski kurmayacaksin.
\par 20 Komsunun karisiyla cinsel iliski kurarak kendini kirletmeyeceksin.
\par 21 Ilah Molek'e* ateste kurban edilmek üzere çocuklarindan hiçbirini vermeyeceksin. Tanrin'in adina leke getirmeyeceksin. RAB benim.
\par 22 Kadinla yatar gibi bir erkekle yatma. Bu igrençtir.
\par 23 Bir hayvanla cinsel iliski kurmayacaksin. Kendini kirletmis olursun. Kadinlar cinsel iliski kurmak amaciyla bir hayvana yaklasmayacak. Sapikliktir bu.
\par 24 "Bu davranislarin hiçbiriyle kendinizi kirletmeyin. Çünkü önünüzden kovacagim uluslar böyle kirlendiler.
\par 25 Onlarin yüzünden ülke bile kirlendi. Günahindan ötürü ülkeyi cezalandirdim. Ülke, üzerinde yasayan halki kusuyor.
\par 26 Ister yerli olsun, ister aranizda yasayan yabancilar olsun kurallarima ve ilkelerime göre yasayacaksiniz. Bu igrençliklerin hiçbirini yapmayacaksiniz.
\par 27 Sizden önce bu ülkede yasayan insanlar bütün bu igrençlikleri yaparak ülkeyi kirlettiler.
\par 28 Eger siz de ülkeyi kirletirseniz, ülke sizden önceki uluslara yaptigi gibi sizi de kusar.
\par 29 "'Kim bu igrençliklerden birini yaparsa halkin arasindan atilacaktir.
\par 30 Buyruklarimi yerine getirin, sizden önceki insanlarin igrenç törelerine uyarak kendinizi kirletmeyin. Tanriniz RAB benim."

\chapter{19}

\par 1 RAB Musa'ya söyle dedi:
\par 2 "Israil topluluguna de ki, 'Kutsal olun, çünkü ben Tanriniz RAB kutsalim.
\par 3 "'Herkes annesine babasina saygi göstersin. Sabat* günlerimi tutun. Tanriniz RAB benim.
\par 4 "'Putlara tapmayin. Kendinize dökme ilahlar yapmayin. Tanriniz RAB benim.
\par 5 "'RAB için esenlik kurbani keseceginiz zaman kabul edilecek biçimde kesin.
\par 6 Kurban eti, kestiginiz gün ya da ertesi gün yenecek. Üçüncü güne kalan et yakilacak.
\par 7 Üçüncü gün yenirse igrenç sayilir. Kabul olmayacaktir.
\par 8 Onu yiyen suçunun cezasini çekecektir. Çünkü RAB'bin gözünde kutsal olani bayagilastirmistir. Halkin arasindan atilacaktir.
\par 9 "'Ülkenizdeki ekinleri biçerken tarlanizi sinirlarina kadar biçmeyeceksiniz. Artakalan basaklari toplamayacaksiniz.
\par 10 Bagbozumunda baginizi tümüyle devsirmeyecek, yere düsen üzümleri toplamayacaksiniz. Onlari yoksullara ve yabancilara birakacaksiniz. Tanriniz RAB benim.
\par 11 "'Çalmayacaksiniz. Hile yapmayacaksiniz. Birbirinize yalan söylemeyeceksiniz.
\par 12 Benim adimla yalan yere ant içmeyeceksiniz. Tanriniz'in adini asagilamis olursunuz. RAB benim.
\par 13 "'Komsuna haksizlik etmeyecek, onu soymayacaksin. Isçinin alacagini sabaha birakmayacaksin.
\par 14 Sagira lanet etmeyecek, körün önüne engel koymayacaksin. Tanrin'dan korkacaksin. RAB benim.
\par 15 "'Yargilarken haksizlik yapmayacaksin. Yoksula ayricalik göstermeyecek, güçlüyü kayirmayacaksin. Komsunu adaletle yargilayacaksin.
\par 16 Halkinin arasinda onu bunu çekistirerek dolasmayacaksin. Komsunun canina zarar vermeyeceksin. RAB benim.
\par 17 "'Kardesine yüreginde nefret beslemeyeceksin. Komsun günah islerse onu uyaracaksin. Yoksa sen de günah islemis olursun.
\par 18 Öç almayacaksin. Halkindan birine kin beslemeyeceksin. Komsunu kendin gibi seveceksin. RAB benim.
\par 19 "'Kurallarimi uygulayin. Farkli cinsten iki hayvani çiftlestirme. Tarlana iki çesit tohum ekme. Üzerine iki tür iplikle dokunmus giysi giyme.
\par 20 "'Bir adam bir cariyeyle yatarsa, eger kadin nisanli, bedeli ödenmemis ya da azat edilmemisse, ikisi de cezalandirilacak ama öldürülmeyecek. Çünkü kadin özgür degildir.
\par 21 Adam RAB'be, Bulusma Çadiri'nin giris bölümüne, suç sunusu* olarak bir koç getirecek.
\par 22 Kâhin bu koçla adamin isledigi günahi RAB'bin önünde bagislatacak ve adam bagislanacak.
\par 23 "'Kenan ülkesine girdiginizde bir meyve agaci dikerseniz, ilk üç yil meyvesini kirli ve yasak sayin, yemeyin.
\par 24 Dördüncü yil agacin bütün meyvesi sükran sunusu olarak RAB için kutsal sayilacak.
\par 25 Besinci yil agacin meyvesini yiyebilirsiniz. Böylece agaç daha bol ürün verir. Tanriniz RAB benim.
\par 26 "'Kanli et yemeyeceksiniz. Kehanette bulunmayacak, falcilik yapmayacaksiniz.
\par 27 Basinizin yan tarafindaki saçlari kesmeyecek, sakalinizin kenarlarina dokunmayacaksiniz.
\par 28 Ölüler için bedeninizi yaralamayacak, dövme yaptirmayacaksiniz. RAB benim.
\par 29 "'Kizini fahiselige sürükleyip rezil etme. Yoksa fahiselik yayilir ve ülke ahlaksizlikla dolup tasar.
\par 30 Sabat günlerimi tutacaksiniz. Tapinagima saygi göstereceksiniz. RAB benim.
\par 31 "'Cincilere, ruh çagiranlara yönelmeyin. Onlara danismayin, kirlenirsiniz. Tanriniz RAB benim.
\par 32 "'Ak saçli insanlarin önünde ayaga kalkacak, yaslilara saygi göstereceksin. Tanrin'dan korkacaksin. RAB benim.
\par 33 "'Ülkenizde sizinle birlikte yasayan bir yabanciya kötü davranmayin.
\par 34 Ona sizden biriymis gibi davranacak ve onu kendiniz kadar seveceksiniz. Çünkü siz de Misir'da yabanciydiniz. Tanriniz RAB benim.
\par 35 "'Yargilarken, uzunluk ve sivi ölçerken, agirlik tartarken haksizlik yapmayin.
\par 36 Dogru terazi, agirlik tasi, efa ve hin kullanin. Misir'dan sizi çikaran Tanriniz RAB benim.
\par 37 Kurallarimin, ilkelerimin tümüne uyacak ve onlari yerine getireceksiniz. RAB benim."

\chapter{20}

\par 1 RAB Musa'ya söyle dedi:
\par 2 "Israil halkina de ki, 'Israilliler'den ya da aranizda yasayan yabancilardan kim çocuklarindan birini ilah Molek'e* sunarsa, kesinlikle öldürülecek. Ülke halki onu taslayacak.
\par 3 Kim çocugunu Molek'e sunarak tapinagimi kirletir, kutsal adima leke sürerse, ona öfkeyle bakacagim. Onu halkimin arasindan atacagim.
\par 4 Adam çocugunu Molek'e sunar da, ülke halki bunu görmezden gelir, onu öldürmezse,
\par 5 adama ve ailesine öfkeyle bakacagim. Hem onu, hem de bana ihanet edip onu izleyerek Molek'e tapanlarin hepsini halkimin arasindan atacagim.
\par 6 "'Kim cincilere, ruh çagiranlara danisir, bana ihanet ederse, ona öfkeyle bakacak, halkimin arasindan atacagim.
\par 7 Kendinizi kutsayin, kutsal olun. Tanriniz RAB benim.
\par 8 Kurallarima uyacak, onlari yerine getireceksiniz. Sizi kutsal kilan RAB benim.
\par 9 "'Annesine ya da babasina lanet eden herkes kesinlikle öldürülecektir. Annesine ya da babasina lanet ettigi için ölümü hak etmistir.
\par 10 "'Biri baska birinin karisiyla, yani komsusunun karisiyla zina ederse, hem kendisi, hem de zina ettigi kadin kesinlikle öldürülecektir.
\par 11 Babasinin karisiyla yatan, babasinin namusuna leke sürmüs olur. Ikisi de kesinlikle öldürülecektir. Ölümü hak etmislerdir.
\par 12 Bir adam geliniyle yatarsa, ikisi de kesinlikle öldürülecektir. Rezillik etmisler, ölümü hak etmislerdir.
\par 13 Bir erkek baska bir erkekle cinsel iliski kurarsa, ikisi de igrençlik etmis olur. Kesinlikle öldürülecekler. Ölümü hak etmislerdir.
\par 14 Bir adam hem bir kizla, hem de kizin annesiyle evlenirse, alçaklik etmis olur. Aranizda böyle alçakliklar olmasin diye üçü de yakilacaktir.
\par 15 Bir hayvanla cinsel iliski kuran adam kesinlikle öldürülecek, hayvansa kesilecektir.
\par 16 Bir kadin cinsel iliski kurmak amaciyla bir hayvana yaklasirsa, kadini da hayvani da kesinlikle öldüreceksiniz. Ölümü hak etmislerdir.
\par 17 "'Bir adam anne ya da baba tarafindan üvey olan kizkardesiyle evlenir, cinsel iliski kurarsa, utançtir. Açikça asagilanip halkin arasindan atilacaklardir. Adam kizkardesiyle iliski kurdugu için suçunun cezasini çekecektir.
\par 18 Âdet gören bir kadinla yatip cinsel iliski kuran adam kadinin akintili yerini açiga çikarmis, kadin da buna katilmis olur. Ikisi de halkin arasindan atilacaktir.
\par 19 "'Teyzenle ya da halanla cinsel iliski kurmayacaksin. Çünkü yakin akrabanin namusudur. Ikiniz de suçunuzun cezasini çekeceksiniz.
\par 20 "'Amcasinin karisiyla cinsel iliski kuran adam, amcasinin namusuna leke sürmüs olur. Ikisi de günahlarinin cezasini çekecek ve çocuk sahibi olmadan öleceklerdir.
\par 21 Kardesinin karisiyla evlenen adam rezillik etmis olur. Kardesinin namusunu lekelemistir. Çocuk sahibi olmayacaklardir.
\par 22 "'Bütün kurallarima, ilkelerime uyacak, onlari yerine getireceksiniz. Öyle ki, yasamak üzere sizi götürecegim ülke sizi disari kusmasin.
\par 23 Önünüzden kovacagim uluslarin törelerine göre yasamayacaksiniz. Çünkü onlar bütün bu kötülükleri yaptilar. Bu yüzden onlardan nefret ettim.
\par 24 Oysa, Siz onlarin topraklarini sahipleneceksiniz. Bal ve süt akan bu ülkeyi size mülk olarak verecegim, dedim. Sizi öteki uluslardan ayri tutan Tanriniz RAB benim.
\par 25 "'Temiz hayvanlarla kuslari kirli olanlardan ayirt edeceksiniz. Sizin için kirli ilan ettigim hayvanlarla, kuslarla, küçük kara hayvanlariyla kendinizi kirletmeyeceksiniz.
\par 26 Benim için kutsal olacaksiniz. Çünkü ben RAB kutsalim. Bana ait olmaniz için sizi öbür halklardan ayri tuttum.
\par 27 "'Cincilik yapan ve ruh çagiran ister erkek olsun, ister kadin olsun kesinlikle öldürülecektir. Onlari taslayacaksiniz. Ölümlerinden kendileri sorumludur."

\chapter{21}

\par 1 RAB Musa'ya söyle dedi: "Harun soyundan gelen kâhinlere de ki, 'Kâhinlerden hiçbiri yakin akrabasi olan annesi, babasi, oglu, kizi ve kardesi disinda, halkindan birinin ölüsüyle kendini kirletmesin.
\par 3 Yaninda kalan evlenmemis kizkardesi için kendini kirletebilir.
\par 4 Halki arasinda bir büyük olarak kendini kirletmemeli, adina leke getirmemeli.
\par 5 "'Kâhinler yas tutarken baslarini tiras etmeyecek, sakallarinin uçlarini kesmeyecek, bedenlerini yaralamayacaklar.
\par 6 Tanrilari için kutsal olacaklar, Tanrilari'nin adini lekelemeyecekler. Çünkü onlar Tanrilari RAB'be yakilan sunu ve yiyecek sunusu sunuyorlar. Kutsal olmalari gerekir.
\par 7 Kâhinler fahiselerle, kirletilmis kadinlarla, bosanmis kadinlarla evlenmeyecek. Çünkü kâhin Tanri için kutsal olmalidir.
\par 8 Onu kutsal sayin. Çünkü yiyecek sunusunu Tanriniz'a o sunuyor. Sizin için kutsaldir. Çünkü ben kutsalim, sizi kutsal kilan RAB benim.
\par 9 Bir kâhinin kizi fahiselik yaparak kendini kirletirse, hem kendini hem de babasini rezil etmis olur. Yakilmalidir.
\par 10 "'Öbür kâhinler arasindan basina mesh yagi dökülen ve özel giysiler giymek üzere atanan baskâhin, saçlarini dagitmayacak, giysilerini yirtmayacak.
\par 11 Hiçbir ölüye yaklasmayacak. Ölen annesi, babasi bile olsa kendini kirletmeyecek.
\par 12 Tapinak hizmetinden ayrilmayacak, Tanrisi'nin Tapinagi'ni kirletmeyecek. Çünkü Tanri'nin buyurdugu mesh yagiyla Tanrisi'na adanmistir. RAB benim.
\par 13 Baskâhinin evlenecegi kadin bakire olmalidir.
\par 14 Dul, bosanmis, kirletilmis ya da fahise bir kadinla evlenmeyecek. Yalniz kendi halkindan bakire bir kizla evlenebilir.
\par 15 Böylece halkinin arasinda çocuklarina leke sürmemis olur. Onu kutsal kilan RAB benim."
\par 16 RAB Musa'ya söyle dedi:
\par 17 "Harun'a de ki, 'Soyundan gelecek kusaklar boyunca kusurlu olan hiç kimse yiyecek sunusu sunmak üzere Tanrisi'na yaklasmasin.
\par 18 Kusurlu olan, sunaga yaklasamaz: Kör, topal, yüzü arizali, organlarindan biri asiri büyümüs,
\par 19 kolu veya ayagi kirik,
\par 20 kambur, cüce, gözü özürlü, uyuz, yarasi kabuk baglamis ya da hadim.
\par 21 Kâhin Harun'un soyundan bu kusurlara sahip hiç kimse RAB için yakilan sunuyu sunmak üzere sunaga yaklasmayacak. Çünkü kusurludur. Tanrisi'na yiyecek sunusu sunmak üzere sunaga yaklasamaz.
\par 22 Böyle bir adam Tanrisi'na sunulan kutsal ve en kutsal yiyecekleri yiyebilir.
\par 23 Ancak perdeye ve sunaga yaklasmayacaktir. Çünkü kusurludur. Tapinagimi kirletmesin. Onlari kutsal kilan RAB benim."
\par 24 Musa Harun'la ogullarina ve bütün Israil halkina bunlari anlatti.

\chapter{22}

\par 1 RAB Musa'ya söyle dedi:
\par 2 "Harun'a ve ogullarina de ki, 'Israil halkinin bana sundugu kutsal sunulardan uzak dursunlar. Kutsal adima leke sürmesinler. RAB benim.
\par 3 Gelecek kusaklar boyunca soyunuzdan biri Israil halkinin bana sundugu kutsal sunulara kirli olarak yaklasirsa, onu huzurumdan atacagim. RAB benim.
\par 4 "'Harun soyundan deri hastaligina yakalanmis ya da akintisi olan biri temiz sayilincaya kadar kutsal sunulari yemeyecektir. Bir cesede degdigi için kirli sayilan bir seye dokunan, menisi akan,
\par 5 insani kirli kilacak küçük kara hayvanlarindan birine ya da herhangi bir nedenle kirli sayilan bir insana dokunan,
\par 6 aksama kadar kirli sayilacak, yikanmadigi sürece kutsal sunulardan yemeyecektir.
\par 7 Günes battiktan sonra temiz sayilir, kutsal sunulari yiyebilir. Çünkü bu onun yiyecegidir.
\par 8 Ölü bulunmus ya da yabanil hayvanlar tarafindan parçalanmis bir lesi yiyerek kendini kirletmeyecektir. RAB benim.
\par 9 "'Kâhinler buyruklarima uymalidir. Yoksa günahlarinin cezasini çeker, buyruklarimi çignedikleri için ölürler. Onlari kutsal kilan RAB benim.
\par 10 "'Kâhin ailesi disinda hiç kimse kutsal sunuyu yemeyecek; kâhinin konugu ve isçisi bile.
\par 11 Ama kâhinin parayla satin aldigi ya da evinde dogan köle onun yemegini yiyebilir.
\par 12 Kâhinin kâhin olmayan bir erkekle evlenen kizi bagislanan kutsal sunulari yemeyecek.
\par 13 Ama dul kalmis ya da bosanmis, çocugu olmamis ve gençliginde kaldigi baba evine geri dönmüs kâhin kizi babasinin ekmegini yiyebilir. Aile disindan yabanci biri asla yiyemez.
\par 14 "'Bilmeden kutsal sunuyu yiyen biri, beste birini üzerine katarak kâhine geri verecek.
\par 15 Kâhinler Israil halkinin RAB'be sundugu kutsal sunulari bayagilastirmayacaklar.
\par 16 Yoksa kutsal sunulari yiyen öbür insanlara büyük suç yüklemis olurlar. Çünkü sunulari kutsal kilan RAB benim." Kabul Edilmeyecek Kurbanlar
\par 17 RAB Musa'ya söyle dedi:
\par 18 "Harun'la ogullarina ve bütün Israil halkina de ki, 'Ister Israilli olsun, ister Israil'de yasayan bir yabanci olsun, biriniz RAB'be dilek adagi ya da gönülden verilen sunu olarak yakmalik sunu* sunmak istiyorsa,
\par 19 sunusunun kabul edilmesi için kusursuz bir erkek sigir, koyun ya da keçi sunmali.
\par 20 Kusurlu olani sunmayacaksiniz. Çünkü kabul edilmeyecektir.
\par 21 Kim gönülden verilen bir sunuyu ya da dilek adagini yerine getirmek için RAB'be esenlik kurbani olarak sigir veya davar sunmak isterse, kabul edilmesi için hayvan kusursuz olmali. Hiçbir eksigi bulunmamali.
\par 22 Kör, sakat, yarali, yarasi irinli, kabuklu ya da uyuz bir hayvani RAB'be sunmayacaksiniz. Yakilan sunu olarak sunak üzerinde RAB'be böyle bir hayvan sunmayacaksiniz.
\par 23 Organlarindan biri asiri büyümüs ya da yeterince gelismemis bir sigiri veya davari dilek adagi olarak sunabilirsiniz. Ama gönülden verilen bir sunu olarak kabul edilmez.
\par 24 Yumurtalari vurulmus, ezilmis, burulmus ya da kesilmis hayvani RAB'be sunmayacak, ülkenizde buna yer vermeyeceksiniz.
\par 25 Böyle bir hayvani bir yabancidan alip yiyecek sunusu olarak Tanriniz'a sunmayacaksiniz. Çünkü sakat ve kusurludur. Kabul edilmeyecektir."
\par 26 RAB Musa'ya söyle dedi:
\par 27 "Bir buzagi, kuzu ya da oglak dogdugu zaman, yedi gün anasinin yaninda kalacaktir. Sekizinci günden itibaren yakilan sunu olarak RAB'be sunulabilir. Kabul edilecektir.
\par 28 Ister inek, ister davar olsun, hayvanla yavrusunu ayni gün kesmeyeceksiniz.
\par 29 "RAB'be sükran kurbani sundugunuz zaman, kabul edilecek biçimde sunun.
\par 30 Eti ayni gün yenecek, sabaha birakilmayacak. RAB benim.
\par 31 "Buyruklarima uyacak, onlari yerine getireceksiniz. RAB benim.
\par 32 Kutsal adima leke sürmeyeceksiniz. Israil halki arasinda kutsal taninacagim. Sizi kutsal kilan RAB benim.
\par 33 Tanriniz olmak için sizi Misir'dan çikardim. RAB benim."

\chapter{23}

\par 1 RAB Musa'ya söyle dedi:
\par 2 "Israil halkina de ki, 'Kutsal toplantilar olarak ilan edeceginiz bayramlarim, RAB'bin bayramlari sunlardir:" Sabat Günü
\par 3 "'Alti gün çalisacaksiniz. Ama yedinci gün olan Sabat* dinlenme ve kutsal toplanti günüdür. Hiçbir is yapmayacaksiniz. Yasadiginiz her yerde Sabat'i RAB'be ayiracaksiniz."
\par 4 "'Belirli zamanlarda kutsal toplantilar olarak ilan edeceginiz RAB'bin bayramlari sunlardir:
\par 5 Birinci ayin* on dördüncü günü aksamüstü RAB'bin Fisih Bayrami* baslar.
\par 6 On besinci gün RAB'bin Mayasiz Ekmek Bayrami'dir*. Yedi gün mayasiz ekmek yiyeceksiniz.
\par 7 Ilk gün kutsal toplanti düzenleyecek, gündelik islerinizi yapmayacaksiniz.
\par 8 Yedi gün RAB için yakilan sunu sunacaksiniz. Yedinci gün kutsal toplanti düzenleyecek, gündelik islerinizi yapmayacaksiniz."
\par 9 RAB Musa'ya söyle dedi:
\par 10 "Israil halkina de ki, 'Size verecegim ülkeye girip ürününü biçtiginiz zaman, ilk yetisen ürününüzden bir demet kâhine götüreceksiniz.
\par 11 Kabul edilmeniz için, kâhin demeti RAB'bin huzurunda sallayacak. Demet Sabat'tan* sonraki gün sallanacak.
\par 12 Demetin sallandigi gün, yakmalik sunu* olarak RAB'be bir yasinda kusursuz bir erkek kuzu sunacaksiniz.
\par 13 Kuzuyla birlikte tahil sunusu* olarak yagla yogrulmus bir efa ince unun onda ikisi sunulacak. RAB için yakilan sunu ve O'nu hosnut eden koku olacak bu. Yakmalik sunuyla birlikte dökmelik sunu olarak bir hin sarabin dörtte birini sunacaksiniz.
\par 14 Tanriniz'a bu sunuyu getireceginiz güne kadar ekmek, kavrulmus bugday, taze basak yemeyeceksiniz. Yasadiginiz her yerde kusaklar boyunca sürekli bir yasa olacak bu."
\par 15 "'Sabat'tan* sonraki gün sallamalik demeti götürdügünüz günden baslayarak tam yedi hafta sayin.
\par 16 Yedinci Sabat'tan sonraki güne kadar elli gün sayin: O gün RAB'be yeni tahil sunusu* sunacaksiniz.
\par 17 Yasadiginiz yerden RAB'be sallamalik sunu olarak iki ekmek getirin. Ekmekler ilk ürünlerden, onda iki efa ince undan yapilacak. Mayayla pisirilip RAB'be öyle sunulacak.
\par 18 Ekmekle birlikte yakmalik sunu* olarak RAB'be bir yasinda kusursuz yedi kuzu, bir boga ve iki koç sunacaksiniz. Tahil sunusu ve dökmelik sunuyla sunulan bu sunu, yakilan sunu ve RAB'bi hosnut eden kokudur.
\par 19 Günah sunusu* olarak bir teke, esenlik kurbani olarak bir yasinda iki kuzu sunacaksiniz.
\par 20 Kâhin iki kuzuyu ilk üründen yapilmis ekmekle birlikte sallamalik sunu olarak RAB'bin huzurunda sallayacak. RAB için kutsal olan bu sunular kâhinindir.
\par 21 O gün kutsal toplanti ilan edecek ve gündelik islerinizi yapmayacaksiniz. Yasadiginiz her yerde kusaklar boyunca sürekli bir yasa olacak bu.
\par 22 "'Ülkenizdeki ekinleri biçerken tarlalarinizi sinirlarina kadar biçmeyin. Artakalan basaklari toplamayin. Onlari yoksullara ve yabancilara birakacaksiniz. Tanriniz RAB benim."
\par 23 RAB Musa'ya söyle dedi:
\par 24 "Israil halkina de ki, 'Yedinci ayin* birinci günü dinlenme günüdür, boru çalinarak anma ve kutsal toplanti günü olacaktir.
\par 25 O gün gündelik islerinizi yapmayacak, RAB için yakilan sunu sunacaksiniz."
\par 26 RAB Musa'ya söyle dedi:
\par 27 "Yedinci ayin* onuncu günü günahlarin bagislanma günüdür. Kutsal bir toplanti düzenleyeceksiniz. Isteklerinizi denetleyecek, RAB için yakilan sunu sunacaksiniz.
\par 28 O gün hiç is yapmayacaksiniz. Çünkü Tanriniz RAB'bin huzurunda günahlarinizi bagislatacaginiz bagislanma günüdür.
\par 29 O gün isteklerini denetlemeyen herkes halkin arasindan atilacaktir.
\par 30 O gün herhangi bir is yapani halkin arasindan yok edecegim.
\par 31 Hiç is yapmayacaksiniz. Yasadiginiz her yerde kusaklar boyunca sürekli yasa olacak bu.
\par 32 O gün sizin için Sabat*, dinlenme günü olacak. Isteklerinizi denetleyeceksiniz. Ayin dokuzuncu günü, aksamdan ertesi aksama kadar Sabat'i kutlayacaksiniz."
\par 33 RAB Musa'ya söyle dedi:
\par 34 "Israil halkina de ki, 'Yedinci ayin* on besinci günü Çardak Bayrami* baslar. Bu bayrami RAB'bin onuruna yedi gün kutlayacaksiniz.
\par 35 Ilk gün kutsal bir toplanti düzenleyecek, gündelik islerinizi yapmayacaksiniz.
\par 36 Yedi gün RAB için yakilan sunu sunacaksiniz. Sekizinci gün kutsal bir toplanti düzenlemeli, RAB için yakilan sunu sunmalisiniz. Bu bayramin son toplantisidir. Gündelik islerinizi yapmayacaksiniz.
\par 37 -"'Kutsal toplantilar olarak ilan edeceginiz RAB'bin bayramlari bunlardir. Bayramlarda RAB için yakilan sunuyu, yakmalik sunuyu*, tahil sunusunu*, kurbani, dökmelik sunulari günün geregine uygun biçimde sunacaksiniz.
\par 38 Bunlar RAB'bin kutlamanizi istedigi Sabat* günlerinin, RAB'be sundugunuz armaganlarin, bütün dilek adaklarinin ve gönülden verilen sunularin disindadir.-
\par 39 "'Yedinci ayin on besinci günü, topraklarinizin ürünlerini devsirdiginiz zaman RAB için yedi gün bayram yapacaksiniz. Birinci ve sekizinci gün dinlenme günleri olacak.
\par 40 Ilk gün meyve agaçlarinin güzel meyvelerini, hurma dallarini, sik yaprakli agaç dallarini, vadi kavaklarini toplayip Tanriniz RAB'bin önünde yedi gün senlik yapacaksiniz.
\par 41 Bunu her yil yedi gün RAB'bin bayrami olarak kutlayacaksiniz. Kusaklar boyunca sürekli bir yasa olacak bu. Bayrami yedinci ay kutlayacaksiniz.
\par 42 Yedi gün çardaklarda kalacaksiniz. Bütün yerli Israilliler çardaklarda yasayacak.
\par 43 Öyle ki, gelecek kusaklar Israil halkini Misir'dan çikardigim zaman çardaklarda barindirdigimi bilsinler. Tanriniz RAB benim."
\par 44 Böylece Musa Israil halkina RAB'bin bayramlarini bildirdi.

\chapter{24}

\par 1 RAB Musa'ya söyle dedi:
\par 2 "Israil halkina buyruk ver, kandilin sürekli yanip isik vermesi için saf sikma zeytinyagi getirsinler.
\par 3 Harun kandilleri benim huzurumda, Bulusma Çadiri'nda, Levha Sandigi'nin önündeki perdenin disinda, aksamdan sabaha kadar sürekli yanar biçimde tutacak. Kusaklar boyunca sürekli bir kural olacak bu.
\par 4 RAB'bin huzurunda saf altin kandillikteki kandiller sürekli yanacaktir."
\par 5 "Ince undan on iki pide pisireceksin. Her biri efanin onda ikisi agirliginda olacak.
\par 6 Bunlari RAB'bin huzurunda iki sira halinde, altisar altisar saf altin masanin üzerine dizeceksin.
\par 7 Iki sira ekmegin yanina anma payi olarak saf günnük koyacaksin. Bu RAB için yakilan sunu olacak ve ekmegin yerini alacak.
\par 8 Bu ekmek, Israil halki adina sonsuza dek sürecek bir antlasma olarak, her Sabat* Günü aksatilmadan RAB'bin huzurunda dizilecek.
\par 9 Ve Harun'la ogullarina ait olacak. Onu kutsal bir yerde yiyecekler. Çünkü çok kutsaldir. RAB için yakilan sunulardan onlarin sürekli bir payi olacak."
\par 10 Israilliler arasinda annesi Israilli babasi Misirli bir adam vardi. Ordugahta onunla bir Israilli arasinda kavga çikti.
\par 11 Israilli kadinin oglu RAB'be sövdü, lanet etti. Onu Musa'ya getirdiler. Annesi Dan oymagindan Divri'nin kizi Selomit'ti.
\par 12 Adami göz altina alip RAB'bin kararini beklediler.
\par 13 RAB Musa'ya söyle dedi:
\par 14 "Onu ordugahin disina çikar. Ettigi laneti duyan herkes elini adamin basina koysun ve bütün topluluk onu taslasin.
\par 15 Israil halkina de ki, 'Kim Tanrisi'na lanet ederse günahinin cezasini çekecektir.
\par 16 RAB'be söven kesinlikle öldürülecektir. Bütün topluluk onu taslayacak. Ister yerli ister yabanci olsun, RAB'be söven herkes öldürülecektir.
\par 17 "'Adam öldüren kesinlikle öldürülecektir.
\par 18 Baskasinin hayvanini öldüren, yerine bir hayvan vererek aldigi canin karsiligini canla ödeyecektir.
\par 19 Kim komsusunu yaralarsa, kendisine de ayni sey yapilacaktir.
\par 20 Kiriga karsilik kirik, göze göz, dise dis olmak üzere, ona ne yaptiysa kendisine de ayni sey yapilacaktir.
\par 21 Hayvan öldüren yerine bir hayvan verecek, adam öldüren öldürülecektir.
\par 22 Yerli yabanci herkes için tek bir yasaniz olacak. Tanriniz RAB benim."
\par 23 Musa bunlari Israil halkina bildirdikten sonra, halk RAB'be lanet eden adami ordugahin disina çikardi ve taslayarak öldürdü. Böylece Israil halki RAB'bin Musa'ya verdigi buyrugu yerine getirmis oldu.

\chapter{25}

\par 1 RAB Sina Dagi'nda Musa'ya söyle dedi:
\par 2 "Israil halkina de ki, 'Size verecegim ülkeye girdiginiz zaman, ülke RAB için Sabat'i* kutlamali.
\par 3 Alti yil tarlani ekeceksin, bagini budayacaksin, ürününü toplayacaksin.
\par 4 Ama yedinci yil toprak dinlenecek. O yil Sabat Yili olacak, RAB'be adanacak. Tarlani ekmemeli, bagini budamamalisin.
\par 5 Hasadinin ardindan süreni biçmeyecek, budanmamis asmanin üzümlerini toplamayacaksin. O yil ülke için dinlenme yili olacak.
\par 6 Sabat Yili'nda ülke ne ürün verirse, sizin için, köleleriniz, cariyeleriniz, yaninizda çalisan ücretliler ve aranizda yasayan yabancilar için yiyecek olacak.
\par 7 Ülkede yetisen ürünler kendi hayvanlarinizi da yabanil hayvanlari da doyuracak."
\par 8 "'Yedi yilda bir kutlanan Sabat yillarinin yedi kez geçmesini bekleyin. Yedi kez geçecek Sabat yillarinin toplami kirk dokuz yildir.
\par 9 Sonra, yedinci ayin* onuncu günü, yani günahlari bagislatma günü, bütün ülkede yüksek sesle boru çalinacak.
\par 10 Ellinci yili kutsal sayacak, bütün ülke halki için özgürlük ilan deceksiniz. O yil sizin için özgürlük yili olacak. Herkes kendi topragina, ailesine dönecek.
\par 11 Ellinci yil sizin için özgürlük yili olacak. O yil ekmeyecek, ürünün ardindan süreni biçmeyecek, budanmamis asmanin üzümlerini toplamayacaksiniz.
\par 12 Çünkü o yil özgürlük yilidir. Sizin için kutsaldir. Yalniz tarlalarda kendiliginden yetiseni yiyebilirsiniz.
\par 13 "'Özgürlük yilinda herkes kendi topragina dönecek.
\par 14 Bir komsuna tarla satar ya da ondan tarla alirsan, birbirinize haksizlik yapmayacaksiniz.
\par 15 Eger sen ondan aliyorsan, özgürlük yilindan sonraki yillarin sayisina göre ödeyeceksin, o da sana ürün yillarinin sayisina göre satacak.
\par 16 Yillarin sayisi çoksa fiyati artiracak, azsa indireceksin. Çünkü sana yillik ürünlerini satiyor.
\par 17 Birbirinize haksizlik yapmayacak, Tanriniz'dan korkacaksiniz. Tanriniz RAB benim.
\par 18 "'Kurallarima uyacak, ilkelerimi özenle yerine getireceksiniz. Böylece ülkede güvenlik içinde yasayacaksiniz.
\par 19 Ülke de ürün verecek, sizi doyuracak ve orada güvenlik içinde oturacaksiniz.
\par 20 Topragimizi ekmez, ürünümüzü toplamazsak, yedinci yil ne yiyecegiz? diye sorarsaniz,
\par 21 altinci yil size öyle bir bereket gönderecegim ki, toprak üç yillik ürün verecek.
\par 22 Sekizinci yil topraginizi ekerken, dokuzuncu yil ürün alincaya kadar eski ürününüzü yiyeceksiniz.
\par 23 "'Tarlaniz temelli olarak satilamaz. Çünkü bana aittir. Sizse yabancisiniz, konugumsunuz.
\par 24 Miras alacaginiz ülkenin her yerinde tarlanin asil sahibine tarlasini geri alma hakki tanimalisiniz.
\par 25 Kardeslerinizden biri yoksullasir, topraginin bir parçasini satmak zorunda kalirsa, en yakin akrabasi gelip topragi geri alabilir.
\par 26 Topragini satin alacak yakin bir akrabasi yoksa, sonradan durumu düzelir, yeterli para bulursa,
\par 27 satis yaptiktan sonra geçen yillari hesaplayacak ve geri kalan parayi topragini sattigi adama ödeyip topragina dönecek.
\par 28 Ancak topragini geri alacak parayi bulamazsa, toprak özgürlük yilina kadar onu satin alan adama ait olacak. O yil topragi elinden çikaracak, satan adam da topragina kavusacak.
\par 29 "'Surlu bir kentte evini satan adamin evi sattiktan tam bir yil sonrasina kadar onu geri alma hakki olacaktir.
\par 30 Eger bir yil içinde evini geri almazsa, ev temelli olarak aliciya geçecek, kusaklar boyunca yeni sahibinin olacaktir. Özgürlük yilinda ev yeni sahibinin elinden alinmayacaktir.
\par 31 Ama surlarla çevrilmemis köylerdeki evler tarlalar gibi islem görecektir. Ilk sahibinin evi geri alma hakki olacak, özgürlük yilinda evi satin alan onu geri verecektir.
\par 32 "'Ancak Levililer kendilerine ait kentlerde sattiklari evleri her zaman geri alma hakkina sahiptirler.
\par 33 Eger bir Levili bu kentlerde sattigi evi geri alamazsa, özgürlük yilinda ev kendisine geri verilecektir. Çünkü Levililer'in kentlerindeki evler onlarin Israil halkinin arasindaki mülkleridir.
\par 34 Kentlerinin çevresindeki otlaklar ise satilamaz. Çünkü bunlar onlarin kalici mülküdür.
\par 35 "'Bir kardesin yoksullasir, muhtaç duruma düserse, ona yardim etmelisin. Aranizda kalan bir yabanci ya da konuk gibi yasayacak.
\par 36 Ondan faiz ve kâr alma. Tanrin'dan kork ki, kardesin yaninda yasamini sürdürebilsin.
\par 37 Ona faizle para vermeyeceksin. Ödünç verdigin yiyecekten kâr almayacaksin.
\par 38 Ben Kenan ülkesini size vermek ve Tanriniz olmak için sizleri Misir'dan çikaran Tanriniz RAB'bim.
\par 39 "'Aranizda yasayan bir kardesin yoksullasir, kendini köle olarak sana satarsa, onu bir köle gibi çalistirmayacaksin.
\par 40 Yaninda çalisan bir isçi ya da yabanci gibi davranacaksin ona. Özgürlük yilina dek yaninda çalisacak.
\par 41 Sonra çocuklariyla birlikte yanindan ayrilip ailesinin yanina, atalarinin topragina dönecek.
\par 42 Çünkü Israilliler benim Misir'dan çikardigim kullarimdir. Köle olarak satilamazlar.
\par 43 Ona efendilik etmeyecek, sert davranmayacaksin. Tanrin'dan korkacaksin.
\par 44 "'Köleleriniz, cariyeleriniz çevrenizdeki uluslardan olmali. Onlardan usak ve cariye satin alabilirsiniz.
\par 45 Ayrica aranizda yasayan yabancilarin çocuklarini, ister ülkenizde dogmus olsun ister olmasin, satin alip onlara sahip olabilirsiniz.
\par 46 Onlari miras olarak çocuklariniza birakabilirsiniz. Yasamlari boyunca size kölelik edecekler. Ancak bir Israilli kardesine efendilik etmeyecek, sert davranmayacaksin.
\par 47 "'Aranizda yasayan bir yabanci ya da geçici olarak kalan biri zenginlesir, buna karsilik bir Israilli kardesin yoksullasip kendini ona ya da ailesinin bir bireyine köle olarak satarsa,
\par 48 satildiktan sonra geri alinma hakki vardir. Kardeslerinden biri, amcasi, amcasinin oglu veya yakin akrabalarindan, ailesinden biri onu geri alabilir. Ya da yeterli para bulursa, kendisi özgürlügünü geri alabilir.
\par 50 Efendisiyle hesap görmeli. Kendisini sattigi yildan özgürlük yilina kadar geçen yillari sayacaklar. Özgürlügünün bedeli, kalan yillarin sayisina göre bir isçinin gündelik ücreti üzerinden hesap edilecektir.
\par 51 Eger geriye çok yil kaliyorsa, buna göre özgürlügünün bedeli olarak satin alindigi fiyatin bir bölümünü ödeyecek.
\par 52 Eger özgürlük yilina yalniz birkaç yil kalmissa, ona göre hesap ederek özgürlügünün bedelini ödemelidir.
\par 53 Efendisinin yaninda yillik sözlesmeyle çalisan bir isçi gibi yasamalidir. Senin önünde efendisinin ona sert davranmamasini saglayacaksin.
\par 54 "'Bu yollardan özgürlügüne kavusamasa bile, özgürlük yilinda çocuklariyla birlikte özgür olacaktir.
\par 55 Çünkü Israilliler benim kullarim, Misir'dan çikardigim kendi kullarimdir. Tanriniz RAB benim."

\chapter{26}

\par 1 "'Put yapmayacaksiniz. Oyma put ya da tas sütun dikmeyeceksiniz. Tapmak için ülkenize putlari simgeleyen oyma taslar koymayacaksiniz. Çünkü Tanriniz RAB benim.
\par 2 Sabat* günlerimi tutacak, tapinagima saygi göstereceksiniz. RAB benim.
\par 3 "'Kurallarima göre yasar, buyruklarimi dikkatle yerine getirirseniz,
\par 4 yagmurlari zamaninda yagdiracagim. Toprak ürün, agaçlar meyve verecek.
\par 5 Bagbozumuna kadar harman dövecek, ekim zamanina kadar baglarinizdan üzüm toplayacaksiniz. Bol bol yiyecek, ülkenizde güvenlik içinde yasayacaksiniz.
\par 6 "'Ülkenize baris saglayacagim. Korku içinde yatmayacaksiniz. Tehlikeli hayvanlari ülkenizden kovacagim. Savas yüzü görmeyeceksiniz.
\par 7 Düsmanlarinizi kovalayacaksiniz. Kiliç darbeleriyle önünüzde yere serilecekler.
\par 8 Besiniz yüz kisinin, yüzünüz on bin kisinin hakkindan gelecek. Düsmanlariniz kiliç darbeleriyle önünüzde yere serilecek.
\par 9 Size iyilikle bakacagim. Sizi verimli kilip çogaltacagim. Sizinle yaptigim antlasmayi sürdürecegim.
\par 10 Eski ürününüz yemekle tükenmeyecek. Yeni ürüne yer bulmak için eskisini bosaltmak zorunda kalacaksiniz.
\par 11 Konutumu aranizda kuracak, size sirt çevirmeyecegim.
\par 12 Aranizda yasayacak, Tanriniz olacagim. Siz de benim halkim olacaksiniz.
\par 13 Ben sizi Misir'da köle olmaktan kurtaran Tanriniz RAB'bim. Boyundurugunuzu kirdim. Sizi basi dik yasattim." Tanri'dan Uzaklasmanin Cezasi
\par 14 "'Ama beni dinlemez, bütün bu buyruklari yerine getirmezseniz, cezalandirilacaksiniz.
\par 15 Kurallarimi çigner, ilkelerimden nefret eder, buyruklarima karsi çikar, antlasmami bozarsaniz,
\par 16 sizi söyle cezalandiracagim: Üzerinize dehset salacagim. Verem ve sitma gözlerinizin ferini söndürecek, caninizi kemirecek. Bosa tohum ekeceksiniz, çünkü ürünlerinizi düsmanlariniz yiyecek.
\par 17 Size öfkeyle bakacagim. Düsmanlariniz sizi bozguna ugratacak. Sizden nefret edenler sizi yönetecek. Kovalayan yokken bile kaçacaksiniz.
\par 18 "'Bütün bunlara karsin beni dinlemezseniz, günahlariniza karsilik cezanizi yedi kat artiracagim.
\par 19 Inatçi gururunuzu kiracagim. Gök demir, yer bakir olacak.
\par 20 Gücünüz tükenecek. Topraklariniz ürün, agaçlariniz meyve vermeyecek.
\par 21 "'Eger karsi çikmaya devam eder, beni dinlemek istemezseniz, günahlariniza karsilik cezanizi yedi kat artiracagim.
\par 22 Üzerinize yabanil hayvanlar gönderecegim. Çocuklarinizi öldürecek, hayvanlarinizi yok edecekler. Sayiniz azalacak, yollariniz issiz kalacak.
\par 23 "'Bununla da yola gelmez, bana karsi çikmaya devam ederseniz,
\par 24 ben de size karsi çikacagim, günahlariniza karsilik sizi yedi kez cezalandiracagim.
\par 25 Bozdugunuz antlasmamin öcünü almak için basiniza savas getirecegim. Kentlerinize çekildiginizde araniza salgin hastalik gönderecegim. Düsman eline düseceksiniz.
\par 26 Ekmeginizi kestigim zaman, on kadin ekmeginizi bir firinda pisirecek. Ekmeginiz azar azar, tartiyla verilecek. Yiyecek ama doymayacaksiniz.
\par 27 "'Bütün bunlardan sonra yine beni dinlemez, bana karsi çikarsaniz,
\par 28 bu kez ben de öfkeyle size karsi çikacagim ve günahlariniza karsilik sizi yedi kat cezalandiracagim.
\par 29 Açliktan çocuklarinizin etini yiyeceksiniz.
\par 30 Tapinma yerlerinizi yikacak, buhur sunaklarinizi yok edecegim. Cesetlerinizi devrilen putlarin üzerine serecek, sizden nefret edecegim.
\par 31 Kentlerinizi viraneye çevirecek, tapinaklarinizi yikacagim. Beni hosnut etmek için sundugunuz kokulari duymayacagim.
\par 32 Ülkenizi viran edecegim, oraya yerlesen düsmanlariniz bile saskina dönecek.
\par 33 Sizi öteki uluslarin arasina dagitacak, kilicimla pesinize düsecegim. Ülkeniz viran olacak, kentleriniz harabeye dönecek.
\par 34 Siz düsmanlarinizin ülkesinde yasarken, ülke issiz kaldigi yillar boyunca Sabatlar'in* sevincini yasayacak. Ancak o zaman dinlenip Sabatlari'nin tadina varacak.
\par 35 Üzerinde yasadiginiz Sabat yillarinda görmedigi rahati issiz kaldigi yillarda görecek.
\par 36 "'Düsman ülkelerinde sag kalanlarinizin yüregine öyle bir korku düsürecegim ki, rüzgarin sürükledigi yapraklarin sesinden bile kaçacaklar. Savastan kaçarcasina kaçacaklar. Peslerinde kovalayan olmadigi halde düsecekler.
\par 37 Kovalayan yokken savastan kaçarcasina birbirlerinin üzerine yikilacaklar. Düsmanlarinizin karsisinda ayakta duramayacaksiniz.
\par 38 Öteki uluslarin arasinda yok olacaksiniz. Düsman ülkeler sizi yutacak.
\par 39 Artakalanlariniz gerek kendi, gerekse atalarinin suçlarindan ötürü düsman ülkelerde eriyip gidecekler.
\par 40 "'Ama isledikleri suçlari, atalarinin suçlarini, bana karsi geldiklerini, ihanet ettiklerini itiraf eder
\par 41 -bu yüzden onlara karsi çikip kendilerini düsman ülkelerine sürmüstüm- inadi birakip alçakgönüllü olur, suçlarinin bedelini öderlerse,
\par 42 ben de Yakup'la, Ishak'la, Ibrahim'le yaptigim antlasmayi ve onlara söz verdigim ülkeyi animsayacagim.
\par 43 Ülke önce issiz birakilacak ve issiz kaldigi sürece Sabatlar'in tadina varacak. Onlar da isledikleri suçlarin bedelini ödeyecekler; çünkü ilkelerimi reddettiler, kurallarimdan nefret ettiler.
\par 44 Bütün bunlara karsin, düsman ülkelerindeyken yine de onlari reddetmeyecek, onlardan nefret etmeyecegim. Böylece hepsini yok etmeyecek, kendileriyle yaptigim antlasmayi bozmayacagim. Çünkü ben onlarin Tanrisi RAB'bim.
\par 45 Tanrilari olmak için öteki uluslarin önünde Misir'dan çikardigim atalariyla yaptigim antlasmayi onlar için animsayacagim. RAB benim."
\par 46 RAB'bin Sina Dagi'nda Musa araciligiyla kendisiyle Israil halki arasina koydugu kurallar, ilkeler, yasalar bunlardir.

\chapter{27}

\par 1 RAB Musa'ya söyle dedi:
\par 2 "Israil halkina de ki, 'Eger bir kimse RAB'be birini adamissa senin biçecegin degeri ödeyerek adagini yerine getirebilir.
\par 3 Bu degerler söyle olacak: Yirmi yasindan altmis yasina kadar erkekler için elli kutsal yerin sekeli gümüs,
\par 4 kadinlar için otuz sekel.
\par 5 Bes yasindan yirmi yasina kadar erkekler için yirmi, kadinlar için on sekel.
\par 6 Bir ayliktan bes yasina kadar oglanlar için bes, kizlar için üç sekel gümüs.
\par 7 Eger altmis ya da daha yukari yasta iseler, erkekler için on bes, kadinlar için on sekel.
\par 8 Ancak adakta bulunan kisi belirtilen parayi ödeyemeyecek kadar yoksulsa, adadigi kisiyi kâhine götürecek; kâhin adakta bulunan kisinin ödeme gücüne göre ona deger biçecektir.
\par 9 "'RAB'be sunulacak adak O'na sunu olarak sunulabilecek hayvanlardan biriyse, kabul edilecektir. O'na böyle sunulan her hayvan kutsaldir.
\par 10 Adakta bulunan kisi RAB'be sunacagi adagi degistirmemeli. Iyisinin yerine kötüsünü ya da kötüsünün yerine iyisini koymamali. Eger hayvani degistirirse, degistirilen hayvanlarin ikisi de kutsal sayilacaktir.
\par 11 Eger adak RAB'be sunulamayacak kirli sayilan hayvanlardan biriyse, kâhine götürülecektir.
\par 12 Hayvan iyi ya da kötü olsun, kâhin ona deger biçecek. Biçilen deger neyse o geçerli olacak.
\par 13 Ama sahibi hayvani geri almak isterse, kâhinin biçtigi degerin üzerine beste bir fazlasini katarak ödemelidir.
\par 14 "'Bir kimse kutsal bir sunu olarak evini RAB'be adarsa, evin iyi ya da kötü olduguna kâhin karar verecektir. Kâhinin biçtigi deger geçerli olacaktir.
\par 15 Eger kisi adadigi evi geri almak isterse, kâhinin biçtigi degerin üzerine beste bir fazlasini katarak ödeyecek, böylece ev kendisine kalacaktir.
\par 16 "'Bir kimse ailesinden kalan tarlanin bir bölümünü RAB'be adamak isterse, tarlaya ekilecek tohum miktarina göre deger biçilecektir. Bir homer arpa tohumu ekilebilecek tarlanin degeri elli sekel gümüs olacaktir.
\par 17 Eger tarlasini özgürlük yilindan hemen sonra adarsa, bu fiyat geçerli olacaktir.
\par 18 Eger özgürlük yilindan daha sonra adarsa, kâhin bir sonraki özgürlük yilina kadar geçecek yillarin sayisina göre tarlaya deger biçecektir. Tarlanin fiyati daha düsük olacaktir.
\par 19 Kisi tarlasini geri almak isterse, tarlaya biçilen degerin üzerine beste bir fazlasini katarak ödeyecek, böylece tarla kendisine kalacaktir.
\par 20 Ama tarlayi geri almadan baska birine satarsa, tarla geri alinamaz.
\par 21 Tarla özgürlük yilinda serbest kaldigi zaman, RAB'be kosulsuz adanmis bir tarla gibi kutsal sayilacak ve kâhinlere ait olacaktir.
\par 22 "'Bir kimse ailesinin mülkü olmayan, sonradan satin aldigi bir tarlayi RAB'be adarsa,
\par 23 kâhin özgürlük yilina kadar geçecek yillara göre ona bir deger biçecektir. O gün kisi biçilen deger üzerinden ödeme yapacak ve para RAB'be ait olacak, kutsal sayilacaktir.
\par 24 Özgürlük yilinda tarla ilk sahibine geri verilecektir.
\par 25 Deger biçilirken kutsal yerin sekeli esas alinacaktir. Bir sekel yirmi geradir.
\par 26 "'Ilk dogan hayvan RAB'be aittir. Ister sigir, ister davar olsun, kimse onu RAB'be adayamaz. Çünkü o RAB'bindir.
\par 27 Ama ilk dogan hayvan kirli sayilan hayvanlardan biriyse, kisi kâhinin biçecegi degerin beste bir fazlasini ödeyerek hayvani geri alabilir. Geri alinmazsa, hayvan biçilen deger üzerinden baska birine satilacaktir.
\par 28 "'Ister insan, ister hayvan, ister aileden kalma tarla olsun, RAB'be kosulsuz adanan hiç bir sey satilmayacak ve geri alinmayacaktir. Çünkü RAB'be kosulsuz adanan her sey RAB için çok kutsaldir.
\par 29 RAB'be kosulsuz adanan insan para karsiliginda kurtarilamayacak, kesinlikle öldürülecektir.
\par 30 "'Ister topragin ürünü, ister agacin meyvesi olsun, toprakta yetisen her seyin ondaligi RAB'be aittir. RAB için kutsaldir.
\par 31 Kim ondaliginin bir bölümünü geri almak isterse, degerinin üzerine beste bir fazlasini katarak ödemelidir.
\par 32 Bütün sigirlarla davarlarin ondaligi, sayimda çoban degneginin altindan geçen her onuncu hayvan RAB için kutsal sayilacaktir.
\par 33 Hayvan sahibi hayvanlari iyi, kötü diye ayirmayacak, birini öbürüyle degistirmeyecektir. Degistirirse, degistirilen hayvanlarin ikisi de kutsal sayilacak ve karsiligi ödenip geri alinamayacaktir."
\par 34 RAB'bin Sina Dagi'nda Israil halki için Musa'ya bildirdigi buyruklar bunlardir.


\end{document}