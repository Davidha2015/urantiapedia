\begin{document}

\title{Rut}


\chapter{1}

\par 1 Hakimlerin egemenlik sürdügü günlerde Israil'de kitlik basladi. Yahuda'nin Beytlehem Kenti'nden bir adam, karisi ve iki ogluyla birlikte geçici bir süre kalmak üzere Moav topraklarina dogru yola çikti.
\par 2 Adamin adi Elimelek, karisinin adi Naomi, ogullarinin adlari da Mahlon ve Kilyon'du. Yahuda'nin Beytlehem Kenti'nden, Efrat boyundan olan bu kisiler, Moav topraklarina gidip orada yasamaya basladilar.
\par 3 Naomi, kocasi Elimelek ölünce iki ogluyla yalniz kaldi.
\par 4 Iki ogul Moav kizlarindan kendilerine birer es aldilar. Kizlardan birinin adi Orpa, ötekinin adi Rut'tu. Orada on yil kadar yasadiktan sonra,
\par 5 Mahlon da, Kilyon da öldü. Böylece kocasiyla iki oglunu yitiren Naomi yapayalniz kaldi.
\par 6 Naomi, Moav topraklarindayken RAB'bin kendi halkinin yardimina yetisip yiyecek sagladigini duyunca gelinleriyle oradan dönmeye hazirlandi.
\par 7 Onlarla birlikte bulundugu yerden ayrildi ve Yahuda ülkesine dönmek üzere yola koyuldu.
\par 8 Yolda onlara, "Analarinizin evine dönün" dedi. "Ölmüslerimize ve bana nasil iyilik ettinizse, RAB de size iyilik etsin.
\par 9 RAB her birinize evinde rahat edeceginiz birer koca versin!" Sonra onlari öptü. Iki gelin hiçkira hiçkira aglayarak,
\par 10 "Hayir, seninle birlikte senin halkina dönecegiz" dediler.
\par 11 Naomi, "Geri dönün, kizlarim" dedi. "Niçin benimle gelesiniz? Size koca olacak ogullarim olabilir mi bundan sonra?
\par 12 Dönün kizlarim, yolunuza gidin. Ben kocaya varamayacak kadar yaslandim. Umudum var desem, bu gece kocaya varip ogullar dogursam,
\par 13 onlar büyüyene kadar bekler miydiniz, kocaya varmaktan vazgeçer miydiniz? Hayir, kizlarim! Benim acim sizinkinden de büyüktür. Çünkü RAB beni felakete ugratti."
\par 14 Gelinler yine hiçkira hiçkira aglamaya basladi. Sonunda Orpa kaynanasini öpüp vedalasti, Rut'sa ona sarilip yaninda kaldi.
\par 15 Naomi Rut'a, "Bak, eltin kendi halkina, kendi ilahina dönüyor. Sen de onun ardindan git" dedi.
\par 16 Rut söyle karsilik verdi: "Seni birakip geri dönmemi isteme! Sen nereye gidersen ben de oraya gidecegim, sen nerede kalirsan ben de orada kalacagim. Senin halkin benim halkim, senin Tanrin benim Tanrim olacak.
\par 17 Sen nerede ölürsen ben de orada ölecegim ve orada gömülecegim. Eger ölümden baska bir nedenle senden ayrilirsam, RAB bana daha kötüsünü yapsin."
\par 18 Naomi, Rut'un kendisiyle gitmeye kesin kararli oldugunu görünce üstelemekten vazgeçti.
\par 19 Böylece ikisi Beytlehem'e kadar yola devam ettiler. Dönüsleri bütün kenti ayaga kaldirdi. Kadinlar birbirlerine, "Naomi bu mu?" diye sordular.
\par 20 Naomi onlara, "Beni, Naomi degil, Mara diye çagirin" dedi. "Çünkü Her Seye Gücü Yeten Tanri bana çok aci verdi.
\par 21 Giderken her seyim vardi, ama RAB beni eli bos döndürdü. Beni niçin Naomi diye çagirasiniz ki? Görüyorsunuz, RAB beni sikintiya soktu, Her Seye Gücü Yeten Tanri basima felaket getirdi."
\par 22 Iste Naomi, Moavli gelini Rut'la birlikte Moav topraklarindan böyle döndü. Beytlehem'e gelisleri, arpanin biçilmeye baslandigi zamana rastlamisti.

\chapter{2}

\par 1 Naomi'nin Boaz adinda bir akrabasi vardi. Kocasi Elimelek'in boyundan olan Boaz, ileri gelen, varlikli bir adamdi.
\par 2 Bir gün Moavli Rut, Naomi'ye söyle dedi: "Izin ver de tarlalara gideyim, iyiliksever bir adamin ardinda basak devsireyim." Naomi, "Git, kizim" diye karsilik verdi.
\par 3 Böylece Rut gidip tarlalarda, orakçilarin ardinda basak devsirmeye basladi. Bir rastlanti sonucu, kendini Elimelek'in boyundan Boaz'in tarlasinda buldu.
\par 4 Bu arada Beytlehem'den gelen Boaz orakçilara, "RAB sizinle olsun" diye seslendi. Onlar da, "RAB seni kutsasin" karsiligini verdiler.
\par 5 Boaz, orakçilarin basinda duran adamina, "Kim bu genç kadin?" diye sordu.
\par 6 Orakçilarin basinda duran adam su karsiligi verdi: "Naomi ile birlikte Moav topraklarindan gelen Moavli genç kadin budur.
\par 7 Bana gelip, 'Izin ver de basak devsireyim, orakçilarin ardindan gidip demetlerin arasindaki artiklari toplayayim dedi. Sabahtan simdiye kadar tarlada çalisip durdu, çardagin altinda pek az dinlendi."
\par 8 Bunun üzerine Boaz Rut'a, "Dinle, kizim" dedi, "Basak devsirmek için baska tarlaya gitme; buradan ayrilma. Burada, benim hizmetçi kizlarla birlikte kal.
\par 9 Gözün, orakçilarin biçtigi tarlada olsun; kizlarin ardindan git. Sana ilismesinler diye adamlarima buyruk verdim. Susayinca var git, kuyudan çektikleri suyla doldurduklari testilerden iç."
\par 10 Rut egilip yüzüstü yere kapandi. Boaz'a, "Bir yabanci oldugum halde bana neden yakinlik gösteriyor, bu iyiligi yapiyorsun?" dedi.
\par 11 Boaz söyle karsilik verdi: "Kocanin ölümünden sonra kaynanan için yaptigin her sey bana bir bir anlatildi. Anneni babani, dogdugun ülkeyi biraktin; önceden hiç tanimadigin bir halkin arasina geldin.
\par 12 RAB yaptiklarinin karsiligini versin. Kanatlari altina siginmak için kendisine geldigin Israil'in Tanrisi RAB seni cömertçe ödüllendirsin."
\par 13 Rut, "Bana çok iyi davrandin, efendim" dedi. "Kölelerinden biri olmadigim halde, söyledigin sözlerle beni teselli ettin, yüregimi oksadin."
\par 14 Yemek vakti gelince Boaz Rut'a, "Buraya yaklas, ekmek al, pekmeze batirip ye" dedi. Rut varip orakçilarin yanina oturdu. Boaz ona kavrulmus basak verdi. Rut bir kismini yedikten sonra doydu, birazini da artirdi.
\par 15 Basak devsirmek için kalkinca, Boaz adamlarina, "Demetler arasinda da basak devsirsin, ona dokunmayin" diye buyurdu.
\par 16 "Hatta onun için demetlerden basak ayirip yere birakin da devsirsin. Sakin onu azarlamayin."
\par 17 Böylece Rut aksama dek tarlada basak devsirdi. Devsirdigi basaklari dövünce bir efa kadar arpasi oldu.
\par 18 Bunu yüklenip kente döndü. Devsirdiklerini gören kaynanasina ayrica, tarlada doyduktan sonra artirdigi basaklari da çikarip verdi.
\par 19 Naomi, "Bugün nerede basak devsirdin, nerede çalistin?" diye sordu. "Sana bunca yakinlik göstermis olan her kimse, kutsansin!" Rut tarlasinda çalistigi adamdan söz ederek kaynanasina, "Bugün tarlasinda çalistigim adamin adi Boaz" dedi.
\par 20 Naomi gelinine, "RAB, sag kalanlardan da ölmüslerden de iyiligini esirgemeyen Boaz'i kutsasin" dedi. Sonra ekledi: "O adam akrabalarimizdan, yakin akrabalarimizdan biridir."
\par 21 Moavli Rut söyle konustu: "Üstelik bana, 'Adamlarim bütün biçme isini bitirinceye kadar onlarla birlikte kal dedi."
\par 22 Naomi, gelini Rut'a, "Kizim, onun kizlariyla gitmen daha iyi. Baska bir tarlada sana zarar gelebilir" dedi.
\par 23 Böylece Rut arpa ile bugday biçimi sonuna kadar Boaz'in hizmetçi kizlarindan ayrilmadi; basak devsirip kaynanasiyla oturmaya devam etti.

\chapter{3}

\par 1 Kaynanasi Naomi bir gün Rut'a, "Kizim, iyiligin için sana rahat edecegin bir yer aramam gerekmez mi?" dedi.
\par 2 "Hizmetçileriyle birlikte bulundugun Boaz akrabamiz degil mi? Bak simdi, bu aksam Boaz harman yerinde arpa savuracak.
\par 3 Yikan, kokular sürün, giyinip harman yerine git. Ama adam yemeyi içmeyi bitirene dek orada oldugunu belli etme.
\par 4 Adam yatip uyudugunda, nerede yattigini belle; sonra gidip onun ayaklarinin üzerindeki örtüyü kaldir ve oracikta yat. Ne yapman gerektigini o sana söyler."
\par 5 Rut ona, "Söyledigin her seyi yapacagim" diye karsilik verdi.
\par 6 Harman yerine giderek kaynanasinin her dedigini yapti.
\par 7 Boaz yiyip içti, keyfi yerine geldi. Sonra harman yigininin dibinde uyumaya gitti. Rut da gizlice yaklasti, onun ayaklarinin üzerindeki örtüyü kaldirip yatti.
\par 8 Gece yarisi adam ürktü; yattigi yerde dönünce ayaklarinin dibinde yatan kadini ayrimsadi.
\par 9 Ona, "Kimsin sen?" diye sordu. Kadin, "Ben kölen Rut'um" diye yanitladi. "Kölenle evlen. Çünkü sen yakin akrabamizsin" dedi.
\par 10 Boaz, "RAB seni kutsasin, kizim" dedi. "Bu son iyiligin, ilkinden de büyük. Çünkü yoksul olsun, zengin olsun, gençlerin pesinden gitmedin.
\par 11 Ve simdi, korkma kizim; her istedigini yapacagim. Bütün kent halki senin erdemli bir kadin oldugunu biliyor.
\par 12 Yakin akrabaniz oldugum dogrudur. Ama benden daha yakin biri var.
\par 13 Geceyi burada geçir. Sabah oldugunda eger adam senin için akrabalik görevini yaparsa ne âlâ, varsin yapsin. Ama o, akrabalik görevini yapmak istemezse, yasayan RAB'be ant içerim ki, bu görevi ben üstlenirim. Sen sabaha kadar yat."
\par 14 Böylece Rut sabaha kadar Boaz'in ayaklari dibinde yatti. Ama ortalik insanlarin birbirini seçebilecegi kadar aydinlanmadan önce kalkti. Çünkü Boaz, 'Harman yerine kadin geldigi bilinmemeli demisti.
\par 15 Boaz Rut'a, "Sirtindaki sali çikar, aç" dedi. Rut sali açinca Boaz içine alti ölçek arpa bosaltip onun sirtina yükledi. Sonra Rut kente döndü.
\par 16 Rut geri dönünce kaynanasi, "Nasil geçti kizim?" diye sordu. Rut, Boaz'in kendisi için yaptigi her seyi anlatti.
\par 17 Sonra ekledi: "'Kaynanana eli bos dönme diyerek bana bu alti ölçek arpayi da verdi."
\par 18 Naomi, "Kizim, bu isin ne olacagini ögreninceye kadar evde kal" dedi. "Çünkü Boaz bugün bu isi bitirmeden rahat edemeyecek."

\chapter{4}

\par 1 Bu arada Boaz kent kapisina gidip oturdu. Sözünü ettigi yakin akraba oradan geçerken ona, "Arkadas, gel suraya otur" diye seslendi. Adam da varip Boaz'in yanina oturdu.
\par 2 Sonra Boaz kentin ileri gelenlerinden on adam topladi. Onlara, "Siz de gelin, oturun" dedi. Adamlar da oturdular.
\par 3 Boaz, yakin akrabadan olan adama söyle dedi: "Moav topraklarindan dönmüs olan Naomi, akrabamiz Elimelek'in tarlasini satiyor.
\par 4 Ben de burada oturanlarin ve halkimin ileri gelenlerinin önünde bunu satin alman için durumu sana açayim dedim. Yakin akrabalik görevini yapmak istiyorsan, yap. Ama sen akrabalik görevini yerine getirmeyeceksen, söyle de bileyim. Çünkü bu görevi yapmak önce sana düser. Senden sonra ben gelirim." Adam, "Yakin akrabalik görevini ben yaparim" diye karsilik verdi.
\par 5 Bunun üzerine Boaz, "Yalniz, tarlayi Naomi'den satin aldigin gün, ölen Mahlon'un adinin biraktigi mirasla sürmesi için dul esi Moavli Rut'u da almalisin" dedi.
\par 6 Adam, "Bu durumda yakin akrabalik görevini yapamam; yaparsam kendi mirasimi tehlikeye atmis olurum" dedi. "Bana düsen akrabalik görevini sen yüklen. Çünkü ben yapamam."
\par 7 Eskiden Israil'de akrabalik görevinin yerine getirildigini ve mülk alim satiminin onaylandigini göstermek için taraflardan biri çarigini çikarip ötekine verirdi. Alisverisi yasallastirmanin yolu buydu.
\par 8 Bu nedenle yakin akrabadan olan adam, "Sen kendin satin al" diyerek çarigini çikarip Boaz'a verdi.
\par 9 Boaz, ileri gelenlere ve bütün halka, "Elimelek'in, Kilyon ile Mahlon'un bütün mülkünü Naomi'den satin aldigima bugün siz tanik oldunuz" dedi.
\par 10 "Mahlon'un dul karisi Moavli Rut'u da kendime es olarak aliyorum. Öyle ki, ölen Mahlon'un adi biraktigi mirasla birlikte sürsün; kardeslerinin arasindan ve yasadigi kentten adi silinmesin. Bugün siz buna tanik oldunuz."
\par 11 Kent kapisinda bulunan bütün halk ve ileri gelenler, "Evet, biz tanigiz" dediler. "RAB senin evine gelen kadini, Israil soyunun o iki ana diregine -Rahel ve Lea'ya- benzer kilsin. Efrat boyunda varlikli, Beytlehem'de ünlü olasin.
\par 12 RAB'bin bu genç kadindan sana verecegi çocuklarla senin soyun, Tamar'in Yahuda'ya dogurdugu Peres'in soyu gibi olsun."
\par 13 Böylece Boaz, Rut'u kendine es olarak aldi ve onunla birlesti. RAB'bin kutsamasiyla gebe kalan Rut bir ogul dogurdu.
\par 14 O zaman kadinlar Naomi'ye, "Bugün seni yakin akrabasiz birakmamis olan RAB'be övgüler olsun. Dogan çocugun ünü Israil'de yayilsin" dediler.
\par 15 "O seni yasama döndürecek, yasliliginda doyuracak. Çünkü onu, seni seven ve senin için yedi oguldan bile daha degerli olan gelinin dogurdu."
\par 16 Naomi çocugu alip bagrina basti ve ona dadilik yapti.
\par 17 Komsu kadinlar, "Naomi'nin bir oglu oldu" diyerek çocuga ad koydular; ona, Ovet adini verdiler. Ovet, Isay'in babasi; Isay ise Davut'un babasidir.
\par 18 Peres'in soyu söyledir: Peres Hesron'un babasi,
\par 19 Hesron Ram'in babasi, Ram Amminadav'in babasi,
\par 20 Amminadav Nahson'un babasi, Nahson Salmon'un babasi,
\par 21 Salmon Boaz'in babasi, Boaz Ovet'in babasi,
\par 22 Ovet Isay'in babasi, Isay da Davut'un babasidir.


\end{document}