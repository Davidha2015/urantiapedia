\begin{document}

\title{1 Samuel}


\chapter{1}

\par 1 Efrayim daglik bölgesindeki Ramatayim Kasabasi'nda yasayan, Efrayim oymaginin Suf boyundan Yeroham oglu Elihu oglu Tohu oglu Suf oglu Elkana adinda bir adam vardi.
\par 2 Elkana'nin Hanna ve Peninna adinda iki karisi vardi. Peninna'nin çocuklari oldugu halde, Hanna'nin çocugu olmuyordu.
\par 3 Elkana Her Seye Egemen RAB'be tapinip kurban sunmak üzere her yil kendi kentinden Silo'ya giderdi. Eli'nin RAB'bin kâhinleri* olan Hofni ve Pinehas adindaki iki oglu da oradaydi.
\par 4 Elkana kurban sundugu gün karisi Peninna'ya ve ogullariyla kizlarina etten birer pay verirken,
\par 5 Hanna'ya iki pay verirdi. Çünkü RAB Hanna'nin rahmini kapamasina karsin, Elkana onu severdi.
\par 6 Ama RAB Hanna'nin rahmini kapadigindan, kumasi Peninna Hanna'yi öfkelendirmek için ona sürekli satasirdi.
\par 7 Bu yillarca böyle sürdü. Hanna RAB'bin Tapinagi'na her gittiginde kumasi ona satasirdi. Böylece Hanna aglar, yemek yemezdi.
\par 8 Kocasi Elkana, "Hanna, neden agliyorsun, neden yemek yemiyorsun?" derdi, "Neden bu kadar üzgünsün? Ben senin için on oguldan daha iyi degil miyim?"
\par 9 Bir gün onlar Silo'da yiyip içtikten sonra, Hanna kalkti. Kâhin Eli RAB'bin Tapinagi'nin kapi sövesi yanindaki sandalyede oturuyordu.
\par 10 Hanna, gönlü buruk, aci aci aglayarak RAB'be yakardi
\par 11 ve su adagi adadi: "Ey Her Seye Egemen RAB, kulunun üzüntüsüne gerçekten bakip beni animsar, kulunu unutmayip bana bir erkek çocuk verirsen, yasami boyunca onu sana adayacagim. Onun basina hiç ustura degmeyecek."
\par 12 Hanna RAB'be yakarisini sürdürürken, Eli onun dudaklarini gözetliyordu.
\par 13 Hanna içinden yakariyor, yalniz dudaklari kimildiyor, sesi duyulmuyordu. Bu yüzden Eli, Hanna'yi sarhos sanarak,
\par 14 "Sarhoslugunu ne zamana dek sürdüreceksin? Artik sarabi birak" dedi.
\par 15 Hanna, "Ah, öyle degil efendim!" diye yanitladi, "Ben yüregi acilarla dolu bir kadinim. Ne sarap içtim, ne de baska bir içki. Sadece yüregimi RAB'be döküyordum.
\par 16 Kulunu kötü bir kadin sanma. Yakarisimi simdiye dek sürdürmemin nedeni çok kaygili, üzüntülü olmamdir."
\par 17 Eli, "Öyleyse esenlikle git" dedi, "Israil'in Tanrisi dilegini yerine getirsin."
\par 18 Hanna, "Senin gözünde lütuf bulayim" deyip yoluna gitti. Sonra yemek yedi. Artik üzgün degildi.
\par 19 Ertesi sabah erkenden kalkip RAB'be tapindilar. Ondan sonra Rama'daki evlerine döndüler. Elkana karisi Hanna'yla birlesti ve RAB Hanna'yi animsadi.
\par 20 Zamani gelince Hanna gebe kaldi ve bir erkek çocuk dogurdu. "Onu RAB'den diledim" diyerek adini Samuel koydu.
\par 21 Elkana RAB'be yillik kurbanini ve adagini sunmak üzere ev halkiyla birlikte Silo'ya gitti.
\par 22 Ama Hanna gitmedi. Kocasina, "Çocuk sütten kesildikten sonra onu RAB'bin hizmetinde bulunmak üzere götürecegim. Yasami boyunca orada kalacak" dedi.
\par 23 Kocasi Elkana, "Nasil istersen öyle yap" diye karsilik verdi, "Çocuk sütten kesilinceye dek burada kal. RAB sözünü yerine getirsin." Böylece Hanna oglu sütten kesilinceye dek evde kalip onu emzirdi.
\par 24 Küçük çocuk sütten kesildikten sonra Hanna üç yasinda bir boga, bir efa un ve bir tulum sarap alarak onu kendisiyle birlikte RAB'bin Silo'daki tapinagina götürdü.
\par 25 Bogayi kestikten sonra çocugu Eli'ye getirdiler.
\par 26 Hanna, "Ey efendim, yasamin hakki için derim ki, burada yaninda durup RAB'be yakaran kadinim ben" dedi,
\par 27 "Bu çocuk için yakarmistim; RAB dilegimi yerine getirdi.
\par 28 Ben de onu RAB'be adiyorum. Yasami boyunca RAB'be adanmis kalacaktir." Sonra çocuk orada RAB'be tapindi.

\chapter{2}

\par 1 Hanna söyle dua etti: "Yüregim RAB'de buldugum sevinçle cosuyor; Gücümü yükselten RAB'dir. Düsmanlarimin karsisinda övünüyor, Kurtarisinla seviniyorum!
\par 2 Kutsallikta RAB'bin benzeri yok, Evet, senin gibisi yok, ya RAB! Tanrimiz gibi dayanak yok.
\par 3 Artik büyük konusmayin, Agzinizdan küstahça sözler çikmasin. Çünkü RAB her seyi bilen Tanri'dir; O'dur davranislari tartan.
\par 4 Güçlülerin yaylari kirilir; Güçsüzlerse güçle donatilir.
\par 5 Toklar yiyecek ugruna gündelikçi olur, Açlar doyurulur. Kisir kadin yedi çocuk dogururken, Çok çocuklu kadin kimsesiz kalir.
\par 6 RAB öldürür de diriltir de, Ölüler diyarina indirir ve çikarir.
\par 7 O kimini yoksul, kimini varsil kilar; Kimini alçaltir, kimini yükseltir.
\par 8 Düskünü yerden kaldirir, Yoksulu çöplükten çikarir; Soylularla oturtsun Ve kendilerine onur tahtini miras olarak bagislasin diye. Çünkü yeryüzünün temelleri RAB'bindir, O dünyayi onlarin üzerine kurmustur.
\par 9 RAB sadik kullarinin adimlarini korur, Ama kötüler karanlikta susturulur. Çünkü güçle zafere ulasamaz insan.
\par 10 RAB'be karsi gelenler paramparça olacak, RAB onlara karsi gökleri gürletecek, Bütün dünyayi yargilayacak, Kralini güçle donatacak, Meshettigi* kralin gücünü yükseltecek."
\par 11 Sonra Elkana Rama'ya, evine döndü. Küçük Samuel ise Kâhin Eli'nin gözetiminde RAB'bin hizmetinde kaldi. Eli'nin Ogullarinin Yaptigi Kötülükler
\par 12 Eli'nin ogullari degersiz kisilerdi. RAB'bi ve kâhinlerin halkla ilgili kurallarini önemsemiyorlardi. Biri sundugu kurbanin etini haslarken, kâhinin hizmetkâri elinde üç disli büyük bir çatalla gelir,
\par 14 çatali kap, tencere, tava ya da kazana daldirirdi. Çatalla çikarilan her sey kâhin için ayirilirdi. Silo'ya gelen Israilliler'in hepsine böyle davranirlardi.
\par 15 Üstelik kurbanin yaglari yakilmadan önce, kâhinin hizmetkâri gelip kurban sunan adama, "Kâhine kizartmalik et ver. Senden haslanmis et degil, çig et alacak" derdi.
\par 16 Kurban sunan, "Önce hayvanin yaglari yakilmali, sonra diledigin kadar al" diyecek olsa, hizmetkâr, "Hayir, simdi vereceksin, yoksa zorla alirim" diye karsilik verirdi.
\par 17 Gençlerin RAB'be karsi isledikleri günah çok büyüktü; çünkü RAB'be sunulan sunulari küçümsüyorlardi.
\par 18 Bu arada genç Samuel, keten efod* giymis, RAB'bin önünde hizmet ediyordu.
\par 19 Yillik kurbani sunmak için annesi her yil kocasiyla birlikte oraya gider, diktigi cüppeyi ogluna getirirdi.
\par 20 Kâhin Eli de, Elkana ile karisina iyi dilekte bulunarak, "Diledigi ve RAB'be adadigi çocugun yerine RAB sana bu kadindan baska çocuklar versin" derdi. Bundan sonra evlerine dönerlerdi.
\par 21 RAB'bin lütfuna eren Hanna gebe kalip üç erkek, iki kiz daha dogurdu. Küçük Samuel ise RAB'bin hizmetinde büyüdü.
\par 22 Eli artik çok yaslanmisti. Ogullarinin Israilliler'e bütün yaptiklarini, Bulusma Çadiri'nin girisinde görevli kadinlarla düsüp kalktiklarini duymustu.
\par 23 Onlara, "Neden böyle seyler yapiyorsunuz?" dedi, "Yaptiginiz kötülükleri herkesten isitiyorum.
\par 24 Olmaz bu, ogullarim! RAB'bin halki arasinda yayildigini duydugum haber iyi degil.
\par 25 Insan insana karsi günah islerse, Tanri onun için aracilik yapar. Ama RAB'be karsi günah isleyeni kim savunacak?" Ne var ki, onlar babalarinin sözünü dinlemediler. Çünkü RAB onlari öldürmek istiyordu.
\par 26 Bu arada giderek büyüyen genç Samuel RAB'bin de halkin da begenisini kazanmaktaydi.
\par 27 O siralarda bir Tanri adami Eli'ye gelip söyle dedi: "RAB diyor ki, 'Atan ve soyu Misir'da firavunun halkina kölelik ederken kendimi onlara açikça göstermedim mi?
\par 28 Sunagima çikmasi, buhur yakip önümde efod* giymesi için bütün Israil oymaklari arasindan yalniz atani kendime kâhin seçtim. Üstelik Israilliler'in yakilan bütün sunularini da atanin soyuna verdim.
\par 29 Öyleyse neden konutum için buyurdugum kurbani ve sunuyu küçümsüyorsunuz? Halkim Israil'in sundugu bütün sunularin en iyi kisimlariyla kendinizi semirterek neden ogullarini benden daha çok sayiyorsun?
\par 30 "Bu nedenle Israil'in Tanrisi RAB söyle diyor: 'Gerçekten, ailen ve atanin soyu sonsuza dek bana hizmet edecekler demistim. Ama simdi RAB söyle buyuruyor: 'Bu benden uzak olsun! Beni onurlandirani ben de onurlandiririm. Ama beni saymayan küçük düsürülecek.
\par 31 Soyundan hiç kimsenin yaslanacak kadar yasamamasi için senin ve atanin soyunun gücünü kiracagim günler yaklasiyor.
\par 32 Israil'e yapilacak bütün iyilige karsin, sen konutumda sikinti göreceksin. Artik soyundan hiç kimse yaslanacak kadar yasamayacak.
\par 33 Sunagimdan bütün soyunu yok edecegim, yalniz bir kisiyi esirgeyecegim. Gözleri aglamaktan kör olacak, yüregi yanacak. Ama soyundan gelenlerin hepsi kiliçla ölecekler.
\par 34 Iki oglun Hofni ile Pinehas'in basina gelecek olay senin için bir belirti olacak: Ikisi de ayni gün ölecek.
\par 35 Isteklerimi ve amaçlarimi yerine getirecek güvenilir bir kâhin çikaracagim kendime. Onun soyunu sürdürecegim; o da meshettigim* kisinin önünde sürekli hizmet edecek.
\par 36 Ailenden sag kalan herkes bir parça gümüs ve bir somun ekmek için gelip ona boyun egecek ve, Ne olur, karin tokluguna beni herhangi bir kâhinlik görevine ata! diye yalvaracak."

\chapter{3}

\par 1 Genç Samuel Eli'nin yönetimi altinda RAB'be hizmet ediyordu. O günlerde RAB'bin sözü seyrek geliyordu; görümler de azalmisti.
\par 2 Bir gece Eli yataginda uyuyordu. Gözleri öyle zayiflamisti ki, güçlükle görebiliyordu.
\par 3 Samuel ise RAB'bin Tapinagi'nda, Tanri'nin Sandigi'nin bulundugu yerde uyuyordu. Tanri'nin kandili daha sönmemisti.
\par 4 RAB Samuel'e seslendi. Samuel, "Buradayim" diye karsilik verdi.
\par 5 Ardindan Eli'ye kosup, "Beni çagirdin, iste buradayim" dedi. Ama Eli, "Ben çagirmadim, dön yat" diye karsilik verdi. Samuel de dönüp yatti.
\par 6 RAB yine, "Samuel!" diye seslendi. Samuel kalkip Eli'ye gitti ve, "Iste, buradayim, beni çagirdin" dedi. Eli, "Çagirmadim, oglum" diye karsilik verdi, "Dön yat."
\par 7 Samuel RAB'bi daha tanimiyordu; RAB'bin sözü henüz ona açiklanmamisti.
\par 8 RAB yine üçüncü kez Samuel'e seslendi. Samuel kalkip Eli'ye gitti. "Iste buradayim, beni çagirdin" dedi. O zaman Eli genç Samuel'e RAB'bin seslendigini anladi.
\par 9 Bunun üzerine Samuel'e, "Git yat" dedi, "Sana yine seslenirse, 'Konus, ya RAB, kulun dinliyor dersin." Samuel gidip yerine yatti.
\par 10 RAB gelip orada durdu ve önceki gibi, "Samuel, Samuel!" diye seslendi. Samuel, "Konus, kulun dinliyor" diye yanitladi.
\par 11 RAB Samuel'e söyle dedi: "Ben Israil'de her duyani saskina çevirecek bir sey yapmak üzereyim.
\par 12 O gün Eli'nin ailesine karsi söyledigim her seyi bastan sona dek yerine getirecegim.
\par 13 Çünkü farkinda oldugu günahtan ötürü ailesini sonsuza dek yargilayacagimi Eli'ye bildirdim. Ogullari Tanri'ya saygisizlik ettiler. Eli de onlara engel olmadi.
\par 14 Bu nedenle, 'Eli'nin ailesinin günahi hiçbir zaman kurban ya da sunuyla bile bagislanmayacaktir diyerek Eli'nin ailesi hakkinda ant içtim."
\par 15 Samuel sabaha kadar yatti, sonra RAB'bin Tapinagi'nin kapilarini açti. Gördügü görümü Eli'ye söylemekten çekiniyordu.
\par 16 Ama Eli ona, "Oglum Samuel!" diye seslendi. Samuel, "Iste buradayim" diye yanitladi.
\par 17 Eli, "RAB sana neler söyledi?" diye sordu, "Lütfen benden gizleme. Sana söylediklerinden birini bile benden gizlersen, Tanri sana aynisini, hatta daha kötüsünü yapsin!"
\par 18 Bunun üzerine Samuel hiçbir sey gizlemeden ona her seyi anlatti. Eli de, "O RAB'dir, gözünde iyi olani yapsin" dedi.
\par 19 Samuel büyürken RAB onunla birlikteydi. RAB ona verdigi sözlerin hiçbirinin bosa çikmasina izin vermedi.
\par 20 Samuel'in RAB'bin bir peygamberi olarak onaylandigini Dan'dan Beer-Seva'ya kadar bütün Israil anladi.
\par 21 RAB Silo'da görünmeyi sürdürdü. Orada sözü araciligiyla kendisini Samuel'e tanitti.

\chapter{4}

\par 1 Samuel'in sözü bütün Israil'de yayildi. Israilliler Filistliler'le savasmak üzere yola çiktilar. Israilliler Even-Ezer'de, Filistliler de Afek'te ordugah kurdu.
\par 2 Filistliler Israil'e karsi savas düzenine girdiler. Savas her yere yayilinca, Filistliler Israilliler'i bozguna ugratti. Savas alaninda dört bine yakin Israilli'yi öldürdüler.
\par 3 Askerler ordugaha dönünce, Israil'in ileri gelenleri, "Neden bugün RAB bizi Filistliler'in önünde bozguna ugratti?" diye sordular, "RAB'bin Antlasma Sandigi'ni* Silo'dan buraya getirelim ki, aramiza geldiginde bizi düsmanlarimizin elinden kurtarsin."
\par 4 Halk Silo'ya adamlar gönderdi. Keruvlar* arasinda taht kurmus, Her Seye Egemen RAB'bin Antlasma Sandigi'ni oradan getirdiler. Eli'nin iki oglu, Hofni ile Pinehas da Tanri'nin Antlasma Sandigi'nin yanindaydilar.
\par 5 RAB'bin Antlasma Sandigi ordugaha varinca, bütün Israilliler öyle yüksek sesle bagirdilar ki, yer yerinden oynadi.
\par 6 Filistliler bagrismalari duyunca, "Ibraniler'in ordugahindaki bu yüksek bagrismalarin anlami ne?" diye sordular. RAB'bin Sandigi'nin ordugaha getirildigini ögrenince,
\par 7 korkarak, "Tanrilar ordugaha gelmis" dediler, "Vay basimiza! Daha önce buna benzer bir olay olmamisti.
\par 8 Vay basimiza! Bu güçlü tanrilarin elinden bizi kim kurtarabilir? Çölde Misirlilar'i her tür belaya çarptiran tanrilar bunlar.
\par 9 Güçlü olun, ey Filistliler! Yigitçe davranin! Yoksa, Ibraniler size nasil boyun egdiyse, siz de onlara öyle boyun egeceksiniz. Bu yüzden yigitçe davranin ve savasin!"
\par 10 Böylece Filistliler savasip Israilliler'i bozguna ugrattilar. Israilliler'in hepsi evlerine kaçti. Yenilgi öyle büyüktü ki, Israilliler otuz bin yaya asker yitirdi,
\par 11 Tanri'nin Sandigi alindi, Eli'nin iki oglu, Hofni ile Pinehas öldü.
\par 12 Benyaminli bir adam savas alanindan kosarak ayni gün Silo'ya ulasti. Giysileri yirtilmis, basi toz toprak içindeydi.
\par 13 Adam Silo'ya vardiginda, Tanri'nin Sandigi için yüregi titreyen Eli, yol kenarinda bir sandalyeye oturmus, kaygiyla bekliyordu. Adam kente girip olup bitenleri anlatinca, kenttekilerin tümü haykirdi.
\par 14 Eli haykirislari duyunca, "Bu gürültünün anlami ne?" diye sordu. Adam olanlari Eli'ye bildirmek için hemen onun yanina geldi.
\par 15 O sirada Eli doksan sekiz yasindaydi. Gözleri zayiflamis, göremiyordu.
\par 16 Adam Eli'ye, "Ben savas alanindan geliyorum" dedi, "Savas alanindan bugün kaçtim." Eli, "Ne oldu, oglum?" diye sordu.
\par 17 Haber getiren adam söyle yanitladi: "Israilliler Filistliler'in önünden kaçti. Askerler büyük bir yenilgiye ugradi. Iki oglun, Hofni'yle Pinehas öldü. Tanri'nin Sandigi da ele geçirildi."
\par 18 Adam Tanri'nin Sandigi'ndan söz edince, Eli sandalyeden geriye, kapinin yanina düstü. Yasli ve sisman oldugundan boynu kirilip öldü. Israil halkini kirk yil süreyle yönetmisti.
\par 19 Eli'nin gelini -Pinehas'in karisi- gebeydi, dogurmak üzereydi. Tanri'nin Sandigi'nin ele geçirildigini, kayinbabasiyla kocasinin öldügünü duyunca birden sancilari tuttu, yere çömelip dogurdu.
\par 20 Ölmek üzereyken ona yardim eden kadinlar, "Korkma, bir oglun oldu" dediler. Ama o aldirmadi, karsilik da vermedi.
\par 21 Tanri'nin Sandigi ele geçirilmis, kayinbabasiyla kocasi ölmüstü. Bu yüzden, "Yücelik Israil'den ayrildi!" diyerek çocuga Ikavot adini verdi.
\par 22 "Yücelik Israil'den ayrildi!" dedi, "Çünkü Tanri'nin Sandigi ele geçirildi."

\chapter{5}

\par 1 Filistliler, Tanri'nin Sandigi'ni ele geçirdikten sonra, onu Even-Ezer'den Asdot'a götürdüler.
\par 2 Tanri'nin Sandigi'ni Dagon Tapinagi'na tasiyip Dagon heykelinin yanina yerlestirdiler.
\par 3 Ertesi gün erkenden kalkan Asdotlular, Dagon'u RAB'bin Sandigi'nin önünde yüzüstü yere düsmüs buldular. Dagon'u alip yerine koydular.
\par 4 Ama ertesi sabah erkenden kalktiklarinda, Dagon'u yine RAB'bin Sandigi'nin önünde yüzüstü yere düsmüs buldular. Bu kez Dagon'un basiyla iki eli kirilmis, esigin üzerinde duruyordu; yalnizca gövdesi kalmisti.
\par 5 Dagon kâhinleri de, Asdot'taki Dagon Tapinagi'na bütün gelenler de bu yüzden bugün de tapinagin esigine basmiyorlar.
\par 6 RAB Asdotlular'i ve çevrelerindeki halki agir biçimde cezalandirdi; onlari urlarla cezalandirip sikintiya soktu.
\par 7 Asdotlular olup bitenleri görünce, "Israil Tanrisi'nin Sandigi yanimizda kalmamali; çünkü O bizi de, ilahimiz Dagon'u da agir bir biçimde cezalandiriyor" dediler.
\par 8 Bunun üzerine ulaklar gönderip bütün Filist beylerini çagirttilar ve, "Israil Tanrisi'nin Sandigi'ni ne yapalim?" diye sordular. Filist beyleri, "Israil Tanrisi'nin Sandigi Gat'a götürülsün" dediler. Böylece Israil Tanrisi'nin Sandigi'ni Gat'a götürdüler.
\par 9 Ama sandik oraya götürüldükten sonra, RAB o kenti de cezalandirdi. Kenti çok büyük bir korku sardi. RAB kent halkini, büyük küçük herkesi urlarla cezalandirdi.
\par 10 Bu yüzden Tanri'nin Sandigi'ni Ekron'a gönderdiler. Tanri'nin Sandigi kente girer girmez Ekronlular, "Bizi ve halkimizi yok etmek için Israil Tanrisi'nin Sandigi'ni bize getirdiler!" diye bagirdilar.
\par 11 Bütün Filist beylerini toplayarak, "Israil Tanrisi'nin Sandigi'ni buradan uzaklastirin" dediler, "Sandik yerine geri gönderilsin; öyle ki, bizi de halkimizi da yok etmesin." Çünkü kentin her yanini ölüm korkusu sarmisti. Tanri'nin onlara verdigi ceza çok agirdi.
\par 12 Sag kalanlarda urlar çikti. Kent halkinin haykirisi göklere yükseldi.

\chapter{6}

\par 1 RAB'bin Sandigi Filist ülkesinde yedi ay kaldiktan sonra,
\par 2 Filistliler kâhinlerle falcilari çagirtip, "RAB'bin Sandigi'ni ne yapalim? Onu nasil yerine gönderecegimizi bize bildirin" dediler.
\par 3 Kâhinlerle falcilar, "Israil Tanrisi'nin Sandigi'ni geri gönderecekseniz, bos göndermeyin" diye yanitladilar, "O'na bir suç sunusu* sunmalisiniz. O zaman iyilesecek ve O'nun sizi neden sürekli cezalandirdigini anlayacaksiniz."
\par 4 Filistliler, "Ona suç sunusu olarak ne göndermeliyiz?" diye sordular. Kâhinlerle falcilar, "Suç sununuz Filist beylerinin sayisina göre bes altin ur ve bes altin fare olsun" diye yanitladilar, "Çünkü ayni bela hepinizin de, beylerinizin de üzerindedir.
\par 5 Onun için, urlarin ve ülkeyi yikan farelerin benzerlerini yapin. Böylelikle Israil'in Tanrisi'ni onurlandirin. Belki sizin, ilahlarinizin ve ülkenizin üzerindeki cezayi hafifletir.
\par 6 Neden Misirlilar'in ve firavunun yaptigi gibi inat ediyorsunuz? Tanri Misirlilar'i alaya aldiktan sonra, Israil halkinin Misir'dan çikmasi için onlari serbest birakmadilar mi?
\par 7 "Simdi yeni bir arabayla boyunduruk vurulmamis, süt veren iki inek hazirlayin. Inekleri arabaya kosun; buzagilarini artlarindan ayirip ahira götürün.
\par 8 RAB'bin Sandigi'ni alip arabaya koyun; suç sunusu olarak O'na göndereceginiz altin esyalari da bir kutuya koyup yanina yerlestirin. Sonra birakin arabayi yoluna gitsin.
\par 9 Ama ardindan gözetleyin. Eger kendi ülkesine, Beytsemes'e giden yoldan ilerlerse, demek ki, üzerimize bu büyük yikimi getiren O'dur. Yoksa bu yikimin O'ndan gelmedigini, bize bir rastlanti oldugunu anlayacagiz."
\par 10 Adamlar denileni yaptilar. Süt veren iki inek getirip arabaya kostular, buzagilarini da ahira kapadilar.
\par 11 Içinde farelerle urlarin altin benzerlerinin bulundugu kutuyu RAB'bin Sandigi'yla birlikte arabaya koydular.
\par 12 Inekler dosdogru Beytsemes yolundan gittiler. Saga sola sapmadan, bögüre bögüre ana yoldan ilerlediler. Filist beyleri onlari Beytsemes sinirina dek izledi.
\par 13 O sirada Beytsemesliler vadide bugday biçiyorlardi. Gözlerini kaldirip sandigi görünce sevindiler.
\par 14 Beytsemesli Yesu'nun tarlasina giren araba oradaki büyük bir tasin yaninda durdu. Beytsemesliler arabanin odununu yardilar, inekleri de RAB'be yakmalik sunu* olarak sundular.
\par 15 Levililer RAB'bin Sandigi'ni ve içinde altin esyalarin bulundugu yanindaki kutuyu indirip büyük tasin üzerine koymuslardi. O gün Beytsemesliler RAB'be yakmalik sunular sunup kurbanlar kestiler.
\par 16 Filistliler'in bes beyi olup bitenleri gördükten sonra ayni gün Ekron'a döndüler.
\par 17 Filistliler Asdot, Gazze, Askelon, Gat ve Ekron kentleri için RAB'be suç sunusu olarak ur biçiminde birer altin gönderdiler.
\par 18 Altin farelerse, surlu kentlerle çevre köyler dahil bes Filistli beye ait kentlerin sayisi kadardi. Beytsemesli Yesu'nun tarlasinda RAB'bin Antlasma Sandigi'nin* üzerine kondugu büyük tas tanik olarak bugün de duruyor.
\par 19 RAB'bin Antlasma Sandigi'nin içine baktiklari için, RAB Beytsemesliler'den bazilarini cezalandirip yetmis kisiyi yok etti. Halk RAB'bin baslarina getirdigi bu büyük yikimdan dolayi yas tuttu.
\par 20 Beytsemesliler, "Bu kutsal Tanri'nin, RAB'bin önünde kim durabilir? Bizden sonra kime gidecek?" diyorlardi.
\par 21 Sonunda Kiryat-Yearim'de oturanlara ulaklar göndererek, "Filistliler RAB'bin Sandigi'ni geri getirdiler; gelin, onu alip götürün" dediler.

\chapter{7}

\par 1 Bunun üzerine Kiryat-Yearim halki varip RAB'bin Sandigi'ni aldi. Onu Avinadav'in tepedeki evine götürdüler. RAB'bin Antlasma Sandigi'na bakmasi için Avinadav oglu Elazar'i görevlendirdiler.
\par 2 Sandik uzun bir süre, yirmi yil boyunca Kiryat-Yearim'de kaldi. Bu arada bütün Israil halki RAB'bin özlemini çekti.
\par 3 Samuel Israil halkina söyle dedi: "Eger bütün yüreginizle RAB'be dönmeye istekliyseniz, yabanci ilahlari ve Astoret'in* putlarini aranizdan kaldirin. Kendinizi RAB'be adayip yalniz O'na kulluk edin. RAB de sizi Filistliler'in elinden kurtaracaktir."
\par 4 Bunun üzerine Israilliler Baal'in* ve Astoret'in putlarini atip yalnizca RAB'be kulluk etmeye basladilar.
\par 5 O zaman Samuel, "Bütün Israil halkini Mispa'da toplayin, ben de sizin için RAB'be yakaracagim" dedi.
\par 6 Mispa'da toplanan Israilliler kuyudan su çekip RAB'bin önüne döktüler. O gün oruç* tuttular ve, "RAB'be karsi günah isledik" dediler. Samuel Mispa'da Israil halkina önderlik etti.
\par 7 Filistliler Israil halkinin Mispa'da toplandigini duydular. Filist beyleri Israilliler'e karsi savasmaya çiktilar. Israilliler bunu duyunca Filistliler'den korktular.
\par 8 Samuel'e, "Bizi Filistliler'in elinden kurtarmasi için Tanrimiz RAB'be yakarmayi birakma" dediler.
\par 9 Bunun üzerine Samuel bir süt kuzusu alip RAB'be tümüyle yakmalik sunu* olarak sundu ve Israilliler adina RAB'be yakardi. RAB de ona karsilik verdi.
\par 10 Samuel yakmalik sunuyu sunarken, Filistliler, Israilliler'e saldirmak üzere yaklasmislardi. Ama RAB o an korkunç bir sesle gürleyerek Filistliler'i öyle saskina çevirdi ki, Israilliler'in önünde bozguna ugradilar.
\par 11 Mispa'dan çikan Israilliler Filistliler'i Beytkar'in altina kadar kovalayip öldürdüler.
\par 12 Samuel bir tas alip Mispa ile Sen arasina dikti. "RAB buraya kadar bize yardim etmistir" diyerek tasa Even-Ezer adini verdi.
\par 13 Yenilgiye ugrayan Filistliler bir daha Israil topraklarina saldirmadilar. Samuel yasadigi sürece RAB Filistliler'in saldirmasini engelledi.
\par 14 Ekron'dan Gat'a kadar Filistliler'in ele geçirdigi kentler Israil'e geri verildi. Bunun yanisira Israil'in sinir topraklari da Filistliler'in elinden kurtarildi. Israilliler'le Amorlular arasinda ise baris vardi.
\par 15 Samuel yasadigi sürece Israil'e önderlik yapti.
\par 16 Her yil gidip Beytel'i, Gilgal'i, Mispa'yi dolasir, bu kentlerden Israil'i yönetirdi.
\par 17 Sonra Rama'daki evine döner, Israil'i oradan yönetirdi. Orada RAB'be bir sunak yapti.

\chapter{8}

\par 1 Samuel yaslaninca ogullarini Israil'e önder atadi.
\par 2 Beer-Seva'da görev yapan ilk oglunun adi Yoel, ikinci oglunun adiysa Aviya'ydi.
\par 3 Ama ogullari onun yolunda yürümediler. Tersine, haksiz kazanca yönelip rüsvet alir, yargida yan tutarlardi.
\par 4 Bu yüzden Israil'in bütün ileri gelenleri toplanip Rama'ya, Samuel'in yanina vardilar.
\par 5 Ona, "Bak, sen yaslandin" dediler, "Ogullarin da senin yolunda yürümüyor. Simdi, öteki uluslarda oldugu gibi, bizi yönetecek bir kral ata."
\par 6 Ne var ki, "Bizi yönetecek bir kral ata" demeleri Samuel'in hosuna gitmedi. Samuel RAB'be yakardi.
\par 7 RAB, Samuel'e su karsiligi verdi: "Halkin sana bütün söylediklerini dinle. Çünkü reddettikleri sen degilsin; krallari olarak beni reddettiler.
\par 8 Onlari Misir'dan çikardigim günden bu yana bütün yaptiklarinin aynisini sana da yapiyorlar. Beni birakip baska ilahlara kulluk ettiler.
\par 9 Simdi onlari dinle. Ancak onlari açikça uyar ve kendilerine krallik yapacak kisinin onlari nasil yönetecegini söyle."
\par 10 Samuel kendisinden kral isteyen halka RAB'bin bütün söylediklerini bildirdi:
\par 11 "Size krallik yapacak kisinin yönetimi söyle olacak: Ogullarinizi alip savas arabalarinda ve atli birliklerinde görevlendirecek. Onun savas arabalarinin önünde kosacaklar.
\par 12 Bazilarini biner, bazilarini elliser kisilik birliklere komutan atayacak. Kimisini topragini sürüp ekinini biçmek, kimisini de silahlarin ve savas arabalarinin donatimini yapmak için görevlendirecek.
\par 13 Kizlarinizi itriyatçi, asçi, firinci olmak üzere alacak.
\par 14 Seçkin tarlalarinizi, baglarinizi, zeytinliklerinizi alip hizmetkârlarina verecek.
\par 15 Tahillarinizin, üzümlerinizin ondaligini alip saray görevlileriyle öbür hizmetkârlarina dagitacak.
\par 16 Kadin erkek kölelerinizi, seçkin bogalarinizi, eseklerinizi alip kendi isinde çalistiracak.
\par 17 Sürülerinizin de ondaligini alacak. Sizler ise onun köleleri olacaksiniz.
\par 18 Bunlar gerçeklestiginde, seçtiginiz kral yüzünden feryat edeceksiniz. Ama RAB o gün size karsilik vermeyecek."
\par 19 Ne var ki, halk Samuel'in sözünü dinlemek istemedi. "Hayir, bizi yönetecek bir kral olsun" dediler,
\par 20 "Böylece biz de bütün uluslar gibi olacagiz. Kralimiz bizi yönetecek, önümüzden gidip savaslarimizi sürdürecek."
\par 21 Halkin bütün söylediklerini dinleyen Samuel, bunlari RAB'be aktardi.
\par 22 RAB Samuel'e, "Onlarin sözünü dinle ve baslarina bir kral ata" diye buyurdu. Bunun üzerine Samuel Israilliler'e, "Herkes kendi kentine dönsün" dedi.

\chapter{9}

\par 1 Benyamin oymagindan Afiyah oglu Bekorat oglu Seror oglu Aviel oglu Kis adinda bir adam vardi. Benyaminli Kis sözü geçen biriydi.
\par 2 Saul adinda genç, yakisikli bir oglu vardi. Israil halki arasinda ondan daha yakisiklisi yoktu. Boyu herkesten bir bas daha uzundu.
\par 3 Bir gün Saul'un babasi Kis'in esekleri kayboldu. Kis, oglu Saul'a, "Hizmetkârlardan birini yanina al da git, esekleri ara" dedi.
\par 4 Saul Efrayim daglik bölgesinden geçip Salisa topraklarini dolasti. Ama esekleri bulamadilar. Saalim bölgesine geçtiler. Esekler orada da yoktu. Sonra Benyamin bölgesinden geçtilerse de, hayvanlari bulamadilar.
\par 5 Suf bölgesine varinca, Saul yanindaki hizmetkârina, "Haydi dönelim! Yoksa babam esekleri düsünmekten vazgeçip bizim için kaygilanmaya baslar" dedi.
\par 6 Hizmetkâr, "Bak, bu kentte saygin bir Tanri adami vardir" diye karsilik verdi, "Bütün söyledikleri bir bir yerine geliyor. Simdi ona gidelim. Belki gidecegimiz yolu o bize gösterir."
\par 7 Saul, "Gidersek, adama ne götürecegiz?" dedi, "Torbalarimizdaki ekmek tükendi. Tanri adamina götürecek bir armaganimiz yok. Neyimiz kaldi ki?"
\par 8 Hizmetkâr, "Bak, bende çeyrek sekel*fm* gümüs var" diye karsilik verdi, "Gidecegimiz yolu bize göstermesi için bunu Tanri adamina verecegim."
\par 9 -Eskiden Israil'de biri Tanri'ya bir sey sormak istediginde, "Haydi, biliciye* gidelim" derdi. Çünkü bugün peygamber denilene o zaman bilici denirdi.-
\par 10 Saul hizmetkârina, "Iyi, haydi gidelim" dedi. Böylece Tanri adaminin yasadigi kente gittiler.
\par 11 Yokustan kente dogru çikarlarken, kuyudan su çekmeye giden kizlarla karsilastilar. Onlara, "Bilici burada mi?" diye sordular.
\par 12 Kizlar, "Evet, ilerde" diye karsilik verdiler, "Simdi çabuk davranin. Kentimize bugün geldi. Çünkü halk bugün tapinma yerinde bir kurban sunacak.
\par 13 Kente girer girmez, yemek için tapinma yerine çikmadan önce onu bulacaksiniz. Kurbani o kutsayacagi için, kendisi gelmeden halk yemek yemez. Çagrili olanlar o geldikten sonra yemeye baslar. Simdi gidin, onu hemen ulursunuz."
\par 14 Saul'la hizmetkâri kente gittiler. Kente girdiklerinde, tapinma yerine çikmaya hazirlanan Samuel onlara dogru ilerliyordu.
\par 15 Saul gelmeden bir gün önce RAB Samuel'e sunu açiklamisti:
\par 16 "Yarin bu saatlerde sana Benyamin bölgesinden birini gönderecegim. Onu halkim Israil'in önderi olarak meshedeceksin*. Halkimi Filistliler'in elinden o kurtaracak. Halkimin durumuna baktim; çünkü haykirislari bana ulasti."
\par 17 Samuel Saul'u görünce, RAB, "Iste sana sözünü ettigim adam!" dedi, "Halkima o önderlik edecek."
\par 18 Saul kent kapisinda duran Samuel'e yaklasti. "Bilicinin evi nerede, lütfen söyler misin?" dedi.
\par 19 Samuel, "Bilici benim" diye yanitladi, "Önümden tapinma yerine çikin. Bugün benimle birlikte yemek yiyeceksiniz. Yarin sabah düsündügün her seyi sana bildirip seni geri gönderirim.
\par 20 Üç gün önce kaybolan eseklerin için kaygilanma. Onlar bulundu. Israil'in özlemi kime yönelik? Sana ve babanin ailesine degil mi?"
\par 21 Saul su karsiligi verdi: "Ben Israil oymaklarinin en küçügü olan Benyamin oymagindan degil miyim? Ait oldugum boy da Benyamin oymagina bagli bütün boylarin en küçügü degil mi? Bana neden böyle seyler söylüyorsun?"
\par 22 Samuel Saul ile hizmetkârini alip yemek odasina götürdü; yaklasik otuz çagrili arasinda ilk sirayi onlara verdi.
\par 23 Sonra asçiya, "Sana verdigim ve bir kenara ayirmani söyledigim payi getir" dedi.
\par 24 Asçi budu getirip Saul'un önüne koydu. Samuel, "Iste senin için ayrilan parça, buyur ye!" dedi, "Çünkü bunu belirtilen gün çagirdigim halkla birlikte yemen için sakladim." O gün Saul Samuel'le yemek yedi.
\par 25 Tapinma yerinden kente indikten sonra Samuel evinin daminda Saul'la konustu.
\par 26 Sabah erkenden, safak sökerken kalktilar. Samuel, damdan Saul'u çagirip, "Hazirlan, seni gönderecegim" dedi. Saul kalkti. Samuel'le birlikte disari çiktilar.
\par 27 Kentin sinirina yaklasirken Samuel Saul'a, "Hizmetkâra önümüzden gitmesini söyle" dedi. Hizmetkâr öne geçince, Samuel, "Ama sen dur" diye ekledi, "Sana Tanri'nin sözünü bildirecegim."

\chapter{10}

\par 1 Sonra Samuel yag kabini alip yagi Saul'un basina döktü. Onu öpüp söyle dedi: "RAB seni kendi halkina önder olarak meshetti.
\par 2 Bugün benden ayrildiktan sonra Benyamin sinirinda, Selsah'taki Rahel'in mezari yaninda iki kisiyle karsilasacaksin. Sana, 'Aramaya çiktigin esekler bulundu diyecekler, 'Baban esekleri düsünmekten vazgeçti, oglum için ne yapsam diye sizin için kaygilanmaya basladi.
\par 3 Oradan daha ilerleyip Tavor'daki mese agacina varacaksin. Orada biri üç oglak, biri üç somun ekmek, öbürü de bir tulum sarapla Tanri'nin huzuruna, Beytel'e çikan üç adamla karsilasacaksin.
\par 4 Seni selamlayip iki somun ekmek verecekler. Sen de kabul edeceksin.
\par 5 Sonra Filist ordugahinin bulundugu Givat-Elohim'e varacaksin. Kente girince, önlerinde çenk, tef, kaval ve lir çalanlarla birlikte peygamberlik ederek tapinma yerinden inen bir peygamber topluluguyla karsilasacaksin.
\par 6 RAB'bin Ruhu senin üzerine güçlü bir biçimde inecek. Onlarla birlikte peygamberlikte bulunacak ve baska bir kisilige bürüneceksin.
\par 7 Bu belirtiler gerçeklestiginde, duruma göre gerekeni yap. Çünkü Tanri seninledir.
\par 8 Simdi benden önce Gilgal'a git. Yakmalik sunulari* sunmak ve esenlik kurbanlarini kesmek için ben de yanina gelecegim. Ancak, ben yanina gelip ne yapacagini bildirene dek yedi gün beklemen gerekecek."
\par 9 Saul, Samuel'in yanindan ayrilmak üzere ona sirtini döner dönmez, Tanri ona baska bir kisilik verdi. O gün bütün bu belirtiler gerçeklesti.
\par 10 Giva'ya varinca, Saul'u bir peygamber toplulugu karsiladi. Tanri'nin Ruhu güçlü bir biçimde üzerine indi ve Saul onlarla birlikte peygamberlikte bulunmaya basladi.
\par 11 Onu önceden taniyanlarin hepsi, peygamberlerle birlikte peygamberlikte bulundugunu görünce, birbirlerine, "Ne oldu Kis ogluna? Saul da mi peygamber oldu?" diye sordular.
\par 12 Orada oturanlardan biri, "Ya onlarin babasi kim?" dedi. Iste, "Saul da mi peygamber oldu?" sözü buradan gelir.
\par 13 Saul peygamberlikte bulunduktan sonra tapinma yerine çikti.
\par 14 Amcasi, Saul ile hizmetkârina, "Nerede kaldiniz?" diye sordu. Saul, "Esekleri ariyorduk" diye karsilik verdi, "Onlari bulamayinca, Samuel'e gittik."
\par 15 Amcasi, "Samuel sana neler söyledi, lütfen bana da anlat" dedi.
\par 16 Saul, "Eseklerin bulundugunu bize açikça bildirdi" diye yanitladi. Ama Samuel'in krallikla ilgili sözlerini amcasina açiklamadi.
\par 17 Sonra Samuel, Israil halkini Mispa'da RAB için bir araya getirip söyle dedi: "Israil'in Tanrisi RAB diyor ki, 'Ben Israilliler'i Misir'dan çikardim. Misirlilar'in ve size baski yapan bütün kralliklarin elinden sizi kurtardim.
\par 19 Ama siz bugün bütün zorluk ve sikintilarinizdan sizi kurtaran Tanriniz'a sirt çevirdiniz ve, 'Hayir, bize bir kral ata dediniz. Simdi RAB'bin önünde oymak oymak, boy boy dizilin."
\par 20 Samuel bütün Israil oymaklarini bir bir öne çikardi. Bunlardan Benyamin oymagi kurayla seçildi.
\par 21 Sonra Benyamin oymagini boy boy öne çagirdi. Matri'nin boyu seçildi. En sonunda da Matri boyundan Kis oglu Saul seçildi. Onu aradilarsa da bulamadilar.
\par 22 Yine RAB'be, "O daha buraya gelmedi mi?" diye sordular. RAB de, "O burada, esyalarin arasinda saklaniyor" dedi.
\par 23 Bunun üzerine kosup Saul'u oradan getirdiler. Saul halkin arasina geldi. Boyu hepsinden bir bas uzundu.
\par 24 Samuel halka, "RAB'bin seçtigi adami görüyor musunuz?" dedi, "Bütün halkin arasinda bir benzeri yok." Bunun üzerine halk, "Yasasin kral!" diye bagirdi.
\par 25 Samuel kralligin ilkelerini halka açikladi. Bunlari kitap haline getirip RAB'bin önüne koydu. Sonra herkesi evine gönderdi.
\par 26 Saul da Giva'ya, kendi evine döndü. Tanri'nin isteklendirdigi yigitler ona eslik ettiler.
\par 27 Ama bazi kötü kisiler, "O bizi nasil kurtarabilir?" diyerek Saul'u küçümsediler ve ona armagan vermediler. Saul ise buna aldirmadi.

\chapter{11}

\par 1 Ammon Krali Nahas Yaves-Gilat üzerine yürüyüp kenti kusatti. Bütün Yavesliler, Nahas'a, "Bizimle bir antlasma yap, sana kulluk ederiz" dediler.
\par 2 Ama Ammonlu Nahas, "Ancak bir kosulla sizinle antlasma yaparim" diye karsilik verdi, "Bütün Israil halkini küçük düsürmek için her birinizin sag gözünü oyup çikaracagim."
\par 3 Yaves Kenti'nin ileri gelenleri ona, "Israil'in her bölgesine ulaklar göndermemiz için bize yedi günlük bir süre tani" dediler, "Eger bizi kurtaracak kimse çikmazsa o zaman sana teslim oluruz."
\par 4 Ulaklar Saul'un yasadigi Giva Kenti'ne gelip olanlari halka bildirince, herkes hiçkira hiçkira aglamaya basladi.
\par 5 Tam o sirada Saul, öküzlerinin ardinda, tarladan dönüyordu. "Halka ne oldu? Neden böyle agliyorlar?" diye sordu. Yavesliler'in söylediklerini ona anlattilar.
\par 6 Saul bu sözleri duyunca, Tanri'nin Ruhu güçlü bir biçimde onun üzerine indi. Saul çok öfkelendi.
\par 7 Bir çift öküz alip parçaladi. Ulaklar araciligiyla Israil'in her bölgesine bu parçalari gönderip söyle dedi: "Saul ile Samuel'in ardinca gelmeyen herkesin öküzlerine de ayni sey yapilacaktir." Halk RAB korkusuyla sarsildi ve tek beden halinde yola çikti.
\par 8 Saul onlari Bezek'te topladi. Israil halki üç yüz bin, Yahudalilar ise otuz bin kisiydi.
\par 9 Oraya gelen Yavesli ulaklara söyle dediler: "Yaves-Gilat halkina, 'Yarin ögleye dogru kurtarilacaksiniz deyin." Ulaklar gidip bu haberi iletince Yavesliler sevindi.
\par 10 Ammonlular'a, "Yarin size teslim olacagiz" dediler, "Bize ne dilerseniz yapin."
\par 11 Ertesi gün Saul adamlarini üç bölüge ayirdi. Adamlar sabah nöbetinde Ammonlular'in ordugahina girdi. Kirim günün en sicak zamanina dek sürdü. Sag kalanlar dagildi; iki kisi bile bir arada kalmadi.
\par 12 Bundan sonra halk Samuel'e, "'Saul mu bize krallik yapacak? diyenler kimdi? Getirin onlari, öldürelim" dedi.
\par 13 Ama Saul, "Bugün hiç kimse öldürülmeyecek" diye yanitladi, "Çünkü RAB bugün Israil halkina kurtulus verdi."
\par 14 Samuel halka, "Haydi, Gilgal'a gidip orada kralligi yeniden onaylayalim" dedi.
\par 15 Böylece bütün halk Gilgal'a gidip RAB'bin önünde Saul'un kral oldugunu onayladi. Orada, RAB'bin önünde esenlik kurbanlari kestiler; Saul da bütün Israilliler de büyük bir sevinç yasadilar.

\chapter{12}

\par 1 Bundan sonra Samuel Israil halkina söyle dedi: "Bana söylediginiz her seye kulak verdim: Size bir kral atadim.
\par 2 Simdi size önderlik yapan bir kraliniz var. Bense yaslandim, saçim agardi. Ogullarim da sizlerle birlikte. Gençligimden bu güne dek size önderlik yaptim.
\par 3 Iste karsinizda duruyorum. Hanginizin öküzünü aldim? Kimin esegine el koydum? Kimi dolandirdim? Kime baski yaptim? Göz yummak için kimden rüsvet aldim? RAB'bin ve O'nun meshettiginin* önünde bana karsi taniklik edin de size karsiligini vereyim."
\par 4 Halk, "Bizi dolandirmadin" diye karsilik verdi, "Bize baski da yapmadin. Kimsenin elinden hiçbir sey almadin."
\par 5 Samuel, "Bana karsi bir sey bulamadiginiza bugün hem RAB, hem de O'nun meshettigi kral taniktir" dedi. "Evet, taniktir" dediler.
\par 6 Samuel konusmasini söyle sürdürdü: "Musa ile Harun'u görevlendiren, atalarinizi Misir'dan çikaran RAB'dir.
\par 7 Simdi burada durun, RAB'bin önünde, O'nun sizi ve atalarinizi tekrar tekrar nasil kurtardigina dair kanitlar göstereyim size.
\par 8 "Yakup Misir'a gittikten sonra, atalariniz RAB'be yakardi. O da atalarinizi Misir'dan çikarip burada yerlesmelerini saglayan Musa ile Harun'u gönderdi.
\par 9 Ama atalariniz Tanrilari RAB'bi unuttular. Bu yüzden RAB onlari Hasor ordusunun komutani Sisera'nin, Filistliler'in ve Moav Krali'nin eline teslim etti. Bunlar atalariniza karsi savastilar.
\par 10 Atalariniz RAB'be, 'Günah isledik; RAB'bi birakip Baal'in* ve Astoret'in* putlarina kulluk ettik. Ama simdi bizi düsmanlarimizin elinden kurtar, sana kulluk edecegiz diye seslendiler.
\par 11 RAB de Yerubbaal'i, Bedan'i, Yiftah'i ve ben Samuel'i gönderdi. Güvenlik içinde yasamaniz için sizi saran düsmanlarinizin elinden kurtardi.
\par 12 "Ama siz Ammon Krali Nahas'in üzerinize yürüdügünü görünce, Tanriniz RAB kraliniz oldugu halde bana, 'Hayir, bize bir kral önderlik yapacak dediniz.
\par 13 Iste seçtiginiz, dilediginiz kral! Evet, RAB size bir kral verdi.
\par 14 Eger RAB'den korkar, O'na kulluk ederseniz, O'nun sözünü dinleyip buyruklarina karsi gelmezseniz, hem siz hem de önderiniz olacak kral Tanriniz RAB'bin ardinca giderseniz, ne âlâ!
\par 15 Ama RAB'bin sözünü dinlemez, buyruklarina karsi gelirseniz, RAB kralinizi cezalandirdigi gibi sizi de cezalandiracaktir.
\par 16 "Simdi oldugunuz yerde durun ve RAB'bin gözlerinizin önünde yapacagi su olaganüstü olayi görün.
\par 17 Bugün bugday biçme zamani degil mi? Gögü gürletsin, yagmur yagdirsin diye RAB'be yalvaracagim. Böylece bir kral istemekle yaptiginiz kötülügün RAB'bin gözünde ne denli büyük oldugunu iyice anlayacaksiniz."
\par 18 Samuel RAB'be yalvardi ve RAB o gün gögü gürletti, yagmur yagdirdi. Halk RAB'den de Samuel'den de çok korktu.
\par 19 Bunun üzerine Samuel'e, "Yok olmayalim diye, biz kullarin için Tanrin RAB'be yakar" dediler, "Çünkü bütün günahlarimiza kendimize bir kral istemek kötülügünü de ekledik."
\par 20 Samuel halka, "Korkmayin" dedi, "Siz bu büyük kötülügü yaptiniz, ama yine de RAB'bin ardinca gitmekten vazgeçmeyin; tersine, bütün yüreginizle RAB'be kulluk edin.
\par 21 Kimseyi kurtaramayan yararsiz putlarin ardinca gitmeyin; çünkü onlar degersizdir.
\par 22 RAB görkemli adinin hatirina halkini birakmayacak. Çünkü sizi kendi halki kilmaktan hosnut kaldi.
\par 23 Bana gelince, sizin için RAB'be yalvarmaktan vazgeçip O'na karsi günah islemek benden uzak olsun! Ancak size iyi ve dogru yolu ögretecegim.
\par 24 Yalniz RAB'den korkun, O'na baglilikla ve bütün yüreginizle kulluk edin. O'nun sizler için ne görkemli isler yaptigini bir düsünün!
\par 25 Ama kötülük yapmayi sürdürürseniz, hem siz yok olacaksiniz, hem de kraliniz."

\chapter{13}

\par 1 Saul Israil'de iki yil krallik yaptiktan sonra
\par 2 halktan üç bin kisi seçti. Bunlardan iki binini Mikmas ve Beytel'in daglik bölgesinde yanina aldi. Binini de Benyamin oymagina ait Giva Kenti'nde Yonatan'in yanina birakti. Halktan geri kalanlari evlerine gönderdi.
\par 3 Yonatan Giva'daki Filist birligini yendi. Filistliler bunu duydular. Saul, bütün ülkede boru çaldirarak, "Ibraniler bu haberi duysun" dedi.
\par 4 Böylece Israilliler'in hepsi Saul'un Filist birligini yendigini ve Filistliler'in Israilliler'den igrendigini duydu. Bunun üzerine halk Gilgal'da Saul'un çevresinde toplandi.
\par 5 Filistliler Israilliler'le savasmak üzere toplandilar. Otuz bin savas arabasi, alti bin atli asker ve kiyilardaki kum kadar kalabalik bir orduya sahiptiler. Gidip Beytaven'in dogusundaki Mikmas'ta ordugah kurdular.
\par 6 Durumlarinin tehlikeli oldugunu ve askerlerinin sikistirildigini gören Israilliler, magaralarda, çaliliklarda, kayaliklarda, çukurlarda, sarniçlarda gizlendiler.
\par 7 Bazi Ibraniler de Seria Irmagi'ndan Gad ve Gilat bölgesine geçti. Ama Saul daha Gilgal'daydi. Bütün askerler onu titreyerek izliyordu.
\par 8 Saul, Samuel tarafindan belirlenen süreye uyarak, yedi gün bekledi. Ama Samuel Gilgal'a gelmeyince, halk Saul'un yanindan dagilmaya basladi.
\par 9 Saul, "Yakmalik sunulari* ve esenlik sunularini* bana getirin" dedi. Sonra yakmalik sunuyu sundu.
\par 10 Saul yakmalik sununun sunulmasini bitirir bitirmez Samuel geldi. Saul selamlamak için onu karsilamaya çikti.
\par 11 Samuel, "Ne yaptin?" diye sordu. Saul, "Halk yanimdan dagiliyordu" diye karsilik verdi, "Sen de belirlenen gün gelmedin. Üstelik Filistliler Mikmas'ta toplandilar. Bunlari görünce,
\par 12 'Simdi Filistliler Gilgal'da üzerime yürüyecek; oysa ben RAB'bin yardimini dilememistim diye düsündüm. Bu nedenle, yakmalik sunuyu sunma gerekliligini duydum."
\par 13 Samuel, "Akilsizca davrandin" dedi, "Tanrin RAB'bin sana verdigi buyruga uymadin; yoksa, RAB Israil üzerinde senin kralliginin sonsuza dek sürmesini saglayacakti.
\par 14 Ama artik kralligin sürmeyecek. RAB kendi gönlüne uygun birini arayip onu kendi halkina önder olarak atamaya kararli. Çünkü sen RAB'bin buyrugunu tutmadin."
\par 15 Bundan sonra Samuel Gilgal'dan ayrilarak Benyaminogullari'nin Giva Kenti'ne gitti. Saul yaninda kalan halki saydi; yaklasik alti yüz kisiydi.
\par 16 Saul, oglu Yonatan ve yanlarindaki halk Benyaminogullari'nin bölgesindeki Giva'da kaliyorlardi. Filistliler ise Mikmas'ta ordugah kurmuslardi.
\par 17 Akincilar üç koldan Filistliler'in ordugahindan çiktilar. Kollardan biri Sual bölgesindeki Ofra'ya,
\par 18 biri Beythoron'a, öbürü ise çöle, Sevoyim Vadisi'ne bakan sinira dogru ilerledi.
\par 19 Bütün Israil ülkesinde bir tek demirci yoktu. Filistliler, "Ibraniler kiliç, mizrak yapmasin" demislerdi.
\par 20 Bu nedenle bütün Israilliler saban demirlerini, kazma, balta ve oraklarini biletmek için Filistliler'e gitmek zorundaydilar.
\par 21 Saban demiriyle kazmanin bileme fiyati, sekelin üçte ikisi kadardi. Beller, baltalar, üvendireler için istenilen fiyat ise sekelin üçte biriydi.
\par 22 Iste bu yüzden, savas sirasinda Saul ile Yonatan disinda, yanlarindaki hiç kimsenin elinde kiliç, mizrak yoktu.
\par 23 O sirada Filistliler'in bir kolu Mikmas Geçidi'ne çikmisti.

\chapter{14}

\par 1 Bir gün Saul oglu Yonatan, silahini tasiyan genç hizmetkârina, "Gel, karsi taraftaki Filist ordugahina geçelim" dedi. Ama bunu babasina haber vermedi.
\par 2 Saul, Giva Kenti yakinindaki Migron'da bir nar agacinin altinda oturmaktaydi. Yaninda alti yüz kadar asker vardi.
\par 3 Efod* giymis olan Ahiya da aralarindaydi. Ahiya Silo'da RAB'bin kâhini olan Eli oglu Pinehas oglu Ikavot'un erkek kardesi Ahituv'un ogluydu. Halk Yonatan'in gittigini farketmemisti.
\par 4 Yonatan'in Filist ordugahina ulasmak için geçmeyi tasarladigi geçidin her iki yaninda iki sivri kaya vardi; birine Boses, öbürüne Sene denirdi.
\par 5 Kayalardan biri kuzeyde Mikmas'a, öbürü güneyde Giva'ya bakardi.
\par 6 Yonatan silahini tasiyan genç hizmetkârina, "Gel, su sünnetsizlerin* ordugahina gidelim" dedi, "Belki RAB bizim için bir seyler yapar. Çünkü gerek çoklukta, gerekse azlikta RAB'bin zafere ulastirmasina engel yoktur."
\par 7 Silahini tasiyan genç, "Ne düsünüyorsan öyle yap" diye yanitladi, "Haydi yürü! Düsündügün her seyde seninleyim."
\par 8 Yonatan, "Bu adamlara gidelim, bizi görsünler" dedi,
\par 9 "Eger bize, 'Yaniniza gelene dek bekleyin derlerse, oldugumuz yerde kaliriz, gitmeyiz.
\par 10 Ama, 'Yanimiza gelin derlerse, gideriz. Çünkü bu, RAB'bin Filistliler'i elimize teslim ettigine iliskin bir belirti olacak bizim için."
\par 11 Böylece ikisi de Filistliler'in askerlerine göründüler. Filistliler, "Bakin! Ibraniler gizlendikleri çukurlardan çikmaya basliyor!" dediler.
\par 12 Sonra Yonatan'la silahini tasiyan gence, "Buraya, yanimiza gelin, size bir sey söyleyecegiz" diye seslendiler. Bunun üzerine Yonatan silahini tasiyana, "Ardimdan gel" dedi, "RAB onlari Israilliler'in eline teslim etti."
\par 13 Yonatan elleriyle ayaklarini kullanarak yukariya tirmandi; silahini tasiyan genç de onu izledi. Yonatan Filistliler'i yenilgiye ugratti. Silahini tasiyan genç de onu izliyor ve Filistliler'i öldürüyordu.
\par 14 Yonatan'la silahini tasiyan genç bu ilk saldirida iki dönümlük bir alanda yirmi kadar asker öldürdüler.
\par 15 Ordugahta ve kirsal alanda bütün Filist halki arasinda dehset hüküm sürüyordu. Askerlerle akincilar bile titriyordu. Derken yer sarsildi; sanki Tanri'dan gelen bir titremeydi bu.
\par 16 Benyamin topraklarindaki Giva Kenti'nde Saul'un nöbetçileri büyük bir kalabaligin oraya buraya dagildigini gördüler.
\par 17 Bunun üzerine Saul yanindaki adamlara, "Yoklama yapin da aramizdan kimin ayrildigini görün" dedi. Yoklama yapilinca Yonatan'la silahini tasiyan gencin orada olmadigini anladilar.
\par 18 Saul Ahiya'ya, "Tanri'nin Sandigi'ni getir" dedi. O sirada Tanri'nin Sandigi Israil halkindaydi.
\par 19 Saul kâhinle konusurken, Filistliler'in ordugahindaki kargasa da giderek artmaktaydi. Bunun üzerine Saul kâhine, "Elini çek" dedi.
\par 20 Saul'la yanindaki askerlerin tümü toplanip savas alanina gittiler. Orada büyük bir kargasa vardi. Herkes birbirine kiliç çekiyordu.
\par 21 Daha önce Filistliler'in yaninda yer alip onlarin ordugahina katilan Ibraniler bile saf degistirerek Saul'la Yonatan'in yanindaki Israil birliklerine katildilar.
\par 22 Efrayim daglik bölgesinde gizlenen Israilliler de Filistliler'in kaçtigini duyunca onlari savas alaninda kovalamaya basladilar.
\par 23 Böylece RAB Israil'i o gün zafere ulastirdi. Savas Beytaven'in ötesine dek yayildi.
\par 24 O gün Israilliler bitkindi. Çünkü Saul, "Ben düsmanlarimdan öç alincaya kadar, aksama dek kim yemek yerse lanetli olsun!" diye halka ant içirmisti. Bu yüzden de kimse bir sey yememisti.
\par 25 Derken, her yani bal dolu bir ormana vardilar. Askerler ormana girince, toprakta akan ballari gördüler. Ne var ki, içtikleri anttan korktuklari için hiçbiri bala dokunmadi.
\par 27 Yonatan babasinin halka ant içirdigini duymamisti. Elindeki degnegi uzatip ucunu bal gümecine batirdi. Biraz bal tadar tatmaz gözleri parladi.
\par 28 Bunun üzerine oradakilerden biri Yonatan'a, "Baban askerlere, 'Bugün kim yemek yerse lanetli olsun diye ant içirdi" dedi, "Askerlerin bitkin düsmesi de bundan."
\par 29 Yonatan, "Babam halka sikinti verdi" diye yanitladi, "Bakin, bu baldan biraz tadinca gözlerim nasil da parladi!
\par 30 Bugün halk düsmanlarindan yagmaladigi yiyeceklerden özgürce yeseydi, çok daha iyi olurdu! O zaman Filistliler'in yenilgisi de daha agir olmaz miydi?"
\par 31 O gün Israilliler, Filistliler'i Mikmas'tan Ayalon'a kadar yenilgiye ugrattilar. Ama Israil askerleri o kadar bitkindi ki,
\par 32 yagmaladiklari mallara saldirdilar; davarlari, sigirlari, buzagilari yakaladiklari gibi hemen oracikta kesip kanini akitmadan yediler.
\par 33 Durumu Saul'a bildirerek, "Bak, askerlerin kanli eti yemekle RAB'be karsi günah isliyor!" dediler. Bunun üzerine Saul, "Hainlik ettiniz!" dedi, "Hemen büyük bir tas yuvarlayin bana."
\par 34 Sonra ekledi: "Halkin arasina varip herkesin öküzünü, koyununu bana getirmesini söyleyin. Onlari burada kesip yesinler. Eti kaniyla birlikte yiyerek RAB'be karsi günah islemeyin." O gece herkes öküzünü getirip orada kesti.
\par 35 O sirada Saul RAB'be bir sunak yapti. RAB'be yaptigi ilk sunakti bu.
\par 36 Saul adamlarina, "Haydi, bu gece Filistliler'e saldiralim" dedi, "Tan agarincaya dek mallarini yagmalayalim, onlardan bir tekini bile sag birakmayalim." Adamlar, "Sence uygun olan neyse onu yap" diye karsilik verdiler. Ama kâhin, "Burada Tanri'ya danisalim" dedi.
\par 37 Bunun üzerine Saul Tanri'ya, "Filistliler'e saldirmaya gideyim mi? Onlari Israilliler'in eline teslim edecek misin?" diye sordu. Ama Tanri o gün yanit vermedi.
\par 38 Bunun için Saul, "Ey halkin önderleri! Buraya yaklasin da bugün islenen bu günahin nasil islendigini ortaya çikaralim" dedi,
\par 39 "Israil'i kurtaran yasayan RAB'bin adiyla derim ki, bu günaha yol açan oglum Yonatan bile olsa kesinlikle öldürülecektir." Ama kimse bir sey söylemedi.
\par 40 Bunun üzerine Saul halka, "Siz bir yanda durun, oglum Yonatan'la ben öbür yanda duracagiz" dedi. Halk, "Sence uygun olan neyse onu yap" diye karsilik verdi.
\par 41 Saul Israil'in Tanrisi RAB'be, "Bana dogru yaniti ver" dedi. Kura Yonatan'la Saul'a düstü, halk aklandi.
\par 42 Saul bu kez, "Benimle oglum Yonatan arasinda kura çekin" dedi. Kura Yonatan'a düstü.
\par 43 Bunun üzerine Saul Yonatan'a, "Söyle bana, ne yaptin?" diye sordu.
\par 44 Saul, "Yonatan, eger seni öldürtmezsem, Tanri bana aynisini, hatta daha kötüsünü yapsin!" dedi.
\par 45 Ama halk Saul'a, "Israil'i bu büyük zafere ulastiran Yonatan'i mi öldürteceksin?" dedi, "Asla! Yasayan RAB'bin adiyla deriz ki, saçinin bir teline bile zarar gelmeyecektir. Çünkü bugün o ne yaptiysa Tanri'nin yardimiyla yapmistir." Böylece halk Yonatan'i öldürülmekten kurtardi.
\par 46 Bundan sonra Saul Filistliler'i kovalamaktan vazgeçti. Filistliler de yerlerine döndüler.
\par 47 Saul Israil'e kral atandiktan sonra, her yandaki düsmanlarina -Moav, Ammon, Edom halklari, Sova krallari ve Filistliler'e- karsi savasti. Gittigi her yerde zafer kazandi.
\par 48 Yigitçe savasarak Amalekliler'i yenilgiye ugratti, Israilliler'i düsmanin yagmasindan kurtardi.
\par 49 Saul'un ogullari Yonatan, Yisvi ve Malkisua idi. Iki kizindan büyügünün adi Merav, küçügünün adi Mikal'di.
\par 50 Karisi, Ahimaas'in kizi Ahinoam'di. Ordusunun baskomutani amcasi Ner oglu Avner'di.
\par 51 Saul'un babasi Kis'le Avner'in babasi Ner, Aviel'in ogullariydi.
\par 52 Saul yasami boyunca Filistliler'le kiyasiya savasti. Nerede yigit, güçlü birini görse kendi ordusuna katti.

\chapter{15}

\par 1 Samuel Saul'a söyle dedi: "RAB seni kendi halki Israil'in Krali olarak meshetmek* için beni gönderdi. Simdi RAB'bin sözlerine kulak ver.
\par 2 Her Seye Egemen RAB diyor ki, 'Israilliler'e yaptiklari kötülükten ötürü Amalekliler'i cezalandiracagim. Çünkü Misir'dan çikan Israilliler'e karsi koydular.
\par 3 Simdi git, Amalekliler'e saldir. Onlara ait her seyi tümüyle yok et, hiçbir seyi esirgeme. Kadin erkek, çoluk çocuk, öküz, koyun, deve, esek hepsini öldür."
\par 4 Bunun üzerine Saul askerlerini toplayip Telaim Kenti'nde saydi. Iki yüz bin yaya askerin yanisira Yahudalilar'dan da on bin kisi vardi.
\par 5 Saul Amalek Kenti'ne varip vadide pusu kurdu.
\par 6 Sonra Kenliler'e su uyariyi gönderdi: "Haydi gidin, Amalekliler'i birakin; öyle ki, sizi de onlarla birlikte yok etmeyeyim. Çünkü siz Misir'dan çikan Israil halkina iyilik ettiniz." Bunun üzerine Kenliler Amalekliler'den ayrildilar.
\par 7 Saul Havila'dan Misir'in dogusundaki Sur'a dek Amalekliler'i yenilgiye ugratti.
\par 8 Amalek Krali Agak'i sag olarak yakaladi. Halkinin tümünü de kiliçtan geçirdi.
\par 9 Ne var ki, Saul ile adamlari Agak'i ve en iyi koyunlari, sigirlari, besili danalari, kuzulari -iyi olan ne varsa hepsini- esirgediler. Bunlari tümüyle yok etmek istemediler. Ancak degersiz ve zayif ne varsa hepsini yok ettiler.
\par 10 RAB Samuel'e söyle seslendi:
\par 11 "Saul'u kral yaptigima pismanim. Beni izlemekten vazgeçti. Buyruklarimi yerine getirmedi." Samuel öfkelendi ve bütün geceyi RAB'be yakarmakla geçirdi.
\par 12 Ertesi sabah Samuel Saul'la görüsmek için erkenden kalkti. Saul'un Karmel Kenti'ne gittigini, orada kendisine bir anit diktikten sonra asagi inip Gilgal'a döndügünü ögrendi.
\par 13 Saul kendisine gelen Samuel'e, "RAB seni kutsasin! Ben RAB'bin buyrugunu yerine getirdim" dedi.
\par 14 Samuel, "Öyleyse nedir kulagima gelen bu koyun melemesi? Nedir bu duydugum sigir bögürmesi?" diye sordu.
\par 15 Saul söyle yanitladi: "Halk bunlari Amalekliler'den getirdi. Tanrin RAB'be kurban sunmak üzere davarlarin, sigirlarin en iyilerini esirgediler. Ama geri kalanlari tümüyle yok ettik."
\par 16 Samuel, "Dur da bu gece RAB'bin bana neler söyledigini sana bildireyim" dedi. Saul, "Söyle" diye karsilik verdi.
\par 17 Samuel konusmasini söyle sürdürdü: "Kendini önemsiz saydigin halde, sen Israil oymaklarinin önderi olmadin mi? RAB seni Israil'e kral meshetti.
\par 18 RAB seni bir göreve gönderip, 'Git, o günahli Amalekliler'i tümüyle yok et; hepsini ortadan kaldirincaya dek onlarla savas dedi.
\par 19 Öyleyse neden RAB'bin sözüne kulak asmadin? Neden yagmalanan mallara saldirarak RAB'bin gözünde kötü olani yaptin?"
\par 20 Saul, "Ama ben RAB'bin sözüne kulak verdim!" diye yanitladi, "RAB'bin beni gönderdigi yere gittim. Amalekliler'i tümüyle yok ettim, Amalek Krali Agak'i da buraya getirdim.
\par 21 Ne var ki askerler, Gilgal'da Tanrin RAB'be kurban sunmak üzere yagmalanmis bazi mallari, yok edilmeye adanmis en iyi davarlarla sigirlari aldilar."
\par 22 Samuel söyle karsilik verdi: "RAB kendi sözünün dinlenmesinden hoslandigi kadar Yakmalik sunulardan*, kurbanlardan hoslanir mi? Iste söz dinlemek kurbandan, Sözü önemsemek de koçlarin yaglarindan daha iyidir.
\par 23 Çünkü baskaldirma, falcilik kadar günahtir Ve dikbaslilik, putperestlik kadar kötüdür. Sen RAB'bin buyrugunu reddettigin için, RAB de senin kral olmani reddetti."
\par 24 Bunun üzerine Saul, "Günah isledim! Evet, RAB'bin buyrugunu da, senin sözlerini de çignedim" dedi, "Halktan korktugum için onlarin sözünü dinledim.
\par 25 Ama simdi yalvaririm, günahimi bagisla ve benimle birlikte dön ki, RAB'be tapinayim."
\par 26 Samuel, "Seninle dönmem" dedi, "Çünkü sen RAB'bin buyrugunu reddettin, RAB de Israil Krali olmani reddetti!"
\par 27 Samuel dönüp gitmeye davraninca, Saul onun cüppesinin etegini tuttu. Cüppe yirtildi.
\par 28 Samuel, "Bugün RAB Israil Kralligi'ni elinden aldi ve senden daha iyi birine verdi" dedi,
\par 29 "Israil'in yüce Tanrisi yalan söylemez, düsüncesini de degistirmez. Çünkü O insan degil ki, düsüncesini degistirsin."
\par 30 Saul, "Günah isledim!" dedi, "Ama ne olur halkimin ileri gelenleri ve Israilliler karsisinda beni onurlandir. Tanrin RAB'be tapinmam için benimle dön."
\par 31 Böylece Samuel Saul'la birlikte geri döndü ve Saul RAB'be tapindi.
\par 32 Samuel, "Amalek Krali Agak'i bana getirin" diye buyurdu. Agak güvenle geldi. Çünkü, "Ölüm tehlikesi kesinlikle geçti" diye düsünüyordu.
\par 33 Ama Samuel, "Kilicin kadinlari nasil çocuksuz biraktiysa Senin annen de kadinlar arasinda Çocuksuz birakilacak" diyerek Agak'i Gilgal'da RAB'bin önünde kiliçla parçaladi.
\par 34 Samuel Rama'ya, Saul da Giva'daki evine gitti.
\par 35 Samuel ölümüne dek Saul'u bir daha görmediyse de, onun için üzüldü. RAB de Saul'u Israil Krali yaptigina pismandi.

\chapter{16}

\par 1 RAB Samuel'e, "Ben Saul'un Israil Krali olmasini reddettim diye sen daha ne zamana dek onun için üzüleceksin?" dedi, "Yag boynuzunu yagla doldurup yola çik. Seni Beytlehemli Isay'in evine gönderiyorum. Çünkü onun ogullarindan birini kral seçtim."
\par 2 Samuel, "Nasil gidebilirim? Saul bunu duyarsa beni öldürür!" dedi. RAB söyle yanitladi: "Yanina bir düve al ve, 'RAB'be kurban sunmak için geldim de.
\par 3 Isay'i kurban törenine çagir. O zaman ne yapman gerektigini ben sana bildirecegim. Sana belirtecegim kisiyi benim adima kral olarak meshedeceksin*."
\par 4 Samuel RAB'bin sözüne uyarak Beytlehem Kenti'ne gitti. Kentin ileri gelenleri onu titreyerek karsiladilar ve, "Baris için mi geldin?" diye sordular.
\par 5 Samuel, "Evet, baris için" diye yanitladi, "RAB'be kurban sunmaya geldim. Kendinizi kutsayip benimle birlikte kurban törenine gelin." Sonra Isay ile ogullarini kutsayip kurban törenine çagirdi.
\par 6 Isay ile ogullari gelince Samuel Eliav'i gördü ve, "Gerçekten RAB'bin önünde duran bu adam O'nun meshettigi kisidir" diye düsündü.
\par 7 Ama RAB Samuel'e, "Onun yakisikli ve uzun boylu olduguna bakma" dedi, "Ben onu reddettim. Çünkü RAB insanin gördügü gibi görmez; insan dis görünüse, RAB ise yürege bakar."
\par 8 Isay, oglu Avinadav'i çagirip Samuel'in önünden geçirdi. Ama Samuel, "RAB bunu da seçmedi" dedi.
\par 9 Bunun üzerine Isay Samma'yi da geçirdi. Samuel yine, "RAB bunu da seçmedi" dedi.
\par 10 Böylece Isay yedi oglunu da Samuel'in önünden geçirdi. Ama Samuel, "RAB bunlardan hiçbirini seçmedi" dedi.
\par 11 Sonra Isay'a, "Ogullarinin hepsi bunlar mi?" diye sordu. Isay, "Bir de en küçügü var" dedi, "Sürüyü güdüyor." Samuel, "Birini gönder de onu getirsin" dedi, "O buraya gelmeden yemege oturmayacagiz."
\par 12 Isay birini gönderip oglunu getirtti. Çocuk kizil saçli, yakisikli, gözleri piril piril bir delikanliydi. RAB Samuel'e, "Kalk, onu meshet. Seçtigim kisi odur" dedi.
\par 13 Samuel yag boynuzunu alip kardeslerinin önünde çocugu meshetti. O günden baslayarak RAB'bin Ruhu Davut'un üzerine güçlü bir biçimde indi. Bundan sonra Samuel kalkip Rama'ya döndü.
\par 14 Bu siralarda RAB'bin Ruhu Saul'dan ayrilmisti. RAB'bin gönderdigi kötü bir ruh ona sikinti çektiriyordu.
\par 15 Hizmetkârlari Saul'a, "Bak, Tanri'nin gönderdigi kötü bir ruh sana sikinti çektiriyor" dediler,
\par 16 "Efendimiz, biz hizmetkârlarina buyruk ver, iyi lir çalan birini bulalim. Öyle ki, Tanri'nin gönderdigi kötü ruh üzerine gelince, o lir çalar, sen de rahatlarsin."
\par 17 Saul hizmetkârlarina, "Iyi lir çalan birini bulup bana getirin" diye buyurdu.
\par 18 Hizmetkârlardan biri, "Beytlehemli Isay'in ogullarindan birini gördüm" dedi, "Iyi lir çalar. Üstelik yürekli, güçlü bir savasçidir; akillica konusur, yakisiklidir. RAB de onunladir."
\par 19 Bunun üzerine Saul Isay'a ulaklar göndererek, "Sürüyü güden oglun Davut'u bana gönder" dedi.
\par 20 Isay ekmek yüklü bir esek, bir tulum sarap, bir de oglak alip oglu Davut'la birlikte Saul'a gönderdi.
\par 21 Davut Saul'un yanina varip onun hizmetine girdi. Saul Davut'u çok sevdi ve ona silahlarini tasima görevini verdi.
\par 22 Saul Isay'a su haberi gönderdi: "Izin ver de Davut hizmetimde kalsin; ondan hosnudum."
\par 23 O günden sonra, Tanri'nin gönderdigi kötü ruh ne zaman Saul'un üzerine gelse, Davut liri alip çalar, Saul rahatlayip kendine gelirdi. Kötü ruh da ondan uzaklasirdi.

\chapter{17}

\par 1 Savasmak üzere ordularini bir araya getiren Filistliler, Yahuda'nin Soko Kenti'nde toplandilar. Soko ile Azeka Kenti arasindaki Efes-Dammim'de ordugah kurdular.
\par 2 Saul ile Israilliler de toplandilar. Ela Vadisi'nde ordugah kurup Filistliler'e karsi savas düzeni aldilar.
\par 3 Filistliler tepenin bir yaninda, Israilliler de karsi tepede yerlerini aldi. Aralarinda vadi vardi.
\par 4 Filist ordugahindan Gatli Golyat adinda usta bir dövüsçü ortaya çikti. Boyu alti arsin bir karisti.
\par 5 Basina tunç* migfer takmis, pullu bir zirh kusanmisti. Tunç zirhin agirligi bes bin sekeldi.
\par 6 Baldirlari zirhlarla korunmustu. Omuzlari arasinda tunç bir pala asiliydi.
\par 7 Mizraginin sapi dokumaci tezgahinin sirigi gibiydi. Mizragin demir basinin agirligi alti yüz sekeldi. Golyat'in önüsira kalkanini tasiyan bir adam yürüyordu.
\par 8 Golyat durup Israil ordusuna, "Neden savas düzeni aldiniz?" diye haykirdi, "Ben Filistli'yim, sizse Saul'un kölelerisiniz. Aranizdan karsima çikacak birini seçin.
\par 9 Dövüste beni yenip öldürebilirse, biz sizin köleniz oluruz. Ama ben üstün gelip onu yok edebilirsem, siz bizim kölemiz olur, bize kulluk edersiniz."
\par 10 Filistli Golyat konusmasini söyle sürdürdü: "Bugün Israil ordusuna meydan okuyorum! Benimle dövüsecek birini çikarin karsima!"
\par 11 Filistli'nin bu sözlerini duyunca, Saul da Israilliler de çok korkup dehset içinde kaldilar.
\par 12 Davut Yahuda'nin Beytlehem Kenti'nden Efratli Isay adinda bir adamin ogluydu. Isay'in sekiz oglu vardi. Saul'un kralligi döneminde Isay'in yasi oldukça ilerlemisti.
\par 13 Isay'in üç büyük oglu Saul'la birlikte savasa katilmisti. Savasa giden en büyük oglunun adi Eliav, ikincisinin adi Avinadav, üçüncüsünün adiysa Samma'ydi.
\par 14 Davut en küçükleriydi. Üç büyük ogul Saul'un yanindaydi.
\par 15 Davut ise babasinin sürüsüne bakmak için Saul'un yanindan ayrilip Beytlehem'e gider gelirdi.
\par 16 Filistli Golyat kirk gün boyunca sabah aksam ortaya çikip meydan okudu.
\par 17 Bir gün Isay, oglu Davut'a söyle dedi: "Kardeslerin için su kavrulmus bir efa bugdayla on somun ekmegi al, çabucak ordugaha, kardeslerinin yanina git.
\par 18 Su on parça peyniri de birlik komutanina götür. Kardeslerinin ne durumda oldugunu ögren ve iyi olduklarina iliskin bir belirti getir.
\par 19 Kardeslerin Saul ve öbür Israilliler'le birlikte Ela Vadisi'nde Filistliler'e karsi savasiyorlar."
\par 20 Ertesi sabah Davut erkenden kalkti. Sürüyü bir çobana birakti. Isay'in buyurdugu gibi erzagi alip yola koyuldu. Ordugaha vardigi sirada askerler savas naralari atarak savas düzenine giriyorlardi.
\par 21 Israilliler'le Filistliler karsi karsiya savas düzeni almislardi.
\par 22 Davut getirdiklerini levazim görevlisine birakip cepheye kostu; kardeslerinin yanina varip onlari selamladi.
\par 23 Davut onlarla konusurken, Gatli Filistli, Golyat adindaki dövüsçü Filist cephesinden ileri çikarak daha önce yaptigi gibi meydan okudu. Davut bunu duydu.
\par 24 Israilliler Golyat'i görünce büyük korkuyla önünden kaçistilar.
\par 25 Birbirlerine, "Israil'e meydan okumak için ortaya çikan su adami görüyorsunuz ya!" diyorlardi, "Kral onu öldürene büyük bir armaganin yanisira kizini da verecek. Babasinin ailesini de Israil'e vergi ödemekten muaf tutacak."
\par 26 Davut yanindakilere, "Bu Filistli'yi öldürüp Israil'den bu utanci kaldiracak kisiye ne verilecek?" diye sordu, "Bu sünnetsiz* Filistli kim oluyor da yasayan Tanri'nin ordusuna meydan okuyor?"
\par 27 Adamlar daha önce verilmis olan söze göre Golyat'i öldürecek kisiye neler verilecegini anlattilar.
\par 28 Agabeyi Eliav Davut'un adamlarla konustugunu duyunca öfkelendi. "Ne isin var burada?" dedi, "Çöldeki üç bes koyunu kime biraktin? Ne kadar kendini begenmis ve ne kadar kötü yürekli oldugunu biliyorum. Sadece savasi görmeye geldin."
\par 29 Davut, "Ne yaptim ki?" dedi, "Bir soru sordum, o kadar."
\par 30 Sonra baska birine dönüp ayni soruyu sordu. Adamlar öncekine benzer bir yanit verdiler.
\par 31 Davut'un söylediklerini duyanlar Saul'a ilettiler. Saul onu çagirtti.
\par 32 Davut Saul'a, "Bu Filistli yüzünden kimse yilmasin! Ben kulun gidip onunla dövüsecegim!" dedi.
\par 33 Saul, "Sen bu Filistli'yle dövüsemezsin" dedi, "Çünkü daha gençsin, o ise gençliginden beri savasçidir."
\par 34 Ama Davut, "Kulun babasinin sürüsünü güder" diye karsilik verdi, "Bir aslan ya da ayi gelip sürüden bir kuzu kaçirinca,
\par 35 pesinden gidip ona saldirir, kuzuyu agzindan kurtaririm. Eger aslan ya da ayi üzerime gelirse, bogazindan tuttugum gibi vurur öldürürüm.
\par 36 Kulun, aslan da ayi da öldürmüstür. Bu sünnetsiz Filistli de onlar gibi olacak. Çünkü yasayan Tanri'nin ordusuna meydan okudu.
\par 37 Beni aslanin, ayinin pençesinden kurtaran RAB, bu Filistli'nin elinden de kurtaracaktir." Saul, "Öyleyse git, RAB seninle birlikte olsun" dedi.
\par 38 Sonra kendi giysilerini Davut'a verdi; basina tunç migfer takti, ona bir zirh giydirdi.
\par 39 Davut giysilerinin üzerine kilicini kusanip yürümeye çalisti. Çünkü bu giysilere alisik degildi. Saul'a, "Bunlarla yürüyemiyorum" dedi, "Çünkü alisik degilim." Sonra giysileri üzerinden çikardi.
\par 40 Degnegini alip dereden bes çakil tasi seçti. Bunlari çoban dagarciginin cebine koyduktan sonra sapanini alip Filistli Golyat'a dogru ilerledi.
\par 41 Filistli de, önünde kalkan tasiyicisi, Davut'a dogru ilerliyordu.
\par 42 Davut'u tepeden tirnaga süzdü. Kizil saçli, yakisikli bir genç oldugu için onu küçümsedi.
\par 43 "Ben köpek miyim ki, üzerime degnekle geliyorsun?" diyerek kendi ilahlarinin adiyla Davut'u lanetledi.
\par 44 "Bana gelsene! Bedenini gökteki kuslara ve kirdaki hayvanlara yem edecegim!" dedi.
\par 45 Davut, "Sen kiliçla, mizrakla, palayla üzerime geliyorsun" diye karsilik verdi, "Bense meydan okudugun Israil ordusunun Tanrisi, Her Seye Egemen RAB'bin adiyla senin üzerine geliyorum.
\par 46 Bugün RAB seni elime teslim edecek. Seni vurup basini gövdenden ayiracagim. Bugün Filistli askerlerin leslerini gökteki kuslarla yerdeki hayvanlara yem edecegim. Böylece bütün dünya Israil'de Tanri'nin var oldugunu anlayacak.
\par 47 Bütün bu topluluk RAB'bin kiliçla, mizrakla kurtarmadigini anlayacak. Çünkü savas zaten RAB'bindir! O sizi elimize teslim edecek."
\par 48 Golyat saldirmak amaciyla Davut'a dogru ilerledi. Davut da onunla dövüsmek üzere hemen Filist cephesine dogru kostu.
\par 49 Elini dagarcigina sokup bir tas çikardi, sapanla firlatti. Tas Filistli'nin alnina çarpip saplandi. Filistli yüzükoyun yere düstü.
\par 50 Böylece Davut Filistli Golyat'i sapan ve tasla yendi. Elinde kiliç olmaksizin onu yere serdi.
\par 51 Sonra kosup üzerine çikti. Golyat'in kilicini tutup kinindan çektigi gibi onu öldürdü ve basini kesti. Kahraman Golyat'in öldügünü gören Filistliler kaçtilar.
\par 52 Israilliler'le Yahudalilar kalkip Gat'in girisine ve Ekron kapilarina kadar nara atarak onlari kovaladilar. Filistliler'in ölüleri Gat'a, Ekron'a kadar Saarayim yolunda yerlere serildi.
\par 53 Filistliler'i kovaladiktan sonra geri dönen Israilliler Filist ordugahini yagmaladilar.
\par 54 Davut Filistli Golyat'in basini alip Yerusalim'e götürdü, silahlarini da kendi çadirina koydu.
\par 55 Saul, Davut'un Golyat'la dövüsmeye çiktigini görünce, ordu komutani Avner'e, "Ey Avner, kimin oglu bu genç?" diye sormustu. Avner de, "Yasamin hakki için, ey kral, bilmiyorum" diye yanitlamisti.
\par 56 Kral Saul, "Bu gencin kimin oglu oldugunu ögren" diye buyurmustu.
\par 57 Davut Golyat'i öldürüp ordugaha döner dönmez, Avner onu alip Saul'a götürdü. Golyat'in kesik basi Davut'un elindeydi.
\par 58 Saul, "Kimin oglusun, delikanli?" diye sordu. Davut, "Kulun Beytlehemli Isay'in ogluyum" diye karsilik verdi.

\chapter{18}

\par 1 Saul'la Davut'un konusmasi sona erdiginde, Saul oglu Yonatan'in yüregi Davut'a baglandi. Yonatan onu cani gibi sevdi.
\par 2 O günden sonra Saul Davut'u yaninda tuttu ve babasinin evine dönmesine izin vermedi.
\par 3 Yonatan, Davut'a besledigi derin sevgiden ötürü, onunla bir dostluk antlasmasi yapti.
\par 4 Üzerinden kaftanini çikarip zirhi, kilici, yayi ve kusagiyla birlikte Davut'a verdi.
\par 5 Davut Saul'un kendisini gönderdigi her yere gitti ve basarili oldu. Bu yüzden Saul ona ordusunda üstün bir rütbe verdi. Bu olay bütün halki, Saul'un görevlilerini bile hosnut etti.
\par 6 Davut'un Filistli Golyat'i öldürmesinden sonra, askerler geri dönerken, Israil'in bütün kentlerinden gelen kadinlar, tef ve çesitli çalgilar çalarak, sevinçli ezgiler söyleyip oynayarak Kral Saul'u karsilamaya çiktilar.
\par 7 Bir yandan oynuyor, bir yandan da su ezgiyi söylüyorlardi: "Saul binlercesini öldürdü, Davut'sa on binlercesini."
\par 8 Bu sözlere gücenen Saul çok öfkelendi. "Davut'a on binlercesini, banaysa ancak binlercesini verdiler. Artik kral olmaktan baska onun ne eksigi kaldi ki?" diye düsündü.
\par 9 Böylece o günden sonra Saul Davut'u kiskanmaya basladi.
\par 10 Ertesi gün Tanri'nin gönderdigi kötü bir ruh Saul'un üzerine güçlü bir biçimde indi. Saul evinde sayiklamaya basladi. Davut her zamanki gibi yine lir çaliyordu. Saul'un elinde bir mizrak vardi.
\par 11 "Davut'u vurup duvara çakacagim" diye düsünerek mizragi ona firlatti. Ama Davut iki kez ondan kurtuldu.
\par 12 Saul Davut'tan korkuyordu. Çünkü RAB Davut'laydi, oysa kendisinden ayrilmisti.
\par 13 Bu yüzden Saul Davut'u yanindan uzaklastirdi. Onu bin kisilik birlige komutan atadi. Davut askerlere öncülük yapiyordu.
\par 14 RAB onunla birlikte oldugundan, yaptigi her iste basariliydi.
\par 15 Davut'un büyük basarisini gördükçe Saul'un korkusu daha da artiyordu.
\par 16 Ne var ki, bütün Israil ve Yahuda halki Davut'u seviyordu; çünkü Davut onlara öncülük ediyordu.
\par 17 Saul Davut'a, "Iste büyük kizim Merav" dedi, "Onu sana es olarak verecegim. Yalniz hatirim için yigitçe davran ve RAB'bin savaslarini sürdür." Çünkü, "Davut'un ölümü benim elimden degil, Filistliler'in elinden olsun" diye düsünüyordu.
\par 18 Davut, "Ben kim oluyorum, Israil'de ailem ve babamin oymagi ne ki, krala damat olayim?" diye karsilik verdi.
\par 19 Ne var ki, Saul'un kizi Merav'in Davut'a verilecegi zaman geldiginde, kiz Davut yerine Meholali Adriel'e es olarak verildi.
\par 20 Bu arada Saul'un öbür kizi Mikal Davut'a gönül vermisti. Bunu duyan Saul sevindi.
\par 21 "Davut'a Mikal'i veririm" diye düsündü, "Öyle ki, Mikal Davut'u tuzaga düsürür; Filistliler de onu öldürür." Davut'a, "Bugün damadim olmak için yine firsatin var" dedi.
\par 22 Sonra görevlilerine, Davut'a gizlice sunlari söylemelerini buyurdu: "Bak, kral senden hosnut, bütün görevlileri de seni seviyor. Kralin damadi olmanin zamani geldi."
\par 23 Saul'un görevlileri bu sözleri Davut'a ilettiler. Davut, "Yoksul ve önemsiz biriyken kralin damadi olmak sizce küçük bir sey mi?" diye karsilik verdi.
\par 24 Görevliler Davut'un dediklerini Saul'a bildirdiler.
\par 25 Saul söyle buyurdu: "Davut'a deyin ki, 'Kral düsmanlarindan öç almak için baslik parasi olarak yüz Filistli'nin sünnet derisinden baska bir sey istemiyor." Davut'un Filistliler'in eline düsüp ölecegini tasarliyordu.
\par 26 Görevliler Saul'un söylediklerini Davut'a ilettiler. Davut, kralin damadi olacagina sevindi. Taninan süre dolmadan
\par 27 Davut'la adamlari gidip iki yüz Filistli öldürdüler. Kralin damadi olabilmek için Davut, öldürülen Filistliler'in sünnet derilerini tam tamina getirip krala sundu. Saul da buna karsilik kizi Mikal'i es olarak ona verdi.
\par 28 Saul, RAB'bin Davut'la birlikte oldugunu ve kizi Mikal'in onu sevdigini apaçik gördü.
\par 29 Bu yüzden Davut'tan daha çok korktu ve yasami boyunca ona düsmanlik besledi.
\par 30 Filistli komutanlar saldirdikça Davut Saul'un öbür komutanlarindan daha basarili oluyordu. Bu yüzden büyük bir üne kavustu.

\chapter{19}

\par 1 Saul, oglu Yonatan'a ve bütün görevlilerine Davut'u öldürmeleri için buyruk verdi. Ama Davut'u çok seven Yonatan ona, "Babam Saul seni öldürmek için firsat kolluyor" diye haber verdi, "Lütfen yarin sabah dikkatli ol; gizlenebilecegin bir yere gidip saklan.
\par 3 Ben de saklandigin tarlaya gidip babamin yaninda duracagim ve onunla senin hakkinda konusacagim. Bir sey ögrenirsem, sana bildiririm."
\par 4 Yonatan babasi Saul'a Davut'u överek sunlari söyledi: "Kral kulu Davut'a haksizlik etmesin. Çünkü o sana hiç haksizlik etmedi ve yaptigi her seyde sana büyük yarari dokundu.
\par 5 Yasamini tehlikeye atarak Filistli'yi öldürdü. RAB de bütün Israil'i büyük bir zafere ulastirdi. Sen de bunu görüp sevindin. Öyleyse neden Davut'u yok yere öldürerek suçsuz birine haksizlik edesin?"
\par 6 Saul Yonatan'in söylediklerinden etkilenerek ant içti: "Yasayan RAB'bin adiyla derim ki, Davut öldürülmeyecektir."
\par 7 Bunun üzerine Yonatan Davut'u çagirip ona her seyi anlatti. Sonra Davut'u Saul'un yanina getirdi. Davut da önceden oldugu gibi kralin hizmetine girdi.
\par 8 Savas yine patlak verdi. Davut gidip Filistliler'e karsi savasti. Onlari öyle büyük bir bozguna ugratti ki, önünden kaçtilar.
\par 9 Bir gün Saul, mizragi elinde evinde oturuyor, Davut da lir çaliyordu. Derken RAB'bin gönderdigi kötü bir ruh Saul'u yakaladi.
\par 10 Saul mizragiyla Davut'u duvara çakmaya çalisti. Ancak Davut yana kaçinca Saul'un mizragi duvara saplandi. O gece Davut kaçip kurtuldu.
\par 11 Saul, Davut'u gözetlemeleri, ertesi sabah da öldürmeleri için evine ulaklar gönderdi. Ama karisi Mikal Davut'a, "Bu gece kaçip kurtulamazsan, yarin öldürüleceksin" dedi.
\par 12 Sonra Davut'u pencereden asagiya indirdi. Böylece Davut kaçip kurtuldu.
\par 13 Mikal aile putunu alip yataga koydu, üstüne yorgani örttü, bas tarafina da keçi kilindan bir yastik yerlestirdi.
\par 14 Saul'un gönderdigi ulaklar Davut'u yakalamaya geldiginde, Mikal, "Davut hasta" dedi.
\par 15 Saul Davut'u görmeleri için ulaklari yeniden göndererek, "Onu yatagiyla buraya getirin de öldüreyim" diye buyurdu.
\par 16 Ulaklar eve girince, yatakta basinda keçi kilindan yastik olan putu gördüler.
\par 17 Saul Mikal'a "Neden beni böyle kandirip düsmanimin kaçmasini sagladin?" diye sordu. Mikal, "Davut bana, 'Birak beni gideyim, yoksa seni öldürürüm dedi" diye yanitladi.
\par 18 Kaçip kurtulan Davut, Rama'da yasayan Samuel'in yanina gitti. Saul'un kendisine bütün yaptiklarini ona anlatti. Sonra Samuel'le birlikte Nayot Mahallesi'ne gidip orada kaldi.
\par 19 Davut'un Rama'nin Nayot Mahallesi'nde oldugu haberi Saul'a ulastirildi.
\par 20 Bunun üzerine Saul Davut'u yakalamalari için ulaklarini oraya gönderdi. Ulaklar Samuel'in önderliginde bir peygamber toplulugunun oynayip costugunu gördüler. Iste o zaman Tanri'nin Ruhu Saul'un ulaklarinin üzerine indi. Onlar da oynayip cosmaya basladilar.
\par 21 Saul olup bitenleri duyunca, baska ulaklar gönderdi. Onlar da oynayip costular*fi*. Saul'un üçüncü kez gönderdigi ulaklar da öncekiler gibi yapti.
\par 22 Sonunda Saul kendisi Rama'ya dogru yola çikti. Seku'daki büyük sarnica varinca, "Samuel'le Davut neredeler?" diye sordu. Biri, "Rama'nin Nayot Mahallesi'nde" dedi.
\par 23 Saul Rama'daki Nayot'a dogru ilerlerken, Tanri'nin Ruhu onun üzerine de indi. Nayot'a varincaya dek yol boyunca oynayip costu*fi*.
\par 24 Giysilerini de çikarip Samuel'in önünde oynayip costu*fi*. Bütün gün ve gece çiplak yatti. Halkin, "Saul da mi peygamber oldu?" demesi bundandir.

\chapter{20}

\par 1 Davut Rama'nin Nayot Mahallesi'nden kaçtiktan sonra Yonatan'a gitti. Ona, "Ne yaptim? Suçum ne?" diye sordu, "Babana karsi ne günah isledim ki, beni öldürmek istiyor?"
\par 2 Yonatan, "Bu senden uzak olsun, ölmeyeceksin!" diye yanitladi, "Babam bana bildirmeden ister büyük, ister küçük olsun hiçbir is yapmaz. Neden bunu benden gizlesin? Olmaz öyle sey!"
\par 3 Ancak Davut ant içerek, "Senin beni sevdigini baban çok iyi biliyor" diye yanitladi, "'Yonatan ne yapacagimi bilmemeli, yoksa üzülür diye düsünmüstür. RAB'bin ve senin yasamin hakki için derim ki, ölüm ile aramda yalniz bir adim var."
\par 4 Yonatan Davut'a, "Ne dilersen dile, senin için yaparim" diye karsilik verdi.
\par 5 Davut Yonatan'a, "Bak, yarin Yeni Ay Töreni" dedi, "Kralla birlikte yemege oturmam gerekir. Ama izin ver, ertesi günün aksamina dek tarlada gizleneyim.
\par 6 Eger baban yoklugumu sezerse ona, 'Davut aceleyle kendi kenti Beytlehem'e gitmek için benden israrla izin istedi; orada bütün ailenin yillik kurban töreni var dersin.
\par 7 Baban, 'Iyi derse, kulun güvenlikte demektir. Ama öfkelenirse, bil ki, bana kötülük yapmaya karar vermistir.
\par 8 Sana gelince, bana yardim et; çünkü RAB'bin önünde benimle antlasma yaptin. Suçluysam, beni sen öldür! Neden beni babana teslim edesin?"
\par 9 Yonatan, "Olmaz öyle sey!" diye yanitladi, "Babamin sana kötülük yapmaya karar verdigini bilsem, sana söylemez miydim?"
\par 10 Davut, "Baban sana sert bir karsilik verirse, kim bana bildirecek?" diye sordu.
\par 11 Yonatan, "Gel, tarlaya gidelim" dedi. Böylece ikisi tarlaya gittiler.
\par 12 Yonatan Davut'la konusmasini sürdürdü: "Israil'in Tanrisi RAB tanik olsun! Yarin ya da öbür gün bu saate kadar babamin ne düsündügünü arastiracagim. Babamin sana karsi tutumu olumluysa, sana haber gönderecegim.
\par 13 Ama babam seni öldürmeyi tasarliyorsa, bunu sana bildirip güvenlik içinde gitmeni saglamazsam, RAB bana aynisini, hatta daha kötüsünü yapsin! RAB önceden babamla oldugu gibi seninle de birlikte olsun!
\par 14 Ama sen yasamim boyunca RAB'bin iyiligini bana göster ki ölmeyeyim.
\par 15 RAB Davut'un bütün düsmanlarini yeryüzünden yok edecegi zaman bile, sen soyuma iyiliklerini sonsuza dek esirgeme."
\par 16 Böylece Yonatan Davut soyuyla bir antlasma yapti ve, "RAB Davut'un düsmanlarini cezalandirsin" dedi.
\par 17 Davut'a besledigi sevgiden ötürü Yonatan ona bir daha ant içirtti. Çünkü onu cani kadar seviyordu.
\par 18 Yonatan Davut'a, "Yarin Yeni Ay Töreni" dedi, "Yerin bos kalacagindan, yoklugun anlasilacak.
\par 19 Öbür gün, geçen sefer gizlendigin yere çabucak git. Ezel Tasi'nin yaninda bekle.
\par 20 Ben hedefe atar gibi tasin bir yanina üç ok atacagim.
\par 21 Sonra hizmetkârimi gönderip, 'Git oklari bul diye buyruk verecegim. Eger özellikle ona, 'Bak, oklar senin bu yaninda, onlari alip buraya getir dersem, gel. Yasayan RAB'bin adiyla derim ki, güvenliktesin, tehlike yok.
\par 22 Ama hizmetkâra, 'Bak, oklar ötende dersem, git; çünkü RAB seni uzaklastirmistir.
\par 23 Birbirimizle yaptigimiz antlasmaya gelince, RAB sonsuza dek seninle benim aramda tanik olsun."
\par 24 Böylece Davut tarlada gizlendi. Yeni Ay Töreni baslayinca, Kral Saul gelip yemege oturdu.
\par 25 Her zamanki gibi duvarin yanindaki yerine oturmustu. Yonatan karsisinda, Avner de yaninda yerlerini aldilar. Davut'un yeriyse bos kaldi.
\par 26 Ama Saul o gün bir sey söylemedi. "Davut'un basina bir sey gelmis olmali. Dinsel açidan kirli olsa gerek, evet dinsel açidan temiz degildir" diye düsündü.
\par 27 Ertesi gün, ayin ikinci günü, Davut'un yeri yine bostu. Bunun üzerine Saul, oglu Yonatan'a, "Isay'in oglu neden dün de, bugün de yemege gelmedi?" diye sordu.
\par 28 Yonatan, "Davut Beytlehem'e gitmek için benden israrla izin istedi" diye karsilik verdi,
\par 29 "'Lütfen izin ver. Çünkü ailemizin kentte bir kurbani var, agabeyim orada bulunmami buyurdu. Gözünde lütuf bulduysam gidip kardeslerimi göreyim dedi. Iste bu yüzden kralin sofrasina gelemedi."
\par 30 Saul Yonatan'a öfkelenerek, "Seni sapik ve dikbasli kadinin oglu!" diye bagirdi, "Isay'in oglunu destekledigini bilmiyor muyum? Bu kendin için de, seni doguran annen için de utanç verici.
\par 31 Çünkü Isay'in oglu yeryüzünde yasadikça ne sen güvenlikte olabilirsin, ne de kralligin. Simdi adam gönder, onu bana getir. O ölmeli!"
\par 32 Yonatan babasi Saul'a, "Neden ölmeli? Ne yapti ki?" diye karsilik verdi.
\par 33 Ama Saul Yonatan'i öldürmek amaciyla mizragini ona firlatti. Böylece Yonatan babasinin Davut'u öldürmeye kararli oldugunu anladi.
\par 34 Büyük bir öfkeyle sofradan kalkti ve ayin ikinci günü hiç yemek yemedi. Babasinin Davut'u böyle asagilamasina üzüldü.
\par 35 Sabahleyin Yonatan Davut'la bulusmak üzere tarlaya gitti. Yanina bir usak almisti.
\par 36 Usaga, "Haydi kos, atacagim oklari bul" dedi. Usak kosarken, Yonatan onun ötesine bir ok atti.
\par 37 Usak Yonatan'in attigi okun düstügü yere varinca, Yonatan, "Ok ötende!" diye seslendi,
\par 38 "Çabuk ol! Kos, yerinde durma!" Yonatan'in usagi oku alip efendisine getirdi.
\par 39 Olup bitenden habersizdi. Olanlari yalniz Yonatan'la Davut biliyordu.
\par 40 Yonatan, silahlarini yanindaki usaga vererek, "Al bunlari kente götür" dedi.
\par 41 Usak gider gitmez, Davut tasin güney yanindan ayaga kalkti ve yüzüstü yere kapanarak üç kez egildi. Iki arkadas birbirlerini öpüp agladilar; ancak Davut daha çok agladi.
\par 42 Yonatan, "Esenlikle yoluna git" dedi, "Ikimiz RAB'bin adiyla ant içmistik. RAB seninle benim aramda ve soylarimiz arasinda sonsuza dek tanik olsun." Bundan sonra Davut yoluna gitti. Yonatan da kente döndü.

\chapter{21}

\par 1 Davut Nov Kenti'ne, Kâhin Ahimelek'in yanina gitti. Ahimelek titreyerek Davut'u karsilamaya çikti. "Neden yalnizsin? Neden yaninda kimse yok?" diye sordu.
\par 2 Davut söyle yanitladi: "Kral bana bir görev verdi. 'Sana verdigim görevden ve buyruklardan kimsenin haberi olmasin dedi. Adamlarima gelince, belli bir yere gitmelerini söyledim.
\par 3 Su an elinde ne var? Bana bes somun ekmek ya da baska ne varsa ver."
\par 4 Kâhin, "Taze ekmegim yok" diye karsilik verdi, "Ama adamlarin kadindan uzak kaldilarsa kutsanmis ekmek*fk* var."
\par 5 Davut, "Yola çiktigimizdan her zaman oldugu gibi, kadindan uzak kaldik" dedi, "Siradan bir yolculuga çiktigimizda bile adamlarim kendilerini temiz tutarlar; özellikle bugün ne kadar daha çok temiz olacaklar."
\par 6 Bunun üzerine kâhin ona kutsanmis ekmek verdi; çünkü orada huzura konan ekmekten baska ekmek yoktu. Bu ekmek RAB'bin huzurundan alindigi gün yerine sicak ekmek konurdu.
\par 7 O gün Saul'un görevlilerinden Edomlu Doek adindaki bas çoban RAB'bin önünde dinsel görevini yerine getirmek üzere orada bulunuyordu.
\par 8 Davut Ahimelek'e, "Yaninda mizrak ya da kiliç yok mu?" diye sordu, "Kralin isi acele oldugundan, yanima ne kilicimi aldim, ne de baska bir silah."
\par 9 Kâhin, "Ela Vadisi'nde öldürdügün Filistli Golyat'in kilici var" diye karsilik verdi, "Efodun* arkasinda beze sarili duruyor. Burada baska silah yok. Istersen onu alabilirsin." Davut, "Onun gibisi yoktur, onu bana ver" dedi.
\par 10 Saul'dan kaçan Davut o gün Gat Krali Akis'e gitti.
\par 11 Akis'in görevlileri, "Bu Israil Krali Davut degil mi?" dediler, "Çalip oynarken, 'Saul binlercesini öldürdü, Davut'sa on binlercesini diye hakkinda ezgiler okuduklari kisi bu degil mi?"
\par 12 Bu sözler Davut'u derin derin düsündürdü. Gat Krali Akis'ten çok korkan Davut, onlarin önünde tutumunu degistirerek deli gibi davrandi. Kentin kapilarini tirmaladi, salyasini sakalina akitti.
\par 14 Akis görevlilerine, "Su adama bakin!" dedi, "Delinin biri! Onu neden bana getirdiniz?
\par 15 Bizde deliler eksik mi ki, önümde delilik yapsin diye bu adami getirdiniz? Bu adamin sarayima girmesi sart mi?"

\chapter{22}

\par 1 Davut Gat'tan ayrilip Adullam Magarasi'na kaçti. Bunu duyan kardesleri ve ailesinin öteki bireyleri yanina gittiler.
\par 2 Sikintisi, borcu, hosnutsuzlugu olan herkes Davut'un çevresinde toplandi. Davut sayisi dört yüze varan bu adamlara önderlik yapti.
\par 3 Davut oradan Moav'daki Mispa Kenti'ne gitti. Moav Krali'ndan, "Tanri'nin bana ne yapacagi belli oluncaya dek annemle babamin gelip yaninizda kalmasina izin verir misin?" diye bir istekte bulundu.
\par 4 Böylece Davut annesiyle babasini Moav Krali'nin yanina birakti. Davut siginakta kaldigi sürece onlar da Moav Krali'nin yaninda kaldilar.
\par 5 Ne var ki, Peygamber Gad Davut'a, "Siginakta kalma. Yahuda ülkesine git" dedi. Bunun üzerine Davut oradan ayrilip Heret Ormani'na gitti.
\par 6 Bu sirada Saul Davut'la yanindakilerin nerede olduklarini ögrendi. Saul elinde mizragiyla Giva'da bir tepedeki ilgin agacinin altinda oturuyordu. Askerleri de çevresinde duruyordu.
\par 7 Saul onlara söyle dedi: "Ey Benyaminliler, simdi dinleyin! Isay'in oglu her birinize tarlalar, baglar mi verecek? Her birinizi binbasi, yüzbasi mi yapacak?
\par 8 Hepiniz bana karsi düzen kurdunuz. Çünkü oglum Isay'in ogluyla antlasma yaptiginda bana haber veren olmadi. Içinizden bana aciyan tek kisi çikmadi. Bugün oldugu gibi, bana pusu kurmasi için oglumun kulum Davut'u kiskirttigini bana bildiren olmadi."
\par 9 Bunun üzerine Saul'un askerlerinin yaninda duran Edomlu Doek, "Isay oglu Davut'un Nov Kenti'ne, Ahituv oglu Kâhin Ahimelek'in yanina geldigini gördüm" dedi,
\par 10 "Ahimelek Davut için RAB'be danisti. Ona hem yiyecek sagladi, hem de Filistli Golyat'in kilicini verdi."
\par 11 Kral Saul, Ahituv oglu Kâhin Ahimelek'i ve babasinin ailesinden Nov'da yasayan bütün kâhinleri çagirmak için ulaklar gönderdi. Hepsi kralin yanina geldi.
\par 12 Saul Ahimelek'e, "Ey Ahituv oglu, beni dinle!" dedi. Ahimelek, "Buyur, efendim" diye yanitladi.
\par 13 Saul, "Neden sen ve Isay oglu bana karsi düzen kurdunuz?" dedi, "Çünkü ona ekmek, kiliç verdin ve onun için Tanri'ya danistin. O da bana karsi ayaklandi ve bugün yaptigi gibi pusu kurdu."
\par 14 Ahimelek, "Bütün görevlilerin arasinda Davut kadar sana bagli biri var mi?" diye karsilik verdi, "Davut senin damadin, muhafiz birligi komutanin ve ailende saygin biridir.
\par 15 Ben Davut için Tanri'ya danismaya o gün mü basladim? Kesinlikle hayir! Kral ben kulunu ve babasinin ailesini suçlamasin. Çünkü kulun bu konuda hiçbir sey bilmiyor."
\par 16 Ama Saul, "Ey Ahimelek, sen de bütün ailen de kesinlikle öleceksiniz" dedi.
\par 17 Sonra yaninda duran nöbetçi askerlere, "Gidin ve Davut'u destekleyen RAB'bin kâhinlerini öldürün!" dedi, "Çünkü onun kaçtigini bildikleri halde bana haber vermediler." Ne var ki, kralin görevlileri el kaldirip RAB'bin kâhinlerini öldürmek istemediler.
\par 18 Bunun üzerine kral, Doek'e, "Sen git, kâhinleri öldür" diye buyurdu. Edomlu Doek de gidip kâhinleri öldürdü. O gün Doek keten efod* giymis seksen bes kisi öldürdü.
\par 19 Kadin erkek, çoluk çocuk demeden kâhinler kenti Nov'un halkini kiliçtan geçirdi. Sigirlari, esekleri, koyunlari da öldürdü.
\par 20 Yalniz Ahituv oglu Kâhin Ahimelek'in ogullarindan Aviyatar adinda biri kurtulup Davut'a kaçti.
\par 21 Aviyatar Saul'un RAB'bin kâhinlerini öldürttügünü Davut'a söyledi.
\par 22 Davut Aviyatar'a, "O gün orada bulunan Edomlu Doek'in olup biteni Saul'a bildirecegini anlamistim zaten" dedi, "Babanin bütün aile bireylerinin ölümüne ben neden oldum.
\par 23 Yanimda kal ve korkma! Seni öldürmek isteyen beni de öldürmek istiyor. Yanimda güvenlikte olursun."

\chapter{23}

\par 1 Davut'a, "Filistliler Keila Kenti'ne saldirip harmanlari yagmaliyorlar" diye haber verdiler.
\par 2 Davut RAB'be, "Gidip su Filistliler'e saldirayim mi?" diye danisti. RAB, "Git, Filistliler'e saldir ve Keila Kenti'ni kurtar" diye yanitladi.
\par 3 Ama adamlari Davut'a, "Bak, biz burada Yahuda'dayken korkuyoruz" dediler, "Keila'ya Filist ordusuna karsi savasmaya gidersek büsbütün korkariz."
\par 4 Bunun üzerine Davut RAB'be bir kez daha danisti. RAB ona yine, "Kalk, Keila'ya git! Çünkü Filistliler'i senin eline ben teslim edecegim" dedi.
\par 5 Böylece Davut'la adamlari Keila'ya gidip Filistliler'e karsi savastilar. Davut onlarin hayvanlarini ele geçirdi. Filistliler'i agir bir yenilgiye ugratarak Keila halkini kurtardi.
\par 6 Ahimelek'in oglu Aviyatar kaçip Keila'da bulunan Davut'a gittiginde, efodu* da birlikte götürmüstü.
\par 7 Saul, Davut'un Keila Kenti'ne gittigini duyunca, "Tanri Davut'u elime teslim etti" dedi, "Davut sürgülü kapilari olan bir kente girmekle kendini hapsetmis oldu."
\par 8 Böylece Saul, Keila'ya yürüyüp Davut'la adamlarini kusatmak amaciyla bütün halki savasa çagirdi.
\par 9 Davut, Saul'un kendisine bir düzen kurdugunu duyunca, Kâhin Aviyatar'a, "Efodu* getir" dedi.
\par 10 Sonra söyle yakardi: "Ey Israil'in Tanrisi RAB! Ben kulun yüzünden Saul'un gelip Keila'yi yikmayi tasarladigina dair kesin haber aldim.
\par 11 Keila halki beni onun eline teslim eder mi? Kulunun duymus oldugu gibi Saul gelecek mi? Ey Israil'in Tanrisi RAB, yalvaririm, kuluna bildir!" RAB, "Saul gelecek" yanitini verdi.
\par 12 Davut RAB'be, "Keila halki beni ve adamlarimi Saul'un eline teslim edecek mi?" diye sordu. RAB, "Teslim edecek" dedi.
\par 13 Bunun üzerine Davut ile yanindaki alti yüz kadar kisi Keila'dan ayrilip oradan oraya yer degistirmeye basladilar. Davut'un Keila'dan kaçtigini ögrenen Saul oraya gitmekten vazgeçti.
\par 14 Davut kirsal bölgedeki siginaklarda ve Zif Çölü'nün daglik kesiminde kaldi. Saul her gün Davut'u aradigi halde, Tanri onu Saul'un eline teslim etmedi.
\par 15 Davut Zif Çölü'nde, Hores'teyken, Saul'un kendisini öldürmek için yola çiktigini ögrendi.
\par 16 Bu arada Saul oglu Yonatan kalkip Hores'e, Davut'un yanina gitti ve onu Tanri'nin adiyla yüreklendirdi.
\par 17 "Korkma!" dedi, "Babam Saul sana dokunmayacak. Sen Israil Krali olacaksin, ben de senin yardimcin olacagim. Babam Saul da bunu biliyor."
\par 18 Ikisi de RAB'bin önünde aralarindaki antlasmayi yenilediler. Sonra Yonatan evine döndü, Davut ise Hores'te kaldi.
\par 19 Zifliler Giva'ya gidip Saul'a, "Davut aramizda" dediler, "Yesimon'un güneyinde, Hakila Tepesi'ndeki Hores siginaklarinda gizleniyor.
\par 20 Ey kral, ne zaman gelmek istersen gel! Davut'u kralin eline teslim etmeyi ise bize birak."
\par 21 Saul, "RAB sizi kutsasin! Bana acidiniz" dedi,
\par 22 "Gidin ve bir daha arastirin; Davut'un genellikle nerelerde gizlendigini, orada onu kimin gördügünü iyice ögrenin. Çünkü onun çok kurnaz oldugunu söylüyorlar.
\par 23 Gizlendigi yerlerin hepsini ögrenip bana kesin bir haber getirin. O zaman ben de sizinle gelirim. Eger Davut o bölgedeyse, bütün Yahuda boylari içinde onu arayip bulacagim."
\par 24 Böylece Zifliler kalkip Saul'dan önce Zif'e gittiler. O sirada Davut'la adamlari Yesimon'un güneyindeki Arava'da, Maon Çölü'ndeydiler.
\par 25 Saul ile adamlarinin kendisini aramaya geldiklerini ögrenince Davut asagiya inip Maon Çölü'ndeki kayaliga sigindi. Saul bunu duyunca Davut'un ardindan Maon Çölü'ne gitti.
\par 26 Saul dagin bir yanindan, Davut'la adamlari ise öbür yanindan ilerliyordu. Davut Saul'dan kaçip kurtulmaya çalisiyordu. Saul'la askerleri Davut'la adamlarini yakalamak üzere yaklasirken,
\par 27 bir ulak gelip Saul'a söyle dedi: "Çabuk gel! Filistliler ülkeye saldiriyor."
\par 28 Bunun üzerine Saul Davut'u kovalamayi birakip Filistliler'le savasmaya gitti. Bu yüzden oraya Sela-Hammahlekot adi verildi.
\par 29 Davut oradan ayrilip Eyn-Gedi bölgesindeki siginaklara gizlendi.

\chapter{24}

\par 1 Saul Filistliler'i kovalamaktan dönünce, Davut'un Eyn-Gedi Çölü'nde oldugu haberini aldi.
\par 2 Saul da Davut'la adamlarini Dag Keçisi Kayaligi dolaylarinda arayip bulmak için, bütün Israil'den üç bin seçme asker alip yola çikti.
\par 3 Yolda koyun agillarina rastladi. Yakinda bir de magara vardi. Saul ihtiyacini gidermek için magaraya girdi. Davut'la adamlari magaranin en iç bölümünde kaliyorlardi.
\par 4 Adamlari, Davut'a, "Iste RAB'bin sana, 'Diledigini yapabilmen için düsmanini eline teslim edecegim dedigi gün bugündür" dediler. Davut kalkip Saul'un cüppesinin eteginden gizlice bir parça kesti.
\par 5 Ama sonradan Saul'un eteginden bir parça kestigi için kendini suçlu buldu.
\par 6 Adamlarina, "Efendime, RAB'bin meshettigi* kisiye karsi böyle bir sey yapmaktan, el kaldirmaktan RAB beni uzak tutsun" dedi, "Çünkü o RAB'bin meshettigi kisidir."
\par 7 Davut bu sözlerle adamlarini engelledi ve Saul'a saldirmalarina izin vermedi. Saul magaradan çikip yoluna koyuldu.
\par 8 O zaman Davut da magaradan çikti. Saul'a, "Efendim kral!" diye seslendi. Saul arkasina bakinca, Davut egilip yüzüstü yere kapandi.
\par 9 "'Davut sana kötülük yapmak istiyor diyenlerin sözlerini neden önemsiyorsun?" dedi,
\par 10 "Bugün RAB'bin magarada seni elime nasil teslim ettigini gözünle görüyorsun. Bazilari seni öldürmemi istedi. Ama ben seni esirgeyip, 'Efendime el kaldirmayacagim, çünkü o RAB'bin meshettigi kisidir dedim.
\par 11 Ey baba, cüppenin eteginden kesilmis, elimdeki su parçaya bak; evet, bak! Cüppenden bir parça kestim, ama seni öldürmedim. Bundan ötürü içimde kötülük ve baskaldirma düsüncesi olmadigini iyice bilesin. Sana kötülük yapmadigim halde sen beni öldürmeye çalisiyorsun.
\par 12 RAB aramizda yargiç olsun ve benim öcümü senden O alsin. Ama ben elimi sana karsi kaldirmayacagim.
\par 13 Eskilerin su, 'Kötülük kötü kisilerden gelir deyisi uyarinca elim sana karsi kalkmayacaktir.
\par 14 Israil Krali kime karsi çikmis? Sen kimi kovaliyorsun? Ölü bir köpek mi? Bir pire mi?
\par 15 RAB yargiç olsun ve hangimizin hakli olduguna O karar versin. RAB davama baksin ve beni savunup senin elinden kurtarsin."
\par 16 Davut söylediklerini bitirince, Saul, "Davut oglum, bu senin sesin mi?" diye sordu ve hiçkira hiçkira aglamaya basladi.
\par 17 Sonra, "Sen benden daha dogru bir adamsin" dedi, "Sana kötülük yaptigim halde sen bana iyilikle karsilik verdin.
\par 18 Bugün bana iyi davrandigini kanitladin: RAB beni eline teslim ettigi halde beni öldürmedin.
\par 19 Düsmanini yakalayan biri onu güvenlik içinde saliverir mi? Bugün bana yaptigin iyilige karsilik RAB de seni iyilikle ödüllendirsin.
\par 20 Simdi anladim ki, sen gerçekten kral olacaksin ve Israil Kralligi senin egemenligin altinda sürecek.
\par 21 Benden sonra soyumu ortadan kaldirmayacagina, babamin ailesinden adimi silmeyecegine dair RAB'bin önünde ant iç."
\par 22 Davut Saul'un istedigi gibi ant içti. Sonra Saul evine döndü. Davut'la adamlari da siginaga gittiler.

\chapter{25}

\par 1 Bu sirada Samuel öldü. Bütün Israilliler toplanip onun için yas tuttular. Onu Rama'daki evine gömdüler. Bundan sonra Davut Maon Çölü'ne gitti.
\par 2 Maon'da çok varlikli bir adam vardi; isi Karmel'deydi. Üç bin koyunu, bin keçisi vardi. O sirada Karmel'de koyunlarini kirkmaktaydi.
\par 3 Adamin adi Naval, karisinin adi da Avigayil'di. Kadin saggörülü ve güzeldi. Ama Kalev soyundan gelen kocasi kaba, kötü huylu biriydi.
\par 4 Davut kirdayken, Naval'in koyunlarini kirktigini duydu.
\par 5 On usagi su buyrukla ona gönderdi: "Karmel'de Naval'in yanina gidin. Benden ona selam söyleyip
\par 6 söyle deyin: 'Ömrün uzun olsun! Sana, ailene ve sana bagli olan herkese esenlik olsun!
\par 7 Simdi koyunlarin kirkma zamani oldugunu duydum. Çobanlarin bizimle birlikteyken, onlari incitmedik. Karmel'de kaldiklari sürece hiçbir kayiplari olmadi.
\par 8 Usaklarina sor, sana söyleyecekler. Bunun için adamlarima yakinlik göster. Çünkü sana senlik zamaninda geldik. Lütfen kullarina ve oglun Davut'a elinden geleni ver."
\par 9 Davut'un adamlari varip Davut adina bu sözleri Naval'a ilettiler ve beklemeye basladilar.
\par 10 Ne var ki, Naval Davut'un adamlarina su karsiligi verdi: "Bu Davut da kim? Isay'in oglu da kim oluyor? Bu günlerde birçok köle efendilerini birakip kaçiyor.
\par 11 Ekmegimi, suyumu, kirkicilarim için kestigim hayvanlarin etini alip nereden geldiklerini bilmedigim kisilere mi vereyim?"
\par 12 Davut'un adamlari geldikleri yoldan döndüler ve Naval'in bütün söylediklerini Davut'a bildirdiler.
\par 13 Davut adamlarina, "Herkes kilicini kusansin!" diye buyruk verdi. Davut da, adamlari da kiliçlarini kusandilar. Yaklasik dört yüz adam Davut'la birlikte gitti; iki yüz kisi de erzagin yaninda kaldi.
\par 14 Naval'in usaklarindan biri, Naval'in karisi Avigayil'e, "Davut efendimiz Naval'a esenlik dilemek için kirdan ulaklar gönderdi" dedi, "Ama Naval onlari tersledi.
\par 15 Oysa adamlar bize çok iyi davrandilar. Bizi incitmediler. Kirda onlarla birlikte kaldigimiz sürece hiçbir seyimiz kaybolmadi.
\par 16 Koyunlarimizi güderken, yanlarinda kaldigimiz sürece gece gündüz bizi korudular.
\par 17 Simdi ne yapman gerektigini iyi düsün. Çünkü efendimize ve bütün ailesine kötülük yapmayi tasarliyorlar. Üstelik efendimiz o kadar kötü ki, kimse ona bir sey söyleyemiyor."
\par 18 Bunun üzerine Avigayil, hiç zaman yitirmeden, iki yüz ekmek, iki tulum sarap, hazirlanmis bes koyun, bes sea kavrulmus bugday, yüz salkim kuru üzüm ve iki yüz parça incir pestili alip eseklere yükledi.
\par 19 Sonra usaklarina, "Önümden gidin, ben arkanizdan geliyorum" dedi. Kocasi Naval'a hiçbir sey söylemedi.
\par 20 Avigayil esege binmis, dagin öbür yolundan inerken, Davut'la adamlari da ona dogru ilerliyorlardi. Avigayil onlarla karsilasti.
\par 21 Davut, "Bu adamin kirdaki malini dogrusu bos yere korudum" demisti, "Onun mallarindan hiçbir sey eksilmedi. Öyleyken bana iyilik yapacagina kötülükle karsilik verdi.
\par 22 Eger sabaha dek adamlarindan tek birini bile sag birakirsam, Tanri bana aynisini, hatta daha kötüsünü yapsin!"
\par 23 Avigayil Davut'u görünce hemen esekten indi; Davut'un önünde egilip yüzüstü yere kapandi.
\par 24 Onun ayaklarina kapanarak söyle yalvardi: "Efendim, suçu ben, yalniz ben üstüme aliyorum. Izin ver, ben kölen seninle konussun, onun söyleyeceklerini dinle.
\par 25 Yalvaririm, efendim, o kötü adam Naval'a aldirma. Çünkü kisiligi tipki adi gibidir. Adi akilsiz anlamina gelir; kendisi de akilsizin biridir. Ben kulun, efendim Davut'un gönderdigi ulaklari görmedim.
\par 26 "Ama simdi, ey efendim, RAB senin kan dökmene ve kendi elinle öç almana engel oldu. Yasayan RAB'bin adi ve senin yasamin hakki için yalvaririm, düsmanlarin ve efendime kötülük tasarlayanlarin tümü Naval gibi olsun.
\par 27 Ben kölenin efendime getirdigi bu armagan, seni izleyen adamlarina verilsin.
\par 28 Lütfen kölenin suçunu bagisla. RAB kesinlikle efendimin soyunu sürdürecektir; çünkü efendim RAB'bin savaslarini sürdürüyor. Yasadigin sürece sende hiçbir haksizlik bulunmasin.
\par 29 Biri kalkip seni öldürmek amaciyla ardina düserse, yasamini Tanrin RAB güven altinda tutacaktir; düsmanlarini sapanla tas atar gibi firlatip atacaktir.
\par 30 RAB, efendime söz verdigi bütün iyilikleri yerine getirip onu Israil'e önder atadiginda,
\par 31 kendi öcünü almak ugruna bos yere kan dökmedigin için pismanlik ve üzüntü duymayacaksin. RAB efendimi basariya ulastirdiginda köleni animsa."
\par 32 Davut, "Bugün seni karsima çikaran Israil'in Tanrisi RAB'be övgüler olsun!" diye karsilik verdi,
\par 33 "Anlayisini kutlarim! Bugün kan dökmemi ve öcümü elimle almami engelledigin için seni kutlarim.
\par 34 Dogrusu sana kötülük etmemi önleyen Israil'in Tanrisi yasayan RAB'bin adiyla derim ki, beni karsilamak için hemen gelmemis olsaydin, gün doguncaya dek Naval'in adamlarindan hiçbiri sag kalmayacakti."
\par 35 Avigayil'in kendisine getirdiklerini kabul eden Davut, "Esenlikle evine dön. Sözlerine kulak verip dilegini kabul ettim" dedi.
\par 36 Avigayil Naval'in yanina döndü. Naval evinde krallara yarasir bir sölen düzenlemisti. Çok sarhos oldugundan neseliydi. Bu yüzden Avigayil sabaha dek ona bir sey söylemedi.
\par 37 Ama ertesi sabah Naval ayilinca karisi ona olup bitenleri anlatti. Iste o an Naval'in kalbi sikisti ve felç oldu.
\par 38 Yaklasik on gün sonra da RAB Naval'i cezalandirip öldürdü.
\par 39 Davut, Naval'in öldügünü duyunca, "Beni küçümseyen Naval'a karsi davama bakan, kulunu kötülük etmekten alikoyan RAB'be övgüler olsun!" dedi, "RAB Naval'in kötülügünü onun basina döndürdü." Sonra Davut Avigayil'e evlenme teklifinde bulunmak için ulaklar gönderdi.
\par 40 Davut'un ulaklari Karmel'e, Avigayil'in yanina varip, "Davut sana evlenme teklifinde bulunmak için bizi gönderdi" dediler.
\par 41 Avigayil yüzüstü yere kapanarak, "Ben kölen sana hizmet etmeye ve efendimin ulaklarinin ayaklarini yikamaya hazirim" diye yanitladi.
\par 42 Hemen kalkip esege bindi. Yanina bes hizmetçisini alip Davut'un ulaklarini izleyerek yola koyuldu. Sonra Davut'un karisi oldu.
\par 43 Davut Yizreelli Ahinoam'i da es olarak almisti. Böylece ikisi de onun karisi oldular.
\par 44 Bu arada Saul, Davut'un karisi olan kizi Mikal'i Gallimli Layis oglu Palti'ye vermisti.

\chapter{26}

\par 1 Zifliler Giva'ya, Saul'un yanina gidip, "Davut Yesimon'a bakan Hakila Tepesi'nde gizleniyor" dediler.
\par 2 Bunun üzerine Saul üç bin seçme Israilli askerle Zif Çölü'nde Davut'u aramaya çikti.
\par 3 Yesimon'a bakan Hakila Tepesi'nde, yol kenarinda ordugah kurdu. Kirda bulunan Davut, Saul'un pesine düstügünü anlayinca,
\par 4 gözcü gönderdi. Böylece Saul'un oraya geldigini saptadi.
\par 5 Bunun üzerine Davut, Saul'un ordugah kurdugu yere gitti ve Saul'la ordusunun baskomutani Ner oglu Avner'in nerede yattiklarini gördü. Saul ordugahin ortasinda, askerler de çevresinde yatiyorlardi.
\par 6 O zaman Davut, Hititli* Ahimelek ile Yoav'in kardesi, Seruya oglu Avisay'a, "Kim benimle ordugaha, Saul'un yanina gelecek?" diye sordu. Avisay, "Ben seninle gelecegim" diye karsilik verdi.
\par 7 Davut'la Avisay o gece ordugaha girdiler. Saul, mizragi basucunda yere saplanmis, ordugahin ortasinda uyuyordu. Avner'le askerler de çevresinde uyuyorlardi.
\par 8 Avisay Davut'a, "Bugün Tanri düsmanini senin eline teslim etti" dedi, "Simdi birak da, onu kendi mizragiyla bir atista yere çakayim. Ikinci kez vurmama gerek kalmayacak."
\par 9 Ne var ki Davut, "Onu öldürme!" dedi, "RAB'bin meshettigi* kisiye kim el uzatirsa, suçlu çikar.
\par 10 Yasayan RAB'bin adiyla derim ki, RAB kendisi onu öldürecektir; ya günü gelince ölecek, ya da savasta vurulup yok olacak.
\par 11 Ama RAB'bin meshettigi kisiye el uzatmaktan RAB beni uzak tutsun! Haydi, Saul'un basucundaki mizrakla su matarasini al da gidelim."
\par 12 Böylece Davut Saul'un basucundan mizragini ve su matarasini aldi. Sonra oradan uzaklastilar. Onlari gören olmadi. Kimse olup bitenin farkina varmadi, uyanan da olmadi. Hepsi uyuyorlardi, çünkü RAB onlara derin bir uyku vermisti.
\par 13 Davut karsi yakaya geçip tepenin üstünde, onlardan uzak bir yerde durdu. Aralarinda epeyce mesafe vardi.
\par 14 Davut askerlere ve Ner oglu Avner'e, "Ey Avner, bana yanit vermeyecek misin?" diye seslendi. Avner, "Sen kimsin ki krala sesleniyorsun?" diye karsilik verdi.
\par 15 Davut, "Sen yigit biri degil misin?" dedi, "Israil'de senin gibisi var mi? Öyleyse neden efendin krali korumadin? Çünkü biri onu öldürmek için ordugaha girdi.
\par 16 Görevini iyi yapmadin. Yasayan RAB'bin adiyla derim ki, hepiniz ölümü hak ettiniz; çünkü efendinizi, RAB'bin meshettigi kisiyi korumadiniz. Bak bakalim, kralin basucundaki mizragiyla su matarasi nerede?"
\par 17 Davut'un sesini taniyan Saul, "Davut, oglum, bu senin sesin mi?" diye sordu. Davut, "Evet, efendim kral, benim sesim" diye karsilik verdi,
\par 18 "Efendim, ben kulunu neden kovaliyorsun? Ne yaptim? Ne suç isledim?
\par 19 Lütfen, efendim kral, kulunun sözlerine kulak ver. Eger seni bana karsi kiskirtan RAB ise, bir sunu O'nu yatistirir. Ama bunu yapan insanlarsa, RAB'bin önünde lanetli olsunlar! Çünkü, 'Git, baska ilahlara kulluk et diyerek, RAB'bin mirasindan bana düsen paydan bugün beni uzaklastirdilar.
\par 20 Ne olur, kanim RAB'den uzak topraklara dökülmesin. Israil Krali, daglarda keklik avlayan avci gibi, bir pireyi avlamaya çikmis!"
\par 21 Bunun üzerine Saul, "Günah isledim" diye karsilik verdi, "Davut, oglum, geri dön. Bugün yasamima deger verdigin için sana bir daha kötülük yapmayacagim. Gerçekten akilsizca davrandim, çok büyük yanlislik yaptim."
\par 22 Davut, "Iste kralin mizragi!" dedi, "Adamlarindan biri gelip alsin.
\par 23 RAB herkesi dogruluguna ve bagliligina göre ödüllendirir. Bugün RAB seni elime teslim ettigi halde, ben RAB'bin meshettigi kisiye elimi uzatmak istemedim.
\par 24 Bugün ben senin yasamina nasil deger verdiysem, RAB de benim yasamima öyle deger versin ve beni her sikintidan kurtarsin."
\par 25 Saul, "Davut, oglum, RAB seni kutsasin!" dedi, "Sen kesinlikle büyük isler yapacak, basarili olacaksin!" Bundan sonra Davut yoluna koyuldu, Saul da evine döndü.

\chapter{27}

\par 1 Davut, "Bir gün Saul'un eliyle yok olacagim" diye düsündü, "Benim için en iyisi hemen Filist topraklarina kaçmak. O zaman Saul Israil'in her yaninda beni aramaktan vazgeçer; ben de onun elinden kurtulmus olurum."
\par 2 Böylece Davut'la yanindaki alti yüz kisi kalkip Gat Krali Maok oglu Akis'in tarafina geçtiler.
\par 3 Aileleriyle birlikte Gat'ta Akis'in yanina yerlestiler. Iki karisi Yizreelli Ahinoam'la Karmelli Naval'in dul karisi Avigayil de Davut'un yanindaydi.
\par 4 Saul Davut'un Gat'a kaçtigini duyunca, artik onu aramaktan vazgeçti.
\par 5 Davut Akis'e, "Benden hosnut kaldiysan, çevre kentlerden birinde bana bir yer versinler de orada oturayim" dedi, "Çünkü ben kulunun seninle birlikte kral kentinde yasamasina gerek yok."
\par 6 Akis o gün ona Ziklak Kenti'ni verdi. Bundan ötürü Ziklak bugün de Yahuda krallarina aittir.
\par 7 Davut Filist topraklarinda bir yil dört ay yasadi.
\par 8 Bu süre içinde Davut'la adamlari gidip Gesurlular'a, Girizliler'e ve Amalekliler'e baskinlar yaptilar. Bunlar uzun zamandan beri Sur'a, hatta Misir'a dek uzanan topraklarda yasiyorlardi.
\par 9 Davut bir bölgeye saldirdiginda kadin erkek demez, kimseyi sag birakmazdi; yalniz davarlari, sigirlari, esekleri, develeri ve giysileri alip Akis'e dönerdi.
\par 10 Akis, "Bugün nerelere baskin düzenlediniz?" diye sorardi. Davut da, "Yahuda'nin güneyine, Yerahmeelliler'in ve Kenliler'in güney bölgesine saldirdik" derdi.
\par 11 Davut, kendisiyle Gat'a kimseyi götürmemek için kadin erkek kimseyi sag birakmazdi. Çünkü, "Gat'a gidip, 'Davut söyle yapti, böyle yapti diyerek bize karsi bilgi aktarmasinlar" diye düsünürdü. Davut, Filist topraklarinda yasadigi sürece bu yöntemi uyguladi.
\par 12 Akis Davut'a güven duymaya basladi. "Davut kendi halki olan Israilliler'in nefretine ugradi. Bundan böyle benim hizmetimde kalacak" diye düsünüyordu.

\chapter{28}

\par 1 O sirada Filistliler Israil'le savasmak için askeri birliklerini topladilar. Akis Davut'a, "Adamlarinla birlikte benim yanimda savasacagini bilmelisin" dedi.
\par 2 Davut, "O zaman sen de kulunun neler yapabilecegini göreceksin!" diye karsilik verdi. Akis, "Iyi!" dedi, "Yasadigin sürece seni kendime koruma görevlisi atayacagim."
\par 3 Samuel ölmüs, bütün Israil halki onun için yas tutmustu. Onu kendi kenti Rama'da gömmüslerdi. Saul da cincilerle ruhlara danisanlari ülkeden kovmustu.
\par 4 Filistliler toplanip Sunem'e gittiler ve orada ordugah kurdular. Saul da bütün Israilliler'i toplayip Gilboa Dagi'nda ordugah kurdu.
\par 5 Saul Filist ordusunu görünce korkup büyük dehsete kapildi.
\par 6 RAB'be danistiysa da, RAB ona ne düslerle, ne Urim*, ne de peygamberler araciligiyla yanit verdi.
\par 7 Bunun üzerine Saul görevlilerine, "Bana bir cinci kadin bulun da varip ona danisayim" diye buyruk verdi. Görevliler, "Eyn-Dor'da bir cinci kadin var" dediler.
\par 8 Böylece Saul baska giysilere bürünüp kiligini degistirdi. Geceleyin yanina iki kisi alip kadinin yasadigi yere gitti. Kadina, "Lütfen benim için ruhlara danis ve sana söyleyecegim kisiyi çagir" dedi.
\par 9 Ama kadin ona su karsiligi verdi: "Saul'un neler yaptigini, cincilerle ruhlara danisanlari ülkeden kovdugunu biliyorsun. Öyleyse neden beni öldürmek için tuzak kuruyorsun?"
\par 10 Saul, "Yasayan RAB'bin adiyla derim ki, bundan sana bir kötülük gelmeyecek" diye ant içti.
\par 11 Bunun üzerine kadin, "Sana kimi çagirayim?" diye sordu. Saul, "Bana Samuel'i çagir" dedi.
\par 12 Kadin, Samuel'i görünce çiglik atarak, "Sen Saul'sun! Neden beni kandirdin?" dedi.
\par 13 Kral ona, "Korkma!" dedi, "Ne görüyorsun?" Kadin, "Yerin altindan çikan bir ilah görüyorum" diye karsilik verdi.
\par 14 Saul, "Neye benziyor?" diye sordu. Kadin, "Cüppe giymis yasli bir adam yukariya çikiyor" dedi. O zaman Saul onun Samuel oldugunu anladi; egilip yüzüstü yere kapandi.
\par 15 Samuel Saul'a, "Neden beni çagirtip rahatsiz ettin?" dedi. Saul, "Büyük sikinti içindeyim" diye yanitladi, "Filistliler bana karsi savasiyor ve Tanri da beni terk etti. Artik bana ne peygamberler araciligiyla, ne de düslerle yanit veriyor. Bu yüzden, ne yapmam gerektigini bana bildirmen için seni çagirttim."
\par 16 Samuel, "RAB seni terk edip sana düsman olduguna göre, neden bana danisiyorsun?" dedi,
\par 17 "RAB benim araciligimla söyledigini yapti, kralligi senden alip soydasin Davut'a verdi.
\par 18 Çünkü sen RAB'bin buyruguna uymadin, O'nun alevlenen öfkesini Amalekliler'e uygulamadin. RAB bugün bunlari bu yüzden basina getirdi.
\par 19 RAB seni de, Israil halkini da Filistliler'in eline teslim edecek. Yarin sen ve ogullarin bana katilacaksiniz. RAB Israil ordusunu da Filistliler'in eline teslim edecek."
\par 20 Saul birden boylu boyunca yere düstü. Samuel'in sözlerinden ötürü büyük korkuya kapildi. Gücü de kalmamisti; çünkü bütün gün, bütün gece yemek yememisti.
\par 21 Kadin Saul'a yaklasti. Onun büyük saskinlik içinde oldugunu görünce, "Bak, kölen sözünü dinledi" dedi, "Canimi tehlikeye atarak benden istedigini yaptim.
\par 22 Simdi lütfen kölenin söyleyecegini dinle. Izin ver de, önüne biraz yemek koyayim. Yoluna devam edecek gücün olmasi için yemek yemelisin."
\par 23 Ama Saul, "Yemem" diyerek reddetti. Ancak hizmetkârlariyla kadin zorlayinca, onlarin dedigini yapti. Yerden kalkip yatagin üzerine oturdu.
\par 24 Kadinin evinde besili bir dana vardi. Kadin onu hemen kesti. Un alip yogurdu ve mayasiz ekmek pisirdi.
\par 25 Sonra Saul'la görevlilerinin önüne koydu. Onlar da yediler. Sonra o gece kalkip gittiler.

\chapter{29}

\par 1 Filistliler bütün ordularini Afek'te topladilar. Israilliler ise Yizreel'deki pinarin yanina kurduklari ordugahta kaliyorlardi.
\par 2 Filist beyleri yüzer ve biner kisilik birliklerle ilerliyordu. Davut'la adamlariysa Akis'le birlikte geriden geliyorlardi.
\par 3 Filistli komutanlar, "Bu Ibraniler'in burada ne isi var?" diye sorunca, Akis su karsiligi verdi: "Bu, Israil Krali Saul'un görevlisi Davut'tur. Bir yildan uzun süredir yanimda kaliyor. Bana geldiginden beri kendisinde hiçbir kötülük bulamadim."
\par 4 Ama Filistli komutanlar Akis'e öfkelendiler. "Adami geri gönder, kendisine verdigin yere dönsün" dediler, "Bizimle birlikte savasa gelmesin; yoksa savas sirasinda bize karsi çikar. Efendisinin begenisini nasil kazanabilir? Adamlarimizin basini ona vermekten daha iyi bir yol bulabilir mi?
\par 5 Çalip oynarken, 'Saul binlercesini öldürdü, Davut'sa on binlercesini diye hakkinda ezgiler okuduklari Davut degil mi bu?"
\par 6 Bunun üzerine Akis, Davut'u çagirip, "Yasayan RAB'bin adiyla derim ki, sen dürüst bir kisisin" dedi, "Benimle birlikte savasa katilmani isterdim. Yanima geldigin günden bu yana ters bir davranisini görmedim. Ama Filist beyleri seni uygun görmedi.
\par 7 Simdi geri dön ve esenlikle git. Filist beylerinin gözünde ters bir davranista bulunma."
\par 8 Davut, "Ama ben ne yaptim?" diye sordu, "Yanina geldigimden bu yana bende ne buldun ki, gidip efendim kralin düsmanlarina karsi savasmayayim?"
\par 9 Akis, "Biliyorum, sen benim gözümde Tanri'nin bir melegi gibi iyisin" diye yanitladi, "Ne var ki Filistli komutanlar, 'Bizimle savasa gelmesin diyorlar.
\par 10 Seninle gelmis olan efendin Saul'un kullariyla birlikte sabah erkenden kalkin ve tan agarir agarmaz gidin."
\par 11 Böylece Davut'la adamlari Filist ülkesine dönmek üzere sabah erkenden kalktilar. Filistliler ise Yizreel'e gittiler.

\chapter{30}

\par 1 Davut'la adamlari üçüncü gün Ziklak Kenti'ne vardilar. Bu arada Amalekliler Negev bölgesiyle Ziklak'a baskin yapmis, Ziklak Kenti'ni yakip yikmislardi.
\par 2 Kimseyi öldürmemislerdi, ama kadinlarla orada yasayan genç, yasli herkesi tutsak etmislerdi. Sonra onlari da yanlarina alip yollarina gitmislerdi.
\par 3 Davut'la adamlari oraya varinca kentin atese verildigini, karilarinin, ogullarinin, kizlarinin tutsak alindigini anladilar.
\par 4 Güçleri tükeninceye dek hiçkira hiçkira agladilar.
\par 5 Davut'un iki karisi, Yizreelli Ahinoam ile Karmelli Naval'in dulu Avigayil de tutsak edilmisti.
\par 6 Davut büyük sikinti içindeydi. Çünkü herkes ogullari, kizlari için aci çekiyor ve, "Davut'u taslayalim" diyordu. Ama Davut, Tanrisi RAB'de güç bularak,
\par 7 Ahimelek oglu Kâhin Aviyatar'a, "Bana efodu* getir" dedi. Aviyatar efodu getirdi.
\par 8 Davut RAB'be danisarak, "Bu akincilarin ardina düsersem, onlara yetisir miyim?" diye sordu. RAB, "Artlarina düs, kesinlikle onlara yetisip tutsaklari kurtaracaksin" diye yanitladi.
\par 9 Bunun üzerine Davut yanindaki alti yüz kisiyle yola çikti. Besor Vadisi'ne geldiler. Vadiyi geçemeyecek kadar bitkin düsen iki yüz kisi orada kaldi. Davut dört yüz kisiyle akincilari kovalamayi sürdürdü.
\par 11 Kirda bir Misirli bulup Davut'a getirdiler. Yiyip içmesi için ona yiyecek, içecek verdiler.
\par 12 Bir parça incir pestili ile iki salkim kuru üzüm de verdiler. Adam yiyince canlandi. Üç gün üç gecedir yiyip içmemisti.
\par 13 Davut ona, "Kime baglisin? Nerelisin?" diye sordu. Genç adam, "Misirli'yim, bir Amalekli'nin kölesiyim" diye yanitladi, "Üç gün önce hastalaninca, efendim beni birakti.
\par 14 Keretliler'in güney sinirlarina, Yahuda topraklarina, Kalev'in güneyine baskinlar düzenlemis, Ziklak Kenti'ni de atese vermistik."
\par 15 Davut, "Beni bu akincilara götürebilir misin?" diye sordu. Misirli genç, "Beni öldürmeyecegine ya da efendimin eline teslim etmeyecegine dair Tanri'nin önünde ant içersen, seni akincilarin oldugu yere götürürüm" diye karsilik verdi.
\par 16 Böylece Misirli Davut'u götürdü. Akincilar dört bir yana dagilmislardi. Filist ve Yahuda topraklarindan topladiklari büyük yagmadan yiyip içiyor, eglenip oynuyorlardi.
\par 17 Davut ertesi gün tan vaktinden aksama dek onlari öldürdü. Develere binip kaçan dört yüz genç disinda içlerinden kurtulan olmadi.
\par 18 Davut Amalekliler'in ele geçirdigi her seyi, bu arada da iki karisini kurtardi.
\par 19 Gençler, yaslilar, ogullar, kizlar, yagmalanan mallar, kisacasi Amalekliler'in aldiklarindan hiçbir sey eksik kalmadi. Davut tümünü geri aldi.
\par 20 Bütün koyunlarla sigirlari da aldi. Adamlari, bunlari öbür hayvanlarin önünden sürerek, "Bunlar Davut'un yagmaladiklari" diyorlardi.
\par 21 Bundan sonra Davut, daha ileriye gidemeyecek kadar bitkin düsüp Besor Vadisi'nde kalan iki yüz kisinin bulundugu yere vardi. Onlar da Davut'la yanindakileri karsilamaya çiktilar. Davut yaklasinca onlara esenlik diledi.
\par 22 Ama Davut'la giden adamlardan kötü ve degersiz olanlarin tümü, "Madem bizimle birlikte gitmediler, geri aldigimiz yagmadan onlara hiçbir pay vermeyecegiz" dediler, "Her biri yalniz karisiyla çocuklarini alip gitsin."
\par 23 Ama Davut, "Hayir, kardeslerim!" dedi, "RAB'bin bize verdikleri konusunda böyle davranamayiz! O bizi korudu ve bize saldiran akincilari elimize teslim etti.
\par 24 Sizin bu söylediklerinizi kim kabul eder? Savasa gidenle esyanin yaninda kalanin payi aynidir. Her sey esit paylasilacak!"
\par 25 O günden sonra Davut bunu Israil için bugüne dek geçerli bir kural ve ilke haline getirdi.
\par 26 Davut Ziklak'a dönünce, dostlari olan Yahuda ileri gelenlerine yagma mallardan göndererek, "Iste RAB'bin düsmanlarindan yagmalanan mallardan size bir armagan" dedi.
\par 27 Sonra Beytel, Negev'deki Ramot, Yattir,
\par 28 Aroer, Sifmot, Estemoa,
\par 29 Rakal, Yerahmeelliler'in, Kenliler'in kentlerinde,
\par 30 Horma, Bor-Asan, Atak,
\par 31 Hevron'da oturanlara ve adamlariyla birlikte sik sik ugradigi yerlerin tümüne yagmalanan mallardan gönderdi.

\chapter{31}

\par 1 Filistliler Israilliler'le savasa tutustu. Israilliler Filistliler'in önünden kaçti. Birçogu Gilboa Dagi'nda ölüp yere serildi.
\par 2 Filistliler Saul'la ogullarinin ardina düstüler. Saul'un ogullari Yonatan'i, Avinadav'i ve Malkisua'yi yakalayip öldürdüler.
\par 3 Saul'un çevresinde savas kizisti. Derken Saul Filistli okçular tarafindan vuruldu ve agir yaralandi.
\par 4 Saul, silahini tasiyan adama, "Kilicini çek de bana sapla" dedi, "Yoksa bu sünnetsizler* gelip bana kiliç saplayacak ve benimle alay edecekler." Ama silah tasiyicisi büyük bir korkuya kapilarak bunu yapmak istemedi. Bunun üzerine Saul kilicini çekip kendini üzerine atti.
\par 5 Saul'un öldügünü görünce, silah tasiyicisi da kendini kilicinin üzerine atti ve Saul'la birlikte öldü.
\par 6 Böylece Saul, üç oglu, silah tasiyicisi ve bütün adamlari ayni gün öldüler.
\par 7 Vadinin öbür tarafinda ve Seria Irmagi'nin karsi yakasinda oturan Israilliler, Israil ordusunun kaçtigini, Saul'la ogullarinin öldügünü anlayinca, kentlerini terk edip kaçmaya basladilar. Filistliler gelip bu kentlere yerlestiler.
\par 8 Ertesi gün Filistliler, öldürülenleri soymak için geldiklerinde, Saul'la üç oglunun Gilboa Dagi'nda öldügünü gördüler.
\par 9 Saul'un basini kesip silahlarini aldilar. Sonra bu iyi haberin putlarinin tapinaginda ve halk arasinda duyurulmasi için Filist ülkesinin her yanina ulaklar gönderdiler.
\par 10 Saul'un silahlarini Astoret'in* tapinagina koyup cesedini Beytsean Kenti'nin suruna çaktilar.
\par 11 Yaves-Gilat halki Filistliler'in Saul'a yaptiklarini duydu.
\par 12 Bütün yigitler geceleyin yola koyularak Beytsean'a gittiler. Saul'la ogullarinin cesetlerini Beytsean surundan indirip Yaves'e götürdüler, orada yaktilar.
\par 13 Sonra kemiklerini toplayip Yaves'teki ilgin agacinin altina gömdüler ve yedi gün oruç* tuttular.


\end{document}