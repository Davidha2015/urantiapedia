\begin{document}

\title{1 Krallar}


\chapter{1}

\par 1 Yillar geçmis, Kral Davut yaslanmisti. Üstünü örtülerle örtmelerine karsin isinamiyordu.
\par 2 Görevlileri, "Efendimiz kral!" dediler, "Yaninda kalip sana bakacak, koynunda yatip seni isitacak genç bir kiz arayalim."
\par 3 Görevliler bütün Israil'i aradilar; sonunda Sunemli Avisak adinda genç ve güzel bir kiz bulup krala getirdiler.
\par 4 Çok güzel olan genç kiz, krala bakip hizmet etti. Ama kral ona hiç dokunmadi.
\par 5 Hagit'in oglu Adoniya kral olmayi düsünüyordu. Bu amaçla ortaya çikip kendine savas arabalari, atlilar ve önünde kosacak elli muhafiz buldu.
\par 6 Babasi Davut hiçbir zaman, "Neden söyle ya da böyle davraniyorsun?" diye ona karsi çikmamisti. Avsalom'dan sonra dünyaya gelen Adoniya çok yakisikliydi.
\par 7 Adoniya, Seruya oglu Yoav ve Kâhin* Aviyatar'la görüsüp onlarin destegini sagladi.
\par 8 Ama Kâhin Sadok, Yehoyada oglu Benaya, Peygamber Natan, Simi, Rei ve Davut'un muhafizlari ona katilmadilar.
\par 9 Adoniya, Eyn-Rogel yakinlarinda Sohelet Kayasi denilen yerde davar, sigir ve besili buzagilar kurban edip bütün kardeslerini, yani kralin ogullarini ve krala hizmet eden bütün Yahudalilar'i çagirdi. Ama Peygamber Natan'i, Benaya'yi, muhafizlari ve kardesi Süleyman'i çagirmadi.
\par 11 Bunun üzerine Natan, Süleyman'in annesi Bat-Seva'ya, "Hagit oglu Adoniya efendimiz Davut'un haberi olmadan kendini kral ilan etmis, duymadin mi?" dedi,
\par 12 "Simdi izin ver de sana kendi caninla oglun Süleyman'in canini nasil kurtaracagina iliskin ögüt vereyim.
\par 13 Git Kral Davut'a söyle, 'Efendim kral, benden sonra oglun Süleyman kral olacak ve tahtima o oturacak diye bana ant içmedin mi? O halde neden Adoniya kral oldu?
\par 14 Sen kralla konusmani bitirmeden ben içeri girip sözlerini dogrulayacagim."
\par 15 Bat-Seva yasi çok ilerlemis olan kralin odasina girdiginde Sunemli Avisak ona hizmet ediyordu.
\par 16 Bat-Seva, kralin önünde diz çöküp yere kapandi. Kral, "Ne istiyorsun?" diye sordu.
\par 17 Bat-Seva söyle karsilik verdi: "Efendim, 'Benden sonra oglun Süleyman kral olacak ve tahtima oturacak diye bana Tanrin RAB adiyla ant içtin.
\par 18 Efendim kral, su anda senin haberin olmadan Adoniya kralligini ilan etti.
\par 19 Sayisiz sigir, davar ve besili buzagi kurban edip kralin bütün ogullarini, Kâhin Aviyatar'i ve ordu komutani Yoav'i çagirdi, ama kulun Süleyman'i çagirmadi.
\par 20 Efendim kral, bütün Israil'in gözü sende. Senden sonra tahtina kimin geçecegini ögrenmek istiyorlar.
\par 21 Yoksa sen ölüp atalarina kavusunca, ben ve oglum Süleyman suçlu sayilacagiz."
\par 22 Bat-Seva daha kralla konusurken Peygamber Natan geldi.
\par 23 Krala, "Peygamber Natan geldi" dediler. Natan kralin huzuruna çikip yüzüstü yere kapandi.
\par 24 Sonra, "Efendim kral, senden sonra Adoniya'nin kral olup tahtina geçecegini söyledin mi?" dedi,
\par 25 "Adoniya bugün gidip sayisiz sigir, davar ve besili buzagi kurban etmis; kralin bütün ogullarini, ordu komutanlarini ve Kâhin Aviyatar'i çagirmis. Su anda hepsi onun önünde yiyip içiyor ve, 'Yasasin Kral Adoniya! diye bagiriyor.
\par 26 Ama Adoniya beni, Kâhin Sadok'u, Yehoyada oglu Benaya'yi ve kulun Süleyman'i çagirmadi.
\par 27 Efendim ve kralim, gerçekten bütün bunlara sen mi karar verdin? Yoksa senden sonra tahtina kimin geçecegini biz kullarina bildirmeden mi bunu yaptin?"
\par 28 Kral Davut, "Bana Bat-Seva'yi çagirin!" dedi. Bat-Seva kralin huzuruna çikip önünde durdu.
\par 29 Kral Bat-Seva'ya, "Beni bütün sikintilardan kurtaran, yasayan RAB'bin adiyla ant içiyorum" dedi,
\par 30 "'Benden sonra oglun Süleyman kral olacak ve tahtima geçecek diye Israil'in Tanrisi RAB'bin adiyla içtigim andi bugün yerine getirecegim."
\par 31 O zaman Bat-Seva kralin önünde diz çöküp yüzüstü yere kapandi ve, "Efendim kral Davut sonsuza dek yasasin!" dedi.
\par 32 Kral Davut, "Kâhin Sadok'u, Peygamber Natan'i ve Yehoyada oglu Benaya'yi bana çagirin" dedi. Hepsi önüne gelince,
\par 33 kral onlara söyle dedi: "Efendinizin görevlilerini yaniniza alin ve oglum Süleyman'i benim katirima bindirip Gihon'a götürün.
\par 34 Orada Kâhin Sadok ve Peygamber Natan onu Israil Krali olarak meshetsinler*. Boru çalip, 'Yasasin Kral Süleyman! Diye bagirin.
\par 35 Onun ardindan gidin; çünkü o gelip tahtima oturacak ve yerime kral olacak. Onu Israil ve Yahuda'ya yönetici atadim."
\par 36 Yehoyada oglu Benaya, "Amin" diye karsilik verdi, "Efendim kralin Tanrisi RAB de bu karari onaylasin.
\par 37 RAB, efendim kralla birlikte oldugu gibi Süleyman'la da birlikte olsun ve onun kralligini Davut'un kralligindan daha basarili kilsin."
\par 38 Kâhin Sadok, Peygamber Natan, Yehoyada oglu Benaya, Keretliler'le Peletliler* gidip Süleyman'i Kral Davut'un katirina bindirdiler ve Gihon'a kadar ona eslik ettiler.
\par 39 Kâhin Sadok, Kutsal Çadir'dan yag dolu boynuz kabi alip Süleyman'i meshetti. Boru çalinca bütün halk "Yasasin Kral Süleyman!" diye bagirdi.
\par 40 Herkes kaval çalarak Süleyman'in ardindan yürüdü. Öyle sevinçliydiler ki, seslerinden adeta yer sarsiliyordu.
\par 41 Adoniya ve yanindaki konuklar yemeklerini bitirirken kalabaligin gürültüsünü duydular. Boru sesini duyan Yoav, "Kentten gelen bu gürültü de ne?" diye sordu.
\par 42 Yoav daha sorusunu tamamlamadan, Kâhin Aviyatar oglu Yonatan çikageldi. Adoniya ona, "Içeri gir, sen yigit bir adamsin. Iyi haberler getirmis olmalisin" dedi.
\par 43 Yonatan, "Hayir, bu kez farkli" diye karsilik verdi, "Efendimiz Kral Davut, Süleyman'i kral atadi.
\par 44 Kral, Kâhin Sadok, Peygamber Natan, Yehoyada oglu Benaya ve Keretliler'le Peletliler'in Süleyman'i kendi katirina bindirip götürmelerini istedi.
\par 45 Gihon'a götürülen Süleyman orada Kâhin Sadok'la Peygamber Natan tarafindan kral olarak meshedildi. Oradan sevinçle döndüler ve sesleri kentte yankilanmaya basladi. Duydugunuz sesler onlarin sesleridir.
\par 46 Üstelik Süleyman krallik tahtina oturdu bile.
\par 47 Ayrica efendimiz Kral Davut'u kutlamaya gelen görevlileri, 'Tanrin, Süleyman'in adini senin adindan daha yüce, kralligini senin kralligindan daha basarili kilsin diyorlar. Kral yataginin üzerine kapanarak,
\par 48 'Henüz gözlerim açikken bugün tahtima oturacak birini veren Israil'in Tanrisi RAB'be övgüler olsun diyor."
\par 49 Bunun üzerine Adoniya'nin bütün konuklari korkuya kapilip kalktilar, her biri kendi yoluna gitti.
\par 50 Adoniya ise, Süleyman'dan korktugu için, gidip sunagin boynuzlarina sarildi.
\par 51 Durumu Süleyman'a anlattilar. "Adoniya senden korkuyor" dediler, "Sunagin boynuzlarina sarilmis ve 'Ilk önce Kral Süleyman ben kulunu kiliçla öldürmeyecegine dair ant içsin diyor."
\par 52 Süleyman, "Eger bana bagli kalirsa, saçinin bir teline bile zarar gelmez" diye yanitladi, "Ama içinde bir kötülük varsa öldürülür."
\par 53 Sonra Kral Süleyman'in gönderdigi adamlar Adoniya'yi sunaktan indirip getirdiler. Adoniya gelip önünde yere kapaninca, ona, "Evine dön!" dedi.

\chapter{2}

\par 1 Davut'un ölümü yaklasinca, oglu Süleyman'a sunlari söyledi:
\par 2 "Herkes gibi ben de yakinda bu dünyadan ayrilacagim. Güçlü ve kararli ol.
\par 3 Tanrin RAB'bin verdigi görevleri yerine getir. Onun yollarinda yürü ve Musa'nin yasasinda yazildigi gibi Tanri'nin kurallarina, buyruklarina, ilkelerine ve ögütlerine uy ki, yaptigin her seyde ve gittigin her yerde basarili olasin.
\par 4 O zaman RAB bana verdigi su sözü yerine getirecektir: 'Eger soyun nasil yasadigina dikkat eder, candan ve yürekten bana bagli kalarak yollarimda yürürse, Israil tahtindan senin soyunun ardi arkasi kesilmeyecektir.
\par 5 "Seruya oglu Yoav'in bana ve Israil ordusunun iki komutani Ner oglu Avner'le Yeter oglu Amasa'ya neler yaptigini biliyorsun. Sanki savas varmis gibi onlari öldürerek baris döneminde kan döktü. Belindeki kemerle ayagindaki çariklara kan bulastirdi.
\par 6 Sen aklina uyani yap, ama onun ak saçli basinin esenlik içinde ölüler diyarina gitmesine izin verme.
\par 7 "Gilatli Barzillay'in ogullarina iyi davran, sofranda yemek yiyenlerin arasinda onlara da yer ver. Çünkü ben agabeyin Avsalom'un önünden kaçtigim zaman onlar bana yardim etmislerdi.
\par 8 "Mahanayim'e gittigim gün beni çok agir biçimde lanetleyen Benyamin oymagindan Bahurimli Gera'nin oglu Simi de yaninda. Beni Seria Irmagi kiyisinda karsilamaya geldiginde, 'Seni kiliçla öldürmeyecegim diye RAB'bin adiyla ona ant içmistim.
\par 9 Ama sen sakin onu cezasiz birakma. Ona ne yapacagini bilecek kadar akillisin. Onun ak saçli basini ölüler diyarina kanlar içinde gönder."
\par 10 Davut ölüp atalarina kavusunca, kendi adiyla bilinen kentte gömüldü.
\par 11 Yedi yil Hevron'da, otuz üç yil Yerusalim'de olmak üzere toplam kirk yil Israil'de krallik yapti.
\par 12 Babasi Davut'un tahtina geçen Süleyman'in kralligi çok saglam temellere oturmustu.
\par 13 Hagit oglu Adoniya, Süleyman'in annesi Bat-Seva'nin yanina gitti. Bat-Seva ona, "Dostça mi geldin?" diye sordu. Adoniya, "Dostça" diye karsilik verdi.
\par 14 Ve ekledi: "Sana söyleyeceklerim var." Bat-Seva, "Söyle!" dedi.
\par 15 Adoniya, "Bildigin gibi, daha önce krallik benim elimdeydi" dedi, "Bütün Israil benim kral olmami bekliyordu. Ancak her sey degisti ve krallik kardesimin eline geçti. Çünkü RAB'bin istegi buydu.
\par 16 Ama benim senden bir dilegim var. Lütfen geri çevirme." Bat-Seva, "Söyle!" dedi.
\par 17 Adoniya, "Kral Süleyman seni kirmaz" dedi, "Lütfen ona söyle, Sunemli Avisak'i bana es olarak versin."
\par 18 Bat-Seva, "Peki, senin için kralla konusacagim" diye karsilik verdi.
\par 19 Bat-Seva, Adoniya'nin dilegini iletmek üzere Kral Süleyman'in yanina gitti. Süleyman annesini karsilamak için ayaga kalkip önünde egildikten sonra tahtina oturdu. Annesi için de sag tarafina bir taht koydurdu.
\par 20 Tahtina oturan annesi, "Senden küçük bir dilegim var, lütfen beni bos çevirme" dedi. Kral, "Söyle anne, seni kirmam" diye karsilik verdi.
\par 21 Bat-Seva, "Sunemli Avisak agabeyin Adoniya'ya es olarak verilsin" dedi.
\par 22 Kral Süleyman, "Neden Sunemli Avisak'in Adoniya'ya verilmesini istiyorsun?" dedi, "Kralligi da ona vermemi iste bari! Ne de olsa o benim büyügüm. Üstelik Kâhin Aviyatar'la Seruya oglu Yoav da ondan yana."
\par 23 Bu olay üzerine Kral Süleyman RAB'bin adiyla ant içti: "Eger Adoniya bu dilegini hayatiyla ödemezse, Tanri bana aynisini, hatta daha kötüsünü yapsin!
\par 24 Beni güçlendirip babam Davut'un tahtina oturtan, verdigi sözü tutup bana bir hanedan kuran, yasayan RAB'bin adiyla ant içerim ki, Adoniya bugün öldürülecek!"
\par 25 Böylece Kral Süleyman Yehoyada oglu Benaya'yi Adoniya'yi öldürmekle görevlendirdi. Benaya da gidip Adoniya'yi öldürdü.
\par 26 Kral, Kâhin Aviyatar'a, "Anatot'taki tarlana dön" dedi, "Aslinda ölümü hak ettin. Ama seni simdi öldürmeyecegim. Çünkü sen babam Davut'un önünde Egemen RAB'bin Antlasma Sandigi'ni* tasidin ve babamin çektigi bütün sikintilari onunla paylastin."
\par 27 Eli'nin ailesi hakkinda RAB'bin Silo'da söyledigi sözün gerçeklesmesi için, Süleyman Aviyatar'i RAB'bin kâhinliginden uzaklastirdi.
\par 28 Haber Yoav'a ulasti. Yoav daha önce ayaklanan Avsalom'u desteklemedigi halde Adoniya'yi destekledi. Bu nedenle RAB'bin Çadiri'na kaçti ve sunagin boynuzlarina sarildi.
\par 29 Yoav'in RAB'bin Çadiri'na kaçip sunagin yaninda oldugu Kral Süleyman'a bildirildi. Süleyman, Yehoyada oglu Benaya'ya, "Git, onu vur!" diye buyruk verdi.
\par 30 Benaya RAB'bin Çadiri'na gitti ve Yoav'a, "Kral disari çikmani buyuruyor!" dedi. Yoav, "Hayir, burada ölmek istiyorum" karsiligini verdi. Benaya gidip Yoav'in kendisini nasil yanitladigini krala bildirdi.
\par 31 Kral, "Onun istedigi gibi yap" dedi, "Onu orada öldür ve göm. Yoav'in bos yere döktügü kanin sorumlulugunu benim ve babamin soyu üstünden kaldirmis olursun.
\par 32 RAB döktügü kandan ötürü onu cezalandiracaktir. Çünkü Yoav babam Davut'un bilgisi disinda, kendisinden daha iyi ve dogru olan iki kisiyi -Israil ordusunun komutani Ner oglu Avner'le Yahuda ordusunun komutani Yeter oglu Amasa'yi- kiliçla öldürdü.
\par 33 Böylece dökülen kanlarinin sorumlulugu sonsuza dek Yoav'in ve soyunun üstünde kalacaktir. Ama RAB, Davut'a, soyuna, ailesine ve tahtina sonsuza dek esenlik verecektir."
\par 34 Yehoyada oglu Benaya gidip Yoav'i öldürdü. Onu issiz bir bölgede bulunan kendi evine gömdüler.
\par 35 Kral, Yoav'in yerine Yehoyada oglu Benaya'yi ordu komutani yapti. Aviyatar'in yerine de Kâhin Sadok'u atadi.
\par 36 Sonra kral haber gönderip Simi'yi çagirtti. Ona, "Yerusalim'de kendine bir ev yap ve orada otur" dedi, "Hiçbir yere gitme.
\par 37 Oradan ayrilip Kidron Vadisi'nden öteye geçtigin gün bil ki öleceksin. Sorumluluk sana ait."
\par 38 Simi krala, "Efendim kral, peki" diye karsilik verdi, "Kulun olarak söylediklerini aynen yapacagim." Simi Yerusalim'de uzun süre yasadi.
\par 39 Aradan üç yil geçmisti, Simi'nin iki kölesi Gat Krali Maaka oglu Akis'in yanina kaçti. Kölelerin Gat'a kaçtigini Simi'ye haber verdiler.
\par 40 Bunun üzerine Simi kalkip esegine palan vurdu ve kölelerini aramak üzere Gat'a Akis'in yanina gitti. Kölelerini bulup Gat'tan geri getirdi.
\par 41 Simi'nin Yerusalim'den Gat'a gidip döndügü Süleyman'a anlatilinca,
\par 42 Süleyman Simi'yi çagirtti. "Sana RAB'bin adiyla ant içirmedim mi?" dedi, "'Kalkip herhangi bir yere gittigin gün ölecegini bil! diye seni uyarmadim mi? Sen de bana: 'Peki, sözünü dinleyecegim demedin mi?
\par 43 Öyleyse neden RAB'bin adina içtigin anda ve buyruguma uymadin?"
\par 44 Kral, Simi'ye karsi sözlerini söyle sürdürdü: "Babam Davut'a yaptigin bütün kötülükleri çok iyi biliyorsun. Bu yaptiklarindan dolayi RAB seni cezalandiracak.
\par 45 Ama Kral Süleyman kutsanacak ve Davut'un tahti RAB'bin önünde sonsuza dek kurulu kalacaktir."
\par 46 Kral, Yehoyada oglu Benaya'ya buyruk verdi. O da gidip Simi'yi öldürdü. Böylece Süleyman'in kralligi iyice pekisti.

\chapter{3}

\par 1 Süleyman, Misir Firavunu'nun kiziyla evlendi. Böylece firavunla müttefik oldu. Esini Davut Kenti'ne götürdü. Kendi sarayi, RAB'bin Tapinagi ve Yerusalim'in çevre surlari tamamlanincaya kadar orada yasadilar.
\par 2 Halk, hâlâ çesitli tapinma yerlerinde RAB'be kurban sunuyordu. Çünkü o güne dek RAB'bin adina yapilmis bir tapinak yoktu.
\par 3 Süleyman babasi Davut'un kurallarina uyarak RAB'be olan sevgisini gösterdi. Ancak hâlâ çesitli tapinma yerlerinde kurban sunuyor, buhur yakiyordu.
\par 4 Tapinma yerlerinin en ünlüsü Givon'daydi. Kral Süleyman oraya giderek sunakta bin yakmalik sunu* sundu.
\par 5 RAB Tanri, Givon'da o gece rüyada Süleyman'a görünüp, "Sana ne vermemi istersin?" diye sordu.
\par 6 Süleyman, "Kulun babam Davut'a büyük iyilikler yaptin" diye karsilik verdi, "O sana bagli, dogru, bütün yüregiyle dürüst biri olarak yolunda yürüdü. Bugün tahtina oturacak bir ogul vermekle ona büyük bir iyilik daha yapmis oldun.
\par 7 "Ya RAB Tanrim! Ben henüz çocuk denecek bir yasta, yöneticilik nedir bilmezken bu kulunu babam Davut'un yerine kral atadin.
\par 8 Iste kulun kendi seçtigin kalabalik halkin, sayilamayacak kadar büyük bir kalabaligin ortasindadir.
\par 9 Bu yüzden bana öyle sezgi dolu bir yürek ver ki, iyi ile kötüyü ayirt edip halkini yönetebileyim. Baska türlü senin bu büyük halkini kim yönetebilir!"
\par 10 Süleyman'in bu istegi Rab'bi hosnut etti.
\par 11 Tanri ona söyle dedi: "Madem kendin için uzun ömür, zenginlik ve düsmanlarinin ölümünü istemedin, bunlarin yerine adil bir yönetim için bilgelik istedin; istegini yerine getirecegim. Sana öyle bir bilgelik ve sezgi dolu bir yürek verecegim ki, benzeri ne senden öncekilerde görülmüstür, ne de senden sonrakilerde görülecektir.
\par 13 Sana istemediklerini de verecegim: Yasadigin sürece öbür krallarin erisemeyecegi bir zenginlik ve onura ulasacaksin.
\par 14 Eger sen de baban Davut gibi kurallarima ve buyruklarima uyup yollarimda yürürsen, sana uzun ömür de verecegim."
\par 15 Süleyman uyaninca bunun bir rüya oldugunu anladi. Sonra Yerusalim'e gitti, Rab'bin Antlasma Sandigi'nin* önünde durup yakmalik sunular ve esenlik sunulari* sundu. Ayrica bütün görevlilerine de bir sölen verdi.
\par 16 Bir gün iki fahise gelip kralin önünde durdu.
\par 17 Kadinlardan biri krala söyle dedi: "Efendim, bu kadinla ben ayni evde kaliyoruz. Birlikte kaldigimiz sirada ben bir çocuk dogurdum.
\par 18 Iki gün sonra da o dogurdu. Evde yalnizdik, ikimizden baska kimse yoktu.
\par 19 Bu kadin geceleyin çocugunun üzerine yattigi için çocuk ölmüs.
\par 20 Gece yarisi, ben kulun uyurken, kalkip çocugumu almis, koynuna yatirmis, kendi ölü çocugunu da benim koynuma koymus.
\par 21 Sabahleyin oglumu emzirmek için kalktigimda, onu ölmüs buldum. Ama sabah aydinliginda dikkatle bakinca, onun benim dogurdugum çocuk olmadigini anladim."
\par 22 Öbür kadin, "Hayir! Yasayan çocuk benim, ölü olan senin!" diye çikisti. Birinci kadin, "Hayir! Ölen çocuk senin, yasayan çocuk benim!" diye diretti. Kralin önünde böyle tartisip durdular.
\par 23 Kral, "Biri, 'Yasayan çocuk benim, ölü olan senin diyor, öbürü, 'Hayir! Ölen çocuk senin, yasayan benim diyor.
\par 24 O halde bana bir kiliç getirin!" dedi. Kiliç getirilince,
\par 25 kral, "Yasayan çocugu ikiye bölüp yarisini birine, yarisini öbürüne verin!" diye buyurdu.
\par 26 Yüregi oglunun acisiyla sizlayan, çocugun gerçek annesi krala, "Aman efendim, sakin çocugu öldürmeyin! Ona verin!" dedi. Öbür kadinsa, "Çocuk ne benim, ne de senin olsun, onu ikiye bölsünler!" dedi.
\par 27 O zaman kral kararini verdi: "Sakin çocugu öldürmeyin! Birinci kadina verin, çünkü gerçek annesi odur."
\par 28 Kralin verdigi bu karari duyan bütün Israilliler hayranlik içinde kaldi. Herkes adil bir yönetim için Süleyman'in Tanri'dan gelen bilgelige sahip oldugunu anladi.

\chapter{4}

\par 1 Süleyman bütün Israil'in kraliydi.
\par 2 Görevlileri ise sunlardi: Kâhin: Sadok oglu Azarya.
\par 3 Yazmanlar: Sisa'nin ogullari Elihoref ve Ahiya. Devlet tarihçisi: Ahilut oglu Yehosafat.
\par 4 Ordu komutani: Yehoyada oglu Benaya. Kâhinler: Sadok ve Aviyatar.
\par 5 Bas vali: Natan oglu Azarya. Kralin özel danismani: Natan oglu Kâhin Zavut.
\par 6 Saray sorumlusu: Ahisar. Angaryacilarin basi: Avda oglu Adoniram.
\par 7 Süleyman'in Israil'de on iki bölge valisi vardi. Bunlar kralin ve sarayin yiyecek içecek gereksinimini karsilardi. Her vali yilda bir ay bu gereksinimleri karsilamakla yükümlüydü.
\par 8 Bu valiler sunlardi: Efrayim'in daglik bölgesinde Ben-Hur;
\par 9 Makaz, Saalvim, Beytsemes ve Elon-Beythanan bölgelerinde Ben-Deker;
\par 10 Arubbot, Soko ve bütün Hefer bölgesinde Ben-Heset;
\par 11 Nafat-Dor bölgesinde Süleyman'in kizi Tafat'la evli olan Ben-Avinadav;
\par 12 Taanak, Megiddo, Yizreel'in altinda Saretan'in yanindaki bütün Beytsean ve Beytsean'dan Avel-Mehola ve Yokmoam'in ötelerine kadar uzanan bölgede Ahilut oglu Baana;
\par 13 Ramot-Gilat, Gilat'ta Manasse oglu Yair'in yerlesim birimleri ve Basan'daki Argov yöresinde surlar ve tunç* sürgülerle güçlendirilmis altmis büyük kentin basinda Ben-Gever;
\par 14 Mahanayim bölgesinde Iddo oglu Ahinadav;
\par 15 Naftali bölgesinde Süleyman'in kizi Basemat'la evlenen Ahimaas;
\par 16 Aser ve Bealot bölgelerinde Husay oglu Baana;
\par 17 Issakar bölgesinde Paruah oglu Yehosafat;
\par 18 Benyamin bölgesinde Ela oglu Simi;
\par 19 Gilat bölgesinde, yani Amorlular'in Krali Sihon'la Basan Krali Og'un eski topraklarinda Uri oglu Gever. Ayrica Yahuda bölgesinin tek valisi vardi.
\par 20 Yahuda ve Israil halki kiyilarin kumu kadar kalabalikti. Herkes yiyip içip sevinç içinde yasiyordu.
\par 21 Süleyman, Firat Irmagi'ndan Filist'e, oradan Misir sinirina kadar bütün ülkelere egemendi. Bu ülkeler Süleyman'in yasami boyunca ona haraç ödeyip hizmet ettiler.
\par 22 Süleyman'in sarayinin bir günlük yiyecek gereksinimi sunlardi: Otuz kor ince, altmis kor kepekli un;
\par 23 onu ahirda, yirmisi çayirda yetistirilmis sigir ve yüz koyun; ayrica geyikler, ceylanlar, karacalar ve semiz kuslar.
\par 24 Tifsah'tan Gazze'ye kadar, Firat Irmagi'nin batisindaki bütün kralliklari Süleyman yönetiyordu. Her tarafta baris vardi.
\par 25 Dan'dan Beer-Seva'ya kadar Yahuda ve Israil halkinin her bireyi Süleyman'in yasami boyunca kendi asmasi ve incir agaci altinda güvenlik içinde yasadi.
\par 26 Süleyman'in savas arabalarinin atlari için kirk bin ahiri ve on iki bin atlisi vardi.
\par 27 Bölge valilerinin her biri kendine düsen bir ay boyunca, Kral Süleyman'a ve sofrasina oturan herkese yiyecek saglar, hiçbir seyi eksik etmezdi.
\par 28 Her vali kendisine verilen buyruk uyarinca, savas arabalarinin atlariyla öbür atlar için belirli bir yere arpa ve saman getirirdi.
\par 29 Tanri, Süleyman'a bilgelik, derin bir sezgi, kiyilardaki kum kadar anlayis verdi.
\par 30 Süleyman'in bilgeligi, bütün dogulularin ve Misirlilar'in bilgeliginden daha üstündü.
\par 31 O, Ezrahli Etan, Mahol'un ogullari Heman, Kalkol ve Darda dahil herkesten daha bilgeydi. Ünü çevredeki bütün uluslara yayilmisti.
\par 32 Üç bin özdeyisi ve bin bes ezgisi vardi.
\par 33 Lübnan sedir agacindan duvarlarda biten mercanköskotuna kadar bütün agaçlardan söz ettigi gibi, hayvanlar, kuslar, sürüngenler* ve baliklardan da söz edebiliyordu.
\par 34 Süleyman'in bilgeligini duyan dünyanin bütün krallari ona adamlarini gönderirdi. Bütün uluslardan insanlar gelir, Süleyman'in bilgece sözlerini dinlerdi.

\chapter{5}

\par 1 Sur Krali Hiram, Süleyman'in babasi Davut'un yerine kral olarak meshedildigini* duyunca, elçilerini Süleyman'a gönderdi. Çünkü Davut'la hep dostça geçinmisti.
\par 2 Süleyman Hiram'a su haberi gönderdi:
\par 3 "Bildigin gibi, babam Davut çevresindeki savaslar yüzünden Tanrisi RAB'bin adina bir tapinak yapamadi. Bu savaslarda RAB, Davut'un düsmanlarini onun ayaklari altina serdi.
\par 4 Oysa simdi Tanrim RAB her yönden bana rahatlik verdi. Ne bir düsmanim var, ne de kötü bir olay.
\par 5 RAB, babam Davut'a, 'Tahtina oturtacagim oglun benim adima bir tapinak yapacak diye söz verdi. Ben de Tanrim RAB'bin adina bir tapinak yapmaya karar verdim.
\par 6 "Simdi bana Lübnan'dan sedir agaçlari kesmeleri için adamlarina buyruk ver. Benim adamlarim da seninkilerle birlikte çalissin. Adamlarin için istedigin ücreti verecegim. Aramizda Saydalilar kadar agaç kesmede usta adamlar olmadigini biliyorsun."
\par 7 Hiram, Süleyman'dan bu haberi alinca çok sevindi ve, "Bugün, o büyük ulusu yönetmek üzere Davut'a bilge bir ogul veren RAB'be övgüler olsun!" dedi.
\par 8 Sonra Hiram Süleyman'a su haberi gönderdi: "Gönderdigin haberi aldim. Sedir ve çam agaçlariyla ilgili bütün dileklerini yerine getirecegim.
\par 9 Adamlarim tomruklari Lübnan'dan denize indirecekler, ben de onlari sallar halinde baglatip belirtecegin yere kadar yüzdürecegim. Orada adamlarim onlari çözer, sen de alip götürürsün. Sarayimin yiyecek gereksinimini karsilamakla, sen de benim dilegimi yerine getirmis olursun."
\par 10 Hiram Süleyman'a istedigi kadar sedir ve çam tomrugu sagladi.
\par 11 Süleyman her yil Hiram'a sarayinin yiyecek gereksinimi olarak yirmi bin kor bugday, yirmi kor saf zeytinyagi verirdi.
\par 12 RAB, verdigi söz uyarinca, Süleyman'a bilgelik verdi. Süleyman'la Hiram arasinda baris vardi. Aralarinda bir antlasma yaptilar.
\par 13 Kral Süleyman angaryasina çalistirmak üzere bütün Israil'den otuz bin adam topladi.
\par 14 Sirayla her ay on binini Lübnan'a gönderiyordu. Bir ay Lübnan'da, iki ay evlerinde kaliyorlardi. Angaryasina çalisan adamlarin basinda Adoniram vardi.
\par 15 Süleyman'in yük tasiyan 70 000, daglarda tas kesen 80 000 adami vardi.
\par 16 Ayrica, isin yürümesini saglayan ve isçileri yöneten 3 300 görevlisi vardi.
\par 17 Isçiler, kralin buyrugu uyarinca, tapinagin temelini yontma taslarla atmak üzere ocaktan büyük ve kaliteli taslar kesip çikardilar.
\par 18 Süleyman'in ve Hiram'in yapicilariyla Gevallilar, tapinagin yapimi için taslarla keresteleri kesip hazirladilar.

\chapter{6}

\par 1 Israil halki Misir'dan çiktiktan dört yüz seksen yil sonra, Süleyman, kralliginin dördüncü yilinin ikinci ayi* olan Ziv ayinda RAB'bin Tapinagi'nin yapimina basladi.
\par 2 Kral Süleyman'in RAB için yaptigi tapinagin uzunlugu altmis, genisligi yirmi, yüksekligi otuz arsindi.
\par 3 Tapinagin ana bölümünün önündeki eyvan tapinagin genisliginde olup yirmi arsindi. Eyvan tapinagin önünden ileriye dogru on arsindi.
\par 4 Süleyman tapinakta disa dogru daralan kafesli pencereler yaptirdi.
\par 5 Tapinagin dis cephesine bitisik, ana bölümün ve iç odanin çevresindeki duvarlara bitisik, odalardan olusan katlar yaptirdi.
\par 6 Alt kat bes arsin, orta kat alti arsin, üst kat yedi arsin genisligindeydi. Kirisler tapinagin duvarlarina girmesin diye duvarlarin çevresinde disariya dogru çikintilar birakti.
\par 7 Tapinagin yapiminda kullanilan taslar tas ocaklarinda yontulmustu. Onun için yapim halindeki tapinakta çekiç ve balta dahil hiçbir demir aletin sesi duyulmadi.
\par 8 Asagi yan katin girisi tapinagin güneyindeydi. Döner bir merdivenle orta kata, oradan da üçüncü kata çikilirdi.
\par 9 Süleyman tapinagi yapip tamamladi. Üstünü sedir agacindan direklerle, kalin tahtalarla kapatti.
\par 10 Dis duvarlara bitisik katlar tapinagin çevresini kapsiyordu. Her birinin yüksekligi bes arsindi. Bunlar sedir agacindan kirislerle tapinaga eklendi.
\par 11 RAB, Süleyman'a söyle seslendi:
\par 12 "Bu tapinagi yapmaktasin. Kurallarima, ilkelerime ve bütün buyruklarima uyup onlara bagli kalirsan, baban Davut'a verdigim sözü senin araciliginla yerine getirecegim.
\par 13 Halkim Israil'in arasinda yasayip onlari hiç terk etmeyecegim."
\par 14 Süleyman tapinagi yapip bitirdi.
\par 15 Tapinagin iç duvarlarinin yüzeyini sedir agaçlariyla döseyip üstlerini tabandan tavana kadar tahtalarla kapladi. Tapinagin zeminini ise çam tahtalarla dösetti.
\par 16 En Kutsal Yer* olarak adlandirilan iç oda, tapinagin arka bölümünde yapildi. Tabandan tavana kadar sedir tahtasindan olusan bir duvarla öbür bölümlerden ayrildi. Bu bölümün uzunlugu yirmi arsindi.
\par 17 Bu iç odanin önündeki ana bölümün uzunlugu ise kirk arsindi.
\par 18 Taslar görünmesin diye tapinagin içi, üzerine sukabagi ve çiçek motifleri oyulmus sedir tahtalariyla kaplandi.
\par 19 Tapinagin içinde RAB'bin Antlasma Sandigi'nin* konacagi iç oda hazirlandi.
\par 20 Bu odanin içten içe uzunlugu, genisligi ve yüksekligi yirmiser arsindi. Süleyman odayi saf altinla kaplatti. Sunagi da sedir tahtalarla dösetti.
\par 21 Tapinagin içini saf altinla kaplatti; iç odanin önüne altin zincirler asip orayi da altinla kaplatti.
\par 22 Böylece iç odadaki sunak dahil, tapinagin içini tamamen altinla kaplatmis oldu.
\par 23 Iç odada her biri on arsin yüksekliginde, igde agacindan iki Keruv* yapti.
\par 24 Her kanadi beser arsin olan Keruv'un açik kanatlari bir uçtan öbür uca toplam on arsindi.
\par 25 Öbür Keruv'un kanat açikligi da on arsindi. Her iki Keruv'un da ölçüsü ve görünüsü ayniydi.
\par 26 Ikisinin de yüksekligi on arsindi.
\par 27 Süleyman Keruvlar'i tapinagin iç odasina yerlestirdi. Keruvlar'dan birinin açik kanadi bir duvara, ötekinin kanadi karsi duvara erisirken, öbür kanatlari da odanin ortasinda birbirine degiyordu.
\par 28 Süleyman Keruvlar'i da altinla kaplatti.
\par 29 Tapinagin iç ve dis odalarinin bütün duvarlarini kabartma Keruvlar, hurma agaçlari ve çiçek motifleriyle süsletti.
\par 30 Tapinagin hem iç, hem de dis odasinin dösemelerini altinla kaplatti.
\par 31 Iç odanin girisine igde agacindan iki kanatli söveli bir kapi yaptirdi. Söveli kapinin genisligi tapinagin genisliginin beste biriydi.
\par 32 Igde agacindan yapilan iki kapi kanadinin üstüne kabartma Keruvlar, hurma agaçlari ve çiçek motifleri oydurdu. Keruvlar ve hurma agaçlarini altinla kaplatti.
\par 33 Ayni biçimde ana bölümün girisine de igde agacindan kapi söveleri yaptirdi. Bu söveli kapi tapinagin genisliginin dörtte biriydi.
\par 34 Ayrica, çam agacindan, her biri iki kanatli, katlanabilen iki kapi yaptirdi.
\par 35 Bunlarin da üstüne Keruvlar, hurma agaçlari ve çiçek motifleri oydurdu. Oymalarin üzerini düzgün biçimde altinla kaplatti.
\par 36 Iç avlunun duvarlarini üç sira yontma tas ve bir sira sedir agaci kirisleriyle yaptirdi.
\par 37 RAB'bin Tapinagi'nin temeli dördüncü yilin Ziv ayinda atildi.
\par 38 On birinci yilin sekizinci ayi olan Bul ayinda tapinak tasarlandigi biçimde bütün ayrintilariyla tamamlandi. Tapinagin yapimi Süleyman'in yedi yilini almisti.

\chapter{7}

\par 1 Süleyman kendine, yapimi on üç yil süren bir saray yaptirdi.
\par 2 Uzunlugu yüz, genisligi elli, yüksekligi otuz arsin olan Lübnan Ormani adinda bir saray daha yaptirdi. Saray sedir kirisler yerlestirilmis dört sira halindeki sedir sütunlarin üzerine yapilmisti.
\par 3 Sütunlarin üstündeki kirk bes kirisin üstü sedir tahtalariyla kaplanmisti. Bir sira on bes kiristen olusuyordu.
\par 4 Kafesli pencereler üç sira halinde birbirine bakacak biçimde yapilmisti.
\par 5 Kapilar ve kapi söveleri dört köseliydi. Pencereler ise üç sira halinde birbirine bakacak biçimde yapilmisti.
\par 6 Süleyman elli arsin uzunlugunda otuz arsin genisliginde sütunlu bir eyvan yaptirdi. Eyvanin önünde sütunlarla desteklenmis asma tavan vardi.
\par 7 Taht Eyvani'ni, yani kararlarin verilecegi Yargi Eyvani'ni da yaptirdi. Bu eyvan da bastan asagi sedir tahtalariyla kapliydi.
\par 8 Eyvanin arkasinda öbür avludaki kendi oturacagi saray da ayni biçimde yapilmisti. Süleyman, karisi olan firavunun kizi için de bu eyvanin benzeri bir saray yaptirdi.
\par 9 Disaridan büyük avluya, temelden çatiya kadar bütün bu yapilar kaliteli taslarla yapilmisti. Taslar testereyle kesilmis, ön ve arka yüzleri yontulmus, belirli ölçülere göre hazirlanmisti.
\par 10 Temeller sekiz ve on arsin uzunlugunda büyük, seçme taslardan atilmisti.
\par 11 Üstlerinde belirli ölçülere göre kesilmis kaliteli taslar ve sedir kirisler vardi.
\par 12 Büyük avlu üç sira yontma tas ve bir sira sedir kirislerinden olusan bir duvarla çevrilmisti. RAB'bin Tapinagi'nin iç avlusuyla eyvanin duvarlari da ayni yapidaydi.
\par 13 Kral Süleyman haber gönderip Sur'dan Hiram'i getirtti.
\par 14 Hiram'in annesi Naftali oymagindan dul bir kadin, babasi ise Surlu bir tunç* isçisiydi. Hiram tunç islemede bilgili, deneyimli, usta biriydi. Gelip Kral Süleyman'in bütün islerini yapti.
\par 15 Hiram her birinin yüksekligi on sekiz arsin ve çevresi on iki arsin olan iki tunç sütun döktü.
\par 16 Sütunlarin üzerine koymak için beser arsin yüksekliginde dökme tunçtan iki sütun basligi yapti.
\par 17 Sütun basliklarinin her biri agla kaplanmisti. Agin üzeri yedi sira örgülü zincirle ve iki sira nar motifiyle bezenmisti.
\par 19 Eyvanda bulunan dört arsin yüksekligindeki sütun basliklari da nilüfer biçimindeydi.
\par 20 Her iki sütun basliginda, örgülü aga yakin çikintinin yukarisinda çepeçevre diziler halinde iki yüz nar motifi vardi.
\par 21 Hiram sütunlari tapinagin eyvanina dikip sagdakine Yakin, soldakine Boaz adini verdi.
\par 22 Sütun basliklari nilüfer biçimindeydi. Böylece sütunlarin isi tamamlanmis oldu.
\par 23 Hiram dökme tunçtan on arsin çapinda, bes arsin derinliginde, çevresi otuz arsin yuvarlak bir havuz yapti.
\par 24 Havuz, kenarlarinin altindaki iki sira sukabagi motifiyle birlikte dökülmüstü. Her arsinda onar tane olan bu motifler havuzu çepeçevre kusatiyordu.
\par 25 Havuz üçü kuzeye, üçü batiya, üçü güneye, üçü de doguya bakan on iki boga heykeli üzerine oturtulmustu. Bogalarin sagrilari içe dönüktü.
\par 26 Havuzun çeperi dört parmak kalinligindaydi; kenarlari kâse kenarlarini, nilüferleri andiriyordu. Iki bin bat su aliyordu.
\par 27 Hiram her biri dört arsin uzunlugunda, dört arsin genisliginde ve üç arsin yüksekliginde on adet tunç ayaklik yapti.
\par 28 Ayakliklar aynaliklarla dösenmis, aynaliklar da çerçeve içine alinmisti.
\par 29 Aynaliklar aslan, boga, Keruv* motifleriyle süslenmisti. Çerçeveler de böyleydi, yalniz aslanlarla bogalarin üstünde ve altinda sarkik çelenk islemeleri vardi.
\par 30 Her bir ayakligin dört tunç tekerlegi ve dingilleri vardi. Dört köseye de kazan için destekler yapilmisti. Her dökme destek çelenklerle süslenmisti.
\par 31 Ayakligin üst yüzeyinde kazan için bir arsin yüksekliginde yuvarlak çerçeveli bir bosluk vardi. Boslugun tabani bir buçuk arsin genisligindeydi. Çevresinde oymalar vardi. Ayakliklarin aynaliklari yuvarlak degil, kareydi.
\par 32 Aynaliklarin altindaki dört tekerlegin dingilleri ayakliklara bagliydi. Her tekerlegin çapi bir buçuk arsindi*fe*.
\par 33 Tekerlekler savas arabalarinin tekerlekleri gibiydi. Dingilleri, jantlari, parmaklari ve göbeklerinin hepsi dökümdü.
\par 34 Her ayakligin dört kösesinde de kendinden dört destek vardi.
\par 35 Ayakliklarin üstünde yarim arsin yüksekliginde yuvarlak birer halka vardi. Ayakliklarin basindaki dayanaklar ve yan aynaliklar da ayakliklara bitisikti.
\par 36 Hiram dayanaklarin ve aynaliklarinin genisligi oraninda her birinin yüzeyine Keruvlar, aslanlar, hurma agaçlari, çevrelerine de çelenkler oydu.
\par 37 Böylece on ayakligi yapti; hepsinin dökümü, ölçüsü ve biçimi ayniydi.
\par 38 Hiram ayrica on ayakligin üzerine oturan dörder arsin genisliginde on tunç kazan yapti. Her kazan kirk bat su aliyordu.
\par 39 Ayakliklarin besini tapinagin güneyine, besini kuzeyine yerlestirdi. Havuzu ise tapinagin güneydogu kösesine yerlestirdi.
\par 40 Hiram kazanlar, kürekler, çanaklar yapti. Böylece Kral Süleyman için üstlenmis oldugu RAB'bin Tapinagi'yla ilgili bütün isleri tamamlamis oldu:
\par 41 Iki sütun ve iki yuvarlak sütun basligi, bu basliklari süsleyen iki örgülü ag,
\par 42 Sütunlarin yuvarlak basliklarini süsleyen iki örgülü agin üzerini ikiser sira halinde süsleyen dört yüz nar motifi,
\par 43 On kazan ve ayakliklari,
\par 44 Havuz ve havuzu tasiyan on iki boga heykeli,
\par 45 Kovalar, kürekler, çanaklar. Hiram'in Kral Süleyman için RAB'bin Tapinagi'na yaptigi bütün bu esyalar parlak tunçtandi.
\par 46 Kral bunlari Seria Ovasi'nda, Sukkot ile Saretan arasindaki killi topraklarda döktürmüstü.
\par 47 Esyalar o kadar çoktu ki, Süleyman hepsini tartmadi. Kullanilan tuncun hesabi tutulmadi.
\par 48 Süleyman'in RAB'bin Tapinagi için yaptirdigi altin esyalar sunlardi: Sunak, ekmeklerin Tanri'nin huzuruna kondugu masa,
\par 49 Iç odanin girisine, besi saga, besi sola yerlestirilen saf altin kandillikler, çiçek süslemeleri, kandiller, masalar,
\par 50 Saf altin taslar, fitil masalari, çanaklar, tabaklar, buhurdanlar. Tapinaktaki iç odanin, yani En Kutsal Yer'in* ve ana bölümün kapi menteseleri de altindandi.
\par 51 RAB'bin Tapinagi'nin yapimi tamamlaninca Kral Süleyman, babasi Davut'un adadigi altin, gümüs ve öbür esyalari getirip tapinagin hazine odalarina yerlestirdi.

\chapter{8}

\par 1 Kral Süleyman RAB'bin Antlasma Sandigi'ni* Davut Kenti olan Siyon'dan getirmek üzere Israil halkinin ileri gelenleriyle bütün oymak ve boy baslarini Yerusalim'e çagirdi.
\par 2 Hepsi yedinci ay* olan Etanim ayindaki bayramda Kral Süleyman'in önünde toplandi.
\par 3 Israil'in bütün ileri gelenleri toplaninca, bazi kâhinler Antlasma Sandigi'ni yerden kaldirdilar.
\par 4 Sandigi, Bulusma Çadiri'ni ve çadirdaki bütün kutsal esyalari kâhinlerle Levililer tapinaga tasidilar.
\par 5 Kral Süleyman ve bütün Israil toplulugu Antlasma Sandigi'nin önünde sayisiz davar ve sigir kurban etti.
\par 6 Kâhinler RAB'bin Antlasma Sandigi'ni tapinagin iç odasina, En Kutsal Yer'e* tasiyip Keruvlar'in* kanatlarinin altina yerlestirdiler.
\par 7 Keruvlar'in kanatlari sandigin kondugu yerin üstüne kadar uzaniyor ve sandigi da, siriklarini da örtüyordu.
\par 8 Siriklar öyle uzundu ki, uçlari iç odanin önündeki Kutsal Yer'den* görünüyordu. Ancak disaridan görünmüyordu. Bunlar hâlâ oradadir.
\par 9 Sandigin içinde Musa'nin Horev Dagi'nda koydugu iki tas levhadan baska bir sey yoktu. Bunlar Misir'dan çikislarinda RAB'bin Israilliler'le yaptigi antlasmanin tas levhalariydi.
\par 10 Kâhinler Kutsal Yer'den çikinca, RAB'bin Tapinagi'ni bir bulut doldurdu.
\par 11 Bu bulut yüzünden kâhinler görevlerini sürdüremediler. Çünkü RAB'bin görkemi tapinagi doldurmustu.
\par 12 O zaman Süleyman söyle dedi: "Ya RAB, karanlik bulutlarda otururum demistin.
\par 13 Senin için görkemli bir tapinak, sonsuza dek yasayacagin bir konut yaptim."
\par 14 Kral ayakta duran bütün Israil topluluguna dönerek onlari kutsadiktan sonra söyle dedi:
\par 15 "Babam Davut'a verdigi sözü tutan Israil'in Tanrisi RAB'be övgüler olsun! RAB demisti ki,
\par 16 'Halkim Israil'i Misir'dan çikardigim günden bu yana, içinde bulunacagim bir tapinak yaptirmak için Israil oymaklarina ait kentlerden hiçbirini seçmedim. Ancak halkim Israil'i yönetmesi için Davut'u seçtim.
\par 17 "Babam Davut Israil'in Tanrisi RAB'bin adina bir tapinak yapmayi yürekten istiyordu.
\par 18 Ama RAB, babam Davut'a, 'Adima bir tapinak yapmayi yürekten istemen iyi bir sey dedi,
\par 19 'Ne var ki, adima yapilacak bu tapinagi sen degil, öz oglun yapacak.
\par 20 "RAB verdigi sözü yerine getirdi. RAB'bin sözü uyarinca, babam Davut'tan sonra Israil tahtina ben geçtim ve Israil'in Tanrisi RAB'bin adina tapinagi ben yaptirdim.
\par 21 Ayrica, RAB'bin atalarimizi Misir'dan çikardiginda onlarla yaptigi antlasmanin içinde korundugu sandik için tapinakta bir yer hazirladim."
\par 22 Süleyman RAB'bin sunaginin önünde, Israil toplulugunun karsisinda durup ellerini göklere açti.
\par 23 "Ya RAB, Israil'in Tanrisi, yerde ve gökte sana benzer baska tanri yoktur" dedi, "Bütün yürekleriyle yolunu izleyen kullarinla yaptigin antlasmaya bagli kalirsin.
\par 24 Agzinla kulun babam Davut'a verdigin sözü bugün ellerinle yerine getirdin.
\par 25 "Simdi, ya RAB, Israil'in Tanrisi, kulun babam Davut'a verdigin öbür sözü de tutmani istiyorum. Ona, 'Senin soyundan Israil tahtina oturacaklarin ardi arkasi kesilmeyecektir; yeter ki, çocuklarin önümde senin gibi dikkatle yürüsünler demistin.
\par 26 Ey Israil'in Tanrisi, simdi kulun babam Davut'a verdigin sözleri yerine getirmeni istiyorum.
\par 27 "Tanri gerçekten yeryüzünde yasar mi? Sen göklere, göklerin göklerine bile sigmazsin. Benim yaptigim bu tapinak ne ki!
\par 28 Ya RAB Tanrim, kulunun bugün ettigi duayi, yalvarisi isit; duasina ve yakarisina kulak ver.
\par 29 Gözlerin gece gündüz, 'Orada bulunacagim! dedigin bu tapinagin üzerinde olsun. Kulunun buraya yönelerek ettigi duayi isit.
\par 30 Buraya yönelerek dua eden kulunun ve halkin Israil'in yalvarisini isit. Göklerden, oturdugun yerden kulak ver; duyunca bagisla.
\par 31 "Biri komsusuna karsi günah isleyip ant içmek zorunda kaldiginda, gelip bu tapinakta, senin sunaginin önünde ant içerse,
\par 32 göklerden kulak ver ve geregini yap. Suçlunun cezasini vererek, suçsuzu hakli çikararak kullarini yargila.
\par 33 "Sana karsi günah isledigi için düsmanlarina yenik düsen halkin Israil yine sana döner, adini anar, bu tapinakta dua edip yakararak önüne çikarsa,
\par 34 göklerden kulak ver, halkin Israil'in günahini bagisla. Onlari atalarina verdigin ülkeye yine kavustur.
\par 35 "Halkin sana karsi günah isledigi için gökler kapanip yagmur yagmazsa, sikintiya düsen halkin buraya yönelip dua eder, adini anar ve günahlarindan dönerse,
\par 36 göklerden kulak ver; kullarinin, halkin Israil'in günahlarini bagisla. Onlara dogru yolda yürümeyi ögret, halkina mülk olarak verdigin ülkene yagmurlarini gönder.
\par 37 "Ülkeyi kitlik, salgin hastalik, samyeli, küf, tirtil ya da çekirgeler kavurdugunda, düsmanlar kentlerden birinde halkini kusattiginda, herhangi bir felaket ya da hastalik ortaligi sardiginda,
\par 38 halkindan bir kisi ya da bütün halkin Israil basina gelen felaketi ayrimsar, dua edip yakararak ellerini bu tapinaga dogru açarsa,
\par 39 göklerden, oturdugun yerden kulak ver ve bagisla. Insanlarin yüreklerini yalnizca sen bilirsin. Onlara yaptiklarina göre davran ki,
\par 40 atalarimiza verdigin bu ülkede yasadiklari sürece senden korksunlar.
\par 41 "Halkin Israil'den olmayan, ama senin yüce adini,
\par 42 gücünü, kudretini duyup uzak ülkelerden gelen yabancilar bu tapinaga gelip dua ederlerse,
\par 43 göklerden, oturdugun yerden kulak ver, yalvarislarini yanitla. Öyle ki, dünyanin bütün uluslari, halkin Israil gibi, senin adini bilsin, senden korksun ve yaptirdigim bu tapinagin sana ait oldugunu ögrensin.
\par 44 "Halkin düsmanlarina karsi gösterdigin yoldan savasa giderken sana, seçtigin bu kente ve adina yaptirdigim bu tapinaga yönelip dua ederse,
\par 45 dualarina, yakarislarina göklerden kulak ver ve onlari kurtar.
\par 46 "Sana karsi günah islediklerinde -günah islemeyen tek kisi yoktur- sen öfkelenip onlari yakin ya da uzak bir ülkeye tutsak olarak götürecek düsmanlarinin eline teslim edersen,
\par 47 onlar da tutsak olduklari ülkede pismanlik duyup günahlarindan döner, 'Günah isledik, yoldan sapip kötülük yaptik diyerek sana yakarirlarsa,
\par 48 tutsak olduklari ülkede candan ve yürekten sana dönerlerse, atalarina verdigin ülkelerine, seçtigin kente ve adina yaptirdigim tapinagina yönelip dua ederlerse,
\par 49 göklerden, oturdugun yerden dualarina, yakarislarina kulak ver, onlari kurtar.
\par 50 Sana karsi günah islemis olan halkini ve isledikleri bütün suçlari bagisla. Düsmanlarinin onlara acimasini sagla.
\par 51 Çünkü onlar demir eritme ocagindan, Misir'dan çikardigin kendi halkin, kendi mirasindir.
\par 52 "Sana her yalvarislarinda onlara kulak ver, bu kulunun ve halkin Israil'in yalvarislarini dinle.
\par 53 Ey Egemen RAB, atalarimizi Misir'dan çikardiginda kulun Musa araciligiyla dedigin gibi, onlari dünyanin bütün halklari arasindan kendine miras olarak seçtin."
\par 54 Süleyman, RAB'be duasini ve yalvarisini bitirince, elleri göklere açik, dizleri üzerine çökmüs oldugu RAB'bin sunaginin önünden kalkti.
\par 55 Ayakta durup bütün Israil toplulugunu yüksek sesle söyle kutsadi:
\par 56 "Sözünü tutup halki Israil'e esenlik veren RAB'be övgüler olsun. Kulu Musa araciligiyla verdigi iyi sözlerin hiçbiri bosa çikmadi.
\par 57 Tanrimiz RAB atalarimizla oldugu gibi bizimle de olsun ve bizi hiç birakmasin, bizden ayrilmasin.
\par 58 Bütün yollarini izlememiz, atalarimiza verdigi buyruklara, kurallara, ilkelere uymamiz için RAB yüreklerimizi kendine yöneltsin.
\par 59 Ya RAB Tanrimiz, önünde yalvarirken söyledigim bu sözleri gece gündüz animsa. Kulunu ve halkin Israil'i her durumda koru.
\par 60 Sonunda dünyanin bütün uluslari bilsinler ki, tek Tanri RAB'dir ve O'ndan baska Tanri yoktur.
\par 61 Bugünkü gibi O'nun kurallarina göre yasamak ve buyruklarina uymak için bütün yüreginizi Tanrimiz RAB'be adayin."
\par 62 Kral ve bütün Israil halki RAB'bin önünde kurban kestiler.
\par 63 Süleyman, esenlik kurbani olarak RAB'be yirmi iki bin sigir, yüz yirmi bin davar kurban etti. Böylece kral ve bütün Israil halki, RAB'bin Tapinagi'ni adama isini tamamlamis oldu.
\par 64 Ayni gün kral, tapinagin önündeki avlunun orta kismini da kutsadi. Yakmalik sunulari*, tahil sunularini* ve esenlik sunularinin* yaglarini orada sundu. Çünkü RAB'bin önündeki tunç* sunak yakmalik sunulari, tahil sunularini ve esenlik sunularinin yaglarini alamayacak kadar küçüktü.
\par 65 Süleyman, Levo-Hamat'tan Misir Vadisi'ne kadar her yerden gelen Israilliler'in olusturdugu büyük toplulukla birlikte Tanrimiz RAB'bin önünde art arda yediser gün, toplam on dört gün bayram yapti.
\par 66 Kral sekizinci gün halki evlerine gönderdi. Onlar da krali kutsayip RAB'bin, kulu Davut ve halki Israil için yapmis oldugu bütün iyiliklerden dolayi sevinç duyarak mutluluk içinde evlerine döndüler.

\chapter{9}

\par 1 Süleyman RAB'bin Tapinagi'ni, sarayi ve yapmayi istedigi bütün isleri bitirince,
\par 2 RAB daha önce Givon'da oldugu gibi ona yine görünerek
\par 3 söyle dedi: "Duani ve yakarisini duydum. Adim sürekli orada bulunsun diye yaptigin bu tapinagi kutsal kildim. Gözlerim onun üstünde, yüregim her zaman orada olacaktir.
\par 4 Sana gelince, baban Davut'un yaptigi gibi, bütün yüreginle ve dogrulukla yollarimi izler, buyurdugum her seyi yapar, kurallarima ve ilkelerime uyarsan,
\par 5 baban Davut'a, 'Israil tahtindan senin soyunun ardi arkasi kesilmeyecektir diye verdigim sözü tutup kralligini sonsuza dek pekistirecegim.
\par 6 "Ama siz ya da çocuklariniz yollarimdan sapar, buyruklarima ve kurallarima uymaz, gidip baska ilahlara kulluk eder, taparsaniz,
\par 7 size verdigim bu ülkeden sizi söküp atacagim, adima kutsal kildigim bu tapinagi terk edecegim; Israil bütün uluslar arasinda asagilanip alay konusu olacak.
\par 8 Bu gösterisli tapinagin önünden geçenler, hayret ve dehset içinde, 'RAB bu ülkeyi ve tapinagi neden bu duruma getirdi? Diye soracaklar.
\par 9 Ve, diyecekler ki, 'Israil halki, atalarini Misir'dan çikaran Tanrilari RAB'bi terk etti; baska ilahlarin ardindan gitti, onlara tapip kulluk etti. RAB bu yüzden bütün bu kötülükleri baslarina getirdi."
\par 10 Süleyman iki yapiyi -RAB'bin Tapinagi'yla kendi sarayini- yirmi yilda bitirdi.
\par 11 Bu yapilar için istedigi sedir ve çam agaçlariyla altini saglayan Sur Krali Hiram'a Celile bölgesinde yirmi kent verdi.
\par 12 Hiram gidip Süleyman'in kendisine verdigi kentleri görünce onlari begenmedi.
\par 13 Süleyman'a, "Bunlar mi bana verdigin kentler, kardesim!" dedi. Bu yüzden o bölge bugün de "Kavul" diye bilinir.
\par 14 Oysa Hiram, Kral Süleyman'a yüz yirmi talant altin göndermisti.
\par 15 Kral Süleyman RAB'bin Tapinagi'ni, kendi sarayini, Millo'yu* ve Yerusalim'in surlarini yaptirmak; ayrica Hasor, Megiddo ve Gezer kentlerini onarip güçlendirmek amaciyla angaryacilari toplamisti.
\par 16 Misir Firavunu gidip Gezer'i ele geçirmis ve atese vermisti. Orada yasayan Kenanlilar'i öldürerek kenti Süleyman'la evlenen kizina armagan etmisti.
\par 17 Süleyman Gezer'i, Asagi Beythoron'u,
\par 18 Baalat'i ve kirsal bir bölgede bulunan Tamar'la birlikte
\par 19 bütün ambarli kentleri, ayrica savas arabalariyla atlarin bulundugu kentleri de onarip güçlendirdi. Böylece Yerusalim'de, Lübnan'da, yönetimi altindaki bütün topraklarda her istedigini yaptirmis oldu.
\par 20 Israil halkindan olmayan Amorlular, Hititler*, Perizliler, Hivliler ve Yevuslular'dan artakalanlara gelince,
\par 21 Süleyman Israil halkinin tamamen yok edemedigi bu insanlarin soyundan gelip ülkede kalanlari angaryaya kostu. Bu durum bugün de sürüyor.
\par 22 Ancak Süleyman Israil halkindan hiç kimseye kölelik yaptirmadi. Onlar savasçi, görevli, komutan, subay, savas arabalariyla atlilarin komutani olarak görev yaptilar.
\par 23 Süleyman'in yapilan islerin basinda duran ve çalisanlari denetleyen bes yüz elli görevlisi vardi.
\par 24 Firavunun kizi, Davut Kenti'nden Süleyman'in kendisi için yaptirdigi saraya tasindiktan sonra, Süleyman Millo'yu yaptirdi.
\par 25 Süleyman RAB için yaptirdigi sunakta yilda üç kez yakmalik sunular* ve esenlik sunulari* sunardi. Ayrica RAB'bin önündeki sunagin üstünde buhur da yakardi. Böylece Süleyman tapinagin yapimini tamamlamis oldu.
\par 26 Kral Süleyman Edomlular'in ülkesinde, Kizildeniz* kiyisinda Eylat yakinlarindaki Esyon-Gever'de gemiler yaptirdi.
\par 27 Hiram denizi bilen gemicilerini Süleyman'in adamlariyla birlikte Ofir'e gönderdi.
\par 28 Ofir'e giden bu gemiciler, Kral Süleyman'a 420 talant altin getirdiler.

\chapter{10}

\par 1 Saba Kraliçesi, RAB'bin adindan ötürü Süleyman'in artan ününü duyunca, onu çetin sorularla sinamaya geldi.
\par 2 Çesitli baharat, çok miktarda altin ve degerli taslarla yüklü büyük bir kervan esliginde Yerusalim'e gelen kraliçe, aklindan geçen her seyi Süleyman'la konustu.
\par 3 Süleyman onun bütün sorularina karsilik verdi. Kralin ona yanit bulmakta güçlük çektigi hiçbir konu olmadi.
\par 4 Süleyman'in bilgeligini, yaptirdigi sarayi, sofrasinin zenginligini, görevlilerinin oturup kalkisini, hizmetkârlarinin özel giysileriyle yaptigi hizmeti, sakilerini ve RAB'bin Tapinagi'nda sundugu yakmalik sunulari* gören Saba Kraliçesi hayranlik içinde kaldi.
\par 6 Krala, "Ülkemdeyken yaptiklarinla ve bilgeliginle ilgili duyduklarim dogruymus" dedi,
\par 7 "Ama gelip kendi gözlerimle görünceye dek inanmamistim. Bunlarin yarisi bile bana anlatilmadi. Bilgeligin de, zenginligin de duyduklarimdan kat kat fazla.
\par 8 Ne mutlu adamlarina! Ne mutlu sana hizmet eden görevlilere! Çünkü sürekli bilgeligine tanik oluyorlar.
\par 9 Senden hosnut kalan, seni Israil tahtina oturtan Tanrin RAB'be övgüler olsun! RAB Israil'e sonsuz sevgi duydugundan, adaleti ve dogrulugu saglaman için seni kral yapti."
\par 10 Saba Kraliçesi krala 120 talant altin, çok büyük miktarda baharat ve degerli taslar armagan etti. Krala o kadar baharat armagan etti ki, bir daha bu kadar çok baharat görülmedi.
\par 11 Bu arada Hiram'in gemileri Ofir'den altin ve büyük miktarda almug*fm* kerestesiyle degerli taslar getirdiler.
\par 12 Kral, RAB'bin Tapinagi'yla sarayin tirabzanlarini, çalgicilarin lirleriyle çenklerini bu almug kerestesinden yaptirdi. Bugüne dek o kadar almug agaci ne gelmis, ne de görülmüstür.
\par 13 Kral Süleyman Saba Kraliçesi'nin her istegini, her dilegini yerine getirdi. Ayrica ona gönülden kopan birçok armagan verdi. Bundan sonra kraliçe adamlariyla birlikte oradan ayrilip kendi ülkesine döndü.
\par 14 Süleyman'a bir yilda gelen altinin miktari 666 talanti buluyordu.
\par 15 Alim satimla ugrasanlarla tüccarlarin kazançlarindan ve Arabistan'in bütün krallariyla Israil valilerinden gelenler bunun disindaydi.
\par 16 Kral Süleyman her biri alti yüz sekel agirliginda dövme altindan iki yüz büyük kalkan yaptirdi.
\par 17 Ayrica her biri üç mina agirliginda dövme altindan üç yüz küçük kalkan yaptirdi. Kral bu kalkanlari Lübnan Ormani adindaki saraya koydu.
\par 18 Kral fildisinden büyük bir taht yaptirip saf altinla kaplatti.
\par 19 Tahtin alti basamagi, arka kisminda yuvarlak bir basligi vardi. Oturulan yerin iki yaninda kollar, her kolun yaninda birer aslan heykeli bulunuyordu.
\par 20 Alti basamagin iki yaninda on iki aslan heykeli vardi. Hiçbir krallikta böylesi yapilmamisti.
\par 21 Kral Süleyman'in kadehleriyle Lübnan Ormani adindaki sarayin bütün esyalari saf altindan yapilmis, hiç gümüs kullanilmamisti. Çünkü Süleyman'in döneminde gümüsün degeri yoktu.
\par 22 Hiram'in gemilerinin yanisira, kralin da denizde ticaret gemileri vardi. Bu gemiler üç yilda bir altin, gümüs, fildisi ve türlü maymunlarla yüklü olarak dönerlerdi.
\par 23 Kral Süleyman dünyanin bütün krallarindan daha zengin, daha bilgeydi.
\par 24 Tanri'nin Süleyman'a verdigi bilgeligi dinlemek için bütün dünya onu görmek isterdi.
\par 25 Onu görmeye gelenler her yil armagan olarak altin ve gümüs esya, giysi, silah, baharat, at, katir getirirlerdi.
\par 26 Süleyman savas arabalariyla atlarini topladi. Bin dört yüz savas arabasi, on iki bin ati vardi. Bunlarin bir kismini savas arabalari için ayrilan kentlere, bir kismini da kendi yanina, Yerusalim'e yerlestirdi.
\par 27 Kralligi döneminde Yerusalim'de gümüs tas degerine düstü. Sedir agaçlari Sefela'daki yabanil incir agaçlari kadar bollasti.
\par 28 Süleyman'in atlari Misir ve Keve'den getirilirdi. Kralin tüccarlari atlari Keve'den satin alirdi.
\par 29 Misir'dan bir savas arabasi alti yüz, bir at yüz elli sekel gümüse getirilirdi. Bunlari bütün Hitit* ve Aram krallarina satarlardi.

\chapter{11}

\par 1 Kral Süleyman firavunun kizinin yanisira Moavli, Ammonlu, Edomlu, Saydali ve Hititli* birçok yabanci kadin sevdi.
\par 2 Bu kadinlar RAB'bin Israil halkina, "Ne siz onlarin arasina girin, ne de onlar sizin araniza girsinler; çünkü onlar kesinlikle sizi kendi ilahlarinin ardinca yürümek üzere saptiracaklardir" dedigi uluslardandi. Buna karsin, Süleyman onlara sevgiyle baglandi.
\par 3 Süleyman'in kral kizlarindan yedi yüz karisi ve üç yüz cariyesi vardi. Karilari onu yolundan saptirdilar.
\par 4 Süleyman yaslandikça, karilari onu baska ilahlarin ardinca yürümek üzere saptirdilar. Böylece Süleyman bütün yüregini Tanrisi RAB'be adayan babasi Davut gibi yasamadi.
\par 5 Saydalilar'in tanriçasi Astoret'e* ve Ammonlular'in igrenç ilahi Molek'e* tapti.
\par 6 Böylece RAB'bin gözünde kötü olani yapti, RAB'bin yolunda yürüyen babasi Davut gibi tam anlamiyla RAB'bi izlemedi.
\par 7 Yerusalim'in dogusundaki tepede Moavlilar'in igrenç ilahi Kemos'a ve Ammonlular'in igrenç ilahi Molek'e tapmak için bir yer yaptirdi.
\par 8 Ilahlarina buhur yakip kurban kesen bütün yabanci karilari için de ayni seyleri yapti.
\par 9 Israil'in Tanrisi RAB, kendisine iki kez görünüp, "Baska ilahlara tapma!" demesine karsin, Süleyman RAB'bin yolundan sapti ve O'nun buyruguna uymadi. Bu yüzden RAB Süleyman'a öfkelenerek,
\par 11 "Seninle yaptigim antlasmaya ve kurallarima bilerek uymadigin için kralligi elinden alacagim ve görevlilerinden birine verecegim" dedi,
\par 12 "Ancak baban Davut'un hatiri için, bunu senin yasadigin sürede degil, oglun kral olduktan sonra yapacagim.
\par 13 Ama oglunun elinden bütün kralligi almayacagim. Kulum Davut'un ve kendi seçtigim Yerusalim'in hatiri için ogluna bir oymak birakacagim."
\par 14 RAB kral soyundan gelen bir düsmani, Edomlu Hadat'i Süleyman'a karsi ayaklandirdi.
\par 15 Daha önce, Davut Edomlular'la savasirken, ölüleri gömmeye giden ordu komutani Yoav Edom'daki bütün erkekleri öldürmüstü.
\par 16 Yoav ile Israilliler Edom'daki erkeklerin hepsini yok edinceye dek, alti ay orada kalmislardi.
\par 17 Ancak genç Hadat, babasinin görevlilerinden bazi Edomlular'la birlikte Misir'a kaçmisti.
\par 18 Sonra Midyan'dan ayrilip Paran'a gitmisler, oradan bazi Paranlilar'i da yanlarina alip Misir'a, firavunun yanina gelmislerdi. Firavun Hadat'a barinak, yiyecek ve toprak saglamisti.
\par 19 Hadat firavunun dostlugunu kazandi. Bunun üzerine firavun, kendi karisi Kraliçe Tahpenes'in kizkardesini Hadat'la evlendirdi.
\par 20 Tahpenes'in kizkardesi Hadat'a Genuvat adli bir ogul dogurdu. Tahpenes çocugu sarayda firavunun çocuklariyla birlikte büyüttü.
\par 21 Hadat Davut'la ordu komutani Yoav'in ölüm haberini Misir'da duydu. Firavuna, "Izin ver, kendi ülkeme döneyim" dedi.
\par 22 Firavun, "Bir eksigin mi var, neden ülkene dönmek istiyorsun?" diye sordu. Hadat, "Hayir, ama lütfen gitmeme izin ver" diye yanitladi.
\par 23 Tanri, efendisi Sova Krali Hadadezer'den kaçan bir düsmani, Elyada oglu Rezon'u Süleyman'a karsi ayaklandirdi.
\par 24 Davut Sovalilar'a saldirdiginda, Rezon çevresine haydutlari toplayip onlara önderlik etmisti. Birlikte Sam'a gitmisler, orada kalip yönetimi ele geçirmislerdi.
\par 25 Hadat'in yaptigi kötülügün yanisira, Rezon Süleyman yasadigi sürece Israil'in düsmani oldu; Aram'da krallik yaparak Israil'den nefret etti.
\par 26 Efrayim oymagindan Nevat oglu Seredali Yarovam Kral Süleyman'a karsi ayaklandi. Yarovam Süleyman'in görevlilerindendi. Annesi Serua adli dul bir kadindi.
\par 27 Yarovam'in krala karsi ayaklanmasinin öyküsü söyleydi: Süleyman Millo'yu* yaptirip babasi Davut Kenti'ndeki surlarin gedigini kapatmisti.
\par 28 Yarovam çok yetenekli biriydi. Süleyman bu genç adamin ne denli çaliskan oldugunu görünce, Yusuf soyunun bütün agir islerinin sorumlulugunu ona verdi.
\par 29 Bir gün Yarovam Yerusalim'in disina çikti. Yolda Silolu Peygamber Ahiya ile karsilasti. Ahiya yeni giysisini giymisti. Ikisi kent disinda yalnizdilar.
\par 30 Ahiya üzerindeki giysiyi yirtip on iki parçaya ayirdi
\par 31 ve Yarovam'a, "On parçayi kendine al" dedi, "Çünkü Israil'in Tanrisi RAB diyor ki, 'Ben, Süleyman'in elinden kralligi alip on oymagi sana verecegim.
\par 32 Ama kulum Davut'un ve Israil oymaklarinin yasadigi kentler arasindan seçtigim Yerusalim Kenti'nin hatiri için bir oymagi onda birakacagim.
\par 33 Çünkü Süleyman bana sirt çevirip Saydalilar'in tanriçasi Astoret'e*, Moavlilar'in ilahi Kemos'a ve Ammonlular'in ilahi Molek'e* tapti. Kurallarima, ilkelerime uyup gözümde dogru olani yapan babasi Davut gibi yollarimi izlemedi.
\par 34 Ama buyruklarima, kurallarima bagli kalan, seçtigim kulum Davut'un hatiri için Süleyman'in elinden bütün kralligi almayacagim. Yasami boyunca onu önder yapacagim.
\par 35 Ancak kralligi oglunun elinden alip on oymagi sana verecegim.
\par 36 Adimi yerlestirmek için kendime seçtigim Yerusalim Kenti'nde kulum Davut için önümde sönmeyen bir isik olmak üzere Süleyman'in ogluna bir oymak verecegim.
\par 37 Sana gelince, seni Israil Krali yapacagim. Israil'i diledigin gibi yöneteceksin.
\par 38 Kulum Davut gibi isteklerimi yerine getirir, kurallarima ve buyruklarima uyar, gözümde dogru olani yapar, yollarimi izlersen, seninle birlikte olacagim. Davut'a yaptigim gibi senin için de güçlü bir hanedan kurup Israil'i sana verecegim.
\par 39 Süleyman'in günahindan ötürü Davut soyunun gururunu kiracagim, ancak sonsuza dek degil."
\par 40 Süleyman Yarovam'i öldürmeye çalisti. Ama Yarovam Misir'a kaçip Misir Krali Sisak'a sigindi. Süleyman'in ölümüne kadar orada kaldi.
\par 41 Süleyman'in kralligi dönemindeki öteki olaylar, bütün yaptiklari ve bilgeligi Süleyman'in tarihinde yazilidir.
\par 42 Süleyman kirk yil süreyle bütün Israil'i Yerusalim'den yönetti.
\par 43 Süleyman ölüp atalarina kavusunca babasi Davut Kenti'nde gömüldü. Yerine oglu Rehavam kral oldu.

\chapter{12}

\par 1 Rehavam Sekem'e gitti. Çünkü bütün Israilliler kendisini kral ilan etmek için orada toplanmislardi.
\par 2 Kral Süleyman'dan kaçip Misir'a yerlesen Nevat oglu Yarovam bunu duyunca Misir'da kalmaya karar verdi.
\par 3 Israil toplulugu Yarovam'i çagirtti. Birlikte gidip Rehavam'a söyle dediler:
\par 4 "Baban üzerimize agir bir boyunduruk koydu. Ama babanin üzerimize yükledigi agir yükü ve boyundurugu hafifletirsen sana kul köle oluruz."
\par 5 Rehavam, "Simdi gidin, üç gün sonra yine gelin" yanitini verince halk yanindan ayrildi.
\par 6 Kral Rehavam, babasi Süleyman'a sagliginda danismanlik yapan ileri gelenlere, "Bu halka nasil yanit vermemi ögütlersiniz?" diye sordu.
\par 7 Ileri gelenler, "Bugün bu halka hizmet eder, olumlu yanit verirsen, sana her zaman kul köle olurlar" diye karsilik verdiler.
\par 8 Ne var ki, Rehavam ileri gelenlerin ögüdünü reddederek birlikte büyüdügü genç görevlilerine danisti:
\par 9 "Siz ne yapmami ögütlersiniz? 'Babanin üzerimize koydugu boyundurugu hafiflet diyen bu halka nasil bir yanit verelim?"
\par 10 Birlikte büyüdügü gençler ona su karsiligi verdiler: "Sana 'Babanin üzerimize koydugu boyundurugu hafiflet diyen halka de ki, 'Benim küçük parmagim babamin belinden daha kalindir.
\par 11 Babam size agir bir boyunduruk yüklediyse, ben boyundurugunuzu daha da agirlastiracagim. Babam sizi kirbaçla yola getirdiyse, ben sizi akreplerle yola getirecegim."
\par 12 Yarovam'la bütün halk, kralin, "Üç gün sonra yine gelin" sözü üzerine, üçüncü gün Rehavam'in yanina geldiler.
\par 13 Ileri gelenlerin ögüdünü reddeden Kral Rehavam, gençlerin ögüdüne uyarak halka sert bir yanit verdi: "Babamin size yükledigi boyundurugu ben daha da agirlastiracagim. Babam sizi kirbaçla yola getirdiyse, ben sizi akreplerle yola getirecegim."
\par 15 Kral halki dinlemedi. Çünkü Silolu Ahiya araciligiyla Nevat oglu Yarovam'a verdigi sözü yerine getirmek için RAB bu olayi düzenlemisti.
\par 16 Kralin kendilerini dinlemedigini görünce, bütün Israilliler, "Isay oglu, Davut'la ne ilgimiz, Ne de payimiz var!" diye bagirdilar, "Ey Israil halki, haydi evimize dönelim! Davut'un soyu basinin çaresine baksin." Böylece herkes evine döndü.
\par 17 Rehavam da yalnizca Yahuda kentlerinde yasayan Israilliler'e krallik yapmaya basladi.
\par 18 Israilliler Kral Rehavam'in gönderdigi angaryacibasi Adoram'i tasa tutup öldürdüler. Bunun üzerine Kral Rehavam savas arabasina atlayip Yerusalim'e kaçti.
\par 19 Israil halki, Davut soyundan gelenlere hep baskaldirdi.
\par 20 Yarovam'in Misir'dan döndügünü duyunca, bütün Israilliler haber gönderip kendisini toplantiya çagirdilar ve onu Israil Krali ilan ettiler. Yahuda oymagindan baska hiç kimse Davut soyunu izlemedi.
\par 21 Süleyman oglu Rehavam Yerusalim'e varinca, Israil oymaklariyla savasip onlari yeniden egemenligi altina almak amaciyla bütün Yahuda ve Benyamin oymaklarindan yüz seksen bin seçkin savasçi topladi.
\par 22 Bu arada Tanri adami Semaya'ya Tanri söyle seslendi:
\par 23 "Süleyman oglu Yahuda Krali Rehavam'a, bütün Yahudalilar'a, Benyaminliler'e ve orada yasayan öteki insanlara sunu söyle:
\par 24 'RAB diyor ki, Israilli kardeslerinize saldirmayin, onlarla savasmayin. Herkes evine dönsün! Çünkü bu olayi ben düzenledim." RAB'bin bu sözlerini duyan halk O'nun buyruguna uyup evine döndü. Beytel ve Dan'daki Altin Buzagilar
\par 25 Yarovam Efrayim'in daglik bölgesindeki Sekem Kenti'ni onarip orada yasamaya basladi. Daha sonra oradan ayrilip Penuel Kenti'ni onardi.
\par 26 Yarovam, "Simdi krallik yine Davut soyunun eline geçebilir" diye düsündü,
\par 27 "Eger bu halk Yerusalim'e gidip RAB'bin Tapinagi'nda kurbanlar sunarsa, yürekleri efendileri, Yahuda Krali Rehavam'a döner. Beni öldürüp yeniden Rehavam'a baglanirlar."
\par 28 Kral, danismanlarina danistiktan sonra, iki altin buzagi yaptirip halkina, "Tapinmak için artik Yerusalim'e gitmenize gerek yok" dedi, "Ey Israil halki, iste sizi Misir'dan çikaran ilahlariniz!"
\par 29 Altin buzagilardan birini Beytel, ötekini Dan Kenti'ne yerlestirdi.
\par 30 Bu günahti. Böylece halk buzagiya tapmak için Dan'a kadar gitmeye basladi.
\par 31 Yarovam ayrica tapinma yerlerinde tapinaklar yaptirdi. Levililer'in disinda her türlü insanlardan kâhinler atadi.
\par 32 Yarovam sekizinci ayin on besinci günü Yahuda'daki bayrama benzer bir bayram baslatti. Dan'daki sunakta ve Beytel'de yaptirdigi altin buzagilara kurbanlar sundu; orada kurmus oldugu tapinma yerlerine kâhinler yerlestirdi.
\par 33 Kendi kendine uydurdugu sekizinci ayin on besinci günü, Beytel'de yaptirdigi sunaga gitti, kurban sunup buhur yakti. Ve o günü Israil halki için bayram ilan etti.

\chapter{13}

\par 1 Yarovam buhur yakmak için sunagin yaninda dururken, bir Tanri adami RAB'bin buyrugu üzerine Yahuda'dan Beytel'e geldi.
\par 2 RAB'bin buyrugu uyarinca sunaga karsi söyle seslendi: "Sunak, ey sunak! RAB diyor ki, 'Davut'un soyundan Yosiya adinda bir erkek çocuk dogacak. Buhur yakan, tapinma yerlerinde görevli kâhinleri senin üstünde kurban edecek. Üstünde insan kemikleri yakilacak."
\par 3 Ayni gün Tanri adami bir belirti göstererek konusmasini söyle sürdürdü: "RAB'bin bana açikladigi belirti sudur: Bu sunak parçalanacak, üstündeki küller çevreye savrulacak."
\par 4 Kral Yarovam, Tanri adaminin Beytel'de sunaga karsi söylediklerini duyunca, elini ona dogru uzatarak, "Yakalayin onu!" diye buyruk verdi. Ancak Tanri adamina uzattigi eli felç oldu ve düzelmedi.
\par 5 Tanri adaminin RAB'bin buyruguyla gösterdigi belirti uyarinca, sunak parçalandi, üstündeki küller çevreye savruldu.
\par 6 O zaman Kral Yarovam, Tanri adamina, "Lütfen benim için dua et, Tanrin RAB'be yalvar ki, elim eski halini alsin" dedi. Tanri adami RAB'be yalvarinca kralin eli iyilesip eski halini aldi.
\par 7 Kral, Tanri adamina, "Benimle eve kadar gel de bir seyler ye" dedi, "Sana bir armagan verecegim."
\par 8 Tanri adami, "Varliginin yarisini bile versen, seninle gelmem" diye karsilik verdi, "Burada ne yer, ne de içerim.
\par 9 Çünkü RAB bana, 'Orada hiçbir sey yiyip içme ve gittigin yoldan dönme diye buyruk verdi."
\par 10 Böylece Tanri adami, Beytel'e gelmis oldugu yoldan degil, baska bir yoldan gitti.
\par 11 O siralarda Beytel'de yasayan yasli bir peygamber vardi. Çocuklari gelip o gün Tanri adaminin Beytel'de yaptiklarini ve krala söylediklerini babalarina anlattilar.
\par 12 Yasli baba, "Tanri adami hangi yoldan gitti?" diye sordu. Çocuklar Yahuda'dan gelen Tanri adaminin hangi yoldan gittigini ona gösterdiler.
\par 13 Bunun üzerine yasli baba, "Esegimi hazirlayin" dedi. Çocuklar esege palan vurunca, binip
\par 14 Tanri adaminin ardina düstü. Onun bir yabanil fistik agacinin altinda oturdugunu görünce, "Yahuda'dan gelen Tanri adami sen misin?" diye sordu. Adam, "Evet, benim" diye karsilik verdi.
\par 15 Yasli peygamber, "Gel benimle eve gidelim, bir seyler ye" dedi.
\par 16 Tanri adami söyle karsilik verdi: "Yolumdan dönüp seninle gelemem. Burada ne yer, ne de içerim.
\par 17 Çünkü RAB bana, 'Orada hiçbir sey yiyip içme ve gittigin yoldan dönme diye buyruk verdi."
\par 18 Bunun üzerine yasli peygamber, "Ben de senin gibi peygamberim" dedi, "RAB'bin buyrugu üzerine bir melek bana, 'Onu evine götür ve yiyip içmesini sagla dedi." Ne var ki yalan söyleyerek Tanri adamini kandirdi.
\par 19 Böylece Tanri adami onunla birlikte geri döndü ve evinde yiyip içmeye basladi.
\par 20 Sofrada otururlarken, RAB, Tanri adamini yolundan döndüren peygambere seslendi.
\par 21 O da Yahuda'dan gelen Tanri adamina söyle dedi: "RAB diyor ki, 'Madem RAB'bin sözünü dinlemedin, Tanrin RAB'bin sana verdigi buyruga uymayip
\par 22 yolundan döndün; sana yiyip içme dedigi yerde yiyip içtin, cesedin atalarinin mezarligina gömülmeyecek."
\par 23 Tanri adami yiyip içtikten sonra yasli peygamber onun için esegi hazirladi.
\par 24 Tanri adami giderken yolda bir aslanla karsilasti. Aslan onu oracikta öldürdü. Esekle aslan yere serilen cesedin yaninda duruyordu.
\par 25 Yoldan geçenler yerde yatan cesetle yaninda duran aslani gördüler. Gidip yasli peygamberin yasadigi kentte gördüklerini anlattilar.
\par 26 Tanri adamini yolundan döndüren yasli peygamber olanlari duyunca, söyle dedi: "RAB'bin sözünü dinlemeyen Tanri adami budur. Bu yüzden RAB, sözü uyarinca, onun üzerine bir aslan gönderdi. Aslan onu parçalayip öldürdü."
\par 27 Peygamber, çocuklarina, "Esegi hazirlayin!" dedi. Çocuklarin hazirladigi esege binip gitti. Esekle aslani yerde yatan Tanri adaminin cesedi basinda buldu. Ancak aslan cesedi yemedigi gibi esege de dokunmamisti.
\par 29 Yasli peygamber Tanri adaminin cesedini esegin sirtina atti ve ona agit yakip gömmek için kendi kentine götürdü.
\par 30 Cesedi kendi mezarina gömdü. "Ah kardesim!" diyerek ardindan agit yaktilar.
\par 31 Tanri adamini gömdükten sonra, yasli peygamber çocuklarina söyle dedi: "Ben ölünce beni bu Tanri adaminin yanina gömün, kemiklerimi de onun kemiklerinin yanina koyun.
\par 32 Çünkü onun Beytel'deki sunaga ve Samiriye kentlerindeki tapinma yerlerinde bulunan bütün tapinaklara karsi RAB'bin buyruguyla yaptigi uyarilar kesinlikle yerine gelecektir."
\par 33 Bu olaya karsin Yarovam gittigi kötü yoldan ayrilmadi. Yine her türlü insani tapinma yerlerine kâhin olarak atadi ve buralara kâhin olmak isteyen herkese görev verdi.
\par 34 Yarovam'in soyu günah isledi. Bu günahlar onlarin yeryüzünden silinip yok edilmesine neden oldu.

\chapter{14}

\par 1 O siralarda Israil Krali Yarovam'in oglu Aviya hastalandi.
\par 2 Yarovam, karisina, "Kalk, Yarovam'in karisi oldugunu anlamamalari için kiligini degistirip Silo'ya git" dedi, "Bu halkin krali olacagimi bana bildiren Peygamber Ahiya orada oturuyor.
\par 3 Ona on ekmek, birkaç çörek, bir tulum bal götür. Çocuga ne olacagini o sana bildirecektir."
\par 4 Yarovam'in karisi denileni yapti; kalkip Silo'ya, Ahiya'nin evine gitti. Ahiya'nin gözleri yasliliktan görmez olmustu.
\par 5 RAB, Ahiya'ya söyle dedi: "Simdi Yarovam'in karisi gelecek. Hastalanan oglunun durumunu senden soracak. Onu söyledigin gibi yanitlayacaksin. O geldiginde kendini sana baska biriymis gibi gösterecek."
\par 6 Ahiya, kapidan içeri giren kadinin ayak seslerini duyunca, "Gel, Yarovam'in karisi!" dedi, "Neden baska kiliga giriyorsun? Sana kötü haberlerim var.
\par 7 Git Yarovam'a de ki, 'Israil'in Tanrisi RAB, Ben seni halkin arasindan seçip kendi halkima, Israilliler'e önder yaptim, diyor,
\par 8 Kralligi Davut'un soyundan alip sana verdim. Ama sen buyruklarima uyan, gözümde yalniz dogru olani yapan ve bütün yüregiyle yollarimi izleyen kulum Davut'a benzemedin.
\par 9 Senden önce yasayanlarin hepsinden çok kötülük yaptin. Beni reddettin; kendine baska ilahlar buldun, dökme putlar yaparak beni öfkelendirdin.
\par 10 "'Bundan dolayi Yarovam'in ailesini sikintilara sokup Israil'de onun soyundan gelen genç yasli*ft* bütün erkekleri öldürecegim. Yarovam'in ailesini gübre yakarcasina kökünden kurutacagim.
\par 11 Yarovam'in ailesinden kentte ölenleri köpekler, kirda ölenleri yirtici kuslar yiyecek. RAB böyle konustu. f t 14:10 "Genç yasli" ya da "Köle olsun, özgür olsun".
\par 12 "Sana gelince, kalk, evine dön. Kente ayak basar basmaz çocuk ölecek.
\par 13 Bütün Israil halki agit yakip onu gömecek. Yarovam'in ailesinden yalniz o gömülecek. Çünkü Yarovam ailesi içinde Israil'in Tanrisi RAB'bi hosnut eden nitelikler yalniz onda bulundu.
\par 14 "RAB Israil'e bir kral atayacak. Bu kral ayni gün Yarovam'in ailesine son verecek. Ne zaman mi? Hemen simdi.
\par 15 RAB Israil halkini cezalandiracak. Israil halki suda sallanan bir kamisa dönecek. RAB onlari atalarina vermis oldugu bu iyi topraklardan söküp Firat Irmagi'nin ötelerine dagitacak. Çünkü Asera* putlarini dikerek RAB'bi öfkelendirdiler.
\par 16 Yarovam'in isledigi ve Israil halkini sürükledigi günahlar yüzünden RAB Israil'i terk edecek."
\par 17 Yarovam'in karisi oradan ayrilip Tirsa'ya döndü. Evinin esigine varinca çocuk öldü.
\par 18 Bütün Israil halki, RAB'bin kulu Peygamber Ahiya araciligiyla söyledigi söz uyarinca, çocugu gömüp onun için agit yakti.
\par 19 Yarovam'in kralligi dönemindeki öteki olaylar, nasil savastigi, ülkesini nasil yönettigi Israil krallarinin tarihinde yazilidir.
\par 20 Yarovam yirmi iki yil krallik yapti. Ölüp atalarina kavusunca, yerine oglu Nadav kral oldu.
\par 21 Süleyman oglu Rehavam Yahuda Krali oldugunda kirk bir yasindaydi. RAB'bin adini yerlestirmek için bütün Israil oymaklarinin yasadigi kentler arasindan seçtigi Yerusalim Kenti'nde on yedi yil krallik yapti. Annesi Ammonlu Naama'ydi.
\par 22 Yahudalilar RAB'bin gözünde kötü olani yaparak, isledikleri günahlarla Tanri'yi atalarindan daha çok öfkelendirdiler.
\par 23 Ayrica kendilerine her yüksek tepenin üstüne ve bol yaprakli her agacin altina tapinma yerleri, dikili taslar ve Asera* putlari yaptilar.
\par 24 Ülkedeki putperest törenlerinde fuhus yapan kadin ve erkekler bile vardi. Yahudalilar RAB'bin Israil halkinin önünden kovdugu uluslarin yaptigi bütün igrençlikleri yaptilar.
\par 25 Rehavam'in kralliginin besinci yilinda Misir Krali Sisak Yerusalim'e saldirdi.
\par 26 Süleyman'in yaptirmis oldugu altin kalkanlar dahil RAB'bin Tapinagi'nin ve sarayin bütün hazinelerini bosaltip götürdü.
\par 27 Kral Rehavam bunlarin yerine tunç* kalkanlar yaptirarak sarayin kapi muhafizlarinin komutanlarina emanet etti.
\par 28 Kral RAB'bin Tapinagi'na her gittiginde, muhafizlar bu kalkanlari tasir, sonra muhafiz odasina götürürlerdi.
\par 29 Rehavam'in kralligi dönemindeki öteki olaylar ve bütün yaptiklari Yahuda krallarinin tarihinde yazilidir.
\par 30 Rehavam'la Yarovam arasinda sürekli savas vardi.
\par 31 Rehavam ölüp atalarina kavusunca, Davut Kenti'nde atalarinin yanina gömüldü. Annesi Ammonlu Naama'ydi. Rehavam'in yerine oglu Aviyam kral oldu.

\chapter{15}

\par 1 Nevat oglu Israil Krali Yarovam'in kralliginin on sekizinci yilinda Aviyam Yahuda Krali oldu.
\par 2 Yerusalim'de üç yil krallik yapti. Annesi Avsalom'un kizi Maaka'ydi.
\par 3 Babasinin kendisinden önce islemis oldugu bütün günahlara Aviyam da katildi. Bütün yüregini Tanrisi RAB'be adayan atasi Davut gibi degildi.
\par 4 Buna karsin Tanrisi RAB, Davut'un hatirina Yerusalim'i güçlendirmek için kendisinden sonra oglunu kral atayarak ona Yerusalim'de bir isik verdi.
\par 5 Çünkü RAB'bin gözünde dogru olani yapan Davut, Hititli* Uriya olayi disinda, yasami boyunca RAB'bin buyruklarinin hiçbirinden sapmamisti.
\par 6 Rehavam'la Yarovam arasindaki savas Aviyam'in yasami boyunca sürüp gitti.
\par 7 Aviyam'in kralligi dönemindeki öteki olaylar ve bütün yaptiklari Yahuda krallarinin tarihinde yazilidir. Aviyam'la Yarovam arasindaki savas sürüp gitti.
\par 8 Aviyam ölüp atalarina kavusunca, Davut Kenti'nde gömüldü, yerine oglu Asa kral oldu.
\par 9 Israil Krali Yarovam'in kralliginin yirminci yilinda Asa Yahuda Krali oldu.
\par 10 Yerusalim'de kirk bir yil krallik yapti. Büyükannesi Avsalom'un kizi Maaka'ydi.
\par 11 Atasi Davut gibi RAB'bin gözünde dogru olani yapan Asa,
\par 12 putperest törenlerinde fuhus yapan kadin ve erkekleri ülkeden kovdu. Atalarinin yapmis oldugu bütün putlari yok etti.
\par 13 Kral Asa annesi Maaka'nin kraliçeligini elinden aldi. Çünkü o Asera* için igrenç bir put yaptirmisti. Asa bu igrenç putu kesip Kidron Vadisi'nde yakti.
\par 14 Ancak puta tapilan yerleri kaldirmadi. Ama yasami boyunca yüregini RAB'be adadi.
\par 15 Babasinin ve kendisinin adadigi altini, gümüsü ve esyalari RAB'bin Tapinagi'na getirdi.
\par 16 Asa'yla Israil Krali Baasa arasindaki savas yasamlari boyunca sürüp gitti.
\par 17 Israil Krali Baasa Yahuda'ya saldirmaya hazirlaniyordu. Yahuda Krali Asa'nin topraklarina giris çikisi engellemek amaciyla, Rama Kenti'ni güçlendirmeye basladi.
\par 18 Bunun üzerine Asa, Sam'da oturan Hezyon oglu Tavrimmon oglu Aram Krali Ben-Hadat'a, RAB'bin Tapinagi'nin ve sarayin hazinelerindeki bütün altin ve gümüsü görevlileri araciligiyla su haberle birlikte gönderdi:
\par 19 "Babamla baban arasinda oldugu gibi seninle benim aramizda da bir antlasma olsun. Sana armagan olarak gönderdigim bu altinlara, gümüslere karsilik, sen de Israil Krali Baasa ile yaptigin antlasmayi boz, topraklarimdan askerlerini çeksin."
\par 20 Kral Asa'nin önerisini kabul eden Ben-Hadat, komutanlarini Israil kentlerinin üzerine gönderdi. Iyon'u, Dan'i, Avel-Beytmaaka'yi ve bütün Naftali bölgesiyle birlikte Kinrot'u ele geçirdi.
\par 21 Baasa bunu duyunca Rama'nin yapimini durdurup Tirsa'ya çekildi.
\par 22 Kral Asa istisnasiz bütün Yahudalilar'i kapsayan bir çagri yapti. Baasa'nin Rama'nin yapiminda kullandigi taslarla keresteleri alip götürdüler. Kral Asa bunlarla Benyamin bölgesindeki Geva ve Mispa kentlerini onardi.
\par 23 Asa'nin kralligi dönemindeki öteki olaylar, basarilari, bütün yaptiklari ve kurdugu kentler Yahuda krallarinin tarihinde yazilidir. Yasliliginda ayaklarindan hastalanan Asa, ölüp atalarina kavusunca, atasi Davut'un Kenti'nde atalarinin yanina gömüldü; yerine oglu Yehosafat kral oldu.
\par 25 Yahuda Krali Asa'nin kralliginin ikinci yilinda Yarovam oglu Nadav Israil Krali oldu ve Israil'de iki yil krallik yapti.
\par 26 O da RAB'in gözünde kötü olani yapti. Babasinin yolunu izledi ve babasinin Israil'i sürükledigi günahlara katildi.
\par 27 Nadav ve Israil ordusu Filistliler'in Gibbeton Kenti'ni kusatirken, Issakar oymagindan Ahiya oglu Baasa, Nadav'a düzen kurup onu Gibbeton'da öldürdü.
\par 28 Yahuda Krali Asa'nin kralliginin üçüncü yilinda Nadav'i öldüren Baasa, onun yerine kral oldu.
\par 29 Baasa kral olur olmaz, Yarovam'in bütün ailesini ortadan kaldirdi. RAB'bin, kulu Silolu Ahiya araciligiyla söyledigi söz uyarinca, Yarovam'in bütün ailesi yok edildi; sag kalan olmadi.
\par 30 Bütün bunlar Israil'in Tanrisi RAB'bi öfkelendiren Yarovam'in isledigi ve Israil'i sürükledigi günahlar yüzünden oldu.
\par 31 Nadav'in kralligi dönemindeki öteki olaylar ve bütün yaptiklari Israil krallarinin tarihinde yazilidir.
\par 32 Yahuda Krali Asa ile Israil Krali Baasa arasindaki savas yasamlari boyunca sürüp gitti.
\par 33 Yahuda Krali Asa'nin kralliginin üçüncü yilinda Ahiya oglu Baasa Tirsa'da bütün Israil'in Krali oldu ve yirmi dört yil krallik yapti.
\par 34 Baasa, RAB'bin gözünde kötü olani yapti. Yarovam'in yolunu izledi ve onun Israil'i sürükledigi günahlara katildi.

\chapter{16}

\par 1 RAB Hanani oglu Yehu araciligiyla Israil Krali Baasa'ya sunlari bildirdi:
\par 2 "Sen önemsiz biriyken, ben seni halkim Israil'e önder yaptim. Ama sen Yarovam'in yolunu izleyip halkim Israil'i günaha sürükledin. Günahlariniz beni öfkelendirdi.
\par 3 Onun için Nevat oglu Yarovam'a yaptigim gibi, senin ve ailenin kökünü kurutacagim.
\par 4 Baasa'nin ailesinden kentte ölenleri köpekler, kirda ölenleri yirtici kuslar yiyecek."
\par 5 Baasa'nin kralligi dönemindeki öteki olaylar, yaptiklari ve basarilari Israil krallarinin tarihinde yazilidir.
\par 6 Baasa ölüp atalarina kavusunca, Tirsa'da gömüldü ve yerine oglu Ela kral oldu.
\par 7 Baasa'nin RAB'bin gözünde yaptigi her kötülükten ötürü RAB, Hanani oglu Peygamber Yehu araciligiyla ona ve ailesine karsi yargisini bildirdi. Baasa yaptigi kötülüklerle RAB'bi öfkelendirmis, Yarovam'in ailesine benzemis ve bu aileyi ortadan kaldirmisti.
\par 8 Yahuda Krali Asa'nin kralliginin yirmi altinci yilinda Baasa oglu Ela Tirsa'da Israil Krali oldu ve iki yil krallik yapti.
\par 9 Savas arabalarinin yarisina komuta eden Zimri adindaki bir görevlisi ona düzen kurdu. Ela o sirada Tirsa'da sarayinin sorumlusu Arsa'nin evinde içip sarhos olmustu.
\par 10 Zimri içeri girip Ela'yi öldürdü ve onun yerine kral oldu. Yahuda Krali Asa'nin kralliginin yirmi yedinci yiliydi.
\par 11 Zimri Israil Krali olup tahta geçince, Baasa'nin bütün ailesini yok etti. Dost ve akrabalarindan hiçbir erkegi sag birakmadi.
\par 12 Baasa'yla oglu Ela, degersiz putlara taptiklari için Israil'in Tanrisi RAB'bi öfkelendirmislerdi. Onlarin isledigi ve Israil'i sürükledikleri günahlardan dolayi RAB'bin Peygamber Yehu araciligiyla Baasa'ya karsi söyledigi söz uyarinca, Zimri Baasa'nin bütün ailesini ortadan kaldirdi.
\par 14 Ela'nin kralligi dönemindeki öteki olaylar ve bütün yaptiklari Israil krallarinin tarihinde yazilidir.
\par 15 Yahuda Krali Asa'nin kralliginin yirmi yedinci yilinda Zimri Tirsa'da yedi gün krallik yapti. Israil ordusu Filistliler'in Gibbeton Kenti yakinlarinda ordugah kurmustu.
\par 16 Ordugahta bulunan Israilliler, Zimri'nin düzen kurup krali öldürdügünü duyunca, ordu komutani Omri'yi o gün orada Israil Krali yaptilar.
\par 17 Omri ve yanindaki bütün Israilliler Gibbeton'dan çikip Tirsa'yi kusattilar.
\par 18 Zimri kentin alindigini görünce, sarayin kalesine girip sarayi atese verdi ve orada öldü.
\par 19 Çünkü o RAB'bin gözünde kötü olani yapmis, Yarovam'in yolunu izlemis, onun isledigi ve Israil'i sürükledigi günahlara katilmisti.
\par 20 Zimri'nin kralligi dönemindeki öteki olaylar ve krala kurdugu düzen Israil krallarinin tarihinde yazilidir.
\par 21 Israil halki ikiye bölündü. Halkin yarisi Ginat'in oglu Tivni'yi kral yapmak isterken, öbür yarisi Omri'yi destekliyordu.
\par 22 Sonunda Omri'yi destekleyenler Ginat oglu Tivni'yi destekleyenlerden daha güçlü çikti. Tivni öldü, Omri kral oldu.
\par 23 Yahuda Krali Asa'nin kralliginin otuz birinci yilinda Omri Israil Krali oldu ve alti yili Tirsa'da olmak üzere toplam on iki yil krallik yapti.
\par 24 Omri, Semer adli birinden Samiriye Tepesi'ni iki talant gümüse satin alip üstüne bir kent yaptirdi. Tepenin eski sahibi Semer'in adindan dolayi kente Samiriye adini verdi.
\par 25 RAB'bin gözünde kötü olani yapan Omri, kendisinden önceki bütün krallardan daha çok kötülük yapti.
\par 26 Nevat oglu Yarovam'in bütün yollarini izledi ve onun Israil'i sürükledigi günahlara katilip degersiz putlara taparak Israil'in Tanrisi RAB'bi öfkelendirdi.
\par 27 Omri'nin kralligi dönemindeki öteki olaylar, yaptiklari ve basarilari Israil krallarinin tarihinde yazilidir.
\par 28 Omri ölüp atalarina kavusunca, Samiriye'de gömüldü ve yerine oglu Ahav kral oldu.
\par 29 Yahuda Krali Asa'nin kralliginin otuz sekizinci yilinda Omri oglu Ahav Israil Krali oldu ve Samiriye'de yirmi iki yil krallik yapti.
\par 30 RAB'bin gözünde kötü olani yapan Omri oglu Ahav, kendisinden önceki bütün krallardan daha çok kötülük yapti.
\par 31 Nevat oglu Yarovam'in günahlarini izlemek yetmezmis gibi, bir de Sayda Krali Etbaal'in kizi Izebel'le evlendi. Gidip Baal'a* hizmet ederek ona tapti.
\par 32 Baal için Samiriye'de yaptirdigi tapinagin içine bir sunak kurdu.
\par 33 Ayrica bir Asera* putu yaptirdi. Ahav Israil'in Tanrisi RAB'bi kendisinden önceki bütün Israil krallarindan daha çok öfkelendirdi.
\par 34 Ahav'in kralligi döneminde, Beytelli Hiel Eriha Kenti'ni yeniden insa etti. RAB'bin Nun oglu Yesu araciligiyla söyledigi söz uyarinca, Hiel ilk oglu Aviram'i kaybetme pahasina kentin temelini atti; en küçük oglu Seguv'u kaybetme pahasina da kentin kapilarini takti. Ilyas ve Kuraklik

\chapter{17}

\par 1 Gilat'in Tisbe Kenti'nden olan Ilyas, Ahav'a söyle dedi: "Hizmet ettigim Israil'in Tanrisi yasayan RAB'bin adiyla derim ki, ben söylemedikçe önümüzdeki yillarda ne yagmur yagacak, ne de çiy düsecek."
\par 2 O zaman RAB, Ilyas'a söyle seslendi:
\par 3 "Buradan ayril, doguya git. Seria Irmagi'nin dogusundaki Kerit Vadisi'nde gizlen.
\par 4 Dereden su içeceksin ve buyruk verdigim kargalarin getirdiklerini yiyeceksin."
\par 5 RAB'bin söylediklerini yapan Ilyas, gidip Seria Irmagi'nin dogusundaki Kerit Vadisi'ne yerlesti.
\par 6 Dereden su içiyor, kargalarin sabah aksam getirdigi et ve ekmekle besleniyordu.
\par 7 Ancak ülkede yagmur yagmadigi için bir süre sonra dere kurudu. Ilyas ve Sarefatli Dul Kadin
\par 8 O zaman RAB, Ilyas'a,
\par 9 "Simdi kalk git, Sayda yakinlarindaki Sarefat Kenti'ne yerles" dedi, "Orada sana yiyecek saglamasi için dul bir kadina buyruk verdim."
\par 10 Sarefat'a giden Ilyas kentin kapisina varinca, orada dul bir kadinin odun topladigini gördü. Kadina: "Bana içmek için biraz su verebilir misin?" dedi.
\par 11 Kadin su getirmeye giderken Ilyas yine seslendi: "Lütfen bir parça da ekmek getir."
\par 12 Kadin, "Senin Tanrin yasayan RAB'bin adiyla ant içerim, hiç ekmegim yok" diye karsilik verdi, "Yalniz küpte bir avuç un, çömlegin dibinde de azicik yag var. Görüyorsun, bir iki parça odun topluyorum. Götürüp oglumla kendim için bir seyler hazirlayacagim. Belki de son yemegimiz olacak, ölüp gidecegiz."
\par 13 Ilyas kadina, "Korkma, git yiyecegini hazirla" dedi, "Yalniz önce bana küçük bir pide yapip getir. Sonra oglunla kendin için yaparsin.
\par 14 Israil'in Tanrisi RAB diyor ki, 'Topraga yagmur düsünceye dek küpten un, çömlekten yag eksilmeyecek."
\par 15 Kadin gidip Ilyas'in söylediklerini yapti. Hep birlikte günlerce yiyip içtiler.
\par 16 RAB'bin Ilyas araciligiyla söyledigi söz uyarinca, küpten un, çömlekten yag eksilmedi.
\par 17 Bir süre sonra ev sahibi dul kadinin oglu gittikçe agirlasan kötü bir hastaliga yakalandi, sonunda öldü.
\par 18 Kadin Ilyas'a, "Ey Tanri adami, alip veremedigimiz nedir?" dedi, "Günahlarimi Tanri'ya animsatip oglumun ölümüne neden olmak için mi buraya geldin?"
\par 19 Ilyas, "Oglunu bana ver" diyerek çocugu kadinin kucagindan aldi, kaldigi yukari odaya çikardi ve yatagina yatirdi.
\par 20 Sonra RAB'be söyle yalvardi: "Ya RAB Tanrim, neden yaninda kaldigim dul kadinin oglunu öldürerek ona bu kötülügü yaptin?"
\par 21 Ilyas üç kez çocugun üzerine kapanip RAB'be söyle dua etti: "Ya RAB Tanrim, bu çocuga yeniden can ver."
\par 22 RAB Ilyas'in yalvarisini duydu. Çocuk dirilip yeniden yasama döndü.
\par 23 Ilyas çocugu yukari odadan indirip annesine verirken, "Iste oglun yasiyor!" dedi.
\par 24 Bunun üzerine kadin, "Simdi anladim ki, sen Tanri adamisin ve söyledigin söz gerçekten RAB'bin sözüdür" dedi.

\chapter{18}

\par 1 Uzun bir süre sonra kurakligin üçüncü yilinda RAB Ilyas'a, "Git, Ahav'in huzuruna çik" dedi, "Topragi yagmursuz birakmayacagim."
\par 2 Ilyas Ahav'in huzuruna çikmaya gitti. Samiriye'de kitlik siddetlenmisti.
\par 3 Ahav sarayinin sorumlusu Ovadya'yi çagirdi. -Ovadya RAB'den çok korkardi.
\par 4 Izebel RAB'bin peygamberlerini öldürdügünde, Ovadya yüz peygamberi yanina alip elliser elliser magaralara gizlemis ve yiyecek, içecek gereksinimlerini karsilamisti.-
\par 5 Ahav, Ovadya'ya, "Haydi gidip ülkedeki bütün su kaynaklariyla vadilere bakalim" dedi, "Belki atlarla katirlarin yasamasini saglayacak kadar ot buluruz da onlari ölüme terk etmemis oluruz."
\par 6 Ahav'la Ovadya, arastirma yapmak üzere ülkeyi aralarinda bölüstükten sonra, her biri yalniz basina bir yöne gitti.
\par 7 Ovadya giderken yolda Ilyas'la karsilasti. Ilyas'i taniyinca yüzüstü yere kapanarak, "Efendim Ilyas sen misin?" diye sordu.
\par 8 Ilyas, "Evet, benim. Git efendine, 'Ilyas burada de" diye karsilik verdi.
\par 9 Ovadya, "Ne günah isledim ki, beni öldürsün diye Ahav'a gönderiyorsun?" dedi ve ekledi:
\par 10 "Tanrin yasayan RAB'bin adiyla derim ki, efendimin seni aramak için adam göndermedigi ulus ve krallik kalmadi. Ahav ülkelerinde olmadigini söyleyen herkese, seni bulamadiklarina dair ant içirdi.
\par 11 Oysa sen simdi, 'Git, efendine Ilyas burada de diyorsun.
\par 12 Ben senin yanindan ayrildigimda, RAB'bin Ruhu seni bilmedigim bir yere götürebilir. Durumu Ahav'a bildirince, gelip seni bulamazsa beni öldürür. Ben kulun gençligimden beri RAB'den korkan biriyim.
\par 13 Efendim, Izebel RAB'bin peygamberlerini öldürdügünde yaptiklarimi duymadin mi? RAB'bin peygamberlerinden yüzünü elliser elliser iki magaraya saklayip onlarin yiyecek, içecek gereksinimlerini karsiladim.
\par 14 Ama sen simdi, 'Git, efendine Ilyas burada de diyorsun. O zaman beni öldürür!"
\par 15 Ilyas söyle karsilik verdi: "Hizmetinde bulundugum yasayan ve Her Seye Egemen RAB'bin adiyla diyorum, bugün Ahav'in huzuruna çikacagim."
\par 16 Ovadya gidip Ahav'i gördü, ona durumu anlatti. Bunun üzerine Ahav Ilyas'i karsilamaya gitti.
\par 17 Ilyas'i görünce, "Ey Israil'i sikintiya sokan adam, sen misin?" diye sordu.
\par 18 Ilyas, "Israil'i sikintiya sokan ben degilim, seninle babanin ailesi Israil'i sikintiya soktunuz" diye karsilik verdi, "RAB'bin buyruklarini terk edip Baallar'in* ardinca gittiniz.
\par 19 Simdi haber sal: Bütün Israil halki, Izebel'in sofrasinda yiyip içen Baal'in dört yüz elli peygamberi ve Asera'nin* dört yüz peygamberi Karmel Dagi'na gelip önümde toplansin."
\par 20 Ahav bütün Israil'e haber salarak peygamberlerin Karmel Dagi'nda toplanmalarini sagladi.
\par 21 Ilyas halka dogru ilerleyip, "Daha ne zamana kadar böyle iki taraf arasinda dalgalanacaksiniz?" dedi, "Eger RAB Tanri'ysa, onu izleyin; yok eger Baal Tanri'ysa, onun ardinca gidin." Halk Ilyas'a hiç karsilik vermedi.
\par 22 Ilyas konusmasini söyle sürdürdü: "RAB'bin peygamberi olarak sadece ben kaldim. Ama Baal'in dört yüz elli peygamberi var.
\par 23 Bize iki boga getirin. Birini Baal'in peygamberleri alip kessinler, parçalayip odunlarin üzerine koysunlar; ama odunlari yakmasinlar. Öbür bogayi da ben kesip hazirlayacagim ve odunlarin üzerine koyacagim; ama odunlari yakmayacagim.
\par 24 Sonra siz kendi ilahiniza yalvarin, ben de RAB'be yalvarayim. Hangisi atesle karsilik verirse, Tanri odur." Bütün halk, "Peki, öyle olsun" dedi.
\par 25 Ilyas, Baal'in peygamberlerine, "Kalabalik oldugunuz için önce siz bogalardan birini seçip hazirlayin ve ilahiniza yalvarin" dedi, "Ama ates yakmayin."
\par 26 Kendilerine verilen bogayi alip hazirlayan Baal'in peygamberleri sabahtan öglene kadar, "Ey Baal, bize karsilik ver!" diye yalvardilar. Ama ne bir ses vardi, ne de bir karsilik. Yaptiklari sunagin çevresinde ziplayip oynadilar.
\par 27 Ögleyin Ilyas onlarla alay etmeye basladi: "Bagirin, yüksek sesle bagirin! O tanriymis. Belki dalgindir, ya da heladadir, belki de yolculuk yapiyor! Yahut uyuyordur da uyandirmak gerekir!"
\par 28 Böylece yüksek sesle bagirdilar. Adetleri uyarinca, kiliç ve mizraklarla kanlarini akitincaya dek bedenlerini yaraladilar.
\par 29 Öglenden aksam sunusu saatine kadar kivrandilar. Ama hâlâ ne bir ses, ne ilgi, ne de bir karsilik vardi.
\par 30 O zaman Ilyas bütün halka, "Bana yaklasin" dedi. Herkes onun çevresinde toplandi. Ilyas RAB'bin yikilan sunagini onarmaya basladi.
\par 31 On iki tas aldi. Bu sayi RAB'bin Yakup'a, "Senin adin Israil olacak" diye bildirdigi Yakupogullari oymaklarinin sayisi kadardi.
\par 32 Ilyas bu taslarla RAB'bin adina bir sunak yaptirdi. Çevresine de iki sea tohum alacak kadar bir hendek kazdi.
\par 33 Sunagin üzerine odunlari dizdi, bogayi parça parça kesip odunlarin üzerine yerlestirdi. "Dört küp su doldurup yakmalik sunuyla* odunlarin üzerine dökün" dedi.
\par 34 Sonra, "Bir daha yapin" dedi. Bir daha yaptilar. "Bir kez daha yapin" dedi. Üçüncü kez ayni seyi yaptilar.
\par 35 O zaman sunagin çevresine akan su hendegi doldurdu.
\par 36 Aksam sunusu saatinde, Peygamber Ilyas sunaga yaklasip söyle dua etti: "Ey Ibrahim'in, Ishak'in ve Israil'in Tanrisi olan RAB! Bugün bilinsin ki, sen Israil'in Tanrisi'sin, ben de senin kulunum ve bütün bunlari senin buyruklarinla yaptim.
\par 37 Ya RAB, bana yanit ver! Yanit ver ki, bu halk senin Tanri oldugunu anlasin. Onlarin yine sana dönmelerini sagla."
\par 38 O anda gökten RAB'bin atesi düstü. Düsen ates yakmalik sunuyu, odunlari, taslari ve topragi yakip hendekteki suyu kuruttu.
\par 39 Halk olanlari görünce yüzüstü yere kapandi. "RAB Tanri'dir, RAB Tanri'dir!" dediler.
\par 40 Ilyas, "Baal'in peygamberlerini yakalayin, hiçbirini kaçirmayin" diye onlara buyruk verdi. Peygamberler yakalandi, Ilyas onlari Kison Vadisi'ne götürüp orada öldürdü.
\par 41 Sonra Ilyas, Ahav'a, "Git, yemene içmene bak; çünkü güçlü bir yagmur sesi var" dedi.
\par 42 Ahav yiyip içmek üzere oradan ayrilinca, Ilyas Karmel Dagi'nin tepesine çikti. Yere kapanarak basini dizlerinin arasina koydu.
\par 43 Sonra usagina, "Haydi git, denize dogru bak!" dedi. Usagi gidip denize bakti ve, "Hiçbir sey görmedim" diye karsilik verdi. Ilyas, usagina yedi kez, "Git, bak" dedi.
\par 44 Yedinci kez gidip bakan usak, "Denizden avuç kadar küçük bir bulut çikiyor" dedi. Ilyas söyle dedi: "Git, Ahav'a, 'Yagmura yakalanmadan arabani al ve geri dön de."
\par 45 Tam o sirada gökyüzü bulutlarla karardi, rüzgar çikti, siddetli bir yagmur basladi. Ahav hemen arabasina binip Yizreel'e gitti.
\par 46 Üzerine RAB'bin gücü inen Ilyas kemerini kusanip Yizreel'e kadar Ahav'in önünde kostu.

\chapter{19}

\par 1 Ahav, Ilyas'in bütün yaptiklarini, peygamberleri nasil kiliçtan geçirdigini Izebel'e anlatti.
\par 2 Izebel, Ilyas'a, "Yarin bu saate kadar senin peygamberlere yaptigini ben de sana yapmazsam, ilahlar bana aynisini, hatta daha kötüsünü yapsin" diye haber gönderdi.
\par 3 Ilyas can korkusuyla Yahuda'nin Beer-Seva Kenti'ne kaçip usagini orada birakti.
\par 4 Bir gün boyunca çölde yürüdü, sonunda bir retem çalisinin altina oturdu ve ölmek için dua etti: "Ya RAB, yeter artik, canimi al, ben atalarimdan daha iyi degilim."
\par 5 Sonra retem çalisinin altina yatip uykuya daldi. Ansizin bir melek ona dokunarak, "Kalk yemek ye" dedi.
\par 6 Ilyas çevresine bakinca yanibasinda, kizgin taslarin üstünde bir pideyle bir testi su gördü. Yiyip içtikten sonra yine uzandi.
\par 7 RAB'bin melegi ikinci kez geldi, ona dokunarak, "Kalk yemegini ye. Gidecegin yol çok uzun" dedi.
\par 8 Ilyas kalkti, yiyip içti. Yediklerinden aldigi güçle kirk gün kirk gece Tanri Dagi Horev'e kadar yürüdü.
\par 9 Geceyi orada bulunan bir magarada geçirdi. RAB, "Burada ne yapiyorsun, Ilyas?" diye sordu.
\par 10 Ilyas, "RAB'be, Her Seye Egemen Tanri'ya büyük bir istekle kulluk ettim" diye karsilik verdi, "Ama Israil halki senin antlasmani reddetti, sunaklarini yikti ve peygamberlerini kiliçtan geçirdi. Yalniz ben kaldim. Beni de öldürmeye çalisiyorlar."
\par 11 RAB, "Daga çik ve önümde dur, yanindan geçecegim" dedi. RAB'bin önünde çok güçlü bir rüzgar daglari yarip kayalari parçaladi. Ancak RAB rüzgarin içinde degildi. Rüzgarin ardindan bir deprem oldu, RAB depremin içinde de degildi.
\par 12 Depremden sonra bir ates çikti, ancak RAB atesin içinde de degildi. Atesten sonra ince, yumusak bir ses duyuldu.
\par 13 Ilyas bu sesi duyunca, cüppesiyle yüzünü örttü, çikip magaranin girisinde durdu. O sirada bir ses, "Burada ne yapiyorsun, Ilyas?" dedi.
\par 14 Ilyas, "RAB'be, Her Seye Egemen Tanri'ya büyük bir istekle kulluk ettim" diye karsilik verdi, "Ama Israil halki senin antlasmani reddetti, sunaklarini yikti ve peygamberlerini kiliçtan geçirdi. Yalniz ben kaldim. Beni de öldürmeye çalisiyorlar."
\par 15 RAB, "Geldigin yoldan geri dön, Sam yakinindaki kirlara git" dedi, "Oraya vardiginda, Hazael'i Aram Krali olarak, Nimsi oglu Yehu'yu Israil Krali olarak, Avel-Meholali Safat'in oglu Elisa'yi da kendi yerine peygamber olarak meshedeceksin*.
\par 17 Hazael'in kilicindan kurtulani Yehu, Yehu'nun kilicindan kurtulani Elisa öldürecek.
\par 18 Ancak Israil'de Baal'in* önünde diz çöküp onu öpmemis yedi bin kisiyi sag birakacagim."
\par 19 Ilyas oradan ayrilip gitti, Safat oglu Elisa'yi buldu. Elisa, on iki çift öküzle saban sürenlerin ardindan on ikinci çifti sürüyordu. Ilyas Elisa'nin yanindan geçerek kendi cüppesini onun üzerine atti.
\par 20 Elisa öküzleri birakip Ilyas'in ardindan kostu ve, "Izin ver, annemle babami öpeyim, sonra seninle geleyim" dedi. Ilyas, "Geri dön, ben sana ne yaptim ki?" diye karsilik verdi.
\par 21 Böylece Elisa gidip sürdügü çiftin öküzlerini kesti. Boyunduruklariyla ates yakip etleri pisirdikten sonra, yesinler diye halka dagitti. Sonra, Ilyas'in ardindan gidip ona hizmet etti.

\chapter{20}

\par 1 Aram Krali Ben-Hadat bütün ordusunu topladi. Atlari, savas arabalari ve kendisini destekleyen otuz iki kralla birlikte Samiriye'nin üzerine yürüyerek kenti kusatti.
\par 2 Ben-Hadat, kentte bulunan Israil Krali Ahav'a haberciler göndererek söyle buyruk verdi:
\par 3 "Ben-Hadat diyor ki, 'Altinini, gümüsünü, karilarini ve en gürbüz çocuklarini bana teslim et."
\par 4 Israil Krali, "Efendim kralin dediklerini kabul ediyorum" diye karsilik verdi, "Beni ve sahip oldugum her seyi alabilirsin."
\par 5 Haberciler yine gelip Ahav'a söyle dediler: "Ben-Hadat diyor ki, 'Sana altinini, gümüsünü, karilarini ve çocuklarini bana vereceksin diye haber göndermistim.
\par 6 Ayrica yarin bu saatlerde sarayinda ve görevlilerinin evlerinde arama yapmak üzere kendi görevlilerimi gönderecegim. Degerli olan her seyini alip getirecekler."
\par 7 Israil Krali ülkenin bütün ileri gelenlerini toplayarak, "Bakin, bu adam nasil bela ariyor!" dedi, "Bana haber gönderip altinimi, gümüsümü, karilarimi, çocuklarimi istedi, reddetmedim."
\par 8 Bütün ileri gelenler ve halk, "Onu dinleme, isteklerini de kabul etme" diye karsilik verdiler.
\par 9 Böylece Ahav, Ben-Hadat'in habercilerine, "Efendimiz krala ilk isteklerinin hepsini kabul edecegimi, ama ikincisini kabul edemeyecegimi söyleyin" dedi. Haberciler gidip Ben-Hadat'a durumu bildirdiler.
\par 10 O zaman Ben-Hadat Ahav'a baska bir haber gönderdi: "O kadar çok adamla senin üstüne yürüyecegim ki, Samiriye'yi yerle bir edecegim. Kentin tozlari askerlerimin avuçlarini bile dolduramayacak. Eger bunu yapmazsam, ilahlar bana aynisini, hatta daha kötüsünü yapsin!"
\par 11 Israil Krali söyle karsilik verdi: "Kraliniza deyin ki, 'Zirhini kusanmadan önce degil, kusandiktan sonra övünsün."
\par 12 Ben-Hadat bunu duydugunda, kendisini destekleyen krallarla birlikte çadirda içki içiyordu. Hemen adamlarina buyruk verdi: "Saldiriya hazirlanin." Böylece Samiriye'ye karsi saldiri hazirliklarina giristiler.
\par 13 O sirada bir peygamber gelip Israil Krali Ahav'a söyle dedi: "RAB diyor ki, 'Bu büyük orduyu görüyor musun? Onlari bugün senin eline teslim edecegim. O zaman benim RAB oldugumu anlayacaksin."
\par 14 Ahav, "Kimin araciligiyla olacak bu?" diye sordu. Peygamber su karsiligi verdi: "RAB diyor ki, 'Ilçe komutanlarinin genç askerleri bunu basaracak." Ahav, "Savasa kim baslayacak?" diye sordu. Peygamber, "Sen baslayacaksin" dedi.
\par 15 Ahav ilçe komutanlarinin genç askerlerini çagirip saydi. Iki yüz otuz iki kisiydiler. Sonra bütün Israil ordusunu toplayip saydi, onlar da yedi bin kisiydiler.
\par 16 Ögleyin Ben-Hadat ile kendisini destekleyen otuz iki kral çadirlarda içip sarhos olmusken Israil saldirisi basladi.
\par 17 Önce genç askerler saldiriya geçti. Ben-Hadat'in gönderdigi gözcüler, "Samiriyeliler geliyor" diye ona haber getirdiler.
\par 18 Ben-Hadat, "Ister baris, ister savas için gelsinler, onlari canli yakalayin" dedi.
\par 19 Genç askerler arkalarindaki Israil ordusuyla birlikte kentten çikip saldiriya geçtiler.
\par 20 Herkes önüne geleni öldürdü. Aramlilar kaçmaya baslayinca, Israilliler peslerine düstü. Ama Aram Krali Ben-Hadat, atina binerek atlilarla birlikte kaçip kurtuldu.
\par 21 Israil Krali atlarla savas arabalarina büyük zararlar vererek Aramlilar'i agir bir yenilgiye ugratti.
\par 22 Daha sonra peygamber gelip Israil Krali'na, "Git, gücünü pekistir ve neler yapman gerektigini iyi düsün" dedi, "Çünkü önümüzdeki ilkbaharda Aram Krali sana yine saldiracak."
\par 23 Bu arada görevlileri Aram Krali'nin kendisine, "Israil'in ilahi dag ilahidir" dediler, "Bu nedenle bizden güçlü çiktilar. Ama ovada savasirsak, onlari kesinlikle yeneriz.
\par 24 Simdi bütün krallari görevlerinden al, onlarin yerine yeni komutanlar ata.
\par 25 Kaybettigin kadar at ve savas arabasi toplayarak kendine yeni bir ordu kur. Israilliler'le ovada savasalim. O zaman onlari kesinlikle yeneriz." Aram Krali Ben-Hadat bütün söylenenleri kabul edip yerine getirdi.
\par 26 Ilkbaharda Aramlilar'i toplayip Israilliler'le savasmak üzere Afek Kenti'ne gitti.
\par 27 Israil halki da toplanip yiyecegini hazirladi. Aramlilar'la savasmak üzere yola çikip onlarin karsisina ordugah kurdu. Ülkeyi dolduran Aramlilar'in karsisinda Israilliler iki küçük oglak sürüsü gibi kaliyordu.
\par 28 Bir Tanri adami gidip Israil Krali Ahav'a söyle dedi: "RAB diyor ki, 'Aramlilar, RAB daglarin Tanrisi'dir, ovalarin degil, dedikleri için bu güçlü ordunun tümünü senin eline teslim edecegim. O zaman benim RAB oldugumu anlayacaksin."
\par 29 Birbirlerine karsi ordugah kuran Aramlilar'la Israilliler yedi gün beklediler. Yedinci gün savas basladi. Israilliler bir gün içinde yüz bin Aramli yaya asker öldürdü.
\par 30 Sag kalanlar Afek Kenti'ne kaçtilar. Orada da yirmi yedi bin kisinin üstüne surlar yikildi. Ben-Hadat kentin içine kaçip bir iç odaya saklandi.
\par 31 Görevlileri Ben-Hadat'a söyle dediler: "Duydugumuza göre, Israil krallari iyi yürekli krallarmis. Haydi bellerimize çul kusanip baslarimiza ip saralim ve Israil Krali'nin huzuruna çikalim. Belki senin canini bagislar."
\par 32 Bellerine çul kusanip baslarina da ip bagladilar ve Israil Krali'nin huzuruna çikarak, "Kulun Ben-Hadat 'Canimi bagisla diye yalvariyor" dediler. Ahav, "Ben-Hadat hâlâ yasiyor mu? O benim kardesim sayilir" diye karsilik verdi.
\par 33 Adamlar bunu olumlu bir belirti sayarak hemen sözü agzindan aldilar ve, "Evet, Ben-Hadat kardesin sayilir!" dediler. Kral, "Gidin, onu getirin" diye buyruk verdi. Ben-Hadat gelince, Ahav onu kendi savas arabasina aldi.
\par 34 Ben-Hadat, "Babamin babandan almis oldugu kentleri geri verecegim" dedi, "Babam nasil Samiriye'de çarsilar kurduysa, sen de Sam'da çarsilar kurabilirsin." Bunun üzerine Ahav, "Ben de bu sartlara dayanarak sana özgürlügünü veriyorum" dedi. Böylece onunla bir antlasma yaparak gitmesine izin verdi.
\par 35 Peygamberlerden biri, RAB'bin sözüne uyarak arkadasina, "Lütfen, beni vur!" dedi. Ama arkadasi onu vurmak istemedi.
\par 36 O zaman peygamber arkadasina söyle dedi: "Sen RAB'bin buyrugunu dinlemedigin için, yanimdan ayrilir ayrilmaz bir aslan seni öldürecek." Adam oradan ayrildiktan sonra aslan onu yakalayip öldürdü.
\par 37 Bunun üzerine ayni peygamber, baska bir adama giderek, "Lütfen beni vur!" dedi. Adam da onu vurup yaraladi.
\par 38 Peygamber gitti, kiligini degistirmek için gözlerini bagladi. Yol kenarinda kralin geçmesini beklemeye basladi.
\par 39 Kral oradan geçerken, peygamber ona söyle seslendi: "Ben kulun, tam savasin içindeyken, askerin biri bana bir tutsak getirip, 'Bu adami iyi koru dedi, 'Kaçacak olursa, karsiligini ya caninla, ya da bir talant gümüsle ödersin.
\par 40 Ama ben oraya buraya bakarken, adam kayboldu." Israil Krali, "Sen kendini yargilamis oldun" diye karsilik verdi, "Cezani çekeceksin."
\par 41 Peygamber, hemen gözlerindeki sargiyi çikardi. O zaman Israil Krali onun bir peygamber oldugunu anladi.
\par 42 Bunun üzerine peygamber krala söyle dedi: "RAB diyor ki, 'Ölüme mahkûm ettigim adami saliverdigin için onun yerine sen öleceksin. Onun halkinin basina gelecekler senin halkinin basina gelecek."
\par 43 Keyfi kaçan Israil Krali öfkeyle Samiriye'deki sarayina döndü.

\chapter{21}

\par 1 Yizreel'de Samiriye Krali Ahav'in sarayinin yaninda Yizreelli Navot'un bir bagi vardi. Bir gün Ahav, Navot'a sunu önerdi: "Bagini bana ver. Sarayima yakin oldugu için orayi sebze bahçesi olarak kullanmak istiyorum. Karsiliginda ben de sana daha iyi bir bag vereyim, ya da istersen degerini gümüs olarak ödeyeyim."
\par 3 Ama Navot, "Atalarimin bana biraktigi mirasi sana vermekten RAB beni esirgesin" diye karsilik verdi.
\par 4 "Atalarimin bana biraktigi mirasi sana vermem" diyen Yizreelli Navot'un bu sözlerine sikilip öfkelenen Ahav sarayina döndü. Asik bir yüzle yatagina uzanip hiçbir sey yemedi.
\par 5 Karisi Izebel yanina gelip, "Neden bu kadar sikiliyorsun? Neden yemek yemiyorsun?" diye sordu.
\par 6 Ahav karisina söyle karsilik verdi: "Yizreelli Navot'a, 'Sen bagini gümüs karsiliginda bana sat, istersen ben de onun yerine sana baska bir bag vereyim dedim. Ama o, 'Hayir, bagimi sana vermem dedi."
\par 7 Izebel, "Sen Israil'e böyle mi krallik yapiyorsun?" dedi, "Kalk, yemegini ye, keyfini bozma. Yizreelli Navot'un bagini sana ben verecegim."
\par 8 Izebel Ahav'in mührünü kullanarak onun adina mektuplar yazdi, Navot'un yasadigi kentin ileri gelenleriyle soylularina gönderdi.
\par 9 Mektuplarda sunlari yazdi: "Oruç* ilan edip Navot'u halkin önüne oturtun.
\par 10 Karsisina da, 'Navot Tanri'ya ve krala sövdü diyen iki yalanci tanik koyun. Sonra onu disari çikarip taslayarak öldürün."
\par 11 Navot'un yasadigi kentin ileri gelenleriyle soylulari Izebel'in gönderdigi mektuplarda yazdiklarini uyguladilar.
\par 12 Oruç ilan edip Navot'u halkin önüne oturttular.
\par 13 Sonra iki kötü adam gelip Navot'un karsisina oturdu ve halkin önünde: "Navot, Tanri'ya ve krala sövdü" diyerek yalan yere taniklik etti. Bunun üzerine onu kentin disina çikardilar ve taslayarak öldürdüler.
\par 14 Sonra Izebel'e, "Navot taslanarak öldürüldü" diye haber gönderdiler.
\par 15 Izebel, Navot'un taslanip öldürüldügünü duyar duymaz, Ahav'a, "Kalk, Yizreelli Navot'un sana gümüs karsiliginda satmak istemedigi bagini sahiplen" dedi, "Çünkü o artik yasamiyor, öldü."
\par 16 Ahav, Yizreelli Navot'un öldügünü duyunca, onun bagini almaya gitti.
\par 17 O zaman RAB, Tisbeli Ilyas'a söyle dedi:
\par 18 "Kalk, Samiriyeli Israil Krali Ahav'i karsilamaya git. Su anda Navot'un bagindadir. Orayi almaya gitti.
\par 19 Ona de ki, RAB söyle diyor: 'Hem adami öldürdün, hem de bagini aldin, degil mi? Navot'un kanini köpekler nerede yaladiysa, senin kanini da orada yalayacak."
\par 20 Ahav, Ilyas'a, "Ey düsmanim, beni buldun, degil mi?" dedi. Ilyas söyle karsilik verdi: "Evet, buldum. Çünkü sen RAB'bin gözünde kötü olani yaparak kendini sattin.
\par 21 RAB diyor ki, 'Seni sikintilara sokacak ve yok edecegim. Israil'de senin soyundan gelen genç yasli bütün erkeklerin kökünü kurutacagim.
\par 22 Beni öfkelendirip Israil'i günaha sürükledigin için senin ailen de Nevat oglu Yarovam'in ve Ahiya oglu Baasa'nin ailelerinin akibetine ugrayacak.
\par 23 "RAB Izebel için de, 'Izebel'i Yizreel Kenti'nin surlari dibinde köpekler yiyecek diyor.
\par 24 'Ahav'in ailesinden kentte ölenleri köpekler, kirda ölenleri yirtici kuslar yiyecek."
\par 25 -Ahav kadar, RAB'bin gözünde kötü olani yaparak kendini satan hiç kimse olmadi. Karisi Izebel onu her konuda kiskirtiyordu.
\par 26 Ahav RAB'bin Israil halkinin önünden kovdugu Amorlular'in her yaptigina uyarak putlarin ardinca yürüdü ve igrenç isler yapti.-
\par 27 Ahav bu sözleri dinledikten sonra, giysilerini yirtti, çula sarinip oruç tutmaya basladi. Çul içinde yatip kalkarak, alçakgönüllü bir yol tuttu.
\par 28 RAB, Tisbeli Ilyas'a söyle dedi:
\par 29 "Ahav'in önümde ne denli alçakgönüllü davrandigini gördün mü? Bu alçakgönüllülügünden ötürü yasami boyunca ben de onu sikintiya sokmayacagim. Ama oglunun zamaninda ailesine sikinti verecegim."

\chapter{22}

\par 1 Üç yil boyunca Aram ile Israil arasinda savas çikmadi.
\par 2 Üçüncü yil Yahuda Krali Yehosafat, Israil Krali'ni görmeye gitti.
\par 3 Israil Krali Ahav, görevlilerine, "Ramot-Gilat'in bize ait oldugunu bilmiyor musunuz?" dedi, "Biz onu Aram Krali'ndan geri almak için bir sey yapmadik."
\par 4 Sonra Yehosafat'a, "Ramot-Gilat'a karsi benimle birlikte savasir misin?" diye sordu. Yehosafat, "Beni kendin, halkimi halkin, atlarimi atlarin say" diye yanitladi,
\par 5 "Ama önce RAB'be danisalim" diye ekledi.
\par 6 Israil Krali dört yüz kadar peygamberi toplayip, "Ramot-Gilat'a karsi savasayim mi, yoksa vaz mi geçeyim?" diye sordu. Peygamberler, "Savas, çünkü Rab kenti senin eline teslim edecek" diye yanitladilar.
\par 7 Ama Yehosafat, "Burada danisabilecegimiz RAB'bin baska peygamberi yok mu?" diye sordu.
\par 8 Israil Krali, "Yimla oglu Mikaya adinda biri daha var" diye yanitladi, "Onun araciligiyla RAB'be danisabiliriz. Ama ben ondan nefret ederim. Çünkü benimle ilgili hiç iyi peygamberlik etmez, yalniz kötü seyler söyler." Yehosafat, "Böyle konusmaman gerekir, ey kral!" dedi.
\par 9 Israil Krali bir görevli çagirip, "Hemen Yimla oglu Mikaya'yi getir!" diye buyurdu.
\par 10 Israil Krali Ahav ile Yahuda Krali Yehosafat kral giysileriyle Samiriye Kapisi'nin girisinde, harman yerine konan tahtlarinda oturuyorlardi. Bütün peygamberler de onlarin önünde peygamberlik ediyordu.
\par 11 Kenaana oglu Sidkiya, yaptigi demir boynuzlari göstererek söyle dedi: "RAB diyor ki, 'Aramlilar'i yok edinceye dek onlari bu boynuzlarla vuracaksin."
\par 12 Öteki peygamberlerin hepsi de ayni seyi söylediler: "Ramot-Gilat'a saldir, kazanacaksin! Çünkü RAB onlari senin eline teslim edecek."
\par 13 Mikaya'yi çagirmaya giden görevli ona, "Bak! Peygamberler bir agizdan kral için olumlu seyler söylüyorlar" dedi, "Rica ederim, senin sözün de onlarinkine uygun olsun; olumlu bir sey söyle."
\par 14 Mikaya, "Yasayan RAB'bin hakki için, RAB bana ne derse onu söyleyecegim" diye karsilik verdi.
\par 15 Mikaya gelince kral, "Mikaya, Ramot-Gilat'a karsi savasa gidelim mi, yoksa vaz mi geçelim?" diye sordu. Mikaya, "Saldir, kazanacaksin! Çünkü RAB onlari senin eline teslim edecek" diye yanitladi.
\par 16 Bunun üzerine kral, "RAB'bin adina bana gerçegin disinda bir sey söylemeyecegine iliskin sana kaç kez ant içireyim?" diye sordu.
\par 17 Mikaya söyle karsilik verdi: "Israilliler'i daglara dagilmis çobansiz koyunlar gibi gördüm. RAB, 'Bunlarin sahibi yok. Herkes güvenlik içinde evine dönsün dedi."
\par 18 Israil Krali Yehosafat'a, "Benimle ilgili iyi peygamberlik etmez, hep kötü seyler söyler dememis miydim?" dedi.
\par 19 Mikaya konusmasini sürdürdü: "Öyleyse RAB'bin sözünü dinle! Gördüm ki, RAB tahtinda oturuyor, bütün göksel varliklar da saginda, solunda duruyordu.
\par 20 RAB sordu: 'Ramot-Gilat'a saldirip ölsün diye Ahav'i kim kandiracak? "Kimi söyle, kimi böyle derken,
\par 21 bir ruh çikip RAB'bin önünde durdu ve, 'Ben onu kandiracagim dedi. "RAB, 'Nasil? diye sordu.
\par 22 "Ruh, 'Aldatici ruh olarak gidip Ahav'in bütün peygamberlerine yalan söyletecegim diye karsilik verdi. "RAB, 'Onu kandirmayi basaracaksin! dedi, 'Git, dedigini yap.
\par 23 "Iste RAB bütün bu peygamberlerin agzina aldatici bir ruh koydu. Çünkü sana kötülük etmeye karar verdi."
\par 24 Kenaana oglu Sidkiya yaklasip Mikaya'nin yüzüne bir tokat atti. "RAB'bin Ruhu nasil benden çikip da seninle konustu?" dedi.
\par 25 Mikaya, "Gizlenmek için bir iç odaya girdigin gün göreceksin" diye yanitladi.
\par 26 Bunun üzerine Israil Krali, "Mikaya'yi kentin yöneticisi Amon'a ve kralin oglu Yoas'a götürün" dedi,
\par 27 "Ben güvenlik içinde dönünceye dek bu adami cezaevinde tutmalarini, ona su ve ekmekten baska bir sey vermemelerini söyleyin!"
\par 28 Mikaya, "Eger sen güvenlik içinde dönersen, RAB benim araciligimla konusmamis demektir" dedi ve, "Herkes bunu duysun!" diye ekledi.
\par 29 Israil Krali Ahav'la Yahuda Krali Yehosafat Ramot-Gilat'a saldirmak için yola çiktilar.
\par 30 Israil Krali, Yehosafat'a, "Ben kilik degistirip savasa öyle girecegim, ama sen kral giysilerini giy" dedi. Böylece Israil Krali kiligini degistirip savasa girdi.
\par 31 Aram Krali, savas arabalarinin otuz iki komutanina, "Israil Krali disinda, büyük küçük hiç kimseye saldirmayin!" diye buyruk vermisti.
\par 32 Savas arabalarinin komutanlari Yehosafat'i görünce, Israil Krali sanip saldirmak için ona döndüler. Yehosafat yakarmaya basladi.
\par 33 Komutanlar onun Israil Krali olmadigini anlayinca pesini biraktilar.
\par 34 O sirada bir asker rasgele attigi bir okla Israil Krali'ni zirhinin parçalarinin birlestigi yerden vurdu. Kral arabacisina, "Dönüp beni savas alanindan çikar, yaralandim" dedi.
\par 35 Savas o gün siddetlendi. Israil Krali, arabasinda Aramlilar'a karsi aksama kadar dayandi ve aksamleyin öldü. Yarasindan akan kanlar arabasinin içinde kaldi.
\par 36 Günes batarken ordugahta, "Herkes kendi kentine, ülkesine dönsün!" diye bagirdilar.
\par 37 Kral ölmüstü. Onu Samiriye'ye getirip orada gömdüler.
\par 38 Arabasi fahiselerin yikandigi Samiriye Havuzu'nun kenarinda temizlenirken RAB'bin sözü uyarinca köpekler kanini yaladi.
\par 39 Ahav'in kralligi dönemindeki öteki olaylar, bütün yaptiklari, yaptirdigi fildisi süslemeli saray ve bütün kentler Israil krallarinin tarihinde yazilidir.
\par 40 Ahav ölüp atalarina kavusunca yerine oglu Ahazya kral oldu.
\par 41 Israil Krali Ahav'in kralliginin dördüncü yilinda Asa oglu Yehosafat Yahuda Krali oldu.
\par 42 Yehosafat otuz bes yasinda kral oldu ve Yerusalim'de yirmi bes yil krallik yapti. Annesi Silhi'nin kizi Azuva'ydi.
\par 43 Babasi Asa'nin bütün yollarini izleyen ve bunlardan sapmayan Yehosafat RAB'bin gözünde dogru olani yapti. Ancak alisilagelen tapinma yerleri kaldirilmadi. Halk hâlâ oralarda kurban kesip buhur yakiyordu.
\par 44 Yehosafat Israil Krali ile baris yapti.
\par 45 Yehosafat'in kralligi dönemindeki öteki olaylar, basarilari ve savaslari Yahuda krallarinin tarihinde yazilidir.
\par 46 Yehosafat babasi Asa'nin döneminden kalan, putperest törenlerinde fuhus yapan kadin ve erkeklerin hepsini ülkeden süpürüp atti.
\par 47 Edom'da kral yoktu, yerine bir vekil bakiyordu.
\par 48 Yehosafat altin almak için Ofir'e gitmek üzere ticaret gemileri yaptirdi. Ancak gemiler oraya gidemeden Esyon-Gever'de parçalandi.
\par 49 O zaman Ahav oglu Ahazya, Yehosafat'a, "Benim adamlarim gemilerde seninkilerle birlikte gitsinler" dedi. Ama Yehosafat kabul etmedi.
\par 50 Yehosafat ölüp atalarina kavustu ve atasi Davut'un Kenti'nde atalarinin yanina gömüldü. Yerine oglu Yehoram kral oldu.
\par 51 Yahuda Krali Yehosafat'in kralliginin on yedinci yilinda Ahav oglu Ahazya Samiriye'de Israil Krali oldu. Iki yil krallik yapti.
\par 52 RAB'bin gözünde kötü olani yapti. Babasinin, annesinin ve Israil'i günaha sürükleyen Nevat oglu Yarovam'in yolunda yürüdü.
\par 53 Baal'a* hizmet edip tapti. Babasinin her yaptigina uyarak Israil'in Tanrisi RAB'bi öfkelendirdi.


\end{document}