\begin{document}

\title{2 Krallar}


\chapter{1}

\par 1 Israil Krali Ahav'in ölümünden sonra Moavlilar Israil'e karsi ayaklandi.
\par 2 Israil Krali Ahazya Samiriye'de yasadigi sarayin üst katindaki kafesli pencereden düsüp yaralandi. Habercilerine, "Gidin, Ekron ilahi Baalzevuv'a* danisin, yaralarimin iyilesip iyilesmeyecegini ögrenin" dedi.
\par 3 Ama RAB'bin melegi, Tisbeli Ilyas'a söyle dedi: "Kalk, Samiriye Krali'nin habercilerini karsila ve onlara de ki, 'Israil'de Tanri yok mu ki Ekron ilahi Baalzevuv'a danismaya gidiyorsunuz?
\par 4 Kraliniza deyin ki, 'RAB, Yattigin yataktan kalkamayacak, kesinlikle öleceksin! diyor." Böylece Ilyas oradan ayrildi.
\par 5 Haberciler kralin yanina döndüler. Kral, "Neden geri döndünüz?" diye sordu.
\par 6 Söyle karsilik verdiler: "Yolda bir adamla karsilastik. Bizededi ki, 'Gidin, sizi gönderen krala RAB söyle diyor deyin:Israil'de Tanri yok mu ki Ekron ilahi Baalzevuv'a danismak için haberciler gönderdin? Bu yüzden yattigin yataktan kalkamayacak, kesinlikle öleceksin!"
\par 7 Kral, "Sizi karsilayip bu sözleri söyleyen nasil bir adamdi?" diye sordu.
\par 8 "Üzerinde tüylü bir giysi, belinde deri bir kusak vardi" diye yanitladilar. Kral, "O Tisbeli Ilyas'tir" dedi.
\par 9 Sonra bir komutanla birlikte elli adamini Ilyas'a gönderdi. Komutan tepenin üstünde oturan Ilyas'in yanina çikip ona, "Ey Tanri adami, kral asagi inmeni istiyor" dedi.
\par 10 Ilyas, "Eger ben Tanri adamiysam, simdi göklerden ates yagacak ve seninle birlikte elli adamini yok edecek!" diye karsilik verdi. O anda göklerden ates yagdi, komutanla birlikte elli adamini yakip yok etti.
\par 11 Bunun üzerine kral, Ilyas'a baska bir komutanla birlikte elli adam daha gönderdi. Komutan Ilyas'a, "Ey Tanri adami, kral hemen asagi inmeni istiyor!" dedi.
\par 12 Ilyas, "Eger ben Tanri adamiysam, göklerden ates yagacak ve seninle birlikte elli adamini yok edecek!" diye karsilik verdi. O anda göklerden ates yagdi, komutanla birlikte elli adamini yakip yok etti.
\par 13 Kral üçüncü kez bir komutanla elli adam gönderdi. Üçüncü komutan çikip Ilyas'in önünde diz çöktü ve ona söyle yalvardi: "Ey Tanri adami, lütfen bana ve adamlarima aci, canimizi bagisla!
\par 14 Göklerden yagan ates daha önce gelen iki komutanla elliser adamini yakip yok etti, ama lütfen bana aci."
\par 15 RAB'bin melegi, Ilyas'a, "Onunla birlikte asagi in, korkma" dedi. Ilyas kalkip komutanla birlikte kralin yanina gitti
\par 16 ve ona söyle dedi: "RAB diyor ki, 'Israil'de danisacak Tanri yok mu ki Ekron ilahi Baalzevuv'a danismak için haberciler gönderdin? Bu yüzden yattigin yataktan kalkamayacak, kesinlikle öleceksin!"
\par 17 RAB'bin Ilyas araciligiyla söyledigi söz uyarinca Kral Ahazya öldü. Oglu olmadigi için yerine kardesi Yoram geçti. Bu olay Yahuda Krali Yehosafat oglu Yehoram'in kralliginin ikinci yilinda oldu.
\par 18 Ahazya'nin kralligi dönemindeki öteki olaylar ve yaptiklari Israil krallarinin tarihinde yazilidir.

\chapter{2}

\par 1 RAB Ilyas'i kasirgayla göklere çikarmadan önce, Ilyas ile Elisa Gilgal'dan ayrilip yola çikmislardi.
\par 2 Ilyas Elisa'ya, "Lütfen sen burada kal, çünkü RAB beni Beytel'e gönderdi" dedi. Elisa, "Yasayan RAB'bin adiyla basin üzerine ant içerim ki, senden ayrilmam" diye karsilik verdi. Böylece Beytel'e birlikte gittiler.
\par 3 Beytel'deki peygamber toplulugu Elisa'nin yanina geldi. "RAB bugün efendini senin basindan alacak, biliyor musun?" diye ona sordular. Elisa, "Evet, biliyorum, konusmayin!" diye karsilik verdi.
\par 4 Ilyas, "Elisa, lütfen burada kal, çünkü RAB beni Eriha'ya gönderdi" dedi. Elisa, "Yasayan RAB'bin adiyla basin üzerine ant içerim ki, senden ayrilmam" diye karsilik verdi. Böylece birlikte Eriha'ya gittiler.
\par 5 Eriha'daki peygamber toplulugu Elisa'nin yanina geldi. "RAB efendini bugün senin basindan alacak, biliyor musun?" diye ona sordular. Elisa, "Evet, biliyorum, konusmayin" diye karsilik verdi.
\par 6 Sonra Ilyas, "Lütfen, burada kal, çünkü RAB beni Seria Irmagi kiyisina gönderdi" dedi. Elisa, "Yasayan RAB'bin adiyla basin üzerine ant içerim ki, senden ayrilmam" diye karsilik verdi. Böylece ikisi birlikte yollarina devam etti.
\par 7 Elli peygamber de onlari Seria Irmagi'na kadar izledi. Ilyas ile Elisa Seria Irmagi'nin kiyisinda durdular. Peygamberler de biraz ötede, onlarin karsisinda durdu.
\par 8 Ilyas cüppesini dürüp sulara vurunca, sular ikiye ayrildi. Elisa ile Ilyas kuru topragin üzerinden yürüyerek karsiya geçtiler.
\par 9 Karsi yakaya geçtikten sonra Ilyas Elisa'ya, "Söyle, yanindan alinmadan önce senin için ne yapabilirim?" dedi. Elisa, "Izin ver, senin ruhundan iki pay miras alayim" diye karsilik verdi.
\par 10 Ilyas, "Zor bir sey istedin" dedi, "Eger yanindan alindigimi görürsen olur, yoksa olmaz."
\par 11 Onlar yürüyüp konusurlarken, ansizin atesten bir atli araba göründü, onlari birbirinden ayirdi. Ilyas kasirgayla göklere alindi.
\par 12 Olanlari gören Elisa söyle bagirdi: "Baba, baba, Israil'in arabasi ve atlilari!" Ilyas'i bir daha göremedi. Giysilerini yirtip paramparça etti.
\par 13 Sonra Ilyas'in üzerinden düsen cüppeyi alip geri döndü ve Seria Irmagi'nin kiyisinda durdu.
\par 14 Ilyas'in üzerinden düsen cüppeyi sulara vurarak, "Ilyas'in Tanrisi RAB nerede?" diye seslendi. Cüppeyi sulara vurunca irmak ikiye ayrildi, Elisa karsi yakaya geçti.
\par 15 Erihali peygamberler karsidan Elisa'yi görünce, "Ilyas'in ruhu Elisa'nin üzerinde!" dediler. Sonra onu karsilamaya giderek önünde yere kapandilar.
\par 16 "Yanimizda elli güçlü adam var" dediler, "Izin ver, gidip efendini arayalim. Belki RAB'bin Ruhu onu daglarin ya da vadilerin birine atmistir." Elisa, "Hayir, onlari göndermeyin" dedi.
\par 17 Ama o kadar direttiler ki, sonunda Elisa dayanamadi, "Peki, gönderin" dedi. Elli adam gidip üç gün Ilyas'i aradilarsa da bulamadilar.
\par 18 Sonra Eriha'ya, Elisa'nin yanina döndüler. Elisa onlara, "Ben size gitmeyin demedim mi?" dedi.
\par 19 Erihalilar Elisa'ya, "Efendimiz, gördügün gibi bu kentin yeri iyi ama suyu kötü, topragi da verimsiz" dediler.
\par 20 Elisa, "Yeni bir kabin içine tuz koyup bana getirin" dedi. Kap getirilince,
\par 21 Elisa suyun kaynagina çikti, tuzu suya atip söyle dedi: "RAB diyor ki, 'Bu suyu pakliyorum, artik onda ölüm ve verimsizlik olmayacak."
\par 22 Elisa'nin söyledigi gibi, su bugüne dek temiz kaldi. Elisa'yi Alaya Alan Çocuklar
\par 23 Elisa oradan ayrilip Beytel'e giderken kentin küçük çocuklari yola döküldüler. "Defol, defol, kel kafali!" diyerek onunla alay ettiler.
\par 24 Elisa arkasina dönüp çocuklara bakti ve RAB'bin adiyla onlari lanetledi. Bunun üzerine ormandan çikan iki disi ayi çocuklardan kirk ikisini parçaladi.
\par 25 Elisa oradan Karmel Dagi'na gitti, sonra Samiriye'ye döndü.

\chapter{3}

\par 1 Yahuda Krali Yehosafat'in kralliginin on sekizinci yilinda Ahav oglu Yoram Samiriye'de Israil Krali oldu ve on iki yil krallik yapti.
\par 2 Yoram RAB'bin gözünde kötü olani yaptiysa da annesiyle babasi kadar kötü degildi. Çünkü babasinin yaptirdigi Baal'i* simgeleyen dikili tasi kaldirip atti.
\par 3 Bununla birlikte Nevat oglu Yarovam'in Israil'i sürükledigi günahlara o da katildi ve bu günahlardan ayrilmadi.
\par 4 Moav Krali Mesa koyun yetistirirdi. Israil Krali'na her yil yüz bin kuzu, yüz bin de koç yünü saglamak zorundaydi.
\par 5 Ama Ahav'in ölümünden sonra, Moav Krali Israil Krali'na karsi ayaklandi.
\par 6 O zaman Kral Yoram Samiriye'den ayrildi ve bütün Israilliler'i bir araya topladi.
\par 7 Yahuda Krali Yehosafat'a da su haberi gönderdi: "Moav Krali bana baskaldirdi, benimle birlikte Moavlilar'a karsi savasir misin?" Yehosafat, "Evet, savasirim. Beni kendin, halkimi halkin, atlarimi atlarin say" dedi.
\par 8 Sonra, "Hangi yönden saldiralim?" diye sordu. Yoram, "Edom kirlarindan" diye karsilik verdi.
\par 9 Israil, Yahuda ve Edom krallari birlikte yola çiktilar. Dolambaçli yollarda yedi gün ilerledikten sonra sulari tükendi. Askerler ve hayvanlar susuz kaldi.
\par 10 Israil Krali, "Eyvah!" diye bagirdi, "RAB, Moavlilar'in eline teslim etmek için mi üçümüzü bir araya topladi?"
\par 11 Yehosafat, "Burada RAB'bin peygamberi yok mu? Onun araciligiyla RAB'be danisalim" dedi. Israil Krali'nin adamlarindan biri, "Safat oglu Elisa burada. Ilyas'in ellerine o su dökerdi" diye yanitladi.
\par 12 Kral Yehosafat, "O, RAB'bin ne düsündügünü bilir" dedi. Bunun üzerine Yehosafat, Israil ve Edom krallari birlikte Elisa'nin yanina gittiler.
\par 13 Elisa Israil Krali'na, "Ne diye bana geldin?" dedi, "Git, annenle babanin peygamberlerine danis." Israil Krali, "Olmaz! Demek RAB üçümüzü Moavlilar'in eline teslim etmek için bir araya toplamis" diye karsilik verdi.
\par 14 Elisa söyle dedi: "Hizmetinde oldugum, Her Seye Egemen, yasayan RAB'bin adiyla derim ki, Yahuda Krali Yehosafat'a saygim olmasaydi, sana ne bakardim, ne de ilgilenirdim.
\par 15 Simdi bana lir çalan bir adam getirin." Getirilen adam lir çalarken, RAB'bin gücü Elisa'nin üzerine indi.
\par 16 Elisa söyle dedi: "RAB diyor ki, 'Bu vadinin basindan sonuna kadar hendekler kazin.
\par 17 Ne rüzgar göreceksiniz, ne yagmur. Öyleyken vadi suyla dolup tasacak. Sizler, sürüleriniz ve öteki hayvanlariniz doyasiya içeceksiniz.
\par 18 RAB için bunu yapmak kolaydir. O, Moavlilar'i da sizin elinize teslim edecek.
\par 19 Onlarin önemli surlu kentlerinin tümünü ele geçireceksiniz. Meyve agaçlarinin hepsini kesecek, su kaynaklarini kurutacak, verimli tarlalarina tas dolduracaksiniz."
\par 20 Ertesi sabah, sununun sunuldugu saatte, Edom yönünden akan sular her yeri doldurdu.
\par 21 Moavlilar krallarin kendilerine saldirmak üzere yola çiktiklarini duydular. Genç, yasli eli silah tutan herkes bir araya toplanip sinirda beklemeye basladi.
\par 22 Ertesi sabah erkenden kalktilar. Günes isinlarinin kizillastirdigi suyu kan sanarak,
\par 23 "Kan bu!" diye haykirdilar, "Krallar kendi aralarinda savasip birbirlerini öldürmüs olsalar gerek. Haydi, Moavlilar, yagmaya!"
\par 24 Ama Moavlilar Israil ordugahina vardiklarinda, Israilliler saldirip onlari püskürttü. Moavlilar kaçmaya basladi. Israilliler peslerine düsüp onlari öldürdüler.
\par 25 Kentlerini yiktilar. Her Israilli verimli tarlalara tas atti. Bütün tarlalar tasla doldu. Su kaynaklarini kuruttular, meyve agaçlarini kestiler. Yalniz Kîr-Hereset'in taslari yerinde kaldi. Sapancilar kenti kusatip saldiriya geçti.
\par 26 Moav Krali, savasi kaybettigini anlayinca, yanina yedi yüz kiliçli adam aldi; Edom kuvvetlerini yarip kaçmak istediyse de basaramadi.
\par 27 Bunun üzerine tahtina geçecek en büyük oglunu surlarin üzerine götürüp yakmalik sunu* olarak sundu. Israilliler bu olaydan dogan büyük öfke karsisinda oradan ayrilip ülkelerine döndüler.

\chapter{4}

\par 1 Bir gün, peygamber toplulugundan bir adamin karisi gidip Elisa'ya söyle yakardi: "Efendim, kocam öldü! Bildigin gibi RAB'be tapinirdi. Simdi bir alacaklisi geldi, iki oglumu benden alip köle olarak götürmek istiyor."
\par 2 Elisa, "Senin için ne yapsam?" diye karsilik verdi, "Söyle bana, evinde neler var?" Kadin, "Azicik zeytinyagi disinda, kulunun evinde hiçbir sey yok" dedi.
\par 3 Elisa, "Bütün komsularina git, ne kadar bos kaplari varsa iste" dedi,
\par 4 "Sonra ogullarinla birlikte eve git. Kapiyi üzerinize kapayin ve bütün kaplari yagla doldurun. Doldurduklarinizi bir kenara koyun."
\par 5 Kadin oradan ayrilip ogullariyla birlikte evine gitti, kapiyi kapadi. Ogullarinin getirdigi kaplari doldurmaya basladi.
\par 6 Bütün kaplar dolunca ogullarindan birine, "Bana bir kap daha getir" dedi. Oglu, "Baska kap kalmadi" diye karsilik verdi. O zaman zeytinyaginin akisi durdu.
\par 7 Kadin gidip durumu Tanri adami Elisa'ya bildirdi. Elisa, "Git, zeytinyagini sat, borcunu öde" dedi, "Kalan parayla da ogullarinla birlikte yasamini sürdür."
\par 8 Elisa bir gün Sunem'e gitti. Orada zengin bir kadin vardi. Elisa'yi yemege alikoydu. O günden sonra Elisa ne zaman Sunem'e gitse, yemek için oraya ugradi.
\par 9 Kadin kocasina, "Bize sik sik gelen bu adamin kutsal bir Tanri adami oldugunu anladim" dedi,
\par 10 "Gel, damda onun için küçük bir oda yapalim; içine yatak, masa, sandalye, bir de kandil koyalim. Bize geldiginde orada kalsin."
\par 11 Bir gün Elisa geldi, yukari odaya çikip uzandi.
\par 12 Usagi Gehazi'ye, "Sunemli kadini çagir" dedi. Gehazi kadini çagirdi. Kadin gelince,
\par 13 Elisa Gehazi'ye söyle dedi: "Ona de ki, 'Bizim için katlandigin bunca zahmetlere karsilik ne yapabilirim? Senin için kralla ya da ordu komutaniyla konusayim mi?" Kadin, "Ben halkimin arasinda mutlu yasiyorum" diye karsilik verdi.
\par 14 Elisa, "Öyleyse ne yapabilirim?" diye sordu. Gehazi, "Kadinin oglu yok, kocasi da yasli" diye yanitladi.
\par 15 Bunun üzerine Elisa, "Kadini çagir" dedi. Gehazi kadini çagirdi. Kadin gelip kapinin esiginde durdu.
\par 16 Elisa, kadina, "Gelecek yil bu zaman kucaginda bir oglun olacak" dedi. Kadin, "Olamaz, efendim!" diye karsilik verdi, "Sen ki bir Tanri adamisin, lütfen kuluna yalan söyleme!"
\par 17 Ama kadin gebe kaldi ve bir yil sonra, Elisa'nin söyledigi günlerde bir ogul dogurdu.
\par 18 Çocuk büyüdü. Bir gün orakçilarin basinda bulunan babasinin yanina gitti.
\par 19 "Basim agriyor, basim!" diye bagirmaya basladi. Babasi usagina, "Onu annesine götür" dedi.
\par 20 Usak çocugu alip annesine götürdü. Çocuk öglene kadar annesinin dizlerinde yattiktan sonra öldü.
\par 21 Annesi onu yukari çikardi, Tanri adaminin yatagina yatirdi, sonra kapiyi kapayip disariya çikti.
\par 22 Kocasini çagirip söyle dedi: "Lütfen bir esekle birlikte usaklarindan birini bana gönder. Tanri adaminin yanina gitmeliyim. Hemen dönerim."
\par 23 Kocasi, "Neden bugün gidiyorsun?" dedi, "Ne Yeni Ay, ne de Sabat* bugün." Kadin, "Zarar yok" karsiligini verdi.
\par 24 Esege palan vurup usagina, "Haydi yürü, ben sana söylemedikçe yavaslama" dedi.
\par 25 Karmel Dagi'na varip Tanri adaminin yanina çikti. Tanri adami, kadini uzaktan görünce, usagi Gehazi'ye, "Bak, Sunemli kadin geliyor!" dedi,
\par 26 "Haydi kos, onu karsila, 'Nasilsin, kocanla oglun nasillar? diye sor." Kadin Gehazi'ye, "Herkes iyi" dedi.
\par 27 Kadin daga çikip Tanri adaminin yanina varinca, onun ayaklarina sarildi. Gehazi kadini uzaklastirmak istediyse de Tanri adami, "Kadini rahat birak!" dedi, "Çünkü aci çekiyor. RAB bunun nedenini benden gizledi, açiklamadi."
\par 28 Kadin ona, "Efendim, ben senden çocuk istedim mi?" dedi, "Beni umutlandirma demedim mi?"
\par 29 Elisa Gehazi'ye, "Hemen kemerini kusan, degnegimi al, kos" dedi, "Biriyle karsilasirsan selam verme, biri seni selamlarsa karsilik verme. Git, degnegimi çocugun yüzüne tut."
\par 30 Çocugun annesi, "Yasayan RAB'bin adiyla basin üzerine ant içerim ki, senden ayrilmayacagim" dedi. Sonra Gehazi'yle birlikte yola çikti.
\par 31 Gehazi önden gidip degnegi çocugun yüzüne tuttu, ama ne bir ses vardi, ne de bir yanit. Bunun üzerine Gehazi geri dönüp Elisa'yi karsiladi ve ona, "Çocuk dirilmedi" diye haber verdi.
\par 32 Elisa eve vardiginda, çocugu yataginda ölü buldu.
\par 33 Içeri girdi, kapiyi kapayip RAB'be yalvarmaya basladi.
\par 34 Sonra agzi çocugun agzinin, gözleriyle elleri de çocugun gözleriyle ellerinin üzerine gelecek biçimde yataga, çocugun üzerine kapandi. Çocugun bedeni isinmaya basladi.
\par 35 Elisa kalkip odanin içinde saga sola gezindi, sonra yine dönüp çocugun üzerine kapandi. Çocuk yedi kez aksirdi ve gözlerini açti.
\par 36 Elisa Gehazi'ye, "Sunemli kadini çagir" diye seslendi. Gehazi kadini çagirdi. Kadin gelince, Elisa, "Al oglunu" dedi.
\par 37 Kadin Elisa'nin ayaklarina kapandi, yerlere kadar egildi, sonra çocugunu alip gitti.
\par 38 Elisa Gilgal'a döndü. Ülkede kitlik vardi. Elisa bir peygamber topluluguyla otururken usagina, "Büyük tencereyi atese koy, peygamberlere çorba pisir" dedi.
\par 39 Biri ot toplamak için tarlaya gitti ve yabanil bir bitki buldu. Bitkiden bir etek dolusu yaban kabagi topladi, getirip tencereye dogradi. Bunlarin ne oldugunu kimse bilmiyordu.
\par 40 Çorba yenmek üzere bosaltildi. Ama adamlar çorbayi tadar tatmaz, "Ey Tanri adami, zehirli bu!" diye bagirdilar ve yiyemediler.
\par 41 Elisa, "Biraz un getirin" dedi. Unu tencereye atip, "Koy önlerine, yesinler" dedi. Tencerede zararli bir sey kalmadi.
\par 42 Baal-Salisa'dan bir adam geldi. Tanri adamina o yil ilk biçilen arpadan yapilmis yirmi ekmekle taze bugday basagi getirdi. Elisa usagina, "Bunlari halka dagit, yesinler" dedi.
\par 43 Usak, "Nasil olur, bu yüz kisinin önüne konur mu?" diye sordu. Elisa, "Halka dagit, yesinler" diye karsilik verdi, "Çünkü RAB diyor ki, 'Yiyecekler, birazi da artacak."
\par 44 Bunun üzerine usak yiyecekleri halkin önüne koydu. RAB'bin sözü uyarinca halk yedi, birazi da artti.

\chapter{5}

\par 1 Aram Krali'nin ordu komutani Naaman efendisinin gözünde saygin, degerli bir adamdi. Çünkü RAB onun araciligiyla Aramlilar'i zafere ulastirmisti. Naaman yigit bir askerdi, ama bir deri hastaligina yakalanmisti.
\par 2 Aramlilar düzenledikleri akinlar sirasinda Israil'den küçük bir kizi tutsak almislardi. Bu kiz Naaman'in karisinin hizmetine verilmisti.
\par 3 Bir gün hanimina, "Keske efendim Samiriye'deki peygamberin yanina gitse! Peygamber onu deri hastaligindan kurtarirdi" dedi.
\par 4 Naaman gidip Israilli kizin söylediklerini efendisi krala anlatti.
\par 5 Aram Krali söyle karsilik verdi: "Kalk git, seninle Israil Krali'na bir mektup gönderecegim." Naaman yanina on talant gümüs, alti bin sekel altin ve on takim giysi alip gitti.
\par 6 Mektubu Israil Krali'na verdi. Mektupta sunlar yaziliydi: "Bu mektupla birlikte sana kulum Naaman'i gönderiyorum. Onu deri hastaligindan kurtarmani dilerim."
\par 7 Israil Krali mektubu okuyunca giysilerini yirtip söyle haykirdi: "Ben Tanri miyim, can alip can vereyim? Nasil bana bir adam gönderip onu deri hastaligindan kurtar der? Görüyor musunuz, açikça benimle kavga çikarmaya çalisiyor!"
\par 8 Israil Krali'nin giysilerini yirttigini duyan Tanri adami Elisa ona su haberi gönderdi: "Neden giysilerini yirttin? Adam bana gelsin, Israil'de bir peygamber oldugunu anlasin!"
\par 9 Böylece Naaman atlari ve savas arabalariyla birlikte gidip Elisa'nin evinin kapisi önünde durdu.
\par 10 Elisa ona su haberi gönderdi: "Git, Seria Irmagi'nda yedi kez yikan. Tenin eski halini alacak, tertemiz olacaksin."
\par 11 Gelgelelim Naaman oradan öfkeyle ayrildi. "Sandim ki disari çikip yanima gelecek, Tanrisi RAB'be yalvararak eliyle hastalikli derime dokunup beni iyilestirecek" dedi,
\par 12 "Sam'in Avana ve Farpar irmaklari Israil'in bütün irmaklarindan daha iyi degil mi? Oralarda yikanip paklanamaz miydim sanki?" Sonra öfkeyle dönüp gitti.
\par 13 Naaman'in görevlileri yanina varip, "Efendim, peygamber senden daha zor bir sey istemis olsaydi, yapmaz miydin?" dediler, "Oysa o sana sadece, 'Yikan, temizlen diyor."
\par 14 Bunun üzerine Naaman Tanri adaminin sözü uyarinca gidip Seria Irmagi'nda yedi kez suya daldi. Teni eski haline döndü, bebek teni gibi tertemiz oldu.
\par 15 Naaman adamlariyla birlikte Tanri adaminin yanina döndü. Onun önünde durup söyle dedi: "Simdi anladim ki, Israil disinda dünyanin hiçbir yerinde Tanri yoktur. Lütfen, bu kulunun armaganini kabul et."
\par 16 Elisa, "Hizmetinde oldugum yasayan RAB'bin adiyla ant içerim ki, hiçbir sey alamam" diye karsilik verdi. Naaman direttiyse de, Elisa almak istemedi.
\par 17 Bunun üzerine Naaman, "Madem armagan istemiyorsun, öyleyse buradan iki katir yükü toprak almama izin ver" dedi, "Çünkü bu kulun artik RAB'bin disinda baska ilahlara yakmalik sunu* ve kurban sunmayacaktir.
\par 18 Ama RAB kulunu bir konuda bagislasin. Efendim tapinmak için Rimmon Tapinagi'na girip kendisine eslik etmemi isteyince, tapinakta onunla birlikte yere kapandigimda RAB bu kulunu bagislasin."
\par 19 Elisa ona, "Esenlikle git" dedi. Naaman oradan ayrilip biraz uzaklasinca,
\par 20 Tanri adami Elisa'nin usagi Gehazi, "Efendim, Aramli Naaman'a çok yumusak davrandi; getirdigi armaganlari kabul etmedi" dedi, "Yasayan RAB'bin hakki için, pesinden kosup ondan bir sey alacagim."
\par 21 Böylece Gehazi Naaman'in pesine düstü. Naaman ardindan birinin kostugunu görünce, arabasindan inip onu karsiladi ve, "Ne oldu?" diye sordu.
\par 22 Gehazi, "Bir sey yok" dedi, "Yalniz efendimin bir ricasi var. Biraz önce Efrayim'in daglik bölgesinden iki genç peygamber geldi. Efendim onlara bir talant gümüsle iki takim giysi vermen için beni gönderdi."
\par 23 Naaman, "Lütfen iki talant al!" dedi ve israrla iki talant gümüsü iki torbaya koyup bagladi. Ayrica iki usagina da birer takim giysi verdi. Usaklar Gehazi'nin önüsira bunlari tasidilar.
\par 24 Tepeye varinca Gehazi esyalari ellerinden alip eve koydu, adamlari da geri gönderdi.
\par 25 Sonra gidip efendisi Elisa'nin huzuruna çikti. Elisa, "Neredeydin, Gehazi?" diye sordu. Gehazi, "Kulun hiçbir yere gitmedi" diye karsilik verdi.
\par 26 Bunun üzerine Elisa, "O adam arabasindan inip seni karsilarken ruhum seninle degil miydi?" diye sordu, "Simdi gümüs ya da giysi, zeytinlik, bag, koyun, sigir, erkek ve kadin köle almanin zamani mi?
\par 27 Bu yüzden Naaman'in deri hastaligi sonsuza dek senin ve soyunun üzerinde kalacak." Böylece Gehazi Elisa'nin huzurundan kar gibi beyaz bir deri hastaligiyla ayrildi.

\chapter{6}

\par 1 Bir gün peygamber toplulugu Elisa'ya, "Bak, yasadigimiz yer bize küçük geliyor" dedi,
\par 2 "Lütfen izin ver, Seria Irmagi kiyisina gidelim, agaç kesip kendimize ev yapalim." Elisa, "Gidin" dedi.
\par 3 Peygamberlerden biri, "Lütfen kullarinla birlikte sen de gel" dedi. Elisa, "Olur, gelirim" diye karsilik verdi
\par 4 ve onlarla birlikte gitti. Seria Irmagi kiyisina varinca agaç kesmeye basladilar.
\par 5 Biri agaç keserken balta demirini suya düsürdü. "Eyvah, efendim! Onu ödünç almistim" diye bagirdi.
\par 6 Tanri adami, "Nereye düstü?" diye sordu. Adam ona demirin düstügü yeri gösterdi. Elisa bir dal kesip oraya atinca, balta demiri su yüzüne çikti.
\par 7 Elisa, "Al onu!" dedi. Adam elini uzatip balta demirini aldi.
\par 8 Aram Krali Israil'le savas halindeydi. Görevlilerine danistiktan sonra, "Ordugahimi kuracak bir yer seçtim" dedi.
\par 9 Tanri adami Elisa, Israil Krali'na su haberi gönderdi: "Sakin oradan geçmeyin, çünkü Aramlilar oraya dogru iniyorlar."
\par 10 Israil Krali adam gönderip oradaki durumu denetledi. Böylece Tanri adami Israil Krali'ni birkaç kez uyardi. Kral da önlem aldi.
\par 11 Bu durum Aram Krali'ni çok öfkelendirdi. Görevlilerini çagirip, "Içinizden hanginizin Israil Krali'ndan yana oldugunu söylemeyecek misiniz?" dedi.
\par 12 Görevlilerden biri, "Hiçbirimiz, efendimiz kral" diye karsilik verdi, "Yalniz Israil'de yasayan Peygamber Elisa senin yatak odanda söylediklerini bile Israil Krali'na bildiriyor."
\par 13 Aram Krali söyle buyurdu: "Gidip onun nerede oldugunu ögrenin. Adam gönderip onu yakalayacagim." Elisa'nin Dotan'da oldugu bildirilince,
\par 14 kral oraya atlilar, savas arabalari ve büyük bir kuvvet gönderdi. Geceleyin varip kenti kusattilar.
\par 15 Tanri adaminin usagi erkenden kalkti. Disariya çikinca kentin askerler, atlilar ve savas arabalarinca kusatildigini gördü. Dönüp Elisa'ya, "Eyvah, efendim, ne yapacagiz?" diye sordu.
\par 16 Elisa, "Korkma, çünkü bizim yandaslarimiz onlarinkinden daha çok" diye karsilik verdi.
\par 17 Sonra söyle dua etti: "Ya RAB, lütfen onun gözlerini aç, görsün!" RAB usagin gözlerini açti. Usak Elisa'nin çevresindeki daglarin atlilarla, atesten savas arabalariyla dolu oldugunu gördü.
\par 18 Aramlilar kendisine dogru ilerleyince Elisa RAB'be söyle yalvardi: "Ya RAB, lütfen bu halki kör et." RAB Elisa'nin yalvarisini duydu ve onlari kör etti.
\par 19 Bunun üzerine Elisa onlara, "Yanlis yoldasiniz" dedi, "Aradiginiz kent bu degil. Beni izleyin, sizi aradiginiz adama götüreyim." Sonra onlari Samiriye'ye götürdü.
\par 20 Samiriye'ye girdiklerinde Elisa söyle dua etti: "Ya RAB, bu adamlarin gözlerini aç, görsünler." RAB gözlerini açinca adamlar Samiriye'nin ortasinda olduklarini anladilar.
\par 21 Israil Krali adamlari görünce Elisa'ya, "Onlari öldüreyim mi? Öldüreyim mi, baba?" dedi.
\par 22 Elisa, "Hayir, öldürme" diye karsilik verdi, "Kendi kiliç ve yayinla tutsak aldigin insanlari nasil öldürürsün. Önlerine yiyecek içecek bir seyler koy, yiyip içtikten sonra izin ver, krallarina dönsünler."
\par 23 Bunun üzerine Israil Krali adamlara büyük bir sölen verdi, yedirip içirdikten sonra da onlari krallarina gönderdi. Aramli akincilar bir daha Israil topraklarina ayak basmadilar.
\par 24 Bir süre sonra, Aram Krali Ben-Hadat bütün ordusunu toplayip Israil'e girdi ve Samiriye'yi kusatti.
\par 25 Samiriye'de büyük bir kitlik oldu. Kusatma sonunda bir esek kellesinin fiyati seksen sekel gümüse, dörtte bir kav güvercin gübresinin fiyati ise bes sekel gümüse çikti.
\par 26 Israil Krali surlarin üzerinde yürürken, bir kadin, "Efendim kral, bana yardim et!" diye seslendi.
\par 27 Kral, "RAB sana yardim etmiyorsa, ben nasil yardim edebilirim ki?" diye karsilik verdi, "Bugday mi, yoksa sarap mi istersin?
\par 28 Derdin ne?" Kadin söyle yanitladi: "Geçen gün su kadin bana dedi ki, 'Oglunu ver, bugün yiyelim, yarin da benim oglumu yeriz.
\par 29 Böylece oglumu pisirip yedik. Ertesi gün ona, 'Oglunu ver de yiyelim dedim. Ama o, oglunu gizledi."
\par 30 Kadinin bu sözlerini duyan kral giysilerini yirtti. Surlarin üzerinde yürürken, halk onun giysilerinin altina çul giydigini gördü.
\par 31 Kral, "Eger bugün Safat oglu Elisa'nin basi yerinde kalirsa, Tanri bana aynisini, hatta daha kötüsünü yapsin!" dedi.
\par 32 Elisa o sirada halkin ileri gelenleriyle birlikte evinde oturuyordu. Kral önden bir haberci gönderdi. Ama daha haberci gelmeden, Elisa ileri gelenlere, "Görüyor musunuz caniyi?" dedi, "Kalkmis, basimi kestirmek için adam gönderiyor! Haberci geldiginde kapiyi kapayin, onu içeri almayin. Çünkü ardindan efendisi kral da gelecek."
\par 33 Elisa konusmasini bitirmeden, haberci yanina geldi ve, "Bu felaket RAB'dendir" dedi, "Neden hâlâ RAB'bi bekleyeyim?"

\chapter{7}

\par 1 Elisa, "RAB'bin sözüne kulak verin!" dedi, "RAB diyor ki, 'Yarin bu saatlerde Samiriye Kapisi'nda bir sea ince un da, iki sea arpa da birer sekele satilacak."
\par 2 Kralin özel yardimcisi olan komutan, Tanri adamina, "RAB göklerin kapaklarini açsa bile olacak sey degil bu!" dedi. Elisa, "Sen herseyi gözlerinle göreceksin, ama onlardan hiçbir sey yiyemeyeceksin!" diye karsilik verdi.
\par 3 Kent kapisinin girisinde deri hastaligina yakalanmis dört adam vardi. Birbirlerine, "Ne diye ölene dek burada kalalim?" diyorlardi,
\par 4 "Kente girelim desek, orada kitlik var, ölürüz; burada kalsak da ölecegiz. Bari gidip Aram ordugahina teslim olalim. Canimizi bagislarlarsa yasariz, öldürürlerse de öldürsünler."
\par 5 Aksam karanliginda kalkip Aram ordugahina dogru gittiler. Ordugaha yaklastiklarinda, orada kimseyi göremediler.
\par 6 Çünkü Rab Aram ordugahinda savas arabalariyla, atlariyla yaklasan büyük bir ordunun çikardigi seslerin duyulmasini saglamisti. Aramlilar da birbirlerine, "Bakin, Israil Krali bize saldirmak için Hitit* ve Misir krallarini kiralamis!" demislerdi.
\par 7 Böylece, gün batarken çadirlarini, atlarini, eseklerini birakip kaçmislar, canlarini kurtarmak için ordugahi oldugu gibi birakmislardi.
\par 8 Deri hastaligina yakalanmis adamlar ordugaha varip çadirlarin birine girdiler. Yiyip içtikten sonra oradaki altin, gümüs ve giysileri götürüp gizlediler. Sonra dönüp baska bir çadira girdiler, orada bulduklarini da götürüp gizlediler.
\par 9 Ardindan birbirlerine, "Yaptigimiz dogru degil" dediler, "Bugün müjde günü. Oysa biz susuyoruz. Gün doguncaya kadar beklersek, cezaya çarptirilacagimiz kesin. Haydi saraya gidip durumu bildirelim."
\par 10 Böylece gidip kent kapisindaki nöbetçilere seslendiler. "Aram ordugahina gittik" dediler, "Hiç kimseyi göremedik; ne de bir insan sesi duyduk. Yalnizca bagli atlar, esekler vardi. Çadirlari da oldugu gibi birakip gitmisler."
\par 11 Kapi nöbetçileri haberi duyurdu. Haber kralin sarayina ulastirildi.
\par 12 Kral gece kalkip görevlilerine, "Aramlilar'in ne tasarladigini size söyleyeyim" dedi, "Aç kaldigimizi biliyorlar. Onun için ordugahlarini birakip kirda gizlenmisler. Kentin disina çiktigimizda, bizi canli yakalayip kenti ele geçirmeyi düsünüyorlar."
\par 13 Görevlilerden biri, "Kentte kalan bes atla birkaç adam gönderelim, o zaman durumu anlariz" dedi, "Nasil olsa gidecek olanlar da burada, kentte kalan nice Israilli gibi ölüme mahkûm!"
\par 14 Adamlar yanlarina iki atli araba aldilar. Kral, "Gidin, ne oldugunu ögrenin" diyerek onlari Aram ordusunun ardindan gönderdi.
\par 15 Adamlar Seria Irmagi'na kadar Aram ordusunu izlediler. Yol bastan sona kadar Aramlilar'in kaçarken attiklari giysi ve esyalarla doluydu. Haberciler dönüp krala durumu bildirdiler.
\par 16 Bunun üzerine halk kentten çikip Aram ordugahini yagmaladi. RAB'bin dedigi gibi, bir sea ince unun da, iki sea arpanin da fiyati bir sekele düstü.
\par 17 Kral özel yardimcisi olan komutani kentin kapisinda birakmisti. Halk onu kapinin agzinda çigneyerek öldürdü. Kral Elisa'nin evine gittiginde, Tanri adami ona olacaklari önceden bildirmisti.
\par 18 Her sey Tanri adaminin krala dedigi gibi oldu. "Yarin bu saatlerde Samiriye Kapisi'nda bir sea ince un da, iki sea arpa da birer sekele satilacak" demisti.
\par 19 Komutan da Tanri adamina söyle karsilik vermisti: "RAB göklerin kapaklarini açsa bile, olacak sey degil bu!" Elisa, "Sen her seyi gözlerinle görecek, ama onlardan hiçbir sey yiyemeyeceksin!" demisti.
\par 20 Tam dedigi gibi oldu. Komutan kentin kapisinda halk tarafindan çignenerek öldü.

\chapter{8}

\par 1 Elisa, oglunu diriltmis oldugu Sunemli kadina söyle demisti: "Kalk, ailenle birlikte buradan git, geçici olarak kalabilecegin bir yer bul. Çünkü RAB ülkeye yedi yil sürecek bir kitlik göndermeye karar verdi."
\par 2 Kadin Tanri adaminin ögüdüne uyarak ailesiyle birlikte kalkip Filist ülkesine gitti ve orada yedi yil kaldi.
\par 3 Yedi yil sonra Filist'ten döndü. Evini, tarlasini geri almak için kraldan yardim istemeye gitti.
\par 4 O sirada kral Tanri adaminin usagi Gehazi'yle konusuyor, "Bana Elisa'nin yaptigi bütün mucizeleri anlat" diyordu.
\par 5 Iste Gehazi tam Elisa'nin ölüyü nasil dirilttigini krala anlatirken, oglu diriltilen kadin eviyle tarlasini geri almak için kraldan yardim istemeye geldi. Gehazi krala, "Efendim kral, sözünü ettigim kadin budur. Yanindaki oglu da Elisa'nin dirilttigi çocuktur" dedi.
\par 6 Kral kadina sorunca kadin her seyi anlatti. Bunun üzerine kral bir görevli çagirtip su buyrugu verdi: "Bu kadina her seyini, ülkeden ayrildigi günden bugüne kadar biriken bütün geliriyle birlikte tarlasini geri verin."
\par 7 Aram Krali Ben-Hadat hastalandigi sirada Elisa Sam'a gitti. Tanri adaminin Sam'a geldigi krala bildirildi.
\par 8 Kral, Hazael'e, "Bir armagan al, Tanri adamini karsilamaya git" dedi, "Onun araciligiyla RAB'be danis, bu hastaliktan kurtulup kurtulamayacagimi sor."
\par 9 Hazael, Sam'in en iyi mallarindan olusan kirk deve yükü armagani yanina alarak, Tanri adamini karsilamaya gitti. Elisa'nin önünde durup söyle dedi: "Kulun Aram Krali Ben-Hadat, hastaligindan kurtulup kurtulamayacagini sormam için beni gönderdi."
\par 10 Elisa, "Git ona, 'Kesinlikle iyileseceksin de; ama RAB bana onun kesinlikle ölecegini açikladi" diye karsilik verdi.
\par 11 Tanri adami, Hazael'i utandirincaya kadar dik dik yüzüne bakti. Ardindan aglamaya basladi.
\par 12 Hazael, "Efendim, niçin agliyorsun?" diye sordu. Elisa, "Senin Israil halkina yapacagin kötülükleri biliyorum" diye yanitladi, "Kalelerini atese verecek, gençlerini kiliçtan geçirecek, çocuklarini yere çalip öldürecek, gebe kadinlarinin karinlarini deseceksin."
\par 13 Hazael, "Bir köpekten farksiz olan bu kulun, bütün bu isleri nasil yapabilir?" dedi. Elisa, "RAB bana senin Aram Krali olacagini gösterdi" diye yanitladi.
\par 14 Bunun üzerine Hazael Elisa'dan ayrilip efendisi Ben-Hadat'in yanina döndü. Ben-Hadat ona, "Elisa sana ne söyledi?" diye sordu. Hazael, "Kesinlikle iyilesecegini söyledi" diye yanitladi.
\par 15 Gelgelelim ertesi gün Hazael islattigi bir örtüyü kralin yüzüne kapatip onu bogdu. Böylece kral öldü, yerine Hazael geçti.
\par 16 Israil Krali Ahav oglu Yoram'in kralliginin besinci yilinda, Yehosafat'in Yahuda Krali oldugu sirada, Yehosafat'in oglu Yehoram Yahuda'yi yönetmeye basladi.
\par 17 Yehoram otuz iki yasinda kral oldu ve Yerusalim'de sekiz yil krallik yapti.
\par 18 Karisi Ahav'in kizi oldugu için, o da Ahav'in ailesi gibi Israil krallarinin yolunu izledi ve RAB'bin gözünde kötü olani yapti.
\par 19 Ama RAB kulu Davut'un hatiri için Yahuda'yi yok etmek istemedi. Çünkü Davut'a ve soyuna sönmeyen bir isik verecegine söz vermisti.
\par 20 Yehoram'in kralligi döneminde Edomlular Yahudalilar'a karsi ayaklanarak kendi kralliklarini kurdular.
\par 21 Yehoram bütün savas arabalariyla Sair'e gitti. Edomlular onu ve savas arabalarinin komutanlarini kusattilar. Ama Yehoram gece kalkip kusatmayi yararak kaçti. Askerleri de kaçarak evlerine döndü.
\par 22 O sirada Livna Kenti ayaklandi. Edomlular'in Yahuda'ya karsi baskaldirmasi bugün de sürüyor.
\par 23 Yehoram'in kralligi dönemindeki öteki olaylar ve bütün yaptiklari Yahuda krallarinin tarihinde yazilidir.
\par 24 Yehoram ölüp atalarina kavustu ve Davut Kenti'nde atalarinin yanina gömüldü. Yerine oglu Ahazya kral oldu.
\par 25 Israil Krali Ahav oglu Yoram'in kralliginin on ikinci yilinda Yehoram oglu Ahazya Yahuda Krali oldu.
\par 26 Ahazya yirmi iki yasinda kral oldu ve Yerusalim'de bir yil krallik yapti. Annesi Israil Krali Omri'nin torunu Atalya'ydi.
\par 27 Ahazya evlilik yoluyla Ahav'a akraba oldugu için Ahav ailesinin yolunu izledi ve onlar gibi RAB'bin gözünde kötü olani yapti.
\par 28 Ahazya, Aram Krali Hazael'le savasmak üzere Ahav oglu Yoram'la birlikte Ramot-Gilat'a gitti. Aramlilar Yoram'i yaraladilar.
\par 29 Kral Yoram Ramot-Gilat'ta Aram Krali Hazael'le savasirken aldigi yaralarin iyilesmesi için Yizreel'e döndü. Yahuda Krali Yehoram oglu Ahazya da yaralanan Ahav oglu Yoram'i görmek için Yizreel'e gitti.

\chapter{9}

\par 1 Peygamber Elisa, peygamberler toplulugundan bir adam çagirip, "Kemerini kusan, bu yag kabini alip Ramot-Gilat'a git" dedi,
\par 2 "Oraya varinca Nimsi oglu, Yehosafat oglu Yehu'yu ara. Onu kardeslerinin arasindan alip baska bir odaya götür.
\par 3 Zeytinyagini basina dök ve ona RAB söyle diyor de: 'Seni Israil Krali olarak meshettim*. Sonra kapiyi aç ve kos, oyalanma!"
\par 4 Böylece peygamberin usagi Ramot-Gilat'a gitti.
\par 5 Oraya vardiginda ordu komutanlarinin bir arada oturduklarini gördü. "Komutanim, sana bir haberim var" dedi. Yehu, "Hangimize söylüyorsun?" diye sordu. Usak, "Sana, efendim" diye yanitladi.
\par 6 Yehu kalkip eve girdi. Usak yagi Yehu'nun basina döküp ona söyle dedi: "Israil'in Tanrisi RAB diyor ki, 'Seni halkim Israil'in krali olarak meshettim.
\par 7 Efendin Ahav'in ailesini öldüreceksin. Bana hizmet eden peygamberlerin ve bütün kullarimin dökülen kaninin öcünü Izebel'den alacagim.
\par 8 Ahav'in bütün soyu ortadan kalkacak. Israil'de genç yasli Ahav'in soyundan gelen bütün erkeklerin kökünü kurutacagim.
\par 9 Nevat oglu Yarovam'la Ahiya oglu Baasa'nin ailelerine ne yaptimsa, Ahav'in ailesine de aynisini yapacagim.
\par 10 Yizreel topraklarinda Izebel'in ölüsünü köpekler yiyecek ve onu gömen olmayacak." Usak bunlari söyledikten sonra kapiyi açip kaçti.
\par 11 Yehu komutan arkadaslarinin yanina döndü. Içlerinden biri, "Her sey yolunda mi? O delinin seninle ne isi vardi?" diye sordu. Yehu, "Onu taniyorsunuz, neler saçmaladigini bilirsiniz" diye karsilik verdi.
\par 12 "Hayir, bilmiyoruz, ne söyledi? Anlat bize!" dediler. Yehu söyle yanitladi: "Bana RAB söyle diyor dedi: 'Seni Israil Krali olarak meshettim."
\par 13 Bunun üzerine hepsi hemen cüppelerini çikarip merdivenin basinda duran Yehu'nun ayaklarina serdi. Boru çalarak, "Yehu kraldir!" diye bagirdilar.
\par 14 Nimsi oglu Yehosafat oglu Yehu Yoram'a karsi bir düzen kurdu. O siralarda Yoram ile Israil halki Aram Krali Hazael'e karsi Ramot-Gilat'i savunuyordu.
\par 15 Ancak Kral Yoram, Aram Krali Hazael'le savasirken Aramlilar onu yaralamisti. Yoram da yaralarin iyilesmesi için Yizreel'e dönmüstü. Yehu arkadaslarina, "Eger siz de benimle ayni görüsteyseniz, hiç kimsenin kentten kaçmasina ve gidip durumu Yizreel'e bildirmesine izin vermeyin" dedi.
\par 16 Yehu savas arabasina binip Yizreel'e gitti. Çünkü Yoram orada hasta yatiyordu. Yahuda Krali Ahazya da Yoram'i görmek için oraya gitmisti.
\par 17 Yizreel'de kulede nöbet tutan gözcü, Yehu'nun ordusuyla yaklastigini görünce, "Bir kalabalik görüyorum!" diye bagirdi. Yoram, "Bir atli gönder, onu karsilasin, baris için gelip gelmedigini sorsun" dedi.
\par 18 Atli Yehu'yu karsilamaya gitti ve ona, "Kralimiz, 'Baris için mi geldin? diye soruyor" dedi. Yehu, "Baristan sana ne! Sen beni izle" diye karsilik verdi. Gözcü durumu krala bildirdi: "Ulak onlara vardi, ama geri dönmedi."
\par 19 Bu kez ikinci bir atli gönderildi. Atli onlara varip, "Kralimiz, 'Baris için mi geldin? diye soruyor" dedi. Yehu, "Baristan sana ne! Sen beni izle" diye karsilik verdi.
\par 20 Gözcü durumu krala bildirdi: "Ulak onlara vardi, ama geri dönmedi. Komutanlari savas arabasini Nimsi oglu Yehu gibi delicesine sürüyor."
\par 21 Kral Yoram, "Arabami hazirlayin!" diye buyruk verdi. Arabasi hazirlandi. Israil Krali Yoram ile Yahuda Krali Ahazya arabalarina binip Yehu'yu karsilamaya gittiler. Yizreelli Navot'un topraklarinda onunla karsilastilar.
\par 22 Yoram Yehu'yu görünce, "Baris için mi geldin?" diye sordu. Yehu, "Annen Izebel'in yaptigi bunca putperestlik ve büyücülük sürüp giderken baristan söz edilir mi?" diye karsilik verdi.
\par 23 Yoram, "Hainlik bu, Ahazya!" diye bagirdi ve arabasinin dizginlerini çevirip kaçti.
\par 24 Yehu var gücüyle yayini çekip Yoram'i sirtindan vurdu. Ok Yoram'in kalbini delip geçti. Yoram arabasinin içine yigilip kaldi.
\par 25 Yehu yardimcisi Bidkar'a, "Onun cesedini al, Yizreelli Navot'un topragina at" dedi, "Animsa, senle ben birlikte Yoram'in babasi Ahav'in ardindan savas arabasiyla giderken, RAB Ahav'a,
\par 26 'Dün Navot'la ogullarinin kanini gördüm. Seni de bu topraklarda cezalandiracagim demisti. Simdi RAB'bin sözü uyarinca, Yoram'in cesedini al, Navot'un topragina at!"
\par 27 Yahuda Krali Ahazya olanlari görünce Beythaggan'a dogru kaçmaya basladi. Yehu ardina takilip, "Onu da öldürün!" diye bagirdi. Ahazya'yi Yivleam yakinlarinda, Gur yolunda, arabasinin içinde vurdular. Yarali olarak Megiddo'ya kadar kaçip orada öldü.
\par 28 Adamlari Ahazya'nin cesedini bir savas arabasina koyup Yerusalim'e götürdüler. Onu Davut Kenti'nde atalarinin yanina, kendi mezarina gömdüler.
\par 29 Ahazya Ahav oglu Yoram'in kralliginin on birinci yilinda Yahuda Krali olmustu.
\par 30 Sonra Yehu Yizreel'e gitti. Izebel bunu duyunca, gözlerine sürme çekti, saçlarini tarayip pencereden disariyi gözlemeye basladi.
\par 31 Yehu kentin kapisindan içeri girince, Izebel, "Ey efendisini öldüren Zimri, baris için mi geldin?" diye seslendi.
\par 32 Yehu pencereye dogru bakip, "Kim benden yana?" diye bagirdi. Iki üç görevli yukaridan ona bakti.
\par 33 Yehu, "Atin onu asagi!" dedi. Görevliler Izebel'i asagiya attilar. Kani surlarin ve bedenini çigneyen atlarin üzerine siçradi.
\par 34 Yehu içeri girip yedi, içti. Sonra, "O lanet olasi kadini alip gömün, ne de olsa bir kral kizidir" dedi.
\par 35 Ama Izebel'i gömmeye giden adamlar basindan, ayaklarindan, ellerinden baska bir sey bulamadilar.
\par 36 Geri dönüp durumu Yehu'ya bildirdiler. Yehu onlara söyle dedi: "Kulu Tisbeli Ilyas araciligiyla konusan RAB'bin sözü yerine geldi. RAB, 'Yizreel topraklarinda Izebel'in ölüsünü köpekler yiyecek demisti.
\par 37 'Izebel'in lesi Yizreel topraklarina gübre olacak ve kimse, bu Izebel'dir, diyemeyecek."

\chapter{10}

\par 1 Ahav'in Samiriye'de yetmis oglu vardi. Yehu mektuplar yazip Samiriye'ye gönderdi. Yizreel'in yöneticilerine, ileri gelenlere ve Ahav'in çocuklarini koruyanlara yazdigi mektuplarda Yehu söyle diyordu:
\par 2 "Efendinizin ogullari sizinle birliktedir. Savas arabalariniz, atlariniz, silahlariniz var. Surlu bir kentte yasiyorsunuz. Bu mektup size ulasir ulasmaz,
\par 3 efendinizin ogullarindan en iyi ve en uygun olani seçip babasinin tahtina oturtun. Ve efendinizin ailesini korumak için savasin."
\par 4 Ama onlar dehsete düstüler. "Iki kral Yehu'yla basa çikamadi, biz nasil çikariz?" dediler.
\par 5 Saray sorumlusu, kent valisi, ileri gelenler ve Ahav'in çocuklarini koruyanlar Yehu'ya su haberi gönderdi: "Biz senin kullariniz, söyleyecegin her seyi yapmaya haziriz. Kimseyi kral yapmaya niyetimiz yok. Kendin için en iyi olan neyse onu yap."
\par 6 Yehu onlara ikinci bir mektup yazdi: "Eger siz benden yana ve bana bagliysaniz, efendinizin ogullarinin basini kesip yarin bu saatlerde Yizreel'e, bana getirin." Kral Ahav'in yetmis oglu, onlari yetistirmekle görevli kent ileri gelenlerinin korumasi altindaydi.
\par 7 Yehu'nun mektubu kent ileri gelenlerine ulasinca, Ahav'in yetmis oglunu öldürüp baslarini küfelere koydular ve Yizreel'e, Yehu'ya gönderdiler.
\par 8 Ulak gelip Yehu'ya, "Kral ogullarinin baslarini getirdiler" diye haber verdi. Yehu, "Onlari iki yigin halinde kent kapisinin girisine birakin, sabaha kadar orada kalsinlar" dedi.
\par 9 Ertesi sabah Yehu halkin önüne çikip söyle dedi: "Efendime düzen kurup onu öldüren benim, sizin suçunuz yok. Ama bunlari kim öldürdü?
\par 10 Bu olay gösteriyor ki, RAB'bin Ahav'in ailesine iliskin söyledigi hiçbir söz bosa çikmayacaktir. RAB, kulu Ilyas araciligiyla verdigi sözü yerine getirdi."
\par 11 Sonra Yizreel'de Ahav'in öteki akrabalarinin hepsini, bütün yüksek görevlilerini, yakin arkadaslarini ve kâhinlerini* öldürdü. Sag kalan olmadi.
\par 12 Yehu Yizreel'den ayrilip Samiriye'ye dogru yola çikti. Yolda çobanlarin Beyteket adini verdigi yerde,
\par 13 Yahuda Krali Ahazya'nin akrabalariyla karsilasti. Onlara, "Siz kimsiniz?" diye sordu. "Biz Ahazya'nin akrabalariyiz" diye karsilik verdiler, "Kralin ve ana kraliçe Izebel'in çocuklarina saygilarimizi sunmaya gidiyoruz."
\par 14 Yehu adamlarina, "Bunlari diri yakalayin!" diye buyruk verdi. Onlari diri yakalayip Beyteket Kuyusu yakininda kiliçtan geçirdiler. Öldürülenler kirk iki kisiydi. Sag kalan olmadi. Ahav'in Öteki Akrabalarinin Öldürülmesi
\par 15 Yehu oradan ayrildi. Yolda kendisine dogru gelen Rekav oglu Yehonadav'la karsilasti. Ona selam vererek, "Ben sana karsi iyi duygular besliyorum, sen de ayni duygulara sahip misin?" diye sordu. Yehonadav, "Evet" diye yanitladi. Yehu, "Öyleyse elini ver" dedi. Yehonadav elini uzatti. Yehu onu arabasina alarak,
\par 16 "Benimle gel ve RAB için nasil çaba harcadigimi gör" dedi. Sonra onu arabasiyla Samiriye'ye götürdü.
\par 17 Samiriye'ye varinca Yehu RAB'bin Ilyas araciligiyla söyledigi söz uyarinca, Ahav'in orada kalan akrabalarinin hepsini öldürdü.
\par 18 Yehu, bütün halki toplayarak, "Ahav Baal'a* az kulluk etti, ben daha çok edecegim" dedi,
\par 19 "Baal'in bütün peygamberlerini, kâhinlerini, ona tapan herkesi çagirin. Hiçbiri gelmemezlik etmesin. Çünkü Baal'a büyük bir kurban sunacagim. Kim gelmezse öldürülecek." Gerçekte Yehu Baal'a tapanlari yok etmek için bir düzen kurmaktaydi.
\par 20 Yehu, "Baal'in onuruna bir toplanti yapilacagini duyurun" dedi. Duyuru yapildi.
\par 21 Yehu bütün Israil'e haber saldi. Baal'a tapanlarin hepsi geldi, gelmeyen kalmadi. Baal'in tapinagi hincahinç doldu.
\par 22 Yehu, kutsal giysiler görevlisine, "Baal'a tapanlarin hepsine giysi çikar" diye buyruk verdi. Görevli herkese giysi getirdi.
\par 23 O zaman Yehu Rekav oglu Yehonadav'la birlikte Baal'in tapinagina girdi. Içerdekilere, "Çevrenize iyi bakin" dedi, "Aranizda RAB'be tapanlardan kimse olmasin, sadece Baal'a tapanlar olsun."
\par 24 Ardindan Yehu'yla Yehonadav kurban ve yakmalik sunu* sunmak üzere içeri girdiler. Yehu tapinagin çevresine seksen kisi yerlestirmis ve onlara su buyrugu vermisti: "Elinize teslim ettigim bu adamlardan biri kaçarsa, bunu caninizla ödersiniz!"
\par 25 Yakmalik sununun sunulmasi biter bitmez, Yehu muhafizlarla komutanlara, "Içeriye girin, hepsini öldürün, hiçbiri kaçmasin!" diye buyruk verdi. Muhafizlarla komutanlar hepsini kiliçtan geçirip ölülerini disari attilar. Sonra Baal'in tapinaginin iç bölümüne girdiler.
\par 26 Baal'in tapinagindaki dikili taslari çikarip yaktilar.
\par 27 Baal'in dikili tasini ve tapinagini ortadan kaldirdilar. Halk orayi helaya çevirdi. Orasi bugüne kadar da öyle kaldi.
\par 28 Böylece Yehu Israil'de Baal'a tapmaya son verdi.
\par 29 Ne var ki, Nevat oglu Yarovam'in Israil'i sürükledigi günahlardan -Beytel ve Dan'daki altin buzagilara tapmaktan- vazgeçmedi.
\par 30 RAB Yehu'ya, "Gözümde dogru olani yaparak basarili oldun" dedi, "Ahav'in ailesine istedigim her seyi yaptin. Bunun için senin soyun dört kusak Israil tahtinda oturacak."
\par 31 Gelgelelim Yehu Israil'in Tanrisi RAB'bin yasasini yürekten izlemedi, önemsemedi. Yarovam'in Israil'i sürükledigi günahlardan ayrilmadi.
\par 32 RAB o günlerde Israil topraklarini küçültmeye basladi. Aram Krali Hazael Seria Irmagi'nin dogusunda Gadlilar, Rubenliler ve Manasseliler'in yasadigi bütün Gilat bölgesini, Arnon Vadisi'ndeki Aroer'den Gilat ve Basan'a kadar bütün Israil topraklarini ele geçirdi.
\par 34 Yehu'nun kralligi dönemindeki öteki olaylar, bütün yaptiklari ve basarilari Israil krallarinin tarihinde yazilidir.
\par 35 Yehu ölüp atalarina kavusunca, Samiriye'de gömüldü. Yerine oglu Yehoahaz kral oldu.
\par 36 Yehu Samiriye'de yirmi sekiz yil Israil kralligi yapti.

\chapter{11}

\par 1 Ahazya'nin annesi Atalya, oglunun öldürüldügünü duyunca, kral soyunun bütün bireylerini yok etmeye çalisti.
\par 2 Ne var ki, Kral Yehoram'in kizi, Ahazya'nin üvey kizkardesi Yehoseva, Ahazya oglu Yoas'i kralin öldürülmek istenen öteki ogullarinin arasindan alip kaçirdi ve dadisiyla birlikte yatak odasina gizledi. Çocugu Atalya'dan gizleyerek kurtarmis oldu.
\par 3 Atalya ülkeyi yönetirken, çocuk alti yil boyunca RAB'bin Tapinagi'nda dadisiyla birlikte gizlendi.
\par 4 Yedinci yil Yehoyada haber gönderip Karyalilar'in ve muhafizlarin yüzbasilarini çagirtti. Onlari RAB'bin Tapinagi'nda toplayarak onlarla bir antlasma yapti. Hepsine RAB'bin Tapinagi'nda ant içirdikten sonra kralin oglu Yoas'i kendilerine gösterdi.
\par 5 Onlara su buyruklari verdi: "Sabat Günü* göreve gidenlerin üçte biri kral sarayini koruyacak,
\par 6 üçte biri Sur Kapisi'nda, üçte biri de muhafizlarin arkasindaki kapida bulunacak. Sirayla tapinak nöbeti tutacaksiniz.
\par 7 Sabat Günü görevleri biten öbür iki bölükteki askerlerin tümü RAB'bin Tapinagi'nin çevresinde durup krali koruyacak.
\par 8 Herkes yalin kiliç kralin çevresini sarsin, yaklasan olursa öldürün. Kral nereye giderse, ona eslik edin."
\par 9 Yüzbasilar Kâhin Yehoyada'nin buyruklarini tam tamina uyguladilar. Sabat Günü göreve gidenlerle görevi biten adamlarini alip Yehoyada'nin yanina gittiler.
\par 10 Kâhin RAB'bin Tapinagi'ndaki Kral Davut'tan kalan mizraklarla kalkanlari yüzbasilara dagitti.
\par 11 Krali korumak için sunagin ve tapinagin çevresine tapinagin güneyinden kuzeyine kadar silahli muhafizlar yerlestirildi.
\par 12 Yehoyada kralin oglu Yoas'i disari çikarip basina taç koydu. Tanri'nin Yasasi'ni da ona verip kralligini ilan ettiler. Onu meshedip* alkislayarak, "Yasasin kral!" diye bagirdilar.
\par 13 Atalya muhafizlarla halkin çikardigi gürültüyü duyunca, RAB'bin Tapinagi'nda toplananlarin yanina gitti.
\par 14 Bakti, kral gelenege uygun olarak sütunun yaninda duruyor; yüzbasilar, borazan çalanlar çevresine toplanmis. Ülke halki sevinç içindeydi, borazanlar çaliniyordu. Atalya giysilerini yirtarak, "Hainlik! Hainlik!" diye bagirdi.
\par 15 Kâhin Yehoyada yüzbasilara, "O kadini aradan çikarin. Ardindan kim giderse kiliçtan geçirin" diye buyruk verdi. Çünkü kadinin RAB'bin Tapinagi'nda öldürülmesini istemiyordu.
\par 16 Atalya yakalandi ve sarayin At Kapisi'na götürülüp öldürüldü.
\par 17 Yehoyada RAB'bin halki olmalari için RAB ile kral ve halk arasinda bir antlasma yapti. Ayrica halkla kral arasinda da bir antlasma yapti.
\par 18 Ülke halki gidip Baal'in* tapinagini yikti. Sunaklarini, putlarini parçaladilar; Baal'in Kâhini Mattan'i da sunaklarin önünde öldürdüler. Kâhin Yehoyada RAB'bin Tapinagi'na nöbetçiler yerlestirdi.
\par 19 Sonra yüzbasilari, Karyalilar'i, muhafizlari ve halki yanina aldi. Krali RAB'bin Tapinagi'ndan getirdiler. Muhafizlar Kapisi'ndan geçerek sarayina götürdüler, kral tahtina oturttular.
\par 20 Ülke halki sevinç içindeydi, ancak kent suskundu. Çünkü Atalya sarayda kiliçla öldürülmüstü.

\chapter{12}

\par 1 Israil Krali Yehu'nun kralliginin yedinci yilinda Yoas Yahuda Krali oldu. Yedi yasinda kral oldu ve Yerusalim'de kirk yil krallik yapti. Annesi Beer-Sevali Sivya'ydi.
\par 2 Yoas Kâhin Yehoyada yasadigi sürece RAB'bin gözünde dogru olani yapti. Çünkü Kâhin Yehoyada ona yol gösteriyordu.
\par 3 Ancak alisilagelen tapinma yerleri henüz kaldirilmamisti ve halk oralarda hâlâ kurban kesip buhur yakiyordu.
\par 4 Yoas kâhinlere söyle dedi: "RAB'bin Tapinagi için yapilan bagislari: Nüfus sayimindan elde edilen geliri, kisi basina düsen vergiyi ve halkin gönüllü olarak RAB'bin Tapinagi'na sundugu paralari toplayin.
\par 5 Her kâhin bunlari hazine görevlilerinden alsin. Tapinagin neresinde yikik bir yer varsa, onarilsin."
\par 6 Yoas'in kralliginin yirmi üçüncü yilinda kâhinler tapinagi hâlâ onarmamislardi.
\par 7 Bunun üzerine Kral Yoas, Kâhin Yehoyada ile öbür kâhinleri çagirip, "Neden RAB'bin Tapinagi'ni onarmiyorsunuz?" diye sordu, "Hazine görevlilerinden artik para almayin. Aldiginiz paralari da RAB'bin Tapinagi'nin onarimina devredin."
\par 8 Böylece kâhinler halktan para toplamamayi ve tapinagin onarim islerine karismamayi kabul ettiler.
\par 9 Kâhin Yehoyada bir sandik aldi. Kapagina bir delik açip sunagin yanina, RAB'bin Tapinagi'na girenlerin sagina yerlestirdi. Kapida görevli kâhinler RAB'bin Tapinagi'na getirilen bütün paralari sandiga atiyorlardi.
\par 10 Sandikta çok para biriktigini görünce kralin yazmaniyla baskâhin RAB'bin Tapinagi'na getirilen paralari sayip torbalara koyarlardi.
\par 11 Sayilan paralar RAB'bin Tapinagi'ndaki islerin basinda bulunan adamlara verilirdi. Onlar da paralari RAB'bin Tapinagi'nda çalisan marangozlara, yapicilara,
\par 12 duvarcilara, tasçilara öder, tapinagi onarmak için kereste ve yontma tas aliminda kullanir ve onarim için gereken öbür malzemelere harcarlardi.
\par 13 RAB'bin Tapinagi'nda toplanan paralar, tapinak için gümüs tas, fitil masalari, çanak, borazan, altin ya da gümüs esya yapiminda kullanilmadi.
\par 14 Bu para yalniz isçilere ödendi ve tapinagin onarimina harcandi.
\par 15 Tapinakta çalisanlara para ödemekle görevli kisiler öyle dürüst insanlardi ki, onlara hesap bile sorulmazdi.
\par 16 Suç ve günah sunusu* olarak verilen paralarsa RAB'bin Tapinagi'na getirilmezdi; bunlar kâhinlere aitti.
\par 17 O sirada Aram Krali Hazael, Gat Kenti'ne saldirip kenti ele geçirdi. Sonra Yerusalim'e saldirmaya karar verdi.
\par 18 Yahuda Krali Yoas, atalari olan öbür Yahuda krallarindan Yehosafat'in, Yehoram'in, Ahazya'nin ve kendisinin RAB'be adamis oldugu bütün kutsal armaganlari, RAB'bin Tapinagi'nda ve sarayin hazinelerinde bulunan bütün altinlari Aram Krali Hazael'e gönderdi. Bunun üzerine Hazael Yerusalim'e saldirmaktan vazgeçti.
\par 19 Yoas'in kralligi dönemindeki öteki olaylar ve bütün yaptiklari Yahuda krallarinin tarihinde yazilidir.
\par 20 Kral Yoas'in görevlileri düzen kurup onu Silla'ya inen yolda, Beytmillo'da öldürdüler.
\par 21 Yoas'i öldürenler, görevlilerinden Simat oglu Yozakar'la Somer oglu Yehozavat'ti. Yoas Davut Kenti'nde atalarinin yanina gömüldü. Yerine oglu Amatsya kral oldu.

\chapter{13}

\par 1 Yahuda Krali Ahazya oglu Yoas'in kralliginin yirmi üçüncü yilinda Yehu oglu Yehoahaz Samiriye'de Israil Krali oldu ve on yedi yil krallik yapti.
\par 2 Yehoahaz RAB'bin gözünde kötü olani yapti ve Nevat oglu Yarovam'in Israil'i sürükledigi günahlara katildi. Bu günahlardan ayrilmadi.
\par 3 Iste bu yüzden RAB'bin Israil'e karsi öfkesi alevlendi ve RAB onlari uzun süre Aram Krali Hazael'le oglu Ben-Hadat'in egemenligi altina soktu.
\par 4 Bunun üzerine Yehoahaz RAB'be yakardi. RAB onun yakarisini kabul etti. Çünkü Israil'in çektigi sikintiyi, Aram Krali'nin onlara neler yaptigini görüyordu.
\par 5 RAB Israil'e bir kurtarici gönderdi. Böylece halk Aramlilar'in egemenliginden kurtuldu ve önceki gibi evlerinde yasamaya basladi.
\par 6 Ne var ki Yarovam ailesinin Israil'i sürükledigi günahlari izlediler, bu günahlardan ayrilmadilar. Asera* putu da Samiriye'de dikili kaldi.
\par 7 Yehoahaz'in elli atli, on savas arabasi, on bin de yaya asker disinda gücü kalmamisti. Çünkü Aram Krali ötekileri harman tozu gibi ezip yok etmisti.
\par 8 Yehoahaz'in kralligi dönemindeki öteki olaylar, bütün yaptiklari ve basarilari Israil krallarinin tarihinde yazilidir.
\par 9 Yehoahaz ölüp atalarina kavusunca, Samiriye'de gömüldü ve yerine oglu Yehoas kral oldu.
\par 10 Yahuda Krali Yoas'in kralliginin otuz yedinci yilinda Yehoahaz oglu Yehoas Samiriye'de Israil Krali oldu ve on alti yil krallik yapti.
\par 11 RAB'bin gözünde kötü olani yapti ve Nevat oglu Yarovam'in Israil'i sürükledigi günahlardan ayrilmadi; onun yolunu izledi.
\par 12 Yehoas'in kralligi dönemindeki öteki olaylar, bütün yaptiklari, Yahuda Krali Amatsya ile savasirken gösterdigi basarilar Israil krallarinin tarihinde yazilidir.
\par 13 Yehoas ölüp atalarina kavusunca, Samiriye'de Israil krallarinin yanina gömüldü ve tahtina Yarovam geçti.
\par 14 Elisa ölümcül bir hastaliga yakalandi. Israil Krali Yehoas gidip onu ziyaret etti, "Baba, baba, Israil'in arabasi ve atlilari!" diyerek agladi.
\par 15 Elisa ona, "Bir yayla birkaç ok al" dedi. Kral yayla oklari aldi.
\par 16 Elisa, "Yayi hazirla" dedi. Kral yayi hazirlayinca, Elisa ellerini kralin elleri üzerine koydu.
\par 17 "Doguya bakan pencereyi aç" dedi. Kral pencereyi açti. Elisa, "Oku at!" dedi. Kral oku atinca, Elisa, "Bu ok Aramlilar'a karsi sizi zafere ulastiracak RAB'bin kurtaris okudur" dedi, "Afek'te onlari kesin bozguna ugratacaksiniz."
\par 18 Sonra, "Öbür oklari al" dedi. Israil Krali oklari alinca, Elisa, "Onlari yere vur" dedi. Kral oklari üç kez yere vurdu ve durdu.
\par 19 Buna öfkelenen Tanri adami Elisa, "Bes alti kez vurmaliydin, o zaman Aramlilar'a karsi kesin bir zafer kazanirdin" dedi, "Ama simdi Aramlilar'i ancak üç kez bozguna ugratacaksin."
\par 20 Elisa öldü ve gömüldü. Her ilkbaharda Moav akincilari Israil topraklarina girerlerdi.
\par 21 Bir keresinde Israilliler, ölü gömerken akincilarin geldigini görünce, ölüyü Elisa'nin mezarina atip kaçtilar. Ölü Elisa'nin kemiklerine dokununca dirilip ayaga kalkti.
\par 22 Aram Krali Hazael Yehoahaz'in kralligi boyunca Israilliler'e baski yapti.
\par 23 Ama RAB, Ibrahim, Ishak ve Yakup'la yaptigi antlasmadan ötürü, Israilliler'e iyilik etti. Onlara aciyip yardim etti. Yok olmalarina izin vermedi. Onlari simdiye kadar huzurundan atmadi.
\par 24 Aram Krali Hazael ölünce yerine oglu Ben-Hadat kral oldu.
\par 25 Bunun üzerine Israil Krali Yehoas, babasi Yehoahaz'in kralligi döneminde Hazael oglu Ben-Hadat'in savasta ele geçirmis oldugu kentleri geri aldi. Onu üç kez bozguna ugratip Israil kentlerine yeniden sahip oldu.

\chapter{14}

\par 1 Israil Krali Yehoahaz oglu Yehoas'in kralliginin ikinci yilinda Yoas oglu Amatsya Yahuda Krali oldu.
\par 2 Amatsya yirmi bes yasinda kral oldu ve Yerusalim'de yirmi dokuz yil krallik yapti. Annesi Yerusalimli Yehoaddan'di.
\par 3 Amatsya RAB'bin gözünde dogru olani yaptiysa da atasi Davut gibi degildi. Her konuda babasi Yoas'i örnek aldi.
\par 4 Ancak alisilagelen tapinma yerleri henüz kaldirilmamisti ve halk oralarda hâlâ kurban kesip buhur yakiyordu.
\par 5 Amatsya kralligini güçlendirdikten sonra, babasini öldüren görevlileri ortadan kaldirdi.
\par 6 Ancak Musa'nin Kitabi'ndaki yasaya uyarak katillerin çocuklarini öldürtmedi. Çünkü RAB, "Ne babalar çocuklarinin yerine öldürülecek, ne de çocuklar babalarinin yerine. Herkes kendi günahi için öldürülecek" diye buyurmustu.
\par 7 Amatsya Tuz Vadisi'nde on bin Edomlu asker öldürdü. Savasarak Sela'yi ele geçirdi ve oraya Yokteel adini verdi. Orasi hâlâ ayni adla anilmaktadir.
\par 8 Bundan sonra Amatsya, Yehu oglu Yehoahaz oglu Israil Krali Yehoas'a, "Gel, yüz yüze görüselim" diye haber gönderdi.
\par 9 Israil Krali Yehoas da karsilik olarak Yahuda Krali Amatsya'ya su haberi gönderdi: "Lübnan'da dikenli bir çali, sedir agacina, 'Kizini ogluma es olarak ver diye haber yollar. O sirada oradan geçen yabanil bir hayvan basip çaliyi çigner.
\par 10 Edomlular'i bozguna ugrattin diye böbürleniyorsun. Bu zafer sana yeter. Otur evinde! Niçin bela ariyorsun? Kendi basini da, Yahuda halkinin basini da derde sokacaksin."
\par 11 Ne var ki, Amatsya dinlemek istemedi. Derken Israil Krali Yehoas Yahuda Krali Amatsya'nin üzerine yürüdü. Iki ordu Yahuda'nin Beytsemes Kenti'nde karsilasti.
\par 12 Israilliler'in önünde bozguna ugrayan Yahudalilar evlerine kaçti.
\par 13 Israil Krali Yehoas, Ahazya oglu Yoas oglu Yahuda Krali Amatsya'yi Beytsemes'te yakaladi. Sonra Yerusalim'e girip Efrayim Kapisi'ndan Köse Kapisi'na kadar Yerusalim surlarinin dört yüz arsinlik bölümünü yiktirdi.
\par 14 RAB'bin Tapinagi'nda ve sarayin hazinelerinde buldugu altini, gümüsü ve bütün esyalari aldi. Ayrica bazi adamlari da rehine olarak yanina alip Samiriye'ye döndü.
\par 15 Yehoas'in kralligi dönemindeki öteki olaylar, yaptiklari ve Yahuda Krali Amatsya ile savasirken gösterdigi basarilar Israil krallarinin tarihinde yazilidir.
\par 16 Yehoas ölüp atalarina kavusunca, Samiriye'de Israil krallarinin yanina gömüldü. Yerine oglu Yarovam kral oldu.
\par 17 Yahuda Krali Yoas oglu Amatsya, Israil Krali Yehoahaz oglu Yehoas'in ölümünden sonra on bes yil daha yasadi.
\par 18 Amatsya'nin kralligi dönemindeki öteki olaylar Yahuda krallarinin tarihinde yazilidir.
\par 19 Yerusalim'de Amatsya'ya bir düzen kurulmustu. Amatsya Lakis'e kaçti. Ardindan adam göndererek onu öldürttüler.
\par 20 Ölüsü at sirtinda Yerusalim'e getirildi, Davut Kenti'nde atalarinin yanina gömüldü.
\par 21 Yahuda halki Amatsya'nin yerine on alti yasindaki oglu Azarya'yi kral yapti.
\par 22 Babasi Amatsya ölüp atalarina kavustuktan sonra Azarya Eylat Kenti'ni onarip Yahuda topraklarina katti.
\par 23 Yahuda Krali Yoas oglu Amatsya'nin kralliginin on besinci yilinda Yehoas oglu Yarovam Samiriye'de Israil Krali oldu ve kirk bir yil krallik yapti.
\par 24 Yarovam RAB'bin gözünde kötü olani yapti ve Nevat oglu Yarovam'in Israil'i sürükledigi günahlardan ayrilmadi.
\par 25 Israil'in Tanrisi RAB'bin, kulu Gat-Heferli Amittay oglu Yunus Peygamber araciligiyla söyledigi söz uyarinca, Yarovam Levo-Hamat'tan Arava Gölü'ne kadar Israil topraklarini yeniden ele geçirdi.
\par 26 RAB Israil'in çektigi sikintiyi görmüstü. Genç yasli herkes aci içinde kivraniyordu. Israil'e yardim edecek kimse yoktu.
\par 27 RAB Israil'in adini yeryüzünden silmek istemiyordu. Onun için, Yehoas oglu Yarovam araciligiyla onlari kurtardi.
\par 28 Yarovam'in kralligi dönemindeki öteki olaylar, bütün yaptiklari, askeri basarilari, eskiden Yaudi'ye ait olan Sam ve Hama kentlerini yeniden Israil topraklarina katisi Israil krallarinin tarihinde yazilidir.
\par 29 Yarovam ölüp atalari olan Israil krallarina kavustu. Yerine oglu Zekeriya kral oldu.

\chapter{15}

\par 1 Israil Krali Yarovam'in kralliginin yirmi yedinci yilinda Amatsya oglu Azarya Yahuda Krali oldu.
\par 2 Azarya on alti yasinda kral oldu ve Yerusalim'de elli iki yil krallik yapti. Annesi Yerusalimli Yekolya'ydi.
\par 3 Babasi Amatsya gibi, Azarya da RAB'bin gözünde dogru olani yapti.
\par 4 Ancak alisilagelen tapinma yerleri henüz kaldirilmamisti ve halk oralarda hâlâ kurban kesip buhur yakiyordu.
\par 5 RAB Kral Azarya'yi cezalandirdi. Kral ölünceye kadar deri hastaligindan kurtulamadi. Bu yüzden ayri bir evde yasadi. Sarayi ve ülke halkini oglu Yotam yönetti.
\par 6 Azarya'nin kralligi dönemindeki öteki olaylar ve bütün yaptiklari Yahuda krallarinin tarihinde yazilidir.
\par 7 Azarya ölüp atalarina kavusunca, Davut Kenti'nde atalarinin yanina gömüldü. Yerine oglu Yotam kral oldu.
\par 8 Yahuda Krali Azarya'nin kralliginin otuz sekizinci yilinda Yarovam oglu Zekeriya Samiriye'de Israil Krali oldu ve alti ay krallik yapti.
\par 9 Atalari gibi, RAB'bin gözünde kötü olani yapti ve Nevat oglu Yarovam'in Israil'i sürükledigi günahlardan ayrilmadi.
\par 10 Yaves oglu Sallum Zekeriya'ya bir düzen kurdu; halkin önünde saldirip onu öldürdü, yerine kendisi kral oldu.
\par 11 Zekeriya'nin kralligi dönemindeki öteki olaylar Israil krallarinin tarihinde yazilidir.
\par 12 Böylece RAB'bin Yehu'ya, "Senin soyun dört kusak Israil tahtinda oturacak" diye verdigi söz yerine gelmis oldu. Sallum'un Israil Kralligi
\par 13 Yahuda Krali Azarya'nin kralliginin otuz dokuzuncu yilinda Yaves oglu Sallum Israil Krali oldu ve Samiriye'de bir ay krallik yapti.
\par 14 Gadi oglu Menahem Tirsa'dan Samiriye'ye gelip Yaves oglu Sallum'a saldirdi. Onu öldürüp yerine kendisi kral oldu.
\par 15 Sallum'un kralligi dönemindeki öteki olaylar ve kurdugu düzen Israil krallarinin tarihinde yazilidir.
\par 16 O sirada Menahem, Tirsa Kenti'nden baslayarak, Tifsah Kenti'ne ve çevresinde yasayan herkese saldirdi. Çünkü kentin kapisini kendisine açmamislardi. Herkesi öldürüp gebe kadinlarin karinlarini bile yardi.
\par 17 Yahuda Krali Azarya'nin kralliginin otuz dokuzuncu yilinda Gadi oglu Menahem Israil Krali oldu ve Samiriye'de on yil krallik yapti.
\par 18 RAB'bin gözünde kötü olani yapti ve yasami boyunca Nevat oglu Yarovam'in Israil'i sürükledigi günahlardan ayrilmadi.
\par 19 Asur Krali Tiglat-Pileser Israil'e saldirdi. Menahem, Tiglat-Pileser'in destegini saglayip kralligini güçlendirmek için, ona bin talant gümüs verdi.
\par 20 Israil'deki bütün zenginleri adam basi elli sekel gümüs ödemekle yükümlü kilarak Asur Krali'na verilen gümüsü karsiladi. Böylece Asur Krali Israil topraklarindan çekilip ülkesine döndü.
\par 21 Menahem'in kralligi dönemindeki öteki olaylar ve bütün yaptiklari Israil krallarinin tarihinde yazilidir.
\par 22 Menahem ölüp atalarina kavusunca, yerine oglu Pekahya kral oldu.
\par 23 Yahuda Krali Azarya'nin kralliginin ellinci yilinda Menahem oglu Pekahya Samiriye'de Israil Krali oldu ve iki yil krallik yapti.
\par 24 Pekahya RAB'bin gözünde kötü olani yapti ve Nevat oglu Yarovam'in Israil'i sürükledigi günahlardan ayrilmadi.
\par 25 Komutanlarindan biri olan Remalya'nin oglu Pekah kendisine düzen kurdu. Argov ve Arye'nin isbirligiyle yanina Gilatli elli adam alarak Pekahya'yi Samiriye'deki sarayin kalesinde öldürdü, yerine kendisi kral oldu.
\par 26 Pekahya'nin kralligi dönemindeki öteki olaylar ve bütün yaptiklari Israil krallarinin tarihinde yazilidir.
\par 27 Yahuda Krali Azarya'nin kralliginin elli ikinci yilinda Remalya oglu Pekah Samiriye'de Israil Krali oldu ve yirmi yil krallik yapti.
\par 28 RAB'bin gözünde kötü olani yapti ve Nevat oglu Yarovam'in Israil'i sürükledigi günahlardan ayrilmadi.
\par 29 Israil Krali Pekah'in kralligi sirasinda, Asur Krali Tiglat-Pileser Israil'in Iyon, Avel-Beytmaaka, Yanoah, Kedes, Hasor kentleriyle Gilat, Celile ve Naftali bölgelerini ele geçirerek halki Asur'a sürdü.
\par 30 Yahuda Krali Azarya oglu Yotam'in kralliginin yirminci yilinda Ela oglu Hosea, Remalya oglu Pekah'a düzen kurdu ve onu öldürüp yerine kendisi kral oldu.
\par 31 Pekah'in kralligi dönemindeki öteki olaylar ve bütün yaptiklari Israil krallarinin tarihinde yazilidir.
\par 32 Israil Krali Remalya oglu Pekah'in kralliginin ikinci yilinda Azarya oglu Yotam Yahuda Krali oldu.
\par 33 Yotam yirmi bes yasinda kral oldu ve Yerusalim'de on alti yil krallik yapti. Annesi Sadok'un kizi Yerusa'ydi.
\par 34 Babasi Azarya gibi, Yotam da RAB'bin gözünde dogru olani yapti.
\par 35 Ancak alisilagelen tapinma yerleri henüz kaldirilmamisti ve halk oralarda hâlâ kurban kesip buhur yakiyordu. Yotam RAB'bin Tapinagi'nin Yukari Kapisi'ni onardi.
\par 36 Yotam'in kralligi dönemindeki öteki olaylar ve yaptiklari Yahuda krallarinin tarihinde yazilidir.
\par 37 RAB iste o günlerde Aram Krali Resin'le Israil Krali Remalya oglu Pekah'i Yahuda üzerine göndermeye basladi.
\par 38 Yotam ölüp atalarina kavusunca, atasi Davut'un Kenti'nde atalarinin yanina gömüldü. Yerine oglu Ahaz kral oldu.

\chapter{16}

\par 1 Israil Krali Remalya oglu Pekah'in kralliginin on yedinci yilinda Yotam oglu Ahaz Yahuda Krali oldu.
\par 2 Ahaz yirmi yasinda kral oldu ve Yerusalim'de on alti yil krallik yapti. Tanrisi RAB'bin gözünde dogru olani yapan atasi Davut gibi davranmadi.
\par 3 Israil krallarinin yolunu izledi; hatta RAB'bin Israil halkinin önünden kovmus oldugu uluslarin igrenç törelerine uyarak oglunu ateste kurban etti.
\par 4 Puta tapilan yerlerde, tepelerde, bol yaprakli her agacin altinda kurban kesip buhur yakti.
\par 5 Aram Krali Resin'le Israil Krali Remalya oglu Pekah Yerusalim'e yürüdüler. Kenti kusattilarsa da Ahaz'i yenemediler.
\par 6 O sirada Aram Krali Resin Eylat'i geri alip Yahudalilar'i oradan sürdü. Edomlular Eylat'a yerlesti. Bugün de orada yasiyorlar.
\par 7 Ahaz, Asur Krali Tiglat-Pileser'e: "Senin kulun kölenim; gel, bana saldiran Aram ve Israil krallarinin elinden beni kurtar" diye ulaklar gönderdi.
\par 8 RAB'bin Tapinagi'nda ve sarayin hazinelerinde bulunan altin ve gümüsü armagan olarak Asur Krali'na gönderdi.
\par 9 Asur Krali Ahaz'in istegini olumlu karsiladi, saldirip Sam'i ele geçirdi. Kent halkini Kîr'e sürüp Resin'i öldürdü.
\par 10 Kral Ahaz, Asur Krali Tiglat-Pileser'i karsilamak için Sam'a gittiginde, oradaki sunagi gördü. Aynisini yaptirmak için sunagin bütün ayrintilarini gösteren bir plani ve maketi Kâhin Uriya'ya gönderdi.
\par 11 Kâhin Uriya, Kral Ahaz Sam'dan dönünceye kadar, onun göndermis oldugu maketin tipkisi bir sunak yapti.
\par 12 Kral Sam'dan dönünce sunagi gördü, yaklasip üzerinde sunular sundu.
\par 13 Yakmalik sunulari* ve tahil sunularini* sundu, dökmelik sunuyu bosaltti, esenlik sunusunun* kanini sunagin üzerine döktü.
\par 14 RAB'bin huzurundaki tunç* sunagi tapinagin önünden, yeni sunakla tapinagin arasindaki yerinden getirtip yeni sunagin kuzeyine yerlestirdi.
\par 15 Kral Ahaz Kâhin Uriya'ya su buyruklari verdi: "Sabahin yakmalik sunusuyla aksamin tahil sunusunu, kralin yakmalik ve tahil sunusunu, ayrica ülke halkinin yakmalik, tahil ve dökmelik sunularini bu büyük sunagin üzerinde sun; yakmalik sunularin ve kurbanlarin kanini onun üzerine dök. Ama tunç sunagi gelecegi bilmek için kendim kullanacagim."
\par 16 Kâhin Uriya Kral Ahaz'in bütün buyruklarini yerine getirdi.
\par 17 Kral Ahaz tapinaktaki kazanlarin üzerine oturdugu ayakliklarin yan aynaliklarini söküp kazanlari kaldirdi. Havuzu tunç bogalarin üzerinden indirip tas bir döseme üzerine yerlestirdi.
\par 18 Asur Krali'ni hosnut etmek için RAB'bin Tapinagi'na konan kral kürsüsünün setini kaldirip kralin tapinaga girmek için kullandigi özel kapiyi kapatti.
\par 19 Ahaz'in kralligi dönemindeki öteki olaylar ve yaptiklari Yahuda krallarinin tarihinde yazilidir.
\par 20 Ahaz ölüp atalarina kavusunca, Davut Kenti'nde atalarinin yanina gömüldü. Yerine oglu Hizkiya kral oldu.

\chapter{17}

\par 1 Yahuda Krali Ahaz'in kralliginin on ikinci yilinda Ela oglu Hosea Samiriye'de Israil Krali oldu ve dokuz yil krallik yapti.
\par 2 RAB'bin gözünde kötü olani yapti, ama kendisinden önceki Israil krallari kadar kötü degildi.
\par 3 Asur Krali Salmaneser Hosea'ya savas açti. Hosea teslim olup haraç ödemeye basladi.
\par 4 Ancak Asur Krali Hosea'nin hainlik yaptigini ögrendi. Çünkü Hosea Misir Firavunu So'nun destegini saglamak için ona ulaklar göndermis, üstelik her yil ödemesi gereken haraçlari da Asur Krali'na ödememisti. Bunun üzerine Asur Krali onu yakalayip cezaevine kapadi.
\par 5 Asur Krali Israil topraklarina saldirdi. Samiriye'yi kusatti. Kusatma üç yil sürdü.
\par 6 Hosea'nin kralliginin dokuzuncu yilinda Asur Krali Samiriye'yi ele geçirdi. Israil halkini Asur'a sürdü. Onlari Halah'a, Habur Irmagi kiyisindaki Gozan'a ve Med kentlerine yerlestirdi.
\par 7 Bütün bunlar kendilerini Misir Firavunu'nun boyundurugundan kurtarip Misir'dan çikaran Tanrilari RAB'be karsi günah isledikleri için Israilliler'in basina geldi. Çünkü baska ilahlara tapmislar,
\par 8 RAB'bin Israil halkinin önünden kovmus oldugu uluslarin törelerine ve Israil krallarinin koydugu kurallara göre yasamislardi.
\par 9 Tanrilari RAB'bin onaylamadigi bu isleri gizlilik içinde yapmislar, gözcü kulelerinden surlu kentlere kadar her yerde tapinma yerleri kurmuslardi.
\par 10 Her yüksek tepenin üzerine, bol yaprakli her agacin altina dikili taslar, Asera* putlari diktiler.
\par 11 RAB'bin onlarin önünden kovmus oldugu uluslarin yaptigi gibi, bütün tapinma yerlerinde buhur yaktilar. Yaptiklari kötülüklerle RAB'bi öfkelendirdiler.
\par 12 RAB'bin, "Bunu yapmayacaksiniz" demis olmasina karsin putlara taptilar.
\par 13 RAB Israil ve Yahuda halkini bütün peygamberler ve biliciler* araciligiyla uyarmis, onlara, "Bu kötü yollarinizdan dönün" demisti, "Atalariniza buyurdugum ve kullarim peygamberler araciligiyla size gönderdigim Kutsal Yasa'nin tümüne uyarak buyruklarimi, kurallarimi yerine getirin."
\par 14 Ama dinlemediler, Tanrilari RAB'be güvenmeyen atalari gibi inat ettiler.
\par 15 Tanri'nin kurallarini, uyarilarini ve atalariyla yaptigi antlasmayi hiçe sayarak degersiz putlarin ardinca gittiler, böylece kendi degerlerini de yitirdiler. Çevrelerindeki uluslar gibi yasamamalari için RAB kendilerine buyruk verdigi halde, uluslarin törelerine göre yasadilar.
\par 16 Tanrilari RAB'bin bütün buyruklarini terk ettiler. Tapinmak için kendilerine iki dökme buzagi ve Asera* putu yaptirdilar. Gök cisimlerine taptilar. Baal'a* kulluk ettiler.
\par 17 Ogullarini, kizlarini ateste kurban ettiler. Falcilik, büyücülük yaptilar. RAB'bin gözünde kötü olani yaptilar, kendilerini kötülüge adayarak O'nu öfkelendirdiler.
\par 18 RAB Israilliler'e çok kizdi, Yahuda oymagi disinda hepsini huzurundan kovdu.
\par 19 Yahudalilar bile Tanrilari RAB'bin buyruklarina uymadilar. Israilliler'in benimsedigi törelere göre yasadilar.
\par 20 Bundan dolayi RAB Israil soyundan olan herkesi reddetti. Çapulcularin eline teslim ederek onlari cezalandirdi. Hepsini huzurundan kovdu.
\par 21 RAB Israil'i Davut soyunun elinden aldiktan sonra, Israilliler Nevat oglu Yarovam'i kral yaptilar. Yarovam Israilliler'i RAB'bin yolundan saptirarak büyük günaha sürükledi.
\par 22 Israilliler Yarovam'in isledigi bütün günahlara katildilar ve bunlardan ayrilmadilar.
\par 23 Sonunda RAB kullari peygamberler araciligiyla uyarmis oldugu gibi, onlari huzurundan kovdu. Israilliler kendi topraklarindan Asur'a sürüldüler. Bugün de orada yasiyorlar.
\par 24 Asur Krali Israilliler'in yerine Babil, Kuta, Avva, Hama ve Sefarvayim'den insanlar getirtip Samiriye kentlerine yerlestirdi. Bunlar Samiriye'yi mülk edinip oradaki kentlerde yasamaya basladilar.
\par 25 Oralara ilk yerlestiklerinde RAB'be tapinmadilar. Bu yüzden RAB aslanlar göndererek bazilarini öldürttü.
\par 26 Asur Krali'na, "Sürdügün ve Samiriye kentlerine yerlestirdigin uluslar Samiriye ilahinin yasasini bilmiyorlar. O da üzerlerine aslanlar gönderiyor" diye haber salindi, "Bu yüzden aslanlara yem oluyorlar. Çünkü ülke ilahinin yasasindan haberleri yok."
\par 27 Bunun üzerine Asur Krali su buyrugu verdi: "Samiriye'den sürülen kâhinlerden birini geri gönderin, gidip orada yasasin ve ülke ilahinin yasasini onlara ögretsin."
\par 28 Samiriye'den sürülen kâhinlerden biri gelip Beytel'e yerlesti ve RAB'be nasil tapinacaklarini onlara ögretmeye basladi.
\par 29 Gelgelelim Samiriye kentlerine yerlesen her ulus kendi ilahlarini yapti. Samiriyeliler'in yapmis oldugu tapinma yerlerindeki yapilara bu ilahlari koydular.
\par 30 Babil halki Sukkot-Benot, Kuta halki Nergal, Hama halki Asima,
\par 31 Avva halki ise Nivhaz ve Tartak adindaki ilahlarini yaptilar. Sefarvayim halki ise ogullarini ilahlari Adrammelek ve Anammelek'e yakarak kurban ettiler.
\par 32 Bir yandan RAB'be tapiniyor, öte yandan tapinma yerlerindeki yapilarda görev yapmak üzere aralarindan rasgele kâhinler seçiyorlardi.
\par 33 Böylece hem RAB'be tapiniyorlar, hem de aralarindan geldikleri uluslarin törelerine göre kendi ilahlarina kulluk ediyorlardi.
\par 34 Bugün de eski törelerine göre yasiyorlar. Ne RAB'be tapiniyorlar, ne de RAB'bin Israil adini verdigi Yakup'un ogullari için koymus oldugu kurallara, ilkelere, yasalara, buyruklara uyuyorlar.
\par 35 RAB Yakupogullari'yla antlasma yapmis ve onlara söyle buyurmustu: "Baska ilahlara tapmayacak, önlerinde egilmeyecek, onlara kulluk etmeyecek, kurban kesmeyeceksiniz.
\par 36 Yalnizca ulu gücüyle her yere erisen eliyle sizleri Misir'dan çikaran RAB'be tapinacaksiniz. O'nun önünde egilip O'na kurban keseceksiniz.
\par 37 Sizler için yazmis oldugu kurallari, ilkeleri, yasalari, buyruklari her zaman yerine getirmeye özen gösterecek ve baska ilahlara tapmayacaksiniz.
\par 38 Sizinle yaptigim antlasmayi unutmayacak ve baska ilahlara tapmayacaksiniz.
\par 39 Yalniz Tanriniz RAB'be tapacaksiniz. O sizi bütün düsmanlarinizin elinden kurtaracak."
\par 40 Ne var ki Samiriye'ye yerlesenler buna kulak asmadilar ve eski törelerine göre yasamaya devam ettiler.
\par 41 Bu uluslar ayni zamanda hem RAB'be, hem de putlarina tapiyorlardi. Çocuklari ve torunlari da bugüne dek atalari gibi yasiyorlar.

\chapter{18}

\par 1 Israil Krali Ela oglu Hosea'nin kralliginin üçüncü yilinda Ahaz oglu Hizkiya Yahuda Krali oldu.
\par 2 Hizkiya yirmi bes yasinda kral oldu ve Yerusalim'de yirmi dokuz yil krallik yapti. Annesi Zekeriya'nin kizi Aviya'ydi.
\par 3 Atasi Davut gibi, o da RAB'bin gözünde dogru olani yapti.
\par 4 Alisilagelen tapinma yerlerini kaldirdi, dikili taslari, Asera* putlarini parçaladi. Musa'nin yapmis oldugu Nehustan adindaki tunç* yilani da parçaladi. Çünkü Israilliler o güne kadar ona buhur yakiyorlardi.
\par 5 Hizkiya Israil'in Tanrisi RAB'be güvendi. Kendisinden önceki ve sonraki Yahuda krallari arasinda onun gibisi yoktu.
\par 6 RAB'be çok bagliydi, O'nun yolundan ayrilmadi, RAB'bin Musa'ya vermis oldugu buyruklari yerine getirdi.
\par 7 RAB onunla birlikteydi. Yaptigi her iste basarili oldu. Asur Krali'na karsi ayaklandi ve ona kulluk etmedi.
\par 8 Gazze ve çevresine, gözcü kulelerinden surlu kentlere kadar her yerde Filistliler'i bozguna ugratti.
\par 9 Hizkiya'nin kralliginin dördüncü yilinda -Israil Krali Ela oglu Hosea'nin kralliginin yedinci yili- Asur Krali Salmaneser Samiriye'ye yürüyerek kenti kusatti.
\par 10 Kusatma üç yil sürdü. Sonunda Samiriye'yi ele geçirdiler. Hizkiya'nin kralliginin altinci yili, Israil Krali Hosea'nin kralliginin dokuzuncu yilinda Samiriye alindi.
\par 11 Asur Krali Israilliler'i Asur'a sürerek Halah'a, Habur Irmagi kiyisindaki Gozan'a ve Med kentlerine yerlestirdi.
\par 12 Çünkü Tanrilari RAB'bin sözünü dinlememisler, O'nun antlasmasini ve RAB'bin kulu Musa'nin buyruklarini çignemislerdi. Ne kulak asmislar, ne de buyruklari yerine getirmislerdi.
\par 13 Hizkiya'nin kralliginin on dördüncü yilinda Asur Krali Sanherib, Yahuda'nin surlu kentlerine saldirip hepsini ele geçirdi.
\par 14 Yahuda Krali Hizkiya, Lakis Kenti'ndeki Asur Krali'na su haberi gönderdi: "Suçluyum, üzerimden kuvvetlerini çek, ne istersen ödeyecegim." Asur Krali Yahuda Krali Hizkiya'yi üç yüz talant*fç* gümüs ve otuz talant*fd* altin ödemekle yükümlü kildi.
\par 15 Hizkiya RAB'bin Tapinagi'nda ve kral sarayinin hazinelerinde bulunan bütün gümüsü ona verdi.
\par 16 Daha önce yaptirmis oldugu RAB'bin Tapinagi'nin kapilariyla kapi pervazlarinin üzerindeki altin kaplamalari da çikarip Asur Krali'na verdi.
\par 17 Asur Krali baskomutan, askeri danisman ve komutanini büyük bir orduyla Lakis'ten Yerusalim'e, Kral Hizkiya'ya gönderdi. Yerusalim'e varan ordu Çirpici Tarlasi yolunda, Yukari Havuz'un su yolunun yaninda durdu.
\par 18 Haber gönderip Kral Hizkiya'yi çagirdilar. Saray sorumlusu Hilkiya oglu Elyakim, Yazman Sevna ve devlet tarihçisi Asaf oglu Yoah Asurlular'i karsilamaya çikti.
\par 19 Komutan onlara söyle dedi: "Hizkiya'ya söyleyin. 'Büyük kral, Asur Krali diyor ki: Güvendigin sey ne, neye güveniyorsun?
\par 20 Savas tasarilarin ve gücün oldugunu söylüyorsun, ama bunlar bos sözler. Kime güveniyorsun da bana karsi ayaklaniyorsun?
\par 21 Iste sen su kirik kamis degnege, Misir'a güveniyorsun. Bu degnek kendisine yaslanan herkesin eline batar, deler. Firavun da kendisine güvenenler için böyledir.
\par 22 Yoksa bana, Tanrimiz RAB'be güveniyoruz mu diyeceksiniz? Hizkiya'nin Yahuda ve Yerusalim halkina, yalniz Yerusalim'de, bu sunagin önünde tapinacaksiniz diyerek tapinma yerlerini, sunaklarini ortadan kaldirdigi Tanri degil mi bu?
\par 23 "Haydi, efendim Asur Krali'yla bahse giris. Binicileri saglayabilirsen sana iki bin at veririm.
\par 24 Misir'in savas arabalariyla atlilari saglayacagina güvensen bile, efendimin en küçük rütbeli komutanlarindan birini yenemezsin!
\par 25 Dahasi var: RAB'bin buyrugu olmadan mi saldirip burayi yikmak için yola çiktigimi saniyorsun? RAB, 'Git, o ülkeyi yik dedi."
\par 26 Hilkiya oglu Elyakim, Sevna ve Yoah, "Lütfen biz kullarinla Aramice* konus" diye karsilik verdiler, "Çünkü biz bu dili anlariz. Yahudi dilinde konusma. Surlarin üzerindeki halk bizi dinliyor."
\par 27 Komutan, "Efendim bu sözleri yalniz size ve efendinize söyleyeyim diye mi gönderdi beni?" dedi, "Surlarin üzerinde oturan bu halka, sizin gibi diskisini yemek, idrarini içmek zorunda kalacak olan herkese gönderdi."
\par 28 Sonra ayaga kalkip Yahudi dilinde bagirdi: "Büyük kralin, Asur Krali'nin söylediklerini dinleyin!
\par 29 Kral diyor ki, 'Hizkiya sizi aldatmasin, o sizi benim elimden kurtaramaz.
\par 30 RAB bizi mutlaka kurtaracak, bu kent Asur Krali'nin eline geçmeyecek diyen Hizkiya'ya kanmayin, RAB'be güvenmeyin.
\par 31 Hizkiya'yi dinlemeyin. Çünkü Asur Krali diyor ki, 'Teslim olun, bana gelin. Böylece ben gelip sizi zeytinyagi ve bal ülkesi olan kendi ülkeniz gibi bir ülkeye -tahil ve yeni sarap, ekmek ve üzüm dolu bir ülkeye- götürene kadar herkes kendi asmasindan, kendi incir agacindan yiyecek, kendi sarnicindan içecek. Yasami seçin, ölümü degil. RAB bizi kurtaracak diyerek sizi aldatmaya çalisan Hizkiya'yi dinlemeyin.
\par 33 Uluslarin ilahlari ülkelerini Asur Krali'nin elinden kurtarabildi mi?
\par 34 Hani nerede Hama'nin, Arpat'in ilahlari? Sefarvayim'in, Hena ve Ivva'nin ilahlari nerede? Samiriye'yi elimden kurtarabildiler mi?
\par 35 Bütün ülkelerin ilahlarindan hangisi ülkesini elimden kurtardi ki, RAB Yerusalim'i elimden kurtarabilsin?"
\par 36 Halk sustu, komutana tek sözle bile karsilik veren olmadi. Çünkü Kral Hizkiya, "Karsilik vermeyin" diye buyurmustu.
\par 37 Sonra saray sorumlusu Hilkiya oglu Elyakim, Yazman Sevna ve devlet tarihçisi Asaf oglu Yoah giysilerini yirttilar ve gidip komutanin söylediklerini Hizkiya'ya bildirdiler.

\chapter{19}

\par 1 Kral Hizkiya olanlari duyunca giysilerini yirtti, çul kusanip RAB'bin Tapinagi'na girdi.
\par 2 Saray sorumlusu Elyakim'i, Yazman Sevna'yi ve ileri gelen kâhinleri Amots oglu Peygamber Yesaya'ya gönderdi. Hepsi çul kusanmisti.
\par 3 Yesaya'ya söyle dediler: "Hizkiya diyor ki, 'Bugün sikinti, azar ve utanç günü. Çünkü çocuklarin dogum vakti geldi, ama doguracak güç yok.
\par 4 Yasayan Tanri'yi asagilamak için efendisi Asur Krali'nin gönderdigi komutanin söylediklerini belki Tanrin RAB duyar da duydugu sözlerden ötürü onlari cezalandirir. Bu nedenle sag kalanlarimiz için dua et."
\par 5 Yesaya, Kral Hizkiya'dan gelen görevlilere söyle dedi: "Efendinize sunlari söyleyin: 'RAB diyor ki, Asur Krali'nin adamlarindan benimle ilgili duydugunuz küfürlerden korkma.
\par 7 Onun içine öyle bir ruh koyacagim ki, bir haber üzerine kendi ülkesine dönecek. Orada onu kiliçla öldürtecegim."
\par 8 Komutan, Asur Krali'nin Lakis'ten ayrilip Livna'ya karsi savastigini duydu. Krala danismak için oraya gitti.
\par 9 Kûs* Krali Tirhaka'nin kendisiyle savasmak üzere yola çiktigini haber alan Asur Krali, Hizkiya'ya yine ulaklar göndererek söyle dedi:
\par 10 "Yahuda Krali Hizkiya'ya deyin ki, 'Güvendigin Tanrin, Yerusalim Asur Krali'nin eline teslim edilmeyecek diyerek seni aldatmasin.
\par 11 Asur krallarinin bütün ülkelere neler yaptigini, onlari nasil yerle bir ettigini duymussundur. Sen kurtulacagini mi saniyorsun?
\par 12 Atalarimin yok ettigi uluslari -Gozanlilar'i, Harranlilar'i, Resefliler'i, Telassar'da yasayan Edenliler'i- ilahlari kurtarabildi mi?
\par 13 Hani nerede Hama ve Arpat krallari? Lair, Sefarvayim, Hena, Ivva krallari nerede?"
\par 14 Hizkiya mektubu ulaklarin elinden alip okuduktan sonra RAB'bin Tapinagi'na çikti. RAB'bin önünde mektubu yere yayarak
\par 15 söyle dua etti: "Ey Keruvlar* arasinda taht kuran Israil'in Tanrisi RAB, bütün dünya kralliklarinin tek Tanrisi sensin. Yeri, gögü sen yarattin.
\par 16 Ya RAB, kulak ver de isit, gözlerini aç da gör, ya RAB; Sanherib'in söylediklerini, yasayan Tanri'yi nasil asagiladigini duy.
\par 17 Ya RAB, gerçek su ki, Asur krallari birçok ulusu ve ülkelerini viraneye çevirdiler.
\par 18 Ilahlarini yakip yok ettiler. Çünkü onlar tanri degil, insan eliyle biçimlendirilmis tahta ve taslardi.
\par 19 Ya RAB Tanrimiz, simdi bizi Sanherib'in elinden kurtar ki, bütün dünya kralliklari senin tek Tanri oldugunu anlasin."
\par 20 Bunun üzerine Amots oglu Yesaya, Hizkiya'ya su haberi gönderdi: "Israil'in Tanrisi RAB söyle diyor: 'Asur Krali Sanherib'le ilgili olarak bana yalvardigin için diyorum ki, "'Erden kiz Siyon seni hor görüyor, Alay ediyor seninle. Yerusalim kizi* ardindan alayla bas salliyor.
\par 22 Sen kimi asagiladin, kime küfrettin? Kime sesini yükselttin? Israil'in Kutsali'na tepeden baktin!
\par 23 Ulaklarin araciligiyla Rab'bi asagiladin. Bir sürü savas arabamla daglarin tepesine, Lübnan'in doruklarina çiktim, dedin. Yüksek sedir agaçlarini, seçme çamlarini kestim, Lübnan'in en iç noktalarina, Gür ormanlarina ulastim.
\par 24 Yabanci ülkelerde kuyular kazdim, sular içtim, Misir'in bütün kanallarini ayagimin tabaniyla kuruttum, dedin.
\par 25 "'Bütün bunlari çoktan yaptigimi, Çok önceden tasarladigimi duymadin mi? Surlu kentleri enkaz yiginlarina çevirmeni Simdi ben gerçeklestirdim.
\par 26 O kentlerde yasayanlarin kolu kanadi kirildi. Yilginlik ve utanç içindeydiler; Kir otuna, körpe filizlere, Damlarda büyümeden kavrulup giden ota döndüler.
\par 27 Senin oturusunu, kalkisini, Ne zaman gidip geldigini, Bana nasil öfkelendigini biliyorum.
\par 28 Bana duydugun öfkeden, Kulagima erisen küstahligindan ötürü Halkami burnuna, gemimi agzina takacak, Seni geldigin yoldan geri çevirecegim.
\par 29 "'Senin için belirti su olacak, ey Hizkiya: Bu yil kendiliginden yetiseni yiyeceksiniz, Ikinci yil ise ardindan biteni. Üçüncü yil ekip biçin, Baglar dikip ürününü yiyin.
\par 30 Yahudalilar'in kurtulup sag kalanlari Yine asagiya dogru kök salacak, Yukariya dogru meyve verecek.
\par 31 Çünkü sag kalanlar Yerusalim'den, Kurtulanlar Siyon Dagi'ndan çikacak. Her Seye Egemen RAB'bin gayretiyle olacak bu.
\par 32 "Bundan dolayi RAB Asur Krali'na iliskin söyle diyor: 'Bu kente girmeyecek, ok atmayacak. Kente kalkanla yaklasmayacak, Karsisinda rampa kurmayacak.
\par 33 Geldigi yoldan dönecek ve kente girmeyecek diyor RAB,
\par 34 'Kendim için ve kulum Davut'un hatiri için Bu kenti savunup kurtaracagim diyor."
\par 35 O gece RAB'bin melegi gidip Asur ordugahinda yüz seksen bes bin kisiyi öldürdü. Ertesi sabah uyananlar salt cesetlerle karsilastilar.
\par 36 Bunun üzerine Asur Krali Sanherib ordugahini birakip çekildi. Ninova'ya döndü ve orada kaldi.
\par 37 Bir gün ilahi Nisrok'un tapinaginda tapinirken, ogullarindan Adrammelek'le Sareser, onu kiliçla öldürüp Ararat ülkesine kaçtilar. Yerine oglu Esarhaddon kral oldu.

\chapter{20}

\par 1 O günlerde Hizkiya ölümcül bir hastaliga yakalandi. Amots oglu Peygamber Yesaya ona gidip söyle dedi: "RAB diyor ki, 'Ev islerini düzene sok. Çünkü iyilesmeyecek, öleceksin."
\par 2 Hizkiya yüzünü duvara dönüp RAB'be yalvardi:
\par 3 "Ya RAB, yürekten bir sadakatle önünde nasil yasadigimi, gözünde iyi olani yaptigimi animsa lütfen." Sonra aci aci aglamaya basladi.
\par 4 Yesaya sarayin orta avlusundan çikmadan önce RAB ona söyle dedi:
\par 5 "Geri dön ve halkimi yöneten Hizkiya'ya sunu söyle: 'Atan Davut'un Tanrisi RAB diyor ki: Duani isittim, gözyaslarini gördüm, seni sagligina kavusturacagim. Üç gün içinde RAB'bin Tapinagi'na çikacaksin.
\par 6 Ömrünü on bes yil daha uzatacagim. Seni de kenti de Asur Krali'nin elinden kurtaracagim. Kendim için ve kulum Davut'un hatiri için bu kenti savunacagim."
\par 7 Sonra Yesaya, "Incir pestili getirin" dedi. Getirip çibanina koydular ve Hizkiya iyilesti.
\par 8 Hizkiya Yesaya'ya, "RAB'bin beni iyilestirecegine ve üç gün içinde RAB'bin Tapinagi'na çikacagima iliskin belirti nedir?" diye sormustu.
\par 9 Yesaya söyle karsilik vermisti: "RAB'bin verdigi sözü tutacagina iliskin belirti su olacak: Gölge on basamak uzasin mi, kisalsin mi?"
\par 10 Hizkiya, "Gölgenin on basamak uzamasi kolaydir, on basamak kisalsin" demisti.
\par 11 Bunun üzerine Peygamber Yesaya RAB'be yakardi ve RAB Ahaz'in merdiveninden asagi düsmüs olan gölgeyi on basamak kisaltmisti.
\par 12 O sirada Hizkiya'nin hastalandigini duyan Baladan oglu Babil Krali Merodak-Baladan, ona mektuplarla birlikte bir armagan gönderdi.
\par 13 Hizkiya elçileri kabul etti. Deposundaki bütün degerli esyalari -altini, gümüsü, baharati, degerli yagi, silah deposunu ve hazine odalarindaki her seyi- elçilere gösterdi. Sarayinda da kralliginda da onlara göstermedigi hiçbir sey kalmadi.
\par 14 Peygamber Yesaya Kral Hizkiya'ya gidip, "Bu adamlar sana ne dediler, nereden gelmisler?" diye sordu. Hizkiya, "Uzak bir ülkeden, Babil'den gelmisler" diye karsilik verdi.
\par 15 Yesaya, "Sarayinda ne gördüler?" diye sordu. Hizkiya, "Sarayimdaki her seyi gördüler, hazinelerimde onlara göstermedigim hiçbir sey kalmadi" diye yanitladi.
\par 16 Bunun üzerine Yesaya söyle dedi: "RAB'bin sözüne kulak ver.
\par 17 RAB diyor ki, 'Gün gelecek, sarayindaki her sey, atalarinin bugüne kadar bütün biriktirdikleri Babil'e tasinacak. Hiçbir sey kalmayacak.
\par 18 Soyundan gelen bazi çocuklar alinip götürülecek, Babil Krali'nin sarayinda hadim edilecek."
\par 19 Hizkiya, "RAB'den ilettigin bu söz iyi" dedi. Çünkü, "Nasil olsa yasadigim sürece baris ve güvenlik olacak" diye düsünüyordu.
\par 20 Hizkiya'nin kralligi dönemindeki öteki olaylar, bütün basarilari, bir havuz ve tünel yaparak suyu kente nasil getirdigi, Yahuda krallarinin tarihinde yazilidir.
\par 21 Hizkiya ölüp atalarina kavusunca, yerine oglu Manasse kral oldu.

\chapter{21}

\par 1 Manasse on iki yasinda kral oldu ve Yerusalim'de elli bes yil krallik yapti. Annesinin adi Hefsivah'ti.
\par 2 RAB'bin Israil halkinin önünden kovmus oldugu uluslarin igrenç törelerine uyarak RAB'bin gözünde kötü olani yapti.
\par 3 Babasi Hizkiya'nin yok ettigi puta tapilan yerleri yeniden yaptirdi. Israil Krali Ahav gibi, Baal* için sunaklar kurdu, Asera* putu yapti. Gök cisimlerine taparak onlara kulluk etti.
\par 4 RAB'bin, "Yerusalim'de bulunacagim" dedigi RAB'bin Tapinagi'nda sunaklar kurdu.
\par 5 Tapinagin iki avlusunda gök cisimlerine tapmak için sunaklar yaptirdi.
\par 6 Oglunu ateste kurban etti; falcilik ve büyücülük yapti. Medyumlara, ruh çagiranlara danisti. RAB'bin gözünde çok kötülük yaparak O'nu öfkelendirdi.
\par 7 Manasse yaptirdigi Asera putunu RAB'bin Tapinagi'na yerlestirdi. Oysa RAB tapinaga iliskin Davut'la oglu Süleyman'a söyle demisti: "Bu tapinakta ve Israil oymaklarinin yasadigi kentler arasindan seçtigim Yerusalim'de adim sonsuza dek anilacak.
\par 8 Buyruklarimi dikkatle yerine getirir ve kulum Musa'nin kendilerine verdigi Kutsal Yasa'yi tam uygularlarsa, Israil halkinin ayagini bir daha atalarina vermis oldugum ülkenin disina çikarmayacagim."
\par 9 Ama halk kulak asmadi. Manasse onlari öylesine yoldan çikardi ki, RAB'bin Israil halkinin önünde yok ettigi uluslardan daha çok kötülük yaptilar.
\par 10 RAB, kullari peygamberler araciligiyla söyle dedi:
\par 11 "Yahuda Krali Manasse bu igrenç isleri yaptigi, kendisinden önceki Amorlular'dan daha çok kötülük ettigi, putlar dikerek Yahudalilar'i günaha sürükledigi için Israil'in Tanrisi RAB, 'Yerusalim'in ve Yahuda'nin basina öyle bir felaket getirecegim ki, duyanlarin hepsi saskina dönecek diyor,
\par 13 'Samiriye'yi ve Ahav'in soyunu nasil cezalandirdimsa, Yerusalim'i cezalandirirken de ayni ölçü ipini, ayni çekülü kullanacagim. Bir adam tabagini nasil temizler, silip yüzüstü kapatirsa, Yerusalim'i de öyle silip süpürecegim.
\par 14 Halkimdan sag kalanlari terk edecegim ve düsmanlarinin eline teslim edecegim. Yerusalim halki yagmalanip ganimet olarak götürülecek.
\par 15 Çünkü gözümde kötü olani yaptilar. Atalarinin Misir'dan çiktigi günden bugüne kadar beni öfkelendirdiler."
\par 16 Manasse RAB'bin gözünde kötü olani yaparak Yahudalilar'i günaha sürüklemekle kalmadi, Yerusalim'i bir uçtan öbür uca dolduracak kadar çok sayida suçsuzun kanini döktü.
\par 17 Manasse'nin kralligi dönemindeki öteki olaylar, bütün yaptiklari ve isledigi günahlar Yahuda krallarinin tarihinde yazilidir.
\par 18 Manasse ölüp atalarina kavusunca, sarayinin Uzza adindaki bahçesine gömüldü. Yerine oglu Amon kral oldu.
\par 19 Amon yirmi iki yasinda kral oldu ve Yerusalim'de iki yil krallik yapti. Annesi Yotvali Harus'un kizi Mesullemet'ti.
\par 20 Amon da babasi Manasse gibi RAB'bin gözünde kötü olani yapti.
\par 21 Babasinin yürüdügü yollarda yürüdü, ayni putlara tapti ve hizmet etti.
\par 22 Atalarinin Tanrisi RAB'bi terk etti, O'nun yolunda yürümedi.
\par 23 Kral Amon'un görevlileri düzen kurup onu sarayinda öldürdüler.
\par 24 Ülke halki Amon'a düzen kuranlarin hepsini öldürdü. Yerine oglu Yosiya'yi kral yapti.
\par 25 Amon'un kralligi dönemindeki öteki olaylar ve yaptiklari Yahuda krallarinin tarihinde yazilidir.
\par 26 Amon Uzza bahçesinde kendi mezarina gömüldü. Yerine oglu Yosiya kral oldu.

\chapter{22}

\par 1 Yosiya sekiz yasinda kral oldu ve Yerusalim'de otuz bir yil krallik yapti. Annesi Boskatli Adaya'nin kizi Yedida'ydi.
\par 2 Yosiya RAB'bin gözünde dogru olani yapti. Saga sola sapmadan atasi Davut'un bütün yollarini izledi.
\par 3 Kral Yosiya, kralliginin on sekizinci yilinda Mesullam oglu Asalya oglu Yazman Safan'i RAB'bin Tapinagi'na gönderirken ona söyle dedi:
\par 4 "Baskâhin Hilkiya'nin yanina çik. Kapi nöbetçilerinin halktan toplayip RAB'bin Tapinagi'na getirdikleri paralari saysin.
\par 5 RAB'bin Tapinagi'ndaki islerin basinda bulunan denetçilere versin. Onlar da paralari RAB'bin Tapinagi'ndaki çatlaklari onaranlara, marangozlara, yapicilara, duvarcilara ödesinler. Tapinagin onarimi için gerekli keresteyi, yontma tasi da bu parayla alsinlar.
\par 7 Onlara verilen paranin hesabi sorulmasin, çünkü dürüstçe çalisiyorlar."
\par 8 Baskâhin Hilkiya Yazman Safan'a, "RAB'bin Tapinagi'nda Yasa Kitabi'ni buldum" diyerek kitabi ona verdi. Safan kitabi okudu.
\par 9 Sonra krala giderek, "Görevlilerin tapinaktaki paralari alip RAB'bin Tapinagi'ndaki islerin basinda bulunan adamlara verdiler" diye durumu bildirdi.
\par 10 Ardindan, "Kâhin Hilkiya bana bir kitap verdi" diyerek kitabi krala okudu.
\par 11 Kral Kutsal Yasa'daki sözleri duyunca giysilerini yirtti.
\par 12 Kâhin Hilkiya'ya, Safan oglu Ahikam'a, Mikaya oglu Akbor'a, Yazman Safan'a ve kendi özel görevlisi Asaya'ya söyle buyurdu:
\par 13 "Gidin, bulunan bu kitabin sözleri hakkinda benim için de, bütün Yahuda halki için de RAB'be danisin. RAB'bin bize karsi alevlenen öfkesi büyüktür. Çünkü atalarimiz bu kitabin sözlerine kulak asmadilar, bizler için yazilan bu sözlere uymadilar."
\par 14 Kâhin Hilkiya, Ahikam, Akbor, Safan ve Asaya varip tapinaktaki giysilerin nöbetçisi Harhas oglu Tikva oglu Sallum'un karisi Peygamber Hulda'ya danistilar. Hulda Yerusalim'de, Ikinci Mahalle'de oturuyordu.
\par 15 Hulda onlara söyle dedi: "Israil'in Tanrisi RAB, 'Sizi bana gönderen adama sunlari söyleyin diyor:
\par 16 'Yahuda Krali'nin okudugu kitapta yazili oldugu gibi, buraya da, burada yasayan halkin basina da felaket getirecegim.
\par 17 Beni terk ettikleri, elleriyle yaptiklari baska ilahlara buhur yakip beni kizdirdiklari için buraya karsi öfkem alevlenecek ve sönmeyecek.
\par 18 "RAB'be danismak için sizi gönderen Yahuda Krali'na söyle deyin: 'Israil'in Tanrisi RAB duydugun sözlere iliskin diyor ki:
\par 19 Madem yikilip lanetle anilacak olan burasi ve burada yasayanlarla ilgili sözlerimi duyunca yüregin yumusadi, önümde kendini alçalttin, giysilerini yirtip huzurumda agladin, ben de yalvarisini isittim.
\par 20 Seni atalarina kavusturacagim, esenlik içinde mezarina gömüleceksin. Buraya getirecegim büyük felaketi görmeyeceksin." Hilkiya ile yanindakiler bu sözleri krala ilettiler.

\chapter{23}

\par 1 Kral Yosiya haber gönderip Yahuda ve Yerusalim'in bütün ileri gelenlerini yanina topladi.
\par 2 Sonra Yahudalilar, Yerusalim'de yasayanlar, kâhinler,peygamberler, büyük küçük herkesle birlikte RAB'bin Tapinagi'na çikti. RAB'bin Tapinagi'nda bulunmus olan Antlasma Kitabi'ni bastan sona kadar herkesin duyacagi biçimde okudu.
\par 3 Sütunun yaninda durarak RAB'bin yolunu izleyecegine, buyruklarini, ögütlerini, kurallarini candan ve yürekten uygulayacagina, bu kitapta yazili antlasmanin kosullarini yerine getirecegine iliskin RAB'bin huzurunda antlasma yapti. Bütün halk bu antlasmayi onayladi.
\par 4 Kral Yosiya Baal*, Asera* ve gök cisimleri için yapilmis olan bütün esyalari RAB'bin Tapinagi'ndan çikarmak üzere Baskâhin Hilkiya'ya, kâhin yardimcilarina ve kapi nöbetçilerine buyruk verdi. Bunlari Yerusalim'in disina çikarip Kidron Vadisi'nde yakti, küllerini Beytel'e götürdü.
\par 5 Yahuda krallarinin kentlerde ve Yerusalim'in çevresindeki tapinma yerlerinde buhur yaksinlar diye atamis oldugu putperest kâhinleri, Baal'a, günese, aya, takimyildizlara -bütün gök cisimlerine- buhur yakanlari ortadan kaldirdi.
\par 6 Asera putunu RAB'bin Tapinagi'ndan çikarip Yerusalim'in disinda Kidron Vadisi'nde yakti, ezip toza çevirdi. Bu tozu siradan halkin mezarlarina serpti.
\par 7 Fuhus yapan kadin ve erkeklerin RAB'bin Tapinagi alanindaki odalarini yikti. Kadinlar orada Asera için kumas dokurlardi.
\par 8 Yosiya Yahuda kentlerinden bütün kâhinleri getirtti. Geva'dan Beer-Seva'ya kadar kâhinlerin buhur yaktiklari tapinma yerlerini kirletti. Adini kent yöneticisinden alan Yesu Kapisi'nin girisinde, kentin ana kapisinin solunda kalan kapilardaki tapinma yerlerini de yikti.
\par 9 Tapinma yerlerinin kâhinleri, Yerusalim'deki RAB'bin sunagina çikmaz, ancak öbür kâhinlerle birlikte mayasiz ekmek yerlerdi.
\par 10 Yosiya, kimse oglunu ya da kizini ilah Molek* için ateste kurban etmesin diye, Ben-Hinnom Vadisi'ndeki Tofet'i* kirletti.
\par 11 Yahuda krallarinin günese adamis oldugu atlari RAB'bin Tapinagi'nin girisinden kaldirdi. Atlar tapinagin avlusunda, hadim Natan-Melek'in odasinin yanindaydi. Yosiya günese adanmis savas arabalarini da atese verdi.
\par 12 Ahaz'in yukari odasinin daminda Yahuda krallarinin yaptirdigi sunaklari da, RAB'bin Tapinagi'nin iki avlusunda Manasse'nin yaptirdigi sunaklari da yikti; onlari kirip parçalayarak tozlarini Kidron Vadisi'ne saçti.
\par 13 Yerusalim'in dogusunda, Yikim Dagi'nin güneyinde Israil Krali Süleyman'in Saydalilar'in igrenç putu Astoret*, Moavlilar'in igrenç putu Kemos ve Ammonlular'in igrenç putu Molek için yaptirmis oldugu tapinma yerlerini kirletti.
\par 14 Dikili taslari, Asera putlarini parçaladi; yerlerini insan kemikleriyle doldurdu.
\par 15 Bundan baska Israil'i günaha sürükleyen Nevat oglu Yarovam'in yaptirdigi Beytel'deki tapinma yerini ve sunagi bile yikti. Tapinma yerini atese verip toz duman etti. Asera putunu yakti.
\par 16 Yosiya çevresine bakindi. Tepedeki mezarlari görünce, adamlarini gönderip mezarlardaki kemikleri çikartti. Olacaklari önceden bildiren Tanri adaminin açikladigi RAB'bin sözü uyarinca, kemikleri sunagin üzerinde yakarak sunagi kirletti.
\par 17 Kral, "Orada görünen anit nedir?" diye sordu. Kent halki, "Orasi Yahuda'dan gelen ve senin Beytel'deki sunaga yaptiklarini bildiren Tanri adaminin mezaridir" diye yanitladi.
\par 18 Kral, "Ona dokunmayin" dedi, "Kimse onun kemiklerini rahatsiz etmesin." Böylece Tanri adaminin kemiklerine de, Samiriye'den gelmis olan peygamberin kemiklerine de dokunmadilar.
\par 19 Yosiya Beytel'de yaptigi gibi, Israil krallarinin Samiriye kentlerinde yaptirdigi RAB'bi öfkelendiren tapinma yerlerindeki bütün yapilari ortadan kaldirdi.
\par 20 O kentlerdeki tapinma yerlerinin bütün kâhinlerini sunaklarin üzerinde kurban etti. Sunaklarin üzerinde insan kemikleri yaktiktan sonra Yerusalim'e döndü.
\par 21 Kral, "Tanriniz RAB için Fisih Bayrami'ni* bu Antlasma Kitabi'nda yazilanlara uygun biçimde kutlayin" diye halka buyruk verdi.
\par 22 Israil'e önderlik etmis olan hakimler döneminden bu yana, ne Israil, ne de Yahuda krallari döneminde, böyle bir Fisih Bayrami kutlanmamisti.
\par 23 RAB için düzenlenen bu Fisih Bayrami Kral Yosiya'nin kralliginin on sekizinci yilinda Yerusalim'de kutlandi.
\par 24 Bundan baska Yosiya, Kâhin Hilkiya'nin RAB'bin Tapinagi'nda buldugu kitapta yazili yasanin ilkelerini yerine getirmek amaciyla, cincileri, ruhçulari, aile putlarini, öteki putlari, ayrica Yahuda ve Yerusalim'de görülen bütün igrençlikleri silip süpürdü.
\par 25 Ne ondan önce, ne de sonra onun gibi candan ve yürekten var gücüyle RAB'be yönelen ve Musa'nin yasasina uyan bir kral çikti.
\par 26 Oysa Manasse isledigi suçlarla RAB'bi öyle öfkelendirmisti ki, RAB Yahuda'ya karsi alevlenen öfkesinden vazgeçmedi
\par 27 ve "Israil'i nasil huzurumdan attimsa, Yahuda'yi da öyle atacagim" dedi, "Seçtigim bu kenti, Yerusalim'i ve 'Orada bulunacagim dedigim tapinagi kendimden uzaklastiracagim."
\par 28 Yosiya'nin kralligi dönemindeki öteki olaylar ve bütün yaptiklari Yahuda krallarinin tarihinde yazilidir.
\par 29 Yosiya'nin kralligi sirasinda Misir Firavunu Neko Asur Krali'na yardim etmek üzere Firat'a dogru yola çikti. Kral Yosiya da Neko'nun üzerine yürüdü. Megiddo'da karsilastilar. Neko Yosiya'yi öldürdü.
\par 30 Görevlileri Yosiya'nin cesedini savas arabasiyla Megiddo'dan Yerusalim'e getirip mezarina gömdüler. Yahuda halki Yosiya'nin oglu Yehoahaz'i meshederek* babasinin yerine kral yapti.
\par 31 Yehoahaz yirmi üç yasinda kral oldu ve Yerusalim'de üç ay krallik yapti. Annesi Livnali Yeremya'nin kizi Hamutal'di.
\par 32 Yehoahaz atalari gibi RAB'bin gözünde kötü olani yapti.
\par 33 Yerusalim'de krallik yapmasin diye, Firavun Neko, Hama ülkesinde, Rivla'da Yehoahaz'i zincire vurdu. Ülke halkini yüz talant gümüs ve bir talant altin ödemekle yükümlü kildi.
\par 34 Firavun Neko Yosiya'nin oglu Elyakim'i babasinin yerine kral yapti ve adini degistirip Yehoyakim koydu. Sonra Yehoahaz'i alip Misir'a döndü. Yehoahaz orada öldü.
\par 35 Yehoyakim firavunun istedigi altin ve gümüsü ödedi. Bu parayi bulmak için firavunun buyruguna uyarak ülkeyi vergiye bagladi. Firavun Neko'ya verilmek üzere Yahuda halkindan herkesin gücü oraninda altin ve gümüs topladi.
\par 36 Yehoyakim yirmi bes yasinda kral oldu ve Yerusalim'de on bir yil krallik yapti. Annesi Rumali Pedaya'nin kizi Zevuda'ydi.
\par 37 Yehoyakim atalari gibi RAB'bin gözünde kötü olani yapti.

\chapter{24}

\par 1 Yahuda Krali Yehoyakim'in kralligi döneminde Babil Krali Nebukadnessar Yahuda'ya saldirdi. Yehoyakim üç yil ona boyun egdiyse de sonradan fikrini degistirerek Nebukadnessar'a baskaldirdi.
\par 2 RAB, kullari peygamberler araciligiyla söyledigi söz uyarinca, Yahuda'yi yok etmek üzere Kildani*, Aramli, Moavli ve Ammonlu akincilari ona karsi gönderdi.
\par 3 Bütün bunlar RAB'bin buyruguyla Yahudalilar'in basina geldi. Manasse'nin isledigi bütün günahlar, döktügü suçsuz kan yüzünden RAB Yahudalilar'i huzurundan atmak istedi. Çünkü Manasse Yerusalim'i suçsuzlarin kaniyla doldurmustu ve RAB bunu bagislamak niyetinde degildi.
\par 5 Yehoyakim'in kralligi dönemindeki öteki olaylar ve bütün yaptiklari Yahuda krallarinin tarihinde yazilidir.
\par 6 Yehoyakim ölüp atalarina kavusunca, yerine oglu Yehoyakin kral oldu.
\par 7 Misir Krali bir daha ülkesinden disari çikamadi. Çünkü Babil Krali Misir Vadisi'nden Firat'a kadar daha önce Misir Firavunu'na ait olan bütün topraklari ele geçirmisti.
\par 8 Yehoyakin on sekiz yasinda kral oldu ve Yerusalim'de üç ay krallik yapti. Annesi Yerusalimli Elnatan'in kizi Nehusta'ydi.
\par 9 O da babasi gibi RAB'bin gözünde kötü olani yapti.
\par 10 O sirada Babil Krali Nebukadnessar'in askerleri Yerusalim'in üzerine yürüyüp kenti kusatti.
\par 11 Kusatma sürerken Nebukadnessar geldi.
\par 12 Yahuda Krali Yehoyakin, annesi, görevlileri, yöneticileri ve hadimlariyla birlikte, Nebukadnessar'a teslim oldu. Babil Krali kralliginin sekizinci yilinda Yehoyakin'i tutsak etti.
\par 13 RAB'bin sözü uyarinca, Nebukadnessar RAB'bin Tapinagi'nin ve kral sarayinin bütün hazinelerini bosaltti, Israil Krali Süleyman'in RAB'bin Tapinagi için yaptirdigi altin esyalarin tümünü parçaladi.
\par 14 Bütün Yerusalim halkini, komutanlari, yigit savasçilari, zanaatçilari, demircileri, toplam on bin kisiyi sürgün etti. Yahuda halkinin en yoksul kesimi disinda kimse kalmadi.
\par 15 Babil Krali Nebukadnessar Yehoyakin'i tutsak olarak Babil'e götürdü. Onunla birlikte annesini, karilarini, hadimlarini ve ülkenin ileri gelenlerini de Yerusalim'den Babil'e sürdü. Ayrica yedi bin deneyimli yigit savasçiyi ve bin zanaatçiyla demirciyi Babil'e sürgün etti.
\par 17 Yehoyakin'in yerine amcasi Mattanya'yi kral yapti ve adini degistirip Sidkiya koydu.
\par 18 Sidkiya yirmi bir yasinda kral oldu ve Yerusalim'de on bir yil krallik yapti. Annesi Livnali Yeremya'nin kizi Hamutal'di.
\par 19 Yehoyakim gibi Sidkiya da RAB'bin gözünde kötü olani yapti.
\par 20 RAB Yerusalim'le Yahuda'ya öfkelendigi için onlari huzurundan atti. Sidkiya Babil Krali'na karsi ayaklandi.

\chapter{25}

\par 1 Sidkiya'nin kralliginin dokuzuncu yilinda, onuncu ayin* onuncu günü, Babil Krali Nebukadnessar bütün ordusuyla Yerusalim önlerine gelip ordugah kurdu. Kentin çevresine rampa yaptilar.
\par 2 Kral Sidkiya'nin kralliginin on birinci yilina kadar kent kusatma altinda kaldi.
\par 3 Dördüncü ayin dokuzuncu günü kentte kitlik öyle siddetlendi ki, halk bir lokma ekmek bulamaz oldu.
\par 4 Sonunda kentin surlarinda bir gedik açildi. Kildaniler* kenti çepeçevre kusatmis olmasina karsin, bütün askerler gece kral bahçesinin yolundan iki duvarin arasindaki kapidan kaçarak Arava yoluna çiktilar.
\par 5 Ama Kildani ordusu kralin ardina düserek Eriha ovalarinda ona yetisti. Sidkiya'nin bütün ordusu dagildi.
\par 6 Kral Sidkiya yakalanip Rivla'da Babil Krali'nin huzuruna çikarildi ve hakkinda karar verildi.
\par 7 Sidkiya'nin gözü önünde ogullarini öldürdüler; kendisinin de gözlerini oydular, zincire vurup Babil'e götürdüler.
\par 8 Babil Krali Nebukadnessar'in kralliginin on dokuzuncu yilinda, besinci ayin* yedinci günü muhafiz birligi komutani, Babil Krali'nin görevlisi Nebuzaradan Yerusalim'e girdi.
\par 9 RAB'bin Tapinagi'ni, sarayi ve Yerusalim'deki bütün evleri atese verip önemli yapilari yakti.
\par 10 Muhafiz birligi komutani önderligindeki Kildani* ordusu Yerusalim'i çevreleyen surlari yikti.
\par 11 Komutan Nebuzaradan kentte sag kalanlari, Babil Krali'nin safina geçen kaçaklari ve geri kalan halki sürgün etti.
\par 12 Ancak bagcilik, çiftçilik yapsinlar diye bazi yoksullari orada birakti.
\par 13 Kildaniler RAB'bin Tapinagi'ndaki tunç* sütunlari, ayakliklari, tunç havuzu parçalayip tunçlari Babil'e götürdüler.
\par 14 Tapinak törenlerinde kullanilan kovalari, kürekleri, fitil masalarini, tabaklari, bütün tunç esyalari aldilar.
\par 15 Muhafiz birligi komutani saf altin ve gümüs buhurdanlari, çanaklari alip götürdü.
\par 16 RAB'bin Tapinagi için Süleyman'in yaptirmis oldugu iki sütun, havuz ve ayakliklar için hesapsiz tunç harcanmisti.
\par 17 Her sütun on sekiz arsin yüksekligindeydi, üzerlerinde tunç birer baslik vardi. Basligin yüksekligi üç arsindi, çevresi tunçtan ag ve nar motifleriyle bezenmisti. Öbür sütun da ag motifleriyle süslenmisti ve ötekine benziyordu.
\par 18 Muhafiz birligi komutani Nebuzaradan Baskâhin Seraya'yi, Baskâhin Yardimcisi Sefanya'yi ve üç kapi nöbetçisini tutsak aldi.
\par 19 Kentte kalan askerlerin komutanini, kralin bes danismanini, ayrica ülke halkini askere yazan ordu komutaninin yazmanini ve ülke halkindan kentte bulunan altmis kisiyi tutsak etti.
\par 20 Hepsini Rivla'ya, Babil Krali'nin yanina götürdü.
\par 21 Babil Krali Hama ülkesinde, Rivla'da onlari idam etti. Böylece Yahuda halki ülkesinden sürülmüs oldu.
\par 22 Babil Krali Nebukadnessar Yahuda'da kalan halkin üzerine Safan oglu Ahikam oglu Gedalya'yi vali atadi.
\par 23 Ordu komutanlariyla adamlari, Babil Krali'nin Gedalya'yi vali atadigini duyunca, Mispa'ya, Gedalya'nin yanina geldiler. Gelenler Netanya oglu Ismail, Kareah oglu Yohanan, Netofali Tanhumet oglu Seraya, Maakali oglu Yaazanya ve adamlariydi.
\par 24 Gedalya onlara ve adamlarina ant içerek, "Kildani* yetkililerden korkmayin" dedi, "Ülkeye yerlesip Babil Krali'na hizmet edin. Böylesi sizin için daha iyi olur."
\par 25 O yilin yedinci ayinda* kral soyundan Elisama oglu Netanya oglu Ismail on adamiyla birlikte Mispa'ya gidip Gedalya'yi öldürdü. Ayrica, Gedalya'yi destekleyen Yahudiler'i ve Kildaniler'i de kiliçtan geçirdi.
\par 26 Bunun üzerine büyük küçük bütün halk ordu komutanlariyla birlikte Misir'a kaçti. Çünkü Kildaniler'den korkuyorlardi.
\par 27 Yahuda Krali Yehoyakin'in sürgündeki otuz yedinci yili Evil-Merodak Babil Krali oldu. Evil-Merodak o yilin on ikinci ayinin* yirmi yedinci günü, Yahuda Krali Yehoyakin'i cezaevinden çikardi.
\par 28 Kendisiyle tatli tatli konustu ve ona Babil'deki öteki sürgün krallardan daha üstün bir yer verdi.
\par 29 Yehoyakin cezaevi giysilerini üstünden çikardi. Yasadigi sürece Babil Krali'nin sofrasinda yer aldi.
\par 30 Yasami boyunca kral tarafindan günlük yiyecegi sürekli karsilandi.


\end{document}