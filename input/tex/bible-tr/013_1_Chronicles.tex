\begin{document}

\title{1 Tarihler}


\chapter{1}

\par 1 Adem, Sit, Enos,
\par 2 Kenan, Mahalalel, Yeret,
\par 3 Hanok, Metuselah, Lemek,
\par 4 Nuh. Nuh'un ogullari: Sam, Ham, Yafet.
\par 5 Yafet'in ogullari: Gomer, Magog, Meday, Yâvan, Tuval, Mesek, Tiras.
\par 6 Gomer'in ogullari: Askenaz, Difat, Togarma.
\par 7 Yâvan'in ogullari: Elisa, Tarsis, Kittim, Rodanim.
\par 8 Ham'in ogullari: Kûs, Misrayim, Pût, Kenan.
\par 9 Kûs'un ogullari: Seva, Havila, Savta, Raama, Savteka. Raama'nin ogullari: Seva, Dedan.
\par 10 Kûs'un Nemrut adinda bir oglu oldu. Yigitligiyle yeryüzüne ün saldi.
\par 11 Misrayim Ludlular'in, Anamlilar'in, Lehavlilar'in, Maftuhlular'in, Patruslular'in, Filistliler'in atalari olan Kasluhlular'in ve Kaftorlular'in atasiydi.
\par 13 Kenan ilk oglu Sidon'un babasi ve Hititler'in*, Yevuslular'in, Amorlular'in, Girgaslilar'in, Hivliler'in, Arklilar'in, Sinliler'in, Arvatlilar'in, Semarlilar'in, Hamalilar'in atasiydi.
\par 17 Sam'in ogullari: Elam, Asur, Arpaksat, Lud, Aram, Ûs, Hul, Geter, Mesek.
\par 18 Arpaksat Selah'in babasiydi. Selah'tan Ever oldu.
\par 19 Ever'in iki oglu oldu. Birinin adi Pelek'ti, çünkü yeryüzündeki insanlar onun yasadigi dönemde bölündü. Kardesinin adi Yoktan'di.
\par 20 Yoktan Almodat'in, Selef'in, Hasarmavet'in, Yerah'in, Hadoram'in, Uzal'in, Dikla'nin, Eval'in, Avimael'in, Seva'nin, Ofir'in, Havila'nin, Yovav'in atasiydi. Bunlarin hepsi Yoktan'in soyundandi.
\par 24 Sam, Arpaksat, Selah,
\par 25 Ever, Pelek, Reu,
\par 26 Seruk, Nahor, Terah,
\par 27 Avram -Ibrahim-.
\par 28 Ibrahim'in ogullari: Ishak, Ismail.
\par 29 Ismailogullari'nin soyu: Ismail'in ilk oglu Nevayot. Sonra Kedar, Adbeel, Mivsam,
\par 30 Misma, Duma, Massa, Hadat, Tema,
\par 31 Yetur, Nafis, Kedema gelir. Bunlar Ismail'in ogullariydi.
\par 32 Ibrahim'in cariyesi Ketura'nin ogullari: Zimran, Yoksan, Medan, Midyan, Yisbak, Suah. Yoksan'in ogullari: Seva, Dedan.
\par 33 Midyan'in ogullari: Efa, Efer, Hanok, Avida, Eldaa. Bunlarin hepsi Ketura'nin soyundandi.
\par 34 Ishak Ibrahim'in ogluydu. Ishak'in ogullari: Esav, Israil.
\par 35 Esav'in ogullari: Elifaz, Reuel, Yeus, Yalam, Korah.
\par 36 Elifaz'in ogullari: Teman, Omar, Sefi, Gatam, Kenaz ve Timna'dan dogan Amalek.
\par 37 Reuel'in ogullari: Nahat, Zerah, Samma, Mizza.
\par 38 Seir'in ogullari: Lotan, Soval, Sivon, Âna, Dison, Eser, Disan.
\par 39 Lotan'in ogullari: Hori, Homam. Timna Lotan'in kizkardesiydi.
\par 40 Soval'in ogullari: Alyan, Manahat, Eval, Sefi, Onam. Sivon'un ogullari: Aya, Âna.
\par 41 Âna'nin oglu: Dison. Dison'un ogullari: Hamran, Esban, Yitran, Keran.
\par 42 Eser'in ogullari: Bilhan, Zaavan, Yaakan. Disan'in ogullari sunlardi: Ûs, Aran.
\par 43 Israilliler'i yöneten bir kralin olmadigi dönemde, Edom'u su krallar yönetti: Beor oglu Bala. Kentinin adi Dinhava'ydi.
\par 44 Bala ölünce, yerine Bosrali Zerah oglu Yovav geçti.
\par 45 Yovav ölünce, Temanlilar ülkesinden Husam kral oldu.
\par 46 Husam ölünce, Midyan'i Moav kirlarinda bozguna ugratan Bedat oglu Hadat kral oldu. Kentinin adi Avit'ti.
\par 47 Hadat ölünce, yerine Masrekali Samla geçti.
\par 48 Samla ölünce, yerine Rehovot-Hannaharli Saul geçti.
\par 49 Saul ölünce, yerine Akbor oglu Baal-Hanan geçti.
\par 50 Baal-Hanan ölünce, yerine Hadat geçti. Kentinin adi Pai'ydi. Karisi, Me-Zahav kizi Matret'in kizi Mehetavel'di.
\par 51 Sonunda Hadat da öldü. Edom beyleri sunlardi: Timna, Alva, Yetet,
\par 52 Oholivama, Ela, Pinon,
\par 53 Kenaz, Teman, Mivsar,
\par 54 Magdiel, Iram. Edom beyleri bunlardi.

\chapter{2}

\par 1 Israil'in ogullari sunlardir: Ruben, Simon, Levi, Yahuda, Issakar, Zevulun,
\par 2 Dan, Yusuf, Benyamin, Naftali, Gad, Aser.
\par 3 Yahuda'nin ogullari: Er, Onan, Sela. Bu üç ogulu Yahuda'ya Kenanli Sua'nin kizi dogurdu. Yahuda'nin ilk oglu Er, RAB'bin gözünde kötüydü. Bu yüzden RAB onu öldürdü.
\par 4 Yahuda'nin gelini Tamar ona Peres ve Zerah'i dogurdu. Yahuda'nin toplam bes oglu vardi.
\par 5 Peres'in ogullari: Hesron, Hamul.
\par 6 Zerahogullari: Zimri, Etan, Heman, Kalkol, Darda. Toplam bes kisiydi.
\par 7 Karmi'nin oglu: Yok edilmeye adanmis esyalar konusunda RAB'be ihanet etmekle Israil'i yikima sürükleyen Akan.
\par 8 Etam'in oglu: Azarya.
\par 9 Hesron'un ogullari: Yerahmeel, Ram, Kalev.
\par 10 Amminadav Ram'in ogluydu. Yahudalilar'in önderi Nahson Amminadav'in ogluydu.
\par 11 Salma*fe* Nahson'un ogluydu. Boaz Salma'nin,
\par 12 Ovet Boaz'in, Isay Ovet'in ogluydu.
\par 13 Isay'in yedi oglu oldu: Birincisi Eliav, ikincisi Avinadav, üçüncüsü Sima,
\par 14 dördüncüsü Netanel, besincisi Radday,
\par 15 altincisi Osem, yedincisi Davut.
\par 16 Kizkardesleri Seruya ile Avigayil'di. Seruya'nin üç oglu vardi: Avisay, Yoav, Asahel.
\par 17 Ismaili Yeter'le evlenen Avigayil Amasa'yi dogurdu.
\par 18 Hesron oglu Kalev'in, karisi Azuva'dan ve Yeriot'tan ogullari oldu. Karisindan dogan ogullar sunlardir: Yeser, Sovav, Ardon.
\par 19 Azuva ölünce, Kalev kendisine Hur'u doguran Efrat'la evlendi.
\par 20 Uri Hur'un ogluydu, Besalel Uri'nin ogluydu.
\par 21 Daha sonra Hesron, Gilat'in babasi Makir'in kiziyla yatti. Altmis yasindayken evlendigi bu kadin ona Seguv'u dogurdu.
\par 22 Yair Seguv'un ogluydu; Gilat ülkesinde yirmi üç kenti vardi.
\par 23 Ama Gesur'la Aram Havvot-Yair'i ve Kenat ile çevresindeki köyleri, toplam altmis kenti ele geçirdiler. Buralarda yasayan bütün halk Gilat'in babasi Makir'in soyundandi.
\par 24 Hesron Kalev-Efrata'da öldükten sonra, karisi Aviya Tekoa'nin kurucusu olan Ashur'u dogurdu.
\par 25 Hesron'un ilk oglu Yerahmeel'in ogullari: Ilk oglu Ram, Buna, Oren, Osem, Ahiya.
\par 26 Yerahmeel'in Atara adinda baska bir karisi vardi; O da Onam'in annesiydi.
\par 27 Yerahmeel'in ilk oglu Ram'in ogullari: Maas, Yamin, Eker.
\par 28 Onam'in ogullari: Sammay, Yada. Sammay'in ogullari: Nadav, Avisur.
\par 29 Avisur'un Avihayil adinda bir karisi vardi; ona Ahban'i ve Molit'i dogurdu.
\par 30 Nadav'in ogullari: Selet, Appayim. Selet çocuksuz öldü.
\par 31 Appayim'in oglu: Yisi. Yisi'nin oglu: Sesan. Sesan'in oglu: Ahlay.
\par 32 Sammay'in kardesi Yada'nin ogullari: Yeter ve Yonatan. Yeter çocuksuz öldü.
\par 33 Yonatan'in ogullari: Pelet, Zaza. Bunlar Yerahmeel'in soyundan geliyordu.
\par 34 Sesan'in oglu olmadiysa da kizlari vardi. Sesan'in Yarha adinda Misirli bir usagi vardi.
\par 35 Sesan kizini usagi Yarha'yla evlendirdi. Kadin ona Attay'i dogurdu.
\par 36 Natan Attay'in ogluydu. Zavat Natan'in,
\par 37 Eflal Zavat'in, Ovet Eflal'in,
\par 38 Yehu Ovet'in, Azarya Yehu'nun,
\par 39 Heles Azarya'nin, Elasa Heles'in,
\par 40 Sismay Elasa'nin, Sallum Sismay'in,
\par 41 Yekamya Sallum'un, Elisama Yekamya'nin ogluydu.
\par 42 Yerahmeel'in kardesi Kalev'in ogullari: Ilk oglu Zif'in kurucusu Mesa ve Hevron'un kurucusu Merasa.
\par 43 Hevron'un ogullari: Korah, Tappuah, Rekem, Sema.
\par 44 Raham Sema'nin ogluydu. Yorkoam Raham'in, Sammay Rekem'in,
\par 45 Maon Sammay'in ogluydu. Beytsur'un kurucusu Maon'du.
\par 46 Kalev'in cariyesi Efa ona Haran'i, Mosa'yi ve Gazez'i dogurdu. Gazez Haran'in ogluydu.
\par 47 Yahday'in ogullari: Regem, Yotam, Gesan, Pelet, Efa, Saaf.
\par 48 Kalev'in öbür cariyesi Maaka ona Sever'i ve Tirhana'yi dogurdu.
\par 49 Maaka Madmanna'nin kurucusu Saaf'i, Makbena ve Giva'nin kurucusu Seva'yi da dogurdu. Kalev'in Aksa adinda bir de kizi oldu.
\par 50 Kalev'in soyundan gelenler: Efrat'in ilk oglu Hur'un ogullari: Kiryat-Yearim'in kurucusu Soval,
\par 51 Beytlehem'in kurucusu Salma, Beytgader'in kurucusu Haref.
\par 52 Kiryat-Yearim'in kurucusu Soval, Haroeliler'in ve Menuhot'ta yasayan halkin yarisinin atasiydi.
\par 53 Kiryat-Yearim'in boylari: Yeterliler, Pûtlular, Sumatlilar, Misralilar. Soralilar'la Estaollular da bu boylarin soyundan geldi.
\par 54 Salma'nin soyundan gelenler: Beytlehemliler, Netofalilar, Atrot-Beytyoavlilar, Manahatlilar'in yarisi ve Sorlular.
\par 55 Yabes'te yasayan yazmanlarin boylari: Tiratlilar, Simatlilar, Sukatlilar. Bunlar Rekav halkinin atasi Hammat'in soyundan gelen Kenliler'dir.

\chapter{3}

\par 1 Davut'un Hevron'da dogan ogullari sunlardi: Ilk oglu Yizreelli Ahinoam'dan Amnon, ikincisi Karmelli Avigayil'den Daniel,
\par 2 üçüncüsü Gesur Krali Talmay'in kizi Maaka'dan Avsalom, dördüncüsü Hagit'ten Adoniya,
\par 3 besincisi Avital'dan Sefatya, altincisi karisi Egla'dan Yitream.
\par 4 Davut'un bu alti oglu Hevron'da dogdular. Davut orada yedi yil alti ay, Yerusalim'de de otuz üç yil krallik yapti.
\par 5 Davut'un Yerusalim'de dogan ogullari: Sima, Sovav, Natan, Süleyman. Bu dördü Ammiel'in kizi Bat-Seva'dan dogdular.
\par 6 Öbür ogullari: Yivhar, Elisama, Elifelet,
\par 7 Nogah, Nefek, Yafia,
\par 8 Elisama, Elyada, Elifelet. Toplam dokuz kisiydiler.
\par 9 Bütün bunlar, Davut'un cariyelerinden doganlar disindaki ogullariydi. Tamar onlarin kizkardesidir.
\par 10 Rehavam Süleyman'in ogluydu. Aviya Rehavam'in, Asa Aviya'nin, Yehosafat Asa'nin,
\par 11 Yehoram Yehosafat'in, Ahazya Yehoram'in, Yoas Ahazya'nin,
\par 12 Amatsya Yoas'in, Azarya Amatsya'nin, Yotam Azarya'nin,
\par 13 Ahaz Yotam'in, Hizkiya Ahaz'in, Manasse Hizkiya'nin,
\par 14 Amon Manasse'nin, Yosiya Amon'un ogluydu.
\par 15 Yosiya'nin ogullari: Ilk oglu Yohanan, Ikincisi Yehoyakim, Üçüncüsü Sidkiya, Dördüncüsü Sallum.
\par 16 Yehoyakim'in ogullari: Yehoyakin ve Sidkiya.
\par 17 Sürgün edilen Yehoyakin'in torunlari: Sealtiel,
\par 18 Malkiram, Pedaya, Senassar, Yekamya, Hosama, Nedavya.
\par 19 Pedaya'nin ogullari: Zerubbabil, Simi. Zerubbabil'in çocuklari: Mesullam, Hananya ve kizkardesleri Selomit.
\par 20 Zerubbabil'in bes oglu daha vardi: Hasuva, Ohel, Berekya, Hasadya, Yusav-Heset.
\par 21 Hananya'nin ogullari: Pelatya, Yesaya. Yesaya Refaya'nin, Refaya Arnan'in, Arnan Ovadya'nin, Ovadya Sekanya'nin babasiydi.
\par 22 Sekanya'nin oglu: Semaya. Semaya'nin ogullari: Hattus, Yigal, Bariah, Nearya, Safat. Toplam alti kisiydi.
\par 23 Nearya'nin ogullari: Elyoenay, Hizkiya, Azrikam. Toplam üç kisiydi.
\par 24 Elyoenay'in ogullari: Hodavya, Elyasiv, Pelaya, Akkuv, Yohanan, Delaya, Anani. Toplam yedi kisiydi.

\chapter{4}

\par 1 Yahuda ogullari: Peres, Hesron, Karmi, Hur, Soval.
\par 2 Soval oglu Reaya Yahat'in babasiydi. Yahat Ahumay'in ve Lahat'in babasiydi. Sorali boylar bunlardi.
\par 3 Etam'in ogullari: Yizreel, Yisma, Yidbas. Kizkardeslerinin adi Haselelponi'dir.
\par 4 Penuel Gedor'un babasiydi. Ezer Husa'nin babasiydi. Bunlar Beytlehem'in kurucusu ve Efrat'in ilk oglu Hur'un torunlariydi.
\par 5 Tekoa'nin kurucusu Ashur'un Helah ve Naara adinda iki karisi vardi.
\par 6 Naara ona Ahuzzam, Hefer, Temeni ve Haahastari'yi dogurdu. Bunlar Naara'nin ogullariydi.
\par 7 Helah'in ogullari: Seret, Yishar, Etnan
\par 8 ve Anuv'la Hassoveva'nin babasi Kos. Kos Harum oglu Aharhel boylarinin atasiydi.
\par 9 Yabes kardeslerinden daha saygin biriydi. Annesi, "Onu aci çekerek dogurdum" diyerek adini Yabes koymustu.
\par 10 Yabes, Israil'in Tanrisi'na, "Ne olur, beni kutsa, sinirlarimi genislet!" diye yakardi, "Elin üzerimde olsun, beni kötülükten koru. Öyle ki, aci çekmeyeyim." Tanri onun yakarisini duydu.
\par 11 Suha'nin kardesi Keluv, Eston'un babasi Mehir'in babasiydi.
\par 12 Eston Beytrafa'nin, Paseah'in ve Ir-Nahas'in kurucusu Tehinna'nin babasiydi. Bunlar Rekalilar'dir.
\par 13 Kenaz'in ogullari: Otniel, Seraya. Otniel'in ogullari: Hatat ve Meonotay. Meonotay Ofra'nin babasiydi. Seraya Ge-Harasim'in kurucusu Yoav'in babasiydi. Bunlar el islerinde becerikli kisilerdi.
\par 15 Yefunne oglu Kalev'in ogullari: Iru, Ela, Naam. Ela'nin oglu: Kenaz.
\par 16 Yehallelel'in ogullari: Zif, Zifa, Tirya, Asarel.
\par 17 Ezra'nin ogullari: Yeter, Meret, Efer, Yalon. Meret'in karilarindan biri Miryam'i, Samma'yi ve Estemoa'nin kurucusu Yisbah'i dogurdu.
\par 18 Bunlar Meret'in evlendigi firavunun kizi Bitya'nin dogurdugu çocuklardir. Meret'in Yahudali karisi, Gedor'un kurucusu Yeret'i, Soko'nun kurucusu Hever'i, Zanoah'in kurucusu Yekutiel'i dogurdu.
\par 19 Hodiya Naham'in kizkardesiyle evlendi. Ondan dogan ogullari Keila'da yasayan Garm ve Estemoa'da yasayan Maaka boylarinin atasiydi.
\par 20 Simon'un ogullari: Amnon, Rinna, Ben-Hanan, Tilon. Yisi'nin torunlari: Zohet ve Ben-Zohet.
\par 21 Yahuda oglu Sela'nin ogullari: Leka'nin kurucusu Er, Maresa'nin kurucusu Lada, Beytasbea'da ince keten isinde çalisanlarin boylari,
\par 22 Yokim, Kozevalilar, Moavli kadinlarla evlenen Yoas'la Saraf ve Yasuvi-Lehem. Bu kayitlar eskidir.
\par 23 Netayim ve Gedera'da oturur, çömlekçilik yapar ve kral için çalisirlardi.
\par 24 Simonogullari: Nemuel, Yamin, Yariv, Zerah, Saul.
\par 25 Sallum Saul'un ogluydu. Mivsam Sallum'un, Misma da Mivsam'in ogluydu.
\par 26 Misma'nin torunlari: Hammuel Misma'nin ogluydu. Zakkur Hammuel'in, Simi de Zakkur'un ogluydu.
\par 27 Simi'nin on alti oglu, alti kizi vardi. Ancak kardeslerinin çok sayida çocuklari yoktu. Bu yüzden Simon oymagi, sayica Yahuda oymagi kadar artmadi.
\par 28 Simonogullari Beer-Seva, Molada, Hasar-Sual, Bilha, Esem, Tolat, Betuel, Horma, Ziklak, Beytmarkavot, Hasar-Susim, Beytbiri ve Saarayim kentlerinde yasadilar. Davut dönemine dek kentleri bunlardi.
\par 32 Ayrica su bes ilçede de yasadilar: Etam, Ayin, Rimmon, Token, Asan.
\par 33 Baalat'a kadar uzanan bu kentlerin çevresindeki bütün köyler Simonogullari'na aitti. Ailelerinin soy kütügünü de tuttular.
\par 34 Mesovav, Yamlek, Amatsya oglu Yosa,
\par 35 Yoel, Asiel oglu Seraya oglu Yosivya oglu Yehu,
\par 36 Elyoenay, Yaakova, Yesohaya, Asaya, Adiel, Yesimiel, Benaya
\par 37 ve Semaya oglu Simri oglu Yedaya oglu Allon oglu Sifi oglu Ziza.
\par 38 Yukarida adi geçenlerin her biri, ait oldugu boyun basiydi. Aileleri sayica çogaldilar.
\par 39 Sürülerine otlak aramak için Gedor yakinlarina, vadinin dogusuna kadar gittiler.
\par 40 Orada çok zengin otlaklar buldular. Ülke genis, rahat ve huzur doluydu. Eskiden orada Ham'in soyundan gelenler yasardi.
\par 41 Yukarida adi yazili olanlar, Yahuda Krali Hizkiya döneminde geldiler. Hamlilar'in çadirlarina ve orada yasayan Meunlular'a saldirip tümünü öldürdüler. Sonra ülkelerine yerlestiler. Bugün de oradalar. Çünkü orada sürüleri için otlak vardi.
\par 42 Yisi'nin ogullari Pelatya, Nearya, Refaya, Uzziel önderliginde Simon oymagindan bes yüz kisi Seir daglik bölgesine gidip
\par 43 Amalekliler'den sag kalanlari öldürdüler. Bugün de orada yasiyorlar.

\chapter{5}

\par 1 Israil'in ilk oglu Ruben'in ogullari. -Ruben ilk dogandir. Babasinin yatagina yatip onu kirlettigi için ilk ogulluk hakki Israil oglu Yusuf'un ogullarina verildi. Bu yüzden Ruben soy kütügüne ilk ogulluk hakkina göre yazilmadi.
\par 2 Yahuda kardesleri arasinda en güçlü olandi, önderlik hep onun soyundan çikti. Ama ilk ogulluk hakki Yusuf'a aitti.-
\par 3 Israil'in ilk oglu Ruben'in ogullari: Hanok, Pallu, Hesron, Karmi.
\par 4 Yoel'in soyu: Semaya Yoel'in, Gog Semaya'nin, Simi Gog'un, Mika Simi'nin, Reaya Mika'nin, Baal Reaya'nin, Beera Baal'in ogluydu. Asur Krali Tiglat-Pileser'in sürgüne gönderdigi Beera Rubenliler'in önderiydi.
\par 7 Boylarina göre aile soy kütügüne yazilan akrabalari sunlardir: Önder Yeiel, Zekeriya,
\par 8 Yoel oglu Sema oglu Azaz oglu Bala. Bunlar Aroer'de, Nevo ve Baal-Meon'a kadar uzanan topraklarda yasadilar.
\par 9 Doguda Firat'tan çöle kadar uzanan topraklara yayildilar. Çünkü Gilat bölgesinde sigirlari çogalmisti.
\par 10 Rubenliler Saul döneminde Hacerliler'e karsi savas açti. Onlari yenilgiye ugratip Gilat'in dogusunda kalan topraklardaki çadirlarini ele geçirdiler.
\par 11 Gadlilar, Basan'da, Selka'ya kadar uzanan topraklarda Rubenliler'in karsisinda yasadilar.
\par 12 Önderleri Yoel'di; ikinci derecede önemli Safam, Basan'da ise Yanay ve Safat'ti.
\par 13 Boylarina göre akrabalari sunlardir: Mikael, Mesullam, Seva, Yoray, Yakan, Zia, Ever. Toplam yedi kisiydi.
\par 14 Bunlar Bûz oglu Yahdo oglu Yesisay oglu Mikael oglu Gilat oglu Yaroah oglu Huri oglu Avihayil'in ogullariydi.
\par 15 Guni oglu Avdiel oglu Ahi bu boylarin basiydi.
\par 16 Gadlilar Gilat'ta, Basan'da, çevredeki köylerde ve Saron'un bütün otlaklarinda yasadilar.
\par 17 Bunlarin hepsi Yahuda Krali Yotam'in ve Israil Krali Yarovam'in döneminde soy kütügüne yazildilar.
\par 18 Rubenliler'in, Gadlilar'in ve Manasse oymaginin yarisinin kalkan ve kiliç kullanabilen, ok atabilen, savas için egitilmis 44 760 yigit askeri vardi.
\par 19 Hacerliler'e, Yetur'a, Nafis'e, Nodav'a karsi savas açtilar.
\par 20 Onlarla savasirken yardim gördüler. Tanri Hacerliler'i ve onlarla birlikte olanlari ellerine teslim etti. Çünkü savas sirasinda Tanri'ya yalvarmislar, O'na güvenmislerdi. Bu yüzden Tanri yalvarislarini yanitladi.
\par 21 Hacerliler'in hayvanlarini ele geçirdiler: Elli bin deve, iki yüz elli bin davar, iki bin esek. Tutsak olarak da yüz bin kisi aldilar.
\par 22 Savas Tanri'nin istegiyle oldugu için düsmandan birçok kisiyi öldürdüler. Sürgün dönemine dek Hacerliler'in topraklarinda yasadilar.
\par 23 Manasse oymaginin yarisi Basan'dan Baal-Hermon'a, Senir, yani Hermon Dagi'na kadar uzanan topraklarda yasadi ve sayica çogaldi.
\par 24 Boy baslari sunlardi: Efer, Yisi, Eliel, Azriel, Yeremya, Hodavya, Yahdiel. Bunlar yigit savasçilar, ünlü kisiler ve boy baslariydi.
\par 25 Ne var ki atalarinin Tanrisi'na bagli kalmadilar. Tanri'ya ihanet ederek önlerinden yok ettigi uluslarin ilahlarina yöneldiler.
\par 26 Bu yüzden Israil'in Tanrisi, Tiglat-Pileser diye bilinen Asur Krali Pûl'u harekete geçirdi. Asur Krali Rubenliler'i, Gadlilar'i, Manasse oymaginin yarisini tutsak edip Halah'a, Habur'a, Hara'ya, Gozan Irmagi'na sürdü. Onlar bugün de oralarda yasiyorlar.

\chapter{6}

\par 1 Levi'nin ogullari: Gerson, Kehat, Merari.
\par 2 Kehat'in ogullari: Amram, Yishar, Hevron, Uzziel.
\par 3 Amram'in çocuklari: Harun, Musa, Miryam. Harun'un ogullari: Nadav, Avihu, Elazar, Itamar.
\par 4 Pinehas Elazar'in ogluydu. Avisua Pinehas'in,
\par 5 Bukki Avisua'nin, Uzzi Bukki'nin,
\par 6 Zerahya Uzzi'nin, Merayot Zerahya'nin,
\par 7 Amarya Merayot'un, Ahituv Amarya'nin,
\par 8 Sadok Ahituv'un, Ahimaas Sadok'un,
\par 9 Azarya Ahimaas'in, Yohanan Azarya'nin,
\par 10 Azarya Yohanan'in ogluydu. Süleyman'in Yerusalim'de yaptigi tapinakta kâhinlik eden oydu.-
\par 11 Amarya Azarya'nin, Ahituv Amarya'nin,
\par 12 Sadok Ahituv'un, Sallum Sadok'un,
\par 13 Hilkiya Sallum'un, Azarya Hilkiya'nin,
\par 14 Seraya Azarya'nin, Yehosadak Seraya'nin ogluydu.
\par 15 RAB Nebukadnessar araciligiyla Yahuda ve Yerusalim halkini sürdügünde Yehosadak da sürgüne gitmisti.
\par 16 Levi'nin ogullari: Gerson, Kehat, Merari.
\par 17 Gerson'un ogullari: Livni, Simi.
\par 18 Kehat'in ogullari: Amram, Yishar, Hevron, Uzziel.
\par 19 Merari'nin ogullari: Mahli, Musi. Soylarina göre yazilan Levi oymaginin boylari sunlardir:
\par 20 Gerson'un soyu: Livni Gerson'un, Yahat Livni'nin, Zimma Yahat'in,
\par 21 Yoah Zimma'nin, Iddo Yoah'in, Zerah Iddo'nun, Yeateray Zerah'in ogluydu.
\par 22 Kehat'in soyu: Amminadav Kehat'in, Korah Amminadav'in, Assir Korah'in,
\par 23 Elkana Assir'in, Evyasaf Elkana'nin, Assir Evyasaf'in,
\par 24 Tahat Assir'in, Uriel Tahat'in, Uzziya Uriel'in, Saul Uzziya'nin ogluydu.
\par 25 Elkana'nin öbür ogullari: Amasay, Ahimot.
\par 26 Elkana Ahimot'un, Sofay Elkana'nin, Nahat Sofay'in,
\par 27 Eliav Nahat'in, Yeroham Eliav'in, Elkana Yeroham'in, Samuel Elkana'nin ogluydu.
\par 28 Samuel'in ogullari: Ilk oglu Yoel, ikincisi Aviya.
\par 29 Merari'nin soyu: Mahli Merari'nin, Livni Mahli'nin, Simi Livni'nin, Uzza Simi'nin,
\par 30 Sima Uzza'nin, Hagiya Sima'nin, Asaya Hagiya'nin ogluydu.
\par 31 Antlasma Sandigi* RAB'bin Tapinagi'na tasindiktan sonra Davut'un orada görevlendirdigi ezgiciler sunlardir.
\par 32 Bunlar Süleyman Yerusalim'de RAB'bin Tapinagi'ni kurana dek Bulusma Çadiri'nda ezgi okuyarak hizmet eder, belirlenmis kurallar uyarinca görevlerini yerine getirirlerdi.
\par 33 Ogullariyla birlikte görev yapan kisiler sunlardir: Kehatogullari'ndan: Ezgici Heman. Heman, Israil oglu Levi oglu Kehat oglu Yishar oglu Korah oglu Evyasaf oglu Assir oglu Tahat oglu Sefanya oglu Azarya oglu Yoel oglu Elkana oglu Amasay oglu Mahat oglu Elkana oglu Suf oglu Toah oglu Eliel oglu Yeroham oglu Elkana oglu Samuel oglu Yoel'in ogluydu.
\par 39 Heman'in sag yaninda görev yapan akrabasi Asaf. Asaf, Levi oglu Gerson oglu Yahat oglu Simi oglu Zimma oglu Etan oglu Adaya oglu Zerah oglu Etni oglu Malkiya oglu Baaseya oglu Mikael oglu Sima oglu Berekya'nin ogluydu.
\par 44 Heman'in solunda görev yapan kardesleri Merariogullari'ndan: Etan. Etan, Levi oglu Merari oglu Musi oglu Mahli oglu Semer oglu Bani oglu Amsi oglu Hilkiya oglu Amatsya oglu Hasavya oglu Malluk oglu Avdi oglu Kiysi'nin ogluydu.
\par 48 Bunlarin Levili akrabalari, çadirin, Tanri'nin Tapinagi'nin bütün görevlerini yerine getirmek üzere atandilar.
\par 49 Ancak, Tanri kulu Musa'nin buyrugu uyarinca, yakmalik sunu* sunaginda ve buhur sunaginda sunu sunanlar Harun'la ogullariydi. En Kutsal Yer'de* yapilan hizmetlerden ve Israilliler'in bagislanmasi için sunulan kurbanlardan onlar sorumluydu.
\par 50 Harunogullari sunlardir: Harun'un oglu Elazar, onun oglu Pinehas, onun oglu Avisua,
\par 51 onun oglu Bukki, onun oglu Uzzi, onun oglu Zerahya,
\par 52 onun oglu Merayot, onun oglu Amarya, onun oglu Ahituv,
\par 53 onun oglu Sadok, onun oglu Ahimaas.
\par 54 Ilk kurayi çeken Kehat boyundan Harunogullari'nin sinirlarina göre yerlesim yerleri sunlardir:
\par 55 Yahuda topraklarindaki Hevron'la çevresindeki otlaklar onlara verildi.
\par 56 Kentin tarlalariyla köyleri ise Yefunne oglu Kalev'e verildi.
\par 57 Siginak kent seçilen Hevron, Livna, Yattir, Estemoa, Hilen, Devir, Asan, Yutta, Beytsemes kentleriyle bunlarin otlaklari Harunogullari'na verildi.
\par 60 Benyamin oymagindan da Givon, Geva, Alemet, Anatot ve bunlarin otlaklari verildi. Kehat boylarina verilen bu kentlerin toplam sayisi on üçtü.
\par 61 Geri kalan Kehatogullari'na Manasse oymaginin yarisina ait boylardan alinan on kent kurayla verildi.
\par 62 Gersonogullari'na boy sayisina göre Issakar, Aser, Naftali ve Basan'daki Manasse oymagindan alinan on üç kent verildi.
\par 63 Merariogullari'na boy sayisina göre Ruben, Gad ve Zevulun oymaklarindan alinan on iki kent kurayla verildi.
\par 64 Israilliler bu kentleri otlaklariyla birlikte Levililer'e verdiler.
\par 65 Yahuda, Simon, Benyamin oymaklarindan alinan ve yukarida adlari sayilan kentler kurayla verildi.
\par 66 Kehat boyundan bazi ailelere Efrayim oymagindan alinan kentler verildi.
\par 67 Efrayim daglik bölgesinde siginak kent seçilen Sekem, Gezer, Yokmoam, Beythoron, Ayalon, Gat-Rimmon ve bunlarin otlaklari verildi.
\par 70 Israilliler Manasse oymaginin yarisindan alinan Aner, Bilam ve bunlarin otlaklarini Kehat boyunun öbür ailelerine verdiler.
\par 71 Asagidaki kentler Gersonogullari'na verildi: Manasse oymaginin yarisina ait Basan'daki Golan, Astarot ve bunlarin otlaklari.
\par 72 Issakar oymagindan Kedes, Daverat, Ramot, Anem ve bunlarin otlaklari.
\par 74 Aser oymagindan Masal, Avdon, Hukok, Rehov ve bunlarin otlaklari.
\par 76 Naftali oymagindan Celile'deki Kedes, Hammon, Kiryatayim ve bunlarin otlaklari.
\par 77 Merariogullari'na -geri kalan Levililer'e- asagidaki kentler verildi: Zevulun oymagindan Rimmono, Tavor ve bunlarin otlaklari.
\par 78 Ruben oymagindan Eriha'nin ötesinde, Seria Irmagi'nin dogusundaki kirda bulunan Beser, Yahsa, Kedemot, Mefaat ve bunlarin otlaklari.
\par 80 Gad oymagindan Gilat'taki Ramot, Mahanayim, Hesbon, Yazer ve bunlarin otlaklari.

\chapter{7}

\par 1 Issakar'in dört oglu vardi: Tola, Pua, Yasuv, Simron.
\par 2 Tola'nin ogullari: Uzzi, Refaya, Yeriel, Yahmay, Yivsam, Samuel. Bunlar Tola boyunun baslariydi. Soy kütügüne göre yigit savasçilardi. Davut döneminde sayilari 22 600'dü.
\par 3 Uzzi'nin oglu: Yizrahya. Yizrahya'nin ogullari: Mikael, Ovadya, Yoel, Yissiya. Besi de boy basiydi.
\par 4 Soy kütügüne göre, aralarinda savasa hazir 36 000 kisi vardi. Hepsinin çok sayida karisi ve çocugu vardi.
\par 5 Soy kütügüne göre Issakar boylarina bagli akrabalarindan savasacak durumda olanlarin sayisi 87 000'di.
\par 6 Benyamin'in üç oglu vardi: Bala, Beker, Yediael.
\par 7 Bala'nin bes oglu vardi: Esbon, Uzzi, Uzziel, Yerimot, Iyri. Bunlar boy basiydi. Soy kütügüne kayitli yigit savasçilarin sayisi 22 034 kisiydi.
\par 8 Beker'in ogullari: Zemira, Yoas, Eliezer, Elyoenay, Omri, Yeremot, Aviya, Anatot, Alemet. Hepsi Beker'in ogullariydi.
\par 9 Bunlarin soy kütügüne kayitli aile baslarinin ve yigit savasçilarin sayisi 20 200'dü.
\par 10 Yediael'in oglu: Bilhan. Bilhan'in ogullari: Yeus, Benyamin, Ehut, Kenaana, Zetan, Tarsis, Ahisahar.
\par 11 Yediael'in bütün ogullari boy baslariydi. Aralarinda savasa hazir 17 200 yigit savasçi vardi.
\par 12 Suppim ve Huppim Ir'in, Husim ise Aher'in ogluydu.
\par 13 Naftali'nin ogullari: Yahasiel, Guni, Yeser, Sallum. Bunlar Bilha'nin soyundandi.
\par 14 Manasse'nin ogullari: Aramli cariyenin dogurdugu Asriel, Makir. Makir Gilat'in babasiydi.
\par 15 Makir Huppim'le Suppim'in kizkardesi Maaka'yi kari olarak aldi. Makir'in ikinci oglunun adi Selofhat'ti. Selofhat'in yalniz kizlari oldu.
\par 16 Makir'in karisi Maaka, Peres ve Seres adinda iki ogul dogurdu. Seres'in de Ulam ve Rekem adinda iki oglu oldu.
\par 17 Ulam'in oglu: Bedan. Manasse oglu Makir oglu Gilat'in ogullari bunlardir.
\par 18 Gilat'in kizkardesi Hammoleket Ishot'u, Aviezer'i, Mahla'yi dogurdu.
\par 19 Semida'nin ogullari: Ahyan, Sekem, Likhi, Aniam.
\par 20 Efrayim'in ogullari: Sutelah, Ezer ve Elat. Beret Sutelah'in, Tahat Beret'in, Elada Tahat'in, Tahat Elada'nin, Zavat Tahat'in, Sutelah da Zavat'in ogluydu. Ülkede dogup büyüyen Gatlilar Ezer'le Elat'i öldürdüler. Çünkü onlarin sürülerini çalmaya gitmislerdi.
\par 22 Babalari Efrayim günlerce yas tuttu. Akrabalari onu avutmaya geldiler.
\par 23 Efrayim karisiyla yine yatti. Kadin gebe kalip bir ogul dogurdu. Evinde talihsizlik var diye babasi oglana Beria adini verdi.
\par 24 Kizi Asagi ve Yukari Beythoron'u, Uzzen-Seera'yi kuran Seera'ydi.
\par 25 Beria'nin Refah adinda bir oglu vardi. Refah Beria'nin, Resef Refah'in, Telah Resef'in, Tahan Telah'in, Ladan Tahan'in, Ammihut Ladan'in, Elisama Ammihut'un, Nun Elisama'nin, Yesu da Nun'un ogluydu.
\par 28 Efrayimliler'in yerlestikleri topraklar Beytel'i ve çevresindeki köyleri, doguda Naaran'i, batida Gezer'i ve köylerini, Sekem, Aya ve köylerini kapsiyordu.
\par 29 Manasse oymaginin sinirinda Beytsean, Taanak, Megiddo, Dor ve bunlara ait köyler vardi. Israil oglu Yusuf'un soyu buralarda yasadi.
\par 30 Aser'in ogullari: Yimna, Yisva, Yisvi, Beria; kizkardesleri Serah.
\par 31 Beriaogullari: Hever ve Birzayit'in kurucusu Malkiel.
\par 32 Hever Yaflet, Somer, Hotam ve kizkardesleri Sua'nin babasiydi.
\par 33 Yafletogullari: Pasak, Bimhal, Asvat. Yaflet'in ogullari bunlardi.
\par 34 Somerogullari: Ahi, Rohga, Yehubba, Aram.
\par 35 Kardesi Helem'in ogullari: Sofah, Yimna, Seles, Amal.
\par 36 Sofahogullari: Suah, Harnefer, Sual, Beri, Yimra,
\par 37 Beser, Hod, Samma, Silsa, Yitran, Beera.
\par 38 Yeterogullari: Yefunne, Pispa, Ara.
\par 39 Ulla'nin ogullari: Arah, Hanniel, Risya.
\par 40 Bunlarin hepsi Aser soyundandi. Boy baslari, seçkin kisiler, yigit savasçilar ve taninmis önderlerdi. Soy kütügüne kayitli savasa hazir olanlarin sayisi yirmi alti bindi.

\chapter{8}

\par 1 Benyamin'in ilk oglu Bala, Ikinci oglu Asbel, Üçüncü oglu Ahrah, Dördüncü oglu Noha,
\par 2 Besinci oglu Rafa'ydi.
\par 3 Bala'nin ogullari: Addar, Gera, Avihut,
\par 4 Avisua, Naaman, Ahoah,
\par 5 Gera, Sefufan, Huram.
\par 6 Ehutogullari Geva'da yasayan ailelerin baslariydi. Oradan Manahat'a sürüldüler. Adlari sunlardir:
\par 7 Naaman, Ahiya ve onlari sürgüne gönderen Gera. Gera Uzza'yla Ahihut'un babasiydi.
\par 8 Saharayim, karilari Husim'le Baara'yi bosadiktan sonra, Moav kirinda çocuk sahibi oldu.
\par 9 Yeni karisi Hodes'ten dogan ogullari sunlardir: Yovav, Sivya, Mesa, Malkam,
\par 10 Yeus, Sakeya, Mirma. Bunlar Saharayim'in ogullariydi. Hepsi de boy baslariydi.
\par 11 Saharayim'in karisi Husim'den de Avituv ve Elpaal adinda iki oglu vardi.
\par 12 Elpaal'in ogullari: Ever, Misam, Ono ve Lod kentleriyle çevrelerindeki köyleri yeniden kuran Semet, Beria, Sema. Beria'yla Sema Ayalon'da yasayan halkin boy baslariydi. Bunlar Gat'ta yasayan halki sürdüler.
\par 14 Ahyo, Sasak, Yeremot,
\par 15 Zevadya, Arat, Eder,
\par 16 Mikael, Yispa ve Yoha Beria'nin ogullariydi.
\par 17 Zevadya, Mesullam, Hizki, Hever,
\par 18 Yismeray, Yizliya ve Yovav Elpaal'in ogullariydi.
\par 19 Yakim, Zikri, Zavdi,
\par 20 Elienay, Silletay, Eliel,
\par 21 Adaya, Beraya ve Simrat Simi'nin ogullariydi.
\par 22 Sasak'in ogullari: Yispan, Ever, Eliel,
\par 23 Avdon, Zikri, Hanan,
\par 24 Hananya, Elam, Antotiya,
\par 25 Yifdeya, Penuel.
\par 26 Yeroham'in ogullari: Samseray, Seharya, Atalya,
\par 27 Yaaresya, Eliya, Zikri.
\par 28 Bunlarin hepsi soy kütügüne göre boy baslari ve önderlerdi. Yerusalim'de yasadilar.
\par 29 Givon'un kurucusu Yeiel Givon'da yasadi. Karisinin adi Maaka'ydi.
\par 30 Yeiel'in ilk oglu Avdon'du. Öbürleri sunlardi: Sur, Kis, Baal, Ner, Nadav,
\par 31 Gedor, Ahyo, Zeker
\par 32 ve Sima'nin babasi Miklot. Bunlar Yerusalim'deki akrabalarinin yaninda yasarlardi.
\par 33 Ner Kis'in, Kis Saul'un babasiydi. Saul Yonatan, Malkisua, Avinadav ve Esbaal'in babasiydi.
\par 34 Yonatan Merib-Baal'in, Merib-Baal Mika'nin babasiydi.
\par 35 Mika'nin ogullari: Piton, Melek, Tarea, Ahaz.
\par 36 Ahaz Yehoadda'nin babasiydi. Yehoadda Alemet, Azmavet, Zimri'nin, Zimri Mosa'nin,
\par 37 Mosa Binea'nin, Binea Rafa'nin, Rafa Elasa'nin, Elasa da Asel'in babasiydi.
\par 38 Asel'in alti oglu vardi. Adlari söyledir: Azrikam, Bokeru, Ismail, Searya, Ovadya, Hanan. Bütün bunlar Asel'in ogullariydi.
\par 39 Asel'in kardesi Esek'in ogullari: Ilk oglu Ulam, ikincisi Yeus, üçüncüsü Elifelet.
\par 40 Ulamogullari ok atmakta usta, yigit savasçilardi. Ulam'in birçok oglu, torunu vardi. Sayilari yüz elli kisiydi. Hepsi Benyamin soyundandi.

\chapter{9}

\par 1 Bütün Israilliler soylarina göre kaydedilmistir. Bu kayitlar Israil krallarinin tarihinde yazilidir. Yahudalilar RAB'be ihanet ettikleri için Babil'e sürüldüler.
\par 2 Kentlerindeki mülklerine dönüp ilk yerlesenler bazi Israilliler, kâhinler*, Levililer ve tapinak görevlileriydi.
\par 3 Yahuda, Benyamin, Efrayim ve Manasse soyundan Yerusalim'de yasayanlar sunlardir:
\par 4 Yahuda oglu Peres soyundan Bani oglu Imri oglu Omri oglu Ammihut oglu Utay.
\par 5 Selaogullari'ndan: Ilk oglu Asaya ve ogullari.
\par 6 Zerahogullari'ndan: Yeuel. Yahudalilar'in toplami 690 kisiydi.
\par 7 Benyamin soyundan gelenler: Hassenua oglu Hodavya oglu Mesullam oglu Sallu,
\par 8 Yeroham oglu Yivneya, Mikri oglu Uzzi oglu Ela, Yivniya oglu Reuel oglu Sefatya oglu Mesullam.
\par 9 Soylarina göre kaydedilenlerin toplami 956 kisiydi. Bunlarin hepsi aile baslariydi.
\par 10 Kâhinler: Yedaya, Yehoyariv, Yakin,
\par 11 Ahituv oglu Merayot oglu Sadok oglu Mesullam oglu Hilkiya oglu tapinak bas görevlisi Azarya,
\par 12 Malkiya oglu Pashur oglu Yeroham oglu Adaya, Immer oglu Mesillemit oglu Mesullam oglu Yahzera oglu Adiel oglu Maasay.
\par 13 Aile baslari olan kâhin kardeslerinin toplami 1 760'ti. Tanri'nin Tapinagi'ndaki hizmetlerden sorumlu yetenekli kisilerdi.
\par 14 Levililer: Merariogullari'ndan Hasavya oglu Azrikam oglu Hassuv oglu Semaya,
\par 15 Bakbakkar, Heres, Galal, Asaf oglu Zikri oglu Mika oglu Mattanya,
\par 16 Yedutun oglu Galal oglu Semaya oglu Ovadya ve Netofalilar'in köylerinde yasayan Elkana oglu Asa oglu Berekya.
\par 17 Tapinak kapi nöbetçileri: Sallum, Akkuv, Talmon, Ahiman ve kardesleri. Sallum baslariydi.
\par 18 Bugüne kadar doguya bakan Kral Kapisi'nda görevlidirler. Levili bölüklere bagli kapi nöbetçileri bunlardi.
\par 19 Korah oglu Evyasaf oglu Kore oglu Sallum ve ailesinden Korahogullari'ndan- olan çalisma arkadaslari Çadir'in giris kapisinin nöbetçileriydi. Bunlarin atalari da RAB'bin ordugahinin giris kapisinin nöbetçileriydi.
\par 20 Önceleri Elazar oglu Pinehas onlarin basiydi. RAB onunlaydi.
\par 21 Bulusma Çadiri'nin kapisinda Meselemya oglu Zekeriya nöbet tutardi.
\par 22 Giris kapisina nöbetçi seçilenlerin toplami 212'ydi. Bunlar, köylerinde bagli olduklari soy kütügüne yazilidir. Davut'la Bilici* Samuel tarafindan bu göreve atanmislardi.
\par 23 Ogullariyla birlikte RAB'bin Tapinagi'nin, yani Çadir'in kapilarinda nöbet tutarlardi.
\par 24 Nöbetçiler dört yanda -doguda, batida, kuzeyde ve güneyde-nöbet tutardi.
\par 25 Köylerdeki kardesleri zaman zaman gelir, yedi günlük bir süre için görevi onlarla paylasirdi.
\par 26 Dört kapinin bas nöbetçileri Levili'ydi. Bunlar Tanri'nin Tapinagi'ndaki odalardan ve hazinelerden sorumluydu.
\par 27 Geceyi Tanri'nin Tapinagi'nin çevresinde geçirirlerdi. Çünkü tapinagi koruma ve her sabah kapilarini açma görevi onlarindi.
\par 28 Bazilari da tapinak hizmetinde kullanilan esyalardan sorumluydu. Esyalari sayarak içeri alir, sayarak disariya çikarirlardi.
\par 29 Öbürleri esyalardan, kutsal yere ait nesnelerden, ince undan, saraptan, zeytinyagindan, günnükten, baharattan sorumluydu.
\par 30 Ancak baharati karistirip hazirlama görevi kâhinlerindi.
\par 31 Korahogullari'ndan Sallum'un ilk oglu Levili Mattitya sacda pide pisirme görevine atanmisti.
\par 32 Kardesleri Kehatogullari'ndan bazilari da her Sabat Günü* adak ekmeklerini* hazirlamakla görevliydi.
\par 33 Levililer'in boy baslari olan ezgiciler tapinagin odalarinda yasardi. Baska is yapmazlardi. Çünkü yaptiklari isten gece gündüz sorumluydular.
\par 34 Bunlarin hepsi soy kütügüne göre Levililer'in boy baslari, önderleriydi ve Yerusalim'de yasarlardi.
\par 35 Givon'un kurucusu Yeiel, Givon'da yasadi. Karisinin adi Maaka'ydi.
\par 36 Yeiel'in ilk oglu Avdon'du. Öbürleri sunlardi: Sur, Kis, Baal, Ner, Nadav,
\par 37 Gedor, Ahyo, Zekeriya, Miklot.
\par 38 Miklot Simam'in babasiydi. Bunlar Yerusalim'de akrabalarinin yaninda yasarlardi.
\par 39 Ner Kis'in, Kis Saul'un babasiydi. Saul Yonatan, Malkisua, Avinadav ve Esbaal'in babasiydi.
\par 40 Yonatan Merib-Baal'in, Merib-Baal Mika'nin babasiydi.
\par 41 Mika'nin ogullari: Piton, Melek, Tahrea, Ahaz.
\par 42 Ahaz Yada'nin babasiydi. Yada Alemet, Azmavet ve Zimri'nin, Zimri Mosa'nin,
\par 43 Mosa Binea'nin, Binea Refaya'nin, Refaya Elasa'nin, Elasa da Asel'in babasiydi.
\par 44 Asel'in alti oglu vardi. Adlari söyledir: Azrikam, Bokeru, Ismail, Searya, Ovadya, Hanan. Bütün bunlar Asel'in ogullariydi.

\chapter{10}

\par 1 Filistliler Israilliler'le savasa tutustu. Israilliler Filistliler'in önünden kaçti. Birçogu Gilboa Dagi'nda ölüp yere serildi.
\par 2 Filistliler Saul'la ogullarinin ardina düstüler. Saul'un ogullari Yonatan'i, Avinadav'i ve Malkisua'yi yakalayip öldürdüler.
\par 3 Saul'un çevresinde savas kizisti. Derken Saul Filistli okçular tarafindan vuruldu ve yaralandi.
\par 4 Saul silahini tasiyan adama, "Kilicini çek de bana sapla" dedi, "Yoksa bu sünnetsizler* gelip benimle alay edecekler." Ama silah tasiyicisi büyük bir korkuya kapilarak bunu yapmak istemedi. Bunun üzerine Saul kilicini çekip kendini üzerine atti.
\par 5 Saul'un öldügünü görünce, silah tasiyicisi da kendini kilicinin üzerine atip öldü.
\par 6 Böylece Saul, üç oglu ve bütün ev halki birlikte öldüler.
\par 7 Vadide oturan Israilliler, Israil ordusunun kaçtigini, Saul'la ogullarinin öldügünü anlayinca, kentlerini terk edip kaçmaya basladilar. Filistliler gelip bu kentlere yerlestiler.
\par 8 Ertesi gün Filistliler, öldürülenleri soymak için geldiklerinde, Saul'la ogullarinin Gilboa Dagi'nda öldügünü gördüler.
\par 9 Saul'u soyduktan sonra basini kesip silahlarini aldilar. Sonra bu iyi haberi putlarina ve halka duyurmalari için Filist ülkesinin her yanina ulaklar gönderdiler.
\par 10 Saul'un silahlarini ilahlarinin tapinagina koyup basini Dagon Tapinagi'na çaktilar.
\par 11 Yaves-Gilat halki Filistliler'in Saul'a yaptiklarini duydu.
\par 12 Bütün yigitler gidip Saul'la ogullarinin cesetlerini Yaves'e getirdiler. Sonra kemiklerini Yaves'teki yabanil fistik agacinin altina gömdüler ve yedi gün oruç* tuttular.
\par 13 Saul RAB'be ihanet ettigi için öldü. RAB'bin sözünü yerine getirmedi. Yol göstermesi için RAB'be danisacagina bir cinciye danisti. Bu yüzden RAB onu öldürdü. Kralligini da Isay oglu Davut'a devretti.

\chapter{11}

\par 1 Israilliler'in tümü Hevron'da bulunan Davut'a gelip söyle dediler: "Biz senin etin, kemiginiz.
\par 2 Geçmiste Saul kralimizken, savasta Israil'e komuta eden sendin. Tanrin RAB sana, 'Halkim Israil'i sen güdecek, onlara sen önder olacaksin diye söz verdi."
\par 3 Israil'in bütün ileri gelenleri Hevron'a, Kral Davut'un yanina gelince, Davut RAB'bin önünde orada onlarla bir antlasma yapti. Onlar da RAB'bin Samuel araciligiyla söyledigi söz uyarinca, Davut'u Israil Krali olarak meshettiler*.
\par 4 Kral Davut'la Israilliler Yevus diye bilinen Yerusalim'e saldirmak için yola çiktilar. Orada yasayan Yevuslular
\par 5 Davut'a, "Sen buraya giremezsin" dediler. Ne var ki, Davut Siyon Kalesi'ni, Davut Kenti'ni ele geçirdi.
\par 6 Davut, "Yevuslular'a ilk saldiran kisi komutan ve önder olacak" demisti. Ilk saldiriyi Seruya oglu Yoav yapti, böylece ordu komutani oldu.
\par 7 Bundan sonra Davut kalede oturmaya basladi. Bunun için oraya "Davut Kenti" adi verildi.
\par 8 Çevredeki bölgeyi, Millo'dan* çevre surlara kadar uzanan kesimi insa etti. Yoav da kentin geri kalan bölümünü onardi.
\par 9 Davut giderek güçleniyordu. Çünkü Her Seye Egemen RAB onunlaydi.
\par 10 RAB'bin Israil'e verdigi söz uyarinca Davut'un yigit askerlerinin komutanlari Israil halkiyla birlikte Davut'u kral yaptilar ve kralliginin güçlenmesi için onu desteklediler.
\par 11 Bunlarin adlari söyledir: Üçler'in önderi Hakmonlu Yasovam, mizragini üç yüz kisiye karsi kaldirip bir saldirida hepsini öldürdü.
\par 12 Ikincisi, üç yigitlerden biri olan Ahohlu Dodo oglu Elazar.
\par 13 Filistliler savas için Pas-Dammim'de toplandiklarinda Elazar Davut'un yanindaydi. Orada bir arpa tarlasi vardi. Israilliler Filistliler'in önünden kaçmisti.
\par 14 Ama Elazar'la Davut tarlanin ortasinda durup orayi savunmus, Filistliler'i öldürmüslerdi. RAB onlara büyük bir zafer saglamisti.
\par 15 Otuzlar'dan üçü Davut'un yanina, Adullam Magarasi'ndaki kayaya gittiler. Bir Filist birligi Refaim Vadisi'nde ordugah kurmustu.
\par 16 Bu sirada Davut hisarda, baska bir Filist birligiyse Beytlehem'deydi.
\par 17 Davut özlemle, "Keske biri Beytlehem'de kapinin yanindaki kuyudan bana su getirse!" dedi.
\par 18 Bu Üçler Filist ordugahinin ortasindan geçerek Beytlehem'de kapinin yanindaki kuyudan su çekip Davut'a getirdiler. Ama Davut içmek istemedi; suyu yere dökerek RAB'be sundu.
\par 19 "Ey Tanrim, bunu yapmak benden uzak olsun!" dedi, "Canlarini tehlikeye atip giden bu üç kisinin kanini mi içeyim?" Canlarini tehlikeye atarak suyu getirdikleri için Davut içmek istemedi. Bu üç kisinin yigitligi iste böyleydi.
\par 20 Yoav'in kardesi Avisay Üçler'in önderiydi. Mizragini kaldirip üç yüz kisiyi öldürdü. Bu yüzden Üçler kadar ünlendi.
\par 21 Üçler'in en saygin kisisiydi ve onlarin önderi oldu. Ama Üçler'den sayilmadi.
\par 22 Yehoyada oglu Kavseelli Benaya yürekli bir savasçiydi. Büyük isler basardi. Aslan yürekli iki Moavli'yi öldürdü. Ayrica karli bir gün çukura inip bir aslan öldürdü.
\par 23 Bes arsin*fü* boyunda iri yari bir Misirli'yi da öldürdü. Misirli'nin elinde dokumaci sirigi gibi bir mizrak vardi. Benaya sopayla onun üzerine yürüdü. Mizragi elinden kaptigi gibi onu kendi mizragiyla öldürdü.
\par 24 Yehoyada oglu Benaya'nin yaptiklari bunlardir. Bu sayede o da üç yigitler kadar ünlendi.
\par 25 Benaya Otuzlar arasinda saygin bir yer edindiyse de, Üçler'den sayilmadi. Davut onu muhafiz birligi komutanligina atadi.
\par 26 Öteki yigitler sunlardir: Yoav'in kardesi Asahel, Beytlehemli Dodo oglu Elhanan,
\par 27 Harorlu Sammot, Pelonlu Heles,
\par 28 Tekoali Ikkes oglu Ira, Anatotlu Aviezer,
\par 29 Husali Sibbekay, Ahohlu Ilay,
\par 30 Netofali Mahray ve Baana oglu Helet,
\par 31 Benyaminogullari'ndan Givali Rivay oglu Ittay, Piratonlu Benaya,
\par 32 Gaas vadilerinden Huray, Arvali Aviel,
\par 33 Baharumlu Azmavet, Saalbonlu Elyahba,
\par 34 Gizonlu Hasem'in ogullari, Hararli Sage oglu Yonatan,
\par 35 Hararli Sakâr oglu Ahiam, Ur oglu Elifal,
\par 36 Mekerali Hefer, Pelonlu Ahiya,
\par 37 Karmelli Hesro, Ezbay oglu Naaray,
\par 38 Natan'in kardesi Yoel, Hacer oglu Mivhar,
\par 39 Ammonlu Selek, Seruya oglu Yoav'in silah tasiyicisi Berotlu Nahray,
\par 40 Yattirli Ira ve Garev,
\par 41 Hititli* Uriya, Ahlay oglu Zavat,
\par 42 Rubenliler'in önderi Rubenli Siza oglu Adina ve ona eslik eden otuz kisi,
\par 43 Maaka oglu Hanan, Mitanli Yosafat,
\par 44 Asterali Uzziya, Aroerli Hotam'in ogullari Sama ve Yeiel,
\par 45 Simri oglu Yediael ve kardesi Tisli Yoha,
\par 46 Mahavli Eliel, Elnaam'in ogullari Yerivay ve Yosavya, Moavli Yitma,
\par 47 Eliel, Ovet, Mesovali Yaasiel.

\chapter{12}

\par 1 Davut'un Kis oglu Saul'dan gizlendigi Ziklak'ta yanina gelenler sunlardir. Bunlar savasta onu destekleyen yigitlerdi.
\par 2 Benyamin oymagindan, Saul'un ailesindendiler. Yay tasir ve yayla ok, sapanla tas atmak için hem sag, hem sol ellerini kullanabilirlerdi.
\par 3 Givali Semaa'nin ogullari Ahiezer'le Yoas'in komutasi altindaydilar. Adlari sunlardi: Azmavet'in ogullari Yeziel'le Pelet, Beraka, Anatotlu Yehu,
\par 4 Otuzlar'dan bir yigit ve Otuzlar'in önderi olan Givonlu Yismaya, Yeremya, Yahaziel, Yohanan, Gederali Yozavat,
\par 5 Eluzay, Yerimot, Bealya, Semarya, Haruflu Sefatya,
\par 6 Elkana, Yissiya, Azarel, Yoezer, Yasovam, Korahlilar,
\par 7 Yoela, Zevadya, Gedorlu Yeroham'in ogullari.
\par 8 Davut çölde saklandigi yerdeyken bazi Gadlilar da ona katildi. Bunlar savasa hazir, kalkan ve mizrak kullanabilen yigit adamlardi. Yüzleri aslan yüzü gibiydi, daglardaki ceylanlar kadar çeviktiler.
\par 9 Önderleri Ezer'di. Ikincisi Ovadya, üçüncüsü Eliav,
\par 10 dördüncüsü Mismanna, besincisi Yeremya,
\par 11 altincisi Attay, yedincisi Eliel,
\par 12 sekizincisi Yohanan, dokuzuncusu Elzavat,
\par 13 onuncusu Yeremya, on birincisi de Makbannay'di.
\par 14 Bu Gadlilar ordu komutanlariydi. En güçsüzleri yüz, güçlüleri bin kisinin yerini tutardi.
\par 15 Birinci ay*, Seria Irmagi her yana tasmisken, karsi yakaya geçtiler. Oradaki vadilerde oturanlarin tümünü doguya, batiya kaçirdilar.
\par 16 Benyamin ve Yahudaogullari'ndan bazi kisiler de Davut'un yanina, saklandigi yere gittiler.
\par 17 Onlari karsilamaya çikan Davut söyle dedi: "Eger bana yardim etmek için esenlikle geldiyseniz, buyrun bize katilin. Ama ben haksizlik yapmamisken beni düsmanlarimin eline teslim etmeye geldiyseniz, atalarimizin Tanrisi bunu görsün ve sizi yargilasin."
\par 18 Tanri'nin Ruhu Otuzlar'in önderi Amasay'in üzerine indi. Amasay söyle dedi: "Ey Davut, seniniz biz! Ey Isay oglu, seninleyiz! Esenlik olsun sana, esenlik! Seni destekleyenlere de esenlik olsun! Tanrin sana yardim edecektir." Davut onlari iyi karsiladi ve akincilarin basi yapti.
\par 19 Davut Filistliler'le birlikte Saul'a karsi savasmaya gidince, Manasse oymagindan da bazi kisiler onun yanina geçtiler. Ne var ki Davut'la adamlari Filistliler'e yardim etmediler. Çünkü Filist beyleri birbirlerine danisip, "Davut efendisi Saul'a dönerse basimizdan oluruz" diyerek onu geri göndermislerdi.
\par 20 Davut Ziklak'a gittiginde yanina geçen Manasseliler sunlardir: Adna, Yozavat, Yediael, Mikael, Yozavat, Elihu, Silletay. Bunlar Manasse'de bin kisilik birliklerin komutanlariydi.
\par 21 Düsman akincilarina karsi Davut'a yardim ettiler. Hepsi de yigit savasçilar, ordu komutanlariydi.
\par 22 Her gün insanlar Davut'a yardim etmeye geliyorlardi. Davut büyük, güçlü bir orduya sahip oluncaya dek bu böyle sürdü.
\par 23 RAB'bin sözü uyarinca Saul'un kralligini Davut'a vermek için Hevron'a, Davut'un yanina gelen savasçilarin sayisi sudur:
\par 24 Savasa hazir, kalkan ve mizrak tasiyan Yahudaogullari'ndan 6 800 kisi.
\par 25 Savasa hazir, yigit Simonogullari'ndan 7 100 kisi.
\par 26 Leviogullari'ndan 4 600 kisi.
\par 27 Harun ailesinin basi Yehoyada ve onunla birlikte olanlar 3 700 kisi.
\par 28 Genç yigit Sadok'la ailesinden 22 subay.
\par 29 Saul'un soyu Benyaminliler'den 3 000 kisi. Benyaminliler'in çogu o zamana dek Saul'un ailesine bagli kalmislardi.
\par 30 Efrayimogullari'ndan yigit savasçi ve boylarinda ün salmis 20 800 kisi.
\par 31 Davut'u kral yapmak için Manasse oymaginin yarisindan özel olarak seçilip gelenler 18 000 kisi.
\par 32 Issakarogullari'ndan 200 kisi. Bunlar Israilliler'in ne zaman ne yapmasi gerektigini bilen kisilerdi. Boy baslariydi ve bütün akrabalarini yönetirlerdi.
\par 33 Zevulun'dan 50 000 kisi. Bunlar savasa hazir, deneyimli askerlerdi. Her tür silahi kullanmada ustaydilar. Davut'a candan ve yürekten yardim ettiler.
\par 34 Naftali'den 1 000 subay ile kalkan ve mizrak tasiyan 37 000 kisi.
\par 35 Danlilar'dan savasa hazir 28 600 kisi.
\par 36 Aser'den savasa hazir, deneyimli 40 000 asker.
\par 37 Seria Irmagi'nin dogusunda yasayan, savas için her tür silahla donatilmis Rubenliler'den, Gadlilar'dan ve Manasse oymaginin yarisindan 120 000 kisi.
\par 38 Bunlarin hepsi savasa hazir yigit askerlerdi. Büyük bir kararlilikla Davut'u bütün Israil'in krali yapmak için Hevron'a geldiler. Geri kalan Israilliler de Davut'u kral yapma konusunda ayni düsüncedeydiler.
\par 39 Adamlar Davut'un yaninda üç gün kaldilar. Orada yiyip içtiler. Gereksinimlerini yakinlari saglamisti.
\par 40 Issakar, Zevulun ve Naftali'ye kadar yayilmis olan yakinlari da yiyecek yüklü eseklerle, develerle, katirlarla, öküzlerle geldiler. Bol miktarda un, incir pestili, kuru üzüm salkimlari, sarap, zeytinyagi, çok sayida sigir ve davar getirdiler. Çünkü Israil'de sevinç vardi.

\chapter{13}

\par 1 Davut binbasilara, yüzbasilara ve subaylarina danisti.
\par 2 Sonra bütün Israil topluluguna söyle seslendi: "Eger onaylarsaniz ve Tanrimiz RAB'bin istegiyse, Israil ülkesinin her yanina yayilmis öbür soydaslarimiza ve onlarla birlikte kendi kentlerinde ve otlaklarinda yasayan kâhinlerle Levililer'e haberciler gönderelim. Onlar da gelip bize katilsinlar.
\par 3 Tanrimiz'in Sandigi'ni geri getirelim. Çünkü Saul'un kralligi döneminde ona gereken önemi vermedik."
\par 4 Topluluk bu öneriyi benimseyerek sandigi geri getirmeye karar verdi.
\par 5 Davut Tanri'nin Antlasma Sandigi'ni* Kiryat-Yearim'den geri getirmek için Misir'daki Sihor Irmagi'ndan Levo-Hamat'a kadar bütün Israilliler'i topladi.
\par 6 Böylece Davut'la Israilliler Keruvlar* arasinda taht kuran RAB Tanri'nin adiyla anilan Tanri'nin Antlasma Sandigi'ni getirmek için Yahuda'daki Baala Kenti'ne -Kiryat-Yearim'e-gittiler.
\par 7 Tanri'nin Sandigi'ni Avinadav'in evinden alip yeni bir arabaya koydular. Arabayi Uzza'yla Ahyo sürüyordu.
\par 8 Bu arada Davut'la bütün Israil halki da Tanri'nin önünde lir, çenk, tef, zil ve borazanlar esliginde ezgiler okuyarak, var güçleriyle bu olayi kutluyorlardi.
\par 9 Kidon'un harman yerine vardiklarinda öküzler tökezledi. Bu nedenle Uzza elini uzatip sandigi tuttu.
\par 10 RAB sandiga elini uzatan Uzza'ya öfkelenerek onu yere çaldi. Uzza orada, Tanri'nin önünde öldü.
\par 11 Davut, RAB'bin Uzza'yi cezalandirmasina öfkelendi. O günden bu yana oraya Peres-Uzza denilir.
\par 12 Davut o gün Tanri'dan korkarak, "Tanri'nin Sandigi'ni nasil yanima getirsem?" diye düsündü.
\par 13 Sandigi yanina, Davut Kenti'ne götürecegine Gatli Ovet-Edom'un evine götürdü.
\par 14 Tanri'nin Sandigi Gatli Ovet-Edom'un evinde üç ay kaldi. RAB Ovet-Edom'un ailesini ve ona ait her seyi kutsadi.

\chapter{14}

\par 1 Sur Krali Hiram Davut'a ulaklar ve bir saray yapmak için sedir tomruklari, tasçilar, marangozlar gönderdi.
\par 2 Böylece Davut RAB'bin kendisini Israil Krali atadigini ve halki Israil'in hatiri için kralligini çok yücelttigini anladi.
\par 3 Davut Yerusalim'de kendine daha birçok kari aldi; bunlardan erkek ve kiz çocuklari oldu.
\par 4 Davut'un Yerusalim'de dogan çocuklarinin adlari sunlardi: Sammua, Sovav, Natan, Süleyman,
\par 5 Yivhar, Elisua, Elpelet,
\par 6 Nogah, Nefek, Yafia,
\par 7 Elisama, Beelyada, Elifelet.
\par 8 Filistliler Davut'un Israil Krali olarak meshedildigini* duyunca, bütün Filist ordusu onu aramak için yola çikti. Bunu duyan Davut onlari karsilamaya gitti.
\par 9 Filistliler gelip Refaim Vadisi'nde baskin yapmislardi.
\par 10 Davut Tanri'ya danisti: "Filistliler'e saldirayim mi? Onlari elime teslim edecek misin?" RAB, "Saldir" dedi, "Onlari eline teslim edecegim."
\par 11 Bunun üzerine Davut'la adamlari Baal-Perasim'e gittiler. Davut orada Filistliler'i bozguna ugratti. Sonra, "Her seyi yarip geçen sular gibi, Tanri düsmanlarimi benim elimle yarip geçti" dedi. Bundan ötürü oraya Baal-Perasim adi verildi.
\par 12 Filistliler putlarini orada biraktilar. Davut'un buyrugu uyarinca putlar yakildi.
\par 13 Filistliler bir kez daha gelip vadiye baskin yaptilar.
\par 14 Davut yine Tanri'ya danisti. Tanri söyle karsilik verdi: "Buradan saldirma! Onlari arkadan çevirip pelesenk agaçlarinin önünden saldir.
\par 15 Pelesenk agaçlarinin tepesinden yürüyüs sesi duyar duymaz, saldiriya geç. Çünkü ben Filist ordusunu bozguna ugratmak için önünsira gitmisim demektir."
\par 16 Davut Tanri'nin kendisine buyurdugu gibi yapti ve Filist ordusunu Givon'dan Gezer'e kadar bozguna ugratti.
\par 17 Böylece Davut'un ünü her yana yayildi. RAB bütün uluslarin ondan korkmasini sagladi.

\chapter{15}

\par 1 Davut kendisine Davut Kenti'nde evler yapti. Ardindan Tanri'nin Antlasma Sandigi* için bir yer hazirlayip çadir kurdu.
\par 2 Sonra, "Tanri'nin Antlasma Sandigi'ni yalniz Levililer tasiyacak" dedi, "Çünkü sandigi tasimak ve sonsuza dek kendisine hizmet etmek için RAB Levililer'i seçti."
\par 3 Davut RAB'bin Antlasma Sandigi'ni hazirlamis oldugu yere getirmek için bütün Israilliler'i Yerusalim'de topladi.
\par 4 Sonra Harunogullari'yla Levililer'i bir araya getirdi:
\par 5 Kehatogullari'ndan: Önder Uriel'le 120 yakini,
\par 6 Merariogullari'ndan: Önder Asaya'yla 220 yakini,
\par 7 Gersonogullari'ndan: Önder Yoel'le 130 yakini,
\par 8 Elisafanogullari'ndan: Önder Semaya'yla 200 yakini,
\par 9 Hevronogullari'ndan: Önder Eliel'le 80 yakini,
\par 10 Uzzielogullari'ndan: Önder Amminadav'la 112 yakini.
\par 11 Bundan sonra Davut Kâhin Sadok'la Kâhin Aviyatar'i, Levili Uriel'i, Asaya'yi, Yoel'i, Semaya'yi, Eliel'i, Amminadav'i yanina çagirdi.
\par 12 Onlara söyle dedi: "Siz Levili boylarin baslarisiniz. Levili kardeslerinizle birlikte kendinizi kutsayin, sonra Israil'in Tanrisi RAB'bin Antlasma Sandigi'ni hazirlamis oldugum yere getirin.
\par 13 Çünkü geçen sefer sandigi siz tasimadiginiz ve biz de kurala uygun davranmadigimiz için Tanrimiz RAB bize öfkelendi."
\par 14 Böylece kâhinlerle Levililer Israil'in Tanrisi RAB'bin Antlasma Sandigi'ni getirmek için kendilerini kutsadilar.
\par 15 RAB'bin sözü uyarinca ve Musa'nin onlara buyurdugu gibi, Tanri'nin Sandigi'nin siriklarini omuzlari üzerinde tasidilar.
\par 16 Davut Levili önderlere, kardeslerinden çenk, lir ve zil gibi çalgilar esliginde yüksek sesle sevinçli ezgiler okuyacak ezgiciler atamalarini söyledi.
\par 17 Levililer de Yoel oglu Heman'i, akrabalarindan Berekya oglu Asaf'i, akrabalari Merariogullari'ndan Kusaya oglu Etan'i atadilar.
\par 18 Onlara yardimci olarak da kapi nöbetçileri kardeslerinden Zekeriya'yi, Yaaziel'i, Semiramot'u, Yehiel'i, Unni'yi, Eliav'i, Benaya'yi, Maaseya'yi, Mattitya'yi, Elifelehu'yu, Mikneya'yi, Ovet-Edom'u, Yeiel'i atadilar.
\par 19 Ezgicilerden Heman, Asaf, Etan tunç* zil;
\par 20 Zekeriya, Aziel, Semiramot, Yehiel, Unni, Eliav, Maaseya, Benaya tiz perdeli çenk çalmak;
\par 21 Mattitya, Elifelehu, Mikneya, Ovet-Edom, Yeiel, Azazya sekiz telli lirle önderlik yapmak üzere seçildiler.
\par 22 Levililer'in önderi Kenanya ise ezgilerden sorumluydu. Bu konuda yetenekliydi.
\par 23 Berekya ile Elkana Antlasma Sandigi'nin bulundugu yerin kapi nöbetçileriydi.
\par 24 Sevanya, Yosafat, Netanel, Amasay, Zekeriya, Benaya, Eliezer adindaki kâhinler Tanri'nin Antlasma Sandigi önünde borazan çaliyordu. Ovet-Edom ile Yehiya da Antlasma Sandigi'nin bulundugu yerin kapi nöbetçileriydi.
\par 25 Böylece Davut, Israil ileri gelenleri ve binbasilar RAB'bin Antlasma Sandigi'ni Ovet-Edom'un evinden sevinçle getirmeye gittiler.
\par 26 Tanri, RAB'bin Antlasma Sandigi'ni tasiyan Levililer'e yardim ettigi için, yedi boga ile yedi koç kurban ettiler.
\par 27 Davut, sandigi tasiyan Levililer, ezgiciler ve ezgilerden sorumlu olan Kenanya ince ketenden cüppeler giyinmislerdi. Davut ince keten efod* da kusanmisti.
\par 28 Böylece Israilliler sevinç naralari atarak, boru, borazan, zil, çenk ve lirler çalarak RAB'bin Antlasma Sandigi'ni getiriyorlardi.
\par 29 RAB'bin Antlasma Sandigi Davut Kenti'ne varinca, Saul'un kizi Mikal pencereden bakti. Oynayip ziplayan Kral Davut'u görünce, onu içinden küçümsedi.

\chapter{16}

\par 1 Tanri'nin Antlasma Sandigi'ni getirip Davut'un bu amaçla kurdugu çadirin içine koydular. Tanri'ya yakmalik sunular* ve esenlik sunulari* sundular.
\par 2 Davut yakmalik sunulari ve esenlik sunularini sunmayi bitirince, RAB'bin adiyla halki kutsadi.
\par 3 Ardindan erkek, kadin her Israilli'ye birer somun ekmekle birer hurma ve üzüm pestili dagitti.
\par 4 RAB'bin Antlasma Sandigi önünde hizmet etmek, Israil'in Tanrisi RAB'bi anmak, O'na sükretmek ve övgüler sunmak için bazi Levililer'i atadi.
\par 5 Bunlarin önderi Asaf, yardimcisi Zekeriya'ydi. Öbürleri Yeiel, Semiramot, Yehiel, Mattitya, Eliav, Benaya, Ovet-Edom ve Yeiel'di. Bunlar çenk ve lir, Asaf yüksek sesli zil,
\par 6 Kâhin Benaya ile Yahaziel de Tanri'nin Antlasma Sandigi önünde sürekli borazan çalacaklardi.
\par 7 O gün Davut RAB'be sükretme isini ilk kez Asaf'la kardeslerine verdi.
\par 8 RAB'be sükredin, O'na yakarin, Halklara duyurun yaptiklarini!
\par 9 O'nu ezgilerle, ilahilerle övün, Bütün harikalarini anlatin!
\par 10 Kutsal adiyla övünün, Sevinsin RAB'be yönelenler!
\par 11 RAB'be ve O'nun gücüne bakin, Durmadan O'nun yüzünü arayin!
\par 12 Ey sizler, kulu Israil'in soyu, Seçtigi Yakupogullari, O'nun yaptigi harikalari, Olaganüstü islerini Ve agzindan çikan yargilari animsayin!
\par 14 Tanrimiz RAB O'dur, Yargilari bütün yeryüzünü kapsar.
\par 15 O'nun antlasmasini, Bin kusak için verdigi sözü, Ibrahim'le yaptigi antlasmayi, Ishak için içtigi andi sonsuza dek animsayin.
\par 17 "Hakkiniza düsen mülk olarak Kenan ülkesini size verecegim" diyerek, Bunu Yakup için bir kural, Israil'le sonsuza dek geçerli bir antlasma yapti.
\par 19 O zaman bir avuç insandiniz, Sayica az ve ülkeye yabanciydiniz.
\par 20 Bir ulustan öbürüne, Bir ülkeden ötekine dolasip durdular.
\par 21 RAB kimsenin onlari ezmesine izin vermedi, Onlar için krallari bile payladi:
\par 22 "Meshettiklerime* dokunmayin, Peygamberlerime kötülük etmeyin!" dedi.
\par 23 Ey bütün dünya, ezgiler söyleyin RAB'be! Her gün duyurun kurtarisini!
\par 24 Görkemini uluslara, Harikalarini bütün halklara anlatin!
\par 25 Çünkü RAB uludur, yalniz O övgüye deger, Ilahlardan çok O'ndan korkulur.
\par 26 Halklarin bütün ilahlari bir hiçtir, Oysa gökleri yaratan RAB'dir.
\par 27 Yücelik, ululuk O'nun huzurundadir, Güç ve sevinç O'nun konutundadir.
\par 28 Ey bütün halklar, RAB'bi övün, RAB'bin gücünü, yüceligini övün,
\par 29 RAB'bin görkemini adina yarasir biçimde övün, Sunular getirip O'nun önüne çikin! Kutsal giysiler içinde RAB'be tapinin!
\par 30 Titreyin O'nun önünde, ey bütün yeryüzündekiler! Dünya saglam kurulmus, sarsilmaz.
\par 31 Sevinsin gökler, cossun yeryüzü, Uluslar arasinda, "RAB egemenlik sürüyor!" densin.
\par 32 Gürlesin deniz içindekilerle birlikte, Bayram etsin kirlar ve üzerindekiler!
\par 33 O zaman RAB'bin önünde ormanin agaçlari Sevinçle haykiracak. Çünkü O yeryüzünü yargilamaya geliyor.
\par 34 RAB'be sükredin, çünkü O iyidir, Sevgisi sonsuzdur.
\par 35 Söyle seslenin: "Kurtar bizi, ey kurtaricimiz Tanri, Topla bizi, uluslarin arasindan çikar. Kutsal adina sükredelim, Yüceliginle övünelim.
\par 36 Israil'in Tanrisi RAB'be Öncesizlikten sonsuza dek övgüler olsun!" Bütün halk, "Amin!" diyerek RAB'be övgüler sundu.
\par 37 Davut RAB'bin Antlasma Sandigi'nin* önünde günlük islerde sürekli hizmet etmeleri için Asaf'la Levili kardeslerini atadi.
\par 38 Onlarla birlikte hizmet etmeleri için Ovet-Edom'la altmis sekiz Levili akrabasini da atadi. Yedutun oglu Ovet-Edom'la Hosa kapi nöbetçileriydi.
\par 39 Davut Kâhin Sadok'la öbür kâhin kardeslerini Givon'daki tapinma yerinde, RAB'bin Çadiri'nin bulundugu yerde görevlendirdi.
\par 40 Bunlar RAB'bin Israil'e verdigi yasada yazilanlar uyarinca, sabah aksam, düzenli olarak yakmalik sunu* sunaginda RAB'be sunular sunacaklardi.
\par 41 Onlarla birlikte Heman'la Yedutun'u ve RAB'bin sonsuz sevgisi için sükretsinler diye özel olarak seçilen öbürlerini de görevlendirdi.
\par 42 Heman'la Yedutun borazanlardan, zillerden ve Tanri'yi öven ezgiler için gereken öbür çalgilardan sorumluydu. Yedutunogullari'ni da kapida nöbetçi olarak görevlendirdi.
\par 43 Sonra herkes evine döndü. Davut da ailesini kutsamak için evine döndü.

\chapter{17}

\par 1 Davut sarayina yerlestikten sonra Peygamber Natan'a, "Bak, ben sedir agacindan yapilmis bir sarayda oturuyorum. Oysa RAB'bin Antlasma Sandigi* bir çadirin altinda duruyor!" dedi.
\par 2 Natan, "Tasarladigin her seyi yap, çünkü Tanri seninledir" diye karsilik verdi.
\par 3 O gece Tanri Natan'a söyle seslendi:
\par 4 "Git, kulum Davut'a söyle de: 'RAB diyor ki, oturmam için bana tapinak yapmayacaksin.
\par 5 Israil halkini Misir'dan çikardigim günden bu yana tapinakta oturmadim. Bir çadirdan öbür çadira, orada burada konaklayarak dolastim.
\par 6 Israilliler'le birlikte dolastigim yerlerin herhangi birinde, halkimi gütmesini buyurdugum Israil önderlerinden birine, neden bana sedir agacindan bir konut yapmadiniz diye hiç sordum mu?
\par 7 "Simdi kulum Davut'a söyle diyeceksin: 'Her Seye Egemen RAB diyor ki: Halkim Israil'e önder olasin diye seni otlaklardan ve koyun gütmekten aldim.
\par 8 Her nereye gittiysen seninleydim. Önünden bütün düsmanlarini yok ettim. Adini dünyadaki büyük adamlarin adi gibi büyük kilacagim.
\par 9 Halkim Israil için bir yurt saglayip onlari oraya yerlestirecegim. Bundan böyle kendi yurtlarinda otursunlar, bir daha rahatsiz edilmesinler. Kötü kisiler de halkim Israil'e hakimler atadigim günden bu yana yaptiklari gibi, bir daha onlara baski yapmasinlar. Bütün düsmanlarinin sana boyun egmesini saglayacagim. RAB'bin senin için bir soy yetistirecek, bilesin.
\par 11 "'Sen ölüp atalarina kavusunca, senden sonra ogullarindan birini ortaya çikarip kralligini pekistirecegim.
\par 12 Benim için tapinak kuracak olan odur. Ben de onun tahtini sonsuza dek sürdürecegim.
\par 13 Ben ona baba olacagim, o da bana ogul olacak. Senden önceki kraldan esirgedigim sevgiyi hiçbir zaman esirgemeyecegim.
\par 14 Onu sonsuza dek tapinagimin ve kralligimin üzerine atayacagim; tahti sonsuza dek sürecektir."
\par 15 Böylece Natan bütün bu sözleri ve görümleri Davut'a aktardi.
\par 16 Bunun üzerine Kral Davut gelip RAB'bin önünde oturdu ve söyle dedi: "Ya RAB Tanri, ben kimim, ailem nedir ki, beni bu duruma getirdin?
\par 17 Ey Tanri, sanki bu yetmezmis gibi, kulunun soyunun gelecegi hakkinda da söz verdin. Benimle de büyük bir adammisim gibi ilgilendin, ya RAB Tanri!
\par 18 Kulunu onurlandirdin, ben sana baska ne diyebilirim ki! Çünkü sen kulunu taniyorsun.
\par 19 Ya RAB, kulunun hatiri için ve istegin uyarinca bu büyüklügü gösterdin ve bu büyük vaatleri bildirdin.
\par 20 "Ya RAB, bir benzerin yok, senden baska tanri da yok! Bunu kendi kulaklarimizla duyduk.
\par 21 Halkin Israil'e benzer tek bir ulus yok dünyada. Kendi halkin olsun diye onlari kurtarmaya gittin. Büyük ve görkemli isler yapmakla ün saldin. Misir'dan kurtardigin halkin önünden uluslari kovdun.
\par 22 Halkin Israil'i sonsuza dek kendi halkin olarak seçtin ve sen de, ya RAB, onlarin Tanrisi oldun.
\par 23 "Simdi, ya RAB, kuluna ve onun soyuna iliskin verdigin sözü sonsuza dek tut, sözünü yerine getir.
\par 24 Öyle ki, insanlar, 'Israil'i kayiran, Her Seye Egemen RAB Tanri Israil'in Tanrisi'dir! diyerek adini sonsuza dek ansinlar, yüceltsinler ve kulun Davut'un soyu da önünde sürsün.
\par 25 "Sen, ey Tanrim, ben kulun için bir soy çikaracagini bana açikladin. Bundan dolayi kulun önünde sana dua etme yürekliligini buldu.
\par 26 Ya RAB, sen Tanri'sin! Kuluna bu iyi sözü verdin.
\par 27 Simdi önünde sonsuza dek sürmesi için kulunun soyunu kutsamayi uygun gördün. Çünkü, ya RAB, onu kutsadigin için sonsuza dek kutlu kilinacak."

\chapter{18}

\par 1 Bir süre sonra, Davut Filistliler'i yenip boyundurugu altina aldi; Gat'i ve çevresindeki köyleri Filistliler'in yönetiminden çikardi.
\par 2 Moavlilar'i da bozguna ugratti. Onlar Davut'un haraç ödeyen köleleri oldular.
\par 3 Davut Firat'a kadar kralligini pekistirmeye giden Sova Krali Hadadezer'i de Hama yakinlarinda yendi.
\par 4 Bin savas arabasini, yedi bin atlisini, yirmi bin yaya askerini ele geçirdi. Yüz savas arabasi için gereken atlarin disindaki bütün atlari da sakatladi.
\par 5 Sova Krali Hadadezer'e yardima gelen Sam Aramlilari'ndan yirmi iki bin kisiyi öldürdü.
\par 6 Sonra Sam Aramlilari'nin ülkesine askeri birlikler yerlestirdi. Onlar da Davut'un haraç ödeyen köleleri oldular. RAB Davut'u gittigi her yerde zafere ulastirdi.
\par 7 Davut Hadadezer'in komutanlarinin tasidigi altin kalkanlari alip Yerusalim'e götürdü.
\par 8 Ayrica Hadadezer'in yönetimindeki Tivhat ve Kun kentlerinden bol miktarda tunç* aldi. Süleyman bunu havuz, sütunlar ve çesitli esyalar yapmak için kullandi.
\par 9 Hama Krali Tou, Davut'un Sova Krali Hadadezer'in bütün ordusunu bozguna ugrattigini duydu.
\par 10 Tou Kral Davut'u selamlamak ve Hadadezer'le savasip yendigi için kutlamak üzere oglu Hadoram'i ona gönderdi. Çünkü Tou Hadadezer'le sürekli savasmisti. Hadoram Davut'a her türlü altin, gümüs, tunç armaganlar getirdi.
\par 11 Kral Davut bu armaganlari bütün uluslardan -Edom, Moav, Ammonlular, Filistliler ve Amalekliler'den- ele geçirdigi altin ve gümüsle birlikte RAB'be adadi.
\par 12 Seruya oglu Avisay Tuz Vadisi'nde on sekiz bin Edomlu öldürdü.
\par 13 Edom'a askeri birlikler yerlestirdi. Edomlular'in tümü Davut'un köleleri oldular. RAB Davut'u gittigi her yerde zafere ulastirdi.
\par 14 Bütün Israil'de krallik yapan Davut halkina dogruluk ve adalet sagladi.
\par 15 Seruya oglu Yoav ordu komutani, Ahilut oglu Yehosafat devlet tarihçisiydi.
\par 16 Ahituv oglu Sadok'la Aviyatar oglu Ahimelek kâhin, Savsa yazmandi.
\par 17 Yehoyada oglu Benaya Keretliler'le Peletliler'in* komutaniydi. Davut'un ogullari da sarayda yüksek görevlere atanmislardi.

\chapter{19}

\par 1 Bir süre sonra Ammon Krali Nahas öldü, yerine oglu kral oldu.
\par 2 Davut, "Babasi bana iyilik ettigi için ben de Nahas oglu Hanun'a iyilik edecegim" diye düsünerek, babasinin ölümünden dolayi bas sagligi dilemek için Hanun'a ulaklar gönderdi. Davut'un ulaklari Hanun'a bas sagligi dilemek için Ammonlular'in ülkesine varinca,
\par 3 Ammon önderleri Hanun'a söyle dediler: "Davut sana bas sagligi dileyen bu adamlari gönderdi diye babana saygi duydugunu mu saniyorsun? Bu ulaklar ülkeyi arastirmak, casusluk etmek, yikmak için buraya geldiler."
\par 4 Bunun üzerine Hanun Davut'un ulaklarini yakalatti. Sakallarini tiras edip giysilerinin kalçayi kapatan kesimini ortadan kesti ve onlari öylece gönderdi.
\par 5 Davut bunu duyunca, ulaklari karsilamak üzere adamlar gönderdi. Çünkü ulaklar çok utaniyorlardi. Kral, "Sakaliniz uzayincaya dek Eriha'da kalin, sonra dönün" diye buyruk verdi.
\par 6 Ammonlular Davut'un nefretini kazandiklarini anlayinca, Hanun'la Ammonlular Aram-Naharayim, Aram-Maaka ve Sova'dan savas arabalariyla atlilar kiralamak için bin talant gümüs gönderdiler.
\par 7 Otuz iki bin savas arabasi ve Maaka Krali'yla askerlerini kiraladilar. Maaka Krali'yla askerleri gelip Medeva'nin yakininda ordugah kurdular. Ammonlular da savasmak üzere kentlerinden çikip bir araya geldiler.
\par 8 Davut bunu duyunca, Yoav'i ve güçlü adamlardan olusan bütün ordusunu onlara karsi gönderdi.
\par 9 Ammonlular çikip kent kapisinda savas düzeni aldilar. Yardima gelen krallar da kirda savas düzenine girdiler.
\par 10 Önde, arkada düsman birliklerini gören Yoav, Israil'in en iyi askerlerinden bazilarini seçerek Aramlilar'in karsisina yerlestirdi.
\par 11 Geri kalan birlikleri de kardesi Avisay'in komutasina vererek Ammonlular'a karsi yerlestirdi.
\par 12 Yoav, "Aramlilar benden güçlü çikarsa, yardimima gelirsin" dedi, "Ama Ammonlular senden güçlü çikarsa, ben sana yardima gelirim.
\par 13 Güçlü ol! Halkimizin ve Tanrimiz'in kentleri ugruna yürekli olalim! RAB gözünde iyi olani yapsin."
\par 14 Yoav'la yanindakiler Aramlilar'a karsi savasmak için ileri atilinca, Aramlilar onlardan kaçti.
\par 15 Aramlilar'in kaçistigini gören Ammonlular da Yoav'in kardesi Avisay'dan kaçarak kente girdiler. Yoav ise Yerusalim'e döndü.
\par 16 Israilliler'in önünde bozguna ugradiklarini gören Aramlilar, ulaklar gönderip Firat Irmagi'nin karsi yakasinda, Hadadezer'in ordu komutani Sofak'in komutasindaki Aramlilar'i çagirdilar.
\par 17 Davut bunu duyunca, bütün Israil ordusunu topladi. Seria Irmagi'ndan geçerek onlara dogru ilerleyip karsilarinda savas düzeni aldi. Davut savasmak için düzen alinca, Aramlilar onunla savastilar.
\par 18 Ne var ki, Aramlilar Israilliler'in önünden kaçtilar. Davut onlardan yedi bin savas arabasi sürücüsü ile kirk bin yaya asker öldürdü. Ordu komutani Sofak'i da öldürdü.
\par 19 Hadadezer'in buyrugundaki krallar Israilliler'in önünde bozguna ugradiklarini görünce, Davut'la baris yaparak ona boyun egdiler. Aramlilar bundan böyle Ammonlular'a yardim etmekten kaçindilar.

\chapter{20}

\par 1 Ilkbaharda, krallarin savasa gittigi dönemde Yoav, komutasindaki orduyla birlikte yola çikti. Ammonlular'in ülkesini yerle bir edip Rabba Kenti'ni kusatirken Davut Yerusalim'de kaliyordu. Yoav Rabba Kenti'ne saldirip onu yerle bir etti.
\par 2 Davut Ammon Krali'nin basindaki taci aldi. Degerli taslarla süslü, agirligi bir talant altini bulan taci Davut'un basina koydular. Davut kentten çok miktarda mal yagmalayip götürdü.
\par 3 Orada yasayan halki disari çikarip testereyle, demir kazma ve baltayla yapilan islerde çalistirdi. Davut bunu bütün Ammon kentlerinde uyguladi. Sonra ordusuyla birlikte Yerusalim'e döndü.
\par 4 Bir süre sonra Filistliler'le Gezer'de savas çikti. Bu savas sirasinda Husali Sibbekay, Rafa soyundan Sippay'i öldürünce, Filistliler boyun egdiler.
\par 5 Israilliler'le Filistliler arasinda çikan bir baska savasta Yair oglu Elhanan, Gatli Golyat'in kardesi Lahmi'yi öldürdü. Golyat'in mizraginin sapi dokumaci tezgahinin sirigi gibiydi.
\par 6 Gat'ta bir kez daha savas çikti. Orada dev gibi bir adam vardi. Elleri, ayaklari altisar parmakliydi. Toplam yirmi dört parmagi vardi. O da Rafa soyundandi.
\par 7 Adam Israilliler'e meydan okuyunca, Davut'un kardesi Sima'nin oglu Yonatan onu öldürdü.
\par 8 Bunlar Gat'taki Rafa soyundandi. Davut'la adamlari tarafindan öldürüldüler.

\chapter{21}

\par 1 Seytan Israilliler'e karsi çikip Israil'de sayim yapmasi için Davut'u kiskirtti.
\par 2 Davut Yoav'la halkin önderlerine, "Gidin, Beer-Seva'dan Dan'a dek Israilliler'i sayin" dedi, "Sonra bana bilgi verin ki, halkin sayisini bileyim."
\par 3 Ama Yoav, "RAB halkini yüz kat daha çogaltsin" diye karsilik verdi, "Ey efendim kral, bunlar hepsi senin kullarin degil mi? Efendim neden bunu istiyor? Neden Israil'i suça sürüklüyor?"
\par 4 Gelgelelim kralin sözü Yoav'in sözünden baskin çikti. Böylece Yoav kralin yanindan ayrilip Israil'in her yanini dolasmaya gitti. Sonra Yerusalim'e dönerek
\par 5 sayimin sonucunu Davut'a bildirdi: Israil'de kiliç kusanabilen bir milyon yüz bin, Yahuda'daysa dört yüz yetmis bin kisi vardi.
\par 6 Yoav Levililer'le Benyaminliler'i saymadi; çünkü kralin bu konudaki buyrugunu benimsememisti.
\par 7 Tanri da yapilani uygun görmedi ve bu yüzden Israilliler'i cezalandirdi.
\par 8 Davut Tanri'ya, "Bunu yapmakla büyük günah isledim!" dedi, "Lütfen kulunun suçunu bagisla. Çünkü çok akilsizca davrandim."
\par 9 RAB Davut'un bilicisi* Gad'a söyle dedi:
\par 10 "Gidip Davut'a de ki, 'RAB söyle diyor: Önüne üç seçenek koyuyorum. Bunlardan birini seç de sana onu yapayim."
\par 11 Gad Davut'a gidip söyle dedi: "RAB diyor ki, 'Hangisini istiyorsun?
\par 12 Üç yil kitlik mi? Yoksa kiliçla seni kovalayan düsmanlarinin önünde üç ay kaçip yok olmak mi? Ya da RAB'bin kilicinin ve RAB'bin meleginin bütün Israil ülkesine üç gün salgin hastalik salmasini mi? Beni gönderene ne yanit vereyim, simdi iyice düsün."
\par 13 Davut, "Sikintim büyük" diye yanitladi, "Insan eline düsmektense, RAB'bin eline düseyim. Çünkü O'nun acimasi çok büyüktür."
\par 14 Bunun üzerine RAB Israil ülkesine salgin hastalik gönderdi. Yetmis bin Israilli öldü.
\par 15 Tanri Yerusalim'i yok etmek için bir melek gönderdi. Ama melek yikima baslayacagi sirada RAB onu gördü. Gönderecegi yikimdan vazgeçerek halki yok eden melege, "Yeter artik! Elini çek" dedi. RAB'bin melegi Yevuslu Ornan'in harman yerinde duruyordu.
\par 16 Davut basini kaldirip bakti. Elinde yalin bir kiliç olan RAB'bin melegini gördü. Melek elini Yerusalim'in üzerine uzatmis, yerle gök arasinda duruyordu. Çula sarinmis Davut'la halkin ileri gelenleri yüzüstü yere kapandilar.
\par 17 Davut Tanri'ya söyle seslendi: "Halkin sayilmasini buyuran ben degil miydim? Günah isleyen benim, kötülük yapan benim. Ama bu koyunlar ne yapti ki? Ya RAB Tanrim, ne olur beni ve babamin soyunu cezalandir. Bu salgin hastaligi halkin üzerinden kaldir."
\par 18 RAB'bin melegi Gad'a, Davut'un Yevuslu Ornan'in harman yerine gidip RAB'be bir sunak kurmasini buyurdu.
\par 19 Davut RAB'bin adiyla konusan Gad'in sözü uyarinca oraya gitti.
\par 20 Harman yerinde bugday döverken, Ornan arkasina dönüp melegi gördü. Yanindaki dört oglu gizlendi.
\par 21 Davut'un yaklastigini gören Ornan, harman yerinden çikti, varip Davut'un önünde yüzüstü yere kapandi.
\par 22 Davut Ornan'a, "RAB'be bir sunak kurmak üzere harman yerini bana sat" dedi, "Öyle ki, salgin hastalik halkin üzerinden kalksin. Harman yerini bana tam degerine satacaksin."
\par 23 Ornan, "Senin olsun!" diye karsilik verdi, "Efendim kral uygun gördügünü yapsin. Iste yakmalik sunular* için öküzleri, odun olarak dövenleri, tahil sunusu* olarak bugday veriyorum. Hepsini veriyorum."
\par 24 Ne var ki, Kral Davut, "Olmaz!" dedi, "Tam degerini ödeyip alacagim. Çünkü senin olani RAB'be vermem. Karsiligini ödemeden yakmalik sunu sunmam."
\par 25 Böylece Davut harman yeri için Ornan'a alti yüz sekel altin ödedi.
\par 26 Davut orada RAB'be bir sunak kurup yakmalik sunulari ve esenlik sunularini* sundu. RAB'be yakardi. RAB yakmalik sunu sunaginda gökten gönderdigi atesle onu yanitladi.
\par 27 Bundan sonra RAB melege kilicini kinina koymasini buyurdu. Melek buyruga uydu.
\par 28 RAB'bin kendisine Yevuslu Ornan'in harman yerinde yanit verdigini gören Davut, orada kurbanlar kesti.
\par 29 Musa'nin çölde RAB için yaptigi çadirla yakmalik sunu sunagi o sirada Givon'daki tapinma yerindeydi.
\par 30 Ama Davut Tanri'ya danismak için oraya gidemedi. Çünkü RAB'bin meleginin kilicindan korkuyordu.

\chapter{22}

\par 1 Davut, "RAB Tanri'nin Tapinagi ve Israil için yakmalik sunu sunagi burada olacak" dedi.
\par 2 Davut Israil'de yasayan yabancilarin toplanmasini buyurdu. Tanri'nin Tapinagi'ni kurmak için onlari yontma taslar hazirlamakla görevlendirdi.
\par 3 Giris kapilarinin çivileri ve kenetleri için çok miktarda demir, tartilamayacak kadar çok tunç* sagladi.
\par 4 Ayrica sayisiz sedir tomrugu da sagladi. Çünkü Saydalilar'la Surlular Davut'a çok sedir tomrugu getirmislerdi.
\par 5 Davut, "Oglum Süleyman genç ve deneyimsiz" dedi, "RAB için kurulacak tapinak bütün uluslarin gözünde çok büyük, ünlü ve görkemli olmali. Onun için hazirlik yapmaliyim." Böylece, ölmeden önce, tapinagin yapimi için büyük hazirlik yapti.
\par 6 Davut, oglu Süleyman'i yanina çagirdi. Onu Israil'in Tanrisi RAB için bir tapinak kurmakla görevlendirdi.
\par 7 Sonra Süleyman'a söyle dedi: "Oglum, Tanrim RAB'bin adina bir tapinak kurmak istedim.
\par 8 Ama RAB bana, 'Sen çok kan döktün, büyük savaslara katildin dedi, 'Benim adima tapinak kurmayacaksin. Çünkü yeryüzünde gözümün önünde çok kan döktün.
\par 9 Ama barissever bir oglun olacak. Onu her yandan kusatan düsmanlarindan kurtarip rahata kavusturacagim. Adi Süleyman olacak. Onun döneminde Israil'in baris ve güvenlik içinde yasamasini saglayacagim.
\par 10 Adima bir tapinak kuracak olan odur. O bana ogul olacak, ben de ona baba olacagim. Onun kralliginin tahtini Israil'de sonsuza dek sürdürecegim.
\par 11 "Simdi, oglum, RAB seninle olsun; Tanrin RAB kendisine tapinak kurman için verdigi söz uyarinca, seni basarili kilsin.
\par 12 Tanrin RAB seni Israil'e önder atadigi zaman yasasini yerine getirmen için sana saggörü ve anlayis versin.
\par 13 RAB'bin Israil'e Musa araciligiyla verdigi kurallara, ilkelere dikkatle uyarsan, basarili olacaksin. Güçlü ve yürekli ol! Korkma, yilma!
\par 14 "Iste sikintilar içinde RAB'bin Tapinagi için yüz bin talant altin, bir milyon talant gümüs, tartilamayacak kadar çok miktarda tunç, demir, tomruk ve tas sagladim. Sen de bunlara ekleyebilirsin.
\par 15 Birçok isçin var. Tasçilarin, duvarcilarin, marangozlarin ve her tür iste hünerli adamlarin var.
\par 16 Ölçülemeyecek kadar altinin, gümüsün, tuncun, demirin de var. Haydi, isi baslat. RAB seninle olsun!"
\par 17 Davut bütün Israil önderlerine oglu Süleyman'a yardim etmelerini buyurdu.
\par 18 Onlara, "Tanriniz RAB sizinle degil mi?" dedi, "Her yanda sizi rahata kavusturmadi mi? Çünkü bu ülkede yasayanlari elime teslim etti. Bu yüzden ülke RAB'bin ve halkinin yönetimi altindadir.
\par 19 Simdi yüreginizi ve caninizi Tanriniz RAB'be adayarak O'na yönelin. RAB Tanri'nin Tapinagi'ni yapmaya baslayin. Öyle ki, RAB'bin adina kuracaginiz tapinaga RAB'bin Antlasma Sandigi'ni* ve Tanri'ya ait kutsal esyalari getiresiniz."

\chapter{23}

\par 1 Davut çok yaslaninca, oglu Süleyman'i Israil Krali yapti.
\par 2 Davut Israil'in bütün önderlerini, kâhinleri, Levililer'i bir araya topladi.
\par 3 Otuz ve daha yukari yastaki Levililer sayildi. Toplami otuz sekiz bin erkekti.
\par 4 Bunlardan yirmi dört bini RAB'bin Tapinagi'nin islerini gözetecek, alti bini memur ve yargiç olacakti;
\par 5 dört bini kapi nöbetçisi olacak, dört bini de Davut'un RAB'bi övmek için sagladigi çalgilari çalacakti.
\par 6 Davut Levililer'i Levi'nin ogullarina göre bölümlere ayirdi: Gerson, Kehat, Merari.
\par 7 Gersonlular: Ladan, Simi.
\par 8 Ladan'in ogullari: Ilk oglu Yehiel, Zetam, Yoel. Toplam üç kisiydi.
\par 9 Bunlar Ladan boyunun boy baslariydi. Simi'nin ogullari: Selomit, Haziel, Haran. Toplam üç kisiydi.
\par 10 Simi'nin öbür ogullari: Yahat, Ziza, Yeus, Beria. Bu dördü Simi'nin ogullariydi.
\par 11 Yahat ilk, Ziza ikinci oguldu. Ancak Yeus'la Beria'nin çok sayida ogullari olmadigi için bir boy sayildilar.
\par 12 Kehat'in ogullari: Amram, Yishar, Hevron, Uzziel. Toplam dört kisi.
\par 13 Amram'in ogullari: Harun, Musa. Harun'la ogullari en kutsal esyalari korumak, RAB'bin önünde buhur yakmak, O'na hizmet etmek ve sonsuza dek O'nun adina halki kutsamak için atandilar.
\par 14 Tanri adami Musa'nin ogullari da Levi oymagindan sayildi.
\par 15 Musa'nin ogullari: Gersom, Eliezer.
\par 16 Gersom'un ogullari: Önder Sevuel.
\par 17 Eliezer'in oglu: Önder Rehavya. Eliezer'in baska oglu yoktu. Rehavya'ninsa birçok oglu vardi.
\par 18 Yishar'in ogullari: Önder Selomit.
\par 19 Hevron'un ogullari: Ilk oglu Yeriya, ikincisi Amarya, üçüncüsü Yahaziel, dördüncüsü Yekamam.
\par 20 Uzziel'in ogullari: Ilk oglu Mika, ikincisi Yissiya.
\par 21 Merari'nin ogullari: Mahli, Musi. Mahli'nin ogullari: Elazar, Kis.
\par 22 Elazar ogul sahibi olmadan öldü. Ama kizlari vardi. Amcalari Kis'in ogullari onlarla evlendi.
\par 23 Musi'nin ogullari: Mahli, Eder, Yeremot. Toplam üç kisi.
\par 24 Boylarina göre Leviogullari bunlardi. Boy baslarinin her biri kendi adiyla sayildi. Yirmi ve daha yukari yastaki Levililer, RAB'bin Tapinagi'nin islerinde görev aldilar.
\par 25 Çünkü Davut, "Israil'in Tanrisi RAB halkini rahata kavusturdu" demisti, "Yerusalim'i de sonsuza dek kendine konut seçti.
\par 26 Onun için Levililer'in RAB'bin Çadiri'ni ve tapinma hizmetinde kullanilan esyalari tasimalarina artik gerek yok."
\par 27 Davut'un son buyrugu uyarinca, yirmi ve daha yukari yastaki Levililer sayildi.
\par 28 Levililer'in görevi RAB'bin Tapinagi'nin hizmetinde Harunogullari'na yardim etmekti: Avlulardan, odalardan, kutsal esyalarin arinmasindan ve Tanri'nin Tapinagi'ndaki öbür hizmetlerden sorumluydular.
\par 29 Adak ekmeklerinden*, tahil sunusu* için kullanilan ince undan, mayasiz ince ekmekten, sacda pisirilen yiyeceklerden ve zeytinyagiyla karistirilan sunulardan, her tür hacim ve uzunluk ölçülerinden onlar sorumluydu.
\par 30 RAB'be sükretmek, övgüler sunmak üzere her sabah ve aksam tapinakta hazir bulunacaklardi.
\par 31 Sabat Günü*, Yeni Ay Töreni ve öbür bayramlarda RAB'be yakmalik sunular* sunuldugunda da hazir bulunacaklardi. RAB'bin önünde, kendileri için belirlenen ilkeler uyarinca uygun sayida sürekli hizmet edeceklerdi.
\par 32 Böylece Levililer Bulusma Çadiri'na ve kutsal yere bakma, RAB'bin Tapinagi'nin hizmetinde kardesleri Harunogullari'na yardim etme görevini üstlendiler.

\chapter{24}

\par 1 Harunogullari'nin bagli olduklari bölükler: Harun'un ogullari: Nadav, Avihu, Elazar, Itamar.
\par 2 Nadav'la Avihu babalarindan önce, ogul sahibi olamadan öldüler. Onun için Elazar'la Itamar kâhinlik yaptilar.
\par 3 Davut Elazar soyundan Sadok'la Itamar soyundan Ahimelek'in yardimiyla Harunogullari'ni yaptiklari göreve göre bölüklere ayirdi.
\par 4 Elazarogullari arasinda Itamarogullari'ndan daha çok önder oldugundan, buna göre bölündüler: Elazarogullari'ndan on alti boy basi, Itamarogullari'ndan ise sekiz boy basi çikti.
\par 5 Gerek Elazarogullari, gerekse Itamarogullari arasinda kutsal yerden ve Tanri'yla ilgili hizmetlerden sorumlu önderler vardi. Bu yüzden atanmalari kayirilmaksizin kurayla yapildi.
\par 6 Levili Netanel oglu Yazman Semaya, kralin ve görevlileri Kâhin Sadok, Aviyatar oglu Ahimelek, kâhinler ve Levililer'in boy baslarinin gözü önünde kura çekimini kaydetti. Kura sirayla, bir Elazar ailesinden, bir Itamar ailesinden çekildi.
\par 7 Birinci kura Yehoyariv'e düstü, Ikincisi Yedaya'ya.
\par 8 Üçüncüsü Harim'e, Dördüncüsü Seorim'e,
\par 9 Besincisi Malkiya'ya, Altincisi Miyamin'e,
\par 10 Yedincisi Hakkos'a, Sekizincisi Aviya'ya,
\par 11 Dokuzuncusu Yesu'ya, Onuncusu Sekanya'ya,
\par 12 On birincisi Elyasiv'e, On ikincisi Yakim'e,
\par 13 On üçüncüsü Huppa'ya, On dördüncüsü Yesevav'a,
\par 14 On besincisi Bilga'ya, On altincisi Immer'e,
\par 15 On yedincisi Hezir'e, On sekizincisi Happises'e,
\par 16 On dokuzuncusu Petahya'ya, Yirmincisi Yehezkel'e,
\par 17 Yirmi birincisi Yakin'e, Yirmi ikincisi Gamul'a,
\par 18 Yirmi üçüncüsü Delaya'ya, Yirmi dördüncüsü Maazya'ya düstü.
\par 19 Israil'in Tanrisi RAB'bin buyrugu uyarinca atalari Harun'un verdigi ilkelere göre RAB'bin Tapinagi'na gidip görev yapma sirasi buydu.
\par 20 Öbür Levililer: Amramogullari'ndan Suvael, Suvaelogullari'ndan Yehdeya.
\par 21 Rehavyaogullari'ndan önder Yissiya.
\par 22 Yisharogullari'ndan Selomot, Selomotogullari'ndan Yahat.
\par 23 Hevron'un ogullari: Ilk oglu Yeriya, ikincisi Amarya, üçüncüsü Yahaziel, dördüncüsü Yekamam.
\par 24 Uzziel'in oglu: Mika. Mika'nin oglu: Samir.
\par 25 Mika'nin kardesi: Yissiya. Yissiya'nin oglu: Zekeriya.
\par 26 Merariogullari: Mahli, Musi, Yaaziya.
\par 27 Merari'nin torunlarindan Yaaziya'nin ogullari: Soham, Zakkur, Ivri.
\par 28 Mahli'den: Elazar. Elazar'in oglu olmadi.
\par 29 Kis'ten: Kis oglu Yerahmeel.
\par 30 Musi'nin ogullari: Mahli, Eder, Yerimot. Levi soyundan gelen boylar bunlardir.
\par 31 Bunlar da kardesleri Harunogullari gibi, Kral Davut'un, Sadok'un, Ahimelek'in, kâhinler ve Levililer'in boy baslarinin gözü önünde kura çekti. Büyük boy baslari da, kardesleri olan küçük boy baslari da kura çektiler.

\chapter{25}

\par 1 Davut'la ordu komutanlari hizmet için Asaf'in, Heman'in, Yedutun'un bazi ogullarini ayirdilar. Bunlar lir, çenk ve ziller esliginde peygamberlikte bulunacaklardi. Bu göreve atananlarin listesi suydu:
\par 2 Asaf'in ogullarindan: Zakkur, Yusuf, Netanya, Asarela. Bunlar kralin buyrugu uyarinca peygamberlikte bulunan Asaf'in yönetimi altindaydilar.
\par 3 Yedutun'un ogullarindan: Gedalya, Seri, Yesaya, Simi, Hasavya, Mattitya. Toplam alti kisiydi. Lir esliginde peygamberlikte bulunan, RAB'be sükür ve övgü sunan babalari Yedutun'un sorumlulugu altindaydilar.
\par 4 Heman'in ogullarindan: Bukkiya, Mattanya, Uzziel, Sevuel, Yerimot, Hananya, Hanani, Eliata, Giddalti, Romamti-Ezer, Yosbekasa, Malloti, Hotir, Mahaziot.
\par 5 Hepsi kralin bilicisi* Heman'in ogullariydi. Tanri'nin sözü uyarinca bu ogullar Heman'i güçlendirmek için ona verilmisti. Tanri Heman'a on dört ogulla üç kiz verdi.
\par 6 Bunlarin tümü babalarinin sorumlulugu altinda RAB Tanri'nin Tapinagi'nda hizmet etmek için zil, çenk ve lirler esliginde ezgi söylerdi. Asaf, Yedutun, Heman kralin sorumlulugu altindaydi.
\par 7 RAB'be ezgi okumak için egitilmis yetenekli Levililer'in toplami 288 kisiydi.
\par 8 Bunlarin her biri, büyük küçük, ögretmen ögrenci ayrimi yapilmaksizin, görev dagitimi için kura çekti.
\par 9 Birinci kura Asaf soyundan Yusuf'a düstü; ogullari ve kardesleriyle birlikte 12 kisi.
\par 10 Üçüncüsü Zakkur'a; ogullari ve kardesleriyle birlikte 12 kisi.
\par 11 Dördüncüsü Yisri'ye; ogullari ve kardesleriyle birlikte 12 kisi.
\par 12 Besincisi Netanya'ya; ogullari ve kardesleriyle birlikte 12 kisi.
\par 13 Altincisi Bukkiya'ya; ogullari ve kardesleriyle birlikte 12 kisi.
\par 14 Yedincisi Yesarela'ya; ogullari ve kardesleriyle birlikte 12 kisi.
\par 15 Sekizincisi Yesaya'ya; ogullari ve kardesleriyle birlikte 12 kisi.
\par 16 Dokuzuncusu Mattanya'ya; ogullari ve kardesleriyle birlikte 12 kisi.
\par 17 Onuncusu Simi'ye; ogullari ve kardesleriyle birlikte 12 kisi.
\par 18 On birincisi Azarel'e; ogullari ve kardesleriyle birlikte 12 kisi.
\par 19 On ikincisi Hasavya'ya; ogullari ve kardesleriyle birlikte 12 kisi.
\par 20 On üçüncüsü Suvael'e; ogullari ve kardesleriyle birlikte 12 kisi.
\par 21 On dördüncüsü Mattitya'ya; ogullari ve kardesleriyle birlikte 12 kisi.
\par 22 On besincisi Yeremot'a; ogullari ve kardesleriyle birlikte 12 kisi.
\par 23 On altincisi Hananya'ya; ogullari ve kardesleriyle birlikte 12 kisi.
\par 24 On yedincisi Yosbekasa'ya; ogullari ve kardesleriyle birlikte 12 kisi.
\par 25 On sekizincisi Hanani'ye; ogullari ve kardesleriyle birlikte 12 kisi.
\par 26 On dokuzuncusu Malloti'ye; ogullari ve kardesleriyle birlikte 12 kisi.
\par 27 Yirmincisi Eliata'ya; ogullari ve kardesleriyle birlikte 12 kisi.
\par 28 Yirmi birincisi Hotir'e; ogullari ve kardesleriyle birlikte 12 kisi.
\par 29 Yirmi ikincisi Giddalti'ye; ogullari ve kardesleriyle birlikte 12 kisi.
\par 30 Yirmi üçüncüsü Mahaziot'a; ogullari ve kardesleriyle birlikte 12 kisi.
\par 31 Yirmi dördüncüsü Romamti-Ezer'e; ogullari ve kardesleriyle birlikte 12 kisi.

\chapter{26}

\par 1 Kapi nöbetçilerinin bölükleri: Korahlilar'dan: Asafogullari'ndan Kore oglu Meselemya.
\par 2 Meselemya'nin ogullari: Ilk oglu Zekeriya, ikincisi Yediael, üçüncüsü Zevadya, dördüncüsü Yatniel,
\par 3 besincisi Elam, altincisi Yehohanan, yedincisi Elyehoenay.
\par 4 Ovet-Edom'un ogullari: Ilk oglu Semaya, ikincisi Yehozavat, üçüncüsü Yoah, dördüncüsü Sakâr, besincisi Netanel,
\par 5 altincisi Ammiel, yedincisi Issakar, sekizincisi Peulletay. Çünkü Tanri Ovet-Edom'u kutsamisti.
\par 6 Ovet-Edom'un oglu Semaya'nin ogullari vardi. Bunlar yetenekli olduklarindan boy baslariydilar.
\par 7 Semaya'nin ogullari: Otni, Refael, Ovet, Elzavat. Semaya'nin akrabalari Elihu ile Semakya da yigit adamlardi.
\par 8 Bunlarin tümü Ovet-Edom soyundandi. Onlar da, ogullariyla akrabalari da yigit, görevlerinde becerikli kisilerdi. Ovet-Edom soyundan 62 kisi vardi.
\par 9 Meselemya'nin 18 becerikli oglu ve akrabasi vardi.
\par 10 Merariogullari'ndan Hosa'nin ogullari: Ilki Simri'ydi. Simri ilk ogul olmadigi halde babasi onu önder atamisti.-
\par 11 Ikincisi Hilkiya, üçüncüsü Tevalya, dördüncüsü Zekeriya. Hosa'nin ogullariyla akrabalarinin toplami 13 kisiydi.
\par 12 RAB'bin Tapinagi'nda Levili kardesleri gibi görev yapan kapi nöbetçisi bölükler ve baslarindaki adamlar bunlardir.
\par 13 Her kapi için her aile büyük, küçük ayirmadan kura çekti.
\par 14 Kurada Dogu Kapisi Selemya'ya düstü. Sonra bilge bir danisman olan oglu Zekeriya için kura çekildi. Ona Kuzey Kapisi düstü.
\par 15 Güney Kapisi Ovet-Edom'a, depo için çekilen kura da ogullarina düstü.
\par 16 Bati Kapisi ile yukari yol üzerindeki Salleket Kapisi için çekilen kuralar Suppim'le Hosa'ya düstü. Her ailenin nöbet zamanlari sirayla birbirini izliyordu.
\par 17 Dogu Kapisi'nda günde alti, Kuzey Kapisi'nda dört, Güney Kapisi'nda dört, depolarda da ikiserden dört Levili nöbetçi vardi.
\par 18 Bati Kapisi'na bakan avlu içinse, yolda dört, avluda iki nöbetçi bekliyordu.
\par 19 Korah ve Merari soyundan gelen kapi nöbetçilerinin bölükleri bunlardi.
\par 20 Levili Ahiya Tanri'nin Tapinagi'nin hazinelerinden ve Tanri'ya adanmis armaganlardan sorumluydu.
\par 21 Ladanogullari: Ladan soyundan gelen Gersonogullari, Ladan boyunun boy baslari Gersonlu Yehieli ve ogullariydi.
\par 22 Yehieli'nin ogullari: Zetam'la kardesi Yoel. Bunlar RAB'bin Tapinagi'nin hazinelerinden sorumluydu.
\par 23 Amram, Yishar, Hevron, Uzziel soylarindan su kisilere görev verildi:
\par 24 Musa oglu Gersom'un soyundan Sevuel tapinak hazinelerinin bas sorumlusuydu.
\par 25 Eliezer soyundan gelen kardesleri: Rehavya Eliezer'in ogluydu, Yesaya Rehavya'nin ogluydu, Yoram Yesaya'nin ogluydu, Zikri Yoram'in ogluydu, Selomit Zikri'nin ogluydu.
\par 26 Selomit'le kardesleri boy baslarinin, Kral Davut'un binbasilarinin, yüzbasilarinin ve öbür ordu komutanlarinin verdigi armaganlarin saklandigi hazinelerden sorumluydular.
\par 27 Bunlar savasta yagmalanan mallardan bir kismini RAB'bin Tapinagi'nin onarimi için ayirdilar.
\par 28 Bilici* Samuel'in, Kis oglu Saul'un, Ner oglu Avner'in, Seruya oglu Yoav'in verdigi armaganlarla öbür armaganlardan da Selomit'le kardesleri sorumluydu.
\par 29 Yisharlilar'dan: Kenanya ile ogullari Israil'de memur ve yargiç olarak tapinak disindaki islere bakmakla görevlendirildiler.
\par 30 Hevronlular'dan: Hasavya ile kardesleri -bin yedi yüz yigit adam- RAB'bin islerine ve kralin hizmetine bakmak için Israil'in Seria Irmagi'nin bati yakasinda kalan bölgesine atanmisti.
\par 31 Aile soy kütügündeki kayitlara göre Yeriya Hevronlular'in boy basiydi. Davut'un kralliginin kirkinci yilinda kayitlar incelendi ve Gilat'taki Yazer'de Hevronlular arasinda yigit savasçilar bulundu.
\par 32 Yeriya'nin aile baslari olan iki bin yedi yüz akrabasi vardi. Kral Davut bu yigit adamlari Tanri'nin ve kralin islerine bakmalari için Rubenliler'e, Gadlilar'a, Manasse oymaginin yarisina yönetici olarak atadi.

\chapter{27}

\par 1 Israil'de görev yapan Israilli boy baslarinin, binbasilarla yüzbasilarin ve görevlilerin listesi. Bunlar degisen birliklerde yil boyunca aydan aya her konuda krala hizmet ederlerdi. Her birlik 24 000 kisiden olusurdu.
\par 2 Birinci ay* için birinci birligin komutani Zavdiel oglu Yasovam'di. Komutasindaki birlik 24 000 kisiden olusuyordu.
\par 3 Birinci ay için görevlendirilen ordunun baskomutani Yasovam, Peres soyundandi.
\par 4 Ikinci ay için ikinci birligin komutani Ahohlu Doday'di. Miklot bu birligin bas görevlisiydi. Doday komutasindaki birlik 24 000 kisiden olusuyordu.
\par 5 Üçüncü ay için üçüncü birligin komutani Kâhin Yehoyada oglu önder Benaya'ydi. Komutasindaki birlik 24 000 kisiden olusuyordu.
\par 6 Otuz yigitlerden biri ve Otuzlar'in önderi olan Benaya'ydi bu. Oglu Ammizavat da onun birliginde görevliydi.
\par 7 Dördüncü ay için dördüncü birligin komutani Yoav'in kardesi Asahel'di. Sonradan yerine oglu Zevadya geçti. Birliginde 24 000 kisi vardi.
\par 8 Besinci ay için besinci birligin komutani Yizrahli Samhut'tu. Komutasindaki birlikte 24 000 kisi vardi.
\par 9 Altinci ay için altinci birligin komutani Tekoali Ikkes oglu Ira'ydi. Komutasindaki birlikte 24 000 kisi vardi.
\par 10 Yedinci ay için yedinci birligin komutani Efrayimogullari'ndan Pelonlu Heles'ti. Komutasindaki birlikte 24 000 kisi vardi.
\par 11 Sekizinci ay için sekizinci birligin komutani Zerahlilar'dan Husali Sibbekay'di. Komutasindaki birlikte 24 000 kisi vardi.
\par 12 Dokuzuncu ay için dokuzuncu birligin komutani Benyaminogullari'ndan Anatotlu Aviezer'di. Komutasindaki birlikte 24 000 kisi vardi.
\par 13 Onuncu ay için onuncu birligin komutani Zerahlilar'dan Netofali Mahray'di. Komutasindaki birlikte 24 000 kisi vardi.
\par 14 On birinci ay için on birinci birligin komutani Efrayimogullari'ndan Piratonlu Benaya'ydi. Komutasindaki birlikte 24 000 kisi vardi.
\par 15 On ikinci ay için on ikinci birligin komutani Otniel soyundan Netofali Helday'di. Komutasindaki birlikte 24 000 kisi vardi.
\par 16 Israil oymaklarinin yöneticileri: Ruben oymagi: Zikri oglu Eliezer. Simon oymagi: Maaka oglu Sefatya.
\par 17 Levi oymagi: Kemuel oglu Hasavya. Harunogullari: Sadok.
\par 18 Yahuda oymagi: Davut'un kardeslerinden Elihu. Issakar oymagi: Mikael oglu Omri.
\par 19 Zevulun oymagi: Ovadya oglu Yismaya. Naftali oymagi: Azriel oglu Yerimot.
\par 20 Efrayimogullari: Azazya oglu Hosea. Manasse oymaginin yarisi: Pedaya oglu Yoel.
\par 21 Gilat'taki Manasse oymaginin öbür yarisi: Zekeriya oglu Yiddo. Benyamin oymagi: Avner oglu Yaasiel.
\par 22 Dan oymagi: Yeroham oglu Azarel. Israil oymaklarinin yöneticileri bunlardi.
\par 23 Davut yirmi ve daha asagidaki yastakilerin sayimini yapmadi. Çünkü RAB Israil'i gökteki yildizlar kadar çogaltacagina söz vermisti.
\par 24 Seruya oglu Yoav da basladigi sayimi bitirmedi. Bu sayimdan ötürü RAB Israil'e öfkelendi. Bu yüzden sayimin sonucu Kral Davut'un tarihinde yazilmadi.
\par 25 Kralin hazinelerine yönetici olarak Adiel oglu Azmavet atanmisti. Açik bölgelerdeki, kentlerdeki, köylerdeki, kalelerdeki depolardan da Uzziya oglu Yehonatan sorumluydu.
\par 26 Topragi süren tarim isçilerinden: Keluv oglu Ezri,
\par 27 Baglardan: Ramali Simi, Üzümlerden ve sarap mahzenlerinden: Sefamli Zavdi,
\par 28 Sefela bölgesindeki zeytinliklerden ve yabanil incir agaçlarindan: Gederli Baal-Hanan, Zeytinyagi depolarindan: Yoas,
\par 29 Saron'da otlatilan sigirlardan: Saronlu Sitray, Vadilerdeki sigirlardan: Adlay oglu Safat,
\par 30 Develerden: Ismaili Ovil, Eseklerden: Meronotlu Yehdeya,
\par 31 Davarlardan: Hacerli Yaziz sorumluydu. Bunlarin hepsi Kral Davut'un servetinden sorumlu yöneticilerdi.
\par 32 Davut'un amcasi Yehonatan anlayisli bir yazman ve danismandi. Hakmoni oglu Yehiel kralin ogullarina bakardi.
\par 33 Ahitofel kralin danismaniydi. Arkli Husay kralin dostuydu.
\par 34 Ahitofel'den sonra yerine Benaya oglu Yehoyada'yla Aviyatar geçti. Yoav kralin ordu komutaniydi.

\chapter{28}

\par 1 Davut Israil'deki bütün yöneticilerin -oymak baslarinin, kralin hizmetindeki birlik komutanlarinin, binbasilarin, yüzbasilarin, kralla ogullarina ait servetten ve sürüden sorumlu kisilerin, saray görevlilerinin, bütün güçlü adamlarin ve yigit savasçilarin- Yerusalim'de toplanmasini buyurdu.
\par 2 Kral Davut ayaga kalkip onlara söyle dedi: "Ey kardeslerim ve halkim, beni dinleyin! RAB'bin Antlasma Sandigi*, Tanrimiz'in ayak basamagi için kalici bir yer yapmak istedim. Konutun yapimi için hazirlik yaptim.
\par 3 Ama Tanri bana, 'Adima bir tapinak kurmayacaksin dedi, 'Çünkü sen savasçi birisin, kan döktün.
\par 4 "Israil'in Tanrisi RAB, sonsuza dek Israil Krali olmam için bütün ailem arasindan beni seçti. Önder olarak Yahuda'yi, Yahuda oymagindan da babamin ailesini seçti. Babamin ogullari arasindan beni bütün Israil'in krali yapmayi uygun gördü.
\par 5 Bütün ogullarim arasindan -RAB bana birçok ogul verdi-Israil'de RAB'bin kralliginin tahtina oturtmak için oglum Süleyman'i seçti.
\par 6 RAB bana söyle dedi: 'Tapinagimi ve avlularimi yapacak olan oglun Süleyman'dir. Onu kendime ogul seçtim. Ben de ona baba olacagim.
\par 7 Bugün yaptigi gibi buyruklarima, ilkelerime dikkatle uyarsa, kralligini sonsuza dek sürdürecegim.
\par 8 "Simdi, Tanrimiz'in önünde, RAB'bin toplulugu olan bütün Israil'in gözü önünde size sesleniyorum: Tanriniz RAB'bin bütün buyruklarina uymaya dikkat edin ki, bu verimli ülkeyi mülk edinip sonsuza dek çocuklariniza miras olarak veresiniz.
\par 9 "Sen, ey oglum Süleyman, babanin Tanrisi'ni tani. Bütün yüreginle ve istekle O'na kulluk et. Çünkü RAB her yüregi arastirir, her düsüncenin ardindaki amaci saptar. Eger O'na yönelirsen, kendisini sana buldurur. Ama O'nu birakirsan, seni sonsuza dek reddeder.
\par 10 Simdi dinle! Tapinagi yapmak için RAB seni seçti. Yürekli ol, ise basla!"
\par 11 Davut tapinaga ait eyvanin, binalarin -hazine odalarinin, yukariyla iç odalarin ve Bagislanma Kapagi'nin bulundugu yerin-tasarilarini oglu Süleyman'a verdi.
\par 12 Ruh araciligiyla kendisine açiklanan bütün tasarilari verdi: RAB'bin Tapinagi'nin avlulariyla çevredeki bütün odalarin, Tanri'nin Tapinagi'nin hazinelerinin, adanmis armaganlarin konulacagi yerlerin ölçülerini verdi.
\par 13 Kâhinlerle Levililer'in bölüklerine iliskin kurallari, RAB'bin Tapinagi'ndaki hizmet ve hizmette kullanilan bütün esyalarla ilgili ilkeleri,
\par 14 degisik hizmetlerde kullanilan altin esyalar için saptanan altin miktarini, degisik hizmetlerde kullanilan gümüs esyalar için saptanan gümüs miktarini,
\par 15 altin kandilliklerle kandiller -her bir altin kandillikle kandil- için saptanan altini, her kandilligin kullanis biçimine göre gümüs kandilliklerle kandiller için saptanan gümüsü,
\par 16 adak ekmeklerinin* dizildigi masalar için belirlenen altini, gümüs masalar için gümüsü,
\par 17 büyük çatallar, çanaklar ve testiler için belirlenen saf altini, her altin tas için saptanan altini, her gümüs tas için saptanan gümüsü,
\par 18 buhur sunagi için saptanan saf altin miktarini bildirdi. Davut arabanin -RAB'bin Antlasma Sandigi'na kanatlarini yayarak onu örten altin Keruvlar'in*- tasarisini da verdi.
\par 19 "Bütün bunlar RAB'bin eli üzerimde oldugu için bana bildirildi" dedi, "Ben de tasarinin bütün ayrintilarini yazili olarak veriyorum."
\par 20 Sonra oglu Süleyman'a, "Güçlü ve yürekli ol!" dedi, "Ise giris. Korkma, yilma. Çünkü benim Tanrim, RAB Tanri seninledir. RAB'bin Tapinagi'nin bütün yapim isleri bitinceye dek seni basarisizliga ugratmayacak, seni birakmayacaktir.
\par 21 Tanri'nin Tapinagi'nin yapim isleri için kâhinlerle Levililer'in bölükleri burada hazir bekliyor. Her tür yapim isinde usta ve istekli adamlar sana yardim edecek. Önderler de bütün halk da senin buyruklarini bekliyor."

\chapter{29}

\par 1 Kral Davut bütün topluluga söyle dedi: "Tanri'nin seçtigi oglum Süleyman genç ve deneyimsizdir. Is büyüktür. Çünkü bu tapinak insan için degil, RAB Tanri içindir.
\par 2 Tanrim'in Tapinagi'na gereç saglamak için var gücümle çalistim. Altin esyalar için altin, gümüs için gümüs, tunç* için tunç, demir için demir, ahsap için agaç sagladim. Ayrica oniks, kakma taslar, süs taslari, çesitli renklerde degerli taslar ve çok miktarda mermer sagladim.
\par 3 Bu kutsal tapinak için sagladiklarimin yanisira, Tanrim'in Tapinagi'na sevgim yüzünden, kisisel servetimden de altin ve gümüs de veriyorum:
\par 4 Üç bin talant Ofir altini ve tapinagin duvarlarini kaplamak için yedi bin talant kaliteli gümüs.
\par 5 Bunlari altin, gümüs gerektiren islerde ve sanatkârlarin el isçiliginde kullanilmak üzere veriyorum. Kim bugün kendini RAB'be adamak istiyor?"
\par 6 Bunun üzerine boy baslari, Israil'in oymak önderleri, binbasilar, yüzbasilar ve saray yöneticileri gönülden armaganlar verdiler.
\par 7 Tanri'nin Tapinagi'nin yapimi için bes bin talant, on bin darik altin, on bin talant gümüs, on sekiz bin talant tunç*, yüz bin talant demir bagisladilar.
\par 8 Degerli taslari olanlar, Gersonlu Yehiel'in denetiminde, bunlari RAB'bin Tapinagi'nin hazinesine verdi.
\par 9 Halk verdigi armaganlar için seviniyordu. Çünkü herkes RAB'be içtenlikle ve gönülden vermisti. Kral Davut da çok sevinçliydi.
\par 10 Davut bütün toplulugun gözü önünde RAB'bi övdü. Söyle dedi: "Ey atamiz Israil'in Tanrisi RAB, Sonsuzluk boyunca sana övgüler olsun!
\par 11 Ya RAB, büyüklük, güç, yücelik, Zafer ve görkem senindir. Gökte ve yerde olan her sey senindir. Egemenlik senindir, ya RAB! Sen her seyden yücesin.
\par 12 Zenginlik ve onur senden gelir. Her seye egemensin. Güç ve yetki senin elindedir. Birini yükseltmek ve güçlendirmek Senin elindedir.
\par 13 Simdi, ey Tanrimiz, sana sükrederiz, Görkemli adini överiz.
\par 14 "Ama ben kimim, halkim kim ki, böyle gönülden armaganlar verebilelim? Her sey sendendir. Biz ancak senin elinden aldiklarimizi sana verdik.
\par 15 Senin önünde garibiz, yabanciyiz atalarimiz gibi. Yeryüzündeki günlerimiz bir gölge gibidir, kalici degildir.
\par 16 Ya RAB Tanrimiz, kutsal adina bir tapinak yapmak için sagladigimiz bu büyük servet senin elindendir, hepsi senindir.
\par 17 Yüregi sinadigini, dogruluktan hoslandigini bilirim. Her seyi içtenlikle, gönülden verdim. Simdi burada olan halkinin sana nasil istekle bagislar verdigini sevinçle gördüm.
\par 18 Ya RAB, atalarimiz Ibrahim'in, Ishak'in, Israil'in Tanrisi, bu istegi sonsuza dek halkinin yüreginde ve düsüncesinde tut, onlarin sana bagli kalmalarini sagla.
\par 19 Oglum Süleyman'a bütün buyruklarina, uyarilarina, kurallarina uymak, hazirligini yaptigim tapinagi kurmak için istekli bir yürek ver."
\par 20 Sonra Davut bütün topluluga, "Tanriniz RAB'bi övün!" dedi. Böylece hepsi atalarinin Tanrisi RAB'bi övdü; Tanri'nin ve kralin önünde baslarini egip yere kapandi.
\par 21 Ertesi gün halk RAB'be kurbanlar kesip yakmalik sunular* sundu: Bin boga, bin koç, bin kuzunun yanisira, bütün Israilliler için dökmelik sunular ve birçok baska kurban.
\par 22 O gün Israilliler büyük bir sevinçle RAB'bin önünde yiyip içtiler. Bundan sonra Davut oglu Süleyman'i ikinci kez kral olarak onayladilar. Süleyman'i RAB'bin önünde önder, Sadok'u da kâhin olarak meshettiler*.
\par 23 Böylece Süleyman babasi Davut'un yerine RAB'bin tahtina oturdu. Basarili oldu. Bütün Israil halki onun sözüne uydu.
\par 24 Yöneticilerin, güçlü kisilerin ve Davut'un ogullarinin tümü Kral Süleyman'a bagli kalacaklarina söz verdiler.
\par 25 RAB bütün Israilliler'in gözünde Süleyman'i çok yükseltti ve daha önce Israil'de hiçbir kralin erisemedigi bir krallik görkemiyle donatti.
\par 26 Isay oglu Davut bütün Israil'de krallik yapti.
\par 27 Yedi yil Hevron'da, otuz üç yil Yerusalim'de olmak üzere toplam kirk yil Israil'de krallik yapti.
\par 28 Güzel bir yaslilik döneminde öldü. Zenginlik ve onur dolu günler yasadi. Yerine oglu Süleyman kral oldu.
\par 29 Kral Davut'un kralligi dönemindeki öteki olaylar, basindan sonuna dek Bilici* Samuel'in, Peygamber Natan'in, Bilici Gad'in tarihinde yazilidir.
\par 30 Kralligi dönemindeki bütün ayrintilar, ne denli güçlü oldugu, basindan geçen olaylar, Israil'de ve çevredeki ülkelerde olup bitenler bu tarihte yazilidir.


\end{document}