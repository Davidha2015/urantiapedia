\begin{document}

\title{2 Tarihler}


\chapter{1}

\par 1 Davut oglu Süleyman kralligini saglamlastirdi. Çünkü Tanrisi RAB onunlaydi ve onu çok yüceltti.
\par 2 Süleyman bütün Israilliler'i -binbasilari, yüzbasilari, yargiçlari, Israil'in boy baslari olan önderleri- çagirtti.
\par 3 Sonra bütün toplulukla birlikte Givon'daki tapinma yerine gitti. Çünkü RAB'bin kulu Musa'nin çölde yaptigi Tanri'yla Bulusma Çadiri oradaydi.
\par 4 Ancak Davut Tanri'nin Antlasma Sandigi'ni* Kiryat-Yearim'den getirip Yerusalim'de hazirladigi çadira koymustu.
\par 5 Hur oglu Uri oglu Besalel'in yaptigi tunç* sunagi da Givon'da RAB'bin Konutu'nun önüne yerlestirmisti. Süleyman'la topluluk orada RAB'be danistilar.
\par 6 Süleyman RAB'bin önüne, Bulusma Çadiri'nin önündeki tunç sunaga çikarak üzerinde bin yakmalik sunu* sundu.
\par 7 Tanri o gece Süleyman'a görünüp, "Sana ne vermemi istersin?" diye sordu.
\par 8 Süleyman, "Babam Davut'a büyük iyilikler yaptin" diye karsilik verdi, "Beni de onun yerine kral atadin.
\par 9 Ya RAB Tanri, babam Davut'a verdigin söz yerine gelsin! Beni yeryüzünün tozu kadar çok olan bir halkin krali yaptin.
\par 10 Simdi bu halki yönetebilmem için bana bilgi ve bilgelik ver. Baska türlü senin bu büyük halkini kim yönetebilir!"
\par 11 Tanri Süleyman'a, "Demek yüreginin dilegi bu" dedi, "Zenginlik, mal mülk, onur ya da senden nefret edenlerin ölümünü istemedin, kendin için uzun ömür de istemedin. Bunlarin yerine seni basina kral yaptigim halkimi yönetmek için bilgi ve bilgelik istedin.
\par 12 Sana bilgi ve bilgelik verilecektir. Sana ayrica öyle bir zenginlik, mal mülk ve onur verecegim ki, benzeri ne senden önceki krallarda görülmüstür, ne de senden sonrakilerde görülecektir."
\par 13 Bundan sonra Süleyman Givon'daki tapinma yerinden, Bulusma Çadiri'ndan ayrilip Yerusalim'e gitti. Israil'i oradan yönetti.
\par 14 Kral Süleyman savas arabalariyla atlarini topladi. Bin dört yüz savas arabasi, on iki bin ati vardi. Bunlarin bir kismini savas arabalari için ayrilan kentlere, bir kismini da kendi yanina, Yerusalim'e yerlestirdi.
\par 15 Kralligi döneminde Yerusalim'de altin ve gümüs tas degerine düstü. Sedir agaçlari Sefela'daki yabanil incir agaçlari kadar bollasti.
\par 16 Süleyman'in atlari Misir ve Keve'den getirilirdi. Kralin tüccarlari atlari Keve'den satin alirdi.
\par 17 Misir'dan bir savas arabasi alti yüz, bir at yüz elli sekel gümüse getirilirdi. Bunlari bütün Hitit* ve Aram krallarina satarlardi.

\chapter{2}

\par 1 Süleyman RAB adina bir tapinak, kendisi için de bir saray yaptirmaya karar verdi.
\par 2 Yük tasimak için yetmis bin, daglarda tas kesmek için seksen bin, bunlara gözcülük etmek için de üç bin alti yüz kisi görevlendirdi.
\par 3 Sur Krali Hiram'a da su haberi gönderdi: "Babam Davut'un oturmasi için saray yapilirken kendisine gönderdigin sedir tomruklarindan bana da gönder.
\par 4 Tanrim RAB'be adamak üzere, O'nun adina bir tapinak yapiyorum. Bu tapinakta hos kokulu buhur yakip adak ekmeklerini* sürekli olarak masaya dizecegiz. Sabah aksam, her Sabat Günü*, her Yeni Ay ve Tanrimiz RAB'bin belirledigi bayramlarda orada yakmalik sunular* sunacagiz. Israil'e bunlari sürekli yapmasi buyruldu.
\par 5 "Yapacagim tapinak büyük olacak. Çünkü Tanrimiz bütün tanrilardan büyüktür.
\par 6 Ama O'na bir tapinak yapmaya kimin gücü yeter? Çünkü O göklere, göklerin göklerine bile sigmaz. Ben kimim ki O'na bir tapinak yapayim! Ancak önünde buhur yakilabilecek bir yer yapabilirim.
\par 7 "Bana bir adam gönder; Yahuda ve Yerusalim'de babam Davut'un yetistirdigi ustalarimla çalissin. Altin, gümüs, tunç* ve demiri islemede; mor, kirmizi, lacivert kumas dokumada, oymacilikta usta olsun.
\par 8 "Bana Lübnan'dan sedir, çam, algum tomruklari da gönder. Adamlarinin oradaki agaçlari kesmekte usta olduklarini biliyorum. Benim adamlarim da seninkilerle birlikte çalissin.
\par 9 Öyle ki, bana çok sayida tomruk saglayabilsinler. Çünkü yapacagim tapinak büyük ve görkemli olacak.
\par 10 Agaç kesen adamlarina yirmi bin kor bulgur, yirmi bin kor arpa, yirmi bin bat sarap, yirmi bin bat zeytinyagi verecegim."
\par 11 Sur Krali Hiram Süleyman'a mektupla su yaniti gönderdi: "RAB halkini sevdigi için, seni onlarin krali yapti."
\par 12 Hiram mektubunu söyle sürdürdü: "Yeri gögü yaratan Israil'in Tanrisi RAB'be övgüler olsun! Kral Davut'a bilge bir ogul verdi; RAB için bir tapinak, kendisi için de bir saray yapacak akilli ve anlayisli bir ogul.
\par 13 "Sana Huram-Avi adinda usta ve akilli birini gönderiyorum.
\par 14 Annesi Danli, babasi Surlu'dur. Altin, gümüs, tunç, demir, tas ve tahta islemekte; ince keten, mor, lacivert ve kirmizi kumas dokumakta ustadir. Her türlü oymacilikta usta oldugu gibi her tasarimi uygulayabilecek yetenektedir. Ustalarinla ve babanin, efendim Davut'un yetistirdigi ustalarla çalisacak.
\par 15 "Efendim, sözünü ettigin bugday, arpa, zeytinyagi ve sarabi kullarina gönder.
\par 16 Biz de sana gereken bütün tomruklari Lübnan'da keser, deniz yoluyla, sallarla Yafa'ya kadar yüzdürürüz. Sonra sen tomruklari alip Yerusalim'e götürürsün."
\par 17 Babasi Davut'un yaptigi sayimdan sonra, Süleyman da Israil'de yasayan bütün yabancilar arasinda bir sayim yapti. Yabancilarin sayisi 153 600 kisi olarak belirlendi.
\par 18 Bunlardan 70 000'_ne yük tasima, 80 000'ine daglarda tas kesme, 3 600'üne de isçileri çalistirma görevi verildi.

\chapter{3}

\par 1 Süleyman bundan sonra RAB'bin Yerusalim'de babasi Davut'a göründügü Moriya Dagi'nda RAB'bin Tapinagi'ni yaptirmaya basladi. Yevuslu Ornan'in olan bu harman yerini Davut saglamisti.
\par 2 Süleyman kralliginin dördüncü yilinin ikinci ayinin* ikinci gününde yapiyi baslatti.
\par 3 Tanri'nin Tapinagi için attigi temel, eski ölçülere göre altmis arsin uzunlugunda, yirmi arsin genisligindeydi.
\par 4 Tapinagin ön cephesini boydan boya kaplayan eyvaninin genisligi yirmi arsin, yüksekligi yüz yirmi arsindi. Süleyman iç duvarlari saf altinla kaplatti.
\par 5 Ana bölümün duvarlarini da önce çam tahtasiyla, sonra saf altinla kaplatti; hurma agaci ve zincir motifleriyle süsletti.
\par 6 Tapinagi degerli taslarla bezetti. Kullanilan altin Parvayim'den getirilmisti.
\par 7 Kirisleri, kapi esiklerini, duvarlarla kapilari altinla kaplatti. Duvarlara Keruvlar* oydurdu.
\par 8 En Kutsal Yer'i* yapti: Uzunlugu tapinagin genisligine esitti; uzunlugu da genisligi de yirmiser arsindi. En Kutsal Yer'in iç duvarlarini alti yüz talant saf altinla kaplatti.
\par 9 Altin çivilerin agirligi elli sekeldi. Süleyman yukari odalari da altinla kaplatti.
\par 10 En Kutsal Yer'de iki Keruv heykeli yaptirarak altinla kaplatti.
\par 11 Keruvlar'in kanatlarinin uzunlugu yirmi arsindi. Keruvlar'dan birinin kanadi bes arsindi ve tapinagin duvarina erisiyordu. Öbür kanat da bes arsindi ve öteki Keruv'un kanadina degiyordu.
\par 12 Ayni sekilde öteki Keruv'un da kanadi bes arsindi*fm* ve tapinagin duvarina erisiyordu. Öbür kanat da bes arsindi*fm* ve birinci Keruv'un kanadina degiyordu.
\par 13 Ayakta duran ve açilmis kanatlarinin uzunlugu yirmi arsin*fl* olan Keruvlar'in yüzü ana bölüme bakiyordu.
\par 14 En Kutsal Yer'in perdesi lacivert, mor, kirmizi kumastan ve ince ketenden yapilmisti. Üzerinde Keruv islemeleri vardi.
\par 15 Süleyman otuz beser arsin yüksekliginde iki sütun yaptirip tapinagin önüne diktirdi. Sütun basliklari beser arsin yüksekligindeydi.
\par 16 Gerdanliga benzer zincirler yaptirarak sütunlarin üzerine koydurdu. Yüz nar motifi yaptirip zincirlere taktirdi.
\par 17 Sütunlari tapinagin önüne diktirip sagdakine Yakin, soldakine Boaz adini verdi.

\chapter{4}

\par 1 Süleyman tunçtan* bir sunak yaptirdi. Sunagin eniyle uzunlugu yirmiser arsin, yüksekligi on arsindi.
\par 2 Dökme tunçtan on arsin çapinda, bes arsin derinliginde, çevresi otuz arsin yuvarlak bir havuz yaptirdi.
\par 3 Havuzun disi boga kabartmalariyla kusatilmisti. Her arsinda onar tane olan bu kabartmalar iki sira halindeydi ve gövdeyle birlikte dökülmüstü.
\par 4 Havuz üçü kuzeye, üçü batiya, üçü güneye, üçü de doguya bakan on iki boga heykeli üzerine oturtulmustu. Bogalarin sagrilari içe dönüktü.
\par 5 Havuzun çeperi dört parmak kalinligindaydi; kenarlari kâse kenarlarini, nilüferleri andiriyordu. Üç bin bat su aliyordu.
\par 6 Süleyman yikama isleri için on kazan yaptirdi. Besini sunagin güneyine, besini kuzeyine yerlestirdi. Yakmalik sunularin* parçalari bunlarda yikanirdi. Havuz ise kâhinlerin* yikanmasi içindi.
\par 7 Süleyman tanima uygun biçimde yaptirdigi on altin kandilligi tapinagin içine, besi sagda, besi solda olmak üzere yerlestirdi.
\par 8 Yaptirdigi on masanin besini de tapinagin sagina, besini soluna yerlestirdi. Ayrica yüz altin çanak yaptirdi.
\par 9 Kâhinlerin avlusunu, büyük avluyla kapilarini yaptirdi. Kapilari tunçla kaplatti.
\par 10 Havuzu ise tapinagin güneydogu kösesine yerlestirdi.
\par 11 Hiram kovalar, kürekler, çanaklar yapti. Böylece Kral Süleyman için üstlenmis oldugu Tanri'nin Tapinagi'yla ilgili isleri tamamlamis oldu:
\par 12 Iki sütun ve iki yuvarlak sütun basligi, bu basliklari süsleyen iki örgülü ag,
\par 13 Sütunlarin yuvarlak basliklarini süsleyen iki örgülü agin üzerini ikiser sira halinde süsleyen dört yüz nar motifi,
\par 14 On kazan ve ayakliklari,
\par 15 Havuz ve havuzu tasiyan on iki boga heykeli,
\par 16 Kovalar, kürekler, büyük çatallar. Huram-Avi'nin Kral Süleyman için RAB'bin Tapinagi'na yaptigi bütün esyalar parlak tunçtandi.
\par 17 Kral bunlari Seria Ovasi'nda, Sukkot ile Seredata arasindaki killi topraklarda döktürmüstü.
\par 18 Süleyman'in yaptirdigi esyalar o kadar çoktu ki, kullanilan tuncun hesabi tutulmadi.
\par 19 Süleyman'in Tanri'nin Tapinagi için yaptirdigi altin esyalar sunlardi: Altin sunak ve ekmeklerin Tanri'nin huzuruna kondugu masalar,
\par 20 Iç odanin önüne yerlestirilen ve kurala uygun olarak yakilan saf altindan kandilliklerle kandilleri,
\par 21 Çiçek süslemeleri, kandiller, masalar. -Bunlar saf altindandi.-
\par 22 Saf altin fitil masalari, çanaklar, tabaklar, buhurdanlar ve tapinagin altin kapilari. En Kutsal Yer'in* ve ana bölümün kapilari da altindandi.

\chapter{5}

\par 1 RAB'bin Tapinagi'nin yapimi tamamlaninca Süleyman, babasi Davut'un adadigi altin, gümüs ve öbür esyalari getirip Tanri'nin Tapinagi'nin hazine odalarina yerlestirdi.
\par 2 Süleyman RAB'bin Antlasma Sandigi'ni* Davut Kenti olan Siyon'dan getirmek üzere Israil halkinin ileri gelenleriyle bütün oymak ve boy baslarini Yerusalim'e çagirdi.
\par 3 Hepsi yedinci aydaki* bayramda kralin önünde toplandi.
\par 4 Israil'in bütün ileri gelenleri toplaninca, Levililer Antlasma Sandigi'ni yerden kaldirdilar.
\par 5 Sandigi, Bulusma Çadiri'ni ve çadirdaki bütün kutsal esyalari Levili kâhinler tapinaga tasidilar.
\par 6 Kral Süleyman ve bütün Israil toplulugu Antlasma Sandigi'nin önünde sayisiz davar ve sigir kurban etti.
\par 7 Kâhinler RAB'bin Antlasma Sandigi'ni tapinagin iç odasina, En Kutsal Yer'e* tasiyip Keruvlar'in* kanatlarinin altina yerlestirdiler.
\par 8 Keruvlar'in kanatlari sandigin kondugu yerin üstüne kadar uzaniyor ve sandigi da, siriklarini da örtüyordu.
\par 9 Siriklar öyle uzundu ki, uçlari iç odanin önünden görünüyordu. Ancak disaridan görünmüyordu. Bunlar hâlâ oradadir.
\par 10 Sandigin içinde Musa'nin Horev Dagi'nda koydugu iki levhadan baska bir sey yoktu. Bunlar Misir'dan çikislarinda RAB'bin Israilliler'le yaptigi antlasmanin levhalariydi.
\par 11 Kâhinler Kutsal Yer'den* çiktilar. Orada bulunan kâhinlerin hepsi, bölüklerinin sirasini beklemeden, kendilerini kutsamislardi.
\par 12 Bütün Levili ezgiciler -Asaf, Heman, Yedutun, ogullariyla kardesleri- zillerle, çenk ve lirlerle, ince keten kusanmis olarak sunagin dogusunda yerlerini almislardi. Borazan çalan yüz yirmi kâhin onlara eslik ediyordu.
\par 13 Borazan çalanlarla ezgiciler tek ses halinde RAB'be sükredip övgüler sunmaya basladilar. Borazan, zil ve çalgilarin esliginde seslerini yükselterek RAB'bi söyle övdüler: "RAB iyidir; Sevgisi sonsuza dek kalicidir." O anda RAB'bin Tapinagi'ni bir bulut doldurdu.
\par 14 Bu bulut yüzünden kâhinler görevlerini sürdüremediler. Çünkü RAB Tanri'nin görkemi tapinagi doldurmustu.

\chapter{6}

\par 1 O zaman Süleyman söyle dedi: "Ya RAB, karanlik bulutlarda otururum demistin.
\par 2 Senin için görkemli bir tapinak, sonsuza dek yasayacagin bir konut yaptim."
\par 3 Kral ayakta duran bütün Israil topluluguna dönerek onlari kutsadiktan sonra
\par 4 söyle dedi: "Babam Davut'a verdigi sözü tutan Israil'in Tanrisi RAB'be övgüler olsun! RAB demisti ki,
\par 5 'Halkimi Misir'dan çikardigim günden bu yana, içinde bulunacagim bir tapinak yaptirmak için Israil oymaklarina ait kentlerden hiçbirini seçmedim. Içlerinden halkim Israil'i yönetecek birini de seçmedim.
\par 6 Ancak adimin içinde bulunacagi yer olarak Yerusalim'i, halkim Israil'i yönetmesi için Davut'u seçtim.
\par 7 "Babam Davut Israil'in Tanrisi RAB'bin adina bir tapinak yapmayi yürekten istiyordu.
\par 8 Ama RAB, babam Davut'a, 'Adima bir tapinak yapmayi yürekten istemen iyi bir sey dedi,
\par 9 'Ne var ki, adima yapilacak bu tapinagi sen degil, öz oglun yapacak.
\par 10 "RAB verdigi sözü yerine getirdi. RAB'bin sözü uyarinca, babam Davut'tan sonra Israil tahtina ben geçtim ve Israil'in Tanrisi RAB'bin adina tapinagi ben yaptirdim.
\par 11 Ayrica RAB'bin Israil halkiyla yaptigi antlasmanin içinde korundugu sandigi oraya yerlestirdim."
\par 12 Süleyman RAB'bin sunaginin önünde, Israil toplulugunun karsisinda durup ellerini göklere açti.
\par 13 Bes arsin uzunlugunda, bes arsin eninde, üç arsin yüksekliginde tunç* bir kürsü yaptirip avlunun ortasina kurdurmustu. Bu kürsünün üstünde durdu, Israil toplulugunun önünde diz çöküp ellerini göklere açti.
\par 14 "Ya RAB, Israil'in Tanrisi, yerde ve gökte sana benzer baska tanri yoktur" dedi, "Bütün yürekleriyle yolunu izleyen kullarinla yaptigin antlasmaya bagli kalirsin.
\par 15 Agzinla kulun babam Davut'a verdigin sözü bugün ellerinle yerine getirdin.
\par 16 "Simdi, ya RAB, Israil'in Tanrisi, kulun babam Davut'a verdigin öbür sözü de tutmani istiyorum. Ona, 'Senin soyundan Israil tahtina oturacaklarin ardi arkasi kesilmeyecektir; yeter ki, çocuklarin yasam uyarinca önümde senin gibi dikkatle yürüsünler demistin.
\par 17 Ya RAB, Israil'in Tanrisi, simdi kulun Davut'a verdigin sözü yerine getirmeni istiyorum.
\par 18 "Tanri gerçekten yeryüzünde, insanlar arasinda yasar mi? Sen göklere, göklerin göklerine bile sigmazsin. Benim yaptigim bu tapinak ne ki!
\par 19 Ya RAB Tanrim, kulunun ettigi duayi, yalvarisi isit; duasina ve yakarisina kulak ver.
\par 20 Gözlerin gece gündüz, 'Adimi oraya yerlestirecegim! dedigin bu tapinagin üzerinde olsun. Kulunun buraya yönelerek ettigi duayi isit.
\par 21 Buraya yönelerek dua eden kulunun ve halkin Israil'in yakarisini isit. Göklerden, oturdugun yerden kulak ver; duyunca bagisla.
\par 22 "Biri komsusuna karsi günah isleyip ant içmek zorunda kaldiginda, gelip bu tapinakta, senin sunaginin önünde ant içerse,
\par 23 göklerden kulak ver ve geregini yap. Suçluya karsiligini vererek, suçsuzu hakli çikararak kullarini yargila.
\par 24 "Sana karsi günah isledigi için düsmanlarina yenik düsen halkin Israil yine sana döner, adini anar, bu tapinakta dua edip yakararak önüne çikarsa,
\par 25 göklerden kulak ver, halkin Israil'in günahini bagisla. Onlari kendilerine ve atalarina verdigin ülkeye yine kavustur.
\par 26 "Halkin sana karsi günah isledigi için gökler kapanip yagmur yagmazsa, sikintiya düsen halkin buraya yönelip dua eder, adini anar ve günahlarindan dönerse,
\par 27 göklerden kulak ver; kullarinin, halkin Israil'in günahlarini bagisla. Onlara dogru yolda yürümeyi ögret, halkina mülk olarak verdigin ülkene yagmurlarini gönder.
\par 28 "Ülkeyi kitlik, salgin hastalik, samyeli, küf, tirtil ya da çekirgeler kavurdugunda, düsmanlar kentlerden birinde halkini kusattiginda, herhangi bir felaket ya da hastalik ortaligi sardiginda,
\par 29 halkindan bir kisi ya da bütün halkin Israil basina gelen felaketi, aciyi kavrar, dua edip yakararak ellerini bu tapinaga dogru açarsa,
\par 30 göklerden, oturdugun yerden kulak ver ve bagisla. Insanlarin yüreklerini yalnizca sen bilirsin. Onlara yaptiklarina göre davran ki,
\par 31 atalarimiza verdigin bu ülkede yasadiklari sürece senden korksunlar ve senin yolunda yürüsünler.
\par 32 "Halkin Israil'den olmayan, ama senin yüce adini, gücünü, kudretini duyup uzak ülkelerden gelen yabancilar bu tapinaga gelip dua ederlerse,
\par 33 göklerden, oturdugun yerden kulak ver, yalvarislarini yanitla. Öyle ki, dünyanin bütün uluslari, halkin Israil gibi, adini bilsin, senden korksun ve yaptirdigim bu tapinagin sana ait oldugunu ögrensin.
\par 34 "Halkin, düsmanlarina karsi gösterdigin yoldan savasa giderken sana, seçtigin bu kente ve adina yaptirdigim bu tapinaga yönelip dua ederse,
\par 35 dualarina, yakarislarina göklerden kulak ver ve onlari kurtar.
\par 36 "Sana karsi günah islediklerinde -günah islemeyen tek kisi yoktur- öfkelenip onlari yakin ya da uzak bir ülkeye tutsak olarak götürecek düsmanlarinin eline teslim edersen,
\par 37 onlar da tutsak olduklari ülkede pismanlik duyup günahlarindan döner, 'Günah isledik, yoldan sapip kötülük yaptik diyerek sana yakarirlarsa,
\par 38 tutsak olduklari ülkede candan ve yürekten sana dönerlerse, atalarina verdigin ülkelerine, seçtigin kente ve adina yaptirdigim tapinaga yönelip dua ederlerse,
\par 39 göklerden, oturdugun yerden dualarina, yakarislarina kulak ver, onlari kurtar. Sana karsi günah islemis olan halkini bagisla.
\par 40 "Simdi, ey Tanrim, bizi gör ve burada edilen duaya kulak ver.
\par 41 "Çik, ya RAB Tanri, yasayacagin yere, Gücünü simgeleyen Sandik'la birlikte. Ya RAB Tanri, kâhinlerin kurtulusu kusansin, Sadik kullarin iyiliklerinle sevinsinler.
\par 42 Ya RAB Tanri, meshettigin* krala yüz çevirme. Kulun Davut'a yaptigin iyilikleri animsa."

\chapter{7}

\par 1 Süleyman duasini bitirince, gökten ates yagdi; yakmalik sunularla* kurbanlari yiyip bitirdi. RAB'bin görkemi tapinagi doldurdu.
\par 2 RAB'bin Tapinagi O'nun görkemiyle dolunca kâhinler tapinaga giremediler.
\par 3 Gökten yagan atesi ve tapinagin üzerindeki RAB'bin görkemini gören Israilliler avluda yüzüstü yere kapandilar; RAB'be tapinarak O'nu övdüler: "RAB iyidir; Sevgisi sonsuza dek kalicidir."
\par 4 Kral ve bütün halk RAB'bin önünde kurban kestiler.
\par 5 Kral Süleyman yirmi iki bin sigir, yüz yirmi bin davar kurban etti. Böylece kral ve halk Tanri'nin Tapinagi'ni adamis oldular.
\par 6 Kâhinler yerlerini almislardi. Kral Davut'un RAB'bi övmek için yaptirdigi ve "RAB'bin sevgisi sonsuza dek kalicidir" diyerek överken kullandigi çalgilari alan Levililer de yerlerini almisti. Levililer'in karsisinda duran kâhinler borazanlarini çaliyorlardi. Bu sirada bütün Israilliler ayakta duruyordu.
\par 7 Süleyman RAB'bin Tapinagi'nin önündeki avlunun orta kismini kutsadi. Yakmalik sunularla esenlik sunularinin* yagli parçalarini orada sundu. Çünkü yaptirdigi tunç* sunak yakmalik sunulari, tahil sunularini* ve yagli parçalari almadi.
\par 8 Süleyman, Levo-Hamat'tan Misir Vadisi'ne kadar her yerden gelen Israilliler'in olusturdugu çok büyük bir toplulukla birlikte bayrami yedi gün kutladi.
\par 9 Sekizinci gün kutsal bir toplanti yaptilar. Sunagi adamaya yedi gün, bayrami kutlamaya da yedi gün ayirdilar.
\par 10 Kral yedinci ayin* yirmi üçüncü günü halki evlerine gönderdi. RAB'bin, Davut, Süleyman ve halki Israil için yapmis oldugu iyilikten dolayi hepsi mutluydu, sevinçle cosuyordu.
\par 11 Süleyman RAB'bin Tapinagi'ni, sarayi ve RAB'bin Tapinagi'yla kendi sarayinda yapmayi istedigi bütün isleri basariyla bitirince,
\par 12 RAB geceleyin ona görünerek söyle dedi: "Duani duydum. Burayi kendime kurban sunulan tapinak olarak seçtim.
\par 13 "Yagmur yagmasin diye gögü kapadigimda, topragin ürününü yiyip bitirmesi için çekirgelere buyruk verdigimde ya da halkimin arasina salgin hastalik gönderdigimde,
\par 14 adimla çagrilan halkim alçakgönüllülügü takinir, bana yönelip dua eder, kötü yollarindan dönerse, gökten onlari duyacagim, günahlarini bagislayip ülkelerini sagliga kavusturacagim.
\par 15 Gözlerim burada edilen duaya açik, kulaklarim isitici olacak.
\par 16 Adim sürekli orada bulunsun diye bu tapinagi seçip kutsal kildim. Gözlerim onun üstünde, yüregim her zaman orada olacaktir.
\par 17 Sana gelince, baban Davut'un yaptigi gibi yollarimi izler, buyurdugum her seyi yapar, kurallarima ve ilkelerime uyarsan,
\par 18 baban Davut'la, 'Israil tahtindan senin soyunun ardi arkasi kesilmeyecektir diye yaptigim antlasmaya bagli kalip kralligini pekistirecegim.
\par 19 "Ama siz yollarimdan sapar, kurallarimi, buyruklarimi birakir, gidip baska ilahlara kulluk eder, taparsaniz,
\par 20 size verdigim ülkeden sizi söküp atacagim, adima kutsal kildigim bu tapinagi terk edecegim; burayi bütün uluslarin asagilayip alay ettigi bir yer durumuna getirecegim.
\par 21 Bu gösterisli tapinagin önünden geçenler hayretle, 'RAB bu ülkeyi ve tapinagi neden bu duruma getirdi? diye soracaklar.
\par 22 Ve diyecekler ki, 'Israil halki, atalarini Misir'dan çikaran Tanrilari RAB'bi terk etti; baska ilahlarin ardindan gitti, onlara tapip kulluk etti. RAB bu yüzden bu kötülükleri baslarina getirdi."

\chapter{8}

\par 1 Süleyman RAB'bin Tapinagi'yla kendi sarayini yirmi yilda bitirdi.
\par 2 Sonra Hiram'in kendisine verdigi kentleri onarip Israilliler'in buralara yerlesmesini sagladi.
\par 3 Ardindan Hamat-Sova üzerine yürüyerek orayi ele geçirdi.
\par 4 Çölde Tadmor Kenti'ni onardi. Hama yöresinde yaptirdigi bütün ambarli kentlerin yapimini bitirdi.
\par 5 Yukari ve Asagi Beythoron'u yeniden kurup çevrelerini surlarla, sürgülü kapilarla saglamlastirdi.
\par 6 Baalat'i ve yönetimindeki bütün ambarli kentleri, savas arabalariyla atlarin bulundugu kentleri de onarip güçlendirdi. Böylece Yerusalim'de, Lübnan'da, yönetimi altindaki bütün topraklarda her istedigini yaptirmis oldu.
\par 7 Israil halkindan olmayan Hititler*, Amorlular, Perizliler, Hivliler ve Yevuslular'dan artakalanlara gelince,
\par 8 Süleyman Israil halkinin yok etmedigi bu insanlarin soyundan gelip ülkede kalanlari angaryaya kostu. Bu durum bugün de sürüyor.
\par 9 Ancak Israil halkindan hiç kimseye kölelik yaptirmadi. Onlar savasçi, birlik komutani, savas arabalariyla atlilarin komutani olarak görev yaptilar.
\par 10 Kral Süleyman adina halki denetleyen iki yüz elli görevli de Israil halkindandi.
\par 11 Süleyman, "Karim Israil Krali Davut'un sarayinda kalmamali. Çünkü RAB'bin Antlasma Sandigi'nin* girdigi yerler kutsaldir" diyerek firavunun kizini Davut Kenti'nden kendisi için yaptirdigi saraya getirtti.
\par 12 Tapinagin eyvaninin önünde, yaptirdigi sunakta RAB'be yakmalik sunular* sundu.
\par 13 Sabat Günü*, Yeni Ay, yilin üç bayrami -Mayasiz Ekmek, Haftalar ve Çardak bayramlari*- için Musa'nin buyurdugu sunulari günü gününe sundu.
\par 14 Babasi Davut'un koydugu kural uyarinca, kâhin bölüklerine ayri ayri görevler verdi. Levililer'i Tanri'yi övme ve her günün gerektirdigi islerde kâhinlere yardim etme görevine atadi. Kapi nöbetçilerini de bölüklerine göre degisik kapilarda görevlendirdi. Çünkü Tanri adami Davut böyle buyurmustu.
\par 15 Bunlar, hazine odalarina iliskin konular dahil, hiçbir konuda kralin kâhinlerle Levililer'e verdigi buyruklardan ayrilmadilar.
\par 16 RAB'bin Tapinagi'nin temelinin atildigi günden bitimine dek Süleyman'in yapmak istedigi her is yerine getirildi. Böylece RAB'bin Tapinagi'nin yapimi tamamlanmis oldu.
\par 17 Bundan sonra Süleyman Esyon-Gever'e, Edom kiyisindaki Eylat'a gitti.
\par 18 Hiram ona denizi bilen gemiciler ve kendi görevlileri araciligiyla gemiler gönderdi. Kral Süleyman'in adamlariyla birlikte Ofir'e giden bu gemiciler, dört yüz elli talant altin getirdiler.

\chapter{9}

\par 1 Saba Kraliçesi, Süleyman'in ününü duyunca, onu çetin sorularla sinamak için Yerusalim'e geldi. Çesitli baharat, çok miktarda altin ve degerli taslarla yüklü büyük bir kervan esliginde gelen kraliçe, aklindan geçen her seyi Süleyman'la konustu.
\par 2 Süleyman onun bütün sorularina karsilik verdi. Kralin ona yanit bulmakta güçlük çektigi hiçbir konu olmadi.
\par 3 Süleyman'in bilgeligini, yaptirdigi sarayi, sofrasinin zenginligini, görevlilerinin oturup kalkisini, hizmetkârlarinin ve sakilerinin özel giysileriyle yaptigi hizmeti, RAB'bin Tapinagi'nda sundugu yakmalik sunulari* gören Saba Kraliçesi hayranlik içinde kaldi.
\par 5 Krala, "Ülkemdeyken, yaptiklarinla ve bilgeliginle ilgili duyduklarim dogruymus" dedi,
\par 6 "Ama gelip kendi gözlerimle görünceye dek anlatilanlara inanmamistim. Büyük bilgeliginin yarisi bile bana anlatilmadi. Duyduklarimdan daha üstünsün.
\par 7 Ne mutlu adamlarina! Ne mutlu sana hizmet eden görevlilere! Çünkü sürekli bilgeligine tanik oluyorlar.
\par 8 Senden hosnut kalan, adina egemenlik sürmen için seni tahta oturtan Tanrin RAB'be övgüler olsun! Tanrin Israil'i sevdigi, sonsuza dek korumak istedigi için, adaleti ve dogrulugu saglaman için seni Israil'e kral yapti."
\par 9 Saba Kraliçesi krala 120 talant altin, çok büyük miktarda baharat ve degerli taslar armagan etti. Krala armagan ettigi baharatin benzeri yoktu.
\par 10 Bu arada Hiram'in adamlariyla Süleyman'in adamlari Ofir'den altin, algum*fe* kerestesiyle degerli taslar getirdiler.
\par 11 Kral, RAB'bin Tapinagi'yla sarayin basamaklarini, çalgicilarin lirleriyle çenklerini bu algum kerestesinden yaptirdi. Yahuda bölgesinde daha önce böylesi görülmemisti.
\par 12 Kral Süleyman Saba Kraliçesi'nin her istegini, her dilegini yerine getirdi. Kraliçenin kendisine getirdiklerinden daha fazlasini ona verdi. Bundan sonra kraliçe adamlariyla birlikte oradan ayrilip kendi ülkesine döndü.
\par 13 Süleyman'a bir yilda gelen altinin miktari 666 talanti* buluyordu.
\par 14 Tüccarlarin ve alim satimla ugrasanlarin getirdigi altin bunun disindaydi. Arabistan'in bütün krallariyla Israil valileri de Süleyman'a altin, gümüs getiriyorlardi.
\par 15 Kral Süleyman dövme altindan her biri alti yüz sekel agirliginda iki yüz büyük kalkan yaptirdi.
\par 16 Ayrica her biri üç yüz sekel agirliginda dövme altindan üç yüz küçük kalkan yaptirdi. Kral bu kalkanlari Lübnan Ormani adindaki saraya koydu.
\par 17 Kral fildisinden büyük bir taht yaptirip saf altinla kaplatti.
\par 18 Tahtin alti basamagi, bir de altin ayak taburesi vardi. Bunlar tahta bagliydi. Oturulan yerin iki yaninda kollar, her kolun yaninda birer aslan heykeli bulunuyordu.
\par 19 Alti basamagin iki yaninda on iki aslan heykeli vardi. Hiçbir krallikta böylesi yapilmamisti.
\par 20 Kral Süleyman'in kadehleriyle Lübnan Ormani adindaki sarayin bütün esyalari saf altindan yapilmis, hiç gümüs kullanilmamisti. Çünkü Süleyman'in döneminde gümüsün degeri yoktu.
\par 21 Kralin gemileri Hiram'in adamlarinin yönetiminde Tarsis'e giderdi. Bu gemiler üç yilda bir altin, gümüs, fildisi ve türlü maymunlarla yüklü olarak dönerlerdi.
\par 22 Kral Süleyman dünyanin bütün krallarindan daha zengin, daha bilgeydi.
\par 23 Tanri'nin Süleyman'a verdigi bilgeligi dinlemek için dünyanin bütün krallari onu görmek isterlerdi.
\par 24 Onu görmeye gelenler her yil armagan olarak altin ve gümüs esya, giysi, silah, baharat, at, katir getirirlerdi.
\par 25 Süleyman'in atlarla savas arabalari için dört bin ahiri, on iki bin atlisi vardi. Bunlarin bir kismini savas arabalari için ayrilan kentlere, bir kismini da kendi yanina, Yerusalim'e yerlestirdi.
\par 26 Firat Irmagi'ndan Filist bölgesine, oradan da Misir sinirina dek uzanan bölgedeki bütün krallara egemendi.
\par 27 Onun kralligi döneminde Yerusalim'de gümüs tas degerine düstü. Sedir agaçlari Sefela'daki yabanil incir agaçlari kadar bollasti.
\par 28 Süleyman'in atlari Misir'dan ve bütün öbür ülkelerden getirilirdi.
\par 29 Süleyman'in yaptigi öbür isler, basindan sonuna dek, Peygamber Natan'in tarihinde, Silolu Ahiya'nin peygamberlik yazilarinda ve Bilici* Iddo'nun Nevat oglu Yarovam'a iliskin görümlerinde yazilidir.
\par 30 Süleyman kirk yil süreyle bütün Israil'i Yerusalim'den yönetti.
\par 31 Süleyman ölüp atalarina kavusunca babasi Davut'un Kenti'nde gömüldü. Yerine oglu Rehavam kral oldu.

\chapter{10}

\par 1 Rehavam Sekem'e gitti. Çünkü bütün Israilliler kendisini kral ilan etmek için orada toplanmislardi.
\par 2 Kral Süleyman'dan kaçip Misir'a yerlesen Nevat oglu Yarovam bunu duyunca Misir'dan döndü.
\par 3 Israilliler Yarovam'i çagirttilar. Birlikte gidip Rehavam'a söyle dediler:
\par 4 "Baban üzerimize agir bir boyunduruk koydu. Ama babanin üzerimize yükledigi agir yükü ve boyundurugu hafifletirsen sana kul köle oluruz."
\par 5 Rehavam, "Üç gün sonra yine gelin" yanitini verince halk yanindan ayrildi.
\par 6 Kral Rehavam, babasi Süleyman'a sagliginda danismanlik yapan ileri gelenlere, "Bu halka nasil yanit vermemi ögütlersiniz?" diye sordu.
\par 7 Ileri gelenler, "Halka iyi davranir, onlari hosnut eder, olumlu yanit verirsen, sana her zaman kul köle olurlar" diye karsilik verdiler.
\par 8 Ne var ki, Rehavam ileri gelenlerin ögüdünü reddederek birlikte büyüdügü genç görevlilerine danisti:
\par 9 "Siz ne yapmami ögütlersiniz? 'Babanin üzerimize koydugu boyundurugu hafiflet diyen bu halka nasil bir yanit verelim?"
\par 10 Birlikte büyüdügü gençler ona su karsiligi verdiler: "Sana, 'Babanin üzerimize koydugu boyundurugu hafiflet diyen halka de ki, 'Benim küçük parmagim, babamin belinden daha kalindir.
\par 11 Babam size agir bir boyunduruk yüklediyse, ben boyundurugunuzu daha da agirlastiracagim. Babam sizi kirbaçla yola getirdiyse, ben sizi akreplerle yola getirecegim."
\par 12 Yarovam'la bütün halk, kralin, "Üç gün sonra yine gelin" sözü üzerine, üçüncü gün Rehavam'in yanina geldiler.
\par 13 Ileri gelenlerin ögüdünü reddeden Kral Rehavam, gençlerin ögüdüne uyarak halka sert bir yanit verdi: "Babamin size yükledigi boyundurugu ben daha da agirlastiracagim. Babam sizi kirbaçla yola getirdiyse, ben sizi akreplerle yola getirecegim."
\par 15 Kral halki dinlemedi. Bu Tanri'dandi. Çünkü Silolu Ahiya araciligiyla Nevat oglu Yarovam'a verdigi sözü yerine getirmek için RAB bu olayi düzenlemisti.
\par 16 Kralin kendilerini dinlemedigini görünce, bütün Israilliler, "Isay oglu Davut'la ne ilgimiz, Ne de payimiz var!" diye bagirdilar, "Ey Israil halki, haydi evimize dönelim! Davut'un soyu basinin çaresine baksin." Böylece herkes evine döndü.
\par 17 Rehavam da yalnizca Yahuda kentlerinde yasayan Israilliler'e krallik yapmaya basladi.
\par 18 Israilliler Kral Rehavam'in gönderdigi angaryacibasi Hadoram'i tasa tutup öldürdüler. Bunun üzerine Kral Rehavam savas arabasina atlayip Yerusalim'e kaçti.
\par 19 Israil halki, Davut soyundan gelenlere hep baskaldirdi.

\chapter{11}

\par 1 Rehavam Yerusalim'e varinca, Israilliler'le savasip onlari yeniden egemenligi altina almak amaciyla Yahuda ve Benyamin oymaklarindan yüz seksen bin seçkin savasçi topladi.
\par 2 Bu arada RAB, Tanri adami Semaya'ya söyle seslendi:
\par 3 "Süleyman oglu Yahuda Krali Rehavam'a, Yahuda ve Benyamin bölgesinde yasayan bütün Israilliler'e sunu söyle:
\par 4 'RAB diyor ki, Israilli kardeslerinize saldirmayin, onlarla savasmayin. Herkes evine dönsün! Çünkü bu olayi ben düzenledim." RAB'bin bu sözlerini duyan halk Yarovam'a karsi savasmaktan vazgeçti.
\par 5 Rehavam Yerusalim'de yasadi ve savunma amaciyla Yahuda'daki su kentleri onardi:
\par 6 Beytlehem, Etam, Tekoa,
\par 7 Beytsur, Soko, Adullam,
\par 8 Gat, Maresa, Zif,
\par 9 Adorayim, Lakis, Azeka,
\par 10 Sora, Ayalon, Hevron. Bunlar Yahuda ve Benyamin bölgesinde surlu kentlerdi.
\par 11 Rehavam bu kentlerin surlarini güçlendirerek buralara komutanlar atadi; yiyecek, zeytinyagi, sarap depoladi.
\par 12 Her kent için kalkanlar, mizraklar sagladi. Kentleri iyice güçlendirdi. Böylece Yahuda ve Benyamin bölgeleri onun denetimi altinda kaldi.
\par 13 Israil'in her bölgesinden kâhinlerle Levililer Rehavam'dan yana geçtiler.
\par 14 Levililer otlaklarini, mallarini birakip Yahuda ve Yerusalim'e gelmislerdi. Çünkü Yarovam'la ogullari onlari RAB'bin kâhinliginden uzaklastirmisti.
\par 15 Yarovam tapinma yerleri ve yaptirdigi teke ve buzagi biçimindeki putlar için kâhinler atamisti.
\par 16 Israil'in her oymagindan RAB'be, Israil'in Tanrisi'na yönelmeye yürekten kararli olanlar ise, atalarinin Tanrisi RAB'be kurban sunmak için kâhinlerle Levililer'in ardindan Yerusalim'e geldiler.
\par 17 Üç yil Davut'la Süleyman'in izinde yürüyen bu kisiler, Süleyman oglu Rehavam'i destekleyerek Yahuda Kralligi'ni güçlendirdiler.
\par 18 Rehavam Davut oglu Yerimot'un kizi Mahalat'la evlendi. Mahalat'in annesi Avihayil, Isay oglu Eliav'in kiziydi.
\par 19 Mahalat Rehavam'a su ogullari dogurdu: Yeus, Semarya, Zaham.
\par 20 Rehavam Mahalat'tan sonra Avsalom'un kizi Maaka ile evlendi. Maaka ona Aviya'yi, Attay'i, Ziza'yi, Selomit'i dogurdu.
\par 21 Rehavam Avsalom'un kizi Maaka'yi öbür esleriyle cariyelerinin hepsinden daha çok severdi. Rehavam'in on sekiz karisi, altmis cariyesi vardi. Bunlardan yirmi sekiz erkek, altmis kiz çocugu oldu.
\par 22 Rehavam Maaka'dan dogan Aviya'yi kral yapmak amaciyla onu kardesleri arasinda önder ve tahta geçecek aday atadi.
\par 23 Bilgece davranarak ogullarinin bazilarini Yahuda ve Benyamin topraklarina, surlu kentlere dagitti. Onlara bol yiyecek sagladi ve birçok kadinla evlendirdi.

\chapter{12}

\par 1 Rehavam kralligini pekistirip güçlenince, Israil halkiyla birlikte RAB'bin Yasasi'na sirt çevirdi.
\par 2 Rehavam'in kralliginin besinci yilinda Misir Krali Sisak Yerusalim'e saldirdi. Çünkü Rehavam'la halk RAB'be ihanet etmisti.
\par 3 Sisak'in bin iki yüz savas arabasi, altmis bin atlisi ve Misir'dan onunla birlikte gelen Luvlu, Suklu, Kûslu* sayisiz askeri vardi.
\par 4 Sisak Yahuda'nin surlu kentlerini ele geçirerek Yerusalim'e kadar geldi.
\par 5 Bu sirada Peygamber Semaya, Rehavam'a ve Sisak yüzünden Yerusalim'de toplanan Yahuda önderlerine gelip söyle dedi: "RAB, 'Siz beni biraktiniz. Ben de sizi birakip Sisak'in eline teslim ettim diyor."
\par 6 Israil önderleriyle kral alçakgönüllü bir tutum takinarak, "RAB adildir" dediler.
\par 7 RAB onlarin alçakgönüllü bir tutum takindiklarini görünce, Semaya'ya söyle dedi: "Madem alçakgönüllü bir tutum takindilar, onlari yok etmeyecegim; biraz da olsa onlari huzura kavusturacagim. Öfkemi Sisak araciligiyla Yerusalim üzerine bosaltmayacagim.
\par 8 Ama onlari Sisak'a köle edecegim. Öyle ki, bana hizmet etmekle öbür uluslarin krallarina hizmet etmek arasindaki farki anlayabilsinler."
\par 9 Misir Krali Sisak Yerusalim'e saldirdiginda, Süleyman'in yaptirmis oldugu altin kalkanlar dahil RAB'bin Tapinagi'nin ve sarayin bütün hazinelerini bosaltip götürdü.
\par 10 Kral Rehavam bunlarin yerine tunç* kalkanlar yaptirarak sarayin kapi muhafizlarinin komutanlarina emanet etti.
\par 11 Kral RAB'bin Tapinagi'na her gittiginde, muhafizlar bu kalkanlari tasiyarak ona eslik eder, sonra muhafiz odasina götürürlerdi.
\par 12 Rehavam'in alçakgönüllü bir tutum takinmasi üzerine RAB'bin öfkesi dindi, onu büsbütün yok etmekten vazgeçti. Yahuda'da bazi iyi davranislar da vardi.
\par 13 Kral Rehavam Yerusalim'de kralligini pekistirerek sürdürdü. Kral oldugunda kirk bir yasindaydi. RAB'bin adini yerlestirmek için bütün Israil oymaklarinin yasadigi kentler arasindan seçtigi Yerusalim Kenti'nde on yedi yil krallik yapti. Annesi Ammonlu Naama'ydi.
\par 14 Rehavam RAB'be yönelmeye yürekten kararli olmadigi için kötülük yapti.
\par 15 Rehavam'in yaptigi isler, basindan sonuna dek, Peygamber Semaya ve Bilici* Iddo'nun soyla ilgili tarihinde yazilidir. Rehavam'la Yarovam arasinda sürekli savas vardi.
\par 16 Rehavam ölüp atalarina kavusunca, Davut Kenti'nde gömüldü. Rehavam'in yerine oglu Aviya kral oldu.

\chapter{13}

\par 1 Israil Krali Yarovam'in kralliginin on sekizinci yilinda Aviya Yahuda Krali oldu.
\par 2 Yerusalim'de üç yil krallik yapti. Annesi Givali Uriel'in kizi Mikaya'ydi. Aviya'yla Yarovam arasinda savas vardi.
\par 3 Aviya seçme yigit askerlerden olusan dört yüz bin kisilik bir orduyla savasa çikti. Yarovam da sekiz yüz bin seçme yigit savasçidan olusan bir orduyla ona karsi savas düzenine girdi.
\par 4 Aviya Efrayim daglik bölgesindeki Semarayim Dagi'na çikip söyle seslendi: "Ey Yarovam ve bütün Israilliler, beni dinleyin!
\par 5 Israil'in Tanrisi RAB'bin bozulmaz bir antlasmayla Israil Kralligi'ni sonsuza dek Davut'a ve soyuna verdigini bilmiyor musunuz?
\par 6 Nevat oglu Yarovam efendisi Davut oglu Süleyman'a baskaldirdi.
\par 7 Bir takim ise yaramaz kötü kisiler çevresinde toplanip Süleyman oglu Rehavam'a baski yaptilar. O sirada Rehavam onlara karsi koyamayacak kadar genç ve deneyimsizdi.
\par 8 "Simdi de siz Davut soyunun elindeki RAB'bin Kralligi'na karsi gelmeyi tasarliyorsunuz. Büyük bir ordusunuz. Üstelik Yarovam'in ilahlariniz olsun diye yaptirdigi altin buzagilar da yaninizda.
\par 9 RAB'bin kâhinlerini, Harunogullari'yla Levililer'i kovmadiniz mi? Onlarin yerine öbür halklar gibi kendinize kâhinler atamadiniz mi? Atanmak için bir boga ve yedi koçla gelen herkes, tanri olmayanlara kâhin olabiliyor.
\par 10 "Ama bizim Tanrimiz RAB'dir, O'nu birakmadik. RAB'be hizmet eden kâhinler Harun soyundandir. Levililer de onlara yardimcidir.
\par 11 Onlar her sabah, her aksam RAB'be yakmalik sunular* sunar, hos kokulu buhur yakar, dinsel açidan temiz masanin üzerine adak ekmeklerini* dizerler. Her aksam altin kandilligin kandillerini yakarlar. Biz Tanrimiz RAB'bin buyruklarini yerine getiriyoruz. Oysa siz O'na sirt çevirdiniz.
\par 12 Tanri bizimledir, O önderimizdir. O'nun kâhinleri borazanlarla size karsi savas çagrisi yapacaklar. Ey Israil halki, atalarinizin Tanrisi RAB'be karsi savasmayin; çünkü basaramazsiniz."
\par 13 Yarovam askerlerinin bir bölümüyle Yahudalilar'i önden karsilarken, öbür bölümünü arkalarinda pusu kurmaya göndermisti.
\par 14 Yahudalilar önden, arkadan kusatildiklarini görünce, RAB'be yakardilar. Kâhinler borazanlarini çaldi.
\par 15 Yahudalilar savas çigliklari attigi anda, Tanri Yarovam'la Israilliler'i Aviya'yla Yahudalilar'in önünde yenilgiye ugratti.
\par 16 Israilliler Yahudalilar'in önünden kaçti. Tanri onlari Yahudalilar'in eline teslim etti.
\par 17 Aviya'yla ordusu Israilliler'i bozguna ugratti. Israilliler'den bes yüz bin seçme asker öldürüldü.
\par 18 Böylece Israilliler yenilgiye ugradi, Yahudalilar'sa zafer kazandi. Çünkü Yahudalilar atalarinin Tanrisi RAB'be güvenmislerdi.
\par 19 Aviya Yarovam'i kovaladi. Yarovam'a ait Beytel, Yesana, Efron kentleriyle çevrelerindeki köyleri ele geçirdi.
\par 20 Aviya'nin kralligi döneminde Yarovam bir daha eski gücünü toparlayamadi. Sonunda RAB onu cezalandirdi, Yarovam öldü.
\par 21 Aviya ise gitgide kralligini güçlendirdi. On dört kadinla evlenip yirmi iki erkek, on alti kiz babasi oldu.
\par 22 Aviya'nin yaptigi öbür isler, uygulamalari ve söyledikleri, Peygamber Iddo'nun yorumunda yazilidir.

\chapter{14}

\par 1 Aviya ölüp atalarina kavusunca, Davut Kenti'nde gömüldü, yerine oglu Asa kral oldu. Ülke Asa'nin yönetimi altinda on yil baris içinde yasadi.
\par 2 Asa Tanrisi RAB'bin gözünde iyi ve dogru olani yapti.
\par 3 Yabanci ilahlarin sunaklarini, puta tapilan yerleri kaldirdi. Dikili taslari parçaladi, Asera* putlarini devirdi.
\par 4 Yahudalilar'dan atalarinin Tanrisi RAB'be yönelmelerini, O'nun yasasina ve buyruklarina uymalarini istedi.
\par 5 Yahuda'nin bütün kentlerinden puta tapilan yerlerle buhur sunaklarini kaldirdi. Ülke onun yönetimi altinda baris içinde yasadi.
\par 6 Ülke baris içinde oldugu için Asa Yahuda'daki bazi kentleri surlarla çevirdi. O yillarda kimse ona karsi savas açmadi. Çünkü RAB ona esenlik vermisti.
\par 7 Asa Yahudalilar'a, "Bu kentleri onaralim" dedi, "Onlari surlarla kusatip kulelerle, kapilarla, sürgülerle güçlendirelim. Ülke hâlâ bizim elimizde, çünkü Tanrimiz RAB'be yöneldik, O da bizi her yandan esenlikle kusatti." Böylece yapim islerini basariyla bitirdiler.
\par 8 Asa'nin Yahudalilar'la Benyaminliler'den olusan bir ordusu vardi. Yahudalilar büyük kalkan ve mizraklarla donanmis üç yüz bin kisiydi. Benyaminliler ise küçük kalkan ve yay tasiyan iki yüz seksen bin kisiydi. Bunlarin hepsi yigit savasçilardi.
\par 9 Kûslu* Zerah binlerce asker ve üç yüz savas arabasiyla Maresa'ya ilerledi.
\par 10 Asa ona karsi durmak için yola çikti. Iki ordu Maresa yakinlarinda Sefata Vadisi'nde savas düzeni aldi.
\par 11 Asa, Tanrisi RAB'be, "Ya RAB, güçlünün karsisinda güçsüze yardim edebilecek senden baska kimse yoktur" diye yakardi, "Ey Tanrimiz RAB, bize yardim et, çünkü sana güveniyoruz. Senin adinla bu kalabaliga karsi çiktik. Ya RAB, sen bizim Tanrimiz'sin. Insanlar sana karsi zafer kazanmasin."
\par 12 RAB Kûslular'i Asa'yla Yahudalilar'in önünde bozguna ugratti. Kûslular kaçmaya basladi.
\par 13 Asa ordusuyla onlari Gerar'a kadar kovaladi. Kûslular'dan kurtulan olmadi. RAB'bin ve ordusunun önünde kirildilar. Yahudalilar çok miktarda mal yagmalayip götürdüler.
\par 14 Gerar'in çevresindeki bütün köyleri yerle bir ettiler. Çünkü RAB'bin dehseti onlari sarmisti. Bu köylerde çok mal oldugundan onlari yagmaladilar.
\par 15 Çobanlarin çadirlarina da saldirdilar. Çok sayida davar ve deveyi alip Yerusalim'e döndüler.

\chapter{15}

\par 1 Tanri'nin Ruhu Odet oglu Azarya'nin üzerine indi.
\par 2 Azarya, Kral Asa'ya gidip söyle dedi: "Ey Asa, ey Yahuda ve Benyamin halki, beni dinleyin! RAB'le birlikte oldugunuz sürece, O da sizinle olacaktir. O'nu ararsaniz bulursunuz. Ama O'nu birakirsaniz, O da sizi birakir.
\par 3 Israil halki uzun süre gerçek Tanri'dan, egitici kâhinlerden ve yasadan uzak yasadi.
\par 4 Ama sikintiya düstüklerinde Israil'in Tanrisi RAB'be döndüler, O'nu arayip buldular.
\par 5 O günlerde yolcularin güvenligi yoktu. Çünkü çevre ülkelerde yasayanlarin tümü büyük kargasa içindeydi.
\par 6 Ulus ulusu, kent kenti ezmeye çalisiyordu. Çünkü Tanri onlari çesitli sikintilarla tedirgin ediyordu.
\par 7 Ama siz güçlü olun, cesaretinizi yitirmeyin. Yaptiklarinizin karsiligini alacaksiniz."
\par 8 Asa bu sözleri, Peygamber Odet'in oglu Azarya'nin peygamberligini duyunca, cesaret buldu. Yahuda ve Benyamin topraklarindan, Efrayim'in daglik bölgesinde ele geçirdigi kentlerden igrenç putlari kaldirdi. RAB'bin Tapinagi'nin eyvaninin önündeki RAB'bin sunagini onardi.
\par 9 Efrayim'den, Manasse'den, Simon'dan gelen birçok Israilli, Tanrisi RAB'bin Asa'yla birlikte oldugunu görünce onun tarafina geçti. Asa bu gelenlerle Yahuda ve Benyamin halkini bir araya topladi.
\par 10 Asa'nin kralliginin on besinci yilinin üçüncü ayinda* Yerusalim'de toplandilar.
\par 11 Yagmalamis olduklari hayvanlardan yedi yüz sigirla yedi bin davari o gün RAB'be kurban ettiler.
\par 12 Bütün yürekleriyle, bütün canlariyla atalarinin Tanrisi RAB'be yönelmek için antlasma yaptilar.
\par 13 Büyük küçük, kadin erkek, kim Israil'in Tanrisi RAB'be yönelmezse öldürülecekti.
\par 14 Yüksek sesle bagirarak, borazan ve boru çalarak RAB'bin önünde ant içtiler.
\par 15 Yahudalilar bütün yürekleriyle içtikleri ant için sevindiler. RAB'bi istekle arayip buldular. O da onlari her yandan esenlikle kusatti.
\par 16 Kral Asa annesi Maaka'nin kraliçeligini elinden aldi. Çünkü o Asera* için igrenç bir put yaptirmisti. Asa bu igrenç putu kesip parçaladiktan sonra Kidron Vadisi'nde yakti.
\par 17 Ancak Israil'den puta tapilan yerleri kaldirmadi. Ama yasami boyunca yüregini RAB'be adadi.
\par 18 Babasinin ve kendisinin adadigi altini, gümüsü ve esyalari Tanri'nin Tapinagi'na getirdi.
\par 19 Asa'nin kralliginin otuz besinci yilina kadar savas olmadi.

\chapter{16}

\par 1 Yahuda Krali Asa'nin kralliginin otuz altinci yilinda Israil Krali Baasa Yahuda'ya saldirmaya hazirlaniyordu. Asa'nin topraklarina giris çikisi engellemek amaciyla, Rama Kenti'ni güçlendirmeye basladi.
\par 2 Bunun üzerine Asa, RAB'bin Tapinagi'nin ve sarayin hazinelerindeki altin ve gümüsü çikararak su haberle birlikte Sam'da oturan Aram Krali Ben-Hadat'a gönderdi:
\par 3 "Babamla baban arasinda oldugu gibi seninle benim aramizda da bir antlasma olsun. Sana gönderdigim bu altinlara, gümüslere karsilik, sen de Israil Krali Baasa ile yaptigin antlasmayi boz, topraklarimdan askerlerini çeksin."
\par 4 Kral Asa'nin önerisini kabul eden Ben-Hadat, ordu komutanlarini Israil kentlerinin üzerine gönderdi. Iyon'u, Dan'i, Avel-Mayim'i, Naftali'nin bütün ambarli kentlerini ele geçirdiler.
\par 5 Baasa bunu duyunca Rama'nin yapimini durdurup ise son verdi.
\par 6 Kral Asa bütün Yahudalilar'i çagirtti; Baasa'nin Rama'nin yapiminda kullandigi taslarla keresteleri alip götürdüler. Asa bunlarla Geva ve Mispa kentlerini onardi.
\par 7 O sirada Bilici* Hanani Yahuda Krali Asa'ya gelip söyle dedi: "Tanrin RAB'be güvenecegine Aram Krali'na güvendin. Bu yüzden Aram Krali'nin ordusu elinden kurtuldu.
\par 8 Kûslular'la* Luvlular, çok sayida savas arabalari, atlilariyla büyük bir ordu degil miydiler? Ama sen RAB'be güvendin, O da onlari eline teslim etti.
\par 9 RAB'bin gözleri bütün yürekleriyle kendisine bagli olanlara güç vermek için her yeri görür. Akilsizca davrandin. Bundan böyle hep savas içinde olacaksin."
\par 10 Asa biliciye öfkelenip onu cezaevine attirdi. Çünkü söyledikleri onu kizdirmisti. Halktan bazi kisilere de baski yapti.
\par 11 Asa'nin yaptigi isler, basindan sonuna dek, Yahuda ve Israil krallarinin tarihinde yazilidir.
\par 12 Asa, kralliginin otuz dokuzuncu yilinda ayaklarindan hastalandi. Durumu çok agirdi. Hastaliginda RAB'be yönelecegine hekimlere basvurdu.
\par 13 Asa kralliginin kirk birinci yilinda ölüp atalarina kavustu.
\par 14 Onu özel olarak hazirlanmis, güzel kokulu çesit çesit baharat dolu bir sedyeye yatirarak Davut Kenti'nde kendisi için yaptirdigi mezara gömdüler. Onuruna çok büyük bir ates yaktilar.

\chapter{17}

\par 1 Asa'nin yerine oglu Yehosafat kral oldu. Yehosafat Israil'e karsi kralligini pekistirdi.
\par 2 Yahuda'nin bütün surlu kentlerine askeri birlikler yerlestirdi; Yahuda'ya ve babasi Asa'nin ele geçirmis oldugu Efrayim kentlerine de asker yerlestirdi.
\par 3 RAB Yehosafat'laydi. Çünkü Yehosafat atasi Davut'un baslangiçtaki yollarini izledi; Baallar'a* yönelecegine,
\par 4 babasinin Tanrisi'na yöneldi; Israil halkinin yaptiklarina degil, Tanri'nin buyruklarina uydu.
\par 5 RAB onun yönetimi altindaki kralligi pekistirdi. Bütün Yahudalilar ona armaganlar getirdi. Böylece Yehosafat büyük zenginlige ve onura kavustu.
\par 6 Yüregi RAB'de cesaret buldu, puta tapilan yerleri ve Asera* putlarini Yahuda'dan kaldirdi.
\par 7 Yehosafat kralliginin üçüncü yilinda halka ögretmek amaciyla görevlilerinden Ben-Hayil'i, Ovadya'yi, Zekeriya'yi, Netanel'i, Mikaya'yi Yahuda kentlerine gönderdi.
\par 8 Onlarla birlikte su Levililer'i de gönderdi: Semaya, Netanya, Zevadya, Asahel, Semiramot, Yehonatan, Adoniya, Toviya, Tov-Adoniya. Kâhinlerden Elisama ile Yehoram'i gönderdi.
\par 9 RAB'bin Yasa Kitabi'ni yanlarina alip Yahuda'da halka ögrettiler. Bütün Yahuda kentlerini dolasarak halka ders verdiler.
\par 10 Yahuda'yi çevreleyen ülkelerin kralliklarini RAB korkusu sardi. Bu yüzden Yehosafat'a karsi savasamadilar.
\par 11 Bazi Filistliler Yehosafat'a haraç olarak armaganlar ve gümüs verdiler. Araplar da ona davar getirdiler: Yedi bin yedi yüz koçla yedi bin yedi yüz teke.
\par 12 Yehosafat giderek güçlendi. Yahuda'da kaleler, ambarli kentler yaptirdi.
\par 13 Yahuda kentlerinde birçok isler yapti. Yerusalim'e deneyimli yigit savasçilar yerlestirdi.
\par 14 Bunlar boylarina göre söyle kaydedilmislerdi: Yahuda'dan binbasilar: Komutan Adna ve komutasinda 300 000 yigit asker;
\par 15 Yaninda Komutan Yehohanan ve komutasinda 280 000 asker;
\par 16 Onun yaninda kendini RAB'bin hizmetine adayan Zikri oglu Amatsya ve komutasinda 200 000 yigit asker.
\par 17 Benyamin oymagindan: Yigit bir savasçi olan Elyada ile komutasinda yay ve kalkanla silahlanmis 200 000 asker;
\par 18 Yaninda Yehozavat ve komutasinda savasmak üzere donatilmis 180 000 asker.
\par 19 Bunlar kralin surlu Yahuda kentlerine yerlestirdigi askerlerin yanisira, ona hizmet ediyorlardi.

\chapter{18}

\par 1 Büyük bir zenginlik ve onura kavusan Yehosafat, evlilik bagiyla Ahav'a akraba oldu.
\par 2 Birkaç yil sonra Samiriye Kenti'nde yasayan Ahav'i görmeye gitti. Ahav onun ve yanindakilerin onuruna birçok davar, sigir keserek Ramot-Gilat'a saldirmak için onu kiskirtti.
\par 3 Israil Krali Ahav, Yahuda Krali Yehosafat'a, "Ramot-Gilat'a karsi benimle birlikte savasir misin?" diye sordu. Yehosafat, "Beni kendin, halkimi halkin say. Savasta sana eslik edecegiz" diye yanitladi,
\par 4 "Ama önce RAB'be danisalim" diye ekledi.
\par 5 Israil Krali Ahav dört yüz peygamber toplayip, "Ramot-Gilat'a karsi savasalim mi, yoksa vaz mi geçeyim?" diye sordu. Peygamberler, "Savas, çünkü Tanri kenti senin eline teslim edecek" diye yanitladilar.
\par 6 Ama Yehosafat, "Burada danisabilecegimiz RAB'bin baska peygamberi yok mu?" diye sordu.
\par 7 Israil Krali, "Yimla oglu Mikaya adinda biri daha var" diye yanitladi, "Onun araciligiyla RAB'be danisabiliriz. Ama ben ondan nefret ederim. Çünkü benimle ilgili hiç iyi peygamberlik etmez, hep kötü seyler söyler." Yehosafat, "Böyle konusmaman gerekir, ey kral!" dedi.
\par 8 Israil Krali bir görevli çagirip, "Hemen Yimla oglu Mikaya'yi getir!" diye buyurdu.
\par 9 Israil Krali Ahav ile Yahuda Krali Yehosafat kral giysileriyle Samiriye Kapisi'nin girisinde, harman yerine konan tahtlarinda oturuyorlardi. Bütün peygamberler de onlarin önünde peygamberlik ediyordu.
\par 10 Kenaana oglu Sidkiya, yaptigi demir boynuzlari göstererek söyle dedi: "RAB diyor ki, 'Aramlilar'i yok edinceye dek onlari bu boynuzlarla vuracaksin."
\par 11 Öteki peygamberlerin hepsi de ayni seyi söylediler: "Ramot- Gilat'a saldir, kazanacaksin! Çünkü RAB onlari senin eline teslim edecek."
\par 12 Mikaya'yi çagirmaya giden görevli ona, "Bak! Peygamberler bir agizdan kral için olumlu seyler söylüyorlar" dedi, "Rica ederim, senin sözün de onlarinkine uygun olsun; olumlu bir sey söyle."
\par 13 Mikaya, "Yasayan RAB'bin hakki için, Tanrim ne derse onu söyleyecegim" diye karsilik verdi.
\par 14 Mikaya gelince kral, "Mikaya, Ramot-Gilat'a karsi savasa gidelim mi, yoksa vaz mi geçeyim?" diye sordu. Mikaya, "Saldirin, kazanacaksiniz! Çünkü onlar sizin elinize teslim edilecek" diye yanitladi.
\par 15 Bunun üzerine kral, "RAB'bin adina bana gerçegin disinda bir sey söylemeyecegine iliskin sana kaç kez ant içireyim?" diye sordu.
\par 16 Mikaya söyle karsilik verdi: "Israilliler'i daglara dagilmis çobansiz koyunlar gibi gördüm. RAB, 'Bunlarin sahibi yok. Herkes güvenlik içinde evine dönsün dedi."
\par 17 Israil Krali Ahav Yehosafat'a, "Benimle ilgili iyi peygamberlik etmez, hep kötü seyler söyler dememis miydim?" dedi.
\par 18 Mikaya konusmasini sürdürdü: "Öyleyse RAB'bin sözünü dinleyin! Gördüm ki, RAB tahtinda oturuyor, bütün göksel varliklar da saginda, solunda duruyordu.
\par 19 RAB sordu: 'Ramot-Gilat'a saldirip ölsün diye Israil Krali Ahav'i kim kandiracak? "Kimi söyle, kimi böyle derken,
\par 20 bir ruh çikip RAB'bin önünde durdu ve, 'Ben onu kandiracagim dedi. "RAB, 'Nasil? diye sordu.
\par 21 "Ruh, 'Aldatici ruh olarak gidip Ahav'in bütün peygamberlerine yalan söyletecegim diye karsilik verdi. "RAB, 'Onu kandirmayi basaracaksin dedi, 'Git, dedigini yap.
\par 22 "Iste RAB bu peygamberlerinin agzina aldatici bir ruh koydu. Çünkü sana kötülük etmeye karar verdi."
\par 23 Kenaana oglu Sidkiya yaklasip Mikaya'nin yüzüne bir tokat atti. "RAB'bin Ruhu nasil benden çikip da seninle konustu?" dedi.
\par 24 Mikaya, "Gizlenmek için bir iç odaya girdigin gün göreceksin" diye yanitladi.
\par 25 Bunun üzerine Israil Krali, "Mikaya'yi kentin yöneticisi Amon'a ve kralin oglu Yoas'a götürün" dedi,
\par 26 "Ben güvenlik içinde dönünceye dek bu adami cezaevinde tutmalarini, ona su ve ekmekten baska bir sey vermemelerini söyleyin!"
\par 27 Mikaya, "Eger sen güvenlik içinde dönersen, RAB benim araciligimla konusmamis demektir" dedi ve, "Herkes bunu duysun!" diye ekledi.
\par 28 Israil Krali Ahav'la Yahuda Krali Yehosafat Ramot-Gilat'a saldirmak için yola çiktilar.
\par 29 Israil Krali, Yehosafat'a, "Ben kilik degistirip savasa öyle girecegim, ama sen kral giysilerini giy" dedi. Böylece Israil Krali kiligini degistirdi, sonra savasa girdiler.
\par 30 Aram Krali, savas arabalarinin komutanlarina, "Israil Krali disinda, büyük küçük hiç kimseye saldirmayin!" diye buyruk vermisti.
\par 31 Savas arabalarinin komutanlari Yehosafat'i görünce, Israil Krali sanip saldirmak için ona döndüler. Yehosafat yakarmaya basladi. RAB Tanri ona yardim edip saldiranlarin yönünü degistirdi.
\par 32 Komutanlar onun Israil Krali olmadigini anlayinca pesini biraktilar.
\par 33 O sirada bir asker rasgele attigi bir okla Israil Krali'ni zirhinin parçalarinin birlestigi yerden vurdu. Kral arabacisina, "Dönüp beni savas alanindan çikar, yaralandim" dedi.
\par 34 Savas o gün siddetlendi. Arabasinda Aramlilar'a karsi aksama kadar dayanan Israil Krali gün batiminda öldü.

\chapter{19}

\par 1 Yahuda Krali Yehosafat ise Yerusalim'deki sarayina güvenlik içinde döndü.
\par 2 Hanani oglu Bilici* Yehu, Kral Yehosafat'i karsilamaya giderek ona söyle dedi: "Kötülere yardim edip RAB'den nefret edenleri mi sevmen gerekir? Bunun için RAB'bin öfkesi senin üstünde olacak.
\par 3 Ancak, bazi iyi yönlerin de var. Ülkeden Asera* putlarini kaldirip attin ve yürekten Tanri'ya yönelmeye karar verdin." Yehosafat'in Yaptigi Yenilikler
\par 4 Yehosafat Yerusalim'de yasadi. Beer-Seva'dan Efrayim daglik bölgesine kadar yine halkin arasinda dolasarak onlari atalarinin Tanrisi RAB'be döndürdü.
\par 5 Ülkeye, Yahuda'nin bütün surlu kentlerine yargiçlar atadi.
\par 6 Onlara, "Yaptiklariniza dikkat edin" dedi, "Çünkü insan adina degil, yargi verirken sizinle olan RAB adina yargilayacaksiniz.
\par 7 Onun için RAB'den korkun, dikkatle yargilayin. Çünkü Tanrimiz RAB kimsenin haksizlik yapmasina, kimseyi kayirmasina, rüsvet almasina göz yummaz."
\par 8 RAB'bin yargilarini uygulamak, davalara bakmak için Yehosafat Yerusalim'e de bazi Levililer'i, kâhinleri, Israil boy baslarini atadi. Bunlar Yerusalim'de yasadilar.
\par 9 Yehosafat onlara su buyruklari verdi: "Görevinizi RAB korkusuyla, baglilikla, bütün yüreginizle yapmalisiniz.
\par 10 Kentlerde yasayan kardeslerinizden gelen her davada -kan davasi, Kutsal Yasa, buyruklar, kurallar ya da ilkelerle ilgili her konuda- onlari RAB'be karsi suç islememeleri için uyarin; öyle ki RAB size de kardeslerinize de öfkelenmesin. Böyle yaparsaniz suç islememis olursunuz.
\par 11 "RAB'be iliskin her konuda Baskâhin Amarya, krala iliskin her konuda ise Yahuda oymaginin önderi Ismail oglu Zevadya size baskanlik edecek. Levililer de görevli olarak size yardimci olacaklar. Yürekli olun, bu buyruklari uygulayin. RAB dogru kisiyle olsun!"

\chapter{20}

\par 1 Bundan sonra Moavlilar, Ammonlular ve Meunlular'in bir kismi Yehosafat'la savasmak için yola çiktilar.
\par 2 Birkaç kisi Yehosafat'a gidip, "Gölün öbür yakasindan, Edom'dan sana saldirmak için büyük bir ordu geliyor. Su anda Haseson-Tamar'da -Eyn-Gedi'de-" dediler.
\par 3 Korkuya kapilan Yehosafat RAB'be danismaya karar verdi ve bütün Yahuda'da oruç* ilan etti.
\par 4 RAB'be yönelmek için Yahuda'nin bütün kentlerinden gelen halk toplanip RAB'den yardim diledi.
\par 5 Yehosafat RAB'bin Tapinagi'nda, yeni avlunun önünde, Yahuda ve Yerusalim toplulugunun arasina gidip durdu.
\par 6 "Ey atalarimizin Tanrisi RAB, sen göklerde oturan Tanri degil misin?" dedi, "Uluslarin bütün kralliklarini yöneten sensin. Güç, kudret senin elinde. Kimse sana karsi duramaz.
\par 7 Ey Tanrimiz, bu ülkede yasayanlari halkin Israil'in önünden kovan ve ülkeyi sonsuza dek dostun Ibrahim'in soyuna veren sen degil misin?
\par 8 Onlar orada yasadilar, adina bir tapinak kurdular ve,
\par 9 'Basimiza bela, savas, yargi, salgin hastalik, kitlik gelirse, adinin bulundugu bu tapinagin ve senin önünde duracagiz dediler, 'Sikintiya düstügümüzde sana yakaracagiz, sen de duyup bizi kurtaracaksin.
\par 10 "Iste Ammonlular, Moavlilar ve Seir daglik bölgesinde yasayanlar! Misir'dan çiktiktan sonra Israilliler'in onlarin ülkesine girmelerine izin vermedin. Bu yüzden atalarimiz baska yöne döndü, onlari yok etmedi.
\par 11 Ama bak, bunun karsiligini bize nasil ödüyorlar! Bize miras olarak vermis oldugun mülkünden bizi kovmaya geliyorlar.
\par 12 Ey Tanrimiz, onlari yargilamayacak misin? Çünkü bize saldiran bu büyük orduya karsi koyacak gücümüz yok. Ne yapacagimizi bilemiyoruz. Gözümüz sende."
\par 13 Bütün Yahudalilar, çoluk çocuklariyla birlikte RAB'bin önünde duruyordu.
\par 14 RAB'bin Ruhu toplulugun ortasinda duran Asaf soyundan Mattanya oglu Yeiel oglu Benaya oglu Zekeriya oglu Levili Yahaziel'in üzerine indi.
\par 15 Yahaziel söyle dedi: "Ey Kral Yehosafat, ey Yahuda halki ve Yerusalim'de oturanlar, dinleyin! RAB size söyle diyor: 'Bu büyük ordudan korkmayin, yilmayin! Çünkü savas sizin degil, Tanri'nindir.
\par 16 Yarin onlarla savasmaya çikin. Onlari vadinin sonunda, Yeruel kirlarinda, Sits Yokusu'nu çikarlarken bulacaksiniz.
\par 17 Bu kez savasmak zorunda kalmayacaksiniz. Yerinizde durup bekleyin, RAB'bin size saglayacagi kurtulusu görün, ey Yahuda ve Yerusalim halki! Korkmayin, yilmayin. Yarin onlara karsi savasa çikin. RAB sizinle olacak!"
\par 18 Yehosafat yüzüstü yere kapandi. Yahuda halkiyla Yerusalim'de oturanlar da RAB'bin önünde yere kapanip O'na tapindilar.
\par 19 Sonra Kehatogullari'ndan ve Korahogullari'ndan bazi Levililer ayaga kalkip Israil'in Tanrisi RAB'bi yüksek sesle övdüler.
\par 20 Ertesi sabah erkenden kalkip Tekoa kirlarina dogru yola çiktilar. Yola koyulduklarinda Yehosafat durup söyle dedi: "Beni dinleyin, ey Yahuda halki ve Yerusalim'de oturanlar! Tanriniz RAB'be güvenin, güvenlikte olursunuz. O'nun peygamberlerine güvenin, basarili olursunuz."
\par 21 Yehosafat halka danistiktan sonra RAB'be ezgi okumak, O'nun kutsalliginin görkemini övmek için adamlar atadi. Bunlar ordunun önünde yürüyerek söyle diyorlardi: "RAB'be sükredin, Çünkü sevgisi sonsuza dek kalicidir!"
\par 22 Onlar ezgi okuyup övgüler sunmaya basladiginda, RAB Yahuda'ya saldiran Ammonlular'a, Moavlilar'a ve Seir daglik bölgesinde yasayanlara pusu kurmustu. Hepsi bozguna ugratildi.
\par 23 Ammonlular'la Moavlilar, Seir daglik bölgesinde yasayan halki büsbütün yok etmek için onlara saldirdilar. Seirliler'i yok ettikten sonra da birbirlerini öldürmeye basladilar.
\par 24 Yahudalilar kirdaki gözcü kulesine varinca, o büyük orduya baktilar, ama sadece yere serilmis cesetler gördüler. Tek kisi kurtulmamisti.
\par 25 Mallari yagmalamaya giden Yehosafat'la askerleri, ölülerin arasinda çok miktarda mal, giysi ve degerli esya buldular. Tasiyabileceklerinden çok mal topladilar. Yagma edilecek o kadar çok mal vardi ki, toplama isi üç gün sürdü.
\par 26 Dördüncü gün Beraka Vadisi'nde toplanarak RAB'be övgüler sundular. Bu yüzden oranin adi bugün de Beraka Vadisi olarak kaldi.
\par 27 Bundan sonra bütün Yahuda ve Yerusalim halki Yehosafat'in önderliginde sevinçle Yerusalim'e döndü. Çünkü RAB düsmanlarini bozguna ugratarak onlari sevindirmisti.
\par 28 Çenk, lir ve borazan çalarak Yerusalim'e, RAB'bin Tapinagi'na gittiler.
\par 29 RAB'bin Israil'in düsmanlarina karsi savastigini duyan ülkelerin kralliklarini Tanri korkusu sardi.
\par 30 Yehosafat'in ülkesi ise baris içindeydi. Çünkü Tanrisi her yandan onu esenlikle kusatmisti.
\par 31 Yehosafat Yahuda'yi yönetti. Otuz bes yasinda kral oldu ve Yerusalim'de yirmi bes yil krallik yapti. Annesi Silhi'nin kizi Azuva'ydi.
\par 32 Babasi Asa'nin yollarini izleyen ve bunlardan sapmayan Yehosafat RAB'bin gözünde dogru olani yapti.
\par 33 Ancak alisilagelen tapinma yerleri kaldirilmadi. Halk hâlâ atalarinin Tanrisi'na bütün yüregiyle yönelmemisti.
\par 34 Yehosafat'in yaptigi öbür isler, basindan sonuna dek, Israil krallari tarihinin bir bölümü olan Hanani oglu Yehu'nun tarihinde yazilidir.
\par 35 Yahuda Krali Yehosafat bir süre sonra kendini günaha veren Israil Krali Ahazya ile anlasmaya vardi.
\par 36 Tarsis'e gidecek gemiler yapmak için anlastilar. Gemileri Esyon-Gever'de yaptilar.
\par 37 Maresali Dodavahu oglu Eliezer, Yehosafat'a karsi söyle peygamberlik etti: "Ahazya ile anlasmaya vardigin için RAB isini bozacak." Gemiler Tarsis'e gidemeden parçalandi.

\chapter{21}

\par 1 Yehosafat ölüp atalarina kavustu ve Davut Kenti'nde atalarinin yanina gömüldü. Yerine oglu Yehoram kral oldu.
\par 2 Yehosafat oglu Yehoram'in kardesleri sunlardi: Azarya, Yehiel, Zekeriya, Azaryahu, Mikael, Sefatya. Bunlarin tümü Israil Krali Yehosafat'in ogullariydi.
\par 3 Babalari onlara çok miktarda altin, gümüs, degerli esyalar armagan etmis, ayrica Yahuda'da surlu kentler vermisti. Ancak, Yehoram ilk oglu oldugundan, kralligi ona birakmisti.
\par 4 Babasinin yerine geçen Yehoram güçlenince, bütün kardeslerini ve bazi Israil*ft* önderlerini kiliçtan geçirtti.
\par 5 Yehoram otuz iki yasinda kral oldu ve Yerusalim'de sekiz yil krallik yapti.
\par 6 Karisi Ahav'in kizi oldugu için, o da Ahav'in ailesi gibi Israil krallarinin yolunu izledi ve RAB'bin gözünde kötü olani yapti.
\par 7 Ama RAB Davut'la yaptigi antlasmadan ötürü onun soyunu yok etmek istemedi. Çünkü Davut'a ve soyuna sönmeyen bir isik verecegine söz vermisti.
\par 8 Yehoram'in kralligi döneminde Edomlular Yahudalilar'a karsi ayaklanarak kendi kralliklarini kurdular.
\par 9 Yehoram komutanlari ve bütün savas arabalariyla oraya gitti. Edomlular onu ve savas arabalarinin komutanlarini kusattilar. Ama Yehoram gece kalkip kusatmayi yararak kaçti.
\par 10 Edomlular'in Yahuda'ya karsi baskaldirmasi bugün de sürüyor. O sirada Livna Kenti de ayaklandi. Çünkü Yehoram atalarinin Tanrisi RAB'bi birakmisti.
\par 11 Yahuda tepelerinde puta tapilan yerler bile yapmis, Yerusalim halkinin putlara baglanmasina önayak olmus, Yahuda halkini günaha sürüklemisti.
\par 12 Yehoram Peygamber Ilyas'tan bir mektup aldi. Mektup söyleydi: "Atan Davut'un Tanrisi RAB söyle diyor: 'Baban Yehosafat'in ve Yahuda Krali Asa'nin yollarini izlemedin.
\par 13 Israil krallarinin yolunu izledin. Ahav'in ailesinin yaptigi gibi Yahuda ve Yerusalim halkinin putlara baglanmasina önayak oldun. Üstelik kendi ailenden, senden daha iyi olan kardeslerini öldürttün.
\par 14 Bu yüzden RAB halkini, ogullarini, karilarini ve senin olan her seyi büyük bir yikimla vuracak.
\par 15 Sen de korkunç bir bagirsak hastaligina yakalanacaksin. Bu hastalik yüzünden bagirsaklarin disari dökülene dek günlerce sürüneceksin."
\par 16 RAB Filistliler'i ve Kûslular'in* yaninda yasayan Araplar'i Yehoram'a karsi kiskirtti.
\par 17 Saldirip Yahuda'ya girdiler. Sarayda bulunan her seyi, kralin ogullariyla karilarini alip götürdüler. Kralin en küçük oglu Yehoahaz'dan baska oglu kalmadi.
\par 18 Sonra RAB Yehoram'i iyilesmez bir bagirsak hastaligiyla cezalandirdi.
\par 19 Zaman geçti, ikinci yilin sonunda hastalik yüzünden bagirsaklari disari döküldü. Siddetli acilar içinde öldü. Halki atalari için ates yaktigi gibi onun onuruna ates yakmadi.
\par 20 Yehoram otuz iki yasinda kral oldu ve Yerusalim'de sekiz yil krallik yapti. Ölümüne kimse üzülmedi. Onu Davut Kenti'nde gömdülerse de, krallarin mezarligina gömmediler.

\chapter{22}

\par 1 Yerusalim halki Yehoram'in en küçük oglu Ahazya'yi babasinin yerine kral yapti. Çünkü Araplar'la ordugaha gelen akincilar büyük kardeslerinin hepsini öldürmüstü. Dolayisiyla Yehoram oglu Ahazya Yahuda Krali oldu.
\par 2 Ahazya yirmi iki yasinda kral oldu ve Yerusalim'de bir yil krallik yapti. Annesi Omri'nin torunu Atalya'ydi.
\par 3 Ahazya da Ahav ailesinin yollarini izledi. Annesi onu kötülüge itiyordu.
\par 4 Ahav ailesi gibi RAB'bin gözünde kötü olani yapti. Çünkü babasinin ölümünden beri Ahav ailesi onu yikima götürecek ögütler veriyordu.
\par 5 Ahazya onlarin ögütlerine uyarak, Aram Krali Hazael'le savasmak üzere Israil Krali Ahav oglu Yoram'la birlikte Ramot-Gilat'a gitti. Aramlilar Yoram'i yaraladilar.
\par 6 Yoram Ramot-Gilat'ta Aram Krali Hazael'le savasirken aldigi yaralarin iyilesmesi için Yizreel'e döndü. Yahuda Krali Yehoram oglu Ahazya da yaralanan Ahav oglu Yoram'i görmek için Yizreel'e gitti.
\par 7 Yoram'i ziyaret eden Ahazya'yi Tanri yikima ugratti. Ahazya oraya varinca, Yoram'la birlikte Nimsi oglu Yehu'yu karsilamaya gitti; RAB Yehu'yu Ahav soyunu ortadan kaldirmak için meshetmisti*.
\par 8 Yehu, Ahav soyunu cezalandirirken, Ahazya'ya hizmet eden Yahuda önderleriyle Ahazya'nin yegenlerini bulup hepsini öldürdü.
\par 9 Sonra Ahazya'yi aramaya koyuldu. Onu Samiriye'de gizlenirken yakalayip Yehu'ya getirdiler, sonra öldürdüler. Ahazya'yi, "Bütün yüregiyle RAB'be yönelen Yehosafat'in torunudur" diyerek gömdüler. Böylece Ahazya'nin soyunda kralligi yönetebilecek güçte kimse kalmadi.
\par 10 Ahazya'nin annesi Atalya, oglunun öldürüldügünü duyunca, Yahuda'da kral soyunun bütün bireylerini yok etmeye çalisti.
\par 11 Ne var ki, Kral Yehoram'in kizi Yehosavat, Ahazya oglu Yoas'i kralin öldürülmek istenen öteki ogullarinin arasindan alip kaçirdi ve dadisiyla birlikte yatak odasina gizledi. Yehosavat Kral Yehoram'in kizi, Kâhin Yehoyada'nin karisi, Ahazya'nin da üvey kizkardesiydi. Öldürülmesin diye çocugu Atalya'dan gizledi.
\par 12 Atalya ülkeyi yönetirken, çocuk alti yil boyunca Tanri'nin Tapinagi'nda onlarin yaninda gizlendi.

\chapter{23}

\par 1 Yedinci yil gücünü gösteren Yehoyada yüzbasi olan Yeroham oglu Azarya, Yehohanan oglu Ismail, Ovet oglu Azarya, Adaya oglu Maaseya, Zikri oglu Elisafat'la bir antlasma yapti.
\par 2 Yahuda'yi dolasarak bütün kentlerden Levililer'le Israil boy baslarini topladilar. Yerusalim'e dönen
\par 3 topluluk Tanri'nin Tapinagi'nda kralla bir antlasma yapti. Yehoyada onlara söyle dedi: "Davut'un soyuna iliskin RAB'bin verdigi söz uyarinca, kralin oglu kral olacak.
\par 4 Siz sunlari yapacaksiniz: Sabat Günü* göreve giden kâhinlerle Levililer'in üçte biri kapilarda nöbet tutacak,
\par 5 üçte biri sarayda, üçte biri de Temel Kapisi'nda duracak. Bütün halk RAB'bin Tapinagi'nin avlularinda toplanacak.
\par 6 Kâhinlerle görevli Levililer'in disinda RAB'bin Tapinagi'na kimse girmesin. Onlar kutsal olduklari için girebilirler. Halk da RAB'bin buyrugunu yerine getirmeye özen gösterecek.
\par 7 Levililer yalin kiliç kralin çevresini saracaklar. Tapinaga yaklasan olursa öldürün. Kral nereye giderse, ona eslik edin."
\par 8 Levililer'le Yahudalilar Kâhin Yehoyada'nin buyruklarini tam tamina uyguladilar. Sabat Günü göreve gidenlerle görevi biten adamlarini aldilar. Çünkü Kâhin Yehoyada bölüklere izin vermemisti.
\par 9 Kâhin Yehoyada Tanri'nin Tapinagi'ndaki Kral Davut'tan kalan mizraklari ve büyük küçük kalkanlari yüzbasilara dagitti.
\par 10 Krali korumak için sunagin ve tapinagin çevresine tapinagin güneyinden kuzeyine kadar bütün silahli adamlari yerlestirdi.
\par 11 Yehoyada'yla ogullari kralin oglu Yoas'i disari çikarip basina taç koydular. Tanri'nin Yasasi'ni da ona verip kralligini ilan ettiler. Onu meshederek*, "Yasasin kral!" diye bagirdilar.
\par 12 Atalya kosusan, krali öven halkin çikardigi gürültüyü duyunca, RAB'bin Tapinagi'nda toplananlarin yanina gitti.
\par 13 Bakti, kral tapinagin girisinde sütunun yaninda duruyor; yüzbasilar, borazan çalanlar çevresine toplanmis. Ülke halki sevinç içindeydi, borazanlar çaliniyor, ezgiciler çalgilariyla övgüleri yönetiyordu. Atalya giysilerini yirtarak, "Hainlik! Hainlik!" diye bagirdi.
\par 14 Kâhin Yehoyada yüzbasilari çagirarak, "O kadini aradan çikarin. Ardindan kim giderse kiliçtan geçirin" diye buyruk verdi. Çünkü, "Onu RAB'bin Tapinagi'nda öldürmeyin" demisti.
\par 15 Atalya yakalandi ve sarayin At Kapisi'na varir varmaz öldürüldü.
\par 16 Yehoyada RAB'bin halki olmalari için kendisiyle halk ve kral arasinda bir antlasma yapti.
\par 17 Bütün halk gidip Baal'in* tapinagini yikti. Sunaklarini, putlarini parçaladilar; Baal'in Kâhini Mattan'i da sunaklarin önünde öldürdüler.
\par 18 Yehoyada Levili kâhinleri RAB'bin Tapinagi'ndaki isin basina getirdi. Onlari Davut tarafindan atanmis olduklari görevleri yerine getirmek, Musa'nin Yasasi'nda yazilanlar uyarinca RAB'be yakmalik sunular* sunmak ve Davut'un koydugu düzene göre sevinçle ezgiler okumakla görevlendirdi.
\par 19 Ayrica dinsel açidan herhangi bir nedenle kirli birinin içeri girmemesi için RAB'bin Tapinagi'nin kapilarina da nöbetçiler yerlestirdi.
\par 20 Sonra yüzbasilari, soylulari, halkin yöneticilerini ve ülke halkini yanina alarak krali RAB'bin Tapinagi'ndan getirdi. Yukari Kapi'dan geçip saraya girerek krali tahta oturttular.
\par 21 Ülke halki sevinç içindeydi, ancak kent suskundu. Çünkü Atalya kiliçla öldürülmüstü.

\chapter{24}

\par 1 Yoas yedi yasinda kral oldu ve Yerusalim'de kirk yil krallik yapti. Annesi Beer-Sevali Sivya'ydi.
\par 2 Yoas Kâhin Yehoyada yasadigi sürece RAB'bin gözünde dogru olani yapti.
\par 3 Yehoyada onu iki kadinla evlendirdi. Yoas'in onlardan ogullari, kizlari oldu.
\par 4 Bir süre sonra Yoas RAB'bin Tapinagi'ni onarmaya karar verdi.
\par 5 Kâhinlerle Levililer'i toplayarak söyle dedi: "Yahuda kentlerine gidip Tanriniz'in Tapinagi'ni onarmak için gerekli yillik parayi bütün Israil halkindan toplayin. Bunu hemen yapin." Ama Levililer hemen ise girismediler.
\par 6 Bunun üzerine kral, Baskâhin Yehoyada'yi çagirip, "RAB'bin kulu Musa'nin ve Israil toplulugunun Levha Sandigi'nin bulundugu çadir için koydugu vergiyi Yahuda ve Yerusalim'den toplasinlar diye Levililer'i neden uyarmadin?" diye sordu.
\par 7 Çünkü o kötü kadinin, Atalya'nin ogullari, RAB Tanri'nin Tapinagi'na zorla girerek bütün adanmis esyalari Baallar* için kullanmislardi.
\par 8 Kralin buyrugu uyarinca bir sandik yapilarak RAB'bin Tapinagi'nin kapisinin disina yerlestirildi.
\par 9 Tanri'nin kulu Musa'nin çölde Israil halkina koydugu vergiyi RAB'be getirmeleri için Yahuda ve Yerusalim halkina bir çagri yapildi.
\par 10 Bütün önderlerle halk vergilerini sevinçle getirdiler, doluncaya dek sandigin içine attilar.
\par 11 Levililer, sandigi kralin görevlilerine getiriyorlardi. Çok para biriktigini görünce kralin yazmaniyla baskâhinin görevlisine haber veriyor, onlar da gelip sandigi bosaltiyor, sonra yine yerine koyuyorlardi. Bunu her gün yaptilar ve çok para topladilar.
\par 12 Kral Yoas'la Yehoyada parayi RAB'bin Tapinagi'nda yapilan islerden sorumlu olanlara veriyorlardi. RAB'bin Tapinagi'ni onarmak için ücretle tasçilar, marangozlar, demir ve tunç* isçileri tuttular.
\par 13 Isten sorumlu kisiler çaliskandi, onarim isi onlar sayesinde ilerledi. Tanri'nin Tapinagi'ni eski durumuna getirip saglamlastirdilar.
\par 14 Onarim isi bitince, geri kalan parayi krala ve Yehoyada'ya getirdiler. Bununla RAB'bin Tapinagi'ndaki hizmet ve yakmalik sunulari* sunmak için gereçler, tabaklar, altin ve gümüs esyalar yaptilar. Yehoyada yasadigi sürece RAB'bin Tapinagi'nda sürekli yakmalik sunu sunuldu.
\par 15 Yehoyada uzun yillar yasadiktan sonra yüz otuz yasinda öldü.
\par 16 Israil'de Tanri'ya ve Tanri'nin Tapinagi'na yaptigi yararli hizmetlerden dolayi kendisini Davut Kenti'nde krallarin yanina gömdüler.
\par 17 Yehoyada'nin ölümünden sonra Yahuda önderleri gelip kralin önünde egildiler. Kral da onlari dinledi.
\par 18 Atalarinin Tanrisi RAB'bin Tapinagi'ni terk ederek Asera* putlariyla öbür putlara taptilar. Suçlari yüzünden RAB Yahuda ve Yerusalim halkina öfkelendi.
\par 19 Kendisine dönmeleri için aralarina peygamberler gönderdi. Bu peygamberler halki uyardilarsa da, onlara kulak asmadilar.
\par 20 O zaman Tanri'nin Ruhu Kâhin Yehoyada oglu Zekeriya'nin üzerine indi. Zekeriya, halkin önünde durup seslendi: "Tanri söyle diyor: 'Niçin buyruklarima karsi geliyorsunuz? Isleriniz iyi gitmeyecek. Çünkü siz beni biraktiniz, ben de sizi biraktim."
\par 21 Bunun üzerine Zekeriya'ya düzen kurdular. Kralin buyruguyla RAB'bin Tapinagi'nin avlusunda tasa tutup onu öldürdüler.
\par 22 Kral Yoas Zekeriya'nin babasi Yehoyada'nin kendisine yaptigi iyiligi unutup oglunu öldürdü. Zekeriya ölürken, "RAB bu yaptiginizi görüp hesabini sorsun" dedi.
\par 23 Yil sonunda Aram ordusu Yoas'a saldirip Yahuda ve Yerusalim'e girdi. Halkin bütün önderlerini öldürdüler. Yagmaladiklari mallarin hepsini Sam Krali'na gönderdiler.
\par 24 Aram ordusu küçük bir ordu olmasina karsin, RAB çok büyük bir orduyu ellerine teslim etmisti. Çünkü Yahuda halki, atalarinin Tanrisi RAB'bi birakmisti. Böylece Aram ordusu Yoas'i cezalandirmis oldu.
\par 25 Aramlilar Yoas'i agir yarali durumda birakip gittiler. Yoas'in görevlileri, Kâhin Yehoyada'nin oglunun kanini döktügü için, düzen kurup onu yataginda öldürdüler. Yoas'i Davut Kenti'nde gömdülerse de, krallarin mezarligina gömmediler.
\par 26 Yoas'a düzen kuranlar sunlardi: Ammonlu Simat adli kadinin oglu Zavat, Moavli Simrit adli kadinin oglu Yehozavat.
\par 27 Yoas'in ogullarinin öyküsü, kendisine Tanri'dan defalarca gelen bildiriler, Tanri'nin Tapinagi'nin onarimiyla ilgili konular krallarin tarihine iliskin yorumda yazilidir. Yerine oglu Amatsya kral oldu.

\chapter{25}

\par 1 Amatsya yirmi bes yasinda kral oldu ve Yerusalim'de yirmi dokuz yil krallik yapti. Annesi Yerusalimli Yehoaddan'di.
\par 2 Amatsya RAB'bin gözünde dogru olani yaptiysa da bütün yüregiyle yapmadi.
\par 3 Kralligini güçlendirdikten sonra, babasini öldüren görevlileri öldürttü.
\par 4 Ancak Musa'nin kitabindaki yasaya uyarak çocuklarini öldürtmedi. Çünkü RAB, "Ne babalar çocuklarinin günahindan ötürü öldürülecek, ne de çocuklar babalarinin. Herkes kendi günahi için öldürülecek" diye buyurmustu.
\par 5 Amatsya Yahudalilar'i topladi. Bütün Yahudalilar'la Benyaminliler'i boylarina göre binbasilarin ve yüzbasilarin denetimine verdi. Yaslari yirmi ve yirminin üstünde olanlarin sayimini yapti. Savasa hazir, mizrak ve kalkan kullanabilen üç yüz bin seçme asker oldugunu saptadi.
\par 6 Ayrica Israil'den yüz talant*fa* gümüs karsiliginda, yüz bin yigit savasçi tuttu.
\par 7 Bu sirada bir Tanri adami gelip krala söyle dedi: "Ey kral, Israil'den gelen askerler seninle savasa gitmesin. Çünkü RAB Israilliler'le, yani Efrayimliler'le birlikte degil.
\par 8 Gidersen, yigitçe savassan bile Tanri seni düsmanin önünde bozguna ugratacak. Tanri'nin yardim etmeye de, bozguna ugratmaya da gücü vardir."
\par 9 Amatsya, "Ya Israil askerleri için ödedigim yüz talant ne olacak?" diye sordu. Tanri adami, "RAB sana bundan çok daha fazlasini verebilir" diye karsilik verdi.
\par 10 Bunun üzerine Amatsya Efrayim'den gelen askerlerin evlerine dönmelerini buyurdu. Yahudalilar'a çok kizan askerler büyük bir öfke içinde evlerine döndüler.
\par 11 Amatsya cesaretini toplayarak ordusunu Tuz Vadisi'ne götürdü. Yahudalilar orada on bin Seirli'yi öldürdü.
\par 12 Sag olarak ele geçirdikleri on bin kisiyi de uçurumun kenarina götürüp oradan asagiya attilar. Hepsi paramparça oldu.
\par 13 Bu arada Amatsya'nin kendisiyle birlikte savasa katilmamalari için geri gönderdigi askerler, Samiriye ile Beythoron arasindaki Yahuda kentlerine saldirdilar. Üç bin kisi öldürüp çok miktarda mal yagmaladilar.
\par 14 Amatsya, Edomlular'i bozguna ugrattiktan sonra, dönüste Seir halkinin putlarini yaninda getirdi. Kendisine ilah olarak diktigi putlara tapti ve buhur yakti.
\par 15 RAB Amatsya'ya öfkelendi. Ona bir peygamber göndererek, "Halkini senin elinden kurtaramayan bu halkin ilahlarina neden tapiyorsun?" diye sordu.
\par 16 Peygamber konusmasini sürdürmekteyken Amatsya ona, "Seni krala danisman mi atadik? Sus! Yoksa öldürüleceksin" dedi. Peygamber, "Tanri'nin seni yok etmeye karar verdigini biliyorum. Çünkü sen bu kötülügü yaptin, ögüdüme de aldiris etmedin!" dedikten sonra sustu.
\par 17 Yahuda Krali Amatsya, danismanlariyla konustuktan sonra Yehu oglu Yehoahaz oglu Israil Krali Yehoas'a, "Gel, yüz yüze görüselim" diye haber gönderdi.
\par 18 Israil Krali Yehoas da karsilik olarak Yahuda Krali Amatsya'ya su haberi gönderdi: "Lübnan'da dikenli bir çali, sedir agacina, 'Kizini ogluma es olarak ver diye haber yollar. O sirada oradan geçen yabanil bir hayvan basip çaliyi çigner.
\par 19 Iste Edomlular'i bozguna ugrattin diye böbürleniyorsun. Otur evinde! Niçin bela ariyorsun? Kendi basini da, Yahuda halkinin basini da derde sokacaksin."
\par 20 Ne var ki, Amatsya dinlemek istemedi. Bunun böyle olmasi Tanri'nin istegiydi. Edom'un ilahlarina taptiklari için Tanri onlari düsmanin eline teslim etmeye karar vermisti.
\par 21 Derken Israil Krali Yehoas Yahuda Krali Amatsya'nin üzerine yürüdü. Iki ordu Yahuda'nin Beytsemes Kenti'nde karsilasti.
\par 22 Israilliler'in önünde bozguna ugrayan Yahudalilar evlerine kaçti.
\par 23 Israil Krali Yehoas, Yehoahaz oglu Yoas oglu Yahuda Krali Amatsya'yi Beytsemes'te yakalayip Yerusalim'e götürdü. Efrayim Kapisi'ndan Köse Kapisi'na kadar Yerusalim surlarinin dört yüz arsinlik bölümünü yiktirdi.
\par 24 Tanri'nin Tapinagi'nda Ovet-Edom soyunun gözetimi altindaki altini, gümüsü, esyalari, sarayin hazinelerini ve bazi adamlari da rehine olarak yanina alip Samiriye'ye döndü.
\par 25 Yahuda Krali Yoas oglu Amatsya, Israil Krali Yehoahaz oglu Yehoas'in ölümünden sonra on bes yil daha yasadi.
\par 26 Amatsya'nin yaptigi öbür isler, basindan sonuna dek, Yahuda ve Israil krallarinin tarihinde yazilidir.
\par 27 RAB'bin ardinca yürümekten vazgeçtigi andan baslayarak Yerusalim'de Amatsya'ya bir düzen kurulmustu. Amatsya Lakis'e kaçti. Ardindan adam göndererek onu öldürttüler.
\par 28 Ölüsü at sirtinda getirildi, Yahuda Kenti'nde atalarinin yanina gömüldü.

\chapter{26}

\par 1 Yahuda halki Amatsya'nin yerine on alti yasindaki oglu Uzziya'yi kral yapti.
\par 2 Babasi Amatsya ölüp atalarina kavustuktan sonra Uzziya Eylat Kenti'ni onarip Yahuda topraklarina katti.
\par 3 Uzziya on alti yasinda kral oldu ve Yerusalim'de elli iki yil krallik yapti. Annesi Yerusalimli Yekolya'ydi.
\par 4 Babasi Amatsya gibi, Uzziya da RAB'bin gözünde dogru olani yapti.
\par 5 Kendisine Tanri korkusunu ögreten Zekeriya'nin günlerinde Tanri'ya yöneldi. RAB'be yöneldigi sürece Tanri onu basarili kildi.
\par 6 Uzziya Filistliler'e savas açti. Gat, Yavne ve Asdot'un surlarini yiktirdi. Sonra Asdot yakinlarinda ve Filist bölgesinde kentler kurdu.
\par 7 Filistliler'e, Gur-Baal'da yasayan Araplar'a ve Meunlular'a karsi Tanri ona yardim etti.
\par 8 Ammonlular Uzziya'ya haraç vermeye basladilar. Gitgide güçlenen Uzziya'nin ünü Misir sinirina dek ulasti.
\par 9 Uzziya Yerusalim'de Köse Kapisi, Dere Kapisi ve surun kösesi üzerinde kuleler kurup bunlari saglamlastirdi.
\par 10 Sefela'da ve ovada çok sayida hayvani oldugundan, kirda gözetleme kuleleri yaptirarak birçok sarniç açtirdi. Daglik bölgelerde, verimli topraklarda da irgatlari, bagcilari vardi; çünkü topragi severdi.
\par 11 Uzziya'nin savasa hazir bir ordusu vardi. Kralin komutanlarindan Hananya'nin denetiminde Yazman Yeiel'le görevli Maaseya'nin yaptigi sayimin sonuçlarina göre, bu ordu bölük bölük savasa girerdi.
\par 12 Yigit savasçilari yöneten boy baslarinin sayisi 2 600'dü.
\par 13 Bunlarin komutasi altinda krala yardim etmek için düsmanla yigitçe savasacak 307 500 askerden olusan bir ordu vardi.
\par 14 Uzziya bütün ordu için kalkan, mizrak, migfer, zirh, yay, sapan tasi sagladi.
\par 15 Yerusalim'de becerikli adamlarca tasarlanmis gereçler yaptirdi. Oklari, büyük taslari firlatmak için bu gereçleri kulelere ve köselere yerlestirdi. Uzziya'nin ünü uzaklara kadar yayildi; çünkü gördügü olaganüstü yardim sayesinde büyük güce kavusmustu.
\par 16 Ne var ki, güçlenince kendisini yikima sürükleyecek bir gurura kapildi. Tanrisi RAB'be ihanet etti. Buhur sunagi üzerinde buhur yakmak için RAB'bin Tapinagi'na girdi.
\par 17 Kâhin Azarya ile RAB'bin yürekli seksen kâhini de ardisira tapinaga girdiler.
\par 18 Kral Uzziya'ya karsi durarak, "Ey Uzziya, RAB'be buhur yakmaya hakkin yok!" dediler, "Ancak Harun soyundan kutsanmis kâhinler buhur yakabilir. Tapinaktan çik! Çünkü sen RAB'be ihanet ettin; RAB Tanri da seni onurlandirmayacak!"
\par 19 Buhur yakmak için elinde buhurdan tutan Uzziya kâhinlere öfkelendi. Öfkelenir öfkelenmez de kâhinlerin önünde, RAB'bin Tapinagi'ndaki buhur sunaginin yaninda duran Uzziya'nin alninda deri hastaligi belirdi.
\par 20 Baskâhin Azarya ile öbür kâhinler ona bakinca alninda deri hastaligi belirdigini gördüler. Onu çabucak oradan çikardilar. Uzziya da çikmaya istekliydi, çünkü RAB onu cezalandirmisti.
\par 21 Kral Uzziya ölünceye kadar deri hastaligindan kurtulamadi. Bu yüzden ayri bir evde yasadi ve RAB'bin Tapinagi'na sokulmadi. Sarayi ve ülke halkini oglu Yotam yönetti.
\par 22 Uzziya'nin yaptigi öbür isleri, basindan sonuna dek, Amots oglu Peygamber Yesaya yazmistir.
\par 23 Uzziya ölüp atalarina kavusunca, deri hastaligina yakalandigi için onu atalarinin yanina, kral mezarliginin ayri bir yerine gömdüler. Yerine oglu Yotam kral oldu.

\chapter{27}

\par 1 Yotam yirmi bes yasinda kral oldu ve Yerusalim'de on alti yil krallik yapti. Annesi Sadok'un kizi Yerusa'ydi.
\par 2 Babasi Uzziya gibi, Yotam da RAB'bin gözünde dogru olani yapti. Ancak RAB'bin Tapinagi'na girmedi. Ne var ki, halk kötülük yapmayi sürdürmekteydi.
\par 3 Yotam RAB'bin Tapinagi'nin Yukari Kapisi'ni onardi. Ofel Tepesi'ndeki surun üzerinde de birçok is yapti.
\par 4 Yahuda'nin daglik bölgesinde kentler, ormanlik tepelerinde kaleler ve kuleler kurdu.
\par 5 Ammon Krali'yla savasarak Ammonlular'i yendi. Ammonlular o yil için kendisine yüz talant gümüs, on bin kor bugday, on bin kor arpa verdiler. Ikinci ve üçüncü yillar için de ayni miktari ödediler.
\par 6 Tanrisi RAB'bin önünde kararli bir sekilde yürüyen Yotam giderek güçlendi.
\par 7 Yotam'in yaptigi öbür isler, bütün savaslari ve uygulamalari, Israil ve Yahuda krallarinin tarihinde yazilidir.
\par 8 Yotam yirmi bes yasinda kral oldu ve Yerusalim'de on alti yil krallik yapti.
\par 9 Yotam ölüp atalarina kavusunca, onu Davut Kenti'nde gömdüler. Yerine oglu Ahaz kral oldu.

\chapter{28}

\par 1 Ahaz yirmi yasinda kral oldu ve Yerusalim'de on alti yil krallik yapti. RAB'bin gözünde dogru olani yapan atasi Davut gibi davranmadi.
\par 2 Israil krallarinin yollarini izledi; Baallar'a* tapmak için dökme putlar bile yaptirdi.
\par 3 Ben-Hinnom Vadisi'nde buhur yakti. RAB'bin Israil halkinin önünden kovmus oldugu uluslarin igrenç törelerine uyarak ogullarini ateste kurban etti.
\par 4 Puta tapilan yerlerde, tepelerde, bol yaprakli her agacin altinda kurban kesip buhur yakti.
\par 5 Iste bu nedenle Tanrisi RAB Ahaz'i Aram Krali'nin eline teslim etti. Aramlilar onu bozguna ugrattilar, halkindan birçogunu tutsak alip Sam'a götürdüler. Ayrica onu agir bir yenilgiye ugratan Israil Krali'nin eline de teslim edildi.
\par 6 Remalya oglu Israil Krali Pekah, Yahuda'da bir günde yüz yirmi bin yigit askeri öldürttü. Çünkü Yahuda halki atalarinin Tanrisi RAB'be sirt çevirmisti.
\par 7 Efrayimli yigit Zikri, kralin oglu Maaseya'yi, saray sorumlusu Azrikam'i ve kraldan sonra ikinci adam olan Elkana'yi öldürdü.
\par 8 Israilliler kardesleri olan Yahudalilar'dan iki yüz bin kadinla çocugu tutsak aldi. Bu arada çok miktarda mali da yagmalayip Samiriye'ye tasidilar.
\par 9 Orada RAB'bin Odet adinda bir peygamberi vardi. Odet Samiriye'ye dönmekte olan ordunun karsisina çikip söyle dedi: "Atalarinizin Tanrisi RAB, Yahuda halkina öfkelendigi için onlari elinize teslim etti. Ama siz göklere erisen bir öfkeyle onlari öldürdünüz.
\par 10 Simdi de Yahuda ve Yerusalim halkini kendinize kadin ve erkek köleler yapmayi düsünüyorsunuz. Tanriniz RAB'be karsi siz hiç suç islemediniz mi?
\par 11 Simdi beni dinleyin! Kardeslerinizden aldiginiz tutsaklari geri gönderin, çünkü RAB'bin kizgin öfkesi üzerinizdedir."
\par 12 Efrayim halkinin önderlerinden Yehohanan oglu Azarya, Mesillemot oglu Berekya, Sallum oglu Yehizkiya ve Hadlay oglu Amasa savastan dönenlerin karsisina çikarak,
\par 13 "Tutsaklari buraya getirmeyin, yoksa RAB'be karsi suç islemis olacagiz" dediler, "Suçlarimizi, günahlarimizi çogaltmak mi istiyorsunuz? Simdiden yeterince suçumuz var zaten. RAB'bin kizgin öfkesi Israil halkinin üzerindedir."
\par 14 Bunun üzerine savasçilar tutsaklarla yagmalanan mallari önderlerin ve toplulugun önüne biraktilar.
\par 15 Görevlendirilen belirli kisiler tutsaklari aldilar, yagmalanmis giysilerle aralarindaki çiplaklarin hepsini giydirdiler. Onlara giysi, çarik, yiyecek, içecek sagladilar. Yaralarina zeytinyagi sürdüler. Yürüyemeyecek durumda olanlari eseklere bindirdiler. Onlari hurma kenti Eriha'ya, kardeslerine geri götürdükten sonra Samiriye'ye döndüler.
\par 16 O sirada Kral Ahaz yardim istemek için Asur Krali'na haber gönderdi.
\par 17 Edomlular yine Yahuda'ya saldirmis, onlari yenip tutsak almislardi.
\par 18 Filistliler ise Sefela bölgesiyle Yahuda'nin Negev bölgesindeki kentlere akinlar düzenleyip Beytsemes, Ayalon, Gederot ile Soko, Timna, Gimzo ve çevre köylerini ele geçirerek buralara yerlesmislerdi.
\par 19 RAB Israil Krali Ahaz yüzünden Yahuda halkini bu ezik duruma düsürmüstü. Çünkü Ahaz Yahuda'yi günaha özendirmis ve RAB'be ihanet etmisti.
\par 20 Asur Krali Tiglat-Pileser Ahaz'a geldi, ama yardim edecegine onu sikintiya düsürdü.
\par 21 Ahaz RAB'bin Tapinagi'ndan, saraydan ve önderlerden aldiklarini Asur Krali'na armagan ettiyse de ona yaranamadi.
\par 22 Iste Ahaz denen bu kral, sikintili günlerinde RAB'be ihanetini artirdi.
\par 23 Daha önce kendisini bozguna ugratan Sam ilahlarina kurbanlar sundu. "Madem ilahlari Aram krallarina yardim etti, onlara kurban sunayim da bana da yardim etsinler" diye düsünüyordu. Ne var ki, bu ilahlar onun da, bütün Israil halkinin da yikimina neden oldu.
\par 24 Ahaz Tanri'nin Tapinagi'ndaki esyalari toplayip parçaladi. RAB'bin Tapinagi'nin kapilarini kapatip Yerusalim'in her kösesinde sunaklar yaptirdi.
\par 25 Baska ilahlara buhur yakmak için Yahuda'nin her kentinde tapinma yerleri yaparak atalarinin Tanrisi RAB'bi öfkelendirdi.
\par 26 Ahaz'in yaptigi öbür isler ve bütün uygulamalari, basindan sonuna dek, Yahuda ve Israil krallarinin tarihinde yazilidir.
\par 27 Ahaz ölüp atalarina kavusunca, onu Israil*fh* krallarinin mezarligina gömeceklerine Yerusalim Kenti'nde gömdüler. Yerine oglu Hizkiya kral oldu.

\chapter{29}

\par 1 Hizkiya yirmi bes yasinda kral oldu ve Yerusalim'de yirmi dokuz yil krallik yapti. Annesi Zekeriya'nin kizi Aviya'ydi.
\par 2 Atasi Davut gibi, o da RAB'bin gözünde dogru olani yapti. Hizkiya Tapinagi Onarip Kutsuyor
\par 3 Hizkiya kralliginin birinci yilinin birinci ayinda* RAB'bin Tapinagi'nin kapilarini açip onardi.
\par 4 Sonra kâhinlerle Levililer'i çagirip tapinagin dogusundaki alanda topladi.
\par 5 Onlara, "Ey Levililer, beni dinleyin!" dedi, "Simdi kendinizi kutsayin; atalarinizin Tanrisi RAB'bin Tapinagi'ni da kutsayin. Igrenç olan her seyi kutsal yerden çikarin.
\par 6 Atalarimiz Tanri'ya ihanet ettiler. Tanrimiz RAB'bin gözünde kötü olani yaparak O'nu biraktilar. Yüzlerini RAB'bin Konutu'ndan ayirip ona sirt çevirdiler.
\par 7 Tapinagin eyvana açilan kapilarini kapattilar, kandilleri sönmeye biraktilar. Kutsal yerde Israil'in Tanrisi'na buhur yakmadilar, yakmalik sunu* da sunmadilar.
\par 8 Yahuda ve Yerusalim halki bu yüzden RAB'bin öfkesine ugradi. Gözlerinizle gördügünüz gibi, RAB'bin onlara yaptigi, baskalarini korkuya, dehsete düsürdü. Alay konusu oldular.
\par 9 Iste bu yüzden, babalarimiz kiliçtan geçirildi; ogullarimiz, kizlarimiz, karilarimiz tutsak alindi.
\par 10 Simdi, bize duydugu kizgin öfkeyi yatistirmak için, Israil'in Tanrisi RAB'le bir antlasma yapmayi tasarliyorum.
\par 11 Ogullarim, artik isi savsaklamayin! Çünkü RAB önünde durmaniz, hizmet etmeniz, hizmetkârlari olmaniz ve buhur yakmaniz için sizi seçti."
\par 12 Ise baslayan Levililer sunlardi: Kehat boyundan Amasay oglu Mahat, Azarya oglu Yoel; Merari boyundan Avdi oglu Kis, Yehallelel oglu Azarya; Gerson boyundan Zimma oglu Yoah, Yoah oglu Eden;
\par 13 Elisafan soyundan Simri, Yeiel; Asaf soyundan Zekeriya, Mattanya;
\par 14 Heman soyundan Yehiel, Simi; Yedutun soyundan Semaya ve Uzziel.
\par 15 Bu Levililer kardeslerini topladilar. Kendilerini kutsadiktan sonra, RAB'bin sözü uyarinca kralin buyruguyla RAB'bin Tapinagi'ni dinsel açidan arindirmak için içeri girdiler.
\par 16 RAB'bin Tapinagi'ni arindirmak için içeri giren kâhinler tapinakta bulduklari bütün kirli sayilan seyleri tapinagin avlusuna çikardilar. Levililer bunlari disari çikarip Kidron Vadisi'ne götürdüler.
\par 17 Birinci ayin ilk günü kutsamaya basladilar; ayin sekizinci günü tapinagin eyvanina vardilar. Tapinagi kutsamayi sekiz gün daha sürdürerek, birinci ayin on altinci günü isi bitirdiler.
\par 18 Sonra Kral Hizkiya'ya giderek, "Bütün RAB'bin Tapinagi'ni, yakmalik sunu sunagiyla takimlarini, adak ekmeklerinin* dizildigi masayla takimlarini arindirdik" dediler,
\par 19 "Ayrica kralligi döneminde ihanet eden Ahaz'in attirdigi bütün takimlari da hazirlayip kutsadik. Hepsi RAB'bin sunaginin önünde duruyor."
\par 20 Ertesi gün Kral Hizkiya erkenden kentin ileri gelenlerini toplayip onlarla birlikte RAB'bin Tapinagi'na gitti.
\par 21 Kral ailesi, tapinak ve Yahuda halki için günah sunusu* olarak yedi boga, yedi koç, yedi kuzu, yedi teke getirdiler. Hizkiya bunlari RAB'bin sunaginin üzerinde sunmalari için Harun soyundan gelen kâhinlere buyruk verdi.
\par 22 Önce bogalar kesildi, kâhinler bogalarin kanini sunagin üzerine döktüler. Sonra sirasiyla koçlari ve kuzulari keserek kanlarini sunagin üzerine döktüler.
\par 23 Tekeleri günah sunusu olarak kralla toplulugun önüne getirdiler. Kralla topluluk ellerini tekelerin üzerine koydu.
\par 24 Sonra kâhinler tekeleri kestiler. Bütün Israil halkinin günahlarini bagislatmak için tekelerin kanini sunagin üzerinde günah sunusu olarak sundular. Çünkü kral yakmalik sunu ve günah sunusunu bütün Israil halki adina sunmalari için buyruk vermisti.
\par 25 Sonra kral Davut'un, bilicisi* Gad'in ve Peygamber Natan'in düzenine göre Levililer'i ziller, çenkler ve lirlerle RAB'bin Tapinagi'na yerlestirdi. RAB bu düzeni peygamberleri araciligiyla vermisti.
\par 26 Böylece Levililer Davut'un çalgilariyla, kâhinler de borazanlariyla yerlerini aldilar.
\par 27 Hizkiya yakmalik sununun sunagin üzerinde yakilmasini buyurdu. Sunu sunulmaya baslayinca, borazanlar ve Israil Krali Davut'un çalgilari esliginde RAB'be ezgiler okumaya koyuldular.
\par 28 Ezgiciler ezgi söylüyor, borazancilar borazan çaliyor, bütün topluluk tapiniyordu. Yakmalik sunu bitinceye dek bu böyle sürüp gitti.
\par 29 Yakmalik sunular sunulduktan sonra, kralla yanindakiler yere kapanip tapindilar.
\par 30 Kral Hizkiya ile önderler, Levililer'e Davut'un ve Bilici Asaf'in sözleriyle RAB'bi övmelerini söylediler. Onlar da sevinçle övgüler sundular, baslarini egip tapindilar.
\par 31 Hizkiya, "Artik kendinizi RAB'be adamis bulunuyorsunuz" dedi, "Gelin, RAB'bin Tapinagi'na kurbanlar, sükran sunulari getirin." Bunun üzerine topluluk kurban ve sükran sunulari getirdi. Içlerindeki istekli kisiler de yakmalik sunular getirdiler.
\par 32 Toplulugun yakmalik sunu olarak getirdigi hayvanlarin sayisi yetmis sigir, yüz koç, iki yüz kuzuydu. Bunlarin tümü RAB'be yakmalik sunu olarak sunulmak içindi.
\par 33 Kurban olarak adanan hayvanlar alti yüz sigir, üç bin davardi.
\par 34 Yakmalik sunu olarak kesilen hayvanlarin derilerini yüzecek kâhinlerin sayisi yetersizdi. Bu nedenle kardesleri Levililer is bitene ve öbür kâhinler kutsanana dek onlara yardim etti. Çünkü Levililer kendilerini kutsamaya kâhinlerden daha çok özen göstermislerdi.
\par 35 Çok sayida yakmalik sununun yanisira esenlik sunularinin* yagi ve yakmalik sunularla birlikte sunulan dökmelik sunular da vardi. Böylece RAB'bin Tapinagi'ndaki hizmet düzeni yeniden kurulmus oldu.
\par 36 Hizkiya'yla bütün halk, Tanri'nin halk için yaptiklari karsisinda sevinç içindeydi; çünkü her sey çabucak tamamlanmisti.

\chapter{30}

\par 1 Hizkiya, bütün Israil ve Yahuda halkina haber göndererek, Efrayim ve Manasse halkina da mektup yazarak, Israil'in Tanrisi RAB için Fisih Bayrami'ni* kutlamak amaciyla Yerusalim'deki RAB'bin Tapinagi'na gelmelerini bildirdi.
\par 2 Kralla önderleri ve Yerusalim'deki topluluk Fisih Bayrami'ni ikinci ay* kutlamaya karar verdiler.
\par 3 Bayrami zamaninda kutlayamamislardi; çünkü ne kendini kutsamis yeterli sayida kâhin vardi, ne de halk Yerusalim'de toplanabilmisti.
\par 4 Bu tasari krala da topluluga da uygun göründü.
\par 5 Herkes Yerusalim'e gelip Israil'in Tanrisi RAB için Fisih Bayrami'ni kutlasin diye Beer-Seva'dan Dan'a kadar bütün Israil ülkesine duyuru yapmaya karar verdiler. Çünkü bayram, yazili oldugu gibi çok sayida insanla kutlanmamisti.
\par 6 Kralin buyrugu uyarinca ulaklar, kral ve önderlerinden aldiklari mektuplarla bütün Israil ve Yahuda'yi kosa kosa dolasarak su duyuruyu yaptilar: "Ey Israil halki! Ibrahim'in, Ishak'in ve Israil'in Tanrisi RAB'be dönün. Öyle ki, O da sag kalanlariniza, Asur krallarinin elinden kurtulanlariniza dönsün.
\par 7 Atalarinin Tanrisi RAB'be ihanet eden atalariniza, kardeslerinize benzemeyin; gördügünüz gibi RAB onlari dehset verici bir duruma düsürdü.
\par 8 Atalariniz gibi inatçi olmayin, RAB'be boyun egin. O'nun sonsuza dek kutsal kildigi tapinaga gelin. Kizgin öfkesinin sizden uzaklasmasi için Tanriniz RAB'be kulluk edin.
\par 9 RAB'be dönerseniz, kardeslerinizi, ogullarinizi tutsak edenler onlara aciyip bu ülkeye dönmelerine izin vereceklerdir. Çünkü Tanriniz RAB lütfeden, aciyan bir Tanri'dir. O'na dönerseniz, O da yüzünü sizden çevirmeyecektir."
\par 10 Ulaklar Efrayim ve Manasse bölgelerinde Zevulun'a dek kent kent dolastilar. Ne var ki, halk gülerek onlarla alay etti.
\par 11 Ama Aser, Manasse ve Zevulun halkindan bazilari alçakgönüllü bir tutum takinarak Yerusalim'e gitti.
\par 12 Birlik ruhu vermek için Tanri'nin eli Yahuda'nin üzerindeydi. Öyle ki, Tanri kral ve önderlerin RAB'bin sözü uyarinca verdikleri buyruga halkin uymasini sagladi.
\par 13 Ikinci ay Mayasiz Ekmek Bayrami'ni* kutlamak için çok büyük bir topluluk Yerusalim'de toplandi.
\par 14 Ise Yerusalim'deki sunaklari kaldirmakla basladilar. Bütün buhur sunaklarini kaldirip Kidron Vadisi'ne attilar.
\par 15 Ikinci ayin on dördüncü günü Fisih* kurbanini kestiler. Kâhinlerle Levililer utanarak kendilerini kutsadilar, sonra RAB'bin Tapinagi'na yakmalik sunular* getirdiler.
\par 16 Bunun ardindan, Tanri adami Musa'nin Yasasi uyarinca, geleneksel yerlerini aldilar. Kâhinler Levililer'in elinden aldiklari kurban kanini sunagin üstüne döktüler.
\par 17 Topluluk arasinda kendini kutsamamis birçok kisi vardi; bu nedenle arinmamis olanlarin Fisih kurbanini kesme ve RAB'be adama görevini Levililer üstlendi.
\par 18 Çok sayida insan, Efrayim, Manasse, Issakar ve Zevulun'dan gelen halkin birçogu kendisini dinsel açidan arindirmamisti; öyleyken yazilana ters düserek Fisih kurbanini yediler. Hizkiya onlar için söyle dua etti: "Kutsal yerin kurallari uyarinca arinmamis bile olsa, RAB Tanri'ya, atalarinin Tanrisi'na yönelmeye yürekten kararli olan herkesi iyi olan RAB bagislasin."
\par 20 RAB Hizkiya'nin yakarisini duydu ve halki bagisladi.
\par 21 Yerusalim'deki Israilliler Mayasiz Ekmek Bayrami'ni yedi gün büyük sevinçle kutladilar. Levililer'le kâhinler RAB'bi yüceltmek amaciyla kullanilan yüksek sesli çalgilarla her gün O'nu övüyorlardi.
\par 22 Hizkiya RAB'be hizmetlerini basariyla yerine getiren Levililer'i övdü. Yedi gün boyunca herkes esenlik kurbanlarini kesip atalarinin Tanrisi RAB'be sükrederek bayramda kendisine ayrilan paydan yedi.
\par 23 Topluluk bayrami yedi gün daha kutlamaya karar verdi. Böylece herkes bayrami yedi gün daha sevinçle kutladi.
\par 24 Yahuda Krali Hizkiya topluluga bin bogayla yedi bin davar sagladi; önderler de topluluga bin bogayla on bin davar daha verdiler. Birçok kâhin kendini kutsadi.
\par 25 Bütün Yahuda toplulugu, kâhinler, Levililer, Israil'den gelen topluluk, Israil'den gelip Yahuda'ya yerlesen yabancilar sevindi.
\par 26 Yerusalim'de büyük sevinç vardi. Çünkü Israil Krali Davut oglu Süleyman'in günlerinden bu yana Yerusalim'de böylesi bir olay yasanmamisti.
\par 27 Levili kâhinler ayaga kalkip halki kutsadilar. Tanri onlari duydu. Çünkü dualari O'nun kutsal konutuna, göklere erismisti.

\chapter{31}

\par 1 Bütün bunlar sona erince, oradaki Israilliler Yahuda kentlerine giderek dikili taslari parçalayip Asera* putlarini devirdiler. Yahuda, Benyamin, Efrayim ve Manasse bölgelerindeki puta tapilan yerlerin ve sunaklarin hepsini yiktilar. Sonra herkes kendi kentine, topragina döndü.
\par 2 Hizkiya kâhinlerle Levililer'i görevlerine göre bölüklere ayirdi. Kimine yakmalik sunulari* ve esenlik sunularini* sunma, kimine hizmet etme, kimine de RAB'bin Konutu'nun kapilarinda sükredip övgüler söyleme görevini verdi.
\par 3 Kral da RAB'bin Yasasi uyarinca sabah aksam sunulmak üzere yakmalik sunular, Sabat* günleri, Yeni Ay ve bayramlarda sunulacak yakmalik sunular için sürüsünden hayvanlar verdi.
\par 4 Yerusalim'de yasayan halka da kâhinlerle Levililer'in paylarina düseni vermelerini buyurdu; öyle ki, RAB'bin Yasasi'na kendilerini adayabilsinler.
\par 5 Kralin bu buyrugunu duyar duymaz Israilliler ilk yetisen tahil, yeni sarap, zeytinyagi, bal ve bütün tarla ürünlerinden bol bol verdiler. Bunun yanisira her seyin ondaligini da bol bol getirdiler.
\par 6 Yahuda kentlerinde yasayan Israilliler'le Yahudalilar da sigirlarinin, davarlarinin, Tanrilari RAB'be adamis olduklari kutsal armaganlarin ondaligini getirip bir araya yigdilar.
\par 7 Ondaliklarin toplanmasi üçüncü aydan* yedinci aya kadar sürdü.
\par 8 Hizkiya ile önderler gelip yiginlari görünce, RAB'be övgüler sunup halki Israil'i kutsadilar.
\par 9 Hizkiya kâhinlerle Levililer'e yiginlari sorunca,
\par 10 Sadok soyundan Baskâhin Azarya su yaniti verdi: "Halk RAB'bin Tapinagi'na bagis getirmeye basladiktan bu yana yiyip doyduk; üstelik artirdik da. Çünkü RAB halkini kutsadi. Bu büyük yigin da artakalandir."
\par 11 Hizkiya RAB'bin Tapinagi'ndaki depolarin hazirlanmasi için buyruk verdi. Depolar hazirlandi.
\par 12 Bagislar, ondaliklar, adanan armaganlar sadakatle içeri getirildi. Bütün bu islerin sorumlusu olarak Levili Konanya atandi; kardesi Simi de yardimcisi oldu.
\par 13 Kral Hizkiya ile Tanri'nin Tapinagi'nin sorumlusu Azarya, Konanya ile kardesi Simi'nin yönetimindeki su denetçileri atadilar: Yehiel, Azazya, Nahat, Asahel, Yerimot, Yozavat, Eliel, Yismakya, Mahat, Benaya.
\par 14 Tapinagin Dogu Kapisi'nin nöbetçisi Yimna'nin oglu Levili Kore, Tanri'ya gönülden verilen sunularin sorumlusuydu. RAB'be adanan bagislari ve kutsal yiyecekleri dagitmak için bu göreve getirilmisti.
\par 15 Eden, Minyamin, Yesua, Semaya, Amarya ve Sekanya kâhinlerin yasadiklari kentlerde bölükleri uyarinca büyük küçük bütün kardeslerine bagislari dagitma isinde Kore'ye sadakatle yardim ettiler.
\par 16 Ayrica soyagacinda adi kayitli üç ve daha yukari yastaki erkeklere, bölüklerine ve sorumluluklarina göre RAB'bin Tapinagi'na girip günün gerektirdigi degisik görevleri yapanlarin tümüne; bagli olduklari boylara göre kütüge kayitli kâhinlere, bölüklerine ve sorumluluklarina göre yirmi ve daha yukari yastaki Levililer'e; kâhinlerle Levililer'in soyagacinda kayitli küçük çocuklara, kadinlara, ogullarla kizlara, yani topluluga yiyecek dagitiyorlardi. Çünkü kendilerini sadakatle Tanri'ya adamislardi.
\par 19 Kentlerinin çevresindeki kirlarda ya da herhangi bir kentte yasayan Harun soyundan kâhinlere gelince, onlardan olan bütün erkeklere ve Levililer'in soyagacinda kayitli herkese yiyecek dagitmak için belirli kisiler atanmisti.
\par 20 Hizkiya bütün Yahuda'da böyle davrandi; Tanrisi RAB'bin önünde iyi, dogru ve hakça olani yapti.
\par 21 Tanri'nin Tapinagi'nda üstlendigi görevi bütün yüregiyle yerine getirdi; Kutsal Yasa'ya, buyruklara uydu; Tanrisi'na yöneldi. Bu sayede basarili oldu.

\chapter{32}

\par 1 Hizkiya'nin RAB'be baglilikla yaptigi bu hizmetlerden sonra Asur Krali Sanherib gelip Yahuda'ya saldirdi. Surlu kentleri ele geçirmek amaciyla onlari kusatti.
\par 2 Sanherib'in Yerusalim'le savasmaya hazirlandigini duyan Hizkiya,
\par 3 kentin disindaki pinarlari kapatma konusunda önderlerine ve ordu komutanlarina danisti. Onlar da onu desteklediler.
\par 4 Böylece birçok kisi toplandi. "Neden Asur Krali gelip bol su bulsun?" diyerek bütün pinarlari ve ülkenin ortasindan akan dereyi kapadilar.
\par 5 Hizkiya gücünü toplayarak surun yikilmis bölümlerini onartti, üstüne kuleler yaptirdi. Disardan bir sur daha yaptirarak Davut Kenti'nde Millo'yu* saglamlastirdi. Bunlarin yanisira, çok sayida silah ve kalkan hazirlatti.
\par 6 Halka komutanlar atadi. Sonra onlari kent kapisi yakinindaki alanda toplayarak söyle yüreklendirdi:
\par 7 "Güçlü ve yürekli olun! Asur Krali'ndan ve yanindaki büyük ordudan korkmayin, yilmayin. Çünkü bizimle olan onunla olandan daha üstündür.
\par 8 Ondaki güç insansaldir; bizdeki güç ise bize yardim eden ve bizden yana savasan Tanrimiz RAB'dir." Yahuda Krali Hizkiya'nin bu sözleri halka güven verdi.
\par 9 Asur Krali Sanherib, bütün ordusuyla Lakis Kenti'ni kusatirken, subaylari araciligiyla Yahuda Krali Hizkiya'ya ve Yerusalim'de yasayan Yahuda halkina su haberi gönderdi:
\par 10 "Asur Krali Sanherib söyle diyor: 'Neye güvenerek Yerusalim'de kusatma altinda kaliyorsunuz?
\par 11 Hizkiya, 'Tanrimiz RAB bizi Asur Krali'nin elinden kurtaracak diyerek sizi kandiriyor. Sizi kitliga ve susuzluga terk edip ölüme sürüklüyor.
\par 12 Tanri'nin tapinma yerlerini, sunaklarini ortadan kaldiran, Yahuda ve Yerusalim halkina, 'Tek bir sunagin önünde tapinacak, onun üzerinde buhur yakacaksiniz diyen Hizkiya degil mi bu?
\par 13 Benim ve atalarimin öbür ülkelerin halklarina neler yaptigimizi bilmiyor musunuz? Uluslarin ilahlarindan hangisi ülkesini benim elimden kurtarabildi?
\par 14 Atalarimin büsbütün yok ettigi bu uluslarin ilahlarindan hangisi halkini elimden kurtarabildi ki, Tanriniz sizi elimden kurtarabilsin?
\par 15 Hizkiya'nin sizi kandirip aldatmasina izin vermeyin! Ona güvenmeyin! Çünkü hiçbir ulusun, kralligin ilahlarindan bir teki bile halkini elimden ya da atalarimin elinden kurtaramamistir! Tanriniz kim ki sizi elimden kurtarsin?"
\par 16 Sanherib'in subaylari RAB Tanri'ya ve kulu Hizkiya'ya karsi daha birçok sey söylediler.
\par 17 Sanherib Israil'in Tanrisi RAB'bi asagilamak için yazdigi mektuplarda da söyle diyordu: "Öteki uluslarin tanrilari halklarini elimden kurtaramadiklari gibi, Hizkiya'nin Tanrisi da halkini elimden kurtaramayacak."
\par 18 Sonra kenti ele geçirmek amaciyla surun üstündeki Yerusalim halkini korkutup yildirmak için Yahudi dilinde bagirdilar.
\par 19 Üstelik dünyanin öteki uluslarinin insan eliyle yapilmis ilahlarindan söz edercesine Yerusalim'in Tanrisi'ndan söz ettiler.
\par 20 Bunun üzerine Kral Hizkiya ile Amots oglu Peygamber Yesaya dua edip Tanri'ya yalvardilar.
\par 21 RAB bir melek göndererek Asur Krali'nin ordugahindaki bütün yigit savasçilari, önderleri, komutanlari yok etti. Asur Krali utanç içinde ülkesine döndü. Bir gün kendi ilahinin tapinagina girdiginde de ogullarindan bazilari onu orada kiliçla öldürdüler.
\par 22 Böylece RAB Hizkiya'yla Yerusalim'de yasayanlari Asur Krali Sanherib'in ve öbür düsmanlarinin elinden kurtararak her yanda güvenlik içinde yasamalarini sagladi.
\par 23 Yerusalim'e gelen birçok kisi RAB'be sunular, Yahuda Krali Hizkiya'ya da degerli armaganlar getirdi. O günden sonra Hizkiya bütün uluslar arasinda sayginlik kazandi.
\par 24 O günlerde Hizkiya ölümcül bir hastaliga yakalaninca, RAB'be yalvardi. RAB yakarisini duyarak ona bir belirti verdi.
\par 25 Ne var ki, Hizkiya kendisine yapilan bu iyilige yarasir biçimde davranmayip büyüklendi. Bu yüzden RAB hem ona, hem Yahuda'ya, hem de Yerusalim'e öfkelendi.
\par 26 Hizkiya ile Yerusalim'de yasayanlar gururu birakip alçakgönüllü davranmaya basladilar. Bu sayede Hizkiya'nin kralligi boyunca RAB'bin öfkesine ugramadilar.
\par 27 Hizkiya çok zengin ve onurlu biriydi. Altini, gümüsü, degerli taslari, baharati, kalkanlari ve çesit çesit degerli esyasi için hazineler yaptirdi.
\par 28 Ayrica tahil ambarlari, yeni sarap ve zeytinyagi depolari, sigirlar için ahirlar, sürüler için agillar yaptirdi.
\par 29 Bunlarin yanisira kendisi için kentler kurdurdu ve çok sayida davar, sigir edindi. Tanri ona çok büyük zenginlik vermisti.
\par 30 Gihon Pinari'nin yukari agzini kapayip suyun Davut Kenti'nin bati yakasindan asagiya akmasini saglayan da Hizkiya'dir. Üstlendigi her iste basarili oldu.
\par 31 Ama Babil önderlerinin ülkede gerçeklestirilen belirtiyi arastirmak için gönderdigi elçiler gelince, Tanri Hizkiya'yi denemek ve aklindan geçenlerin hepsini ögrenmek için onu birakti.
\par 32 Hizkiya'nin yaptigi öbür isler ve RAB'be bagliligi Amots oglu Peygamber Yesaya'nin Yahuda ve Israil krallarinin tarihindeki görümünde yazilidir.
\par 33 Hizkiya ölüp atalarina kavusunca, Davutogullari'nin mezarlarinin üst kesimine gömüldü. Bütün Yahuda ve Yerusalim halki onu saygiyla andi. Yerine oglu Manasse kral oldu.

\chapter{33}

\par 1 Manasse on iki yasinda kral oldu ve Yerusalim'de elli bes yil krallik yapti.
\par 2 RAB'bin Israil halkinin önünden kovmus oldugu uluslarin igrenç törelerine uyarak RAB'bin gözünde kötü olani yapti.
\par 3 Babasi Hizkiya'nin ortadan kaldirdigi puta tapilan yerleri yeniden yaptirdi. Baallar* için sunaklar kurdu, Asera* putlari yapti. Gök cisimlerine taparak onlara kulluk etti.
\par 4 RAB'bin, "Adim sonsuza dek Yerusalim'de bulunacaktir" dedigi RAB'bin Tapinagi'nda sunaklar kurdu.
\par 5 Tapinagin iki avlusunda gök cisimlerine tapmak için sunaklar yaptirdi.
\par 6 Ogullarini Ben-Hinnom Vadisi'nde ateste kurban etti; falcilik ve büyücülük yapti. Medyumlara, ruh çagiranlara danisti. RAB'bin gözünde çok kötülük yaparak O'nu öfkelendirdi.
\par 7 Manasse yaptirdigi putu Tanri'nin Tapinagi'na yerlestirdi. Oysa Tanri tapinaga iliskin Davut'la oglu Süleyman'a söyle demisti: "Bu tapinakta ve Israil oymaklarinin yasadigi kentler arasindan seçtigim Yerusalim'de adim sonsuza dek anilacak.
\par 8 Kendilerine Musa araciligiyla buyurdugum Kutsal Yasa'yi, kurallari, ilkeleri dikkatle yerine getirirlerse, Israil halkinin ayagini bir daha atalarina vermis oldugum ülkenin disina çikarmayacagim."
\par 9 Ancak Manasse Yahudalilar'la Yerusalim'de yasayanlari öylesine yoldan çikardi ki, RAB'bin Israil halkinin önünde yok ettigi uluslardan daha çok kötülük yaptilar.
\par 10 RAB Manasse'yle halkini uyardiysa da aldiris etmediler.
\par 11 Bunun üzerine RAB Asur Krali'nin ordu komutanlarini onlarin üzerine gönderdi. Manasse'yi tutsak alip burnuna çengel taktilar; tunç* zincirlerle baglayip Babil'e götürdüler.
\par 12 Ne var ki, Manasse sikintisinda Tanrisi RAB'be yakardi ve atalarinin Tanrisi önünde son derece alçakgönüllü davrandi.
\par 13 RAB'be yalvarinca RAB yakarisini, duasini duydu ve onu yine Yerusalim'e, kralligina getirdi. Iste o zaman Manasse RAB'bin Tanri oldugunu anladi.
\par 14 Sonra Davut Kenti için vadideki Gihon Pinari'nin batisindan Balik Kapisi'nin girisine kadar yüksek bir dis sur yapti; Ofel Tepesi'ni de bu surla çevirdi. Yahuda'nin bütün surlu kentlerine komutanlar yerlestirdi.
\par 15 RAB'bin Tapinagi'ndan yabanci ilahlari ve diktirdigi putu çikardi; tapinagin bulundugu tepede ve Yerusalim'de yaptirdigi bütün sunaklari kaldirip kentin disina atti.
\par 16 RAB'bin sunagini yeniden kurup üzerinde esenlik ve sükran kurbanlari kesti; Yahuda halkina Israil'in Tanrisi RAB'be kulluk etmeleri için buyruk verdi.
\par 17 Ne var ki, halk tapinma yerlerinde kurban sunmayi sürdürdü; ancak yalniz Tanrilari RAB'be kurban sunuyorlardi.
\par 18 Manasse'nin yaptigi öbür isler, Tanrisi'na yakarisi ve Israil'in Tanrisi RAB'bin adina onu uyaran bilicilerin* sözleri, Israil krallarinin tarihinde yazilidir.
\par 19 Duasi, Tanri'nin ona yaniti, gururundan dönmeden önce isledigi bütün günahlar, ihaneti, yaptirdigi tapinma yerleri, Asera putlariyla öbür putlari diktirdigi yerler Hozay'in tarihinde yazilidir.
\par 20 Manasse ölüp atalarina kavusunca, kendi sarayina gömüldü. Yerine oglu Amon kral oldu.
\par 21 Amon yirmi iki yasinda kral oldu ve Yerusalim'de iki yil krallik yapti.
\par 22 Amon da babasi Manasse gibi RAB'bin gözünde kötü olani yapti. Babasi Manasse'nin yaptirdigi putlara kurban sundu, onlara tapti.
\par 23 Ancak RAB'bin önünde alçakgönüllülügü takinan babasi Manasse'nin tersine, giderek suçunu artirdi.
\par 24 Görevlileri düzen kurup onu sarayinda öldürdüler.
\par 25 Ülke halki Kral Amon'a düzen kuranlarin hepsini öldürdü. Yerine oglu Yosiya'yi kral yapti.

\chapter{34}

\par 1 Yosiya sekiz yasinda kral oldu ve Yerusalim'de otuz bir yil krallik yapti.
\par 2 RAB'bin gözünde dogru olani yapti. Saga sola sapmadan atasi Davut'un yollarini izledi.
\par 3 Yosiya kralliginin sekizinci yilinda, daha gençken, atasi Davut'un Tanrisi'na yönelmeye basladi. Kralliginin on ikinci yilinda da Yahuda ve Yerusalim'i puta tapilan yerlerden, Asera* putlarindan, oyma ve dökme putlardan arindirmaya basladi.
\par 4 Yosiya'nin gözetiminde Baallar'in* sunaklarini yiktirip üzerlerindeki buhur sunaklarini parçalatti. Asera putlarini, oyma ve dökme putlari da parçalayip ezdikten sonra tozlarini onlara kurban sunanlarin mezarlarina serpti.
\par 5 Kâhinlerin kemiklerini kendi sunaklarinin üstünde yakti. Böylece Yahuda ve Yerusalim'i arindirdi.
\par 6 Naftali'ye dek Manasse, Efrayim ve Simon oymaklarinin kentleriyle çevredeki yikintilarda bulunan
\par 7 sunaklari, Asera putlarini yikti; toz haline gelinceye dek putlari ezdi. Israil ülkesinin her yanindaki buhur sunaklarini paramparça etti. Sonra Yerusalim'e döndü.
\par 8 Yosiya ülkeyi ve tapinagi arindirdiktan sonra, kralliginin on sekizinci yilinda Asalya oglu Safan'i, kent yöneticisi Maaseya'yi ve Yoahaz oglu devlet tarihçisi Yoah'i Tanrisi RAB'bin Tapinagi'ni onarmaya gönderdi.
\par 9 Bunlar Tanri'nin Tapinagi'na getirilen paralari götürüp Baskâhin Hilkiya'ya verdiler. Kapi nöbetçileri olan Levililer bu parayi Manasse ve Efrayim halkindan, Israil'in geri kalanindan, bütün Yahudalilar'la Benyaminliler'den, Yerusalim'de yasayanlardan toplamislardi.
\par 10 Paralari RAB'bin Tapinagi'ndaki islerin basinda bulunan denetçilere verdiler. Onlar da tapinagi yenileme ve onarma isinde çalisan isçilere ödediler.
\par 11 Yontma tas, Yahuda krallarinin yikilmaya terk ettigi yapilarin kiris ve baglanti yerlerinin onarimi için kereste almalari için marangozlara, yapicilara ödeme yapildi.
\par 12 Çalisanlar isi özenle yaptilar. Baslarinda yönetici olarak Levililer'den su denetçiler vardi: Merari boyundan Yahat'la Ovadya, Kehat boyundan Zekeriya'yla Mesullam. Çalgi çalmakta usta olan Levililer yük tasiyan isçilerin sorumlulugunu aldilar ve her isi yapan isçileri denetlediler. Levililer'den bazilari da yazman, görevli, kapi nöbetçisi olarak çalistilar.
\par 14 RAB'bin Tapinagi'na getirilen parayi çikarirlarken, Kâhin Hilkiya Musa araciligiyla verilmis olan RAB'bin Yasa Kitabi'ni buldu.
\par 15 Yazman Safan'a, "RAB'bin Tapinagi'nda Yasa Kitabi'ni buldum" diyerek kitabi ona verdi.
\par 16 Safan kitabi krala götürerek, "Görevlilerin kendilerine verilen her isi yapiyorlar" diye haber verdi,
\par 17 "RAB'bin Tapinagi'ndaki paralari alip denetçilerle isçilere verdiler."
\par 18 Ardindan, "Kâhin Hilkiya bana bir kitap verdi" diyerek kitabi krala okudu.
\par 19 Kral Kutsal Yasa'daki sözleri duyunca giysilerini yirtti.
\par 20 Hilkiya'ya, Safan oglu Ahikam'a, Mika oglu Avdon'a, Yazman Safan'a ve kendi özel görevlisi Asaya'ya söyle buyurdu:
\par 21 "Gidin, bulunan bu kitabin sözleri hakkinda benim için de, Israil ve Yahuda halkinin geri kalani için de RAB'be danisin. RAB'bin bize karsi alevlenen öfkesi büyüktür. Çünkü atalarimiz RAB'bin sözüne kulak asmadilar, bu kitapta yazilanlara uymadilar."
\par 22 Hilkiya ile kralin gönderdigi adamlar varip tapinaktaki giysilerin nöbetçisi Hasra oglu Tokhat oglu Sallum'un karisi Peygamber Hulda'ya danistilar. Hulda Yerusalim'de, Ikinci Mahalle'de oturuyordu.
\par 23 Hulda onlara söyle dedi: "Israil'in Tanrisi RAB, 'Sizi bana gönderen adama sunlari söyleyin diyor:
\par 24 'Yahuda Krali'nin önünde okunan kitapta yazili bütün lanetleri, felaketi buraya da, burada yasayan halkin basina da getirecegim.
\par 25 Beni terk ettikleri, elleriyle yaptiklari baska ilahlara buhur yakip beni kizdirdiklari için buraya karsi öfkem alevlenecek ve sönmeyecek.
\par 26 "RAB'be danismak için sizi gönderen Yahuda Krali'na söyle deyin: 'Israil'in Tanrisi RAB duydugun sözlere iliskin diyor ki:
\par 27 Madem burasi ve burada yasayanlarla ilgili sözlerimi duyunca yüregin yumusadi, kendini alçalttin, evet, önümde kendini alçalttin, giysilerini yirtip huzurumda agladin, ben de yalvarisini isittim. Böyle diyor RAB.
\par 28 Seni atalarina kavusturacagim, esenlik içinde mezarina gömüleceksin. Buraya ve burada yasayanlara getirecegim büyük felaketi görmeyeceksin." Hilkiya ile yanindakiler bu sözleri krala ilettiler.
\par 29 Kral Yosiya haber gönderip Yahuda ve Yerusalim'in bütün ileri gelenlerini topladi.
\par 30 Sonra Yahudalilar, Yerusalim'de yasayanlar, kâhinler, Levililer, büyük küçük herkesle birlikte RAB'bin Tapinagi'na çikti. RAB'bin Tapinagi'nda bulunmus olan Antlasma Kitabi'ni bastan sona kadar herkesin duyacagi biçimde okudu.
\par 31 Özel yerinde durarak RAB'bin yolunu izleyecegine, buyruklarini, ögütlerini, kurallarini candan ve yürekten uygulayacagina, bu kitapta yazili antlasmanin kosullarini yerine getirecegine iliskin RAB'bin huzurunda antlasma yapti.
\par 32 Sonra oradaki Yerusalim ve Benyamin halkina bu antlasmaya bagli kalacaklarina iliskin ant içirtti. Yerusalim'de yasayanlar Tanri'nin, atalarinin Tanrisi'nin antlasmasina bagli kaldilar.
\par 33 Yosiya Israil topraklarindan bütün igrenç putlari kaldirtti. Israil'de kalan halkin Tanrilari RAB'be kulluk etmelerini sagladi. Kral yasadigi sürece halk atalarinin Tanrisi RAB'bin ardinca yürümekten vazgeçmedi.

\chapter{35}

\par 1 Yosiya Yerusalim'de RAB için Fisih Bayrami'ni* kutladi. Birinci ayin* on dördüncü günü Fisih* kurbani kesildi.
\par 2 Yosiya kâhinleri görevlerine atayarak RAB'bin Tapinagi'nda hizmet etmeleri için yüreklendirdi.
\par 3 Bütün Israil halkini egiten RAB'be adanmis Levililer'e, "Kutsal sandigi Israil Krali Davut oglu Süleyman'in yaptirdigi tapinaga koyun" dedi, "Bundan böyle onu omuzlarinizin üzerinde tasimayacaksiniz. Simdi Tanriniz RAB'be ve halki Israil'e hizmet edin.
\par 4 Israil Krali Davut'la oglu Süleyman'in yazili düzeni uyarinca, boylariniza, bölüklerinize göre hazirlanin.
\par 5 Kardesleriniz olan halkin boylarina bagli bölüklere yardim etmek üzere siz Levililer takimlar halinde kutsal yerde durun.
\par 6 Kendinizi kutsayin ve RAB'bin Musa araciligiyla buyurdugu gibi Fisih kurbanlarini kesip kardesleriniz için hazirlayin."
\par 7 Yosiya Fisih kurbanini sunmalari için oradaki halka sürüsünden otuz bin kuzuyla oglak, üç bin de sigir bagisladi.
\par 8 Kralin önderleri de halka, kâhinlere ve Levililer'e gönülden bagista bulundular. Tanri Tapinagi'nin yöneticileri olan Hilkiya, Zekeriya, Yehiel de Fisih kurbani olarak kâhinlere iki bin alti yüz kuzuyla oglak, üç yüz sigir verdiler.
\par 9 Konanya, kardesleri Semaya ile Netanel, Levililer'in önderleri Hasavya, Yeiel ve Yozavat da Fisih kurbani olarak Levililer'e bes bin kuzuyla oglak, bes yüz de sigir bagisladilar.
\par 10 Hizmetle ilgili hazirliklar tamamlaninca, kralin buyrugu uyarinca kâhinlerle bölüklerine göre Levililer yerlerini aldilar.
\par 11 Fisih kurbanlari kesildi. Kâhinler kendilerine verilen kani sunagin üzerine döktüler; Levililer de hayvanlarin derisini yüzdüler.
\par 12 Musa'nin kitabinda yazilanlar uyarinca, RAB'be sunsunlar diye yakmalik sunular* halk boylarinin bölüklerine verilmek üzere bir yana koyuldu. Sigirlara da aynisini yaptilar.
\par 13 Kural uyarinca, Fisih kurbanlarini ateste kizarttilar; kutsal sunulari da tencerelerde, kazanlarda, tavalarda haslayip çabucak halka dagittilar.
\par 14 Bundan sonra Levililer hem kendileri, hem de kâhinler adina hazirlik yaptilar. Çünkü Harun soyundan kâhinler aksam geç vakte kadar yakmalik sunu ve yag sunmakla ugrasiyorlardi. Bu nedenle Levililer hem kendileri, hem de Harun soyundan gelen kâhinler için hazirlik yaptilar.
\par 15 Asaf soyundan gelen ezgiciler Davut, Asaf, Heman ve kralin bilicisi* Yedutun'un buyrugu uyarinca yerlerinde durdular. Kapi nöbetçileri de görevlerini birakmak zorunda kalmadi, çünkü onlar için hazirligi kardesleri Levililer yapmisti.
\par 16 Böylece o gün Kral Yosiya'nin buyrugu dogrultusunda Fisih Bayrami'ni kutlamak ve RAB'bin sunagi üzerinde yakmalik sunular sunmak için RAB'bin hizmetiyle ilgili bütün çalismalar tamamlandi.
\par 17 O gün orada bulunan Israil halki Fisih Bayrami'ni kutladi. Mayasiz Ekmek Bayrami'ni* da yedi gün boyunca kutladilar.
\par 18 Peygamber Samuel'in döneminden bu yana, Israil'de böyle bir Fisih Bayrami kutlanmamisti. Hiçbir Israil krali da Yosiya'nin, kâhinlerin, Levililer'in, bütün Yahuda halkiyla oradaki Israilliler'in ve Yerusalim'de yasayanlarin kutladigi gibi bir Fisih Bayrami kutlamamisti.
\par 19 Bu Fisih Bayrami Yosiya'nin kralliginin on sekizinci yilinda kutlandi.
\par 20 Yosiya'nin tapinagi düzenlemesinden sonra, Misir Krali Neko savasmak üzere Firat kiyisindaki Karkamis'a yürüdü. Yosiya da onunla savasmak için yola çikti.
\par 21 Ama Neko ulaklar araciligiyla su haberi gönderdi: "Benimle senin aranda bir anlasmazlik yok, ey Yahuda Krali! Bugün sana degil, savas açtigim ülkeye karsi savasmaya geldim. Tanri ivedi davranmami buyurdu. Benden yana olan Tanri'dan sakin. Yoksa seni yok eder!"
\par 22 Ne var ki, Yosiya onunla savasmaktan vazgeçmedigi gibi, Tanri'nin Neko araciligiyla söyledigi sözlere de aldiris etmedi. Kilik degistirip Megiddo Ovasi'nda Neko ile savasmak üzere yola çikti.
\par 23 Okçular Kral Yosiya'yi vurunca, kral görevlilerine, "Beni buradan götürün, agir yaraliyim!" dedi.
\par 24 Görevlileri onu savas arabasindan çikarip kendisine ait baska bir arabaya koyarak Yerusalim'e götürdüler. Yosiya öldü ve atalarinin mezarligina gömüldü. Bütün Yahuda ve Yerusalim halki onun için yas tuttu.
\par 25 Yeremya Yosiya için bir agit yazdi. Kadin, erkek bütün ozanlar bugüne dek agitlarinda Yosiya'yi anarlar. Israil'de bir gelenek haline gelen bu agitlar Agitlar Kitabi'nda yazilidir.
\par 26 Yosiya'nin yaptigi öbür isler, RAB'bin Yasasi'nda yazilanlara uygun bagliligi,
\par 27 uygulamalari, basindan sonuna dek Israil ve Yahuda krallarinin tarihinde yazilidir.

\chapter{36}

\par 1 Yahuda halki babasi Yosiya'nin yerine oglu Yehoahaz'i Yerusalim'e kral yapti.
\par 2 Yehoahaz yirmi üç yasinda kral oldu ve Yerusalim'de üç ay krallik yapti.
\par 3 Misir Krali Neko Yerusalim'de onu tahttan indirerek ülke halkini yüz talant gümüs ve bir talant altin ödemekle yükümlü kildi. Yehoahaz'in agabeyi Elyakim'i de Yahuda'yla Yerusalim Krali yapti ve adini degistirip Yehoyakim koydu. Sonra kardesi Yehoahaz'i alip Misir'a döndü.
\par 5 Yehoyakim yirmi bes yasinda kral oldu ve Yerusalim'de on bir yil krallik yapti. Tanrisi RAB'bin gözünde kötü olani yapti.
\par 6 Yehoyakim'e saldiran Babil Krali Nebukadnessar Babil'e götürmek için onu tunç* zincirlerle bagladi.
\par 7 RAB'bin Tapinagi'ndaki bazi esyalari da alip Babil'de kendi tapinagina yerlestirdi.
\par 8 Yehoyakim'in yaptigi öbür isler, igrençlikleri, onunla ilgili açiga çikan kötülükler Israil ve Yahuda krallarinin tarihinde yazilidir. Yerine oglu Yehoyakin kral oldu.
\par 9 Yehoyakin on sekiz yasinda kral oldu ve Yerusalim'de üç ay on gün krallik yapti. O da RAB'bin gözünde kötü olani yapti.
\par 10 Ilkbaharda Kral Nebukadnessar onu ve RAB'bin Tapinagi'ndaki bazi degerli esyalari Babil'e getirtti. Yehoyakin'in yerine akrabasi Sidkiya'yi Yahuda ve Yerusalim Krali yapti.
\par 11 Sidkiya yirmi bir yasinda kral oldu ve Yerusalim'de on bir yil krallik yapti.
\par 12 Tanrisi RAB'bin gözünde kötü olani yapti. RAB'bin sözünü bildiren Peygamber Yeremya'nin karsisinda alçakgönüllü davranmadi.
\par 13 Sidkiya Tanri adiyla kendisine bagli kalacagina ant içiren Kral Nebukadnessar'a karsi ayaklandi. Israil'in Tanrisi RAB'be dönmemek için direnerek inat etti.
\par 14 Üstelik kâhinlerin ve halkin önderleri de öteki uluslarin igrenç törelerine uyarak ihanetlerini gitgide artirdilar ve RAB'bin Yerusalim'de kutsal kildigi tapinagini kirlettiler.
\par 15 Atalarinin Tanrisi RAB, halkina ve konutuna acidigi için onlari ulaklari araciligiyla defalarca uyardi.
\par 16 Ama onlar Tanri'nin ulaklariyla alay ederek sözlerini küçümsediler, peygamberlerini asagiladilar. Sonunda RAB'bin halkina karsi öfkesi kurtulus yolu birakmayacak kadar alevlendi.
\par 17 RAB Kildan* Krali'ni onlarin üzerine saldirtti. Kildani ordusu gençlerini tapinakta kiliçtan geçirdi. Ne delikanliya, ne genç kiza, ne yasliya, ne aksaçliya acidi. RAB hepsini Kildan Krali'nin eline teslim etti.
\par 18 Kral, RAB Tanri'nin Tapinagi'ndaki büyük küçük bütün esyalari, tapinagin, Yahuda Krali'nin ve önderlerinin hazinelerini Babil'e tasitti.
\par 19 Tanri'nin Tapinagi'ni atese verdiler, Yerusalim surlarini yikip bütün saraylari yaktilar, degerli olan herseyi yok ettiler.
\par 20 Kildan Krali kiliçtan kurtulanlari Babil'e sürdü. Bunlar Pers kralligi egemen oluncaya dek onun ve ogullarinin köleleri olarak yasadilar.
\par 21 Böylece RAB'bin Yeremya araciligiyla söyledigi söz yerine geldi: "Ülke tutulmayan Sabat yillarini tamamlayincaya, yetmis yil doluncaya kadar issiz kalip dinlenecek."
\par 22 Pers Krali Kores'in kralliginin birinci yilinda RAB, Yeremya araciligiyla bildirdigi sözü yerine getirmek amaciyla, Pers Krali Kores'i harekete geçirdi. Kores yönetimi altindaki bütün halklara su yazili bildiriyi duyurdu:
\par 23 "Pers Krali Kores söyle diyor: 'Göklerin Tanrisi RAB yeryüzünün bütün kralliklarini bana verdi. Beni Yahuda'daki Yerusalim Kenti'nde kendisi için bir tapinak yapmakla görevlendirdi. Aranizda O'nun halkindan kim varsa oraya gitsin. Tanrisi RAB onunla olsun!"


\end{document}