\begin{document}

\title{Nehemya}


\chapter{1}

\par 1 Hakalya oglu Nehemya'nin anlattiklari: Pers Krali Artahsasta'nin kralliginin yirminci yili, Kislev ayinda* Sus Kalesi'ndeydim.
\par 2 Kardeslerimden Hanani ve bazi Yahudalilar yanima geldi. Onlara sürgünden kurtulup sag kalan Yahudiler'i ve Yerusalim'in durumunu sordum.
\par 3 "Sürgünden kurtulup Yahuda Ili'ne dönenler büyük sikinti ve utanç içinde" diye karsilik verdiler, "Üstelik Yerusalim surlari yikilmis, kapilari yakilmis."
\par 4 Bunlari duyunca oturup agladim, günlerce yas tuttum. Oruç* tutup Göklerin Tanrisi'na dua ettim:
\par 5 "Ey Göklerin Tanrisi RAB! Yüce ve görkemli Tanri! Seni sevenlerle, buyruklarina uyanlarla yaptigin antlasmaya bagli kalirsin.
\par 6 Ya RAB, halimi gör, gece gündüz kullarin Israil halki için ettigim duaya kulak ver. Itiraf ediyorum, Israil halki günah isledi, ben ve atalarim günah isledik.
\par 7 Sana çok kötülük yaptik. Kulun Musa'ya verdigin buyruklara, kurallara, ilkelere uymadik.
\par 8 "Kulun Musa'ya söylediklerini animsa. Dedin ki, `Eger bana ihanet ederseniz, sizi uluslarin arasina dagitacagim.
\par 9 Ama bana döner, buyruklarimi özenle yerine getirirseniz, dünyanin öbür ucuna sürülmüs olsaniz bile sizleri toplayip seçtigim yere, bulunacagim yere getirecegim.'
\par 10 "Onlar senin kullarin, kendi halkindir. Yüce kudretin ve güçlü elinle onlari kurtardin.
\par 11 Ya Rab, bu kulunun, adini yüceltmekten sevinç duyan öbür kullarinin dualarina kulak ver. Beni bugün basarili kil ve kralin önerimi kabul etmesini sagla." O günlerde kralin sakisiydim.

\chapter{2}

\par 1 Kral Artahsasta'nin kralliginin yirminci yili, Nisan ayiydi. Krala getirilen sarabi alip kendisine sundum. O güne kadar beni hiç üzgün görmemisti.
\par 2 Bu yüzden, "Neden böyle üzgün görünüyorsun?" diye sordu, "Hasta olmadigina göre, bir derdin olmali." Çok korktum.
\par 3 Krala, "Tanri sana uzun ömürler versin" dedim, "Atalarimin gömüldügü kent yikildi, kapilari yakildi. Nasil üzülmem?"
\par 4 Kral, "Dilegin ne?" diye sordu. Göklerin Tanrisi'na dua edip krala söyle dedim: "Eger uygun görüyorsan, benden hosnut kaldinsa, lütfen beni Yahuda'ya, atalarimin gömüldügü kente gönder; kenti onarayim."
\par 6 Kral kraliçeyle birlikte oturuyordu. "Yolculugun ne kadar sürer?" diye sordu, "Ne zaman dönersin?" Böylece kral dilegimi uygun buldu ve beni göndermeyi kabul etti. Ona ne zaman dönecegimi söyledim.
\par 7 Sonra söyle dedim: "Uygun görüyorsan, Yahuda'ya varmami saglamalari için, Firat'in bati yakasindaki valilere birer mektup yazilsin.
\par 8 Bir de kralin orman sorumlusu Asaf'a bir mektup götürmek istiyorum. Tapinagin yanindaki kalenin kapilari, kent surlari ve oturacagim evin yapimi için bana kereste versin." Tanrim bana destek oldugu için kral dileklerimi yerine getirdi.
\par 9 Firat'in bati yakasindaki valilere gidip kralin mektuplarini verdim. Kral benimle birlikte komutanlar ve atlilar göndermisti.
\par 10 Horonlu Sanballat ile Ammonlu görevlilerden Toviya, Israil halkinin iyiligi için birinin çalismaya geldigini duyunca çok sikildilar.
\par 11 Yerusalim'e gittim. Orada üç gün kaldiktan sonra,
\par 12 gece kalkip birkaç adamla birlikte ise koyuldum. Yerusalim için yapacaklarima iliskin Tanri'dan aldigim esini kimseye açiklamadim. Bindigim hayvandan baska hayvan götürmemistim.
\par 13 Hâlâ karanlikti. Dere Kapisi'ndan Ejder Pinari'na, oradan Gübre Kapisi'na gittim. Yerusalim'in yikilan surlarini, yanip kül olan kapilarini gözden geçirdim.
\par 14 Sonra Pinar Kapisi'na, Kral Havuzu'na dogru gittim. Ne var ki, yol bindigim hayvanin geçmesine uygun degildi.
\par 15 Gece karanliginda dere boyunca ilerledim, surlari gözden geçirip geri geldim. Sonunda Dere Kapisi'ndan girip yerime döndüm.
\par 16 Yetkililer nereye gittigimi, ne yaptigimi bilmiyorlardi. Çünkü Yahudiler'e, kâhinlere*, soylulara, yetkililere ve öteki görevlilere henüz hiçbir sey söylememistim.
\par 17 Sonra onlara, "Içine düstügümüz yikimi görüyorsunuz" dedim, "Yerusalim yikilmis, kapilari atese verilmis. Gelin, Yerusalim surlarini onaralim, utancimiza son verelim."
\par 18 Onlara Tanri'nin bana nasil destek oldugunu ve kralin söylediklerini anlattim. Onlar da, "Haydi, onarmaya baslayalim" dediler. Var güçleriyle bu hayirli ise basladilar.
\par 19 Ama Horonlu Sanballat, Ammonlu görevlilerden Toviya, Arap Gesem yapacaklarimizi duyunca, bizi küçümseyip alay ettiler. "Ne yapiyorsunuz? Krala bas mi kaldiriyorsunuz?" dediler.
\par 20 Onlari söyle yanitladim: "Göklerin Tanrisi bizi basarili kilacaktir. Biz O'nun kullari olarak onarimi baslatacagiz. Ama sizin Yerusalim üzerinde ne hakkiniz, ne de payiniz olacak, adiniz bile anilmayacak."

\chapter{3}

\par 1 Baskâhin Elyasiv ile öbür kâhinler ise koyulup Koyun Kapisi'ni onardilar, kapi kanatlarini kutsayip yerine taktilar. Hammea Kulesi'ne ve Hananel Kulesi'ne kadar surlari kutsadilar.
\par 2 Bitisik bölümü Erihalilar, onun yanindakini de Imri oglu Zakkur onardi.
\par 3 Balik Kapisi'ni Senaalilar onardi. Kirisleri yerlestirip kapi kanatlarini yerine koydular, sürgülerle kapi kollarini taktilar.
\par 4 Bitisik bölümü Hakkos oglu Uriya oglu Meremot onardi. Onun yanindakini Mesezavel oglu Berekya oglu Mesullam onardi. Onun yanindakini Baana oglu Sadok onardi.
\par 5 Onun yanindakini Tekoalilar onardi; ama soylular efendilerinin buyurdugu islere el atmadilar.
\par 6 Eski Kapi'yi Paseah oglu Yoyada ile Besodya oglu Mesullam onardi. Kirisleri yerlestirip kapi kanatlarini yerine koydular, sürgülerle kapi kollarini taktilar.
\par 7 Bitisik bölümü Givonlu Melatya, Meronotlu Yadon ve Firat'in bati yakasindaki bölge valisinin yönetiminde yasayan Givonlular'la Mispalilar onardi.
\par 8 Onlarin yanindakini kuyumculardan biri olan Harhaya oglu Uzziel onardi. Onun yanindakini baharatçi Hananya onardi. Yerusalim surlarini Genis Duvar'a kadar onardilar.
\par 9 Onlarin yanindaki bölümü Yerusalim'in yarisini yöneten Hur oglu Refaya onardi.
\par 10 Bunun yanindaki bölüm Harumaf oglu Yedaya'nin evinin karsisina düsüyordu. O bölümü Yedaya onardi. Onun yanindakini Hasavneya oglu Hattus onardi.
\par 11 Harim oglu Malkiya ile Pahat-Moav oglu Hassuv baska bir bölümü ve Firinlar Kulesi'ni onardilar.
\par 12 Onlarin yanindaki bölümü Yerusalim'in öbür yarisini yöneten Hallohes oglu Sallum kizlariyla birlikte onardi.
\par 13 Dere Kapisi'ni Hanun ve Zanoah'ta yasayanlar onardi. Kapi kanatlarini yerlerine yerlestirip sürgülerle kapi kollarini taktilar. Ayrica Gübre Kapisi'na kadar uzanan surlarda bin arsinlik yer onardilar.
\par 14 Gübre Kapisi'ni Beythakkerem bölgesini yöneten Rekav oglu Malkiya onardi. Kapi kanatlarini yerine yerlestirip sürgülerle kapi kollarini taktilar.
\par 15 Çesme Kapisi'ni Mispa bölgesini yöneten Kol-Hoze oglu Sallun onardi. Üzerini bir saçakla kapadi. Kapi kanatlarini yerine yerlestirip sürgülerle kapi kollarini takti. Kral Bahçesi'nin yanindaki Selah Havuzu'nun duvarini Davut Kenti'nden inen merdivenlere kadar onardi.
\par 16 Oradan ötesini Beytsur bölgesinin yarisini yöneten Azbuk oglu Nehemya, Davut'un aile mezarligindan yapay havuza ve Yigitler Evi'ne kadar onardi.
\par 17 Surlarin sonraki bölümünü asagidaki Levililer onardi: Bani oglu Rehum bir sonraki bölümü onardi. Bitisigini Keila bölgesinin yarisini yöneten Hasavya kendi bölgesi adina onardi.
\par 18 Ondan sonraki bölümü, Keila bölgesinin öbür yarisini yöneten Henadat oglu Bavvay yönetiminde kardesleri onardilar.
\par 19 Onun bitisigini -silah deposuna, surun döndügü yere kadar çikan yolun karsisini- Mispa'yi yöneten Yesu oglu Ezer onardi.
\par 20 Ondan sonraki, surun döndügü yerden Baskâhin Elyasiv'in evinin kapisina kadar uzanan bölümü büyük çaba harcayarak Zakkay oglu Baruk onardi.
\par 21 Ondan sonrasini, Elyasiv'in evinin kapisindan evin sonuna kadar uzanan bölümü Hakkos oglu Uriya oglu Meremot onardi.
\par 22 Surlarin sonraki bölümünü çevrede yasayan kâhinler onardi.
\par 23 Evlerinin karsisina düsen bölümü Benyamin ile Hassuv onardilar. Onlardan sonra, evinin bitisigi olan bölümü Ananya oglu Maaseya oglu Azarya onardi.
\par 24 Ondan sonraki, Azarya'nin evinden surun döndügü köseye kadar uzanan bölümü Henadat oglu Binnuy onardi.
\par 25 Uzay oglu Palal surun döndügü köseden sonraki bölümü ve yukari saray muhafiz avlusunun yanindaki gözetleme kulesini onardi. Ondan sonraki bölümü Paros oglu Pedaya onardi;
\par 26 Pedaya ile Ofel Tepesi'nde yasayan tapinak görevlileri doguya dogru Su Kapisi'nin önüne ve gözetleme kulesine kadarki bölümü onardilar.
\par 27 Ondan sonraki, büyük gözetleme kulesinden Ofel surlarina kadar uzanan bölümü Tekoalilar onardi.
\par 28 At Kapisi'nin yukarisini kâhinler onardi. Her biri kendi evinin karsisini yapti.
\par 29 Onlardan sonra, evinin karsisina düsen bölümü Immer oglu Sadok onardi. Ondan sonrasini Dogu Kapisi'nin nöbetçisi Sekanya oglu Semaya onardi.
\par 30 Ondan sonraki bölümü Selemya oglu Hananya ile Salaf'in altinci oglu Hanun onardi. Ondan sonra, odasinin karsisina düsen bölümü Berekya oglu Mesullam onardi.
\par 31 Ondan sonraki bölümü, tapinak görevlileriyle tüccarlarin kaldigi eve, oradan Mifkat Kapisi'na, surun yukari kösesindeki odaya kadar, kuyumculardan biri olan Malkiya onardi.
\par 32 Surun kösesindeki odayla Koyun Kapisi arasindaki bölümü kuyumcularla tüccarlar onardilar.

\chapter{4}

\par 1 Sanballat surlari onardigimizi duyunca öfkeden deliye döndü. Bizimle alay etmeye basladi.
\par 2 Dostlarinin ve Samiriye ordusunun önünde, "Bu zavalli Yahudiler ne yaptiklarini saniyorlar?" dedi, "Onlara izin verirler mi? Kurban mi kesecekler? Bir günde mi bitirecekler? Küle dönmüs molozlarin arasindan taslari mi canlandiracaklar?"
\par 3 Yaninda duran Ammonlu Toviya, "Yaptiklari su tas duvara bak!" dedi, "Üzerine bir tilki çiksa yikilir."
\par 4 O zaman söyle dua ettim: "Ey Tanrimiz, bize kulak ver! Hor görüyorlar bizi. Onlarin asagilamalarini kendi baslarina döndür. Sürüldükleri ülkede yagmaya ugrasinlar.
\par 5 Suçlarini bagislama, günahlarini unutma. Çünkü biz çalisanlari asagiladilar."
\par 6 Surun onarimina devam ettik; yari yükseklige kadar suru tamamladik. Çünkü herkes canla basla çalisiyordu.
\par 7 Sanballat, Toviya, Araplar, Ammonlular ve Asdotlular Yerusalim surlarindaki onarimin ilerledigini, gediklerin kapanmaya basladigini duyunca çok öfkelendiler.
\par 8 Hepsi bir araya gelerek Yerusalim'e karsi savasmak ve kentte karisiklik çikarmak için düzen kurdular.
\par 9 Ama biz Tanrimiz'a dua ettik ve gece gündüz onlari gözetlesinler diye nöbetçiler diktik.
\par 10 O sirada Yahudalilar, "Yük tasiyanlarin gücü tükendi" dediler, "O kadar moloz var ki, artik surlarin onarimini sürdüremiyoruz."
\par 11 Düsmanlarimiz ise, "Onlar anlamadan, bizi görmeden aralarina girip hepsini öldürerek bu ise son verelim" diye düsünüyorlardi.
\par 12 Çevrede yasayan Yahudiler gelip on kez bizi uyardilar. "Yanimiza gelin, yoksa size her yönden saldiracaklar" dediler.
\par 13 Bu yüzden, surlarin en alçak yerlerinin arkasina, tamamlanmamis yerlere, çesitli boylardan kiliçli, mizrakli, yayli adamlar yerlestirdim.
\par 14 Durumu görünce ayaga kalktim; soylulara, görevlilere ve geri kalan herkese, "Onlardan korkmayin!" dedim, "Yüce ve görkemli Rab'bi animsayin. Kardesleriniz, ogullariniz, kizlariniz, karilariniz, evleriniz için savasin."
\par 15 Kurduklari düzeni anladigimiz düsmanlarimizin kulagina gitti. Tanri düzenlerini bosa çikarmisti. O zaman hepimiz surlara, isimizin basina döndük.
\par 16 O günden sonra adamlarimin yarisi çalisirken öbür yarisi mizrakli, kalkanli, yayli ve zirhli olarak nöbet tuttu. Önderler Yahudalilar'in arkasinda yer almisti.
\par 17 Duvarcilar, yükleri tasiyanlar, yükleyenler bir eliyle çalisiyor, bir eliyle silah tutuyordu.
\par 18 Yapicilar kiliç kusanmis, öyle çalisiyorlardi. Boru çalansa benim yanimdaydi.
\par 19 Soylulara, görevlilere ve geri kalan herkese, "Is çok büyük ve daginik" dedim, "Surlarin üzerinde her birimiz ayri yerde, birbirimizden uzaktayiz.
\par 20 Nereden boru sesini isitirseniz, orada bize katilin. Tanrimiz bizim için savasacak."
\par 21 Iste böyle çalisiyorduk. Yarimiz gün dogumundan yildizlar görünene kadar mizraklarla nöbet tutuyordu.
\par 22 O sirada halka, "Herkes geceyi yardimcisiyla birlikte Yerusalim'de geçirsin" dedim, "Gece bizim için nöbet tutsunlar, gündüz de çalissinlar."
\par 23 Ne ben, ne kardeslerim, ne adamlarim, ne de yanimdaki nöbetçiler, giysilerimizi çikarmadik. Herkes suya bile silahiyla gitti.

\chapter{5}

\par 1 Bir süre sonra kadinli erkekli halk Yahudi kardeslerinden siddetle yakinmaya basladi.
\par 2 Bazilari, "Biz kalabaligiz" diyordu, "Ogullarimiz, kizlarimiz çok. Yasamak için bugdaya ihtiyacimiz var."
\par 3 Bazilari da, "Kitlikta bugday almak için tarlalarimizi, baglarimizi, evlerimizi ipotek ediyoruz" diyordu.
\par 4 Bazilari ise, "Krala vergi ödemek için tarlalarimizi, baglarimizi karsilik gösterip borç para aldik" diyordu,
\par 5 "Yahudi kardeslerimizle ayni kani tasimiyor muyuz? Bizim çocuklarimizin onlarinkinden ne farki var? Ogullarimizi kizlarimizi köle olarak satmak zorunda kaldik. Kizlarimizdan bazilari cariye olarak satildi bile. Çaresiz kaldik. Çünkü tarlalarimiz, baglarimiz baskalarinin elinde."
\par 6 Onlarin bu dertlerini, yakinmalarini duyunca çok öfkelendim.
\par 7 Düsününce soylularla yetkilileri suçlu buldum. Onlara, "Kardeslerinizden faiz aliyorsunuz!" dedim. Onlara karsi herkesi bir araya topladim. Sonra söyle dedim:
\par 8 "Biz yabancilara satilan Yahudi kardeslerimizi elimizden geldigince geri almaya çalisirken siz kardeslerinizi satiyorsunuz. Yine bize satilsinlar diye mi?" Sustular, söyleyecek söz bulamadilar.
\par 9 Sonra, "Yaptiginiz dogru degil" dedim, "Düsmanlarimiz olan öteki uluslarin asagilamalarindan kaçinmak için Tanri korkusuyla yasamaniz gerekmez mi?
\par 10 Kardeslerim, adamlarim ve ben ödünç olarak halka para ve bugday veriyoruz. Lütfen faiz almaktan vazgeçelim!
\par 11 Tarlalarini, baglarini, zeytinliklerini, evlerini onlara hemen geri verin. Bir de faiz olarak aldiginiz gümüsün, bugdayin, yeni sarabin, zeytinyaginin yüzde birini verin."
\par 12 "Veririz" dediler, "Artik onlardan hiçbir sey istemeyecegiz. Ne diyorsan öyle yapacagiz." Kâhinleri çagirdim ve yetkililere kâhinlerin önünde verdikleri sözü tutacaklarina iliskin ant içirdim.
\par 13 Sonra etegimi silktim ve dedim ki, "Kim verdigi sözü tutmazsa, Tanri da onu böyle silksin; malini mülkünü elinden alsin; tamtakir biraksin." Herkes buna, "Amin" dedi ve RAB'be övgüler sundu. Ve sözlerini tuttular.
\par 14 Yahuda'da valilik yaptigim on iki yil boyunca, ilk atandigim günden son güne kadar, Artahsasta'nin kralliginin yirminci yilindan otuz ikinci yilina dek, ne ben, ne kardeslerim valilige ayrilan yiyecek bütçesine dokunmadik.
\par 15 Benden önce görev yapan valiler halka yük oldular. Onlardan kirk sekel gümüsün yanisira yiyecek ve sarap da aldilar. Usaklari bile halki ezdi. Ama ben Tanri'dan korktugum için böyle davranmadim.
\par 16 Surlarin onarimini sürdürdüm. Adamlarimin hepsi isin basinda durdu. Bir tarla bile satin almadik.
\par 17 Çevremizdeki uluslardan bize gelenlerin disinda Yahudiler'den ve yetkililerden yüz elli kisi soframa otururdu.
\par 18 Benim için her gün bir boga, alti seçme koyun, tavuklar kesilir, on günde bir de her türden bolca sarap hazirlanirdi. Bütün bunlara karsin valiligin yiyecek bütçesine dokunmadim. Çünkü halk agir yük altindaydi.
\par 19 Ey Tanrim, bu halk ugruna yaptiklarim için beni iyilikle an.

\chapter{6}

\par 1 Surlari onardigim, gediklerini kapadigim haberi Sanballat'a, Toviya'ya, Arap Gesem'e ve öbür düsmanlarimiza ulasti. O sirada kapi kanatlarini henüz takmamistim.
\par 2 Sanballat ile Gesem bana haber göndererek, "Gel, Ono Ovasi'ndaki köylerden birinde bulusalim" dediler. Bana kötülük yapmayi düsünüyorlardi.
\par 3 Onlara haberciler göndererek, "Büyük bir is yapiyorum, gelemem" dedim, "Yaniniza gelirsem isi birakmis olurum; niçin is dursun?"
\par 4 Bana dört kez bu haberi gönderdiler, ben de hep ayni yaniti verdim.
\par 5 Sanballat besinci kez ayni öneriyle habercisini gönderdi. Habercinin elinde açik bir mektup vardi.
\par 6 Içinde sunlar yaziliydi: "Çevredeki uluslar arasinda Gesem'in de dogruladigi bir söylenti var. Sen ve Yahudiler ayaklanmayi düsündügünüz için surlari onariyormussunuz. Anlatilanlara göre kral olmak üzeresin.
\par 7 Yahuda Krali oldugunu Yerusalim'e duyurmak için peygamberler bile atamissin. Bütün bunlar kralin kulagina gidecek. Onun için, gel de görüselim."
\par 8 Ona su yaniti gönderdim: "Söylediklerinin hiçbiri dogru degil. Hepsini kendin uyduruyorsun."
\par 9 Hepsi bizi korkutmaya çalisiyorlardi. "Isi birakacaklar, onarim duracak" diye düsünüyorlardi. Ama ben, "Tanrim, ellerime güç ver" diye dua ettim.
\par 10 Bir gün Mehetavel oglu Delaya oglu Semaya'nin evine gittim. Evine kapanmisti. Bana, "Tanri'nin evinde, tapinakta bulusalim" dedi, "Tapinagin kapilarini kapatalim, çünkü seni öldürmeye gelecekler. Gece seni öldürmeye gelecekler."
\par 11 Ona, "Ben kaçacak adam degilim" dedim, "Benim gibi biri canini kurtarmak için tapinaga siginir mi? Gelmeyecegim."
\par 12 Anladim ki, onu Tanri göndermemis. Bu sözleri bir peygamber gibi, benim kötülügüm için söylemisti. Toviya ile Sanballat onu satin almislardi.
\par 13 Bu yolla gözümü korkutup bana günah isleteceklerini düsünüyorlardi. Böylece beni kötülemek için ellerine firsat geçmis olacakti.
\par 14 "Ey Tanrim, Toviya'yla Sanballat'in yaptigi kötülügü unutma" diye dua ettim, "Beni korkutmak isteyen kadin peygamber Noadya'yla öbür peygamberlerin yaptiklarini da unutma."
\par 15 Surlarin onarimi elli iki günde, yirmi bes Elul'da bitti.
\par 16 Bütün düsmanlarimiz bunu duydu, çevremizdeki uluslari korku sardi. Böylece düsmanlarimiz özgüvenlerini büsbütün yitirdiler. Çünkü bu isi Tanrimiz'in yardimiyla basardigimizi anladilar.
\par 17 O günlerde Yahuda soylulariyla Toviya sik sik yazisiyorlardi.
\par 18 Birçok Yahudali Toviya'ya bagli kalacagina ant içmisti. Çünkü Toviya, Arah oglu Sekanya'nin damadiydi. Oglu Yehohanan da Berekya oglu Mesullam'in kizini almisti.
\par 19 Soylular Toviya'nin iyiliklerini bana anlatiyor, benim söylediklerimi de ona iletiyorlardi. Toviya beni yildirmak için sürekli mektup gönderiyordu.

\chapter{7}

\par 1 Surlarin onarimi bitip kapilar yerine takildiktan sonra, kapi nöbetçileri, ezgiciler ve Levililer göreve atandi.
\par 2 Kardesim Hanani'yle kale komutani Hananya'yi Yerusalim'e yönetici atadim. Hananya güvenilir bir kisiydi. Çogu insandan daha çok Tanri'dan korkardi.
\par 3 Onlara, "Günes ortaligi isitincaya kadar Yerusalim kapilari açilmasin" dedim, "Kapi nöbetçileri görev basindayken kapilari kapali tutsunlar. Kapilari siz sürgüleyin ve Yerusalim'de oturanlara nöbet görevi verin. Bazilari bu görevi yapsin, bazilari da evlerinin çevresinde nöbet tutsun."
\par 4 Yerusalim genis, büyük bir kentti, ama nüfusu azdi. Içindeki evler henüz onarilmamisti.
\par 5 Tanrim soylarina göre halkin sayimi yapilabilsin diye soylulari, yetkilileri ve bütün halki toplamami istedi. Sürgünden ilk dönenlerin soy kütügünü buldum. Içinde sunlar yaziliydi:
\par 6 Babil Krali Nebukadnessar'in sürgün ettigi insanlar yasadiklari ilden Yerusalim ve Yahuda'daki kendi kentlerine döndü.
\par 7 Bunlar Zerubbabil, Yesu, Nehemya, Azarya, Raamya, Nahamani, Mordekay, Bilsan, Misperet, Bigvay, Nehum ve Baana'nin önderliginde geldiler.
\par 8 Parosogullari: 2172
\par 9 Sefatyaogullari: 372
\par 10 Arahogullari: 652
\par 11 Yesu ve Yoav soyundan Pahat-Moavogullari: 2818
\par 12 Elamogullari: 1254
\par 13 Zattuogullari: 845
\par 14 Zakkayogullari: 760
\par 15 Binnuyogullari: 648
\par 16 Bevayogullari: 628
\par 17 Azgatogullari: 2322
\par 18 Adonikamogullari: 667
\par 19 Bigvayogullari: 2067
\par 20 Adinogullari: 655
\par 21 Hizkiya soyundan Aterogullari: 98
\par 22 Hasumogullari: 328
\par 23 Besayogullari: 324
\par 24 Harifogullari: 112
\par 25 Givonlular: 95
\par 26 Beytlehemliler ve Netofalilar: 188
\par 27 Anatotlular: 128
\par 28 Beytazmavetliler: 42
\par 29 Kiryat-Yearimliler, Kefiralilar ve Beerotlular: 743
\par 30 Ramalilar ve Gevalilar: 621
\par 31 Mikmaslilar: 122
\par 32 Beytel ve Ay kentlerinden olanlar: 123
\par 33 Öbür Nevo Kenti'nden olanlar: 52
\par 34 Öbür Elam Kenti'nden olanlar: 1254
\par 35 Harimliler: 320
\par 36 Erihalilar: 345
\par 37 Lod, Hadit ve Ono kentlerinden olanlar: 721
\par 38 Senaalilar: 3930
\par 39 Kâhinler: Yesu soyundan Yedayaogullari: 973
\par 40 Immerogullari: 1052
\par 41 Pashurogullari: 1247
\par 42 Harimogullari: 1017
\par 43 Levililer: Kadmiel ve Hodeva soyundan gelen Yesuogullari: 74
\par 44 Ezgiciler: Asafogullari: 148
\par 45 Tapinak kapi nöbetçileri: Sallumogullari, Aterogullari, Talmonogullari, Akkuvogullari, Hatitaogullari, Sovayogullari: 138
\par 46 Tapinak görevlileri: Sihaogullari, Hasufaogullari, Tabbaotogullari,
\par 47 Kerosogullari, Siaogullari, Padonogullari,
\par 48 Levanaogullari, Hagavaogullari, Salmayogullari,
\par 49 Hananogullari, Giddelogullari, Gaharogullari,
\par 50 Reayaogullari, Resinogullari, Nekodaogullari,
\par 51 Gazzamogullari, Uzzaogullari, Paseahogullari,
\par 52 Besayogullari, Meunimogullari, Nefisesimogullari,
\par 53 Bakbukogullari, Hakufaogullari, Harhurogullari,
\par 54 Baslitogullari, Mehidaogullari, Harsaogullari,
\par 55 Barkosogullari, Siseraogullari, Temahogullari,
\par 56 Nesiahogullari, Hatifaogullari.
\par 57 Süleyman'in kullarinin soyu: Sotayogullari, Soferetogullari, Peridaogullari,
\par 58 Yalaogullari, Darkonogullari, Giddelogullari,
\par 59 Sefatyaogullari, Hattilogullari, Pokeret-Hassevayimogullari, Amonogullari.
\par 60 Tapinak görevlileriyle Süleyman'in kullarinin soyundan olanlar: 392
\par 61 Tel-Melah, Tel-Harsa, Keruv, Addon ve Immer'den dönen, ancak hangi aileden olduklarini ve soylarinin Israil'den geldigini kanitlayamayanlar sunlardir:
\par 62 Delayaogullari, Toviyaogullari, Nekodaogullari: 642
\par 63 Kâhinlerin soyundan: Hovayaogullari, Hakkosogullari ve Gilatli Barzillay'in kizlarindan biriyle evlenip kayinbabasinin adini alan Barzillay'in ogullari.
\par 64 Bunlar soy kütüklerini aradilar. Ama yazili bir kayit bulamayinca, kâhinlik görevi ellerinden alindi.
\par 65 Vali, Urim ile Tummim'i* kullanan bir kâhin çikincaya dek en kutsal yiyeceklerden yememelerini buyurdu.
\par 66 Bütün halk toplam 42 360 kisiydi.
\par 67 Ayrica 7 337 erkek ve kadin köle, kadinli erkekli 245 ezgici, 736 at, 245 katir, 435 deve, 6 720 esek vardi.
\par 70 Bazi aile baslari onarim isi için bagista bulundu: Vali hazineye 1 000 darik altin, 50 çanak, 530 kâhin mintani bagisladi.
\par 71 Bazi aile baslari da is için hazineye 20 000 darik altin, 2 200 mina gümüs verdiler.
\par 72 Halkin geri kalani ise, toplam 20 000 darik*fi* altin, 2000 mina*fj* gümüs ve 67 kâhin mintani verdi.
\par 73 Kâhinler, Levililer, tapinak görevlileri ve kapi nöbetçileri, ezgiciler, siradan insanlar ve bütün Israilliler kentlerine yerlestiler.

\chapter{8}

\par 1 Israilliler kentlerine yerlestikten sonra, yedinci ay* tek vücut halinde Su Kapisi'nin karsisindaki alanda toplandilar. Bilgin Ezra'ya RAB'bin Musa araciligiyla Israil halkina verdigi buyruklari içeren Yasa Kitabi'ni getirmesini söylediler.
\par 2 Yedinci ayin birinci günü Kâhin Ezra Yasa Kitabi'ni halkin toplandigi yere getirdi. Dinleyip anlayabilecek kadin erkek herkes oradaydi.
\par 3 Ezra Su Kapisi'nin karsisindaki alanda kadinlarin, erkeklerin ve anlayabilecek yastaki çocuklarin önünde, sabahtan öglene kadar Yasa Kitabi'ni okudu. Herkes dikkatle dinledi.
\par 4 Bilgin Ezra toplanti için hazirlanmis ahsap bir zemin üzerinde duruyordu. Saginda Mattitya, Sema, Anaya, Uriya, Hilkiya ve Maaseya vardi. Solunda ise Pedaya, Misael, Malkiya, Hasum, Hasbaddana, Zekeriya ve Mesullam duruyordu.
\par 5 Ezra halkin gözü önünde kitabi açti. Halktan daha yüksek bir yerde duruyordu. Kitabi açar açmaz herkes ayaga kalkti.
\par 6 Ezra yüce Tanri'ya, RAB'be övgüler sundu. Bütün halk ellerini kaldirarak, "Amin! Amin!" diye karsilik verdi. Hep birlikte egilip yere kapanarak RAB'be tapindilar.
\par 7 Levililer'den Yesu, Bani, Serevya, Yamin, Akkuv, Sabbetay, Hodiya, Maaseya, Kelita, Azarya, Yozavat, Hanan ve Pelaya ayakta duran halka yasayi anlattilar.
\par 8 Tanri'nin Yasa Kitabi'ni okuyup açikladilar, herkesin anlamasini saglayacak biçimde yorumladilar.
\par 9 Vali Nehemya, Kâhin ve Bilgin Ezra ve halka ögretmenlik yapan Levililer, "Bugün Tanriniz RAB için kutsal bir gündür. Yas tutup aglamayin" dediler. Çünkü bütün halk Kutsal Yasa'yi dinlerken agliyordu.
\par 10 Nehemya da, "Gidin, yagli yiyip tatli için" dedi, "Hazirligi olmayanlara da bir pay gönderin. Çünkü bugün Rabbimiz için kutsal bir gündür. Üzülmeyin. RAB'bin verdigi sevinç sizi güçlü kilar."
\par 11 Levililer, "Sakin olun, bugün kutsal bir gündür, üzülmeyin" diyerek halki yatistirdilar.
\par 12 Böylece herkes yiyip içmek, yiyeceklerini baskalariyla paylasmak ve büyük senlik yapmak üzere evinin yolunu tuttu. Çünkü kendilerine okunanlari anlamislardi.
\par 13 Ertesi gün bütün aile baslari, kâhinler ve Levililer Kutsal Yasa'nin buyruklarini ögrenmek için Bilgin Ezra'nin çevresine toplandilar.
\par 14 Yasada RAB'bin Musa araciligiyla verdigi su buyrugu buldular: Yedinci ayda* kutlanan bayramda Israilliler çardaklarda oturmali.
\par 15 Bütün kentlerde ve Yerusalim'de su duyuru yapilsin: "Daglara çikin; yasada yazilana uygun olarak, çardak yapmak üzere zeytin, igde, mersin ve hurma dallari, sik yaprakli agaç dallari getirin."
\par 16 Böylece halk dallari getirip damlarinda, evlerinin ve Tanri Tapinagi'nin avlularinda, Su Kapisi ve Efrayim Kapisi alanlarinda çardaklar yapti.
\par 17 Sürgünden dönen herkes yaptigi çardakta oturdu. Israilliler Nun oglu Yesu'nun döneminden beri böyle bir kutlama yapmamislardi. Herkes büyük sevinç içindeydi.
\par 18 Ezra ilk günden son güne kadar, her gün Tanri'nin Yasa Kitabi'ni okudu. Yedi gün bayram yaptilar. Sekizinci gün kural uyarinca kutsal toplanti yapildi.

\chapter{9}

\par 1 Ayni ayin yirmi dördüncü günü Israilliler toplandi. Hepsi oruç* tutmus, çul kusanmis, basina toprak serpmisti.
\par 2 Israil soyundan gelenler bütün yabancilardan ayrilmisti. Günahlarini ve atalarinin yaptigi kötülükleri ayakta itiraf ettiler.
\par 3 Olduklari yerde durup günün dörtte biri boyunca Tanrilari RAB'bin Yasa Kitabi'ni okudular. Günün öbür dörtte birindeyse günahlarini itiraf ederek Tanrilari RAB'be tapindilar.
\par 4 Levililer'e yüksekçe bir yer ayrilmisti. Yesu, Bani, Kadmiel, Sevanya, Bunni, Serevya, Bani ve Kenani orada oturuyordu. Ayaga kalkip yüksek sesle Tanrilari RAB'be yakardilar.
\par 5 Levililer'den Yesu, Kadmiel, Bani, Hasavneya, Serevya, Hodiya, Sevanya ve Petahya halka, "Ayaga kalkin!" dediler, "Baslangiçtan sonsuza kadar var olan Tanriniz RAB'be övgüler olsun. `Ya Rab senin kutsal adin öyle yücedir ki, bizim yüceltmelerimiz, övgülerimiz yetersiz kalir.'"
\par 6 Halk söyle dua etti: "Tek RAB sensin. Gökleri, göklerin göklerini, bütün gök cisimlerini, yeryüzünü ve içindeki her seyi, denizleri ve içlerindeki her seyi sen yarattin. Hepsine sen can verdin. Bütün gök cisimleri sana tapinir.
\par 7 "Ya RAB, Avram'i seçen, onu Kildaniler'in* Ur Kenti'nden çikaran, ona Ibrahim adini veren Tanri sensin.
\par 8 Onu kendine yürekten bagli buldun ve onunla bir antlasma yaptin. Kenanli, Hitit*, Amorlu, Perizli, Yevus ve Girgas topraklarini onun soyuna verecegim deyip sözünü tuttun. Çünkü sen dogrusun.
\par 9 "Atalarimizin Misir'da çektiklerini gördün, Kizildeniz'de* yakarislarini isittin.
\par 10 Firavuna, görevlilerine ve ülkesinin halkina karsi mucizeler, harikalar yarattin. Çünkü atalarimizi nasil ezdiklerini biliyordun. Bugün oldugu gibi ün kazandin.
\par 11 Denizi yararak atalarimiza yol açtin. Denizin ortasindan, kuru topraktan geçip gittiler. Onlari kovalayanlari ise bir tas gibi azgin derin sulara firlattin.
\par 12 Gündüzün bir bulut sütunuyla, geceleyin yollarina isik tutmak için bir ates sütunuyla atalarimiza yol gösterdin.
\par 13 "Sina Dagi'na indin, onlarla göklerden konustun. Onlara dogru ilkeler, adil yasalar, iyi kurallar, buyruklar verdin.
\par 14 Kutsal Sabat Günü'nü* bildirdin. Kulun Musa araciligiyla buyruklar, kurallar, yasalar verdin.
\par 15 Aciktiklarinda gökten ekmek verdin, susadiklarinda kayadan su çikardin. Onlara vermeye ant içtigin ülkeye girmelerini, orayi mülk edinmelerini buyurdun.
\par 16 "Ama atalarimiz gurura kapildi; dikbaslilik edip buyruklarina uymadilar.
\par 17 Söz dinlemek istemediler, aralarinda yaptigin harikalari unuttular. Dikbaslilik ettiler, eski kölelik yasamlarina dönmek için kendilerine bir önder bularak baskaldirdilar. Ama sen bagislayan, iyilik yapan, aciyan, tez öfkelenmeyen, sevgisi engin bir Tanri'sin. Onlari terk etmedin.
\par 18 Kendilerine buzagi biçiminde dökme bir put yaptilar, `Sizi Misir'dan çikaran Tanriniz budur!' diyerek seni çok asagiladilar.
\par 19 Yine de, yüce merhametinden ötürü onlari çölde birakmadin. Gündüzün yol göstermek için bulut sütununu, geceleyin yollarina isik tutmak için ates sütununu önlerinden eksik etmedin.
\par 20 Onlari egitmek için iyi Ruhun'u verdin. Agizlarindan mani* eksiltmedin. Susadiklarinda onlara su verdin.
\par 21 Kirk yil onlari çölde besledin. Hiç eksikleri olmadi. Ne giysileri eskidi, ne de ayaklari sisti.
\par 22 "Onlara ülkeler, uluslar verdin, aralarinda bölüstürdün. Hesbon Krali Sihon'un, Basan Krali Og'un ülkesini mülk edindiler.
\par 23 Onlara gökteki yildizlar kadar çocuk verdin. Onlari, mülk edinmek üzere atalarina söz verdigin ülkeye getirdin.
\par 24 Çocuklari Kenan ülkesini ele geçirip mülk edindiler. Ülke halkinin onlara boyun egmesini sagladin. Krallarini ve ülkedeki halklari istediklerini yapsinlar diye ellerine teslim ettin.
\par 25 Surlu kentler, verimli topraklar ele geçirdiler. Güzel esyalarla dolu evlere, kazilmis sarniçlara, baglara, zeytinliklere, çok sayida meyve agacina sahip oldular. Yediler, doydular, beslendiler ve onlara yaptigin büyük iyiliklere sevindiler.
\par 26 "Ama halkin söz dinlemedi, sana baskaldirdi. Yasana sirt çevirdiler, sana dönmeleri için kendilerini uyaran peygamberleri öldürdüler. Seni çok asagiladilar.
\par 27 Bu yüzden onlari düsmanlarinin eline teslim ettin. Düsmanlari onlari ezdi. Sikintiya düsünce sana feryat ettiler. Onlari göklerden duydun, yüce merhametinden ötürü kurtaricilar gönderdin. Bunlar halki düsmanlarinin elinden kurtardi.
\par 28 "Ne var ki Israil halki rahata kavusunca yine senin gözünde kötü olani yapti. Bu yüzden onlari düsmanlarinin eline terk ettin. Düsmanlari onlara egemen oldu. Yine sana yönelip feryat ettiler. Onlari göklerden duydun ve merhametinden ötürü defalarca kurtardin.
\par 29 "Onlari Kutsal Yasan'a dönmeleri için uyardinsa da, gurura kapilarak buyruklarina karsi geldiler. Kurallarini çigneyip günah islediler. Oysa kim kurallarina bagli kalirsa yasam bulur. Inatla sana sirt çevirdiler, dinlemek istemediler.
\par 30 Yillarca onlara katlandin. Ruhun'la, peygamberlerin araciligiyla onlari uyardin. Ama kulak asmadilar. Bunun üzerine onlari çesitli ülke halklarinin ellerine teslim ettin.
\par 31 Yüce merhametinden ötürü yok olmalarina izin vermedin. Onlari terk etmedin. Çünkü sen iyilik yapan, aciyan bir Tanri'sin.
\par 32 "Ey Tanrimiz! Sen antlasmana bagli kalirsin. Güçlü, görkemli, yüce bir Tanri'sin. Asur krallarinin döneminden bugüne kadar krallarimiz, önderlerimiz, kâhinlerimiz, peygamberlerimiz, atalarimiz ve bütün halk aci çekti. Çektiklerimizi küçümseme.
\par 33 Basimiza gelen bütün olaylarda sen hep adil davrandin, dogru olani yaptin, bizse kötülük yaptik.
\par 34 Krallarimiz, önderlerimiz, kâhinlerimiz, atalarimiz yasana göre yasamadilar. Verdigin buyruklari, yaptigin uyarilari dinlemediler.
\par 35 Ülkelerinde onlara sagladigin bolluk içinde, önlerine serdigin genis, verimli topraklarda sana kulluk etmediler, kötülüklerinden dönmediler.
\par 36 "Bak, bugün köleyiz. Meyvelerini, iyi ürünlerini yesinler diye atalarimiza verdigin ülkede köle olduk.
\par 37 Günahlarimiz yüzünden ürünlerimizin çogunu basimiza getirdigin krallara veriyoruz. Bizi de, hayvanlarimizi da istedikleri gibi kullaniyorlar. Büyük sikinti içindeyiz."
\par 38 "Bütün bu olanlardan ötürü biz Israil halki olarak kesin bir yazili antlasma yapiyoruz. Önderlerimiz, Levililerimiz ve kâhinlerimiz de antlasmayi mühürlüyor."

\chapter{10}

\par 1 Antlasmayi mühürleyenler sunlardi: Hakalya oglu Vali Nehemya ve Sidkiya.
\par 2 Kâhinler: Seraya, Azarya, Yeremya,
\par 3 Pashur, Amarya, Malkiya,
\par 4 Hattus, Sevanya, Malluk,
\par 5 Harim, Meremot, Ovadya,
\par 6 Daniel, Ginneton, Baruk,
\par 7 Mesullam, Aviya, Miyamin,
\par 8 Maazya, Bilgay, Semaya.
\par 9 Levililer: Azanya oglu Yesu, Henadat ogullarindan Binnuy, Kadmiel;
\par 10 arkadaslari Sevanya, Hodiya, Kelita, Pelaya, Hanan,
\par 11 Mika, Rehov, Hasavya,
\par 12 Zakkur, Serevya, Sevanya,
\par 13 Hodiya, Bani, Beninu.
\par 14 Halk önderleri: Paros, Pahat-Moav, Elam, Zattu, Bani,
\par 15 Bunni, Azgat, Bevay,
\par 16 Adoniya, Bigvay, Adin,
\par 17 Ater, Hizkiya, Azzur,
\par 18 Hodiya, Hasum, Besay,
\par 19 Harif, Anatot, Nevay,
\par 20 Magpias, Mesullam, Hezir,
\par 21 Mesezavel, Sadok, Yaddua,
\par 22 Pelatya, Hanan, Anaya,
\par 23 Hosea, Hananya, Hassuv,
\par 24 Hallohes, Pilha, Sovek,
\par 25 Rehum, Hasavna, Maaseya,
\par 26 Ahiya, Hanan, Anan,
\par 27 Malluk, Harim, Baana.
\par 28 "Halkin geri kalani, kâhinler, Levililer, tapinak görevlileri ve kapi nöbetçileri, ezgiciler, Tanri'nin Yasasi ugruna çevre halklardan ayrilmis olan herkes, karilari ve anlayip kavrayacak yastaki ogullariyla, kizlariyla birlikte
\par 29 soylu kardeslerine katildilar. Tanri'nin, kulu Musa araciligiyla verdigi yasaya göre yasamak, Egemenimiz RAB'bin bütün buyruklarina, ilkelerine, kurallarina uymak üzere ant içtiler, uymayacaklara lanet okudular.
\par 30 "Çevremizdeki halklara kiz verip kiz almayacagiz.
\par 31 "Çevre halklardan Sabat Günü* ya da kutsal bir gün esya veya tahil satmak isteyen olursa almayacagiz. Yedi yilda bir topragi sürmeyecegiz ve bütün alacaklarimizi silecegiz.
\par 32 "Tanrimiz'in Tapinagi'nin giderlerini karsilamak üzere hepimiz sorumluluk aliyoruz. Her yil sekelin üçte birini verecegiz. Bu para adak ekmekleri*, günlük tahil sunusu* ve yakmalik sunular*, Sabat günleri, Yeni Aylar ve öbür bayramlarda sunulan kurbanlar, kutsal sunular, Israil'in günahlarini bagislatacak sunular ve Tanrimiz'in Tapinagi'nin öteki isleri için harcanacak.
\par 34 "Kâhinler, Levililer ve halk hep birlikte kura çektik. Aileler her yil Kutsal Yasa'ya uygun olarak Tanrimiz RAB'bin sunaginda yakilmak üzere belirli zamanlarda odun getirecek.
\par 35 "Ayrica her yil topragimizin ve meyve agaçlarimizin ilk ürününü RAB'bin Tapinagi'na götürecegiz.
\par 36 "Yasaya uygun olarak, ilk dogan ogullarimizi, hayvanlarimizi, ilk dogan sigirlarimizi ve davarlarimizi Tanrimiz'in Tapinagi'na, tapinakta hizmet eden kâhinlere götürecegiz.
\par 37 "Hamurlu yiyeceklerimizin, kaldirdigimiz ürünlerin, bütün agaçlarimizin meyvelerinin, yeni sarabimizin, zeytinyagimizin ilkini Tanrimiz'in Tapinagi'nin depolarina getirip kâhinlere verecegiz. Topragimizin ondaligini Levililer'e verecegiz, çünkü çalistigimiz bütün kentlerde ondaliklari onlar topluyor.
\par 38 Levililer ondaliklari toplarken Harun soyundan bir kâhin yanlarinda bulunacak. Levililer topladiklari ondaligin onda birini Tanrimiz'in Tapinagi'ndaki depolara, hazine odalarina birakacak.
\par 39 Israil halkiyla Levililer, bugdaydan, yeni saraptan, zeytinyagindan verilen armaganlari, tapinak esyalarinin, kâhinlerin, tapinak kapi nöbetçilerinin ve ezgicilerin bulundugu odalara koyacaklar. "Artik Tanrimiz'in Tapinagi'ni göz ardi etmeyecegiz."

\chapter{11}

\par 1 Halkin önderleri Yerusalim'e yerlesti. Geri kalanlar aralarinda kura çektiler. Her on kisiden biri kutsal kente, Yerusalim'e yerlesecek, öteki dokuz kisiyse kendi kentlerinde kalacaklardi.
\par 2 Halk Yerusalim'de yasamaya gönüllü olanlarin hepsini kutladi.
\par 3 Yerusalim'e yerlesen bölge önderleri sunlardir: Ancak bazi Israilliler, kâhinler, Levililer, tapinak görevlileri, Süleyman'in kullarinin soyundan gelenler Yahuda kentlerine, her biri kendi kentindeki kendi mülküne yerlesti.
\par 4 Yahuda ve Benyamin halkindan bazilariysa Yerusalim'de kaldi. Yahuda soyundan gelenler: Peres soyundan Mahalalel oglu Sefatya oglu Amarya oglu Zekeriya oglu Uzziya oglu Ataya,
\par 5 Sela soyundan Zekeriya oglu Yoyariv oglu Adaya oglu Hazaya oglu Kol-Hoze oglu Baruk oglu Maaseya.
\par 6 Peresogullari'ndan Yerusalim'e 468 yigit yerlesti.
\par 7 Benyamin soyundan gelenler: Yesaya oglu Itiel oglu Maaseya oglu Kolaya oglu Pedaya oglu Yoet oglu Mesullam oglu Sallu.
\par 8 Onu Gabbay ve Sallay izledi; toplam 928 yigit.
\par 9 Zikri oglu Yoel onlara önderlik ediyordu, Hassenua oglu Yahuda ise kentte vali yardimcisiydi.
\par 10 Kâhinler: Yoyariv oglu Yedaya, Yakin,
\par 11 Ahituv oglu Merayot oglu Sadok oglu Mesullam oglu Hilkiya oglu tapinak bas görevlisi Seraya
\par 12 ve tapinaga hizmet eden kardesleri; toplam 822 kisi. Malkiya oglu Pashur oglu Zekeriya oglu Amsi oglu Pelalya oglu Yeroham oglu Adaya
\par 13 ve aile baslari olan kardesleri; toplam 242 kisi. Immer oglu Mesillemot oglu Ahzay oglu Azarel oglu Amassay ve
\par 14 kardeslerinden olusan 128 cesur yigit. Haggedolim oglu Zavdiel onlara önderlik ediyordu.
\par 15 Levililer: Bunni oglu Hasavya oglu Azrikam oglu Hassuv oglu Semaya.
\par 16 Levililer'in önderlerinden Sabbetay'la Yozavat Tanri Tapinagi'nin dis islerini yönetiyordu.
\par 17 Asaf oglu Zavdi oglu Mika oglu Mattanya sükran duasini okuyan tapinak korosunu yönetiyordu. Kardeslerinden Bakbukya ise ikinci derecede görevliydi. Ayrica Yedutun oglu Galal oglu Sammua oglu Avda vardi.
\par 18 Kutsal kentte* yasayan Levililer 284 kisiydi.
\par 19 Tapinak kapi nöbetçileri: Kapilarda Akkuv, Talmon ve kardesleri nöbet tutardi. Toplam 172 kisiydiler.
\par 20 Israilliler'in geri kalani, kâhinlerle Levililer ise Yahuda'nin öbür kentlerine dagilmisti. Herkes kendi mülküne yerlesmisti.
\par 21 Tapinak görevlileri Ofel'de yasiyordu. Önderleri Siha ile Gispa idi.
\par 22 Tanri'nin Tapinagi'nda ezgi söyleyenlere Asaf soyundan gelenler önderlik ediyordu. Bu soydan Mika oglu Mattanya oglu Hasavya oglu Bani oglu Uzzi Yerusalim'de, Levililer'in basinda bulunuyordu.
\par 23 Pers Krali'nin ezgicilerle ilgili buyrugu vardi. Düzenli olarak her gün ücretlerini alacaklardi.
\par 24 Yahuda oglu Zerah'in soyundan Mesezavel oglu Petahya Israil halkinin genel temsilcisi olarak Pers Krali'na yardimci oluyordu.
\par 25 Kirsal bölgelerde, köylerde yasayanlara gelince: Bazi Yahudalilar Kiryat-Arba ve köylerinde, bazilari Divon ve köylerinde, bazilari Yekavseel ve köylerinde,
\par 26 bazilari Yesua'da, Molada'da, Beytpelet'te,
\par 27 Hasar-Sual'da, Beer-Seva ve köylerinde,
\par 28 bazilari Ziklak'ta, Mekona ve köylerinde,
\par 29 bazilari Eyn-Rimmon'da, Sora'da, Yarmut'ta,
\par 30 Zanoah'ta, Adullam ve köylerinde, bazilari Lakis ve çevresinde, bazilari da Azeka ve köylerinde yasiyordu. Yahudalilar'in yasadigi bu yerler Beer-Seva ile Hinnom Vadisi arasindaki topraklari kapsiyordu.
\par 31 Benyamin soyundan olanlar Geva'da, Mikmas'ta, Aya'da, Beytel ve köylerinde,
\par 32 Anatot'ta, Nov'da, Ananya'da,
\par 33 Hasor'da, Rama'da, Gittayim'de,
\par 34 Hadit'te, Sevoim'de, Nevallat'ta,
\par 35 Lod'da, Ono'da ve Esnaf Vadisi'nde yasiyordu.
\par 36 Bölükler halinde Yahuda'dan gelen bazi Levililer de Benyamin'e yerlesti.

\chapter{12}

\par 1 Sealtiel oglu Zerubbabil ve Yesu ile birlikte sürgünden dönen kâhinlerle Levililer sunlardir: Kâhinler: Seraya, Yeremya, Ezra,
\par 2 Amarya, Malluk, Hattus,
\par 3 Sekanya, Rehum, Meremot,
\par 4 Iddo, Ginneton, Aviya,
\par 5 Miyamin, Maadya, Bilga,
\par 6 Semaya, Yoyariv, Yedaya,
\par 7 Sallu, Amok, Hilkiya, Yedaya. Bunlar Yesu'nun döneminde kâhinlere ve öbür kardeslerine önderlik ediyorlardi.
\par 8 Levililer: Yesu, Binnuy, Kadmiel, Serevya, Yahuda ve sükran ezgileri sorumlusu Mattanya ile kardesleri.
\par 9 Öbür kardesleri Bakbukya ile Unni ezgiler söylenirken onlarin karsisinda dururdu.
\par 10 Yesu Yoyakim'in babasiydi. Yoyakim Elyasiv'in babasi, Elyasiv Yoyada'nin babasi,
\par 11 Yoyada Yonatan'in babasi, Yonatan Yaddua'nin babasiydi.
\par 12 Yoyakim'in döneminde, kâhin ailelerinin baslari sunlardi: Seraya ailesinin basinda Meraya, Yeremya'nin Hananya,
\par 13 Ezra'nin Mesullam, Amarya'nin Yehohanan,
\par 14 Meliku'nun Yonatan, Sevanya'nin Yusuf,
\par 15 Harim'in Adna, Merayot'un Helkay,
\par 16 Iddo'nun Zekeriya, Ginneton'un Mesullam,
\par 17 Aviya'nin Zikri, Minyamin'in, Moadya'nin Piltay,
\par 18 Bilga'nin Sammua, Semaya'nin Yehonatan,
\par 19 Yoyariv'in Mattenay, Yedaya'nin Uzzi,
\par 20 Sallay'in Kallay, Amok'un Ever,
\par 21 Hilkiya'nin Hasavya, Yedaya'nin Netanel.
\par 22 Levililer'den Elyasiv, Yoyada, Yohanan ve Yaddua'nin yasadigi günlerde, Pers Krali Darius'un döneminde, Levili aile baslarinin ve kâhinlerin kaydi tutuldu.
\par 23 Levili aile baslarinin listesi Elyasiv oglu Yohanan'in yasadigi döneme kadar tarihler kitabina yazildi.
\par 24 Levili önderlerden Hasavya, Serevya ve Kadmiel oglu Yesu bir yanda, kardesleri öbür yanda durur, Tanri adami Davut'un buyrugu uyarinca karsilikli övgüler ve sükürler sunarlardi.
\par 25 Mattanya, Bakbukya, Ovadya, Mesullam, Talmon ve Akkuv kapilarda nöbet tutarak kapilara yakin ambarlari korumakla görevliydiler.
\par 26 Yosadak oglu Yesu oglu Yoyakim'in ve Vali Nehemya ile Kâhin ve Bilgin Ezra'nin yasadigi dönemde bu insanlar görev yapti.
\par 27 Yerusalim surlari Tanri'ya adanacagi zaman, nerede bir Levili varsa aranip bulundu ve Yerusalim'e getirildi. Çünkü surlari sevinçle, sükranla, ezgilerle, zil, çenk ve lirlerle adamak istiyorlardi.
\par 28 Ezgiciler Yerusalim çevresindeki bölgelerden, Netofalilar'in köylerinden, Beytgilgal'dan, Geva ve Azmavet çevresinden toplandi. Yerusalim çevresinde köyler kurmuslardi.
\par 30 Kâhinlerle Levililer önce kendilerini, sonra halki, kapilari ve surlari paklama görevini yerine getirdiler.
\par 31 Yahudali önderleri surlarin üzerine çikardim. Sükrederek yürüsünler diye iki büyük gruba ayirdim. Birinci grup sagdan Gübre Kapisi'na dogru yürüdü.
\par 32 Arkalarindan Hosaya ve Yahudali önderlerin yarisi,
\par 33 Azarya, Ezra, Mesullam,
\par 34 Yahuda, Benyamin, Semaya, Yeremya
\par 35 ve borazan çalan bazi kâhinler izliyordu. Asaf oglu Zakkur oglu Mikaya oglu Mattanya oglu Semaya oglu Yonatan oglu Zekeriya
\par 36 ve kardesleri Semaya, Azarel, Milalay, Gilalay, Maay, Netanel, Yahuda ve Hanani Tanri adami Davut gibi çalgilariyla yürüyorlardi. Bilgin Ezra onlara öncülük ediyordu.
\par 37 Pinar Kapisi'ndan Davut Kenti'nin merdivenlerinden dosdogru surlara çiktilar; Davut'un sarayinin üst tarafindan geçerek doguya dogru, Su Kapisi'na kadar yürüdüler.
\par 38 Sükürler sunarak yürüyen öbür grupsa surlarin üzerinde sola dogru ilerliyordu. Halkin yarisiyla birlikte ben de onlari izledim. Firinlar Kulesi'nden geçip Genis Duvar'a kadar yürüdük.
\par 39 Efrayim Kapisi'ni, Eski Kapi'yi, Balik Kapisi'ni, Hananel Kulesi'ni, Hammea Kulesi'ni geçip Koyun Kapisi'na kadar gittik. Muhafizlar Kapisi'nda durduk.
\par 40 Sükürler sunarak yürüyen bu iki grup Tanri Tapinagi'nda durdu. Görevlilerin yarisiyla birlikte ben de durdum.
\par 41 Benim grubumda borazan çalan su kâhinler vardi: Elyakim, Maaseya, Minyamin, Mikaya, Elyoenay, Zekeriya, Hananya.
\par 42 Ayrica Maaseya, Semaya, Elazar, Uzzi, Yehohanan, Malkiya, Elam, Ezer. Ezgiciler Yizrahya'nin öncülügünde yüksek sesle ezgiler söylediler.
\par 43 O gün pek çok kurban kesildi. Halk cosku içindeydi, çünkü Tanri onlara büyük sevinç vermisti. Kadinlarla çocuklar da bu sevince katildilar. Yerusalim'den gelen sevinç sesleri uzaklardan duyulabiliyordu.
\par 44 Bu arada bagislarin, ilk ürünlerin ve ondaliklarin konacagi ambarlari gözetecek bazi kisiler görevlendirildi. Bunlar Kutsal Yasa'nin kâhinler ve Levililer için öngördügü yardimlari kentlerin çevresindeki kirsal bölgelerden toplayip ambarlara getirmekle sorumluydu. Yahudalilar kâhinlerle Levililer'in hizmetinden hosnuttu.
\par 45 Çünkü onlar Tanrilari'nin hizmetini ve paklama görevini yerine getiriyorlardi. Ezgicilerle kapi nöbetçileri de Davut'la oglu Süleyman'in buyruguna uygun olarak sorumluluklarini yerine getirdiler.
\par 46 Çünkü eskiden, Davut'un ve Asaf'in yasadigi yillarda, ezgicileri yönetenler vardi. Tanri'ya övgü ve sükür ezgileri söylenirdi.
\par 47 Zerubbabil'in ve Nehemya'nin yasadigi dönemde bütün Israil halki bagislariyla ezgicilerin ve kapi nöbetçilerinin ücretlerini gününde karsiladi. Levililer'in hakkini ayirdilar, Levililer de Harun soyundan gelenlerin hakkini ayirdi.

\chapter{13}

\par 1 O gün Musa'nin Kitabi halka okundu. Kitapta Ammonlular'la Moavlilar'in sonsuza dek Tanri'nin topluluguna giremeyecegi yaziliydi.
\par 2 Çünkü onlar Israil halkina ekmek ve su vermemekle kalmamis, Israilliler'e lanet okumasi için Balam'a da para vermislerdi. Ancak Tanrimiz laneti kutsamaya çevirmisti.
\par 3 Israil halki bu yasayi duyunca, bütün yabancilari ayri tutmaya basladi.
\par 4 Tanrimiz'in Tapinagi'nin ambarlarina Kâhin Elyasiv bakiyordu. Elyasiv Toviya'nin akrabasiydi.
\par 5 Bu yüzden ona büyük bir oda vermisti. Eskiden bu odaya tahil sunulari*, günnük, tapinak esyalari, ayrica Kutsal Yasa uyarinca Levililer'e, ezgicilere, tapinak kapi nöbetçilerine verilen bugdayin, yeni sarabin, zeytinyaginin ondaliklari ve kâhinlere verilen bagislar konulurdu.
\par 6 Ama bütün bunlar olup biterken ben Yerusalim'de degildim. Babil Krali Artahsasta'nin kralliginin otuz ikinci yilinda, onun yanina gitmistim. Bir süre sonra yine izin istedim
\par 7 ve Yerusalim'e döndüm. O zaman Elyasiv'in yaptigi kötülügü ögrendim. Tanri Tapinagi'nin avlusunda Toviya'ya oda vermisti.
\par 8 Buna çok canim sikildi. Toviya'nin bütün esyalarini odadan attim.
\par 9 Odalari temizlemeleri için buyruk verdim. Tanri Tapinagi'nin esyalarini, tahil sunularini, günnügü yine oraya koydurdum.
\par 10 Ayrica ögrendim ki, Levililer'in alacaklari verilmemis. Hizmeti yürüten Levililer'le ezgiciler tarlalarina geri dönmüsler.
\par 11 Görevlileri azarladim. "Tanri'nin Tapinagi neden ihmal edilmis?" diye sordum. Sonra bütün gidenleri toplayip islerinin basina koydum.
\par 12 Bütün Yahuda halki bugdayin, yeni sarabin, zeytinyaginin ondaligini yine ambarlara getirmeye basladi.
\par 13 Bu kez ambarlarin basina Kâhin Selemya'yi, Bilgin Sadok'u ve Levililer'den Pedaya'yi koydum. Mattanya oglu Zakkur oglu Hanan onlarin yardimcisiydi. Bunlar güvenilir insanlardi. Görevleri kardeslerinin paylarini bölüstürmekti.
\par 14 Ey Tanrim, beni animsa. Tapinagin için ve oradaki hizmetler için yaptigim iyi isleri hiçe sayma.
\par 15 O günlerde Yahuda'da bazi adamlarin Sabat Günü* üzüm siktiklarini gördüm. Bazilari da demet demet tahillarini eseklere yüklüyor, sarap, üzüm, incir ve çesitli yüklerle birlikte Sabat Günü Yerusalim'e getiriyorlardi. Sabat Günü bunlari sattiklari için onlari azarladim.
\par 16 Yerusalim'de yasayan Surlular balik ve çesitli mallar getirip Sabat Günü kentte Yahudalilar'a satiyorlardi.
\par 17 Yahudali soylulari azarlayarak, "Yaptiginiz kötülüge bakin!" dedim, "Sabat Günü'nü hiçe sayiyorsunuz.
\par 18 Atalariniz da ayni seyi yapmadi mi? Bu yüzden Tanrimiz basimiza ve bu kente bela yagdirmadi mi? Siz Sabat Günü'nü hiçe sayarak Tanri'nin öfkesini Israil'e karsi alevlendiriyorsunuz."
\par 19 Sabat'tan önceki aksam Yerusalim kapilarina gölge düsünce, kapilarin kapatilmasi ve Sabat sona erinceye kadar açilmamasi için buyruk verdim. Sabat Günü kente yük sokulmasin diye bazi adamlarimi kapilara yerlestirdim.
\par 20 Tüccarlarla çesitli esya saticilari bir iki kez geceyi Yerusalim'in disinda geçirdiler.
\par 21 Onlari uyardim: "Niçin surun dibinde geceliyorsunuz? Bir daha yaparsaniz size karsi zor kullanacagim." Bir daha Sabat Günü gelmediler.
\par 22 Sabat Günü'nün kutsalligini korumak için Levililer'e kendilerini paklasinlar ve gidip kapilarda nöbet tutsunlar diye buyruk verdim. Ey Tanrim, bunun için de beni animsa ve yüce sevgin uyarinca bana merhamet et.
\par 23 Ayrica o günlerde Asdotlu, Ammonlu, Moavli kadinlarla evlenmis Yahudiler gördüm.
\par 24 Çocuklarinin yarisi Asdot dilini ya da öbür halklarin dilini konusuyor, Yahudi dilini bilmiyorlardi.
\par 25 Adamlari azarladim, lanet okudum. Bazilarini dövüp saçlarini yoldum. Tanri'nin adiyla onlara ant içirdim ve, "Yabancilara kiz verip kiz almayacaksiniz" dedim,
\par 26 "Kral Süleyman bu yabanci kadinlar yüzünden günaha girmedi mi? Onca ulusun krallari arasinda Süleyman gibisi yoktu. Tanri onu öyle sevdi ki, bütün Israil'e kral yapti. Ama yabanci kadinlar onu bile günaha sürükledi.
\par 27 Simdi de siz yabanci kadinlarla evlenerek Tanrimiz'a ihanet ediyorsunuz. Yaptiginiz bu büyük kötülüge göz mü yumalim?"
\par 28 Baskâhin Elyasiv oglu Yoyada'nin ogullarindan biri Horonlu Sanballat'in kiziyla evliydi. Bu yüzden onu yanimdan kovdum.
\par 29 Ey Tanrim, onlari animsa; çünkü kâhinligi lekelediler, kâhinlerle ve Levililer'le yaptigin antlasmayi bozdular.
\par 30 Halki bütün yabancilardan arindirdim. Kâhinlerle Levililer'e görevlerini tek tek bildirdim.
\par 31 Belirli zamanlarda yakilmak için armagan edilen odunlari, getirilen ilk ürünleri düzene koydum. Ey Tanrim, bütün bunlari iyiligim için animsa.


\end{document}