\begin{document}

\title{Ester}


\chapter{1}

\par 1 Ahasveros Hoddu'dan Kûs'a uzanan bölgedeki yüz yirmi yedi ilin kraliydi.
\par 2 O sirada ülkeyi Sus Kalesi'ndeki tahtindan yönetiyordu.
\par 3 Kralliginin üçüncü yilinda bütün önderlerinin ve görevlilerinin onuruna bir sölen verdi. Pers ve Med ordu komutanlari, ileri gelenler ve il valileri de oradaydi.
\par 4 Ahasveros tam yüz seksen gün süren senliklerle kralliginin sonsuz zenginligini, büyüklügünün görkemini ve yüceligini gösterdi.
\par 5 Bunun ardindan, sarayinin avlusunda küçük büyük ayirmadan, Sus Kalesi'nde bulunan bütün halka yedi gün süren bir sölen verdi.
\par 6 Mermer sütunlar üzerindeki gümüs çemberlere mor ve beyaz renkli iplikten yapilmis sicimlerle baglanmis beyaz ve lacivert kumaslar asilmisti. Somaki, mermer, sedef ve pahali taslar dösenmis avluya altin ve gümüs sedirler yerlestirilmisti.
\par 7 Sarayin en iyi sarabi kralin cömertligine yarasir biçimde bol bol ve her biri degisik altin kupalar içinde sunuluyordu.
\par 8 Kralin buyrugu uyarinca, konuklar içki içmeye zorlanmadi. Kral saray hizmetkârlarina konuklarin dileklerini yerine getirmeleri için buyruk vermisti.
\par 9 O sirada Kraliçe Vasti de Kral Ahasveros'un sarayindaki kadinlara bir sölen veriyordu.
\par 10 Yedinci gün, sarabin etkisiyle keyiflenen Kral Ahasveros, hizmetindeki yedi haremagasina -Mehuman, Bizta, Harvona, Bigta, Avagta, Zetar ve Karkas'a- Kraliçe Vasti'yi basinda taciyla huzuruna getirmelerini buyurdu. Kraliçe Vasti güzeldi. Kral halka ve önderlere onun ne kadar güzel oldugunu göstermek istiyordu.
\par 12 Ama Kraliçe Vasti haremagalarinin kraldan getirdigi buyrugu reddedip gitmedi. Bunun üzerine kral çok kizdi, öfkesinden küplere bindi.
\par 13 Kral yasalari bilen bilge kisilerle görüstü. Çünkü kralin, yasalari ve adaleti bilen kisilere danismasi gelenektendi.
\par 14 Kendisine en yakin olan Karsena, Setar, Admata, Tarsis, Meres, Marsena ve Memukan onunla yüzyüze görüsebiliyorlardi. Pers ve Med Imparatorlugu'nun bu yedi önderi kralligin en üst yöneticileriydi.
\par 15 Kral Ahasveros onlara, "Kralin haremagalari araciligiyla gönderdigi buyruga uymayan Kraliçe Vasti'ye yasaya göre ne yapmali?" diye sordu.
\par 16 Memukan, kralin ve önderlerin önünde su yaniti verdi: "Kraliçe Vasti yalniz krala karsi degil, bütün önderlere ve kralin bütün illerindeki halklara karsi suç isledi.
\par 17 Bütün kadinlar, kraliçenin davranisiyla ilgili haberi duyunca, 'Kral Ahasveros Kraliçe Vasti'nin huzuruna getirilmesini buyurdu, ama kraliçe gitmedi' diyerek kocalarini küçümsemeye baslayacaklar.
\par 18 Bugün kraliçenin davranisini ögrenen Pers ve Medli soylu kadinlar da kralin soylu adamlarina ayni biçimde davranacak. Bu da alabildigine kadinlarin küçümsemesine, erkeklerin de öfkelenmesine yol açacak.
\par 19 Kral uygun görüyorsa ferman çikarsin; bu ferman Persler'le Medler'in degismeyen yasalarina eklensin. Buna göre Vasti bir daha Kral Ahasveros'un huzuruna çikmasin ve kral ondan daha iyi birini kraliçelige seçsin.
\par 20 Kralin fermani büyük kralliginin dört bir yanina ulasinca, ister soylu ister halktan olsun, bütün kadinlar kocalarina saygi gösterecektir."
\par 21 Bu sözler kralin ve önderlerinin hosuna gitti. Kral, Memukan'in önerisine uyarak,
\par 22 kralligin bütün illerine yazili buyruklar gönderdi. Her ile kendi isaretleriyle ve her halka kendi diliyle yazildi. Her erkegin kendi evinin egemeni oldugu her dilde vurgulandi.

\chapter{2}

\par 1 Bu olaylardan sonra öfkesi dinen Kral Ahasveros, Vasti'yi, yaptiklarini ve ona karsi alinan karari animsadi.
\par 2 Kralin özel hizmetkârlari, "Kral için genç, güzel, el degmemis kizlar aransin" dediler,
\par 3 "Kral, egemen oldugu bütün illerde görevliler atasin. Bu görevliler bütün genç, güzel, el degmemis kizlari toplayip Sus Kalesi'ndeki hareme getirsinler, kralin kizlardan sorumlu haremagasi Hegay'a teslim etsinler. Güzellesmeleri için ne gerekiyorsa verilsin.
\par 4 Sonunda kralin hosuna giden kiz, Vasti'nin yerine kraliçe olsun." Kral bu öneriyi begendi ve söyleneni yapti.
\par 5 Sus Kalesi'nde Mordekay adinda bir Yahudi vardi. Benyamin oymagindan olan Mordekay, Kis oglu Simi oglu Yair'in ogluydu.
\par 6 Kis, Babil Krali Nebukadnessar'in Yahuda Krali Yehoyakin ile birlikte Yerusalim'den sürgün ettigi kisilerden biriydi.
\par 7 Mordekay'in Hadassa adinda bir amca kizi vardi. Annesiyle babasini yitiren Hadassa'yi Mordekay evlat edinip büyütmüstü. Hadassa'nin öbür adi Ester'di; endami ve yüzü güzeldi.
\par 8 Kralin buyrugu ve fermani yayinlandiktan sonra çok sayida genç kiz Sus Kalesi'ne getirilip harem sorumlusu Hegay'a teslim edildi. Saraya getirilen kizlar arasinda Ester de vardi.
\par 9 Hegay Ester'i begendi ve ona ayricalik tanidi. En iyi biçimde beslenip güzellesmesi için ne gerekiyorsa hemen sagladi; ayrica kralin sarayindan seçilen yedi hizmetçiyi buyruguna verdi. Sonra onu hizmetçileriyle birlikte haremin en güzel bölümüne yerlestirdi.
\par 10 Ester halkini da, soyunu da açiklamadi. Çünkü Mordekay bunlari açiklamasini yasaklamisti.
\par 11 Mordekay, Ester'in nasil oldugunu ve ona nasil davranildigini ögrenmek için her gün haremin avlusunun önünde gezinip dururdu.
\par 12 Her genç kiz sirasi geldiginde Kral Ahasveros'un huzuruna çikacakti. Ama kizlarla ilgili kurallar uyarinca, önce on iki ay süren güzellik bakimini tamamlamasi gerekiyordu. Alti ay süreyle her kiza mür* yagi sürülüyor, alti ay da kremler, losyonlar uygulaniyordu.
\par 13 Kralin yanina girme sirasi gelen genç kiz, haremden her istedigini alip birlikte saraya götürürdü.
\par 14 Aksam kralin yanina giren kiz, ertesi sabah ikinci hareme, cariyelerden sorumlu haremagasi Saasgaz'in yönetimindeki hareme dönerdi. Yalniz kralin begendigi, adiyla çagirdigi kiz yeniden onun yanina girebilirdi.
\par 15 Kralin yanina girme sirasi Mordekay'in evlat edindigi Ester'e -Mordekay'in amcasi Avihayil'in kizina- gelince, Ester, kralin kizlardan sorumlu haremagasi Hegay'in kendisine önerdiklerinden baska bir sey istemedi. Kendisini gören herkesin begenisini kazandi.
\par 16 Ahasveros'un kralliginin yedinci yilinda, Tevet diye adlandirilan onuncu ayda*, Ester saraya, kralin yanina götürüldü.
\par 17 Kral Ester'i öbür kizlardan daha çok sevdi, en çok ondan hoslandi, en çok ona ayricalik tanidi. Kraliçelik tacini ona giydirip Vasti'nin yerine kraliçe yapti.
\par 18 Ardindan Ester'in onuruna büyük bir sölen verdi. Bu sölende bütün önderler ve görevliler hazir bulundu. Kral bütün illerde bayram ilan etti ve krallara yarasir cömertlikle armaganlar dagitti.
\par 19 Kizlar ikinci kez toplandiklarinda Mordekay, kralin kapi görevlilerinden biri olmustu.
\par 20 Ester, Mordekay'in verdigi buyruk uyarinca, soyunu ve halkini henüz açiklamamisti; kendisini büyüttügü günlerde oldugu gibi, Mordekay'in sözünü dinlemeye devam etti.
\par 21 Mordekay kralin kapi görevlisiyken, kapi nöbetçilerinden ikisi, Bigtan ve Teres, Kral Ahasveros'a öfkelendiler; onu öldürmek için firsat kollamaya basladilar.
\par 22 Durumu ögrenen Mordekay bunu Kraliçe Ester'e iletti; o da Mordekay adina krala bildirdi.
\par 23 Durum arastirildi; dogru oldugu anlasilinca da iki adam daragacina asildi ve olay kralin önünde tarih kayitlarina geçirildi.

\chapter{3}

\par 1 Bu olaylardan sonra Kral Ahasveros, Agakli Hammedata'nin oglu Haman'i yüksek bir göreve atayip onurlandirdi. Onu bütün önderlerden daha yetkili kildi.
\par 2 Kralin buyrugu üzerine saray kapisinda çalisan herkes Haman'in önünde egilip yere kapanirdi. Ama Mordekay ne egildi, ne de yere kapandi.
\par 3 Kralin kapi görevlileri Mordekay'a, "Kralin buyruguna neden karsi geliyorsun?" diye sordular.
\par 4 Görevliler ona bu soruyu her gün sordularsa da Mordekay onlara kulak asmadi. Bunun üzerine durumu Haman'a bildirdiler. Çünkü Mordekay onlara kendisinin Yahudi oldugunu söylemisti ve böyle davranmaya devam edip etmeyecegini görmek istiyorlardi.
\par 5 Haman, Mordekay'in egilip yere kapanmadigini görünce öfkeden kudurdu.
\par 6 Yalniz onu öldürmeyi düsünmekle kalmadi, onun hangi halktan geldigini bildigi için bütün halkini, Ahasveros'un egemenliginde yasayan bütün Yahudiler'i ortadan kaldirmaya karar verdi.
\par 7 Bu ise en uygun ayi ve günü belirlemek için Ahasveros'un kralliginin on ikinci yilinda, birinci ay* olan Nisan ayinda Haman'in önünde pur, yani kura çekildi. Kura, on ikinci ay olan Adar ayina düstü.
\par 8 Haman Kral Ahasveros'a söyle dedi: "Kralliginin bütün illerinde, öbür halklarin arasina dagilmis, onlardan ayri yasayan bir halk var. Yasalari bütün öbür halklarinkinden farkli; kendileri de kralin yasalarina uymazlar. Onlari kendi hallerine birakmak kralin çikarlarina uygun düsmez.
\par 9 Kral uygun görüyorsa, yok edilmeleri için yazili bir buyruk verilsin. Ben de hazineye ödenmek üzere kralin memurlarina on bin talant gümüs verecegim."
\par 10 Bunun üzerine kral mühür yüzügünü parmagindan çikartip Agakli Hammedata'nin oglu Yahudi düsmani Haman'a verdi.
\par 11 Ona, "Para sende kalsin; o halka da ne istersen yap" dedi.
\par 12 Birinci ayin on üçüncü günü kralin yazmanlari çagrildi ve Haman'in buyrugu her ile kendi isaretleriyle ve her halka kendi diliyle yazilarak satraplara*, il valilerine ve bütün halk önderlerine gönderildi. Buyruk Kral Ahasveros'un adini ve yüzügünün mührünü tasiyordu.
\par 13 Kralligin bütün illerine ulaklar araciligiyla mektuplar gönderildi. Bu mektuplar, on ikinci ay olan Adar ayinin on üçüncü günü, genç, yasli, kadin, çocuk, bütün Yahudiler'in bir günde öldürülüp yok edilmesini, kökünün kurutulup mal mülklerinin de yagmalanmasini buyuruyordu.
\par 14 Bu fermanin metni her ilde yasa olarak duyurulacak ve bütün halklara bildirilecekti. Öyle ki, herkes belirlenen gün için hazir olsun.
\par 15 Ulaklar kralin buyruguyla hemen yola çiktilar. Ferman Sus Kalesi'nde de duyuruldu. Sus halki saskinlik içindeyken kral ile Haman oturmus içki içiyorlardi.

\chapter{4}

\par 1 Mordekay olup bitenleri ögrenince giysilerini yirtti, çula sarinip basindan asagi kül döktü, yüksek sesle ve aciyla feryat ederek kent merkezine geldi.
\par 2 Varip sarayin kapisinda durdu. Çünkü çula sarinmis hiç kimse bu kapidan içeri giremezdi.
\par 3 Kralin buyrugunun ve fermaninin ulastigi her ilde Yahudiler büyük yas tuttular, aglayip feryat ettiler, oruç* tuttular. Birçogu da çula sarinip kül içinde yatti.
\par 4 Hizmetçileriyle haremagalari gelip Mordekay'in durumunu anlatinca, Kraliçe Ester çok sarsildi. Çulunu çikartip giyinmesi için Mordekay'a giysiler gönderdi, ama Mordekay bunlari kabul etmedi.
\par 5 Bunun üzerine Ester kralin kendi hizmetine atadigi haremagalarindan biri olan Hatak'i çagirtti; Mordekay'dan ne olup bittigini ve nedenini ögrenmesini buyurdu.
\par 6 Hatak saray kapisinin açildigi kent meydanina, Mordekay'in yanina gitti.
\par 7 Mordekay basina gelen her seyi ona anlatti. Yahudiler'in yok edilmesi için Haman'in saray hazinesine vaat ettigi paranin miktarini bile tam tamina ona bildirdi.
\par 8 Ester'e gösterip açiklamasi için Sus'ta yayimlanan, Yahudiler'in kökünün kurutulmasini isteyen fermanin bir kopyasini da ona verdi. Ester'in krala çikmasini, ondan merhamet dileyip kendi halki için yalvarmasini istedi.
\par 9 Hatak geri dönüp Mordekay'in söylediklerini Ester'e bildirdi.
\par 10 Ester Mordekay'a su haberi götürmesini buyurdu:
\par 11 "Kralin bütün adamlari ve illerinde yasayan halk biliyor ki, çagrilmadan sarayin iç avlusuna girip kralin yanina yaklasan her erkek ya da kadin için tek bir ceza vardir. Kral altin asasini uzatip canlarini bagislamadikça bu kisiler ölüme çarptirilir. Ben de otuz gündür kralin huzuruna çagrilmis degilim."
\par 12 Ester'in bu sözleri kendisine iletilince,
\par 13 Mordekay ona su yaniti götürmelerini istedi: "Sarayda yasadigin için bütün Yahudiler içinde kurtulacak tek kisinin sen olacagini sanma.
\par 14 Su anda susarsan, Yahudiler'e yardim ve kurtulus baska yerden gelecektir; ama sen ve babanin ev halki yok olacaksiniz. Kim bilir, belki de böyle bir gün için kraliçe oldun."
\par 15 Bunun üzerine Ester Mordekay'a su yaniti gönderdi:
\par 16 "Git, Sus'taki bütün Yahudiler'i topla; benim için oruç tutun; üç gün, üç gece hiçbir sey yemeyin, içmeyin. Hizmetçilerimle ben de sizin gibi oruç tutacagiz. Ardindan, kurala aykiri oldugu halde kralin huzuruna çikacagim; ölürsem ölürüm."
\par 17 Mordekay oradan ayrildi ve Ester'in söyledigi her seyi yapti.

\chapter{5}

\par 1 Üçüncü gün Ester kraliçe giysilerini kusanip sarayin iç avlusunda, taht odasinin önünde durdu. Kral bu odanin giris kapisinin karsisindaki tahtinda oturuyordu.
\par 2 Avluda bekleyen Kraliçe Ester'i görünce onu hosgörüyle karsilayip elindeki altin asayi ona dogru uzatti. Ester yaklasip asanin ucuna dokundu.
\par 3 Kral ona, "Ne istiyorsun Kraliçe Ester, dilegin ne?" diye sordu. "Kralligin yarisini bile istesen sana verilecektir."
\par 4 Ester, "Kral uygun görüyorsa, bugün kendisi için verecegim sölene Haman'la birlikte gelsin" diye karsilik verdi.
\par 5 Kral adamlarina, "Ester'in istegini yerine getirmek için Haman'i hemen çagirin" dedi. Böylece kralla Haman Ester'in verdigi sölene gittiler.
\par 6 Sarap içerlerken kral yine Ester'e sordu: "Söyle, ne istiyorsun? Ne istersen verilecek. Dilegin nedir? Kralligin yarisini bile istesen sana bagislanacak."
\par 7 Ester, "Istegim ve dilegim su" diye yanitladi,
\par 8 "Kral benden hosnutsa, istedigimi vermek, dilegimi yerine getirmek istiyorsa, kral ve Haman yarin kendileri için verecegim sölene gelsinler, o zaman kralin sorusunu yanitlarim."
\par 9 Haman o gün sölenden mutlu ve sevinçli ayrildi. Ama Mordekay'i sarayin kapisinda görünce ve onun ayaga kalkmadigini, kendisine saygi göstermedigini farkedince öfkeden kudurdu.
\par 10 Yine de kendini tuttu ve evine gitti. Sonra dostlarini ve esi Zeres'i çagirtti.
\par 11 Onlara sonsuz zenginliginden, çok sayidaki ogullarindan, kralin, kendisini nasil onurlandirdigindan, öbür önderlerinden ve görevlilerinden üstün tuttugundan söz etti.
\par 12 "Üstelik, Kraliçe Ester, verdigi sölene kralin yanisira yalniz beni çagirdi" diye ekledi, "Yarinki sölene de kralla birlikte beni davet etti.
\par 13 Ne var ki, o Yahudi Mordekay'i sarayin kapisinda otururken gördükçe bunlardan hiçbirinin gözümde degeri kalmiyor."
\par 14 Karisi Zeres ve bütün dostlari Haman'a söyle dediler: "Elli arsin yüksekliginde bir daragaci kurulsun. Sabah olunca kraldan Mordekay'i oraya astirmasini iste. Sonra da sevinç içinde kralla birlikte sölene gidersin." Haman öneriyi begendi ve daragacini hemen kurdurdu.

\chapter{6}

\par 1 O gece kralin uykusu kaçti; tarih kayitlarinin getirilip kendisine okunmasini buyurdu.
\par 2 Kayitlar Kral Ahasveros'u öldürmeyi tasarlamis olan iki görevliden söz ediyordu. Kapi nöbetçisi olarak görev yapmis olan Bigtan ve Teres adindaki bu iki adami Mordekay ele vermisti.
\par 3 Kral, "Bu yaptiklarindan dolayi Mordekay nasil onurlandirildi, ona ne ödül verildi?" diye sordu. Hizmetkârlar, "Onun için hiçbir sey yapilmadi" diye yanitladilar.
\par 4 Kral, "Avluda kim var?" diye sordu. O sirada Haman sarayin dis avlusuna yeni girmisti. Kraldan, hazirlattigi daragacina Mordekay'in asilmasini isteyecekti.
\par 5 Hizmetkârlar krala, "Haman avluda bekliyor" dediler. Kral, "Buraya gelsin" dedi.
\par 6 Haman içeri girince kral ona, "Kralin onurlandirmak istedigi biri için ne yapilmali?" diye sordu. "Kral benden baska kimi onurlandirmak isteyebilir ki?" diye düsünen Haman su yaniti verdi: "Kral onurlandirmak istedigi kisi için kendi giydigi bir kral giysisini ve üzerine bindigi sorguçlu ati getirtir,
\par 9 giysiyi ve ati en üst yöneticilerinden birine verir; o da kralin onurlandirmak istedigi kisiyi giydirip atin üstünde kent meydaninda gezdirir. Önden giderek, 'Kralin onurlandirmak istedigi kisiye böyle davranilir' diye bagirir."
\par 10 Kral Haman'a, "Hemen git" dedi, "Giysiyle ati al ve söylediklerini kralin kapi görevlisi Yahudi Mordekay için yap. Söylediklerinin hiçbirinde kusur etme."
\par 11 Böylece Haman giysiyi ve ati aldi, Mordekay'i giydirip atin üstünde kent meydaninda gezdirmeye basladi. Önden giderek, "Kralin onurlandirmak istedigi kisiye böyle davranilir" diye bagiriyordu.
\par 12 Sonra Mordekay saray kapisina döndü. Haman ise utanç içinde basini örterek çabucak evine gitti.
\par 13 Basina gelenleri karisi Zeres'e ve bütün dostlarina anlatti. Karisi Zeres ve danismanlari ona söyle dediler: "Önünde gerilemeye basladigin Mordekay Yahudi soyundansa, ona gücün yetmeyecek, önünde yok olup gideceksin."
\par 14 Onlar daha konusurken, kralin haremagalari gelip Haman'i apar topar Ester'in verecegi sölene götürdüler.

\chapter{7}

\par 1 Böylece kral ve Haman, Kraliçe Ester'in sölenine gittiler.
\par 2 O gün sarap içerlerken kral Ester'e yine sordu: "Istegin nedir, Kraliçe Ester? Ne istersen verilecek. Dilegin nedir? Kralligin yarisini bile istesen sana bagislanacak."
\par 3 Kraliçe Ester söyle yanitladi: "Ey kralim, eger benden hosnutsan ve uygun görüyorsan, istegim canimi bagislaman, dilegim de halkimi esirgemendir.
\par 4 Çünkü ben ve halkim öldürülüp yok edilmek, yeryüzünden silinmek üzere satildik. Eger köle ve cariye olarak satilmis olsaydik sesimi çikartmazdim; böyle bir sorun için krali rahatsiz etmek uygun olmazdi."
\par 5 Kral Ahasveros Kraliçe Ester'e, "Böyle bir seyi yapmaya cüret eden kim, nerede bu adam?" diye sordu.
\par 6 Ester, "Düsmanimiz, hasmimiz, iste bu kötü Haman'dir!" dedi. Haman kralla kraliçenin önünde dehsete kapildi.
\par 7 Kral öfkeyle içki masasindan kalkip sarayin bahçesine çikti. Haman ise Kraliçe Ester'den canini bagislamasini istemek için içerde kaldi. Çünkü kralin kendisini yok etmeye kararli oldugunu anlamisti.
\par 8 Kral sarayin bahçesinden sölen salonuna dönünce, Haman'i Ester'in uzandigi sedire kapanmis olarak gördü ve, "Bu adam sarayda, gözümün önünde kraliçeye bile el uzatmaya mi kalkiyor?" diye bagirdi. Kral sözlerini bitirir bitirmez Haman'in yüzünü örttüler.
\par 9 Krala hizmet eden haremagalarindan biri olan Harvona söyle dedi: "Bakin, krali uyarip hayatini kurtaran Mordekay için Haman'in hazirlattigi elli arsin yüksekligindeki daragaci Haman'in evinin önünde hazir duruyor." Kral, "Haman o daragacina asilsin!" diye buyurdu.
\par 10 Böylece Haman Mordekay için hazirlattigi daragacina asildi; kralin öfkesi de yatisti.

\chapter{8}

\par 1 O gün Kral Ahasveros Yahudi düsmani Haman'in malini mülkünü Kraliçe Ester'e verdi. Ester'in Mordekay'a yakinligini açiklamasi üzerine Mordekay kralin huzuruna kabul edildi.
\par 2 Kral, Haman'dan geri almis oldugu mühür yüzügünü parmagindan çikarip Mordekay'a verdi. Ester de onu Haman'in malinin mülkünün yöneticisi atadi.
\par 3 Ester yine kralla görüstü. Aglayarak onun ayaklarina kapandi. Agakli Haman'in Yahudiler'e karsi kurdugu düzene ve kötü tasariya engel olmasi için yalvardi.
\par 4 Kral altin asasini Ester'e dogru uzatinca Ester ayaga kalkip kralin önünde durdu
\par 5 ve söyle dedi: "Kral benden hosnutsa ve uygun görüyorsa, benden hoslaniyorsa ve dilegimi uygun buluyorsa, Agakli Hammedata oglu Haman'in kralligin bütün illerinde yasayan Yahudiler'in yok edilmesini buyurmak için yazdirdigi mektuplari yazili olarak geçersiz kilsin.
\par 6 Halkimin felakete ugradigini görmeye nasil dayanirim? Soydaslarimin öldürülmesine tanik olmaya nasil dayanirim?"
\par 7 Kral Ahasveros, Kraliçe Ester'e ve Yahudi Mordekay'a, "Bakin" dedi, "Haman'in malini mülkünü Ester'e verdim ve Yahudiler'i yok etmeyi tasarladigi için Haman'i daragacina astirdim.
\par 8 Ama kral adina yazilmis ve onun yüzügüyle mühürlenmis yaziyi kimse geçersiz kilamaz. Bunun için, uygun gördügünüz biçimde kral adina Yahudi sorunu konusunda simdi siz yazin ve kralin yüzügüyle mühürleyin."
\par 9 Bunun üzerine üçüncü ay* olan Sivan ayinin yirmi üçüncü günü kralin yazmanlari çagrildi. Mordekay'in buyurdugu her sey, Hoddu'dan Kûs'a dek uzanan bölgedeki yüz yirmi yedi ilde yasayan Yahudiler'e, satraplara*, vali ve önderlere yazildi. Her il için kendi isaretleri, her halk için kendi dili kullanildi. Yahudiler'e de kendi alfabelerinde ve kendi dillerinde yazildi.
\par 10 Mordekay Kral Ahasveros adina yazdirdigi mektuplari kralin yüzügüyle mühürledi ve kralin hizmetinde kullanilmak üzere yetistirilen atlara binmis ulaklarla her yere gönderdi.
\par 11 Kral mektuplarda Yahudiler'e bütün kentlerde toplanma ve kendilerini koruma hakkini veriyordu. Ayrica kendilerine, çocuklarina ve kadinlarina saldirabilecek herhangi bir düsman halkin ya da ilin silahli güçlerini öldürüp yok etmelerine, kökünü kurutmalarina ve mallarini mülklerini yagmalamalarina izin veriyordu.
\par 12 Bu izin Kral Ahasveros'un bütün illerinde tek bir gün -on ikinci ayin, yani Adar ayinin on üçüncü günü- geçerli olacakti.
\par 13 Bütün halklara duyurulan bu fermanin metni her ilde yasa yerine geçecekti. Böylece Yahudiler belirlenen gün düsmanlarindan öç almaya hazir olacaklardi.
\par 14 Kralin hizmetindeki atlara binen ulaklar, kralin buyruguna uyarak hemen dörtnala yola koyuldular. Ferman Sus Kalesi'nde de okundu.
\par 15 Mordekay, lacivert ve beyaz bir krallik giysisiyle, basinda büyük bir altin taç ve sirtinda ince ketenden mor bir pelerinle kralin huzurundan ayrildi. Sus Kenti sevinç çigliklariyla yankilandi.
\par 16 Yahudiler için aydinlik ve sevinç, mutluluk ve onur dolu günler baslamisti.
\par 17 Kralin buyrugu ve fermani ulastigi her ilde ve her kentte Yahudiler arasinda sevinç ve mutluluga yol açti. Sölenler düzenlendi, bir bayram havasi dogdu. Ülkedeki halklardan çok sayida kisi Yahudi oldu; çünkü Yahudi korkusu hepsini sarmisti.

\chapter{9}

\par 1 Kralin buyrugu ve fermani, on ikinci ay* olan Adar ayinin on üçüncü günü yerine getirilecekti. Yahudi düsmanlari o gün Yahudiler'i alt etmeyi ummuslardi, ama tam tersi oldu; Yahudiler kendilerinden nefret edenleri alt ettiler.
\par 2 Yahudiler kendilerini yok etmeyi tasarlayanlara saldirmak üzere Kral Ahasveros'un bütün illerindeki kentlerde bir araya geldiler. Hiç kimse onlara karsi koyamadi. Çünkü Yahudi korkusu bütün halklari sarmisti.
\par 3 Il önderleri, satraplar*, valiler ve kralin memurlari, Mordekay'dan korktuklari için Yahudiler'i desteklediler.
\par 4 Mordekay sarayda güçlü biriydi artik; ünü bütün illere ulasmisti. Gücü gittikçe artiyordu.
\par 5 Yahudiler bütün düsmanlarini kiliçtan geçirdiler, öldürdüler, yok ettiler. Kendilerinden nefret edenlere dilediklerini yaptilar.
\par 6 Sus Kalesi'nde bes yüz kisiyi öldürüp yok ettiler.
\par 7 Yahudi düsmani Hammedata oglu Haman'in on oglunu -Parsandata, Dalfon, Aspata, Porata, Adalya, Aridata, Parmasta, Arisay, Ariday ve Vayzata'yi- öldürdüler. Ama yagmaya girismediler.
\par 11 Sus Kalesi'nde öldürülenlerin sayisi ayni gün krala bildirildi.
\par 12 O da Kraliçe Ester'e, "Yahudiler Sus Kalesi'nde Haman'in on oglu dahil bes yüz kisiyi öldürüp yok etmisler" dedi, "Kim bilir, öbür illerimde neler yapmislardir? Istedigin nedir, sana vereyim; baska dilegin var mi, yerine getirilecektir."
\par 13 Ester, "Eger kral uygun görüyorsa, Sus'taki Yahudiler bugünkü fermanini yarin da uygulasinlar" dedi, "Haman'in on oglunun cesetleri de daragacina asilsin."
\par 14 Kral bu isteklerin yerine getirilmesini buyurdu. Sus'ta ferman çikarildi ve Haman'in on oglu asildi.
\par 15 Sus'taki Yahudiler Adar ayinin on dördüncü günü yeniden toplanarak kentte üç yüz kisi daha öldürdüler; ama yagmaya girismediler.
\par 16 Kralligin illerinde yasayan öbür Yahudiler de canlarini korumak ve düsmanlarindan kurtulmak için bir araya geldiler. Kendilerinden nefret edenlerden yetmis bes bin kisiyi öldürdüler, ama yagmaya girismediler.
\par 17 Bütün bunlar Adar ayinin on üçüncü günü oldu. Yahudiler on dördüncü gün dinlendiler ve o günü sölen ve eglence günü ilan ettiler.
\par 18 Sus'taki Yahudiler ise kendilerini savunmak için on üçüncü ve on dördüncü günler bir araya geldiler. On besinci gün de dinlendiler. O günü sölen ve eglence günü ilan ettiler.
\par 19 Tasradaki kentlerde yasayan Yahudiler iste bu nedenle Adar ayinin* on dördüncü gününü sölen ve eglence günü olarak kutlarlar ve birbirlerine yemek sunarlar.
\par 20 Mordekay bu olaylari kayda geçirdi. Ardindan Kral Ahasveros'un uzak, yakin bütün illerinde yasayan Yahudiler'e mektuplar gönderdi.
\par 21 Her yil Adar ayinin on dördüncü ve on besinci günlerini kutlamalarini buyurdu.
\par 22 Çünkü o günler, Yahudiler'in düsmanlarindan kurtuldugu günlerdir. O ay kederlerinin sevince, yaslarinin mutluluga dönüstügü aydir. Mordekay o günlerde sölenler düzenleyip eglenmelerini, birbirlerine yemek sunmalarini, yoksullara armaganlar vermelerini buyurdu.
\par 23 Böylece Yahudiler, Mordekay'in buyrugunu kabul ederek baslattiklari kutlamalari sürdürdüler.
\par 24 Çünkü bütün Yahudiler'in düsmani Agakli Hammedata oglu Haman onlari yok etmek için düzen kurmustu. Onlari ezip yok etmek için pur, yani kura çekmisti.
\par 25 Ama kral durumu ögrenince, Haman'in Yahudiler'e karsi kurdugu düzen geri tepti; kral, Haman'in ve ogullarinin daragacina asilmalari için yazili buyruklar verdi.
\par 26 Pur sözcügünden ötürü bu günlere Purim adi verildi. Böylece Yahudiler, Mordekay'in mektubunda yazili olanlardan, görüp geçirdiklerinden ve baslarina gelenlerden ötürü bu iki günü buyruldugu biçimde ve günlerde her yil kutlamayi kabul ettiler. Bu gelenek kendileri için, soylarindan olanlar ve onlara katilan herkes için geçerli olacakti.
\par 28 Böylece bu günler her ilde, her kentte ve her ailede kusaktan kusaga animsanacak ve kutlanacakti. Purim günleri Yahudiler için son bulmayacak ve bu günlerin anisi kusaklar boyu sürecekti.
\par 29 Avihayil'in kizi Kraliçe Ester ve Yahudi Mordekay Purim'le ilgili bu ikinci mektubu tam yetkiyle yazip uygulamaya koydular.
\par 30 Mordekay, Ahasveros'un egemenligi altindaki yüz yirmi yedi ilde yasayan Yahudiler'e esenlik ve güvenlik dilekleriyle dolu mektuplar gönderdi.
\par 31 Kraliçe Ester'le birlikte daha önce kararlastirdiklari gibi, Purim günlerini belirlenen tarihte kutlamalarini buyuruyordu. Bu kutlamalara kendilerinin de, soylarindan gelenlerin de katilmalarini, oruç* tutmada ve agit yakmada belirlenen kurallara uymalarini istedi.
\par 32 Purim'e iliskin bu düzenlemeler Ester'in buyruguyla onaylandi ve kayda geçirildi.

\chapter{10}

\par 1 Kral Ahasveros ülkeyi en uzak kiyilarina dek haraca baglamisti.
\par 2 Büyüklügü, kahramanliklari ve Mordekay'i her bakimdan nasil onurlandirdigi Pers ve Med krallarinin tarihinde yazilidir.
\par 3 Yahudi Mordekay, Kral Ahasveros'tan sonra ikinci adam olmustu. Yahudi soydaslari arasinda saygi gören ve çogunluk tarafindan sevilen biriydi. Çünkü halkinin iyiligini düsünüyor, bütün soydaslarinin esenligi için çaba gösteriyordu.


\end{document}