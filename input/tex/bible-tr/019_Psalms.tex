\begin{document}

\title{Mezmurlar}


\chapter{1}

\par 1 Ne mutlu o insana ki, kötülerin ögüdüyle yürümez, Günahkârlarin yolunda durmaz, Alaycilarin arasinda oturmaz.
\par 2 Ancak zevkini RAB'bin Yasasi'ndan alir Ve gece gündüz onun üzerinde derin derin düsünür.
\par 3 Böylesi akarsu kiyilarina dikilmis agaca benzer, Meyvesini mevsiminde verir, Yapragi hiç solmaz. Yaptigi her isi basarir.
\par 4 Kötüler böyle degil, Rüzgarin savurdugu saman çöpüne benzerler.
\par 5 Bu yüzden yargilaninca aklanamaz, Dogrular toplulugunda yer bulamaz günahkârlar.
\par 6 Çünkü RAB dogrularin yolunu gözetir, Kötülerin yolu ise ölüme götürür.

\chapter{2}

\par 1 Nedir uluslar arasindaki bu kargasa, Neden bos düzenler kurar bu halklar?
\par 2 Dünyanin krallari saf bagliyor, Hükümdarlar birlesiyor RAB'be ve meshettigi* krala karsi.
\par 3 "Koparalim onlarin kayislarini" diyorlar, "Atalim üzerimizden baglarini."
\par 4 Göklerde oturan Rab gülüyor, Onlarla egleniyor.
\par 5 Sonra öfkeyle uyariyor onlari, Gazabiyla dehsete düsürüyor
\par 6 Ve, "Ben kralimi Kutsal dagim Siyon'a oturttum" diyor.
\par 7 RAB'bin bildirisini ilan edecegim: Bana, "Sen benim oglumsun" dedi, "Bugün ben sana baba oldum.
\par 8 Dile benden, miras olarak sana uluslari, Mülk olarak yeryüzünün dört bucagini vereyim.
\par 9 Demir çomakla kiracaksin onlari, Çömlek gibi parçalayacaksin."
\par 10 Ey krallar, akilli olun! Ey dünya önderleri, ders alin!
\par 11 RAB'be korkuyla hizmet edin, Titreyerek sevinin.
\par 12 Ogulu öpün ki öfkelenmesin, Yoksa izlediginiz yolda mahvolursunuz. Çünkü öfkesi bir anda alevleniverir. Ne mutlu O'na siginanlara!

\chapter{3}

\par 1 Ya RAB, düsmanlarim ne kadar çogaldi, Hele bana karsi ayaklananlar!
\par 2 Birçogu benim için: "Tanri katinda ona kurtulus yok!" diyor.
\par 3 Ama sen, ya RAB, çevremde kalkansin, Onurum, basimi yukari kaldiran sensin.
\par 4 RAB'be seslenirim, Yanitlar beni kutsal dagindan.
\par 5 Yatar uyurum, Uyanir kalkarim, RAB destektir bana.
\par 6 Korkum yok Çevremi saran binlerce düsmandan.
\par 7 Ya RAB, kalk, ey Tanrim, kurtar beni! Vur bütün düsmanlarimin çenesine, Kir kötülerin dislerini.
\par 8 Kurtulus RAB'dedir, Halkinin üzerinde olsun bereketin!

\chapter{4}

\par 1 Sana seslenince yanitla beni, Ey adil Tanrim! Ferahlat beni sikintiya düstügümde, Lütfet bana, kulak ver duama.
\par 2 Ey insanlar, ne zamana dek Onurumu utanca çevireceksiniz? Ne zamana dek bos seylere gönül verecek, Yalan pesinde kosacaksiniz?
\par 3 Bilin ki, RAB sadik kulunu kendine ayirmistir, Ne zaman seslensem, duyar beni.
\par 4 Öfkelenebilirsiniz, ama günah islemeyin; Iyi düsünün yataginizda, susun.
\par 5 Dogruluk kurbanlari sunun RAB'be, O'na güvenin.
\par 6 "Kim bize iyilik yapacak?" diyen çok. Ya RAB, yüzünün isigiyla bizi aydinlat!
\par 7 Öyle bir sevinç verdin ki bana, Onlarin bol tahil ve yeni saraptan aldigi sevinçten fazla.
\par 8 Esenlik içinde yatar uyurum, Çünkü yalniz sen, ya RAB, Güvenlik içinde tutarsin beni.

\chapter{5}

\par 1 Sözlerime kulak ver, ya RAB, Iniltilerimi isit.
\par 2 Feryadimi dinle, ey Kralim ve Tanrim! Duam sanadir.
\par 3 Sabah sesimi duyarsin, ya RAB, Her sabah sana duami sunar, umutla beklerim.
\par 4 Çünkü sen kötülükten hoslanan Tanri degilsin, Kötülük senin yaninda barinmaz.
\par 5 Böbürlenenler önünde duramaz, Bütün suç isleyenlerden nefret duyar,
\par 6 Yalan söyleyenleri yok edersin; Ya RAB, sen eli kanlilardan, Aldaticilardan tiksinirsin.
\par 7 Bense bol sevgin sayesinde Kutsal tapinagina girecegim; Oraya dogru saygiyla egilecegim.
\par 8 Yol göster bana dogrulugunla, ya RAB, Düsmanlarima karsi! Yolunu önümde düzle.
\par 9 Çünkü onlarin sözüne güvenilmez, Yürekleri yikim dolu. Agizlari açik birer mezardir, Yaltaklanir dururlar.
\par 10 Ey Tanri, onlari suçlu çikar! Kurduklari düzen yikimlarina yol açsin. Kov onlari sayisiz isyanlari yüzünden. Çünkü sana karsi ayaklandilar.
\par 11 Sevinsin sana siginan herkes, Sevinç çigliklari atsin sürekli, Kanat ger üzerlerine; Sevinçle cossun adini sevenler sende.
\par 12 Çünkü sen dogru kisiyi kutsarsin, ya RAB, Çevresini kalkan gibi lütfunla sararsin.

\chapter{6}

\par 1 Ya RAB, öfkeyle azarlama beni, Gazapla yola getirme.
\par 2 Lütfet bana, ya RAB, bitkinim; Sifa ver bana, ya RAB, kemiklerim sizliyor,
\par 3 Çok aci çekiyorum. Ah, ya RAB! Ne zamana dek sürecek bu?
\par 4 Gel, ya RAB, kurtar beni, Yardim et sevginden dolayi.
\par 5 Çünkü ölüler arasinda kimse seni anmaz, Kim sükür sunar sana ölüler diyarindan?
\par 6 Inleye inleye bittim, Dösegim su içinde bütün gece aglamaktan, Yatagim sirilsiklam gözyaslarimdan.
\par 7 Kederden gözlerimin feri sönüyor, Zayifliyor gözlerim düsmanlarim yüzünden.
\par 8 Ey kötülük yapanlar, Uzak durun benden, Çünkü RAB aglayisimi isitti.
\par 9 Yalvarisimi duydu, Duami kabul etti.
\par 10 Bütün düsmanlarim utanacak, Hepsini dehset saracak, Ansizin geri dönecekler utanç içinde.

\chapter{7}

\par 1 Sana siginiyorum, ya RAB Tanrim! Pesime düsenlerden kurtar beni, Özgür kil.
\par 2 Yoksa aslan gibi parçalayacaklar beni, Kurtaracak biri yok diye, Lime lime edecekler etimi.
\par 3 Ya RAB Tanrim, eger sunu yaptiysam: Birine haksizlik ettiysem,
\par 4 Dostuma ihanet ettiysem, Düsmanimi nedensiz soyduysam,
\par 5 Ardima düssün düsman, Yakalasin beni, Canimi yerde çignesin, Ayak altina alsin onurumu. *
\par 6 Öfkeyle kalk, ya RAB! Düsmanlarimin gazabina karsi çik! Benim için uyan! Buyur, adalet olsun.
\par 7 Uluslar toplulugu çevreni sarsin, Onlari yüce katindan yönet.
\par 8 RAB halklari yargilar; Beni de yargila, ya RAB, Dogruluguma, dürüstlügüme göre.
\par 9 Ey adil Tanrim! Kötülerin kötülügü son bulsun, Dogrular güvene kavussun, Sen ki akillari, gönülleri sinarsin.
\par 10 Tanri kalkan gibi yanibasimda, Temiz yüreklileri O kurtarir.
\par 11 Tanri adil bir yargiçtir, Öyle bir Tanri ki, her gün öfke saçar.
\par 12 Kötüler yola gelmezse, Tanri kilicini biler, Yayini gerip hedefine kurar.
\par 13 Hazir bekler ölümcül silahlari, Alevli oklari.
\par 14 Iste kötü insan kötülük sancilari çekiyor, Fesada gebe kalmis, Yalan doguruyor.
\par 15 Bir kuyu açip kaziyor, Kazdigi kuyuya kendisi düsüyor.
\par 16 Kötülügü kendi basina gelecek, Zorbaligi kendi tepesine inecek.
\par 17 Sükredeyim dogrulugu için RAB'be, Yüce RAB'bin adini ilahilerle öveyim.

\chapter{8}

\par 1 Ey Egemenimiz RAB, Ne yüce adin var yeryüzünün tümünde! Gökyüzünü görkeminle kapladin.
\par 2 Çocuklarin, hatta emziktekilerin sesiyle Set çektin hasimlarina, Düsmani, öç alani yok etmek için.
\par 3 Seyrederken ellerinin eseri olan gökleri, Oraya koydugun ayi ve yildizlari,
\par 4 Soruyorum kendi kendime: "Insan ne ki, onu anasin, Ya da insanoglu ne ki, ona ilgi gösteresin?"
\par 5 Nerdeyse bir tanri yaptin onu, Basina yücelik ve onur tacini koydun.
\par 6 Ellerinin yapitlari üzerine onu egemen kildin, Her seyi ayaklarinin altina serdin;
\par 7 Davarlari, sigirlari, Yabanil hayvanlari,
\par 8 Gökteki kuslari, denizdeki baliklari, Denizde kipirdasan bütün canlilari.
\par 9 Ey Egemenimiz RAB, Ne yüce adin var yeryüzünün tümünde!

\chapter{9}

\par 1 Ya RAB, bütün yüregimle sana sükredecegim, Yaptigin harikalarin hepsini anlatacagim.
\par 2 Sende sevinç bulacak, cosacagim, Adini ilahilerle övecegim, ey Yüceler Yücesi!
\par 3 Düsmanlarim geri çekilirken, Sendeleyip ölüyorlar senin önünde.
\par 4 Çünkü hakkimi, davami sen savundun, Adil yargiç olarak tahta oturdun.
\par 5 Uluslari azarladin, kötüleri yok ettin, Sonsuza dek adlarini sildin.
\par 6 Yok olup gitti düsmanlar sonsuza dek, Kökünden söktün kentlerini, Anilari bile silinip gitti.
\par 7 Oysa RAB sonsuza dek egemenlik sürer, Yargi için kurmustur tahtini;
\par 8 O yönetir dogrulukla dünyayi, O yargilar adaletle halklari.
\par 9 RAB ezilenler için bir siginak, Sikintili günlerde bir kaledir.
\par 10 Seni taniyanlar sana güvenir, Çünkü sana yönelenleri hiç terk etmedin, ya RAB.
\par 11 Siyon'da oturan RAB'bi ilahilerle övün! Yaptiklarini halklar arasinda duyurun!
\par 12 Çünkü dökülen kanin hesabini soran animsar, Ezilenlerin feryadini unutmaz.
\par 13 Aci bana, ya RAB! Ey beni ölümün esiginden kurtaran, Benden nefret edenler yüzünden çektigim sikintiya bak!
\par 14 Öyle ki, övgüye deger islerini anlatayim, Siyon Kenti'nin kapilarinda Sagladigin kurtulusla sevineyim.
\par 15 Uluslar kendi kazdiklari kuyuya düstü, Ayaklari gizledikleri aga takildi.
\par 16 Adil yargilariyla RAB kendini gösterdi, Kötüler kendi kurduklari tuzaga düstü. -Higayon
\par 17 Kötüler ölüler diyarina gidecek, Tanri'yi unutan bütün uluslar...
\par 18 Ama yoksul büsbütün unutulmayacak, Mazlumun umudu sonsuza dek kirilmayacak.
\par 19 Kalk, ya RAB! Insan galip çikmasin, Huzurunda yargilansin uluslar!
\par 20 Onlara dehset saç, ya RAB! Sadece insan olduklarini bilsin uluslar.

\chapter{10}

\par 1 Ya RAB, neden uzak duruyorsun, Sikintili günlerde kendini gizliyorsun?
\par 2 Kötüler gururla mazlumlari avliyor, Mazlumlar kötülerin kurdugu tuzaga düsüyor.
\par 3 Kötü insan içindeki isteklerle övünür, Açgözlü insan RAB'be lanet okur, O'nu hor görür.
\par 4 Kendini begenmis kötü insan Tanri'ya yönelmez, Hep, "Tanri yok!" diye düsünür.
\par 5 Kötülerin yollari her zaman basariya götürür. Öyle yücedir ki senin yargilarin, Kötüler anlayamaz, düsmanina burun kivirir.
\par 6 Içinden, "Ben sarsilmam" der, "Hiçbir zaman sikintiya düsmem."
\par 7 Agzi lanet, hile ve zulüm dolu, Dilinin altinda kötülük ve fesat sakli.
\par 8 Köylerin çevresinde pusu kurar, Masumu gizli yerlerde öldürür, Çaresizi sinsice gözler.
\par 9 Gizli yerlerde pusuya yatar Çaliliktaki aslan gibi, Kapmak için mazlumu bekler Ve agina düsürüp yakalar.
\par 10 Kurbanlari çaresiz çöker, Saldiranin üstün gücü altinda ezilir.
\par 11 Kötü insan içinden, "Tanri unuttu" der, "Örttü yüzünü, asla göremez."
\par 12 Kalk, ya RAB, kaldir elini, ey Tanri! Mazlumlari unutma!
\par 13 Neden kötü insan seni hor görsün, Içinden, "Tanri hesap sormaz" desin?
\par 14 Oysa sen sikinti ve aci çekenleri görürsün, Yardim etmek için onlari izlersin; Çaresizler sana dayanir, Öksüzün yardimcisi sensin.
\par 15 Kötünün, haksizin kolunu kir, Sormadik hesap kalmasin yaptigi kötülükten.
\par 16 RAB sonsuza dek kral kalacak, Uluslar O'nun ülkesinden temizlenecek.
\par 17 Mazlumlarin dilegini duyarsin, ya RAB, Yüreklendirirsin onlari, Kulagin hep üzerlerinde;
\par 18 Öksüze, düsküne hakkini vermek için, Bir daha dehset saçmasin ölümlü insan.

\chapter{11}

\par 1 Ben RAB'be siginirim, Nasil dersiniz bana, "Kus gibi kaç daglara.
\par 2 Bak, kötüler yaylarini geriyor, Temiz yürekli insanlari Karanlikta vurmak için Oklarini kirisine koyuyor.
\par 3 Temeller yikilirsa, Ne yapabilir dogru insan?"
\par 4 RAB kutsal tapinagindadir, O'nun tahti göklerdedir, Bütün insanlari görür, Herkesi sinar.
\par 5 RAB dogru insani sinar, Kötüden, zorbaligi sevenden tiksinir.
\par 6 Kötülerin üzerine kizgin korlar ve kükürt yagdiracak, Paylarina düsen kâse* kavurucu rüzgar olacak.
\par 7 Çünkü RAB dogrudur, dogrulari sever; Dürüst insanlar O'nun yüzünü görecek.

\chapter{12}

\par 1 Kurtar beni, ya RAB, sadik kulun kalmadi, Güvenilir insanlar yok oldu.
\par 2 Herkes birbirine yalan söylüyor, Dalkavukluk, ikiyüzlülük ediyor.
\par 3 Sustursun RAB dalkavuklarin agzini, Büyüklenen dilleri.
\par 4 Onlar ki, "Dilimizle kazaniriz, Dudaklarimiz emrimizde, Kim bize efendilik edebilir?" derler.
\par 5 "Simdi kalkacagim" diyor RAB, "Çünkü mazlumlar eziliyor, Yoksullar inliyor, Özledikleri kurtulusu verecegim onlara."
\par 6 RAB'bin sözleri pak sözlerdir; Toprak ocakta eritilmis, Yedi kez aritilmis gümüse benzer.
\par 7 Sen onlari koru, ya RAB, Bu kötü kusaktan hep uzak tut!
\par 8 Insanlar arasinda alçaklik ragbet görünce, Kötüler her yanda dolasir oldu.

\chapter{13}

\par 1 Ne zamana dek, ya RAB, Sonsuza dek mi beni unutacaksin? Ne zamana dek yüzünü benden gizleyeceksin?
\par 2 Ne zamana dek içimde tasa, Yüregimde hep keder olacak? Ne zamana dek düsmanim bana üstün çikacak?
\par 3 Gör halimi, ya RAB, yanitla Tanrim, Gözlerimi aç, ölüm uykusuna dalmayayim.
\par 4 Düsmanlarim, "Onu yendik!" demesin, Sarsildigimda hasimlarim sevinmesin.
\par 5 Ben senin sevgine güveniyorum, Yüregim kurtarisinla cossun.
\par 6 Ezgiler söyleyecegim sana, ya RAB, Çünkü iyilik ettin bana.

\chapter{14}

\par 1 Akilsiz içinden, "Tanri yok!" der. Insanlar bozuldu, igrençlik aldi yürüdü, Iyilik eden yok.
\par 2 RAB göklerden bakar oldu insanlara, Akilli, Tanri'yi arayan biri var mi diye.
\par 3 Hepsi sapti, Tümü yozlasti, Iyilik eden yok, Bir kisi bile!
\par 4 Suç isleyenlerin hiçbiri görmüyor mu? Halkimi ekmek yer gibi yiyor, RAB'be yakarmiyorlar.
\par 5 Dehsete düsecekler yeryüzünde, Çünkü Tanri dogrularin yanindadir.
\par 6 Mazlumun tasarilarini bosa çikarirdiniz, Ama RAB onun siginagidir.
\par 7 Keske Israil'in kurtulusu Siyon'dan gelse! RAB halkini eski gönencine kavusturunca, Yakup soyu sevinecek, Israil halki cosacak.

\chapter{15}

\par 1 Ya RAB, çadirina kim konuk olabilir? Kutsal daginda kim oturabilir?
\par 2 Kusursuz yasam süren, adil davranan, Yürekten gerçegi söyleyen.
\par 3 Iftira etmez, Dostuna zarar vermez, Komsusuna kara çalmaz böylesi.
\par 4 Asagilik insanlari hor görür, Ama RAB'den korkanlara saygi duyar. Kendi zararina ant içse bile, dönmez andindan.
\par 5 Parasini faize vermez, Suçsuza karsi rüsvet almaz. Böyle yasayan asla sarsilmayacak.

\chapter{16}

\par 1 Koru beni, ey Tanri, Çünkü sana siginiyorum.
\par 2 RAB'be dedim ki, "Efendim sensin. Senden öte mutluluk yok benim için."
\par 3 Ülkedeki kutsallara gelince, Soyludur onlar, biricik zevkim onlardir.
\par 4 Baska ilahlarin ardinca kosanlarin derdi artacak. Onlarin kan sunularini dökmeyecegim, Adlarini agzima almayacagim.
\par 5 Benim payima, Benim kâseme düsen sensin, ya RAB; Yasamim senin ellerinde.
\par 6 Payima ne güzel yerler düstü, Ne harika bir mirasim var!
\par 7 Övgüler sunarim bana ögüt veren RAB'be, Geceleri bile vicdanim uyarir beni.
\par 8 Gözümü RAB'den ayirmam, Sagimda durdugu için sarsilmam.
\par 9 Bu nedenle içim sevinç dolu, yüregim cosuyor, Bedenim güven içinde.
\par 10 Çünkü sen beni ölüler diyarina terk etmezsin, Sadik kulunun çürümesine izin vermezsin.
\par 11 Yasam yolunu bana bildirirsin. Bol sevinç vardir senin huzurunda, Sag elinden mutluluk eksilmez.

\chapter{17}

\par 1 Hakli davami dinle, ya RAB, Feryadimi isit! Hilesiz dudaklardan çikan duama kulak ver!
\par 2 Hakli çikar beni, Çünkü sen gerçegi görürsün.
\par 3 Yüregimi yokladin, Gece denedin, Sinadin beni, Kötü bir sey bulmadin; Kararliyim, agzimdan kötü söz çikmaz,
\par 4 Baskalarinin yaptiklarina gelince, Ben senin sözlerine uyarak Siddet yollarindan kaçindim.
\par 5 Siki adimlarla senin yollarini tuttum, Kaymadi ayaklarim.
\par 6 Sana yakariyorum, ey Tanri, Çünkü beni yanitlarsin; Kulak ver bana, dinle söylediklerimi!
\par 7 Göster harika sevgini, Ey sana siginanlari saldirganlardan sag eliyle kurtaran!
\par 8 Koru beni gözbebegi gibi; Kanatlarinin gölgesine gizle
\par 9 Kötülerin saldirisindan, Çevremi saran ölümcül düsmanlarimdan.
\par 10 Yürekleri yag baglamis, Agizlari büyük laflar ediyor.
\par 11 Izimi buldular, üzerime geliyorlar, Yere vurmak için gözetliyorlar.
\par 12 Tipki parçalamak için sabirsizlanan bir aslan, Pusuya yatan genç bir aslan gibi.
\par 13 Kalk, ya RAB, kes önlerini, eg baslarini! Kilicinla kurtar canimi kötülerden,
\par 14 Elinle bu insanlardan, ya RAB, Yasam payi bu dünyada olan insanlardan. Varsin karinlari verecegin cezalara doysun, Çocuklari da yiyip doysun, Artani torunlarina kalsin!
\par 15 Ama ben dogruluk sayesinde yüzünü görecegim senin, Uyaninca suretini görmeye doyacagim.

\chapter{18}

\par 1 Seni seviyorum, gücüm sensin, ya RAB!
\par 2 RAB benim kayam, siginagim, kurtaricimdir, Tanrim, kayam, siginacak yerimdir, Kalkanim, güçlü kurtaricim, korunagimdir!
\par 3 Övgüye deger RAB'be seslenir, Kurtulurum düsmanlarimdan.
\par 4 Ölüm iplerine dolanmistim, Yikim selleri basmisti beni,
\par 5 Ölüler diyarinin baglari sarmisti, Ölüm tuzaklari çikmisti karsima.
\par 6 Sikinti içinde RAB'be yakardim, Yardima çagirdim Tanrim'i. Tapinagindan sesimi duydu, Haykirisim kulaklarina ulasti.
\par 7 O zaman yeryüzü sarsilip sallandi, Titreyip sarsildi daglarin temelleri, Çünkü RAB öfkelenmisti.
\par 8 Burnundan duman yükseldi, Agzindan kavurucu ates Ve korlar fiskirdi.
\par 9 Kara buluta basarak Gökleri yarip indi.
\par 10 Bir Keruv'a* binip uçtu, Rüzgar kanatlar takarak hizla geldi.
\par 11 Karanligi örtündü, Kara bulutlari kendine çardak yapti.
\par 12 Varliginin pariltisindan, Bulutlardan dolu ve korlar savruluyordu.
\par 13 RAB göklerden gürledi, Duyurdu sesini Yüceler Yücesi, Dolu ve alevli korlarla.
\par 14 Savurup oklarini düsmanlarini dagitti, Simsek çaktirarak onlari saskina çevirdi.
\par 15 Denizin dibi göründü, Yeryüzünün temelleri açiga çikti, ya RAB, Senin azarlamandan, Burnundan çikan güçlü soluktan.
\par 16 RAB yukaridan elini uzatip tuttu, Çikardi beni derin sulardan.
\par 17 Beni zorlu düsmanimdan, Benden nefret edenlerden kurtardi, Çünkü onlar benden güçlüydü.
\par 18 Felaket günümde karsima dikildiler, Ama RAB bana destek oldu.
\par 19 Beni huzura kavusturdu, Kurtardi, çünkü benden hosnut kaldi.
\par 20 RAB dogrulugumun karsiligini verdi, Beni temiz ellerime göre ödüllendirdi.
\par 21 Çünkü RAB'bin yolunda yürüdüm, Tanrim'dan uzaklasarak kötülük yapmadim.
\par 22 O'nun bütün ilkelerini göz önünde tuttum, Kurallarindan ayrilmadim.
\par 23 O'nun gözünde kusursuzdum, Suç islemekten sakindim.
\par 24 Bu yüzden RAB beni dogruluguma Ve gözünde pak olan ellerime göre ödüllendirdi.
\par 25 Sadik kuluna sadakat gösterir, Kusursuz olana kusursuz davranirsin.
\par 26 Pak olanla pak olur, Egriye egri davranirsin.
\par 27 Alçakgönüllüleri kurtarir, Gururlularin basini egersin.
\par 28 Isigimin kaynagi sensin, ya RAB, Tanrim! Karanligimi aydinlatirsin.
\par 29 Desteginle akincilara saldirir, Seninle surlari asarim, Tanrim.
\par 30 Tanri'nin yolu kusursuzdur, RAB'bin sözü aridir. O kendisine siginan herkesin kalkanidir.
\par 31 Var mi RAB'den baska tanri? Tanrimiz'dan baska kaya var mi?
\par 32 Tanri beni güçle donatir, Yolumu kusursuz kilar.
\par 33 Ayaklar verdi bana, geyiklerinki gibi, Doruklarda tutar beni.
\par 34 Bana savasmayi ögretti, Kollarimla tunç* bir yayi gereyim diye.
\par 35 Bana zafer kalkanini bagislarsin, Sag elin destekler, Alçakgönüllülügün yüceltir beni.
\par 36 Bastigim yerleri genisletirsin, Burkulmaz bileklerim.
\par 37 Kovalayip yetistim düsmanlarima, Hepsi yok olmadan geri dönmedim.
\par 38 Ezdim onlari, kalkamaz oldular, Ayaklarimin altina serildiler.
\par 39 Savas için beni güçle donattin, Bana baskaldiranlari önümde yere serdin.
\par 40 Düsmanlarimi kaçmak zorunda biraktin, Benden nefret edenleri yok ettim.
\par 41 Feryat ettiler, ama kurtaran çikmadi; RAB'bi çagirdilar, ama O yanit vermedi.
\par 42 Ezdim onlari, rüzgarin savurdugu toza döndüler, Sokak çamuru gibi savurup attim.
\par 43 Halkimin çekismelerinden beni kurtardin, Uluslarin önderi yaptin, Tanimadigim halklar bana kulluk ediyor.
\par 44 Duyar duymaz sözümü dinlediler, Yabancilar bana yaltaklandilar.
\par 45 Yabancilarin betleri benizleri atti, Titreyerek çiktilar kalelerinden.
\par 46 RAB yasiyor! Kayam'a övgüler olsun! Yücelsin kurtaricim Tanri!
\par 47 O'dur öcümü alan, Halklari bana bagimli kilan.
\par 48 Düsmanlarimdan kurtarir, Baskaldiranlardan üstün kilar beni, Zorbalarin elinden alir.
\par 49 Bunun için uluslar arasinda sana sükredecegim, ya RAB, Adini ilahilerle övecegim.
\par 50 RAB kralini büyük zaferlere ulastirir, Meshettigi* krala, Davut'a ve soyuna Sonsuza dek sevgi gösterir.

\chapter{19}

\par 1 Gökler Tanri'nin görkemini açiklamakta, Gökkubbe ellerinin eserini duyurmakta.
\par 2 Gün güne söz söyler, Gece geceye bilgi verir.
\par 3 Ne söz geçer orada, ne de konusma, Sesleri duyulmaz.
\par 4 Ama sesleri yeryüzünü dolasir, Sözleri dünyanin dört bucagina ulasir. Günes için göklerde çadir kurdu Tanri.
\par 5 Gerdekten çikan güveye benzer günes, Kosuya çikacak atlet gibi sevinir.
\par 6 Gögün bir ucundan çikar, Öbür ucuna döner, Hiçbir sey gizlenmez sicakligindan.
\par 7 RAB'bin yasasi yetkindir, cana can katar, RAB'bin buyruklari güvenilirdir, Saf adama bilgelik verir,
\par 8 RAB'bin kurallari dogrudur, yüregi sevindirir, RAB'bin buyruklari aridir, gözleri aydinlatir.
\par 9 RAB korkusu paktir, sonsuza dek kalir, RAB'bin ilkeleri gerçek, tamamen adildir.
\par 10 Onlara altindan, bol miktarda saf altindan çok istek duyulur, Onlar baldan, süzme petek balindan tatlidir.
\par 11 Uyarirlar kulunu, Onlara uyanlarin ödülü büyüktür.
\par 12 Kim yanlislarini görebilir? Bagisla göremedigim kusurlarimi,
\par 13 Bilerek islenen günahlardan koru kulunu, Izin verme bana egemen olmalarina! O zaman büyük isyandan uzak, Kusursuz olurum.
\par 14 Agzimdan çikan sözler, Yüregimdeki düsünceler, Kabul görsün senin önünde, Ya RAB, kayam, kurtaricim benim!

\chapter{20}

\par 1 Sikintili gününde RAB seni yanitlasin, Yakup'un Tanrisi'nin adi seni korusun!
\par 2 Yardim göndersin sana kutsal yerden, Siyon'dan destek versin.
\par 3 Bütün tahil sunularini* animsasin, Yakmalik sunularini* kabul etsin! *
\par 4 Gönlünce versin sana, Bütün tasarilarini gerçeklestirsin!
\par 5 O zaman zaferini sevinç çigliklariyla kutlayacagiz, Tanrimiz'in adiyla sancaklarimizi dikecegiz. RAB senin bütün dileklerini yerine getirsin.
\par 6 Simdi anladim ki, RAB meshettigi* krali kurtariyor, Sag elinin kurtarici gücüyle Kutsal göklerinden ona yanit veriyor.
\par 7 Bazilari savas arabalarina, Bazilari atlarina güvenir, Bizse Tanrimiz RAB'be güveniriz.
\par 8 Onlar çöküyor, düsüyorlar; Bizse kalkiyor, dimdik duruyoruz.
\par 9 Ya RAB, krali kurtar! Yanitla bizi sana yakardigimiz gün!

\chapter{21}

\par 1 Ya RAB, kral seviniyor gösterdigin güce. Sevinçten cosuyor verdigin zaferle!
\par 2 Gönlünün istedigini verdin, Agzindan çikan dilegi geri çevirmedin. *
\par 3 Onu güzel armaganlarla karsiladin, Basina saf altindan taç koydun.
\par 4 Senden yasam istedi, verdin ona: Uzun, sonsuz bir ömür.
\par 5 Sagladigin zaferle büyük yücelige eristi, Onu görkem ve büyüklükle donattin.
\par 6 Üzerine sürekli bereket yagdirdin, Varliginla onu sevince bogdun.
\par 7 Çünkü kral RAB'be güvenir, Yüceler Yücesi'nin sevgisi sayesinde sarsilmaz.
\par 8 Elin bütün düsmanlarina erisecek, Sag elin senden nefret edenlere uzanacak.
\par 9 Öfkelendigin an, ya RAB, Kizgin firina döndüreceksin onlari; Gazapla yutacak, Atesle tüketeceksin.
\par 10 Yok edeceksin çocuklarini yeryüzünden, Soylarini insanlar arasindan.
\par 11 Düzenler kursalar sana, Aldatmaya çalissalar, Yine de basarili olamazlar.
\par 12 Çünkü sirtlarini döndüreceksin, Yayini yüzlerine dogru gerince.
\par 13 Yüceligini göster, ya RAB, gücünle! Ezgiler söyleyip ilahilerle övecegiz kudretini.

\chapter{22}

\par 1 Tanrim, Tanrim, beni neden terk ettin? Niçin bana yardim etmekten, Haykirisima kulak vermekten uzak duruyorsun?
\par 2 Ey Tanrim, gündüz sesleniyorum, yanit vermiyorsun, Gece sesleniyorum, yine rahat yok bana.
\par 3 Oysa sen kutsalsin, Israil'in övgüleri üzerine taht kuran sensin.
\par 4 Sana güvendiler atalarimiz, Sana dayandilar, onlari kurtardin.
\par 5 Sana yakarip kurtuldular, Sana güvendiler, aldanmadilar.
\par 6 Ama ben insan degil, toprak kurduyum, Insanlar beni küçümsüyor, halk hor görüyor.
\par 7 Beni gören herkes alay ediyor, Siritip bas sallayarak diyorlar ki,
\par 8 "Sirtini RAB'be dayadi, kurtarsin bakalim onu, Madem onu seviyor, yardim etsin!"
\par 9 Oysa beni ana rahminden çikaran, Ana kucagindayken sana güvenmeyi ögreten sensin.
\par 10 Dogusumdan beri sana teslim edildim, Ana rahminden beri Tanrim sensin.
\par 11 Benden uzak durma! Çünkü sikinti yanibasimda, Yardim edecek kimse yok.
\par 12 Bogalar kusatiyor beni, Azgin Basan bogalari sariyor çevremi.
\par 13 Kükreyerek avini parçalayan aslanlar gibi Agizlarini açiyorlar bana.
\par 14 Su gibi dökülüyorum, Bütün kemiklerim oynaklarindan çikiyor; Yüregim balmumu gibi içimde eriyor.
\par 15 Gücüm çömlek parçasi gibi kurudu, Dilim damagima yapisiyor; Beni ölüm topragina yatirdin.
\par 16 Köpekler kusatiyor beni, Kötüler sürüsü çevremi sariyor, Ellerimi, ayaklarimi deliyorlar.
\par 17 Bütün kemiklerimi sayar oldum, Gözlerini dikmis, bana bakiyorlar.
\par 18 Giysilerimi aralarinda paylasiyor, Elbisem için kura çekiyorlar.
\par 19 Ama sen, ya RAB, uzak durma; Ey gücüm benim, yardimima kos!
\par 20 Canimi kiliçtan, Biricik hayatimi köpegin pençesinden kurtar!
\par 21 Kurtar beni aslanin agzindan, Yaban öküzlerinin boynuzundan. Yanit ver bana!
\par 22 Adini kardeslerime duyurayim, Toplulugun ortasinda sana övgüler sunayim:
\par 23 Ey sizler, RAB'den korkanlar, O'na övgüler sunun! Ey Yakup soyu, O'nu yüceltin! Ey Israil soyu, O'na saygi gösterin!
\par 24 Çünkü O mazlumun çektigi sikintiyi hafife almadi, Ondan tiksinmedi, yüz çevirmedi; Kendisini yardima çagirdiginda ona kulak verdi.
\par 25 Övgü konum sen olacaksin büyük toplulukta, Senden korkanlarin önünde yerine getirecegim adaklarimi.
\par 26 Yoksullar yiyip doyacak, RAB'be yönelenler O'na övgü sunacak. Sonsuza dek ömrünüz tükenmesin!
\par 27 Yeryüzünün dört bucagi animsayip RAB'be dönecek, Uluslarin bütün soylari O'nun önünde yere kapanacak.
\par 28 Çünkü egemenlik RAB'bindir, Uluslari O yönetir.
\par 29 Yeryüzündeki bütün zenginler doyacak Ve O'nun önünde yere kapanacak, Topraga gidenler, Ölümlerine engel olamayanlar, Egilecekler O'nun önünde.
\par 30 Gelecek kusaklar O'na kulluk edecek, Rab yeni kusaklara anlatilacak.
\par 31 O'nun kurtarisini, "Rab yapti bunlari" diyerek, Henüz dogmamis bir halka duyuracaklar.

\chapter{23}

\par 1 RAB çobanimdir, Eksigim olmaz.
\par 2 Beni yemyesil çayirlarda yatirir, Sakin sularin kiyisina götürür.
\par 3 Içimi tazeler, Adi ugruna bana dogru yollarda öncülük eder.
\par 4 Karanlik ölüm vadisinden geçsem bile, Kötülükten korkmam. Çünkü sen benimlesin. Çomagin, degnegin güven verir bana.
\par 5 Düsmanlarimin önünde bana sofra kurarsin, Basima yag sürersin, Kâsem tasiyor.
\par 6 Ömrüm boyunca yalniz iyilik ve sevgi izleyecek beni, Hep RAB'bin evinde oturacagim.

\chapter{24}

\par 1 RAB'bindir yeryüzü ve içindeki her sey, Dünya ve üzerinde yasayanlar;
\par 2 Çünkü O'dur denizler üzerinde onu kuran, Sular üzerinde durduran.
\par 3 RAB'bin dagina kim çikabilir, Kutsal yerinde kim durabilir?
\par 4 Elleri pak, yüregi temiz olan, Gönlünü putlara kaptirmayan, Yalan yere ant içmeyen.
\par 5 RAB kutsar böylesini, Kurtaricisi Tanri aklar.
\par 6 O'na yönelenler, Yakup'un Tanrisi'nin yüzünü arayanlar Iste böyledir. *
\par 7 Kaldirin basinizi, ey kapilar! Açilin, ey eski kapilar! Yüce Kral girsin içeri!
\par 8 Kimdir bu Yüce Kral? O RAB'dir, güçlü ve yigit, Savasta yigit olan RAB.
\par 9 Kaldirin basinizi, ey kapilar! Açilin, ey eski kapilar! Yüce Kral girsin içeri!
\par 10 Kimdir bu Yüce Kral? Her Seye Egemen RAB'dir bu Yüce Kral!

\chapter{25}

\par 1 Ya RAB, bütün varligimla sana yaklasiyorum,
\par 2 Ey Tanrim, sana güveniyorum, utandirma beni, Düsmanlarim zafer kahkahasi atmasin!
\par 3 Sana umut baglayan hiç kimse utanca düsmez; Nedensiz hainlik edenler utanir.
\par 4 Ya RAB, yollarini bana ögret, Yönlerini bildir.
\par 5 Bana gerçek yolunda öncülük et, egit beni; Çünkü beni kurtaran Tanri sensin. Bütün gün umudum sende.
\par 6 Ya RAB, sevecenligini ve sevgini animsa; Çünkü onlar öncesizlikten beri aynidir.
\par 7 Gençlik günahlarimi, isyanlarimi animsama, Sevgine göre animsa beni, Çünkü sen iyisin, ya RAB.
\par 8 RAB iyi ve dogrudur, Onun için günahkârlara yol gösterir.
\par 9 Alçakgönüllülere adalet yolunda öncülük eder, Kendi yolunu ögretir onlara.
\par 10 RAB'bin bütün yollari sevgi ve sadakate dayanir Antlasmasindaki buyruklara uyanlar için.
\par 11 Ya RAB, adin ugruna Suçumu bagisla, çünkü suçum büyük.
\par 12 Kim RAB'den korkarsa, RAB ona seçecegi yolu gösterir.
\par 13 Gönenç içinde yasayacak o insan, Soyu ülkeyi sahiplenecek.
\par 14 RAB kendisinden korkanlarla paylasir sirrini, Onlara açiklar antlasmasini.
\par 15 Gözlerim hep RAB'dedir, Çünkü ayaklarimi agdan O çikarir.
\par 16 Halime bak, lütfet bana; Çünkü garip ve mazlumum.
\par 17 Yüregimdeki sikintilar artiyor, Kurtar beni dertlerimden!
\par 18 Üzüntüme, acilarima bak, Bütün günahlarimi bagisla!
\par 19 Düsmanlarima bak, ne kadar çogaldilar, Nasil da benden nefret ediyorlar!
\par 20 Canimi koru, kurtar beni! Hayal kirikligina ugratma, çünkü sana siginiyorum!
\par 21 Dürüstlük, dogruluk korusun beni, Çünkü umudum sendedir.
\par 22 Ey Tanri, kurtar Israil'i Bütün sikintilarindan!

\chapter{26}

\par 1 Beni hakli çikar, ya RAB, Çünkü dürüst bir yasam sürdüm; Sarsilmadan RAB'be güvendim.
\par 2 Dene beni, ya RAB, sina; Duygularimi, düsüncelerimi yokla.
\par 3 Çünkü sevgini hep göz önünde tutuyor, Senin gerçegini yasiyorum ben.
\par 4 Yalancilarla oturmam, Ikiyüzlülerin suyuna gitmem.
\par 5 Kötülük yapanlar toplulugundan nefret ederim, Fesatçilarin arasina girmem.
\par 6 Suçsuzlugumu göstermek için ellerimi yikar, Sunaginin çevresinde dönerim, ya RAB,
\par 7 Yüksek sesle sükranimi duyurmak Ve bütün harikalarini anlatmak için.
\par 8 Severim, ya RAB, yasadigin evi, Görkeminin bulundugu yeri.
\par 9 Günahkârlarin, Eli kanli adamlarin yanisira canimi alma.
\par 10 Onlarin elleri kötülük aletidir, Sag elleri rüsvet doludur.
\par 11 Ama ben dürüst yasarim, Kurtar beni, lütfet bana!
\par 12 Ayagim emin yerde duruyor. Topluluk içinde sana övgüler sunacagim, ya RAB.

\chapter{27}

\par 1 RAB benim isigim, kurtulusumdur, Kimseden korkmam. RAB yasamimin kalesidir, Kimseden yilmam.
\par 2 Hasimlarim, düsmanlarim olan kötüler, Beni yutmak için üzerime gelirken Tökezleyip düserler.
\par 3 Karsimda bir ordu konaklasa, Kilim kipirdamaz, Bana karsi savas açilsa, Yine güvenimi yitirmem.
\par 4 RAB'den tek dilegim, tek istegim su: RAB'bin güzelligini seyretmek, Tapinaginda O'na hayran olmak için Ömrümün bütün günlerini O'nun evinde geçirmek.
\par 5 Çünkü O kötü günde beni çardaginda gizleyecek, Çadirinin emin yerinde saklayacak, Yüksek bir kaya üzerine çikaracak beni.
\par 6 O zaman çevremi saran düsmanlarima karsi Basim yukari kalkacak, Sevinçle haykirarak kurbanlar sunacagim O'nun çadirinda, O'nu ezgilerle, ilahilerle övecegim.
\par 7 Sana yakariyorum, ya RAB, kulak ver sesime, Lütfet, yanitla beni!
\par 8 Ya RAB, içimden bir ses duydum: "Yüzümü ara!" dedin, Iste yüzünü ariyorum.
\par 9 Yüzünü benden gizleme, Kulunu öfkeyle geri çevirme! Bana hep yardimci oldun; Birakma, terk etme beni, Ey beni kurtaran Tanri!
\par 10 Annemle babam beni terk etseler bile, RAB beni kabul eder.
\par 11 Ya RAB, yolunu ögret bana, Düsmanlarima karsi Düz yolda bana öncülük et.
\par 12 Beni hasimlarimin keyfine birakma, Çünkü yalanci taniklar dikiliyor karsima, Agizlari siddet saçiyor.
\par 13 Yasam diyarinda RAB'bin iyiligini görecegimden kuskum yok.
\par 14 Umudunu RAB'be bagla, Güçlü ve yürekli ol; Umudunu RAB'be bagla!

\chapter{28}

\par 1 Ya RAB, sana yakariyorum, Kayam benim, kulak tikama sesime; Çünkü sen sessiz kalirsan, Ölüm çukuruna inen ölülere dönerim ben.
\par 2 Seni yardima çagirdigimda, Ellerimi kutsal konutuna dogru açtigimda, Kulak ver yalvarislarima.
\par 3 Beni kötülerle, haksizlik yapanlarla Ayni kefeye koyup cezalandirma. Onlar komsulariyla dostça konusur, Ama yüreklerinde kötülük beslerler.
\par 4 Eylemlerine, yaptiklari kötülüklere göre onlari yanitla; Yaptiklarinin, hak ettiklerinin karsiligini ver.
\par 5 Onlar RAB'bin yaptiklarina, Ellerinin eserine önem vermezler; Bu yüzden RAB onlari yikacak, Bir daha ayaga kaldirmayacak.
\par 6 RAB'be övgüler olsun! Çünkü yalvarisimi duydu.
\par 7 RAB benim gücüm, kalkanimdir, O'na yürekten güveniyor ve yardim görüyorum. Yüregim cosuyor, Ezgilerimle O'na sükrediyorum.
\par 8 RAB halkinin gücüdür, Meshettigi* kralin zafer kalesidir.
\par 9 Halkini kurtar, kendi halkini kutsa; Çobanlik et onlara, sürekli destek ol!

\chapter{29}

\par 1 Ey ilahi varliklar, RAB'bi övün, RAB'bin gücünü, yüceligini övün,
\par 2 RAB'bin görkemini adina yarasir biçimde övün, Kutsal giysiler içinde RAB'be tapinin!
\par 3 RAB'bin sesi sulara hükmediyor, Yüce Tanri gürlüyor, RAB engin sulara hükmediyor.
\par 4 RAB'bin sesi güçlüdür, RAB'bin sesi görkemlidir.
\par 5 RAB'bin sesi sedir agaçlarini kirar, Lübnan sedirlerini parçalar.
\par 6 Lübnan'i buzagi gibi, Siryon Dagi'ni yabanil öküz yavrusu gibi siçratir.
\par 7 RAB'bin sesi simsek gibi çakar,
\par 8 RAB'bin sesi çölü titretir, RAB Kades Çölü'nü sarsar.
\par 9 RAB'bin sesi geyikleri dogurtur, Ormanlari çiplak birakir. O'nun tapinaginda herkes "Yücesin!" diye haykirir.
\par 10 RAB tufan üstünde taht kurdu, O sonsuza dek kral kalacak.
\par 11 RAB halkina güç verir, Halkini esenlikle kutsar!

\chapter{30}

\par 1 Seni yüceltmek istiyorum, ya RAB, Çünkü beni kurtardin, Düsmanlarimi bana güldürmedin.
\par 2 Ya RAB Tanrim, Sana yakardim, bana sifa verdin.
\par 3 Ya RAB, beni ölüler diyarindan çikardin, Yasam verdin bana, ölüm çukuruna düsürmedin.
\par 4 Ey RAB'bin sadik kullari, O'nu ilahilerle övün, Kutsalligini anarak O'na sükredin.
\par 5 Çünkü öfkesi bir an sürer, Lütfu ise bir ömür; Gözyaslariniz belki bir gece akar, Ama sabahla sevinç dogar.
\par 6 Huzur duyunca dedim ki, "Asla sarsilmayacagim!"
\par 7 Ya RAB, lütfunla beni güçlü bir dag gibi Sarsilmaz kildin; Ama sen yüzünü gizleyince, Dehsete düstüm.
\par 8 Ya RAB, sana sesleniyorum, Rab'be yalvariyorum:
\par 9 "Ne yarari olur senin için dökülen kanimin, Ölüm çukuruna inersem? Toprak sana övgüler sunar mi, Senin sadakatini ilan eder mi?
\par 10 Dinle, ya RAB, aci bana; Yardimcim ol, ya RAB!"
\par 11 Yasimi senlige döndürdün, Çulumu çikarip beni sevinçle kusattin.
\par 12 Öyle ki, gönlüm seni ilahilerle övsün, susmasin! Ya RAB Tanrim, sana sürekli sükredecegim.

\chapter{31}

\par 1 Ya RAB, sana siginiyorum. Utandirma beni hiçbir zaman! Adaletinle kurtar beni!
\par 2 Kulak ver bana, Çabuk yetis, kurtar beni; Bir kaya ol bana siginmam için, Güçlü bir kale ol kurtulmam için!
\par 3 Madem kayam ve kalem sensin, Öncülük et, yol göster bana Kendi adin ugruna.
\par 4 Bana kurduklari tuzaktan uzak tut beni, Çünkü siginagim sensin.
\par 5 Ruhumu ellerine birakiyorum, Ya RAB, sadik Tanri, kurtar beni.
\par 6 Degersiz putlara bel baglayanlardan tiksinirim, RAB'be güvenirim ben.
\par 7 Sadakatinden ötürü sevinip cosacagim, Çünkü düskün halimi görüyor, Çektigim sikintilari biliyorsun,
\par 8 Beni düsman eline düsürmedin, Bastigim yerleri genislettin.
\par 9 Aci bana, ya RAB, sikintidayim, Üzüntü gözümü, canimi, içimi kemiriyor.
\par 10 Ömrüm aciyla, Yillarim iniltiyle tükeniyor, Suçumdan ötürü gücüm zayifliyor, Kemiklerim eriyor.
\par 11 Düsmanlarim yüzünden rezil oldum, Özellikle komsularima. Tanidiklarima dehset salar oldum; Beni sokakta görenler benden kaçar oldu.
\par 12 Gönülden çikmis bir ölü gibi unutuldum, Kirilmis bir çömlege döndüm.
\par 13 Birçogunun fisildastigini duyuyorum, Her yer dehset içinde, Bana karsi anlastilar, Canimi almak için düzen kurdular.
\par 14 Ama ben sana güveniyorum, ya RAB, "Tanrim sensin!" diyorum.
\par 15 Hayatim senin elinde, Kurtar beni düsmanlarimin pençesinden, Ardima düsenlerden.
\par 16 Yüzün kulunu aydinlatsin, Sevgi göster, kurtar beni!
\par 17 Utandirma beni, ya RAB, sana sesleniyorum; Kötüler utansin, ölüler diyarinda sesleri kesilsin.
\par 18 Sussun o yalanci dudaklar; Dogru insana karsi Gururla, tepeden bakarak, Küçümseyerek konusan dudaklar.
\par 19 Iyiligin ne büyüktür, ya RAB, Onu senden korkanlar için saklarsin, Herkesin gözü önünde, Sana siginanlara iyi davranirsin.
\par 20 Insanlarin düzenlerine karsi, Koruyucu huzurunla üzerlerine kanat gerersin; Saldirgan dillere karsi Onlari çardaginda gizlersin.
\par 21 RAB'be övgüler olsun, Kusatilmis bir kentte Sevgisini bana harika biçimde gösterdi.
\par 22 Telas içinde demistim ki, "Huzurundan atildim!" Ama yardima çagirinca seni, Yalvarisimi isittin.
\par 23 RAB'bi sevin, ey O'nun sadik kullari! RAB kendisine bagli olanlari korur, Büyüklenenlerin ise tümüyle hakkindan gelir.
\par 24 Ey RAB'be umut baglayanlar, Güçlü ve yürekli olun!

\chapter{32}

\par 1 Ne mutlu isyani bagislanan, Günahi örtülen insana!
\par 2 Suçu RAB tarafindan sayilmayan, Ruhunda hile bulunmayan insana ne mutlu!
\par 3 Sustugum sürece Kemiklerim eridi, Gün boyu inlemekten.
\par 4 Çünkü gece gündüz Elin üzerimde agirlasti. Dermanim tükendi yaz sicaginda gibi. *
\par 5 Günahimi açikladim sana, Suçumu gizlemedim. "RAB'be isyanimi itiraf edecegim" deyince, Günahimi, suçumu bagisladin.
\par 6 Bu nedenle her sadik kulun Ulasilir oldugun zaman sana dua etsin. Engin sular tassa bile ona erisemez.
\par 7 Siginagim sensin, Beni sikintidan korur, Çevremi kurtulus ilahileriyle kusatirsin.
\par 8 Egitecegim seni, gidecegin yolu gösterecegim, Ögüt verecegim sana, Gözüm sendedir.
\par 9 At ya da katir gibi anlayissiz olmayin; Onlari idare etmek için gem ve dizgin gerekir, Yoksa sana yaklasmazlar.
\par 10 Kötülerin acisi çoktur, Ama RAB'be güvenenleri O'nun sevgisi kusatir.
\par 11 Ey dogru insanlar, sevinç kaynaginiz RAB olsun, cosun; Ey yüregi temiz olanlar, Hepiniz sevinç çigliklari atin!

\chapter{33}

\par 1 Ey dogru insanlar, RAB'be sevinçle haykirin! Dürüstlere O'nu övmek yarasir.
\par 2 Lir çalarak RAB'be sükredin, On telli çenk esliginde O'nu ilahilerle övün.
\par 3 O'na yeni bir ezgi söyleyin, Sevinç çigliklariyla sazinizi konusturun.
\par 4 Çünkü RAB'bin sözü dogrudur, Her isi sadakatle yapar.
\par 5 Dogrulugu, adaleti sever, RAB'bin sevgisi yeryüzünü doldurur.
\par 6 Gökler RAB'bin sözüyle, Gök cisimleri agzindan çikan solukla yaratildi.
\par 7 Deniz sularini bir araya toplar, Engin sulari ambarlara depolar.
\par 8 Bütün yeryüzü RAB'den korksun, Dünyada yasayan herkes O'na saygi duysun.
\par 9 Çünkü O söyleyince, her sey var oldu; O buyurunca, her sey belirdi.
\par 10 RAB uluslarin planlarini bozar, Halklarin tasarilarini bosa çikarir.
\par 11 Ama RAB'bin planlari sonsuza dek sürer, Yüregindeki tasarilar kusaklar boyunca degismez.
\par 12 Ne mutlu Tanrisi RAB olan ulusa, Kendisi için seçtigi halka!
\par 13 RAB göklerden bakar, Bütün insanlari görür.
\par 14 Oturdugu yerden, Yeryüzünde yasayan herkesi gözler.
\par 15 Herkesin yüregini yaratan, Yaptiklari her seyi tartan O'dur.
\par 16 Ne büyük ordulariyla zafer kazanan kral var, Ne de büyük gücüyle kurtulan yigit.
\par 17 Zafer için at bos bir umuttur, Büyük gücüne karsin kimseyi kurtaramaz.
\par 18 Ama RAB'bin gözü kendisinden korkanlarin, Sevgisine umut baglayanlarin üzerindedir;
\par 19 Böylece onlari ölümden kurtarir, Kitlikta yasamalarini saglar.
\par 20 Umudumuz RAB'dedir, Yardimcimiz, kalkanimiz O'dur.
\par 21 O'nda sevinç bulur yüregimiz, Çünkü O'nun kutsal adina güveniriz.
\par 22 Madem umudumuz sende, Sevgin üzerimizde olsun, ya RAB!

\chapter{34}

\par 1 Her zaman RAB'be övgüler sunacagim, Övgüsü dilimden düsmeyecek.
\par 2 RAB'le övünürüm, Mazlumlar isitip sevinsin!
\par 3 Benimle birlikte RAB'bin büyüklügünü duyurun, Adini birlikte yüceltelim.
\par 4 RAB'be yöneldim, yanit verdi bana, Bütün korkularimdan kurtardi beni.
\par 5 O'na bakanlarin yüzü isil isil parlar, Yüzleri utançtan kizarmaz.
\par 6 Bu mazlum yakardi, RAB duydu, Bütün sikintilarindan kurtardi onu.
\par 7 RAB'bin melegi O'ndan korkanlarin çevresine ordugah kurar, Kurtarir onlari.
\par 8 Tadin da görün, RAB ne iyidir, Ne mutlu O'na siginan adama!
\par 9 RAB'den korkun, ey O'nun kutsallari, Çünkü O'ndan korkanin eksigi olmaz.
\par 10 Genç aslanlar bile aç ve muhtaç olur; Ama RAB'be yönelenlerden hiçbir iyilik esirgenmez.
\par 11 Gelin, ey çocuklar, dinleyin beni: Size RAB korkusunu ögreteyim.
\par 12 Kim yasamdan zevk almak, Iyi günler görmek istiyorsa,
\par 13 Dilini kötülükten, Dudaklarini yalandan uzak tutsun.
\par 14 Kötülükten sakinin, iyilik yapin; Esenligi amaçlayin, ardinca gidin.
\par 15 RAB'bin gözleri dogru kisilerin üzerindedir, Kulaklari onlarin yakarisina açiktir.
\par 16 RAB kötülük yapanlara karsidir, Onlarin anisini yeryüzünden siler.
\par 17 Dogrular yakarir, RAB duyar; Bütün sikintilarindan kurtarir onlari.
\par 18 RAB gönlü kiriklara yakindir, Ruhu ezginleri kurtarir.
\par 19 Dogrunun dertleri çoktur, Ama RAB hepsinden kurtarir onu.
\par 20 Bütün kemiklerini korur, Hiçbiri kirilmaz.
\par 21 Kötü insanin sonu kötülükle biter, Cezasini bulur dogrulardan nefret edenler.
\par 22 RAB kullarini kurtarir, O'na siginanlarin hiçbiri ceza görmez.

\chapter{35}

\par 1 Ya RAB, benimle ugrasanlarla sen ugras, Benimle savasanlarla sen savas!
\par 2 Al küçük kalkanla büyük kalkani, Yardimima kos!
\par 3 Kaldir mizragini, kargini beni kovalayanlara, "Seni ben kurtaririm" de bana!
\par 4 Canima kastedenler utanip rezil olsun! Utançla geri çekilsin bana kötülük düsünenler!
\par 5 Rüzgarin sürükledigi saman çöpüne dönsünler, RAB'bin melegi artlarina düssün!
\par 6 Karanlik ve kaygan olsun yollari, RAB'bin melegi kovalasin onlari!
\par 7 Madem neden yokken bana gizli aglar kurdular, Nedensiz çukur kazdilar,
\par 8 Baslarina habersiz felaket gelsin, Gizledikleri aga kendileri tutulsun, Felakete ugrasinlar.
\par 9 O zaman RAB'de sevinç bulacagim, Beni kurtardigi için cosacagim.
\par 10 Bütün varligimla söyle diyecegim: "Senin gibisi var mi, ya RAB, Mazlumu zorbanin elinden, Mazlumu ve yoksulu soyguncudan kurtaran?"
\par 11 Kötü niyetli taniklar türüyor, Bilmedigim konulari soruyorlar.
\par 12 Iyiligime karsi kötülük ediyor, Yalnizliga itiyorlar beni.
\par 13 Oysa onlar hastalaninca ben çula sarinir, Oruç* tutup alçakgönüllü olurdum. Duam yanitsiz kalinca, Bir dost, bir kardes yitirmis gibi dolasirdim. Kederden belim bükülürdü, Annesi için yas tutan biri gibi.
\par 15 Ama ben sendeleyince toplanip sevindiler, Toplandi bana karsi tanimadigim alçaklar, Durmadan didiklediler beni.
\par 16 Tanritanimaz, alayci soytarilar gibi, Dis gicirdattilar bana.
\par 17 Ne zamana dek seyirci kalacaksin, ya Rab? Kurtar canimi bunlarin saldirisindan, Hayatimi bu genç aslanlardan!
\par 18 Büyük toplantida sana sükürler sunacagim, Kalabaligin ortasinda sana övgüler dizecegim.
\par 19 Sevinmesin bos yere bana düsman olanlar, Göz kirpmasinlar birbirlerine Nedensiz benden nefret edenler.
\par 20 Çünkü baris sözünü etmez onlar, Kurnazca düzen kurarlar ülkenin sakin insanlarina.
\par 21 Beni suçlamak için agizlarini ardina kadar açtilar: "Oh! Oh!" diyorlar, "Iste kendi gözümüzle gördük yaptiklarini!"
\par 22 Olup biteni sen de gördün, ya RAB, sessiz kalma, Ya Rab, benden uzak durma!
\par 23 Uyan, kalk savun beni, Ugras hakkim için, ey Tanrim ve Rab'bim!
\par 24 Adaletin uyarinca hakli çikar beni, ya RAB, Tanrim benim! Gülmesinler halime!
\par 25 Demesinler içlerinden: "Oh! Iste buydu dilegimiz!", Konusmasinlar ardimdan: "Yedik basini!" diye.
\par 26 Utansin kötü halime sevinenler, Kizarsin yüzleri hepsinin; Gururla karsima dikilenler Utanca, rezalete bürünsün.
\par 27 Benim hakli çikmami isteyenler, Sevinç çigliklari atip cossunlar; Söyle desinler sürekli: "Kulunun esenliginden hoslanan RAB yücelsin!"
\par 28 O zaman gün boyu adaletin, Övgülerin dilimden düsmeyecek.

\chapter{36}

\par 1 Günah fisildar kötü insana, Yüreginin dibinden: Tanri korkusu yoktur onda.
\par 2 Kendini öyle begenmis ki, Suçunu görmez, ondan tiksinmez.
\par 3 Agzindan kötülük ve yalan akar, Akillanmaktan, iyilik yapmaktan vazgeçmis.
\par 4 Yataginda bile fesat düsünür, Olumsuz yolda direnir, reddetmez kötülügü.
\par 5 Ya RAB, sevgin göklere, Sadakatin gökyüzüne erisir.
\par 6 Dogrulugun ulu daglara benzer, Adaletin uçsuz bucaksiz enginlere. Insani da, hayvani da koruyan sensin, ya RAB.
\par 7 Sevgin ne degerli, ey Tanri! Kanatlarinin gölgesine siginir insanoglu.
\par 8 Evindeki bolluga doyarlar, Zevklerinin irmagindan içirirsin onlara.
\par 9 Çünkü yasam kaynagi sensin, Senin isiginla aydinlaniriz.
\par 10 Sürekli göster Seni taniyanlara sevgini, Yüregi temiz olanlara dogrulugunu.
\par 11 Gururlunun ayagi bana varmasin, Kötülerin eli beni kovmasin.
\par 12 Kötülük yapanlar oracikta düstüler, Yikildilar, kalkamazlar artik.

\chapter{37}

\par 1 Kötülük edenlere kizip üzülme, Suç isleyenlere özenme!
\par 2 Çünkü onlar ot gibi hemen solacak, Yesil bitki gibi kuruyup gidecek.
\par 3 Sen RAB'be güven, iyilik yap, Ülkede otur, sadakatle çalis.
\par 4 RAB'den zevk al, O senin içindeki istekleri yerine getirecektir.
\par 5 Her seyi RAB'be birak, O'na güven, O gerekeni yapar.
\par 6 O senin dogrulugunu isik gibi, Hakkini ögle günesi gibi Aydinliga çikarir.
\par 7 RAB'bin önünde sakin dur, sabirla bekle; Kizip üzülme isi yolunda olanlara, Kötü amaçlarina kavusanlara.
\par 8 Kizmaktan kaçin, birak öfkeyi, Üzülme, yalniz kötülüge sürükler bu seni.
\par 9 Çünkü kötülerin kökü kazinacak, Ama RAB'be umut baglayanlar ülkeyi miras alacak.
\par 10 Yakinda kötünün sonu gelecek, Yerini arasan da bulunmayacak.
\par 11 Ama alçakgönüllüler ülkeyi miras alacak, Derin bir huzurun zevkini tadacak.
\par 12 Kötü insan dogru insana düzen kurar, Dis gicirdatir.
\par 13 Ama Rab kötüye güler, Çünkü bilir onun sonunun geldigini.
\par 14 Kiliç çekti kötüler, yaylarini gerdi, Mazlumu, yoksulu yikmak, Dogru yolda olanlari öldürmek için.
\par 15 Ama kiliçlari kendi yüreklerine saplanacak, Yaylari kirilacak.
\par 16 Dogrunun azicik varligi, Pek çok kötünün servetinden iyidir.
\par 17 Çünkü kötülerin gücü kirilacak, Ama dogrulara RAB destek olacak.
\par 18 RAB yetkinlerin her gününü gözetir, Onlarin mirasi sonsuza dek sürecek.
\par 19 Kötü günde utanmayacaklar, Kitlikta karinlari doyacak.
\par 20 Ama kötüler yikima ugrayacak; RAB'bin düsmanlari kir çiçekleri gibi kuruyup gidecek, Duman gibi dagilip yok olacak.
\par 21 Kötüler ödünç alir, geri vermez; Dogrularsa cömertçe verir.
\par 22 RAB'bin kutsadigi insanlar ülkeyi miras alacak, Lanetledigi insanlarin kökü kazinacak.
\par 23 RAB insana saglam adim attirir, Insanin yolundan hosnut olursa.
\par 24 Düsse bile yikilmaz insan, Çünkü elinden tutan RAB'dir.
\par 25 Gençtim, ömrüm tükendi, Ama hiç görmedim dogru insanin terk edildigini, Soyunun ekmek dilendigini.
\par 26 O hep cömertçe ödünç verir, Soyu kutsanir.
\par 27 Kötülükten kaç, iyilik yap; Sonsuz yasama kavusursun.
\par 28 Çünkü RAB dogruyu sever, Sadik kullarini terk etmez. Onlar sonsuza dek korunacak, Kötülerinse kökü kazinacak.
\par 29 Dogrular ülkeyi miras alacak, Orada sonsuza dek yasayacak.
\par 30 Dogrunun agzindan bilgelik akar, Dilinden adalet damlar.
\par 31 Tanrisi'nin yasasi yüregindedir, Ayaklari kaymaz.
\par 32 Kötü, dogruya pusu kurar, Onu öldürmeye çalisir.
\par 33 Ama RAB onu kötünün eline düsürmez, Yargilanirken mahkûm etmez.
\par 34 RAB'be umut bagla, O'nun yolunu tut, Ülkeyi miras almak üzere seni yükseltecektir. Kötülerin kökünün kazindigini göreceksin.
\par 35 Kötü ve acimasiz adami gördüm, Ilk dikildigi toprakta yeseren agaç gibi Dal budak saliyordu;
\par 36 Geçip gitti, yok oldu, Aradim, bulunmaz oldu.
\par 37 Yetkin adami gözle, dogru adama bak, Çünkü yarinlar barisseverindir.
\par 38 Ama baskaldiranlarin hepsi yok olacak, Kötülerin kökü kazinacak.
\par 39 Dogrularin kurtulusu RAB'den gelir, Sikintili günde onlara kale olur.
\par 40 RAB onlara yardim eder, kurtarir onlari, Kötülerin elinden alip özgür kilar, Çünkü kendisine siginirlar.

\chapter{38}

\par 1 Ya RAB, öfkelenip azarlama beni, Gazapla yola getirme!
\par 2 Oklarin içime saplandi, Elin üzerime indi.
\par 3 Öfken yüzünden sagligim bozuldu, Günahim yüzünden rahatim kaçti.
\par 4 Çünkü suçlarim basimdan asti, Tasinmaz bir yük gibi sirtimda agirlasti.
\par 5 Akilsizligim yüzünden Yaralarim igrenç, irinli.
\par 6 Egildim, iki büklüm oldum, Gün boyu yasli dolasiyorum.
\par 7 Çünkü belim ates içinde, Sagligim bozuk.
\par 8 Tükendim, ezildim alabildigine, Inliyorum yüregimin acisindan.
\par 9 Ya Rab, bütün özlemlerimi bilirsin, Iniltilerim senden gizli degil.
\par 10 Yüregim çarpiyor, gücüm tükeniyor, Gözlerimin feri bile söndü.
\par 11 Esim dostum kaçar oldu derdimden, Yakinlarim uzak duruyor benden.
\par 12 Canima susayanlar bana tuzak kuruyor, Zararimi isteyenler kuyumu kaziyor, Gün boyu hileler düsünüyorlar.
\par 13 Ama ben bir sagir gibi duymuyorum, Bir dilsiz gibi agzimi açmiyorum;
\par 14 Duymaz, Agzinda yanit bulunmaz bir adama döndüm.
\par 15 Umudum sende, ya RAB, Sen yanitlayacaksin, ya Rab, Tanrim benim!
\par 16 Çünkü dua ediyorum: "Halime sevinmesinler, Ayagim kayinca böbürlenmesinler!"
\par 17 Düsmek üzereyim, Acim hep içimde.
\par 18 Suçumu itiraf ediyorum, Günahim yüzünden kaygilaniyorum.
\par 19 Ama düsmanlarim güçlü ve dinç, Yok yere benden nefret edenler çok.
\par 20 Iyilige karsi kötülük yapanlar bana karsi çikar, Iyiligin pesinde oldugum için.
\par 21 Beni terk etme, ya RAB! Ey Tanrim, benden uzak durma!
\par 22 Yardimima kos, Ya Rab, kurtulusum benim!

\chapter{39}

\par 1 Karar verdim: "Adimlarima dikkat edecegim, Dilimi günahtan sakinacagim; Karsimda kötü biri oldukça, Agzima gem vuracagim."
\par 2 Dilimi tutup sustum, Hep kaçindim konusmaktan, yarari olsa bile. Acim alevlendi,
\par 3 Yüregim tutustu içimde, Ates aldi derin derin düsünürken, Su sözler döküldü dilimden:
\par 4 "Bildir bana, ya RAB, sonumu, Sayili günlerimi; Bileyim ömrümün ne kadar kisa oldugunu!
\par 5 Yalniz bir karis ömür verdin bana, Hiç kalir hayatim senin önünde. Her insan bir soluktur sadece, En güçlü çaginda bile. *
\par 6 "Bir gölge gibi dolasir insan, Bos yere çirpinir, Mal biriktirir, kime kalacagini bilmeden.
\par 7 "Ne bekleyebilirim simdi, ya Rab? Umudum sende.
\par 8 Kurtar beni bütün isyanlarimdan, Aptallarin hakaretine izin verme.
\par 9 Sustum, açmayacagim agzimi; Çünkü sensin bunu yapan.
\par 10 Uzaklastir üzerimden yumruklarini, Tokadinin altinda mahvoldum.
\par 11 Sen insani suçundan ötürü Azarlayarak yola getirirsin, Güve gibi tüketirsin sevdigi seyleri. Her insan bir soluktur sadece.
\par 12 "Duami isit, ya RAB, Kulak ver yakarisima, Gözyaslarima kayitsiz kalma! Çünkü ben bir garibim senin yaninda, Bir yabanci, atalarim gibi.
\par 13 Uzaklastir üzerimden bakislarini, Göçüp yok olmadan mutlu olayim!"

\chapter{40}

\par 1 RAB'bi sabirla bekledim; Bana yönelip yakarisimi duydu.
\par 2 Ölüm çukurundan, Balçiktan çikardi beni, Ayaklarimi kaya üzerinde tuttu, Kaymayayim diye.
\par 3 Agzima yeni bir ezgi, Tanrimiz'a bir övgü ilahisi koydu. Çoklari görüp korkacak Ve RAB'be güvenecekler.
\par 4 Ne mutlu RAB'be güvenen insana, Gururluya, yalana sapana ilgi duymayana.
\par 5 Ya RAB, Tanrim, Harikalarin, düsüncelerin ne çoktur bizim için; Sana es kosulmaz! Duyurmak, anlatmak istesem yaptiklarini, Saymakla bitmez.
\par 6 Kurbandan, sunudan hosnut olmadin, Ama kulaklarimi açtin. Yakmalik sunu*, günah sunusu* da istemedin.
\par 7 O zaman söyle dedim: "Iste geldim; Kutsal Yazi tomarinda benim için yazilmistir. Ey Tanrim, senin istegini yapmaktan zevk alirim ben, Yasan yüregimin derinligindedir."
\par 9 Büyük toplantida müjdelerim senin zaferini, Sözümü esirgemem, Ya RAB, bildigin gibi!
\par 10 Zaferini içimde gizlemem, Bagliligini ve kurtarisini duyururum, Sevgini, sadakatini saklamam büyük topluluktan.
\par 11 Ya RAB, esirgeme sevecenligini benden! Sevgin, sadakatin hep korusun beni!
\par 12 Sayisiz belalar çevremi sardi, Suçlarim bana yetisti, önümü göremiyorum; Basimdaki saçlardan daha çoklar, Çaresiz kaldim.
\par 13 Ne olur, ya RAB, kurtar beni! Yardimima kos, ya RAB!
\par 14 Utansin canimi almaya çalisanlar, Yüzleri kizarsin! Geri dönsün zararimi isteyenler, Rezil olsunlar!
\par 15 Bana, "Oh! Oh!" çekenler Dehsete düssün utançlarindan!
\par 16 Sende nese ve sevinç bulsun Bütün sana yönelenler! "RAB yücedir!" desin hep Senin kurtarisini özleyenler!
\par 17 Bense mazlum ve yoksulum, Düsün beni, ya Rab. Yardimcim ve kurtaricim sensin, Geç kalma, ey Tanrim!

\chapter{41}

\par 1 Ne mutlu yoksulu düsünene! RAB kurtarir onu kötü günde.
\par 2 Korur RAB, yasatir onu, Ülkede mutlu kilar, Terk etmez düsmanlarinin eline.
\par 3 Destek olur RAB ona Yataga düsünce; Hastalandiginda sagliga kavusturur onu.
\par 4 "Aci bana, ya RAB!" dedim, "Sifa ver bana, çünkü sana karsi günah isledim!"
\par 5 Kötü konusuyor düsmanlarim ardimdan: "Ne zaman ölecek adi batasi?" diyorlar.
\par 6 Biri beni görmeye geldi mi, bos laf ediyor, Fesat topluyor içinde, Sonra disari çikip fesadi yayiyor.
\par 7 Benden nefret edenlerin hepsi Fisildasiyor aralarinda bana karsi, Zararimi düsünüyorlar,
\par 8 "Basina öyle kötü bir sey geldi ki" diyorlar, "Yatagindan kalkamaz artik."
\par 9 Ekmegimi yiyen, güvendigim yakin dostum bile Ihanet etti bana.
\par 10 Bari sen aci bana, ya RAB, kaldir beni. Bunlarin hakkindan geleyim.
\par 11 Düsmanim zafer çigligi atmazsa, O zaman anlarim benden hosnut kaldigini.
\par 12 Dürüstlügümden ötürü bana destek olur, Sonsuza dek beni huzurunda tutarsin.
\par 13 Öncesizlikten sonsuza dek, Övgüler olsun Israil'in Tanrisi RAB'be! Amin! Amin!

\chapter{42}

\par 1 Geyik akarsulari nasil özlerse, Canim da seni öyle özler, ey Tanri!
\par 2 Canim Tanri'ya, yasayan Tanri'ya susadi; Ne zaman görmeye gidecegim Tanri'nin yüzünü?
\par 3 Gözyaslarim ekmegim oldu gece gündüz, Gün boyu, "Nerede senin Tanrin?" dedikleri için.
\par 4 Animsayinca içim içimi yiyor, Nasil toplulukla birlikte yürür, Tanri'nin evine kadar alaya öncülük ederdim, Sevinç ve sükran sesleri arasinda, Bayram eden bir kalabalikla birlikte.
\par 5 Neden üzgünsün, ey gönlüm, Neden içim huzursuz? Tanri'ya umut bagla, Çünkü O'na yine övgüler sunacagim; O benim kurtaricim ve Tanrim'dir.
\par 6 Gönlüm üzgün, Bu yüzden seni animsiyorum, ey Tanrim. Seria yöresinde, Hermon ve Misar daglarinda Çaglayanlarin gümbürdeyince, Enginler birbirine sesleniyor, Bütün dalgalarin, sellerin üzerimden geçiyor.
\par 8 Gündüz RAB sevgisini gösterir, Gece ilahi söyler, dua ederim Yasamimin Tanrisi'na.
\par 9 Kayam olan Tanrim'a diyorum ki, "Neden beni unuttun? Niçin düsmanlarimin baskisi altinda Yasli gezeyim?"
\par 10 Gün boyu hasimlarim: "Nerede senin Tanrin?" diyerek Bana satastikça, Kemiklerim kiriliyor sanki.
\par 11 Neden üzgünsün, ey gönlüm, Neden içim huzursuz? Tanri'ya umut bagla, Çünkü O'na yine övgüler sunacagim; O benim kurtaricim ve Tanrim'dir.

\chapter{43}

\par 1 Hakkimi ara, ey Tanri, Savun beni vefasiz ulusa karsi, Kurtar hileci, haksiz insandan.
\par 2 Çünkü sen Tanrim, kalemsin; Neden beni reddettin? Niçin düsmanlarimin baskisi altinda Yasli gezeyim?
\par 3 Gönder isigini, gerçegini, Yol göstersinler bana, Senin kutsal dagina, konutuna götürsünler beni.
\par 4 O zaman Tanri'nin sunagina, Nese, sevinç kaynagim Tanri'ya gidecegim Ve sana, ey Tanri, Tanrim benim, Lirle sükredecegim.
\par 5 Neden üzgünsün, ey gönlüm, Neden içim huzursuz? Tanri'ya umut bagla, Çünkü O'na yine övgüler sunacagim; O benim kurtaricim ve Tanrim'dir.

\chapter{44}

\par 1 Ey Tanri, kulaklarimizla duyduk, Atalarimiz anlatti bize, Neler yaptigini onlarin gününde, eski günlerde.
\par 2 Elinle uluslari kovdun, Ama atalarimiza yer verdin; Halklari kirdin, Ama atalarimizin yayilmasini sagladin.
\par 3 Onlar ülkeyi kiliçla kazanmadilar, Kendi bilekleriyle zafere ulasmadilar. Senin sag elin, bilegin, yüzünün isigi sayesinde oldu bu; Çünkü sen onlari sevdin.
\par 4 Ey Tanri, kralim sensin, Buyruk ver de Yakup soyu kazansin!
\par 5 Senin sayende düsmanlarimizi püskürtecegiz, Senin adinla karsitlarimizi ezecegiz.
\par 6 Çünkü ben yayima güvenmem, Kilicim da beni kurtarmaz;
\par 7 Ancak sensin bizi düsmanlarimizdan kurtaran, Bizden nefret edenleri utanca bogan.
\par 8 Her gün Tanri'yla övünür, Sonsuza dek adina sükran sunariz. *
\par 9 Ne var ki, reddettin bizi, asagiladin, Artik ordularimizla savasa çikmiyorsun.
\par 10 Düsman karsisinda bizi gerilettin, Bizden tiksinenler bizi soydu.
\par 11 Kasaplik koyuna çevirdin bizi, Uluslarin arasina dagittin.
\par 12 Yok pahasina sattin halkini, Üstelik satistan hiçbir sey kazanmadan.
\par 13 Bizi komsularimizin yüzkarasi, Çevremizdekilerin eglencesi, alay konusu ettin.
\par 14 Uluslarin diline düsürdün bizi, Gülüyor halklar halimize.
\par 15 Rezilligim gün boyu karsimda, Utancimdan yerin dibine geçtim
\par 16 Hakaret ve sövgü duya duya, Öç almak isteyen düsman karsisinda.
\par 17 Bütün bunlar basimiza geldi, Yine de seni unutmadik, Antlasmana ihanet etmedik,
\par 18 Döneklik etmedik, Adimlarimiz senin yolundan sapmadi.
\par 19 Oysa sen bizi ezdin, ülkemizi çakallarin ugragi ettin, Üstümüzü koyu karanlikla örttün.
\par 20 Eger Tanrimiz'in adini unutsaydik, Yabanci bir ilaha ellerimizi açsaydik,
\par 21 Tanri bunu ortaya çikarmaz miydi? Çünkü O yürekteki gizleri bilir.
\par 22 Senin ugruna her gün öldürülüyoruz, Kasaplik koyun sayiliyoruz.
\par 23 Uyan, ya Rab! Niçin uyuyorsun? Kalk! Sonsuza dek terk etme bizi!
\par 24 Niçin yüzünü gizliyorsun? Neden mazlum halimizi, üzerimizdeki baskiyi unutuyorsun?
\par 25 Çünkü yere serildik, Bedenimiz topraga yapisti.
\par 26 Kalk, yardim et bize! Kurtar bizi sevgin ugruna!

\chapter{45}

\par 1 Yüregimden güzel sözler tasiyor, Kral için söylüyorum siirlerimi, Dilim usta bir yazarin kalemi gibi olsun.
\par 2 Sen insanlarin en güzelisin, Lütuf saçilmis dudaklarina. Çünkü Tanri seni sonsuza dek kutsamis.
\par 3 Ey yigit savasçi, kusan kilicini beline, Görkemine, yüceligine bürün.
\par 4 At sirtinda görkeminle, zaferle ilerle, Gerçek ve adalet ugruna Sag elin korkunç isler göstersin.
\par 5 Oklarin sivridir, Kral düsmanlarinin yüregine saplanir, Halklar ayaklarinin altina serilir.
\par 6 Ey Tanri, tahtin sonsuzluklar boyunca kalicidir, Kralliginin asasi adalet asasidir.
\par 7 Dogrulugu sever, kötülükten nefret edersin. Bunun için Tanri, senin Tanrin, Seni sevinç yagiyla Arkadaslarindan daha çok meshetti*.
\par 8 Giysilerinin tümü mür*, öd, tarçin kokuyor; Fildisi saraylardan gelen çalgi sesleri seni eglendiriyor!
\par 9 Kral kizlari senin saygin kadinlarin arasinda, Kraliçe, Ofir altinlari içinde senin saginda duruyor.
\par 10 Dinle, ey kral kizi, bak, kulak ver, Halkini, baba evini unut.
\par 11 Kral senin güzelligine vuruldu, Efendin oldugu için önünde egil.
\par 12 Sur halki armagan getirecek, Halkin zenginleri lütfunu kazanmak isteyecek.
\par 13 Kral kizi odasinda isil isil parildiyor, Giysisi altinla dokunmus.
\par 14 Islemeli giysiler içinde kralin önüne çikarilacak, Arkadaslari, ona eslik eden kizlar sana getirilecek.
\par 15 Sevinç ve coskuyla götürülecek, Kralin sarayina girecekler.
\par 16 Atalarinin yerini ogullarin alacak, Onlari önder yapacaksin bütün ülkeye.
\par 17 Adini kusaklar boyunca yasatacagim, Böylece halklar sonsuza dek övecek seni.

\chapter{46}

\par 1 Tanri siginagimiz ve gücümüzdür, Sikintida hep yardima hazirdir.
\par 2 Bu yüzden korkmayiz yeryüzü altüst olsa, Daglar denizlerin bagrina devrilse,
\par 3 Sular kükreyip köpürse, Kabaran deniz daglari titretse bile. *
\par 4 Bir irmak var ki, sulari sevinç getirir Tanri kentine, Yüceler Yücesi'nin kutsal konutuna.
\par 5 Tanri onun ortasindadir, Sarsilmaz o kent. Gün dogarken Tanri ona yardim eder.
\par 6 Uluslar kükrüyor, kralliklar sarsiliyor, Tanri gürleyince yeryüzü eriyip gidiyor.
\par 7 Her Seye Egemen RAB bizimledir, Yakup'un Tanrisi kalemizdir.
\par 8 Gelin, görün RAB'bin yaptiklarini, Yeryüzüne getirdigi yikimlari.
\par 9 Savaslari durdurur yeryüzünün dört bucaginda, Yaylari kirar, mizraklari parçalar, Kalkanlari yakar.
\par 10 "Sakin olun, bilin ki, Tanri benim! Uluslar arasinda yücelecegim, Yeryüzünde yücelecegim!"
\par 11 Her Seye Egemen RAB bizimledir, Yakup'un Tanrisi kalemizdir.

\chapter{47}

\par 1 Ey bütün uluslar, el çirpin! Sevinç çigliklari atin Tanri'nin onuruna!
\par 2 Ne müthistir yüce RAB, Bütün dünyanin ulu Krali.
\par 3 Halklari altimiza, Uluslari ayaklarimizin dibine serer.
\par 4 Sevdigi Yakup'un gururu olan mirasimizi O seçti bizim için. *
\par 5 RAB Tanri sevinç çigliklari, Boru sesleri arasinda yükseldi.
\par 6 Ezgiler sunun Tanri'ya, ezgiler; Ezgiler sunun Kralimiz'a, ezgiler!
\par 7 Çünkü Tanri bütün dünyanin kralidir, Maskil sunun!
\par 8 Tanri kutsal tahtina oturmus, Krallik eder uluslara.
\par 9 Uluslarin önderleri Ibrahim'in Tanrisi'nin halkiyla bir araya gelmis; Çünkü Tanri'ya aittir yeryüzü krallari. O çok yücedir.

\chapter{48}

\par 1 RAB büyüktür ve yalniz O övülmeye deger Tanrimiz'in kentinde, kutsal daginda.
\par 2 Yükselir zarafetle, Bütün yeryüzünün sevinci Siyon Dagi, Safon'un dorugu, ulu Kral'in kenti.
\par 3 Tanri onun kalelerinde Saglam kule olarak gösterdi kendini.
\par 4 Krallar toplandi, Birlikte Siyon'un üzerine yürüdüler.
\par 5 Ama onu görünce saskina döndüler, Dehsete düsüp kaçtilar.
\par 6 Dogum sancisi tutan kadin gibi, Bir titreme aldi onlari orada.
\par 7 Dogu rüzgarinin parçaladigi ticaret gemileri gibi Yok ettin onlari.
\par 8 Her Seye Egemen RAB'bin kentinde, Tanrimiz'in kentinde, Nasil duyduksa, öyle gördük. Tanri onu sonsuza dek güvenlik içinde tutacak. *
\par 9 Ey Tanri, tapinaginda, Ne kadar vefali oldugunu düsünüyoruz.
\par 10 Adin gibi, ey Tanri, övgün de Dünyanin dört bucagina variyor. Sag elin zafer dolu.
\par 11 Sevinsin Siyon Dagi, Cossun Yahuda beldeleri Senin yargilarinla!
\par 12 Siyon'un çevresini gezip dolanin, Kulelerini sayin,
\par 13 Surlarina dikkatle bakin, Kalelerini yoklayin ki, Gelecek kusaga anlatasiniz:
\par 14 Bu Tanri sonsuza dek bizim Tanrimiz olacak, Bize hep yol gösterecektir.

\chapter{49}

\par 1 Ey bütün halklar, dinleyin! Kulak verin hepiniz, ey dünyada yasayanlar,
\par 2 Halk çocuklari, bey çocuklari, Zenginler, yoksullar!
\par 3 Bilgelik dökülecek agzimdan, Anlayis saglayacak içimdeki düsünceler,
\par 4 Kulak verecegim özdeyislere, Lirle yorumlayacagim bilmecemi.
\par 5 Niçin korkayim kötü günlerde Niyeti bozuk düsmanlarim çevremi sarinca?
\par 6 Onlar varliklarina güvenir, Büyük servetleriyle böbürlenirler.
\par 7 Kimse kimsenin hayatinin bedelini ödeyemez, Tanri'ya fidye veremez.
\par 8 Çünkü hayatin fidyesi büyüktür, Kimse ödemeye yeltenmemeli.
\par 9 Böyle olmasa, Sonsuza dek yasar insan, Mezar yüzü görmez.
\par 10 Kuskusuz herkes biliyor bilgelerin öldügünü, Aptallarla budalalarin yok oldugunu. Mallarini baskalarina birakiyorlar.
\par 11 Mezarlari, sonsuza dek evleri, Kusaklar boyu konutlari olacak, Topraklarina kendi adlarini verseler bile.
\par 12 Bütün gösterisine karsin geçicidir insan, Ölüp giden hayvanlar gibi.
\par 13 Budalalarin yolu, Onlarin sözünü onaylayanlarin sonu budur. *
\par 14 Sürü gibi ölüler diyarina sürülecekler, Ölüm güdecek onlari. Tan agarinca dogrular onlara egemen olacak, Cesetleri çürüyecek, Ölüler diyari onlara konut olacak.
\par 15 Ama Tanri beni Ölüler diyarinin pençesinden kurtaracak Ve yanina alacak.
\par 16 Korkma biri zenginlesirse, Evinin görkemi artarsa.
\par 17 Çünkü ölünce hiçbir sey götüremez, Görkemi onunla mezara gitmez.
\par 18 Yasarken kendini mutlu saysa bile, Basarili olunca övülse bile.
\par 19 Atalarinin kusagina katilacak, Onlar ki asla isik yüzü görmeyecekler.
\par 20 Bütün gösterisine karsin anlayissizdir insan, Ölüp giden hayvanlar gibi.

\chapter{50}

\par 1 Güçlü olan Tanri, RAB konusuyor; Günesin dogdugu yerden battigi yere kadar Yeryüzünün tümüne sesleniyor.
\par 2 Güzelligin dorugu Siyon'dan Parildiyor Tanri.
\par 3 Tanrimiz geliyor, sessiz kalmayacak, Önünde yanan ates her seyi kül ediyor, Çevresinde siddetli bir firtina esiyor.
\par 4 Halkini yargilamak için Yere göge sesleniyor:
\par 5 "Toplayin önüme sadik kullarimi, Kurban keserek benimle antlasma yapanlari."
\par 6 Gökler O'nun dogrulugunu duyuruyor, Çünkü yargiç Tanri'nin kendisidir. *
\par 7 "Ey halkim, dinle de konusayim, Ey Israil, sana karsi taniklik edeyim: Ben Tanri'yim, senin Tanrin'im!
\par 8 Kurbanlarindan ötürü seni azarlamiyorum, Yakmalik sunularin* sürekli önümde.
\par 9 Ne evinden bir boga, Ne de agillarindan bir teke alacagim.
\par 10 Çünkü bütün orman yaratiklari, Daglardaki bütün hayvanlar benimdir.
\par 11 Daglardaki bütün kuslari korurum, Kirlardaki bütün yabanil hayvanlar benimdir.
\par 12 Aciksam sana söylemezdim, Çünkü bütün dünya ve içindekiler benimdir.
\par 13 Ben boga eti yer miyim? Ya da keçi kani içer miyim?
\par 14 Tanri'ya sükran kurbani sun, Yüceler Yücesi'ne adadigin adaklari yerine getir.
\par 15 Sikintili gününde seslen bana, Seni kurtaririm, sen de beni yüceltirsin.
\par 16 Ama Tanri kötüye söyle diyor: "Kurallarimi ezbere okumaya Ya da antlasmami agzina almaya ne hakkin var?
\par 17 Çünkü yola getirilmekten nefret ediyor, Sözlerimi arkana atiyorsun.
\par 18 Hirsiz görünce onunla dost oluyor, Zina edenlere ortak oluyorsun.
\par 19 Agzini kötülük için kullaniyor, Dilini yalana kosuyorsun.
\par 20 Oturup kardesine karsi konusur, Annenin ogluna kara çalarsin.
\par 21 Sen bunlari yaptin, ben sustum, Beni kendin gibi sandin. Seni azarliyorum, Suçlarini gözünün önüne seriyorum.
\par 22 "Dikkate alin bunu, ey Tanri'yi unutan sizler! Yoksa parçalarim sizi, kurtaran olmaz.
\par 23 Kim sükran kurbani sunarsa beni yüceltir; Yolunu düzeltene kurtarisimi gösterecegim."

\chapter{51}

\par 1 Ey Tanri, lütfet bana, Sevgin ugruna; Sil isyanlarimi, Sinirsiz merhametin ugruna.
\par 2 Tümüyle yika beni suçumdan, Arit beni günahimdan.
\par 3 Çünkü biliyorum isyanlarimi, Günahim sürekli karsimda.
\par 4 Sana karsi, yalniz sana karsi günah isledim, Senin gözünde kötü olani yaptim. Öyle ki, konusurken hakli, Yargilarken adil olasin.
\par 5 Nitekim suç içinde dogdum ben, Günah içinde annem bana hamile kaldi.
\par 6 Madem sen gönülde sadakat istiyorsun, Bilgelik ögret bana yüregimin derinliklerinde.
\par 7 Beni mercanköskotuyla arit, paklanayim, Yika beni, kardan beyaz olayim.
\par 8 Nese, sevinç sesini duyur bana, Bayram etsin ezdigin kemikler.
\par 9 Bakma günahlarima, Sil bütün suçlarimi.
\par 10 Ey Tanri, temiz bir yürek yarat, Yeniden kararli bir ruh var et içimde.
\par 11 Beni huzurundan atma, Kutsal Ruhun'u benden alma.
\par 12 Geri ver bana sagladigin kurtulus sevincini, Bana destek ol, istekli bir ruh ver.
\par 13 Baskaldiranlara senin yollarini ögreteyim, Günahkârlar geri dönsün sana.
\par 14 Kurtar beni kan dökme suçundan, Ey Tanri, beni kurtaran Tanri, Dilim senin kurtarisini ilahilerle övsün.
\par 15 Ya Rab, aç dudaklarimi, Agzim senin övgülerini duyursun.
\par 16 Çünkü sen kurbandan hoslanmazsin, Yoksa sunardim sana, Yakmalik sunudan* hosnut kalmazsin.
\par 17 Senin kabul ettigin kurban alçakgönüllü bir ruhtur, Alçakgönüllü ve pisman bir yüregi hor görmezsin, ey Tanri.
\par 18 Lütfet, Siyon'a iyilik yap, Yerusalim'in surlarini onar.
\par 19 O zaman dogru sunulan kurbanlar, Yakmalik sunular, tümüyle yakmalik sunular, Seni hosnut kilar; O zaman sunaginda bogalar sunulur.

\chapter{52}

\par 1 Niçin kötülügünle böbürlenirsin, ey kabadayi, Tanri'nin sadik kullarina karsi Bütün gün dilin yikim tasarlar Keskin ustura gibi, ey hilekâr.
\par 3 Iyilikten çok kötülügü, Dogru konusmaktan çok yalani seversin. *
\par 4 Seni hileli dil seni! Her yikici sözü seversin.
\par 5 Ama Tanri seni sonsuza dek yikacak, Seni kapip çadirindan firlatacak, Yasam diyarindan kökünü sökecek.
\par 6 Dogrular bunu görünce korkacak, Gülerek söyle diyecekler:
\par 7 "Iste bu adam, Tanri'ya siginmak istemedi, Servetinin bolluguna güvendi, Baskalarini yikarak güçlendi!"
\par 8 Ama ben Tanri'nin evinde yeseren zeytin agaci gibiyim, Sonsuza dek Tanri'nin sevgisine güvenirim.
\par 9 Sürekli sana sükrederim yaptiklarin için, Sadik kullarinin önünde umut baglarim, Çünkü adin iyidir.

\chapter{53}

\par 1 Akilsiz içinden, "Tanri yok!" der. Insanlar bozuldu, igrençlik aldi yürüdü, Iyilik eden yok.
\par 2 Tanri göklerden bakar oldu insanlara, Akilli, Tanri'ya yönelen biri var mi diye.
\par 3 Hepsi sapti, Tümü yozlasti, Iyilik eden yok, Bir kisi bile!
\par 4 Suç isleyenler görmüyor mu? Halkimi ekmek yer gibi yiyor, Tanri'ya yakarmiyorlar.
\par 5 Ama korkulmayacak yerde korkacaklar, Çünkü Tanri seni kusatanlarin kemiklerini dagitacak, Onlari reddettigi için hepsini utandiracak.
\par 6 Keske Israil'in kurtulusu Siyon'dan gelse! Tanri halkini eski gönencine kavusturunca, Yakup soyu sevinecek, Israil halki cosacak.

\chapter{54}

\par 1 Ey Tanri, beni adinla kurtar, Gücünle akla beni!
\par 2 Ey Tanri, duami dinle, Kulak ver agzimdan çikan sözlere.
\par 3 Çünkü küstahlar bana saldiriyor, Zorbalar canimi almak istiyor, Tanri'ya aldirmiyorlar. *
\par 4 Iste Tanri benim yardimcimdir, Tek destegim Rab'dir.
\par 5 Düsmanlarim yaptiklari kötülügün cezasini bulsun, Sadakatin uyarinca yok et onlari.
\par 6 Ya RAB, sana gönülden bir kurban sunacagim, Adina sükredecegim, çünkü adin iyidir.
\par 7 Beni bütün sikintilarimdan kurtardin, Gözlerim düsmanlarimin yok olusunu gördü.

\chapter{55}

\par 1 Ey Tanri, kulak ver duama, Sirt çevirme yalvarisima!
\par 2 Dikkatini çevir, yanit ver bana. Düsüncelerim beni rahatsiz ediyor, saskinim
\par 3 Düsman sesinden, kötünün baskisindan; Çünkü sikintiya sokuyorlar beni, Öfkeyle üstüme üstüme geliyorlar.
\par 4 Yüregim sizliyor içimde, Ölüm dehseti çöktü üzerime.
\par 5 Korku ve titreme sardi beni, Ürperti kapladi içimi.
\par 6 "Keske güvercin gibi kanatlarim olsaydi!" Dedim kendi kendime, "Uçar, rahatlardim.
\par 7 Uzaklara kaçar, Çöllerde konaklardim. *
\par 8 Sert rüzgara, kasirgaya karsi Hemen bir barinak bulurdum."
\par 9 Saskina çevir kötüleri, ya Rab, karistir dillerini, Çünkü kentte siddet ve çatisma görüyorum.
\par 10 Gece gündüz kent surlari üzerinde dolasirlar, Haksizlik, fesat dolu kentin içi.
\par 11 Yikicilik kentin göbeginde, Zorbalik, hile eksilmez meydanindan.
\par 12 Beni asagilayan bir düsman olsaydi, Katlanabilirdim; Bana küstahlik eden bir hasim olsaydi, Gizlenebilirdim.
\par 13 Ama sensin, bana denk, Yoldasim, yakin arkadasim.
\par 14 Birlikte tatli tatli yarenlik eder, Toplulukla Tanri'nin evine giderdik.
\par 15 Ölüm yakalasin düsmanlarimi ansizin, Diri diri ölüler diyarina insinler; Çünkü içleri ve evleri kötülük dolu.
\par 16 Bense Tanri'ya seslenirim, RAB kurtarir beni.
\par 17 Sabah, öglen, aksam kederimden feryat ederim, O isitir sesimi.
\par 18 Bana karsi girisilen savastan Esenlikle kurtarir canimi, Sayisi çok da olsa karsitlarimin.
\par 19 Öncesizlikten bu yana tahtinda oturan Tanri, Duyacak ve ezecek onlari. Çünkü hiç degismiyor Ve Tanri'dan korkmuyorlar.
\par 20 Yoldasim dostlarina saldirarak Yaptigi antlasmayi bozdu.
\par 21 Agzindan bal damlar, Ama yüreginde savas var. Sözleri yagdan yumusak, Ama yalin birer kiliçtir.
\par 22 Yükünü RAB'be birak, O sana destek olur. Asla izin vermez Dogru insanin sarsilmasina.
\par 23 Ama sen, ey Tanri, ölüm çukuruna atacaksin kötüleri, Günlerinin yarisini görmeyecek katillerle hainler; Bense sana güveniyorum.

\chapter{56}

\par 1 Aci bana, ey Tanri, Çünkü ayak altinda çigniyor insanlar beni, Gün boyu saldirip eziyorlar.
\par 2 Düsmanlarim ayak altinda çigniyor beni her gün, Küstahça saldiriyor çogu.
\par 3 Sana güvenirim korktugum zaman.
\par 4 Tanri'ya, sözünü övdügüm Tanri'ya Güvenirim ben, korkmam. Insan bana ne yapabilir?
\par 5 Gün boyu sözlerimi çarpitiyorlar, Hakkimda hep kötülük tasarliyorlar.
\par 6 Fesatlik için ugrasiyor, pusuya yatiyor, Adimlarimi gözlüyor, canimi almak istiyorlar.
\par 7 Kötülüklerinin cezasindan kurtulacaklar mi? Ey Tanri, halklari öfkeyle yere çal!
\par 8 Çektigim acilari kaydettin, Gözyaslarimi tulumunda biriktirdin! Bunlar defterinde yazili degil mi?
\par 9 Seslendigim zaman, Düsmanlarim geri çekilecek. Biliyorum, Tanri benden yana.
\par 10 Sözünü övdügüm Tanri'ya, Sözünü övdügüm RAB'be,
\par 11 Tanri'ya güvenirim ben, korkmam; Insan bana ne yapabilir?
\par 12 Ey Tanri, sana adaklar adamistim, Sükran kurbanlari sunmaliyim simdi.
\par 13 Çünkü canimi ölümden kurtardin, Ayaklarimi tökezlemekten korudun; Iste yasam isiginda, Tanri huzurunda yürüyorum.

\chapter{57}

\par 1 Aci bana, ey Tanri, aci, Çünkü sana siginiyorum; Felaket geçinceye kadar, Kanatlarinin gölgesine siginacagim.
\par 2 Yüce Tanri'ya, Benim için her seyi yapan Tanri'ya sesleniyorum.
\par 3 Gökten gönderip beni kurtaracak, Beni ezmek isteyenleri azarlayacak, * Sevgisini, sadakatini gösterecektir.
\par 4 Aslanlarin arasindayim, Alev kusan insanlar arasinda yatarim, Mizrak gibi, ok gibi disleri, Keskin kiliç gibi dilleri.
\par 5 Yüceligini göster göklerin üstünde, ey Tanri, Görkemin bütün yeryüzünü kaplasin!
\par 6 Ayaklarim için ag serdiler, Çöktüm; Yoluma çukur kazdilar, Içine kendileri düstüler.
\par 7 Kararliyim, ey Tanri, kararliyim, Ezgiler, ilahiler söyleyecegim.
\par 8 Uyan, ey canim, Uyan, ey lir, ey çenk, Seheri ben uyandirayim!
\par 9 Halklarin arasinda sana sükürler sunayim, ya Rab, Uluslarin arasinda seni ilahilerle öveyim.
\par 10 Çünkü sevgin göklere erisir, Sadakatin gökyüzüne ulasir.
\par 11 Yüceligini göster göklerin üstünde, ey Tanri, Görkemin bütün yeryüzünü kaplasin!

\chapter{58}

\par 1 Ey yöneticiler, gerçekten adil mi karar verirsiniz? Dogru mu yargilarsiniz insanlari?
\par 2 Hayir! Hep haksizlik tasarlarsiniz içinizde, Zorbalik saçar elleriniz yeryüzüne.
\par 3 Kötüler daha ana rahmindeyken yoldan çikar, Dogdu dogali yalan söyleyerek sapar.
\par 4 Zehirleri yilan zehiri gibidir. Kulaklari tikali bir kobrayi andirirlar,
\par 5 Usta büyücülerin, Afsuncularin sesini duymak istemeyen bir kobrayi.
\par 6 Ey Tanri, kir onlarin agzinda dislerini, Sök genç aslanlarin azi dislerini, ya RAB!
\par 7 Akip giden su gibi yok olsunlar. Yaylarini gerince oklarinin ucu kirilsin.
\par 8 Süründükçe eriyen sümüklüböcege dönsünler. Düsük çocuk gibi günes yüzü görmesinler.
\par 9 Kazanlariniz diken atesini daha duymadan, Yasi da kurusu da kasirgayla savrulacak kötülerin.
\par 10 Dogru adam alinan öcü görünce sevinecek Ve ayaklarini kötünün kaninda yikayacak.
\par 11 O zaman insanlar, "Gerçekten dogrulara ödül var" diyecek, "Gerçekten dünyayi yargilayan bir Tanri var."

\chapter{59}

\par 1 Kurtar beni düsmanlarimdan, ey Tanrim, Kalem ol hasimlarima karsi.
\par 2 Kurtar beni suç isleyenlerden, Uzak tut kanli katillerden.
\par 3 Bak, canimi almak için pusu kuruyorlar, Güçlüler bana karsi birlesiyorlar, Oysa baskaldirmadim, günahim yok, ya RAB.
\par 4 Suç islemedigim halde, Kosusup hazirlaniyorlar. Kalk bana yardim etmek için, halime bak!
\par 5 Sen, ya RAB, Her Seye Egemen Tanri, Israil'in Tanrisi, Uyan bütün uluslari cezalandirmak için, Acima bu suçlu hainlere! *
\par 6 Aksam döner, köpek gibi hirlayip Sinsi sinsi kenti dolasirlar.
\par 7 Bak, neler dökülür agizlarindan, Kiliç çikar dudaklarindan. "Kim duyacak?" derler.
\par 8 Ama sen onlara gülersin, ya RAB, Bütün uluslarla eglenirsin.
\par 9 Gücüm sensin, seni gözlüyorum, Çünkü kalemsin, ey Tanri.
\par 10 Tanrim sevgisiyle karsilar beni, Bana düsmanlarimin yikimini gösterir.
\par 11 Onlari öldürme, yoksa halkim unutur, Gücünle dagit ve alçalt onlari, Ya Rab, kalkanimiz bizim.
\par 12 Agizlarinin günahi, dudaklarindan çikan söz yüzünden, Gururlarinin tuzagina düssünler. Okuduklari lanet, söyledikleri yalan yüzünden
\par 13 Yok et onlari gazabinla, yok et, tükensinler; Bilsinler ki, Tanri'nin egemenligi Yakup soyundan Yeryüzünün ucuna kadar ulasir.
\par 14 Aksam döner, köpek gibi hirlayip Sinsi sinsi kenti dolasirlar.
\par 15 Yiyecek bulmak için gezerler, Doymazlarsa ulurlar.
\par 16 Bense gücün için sabah ezgiler söyleyecek, Sevgini sevinçle dile getirecegim. Çünkü sen bana kale, Sikintili günümde siginak oldun.
\par 17 Gücüm sensin, seni ilahilerle övecegim, Çünkü kalem, beni seven Tanri sensin.

\chapter{60}

\par 1 Bizi reddettin, parladin bize karsi, ey Tanri, Öfkelendin; eski halimize döndür bizi!
\par 2 Salladin yeri, yariklar açtin; Onar çatlaklarini, çünkü yer sarsiliyor.
\par 3 Halkina sikinti çektirdin, Sersemletici bir sarap içirdin bize.
\par 4 Sancak verdin senden korkanlara, Okçulara karsi açsinlar diye. *
\par 5 Kurtar bizi sag elinle, yardim et, Sevdiklerin özgürlüge kavussun diye!
\par 6 Tanri söyle konustu kutsal yerinde: "Sekem'i sevinçle bölüstürecek, Sukkot Vadisi'ni ölçecegim.
\par 7 Gilat benimdir, Manasse de benim, Efrayim migferim, Yahuda asam.
\par 8 Moav yikanma legenim, Edom'un üzerine çarigimi firlatacagim, Filist'e zaferle haykiracagim."
\par 9 Kim beni surlu kente götürecek? Kim bana Edom'a kadar yol gösterecek?
\par 10 Ey Tanri, sen bizi reddetmedin mi? Ordularimiza öncülük etmiyor musun artik?
\par 11 Yardim et bize düsmana karsi, Çünkü bostur insan yardimi.
\par 12 Tanri'yla zafer kazaniriz, O çigner düsmanlarimizi.

\chapter{61}

\par 1 Ey Tanri, yakarisimi isit, Duama kulak ver!
\par 2 Sana seslenirim yeryüzünün öbür ucundan, Yüregime hüzün çökünce. Erisemeyecegim yüksek bir kayaya çikar beni,
\par 3 Çünkü sen benim için siginak, Düsmana karsi güçlü bir kule oldun.
\par 4 Çadirinda sonsuza dek oturmak Ve kanatlarinin gölgesine siginmak isterim. *
\par 5 Çünkü sen, ey Tanri, adaklarimi duydun, Adindan korkanlarin mirasini bana verdin.
\par 6 Kralin günlerine gün kat, Yillari yüzyillar olsun!
\par 7 Tanri'nin huzurunda sonsuza dek tahtinda otursun; Onu sevgin ve sadakatinle koru!
\par 8 O zaman adini hep ilahilerle övecegim, Her gün adaklarimi yerine getirecegim.

\chapter{62}

\par 1 Canim yalniz Tanri'da huzur bulur, Kurtulusum O'ndan gelir.
\par 2 Tek kayam, kurtulusum, Kalem O'dur, asla sarsilmam.
\par 3 Birini ezmek için daha ne vakte kadar Hep birlikte üstüne saldiracaksiniz, Egri bir duvara, Yerinden oynamis bir çite saldirir gibi?
\par 4 Tek düsünceleri onu doruktan indirmektir. Yalandan zevk alirlar. Agizlariyla hayirdua ederken, Içlerinden lanet okurlar. *
\par 5 Ey canim, yalniz Tanri'da huzur bul, Çünkü umudum O'ndadir.
\par 6 Tek kayam, kurtulusum, Kalem O'dur, sarsilmam.
\par 7 Kurtulusum ve onurum Tanri'ya baglidir, Güçlü kayam, siginagim O'dur.
\par 8 Ey halkim, her zaman O'na güven, Içini dök O'na, Çünkü Tanri siginagimizdir.
\par 9 Siradan insan ancak bir soluk, Soylu insansa bir yalandir. Tartiya kondugunda ikisi birlikte soluktan hafiftir.
\par 10 Zorbaliga güvenmeyin, yagma malla övünmeyin; Varliginiz artsa bile, ona gönül baglamayin.
\par 11 Tanri bir sey söyledi, Ben iki sey duydum: Güç Tanri'nindir,
\par 12 Sevgi de senin, ya Rab! Çünkü sen herkese, yaptiginin karsiligini verirsin.

\chapter{63}

\par 1 Ey Tanri, sensin benim Tanrim, Seni çok özlüyorum, Canim sana susamis, Kurak, yorucu, susuz bir diyarda, Bütün varligimla seni ariyorum.
\par 2 Kutsal yerde baktim sana, Gücünü, görkemini görmek için.
\par 3 Senin sevgin yasamdan iyidir, Bu yüzden dudaklarim seni yüceltir.
\par 4 Ömrümce sana övgüler sunacagim, Senin adinla ellerimi kaldiracagim.
\par 5 Zengin yiyeceklere doyarcasina doyacagim sana, Sakiyan dudaklarla agzim sana övgüler sunacak.
\par 6 Yatagima uzaninca seni anarim, Gece boyunca derin derin seni düsünürüm.
\par 7 Çünkü sen bana yardimci oldun, Kanatlarinin gölgesinde sevincimi dile getiririm.
\par 8 Canim sana simsiki sarilir, Sag elin bana destek olur.
\par 9 Ama canimi almak isteyenler, Yerin derinliklerine inecek,
\par 10 Kilicin agzina atilacak, Çakallara yem olacak.
\par 11 Kralsa Tanri'da sevinç bulacak. Tanri'nin adiyla ant içenlerin hepsi övünecek, Yalancilarin agziysa kapanacak.

\chapter{64}

\par 1 Ey Tanri, kulak ver sesime yakindigim zaman, Hayatimi düsman korkusundan koru.
\par 2 Kötülerin gizli tasarilarindan, O suçlu güruhun samatasindan esirge beni.
\par 3 Onlar dillerini kiliç gibi bilemis, Aci sözlerini ok gibi hedefe yöneltmisler,
\par 4 Pusularindan masum insanin üzerine atmak için. Ansizin vururlar, hiç çekinmeden.
\par 5 Birbirlerini kötülük yapmaya iter, Gizli tuzaklar tasarlarken, "Kim görecek?" derler.
\par 6 Haksizlik yapmayi düsünür, "Kusursuz bir plan yaptik!" derler. Insanin içi ve yüregi derin bir sirdir, bilinmez.
\par 7 Ama Tanri onlara ok atacak, Ansizin yaralanacaklar.
\par 8 Dilleri yüzünden yikima ugrayacaklar, Hallerini gören herkes alayla bas sallayacak.
\par 9 Bütün insanlar korkuya kapilacak, Tanri'nin isini duyuracak, O'nun yaptiklari üzerinde düsünecekler.
\par 10 Dogru insan RAB'de sevinç bulacak, O'na siginacak, Bütün temiz yürekliler O'nu övecek.

\chapter{65}

\par 1 Ey Tanri, Siyon'da seni övgü bekliyor, Yerine getirilecek sana adanan adaklar.
\par 2 Ey sen, dualari isiten, Bütün insanlar sana gelecek.
\par 3 Suçlarimizin altinda ezildik, Ama sen isyanlarimizi bagislarsin.
\par 4 Ne mutlu avlularinda otursun diye Seçip kendine yaklastirdigin kisiye! Evinin, kutsal tapinaginin Nimetlerine doyacagiz.
\par 5 Ey bizi kurtaran Tanri, Müthis isler yaparak Zaferle yanitlarsin bizi. Sen yeryüzünün dört bucaginda, Uzak denizlerdekilerin umudusun;
\par 6 Kudret kusanan, Gücüyle daglari kuran,
\par 7 Denizlerin kükremesini, Dalgalarin gümbürtüsünü, Halklarin kargasasini yatistiran sensin.
\par 8 Dünyanin öbür ucunda yasayanlar Korkuya kapilir senin belirtilerin karsisinda. Dogudan batiya kadar insanlara Sevinç çigliklari attirirsin.
\par 9 Topraga bakar, çok verimli kilarsin, Onu zenginlige bogarsin. Ey Tanri, irmaklarin suyla doludur, Insanlara tahil saglarsin, Çünkü sen topragi söyle hazirlarsin:
\par 10 Sabanin açtigi yariklari bolca sular, Sirtlarini düzlersin. Yagmurla topragi yumusatir, Ürünlerine bereket katarsin.
\par 11 Iyiliklerinle yili taçlandirirsin, Arabalarinin geçtigi yollardan bolluk akar,
\par 12 Otlaklar yesillenir, Tepeler sevince bürünür,
\par 13 Çayirlar sürülerle bezenir, Vadiler ekinle örtünür, Sevinçten haykirir, ezgi söylerler.

\chapter{66}

\par 1 Ey yeryüzündeki bütün insanlar, Tanri'ya sevinç çigliklari atin!
\par 2 Adinin yüceligine ilahiler söyleyin, O'na görkemli övgüler sunun!
\par 3 "Ne müthis islerin var!" deyin Tanri'ya, "Öyle büyük gücün var ki, Düsmanlarin egiliyor önünde.
\par 4 Bütün yeryüzü sana tapiniyor, Ilahiler okuyor, adini ilahilerle övüyor." *
\par 5 Gelin, bakin Tanri'nin neler yaptigina! Ne müthis isler yapti insanlar arasinda:
\par 6 Denizi karaya çevirdi, Atalarimiz yaya geçtiler irmaktan. Yaptigina sevindik orada.
\par 7 Kudretiyle sonsuza dek egemenlik sürer, Gözleri uluslari süzer; Baskaldiranlar gurura kapilmasin!
\par 8 Ey halklar, Tanrimiz'a sükredin, Övgülerini duyurun.
\par 9 Hayatimizi koruyan, Ayaklarimizin kaymasina izin vermeyen O'dur.
\par 10 Sen bizi sinadin, ey Tanri, Gümüs aritir gibi arittin.
\par 11 Aga düsürdün bizi, Sirtimiza agir yük vurdun.
\par 12 Insanlari basimiza çikardin, Atesten, sudan geçtik. Ama sonra bizi bolluga kavusturdun.
\par 13 Yakmalik sunularla* evine girecegim, Adaklarimi yerine getirecegim,
\par 14 Sikinti içindeyken dudaklarimdan dökülen, Agzimdan çikan adaklari.
\par 15 Yakilan koçlarin dumaniyla semiz hayvanlardan Sana yakmalik sunular sunacagim, Tekeler, sigirlar kurban edecegim.
\par 16 Gelin, dinleyin, ey sizler, Tanri'dan korkanlar, Benim için neler yaptigini size anlatayim.
\par 17 Agzimla O'na yakardim, Övgüsü dilimden düsmedi.
\par 18 Yüregimde kötülüge yer verseydim, Rab beni dinlemezdi.
\par 19 Oysa Tanri dinledi beni, Kulak verdi duamin sesine.
\par 20 Övgüler olsun Tanri'ya, Çünkü duami geri çevirmedi, Sevgisini benden esirgemedi.

\chapter{67}

\par 1 Tanri bize lütfetsin, bolluk versin, Yüzünün isigi üzerimize parlasin. *
\par 2 Öyle ki, yeryüzünde yolun, Bütün uluslar arasinda kurtarici gücün bilinsin.
\par 3 Halklar sana sükretsin, ey Tanri, Bütün halklar sana sükretsin!
\par 4 Uluslar sevinsin, sevinçten çiglik atsin, Çünkü sen halklari adaletle yargilarsin, Yeryüzündeki uluslara yol gösterirsin.
\par 5 Halklar sana sükretsin, ey Tanri, Bütün halklar sana sükretsin!
\par 6 Toprak ürününü verdi, Tanri, Tanrimiz, bizi bolluga kavustursun.
\par 7 Tanri bize bolluk versin, Dünyanin dört bucagindakiler O'ndan korksun! Müzik sefi için - Davut'un mezmuru - Ilahi

\chapter{68}

\par 1 Kalksin Tanri, dagilsin düsmanlari, Kaçsin önünden O'ndan nefret edenler!
\par 2 Dagitsin onlari dagilan duman gibi; Atesin karsisinda eriyen balmumu gibi Yok olsun kötüler Tanri'nin önünde!
\par 3 Ancak dogrular sevinsin, Bayram etsinler Tanri'nin önünde, Neseyle cossunlar.
\par 4 Tanri'ya ezgiler söyleyin, adini ilahilerle övün, Çölleri geçecek biniciye yol hazirlayin; O'nun adi RAB'dir, bayram edin önünde!
\par 5 Kutsal konutundaki Tanri, Öksüzlerin babasi, dul kadinlarin savunucusudur.
\par 6 Tanri kimsesizlere ev verir, Tutsaklari özgürlüge ve gönence kavusturur, Ama baskaldiranlar kurak yerde oturur.
\par 7 Ey Tanri, sen halkina öncülük ettiginde, Çölde yürüdügünde, *
\par 8 Yer sarsildi, Göklerden yagmur bosandi Tanri'nin önünde, Sina Dagi sarsildi Tanri'nin, Israil'in Tanrisi'nin önünde.
\par 9 Bol yagmurlar yagdirdin, ey Tanri, Canlandirdin yorgun düsen yurdunu.
\par 10 Halkin oraya yerlesti, Iyiliginle mazlumlarin geçimini sagladin, ey Tanri.
\par 11 Rab buyruk verdi, Büyük bir kadin toplulugu duyurdu müjdeyi:
\par 12 "Kaçiyor, kaçiyor ordularin krallari! Evi bekleyen kadinlar ganimeti paylasiyor.
\par 13 Agillarin arasinda uyurken, Kanatlari gümüs, tüyleri piril piril altinla kapli Bir güvercine benzersiniz."
\par 14 Her Seye Gücü Yeten, krallari dagitirken, Sanki Salmon Dagi'na kar yagiyordu.
\par 15 Ey Basan Dagi, Tanri Dagi! Ey Basan Dagi, doruklari ulu dag!
\par 16 Ey ulu daglar, niçin yan gözle bakiyorsunuz Tanri'nin yerlesmek için seçtigi daga? Evet, RAB orada sonsuza dek oturacaktir.
\par 17 Tanri'nin savas arabalari sayisizdir, Rab kutsallik içinde Sina'dan geldi.
\par 18 Sen yüksege çiktin, tutsaklari pesine taktin, Insanlardan, baskaldiranlardan bile armaganlar aldin, Oraya yerlesmek için, ya RAB Tanri.
\par 19 Her gün yükümüzü tasiyan Rab'be, Bizi kurtaran Tanri'ya övgüler olsun.
\par 20 Tanrimiz kurtarici bir Tanri'dir, Ölümden kurtaris yalniz Egemen RAB'be özgüdür.
\par 21 Kuskusuz Tanri düsmanlarinin basini, Suçlu yasayanlarin killi kafasini ezer.
\par 22 Rab, "Onlari Basan'dan, Denizin derinliklerinden geri getirecegim" der,
\par 23 "Öyle ki, ayaklarini düsmanlarinin kanina batirasin, Köpeklerinin dili de onlardan payini alsin."
\par 24 Ey Tanri, senin zafer alayini, Tanrim'in, Kralim'in kutsal yere törenle gelisini gördüler:
\par 25 Basta okuyucular, arkada çalgicilar, Ortada tef çalan genç kizlar.
\par 26 "Ey sizler, Israil soyundan gelenler, Toplantilarinizda Tanri'ya, RAB'be övgüler sunun!"
\par 27 Önde en küçük oymak Benyamin, Kalabalik halinde Yahuda önderleri, Zevulun ve Naftali önderleri oradalar!
\par 28 Ey Tanri, Yerusalim'deki tapinagindan göster gücünü, Bizim için kullandigin gücünü, ey Tanri. Krallar sana armaganlar sunacak.
\par 30 Azarla kamislar arasinda yasayan hayvani, Halklarin buzagilariyla bogalar sürüsünü, Çigne ayaklarinla gümüse gönül verenleri, Dagit savastan zevk alan halklari!
\par 31 Misir'dan elçiler gelecek, Kûslular* ellerini Tanri'ya dogru kaldiriverecek.
\par 32 Ey yeryüzünün kralliklari, Tanri'ya ezgiler söyleyin, Ilahilerle övün Rab'bi,
\par 33 Göklere, kadim göklere binmis olani. Iste sesiyle, güçlü sesiyle gürlüyor!
\par 34 Tanri'nin gücünü taniyin; O'nun yüceligi Israil'in üzerinde, Gücü göklerdedir.
\par 35 Ne heybetlisin, ey Tanri, tapinaginda! Israil'in Tanrisi'na, Halkina güç, kudret veren Tanri'ya övgüler olsun!

\chapter{69}

\par 1 Kurtar beni, ey Tanri, Sular boyuma ulasti.
\par 2 Dipsiz bataga gömülüyorum, Basacak yer yok. Derin sulara battim, Sellere kapildim.
\par 3 Tükendim feryat etmekten, Bogazim kurudu; Gözlerimin feri sönüyor Tanrim'i beklemekten.
\par 4 Yok yere benden nefret edenler Saçlarimdan daha çok. Kalabaliktir canima kasteden haksiz düsmanlarim. Çalmadigim mali nasil geri verebilirim?
\par 5 Akilsizligimi biliyorsun, ey Tanri, Suçlarim senden gizli degil.
\par 6 Ya Rab, Her Seye Egemen RAB, Utanmasin sana umut baglayanlar benim yüzümden! Ey Israil'in Tanrisi, Benim yüzümden sana yönelenler rezil olmasin!
\par 7 Senin ugruna hakarete katlandim, Utanç kapladi yüzümü.
\par 8 Kardeslerime yabanci, Annemin öz ogullarina uzak kaldim.
\par 9 Çünkü evin için gösterdigim gayret beni yiyip bitirdi, Sana edilen hakaretlere ben ugradim.
\par 10 Oruç* tutup aglayinca, Yine hakarete ugradim.
\par 11 Çula büründügüm zaman Alay konusu oldum.
\par 12 Kent kapisinda oturanlar beni çekistiriyor, Sarhoslarin türküsü oldum.
\par 13 Ama benim duam sanadir, ya RAB. Ey Tanri, sevginin bolluguyla, Güvenilir kurtarisinla uygun gördügünde Yanitla beni.
\par 14 Beni çamurdan kurtar, Izin verme batmama; Benden nefret edenlerden, Derin sulardan kurtulayim.
\par 15 Seller beni sürüklemesin, Engin beni yutmasin, Ölüm çukuru agzini üstüme kapamasin.
\par 16 Yanit ver bana, ya RAB, Çünkü sevgin iyidir. Yüzünü çevir bana büyük merhametinle!
\par 17 Kulundan yüzünü gizleme, Çünkü sikintidayim, hemen yanitla beni!
\par 18 Yaklas bana, kurtar canimi, Al basimdan düsmanlarimi.
\par 19 Bana nasil hakaret edildigini, Utandigimi, rezil oldugumu biliyorsun; Düsmanlarimin hepsi senin önünde.
\par 20 Hakaret kalbimi kirdi, dertliyim, Acilarimi paylasacak birini bekledim, çikmadi, Avutacak birini aradim, bulamadim.
\par 21 Yiyecegime zehir kattilar, Sirke içirdiler susadigimda.
\par 22 Önlerindeki sofra tuzak olsun onlara, Yandaslari için kapan olsun!
\par 23 Gözleri kararsin, göremesinler! Bellerini hep bükük tut!
\par 24 Gazabini yagdir üzerlerine, Öfkenin atesi yapissin yakalarina!
\par 25 Issiz kalsin konaklari, Çadirlarinda oturan olmasin!
\par 26 Çünkü senin vurdugun insanlara zulmediyor, Yaraladigin insanlarin acisini konusuyorlar.
\par 27 Ceza yagdir baslarina, Senin tarafindan aklanmasinlar!
\par 28 Yasam kitabindan silinsin adlari, Dogrularla yan yana yazilmasinlar!
\par 29 Bense ezilmis ve kederliyim, Senin kurtarisin, ey Tanri, bana bir kale olsun!
\par 30 Tanri'nin adini ezgilerle övecegim, ªükranlarimla O'nu yüceltecegim.
\par 31 RAB'bi bir öküzden, Boynuzlu, tirnakli bir bogadan Daha çok hosnut eder bu.
\par 32 Mazlumlar bunu görünce sevinsin, Ey Tanri'ya yönelen sizler, yüreginiz canlansin.
\par 33 Çünkü RAB yoksullari isitir, Kendi tutsak halkini hor görmez.
\par 34 O'na övgüler sunun, ey yer, gök, Denizler ve onlardaki bütün canlilar!
\par 35 Çünkü Tanri Siyon'u kurtaracak, Yahuda kentlerini onaracak; Halk oraya yerlesip sahibi olacak.
\par 36 Kullarinin çocuklari orayi miras alacak, O'nun adini sevenler orada oturacak.

\chapter{70}

\par 1 Ey Tanri, kurtar beni! Yardimima kos, ya RAB!
\par 2 Utansin canimi almaya çalisanlar, Yüzleri kizarsin! Geri dönsün zararimi isteyenler, Rezil olsunlar!
\par 3 Bana, "Oh! Oh!" çekenler Geri çekilsin utançlarindan!
\par 4 Sende nese ve sevinç bulsun Bütün sana yönelenler! "Tanri yücedir!" desin hep Senin kurtarisini özleyenler!
\par 5 Bense, mazlum ve yoksulum, Ey Tanri, yardimima kos! Yardimcim ve kurtaricim sensin, Geç kalma, ya RAB!

\chapter{71}

\par 1 Ya RAB, sana siginiyorum, Utandirma beni hiçbir zaman!
\par 2 Adaletinle kurtar beni, tehlikeden uzaklastir, Kulak ver bana, kurtar beni!
\par 3 Siginacak kayam ol, Her zaman basvurabilecegim; Buyruk ver, kurtulayim, Çünkü kayam ve kalem sensin.
\par 4 Ey Tanrim, kurtar beni Kötünün elinden, haksizin, gaddarin pençesinden!
\par 5 Çünkü umudum sensin, ey Egemen RAB, Gençligimden beri dayanagim sensin.
\par 6 Dogdugum günden beri sana güveniyorum, Beni ana rahminden çikaran sensin. Övgülerim hep sanadir.
\par 7 Birçoklari için iyi bir örnek oldum, Çünkü sen güçlü siginagimsin.
\par 8 Agzimdan sana övgü eksilmez, Gün boyu yüceligini anarim.
\par 9 Yaslandigimda beni reddetme, Gücüm tükendiginde beni terk etme!
\par 10 Çünkü düsmanlarim benden söz ediyor, Beni öldürmek isteyenler birbirine danisiyor,
\par 11 "Tanri onu terk etti" diyorlar, "Kovalayip yakalayin, Kurtaracak kimsesi yok!"
\par 12 Ey Tanri, benden uzak durma, Tanrim, yardimima kos!
\par 13 Utansin, yok olsun beni suçlayanlar, Utanca, rezalete bürünsün kötülügümü isteyenler.
\par 14 Ama ben her zaman umutluyum, Sana övgü üstüne övgü dizecegim.
\par 15 Gün boyu senin zaferini, Kurtarisini anlatacagim, Ölçüsünü bilmesem de.
\par 16 Ey Egemen RAB, gelip yigitliklerini, Senin, yalniz senin zaferini duyuracagim.
\par 17 Ey Tanri, çocuklugumdan beri beni sen yetistirdin, Senin harikalarini hâlâ anlatiyorum.
\par 18 Yaslanip saçlarima ak düsse bile Terk etme beni, ey Tanri, Gücünü gelecek kusaga, Kudretini sonrakilere anlatana dek.
\par 19 Ey Tanri, dogrulugun göklere erisiyor, Büyük isler yaptin, Senin gibisi var mi, ey Tanri?
\par 20 Sen ki, bana birçok kötü sikinti gösterdin, Bana yeniden yasam verecek, Beni topragin derinliklerinden çikaracaksin.
\par 21 Sayginligimi artiracak, Yine beni avutacaksin.
\par 22 Ben de seni, Senin sadakatini çenkle övecegim, ey Tanrim, Lir çalarak seni ilahilerle övecegim, Ey Israil'in Kutsali!
\par 23 Seni ilahilerle överken, Dudaklarimla, varligimla sevincimi dile getirecegim, Çünkü sen beni kurtardin.
\par 24 Dilim gün boyu senin zaferinden söz edecek, Çünkü kötülügümü isteyenler Utanip rezil oldu.

\chapter{72}

\par 1 Ey Tanri, adaletini krala, Dogrulugunu kral ogluna armagan et.
\par 2 Senin halkini dogrulukla, Mazlum kullarini adilce yargilasin!
\par 3 Daglar, tepeler, Halka adilce gönenç getirsin!
\par 4 Mazlumlara hakkini versin, Yoksullarin çocuklarini kurtarsin, Zalimleriyse ezsin!
\par 5 Günes ve ay durdukça, Kral kusaklar boyunca yasasin;
\par 6 Yeni biçilmis çayira düsen yagmur gibi, Topragi sulayan bereketli yagmurlar gibi olsun!
\par 7 Onun günlerinde dogruluk serpilip gelissin, Ay isidigi sürece esenlik artsin!
\par 8 Egemenlik sürsün denizden denize, Firat'tan yeryüzünün ucuna dek!
\par 9 Çöl kabileleri diz çöksün önünde, Düsmanlari toz yalasin.
\par 10 Tarsis'in ve kiyi ülkelerinin krallari Ona haraç getirsin, Saba ve Seva krallari armaganlar sunsun!
\par 11 Bütün krallar önünde yere kapansin, Bütün uluslar ona kulluk etsin!
\par 12 Çünkü yardim isteyen yoksulu, Dayanagi olmayan düskünü o kurtarir.
\par 13 Yoksula, düsküne acir, Düskünlerin canini kurtarir.
\par 14 Baskidan, zorbaliktan özgür kilar onlari, Çünkü onun gözünde onlarin kani degerlidir.
\par 15 Yasasin kral! Ona Saba altini versinler; Durmadan dua etsinler onun için, Gün boyu onu övsünler!
\par 16 Ülkede bol bugday olsun, Dag baslarinda dalgalansin! Basaklari Lübnan gibi verimli olsun, Kent halki ot gibi serpilip çogalsin!
\par 17 Kralin adi sonsuza dek yasasin, Günes durdukça adi var olsun, Onun araciligiyla insanlar kutsansin, Bütün uluslar "Ne mutlu ona" desin!
\par 18 RAB Tanri'ya, Israil'in Tanrisi'na övgüler olsun, Harikalar yaratan yalniz O'dur.
\par 19 Yüce adina sonsuza dek övgüler olsun, Bütün yeryüzü O'nun yüceligiyle dolsun! Amin! Amin!
\par 20 Isay oglu Davut'un dualari burada bitiyor.

\chapter{73}

\par 1 Tanri gerçekten Israil'e, Yüregi temiz olanlara karsi iyidir.
\par 2 Ama benim ayaklarim neredeyse tökezlemis, Adimlarim az kalsin kaymisti.
\par 3 Çünkü kötülerin gönencini gördükçe, Küstahlari kiskaniyordum.
\par 4 Onlar aci nedir bilmezler, Bedenleri saglikli ve semizdir.
\par 5 Baskalarinin derdini bilmez, Onlar gibi çile çekmezler.
\par 6 Bu yüzden gurur onlarin gerdanligi, Zorbalik onlari örten bir giysi gibidir.
\par 7 ªismanliktan gözleri disari firlar, Içleri kötülük kazani gibi kaynar.
\par 8 Insanlarla eglenir, kötü niyetle konusur, Tepeden bakar, baskiyla tehdit ederler.
\par 9 Göklere karsi agizlarini açarlar, Bos sözleri yeryüzünü dolasir.
\par 10 Bu yüzden halk onlardan yana döner, Sözlerini agzi açik dinler.
\par 11 Derler ki, "Tanri nasil bilir? Bilgisi var mi Yüceler Yücesi'nin?"
\par 12 Iste böyledir kötüler, Hep tasasiz, sürekli varliklarini artirirlar.
\par 13 Anlasilan bos yere yüregimi temiz tutmusum, Ellerimi yikamisim suçsuzum diye.
\par 14 Gün boyu içim içimi yiyor, Her sabah azap çekiyorum.
\par 15 "Ben de onlar gibi konusayim" deseydim, Senin çocuklarina ihanet etmis olurdum.
\par 16 Bunu anlamak için düsündügümde, Zor geldi bana,
\par 17 Tanri'nin Tapinagi'na girene dek; O zaman anladim sonlarinin ne olacagini.
\par 18 Gerçekten onlari kaygan yere koyuyor, Yikima sürüklüyorsun.
\par 19 Nasil da bir anda yok oluyor, Siliniveriyorlar dehset içinde!
\par 20 Uyanan birisi için rüya nasilsa, Sen de uyaninca, ya Rab, Hor göreceksin onlarin görüntüsünü.
\par 21 Kalbim kirildiginda, Içim aci doldugunda,
\par 22 Akilsiz ve bilgisizdim, Karsinda bir hayvan gibi.
\par 23 Yine de sürekli seninleyim, Sag elimden tutarsin beni.
\par 24 Ögütlerinle yol gösterir, Beni sonunda yücelige eristirirsin.
\par 25 Senden baska kimim var göklerde? Istemem senden baskasini yeryüzünde.
\par 26 Bedenim ve yüregim tükenebilir, Ama Tanri yüregimde güç, Bana düsen paydir sonsuza dek.
\par 27 Kuskusuz yok olacak senden uzak duranlar, Ortadan kaldiracaksin sana vefasizlik edenleri.
\par 28 Ama benim için en iyisi Tanri'ya yakin olmaktir; Bütün islerini duyurayim diye Siginak yaptim Egemen RAB'bi kendime.

\chapter{74}

\par 1 Ey Tanri, neden bizi sonsuza dek reddettin? Niçin otlaginin koyunlarina karsi öfken tütmekte?
\par 2 Animsa geçmiste sahiplendigin toplulugu, Kendi halkin olsun diye kurtardigin oymagi Ve üzerine konut kurdugun Siyon Dagi'ni.
\par 3 Yönelt adimlarini su onarilmaz yikintilara dogru, Düsman kutsal yerdeki her seyi yikti.
\par 4 Düsmanlarin bizimle bulustugun yerde kükredi, Zafer simgesi olarak kendi bayraklarini dikti.
\par 5 Gür bir ormana Baltayla dalar gibiydiler.
\par 6 Baltayla, balyozla kirdilar, Bütün oymalari.
\par 7 Atese verdiler tapinagini, Yerle bir edip kutsalligini bozdular Adinin yasadigi konutun.
\par 8 Içlerinden, "Hepsini ezelim!" dediler. Ülkede Tanri'yla bulusma yerlerinin tümünü yaktilar.
\par 9 Artik kutsal simgelerimizi görmüyoruz, Peygamberler de yok oldu, Içimizden kimse bilmiyor ne zamana dek...
\par 10 Ey Tanri, ne zamana dek düsman sana sövecek, Hasmin senin adini hor görecek?
\par 11 Niçin geri çekiyorsun elini? Çikar sag elini bagrindan, yok et onlari!
\par 12 Ama geçmisten bu yana kralim sensin, ey Tanri, Yeryüzünde kurtulus sagladin.
\par 13 Gücünle denizi yardin, Canavarlarin kafasini sularda parçaladin.
\par 14 Livyatan'in* baslarini ezdin, Çölde yasayanlara onu yem ettin.
\par 15 Kaynaklar, dereler fiskirttin, Sürekli akan irmaklari kuruttun.
\par 16 Gün senindir, gece de senin, Ay ve günesi sen yerlestirdin,
\par 17 Yeryüzünün bütün sinirlarini sen saptadin, Yazi da kisi da yaratan sensin.
\par 18 Animsa, ya RAB, düsmanin sana nasil sövdügünü, Akilsiz bir halkin, adini nasil hor gördügünü.
\par 19 Canavara teslim etme kumrunun canini, Asla unutma düskün kullarinin yasamini.
\par 20 Yaptigin antlasmayi gözönüne al, Çünkü ülkenin her karanlik kösesi Zorbalarin inleriyle dolmus.
\par 21 Düskünler boynu bükük geri çevrilmesin, Mazlumlar, yoksullar adina övgüler dizsin.
\par 22 Kalk, ey Tanri, davani savun! Animsa akilsizlarin gün boyu sana nasil sövdügünü!
\par 23 Unutma hasimlarinin yaygarasini, Sana baskaldiranlarin durmadan yükselen patirtisini!

\chapter{75}

\par 1 Sana sükrederiz, ey Tanri, ªükrederiz, çünkü sen yakinsin, Harikalarin bunu gösterir.
\par 2 "Belirledigim zaman gelince, Dogrulukla yargilayacagim" diyor Tanri,
\par 3 "Yeryüzü altüst olunca üzerindekilerle, Ben pekistirecegim onun direklerini. *
\par 4 Övünenlere, 'Övünmeyin artik! dedim; Kötülere, 'Kaldirmayin basinizi!
\par 5 Kaldirmayin basinizi! Tepeden konusmayin!"
\par 6 Çünkü ne dogudan, ne batidan, Ne de çöldeki daglardan dogar yargi.
\par 7 Yargiç ancak Tanri'dir, Birini alçaltir, birini yükseltir.
\par 8 RAB elinde dolu bir kâse* tutuyor, Köpüklü, baharat karistirilmis sarap döküyor; Yeryüzünün bütün kötüleri Tortusuna dek yalayip onu içiyor.
\par 9 Bense sürekli duyuracagim bunu, Yakup'un Tanrisi'ni ilahilerle övecegim:
\par 10 "Kiracagim kötülerin bütün gücünü, Dogrularin gücüyse yükseltilecek."

\chapter{76}

\par 1 Yahuda'da Tanri bilinir, Israil'de adi uludur;
\par 2 Konutu ªalem'dedir, Yasadigi yer Siyon'da.
\par 3 Orada kirdi alevli oklari, Kalkani, kilici, savas silahlarini. *
\par 4 Isil isil parildiyorsun, Avi bol daglardan daha görkemli.
\par 5 Yagmaya ugradi yigitler, Uykularina daldilar, En güçlüleri bile elini kipirdatamaz oldu.
\par 6 Ey Yakup'un Tanrisi, sen kükreyince, Atlarla atlilar son uykularina daldilar.
\par 7 Yalniz sensin korkulmasi gereken, Öfkelenince kim durabilir karsinda?
\par 8 Yargini göklerden açikladin, Yeryüzü korkup sessizlige büründü,
\par 9 Ey Tanri, sen yargilamaya, Ülkedeki mazlumlari kurtarmaya kalkinca.
\par 10 Insanlarin gazabi bile sana övgüler doguruyor, Gazabindan kurtulanlari çevrene topluyorsun.
\par 11 Adaklar adayin Tanri'niz RAB'be, Yerine getirin adaklarinizi, Armaganlar sunun korkulmasi gereken Tanri'ya, Bütün çevresindekiler.
\par 12 RAB önderlerin solugunu keser, Korku salar yeryüzü krallarina.

\chapter{77}

\par 1 Yüksek sesle Tanri'ya yakariyorum, Haykiriyorum beni duysun diye.
\par 2 Sikintili günümde Rab'be yönelir, Gece hiç durmadan ellerimi açarim, Gönlüm avunmaz bir türlü.
\par 3 Tanri'yi animsayinca inlerim, Düsündükçe içim daralir. *
\par 4 Açik tutuyorsun göz kapaklarimi, Sikintidan konusamiyorum.
\par 5 Geçmis günleri, Yillar öncesini düsünüyorum.
\par 6 Gece ilahilerimi anacagim, Kendi kendimle konusacagim, Inceden inceye soracagim:
\par 7 "Rab sonsuza dek mi bizi reddedecek? Lütfunu bir daha göstermeyecek mi?
\par 8 Sevgisi sonsuza dek mi yok oldu? Sözü geçerli degil mi artik?
\par 9 Tanri unuttu mu acimayi? Sevecenliginin yerini öfke mi aldi?"
\par 10 Sonra kendi kendime, "Iste benim derdim bu!" dedim, "Yüceler Yücesi gücünü göstermiyor artik."
\par 11 RAB'bin islerini anacagim, Evet, geçmisteki harikalarini anacagim.
\par 12 Yaptiklari üzerinde derin derin düsünecegim, Bütün islerinin üzerinde dikkatle duracagim.
\par 13 Ey Tanri, yolun kutsaldir! Hangi ilah Tanri kadar uludur?
\par 14 Harikalar yaratan Tanri sensin, Halklar arasinda gücünü gösterdin.
\par 15 Güçlü bileginle kendi halkini, Yakup ve Yusuf ogullarini kurtardin.
\par 16 Sular seni görünce, ey Tanri, Sular seni görünce çalkalandi, Enginler titredi.
\par 17 Bulutlar suyunu bosaltti, Gökler gürledi, Her yanda oklarin uçustu.
\par 18 Kasirgada gürleyisin duyuldu, ªimsekler dünyayi aydinlatti, Yer titreyip sarsildi.
\par 19 Kendine denizde, Derin sularda yollar açtin, Ama ayak izlerin belli degildi.
\par 20 Musa ve Harun'un eliyle Halkini bir sürü gibi güttün.

\chapter{78}

\par 1 Dinle, ey halkim, ögrettiklerimi, Kulak ver agzimdan çikan sözlere.
\par 2 Özdeyislerle söze baslayacagim, Eski sirlari anlatacagim,
\par 3 Duydugumuzu, bildigimizi, Atalarimizin bize anlattigini.
\par 4 Torunlarindan bunlari gizlemeyecegiz; RAB'bin övgüye deger islerini, Gücünü, yaptigi harikalari Gelecek kusaga duyuracagiz.
\par 5 RAB Yakup soyuna kosullar bildirdi, Israil'e yasa koydu. Bunlari çocuklarina ögretsinler diye Atalarimiza buyruk verdi.
\par 6 Öyle ki, gelecek kusak, yeni dogacak çocuklar bilsinler, Onlar da kendi çocuklarina anlatsinlar,
\par 7 Tanri'ya güven duysunlar, Tanri'nin yaptiklarini unutmasinlar, O'nun buyruklarini yerine getirsinler;
\par 8 Atalari gibi inatçi, baskaldirici, Yüregi kararsiz, Tanri'ya sadakatsiz bir kusak olmasinlar.
\par 9 Oklarla, yaylarla kusanmis Efrayimogullari Savas günü sirtlarini döndüler.
\par 10 Tanri'nin antlasmasina uymadilar, O'nun yasasina göre yasamayi reddettiler.
\par 11 Unuttular O'nun islerini, Kendilerine gösterdigi harikalari.
\par 12 Misir'da, Soan bölgesinde Tanri harikalar yapmisti atalarinin önünde.
\par 13 Denizi yarip geçirmisti onlari, Bir duvar gibi ayakta tutmustu sulari.
\par 14 Gündüz bulutla, Gece ates isigiyla onlara yol göstermisti.
\par 15 Çölde kayalari yarmis, Sanki dipsiz kaynaklardan Onlara kana kana su içirmisti.
\par 16 Kayadan akarsular fiskirtmis, Sulari irmak gibi akitmisti.
\par 17 Ama onlar çölde Yüceler Yücesi'ne baskaldirarak Günah islemeye devam ettiler.
\par 18 Canlarinin çektigi yiyecegi isteyerek Içlerinde Tanri'yi denediler.
\par 19 "Tanri çölde sofra kurabilir mi?" diyerek, Tanri'ya karsi konustular.
\par 20 "Bak, kayaya vurunca sular fiskirdi, Dereler tasti. Peki, ekmek de verebilir mi, Et saglayabilir mi halkina?"
\par 21 RAB bunu duyunca çok öfkelendi, Yakup'a ates püskürdü, Öfkesi tirmandi Israil'e karsi;
\par 22 Çünkü Tanri'ya inanmiyorlardi, O'nun kurtariciligina güvenmiyorlardi.
\par 23 Yine de RAB buyruk verdi bulutlara, Kapaklarini açti göklerin;
\par 24 Man* yagdirdi onlari beslemek için, Göksel tahil verdi onlara.
\par 25 Meleklerin ekmegini yedi her biri, Doyasiya yiyecek gönderdi onlara.
\par 26 Dogu rüzgarini estirdi göklerde, Gücüyle güney rüzgarina yol gösterdi.
\par 27 Toz gibi et yagdirdi baslarina, Deniz kumu kadar kus;
\par 28 Ordugahlarinin ortasina, Konakladiklari yerin çevresine düsürdü.
\par 29 Yediler, tika basa doydular, Isteklerini yerine getirdi Tanri.
\par 30 Ancak onlar isteklerine doymadan, Daha agizlari doluyken,
\par 31 Tanri'nin öfkesi parladi üzerlerine. En güçlülerini öldürdü, Yere serdi Israil yigitlerini.
\par 32 Yine de günah islemeye devam ettiler, O'nun harikalarina inanmadilar.
\par 33 Bu yüzden Tanri onlarin günlerini bosluk, Yillarini dehset içinde bitirdi.
\par 34 Tanri onlari öldürdükçe O'na yönelmeye, Istekle O'nu yeniden aramaya basliyorlardi.
\par 35 Tanri'nin kayalari oldugunu, Yüce Tanri'nin kurtaricilari oldugunu animsiyorlardi.
\par 36 Oysa agizlariyla O'na yaltaklaniyor, Dilleriyle yalan söylüyorlardi.
\par 37 O'na yürekten bagli degillerdi, Antlasmasina sadik kalmadilar.
\par 38 Yine de Tanri sevecendi, Suçlarini bagisliyor, onlari yok etmiyordu; Çok kez öfkesini tuttu, Bütün gazabini göstermedi.
\par 39 Onlarin yalnizca insan oldugunu animsadi, Geçip giden, dönmeyen bir rüzgar gibi.
\par 40 Çölde kaç kez O'na baskaldirdilar, Issiz yerlerde O'nu gücendirdiler!
\par 41 Defalarca denediler Tanri'yi, Incittiler Israil'in Kutsali'ni.
\par 42 Animsamadilar O'nun güçlü elini, Kendilerini düsmandan kurtardigi günü,
\par 43 Misir'da gösterdigi belirtileri, Soan bölgesinde yaptigi sasilasi isleri.
\par 44 Misir'in kanallarini kana çevirdi, Sularini içemediler.
\par 45 Gönderdigi at sinekleri yedi halki, Gönderdigi kurbagalar yok etti ülkeyi.
\par 46 Ekinlerini tirtillara, Emeklerinin ürününü çekirgelere verdi.
\par 47 Asmalarini doluyla, Yabanil incir agaçlarini iri dolu taneleriyle yok etti.
\par 48 Büyükbas hayvanlarini kirgina, Küçükbas hayvanlarini yildirima teslim etti.
\par 49 Üzerlerine kizgin öfkesini, Gazap, hisim, bela Ve bir alay kötülük melegi gönderdi.
\par 50 Yol verdi öfkesine, Canlarini ölümden esirgemedi, Onlari salgin hastaligin pençesine düsürdü.
\par 51 Misir'da bütün ilk doganlari, Ham'in çadirlarinda bütün ilk çocuklari vurdu.
\par 52 Kendi halkini davar gibi götürdü, Çölde onlari bir sürü gibi güttü.
\par 53 Onlara güvenlik içinde yol gösterdi, korkmadilar; Düsmanlariniysa deniz yuttu.
\par 54 Böylece onlari kendi kutsal topraklarinin sinirina, Sag elinin kazandigi daglik bölgeye getirdi.
\par 55 Önlerinden uluslari kovdu, Mülk olarak topraklarini Israil oymaklari arasinda bölüstürdü. Halkini konutlarina yerlestirdi.
\par 56 Ama onlar yüce Tanri'yi denediler, O'na baskaldirdilar, Kosullarina uymadilar.
\par 57 Döneklik edip atalari gibi ihanet ettiler, Güvenilmez bir yay gibi bozuk çiktilar.
\par 58 Puta taptiklari yerlerle O'nu kizdirdilar, Putlariyla O'nu kiskandirdilar.
\par 59 Tanri bunlari duyunca çok öfkelendi, Israil'i büsbütün reddetti.
\par 60 Insanlar arasinda kurdugu çadiri, ªilo'daki konutunu terk etti.
\par 61 Kudretini tutsakliga, Görkemini düsman eline teslim etti.
\par 62 Halkini kiliç önüne sürdü, Öfkesini kendi halkindan çikardi.
\par 63 Gençlerini ates yuttu, Kizlarina dügün türküsü söylenmez oldu.
\par 64 Kâhinleri* kiliç altinda öldü, Dul kadinlari aglayamadi.
\par 65 O zaman Rab uykudan uyanir gibi, ªarabin rehavetinden ayilan bir yigit gibi oldu.
\par 66 Düsmanlarini püskürttü, Onlari sonsuz utanca bogdu.
\par 67 Tanri Yusuf soyunu reddetti, Efrayim oymagini seçmedi;
\par 68 Ancak Yahuda oymagini, Sevdigi Siyon Dagi'ni seçti.
\par 69 Tapinagini doruklar gibi, Sonsuzluk için kurdugu yeryüzü gibi yapti.
\par 70 Kulu Davut'u seçti, Onu koyun agilindan aldi.
\par 71 Halki Yakup'u, kendi halki Israil'i gütmek için, Onu yavru kuzularin ardindan getirdi.
\par 72 Böylece Davut onlara dürüstçe çobanlik etti, Becerikli elleriyle onlara yol gösterdi.

\chapter{79}

\par 1 Ey Tanri, uluslar senin yurduna saldirdi, Kutsal tapinagini kirletti, Yerusalim'i tas yiginina çevirdi.
\par 2 Kullarinin ölülerini yem olarak yirtici kuslara, Sadik kullarinin etini yabanil hayvanlara verdiler.
\par 3 Kanlarini su gibi akittilar Yerusalim'in çevresine, Onlari gömecek kimse yok.
\par 4 Komsularimiza yüzkarasi, Çevremizdekilere eglence ve oyuncak olduk.
\par 5 Ne zamana dek, ya RAB? Sonsuza dek mi sürecek öfken, Alev gibi yanan kiskançligin?
\par 6 Öfkeni seni tanimayan uluslarin, Adini anmayan ülkelerin üzerine dök.
\par 7 Çünkü onlar Yakup soyunu yiyip bitirdiler, Yurdunu viraneye çevirdiler.
\par 8 Atalarimizin suçlarini artik önümüze sürme, Sevecenligini hemen göster bize, Çünkü tükendikçe tükendik.
\par 9 Yardim et bize yüce adin ugruna, ey bizi kurtaran Tanri, Kurtar bizi adin ugruna, bagisla günahlarimizi!
\par 10 Niçin uluslar, "Nerede onlarin Tanrisi?" diye konussun, Kullarinin dökülen kaninin öcünü alacagini bilsinler, Gözlerimizle bunu görelim!
\par 11 Tutsaklarin iniltisi senin katina erissin, Koru büyük gücünle ölüme mahkûm olanlari.
\par 12 Komsularimizin sana ettikleri hakareti Yedi kat iade et bagirlarina, ya Rab!
\par 13 Bizler, kendi halkin, otlaginin koyunlari Sonsuza dek sükredecegiz sana, Kusaklar boyunca övgülerini dilimizden düsürmeyecegiz.

\chapter{80}

\par 1 Kulak ver, ey Israil'in çobani, Ey Yusuf'u bir sürü gibi güden, Keruvlar* arasinda taht kuran, Saç isigini,
\par 2 Efrayim, Benyamin, Manasse önünde Uyandir gücünü, Gel, kurtar bizi!
\par 3 Bizi eski halimize kavustur, ey Tanri, Yüzünün isigiyla aydinlat, kurtulalim!
\par 4 Ya RAB, Her Seye Egemen Tanri, Ne zamana dek halkinin dualarina ates püsküreceksin?
\par 5 Onlara ekmek yerine gözyasi verdin, Ölçekler dolusu gözyasi içirdin.
\par 6 Kavga nedeni ettin bizi komsularimiza, Düsmanlarimiz alay ediyor bizimle.
\par 7 Bizi eski halimize kavustur, Ey Her Seye Egemen Tanri, Yüzünün isigiyla aydinlat, kurtulalim!
\par 8 Misir'dan bir asma çubugu getirdin, Uluslari kovup onu diktin.
\par 9 Onun için topragi hazirladin, Kök saldi, bütün ülkeye yayildi.
\par 10 Gölgesi daglari, Dallari koca sedir agaçlarini kapladi.
\par 11 Sürgünleri Akdeniz'e, Filizleri Firat'a dek uzandi.
\par 12 Niçin yiktin bagin duvarlarini? Yoldan geçen herkes üzümünü kopariyor,
\par 13 Orman domuzlari onu yoluyor, Yabanil hayvanlar onunla besleniyor.
\par 14 Ey Her Seye Egemen Tanri, ne olur, dön bize! Göklerden bak ve gör, Ilgilen bu asmayla.
\par 15 Ilgilen sag elinin diktigi filizle, Kendine seçtigin ogulla!
\par 16 Asman kesilmis, yakilmis, Öfkeli bakislarin yok etsin düsmanlarini!
\par 17 Elin, sag kolun olan adamin üzerinde, Kendine seçtigin insanin üzerinde olsun!
\par 18 O zaman senden asla ayrilmayacagiz; Yasam ver bize, adini analim!
\par 19 Ya RAB, ey Her Seye Egemen Tanri, Bizi eski halimize kavustur, Yüzünün isigiyla aydinlat, kurtulalim!

\chapter{81}

\par 1 Sevincinizi dile getirin gücümüz olan Tanri'ya, Sevinç çigliklari atin Yakup'un Tanrisi'na!
\par 2 Çalgiya baslayin, tef çalin, Tatli sesli lir ve çenk çinlatin.
\par 3 Yeni Ay'da, dolunayda, Boru çalin bayram günümüzde.
\par 4 Çünkü bu Israil için bir kuraldir, Yakup'un Tanrisi'nin ilkesidir.
\par 5 Tanri Misir'a karsi yürüdügünde, Yusuf soyuna koydu bu kosulu. Orada tanimadigim bir ses isittim:
\par 6 "Sirtindaki yükü kaldirdim, Ellerin küfeden kurtuldu" diyordu,
\par 7 "Sikintiya düsünce seslendin, seni kurtardim, Gök gürlemesinin ardindan sana yanit verdim, Meriva sularinda seni sinadim. *
\par 8 "Dinle, ey halkim, seni uyariyorum; Ey Israil, keske beni dinlesen!
\par 9 Aranizda yabanci ilah olmasin, Baska bir ilaha tapmayin!
\par 10 Seni Misir'dan çikaran Tanrin RAB benim. Agzini iyice aç, doldurayim!
\par 11 "Ama halkim sesimi dinlemedi, Israil bana boyun egmek istemedi.
\par 12 Ben de onlari inatçi yürekleriyle bas basa biraktim, Bildikleri gibi yasasinlar diye.
\par 13 Keske halkim beni dinleseydi, Israil yollarimda yürüseydi!
\par 14 Düsmanlarini hemen yere serer, Hasimlarina el kaldirirdim!
\par 15 Benden nefret edenler bana boyun egerdi, Bu böyle sonsuza dek sürerdi.
\par 16 Oysa sizleri en iyi bugdayla besler, Kayadan akan balla doyururdum."

\chapter{82}

\par 1 Tanri yerini aldi tanrisal kurulda, Yargisini açikliyor ilahlarin ortasinda:
\par 2 "Ne zamana dek haksiz karar verecek, Kötüleri kayiracaksiniz? *
\par 3 Zayifin, öksüzün davasini savunun, Mazlumun, yoksulun hakkini arayin.
\par 4 Zayifi, düskünü kurtarin, Onlari kötülerin elinden özgür kilin."
\par 5 Bilmiyor, anlamiyorlar, Karanlikta dolasiyorlar. Yeryüzünün temelleri sarsiliyor.
\par 6 "'Siz ilahlarsiniz diyorum, 'Yüceler Yücesi'nin ogullarisiniz hepiniz!
\par 7 Yine de insanlar gibi öleceksiniz, Siradan bir önder gibi düseceksiniz!"
\par 8 Kalk, ey Tanri, yargila yeryüzünü! Çünkü bütün uluslar senindir.

\chapter{83}

\par 1 Ey Tanri, susma, Sessiz, hareketsiz kalma!
\par 2 Bak, düsmanlarin kargasa çikariyor, Senden nefret edenler boy gösteriyor.
\par 3 Halkina karsi kurnazlik pesindeler, Korudugun insanlara dolap çeviriyorlar.
\par 4 "Gelin, bu ulusun kökünü kaziyalim" diyorlar, "Israil'in adi bir daha anilmasin!"
\par 5 Hepsi sözbirligi etmis, düzen kuruyor, Sana karsi anlasmaya vardi:
\par 6 Edomlular, Ismaililer, Moavlilar, Hacerliler,
\par 7 Geval, Ammon, Amalek, Filist ve Sur halki.
\par 8 Asur da onlara katildi, Lutogullari'na güç verdiler. *
\par 9 Onlara Midyan'a, Kison Vadisi'nde Sisera'ya ve Yavin'e yaptigini yap:
\par 10 Onlar Eyn-Dor'da yok oldular, Toprak için gübreye döndüler.
\par 11 Onlarin soylularina Orev ve Zeev'e yaptigini, Beylerine Zevah ve Salmunna'ya yaptigini yap.
\par 12 Onlar: "Gelin, sahiplenelim Tanri'nin otlaklarini" demislerdi.
\par 13 Ey Tanrim, savrulan toza, Rüzgarin sürükledigi saman çöpüne çevir onlari!
\par 14 Orman yangini gibi, Daglari tutusturan alev gibi,
\par 15 Firtinanla kovala, Kasirganla dehsete düsür onlari!
\par 16 Utançla kapla yüzlerini, Sana yönelsinler, ya RAB.
\par 17 Sonsuza dek utanç ve dehset içinde kalsinlar, Rezil olup yok olsunlar.
\par 18 Senin adin RAB'dir, Anlasinlar yalniz senin yeryüzüne egemen en yüce Tanri oldugunu.

\chapter{84}

\par 1 Ey Her Seye Egemen RAB, Ne kadar severim konutunu!
\par 2 Canim senin avlularini özlüyor, Içim çekiyor, Yüregim, bütün varligim Sana, yasayan Tanri'ya sevinçle haykiriyor.
\par 3 Kuslar bile bir yuva, Kirlangiç, yavrularini koyacak bir yer buldu Senin sunaklarinin yaninda, Ey Her Seye Egemen RAB, Kralim ve Tanrim!
\par 4 Ne mutlu senin evinde oturanlara, Seni sürekli överler! *
\par 5 Ne mutlu gücünü senden alan insana! Akli hep Siyon'u ziyaret etmekte.
\par 6 Baka Vadisi'nden geçerken, Pinar basina çevirirler orayi, Ilk yagmurlar orayi berekete bogar.
\par 7 Gittikçe güçlenir, Siyon'da Tanri'nin huzuruna çikarlar.
\par 8 Ya RAB, Her Seye Egemen Tanri, duami dinle, Kulak ver, ey Yakup'un Tanrisi!
\par 9 Ey Tanri, kalkanimiza bak, Meshettigin* krala lütfet!
\par 10 Senin avlularinda bir gün, Baska yerdeki bin günden iyidir; Kötülerin çadirinda yasamaktansa, Tanrim'in evinin esiginde durmayi yeglerim.
\par 11 Çünkü RAB Tanri bir günes, bir kalkandir. Lütuf ve yücelik saglar; Dürüstçe yasayanlardan hiçbir iyiligi esirgemez.
\par 12 Ey Her Seye Egemen RAB, Ne mutlu sana güvenen insana!

\chapter{85}

\par 1 Ya RAB, ülkenden hosnut kaldin, Yakup soyunu eski gönencine kavusturdun.
\par 2 Halkinin suçlarini bagisladin, Bütün günahlarini yok saydin. *
\par 3 Bütün gazabini bir yana koydun, Kizgin öfkenden vazgeçtin.
\par 4 Ey bizi kurtaran Tanri, bizi eski halimize getir, Bize karsi öfkeni dindir!
\par 5 Sonsuza dek mi öfkeleneceksin bize? Kusaktan kusaga mi sürdüreceksin öfkeni?
\par 6 Halkin sende sevinç bulsun diye Bize yeniden yasam vermeyecek misin?
\par 7 Ya RAB, sevgini göster bize, Kurtarisini bagisla!
\par 8 Kulak verecegim RAB Tanri'nin ne diyecegine; Halkina, sadik kullarina esenlik sözü verecek, Yeter ki, bir daha akilsizlik etmesinler.
\par 9 Evet, O kendisinden korkanlari kurtarmak üzeredir, Görkemi ülkemizde yasasin diye.
\par 10 Sevgiyle sadakat bulusacak, Dogrulukla esenlik öpüsecek.
\par 11 Sadakat yerden bitecek, Dogruluk gökten bakacak.
\par 12 Ve RAB iyi olan neyse, onu verecek, Topragimizdan ürün fiskiracak.
\par 13 Dogruluk önüsira yürüyecek, Adimlari için yol yapacak.

\chapter{86}

\par 1 Kulak ver, ya RAB, yanitla beni, Çünkü mazlum ve yoksulum.
\par 2 Koru canimi, çünkü senin sadik kulunum. Ey Tanrim, kurtar sana güvenen kulunu!
\par 3 Aci bana, ya Rab, Çünkü gün boyu sana yakariyorum.
\par 4 Sevindir kulunu, ya Rab, Çünkü dualarimi sana yükseltiyorum.
\par 5 Sen iyi ve bagislayicisin, ya Rab, Sana yakaran herkese bol sevgi gösterirsin.
\par 6 Kulak ver duama, ya RAB, Yalvarislarimi dikkate al!
\par 7 Sikintili günümde sana yakaririm, Çünkü yanitlarsin beni.
\par 8 Ilahlar arasinda senin gibisi yok, ya Rab, Essizdir islerin.
\par 9 Yarattigin bütün uluslar gelip Sana tapinacaklar, ya Rab, Adini yüceltecekler.
\par 10 Çünkü sen ulusun, harikalar yaratirsin, Tek Tanri sensin.
\par 11 Ya RAB, yolunu bana ögret, Senin gerçegine göre yürüyeyim, Kararli kil beni, yalniz senin adindan korkayim.
\par 12 Ya Rab Tanrim, bütün yüregimle sana sükredecegim, Adini sonsuza dek yüceltecegim.
\par 13 Çünkü bana sevgin büyüktür, Canimi ölüler diyarinin derinliklerinden sen kurtardin.
\par 14 Ey Tanri, küstahlar bana saldiriyor, Zorbalar sürüsü, sana aldirmayanlar Canimi almak istiyor,
\par 15 Oysa sen, ya Rab, Sevecen, lütfeden, tez öfkelenmeyen, Sevgisi ve sadakati bol bir Tanri'sin.
\par 16 Yönel bana, aci halime, Kuluna kendi gücünü ver, Kurtar hizmetçinin oglunu.
\par 17 Iyiliginin bir belirtisini göster bana; Benden nefret edenler görüp utansin; Çünkü sen, ya RAB, bana yardim ettin, Beni avuttun.

\chapter{87}

\par 1 RAB Siyon'u kutsal daglar üzerine kurdu.
\par 2 Siyon'un kapilarini Yakup soyunun bütün konutlarindan daha çok sever.
\par 3 Ey Tanri kenti, senin için ne yüce sözler söylenir: *
\par 4 "Beni taniyanlar arasinda Rahav*fl* ve Babil'i anacagim, Filist'i, Sur'u, Kûs'u* da; 'Bu da Siyon'da dogdu diyecegim."
\par 5 Evet, Siyon için söyle denecek: "Su da orada dogmus, bu da, Yüceler Yücesi onu sarsilmaz kilacak."
\par 6 RAB halklari kaydederken, "Bu da Siyon'da dogmus" diye yazacak.
\par 7 Okuyucular, kavalcilar, "Bütün kaynaklarim sendedir!" diyecek.

\chapter{88}

\par 1 Ya RAB, beni kurtaran Tanri, Gece gündüz sana yakariyorum.
\par 2 Duam sana erissin, Kulak ver yakarisima.
\par 3 Çünkü sikintiya doydum, Canim ölüler diyarina yaklasti.
\par 4 Ölüm çukuruna inenler arasinda sayiliyorum, Tükenmis gibiyim;
\par 5 Ölüler arasina atilmis, Artik animsamadigin, Ilginden yoksun, Mezarda yatan cesetler gibiyim.
\par 6 Beni çukurun dibine, Karanliklara, derinliklere attin.
\par 7 Öfken üzerime çöktü, Dalga dalga kizginliginla beni ezdin. *
\par 8 Yakinlarimi benden uzaklastirdin, Igrenç kildin beni gözlerinde. Kapali kaldim, çikamiyorum.
\par 9 Üzüntüden gözlerimin feri sönüyor, Her gün sana yakariyorum, ya RAB, Ellerimi sana açiyorum.
\par 10 Harikalarini ölülere mi göstereceksin? Ölüler mi kalkip seni övecek?
\par 11 Sevgin mezarda, Sadakatin yikim diyarinda duyurulur mu?
\par 12 Karanliklarda harikalarin, Unutulmusluk diyarinda dogrulugun bilinir mi?
\par 13 Ama ben, ya RAB, yardima çagiriyorum seni, Sabah duam sana variyor.
\par 14 Niçin beni reddediyorsun, ya RAB, Neden yüzünü benden gizliyorsun?
\par 15 Düskünüm, gençligimden beri ölümle burun burunayim, Dehsetlerinin altinda tükendim.
\par 16 Siddetli gazabin üzerimden geçti, Saçtigin dehset beni yedi bitirdi.
\par 17 Bütün gün su gibi kusattilar beni, Çevremi tümüyle sardilar.
\par 18 Esi dostu benden uzaklastirdin, Tek dostum karanlik kaldi.

\chapter{89}

\par 1 RAB'bin sevgisini sonsuza dek ezgilerle övecegim, Sadakatini bütün kusaklara bildirecegim.
\par 2 Sevgin sonsuza dek ayakta kalir diyecegim, Sadakatini gökler kadar kalici kildin.
\par 3 Dedin ki, "Seçtigim adamla antlasma yaptim, Kulum Davut'a söyle ant içtim:
\par 4 'Soyunu sonsuza dek sürdürecegim, Tahtini kusaklar boyunca sürekli kilacagim." *
\par 5 Ya RAB, gökler över harikalarini, Kutsallar toplulugunda övülür sadakatin.
\par 6 Çünkü göklerde RAB'be kim es kosulur? Kim benzer RAB'be ilahi varliklar arasinda?
\par 7 Kutsallar toplulugunda Tanri korku uyandirir, Çevresindekilerin hepsinden ulu ve müthistir.
\par 8 Ya RAB, Her Seye Egemen Tanri, Senin gibi güçlü RAB var mi? Sadakatin çevreni sarar.
\par 9 Sen kudurmus denizler üzerinde egemenlik sürer, Dalgalar kabardikça onlari dindirirsin.
\par 10 Sen Rahav'i les ezer gibi ezdin, Güçlü kolunla düsmanlarini dagittin.
\par 11 Gökler senindir, yeryüzü de senin; Dünyanin ve içindeki her seyin temelini sen attin.
\par 12 Kuzeyi, güneyi sen yarattin, Tavor ve Hermon daglari Sana sevincini dile getiriyor.
\par 13 Kolun güçlüdür, Elin kudretli, sag elin yüce.
\par 14 Tahtin adalet ve dogruluk üzerine kurulu, Sevgi ve sadakat önünsira gider.
\par 15 Ne mutlu sevinç çigliklari atmasini bilen halka, ya RAB! Yüzünün isiginda yürürler.
\par 16 Gün boyu senin adinla sevinir, Dogrulugunla yücelirler.
\par 17 Çünkü sen onlarin gücü ve yüceligisin, Lütfun sayesinde gücümüz artar.
\par 18 Kalkanimiz RAB'be, Kralimiz Israil'in Kutsali'na aittir.
\par 19 Geçmiste bir görüm araciligiyla, Sadik kullarina söyle dedin: "Bir yigide yardim ettim, Halkin içinden bir genci yükselttim.
\par 20 Kulum Davut'u buldum, Kutsal yagimla onu meshettim*.
\par 21 Elim ona destek olacak, Kolum güç verecek.
\par 22 Düsman onu haraca baglayamayacak, Kötüler onu ezmeyecek.
\par 23 Düsmanlarini onun önünde kiracagim, Ondan nefret edenleri vuracagim.
\par 24 Sadakatim, sevgim ona destek olacak, Benim adimla gücü yükselecek.
\par 25 Sag elini denizin, Irmaklarin üzerine egemen kilacagim.
\par 26 'Babam sensin diye seslenecek bana, 'Tanrim, kurtulusumun kayasi.
\par 27 Ben de onu ilk oglum, Dünyadaki krallarin en yücesi kilacagim.
\par 28 Sonsuza dek ona sevgi gösterecegim, Onunla yaptigim antlasma hiç bozulmayacak.
\par 29 Soyunu sonsuza dek, Tahtini gökler durdugu sürece sürdürecegim.
\par 30 "Çocuklari yasamdan ayrilir, Ilkelerime göre yasamazsa;
\par 31 Kurallarimi bozar, Buyruklarima uymazsa,
\par 32 Isyanlarini sopayla, Suçlarini dayakla cezalandiracagim.
\par 33 Ama onu sevmekten vazgeçmeyecek, Sadakatime sirt çevirmeyecegim.
\par 34 Antlasmami bozmayacak, Agzimdan çikan sözü degistirmeyecegim.
\par 35 Bir kez kutsalligim üstüne ant içtim, Davut'a yalan söylemeyecegim.
\par 36 Onun soyu sonsuza dek sürecek, Tahti karsimda günes gibi duracak,
\par 37 Göklerde güvenilir bir tanik olan ay gibi Sonsuza dek kalacak."
\par 38 Ama sen reddettin, sirt çevirdin, Çok öfkelendin meshettigin* krala.
\par 39 Kulunla yaptigin antlasmadan vazgeçtin, Onun tacini yere atip kirlettin.
\par 40 Yiktin bütün surlarini, Viran ettin kalelerini.
\par 41 Yoldan geçen herkes onu yagmaladi, Yüzkarasi oldu komsularina.
\par 42 Hasimlarinin sag elini onun üstüne kaldirdin, Bütün düsmanlarini sevindirdin.
\par 43 Kilicinin agzini baska yöne çevirdin, Savasta ona yan çikmadin.
\par 44 Görkemine son verdin, Tahtini yere çaldin.
\par 45 Gençlik günlerini kisalttin, Onu utanca bogdun.
\par 46 Ne zamana dek, ya RAB? Sonsuza dek mi gizleneceksin? Ne zamana dek öfken alev alev yanacak?
\par 47 Animsa ömrümün ne çabuk geçtigini, Ne bos yaratmissin insanoglunu!
\par 48 Var mi yasayip da ölümü görmeyen, Ölüler diyarinin pençesinden canini kurtaran?
\par 49 Ya Rab, nerede o eski sevgin? Davut'a gösterecegine ant içtigin o sadik sevgin!
\par 50 Animsa, ya Rab, kullarinin nasil rezil oldugunu, Bütün halklarin hakaretini bagrimda nasil tasidigimi, Düsmanlarinin hakaretini, ya RAB, Meshettigin kralin attigi adima edilen hakaretleri.
\par 52 Sonsuza dek övgüler olsun RAB'be! Amin! Amin!

\chapter{90}

\par 1 Ya Rab, barinak oldun bize Kusaklar boyunca.
\par 2 Daglar var olmadan, Daha evreni ve dünyayi yaratmadan, Öncesizlikten sonsuzluga dek Tanri sensin.
\par 3 Insani topraga döndürürsün, "Ey insanogullari, topraga dönün!" diyerek.
\par 4 Çünkü senin gözünde bin yil Geçmis bir gün, dün gibi, Bir gece nöbeti gibidir.
\par 5 Insanlari bir düs gibi siler, süpürürsün, Sabah biten ot misali:
\par 6 Sabah filizlenir, büyür, Aksam solar, kurur.
\par 7 Eriyip bitiyoruz senin öfkenden, Kizginligindan dehsete düsüyoruz.
\par 8 Suçlarimizi önüne, Gizli günahlarimizi yüzünün isigina çikardin.
\par 9 Gazabindan kisaliyor günlerimiz, Bir soluk gibi tükeniyor yillarimiz.
\par 10 Ömrümüz yetmis yil sürüyor, Bilemedin seksen, o da saglikliysak; En güzel yillar da zahmetle, kederle geçiyor, Çabucak bitiyor, uçup gidiyoruz.
\par 11 Kim bilir gazabinin gücünü? Çünkü öfken sana duyulan korku kadar güçlüdür.
\par 12 Bu yüzden günlerimizi saymayi bize ögret ki, Bilgelik kazanalim.
\par 13 Vazgeç, ya RAB! Öfken ne zamana dek sürecek? Aci kullarina!
\par 14 Sabah bizi sevginle doyur, Ömrümüz boyunca sevinçle haykiralim.
\par 15 Kaç gün bizi sikintiya soktunsa, Kaç yil çile çektirdinse, O kadar sevindir bizi.
\par 16 Yaptiklarin kullarina, Görkemin onlarin çocuklarina görünsün.
\par 17 Tanrimiz Rab bizden hosnut kalsin. Ellerimizin emegini bosa çikarma. Evet, ellerimizin emegini bosa çikarma.

\chapter{91}

\par 1 Yüceler Yücesi'nin barinaginda oturan, Her Seye Gücü Yeten'in gölgesinde barinir.
\par 2 "O benim siginagim, kalemdir" derim RAB için, "Tanrim'dir, O'na güvenirim."
\par 3 Çünkü O seni avci tuzagindan, Ölümcül hastaliktan kurtarir.
\par 4 Seni kanatlarinin altina alir, Onlarin altina siginirsin. O'nun sadakati senin kalkanin, siperin olur.
\par 5 Ne gecenin dehsetinden korkarsin, Ne gündüz uçan oktan, Ne karanlikta dolasan hastaliktan, Ne de ögleyin yok eden kirgindan.
\par 7 Yaninda bin kisi, Saginda on bin kisi kirilsa bile, Sana dokunmaz.
\par 8 Sen yalniz kendi gözlerinle seyredecek, Kötülerin cezasini göreceksin.
\par 9 Sen RAB'bi kendine siginak, Yüceler Yücesi'ni konut edindigin için,
\par 10 Basina kötülük gelmeyecek, Çadirina felaket yaklasmayacak.
\par 11 Çünkü Tanri meleklerine buyruk verecek, Gidecegin her yerde seni korusunlar diye.
\par 12 Elleri üzerinde tasiyacaklar seni, Ayagin bir tasa çarpmasin diye.
\par 13 Aslanin, kobranin üzerine basip geçeceksin, Genç aslani, yilani çigneyeceksin.
\par 14 "Beni sevdigi için Onu kurtaracagim" diyor RAB, "Beni iyi tanidigi için Ona kale olacagim.
\par 15 Bana seslenince onu yanitlayacagim, Sikintida onun yaninda olacagim, Kurtarip yüceltecegim onu.
\par 16 Onu uzun ömürle doyuracak, Ona kurtarisimi gösterecegim."

\chapter{92}

\par 1 Ya RAB, sana sükretmek, Ey Yüceler Yücesi, adini ilahilerle övmek, Sabah sevgini, Gece sadakatini, On telli sazla, çenk ve lirle duyurmak ne güzel!
\par 4 Çünkü yaptiklarinla beni sevindirdin, ya RAB, Ellerinin isi karsisinda sevinç ilahileri okuyorum.
\par 5 Yaptiklarin ne büyüktür, ya RAB, Düsüncelerin ne derin!
\par 6 Aptal insan bilemez, Budala akil erdiremez:
\par 7 Kötüler mantar gibi bitse, Suçlular pitrak gibi açsa bile, Bu onlarin sonsuza dek yok olusu demektir.
\par 8 Ama sen sonsuza dek yücesin, ya RAB.
\par 9 Ya RAB, düsmanlarin kesinlikle, Evet, kesinlikle yok olacak, Suç isleyen herkes dagilacak.
\par 10 Beni yaban öküzü kadar güçlü kildin, Taze zeytinyagini basima döktün.
\par 11 Gözlerim düsmanlarimin bozgununu gördü, Kulaklarim bana saldiran kötülerin sonunu duydu.
\par 12 Dogru insan hurma agaci gibi serpilecek, Lübnan sediri gibi yükselecek.
\par 13 RAB'bin evinde dikilmis, Tanrimiz'in avlularinda serpilecek.
\par 14 Böyleleri yaslaninca da meyve verecek, Taptaze ve yesil kalacaklar.
\par 15 "RAB dogrudur! Kayamdir benim! O'nda haksizlik bulunmaz!" diye duyuracaklar.

\chapter{93}

\par 1 RAB egemenlik sürüyor, görkeme bürünmüs, Kudret giyinip kusanmis. Dünya saglam kurulmus, sarsilmaz.
\par 2 Ya RAB, tahtin öteden beri kurulmus, Varligin öncesizlige uzanir.
\par 3 Denizler gürlüyor, ya RAB, Denizler gümbür gümbür gürlüyor, Denizler dalgalarini çinlatiyor.
\par 4 Yücelerdeki RAB engin sularin gürleyisinden, Denizlerin azgin dalgalarindan Daha güçlüdür.
\par 5 Kosullarin hep geçerlidir; Tapinagina kutsallik yarasir Sonsuza dek, ya RAB.

\chapter{94}

\par 1 Ya RAB, öç alici Tanri, Saç isigini, ey öç alici Tanri!
\par 2 Kalk, ey yeryüzünün yargici, Küstahlara hak ettikleri cezayi ver!
\par 3 Kötüler ne zamana dek, ya RAB, Ne zamana dek sevinip cosacak?
\par 4 Agizlarindan küstahlik dökülüyor, Suç isleyen herkes övünüyor.
\par 5 Halkini eziyorlar, ya RAB, Kendi halkina eziyet ediyorlar.
\par 6 Dulu, garibi bogazliyor, Öksüzleri öldürüyorlar.
\par 7 "RAB görmez" diyorlar, "Yakup'un Tanrisi dikkat etmez."
\par 8 Ey halkin içindeki budalalar, dikkat edin; Ey aptallar, ne zaman akillanacaksiniz?
\par 9 Kulagi yaratan isitmez mi? Göze biçim veren görmez mi?
\par 10 Uluslari yola getiren yargilamaz mi? Insani egiten bilmez mi?
\par 11 RAB insanin düsüncelerinin Bos oldugunu bilir.
\par 12 Ne mutlu, ya RAB, yola getirdigin, Yasani ögrettigin insana!
\par 13 Kötüler için çukur kazilincaya dek, Onu sikintili günlerden kurtarip rahatlatirsin.
\par 14 Çünkü RAB halkini reddetmez, Kendi halkini terk etmez.
\par 15 Adalet yine dogruluk üzerine kurulacak, Yüregi temiz olan herkes ona uyacak.
\par 16 Kötülere karsi beni kim savunacak? Kim benim için suçlulara karsi duracak?
\par 17 RAB yardimcim olmasaydi, Simdiye dek sessizlik diyarina göçmüstüm bile.
\par 18 "Ayagim kayiyor" dedigimde, Sevgin ayakta tutar beni, ya RAB.
\par 19 Kaygilar içimi sarinca, Senin avutmalarin gönlümü sevindirir.
\par 20 Yasaya dayanarak haksizlik yapan koltuk sahibi Seninle bagdasir mi?
\par 21 Onlar dogruya karsi birlesiyor, Suçsuzu ölüme mahkûm ediyorlar.
\par 22 Ama RAB bana kale oldu, Tanrim sigindigim kaya oldu.
\par 23 Tanrimiz RAB yaptiklari kötülügü Kendi baslarina getirecek, Kötülükleri yüzünden köklerini kurutacak, Evet, köklerini kurutacak.

\chapter{95}

\par 1 Gelin, RAB'be sevinçle haykiralim, Bizi kurtaran kayaya sevinç çigliklari atalim,
\par 2 Sükranla huzuruna çikalim, O'na sevinç ilahileri yükseltelim!
\par 3 Çünkü RAB ulu Tanri'dir, Bütün ilahlarin üstünde ulu kraldir.
\par 4 Yerin derinlikleri O'nun elindedir, Daglarin doruklari da O'nun.
\par 5 Deniz O'nundur, çünkü O yaratti, Karaya da O'nun elleri biçim verdi.
\par 6 Gelin, tapinalim, egilelim, Bizi yaratan RAB'bin önünde diz çökelim.
\par 7 Çünkü O Tanrimiz'dir, Bizse O'nun otlaginin halki, Elinin altindaki koyunlariz. Bugün sesini duyarsaniz,
\par 8 Meriva'da, o gün çölde, Massa'da oldugu gibi, Yüreklerinizi nasirlastirmayin.
\par 9 Yaptiklarimi görmelerine karsin, Atalariniz orada beni sinayip denediler.
\par 10 Kirk yil o kusaktan hep igrendim, "Yüregi kötü yola sapan bir halktir" dedim, "Yollarimi bilmiyorlar."
\par 11 Bu yüzden öfkeyle ant içtim: "Huzur diyarima asla girmeyecekler!"

\chapter{96}

\par 1 Yeni bir ezgi söyleyin RAB'be! Ey bütün dünya, RAB'be ezgiler söyleyin!
\par 2 Ezgi söyleyin, RAB'bin adini övün, Her gün duyurun kurtarisini!
\par 3 Görkemini uluslara, Harikalarini bütün halklara anlatin!
\par 4 Çünkü RAB uludur, yalniz O övgüye deger, Ilahlardan çok O'ndan korkulur.
\par 5 Halklarin bütün ilahlari bir hiçtir, Oysa gökleri yaratan RAB'dir.
\par 6 Yücelik, ululuk O'nun huzurundadir, Güç ve güzellik O'nun tapinagindadir.
\par 7 Ey bütün halklar, RAB'bi övün, RAB'bin gücünü, yüceligini övün,
\par 8 RAB'bin görkemini adina yarasir biçimde övün, Sunular getirip avlularina girin!
\par 9 Kutsal giysiler içinde RAB'be tapinin! Titreyin O'nun önünde, ey bütün yeryüzündekiler!
\par 10 Uluslara, "RAB egemenlik sürüyor" deyin. Dünya saglam kurulmus, sarsilmaz. O halklari adaletle yargilar.
\par 11 Sevinsin gökler, cossun yeryüzü! Gürlesin deniz içindekilerle birlikte!
\par 12 Bayram etsin kirlar ve üzerindekiler! O zaman RAB'bin önünde bütün orman agaçlari Sevinçle haykiracak. Çünkü O geliyor! Yeryüzünü yargilamaya geliyor. Dünyayi adaletle, Halklari kendi gerçegiyle yönetecek.

\chapter{97}

\par 1 RAB egemenlik sürüyor, cossun yeryüzü, Bütün kiyi halklari sevinsin!
\par 2 Bulut ve zifiri karanlik sarmis çevresini, Dogruluk ve adalettir tahtinin temeli.
\par 3 Ates yürüyor O'nun önünde, Düsmanlarini yakiyor çevrede.
\par 4 Simsekleri dünyayi aydinlatir, Yeryüzü görüp titrer.
\par 5 Daglar balmumu gibi erir, RAB'bin, bütün yeryüzünün Rab'bi önünde.
\par 6 Gökler O'nun dogrulugunu duyurur, Bütün halklar görkemini görür.
\par 7 Utansin puta tapanlar, Degersiz putlarla övünenler! RAB'be tapin, ey bütün ilahlar!
\par 8 Siyon seviniyor yargilarini duyunca, ya RAB, Yahuda kentleri cosuyor.
\par 9 Çünkü sensin, ya RAB, bütün yeryüzünün en yücesi, Bütün ilahlarin üstündesin, çok ulusun.
\par 10 Ey sizler, RAB'bi sevenler, kötülükten tiksinin. O sadik kullarinin canini korur, Onlari kötülerin elinden kurtarir.
\par 11 Dogrulara isik, Temiz yüreklilere sevinç saçar.
\par 12 Ey dogrular, RAB'de sevinç bulun, Kutsalligini anarak O'na sükredin!

\chapter{98}

\par 1 Yeni bir ezgi söyleyin RAB'be. Çünkü harikalar yapti, Zaferler kazandi sag eli ve kutsal koluyla.
\par 2 RAB uluslarin gözü önüne serdi kurtarisini, Zaferini bildirdi.
\par 3 Israil halkina sevgisini, Sadakatini animsadi; Tanrimiz'in zaferini gördü dünyanin dört bucagi.
\par 4 Sevinç çigliklari yükseltin RAB'be, ey yeryüzündekiler! Sevinç ilahileriyle yeri gögü çinlatin!
\par 5 Lirle ezgiler sunun RAB'be, Lir ve müzik esliginde!
\par 6 Boru ve borazan esliginde Sevinç çigliklari atin Kral olan RAB'bin önünde.
\par 7 Gürlesin deniz ve içindekiler, Gürlesin yeryüzü ve üzerindekiler.
\par 8 El çirpsin irmaklar, Sevinçle haykirsin daglar RAB'bin önünde! Çünkü O geliyor Yeryüzünü yönetmeye. Dünyayi adaletle, Halklari dogrulukla yönetecek.

\chapter{99}

\par 1 RAB egemenlik sürüyor, titresin halklar! Keruvlar* arasinda tahtina oturmus, Sarsilsin yeryüzü!
\par 2 RAB Siyon'da uludur, Yücedir O, bütün halklara egemendir.
\par 3 Övsünler büyük, müthis adini! O kutsaldir.
\par 4 Ey adaleti seven güçlü kral, Esitligi sen sagladin, Yakup soyunda dogru ve adil olani sen yaptin.
\par 5 Yüceltin Tanrimiz RAB'bi, Ayaklarinin taburesi önünde tapinin! O kutsaldir.
\par 6 Musa'yla Harun O'nun kâhinlerindendi, Samuel de O'na yakaranlar arasinda. RAB'be seslenirlerdi, O da yanitlardi.
\par 7 Bulut sütunu içinden onlarla konustu, Uydular O'nun buyruklarina, Kendilerine verdigi kurallara.
\par 8 Ya RAB Tanrimiz, yanit verdin onlara; Bagislayici bir Tanri oldun, Ama yaptiklari kötülügü cezasiz birakmadin.
\par 9 Tanrimiz RAB'bi yüceltin, Tapinin O'na kutsal daginda! Çünkü Tanrimiz RAB kutsaldir.

\chapter{100}

\par 1 Ey bütün dünya, RAB'be sevinç çigliklari yükseltin!
\par 2 O'na neseyle kulluk edin, Sevinç ezgileriyle çikin huzuruna!
\par 3 Bilin ki RAB Tanri'dir. Bizi yaratan O'dur, biz de O'nunuz, O'nun halki, otlaginin koyunlariyiz.
\par 4 Kapilarina sükranla, Avlularina övgüyle girin! Sükredin O'na, adina övgüler sunun!
\par 5 Çünkü RAB iyidir, Sevgisi sonsuzdur. Sadakati kusaklar boyunca sürer.

\chapter{101}

\par 1 Sevgini ve adaletini ezgilerle anacagim, Seni ilahilerle övecegim, ya RAB.
\par 2 Dürüst davranmaya özen gösterecegim, Ne zaman geleceksin bana? Temiz bir yasam sürecegim evimde,
\par 3 Önümde alçakliga izin vermeyecegim. Tiksinirim döneklerin isinden, Etkilemez beni.
\par 4 Uzak olsun benden sapiklik, Tanimak istemem kötülügü.
\par 5 Yok ederim dostunu gizlice çekistireni, Katlanamam tepeden bakan, gururlu insana.
\par 6 Gözüm ülkenin sadik insanlari üzerinde olacak, Yanimda oturmalarini isterim; Bana dürüst yasayan kisi hizmet edecek.
\par 7 Dolap çeviren evimde oturmayacak, Yalan söyleyen gözümün önünde durmayacak.
\par 8 Her sabah ülkedeki kötüleri yok ederek Bütün haksizlari RAB'bin kentinden söküp atacagim.

\chapter{102}

\par 1 Ya RAB, duami isit, Yakarisim sana erissin.
\par 2 Sikintili günümde yüzünü benden gizleme, Kulak ver sesime, Seslenince yanit ver bana hemen.
\par 3 Çünkü günlerim duman gibi yok oluyor, Kemiklerim ates gibi yaniyor.
\par 4 Yüregim kirgin yemis ot gibi kurudu, Ekmek yemeyi bile unuttum.
\par 5 Bir deri bir kemige döndüm Aci aci inlemekten.
\par 6 Issiz yerlerdeki ishakkusunu andiriyorum, Viranelerdeki kukumav gibiyim.
\par 7 Gözüme uyku girmiyor, Damda yalniz kalmis bir kus gibiyim.
\par 8 Düsmanlarim bütün gün bana hakaret ediyor, Bana dil uzatanlar adimi lanet için kullaniyor.
\par 9 Kizip öfkelendigin için Külü ekmek gibi yiyor, Içecegime gözyasi katiyorum. Beni kaldirip bir yana attin.
\par 11 Günlerim aksam uzayan gölge gibi yitmekte, Ot gibi sararmaktayim.
\par 12 Ama sen, sonsuza dek tahtinda oturursun, ya RAB, Ünün kusaklar boyu sürer.
\par 13 Kalkip Siyon'a sevecenlik göstereceksin, Çünkü onu kayirmanin zamanidir, beklenen zaman geldi.
\par 14 Kullarin onun taslarindan hoslanir, Tozunu bile severler.
\par 15 Uluslar RAB'bin adindan, Yeryüzü krallari görkeminden korkacak.
\par 16 Çünkü RAB Siyon'u yeniden kuracak, Görkem içinde görünecek.
\par 17 Yoksullarin duasina kulak verecek, Yalvarislarini asla hor görmeyecek.
\par 18 Bunlar gelecek kusak için yazilsin, Öyle ki, henüz dogmamis insanlar RAB'be övgüler sunsun.
\par 19 RAB yücelerdeki kutsal katindan asagi bakti, Göklerden yeryüzünü gözetledi,
\par 20 Tutsaklarin iniltisini duymak, Ölüm mahkûmlarini kurtarmak için.
\par 21 Böylece halklar ve kralliklar RAB'be tapinmak için toplaninca, O'nun adi Siyon'da, Övgüsü Yerusalim'de duyurulacak.
\par 23 RAB gücümü kirdi yasam yolunda, Ömrümü kisaltti.
\par 24 "Ey Tanrim, ömrümün ortasinda canimi alma!" dedim. "Senin yillarin kusaklar boyu sürer!
\par 25 "Çok önceden attin dünyanin temellerini, Gökler de senin ellerinin yapitidir.
\par 26 Onlar yok olacak, ama sen kalicisin. Hepsi bir giysi gibi eskiyecek. Onlari bir kaftan gibi degistireceksin, Geçip gidecekler.
\par 27 Ama sen hep aynisin, Yillarin tükenmeyecek.
\par 28 Gözetiminde yasayacak kullarinin çocuklari, Senin önünde duracak soylari."

\chapter{103}

\par 1 RAB'be övgüler sun, ey gönlüm! O'nun kutsal adina övgüler sun, ey bütün varligim!
\par 2 RAB'be övgüler sun, ey canim! Iyiliklerinin hiçbirini unutma!
\par 3 Bütün suçlarini bagislayan, Bütün hastaliklarini iyilestiren,
\par 4 Canini ölüm çukurundan kurtaran, Sana sevgi ve sevecenlik taci giydiren,
\par 5 Yasam boyu seni iyiliklerle doyuran O'dur, Bu nedenle gençligin kartalinki gibi tazelenir.
\par 6 RAB bütün düskünlere Hak ve adalet saglar.
\par 7 Kendi yöntemlerini Musa'ya, Islerini Israilliler'e açikladi.
\par 8 RAB sevecen ve lütfedendir, Tez öfkelenmez, sevgisi engindir.
\par 9 Sürekli suçlamaz, Öfkesini sonsuza dek sürdürmez.
\par 10 Bize günahlarimiza göre davranmaz, Suçlarimizin karsiligini vermez.
\par 11 Çünkü gökler yeryüzünden ne kadar yüksekse, Kendisinden korkanlara karsi sevgisi de o kadar büyüktür.
\par 12 Dogu batidan ne kadar uzaksa, O kadar uzaklastirdi bizden isyanlarimizi.
\par 13 Bir baba çocuklarina nasil sevecen davranirsa, RAB de kendisinden korkanlara öyle sevecen davranir.
\par 14 Çünkü mayamizi bilir, Toprak oldugumuzu animsar.
\par 15 Insana gelince, ota benzer ömrü, Kir çiçegi gibi serpilir;
\par 16 Rüzgar üzerine esince yok olur gider, Bulundugu yer onu tanimaz.
\par 17 Ama RAB kendisinden korkanlari sonsuza dek sever, Antlasmasina uyan Ve buyruklarina uymayi animsayan soylarina adil davranir.
\par 19 RAB tahtini göklere kurmustur, O'nun egemenligi her yeri kapsar.
\par 20 RAB'be övgüler sunun, ey sizler, O'nun melekleri, O'nun sözünü dinleyen, Söylediklerini yerine getiren güç sahipleri!
\par 21 RAB'be övgüler sunun, ey sizler, O'nun bütün göksel ordulari, Istegini yerine getiren kullari!
\par 22 RAB'be övgüler sunun, Ey O'nun egemen oldugu yerlerdeki bütün yaratiklar! RAB'be övgüler sun, ey gönlüm!

\chapter{104}

\par 1 RAB'be övgüler sun, ey gönlüm! Ya RAB Tanrim, ne ulusun! Görkem ve yücelik kusanmissin,
\par 2 Bir kaftana bürünür gibi isiga bürünmüssün. Gökleri bir çadir gibi geren,
\par 3 Evini yukaridaki sular üzerine kuran, Bulutlari kendine savas arabasi yapan, Rüzgarin kanatlari üzerinde gezen,
\par 4 Rüzgarlari kendine haberci, Yildirimlari hizmetkâr eden sensin.
\par 5 Yeryüzünü temeller üzerine kurdun, Asla sarsilmasin diye.
\par 6 Engini ona bir giysi gibi giydirdin, Sular daglarin üzerinde durdu.
\par 7 Sen kükreyince sular kaçti, Gögü gürletince hemen çekildi.
\par 8 Daglari asip derelere akti, Onlar için belirledigin yerlere dogru.
\par 9 Bir sinir koydun önlerine, Geçmesinler, gelip yeryüzünü bir daha kaplamasinlar diye.
\par 10 Vadilerde fiskirttigin pinarlar, Daglarin arasindan akar.
\par 11 Bütün kir hayvanlarini suvarir, Yaban eseklerinin susuzlugunu giderirler.
\par 12 Kuslar yanlarinda yuva kurar, Dallarin arasinda ötüsürler.
\par 13 Gökteki evinden daglari sularsin, Yeryüzü islerinin meyvesine doyar.
\par 14 Hayvanlar için ot, Insanlarin yarari için bitkiler yetistirirsin; Insanlar ekmegini topraktan çikarsin diye,
\par 15 Yüreklerini sevindiren sarabi, Yüzlerini güldüren zeytinyagini, Güçlerini arttiran ekmegi hep sen verirsin.
\par 16 RAB'bin agaçlari, Kendi diktigi Lübnan sedirleri suya doyar.
\par 17 Kuslar orada yuva yapar, Leylegin evi ise çamlardadir.
\par 18 Yüksek daglar dag keçilerinin ugragi, Kayalar kaya tavsanlarinin* siginagidir.
\par 19 Mevsimleri göstersin diye ayi, Batacagi zamani bilen günesi yarattin.
\par 20 Karartirsin ortaligi, gece olur, Baslar kipirdamaya orman hayvanlari.
\par 21 Genç aslan av pesinde kükrer, Tanri'dan yiyecek ister.
\par 22 Günes dogunca Inlerine çekilir, yatarlar.
\par 23 Insan isine gider, Aksama dek çalismak için.
\par 24 Ya RAB, ne çok eserin var! Hepsini bilgece yaptin; Yeryüzü yarattiklarinla dolu.
\par 25 Iste uçsuz bucaksiz denizler, Içinde kaynasan sayisiz canlilar, Büyük küçük yaratiklar.
\par 26 Orada gemiler dolasir, Içinde oynassin diye yarattigin Livyatan* da orada.
\par 27 Hepsi seni bekliyor, Yiyeceklerini zamaninda veresin diye.
\par 28 Sen verince onlar toplar, Sen elini açinca onlar iyilige doyar.
\par 29 Yüzünü gizleyince dehsete kapilirlar, Soluklarini kesince ölüp toprak olurlar.
\par 30 Ruhun'u gönderince var olurlar, Yeryüzüne yeni yasam verirsin.
\par 31 RAB'bin görkemi sonsuza dek sürsün! Sevinsin RAB yaptiklariyla!
\par 32 O bakinca yeryüzü titrer, O dokununca daglar tüter.
\par 33 Ömrümce RAB'be ezgiler söyleyecek, Var oldukça Tanrim'i ilahilerle övecegim.
\par 34 Düsüncem ona hos görünsün, Sevincim RAB olsun!
\par 35 Tükensin dünyadaki günahlilar, Yok olsun artik kötüler! RAB'be övgüler sun, ey gönlüm! RAB'be övgüler sunun!

\chapter{105}

\par 1 RAB'be sükredin, O'na yakarin, Halklara duyurun yaptiklarini!
\par 2 O'nu ezgilerle, ilahilerle övün, Bütün harikalarini anlatin!
\par 3 Kutsal adiyla övünün, Sevinsin RAB'be yönelenler!
\par 4 RAB'be ve O'nun gücüne bakin, Durmadan O'nun yüzünü arayin!
\par 5 Ey sizler, kulu Ibrahim'in soyu, Seçtigi Yakupogullari, O'nun yaptigi harikalari, Olaganüstü islerini Ve agzindan çikan yargilari animsayin!
\par 7 Tanrimiz RAB O'dur, Yargilari bütün yeryüzünü kapsar.
\par 8 O antlasmasini, Bin kusak için verdigi sözü, Ibrahim'le yaptigi antlasmayi, Ishak için içtigi andi sonsuza dek animsar.
\par 10 "Hakkiniza düsen mülk olarak Kenan ülkesini size verecegim" diyerek, Bunu Yakup için bir kural, Israil'le sonsuza dek geçerli bir antlasma yapti.
\par 12 O zaman bir avuç insandilar, Sayica az ve ülkeye yabanciydilar.
\par 13 Bir ulustan öbürüne, Bir ülkeden ötekine dolasip durdular.
\par 14 RAB kimsenin onlari ezmesine izin vermedi, Onlar için krallari bile payladi:
\par 15 "Meshettiklerime* dokunmayin, Peygamberlerime kötülük etmeyin!" dedi.
\par 16 Ülkeye kitlik gönderdi, Bütün yiyeceklerini yok etti.
\par 17 Önlerinden bir adam göndermisti, Köle olarak satilan Yusuf'tu bu.
\par 18 Zincir vurup incittiler ayaklarini, Demir halka geçirdiler boynuna,
\par 19 Söyledikleri gerçeklesinceye dek, RAB'bin sözü onu sinadi.
\par 20 Kral adam gönderip Yusuf'u saliverdi, Halklara egemen olan onu özgür kildi.
\par 21 Onu kendi sarayinin efendisi, Bütün varliginin sorumlusu yapti;
\par 22 Önderlerini istedigi gibi egitsin, Ileri gelenlerine akil versin diye.
\par 23 O zaman Israil Misir'a gitti, Yakup Ham ülkesine yerlesti.
\par 24 RAB halkini alabildigine çogaltti, Düsmanlarindan sayica arttirdi onlari.
\par 25 Sonunda tutumunu degistirdi düsmanlarinin: Halkindan tiksindiler, Kullarina kurnazca davrandilar.
\par 26 Kulu Musa'yi, Seçtigi Harun'u gönderdi aralarina.
\par 27 Onlar gösterdiler RAB'bin belirtilerini, Ham ülkesinde sasilasi islerini.
\par 28 Karanlik gönderip ülkeyi karanliga bürüdü RAB, Çünkü Misirlilar O'nun sözlerine karsi gelmisti.
\par 29 Kana çevirdi sularini, Öldürdü baliklarini.
\par 30 Ülkede kurbagalar kaynasti Krallarinin odalarina kadar.
\par 31 RAB buyurunca sinek sürüleri, Sivrisinekler üsüstü ülkenin her yanina.
\par 32 Dolu yagdirdi yagmur yerine, Simsekler çaktirdi ülkelerinde.
\par 33 Baglarini, incir agaçlarini vurdu, Parçaladi ülkenin agaçlarini.
\par 34 O buyurunca çekirgeler, Sayisiz yavrular kaynadi.
\par 35 Ülkenin bütün bitkilerini yediler, Topragin ürününü yiyip bitirdiler.
\par 36 RAB ülkede ilk doganlarin hepsini, Ilk çocuklarini öldürdü.
\par 37 Israilliler'i ülkeden altin ve gümüsle çikardi, Oymaklarindan tek kisi bile tökezlemedi.
\par 38 Onlar gidince Misir sevindi, Çünkü Israil korkusu çökmüstü Misir'in üzerine.
\par 39 RAB bulutu bir örtü gibi yaydi üzerlerine, Gece ates verdi yollarini aydinlatsin diye.
\par 40 Istediler, bildircin gönderdi, Göksel ekmekle doyurdu karinlarini.
\par 41 Kayayi yardi, sular fiskirdi, Çorak topraklarda bir irmak gibi akti.
\par 42 Çünkü kutsal sözünü, Kulu Ibrahim'e verdigi sözü animsadi.
\par 43 Halkini sevinç içinde, Seçtiklerini sevinç çigliklariyla ülkeden çikardi.
\par 44 Uluslarin topraklarini verdi onlara. Halklarin emegini miras aldilar;
\par 45 Kurallarini yerine getirsinler, Yasalarina uysunlar diye. RAB'be övgüler sunun!

\chapter{106}

\par 1 Övgüler sunun, RAB'be! RAB'be sükredin, çünkü O iyidir, Sevgisi sonsuzdur.
\par 2 RAB'bin büyük islerini kim anlatabilir, Kim O'na yeterince övgü sunabilir?
\par 3 Ne mutlu adalete uyanlara, Sürekli dogru olani yapanlara!
\par 4 Ya RAB, halkina lütfettiginde animsa beni, Onlari kurtardiginda ilgilen benimle.
\par 5 Öyle ki, seçtiklerinin gönencini göreyim, Ulusunun sevincini, Kendi halkinin kivancini paylasayim.
\par 6 Atalarimiz gibi biz de günah isledik, Suç isledik, kötülük ettik.
\par 7 Atalarimiz Misir'dayken Yaptigin harikalari anlamadi, Çok kez gösterdigin sevgiyi animsamadi, Denizde, Kizildeniz'de* baskaldirdilar.
\par 8 Buna karsin RAB gücünü göstermek için, Adi ugruna kurtardi onlari.
\par 9 Kizildeniz'i azarladi, kurudu deniz, Yürüdüler enginde O'nun öncülügünde, Çölde yürür gibi.
\par 10 Kendilerinden nefret edenlerin elinden aldi onlari, Düsmanlarinin pençesinden kurtardi.
\par 11 Sular yuttu hasimlarini, Hiçbiri kurtulmadi.
\par 12 O zaman atalarimiz O'nun sözlerine inandilar, Ezgiler söyleyerek O'nu övdüler.
\par 13 Ne var ki, RAB'bin yaptiklarini çabucak unuttular, Ögüt vermesini beklemediler.
\par 14 Özlemle kivrandilar çölde, Tanri'yi denediler issiz yerlerde.
\par 15 Tanri onlara istediklerini verdi, Ama üzerlerine yipratici bir hastalik gönderdi.
\par 16 Onlar ordugahlarinda Musa'yi, RAB'bin kutsal kulu Harun'u kiskaninca,
\par 17 Yer yarildi ve Datan'i yuttu, Aviram'la yandaslarinin üzerine kapandi.
\par 18 Ates kavurdu onlari izleyenleri, Alev yakti kötüleri.
\par 19 Bir buzagi heykeli yaptilar Horev'de, Dökme bir puta tapindilar.
\par 20 Tanri'nin yüceligini, Ot yiyen öküz putuna degistirdiler.
\par 21 Unuttular kendilerini kurtaran Tanri'yi, Misir'da yaptigi büyük isleri,
\par 22 Ham ülkesinde yarattigi harikalari, Kizildeniz kiyisinda yaptigi müthis isleri.
\par 23 Bu yüzden onlari yok edecegini söyledi Tanri, Ama seçkin kulu Musa O'nun önündeki gedikte durarak, Yok edici öfkesinden vazgeçirdi O'nu.
\par 24 Ardindan hor gördüler güzelim ülkeyi, Tanri'nin verdigi söze inanmadilar.
\par 25 Çadirlarinda söylendiler, Dinlemediler RAB'bin sesini.
\par 26 Bu yüzden RAB elini kaldirdi Ve çölde onlari yere serecegine, Soylarini uluslarin arasina saçacagina, Onlari öteki ülkelere dagitacagina ant içti.
\par 28 Sonra Baal-Peor'a bel bagladilar, Ölülere sunulan kurbanlari yediler.
\par 29 Öfkelendirdiler RAB'bi yaptiklariyla, Salgin hastalik çikti aralarinda.
\par 30 Ama Pinehas kalkip araya girdi, Felaketi önledi.
\par 31 Bu dogruluk sayildi ona, Kusaklar boyu, sonsuza dek sürecek bu.
\par 32 Yine RAB'bi öfkelendirdiler Meriva sulari yaninda, Musa'nin basina dert açildi onlar yüzünden;
\par 33 Çünkü onu sinirlendirdiler, O da düsünmeden konustu.
\par 34 RAB'bin onlara buyurdugu gibi Yok etmediler halklari,
\par 35 Tersine öteki uluslara karistilar, Onlarin törelerini ögrendiler.
\par 36 Putlarina taptilar, Bu da onlara tuzak oldu.
\par 37 Ogullarini, kizlarini Cinlere kurban ettiler.
\par 38 Kenan putlarina kurban olsun diye Ogullarinin, kizlarinin kanini, Suçsuzlarin kanini döktüler; Ülke onlarin kaniyla kirlendi.
\par 39 Böylece yaptiklariyla kirli sayildilar, Vefasiz duruma düstüler töreleriyle.
\par 40 RAB'bin öfkesi parladi halkina karsi, Tiksindi kendi halkindan.
\par 41 Onlari uluslarin eline teslim etti. Onlardan nefret edenler onlara egemen oldu.
\par 42 Düsmanlari onlari ezdi, Boyun egdirdi hepsine.
\par 43 RAB onlari birçok kez kurtardi, Ama akillari fikirleri baskaldirmaktaydi Ve alçaltildilar suçlari yüzünden.
\par 44 RAB yine de ilgilendi sikintilariyla Yakarislarini duyunca.
\par 45 Antlasmasini animsadi onlar ugruna, Essiz sevgisinden ötürü vazgeçti yapacaklarindan.
\par 46 Merhamet koydu onlari tutsak alanlarin yüregine.
\par 47 Kurtar bizi, ey Tanrimiz RAB, Topla bizi uluslarin arasindan. Kutsal adina sükredelim, Yüceliginle övünelim.
\par 48 Öncesizlikten sonsuza dek, Israil'in Tanrisi RAB'be övgüler olsun! Bütün halk, "Amin!" desin. RAB'be övgüler olsun!

\chapter{107}

\par 1 RAB'be sükredin, çünkü O iyidir, Sevgisi sonsuzdur.
\par 2 Böyle desin RAB'bin kurtardiklari, Düsman pençesinden özgür kildiklari,
\par 3 Dogudan, batidan, kuzeyden, güneyden, Bütün ülkelerden topladiklari.
\par 4 Issiz çöllerde dolastilar, Yerlesecekleri kente giden bir yol bulamadilar.
\par 5 Aç, susuz, Sefil oldular.
\par 6 O zaman sikinti içinde RAB'be yakardilar, RAB kurtardi onlari dertlerinden.
\par 7 Yerlesecekleri bir kente varincaya dek, Onlara dogru yolda öncülük etti.
\par 8 Sükretsinler RAB'be sevgisi için, Insanlar yararina yaptigi harikalar için.
\par 9 Çünkü O susamis canin susuzlugunu giderir, Aç cani iyiliklerle doyurur.
\par 10 Zincire vurulmus, aciyla kivranan tutsaklar, Karanlikta, zifiri karanlikta oturmustu.
\par 11 Çünkü Tanri'nin buyruklarina karsi çikmislardi, Küçümsemislerdi Yüceler Yücesi'nin ögüdünü.
\par 12 Agir islerle hayati onlara zehir etti, Çöktüler, yardim eden olmadi.
\par 13 O zaman sikinti içinde RAB'be yakardilar, RAB kurtardi onlari dertlerinden;
\par 14 Çikardi karanliktan, zifiri karanliktan, Kopardi zincirlerini.
\par 15 Sükretsinler RAB'be sevgisi için, Insanlar yararina yaptigi harikalar için!
\par 16 Çünkü tunç* kapilari kirdi, Demir kapi kollarini parçaladi O.
\par 17 Cezalarini buldu aptallar, Suçlari, isyanlari yüzünden.
\par 18 Igrenir olmuslardi bütün yemeklerden, Ölümün kapilarina yaklasmislardi.
\par 19 O zaman sikinti içinde RAB'be yakardilar, RAB kurtardi onlari dertlerinden.
\par 20 Sözünü gönderip iyilestirdi onlari, Kurtardi ölüm çukurundan.
\par 21 Sükretsinler RAB'be sevgisi için, Insanlar yararina yaptigi harikalar için!
\par 22 Sükran kurbanlari sunsunlar Ve sevinç çigliklariyla duyursunlar O'nun yaptiklarini!
\par 23 Gemilerle denize açilanlar, Okyanuslarda is yapanlar,
\par 24 RAB'bin islerini, Derinliklerde yaptigi harikalari gördüler.
\par 25 Çünkü O buyurunca siddetli bir firtina koptu, Dalgalar saha kalkti.
\par 26 Göklere yükselip diplere indi gemiler, Sikintidan canlari burunlarina geldi gemicilerin,
\par 27 Sarhos gibi sallanip sendelediler, Ustaliklari ise yaramadi.
\par 28 O zaman sikinti içinde RAB'be yakardilar, RAB kurtardi onlari dertlerinden.
\par 29 Firtinayi limanliga çevirdi, Yatisti dalgalar;
\par 30 Rahatlayinca sevindiler, Diledikleri limana götürdü RAB onlari.
\par 31 Sükretsinler RAB'be sevgisi için, Insanlar yararina yaptigi harikalar için!
\par 32 Yüceltsinler O'nu halk toplulugunda, Övgüler sunsunlar ileri gelenlerin toplantisinda.
\par 33 Irmaklari çöle çevirir, Pinarlari kurak topraga,
\par 34 Verimli topragi çorak alana, Orada yasayanlarin kötülügü yüzünden.
\par 35 Çölü su birikintisine çevirir, Kuru topragi pinara.
\par 36 Açlari yerlestirir oraya; Oturacak bir kent kursunlar,
\par 37 Tarlalar ekip baglar diksinler, Bol ürün alsinlar diye.
\par 38 RAB'bin kutsamasiyla, Çogaldilar alabildigine, Eksiltmedi hayvanlarini.
\par 39 Sonra azaldilar, alçaldilar, Baski, sikinti ve aci yüzünden.
\par 40 RAB rezalet saçti soylular üzerine, Yolu izi belirsiz bir çölde dolastirdi onlari.
\par 41 Ama yoksulu sefaletten kurtardi, Davar sürüsü gibi çogaltti ailelerini.
\par 42 Dogru insanlar görüp sevinecek, Kötülerse agzini kapayacak.
\par 43 Akli olan bunlari göz önünde tutsun, RAB'bin sevgisini dikkate alsin.

\chapter{108}

\par 1 Kararliyim, ey Tanri, Bütün varligimla sana ezgiler, ilahiler söyleyecegim!
\par 2 Uyan, ey lir, ey çenk, Seheri ben uyandirayim!
\par 3 Halklarin arasinda sana sükürler sunayim, ya RAB, Uluslarin arasinda seni ilahilerle öveyim.
\par 4 Çünkü sevgin göklere erisir, Sadakatin gökyüzüne ulasir.
\par 5 Yüceligini göster göklerin üstünde, ey Tanri, Görkemin bütün yeryüzünü kaplasin!
\par 6 Kurtar bizi sag elinle, yardim et, Sevdiklerin özgürlüge kavussun diye!
\par 7 Tanri söyle konustu kutsal yerinde: "Sekem'i sevinçle bölüstürecek, Sukkot Vadisi'ni ölçecegim.
\par 8 Gilat benimdir, Manasse de benim, Efrayim migferim, Yahuda asam.
\par 9 Moav yikanma legenim, Edom'un üzerine çarigimi firlatacagim, Filist'e zaferle haykiracagim."
\par 10 Kim beni surlu kente götürecek? Kim bana Edom'a kadar yol gösterecek?
\par 11 Ey Tanri, sen bizi reddetmedin mi? Ordularimiza öncülük etmiyor musun artik?
\par 12 Yardim et bize düsmana karsi, Çünkü bostur insan yardimi.
\par 13 Tanri'yla zafer kazaniriz, O çigner düsmanlarimizi.

\chapter{109}

\par 1 Ey övgüler sundugum Tanri, Sessiz kalma!
\par 2 Çünkü kötüler, yalancilar Bana karsi agzini açti, Karaliyorlar beni.
\par 3 Nefret dolu sözlerle beni kusatip Yok yere bana savas açtilar.
\par 4 Sevgime karsilik bana düsman oldular, Bense dua etmekteyim.
\par 5 Iyiligime kötülük, Sevgime nefretle karsilik verdiler.
\par 6 Kötü bir adam koy düsmanin basina, Saginda onu suçlayan biri dursun!
\par 7 Yargilaninca suçlu çiksin, Duasi bile günah sayilsin!
\par 8 Ömrü kisa olsun, Görevini bir baskasi üstlensin!
\par 9 Çocuklari öksüz, Karisi dul kalsin!
\par 10 Çocuklari avare gezip dilensin, Yikik evlerinden uzakta yiyecek arasin!
\par 11 Bütün mallari tefecinin agina düssün, Emegini yabancilar yagmalasin!
\par 12 Kimse ona sevgi göstermesin, Öksüzlerine aciyan olmasin!
\par 13 Soyu kurusun, Bir kusak sonra adi silinsin!
\par 14 Atalarinin suçlari RAB'bin önünde anilsin, Annesinin günahi silinmesin!
\par 15 Günahlari hep RAB'bin önünde dursun, RAB anilarini yok etsin yeryüzünden!
\par 16 Çünkü düsmanim sevgi göstermeyi düsünmedi, Ölesiye baski yapti mazluma, yoksula, Yüregi kirik insana.
\par 17 Sevdigi lanet basina gelsin! Madem kutsamaktan hoslanmiyor, Uzak olsun ondan kutsamak!
\par 18 Laneti bir giysi gibi giydi, Su gibi içine, yag gibi kemiklerine islesin lanet!
\par 19 Bir giysi gibi onu örtünsün, Bir kusak gibi hep onu sarsin!*fu*
\par 20 Düsmanlarima, beni kötüleyenlere, RAB böyle karsilik versin!
\par 21 Ama sen, ey Egemen RAB, Adin ugruna bana ilgi göster; Kurtar beni, iyiligin, sevgin ugruna!
\par 22 Çünkü düskün ve yoksulum, Yüregim yarali içimde.
\par 23 Batan günes gibi geçip gidiyorum, Çekirge gibi silkilip atiliyorum.
\par 24 Dizlerim titriyor oruç* tutmaktan; Bir deri bir kemige döndüm.
\par 25 Düsmanlarima yüzkarasi oldum; Beni görünce kafalarini salliyorlar!
\par 26 Yardim et bana, ya RAB Tanrim; Kurtar beni sevgin ugruna!
\par 27 Bilsinler bu iste senin elin oldugunu, Bunu senin yaptigini, ya RAB!
\par 28 Varsin lanet etsin onlar, sen kutsa beni, Bana saldiranlar utanacak, Ben kulunsa sevinecegim.
\par 29 Rezillige bürünsün beni suçlayanlar, Kaftan giyer gibi utançlariyla örtünsünler!
\par 30 RAB'be çok sükredecegim, Kalabaligin arasinda O'na övgüler dizecegim;
\par 31 Çünkü O yoksulun saginda durur, Onu yargilayanlardan kurtarmak için.

\chapter{110}

\par 1 RAB efendime: "Ben düsmanlarini ayaklarinin altina serinceye dek Sagimda otur" diyor.
\par 2 RAB Siyon'dan uzatacak kudret asani, Düsmanlarinin ortasinda egemenlik sür!
\par 3 Savasacagin gün Gönüllü gidecek askerlerin. Seherin bagrindan dogan çiy gibi Kutsal giysiler içinde Sana gelecek gençlerin.
\par 4 RAB ant içti, kararindan dönmez: "Melkisedek düzeni uyarinca Sonsuza dek kâhinsin sen!" dedi.
\par 5 Rab senin sagindadir, Krallari ezecek öfkelendigi gün.
\par 6 Uluslari yargilayacak, ortaligi cesetler dolduracak, Dünyanin dört bucaginda baslari ezecek.
\par 7 Yol kenarindaki dereden su içecek; Bu yüzden basini dik tutacak.

\chapter{111}

\par 1 Övgüler sunun RAB'be! Dogru insanlarin toplantisinda, Topluluk içinde, Bütün yüregimle RAB'be sükredecegim.
\par 2 RAB'bin isleri büyüktür, Onlardan zevk alanlar hep onlari düsünür.
\par 3 O'nun yaptiklari yüce ve görkemlidir, Dogrulugu sonsuza dek sürer.
\par 4 RAB unutulmayacak harikalar yapti, O sevecen ve lütfedendir.
\par 5 Kendisinden korkanlari besler, Antlasmasini sonsuza dek animsar.
\par 6 Uluslarin topraklarini kendi halkina vermekle Gösterdi onlara islerinin gücünü.
\par 7 Yaptigi her iste sadik ve adildir, Bütün kosullari güvenilirdir;
\par 8 Sonsuza dek sürer, Sadakat ve dogrulukla yapilir.
\par 9 O halkinin kurtulusunu sagladi, Antlasmasini sonsuza dek geçerli kildi. Adi kutsal ve müthistir.
\par 10 Bilgeligin temeli RAB korkusudur, O'nun kurallarini yerine getiren herkes Sagduyu sahibi olur. O'na sonsuza dek övgü sunulur!

\chapter{112}

\par 1 Övgüler sunun RAB'be! Ne mutlu RAB'den korkan insana, O'nun buyruklarindan büyük zevk alana!
\par 2 Soyu yeryüzünde güç kazanacak, Dogrularin kusagi kutsanacak.
\par 3 Bolluk ve zenginlik eksilmez evinden, Sonsuza dek sürer dogrulugu.
\par 4 Karanlikta isik dogar dürüstler için, Lütfeden, sevecen, dogru insanlar için.
\par 5 Ne mutlu eli açik olan, ödünç veren, Islerini adaletle yürüten insana!
\par 6 Asla sarsilmaz, Sonsuza dek anilir dogru insan.
\par 7 Kötü haberden korkmaz, Yüregi sarsilmaz, RAB'be güvenir.
\par 8 Gözü pektir, korku nedir bilmez, Sonunda düsmanlarinin yenilgisini görür.
\par 9 Armaganlar dagitti, yoksullara verdi; Dogrulugu sonsuza dek kalicidir, Gücü ve sayginligi artar.
\par 10 Kötü kisi bunu görünce kudurur, Dislerini gicirdatir, kendi kendini yer, bitirir. Kötülerin dilegi bosa çikar.

\chapter{113}

\par 1 Övgüler sunun RAB'be! Övgüler sunun, ey RAB'bin kullari, RAB'bin adina övgüler sunun!
\par 2 Simdiden sonsuza dek RAB'bin adina sükürler olsun!
\par 3 Günesin dogdugu yerden battigi yere kadar RAB'bin adina övgüler sunulmali!
\par 4 RAB bütün uluslara egemendir, Görkemi gökleri asar.
\par 5 Var mi Tanrimiz RAB gibi, Yücelerde oturan,
\par 6 Göklerde ve yeryüzünde olanlara Bakmak için egilen?
\par 7 Düskünü yerden kaldirir, Yoksulu çöplükten çikarir;
\par 8 Soylularla, Halkinin soylulariyla birlikte oturtsun diye.
\par 9 Kisir kadini evde oturtur, Çocuk sahibi mutlu bir anne kilar. RAB'be övgüler sunun!

\chapter{114}

\par 1 Israil Misir'dan çiktiginda, Yakup'un soyu yabanci dil konusan bir halktan ayrildiginda,
\par 2 Yahuda Rab'bin kutsal yeri oldu, Israil de O'nun kralligi.
\par 3 Deniz olani görüp geri çekildi, Seria Irmagi tersine akti.
\par 4 Daglar koç gibi, Tepeler kuzu gibi siçradi.
\par 5 Ey deniz, sana ne oldu da kaçtin? Ey Seria, neden tersine aktin?
\par 6 Ey daglar, niçin koç gibi, Ey tepeler, niçin kuzu gibi siçradiniz?
\par 7 Titre, ey yeryüzü, Kayayi havuza, Çakmaktasini pinara çeviren Rab'bin önünde, Yakup'un Tanrisi'nin huzurunda.

\chapter{115}

\par 1 Bizi degil, ya RAB, bizi degil, Sevgin ve sadakatin ugruna, Kendi adini yücelt!
\par 2 Niçin uluslar: "Hani, nerede onlarin Tanrisi?" desin.
\par 3 Bizim Tanrimiz göklerdedir, Ne isterse yapar.
\par 4 Oysa onlarin putlari altin ve gümüsten yapilmis, Insan elinin eseridir.
\par 5 Agizlari var, konusmazlar, Gözleri var, görmezler,
\par 6 Kulaklari var, duymazlar, Burunlari var, koku almazlar,
\par 7 Elleri var, hissetmezler, Ayaklari var, yürümezler, Bogazlarindan ses çikmaz.
\par 8 Onlari yapan, onlara güvenen herkes Onlar gibi olacak!
\par 9 Ey Israil halki, RAB'be güven, O'dur yardimciniz ve kalkaniniz!
\par 10 Ey Harun soyu, RAB'be güven, O'dur yardimciniz ve kalkaniniz!
\par 11 Ey RAB'den korkanlar, RAB'be güvenin, O'dur yardimciniz ve kalkaniniz!
\par 12 RAB bizi animsayip kutsayacak, Israil halkini, Harun soyunu kutsayacak.
\par 13 Küçük, büyük, Kendisinden korkan herkesi kutsayacak.
\par 14 RAB sizi, Sizi ve çocuklarinizi çogaltsin!
\par 15 Yeri gögü yaratan RAB Sizleri kutsasin.
\par 16 Göklerin öteleri RAB'bindir, Ama yeryüzünü insanlara vermistir.
\par 17 Ölüler, sessizlik diyarina inenler, RAB'be övgüler sunmaz;
\par 18 Biziz RAB'bi öven, Simdiden sonsuza dek. RAB'be övgüler sunun!

\chapter{116}

\par 1 RAB'bi seviyorum, Çünkü O feryadimi duyar.
\par 2 Bana kulak verdigi için, Yasadigim sürece O'na seslenecegim.
\par 3 Ölüm iplerine dolasmistim, Ölüler diyarinin kâbusu yakama yapismisti, Sikintiya, aciya gömülmüstüm.
\par 4 O zaman RAB'be yakardim, "Aman, ya RAB, kurtar canimi!" dedim.
\par 5 RAB lütufkâr ve adildir, Sevecendir Tanrimiz.
\par 6 RAB saf insanlari korur, Tükendigim zaman beni kurtardi.
\par 7 Ey canim, yine huzura kavus, Çünkü RAB sana iyilik etti.
\par 8 Sen, ya RAB, canimi ölümden, Gözlerimi yastan, Ayaklarimi sürçmekten kurtardin.
\par 9 Yasayanlarin diyarinda, RAB'bin huzurunda yürüyecegim.
\par 10 Iman ettim, "Büyük aci çekiyorum" dedigim zaman bile.
\par 11 Saskinlik içinde, "Bütün insanlar yalanci" dedim.
\par 12 Ne karsilik verebilirim RAB'be, Bana yaptigi onca iyilik için?
\par 13 Kurtulus sunusu olarak kadeh kaldirip RAB'be seslenecegim.
\par 14 Bütün halkinin önünde, RAB'be adadiklarimi yerine getirecegim.
\par 15 RAB'bin gözünde degerlidir Sadik kullarinin ölümü.
\par 16 Ya RAB, ben gerçekten senin kulunum; Kulun, hizmetçinin ogluyum, Sen çözdün baglarimi.
\par 17 Ya RAB, sana seslenecek, Sükran kurbani sunacagim.
\par 18 RAB'be adadiklarimi yerine getirecegim Bütün halkinin önünde,
\par 19 RAB'bin Tapinagi'nin avlularinda, Senin orta yerinde, ey Yerusalim! RAB'be övgüler sunun!

\chapter{117}

\par 1 Ey bütün uluslar, RAB'be övgüler sunun! Ey bütün halklar, O'nu yüceltin!
\par 2 Çünkü bize besledigi sevgi büyüktür, RAB'bin sadakati sonsuza dek sürer. RAB'be övgüler sunun!

\chapter{118}

\par 1 RAB'be sükredin, çünkü O iyidir, Sevgisi sonsuzdur.
\par 2 "Sonsuzdur sevgisi!" desin Israil halki.
\par 3 "Sonsuzdur sevgisi!" desin Harun'un soyu.
\par 4 "Sonsuzdur sevgisi!" desin RAB'den korkanlar.
\par 5 Sikinti içinde RAB'be seslendim; Yanitladi, rahata kavusturdu beni.
\par 6 RAB benden yana, korkmam; Insan bana ne yapabilir?
\par 7 RAB benden yana, benim yardimcim, Benden nefret edenlerin sonuna zaferle bakacagim.
\par 8 RAB'be siginmak Insana güvenmekten iyidir.
\par 9 RAB'be siginmak Soylulara güvenmekten iyidir.
\par 10 Bütün uluslar beni kusatti, RAB'bin adiyla püskürttüm onlari.
\par 11 Kusattilar, sardilar beni, RAB'bin adiyla püskürttüm onlari.
\par 12 Arilar gibi sardilar beni, Ama diken atesi gibi sönüverdiler; RAB'bin adiyla püskürttüm onlari.
\par 13 Itilip kakildim, düsmek üzereydim, Ama RAB yardim etti bana.
\par 14 RAB gücüm ve ezgimdir, O kurtardi beni.
\par 15 Sevinç ve zafer çigliklari Çinliyor dogrularin çadirlarinda: "RAB'bin sag eli güçlü isler yapar!
\par 16 RAB'bin sag eli üstündür, RAB'bin sag eli güçlü isler yapar!"
\par 17 Ölmeyecek, yasayacagim, RAB'bin yaptiklarini duyuracagim.
\par 18 RAB beni siddetle yola getirdi, Ama ölüme terk etmedi.
\par 19 Açin bana adalet kapilarini, Girip RAB'be sükredeyim.
\par 20 Iste budur RAB'bin kapisi! Dogrular girebilir oradan.
\par 21 Sana sükrederim, çünkü bana yanit verdin, Kurtaricim oldun.
\par 22 Yapicilarin reddettigi tas, Kösenin bas tasi oldu.
\par 23 RAB'bin isidir bu, Gözümüzde harika bir is!
\par 24 Bugün RAB'bin yarattigi gündür, Onun için sevinip cosalim!
\par 25 Ne olur, ya RAB, kurtar bizi, Ne olur, basarili kil bizi!
\par 26 Kutsansin RAB'bin adiyla gelen! Kutsuyoruz sizi RAB'bin evinden.
\par 27 RAB Tanri'dir, aydinlatti bizi. Iplerle baglayin bayram kurbanini, Ilerleyin sunagin boynuzlarina kadar.
\par 28 Tanrim sensin, sükrederim sana, Tanrim sensin, yüceltirim seni.
\par 29 RAB'be sükredin, çünkü O iyidir, Sevgisi sonsuzdur.

\chapter{119}

\par 1 Ne mutlu yollari temiz olanlara, RAB'bin yasasina göre yasayanlara!
\par 2 Ne mutlu O'nun ögütlerine uyanlara, Bütün yüregiyle O'na yönelenlere!
\par 3 Hiç haksizlik etmezler, O'nun yolunda yürürler.
\par 4 Koydugun kosullara Dikkatle uyulmasini buyurdun.
\par 5 Keske kararli olsam Senin kurallarina uymakta!
\par 6 Hiç utanmayacagim, Bütün buyruklarini izledikçe.
\par 7 Sükredecegim sana temiz yürekle, Adil hükümlerini ögrendikçe.
\par 8 Kurallarini yerine getirecegim, Birakma beni hiçbir zaman!
\par 9 Genç insan yolunu nasil temiz tutar? Senin sözünü tutmakla.
\par 10 Bütün yüregimle sana yöneliyorum, Izin verme buyruklarindan sapmama!
\par 11 Aklimdan çikarmam sözünü, Sana karsi günah islememek için.
\par 12 Övgüler olsun sana, ya RAB, Bana kurallarini ögret.
\par 13 Agzindan çikan bütün hükümleri Dudaklarimla yineliyorum.
\par 14 Sevinç duyuyorum ögütlerini izlerken, Sanki benim oluyor bütün hazineler.
\par 15 Kosullarini derin derin düsünüyorum, Yollarini izlerken.
\par 16 Zevk aliyorum kurallarindan, Sözünü unutmayacagim.
\par 17 Ben kuluna iyilik et ki yasayayim, Sözüne uyayim.
\par 18 Gözlerimi aç, Yasandaki harikalari göreyim.
\par 19 Garibim bu dünyada, Buyruklarini benden gizleme!
\par 20 Içim tükeniyor, Her an hükümlerini özlemekten.
\par 21 Buyruklarindan sapan Lanetli küstahlari azarlarsin.
\par 22 Uzaklastir benden küçümsemeleri, hakaretleri, Çünkü ögütlerini tutuyorum.
\par 23 Önderler toplanip beni kötüleseler bile, Ben kulun senin kurallarini derin derin düsünecegim.
\par 24 Ögütlerin benim zevkimdir, Bana akil verirler.
\par 25 Toza topraga serildim, Sözün uyarinca yasam ver bana.
\par 26 Yaptiklarimi açikladim, beni yanitladin; Kurallarini ögret bana!
\par 27 Kosullarini anlamami sagla ki, Harikalarinin üzerinde düsüneyim.
\par 28 Içim eriyor kederden, Sözün uyarinca güçlendir beni!
\par 29 Yalan yoldan uzaklastir, Yasan uyarinca lütfet bana.
\par 30 Ben sadakat yolunu seçtim, Hükümlerini uygun gördüm.
\par 31 Ögütlerine dört elle sarildim, ya RAB, Utandirma beni!
\par 32 Içime huzur verdigin için Buyruklarin dogrultusunda kosacagim.
\par 33 Kurallarini nasil izleyecegimi ögret bana, ya RAB, Öyle ki, onlari sonuna kadar izleyeyim.
\par 34 Anlamami sagla, yasana uyayim, Bütün yüregimle onu yerine getireyim.
\par 35 Buyruklarin dogrultusunda yol göster bana, Çünkü yolundan zevk alirim.
\par 36 Yüregimi haksiz kazanca degil, Kendi ögütlerine yönelt.
\par 37 Gözlerimi bos seylerden çevir, Beni kendi yolunda yasat.
\par 38 Senden korkulmasi için Ben kuluna verdigin sözü yerine getir.
\par 39 Korktugum hakaretten uzak tut beni, Çünkü senin ilkelerin iyidir.
\par 40 Çok özlüyorum senin kosullarini! Beni dogrulugunun içinde yasat!
\par 41 Bana sevgini göster, ya RAB, Sözün uyarinca kurtar beni!
\par 42 O zaman beni asagilayanlara Gereken yaniti verebilirim, Çünkü senin sözüne güvenirim.
\par 43 Gerçegini agzimdan düsürme, Çünkü senin hükümlerine umut bagladim.
\par 44 Yasana sürekli, Sonsuza dek uyacagim.
\par 45 Özgürce yürüyecegim, Çünkü senin kosullarina yöneldim ben.
\par 46 Krallarin önünde senin ögütlerinden söz edecek, Utanç duymayacagim.
\par 47 Senin buyruklarindan zevk aliyor, Onlari seviyorum.
\par 48 Saygi ve sevgi duyuyorum buyruklarina, Derin derin düsünüyorum kurallarini.
\par 49 Kuluna verdigin sözü animsa, Bununla umut verdin bana.
\par 50 Aci çektigimde beni avutan budur, Sözün bana yasam verir.
\par 51 Çok eglendiler küstahlar benimle, Yine de yasandan sasmadim.
\par 52 Geçmiste verdigin hükümleri animsayinca, Avundum, ya RAB.
\par 53 Çileden çikiyorum, Yasani terk eden kötüler yüzünden.
\par 54 Senin kurallarindir ezgilerimin konusu, Konuk oldugum bu dünyada.
\par 55 Gece adini anarim, ya RAB, Yasana uyarim.
\par 56 Tek yaptigim, Senin kosullarina uymak.
\par 57 Benim payima düsen sensin, ya RAB, Sözlerini yerine getirecegim, dedim.
\par 58 Bütün yüregimle sana yakardim. Lütfet bana, sözün uyarinca.
\par 59 Tuttugum yollari düsündüm, Senin ögütlerine göre adim attim.
\par 60 Buyruklarina uymak için Elimi çabuk tuttum, oyalanmadim.
\par 61 Kötülerin ipleri beni sardi, Yasani unutmadim.
\par 62 Dogru hükümlerin için Gece yarisi kalkip sana sükrederim.
\par 63 Dostuyum bütün senden korkanlarin, Kosullarina uyanlarin.
\par 64 Yeryüzü sevginle dolu, ya RAB, Kurallarini ögret bana!
\par 65 Ya RAB, iyilik ettin kuluna, Sözünü tuttun.
\par 66 Bana sagduyu ve bilgi ver, Çünkü inaniyorum buyruklarina.
\par 67 Aci çekmeden önce yoldan sapardim, Ama simdi sözüne uyuyorum.
\par 68 Sen iyisin, iyilik edersin; Bana kurallarini ögret.
\par 69 Küstahlar yalanlarla beni lekeledi, Ama ben bütün yüregimle senin kosullarina uyarim.
\par 70 Onlarin yüregi yag bagladi, Bense zevk alirim yasandan.
\par 71 Iyi oldu aci çekmem; Çünkü kurallarini ögreniyorum.
\par 72 Agzindan çikan yasa benim için Binlerce altin ve gümüsten daha degerlidir.
\par 73 Senin ellerin beni yaratti, biçimlendirdi. Anlamami sagla ki buyruklarini ögreneyim.
\par 74 Senden korkanlar beni görünce sevinsin, Çünkü senin sözüne umut bagladim.
\par 75 Biliyorum, ya RAB, hükümlerin adildir; Bana aci çektirirken bile sadiksin.
\par 76 Ben kuluna verdigin söz uyarinca, Sevgin beni avutsun.
\par 77 Sevecenlik göster bana, yasayayim, Çünkü yasandan zevk aliyorum.
\par 78 Utansin küstahlar beni yalan yere suçladiklari için. Bense senin kosullarini düsünüyorum.
\par 79 Bana dönsün senden korkanlar, Ögütlerini bilenler.
\par 80 Yüregim kusursuz uysun kurallarina, Öyle ki, utanç duymayayim.
\par 81 Içim tükeniyor senin kurtarisini özlerken, Senin sözüne umut bagladim ben.
\par 82 Gözümün feri sönüyor söz verdiklerini beklemekten, "Ne zaman avutacaksin beni?" diye soruyorum.
\par 83 Dumandan kararmis tuluma döndüm, Yine de unutmuyorum kurallarini.
\par 84 Daha ne kadar bekleyecek kulun? Ne zaman yargilayacaksin bana zulmedenleri?
\par 85 Çukur kazdilar benim için Yasana uymayan küstahlar.
\par 86 Bütün buyruklarin güvenilirdir; Haksiz yere zulmediyorlar, yardim et bana!
\par 87 Nerdeyse sileceklerdi beni yeryüzünden, Ama ben senin kosullarindan ayrilmadim.
\par 88 Koru canimi sevgin uyarinca, Tutayim agzindan çikan ögütleri.
\par 89 Ya RAB, sözün Göklerde sonsuza dek duruyor.
\par 90 Sadakatin kusaklar boyu sürüyor, Kurdugun yeryüzü sapasaglam duruyor.
\par 91 Bugün hükümlerin uyarinca ayakta duran her sey Sana kulluk ediyor.
\par 92 Eger yasan zevk kaynagim olmasaydi, Çektigim acilardan yok olurdum.
\par 93 Kosullarini asla unutmayacagim, Çünkü onlarla bana yasam verdin.
\par 94 Kurtar beni, çünkü seninim, Senin kosullarina yöneldim.
\par 95 Kötüler beni yok etmeyi beklerken, Ben senin ögütlerini inceliyorum.
\par 96 Kusursuz olan her seyin bir sonu oldugunu gördüm, Ama senin buyrugun sinir tanimaz.
\par 97 Ne kadar severim yasani! Bütün gün düsünürüm onun üzerinde.
\par 98 Buyruklarin beni düsmanlarimdan bilge kilar, Çünkü her zaman aklimdadir onlar.
\par 99 Bütün ögretmenlerimden daha akilliyim, Çünkü ögütlerin üzerinde düsünüyorum.
\par 100 Yaslilardan daha bilgeyim, Çünkü senin kosullarina uyuyorum.
\par 101 Sakinirim her kötü yoldan, Senin sözünü tutmak için.
\par 102 Ayrilmam hükümlerinden, Çünkü bana sen ögrettin.
\par 103 Ne tatli geliyor verdigin sözler damagima, Baldan tatli geliyor agzima!
\par 104 Senin kosullarina uymakla bilgelik kazaniyorum, Bu yüzden nefret ediyorum her yanlis yoldan.
\par 105 Sözün adimlarim için çira, Yolum için isiktir.
\par 106 Adil hükümlerini izleyecegime ant içtim, Andimi tutacagim.
\par 107 Çok sikinti çektim, ya RAB; Koru hayatimi sözün uyarinca.
\par 108 Agzimdan çikan içten övgüleri Kabul et, ya RAB, Bana hükümlerini ögret.
\par 109 Hayatim her an tehlikede, Yine de unutmam yasani.
\par 110 Kötüler tuzak kurdu bana, Yine de sapmadim senin kosullarindan.
\par 111 Ögütlerin sonsuza dek mirasimdir, Yüregimin sevincidir onlar.
\par 112 Kararliyim Sonuna kadar senin kurallarina uymaya.
\par 113 Döneklerden tiksinir, Senin yasani severim.
\par 114 Siginagim ve kalkanim sensin, Senin sözüne umut baglarim.
\par 115 Ey kötüler, benden uzak durun, Tanrim'in buyruklarini yerine getireyim.
\par 116 Sözün uyarinca destek ol bana, yasam bulayim; Umudumu bosa çikarma!
\par 117 Siki tut beni, kurtulayim, Her zaman kurallarini dikkate alayim.
\par 118 Kurallarindan sapan herkesi reddedersin, Çünkü onlarin hileleri bostur.
\par 119 Dünyadaki kötüleri cüruf gibi atarsin, Bu yüzden severim senin ögütlerini.
\par 120 Bedenim ürperiyor dehsetinden, Korkuyorum hükümlerinden.
\par 121 Adil ve dogru olani yaptim, Gaddarlarin eline birakma beni!
\par 122 Güven altina al kulunun mutlulugunu, Baski yapmasin bana küstahlar.
\par 123 Gözümün feri sönüyor, Beni kurtarmani, Adil sözünü yerine getirmeni beklemekten.
\par 124 Kuluna sevgin uyarinca davran, Bana kurallarini ögret.
\par 125 Ben senin kulunum, bana akil ver ki, Ögütlerini anlayabileyim.
\par 126 Ya RAB, harekete geçmenin zamanidir, Yasani çigniyorlar.
\par 127 Bu yüzden senin buyruklarini, Altindan, saf altindan daha çok seviyorum;
\par 128 Koydugun kosullarin hepsini dogru buluyorum, Her yanlis yoldan tiksiniyorum.
\par 129 Harika ögütlerin var, Bu yüzden onlara candan uyuyorum.
\par 130 Sözlerinin açiklanisi aydinlik saçar, Saf insanlara akil verir.
\par 131 Agzim açik, soluk solugayim, Çünkü buyruklarini özlüyorum.
\par 132 Bana lütufla bak, Adini sevenlere her zaman yaptigin gibi.
\par 133 Adimlarimi pekistir verdigin söz uyarinca, Hiçbir suç bana egemen olmasin.
\par 134 Kurtar beni insan baskisindan, Kosullarina uyabileyim.
\par 135 Yüzün aydinlik saçsin kulunun üzerine, Kurallarini ögret bana.
\par 136 Oluk oluk yas akiyor gözlerimden, Çünkü uymuyorlar yasana.
\par 137 Sen adilsin, ya RAB, Hükümlerin dogrudur.
\par 138 Buyurdugun ögütler dogru Ve tam güvenilirdir.
\par 139 Gayretim beni tüketti, Çünkü düsmanlarim unuttu senin sözlerini.
\par 140 Sözün çok güvenilirdir, Kulun onu sever.
\par 141 Önemsiz ve horlanan biriyim ben, Ama kosullarini unutmuyorum.
\par 142 Adaletin sonsuza dek dogrudur, Yasan gerçektir.
\par 143 Sikintiya, darliga düstüm, Ama buyruklarin benim zevkimdir.
\par 144 Ögütlerin sonsuza dek dogrudur; Bana akil ver ki, yasayayim.
\par 145 Bütün yüregimle haykiriyorum, Yanitla beni, ya RAB! Senin kurallarina uyacagim.
\par 146 Sana sesleniyorum, Kurtar beni, Ögütlerine uyayim.
\par 147 Gün dogmadan kalkip yardim dilerim, Senin sözüne umut bagladim.
\par 148 Verdigin söz üzerinde düsüneyim diye, Gece boyunca uyku girmiyor gözüme.
\par 149 Sevgin uyarinca sesime kulak ver, Hükümlerin uyarinca, ya RAB, yasam ver bana!
\par 150 Yaklasiyor kötülük ardinca kosanlar, Yasandan uzaklasiyorlar.
\par 151 Oysa sen yakinsin, ya RAB, Bütün buyruklarin gerçektir.
\par 152 Çoktan beri anladim Ögütlerini sonsuza dek verdigini.
\par 153 Çektigim sikintiyi gör, kurtar beni, Çünkü yasani unutmadim.
\par 154 Davami savun, özgür kil beni, Sözün uyarinca koru canimi.
\par 155 Kurtulus kötülerden uzaktir, Çünkü senin kurallarina yönelmiyorlar.
\par 156 Çok sevecensin, ya RAB, Hükümlerin uyarinca koru canimi.
\par 157 Bana zulmedenler, düsmanlarim çok, Yine de sapmadim senin ögütlerinden.
\par 158 Tiksinerek bakiyorum hainlere, Çünkü uymuyorlar senin sözüne.
\par 159 Bak, ne kadar seviyorum kosullarini, Sevgin uyarinca, ya RAB, koru canimi.
\par 160 Sözlerinin temeli gerçektir, Dogru hükümlerinin tümü sonsuza dek sürecektir.
\par 161 Yok yere zulmediyor bana önderler, Oysa yüregim senin sözünle titrer.
\par 162 Ganimet bulan biri gibi Verdigin sözlerde sevinç bulurum.
\par 163 Tiksinir, igrenirim yalandan, Ama senin yasani severim.
\par 164 Dogru hükümlerin için Seni günde yedi kez överim.
\par 165 Yasani sevenler büyük esenlik bulur, Hiçbir sey sendeletmez onlari.
\par 166 Ya RAB, kurtarisina umut baglar, Buyruklarini yerine getiririm.
\par 167 Ögütlerine candan uyar, Onlari çok severim.
\par 168 Ögütlerini, kosullarini uygularim, Çünkü bütün davranislarimi görürsün sen.
\par 169 Feryadim sana erissin, ya RAB, Sözün uyarinca akil ver bana!
\par 170 Yalvarisim sana ulassin; Verdigin söz uyarinca kurtar beni!
\par 171 Dudaklarimdan övgüler aksin, Çünkü bana kurallarini ögretiyorsun.
\par 172 Dilimde sözün ezgilere dönüssün, Çünkü bütün buyruklarin dogrudur.
\par 173 Elin bana yardima hazir olsun, Çünkü senin kosullarini seçtim ben.
\par 174 Kurtarisini özlüyorum, ya RAB, Yasan zevk kaynagimdir.
\par 175 Beni yasat ki, sana övgüler sunayim, Hükümlerin bana yardimci olsun.
\par 176 Kaybolmus koyun gibi avare dolasiyordum; Kulunu ara, Çünkü buyruklarini unutmadim ben.

\chapter{120}

\par 1 Sikintiya düsünce RAB'be seslendim; Yanitladi beni.
\par 2 Ya RAB, kurtar canimi yalanci dudaklardan, Aldatici dillerden!
\par 3 Ey aldatici dil, RAB ne verecek sana, Daha ne verecek?
\par 4 Yigidin sivri oklariyla Retem çalisindan alevli korlar!
\par 5 Vay bana, Mesek'te garip kaldim sanki, Kedar çadirlari arasinda oturdum.
\par 6 Fazla kaldim Baristan nefret edenler arasinda.
\par 7 Ben baris yanlisiyim, Ama söze basladigimda, Onlar savasa kalkiyor!

\chapter{121}

\par 1 Gözlerimi daglara kaldiriyorum, Nereden yardim gelecek?
\par 2 Yeri gögü yaratan RAB'den gelecek yardim.
\par 3 O ayaklarinin kaymasina izin vermez, Seni koruyan uyuklamaz.
\par 4 Israil'in koruyucusu ne uyur ne uyuklar.
\par 5 Senin koruyucun RAB'dir, O sag yaninda sana gölgedir.
\par 6 Gündüz günes, Gece ay sana zarar vermez.
\par 7 RAB her kötülükten seni korur, Esirger canini.
\par 8 Simdiden sonsuza dek RAB koruyacak gidisini, gelisini.

\chapter{122}

\par 1 Bana: "RAB'bin evine gidelim" dendikçe Sevinirim.
\par 2 Ayaklarimiz senin kapilarinda, Ey Yerusalim!
\par 3 Bitisik nizamda kurulmus bir kenttir Yerusalim!
\par 4 Oymaklar çikar oraya, RAB'bin oymaklari, Israil'e verilen ögüt uyarinca, RAB'bin adina sükretmek için.
\par 5 Çünkü orada yargi tahtlari, Davut soyunun tahtlari kurulmustur.
\par 6 Esenlik dileyin Yerusalim'e: "Huzur bulsun seni sevenler!
\par 7 Surlarina esenlik, Saraylarina huzur egemen olsun!"
\par 8 Kardeslerim, dostlarim için, "Esenlik olsun sana!" derim.
\par 9 Tanrimiz RAB'bin evi için Iyilik dilerim sana.

\chapter{123}

\par 1 Gözlerimi sana kaldiriyorum, Ey göklerde taht kuran!
\par 2 Nasil kullarin gözleri efendilerinin, Hizmetçinin gözleri haniminin eline bakarsa, Bizim gözlerimiz de RAB Tanrimiz'a öyle bakar, O bize aciyincaya dek.
\par 3 Aci bize, ya RAB, aci; Gördügümüz hakaret yeter de artar.
\par 4 Rahat yasayanlarin alaylari, Küstahlarin hakareti Canimiza yetti.

\chapter{124}

\par 1 RAB bizden yana olmasaydi, Desin simdi Israil:
\par 2 RAB bizden yana olmasaydi, Insanlar bize saldirdiginda,
\par 3 Diri diri yutarlardi bizi, Öfkeleri bize karsi alevlenince.
\par 4 Sular silip süpürürdü bizleri, Seller geçerdi üzerimizden.
\par 5 Kabaran sular Asardi basimizdan.
\par 6 Övgüler olsun Bizi onlarin agzina yem etmeyen RAB'be!
\par 7 Bir kus gibi Kurtuldu canimiz avcinin tuzagindan, Kirildi tuzak, kurtulduk.
\par 8 Yeri gögü yaratan RAB'bin adi yardimcimizdir.

\chapter{125}

\par 1 RAB'be güvenenler Siyon Dagi'na benzer, Sarsilmaz, sonsuza dek durur.
\par 2 Daglar Yerusalim'i nasil kusatmissa, RAB de halkini öyle kusatmistir, Simdiden sonsuza dek.
\par 3 Kalmayacak kötülerin asasi, Dogrularin payina düsen toprakta, Yoksa dogrular haksizliga el uzatabilir.
\par 4 Iyilik et, ya RAB, Iyilere, yüregi temiz olanlara.
\par 5 Ama kendi halkindan egri yollara sapanlari, RAB kötü uluslarla birlikte kovacak. Israil'e esenlik olsun!

\chapter{126}

\par 1 RAB sürgünleri Siyon'a geri getirince, Rüya gibi geldi bize.
\par 2 Agzimiz gülüslerle, Dilimiz sevinç çigliklariyla doldu. "RAB onlar için büyük isler yapti" Diye konusuldu uluslar arasinda.
\par 3 RAB bizim için büyük isler yapti, Sevinç doldu içimiz.
\par 4 Ya RAB, eski gönencimize kavustur bizi, Negev'de suya kavusan vadiler gibi.
\par 5 Gözyaslari içinde ekenler, Sevinç çigliklariyla biçecek;
\par 6 Aglayarak tohum çuvalini tasiyip dolasan, Sevinç çigliklari atarak demetlerle dönecek.

\chapter{127}

\par 1 Evi RAB yapmazsa, Yapicilar bosuna didinir. Kenti RAB korumazsa, Bekçi bosuna bekler.
\par 2 Bosuna erken kalkip Geç yatiyorsunuz. Ey zahmetle kazanilan ekmegi yiyenler, RAB sevdiklerinin rahat uyumasini saglar.
\par 3 Çocuklar RAB'bin verdigi bir armagandir, Rahmin ürünü bir ödüldür.
\par 4 Yigidin elinde nasilsa oklar, Öyledir gençlikte dogan çocuklar.
\par 5 Ne mutlu ok kilifi onlarla dolu insana! Kent kapisinda hasimlariyla tartisirken Utanç duymayacaklar.

\chapter{128}

\par 1 Ne mutlu RAB'den korkana, O'nun yolunda yürüyene!
\par 2 Emeginin ürününü yiyeceksin, Mutlu ve basarili olacaksin.
\par 3 Esin evinde verimli bir asma gibi olacak; Çocuklarin zeytin filizleri gibi sofranin çevresinde.
\par 4 Iste RAB'den korkan kisi Böyle kutsanacak.
\par 5 RAB seni Siyon'dan kutsasin! Yerusalim'in gönencini göresin, Bütün yasamin boyunca!
\par 6 Çocuklarinin çocuklarini göresin! Israil'e esenlik olsun!

\chapter{129}

\par 1 Gençligimden beri bana sik sik saldirdilar; Simdi söylesin Israil:
\par 2 "Gençligimden beri bana sik sik saldirdilar, Ama yenemediler beni.
\par 3 Çiftçiler saban sürdüler sirtimda, Upuzun iz biraktilar."
\par 4 Ama RAB adildir, Kesti kötülerin baglarini.
\par 5 Siyon'dan nefret eden herkes Utanç içinde geri çekilsin.
\par 6 Damlardaki ota, Büyümeden kuruyan ota dönsünler.
\par 7 Orakçi avucunu, Demetçi kucagini dolduramaz onunla.
\par 8 Yoldan geçenler de, "RAB sizi kutsasin, RAB'bin adiyla sizi kutsariz" demezler.

\chapter{130}

\par 1 Derinliklerden sana sesleniyorum, ya RAB,
\par 2 Sesimi isit, ya Rab, Yalvarisima iyi kulak ver!
\par 3 Ya RAB, sen suçlarin hesabini tutsan, Kim ayakta kalabilir, ya Rab?
\par 4 Ama sen bagislayicisin, Öyle ki senden korkulsun.
\par 5 RAB'bi gözlüyorum, Canim RAB'bi gözlüyor, Umut bagliyorum O'nun sözüne.
\par 6 Sabahi gözleyenlerden, Evet, sabahi gözleyenlerden daha çok, Canim Rab'bi gözlüyor.
\par 7 Ey Israil, RAB'be umut bagla! Çünkü RAB'de sevgi, Tam kurtulus vardir.
\par 8 Israil'i bütün suçlarindan Fidyeyle O kurtaracaktir.

\chapter{131}

\par 1 Ya RAB, yüregimde gurur yok, Gözüm yükseklerde degil. Büyük islerle, Kendimi asan harika islerle ugrasmiyorum.
\par 2 Tersine, ana kucaginda sütten kesilmis çocuk gibi, Kendimi yatistirip huzur buldum, Sütten kesilmis çocuga döndüm.
\par 3 Ey Israil, RAB'be umut bagla Simdiden sonsuza dek!

\chapter{132}

\par 1 Ya RAB, Davut'un hatiri için, Çektigi bütün zorluklari, Sana nasil ant içtigini, Yakup'un güçlü Tanrisi'na adak adadigini animsa:
\par 3 "Evime gitmeyecegim, Yatagima uzanmayacagim,
\par 4 Gözlerime uyku girmeyecek, Göz kapaklarim kapanmayacak,
\par 5 RAB'be bir yer, Yakup'un güçlü Tanrisi'na bir konut buluncaya dek."
\par 6 Antlasma Sandigi'nin* Efrata'da oldugunu duyduk, Onu Yaar kirlarinda bulduk.
\par 7 "RAB'bin konutuna gidelim, Ayaginin taburesi önünde tapinalim" dedik.
\par 8 Çik, ya RAB, yasayacagin yere, Gücünü simgeleyen sandikla birlikte.
\par 9 Kâhinlerin dogrulugu kusansin, Sadik kullarin sevinç çigliklari atsin.
\par 10 Kulun Davut'un hatiri için, Meshettigin* krala yüz çevirme.
\par 11 RAB Davut'a kesin ant içti, Andindan dönmez: "Senin soyundan birini tahtina oturtacagim.
\par 12 Eger ogullarin antlasmama, Verecegim ögütlere uyarlarsa, Onlarin ogullari da sonsuza dek Senin tahtina oturacak."
\par 13 Çünkü RAB Siyon'u seçti, Onu konut edinmek istedi.
\par 14 "Sonsuza dek yasayacagim yer budur" dedi, "Burada oturacagim, çünkü bunu kendim istedim.
\par 15 Çok bereketli kilacagim erzagini, Yiyecekle doyuracagim yoksullarini.
\par 16 Kurtulusla donatacagim kâhinlerini; Hep sevinç ezgileri söyleyecek sadik kullari.
\par 17 Burada Davut soyundan güçlü bir kral çikaracagim, Meshettigim kralin soyunu Isik olarak sürdürecegim.
\par 18 Düsmanlarini utanca bürüyecegim, Ama onun basindaki taç parildayacak."

\chapter{133}

\par 1 Ne iyi, ne güzeldir, Birlik içinde kardesçe yasamak!
\par 2 Basa sürülen degerli yag gibi, Sakaldan, Harun'un sakalindan Kaftaninin yakasina dek inen yag gibi.
\par 3 Hermon Dagi'na yagan çiy Siyon daglarina yagiyor sanki. Çünkü RAB orada bereketi, Sonsuz yasami buyurdu.

\chapter{134}

\par 1 Ey sizler, RAB'bin bütün kullari, RAB'bin Tapinagi'nda gece hizmet edenler, O'na övgüler sunun!
\par 2 Ellerinizi kutsal yere dogru kaldirip RAB'be övgüler sunun!
\par 3 Yeri gögü yaratan RAB kutsasin sizi Siyon'dan.

\chapter{135}

\par 1 RAB'be övgüler sunun! RAB'bin adina övgüler sunun, Ey RAB'bin kullari! Ey sizler, RAB'bin Tapinagi'nda, Tanrimiz'in Tapinagi'nin avlularinda hizmet edenler, Övgüler sunun!
\par 3 RAB'be övgüler sunun, Çünkü RAB iyidir. Adini ilahilerle övün, Çünkü hostur bu.
\par 4 RAB kendine Yakup soyunu, Öz halki olarak Israil'i seçti.
\par 5 Biliyorum, RAB büyüktür, Rabbimiz bütün ilahlardan üstündür.
\par 6 RAB ne isterse yapar, Göklerde, yeryüzünde, Denizlerde, bütün derinliklerde.
\par 7 Yeryüzünün dört bucagindan bulutlar yükseltir, Yagmur için simsek çaktirir, Ambarlarindan rüzgar estirir.
\par 8 Insanlardan hayvanlara dek Misir'da ilk doganlari öldürdü.
\par 9 Ey Misir, senin orta yerinde, Firavunla bütün görevlilerine Belirtiler, sasilasi isler gösterdi.
\par 10 Birçok ulusu bozguna ugratti, Güçlü krallari öldürdü:
\par 11 Amorlu kral Sihon'u, Basan Krali Og'u, Bütün Kenan krallarini.
\par 12 Topraklarini mülk, Evet, mülk olarak halki Israil'e verdi.
\par 13 Ya RAB, adin sonsuza dek sürecek, Bütün kusaklar seni anacak.
\par 14 RAB halkini hakli çikarir, Kullarina acir.
\par 15 Uluslarin putlari altin ve gümüsten yapilmis, Insan elinin eseridir.
\par 16 Agizlari var, konusmazlar, Gözleri var, görmezler,
\par 17 Kulaklari var, duymazlar, Soluk alip vermezler.
\par 18 Onlari yapan, onlara güvenen herkes Onlar gibi olacak!
\par 19 Ey Israil halki, RAB'be övgüler sun! Ey Harun soyu, RAB'be övgüler sun!
\par 20 Ey Levi soyu, RAB'be övgüler sun! RAB'be övgüler sunun, ey RAB'den korkanlar!
\par 21 Yerusalim'de oturan RAB'be Siyon'dan övgüler sunulsun! RAB'be övgüler sunun!

\chapter{136}

\par 1 Sükredin RAB'be, çünkü O iyidir, Sevgisi sonsuzdur;
\par 2 Sükredin tanrilar Tanrisi'na, Sevgisi sonsuzdur;
\par 3 Sükredin rabler Rabbi'ne, Sevgisi sonsuzdur;
\par 4 Büyük harikalar yapan tek varliga, Sevgisi sonsuzdur;
\par 5 Gökleri bilgece yaratana, Sevgisi sonsuzdur;
\par 6 Yeri sular üzerine yayana, Sevgisi sonsuzdur;
\par 7 Büyük isiklar yaratana, Sevgisi sonsuzdur;
\par 8 Gündüze egemen olsun diye günesi, Sevgisi sonsuzdur;
\par 9 Geceye egemen olsun diye ayi ve yildizlari yaratana, Sevgisi sonsuzdur;
\par 10 Misir'da ilk doganlari öldürene, Sevgisi sonsuzdur;
\par 11 Güçlü eli, kudretli koluyla Sevgisi sonsuzdur; Israil'i Misir'dan çikarana, Sevgisi sonsuzdur;
\par 13 Kizildeniz'i* ikiye bölene, Sevgisi sonsuzdur;
\par 14 Israil'i ortasindan geçirene, Sevgisi sonsuzdur;
\par 15 Firavunla ordusunu Kizildeniz'e dökene, Sevgisi sonsuzdur;
\par 16 Kendi halkini çölde yürütene, Sevgisi sonsuzdur;
\par 17 Büyük krallari vurana, Sevgisi sonsuzdur;
\par 18 Güçlü krallari öldürene, Sevgisi sonsuzdur;
\par 19 Amorlu kral Sihon'u, Sevgisi sonsuzdur;
\par 20 Basan Krali Og'u öldürene, Sevgisi sonsuzdur;
\par 21 Topraklarini mülk olarak, Sevgisi sonsuzdur; Kulu Israil'e mülk verene, Sevgisi sonsuzdur;
\par 23 Düskün günlerimizde bizi animsayana, Sevgisi sonsuzdur;
\par 24 Düsmanlarimizdan bizi kurtarana, Sevgisi sonsuzdur;
\par 25 Bütün canlilara yiyecek verene, Sevgisi sonsuzdur;
\par 26 Sükredin Göklerin Tanrisi'na, Sevgisi sonsuzdur.

\chapter{137}

\par 1 Babil irmaklari kiyisinda oturup Siyon'u andikça agladik;
\par 2 Çevredeki kavaklara Lirlerimizi astik.
\par 3 Çünkü orada bizi tutsak edenler bizden ezgiler, Bize zulmedenler bizden senlik istiyor, "Siyon ezgilerinden birini okuyun bize!" diyorlardi.
\par 4 Nasil okuyabiliriz RAB'bin ezgisini El topraginda?
\par 5 Ey Yerusalim, seni unutursam, Sag elim kurusun.
\par 6 Seni anmaz, Yerusalim'i en büyük sevincimden üstün tutmazsam, Dilim damagima yapissin!
\par 7 Yerusalim'in düstügü gün, "Yikin onu, yikin temellerine kadar!" Diyen Edomlular'in tavrini animsa, ya RAB.
\par 8 Ey sen, yikilasi Babil kizi, Bize yaptiklarini Sana ödetecek olana ne mutlu!
\par 9 Ne mutlu senin yavrularini tutup Kayalarda parçalayacak insana!

\chapter{138}

\par 1 Bütün yüregimle sana sükrederim, ya RAB, Ilahlar önünde seni ilahilerle överim.
\par 2 Kutsal tapinagina dogru egilir, Adina sükrederim, Sevgin, sadakatin için. Çünkü adini ve sözünü her seyden üstün tuttun.
\par 3 Seslendigim gün bana yanit verdin, Içime güç koydun, beni yüreklendirdin.
\par 4 Sükretsin sana, ya RAB, yeryüzü krallarinin tümü, Çünkü agzindan çikan sözleri isittiler.
\par 5 Yaptigin isleri ezgilerle övsünler, ya RAB, Çünkü çok yücesin.
\par 6 RAB yüksekse de, Alçakgönüllüleri gözetir, Küstahlari uzaktan tanir.
\par 7 Sikintiya düsersem, canimi korur, Düsmanlarimin öfkesine karsi el kaldirirsin, Sag elin beni kurtarir.
\par 8 Ya RAB, her seyi yaparsin benim için. Sevgin sonsuzdur, ya RAB, Elinin eserini birakma!

\chapter{139}

\par 1 Ya RAB, sinayip tanidin beni.
\par 2 Oturup kalkisimi bilirsin, Niyetimi uzaktan anlarsin.
\par 3 Gittigim yolu, yattigim yeri inceden inceye elersin, Bütün yaptiklarimdan haberin var.
\par 4 Daha sözü agzima almadan, Söyleyecegim her seyi bilirsin, ya RAB.
\par 5 Beni çepeçevre kusattin, Elini üzerime koydun.
\par 6 Kaldiramam böylesi bir bilgiyi, Basa çikamam, erisemem.
\par 7 Nereye gidebilirim senin Ruhun'dan, Nereye kaçabilirim huzurundan?
\par 8 Göklere çiksam, oradasin, Ölüler diyarina yatak sersem, yine oradasin.
\par 9 Seherin kanatlarini alip uçsam, Denizin ötesine konsam,
\par 10 Orada bile elin yol gösterir bana, Sag elin tutar beni.
\par 11 Desem ki, "Karanlik beni kaplasin, Çevremdeki aydinlik geceye dönsün."
\par 12 Karanlik bile karanlik sayilmaz senin için, Gece, gündüz gibi isildar, Karanlikla aydinlik birdir senin için.
\par 13 Iç varligimi sen yarattin, Annemin rahminde beni sen ördün.
\par 14 Sana övgüler sunarim, Çünkü müthis ve harika yaratilmisim. Ne harika islerin var! Bunu çok iyi bilirim.
\par 15 Gizli yerde yaratildigimda, Yerin derinliklerinde örüldügümde, Bedenim senden gizli degildi.
\par 16 Henüz döl yatagindayken gözlerin gördü beni; Bana ayrilan günlerin hiçbiri gelmeden, Hepsi senin kitabina yazilmisti.
\par 17 Hakkimdaki düsüncelerin ne degerli, ey Tanri, Sayilari ne çok!
\par 18 Kum tanelerinden fazladir saymaya kalksam. Uyaniyorum, hâlâ seninleyim.
\par 19 Ey Tanri, keske kötüleri öldürsen! Ey eli kanli insanlar, uzaklasin benden!
\par 20 Çünkü senin için kötü konusuyorlar, Adini kötüye kullaniyor düsmanlarin.
\par 21 Ya RAB, nasil tiksinmem senden tiksinenlerden? Nasil igrenmem sana baskaldiranlardan?
\par 22 Onlardan tümüyle nefret ediyor, Onlari düsman sayiyorum.
\par 23 Ey Tanri, yokla beni, tani yüregimi, Sina beni, ögren kaygilarimi.
\par 24 Bak, seni gücendiren bir yönüm var mi, Öncülük et bana sonsuz yasam yolunda!

\chapter{140}

\par 1 Ya RAB, kurtar beni kötü insandan, Koru beni zorbadan.
\par 2 Onlar yüreklerinde kötülük tasarlar, Savasi sürekli körükler,
\par 3 Yilan gibi dillerini bilerler, Engerek zehiri var dudaklarinin altinda. *
\par 4 Ya RAB, sakin beni kötünün elinden, Koru beni zorbadan; Bana çelme takmayi tasarliyorlar.
\par 5 Küstahlar benim için tuzak kurdu, Haydutlar ag gerdi; Yol kenarina kapan koydular benim için.
\par 6 Sana diyorum, ya RAB: "Tanrim sensin." Yalvarisima kulak ver, ya RAB.
\par 7 Ey Egemen RAB, güçlü kurtaricim, Savas gününde basimi korudun.
\par 8 Kötülerin dileklerini yerine getirme, ya RAB, Tasarilarini ileri götürme! Yoksa gurura kapilirlar.
\par 9 Beni kusatanlarin basini, Dudaklarindan dökülen fesat kaplasin.
\par 10 Kizgin korlar yagsin üzerlerine! Atese, dipsiz çukurlara atilsinlar, Bir daha kalkamasinlar.
\par 11 Iftiracilara ülkede hayat kalmasin, Felaket zorbalari amansizca avlasin.
\par 12 Biliyorum, RAB mazlumun davasini savunur, Yoksullari hakli çikarir.
\par 13 Kuskusuz dogrular senin adina sükredecek, Dürüstler senin huzurunda oturacak.

\chapter{141}

\par 1 Seni çagiriyorum, ya RAB, yardimima kos! Sana yakarinca sesime kulak ver!
\par 2 Duam önünde yükselen buhur gibi, El açisim aksam sunusu gibi kabul görsün!
\par 3 Ya RAB, agzima bekçi koy, Dudaklarimin kapisini koru!
\par 4 Yüregim kötülüge egilim göstermesin, Suç isleyenlerin fesadina bulasmayayim; Onlarin nefis yemeklerini tatmayayim.
\par 5 Dogru insan bana vursa, iyilik sayilir, Azarlasa, basa sürülen yag gibidir, Basim reddetmez onu. Çünkü duam hep kötülere karsidir.
\par 6 Önderleri kayalardan asagi atilinca, Dinleyecekler tatli sözlerimi.
\par 7 Sabanla sürülüp yarilmis toprak gibi, Saçilmis kemiklerimiz ölüler diyarinin agzina.
\par 8 Ancak gözlerim sende, ey Egemen RAB, Sana siginiyorum, beni savunmasiz birakma!
\par 9 Koru beni kurduklari tuzaktan, Suç isleyenlerin kapanlarindan.
\par 10 Ben güvenlik içinde geçip giderken, Kendi aglarina düssün kötüler.

\chapter{142}

\par 1 Yüksek sesle yakariyorum RAB'be, Yüksek sesle RAB'be yalvariyorum.
\par 2 Önüne döküyorum yakinmalarimi, Önünde anlatiyorum sikintilarimi.
\par 3 Bunalima düstügümde, Gidecegim yolu sen bilirsin. Tuzak kurdular yürüdügüm yola.
\par 4 Sagima bak da gör, Kimse saymiyor beni, Siginacak yerim kalmadi, Kimse aramiyor beni.
\par 5 Sana haykiriyorum, ya RAB: "Siginagim, Yasadigimiz bu dünyada nasibim sensin" diyorum.
\par 6 Haykirisima kulak ver, Çünkü çok çaresizim; Kurtar beni ardima düsenlerden, Çünkü benden güçlüler.
\par 7 Çikar beni zindandan, Adina sükredeyim. O zaman dogrular çevremi saracak, Bana iyilik ettigin için.

\chapter{143}

\par 1 Duami isit, ya RAB, Yalvarislarima kulak ver! Sadakatinle, dogrulugunla yanitla beni!
\par 2 Kulunla yargiya girme, Çünkü hiçbir canli senin karsinda aklanmaz.
\par 3 Düsman beni kovaliyor, Ezip yere seriyor. Çoktan ölmüs olanlar gibi, Beni karanliklarda oturtuyor.
\par 4 Bu yüzden bunalima düstüm, Yüregim perisan.
\par 5 Geçmis günleri aniyor, Bütün yaptiklarini derin derin düsünüyor, Ellerinin isine bakip daliyorum.
\par 6 Ellerimi sana açiyorum, Canim kurak toprak gibi sana susamis. *
\par 7 Çabuk yanitla beni, ya RAB, Tükeniyorum. Çevirme benden yüzünü, Yoksa ölüm çukuruna inen ölülere dönerim.
\par 8 Sabahlari duyur bana sevgini, Çünkü sana güveniyorum; Bana gidecegim yolu bildir, Çünkü duam sanadir.
\par 9 Düsmanlarimdan kurtar beni, ya RAB; Sana siginiyorum.
\par 10 Bana istemini yapmayi ögret, Çünkü Tanrim'sin benim. Senin iyi Ruhun Düz yolda bana öncülük etsin!
\par 11 Ya RAB, adin ugruna yasam ver bana, Dogrulugunla kurtar beni sikintidan.
\par 12 Sevginden ötürü, Öldür düsmanlarimi, Yok et bütün hasimlarimi, Çünkü senin kulunum ben.

\chapter{144}

\par 1 Ellerime vurusmayi, Parmaklarima savasmayi ögreten Kayam RAB'be övgüler olsun!
\par 2 O'dur benim vefali dostum, kalem, Kurtaricim, kulem, Kalkanim, O'na siginirim; O'dur halklari bana boyun egdiren!
\par 3 Ya RAB, insan ne ki, onu gözetesin, Insan soyu ne ki, onu düsünesin?
\par 4 Insan bir solugu andirir, Günleri geçici bir gölge gibidir.
\par 5 Ya RAB, gökleri yar, asagiya in, Dokun daglara, tütsünler.
\par 6 Simsek çaktir, dagit düsmani, Savur oklarini, saskina çevir onlari.
\par 7 Yukaridan elini uzat, kurtar beni; Çikar derin sulardan, Al eloglunun elinden.
\par 8 Onlarin agzi yalan saçar, Sag ellerini kaldirir, yalan yere ant içerler.
\par 9 Ey Tanri, sana yeni bir ezgi söyleyeyim, Seni on telli çenkle, ilahilerle öveyim.
\par 10 Sensin krallari zafere ulastiran, Kulun Davut'u kötülük kilicindan kurtaran.
\par 11 Kurtar beni, özgür kil Eloglunun elinden. Onlarin agzi yalan saçar, Sag ellerini kaldirir, yalan yere ant içerler.
\par 12 O zaman gençliginde Saglikli yetisen fidan gibi olacak ogullarimiz, Sarayin oymali sütunlari gibi olacak kizlarimiz.
\par 13 Her türlü ürünle dolup tasacak ambarlarimiz; Binlerce, on binlerce yavrulayacak Çayirlarda davarlarimiz.
\par 14 Semiz olacak sigirlarimiz; Surlarimiza gedik açilmayacak, Insanlarimiz sürgün edilmeyecek, Meydanlarimizda feryat duyulmayacak!
\par 15 Ne mutlu bunlara sahip olan halka! Ne mutlu Tanrisi RAB olan halka!

\chapter{145}

\par 1 Ey Tanrim, ey Kral, seni yüceltecegim, Adini sonsuza dek övecegim.
\par 2 Seni her gün övecek, Adini sonsuza dek yüceltecegim.
\par 3 RAB büyüktür, yalniz O övgüye yarasiktir, Akil ermez büyüklügüne.
\par 4 Yaptiklarin kusaktan kusaga sükranla anilacak, Güçlü islerin duyurulacak.
\par 5 Düsünecegim harika islerini, Insanlar büyüklügünü, yüce görkemini konusacak.
\par 6 Yaptigin müthis islerin gücünden söz edecekler, Ben de senin büyüklügünü duyuracagim.
\par 7 Essiz iyiliginin anilarini kutlayacak, Sevinç ezgileriyle övecekler dogrulugunu.
\par 8 RAB lütufkâr ve sevecendir, Tez öfkelenmez, sevgisi engindir.
\par 9 RAB herkese iyi davranir, Sevecenligi bütün yapitlarini kapsar.
\par 10 Bütün yapitlarin sana sükreder, ya RAB, Sadik kullarin sana övgüler sunar.
\par 11 Kralliginin yüceligini anlatir, Kudretini konusur;
\par 12 Herkes senin gücünü, Kralliginin yüce görkemini bilsin diye.
\par 13 Senin kralligin ebedi kralliktir, Egemenligin kusaklar boyunca sürer. RAB verdigi bütün sözleri tutar, Her davranisi sadiktir.
\par 14 RAB her düsene destek olur, Iki büklüm olanlari dogrultur.
\par 15 Herkesin umudu sende, Onlara yiyeceklerini zamaninda veren sensin.
\par 16 Elini açar, Bütün canlilari doyurursun dilediklerince.
\par 17 RAB bütün davranislarinda adil, Yaptigi bütün islerde sevecendir.
\par 18 RAB kendisine yakaran, Içtenlikle yakaran herkese yakindir.
\par 19 Dilegini yerine getirir kendisinden korkanlarin, Feryatlarini isitir, onlari kurtarir.
\par 20 RAB korur kendisini seven herkesi, Yok eder kötülerin hepsini.
\par 21 RAB'be övgüler sunsun agzim! Bütün canlilar O'nun kutsal adina, Sonsuza dek övgüler dizsin.

\chapter{146}

\par 1 RAB'be övgüler sunun! Ey gönlüm, RAB'be övgüler sun.
\par 2 Yasadikça RAB'be övgüler sunacak, Var oldukça Tanrim'a ilahiler söyleyecegim.
\par 3 Önderlere, Sizi kurtaramayacak insanlara güvenmeyin.
\par 4 O son solugunu verince topraga döner, O gün tasarilari da biter.
\par 5 Ne mutlu yardimcisi Yakup'un Tanrisi olan insana, Umudu Tanrisi RAB'de olana!
\par 6 Yeri gögü, Denizi ve içindeki her seyi yaratan, Sonsuza dek sadik kalan,
\par 7 Ezilenlerin hakkini alan, Açlara yiyecek saglayan O'dur. RAB tutsaklari özgür kilar,
\par 8 Körlerin gözünü açar, Iki büklüm olanlari dogrultur, Dogrulari sever.
\par 9 RAB garipleri korur, Öksüze, dul kadina yardim eder, Kötülerin yolunuysa saptirir.
\par 10 RAB Tanrin sonsuza dek, ey Siyon, Kusaklar boyunca egemenlik sürecek. RAB'be övgüler sunun!

\chapter{147}

\par 1 RAB'be övgüler sunun! Ne güzel, ne hos Tanrimiz'i ilahilerle övmek! O'na övgü yarasir.
\par 2 RAB yeniden kuruyor Yerusalim'i, Bir araya topluyor Israil'in sürgünlerini.
\par 3 O kirik kalplileri iyilestirir, Yaralarini sarar.
\par 4 Yildizlarin sayisini belirler, Her birini adiyla çagirir.
\par 5 Rabbimiz büyük ve çok güçlüdür, Sinirsizdir anlayisi.
\par 6 RAB mazlumlara yardim eder, Kötüleri yere çalar.
\par 7 RAB'be sükran ezgileri okuyun, Tanrimiz'i lirle, ilahilerle övün.
\par 8 O'dur gökleri bulutlarla kaplayan, Yeryüzüne yagmur saglayan, Daglarda ot bitiren.
\par 9 O yiyecek saglar hayvanlara, Bagrisan kuzgun yavrularina.
\par 10 Ne atin gücünden zevk alir, Ne de insanin yigitliginden hoslanir.
\par 11 RAB kendisinden korkanlardan, Sevgisine umut baglayanlardan hoslanir.
\par 12 RAB'bi yücelt, ey Yerusalim! Tanrin'a övgüler sun, ey Siyon!
\par 13 Çünkü senin kapilarinin kol demirlerine güç katar, Içindeki halki kutsar.
\par 14 Sinirlarini esenlik içinde tutar, Seni en iyi bugdayla doyurur.
\par 15 Yeryüzüne buyrugunu gönderir, Sözü çarçabuk yayilir.
\par 16 Yapagi gibi kar yagdirir, Kiragiyi kül gibi saçar.
\par 17 Asagiya iri iri dolu savurur, Kim dayanabilir soguguna?
\par 18 Buyruk verir, eritir buzlari, Rüzgarini estirir, sular akmaya baslar.
\par 19 Sözünü Yakup soyuna, Kurallarini, ilkelerini Israil'e bildirir.
\par 20 Baska hiçbir ulus için yapmadi bunu, Onlar O'nun ilkelerini bilmezler. RAB'be övgüler sunun!

\chapter{148}

\par 1 RAB'be övgüler sunun! Göklerden RAB'be övgüler sunun, Yücelerde O'na övgüler sunun!
\par 2 Ey bütün melekleri, O'na övgüler sunun, Övgüler sunun O'na, ey bütün göksel ordulari!
\par 3 Ey günes, ay, O'na övgüler sunun, Övgüler sunun O'na, ey isildayan bütün yildizlar!
\par 4 Ey göklerin gökleri Ve göklerin üstündeki sular, O'na övgüler sunun!
\par 5 RAB'bin adina övgüler sunsunlar, Çünkü O buyruk verince, var oldular;
\par 6 Bozulmayacak bir kural koyarak, Onlari sonsuza dek yerlerine oturttu.
\par 7 Yeryüzünden RAB'be övgüler sunun, Ey deniz canavarlari, bütün enginler,
\par 8 Simsek, dolu, kar, bulutlar, O'nun buyruguna uyan firtinalar,
\par 9 Daglar, bütün tepeler, Meyve agaçlari, sedir agaçlari,
\par 10 Yabanil ve evcil hayvanlar, Sürüngenler*, uçan kuslar,
\par 11 Yeryüzünün krallari, bütün halklar, Önderler, yeryüzünün bütün yöneticileri,
\par 12 Delikanlilar, genç kizlar, Yaslilar, çocuklar!
\par 13 RAB'bin adina övgüler sunsunlar, Çünkü yalniz O'nun adi yücedir. O'nun yüceligi yerin gögün üstündedir.
\par 14 RAB kendi halkini güçlü kildi, Bütün sadik kullarina, Kendisine yakin olan halka, Israilliler'e ün kazandirdi. RAB'be övgüler sunun!

\chapter{149}

\par 1 RAB'be övgüler sunun! RAB'be yeni bir ezgi söyleyin, Sadik kullarinin toplantisinda O'nu ezgilerle övün!
\par 2 Israil Yaraticisi'nda sevinç bulsun, Siyon halki Krallari'yla cossun!
\par 3 Dans ederek övgüler sunsunlar O'nun adina, Tef ve lir çalarak O'nu ilahilerle övsünler!
\par 4 Çünkü RAB halkindan hoslanir, Alçakgönüllüleri zafer taciyla süsler.
\par 5 Bu onurla mutlu olsun sadik kullari, Sevinç ezgileri okusunlar yataklarinda!
\par 6 Agizlarinda Tanri'ya yüce övgüler, Ellerinde iki agizli kiliçla
\par 7 Uluslardan öç alsinlar, Halklari cezalandirsinlar,
\par 8 Krallarini zincire, Soylularini prangaya vursunlar!
\par 9 Yazilan karari onlara uygulasinlar! Bütün sadik kullari için onurdur bu. RAB'be övgüler sunun!

\chapter{150}

\par 1 RAB'be övgüler sunun! Kutsal yerde Tanri'ya övgüler sunun! Gücünü gösteren göklerde övgüler sunun O'na!
\par 2 Övgüler sunun O'na güçlü isleri için! Övgüler sunun O'na essiz büyüklügüne yarasir biçimde!
\par 3 Boru çalarak O'na övgüler sunun! Çenkle ve lirle O'na övgüler sunun!
\par 4 Tef ve dansla O'na övgüler sunun! Saz ve neyle O'na övgüler sunun!
\par 5 Zillerle O'na övgüler sunun! Çinlayan zillerle O'na övgüler sunun!
\par 6 Bütün canli varliklar RAB'be övgüler sunsun! RAB'be övgüler sunun!


\end{document}